\article{百官公卿表}

\begin{pinyinscope}
易敘宓羲、神農、皇帝作教化民,而傳述其官,以為宓羲龍師名官,神農火師火名,黃帝雲師雲名,少昊鳥師鳥名。自顓頊以來,為民師而命以民事,有重黎、句芒、祝融、后土、蓐收、玄冥之官,然已上矣。書載唐虞之際,命羲和四子順天文,授民時;咨四岳,以舉賢材,揚側陃;十有二牧,柔遠能邇;禹作司空,平水土;棄作后稷,播百穀;镨作司徒,敷五教;咎繇作士,正五刑;垂作共工,利器用;籄作朕虞,育草木鳥獸;伯夷作秩宗,典三禮;夔典樂,和神人;龍作納言,出入帝命。夏、殷亡聞焉,周官則備矣。天官冢宰,地官司徒,春官宗伯,夏官司馬,秋官司寇,冬官司空,是為六卿,各有徒屬職分,用於百事。太師、太傅、太保,是為三公,蓋參天子,坐而議政,無不總統,故不以一職為官名。又立三少為之副,少師、少傅、少保,是為孤卿,與六卿為九焉。記曰三公無官,言有其人然後充之,舜之於堯,伊尹於湯,周公、召公於周,是也。或說司馬主天,司徒主人,司空主土,是為三公。四岳謂四方諸侯。自周衰,官失而百職亂,戰國並爭,各變異。秦兼天下,建皇帝之號,立百官之職。漢因循而不革,明簡易,隨時宜也。其後頗有所改。王莽篡位,慕從古官,而吏民弗安,亦多虐政,遂以亂亡。故略表舉大分,以通古今,備溫故知新之義云。

相國、丞相,皆秦官,金印紫綬,掌丞天子助理萬機。秦有左右,高帝即位,置一丞相,十一年更名相國,綠綬。孝惠、高后置左右丞相,文帝二年復置一丞相。有兩長史,秩千石。哀帝元壽二年更名大司徒。武帝元狩五年初置司直,秩比二千石,掌佐丞相舉不法。

太尉,秦官,金印紫綬,掌武事。武帝建元二年省。元狩四年初置大司馬,以冠將軍之號。宣帝地節三年置大司馬,不冠將軍,亦無印綬官屬。成帝綏和元年賜大司馬金印紫綬,置官屬,祿比丞相,去將軍。哀帝建平二年復去大司馬印綬、官屬,冠將軍如故。元壽二年復賜大司馬印綬,置官屬,去將軍,位在司徒上。有長史,秩千石。

御史大夫,秦官,位上卿,銀印青綬,掌副丞相。有兩丞,秩千石。一曰中丞,在殿中蘭臺,掌圖籍祕書,外督部刺史,內領侍御史員十五人,受公卿奏事,舉劾按章。成帝綏和元年更名大司空,金印紫綬,祿比丞相,置長史如中丞,官職如故。哀帝建平二年復為御史大夫,元壽二年復為大司空,御史中丞更名御史長史。侍御史有繡衣直指,出討姦猾,治大獄,武帝所制,不常置。

太傅,古官,高后元年初置,金印紫綬。後省,八年復置。後省,哀帝元壽二年復置。位在三公上。

太師、太保,皆古官,平帝元始元年皆初置,金印紫綬。太師位在太傅上,太保次太傅。

前後左右將軍,皆周末官,秦因之,位上卿,金印紫綬。漢不常置,或有前後,或有左右,皆掌兵及四夷。有長史,秩千石。

奉常,秦官,掌宗廟禮儀,有丞。景帝中六年更名太常。屬官有太樂、太祝、太宰、太史、太卜、太醫六令丞,又均官、都水兩長丞,又諸廟寢園食官令長丞,有廱太宰、太祝令丞,五畤各一尉。又博士及諸陵縣皆屬焉。景帝中六年更名太祝為祠祀,武帝太初元年更曰廟祀,初置太卜。博士,秦官,掌通古今,秩比六百石,員多至數十人。武帝建元五年初置五經博士,宣帝黃龍元年稍增員十二人。元帝永光元年分諸陵邑屬三輔。王莽改太常曰秩宗。

郎中令,秦官,掌宮殿掖門戶,有丞。武帝太初元年更名光祿勳。屬官有大夫、郎、謁者,皆秦官。又期門、羽林皆屬焉。大夫掌論議,有太中大夫、中大夫、諫大夫,皆無員,多至數十人。武帝元狩五年初置諫大夫,秩比八百石,太初元年更名中大夫為光祿大夫,秩比二千石,太中大夫秩比千石如故。郎掌守門戶,出充車騎,有議郎、中郎、侍郎、郎中,皆無員,多至千人。議郎、中郎秩比六百石,侍郎比四百石,郎中比三百石。中郎有五官、左、右三將,秩皆比二千石。郎中有車、戶、騎三將,秩皆比千石。謁者掌賓讚受事,員七十人,秩比六百石,有僕射,秩比千石。期門掌執兵送從,武帝建元三年初置,比郎,無員,多至千人,有僕射,秩比千石。平帝元始元年更名虎賁郎,置中郎將,秩比二千石。羽林掌送從,次期門,武帝太初元年初置,名曰建章營騎,後更名羽林騎。又取從軍死事之子孫養羽林,官教以五兵,號曰羽林孤兒。羽林有令丞。宣帝令中郎將、騎都尉監羽林,秩比二千石。僕射,秦官,自侍中、尚書、博士、郎皆有。古者重武官,有主射以督課之,軍屯吏、騶、宰、永巷宮人皆有,取其領事之號。

衛尉,秦官,掌宮門衛屯兵,有丞。景帝初更名中大夫令,後元年復為衛尉。屬官有公車司馬、衛士、旅賁三令丞。衛士三丞。又諸屯衛候、司馬二十二官皆屬焉。長樂、建章、甘泉衛尉皆掌其宮,職略同,不常置。

太僕,秦官,掌輿馬,有兩丞。屬官有大廄、未央、家馬三令,各五丞一尉。又車府、路軨、騎馬、駿馬四令丞;又龍馬、閑駒、橐泉、騊駼、承華五監長丞;又邊郡六牧師菀令,各三丞;又牧橐、昆蹄令丞皆屬焉。中太僕掌皇太后輿馬,不常置也。武帝太初元年更名家馬為挏馬,初置路軨。

廷尉,秦官,掌刑辟,有正、左右監,秩皆千石。景帝中六年更名大理,武帝建元四年復為廷尉。宣帝地節三年初置左右平,秩皆六百石。哀帝元壽二年復為大理。王莽改曰作士。

典客,秦官,掌諸歸義蠻夷,有丞。景帝中六年更名大行令,武帝太初元年更名大鴻臚。屬官有行人、譯官、別火三令丞及郡邸長丞。武帝太初元年更名行人為大行令,初置別火。王莽改大鴻臚曰典樂。初,置郡國邸屬少府,中屬中尉,後屬大鴻臚。

宗正,秦官,掌親屬,有丞。平帝元始四年更名宗伯。屬官有都司空令丞,內官長丞。又諸公主家令、門尉皆屬焉。王莽并其官於秩宗。初,內官屬少府,中屬主爵,後屬宗正。

治粟內史,秦官,掌穀貨,有兩丞。景帝後元年更名大農令,武帝太初元年更名大司農。屬官有太倉、均輸、平準、都內、籍田五令丞,斡官、鐵市兩長丞。又郡國諸倉農監、都水六十五官長丞皆屬焉。騪粟都尉,武帝軍官,不常置。王莽改大司農曰羲和,後更為納言。初,斡官屬少府,中屬主爵,後屬大司農。

少府,秦官,掌山海池澤之稅,以給共養,有六丞。屬官有尚書、符節、太醫、太官、湯官、導官、樂府、若盧、考工室、左弋、居室、甘泉居室、左右司空、東織、西織、東園匠十二官令丞,又胞人、都水、均官三長丞,又上林中十池監,又中書謁者、黃門、鉤盾、尚方、御府、永巷、內者、宦者七官令丞。諸僕射、署長、中黃門皆屬焉。武帝太初元年更名考工室為考工,左弋為佽飛,居室為保宮,甘泉居室為昆臺,永巷為掖廷。佽飛掌弋射,有九丞兩尉,太官七丞,昆臺五丞,樂府三丞,掖廷八丞,宦者七丞,鉤盾五丞兩尉。成帝建始四年更名中書謁者令為中謁者令,初置尚書,員五人,有四丞。河平元年省東織,更名西織為織室。綏和二年,哀帝省樂府。王莽改少府曰共工。

中尉,秦官,掌徼循京師,有兩丞、候、司馬、千人。武帝太初元年更名執金吾。屬官有中壘、寺互、武庫、都船四令丞。都船、武庫有三丞,中壘兩尉。又式道左右中候、候丞及左右京輔都尉、尉丞兵卒皆屬焉。初,寺互屬少府,中屬主爵,後屬中尉。

自太常至執金吾,秩皆中二千石,丞皆千石。

太子太傅、少傅,古官。屬官有太子門大夫、庶子、先馬、舍人。

將作少府,秦官,掌治宮室,有兩丞、左右中候。景帝中六年更名將作大匠。屬官有石庫、東園主章、左右前後中校七令丞,又主章長丞。武帝太初元年更名東園主章為木工。成帝陽朔三年省中候及左右前後中校五丞。

詹事,秦官,掌皇后、太子家,有丞。屬官有太子率更、家令丞,僕、中盾、衛率、廚廄長丞,又中長秋、私府、永巷、倉、廄、祠祀、食官令長丞。諸宦官皆屬焉。成帝鴻嘉三年省詹事官,并屬大長秋。長信詹事掌皇太后宮,景帝中六年更名長信少府,平帝元始四年更名長樂少府。

將行,秦官,景帝中六年更名大長秋,或用中人,或用士人。

典屬國,秦官,掌蠻夷降者。武帝元狩三年昆邪王降,復增屬國,置都尉、丞、候、千人。屬官,九譯令。成帝河平元年省并大鴻臚。

水衡都尉,武帝元鼎二年初置,掌上林苑,有五丞。屬官有上林、均輸、御羞、禁圃、輯濯、鍾官、技巧、六廄、辯銅九官令丞。又衡官、水司空、都水、農倉,又甘泉上林、都水七官長丞皆屬焉。上林有八丞十二尉,均輸四丞,御羞兩丞,都水三丞,禁圃兩尉,甘泉上林四丞。成帝建始二年省技巧、六廄官。王莽改水衡都尉曰予虞。初,御羞、上林、衡官及鑄錢皆屬少府。

內史,周官,秦因之,掌治京師。景帝二年分置左內史。右內史武帝太初元年更名京兆尹,屬官有長安市、廚兩令丞,又都水、鐵官兩長丞。左內史更名左馮翊,屬官有廩犧令丞尉。又左都水、鐵官、雲壘、長安四市四長丞皆屬焉。

主爵中尉,秦官,掌列侯。景帝中六年更名都尉,武帝太初元年更名右扶風,治內史右地。屬官有掌畜令丞。又有都水、鐵官、廄、廱廚四長丞皆屬焉。與左馮翊、京兆尹是為三輔,皆有兩丞。列侯更屬大鴻臚。元鼎四年更置二輔都尉、都尉丞各一人。

自太子太傅至右扶風,皆秩二千石,丞六百石。

護軍都尉,秦官,武帝元狩四年屬大司馬,成帝綏和元年居大司馬府比司直,哀帝元壽元年更名司寇,平帝元始元年更名護軍。

司隸校尉,周官,武帝征和四年初置。持節,從中都官徒千二百人,捕巫蠱,督大姦猾。後罷其兵。察三輔、三河、弘農。元帝初元四年去節。成帝元延四年省。綏和二年,哀帝復置,但為司隸,冠進賢冠,屬大司空,比司直。

城門校尉掌京師城門屯兵,有司馬、十二城門候。中壘校尉掌北軍壘門內,外掌西域。屯騎校尉掌騎士。步兵校尉掌上林苑門屯兵。越騎校尉掌越騎。長水校尉掌長水宣曲胡騎。又有胡騎校尉,掌池陽胡騎,不常置。射聲校尉掌待詔射聲士。虎賁校尉掌輕車。凡八校尉,皆武帝初置,有丞、司馬。自司隸至虎賁校尉,秩皆二千石。西域都護加官,宣帝地節二年初置,以騎都尉、諫大夫使護西域三十六國,有副校尉,秩比二千石,丞一人,司馬、候、千人各二人。戊己校尉,元帝初元元年置,有丞、司馬各一人,候五人,秩比六百石。

奉軍都尉掌御乘輿車,駙馬都尉掌駙馬,皆武帝初置,秩比二千石。侍中、左右曹、諸吏、散騎、中常侍,皆加官,所加或列侯、將軍、卿大夫、將、都尉、尚書、太醫、太官令至郎中,亡員,多至數十人。侍中、中常侍得入禁中,諸曹受尚書事,諸吏得舉法,散騎騎並乘輿車。給事中亦加官,所加或大夫、博士、議郎,掌顧問應對,位次中常侍。中黃門有給事黃門,位從將大夫。皆秦制。

爵:一級曰公士,二上造,三簪裊,四不更,五大夫,六官大夫,七公大夫,八公乘,九五大夫,十左庶長,十一右庶長,十二左更,十三中更,十四右更,十五少上造,十六大上造,十七駟車庶長,十八大庶長,十九關內侯,二十徹侯。皆秦制,以賞功勞。徹侯金印紫綬,避武帝諱,曰通侯,或曰列侯,改所食國令長名相,又有家丞、門大夫、庶子。

諸侯王,高帝初置,金璽盭綬,掌治其國。有太傅輔王,內史治國民,中尉掌武職,丞相統眾官,群卿大夫都官如漢朝。景帝中五年令諸侯王不得復治國,天子為置吏,改丞相曰相,省御史大夫、廷尉、少府、宗正、博士官,大夫、謁者、郎諸官長丞皆損其員。武帝改漢內史為京兆尹,中尉為執金吾,郎中令為光祿勳,故王國如故。損其郎中令,秩千石;改太僕曰僕,秩亦千石。成帝綏和元年省內史,更令相治民,如郡太守,中尉如郡都尉。

監御史,秦官,掌監郡。漢省,丞相遣史分刺州,不常置。武帝元封五年初置部刺史,掌奉詔條察州,秩六百石,員十三人。成帝綏和元年更名牧,秩二千石。哀帝建平二年復為刺史,元壽二年復為牧。

郡守,秦官,掌治其郡,秩二千石。有丞,邊郡又有長史,掌兵馬,秩皆六百石。景帝中二年更名太守。

郡尉,秦官,掌佐守典武職甲卒,秩比二千石。有丞,秩皆六百石。景帝中二年更名都尉。

關都尉,秦官。農都尉、屬國都尉,皆武帝初置。

縣令、長,皆秦官,掌治其縣。萬戶以上為令,秩千石至六百石。減萬戶為長,秩五百石至三百石。皆有丞、尉,秩四百石至二百石,是為長吏。百石以下有斗食、佐史之秩,是為少吏。大率十里一亭,亭有長。十亭一鄉,鄉有三老、有秩、嗇夫、游徼。三老掌教化。嗇夫職聽訟,收賦稅。游徼徼循禁賊盜。縣大率方百里,其民稠則減,稀則曠,鄉、亭亦如之,皆秦制也。列侯所食縣曰國,皇太后、皇后、公主所食曰邑,有蠻夷曰道。凡縣、道、國、邑千五百八十七,鄉六千六百二十二,亭二萬九千六百三十五。

凡吏秩比二千石以上,皆銀印青綬,光祿大夫無。秩比六百石以上,皆銅印黑綬,大夫、博士、御史、謁者、郎無。其僕射、御史治書尚符璽者,有印綬。比二百石以上,皆銅印黃綬。成帝陽朔二年除八百石、五百石秩。綏和元年,長、相皆黑綬。哀帝建平二年,復黃綬。吏員自佐史至丞相,十二萬二百八十五人。

公元前相國左內史


\end{pinyinscope}