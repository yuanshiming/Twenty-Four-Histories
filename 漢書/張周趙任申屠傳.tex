\article{張周趙任申屠傳}

\begin{pinyinscope}
張蒼,陽武人也,好書律曆。秦時為御史,主柱下方書。有罪,亡歸。及沛公略地過陽武,蒼以客從攻南陽。蒼當斬,解衣伏質,身長大,肥白如瓠,時王陵見而怪其美士,乃言沛公,赦勿斬。遂西入武關,至咸陽。

沛公立為漢王,入漢中,還定三秦。陳餘擊走常山王張耳,耳歸漢,漢以蒼為常山守。從韓信擊趙,蒼得陳餘。趙地已平,漢王以蒼為代相,備邊寇。已而徙為趙相,相趙王耳。耳卒,相其子敖。復徙相代。燕王臧荼反,蒼以代相從攻荼有功,六年封為北平侯,食邑千二百戶。

遷為計相,一月,更以列侯為主計四歲。是時蕭何為相國,而蒼乃自秦時為柱下御史,明習天下圖書計籍,又善用算律曆,故令蒼以列侯居相府,領主郡國上計者。黥布反,漢立皇子長為淮南王,而蒼相之。十四年,遷為御史大夫。

周昌者,沛人也。其從兄苛,秦時皆為泗水卒史。及高祖沛起,擊破泗水守監,於是苛、昌自卒史從沛公,沛公以昌為職志,苛為客。從入關破秦。沛公立為漢王,以苛為御史大夫,昌為中尉。

漢三年,楚圍漢王滎陽急,漢王出去,而使苛守滎陽城。楚破滎陽城,欲令苛將,苛罵曰:「若趣降漢王!不然,今為慮矣!」項羽怒,亨苛。漢王於是拜昌為御史大夫。常從擊破項籍。六年,與蕭、曹等俱封,為汾陰侯。苛子成以父死事,封為高景侯。

昌為人強力,敢直言,自蕭、曹等皆卑下之。昌嘗燕入奏事,高帝方擁戚姬,昌還走。高帝逐得,騎昌項,上問曰:「我何如主也?」昌仰曰:「陛下即桀紂之主也。」於是上笑之,然尤憚昌。及高帝欲廢太子,而立戚姬子如意為太子,大臣固爭莫能得,上以留侯策止。而昌庭爭之強,上問其說,昌為人吃,又盛怒,曰:「臣口不能言,然臣心知其其不可。陛下欲廢太子,臣期期不奉詔。」上欣然而笑,即罷。呂后側耳於東箱聽,見昌,為跪謝曰:「微君,太子幾廢。」

是歲,戚姬子如意為趙王,年十歲,高祖憂萬歲之後不全也。趙堯為符璽御史,趙人方與公謂御史大夫周昌曰:「君之史趙堯,年雖少,然奇士,君必異之,是且代君之位。」昌笑曰:「堯年少,刀筆吏耳,何至是乎!」居頃之,堯侍高祖,高祖獨心不樂,悲歌,群臣不知上所以然。堯進請間曰:「陛下所為不樂,非以趙王年少,而戚夫人與呂后有隙,備萬歲之後而趙王不能自全乎?」高祖曰:「我私憂之,不知所出。」堯曰:「陛下獨為趙王置貴彊相,及呂后、太子、群臣素所敬憚者乃可。」高祖曰:「然。吾念之欲如是,而群臣誰可者?」堯曰:「御史大夫昌,其人堅忍伉直,自呂后、太子及大臣皆素嚴憚之。獨昌可。」高祖曰:「善。」於是召昌謂曰:「吾固欲煩公,公彊為我相趙。」昌泣曰:「臣初起從陛下,陛下獨奈何中道而棄之於諸侯乎?」高祖曰:「吾極知其左遷,然吾私憂趙,念非公無可者。公不得已強行!」於是徙御史大夫昌為趙相。

既行久之,高祖持御史大夫印弄之,曰:「誰可以為御史大夫者?」孰視堯曰:「無以易堯。」遂拜堯為御史大夫。堯亦前有軍功食邑,及以御史大夫從擊陳豨有功,封為江邑侯。

高祖崩,太后使使召趙王,其相昌令王稱疾不行。使者三反,昌曰:「高帝屬臣趙王,王年少,竊聞太后怨戚夫人,欲召趙王并誅之。臣不敢遣王,王且亦疾,不能奉詔。」太后怒,乃使使召趙相。相至,謁太后,太后罵昌曰:「爾不知我之怨戚氏乎?而不遣趙王!」昌既被徵,高后使使召趙王。王果來,至長安月餘,見鴆殺。昌謝病不朝見,三歲而薨,諡曰悼侯。傳子至孫意,有罪,國除。景帝復封昌孫左車為安陽侯,有罪,國除。

初,趙堯既代周昌為御史大夫,高祖崩,事惠帝終世。高后元年,怨堯前定趙王如意之畫,乃抵堯罪,以廣阿侯任敖為御史大夫。

任敖,沛人也,少為獄吏。高祖嘗避吏,吏繫呂后,遇之不謹。任敖素善高祖,怒,擊傷主呂后吏。及高祖初起,敖以客從為御史,守豐二歲。高祖立為漢王,東擊項羽,敖遷為上黨守。陳豨反,敖堅守,封為廣阿侯,食邑千八百戶。高后時為御史大夫,三歲免。孝文元年薨,諡曰懿侯。傳子至曾孫越人,坐為太常廟酒酸不敬,國除。

初任敖免,平陽侯曹窋代敖為御史大夫。高后崩,與大臣共誅諸呂。後坐事免,以淮南相張蒼為御史大夫。蒼與絳侯等尊立孝文皇帝,四年,代灌嬰為丞相。

漢興二十餘年,天下初定,公卿皆軍吏。蒼為計相時,緒正律曆。以高祖十月始至霸上,故因秦時本十月為歲首,不革。推五德之運,以為漢當水德之時,上黑如故。吹律調樂,入之音聲,及以比定律令。若百工,天下作程品。至於為丞相,卒就之。故漢家言律曆者本張蒼。蒼尤好書,無所不觀,無所不通,而尤邃律曆。

蒼德安國侯王陵,及貴,父事陵。陵死後,蒼為丞相,洗沐,常先朝陵夫人上食,然後敢歸家。

蒼為丞相十餘年,魯人公孫臣上書,陳終始五德傳,言漢土德時,其符黃龍見,當改正朔,易服色。事下蒼,蒼以為非是,罷之。其後黃龍見成紀,於是文帝召公孫臣以為博士,草立土德時曆制度,更元年。蒼由此自絀,謝病稱老。蒼任人為中候,大為姦利,上以為讓,蒼遂病免。孝景五年薨,諡曰文侯。傳子至孫類,有罪,國除。

初蒼父長不滿五尺,蒼長八尺餘,蒼子復長八尺,及孫類長六尺餘。蒼免相後,口中無齒,食乳,女子為乳母。妻妾以百數,嘗孕者不復幸。年百餘歲乃卒。著書十八篇,言陰陽律曆事。

申屠嘉,梁人也。以材官蹶張從高帝擊項籍,遷為隊率。從擊黥布,為都尉。孝惠時,為淮陽守。孝文元年,舉故以二千石從高祖者,悉以為關內侯,食邑二十四人,而嘉食邑五百戶。十六年,遷為御史大夫。張蒼免相,文帝以皇后弟竇廣國賢有行,欲相之,曰:「恐天下以吾私廣國。」久念不可,而高帝時大臣餘見無可者,乃以御史大夫嘉為丞相,因故邑封為故安侯。

嘉為人廉直,門不受私謁。是時太中大夫鄧通方愛幸,賞賜累鉅萬。文帝常燕飲通家,其見寵如是。是時嘉入朝,而通居上旁,有怠慢之禮,嘉奏事畢,因言曰:「陛下幸愛群臣則富貴之,至於朝廷之禮,不可以不肅!」上曰:「君勿言,吾私之。」罷朝坐府中,嘉為檄召通詣丞相府,不來,且斬通。通恐,入言上。上曰:「汝弟往,吾今使人召若。」通至詣丞相府,免冠,徒跣,頓首謝嘉。嘉坐自如,弗為禮,責曰:「夫朝廷者,高皇帝之朝廷也,通小臣,戲殿上,大不敬,當斬。史今行斬之!」通頓首,首盡出血,不解。上度丞相已困通,使使持節召通,而謝丞相:「此吾弄臣,君釋之。」鄧通既至,為上泣曰:「丞相幾殺臣。」

嘉為丞相五歲,文帝崩,孝景即位。二年,晁錯為內史,貴幸用事,諸法令多所請變更,議以適罰侵削諸侯。而丞相嘉自絀,所言不用,疾錯。錯為內史,門東出,不便,更穿一門,南出。南出者,太上皇廟堧垣也。嘉聞錯穿宗廟垣,為奏請誅錯。客有語錯,錯恐,夜入宮上謁,自歸上。至朝,嘉請誅內史錯。上曰:「錯所穿非真廟垣,乃外堧垣,故冗官居其中,且又我使為之,錯無罪。」罷朝,嘉謂長史曰:「吾悔不先斬錯乃請之,為錯所賣。」至舍,因歐血而死。諡曰節侯。傳子至孫臾,有罪,國除。

自嘉死後,開封侯陶青、桃侯劉舍及武帝時柏至侯許昌、平棘侯薛澤、武彊侯莊青翟、商陵侯趙周,皆以列侯繼踵,愦愦廉謹,為丞相備員而已,無所能發明功名著於世者。

贊曰:張蒼文好律曆,為漢名相,而專遵用秦之顓頊曆,何哉?周昌,木強人也。任敖以舊德用。申屠嘉可謂剛毅守節,然無術學,殆與蕭、曹、陳平異矣。


\end{pinyinscope}