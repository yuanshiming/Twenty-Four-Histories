\article{賈誼傳}

\begin{pinyinscope}
賈誼,雒陽人也,年十八,以能誦詩書屬文稱於郡中。河南守吳公聞其秀材,召置門下,甚幸愛。文帝初立,聞河南守吳公治平為天下第一,故與李斯同邑,而嘗學事焉,徵以為廷尉。廷尉乃言誼年少,頗通諸家之書。文帝召以為博士。

是諸,誼年二十餘,最為少。每詔令議下,諸老先生未能言,誼盡為之對,人人各如其意所出。諸生於是以為能。文帝說之,超遷,歲中至太中大夫。

誼以為漢興二十餘年,天下和洽,宜當改正朔,易服色制度,定官名,興禮樂。乃草具其儀法,色上黃,數用五,為官名悉更,奏之。文帝謙讓未皇也。然諸法令所更定,及列侯就國,其說皆誼發之。於是天子議以誼任公卿之位。絳、灌、東陽侯、馮敬之屬盡害之,乃毀誼曰:「雒陽之人年少初學,專欲擅權,紛亂諸事。」於是天子後亦疏之,不用其議,以誼為長沙王太傅。

誼既以適去,意不自得,及度湘水,為賦以弔屈原。屈原,楚賢臣也,被讒放逐,作離騷賦,其終篇曰:「已矣!國亡人,莫我知也。」遂自投江而死。誼追傷之,因以自諭。其辭曰:

恭承嘉惠兮,俟罪長沙。仄聞屈原兮,自湛汨羅。造託湘流兮,敬弔先生。遭世罔極兮,迺隕厥身。烏虖哀哉兮,逢時不祥!鸞鳳伏竄兮,鴟鴞翱翔。闒茸尊顯兮,讒諛得志;賢聖逆曳兮,方正倒植。謂隨、夷溷兮,謂跖、蹻廉;莫邪為鈍兮,鈆刀為銛。于嗟默默,生之亡故兮!斡棄周鼎,寶康瓠兮。騰駕罷牛,驂蹇驢兮;驥垂兩耳,服鹽車兮。章父薦屨,漸不可久兮;嗟若先生,獨離此咎兮!

誶曰:已矣!國其莫吾知兮,子獨壹鬱其誰語?鳳縹縹其高逝兮,夫固自引而遠去。襲九淵之神龍兮,沕淵潛以自珍;偭蟂獺以隱處兮,夫豈從蝦與蛭螾?所貴聖之神德兮,遠濁世而自臧。使麒麟可係而羈兮,豈云異夫犬羊?般紛紛其離此郵兮,亦夫子之故也!歷九州而相其君兮,何必懷此都也?鳳皇翔于千仞兮,覽德煇而下之;見細德之險微兮,遙增擊而去之。彼尋常之汙瀆兮,豈容吞舟之魚!橫江湖之鱣鯨兮,固將制於螻螘。

誼為長沙傅三年,有服飛入誼舍,止於坐隅。服似鴞,不祥鳥也。誼既以適居長沙,長沙卑濕,誼自傷悼,以為壽不得長,乃為賦以自廣。其辭曰:

單閼之歲,四月孟夏,庚子日斜,服集余舍,止于坐隅,貌甚閒暇。異物來聂,私怪其故,發書占之,讖言其度。曰「野鳥入室,主人將去。」問于子服:「余去何之?吉虖告我,凶言其災。淹速之度,語余其期。」

服乃太息,舉首奮翼,口不能言,請對以意。萬物變化,固亡休息。斡流而遷,或推而還。形氣轉續,變化而嬗。沕穆亡間,胡可勝言!禍兮福所倚,福兮禍所伏;憂喜聚門,吉凶同域。彼吳彊大,夫差以敗;粵棲會稽,句踐伯世。斯遊遂成,卒被五刑;傅說胥靡,乃相武丁。夫禍之與福,何異糾纆!命不可說,孰知其極?水激則旱,矢激則遠。萬物回薄,震蕩相轉。雲烝雨降,糾錯相紛。大鈞播物,坱圠無垠。天不可與慮,道不可與謀。遲速有命,烏識其時?

且夫天地為鑪,造化為工;陰陽為炭,萬物為銅,合散消息,安有常則?千變萬化,未始有極。忽然為人,何足控揣;化為異物,又何足患!小智自私,賤彼貴我;達人大觀,物亡不可。貪夫徇財,列士徇名;夸者死權,品庶每生。怵迫之徒,或趨西東;大人不曲,意變齊同。愚士繫俗,時若囚拘;至人遺物,獨與道俱。眾人惑惑,好惡積意;真人恬漠,獨與道息。釋智遺形,超然自喪;寥廓忽荒,與道翱翔。乘流則逝,得坎則止;縱軀委命,不私與己。其生兮若浮,其死兮若休。澹虖若深淵之靚,氾虖若不繫之舟。不以生故自保,養空而浮。德人無累,知命不憂。細故蔕芥,何足以疑!

後歲餘,文帝思誼,徵之。至,入見,上方受釐,坐宣室。上因感鬼神事,而問鬼神之本。誼具道所以然之故。至夜半,文帝前席。既罷,曰:「吾久不見賈生,自以為過之,今不及也。」乃拜誼為梁懷王太傅。懷王,上少子,愛,而好書,故令誼傅之,數問以得失。

是時,匈奴彊,侵邊。天下初定,制度疏闊。諸侯王僭儗,地過古制,淮南、濟北王皆為逆誅。誼數上疏陳政事,多所欲匡建,其大略曰:

臣竊惟事勢,可為痛哭者一,可為流涕者二,可為長太息者六,若其它背理而傷道者,難遍以疏舉。進言者皆曰天下已安已治矣,臣獨以為未也。曰安且治者,非愚則諛,皆非事實知治亂之體者也。夫抱火厝之積薪之下而寢其上,火未及燃,因謂之安,方今之勢,何以異此!本末舛逆,首尾衡決,國制搶攘,非甚有紀,胡可謂治!陛下何不壹令臣得孰數之於前,因陳治安之策,試詳擇焉!

夫射獵之娛,與安危之機孰急?使為治,勞智慮,苦身體,乏鍾鼓之樂,勿為可也。樂與今同,而加之諸侯軌道,兵革不動,民保首領,匈奴賓服,四荒鄉風,百姓素朴,獄訟衰息,大數既得,則天下順治,海內之氣清和咸理,生為明帝,沒為明神,名譽之美,垂於無窮。禮祖有功而宗有德,使顧成之廟稱為太宗,上配太祖,與漢亡極。建久安之勢,成長治之業,以承祖廟,以奉六親,至孝也;以幸天下,以育群生,至仁也;立經陳紀,輕重同得,後可以為萬世法程,雖有愚幼不肖之嗣,猶得蒙業而安,至明也。以陛下之明達,因使少知治體者得佐下風,致此非難也。其具可素陳於前,願幸無忽。臣謹稽之天地,驗之往古,按之當今之務,日夜念此至孰也,雖使禹舜復生,為陛下計,亡以易此。

夫樹國固必相疑之勢,下數被其殃,上數爽其憂,甚非所以安上而全下也。今或親弟謀為東帝,親兄之子西鄉而擊,今吳又見告矣。天子春秋鼎盛,行義未過,德澤有加焉,猶尚如是,況莫大諸侯,權力且十此者虖!

然而天下少安,何也?大國之王幼弱未壯,漢之所置傅相方握其事。數年之後,諸侯之王大抵皆冠,血氣方剛,漢之傅相稱病而賜罷,彼自丞尉以上偏置私人,如此,有異淮南、濟北之為邪!此時而欲為治安,雖堯舜不治。

黃帝曰:「日中必熭,操刀必割。」今令此道順而全安,甚易,不肯早為,已乃墮骨肉之屬而抗剄之,豈有異秦之季世虖!夫以天子之位,乘今之時,因天之助,尚憚以危為安,以亂為治,假設陛下居齊桓之處,將不合諸侯而匡天下乎?臣又以知陛下有所必不能矣。假設天下如曩時,淮陰侯尚王楚,黥布王淮南,彭越王梁,韓信王韓,張敖王趙,貫高為相,盧綰王燕,陳豨在代,令此六七公者皆亡恙,當是時而陛下即天子位,能自安乎?臣有以知陛下之不能也。天下殽亂,高皇帝與諸公併起,非有仄室之勢以豫席之也。諸公幸者,乃為中涓,其次廑得舍人,材之不逮至遠也。高皇帝以明聖威武即天子位,割膏腴之地以王諸公,多者百餘城,少者乃三四十縣,德至渥也,然其後十年之間,反者九起。陛下之與諸公,非親角材而臣之也,又非身封王之也,自高皇帝不能以是一歲為安,故臣知陛下之不能也。然尚有可諉者,曰疏,臣請試言其親者。假令悼惠王王齊,元王王楚,中子王趙,幽王王淮陽,共王王梁,靈王王燕,厲王王淮南,六七貴人皆亡恙,當是時陛下即位,能為治虖?臣又知陛下之不能也。若此諸王,雖名為臣,實皆有布衣昆弟之心,慮亡不帝制而天子自為者。擅爵人,赦死罪,甚者或戴黃屋,漢法令非行也。雖行不軌如厲王者,令之不肯聽,召之安可致乎!幸而來至,法安可得加!動一親戚,天下圜視而起,陛下之臣雖有悍如馮敬者,適啟其口,匕首已陷其匈矣。陛下雖賢,誰與領此?故疏者必危,親者必亂,已然之效也。其異姓負彊而動者,漢已幸勝之矣,又不易其所以然。同姓襲是跡而動,既有徵矣,其勢盡又復然。殃禍之變,未知所移,明帝處之尚不能以安,後世將如之何!

屠牛坦一朝解十二牛,而芒刃不頓者,所排擊剝割,皆眾理解也。至於髖髀之所,非斤則斧。夫仁義恩厚,人主之芒刃也;權勢法制,人主之斤斧也。今諸侯王皆眾髖髀也,釋斤斧之用,而欲嬰以芒刃,臣以為不缺則折。胡不用之淮南、濟北?勢不可也。

臣竊跡前事,大抵彊者先反。淮陰王楚最彊,則最先反;韓信倚胡,則又反;貫高因趙資,則又反;陳豨兵精,則又反;彭越用梁,則又反;黥布用淮南,則又反;盧綰最弱,最後反。長沙乃在二萬五千戶耳,功少而最完,勢疏而最忠,非獨性異人也,亦形勢然也。曩令樊、酈、絳、灌據數十城而王,今雖以殘亡可也;令信、越之倫列為徹侯而居,雖至今存可也。然則天下之大計可知已。欲諸王之皆忠附,則莫若令如長沙王;欲臣子之勿菹醢,則莫若令如樊、酈等;欲天下之治安,莫若眾建諸侯而少其力。力少則易使以義,國小則亡邪心。令海內之勢如身之使臂,臂之使指,莫不制從,諸侯之君不敢有異心,輻湊並進而歸命天子,雖在細民,且知其安,故天下咸知陛下之明。割地定制,令齊、趙、楚各為若干國,使悼惠王、幽王、元王之子孫畢以次各受祖之分地,地盡而止,及燕、梁它國皆然。其分地眾而子孫少者,建以為國,空而置之,須其子孫生者,舉使君之。諸侯之地其削頗入漢者,為徙其侯國及封其子孫也,所以數償之:一寸之地,一人之眾,天子亡所利焉,誠以定治而已,故天下咸知陛下之廉。地制壹定,宗室子孫莫慮不王,下無倍畔之心,上無誅伐之志,故天下咸知陛下之仁。法立而不犯,令行而不逆,貫高、利幾之謀不生,柴奇、開章之計不萌,細民鄉善,大臣致順,故天下咸知陛下之義。臥赤子天下之上而安,植遺腹,朝委裘,而天下不亂,當時大治,後世誦聖。壹動而五業附,陛下誰憚而久不為此?

天下之勢方病大拶。一脛之大幾如要,一指之大幾如股,平居不可屈信,一二指搐,身慮亡聊。失今不治,必為錮疾,後雖有扁鵲,不能為已。病非徒拶也,又苦蹠盭。元王之子,帝之從弟也;今之王者,從弟之子也。惠王,親兄子也;今之王者,兄子之子也。親者或亡分地以安天下,疏者或制大權以偪天子,臣故曰非徒病拶也,又苦蹠盭。可痛哭者,此病是也。

天下之勢方倒縣。凡天子者,天下之首,何也?上也。蠻夷者,天下之足,何也?下也。今匈奴嫚厉侵掠,至不敬也,為天下患,至亡已也,而漢歲致金絮采繒以奉之。夷狄徵令,是主上之操也;天子共貢,是臣下之禮也。足反居上,首顧居下,倒縣如此,莫之能解,猶為國有人乎?非亶倒縣而已,又類辟,且病痱。夫辟者一面病,痱者一方痛。今西邊北邊之郡,雖有長爵不輕得復,五尺以上不輕得息,斥候望烽燧不得臥,將吏被介冑而睡,臣故曰一方病矣。醫能治之,而上不使,可為流涕者此也。

陛下何忍以帝皇之號為戎人諸侯,勢既卑辱,而禍不息,長此安窮!進謀者率以為是,固不可解也,亡具甚矣。臣竊料匈奴之眾不過漢一大縣,以天下之大困於一縣之眾,甚為執事者羞之。陛下何不試以臣為屬國之官以主匈奴?行臣之計,請必係單于之頸而制其命,伏中行說而笞其背,舉匈奴之眾唯上之令。今不獵猛敵而獵田彘,不搏反寇而搏畜菟,翫細娛而不圖大患,非所以為安也。德可遠施,威可遠加,而直數百里外威令不信,可為流涕者此也。

今民賣僮者,為之繡衣絲履偏諸緣,內之閑中,是古天子后服,所以廟而不晏者也,而庶人得以衣婢妾。白縠之表,薄紈之裏,緁以偏諸,美者黼繡,是古天子之服,今富人大賈嘉會召客者以被牆。古者以奉一帝一后而節適,今庶人屋壁得為帝服,倡優下賤得為后飾,然而天下不屈者,殆未有也。且帝之身自衣皁綈,而富民牆屋被文繡;天子之后以緣其領,庶人键妾緣其履:此臣所謂舛也。夫百人作之不能衣一人,欲天下亡寒,胡可得也?一人耕之,十人聚而食之,欲天下亡飢,不可得也。飢寒切於民之肌膚,欲其亡為姦邪,不可得也。國已屈矣,盜賊直須時耳,然而獻計者曰「毋動」,為大耳。夫俗至大不敬也,至亡等也,至冒上也,進計者猶曰「毋為」,可為長太息者此也。

商君遺禮義,棄仁恩,并心於進取,行之二歲,秦俗日敗。故秦人家富子壯則出分,家貧子壯則出贅。借父耰鉏,慮有德色;母取箕嶹,立而誶語。抱哺其子,與公併倨;婦姑不相說,則反脣而相稽。其慈子耆利,不同禽獸者亡幾耳。然并心而赴時,猶曰蹶六國,兼天下。功成求得矣,終不知反廉愧之節,仁義之厚。信并兼之法,遂進取之業,天下大敗;眾掩寡,智欺愚,勇威怯,壯陵衰,其亂至矣。是以大賢起之,威震海內,德從天下。曩之為秦者,今轉而為漢矣。然其遺風餘俗,猶尚未改。今世以侈靡相競,而上亡制度,棄禮誼,捐廉恥,日甚,可謂月異而歲不同矣。逐利不耳,慮非顧行也,今其甚者殺父兄矣。盜者剟寢戶之簾,搴兩廟之器,白晝大都之中剽吏而奪之金。矯偽者出幾十萬石粟,賦六百餘萬錢,乘傳而行郡國,此其亡行義之先至者也。而大臣特以簿書不報,期會之間,以為大故。至於俗流失,世壞敗,因恬而不知怪,慮不動於耳目,以為是適然耳。夫移風易俗,使天下回心而鄉道,類非俗吏之所能為也。俗吏之所務,在於刀筆筐篋,而不知大禮。陛下又不自憂,竊為陛下惜之。

夫立君臣,等上下,使父子有禮,六親有紀,此非天之所為,人之所設也。夫人之所設,不為不立,不植則僵,不修則壞。筦子曰:「禮義廉恥,是謂四維;四維不張,國乃滅亡。」使筦子愚人也則可,筦子而少知治體,則是豈可不為寒心哉!秦滅四維而不張,故君臣乖亂,六親殃戮,姦人並起,萬民離叛,凡十三歲,社稷為虛。今四維猶未備也,故姦人幾幸,而眾心疑惑。豈如今定經制,令君君臣臣,上下有差,父子六親各得其宜,姦人亡所幾幸,而群臣眾信,上不疑惑!此業壹定,世世常安,而後有所持循矣。若夫經制不定,是猶度江河亡維楫,中流而遇風波,船必覆矣。可為長太息者此也。

夏為天子,十有餘世,而殷受之。殷為天子,二十餘世,而周受之。周為天子,三十餘世,而秦受之。秦為天子,二世而亡。人性不甚相遠也,何三代之君有道之長,而秦無道之暴也?其故可知也。古之王者,太子乃生,固舉以禮,使士負之,有司齊肅端冕,見之南郊,見于天也。過闕則下,過廟則趨,孝子之道也。故自為赤子而教固已行矣。昔者成王幼在繈抱之中,召公為太保,周公為太傅,太公為太師。保,保其身體;傅,傅之德意;師,道之教訓:此三公之職也。於是為置三少,皆上大夫也,曰少保、少傅、少師,是與太子宴者也。故乃孩提有識,三公、三少固明孝仁禮義以道習之,逐去邪人,不使見惡行。於是皆選天下之端士孝悌博聞有道術者以衛翼之,使與太子居處出入。故太子乃生而見正事,聞正言,行正道,左右前後皆正人也。夫習與正人居之,不能毋正,猶生長於齊不能不齊言也;習與不正人居之,不能毋不正,猶生長於楚之地不能不楚言也。故擇其所耆,必先受業,乃得嘗之;擇其所樂,必先有習,乃得為之。孔子曰:「少成若天性,習貫如自然。」及太子少長,知妃色,則入于學。學者,所學之官也。學禮曰:「帝入東學,上親而貴仁,則親疏有序而恩相及矣;帝入南學,上齒而貴信,則長幼有差而民不誣矣;帝入西學,上賢而貴德,則聖智在位而功不遺矣;帝入北學,上貴而尊爵,則貴賤有等而下不隃矣;帝入太學,承師問道,退習而考於太傅,太傅罰其不則而匡其不及,則德智長而治道得矣。此五學者既成於上,則百姓黎民化輯於下矣。」及太子既冠成人,免於保傅之嚴,則有記過之史,徹膳之宰,進善之旌,誹謗之木,敢諫之鼓。瞽史誦詩,工誦箴諫,大夫進謀,士傳民語。習與智長,故切而不媿;化與心成,故中道若性。三代之禮:春朝朝日,秋暮夕月,所以明有敬也;春秋入學,坐國老,執醬而親餽之,所以明有孝也;行以鸞和,步中采齊,趣中肆夏,所以明有度也;其於禽獸,見其生不食其死,聞其聲不食其肉,故遠庖廚,所以長恩,且明有仁也。

夫三代之所以長久者,以其輔翼太子有此具也。及秦而不然。其俗固非貴辭讓也,所上者告訐也;固非貴禮義也,所上者刑罰也。使趙高傅胡亥而教之獄,所習者非斬劓人,則夷人之三族也。故胡亥今日即位而明日射人,忠諫者謂之誹謗,深計者謂之妖言,其視殺人若艾草菅然。豈惟胡亥之性惡哉?彼其所以道之者非其理故也。

鄙諺曰:「不習為吏,視已成事。」又曰:「前車覆,後車誡。」夫三代之所以長久者,其已事可知也;然而不能從者,是不法聖智也。秦世之所以亟絕者,其轍跡可見也;然而不避,是後車又將覆也。夫存亡之變,治亂之機,其要在是矣。天下之命,縣於太子;太子之善,在於早諭教與選左右。夫心未濫而先諭教,則化易成也;開於道術智誼之指,則教之力也。若其服習積貫,則左右而已。夫胡、粵之人,生而同聲,耆欲不異,及其長而成俗,累數譯而不能相通,行者雖死而不相為者,則教習然也。臣故曰選左右早諭教最急。夫教得而左右正,則太子正矣,太子正而天下定矣。書曰:「一人有慶,兆民賴之。」此時務也。

凡人之智,能見已然,不能見將然。夫禮者禁於將然之前,而法者禁於已然之後,是故法之所用易見,而禮之所為生難知也。若夫慶賞以勸善,刑罰以懲惡,先王執此之政,堅如金石,行此之令,信如四時,據此之公,無私如天地耳,豈顧不用哉?然而曰禮云禮云者,貴絕惡於未萌,而起教於微眇,使民日遷善遠罪而不自知也。孔子曰:「聽訟,吾猶人也,必也使毋訟乎!」為人主計者,莫如先審取舍;取舍之極定於內,而安危之萌應於外矣。安者非一日而安也,危者非一日而危也,皆以積漸然,不可不察也。人主之所積,在其取舍。以禮義治之者,積禮義;以刑罰治之者,積刑罰。刑罰積而民怨背,禮義積而民和親。故世主欲民之善同,而所以使民善者或異。或道之以德教,或蓝之以法令。道之以德教者,德教洽而民氣樂;蓝之以法令者,法令極而民風哀。哀樂之感,禍福之應也。秦王之欲尊宗廟而安子孫,與湯武同,然而湯武廣大其德行,六七百歲而弗失,秦王治天下,十餘歲則大敗。此亡它故矣,湯武之定取舍審而秦王之定取舍不審矣。夫天下,大器也。今人之置器,置諸安處則安,置諸危處則危。天下之情與器亡以異,在天子之所置之。湯武置天下於仁義禮樂,而德澤洽,禽獸草木廣裕,德被蠻貊四夷,累子孫數十世,此天下所共聞也。秦王置天下於法令刑罰,德澤亡一有,而怨毒盈於世,下憎惡之如仇讎,禍幾及身,子孫誅絕,此天下之所共見也。是非其明效大驗邪!人之言曰:「聽言之道,必以其事觀之,則言者莫敢妄言。」今或言禮誼之不如法令,教化之不如刑罰,人主胡不引殷、周、秦事以觀之也?

人主之尊譬如堂,群臣如陛,眾庶如地。故陛九級上,廉遠地,則堂高;陛亡級,廉近地,則堂卑。高者難攀,卑者易陵,理勢然也。故古者聖王制為等列,內有公卿大夫士,外有公侯伯子男,然後有官師小吏,延及庶人,等級分明,而天子加焉,故其尊不可及也。里諺曰:「欲投鼠而忌器。」此善諭也。鼠近於器,尚憚不投,恐傷其器,況於貴臣之近主乎!廉恥節禮以治君子,故有賜死而亡戮辱。是以黥劓之罪不及大夫,以其離主上不遠也。禮不敢齒君之路馬,蹴其芻者有罰;見君之几杖則起,遭君之乘車則下,入正門則趨;君之寵臣雖或有過,刑戮之罪不加其身者,尊君之故也。此所以為主上豫遠不敬也,所以體貌大臣而厲其節也。今自王侯三公之貴,皆天子之所改容而禮之也,古天子之所謂伯父、伯舅也,而令與眾庶同黥劓髡刖笞傌棄巿之法,然則堂不亡陛虖?被戮辱者不泰迫虖?廉恥不行,大臣無乃握重權,大官而有徒隸亡恥之心虖?夫望夷之事,二世見當以重法者,投鼠而不忌器之習也。

臣聞之,履雖鮮不加於枕,冠雖敝不以苴履。夫嘗已在貴寵之位,天子改容而體貌之矣,吏民嘗俯伏以敬畏之矣,今而有過,帝令廢之可也,退之可也,賜之死可也,滅之可也;若夫束縛之,係惞之,輸之司寇,編之徒官,司寇小吏詈罵而榜笞之,殆非所以令眾庶見也。夫卑賤者習知尊貴者之一旦吾亦乃可以加此也,非所以習天下也,非尊尊貴貴之化也。夫天子之所嘗敬,眾庶之所嘗寵,死而死耳,賤人安宜得如此而頓辱之哉!

豫讓事中行之君,智伯伐而滅之,移事智伯。及趙滅智伯,豫讓舗面吞炭,必報襄子,五起而不中。人問豫子,豫子曰:「中行眾人畜我,我故眾人事之;智伯國士遇我,我故國士報之。」故此一豫讓也,反君事讎,行若狗彘,已而抗節致忠,行出虖列士,人主使然也。故主上遇其大臣如遇犬馬,彼將犬馬自為也;如遇官徒,彼將官徒自為也。頑頓亡恥绕詬亡節,廉恥不立,且不自好,苟若而可,故見利則逝,見便則奪。主上有敗,則因而挻之矣;主上有患,則吾苟免而已,立而觀之耳;有便吾身者,則欺賣而利之耳。人主將何便於此?群下至眾,而主上至少也,所託財器職業者粹於群下也。俱亡恥,俱苟妄,則主上最病。故古者禮不及庶人,刑不至大夫,所以厲寵臣之節也。古者大臣有坐不廉而廢者,不謂不廉,曰「簠簋不飾」;坐汙穢淫亂男女亡別者,不曰汙穢,曰「帷薄不修」;坐罷軟不勝任者,不謂罷軟,曰「下官不職」。故貴大臣定有其罪矣,猶未斥然正以謼之也,尚遷就而為之諱也。故其在大譴大何之域者,聞譴何則白冠氂纓,盤水加劍,造請室而請罪耳,上不執縛係引而行也。其有中罪者,聞命而自弛,上不使人頸盭而加也。其有大罪者,聞命則北面再拜,跪而自裁,上不使捽抑而刑之也,曰:「子大夫自有過耳!吾遇子有禮矣。」遇之有禮,故群臣自憙;嬰以廉恥,故人矜節行。上設廉恥禮義以遇其臣,而臣不以節行報其上者,則非人類也。故化成俗定,則為人臣者主耳忘身,國耳忘家,公耳忘私,利不苟就,害不苟去,唯義所在。上之化也,故父兄之臣誠死宗廟,法度之臣誠死社稷,輔翼之臣誠死君上,守圄扞敵之臣誠死城郭封疆。故曰聖人有金城者,比物此志也。彼且為我死,故吾得與之俱生;彼且為我亡,故吾得與之俱存;夫將為我危,故吾得與之皆安。顧行而忘利,守節而仗義,故可以託不御之權,可以寄六尺之孤。此厲廉恥行禮誼之所致也,主上何喪焉!此之不為,而顧彼之久行,故曰可為長太息者此也。

是時丞相絳侯周勃免就國,人有告勃謀反,逮繫長安獄治,卒亡事,復爵邑,故賈誼以此譏上。上深納其言,養臣下有節。是後大臣有罪,皆自殺,不受刑。至武帝時,稍復入獄,自甯成始。

初,文帝以代王入即位,後分代為兩國,立皇子武為代王,參為太原王,小子勝則梁王矣。後又徙代王武為淮陽王,而太原王參為代王,盡得故地。居數年,梁王勝死,亡子。誼復上疏曰:

陛下即不定制,如今之勢,不過一傳再傳,諸侯猶且人恣而不制,豪植而大強,漢法不得行矣。陛下所以為蕃扞及皇太子之所恃者,唯淮陽、代二國耳。代北邊匈奴,與強敵為鄰。能自完則足矣。而淮陽之比大諸侯,势如黑子之著面,適足以餌大國耳,不足以有所禁禦。方今制在陛下,制國而令子適足以為餌,豈可謂工哉!人主之行異布衣。布衣者,飾小行,競小廉,以自託於鄉黨,人主唯天下安社稷固不耳。高皇帝瓜分天下以王功臣,反者如蝟毛而起,以為不可,故蔪去不義諸侯而虛其國。擇良日,立諸子雒陽上東門之外,畢以為王,而天下安。故大人者,不牽小行,以成大功。

今淮南地遠者或數千里,越兩諸侯,而縣屬於漢。其吏民繇役往來長安者,自悉而補,中道衣敝,錢用諸費稱此,其苦屬漢而欲得王至甚,逋逃而歸諸侯者已不少矣。其勢不可久。臣之愚計,願舉淮南地以益淮陽,而為梁王立後,割淮陽北邊二三列城與東郡以益梁;不可者,可徙代王而都睢陽。梁起於新郪以北著之河,淮陽包陳以南揵之江,則大諸侯之有異心者,破膽而不敢謀。梁足以扞齊、趙,淮陽足以禁吳、楚,陛下高枕,終亡山東之憂矣,此二世之利也。當今恬然,適遇諸侯之皆少,數歲之後,陛下且見之矣。夫秦日夜苦心勞力以除六國之禍,今陛下力制天下,頤指如意,高拱以成六國之禍,難以言智。苟身亡事,畜亂宿禍,孰視而不定,萬年之後,傳之老母弱子,將使不寧,不可謂仁。臣聞聖主言問其臣而不自造事,故使人臣得畢其愚忠。唯陛下財幸!

文帝於是從誼計,乃徙淮陽王武為梁王,北界泰山,西至高陽,得大縣四十餘城;徙城陽王喜為淮陽王,撫其民。

時又封淮南厲王四子皆為列侯。誼知上必將復王之也,上疏諫曰:「竊恐陛下接王淮南諸子,曾不與如臣者孰計之也。淮南王之悖逆亡道,天下孰不知其罪?陛下幸而赦遷之,自疾而死,天下孰以王死之不當?今奉尊罪人之子,適足以負謗於天下耳。此人少壯,豈能忘其父哉?白公勝所為父報仇者,大父與伯父、叔父也。白公為亂,非欲取國代主也,發憤快志,剡手以衝仇人之匈,固為俱靡而已。淮南雖小,黥布嘗用之矣,漢存特幸耳。夫擅仇人足以危漢之資,於策不便。雖割而為四,四子一心也。予之眾,積之財,此非有子胥、白公報於廣都之中,即疑有剸諸、荊軻起於兩柱之間,所謂假賊兵為虎翼者也。願陛下少留計!」

梁王勝墜馬死,誼自傷為傅無狀,常哭泣,後歲餘,亦死。賈生之死,年三十三矣。

後四歲,齊文王薨,亡子。文帝思賈生之言,乃分齊為六國,盡立悼惠王子六人為王;又遷淮南王喜於城陽,分淮南為三國,盡立厲王三子以王之。後十年,文帝崩,景帝立,三年而吳、楚、趙與四齊王合從舉兵,西鄉京師,梁王扞之,卒破七國。至武帝時,淮南厲王子為王者兩國亦反誅。

孝武初立,舉賈生之孫二人至郡守。賈嘉最好學,世其家。

贊曰:劉向稱「賈誼言三代與秦治亂之意,其論甚美,通達國體,雖古之伊、管未能遠過也。使時見用,功化必盛。為庸臣所害,甚可悼痛。」追觀孝文玄默躬行以移風俗,誼之所陳略施行矣。及欲改定制度,以漢為土德,色上黃,數用五,及欲試屬國,施五餌三表以係單于,其術固以疏矣。誼以夭年早終,雖不至公卿,未為不遇也。凡所著述五十八篇,掇其切於世事者著于傳云。


\end{pinyinscope}