\article{公孫劉田王楊蔡陳鄭傳}

\begin{pinyinscope}
公孫賀字子叔,北地義渠人也。賀祖父昆邪,景帝時為隴西守,以將軍擊吳楚有功,封平曲侯,著書十餘篇。

賀少為騎士,從軍數有功。自武帝為太子時,賀為舍人,及武帝即位,遷至太僕。賀夫人君孺,衛皇后姊也,賀由是有寵。元光中為輕車將軍,軍馬邑。後四歲,出雲中。後五歲,以車騎將軍從大將軍青出,有功,封南窌侯。後再以左將軍出定襄,無功,坐酎金,失侯。復以浮沮將軍出五原二千餘里,無功。後八歲,遂代石慶為丞相,封葛繹侯。時朝廷多事,督責大臣。自公孫弘後,丞相李蔡、嚴青翟、趙周三人比坐事死。石慶雖以謹得終,然數被譴。初賀引拜為丞相,不受印綬,頓首涕泣,曰:「臣本邊鄙,以鞍馬騎射為官,材誠不任宰相。」上與左右見賀悲哀,感動下泣,曰:「扶起丞相。」賀不肯起,上乃起去,賀不得已拜。出,左右問其故,賀曰:「主上賢明,臣不足以稱,恐負重責,從是殆矣。」

賀子敬聲,代賀為太僕,父子並居公卿位。敬聲以皇后姊子,驕奢不奉法,征和中擅用北軍錢千九百萬,發覺,下獄。是時詔捕陽陵朱安世不能得,上求之急,賀自請逐捕安世以贖敬聲罪。上許之。後果得安世。安世者,京師大俠也,聞賀欲以贖子,笑曰:「丞相禍及宗矣。南山之竹不足受我辭,斜谷之木不足為我械。」安世遂從獄中上書,告敬聲與陽石公主私通,及使人巫祭祠詛上,且上甘泉當馳道埋偶人,祝詛有惡言。下有司案驗賀,窮治所犯,遂父子死獄中,家族。

巫蠱之禍起自朱安世,成於江充,遂及公主、皇后、太子,皆敗。語在江充、戾園傳。

劉屈氂,武帝庶兄中山靖王子也,不知其始所以進。

征和二年春,制詔御史:「故丞相賀倚舊故乘高勢而為邪,興美田以利子弟賓客,不顧元元,無益邊穀,貨賂上流,朕忍之久矣。終不自革,乃以邊為援,使內郡自省作車,又令耕者自轉,以困農煩擾畜者,重馬傷秏,武備衰減;下吏妄賦,百姓流亡;又詐為詔書,以姦傳朱安世。獄已正於理。其以涿郡太守屈氂為左丞相,分丞相長史為兩府,以待天下遠方之選。夫親親任賢,周唐之道也。以澎戶二千二百封左丞相為澎侯。」

其秋,戾太子為江充所譖,殺充,發兵入丞相府,屈氂挺身逃,亡其印綬。是時上避暑在甘泉宮,丞相長史乘疾置以聞。上問「丞相何為?」對曰:「丞相祕之,未敢發兵。」上怒曰:「

事籍籍如此,何謂祕也?丞相無周公之風矣。周公不誅管蔡乎?」乃賜丞相璽書曰:「捕斬反省,自有賞罰。以牛車為櫓,毋接短兵,多殺傷士眾。堅閉城門,毋令反者得出。」

太子既誅充發兵,宣言帝在甘泉病困,疑有變,姦臣欲作亂。上於是從甘泉來,幸城西建章宮,詔發三輔近縣兵,部中二千石以下,丞相兼將。太子亦遣使者撟制赦長安中都官囚徒,發武庫兵,命少傅石德及賓客張光等分將,使長安囚如侯持節發長水及宣曲胡騎,皆以裝會。侍郎莽通使長安,因追捕如侯,告胡人曰:「節有詐,勿聽也。」遂斬如侯,引騎入長安,又發輯濯士,以予大鴻臚商丘成。初,漢節純赤,以太子持赤節,故更為黃旄加上以相別。太子召監北軍使者任安發北軍兵,安受節已閉軍門,不肯應太子。太子引兵去,蓝四市人凡數萬眾,至長樂西闕下,逢丞相軍,合戰五日,死者數萬人,血流入溝中。丞相附兵浸多,太子軍敗,南奔覆盎城門,得出。會夜司直田仁部閉城門,坐令太子得出,丞相欲斬仁。御史大夫暴勝之謂丞相曰:「司直,吏二千石,當先請,柰何擅斬之。」丞相釋仁。上聞而大怒,下吏責問御史大夫曰:「司直縱反者,丞相斬之,法也,大夫何以擅止之?」勝之皇恐,自殺。及北軍使者任安,坐受太子節,懷二心,司直田仁縱太子,皆要斬。上曰:「侍郎莽通獲反將如侯,長安男子景建從通獲少傅石德,可謂元功矣。大鴻臚商丘成力戰獲反將張光。其封通為重合侯,建為德侯,成為秺侯。」諸太子賓客,嘗出入宮門,皆坐誅。其隨太子發兵,以反法族。吏士劫略者,皆徙敦煌郡。以太子在外,始置屯兵長安諸城門。後二十餘日,太子得於湖。語在太子傳。

其明年,貳師將軍李廣利將兵出擊匈奴,丞相為祖道,送至渭橋,與廣利辭決。廣利曰:「願君侯早請昌邑王為太子。如立為帝,君侯長何憂乎?」屈氂許諾。昌邑王者,貳師將軍女弟李夫人子也。貳師女為屈氂子妻,故共欲立焉。是時治巫蠱獄急,內者令郭穰告丞相夫人以丞相數有譴,使巫祠社,祝詛主上,有惡言,及與貳師共禱祠,欲令昌邑王為帝。有司奏請案驗,罪至大逆不道。有詔載屈氂廚車以徇,要斬東市,妻子梟首華陽街。貳師將軍妻子亦收。貳師聞之,降匈奴,宗族遂滅。

車千秋,本姓田氏,其先齊諸田徙長陵。千秋為高寑郎。會衛太子為江充所譖敗,久之,千秋上急變訟太子冤,曰:「子弄父兵,罪當笞;天子之子過誤殺人,當何罪哉!臣嘗夢見一白頭翁教臣言。」是時,上頗知太子惶恐無他意,乃大感寤,召見千秋。至前,千秋長八尺餘,體貌甚麗,武帝見而說之,謂曰:「父子之間,人所難言也,公獨明其不然。此高廟神靈使公教我,公當遂為吾輔佐。」立拜千秋為大鴻臚。數月,遂代劉屈氂為丞相,封富民侯。千秋無他材能術學,又無伐閱功勞,特以一言寤意,旬月取宰相封侯,世未嘗有也。後漢使者至匈奴,單于問曰:「聞漢新拜丞相,何用得之?」使者曰:「以上書言事故。」單于曰:「苟如是,漢置丞相,非用賢也,妄一男子上書即得之矣。」使者還,道單于語。武帝以為辱命,欲下之吏。良久,乃貰之。

然千秋為人敦厚有智,居位自稱,踰於前後數公。初,千秋始視事,見上連年治太子獄,誅罰尤多,群下恐懼,思欲寬廣上意,尉安眾庶。乃與御史、中二千石共上壽頌德美。勸上施恩惠,緩刑罰,玩聽音樂,養志和神,為天下自虞樂。上報曰:「

朕之不德,自左丞相與貳師陰謀逆亂,巫蠱之禍流及士大夫。朕日一食者累月,乃何樂之聽?痛士大夫常在心,既事不咎。雖然,巫蠱始發,詔丞相、御史督二千石求捕,廷尉治,未聞九卿廷尉有所鞫也。曩者,江充先治甘泉宮人,轉至未央椒房,以及敬聲之疇、李禹之屬謀入匈奴,有司無所發,今丞相親掘蘭臺蠱驗,所明知也。至今餘巫頗脫不止,陰賊侵身,遠近為蠱,朕媿之甚,何壽之有?敬不舉君之觴!謹謝丞相、二千石各就館。書曰:『毋偏毋黨,王道蕩蕩。』毋有復言。」

後歲餘,武帝疾,立皇子鉤弋夫人男為太子,拜大將軍霍光、車騎將軍金日磾、御史大夫桑弘羊及丞相千秋,並受遺詔,輔道少主。武帝崩,昭帝初即位,未任聽政,政事壹決大將軍光。千秋居丞相位,謹厚有重德。每公卿朝會,光謂千秋曰:「始與君侯俱受先帝遺詔,今光治內,君侯治外,宜有以教督,使光毋負天下。」千秋曰:「唯將軍留意,即天下幸甚。」終不肯有所言。光以此重之。每有吉祥嘉應,數褒賞丞相。訖昭帝世,國家少事,百姓稍益充實。始元六年,詔郡國舉賢良文學士,問以民所疾苦,於是鹽鐵之議起焉。

千秋為相十二年,薨,諡曰定侯。初,千秋年老,上優之,朝見,得乘小車入宮殿中,故因號曰「車丞相」。子順嗣侯,官至雲中太守,宣帝時以虎牙將軍擊匈奴,坐盜增鹵獲自殺,國除。

桑弘羊為御史大夫八年,自以為國家興榷筦之利,伐其功,欲為子弟得官,怨望霍光,與上官桀等謀反,遂誅滅。

王訢,濟南人也。以郡縣吏積功,稍遷為被陽令。武帝末,軍旅數發,郡國盜賊群起,繡衣御史暴勝之使持斧逐捕盜賊,以軍興從事,誅二千石以下。勝之過被陽,欲斬訢,訢已解衣伏質,仰言曰:「使君顓殺生之柄,威震郡國,今復斬一訢,不足以增威,不如時有所寬,以明恩貸,令盡死力。」勝之壯其言,貰不誅,因與訢相結厚。

勝之使還,薦訢,徵為右輔都尉,守右扶風。上數出幸安定、北地,過扶風,宮館馳道脩治,供張辦。武帝嘉之,駐車,拜訢為真,視事十餘年。昭帝時為御史大夫,代車千秋為丞相,封宜春侯。明年薨,諡曰敬侯。

子譚嗣,以列侯與謀廢昌邑王立宣帝,益封三百戶。薨,子咸嗣。王莽妻即咸女,莽篡位,宜春氏以外戚寵。自訢傳國至玄孫,莽敗,乃絕。

楊敞,華陰人也。給事大將軍莫府,為軍司馬,霍光愛厚之,稍遷至大司農。元鳳中,稻田使者燕蒼知上官桀等反謀,以告敞。敞素謹畏事,不敢言,乃移病臥。以告諫大夫杜延年,延年以聞。蒼、延年皆封,敞以九卿不輒言,故不得侯。後遷御史大夫,代王訢為丞相,封安平侯。

明年,昭帝崩。昌邑王徵即位,淫亂,大將軍光與車騎將軍張安世謀欲廢王更立。議既定,使大司農田延年報敞。敞驚懼,不知所言,汗出洽背,徒唯唯而已。延年起至更衣,敞夫人遽從東箱謂敞曰:「此國大事,今大將軍議已定,使九卿來報君侯。君侯不疾應,與大將軍同心,猶與無決,先事誅矣。」延年從更衣還,敞、夫人與延年參語許諾,請奉大將軍教令,遂共廢昌邑王,立宣帝。宣帝即位月餘,敞薨,諡曰敬侯。子忠嗣,以敞居位定策安宗廟,益封三千五百戶。

忠弟惲,字子幼,以忠任為郎,補常侍騎。惲母,司馬遷女也。惲始讀外祖太史公記,頗為春秋。以材能稱。好交英俊諸儒,名顯朝廷,擢為左曹。霍氏謀反,惲先聞知,因侍中金安上以聞,召見言狀。霍氏伏誅,惲等五人皆封,惲為平通侯,遷中郎將。

郎官故事,令郎出錢市財用,給文書,乃得出,名曰「山郎」。移病盡一日,輒償一沐,或至歲餘不得沐。其豪富郎,日出游戲,或行錢得善部。貨賂流行,傳相放效。惲為中郎將,罷山郎,移長度大司農,以給財用。其疾病休謁洗沐,皆以法令從事。郎、謁者有罪過,輒奏免,薦舉其高弟有行能者,至郡守九卿。郎官化之,莫不自厲,絕請謁貨賂之端,令行禁止,宮殿之內翕然同聲。由是擢為諸吏光祿勳,親近用事。

初,惲受父財五百萬,及身封侯,皆以分宗族。後母無子,財亦數百萬,死皆予惲,惲盡復分後母昆弟。再受訾千餘萬,皆以分施。其輕財好義如此。

惲居殿中,廉絜無私,郎官稱公平。然惲伐其行治,又性刻害,好發人陰伏,同位有忤己者,必欲害之,以其能高人。由是多怨於朝廷,與太僕戴長樂相失,卒以是敗。

長樂者,宣帝在民間時與相知,及即位,拔擢親近。長樂嘗使行事隸宗廟,還謂掾史曰:「我親面見受詔,副帝

隸,秺侯御。」人有上書告長樂非所宜言,事下廷尉。長樂疑惲教人告之,亦上書告惲罪:「高昌侯車奔入北掖門,惲語富平侯張延壽曰:『聞前曾有奔車抵殿門,門關折,馬死,而昭帝崩。今復如此,天時,非人力也。』左馮翊韓延壽有罪下獄,惲上書訟延壽。郎中丘常謂惲曰:『聞君侯訟韓馮翊,當得活乎?』惲曰:『事何容易!脛脛者未必全也。我不能自保,真人所謂鼠不容穴銜窶數者也。』又中書謁者令宣持單于使者語,視諸將軍、中朝二千石。惲曰:『冒頓單于得漢美食好物,謂之殠惡,單于不來明甚。』惲上觀西閣上畫人,指桀紂畫謂樂昌侯王武曰:『天子過此,一二問其過,可以得師矣。』畫人有堯舜禹湯,不稱而舉桀紂。惲聞匈奴降者道單于見殺,惲曰:『得不肖君,大臣為畫善計不用,自令身無處所。若秦時但任小臣,誅殺忠良,竟以滅亡;令親任大臣,即至今耳。古與今如一丘之貉。』惲妄引亡國以誹謗當世,無人臣禮。又語長樂曰:『正月以來,天陰不雨,此春秋所記,夏侯君所言。行必不至河東矣。』以主上為戲語,尤悖逆絕理。

「事下廷尉。廷尉定國考問,左驗明白,奏惲不服罪,而召戶將尊,欲令戒飭富平侯延壽,曰『太僕定有死罪數事,朝暮人也。惲幸與富平侯婚姻,今獨三人坐語,侯言「時不聞惲語」,自與太僕相觸也』。尊曰:『不可』。惲怒,持大刀,曰:『蒙富平侯力,得族罪!毋泄惲語,令太僕聞之亂餘事。』惲幸得列九卿諸吏,宿衛近臣,上所信任,與聞政事,不竭忠愛,盡臣子義,而妄怨望,稱引為訞惡言,大逆不道,請逮捕治。」上不忍加誅,有詔皆免惲、長樂為庶人。

惲既失爵位,家居治產業,起室宅,以財自娛。歲餘,其友人安定太守西河孫會宗,知略士也,與惲書諫戒之,為言大臣廢退,當闔門惶懼,為可憐之意,不當治產業,通賓客,有稱舉。惲宰相子,少顯朝廷,一朝晻昧語言見廢,內懷不服,報會宗書曰:

惲材朽行穢,文質無所底,幸賴先人餘業得備宿衛,遭遇時變以獲爵位,終非其任,卒與禍會。足下哀其愚,蒙賜書,教督以所不及,殷勤甚厚。然竊恨足下不深惟其終始,而猥隨俗之毀譽也。言鄙陋之愚心,若逆指而文過,默而息乎,恐違孔氏「各言爾志」之義,故敢略陳其愚,唯君子察焉!

惲家方隆盛時,乘朱輪者十人,位在列卿,爵為通侯,總領從官,與聞政事,曾不能以此時有所建明,以宣德化,又不能與群僚同心并力,陪輔朝廷之遺忘,已負竊位素餐之責久矣。懷祿貪勢,不能自退,遭遇變故,橫被口語,身幽北闕,妻子滿獄。當此之時,自以夷滅不足以塞責,豈意得全首領,復奉先人之丘墓乎?伏惟聖主之恩,不可勝量。君子游道,樂以忘憂;小人全軀,說以忘罪。竊自思念,過已大矣,行已虧矣,長為農夫以沒世矣。是故身率妻子,戮力耕桑,灌園治產,以給公上,不意當復用此為譏議也。

夫人情所不能止者,聖人弗禁,故君父至尊親,送其終也,有時而既。臣之得罪,已三年矣。田家作苦,歲時伏臘,亨羊炰羔,斗酒自勞。家本秦也,能為秦聲。婦,趙女也,雅善鼓瑟。奴婢歌者數人,酒後耳熱,仰天拊缶而呼烏烏。其《詩》曰:「田彼南山,蕪穢不治,種一頃豆,落而為萁。人生行樂耳,須富貴何時!」是日也,拂衣而喜,奮褎低卬,頓足起舞,誠淫荒無度,不知其不可也。惲幸有餘祿,方糴賤販貴,逐什一之利,此賈豎之事,汙辱之處,惲親行之。下流之人,眾毀所歸,不寒而栗。雖雅知惲者,猶隨風而靡,尚何稱譽之有!董生不云乎?「明明求仁義,常恐不能化民者,卿大夫意也;明明求財利,常恐困乏者,庶人之事也。」故「道不同,不相為謀。」今子尚安得以卿大夫之制而責僕哉!

夫西河魏土,文侯所興,有段干木、田子方之遺風,漂然皆有節概,知去就之分。頃者,足下離舊土,臨安定,安定山谷之間,昆戎舊壤,子弟貪鄙,豈習俗之移人哉?於今乃睹子之志矣。方當盛漢之隆,願勉旃,毋多談。

又惲兄子安平侯譚為典屬國,謂惲曰:「西河太守建平杜侯前以罪過出,今徵為御史大夫。侯罪薄,又有功,且復用。」惲曰:「有功何益?縣官不足為盡力。」惲素與蓋寬饒、韓延壽善,譚即曰:「縣官實然,蓋司隸、韓馮翊皆盡力吏也,俱坐事誅。」會有日食變,騶馬猥佐成上書告惲「驕奢不悔過,日食之咎,此人所致。」章下廷尉案驗,得所予會宗書,宣帝見而惡之。廷尉當惲大逆無道,要斬。妻子徙酒泉郡。譚坐不諫正惲,與相應,有怨望語,免為庶人。召拜成為郎,諸在位與惲厚善者,未央衛尉韋玄成、京兆尹張敞及孫會宗等,皆免官。

蔡義,河內溫人也。以明經給事大將軍莫府。家貧,常步行,資禮不逮眾門下,好事者相合為義買犢車,令乘之。數歲,遷補覆盎城門候。

久之,詔求能為韓詩者,徵義待詔,久不進見。義上疏曰:「臣山東草萊之人,行能亡所比,容貌不及眾,然而不棄人倫者,竊以聞道於先師,自託於經術也。願賜清閒之燕,得盡精思於前。」上召見義,說詩,甚說之,擢為光祿大夫給事中,進授昭帝。數歲,拜為少府,遷御史大夫,代楊敞為丞相,封陽平侯。又以定策安宗廟益封,加賜黃金二百斤。

義為丞相時年八十餘,短小無須眉,貌似老嫗,行步俛僂,常兩吏扶夾乃能行。時大將軍光秉政,議者或言光置宰相不選賢,苟用可顓制者。光聞之,謂侍中左右及官屬曰:「以為人主師當為宰相,何謂云云?此語不可使天下聞也。」

義為相四歲,薨,諡曰節侯。無子,國除。

陳萬年字幼公,沛郡相人也。為郡吏,察舉,至縣令,遷廣陵太守,以高弟入為右扶風,遷太僕。

萬年廉平,內行修,然善事人,賂遺外戚許、史,傾家自盡,尤事樂陵侯史高。丞相丙吉病,中二千石上謁問疾。遣家丞出謝,謝已皆去,萬年獨留,昏夜乃歸。及吉病甚,上自臨,問以大臣行能。吉薦于定國、杜延年及萬年。萬年竟代定國為御史大夫,八歲病卒。

子咸字子康,年十八,以萬年任為郎。有異材,抗直,數言事,刺譏近臣,書數十上,遷為左曹。萬年嘗病,召咸教戒於床下,語至夜半,咸睡,頭觸屏風。萬年大怒,欲杖之,曰:「乃公教戒汝,汝反睡,不聽吾言,何也?」咸叩頭謝曰:「具曉所言,大要教咸諂也。」萬年乃不復言。

萬年死後,元帝擢咸為御史中丞,總領州郡奏事,課第諸刺史,內執法殿中,公卿以下皆敬憚之。是時中書令石顯用事顓權,咸頗言顯短,顯等恨之。時槐里令朱雲殘酷殺不辜,有司舉奏,未下。咸素善雲,雲從刺候,教令上書自訟。於是石顯微伺知之,白奏咸漏泄省中語,下獄掠治,減死,髡為城旦,因廢。

成帝初即位,大將軍王鳳以咸前指言石顯,有忠直節,奏請咸補長史。遷冀州刺史,奉使稱意,徵為諫大夫。復出為楚內史,北海、東郡太守。坐為京兆尹王章所薦,章誅,咸免官。起家復為南陽太守。所居以殺伐立威,豪猾吏及大姓犯法,輒論輸府,以律程作司空,為地臼木杵,舂不中程,或私解脫鉗釱,衣服不如法,輒加罪笞。督作劇,不勝痛,自絞死,歲數百千人,久者蟲出腐爛,家不得收。其治放嚴延年,其廉不如。所居調發屬縣所出食物以自奉養,奢侈玉食。然操持掾史,郡中長吏皆令閉門自斂,不得踰法。公移敕書曰:「即各欲求索自快,是一郡百太守也,何得然哉!」下吏畏之,豪彊執服,令行禁止,然亦以此見廢。咸,三公子,少顯名於朝廷,而薛宣、朱博、翟方進、孔光等仕宦絕在咸後,皆以廉儉先至公卿,而咸滯於郡守。

時車騎將軍王音輔政,信用陳湯。咸數賂遺湯,予書曰:「即蒙子公力,得入帝城,死不恨。」後竟徵入為少府。少府多寶物,屬官咸皆鉤校,發其姦臧,沒入辜榷財物。官屬及諸中宮黃門、鉤盾、掖庭官吏,舉奏按論,畏咸,皆失氣。為少府三歲,與翟方進有隙。方進為丞相,奏「咸前為郡守,所在殘酷,毒螫加於吏民。主守盜,受所監。而官媚邪臣陳湯以求薦舉。苟得無恥,不宜處位。」咸坐免。頃之,紅陽侯立舉咸方正,為光祿大夫給事中,方進復奏免之。後數年,立有罪就國,方進奏歸咸故郡,以憂死。

鄭弘字稚卿,泰山剛人也。兄昌字次卿,亦好學,皆明經,通法律政事。次卿為太原、涿郡太守,弘為南陽太守,皆著治跡,條教法度,為後所述。次卿用刑罰深,不如弘平。遷淮陽相,以高弟入為右扶風,京師稱之。代韋玄成為御史大夫。六歲,坐與京房論議免,語在房傳。

贊曰:所謂鹽鐵議者,起始元中,徵文學賢良問以治亂,皆對願罷郡國鹽鐵酒榷均輸,務本抑末,毋與天下爭利,然後化可興。御史大夫弘羊以為此乃所以安邊竟,制四夷,國家大業,不可廢也。當時相詰難,頗有其議文。至宣帝時,汝南相寬次公治公羊春秋,舉為郎,至廬江太守丞,博通善屬文,推衍鹽鐵之議,增廣條目,極其論難,著數萬言,亦欲以究治亂,成一家之法焉。其辭曰:「觀公卿賢良文學之議,『異乎吾所聞』。聞汝南朱生言,當此之時,英俊並進,賢良茂陵唐生、文學魯國萬生之徒六十有餘人咸聚闕庭,舒六藝之風,陳治平之原,知者贊其慮,仁者明其施,勇者見其斷,辯者騁其辭,齗齗焉,行行焉,雖未詳備,斯可略觀矣。中山劉子推言王道,撟當世,反諸正,彬彬然弘博君子也。九江祝生奮史魚之節,發憤懣,譏公卿,介然直而不撓,可謂不畏彊圉矣。桑大夫據當世,合時變,上權利之略,雖非正法,鉅儒宿學不能自解,博物通達之士也。然攝公卿之柄,不師古始,放於末利,處非其位,行非其道,果隕其性,以及厥宗。車丞相履伊呂之列,當軸處中,括囊不言,容身而去,彼哉!彼哉!若夫丞相、御史兩府之士,不能正議以輔宰相,成同類,長同行,阿意苟合,以說其上,『斗筲之徒,何足選也!』」


\end{pinyinscope}