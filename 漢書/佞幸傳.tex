\article{佞幸傳}

\begin{pinyinscope}
漢興,佞幸寵臣,高祖時則有籍孺,孝惠有閎孺。此兩人非有材能,但以婉媚貴幸,與上臥起,公卿皆因關說。故孝惠時,郎侍中皆冠鵔鸃,貝帶,傅脂粉,化閎、籍之屬也。兩人徙家安陵。其後寵臣,孝文時士人則鄧通,宦者則趙談、北宮伯子;孝武時士人則韓嫣,宦者則李延年;孝元時宦者則弘恭、石顯;孝成時士人則張放、淳于長;孝哀時則有董賢。孝景、昭、宣時皆無寵臣。景帝唯有郎中令周仁。昭帝時,駙馬都尉秺侯金賞嗣父車騎將軍日磾爵為侯,二人之寵取過庸,不篤。宣帝時,侍中中郎將張彭祖少與帝微時同席研書,及帝即尊位,彭祖以舊恩封陽都侯,出常參乘,號為愛幸。其人謹敕,無所虧損,為其小妻所毒薨,國除。

鄧通,蜀郡南安人也,以濯船為黃頭郎。文帝嘗夢欲上天,不能,有一黃頭郎推上天,顧見其衣尻帶後穿。覺而之漸臺,以夢中陰目求推者郎,見鄧通,其衣後穿,夢中所見也。召問其名姓,姓鄧,名通。鄧猶登也,文帝甚說,尊幸之,日日異。通亦愿謹,不好外交,雖賜洗沐,不欲出。於是文帝賞賜通鉅萬以十數,官至上大夫。

文帝時間如通家游戲,然通無他伎能,不能有所薦達,獨自謹身以媚上而已。上使善相人者相通,曰:「當貧餓死。」上曰:「能富通者在我,何說貧?」於是賜通蜀嚴道銅山,得自鑄錢。鄧氏錢布天下,其富如此。

文帝嘗病癰,鄧通常為上嗽吮之。上不樂,從容問曰:「天下誰最愛我者乎?」通曰:「宜莫若太子。」太子入問疾,上使太子齰癰太子嗽癰而色難之。已而聞通嘗為上齰,太子慚,繇是心恨通。

及文帝崩,景帝立,鄧通免,家居。居無何,人有告通盜出徼外鑄錢,下吏驗問,頗有,遂竟案,盡沒入之,通家尚負責數鉅萬。長公主賜鄧通,吏輒隨沒入之,一簪不得著身。於是長公主乃令假衣食。竟不得名一錢,寄死人家。

趙談者,以星氣幸,北宮伯子長者愛人,故親近,然皆不比鄧通。

韓嫣字王孫,弓高侯穨當之孫也。武帝為膠東王時,嫣與上學書相愛。及上為太子,愈益親嫣。嫣善騎射,聰慧。上即位,欲事伐胡,而嫣先習兵,以故益尊貴,官至上大夫,賞賜儗鄧通。

始時,嫣常與上共臥起。江都王入朝,從上獵上林中。天子車駕䟆道未行,先使嫣乘副車,從數十百騎馳視獸。江都王望見,以為天子,辟從者,伏謁道旁。嫣驅不見。既過,江都王怒,為皇太后泣,請得歸國入宿衛,比韓嫣。太后繇此銜嫣。

嫣侍,出入永巷不禁,以姦聞皇太后。太后怒,使使賜嫣死。上為謝,終不能得,嫣遂死。

嫣弟說,亦愛幸,以軍功封案道侯,巫蠱時為戾太子所殺。子增封龍雒侯,大司馬車騎將軍,自有傳。

李延年,中山人,身及父母兄弟皆故倡也。延年坐法腐刑,給事狗監中。女弟得幸於上,號李夫人,列外戚傳。延年善歌,為新變聲。是時上方興天地諸祠,欲造樂,令司馬相如等作詩頌。延年輒承意弦歌所造詩,為之新聲曲。而李夫人產昌邑王,延年繇是貴為協律都尉,佩二千石印綬,而與上臥起,其愛幸埒韓嫣。久之,延年弟季與中人亂,出入驕恣。及李夫人卒後,其愛弛,上遂誅延年兄弟宗族。

是後寵臣,大氐外戚之家也。衛青、霍去病皆愛幸,然亦以功能自進。

石顯字君房,濟南人;弘恭,沛人也。皆少坐法腐刑,為中黃門,以選為中尚書。宣帝時任中書官,恭明習法令故事,善為請奏,能稱其職。恭為令,顯為僕射。元帝即位數年,恭死,顯代為中書令。

是時,元帝被疾,不親政事,方隆好於音樂,以顯久典事,中人無外黨,精專可信任,遂委以政。事無小大,因顯白決,貴幸傾朝,百僚皆敬事顯。顯為人巧慧習事,能探得人主微指,內深賊,持詭辯以中傷人,忤恨睚眥,輒被以危法。初元中,前將軍蕭望之及光祿大夫周堪、宗正劉更生皆給事中。望之領尚書事,知顯專權邪辟,建白以為「尚書百官之本,國家樞機,宜以通明公正處之。武帝游宴後庭,故用宦者,非古制也。宜罷中書宦官,應古不近刑人。」元帝不聽,繇是大與顯忤。後皆害焉,望之自殺,堪、更生廢錮,不得復進用,語在望之傳。後太中大夫張猛、魏郡太守京房、御史中丞陳咸、待詔賈捐之皆嘗奏封事,或召見,言顯短。顯求索其罪,房、捐之棄市,猛自殺於公車,咸抵罪,髡為城旦。及鄭令蘇建得顯私書奏之,後以它事論死。自是公卿以下畏顯,重足一跡。

顯與中書僕射牢梁、少府五鹿充宗結為黨友,諸附倚者皆得寵位。民歌之曰:「牢邪石邪,五鹿客邪!印何纍纍,綬若若邪!」言其兼官據勢也。

顯見左將軍馮奉世父子為公卿著名,女又為昭儀在內,顯心欲附之,薦言昭儀兄謁者逡修敕宜帷幄。天子召見,欲以為侍中,逡請間言事。上聞逡言顯顓權,天子大怒,罷逡歸郎官。其後御史大夫缺,群臣皆舉逡兄大鴻臚野王行能第一,天子以問顯,顯曰:「九卿無出野王者。然野王親昭儀兄,臣恐後世必以陛下度越眾賢,私後宮親以為三公。」上曰:「善,吾不見是。」乃下詔嘉美野王,廢而不用,語在野王傳。

顯內自知擅權事柄在掌握,恐天子一旦納用左右耳目,有以間己,乃時歸誠,取一信以為驗。顯嘗使至諸官有所徵發,顯先自白,恐後漏盡宮門閉,請使詔吏開門。上許之。顯故投夜還,稱詔開門入。後果有上書告顯顓命矯詔開宮門,天子聞之,笑以其書示顯。顯因泣曰:「陛下過私小臣,屬任以事,群下無不嫉妒欲陷害臣者,事類如此非一,唯獨明主知之。愚臣微賤,誠不能以一軀稱快萬眾,任天下之怨,臣願歸樞機職,受後宮掃除之役,死無所恨,唯陛下哀憐財幸,以此全活小臣。」天子以為然而憐之,數勞勉顯,加厚賞賜,賞賜及賂遺訾一萬萬。

初,顯聞眾人匈匈,言己殺前將軍蕭望之。望之當世名儒,顯恐天下學士姍己,病之。是時,明經著節士琅邪貢禹為諫大夫,顯使人致意,深自結納。顯因薦禹天子,歷位九卿,至御史大夫,禮事之甚備。議者於是稱顯,以為不妒譖望之矣。顯之設變詐以自解免取信人主者,皆此類也。

元帝晚節寢疾,定陶恭王愛幸,顯擁祐太子頗有力。元帝崩,成帝初即位,遷顯為長信中太僕,秩中二千石。顯失倚,離權數月,丞相御史條奏顯舊惡,及其黨牢梁、陳順皆免官。顯與妻子徙歸故郡,憂滿不食,道病死。諸所交結,以顯為官,皆廢罷。少府五鹿充宗左遷玄菟太守,御史中丞伊嘉為鴈門都尉。長安謠曰:「伊徙鴈,鹿徙菟,去牢與陳實無賈。」

淳于長字子篓,魏郡元城人也。少以太后姊子為黃門郎,未進幸。會大將軍王鳳病,長侍病,晨夜扶丞左右,甚有甥舅之恩。鳳且終,以長屬託太后及帝。帝嘉長義,拜為列校尉諸曹,遷水衡都尉侍中,至衛尉九卿。

久之,趙飛燕貴幸,上欲立以為皇后,太后以其所出微,難之。長主往來通語東宮。歲餘,趙皇后得立,上甚德之,乃追顯長前功,下詔曰:「前將作大匠解萬年奏請營作昌陵,罷弊海內,侍中衛尉長數白宜止徙家反故處,朕以長言下公卿,議者皆合長計。首建至策,民以康寧。其賜長爵關內侯。」後遂封為定陵侯,大見信用,貴傾公卿。外交諸侯牧守,賂遺賞賜亦絫鉅萬。多畜妻妾,淫於聲色,不奉法度。

初,許皇坐執左道廢處長定宮,而后姊缅為龍哣思侯夫人,寡居。長與缅私通,因取為小妻。許后因缅賂遺長,欲求復為婕妤。長受許后金錢乘輿服御物前後千餘萬,詐許為白上,立以為左皇后。缅每入長定宮,輒與缅書,戲侮許后,嫚易無不言。交通書記,賂遺連年,是時,帝舅曲陽侯王根為大司馬票騎將軍,輔政數歲,久病,數乞骸骨。長以外親居九卿位,次第當代根。根兄子新都侯王莽心害長寵,私聞長取許缅,受長定宮賂遺。莽侍曲陽侯疾,因言「長見將軍久病,意喜,自以當代輔政,至對衣冠議語署置。」具言其罪過。根怒曰:「即如是,何不白也?」莽曰:「未知將軍意,故未敢言。」根曰:「趣白東宮。」莽求見太后,具言長驕佚,欲代曲陽侯,對莽母上車,私與長定貴人姊通,受取其衣物。太后亦怒曰:「兒至如此!往白之帝!」莽白上,上乃免長官,遣就國。

初,長為侍中,奉兩宮使,親密。紅陽侯立獨不得為大司馬輔政,立自疑為長毀譖,常怨毒長。上知之。及長當就國也,立嗣子融從長請車騎,長以珍寶因融重遺立,立因為長言。於是天子疑焉,下有司案驗。吏捕融,立令融自殺以滅口。上愈疑其有大姦,遂逮長繫洛陽詔獄窮治。長具服戲侮長定宮,謀立左皇后,罪至大逆,死獄中。妻子當坐者徙合浦,母若歸故郡。紅陽侯立就國。將軍卿大夫郡守坐長免罷者數十人。莽遂代根為大司馬。久之,還長母及子酺於長安。後酺有罪,莽復殺之,徙其家屬故郡。

始長以外親親近,其愛幸不及富平侯張放。放常與上臥起,俱為微行出入。

董賢字聖卿,雲陽人也。父恭,為御史,任賢為太子舍人。哀帝立,賢隨太子官為郎。二歲餘,賢傳漏在殿下,為人美麗自喜,哀帝望見,說其儀貌,識而問之,曰:「是舍人董賢邪?」因引上與語,拜為黃門郎,繇是始幸。問及其父為雲中侯,即日徵為霸陵令,遷光祿大夫。賢寵愛日甚,為駙馬都尉侍中,出則參乘,入御左右,旬月間賞賜絫鉅萬,貴震朝廷。常與上臥起。嘗晝寢,偏藉上袖,上欲起,賢未覺,不欲動賢,乃斷袖而起。其恩愛至此。賢亦性柔和便辟,善為媚以自固。每賜洗沐,不肯出,嘗留中視醫藥。上以賢難歸,詔令賢妻得通引籍殿中,止賢廬,若吏妻子居官寺舍。又詔賢女弟以為昭儀,位次皇后,更名其舍為椒風,以配椒房云。昭儀及賢與妻旦夕上下,並侍左右。賞賜昭儀及賢妻亦各千萬數。遷賢父為少府,賜爵關內侯,食邑,復徙為衛尉。又以賢妻父為將作大匠,弟為執金吾。詔將作大匠為賢起大第北闕下,重殿洞門,木土之功窮極技巧,柱檻衣以綈錦。下至賢家僮僕皆受上賜,及武庫禁兵,上方珍寶。其選物上弟盡在董氏,而乘輿所服乃其副也。及至東園祕器,珠襦玉柙,豫以賜賢,無不備具。又令將作為賢起冢塋義陵旁,內為便房,剛柏題湊,外為徼道,周垣數里,門闕罘罳甚盛。

上欲侯賢而未有緣。會待詔孫寵、息夫躬等告東平王雲后謁祠祀祝詛,下有司治,皆伏其辜。上於是令躬、寵為因賢告東平事者,乃以其功下詔封賢為高安侯,躬宜陵侯,寵方陽侯,食邑各千戶。頃之,復益封賢二千戶。丞相王嘉內疑東平事冤,甚惡躬等,數諫爭,以賢為亂國制度,嘉竟坐言事下獄死。

上初即位,祖母傅太后、母丁太后皆在,兩家先貴。傅太后從弟喜先為大司馬輔政,數諫,失太后指,免官。上舅丁明代為大司馬,亦任職,頗害賢寵,及丞相王嘉死,明甚憐之。上浸重賢,欲極其位,而恨明如此,遂冊免明曰:「前東平王雲貪欲上位,祠祭祝詛,雲后舅伍宏以醫待詔,與校祕書郎楊閎結謀反逆,禍甚迫切。賴宗廟神靈,董賢等以聞,咸伏其辜。將軍從弟侍中奉車都尉吳、族父左曹屯騎校尉宣皆知宏及栩丹諸侯王后親,而宣除用丹為御屬,吳與宏交通厚善,數稱薦宏。宏以附吳得興其惡心,因醫技進,幾危社稷,朕以恭皇后故,不忍有云。將軍位尊任重,既不能明威立義,折消未萌,又不深疾雲、宏之惡,而懷非君上,阿為宣、吳,反痛恨雲等揚言為群下所冤,又親見言伍宏善醫,死可惜也,賢等獲封極幸。嫉妒忠良,非毀有功,於戲傷哉!蓋『君親無將,將而誅之』。是以季友鴆叔牙,春秋賢之;趙盾不討賊,謂之弒君。朕閔將軍陷于重刑,故以書飭。將軍遂非不改,復與丞相嘉相比,令嘉有依,得以罔上。有司致法將軍請獄治,朕惟噬膚之恩未忍,其上票騎將軍印綬,罷歸就第。」遂以賢代明為大司馬衛將軍,冊曰:「朕承天序,惟稽古建爾于公,以為漢輔。往悉爾心,統辟元戎,折衝綏遠,匡正庶事,允執其中。天下之眾,受制於朕,以將為命,以兵為威,可不慎與!」是時賢年二十二,雖為三公,常給事中,領尚書,百官因賢奏事。以父恭不宜在卿位,徙為光祿大夫,秩中二千石。弟寬信代賢為駙馬都尉。董氏親屬皆侍中諸曹奉朝請,寵在丁、傅之右矣。

明年,匈奴單于來朝,宴見,群臣在前。單于怪賢年少,以問譯,上令譯報曰:「大司馬年少,以大賢居位。」單于乃起拜,賀漢得賢臣。

初,丞相孔光為御史大夫,時賢父恭為御史,事光。及賢為大司馬,與光並為三公,上故令賢私過光。光雅恭謹。知上欲尊寵賢,及聞賢當來也,光警戒衣冠出門待,望見賢車乃卻入。賢至中門,光入閤,既下車,乃出拜謁,送迎甚謹,不敢以賓客均敵之禮。賢歸,上聞之喜,立拜光兩兄子為諫大夫常侍。賢繇是權與人主侔矣。

是時,成帝外家王氏衰廢,唯平阿侯譚子去疾,哀帝為太子時為庶子得幸,及即位,為侍中騎都尉。上以王氏亡在位者,遂用舊恩親近去疾,復進其弟閎為中常侍。閎妻父蕭咸,前將軍望之子也,久為郡守,病免,為中郎將。兄弟並列,賢父恭慕之,欲與結婚姻。閎為賢弟駙馬都尉寬信求咸女為婦,咸惶恐不敢當,私謂閎曰:「董公為大司馬,冊文言『允執其中』,此乃堯禪舜之文,非三公故事,長老見者,莫不心懼。此豈家人子所能堪邪!」閎性有知略,聞咸言,心亦悟。乃還報恭,深達咸自謙薄之意。恭歎曰:「我家何用負天下,而為人所畏如是!」意不說。後上置酒麒麟殿,賢父子親屬宴飲,王閎兄弟侍中中常侍皆在側。上有酒所,從容視賢笑,曰:「吾欲法堯禪舜,何如?」閎進曰:「天下乃高皇帝天下,非陛下之有也。陛下承宗廟,當傳子孫於亡窮。統業至重,天子亡戲言!」上默然不說,左右皆恐。於是遣閎出,後不得復侍宴。

賢第新成,功堅,其外大門無故自壞,賢心惡之。後數月,哀帝崩。太皇太后召大司馬賢,引見東廂,問以喪事調度。賢內憂,不能對,免冠謝。太后曰:「新都侯莽前以大司馬奉送先帝大行,曉習故事,吾令莽佐君。」賢頓首幸甚。太后遣使者召莽。既至,以太后指使尚書劾賢帝病不親醫藥,禁止賢不得入出宮殿司馬中。賢不知所為,詣闕免冠徒跣謝。莽使謁者以太后詔即闕下冊賢曰:「間者以來,陰陽不調,菑害並臻,元元蒙辜。夫三公,鼎足之輔也,高安侯賢未更事理,為大司馬不合眾心,非所以折衝綏遠也。其收大司馬印綬,罷歸第。」即日賢與妻皆自殺,家惶恐夜葬。莽疑其詐死,有司奏請發賢棺,至獄診視。莽復風大司徒光奏「賢質性巧佞,翼姦以獲封侯,父子專朝,兄弟並寵,多受賞賜,治第宅,遠冢壙,放效無極,不異王制,費以萬萬計,國家為空虛。父子驕蹇。至不為使者禮,受賜不拜,罪惡暴著。賢自殺伏辜,死後父恭等不悔過,乃復以沙畫棺四時之色,左蒼龍,右白虎,上著金銀日月,玉衣珠璧以棺,至尊無以加。恭等幸得免於誅,不宜在中土。臣請收沒入財物縣官。諸以賢為官者皆免。」父恭、弟寬信與家屬徙合浦,母別歸故郡鉅鹿。長安中小民讙譁,鄉其弟哭,幾獲盜之。縣官斥賣董氏財凡四十三萬萬。賢既見發,臝診其尸,因埋獄中。

賢所厚吏沛朱詡自劾去大司馬府,買棺衣收賢尸葬之。王莽聞之而大怒,以它罪擊殺詡。詡子浮建武中貴顯,至大司馬,司空,封侯。而王閎王莽時為牧守,所居見紀,莽敗乃去官。世祖下詔曰:「武王克殷,表商容之閭。閎修善謹敕,兵起,吏民獨不爭其頭首。今以閎子補吏。」至墨綬率官,蕭咸外孫云。

贊曰:柔曼之傾意,非獨女德,蓋亦有男色焉。觀籍、閎、鄧、韓之徒非一,而董賢之寵尤盛,父子並為公卿,可謂貴重人臣無二矣。然進不繇道,位過其任,莫能有終,所謂愛之適足以害之者也。漢世衰於元、成,壞於哀、平。哀、平之際,國多釁矣。主疾無嗣,弄臣為輔,鼎足不彊,棟幹微撓。一朝帝崩,姦臣擅命,董賢縊死,丁、傅流放,辜及母后,奪位幽廢,咎在親便嬖,所任非仁賢。故仲尼著「損者三友」,王者不私人以官,殆為此也。


\end{pinyinscope}