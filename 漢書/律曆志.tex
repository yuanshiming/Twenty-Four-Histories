\article{律曆志}

\begin{pinyinscope}
《虞書》曰「乃同律度量衡」,所以齊遠近立民信也。自伏戲畫八卦,由數起,至黃帝、堯、舜而大備。三代稽古,法度章焉。周衰官失,孔子陳後王之法,曰:「謹權量,審法度,修廢官,舉逸民,四方之政行矣。」漢興,北平侯張蒼首律曆事,孝武帝時樂官考正。至元始中王莽秉政,欲燿名譽,徵天下通知鐘律者百餘餘人,使羲和劉歆等典領條奏,言之最詳。故刪其偽辭,取正義,著于篇。

一曰備數,二曰和聲,三曰審度,四曰嘉量,五曰權衡。參五以變,錯綜其數,稽之於古今,效之於氣物,和之於心耳,考之於經傳,咸得其實,靡不協同。

數者,一、十、百、千、萬也,所以算數事物,順性命之理也。《書》曰:「先其算命。」本起於黃鐘之數,始於一而三之,三三積之,歷十二辰之數,十有七萬七千一百四十七,而五數備矣。其算法用竹,徑一分,長六寸,二百七十一枚而成六觚,為一握。徑象乾律黃鐘之一,而長象坤呂林鐘之長。其數以易大衍之數五十,其用四十九,成陽六爻,得周流六虛之象也。夫推曆生律制器,規圜矩方,權重衡平,準繩嘉量,探賾索隱,鉤深致遠,莫不用焉。度長短者不失豪氂,量多少者不失圭撮,權輕重者不失黍絫。紀於一,協於十,長於百,大於千,衍於萬,其法在算術。宣於天下,小學是則。職在太史,羲和掌之。

聲者,宮、商、角、徵、羽也。所以作樂者,諧八音,蕩降人之邪意,全其正性,移風易俗也。八音:土曰塤,匏曰笙,皮曰鼓,竹曰管,絲曰絃,石曰磬,金曰鐘,木曰柷。五聲和,八音諧,而樂成。商之為言章也,物成孰可章度也。角,觸也,物觸地而出,戴芒角也。宮,中也,居中央,暢四方,唱始施生,為四聲綱也。徵,祉也,物盛大而茇祉也。羽,宇也,物聚臧宇覆之也。夫聲者,中於宮,觸於角,祉於徵,章於商,宇於羽,故四聲為宮紀也。協之五行,則角為木,五常為仁,五事為貌。商為金為義為言,徵為火為禮為視,羽為水為智為聽,宮為土為信為思。以君臣民事物言之,則宮為君,商為臣,角為民,徵為事,羽為物。唱和有象,故言君臣位事之體也。

五聲之本,生於黃鐘之律。九寸為宮,或損或益,以定商、角、徵、羽。九六相生,陰陽之應也。律十有二,陽六為律,陰六為呂。律以統氣類物,一曰黃鐘,二曰太族,三曰姑洗,四曰蕤賓,五曰夷則,六曰亡射。呂以旅陽宣氣,一曰林鐘,二曰南呂,三曰應鐘,四曰大呂,五曰夾鐘,六曰中呂。有三統之義焉。其傳曰,黃帝之所作也。黃帝使泠綸,自大夏之西,昆侖之陰,取竹之解谷生,其竅厚均者,斷兩節間而吹之,以為黃鐘之宮。制十二筩以聽鳳之鳴,其雄鳴為六,雌鳴亦六,比黃鐘之宮,而皆可以生之,是為律本。至治之世,天地之氣合以生風;天地之風氣正,十二律定。黃鐘:黃者,中之色,君之服也;鐘者,種也。天之中數五,五為聲,聲上宮,五聲莫大焉。地之中數六,六為律,律有形有色,色上黃,五色莫盛焉。故陽氣施種於黃泉,孳萌萬物,為六氣元也。以黃色名元氣律者,著宮聲也。宮以九唱六,變動不居,周流六虛。始於子,在十一月。大呂:呂,旅也,言陰大,旅助黃鐘宮氣而牙物也。位於丑,在十二月。太族:族,奏也,言陽氣大,奏地而達物也。位於寅,在正月。夾鐘,言陰夾助太族宣四方之氣而出種物也。位於卯,在二月。姑洗:洗,絜也,言陽氣洗物辜絜之也。位於辰,在三月。中呂,言微陰始起未成,著於其中旅助姑洗宣氣齊物也。位於巳,在四月。蕤賓:蕤,繼也,賓,導也,言陽始導陰氣使繼養物也。位於午,在五月。林鐘:林,君也,言陰氣受任,助蕤賓君主種物使長大楙盛也。位於未,在六月。夷則:則,法也,言陽氣正法度而使陰氣夷當傷之物也。位於申,在七月。南呂:南,任也,言陰氣旅助夷則任成萬物也。位於酉,在八月。亡射:射,厭也,言陽氣究物而使陰氣畢剝落之,終而復始,亡厭已也。位於戌,在九月。應鐘,言陰氣應亡射,該臧萬物而雜陽閡種也。位於亥,在十月。

三統者,天施,地化,人事之紀也。十一月,乾之初九,陽氣伏於地下,始著為一,萬物萌動,鐘於太陰,故黃鐘為天統,律長九寸。九者,所以究極中和,為萬物元也。《易》曰:「立天之道,曰陰與陽。」六月,坤之初六,陰氣受任於太陽,繼養化柔,萬物生長,楙之於未,令種剛彊大,故林鐘為地統,律長六寸。六者,所以含陽之施,楙之於六合之內,令剛柔有體也。「立地之道,曰柔與剛。」「乾知太始,坤作成物。」正月,乾之九三,萬物棣通,族出於寅,人奉而成之,仁以養之,義以行之,令事物各得其理。寅,木也,為仁;其聲,商也,為義。故太族為人統,律長八寸,象八卦,宓戲氏之所以順天地,通神明,類萬物之情也。「立人之道,曰仁與義。」「在天成象,在地成形。」「后以裁成天地之道,輔相天地之宜,以左右民。」此三律之謂矣,是為三統。

其於三正也,黃鐘子為天正,林鐘未之衝丑為地正,太族寅為人正。三正正始,是以地正適其始紐於陽東北丑位。《易》曰「東北喪朋,乃終有慶」,答應之道也。及黃鐘為宮,則太族、姑洗、林鐘、南呂皆以正聲應,無有忽微,不復與它律為役者,同心一統之義也。非黃鐘而它律,雖當其月自宮者,則其和應之律有空積忽微,不得其正。此黃鐘至尊,亡與並也。

《易》曰:「參天兩地而倚數。」天之數始於一,終於二十有五。其義紀之以三,故置一得三,又二十五分之六,凡二十五置,終天之數,得八十一,以天地五位之合終於十者乘之,為八百一十分,應曆一統千五百三十九歲之章數,黃鐘之實也。繇此之義,起十二律之周徑。地之數始於二,終於三十。其義紀之以兩,故置一得二,凡三十置,終地之數,得六十,以地中數六乘之,為三百六十分,當期之日,林鐘之實。人者,繼天順地,序氣成物,統八卦,調八風,理八政,正八節,諧八音,舞八佾,監八方,被八荒,以終天地之功,故八八六十四。其義極天地之變,以天地五位之合終於十者乘之,為六百四十分,以應六十四卦,大族之實也。《書》曰:「天功人其代之。」天兼地,人則天,故以五位之合乘焉,「唯天為大,唯堯則之」之象也。地以中數乘者,陰道理內,在中餽之象也。三統相通,故黃鐘、林鐘、太族律長皆全寸而亡餘分也。

天之中數五,地之中數六,而二者為合。六為虛,五為聲,周流於六虛。虛者,爻律夫陰陽,登降運行,列為十二,而律呂和矣。太極元氣,函三為一。極,中也。元,始也。行於十二辰,始動於子。參之於丑,得三。又參之於寅,得九。又參之於卯,得二十七。又參之於辰,得八十一。又參之於巳,得二百四十三。又參之於午,得七百二十九。又參之於未,得二千一百八十七。又參之於申,得六千五百六十一。又參之於酉,得萬九千六百八十三。又參之於戌,得五萬九千四十九。又參之於亥,得十七萬七千一百四十七。此陰陽合德,氣鐘於子,化生萬物者也。故孳萌於子,紐牙於丑,引達於寅,冒茆於卯,振美於辰,已盛於巳,咢布於午,昧薆於未,申堅於申,留孰於酉,畢入於戌,該閡於亥。出甲於甲,奮軋於乙,明炳於丙,大盛於丁,豐楙於戊,理紀於己,斂更於庚,悉新於辛,懷任於壬,陳揆於癸。故陰陽之施化,萬物之終始,既類旅於律呂,又經歷於日辰,而變化之情可見矣。

玉衡杓建,天之綱也;日月初纏,星之紀也。綱紀之交,以原始造設,合樂用焉。律呂唱和,以育生成化,歌奏用焉。指顧取象,然後陰陽萬物靡不條鬯該成。故以成之數忖該之積,如法為一寸,則黃鐘之長也。參分損一,下生林鐘。參分林鐘益一,上生太族。參分太族損一,下生南呂。參分南呂益一,上生姑洗。參分姑洗損一,下生應鐘。參分應鐘益一,上生蕤賓。參分蕤賓損一,下生大呂。參分大呂益一,上生夷則。參分夷則損一,下生夾鐘。參分夾鐘益一,上生亡射。參分亡射損一,下生中呂。陰陽相生,自黃鐘始而左旋,八八為伍。其法皆用銅。職在大樂,太常掌之。

度者,分、寸、尺、丈、引也,所以度長短也。本起黃鐘之長。以子穀秬黍中者,一黍之廣,度之九十分,黃鐘之長。一為一分,十分為寸,十寸為尺,十尺為丈,十丈為引,而五度審矣。其法用銅,高一寸,廣二寸,長一丈,而分寸尺丈存焉。用竹為引,高一分,廣六分,長十丈,其方法矩,高廣之數,陰陽之象也。分者,自三微而成著,可分別也。寸者,忖也。尺者,卺也。丈者,張也。引者,信也。夫度者,別於分,忖於寸,卺於尺,張於丈,信於引。引者,信天下也。職在內官,廷尉掌之。

量者,龠、合、升、斗、斛也,所以量多少也。本起於黃鐘之龠,用度數審其容,以子穀秬黍中者千有二百實其龠,以井水準其概。合龠為合,十合為升,十升為斗,十斗為斛,而五量嘉矣。其法用銅,方尺而圜其外,旁有庣焉。其上為斛,其下為斗。左耳為升,右耳為合龠。其狀似爵,以縻爵祿。上三下二,參天兩地,圜而函方,左一右二,陰陽之象也。其圜象規,其重二鈞,備氣物之數,合萬有一千五百二十。聲中黃鐘,始於黃鐘而反覆焉,君制器之象也。龠者,黃鐘律之實也,躍微動氣而生物也。合者,合龠之量也。升者,登合之量也。斗者,聚升之量也。斛者,角斗平多少之量也。夫量者,躍於龠,合於合,登於升,聚於斗,角於斛也。職在太倉,大司農掌之。

衡權者,衡,平也,權,重也,衡所以任權而均物平輕重也。其道如底,以見準之正,繩之直,左旋見規,右折見矩。其在天也,佐助旋機,斟酌建指,以齊七政,故曰玉衡。論語云:「

立則見其參於前也,在車則見其倚於衡也。」又曰:「齊之以禮。」此衡在前居南方之義也。

權者,銖、兩、斤、鈞、石也,所以稱物平施,知輕重也。本起於黃鐘之重。一龠容千二百黍,重十二銖,兩之為兩。二十四銖為兩。十六兩為斤。三十斤為鈞。四鈞為石。忖為十八,易十有八變之象也。五權之制,以義立之,以物鈞之,其餘小大之差,以輕重為宜。圜而環之,令之肉倍好者,周旋無端,終而復始,無窮已也。銖者,物繇忽微始,至於成著,可殊異也。兩者,兩黃鐘律之重也。二十四銖而成兩者,二十四氣之象也。斤者,明也,三百八十四銖,易二篇之爻,陰陽變動之象也。十六兩成斤者,四時乘四方之象也。鈞者,均也,陽施其氣,陰化其物,皆得其成就平均也。權與物均,重萬一千五百二十銖,當萬物之象也。四百八十兩者,六旬行八節之象也。三十斤成鈞者,一月之象也。石者,大也,權之大者也。始於銖,兩於兩,明於斤,均於鈞,終於石,物終石大也。四鈞為石者,四時之象也。重百二十斤者,十二月之象也。終於十二辰而復於子,黃鐘之象也。千九百二十兩者,陰陽之數也。三百八十四爻,五行之象也。四萬六千八十銖者,萬一千五百二十物歷四時之象也。而歲功成就,五權謹矣。

權與物鈞而生衡,衡運生規,規圜生矩,矩方生繩,繩直生準,準正則平衡而鈞權矣。是為五則。規者,所以規圜器械,令得其類也。矩者,所以矩方器械,令不失其形也。規矩相須,陰陽位序,圜方乃成。準者,所以揆平取正也。繩者,上下端直,經緯四通也。準繩連體,衡權合德,百工繇焉,以定法式,輔弼執玉,以翼天子。《詩》云:「尹氏大師,秉國之鈞,四方是維,天子是毗,俾民不迷。」咸有五象,其義一也。以陰陽言之,大陰者,北方。北,伏也,陽氣伏於下,於時為冬。冬,終也,物終臧,乃可稱。水潤下。知者謀,謀者重,故為權也。大陽者,南方。南,任也,陽氣任養物,於時為夏。夏,假也,物假大,乃宣平。火炎上。禮者齊,齊者平,故為衡也。少陰者,西方。西,遷也,陰氣遷落物,於時為秋。秋,胆也,物呙斂,乃成孰。金從革,改更也。義者成,成者方,故為矩也。少陽者,東方。東,動也,陽氣動物,於時為春。春,蠢也,物蠢生,乃動運。木曲直。仁者生,生者圜,故為規也。中央者,陰陽之內,四方之中,經緯通達,乃能端直,於時為四季。土稼嗇蕃息。信者誠,誠者直,故為繩也。五則揆物,有輕重圜方平直陰陽之義,四方四時之體,五常五行之象。厥法有品,各順其方而應其行。職在大行,鴻臚掌之。

《書》曰:「予欲聞六律、五聲、八音、七始詠,以出內五言,女聽。」予者,帝舜也。言以律呂和五聲,施之八音,合之成樂。七者,天地四時人之始也。順以歌詠五常之言,聽之則順乎天地,序乎四時,應人倫,本陰陽,原情性,風之以德,感之以樂,莫不同乎一。唯聖人為能同天下之意,故帝舜欲聞之也。今廣延群儒,博謀講道,修明舊典,同律,審度,嘉量,平衡,鈞權,正準,直繩,立于五則,備數和聲,以利兆民,貞天下於一,同海內之歸。凡律度量衡用銅者,名自名也,所以同天下,齊風俗也。銅為物之至精,不為燥溼寒暑變其節,不為風雨暴露改其形,介然有常,有似於士君子之行,是以用銅也。用竹為引者,事之宜也。

曆數之起上矣。傳述顓頊命南正重司天,火正黎司地,其後三苗亂德,二官咸廢,而閏餘乖次,孟陬殄滅,攝提失方。堯復育重、黎之後,使纂其業,故《書》曰:「乃命羲、和,欽若昊天,曆象日月星辰,敬授民時。」「歲三百有六旬有六日,以閏月定四時成歲,允釐百官,眾功皆美。」其後以授舜曰:「咨爾舜,天之曆數在爾躬。」「舜亦以命禹。」至周武王訪箕子,箕子言大法九章,而五紀明曆法。故自殷周,皆創業改制,咸正曆紀,服色從之,順其時氣,以應天道。三代既沒,五伯之末史官喪紀,疇人子弟分散,或在夷狄,故其所記,有黃帝、顓頊、夏、殷、周及魯曆。戰國擾攘,秦兼天下,未皇暇也,亦頗推五勝,而自以為獲水德,乃以十月為正,色上黑。

漢興,方綱紀大基,庶事草創,襲秦正朔。以北平侯張蒼言,用顓頊曆,比於六曆,疏闊中最為微近。然正朔服色,未睹其真,而朔晦月見,弦望滿虧,多非是。

至武帝元封七年,漢興百二歲矣,大中大夫公孫卿、壺遂、太史令司馬遷等言「曆紀壞廢,宜改正朔」。是時御史大夫兒寬明經術,上乃詔寬曰:「與博士共議,今宜何以為正朔?服色何上?」寬與博士賜等議,皆曰:「帝王必改正朔,易服色,所以明受命於天也。創業變改,制不相復,推傳序文,則今夏時也。臣等聞學褊陋,不能明。陛下躬聖發憤,昭配天地,臣愚以為三統之制,後聖復前聖者,二代在前也。今二代之統絕而不序矣,唯陛下發聖德,宣考天地四時之極,則順陰陽以定大明之制,為萬世則。」於是乃詔御史曰:「乃者有司言曆未定,廣延宣問,以考星度,未能讎也。蓋聞古者黃帝合而不死,名察發斂,定清濁,起五部,建氣物分數。然則上矣。書缺樂弛,朕甚難之。依違以惟,未能修明。其以七年為元年。」遂詔卿、遂、遷與侍郎尊、大典星射姓等議造漢曆。乃定東西,立晷儀,下漏刻,以追二十八宿相距於四方,舉終以定朔晦分至,躔離弦望。乃以前曆上元泰初四千六百一十七歲,至於元封七年,復得閼逢攝提格之歲,中冬十一月甲子朔旦冬至,日月在建星,太歲在子,已得太初本星度新正。姓等奏不能為算,願募治曆者,更造密度,各自增減,以造漢太初曆。乃選治曆鄧平及長樂司馬可、酒泉候宜君、侍郎尊及與民間治曆者,凡二十餘人,方士唐都、巴郡落下閎與焉。都分天部,而閎運算轉曆。其法以律起曆,曰:「律容一龠,積八十一寸,則一日之分也。與長相終。律長九寸,百七十一分而終復。三復而得甲子。夫律陰陽九六,爻象所從出也。故黃鐘紀元氣之謂律。律,法也,莫不取法焉。」與鄧平所治同。於是皆觀新星度、日月行,更以算推,如閎、平法。法,一月之日二十九日八十一分日之四十三。先藉半日,名曰陽曆;不藉,名曰陰曆。所謂陽曆者,先朔月生;陰曆者,朔而後月乃生。平曰:「陽曆朔皆先旦月生,以朝諸侯王群臣便。」乃詔遷用鄧平所造八十一分律曆,罷廢尤疏遠者十七家,復使校曆律昏明。宦者淳于陵渠復覆太初曆晦朔弦望,皆最密,日月如合璧,五星如連珠。陵渠奏狀,遂用鄧平曆,以平為太史丞。

後二十七年,元鳳三年,太史令張壽王上書言:「曆者天地之大紀,上帝所為。傳黃帝調律曆,漢元年以來用之。今陰陽不調,宜更曆之過也。」詔下主曆使者鮮于妄人詰問,壽王不服。妄人請與治曆大司農中丞麻光等二十餘人雜候日月晦朔弦望、八節二十四氣,鈞校諸曆用狀。奏可。詔與丞相、御史、大將軍、右將軍史各一人雜候上林清臺,課諸曆疏密,凡十一家。以元鳳三年十一月朔旦冬至,盡五年十二月,各有第。壽王課疏遠。案漢元年不用黃帝調曆,壽王非漢曆,逆天道,非所宜言,大不敬。有詔勿劾。復候,盡六年。太初曆第一,即墨徐萬且、長安徐禹治太初曆亦第一。壽王及待詔李信治黃帝調曆,課皆疏闊,又言黃帝至元鳳三年六千餘歲。丞相屬寶、長安單安國、安陵桮育治終始,言黃帝以來三千六百二十九歲,不與壽王合。壽王又移帝王錄,舜、禹年歲不合人年。壽王言化益為天子代禹,驪山女亦為天子,在殷周間,皆不合經術。壽王曆乃太史官殷曆也。壽王猥曰安得五家曆,又妄言太初曆虧四分日之三,去小餘七百五分,以故陰陽不調,謂之亂世。劾壽王吏八百石,古之大夫,服儒衣,誦不詳之辭,作祅言欲亂制度,不道。奏可。壽王候課,比三年下,終不服。再劾死,更赦勿劾,遂不更言,誹謗益甚,竟以下吏。故曆本之驗在於天,自漢曆初起,盡元鳳六年,三十六歲,而是非堅定。

至孝成世,劉向總六曆,列是非,作五紀論。向子歆究其微眇,作三統曆及譜以說春秋,推法密要,故述焉。

夫曆春秋者,天時也,列人事而目以天時。傳曰:「

民受天地之中以生,所謂命也。是故有禮誼動作威儀之則以定命也,能者養以之福,不能者敗以取禍。」故列十二公二百四十二年之事,以陰陽之中制其禮。故春為陽中,萬物以生;秋為陰中,萬物以成。是以事舉其中,禮取其和,曆數以閏正天地之中,以作事厚生,皆所以定命也。易金火相革之卦曰「湯武革命,順乎天而應乎人」,又曰「治曆明時」,所以和人道也。

周道既衰,幽王既喪,天子不能班朔,魯曆不正,以閏餘一之歲為蔀首。故春秋刺「十一月乙亥朔,日有食之」。於是辰在申,而司曆以為在建戌,史書建亥。哀十二年,亦以建申流火之月為建亥,而怪蟄蟲之不伏也。自文公閏月不告朔,至此百有餘年,莫能正曆數。故子貢欲去其餼羊,孔子愛其禮,而著其法於春秋。經曰:「冬十月朔,日有食之。」傳曰:「不書日,官失之也。天子有日官,諸侯有日御,日官居卿以厎日,禮也。日御不失日以授百官於朝。」言告朔也。元典曆始曰元。傳曰:「元,善之長也。」共養三德為善。又曰:「元,體之長也。」合三體而為之原,故曰元。於春三月,每月書王,元之三統也。三統合於一元,故因元一而九三之以為法,十一三之以為實。實如法得一。黃鐘初九,律之首,陽之變也。因而六之,以九為法,得林鐘初六,呂之首,陰之變也。皆參天兩地之法也。上生六而倍之,下生六而損之,皆以九為法。九六,陰陽夫婦子母之道也。律娶妻而呂生子,天地之情也。六律六呂,而十二辰立矣。五聲清濁,而十日行矣。傳曰「天六地五」,數之常也。天有六氣,降生五味。夫五六者,天地之中合,而民所受以生也。故日有六甲,辰有五子,十一而天地之道畢,言終而復始。太極中央元氣,故為黃鐘,其實一龠,以其長自乘,故八十一為日法,所以生權衡度量,禮樂之所繇出也。經元一以統始,易太極之首也。春秋二以目歲,易兩儀之中也。於春每月書王,易三極之統也。於四時雖亡事必書時月,易四象之節也。時月以建分至啟閉之分,易八卦之位也。象事成敗,易吉凶之效也。朝聘會盟,易大業之本也。故易與春秋,天人之道也。傳曰:「龜,象也。筮,數也。物生而後有象,象而後有滋,滋而後有數。」

是故元始有象一也,春秋二也,三統三也,四時四也,合而為十,成五體。以五乘十,大衍之數也,而道據其一,其餘四十九,所當用也,故蓍以為數。以象兩兩之,又以象三三之,又以象四四之,又歸奇象閏十九及所據一加之,因以再扐兩之,是為月法之實。如日法得一,則一月之日數也,而三辰之會交矣,是以能生吉凶。故《易》曰:「天一地二,天三地四,天五地六,天七地八,天九地十。天數五,地數五,五位相得而各有合。天數二十有五,地數三十,凡天地之數五十有五,此所以成變化而行鬼神也。」并終數為十九,易窮則變,故為閏法。參天九,兩地十,是為會數。參天數二十五,兩地數三十,是為朔望之會。以會數乘之,則周於朔旦冬至,是為會月。九會而復元,黃鐘初九之數也。經於四時,雖亡事必書時月。時所以記啟閉也,月所以紀分至也。啟閉者,節也。分至者,中也。節不必在其月,故時中必在正數之月。故傳曰:「先王之正時也,履端於始,舉正於中,歸餘於終。履端於始,序則不愆;舉正於中,民則不惑;歸餘於終,事則不誖。」此聖王之重閏也。以五位乘會數,而朔旦冬至,是為章月。四分月法,以其一乘章月,是為中法。參閏法為周至,以乘月法,以減中法而約之,則六扐之數,為一月之閏法,其餘七分,此中朔相求之術也。朔不得中,是謂閏月,言陰陽雖交,不得中不生。故日法乘閏法,是為統歲。三統,是為元歲。元歲之閏,陰陽災,三統閏法。易九厄曰:初入元,百六,陽九;次三百七十四,陰九;次四百八十,陽九;次七百二十,陰七;次七百二十,陽七;次六百,陰五;次六百,陽五;次四百八十,陰三;次四百八十,陽三。凡四千六百一十七歲,與一元終。經歲四千五百六十,災歲五十七。是以春秋曰:「舉正於中。」又曰:「閏月不告朔,非禮也。閏以正時,時以作事,事以厚生,生民之道於是乎在矣。不告閏朔,棄時正也,何以為民?」故善僖「五年春王正月辛亥朔,日南至,公既視朔,遂登觀臺以望,而書,禮也。凡分至啟閉,必書雲物,為備故也。」至昭二十年二月己丑,日南至,失閏,至在非其月。梓慎望氛氣而弗正,不履端於始也。故傳不曰冬至,而曰日南至。極於牽牛之初,日中之時景最長,以此知其南至也。

斗綱之端連貫營室,織女之紀指牽牛之初,以紀日月,故曰星紀。五星起其初,日月起其中,凡十二次。日至其初為節,至其中斗建下為十二辰。視其建而知其次。故曰「制禮上物,不過十二,天之大數也」。經曰春王正月,傳曰周正月「火出,於夏為三月,商為四月,周為五月。夏數得天」,得四時之正也。三代各據一統,明三統常合,而迭為首,登降三統之首,周還五行之道也。故三五相包而生。天統之正,始施於子半,日萌色赤。地統受之於丑初,日肇化而黃,至丑半,日牙化而白。人統受之於寅初,日孽成而黑,至寅半,日生成而青。天施復於子,地化自丑畢於辰,人生自寅成於申。故曆數三統,天以甲子,地以甲辰,人以甲申。孟仲季迭用事為統首。三微之統既著,而五行自青始,其序亦如之。五行與三統相錯。傳曰「天有三辰,地有五行」,然則三統五星可知也。《易》曰:「參五以變,錯綜其數。通其變,遂成天下之文;極其數,遂定天下之象。」太極運三辰五星於上。而元氣轉三統五行於下。其於人,皇極統三德五事。故三辰之合於三統也,日合於天統,月合於地統,斗合於人統。五星之合於五行,水合於辰星,火合於熒惑,金合於太白,木合於歲星,土合於填星。三辰五星而相經緯也。天以一生水,地以二生火,天以三生木,地以四生金,天以五生土。五勝相乘,以生小周,以乘乾坤之策,而成大周。陰陽比類,交錯相成,故九六之變登降於六體。三微而成著,三著而成象,二象十有八變而成卦,四營而成易,為七十二,參三統兩四時相乘之數也。參之則得乾之策,兩之則得坤之策。以陽九九之,為六百四十八,以陰六六之,為四百三十二,凡一千八十,陰陽各一卦之微算策也。八之,為八千六百四十,而八卦小成。引而信之,又八之,為六萬九千一百二十,天地再之,為十三萬八千二百四十,然後大成。五星會終,觸類而長之,以乘章歲,為二百六十二萬六千五百六十,而與日月會。三會為七百八十七萬九千六百八十,而與三統會。三統二千三百六十三萬九千四十,而復於太極上元。九章歲而六之為法,太極上元為實,實如法得一,陰一陽各萬一千五百二十,當萬物氣體之數,天下之能事畢矣。

統母日法八十一。元始黃鐘初九自乘,一龠之數,得日法。

閏法十九,因為章歲。合天地終數,得閏法。

統法千五百三十九。以閏法乘日法,得統法。

元法四千六百一十七。參統法,得元法。

會數四十七。參天九,兩地十,得會數。

章月二百三十五。五位乘會數,得章月。

月法二千三百九十二。推大衍象,得月法。

通法五百九十八。四分月法,得通法。

中法十四萬五百三十。以章月乘通法,得中法。

周天五十六萬二千一百二十。以章月乘月法,得周天。

歲中十二。以三統乘四時,得歲中。

月周二百五十四。以章月加閏法,得月周。

朔望之會百三十五。參天數二十五,兩地數三十,得朔望之會。

會月六千三百四十五。以會數乘朔望之會,得會月。

統月萬九千三十五。參會月,得統月。

元月五萬七千一百五。參統月,得元月。

章中二百二十八。以閏法乘歲中,得章中。

統中萬八千四百六十八。以日法乘章中,得統中。

元中五萬五千四百四。參統中,得元中。

策餘八千八十。什乘元中,以減周天,得策餘。

周至五十七。參閏法,得周至。

統母。

木金相乘為十二,是為歲星小周。小周乘巛策,為千七百二十八,是為歲星歲數。

見中分二萬七百三十六。

積中十三,中餘百五十七。

見中法千五百八十三。

見閏分萬二千九十六。

積月十三,月餘萬五千七十九。

見月法三萬七十七。

見中日法七百三十萬八千七百一十一。

見月日法二百四十三萬六千二百三十七。

金火相乘為八,又以火乘之為十六而小復。小復乘乾策,為三千四百五十六,是為太白歲數。

見中分四萬一千四百七十二。

積中十九,中餘四百一十三。

見中法二千一百六十一。

見閏分二萬四千一百九十二。

積月十九,月餘三萬二千三十九。

見月法四萬一千五十九。

晨中分二萬三千三百二十八。

積中十,中餘千七百一十八。「十」一作「七」

夕中分萬八千一百四十四。

積中八,中餘八百五十六。

晨閏分萬三千六百八。

積月十一,月餘五千一百九十一。

夕閏分萬五百八十四。

積月八,月餘二萬六千八百四十八。

見中日法九百九十七萬七千三百三十七。

見月日法三百三十二萬五千七百七十九。

土木相乘而合經緯為三十,是為鎮星小周。小周乘巛策,為四千三百二十,是為鎮星歲數。

見中分五萬一千八百四十。

積中十二,中餘千七百四十。

見中法四千一百七十五。

見閏分三萬二百四十。

積月十二,月餘六萬三千三百。

見月法七萬九千三百二十五。

見中日法千九百二十七萬五千九百七十五。

見月日法六百四十二萬五千三百二十五。

火經特成,故二歲而過初,三十二過初為六十四歲而小周。小周乘乾策,則太陽大周,為萬三千八百二十四歲,是為熒惑歲數。

見中分十六萬五千八百八十八。

積中二十五,中餘四千一百六十三。

見中法六千四百六十九。

見閏分九萬六千七百六十八。

積月二十六,月餘五萬二千九百五十四。

見月法十二萬二千九百一十一。「二千」一作「一千」

見中日法二千九百八十六萬七千三百七十三。

見月日法九百九十五萬五千七百九十一。

水經特成,故一歲而及初,六十四及初而小復。小復乘巛策,則太陰大周,為九千二百一十六歲,是為辰星歲數。

見中分十一萬五百九十二。

積中三,中餘三萬二千四百六十九。

見中法二萬九千四十一。

見閏分六萬四千五百一十二。

積月三,月餘五十一萬四百二十三。

見月法五十五萬一千七百七十九。

晨中分六萬二千二百八。

積中二,中餘四千一百二十六。

夕中分四萬八千三百八十四。

積中一,中餘萬九千三百四十三。

晨閏分三萬六千二百八十八。

積月二,月餘十一萬四千六百八十二。

夕閏分二萬八千二百二十四。

積月一,月餘三十九萬五千七百四十一。

見中日法一億三千四百八萬二千二百九十七。

見月日法四千四百六十九萬四千九十九。

合太陰太陽之歲數而中分之,各萬一千五百二十。陽施其氣,陰成其物。

以星行率減歲數,餘則見數也。

東九西七乘歲數,并九七為法,得一,金、水晨夕歲數。

以歲中乘歲數,是為星見中分。

星見數,是為見中法。

以歲閏乘歲數,是為星見閏分。

以章歲乘見數,是為見月法。

以元法乘見數,是為見中日法。

以統法乘見數,是為見月日法。

五步木,晨始見,去日半次。順,日行十一分度二,百二十一日。始留,二十五日而旋。逆,日行七分度一,八十四日。復留,二十四日三分而旋。復順,日行十一分度二,百一十一日有百八十二萬八千三百六十二分而伏。凡見三百六十五日有百八十二萬八千三百六十五分,除逆,定行星三十度百六十六萬一千二百八十六分。凡見一歲,行一次而後伏。日行不盈十一分度一。伏三十三日三百三十三萬四千七百三十七分,行星三度百六十七萬三千四百五十一一作「三」分。一見,三百九十八日五百一十六萬三千一百二分,行星三十三度三百三十三萬四千七百三十七分。通其率,故曰日行千七百二十八分度之百四十五。

金,晨始見,去日半次。逆,日行二分度一,六日。始留,八日而旋。始順,日行四十六分度三十三,四十六日。順,疾,日行一度九十二分度十五,百八十四日而伏。凡見二百四十四日,除逆,定行星二百四十四度。伏,日行一度九十二分度三十三有奇。伏八十三日,行星百一十三度四百三十六萬五千二百二十分。凡晨見、伏三百二十七日,行星三百五十七度四百三十六萬五千二百二十分。夕始見,去日半次。順,日行一度九十二分度十五,百八十一日百七分日四十五。順,遲,日行四十六分度三一作「四」十三,四十六日。始留,七日百七分日六十二分而旋。逆,日行二一作「三」分度一,六日而伏。凡見二百四十一日,除逆,定行星二百四十一度。伏,逆,日行八分度七有奇。伏十六「一作六十」日百二十九萬五千三百五十二分,行星十四度三百六萬九千八百六十八分。一凡夕見伏,二百五十七日百二十九萬五千三百五十二一作「一」分,行星二百二十六度六百九十萬七千四百六十九分。一復,五百八十四日百二十九萬五千三百五十二分。行星亦如之,故曰日行一度。

土,晨始見,去日半次。順,日行十五分度一,八十七日,始留,三十四日而旋。逆,日行八十一分度五,百一日。復留,三十三日八十六萬二千四百五十五分而旋。復順,日行十五分度一,八十五日而伏。凡見三百四十日八十六萬二千四百五十五分,除逆,定一多「餘」字行星五度四百四十七萬三千九百三十分。伏,日行不盈十五分度三。百三十七日千七百一十七萬一百七十分,行星七度八百七十三萬六千五百七十分。一見,三百七十七日千八百三萬二千六百二十五分,行星十二度千三百二十一萬五百分。通其率,故曰日行四千三百二十分度之百四十五。

火,晨始見,去日半次。順,日行九十二分度五十三,二百七十六日,始留,十日而旋。逆,日行六十二分度十七,六十二日。復留,十日而旋。復順,日行九十二分度五十三,二百七十六日而伏。凡見六百三十四日,除逆,定行星三百一度。伏,日行不盈九十二分度七十三分,伏百四十六日千五百六十八萬九千七百分,行星百一十四度八百二十一萬八千五分。一見,七百八十日千五百六十八萬九千七百分,凡行星四百一十五度八百二十一萬八千五分。通其率,故曰日行萬三千八百二十四分度之七千三百五十五。

水,晨始見,去日半次。逆,日行二度,一日。始留,二日而旋。順,日行七分度六,一多「十」字七日。順,疾,日行一度三分度一,一多「一」字十八日而伏。凡見二十八日,除逆,定行星二十八度。伏,日行一度九分度七有奇,三十七日一億二千二百二萬九千六百五分,行星六十八度四千六百六十一萬一百二十八分。凡晨見、伏,六十五日一億二千二百二萬九千六百五分,行星九十六度四千六百六十一萬一百二十八分。夕始見,去日半次。順,疾,日行一度三分度一,十六日二分日一。順,遲,日行七分度六,七一作「十」日。留,一日二分日一而旋。逆,日行二度,一日而伏。凡見二十六日,除逆,定行星二十六度。伏,逆,日行十五分度四有奇,二十四日,行星六度五千八百六十六萬二千八百二十分。凡夕見伏,五十日,行星十九度七千五百四十一萬九千四百七十七分。一復,百一十五日一億二千二百二萬九千六百五分。行星亦如之,故曰日行一度。

統術推日月元統,置太極上元以來,外所求年,盈元法除之,餘不盈統者,則天統甲子以來年數也。盈統,除之,餘則地統甲辰以來年數也。又盈統,除之,餘則人統甲申以來年數也。各以其統首日為紀。

推天正,以章月乘人統歲數,盈章歲得一,名曰積月,不盈者名曰閏餘。閏餘十二以上,歲有閏。求地正,加積月一;求人正,加二。

推正月朔,以月法乘積月,盈日法得一,名曰積日,不盈者名曰小餘。小餘三十八以上,其月大。積日盈六十,除之,不盈者名曰大餘。數從統首日起,算外,則朔日也。求其次月,加大餘二十九,小餘四十三。小餘盈日法得一,從大餘,數除如法。求弦,加大餘七,小餘三十一。求望,倍弦。

推閏餘所在,以十二乘閏餘,加十得一。盈章中,數所得,起冬至,算外,則中至終閏盈。中氣在朔若二日,則前月閏也。

推冬至,以算餘乘人統歲數,盈統法得一,名曰大餘,不盈者名曰小餘。除數如法,則所求冬至日也。

求八節,加大餘四十五,小餘千一百。求二十四氣,三其小餘,加大餘十五,小餘千一十。

推中部二十四氣,皆以元為法。

推五行,其四行各七十三日,統歲分之七十七。中央各十八日,統法分之四百四。冬至後,中央二十七日六百六分。

推合晨所在星,置積日,以統法乘之,以十九乘小餘而并之。盈周天,除去之;不盈者,令盈統法得一度。數起牽牛,算外,則合晨所入星度也。

推其日夜半所在星,以章歲乘月小餘,以減合晨度。小餘不足者,破全度。

推其月夜半所在星,以月周乘月小餘,盈統法得一度,以減合晨度。

推諸加時,以十二乘小餘為實,各盈分母為法,數起於子,算外,則所加辰也。

推月食,置會餘歲積月,以二十三乘之,盈百三十五,除之。不盈者,加二十三得一月,盈百三十五,數所得,起其正,算外,則食月也。加時,在望日衝辰。

紀術推五星見復,置太極上元以來,盡所求年,乘大統見復數,盈歲數得一,則定見復數也。不盈者名曰見復餘。見復餘盈其見復數,一以上見在往年,倍一以上,又在前往年,不盈者在今年也。

推星所一多「在」字見中次,以見中分乘定見復數,盈見中法得一,則積中法也。不盈者名曰中餘。以元中除積中,餘則中元餘也。以章中除之,餘則入章中數也。以十二除之,餘則星見中次也。中數從冬至起,次數從星紀起,算外,則星所見中次也。

推星見月,以閏分乘定見,以章歲乘中餘從之,盈見月法得一,并積中,則積月也。不盈者名曰月中餘。以元月除積月餘,名曰月元餘。以章月除月元餘,則入章月數也。以十二除之,至有閏之歲,除十三入章。三歲一閏,六歲二閏,九歲三閏,十一歲四閏,十四歲五閏,十七歲六閏,十九歲七閏。不盈者數起於天正,算外,則星所見月也。

推至日,以中法乘中元餘,盈元法得一,名曰積日,不盈者名曰小餘。小餘盈二千五百九十七以上,中大。數除積日如法,算外,則冬至也。

推朔日,以月法乘月元餘,盈日法得一,名曰積日,餘名曰小餘。小餘三十八以上,月大。數除積日如法,算外,則星見月朔日也。

推入中次日度數,以中法乘中餘,以見中法乘其小餘并之,盈見中日法得一,則入中日入次度數也。中次至日數,次以次初數,算外,則星所見及日所在度數也。求夕,在日後十五度。

推入月日數,以月法乘月餘,以見月法乘其小餘并之,盈見月日法得一,則入月日數也。并之大餘,數除如法,則見日也。

推後見中,加積中於中元餘,加後餘於中餘,盈其法得一,從中元餘,數如法,則見也。

推後見月,加積月於月元餘,加後月餘於月餘,盈其法得一,從月元餘,除數如法,則後見月也。

推至日及入中次度數,如上法。

推朔日及入月數,如上法

推晨見加夕,夕見加晨,皆如上法。

推五步,置始見以來日數,至所求日,各以其行度數乘之。其星若日有分者,分子乘全為實,分母為法。其兩有分者,分母分度數乘全,分子從之,令相乘為實,分母相乘為法,實如法得一,名曰積度。數起星初見星宿所在宿度,算外,則星所在宿度也。

歲術推歲所在,置上元以來,外所求年,盈歲數,除去之,不盈者以百四十五乘之,以百四十四為法,如法得一,名曰積次,不盈者名曰次餘。積次盈十二,除去之,不盈者名曰定次。數從星紀起,算盡之外,則所在次也。欲知太歲,以六十除餘積次,餘不盈者,數從丙子起,算盡之外,則太歲日也。

贏縮。傳曰:「歲棄其次而旅於明年之次,以害鳥帑,周楚惡之。」五星之盈縮不是過也。過次者殃大,過舍者災小,不過者亡咎。次度。六物者,歲時數日月星辰也。辰者,日月之會而建所指也。

星紀,初斗十二度,大雪。中牽牛初,冬至。終於婺女七度。

玄枵,初婺女八度,小寒。中危初,大寒。終於危十五度。

諏訾,初危十六度,立春。中營室十四度,驚蟄。終於奎四度。

降婁,初奎五度,雨水。中婁四度,春分。終於胃六度。

大梁,初胃七度,穀雨。中昴八度,清明。終於畢十一度。

實沈、初畢十二度,立夏。中井初,小滿。終於井十五度。

鶉首,初井十六度,芒種。中井三十一度,夏至。終於柳八度。

鶉火,初柳九度,小暑。中張三度,大暑。終於張十七度。

鶉尾,初張十八度,立秋。中翼十五度,處暑。

終於軫十一度。

壽星,初軫十二度,白露。中角十度,秋分。終於氐四度。

大火,初氐五度,寒露。中房五度,霜降。終於尾九度。

析木,初尾十度,立冬。中箕七度,小雪。終於斗十一度。

角十二。亢九。氐十五。房五。心五。尾十八。箕十一。東七十五度。斗二十六。牛八。女十二。虛十。危十七。營室十六。壁九。北九十八度。奎十六。婁十二。胃十四。昴十一。畢十六。觜二。參九。西八十度。井三十三。鬼四。柳十五。星七。張十八。翼十八。軫十七。南百一十二度。

九章歲為百七十一歲,而九道小終。九終千五百三十九歲而大終。三終而與元終。進退於牽牛之前四度五分。九會。陽以九終,故日有九道。陰兼而成之,故月有十九道。陽名成功,故九會而終。四營而成易,故四歲中餘一,四章而朔餘一,為篇首,八十一章而終一統。

一,甲子元首。十,辛酉。十九,己未。二十八,丁巳。三十七,乙卯。四十六,壬子。五十五,庚戌。六十四,戊申。七十三,丙午,中。

甲辰二統。辛丑。己亥。丁酉。乙未。壬辰。庚寅。戊子。丙戌,季。

甲申三統。辛巳。己卯。丁丑。乙亥。壬申。庚午。戊辰。丙寅,孟。

二,癸卯。十一,辛丑。二十,己亥。二十九,丁酉。二十八,甲午。四十七,壬辰。五十六,庚寅。六十五,戊子。七十四,乙酉,中。

癸未。辛巳。己卯。丁丑。甲戌。壬申。庚午。戊辰。乙丑,季。

癸亥。辛酉。己未。丁巳。甲寅。壬子。庚戌。戊申乙巳,孟。

三,癸未。十二,辛巳。二十一,己卯。三十,丙子。三十九,甲戌。四十八,壬申。五十七,庚子。六十六,丁卯。七十五,乙丑,中。

癸亥。辛酉。己未。丙辰。甲寅。壬子。庚戌。丁未。乙巳,季。

癸卯。辛丑。己亥。丙申。甲午。壬辰。庚寅。丁亥。乙酉,孟。

四,癸亥。

十三,辛酉。二十二,戊午。三十一,丙辰。四十,甲寅。四十九,壬子。五十八,己酉。六十七,丁未。七十六,乙巳,中。

癸卯。辛丑。戊戌。丙申。甲午。壬辰。己丑。丁亥。乙酉,季。

癸未。辛巳。戊寅。丙子。甲戌。壬申。己巳。丁卯。乙丑,孟。

五,癸卯。

十四,庚子。二十三,戊戌。三十二,丙申。四十一,甲午。五十,辛卯。五十九,己丑。六十八,丁亥。七十七,乙酉,中。

癸未。庚辰。戊寅。丙子。甲戌。辛未。己巳。丁卯。乙丑,季。

癸亥。庚申。戊午。丙辰。甲寅。辛亥。己酉。丁未。乙巳,孟。

六,壬午。十五,庚辰。二十四,戊寅。三十三,丙子。四十二,癸酉。五十一,辛未。六十,己巳。六十九,丁卯。七十八,甲子,中。

壬戌。庚申。戊午。丙辰。癸丑。辛亥。己酉。丁未。甲辰,季。

壬寅。庚子。戊戌。丙申。癸巳。辛卯。己丑。丁亥。甲申,孟。

七,壬戌。

十六,庚申。二十五,戊午。三十四,乙卯。四十三,癸丑。五十二,辛亥。六十一,己酉。七十,丙午。七十九,甲辰,中。

壬寅。庚子。戊戌。乙未。癸巳。辛卯。己丑。丙戌。甲申,季。

壬午。庚辰。戊寅。乙亥。癸酉。辛未。己巳。丙寅。甲子,孟。

八,壬寅。十七,庚子。二十六,丁酉。三十五,乙未。四十四,癸巳。五十三,辛卯。六十二,戊子。七十一,丙戌。八十,甲申,中。

壬午。庚辰。丁丑。乙亥。癸酉。辛未。戊辰。丙寅。甲子,季。

壬戌。庚申。丁巳。乙卯。癸丑。辛亥。戊申。丙午。甲辰,孟。

九,壬午。十八,己卯。二十七,丁丑。三十六,乙亥。四十五,癸酉。五十四,庚午。六十三,戊辰。七十二,丙寅。八十一,甲子,中。

壬戌。己未。丁巳。乙卯。癸丑。庚戌。戊申。丙午。甲辰,季。

壬寅。己亥。丁酉。乙未。癸巳。庚寅。戊子。丙戌。甲申,孟。

推章首朔旦冬至日,置大餘三十九,小餘六十一,數除如法,各從其統首起。求其後章,當加大餘三十九,小餘六十一,各盡其八十一章。

推篇,大餘亦如之,小餘加一。求周至,加大餘五十九,小餘二十一。

世經春秋昭公十七年「郯子來朝」,傳曰昭子問少昊氏鳥名何故,對曰:「吾祖也,我知之矣。昔者,黃帝氏以雲紀,故為雲師而雲名;炎帝氏以火紀,故為火師而火名;共工氏以水紀,故為水師而水名;太昊氏以龍紀,故為龍師而龍名。我高祖少昊縶之立也,鳳鳥適至,故紀於鳥,為鳥師而鳥名。」言郯子據少昊受黃帝,黃帝受炎帝,炎帝受共工,共工受太昊,故先言黃帝,上及太昊。稽之於易,炮犧、神農、黃帝相繼之世可知。

太昊帝易曰:「炮犧氏之王天下也。」言炮犧繼天而王,為百王先,首德始於木,故為帝太昊。作罔罟以田漁,取犧牲,故天下號曰炮犧氏。祭典曰:「共工氏伯九域。」言雖有水德,在火木之間,非其序也。任知刑以彊,故伯而不王。秦以水德,在周、漢木火之間。周人俣其行序,故易不載。

炎帝易曰:「炮犧氏沒,神農氏作。」言共工伯而不王,雖有水德,非其序也。以火承木,故為炎帝。教民耕農,故天下號曰神農氏。

黃帝易曰:「神農氏沒,黃帝氏作。」火生土,故為土德。與炎帝之後戰於阪泉,遂王天下。始垂衣裳,有軒冕之服,故天下號曰軒轅氏。

少昊帝考德曰少昊曰清。清者,黃帝之子清陽也,是其子孫名摯立。土生金,故為金德,天下號曰金天氏。周俣其樂,故易不載,序於行。

顓頊帝春秋外傳曰,少昊之衰,九黎亂德,顓頊受之,乃命重黎。蒼林昌意之子也。金生水,故為水德。天下號曰高陽氏。周俣其樂,故易不載,序於行。

帝嚳春秋外傳曰,顓頊之所建,帝嚳受之。清陽玄囂之孫也。水生木,故為木德。天下號曰高辛氏。帝摯繼之,不知世數。周俣其樂,故易不載。周人禘之。

唐帝帝系曰,帝嚳四妃,陳豐生帝堯,封於唐。蓋高辛氏衰,天下歸之。木生火,故為火德,天下號曰陶唐氏。讓天下於虞,使子朱處于丹淵為諸侯。即位七十載。

虞帝帝系曰,顓頊生窮蟬,五世而生瞽叟,瞽叟生帝舜,處虞之媯汭,堯嬗以天下。火生土,故為土德。天下號曰有虞氏。讓天下於禹,使子商均為諸侯。即位五十載。

伯禹帝系曰,顓頊五世而生鯀,鯀生禹,虞舜嬗以天下。土生金,故為金德。天下號曰夏后氏。繼世十七王,四百三十二歲。

成湯書經湯誓湯伐夏桀。金生水,故為水德。天下號曰商,後曰殷。

三統,上元至伐桀之歲,十四萬一千四百八十歲,歲在大火房五度,故傳曰:「大火,閼伯之星也,實紀商人。」後為成湯,方即世崩沒之時,為天子用事十三年矣。商十二月乙丑朔旦冬至,故書序曰:「成湯既沒,太甲元年,使伊尹作伊訓。」伊訓篇曰:「惟太甲元年十有二月乙丑朔,伊尹祀于先王,誕資有牧方明。」言雖有成湯、太丁、外丙之服,以冬至越茀祀先王于方明以配上帝,是朔旦冬至之歲也。後九十五歲,商十二月甲申朔旦冬至,亡餘分,是為孟統。自伐桀至武王伐紂,六百二十九歲,故傳曰殷「載祀六百」。

殷曆曰,當成湯方即世用事十三年,十一月甲子朔旦冬至,終六府首。當周公五年,則為距伐桀四百五十八歲,少百七十一歲,不盈六百二十九。又以夏時乙丑為甲子,計其年乃孟統後五章,癸亥朔旦冬至也。以為甲子府首,皆非是。凡殷世繼嗣三十一王,六百二十九歲。

四分,上元至伐桀十三萬二千一百一十三歲,其八十八紀,甲子府首,入伐桀後百二十七歲。

春秋曆,周文王四十二年十二月丁丑朔旦冬至,孟統之二會首也。後八歲而武王伐紂。

武王書經牧誓武王伐商紂。水生木,故為木德。天下號曰周室。

三統,上元至伐紂之歲,十四萬二千一百九歲,歲在鶉火張十三度。文王受命九年而崩,再期,在大祥而代紂,故書序曰:「惟十有一年,武王伐紂,太誓。」八百諸侯會。還歸二年,乃遂伐紂克殷,以箕子歸,十三年也。故書序曰:「武王克殷,以箕子歸,作洪範。」洪範篇曰:「惟十有三祀,王訪于箕子。」自文王受命而至此十三年,歲亦在鶉火,故傳曰:「歲在鶉火,則我有周之分野也。」師初發,以殷十一月戊子,日在析木箕七度,故傳曰:「日在析木。」是夕也,月在房五度。房為天駟,故傳曰:「月在天駟。」後三日得周正月辛卯朔,合辰在斗前一度,斗柄也,故傳曰:「辰在斗柄。」明日壬辰,晨星始見。癸巳武王始發,丙午還師,戊午度于孟津。孟津去周九百里,師行三十里,故三十一日而度。明日己未冬至,坻星與婺女伏,歷建星及牽牛,至於婺女天黿之首,故傳曰:「星在天黿。」周書武成篇:「惟一月壬辰,旁死霸,若翌日癸巳,武王乃朝步自周,于征伐紂。」序曰:「一月戊午,師度于孟津。」至庚申,二月朔日也。四日癸亥,至牧野,夜陳,甲子昧爽而合矣。故外傳曰:「王以二月癸亥夜陳。」武成篇曰:「粵若來三月,既死霸,粵五日甲子,咸劉商王紂。」是歲也,閏數餘十八,正大寒中,在周二月己丑晦。明日閏月庚寅朔。三月二日庚申驚蟄。四月己丑朔死霸。死霸,朔也。生霸,望也。是月甲辰望,乙巳,旁之。故武成篇曰:「惟四月既旁生霸,粵六日庚戌,武王燎于周廟。翌日辛亥,祀于天位。粵五日乙卯,乃以庶國祀馘于周廟。」文王十五而生武王,受命九年而崩,崩後四年而武王克殷。克殷之歲八十六矣,後七歲而崩。故禮記文王世子曰:「文王九十七而終,武王九十三而終。」凡武王即位十一年,周公攝政五年,正月丁巳朔旦冬至,殷曆以為六年戊午,距煬公七十六歲,入孟統二十九章首也。後二歲,得周公七年「復子明辟」之歲。是歲二月乙亥朔,庚寅望,後六日得乙未。故召誥曰:「惟二月既望,粵六日乙未。」又其三月甲辰朔,三日丙午。召誥曰:「惟三月丙午朏。」古文月采篇曰「三日曰朏」。是歲十二月戊辰晦,周公以反政。故洛誥篇曰:「戊辰,王在新邑,烝祭歲,命作策,惟周公誕保文武受命,惟七年。」

成王元年正月己巳朔,此命伯禽俾侯于魯之歲也。後三十年四月庚戌朔,十五日甲子哉生霸。故顧命曰「惟四月哉生霸,王有疾不豫,甲子,王乃洮沬水」,作顧命。翌日乙丑,成王崩。康王十二年六月戊辰朔,三日庚午,故畢命豐刑曰:「惟十

月二年六月庚午朏,王命作策豐刑。」

春秋、殷曆皆以殷,魯自周昭王以下亡年數,故據周公、伯禽以下為紀。魯公伯禽,推即位四十六年,至康王十六年而薨。故傳曰:「燮父、禽父並事康王」,言晉侯燮、魯公伯禽俱事康王也。子考公就立,酋。考公,世家即位四年,及煬公熙立。煬公二十四年正月丙申朔旦冬至,殷曆以為丁酉,距微公七十六歲。

世家,煬公即位六十年,子幽公宰立。幽公,世家即位十四年,及微公茀立,泺。微公二十六年正月乙亥朔旦冬至,殷曆以為丙子,距獻公七十六歲。

世家,微公即位五十年,子厲公翟立,擢。厲公,世家即位三十七年,及獻公具立。獻公十五年正月甲寅朔旦冬至,殷曆以為乙卯,距懿公七十六歲。

世家,獻公即位五十年,子慎公埶立,撸。慎公,世家即位三十年,及武公敖立。武公,世家即位二年,子懿公被立,戲。懿公九年正月癸巳朔旦冬至,殷曆以為甲午,距惠公七十六歲。

世家,懿公即位九年,兄子柏御立。柏御,世家即位十一年,叔父孝公稱立。孝公,世家即位二十七年,子惠公皇立。惠公三十八年正月壬申朔旦冬至,殷曆以為癸酉,距釐公七十六歲。

世家,惠公即位四十六年,子隱公息立。

凡伯禽至春秋,三百八十六年。

春秋隱公,春秋即位十一年,及桓公軌立。此元年上距伐紂四百歲。

桓公,春秋即位十八年,子莊公同立。

莊公,春秋即位三十二年,子愍公啟方立。

愍公,春秋即位二年,及釐公申立。釐公五年正月辛亥朔旦冬至,殷曆以為壬子,距成公七十六歲。

是歲距上元十四萬二千五百七十七歲,得孟統五十三章首。故傳曰:「五年春王正月辛亥朔,日南至。」「八月甲午,晉侯圍上陽。」章謠云:「丙子之辰,龍尾伏辰,袀服振振,取虢之旂。鶉之賁賁,天策焞焞,火中成軍,虢公其奔。」卜偃曰:「其九月十月之交乎?丙子旦,日在尾,月在策,鶉火中,必是時也。」冬十二月丙子滅虢。言曆者以夏時,故周十二月,夏十月也。是歲,歲在大火。故傳曰晉侯使寺人披伐蒲,重耳奔狄。董因曰:「君之行,歲在大火。」後十二年,釐之十六歲,歲在壽星。故傳曰重耳處狄十二年而行,過衛五鹿,乞食於野人,野人舉塊而與之。子犯曰:「天賜也,後十二年,必獲此土。歲復於壽星,必獲諸侯。」後八歲,釐之二十四年也,歲在實沈,秦伯納之。故傳曰董因云:「君以辰出,而以參入,必獲諸侯。」

春秋,釐公即位三十三年,子文公興立。文公元年,距辛亥朔旦冬至二十九歲。是歲閏餘十三,正小雪,閏當在十一月後,而在三月,故傳曰「非禮也」。後五年,閏餘十,是歲亡閏,而置閏。閏,所以正中朔也。亡閏而置閏,又不告朔,故經曰「閏月不告朔」,言亡此月也。傳曰:「不告朔,非禮也。」

春秋,文公即位十八年,子宣公倭立。

宣公,春秋即位十八年,子成公黑肱立。成公十二年正月庚寅朔旦冬至,殷曆以為辛卯,距定公七年七十六歲。

春秋,成公即位十八年,子襄公午立。襄公二十七年,距辛亥百九歲。九月乙亥朔,是建申之月也。魯史書:「十二月乙亥朔,日有食之。」傳曰:「冬十一月乙亥朔,日有食之,於是辰在申,司曆過也,再失閏矣。」言時實行以為十一月也,不察其建,不考之於天也。二十八年距辛亥百一十歲,歲在星紀,故經曰:「春無冰。」傳曰:「歲在星紀,而淫於玄枵。」三十年歲在娵訾。三十一年歲在降婁。是歲距辛亥百一十三年,二月有癸未,上距文公十一年會于承匡之歲夏正月甲子朔凡四百四十有五甲子,奇二十日,為日二萬六千六百有六旬。故傳曰絳縣老人曰:「臣生之歲,正月甲子朔,四百四十有五甲子矣。其季於今,三之一也。」師曠曰:「郤成子會于承匡之歲也,七十三年矣。」史趙曰:「亥有二首六身,下二如身,則其日數也。」士文伯曰:「然則二萬六千六百有六旬也。」

春秋,襄公即位三十一年,子昭公稠立。昭公八年歲在析木,十年歲在顓頊之虛,玄枵也。十八年距辛亥百三十一歲,五月有丙子、戊寅、壬午,火始昏見,宋、衛、陳、鄭火。二十年春王正月,距辛亥百三十三歲,是辛亥後八章首也。正月己丑朔旦冬至,失閏。故傳曰:「二月己丑,日南至。」三十二年,歲在星紀,距辛亥百四十五歲,盈一次矣。故傳曰:「越得歲,吳伐之,必受其咎。」

春秋,昭公即位三十二年,及定公宋立。定公七年,正月己巳朔旦冬至,殷曆以為庚午,距元公七十六歲。

春秋,定公即位十五年,子哀公將立。哀公十二年冬十二月流火,非建戌之月也。是月也螽,故傳曰:「火伏而後蟄者畢,今火猶西流,司曆過也。」《詩》曰:「七月流火。」春秋,哀公即位二十七年。自春秋盡哀十四年,凡二百四十二年。

六國春秋哀公後十三年遜于邾,子悼公曼立,寧。悼公,世家即位三十七年,子元公嘉立。元公四年正月戊申朔旦冬至,殷曆以為己酉,距康公七十六歲。元公,世家即位二十一年,子穆公衍立,顯。穆公,世家即位三十三年,子恭公奮立。恭公,世家即位二十二年,子康公毛立。康公四年正月丁亥朔旦冬至,殷曆以為戊子,距緡公七十六歲。康公,世家即位九年,子景公偃立。景公,世家即位二十九年,子平公旅立。平公,世家即位二十年,子緡公賈立。緡公二十二年正月丙寅朔旦冬至,殷曆以為丁卯,距楚元七十六歲。緡公,世家即位二十三年,子頃公讎立。頃公,表十八年,秦昭王之五十一年也,秦始滅周。周凡三十六王,八百六十七歲。

秦伯昭公,本紀無天子五年。孝文王,本紀即位一年。元年,楚考烈王滅魯頃公為家人,周滅後六年也。莊襄王,本紀即位三年。始皇,本紀即位三十七年。二世,本紀即位三年。凡秦伯五世,四十九歲。

漢高祖皇帝,著紀,伐秦繼周。木生火,故為火德。天下號曰漢。距上元年十四萬三千二十五歲,歲在大棣之東井二十二度,鶉首之六度也。故漢志曰歲在大棣,名曰敦牂,太歲在午。八年十一月乙巳朔旦冬至,楚元三年也。故殷曆以為丙午,距元朔七十六歲。著紀,高帝即位十二年。

惠帝,著紀即位七年。

高帝,著紀即位八年。

文帝,前十六年,後七年,著紀即位二十三年。

景帝,前七年,中六年,後三年,著紀即位十六年。

武帝建元、元光、元朔各六年。元朔六年十一月甲申朔旦冬至,殷曆以為乙酉,距初元七十六歲。元狩、元鼎、元封各六年。漢曆太初元年,距上元十四萬三千一百二十七歲。前十一月甲子朔旦冬至,歲在星紀婺女六度,故漢志曰歲名困敦,正月歲星出婺女。太初、天漢、太始、征和各四年,後二年,著紀即位五十四年。

昭帝始元、元鳳各六年,元平一年,著紀即位十三年。

宣帝本始、地節、元康、神爵、五鳳、甘露各四年,黃龍一年,著紀即位二十五年。元帝初元二年十一月癸亥朔旦冬至,殷曆以為甲子,以為紀首。是歲也,十月日食,非合辰之會,不得為紀首。距建武七十六歲。初元、永光、建昭各五年,竟寧一年,著紀即位十六年。

成帝建始、河平、陽朔、鴻嘉、永始、元延各四年,綏和二年,著紀即位二十六年。

哀帝建平四年,元壽二年,著紀即位六年。

平帝,著紀即位元始五年,以宣帝玄孫嬰為嗣,謂之孺子。孺子,著紀新都侯王莽居攝三年,王莽居攝,盜襲帝位,竊號曰新室。始建國五年,天鳳六年,地皇三年,著紀盜位十四年。更始帝,著紀以漢宗室滅王莽,即位二年。赤眉賊立宗室劉盆子,滅更始帝。自漢元年訖更始二年,凡二百三十歲。

光武皇帝,著紀以景帝後高祖九世孫受命中興復漢,改元曰建武,歲在鶉尾之張度。建武三十一年,中元二年,即位三十三年。


\end{pinyinscope}