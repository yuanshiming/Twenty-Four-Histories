\article{傅常鄭甘陳段傳}

\begin{pinyinscope}
傅介子,北地人也,以從軍為官。先是龜茲、樓蘭皆嘗殺漢使者,語在西域傳。至元鳳中,介子以駿馬監求使大宛,因詔令責樓蘭、龜茲國。

介子至樓蘭,責其王教匈奴遮殺漢使:「大兵方至,王苟不教匈奴,匈奴使過至諸國,何為不言?」王謝服,言「匈奴使屬過,當至烏孫,道過龜茲。」介子至龜茲,復責其王,王亦服罪。介子從大宛還到龜茲,龜茲言「匈奴使從烏孫還,在此。」介子因率其吏士共誅斬匈奴使者。還奏事,詔拜介子為中郎,遷平樂監。

介子謂大將軍霍光曰:「樓蘭、龜茲數反覆而不誅,無所懲艾。介子過龜茲時,其王近就人,易得也,願往刺之,以威示諸國。」大將軍曰:「龜茲道遠,且驗之於樓蘭。」於是白遣之。

介子與士卒俱齎金幣,揚言以賜外國為名。至樓蘭,樓蘭王意不親介子,介子陽引去,至其西界,使譯謂曰:「漢使者持黃金錦繡行賜諸國,王不來受,我去之西國矣。」即出金幣以示譯。譯還報王,王貪漢物,來見使者。介子與坐飲,陳物示之。飲酒皆醉,介子謂王曰:「天子使我私報王。」王起隨介子入帳中,屏語,壯士二人從後刺之,刃交胸,立死。其貴人左右皆散走。介子告諭以「王負漢罪,天子遣我來誅王,當更立前太子質在漢者。漢兵方至,毋敢動,動,滅國矣!」遂持王首還詣闕,公卿將軍議者咸嘉其功。上乃下詔曰:「樓蘭王安歸嘗為匈奴間,候遮漢使者,發兵殺略衛司馬安樂、光祿大夫忠、期門郎遂成等三輩,及安息、大宛使,盜取節印獻物,甚逆天理。平樂監傅介子持節使誅斬樓蘭王安歸首,縣之北闕,以直報怨,不煩師眾。其封介子為義陽侯,食邑七百戶。士刺王者皆補侍郎。」

介子薨,子敞有罪不得嗣,國除。元始中,繼功臣世,復封介子曾孫長為義陽侯,王莽敗,乃絕。

常惠,太原人也。少時家貧,自奮應募,隨栘中監蘇武使匈奴,并見拘留十餘年,昭帝時乃還。漢嘉其勤勞,拜為光祿大夫。

是時,烏孫公主上書言「匈奴發騎田車師,車師與匈奴為一,共侵烏孫,唯天子救之!」漢養士馬,議欲擊匈奴。會昭帝崩,宣帝初即位,本始二年,遣惠使烏孫。公主及昆彌皆遣使,因惠言「

匈奴連發大兵擊烏孫,取車延、惡師地,收其人民去,使使脅求公主,欲隔絕漢。昆彌願發國半精兵,自給人馬五萬騎,盡力擊匈奴。唯天子出兵以救公主、昆彌!」於是漢大發十五萬騎,五將軍分道出,語在匈奴傳。

以惠為校尉,持節護烏孫兵。昆彌自將翕侯以下五萬餘騎從西方入至右谷蠡庭,獲單于父行及嫂居次,名王騎將以下三萬九千人,得馬牛驢执橐佗五萬餘匹,羊六十餘萬頭,烏孫皆自取鹵獲。惠從吏卒十餘人隨昆彌還,未至烏孫,烏孫人盜惠印綬節。惠還,自以當誅。時漢五將皆無功,天子以惠奉使克獲,遂封惠為長羅侯。復遣惠持金幣還賜烏孫貴人有功者,惠因奏請龜茲國嘗殺校尉賴丹,未伏誅,請便道擊之,宣帝不許。大將軍霍光風惠以便宜從事。惠與吏士五百人俱至烏孫,還過,發西國兵二萬人,令副使發龜茲東國二萬人,烏孫兵七千人,從三面攻龜茲,兵未合,先遣人責其王以前殺漢使狀。王謝曰:「乃我先王時為貴人姑翼所誤耳,我無罪。」惠曰:「即如此,縛姑翼來,吾置王。」王執姑翼詣惠,惠斬之而還。

後代蘇武為典屬國,明習外國事,勤勞數有功。甘露中,後將軍趙充國薨,天子遂以惠為右將軍,典屬國如故。宣帝崩,惠事元帝,三歲薨,諡曰壯武侯。傳國至曾孫,建武中乃絕。

鄭吉,會稽人也,以卒伍從軍,數出西域,由是為郎。吉為人彊執,習外國事。自張騫通西域,李廣利征伐之後,初置校尉,屯田渠黎。至宣帝時,吉以侍郎田渠黎,積穀,因發諸國兵攻破車師,遷衛司馬,使護鄯善以西南道。

神爵中,匈奴乖亂,日逐王先賢撣欲降漢,使人與吉相聞。吉發渠黎、龜茲諸國五萬人迎日逐王,口萬二千人、小王將十二人隨吉至河曲,頗有亡者,吉追斬之,遂將詣京師。漢封日逐王為歸德侯。

吉既破車師,降日逐,威震西域,遂并護車師以西北道,故號都護。都護之置自吉始焉。

上嘉其功效,乃下詔曰:「都護西域騎都尉鄭吉,拊循外蠻,宣明威信,迎匈奴單于從兄日逐王眾,擊破車師兜訾城,功效茂著。其封吉為安遠侯,食邑千戶。」吉於是中西域而立莫府,治烏壘城,鎮撫諸國,誅伐懷集之。漢之號令班西域矣,始自張騫而成於鄭吉。語在西域傳。

吉薨,諡曰繆侯。子光嗣,薨,無子,國除。元始中,錄功臣不以罪絕者,封吉曾孫永為安遠侯。

甘延壽字君況,北地郁郅人也。少以良家子善騎射為羽林,投石拔距絕於等倫,嘗超踰羽林亭樓,由是遷為郎。試弁,為期門,以材力愛幸。稍遷至遼東太守,免官。車騎將軍許嘉薦延壽為郎中諫大夫,使西域都護騎都尉,與副校尉陳湯共誅斬郅支單于,封義成侯。薨,諡曰壯侯。傳國至曾孫,王莽敗,乃絕。

陳湯字子公,山陽瑕丘人也。少好書,博達善屬文。家貧饨貣無節,不為州里所稱。西至長安求官,得太官獻食丞。數歲,富平侯張勃與湯交,高其能。初元二年,元帝詔列侯舉茂材,勃舉湯。湯待遷,父死不奔喪,司隸奏湯無循行,勃選舉故不以實,坐削二百戶,會薨,因賜諡曰繆侯。湯下獄論。後復以薦為郎,數求使外國。久之,遷西域副校尉,與甘延壽俱出。

先是,宣帝時匈奴乖亂,五單于爭立,呼韓邪單于與郅支單于俱遣子入侍,漢兩受之。後呼韓邪單于身入稱臣朝見,郅支以為呼韓邪破弱降漢,不能自還,即西收右地。會漢發兵送呼韓邪單于,郅支由是遂西破呼偈、堅昆、丁令,兼三國而都之。怨漢擁護呼韓邪而不助己,困辱漢使者江乃始等。初元四年,遣使奉獻,因求侍子,願為內附。漢議遣衛司馬谷吉送之。御史大夫貢禹、博士匡衡以為春秋之義「許夷狄者不壹而足」,今郅支單于鄉化未淳,所在絕遠,宜令使者送其子至塞而還。吉上書言:「中國與夷狄有羈靡不絕之義,今既養全其子十年,德澤甚厚,空絕而不送,近從塞還,示捐棄不畜,使無鄉從之心。棄前恩,立後怨,不便。議者見前江乃始無應敵之數,知勇俱困,以致恥辱,即豫為臣憂。臣幸得建彊漢之節,承明聖之詔,宣諭厚恩,不宜敢桀。若懷禽獸,加無道於臣,則單于長嬰大罪,必遁逃遠舍,不敢近邊。沒一使以安百姓,國之計,臣之願也。願送至庭。」上以示朝者,禹復爭,以為吉往必為國取悔生事,不可許。右將軍馮奉世以為可遣,上許焉。既至,郅支單于怒,竟殺吉等。自知負漢,又聞呼韓邪益彊,遂西奔康居。康居王以女妻郅支,郅支亦以女予康居王。康居甚尊敬郅支,欲倚其威以脅諸國。郅支數借兵擊烏孫,深入至赤谷城,殺略民人,歐畜產,烏孫不敢追,西邊空虛,不居者且千里。郅支單于自以大國,威名尊重,又乘勝驕,不為康居王禮,怒殺康居王女及貴人、人民數百,或支解投都賴水中。發民作城,日作五百人,二歲乃已。又遣使責闔蘇、大宛諸國歲遺,不敢不予。漢遣使三輩至康居求谷吉等死,郅支困辱使者,不肯奉詔,而因都護上書言:「居困厄,願歸計彊漢,遣子入侍。」其驕嫚如此。

建昭三年,湯與延壽出西域。湯為人沈勇有大慮,多策謀,喜奇功,每過城邑山川,常登望。既領外國,與延壽謀曰:「夷狄畏服大種,其天性也。西域本屬匈奴,今郅支單于威名遠聞,侵陵烏孫、大宛,常為康居畫計,欲降服之。如得此二國,北擊伊列,西取安息,南排月氏、山離烏弋,數年之間,城郭諸國危矣。且其人剽悍,好戰伐,數取勝,久畜之,必為西域患。郅支單于雖所在絕遠,蠻夷無金城強弩之守,如發屯田吏士,驅從烏孫眾兵,直指其城下,彼亡則無所之,守則不足自保,千載之功可一朝而成也。」延壽亦以為然,欲奏請之,湯曰:「國家與公卿議,大策非凡所見,事必不從。」延壽猶與不聽。會其久病,湯獨矯制發城郭諸國兵、車師戊己校尉屯田吏士。延壽聞之,驚起,欲止焉。湯怒,按劍叱延壽曰:「大眾已集會,豎子欲沮眾邪?」延壽遂從之,部勒行陳,益置揚威、白虎、合騎之校,漢兵胡兵合四萬餘人,延壽、湯上疏自劾奏矯制,陳言兵狀。

即日引軍分行,別為六校,其三校從南道踰蔥領徑大宛,其三校都護自將,發溫宿國,從北道入赤谷,過烏孫,涉康居界,至闐池西。而康居副王抱闐將數千騎,寇赤谷城東,殺略大昆彌千餘人,驅畜產甚多。從後與漢軍相及,頗寇盜後重。湯縱胡兵擊之,殺四百六十人,得其所略民四百七十人,還付大昆彌,其馬牛羊以給軍食。又捕得抱闐貴人伊奴毒。

入康居東界,令軍不得為寇。間呼其貴人屠墨見之,諭以威信,與飲盟遣去。徑引行,未至單于城可六十里,止營。復捕得康居貴人貝色子男開牟以為導。貝色子即屠墨母之弟,皆怨單于,由是具知郅支情。

明日引行,未至城三十里,止營。單于遣使問:「漢兵何以來?」應曰:「單于上書言居困阨,願歸計彊漢,身入朝見。天子哀閔單于棄大國,屈意康居,故使都護將軍來迎單于妻子,恐左右驚動,故未敢至城下。」使數往來相答報。延壽、湯因讓之:「我為單于遠來,而至今無名王大人見將軍受事者,何單于忽大計,失客主之禮也!兵來道遠,人畜罷極,食度且盡,恐無以自還,願單于與大臣審計策。」

明日,前至郅支城都賴水上,離城三里,止營傅陳。望見單于城上立五采幡織,數百人披甲乘城,又出百餘騎往來馳城下,步兵百餘人夾門魚鱗陳,講習用兵。城上人更招漢軍曰「鬥來!」百餘騎馳赴營,營皆張弩持滿指之,騎引卻。頗遣吏士射城門騎步兵,騎步兵皆入。延壽、湯令軍聞鼓音皆薄城下,四面圍城,各有所守,穿塹,塞門戶,鹵楯為前,戟弩為後,卬射城中樓上人,樓上人下走。土城外有重木城,從木城中射,頗殺傷外人。外人發薪燒木城。夜,數百騎欲出外,迎射殺之。

初,單于聞漢兵至,欲去,疑康居怨己,為漢內應,又聞烏孫諸國兵皆發,自以無所之。郅支已出,復還,曰:「不如堅守。漢兵遠來,不能久攻。」單于乃被甲在樓上,諸閼氏夫人數十皆以弓射外人。外人射中單于鼻,諸夫人頗死。單于下騎,傳戰大內。夜過半,木城穿,中人卻入土城,乘城呼。時康居兵萬餘騎分為十餘處,四面環城,亦與相應和。夜,數奔營,不利,輒卻。平明,四面火起,吏士喜,大呼乘之,鉦鼓聲動地。康居兵引卻。漢兵四面推鹵楯,並入土城中。單于男女百餘人走入大內。漢兵縱火,吏士爭入,單于被創死。軍候假丞杜勳斬單于首,得漢使節二及谷吉等所齎帛書。諸鹵獲以畀得者。凡斬閼氏、太子、名王以下千五百一十八級,生虜百四十五人,降虜千餘人,賦予城郭諸國所發十五王。

於是延壽、湯上疏曰:「臣聞天下之大義,當混為一,昔有唐虞,今有彊漢。匈奴呼韓邪單于已稱北藩,唯郅支單于叛逆,未伏其辜,大夏之西,以為彊漢不能臣也。郅支單于慘毒行於民,大惡通于天。臣延壽、臣湯將義兵,行天誅,賴陛下神靈,陰陽並應,天氣精明,陷陳克敵,斬郅支首及名王以下。宜縣頭槁街蠻夷邸間,以示萬里,明犯彊漢者,雖遠必誅。」事下有司。丞相匡衡、御史大夫繁延壽以為「郅支及名王首更歷諸國,蠻夷莫不聞知。月令春『掩骼埋胔』之時,宜勿縣。」車騎將軍許嘉、右將軍王商以為「春秋夾谷之會,優施笑君,孔子誅之,方盛夏,首足異門而出。宜縣十日乃埋之。」有詔將軍議是。

初,中書令石顯嘗欲以姊妻延壽,延壽不取。及丞相、御史亦惡其矯制,皆不與湯。湯素貪,所鹵獲財物入塞多不法。司隸校尉移書道上,繫吏士按驗之。湯上疏言:「臣與吏士共誅郅支單于,幸得禽滅,萬里振旅,宜有使者迎勞道路。今司隸反逆收繫按驗,是為郅支報讎也!」上立出吏士,令縣道具酒食以過軍。既至,論功,石顯、匡衡以為「延壽、湯擅興師矯制,幸得不誅,如復加爵土,則後奉使者爭欲乘危徼幸,生事於蠻夷,為國招難,漸不可開。」元帝內嘉延壽、湯功,而重違衡、顯之議,議久不決。

故宗正劉向上疏曰:「郅支單于囚殺使者吏士以百數,事暴揚外國,傷威毀重,群臣皆閔焉。陛下赫然欲誅之,意未嘗有忘。西域都護延壽、副校尉湯承聖指,倚神靈,總百蠻之君,髓城郭之兵,出百死,入絕域,遂蹈康居,屠五重城,搴歙侯之旗,斬郅支之首,縣旌萬里之外,揚威昆山之西,掃谷吉之恥,立昭明之功,萬夷慴伏,莫不懼震。呼韓邪單于見郅支已誅,且喜且懼,鄉風馳義,稽首來賓,願守北藩,累世稱臣。立千載之功,建萬世之安,群臣之勳莫大焉。昔周大夫方叔、吉甫為宣王誅獫狁而百蠻從,其《詩》曰:『嘽嘽焞焞,如霆如雷,顯允方叔,征伐獫狁,蠻荊來威。』《易》曰:『有嘉折首,獲非其醜。』言美誅首惡之人,而諸不順者皆來從也。今延壽、湯所誅震,雖易之折首、詩之雷霆不能及也。論大功者不錄小過,舉大美者不疵細瑕。司馬法曰『軍賞不踰月』,欲民速得為善之利也。蓋急武功,重用人也。吉甫之歸,周厚賜之,其《詩》曰:『吉甫燕喜,既多受祉,來歸自鎬,我行永久。』千里之鎬猶以為遠,況萬里之外,其勤至矣!延壽、湯既未獲受祉之報,反屈捐命之功,久挫於刀筆之前,非所以勸有功厲戎士也。昔齊桓公前有尊周之功,後有滅項之罪,君子以功覆過而為之諱行事。貳師將軍李廣利捐五萬之師,靡億萬之費,經四年之勞,而势獲駿馬三十匹,雖斬宛王毌鼓之首,猶不足以復費,其私罪惡甚多。孝武以為萬里征伐,不錄其過,遂封拜兩侯、三卿、二千石百有餘人。今康居國彊於大宛,郅支之號重於宛王,殺使者罪甚於留馬,而延壽、湯不煩漢士,不費斗糧,比於貳師,功德百之。且常惠隨欲擊之烏孫,鄭吉迎自來之日逐,猶皆裂土受爵。故言威武勤勞則大於方叔、吉甫,列功覆過則優於齊桓、貳師,近事之功則高於安遠、長羅,而大功未著,小惡數布,臣竊痛之!宜以時解縣通籍,除過勿治,尊寵爵位,以勸有功。」

於是天子下詔曰:「匈奴郅支單于背畔禮義,留殺漢使者、吏士,甚逆道理,朕豈忘之哉!所以優游而不征者,重動師眾,勞將帥,故隱忍而未有云也。今延壽、湯睹便宜,乘時利,結城郭諸國,擅興師矯制而征之,賴天地宗廟之靈,誅討郅支單于,斬獲其首,及閼氏貴人名王以下千數。雖踰義干法,內不煩一夫之役,不開府庫之臧,因敵之糧以贍軍用,立功萬里之外,威震百蠻,名顯四海。為國除殘,兵革之原息,邊竟得以安。然猶不免死亡之患,罪當在於奉憲,朕甚閔之!其赦延壽、湯罪,勿治。」詔公卿議封焉。議者皆以為宜如軍法捕斬單于令。匡衡、石顯以為「郅支本亡逃失國,竊號絕域,非真單于。」元帝取安遠侯鄭吉故事,封千戶,衡、顯復爭。乃封延壽為義成侯,賜湯爵關內侯,食邑各三百戶,加賜黃金百斤。告上帝、宗廟,大赦天下。拜延壽為長水校尉,湯為射聲校尉。

延壽遷城門校尉,護軍都尉,薨於官。成帝初即位,丞相衡復奏「湯以吏二千石奉使,顓命蠻夷中,不正身以先下,而盜所收康居財物,戒官屬曰絕域事不覆校。雖在赦前,不宜處位。」湯坐免。

後湯上書言康居王侍子非王子也。按驗,實王子也。湯下獄當死。太中大夫谷永上疏訟湯曰:「臣聞楚有子玉得臣,文公為之仄席而坐;趙有廉頗、馬服,彊秦不敢窺兵井陘;近漢有郅都、魏尚,匈奴不敢南鄉沙幕。由是言之,戰克之將,國之爪牙,不可不重也。蓋『君子聞鼓鼙之聲,則思將率之臣』。竊見關內侯陳湯,前使副西域都護,忿郅支之無道,閔王誅之不加,策慮愊億,義勇奮發,卒興師奔逝,橫厲烏孫,踰集都賴,屠三重城,斬郅支首,報十年之逋誅,雪邊吏之宿恥,威震百蠻,武暢西海,漢元以來,征伐方外之將,未嘗有也。今湯坐言事非是,幽囚久繫,歷時不決,執憲之吏欲致之大辟。昔白起為秦將,南拔郢都,北阬趙括,以纖介之過,賜死杜郵,秦民憐之,莫不隕涕。今湯親秉鉞,席卷喋血萬里之外,薦功宗廟,告類上帝,介冑之士靡不慕義。以言事為罪,無赫赫之惡。周書曰:『記人之功,忘人之過,宜為君者也。』夫犬馬有勞於人,尚加帷蓋之報,況國之功臣者哉!竊恐陛下忽於鼓鼙之聲,不察周書之意,而忘帷蓋之施,庸臣遇湯,卒從吏議,使百姓介然有秦民之恨,非所以厲死難之臣也。」書奏,天子出湯,奪爵為士伍。

後數歲,西域都護段會宗為烏孫兵所圍,驛騎上書,願發城郭敦煌兵以自救。丞相王商、大將軍王鳳及百僚議數日不決。鳳言「湯多籌策,習外國事,可問。」上召湯見宣室。湯擊郅支時中寒病,兩臂不詘申。湯入見,有詔毋拜,示以會宗奏。湯辭謝,曰:「將相九卿皆賢材通明,小臣罷癃,不足以策大事。」上曰:「國家有急,君其毋讓。」對曰:「臣以為此必無可憂也。」上曰:「何以言之?」湯曰:「夫胡兵五而當漢兵一,何者?兵刃朴鈍,弓弩不利。今聞頗得漢巧,然猶三而當一。又兵法曰『客倍而主人半然後敵』,今圍會宗者人眾不足以勝會宗,唯陛下勿憂!且兵輕行五十里,重行三十里,今會宗欲發城郭敦煌,歷時乃至,所謂報讎之兵,非救急之用也。」上曰:「柰何?其解可必乎?度何時解?」湯知烏孫瓦合,不能久攻,故事不過數日,因對曰:「已解矣!」詘指計其日,曰:「不出五日,當有吉語聞。」居四日,軍書到,言已解。大將軍鳳奏以為從事中郎,莫府事壹決於湯。湯明法令,善因事為勢,納說多從。常受人金錢作章奏,卒以此敗。

初,湯與將作大匠解萬年相善。自元帝時,渭陵不復徙民起邑。成帝起初陵,數年後,樂霸陵曲亭南,更營之。萬年與湯議,以為「

武帝時工楊光以所作數可意自致將作大匠,及大司農中丞耿壽昌造杜陵賜爵關內侯,將作大匠乘馬延年以勞苦秩中二千石;今作初陵而營起邑居,成大功,萬年亦當蒙重賞。子公妻家在長安,兒子生長長安,不樂東方,宜求徙,可得賜田宅,俱善。」湯心利之,即上封事言:「初陵,京師之地,最為肥美,可立一縣。天下民不徙諸陵三十餘歲矣,關東富人益眾,多規良田,役使貧民,可徙初陵,以彊京師,衰弱諸侯,又使中家以下得均貧富。湯願與妻子家屬徙初陵,為天下先。」於是天子從其計,果起昌陵邑,後徙內郡國民。萬年自詭三年可成,後卒不就,群臣多言其不便者。下有司議,皆曰:「昌陵因卑為高,積土為山,度便房猶在平地上,客土之中不保幽冥之靈,淺外不固,卒徒工庸以鉅萬數,至然脂火夜作,取土東山,且與穀同賈。作治數年,天下遍被其勞,國家罷敝,府臧空虛,下至眾庶,熬熬苦之。故陵因天性,據真土,處勢高敞,旁近祖考,前又已有十年功緒,宜還復故陵,勿徙民。」上乃下詔罷昌陵,語在成紀。丞相御史請廢昌陵邑中室,奏未下,人以問湯:「第宅不

得徹毋復發徙?」湯曰:「縣官且順聽群臣言,猶且復發徙之也。」

時成都侯商新為大司馬衛將軍輔政,素不善湯。商聞此語,白湯惑眾,下獄治,按驗諸所犯。湯前為騎都尉王莽上書言:「父早死,犯不封,母明君共養皇太后,尤勞苦,宜封竟為新都侯。」後皇太后同母弟苟參為水衡都尉,死,子伋為侍中,參妻欲為伋求封,湯受其金五十斤,許為求比上奏。弘農太守張匡坐臧百萬以上,狡猾不道,有詔即訊,恐下獄,使人報湯。湯為訟罪,得踰冬月,許謝錢二百萬,皆此類也。事在赦前。後東萊郡黑龍冬出,人以問湯,湯曰:「是所謂玄門開。微行數出,出入不時,故龍以非時出也。」又言當復發徙,傳相語者十餘人。丞相御史奏「湯惑眾不道,妄稱詐歸異於上,非所宜言,大不敬。」廷尉增壽議,以為「不道無正法,以所犯劇易為罪,臣下丞用失其中,故移獄廷尉,無比者先以聞,所以正刑罰,重人命也。明主哀憫百姓,下制書罷昌陵勿徙吏民,已申布。湯妄以意相謂且復發徙,雖頗驚動,所流行者少,百姓不為變,不可謂惑眾。湯稱詐,虛設不然之事,非所宜言,大不敬也。」制曰:「廷尉增壽當是。湯前有討郅支單于功,其免湯為庶人,徙邊。」又曰:「故將作大匠萬年佞邪不忠,妄為巧詐,多賦斂,煩繇役,興卒暴之作,卒徒蒙辜,死者連屬,毒流眾庶,海內怨望。雖蒙赦令,不宜居京師。」於是湯與萬年俱徙敦煌。

久之,敦煌太守奏「湯前親誅郅支單于,威行外國,不宜近邊塞。」詔徙安定。

議郎耿育上書言便宜,因冤訟湯曰:「延壽、湯為聖漢揚鉤深致遠之威,雪國家累年之恥,討絕域不羈之君,係萬里難制之虜,豈有比哉!先帝嘉之,仍下明詔,宣著其功,改年垂曆,傳之無窮。應是,南郡獻白虎,邊陲無警備。會先帝寢疾,然猶垂意不忘,數使尚書責問丞相,趣立其功。獨丞相匡衡排而不予,封延壽、湯數百戶,此功臣戰士所以失望也。孝成皇帝承建業之基,乘征伐之威,兵革不動,國家無事,而大臣傾邪,讒佞在朝,曾不深惟本末之難,以防未然之戒,欲專主威,排妒有功,使湯塊然被冤拘囚,不能自明,卒以無罪,老棄敦煌,正當西域通道,令威名折衝之臣旋踵及身,復為郅支遺虜所笑,誠可悲也!至今奉使外蠻者,未嘗不陳郅支之誅以揚漢國之盛。夫援人之功以懼敵,棄人之身以快讒,豈不痛哉!且安不忘危,盛必慮衰,今國家素無文帝累年節儉富饒之畜,又無武帝薦延梟俊禽敵之臣,獨有一陳湯耳!假使異世不及陛下,尚望國家追錄其功,封表其墓,以勸後進也。湯幸得身當聖世,功曾未久,反聽邪臣鞭逐斥遠,使亡逃分竄,死無處所。遠覽之士,莫不計度,以為湯功累世不可及,而湯過人情所有,湯尚如此,雖復破絕筋骨,暴露形骸,猶復制於脣舌,為嫉妒之臣所係虜耳。此臣所以為國家尤戚戚也。」書奏,天子還湯,卒於長安。

死後數年,王莽為安漢公秉政,既內德湯舊恩,又欲諂皇太后,以討郅支功尊元帝廟稱高宗。以湯、延壽前功大賞薄,及候丞杜勳不賞,乃益封延壽孫遷千六百戶,追諡湯曰破胡壯侯,封湯子馮為破胡侯,勳為討狄侯。

段會宗字子松,天水上邽人也。竟寧中,以杜陵令五府舉為西域都護、騎都尉光祿大夫,西域敬其威信。三歲,更盡還,拜為沛郡太守。以單于當朝,徙為雁門太守。數年,坐法免。西域諸國上書願得會宗,陽朔中復為都護。

會宗為人好大節,矜功名,與谷永相友善。谷永閔其老復遠出,予書戒曰:「足下以柔遠之令德,復典都護之重職,甚休甚休!若子之材,可優遊都城而取卿相,何必勒功昆山之仄,總領百蠻,懷柔殊俗?子之所長,愚無以喻。雖然,朋友以言贈行,敢不略意。方今漢德隆盛,遠人賓服,傅、鄭、甘、陳之功沒齒不可復見,願吾子因循舊貫,毋求奇功,終更亟還,亦足以復雁門之踦。萬里之外以身為本。願詳思愚言。」

會宗既出。諸國遣子弟郊迎。小昆彌安日前為會宗所立,德之,欲往謁,諸翕侯止不聽,遂至龜茲謁。城郭甚親附。康居太子保蘇匿率眾萬餘人欲降,會宗奏狀,漢遣衛司馬逢迎。會宗發戊己校尉兵隨司馬受降。司馬畏其眾,欲令降者皆自縛,保蘇匿怨望,舉眾亡去。會宗更盡還,以擅發戊己校尉之兵乏興,有詔贖論。拜為金城太守,以病免。

歲餘,小昆彌為國民所殺,諸翕侯大亂。徵會宗為左曹中郎將光祿大夫,使安輯烏孫,立小昆彌兄末振將,定其國而還。

明年,末振將殺大昆彌,會病死,漢恨誅不加。元延中,復遣會宗發戊己校尉諸國兵,即誅末振將太子番丘。會宗恐大兵入烏孫,驚番丘,亡逃不可得,即留所發兵墊婁地,選精兵三十弩,徑至昆彌所在,召番丘,責以「末振將骨肉相殺,殺漢公主子孫,未伏誅而死,使者受詔誅番丘。」即手劍擊殺番丘。官屬以下驚恐,馳歸。小昆彌烏犁靡者,末振將兄子也,勒兵數千騎圍會宗,會宗為言來誅之意:「今圍守殺我,如取漢牛一毛耳。宛王郅支頭縣槁街,烏孫所知也。」昆彌以下服,曰:「末振將負漢,誅其子可也,獨不可告我,令飲食之邪?」會宗曰:「豫告昆彌,逃匿之,為大罪。即飲食以付我,傷骨肉恩,故不先告。」昆彌以下號泣罷去。會宗還奏事,公卿議會宗權得便宜,以輕兵深入烏孫,即誅番丘,宣明國威,宜加重賞。天子賜會宗爵關內侯,黃金百斤。

是時,小昆彌季父卑爰联擁眾欲害昆彌,漢復遣會宗使安輯,與都護孫建并力。明年,會宗病死烏孫中,年七十五矣,城郭諸國為發喪立祠焉。

贊曰:自元狩之際,張騫始通西域,至于地節,鄭吉建都護之號,訖王莽世,凡十八人,皆以勇略選,然其有功跡者具此。廉褒以恩信稱,郭舜以廉平著,孫建用威重顯,其餘無稱焉。陳湯儻潇,不自收斂,卒用困窮,議者閔之,故備列云。


\end{pinyinscope}