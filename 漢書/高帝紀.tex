\article{高帝紀}

\begin{pinyinscope}
高祖,沛豐邑中陽里人也,姓劉氏。母媼嘗息大澤之陂,夢與神遇。是時雷電晦冥,父太公往視,則見交龍於上。已而有娠,遂產高祖。

高祖為人,隆準而龍顏,美須髯,左股有七十二黑子。寬仁愛人,意豁如也。常有大度,不事家人生產作業。及壯,試吏,為泗上亭長,廷中吏無所不狎侮。好酒及色。常從王媼、武負貰酒,時飲醉臥,武負、王媼見其上常有怪。高祖每酤留飲,酒讎數倍。及見怪,歲竟,此兩家常折券棄責。

高祖常繇咸陽,縱觀秦皇帝,喟然大息,曰:「嗟乎,大丈夫當如此矣!」

單父人呂公善沛令,辟仇,從之客,因家焉。沛中豪傑吏聞令有重客,皆往賀。蕭何為主吏,主進,令諸大夫曰:「進不滿千錢,坐之堂下。」高祖為亭長,素易諸吏,乃紿為謁曰「賀錢萬」,實不持一錢。謁入,呂公大驚,起,迎之門。呂公者,好相人,見高祖狀貌,因重敬之,引入坐上坐。蕭何曰:「劉季固多大言,少成事。」高祖因狎侮諸客,遂坐上坐,無所詘。酒闌,呂公因目固留高祖。竟酒,後。呂公曰:「臣少好相人,相人多矣,無如季相,願季自愛。臣有息女,願為箕帚妾。」酒罷,呂媼怒呂公曰:「公始常欲奇此女,與貴人。沛令善公,求之不與,何自妄許與劉季?」呂公曰:「此非兒女子所知。」卒與高祖。呂公女即呂后也,生孝惠帝、魯元公主。

高祖嘗告歸之田。呂后與兩子居田中,有一老父過請飲,呂后因餔之。老父相后曰:「夫人天下貴人也。」令相兩子,見孝惠帝,曰:「夫人所以貴者,乃此男也。」相魯元公主,亦皆貴。老父已去,高祖適從旁舍來,呂后具言客有過,相我子母皆大貴。高祖問,曰:「未遠。」乃追及,問老父。老父曰:「鄉者夫人兒子皆以君,君相貴不可言。」高祖乃謝曰:「誠如父言,不敢忘德。」及高祖貴,遂不知老父處。

高祖為亭長,乃鲈竹皮為冠,令求盜之薛治,時時冠之,及貴常冠,所謂「劉氏冠」也。

高祖以亭長為縣送徒驪山,徒多道亡。自度比至皆亡之,到豐西澤中亭,止飲,夜皆解縱所送徒。曰:「

公等皆去,吾亦從此逝矣!」徒中壯士願從諸十餘人。高祖被酒,夜徑澤中,令一人行前。行前者還報曰:「

前有大蛇當徑,願還。」高祖醉,曰:「壯士行,何畏!」乃前,拔劍斬蛇。蛇分為兩,道開。行數里,醉困臥。後人來至蛇所,有一老嫗夜哭。人問嫗何哭,嫗曰:「人殺吾子。」人曰:「嫗子何為見殺?」嫗曰:「吾子,白帝子也,化為蛇,當道,今者赤帝子斬之,故哭。」人乃以嫗為不誠,欲苦之,嫗因忽不見。後人至,高祖覺。告高祖,高祖乃心獨喜,自負。諸從者日益畏之。

秦始皇帝嘗曰「東南有天子氣」,於是東游以猒當之。高祖隱於芒、碭山澤間,呂后與人俱求,常得之。高祖怪問之。呂后曰:「季所居上常有雲氣,故從往常得季。」高祖又喜。沛中子弟或聞之,多欲附者矣。

秦二世元年秋七月,陳涉起蘄,至陳,自立為楚王,遣武臣、張耳、陳餘略趙地。八月,武臣自立為趙王。郡縣多殺長吏以應涉。九月,沛令欲以沛應之。掾、主吏蕭何、曹參曰:「君為秦吏,今欲背之,帥沛子弟,恐不聽。願君召諸亡在外者,可得數百人,因以劫眾,眾不敢不聽。」乃令樊噲召高祖。高祖之眾已數百人矣。

於是樊噲從高祖來。沛令後悔,恐其有變,乃閉城城守,欲誅蕭、曹。蕭、曹恐,踰城保高祖。高祖乃書帛射城上,與沛父老曰:「天下同苦秦久矣。今父老雖為沛令守,諸侯並起,今屠沛。沛今共誅令,擇可立立之,以應諸侯,即室家完。不然,父子俱屠,無為也。」父老乃帥子弟共殺沛令,開城門迎高祖,欲以為沛令。高祖曰:「天下方擾,諸侯並起,令置將不善,一敗塗地。吾非敢自愛,恐能薄,不能完父兄子弟。此大事,願吏擇可者。」蕭、曹等皆文吏,自愛,恐事不就,後秦種族其家,盡讓高祖。諸父老皆曰:「平生所聞劉季奇怪,當貴,且卜筮之,莫如劉季最吉。」高祖數讓。眾莫肯為,高祖乃立為沛公。祠黃帝,祭蚩尤於沛廷,而釁鼓旗。幟皆赤,由所殺蛇白帝子,所殺者赤帝子故也。於是少年豪吏如蕭、曹、樊噲等皆為收沛子弟,得三千人。

是月,項梁與兄子羽起吳。田儋與從弟榮、橫起齊,自立為齊王。韓廣自立為燕王。魏咎自立為魏王。陳涉之將周章西入關,至戲,秦將章邯距破之。

秦二年十月,沛公攻胡陵、方與,還守豐。秦泗川監平將兵圍豐。二日,出與戰,破之。令雍齒守豐。十一月,沛公引兵之薜。秦泗川守壯兵敗於薛,走至戚,沛公左司馬得殺之。沛公還軍亢父,至方與。趙王武臣為其將所殺。十二月,楚王陳涉為其御莊賈所殺。魏人周市略地豐沛,使人謂雍齒曰:「豐,故梁徙也,今魏地已定者數十城。齒今下魏,魏以齒為侯守豐;不下,且屠豐。」雍齒雅不欲屬沛公,及魏招之,即反為魏守豐。沛公攻豐,不能取。沛公還之沛,怨雍齒與豐子弟畔之。

正月,張耳等立趙後趙歇為趙王。東陽甯君、秦嘉立景駒為楚王,在留。沛公往從之,道得張良,遂與俱見景駒,請兵以攻豐。時章邯從陳,別將司馬镛將兵北定楚地,屠相,至碭。東陽甯君、沛公引兵西,與戰蕭西,不利,還收兵聚留。二月,攻碭,三日拔之。收碭兵,得六千人,與故合九千人。三月,攻下邑,拔之。還擊豐,不下。四月,項梁擊殺景駒、秦嘉,止薛,沛公往見之。項梁益沛公卒五千人,五大夫將十人。沛公還,引兵攻豐,拔之。雍齒奔魏。

五月,項羽拔襄城還。項梁盡召別將。六月,沛公如薛,與項梁共立楚懷王孫心為楚懷王。章邯破殺魏王咎、齊王田儋於臨濟。七月,大霖雨。沛公攻亢父。章邯圍田榮於東阿。沛公與項梁共救田榮,大破章邯東阿。田榮歸,沛公、項羽追北,至城陽,攻屠其城。軍濮陽東,復與章邯戰,又破之。

章邯復振,守濮陽,環水。沛公、項羽去攻定陶。八月,田榮立田儋子市為齊王。定陶未下,沛公與項羽西略地至雍丘,與秦軍戰,大敗之,斬三川守李由。還攻外黃,外黃未下。

項梁再破秦軍,有驕色。宋義諫,不聽。秦益章邯兵。九月,章邯夜銜枚擊項梁定陶,大破之,殺項梁。時連雨自七月至九月。沛公、項羽方攻陳留,聞梁死,士卒恐,乃與將軍呂臣引兵而東,徙懷王自盱台都彭城。呂臣軍彭城東,項羽軍彭城西,沛公軍碭。魏咎弟豹自立為魏王。後九月,懷王并呂臣、項羽軍自將之。以沛公為碭郡長,封武安侯,將碭郡兵。以羽為魯公,封長安侯,呂臣為司徒,其父呂青為令尹。

章邯已破項梁,以為楚地兵不足憂,乃渡河北擊趙王歇,大破之。歇保鉅鹿城,秦將王離圍之。趙數請救,懷王乃以宋義為上將,項羽為次將,范增為末將,北救趙。

初,懷王與諸將約,先入定關中者王之。當是時,秦兵彊,常乘勝逐北,諸將莫利先入關。獨羽怨秦破項梁,奮勢,願與沛公西入關。懷王諸老將皆曰:「項羽為人慓悍禍賊,嘗攻襄城,襄城無唣類,所過無不殘滅。且楚數進取,前陳王、項梁皆敗,不如更遣長者扶義而西,告諭秦父兄。秦父兄苦其主久矣,今誠得長者往,毋侵暴,宜可下。項羽不可遣,獨沛公素寬大長者。」卒不許羽,而遣沛公西收陳王、項梁散卒。乃道碭至陽城與杠里,攻秦軍壁,破其二軍。

秦三年十月,齊將田都畔田榮,將兵助項羽救趙。沛公攻破東郡尉於成武。十一月,項羽殺宋義,并其兵渡河,自立為上將軍,諸將黥布等皆屬。十二月,沛公引兵至栗,遇剛武侯,奪其軍四千餘人,并之,與魏將皇欣、武滿軍合,攻秦軍,破之。故齊王建孫田安下濟北,從項羽救趙。羽大破秦軍鉅鹿下,虜王離,走章邯。

二月,沛公從碭北攻昌邑,遇彭越。越助攻昌邑,未下。沛公西過高陽,酈食其為里監門,曰:「諸將過此者多,吾視沛公大度。」乃求見沛公。沛公方踞床,使兩女子洗。酈生不拜,長揖曰:「足下必欲誅無道秦,不宜踞見長者。」於是沛公起,攝衣謝之,延上坐。食其說沛公襲陳留。沛公以為廣野君,以其弟商為將,將陳留兵。三月,攻開封,未拔。西與秦將楊熊會戰白馬,又戰曲遇東,大破之。楊熊走之滎陽,二世使使斬之以徇。四月,南攻潁川,屠之。因張良遂略韓地。

時趙別將司馬卬方欲渡河入關,沛公乃北攻平陰,絕河津。南,戰雒陽東,軍不利,從轘轅至陽城,收軍中馬騎。六月,與南陽守齮戰犨東,大破之。略南陽郡,南陽守走,保城守宛。沛公引兵過宛西。張良諫曰:「沛公雖欲急入關,秦兵尚眾,距險。今不下宛,宛從後擊,彊秦在前,此危道也。」於是沛公乃夜引軍從他道還,偃旗幟,遲明,圍宛城三匝。南陽守欲自剄,其舍人陳恢曰:「死未晚也。」乃踰城見沛公,曰:「臣聞足下約先入咸陽者王之,今足下留守宛。宛郡縣連城數十,其吏民自以為降必死,故皆堅守乘城。今足下盡日止攻,士死傷者必多;引兵去宛,宛必隨足下。足下前則失咸陽之約,後有彊宛之患。為足下計,莫若約降,封其守,因使止守,引其甲卒與之西。諸城未下者,聞聲爭開門而待足下,足下通行無所累。」沛公曰:「善。」七月,南陽守齮降,封為殷侯,封陳恢千戶。引兵西,無不下者。至丹水,高武侯鰓、襄侯王陵降。還攻胡陽,遇番君別將梅鋗,與偕攻析、酈,皆降。所過毋得鹵掠,秦民喜。遣魏人甯昌使秦。是月章邯舉軍降項羽,羽以為雍王。瑕丘申陽下河南。

八月,沛公攻武關,入秦。秦相趙高恐,乃殺二世,使人來,欲約分王關中,沛公不許。九月,趙高立二世兄子子嬰為秦王。子嬰誅滅趙高,遣將將兵距嶢關。沛公欲擊之,張良曰:「秦兵尚彊,未可輕。願先遣人益張旗幟於山上為疑兵,使酈食其、陸賈往說秦將,啗以利。」秦將果欲連和,沛公欲許之。張良曰:「此獨其將欲叛,恐其士卒不從,不如因其怠懈擊之。」沛公引兵繞嶢關,踰蕢山,擊秦軍,大破之藍田南。遂至藍田,又戰其北,秦兵大敗。

元年冬十月,五星聚于東井。沛公至霸上。秦王子嬰素車白馬,係頸以組,封皇帝璽符節,降枳道旁。諸將或言誅秦王,沛公曰:「始懷王遣我,固以能寬容,且人已服降,殺之不祥。」乃以屬吏。遂西入咸陽,欲止宮休舍,樊噲、張良諫,乃封秦重寶財物府庫,還軍霸上。蕭何盡收秦丞相府圖籍文書。十一月,召諸縣豪桀曰:「父老苦秦苛法久矣,誹謗者族,耦語者棄市。吾與諸侯約,先入關者王之,吾當王關中。與父老約,法三章耳:殺人者死,傷人及盜抵罪。餘悉除去秦法。吏民皆按堵如故。凡吾所以來,為父兄除害,非有所侵暴,毋恐!且吾所以軍霸上,待諸侯至而定要束耳。」乃使人與秦吏行至縣鄉邑告諭之。秦民大喜,爭持牛羊酒食獻享軍士。沛公讓不受,曰:「倉粟多,不欲費民。」民又益喜,唯恐沛公不為秦王。

或說沛公曰:「秦富十倍天下,地形彊。今聞章邯降項羽,羽號曰雍王,王關中。即來,沛公恐不得有此。可急使守函谷關,毋內諸侯軍,稍徵關中兵以自益,距之。」沛公然其計,從之。十二月,項羽果帥諸侯兵欲西入關,關門閉。聞沛公已定關中,羽大怒,使黥布等攻破函谷關,遂至戲下。沛公左司馬曹毋傷聞羽怒,欲攻沛公,使人言羽曰:「沛公欲王關中,令子嬰相,珍寶盡有之。」欲以求封。亞父范增說羽曰:「沛公居山東時,貪財好色,今聞其入關,珍物無所取,婦女無所幸,此其志不小。吾使人望其氣,皆為龍,成五色,此天子氣。急擊之,勿失。」於是饗士,旦日合戰。是時,羽兵四十萬,號百萬。沛公兵十萬,號二十萬,力不敵。會羽季父左尹項伯素善張良,夜馳見張良,具告其實,欲與俱去,毋特俱死。良曰:「臣為韓王送沛公,不可不告,亡去不義。」乃與項伯俱見沛公。沛公與伯約為婚姻,曰:「吾入關,秋豪無所敢取,籍吏民,封府庫,待將軍。所以守關者,備他盜也。日夜望將軍到,豈敢反邪!願伯明言不敢背德。」項伯許諾,即夜復去。戒沛公曰:「旦日不可不早自來謝。」項伯還,具以沛公言告羽,因曰:「沛公不先破關中兵,公巨能入乎?且人有大功,擊之不祥,不如因善之。」羽許諾。

沛公旦日從百餘騎見羽鴻門,謝曰:「臣與將軍戮力攻秦,將軍戰河北,臣戰河南,不自意先入關,能破秦,與將軍復相見。今者有小人言,令將軍與臣有隙。」羽曰:「此沛公左司馬曹毋傷言之,不然,籍何以生此?」羽因留沛公飲。范增數目羽擊沛公,羽不應。范增起,出謂項莊曰:「

君王為人不忍,汝入以劍舞,因擊沛公,殺之。不者,汝屬且為所虜。」莊入為壽。壽畢,曰:「軍中無以為樂,請以劍舞。」因拔劍舞。項伯亦起舞,常以身翼蔽沛公。樊噲聞事急,直入,怒甚。羽壯之,賜以酒。噲因譙讓羽。有頃,沛公起如廁,招樊噲出,置車官屬,獨騎,與樊噲、靳彊、滕公、紀成步,從間道走軍,使張良留謝羽。羽問:「沛公安在?」曰:「聞將軍有意督過之,脫身去,間至軍,故使臣獻璧。」羽受之。又獻玉斗范增。增怒,撞其斗,起曰:「吾屬今為沛公虜矣!」

沛公歸數日,羽引兵西屠咸陽,殺秦降王子嬰,燒秦宮室,所過無不殘滅。秦民大失望。羽使人還報懷王,懷王曰:「如約。」羽怨懷王不肯令與沛公俱西入關,而北救趙,後天下約。乃曰:「

懷王者,吾家所立耳,非有功伐,何以得專主約!本定天下,諸將與籍也。」春正月,陽尊懷王為義帝,實不用其命。

二月,羽自立為西楚霸王,王梁、楚地九郡,都彭城。背約,更立沛公為漢王,王巴、蜀、漢中四十一縣,都南鄭。三分關中,立秦三將:章邯為雍王,都廢丘;司馬欣為塞王,都櫟陽;董翳為翟王,都高奴。楚將瑕丘申陽為河南王,都洛陽。趙將司馬卬為殷王,都朝歌。當陽君英布為九江王,都六。懷王柱國共敖為臨江王,都江陵。番君吳芮為衡山王,都邾。故齊王建孫田安為濟北王。徙魏王豹為西魏王,都平陽。徙燕王韓廣為遼東王。燕將臧荼為燕王,都薊。徙齊王田市為膠東王。齊將田都為齊王,都臨菑。徙趙王歇為代王。趙相張耳為常山王。漢王怨羽之背約,欲攻之,丞相蕭何諫,乃止。

夏四月,諸侯罷戲下,各就國。羽使卒三萬人從漢王,楚子、諸侯人之慕從者數萬人,從杜南入蝕中。張良辭歸韓,漢王送至褒中,因說漢王燒絕棧道,以備諸侯盜兵,亦視項羽無東意。

漢王既至南鄭,諸將及士卒皆歌謳思東歸,多道亡還者。韓信為治粟都尉,亦亡去,蕭何追還之,因薦於漢王,曰:「

必欲爭天下,非信無可與計事者。」於是漢王齊戒設壇場,拜信為大將軍,問以計策。信對曰:「項羽背約而王君王於南鄭,是遷也。吏卒皆山東之人,日夜企而望歸,及其鋒而用之,可以有大功。天下已定,民皆自寧,不可復用。不如決策東向。」因陳羽可圖三秦易并之計。漢王大說,遂聽信策,部署諸將。留蕭何收巴蜀租,給軍食。

五月,漢王引兵從故道出襲雍。雍王邯迎擊漢陳倉,雍兵敗,還走;戰好畤,又大敗,走廢丘。漢王遂定雍地。東如咸陽,引兵圍雍王廢丘,而遣諸將略地。

田榮聞羽徙齊王市於膠東而立田都為齊王,大怒,以齊兵迎擊田都。都走降楚。六月,田榮殺田市,自立為齊王。時彭城在鉅野,眾萬餘人,無所屬。榮與越將軍印,因令反梁地。越擊殺濟北王安,榮遂并三齊之地。燕王韓廣亦不肯徙遼東。秋八月,臧荼殺韓廣,并其地。塞王欣、翟王翳皆降漢。

初,項梁立韓後公子成為韓王,張良為韓司徒。羽以良從漢王,韓王成又無功,故不遣就國,與俱至彭城,殺之。及聞漢王并關中,而齊、梁畔之,羽大怒,乃以故吳令鄭昌為韓王,距漢。令蕭公角擊彭越,越敗角兵。時張良徇韓地,遺羽書曰:「漢欲得關中,如約即止,不敢復東。」羽以故無西意,而北擊齊。

九月,漢王遣將軍薛歐、王吸出武關,因王陵兵,從南陽迎太公、呂后於沛。羽聞之,發兵距之陽夏,不得前。

二年冬十月,項羽使九江王布殺義帝於郴。陳餘亦怨羽獨不王己,從田榮藉助兵,以擊常山王張耳。耳敗走降漢,漢王厚遇之。陳餘迎代王歇還趙,歇立餘為代王。張良自韓間行歸漢,漢王以為成信侯。

漢王如陝,鎮撫關外父老。河南王申陽降,置河南郡。使韓太尉韓信擊韓,韓王鄭昌降。十一月,立韓太尉信為韓王。漢王還歸,都櫟陽,使諸將略地,拔隴西。以萬人若一郡降者,封萬戶。繕治河上塞。故秦苑囿園池,令民得田之。

春正月,羽擊田榮城陽,榮敗走平原,平原民殺之。齊皆降楚,楚焚其城郭,齊人復畔之。諸將拔北地,虜雍王弟章平。赦罪人。二月癸未,令民除秦社稷,立漢社稷。施恩德,賜民爵。蜀漢民給軍事勞苦,復勿租稅二歲。關中卒從軍者,復家一歲。舉民年五十以上,有脩行,能帥眾為善,置以為三老,鄉一人。擇鄉三老一人為縣三老,與縣令丞尉以事相教,復勿繇戍。以十月賜酒肉。

三月,漢王自臨晉渡河,魏王豹降,將兵從。下河內,虜殷王卬,置河內郡。至脩武,陳平亡楚來降。漢王與語,說之,使參乘,監諸將。南渡平陰津,至洛陽,新城三老董公遮說漢王曰:「臣聞『順德者昌,逆德者亡』,『兵出無名,事故不成』。故曰:『明其為賊,敵乃可服。』項羽為無道,放殺其主,天下之賊也。夫仁不以勇,義不以力,三軍之眾為之素服,以告之諸侯,為此東伐,四海之內莫不仰德。此三王之舉也。」漢王曰:「善,非夫子無所聞。」於是漢王為義帝發喪,袒而大哭,哀臨三日。發使告諸侯曰:「

天下共立義帝,北面事之。今項羽放殺義帝江南,大逆無道。寡人親為發喪,兵皆縞素。悉發關中兵,收三河士,南浮江漢以下,願從諸侯王擊楚之殺義帝者。」

夏四月,田榮弟橫收得數萬人,立榮子廣為齊王。羽雖聞漢東,既擊齊,欲遂破之而後擊漢,漢王以故得劫五諸侯兵,東伐楚。到外黃,彭越將三萬人歸漢。漢王拜越為魏相國,令定梁地。漢王遂入彭城,收羽美人貨賂,置酒高會。羽聞之,令其將擊齊,而自以精兵三萬人從魯出胡陵,至蕭,晨擊漢軍,大戰彭城靈壁東睢水上,大破漢軍,多殺士卒,睢水為之不流。圍漢王三匝。大風從西北起,折木發屋,揚砂石,晝晦,楚軍大亂,而漢王得與數十騎遁去。過沛,使人求室家,室家亦已亡,不相得。漢王道逢孝惠、魯元,載行。楚騎追漢王,漢王急,推墮二子。滕公下收載,遂得脫。審食其從太公、呂后間行,反遇楚軍,羽常置軍中以為質。諸侯見漢敗,皆亡去。塞王欣、翟王翳降楚,殷王卬死。

呂后兄周呂侯將兵居下邑,漢王往從之。稍收士卒,軍碭。

漢王西過梁地,至虞,謂謁者隨何曰:「公能說九江王布使舉兵畔楚,項王必留擊之。得留數月,吾取天下必矣。」隨何往說布,果使畔楚。

五月,漢王屯滎陽,蕭何發關中老弱未傅者悉詣軍。韓信亦收兵與漢王會,兵復大振。與楚戰滎陽南京、索間,破之。築甬道,屬河,以取敖倉粟。魏王豹謁歸視親疾。至則絕河津,反為楚。

六月,漢王還櫟陽。壬午,立太子,赦罪人。令諸侯子在關中者皆集櫟陽為衛。引水灌廢丘,廢丘降,章邯自殺。雍州定,八十餘縣,置河上、渭南、中地、隴西、上郡。令祠官祀天地四方上帝山川,以時祠之。興關中卒乘邊塞。關中大飢,米斛萬錢,人相食。令民就食蜀漢。

秋八月,漢王如滎陽,謂酈食其曰:「緩頰往說魏王豹,能下之,以魏地萬戶封生。」食其往,豹不聽。漢王以韓信為左丞相,與曹參、灌嬰俱擊魏。食其還,漢王問:「魏大將誰也?」對曰:「柏直。」王曰:「是口尚乳臭,不能當韓信。騎將誰也?」曰:「馮敬。」曰:「是秦將馮無擇子也,雖賢,不能當灌嬰。步卒將誰也?」曰:「項它。」曰:「是不能當曹參。吾無患矣。」九月,信等虜豹,傳詣滎陽。定魏地,置河東、太原、上黨郡。信使人請兵三萬人,願以北舉燕趙,東擊齊,南絕楚糧道。漢王與之。

三年冬十月,韓信、張耳東下井陘擊趙,斬陳餘,獲趙王歇。置常山、代郡。甲戌晦,日有食之。十一月癸卯晦,日有食之。

隨何既說黥布,布起兵攻楚。楚使項聲、龍且攻布,布戰不勝。十二月,布與隨何間行歸漢。漢王分之兵,與俱收兵至成皋。

項羽數侵奪漢甬道,漢軍乏食,與酈食其謀橈楚權。食其欲立六國後以樹黨,漢王刻印,將遣食其立之。以問張良,良發八難。漢王輟飯吐哺,曰:「豎儒幾敗乃公事!」令趨銷印。又問陳平,乃從其計,與平黃金四萬斤,以間疏楚君臣。

夏四月,項羽圍漢滎陽,漢王請和,割滎陽以西者為漢。亞父勸項羽急攻滎陽,漢王患之。陳平反間既行,羽果疑亞父。亞父大怒而去,發病死。

五月,將軍紀信曰:「事急矣!臣請誑楚,可以間出。」於是陳平夜出女子東門二千餘人,楚因四面擊之。紀信乃乘王車,黃屋左纛,曰:「食盡,漢王降楚。」楚皆呼萬歲,之城東觀,以故漢王得與數十騎出西門遁。令御史大夫周苛、魏豹、樅公守滎陽。羽見紀信,問:「漢王安在?」曰:「已出去矣。」羽燒殺信。而周苛、樅公相謂曰:「反國之王,難與守城。」因殺魏豹。

漢王出滎陽,至成皋。自成皋入關,收兵欲復東。轅生說漢王曰:「漢與楚相距滎陽數歲,漢常困。願君王出武關,項王必引兵南走,王深壁,令滎陽成皋間且得休息。使韓信等得輯河北趙地,連燕齊,君王乃復走滎陽。如此,則楚所備者多,力分。漢得休息,復與之戰,破之必矣。」漢王從其計,出軍宛葉間,與黥布行收兵。

羽聞漢王在宛,果引兵南,漢王堅壁不與戰。是月,彭越渡睢,與項聲、薛公戰下邳,破殺薛公。羽使終公守成皋,而自東擊彭越。漢王引兵北,擊破終公,復軍成皋。六月,羽已破走彭越,聞漢復軍成皋,乃引兵西拔滎陽城,生得周苛。羽謂苛:「為我將,以公為上將軍,封三萬戶。」周苛罵曰:「若不趨降漢,今為虜矣!若非漢王敵也。」羽亨周苛,并殺樅公,而虜韓王信,遂圍成皋。漢王跳,獨與滕公共車出成皋玉門,北渡河,宿小脩武。自稱使者,晨馳入張耳、韓信壁,而奪之軍。乃使張耳北收兵趙地。

秋七月,有星孛于大角。漢王得韓信軍,復大振。八月,臨河南鄉,軍小脩武,欲復戰。郎中鄭忠說止漢王,高壘深塹勿戰。漢王聽其計,使盧綰、劉賈將卒二萬人,騎數百,渡白馬津入楚地,佐彭越燒楚積聚,復擊破楚軍燕郭西,攻下睢陽、外黃十七城。九月,羽謂海春侯大司馬曹咎曰:「謹守成皋。即漢王欲挑戰,慎勿與戰,勿令得東而已。我十五日必定梁地,復從將軍。」羽引兵東擊彭越。

漢王使酈食其說齊王田廣,罷守兵與漢和。

四年冬十月,韓信用蒯通計,襲破齊。齊王亨酈生,東走高密。項羽聞韓信破齊,且欲擊楚,使龍且救齊。

漢果數挑成皋戰,楚軍不出,使人辱之數日,大司馬咎怒,渡兵汜水。士卒半渡,漢擊之,大破楚軍,盡得楚國金玉貨賂。大司馬咎、長史欣皆自剄汜水上。漢王引兵渡河,復取成皋,軍廣武,就敖倉食。

羽下梁地十餘城,聞海春侯破,乃引兵還。漢軍方圍鍾離辚於滎陽東,聞羽至,盡走險阻。羽亦軍廣武,與漢相守。丁壯苦軍旅,老弱罷轉餉。漢王、羽相與臨廣武之間而語。羽欲與漢王獨身挑戰,漢王數羽曰:「吾始與羽俱受命懷王,曰先定關中者王之。羽負約,王我於蜀漢,罪一也。羽矯殺卿子冠軍,自尊,罪二也。羽當以救趙還報,而擅劫諸侯兵入關,罪三也。懷王約入秦無暴掠,羽燒秦宮室,掘始皇帝冢,收私其財,罪四也。又彊殺秦降王子嬰,罪五也。詐阬秦子弟新安二十萬,王其將,罪六也。皆王諸將善地,而徙逐故主,令臣下爭畔逆,罪七也。出逐義帝彭城,自都之,奪韓王地,并王梁楚,多自與,罪八也。使人陰殺義帝江南,罪九也。夫為人臣而殺其主,殺其已降,為政不平,主約不信,天下所不容,大逆無道,罪十也。吾以義兵從諸侯誅殘賊,使刑餘罪人擊公,何苦乃與公挑戰!」羽大怒,伏弩射中漢王。漢王傷胸,乃捫足曰:「虜中吾指!」漢王病創臥,張良彊請漢王起行勞軍,以安士卒,毋令楚乘勝。漢王出行軍,疾甚,因馳入成皋。

十一月,韓信與灌嬰擊破楚軍,殺楚將龍且,追至城陽,虜齊王廣。齊相田橫自立為齊王,奔彭越。漢立張耳為趙王。

漢王疾瘉,西入關,至櫟陽,存問父老,置酒。梟故塞王欣頭櫟陽市。留四日,復如軍,軍廣武。關中兵益出,而彭越、田橫居梁地,往來苦楚兵,絕其糧食。

韓信已破齊,使人言曰:「齊邊楚,權輕,不為假王,恐不能安齊。」漢王怒,欲攻之。張良曰:「不如因而立之,使自為守。」春二月,遣張良操印,立韓信為齊王。秋七月,立黥布為淮南王。八月,初為算賦。北貉、燕人來致梟騎助漢。漢王下令:軍士不幸死者,吏為衣衾棺斂,轉送其家。四方歸心焉。

項羽自知少助食盡,韓信又進兵擊楚,羽患之。漢遣陸賈說羽,請太公,羽弗聽。漢復使侯公說羽,羽乃與漢約,中分天下,割鴻溝以西為漢,以東為楚。九月,歸太公、呂后,軍皆稱萬歲。乃封侯公為平國君。羽解而東歸。漢王欲西歸,張良、陳平諫曰:「今漢有天下太半,而諸侯皆附,楚兵罷食盡,此天亡之時,不因其幾而遂取之,所謂養虎自遺患也。」漢王從之。

五年冬十月,漢王追項羽至陽夏南止軍,與齊王信、魏相國越期會擊楚,至固陵,不會。楚擊漢軍,大破之。漢王復入壁,深塹而守。謂張良曰:「諸侯不從,柰何?」良對曰:「楚兵且破,未有分地,其不至固宜。君王能與共天下,可立致也。齊王信之立,非君王意,信亦不自堅。彭越本定梁地,始君王以魏豹故,拜越為相國。今豹死,越亦望王,而君王不早定。今能取睢陽以北至穀城皆以王彭越,從陳以東傅海與齊王信,信家在楚,其意欲復得故邑。能出捐此地以許兩人,使各自為戰,則楚易敗也。」於是漢王發使使韓信、彭越。至,皆引兵來。

十一月,劉賈入楚地,圍壽春。漢亦遣人誘楚大司馬周殷。殷畔楚,以舒屠六,舉九江兵迎黥布,並行屠城父,隨劉賈皆會。

十二月,圍羽垓下。羽夜聞漢軍四面皆楚歌,知盡得楚地,羽與數百騎走,是以兵大敗。灌嬰追斬羽東城。楚地悉定,獨魯不下。漢王引天下兵欲屠之,為其守節禮義之國,乃持羽頭示其父兄,魯乃降。初,懷王封羽為魯公,及死,魯又為之堅守,故以魯公葬羽於穀城。漢王為發葬,哭臨而去。封項伯等四人為列侯,賜姓劉氏。諸民略在楚者皆歸之。漢王還至定陶,馳入齊王信壁,奪其軍。初項羽所立臨江王共敖前死,子尉嗣立為王,不降。遣盧綰、劉賈擊虜尉。

春正月,追尊兄伯號曰武哀侯。下令曰:「楚地已定,義帝亡後,欲存恤楚眾,以定其主。齊王信習楚風俗,更立為楚王,王淮北,都下邳。魏相國建城侯彭越勤勞魏民,卑下士卒,常以少擊眾,數破楚軍,其以魏故地王之,號曰梁王,都定陶。」又曰:「兵不得休八年,萬民與苦甚,今天下事畢,其赦天下殊死以下。」

於是諸侯上疏曰:「楚王韓信、韓王信、淮南王英布、梁王彭越、故衡山王吳芮、趙王張敖、燕王臧荼昧死再拜言,大王陛下:先時秦為亡道,天下誅之。大王先得秦王,定關中,於天下功最多。存亡定危,救敗繼絕,以安萬民,功盛德厚。又加惠於諸侯王有功者,使得立社稷。地分已定,而位號比儗,亡上下之分,大王功德之著,於後世不宣。昧死再拜上皇帝尊號。」漢王曰:「寡人聞帝者賢者有也,虛言亡實之名,非所取也。今諸侯王皆推高寡人,將何以處之哉?」諸侯王皆曰:「大王起於細微,滅亂秦,威動海內。又以辟陋之地,自漢中行威德,誅不義,立有功,平定海內,功臣皆受地食邑,非私之也。大王德施四海,諸侯王不足以道之,居帝位甚實宜,願大王以幸天下。」漢王曰:「諸侯王幸以為便於天下之民,則可矣。」於是諸侯王及太尉長安侯臣綰等三百人,與博士稷嗣君叔孫通謹擇良日二月甲午,上尊號。漢王即皇帝位于氾水之陽。尊王后曰皇后,太子曰皇太子,追尊先媼曰昭靈夫人。

詔曰:「故衡山王吳芮與子二人、兄子一人,從百粵之兵,以佐諸侯,誅暴秦,有大功,諸侯立以為王。項羽侵奪之地,謂之番君。其以長沙、豫章、象郡、桂林、南海立番君芮為長沙王。」又曰:「故粵王亡諸世奉粵祀,秦侵奪其地,使其社稷不得血食。諸侯伐秦,亡諸身帥閩中兵以佐滅秦,項羽廢而弗立。今以為閩粵王,王閩中地,勿使失職。」

帝乃西都洛陽。夏五月,兵皆罷歸家。詔曰:「諸侯子在關中者,復之十二歲,其歸者半之。民前或相聚保山澤,不書名數,今天下已定,令各歸其縣,復故爵田宅,吏以文法教訓辨告,勿笞辱。民以飢餓自賣為人奴婢者,皆免為庶人。軍吏卒會赦,其亡罪而亡爵及不滿大夫者,皆賜爵為大夫。故大夫以上賜爵各一級,其七大夫以上,皆令食邑,非七大夫以下,皆復其身及戶,勿事。」又曰:「七大夫、公乘以上,皆高爵也。諸侯子及從軍歸者,甚多高爵,吾數詔吏先與田宅,及所當求於吏者,亟與。爵或人君,上所尊禮,久立吏前,曾不為決,甚亡謂也。異日秦民爵公大夫以上,令丞與亢禮。今吾於爵非輕也,吏獨安取此!且法以有功勞行田宅,今小吏未嘗從軍者多滿,而有功者顧不得,背公立私,守尉長吏教訓甚不善。其令諸吏善遇高爵,稱吾意。且廉問,有不如吾詔者,以重論之。」

帝置酒雒陽南宮。上曰:「通侯諸將毋敢隱朕,皆言其情。吾所以有天下者何?項氏之所以失天下者何?」高起、王陵對曰:「陛下嫚而侮人,項羽仁而敬人。然陛下使人攻城略地,所降下者,因以與之,與天下同利也。項羽妒賢嫉能,有功者害之,賢者疑之,戰勝而不與人功,得地而不與人利,此其所以失天下也。」上曰:「公知其一,未知其二。夫運籌帷幄之中,決勝千里之外,吾不如子房;填國家,撫百姓,給餉餽,不絕糧道,吾不如蕭何;連百萬之眾,戰必勝,攻必取,吾不如韓信。三者皆人傑,吾能用之,此吾所以取天下者也。項羽有一范增而不能用,此所以為我禽也。」群臣說服。

初,田橫歸彭越。項羽已滅,橫懼誅,與賓客亡入海。上恐其久為亂,遣使者赦橫,曰:「橫來,大者王,小者侯;不來,且發兵加誅。」橫懼,乘傳詣雒陽,未至三十里,自殺。上壯其節,為流涕,發卒二千人,以王禮葬焉。

戍卒婁敬求見,說上曰:「陛下取天下與周異,而都雒陽,不便,不如入關,據秦之固。」上以問張良,良因勸上。是日,車駕西都長安。拜婁敬為奉春君,賜姓劉氏。六月壬辰,大赦天下。

秋七月,燕王臧荼反,上自將征之。九月,虜荼。詔諸侯王視有功者立以為燕王。荊王臣信等十人皆曰:「太尉長安侯盧綰功最多,請立以為燕王。」使丞相噲將兵平代地。

利幾反,上自擊破之。利幾者,項羽將。羽敗,利幾為陳令,降,上侯之潁川。上至雒陽,舉通侯籍召之,而利幾恐,反。

後九月,徙諸侯子關中。治長樂宮。

六年冬十月,令天下縣邑城。

人告楚王信謀反,上問左右,左右爭欲擊之。用陳平計,乃偽游雲夢。十二月,會諸侯于陳,楚王信迎謁,因執之。詔曰:「

天下既安,豪桀有功者封侯,新立,未能盡圖其功。身居軍九年,或未習法令,或以其故犯法,大者死刑,吾甚憐之。其赦天下。」田肯賀上曰:「甚善,陛下得韓信,又治秦中。秦,形勝之國也,帶河阻山,縣隔千里,持戟百萬,秦得百二焉。地勢便利,其以下兵於諸侯,譬猶居高屋之上建瓴水也。夫齊,東有琅邪、即墨之饒,南有泰山之固,西有濁河之限,北有勃海之利,地方二千里,持戟百萬,縣隔千里之外,齊得十二焉。此東西秦也。非親子弟,莫可使王齊者。」上曰:「善。」賜金五百斤。上還至雒陽,赦韓信,封為淮陰侯。

甲申,始剖符封功臣曹參等為通侯。詔曰:「齊,古之建國也,今為郡縣,其復以為諸侯。將軍劉賈數有大功,及擇寬惠脩絜者,王齊、荊地。」春正月丙午,韓王信等奏請以故東陽郡、鄣郡、吳郡五十三縣立劉賈為荊王,以碭郡、薛郡、郯郡三十六縣立弟文信君交為楚王。壬子,以雲中、鴈門、代郡五十三縣立兄宜信侯喜為代王,以膠東、膠西、臨淄、濟北、博陽、城陽郡七十三縣立子肥為齊王,以太原郡三十一縣為韓國,徙韓王信都晉陽。

上已封大功臣三十餘人,其餘爭功,未得行封。上居南宮,從復道上見諸將往往耦語,以問張良。良曰:「陛下與此屬共取天下,今已為天子,而所封皆故人所愛,所誅皆平生仇怨。今軍吏計功,以天下為不足用遍封,而恐以過失及誅,故相聚謀反耳。」上曰:「為之奈何?」良曰:「取上素所不快,計群臣所共知最甚者一人,先封以示群臣。」三月,上置酒,封雍齒,因趣丞相急定功行封。罷酒,群臣皆喜,曰:「雍齒且侯,吾屬亡患矣!」

上歸櫟陽,五日一朝太公。太公家令說太公曰:「天亡二日,土亡二王。皇帝雖子,人主也;太公雖父,人臣也。奈何令人主拜人臣!如此,則威重不行。」後上朝,太公擁彗,迎門卻行。上大驚,下扶太公。太公曰:「帝,人主,奈何以我亂天下法!」於是上心善家令言,賜黃金五百斤。夏五月丙午,詔曰:「人之至親,莫親於父子,故父有天下傳歸於子,子有天下尊歸於父,此人道之極也。前日天下大亂,兵革並起,萬民苦殃,朕親被堅執銳,自帥士卒,犯危難,平暴亂,立諸侯,偃兵息民,天下大安,此皆太公之教訓也。諸王、通侯、將軍、群卿、大夫已尊朕為皇帝,而太公未有號。今上尊太公曰太上皇。」

秋九月,匈奴圍韓王信於馬邑,信降匈奴。

七年冬十月,上自將擊韓王信於銅鞮,斬其將。信亡走匈奴,與其將曼丘臣、王黃共立故趙後趙利為王,收信散兵,與匈奴共距漢。上從晉陽連戰,乘勝逐北,至樓煩,會大寒,士卒墮指者什二三。遂至平城,為匈奴所圍,七日,用陳平祕計得出。使樊噲留定代地。

十二月,上還過趙,不禮趙王。是月,匈奴攻代,代王喜棄國,自歸雒陽,赦為合陽侯。辛卯,立子如意為代王。

春,令郎中有罪耐以上,請之。民產子,復勿事二歲。

二月,至長安。蕭何治未央宮,立東闕、北闕、前殿、武庫、大倉。上見其壯麗,甚怒,謂何曰:「天下匈匈,勞苦數歲,成敗未可知,是何治宮室過度也!」何曰:「天下方未定,故可因以就宮室。且夫天子以四海為家,非令壯麗亡以重威,且亡令後世有以加也。」上說。自櫟陽徙都長安。置宗正宮以序九族。夏四月,行如雒陽。

八年冬,上東擊韓信餘寇於東垣。還過趙,趙相貫高等恥上不禮其王,陰謀欲弒上。上欲宿,心動,問「縣名何?」曰:「柏人。」上曰:「柏人者,迫於人也。」去弗宿。

十一月,令士卒從軍死者為槥,歸其縣,縣給衣衾棺葬具,祠以少牢,長吏視葬。十二月,行自東垣至。

春三月,行如雒陽。令吏卒從軍至平城及守城邑者皆復終身勿事。爵非公乘以上毋得冠劉氏冠。賈人毋得衣錦繡綺縠絺紵钛,操兵,乘騎馬。秋八月,吏有罪未發覺者,赦之。九月,行自雒陽至,淮南王、梁王、趙王、楚王皆從。

九年冬十月,淮南王、梁王、趙王、楚王朝未央宮,置酒前殿。上奉玉卮為太上皇壽,曰:「始大人常以臣亡賴,不能治產業,不如仲力。今某之業所就孰與仲多?」殿上群臣皆稱萬歲,大笑為樂。

十一月,徙齊楚大族昭氏、屈氏、景氏、懷氏、田氏五姓關中,與利田宅。十二月,行如雒陽。

貫高等謀逆發覺,逮捕高等,并捕趙王敖下獄。詔敢有隨王,罪三族。郎中田叔、孟舒等十人自髡鉗為王家奴,從王就獄。王實不知其謀。春正月,廢趙王敖為宣平侯。徙代王如意為趙王,王趙國。丙寅,前有罪殊死以下,皆赦之。

二月,行自雒陽至。賢趙臣田叔、孟舒等十人,召見與語,漢廷臣無能出其右者。上說,盡拜為郡守、諸侯相。

夏六月乙未晦,日有食之。

十年冬十月,淮南王、燕王、荊王、梁王、楚王、齊王、長沙王來朝。

夏五月,太上皇后崩。秋七月癸卯,太上皇崩,葬萬年。赦櫟陽囚死罪以下。八月,令諸侯王皆立太上皇廟于國都。

九月,代相國陳豨反。上曰:「豨嘗為吾使,甚有信。代地吾所急,故封豨為列侯,以相國守代,今乃與王黃等劫掠代地!吏民非有罪也,能去豨、黃來歸者,皆赦之。」上自東,至邯鄲。上喜曰:「豨不南據邯鄲而阻漳水,吾知其亡能為矣。」趙相周昌奏常山二十五城亡其二十城,請誅守尉。上曰:「守尉反乎?」對曰:「不。」上曰:「是力不足,亡罪。」上令周昌選趙壯士可令將者,白見四人。上嫚罵曰:「豎子能為將乎!」四人慚,皆伏地。上封各千戶,以為將。左右諫曰:「從入蜀漢,伐楚,賞未遍行,今封此,何功?」上曰:「非汝所知。陳豨反,趙代地皆豨有。吾以羽檄徵天下兵,未有至者,今計唯獨邯鄲中兵耳。吾何愛四千戶,不以慰趙子弟!」皆曰:「善。」又求「樂毅有後乎?」得其孫叔,封之樂鄉,號華成君。問豨將,皆故賈人。上曰:「吾知與之矣。」乃多以金購豨將,豨將多降。

十一年冬,上在邯鄲。豨將侯敞將萬餘人游行,王黃將騎千餘軍曲逆,張春將卒萬餘人度河攻聊城。漢將軍郭蒙與齊將擊,大破之。太尉周勃道太原入定代地,至馬邑,馬邑不下,攻殘之。豨將趙利守東垣,高祖攻之不下。卒罵,上怒。城降,卒罵者斬之。諸縣堅守不降反寇者,復租賦三歲。

春正月,淮陰侯韓信謀反長安,夷三族。將軍柴武斬韓王信於參合。

上還雒陽。詔曰:「代地居常山之北,與夷狄邊,趙乃從山南有之,遠,數有胡寇,難以為國。頗取山南太原之地益屬代,代之雲中以西為雲中郡,則代受邊寇益少矣。王、相國、通侯、吏二千石擇可立為代王者。」燕王綰、相國何等三十三人皆曰:「子恆賢知溫良,請立以為代王,都晉陽。」大赦天下。

二月,詔曰:「欲省賦甚。今獻未有程,吏或多賦以為獻,而諸侯王尤多,民疾之。令諸侯王、通侯常以十月朝獻,及郡各以其口數率,人歲六十三錢,以給獻費。」又曰:「蓋聞王者莫高於周文,伯者莫高於齊桓,皆待賢人而成名。今天下賢者智能豈特古之人乎?患在人主不交故也,士奚由進!今吾以天之靈,賢士大夫定有天下,以為一家,欲其長久,世世奉宗廟亡絕也。賢人已與我共平之矣,而不與吾共安利之,可乎?賢士大夫有肯從我游者,吾能尊顯之。布告天下,使明知朕意。御史大夫昌下相國,相國酇侯下諸侯王,御史中執法下郡守,其有意稱明德者,必身勸,為之駕,遣詣相國府,署行、義、年。有而弗言,覺,免。年老癃病,勿遣。」

三月,梁王彭越謀反,夷三族。詔曰:「擇可以為梁王、淮陽王者。」燕王綰、相國何等請立子恢為梁王,子友為淮陽王。罷東郡,頗益梁;罷潁川郡,頗益淮陽。

夏四月,行自雒陽至。令豐人徙關中者皆復終身。

五月,詔曰:「粵人之俗,好相攻擊,前時秦徙中縣之民南方三郡,使與百粵雜處。會天下誅秦,南海尉它居南方長治之,甚有文理,中縣人以故不耗減,粵人相攻擊之俗益止,俱賴其力。今立它為南粵王。」使陸賈即授璽綬。它稽首稱臣。

六月,令士卒從入蜀、漢、關中者皆復終身。

秋七月,淮南王布反。上問諸將,滕公言故楚令尹薛公有籌策。上見公,薛公言布形勢,上善之,封薛公千戶。詔王、相國擇可立為淮南王者,群臣請立子長為王。上乃發上郡、北地、隴西車騎,巴蜀材官及中尉卒三萬人為皇太子衛,軍霸上。布果如薛公言,東擊殺荊王劉賈,劫其兵,度淮擊楚,楚王交走入薛。上赦天下死罪以下,皆令從軍;徵諸侯兵,上自將以擊布。

十二年冬十月,上破布軍于會缶,布走,令別將追之。

上還,過沛,留,置酒沛宮,悉召故人父老子弟佐酒。發沛中兒得百二十人,教之歌。酒酣,上擊筑,自歌曰:「大風起兮雲飛揚,威加海內兮歸故鄉,安得猛士兮守四方!」令兒皆和習之。上乃起舞,忼慨傷懷,泣數行下。謂沛父兄曰:「游子悲故鄉。吾雖都關中,萬歲之後吾魂魄猶思樂沛。且朕自沛公以誅暴逆,遂有天下,其以沛為朕湯沐邑,復其民,世世無有所與。」沛父老諸母故人日樂飲極歡,道舊故為笑樂。十餘日,上欲去,沛父兄固請。上曰:「吾人眾多,父兄不能給。」乃去。沛中空縣皆之邑西獻。上留止,張飲三日。沛父兄皆頓首曰:「沛幸得復,豐未得,唯陛下哀矜。」上曰:「豐者,吾所生長,極不忘耳。吾特以其為雍齒故反我為魏。」沛父兄固請之,乃并復豐,比沛。

漢別將擊布軍洮水南北,皆大破之,追斬布番陽。

周勃定代,斬陳豨於當城。

詔曰:「吳,古之建國也,日者荊王兼有其地,今死亡後。朕欲復立吳王,其議可者。」長沙王臣等言:「沛侯濞重厚,請立為吳王。」已拜,上召謂濞曰:「汝狀有反相。」因拊其背,曰:「漢後五十年東南有亂,豈汝邪?然天下同姓一家,汝慎毋反。」濞頓首曰:「不敢。」

十一月,行自淮南還。過魯,以大牢祠孔子。

十二月,詔曰:「秦皇帝、楚隱王、魏安釐王、齊愍王、趙悼襄王皆絕亡後。其與秦始皇帝守冢二十家,楚、魏、齊各十家,趙及魏公子亡忌各五家,令視其冢,復亡與它事。」

陳豨降將言豨反時燕王盧綰使人之豨所陰謀。上使辟陽侯審食其迎綰,綰稱疾。食其言綰反有端。春二月,使樊噲、周勃將兵擊綰。詔曰:「燕王綰與吾有故,愛之如子,聞與陳豨有謀,吾以為亡有,故使人迎綰。綰稱疾不來,謀反明矣。燕吏民非有罪也,賜其吏六百石以上爵各一級。與綰居,去來歸者,赦之,加爵亦一級。」詔諸侯王議可立為燕王者,長沙王臣等請立子建為燕王。

詔曰:「南武侯織亦粵之世也,立以為南海王。」

三月,詔曰:「吾立為天子,帝有天下,十二年于今矣。與天下之豪士賢大夫共定天下,同安輯之。其有功者上致之王,次為列侯,下乃食邑。而重臣之親,或為列侯,皆令自置吏,得賦斂,女子公主。為列侯食邑者,皆佩之印,賜大第室。吏二千石,徙之長安,受小第室。入蜀漢定三秦者,皆世世復。吾於天下賢士功臣,可謂亡負矣。其有不義背天子擅起兵者,與天下共伐誅之。布告天下,使明知朕意。」

上擊布時,為流矢所中,行道疾。疾甚,呂后迎良醫。醫入見,上問醫。曰:「疾可治不醫曰可治。」於是上嫚罵之,曰:「吾以布衣提三尺取天下,此非天命乎?命乃在天,雖扁鵲何益!」遂不使治疾,賜黃金五十斤,罷之。呂后問曰:「陛下百歲後,蕭相國既死,誰令代之?」上曰:「曹參可。」問其次,曰:「

王陵可,然少戇,陳平可以助之。陳平知有餘,然難獨任。周勃重厚少文,然安劉氏者必勃也,可令為太尉。」呂后復問其次,上曰:「此後亦非乃所知也。」

盧綰與數千人居塞下候伺,幸上疾愈,自入謝。夏四月甲辰,帝崩于長樂宮。盧綰聞之,遂亡入匈奴。

呂后與審食其謀曰:「諸將故與帝為編戶民,北面為臣,心常鞅鞅,今乃事少主,非盡族是,天下不安。」以故不發喪。人或聞,以語酈商。酈商見審食其曰:「聞帝已崩,四日不發喪,欲誅諸將。誠如此,天下危矣。陳平、灌嬰將十萬守滎陽,樊噲、周勃將二十萬定燕代,此聞帝崩,諸將皆誅,必連兵還鄉,以攻關中。大臣內畔,諸將外反,亡可蹻足待也。」審食其入言之,乃以丁未發喪,大赦天下。

五月丙寅,葬長陵。已下,皇太子群臣皆反至太上皇廟。群臣曰:「帝起細微,撥亂世反之正,平定天下,為漢太祖,功最高。」上尊號曰高皇帝。

初,高祖不脩文學,而性明達,好謀,能聽,自監門戍卒,見之如舊。初順民心作三章之約。天下既定,命蕭何次律令,韓信申軍法,張蒼定章程,叔孫通制禮儀,陸賈造新語。又與功臣剖符作誓,丹書鐵契,金匱石室,藏之宗廟。雖日不暇給,規摹弘遠矣。

贊曰:春秋晉史蔡墨有言,陶唐氏既衰,其後有劉累,學擾龍,事孔甲,范氏其後也。而大夫范宣子亦曰:「祖自虞以上為陶唐氏,在夏為御龍氏,在商為豕韋氏,在周為唐杜氏,晉主夏盟為范氏。」范氏為晉士師,魯文公世奔秦。後歸于晉,其處者為劉氏。劉向云戰國時劉氏自秦獲於魏。秦滅魏,遷大梁,都于豐,故周巿說雍齒曰「豐,故梁徙也」。是以頌高祖云:「漢帝本系,出自唐帝。降及于周,在秦作劉。涉魏而東,遂為豐公。」豐公,蓋太上皇父。其遷日淺,墳墓在豐鮮焉。及高祖即位,置祠祀官,則有秦、晉、梁、荊之巫,世祠天地,綴之以祀,豈不信哉!由是推之,漢承堯運,德祚已盛,斷蛇著符,旗幟上赤,協于火德,自然之應,得天統矣。


\end{pinyinscope}