\article{司馬遷傳}

\begin{pinyinscope}
昔在顓頊,命南正重司天,火正黎司地。唐虞之際,紹重黎之後,使復典之,至于夏商,故重黎氏世序天地。其在周,程伯休甫其後也。當宣王時,官失其守而為司馬氏。司馬氏世典周史。惠襄之間,司馬氏適晉。晉中軍隨會奔魏,而司馬氏入少梁。

自司馬氏去周適晉,分散,或在衛,或在趙,或在秦。其在衛者,相中山。在趙者,以傳劍論顯,蒯聵其後也。在秦者錯,與張儀爭論,於是惠王使錯將兵伐蜀,遂拔,因而守之。錯孫蘄,事武安君白起。而少梁更名夏陽。蘄與武安君阬趙長平軍,還而與之俱賜死杜郵,葬於華池。蘄孫昌,為秦王鐵官。當始皇之時,蒯聵玄孫卬為武信君將而徇朝歌。諸侯之相王,王卬於殷。漢之伐楚,卬歸漢,以其地為河內郡。昌生毋懌,毋懌為漢市長。毋懌生喜,喜為五大夫,卒,皆葬高門。喜生談,談為太史公。

太史公學天官於唐都,受易於楊何,習道論於黃子。太史公仕於建元、元封之間,愍學者不達其意而師誖,乃論六家之要指曰:

易大傳曰:「天下一致而百慮,同歸而殊塗。」夫陰陽、儒、墨、名、法、道德,此務為治者也,直所從言之異路,有省不省耳。嘗竊觀陰陽之術,大詳而眾忌諱,使人拘而多畏,然其序四時之大順,不可失也。儒者博而寡要,勞而少功,是以其事難盡從,然其敘君臣父子之禮,列夫婦長幼之別,不可易也。墨者儉而難遵,是以其事不可遍循,然其彊本節用,不可廢也。法家嚴而少恩,然其正君臣上下之分,不可改也。名家使人儉而善失真,然其正名實,不可不察也。道家使人精神專一,動合無形,澹足萬物,其為術也,因陰陽之大順,采儒墨之善,撮名法之要,與時遷徙,應物變化,立俗施事,無所不宜,指約而易操,事少而功多。儒者則不然,以為人主天下之儀表也,君唱臣和,主先臣隨。如此,則主勞而臣佚。至於大道之要,去健羨,黜聰明,釋此而任術。夫神大用則竭,形大勞則敝;神形蚤衰,欲與天地長久,非所聞也。

夫陰陽,四時、八位、十二度、二十四節各有教令,曰順之者昌,逆之者亡,未必然也,故曰「使人拘而多畏」。夫春生夏長,秋收冬臧,此天道之大經也,弗順則無以為天下紀綱,故曰「四時之大順,不可失也」。

夫儒者,以六藝為法,六藝經傳以千萬數,累世不能通其學,當年不能究其禮,故曰「博而寡要,勞而少功」。若夫列君臣父子之禮,序夫婦長幼之別,雖百家弗能易也。

墨者亦上堯舜,言其德行曰:「堂高三尺,土階三等,茅茨不翦,棌椽不斲;飯土簋,歠土刑,畴粱之食,藜藿之羹;夏日葛衣,冬日鹿裘。」其送死,桐棺三寸,舉音不盡其哀。教喪禮,必以此為萬民率。故天下共若此,則尊卑無別也。夫世異時移,事業不必同,故曰「儉而難遵」也。要曰彊本節用,則人給家足之道也。此墨子之所長,雖百家不能廢也。

法家不別親疏,不殊貴賤,壹斷於法,則親親尊尊之恩絕矣,可以行一時之計,而不可長用也,故曰「嚴而少恩」。若尊主卑臣,明分職不得相踰越,雖百家不能改也。

名家苛察繳繞,使人不得反其意,剸決於名,時失人情,故曰「使人儉而善失真」。若夫控名責實,參伍不失,此不可不察也。

道家無為,又曰無不為,其實易行,其辭難知。其術以虛無為本,以因循為用。無成勢,無常形,故能究萬物之情。不為物先後,故能為萬物主。有法無法,因時為業;有度無度,因物興舍。故曰「聖人不巧,時變是守」。虛者道之常也,因者君之綱也。群臣並至,使各自明也。其實中其聲者謂之端,實不中其聲者謂之款。款言不聽,姦乃不生,賢不肖自分,白黑乃形。在所欲用耳,何事不成!乃合大道,混混冥冥。光燿天下,復反無名。凡人所生者神也,所託者形也。神大用則竭,形大勞則敝,形神離則死。死者不可復生,離者不可復合,故聖人重之。由此觀之,神者生之本,形者生之具。不先定其神形,而曰「我有以治天下」,何由哉?

太史公既掌天官,不治民。有子曰遷。

遷生龍門,耕牧河山之陽。年十歲則誦古文。二十而南游江淮,上會稽,探禹穴,窺九疑,浮沅湘。北涉汶泗,講業齊魯之都,觀夫子遺風,鄉射鄒嶧;阨困蕃、薛、彭城,過梁楚以歸。於是遷仕為郎中,奉使西征巴蜀以南,略邛、筰、昆明,還報命。

是歲,天子始建漢家之封,而太史公留滯周南,不得與從事,發憤且卒。而子遷適反,見父於河雒之間。太史公執遷手而泣曰:「予先,周室之太史也。自上世嘗顯功名虞夏,典天官事。後世中衰,絕於予乎?汝復為太史,則續吾祖矣。今天子接千歲之統,封泰山,而予不得從行,是命也夫!命也夫!予死,爾必為太史;為太史,毋忘吾所欲論著矣。且夫孝,始於事親,中於事君,終於立身;揚名於後世,以顯父母,此孝之大也。夫天下稱周公,言其能論歌文武之德,宣周召之風,達大王王季思慮,爰及公劉,以尊后稷也。幽厲之後,王道缺,禮樂衰,孔子脩舊起廢,論詩書,作春秋,則學者至今則之。自獲麟以來四百有餘歲,而諸侯相兼,史記放絕。今漢興,海內壹統,明主賢君,忠臣義士,予為太史而不論載,廢天下之文,予甚懼焉,爾其念哉!」遷俯首流涕曰:「小子不敏,請悉論先人所次舊聞,不敢闕。」卒三歲,而遷為太史令,紬史記石室金鐀之書。五年而當太初元年,十一月甲子朔旦冬至,天曆始改,建於明堂,諸神受記。

太史公曰:「先人有言:『自周公卒五百歲而有孔子,孔子至於今五百歲,有能紹而明之,正易傳,繼春秋,本詩書禮樂之際。』意在斯乎!意在斯乎!小子何敢攘焉!」

上大夫壺遂曰:「昔孔子為何作春秋哉?」太史公曰:「余聞之董生:『周道廢,孔子為魯司寇,諸侯害之,大夫壅之。孔子知時之不用,道之不行也,是非二百四十二年之中,以為天下儀表,貶諸侯,討大夫,以達王事而已矣。』子曰:『我欲載之空言,不如見之於行事之深切著明也。』春秋上明三王之道,下辨人事之經紀,別嫌疑,明是非,定猶與,善善惡惡,賢賢賤不肖,存亡國,繼絕世,補弊起廢,王道之大者也。易著天地陰陽四時五行,故長於變;禮綱紀人倫,故長於行;書記先王之事,故長於政;詩記山川谿谷禽獸草木牝牡雌雄,故長於風;樂樂所以立,故長於和;春秋辯是非,故長於治人。是故禮以節人,樂以發和,書以道事,詩以達意,易以道化,春秋以道義。撥亂世反之正,莫近於春秋。春秋文成數萬,其指數千。萬物之散聚皆在春秋。春秋之中,弒君三十六,亡國五十二,諸侯奔走不得保社稷者不可勝數。察其所以,皆失其本已。故《易》曰『差以豪氂,謬以千里』。故『臣弒君,子弒父,非一朝一夕之故,其漸久矣』。有國者不可以不知春秋,前有讒而不見,後有賊而不知。為人臣者不可以不知春秋,守經事而不知其宜,遭變事而不知其權。為人君父者而不通於春秋之義者,必蒙首惡之名。為人臣子不通於春秋之義者,必陷篡弒誅死之罪。其實皆以善為之,而不知其義,被之空言不敢辭。夫不通禮義之指,至於君不君,臣不臣,父不父,子不子。夫君不君則犯,臣不臣則誅,父不父則無道,子不子則不孝。此四行者,天下之大過也。以天下大過予之,受而不敢辭。故春秋者,禮義之大宗也。夫禮禁未然之前,法施已然之後;法之所為用者易見,而禮之所為禁者難知。」

壺遂曰:「孔子之時,上無明君,下不得任用,故作春秋,垂空文以斷禮義,當一王之法。今夫子上遇明天子,下得守職,萬事既具,咸各序其宜,夫子所論,欲以何明?」太史公曰:「唯唯,否否,不然。余聞之先人曰:『虙戲至純厚,作易八卦。堯舜之盛,尚書載之,禮樂作焉。湯武之隆,詩人歌之。春秋采善貶惡,推三代之德,褒周室,非獨刺譏而已也。』漢興已來,至明天子,獲符瑞,封禪,改正朔,易服色,受命於穆清,澤流罔極,海外殊俗重譯款塞,請來獻見者,不可勝道。臣下百官力誦聖德,猶不能宣盡其意。且士賢能矣,而不用,有國者恥也;主上明聖,德不布聞,有司之過也。且余掌其官,廢明聖盛德不載,滅功臣賢大夫之業不述,墮先人所言,罪莫大焉。余所謂述故事,整齊其世傳,非所謂作也,而君比之春秋,謬矣。」

於是論次其文。十年而遭李陵之禍,幽於纍紲。乃喟然而歎曰:「是余之罪夫!身虧不用矣。」退而深惟曰:「

夫詩書隱約者,欲遂其志之思也。」卒述陶唐以來,至於麟止,自黃帝始。五帝本紀第一,夏本紀第二,殷本紀第三,周本紀第四,秦本紀第五,始皇本紀第六,項羽本紀第七,高祖本紀第八,呂后本紀第九,孝文本紀第十,孝景本紀第十一,今上本紀第十二。三代世表第一,十二諸侯年表第二,六國年表第三,秦楚之際月表第四,漢諸侯年表第五,高祖功臣年表第六,惠景間功臣年表第七,建元以來侯者年表第八,王子侯者年表第九,漢興以來將相名臣年表第十。禮書第一,樂書第二,律書第三,曆書第四,天官書第五,封禪書第六,河渠書第七,平準書第八。吳太伯世家第一,齊太公世家第二,魯周公世家第三,燕召公世家第四,管蔡世家第五,陳杞世家第六,衛康叔世家第七,宋微子世家第八,晉世家第九,楚世家第十,越世家第十一,鄭世家第十二,趙世家第十三,魏世家第十四,韓世家第十五,田完世家第十六,孔子世家第十七,陳涉世家第十八,外戚世家第十九,楚元王世家第二十,荊燕王世家第二十一,齊悼惠王世家第二十二,蕭相國世家第二十三,曹相國世家第二十四,留侯世家第二十五,陳丞相世家第二十六,絳侯世家第二十七,梁孝王世家第二十八,五宗世家第二十九,三王世家第三十。伯夷列傳第一,管晏列傳第二,老子韓非列傳第三,司馬穰苴列傳第四,孫子吳起列傳第五,伍子胥列傳第六,仲尼弟子列傳第七,商君列傳第八,蘇秦列傳第九,張儀列傳第十,樗里甘茂列傳第十一,穰侯列傳第十二,白起王翦列傳第十三,孟子荀卿列傳第十四,平原虞卿列傳第十五,孟嘗君列傳第十六,魏公子列傳第十七,春申君列傳第十八,范睢蔡澤列傳第十九,樂毅列傳第二十,廉頗藺相如列傳第二十一,田單列傳第二十二,魯仲連列傳第二十三,屈原賈生列傳第二十四,呂不韋列傳第二十五,刺客列傳第二十六,李斯列傳第二十七,蒙恬列傳第二十八,張耳陳餘列傳第二十九,魏豹彭越列傳第三十,黥布列傳第三十一,淮陰侯韓信列傳第三十二,韓王信盧綰列傳第三十三,田儋列傳第三十四,樊酈滕灌列傳第三十五,張丞相倉列傳第三十六,酈生陸賈列傳第三十七,傅靳糨成侯列傳第三十八,劉敬叔孫通列傳第三十九,季布欒布列傳第四十,爰盎朝錯列傳第四十一,張釋之馮唐列傳第四十二,萬石張叔列傳第四十三,田叔列傳第四十四,扁鵲倉公列傳第四十五,吳王濞列傳第四十六,魏其武安列傳第四十七,韓長孺列傳第四十八,李將軍列傳第四十九,衛將軍驃騎列傳第五十,平津主父列傳第五十一,匈奴列傳第五十二,南越列傳第五十三,閩越列傳第五十四,朝鮮列傳第五十五,西南夷列傳第五十六,司馬相如列傳第五十七,淮南衡山列傳第五十八,循吏列傳第五十九,汲鄭列傳第六十,儒林列傳第六十一,酷吏列傳第六十二,大宛列傳第六十三,游俠列傳第六十四,佞幸列傳第六十五,滑稽列傳第六十六,日者列傳第六十七,龜策列傳第六十八,貨殖列傳第六十九。

惟漢繼五帝末流,接三代絕業。周道既廢,秦撥去古文,焚滅詩書,故明堂石室金鐀玉版圖籍散亂。漢興,蕭何次律令,韓信申軍法,張蒼為章程,叔孫通定禮儀,則文學彬彬稍進,詩書往往間出。自曹參薦蓋公言黃老,而賈誼、朝錯明申韓,公孫弘以儒顯,百年之間,天下遺文古事靡不畢集。太史公仍父子相繼篹其職,曰:「於戲!余維先人嘗掌斯事,顯於唐虞。至於周,復典之。故司馬氏世主天官,至於余乎,欽念哉!」罔羅天下放失舊聞,王跡所興,原始察終,見盛觀衰,論考之行事,略三代,錄秦漢,上記軒轅,下至於茲,著十二本紀,既科條之矣。並時異世,年差不明,作十表。禮樂損益,律曆改易,兵權山川鬼神,天人之際,承敝通變,作八書。二十八宿環北辰,三十輻共一轂,運行無窮,輔弼股肱之臣配焉,忠信行道以奉主上,作三十世家。扶義俶儻,不令己失時,立功名於天下,作七十列傳。凡百三十篇,五十二萬六千五百字,為太史公書。序略,以拾遺補蓺,成一家言,協六經異傳,齊百家雜語,臧之名山,副在京師,以俟後聖君子。第七十,遷之自敘云爾。而十篇缺,有錄無書。

遷既被刑之後,為中書令,尊寵任職。故人益州刺史任安予遷書,責以古賢臣之義。遷報之曰:

少卿足下:曩者辱賜書,教以慎於接物,推賢進士為務,意氣勤勤懇懇,若望僕不相師用,而流俗人之言。僕非敢如是也。雖罷駑,亦嘗側聞長者遺風矣。顧自以為身殘處穢,動而見尤,欲益反損,是以抑鬱而無誰語。諺曰:「誰為為之?孰令聽之?」蓋鍾子期死,伯牙終身不復鼓琴。何則?士為知己用,女為說己容。若僕大質已虧缺,雖材懷隨和,行若由夷,終不可以為榮,適足以發笑而自點耳。

書辭宜答,會東從上來,又迫賤事,相見日淺,卒卒無須臾之間得竭指意。今少卿抱不測之罪,涉旬月,迫季冬,僕又薄從上上雍,恐卒然不可諱。是僕終已不得舒憤懣以曉左右,則長逝者魂魄私恨無窮。請略陳固陋。闕然不報,幸勿過。

僕聞之,修身者智之府也,愛施者仁之端也,取予者義之符也,恥辱者勇之決也,立名者行之極也。士有此五者,然後可以託於世,列於君子之林矣。故禍莫憯於欲利,悲莫痛於傷心,行莫醜於辱先,而詬莫大於宮刑。刑餘之人,無所比數,非一世也,所從來遠矣。昔衛靈公與雍渠載,孔子適陳;商鞅因景監見,趙良寒心;同子參乘,爰絲變色:自古而恥之。夫中材之人,事關於宦豎,莫不傷氣。況忼慨之士乎!如今朝雖乏人,柰何令刀鋸之餘薦天下豪雋哉!僕賴先人緒業,得待罪輦轂下,二十餘年矣。所以自惟:上之,不能納忠效信,有奇策材力之譽,自結明主;次之,又不能拾遺補闕,招賢進能,顯巖穴之士;外之,不能備行伍,攻城戰野,有斬將搴旗之功;下之,不能累日積勞,取尊官厚祿,以為宗族交遊光寵。四者無一遂,苟合取容,無所短長之效,可見於此矣。鄉者,僕亦嘗廁下大夫之列,陪外廷末議。不以此時引維綱,盡思慮,今已虧形為埽除之隸,在闒茸之中,乃欲卬首信眉,論列是非,不亦輕朝廷,羞當世之士邪!嗟乎!嗟乎!如僕,尚何言哉!尚何言哉!

且事本末未易明也。僕少負不羈之才,長無鄉曲之譽,主上幸以先人之故,使得奉薄技,出入周衛之中。僕以為戴盆何以望天,故絕賓客之知,忘室家之業,日夜思竭其不肖之材力,務壹心營職,以求親媚於主上。而事乃有大謬不然者。夫僕與李陵俱居門下,素非相善也,趣舍異路,未嘗銜盃酒接殷勤之歡。然僕觀其為人自奇士,事親孝,與士信,臨財廉,取予義,分別有讓,恭儉下人,常思奮不顧身以徇國家之急。其素所畜積也,僕以為有國士之風。夫人臣出萬死不顧一生之計,赴公家之難,斯已奇矣。今舉事壹不當,而全軀保妻子之臣隨而媒孽其短,僕誠私心痛之。且李陵提步卒不滿五千,深踐戎馬之地,足歷王庭,垂餌虎口,橫挑彊胡,卬億萬之師,與單于連戰十餘日,所殺過當。虜救死扶傷不給,旃裘之君長咸震怖,乃悉徵左右賢王,舉引弓之民,一國共攻而圍之。轉鬥千里,矢盡道窮,救兵不至,士卒死傷如積。然李陵一呼勞軍,士無不起,躬流涕,沬血飲泣,張空弮,冒白刃,北首爭死敵。陵未沒時,使有來報,漢公卿王侯奉觴上壽。後數日,陵敗書聞,主上為之食不甘味,聽朝不怡。大臣憂懼,不知所出。僕竊不自料其卑賤,見主上慘悽怛悼,誠欲效其款款之愚。以為李陵素與士大夫絕甘分少,能得人之死力,雖古名將不過也。身雖陷敗,彼觀其意,且欲得其當而報漢。事已無可柰何,其所摧敗,功亦足以暴於天下。僕懷欲陳之,而未有路。適會召問,即以此指推言陵功,欲以廣主上之意,塞睚眥之辭。未能盡明,明主不深曉,以為僕沮貳師,而為李陵游說,遂下於理。拳拳之忠,終不能自列,因為誣上,卒從吏議。家貧,財賂不足以自贖,交遊莫救,左右親近不為壹言。身非木石,獨與法吏為伍,深幽囹圄之中,誰可告愬者!此正少卿所親見,僕行事豈不然邪?李陵既生降,隤其家聲,而僕又茸以蠶室,重為天下觀笑。悲夫!悲夫!

事未易一二為俗人言也。僕之先人非有剖符丹書之功,文史星曆近乎卜祝之間,固主上所戲弄,倡優畜之,流俗之所輕也。假令僕伏法受誅,若九牛亡一毛,與螻螘何異?而世又不與能死節者比,特以為智窮罪極,不能自免,卒就死耳。何也?素所自樹立使然。人固有一死,死有重於泰山,或輕於鴻毛,用之所趨異也。太上不辱先,其次不辱身,其次不辱理色,其次不辱辭令,其次詘體受辱,其次易服受辱,其次關木索被箠楚受辱,其次鬄毛髮嬰金鐵受辱,其次毀肌膚斷支體受辱,最下腐刑,極矣。傳曰「刑不上大夫」,此言士節不可不厲也。猛虎處深山,百獸震恐,及其在阱檻之中,搖尾而求食,積威約之漸也。故士有畫地為牢勢不入,削木為吏議不對,定計於鮮也。今交手足,受木索,暴肌膚,受榜箠,幽於圜牆之中,當此之時,見獄吏則頭槍地,視徒隸則心惕息。何者?積威約之勢也。及已至此,言不辱者,所謂彊顏耳,曷足貴乎!且西伯,伯也,拘牖里;李斯,相也,具五刑;淮陰,王也,受械於陳;彭越、張敖南鄉稱孤,繫獄具罪;絳侯誅諸呂,權傾五伯,囚於請室;魏其,大將也,衣赭關三木;季布為朱家鉗奴;灌夫受辱居室。此人皆身至王侯將相,聲聞鄰國,及罪至罔加,不能引決自財。在塵埃之中,古今一體,安在其不辱也!由此言之,勇怯,勢也;彊弱,形也。審矣,曷足怪乎!且人不能蚤自財繩墨之外,已稍陵夷至於鞭箠之間,乃欲引節,斯不亦遠乎!古人所以重施刑於大夫者,殆為此也。夫人情莫不貪生惡死,念親戚,顧妻子,至激於義理者不然,乃有不得已也。今僕不幸,蚤失二親,無兄弟之親,獨身孤立,少卿視僕於妻子何如哉?且勇者不必死節,怯夫慕義,何處不勉焉!僕雖怯耎欲苟活,亦頗識去就之分矣,何至自湛溺累紲之辱哉!且夫臧獲婢妾猶能引決,況若僕之不得已乎!所以隱忍苟活,函糞土之中而不辭者,恨私心有所不盡,鄙沒世而文采不表於後也。

古者富貴而名摩滅,不可勝記,唯俶儻非常之人稱焉。蓋西伯拘而演周易;仲尼厄而作春秋;屈原放逐,乃賦離騷;左丘失明,厥有國語;孫子髕腳,兵法修列;不韋遷蜀,世傳呂覽;韓非囚秦,說難、孤憤。詩三百篇,大氐賢聖發憤之所為作也。此人皆意有所鬱結,不得通其道,故述往事,思來者。及如左丘明無目,孫子斷足,終不可用,退論書策以舒其憤,思垂空文以自見。僕竊不遜,近自託於無能之辭,網羅天下放失舊聞,考之行事,稽其成敗興壞之理,凡百三十篇,亦欲以究天人之際,通古今之變,成一家之言。草創未就,適會此禍,惜其不成,是以就極刑而無慍色。僕誠已著此書,藏之名山,傳之其人通邑大都,則僕償前辱之責,雖萬被戮,豈有悔哉!然此可為智者道,難為俗人言也。

且負下未易居,下流多謗議。僕以口語遇遭此禍,重為鄉黨戮笑,汙辱先人,亦何面目復上父母之丘墓乎?雖累百世,垢彌甚耳!是以腸一日而九回,居則忽忽若有所亡,出則不知所如往。每念斯恥,汗未嘗不發背霑衣也。身直為閨閤之臣,寧得自引深臧於巖穴邪!故且從俗浮湛,與時俯仰,以通其狂惑。今少卿乃教以推賢進士,無乃與僕之私指謬乎。今雖欲自彫瑑,曼辭以自解,無益,於俗不信,祗取辱耳。要之死日,然後是非乃定。書不能盡意,故略陳固陋。

遷既死後,其書稍出。宣帝時,遷外孫平通侯楊惲祖述其書,遂宣布焉。至王莽時,求封遷後,為史通子。

贊曰:自古書契之作而有史官,其載籍博矣。至孔氏篹之,上繼唐堯,下訖秦繆。唐虞以前雖有遺文,其語不經,故言黃帝、顓頊之事未可明也。及孔子因魯史記而作春秋,而左丘明論輯其本事以為之傳,又篹異同為國語。又有世本,錄黃帝以來至春秋時帝王公侯卿大夫祖世所出。春秋之後,七國並爭,秦兼諸侯,有戰國策。漢興伐秦定天下,有楚漢春秋。故司馬遷據左氏、國語,采世本、戰國策,述楚漢春秋,接其後事,訖于

大漢。其言秦漢,詳矣。至於采經摭傳,分散數家之事,甚多疏略,或有抵梧。亦其涉獵者廣博,貫穿經傳,馳騁古今,上下數千載間,斯以勤矣。又其是非頗繆於聖人,論大道則先黃老而後六經,序遊俠則退處士而進姦雄,述貨殖則崇勢利而羞賤貧,此其所蔽也。然自劉向、揚雄博極群書,皆稱遷有良史之材,服其善序事理,辨而不華,質而不俚,其文直,其事核,不虛美,不隱惡,故謂之實錄。烏呼!以遷之博物洽聞,而不能以知自全,既陷極刑,幽而發憤,書亦信矣。跡其所以自傷悼,小雅巷伯之倫。夫唯大雅「既明且哲,能保其身」,難矣哉!


\end{pinyinscope}