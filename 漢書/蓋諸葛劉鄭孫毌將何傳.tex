\article{蓋諸葛劉鄭孫毌將何傳}

\begin{pinyinscope}
蓋寬饒字次公,魏郡人也。明經為郡文學,以孝廉為郎。舉方正,對策高第,遷諫大夫,行郎中戶將事。劾奏衛將軍張安世子侍中陽都侯彭祖不下殿門,并連及安世居位無補。彭祖時實下門,寬饒坐舉奏大臣非是,左遷為衛司馬。

先是時,衛司馬在部,見衛尉拜謁,常為衛官繇使巿買。寬饒視事,案舊令,遂揖官屬以下行衛者。衛尉私使寬饒出,寬饒以令詣官府門上謁辭。尚書責問衛尉,由是衛官不復私使候、司馬。候、司馬不拜,出先置衛,輒上奏辭,自此正焉。

寬饒初拜為司馬,未出殿門,斷其襌衣,令短離地,冠大冠,帶長劍,躬案行士卒廬室,視其飲食居處,有疾病者身自撫循臨問,加致醫藥,遇之甚有恩。及歲盡交代,上臨饗罷衛卒,衛卒數千人皆叩頭自請,願復留共更一年,以報寬饒厚德。宣帝嘉之,以寬饒為太中大夫,使行風俗,多所稱舉貶黜,奉使稱意。擢為司隸校尉,刺舉無所迴避,小大輒舉,所劾奏眾多,廷尉處其法,半用半不用,公卿貴戚及郡國吏繇使至長安,皆恐懼莫敢犯禁,京師為清。

平恩侯許伯入第,丞相、御史、將軍、中二千石皆賀,寬饒不行。許伯請之,乃往,從西階上,東鄉特坐。許伯自酌曰:「蓋君後至。」寬饒曰:「無多酌我,我乃酒狂。」丞相魏侯笑曰:「次公醒而狂,何必酒也?」坐者皆屬目卑下之。酒酣樂作,長信少府檀長卿起舞,為沐猴與狗鬥,坐皆大笑。寬饒不說,卬視屋而歎曰:「美哉!然富貴無常,忽則易人,此如傳舍,所閱多矣。唯謹慎為得久,君侯可不戒哉!」因起趨出,劾奏長信少府以列卿而沐猴舞,失禮不敬。上欲罪少府,許伯為謝,良久,上乃解。

寬饒為人剛直高節,志在奉公。家貧,奉錢月數千,半以給吏民為耳目言事者。身為司隸,子常步行自戍北邊,公廉如此。然深刻喜陷害人,在位及貴戚人與為怨,又好言事刺譏,奸犯上意。上以其儒者,優容之,然亦不得遷。同列後進或至九卿,寬饒自以行清能高,有益於國,而為凡庸所越,愈失意不快,數上疏諫爭。太子庶子王生高寬饒節,而非其如此,予書曰:「明主知君絜白公正,不畏彊禦,故命君以司察之位,擅君以奉使之權,尊官厚祿已施於君矣。君宜夙夜惟思當世之務,奉法宣化,憂勞天下,雖日有益,月有功,猶未足以稱職而報恩也。自古之治,三王之術各有制度。今君不務循職而已,乃欲以太古久遠之事匡拂天子,數進不用難聽之語以摩切左右,非所以揚令名全壽命者也。方今用事之人皆明習法令,言足以飾君之辭,文足以成君之過,君不惟蘧氏之高蹤,而慕子胥之末行,用不訾之軀,臨不測之險,竊為君痛之。夫君子直而不挺,曲而不詘。大雅云:『既明且哲,以保其身。』狂夫之言,聖人擇焉。唯裁省覽。」寬饒不納其言。

是時上方用刑法,信任中尚書宦官,寬饒奏封事曰:「方今聖道浸廢,儒術不行,以刑餘為周召,以法律為詩書。」又引韓氏易傳言:「五帝官天下,三王家天下,家以傳子,官以傳賢,若四時之運,功成者去,不得其人則不居其位。」書奏,上以寬饒怨謗終不改,下其書中二千石。時執金吾議,以為寬饒指意欲求禪,大逆不道。諫大夫鄭昌愍傷寬饒忠直憂國,以言事不當意而為文吏所詆挫,上書頌寬饒曰:「臣聞山有猛獸,藜藿為之不采;國有忠臣,姦邪為之不起。司隸校尉寬饒居不求安,食不求飽,進有憂國之心,退有死節之義,上無許、史之屬,下無金、張之託,職在司察,直道而行,多仇少與,上書陳國事,有司劾以大辟,臣幸得從大夫之後,官以諫為名,不敢不言。」上不聽,遂下寬饒吏。寬饒引佩刀自剄北闕下,眾莫不憐之。

諸葛豐字少季,琅邪人也。以明經為郡文學,名特立剛直。貢禹為御史大夫,除豐為屬,舉侍御史。元帝擢為司隸校尉,刺舉無所避,京師為之語曰:「間何闊,逢諸葛。」上嘉其節,加豐秩光祿大夫。

時侍中許章以外屬貴幸,奢淫不奉法度,賓客犯事,與章相連。豐案劾章,欲奏其事,適逢許侍中私出,豐駐車舉節詔章曰:「下!」欲收之。章迫窘,馳車去,豐追之。許侍中因得入宮門,自歸上。豐亦上奏,於是收豐節。司隸去節自豐始。

豐上書謝曰:「臣豐駑怯,文不足以勸善,武不足以執邪。陛下不量臣能否,拜為司隸校尉,未有以自效,復秩臣為光祿大夫,官尊責重,非臣所當處也。又迫年歲衰暮,常恐卒填溝渠,德無以報厚,使論議士譏臣無補,長獲素餐之名。故常願捐一旦之命,不待時而斷姦臣之首,縣於都市,編書其罪,使四方明知為惡之罰,然後卻就斧鉞之誅,誠臣所甘心也。夫以布衣之士,尚猶有刎頸之交,今以四海之大,曾無伏節死誼之臣,率盡苟合取容,阿黨相為,念私門之利,忘國家之政。邪穢濁溷之氣上感于天,是以災變數見,百姓困乏。此臣下不忠之效也,臣誠恥之亡已。凡人情莫不欲安存而惡危亡,然忠臣直士不避患害者,誠為君也。今陛下天覆地載,物無不容,使尚書令堯賜臣豐書曰:『夫司隸者刺舉不法,善善惡惡,非得顓之也。免處中和,順經術意。』恩深德厚,臣豐頓首幸甚。臣竊不勝憤懣,願賜清宴,唯陛下裁幸。」上不許。

是後所言益不用,豐復上書言:「臣聞伯奇孝而棄於親,子胥忠而誅於君,隱公慈而殺於弟,叔武弟而殺於兄。夫以四子之行,屈平之材,然猶不能自顯而被刑戮,豈不足以觀哉!使臣殺身以安國,蒙誅以顯君,臣誠願之。獨恐未有云補,而為眾邪所排,令讒夫得遂,正直之路雍塞,忠臣沮心,智士杜口,此愚臣之所懼也。」

豐以春夏繫治人,在位多言其短。上徙豐為城門校尉,豐上書告光祿勳周堪、光祿大夫張猛。上不直豐,乃制詔御史:「城門校尉豐,前與光祿勳堪、光祿大夫猛在朝之時,數稱言堪、猛之美。豐前為司隸校尉,不順四時,修法度,專作苛暴,以獲虛威,朕不忍下吏,以為城門校尉。不內省諸己,而反怨堪、猛,以求報舉,告案無證之辭,暴揚難驗之罪,毀譽恣意,不顧前言,不信之大者也。朕憐豐之耆老,不忍加刑,其免為庶人。」終於家。

劉輔,河間宗室也。舉孝廉,為襄賁令。上書言得失,召見,上美其材,擢為諫大夫。會成帝欲立趙婕妤為皇后,先下詔封婕妤父臨為列侯。輔上書言:「臣聞天之所與必先賜以符瑞,天之所違必先降以災變,此神明之徵應,自然之占驗也。昔武王、周公承順天地,以饗魚烏之瑞,然猶君臣祗懼,動色相戒,況於季世,不蒙繼嗣之福,屢受威怒之異者虖!雖夙夜自責,改過易行,畏天命,念祖業,妙選有德之世,考卜窈窕之女,以承宗廟,順神祇心,塞天下望,子孫之祥猶恐晚暮,今乃觸情縱欲,傾於卑賤之女,欲以母天下,不畏于天,不媿于人,惑莫大焉。里語曰:『腐木不可以為柱,卑人不可以為主。』天人之所不予,必有禍而無福,市道皆共知之,朝廷莫肯壹言,臣竊傷心。自念得以同姓拔擢,尸祿不忠,污辱諫爭之官,不敢不盡死,唯陛下深察。」書奏,上使侍御史收縛輔,繫掖庭祕獄,群臣莫知其故。

於是中朝左將軍辛慶忌、右將軍廉褒、光祿勳師丹、太中大夫谷永俱上書曰:「臣聞明王垂寬容之聽,崇諫爭之官,廣開忠直之路,不罪狂狷之言,然後百僚在位,竭忠盡謀,不懼後患,朝廷無諂諛之士,元首無失道之伥。竊見諫大夫劉輔,前以縣令求見,擢為諫大夫,此其言必有卓詭切至,當聖心者,故得拔至於此。旬日之間,收下祕獄,臣等愚,以為輔幸得託公族之親,在諫臣之列,新從下土來,未知朝廷體,獨觸忌諱,不足深過。小罪宜隱忍而已,如有大惡,宜暴治理官,與眾共之。昔趙簡子殺其大夫鳴犢,孔子臨河而還。今天心未豫,災異屢降,水旱迭臻,方當隆寬廣問,褒直盡下之時也。而行慘急之誅於諫爭之臣,震驚群下,失忠直心。假令輔不坐直言,所坐不著,天下不可戶曉。同姓近臣本以言顯,其於治親養忠之義誠不宜幽囚于掖庭獄。公卿以下見陛下進用輔亟,而折傷之暴,人有懼心,精銳銷耎,莫敢盡節正言,非所以昭有虞之聽,廣德美之風也。臣等竊深傷之,唯陛下留神省察。」

上乃徙繫輔共工獄,減死罪一等,論為鬼薪。終於家。

鄭崇字子游,本高密大族,世與王家相嫁娶。祖父以訾徙平陵。父賓明法令,為御史,事貢公,名公直。崇少為郡文學史,至丞相大車屬。弟立與高武侯傅喜同門學,相友善。喜為大司馬,薦崇,哀帝擢為尚書僕射。數求見諫爭,上初納用之。每見曳革履,上笑曰:「我識鄭尚書履聲。」

久之,上欲封祖母傅太后從弟商,崇諫曰:「孝成皇帝封親舅五侯,天為赤黃晝昏,日中有黑氣。今祖母從昆弟二人已侯。孔鄉侯,皇后父;高武侯以三公封,尚有因緣。今無故欲復封商,壞亂制度,逆天人心,非傅氏之福也。臣聞師曰:『逆陽者厥極弱,逆陰者厥極凶短折,犯人者有亂亡之患,犯神者有疾夭之禍。』故周公著戒曰:『惟王不知艱難,唯耽樂是從,時亦罔有克壽。』故衰世之君夭折蚤沒,此皆犯陰之害也。臣願以身命當國咎。」崇因持詔書案起。傅太后大怒曰:「何有為天子乃反為一臣所顓制邪!」上遂下詔曰:「朕幼而孤,皇太太后躬自養育,免于襁褓,教道以禮,至於成人,惠澤茂焉。『欲報之德,皞天罔極。』前追號皇太太后父為崇祖侯,惟念德報未殊,朕甚恧焉。侍中光祿大夫商,皇太太后父同產子,小自保大,恩義最親。其封商為汝昌侯,為崇祖侯後,更號崇祖侯為汝昌哀侯。」

崇又以董賢貴寵過度諫,由是重得罪。數以職事見責,發疾頸癰,欲乞骸骨,不敢。尚書令趙昌佞諂,素害崇,知其見疏,因奏崇與宗族通,疑有姦,請治。上責崇曰:「君門如巿人,何以欲禁切主上?」崇對曰:「臣門如巿,臣心如水。願得考覆。」上怒,下崇獄,窮治,死獄中。

孫寶字子嚴,潁川鄢陵人也。以明經為郡吏。御史大夫張忠辟寶為屬,欲令授子經,更為除舍,設儲偫。寶自劾去,忠固還之,心內不平。後署寶主簿,寶徙入舍,祭灶請比鄰。忠陰察,怪之,使所親問寶:「前大夫為君設除大舍,子自劾去者,欲為高節也。今兩府高士俗不為主簿,子既為之,徙舍甚說,何前後不相副也?」寶曰:「高士不為主簿,而大夫君以寶為可,一府莫言非,士安得獨自高?前日君男欲學文,而移寶自近。禮有來學,義無往教;道不可詘,身詘何傷?且不遭者可無不為,況主簿乎!」忠聞之,甚慚,上書薦寶經明質直,宜備近臣。為議郎,遷諫大夫。

鴻嘉中,廣漢群盜起,選為益州刺史。廣漢太守扈商者,大司馬車騎將軍王音姊子,軟弱不任職。寶到部,親入山谷,諭告群盜,非本造意。渠率皆得悔過自出,遣歸田里。自劾矯制,奏商為亂首,春秋之義,誅首惡而已。商亦奏寶所縱或有渠率當坐者。商徵下獄,寶坐失死罪免。益州吏民多陳寶功效,言為車騎將軍所排。上復拜寶為冀州刺史,遷丞相司直。

時帝舅紅陽侯立使客因南郡太守李尚占墾草田數百頃,頗有民所假少府陂澤,略皆開發,上書願以入縣官。有詔郡平田予直,錢有貴一萬萬以上。寶聞之,遣丞相史按驗,發其姦,劾奏立、尚懷姦罔上,狡猾不道。尚下獄死。立雖不坐,後兄大司馬衛將軍商薨,次當代商,上度立而用其弟曲陽侯根為大司馬票騎將軍。

會益州蠻夷犯法,巴蜀頗不安,上以寶著名西州,拜為廣漢太守,秩中二千石,賜黃金三十斤。蠻夷安輯,吏民稱之。

徵為京兆尹。故吏侯文以剛直不苟合常稱疾不肯仕,寶以恩禮請文,欲為布衣友,日設酒食,妻子相對。文求受署為掾,進見如賓禮。數月,以立秋日署文東部督郵。入見,敕曰:「今日鷹隼始擊,當順天氣取姦惡,以成嚴霜之誅,掾部渠有其人乎?」文卬曰:「無其人不敢空受職。」寶曰:「誰也?」文曰:「霸陵杜稚季。」寶曰:「其次。」文曰:「豺狼橫道,不宜復問狐狸。」寶默然。稚季者大俠,與衛尉淳于長、大鴻臚蕭育等皆厚善。寶前失車騎將軍,與紅陽侯有卻,自恐見危,時淳于長方貴幸,友寶,寶亦欲附之,始視事而長以稚季託寶,故寶窮,無以復應文。文怪寶氣索,知其有故,因曰:「明府素著威名,今不敢取稚季,當且闔閤,勿有所問。如此竟歲,吏民未敢誣明府也。即度稚季而譴它事,眾口讙譁,終身自墮。」寶曰:「受教。」稚季耳目長,聞知之,杜門不通水火,穿舍後牆為小戶,但持鉏自治園,因文所厚自陳如此。文曰:「我與稚季幸同土壤,素無睚眥,顧受將命,分當相直。誠能自改,嚴將不治前事,即不更心,但更門戶,適趣禍耳。」稚季遂不敢犯法,寶亦竟歲無所譴。明年,稚季病死。寶為京兆尹三歲,京師稱之。會淳于長敗,寶與蕭育等皆坐免官。文復去吏,死於家。稚季子杜蒼,字君敖,名出稚季右,在游俠中。

哀帝即位,徵寶為諫大夫,遷司隸。初,傅太后與中山孝王母馮太后俱事元帝,有卻,傅太后使有司考馮太后,令自殺,眾庶冤之。寶奏請覆治,傅太后大怒,曰:「帝置司隸,主使察我。馮氏反事明白,故欲擿觖以揚我惡。我當坐之。」上乃順指下寶獄。尚書僕射唐林爭之,上以林朋黨比周,左遷敦煌魚澤障候。大司馬傅喜、光祿大夫龔勝固爭,上為言太后,出寶復官。

頃之,鄭崇下獄,寶上書曰:「臣聞疏不圖親,外不慮內。臣幸得銜命奉使,職在刺舉,不敢避貴幸之勢,以塞視聽之明。按尚書令昌奏僕射崇,下獄覆治,榜掠將死,卒無一辭,道路稱冤。疑昌與崇內有纖介,浸潤相陷,自禁門內樞機近臣,蒙受冤譖,虧損國家,為謗不小。臣請治昌,以解眾心。」書奏,天子不說,以寶名臣不忍誅,乃制詔丞相大司空:「司隸寶奏故尚書僕射崇冤,請獄治尚書令昌。案崇近臣,罪惡暴著,而寶懷邪,附下罔上,以春月作詆欺,遂其姦心,蓋國之賊也。傳不云乎?『

惡利口之覆國家。』其免寶為庶人。」

哀帝崩,王莽白王太后徵寶以為光祿大夫,與王舜等俱迎中山王。平帝立,寶為大司農。會越嶲郡上黃龍游江中,太師孔光、大司徒馬宮等咸稱莽功德比周公,宜告祠宗廟。寶曰:「周公上聖,召公大賢。尚猶有不相說,著於經典,兩不相損。今風雨未時,百姓不足,每有一事,群臣同聲,得無非其美者。」時大臣皆失色,侍中奉車都尉甄邯即時承制罷議者。會寶遣吏迎母,母道病,留弟家,獨遣妻子。司直陳崇以奏寶,事下三公即訊。寶對曰:「年七十誖眊,恩衰共養,營妻子,如章。」寶坐免,終於家。建武中,錄舊德臣,以寶孫伉為諸長。

毌將隆字君房,東海蘭陵人也。大司馬車騎將軍王音內領尚書,外典兵馬,踵故選置從事中郎與參謀議,奏請隆為從事中郎,遷諫大夫。成帝末,隆奏封事言:「古者選諸侯入為公卿,以褒功德,宜徵定陶王使在國邸,以填萬方。」其後上竟立定陶王為太子,隆遷冀州牧、潁川太守。哀帝即位,以高第入為京兆尹,遷執金吾。

時侍中董賢方貴,上使中黃門發武庫兵,前後十輩,送董賢及上乳母王阿舍。隆奏言:「武庫兵器,天下公用,國家武備,繕治造作,皆度大司農錢。大司農錢自乘輿不以給共養,共養勞賜,壹出少府。蓋不以本臧給末用,不以民力共浮費,別公私,示正路也。古者諸侯方伯得顓征伐,乃賜斧鉞。漢家邊吏,職在距寇,亦賜武庫兵,皆任其事然後蒙之。春秋之誼,家不臧甲,所以抑臣威,損私力也。今賢等便僻弄臣,私恩微妾,而以天下公用給其私門,契國威器共其家備。民力分於弄臣,武兵設於微妾,建立非宜,以廣驕僭,非所以示四方也。孔子曰:『奚取於三家之堂!』臣請收還武庫。」上不說。

頃之,傅太后使謁者買諸官婢,賤取之,復取執金吾官婢八人。隆奏言賈賤,請更平直。上於是制詔丞相、御史大夫:「交讓之禮興,則虞芮之訟息。隆位九卿,既無以匡朝廷之不逮,而反奏請與永信宮爭貴賤之賈,程奏顯言,眾莫不聞。舉錯不由誼理,爭求之名自此始,無以示百僚,傷化失俗。」以隆前有安國之言,左遷為沛郡都尉,遷南郡太守。

王莽少時,慕與隆交,隆不甚附。哀帝崩,莽秉政,使大司徒孔光奏隆前為冀州牧治中山馮太后獄冤陷無辜,不宜處位在中土。本中謁者令史立、侍御史丁玄自典考之,但與隆連名奏事。史立時為中太僕,丁玄泰山太守,及尚書令趙昌譖鄭崇者為河內太守,皆免官,徙合浦。

何並字子廉,祖父以吏二千石自平輿徙平陵。並為郡吏,至大司空掾,事何武。武高其志節,舉能治劇,為長陵令,道不拾遺。

初,邛成太后外家王氏貴,而侍中王林卿通輕俠,傾京師。後坐法免,賓客愈盛,歸長陵上冢,因留飲連日。並恐其犯法,自造門上謁,謂林卿曰:「冢間單外,君宜以時歸。」林卿曰:「諾。」先是林卿殺婢婿埋冢舍,並具知之,以非己時,又見其新免,故不發舉,欲無令留界中而已,即且遣吏奉謁傳送。林卿素驕,慚於賓客,並度其為變,儲兵馬以待之。林卿既去,北度涇橋,令騎奴還至寺門,拔刀剝其建鼓。並自從吏兵追林卿。行數十里,林卿迫窘,乃令奴冠其冠被其襜褕自代,乘車從童騎,身變服從間徑馳去。會日暮追及,收縛冠奴,奴曰:「我非侍中,奴耳。」並心自知已失林卿,乃曰:「王君困,自稱奴,得脫死邪?」叱吏斷頭持還,縣所剝鼓置都亭下,署曰:「故侍中王林卿坐殺人埋冢舍,使奴剝寺門鼓。」吏民驚駭。林卿因亡命,眾庶讙譁,以為實死。成帝太后以邛成太后愛林卿故,聞之涕泣,為言哀帝。哀帝問狀而善之,遷並隴西太守。

徙潁川太守,代陵陽嚴詡。詡本以孝行為官,謂掾史為師友,有過輒閉閤自責,終不大言。郡中亂,王莽遣使徵詡,官屬數百人為設祖道,詡據地哭。掾史曰:「明府吉徵,不宜若此。」詡曰:「吾哀潁川士,身豈有憂哉!我以柔弱徵,必選剛猛代。代到,將有僵仆者,故相弔耳。」詡至,拜為美俗使者。是時潁川鍾元為尚書令,領廷尉,用事有權。弟威為郡掾,臧千金。並為太守,故辭鍾廷尉,廷尉免冠為弟請一等之罪,願蚤就髡鉗。並曰:「罪在弟身與君律,不在於太守。」元懼,馳遣人呼弟。陽翟輕俠趙季、李款多畜賓客,以氣力漁食閭里,至姦人婦女,持吏長短,從橫郡中,聞並且至,皆亡去。並下車求勇猛曉文法吏且十人,使文吏治三人獄,武吏往捕之,各有所部。敕曰:「三人非負太守,乃負王法,不得不治。鍾威所犯多在赦前,驅使入函谷關,勿令汙民間;不入關,乃收之。趙、李桀惡,雖遠去,當得其頭,以謝百姓。」鍾威負其兄,止雒陽,吏格殺之。亦得趙、李它郡,持頭還,並皆縣頭及其具獄於市。郡中清靜,表善好士,見紀潁川,名次黃霸。性清廉,妻子不至官舍。數年,卒。疾病,召丞掾作先令書,曰:「告子恢,吾生素餐日久,死雖當得法賻,勿受。葬為小槨,亶容下棺。」恢如父言。王莽擢恢為關都尉。建武中以並孫為郎。

贊曰:蓋寬饒為司臣,正色立於朝,雖詩所謂「國之司直」無以加也。若采王生之言以終其身,斯近古之賢臣矣。諸葛、劉、鄭雖云狂瞽,有異志焉。孔子曰:「吾未見剛者。」以數子之名跡,然毌將汙於冀州,孫寶橈於定陵,況俗人乎!何並之節,亞尹翁歸云。


\end{pinyinscope}