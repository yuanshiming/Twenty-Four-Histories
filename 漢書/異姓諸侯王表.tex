\article{異姓諸侯王表}

\begin{pinyinscope}
昔詩書述虞夏之際,舜禹受襢,積德累功,洽於百姓,攝位行政,考之于天,經數十年,然後在位。殷周之王,乃繇镨稷,修仁行義,歷十餘世,至于湯武,然後放殺。秦起襄公,章文、繆,獻、孝、昭、嚴,稍蠶食六國,百有餘載,至始皇,乃并天下。以德若彼,用力如此其艱難也。

秦既稱帝,患周之敗,以為起於處士橫議,諸侯力爭,四夷交侵,以弱見奪。於是削去五等,墮城銷刃,箝語燒書,內鋤雄俊,外攘胡粵,用壹威權,為萬世安。然十餘年間,猛敵橫發乎不虞,適戍彊於五伯,閭閻偪於戎狄,嚮應钇於謗議,奮臂威於甲兵。鄉秦之禁,適所以資豪桀而速自斃也。是以漢亡尺土之階,繇一劍之任,五載而成帝業。書傳所記,未嘗有焉。何則?古世相革,皆承聖王之烈,今漢獨收孤秦之弊。鐫金石者難為功,摧枯朽者易為力,其勢然也。故據漢受命,譜十八王,月而列之,天下一統,乃以年數。訖于孝文,異姓盡矣。

公元前漢楚趙齊雍塞翟燕魏韓


\end{pinyinscope}