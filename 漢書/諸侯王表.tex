\article{諸侯王表}

\begin{pinyinscope}
昔周監於二代,三聖制法,立爵五等,封國八百,同姓五十有餘。周公、康叔建於魯、衛,各數百里;太公於齊,亦五侯九伯之地。詩載其制曰:「介人惟藩,大師惟垣。大邦惟屏,大宗惟翰。懷德惟寧,宗子惟城。母俾城壞,毋獨斯畏。」所以親親賢賢,褒表功德,關諸盛衰,深根固本,為不可拔者也。故盛則周、邵相其治,致刑錯;衰則五伯扶其弱,與共守。自幽、平之後,日以陵夷,至虖阨挢河洛之間,分為二周,有逃責之臺,被竊鈇之言。然天下謂之共主,彊大弗之敢傾。歷載八百餘年,數極德盡,既於王赧,降為庶人,用天年終。號位已絕於天下,尚猶枝葉相持,莫得居其虛位,海內無主,三十餘年。

秦據勢勝之地,騁狙詐之兵,蠶食山東,壹切取勝。因矜其所習,自任私知,姍笑三代,盪滅古法,竊自號為皇帝,而子弟為匹夫,內亡骨肉本根之輔,外亡尺土藩翼之衛。陳、吳奮其白挺,劉、項隨而斃之。故曰,周過其曆,秦不及期,國勢然也。

漢興之初,海內新定,同姓寡少,懲戒亡秦孤立之敗,於是剖裂疆土,立二等之爵。功臣侯者百有餘邑,尊王子弟,大啟九國。自鴈門以東,盡遼陽,為燕、代。常山以南,太行左轉,度河、濟,漸于海,為齊、趙。穀、泗以往,奄有龜、蒙,為梁、楚。東帶江、湖,薄會稽,為荊吳。北界淮瀕,略廬、衡,為淮南。波漢之陽,亙九嶷,為長沙。諸侯北境,周市三垂,外接胡越。天子自有三河、東郡、潁川、南陽,自江陵以西至巴蜀,北自雲中至隴西,與京師內史凡十五郡,公主、列侯頗邑其中。而藩國大者夸州兼郡,連城數十,宮室百官同制京師,可謂撟枉過其正矣。雖然,高祖創業,日不暇給,孝惠享國又淺,高后女主攝位,而海內晏如,亡狂狡之憂,卒折諸呂之難,成太宗之業者,亦賴之於諸侯也。

然諸侯原本以大,末流濫以致溢,小者淫荒越法,大者睽孤橫逆,以害身喪國。故文帝采賈生之議分齊、趙,景帝用晁錯之計削吳、楚。武帝施主父之冊,下推恩之令,使諸侯王得分戶邑以封子弟,不行黜陟,而藩國自析。自此以來,齊分為七趙分為六,梁分為五,淮南分為三。皇子始立者,大國不過十餘城。長城、燕、代雖有舊名,皆亡南北邊矣。景遭七國之難,抑損諸侯,減黜其官。武有衡山、淮南之謀,作左官之律,設附益之法,諸侯惟得衣食稅租,不與政事。

至於哀、平之際,皆繼體苗裔,親屬疏遠,生於帷牆之中,不為士民所尊,勢與富室亡異。而本朝短世,國統三絕,是故王莽知漢中外殫微,本末俱弱,亡所忌憚,生其姦心;因母后之權,假伊周之稱,顓作威福廟堂之上,不降階序而運天下。詐謀既成,遂據南面之尊,分遣五威之吏,馳傳天下,班行符命。漢諸侯王厥角绷首,奉上璽韍,惟恐在後,或乃稱美頌德,以求容媚,豈不哀哉!是以究其終始彊弱之變,明監戒焉。

號諡屬始封子孫曾孫玄孫六世七世


\end{pinyinscope}