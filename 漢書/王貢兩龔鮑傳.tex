\article{王貢兩龔鮑傳}

\begin{pinyinscope}
昔武王伐紂,遷九鼎於雒邑,伯夷、叔齊薄之,餓于首陽,不食其祿,周猶稱盛德焉。然孔子賢此二人,以為「不降其志,不辱其身」也。而孟子亦云:「聞伯夷之風者,貪夫廉,懦夫有立志;」「奮乎百世之上,行乎百世之下莫不興起,非賢人而能若是乎!」

漢興有園公、綺里季、夏黃公、甪里先生,此四人者,當秦之世,避而入商雒深山,以待天下之定也。自高祖聞而召之,不至。其後呂后用留侯計,使皇太子卑辭束帛致禮,安車迎而致之。四人既至,從太子見,高祖客而敬焉,太子得以為重,遂用自安。語在留侯傳。

其後谷口有鄭子真,蜀有嚴君平,皆修身自保,非其服弗服,非其食弗食。成帝時,元舅大將軍王鳳以禮聘子真,子真遂不詘而終。君平卜筮於成都巿,以為「卜筮者賤業,而可以惠眾人。有邪惡非正之問,則依蓍龜為言利害。與人子言依於孝,與人弟言依於順,與人臣言依於忠,各因勢導之以善,從吾言者,已過半矣。」裁日閱數人,得百錢足自養,則閉肆下簾而授老子。博覽亡不通,依老子、嚴周之指著書十餘萬言。楊雄少時從遊學,以而仕京師顯名,數為朝廷在位賢者稱君平德。杜陵李彊素善雄,久之為益州牧,喜謂雄曰:「吾真得嚴君平矣。」雄曰:「君備禮以待之,彼人可見而不可得詘也。」彊心以為不然。及至蜀,致禮與相見,卒不敢言以為從事,乃歎曰:「楊子雲誠知人!」君平年九十餘,遂以其業終,蜀人愛敬,至今稱焉。及雄著書言當世士,稱此二人。其論曰:「或問:君子疾沒世而名不稱,盍勢諸?名,卿可幾。曰:君子德名為幾。梁、齊、楚、趙之君非不富且貴也,惡虖成其名!谷口鄭子真不詘其志,耕於巖石之下,名震於京師,豈其卿?豈其卿?楚兩龔之絜,其清矣乎!蜀嚴湛冥,不作苟見,不治苟得,久幽而不改其操,雖隨、和何以加諸?舉茲以旃,不亦寶乎!」

自園公、綺里季、夏黃公、甪里先生、鄭子真、嚴君平皆未嘗仕,然其風聲足以激貪厲俗,近古之逸民也。若王吉、貢禹、兩龔之屬,皆以禮讓進退云。

王吉字子陽,琅玡皋虞人也。少時學明經,以郡吏舉孝廉為郎,補若盧右丞,遷雲陽令。舉賢良為昌邑中尉,而王好遊獵,驅馳國中,動作亡節,吉上疏諫,曰:

臣聞古者師日行三十里,吉行五十里。《詩》云:「匪風發兮,匪車揭兮,顧瞻周道,中心怛兮。」說曰:是非古之風也,發發者;是非古之車也,揭揭者。蓋傷之也。今者大王幸方與,曾不半日而馳二百里,百姓頗廢耕桑,治道牽馬,臣愚以為民不可數變也。昔召公述職,當民事時,舍於棠下而聽斷焉。是時人皆得其所,後世思其仁恩,至虖不伐甘棠,甘棠之詩是也。

大王不好書術而樂逸游,馮式撙銜,馳騁不止,口倦乎叱吒,手苦於箠轡,身勞乎車輿;朝則冒霧露,晝則被塵埃,夏則為大暑之所暴炙,冬則為風寒之所匽薄。數以耎脆之玉體犯勤勞之煩毒,非所以全壽命之宗也,又非所以進仁義之隆也。

夫廣夏之下,細旃之上,明師居前,勸誦在後,上論唐虞之際,下及殷周之盛,考仁聖之風,習治國之道,訢訢焉發憤忘食,日新厥德,其樂豈徒銜橛之間哉!休則俛仰詘信以利形,進退步趨以實下,吸新吐故以練臧,專意積精以適神,於以養生,豈不長哉!大王誠留意如此,則心有堯舜之志,體有喬松之壽,美聲廣譽登而上聞,則福祿其轃而社稷安矣。

皇帝仁聖,至今思慕未怠,於宮館囿池弋獵之樂未有所幸,大王宜夙夜念此,以承聖意。諸侯骨肉,莫親大王,大王於屬則子也,於位則臣也,一身而二任之責加焉,恩愛行義孅介有不具者,於以上聞,非饗國之福也。臣吉愚戇,願大王察之。

王賀雖不遵道,然猶知敬禮吉,乃下令曰:「寡人造行不能無惰,中慰甚忠,數輔吾過。使謁者千秋賜中尉牛肉五百斤,酒五石,脯五束。」其後復放從自若。吉輒諫爭,甚得輔弼之義,雖不治民,國中莫不敬重焉。

久之,昭帝崩,亡嗣,大將軍霍光秉政,遣大鴻臚宗正迎昌邑王。吉即奏書戒王曰:「臣聞高宗諒闇,三年不言。今大王以喪事徵,宜日夜哭泣悲哀而已,慎毋有所發。且何獨喪事,凡南面之君何言哉?天不言,四時行焉,百物生焉,願大王察之。大將軍仁愛勇智,忠信之德天下莫不聞,事孝武皇帝二十餘年未嘗有過。先帝棄群臣,屬以天下,寄幼孤焉,大將軍抱持幼君襁褓之中,布政施教,海內晏然,雖周公、伊尹亡以加也。今帝崩亡嗣,大將軍惟思可以奉宗廟者,攀援而立大王,其仁厚豈有量哉!臣願大王事之敬之,政事壹聽之,大王垂拱南面而已。願留意,嘗以為念。」

王既到,即位二十餘日以行淫亂廢。昌邑群臣坐在國時不舉奏王罪過,令漢朝不聞知,又不能輔道,陷王大惡,皆下獄誅。唯吉與郎中令龔遂以忠直數諫正得減死,髡為城旦。

起家復為益州刺史,病去官,復徵為博士諫大夫。是時宣帝頗修武帝故事,宮室車服盛於昭帝。時外戚許、史、王氏貴寵,而上躬親政事,任用能吏。吉上疏言得失,曰:

陛下躬聖質,總萬方,帝王圖籍日陳于前,惟思世務,將興太平。詔書每下,民欣然若更生。臣伏而思之,可謂至恩,未可謂本務也。

欲治之主不世出,公卿幸得遭遇其時,言聽諫從,然未有建萬世之長策,舉明主於三代之隆者也。其務在於期會簿書,斷獄聽訟而已,此非太平之基也。

臣聞聖王宣德流化,必自近始。朝廷不備,難以言治;左右不正,難以化遠。民者,弱而不可勝,愚而不可欺也。聖主獨行於深宮,得則天下稱誦之,失則天下咸言之。行發於近,必見於遠,故謹選左右,審擇所使;左右所以正身也,所使所以宣德也。《詩》云:「濟濟多士,文王以寧。」此其本也。

春秋所以大一統者,六合同風,九州共貫也。今俗吏所以牧民者,非有禮義科指可世世通行者也,獨設刑法以守之。其欲治者,不知所繇,以意穿鑿,各取一切,權譎自在,故一變之後不可復修也。是以百里不同風,千里不同俗,戶異政,人殊服,詐偽萌生,刑罰亡極,質樸日銷,恩愛寖薄。孔子曰「安上治民,莫善於禮」,非空言也。王者未制禮之時,引先王禮宜於今者而用之。臣願陛下承天心,發大業,與公卿大臣延及儒生,述舊禮,明王制,驅一世之民濟之仁壽之域,則俗何以不若成康,壽何以不若高宗?竊見當世趨務不合於道者,謹條奏,唯陛下財擇焉。

吉意以為「夫婦,人倫大綱,夭壽之萌也。世俗嫁娶太早,未知為人父母之道而有子,是以教化不明而民多夭。聘妻送女亡節,則貧人不及,故不舉子。又漢家列侯尚公主,諸侯則國人承翁主,使男事女,夫詘於婦,逆陰陽之位,故多女亂。古者衣服車馬貴賤有章,以褒有德而別尊卑,今上下僭差,人人自制,是以貪財趨利,不畏死亡。周之所以能致治,刑措而不用者,以其禁邪於冥冥,絕惡於未萌也。」又言「舜、湯不用三公九卿之世而舉皋陶、伊尹,不仁者遠。今使俗吏得任子弟,率多驕驁,不通古今,至於積功治人,亡益於民,此伐檀所為作也。宜明選求賢,除任子之令。外家及故人可厚以財,不宜居位。去角抵,減樂府,省尚方,明視天下以儉。古者工不造琱瑑,商不通侈靡,非工商之獨賢,政教使之然也。民見儉則歸本,本立而末成。」其指如此,上以其言迂闊,不甚寵異也。吉遂謝病歸琅邪。

始吉少時學問,居長安。東家有大棗樹垂吉庭中,吉婦取棗以啖吉。吉後知之,乃去婦。東家聞而欲伐其樹,鄰里共止之,因固請吉令還婦。里中為之語曰:「東家有樹,王陽婦去;東家棗完,去婦復還。」其厲志如此。

吉與貢禹為友,世稱「王陽在位,貢公彈冠」,言其取舍同也。元帝初即位,遣使者徵貢禹與吉。吉年老,道病卒,上悼之,復遣使者弔祠云。

初,吉兼通五經,能為騶氏春秋,以詩、論語教授,好梁丘賀說易,令子駿受焉。駿以孝廉為郎。左曹陳咸薦駿賢父子,經明行修,宜顯以厲俗。光祿勳匡衡亦舉駿有專對材。遷諫大夫,使責淮陽憲王。遷趙內史。吉坐昌邑王被刑後,戒子孫毋為王國吏,故駿道病,免官歸。起家復為幽州刺史,遷司隸校尉,奏免丞相匡衡,遷少府。八歲,成帝欲大用之,出駿為京兆尹,試以政事。先是京兆有趙廣漢、張敞、王尊、王章,至駿皆有能名,故京師稱曰:「前有趙、張,後有三王。」而薛宣從左馮翊代駿為少府,會御史大夫缺,谷永奏言:「聖王不以名譽加於實效。考績用人之法,薛宣政事已試。」上然其議。宣為少府月餘,遂超御史大夫,至丞相。駿乃代宣為御史大夫,並居位。六歲病卒,翟方進代駿為大夫。數月,薛宣免,遂代為丞相。眾人為駿恨不得封侯。駿為少府時,妻死,因不復娶,或問之,駿曰:「德非曾參,子非華、元,亦何敢娶?」

駿子崇以父任為郎,歷刺史、郡守,治有能名。建平三年,以河南太守徵入為御史大夫數月。是時成帝舅安成恭侯夫人放寡居,共養長信宮,坐祝詛下獄,崇奏封事,為放言。放外家解氏與崇為昏,哀帝以崇為不忠誠,策詔崇曰:「朕以君有累世之美,故踰列次。在位以來,忠誠匡國未聞所繇,反懷詐諼之辭,欲以攀救舊姻之家,大逆之辜,舉錯專恣,不遵法度,亡以示百僚。」左遷為大司農,後徙衛尉左將軍。平帝即位,王莽秉政,大司空彭宣乞骸骨罷,崇代為大司空,封扶平侯。歲餘,崇復謝病乞骸骨,皆避王莽,莽遣就國。歲餘,為傅婢所毒,薨,國除。

自吉至崇,世名清廉,然材器名稱稍不能及父,而祿位彌隆。皆好車馬衣服,其自奉養極為鮮明,而亡金銀錦繡之物。及遷徙去處,所載不過囊衣,不畜積餘財。去位家居,亦布衣疏食。天下服其廉而怪其奢,故俗傳「王陽能作黃金」。

貢禹字少翁,琅邪人也。以明經絜行著聞,徵為博士,涼州刺史,病去官。復舉賢良為河南令。歲餘,以職事為府官所責,免冠謝。禹曰:「冠壹免,安復可冠也!」遂去官。

元帝初即位,徵禹為諫大夫,數虛己問以政事。是時年歲不登,郡國多困,禹奏言:

古者宮室有制,宮女不過九人,秣馬不過八匹;牆塗而不琱,木摩而不刻,車輿器物皆不文畫,苑囿不過數十里,與民共之;任賢使能,什一而稅,亡它賦斂繇戍之役,使民歲不過三日,千里之內自給,千里之外各置貢職而已。故天下家給人足,頌聲並作。

至高祖、孝文、孝景皇帝,循古節儉,宮女不過十餘,廄馬百餘匹。孝文皇帝衣綈履革,器亡琱文金銀之飾。後世爭為奢侈,轉轉益盛,臣下亦相放效,衣服履恊刀劍亂於主上,主上時臨朝入廟,眾人不能別異,甚非其宜。然非自知奢僭也,猶魯昭公曰:「吾何僭矣?」

今大夫僭諸侯,諸侯僭天子,天子過天道,其日久矣。承衰救亂,矯復古化,在於陛下。臣愚以為盡如太古難,宜少放古以自節焉。論語曰:「君子樂節禮樂。」方今宮室已定,亡可奈何矣,其餘盡可減損。故時齊三服官輸物不過十笥,方今齊三服官作工各數千人,一歲費數鉅萬。蜀廣漢主金銀器,歲各用五百萬。三工官官費五千萬,東西織室亦然。廄馬食粟將萬匹。臣禹嘗從之東宮,見賜杯案,盡文畫金銀飾,非當所以賜食臣下也。東宮之費亦不可勝計。天下之民所為大飢餓死者,是也。今民大飢而死,死又不葬,為犬豬所食。人至相食,而廄馬食粟,苦其大肥,氣盛怒至,乃日步作之。王者受命於天,為民父母,固當若此乎!天不見邪?武帝時,又多取好女至數千人,以填後宮。及棄天下,昭帝幼弱,霍光專事,不知禮正,妄多臧金錢財物,鳥獸魚鱉牛馬虎豹生禽,凡百九十物,盡瘞臧之,又皆以後宮女置於園陵,大失禮,逆天心,又未必稱武帝意也。昭帝晏駕,光復行之。至孝宣皇帝時,陛下烏有所言,群臣亦隨故事,甚可痛也!故使天下承化,取女皆大過度,諸侯妻妾或至數百人,豪富吏民畜歌者至數十人,是以內多怨女,外多曠夫。及眾庶葬埋,皆虛地上以實地下。其過自上生,皆在大臣循故事之罪也。

唯陛下深察古道,從其儉者,大減損乘輿服御器物,三分去二。子產多少有命,審察後宮,擇其賢者留二十人,餘悉歸之。及諸陵園女亡子者,宜悉遣。獨杜陵宮人數百,誠可哀憐也。廄馬可亡過數十匹。獨舍長安城南苑地以為田獵之囿,自城西南至山西至鄠皆復其田,以與貧民。方今天下飢饉,可亡大自損減以救之,稱天意乎?天生聖人,蓋為萬民,非獨使自娛樂而已也。故《詩》曰:「天難諶斯,不易惟王;」「上帝臨女,毋貳爾心。」「當仁不讓」,獨可以聖心參諸天地,揆之往古,不可與臣下議也。若其阿意順指,隨君上下,臣禹不勝拳拳,不敢不盡愚心。

天子納善其忠,乃下詔令太僕減食穀馬,水衡減食肉獸,省宜春下苑以與貧民,又罷角抵諸戲及齊三服官。遷禹為光祿大夫。

頃之,禹上書曰:「臣禹年老貧窮,家訾不滿萬錢,妻子㐄豆不贍,裋褐不完。有田百三十畝,陛下過意徵臣,臣賣田百畝以供車馬。至,拜為諫大夫,秩八百石,奉錢月九千二百。廩食太官,又蒙賞賜四時雜繒綿絮衣服酒肉諸果物,德厚甚深。疾病侍醫臨治,賴陛下神靈,不死而活。又拜為光祿大夫,秩二千石,奉錢月萬二千。祿賜愈多,家日以益富,身日以益尊,誠非屮茅愚臣所當蒙也。伏自念終亡以報厚恩,日夜慚愧而已。臣禹犬馬之齒八十一,血氣衰竭,耳目不聰明,非復能有補益,所謂素餐尸祿洿朝之臣也。自痛去家三千里,凡有一子,年十二,非有在家為臣具棺槨者也。誠恐一旦蹎仆氣竭,不復自還,洿席薦於宮室,骸骨棄捐,孤魂不歸。不勝私願,願乞骸骨,及身生歸鄉里,死亡所恨。」

天子報曰:「朕以生有伯夷之廉,史魚之直,守經據古,不阿當世,孳孳於民,俗之所寡,故親近生,幾參國政。今未得久聞生之奇論也,而云欲退,意豈有所恨與?將在位者與生殊乎?往者嘗令金敞語生,欲及生時祿生之子,既已諭矣,今復云子少。夫以王命辨護生家,雖百子何以加?傳曰亡懷土,何必思故鄉!生其強飯慎疾以自輔。」後月餘,以禹為長信少府。會御史大夫陳萬年卒,禹代為御史大夫,列於三公。

自禹在位,數言得失,書數十上。禹以為古民亡賦算口錢,起武帝征伐四夷,重賦於民,民產子三歲則出口錢,故民重困,至於生子輒殺,甚可悲痛。宜令兒七歲去齒乃出口錢,年二十乃算。

又言古者不以金錢為幣,專意於農,故一夫不耕,必有受其飢者。今漢家鑄錢,及諸鐵官皆置吏卒徒,攻山取銅鐵,一歲功十萬人已上,中農食七人,是七十萬人常受其飢也。鑿地數百丈,銷陰氣之精,地臧空虛,不能含氣出雲,斬伐林木亡有時禁,水旱之災未必不繇此也。自五銖錢起已來七十餘年,民坐盜鑄錢被刑者眾,富人積錢滿室,猶亡厭足。民心搖動,商賈求利,東西南北各用智巧,好衣美食,歲有十二之利,而不出租稅。農夫父子暴露中野,不避寒暑,捽屮杷土,手足胼胝,已奉穀租,又出槁稅,鄉部私求,不可勝供。故民棄本逐末,耕者不能半。貧民雖賜之田,猶賤賣以賈,窮則起為盜賊。何者?末利深而惑於錢也。是以姦邪不可禁,其原皆起於錢也。疾其末者絕其本,宜罷採珠玉金銀鑄錢之官,亡復以為幣。市井勿得販賣,除其租銖之律,租稅祿賜皆以布帛及穀。使百姓壹歸於農,復古道便。

又言諸離宮及長樂宮衛可減其太半,以寬繇役。又諸官奴婢十萬餘人戲遊亡事,稅良民以給之,歲費五六鉅萬,宜免為庶人,廩食,令代關東戍卒,乘北邊亭塞候望。

又欲令近臣自諸曹侍中以上,家亡得私販賣,與民爭利,犯者輒免官削爵,不得仕宦。禹又言:

孝文皇帝時,貴廉絜,賤貪汙,賈人贅婿及吏坐贓者皆禁錮不得為吏,賞善罰惡,不阿親戚,罪白者伏其誅,疑者以與民,亡贖罪之法,故令行禁止,海內大化,天下斷獄四百,與刑錯亡異。武帝始臨天下,尊賢用士,闢地廣境數千里,自見功大威行,遂從耆欲,用度不足,乃行壹切之變,使犯法者贖罪,入穀者補吏,是以天下奢侈,官亂民貧,盜賊並起,亡命者眾。郡國恐伏其誅,則擇便巧史書習於計簿能欺上府者,以為右職;姦軌不勝,則取勇猛能操切百姓者,以苛暴威服下者,使居大位。故亡義而有財者顯於世,欺謾而善書者尊於朝,誖逆而勇猛者貴於官。故俗皆曰:「何以孝弟為?財多而光榮。何以禮義為?史書而仕宦。何以謹慎為?勇猛而臨官。」故黥劓而髡鉗者猶復攘臂為政於世,行雖犬彘,家富勢足,目指氣使,是為賢耳。故謂居官而置富者為雄桀,處姦而得利者為壯士,兄勸其弟,父勉其子,俗之壞敗,乃至於是!察其所以然者,皆以犯法得贖罪,求士不得真賢,相守崇財利,誅不行之所致也。

今欲興至治,致太平,宜除贖罪之法。相守選舉不以實,及有臧者,輒行其誅,亡但免官,則爭盡力為善,貴孝弟,賤賈人,進真賢,舉實廉,而天下治矣。孔子,匹夫之人耳,以樂道正身不解之故,四海之內,天下之君,微孔子之言亡所折中。況乎以漢地之廣,陛下之德,處南面之尊,秉萬乘之權,因天地之助,其於變世易俗,調和陰陽,陶冶萬物,化正天下,易於決流抑隊。自成康以來,幾且千歲,欲為治者甚眾,然而太平不復興者,何也?以其舍法度而任私意,奢侈行而仁義廢也。

陛下誠深念高祖之苦,醇法太宗之治,正己以先下,選賢以自輔,開進忠正,致誅姦臣,遠放諂佞,放出園陵之女,罷倡樂,絕鄭聲,去甲乙之帳,退偽薄之物,修節儉之化,驅天下之民皆歸於農,如此不解,則三王可侔,五帝可及。唯陛下留意省察,天下幸甚。

天子下其議,令民產子七歲乃出口錢,自此始。又罷上林宮館希幸御者,及省建章、甘泉宮衛卒,減諸侯王廟衛卒省其半。餘雖未盡從,然嘉其質直之意。禹又奏欲罷郡國廟,定漢宗廟迭毀之禮,皆未施行。

為御史大夫數月卒,天子賜錢百萬,以其子為郎,官至東郡都尉。禹卒後,上追思其議,竟下詔罷郡國廟,定迭毀之禮。語在韋玄成傳。

兩龔皆楚人也,勝字君賓,舍字君倩。二人相友,並著名節,故世謂之楚兩龔。少皆好學明經,勝為郡吏,舍不仕。

久之,楚王入朝,聞舍高明,聘舍為常侍,不得已隨王,歸國固辭,願卒學,復至長安。而勝為郡吏,三舉孝廉,以王國人不得宿衛補吏。再為尉,壹為丞,勝輒至官乃去。州舉茂材,為重泉令,病去官。大司空何武、執金吾閻崇薦勝,哀帝自為定陶王固已聞其名,徵為諫大夫。引見,勝薦龔舍及亢父甯壽、濟陰侯嘉,有詔皆徵。勝曰:「竊見國家徵醫巫,常為駕,徵賢者宜駕。」上曰:「大夫乘私車來邪?」勝曰:「唯唯。」有詔為駕。龔舍、侯嘉至,皆為諫大夫。甯壽稱疾不至。

勝居諫官,數上書求見,言百姓貧,盜賊多,吏不良,風俗薄,災異數見,不可不憂。制度泰奢,刑罰泰深,賦斂泰重,宜以儉約先下。其言祖述王吉、貢禹之意。為大夫二歲餘,遷丞相司直,徙光祿大夫,守右扶風。數月,上知勝非撥煩吏,乃復還勝光祿大夫諸吏給事中。勝言董賢亂制度,繇是逆上指。

後歲餘,丞相王嘉上書薦故廷尉梁相等,尚書劾奏嘉「言事恣意,迷國罔上,不道。」下將軍中朝者議,左將軍公孫祿、司隸鮑宣、光祿大夫孔光等十四人皆以為嘉應迷國不道法。勝獨書議曰:「嘉資性邪僻,所舉多貪殘吏。位列三公,陰陽不和,諸事並廢,咎皆繇嘉,迷國不疑,今舉相等,過微薄。」日暮議者罷。明旦復會,左將軍祿問勝:「君議亡所據,今奏當上,宜何從?」勝曰:「將軍以勝議不可者,通劾之。」博士夏侯常見勝應祿不和,起至勝前謂曰:「宜如奏所言。」勝以手推常曰:「去!」

後數日,復會議可復孝惠、孝景廟不,議者皆曰宜復。勝曰:「

當如禮。」常復謂勝:「禮有變。」勝疾言曰:「去!是時之變。」常恚,謂勝曰:「我視君何若,君欲小與眾異,外以采名,君乃申徒狄屬耳!」

先是常又為勝道高陵有子殺母者。勝白之,尚書問:「誰受?」對曰:「受夏侯常。」尚書使勝問常,常連恨勝,即應曰:「聞之白衣,戒君勿言也。奏事不詳,妄作觸罪。」勝窮,亡以對尚書,即自劾奏與常爭言,洿辱朝廷。事下御史中丞,召詰問,劾奏「勝吏二千石,常位大夫,皆幸得給事中,與論議,不崇禮義,而居公門下相非恨,疾言辯訟,惰謾亡狀,皆不敬。」制曰:「貶秩各一等。」勝謝罪,乞骸骨。上乃復加賞賜,以子博為侍郎,出勝為渤海太守。勝謝病不任之官,積六月免歸。

上復徵為光祿大夫。勝常稱疾臥。數使子上書乞骸骨,會哀帝崩。

初,琅邪邴漢亦以清行徵用,至京兆尹,後為太中大夫。王莽秉政,勝與漢俱乞骸骨。自昭帝時,涿郡韓福以德行徵至京師,賜策書束帛遣歸。詔曰:「朕閔勞以官職之事,其務修孝弟以教鄉里。行道舍傳舍,縣次具酒肉,食從者及馬。長吏以時存問,常以歲八月賜羊一頭,酒二斛。不幸死者,賜複衾一,祠以中牢。」於是王莽依故事,白遣勝、漢。策曰:「惟元始二年六月庚寅,光祿大夫、太中大夫耆艾二人以老病罷。太皇太后使謁者僕射策詔之曰:蓋聞古者有司年至則致仕,所以恭讓而不盡其力也。今大夫年至矣,朕愍以官職之事煩大夫,其上子若孫若同產、同產子一人。大夫其修身守道,以終高年。賜帛及行道舍宿,歲時羊酒衣衾,皆如韓福故事。所上子男皆除為郎。」於是勝、漢遂歸老于鄉里。漢兄子曼容亦養志自修,為官不肯過六百石,輒自免去,其名過出於漢。

初,龔舍以龔勝薦,徵為諫大夫,病免。復徵為博士,又病去。頃之,哀帝遣使者即楚拜舍為太山太守。舍家居在武原,使者至縣請舍,欲令至廷拜授印綬。舍曰:「王者以天下為家,何必縣官?」遂於家受詔,便道之官。既至數月,上書乞骸骨。上徵舍,至京兆東湖界,固稱病篤。天子使使者收印綬,拜舍為光祿大夫。數賜告,舍終不肯起,乃遣歸。

舍亦通五經,以魯詩教授。舍、勝既歸鄉里,郡二千石長吏初到官皆至其家,如師弟子之禮。舍年六十八,王莽居攝中卒。

莽既篡國,遣五威將帥行天下風俗,將帥親奉羊酒存問勝。明年,莽遣使者即拜勝為講學祭酒,勝稱疾不應徵。後二年,莽復遣使者奉璽書,太子師友祭酒印綬,安車駟馬迎勝,即拜,秩上卿,先賜六月祿直以辦裝,使者與郡太守、縣長吏、三老官屬、行義諸生千人以上入勝里致詔。使者欲令勝起迎,久立門外。勝稱病篤,為床室中戶西南牖下,東首加朝服癴紳。使者入戶,西行南面立,致詔付璽書,遷延再拜奉印綬,內安車駟馬,進謂勝曰:「聖朝未嘗忘君,制作未定,待君為政,思聞所欲施行,以安海內。」勝對曰:「素愚,加以年老被病,命在朝夕,隨使君上道,必死道路,無益萬分。」使者要說,至以印綬就加勝身,勝輒推不受。使者即上言:「方盛夏暑熱,勝病少氣,可須秋涼乃發。」有詔許。使者五日壹與太守俱問起居,為勝兩子及門人高暉等言:「朝廷虛心待君以茅土之封,雖疾病,宜動移至傳舍,示有行意,必為子孫遺大業。」暉等白使者語,勝自知不見聽,即謂暉等:「吾受漢家厚恩,亡以報,今年老矣,旦暮入地,誼豈以一身事二姓,下見故主哉?」勝因敕以棺斂喪事:「衣周於身,棺周於衣。勿隨俗動吾冢,種柏,作祠堂。」語畢,遂不復開口飲食,積十四日死,死時七十九矣。使者、太守臨斂,賜複衾祭祠如法。門人衰絰治喪者百數。有老父來弔,哭甚哀,既而曰:「嗟虖!薰以香自燒,膏以明自銷。龔生竟夭天年,非吾徒也。」遂趨而出,莫知其誰。勝居彭城廉里,後世刻石表其里門。

鮑宣字子都,渤海高城人也。好學明經,為縣鄉嗇夫,守束州丞。後為都尉太守功曹,舉孝廉為郎,病去官,復為州從事。大司馬衛將軍王商辟宣,薦為議郎,後以病去。哀帝初,大司空何武除宣為西曹掾,甚敬重焉,薦宣為諫大夫,遷豫州牧。歲餘,丞相司直郭欽奏「宣舉錯煩苛,代二千石署吏聽訟,所察過詔條。行部乘傳去法駕,駕一馬,舍宿鄉亭,為眾所非。」宣坐免。歸家數月,復徵為諫大夫。

宣每居位,常上書諫爭,其言少文多實。是時帝祖母傅太后欲與成帝母俱稱尊號,封爵親屬,丞相孔光、大司空師丹、何武、大司馬傅喜始執正議,失傅太后指,皆免官。丁、傅子弟並進,董賢貴幸,宣以諫大夫從其後,上書諫曰:

竊見孝成皇帝時,外親持權,人人牽引所私以充塞朝廷,妨賢人路,濁亂天下,奢泰亡度,窮困百姓,是以日蝕且十,彗星四起。危亡之徵,陛下所親見也,今柰何反覆劇於前乎!朝臣亡有大儒骨鯁,白首耆艾,魁壘之士;論議通古今,喟然動眾心,憂國如飢渴者,臣未見也。敦外親小童及幸臣董賢等在公門省戶下,陛下欲與此共承天地,安海內,甚難。今世俗謂不智者為能,謂智者為不能。昔堯放四罪而天下服,今除一吏而眾皆惑;古刑人尚服,今賞人反惑。請寄為姦,群小日進。國家空虛,用度不足。民流亡,去城郭,盜賊並起,吏為殘賊,歲增於前。

凡民有七亡:陰陽不和,水旱為災,一亡也;縣官重責更賦租稅,二亡也;貪吏並公,受取不已,三亡也;豪強大姓蠶食亡厭,四亡也;苛吏繇役,失農桑時,五亡也;部落鼓鳴,男女遮迣,六亡也;盜賊劫略,取民財物,七亡也。七亡尚可,又有七死:酷吏毆殺,一死也;治獄深刻,二死也;冤陷亡辜,三死也;盜賊橫發,四死也;怨讎相殘,五死也;歲惡飢餓,六死也;時氣疾疫,七死也。民有七亡而無一得,欲望國安,誠難;民有七死而無一生,欲望刑措,誠難。此非公卿守相貪殘成化之所致邪?群臣幸得居尊官,食重祿,豈有肯加惻隱於細民,助陛下流教化者邪?志但在營私家,稱賓客,為姦利而已。以苟容曲從為賢,以拱默尸祿為智,謂如臣宣等為愚。陛下擢臣巖穴,誠冀有益豪毛,豈徒欲使臣美食大官,重高門之地哉!

天下乃皇天之天下也,陛下上為皇天子,下為黎庶父母,為天牧養元元,視之當如一,合尸鳩之詩。今貧民菜食不厭,衣又穿空,父子夫婦不能相保,誠可為酸鼻。陛下不救,將安所歸命乎?奈何獨私養外親與幸臣董賢,多賞賜以大萬數,使奴從賓客漿酒霍肉,蒼頭廬兒皆用致富!非天意也。及汝昌侯傅商亡功而封。夫官爵非陛下之官爵,乃天下之官爵也。陛下取非其官,官非其人,而望天說民服,豈不難哉!

方陽侯孫寵、宜陵侯息夫躬辯足以移眾,彊可用獨立,姦人之雄,或世尤劇者也,宜以時罷退。及外親幼童未通經術者,皆宜令休就師傅。急徵故大司馬傅喜使領外親。故大司空何武、師丹、故丞相孔光、故左將軍彭宣,經皆更博士,位皆歷三公,智謀威信,可與建教化,圖安危。龔勝為司直,郡國皆慎選舉,三輔委輸官不敢為姦,可大委任也。陛下前以小不忍退武等,海內失望。陛下尚能容亡功德者甚眾,曾不能忍武等邪!治天下者當用天下之心為心,不得自專快意而已也。上之皇天見譴,下之黎庶怨恨,次有諫爭之臣,陛下苟欲自薄而厚惡臣,天下猶不聽也。臣雖愚戇,獨不知多受祿賜,美食太官,廣田宅,厚妻子,不與惡人結讎怨以安身邪?誠迫大義,官以諫爭為職,不敢不竭愚。惟陛下少留神明,覽五經之文,原聖人之至意,深思天地之戒。臣宣吶鈍於辭,不勝惓惓,盡死節而已。

上以宣名儒,優容之。

是時郡國地震,民訛言行籌,明年正月朔日蝕,上乃徵孔光,免孫寵、息夫躬,罷侍中諸曹黃門郎數十人。宣復上書言:

陛下父事天,母事地,子養黎民,即位已來,父虧明,母震動,子訛言相驚恐。今日蝕於三始,誠可畏懼。小民正月朔日尚恐毀敗器物,何況於日虧乎!陛下深內自責,避正殿,舉直言,求過失,罷退外親及旁仄素餐之人,徵拜孔光為光祿大夫,發覺孫寵、息夫躬過惡,免官遣就國,眾庶歙然,莫不說喜。天人同心,人心說則天意解矣。乃二月丙戌,白虹虷日,連陰不雨,此天有憂結未解,民有怨望未塞者也。

侍中駙馬都尉董賢本無葭莩之親,但以令色諛言自進,賞賜亡度,竭盡府藏,并合三第尚以為小,復壞暴室。賢父子坐使天子使者將作治第,行夜吏卒皆得賞賜。上冢有會,輒太官為供。海內貢獻當養一君,今反盡之賢家,豈天意與民意邪!天下可久負,厚之如此,反所以害之也。誠欲哀賢,宜為謝過天地,解讎海內,免遣就國,收乘輿器物,還之縣官。如此,可以父子終其性命;不者,海內之所讎,未有得久安者也。

孫寵、息夫躬不宜居國,可皆免以視天下。復徵何武、師丹、彭宣、傅喜,曠然使民易視,以應天心,建立大政,以興太平之端。

高門去省戶數十步,求見出入,二年未省,欲使海瀕仄陋自通,遠矣!願賜數刻之間,極竭毣毣之思,退入三泉,死亡所恨。

上感大異,納宣言,徵何武、彭宣,旬月皆復為三公。拜宣為司隸。時哀帝改司隸校尉但為司隸,官比司直。

丞相孔光四時行園陵,官屬以令行馳道中,宣出逢之,使吏鉤止丞相掾史,沒入其車馬,摧辱宰相。事下御史,中丞侍御史至司隸官,欲捕從事,閉門不肯內。宣坐距閉使者,亡人臣禮,大不敬,不道,下廷尉獄。博士弟子濟南王咸舉幡太學下,曰:「欲救鮑司隸者會此下。」諸生會者千餘人。朝日,遮丞相孔光自言,丞相車不得行,又守闕上書。上遂抵宣罪減死一等,髡鉗。宣既被刑,乃徙之上黨,以為其地宜田牧,又少豪俊,易長雄,遂家于長子。

平帝即位,王莽秉政,陰有篡國之心,乃風州郡以罪法案誅諸豪桀,及漢忠直臣不附己者,宣及何武等皆死。時名捕隴西辛興,興與宣女婿許紺俱過宣,一飯去,宣不知情,坐繫獄,自殺。

自成帝至王莽時,清名之士,琅邪又有紀逡王思,齊則薛方子容,太原則郇越臣仲、郇相稚賓,沛郡則唐林子高、唐尊伯高,皆以明經飭行顯名於世。

紀逡、兩唐皆仕王莽,封侯貴重,歷公卿位。唐林數上疏諫正,有忠直節。唐尊衣敝履空,以瓦器飲食,又以歷遺公卿,被虛偽名。

郇越、相,同族昆弟也,並舉州郡孝廉茂材,數病,去官。越散其先人訾千餘萬,以分施九族州里,志節尤高。相王莽時徵為太子四友,病死,莽太子遣使裞以衣衾,其子攀棺不聽,曰:「死父遺言,師友之送勿有所受,今於皇太子得託友官,故不受也。」京師稱之。

薛方嘗為郡掾祭酒,嘗徵不至,及莽以安車迎方,方因使者辭謝曰:「堯舜在上,下有巢由,今明主方隆唐虞之德,小臣欲守箕山之節也。」使者以聞,莽說其言,不強致。方居家以經教授,喜屬文,著詩賦數十篇。

始隃麋郭欽,哀帝時為丞相司直,奏免豫州牧鮑宣、京兆尹薛修等,又奏董賢,左遷盧奴令,平帝時遷南郡太守。而杜陵蔣詡元卿為兗州刺史,亦以廉直為名。王莽居攝,欽、詡皆以病免官,歸鄉里,臥不出戶,卒於家。

齊栗融客卿、北海禽慶子夏、蘇章游卿、山陽曹竟子期皆儒生,去官不仕於莽。莽死,漢更始徵竟以為丞相,封侯,欲視致賢人,銷寇賊。竟不受侯爵。會赤眉入長安,欲降竟,竟手劍格死。

世祖即位,徵薛方,道病卒。兩龔、鮑宣子孫皆見褒表,至大官。

贊曰:易稱「君子之道,或出或處,或默或語」,言其各得道之一節,譬諸草木,區以別矣。故曰山林之士往而不能反,朝廷之士入而不能出,二者各有所短。春秋列國卿大夫至漢興將相名臣,懷祿耽寵以失其世者多矣!是故清節之士於是為貴。然大率多能自治而不能治人。王、貢之材,優於龔、鮑。守死善道,勝實蹈焉。貞而不諒,薛方近之。郭欽、蔣詡好遯不汙,絕紀、唐矣!


\end{pinyinscope}