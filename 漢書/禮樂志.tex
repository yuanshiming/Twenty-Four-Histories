\article{禮樂志}

\begin{pinyinscope}
六經之道同歸,而禮樂之用為急。治身者斯須忘禮,則暴嫚入之矣;為國者一朝失禮,則荒亂及之矣。人函天地陰陽之氣,有喜怒哀樂之情。天稟其性而不能節也,聖人能為之節而不能絕也,故象天地而制禮樂,所以通神明,立人倫,正情性,節萬事者也。

人性有男女之情,妒忌之別,為制婚姻之禮;有交接長幼之序,為制鄉飲之禮;有哀死思遠之情,為制喪祭之禮;有尊尊敬上之心,為制朝覲之禮。哀有哭踊之節,樂有歌舞之容,正人足以副其誠,邪人足以防其失。故婚姻之禮廢,則夫婦之道苦,而淫辟之罪多;鄉飲之禮廢,則長幼之序亂,而爭鬥之獄蕃;喪祭之禮廢,則骨肉之恩薄,而背死忘先者眾;朝聘之禮廢,則君臣之位失,而侵陵之漸起。故孔子曰:「安上治民,莫殒於禮;移風易俗,莫殒於樂。」禮節民心,樂和民聲,政以行之,刑以防之。禮樂政刑四達而不誖,則王道備矣。

樂以治內而為同,禮以修外而為異;同則和親,異則畏敬;和親則無怨,畏敬則不爭。揖讓而天下治者,禮樂之謂也。二者並行,合為一體。畏敬之意難見,則著之於享獻辭受,登降跪拜;和親之說難形,則發之於詩歌詠言,鐘石筦弦。蓋嘉其敬意而不及其財賄,美其歡心而不流其聲音。故孔子曰:「

禮云禮云,玉帛云乎哉?樂云樂云,鐘鼓云乎哉?」此禮樂之本也。故曰:「知禮樂之情者能作,識禮樂之文者能述;作者之謂聖,述者之謂明。明聖者,述作之謂也。」

王者必因前王之禮,順時施宜,有所損益,即民之心,稍稍制作,至太平而大備。周監於二代,禮文尤具,事為之制,曲為之防,故稱禮經三百,威儀三千。於是教化浹洽,民用和睦,災害不生,禍亂不作,囹圄空虛,四十餘年。孔子美之曰:「郁郁乎文哉!吾從周。」及其衰也,諸侯踰越法度,惡禮制之害己,去其篇籍。遭秦滅學,遂以亂亡。

漢興,撥亂反正,日不暇給,猶命叔孫通制禮儀,以正君臣之位。高祖說而歎曰:「吾乃今日知為天子之貴也!」以通為奉常,遂定儀法,未盡備而通終。

至文帝時,賈誼以為「漢承秦之敗俗,廢禮義,捐廉恥,今其甚者殺父兄,盜者取廟器,而大臣特以簿書不報期會為故,至於風俗流溢,恬而不怪,以為是適然耳。夫移風易俗,使天下回心而鄉道,類非俗吏之所能為也。夫立君臣,等上下,使綱紀有序,六親和睦,此非天之所為,人之所設也。人之所設,不為不立,不修則壞。漢興至今二十餘年,宜定制度,興禮樂,然後諸侯軌道,百姓素樸,獄訟衰息」。乃草具其儀,天子說焉。而大臣絳、灌之屬害之,故其議遂寢。

至武帝即位,進用英雋,議立明堂,制禮服,以興太平。會竇太后好黃老言,不說儒術,其事又廢。後董仲舒對策言:「王者欲有所為,宜求其端於天。天道大者,在於陰陽。陽為德,陰為刑。天使陽常居大夏而以生育長養為事,陰常居大冬而積於空虛不用之處,以此見天之任德不任刑也。陽出布施於上而主歲功,陰入伏藏於下而時出佐陽。陽不得陰之助,亦不能獨成歲功。王者承天意以從事,故務德教而省刑罰。刑罰不可任以治世,猶陰之不可任以成歲也。今廢先王之德教,獨用執法之吏治民,而欲德化被四海,故難成也。是故古之王者莫不以教化為大務,立大學以教於國,設庠序以化於邑。教化已明,習俗已成,天下嘗無一人之獄矣。至周末世,大為無道,以失天下。秦繼其後,又益甚之。自古以來,未嘗以亂濟亂,大敗天下如秦者也。習俗薄惡,民人抵冒。今漢繼秦之後,雖欲治之,無可柰何。法出而姦生,令下而詐起,一歲之獄以萬千數,如以湯止沸,沸俞甚而無益。辟之琴瑟不調,甚者必解而更張之,乃可鼓也。為政而不行,甚者必變而更化之,乃可理也。故漢得天下以來,常欲善治,而至今不能勝殘去殺者,失之當更化而不能更化也。古人有言:『臨淵羨魚,不如歸而結網。』今臨政而願治七十餘歲矣,不如退而更化。更化則可善治,而災害日去,福祿日來矣。」是時,上方征討四夷,銳志武功,不暇留意禮文之事。

至宣帝時,琅邪王吉為諫大夫,又上疏言:「欲治之主不世出,公卿幸得遭遇其時,未有建萬世之長策,舉明主於三代之隆者也。其務在於簿書斷獄聽訟而已,此非太平之基也。今俗吏所以牧民者,非有禮義科指可世世通行者也,以意穿鑿,各取一切。是以詐偽萌生,刑罰無極,質樸日消,恩愛寖薄。孔子曰『安上治民,莫善於禮』,非空言也。願與大臣延及儒生,述舊禮,明王制,驅一世之民,濟之仁壽之域,則俗何以不若成康?壽何以不若高宗?」上不納其言,吉以病去。

至成帝時,犍為郡於水濱得古磬十六枚,議者以為殒祥。劉向因是說上:「宜興辟雍,設庠序,陳禮樂,隆雅頌之聲,盛揖攘之容,以風化天下。如此而不治者,未之有也。或曰,不能具禮。禮以養人為本,如有過差,是過而養人也。刑罰之過,或至死傷。今之刑,非皋陶之法也,而有司請定法,削則削,筆則筆,救時務也。至於禮樂,則曰不敢,是敢於殺人不敢於養人也。為其俎豆筦弦之間小不備,因是絕而不為,是去小不備而就大不備,大不備或莫甚焉。夫教化之比於刑法,刑法輕,是舍所重而急所輕也。且教化,所恃以為治也,刑法所以助治也。今廢所恃而獨立其所助,非所以致太平也。自京師有誖逆不順之子孫,至於陷大辟受刑戮者不絕,繇不習五常之道也。夫承千歲之衰周,繼暴秦之餘敝,民漸漬惡俗,貪饕險詖,不閑義理,不示以大化,而獨敺以刑罰,終已不改。故曰:『

導之以禮樂,而民和睦。』初,叔孫通將制定禮儀,見非於齊魯之士,然卒為漢儒宗,業垂後嗣,斯成法也。」成帝以向言下公卿議,會向病卒,丞相大司空奏請立辟雍。案行長安城南,營表未作,遭成帝崩,群臣引以定諡。

及王莽為宰衡,欲燿眾庶,遂興辟廱,因以篡位,海內畔之。世祖受命中興,撥亂反正,改定京師于土中。即位三十年,四夷賓服,百姓家給,政教清明,乃營立明堂、辟廱。顯宗即位,躬行其禮,宗祀光武皇帝于明堂,養三老五更於辟廱,威儀既盛美矣。然德化未流洽者,禮樂未具,群下無所誦說,而庠序尚未設之故也。孔子曰:「辟如為山,未成一匱,止,吾止也。」今叔孫通所撰禮儀,與律令同錄,臧於理官,法家又復不傳。漢典寢而不著,民臣莫有言者。又通沒之後,河間獻王采禮樂古事,稍稍增輯,至五百餘篇。今學者不能昭見,但推士禮以及天子,說義又頗謬異,故君臣長幼交接之道娅以不章。

樂者,聖人之所樂也,而可以善民心。其感人深,其移風易俗易,故先王著其教焉。

夫民有血氣心知之性,而無哀樂喜怒之常,應感而動,然後心術形焉。是以纖微虍瘁一作「衰」之音作,而民思憂;闡諧嫚易之音作,而民康樂;麤厲猛奮之音作,而民剛毅;廉直正誠之音作,而民肅敬;寬裕和順之音作,而民慈愛;流辟邪散之音作,而民淫亂。先王恥其亂也,故制雅頌之聲,本之情性,稽之度數,制之禮儀,合生氣之和,導五常之行,使之陽而不散,陰而不集,剛氣不怒,柔氣不懾,四暢交於中,而發作於外,皆安其位而不相奪也,足以感動人之善心而,不使邪氣得接焉,是先王立樂之方也。

王者未作樂之時,因先王之樂以教化百姓,說樂其俗,然後改作,以章功德。《易》曰:「先王以作樂崇德,殷薦之上帝,以配祖考。」昔黃帝作咸池,顓頊作六莖,帝嚳作五英,堯作大章,舜作招,禹作夏,湯作濩,武王作武,周公作勺。勺,言能勺先祖之道也。武,言以功定天下也。濩,言救民也。夏,大承二帝也。招,繼堯也。大章,章之也。五英,英華茂也。六莖,及根莖也。咸池,備矣。自夏以往,其流不可聞已,殷頌猶有存者。周詩既備,而其器用張陳,周官具焉。典者自卿大夫師瞽以下,皆選有道德之人,朝夕習業,以教國子。國子者,卿大夫之子弟也。皆學歌九德,誦六詩,習六舞、五聲、八音之和。故帝舜命夔曰:「女典樂,教冑子,直而溫,寬而栗,剛而無虐,簡而無敖。詩言志,歌詠言,聲依詠,律和聲,八音克諧。」此之謂也。又以外賞諸侯德盛而教尊者。其威儀足以充目,音聲足以動耳,詩語足以感心,故聞其音而德和,省其詩而志正,論其數而法立。是以薦之郊廟則鬼神饗,作之朝廷則群臣和,立之學官則萬民協。聽者無不虛己竦神,說而承流,是以海內遍知上德,被服其風,光煇日新,化上遷善,而不知所以然,至於萬物不夭,天地順而嘉應降。故《詩》曰:「鐘鼓鍠鍠,磬管鏘鏘,降福穰穰。」《書》云:「擊石拊石,百獸率舞。」鳥獸且猶感應,而況於人乎?況於鬼神乎?故樂者,聖人之所以感天地,通神明,安萬民,成性類者也。然自雅頌之興,而所承衰亂之音猶在,是謂淫過凶嫚之聲,為設禁焉。世衰民散,小人乘君子,心耳淺薄,則邪勝正。故書序「殷紂斷棄先祖之樂,乃作淫聲,用變亂正聲,以說婦人。」樂官師瞽抱其器而奔散,或適諸侯,或入河海。夫樂本情性,浹肌膚而臧骨髓,雖經乎千載,其遺風餘烈尚猶不絕。至春秋時,陳公子完奔齊。陳,舜之後,招樂存焉。故孔子適齊聞招,三月不知肉味,曰「不圖為樂之至於斯!」美之甚也。

周道始缺,怨刺之詩起。王澤既竭,而詩不能作。王官失業,雅頌相錯,孔子論而定之,故曰:「吾自衛反魯,然後樂正,雅頌各得其所。」是時,周室大壞,諸侯恣行,設兩觀,乘大路。陪臣管仲、季氏之屬,三歸雍徹,八佾舞廷。制度遂壞,陵夷而不反,桑間、濮上,鄭、衛、宋、趙之聲並出,內則致疾損壽,外則亂政傷民。巧偽因而飾之,以營亂富貴之耳目。庶人以求利,列國以相間。故秦穆遺戎而由余去,齊人餽魯而孔子行。至於六國,魏文侯最為好古,而謂子夏曰:「寡人聽古樂則欲寐,及聞鄭、衛,余不知倦焉。」子夏辭而辨之,終不見納,自此禮樂喪矣。

漢興,樂家有制氏,以雅樂聲律世世在大樂官,但能紀其鏗鎗鼓舞,而不能言其義。高祖時,叔孫通因秦樂人制宗廟樂。大祝迎神于廟門,奏嘉至,猶古降神之樂也。皇帝入廟門,奏永至,以為行步之節,猶古采薺、肆夏也。乾豆上,奏登歌,獨上歌,不以筦弦亂人聲,欲在位者遍聞之,猶古清廟之歌也。登歌再終,下奏休成之樂,美神明既饗也。皇帝就酒東廂,坐定,奏永安之樂,美禮已成也。又有房中祠樂,高祖唐山夫人所作也。周有房中樂,至秦名曰壽人。凡樂,樂其所生,禮不忘本。高祖樂楚聲,故房中樂楚聲也。孝惠二年,使樂府令夏侯寬備其簫管,更名曰安世樂。

高祖廟奏武德、文始、五行之舞;孝文廟奏昭德、文始、四時、五行之舞;孝武廟奏盛德、文始、四時、五行之舞。武德舞者,高祖四年作,以象天下樂己行武以除亂也。文始舞者,曰本舜招舞也,高祖六年更名曰文始,以示不相襲也。五行舞者,本周舞也,秦始皇二十六年更名曰五行也。四時舞者,孝文所作,以明示天下之安和也。蓋樂己所自作,明有制也;樂先王之樂,明有法也。孝景采武德舞以為昭德,以尊大宗廟。至孝宣,采昭德舞為盛德,以尊世宗廟。諸帝廟皆常奏文始、四時、五行舞云。高祖六年又作昭容樂、禮容樂。昭容者,猶古之昭夏也,主出武德舞。禮容者,主出文始、五行舞。舞人無樂者,將至至尊之前不敢以樂也;出用樂者,言舞不失節,能以樂終也。大氐皆因秦舊事焉。

初,高祖既定天下,過沛,與故人父老相樂,醉酒歡哀,作「風起」之詩,令沛中僮兒百二十人習而歌之。至孝惠時,以沛宮為原廟,皆令歌兒習吹以相和,常以百二十人為員。文、景之間,禮官肄業而已。至武帝定郊祀之禮,祠太一於甘泉,就乾位也;祭后土於汾陰,澤中方丘也。乃立樂府,采詩夜誦,有趙、代、秦、楚之謳。以李延年為協律都尉,多舉司馬相如等數十人造為詩賦,略論律呂,以合八音之調,作十九章之歌。以正月上辛用事甘泉圜丘,使童男女七十人俱歌,昏祠至明。夜常有神光如流星止集于祠壇,天子自竹宮而望拜,百官侍祠者數百人皆肅然動心焉。

安世房中歌十七章,其詩曰:

大孝備矣,休德昭清。高張四縣,樂充宮庭。芬樹羽林,雲景杳冥,金支秀華,庶旄翠旌。

七始華始,肅倡和聲。神來宴娭,庶幾是聽。粥粥音送,細齊人情。忽乘青玄,熙事備成。清思眑眑,經緯冥冥。

我定曆數,人告其心。敕身齊戒,施教申申。乃立祖廟,敬明尊親。大矣孝熙,四極爰轃。

王侯秉德,其鄰翼翼,顯明昭式。清明鬯矣,皇帝孝德。竟全大功,撫安四極。

海內有姦,紛亂東北。詔撫成師,武臣承德。行樂交逆,簫、勺群慝。肅為濟哉,蓋定燕國。

大海蕩蕩水所歸,高賢愉愉民所懷。大山崔,百卉殖。民何貴?貴有德。

安其所,樂終產。樂終產,世繼緒。飛龍秋,游上天。高賢愉,樂民人。

豐草葽,女羅施。殒何如,誰能回!大莫大,成教德;長莫長,被無極。

雷震震,電燿燿。明德鄉,治本約。治本約,澤弘大。加被寵,咸相保。德施大,世曼壽。

都荔遂芳,窅窊桂華。孝奏天儀,若日月光。乘玄四龍,回馳北行。羽旄殷盛,芬哉芒芒。孝道隨世,我署文章。桂華。

馮馮翼翼,承天之則。吾易久遠,燭明四極。慈惠所愛,美若休德。杳杳冥冥,克綽永福。美芳。

磑磑即即,師象山則。烏呼孝哉,案撫戎國。蠻夷竭歡,象來致福。兼臨是愛,終無兵革。

嘉薦芳矣,告靈饗矣。告靈既饗,德音孔臧。惟德之臧,建侯之常。承保天休,令問不忘。

皇皇鴻明,蕩侯休德。嘉承天和,伊樂厥福。在樂不荒,惟民之則。

浚則師德,下民咸殖。令問在舊,孔容翼翼。

孔容之常,承帝之明。下民之樂,子孫保光。承順溫良,受帝之光。嘉薦令芳,壽考不忘。

承帝明德,師象山則。雲施稱民,永受厥福。承容之常,承帝之明。下民安樂,受福無疆。

郊祀歌十九章,其詩曰:

練時日,侯有望,巽膋蕭,延四方。九重開,靈之斿,垂惠恩,鴻祜休。靈之車,結玄雲,駕飛龍,羽旄紛。靈之下,若風馬,左倉龍,右白虎。靈之來,神哉沛,先以雨,般裔裔。靈之至,慶陰陰,相放怫,震澹心。靈已坐,五音飭,虞至旦,承靈億。牲繭栗,粢盛香,尊桂酒,賓八鄉。靈安留,吟青黃,遍觀此,眺瑤堂。眾嫭並,綽奇麗,顏如荼,兆逐靡。被華文,廁霧縠,曳阿錫,佩珠玉。俠嘉夜,仂蘭芳,澹容與,獻嘉觴。練時日一

帝臨中壇,四方承宇,繩繩意變,備得其所。清和六合,制數以五。海內安寧,興文匽武。后土富媼,昭明三光。穆穆優游,嘉服上黃。帝臨二

青陽開動,根荄以遂,膏潤并愛,跂行畢逮。霆聲發榮,壧處頃聽,枯槁復產,乃成厥命。眾庶熙熙,施及夭胎,群生啿啿,惟春之祺。

青陽三鄒子樂。

朱明盛長,归與萬物,桐生茂豫,靡有所詘。敷華就實,既阜既昌,登成甫田,百鬼迪嘗。廣大建祀,肅雍不忘,神若宥之,傳世無疆。

朱明四鄒子樂。

西顥沆碭,秋氣肅殺,含秀垂穎,續舊不廢。姦偽不萌,祅孽伏息,隅辟越遠,四貉咸服。既畏茲威,惟慕純德,附而不驕,正心翊翊。

西顥五鄒子樂。

玄冥陵陰,蟄蟲蓋臧,屮木零落,抵冬降霜。易亂除邪,革正異俗,兆民反本,抱素懷樸。條理信義,望禮五嶽。籍斂之時,掩收嘉穀。

玄冥六鄒子樂。

惟泰元尊,媼神蕃釐,經緯天地,作成四時。精建日月,星辰度理,陰陽五行,周而復始。雲風雷電,降甘露雨,百姓蕃滋,咸循厥緒。繼統共勤,順皇之德,鸞路龍鱗,罔不肸飾。嘉籩列陳,庶幾宴享,滅除凶災,列騰八荒。鐘鼓竽笙,雲舞翔翔,招搖靈旗,九夷賓將。

惟泰元七建始元年,丞相匡衡奏罷「鸞路龍鱗」,更定詩曰「涓選休成」。

天地並況,惟予有慕,爰熙紫壇,思求厥路。恭承禋祀,縕豫為紛,黼繡周張,承神至尊。千童羅舞成八溢,合好效歡虞泰一。九歌畢奏斐然殊,鳴琴竽瑟會軒朱。璆磬金鼓,靈其有喜,百官濟濟,各敬厥事。盛牲實俎進聞膏,神奄留,臨須搖。長麗前掞光燿明,寒暑不忒況皇章。展詩應律鋗玉鳴,函宮吐角激徵清。發梁揚羽申以商,造茲新音永久長。聲氣遠條鳳鳥启,神夕奄虞蓋孔享。

天地八丞相匡衡奏罷「黼繡周張」,更定詩曰「肅若舊典」。

日出入安窮?時世不與人同。故春非我春,夏非我夏,秋非我秋,冬非我冬。泊如四海之池,遍觀是邪謂何?吾知所樂,獨樂六龍,六龍之調,使我心若。訾黃其何不徠下!

日出入九

太一況,天馬下,霑赤汗,沬流赭。志俶儻,精權奇,逦浮雲,晻上馳。體容與,迣萬里,今安匹,龍為友。元狩三年馬生渥洼水中作。

天馬徠,從西極,涉流沙,九夷服。天馬徠,出泉水,虎脊兩,化若鬼。天馬徠,歷無草,徑千里,循東道。天馬徠,執徐時,將搖舉,誰與期?天馬徠,開遠門,竦予身,逝昆侖。天馬徠,龍之媒,游閶闔,觀玉臺。太初四年誅宛王獲宛馬作。

天馬十

天門開,詄蕩蕩,穆並騁,以臨饗。光夜燭,德信著,靈娅平而鴻,長生豫。大朱涂廣,夷石為堂,飾玉梢以舞歌,體招搖若永望。星留俞,塞隕光,照紫幄,珠熉黃。幡比还集,貳雙飛常羊。月穆穆以金波,日華燿以宣明。假清風軋忽,激長至重觴。神裴回若留放,殣冀親以肆章。函蒙祉福常若期,寂漻上天知厥時。泛泛滇滇從高斿,殷勤此路臚所求。佻正嘉吉弘以昌,休嘉砰隱溢四方。專精厲意逝九閡,紛云六幕浮大海。

天門十一

景星顯見,信星彪列,象載昭庭,日親以察。參侔開闔,爰推本紀,汾脽出鼎,皇祜元始。五音六律,依韋饗昭,雜變並會,雅聲遠姚。空桑琴瑟結信成,四興遞代八風生。殷殷鐘石羽籥鳴。河龍供鯉醇犧牲。百末旨酒布蘭生。泰尊柘漿析朝酲。微感心攸通修名,周流常羊思所并。穰穰復正直往甯,馮蠵切和疏寫平。上天布施后土成,穰穰豐年四時榮。

景星十二元鼎五年得鼎汾陰作。

齊房產草,九莖連葉,宮童效異,披圖案諜。玄氣之精,回復此都,蔓蔓日茂,芝成靈華。

齊房十三元封二年芝生甘泉齊房作。

后皇嘉壇,立玄黃服,物發冀州,兆蒙祉福。沇沇四塞,徦狄合處,經營萬億,咸遂厥宇。

后皇十四

華鳞鳞,固靈根。神之斿,過天門,車千乘,敦昆侖。神之出,排玉房,周流雜,拔蘭堂。神之行,旌容容,騎沓沓,般縱縱。神之徠,泛翊翊,甘露降,慶雲集。神之揄,臨壇宇,九疑賓,夔龍舞。神安坐,启吉時,共翊翊,合所思。神嘉虞,申貳觴,福滂洋,邁延長。沛施祐,汾之阿,揚金光,橫泰河,莽若雲,增陽波。遍臚驩,騰天歌。

華鳞鳞十五

五神相,包四鄰,土地廣,揚浮雲。扢嘉壇,椒蘭芳,璧玉精,垂華光。益億年,美始興,交於神,若有承。廣宣延,咸畢觴,靈輿位,偃蹇驤。卉汨臚,析奚道?淫淥澤,鬓然歸。

五神十六

朝隴首,覽西垠,剨電抠,獲白麟。爰五止,顯黃德,圖匈虐,熏鬻殛。闢流離,抑不詳,賓百僚,山河饗。掩回轅,鬗長馳,騰雨師,洒路陂。流星隕,感惟風,钆歸雲,撫懷心。

朝隴首十七元狩元年行幸雍獲白麟作。

象載瑜,白集西,食甘露,飲榮泉。赤鴈集,六紛員,殊翁雜,五采文。神所見,施祉福,登蓬萊,結無極。

象載瑜十八太始三年行幸東海獲赤鴈作。

赤蛟綏,黃華蓋,露夜零,晝晻濭。百君禮,六龍位,勺椒漿,靈已醉。靈既享,錫吉祥,芒芒極,降嘉觴。靈殷殷,爛揚光,延壽命,永未央。杳冥冥,塞六合,澤汪濊,輯萬國。靈禗禗,象輿轙,票然逝,旗逶蛇。禮樂成,靈將歸,託玄德,長無衰。

赤蛟十九

其餘巡狩福應之事,不序郊廟,故弗論。

是時,河間獻王有雅材,亦以為治道非禮樂不成,因獻所集雅樂。天子下大樂官,常存肄之,歲時以備數,然不常御,常御及郊廟皆非雅聲。然詩樂施於後嗣,猶得有所祖述。昔殷周之雅頌,乃上本有娀、姜原,镨、稷始生,玄王、公劉、古公、大伯、王季、姜女、大任、太姒之德,乃及成湯、文、武受命,武丁、成、康、宣王中興,下及輔佐阿衡、周、召、太公、申伯、召虎、仲山甫之屬,君臣男女有功德者,靡不褒揚。功德既信美矣,褒揚之聲盈乎天地之間,是以光名著於當世,遺譽垂於無窮也。今漢郊廟詩歌,未有祖宗之事,八音調均,又不協於鐘律,而內有掖庭材人,外有上林樂府,皆以鄭聲施於朝廷。

至成帝時,謁者常山王禹世受可間樂,能說其義,其弟子宋嘱等上書言之,下大夫博士平當等考試。當以為「漢承秦滅道之後,賴先帝聖德,博受兼聽,修廢官,立大學,河間獻王聘求幽隱,修興雅樂以助化。時,大儒公孫弘、董仲舒等皆以為音中正雅,立之大樂。春秋鄉射,作於學官,希闊不講。故自公卿大夫觀聽者,但聞鑑鎗,不曉其意,而欲以風諭眾庶,其道無由。是以行之百有餘年,德化至今未成。今嘱等守習孤學,大指歸於興助教化。衰微之學,興廢在人。宜領屬雅樂,以繼絕表微。孔子曰:『人能弘道,非道弘人。』河間區區,不國藩臣,以好學修古,能有所存,民到于今稱之,況於聖主廣被之資,修起舊文,放鄭近雅,述而不作,信而好古,於以風示海內,揚名後世,誠非小功小美也。」事下公卿,以為久遠難分明,當議復寢。

是時,鄭聲尤甚。黃門名倡丙彊、景武之屬富顯於世,貴戚五侯定陵、富平外戚之家淫侈過度,至與人主爭女樂。哀帝自為定陶王時疾之,又性不好音,及即位,下詔曰:「惟世俗奢泰文巧,而鄭衛之聲興。夫奢泰則下不孫而國貧,文巧則趨末背本者眾,鄭衛之聲興則淫辟之化流,而欲黎庶敦朴家給,猶濁其源而求其清流,豈不難哉!孔子不云乎?『放鄭聲,鄭聲淫。』其罷樂府官。郊祭樂及古兵法武樂,在經非鄭衛之樂者,條奏,別屬他官。」丞相孔光、大司空何武奏:「郊祭樂人員六十二人,給祠南北郊。大樂鼓員六人,嘉至鼓員十人,邯鄲鼓員二人,騎吹鼓員三人,江南鼓員二人,淮南鼓員四人,巴俞鼓員三十六人,歌鼓員二十四人,楚嚴鼓員一人,梁皇鼓員四人,臨淮鼓員三十五人,茲邡鼓員三人,凡鼓十二,員百二十八人,朝賀置酒陳殿下,應古兵法。外郊祭員十三人,諸族樂人兼雲招給祠南郊用六十七人,兼給事雅樂用四人,夜誦員五人,剛、別柎員二人,給盛德主調篪員二人,聽工以律知日冬夏至一人,鐘工、磬工、簫工員各一人,僕射二人主領諸樂人,皆不可罷。竽工員三人,一人可罷。琴工員五人,三人可罷。柱工員二人,一人可罷。繩弦工員六人,四人可罷。鄭四會員六十二人,一人給事雅樂,六十一人可罷。張瑟員八人,七人可罷。安世樂鼓員二十人,十九人可罷。沛吹鼓員十二人,族歌鼓員二十七人,陳吹鼓員十三人,商樂鼓員十四人,東海鼓員十六人,長樂鼓員十三人,縵樂鼓員十三人,凡鼓八,員百二十八人,朝賀置酒,陳前殿房中,不應經法。治竽員五人,楚鼓員六人,常從倡三十人,常從象人四人,詔隨常從倡十六人,秦倡員二十九人,秦倡象人員三人,詔隨秦倡一人,雅大人員九人,朝賀置酒為樂。楚四會員十七人,巴四會員十二人,銚四會員十二人,齊四會員十九人,蔡謳員三人,齊謳員六人,竽瑟鐘磬員五人,皆鄭聲,可罷。師學百四十二人,其七十二人給大官挏馬酒,其七十人可罷。大凡八百二十九人,其三百八十八人不可罷,可領屬大樂,其四百四十一人不應經法,或鄭衛之聲,皆可罷。」奏可。然百姓漸漬日久,又不制雅樂有以相變,豪富吏民湛沔自若,陵夷壞于王莽。

今海內更始,民人歸本,戶口歲息,平其刑辟,牧以賢良,至於家給,既庶且富,則須庠序禮樂之教化矣。今幸有前聖遺制之威儀,誠可法象而補備之,經紀可因緣而存著也。

孔子曰:「殷因於夏禮,所損益,可知也;周因於殷禮,所損益,可知也;其或繼周者,百世可知也。」今大漢繼周,久曠大儀,未有立禮成樂,此賈宜、仲舒、王吉、劉向之徒所為發憤而增嘆也。


\end{pinyinscope}