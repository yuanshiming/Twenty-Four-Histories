\article{哀帝紀}

\begin{pinyinscope}
孝哀皇帝,元帝庶孫,定陶恭王子也。母曰丁姬。年三歲嗣立為王,長好文辭法律。元延四年入朝,盡從傅、相、中尉。時成帝少弟中山孝王亦來朝,獨從傅。上怪之,以問定陶王,對曰:「令,諸侯王朝,得從其國二千石。傅、相、中尉皆國二千石,故盡從之。」上令誦詩,通習,能說。他日問中山王:「

獨從傅在何法令?」不能對。令誦尚書,又廢。及賜食於前,後飽;起下,陉係解。成帝由此以為不能,而賢定陶王,數稱其材。時王祖母傅太后隨王來朝,私賂遺上所幸趙昭儀及帝舅票騎將軍曲陽侯王根。昭儀及根見上亡子,亦欲豫自結為長久計,皆更稱定陶王,勸帝以為嗣。成帝亦自美其材,為加元服而遣之,時年十七矣。明年,使執金吾任宏守大鴻臚,持節徵定陶王,立為皇太子。謝曰:「臣幸得繼父守藩為諸侯王,材質不足以假充太子之宮。陛下聖德寬仁,敬承祖宗,奉順神祇,宜蒙福祐子孫千億之報。臣願且得留國邸,旦夕奉問起居,俟有聖嗣,歸國守藩。」書奏,天子報聞。後月餘,立楚孝王孫景為定陶王,奉恭王祀,所以獎厲太子專為後之誼。語在外戚傳。

綏和二年三月,成帝崩。四月丙午,太子即皇帝位,謁高廟。尊皇太后曰太皇太后,皇后曰皇太后。大赦天下。賜宗室王子有屬者馬各一駟,吏民爵,百戶牛酒,三老、孝弟力田、鰥寡孤獨帛。太皇太后詔尊定陶恭王為恭皇。

五月丙戌,立皇后傅氏。詔曰:「春秋『母以子貴』,尊定陶太后曰恭皇太后,丁姬曰恭皇后,各置左右詹事,食邑如長信宮、中宮。」追尊傅父為崇祖侯、丁父為褒德侯。封舅丁明為陽安侯,舅子滿為平周侯。追諡滿父忠為平周懷侯,皇后父晏為孔鄉侯,皇太后弟侍中光祿大夫趙欽為新成侯。

六月,詔曰:「鄭聲淫而亂樂,聖王所放,其罷樂府。」

曲陽侯根前以大司馬建社稷策,益封二千戶。太僕安陽侯舜輔導有舊恩,益封五百戶,及丞相孔光、大司空氾鄉侯何武益封各千戶。

詔曰:「河間王良喪太后三年,為宗室儀表,益封萬戶。」

又曰:「制節謹度以防奢淫,為政所先,百王不易之道也。諸侯王、列侯、公主、吏二千石及豪富民多畜奴婢,田宅亡限,與民爭利,百姓失職,重困不足。其議限列。」有司條奏:「諸王、列侯得名田國中,列侯在長安及公主名田縣道,關內侯、吏民名田,皆無得過三十頃。諸侯王奴婢二百人,列侯、公主百人,關內侯、吏民三十人。年六十以上,十歲以下,不在數中。賈人皆不得名田、為吏,犯者以律論。諸名田畜奴婢過品,皆沒入縣官。齊三服官、諸官織綺繡,難成,害女紅之物,皆止,無作輸。除任子令及誹謗詆欺法。掖庭宮人年三十以下,出嫁之。官奴婢五十以上,免為庶人。禁郡國無得獻名獸。益吏三百石以下奉。察吏殘賊酷虐者,以時退。有司無得舉赦前往事。博士弟子父母死,予寧三年。」

秋,曲陽侯王根、成都侯王況皆有罪。根就國,況免為庶人,歸故郡。

詔曰:「朕承宗廟之重,戰戰兢兢,懼失天心。間者日月亡光,五星失行,郡國比比地動。乃者河南、穎川郡水出,流殺人民,壞敗廬舍。朕之不德,民反蒙辜,朕甚懼焉。已遣光祿大夫循行舉籍,賜死者棺錢,人三千。其令水所傷縣邑及他郡國災害什四以上,民貲不滿十萬,皆無出今年租賦。」

建平元年春正月,赦天下。侍中騎都尉新成侯趙欽、成陽侯趙訢皆有罪,免為庶人,徙遼西。

太皇太后詔外家王氏田非冢塋,皆以賦貧民。

二月,詔曰:「蓋聞聖王之治,以得賢為首。其與大司馬、列侯、將軍、中二千石、州牧、守、相舉孝弟惇厚能直言通政事,延于側陋可親民者,各一人。」

三月,賜諸侯王、公主、列侯、丞相、將軍、中二千石、中都官郎吏金錢帛,各有差。

冬,中山孝王太后媛、弟宜鄉侯馮參有罪,皆自殺。

二年春三月,罷大司空,復御史大夫。

夏四月,詔曰:「漢家之制,推親親以顯尊尊。定陶恭皇之號不宜復稱定陶。尊恭皇太后曰帝太太后,稱永信宮;恭皇后曰帝太后,稱中安宮。立恭皇廟于京師。赦天下徒。」

罷州牧,復刺史。

六月庚申,帝太后丁氏崩。上曰:「朕聞夫婦一體。《詩》云:『穀則異室,死則同穴。』昔季武子成寑,杜氏之殯在西階下,請合葬而許之。附葬之禮,自周興焉。『郁郁乎文哉!吾從周。』孝子事亡如事存。帝太后宜起陵恭皇之園。」遂葬定陶。發陳留、濟陰近郡國五萬人穿復土。

侍詔夏賀良等言赤精子之讖,漢家曆運中衰,當再受命,宜改元易號。詔曰:「漢興二百載,曆數開元。皇天降非材之佑,漢國再獲受命之符,朕之不德,曷敢不通!夫基事之元命,必與天下自新,其大赦天下。以建平二年為太初元將元年。號曰陳聖劉太平皇帝。漏刻以百二十為度。」

七月,以渭城西北原上永陵亭部為初陵。勿徙郡國民,使得自安。

八月,詔曰:「時詔夏賀良等建言改元易號,增益漏刻,可以永安國家。朕過聽賀良等言,冀為海內獲福,卒亡嘉應。皆違經背古,不合時宜。六月甲子制書,非赦令也,皆蠲除之。賀良等反道惑眾,下有司。」皆伏辜。

丞相博、御史大夫玄、孔鄉侯晏有罪。博自殺,玄減死二等論,晏削戶四分之一。語在博傳。

三年春正月,立廣德夷王弟廣漢為廣平王。

癸卯,帝太太后所居桂宮正殿火。

三月己酉,丞相當薨。有星孛于河鼓。

夏六月,立魯頃王子郚鄉侯閔為王。

冬十一月壬子,復甘泉泰畤、汾陰后土祠,罷南北郊。

東平王雲、雲后謁、安成恭侯夫人放皆有罪。雲自殺,謁、放棄市。

四年春,大旱。關東民傳行西王母籌,經歷郡國,西入關至京師。民又會聚祠西王母,或夜持火上屋,擊鼓號呼相驚恐。

二月,封帝太太后從弟侍中傅商為汝昌侯,太后同母弟子侍中鄭業為陽信侯。

三月,侍中駙馬都尉董賢、光祿大夫息夫躬、南陽太守孫寵皆以告東平王封列侯。語在賢傳。

夏五月,賜中二千石至六百石及天下男子爵。

六月,尊帝太太后為皇太太后。

秋八月,恭皇園北門災。

冬,詔將軍、中二千石舉明兵法有大慮者。

元壽元年春正月辛丑朔,日有蝕之。詔曰:「朕獲保宗廟,不明不敏,宿夜憂勞,未皇寧息。惟陰陽不調,元元不贍,未睹厥咎。婁敕公卿,庶幾有望。至今有司執法,未得其中,或上暴虐,假勢獲名,溫良寬柔,陷於亡滅。是故殘賊彌長,和睦日衰,百姓愁怨,靡所錯躬。乃正月朔,日有蝕之,厥咎不遠,在余一人。公卿大夫其各悉心勉帥百寮,敦任仁人,黜遠殘賊,期於安民。陳朕之過失,無有所諱。其與將軍、列侯、中二千石舉賢良方正能直言者各一人。大赦天下。」

丁巳,皇太太后傅氏崩。

三月,丞相嘉有罪,下獄死。

秋九月,大司馬票騎將軍丁明免。

孝元廟殿門銅龜蛇鋪首鳴。

二年春正月,匈奴單于、烏孫大昆彌來朝。二月,歸國,單于不說。語在匈奴傳。

夏四月壬辰晦,日有蝕之。

五月,正三公官分職。大司馬衛將軍董賢為大司馬,丞相孔光為大司徒,御史大夫彭宣為大司空,封長平侯。正司直、司隸,造司寇職,事未定。

六月戊午,帝崩于未央宮。秋九月壬寅,葬義陵。

贊曰:孝哀自為藩王及充太子之宮,文辭博敏,幼有令聞。睹孝成世祿去王室,權柄外移,是故臨朝婁誅大臣,欲彊主威,以則武、宣。雅性不好聲色,時覽卞射武戲。即位痿痺,末年寖劇。饗國不永,哀哉!


\end{pinyinscope}