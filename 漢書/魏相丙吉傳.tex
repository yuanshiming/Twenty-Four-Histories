\article{魏相丙吉傳}

\begin{pinyinscope}
魏相字弱翁,濟陰定陶人也,徙平陵。少學易,為郡卒史,舉賢良,以對策高第,為茂陵令。頃之,御史大夫桑弘羊客詐稱御史止傳,丞不以時謁,客怒縛丞。相疑其有姦,收捕,案致其罪,論棄客市,茂陵大治。

後遷河南太守,禁止姦邪,豪彊畏服。會丞相車千秋死,先是千秋子為雒陽武庫令,自見失父,而相治郡嚴,恐久獲罪,乃自免去。相使掾追呼之,遂不肯還。相獨恨曰:「大將軍聞此令去官,必以為我用丞相死不能遇其子。使當世貴人非我,殆矣!」武庫令西至長安,大將軍霍光果以責過相曰:「幼主新立,以為函谷京師之固,武庫精兵所聚,故以丞相弟為關都尉,子為武庫令。今河南太守不深惟國家大策,苟見丞相不在而斥逐其子,何淺薄也!」後人有告相賊殺不辜,事下有司。河南卒戍中都官者二三千人,遮大將軍,自言願復留作一年以贖太守罪。河南老弱萬餘人守關欲入上書,關吏以聞。大將軍用武庫令事,遂下相廷尉獄。久繫踰冬,會赦出。復有詔守茂陵令,遷楊州刺史。考案郡國守相,多所貶退。相與丙吉相善,時吉為光祿大夫,與相書曰:「朝廷已深知弱翁

行治,方且大用矣。願少慎事自重,臧器于身。」相心善其言,為霽威嚴。居部二歲,徵為諫大夫,復為河南太守。

數年,宣帝即位,徵相入為大司農,遷御史大夫。四歲,大將軍霍光薨,上思其功德,以其子禹為右將軍,兄子樂平侯山復領尚書事。相因平恩侯許伯奏封事,言:「春秋譏世卿,惡宋三世為大夫,及魯季孫之專權,皆危亂國家。自後元以來,祿去王室,政繇冢宰。今光死,子復為大將軍,兄子秉樞機,昆弟諸婿據權勢,在兵官。光夫人顯及諸女皆通籍長信宮,或夜詔門出入,驕奢放縱,恐寖不制。宜有以損奪其權,破散陰謀,以固萬世之基,全功臣之世。」又故事諸上書者皆為二封,署其一曰副,領尚書者先發副封,所言不善,屏去不奏。相復因許伯白,去副封以防雍蔽。宣帝善之,詔相給事中,皆從其議。霍氏殺許后之謀始得上聞。乃罷其三侯,令就第,親屬皆出補吏。於是韋賢以老病免,相遂代為丞相,封高平侯,食邑八百戶。及霍氏怨相,又憚之,謀矯太后詔,先召斬丞相,然後廢天子。事發覺,伏誅。宣帝始親萬機,厲精為治,練群臣,核名實,而相總領眾職,甚稱上意。

元康中,匈奴遣兵擊漢屯田車師者,不能下。上與後將軍趙充國等議,欲因匈奴衰弱,出兵擊其右地,使不敢復擾西域。相上書諫曰:「臣聞之,救亂誅暴,謂之義兵,兵義者王;敵加於己,不得已而起者,謂之應兵,兵應者勝;爭恨小故,不忍憤怒者,謂之忿兵,兵忿者敗;利人土地貨寶者,謂之貪兵,兵貪者破;恃國家之大,矜民人之眾,欲見威於敵者,謂之驕兵,兵驕者滅:此五者,非但人事,乃天道也。間者匈奴嘗有善意,所得漢民輒奉歸之,未有犯於邊境,雖爭屯田車師,不足致意中。今聞諸將軍欲興兵入其地,臣愚不知此兵何名者也。今邊郡困乏,父子共犬羊之裘,食草萊之實,常恐不能自存,難於動兵。『軍旅之後,必有凶年,』言民以其愁苦之氣,傷陰陽之和也。出兵雖勝,猶有後憂,恐災害之變因此以生。今郡國守相多不實選,風俗尤薄,水旱不時。案今年計,子弟殺父兄、妻殺夫者,凡二百二十二人,臣愚以為此非小變也。今左右不憂此,乃欲發兵報纖介之忿於遠夷,殆孔子所謂『吾恐季孫之憂不在顓臾而在蕭牆之內』也。願陛下與平昌侯、樂昌侯、平恩侯及有識者詳議乃可。」上從其言而止。

相明易經,有師法,好觀漢故事及便宜章奏,以為古今異制,方今務在奉行故事而已。數條漢興已來國家便宜行事,及賢臣賈誼、晁錯、董仲舒等所言,奏請施行之,曰:「臣聞明主在上,賢輔在下,則君安虞而民和睦。臣相幸得備位,不能奉明法,廣教化,理四方,以宣聖德。民多背本趨末,或有飢寒之色,為陛下之憂,臣相罪當萬死。臣相知能淺薄,不明國家大體,時用之宜,惟民終始,未得所繇。竊伏觀先帝聖德仁恩之厚,勤勞天下,垂意黎庶,憂水旱之災,為民貧窮發倉廩,賑乏餧;遣諫大夫博士巡行天下,察風俗,舉賢良,平冤獄,冠蓋交道;省諸用,寬租賦,弛山澤波池,禁秣馬酤酒貯積:所以周急繼困,慰安元元,便利百姓之道甚備。臣相不能悉陳,昧死奏故事詔書凡二十三事。臣謹案王法必本於農而務積聚,量入制用以備凶災,亡六年之畜,尚謂之急。元鼎二年,平原、勃海、太山、東郡溥被災害,民餓死於道路。二千石不豫慮其難,使至於此,賴明詔振捄,乃得蒙更生。今歲不登,穀暴騰踴,臨秋收斂猶有乏者,至春恐甚,亡以相恤。西羌未平,師旅在外,兵革相乘,臣竊寒心,宜蚤圖其備。唯陛下留神元元,帥繇先帝盛德以撫海內。」上施行其策。

又數表采易陰陽及明堂月令奏之,曰:「臣相幸得備員,奉職不修,不能宣廣教化。陰陽未和,災害未息,咎在臣等。臣聞《易》曰:『天地以順動,故日月不過,四時不忒;聖王以順動,故刑罰清而民服。』天地變化,必繇陰陽,陰陽之分,以日為紀。日冬夏至,則八風之序立,萬物之性成,各有常職,不得相干。東方之神太昊,乘震執規司春;南方之神炎帝,乘離執衡司夏;西方之神少昊,乘兌執矩司秋;北方之神顓頊,乘坎執權司冬;中央之神黃帝,乘坤艮執繩司下土。茲五帝所司,各有時也。東方之卦不可以治西方,南方之卦不可以治北方。春興兌治則飢,秋興震治則華,冬興離治則泄,夏興坎治則雹。明王謹於尊天,慎于養人,故立羲和之官以乘四時,節授民事。君動靜以道,奉順陰陽,則日月光明,風雨時節,寒暑調和。三者得敘,則災害不生,五穀熟,絲麻遂,屮木茂,鳥獸蕃,民不夭疾,衣食有餘。若是,則君尊民說,上下亡怨,政教不違,禮讓可興。夫風雨不時,則傷農桑;農桑傷,則民飢寒;飢寒在身,則亡廉恥,寇賊姦宄所繇生也。臣愚以為陰陽者,王事之本,群生之命,自古賢聖未有不繇者也。天子之義,必純取法天地,而觀於先聖。高皇帝所述書天子所服第八曰:『大謁者臣章受詔長樂宮,曰:「令群臣議天子所服,以安治天下。」相國臣何、御史大夫臣昌謹與將軍臣陵、太子太傅臣通等議:「春夏秋冬天子所服,當法天地之數,中得人和。故自天子王侯有土之君,下及兆民,能法天地,順四時,以治國家,身亡禍殃,年壽永究,是奉宗廟安天下之大禮也。臣請法之。中謁者趙堯舉春,李舜舉夏,兒湯舉秋,貢禹舉冬,四人各職一時。」大謁者襄章奏,制曰:「可。」』孝文皇帝時,以二月施恩惠於天下,賜孝弟力田及罷軍卒,祠死事者,頗非時節。御史大夫朝錯時為太子家令,奏言其狀。臣相伏念陛下恩澤甚厚,然而災氣未息,竊恐詔令有未合當時者也。願陛下選明經通知陰陽者四人,各主一時,時至明言所職,以和陰陽,天下幸甚!」相數陳便宜,上納用焉。

相敕掾史案事郡國及休告從家還至府,輒白四方異聞,或有逆賊風雨災變,郡不上,相輒奏言之。時丙吉為御史大夫,同心輔政,上皆重之。相為人嚴毅,不如吉寬。視事九歲,神爵三年薨,諡曰憲侯。子弘嗣,甘露中有罪削爵為關內侯。

丙吉字少卿,魯國人也。治律令,為魯獄史。積功勞,稍遷至廷尉右監。坐法失官,歸為州從事。武帝末,巫蠱事起,吉以故廷尉監徵,詔治巫蠱郡邸獄。時宣帝生數月,以皇曾孫坐衛太子事繫,吉見而憐之。又心知太子無事實,重哀曾孫無辜,吉擇謹厚女徒,令保養曾孫,置閒燥處。吉治巫蠱事,連歲不決。後元二年,武帝疾,往來長楊、五柞宮,望氣者言長安獄中有天子氣,於是上遣使者分條中都官詔獄繫者,亡輕重一切皆殺之。內謁者令郭穰夜到郡邸獄,吉閉門拒使者不納,曰:「皇曾孫在。他人亡辜死者猶不可,況親曾孫乎!」相守至天明不得入,穰還以聞,因劾奏吉。武帝亦寤,曰:「天使之也。」因赦天下。郡邸獄繫者獨賴吉得生,恩及四海矣。曾孫病,幾不全者數焉,吉數敕保養乳母加致醫藥,視遇甚有恩惠,以私財物給其衣食。

後吉為車騎將軍軍市令,遷大將軍長史,霍光甚重之,入為光祿大夫給事中。昭帝崩,亡嗣,大將軍光遣吉迎昌邑王賀。賀即位,以行淫亂廢,光與車騎將軍張安世諸大臣議所立,未定。吉奏記光曰:「將軍事孝武皇帝,受襁褓之屬,任天下之寄,孝昭皇帝早崩亡嗣,海內憂懼,欲亟聞嗣主,發喪之日以大誼立後,所立非其人,復以大誼廢之,天下莫不服焉。方今社稷宗廟群生之命在將軍之壹舉。竊伏聽於眾庶,察其所言,諸侯宗室在列位者,未有所聞於民間也。而遺詔所養武帝曾孫名病已在掖庭外家者,吉前使居郡邸時見其幼少,至今十八九矣,通經術,有美材,行安而節和。願將軍詳大議,參以蓍龜,豈宜褒顯,先使入侍,令天下昭然知之,然後決定大策,天下幸甚!」光覽其議,遂尊立皇曾孫,遣宗正劉德與吉迎曾孫於掖庭。宣帝初即位,賜吉爵關內侯。

吉為人深厚,不伐善。自曾孫遭遇,吉絕口不道前恩,故朝廷莫能明其功也。地節三年,立皇太子,吉為太子太傅,數月,遷御史大夫。及霍氏誅,上躬親政,省尚書事。是時,掖庭宮婢則令民夫上書,自陳嘗有阿保之功。章下掖庭令考問,則辭引使者丙吉知狀。掖庭令將則詣御史府以視吉。吉識,謂則曰:「汝嘗坐養皇曾孫不謹督笞,汝安得有功?獨渭城胡組、淮陽郭徵卿有恩耳。」分別奏組等共養勞苦狀。詔吉求組、徵卿,已死,有子孫,皆受厚賞。詔免則為庶人,賜錢十萬。上親見問,然後知吉有舊恩,而終不言。上大賢之,制詔丞相:「朕微眇時,御史大夫吉與朕有舊恩,厥德茂焉。詩不云虖?『亡德不報。』其封吉為博陽侯,邑千三百戶。」臨當封,吉疾病,上將使人加紼而封之,及其生存也。上憂吉疾不起,太子太傅夏侯勝曰:「此未死也。臣聞有陰德者,必饗其樂以及子孫。今吉未獲報而疾甚,非其死疾也。」後病果瘉。吉上書固辭,自陳不宜以空名受賞。上報曰:「朕之封君,非空名也,而君上書歸侯印,是顯朕之不德也。方今天下少事,君其專精神,省思慮,近醫藥,以自持。」後五歲,代魏相為丞相。

吉本起獄法小吏,後學詩、禮,皆通大義。及居相位,上寬大,好禮讓。掾史有罪臧,不稱職,輒予長休告,終無所案驗。客或謂吉曰:「君侯為漢相,姦吏成其私,然無所懲艾。」吉曰:「夫以三公之府有案吏之名,吾竊陋焉。」後人代吉,因以為故事,公府不案吏,自吉始。

於官屬掾史,務掩過揚善。吉馭吏耆酒,數逋蕩,嘗從吉出,醉歐丞相車上。西曹主吏白欲斥之,吉曰:「以醉飽之失去士,使此人將復何所容?西曹地忍之,此不過汙丞相車茵耳。」遂不去也。此馭吏邊郡人,習知邊塞發奔命警備事,嘗出,適見驛騎持赤白囊,邊郡發奔命書馳來至。馭吏因隨驛騎至公車刺取,知虜入雲中、代郡,遽歸府見吉白狀,因曰:「恐虜所入邊郡,二千石長吏有老病不任兵馬者,宜可豫視。」吉善其言,召東曹案邊長吏,瑣科條其人。未已,詔召丞相、御史,問以虜所入郡吏,吉具對。御史大夫卒遽不能詳知,以得譴讓。而吉見謂憂邊思職,馭吏力也。吉乃歎曰:「士亡不可容,能各有所長。嚮使丞相不先聞馭吏言,何見勞勉之有?」掾史繇是益賢吉。

吉又嘗出,逢清道群鬥者,死傷橫道,吉過之不問,掾史獨怪之。吉前行,逢人逐牛,牛喘吐舌。吉止駐,使騎吏問:「逐牛行幾里矣?」掾史獨謂丞相前後失問,或以譏吉,吉曰:「民鬥相殺傷,長安令、京兆尹職所當禁備逐捕,歲竟丞相課其殿最,奏行賞罰而已。宰相不親小事,非所當於道路問也。方春少陽用事,未可大熱,恐牛近行用暑故喘,此時氣失節,恐有所傷害也。三公典調和陰陽,職所當憂,是以問之。」掾史乃服,以吉知大體。

五鳳三年春,吉病篤。上自臨問吉,曰:「君即有不諱,誰可以自代者?」吉辭謝曰:「群臣行能,明主所知,愚臣無所能識。」上固問,吉頓首曰:「西河太守杜延年明於法度,曉國家故事,前為九卿十餘年,今在郡治有能名。廷尉于定國執憲詳平,天下自以不冤。太僕陳萬年事後母孝,惇厚備於行止。此三人能皆在臣右,唯上察之。」上以吉言皆是而許焉。及吉薨,御史大夫黃霸為丞相,徵西河太守杜延年為御史大夫,會其年老,乞骸骨,病免。以廷尉于定國代為御史大夫。黃霸薨,而定國為丞相,太僕陳萬年代定國為御史大夫,居位皆稱職,上稱吉為知人。

吉薨,諡曰定侯。子顯嗣,甘露中有罪削爵為關內侯,官至衛尉太僕。始顯少為諸曹,嘗從祠高廟,至夕牲日,乃使出取齋衣。丞相吉大怒,謂其夫人曰:「宗廟至重,而顯不敬慎,亡吾爵者必顯也。」夫人為言,然後乃已。吉中子禹為水衡都尉。少子高為中壘校尉。

元帝時,長安士伍尊上書,言「臣少時為郡邸小吏,竊見孝宣皇帝以皇曾孫在郡邸獄。是時治獄使者丙吉見皇曾孫遭離無辜,吉仁心感動,涕泣悽惻,選擇復作胡組養視皇孫,吉常從。臣尊日再侍臥庭上。後遭條獄之詔,吉扞拒大難,不避嚴刑峻法。既遭大赦,吉謂守丞誰如,皇孫不當在官,使誰如移書京兆尹,遣與胡組俱送京兆尹,不受,復還。及組日滿當去,皇孫思慕,吉以私錢顧組,令留與郭徵卿並養數月,乃遣組去。後少內嗇夫白吉曰:『

食皇孫亡詔令。』時吉得食米肉,月月以給皇孫。吉即時病,輒使臣尊朝夕請問皇孫,視省席蓐燥濕。候伺組、徵卿,不得令晨夜去皇孫敖盪,數奏甘毳食物。所以擁全神靈,成育聖躬,功德已亡量矣。時豈豫知天下之福,而徼其報哉!誠其仁恩內結於心也。雖介之推割肌以存君,不足比也。孝宣皇帝時,臣上書言狀,幸得下吉,吉謙讓不敢自伐,刪去臣辭,專歸美於組、徵卿。組、徵卿皆以受田宅賜錢,吉封為博陽侯。臣尊不得比組、徵卿。臣年老居貧,死在旦暮,欲終不言,恐使有功不著。吉子顯坐微文奪爵為關內侯,臣愚以為宜復其爵邑,以報先人功德。」先是顯為太僕十餘年,與官屬大為姦利,臧千餘萬,司隸校尉昌案劾,罪至不道,奏請逮捕。上曰:「故丞相吉有舊恩,朕不忍絕。」免顯官,奪邑四百戶。後復以為城門校尉。顯卒,子昌嗣爵關內侯。

成帝時,修廢功,以吉舊恩尤重,鴻嘉元年制詔丞相御史:「蓋聞褒功德,繼絕統,所以重宗廟,廣賢聖之路也。故博陽侯吉以舊恩有功而封,今其祀絕,朕甚憐之。夫善善及子孫,古今之通誼也,其封吉孫中郎將關內侯昌為博陽侯,奉吉後。」國絕三十二歲復續云。昌傳子至孫,王莽時乃絕。

贊曰:古之制名,必繇象類,遠取諸物,近取諸身。故經謂君為元首,臣為股肱,明其一體,相待而成也。是故君臣相配,古今常道,自然之勢也。近觀漢相,高祖開基,蕭、曹為冠,孝宣中興,丙、魏有聲。是時黜陟有序,眾職修理,公卿多稱其位,海內興於禮讓。覽其行事,豈虛虖哉!


\end{pinyinscope}