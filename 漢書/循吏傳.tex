\article{循吏傳}

\begin{pinyinscope}
漢興之初,反秦之敝,與民休息,凡事簡易,禁罔疏闊,而相國蕭、曹以寬厚清靜為天下帥,民作「畫一」之歌。孝惠垂拱,高后女主,不出房闥,而天下晏然,民務稼穡,衣食滋殖。至於文、景,遂移風易俗。是時循吏如河南守吳公、蜀守文翁之屬,皆謹身帥先,居以廉平,不至於嚴,而民從化。

孝武之世,外攘四夷,內改法度,民用彫敝,姦軌不禁。時少能以化治稱者,惟江都相董仲舒、內史公孫弘、兒寬,居官可紀。三人皆儒者,通於世務,明習文法,以經術潤飾吏事,天子器之。仲舒數謝病去,弘、寬至三公。

孝昭幼沖,霍光秉政,承奢侈師旅之後,海內虛耗,光因循守職,無所改作。至於始元、元鳳之間,匈奴鄉化,百姓益富,舉賢良文學,問民所疾苦,於是罷酒榷而議鹽鐵矣。

及至孝宣,繇仄陋而登至尊,興于閭閻,知民事之谡難。自霍光薨後始躬萬機,厲精為治,五日一聽事,自丞相已下各奉職而進。及拜刺史守相,輒親見問,觀其所繇,退而考察所行以質其言,有名實不相應,必知其所以然。常稱曰:「庶民所以安其田里而亡歎息愁恨之心者,政平訟理也。與我共此者,其唯良二千石乎!」以為太守,吏民之本也,數變易則下不安,民知其將久,不可欺罔,乃服從其教化。故二千石有治理效,輒以璽書勉厲,增秩賜金,或爵至關內侯,公卿缺則選諸所表以次用之。是故漢世良吏,於是為盛,稱中興焉。若趙廣漢、韓延壽、尹翁歸、嚴延年、張敞之屬,皆稱其位,然任刑罰,或抵罪誅。王成、黃霸、朱邑、龔遂、鄭弘、召信臣等,所居民富,所去見思,生有榮號,死見奉祀,此廩廩庶幾德讓君子之遺風矣。

文翁,廬江舒人也。少好學,通春秋,以郡縣吏察舉。景帝末,為蜀郡守,仁愛好教化。見蜀地辟陋有蠻夷風,文翁欲誘進之,乃選郡縣小吏開敏有材者張叔等十餘人親自飭厲,遣詣京師,受業博士,或學律令。減省少府用度,買刀布蜀物,齎計吏以遺博士。數歲,蜀生皆成就還歸,文翁以為右職,用次察舉,官有至郡守刺史者。

又修起學官於成都市中,招下縣子弟以為學官弟子,為除更繇,高者以補郡縣吏,次為孝弟力田。常選學官僮子,使在便坐受事。每出行縣,益從學官諸生明經飭行者與俱,使傳教令,出入閨閤。縣邑吏民見而榮之,數年,爭欲為學官弟子,富人至出錢以求之。繇是大化,蜀地學於京師者比齊魯焉。至武帝時,乃令天下郡國皆立學校官,自文翁為之始云。

文翁終於蜀,吏民為立祠堂,歲時祭祀不絕。至今巴蜀好文雅,文翁之化也。

王成,不知何郡人也。為膠東相,治甚有聲。宣帝最先褒之,地節三年下詔曰:「蓋聞有功不賞,有罪不誅,雖唐虞不能以化天下。今膠東相成,勞來不怠,流民自占八萬餘口,治有異等之效。其賜成爵關內侯,秩中二千石。」未及徵用,會病卒官。後詔使丞相御史問郡國上計長吏守丞以政令得失,或對言前膠東相成偽自增加,以蒙顯賞,是後俗吏多為虛名云。

黃霸字次公,淮陽陽夏人也,以豪桀役使徙雲陵。霸少學律令,喜為吏,武帝末以待詔入錢賞官,補侍郎謁者,坐同產有罪劾免。後復入穀沈黎郡,補左馮翊二百石卒史。馮翊以霸入財為官,不署右職,使領郡錢穀計。簿書正,以廉稱,察補河東均輸長,復察廉為河南太守丞。霸為人明察內敏,又習文法,然溫良有讓,足知,善御眾。為丞,處議當於法,合人心,太守甚任之,吏民愛敬焉。

自武帝末,用法深。昭帝立,幼,大將軍霍光秉政,大臣爭權,上官桀等與燕王謀作亂,光既誅之,遂遵武帝法度,以刑罰痛繩群下,繇是俗吏上嚴酷以為能,而霸獨用寬和為名。

會宣帝即位,在民間時知百姓苦吏急也,聞霸持法平,召以為廷尉正,數決疑獄,庭中稱平。守丞相長史,坐公卿大議廷中知長信少府夏侯勝非議詔書大不敬,霸阿從不舉劾,皆下廷尉,繫獄當死。霸因從勝受尚書獄中,再隃冬,積三歲乃出,語在勝傳。勝出,復為諫大夫,令左馮翊宋畸舉霸賢良。勝又口薦霸於上,上擢霸為揚州刺史。三歲,宣帝下詔曰:「制詔御史:其以賢良高第揚州刺史霸為潁川太守,秩比二千石,居官賜車蓋,特高一丈,別駕主簿車,緹油屏泥於軾前,以章有德。」

時上垂意於治,數下恩澤詔書,吏不奉宣。太守霸為選擇良吏,分部宣布詔令,令民咸知上意。使郵亭鄉官皆畜雞豚,以贍鰥寡貧窮者,然後為條教,置父老師帥伍長,班行之於民間,勸以為善防姦之意,及務耕桑,節用殖財,種樹畜養,去食穀馬。米鹽靡密,初若煩碎,然霸精力能推行之。吏民見者,語次尋繹,問它陰伏,以相參考。嘗欲有所司察,擇長年廉吏遣行,屬令周密。吏出,不敢舍郵亭,食於道旁,烏攫其肉。民有欲詣府口言事者適見之,霸與語道此。後日吏還謁霸,霸見迎勞之,曰:「甚苦!食於道旁乃為烏所盜肉。」吏大驚,以霸具知其起居,所問豪氂不敢有所隱。鰥寡孤獨有死無以葬者,鄉部書言,霸具為區處,某所大木可以為棺,某亭豬子可以祭,吏往皆如言。其識事聰明如此,吏民不知所出,咸稱神明。姦人去入它郡,盜賊日少。

霸力行教化而後誅罰,務在成就全安長吏。許丞老,病聾,督郵白欲逐之,霸曰:「許丞廉吏,雖老,尚能拜起送迎,正頗重聽,何傷?且善助之,毋失賢者意。」或問其故,霸曰:「數易長吏,送故迎新之費及姦吏緣絕簿書盜財物,公私費耗甚多,皆當出於民,所易新吏又未必賢,或不如其故,徒相益為亂。凡治道,去其泰甚者耳。」

霸以外寬內明得吏民心,戶口歲增,治為天下第一。徵守京兆尹,秩二千石。坐發民治馳道不先以聞,又發騎士詣北軍馬不適士,劾乏軍興,連貶秩。有詔歸潁川太守官,以八百石居治如其前。前後八年,郡中愈治。是時鳳皇神爵數集郡國,潁川尤多。天子以霸治行終長者,下詔稱揚曰:「穎川太守霸,宣布詔令,百姓鄉化,孝子弟弟貞婦順孫日以眾多,田者讓畔,道不拾遺,養視鰥寡,贍助貧窮,獄或八年亡重罪囚,吏民鄉于教化,興於行誼,可謂賢人君子矣。書不云乎?『股肱良哉!』其賜爵關內侯,黃金百斤,秩中二千石。」而潁川孝弟有行義民、三老、力田,皆以差賜爵及帛。後數月,徵霸為太子太傅,遷御史大夫。

五鳳三年,代丙吉為丞相,封建成侯,食邑六百戶。霸材長於治民,及為丞相,總綱紀號令,風采不及丙、魏、于定國,功名損於治郡。時京兆尹張敞舍鶡雀飛集丞相府,霸以為神雀,議欲以聞。敞奏霸曰:「竊見丞相請與中二千石博士雜問郡國上計長吏守丞,為民興利除害成大化條其對,有耕者讓畔,男女異路,道不拾遺,及舉孝子弟弟貞婦者為一輩,先上殿,舉而不知其人數者次之,不為條教者在後叩頭謝。丞相雖口不言,而心欲其為之也。長吏守丞對時,臣敞舍有鶡雀飛止丞相府屋上,丞相以下見者數百人。邊吏多知鶡雀者,問之,皆陽不知。丞相圖議上奏曰:『臣問上計長吏守丞以興化條,皇天報下神雀。』後知從臣敞舍來,乃止。郡國吏竊笑丞相仁厚有知略,微信奇怪也。昔汲黯為淮陽守,辭去之官,謂大行李息曰:『御史大夫張湯懷詐阿意,以傾朝廷,公不早白,與俱受戮矣。』息畏湯,終不敢言。後湯誅敗,上聞黯與息語,乃抵息罪而秩黯諸侯相,取其思竭忠也。臣敞非敢毀丞相也,誠恐群臣莫白,而長吏守丞畏丞相指,歸舍法令,各為私教,務相增加,澆淳散樸,並行偽貌,有名亡實,傾搖解怠,甚者為妖。假令京師先行讓畔異路,道不拾遺,其實亡益廉貪貞淫之行,而以偽先天下,固未可也;即諸侯先行之,偽聲軼於京師,非細事也。漢家承敝通變,造起律令,所以勸善禁姦,條貫詳備,不可復加。宜令貴臣明飭長吏守丞,歸告二千石,舉三老孝弟力田孝廉廉吏務得其人,郡事皆以義法令撿式,毋得擅為條教;敢挾詐偽以奸名譽者,必先受戮,以正明好惡。」天子嘉納敞言,召上計吏,使侍中臨飭如敞指意。霸甚慚。

又樂陵侯史高以外屬舊恩侍中貴重,霸薦高可太尉。天子使尚書召問霸:「太尉官罷久矣,丞相兼之,所以偃武興文也。如國家不虞,邊境有事,左右之臣皆將率也。夫宣明教化,通達幽隱,使獄無冤刑,邑無盜賊,君之職也。將相之官,朕之任焉。侍中樂陵侯高帷幄近臣,朕之所自親,君何越職而舉之?」尚書令受丞相對,霸免冠謝罪,數日乃決。自是後不敢復有所請。然自漢興,言治民吏,以霸為首。

為丞相五歲,甘露三年薨,諡曰定侯。霸死後,樂陵侯高竟為大司馬。霸子思侯賞嗣,為關都尉。薨,子忠侯輔嗣,至衛尉九卿。薨,子忠嗣侯,訖王莽乃絕。子孫為吏二千石者五六人。

始霸少為陽夏游徼,與善相人者共載出,見一婦人,相者言「此婦人當富貴,不然,相書不可用也。」霸推問之,乃其鄉里巫家女也。霸即取為妻,與之終身。為丞相後徙杜陵。

朱邑字仲卿,廬江舒人也。少時為舒桐鄉嗇夫,廉平不苛,以愛利為行,未嘗笞辱人,存問耆老孤寡,遇之有恩,所部吏民愛敬焉。遷補太守卒史,舉賢良為大司農丞,遷北海太守,以治行第一入為大司農。為人淳厚,篤於故舊,然性公正,不可交以私。天子器之,朝延敬焉。

是時張敞為膠東相,與邑書曰:「明主游心太古,廣延茂士,此誠忠臣謁思之時也。直敞遠守劇郡,馭於繩墨,匈臆約結,固亡奇也。雖有,亦安所施?足下以清明之德,掌周稷之業,猶飢者甘糟糠,穰歲餘粱肉。何則?有亡之勢異也。昔陳平雖賢,須魏倩而後進;韓信雖奇,賴蕭公而後信。故事各達其時之英俊,若必伊尹、呂望而後薦之,則此人不因足下而進矣。」邑感敞言,貢薦賢士大夫,多得其助者。身為列卿,居處儉節,祿賜以共九族鄉黨,家亡餘財。

神爵元年卒。天子閔惜,下詔稱揚曰:「大司農邑,廉潔守節,退食自公,亡彊外之交,束脩之餽,可謂淑人君子。遭離凶災,朕甚閔之。其賜邑子黃金百斤,以奉其祭祀。」

初邑病且死,屬其子曰:「我故為桐鄉吏,其民愛我,必葬我桐鄉。後世子孫奉嘗我,不如桐鄉民。」及死,其子葬之桐鄉西郭外,民果然共為邑起冢立祠,歲時祠祭,至今不絕。

龔遂字少卿,山陽南平陽人也。以明經為官,至昌邑郎中令,事王賀。賀動作多不正,遂為人忠厚,剛毅有大節,內諫爭於王,外責傅相,引經義,陳禍福,至於涕泣,蹇蹇亡已。面刺王過,王至掩耳起走,曰「郎中令善媿人。」及國中皆畏憚焉。王嘗久與騶奴宰人游戲飲食,賞賜亡度,遂入見王,涕泣膝行,左右侍御皆出涕。王曰:「郎中令何為哭?」遂曰:「臣痛社稷危也!願賜清閒竭愚。」王辟左右,遂曰:「大王知膠西王所以為無道亡乎?」王曰:「不知也。」曰:「臣聞膠西王有諛臣侯得,王所為儗於桀紂也,得以為堯舜也。王說其諂諛,嘗與寑處,唯得所言,以至於是。今大王親近群小,漸漬邪惡所習,存亡之機,不可不慎也。臣請選郎通經術有行義者與王起居,坐則誦詩書,立則習禮容,宜有益。」王許之。遂乃選郎中張安等十人侍王。居數日,王皆去逐安等。久之,宮中數有妖怪,王以問遂,遂以為有大憂,宮室將空,語在昌邑王傳。會昭帝崩,亡子,昌邑王賀嗣立,官屬皆徵入。王相安樂遷長樂衛尉,遂見安樂,流涕謂曰:「王立為天子,日益驕溢,諫之不復聽,今哀痛未盡,日與近臣飲食作樂,鬥虎豹,召皮軒,車九流,驅馳東西,所為誖道。古制寬,大臣有隱退,今去不得,陽狂恐知,身死為世戮,奈何?君,陛下故相,宜極諫爭。」王即位二十七日,卒以淫亂廢。昌邑群臣坐陷王於惡不道,皆誅,死者二百餘人,唯遂與中尉王陽以數諫爭得減死,髡為城旦。

宣帝即位,久之,渤海左右郡歲飢,盜賊並起,二千石不能禽制。上選能治者,丞相御史舉遂可用,上以為渤海太守。時遂年七十餘,召見,形貌短小,宣帝望見,不副所聞,心內輕焉,謂遂曰:「渤海廢亂,朕甚憂之。君欲何以息其盜賊,以稱朕意?」遂對曰:「海瀕遐遠,不霑聖化,其民困於飢寒而吏不恤,故使陛下赤子盜弄陛下之兵於潢池中耳。今欲使臣勝之邪,將安之也?」上聞遂對,甚說,答曰:「選用賢良,固欲安之也。」遂曰:「臣聞治亂民猶治亂繩,不可急也;唯緩之,然後可治。臣願丞相御史且無拘臣以文法,得一切便宜從事。」上許焉,加賜黃金,贈遣乘傳。至渤海界,郡聞新太守至,發兵以迎,遂皆遣還,移書敕屬縣悉罷逐捕盜賊吏。諸持鉏鉤田器者皆為良民,吏無得問,持兵者乃為盜賊。遂單車獨行至府,郡中翕然,盜賊亦皆罷。渤海又多劫略相隨,聞遂教令,即時解散,棄其兵弩而持鉤鉏。盜賊於是悉平,民安土樂業。遂乃開倉廩假貧民,選用良吏,尉安牧養焉。

遂見齊俗奢侈,好末技,不田作,乃躬率以儉約,勸民務農桑,令口種一樹榆、百本筹、五十本蔥、一畦韭,家二母彘、五雞。民有帶持刀劍者,使賣劍買牛,賣刀買犢,曰:「何為帶牛佩犢!」春夏不得不趨田畝,秋冬課收斂,益蓄困實蔆芡。勞來循行,郡中皆有畜積,吏民皆富實。獄訟止息。

數年,上遣使者徵遂,議曹王生願從。功曹以為王生素耆酒,亡節度,不可使。遂不忍逆,從至京師。王生日飲酒,不視太守。會遂引入宮,王生醉,從後呼,曰:「明府且止,願有所白。」遂還問其故,王生曰:「天子即問君何以治渤海,君不可有所陳對,宜曰『皆聖主之德,非小臣之力也』。」遂受其言。既至前,上果問以治狀,遂對如王生言。天子說其有讓,笑曰:「君安得長者之言而稱之?」遂因前曰:「臣非知此,乃臣議曹教戒臣也。」上以遂年老不任公卿,拜為水衡都尉,議曹王生為水衡丞,以褒顯遂云。水衡典上林禁苑,共張宮館,為宗廟取牲,官職親近,上甚重之,以官壽卒。

召信臣字翁卿,九江壽春人也。以明經甲科為郎,出補穀陽長。舉高第,遷上蔡長。其治視民如子,所居見稱述。超為零陵太守,病歸。復徵為諫大夫,遷南陽太守,其治如上蔡。

信臣為人勤力有方略,好為民興利,務在富之。躬勸耕農,出入阡陌,止舍離鄉亭,稀有安居時。行視郡中水泉,開通溝瀆,起水門提閼凡數十處,以廣溉灌,歲歲增加,多至三萬頃。民得其利,畜積有餘。信臣為民作均水約束,刻石立於田畔,以防分爭。禁止嫁娶送終奢靡,務出於儉約。府縣吏家子弟好游敖,不以田作為事,輒斥罷之,甚者案其不法,以視好惡。其化大行,郡中莫不耕稼力田,百姓歸之,戶口增倍,盜賊獄訟衰止。吏民親愛信臣,號之曰召父。荊州刺史奏信臣為百姓興利,郡以殷富,賜黃金四十斤。遷河南太守,治行常為第一,復數增秩賜金。

竟寧中,徵為少府,列於九卿,奏請上林諸離遠宮館稀幸御者,勿復繕治共張,又奏省樂府黃門倡優諸戲,及宮館兵弩什器減過泰半。太官園種冬生蔥韭菜茹,覆以屋廡,晝夜然蘊火,待溫氣乃生,信臣以為此皆不時之物,有傷於人,不宜以奉供養,及它非法食物,悉奏罷,省費歲數千萬。信臣年老以官卒。

元始四年,詔書祀百辟卿士有益於民者,蜀郡以文翁,九江以召父應詔書。歲時郡二千石率官屬行禮,奉祠信臣冢,而南陽亦為立祠。


\end{pinyinscope}