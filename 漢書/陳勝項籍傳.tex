\article{陳勝項籍傳}

\begin{pinyinscope}
陳勝字涉,陽城人。吳廣,字叔,陽夏人也。勝少時,嘗與人傭耕。輟耕之壟上,悵然甚久,曰:「苟富貴,無相忘!」傭者笑而應曰:「若為傭耕,何富貴也?」勝太息曰:「嗟乎,燕雀安知鴻鵠之志哉!」

秦二世元年秋七月,發閭左戍漁陽九百人,勝、廣皆為屯長。行至鄿大澤鄉,會天大雨,道不通,度已失期。失期法斬,勝、廣乃謀曰:「今亡亦死,舉大計亦死,等死,死國可乎?」勝曰:「天下苦秦久矣。吾聞二世,少子,不當立,當立者乃公子扶蘇。扶蘇以數諫故不得立,上使外將兵。今或聞無罪,二世殺之。百姓多聞其賢,未知其死。項燕為楚將,數有功,愛士卒,楚人憐之。或以為在。今誠以吾眾為天下倡,宜多應者。」廣以為然。乃行卜。卜者知其指意,曰:「足下事皆成,有功。然足下卜之鬼乎!」勝、廣喜,念鬼,曰:「此教我先威眾耳。」乃丹書帛曰「陳勝王」,置人所罾魚腹中。卒買魚亨食,得書,已怪之矣。又間令廣之次所旁叢祠中,夜構火,狐鳴呼曰:「大楚興,陳勝王。」卒皆夜驚恐。旦日,卒中往往指目勝、廣。

勝、廣素愛人,士卒多為用。將尉醉,廣故數言欲亡,忿尉,令辱之,以激怒其眾。尉果笞廣。尉劍挺,廣起奪而殺尉。勝佐之,并殺兩尉。召令徒屬曰:「公等遇雨,皆已失期,當斬。藉弟令毋斬,而戍死者固什六七。且壯士不死則已,死則舉大名耳。侯王將相,寧有種乎!」徒屬皆曰:「敬受令。」乃詐稱公子扶蘇、項燕,從民望也。袒右,稱大楚。為壇而盟,祭以尉首。勝自立為將軍,廣為都尉。攻大澤鄉,拔之。收兵而攻鄿,蘄下。乃令符離人葛嬰將兵徇蘄以東,攻銍、酇、苦、柘、譙,皆下之。行收兵,比至陳,兵車六七百乘,騎千餘,卒數萬人。攻陳,陳守令皆不在,獨守丞與戰譙門中。不勝,守丞死。乃入據陳。數日,號召三老豪桀會計事。皆曰:「將軍身被堅執銳,伐無道,誅暴秦,復立楚之社稷,功宜為王。」勝乃立為王,號為張楚。

於是諸郡縣苦秦吏暴,皆殺其長吏,將以應勝。乃以廣為假王,監諸將以西擊滎陽。令陳人武臣、張耳、陳餘徇趙,汝陰人鄧宗徇九江郡。當此時,楚兵數千人為聚者不可勝數。

葛嬰至東城,立襄彊為楚王。後聞勝已立,因殺襄彊,還報。至陳,勝殺嬰,令魏人周市北徇魏地。廣圍滎陽。李由為三川守守滎陽,廣不能下。勝徵國之豪桀與計,以上蔡人房君蔡賜為上柱國。

周文,陳賢人也,嘗為項燕軍視日,事春申君,自言習兵。勝與之將軍印,西擊秦。行收兵至關,車千乘,卒十萬,至戲,軍焉。秦令少府章邯免驪山徒、人奴產子,悉發以擊楚軍,大敗之。周文走出關,止屯曹陽。二月餘,章邯追敗之,復走黽池。十餘日,章邯擊,大破之。周文自剄,軍遂不戰。

武臣至邯鄲,自立為趙王,陳餘為大將軍,張耳、召騷為左右丞相。勝怒,捕繫武臣等家室,欲誅之。柱國曰:「秦未亡而誅趙王將相家屬,此生一秦,不如因立之。」勝乃遣使者賀趙,而徙繫武臣等家屬宮中。而封張耳子敖為成都君,趣趙兵亟入關。趙王將相相與謀曰:「王王趙,非楚意也。楚已誅秦,必加兵於趙。計莫如毋西兵,使使北徇燕地以自廣。趙南據大河,北有燕代,楚雖勝秦,不敢制趙,若不勝秦,必重趙。趙承秦楚之敝,可以得志於天下。」趙王以為然,因不西兵,而遣故上谷卒史韓廣將兵北徇燕。

燕地貴人豪桀謂韓廣曰:「楚趙皆已立王。燕雖小,亦萬乘之國也,願將軍立為王。」韓廣曰:「廣母在趙,不可。」燕人曰:「趙方西憂秦,南憂楚,其力不能禁我。且以楚之強,不敢害趙王將相之家,今趙又安敢害將軍之家乎?」韓廣以為然,乃自立為燕王。居數月,趙奉燕王母家屬歸之。

是時,諸將徇地者不可勝數。周市北至狄,狄人田儋殺狄令,自立為齊王,反擊周市。市軍散,還至魏地,立魏後故甯陵君咎為魏王。咎在勝所,不得之魏。魏地已定,欲立周市為王,市不肯。使者五反,勝乃立甯陵君為魏王,遣之國。周市為相。

將軍田臧等相與謀曰:「周章軍已破,秦兵且至,我守滎陽城不能下,秦軍至,必大敗。不如少遺兵,足以守滎陽,悉精兵迎秦軍。今假王驕,不知兵權,不可與計,非誅之,事恐敗。」因相與矯陳王令以誅吳廣,獻其首於勝。勝使賜田臧楚令尹印,使為上將。田臧乃使諸將李歸等守滎陽城,自以精兵西迎秦軍於敖倉。與戰,田臧死,軍破。章邯進擊李歸等滎陽下,破之,李歸死。

陽城人鄧說將兵居郯,章邯別將擊破之,鄧說走陳。銍人五逢將兵居許,章邯擊破之。五逢亦走陳。勝誅鄧說。

勝初立時,淩人秦嘉、銍人董惞、符離人朱雞石、取慮人鄭布、徐人丁疾等皆特起,將兵圍東海守於郯。勝聞,乃使武平君畔為將軍,監郯下軍。秦嘉自立為大司馬,惡屬人,告軍吏曰:「武平君年少,不知兵事,勿聽。」因矯以王命殺武平君畔。

章邯已破五逢,擊陳,柱國房君死。章邯又進擊陳西張賀軍。勝出臨戰,軍破,張賀死。

臘月,勝之汝陰,還至下城父,其御莊賈殺勝以降秦。葬碭,諡曰隱王。

勝故涓人將軍呂臣為蒼頭軍,起新陽,攻陳下之,殺莊賈,復以陳為楚。

初,勝令銍人宋留將兵定南陽,入武關。留已徇南陽,聞勝死,南陽復為秦。宋留不能入武關,乃東至新蔡,遇秦軍,宋留以軍降秦。秦傳留至咸陽,車裂留以徇。

秦嘉等聞勝軍敗,乃立景駒為楚王,引兵之方與,欲擊秦軍濟陰下。使公孫慶使齊王,欲與并力俱進。齊王曰:「陳王戰敗,未知其死生,楚安得不請而立王?」公孫慶曰:「齊不請楚而立王,楚何故請齊而立王?且楚首事,當令於天下。」田儋殺公孫慶。

秦左右校復攻陳,下之。呂將軍走,徼兵復聚,與番盜英布相遇,攻擊秦左右校,破之青波,復以陳為楚。會項梁立懷王孫心為楚王。

陳勝王凡六月。初為王,其故人嘗與傭耕者聞之,乃之陳,叩宮門曰:「吾欲見涉。」宮門令欲縛之。自辯數,乃置,不肯為通。勝出,遮道而呼涉。乃召見,載與歸。入宮,見殿屋帷帳,客曰:「夥,涉之為王沈沈者!」楚人謂多為夥,故天下傳之,「夥涉為王」,由陳涉始。客出入愈益發舒,言勝故情。或言「

客愚無知,專妄言,輕威。」勝斬之。諸故人皆自引去,由是無親勝者。以朱防為中正,胡武為司過,主司群臣。諸將徇地,至,令之不是者,繫而罪之。以苛察為忠。其所不善者,不下吏,輒自治。勝信用之,諸將以故不親附。此其所以敗也。

勝雖已死,其所置遣侯王將相竟亡秦。高祖時為勝置守冢于碭,至今血食。王莽敗,乃絕。

項籍字羽,下相人也。初起,年二十四。其季父梁,梁父即楚名將項燕者也。家世楚將,封於項,故姓項氏。

籍少時,學書不成,去;學劍又不成,去。梁怒之。籍曰:「書足記姓名而已。劍一人敵,不足學,學萬人敵耳。」於是梁奇其意,乃教以兵法。籍大喜,略知其意,又不肯竟。梁嘗有櫟陽逮,請蘄獄掾曹咎書抵櫟陽獄史司馬欣,以故事皆已。梁嘗殺人,與籍避仇吳中。吳中賢士大夫皆出梁下。每有大繇役及喪,梁常主辦,陰以兵法部勒賓客子弟,以知其能。秦始皇帝東遊會稽,渡浙江,梁與籍觀。籍曰:「彼可取而代也。」梁掩其口,曰:「無妄言,族矣!」梁以此奇籍。籍長八尺二寸,力扛鼎,才氣過人。吳中弟子皆憚籍。

秦二世元年,陳勝起。九月,會稽假守通素賢梁,乃召與計事。梁曰:「方今江西皆反秦,此亦天亡秦時也。先發制人,後發制於人。」守歎曰:「聞夫子楚將世家,唯足下耳!」梁曰:「吳有奇士桓楚,亡在澤中,人莫知其處,獨籍知之。」梁乃戒籍持劍居外待。梁復入,與守語曰:「請召籍,使受令召桓楚。」籍入,梁眴籍曰:「可行矣!」籍遂拔劍擊斬守。梁持守頭,佩其印綬。門下驚擾,籍所擊殺數十百人。府中皆讋伏,莫敢復起。梁乃召故人所知豪吏,諭以所為,遂舉吳中兵。使人收下縣,得精兵八千人,部署豪桀為校尉、候、司馬。有一人不得官,自言。梁曰:「某時某喪,使公主某事,不能辦,以故不任公。」眾乃皆服。梁為會稽將,籍為裨將,徇下縣。

秦二年,廣陵人召平為陳勝徇廣陵,未下。聞陳勝敗走,秦將章邯且至,乃渡江矯陳王令,拜梁為楚上柱國,曰:「江東已定,急引兵西擊秦。」梁乃以八千人渡江而西。聞陳嬰已下東陽,使使欲與連和俱西。陳嬰者,故東陽令史,居縣,素信,為長者。東陽少年殺其令,相聚數千人,欲立長,無適用,乃請陳嬰。嬰謝不能,遂強立之,縣中從之者得二萬人。欲立嬰為王,異軍蒼頭特起。嬰母謂嬰曰:「自吾為乃家婦,聞先故未曾貴。今暴得大名,不祥。不如有所屬,事成猶得封侯,事敗易以亡,非世所指名也。」嬰乃不敢為王,謂其軍曰:「項氏世世將家,有名於楚,今欲舉大事,將非其人,不可。我倚名族,亡秦必矣。」其眾從之,乃以其兵屬梁。梁渡淮,英布、蒲將軍亦以其兵屬焉。凡六七萬人,軍下邳。

是時,秦嘉已立景駒為楚王,軍彭城東,欲以距梁。梁謂軍吏曰:「陳王首事,戰不利,未聞所在。今秦嘉背陳王立景駒,大逆亡道。」乃引兵擊秦嘉。軍敗走,追至胡陵。嘉還戰一日,嘉死,軍降。景駒走死梁地。梁已并秦嘉軍,胡陵,將引而西。章邯至栗,梁使別將朱雞石、餘樊君與戰。餘樊君死。朱雞石敗,亡走胡陵。梁乃引兵入薛,誅朱雞石。梁前使羽別攻襄城,襄城堅守不下。已拔,皆阬之,還報梁。聞陳王定死,召諸別將會薛計事。時沛公亦從沛往。

居鄛人范增年七十,素好奇計,往說梁曰:「陳勝敗固當。夫秦滅六國,楚最亡罪,自懷王入秦不反,楚人憐之至今,故南公稱曰『楚雖三戶,亡秦必楚』。今陳勝首事,不立楚後,其勢不長。今君起江東,楚蠭起之將皆爭附君者,以君世世楚將,為能復立楚之後也。」於是梁乃求楚懷王孫心,在民間為人牧羊,立以為楚懷王,從民望也。陳嬰為上柱國,封五縣。與懷王都盱台。梁自號武信君,引兵攻亢父。

初,章邯既殺齊王田儋於臨菑,田假復自立為齊王。儋弟榮走保東阿,章邯追圍之。梁引兵救東阿,大破秦軍東阿,田榮即引兵歸,逐王假。假亡走楚,相田角亡走趙。角弟閒,故將,居趙不敢歸。田榮立儋子市為齊王。梁已破東阿下軍,遂追秦軍。數使使趣齊兵俱西。榮曰:「楚殺田假,趙殺田角、田閒,乃發兵。」梁曰:「田假與國之王,窮來歸我,不忍殺。」趙亦不殺角、閒以市於齊。齊遂不肯發兵助楚。梁使羽與沛公別攻城陽,屠之。西破秦軍濮陽東,秦兵收入濮陽。沛公、羽攻定陶。定陶未下,去,西略地至雍丘,大破秦軍,斬李由。還攻外黃,外黃未下。

梁起東阿,比至定陶,再破秦軍,羽等又斬李由,益輕秦,有驕色。宋義諫曰:「戰勝而將驕卒惰者敗。今少惰矣,秦兵日益,臣為君畏之。」梁不聽。乃使宋義於齊。道遇齊使者高陵君顯,曰:「公將見武信君乎?」曰:「然。」義曰:「臣論武信君軍必敗。公徐行則免,疾行則及禍。」秦果悉起兵益章邯,夜銜枚擊楚,大破之定陶,梁死。沛公與羽去外黃,攻陳留,陳留堅守不下。沛公、羽相與謀曰:「今梁軍敗,士卒恐。」乃與呂臣俱引兵而東。呂臣軍彭城東,羽軍彭城西,沛公軍碭。

章邯已破梁軍,則以為楚地兵不足憂,乃渡河北擊趙,大破之。當此之時,趙歇為王,陳餘為將,張耳為相,走入鉅鹿城。秦將王離、涉閒圍鉅鹿,章邯軍其南,築甬道而輸之粟。陳餘將卒數萬人軍鉅鹿北,所謂河北軍也。

宋義所遇齊使者高陵君顯見楚懷王曰:「宋義論武信君必敗,數日果敗。軍未戰先見敗徵,可謂知兵矣。」王召宋義與計事而說之,因以為上將軍;羽為魯公,為次將,范增為末將。諸別將皆屬,號卿子冠軍。北救趙,至安陽,留不進。秦三年,羽謂宋義曰:「今秦軍圍鉅鹿,疾引兵渡河,楚擊其外,趙應其內,破秦軍必矣。」宋義曰:「不然。夫搏牛之虻不可以破蝨。今秦攻趙,戰勝則兵罷,我承其敝;不勝,則我引兵鼓行而西,必舉秦矣。故不如先鬥秦、趙。夫擊輕銳,我不如公;坐運籌策,公不如我。」因下令軍中曰:「猛如虎,佷如羊,貪如狼,強不可令者,皆斬。」遣其子襄相齊,身送之無鹽,飲酒高會。天寒大雨,士卒凍飢。羽曰:「將戮力而攻秦,久留不行。今歲飢民貧,卒食半菽,軍無見糧,乃飲酒高會,不引兵渡河因趙食,與并力擊秦,乃曰『承其敝』。夫以秦之強,攻新造之趙,其勢必舉趙。趙舉秦強,何敝之承!且國兵新破,王坐不安席,掃境內而屬將軍,國家安危,在此一舉。今不卹士卒而徇私宴,非社稷之臣也。」羽晨朝上將軍宋義,即其帳中斬義頭。出令軍中曰:「宋義與齊謀反楚,楚王陰令籍誅之。」諸將讋服,莫敢枝梧。皆曰:「首立楚者,將軍家也。今將軍誅亂。」乃相與共立羽為假上將軍。使人追宋義子,及之齊,殺之。使桓楚報命於王。王因使使立羽為上將軍。

羽已殺卿子冠軍,威震楚國,名聞諸侯。乃遣當陽君、蒲將軍將卒二萬人渡河救鉅鹿。戰少利,陳餘復請兵。羽乃悉引兵渡河。已渡,皆湛舡,破釜甑,燒廬舍,持三日糧,視士必死,無還心。於是至則圍王離,與秦軍遇,九戰,絕甬道,大破之,殺蘇角,虜王離。涉閒不降,自燒殺。當是時,楚兵冠諸侯。諸侯軍救鉅鹿者十餘壁,莫敢縱兵。及楚擊秦,諸侯皆從壁上觀。楚戰士無不一當十,呼聲動天地。諸侯軍人人惴恐。於是楚已破秦軍,羽見諸侯將,入轅門,膝行而前,莫敢仰視。羽繇是始為諸侯上將軍,兵皆屬焉。

章邯軍棘原,羽軍漳南,相持未戰。秦軍數卻,二世使人讓章邯。章邯恐,使長史欣請事。至咸陽,留司馬門三日,趙高不見,有不信之心。長史欣恐,還走,不敢出故道。趙高果使人追之,不及。欣至軍,報曰:「事亡可為者。相國趙高顓國主斷。今戰而勝,高嫉吾功;不勝,不免於死。願將軍熟計之。」陳餘亦遺章邯書曰:「白起為秦將,南并鄢郢,北阬馬服,攻城略地,不可勝計,而卒賜死。蒙恬為秦將,北逐戎人,開榆中地數十里,竟斬陽周。何者?功多,秦不能封,因以法誅之。今將軍為秦將三歲矣,所亡失已十萬數,而諸侯並起茲益多。彼趙高素諛日久,今事急,亦恐二世誅之,故欲以法誅將軍以塞責,使人更代以脫其禍。將軍居外久,多內隙,有功亦誅,亡功亦誅。且天之亡秦,無愚智皆知之。今將軍內不能直諫,外為亡國將,孤立而欲長存,豈不哀哉!將軍何不還兵與諸侯為從,南面稱孤,孰與身伏斧質,妻子為戮乎?」章邯狐疑,陰使候始成使羽,欲約。約未成,羽使蒲將軍引兵渡三戶,軍漳南,與秦戰,再破之。羽悉引兵擊秦軍汙水上,大破之。

邯使使見羽,欲約。羽召軍吏謀曰:「糧少,欲聽其約。」軍吏皆曰:「善。」羽乃與盟洹水南殷虛上。已盟,章邯見羽流涕,為言趙高。羽乃立章邯為雍王,置軍中。使長史欣為上將,將秦軍行前。

漢元年,羽將諸侯兵三十餘萬,行略地至河南,遂西到新安。異時諸侯吏卒徭役屯戍過秦中,秦中遇之多亡狀,及秦軍降諸侯,諸侯吏卒乘勝奴虜使之,輕重折辱秦吏卒。吏卒多竊言:「章將軍詐吾屬降諸侯,今能入關破秦,大善;即不能,諸侯虜吾屬而東,秦又盡誅吾父毌妻子。」諸將微聞其計,以告羽。羽乃召英布、蒲將軍計曰:「秦吏卒尚眾,其心不服,至關不聽,事必危,不如擊之,獨與章邯、長史欣、都尉翳入秦。」於是夜擊阬秦軍二十餘萬人。

至函谷關,有兵守,不得入。聞沛公已屠咸陽,羽大怒,使當陽君擊關。羽遂入,至戲西鴻門,聞沛公欲王關中,獨有秦府庫珍寶。亞父范增亦大怒,勸羽擊沛公。饗士,旦日合戰。羽季父項伯素善張良。良時從沛公,項伯夜以語良。良與俱見沛公,因伯自解於羽。明日,沛公從百餘騎至鴻門謝羽,自陳「封秦府庫,還軍霸上以待大王,閉關以備他盜,不敢背德。」羽意既解,范增欲害沛公,賴張良、樊噲得免。語在高紀。

後數日,羽乃屠咸陽,殺秦降王子嬰,燒其宮室,火三月不滅;收其寶貨,略婦女而東。秦民失望。於是韓生說羽曰:「關中阻山帶河,四塞之地,肥饒,可都以伯。」羽見秦宮室皆已燒殘,又懷思東歸,曰:「富貴不歸故鄉,如衣錦夜行。」韓生曰:「人謂楚人沐猴而冠,果然。」羽聞之,斬韓生。

初,懷王與諸將約,先入關者王其地。羽既背約。使人致命於懷王。懷王曰:「如約。」羽乃曰:「懷王者,吾家武信君所立耳,非有功伐,何以得顓主約?天下初發難,假立諸侯後以伐秦。然身被堅執銳首事,暴露於野三年,滅秦定天下者,皆將相諸君與籍力也。懷王亡功,固當分其地王之。」諸將皆曰:「善。」羽乃陽尊懷王為義帝,曰:「古之王者,地方千里,必居上游。」徙之長沙,都郴。乃分天下以王諸侯。

羽與范增疑沛公,業已講解,又惡背約,恐諸侯叛之,陰謀曰:「巴、蜀道險,秦之遷民皆居之。」乃曰:「巴、蜀亦關中地。」故立沛公為漢王,王巴、蜀、漢中。而參分關中,王秦降將以距塞漢道。乃立章邯為雍王,王咸陽以西。長史司馬欣,故櫟陽獄吏,嘗有德於梁;都尉董翳,本勸章邯降。故立欣為塞王,王咸陽以東至河;立翳為翟王,王上郡。徙魏王豹為西魏王,王河東。瑕丘公申陽者,張耳嬖臣也,先下河南,迎楚河上。立陽為河南王。趙將司馬卬定河內,數有功。立卬為殷王,王河內。徙趙王歇王代。趙相張耳素賢,又從入關,立為常山王,王趙地。當陽君英布為楚將,常冠軍。立布為九江王。番君吳芮帥百粵佐諸侯從入關。立芮為衡山王。義帝柱國共敖將兵擊南郡,功多,因立為臨江王。徙燕王韓廣為遼東王。燕將臧荼從楚救趙,因從入關。立荼為燕王。徙齊王田市為膠東王。齊將田都從共救趙,入關。立都為齊王。故秦所滅齊王建孫田安,羽方渡河救趙,安下濟北數城,引兵降羽。立安為濟北王。田榮者,背梁不肯助楚擊秦,以故不得封。陳餘棄將印去,不從入關,然素聞其賢,有功於趙,聞其在南皮,故因環封之三縣。番君將梅鋗功多,故封十萬戶侯。羽自立為西楚伯王,王梁楚地九郡,都彭城。

諸侯各就國。田榮聞羽徙齊王市膠東,而立田都為齊王,大怒,不肯遣市之膠東,因以齊反,迎擊都。都走楚。市畏羽,乃亡之膠東就國。榮怒,追殺之即墨,自立為齊王。予彭越將軍印,今反梁地。越乃擊殺濟北王田安。田榮遂并王三齊之地。時漢王還定三秦。羽聞漢并關中,且東,齊、梁畔之,大怒,乃以故吳令鄭昌為韓王以距漢,令蕭公角等擊彭越。越敗蕭公角等。時,張良徇韓,遺項王書曰:「漢王失職,欲得關中,如約即止,不敢東。」又以齊、梁反書遺羽,羽以此故無西意,而北擊齊。徵兵九江王布。布稱疾不行,使將將數千人往。二年,羽陰使九江王布殺義帝。陳餘使張同、夏說說齊王榮,曰:「項王為天下宰不平,今盡王故王於醜地,而王群臣諸將善地,逐其故主趙王,乃北居代,餘以為不可。聞大王起兵,且不聽不義,願大王資餘兵,使擊常山,以復趙王,請以國為扞蔽。」齊王許之,因遣兵往。陳餘悉三縣兵,與齊併力擊常山,大破之。張耳走歸漢。陳餘迎故趙王歇反之趙。趙王因立餘為代王。羽至城陽,田榮亦將兵會戰。榮不勝,走至平原,平原民殺之。羽遂北燒夷齊城郭室屋,皆阬降卒,係虜老弱婦女。徇齊至北海,所過殘滅。齊人相聚而畔之。於是田榮弟橫收得亡卒數萬人,反城陽。羽因留,連戰未能下。

漢王劫五諸侯兵,凡五十六萬人,東伐楚。羽聞之,即令諸將擊齊,而自以精兵三萬人南從魯出胡陵。漢王皆已破彭城,收其貨賂美人,日置酒高會。羽乃從蕭晨擊漢軍而東,至彭城,日中,大破漢軍。漢軍皆走,迫之穀、泗水。漢軍皆南走山,楚又追擊至靈辟東睢水上。漢軍卻,為楚所擠,多殺。漢卒十餘萬皆入睢水,睢水為不流。漢王乃與數十騎遁去。語在高紀。太公、呂后間求漢王,反遇楚軍。楚軍與歸,羽常置軍中。

漢王稍收散卒,蕭何亦發關中卒悉詣滎陽,戰京、索間,敗楚。楚以故不能過滎陽而西。漢軍滎陽,築甬道,取敖倉食。三年,羽數擊絕漢甬道,漢王食乏,請和,割滎陽以西為漢。羽欲聽之。歷陽侯范增曰:「漢易與耳,今不取,後必悔之。」羽乃急圍滎陽。漢王患之,乃與陳平金四萬斤以間楚君臣。語在陳平傳。項羽以故疑范增,稍奪之權。范增怒曰:「天下事大定矣,君王自為之!願賜骸骨歸。」行未至彭城,疽發背死。於是漢將紀信詐為漢王出降,以誑楚軍,故漢王得與數十騎從西門出。令周苛、樅公、魏豹守滎陽。漢王西入關收兵,還出宛、葉間,與九江王黥布行收兵。羽聞之,即引兵南。漢王堅壁不與戰。

是時,彭越渡睢,與項聲、薛公戰下邳,殺薛公。羽乃東擊彭越。漢王亦引兵北軍成皋。羽已破走彭越,引兵西下滎陽城,亨周苛,殺樅公,虜韓王信,進圍成皋。漢王跳,獨與滕公得出。北渡河,至修武,從張耳、韓信。楚遂拔成皋。漢王得韓信軍,留止,使盧綰、劉賈渡白馬津入楚地,佐彭越共擊破楚軍燕郭西,燒其積聚,攻下梁地十餘城。羽聞之,謂海春侯大司馬曹咎曰:「

謹守成皋。即漢欲挑戰,慎毋與戰,勿令得東而已。我十五日必定梁地,復從將軍。」於是引兵東。

四年,羽擊陳留、外黃,外黃不下。數日降,羽悉令男子年十五以上詣城東,欲阬之。外黃令舍人兒年十三,往說羽曰:「彭越強劫外黃,外黃恐,故且降,待大王。大王至,又皆阬之,百姓豈有所歸心哉!從此以東,梁地十餘城皆恐,莫肯下矣。」羽然其言,乃赦外黃當阬者。而東至睢陽,聞之皆爭下。

漢果數挑楚軍戰,楚軍不出。使人辱之,五六日,大司馬怒,渡兵氾水。卒半渡,漢擊,大破之,盡得楚國金玉貨賂。大司馬咎、長史欣皆自剄氾水上。咎故蘄獄掾,欣故塞王,羽信任之。羽至睢陽,聞咎等破,則引兵還,漢軍方圍鍾離昧於滎陽東,羽軍至,漢軍畏楚,盡走險阻。羽亦軍廣武相守,乃為高俎,置太公其上,告漢王曰:「今不急下,吾亨太公。」漢王曰:「吾與若俱北面受命懷王,約為兄弟,吾翁即汝翁。必欲亨乃翁,幸分我一盃羹。」羽怒,欲殺之。項伯曰:「天下事未可知。且為天下者不顧家,雖殺之無益,但益怨耳。」羽從之。乃使人謂漢王曰:「天下匈匈,徒以吾兩人願與王挑戰,決雌雄,毋徒罷天下父子為也。」漢王笑謝曰:「吾寧鬥智,不能鬥力。」羽令壯士出挑戰。漢有善騎射曰樓煩,楚挑戰,三合,樓煩輒射殺之。羽大怒,自被甲持戟挑戰。樓煩欲射,羽瞋目叱之。樓煩目不能視,手不能發,走還入壁,不敢復出。漢王使間問之,乃羽也。漢王大驚。於是羽與漢王相與臨廣武間而語。漢王數羽十罪。語在高紀。羽怒,伏弩射傷漢王。漢王入成皋。

時彭越數反梁地,絕楚糧食,又韓信破齊,且欲擊楚。羽使從兄子項它為大將,龍且為裨將,救齊。韓信破殺龍且,追至成陽,虜齊王廣。信遂自立為齊王。羽聞之,恐,使武涉往說信。語在信傳。

時,漢關中兵益出,食多,羽兵食少。漢王使侯公說羽,羽乃與漢王約,中分天下,割鴻溝而西者為漢,東者為楚,歸漢王父母妻子。已約,羽解而東。五年,漢王進兵追羽,至故陵,復為羽所敗。漢王用張良計,致齊王信、建成侯、彭越兵,乃劉賈入楚地,圍壽春。大司馬周殷叛楚,舉九江兵隨劉賈,迎黥布,與齊梁諸侯皆大會。

羽壁垓下,軍少食盡。漢帥諸侯兵圍之數重。羽夜聞漢軍四面皆楚歌,乃驚曰:「漢皆已得楚乎?是何楚人多也!」起飲帳中。有美人姓虞氏,常幸從;駿馬名騅,常騎。乃悲歌忼慨,自為歌詩曰:「力拔山兮氣蓋世,時不利兮騅不逝。騅不逝兮可柰何!虞兮虞兮柰若何!」歌數曲,美人和之。羽泣下數行,左右皆泣,莫能仰視。

於是羽遂上馬,戲下騎從者八百餘人,夜直潰圍南出馳。平明,漢軍乃覺之,令騎將灌嬰以五千騎追羽。羽渡淮,騎能屬者百餘人。羽至陰陵,迷失道,問一田父,田父紿曰「左」。左,乃陷大澤中,以故漢追及之。羽復引而東,至東城,乃有二十八騎。追者數千,羽自度不得脫,謂其騎曰:「吾起兵至今八歲矣,身七十餘戰,所當者破,所擊者服,未嘗敗北,遂伯有天下。然今卒困於此,此天亡我,非戰之罪也。今日固決死,願為諸軍快戰,必三勝,斬將,艾旗,乃後死,使諸君知我非用兵罪,天亡我也。」於是引其騎因四隤山而為圜陳外嚮。漢騎圍之數重。羽謂其騎曰:「吾為公取彼一將。」令四面騎馳下,期山東為三處。於是羽大呼馳下,漢軍皆披靡。遂殺漢一將。是時,楊喜為郎騎,追羽,羽還叱之,喜人馬俱驚,辟易數里。與其騎會三處。漢軍不知羽所居,分軍為三,復圍之。羽乃馳,復斬漢一都尉,殺數十百人。復聚其騎,亡兩騎。乃謂騎曰:「何如?」騎皆服曰:「如大王言。」

於是羽遂引東,欲渡烏江。烏江亭長檥船待,謂羽曰:「江東雖小,地方千里,眾數十萬,亦足王也。願大王急渡。今獨臣有船,漢軍至,亡以渡。」羽笑曰:「乃天亡我,何渡為!且籍與江東子弟八千人渡而西,今亡一人還,縱江東父兄憐而王我,我何面目見之哉?縱彼不言,籍獨不愧於心乎!」謂亭長曰:「吾知公長者也,吾騎此馬五歲,所當亡敵,嘗一日千里,吾不忍殺,以賜公。」乃令騎皆去馬,步持短兵接戰。羽獨所殺漢軍數百人。羽亦被十餘創。顧見漢騎司馬呂馬童曰:「若非吾故人乎?」馬童面之,指王翳曰:「此項王也。」羽乃曰:「吾聞漢購我頭千金,邑萬戶,吾為公得。」乃自剄。王翳取其頭,亂相輮蹈爭羽相殺者數十人。最後楊喜、呂馬童、郎中呂勝、楊武各得其一體。故分其地以封五人,皆為列侯。

漢王乃以魯公號葬羽於穀城。諸項支屬皆不誅。封項伯等四人為列侯,賜姓劉氏。

贊曰:昔賈生之過秦曰:

秦孝公據殽函之固,擁雍州之地,君臣固守而闚周室,有席卷天下,包舉宇內,囊括四海,并吞八荒之心。當是時也,商君佐之,內立法度,務耕織,修守戰之備,外連衡而鬥諸侯。於是秦人拱手而取西河之外。

孝公既沒,惠文、武、昭襄蒙故業,因遺策,南取漢中,西舉巴蜀,東割膏腴之地,收要害之郡。諸侯恐懼,會盟而謀弱秦,不愛珍器重寶肥饒之地,以致天下之士。合從締交,相與為一。當此之時,齊有孟嘗,趙有平原,楚有春申,魏有信陵。此四賢者,皆明智而忠信,寬厚而愛人,尊賢重士,約從離橫,兼韓、魏、燕、趙、宋、衛、中山之眾。於是六國之士有甯越、徐尚、蘇秦、杜赫之屬為之謀,齊明、周最、陳軫、召滑、樓緩、翟景、蘇厲、樂毅之徒通其意,吳起、孫臏、帶他、兒良、王廖、田忌、廉頗、趙奢之朋制其兵。常以十倍之地,百萬之軍,仰關而攻秦。秦人開關延敵,九國之師遁巡而不敢進。秦無亡矢遺鏃之費,而天下已困矣。於是從散約敗,爭割地而賂秦。秦有餘力而制其弊,追亡逐北,伏尸百萬,流血漂鹵,因利乘便,宰割天下,分裂山河;強國請服,弱國入朝。

施及孝文、莊襄王,享國之日淺,國家亡事。

及至始皇,奮六世之餘烈,振長策而馭宇內,吞二周而亡諸侯,履至尊而制六合,執敲扑以鞭笞天下,威震四海。南取百粵之地,以為桂林、象郡。百粵之君頫首係頸,委命下吏。乃使蒙恬北築長城而守藩籬,卻匈奴七百餘里,胡人不敢南下而牧馬,士不敢彎弓而報怨。於是廢先王之道,焚百家之言,以愚黔首。墮名城,殺豪俊,收天下之兵聚之咸陽,銷鋒鍉鑄以為金人十二,以弱天下之民。然後踐華為城,因河為池,據億丈之城,臨不測之川,以為固。良將勁弩,守要害之處,信臣精卒,陳利兵而誰何。天下已定,始皇之心,自以為關中之固,金城千里,子孫帝王萬世之業也。

始皇既沒,餘威震于殊俗。然而陳涉,甕牖繩樞之子,甿隸之人,遷徙之徒也,材能不及中庸,非有仲尼、墨翟之知,陶朱、猗頓之富。躡足行伍之間,而免起阡陌之中,帥罷散之卒,將數百之眾,轉而攻秦。斬木為兵,揭竿為旗,天下雲合嚮應,贏糧而景從,山東豪俊遂並起而亡秦族矣。

且天下非小弱也;雍州之地,殽函之固,自若也。陳涉之位,不齒於齊、楚、燕、趙、韓、魏、宋、衛、中山之君;鉏耰棘矜,不敵於鉤戟長鎩;適戍之眾,不亢於九國之師;深謀遠慮,行軍用兵之道,非及曩時之士也。然而成敗異變,功業相反,何也?試使山東之國與陳涉度長絜大,比權量力,不可同年而語矣。然秦以區區之地,致萬乘之權,招八州而朝同列,百有餘年,然后以六合為家,殽函為宮。一夫作難而七廟墮,身死人手,為天下笑者,何也?仁誼不施,而攻守之勢異也。

周生亦有言,「舜蓋重童子」,項羽又重童子,豈其苗裔邪?何其興之暴也!夫秦失其政,陳涉首難,豪桀蜂起,相與並爭,不可勝數。然羽非有尺寸,乘勢拔起隴畝之中,三年,遂將五諸侯兵滅秦,分裂天下而威海內,封立王侯,政繇羽出,號為「伯王」,位雖不終,近古以來未嘗有也。及羽背關懷楚,放逐義帝,而怨王侯畔己,難矣。自矜功伐,奮其私智而不師古,始霸王之國,欲以力征經營天下,五年卒亡其國,身死東城,尚不覺寤,不自責過失,乃引「天亡我,非用兵之罪」,豈不謬哉!


\end{pinyinscope}