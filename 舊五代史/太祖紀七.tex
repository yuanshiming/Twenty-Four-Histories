\article{太祖紀七}

\begin{pinyinscope}

 乾化二年正月,宣:「上元夜,任諸市及坊市各點彩燈,金吾不用禁夜。」近年以來,以都下聚兵太廣,未嘗令坊市點燈故也。甲申,以時雪久愆,命丞相及三省官群望祈
 禱。詔曰:「謗木求規,集囊貢事,將裨理道,豈限側言。應內外文武百官及草澤,並許上封事,極言得失。」以丁審衢為陳州,而審衢厚以鞍馬、金帛為謝恩之獻,帝慮其漁民,復其獻而停之。封保義節度使王檀為琅琊郡王。命供奉官朱嶠于河南府宣取先收禁定州進奉官崔騰並傔從一十四人,並釋放,仍命押領送至貝。騰,唐戶部侍郎潔之子也。廣明喪亂,客于北諸侯,為定州節度使王處存所辟,去載領貢獻至闕。未幾,其帥稱兵,遂縶之。
 至是,帝念賓介之來,又已出境,特命縱而歸焉。丙戌,有司以孟春太廟薦享上言,命丞相杜曉攝祭行事。丙申夕,熒惑犯房第二星。


二月庚戌,中和節,御崇勳殿,召丞相、大學士、河南尹,略封訖,于萬春門外廡賜以酒食。
 \gezhu{
  《五代會要》,二月,追封故魏博節度使羅宏信為趙王。}
 癸丑,敕曰:「今載春寒頗甚,雨澤仍愆,司天監占以夏秋必多霖潦,宜令所在郡縣告喻百姓,備淫雨之患。」庚申,御宣威殿開宴,丞相洎文武官屬咸被召列侍,竟日而罷。壬戌,帝將巡按北境,中外戒
 嚴,詔以河南尹、守中書令、判六軍事張宗奭為大內留守。中書門下奏,差定文武官領務尤切宜扈駕者三十八人。詔工部尚書李皎、左散騎常侍孫騭、右諫議大夫張衍、兵部侍郎劉邈、兵部郎中張俊、光祿少卿盧秉彞並令扈蹕。甲子,發自洛師,夕次河陽。
 \gezhu{
  《通鑒》云:至白馬屯,賜從官食,多未至,遣騎趣之于路。左散騎常侍孫騭、右諫議大夫張衍、兵部郎中張俊最後至,帝命撲殺之。}
 乙丑,次溫縣。丙寅,次武陟。懷州刺史段明遠迎拜于境上,其內外所備,咸豐霈焉。丁卯,次獲嘉。戊辰,次衛州之新鄉。己巳,
 晨發衛州,夕止淇門,內衙十將使以十指揮兵士至於行在。辛未,駐蹕黎陽。癸酉,發自黎陽,夕次內黃。甲戌,次昌樂縣。丁丑,次于永濟縣。青州節度使賀德倫奏,統領兵士赴歷亭軍前。戊寅,至貝州,命四丞相及學士李琪、盧文度、知制誥竇賞等十五人扈從,其左常侍韋戩等二十三人止焉。己卯,發自貝州,夕駐蹕于野落。


三月庚辰朔,次于棗強縣之西原。
 \gezhu{
  《通鑑》:辛巳,至下博南,登觀津塚。趙將符暕引數百騎巡邏,不知是帝,遽前逼之。或告曰:「晉兵大至矣!」帝行幄,亟引兵趣棗彊,與楊師厚軍合。}
 丙戌,鎮、定
 諸軍招討使楊師厚奏下棗彊縣,車駕即日疾馳南還。丁亥,復至貝州。庚寅,楊師厚與副招討李周彝等準詔來朝。辛卯,詔丞相、翰林六學士、文武從官、都招討使及諸軍統指揮使等,賜食于行殿。壬辰,命以羊酒等各賜從官。甲午,幸貝州之東闉閱武。乙未,帝復幸東闉閱騎軍。敕以攻下棗彊縣有功將校杜暉等一十一人,並超加檢校官,衙官宋彥等二十五人並超授軍職。丙午,次濟源縣。詔曰:「淑律將遷,亢陽頗甚,宜令魏州差官祈禱
 龍潭。」戊申,詔曰:「雨澤愆期,祈禱未應,宜令宰臣各于魏州靈祠精加祈禱。」
 \gezhu{
  《五代會要》:三月,詔曰:「夫隆興邦國,必本于人民;惠養疲羸,凡資于令長。茍選求之踰濫,固撫理之乖違。如聞吏部擬官,中書除授,或緣親舊所請,或為勢要所干,姑徇私情,靡求才實,念茲蠹弊,宜舉條章。今後應中書用人及吏部注擬,並宜省籓身之才業,驗為政之否臧,必有可觀,方可任用。如或尚行請說,猶假貨財,其所司人吏,必當推窮,重加懲斷。」}


四月己酉,幸魏州。金波亭,賜宴宰臣、文武官及六學士。甲寅夕,月掩心大星。丙辰,敕:「近者星辰違度,式在修禳,宜令兩京及宋州、魏州取此月至五月禁斷屠宰。仍各于佛寺開建道場,以迎福
 應。」己未,次黎陽縣。
 \gezhu{
  《通鑒》:乙卯,博王友文來朝,請帝還東都。丁巳,發魏州。己未,至黎陽,以疾淹留。}
 東都留守官吏奉表起居,賜丞相、從官酒食有差。己巳,至東都,博王友文以新創食殿上言,并進準備內宴錢三千貫、銀器一千五百兩。辛未,宴于食殿,召丞相及文武從官等侍焉。帝泛九曲池,御舟傾,帝墮溺于池中,宮女侍官扶持登岸,驚悸久之。制加建昌宮使、金紫光祿大夫、檢校司徒、開封尹、博王友文為特進、檢校太保,兼開封尹,依前建昌宮使,充東都留守。戊寅,車駕發自
 東京,夕次中牟縣。


五月己卯朔,從官文武自丞相而下,並詣行殿起居,親王及諸道籓帥咸奉表來上。庚辰,發自鄭州,至滎陽,河南尹魏王宗奭望塵迎拜;河陽留後邵贊、懷州刺史段明遠等邐迤來迎。夕次汜水縣,帝召魏王宗奭入對,便于御前賜食,數刻乃退。壬午,駐蹕于汜水,宰臣、河南尹、六學士並于內殿起居,敕以建昌宮事委宰臣于兢領之。《五代會要》:其年六月,廢建昌宮,以河南尹、魏王張宗奭為國計使,凡天下金穀兵戎舊隸建昌宮者,悉主之。癸未,帝發自汜水,宣令邵贊、段明遠各歸所理。午憩任
 村屯,夕次孝義宮。留都文武禮部尚書孔續而下道左迎拜。次偃師。甲申,至都,文武臣奉迎于東郊。渤海遣使朝貢。宰臣薛貽矩抱恙在假,不克扈從,宣問旁午,仍命且駐東京以俟良愈。及薨,帝震悼頗久,命雒苑使曹守璫往弔祭之,又命輟六日、七日、八日朝參,丞相、文武並詣上閣門進名奉慰。丁亥,以彗星謫見,詔兩京見禁囚徒大辟罪以下,遞減一等,限三日內疏理訖聞奏。
 \gezhu{
  《五代會要》:彗星見于靈臺之西,至五月始降赦宥罪,以答天譴。又云:五月壬戌夜,熒惑犯心大星,去心四度,順行。司天奏:「大星為帝王之星,宜修省以答天譴。」}
 詔曰:「生育之人,爰當暑月,乳哺
 之愛,方及薰風。儻肆意于刲屠,豈推恩于長養,俾無殄暴,以助發生。宜令兩京及諸州府,夏季內禁斷屠宰及採捕。天民之窮,諒由賦分;國章所在,亦務興仁。所在鰥寡孤獨、廢疾不濟者,委長吏量加賑恤。史載葬枯,用彰軫恤;禮稱掩骼,將致和平。應兵戈之地,有暴露骸骨,委所在長吏差人專攻收瘞。國癘之文,尚標七祀;良藥之市,亦載三醫。用憐無告之人,宜徵有喜之術。凡有疫之處,委長吏檢尋醫方,于要路曉示。如有家無骨肉兼困
 窮不濟者,即仰長吏差醫給藥救療之。辛卯,詔曰:「亢陽滋甚,農事已傷,宜令宰臣于兢赴中嶽,杜曉赴西嶽,精切祈禱。其近京靈廟,宜委河南尹,五帝壇、風師雨師、九宮貴神,委中書各差官祈之。」《通鑒》:閏月壬戌,帝疾甚,謂近臣曰:「我經營天下三十年,不意太原餘孽更昌熾如此!吾觀其志不小,天復奪我年,我死,諸兒非彼敵也,吾無葬地矣!」因哽咽,絕而復蘇。帝長子郴王友裕早卒。次假子友文,帝特愛之,常留守東都,兼建昌宮使。次郢王友珪,其母亳州營倡也,為左右控鶴都指揮使。次均王友貞,為東都馬步都指揮使。帝雖未以友文為太子,意常
 屬之。六月丁丑朔,帝命敬翔出友珪為萊州刺史,即命之官。已宣旨,未行敕。時左遷者多追賜死,友珪益恐。戊寅,友珪易服微行入左龍虎軍,見統軍韓勍,以情告之。勍亦見功臣宿將多以小過被誅,懼不自保,遂相與合謀。勍以牙兵五百人從友珪雜控鶴士入,伏于禁中;夜斬關入,至寢殿,侍疾者皆散走。帝驚起,問:「反者為誰?」友珪曰:「非他人也。」帝曰:「我固疑此賊,恨不早殺之。汝悖逆如此,天地豈容汝乎!」友珪曰:「老賊萬段!」友珪僕夫馮廷諤刺帝腹,刃出于背。友珪自以敗氈裹之,瘞于寢殿,秘不發喪。遣供奉官丁昭溥馳詣東都,命均王友貞殺友文。己卯,矯詔稱:「博王友文謀逆,遣兵突入殿中,賴郢王友珮忠孝,將兵誅之,保
 全朕躬。然疾因震驚,彌致危殆,宜令友珪權主軍國之務。」韓勍為友珪謀,多出府庫金帛,賜諸軍及百官以取悅。辛巳,丁昭溥還,聞友文已死,乃發喪,宣遺制,友珪即皇帝位。友珪葬太祖于伊闕縣,號宣陵。《五代史補》:太祖朱全忠,黃巢之先鋒。巢入長安,以刺史王鐸圍同州,太祖遂降,鐸承制拜同州刺史。黃巢滅,淮、蔡間秦宗權復盛,朝廷以淮、蔡與汴州相接,太祖汴人,必究其能否,遂移授宣武軍節度使以討宗權,未凡滅之。自是威福由己,朝廷不能制,遂有天下。先是,民間傳讖曰:「五公符」,又謂之「李淳風轉天歌」,其字有「八牛之年」,識者以「八牛」乃「硃」字,則太祖革命之應焉。太祖之用兵也,法令嚴峻,每戰,逐隊主帥或有沒而不反者,其餘皆斬之,謂之:「跋隊斬」。自是戰無不勝。然健兒且多竄匿州郡,疲于追捕,因下令
 文面,健兒文面自此始也。《五代史闕文》:世傳梁太祖迎昭宗於鳳翔,素服待罪,昭宗佯為鞋系脫,呼梁祖曰:「全忠為吾系鞋。」梁祖不得已,跪而結之,汗流浹背。時天子扈蹕尚有衛兵,昭宗意謂左右擒梁祖以殺之,其如無敢動者。自是梁祖被召多不至,盡去昭宗禁衛,皆用汴人矣。臣謹案:梁祖以天復三年迎唐昭宗於岐下,歲在甲子,其年改天祐,至國初建隆庚申歲,才五十六年矣,然則乾德七十歲人皆目睹其事。蓋唐室自懿宗失政,天下亂離,故武宗以下實錄,不傳於世。昭宗一朝,全無記注。梁祖在位止及六年,均帝朝詔史臣修梁祖實錄,岐下系鞋之事,恥而不書。晉天福中,史臣張昭重修《唐史》,始有《昭宗本紀》,但云即位之始,有《會昌》之風,岐陽事跡,不能追補。此亦明唐昭宗有英睿之氣,而衰運不振;又明左右無忠義奮發之臣,致梁祖得行其志。有所警誡,不可不書。



\end{pinyinscope}