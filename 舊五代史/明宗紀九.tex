\article{明宗紀九}

\begin{pinyinscope}

 長興三年春正月癸未朔,帝御明堂殿受朝賀,仗衛如式。丁亥,陜州節度使安從進移鎮延州。己丑,遣邠州節度使藥彥稠、靈
 武節度使康福率步騎七千往方渠討黨項之叛者。庚寅,以前北京副留守呂夢奇為戶部侍郎。辛卯,以
 前彰
 國
 軍留後孫漢韶為利州節度使,充西面行營副部署兼步軍都指揮使。庚子,契丹遣使朝貢。辛丑,秦王從榮加開府儀同三司、兼中書令。戊申,詔選人文解不合式樣,罪在發解官吏,舉人落第,次年免取文解。中書門下奏:「請親王官至兼侍中、中書令,則與見任宰臣分班定位,宰臣居左,諸親王居右。如親王及諸使守侍中、中書令,亦分行居右,其餘使相依舊。」從之。渤
 海、回鶻、吐蕃遣使朝貢。大理正張居琭上言:「所頒諸州新定格式、律令,請委逐處各差法直官一人,專掌檢討。」從之。



 二月乙卯,制晉國夫人夏氏追冊為皇后。丙辰,幸龍門。詔故皇城使李從璨可贈太保。詔出選門官,罷任後周年方許擬議,自于所司投狀磨勘送中書。又詔罷城南稻田務,以其所費多而所收少,欲復其水利,資於民間碾磑故也。秦州奏:「州界三縣之外,別有一十一鎮人戶,係鎮將徵科,欲隨其便,宜復置隴城、天水二縣以
 隸之。」詔從之。甲子,幸至德宮。以右衛大將軍高居貞為右監門衛上將軍。庚午,以前華州節度使李從昶為左驍衛大將軍,以前夔州節度使安崇阮為右驍衛大將軍,以前新州節度使翟璋為右領軍上將軍,以右領軍上將軍羅周敬為右威衛上將軍。辛未,中書奏:「請依石經文字刻《九經》印板。」從之。《五代會要》:長興三年二月,中書門下奏:「請依石經文字刻《九經》印板,敕令國子監集博士儒徒,將西京石經本,各以所業本經,廣為抄寫,仔細看讀,然後雇召能雕字匠人,各部隨帙刻印板,廣頒天下。如諸色人要寫經書,並請依所印刻本,不得更使雜本交錯。」《愛日齋叢鈔》
 云:《通鑑》載:「後唐長興三年二月辛未,初令國子監校定《九經》,雕印賣之。」又曰:「自唐末以來,所在學校廢絕,蜀毋昭裔出私財百萬營學館,且請板刻《九經》,蜀主從之。由是蜀中文學復盛。」又曰:「唐明宗之世,宰相馮道、李愚請令判國子監田敏校定《九經》,刻板印賣,從之。後周廣順三年六月丁巳,板成,獻之。由是雖亂世,《九經》傳布甚廣。王仲言《揮塵錄》云:毋昭裔貧賤時,嘗借《文選》于交游間,其人有難色,發憤,異日若貴,當板以鏤之遺學者。後仕王蜀為宰相,遂踐其言,刊之,印行書籍,創見於此。事載陶岳《五代史補》。後唐平蜀,明宗命太學博士李鍔書《五經》,仿其制作,刊板于國子監,為監中刻書之始。《猗覺寮雜記》云:雕印文字,唐以前無之,唐末,益州始有墨板,後唐方鏤《九經》,悉收人間所有經史,以鏤板為正。見《兩朝國史》。此則印書已始自唐末矣。案《柳氏家訓》序:中和三年癸卯夏,鑾輿在蜀之三年也,餘為中書舍人,旬
 休,閱書於重城之東南,其書多陰陽雜記、占夢相宅、九宮五緯之流。又有字書小學,率雕板,印紙浸染,不可盡曉。葉氏《燕語》正以此證刻書不始於馮道,而沈存中又謂板印書籍,唐人尚未盛行為之,自馮瀛王始印《五經》,自後典籍皆為板本。大概唐末漸有印書,特未盛行,後人遂以為始于蜀也。當五季亂離之際,經籍方有托而流布於四方,天之不絕斯文,信矣。甲戌,靈武奏,都指揮使許審環等謀亂伏誅。藥彥稠奏,誅黨項阿埋等十族,與康福入白魚谷追襲叛黨,獲大首領六人、諸羌二千餘人、孳畜數千,及先劫掠到回鶻物貨。詔彥稠軍士,所獲並令自收,勿得箕斂。己卯,以前河中節度使索自通為鄜州節度使。懷化軍節度使李贊華進契
 丹地圖。詔司天臺,除密奏留中外,應奏歷象、雲物、水旱,及十曜細行、諸州災祥,一一並報史館,以備編修。壬午,藥彥稠進回鶻可汗先送秦王金裝胡錄,為黨項所掠,至是得之以獻。帝曰:「先詔所獲令軍士自收,今何進也?」令彥稠卻與獲者。



 三月甲申,契丹遣使朝貢。靈武軍將裴昭隱等二人與進奏官阮順之隱官馬一匹,有司論罪合抵法,帝曰:「不可以一馬殺三人命。」笞而釋之。丙申,西京奏,百姓侯可洪于楊廣城內掘得宿藏玉四團進
 納。賜可洪二百緡、絹二百匹。庚子,以前鄜州節度使孫璋卒廢朝。癸卯,帝顧謂宰臣曰:「春雨稍多,久未晴霽,何也?」馮道對曰:「水旱作沴,雖是天之常道,然季春行秋令,臣之罪也。更望陛下廣敷恩宥,久雨無妨于聖政也。」丁未,以神捷、神威、雄武、廣捷已下指揮改為左右羽林軍,置四十指揮,每十指揮立為一軍,軍置都指揮使一人。庚戌,帝觀稼於近郊。民有父子三人同挽犁耕者,帝閔之,賜耕牛三頭。高麗國遣使朝貢。以右領軍上將軍翟
 璋為右羽林統軍,以前安州留後周知裕為左神武統軍。



 夏四月甲寅,詔諸道節度使未帶使相及防禦、團練使、刺史,班位居檢校官高者上為,加檢校官同,以先授者為上,前資在見任之下。新羅王金溥遣使貢方物。戊午,中書奏:「準敕重定三京、諸州府地望次第者。舊制以王者所都之地為上,今都洛陽,請以河南道為上,關內道為第二,河東道為第三,餘依舊制。其五府,按《十道圖》,以鳳翔為首,河中、成都、江陵、興元為次。中興初,升魏州
 為興唐府,鎮州為真定府,望升二府在五府之上,合為七州,餘依舊制。又天下舊有八大都督府,以靈州為首,陜、幽、魏、揚、潞、鎮、徐為次,其魏、鎮已升為七府兼具員內,相次升越、杭、福、潭等州為都督,望以十大都督府為額,仍據升降次第,以陜為首,餘依舊制。《十道圖》有大都護,請以安東大都護為首。防禦、團練等使,自來升降極多,今具見在,其員依新定《十道圖》以次第為定。」從之。契丹累遣使求歸扎拉、特哩袞等,幽州趙德鈞奏請不俞允。
 帝顧問侍臣,亦以為不可與。帝意欲歸之,會冀州刺史楊檀罷郡至闕,帝問其事,奏曰:「此輩來援王都,謀危社稷,陛下寬慈,貸其生命。茍若歸之,必復向南放箭,既知中國事情,為患深矣。」帝然之。既而遣哲爾格錫里隨使歸蕃,不欲全拒其請也。詔贈皇后曹氏曾祖父母已下為太傅、太尉、太師、國夫人,淑妃王氏曾祖父母已下為太子太保、太傅、太師、國夫人。壬戌,前樞密使、驃騎大將軍馬紹宏卒。癸亥,以懷化軍節度使李贊華為滑州節
 度使。初,帝欲以贊華為籓鎮,范延光等奏,以為不可。帝曰:「吾與其先人約為兄弟,故贊華來附。吾老矣,儻後世有守文之主,則此輩招之亦不來矣。」由是近臣不能抗議。甲子,以太子賓客蕭遽為戶部尚書致仕。乙丑,以天雄軍節度使、宋王從厚兼中書令。辛未,以幽州節度使趙德鈞兼中書令。



 五月壬午朔,帝御文明殿受朝。詔禁網羅、彈射、弋獵。丁亥,以二王后前詹事府司直楊延紹為右贊善大夫,仍襲封酅國公,食邑二千戶。丁酉,以太
 子太師致仕孔勍卒廢朝。興元奏,東、西兩川各舉兵相持。甲辰,以文宣王四十三代孫曲阜縣主簿孔仁玉為兗州龔邱令,襲文宣公。戊申,襄州奏,漢江大漲,水入州城,壞民廬舍。樞密使奏:「近知兩川交惡,如令一賊兼有兩川,撫眾守險,恐難討除,欲令王思同以興元之師伺便進取。」詔從之。



 六月壬子朔,幽州趙德鈞奏:「新開東南河,自王馬口至淤口,長一百六十五里,闊六十五步,深一丈二尺,以通漕運,舟勝千石,畫圖以獻。」甲寅,以權知
 高麗國事王建為檢校太保,封高麗國王。丁巳,衛州奏,河水壞堤,東北流入御河。戊午,荊南奏:「東川董璋領兵至漢州,西川孟知祥出兵逆戰,璋大敗,得部下人二十餘,走入東川城,尋為前陵州刺史王暉所殺,孟知祥已入梓州。」辛酉,范延光奏曰:「孟知祥兼有兩川,彼之軍眾皆我之將士,料其外假朝廷形勢以制之,然陛下茍不能屈意招攜,彼亦無由革面。」帝曰:「知祥予故人也,以賊臣間諜,故茲阻隔,今因而撫之,何屈意之有!」由是遣供
 奉官李瑰使西川,齎詔以賜知祥。詔以霖雨積旬,久未晴霽,京城諸司繫囚,並宜釋放。甲子,以大雨未止,放朝參兩日。洛水漲泛二丈,廬舍居民有溺死者。以前濮州刺史武延翰為右領軍上將軍,前階州刺史王宏贄為左千牛上將軍。金、徐、安、潁等州大水,鎮州旱。詔應水旱州郡,各遣使人存問。



 秋七月辛巳朔,以天下兵馬元帥、尚父、吳越國王錢鏐薨,廢朝三日。丙戌,詔賜諸軍救接錢有差。戊子,正衙命使冊高麗國王王建。靈武奏,夏州
 界黨項七百騎侵擾,當道出師擊破之,生擒五十騎,追至賀蘭山下。己丑,兩浙節度使錢元璙起復,加守尚書令。青州節度使王晏球加兼中書令。秦、鳳、兗、宋、亳、潁、鄧大水,漂邑屋,損苗稼。夔州赤甲山崩。壬辰,以前太僕卿鄭繢為鴻臚卿,以前兗州行軍司馬李鈴為戶部尚書。乙未,福建節度使王延鈞進絹表云:「吳越王錢鏐薨,乞封臣為吳越王。湖南馬殷官是尚書令,殷薨,請授臣尚書令。」不報。戊戌,太子賓客李光憲以禮部尚書致仕。己
 亥,以前靈武節度使康福為涇州節度使。幽州衙將潘杲上言,知故使劉仁恭于大安山藏錢之所,樞密院差人監往發之,竟無所得。以皇子西京留守、京兆尹從珂為鳳翔節度使。廢鳳州武興軍節制為防禦使,并所管興、文二州並依舊隸興元府。丁未,以門下侍郎兼吏部尚書、同平章事、監修國史趙鳳為檢校太傅、同平章事,充邢州節度使。詔諸州府遭水人戶各支借麥種及等第賑貸。



 八月辛亥,青州節度使王晏球卒,廢朝二日。以
 利州節度使孫漢韶兼西面行營招討使。甲寅,以前振武節度使張萬進為鄧州節度使。己未,以鄆州節度使房知溫兼中書令,移鎮青州。丙寅,以宰臣李愚為門下侍郎、平章事、監修國史。癸亥,以湖南節度使馬希聲卒廢朝。己卯,吐蕃遣使朝貢。



 九月壬午,以鎮南軍節度使、檢校太尉馬希範為湖南節度使、檢校太尉、兼侍中。甲申,荊南節度使、檢校太傅、兼中書令高從誨加檢校太尉、兼中書令。壬辰,供奉官李瑰自西川回,節度使孟知
 祥附表陳敘隔絕之由,并進物,先賜金器等。瑰,知祥甥也,母在蜀,故今瑰往焉。瑰至蜀,具述朝廷厚待之意,知祥稱籓如初,奏福慶長公主以今年正月十二日薨。又奏五月三日,大破東川董璋之眾於漢州,收下東川。又表立功將校趙季良等五人,乞授節鉞;部內刺史令錄已下官,乞許墨制補授。帝遣閣門使劉政恩充西川宣諭使。乙巳,契丹遣使自幽州進馬。秦州地震。



 冬十月己酉朔,再遣供奉官李瑰使西川,押賜故福慶長公主
 祭贈絹三千匹,并賜知祥玉帶。先是,兩川隔遠,朝廷兵士不下三萬人,至是,知祥上表乞發遣兵士家屬入川,詔報不允。知祥所奏兩川部內文武將吏,乞許權行墨制除補訖奏,詔許之。知祥所奏立功大將趙季良等五人正授節鉞,續有處分。襄州奏,漢水溢,壞民廬舍。癸丑,以太常卿劉岳卒廢朝。己未,以兵部侍郎張文寶為吏部侍郎,以戶部侍郎藥縱之為兵部侍郎。庚申,幸至德宮,因幸石敬瑭、李從昶、李從敏之第。壬申,大理少卿康
 澄上疏曰:「臣聞安危得失,治亂興亡,誠不繫于天時,固非由于地利,童謠非禍福之本,妖祥豈隆替之源!故雊雉升鼎而桑穀生朝,不能止殷宗之盛;神馬長嘶而玉龜告兆,不能延晉祚之長。是知國家有不足懼者五,有深可畏者六。陰陽不調不足懼,三辰失行不足懼,小人訛言不足懼,山崩川涸不足懼,蟊賊傷稼不足懼,此不足懼者五也。賢人藏匿深可畏,四民遷業深可畏,上下相徇深可畏,廉恥道消深可畏,毀譽亂真深可畏,直言
 蔑聞深可畏,此深可畏者六也。伏惟陛下尊臨萬國,奄有八紘,蕩三季之澆風,振百王之舊典,設四科而羅俊彥,提二柄而御英雄。所以不軌不物之徒,咸思革面;無禮無儀之輩,相率悛心。然而不足懼者,願陛下存而勿論;深可畏者,願陛下修而靡忒。加以崇三綱五常之教,敷六府三事之歌,則鴻基與五岳爭高,盛業共磐石永固。」優詔獎之。澄言可畏六事,實中當時之病,識者許之。癸酉,湖南馬希範、荊南高重誨並進銀及茶,乞賜戰馬,
 帝還其直,各賜馬有差。丁丑,帝謂范延光曰:「如聞禁軍戍守,多不稟籓臣之命,緩急如何驅使?」延光曰:「承前禁軍出戍,便令逐處守臣管轄斷決,近似簡易。」帝曰:「速以宣命條舉之。」



 十一月辛巳,以三司使、左武衛大將軍孟鵠為許州節度使,以前許州節度使馮贇為宣徽使、判三司,以宣徽北院使孟漢瓊判院事。壬午,史館奏:「宣宗已下四廟未有實錄,請下兩浙、荊湖購募野史及除目報狀。」從之。:《五代會要》載十一月四日,史館奏:當館昨為大中以來,迄于天祐,四朝實錄,尚未纂
 修,尋具奏聞,謹行購募。敕命雖頒于數月,圖書未貢於一編。蓋以北土州城,久罹兵火,遂成滅絕,難可訪求。切恐歲月漸深,耳目不接,長為闕典,過在攸司。伏念江表列籓,湖南奧壞,至於閩、越,方屬勳賢。戈鋌自擾于中原,屏翰悉全于外府,固多奇士,富有群書。其兩浙、福建、湖廣伏乞詔旨,委各于本道采訪宣宗、懿宗、僖宗、昭宗以上四朝野史,及逐朝日歷、銀臺事宜、內外制詞、百司沿革簿籍,不限卷數,據有者抄錄上進。若民間收得,或隱士撰成,即令各列姓名,請議爵賞。癸未,以左僕射致仕鄭玨卒廢朝。丁亥,以河陽節度使兼六軍都衛副使石敬瑭為河東節度使,兼大同、彰國、振武、威塞等軍蕃漢馬步總管。時契丹帳族在雲州境上,與群臣議擇威望大臣以制北方,故
 有是命。己丑,樞密使趙延壽加同平章事。詔在京臣僚,不得進奉賀長至馬及諸物。甲午,日南至,帝御文明殿受朝賀。己亥,河中節度使李從璋加檢校太傅,以右散騎常侍楊凝式為工部侍郎。庚子,以秘書監盧文紀為工部尚書,以工部尚書崔居儉為太常卿,以工部侍郎鄭韜光為禮部侍郎。乙巳,雲州奏,契丹主在黑榆林南納喇泊造攻城之具。帝遣使賜契丹主銀器彩帛。



 十二月戊申朔,供奉官丁延徽、倉官田繼勳並棄市,坐擅出
 倉粟數百斛故也。教坊伶官敬新磨受賄,為人告,帝令御史臺徵還其錢而後撻之。癸丑,幸龍門,觀修伊水石堰,賜丁夫酒食。後數日,有司奏:「丁夫役限十五日已滿,工未畢,請更役五日。」帝曰:「不惟時寒,且不可失信于小民。」即止其役。甲寅,以太子賓客歸藹卒廢朝。戊午,以前宣徽使朱宏昭為襄州節度使;康義誠為河陽節度使,充侍衛親軍馬步軍都指揮使。壬戌,以吏部侍郎姚顗為尚書左丞,以尚書左丞王權為禮部尚書,以兵部侍郎
 藥縱之為吏部侍郎,以翰林學士、中書舍人程遜為戶部侍郎,依前充職。戊辰,帝畋於近郊,射中奔鹿。是冬無雪。



\end{pinyinscope}