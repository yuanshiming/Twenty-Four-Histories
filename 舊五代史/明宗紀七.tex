\article{明宗紀七}

\begin{pinyinscope}

 長興元年春正月丙寅朔,帝御明堂殿受朝賀,仗衛如常儀。乙亥,國子監請以監學生束脩及光學錢備監中修葺公用,從之。丙子,帝謂宰臣曰:「時雪未降,如何?」馮道
 曰:「陛下恭行儉德,憂及蒸民,上合天心,必有春澤。」是夜,降雪。其夕,右散騎常侍蕭希甫封狀申樞密稱,得河堰衙官狀,告本都將校二十餘人欲謀不軌,至旦追問無狀,斬所告人。是日,幸至德宮。辛卯,中書奏,郊天有日,合差大內留守。詔以宣徽南院使硃宏昭充。



 二月戊戌,幸稻田莊。己亥,黑水國主兀兒遣使貢方物。翰林學士劉煦奏:「新學士入院,舊試五題,請今後停試詩賦者,西漢始稱名家。《漢書·藝文志》列為九流之一。主要有,只試麻制、答蕃書、批答共三道。仍請內賜題目,定字數,付本院
 召試。」從之。《五代會要》載劉煦原奏云:「舊例學士入院,除中書舍人不試,餘官皆先試麻制、答蕃、批答各一道,詩賦各一道,號曰五題,並于當日呈納。從前每遇召試,多預出五題,潛令宿構,其無黨援者,即日起草,罕能成功。今請權停詩賦,只試三道,仍內賜題目,兼定字數。」從之。有司奏:「皇帝致齋於明堂,按舊服通天冠、絳紗袍,文武五品已上著褲褶,近例只著朝服。」從之。乙巳,中書奏:「皇帝朝獻太微宮、太廟,祭天地于圜丘,準禮例親王為亞獻行事,受誓戒。」從之。以天雄軍節度使石敬瑭為御營使。壬子,帝宿齋于明堂殿。癸丑,朝獻太微宮。是日,宿齋于太廟。詰旦,請行饗
 禮。甲寅,赴南郊齋宮。是夜微雨,三鼓後晴明如晝。乙卯,祀昊天上帝于圜丘,柴燎禮畢,郊宮受賀。是日,御五鳳樓,宣制:改天成五年為長興元年;大赦天下,除十惡五逆、放火劫舍、屠牛、官典犯贓、偽行印信、合造毒藥外,罪無輕重,咸赦除之。天成四年終諸道所欠殘稅及場院欠折,並特放免。群臣職位帶平章事、侍中、中書令,並與改鄉名里號。朝臣及蕃侯郡守亡父母,及父母在并妻室未沾恩命者,並與恩澤。應私債出利已經倍者,只許微
 本;已經兩倍者,本利並放。河陽管內人戶,每畝舊征橋道錢五文,今後不征。諸道州府每畝先征曲錢五文,今特放二文云。商州吏民以刺史郭知瓊善政聞,詔褒之。



 三月丁卯,幸會節園,遂幸河南府。靈武奏,殺戮蕃賊二千人。壬申,鳳翔節度使李從嚴進封岐國公心。心何以知?曰:虛壹而靜。」意謂專一、虛心而冷靜地觀,移鎮汴州。甲戌,延州節度使高允韜移鎮邢州。丙子,以宣徽使朱宏昭為鳳翔節度使;潞州節度使朱漢賓加檢校太傅,移鎮晉州;徐州節度使房知溫移鎮鄆州;鄆州節度
 使王晏球移鎮青州。宰臣馮道率百僚拜表,請上尊號曰聖明神武文德恭孝皇帝,詔報不允。壬午,許州節度使孔循移鎮滄州;陜州節度使張延朗移鎮許州,加檢校太傅;滄州節度使張虔釗移鎮徐州,加檢校太保。癸未,詔貶右散騎常侍、集賢殿學士、判院事蕭希甫為嵐州司戶參軍,仍馳驛發遣,坐誣告之罪也。宰臣馮道等再請上尊號,詔允之。丙戌,以侍衛親軍馬步軍都指揮使、河陽節度使康義誠為襄州節度使、檢校太傅;以左
 武衛上將軍劉彥琮為陜州節度使、檢校太保。庚寅,制淑妃曹氏可立為皇后,仍令擇日冊命。



 夏四月甲午朔,國子司業張溥奏,請復八館,以廣生徒。按《六典》,監有六學,國子、太學、四門、律學、書學、算學是也,而溥云八館,謬矣。丁酉,前汴州節度使、檢校太尉、兼侍中符習加太子太師致仕,進封衛國公。戊戌,遂州節度使夏魯奇加同平章事,皇子河中節度使從珂進位檢校太尉,封開國公。自是諸道節鎮皆次第加恩,以郊禋覃慶澤故也。己
 亥,幸會節園。壬寅,以樞密使安重誨為留守、太尉、兼中書令,使如故。青州節度使王建立加侍中,移鎮潞州。皇子河中節度使從珂奏:「臣今月五日,閱馬于黃龍莊,衙內指揮使楊彥溫據城叛,臣尋時詰問,稱奉宣命。胡三省《通鑒注》云:樞密院用宣,三省用堂帖。臣見在虞鄉縣。」帝遣西京留守索自通、侍衛步軍都指揮使藥彥稠等攻之,仍授彥溫絳州刺史,冀誘而擒之也。詔從珂赴闕。丁未,以戶部尚書李鈴為兗州行軍司馬,坐引淮南覘人貽安重誨寶帶也。
 戊申,宰臣馮道加右僕射,趙鳳加吏部尚書。乙酉,以左龍武統軍劉君鐸卒廢朝。



 癸丑,索自通、藥彥稠等奏,收復河中,斬楊彥溫,傳首來獻。初,彥稠出師,帝戒之曰:「與朕生致彥溫,吾將自訊之。」及收城,斬首傳送,帝怒彥稠等。時議皆以為安重誨方弄國權,從榮諸王敬事不暇,獨忌從珂威名,每于帝前屢言其短,巧作窺圖,冀能傾陷。彥溫既誅,從珂歸清化里第。重誨謂馮道等曰:「蒲帥失守,責帥之義,法當如何?」翼日,道等奏:「合行朝典。」帝不
 悅,趙鳳堅奏:「故事有責帥之義,所以激勵籓守。」帝曰:「皆非公等意也。」後數日,帝于中興殿見宰臣,趙鳳承重誨意,又再論列,帝默然。翼日,重誨復自論奏,帝極言以拒之,語在《末帝紀》中。帝又曰:「卿欲如何制置?」重誨曰:「于陛下父子之間,臣不合言,一稟聖旨。」帝曰:「從他私第閒坐,何煩奏也!」乃止。以前邢州節度使、檢校司徒李從溫為左武衛上將軍。丙辰,以西京留守、檢校司徒索自通為河中節度使。丁巳,雲州奏:掩襲契丹,獲頭口萬計。



 戊午,
 帝御文明殿受冊徽號,冊曰:「維長興元年,歲次庚寅,四月甲午朔,二十五日戊午,金紫光祿大夫、守尚書左僕射兼門下侍郎、同中書門下平章事、充太微宮使、宏文館大學士、上柱國、始平郡開國侯、食邑一千五百戶、食實封一百戶臣馮道,銀青光祿大夫、門下侍郎兼吏部尚書、同中書門下平章事、監修國史、判集賢院事、上柱國、天水郡開國伯、食邑七百戶臣趙鳳,及文武百官特進、太子少傅、上柱國、酒泉郡開國侯、食邑一千戶臣李
 琪等五千八百九十七人言:



 臣聞天不稱高而體尊,地不矜厚而形大,厚無不載,高無不覆。四時行于內,萬物生其間,總神祗之靈,葉帝王之運。日出而星辰自戢,龍飛而雷雨皆行,元氣和而天下和,庶事正而天下正。



 伏惟皇帝陛下,天授一德,時歷多艱。翊太祖以興邦,佐先皇而定難,拯嗣昭于潞困,救德威于燕危,遏思遠而全鄴都,誅彥章而下梁苑。成再造之業,由四征之功。洎纂鴻圖,每敷皇化。去內庫而省庖膳,出宮人而減伶官,輕
 寶玉之珍,卻鷹鸇之貢。淳風既洽,嘉瑞自臻。故登極之前,人皆不足;改元之後,時便有年。遐荒旋斃于戎王,重譯徑來于蠻子,東巡而守殷殪,北討而王都殲,破契丹而燕、趙無虞,控靈武而瓜、沙並復。



 近以饗上元而薦太廟,就吉土而配昊天,輅已降而雨霑,事欲行而月見。燔柴禮畢,作解恩覃義》云:「義理存乎識,辭章存乎才,微實存乎學。」,帝命咸均,人情普悅。非陛下有道有德,至聖至明,動不疑人,靜惟恭己,常敦孝禮,每納忠言,則何以臨御五年,澄清四海!時久纏于災害,民驟見于
 和平。休征備載于簡編,徽號過持于謙讓。三年不允,眾志皆堅。天不以上帝自崇,日不以大明自貴,于烝民有惠,于元后同符,列聖皆然,舊章斯在。今以明庭百辟,列土諸侯,中外同辭,再三瀝懇。臣等不勝大願,謹奉玉寶玉冊,上號曰聖明神武文德恭孝皇帝。



 伏惟皇帝陛下,體堯、舜之至道,法日月于太虛,威于夷狄,恩及蟲魚。奉國者繼加榮寵,違天者咸就誅鋤。典禮當告成之後,夙夜思即位之初,千秋萬歲,永混車書。



 宰臣馮道之辭也。
 庚申,以左多吾上將軍史敬熔為鄧州節度使,以右金吾上將軍符彥超為兗州節度使,以驍衛上將軍張敬詢為滑州節度使,以閬州防禦使孫岳為鳳州節度使。詔改鳳翔管內應州為匡州,信州為晏州,改新州管內武州為毅州。



 五月乙丑,鄭州防禦使張進、副使咸繼威並停任,以盜掠城中居人故也。丙寅,以少府監韋肅為洺州刺史,以潞州節度使王建立為太傅致仕。建立素與安重誨不協,因其入朝,乃言建立自鎮歸朝過鄴都,
 日有扇搖之言,以是罪之,故令致仕。丁卯,以前興元節度使劉仲殷權知潞州軍州事。戊辰,以安州節度使高行珪卒輟朝。有司上言:「皇后受冊,內外命婦並合奉賀。今未有命婦準例上表稱賀。中書諸道節度使但進表上言皇帝,外命婦上皇后賀箋表,進呈訖,無報。應皇親或有慶賀及起居章表,內中進呈後,只宣示來使,並不合答復。」從之。壬申,以權知昭義軍軍州事劉仲殷為潞州節度使、檢校太傅。丁丑,帝臨軒,命使冊淑妃曹氏為
 皇后。禮院上言,百官上疏于皇后曰「皇后殿下」,及六宮及率土婦人慶賀只呼「殿下」,不言「皇后」。中書覆奏,若只呼「殿下」,恐與皇太子無所分別,凡上中宮表章呼「皇后殿下」,若不形文字,尋常只呼「皇后」。從之。癸未,太子少傅蕭頃卒,廢朝。甲申,回鶻可汗仁喻遣使貢方物。辛卯,以翰林承旨、兵部侍郎李愚為太常卿。壬辰,以前滑州節度使李從璋為左驍衛上將軍。



 六月丁酉,以護駕馬軍都指揮使、貴州刺史安從進為宣州節度使,充護駕馬
 軍都指揮使;以護駕步軍都指揮使、澄州刺史藥彥稠為壽州節度使兼護駕步軍都指揮使。甲辰,以皇城使安崇緒為河陽留後,重誨子也。鳳翔奏:「所管良、晏、匡三州並無屬縣庫」,是這一思潮研究中心。,請卻改為縣。」從之,仍舊為軍鎮。前振武節度使安金全卒。壬子,中書門下奏:「詳覆到禮部送今年及第進士李飛、樊吉、夏侯珙、吳沺、王德柔、李谷等六人,望放及第。其盧價等七人及賓貢鄭朴,望許令將來就試。知貢舉張文寶試士不得精當,望罰一季俸。」從之。丁巳,
 皇子北京留守、河東節度使從厚移領鎮州,以左武衛上將軍李從溫為許州節度使。



 秋七月甲子,以宣徽南院使、行右衛上將軍、判三司馮贇為北京留守、太原尹。己巳,以鄧州節度使史敬鎔卒廢朝。甲戌,以左威衛上將軍梁漢顒為鄧州節度使,前兗州節度使趙在禮為左驍衛上將軍。庚辰,奉國軍節度使兼威武軍節度副使、檢校太尉、兼侍中王延稟加兼中書令。詔:「諸州得替防禦、團練使、刺史並宜于班行比擬,如未有員闕,可隨
 常參官逐日立班。」新例也。辛巳,詔揀年少宮人及西川宮人並還其家,無家可歸者,任從所適。甲申,以前齊州防禦使孫璋為鄜州節度使。戊子,以右散騎常侍陸崇卒廢朝。崇為福建冊使,卒于明州,贈兵部尚書。宿州進白兔,安重誨謂其使曰:「豐年為上瑞,兔懷狡性,雖白何為!」命退歸。



 八月甲午,以前鄧州節度使盧文進為左衛上將軍。北京奏,吐渾千餘帳內附,于天池川安置。禁在京百司影射州縣稅戶。乙未,捧聖軍使李行德、十將張
 儉、告密人邊彥溫並族誅經史諸子之學,兼工詩文、書畫、金石、醫學。著作有《霜,以其誣告安重誨私市兵仗故也。以前許州節度使張延朗為檢校太傅、行兵部尚書,充三司使。三司之有使額,自延朗始也。初,中書覆奏,授延朗諸道鹽鐵轉運等使,兼判戶部度支事。奏入,宣旨曰:「會計之司,國朝重事,將總成其事額,俾專委于近臣,貴便一時,何循往例,兼移內職,可示新規。張延朗可充三司使,班在宣徽使下。」癸卯,北京奏,生吐渾內附,欲于嵐州安族帳。都官員外郎、知制誥張昭遠奏:「請依國
 朝舊以例,選郎官、御史分行天下,宣問風俗,興利除害。」不報。


壬寅,皇子河南尹、判六軍諸衛事從榮封秦王,仍令所司擇日冊命。
 \gezhu{
  《五代會要》:長興元年九月,太常禮院奏,草定冊秦王儀注。博士段顒議曰:據《開元禮》,臨軒冊命諸王大臣,其日受冊者,朝服從第鹵簿,與百官俱集朝堂,就次受冊訖,通事舍人引,不載謁朝還第之儀。自開元以後,冊拜諸王皆正衙命使,詣延英進冊,皇帝御內殿,高品引王入立于位,高品宣制讀冊,王受冊訖,歸院,亦無乘輅謁朝之禮。臣按《五禮精義》云:「古者皆因禘嘗而頒爵祿,所以示無自專,稟之於祖宗也。」今雖冊命,不在烝嘗,然拜大官、封大邑,必至殿廷,敬慎之道也。今當司欲準《開元禮》,其日秦王服朝服,自理所乘輅車、備鹵簿,與群臣俱集朝堂,就次受冊訖,至應天門外,奉冊置于載冊之車,秦王升輅,出謁太廟訖,
  歸理所,儀仗鹵簿如來時之儀。從之。}
 戊申,兗州奏:「淮南海州都指揮使王傳拯殺本州刺史陳宣,焚燒州城,以所部兵士及家口五千人歸國,至沂州。」帝遣使慰納之。庚戌,正衙命使冊福慶長公主孟氏。以前雄武軍節度使王思同為左武衛上將軍,以前鳳州節度使陳皋為右威衛上將軍。壬子,正衙命使赴太原,冊永寧公主石氏。乙卯,以左監門衛上將軍陳延福卒廢朝。丙辰,皇子鎮州節度使從厚封宋王,仍令擇日冊命。



 九月乙丑,階州刺史王宏贄上言:「
 一州主客戶纔及千戶,並無縣局,臣今檢括得新舊主客已及三千二百,欲依舊額立將利、福津二縣,請置令佐。」從之。丁丑,詔天下諸州府,不得奏薦著紫衣官員為州縣官。戊寅,升尚書右丞為正四品。癸未,利、閬、遂三州奏,東川節度使董璋謀叛,結連西川孟知祥。甲申,以鎮州節度使范延光為檢校太傅、守刑部尚書,充樞密使。利州、閬州進納東川檄書,言將兵擊利、閬,責以間諜朝廷為名。乙酉,以左驍衛上將軍趙在禮為同州節度使
 兼四面行營馬步軍都指揮使。樞密院直學士、守工部侍郎閻至,樞密院直學士、守尚書右丞史圭,並轉戶部侍郎,依前充職。以翰林學士、守戶部侍郎李懌為尚書右丞;以翰林學士、戶部侍郎劉煦為兵部侍郎;以翰林學士、中書舍人竇夢徵為工部侍郎,依前充職。以中書舍人劉贊為御史中丞,以御史中丞許光義為兵部侍即,以兵部侍郎姚顗為吏部侍郎。丙戌,詔東川節度使董璋可削奪在身官爵,仍徵兵進討。丁亥,以西川節度
 使孟知祥兼西南面供饋使,天雄軍節度使石敬瑭兼東川行營都招討使,以遂州節度使夏魯奇兼東川行營招討副使。庚寅,以右衛上將軍王思同為京兆尹,充西京留守兼西南行營馬步都虞候。



 冬十月壬辰,以太子少傅李琪卒廢朝。癸巳,以鄜州節度使米君立卒廢朝。詔:「凡賻贈布帛,言段不言端匹,段者二丈也,宜令三司依此給付。」甲午,正衙命使冊興平公主于宋州節度使、駙馬都尉趙延壽之私第。己亥,以左驍衛上將軍李
 從璋為陜州節度使,陜州口度使劉彥琮移鎮邠州。尚書博士田敏請依舊典藏冰、頒冰,以銷陰陽愆伏之沴,詔從之。《五代會要》載原敕云:「藏冰之制,載在前經,獻廟之儀,廢于近代,既朝臣之特舉,案典禮以宜行。田敏所奏祭司寒獻羔事宜依。其桃弧棘矢,事久不行,理難備創。其諸侯亦宜準往制藏冰。」乙巳,供奉官張仁暉自利州迴,奏董璋攻陷閬州,節度使李仁矩舉家遇害。丁未,宮苑使董光業并妻子並斬于都市,璋之子也。辛亥,以武安軍節度副使、洪鄂道行營副都統、檢校太尉馬希聲為武安軍節度使,加兼侍中。時湖南
 馬殷奏,久病不任軍政,乞以男希聲為帥,故有是命。中書奏:「吏部流內銓諸色選人,所試判兩節,欲委定其等第,文優者超一資,其次者次資,又次者以同類,道理全疏者于同類中少人戶處注擬。」從之。



 十一月庚申朔,帝御文明殿,冊皇子秦王,仗衛樂懸如儀。甲子,正衙命使冊皇子宋王于鎮州。是日,幸龍門。翼日,馮道奏曰:「陛下宮中無事,遊幸近郊則可矣,若涉歷山險,萬一馬足蹉跌,則貽臣下之憂。臣聞千金之子,坐不垂堂;百金之
 子,立不倚衡。況貴為天子,豈可自輕哉!」帝斂容謝之。退令小黃門至中書問道垂堂、倚衡之義,道因注解以聞,帝深納之。己巳,故太子少保致仕封舜卿贈太子少傅。庚午,應州節度使張敬達移雲州,以捧聖都指揮使、守恩州刺史沙彥詢為應州節度使;以潁州團練使高行周為安北都護,充振武節度使。壬申,黔南節度使楊漢章棄城奔忠州,為董璋所攻也。乙亥,制西川節度使孟知祥削奪官爵,以其同董璋叛也。丙子,以前同州節度
 使羅周敬為左監門上將軍。丁丑,故兵部侍郎許光義加贈禮部尚書。辛巳,西面軍前奏,今月十三日,階州刺史王宏贄、瀘州刺史馮暉,自利州取山路出劍門關外倒下,殺敗董璋守關兵士三千餘人,收復劍州。甲申,日南至,帝御文明殿受朝賀。丙戌,以給事中鄭韜光為左散騎常侍。青州奏,得登州狀,契丹安巴堅男東丹王托云越海來歸國。《契丹國志》:時東丹王失職怨望,因率其部四十餘人越海歸唐。



 十二月乙未,荊南奏,湖南節度使、楚國王馬殷薨,廢朝三日。庚子,以前
 襄州節度使安元信為宋州節度使。辛丑和年表。,幸苑中。丁未,以二王後祕書丞、襲酅國公楊仁矩卒輟朝,贈工部郎中。庚戌,湖南節度使馬希聲起復,加兼中書令。壬子,以樞密院直學士、戶部侍郎閻至為澤州刺史;樞密使直學士、戶部侍郎史圭為貝州刺史。甲寅,遣樞密使安重誨赴西面軍前。時帝以蜀路險阻,進兵艱難,潼關已西,物價甚賤,百姓挽運至利州,率一斛不得一斗,謂侍臣曰:「關西勞擾,未有成功,誰能辦吾事者!朕須自行。」安重
 誨曰:「此臣之責也,臣請行。」帝許之。言訖而辭,翼日遂行。甲寅,故西川兵馬都監、泗州防禦使李嚴贈太傅。丙辰,車駕畋於西山,臘也。丁巳,回鶻遣使來朝貢。戊午,故荊南節度使、檢校太尉、兼尚書令、南平王高季興贈太尉。



\end{pinyinscope}