\article{太祖紀三}

\begin{pinyinscope}

 開平元年正月丁亥,帝回自長蘆,次於魏州。節度使羅紹威以帝迴軍,慮有不測之患,由是供億甚至,因密以天人之望切陳之。帝雖拒而不納,然心德之。壬寅,帝至
 自長蘆。是日,有慶雲覆於府署之上。甲辰,天子遣御史大夫薛貽矩來傳禪代之意。貽矩謁帝,陳北面之禮,帝揖之升階。貽矩曰:「殿下功德及人,三靈所卜已定。皇帝方議裁詔,行舜、禹之事,臣安敢違。」既而拜伏於砌下,帝側躬以避之。



 二月戊申,帝之家廟棟間有五色芝生焉,狀若芙蓉,紫煙蒙護,數日不散。又是月,家廟第一室神主上,有五色衣自然而生,識者知梁運之興矣。唐乾符中,木星入南斗,數夕不退,諸道都統晉國公王鐸觀之,
 問諸知星者吉凶安在,咸曰:「金火土犯斗即為災,唯木當為福耳!」或亦然之。時有術士邊岡者,洞曉天文,博通陰陽歷數之妙,窮天下之奇秘,有先見之明,雖京房、管輅不能過也。鐸召而質之,岡曰:「惟木為福神,當以帝王占之。然則非福於今,必當有驗於後,未敢言之,請他日證其所驗。」一日,又密召岡,因堅請語其詳,至於三四,岡辭不獲。鐸乃屏去左右,岡曰:「木星入斗,帝王之兆也。木在斗中,『硃』字也。以此觀之,將來當有朱氏為君者也,天
 戒之矣。且木之數三,其禎也應在三紀之內乎!」鐸聞之,不復有言。天后朝有讖辭云:「首尾三鱗六十年,兩角犢子自狂顛,龍蛇相鬥血成川。」當時好事者解云:「兩角犢子,牛也,必有牛姓乾唐祚。」故周子諒彈牛仙客,李德裕謗牛僧孺,皆以應圖讖為辭。然「朱」字「牛」下安「八」,八即角之象也,故朱滔、硃泚構喪亂之禍,冀無妄之福,豈知應之帝也。


四月,唐帝御札敕宰臣張文蔚等備法駕奉迎梁朝。宋州刺史王皋進赤烏一雙。又,宰臣張文蔚正押
 傳國寶、玉冊、金寶及文武群官、諸司儀仗法物及金吾左右二軍離鄭州。丙辰,達上源驛。是日,慶雲見。令曰:「王者創業興邦,立名傳世,必難知而示訓,從易避以便人。
 \gezhu{
  案:此下有闕文。}
 或稽其符命,應彼開基之義,垂諸象德之言。爰考簡書,求於往代,周王昌、發之號,漢帝詢、衍之文,或從一德以徽稱,或為二名而更易。先王令典,布在縑緗。寡人本名,兼於二字,且異帝王之號,仍兼易之難,郡職縣官,多須改換。況宗廟不遷之業,憲章百世之規,事葉
 典儀,豈憚革易。寡人今改名晃,是以天意雅符於明德,日光顯契於瑞文,昭融萬邦,理斯在是。庶順昊穹之意,永臻康濟之期。宜令有司分告天地宗廟,其舊名,中外章疏不得更有迴避。」時將受禪,下教以本名二字異帝王之稱,故改名。己未,賜文武百官一百六十人本色衣一副。戊辰,即位。制曰:


王者受命於天,光宅四海,祗事上帝,寵綏下民。革故鼎新,諒歷數而先定必有名世者」
 \gezhu{
  《孟子·公孫丑下》}
 。並歷數堯、舜、湯、文,創業垂統,知圖籙以無差。神器所歸,祥符合應。是以三正互用,五運相
 生,前朝道消,中原政散,瞻烏莫定,失鹿難追。朕經緯風雷,沐浴霜露,四徵七伐,垂三十年,糾合齊盟,翼戴唐室。隨山刊木,罔憚胼胝;投袂揮戈,不遑寢處。洎上穹之所贊,知廣運之不興,莫諧輔漢之謀,徒罄事殷之禮。唐主知英華已竭,算祀有終,釋龜鼎以如遺,推劍紱而相授。朕懼德弗嗣,執謙允恭,避駿命於南河,眷清風於穎水。而乃列岳群后,盈廷庶官,東西南北之人,斑白緇黃之眾,謂朕功蓋上下,澤被幽深,宜應天以順時,俾化家而
 為國。拒彼億兆,至於再三。且曰七政已齊,萬幾難曠。勉遵令典,爰正鴻名,告天地神祗,建宗廟社稷。



 顧惟涼德,曷副樂推,慄若履冰,懍如馭朽。金行啟祚,玉歷建元,方宏經治之規,宜布惟新之令。可改唐天祐四年為開平元年,國號大梁。《書》載虞賓,斯為令範,《詩》稱周客,蓋有明文。是用先封,以禮後嗣,宜以曹州濟陰之邑奉唐主,封為濟陰王。凡曰軌儀,並遵故實。姬庭多士,比是殷臣;楚國群材,終為晉用。歷觀前載,自有通規,但遵故事之文,
 勿替在公之效。應是唐朝中外文武舊臣,見任前資官爵,一切仍舊。凡百有位,無易厥章,陳力濟時,盡瘁事我。古者興王之地,受命之邦,集大勳有異庶方,霑慶澤所宜加等。故豐沛著啟祚之美,穰鄧有建都之榮,用壯鴻基,且旌故里,爰遵令典,先示殊恩。宜升汴州為開封府,建名東都。其東都改為西都,仍廢京兆府為雍州佑國軍節度使。《五代會要》:四月,改京兆府為大安府,長安縣為大安縣,萬年縣為大年縣,仍置佑國軍節度使額。始命韓建為佑國軍節度使。



 是日大酺,賞賜有差。《通鑒》:甲辰,唐昭宣帝降御札禪
 位于梁。以攝中書令張文蔚為冊禮使,禮部尚書蘇循副之;攝侍中楊涉為押傳國寶使,翰林院學士張策副之;御史大夫薛貽矩為押金寶使,尚書左丞趙光逢副之。帥百官備法駕,詣大梁。甲子,張文蔚、楊涉乘輅自上源驛至,從冊寶諸司各備儀衛鹵簿前導,百官從其後,至金祥殿前陳之。王被袞冕,即皇帝位。張文蔚、蘇循奉冊升殿進讀,楊涉、張策、薛貽矩、趙光逢以次奉寶升殿,讀已,降,帥百官舞蹈稱駕。帝遂與文蔚等宴於元德殿。帝舉酒曰:「朕輔政未久,此皆諸公推戴之力。」文蔚等慚懼,俯伏不能對,獨蘇循、薛貽矩及刑部尚書張禕盛稱帝功德,宜應天順人。宋州刺史王皋進兩岐麥,陳州袁象先進白兔一,付史館編錄,兼示百官。詔在京司及諸軍州縣印一例鑄換,其篆文則各如舊。辛未,武安軍節度使馬殷
 進封楚王。以太府卿敬翔知崇政院,翔與帷幄之謀,故首擢焉。追尊四代廟號:高祖媯州府君上謚曰宣元皇帝,廟號肅祖,太廟第一室,陵號興極陵,祖妣高平縣君范氏追謚宣僖皇后;皇曾祖宣惠王上謚曰光獻皇帝,廟號敬祖,第二室,陵號永安,祖妣秦國夫人楊氏追謚光孝皇后;皇祖武元王上謚曰昭武皇帝,廟號憲祖,第三室,陵號光天,祖妣吳國夫人劉氏追謚昭懿皇后;皇考文明王上謚曰文穆皇帝,廟號烈祖,第四室,陵號咸
 寧,皇妣晉國太夫人王氏追謚文惠皇后。以宣武節度副使皇子友文為開封尹,判建昌院事。友文,本康氏子也,帝養以為子。



 是月,制宮殿門及都門名額:正殿為崇元殿,東殿為元德殿,內殿為金祥殿,萬歲堂為萬歲殿,門如殿名。帝自謂以金德王,又以福建上獻鸚鵡,諸州相繼上白烏、白兔洎白蓮之合蒂者,以為金行應運之兆,故名殿曰金祥。以大內正門為元化門,皇墻南門為建國門,滴漏門為啟運門,下馬門為升龍門,元德殿前
 門為崇明門,正殿東門為金烏門,西門為玉兔門,正衙東門為崇禮門,東偏門為銀臺門,宴堂門為德陽門,天王門為賓天門,皇墻東門為寬仁門,浚儀門為厚載門,皇墻西門為神獸門,望京門為金鳳門,宋門為觀化門,尉氏門為高明門,鄭門為開明門,梁門為乾象門,酸棗門為興和門,封丘門為含耀門,曹門為建陽門。升開封、浚儀為赤縣,尉氏、封丘、雍丘、陳留為畿縣。《五代會要》:四月,改左右長直為左右龍虎軍,左右內衙為左右羽林軍,左右堅銳夾馬突將為左右神武軍,左右親隨軍將馬軍為左右龍驤軍。



 五
 月,以唐朝宰臣張文蔚、楊涉並為門下侍郎、平章事;以御史大夫薛貽矩為中書侍郎、平章事。帝初受禪,求理尤切,委宰臣搜訪賢良。或有在下位抱負器業久不得伸者,特加擢用;有明政理得失之道規救時病者,可陳章疏,當親鑒擇利害施行,然後賞以爵秩;有晦跡丘園不求聞達者,令彼長吏備禮邀致,冀無遺逸之恨。進封河南尹兼河陽節度使張全義為魏王,兩浙節度使錢鏐進封吳越王。辛巳,有司奏,以降誕之日為大明節,休
 假前後各一日。壬午,保義軍節度使朱友謙進百官衣二百副。乙酉,立皇兄全昱為廣王,皇子友文為博王,友珪為郢王,友璋為福王,友雍為賀王,友徽為建王。辛卯,以東都舊第為建昌宮,改判建昌院事為建昌宮使。初,帝創業之時,以四鎮兵馬倉庫籍繁,因總置建昌院以領之,至是改為宮,蓋重其事也。甲午,詔天下管屬及州縣官名犯廟諱者,各宜改換:城門郎改為門局郎,茂州改為汶州,桂州慕化縣改為歸化縣,潘州茂名縣改為
 越裳縣。魏泰《東軒筆錄》:京師呼城外為州東、州西、州南、州北,而韋城、相城、胙城等縣,但呼韋縣、相縣、胙縣,蓋沿梁時避諱之舊也。詔樞密院宜改為崇政院,以知院事敬翔為院使。改文思院為乾文院,同和院改為佐鸞院。《五代會要》:五月,改御食使為司膳使,小馬坊使為天驥使。以西都水北宅為大昌宮,廢雍州太清宮,改西都太微宮,亳州太清宮皆為觀,諸州紫極宮皆為老君廟。泉州僧智宣自西域回,進辟支佛骨及梵夾經律。丙申,御元德殿,宴犒諸軍使劉捍、符道昭已下,賜物有差。



 是月,青州、許州、定州三鎮節度使請開內宴,各賜方物。以青州節度使韓建守司徒、平
 章事。帝以建有文武材,且詳於稼穡利害、軍旅之事、籌度經費,欲盡詢焉,恩澤特異於時,罕有比者;隨拜為上相,賜賚甚厚。宿州刺史王儒進白兔一,濮州刺史圖嘉禾瑞麥以進。廣州進奇寶名藥,品類甚多。河南尹張全義進開平元年已前羨餘錢十萬貫、綢六千匹、綿三十萬兩,仍請每年上供定額每歲貢絹三萬匹,以為常式。荊南高季昌進瑞橘數十顆,質狀百味,倍勝常貢。且橘當冬熟,今方仲夏,時人咸異其事,因稱為瑞。



 六月,幸乾
 元院,宴召宰臣、學士及諸道入貢陪臣。己亥,帝御崇元殿,內出追尊四廟上謚號玉冊寶共八副,宰臣文武百官儀仗鼓吹導引至太廟行事。癸卯,司天監奏:「日辰內有『戊』字,請改為『武』。」從之。癸亥,詔以前朝官僚,譴逐南荒,積年未經昭雪,其間有懷抱材器為時所嫉者,深負冤抑。仍令錄其名姓,盡復官資,兼告諭諸道令津致赴闕。如已亡歿,並許歸葬,以明恩蕩。以西都徽安門北路逼近大內宮垣,兼非民便,令移自榆林直趨端門之南。改
 耀州報恩禪院為興國寺。馬殷奏破淮寇;靜海軍節度使曲裕卒。


七月丙申,以靜海軍行營司馬權知留後曲顥起復為安南都護,充節度使。
 \gezhu{
  《五代會要》:七月,敕云:「建國遷都,俾新其制,況山河之險,表裏為防。今二京俱在關東以內,仍以潼關隸陜州,復置河潼軍使,命虢州刺史兼領之。」其月,敕改虎牢關為軍,仍置虎牢關軍使。}
 己亥,追尊皇妣為皇太后。



 八月,以潞州軍前屯師旅,壁壘未收,乃別議戎帥,於是以亳州刺史李思安充潞州行營都統。敕:「朝廷之儀,封冊為重,用報勳烈,以隆恩榮,固合親臨,式光典禮。舊章久缺,自我復
 行。今後每封冊大臣,宜令有司備臨軒之禮。」《五代會要》:八月,敕云:「諸道所有軍事申奏,令直至右銀臺門,委客省使畫時引進,尋常公事依前四方館收接。」甲子平明前,老人星見於南極。壬申,密州進嘉禾,又有合歡榆樹,並圖形以獻。是月,隰州奏,大寧縣至固鎮上下二百里,今月八日,黃河清,至十月如故。



 九月辛丑,西京大內放出兩宮內人及前朝宮人,任其所適。敕以近年文武官諸道奉使,皆於所在分外停住,踰年涉歲,未聞歸闕。非惟勞費州郡,抑且侮慢國經。臣節既虧,憲章安在。自今
 後兩浙、福建、廣州、安南、邕、容等道使到發許住一月;湖南、洪、鄂、黔、桂許住二十日;荊、襄、同、雍、鎮、定、青、滄許住十日;其餘側近不過三五日。凡往來道路,據遠近里數,日行兩驛。如遇疾患及江河阻隔,委所在長吏具事由奏聞。如或有違,當行朝典,命御史點檢糾察,以人敬慢官。魏博羅紹威二男廷望、廷矩,年在幼稚,皆有材器,帝以其籓屏勳臣之胄,宜受非次之用,皆擢為郎。恩命既行之後,二子亦就班列。紹威乃上章,以齒幼未任公事,乞免
 主印、宿直。從之。封鎮東軍神祠為崇福侯。浙西奏,道門威儀鄭章、道士夏隱言,焚修精志,妙達希夷,推諸輩流實有道業。鄭章宜賜號貞一大師,仍名元章,隱言賜紫衣。《五代會要》:九月,置左右天興、左右廣勝軍,仍以親王為軍使。



 十月,帝以用軍,未暇西幸,文武百官等久居東京,漸及疑訝盡言,此非言乎系表者也。」,令就便各許歸安,只留韓建、薛貽矩,翰林學士張策、韋郊、杜曉,中書舍人封舜聊、張袞並左右御史、司天監、宗正寺,兼要當諸司節級外,其宰臣張文蔚已下文武百官,並先於西京
 祗候。庚午,大明節,內外臣僚各以奇貨良馬上壽。故事,內殿開宴,召釋、道二教對御談論,宣旨罷之。命閣門使以香合賜宰臣佛寺行香。駕幸繁臺講武。癸酉,御史司憲薛廷珪奏請文武百官仍舊朝參。先是,帝欲親征河東,命朝臣先赴洛都,至是緩其期,乃允所奏。宰臣請每月初入閣,望日延英聽政,永為常式。山南東道節度使楊師厚進納越匡凝東第書籍。先是,收復襄、漢,帝閱其圖書,至是命師厚進焉。廣州進獻助軍錢二十萬,又進龍
 腦、腰帶、珍珠枕、玳瑁、香藥等。



 十一月壬寅,帝以征討未罷,調補為先,遂命盡赦逃亡背役髡黥之人,各許歸鄉里。廣州進龍形通犀腰帶、金托裏含稜玳瑁器百餘副,香藥珍巧甚多。廣南管內獲白鹿,並圖形來獻,耳有兩缺。按《符瑞圖》,鹿壽千歲變白,耳一缺。今驗此鹿耳有二缺,其獸與色皆應金行,實表嘉瑞。



 十二月辛亥,詔曰:「潞寇未平,王師在野。攻戰之勢,難緩於寇圍;飛免之勤,實勞於人力。永言輟耒,深用軫懷。宜令長吏,丁寧布告,期
 以兵罷之日,給復賦租。」於是人戶聞之,皆忘其倦。詔故荊南節度使、守中書令、上谷王周汭贈太師,故武昌軍節度使、兼中書令、西平王杜洪贈太傅。先是,鄂渚再為淮夷所侵,攻圍甚急,杜洪以兵食將盡,繼來乞師。帝料其隔越大江,難以赴援,兼以荊州據上游,多戰艦,去江夏甚邇,因命周汭舉舟師沿流以救之。汭於是引兵東下,纔及鄂界,遇朗州背盟作亂,乘江陵之虛,縱兵襲破之,俘掠且盡。既而汭士卒知之,皆顧其家,咸無鬥志,遂
 為淮寇所敗,將卒潰散,汭忿恚自投於江。汭之本姓犯文穆皇帝廟諱,至是因追贈,以其系出周文,故賜姓周氏。及汭兵敗之後,武昌以重圍經年,糧盡力困,救援不至,訖為淮寇所陷,載洪以送淮師,遂殺之。此二鎮,皆以忠貞歿於王事。帝每言諸籓屏翰經綸之業,必首痛汭、洪之薨,至是追贈之,深加軫悼,各以其子孫宗屬錄用焉。棣州蒲臺縣百姓王知嚴妹,以亂離並失怙恃,因舉哀追感,自截兩指以祭父母。帝以遺體之重,不合毀傷,
 言念村閭,何知禮教。自今後所在郡縣,如有截指割股,不用奏聞。


是年,諸道多奏軍人百姓割股,青、齊、河朔尤多。帝曰:「此若因心,亦足為孝。但茍免徭役,自殘肌膚,欲以庇身,何能療疾?並宜止絕。」
 \gezhu{
  《五代會要》:十二月,於輝州碭山縣置崇德軍。太祖榆社在碭山,置使以領之,始命硃彥讓為軍使。}



\end{pinyinscope}