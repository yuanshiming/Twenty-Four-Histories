\article{莊宗紀一}

\begin{pinyinscope}

 莊宗光聖神閔孝皇帝,諱存勖,武皇帝之長子也。母曰貞簡皇后曹氏,以唐光啟元年歲在乙巳,冬十月二十二日癸亥,生帝于晉陽宮。妊時,曹后嘗夢神人,黑衣擁
 扇,夾侍左右。載誕之辰,紫氣出于窗戶。及為嬰兒,體貌奇特,沈厚不群,武皇特所鐘愛。及武皇之討王行瑜,帝時年十一,從行。初令入覲獻捷,迎駕還宮,昭宗一見駭之,曰:「此兒有奇表。」因撫其背曰:「兒將來之國棟也,勿忘忠孝于予家。」因賜鸂鶒酒卮、翡翠盤。《北夢瑣言》云:昭宗曰:「此子可亞其父。」時人號曰「亞子」。賊平,授檢校司空、隰州刺史,改汾、晉二郡,皆遙領之。帝洞曉音律,常令歌舞于前。十三習《春秋》,手自繕寫,略通大義。及壯,便射騎,膽略絕人,其心豁如也。



 武皇
 起義雲中,部下皆北邊勁兵,及破賊迎鑾,功居第一。由是稍優寵士伍,因多不法,或陵侮官吏,豪奪士民,白晝剽攘,酒博喧競。武皇緩于禁制,惟帝不平之,因從容啟于武皇,武皇依違之。及安塞不利之後,時事多難,梁將氏叔琮、康懷英頻犯郊圻,土疆日蹙,城門之外,鞠為戰場,武皇憂形于色。帝因啟曰:「夫盛衰有常理,禍福繫神道。家世三代,盡忠王室,勢窮力屈,無所愧心。物不極則不反,惡不極則不亡。今朱氏攻逼乘輿,窺伺神器,陷害
 良善,誣誑神祇。以臣觀之,殆其極矣。大人當遵養時晦,以待其衰,何事輕為沮喪!」太祖釋然,因奉觴作樂而罷。



 及滄州劉守文為梁朝所攻,其父仁恭遣使乞師,武皇恨其翻覆,不時許之。帝白曰:「此吾復振之道也,不得以嫌怨介懷。且九分天下,朱氏今有六七,趙、魏、中山在他廡下,賊所憚者,惟我與仁恭爾;我之興衰,系此一舉,不可失也。」太祖乃徵兵于燕,攻取潞州,既而丁會果以城來降。



 天祐五年春正月,武皇疾篤,召監軍張承業、大將
 吳珙謂曰:「吾常愛此子志氣遠大,可付後事,惟卿等所教。」及武皇厭代,帝乃嗣王位于晉陽,時年二十有四。



 汴人方寇潞州,周德威宿兵于亂柳,以軍城易帥,竊議忷忷,訛言播于行路。帝方居喪,將吏不得謁見,監軍使張承業排闥至廬所,言曰:「夫孝在不墜家業,不同匹夫之孝。且君父厭世,嗣主未立,竊慮兇猾不逞之徒,有懷覬望。又汴寇壓境,利我凶衰,茍或搖動,則倍張賊勢,訛言不息,懼有變生。請依顧命,墨縗聽政,保家安親,此惟大
 孝。」帝于是始聽斷大事,



 時振武節度使克寧,即帝之季父也,為管內蕃漢馬步都知兵馬使,典握兵柄。帝以軍府事讓季父,曰:「兒年幼稚,未通庶政,雖承遺命,恐未能彈壓。季父勳德俱高,眾情推伏,且請制置軍府,俟兒有立,聽季父處分。」克寧曰:「亡兄遺命,屬在我兒,孰敢異議!」因率先拜賀。初,武皇獎勵戎功,多畜庶孽,衣服禮秩如嫡者六七輩,比之嗣王,年齒又長,部下各綰強兵,朝夕聚議,欲謀為亂。及帝紹統,或強項不拜,鬱鬱憤惋,託疾
 廢事。會李存顥以陰計干克寧曰:「兄亡弟立,古今舊事,季父拜侄,理所未安。」克寧妻素剛狠,因激怒克寧,陰圖禍亂。存顥欲于克寧之第謀害張承業、李存璋等,以并、汾九州歸附于梁,送貞簡太后為質。克寧意將激發,乃擅殺大將李存質,請授己雲州節度使,割蔚、朔、應三州為屬郡,帝悉俞允,然知其陰禍有日矣。克寧俟帝過其第,則圖竊發。時幸臣史敬熔者,亦為克寧所誘,盡得其情,乃來告帝。帝謂張承業曰:「季父所為如此,無猶子之
 情,骨肉不可自相魚肉,予當避路,則禍亂不作矣!」承業曰:「臣受命先王,言猶在耳。存顥輩欲以太原降賊,王欲何路求生?不即誅除,亡無日矣。」因召吳珙、李存璋、李存敬、朱守殷諭其謀,眾咸憤怒。



 二月壬戌,命存璋伏甲以誅克寧,遂靖其難。是月,唐少帝崩于曹州,梁祖使人鴆之也。帝聞之,舉哀號慟。



 三月,周德威尚在亂柳,梁將李思安屢為德威所敗,閉壁不出。是時,梁祖自將兵至澤州,以劉知俊為招討使以代思安,以范君實、劉重霸為
 先鋒,牛存節為撫遏,統大軍營于長子。



 四月,帝召德威軍歸晉陽。汴人既見班師,知我國禍,以為潞州必取,援軍無俟再舉,遂停斥候。梁祖亦自澤州歸洛。帝知其無備,乃謂將佐曰:「汴人聞我有喪,必謂不能興師,人以我少年嗣位,未習戎事,必有驕怠之心。若簡練兵甲,倍道兼行,出其不意,以吾憤激之眾,擊彼驕惰之師,拉朽摧枯,未云其易,解圍定霸,在此一役。」甲子,軍發自太原。己巳,至潞州北黃碾下營。



 五月辛未朔,晨霧晦暝,帝率親軍
 伏三垂崗下。詰旦,天復昏霧,進軍直抵夾城。時李嗣源總帳下親軍攻東北隅;李存璋、王霸率丁夫燒寨,劚夾城為二道;周德威、李存審各分道進攻,軍士鼓噪,三道齊進。李嗣源壞夾城東北隅,率先掩擊,梁軍大恐,南向而奔,投戈委甲,噎塞行路,斬萬餘級,獲其將副招討使符道昭洎大將三百人,芻粟百萬。梁招討使康懷英得百餘騎,出天井關而遁。梁祖聞其敗也,既懼而歎曰:「生子當如是,李氏不亡矣!吾家諸子乃豚犬爾。」初,唐龍紀
 元年,帝纔五歲,從武皇校獵于三垂崗,崗上有明皇原廟在焉。武皇于祠前置酒,樂作,伶人奏《百年歌》者,陳其衰老之狀,聲調心妻苦。武皇引滿,捋鬚指帝曰:「老夫壯心未已,二十年後,此子必戰于此。」及是役也,果符其言焉。



 是月,周德威乘勝攻澤州,刺史王班登城拒守。梁將劉知俊自晉、絳將兵赴援,德威退保高平。帝遂班師于晉陽,告廟飲至,賞勞有差。乃下令于國中,禁賊盜,恤孤寡,徵隱逸,止貪暴,峻隄防,寬獄訟,期月之間,其俗丕變。帝
 每出,于路遇饑寒者,必駐馬而臨問之,由是人情大悅,王霸之業,自茲而基矣。



 六月,鳳翔李茂貞、邠州楊崇本合四川王建之師五萬,以攻長安,遣使會兵于帝,帝遣張承業率師赴之。



 九月,邠、岐、蜀三鎮復大舉攻長安,遣李嗣昭、周德威將兵三萬攻晉州以應之。德威與梁將尹皓戰于神山北,梁人大敗。是時,晉之騎將夏侯敬受以一軍奔于梁,德威乃退保隰州。



 天祐六年秋七月,邠、岐二帥及梁之叛將劉知俊俱遣使來告,將大舉以伐
 靈、夏,兼收關輔,請出兵晉、絳,以張兵勢。八月,帝御軍南征,先遣周德威、李存審、丁會統大軍出陰地關,攻晉州,為地道,壞城二十餘步,城中血戰拒守。梁祖遣楊師厚領兵赴援,德威乃收軍而退。《通鑒》引《莊宗實錄》云:汴軍至蒙坑,周德威逆戰,敗之,斬首三百級,楊師厚退保絳州。是役也,小將蕭萬通戰歿,師厚進營平陽,德威收軍而退。



 天祐七年秋七月,鳳翔李茂貞、邠州楊崇本皆遣師來會兵,同討靈、夏。且言劉知俊三敗汴軍于寧州,靈、夏危蹙,岐、隴之師大舉,決取河西。帝令周德威將兵萬人,西渡河以應之。是
 役也,劉知俊為岐人所構,乃自退。



 九月,德威班師。冬十月,梁祖遣大將李思安、楊師厚率師營于澤州,以攻上黨。十一月,鎮州王鎔遣使來求援。是時,梁祖以羅紹威初卒,全有魏博之地,因欲兼并鎮、定,遣供奉官杜廷隱、丁延徽督魏軍三千人入于深、冀,鎮人懼,故來告難。帝集軍吏議之,咸欲按甲治兵,徐觀勝負,惟帝獨斷,堅欲救之,乃遣周德威率軍屯于趙州。是月,行營都招討使丁會卒。



 十二月丁巳朔,梁祖聞帝軍屯趙州,命寧國軍
 節度使王景仁為北面行營招討使,韓勍為副,相州刺史李思安為前鋒,會魏州之兵以討王鎔;又令閻寶、王彥章率二千騎,會景仁于邢、洺。丁丑,景仁營于柏鄉,帝遂親征,自贊皇縣東下。辛巳,至趙州,與周德威兵合。帝令史建瑭以輕騎嘗寇,獲芻牧者二百人,問其兵數,精兵七萬。是日,帝觀兵于石橋南。詰旦,進軍,距柏鄉一舍,周德威、史建瑭率蕃落勁騎以挑戰,四面馳射,梁軍閉壁不出,乃退。翼日,進軍,距柏鄉五里,遣騎軍逼其營。梁
 將韓勍、李思安率步騎三萬,鎧甲炫曜,其勢甚盛,分道以薄帝軍。德威且戰且退,距河而止。既而德威偵知梁人造浮橋,乃退保高邑。乙酉,致師于柏鄉,帝禱戰于光武廟。柏鄉無芻粟之備,梁軍以樵采為給,為帝之游軍所獲,由是堅壁不出,剉屋茅坐席以秣其馬,眾心益恐。



 天祐八年正月丁亥,周德威、史建瑭帥三千騎致師于柏鄉,設伏于村塢間,遣三百騎直壓其營。梁將怒,悉其軍結陣而來,德威與之轉戰至高邑南,梁軍列陣,橫亙
 六七里。時帝軍未成列,李存璋引諸軍陣于野河之上,梁以五百人爭橋,鎮、定之師與血戰,梁軍敗而復整者數四。帝與張承業登高觀望,梁人戈矛如束,申令之後,囂聲若雷,王師進退有序,步騎嚴整,寂然無聲。帝臨陣誓眾,人百其勇,短兵既接,無不奮力。梁有龍驤、神威、拱宸等軍,皆武勇之士也,每一人鎧仗,費數十萬,裝以組繡,飾以金銀,人望而畏之。自巳及午,騎軍接戰,至晡,梁軍欲抽退,塵埃漲天,德威周麾而呼曰:「汴人走矣!」帝
 軍齊噪以進,魏人收軍漸退。李嗣源率親軍與史建瑭、安金全兼北部吐渾諸軍衝陣夾攻,梁軍大敗,棄鎧投仗之聲,震動天地,龍驤、神威、神捷諸軍,殺戮殆盡。自陣至柏鄉數十里,僵屍枕籍,敗旗折戟,所在蔽地。夜漏一鼓,帝軍入柏鄉,梁軍輜重、帳幄、資財、奴僕,皆為帝軍所有。梁將王景仁、韓敬、李思安等以數十騎夜遁。是役也,斬首二萬級,獲馬三千匹,鎧甲兵仗七萬,輜車鍋幕不可勝計。擒梁將陳思權以下二百八十五人。帝號令收
 軍于趙州。既而梁人棄深、冀二州而遁。



 初,杜廷隱之襲深、冀也,聲言分兵就食。時王鎔將石公立戍深州,欲杜關不納,鎔遽令啟關,命公立移車于外,廷隱遂據其城。公立既出,指城闉而言曰:「開門納盜,後悔何追,此城數萬生靈,生為俘馘矣!」因投刃泣下。數日,廷隱閉城殺鎮兵數千人,遂登陴拒守,王鎔方命公立攻之,即有備矣。及柏鄉之敗,兩州之人悉為奴擄,老弱者皆坑之。己亥,遣史建瑭、周德威徇地于邢、魏,先馳檄以諭之。《冊府元龜》載晉
 王諭邢、洺、魏、博、衛、滑諸郡縣檄。天祐八年正月,周德威等破賊,徇地邢、洺,先馳檄諭邢、洺、魏、博、衛、滑諸郡縣曰:「王室遇屯,七廟被陵夷之酷;昊天不弔,萬民罹塗炭之災。必有英主奮庸,忠臣仗順,斬長鯨而清四海,靖襖祲以泰三靈。予位忝維城,任當分閫,念茲顛覆,詎可宴安。故仗桓、文輔合之規,問羿、浞凶狂之罪。逆溫碭山庸隸,巢孽餘兇,當僖宗奔播之初,我太祖掃平之際,束身泥首,請命牙門,苞藏姦詐之心,惟示婦人之態。我太祖俯憐窮鳥,曲為開懷,特發表章,請帥梁汴,才出萑蒲之澤,便居茅社之尊,殊不感恩,遽行猜忍。我國家祚隆周、漢,跡盛伊、唐,二十聖之基,三百年之文物。外則五侯九伯,內則百辟千官,或代襲簪纓,或門傳忠孝,皆遭陷害,永抱沉冤。且鎮、定兩籓,國家巨鎮,冀安民而保族,咸屈節以稱籓。逆溫唯伏陰謀,專行不義,欲全吞噬,先據屬州。趙州特發使車,來求援助。予情惟蕩寇,
 義切親仁,躬率賦輿,赴茲盟約。賊將王景仁將兵十萬,屯據柏鄉,遂驅三鎮之師,授以七擒之略。鸛鵝才列,梟獍大奔,易如走阪之丸,勢若燎原之火。殭尸僕地,流血成川。組甲雕戈,皆投草莽,謀夫猛將,盡作俘囚。群兇既快于天誅,大憝須垂于鬼錄。今則選搜兵甲,簡練車徒,乘勝長驅,翦除元惡。凡爾魏、博、邢、洺之眾,感恩懷義之人,乃祖乃孫,為聖唐赤子,豈徇虎狼之黨,遂忘覆載之恩。蓋以封豕長蛇,憑陵薦食,無方逃難,遂被脅從。空嘗膽以銜冤,竟無門而雪憤,既聞告捷,想所慰懷。今義旅徂征,止于招撫。昔耿純焚廬而向順,蕭何舉族以從軍,皆審料興亡,能圖富貴,殊勳茂業,翼子貽孫,轉禍見機,決在今日。若能詣轅門而效順,開城堡以迎降,長官則改補官資,百姓則優加賞賜,所經詿誤,更不推窮。三鎮諸軍,已申嚴令,不得焚燒廬舍,剽掠馬牛,但仰所在生靈,各安耕織。予恭行天罰,罪止元凶,已外歸明,一切不問,
 凡爾士眾,咸諒予懷。」帝御親軍南征。庚子,至洺州,梁祖令其將徐仁浦將兵五百,夜入邢州。張承業、李存璋以三鎮步兵攻邢州,遣周德威、史建瑭將三千騎,長驅至澶魏,帝與李嗣源率親軍繼進。



 二月戊午,師次洹水,周德威進至臨河。己未,魏帥羅周翰出兵五千,塞石灰窯口,周德威以騎掩擊,迫入觀音門。是日,王師迫魏州,帝舍于狄公祠西。周翰閉壁自固,帝軍攻之,其城幾陷。帝歎曰:「予為兒童時,從先王渡河,今其忘矣。方春桃花水滿,思一觀之,誰從予者?」癸亥,帝觀河于黎陽。是時,梁祖發兵萬餘將
 渡河,聞王師至,棄舟而退。黎陽都將張從楚、曹儒以部下兵三千人來降,立其軍為左右匡霸使。乙丑,周德威自臨清徇地貝郡,攻博州,下東武、朝城。時澶州刺史張可臻棄城而遁,遂攻黎陽,下臨河、淇門。庚午,梁祖在洛,聞王師將攻河陽,率親軍屯白馬坡。壬申,帝下令班師。帝至趙州,王鎔迎謁。翼日,大饗諸軍。壬午,帝發趙州,歸晉陽,留周德威戍趙州。



 三月己丑,鎮、定州各遣使言幽州劉守光兇僭之狀,請推為尚父,以稔其惡。乙未,帝至
 晉陽宮,召監軍張承業諸將等議幽州之事,乃遣牙將戴漢超齎墨制并六鎮書,推劉守光為尚書令、尚父;守光由是凶熾日甚,遂邀六鎮奉冊。



 五月,六鎮使至幽州,梁使亦集。《通鑒考異》引《莊宗實錄》云:三月己丑,鎮州遣押衙劉光業至,言劉守光兇淫縱毒,欲自尊大,請稔其惡以咎之,推為尚父。乙未,上至晉陽宮,召張承業諸將等議討燕之謀,諸將亦云宜稔其惡。上令押衙戴漢超持墨制及六鎮書如幽州,其辭曰:「天祐八年三月二十七日,天德軍節度使宋瑤、振武節度使周德威、昭義節度使李嗣昭、易定節度使王處直、鎮州節度使王鎔、河東節度使尚書令晉王謹奉冊進盧龍橫海等軍節度、檢校大尉、中書令、燕王為尚書令、尚父。」五月,六鎮使至,汴使亦集。六月,守光令有司定尚父、採
 訪使議。是月,梁祖遣都招討使楊師厚將兵三萬屯邢州,帝令李嗣昭出師掠相、衛而還。



 秋七月,帝會王鎔于承天軍。鎔,武皇之友也,帝奉之盡敬,捧卮酒為壽,鎔亦捧酒醻帝。鎔幼子昭誨從行,因許為婚。八月甲子,幽州劉守光僭稱大燕皇帝,年號應天。九月庚子,梁祖將親軍自洛渡河而北,至相州,聞帝軍未出,乃止。十月,幽州劉守光殺帝之行人李承勳,忿其不行朝禮也。



 十一月辛丑,燕人侵易、定,王處直來告難。十二月甲子,帝遣周德
 威、劉光浚、李嗣源及諸將率蕃漢之兵發晉陽,伐劉守光于幽州。



\end{pinyinscope}