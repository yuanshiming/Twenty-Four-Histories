\article{武皇紀上}

\begin{pinyinscope}
太祖武皇帝,諱克用,
 本姓硃耶氏,其先隴右金城人也。
 始祖拔野,唐貞觀中為墨離軍使,從太宗討高麗、薛延陀有功,為金方道副都護,因家於瓜州。太宗平薛延陀諸部,於安西、北庭置都護屬之,分同
 羅、
 僕骨之人,置沙陀督府。蓋北庭有磧曰沙陀,故因以為名焉。永徽中,以拔野為都督,其後子孫五世相承。曾祖盡忠,貞元中,繼為沙陀府都督。既而為吐蕃所陷,乃舉其族七千帳徙于甘州。盡忠尋率部眾三萬東奔,俄而吐蕃追兵大至,盡忠戰歿。祖執宜,即盡忠之長子也,收合餘眾,至於靈州,德宗命為陰山府都督。元和初,入為金吾將軍,遷蔚州刺史、代北行營招撫使。《新唐書·沙陀傳》:元和三年,盡忠款靈州塞,詔處其部
 鹽州置陰山府,以執宜為府兵馬使。朝長安,授特進、金吾衛將軍。從攻鎮州,進蔚州刺史。破吳元濟,授檢校刑部尚書。長慶初,破賊深州,入朝留宿衛,拜金吾衛將軍。太和中,授陰山府都督、代北行營招撫使。莊宗即位,追謚為昭烈皇帝,廟號懿祖。烈考國昌,本名赤心,唐朔州刺史。咸通中,討龐勛有功,入為金吾上將軍,賜姓李氏,名國昌,
 \gezhu{
  代州有《唐故龍武軍統軍檢校司徒贈太保隴四李公神道碑》云:公諱國昌,字德興。}
 仍係鄭王房。出為振武節度使,尋為吐渾所襲,退保于神武川。武皇鎮太原,表為代北軍節度使。中和三年薨。莊宗即位,追謚為文皇,廟號獻祖。



 武皇即獻祖之第
 三子也。母秦氏,以大中十年丙子歲九月二十二日,生于神武川之新城。在妊十三月,載誕之際,母艱危者竟夕。族人憂駭,市藥于雁門,遇神叟告曰:「非巫醫所及,可馳歸,盡率部人,被甲持旄,擊鉦鼓,躍馬大噪,環所居三周而止。」族人如其教,果無恙而生。是時,虹光燭室,白氣充庭,井水暴溢。武皇始言,喜軍中語,齠齔善騎射,與儕類馳騁嬉戲,必出其右。年十三,見雙鳧翔于空,射之連中,眾皆臣伏。新城北有毗沙天王祠,祠前井一日沸溢,
 未察其禍福,惟天王若有神奇,可與僕交卮談。」奠酒未已,有神人被金甲持戈,隱然出於壁間,見者大驚走,惟武皇從容而退,由是益自負。



 獻祖之討龐勛也,武皇年十五,從征,摧鋒陷陣,出諸將之右,軍中目為「飛虎子」。賊平,獻祖授振武節度使,武皇為雲中牙將。嘗在雲中,宿於別館,擁妓醉寢,有俠兒持刃欲害武皇;及突入曲室,但見烈火熾赫於帳中,俠兒駭異而退。又嘗與達靼部人角勝,達靼指雙雕于空曰:「公能一發中否?」武皇即彎
 弧發矢,連貫雙雕,邊人拜伏。及壯,為雲中守捉使,事防禦使支謨,與同列晨集廨舍,因戲升郡閣,踞謨之座,謨亦不敢詰。



 乾符三年,朝廷以段文楚為代北水陸發運、雲州防禦使。時歲薦飢,文楚稍削軍食,諸軍咸怨。武皇為雲中防邊督將,部下爭訴以軍食不充,邊校程懷素、王行審、蓋寓、李存璋、薛鐵山、康君立等,即擁武皇入雲州,眾且萬人,營於鬥雞臺,城中械文楚出,以應於外。諸將列狀以聞,請授武皇旄鉞,朝廷不允,徵諸道兵以討
 之。



 乾符五年,黃巢渡江,其勢滋蔓,天子乃悟其事,以武皇為大同軍節度使、檢校工部尚書。冬,獻祖出師討黨項,吐渾赫連鐸乘虛陷振武,舉族為吐渾所擄。武皇至定邊軍迎獻祖歸雲州,雲州守將拒關不納。武皇略蔚、朔之地,得三千人,屯神武川之新城。赫邊鐸晝夜攻圍,武皇昆弟三人四面應賊,俄而獻祖自蔚州引軍至,吐渾退走,自是軍勢復振。天子以赫連鐸為大同軍節度使,仍命進軍以討武皇。



 乾符六年春,朝廷以昭義節度
 使李鈞充北面招討使,將上黨、太原之師過石嶺關,屯于代州,與幽州李可舉會赫連鐸同攻蔚州。獻祖以一軍禦之,武皇以一軍南抵遮虜城以拒李鈞。是冬大雪,弓弩弦折,南軍苦寒,臨戰大敗,奔歸代州,李鈞中流矢而卒。



 廣明元年春,天子復命元帥李涿率兵數萬屯代州。武皇令軍使傅文達起兵於蔚州,朔州刺史高文集與薛葛、安慶等部將縛文達送於李涿。六月,李涿引大軍攻蔚州,獻祖戰不利,乃率其族奔於達靼部。居數月,吐
 渾赫連鐸密遣人賂達靼以離間獻祖,既而漸生猜阻。武皇知之,每召其豪右射獵於野,或與之百步馳射馬鞭,或以懸針樹葉為的,中之如神,由是部人心伏,不敢竊發。俄而黃巢自江、淮北渡,武皇椎牛釃酒,饗其酋首。酒酣,諭之曰:「予父子為賊臣讒間,報國無由。今聞黃巢北犯江、淮,必為中原之患。一日天子赦宥,有詔徵兵,僕與公等向南而定天下,是予心也。人生世間,光景幾何,曷能終老沙堆中哉!公等勉之。」達靼知無留意,皆釋然
 無間。


是歲十一月,黃巢寇潼關,天子令河東監軍陳景思為代北起軍使,收兵破賊。十二月,黃巢犯長安,僖宗幸蜀,陳景思與李友金發沙陀諸部五千騎南赴京師。友金即武皇之族父也。
 \gezhu{
  《通鑒》:友金初與高文集並降於李琢,故得與陳景思南赴京師。}


中和元年二月,友金軍至絳州,將渡河,刺史瞿稹謂陳景思曰:「巢賊方盛,不如且還代北,徐圖利害。」四月,友金旋軍鴈門,瞿稹至代州,半月之間,募兵三萬,營于崞縣之西。其軍皆北邊五部之眾,不閑軍法,瞿稹、李友金不
 能制。友金謂景思曰:「興大眾,成大事,當威名素著,則可以伏人。今軍雖數萬,茍無善帥,進亦無功。吾兄李司徒父子,去歲獲罪於國家,今寄北部,雄武之略,為眾所推。若驃騎急奏召還,代北之人,一麾響應,則妖賊不足平也。」景思然之,促奏行在。天子乃以武皇為鴈門節度使,仍令以本軍討賊。
 \gezhu{
  《新唐書·王重榮傳》:重榮懼黃巢復振,憂之,與復光計,復光曰:「我世與李克用共憂患,其人忠不顧難,死義如己,若乞師焉,事蔑不濟。」乃遣使者約連和。}
 李友金發五百騎齎詔召武皇於達靼,武皇即率達靼諸部萬人趨鴈
 門。五月,整兵二萬,南嚮京師。太原鄭從讜以兵守石嶺關,武皇乃引軍出他道;至太原城下,會大雨,班師於鴈門。



 中和二年八月,獻祖自達靼部率其族歸代州。十月,武皇率忻、代、蔚、朔、達靼之軍三萬五千騎赴難於京師。先移檄太原,鄭從讜拒關不納,武皇以兵擊之,進軍至城下,遣人齎幣馬遺從讜;從讜亦遣人饋武皇貨幣、饗餼、軍器。武皇南去,自陰地趨晉、絳。十二月,武皇至河中。



 中和三年正月,晉國公王鐸承制授武皇東北面行營
 都統。武皇令其弟克修領前鋒五百騎渡河視賊,黃巢遣將米重威齎重賂及偽詔以賜武皇;武皇納其賂以給諸將,燔其偽詔。是時,諸道勤王之師雲集京畿學說。,然以賊勢尚熾,未敢爭鋒。及武皇將至,賊帥相謂曰:「鴉兒軍至,當避其鋒。」武皇以兵自夏陽濟河。二月,營于乾坑店。黃巢大將尚讓、林言、王璠、趙璋等引軍十五萬屯於梁田坡。翼日,大軍合戰,自午及晡,巢賊大敗。是夜,賊眾遁據華州。武皇進軍圍之,巢弟黃鄴、黃揆固守。三月,尚讓
 引大軍赴援,武皇率兵萬餘逆戰於零口,巢軍大敗,武皇進軍渭橋。翼日,黃揆棄華州而遁。王鐸承制授武皇鴈門節度使、檢校尚書左僕射。四月,黃巢燔長安,收其餘眾,東走藍關。武皇時收京師。七月,天子授武皇金紫光祿大夫、檢校左僕射、河東節度使。《舊唐書·僖宗紀》:五月,制以鴈門以北行營節度、忻代蔚朔等州觀察處置等使、檢校尚書左僕射、代州刺史、上柱國、食邑七百戶李克用檢校司空、同平章事兼太原尹、北京留守,充河東節度、管內觀察處置等使。《新唐書·沙陀傳》云:收京師功第一,進同中書門下平章事、隴西郡公。未幾,以克用領河東節度。



 是時,武皇既收長安,軍勢甚
 雄,諸侯之師皆畏之。武皇一目微眇,故其時號為「獨眼龍」。是月,武皇仗節赴鎮。遣使報鄭從讜,請治裝歸朝。武皇次於郊外,因往赴鴈門寧覲獻祖。八月,自鴈門赴鎮河東,時年二十有八。十一月,平潞州,表其弟克修為昭義節度使。潞帥孟方立退保於邢州。



 十二月,許帥田從異、汴帥朱溫、徐帥時溥、陳州刺史趙犨各遣使來告,以巢、蔡合從,凶鋒尚熾,請武皇共力討賊。



 中和四年春,武皇率蕃漢之師五萬,自澤、潞將下天井關;河陽節度使
 諸葛爽辭以河橋不完,乃屯兵於萬善。數日。移軍自河中南渡,趨汝、洛。四月,武皇合徐、汴之師破尚讓于太康,斬獲萬計,進攻賊於西華,賊將黃鄴棄營而遁。是夜大雨,巢營中驚亂,乃棄西華之壘,退營陳州北故陽里。五月癸亥,大雨震電,平地水深數尺,賊營為水所漂而潰。戊辰,武皇引軍營於中牟,大破賊於王滿渡。庚午,巢賊大至,濟汴而北。是夜,復大雨,賊黨驚潰。武皇營于鄭州,賊眾分寇汴境。武皇渡汴,遇賊將渡而南,半濟擊之,大
 敗之,臨陣斬賊將李周、王濟安、陽景彪等。是夜,賊大敗,殘眾保於胙縣、冤句。大軍躡之,黃巢乃攜妻子兄弟千餘人東走,武皇追賊至於曹州。



 是月,班師過汴,汴帥迎勞于封禪寺,請武皇休于府第,乃以從官三百人及監軍使陳景思館於上源驛。是夜,張樂陳宴席,汴帥自佐饗,出珍幣侑勸。武皇酒酣,戲諸侍妓,與汴帥握手,敘破賊事以為樂。汴帥素忌武皇,乃與其將楊彥洪密謀竊發,彥洪于巷陌連車樹柵,以扼奔竄之路。時武皇之從
 官皆醉,俄而伏兵竄發,來攻傳舍。武皇方大醉,噪聲動地,從官十餘人捍賊。侍人郭景銖滅燭扶武皇,以茵幕裹之,匿于床下,以水灑面,徐曰:「汴帥謀害司空!」武皇方張目而起,引弓抗賊。有頃,煙火四合,復大雨震電,武皇得從者薛鐵山、賀回鶻等數人而去。雨水如澍,不辨人物,隨電光登尉氏門,縋城而出,得還本營。監軍陳景思、大將史敬思並遇害。武皇既還營,與劉夫人相向慟哭。詰旦,欲勒軍攻汴,夫人曰:「司空比為國家討賊,赴東諸
 侯之急,雖汴人謀害,自有朝廷論列。若反戈攻城,則曲在我也,人得以為辭。」乃收軍而去,馳檄於汴帥。汴帥報曰:「竊發之夜,非僕本心,是朝廷遣天使與牙將楊彥洪同謀也。」武皇自武牢關西趨蒲、陜而旋。秋七月,至太原。武皇自以累立大功,為汴帥怨圖,陷沒諸將,乃上章申理。及武皇表至,朝廷大恐,遣內臣宣諭,尋加守太傅、同平章事、隴西郡王。



 光啟元年三月,幽州李可舉、鎮州王景崇連兵寇定州,節度使王處存求援於武皇;武皇遣
 大將康君立、安老、薛可、郭啜率兵赴之。五月,鎮人攻無極,武皇親領兵救之。鎮人退保新城,武皇攻之,斬首萬餘級,獲馬千匹。王處存亦敗燕軍於易州。


十一月,河中王重榮遣使來乞師,且言邠州朱玫、鳳翔李符將加兵于己。初,武皇與汴人構怨,前後八表,請削奪汴帥官爵,自以本軍進討。天子累遣內臣楊復恭宣旨,令且全大體,武皇不時奉詔,天子頗右汴帥。時觀軍容使田令孜君側擅權,惡王重榮與武皇膠固,將離其勢,乃移重榮
 于定州。重榮告于武皇,武皇上章言:「李符、朱玫挾邪忌正,黨庇朱溫。臣已點檢蕃漢軍五萬,取來年渡河,先斬朱玫、李符,然後平蕩朱溫。」
 \gezhu{
  《新唐書·王重榮傳》:詔克用將兵援河中,重榮貽克用書,且言:「奉密詔,須公到,使我圖公,此令孜、朱全忠、朱玫之惑上也。」因示偽詔。克用方與全忠有隙,信之,請討全忠及玫。}
 天子覽表,遣使譬喻百端,軺傳相望。既而朱玫引邠、鳳之師攻河中,王重榮出師拒戰。朱玫軍於沙苑,對壘月餘。十二月,武皇引軍渡河,與朱玫決戰,玫大敗,收軍夜遁,入於京師。時京城大駭。天子幸鳳翔,武皇退軍于
 河中。



 光啟二年正月,僖宗駐蹕於寶雞,武皇自河中遣使上章,請車駕還京,且言大軍止誅凶黨。時田令孜請僖宗南幸興元,武皇遂班師。朱玫于鳳翔立嗣襄王煴為帝,以偽詔賜武皇,武皇燔之,械其使,馳檄諸方鎮,遣使奉表於行在。九月,武皇遣昭義節度使李克修討孟方立於邢州,大敗方立之眾於焦崗,斬首數千級。以大將安金俊為邢州刺史,以撫其降人。十月,進攻邢州,邢人出戰,又敗之。盂方立求援於鎮州,鎮人出兵三萬
 以援方立,克修班師。



 光啟三年六月,河中節度使王重榮為部將常行儒所殺,武皇表重榮兄重盈為帥。七月,武皇以安金俊為澤州刺史。時張全義自河陽據澤州,及李罕之收復河陽,召全義令守洛陽,全義乃棄澤州而去,故以金俊守之。



 文德元年二月,僖宗自興元還京。三月,僖宗崩,昭宗即位,以武皇為開府儀同三司、檢校太師、兼侍中、隴西郡王,食邑七千戶,食實封二百戶。河南尹張全義潛兵夜襲李罕之於河陽,城陷,舉族為全
 義所擄;罕之踰垣獲免,遂來歸於武皇。遣李存孝、薛阿檀、史儼兒、安金俊、安休休將七千騎送罕之至河陽。汴將丁會、牛存節、葛從周將兵赴援,李存孝率精騎逆戰于溫縣。汴人既扼太行之路,存孝殿軍而退。騎將安休休以戰不利,奔於蔡。武皇以罕之為澤州刺史,遙領河陽節度使。十月,邢州孟方立遣大將奚忠信將兵三萬寇遼州,武皇大破之,斬首萬級,生擒奚忠信。



 龍紀元年五月,遣李罕之、李存孝攻邢州。六月,下磁州。邢將馬溉
 率兵數萬來拒戰,罕之敗之於琉璃陂,生擒馬溉,徇於城下。孟方立恚恨,飲鴆而死。三軍立其侄遷為留後,使求援于汴。汴將王虔裕率精甲數百入于邢州,罕之等班師。



 大順元年,遣李存孝攻邢州,孟遷以邢、洺、磁三州降,執汴將王虔裕三百人以獻。武皇徙孟遷於太原,以安金俊為邢洺團練使。三月,昭義軍節度使李克修卒,以李克恭為潞州節度使。是月,武皇攻雲州,拔其東城。赫連鐸求援於燕,燕帥李匡威將兵三萬以赴之,戰於
 城下,燕軍大敗。時徐州時溥為汴軍所攻,遣使來求援,武皇命石君和由兗、鄆以赴之。



 五月,潞州軍亂,殺節度使李克恭,州人推牙將安居受為留後,南結汴將。時潞之小將馮霸擁叛徒三千騎駐于沁水,居受使人召之,馮霸不至。居受懼,出奔至長子,為村胥所殺,傳首于霸;霸遂入潞州,自為留後。武皇遣大將康君立、李存孝等攻之,汴將朱崇節、葛從周率兵入潞州以固之。是時,幽州李匡威、雲州赫連鐸與汴帥協謀,連上表請加兵於
 太原,宰相張濬、孔緯贊成其事。六月,天子削奪武皇官爵,以張浚為招討使,以京兆尹孫揆為副,華州韓建為行營都虞候,以汴帥為河東南面招討使,幽州李匡威為河東北面招討使,雲州赫連鐸為副。汴將朱友裕將兵屯晉、絳,時汴軍已據潞州,又遣大將李讜等率軍數萬,急攻澤州,武皇遣李存孝自潞州將三千騎以援之。汴將鄧季筠以一軍犯陣,存孝追擊,擒其都將十數人,獲馬千餘匹。是夜,李讜收軍而退,大軍掩擊至馬牢關,
 斬首萬餘級,追襲至懷州而還。存孝復引軍攻潞州。



 八月,存孝擒新授昭義節度使孫揆。初,朝廷授揆節鉞,以本軍取刀黃嶺路赴任,存孝偵知之,引騎三百伏於長子縣崖谷間。揆建牙持節,褒衣大蓋,擁眾而行,存孝突出谷口,遂擒揆及中使韓歸範,并將校五百人。存孝械揆等,以組練繫之,環於潞州,遂獻於武皇。武皇謂揆曰:「公搢紳之士,安言徐步可至達官,何用如是!」揆無以對,令繫於晉陽獄。武皇將用為副使,使人誘之,揆言不遜,
 遂殺之。



 九月,汴將葛從周棄潞州而遁,武皇以康君立為潞州節度使,以李存孝為汾州刺史。十月,張浚之師入晉州,游軍至汾、隰。武皇遣薛鐵山、李承嗣將騎三千出陰地關,營於洪洞,遣李存孝將兵五千,營於趙城。華州韓建以壯士三百人冒犯存孝之營,存孝追擊,直壓晉州西門,張浚之師出戰,為存孝所敗,自是閉壁不出。存孝引軍攻絳州。十二月,晉州刺史張行恭棄城而奔,韓建、張濬由含山路遁去。


大順二年春正月,武皇上章
 申理,其略曰:「臣今身無官爵,名是罪人,不敢歸陛下籓方,且欲於河中寄寓,進退行止,伏候聖裁。」天子尋就加守中書令。
 \gezhu{
  《歐陽史》:二月,復拜克用河東節度使、隴西郡王,加檢校太師、兼中書令。}
 是月,魏博為汴將葛從周所寇,節度使羅宏信遣使來求援,武皇出師以赴之。



 三月,邢州節度使安知建叛,奔青州。天子以知建為神武統軍,自棣州溯河歸朝。鄆州朱瑄邀斬于河上,傳首晉陽。以李存孝為邢州節度使。四月,武皇大舉兵討赫連鐸于雲州,遣騎將薛阿檀率前軍以進
 攻,武皇設伏兵于御河之上,大破之,因塹守其城。七月,武皇進軍柳會,赫連鐸力屈食盡,奔于吐渾部,遂歸幽州,雲州平。武皇表石善友為大同軍防禦使。邢州節度使李存孝以鎮州王熔託附汴人,謀亂河朔,北連燕寇,請乘雲、代之捷,平定燕、趙,武皇然之。八月,大搜于晉陽,遂南巡澤、潞,略地懷、孟,河陽趙克裕望風送款,請修鄰好。九月,蒐于邢州。十月,李存孝董前軍攻臨城,鎮人五萬營于臨城西北龍尾崗。武皇令李存審、李存賢以步
 軍攻之,鎮人大敗,殺獲萬計,拔臨城,進攻元氏。幽州李匡威以步騎五萬營于鄗邑,以援鎮州,武皇分兵大掠,旋軍邢州。



\end{pinyinscope}