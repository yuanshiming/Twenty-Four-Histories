\article{列傳六}

\begin{pinyinscope}

 王鎔,其先回鶻部人也。遠祖沒諾干,唐至德中,事鎮州節度使王武俊為騎將。武俊嘉其勇干,畜為假子,號王五哥,其後子孫以王為氏。四代祖廷湊,事鎮帥王承宗
 為牙將。長慶初,承宗卒,穆宗命田宏正為成德軍節度使。既而鎮人殺宏正,推廷湊為留後,朝廷不能制,因以旄鉞授之。廷湊卒,子元逵尚文宗女壽安公主。元逵卒,子紹鼎立。紹鼎卒,子景崇立。皆世襲鎮州節度使,並前史有傳。景崇位至太尉、中書令,封常山王,中和二年卒。


鎔即景崇之子也,年十歲,三軍推襲父位。大順中,武皇將李存孝既平邢、洺,因獻謀於武皇,欲兼並鎮、定,乃連年出師以擾鎮之屬邑。鎔苦之,遣使求救於幽州。
 \gezhu{
  《舊唐書》云:
  時天子蒙塵,九州鼎沸,河東節度使李克用虎視山東,方謀吞據。鎔以重賂結納,請以修和好。晉軍討孟方立於邢州,鎔常奉以芻糧。及方立平,晉將李存孝侵鎔於南部,鎔求援於幽州。}
 自是燕帥李匡威頻歲出軍,以為鎔援。時匡威兵勢方盛,以鎔沖弱,將有窺圖之志。


景福二年春,匡威率精騎數萬,再來赴援,會匡威弟匡儔奪據兄位,匡威退無歸路,鎔乃延入府第,館於寶壽佛寺。鎔以匡威因己而失國,又感其援助之力,事之如父。五月,鎔謁匡威於其館,匡威陰遣部下伏甲劫鎔,抱持之。鎔曰:「公戒部人勿造次。吾國為晉人所
 侵,垂將覆滅,賴公濟援之力,幸而獲存。今日之事,本所甘心。」即並轡歸府舍。鎔軍拒之,竟殺匡威。鎔本疏瘦,時年始十七,當與匡威並轡之時,電雨驟作,屋瓦皆飛。有一人於缺垣中望見鎔,鎔就之,遽挾於馬上,肩之而去。翼日,鎔但覺項痛頭偏,蓋因為有力者所挾,不勝其苦故也。既而訪之,則曰墨君和,乃鼓刀之士也,遂厚賞之。
 \gezhu{
  《太平廣記》引《劉氏耳目記》云:真定墨君和,幼名三旺。眉目棱岸,肌膚若鐵,年十五六。趙王鎔初即位,曾見之,悅而問曰:「此中何得昆侖兒也?」問其姓,與形質相應,即呼為墨昆侖,因以皁衣賜之。是時,常山縣邑屢為並州中
  軍所侵掠,趙之將卒疲於戰敵。告急於燕王,李匡威率師五萬來救之。並人攻陷數城。燕王聞之,躬領五萬騎徑與晉師戰於元氏,晉師敗績。趙王感燕王之德,椎牛灑酒,大犒於稾城,輦金二十萬以謝之。燕王歸國,比及境上,為其弟匡儔所拒,趙人以其有德於我,遂營東圃以居之。燕王自以失國,又見趙王之幼,乃圖之。遂伏甲俟趙王,旦至,即使擒之。趙王請曰:「某承先代基構,主此山河,每被鄰寇侵漁,困於守備,賴大王武略,累挫戎鋒,獲保宗祧,實資恩力。顧惟幼懦,夙有卑誠,望不匆匆,可伸交讓。願與大王同歸衙署,即軍府必不拒違。」燕王以為然,遂與趙王並轡而進。俄有大風並黑雲起于城上,大雨雷電,至東角門內,有勇夫袒臂旁來,拳毆燕之介士,即挾負趙王踰垣而走,遂得歸公府。問其姓名,君和恐其難記,但言曰:「硯中之物。」王心志之。左右軍士既見主免難,遂逐燕王。燕王退走于東圃,趙人圍而殺之。趙王召墨生以千金賞之,兼賜上第一區,良田萬畝,仍恕
  其十死,奏授光祿大夫。}


鎔既失燕軍之援,會武皇出師以逼真定,鎔遣使謝罪,出絹二十萬匹,及具牛酒犒軍,自是與鎔俱修好如初。洎梁祖兼有山東,虎視天下,鎔卑辭厚禮,以通和好。
 \gezhu{
  《新唐書》:羅紹威諷鎔絕太原,共尊全忠,鎔依違,全忠不悅。}
 光化三年秋,梁祖將吞河朔,乃親征鎮、定,縱其軍燔鎮之關城。鎔謂賓佐曰:「事急矣,謀其所向。」判官周式者,有口辯,出見梁祖。梁祖盛怒,逆謂式曰:「王令公朋附并汾,違盟爽信,敝賦業已及此,期于無捨!」式曰:「公為唐室之桓、文,當以禮義
 而成霸業,反欲窮兵黷武,天下其謂公何!」《新唐書》:李嗣昭攻洺州,全忠自將擊走之,得鎔與嗣昭書,全忠怒,引軍攻鎔。周式請見全忠,全忠即出書示式曰:「嗣昭在者,宜速遣。」式曰:「王公所與和者,息人鋒鏑間耳。況繼奉天子詔和解,能無一番紙墜北路乎?太原與趙本無恩,嗣昭庸肯入耶!」梁祖喜,引式袂而慰之曰:「前言戲之耳!」即送牛酒貨幣以犒軍。式請鎔子昭祚及大將梁公儒、李宏規子各一人往質于汴。梁祖以女妻昭祚。及梁祖稱帝,鎔不得已,行其正朔。



 其後梁祖常慮河朔悠久難制,會羅紹威卒,因欲除移鎮、定。先遣親軍三千,分據鎔深、冀二郡,以鎮
 守為名。又遣大將王景仁、李思安率師七萬,營于柏鄉。鎔遣使告急莊宗,莊宗命周德威率兵應之;鎔復奉唐朝正朔,稱天祐七年。及破梁軍于高邑,我軍大振,自是遣大將王德明率三十七都從莊宗征伐,收燕降魏,皆預其功,然鎔未嘗親軍遠出。八年七月,鎔至承天軍,與莊宗合宴同盟,奉觴獻壽,以申感概。莊宗以鎔父友,曲加敬異,為之聲歌,鎔亦報之,謂莊宗為四十六舅。中飲,莊宗抽佩刀斷衿為盟,許女妻鎔子昭誨。因茲堅附於
 莊宗矣。



 鎔自幼聰悟,然仁而不武,征伐出于下,特以作籓數世。專制四州,高屏塵務,不親軍政,多以閹人秉權,出納決斷,悉聽所為。皆雕靡第舍,崇飾園池,植奇花異木,遞相誇尚。人士皆裒衣博帶,高車大蓋,以事嬉遊,籓府之中,當時為盛。鎔宴安既久,惑于左道,專求長生之要,常聚緇黃,合煉仙丹,或講說佛經,親受符籙。西山多佛寺,又有王母觀,鎔增置館宇,雕飾土木。道士王若訥者,誘鎔登山臨水,訪求仙跡,每一出,數月方歸,百姓勞
 弊。王母觀石路既峻,不通輿馬,每登行,命僕妾數下人維錦繡牽持而上。有閹人石希蒙者,姦寵用事,為鎔所嬖,恒與之臥起。



 天祐八年冬十二月,鎔自西山回,宿於鶻營莊,將歸府第,希蒙勸之他所。宦者李宏規謂鎔曰:「方今晉王親當矢石,櫛沐風雨,王殫供軍之租賦,為不急之遊盤,世道未夷,人心多梗,久虛府第,遠出遊從,如樂禍之徒,翻然起變,拒門不納,則王欲何歸!」鎔懼,促歸。希蒙譖宏規專作威福,多蓄猜防,鎔由是復無歸志。宏
 規聞之怒,使親事偏將蘇漢衡率兵擐甲遽至鎔前,露刃謂鎔曰:「軍人在外已久,願從王歸。」宏規進曰:「石希蒙說王遊從,勞弊士庶,又結構陰邪,將為大逆。臣已偵視情狀不虛,請王殺之,以除禍本。」鎔不聽。宏規因命軍士聚噪,斬希蒙首抵于前。鎔大恐,遂歸。是日,令其子昭祚與張文禮以兵圍李宏規及行軍司馬李藹宅,並族誅之,詿誤者凡數十家。又殺蘇漢衡,收部下偏將下獄,窮其反狀,親軍皆恐,復不時給賜,眾益懼。文禮因其
 反側,密諭之曰;「王將坑爾曹,宜自圖之。」眾皆掩泣相謂曰:「王待我如是,我等焉能效忠?」是夜,親事軍十餘人,自子城西門逾垣而入,鎔方焚香受籙,軍士二人突入,斷其首,袖之而出,遂焚其府第,煙焰亙天,兵士大亂。鎔姬妾數百,皆赴水投火而死。軍校有張友順者,率軍人至張文禮之第,請為留後。遂盡殺王氏之族。鎔于昭宗朝賜號敦睦保定久大功臣,位至成德軍節度使、守太師、中書令、趙王,梁祖加尚書令。初,鎔之遇害,不獲其屍,及
 莊宗攻下鎮州,鎔之舊人于所焚府第灰間方得鎔之殘骸。莊宗命幕客致祭,葬於王氏故塋。



 鎔長子昭祚,亂之翼日,張文禮索之,斬于軍門。次子昭誨。當鎔被禍之夕,昭誨為軍人攜出府第,置之地穴十餘日,乃髡其髮,被以僧衣。屬湖南綱官李震南還,軍士以昭誨托于震,震置之茶褚中。既至湖湘,乃令依南嶽寺僧習業,歲給其費。昭誨年長思歸,震即齎送而還。時鎔故將符習為汴州節度使,會昭誨來投,即表其事曰:「故趙王王鎔小
 男昭誨,年十餘歲遇禍,為人所匿免,今尚為僧,名崇隱,謹令赴闕。」明宗賜衣一襲,令脫僧服。頃之,昭誨稱前成德軍中軍使、檢校太傅,詣中書陳狀,特授朝議大夫、檢校考功郎中、司農少卿,賜金紫。符習因以女妻之。其後,累歷少列,周顯德中,遷少府監。



 王處直。《王處直傳》,原本止存王都廢立之事,而處直事闕佚。今考《舊唐書》列傳云:處直,字允明,處存母弟也。初為定州後院軍都知兵馬使,汴人入寇,處直拒戰,不利而退,三軍大噪,推處直為帥,乃權知留後事。汴將張存敬攻城,梯衝雲合,處直登城呼曰:「敝邑於朝廷未嘗不忠,于籓鄰未嘗失禮,不虞君之涉吾地,何也?」
 朱溫使人報之曰:「何以附太原而弱鄰道?」處直報曰:「吾兄與太原同時立勳王室,地又親鄰,修好往來,常道也。請從此改圖。」溫許之,仍歸罪于孔目吏梁問,出絹十萬匹,牛酒以犒汴軍,存敬修盟而退;溫因表授旄鉞、檢校左僕射。天祐元年,加太保,封太原王。後仕偽梁,授北平王、檢校太尉,不數歲,復歸于莊宗。後十餘年,為其子都廢歸私第,尋卒,年六十一。



 王都,本姓劉,小字云郎,中山陘邑人也。初,有妖人李應之得於村落間,養為己子。及處直有疾,應之以左道醫之,不久病間,處直神之,待為羽人。始假幕職,出入無間,漸署為行軍司馬,軍府之事,咸取決焉。處直時未有子,
 應之以都遺於處直曰:「此子生而有異。」因是都得為處直之子。其後應之閱白丁于管內,別置新軍,起第于博陵坊,面開一門,動皆鬼道。處直信重日隆,將校相慮,變在朝夕,謀先事為禍。會燕師假道,伏甲于外城,以備為不虞,昧旦入郭,諸校因引軍以圍其第,應之死于亂兵,咸云不見其屍,眾不解甲。乃逼牙帳請殺都,處直堅靳之,久乃得免。翼日賞勞,籍其兵于臥內,自隊長已上記于別簿,漸以他事孥戮。迨二十年,別簿之記,略無孑遺。都
 既成長,總其兵柄,姦詐巧佞,生而知之。處直愛養,漸有付託之意,時處直諸子尚幼,乃以都為節度副大使。



 王郁者,亦處直之孽子也。案:以下有闕文。



 天祐十八年十二月,莊宗親征鎮州,敗契丹于沙河。明年正月,乘勝追敵,過定州,都馬前奉迎,莊宗幸其府第曲宴。都有愛女,十餘歲,莊宗與之論婚,許為皇子繼岌妻之。自是恩寵特異,奏請無不從。同光三年,莊宗幸鄴都,都來朝覲,留宴旬日,錫賚鉅萬,遷太尉、侍中。時周元豹見之曰;「形若鯉魚,難
 免刀匕。」及明宗嗣位,加中書令,然以其奪據父位,深心惡之。



 初,同光中,祁、易二州刺史,都奏部下將校為之,不進戶口,租賦自贍本軍,天成初仍舊。既而安重誨用事,稍以朝政釐之。時契丹犯塞,諸軍多屯幽、易間,大將往來,都陰為之備,屢廢迎送,漸成猜間。和昭訓為都籌畫曰:「主上新有四海,其勢易離,可圖自安之計。」會朱守殷據汴州反,鎮州節度使王建立與安重誨不協,心懷怨嫉。都陰知之,乃遣人說建立謀叛,建立偽許之,密以狀
 聞。都又與青、徐、岐、潞、梓五帥蠟書以離聞之。



 三年四月,制削都在身官爵,遣宋州節度使王晏球率師討之。都急與王郁謀,引契丹為援。洎王師攻城,契丹將托諾率騎萬人來援,都與契丹合兵大戰于嘉山,為王師所敗,惟托諾以二千騎奔入定州。都仗之守城,呼為諾王,屈身瀝懇,冀其盡力。孤壘周年,亦甚有備,諸校或思歸嚮,以其訪察嚴密,殺人相繼,人無宿謀,故數構不就。



 都好聚圖書,自常山始破,梁國初平,令人廣將金帛收市,以
 得為務,不責貴賤,書至三萬卷,名畫樂器各數百,皆四方之精妙者,萃于其府。四年三月,晏球拔定州,時都校馬讓能降于曲陽門,都巷戰而敗,奔馬歸於府第,縱火焚之,府庫妻孥,一夕俱燼,惟擒托諾并其男四人、弟一人獻于行在。



 李繼陶者,莊宗初略地河朔,俘而得之,收養于宮中,故名曰得得。天成初,安重誨知其本末,付段徊養之為兒;佪知其不稱,許其就便。王都素蓄異志,潛取以歸,呼為莊宗太子。及都叛,遂僭其服裝,時俾乘墉,
 欲惑軍士,人咸知其偽,競詬辱之。城陷,晏球獲之,拘送於闕下,行至邢州,遣使戮焉。



 史臣曰:王鎔據鎮、冀以稱王,治將數世;處直分易、定以為帥,亦既重侯。一則惑佞臣而覆其宗,一則嬖孽子而失其國,其故何哉?蓋富貴斯久,仁義不修,目眩於妖妍,耳惑於絲竹,故不能防奸於未兆,察禍於未萌,相繼敗亡,又誰咎也!



\end{pinyinscope}