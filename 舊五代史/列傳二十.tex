\article{列傳二十}

\begin{pinyinscope}
薛廷珪,其先河東人也。父逢,咸通中為秘書監,以才名著於時。廷珪,中和年在西川登進士第,累歷臺省。
 \gezhu{
  《舊唐書》:大順初,累遷司勛員外郎、知制誥。}
 乾寧中,為中書舍人。駕在華州,改散騎
 常侍,尋請致仕,客游蜀川。昭宗遷洛陽,徵為禮部侍郎。
 \gezhu{
  《舊唐書》:光化中,復為中書舍人,遷邢部、吏部二侍郎,權知禮部貢舉,拜尚書左丞。}
 時柳璨屠害朝士,衣冠畢罹其毒,廷珪以居常退讓獲全。
 \gezhu{
  《新唐書》:硃全忠兼四鎮,廷珪以官告使至汴,客將先見,諷其拜。廷珪佯不曉,曰:「吾何德,敢受令公拜乎!」及見,卒不肯加禮。}
 入梁為禮部尚書。莊宗平定河南,以廷珪年老,除太子少師致仕。
 \gezhu{
  《通鑒》:廷珪與李琪嘗為太祖冊禮使。}
 同光三年九月卒。贈右僕射。所著《鳳閣詞書》十卷、《克家志》五卷,並行于世。初,廷珪父逢,著《鑿混沌》、《真珠簾》等賦,大為時人所稱。廷珪既壯,亦
 著賦數十篇,同為一集,故目曰《克家志》。



 崔沂,《新唐書·宰相世系表》:沂,字德潤。大中時宰相魏公鉉之幼子也。兄沆,廣明初亦為宰輔。沂舉進士第,歷監察、補闕。昭宗時,累遷至員外郎、知制誥。性抗厲守道,而文藻非優,嘗與同舍顏蕘、錢珝俱秉筆,見蕘、珝贍速,草制數十,無妨譚笑,而沂自愧。翌日,謁國相訴曰:「沂疏淺,不足以供詞翰之職。」相輔然之,移為諫議大夫。入梁,為御史司憲,糾繆繩違,不避豪右。開平中,金吾街使寇彥卿入朝,過天津橋,
 市民梁現者不時回避,前導伍伯捽之,投石欄以致斃。彥卿自前白于梁祖,梁祖命通事舍人趙可封宣諭,令出私財與死者之家,以贖其罪。沂奏劾曰:「彥卿位是人臣,無專殺之理。況天津橋御路之要,正對端門,當車駕出入之途,非街使震怒之所。況梁現不時回避,其過止于鞭笞。捽首投驅,深乖朝憲,請論之以法。」梁祖惜彥卿,令沂以過失論,沂引鬥競律,以怙勢力為罪首,下手者減一等。又鬥毆條,不鬥故毆傷人者,加傷罪一等。沂表
 入,責授彥卿游擊將軍、左衛中郎將。沂剛正守法,人士多之。遷左司侍郎,改太常卿,轉禮部尚書。貞明中,帶本官充西京副留守。時張全義留守、天下兵馬副元帥、河南尹、判六軍諸衛事、守太尉、中書令、魏王,名位之重,冠絕中外。沂至府,客將白以副留守合行廷禮,沂曰:「張公官位至重,然尚帶府尹之名,不知副留守見尹之儀何如?」全義知之,遽引見沂,勞曰:「彼此有禮,俱老矣,勿相勞煩。」莊宗興復唐室,復用為左丞,判吏部尚書銓選司,坐
 累謫石州司馬。明宗即位,召還,復為左丞。以衰疾告老,授太子少保致仕。卒于龍門之別墅,時年七十餘。贈太子少傅。



 劉岳,字昭輔。其先遼東襄平人,元魏平定遼東,徙家于代,隨孝文遷洛,遂為洛陽人。八代祖民部尚書渝國公政會,武德時功臣。祖符,蔡州刺史。父珪,洪洞縣令。符有子八人,皆登進士第。珪之母弟瑰、玕,異母弟崇夷、崇龜、崇望、崇魯、崇謨。崇龜,乾寧中廣南節度使;崇望,乾寧中
 宰相;崇魯、崇謨、崇夷,並歷朝省。岳少孤,亦進士擢第,歷戶部巡官、鄭縣簿、直史館,轉左拾遺、侍御史。梁貞明初,召入翰林為學士。岳為文敏速,尤善談諧,在職累遷戶部侍郎,在翰林十二年。莊宗入汴,隨例貶均州司馬,尋丁母憂,許自貶所奔喪,服闋,授太子詹事。明宗即位,歷兵部吏部侍郎、祕書監、太常卿。卒年五十六。贈吏部尚書。岳文學之外,通于典禮。天成中,奉詔撰《新書儀》一部,文約而理當,今行于世。



 子溫叟,仕至御史中丞。《國老談苑》
 云:劉溫叟方正守道,以名教為己任。幼孤,事母以孝聞,其母甚賢。初為翰林學士,私庭拜母,母即命二婢箱擎公服、金帶,置于階下,謂溫叟曰:「此汝父長興中入翰林時所賜也。自先君子薨背以來,嘗懼家門替墜,今汝能自致青雲,繼父之職,可服之無愧矣!」因欷歔掩泣。溫叟伏地號慟,退就別寢,素衣蔬食,追慕數日,然後服之,士大夫以為得禮。



 封舜卿,案:原本有闕文。據《新唐書·宰相世系表》,封氏世居渤海蓚縣。舜卿,字贊聖,父敖,字碩夫,戶部尚書、渤海縣男。《唐書》有傳。仕梁,為禮部侍郎,知貢舉。開平三年,奉使幽州,以門生鄭致雍從行,復命之日,又與致雍同受命入翰林為學士。致雍有俊才,舜卿雖有文辭,才思拙澀,
 及試五題,不勝困弊,因託致雍秉筆,當時譏者以為座主辱門生。案:以下有闕文。莊宗同光已來,累歷清顯。封氏自太和以來,世居兩制,以文筆稱于時。舜卿從子渭,《世系表》:渭,字希叟。昭宗遷洛時,為翰林學士,舜卿為中書舍人,叔姪對掌內外制。



 從子翹,于梁貞明中亦為翰林學士。天成中,為給事中,因轉對上言,以星辰合度,風雨應時,請御前香一合,帝親爇一炷,餘令于塔廟中焚之,貴表精至。議者以翹時推名族,出朝苑,登瑣闈,甚有巖廊之望,而忽
 有此請,乃近諸妖佞耳,物望由是減之。案:以下殘闕。



 竇夢徵,同州人。少苦心為文,登進士第,歷校書郎,自拾遺召入翰林,充學士。梁貞明中,加兩浙錢鏐元帥之命。夢徵以鏐無功於中原,兵柄不宜虛授,其言切直。梁末帝以觸時忌,左授外任。《玉堂閒話》:竇以錢公無功于本朝,僻在一方,坐邀恩澤,不稱是命,乃抱麻哭于朝。翌日,竇謫掾于東州。有頃,復召為學士。及莊宗入汴,夢徵以例貶沂州,居嘗感梁末帝舊恩,因為《祭故君文》云:「嗚呼!四海九州,天回眷命,一女二夫,人之不幸。當革故
 以鼎新,若金銷而火盛,必然之理,夫何足競」云。秉筆者皆許之,尋量移宿州。天成初,遷中書舍人,復入為翰林學士、工部侍郎。卒,贈禮部尚書。《玉堂閑話》:竇失意被謫,嘗鬱鬱不樂,曾夢有人謂曰:「君無自苦,不久當復故職。然將來慎勿為丞相,茍有是命,當萬計避之。」其後竇復居禁職。有頃,遷工部侍郎。竇忽憶夢中所言,深惡其事。然已受命,不能遜避,未幾果卒。夢徵隨計之秋,文稱甚高,尤長于箋啟,編為十卷,目曰《東堂集》,行于世。



 李保殷,河南洛陽人也。昭宗朝,自處士除太子正字,改錢塘縣尉。浙東帥董昌辟為推官,調補河府兵曹參軍,
 歷長水令、《毛詩》博士,累官至太常少卿、端王傅。入為大理卿,撰《刑律總要》十二卷;與兵部侍郎郗殷象論刑法事。左降房州司馬。同光初,授殿中監,以其素有明法律之譽,拜大理卿;未滿秩,屬為人所制。保殷曰:「人之多辟,無自立辟。」乃謝病以歸,卒于洛陽。



 歸藹,字文彥,吳郡人也。曾祖登,祖融,父仁澤,位皆至列曹尚書、觀察使。藹登進士第,及昇朝,遍歷三署。案:以下疑有闕文。據《舊唐書·昭宗紀》:天祐元年七月,宴于文思殿。硃全忠入,百官或坐于廊下,全忠怒,笞通引官何凝。丙寅,制
 金紫光祿大夫、行御史中丞、上柱國韓儀責授棣州司馬,侍御史歸藹責授登州司戶,坐百官傲全忠也。同光初,為尚書右丞,遷刑、戶二部侍郎,以太子賓客致仕,卒年七十六。



 孔邈,文宣王四十一代孫。身長七尺餘,神氣溫厚。登進士第,歷校書郎、萬年尉,充集賢校理,為諫議大夫,以年老致仕。案:《孔邈傳》,原本殘闕。考《冊府元龜》云:乾寧五年,登進士第,除校書郎。崔遠在中書,奏為萬年尉,充集賢校理,以親舅獨孤損方在廊廟,避嫌不赴職。



 張文寶,昭宗朝諫議大夫顗之子也。文寶初,依河中硃
 友謙為從事。莊宗即位于魏州,以文寶知制誥,歷中書舍人、刑部侍郎、左散騎常侍、知貢舉,遷吏部侍郎。文寶性雅淡稽古。長興初,奉使浙中,泛海船壞,水工以小舟救,文寶與副使吏部郎中張絢信風至淮南界,偽吳楊溥禮待甚至,兼厚遺錢幣、食物。文寶受其食物,反其錢幣,吳人善之,送文寶等復至杭州宣國命,還青州,卒。



 子吉,嗣位邑宰。



 陳乂,薊門人也。少好學,善屬文。因避亂,客于浮陽,轉徙
 于大梁。梁將張漢傑延于私邸,表授太子舍人。莊宗平梁,郭崇韜遙領常山,召居賓榻。崇韜從魏王繼岌伐蜀,署為招討判官。崇韜死,明宗即位,隨任圜歸闕,圜薦之于朝,除膳部員外郎、知制誥,累遷中書舍人。乂性陰僻,寡與人合,不為當路所與。尋移左散騎常侍,由是忿以成疾,踰月而卒。



 乂微有才術,嘗自恃其能。為判官日,人有造者,垂帷深處,罕見其面。及居西掖,而姿態愈倨,位竟不至公卿,蓋器度促狹者也。然乂性孤執,尤廉于財。
 長興中,嘗自舍人銜命冊晉國公主石氏于太原,晉高祖善待之,但訝其高岸。人或有獻可于乂,宜陳一謳頌以稱晉高祖之美,可邀其厚賄耳。乂曰:「人生貧富,咸有定分,未有持天子命違禮以求利,既損國綱,且虧士行,乂今生所不為也。」聞者嘉之。晉高祖即位,贈禮部尚書。



 劉贊,魏州人也。幼有文性。父玭,為令錄,誨以詩書,夏月令服青襦單衫。玭每肉食,別置蔬食以飯贊,謂之曰:「肉食,君之祿也。爾欲食肉,當苦心文藝,自可致之,吾祿不
 可分也。」由是贊及冠有文辭,年三十餘登進士第。魏州節度使羅紹威署巡官,罷歸京師,依開封尹劉鄩。久之,租庸使趙巖表為巡官,累遷至戶部員外郎,職如故。莊宗入汴,租庸副使孔謙以贊里人,表為鹽鐵判官。天成中,歷知制誥、中書舍人。與學士竇夢徵同年登第,鄰居友善,夢徵卒,贊與同年楊凝式緦麻為位而哭,其家無嫡長,與視喪事,恤其孀稚,人士稱之。改御史中丞、刑部侍郎。



 贊性雍和,與物無忤,居官畏慎,人若以私干之,雖
 權豪不能移其操。未幾,改祕書監,兼秦王傅。《冊府元龜》:秦王為元帥,秦王府判官、太子詹事王居敏與贊鄉曲之舊,以秦王盛年自恣,須朝中選端士納誨,冀其稟畏,乃奏薦贊焉。贊節概貞素,忽聞其命,掩泣固辭,竟不能止。案《通鑒》:瓚自以左遷,泣訴,不得免。胡三省注云:唐制,六部侍郎除吏部之外,餘皆從四品下;王傅從三品。然六部侍郎為嚮用,王傅為左遷,以職事有閑劇之不同也。當是時,從榮地居儲副,則秦王傅不可以閑官。言蓋以從榮輕佻峻急,恐豫其禍,故求脫耳。時秦王參佐,皆新進小生,動多輕脫,每稱頌秦王功德,阿意順旨,只奉談笑,惟贊從容諷議,必獻嘉言。秦王常接見賓僚及遊客,于酒筵之中,悉令秉筆賦
 詩。《冊府元龜》:時從榮溺於篇章,凡門客及通謁遊士,必坐于客次,自出題目,令賦一章,然後接見。贊為師傅,亦與諸客混,然容狀不悅。秦王知其意,自是戒典客,贊至勿通,令每月一度至衙。《言行龜鑒》載:劉贊諫秦王曰:「殿下宜以孝敬為職,浮華非所尚也。」秦王不悅,戒閽者後弗引進。贊既官係王府,不敢朝參,不通慶弔,但閉關喑嗚而已,及秦王得罪,或言贊止于朝降,而贊已服麻衣備驢乘在門矣。聞其言曰:「豈有國君之嗣,一旦舉室塗地,而賓佐朝降,得免死,幸也。」俄而臺史示敕,長流嵐州,即時赴貶所。在嵐州踰年,清泰二年春,詔
 歸田里。妻紇干氏塗中卒,贊比羸瘠,慟哭殆絕,因之亦病,行及石會關而卒,時年六十餘。



 史臣曰:自唐祚橫流,衣冠掃地,茍無端士,孰恢素風。如廷珪之文學,崔沂之剛正,劉岳之典禮,舜卿之掌誥,洎夢徵而下,皆蔚有貞規,無虧懿範,固可以為搢紳之圭表,聳朝廷之羽儀,以之垂名,夫何不韙。



\end{pinyinscope}