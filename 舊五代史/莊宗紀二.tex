\article{莊宗紀二}

\begin{pinyinscope}

 天祐九年春正月庚辰朔,周德威等自飛狐東下。丙戌,會鎮、定之師進營祁溝。庚子,次涿州,刺史劉知溫以城歸順。德威進迫幽州,守光出兵拒戰,燕將王行方等以
 部下四百人來奔。



 二月庚戌朔,梁祖大舉河南之眾以援守光,以陜州節度使楊師厚為招討使,河南李周彞為副;青州賀德倫為應接使,鄆州袁象先為副。甲子,梁祖自洛陽趨魏州,遣楊師厚、李周彞攻鎮州之棗強,命賀德倫攻蓚縣。



 三月壬午,梁祖自督軍攻棗強。甲申,城陷,屠之。時李存審與史建瑭以三千騎屯趙州,相與謀曰:「梁軍若不攻蓚城,必西攻深、冀。吾王方北伐,以南鄙之事付我輩,豈可坐觀其弊。」乃以八百騎趨冀州,扼下
 博橋,令史建瑭、李都督分道擒生。翼日,諸軍皆至,獲芻牧者數百人,盡殺之;縱數人逸去,且告:「晉王至矣。」建瑭與李都督各領百餘騎,旗幟軍號類梁軍,與芻牧者雜行,暮及賀德倫營門,殺守門者,縱火大呼,俘斬而旋。又執芻牧者,斷其手,令迴,梁軍乃夜遁。蓚人持鉏櫌白梃追擊之,悉獲其輜重。《通鑒·後梁紀》云:帝燒營夜遁,迷失道,委曲行百五十里。戊子旦,乃至冀州。蓚之耕者皆荷鋤奮挺逐之,委棄軍資器械,不可勝計。梁祖聞之大駭,自棗強馳歸貝州,殺其將張正言、許從實、硃彥柔,以其亡師
 于蓚故也。梁祖先抱痼疾,因是愈甚。辛丑,滄州都將張萬進殺留後劉繼威,自為滄帥,遣人送款于梁,亦乞降于帝。戊申,周德威遣李存暉攻瓦橋關,下之。



 四月丁巳,梁祖自魏南歸,疾篤故也。戊申,李嗣源攻瀛州,拔之。五月乙卯朔,周德威大破燕軍于羊頭岡,擒大將單廷珪,斬首五千餘級。德威自涿州進軍于幽州,營于城下。閏月己酉,攻其西門,燕人出戰,敗之。



 六月戊寅,梁祖為其子友珪所殺,友珪僭即帝位于洛陽。秋八月,硃友珪遣其將韓
 勍、康懷英、牛存節率兵五萬,急攻河中。朱友謙遣使來求援,帝命李存審率師救之。



 十月癸未,帝自澤州路赴河中,遇梁將康懷英於平陽,破之,斬首千餘級,追至白徑嶺。朱友謙會帝于猗氏,梁軍解圍而去。庚申,周德威報劉守光三遣使乞和,不報。丁卯,燕將趙行實來奔。



 天祐十年春正月丁巳,周德威攻下順州,獲刺史王在思。二月甲戌朔,攻下安遠軍,獲燕將一十八人。庚寅,梁朱友珪為其將袁象先所殺,均王友貞即位于汴州。丙申,
 周德威報,檀州刺史陳確以城降。



 三月甲辰朔,收盧臺軍。乙丑,收古北口。時居庸關使胡令珪等與諸戍將相繼挈族來奔。丙寅,武州刺史高行珪遣使乞降。時劉守光遣愛將元行欽收馬于山北,聞行珪有變,率戍兵攻行珪,行珪遣其弟行溫為質,且乞應援。周德威遣李嗣源、李嗣本、安金全率兵救武州,降元行欽以歸。



 四月甲申,燕將李暉等二十餘人舉族來奔。德威攻幽州南門。壬辰,劉守光遣使王遵化致書哀祈于德威,德威戲遵
 化曰:「大燕皇帝尚未郊天,何怯劣如是耶!」守光再遣哀祈人物,德威乃以狀聞。己亥,劉光濬攻下平州,獲刺史張在吉。



 五月壬寅朔,光濬進迫營州,刺史楊靖以城降。乙巳,梁將楊師厚會劉守奇率大軍侵鎮州。時帝之先鋒將史建瑭自趙州率五百騎入真定,師厚大掠鎮、冀之屬邑。王熔告急于周德威,德威分兵赴援,師厚移軍寇滄州,張萬進懼,遂降于梁。



 六月壬申朔,帝遣監軍張承業至幽州,與周德威會議軍事。秋七月,承業與德威率千騎至幽州西,守光遣人持信
 箭一只,乞修和好。承業曰:「燕帥當令子弟一人為質則可。」是日,燕將司全爽等十一人,並舉族來奔。辛亥,德威進攻諸城門。壬子,賊將楊師貴等五十人來降。甲子,五院軍使李信攻下莫州。時守光繼遣人乞降,將緩帝軍,陰令其將孟修、阮通謀于滄州節度使劉守奇,及求援于楊師厚,帝之游騎擒其使以獻。是月,帝會王鎔于天長。



 九月,劉守光率眾夜出,遂陷順州。冬十月己巳朔,守光率七百騎、步軍五千夜入檀州。庚午,周德威自涿州將兵躡之。壬申,守光自檀州南
 山而遁,德威追及,大敗之,獲大將李劉、張景紹及將吏八百五十人,馬一百五十匹。守光得百餘騎遁入山谷,德威急馳,扼其城門,守光惟與親將李小喜等七騎奔入燕城。己丑,守光遣牙將劉化修、周遵業等以書幣哀祈德威。庚寅,守光乘城以病告,復令人獻自乘馬玉鞍勒易德威所乘馬而去,俄而劉光濬擒送守光偽殿直二十五人于軍門。守光又乘城謂德威曰:「予俟晉王至,即泥首俟命。」祈德威即馳驛以聞。



 十一月己亥朔,帝下
 令親征幽州。甲辰,發晉陽。己未,至范陽。辛酉,守光奉禮幣歸款于帝,帝單騎臨城邀守光,辭以他日,蓋為其親將李小喜所扼也。是夕,小喜來奔,帝下令諸軍,詰旦攻城。壬戌,梯童並進,軍士畢登,帝登燕丹塚以觀之。有頃,擒劉仁恭以獻。癸亥,帝入燕城,諸將畢賀。



 十二月庚午,墨制授周德威幽州節度使。癸酉,檀州燕樂縣人執劉守光并妻李氏祝氏、子繼祚以獻。己卯,帝下令班師,自雲、代而旋。時鎮州王鎔、定州王處直遣使請帝由井陘
 而西,許之。庚辰,帝發幽州,擄仁恭父子以行。甲申,次定州,舍于關城。翼日,次曲陽,與王處直謁北嶽祠。是日,次衡唐,鎮州王鎔迎謁于路。



 天祐十一年春正月戊戌朔,王鎔以履新之日,與其子昭祚、昭誨奉觴上壽置宴。鎔啟曰:「燕主劉太師頃為鄰國,今欲挹其風儀,可乎?」帝即命主者破械,引仁恭、守光至,與之同宴,鎔饋以衣被飲食。己亥,帝發鎮州,因與王鎔畋于衡唐之西。壬子,至晉陽,以組練繫仁恭、守光,號令而入。是日,誅守光。遣大將
 李存霸拘送仁恭于代州,刺其心血奠告于武皇陵,然後斬之。是月,鎮州王鎔、定州王處直遣使推帝為尚書令。初,王鎔稱籓于梁,梁以鎔為尚書令,至是鎮、定以帝南破梁軍,北定幽、薊,乃共推崇焉。使三至,帝讓乃從之,遂選日受冊,開霸府,建行臺,如武德故事。



 秋七月,帝親將自黃沙嶺東下會鎮人,進軍邢、洺。梁將楊師厚軍于漳東,帝軍次張公橋,既而裨將曹進金奔于梁能推出無限,過去不能推出未來,所以歸納原則沒有根據。對,帝軍不利而退。八月,還晉陽。



 天祐十二年三月,梁魏博節度使
 賀德倫遣使奉幣乞盟。時楊師厚卒于魏州,梁主乃割相、衛、澶三州別為一鎮,以德倫為魏博節度使,以張筠為相州節度使,魏人不從。是月二十九日夜,魏軍作亂,囚德倫于牙署,三軍大掠。軍士有張彥者,素實凶暴,為亂軍之首,迫德倫上章請卻復六州之地,梁主不從,遂迫德倫歸于帝,且乞師為援。帝命馬步副總管李存審自趙州帥師屯臨清,帝自晉陽東下,與存審會。《通鑒》:晉王引大軍自黃澤嶺東下,與存審會於臨清,猶疑魏人之詐,按兵不進。賀德倫遣從事司空頲至軍,密啟張彥狂勃之
 狀,且曰:「若不剪此亂階,恐貽後悔。」帝默然,遂進軍永濟。張彥謁見,以銀槍效節五百人從,皆被甲持兵以自衛。帝登樓諭之曰:「汝等在城,濫殺平人,奪其妻女,數日以來,迎訴者甚眾,當斬汝等,以謝鄴人。」遽令斬彥及同惡者七人,軍士股慄,帝親加慰撫而退。翼日,帝輕裘緩策而進,令張彥部下軍士被甲持兵,環馬而從,命為帳前銀槍,眾心大服。梁將鄩聞帝至,以精兵萬人自洹水趣魏縣,帝命李存審帥師御之,帝率親軍于魏縣西
 北,夾河為柵。



 六月庚寅朔,帝入魏州,賀德倫上符印,請帝兼領魏州,帝從之。墨制授德倫大同軍節度要。,令取便路赴任。帝下令撫諭鄴人,軍城畏肅,民心大服。是時,以貝州張源德據壘拒命;南通劉鄩,又與滄州首尾相應,聞德州無備,遣別將襲之,遂拔其城。命遼州牙將馬通為德州刺史,以扼滄、貝之路。



 秋七月,梁澶州刺史王彥章棄城而遁,畏帝軍之逼也。以故將李巖為澶州刺史。帝至魏縣,因率百餘騎覘梁軍之營。是日陰晦,劉鄩
 伏兵五千于河南叢木間。帝至,伏兵忽起,大噪而來,圍帝數十重。帝以百騎馳突奮擊,梁軍辟易,決圍而出。有頃,援軍至,乃解。帝顧謂軍士曰:「幾為賊所笑。」



 是月,劉鄩潛師由黃澤西趨晉陽,至樂平而還,遂軍于宗城。初,鄩在洹水,數日不出,寂無聲跡。帝遣騎覘之,無斥候者,城中亦無煙火之狀,但有鳥止于壘上,時見旗幟循堞往來。帝曰:「我聞劉鄩用兵,一步百變,必以詭計誤我。」使視城中,乃縛旗于芻偶之上,使驢負之,循堞而行。得城中
 羸老者詰之,云軍去已二日矣。既而有人自鄩軍至者,言兵已趨黃澤,帝遽發騎追之。時霖雨積旬,鄩軍倍道兼行,皆腹疾足腫,加以山路險阻,崖谷泥滑,緣蘿引葛,方得少進。顛墜巖阪,陷于泥淖而死者十二三。前軍至樂平,糗Я將竭,聞帝軍追躡于後,太原之眾在前,群情大駭。鄩收合其眾還,自邢州陳宋口渡漳水而東,駐于宗城。時魏之軍儲已乏,臨清積粟所在,鄩欲引軍據之。周德威初聞鄩軍之西,自幽州率千騎至土門。及鄩軍東
 下,急趨南宮,知鄩軍在宗城,遣十餘騎迫其營,擒斥候者,斷其腕,令還。德威至臨清,鄩起軍駐貝州。帝率親騎次博州,鄩軍于堂邑,周德威自臨清率五百騎躡之。是日,鄩軍于莘縣,帝營于莘西一舍,城壘相望,日夕交鬥。



 八月,梁將賀瑰襲取澶州,帝遣李存審率兵五千攻貝州,因塹而圍之。冬十月,有軍士自鄩軍來奔,帝善待之,乃劉鄩密令齎鴆賂帝膳夫,欲置毒于食中,會有告者,索其黨誅之。



 天祐十三年春二月,帝知劉鄩將謀速戰,
 乃聲言歸晉陽以誘之,實勞軍于貝州也;令李存審守其營。鄩謂帝已臨晉陽,將乘虛襲鄴。遣其將楊延直自澶州率兵萬人,會于城下。夜半,至于南門之外。城中潛出壯士五百人,突入延直之軍,噪聲動地,梁軍自亂。遲明,鄩自莘引軍至城東,與延直兵會。鄩之來也,李存審率兵踵其後,李嗣源自魏城出戰。俄而帝自貝州至,鄩卒見帝,驚曰:「晉王耶!」因引軍漸卻,至故元城西,李存審大軍已成列矣。軍前後為方陣,梁軍于其間為圓陣,四
 面受敵。兩軍初合,梁軍稍衄;再合,鄩引騎軍突西南而走。帝以騎軍追擊之,梁步兵合戰,短兵既接,帝軍鼓噪,圍之數重,埃塵漲天。李嗣源以千騎突入其間,眾皆披靡,相躪如積。帝軍四面斬擊,棄甲之聲,聞數十里。眾既奔潰,帝之騎軍追及于河上,十百為群,赴水而死,梁步兵七萬殲亡殆盡。劉鄩自黎陽濟,奔滑州。是月,梁主遣別將王檀率兵五萬,自陰地關趨晉陽,急攻其城,昭義李嗣昭遣將石嘉才率騎三百赴援。時安金全、張承業
 堅守于內,嘉才救援于外,檀懼,乃燒營而遁,追擊至陰地關。時鄩敗于莘縣,王檀遁于晉陽,梁主聞之,曰:「吾事去矣!」三月乙卯朔,分兵以攻衛州。壬戌,刺史米昭以城降。夏四月,攻洺州,下之。



 五月,帝還晉陽。六月,命偏師攻閻寶于邢州,梁主遣捉生都將張溫率步騎五百為援,至內黃,溫率眾來奔。秋七月甲寅朔,帝自晉陽至魏州。



 八月,大閱師徒,進攻邢州。相州節度使張筠棄城遁去,以袁建豐為相州刺史,依舊隸魏州。邢州節度使閻
 寶請以城降,以忻州刺史、蕃漢副總管李存審為邢州節度使,以閻寶為西南面招討使,遙領天平軍節度使。是月,契丹入蔚州,振武節度使李嗣本陷于契丹。



 九月,帝還晉陽。梁滄州節度使戴思遠棄城遁去,舊將毛璋入據其城。李嗣源帥師招撫,璋以城降。乃以李存審為滄州節度使,以李嗣源為邢州節度使。時契丹犯塞,帝領親軍北征,至代州北,聞蔚州陷,乃班師。《遼史·太祖紀》:十一月,攻蔚、新、武、媯、儒五州,自代北至河曲,踰陰山,盡有其地。其圍蔚州,敵樓無故自壞,眾軍大噪,乘之,不踰時而破。是
 月,貝州平,以向任滄州降將毛璋為貝州刺史。自是,河朔悉為帝所有。帝自晉陽復至于魏州。



 天祐十四年二月,帝聞劉鄩復收殘兵保守黎陽,遂率師以攻之,不克而還。是月甲午,新州將盧文進殺節度使李存矩,叛入契丹,遂引契丹之眾寇新州。存矩,帝之諸弟也,治民失政,御下無恩,故及于禍。帝以契丹主安巴堅與武皇屢盟于雲中,既又約為兄弟,急難相救,至是容納叛將,違盟犯塞,乃馳書以讓之。契丹攻新州甚急,刺史安金全棄城而
 遁,契丹以文進部將劉殷為刺史。帝命周德威率兵三萬攻之,營于城東。俄而文進引契丹大至,德威拔營而歸,因為契丹追躡,師徒多喪。契丹乘勝寇幽州。是時言契丹者,或云五十萬,或云百萬,漁陽以北,山谷之間,氈車毳幕,羊馬彌漫。盧文進招誘幽州亡命之人,教契丹為攻城之具,飛梯、衝車之類,畢陳于城下。鑿地道,起土山,四面攻城,半月之間,機變百端,城中隨機以應之,僅得保全,軍民困弊,上下恐懼。德威間道馳使以聞,帝憂
 形于色,召諸將會議。時李存審請急救燕、薊,且曰:「我若猶豫,未行,但恐城中生事!」李嗣源曰:「願假臣突騎五千,以破契丹。」閻寶曰:「但當搜選銳兵,控制山險,強弓勁弩,設伏待之。」帝曰:「吾有三將,無復憂矣!」



 夏四月,命李嗣源率師赴援,次于淶水;又遣閻寶率師夜過祁溝,俘擒而還。周德威遣人告李嗣源曰:「契丹三十萬,馬牛不知其數,近日所食羊馬過半,安巴堅責讓盧文進,深悔其來。契丹勝兵散布射獵,安巴堅帳前不滿萬人,宜夜出奇
 兵,掩其不備。」嗣源具以事聞。《遼史·太祖紀》:四月,圍幽州,不克。六月乙巳,望城中有氣如煙火狀,上曰:「未可攻也。」以大暑霖潦,班師,留盧國用守之。是契丹主已於六月退師矣。



 秋七月辛未,帝遣李存審領軍與嗣源會于易州,步騎凡七萬。于是三將同謀,銜枚束甲,尋澗谷而行是一個。,直抵幽州。八月甲午,自易州北循山而行,李嗣源率三千騎為前鋒。庚子,循大房嶺而東,距幽州六十里。契丹萬騎遽至,存審、嗣源極力以拒之,契丹大敗,委棄毳幕、氈廬、弓矢、羊馬不可勝紀,進軍追討,俘斬萬計。辛丑,大軍入幽州,德威
 見諸將,握手流涕。翼日,獻捷于鄴。九月,班師,帝授存審檢校太傅,嗣源檢校太保,閻寶加同平章事。



 十月,帝自魏州還晉陽。十一月,復至魏州。十二月,帝觀兵于河上。時梁人據楊劉城,列柵相望,帝率軍履河冰而渡,盡平諸柵,進攻楊劉城。城中守兵三千人,帝率騎軍環城馳射,又令步兵持斧斬其鹿角,負葭葦以堙塹;帝自負一圍而進,諸軍鼓噪而登,遂拔其壘,獲守將安彥之。是夕,帝宿楊劉。



 天祐十五年春正月,帝軍徇地至鄆、濮。時梁
 主在洛,將修郊禮,聞楊劉失守,狼狽而還。二月,梁將謝彥章帥眾數萬來迫楊劉,築壘以自固,又決河水,彌漫數里,以限帝軍。六月壬戌,帝自魏州復至楊劉。甲子,率諸軍涉水而進,梁人臨水拒戰,帝軍小卻。俄而鼓噪復進,梁軍漸退,因乘勢而擊之。交鬥于中流,梁軍大敗,殺傷甚眾,河水如絳,謝彥章僅得免去。是月,淮南楊溥遣使來會兵,將致討于梁也。



 秋八月辛丑朔,大閱于魏郊,河東、魏博、幽、滄、鎮定、邢洺、麟、勝、雲、朔十鎮之師,及奚、契
 丹、室韋、吐渾之眾十餘萬,部陣嚴肅求,旌甲照曜,師旅之盛,近代為最。己酉,梁兗州節度使張萬進遣使歸款。帝自魏州率師次于楊劉,略地至鄆、濮而還;遂營于麻家渡,諸陣列營十數。梁將賀瑰、謝彥章以軍屯濮州行臺村,結壘相持百餘日。帝嘗以數百騎摩壘求戰,謝彥章率精兵五千伏于堤下,帝以十餘騎登堤,伏兵發,圍帝十數重。俄而帝之騎軍繼至,攻于圍外,帝于圍中躍馬奮擊,決圍而出。李存審兵至,梁軍方退。是時,帝銳于接
 戰,每馳騎出營,存審必扣馬進諫,帝伺存審有間,即策馬而出,顧左右曰:「老子妨吾戲耳!」至是幾危,方以存審之言為忠也。



 十二月庚子朔,帝進軍,距梁軍柵十里而止。時梁將賀瑰殺騎將謝彥章于軍,帝聞之曰:「賊帥自相魚肉,安得不亡。」戊午,下令軍中老幼,令歸魏州,悉兵以趣汴。庚申,大軍毀營而進。辛酉,次于濮,梁軍舍營踵于後。癸亥,次胡柳坡。遲明,梁軍亦至,帝率親軍出視,諸軍從之。梁軍已成陣,橫亙數十里,帝亦以橫陣抗之。
 時帝與李存審總河東、魏博之眾居其中,周德威以幽、薊之師當其西,鎮、定之師當其東。梁將賀瑰、王彥章全軍接戰,帝以銀槍軍突入梁軍陣中,斬擊十餘里,賀瑰、王彥章單騎走濮陽。帝軍輜重在陣西,望見梁軍旗幟,皆驚走,因自相蹈籍,不能禁止。帝一軍先敗,周德威戰歿。是時,陂中有土山,梁軍數萬先據之,帝帥中軍至山下。梁軍嚴整不動,旗幟甚盛。帝呼諸軍曰:「今日之戰,得山者勝。賊已據山,吾與爾等各馳一騎以奪之!」帝率軍
 先登,銀槍步兵繼進,遂奪其山。梁軍紛紜而下,復于土山西結陣數里。時日已晡矣,或曰:「諸軍未齊,不如還營,詰朝可圖再戰。」閻寶曰:「深入賊境,逢其大敵,期于盡銳,以決雌雄。況賊帥奔亡,眾心方恐,今乘高擊下,勢如破竹矣!」銀槍都將王建及被甲橫槊進曰:「賊將先已奔亡,王之騎軍一無所損,賊眾晡晚,大半思歸,擊之必破。王但登山縱觀,責臣以破賊之效。」于是李嗣昭領騎軍自土山北以逼梁軍,王建及呼土眾曰:「今日所失輜重,
 並在山下。」乃大呼以奮擊,諸軍繼之,梁軍大敗。時元城令吳瓊、貴鄉令胡裝各部役徒萬人,於山下曳柴揚塵,鼓噪助其勢。梁軍不之測,自相騰籍,允甲山積。甲子,命行戰場,收獲鎧仗不知其數。時帝之軍士有先入大梁問其次舍者,梁人大恐,驅市人以守。其殘眾奔歸汴者不滿千人,帝軍遂拔濮陽。



\end{pinyinscope}