\article{列傳一 後妃}

\begin{pinyinscope}

 武皇帝貞簡皇后曹氏,莊宗之母也,太原人,以良家子嬪于武皇。姿質閒麗,性謙退而明辨,雅為秦國夫人所重。常從容謂武皇曰:「妾觀曹姬非常婦人,王其厚待之。」
 武皇多內寵,乾寧初,平燕薊,得李匡儔妻張氏,姿色絕代,嬖幸無雙。時姬侍盈室,罕得進御,唯太后恩顧不衰。武皇性嚴急,左右有過,必峻于譴罰,無敢言者,唯太后從容救諫,即為解顏。及莊宗載誕,體貌奇傑,武皇異而憐之,太后益寵貴,諸夫人咸出其下,后亦恭勤內助,左右稱之。



 武皇薨,莊宗嗣晉王位,時李克寧、李存顥謀變,人情危懼。太后召監軍張承業,指莊宗謂之曰;「先人把臂授公此兒,如聞外謀,欲孤付托,公等但置予母子有
 地,毋令乞食于沛,幸矣。」承業因誅存顥、克寧,以清內難。莊宗善音律,喜伶人謔浪,太后常提耳誨之。天祐七年,鎮、定求援,莊宗促命治兵,太后曰:「予齒漸衰,兒但不墜先人之業為幸矣,何事櫛風沐雨,離我晨昏!」莊宗曰:「稟先王遺旨,須滅仇讎。山東之事,機不可失。」及發,太后餞于汾橋,悲不自勝。莊宗平定趙、魏,駐于鄴城,每一歲之內,馳駕歸寧者數四,民士服其仁孝。



 太后初封晉國夫人,莊宗即位,命宰臣盧損奉冊書上皇太后尊號。其年
 平定河南,西幸洛陽,令皇弟存渥、皇子繼岌就太原迎奉。莊宗親至懷州,迎歸長壽宮,太后素與劉太妃善,分訣之後,悒然不樂。俄聞太妃寢疾,尚醫中使,問訊結轍。既而謂莊宗曰:「吾與太妃恩如伯仲,彼經年抱疾,但見吾面,差足慰心,吾暫至晉陽,旬朔與之俱來。」莊宗曰:「時方暑毒,山路崎嶇,無煩往復,可令存渥輩迎侍太妃。」乃止。及凶問至,太后慟哭累旬,由是不豫,尋崩于長壽宮。同光三年冬十月,上謚曰貞簡皇太后,葬于壽安陵。


太妃劉氏,武皇之正室也。
 \gezhu{
  案:《劉太妃傳》,原本闕佚。考《北夢瑣言》云:晉王李克用妻劉夫人,常隨軍行,至于軍機,多所宏益。先是,汴州上源驛有變,晉王憤恨,欲回軍攻之。夫人曰:「公為國討賊,而以杯酒私忿,必若攻城,即曲在于我。不如回師,自有朝廷可以論列。」于是班退。天復中,周德威為汴軍所敗,三軍潰散,汴軍乘我,晉王危懼,與李存信議,欲出保雲州。夫人曰:「存信本北方牧羊兒也,焉顧成敗!王常笑王行瑜棄城失勢,被人屠割,今復欲效之,何也?王頃歲避難塞外,幾遭陷害,賴遇朝廷多事,方得復歸。今一旦出城,便有不測之變,焉能遠及!」晉王止行。居數日,亡散之士復集,軍城安定,夫人之力也。《五代會要》云:同光元年四月,冊為皇太妃。《歐陽史》云:莊宗即位,冊尊曹氏為皇太后,而以嫡母劉氏為皇太妃。太妃往謝太后,太后有慚色。太妃曰:「願吾兒享國無窮,使吾獲沒於地以從先君幸矣。復何言哉!」莊宗滅梁入洛,使人迎太后歸洛,居長壽宮,
  而太妃獨留晉陽。同光三年五月,太妃薨。}



 魏國夫人陳氏,襄州人,本昭宗之宮嬪也。乾寧二年,武皇奉詔討王行瑜,駐軍于渭北,昭宗降硃書御札,出陳氏及內妓四人以賜武皇。陳氏素知書,有才貌,武皇深加寵重。及光化之後,時事多難,武皇常獨居深念,嬪媵鮮得侍謁,唯陳氏得召見。陳氏性既靜退,不以寵侍自侈,武皇常呼為阿媎。及武皇大漸之際,陳氏侍醫藥,垂泣言:「妾為王執掃除之役,十有四年矣,王萬一不幸,妾
 將何托!既不能以身為殉,願落發為尼,為王讀一藏佛經,以報平昔。」武皇為之流涕。及武皇薨,陳氏果落髮持經,法名智願,後居于洛陽佛寺,莊宗賜號建法大師。天成中,明帝幸其院,改賜圓惠大師。晉天福中,卒于太原。追謚光國大師,塔以惠寂為名也。


莊宗神閔敬皇后劉氏。
 \gezhu{
  案:《劉后傳》,原本闕佚。考《北夢瑣言》云:莊宗劉皇后,魏州成安人,家世寒微。太祖攻魏州,取成安,得后,時年五六歲。歸晉陽宮,為太后侍者,教吹笙。及笄,姿色絕眾,聲伎亦所長。太后賜莊宗,為韓國夫人侍者。後誕皇子繼岌,寵待日隆。他日,成安人劉叟詣鄴宮見上,稱夫人之父。有內臣
  劉建豐認之,即昔日黃須丈人,后之父也。劉氏方與嫡夫人爭寵,皆以門族誇尚,劉氏恥為寒家,白莊宗曰:「妾去鄉之時,妾父死于亂兵,是時環屍而哭。妾固無父,是何田舍翁詐偽及此!」乃于宮門笞之,其實后即叟之長女也。莊宗好俳優,宮中暇日,自負蓍囊藥篋,令繼岌相隨,以后父劉叟以醫卜為業也。后方晝眠,及造其臥內,自稱劉衙推訪女,后大恚,笞繼岌。然為太后不禮,復以韓夫人居正,無以發明。大臣希旨請冊劉氏為皇后,議者以后出于寒賤,好興利聚財,初在鄴都,令人設法稗販,所鬻樵蘇果茹亦以皇后為名。正位之後,凡貢奉先入後宮,惟寫佛經施尼師,他無所賜,闕下諸軍困乏,以至妻子饑殍,宰相請出內庫表給,后將出妝具銀盆兩口、皇子滿喜等三人,令鬻以贍軍。一旦作亂,亡國滅族,與夫褒姒、妲己無異也。先是,莊宗自為俳優,名曰李天下,雜于塗粉優雜之間,時為諸優撲扶摑搭,竟為嚚婦惡伶之傾玷,有國者得不以為前鑒!劉氏以囊盛金合
  犀帶四,欲於太原造寺為尼,沿路復通皇弟存渥,同簀而寢,明宗聞其穢,即令自殺。考《歐陽史》,作裨將袁建豐得后,納之晉宮,而《北夢瑣言》作內臣劉建豐,亦傳聞之異辭也。}


淑妃韓氏,莊宗正室。
 \gezhu{
  案:《韓淑妃傳》,原本闕佚。考《五代會要》云:同光二年十二月冊,以宰臣豆盧革、韋說為冊使,出應天門,登路車,鹵簿鼓吹前導,至于永福門降車,入右銀臺門,至淑妃宮,受冊于內,文武百官立班稱賀。}


德妃伊氏,莊宗次室。
 \gezhu{
  案:《伊德妃傳》,原本闕佚。考《北夢瑣言》云:莊宗皇帝嫡夫人韓氏,後為淑妃,伊氏為德妃。又言夫人夏氏,後嫁李贊華,所謂東丹王,即安巴堅長子,性酷毒,侍婢微過,即以刀刲火灼。夏氏少長宮掖,不忍其凶,求離婚,歸河陽節度使夏魯奇家,後為尼也。《歐陽史·家人傳》:明宗立,悉放莊宗時
  宮人。虢國夫人夏氏,歸夏魯奇家,後嫁李贊華。與《北夢瑣言》微異。《遼史》又以夏氏為莊宗皇后,疑誤。又案《五代會要》:莊宗朝內職,又有昭儀侯氏封汧國夫人,昭媛白氏封沂國夫人,出使美宣鄧氏封珝國夫人,御正楚真張氏封涼國夫人,司簿德美周氏封宋國夫人,侍真吳氏封延陵郡夫人,懿才王氏封太原郡夫人,咸一韓氏封昌黎郡夫人,瑤芳張氏封清河郡夫人,懿德王氏封瑯琊郡夫人,宣一馬氏封扶風郡夫人,並同光二年十一月敕。}



 明宗昭懿皇后夏氏,生秦王從榮及閔帝。同光初,后以疾崩,明宗即位,追封為晉國夫人。長興中,明宗以秦、宋二王位望既隆,因思從貴之義,乃下制曰:「故晉國夫人
 夏氏,素推仁德,信睦宗親,嘗施內助之方,不見中興之盛,予當御極,子並為王,有鵲巢之高,無翬衣之貴,貞魂永逝,懿範常存,考本朝之文,沿追冊之制,將慰懷于九族,冀葉慶于四星。宜追冊為皇后,兼定懿號。」既而有司上謚曰昭懿。


和武顯皇后曹氏。
 \gezhu{
  案:《曹后傳》,原本闕佚。考《五代會要》云:天成三年正月,冊為淑妃,長興元年五月十四日,冊為皇后;應順元年閏正月,冊為皇太后。至清泰三年閏十二月,隨末帝崩于後樓。晉高祖使人護葬,至天福五年正月二十八日,追冊曰和武顯皇后。}



 宣憲皇后魏氏。案:《魏後傳》原本闕佚。據《通鑒考異》引《唐廢帝實錄》云:宣憲皇后魏氏,鎮州平山人。中和末,明宗徇地山東,留戍平山,得魏后。又云:明宗為裨將,性闊達,不能治生,曹后亦疏于畫略,生計所資,惟宣憲而已。《五代會要》云:初封魯國太夫人,清泰二年二月,中書門下奏:「臣聞漢昭帝承祚御歷,奉尊謚于雲陽;魏明帝繼體守文,思外家于甄館,而皆追崇微號,祔饗廟庭,克隆敬本之文,式葉愛親之道。臣等又覽國史,竊見明皇帝母曰昭成皇后竇氏,代宗皇帝母曰章敬太后吳氏,始嬪朱邸,俄閟幽宮,鴻圖既屬于明君,尊號咸追于聖母。伏以魯國夫人發祥沙麓,貽慶河洲。三后最賢,周母允成于天統;四妃有子,唐宮先啟于帝基,仰惟當宁之情,彌軫寒泉之思。久虛殷薦,慮損皇猷。臣等謹上尊謚曰宣憲皇太后,請依昭成皇太后故事,擇
 日備禮冊命。又,臣等伏聞先太后舊陵未祔先祠,則都下難崇別廟,既追尊謚,合創閟宮。按漢朝故事,園寢不在王畿,或就陵所便立寢祠。今商量上謚後,權立同廟,以申告獻,配祠之禮,請俟他年。」從之。據《歐陽史》云:議建陵寢,而太原石敬瑭反,乃于京師河南府東立寢宮。又案:《五代會要》所載明宗時內職,德妃王氏,天成三年正月冊立,長興二年四月進號淑妃,應順元年閏正月十三日冊為太妃,至周廣順元年四月追謚賢妃。昭儀王氏封齊國夫人,昭容葛氏封周國夫人,昭媛劉氏封趙國夫人,孫氏封楚國夫人,御正張氏封曹國夫人,司寶郭氏封魏國夫人,司贊于氏封鄭國夫人,尚服王氏封衛國夫人,司記崔氏封蔡國夫人,司膳翟氏封滕國夫人,司醖吳氏封莒國夫人,婕妤高氏封渤海郡夫人,美人沈氏封太原郡夫人,順御硃氏封吳郡夫人,司飾聊氏封潁川郡夫人,司衣劉氏封彭城郡夫人,司藥孟氏封
 咸陽郡夫人,梳篦張氏封清河郡夫人,司服王氏封太原郡夫人,櫛篦傅氏封潁川郡夫人,知客張氏賜號尚書,故江氏追封濟陽郡夫人。以上皆長興三年九月敕。其名號皆中書門下按《六典》內職仿而行之。內人李氏封隴西縣君,崔氏封清河縣君,李氏封成紀縣君,田氏封咸陽縣君,白氏封南陽縣君,並長興四年二月敕。前代內職,皆無封君之禮,此一時之制。



 閔帝哀皇后孔氏。案:《孔後傳》,原本闕佚。據《通鑒》云:孔循陰遣人結王德妃,求納其女;德妃請娶循女為從厚妃,帝許之。庚寅,皇子從厚納孔循女為妃。《五代會要》云:初封魯國夫人聽言動「無一是我自家氣質,如此便是格物物格,致知知至,應順元年四月,為末帝所害。晉天福五年正月二十八日,追謚為哀皇后。



 末帝劉皇后,應州人也。天成中,封為沛國夫人。清泰初,百官三上表,請立中宮,遂立為皇后。後性強戾,末帝甚憚之,故其弟延皓,自鳳翔牙校環歲之間歷樞密使,出
 為鄴都留守,皆由後內政之力也。及延皓為張令昭所逐,執政請行朝典,後力制之,止從罷免而已。晉高祖入洛。后與末帝俱就燔焉。



 史臣曰:昔三代之興亡,雖由于帝王,亦繫于妃后。故夏之興也以塗山,及其亡也以妹嬉;商之興也以簡狄,及其亡民以妲己;周之興也以文母,及其亡也以褒姒。觀夫貞簡之為人也,雖未偕于前代,亦無虧于懿範。而劉后以牝雞之晨,皇業斯墜,則與夫三代之興亡同矣。餘
 無進賢輔佐之德,又何足以道哉!案:《五代史》無《外戚傳》。據《五代會要》,武皇長女瓊華長公主,降孟知祥,同光三年十二月封。第二女瑤英長公主,降張延釗,同光三年十二月封。明宗長女永寧公主,降晉高祖。第十三女興平公主,降趙延壽,天成三年四月封,至長興四年九月改封齊國公主,至清泰三年二月進封燕國長公主。第十四女壽安公主,長興四年六月封。第十五女永樂公主,長興四年六月封。今考《會要》所載,亦多舛互,如瓊華公主,《十國春秋》諸書作太祖弟克讓之女,《會要》以為武皇長女,此傳聞之異辭也。莊宗女義寧公主,降宋廷浩。廷浩仕至房州刺史,晉初為汜水關使,張從賓之叛,戰死。見《東都事略》及《宋史》。又,王禹偁《小畜集》有《宋渥神道碑》云:母義寧公主,天福中,晉祖以嘗事莊宗,有舊君之禮,每貴主入見,聽其不拜。時兵戎方熾,經費不充,惟公主之家,賜予甚厚,盡而復取,亦無倦色。一日,晉祖從容謂
 貴主曰:「朕於主家無所愛惜,但朝廷多事,府庫甚虛,主所知矣。今輦轂之下,桂玉為憂,可命渥分司西京,以豐就養。」因厚遣之,且敕留司具晨昏伏臘之用,至於醯醢,率有備焉。《會要》不載莊宗女幾人,是其闕略也。惟明宗諸女記之,稍詳,然考薛史《趙延壽傳》,其娶明宗小女為繼室。《歐陽史》亦云有耶律德光為延壽娶從益妹,是為永安公主。而《五代會要》不載,則其闕漏者亦多矣。



\end{pinyinscope}