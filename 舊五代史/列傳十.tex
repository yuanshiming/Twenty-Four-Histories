\article{列傳十}

\begin{pinyinscope}

 趙光逢,字延吉。曾祖植,嶺南節度使。祖存約,興元府推官。父隱,右僕射。光逢與弟光裔,皆以文學德行知名。《舊唐書》:光裔,光啟三年進士擢第,累遷司勳郎中、弘文館學士,改膳部郎中、知制誥。季述廢立之後,旅游江表以避
 患,嶺南劉隱深禮之,奏為副使,因家嶺外。光逢幼嗜墳典,動守規檢,議者目之為「玉界尺」。僖宗朝,登進士第。踰月,辟度支巡官,歷官臺省,內外兩制,俱有能名,轉尚書左丞、翰林承旨。昭宗幸石門,光逢不從,昭宗遣內養戴知權詔赴行在,稱疾解官。駕在華州,拜御史中丞。時有道士許巖士、瞽者馬道殷出入禁庭,驟至列卿宮相,因此以左道求進者眾,光逢持憲紀治之,皆伏法,自是其徒頗息。改禮部侍郎、知貢舉。光化中,王道浸衰,南北司為黨,光逢素惟慎靜,
 慮禍及己,因挂冠伊洛,屏絕交遊,凡五六年。門人柳璨登庸,除吏部侍郎、太常卿。《唐摭言》云:光化二年,趙光逢放柳璨及第,後三年不遷,時璨自內庭大拜,光逢始以左丞徵入。入梁為中書侍郎、平章事,累轉左僕射兼租庸使,上章求退,以太子太保致仕。梁末帝愛其才,徵拜司空、平章事。無幾以疾辭,授司徒致仕。《唐摭言》云:光逢膺大用,居重地十餘歲,七表乞骸,守司空致仕。居二年,復徵拜上相。



 同光初,弟光允為平章事,時謁問於私第,嘗語及政事,他日,光逢署其戶曰「請不言中書事」,其清凈寡慾端默如此。嘗有女冠寄
 黃金一鎰于其室家,時屬亂離,女冠委化於他土。後二十年,金無所歸,納於河南尹張全義,請付諸宮觀,其舊封尚在。兩登廊廟,四退邱園,百行五常,不欺暗室,搢紳咸仰以為名教主。天成初,遷太保致仕,封齊國公,卒於洛陽。詔贈太傅。


光允,光逢之弟也,
 \gezhu{
  新舊《唐書》俱云:趙隱子三人,光逢、光裔、光允。為後唐相者,光允也。原本避宋諱,稱光亂為光裔,似混二人為一,今改正。}
 俱以詞藝知名,亦登進士第。
 \gezhu{
  《舊唐書》云:大順二年,進士登第。天祐初,累官至駕部郎中。}
 光允仕梁,歷清顯,伯仲之間,
 咸以方雅自高,北人聞其名者,皆望風欽重。及莊宗平定汴、洛,時盧程以狂妄免,郭崇韜自勳臣拜,議者以為國朝典禮故實,須訪前代名家,咸曰光允有宰相器。薛廷珪、李琪當武皇為晉王時,嘗因為冊使至太原,故皆有宿望,當時咸謂宜處台司。郭崇韜採言事者云,廷珪朽老,浮華無相業;琪雖文學高,傾險無士風,皆不可相,乃止。同光元年十一月,光允與韋說並拜平章事。



 光允生於季末,漸染時風,雖欲躍鱗振翮,仰希前輩,然才力無
 餘,未能恢遠,朝廷每有禮樂制度、沿革擬議,以為己任;同列既匪博通,見其浮譚橫議,莫之測也。豆盧革雖憑門地,在本朝時,仕進尚微,久從使府,朝章典禮,未能深悉。光允每有發論,革但唯唯而已。後革奏議或當,光允謂群官曰:「昨有所議,前座一言粗當,近日差進,學者其可已乎!」其自負如此。



 先是,條制:「權豪強買人田宅,或陷害籍沒,顯有屈塞者,許人自理。」內官楊希朗者,故觀軍容使復恭從孫也,援例理復恭舊業。事下中書,光允謂
 崇韜曰:「復恭與山南謀逆,顯當國法,本朝未經昭雪,安得論理?」崇韜私抑宦者,因具奏聞。希朗泣訴於莊宗,莊宗令自見光允言之。希朗陳訴:「叔祖復光有大功於王室,伯祖復恭為張浚所構,得罪前朝,當時強臣掣肘,國命不行,及王行瑜伏誅,德音昭洗,制書尚在,相公本朝世族,諳練故事,安得謂之未雪耶!若言未雪,吾伯氏彥博,洎諸昆仲,監護軍鎮,何途得進!」漸至聲色俱厲。光允方恃名德,為其所折,悒然不樂。又以希朗幸臣,慮摭他
 事危己,心不自安。三年夏四月,病疽卒。贈左僕射。


鄭玨,昭宗朝宰相綮之姪孫。父徽,河南尹張全義判官。光化中,登進士第,
 \gezhu{
  《歐陽史》云:玨舉進士數不中,全義以玨屬有司,乃得及第。}
 歷弘文館校書、集賢校理、監察御史,入梁為補闕、起居郎,召入翰林,累遷禮部侍郎充職。玨文章美麗,旨趣雍容,自策名登朝,張全義皆有力焉。貞明中,拜平章事。莊宗入汴,責授萊州司戶,未幾,量移曹州司馬。張全義言於郭崇韜,將復相之,尋入為太子賓客。明宗即位,任圜自蜀
 至,安重誨不欲圜獨拜宰輔,共議朝望一人共之。孔循言玨貞明時久在中書,性畏慎而長者,美詞翰,好人物,重誨即奏與任圜並命為相。有頃,玨以老病耳疾,不任中書事,四上章請,明宗惜之,久而方允,乃授開府儀同三司,行尚書左僕射致仕,仍賜鄭州莊一區。明宗自汴還洛陽,遣中使撫問,賜錢二十萬,食羊百口。長興初卒。贈司空。



 初,玨應進士,十九年方登第,名姓為第十九人,自登第凡十九年為宰相,又昆仲之次第十九,時亦異之。



 子
 遘,太平興國中任正郎。



 崔協,字思化。遠祖清河太守第二子寅,仕後魏為太子洗馬,因為清河小房,至唐朝盛為流品。曾祖邠,太常卿,祖瓘,吏部尚書。父彥融,楚州刺史。彥融素與崔蕘善,嘗為萬年令,蕘謁於縣,彥融未出,見案上有尺題,皆賂遺中貴人,蕘知其由徑,始惡其為人。及除司勳郎中,蕘為左丞,通刺不見,蕘謂曰:「郎中行止鄙雜,故未見。」宰相知之,改楚州刺史,卒於任。誡其子曰:「世世無忘蕘。」故其子
 弟常與蕘仇。



 協即彥融之子也。幼有孝行,登進士第,釋褐為度支巡官、渭南尉,直史館,歷三署,入梁為左司郎中、萬年令、給事中,累官至兵部侍郎。與中書舍人崔居儉相遇於幕次,協厲聲而言曰;「崔蕘之子,何敢相見!」居儉亦報之。左降太子詹事,俄拜吏部侍郎。同光初,改御史中丞,憲司舉奏,多以文字錯誤,屢受責罰。協器宇宏爽,高談虛論,多不近理,時人以為虛有其表。天成初,遷禮部尚書、太常卿,因樞密使孔循保薦,拜平章事。



 初,豆盧
 革、韋說得罪,執政議命相,樞密使孔循意不欲河朔人居相位,任圜欲相李琪,而鄭玨素與琪不協,孔循亦惡琪,謂安重誨曰:「李琪非無藝學,但不廉耳。朝論莫若崔協。」重誨然之,因奏擇相。明宗曰:「誰可?」乃以協對。任圜奏曰:「重誨被人欺賣,如崔協者,少識文字,時人謂之『沒字碑』。臣比不知書,無才而進,已為天下笑,何容中書之內,更有笑端!」明宗曰;「易州刺史韋肅,人言名家,待我嘗厚,置於此位何如?肅茍未可,則馮書記是先朝判官,稱為
 長者,與物無競,可以相矣。」道嘗為莊宗霸府書記,故明宗呼之。朝退,宰臣樞密使休於中興殿之廡下,孔循拂衣而去,曰:「天下事一則任圜,二則任圜,崔協暴死則已,不死會居此位。」重誨私謂圜曰:「今相位缺人,協且可乎?」圜曰:「朝廷有李琪者,學際天人,奕葉軒冕,論才校藝,可敵時輩百人。而讒夫巧沮,忌害其能,必舍琪而相協,如棄蘇合之丸,取蛣蜣之轉也。」重誨笑而止。然重誨與循同職,循日言琪之短、協之長,故重誨竟從之。而協登庸
 之後,廟堂代筆,假手於人。朝廷以國庠事重,命協兼判祭酒事,協上奏每歲補監生二百為定,物議非之。《北夢瑣言》:明宗問宰相馮道:「盧質近日吃酒否?」對曰:「質曾到臣居,亦飲數爵,臣勸不令過度,事亦如酒,過則患生。」崔協強言於坐曰:「臣聞『食醫心鏡』,酒極好,不加藥餌,足以安心神。」左右見其膚淺,不覺哂之。四年春,駕自夷門還京,從至須水驛,中風暴卒。詔贈尚書左僕射,謚曰恭靖。



 子頎、頌、壽貞,惟頌仕皇朝,官至左諫議大夫,終於鄜州行軍司馬。



 李琪,字台秀。五代祖憕,天寶末,禮部尚書、東部留守。安
 祿山陷東都,遇害,累贈太尉,謚曰忠懿。憕孫寀,元和朝,位至給事中。寀子敬方,文宗朝,諫議大夫。敬方子縠,廣明中,為晉公王鐸都統判官,以收復功為諫議大夫。



 琪即縠之子也,年十三,詞賦詩頌,大為王鐸所知,然亦疑其假手。一日,鐸召縠宴于公署,密遣人以《漢祖得三傑賦》題就其第試之,琪援筆立成。賦尾云:「得士則昌,非賢罔共,龍頭之友斯貴,鼎足之臣可重,宜哉項氏之敗亡,一范增而不能用。」鐸覽而駭之,曰:「此兒大器也,將擅文
 價。」《太平廣記》:琪總角謁鐸。鐸顧曰:「適蜀中詔到,用夏州拓跋思恭為收復都統,可作一詩否?」即秉筆立製,云:「飛騎經巴棧,洪恩及夏臺。將從天上去,人自日邊來。此處金門遠,何時玉輦回。蚤平關右賊,莫待詔書催。」鐸益奇之,因執琪手曰:「此真鳳手也。」時年十四。明年,丁母憂,因流寓青、齊。然糠照薪,俾夜作畫,覽書數千卷,間為詩賦。唐僖宗再幸梁、洋,竊賦云:「哀痛不下詔,登封誰上書。」



 昭宗時,李谿父子以文學知名。琪年十八,袖賦一軸謁谿。谿覽賦驚異,倒屣迎門,出琪《調啞鐘》、《捧日》等賦,謂琪曰:「余嘗患近年文士辭賦,皆數句之後,未見賦題,吾子入句見題,偶屬典麗,吁!可畏也。」琪由是益知名,舉進士第。天復初,應博學弘詞,居
 第四等,授武功縣尉,辟轉運巡官,遷左拾遣、殿中侍御史。自琪為諫官憲職,凡時政有所不便,必封章論列,文章秀麗,覽之者忘倦。



 琪兄珽,亦登進士第,才藻富贍,兄弟齊名,而尤為梁祖所知,以珽為崇政學士。琪自左補闕入為翰林學士,《北夢瑣言》:梁李相國琪,唐末以文學策名,仕至御史。昭宗播遷,衣冠蕩析,琪藏跡於荊、楚間,自晦其迹,號華原李長官。其堂兄光符宰宜都,嘗厭薄之。琪寂寞,每臨流踞石,摘樹葉而試草制詞,吁嗟怏悵,而投葉水中。梁祖受禪,徵入,拜翰林學士。累遷戶部侍郎、翰林承旨。梁祖西抗邠、岐,北攻澤、潞,出師燕、趙,經略四方,暫無
 寧歲,而琪以學士居帳中,專掌文翰,下筆稱旨,寵遇逾倫。是時,琪之名播於海內。琪重然諾,憐才獎善,家門雍睦。貞明、龍德中,歷兵、禮、吏侍郎,受命與馮錫嘉、張充、郗殷象同撰《梁太祖實錄》三十卷,遷御史中丞,累擢尚書左丞、中書門下平章事。時琪與蕭頃同為宰相,頃性畏慎深密,琪倜儻負氣,不拘小節,中書奏覆,多行其志,而頃專掎摭其咎。會琪除吏是試攝名銜,改「攝」為「守」,為頃所奏,梁帝大怒,將投諸荒裔,而為趙巖輩所援,罷相,為
 太子少保。



 莊宗入汴,素聞琪名,因欲大任。同光初,歷太常卿、吏部尚書。三年秋,天下大水,國計不充,莊宗詔百僚許上封事,陳經國之要。琪因上疏曰:



 臣聞王者富有兆民,深居九重,所重患者,百姓凋耗而不知,四海困窮而莫救,下情不得上達,群臣不敢指言。今陛下以水潦之災,軍食乏闕,焦勞罪己,迫切疚懷,避正殿以責躬,訪多士而求理,則何思而不獲,何議而不臧?止在改而行之,足以擇其善者。



 臣聞古人有言曰:穀者,人之司命也;地
 者,穀之所生也;人者,君之所理也。有其穀則國力備,定其地則人食足,察其人則徭役均,知此三者,為國之急務也。軒黃已前,不可詳記。自堯湮洪水,禹作司空,於時辨九等之田,收什一之稅,其時戶一千三百餘萬,定墾地約九百二十萬頃,最為太平之盛。及商革夏命,重立田制,每私田十畝,種公田一畝,水旱同之,亦什一之義也。洎乎周室,立井田之法,大約百里之國,提封萬井,出車百乘,戎馬四百匹。畿內兵車萬乘,馬四萬匹,以田法
 論之,亦什一之制也。故當成、康之世,比堯、舜之朝,戶口更增二十餘萬,非他術也,蓋三代以前,皆量入以為出,計農以立軍,雖逢水旱之災,而有凶荒之備。



 降及秦、漢,重稅工商,急關市之徵,倍舟車之算,人戶既以減耗,古制猶以兼行,按此時戶口,尚有千二百餘萬,墾田亦八百萬頃。至乎三國並興,兩晉之後,則農夫少於軍眾,戰馬多於耕牛,供軍須奪於農糧,秣馬必侵於牛草,於是天下戶口,只有二百四十餘萬。洎隋文之代,兩漢比隆,
 及煬帝之年,又三分之一。



 我唐太宗文皇帝,以四夷初定,百姓未豐,延訪群臣,各陳所見,惟魏徵獨勸文皇力行王道,由是輕徭薄賦,不奪農時,進賢良,悅忠直,天下粟價,斗直兩錢。自貞觀至於開元,將及一千九百萬戶,五千三百萬口,墾田一千四百萬頃,比之堯、舜,又極增加,是知救人瘼者,以重斂為病源;料兵食者,以惠農為軍政。仲尼云:「百姓足,君孰與不足。」臣之此言,是魏徵所以勸文皇也,伏惟深留宸鑒。如以六軍方闕,不可輕徭,
 兩稅之餘,猶須重斂,則但不以折納為事,一切以本色輸官,又不以紐配為名,止以正耗加納,猶應感悅,未至流亡。況今東作是時,羸牛將駕,數州之地,千里運糧,有此差徭,必妨春種,今秋若無糧草,保以贍軍。



 臣伏思漢文帝時,欲人務農,乃募人入粟,得拜爵及贖罪,景帝亦如之。後漢安帝時,水旱不足,三公奏請,富人入粟,得關內侯及公卿以下散官。本朝乾元中,亦曾如此。今陛下縱不欲入粟授官,願明降制旨下諸道,合差百姓轉倉
 之處,有能出力運官物到京師,五百石以上,白身授一初任州縣官,有官者依資遷授,欠選者便與放選。千石以上至萬石,不拘文武,明示賞酬。免令方春農人流散,斯亦救民轉倉贍軍之一術也。



 莊宗深重之,尋命為國計使,垂為輔相,俄遇蕭牆之難而止。



 及明宗即位,豆盧革、韋說得罪,任圜陳奏,請命琪為相,為孔循、鄭玨排沮,乃相崔協。琪時為御史大夫,安重誨於臺門前專殺殿直馬延。雖曾彈奏,而依違詞旨,不敢正言其罪,以是
 托疾,三上章請老;朝旨不允,除授尚書左僕射。自是之後,尤為宰執所忌,凡有奏陳,靡不望風橫沮。天成末,明宗自汴州還洛,琪為東都留司官班首,奏請至偃師奉迎。時琪奏中有「敗契丹之兇黨,破真定之逆賊」之言,詔曰:「契丹即為兇黨,真定不是逆賊,李琪罰一月俸。」又嘗秦敕撰《霍彥威神道碑》文。琪,梁之故相也,敘彥威仕梁歷任,不言其偽。中書奏曰:「不分真偽,是混功名,望令改撰。」詔從之。多此類也。



 琪雖博學多才,拙于遵養時晦,知時
 不可為,然猶多岐取進,動而見排,由己不能鎮靖也。以太子太傅致仕。長興中,卒於福善里第,時年六十。子貞,官至邑宰。琪以在內署時所為制詔,編為十卷,目曰《金門集》,大行於世。



 蕭頃,字子澄,京兆萬年人。故相仿之孫,京兆尹廩之子。頃聰悟善屬文,昭宗朝擢進士第,歷度支巡官、太常博士、右補闕。時國步艱難,連帥倔強,率多奏請,欲立家廟於本鎮,頃上章論奏,乃止。累遷吏部員外郎。先是,張濬
 自中書出為右僕射,梁祖判官高劭使梁祖廕求一子出身官,省寺皆稱無例,浚曲為行之,指揮甚急,吏徒惶恐。頃判云:「僕射未集郎官,赴省上指揮公事,且非南宮舊儀。」濬聞之,慚悚致謝,頃由是知名,梁祖亦獎之。頃入梁,歷給諫、御史中丞、禮部侍郎、知貢舉,咸有能名。自吏部侍郎拜中書門下平章事,與李琪同輔梁室,事多矛盾。莊宗入汴,頃坐貶登州司戶,量移濮州司馬。數年,遷太子賓客。天成初,為禮部尚書、太常卿、太子少保致仕。
 卒時年六十九。輟朝一日,贈太子少師。



 史臣曰:夫相輔之才,從古難得,蓋文學政事,履行謀猷,不可缺一故也。如數君子者,皆互有所長,亦近代之良相也。如齊公之明節,李琪之文章,足以圭表搢紳,笙簧典誥,陟之廊廟,宜無愧焉!



\end{pinyinscope}