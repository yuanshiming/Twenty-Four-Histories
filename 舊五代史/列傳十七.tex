\article{列傳十七}

\begin{pinyinscope}

 李建及,許州人。本姓王,父質。建及少事李罕之為紀綱,光啟中,罕之謁武皇于晉陽春秋末期鄧析。著作《公孫龍子》,今存六篇。,因選部下驍勇者百人以獻,建及在籍中。後以功署牙職,典義兒軍,及賜姓名。天
 祐七年,改匡衛軍都校。柏鄉之役,汴將韓勍追周德威至高邑南野河上,鎮、定兵扼橋道,韓勍選精兵先奪之。莊宗登高而望,鎮、定兵將衄,謂建及曰:「如賊過橋,則勢不可遏,卿計若何?」建及于部選士二百,挺槍大噪,禦汴軍,卻之于橋下。二月,王師攻魏,魏人夜出犯我營,建及設伏待之,扼其歸路,盡殪之。劉鄩之營莘縣,月餘不出,忽一旦縱兵攻鎮、定之營,軍中騰亂,建及率銀槍勁兵千人赴之,擊敗汴軍,追奔至其壘。元城之戰,建及首陷
 其陣,授天雄軍教練使。八月,遷遼州刺史。十四年,從擊契丹于幽州,破之。十二月,從攻楊劉,自寅至午,汴軍嬰城拒守,建及自負葭葦堙塹,率先登梯,遂拔之。胡柳之役,前軍逗撓,際晚,汴軍登土山,建及一戰奪之。莊宗欲收軍,詰朝合戰。建及橫槊當前,曰:「賊大將已亡,乘此易擊,王但登山,觀臣破賊!」即引銀槍效節大呼奮擊,三軍增氣,由是王師復振,以功授檢校司空、魏博內外衙都將。



 十六年,汴將賀瑰攻德勝南城,以戰船十餘艘,竹笮維
 之,扼斷津路,王師不得渡。城中矢石將盡,守城將氏延賞危急,莊宗令積帛軍門,召能破賊船者。津人有馬破龍者,能水游,乃令往見延賞,延賞言:「危窘極矣,所爭晷刻。」時棹船滿河,流矢雨集,建及被重鎧,執槊呼曰:「豈有一衣帶水,縱賊如此!」乃以二船實甲士,皆短兵持斧,徑抵梁之戰艦,斧其笮;又令上流具甕,積薪其上,順流縱火,以攻其艦。須臾,煙焰騰熾,梁軍斷纜而遁;建及乃入南城,賀瑰解圍而去。其年十二月,與汴將王瓚戰于戚城,
 建及傷手,莊宗解御衣金帶賜之。



 建及有膽氣,慷慨不群,臨陣鞠旅,意氣橫壯,自莊宗至魏州,建及都總內外衙銀槍效節帳前親軍,善于撫御,所得賞賜,皆分給部下,絕甘分少,頗洽軍情。又累立戰功,雄勇冠絕,雌劣者忌讒之。時宦官韋令圖監建及軍,每于莊宗前言:「建及以家財驟施,其趨向志意不小,不可令典衙兵。」莊宗因猜之。建及性既忠藎,雖知讒構,不改其操。



 十七年三月,授代州刺史。八月,與李存審赴河中,解同州之圍。建及
 少遇禍亂,久從戰陣,矢石所中,肌無完膚,後有功見疑,私心憤鬱。是歲,卒于太原,時年五十七。



 石君立,趙州昭慶人也,亦謂之石家財。初事代州刺史李克柔,後隸李嗣昭為牙校,歷典諸軍。夾城之役,君立每出挑戰,壞汴軍柵壘,俘擒而還。八年,與汴軍戰于龍化園,敗之,獲其大將卜渥以獻。嗣昭每出征,俾君立為前鋒,敵人畏之。王檀之逼晉陽也,城中無備,安金全驅市人以登陴,保聚不完。時莊宗在魏博,救應不暇,人心
 危懼,嗣昭遣君立率五百騎,自上黨朝發暮至。王檀游軍扼汾橋,君立一戰敗之,徑至城下,馳突斬擊,出入如神,大呼曰:「昭義侍中大軍至矣!」是夜入城,與安金全等分出諸門擊殺于外,遲明,梁軍敗走。十七年,將兵屯德勝。時汴軍自滑州轉餉以給楊村寨,莊宗親率騎軍于河外,循岸而上,邀擊之。汴人距楊村五十里,于河曲潘張村築壘以貯軍儲,莊宗令諸軍攻之。汴人設伏于要路,逆戰偽敗,王師乘之,蹙入壘門,梁伏兵起,因與血戰。
 君立與鎮州大將王釗陷入賊壘,時諸將部校陷賊者十餘人,君立被執,送于汴。梁主素知其驍勇,欲用之為將,械而下獄。久之,梁主遣人誘之,君立曰:「敗軍之將,難與議勇,如欲將我,我雖真誠效命,能信我乎?人皆有君,吾何忍反為仇人哉!」既而諸將被戮,尚惜君立不之害。同光元年,莊宗至汴前一日,梁主始令殺之。



 高行珪,燕人也。家世勇悍,與弟行周俱有武藝,初仕燕為騎將,驍果出諸將之右。燕帥劉守光僭逆不道,莊宗
 令周德威征之,守光大懼,以行珪為武州刺史,令張犄角之勢。時明宗將兵助德威平燕,俄聞行珪至,率騎以禦之。明宗諭以逆順之理,行珪乃降。守光將元行欽在山北,聞行珪有變,即率部下軍眾以攻行珪。行珪遣弟行周告急于周德威,德威命明宗、李嗣本、安金全將兵援之。明宗破行欽于廣邊軍,行欽亦降。尋以行珪為朔州刺史,歷忻、嵐二郡,遷雲州留後。天成初,授鄧州節度使,尋移鎮安州。行珪性貪鄙,短于為政,在安州日,行事
 多不法。副使范延策者,幽州人也,性剛直,累為賓職,及佐行珪,睹其貪猥,因強諫之,行珪不從。後延策因入奏,獻封章于闕下,事有三條:一請不禁過淮豬羊,而禁絲綿匹帛,以實中國;一請于山林要害置軍鎮,以絕寇盜;一述籓侯之弊,請敕從事明諫諍之,不從,令諸軍校列班廷諍。行珪聞之,深銜之。後因戍兵作亂,誣奏延策與之同謀,父子俱戮于汴,聞者冤之。未幾,行珪以疾卒。詔贈太尉。



 張廷裕,代北人也。幼事武皇于雲中,從平黃巢,討王行瑜,自行間漸升為小將。莊宗定魏,補天雄軍左廂馬步都虞候,歷蔚、慈、隰三州刺史。同光三年,除新州節度使。塞上多事,廷裕無控制之術,邊鄙常聳。天成三年,卒于治所。詔贈太保。



 王思同,幽州人也。父敬柔,歷瀛、平、儒、檀、營五州刺史。思同母即劉仁恭之女也,故思同初事仁恭為帳下軍校。會劉守光攻仁恭于大安山,思同以部下兵歸太原,時
 年十六,武皇命為飛騰指揮使。從莊宗平定山東,累典諸軍。



 思同性疏俊,粗有文,性喜為詩什,與人唱和,自稱薊門戰客。魏王繼岌待之若子。時內養呂知柔侍興聖宮,頗用事,思同不平之。呂為終南山詩,末句有「頭」字,思同和曰:「料伊直擬沖霄漢,賴有青天壓著頭。」其所為詩句,皆此類也。每從征,必在興聖帳下,然同光朝,位止鄭州刺史。明宗在軍時,素知之,即位後,用為同州節度使,未幾,移鎮隴右。思同好文士,無賢不肖,必館接賄遺,歲
 費數十萬。在秦州累年,邊民懷惠,華戎寧息。長興元年,入朝,見于中興殿。明宗問秦州邊事,對曰:「秦州與吐蕃接境,蕃部多違法度。臣設法招懷,沿邊置寨四十餘所,控其要害。每蕃人互市、飲食之界上,令納器械。」因手指畫秦州山川要害控扼處。明宗曰:「人言思同不管事,豈及此耶!」時兩川叛,欲用之,且留左右,故授右武衛將軍。八月,授西南面行營馬步都虞候。九月,遷京兆尹、西京留守。伐蜀之役,為先鋒指揮使。石敬瑭入大散關。思同
 恃勇先入劍門,大軍未相繼,復被董璋兵逐出之。及敬瑭班師,思同以曾獲劍門之功,移鎮山南西道。



 三年,兩川交兵,明宗慮併在一人,則朝廷難制,密詔思同相度形勢,即乘間用軍,事未行而董璋敗。八月,復為京兆尹兼西京留守。時潞王鎮鳳翔,與之鄰境,及潞王不稟朝旨,致書于秦、涇、雍、梁、邠諸帥,言:「賊臣亂政,屬先帝疾篤,謀害秦王,迎立嗣君,自擅權柄,以致殘害骨肉,搖動籓垣。懼先人基業,忽焉墜地,故誓心入朝,以除君側,事濟
 之後,謝病歸籓。然籓邸素貧。兵力俱困,欲希國士,共濟急難。」乃令小伶女十人以五弦技見思同,因歡諷動,又令軍校宋審溫者,請使于雍,若不從命,即獨圖之。又令推官郝昭、府吏朱延乂以書檄起兵。會副部署藥彥稠至,方宴,而妓、使適至,乃擊之于獄。彥稠請誅審溫,拘送昭赴闕。時思同已遣其子入朝言事,朝廷嘉之,乃以思同為鳳翔行營都部署,起軍營于扶風。



 三月十四日,與張虔釗會于岐下,梯衝大集。十五日,進收東西關城,城中
 戰備不完,然死力禦扞,外兵傷夷者十二三。十六日,復進攻其城,潞王登陴泣諭于外,聞者悲之。張虔釗性褊,詰旦,西南用軍,與都監皆血刃以督軍士,軍士齊詬,反攻虔釗,虔釗躍馬避之。時羽林指揮使楊思權引軍自西門先入,思同未之知,猶督士登城。俄而嚴衛指揮使尹暉呼曰:「西城軍已入城受賞矣,軍士可解甲!」棄仗之聲,振動天地。日午,亂軍畢集,涇州張從賓、邠州康福、河中安彥威皆遁去。十七日,思同與藥彥稠、萇從簡俱至
 長安,劉遂雍閉關不內,乃奔潼關。



 二十二日,潞王至昭應,前鋒執思同來獻。王謂左右曰:「思同計乖于事,然盡心于所奉,亦可嘉也。」顧謂趙守鈞曰:「思同爾之故人,可行迓之于路,達予撫慰之意。」思同至,潞王讓之曰:「賊臣傾我國家,殘害骨肉,非予弟之過。我起兵岐山,蓋誅一二賊臣耳,爾何首鼠兩端,多方誤我,今日之罪,其可逃乎!」思同曰:「臣起自行間,受先朝爵命,秉旄仗鉞,累歷重籓,終無顯效以答殊遇。臣非不知攀龍附鳳則福多,扶
 衰救弱則禍速,但恐瞑目之後,無面見先帝。釁鼓膏原,縲囚之常分也。」潞王為之改容,徐謂之曰:「且憩歇。」潞王欲用之,而楊思權之徒恥見其面,屢啟劉延朗,言「思同不可留,慮失士心。」又,潞王入長安時,尹暉盡得思同家財及諸妓女,故尤惡思同,與劉延朗亟言之。屬王醉,不待報,殺思同并其子德勝。潞王醒,召思同,左右報已誅之矣。潞王怒延朗,累日嗟惜之。及漢高祖即位,詔贈侍中。



 索自通,字得之,太原清源人也。父繼昭,以自通貴,授國子監祭酒致仕。自通少能騎射,嘗于山墅射獵,莊宗鎮太原時,遇之于野,訊其姓名,即補右番直軍使。後因從獵,射中走鹿,轉指揮使。佐周德威攻燕軍于涿州,擒燕將郭在鈞。從莊宗定魏博,改突騎指揮使。明宗即位,自隨駕左右廂馬軍都指揮授忻州刺史。歲餘召還,復典禁兵,領韶州刺史,出為大同軍節度使。累歲移鎮忠武,改京兆尹、西京留守。楊彥溫據河中作亂,自通率師
 討平之,授河中節度使。尋自鄜州入為右龍武統軍。初,自通既平楊彥溫,代末帝鎮河中,臨事失於周旋,末帝深銜之。《通鑒》:自通至鎮,承安重誨指,籍軍府甲仗數上之,以為從珂私造,賴王德妃居中保護,從珂由是得免。及末帝即位,自通憂悸求死。清泰元年七月,因朝退涉洛,自溺而卒。



 子萬進,周顯德中,歷任方鎮。



\end{pinyinscope}