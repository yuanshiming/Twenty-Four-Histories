\article{列傳三}

\begin{pinyinscope}

 硃瑄,
 宋州下邑人也。父慶,里之豪右,以攻剽販鹽為事,吏捕之伏法。瑄坐父罪以笞免。因入王敬武軍為小校。唐中和二年,諫議大夫張濬徵兵於青州,敬武遣將曹
 全晸率軍赴之,以瑄隸焉。以戰功累遷列校。賊敗出關,全晸以本軍還鎮。會鄆帥薛崇卒,部將崔君預據城叛,全晸攻之,殺君預自為留後。瑄以功授濮州刺史、鄆州馬步軍都將。光啟初,魏博韓允中攻鄆,全晸為其所害。瑄據城自若。三軍推為留後。允中敗,朝廷以瑄為天平軍節度使,累加官至檢校太尉、同平章事。太祖初鎮大梁,兵威未振,連歲為秦宗權所圍,士不解甲,危殆日數四。太祖以瑄同宗,早兄事之,乃遣使求援于瑄。光啟末,
 宗權急攻大梁,瑄與弟瑾率兗、鄆之師來援,大敗蔡賊,解圍而遁。太祖感其力,厚禮以歸之。先是,瑄、瑾駐于大梁,睹太祖軍士驍勇,私心愛之。及歸,厚懸金帛于界上以誘焉。諸軍貪其厚利,私遁者甚眾。太祖移牒以讓之,瑄來詞不遜,由是始構隙焉。及秦宗權敗,太祖移軍攻時溥于徐州。時瑄方右溥,乃遣使來告太祖曰:「巢、權繼為蛇虺,毒螫中原,與君把臂同盟,輔車相依。今賊已平殄,人粗聊生,吾弟宜念遠圖,不可自相魚肉。或行人之
 失辭,疆吏之踰法,可以理遣,未得便睽和好。投鼠忌器,弟幸思之。」太祖方怒時溥通于孫儒,不從其言。及龐師古攻徐州,瑄出師來援,太祖深銜之。徐既平,太祖併兵以攻鄆,自景福元年冬遣朱友裕領軍渡濟,至乾寧三年宿軍齊、鄆間,大小凡數十戰,語在《太祖紀》中。自是野無人耕,屬城悉為我有。瑄乃遣人求救于太原,李克用遣其將李承嗣、史儼等援之。尋為羅宏信所扼,援路既絕,瑄、瑾竟敗。乾寧四年正月,龐師古攻陷鄆州,遁至中
 都北,匿于民家,為其所箠,并妻榮氏擒之來獻,俱斬于汴橋下。



 朱瑾,瑄從父弟。雄武絕倫,性頗殘忍。光啟中,瑾與兗州節度使齊克讓婚,瑾自鄆盛飾車服,私藏兵甲,以赴禮會。親迎之夜,甲士竊發,擄克讓,自稱留後。及蔡賊鴟張,瑾與太祖連衡,同討宗權,前後屢捷,以功正授兗州節度使。既得士心,有兼并天下之意,太祖亦忌之。瑾以厚利招誘太祖軍士,以為間諜。及太祖攻鄆,瑾出師來援,
 累與太祖接戰。乾寧二年春,太祖令大將朱友恭攻瑾,掘塹柵以環之。朱瑄遣將賀瑰及蕃將何懷寶赴援,為友恭所擒。十一月,瑾從兄齊州刺史瓊以州降。太祖令執賀瑰、懷寶及瓊以徇于城下,語曰:「卿兄已敗,早宜效順。」瑾偽遣牙將瑚兒持書幣送降,太祖自至延壽門外,與瑾交語。瑾謂太祖曰:「欲令大將送符印,願得兄瓊來押領,所貴骨肉,盡布腹心也。」太祖遣瓊與客將劉捍取符笥,瑾單馬立于橋上,揮手謂捍曰:「可令兄來,餘有密
 款。」即令瓊往。瑾先令騎士董懷進伏于橋下,及瓊至,懷進突出,擒瓊而入,俄而斬瓊首投于城外,太祖乃班師。



 及鄆州陷,龐師古乘勝攻兗,瑾與李承嗣方出兵求芻粟于豐沛間,瑾之二子及大將康懷英、判官辛綰、小校閻寶以城降師古。瑾無歸,即與承嗣將麾下士將保沂州,刺史尹處賓拒關不納,乃保海州。為師古所迫,遂擁州民渡淮依楊行密。行密表瑾領徐州節度使。龐師古渡淮,行密令瑾率師以禦之,清口之敗,瑾有力焉。自是
 瑾率淮軍連歲北寇徐、宿,大為東南之患。


及行密卒,子渭繼立,以徐溫子知訓為行軍副使,寵遇頗深。後楊溥僭號,知訓為樞密使,知政事,以瑾為同平章事,仍督親軍。時徐溫父子恃寵專政,慮瑾不附己。
 \gezhu{
  陳彭年《江南別錄》云:徐知訓初學兵法於朱瑾,瑾悉心教之。後與瑾有隙,夜遣壯士殺瑾,瑾手刃數人,埋于舍後。}
 貞明四年六月,出瑾為淮寧軍節度使。知訓設家宴以餞瑾,瑾事之逾遜。翼日,詣知訓第謝,留門久之,知訓家僮私謂瑾曰:「政事相公此夕在白牡丹妓院,侍者無得往。」瑾謂典謁
 曰:「吾不奈朝饑,且歸。」既而知訓聞之,愕然曰:「晚當過瑾。」瑾厚備供帳。瑾有所乘名馬,冬以錦帳貯之,夏以羅幬護之。愛妓桃氏,有絕色,善歌舞。及知訓至,奉卮酒為壽,初以名馬奉,知訓喜而言曰:「相公出鎮,與吾暫別,離恨可知,願此盡歡。」瑾即延知訓于中堂,出桃氏,酒既醉,瑾斬知訓首,示其部下。馬令《南唐書》云:知訓因求馬于瑾,瑾不與,遂有隙。俄出瑾為靜淮節度使。瑾詣知訓別,且願獻前馬。知訓喜,往謁瑾家。瑾妻出拜,知訓答拜,瑾以笏擊踣,遂斬知訓。因以其眾急趨衙城,知訓之黨已闔門矣,唯瑾得獨入,與衙兵
 戰。復踰城而出,傷足,求馬不獲,遂自刎。暴其屍于市,盛夏無蠅蛆,徐溫令投之于江,部人竊收葬之。溫疾亟,夢瑾被髮引滿將射之。溫乃為之禮葬,立祠以祭之。
 \gezhu{
  馬令《南唐書》云:初,宿衛將李球、馬謙挾楊隆演登樓,取庫兵以誅知訓,陣于門橋。知訓與戰,頻卻。朱瑾適自外來,以一騎前視其陣,曰:「不足為也。」因反顧一麾,外兵爭進,遂斬球、謙,亂兵皆潰。瑾嘗有德于知訓者也,及其凶終,吳人皆謂曲在知訓。《五代史補》:瑾之奔淮南也,時行密方圖霸,其為禮待,有加于諸將數等。瑾感行密見知,欲立奇功為報,但憾無入陣馬,忽忽不樂。一日晝寢,夢老叟,眉髮皓然,謂謹曰:「君長憾無入陣馬,今馬生矣!」及廄隸報,適退槽馬生一駒,見臥未能起。瑾驚曰:「何應之速也!」行往視之,見骨目皆非常馬,大喜曰:「事辦矣!」其後破杜
  洪,取鐘傳,未嘗不得力焉。初,瑾之來也,徐溫睹其英烈,深忌之,故瑾不敢預政。及行密死,子溥嗣位,溫與張鎬爭權,襲殺鎬,自是事無大小,皆決于溫。既而溫復為自安之計,乃以子知訓自代,然後引兵出居金陵,實欲控制中外。知訓尤恣橫,瑾居常嫉之。一旦知訓欲得瑾所乘馬,瑾怒,遂擊殺知訓,提其首請溥起兵誅溫。溥素怯懦,見之掩面而走。瑾曰:「老婢兒不足為計!」亦自殺,中外大駭且懼。溫至,遽以瑾尸暴之市中。時盛暑,肌肉累日不壞,至青蠅無敢輒泊。人有病者,或於暴屍處取土煎而服之,無不愈。}



 時溥,徐州人,初為州之驍將。唐中和初,秦宗權據蔡州,侵寇鄰籓,節度使支詳命溥率師以討之,徐軍屢捷,軍情歸順,以節鉞授之。《舊唐書》列傳云:時溥,彭城人,徐之牙將。黃巢據長安,詔徵天下兵進討。
 中和二年,武寧軍節度使支詳遣溥與副將陳璠率師五千赴難,行至河陰,軍亂,剽河陰縣回。溥招合撫諭,其眾復集,懼罪,屯于境上。詳遣人迎犒,悉恕之,溥乃移軍向徐州。既入,軍人大呼,推溥為留後。送詳于大彭館。溥大出資裝,遣陳璠援詳歸京。詳宿七里亭,其夜為璠所殺,舉家屠害。溥以璠為宿州刺史,竟以違命殺詳,溥誅璠,又令別將帥軍三千赴難京師。天子還宮,授之節鉞。及黃巢攻陳州,秦宗權據蔡州,與賊連結,徐、蔡相近,溥出師討之,軍鋒益盛,每戰屢捷。黃巢之敗也,其將尚讓以數千人降溥。後林言又斬黃巢首歸徐州。時溥功居第一,詔授檢校太尉、中書令、鉅鹿郡王。宗權未平,仍授溥徐州行營兵馬都統。蔡賊平,朱全忠與之爭功,遂相嫌怨。淮南亂,朝廷以全忠遙領淮南節度,以平孫儒、行密之亂。汴人應援,路出徐方,溥阻之。全忠怒,出師攻徐。自光啟至大順,六七年間,汴軍四集,徐、泗三郡,民無耕
 稼,頻歲水災,人喪十六七。溥窘蹙,求和於汴,全忠曰:「移鎮則可。」朝廷以尚書劉崇望代溥,以溥為太子太師。溥懼出城見害,不受代。汴將龐師古陳兵於野,溥求援於兗州,朱瑾出兵救之;值大雪,糧盡而還。城中守陴者飢甚,加之疾疫,汴將王重師、牛存節夜乘梯而入,溥與妻子登樓自焚而卒,實景福二年也。地入于汴。



 王師範,青州人。父敬武,初為平盧牙將。唐廣明元年,無棣人洪霸郎合群盜于齊、棣間,節度使安師儒遣敬武討平之。及巢賊犯長安,諸籓擅易主帥,敬武乃逐師儒,自為留後。王鐸承制授以節鉞,後以出師勤王功,加太尉、平章事。龍紀中,敬武卒,師範年幼,三軍推之為帥,棣
 州刺史張蟾叛于師範,不受節度,朝廷乃以崔安潛為平盧帥,師範拒命。張蟾迎安潛至郡,同討師範。師範遣將盧宏將兵攻蟾,宏復叛,與蟾通謀,偽旋軍將襲青州。師範知之,遣重賂迎宏,謂之曰:「吾以先人之故,為軍府所推,年方幼少,未能幹事。如公以先人之故,令不乏祀,公之仁也。如以為難與成事,乞保首領,以守先人墳墓,亦惟命。」宏以師範年幼,必無能為,不為之備。師範伏兵要路,迎而享之,預謂紀綱劉鄩曰:「翼日盧宏至,爾即
 斬之,酬爾以軍校。」鄩如其言,斬宏于座上,及同亂者數人。因戒厲士眾,大行頒賞,與之誓約,自率之以攻棣州,擒張蟾,斬之。安潛遁還長安。師範雅好儒術,少負縱橫之學,故安民禁暴,各有方略,當時籓翰咸稱之。



 及太祖平兗、鄆,遣硃友恭攻之,師範乞盟,遂與通好。天復元年冬,李茂貞劫遷車駕幸鳳翔,韓全誨矯詔加罪于太祖,令方鎮出師赴難。詔至青州,師範承詔泣下曰:「吾輩為天子籓籬,君父有難,略無奮力者,皆強兵自衛,縱賊如
 此,使上失守宗祧,危而不持,是誰之過,吾今日成敗以之!」乃發使通楊行密,遣將劉鄩襲兗州,別將襲齊。時太祖方圍鳳翔,師範遣將張居厚部輿夫二百,言有獻于太祖。至華州城東,華將婁敬思疑其有異,剖輿視之,乃兵仗也。居厚等因呼,殺敬思,聚眾攻西城。時崔允在華州,遣部下閉關拒之,遂遁去。是日,劉鄩下兗州,河南數十郡同日發。太祖怒,遣朱友寧率軍討之。既而友寧為青軍所敗,臨陣被擒,傳首于淮南。天復三年七月,太祖
 復令楊師厚進攻,屯于臨朐。師厚屢敗青軍,遂進寨于城下。師範懼,乃令副使李嗣業詣師厚乞降,《新唐書》云:師厚圍青州,敗師範兵於臨朐,執諸將,又獲其弟師克。是時師範眾尚十餘萬,諸將請決戰,而師範以弟故,乃請降。太祖許之。歲餘,遣李振權典青州事,因令師範舉家徙汴。師範將至,縞素乘驢,請罪于太祖。太祖以禮待之,尋表為河陽節度使。會韓建移鎮青州,太祖帳餞于郊,師範預焉。太祖謂建曰:「公頃在華陰,政事之暇,省覽經籍,此亦士君子之大務。今之青土,政簡務暇,可復修華陰
 之故事。」建捴謙而已。太祖又曰:「公讀書必須精意,勿錯用心。」太祖以師範好儒,前以青州叛,故以此言譏之。及太祖即位,徵為金吾上將軍。



 開平初,太祖封諸子為王,友寧妻號訴于太祖曰:「陛下化家為國,人人皆得封崇。妾夫早預艱難,粗立勞效,不幸師範反逆,亡夫橫屍疆場。冤仇尚在朝廷,受陛下恩澤,亡夫何罪!」太祖淒然泣下曰:「幾忘此賊。」即遣人族師範于洛陽。先掘坑于第側,乃告之,其弟師誨、兄師悅及兒姪二百口,咸盡戮焉。時
 使者宣詔訖,師範盛啟宴席,令昆仲子弟列座,謂使者曰:「死者人所不能免,況有罪乎!然予懼坑屍于下,少長失序,有愧于先人。」行酒之次,令少長依次于坑所受戮,人士痛之。後唐同光三年三月,詔贈太尉。



 劉知俊,字希賢,徐州沛縣人也。姿貌雄傑,倜儻有大志。始事徐帥時溥,為列校應是凝靜專一而固守其道。,溥甚器之,後以勇略見忌。唐大順二年冬,率所部二千人來降,即署為軍校。知俊披甲上馬,輪劍入敵,勇冠諸將。太祖命左右義勝兩軍隸之,
 尋用為左開道指揮使,故當時人謂之「劉開道」。後討秦宗權及攻徐州,皆有功,尋補徐州馬步軍都指揮使。攻海州下之,遂奏授刺史。天復初,歷典懷、鄭二州,從平青州,以功奏授同州節度使。天祐三年冬,以兵五千破岐軍六萬于美原。自是連克鄜、延等五州,乃加檢校太傅、平章事。開平二年春三月,命為潞州行營招討使。知俊未至潞,夾寨已陷,晉人引軍方攻澤州,聞知俊至,乃退。尋改西路招討使。六月,大破岐軍于幕谷,俘斬千計,李
 茂貞僅以身免。三年五月,加檢校太尉、兼侍中,封大彭郡王。



 時知俊威望益隆,太祖雄猜日甚,會佑國軍節度使王重師無罪見誅,知俊居不自安,乃據同州叛,《鑒戒錄》云:彭城王劉知俊鎮同州日,因築營墻,掘得一物,重八十餘斤,狀若油囊,召賓幕將校問之。劉源曰:「此是冤氣所結,古來囹圄之地或有焉。昔王充據洛陽,修河南府獄,亦獲此物。源聞酒能忘憂,奠以醇醪,或可消釋耳。然此物之出,亦非吉徵也。」知俊命具酒饌祝酹,復瘞之。尋有叛城背主之事。送款於李茂貞。又分兵以襲雍、華,雍州節度使劉捍被擒,送鳳翔害之,華州蔡敬思被傷獲免。太祖聞知俊叛,遣近臣諭之曰:「朕
 待卿甚厚,何相負耶?」知俊報曰:「臣非背德,但畏死耳!王重師不負陛下,而致族滅!」太祖復遣使謂知俊曰:「朕不料卿為此。昨重師得罪,蓋劉捍言陰結邠、鳳,終不為國家用。我今雖知枉濫,悔不可追,致卿如斯,我心恨恨,蓋劉捍誤予事也,捍一死固未塞責。」知俊不報,遂分兵以守潼關。太祖命劉鄩率兵進討,攻潼關,下之。時知俊弟知浣為親衛指揮使,聞知俊叛,自洛奔至潼關,為鄩所擒,害之。尋而王師繼至,知俊乃舉族奔于鳳翔;李茂貞
 厚待之,偽加檢校太尉、兼中書令,以土疆不廣,無籓鎮以處之,但厚給俸祿而已。尋命率兵攻圍靈武,且圖牧圉之地。靈武節度使韓遜遣使來告急,太祖令康懷英率師救之,師次邠州長城嶺,為知俊邀擊,懷英敗歸。《九國志》云:李彥琦、劉知俊自靈武班師,塗經長城嶺,梁師率精銳數萬躡其後,彥琦與知俊同設方略,擊敗之。茂貞悅,署為涇州節度使。復命率眾攻興元,進圍西縣,會蜀軍救至,乃退。《九國志·王宗金歲傳》云:岐將劉知俊等領大軍分路來攻,由階、成路奪固鎮糧,王宗侃、唐襲等禦之,至青泥嶺,為知俊所敗,退保西縣。會大雨,漢江漲,宗金歲自羅村得鄉導,緣山而行數百里,與
 宗播遇于鐵谷,合軍出湯頭。時知俊自斜谷山南直抵興州,圍西縣,軍人散掠巴中,宗金歲與宗播襲之。會王建亦至,遂解西縣之圍。



 既而為茂貞左右石簡顒等間之,免其軍政,寓于岐下,掩關歷年。茂貞猶子繼崇鎮秦州,因來寧覲「心」為意識本身,「權,然後知輕重;度,然後知長短;然,言知俊途窮至此,不宜以讒嫉見疑,茂貞乃誅簡顒等以安其心。繼崇又請令知俊挈家居秦州,以就豐給,茂貞從之。未幾,邠州亂,茂貞命知俊討之。時邠州都校李保衡納款于朝廷,末帝遣霍彥威率眾先入于邠,知俊遂圍其城,半載不能下。會李繼崇以秦州降于蜀,知俊
 妻孥皆遷于成都,遂解邠州之圍而歸岐陽。以舉家入蜀,終慮猜忌,因與親信百餘人夜斬關奔蜀。王建待之甚至,即授偽武信軍節度使,尋命將兵伐岐,不克,班師,因圍隴州,獲其帥桑宏志以歸。久之,復命為都統,再領軍伐岐。時部將皆王建舊人,多違節度,不成功而還,蜀人因而毀之。先是,王建雖加寵待,然亦忌之,嘗謂近侍曰:「吾漸衰耗,恒思身後。劉知俊非爾輩能駕馭,不如早為之所。」又嫉其名者于里巷間作謠言云:「黑牛出圈棕
 繩斷。」知俊色黔而丑生,棕繩者,王氏子孫皆以「宗」、「承」為名,故以此構之。偽蜀天漢元年冬十二月,建遣人捕知俊,斬于成都府之炭市。及王衍嗣偽位,以其子嗣禋尚偽峨眉長公主,拜駙馬都尉。後唐同光末,隨例遷于洛,卒。



 知俊族子嗣彬,幼從知俊征行,累遷為軍校。及知俊叛,以不預其謀,得不坐。貞明末若干斷片。,大軍與晉王對壘于德勝,久之,嗣彬率數騎奔于晉,具言朝廷軍機得失;又以家世仇怨,將以報之。晉王深信之,即厚給田宅,仍賜錦
 衣玉帶,軍中目為「劉二哥」。居一年,復來奔,當時晉人謂是刺客,以晉王恩澤之厚,故不竊發。龍德三年冬,從王彥章戰于中都,軍敗,為晉人所擒。晉王見之,笑謂嗣彬曰:「爾可還予玉帶。」嗣彬惶恐請死,遂誅之。



 楊崇本,不知何許人,幼為李茂貞之假子,因冒姓李氏,名繼徽。唐光化中,茂貞表為邠州節度使。天復元年冬,太祖自鳳翔移軍北伐,駐旆于邠郊,命諸軍攻其城。崇本懼,出城請降。太祖復置為邠州節度使,仍令復其本
 姓名焉。及師還,遷其族于河中。其後,太祖因統戎往來由于蒲津,以崇本妻素有姿色,嬖之于別館。其婦素剛烈,私懷愧恥,遣侍者讓崇本曰:「丈夫擁旄仗鉞,不能庇其伉儷,我已為朱公婦,今生無面目對卿,期于刀繩而已。」崇本聞之,但灑淚含怒。及昭宗自鳳翔回京,崇本之家得歸邠州。崇本恥其妻見辱,因茲復貳于太祖。乃遣使告茂貞曰:「朱氏兆亂,謀危唐祚,父為國家磐石,不可坐觀其禍,宜于此時畢命興復,事茍不濟,死為社稷可
 也。」茂貞乃遣使會兵於太原。時西川王建亦令大將出師以助之,岐、蜀連兵以攻雍、華,關西大震。太祖遣郴王友裕帥師禦之,會友裕卒于行,乃班師。天祐三年冬十月,崇本復領鳳翔、邠、涇、秦、隴之師,會延州胡章之眾,合五六萬,屯于美原,列柵十五,其勢甚盛。太祖命同州節度使劉知俊及康懷英帥師拒之,崇本大敗,復歸于邠州,自是垂翅久之。乾化元年冬,為其子彥魯所毒而死。



 彥魯自稱留後,領其軍事,凡五十餘日,為崇本養子李
 保衡所殺。保衡舉其城來降,末帝命霍彥威為邠帥,由是邠、寧復為末帝所有。



 蔣殷,不知何許人。幼孤,隨其母適于河中節度使王重盈之家,重盈憐之,畜為己子。唐天復初,太祖既平蒲、陜,殷與從兄珂舉族遷于大梁。太祖感王重榮之舊恩,凡王氏諸子,皆錄用焉,殷由是繼歷內職,累遷至宣徽院使。殷素與庶人友珪善,友珪篡立,命為徐州節度使。乾化四年秋,末帝以福王友璋鎮徐方,殷自以為友珪之
 黨,懼不受代,遂堅壁以拒命。時華州節度使王瓚,殷之從弟也,懼其連坐,上章言殷本姓蔣,非王氏之子也。末帝乃下詔削奪殷在身官爵,仍令卻還本姓,命牛存節、劉鄩等率軍討之。是時,殷求救于淮南,楊溥遣朱瑾率眾來援,存節等逆擊,敗之。貞明元年春,存節、劉鄩攻下徐州,殷舉族自燔而死,于火中得其屍,梟首以獻之。



 張萬進,雲州人。初為本州小校,亡命投幽州,劉守光厚遇之,任為裨將。滄州劉守文,以弟守光囚父而竊據其
 位,自領兵問罪,尋敗于雞蘇。守光遂兼有滄、景之地,令其子繼威主留務。繼威年幼,未能政事,以萬進佐之,凡關軍政,一皆委任。繼威兇虐類父,嘗淫亂于萬進之家,萬進怒而殺之,《通鑒》云:乾化二年九月庚子,萬進遣使奉表降于梁。辛丑,以萬進為義昌留後。甲辰,改義昌為順化軍,以萬進為節度使。此傳疑有闕文。又遣使歸于晉。既而末帝遣楊師厚、劉守奇潛兵掠鎮、冀,因東攻滄州,萬進乞降。師厚表青州節度使,俄遷兗州,仍賜名守進。萬進性既輕險,專圖反側。貞明四年冬,據城叛命,遣使送款於
 晉王。末帝降制削其官爵,仍復其本名,遣劉鄩討之,晉人不能救。五年冬,萬進危蹙,小將邢師遇潛謀內應,開門以納王師,遂拔其城,萬進族誅。



 史臣曰:夫雲雷構屯,龍蛇起陸,勢均者交鬥,力敗者先亡,故瑄、瑾、時溥之流,皆梁之吞噬,斯亦理之常也。惟瑾始以竊發有土,終以竊發亡身,《傳》所謂「君以此始,必以此終」者乎!師範屬衰季之運,以興復為謀,事雖不成,忠則可尚,雖貽族滅之禍,亦可以與臧洪遊于地下矣。知
 俊驍武有餘,奔亡不暇,六合雖大,無所容身,夫如是則豈若義以為勇者乎!崇本而下,俱以叛滅,又何足以道哉!



\end{pinyinscope}