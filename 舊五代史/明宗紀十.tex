\article{明宗紀十}

\begin{pinyinscope}

 長興四年春正月戊寅朔,帝御明堂殿受朝賀,仗衛如式。是日雪盈尺。戊子,秦王從榮加守尚書令、兼侍中,依前河南尹,判六軍諸衛事。庚寅,以端明殿學士、尚書兵
 部侍郎劉煦為中書侍郎、平章事。甲午,正衙命使冊故福慶長公主孟氏為晉國雍順長公主,遣太常卿崔居儉赴西川行冊禮。突厥內附。庚子,以前河東節度使李從溫為鄆州節度使。



 二月癸丑朔,帝於便殿問範延光內外見管馬數,對曰:「三萬五千匹。」帝嘆曰:「太祖在太願,騎軍不過七千,先皇自始至終馬才及萬。今有鐵馬如是,而不能使九州混一,是吾養士練將之不至也。吾老矣,馬將奈何!」延光奏曰:「臣每思之,國家養馬太多,試計
 一騎士之費,可贍步軍五人,三萬五千騎抵十五萬步軍,既無所施,虛耗國力,臣恐日久難繼。」帝曰:「誠如卿言,肥騎士而瘠吾民,何益哉!」《五代會要》:上問見管馬數,樞密使範延光奏:「天下常支草粟者近五萬匹。見今西北諸道蕃賣馬者往來如市,其郵傳之費、中估之直,日以四十五貫,以臣計之,國力十耗其七,馬無所使,財賦漸消,朝廷甚非所利。」上善之。十月,敕沿邊籓鎮,或有蕃部賣馬,可擇其良壯給券,具數以聞。丁巳,以虔州節度使、檢校太尉、兼侍中馬希振為洪州節度使;以鄂州節度使馬希廣為檢校太尉、同平章事,充桂州節度使;以盧州節度使兼武安軍副使姚彥
 章為檢校太尉、同平章事;以靜江節度副使馬希範為鄂州節度使。故潞州節度使、檢校太保康君立贈太傅。己未,宋州節度使安元信加兼侍中。濮州進重修河堤圖,沿河地名,歷歷可數。帝覽之,愀然曰:「吾佐先朝定天下,於此堤塢間小大數百戰。」又指一邱曰:「此吾擐甲臺也。時事如昨,奄忽一紀,令人悲嘆耳!」癸亥,以西川節度使孟知祥為劍南東、西兩川節度使,封蜀王。三司奏:「當省有諸道鹽鐵轉運使衙職員都押衙、正押衙、同押
 衙、通引、衙前虞候、子弟,今欲列為三司職名。」從之。庚午,以御史中丞崔衍為兵部侍郎,以右諫議大夫龍敏為御史中丞。



 三月己卯,幸龍門。延州節度使安從進奏,夏州節度使李仁福卒,其子彞超自稱留後。甲申,鎮州奏,行軍司馬趙瑰、節度判官陸浣、元從押衙高知柔等並棄市,坐受賂枉法殺人也。節度使李從敏罰一季俸。乙酉,以西川節度副使、知武泰軍節度兵馬留後趙季良為檢校太保、黔南節度使;以西川諸軍馬步都指揮使、
 知武信軍節度兵馬留後李仁罕為檢校太傅、遂州節度使;以西川左廂馬步指揮使、知保寧軍節度兵馬留後趙廷隱為檢校太保、閬州節度使;以西川右廂馬步都指揮使、知寧江軍兵馬留後張知業為檢校司徒、夔州節度使;以西川衙內馬步都指揮使、知昭武軍兵馬留後李肇為檢校太保、利州節度使,從孟知祥之請也。丙戌,賜宰相李愚絹百匹、錢十萬、鋪陳物一十三件。時愚病,帝令近臣翟光鄴宣問,所居寢室,蕭然四壁,病榻弊
 氈而已。光鄴具言其事,故有是賜。戊子,以延州節度使安從進為夏州留後,以夏州左都押衙、四州防遏使李彞超為延州留後,仍命邠州節度使藥彥稠、宮苑使安重益帥師援送從進赴鎮。以左衛上將軍盧文進為潞州節度使,以右龍武統軍張溫為雲州節度使。庚寅,以鳳翔行軍司馬李彥琮為鹽州防禦使。時范延光等奏,請因夏州之師制置鹽州,故有是命。癸巳,以右威衛上將軍安重霸為同州節度使。己亥,以左龍武統軍符彥
 超為安州節度使。詔除放京兆、秦、岐、邠、涇、延、慶、同、華、興元十州長興元年二年系欠夏秋稅物,及營田莊宅務課利,以其曾輦運供軍糧料也。甲辰,故晉國夫人夏氏追冊皇后,有司上謚曰昭懿,從之。



 夏四月戊申,李彞超奏:「奉詔除延州留後,已受恩命訖,三軍百姓擁隔,未遂赴任。」帝遣閣門使蘇繼顏齎詔促彞超赴任。癸丑竹,後改字青主,別字公它。又有真山、濁翁、石山等別名。,以刑部侍郎劉贊為秘書監、秦王傅。《五代會要》:長興四年四月,以秘書監劉贊為秦王傅,前忠武軍節度判官蘇瓚為秦王友,前襄州觀察使魚崇遠為秦王府記室參軍。時言事者請為秦
 王置師傅,上顧問近臣,皆以秦王名勢隆盛,不敢置議,請自選擇,乃降是命。甲寅,前鄧州節度使梁漢顒以太子少師致仕,太子賓客裴皞以兵部尚書致仕。戊午,追冊昭宗皇后何氏為宣穆皇后,祔饗太廟,百僚進名奉慰,廢朝三日。己巳,以左散騎常侍任贊為戶部侍郎,以吏部侍郎藥縱之為曹州刺史。癸酉,延州奏,蕃部劫掠餉運及攻城之具,守蘆關兵士退守全明鎮。


五月丙子朔,帝御文明殿受朝。戊寅,皇子鳳翔節度使從珂封潞王。新授戶部侍郎任贊改刑部侍郎,
 贊訴以所授官是丁憂闕,故改正。皇子從益封許王,鄆州節度使李從溫封兗王,河中節度使李從璋封洋王,鎮州節度使李從敏封涇王。甲申,帝避暑于九曲池,既而登樓,風毒暴作,聖體不豫,翼日而愈。
 \gezhu{
  《北夢瑣言》云:上聖體乖和,馮道對寢膳之間,動思調衛,因指御前果實曰:「如食桃不康,翼日見李而思戒可也。」初,上因御李,暴得風虛之疾,馮道不敢斥言,因奏事諷悟上意。}
 丙戌,契丹遣使朝貢。丁酉,安從進奏,大軍已至夏州,攻城,以其不受命也。庚子,以靈武留後張希崇為本州節度使。辛丑,故夏州節度使、朔方郡王李仁
 福追封虢王。壬寅,以前晉州留後薄文為本州節度使。



 六月丙午朔,文武百僚、宰臣馮道等拜章,請於尊號內加「廣運法天」四字,凡拜三章,詔允之。詔宮西新園宜名永芳園,其間新殿宜名和慶殿。丙辰,秦王從榮加食邑至萬戶,實封二千戶。丁巳,以右驍衛上將軍李從昶為左龍武統軍,以前邢州節度使高允韜為右龍武統軍,以右驍衛上將軍羅周敬為左羽林統軍,以右監門上將軍婁繼英為金州刺史。戊午,宋王從厚加食邑至
 萬戶,實封一千戶。壬戌,以前涇州節度使李金全為滄州節度使。癸亥,詔御史中丞龍敏等詳定《大中統類》。甲子,第十四女封壽安公主,第十五女封永樂公主。戊辰,以前利州節度使孫漢韶為洋州節度使。壬申,永寧軍節度使、容州管內觀察使、檢校太尉、兼侍中馬存加食邑實封。甲戌,帝復不豫。



 秋七月丁丑,以著作佐郎尹拙為左拾遺,直史館。國朝舊制,皆以畿赤尉直史館,今用諫官自拙始,從監修李愚奏也。己卯,東岳三郎神贈威
 雄大將軍。初,帝不豫,前淄州刺史劉遂清薦泰山僧一人,云善醫,及召見,乃庸僧耳。問方藥,僧曰:「不工醫,嘗于泰山中親睹嶽神,謂僧曰:『吾第三子威靈可愛,而未有爵秩,師為我請之。』」宮中神其事,故有是命,識者嫉遂清之妖佞焉。詔應臺官出行,須令人訶引,使軍巡職掌等規避。壬午,詔安從進班師,時王師攻夏州無功故也。乙酉,以許州節度使孟鵠卒廢朝,贈太傅。詔賜在京諸軍將校優給有差。時帝疾未痊,軍士有流言故也。丁亥,兩
 浙節度使、檢校太傅、守中書令錢元瓘封吳王。



 八月戊申,帝被袞冕,御明堂殿受冊,徽號曰聖明神武廣運法天文德恭孝皇帝。禮畢,制大赦天下,常赦所不原者咸赦除之。己酉,賜侍衛諸軍優給有差。時月內再有頒給,自茲府藏無餘積矣。辛亥,以晉州節度使薄文卒廢朝。丁巳,以右龍武統軍李從昶為許州節度使。戊午,以秘書監高輅卒廢朝。辛酉,以太子太師致仕符習卒廢朝,贈太師。辛未,秦王從榮以本官充天下兵馬大元帥,加
 食邑萬戶,實封三千戶;以右羽林統軍翟璋為晉州節度使;以太子賓客馬鎬為戶部侍郎。壬申,幸至德宮。



 九月甲戌,以戶部尚書李鈴為兵部尚書,以前戶部尚書韓彥惲為戶部尚書。丙子,幸至德宮。戊寅,樞密使范延光、趙延壽並加兼侍中,依前充使。中書奏:「元帥儀注,諸道節度使以下帶兵權者,階下具軍禮參見;其帶使相者,初見亦展一度公禮。天下軍務公事,元帥府行帖指揮,其判六軍諸衛事則公牒往來,其官屬軍職,委元帥
 府奏請。」從之。癸未,以兵部侍郎盧詹為吏部侍郎。丙戌,宰臣馮道加左僕射,李愚加吏部尚書,劉煦加刑部尚書。戊子,河陽節度使兼侍衛親軍都指揮使康義誠、山南西道節度使檢校太傅張虔釗並加同平章事。宣徽南院使、判三司馮贇依前檢校太傅、同平章事中書門下二品,充三司使。贇亡父名章,故改平章事為同二品。壬戌,永寧公主石氏進封魏國公主,興平公主趙氏進封齊國公主;皇孫重光、重哲並授銀青光祿大夫、檢
 校工部尚書,秦王、宋王子也。前洋州節度使梁漢顒以太子少傅致仕。丁酉,以右龍武統軍高允韜為滑州節度使,以韶州刺史、檢校司空王萬榮為華州節度使。萬榮,王妃之父也。戊戌,以樞密使趙延壽為汴州節度使,以襄州節度使硃宏昭為檢校太尉、同平章事,充樞密使。時范延光、趙延壽相繼辭退樞密務,及朱宏昭有樞密之命,又面辭訴,帝叱之曰:「爾輩皆欲離朕左右,怕在眼前,素養爾輩,將何用也!」宏昭退謝,不復敢言。吏部侍
 郎張文寶卒。庚子,清海軍節度使錢元璹加檢校太傅、同平章事;中吳、建武等軍節度使錢元璙加檢校太師、兼中書令。以前滑州節度使李贊華遙領虔州節度使。辛丑,詔天下兵馬大元帥、秦王從榮班宜在宰臣之上。《五代會要》:秦王從榮加兼中書令,與宰臣分班左右定位,及為天下兵馬元帥。敕曰:「秦王位隆將相,望重磐維,委任既崇,等威合異,班位宜在宰臣之上。」壬寅,以北面行營都指揮使、易州刺史楊檀為振武軍節度使。



 冬十月丙午,以前同州節度使趙在禮為襄州節度使。丁未,以前滑州節度使張
 敬詢卒廢朝。以刑部侍郎任贊為兵部侍郎,充元帥府判官。戊午,以前鳳翔節度使孫岳為三司使。庚申,以樞密使范延光為鎮州節度使,以三司使馮贇為樞密使。辛酉,以前潞州節度使李承約為左龍武統軍;以前威寨軍節度使王景戡為右龍武統軍;以左驍衛上將軍安崇阮為左神武統軍;以右監門上將軍高允貞為右神武統軍。壬戌,以權知夏州事、檢校司空李彞超為夏州節度使、檢校司徒。丙寅,詔在朝文武臣僚並與加恩,
 以受冊尊號也。戊辰,以前安州節度使楊漢章為兗州節度使,以前雲州節度使張敬達為徐州節度使。庚午,以前兗州節度使張延朗為秦州節度使。壬申,秦州節度使劉仲殷移鎮宋州。



 十一月丙子,以前滄州節度使盧質為右僕射。庚辰,改慎州懷化軍為昭化軍,升洮州為保順軍。辛巳,以保大軍節度使、檢校太尉鮑君福為保順軍節度、洮鄯等州觀察等使;以彰義軍節度使、檢校太尉、同平章事杜建徽為昭化軍節度、慎瑞司等州
 觀察使。乙酉,以前汴州節度使李從嚴為鄆州節度使,以鄆州節度使李從溫為定州節度使。丙戌,新授右僕射盧質奏:「臣忝除官,合赴省上事,若準舊例,左右僕射上事儀注所費極多,欲從權務簡,只取尚書丞、郎上事例,止集南省屬僚及兩省官送上,亦不敢輒援往例,有費官用,自量力排比;兼不敢自臣隳廢前規,他時任行舊制。」從之。



 戊子,帝不豫。己丑,大漸,自廣壽殿移居雍和殿。是夜四鼓後,帝自御榻蹶然而興,顧謂知漏宮女曰:「
 今夜漏幾何?」對曰:「四更。」因奏曰:「官家省事否?」帝曰:「省。」因唾出肉片如肺者數片,便溺升餘。六宮皆至,慶躍而奏曰:「官家今日實還魂也。」已進粥一器,侍醫進湯膳。至曙,帝小康。壬辰,天下大元帥、守尚書令、兼侍中、秦王從榮領兵陣于天津橋,內出禁軍拒之。從榮敗奔河南府,遇害。帝聞之,悲駭,幾落御榻,氣絕而蘇者再,由是不豫有加。癸巳,馮道率百僚見帝于雍和殿,帝雨泣哽噎,曰:「吾家事若此,慚見卿等!」百僚皆泣下沾襟。甲午,賜宰臣、樞密
 使御衣玉帶,康義誠已下錦帛鞍馬有差。遣宣徽使孟漢瓊召宋王于鄴都。乙未,以三司使孫岳為亂兵所害廢朝。丁酉,敕秦王府官屬,除諮議參軍高輦已處斬外,元帥府判官、兵部侍郎任贊配武州,祕書監兼秦王傅劉贊配嵐州,河南少尹劉陟配均州,並為長流百姓,縱逢恩赦,不在放還。河南少尹李蕘配石州,河南府判官司徒詡配寧州,秦王友蘇瓚配萊州,記室參軍魚崇遠配慶州,河南府推官王說配隨州,並為長流百姓。河南
 府推官尹諲,六軍巡官董裔、張九思,河南府巡官張沆、李潮、江文蔚並勒歸田里。應長流人並除名。六軍判官、殿中監王居敏責授復州司馬,六軍推官郭晙責授坊州司戶,並員外置,所在馳驛發遣。時宰相、樞密使共議任贊等已下罪,馮道等曰:「任贊前在班行,比與從榮無舊,除官未及月餘,便逢此禍。王居敏、司徒詡疾病請假,將近半年,近日之事,計不同謀。從榮所款暱者高輦、劉陟、王說三人,昨從榮稱兵指闕之際,沿路只與劉陟、高
 輦並轡耳語,至天津橋南,指日影謂諸判官曰:『明日如今,已誅王居敏矣。』則知其冗泛之徒,不可一例從坐。」硃宏昭意欲盡誅任贊已下,馮贇力爭之乃已。戊戌,帝崩于大內之雍和殿,壽六十七。



 十二月癸卯朔,遷梓宮于二儀殿,宋王從厚自鄴都至。是日發哀,百僚縞素于位,中書侍郎、平章事劉煦宣遺制,宋王從厚于柩前即自帝位,服紀以日易月,一如舊制云。明年四月,太常卿盧文紀上謚議曰聖智仁德欽孝皇帝,廟號明宗,宰臣馮
 道議請改「聖智仁德」四字,為聖德和武欽孝皇帝。宰臣劉煦撰謚冊文,宰臣李愚撰哀冊文,是月二十七日葬於徽陵。《五代史補》:明宗之在位也,一日幸倉場觀納,時主者以車駕親臨,懼得罪,其較量甚輕。明宗因謂之曰:「且朕自省事以來,倉場給散,動經一二十年未畢,今輕量如此,其後銷折將何以償之?」對曰:「竭盡家產,不足則繼之以身命。」明宗愴然曰:「只聞百姓養一家,未聞一家養百姓。今後每石加二斗耗,以備鼠雀侵蠹,謂之鼠雀耗。」倉糧加耗,自此始也。《五代史闕文》:明宗出自邊地,老于戰陳,即位之歲,年已六旬,純厚仁慈,本乎天性。每夕宮中焚香仰天禱祝云「某蕃人也,遇世亂為眾推戴,事不獲已,願上天早生聖人,與百姓為主。」故天成、長興間,比歲豐登,中原無事,言于五代,粗為小康。



 史臣曰:明宗戰伐之勛,雖高佐命,潛躍之事,本不經心。會王室之多艱,屬神器之自至,諒由天贊,匪出人謀。及應運以君臨,能力行乎王化,政皆中道,時亦小康,近代已來,亦可宗也。儻使重誨得房、杜之術,從榮有啟、誦之賢,則宗祧未至於危亡,載祀或期於綿遠矣。惜乎!君親可輔,臣子非才,遽泯烝嘗,良可深嘆矣!



\end{pinyinscope}