\article{列傳十五}

\begin{pinyinscope}

 張
 全義,字國維,濮州臨濮人。初名居言,賜名全義,梁祖改為宗奭;莊宗定河南,復名全義。祖璉,父誠,世為田農。全義為縣嗇夫,嘗為令所辱。乾符末,黃巢起冤句,全義
 亡命入巢軍。巢入長安,以全義為吏部尚書,充水運使。巢敗,依諸葛爽于河陽,累遷至裨校,屢有戰功,爽表為澤州刺史。光啟初,爽卒,其子仲方為留後。部將劉經與李罕之爭據洛陽,罕之敗經于聖善寺,乘勝欲攻河陽,營于洛口。經遣全義拒之,全義乃與罕之同盟結義,返攻經于河陽,為經所敗,收合餘眾,與罕之據懷州,乞師于武皇。武皇遣澤州刺史安金俊助之,進攻河陽,劉經、仲方委城奔汴,罕之遂自領河陽,表全義為河南尹。



 全
 義性勤儉,善撫軍民,雖賊寇充斥,而勸耕務農,由是倉儲殷積。王始至洛,于麾下百人中,選可使者一十八人,命之曰屯將。每人給旗一口,榜一道,于舊十八縣中,令招農戶,令自耕種,流民漸歸。王于百人中,又選可使者十八人,命之曰屯副,民之來者撫綏之,除殺人者死,餘但加杖而已,無重刑,無租稅,流民之歸漸眾。王又于麾下選書計一十八人,命之曰屯判官。不一二年,十八屯中每屯戶至數千。王命農隙,選丁夫授以弓矢槍劍,為坐作進退之法。行之一二年,每屯增戶。大者六七千,次者四千,下之二三千,共得丁夫閑弓矢、槍劍者二萬餘人。有賊盜即時擒捕之,刑寬事簡,遠近歸之如市。五年之內,號為富庶,于是奏每縣除令簿主之。罕之貪暴不法,軍中乏食,每取給于全義。二人初相得甚歡,而至是求取無厭,
 動加凌轢,全義苦之。文德元年四月,罕之出軍寇晉、絳,全義乘其無備,潛兵襲取河陽,全義乃兼領河陽節度。《洛陽搢紳舊聞記》云:罕之鎮三城,知王專以教民耕織為務,常宣言于眾曰:「田舍翁何足憚。」王聞之,蔑如也。每飛尺書于王,求軍食及縑帛,王曰:「李太傅所要,不得不奉之。」左右及賓席咸以為不可與,王曰:「第與之。」似若畏之者,左右不曉。罕之謂王畏己,不設備。因罕之舉兵收懷、澤,王乃密召屯兵,潛師夜發,遲明入三城。罕之乃逃遁投河東,朝廷即授王兼鎮三城。罕之求援于武皇,武皇復遣兵攻敗河陽,會汴人救至而退。梁祖以丁會守河陽,全義復為河南尹、檢校司空。全義感梁祖援助之恩,自是依附,皆從
 其制。



 初,蔡賊孫儒、諸葛爽爭據洛陽,迭相攻伐,七八年間,都城灰燼,滿目荊榛。全義初至,惟與部下聚居故市,井邑窮民,不滿百戶。全義善於撫納,課部人披榛種藝,且耕且戰,以粟易牛,歲滋墾闢,招復流散,待之如子。每農祥勸耕之始,全義必自立畎畝,餉以酒食,政寬事簡,吏不敢欺。數年之間,京畿無閒田,編戶五六萬。乃築壘於故市,建置府署,以防外寇。《洛陽縉紳舊聞記》:王每喜民力耕織者,某家今年蠶麥善,去都城一舍之內,必馬足及之,悉召其家老幼,親慰勞之,賜以酒食茶彩,丈夫遺之布褲,婦人裙衫,時民
 間尚衣青,婦人皆青絹為之。取其新麥新繭,對之喜動顏色,民間有竊言者曰:「大王見好聲妓,等閒不笑,惟見好蠶麥即笑爾。」其真樸皆此類。每觀秋稼,見田中無草者,必下馬命賓客觀之,召田主慰勞之,賜之衣物。若見禾中有草,地耕不熟,立召田主集眾決責之。若苗荒地生,詰之,民訴以牛疲或闕人耕鋤,則田邊下馬,立召其鄰仵責之曰:「此少人牛,何不眾助之。」鄰仵皆伏罪,即赦之。自是洛陽之民無遠近,民之少牛者相率助之,少人者亦然。田夫田婦,相勸以耕桑為務,是以家有蓄積,水旱無饑民。王誠信,每水旱祈祭,必具湯沐,素食別寢,至祠祭所,儼然若對至尊,容如不足。遇旱,祈禱未雨,左右必曰:「王可開塔」,即無畏師塔也,在龍門廣化寺。王即依言而開塔,未嘗不澍雨,故當時俚諺云:「王禱雨,買雨具。」



 梁祖迫昭宗東遷,命全義繕治洛陽宮城,累年方集。昭宗至洛陽,梁祖將圖禪代,慮
 全義心有異同,乃以判官韋震為河南尹,遂移全義為天平軍節度使、守中書令、東平王。《洛陽搢紳舊聞記》:齊王與梁祖互為中書令、尚書令,及梁祖兼四鎮,齊王累表讓兼鎮,蓋潛識梁祖奸雄,避其權位,欲圖自全之計。梁祖經營霸業,外則干戈屢動,內則帑庾俱虛,齊王悉心盡力,傾竭財資助之。其年八月,昭宗遇弒,輝王即位。十月,復以全義為河南尹,兼忠武軍節度使、判六軍諸衛事。梁祖建號,以全義兼河陽節度使,封魏王。開平二年,冊拜太保、兼陜虢節度使、河陽尹。四年,冊拜太傅、河南尹、判六軍,兼鄭、滑等州節度使。乾化元年,冊拜
 太師。二年,朱友珪篡逆,以全義為守太尉、河南尹、宋亳節度使兼國計使。梁末帝嗣位於汴,以全義為洛京留守,兼鎮河陽。未幾,授天下兵馬副元帥。



 末帝季年,趙、張用事,段凝為北面招討使,驟居諸將之右。全義知其不可,遣使啟梁末帝曰:「老臣受先朝重顧,蒙陛下委以副元帥之名。臣雖遲暮,尚可董軍,請付北面兵柄,庶分宵旰。段凝晚進,德未服人,恐人情不和,敗亂國政。」不聽。全義托朱氏垂三十年,梁祖末年,猜忌宿將,欲害全義者
 數四,全義單身曲事,悉以家財貢奉。洎梁祖河朔喪師之後,月獻鎧馬,以補其軍;又以服勤盡瘁,無以加諸,故竟免於禍。全義妻儲氏,明敏有才略。梁祖自柏鄉失律後,連年親征河朔,心疑全義,或左右讒間,儲氏每入宮,委曲伸理。有時怒不可測,急召全義,儲氏謁見梁祖,厲聲言曰:「宗奭種田叟耳,三十餘年,洛城四面,開荒劚棘,招聚軍賦,資陛下創業。今年齒衰朽,指景待盡,而大家疑之,何也?」梁祖遽笑而謂曰:「我無惡心,嫗勿多言。」《洛陽
 搢紳舊聞記》云:梁祖猜忌王,慮為後患,前後欲殺之者數四,夫人儲氏面請梁祖得免,梁祖遂以其子福王納齊王之女。



 莊宗平梁,全義自洛赴覲,泥首待罪。莊宗撫慰久之,以其年老,令人掖而昇殿,宴賜盡歡,詔皇子繼岌、皇弟存紀等皆兄事之。先是,天祐十五年,梁末帝自汴趨洛,將祀於圓丘。時王師攻下楊劉,徇地曹、濮,梁末帝懼,急歸于汴,其禮不遂,然其法物咸在。至是,全義乃奏曰:「請陛下便幸洛陽,臣已有郊禮之備。」翌日,制以全義復為尚書令、魏王、河南尹。明年二月,郊禋禮畢,以全義為守太
 尉中書令、河南尹,改封齊王,兼領河陽。先是,朱梁時供御所費,皆出河南府,其後孔謙侵削其權,中官各領內司使務,或豪奪其田園居第,全義乃悉錄進納。四年,落河南尹,授忠武軍節度使、檢校太師、尚書令。會趙在禮據魏州,都軍進討無功。時明宗已為群小間諜,端居私第。全義以臥疾聞變,憂懼不食,薨于洛陽私第,時年七十五。天成初,冊贈太師,謚曰忠肅。



 全義歷守太師、太傅、太尉、中書令,封王,邑萬三千戶。凡領方鎮洛、鄆、陜、滑、宋,
 三蒞河陽,再領許州,內外官歷二十九任,尹正河、洛,凡四十年,位極人臣,善保終吉者,蓋一人而已。全義樸厚大度,敦本務實,起戰士而忘功名,尊儒業而樂善道。家非士族,而獎愛衣冠,開幕府辟士,必求望實,屬邑補奏,不任吏人。位極王公,不衣羅綺,心奉釋、老,而不溺左道。如是數者,人以為難。自莊宗至洛陽,趨向者皆由徑以希恩寵,全義不改素履,盡誠而已。言事者以梁祖為我世讎,宜斫棺燔柩,全義獨上章申理,議者嘉之。



 劉皇后嘗
 從莊宗幸其第,奏云:「妾孩幼遇亂,失父母,欲拜全義為義父。」許之。全義稽首奏曰:「皇后萬國之母儀,古今未有此事,臣無地自處。」莊宗敦逼再三,不獲已,乃受劉后之拜。既非所願,君子不以為非。然全義少長軍中,立性樸滯,凡百姓有詞訟,以先訴者為得理,以是人多枉濫,為時所非。又嘗怒河南縣令羅貫,因憑劉后譖于莊宗,俾貫非罪而死,露屍于府門,冤枉之聲,聞于遠近,斯亦良玉之微瑕也。《五代史闕文》:梁乾化元年七月辛丑,梁祖幸全義私第。甲辰,歸大內。梁史稱:「上不豫,
 厭秋暑,幸宗奭私第數日,宰臣視事于仁政亭,崇政諸司並止于河南府廨署。」世傳梁祖亂全義之家,婦女悉皆進御,其子繼祚不勝憤恥,欲剚刃于梁祖。全義止之曰:「吾頃在河陽,遭李罕之之難,引太原軍圍閉經年,啖木屑以度朝夕,死在頃刻,得他救援,以至今日,此恩不可負也。」其子乃止。梁史云云者,諱國惡也。臣謹案,《春秋》莊二年,《經》曰:「十有二月,夫人姜氏會齊侯于禚。」《傳》曰:「書姦也。」夫《經》言會者,諱惡,禮也;《傳》書姦者,暴其罪以垂誡也。又《莊》二十二年,《傳》書:陳完飲桓公酒,公曰:「以火繼之。」辭曰:「臣卜其晝,未卜其夜。」豈有天子幸人臣之家,留止數日,姦亂萌矣。況全義本出巢賊,敗依河陽節度使諸葛爽,爽用為澤州刺史,及爽死,全義事爽子仲方,即與李罕之同逐仲方,罕之帥河陽,全義為河南尹,未幾,又逐罕之,自據河陽,其翻覆也如此。自是托跡朱梁,斲喪唐室,惟勤勸課,其實斂民附賊,以固恩寵。梁時,月進鎧
 馬,以補軍實。及梁祖為友珪所弒,首進錢一百萬,以助山陵。莊宗平中原,全義合與敬翔、李振等族誅,又通賂於劉皇后,乘莊宗幸洛,言臣已有郊天費用。夫全義匹夫也,豈能自殖財賦,其剝下奉上也又如此。晚年保證明宗,欲為子孫之福,師方渡河,鄴都兵亂,全義憂憾不食,終以餓死。未死前,其子繼業訟弟汝州防禦使繼孫,莊宗貶房州司戶,賜自盡。其制略曰:「侵奪父權,惑亂家事,繼鳥獸之行,畜梟獍之心。」其御家無法也又如此。河南令羅貫,方正文章之士,事全義稍慢,全義怒告劉皇后,斃貫于枯木之下,朝野冤之。洛陽監軍使嘗收得李太尉平泉莊醒酒石,全義求之,監軍不與,全義立殺之,其附勢作威也又如此。蓋亂世賊臣耳,得保首領,為幸已多。晉天福中,其子繼祚謀反伏誅,識者知餘殃在其子孫也。臣讀《莊宗實錄》,見史官敘《全義傳》,虛美尤甚,至今負俗無識之士,尚以全義為名臣,故因補闕文,粗論事
 跡云。



 硃友謙,字德光,許州人,本名簡。祖巖,父琮,世為陳、許小校。廣明之亂,簡去鄉里,事澠池鎮將柏夔為部隸。嘗為盜于石壕、三鄉之間,剽劫行旅。後事陜州節度使王珙,積勞至軍校。珙性嚴急,御下無恩,牙將李璠者,珙深所倚愛,小有違忤,暴加箠擊,璠陰銜之。光化元年,珙與弟河中節度使珂相持,干戈日尋,珙兵屢敗,部伍離心。二年六月,璠殺珙,歸附汴人,梁祖表璠為陜州節度使。璠亦苛慘,軍情不葉,簡復攻璠,璠冒刃獲免,逃歸于汴。三
 年,梁祖表簡為陜州留後。九月,天子授以旄鉞。車駕在鳳翔,梁祖往來,簡事之益謹,奏授平章事。天復末,昭宗遷都洛陽,駐蹕于陜。時朝士經亂,簪裳不備,簡獻上百副,請給百官,朝容稍備。以迎奉功,遷檢校侍中。簡與梁祖同宗,乃陳情於梁祖曰:「僕位崇將相,比無勳勞,皆元帥令公生成之造也。願以微生灰粉為效,乞以姓名,肩隨宗室。」梁祖深賞其心,乃名之為友謙,編入屬籍,待遇同於己子。友謙亦盡心葉贊,功烈居多。



 梁祖建號,移授
 河中節度使、檢校太尉,累拜中書令,封冀王。及朱友珪弒逆,友謙意不懌,雖勉奉偽命,中懷怏怏。友珪征之,友謙辭以北面侵軼,謂賓友曰:「友珪是先帝假子,敢行大逆,餘位列維城,恩踰父子,論功校德,何讓伊人,詎以平生附託之恩,屈身于逆豎之手!」遂不奉命。其年八月,友珪遣大將牛存節、康懷英、韓勍攻之,友謙乞師于莊宗。莊宗親總軍赴援,與汴軍遇於平陽,大破之。《歐陽史》:晉王出澤潞以救之,追懷英于解縣,大敗之。追至白逕嶺,夜秉炬擊之,懷英又敗。因與友謙會於猗氏,友
 謙盛陳感慨,願敦盟約,莊宗歡甚。友謙乘醉鼾寢于帳中,莊宗熟視之,謂左右曰:「冀王真貴人也,但憾其臂短耳。」及梁末帝嗣位,以恩禮結其心;友謙亦遜辭稱籓,行其正朔。



 天祐十七年,友謙襲取同州,以其子令德為帥,請節鉞于梁,不獲。友謙即請之於莊宗,令幕客王正言以節旄賜之,梁將劉鄩、尹皓攻同州,友謙來告急,莊宗遣李嗣昭、李存審將兵赴之,敗汴軍于滑北,解圍而還。初,劉鄩兵至蒲中,倉儲匱乏,人心離貳,軍民將校,咸
 欲歸梁。友謙諸子令錫等亦說其父曰:「晉王雖推心于我,然懸兵赴援,急維相應,寧我負人,擇福宜重。請納款于梁,候劉鄩兵退後,與晉王修好。」友謙曰:「晉王親赴予急,夜半秉燭戰賊,面為盟誓,不負初心。昨聞吾告難,命將星行,助我資糧,分我衣屨,而欲翻覆背惠,所謂鄧祁侯云『人將不食吾餘』也。」及破梁軍,加守太尉、西平王。



 同光元年,莊宗滅梁,友謙覲于洛陽。莊宗置宴饗勞,寵錫無算,親酌觴屬友謙曰:「成吾大業者,公之力也。」既歸籓,
 請割慈、隰二郡,依舊隸河中,不許,詔以絳州隸之。又請解縣兩池榷鹽,每額輸省課,許之。及郊禮畢,以友謙為守太師、尚書令,進食邑至萬八千戶。三年,賜姓,名繼麟,編入屬籍,賜之鐵券,恕死罪。以其子令德為遂州節度使,令錫為許州節度使。一門三鎮,諸子為刺史者六七人,將校剖竹者又五六人,恩寵之盛,時無與比。



 莊宗季年,稍怠庶政,巷伯伶官,干預國事。時方面諸侯皆行賂遺,或求賂于繼麟,雖僶俛應奉,不滿其請。且曰:「河中土
 薄民貧,厚貺難辦。」由是群小咸怨,遂加誣構。郭崇韜討巴、蜀,徵師於河中,繼麟令其子令德率師赴之。伶官景進與其黨構曰:「昨王師初起,繼麟以為討己,頗有拒命之意,若不除移,如國家有急,必為後患。」郭崇韜既誅,宦官愈盛,遂構成其罪,謂莊宗曰:「崇韜強項于蜀,蓋與河中響應。」繼麟聞之懼,將赴京師,面訴其事。其部將曰:「王有大功於國,密邇京城,群小流言,何足介意。端居奉職,讒邪自銷,不可輕行。」繼麟曰:「郭公功倍於我,尚為人構
 陷,吾若得面天顏,自陳肝膈,則流言者獲罪矣。」四年正月,繼麟入覲。景進謂莊宗曰:「河中人有告變者,言繼麟與崇韜謀叛,聞崇韜死,又與李存乂構松逆,當斷不斷,禍不旋踵。」群閹異口同辭,莊宗駭惑不能決。是月二十三日,授繼麟滑州節度使。是夜,令朱守殷以兵圍其第,擒之,誅於徽安門外;詔繼岌誅令德于遂州,王思同誅令錫于許州,吳縝《篡誤》云:《伶官史彥瓊傳》,友謙有子建徽被殺。傳中未載。命夏魯奇誅其族于河中。初,魯奇至,友謙妻張氏率其家屬二百餘
 口見魯奇曰:「請疏骨肉名字,無致他人橫死。」將刑,張氏持先賜鐵券授魯奇曰:「皇帝所賜也。」是時,百口塗地,冤酷之聲,行路流涕。



 先是,河中衙城閽者夜見婦人數十,袨服靚妝,僕馬炫耀,自外馳騁,笑語趨衙城。閽者不知其故,不敢詰,至門排騎而入,既而扃鎖如故,復無人迹,乃知妖鬼也。又繼麟登逍遙樓,聞哭聲四合,詰日訊之,巷無喪者,隔歲乃族誅。及明宗即位,始下詔昭雪焉。



 史臣曰:全義一逢亂世,十領名籓,而能免梁祖之雄猜,
 受莊宗之厚遇,雖由恭順,亦系貨財。《傳》所謂「貨以籓身」者,全義得之矣。友謙向背為謀,二三其德,考其行事,亦匪純臣。然全族之誅,禍斯酷矣,得非鬼神害盈,而天道惡滿乎!



\end{pinyinscope}