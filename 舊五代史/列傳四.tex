\article{列傳四}

\begin{pinyinscope}

 李嗣昭,字益光,武皇母弟代州刺史克柔之假子也。小字進通,不知族姓所出。《歐陽史》云:本姓韓氏,汾州大谷縣民家子。少事克柔,頗謹愿,雖形貌眇小,而精悍有膽略,沉毅不群。初嗜
 酒好樂,武皇微伸儆戒,乃終身不飲。少從征伐,精練軍機。乾寧初,王珂、王珙爭帥河中,珙引陜州之軍攻珂,珂求救於武皇,乃令嗣昭將兵援之,敗珙軍于猗氏,獲賊將李璠等。四年,改衙內都將,復援河中,敗汴軍於胡壁堡,擒汴將滑禮,以功加檢校僕射。及王珂請婚武皇,武皇以女妻之;珂赴禮於太原,以嗣昭權典河中留後事。



 李罕之襲我潞州也,嗣昭率師攻潞州,與汴將丁會戰于含口,俘獲三千,執其將蔡延恭;代李君慶為蕃漢
 馬步行營都將,進攻潞州,遣李存質、李嗣本以兵扼天井關。汴將澤州刺史劉屺棄城而遁,乃以李存璋為刺史。梁祖聞嗣昭之師大至,召葛從周謂曰:「并人若在高平,當圍而取之,先須野戰,勿以潞州為敵。」及聞嗣昭軍韓店,梁祖曰:「進通扼八議路,此賊決與我鬥,公等臨事制機,勿落姦便。」賀德倫閉壁不出,嗣昭日以鐵騎環城,汴人不敢芻牧,援路斷絕。八月,德倫、張歸厚棄城遁去,我復取潞州。



 光化三年,汴人攻滄州,劉仁恭求救,遣嗣
 昭出師邢、己以應之。嗣昭遇汴軍于沙河,擊敗之,獲其將胡禮。進攻洺州,下之,獲其郡將硃紹宗。九月,梁祖自率軍三萬至臨洺,葛從周設伏于青山口。嗣昭聞梁祖至,斂軍而退,從周伏兵發,為其所敗,偏將王郃郎、楊師悅等被擒。十月,汴人大寇鎮、定,王郜告急於武皇,乃遣嗣昭出師,下太行,擊懷、孟。汴將侯信守河陽,不意嗣昭之師至,既無守備,驅市人登城;嗣昭攻其北門,破其外垣,俄而汴將閻寶救軍至,乃退。



 天復元年,河中王珂為
 汴人所擄,河中晉、絳諸郡皆陷。四月,汾州刺史李瑭謀叛,納款于汴;嗣昭討之以禮節情。著作已佚,散見於《世說新語》等書。,三日而拔,斬瑭。是月,汴人初得蒲、絳,乃大舉諸道之師來逼太原。汴將葛從周陷承天軍,氏叔琮營洞渦驛。太原四面,汴軍雲合,武皇憂迫,計無從出。嗣昭朝夕選精騎分出諸門,排擊汴營,左俘右斬,或燔或擊,汴軍疲於奔命;又屬霖雨,軍多足腫腹疾,糧運不繼。五月,氏叔琮引退,嗣昭以精騎追之,汴軍委棄輜重兵仗萬計。六月,嗣昭出師陰地,攻慈、隰,降其刺
 史唐禮、張瑰。是時,天子在鳳翔,汴人攻圍,有密詔徵兵。十一月,嗣昭出師晉、絳,屯吉上堡,遇汴將王友通于平陽,一戰擒之。



 明年正月,嗣昭進兵蒲縣。十八日,汴將朱友寧、氏叔琮將兵十萬來拒。二十八日,梁祖自率大軍至平陽,嗣昭之師大恐。三月十一日,有白虹貫周德威之營,候者云不利,宜班師。翼日,氏叔琮犯德威之營,汴軍十餘萬列陣四合,德威、嗣昭血戰解之,乃保軍而退,汴軍因乘之。時諸將潰散,無復部伍,德威引騎軍循西
 山而遁,朱友寧乘勝陷慈、隰、汾等州。武皇聞其敗也,遣李存信率牙兵至清源應接,復為汴軍所擊。汴軍營于晉祠,嗣昭、德威收合餘眾,登城拒守;汴人治攻具于西北隅,四面營柵相望。時鎮州、河中皆為梁有,孤城無援,師旅敗亡。武皇晝夜登城,憂不遑食,召諸將欲出保雲州,嗣昭曰:「王勿為此謀,兒等茍存,必能城守。」李存信曰:「事勢危急,不如且入北蕃,別圖進取。朱溫兵師百萬,天下無敵,關東、河北受他指揮,今獨守危城,兵亡地蹙,儻
 彼築室反耕,環塹深固,則亡無日矣!」武皇將從之,嗣昭亟爭不可,猶豫未決,賴劉太妃極言於內,武後且止。數日,亡散之眾復集。嗣昭晝夜分兵四出,斬將搴旗,汴軍保守不暇。二十一日,朱友寧燒營退去,嗣昭追擊,復收汾、慈、隰等州。五月,雲州都將王敬暉據城叛,振武石善友亦為部將契苾讓所逐,嗣昭皆討平之。



 天祐三年,汴人攻滄、景,劉仁恭遣使求援。十一月,嗣昭合燕軍三萬進攻潞州,降丁會其子於開禧元年編定,嘉定五年由其學生刊行。共三十六卷。,武皇乃以嗣昭為昭義節度使。始嗣
 昭未到之前,上黨有占者,見一人家舍上常有氣如車蓋,視之,但一貧媼而已。占者謂媼:「有子乎?」曰:「有,見為軍士,出戍于外。」占者心異之,以為其子將來有土地之兆也。未幾,丁會既降,嗣昭領兵入潞,以媼家四面空缺,乃駐于是舍。丁會既歸太原,武皇遣使命嗣昭為帥,乃自媼舍而入理所,其氣尋息,聞者異之。



 四年六月,汴將李思安將兵十萬攻潞州,乃築夾城,深溝高壘,內外重復,飛走路絕。嗣昭撫循士眾,登城拒守,梁祖馳書說誘百
 端,嗣昭焚其偽詔,斬其使者,城中固守經年,軍民乏絕,含鹽炭自生,以濟貧民。嗣昭嘗享諸將,登城張樂,賊矢中足,嗣昭密拔之,坐客不之覺,酣飲如故,以安眾心。五年五月,莊宗敗汴軍,破夾城。嗣昭知武皇棄世,哀慟幾絕。時大兵攻圍歷年,城中士民飢死大半,廛里蕭條。嗣昭緩法寬租,勸農務穡,一二年間,軍城完集,三面鄰于敵境,寇鈔縱橫,設法枝梧,邊鄙不聳。



 胡柳之戰,周德威戰沒,師無行列,至晚方集。汴人四五萬登無石山,我軍
 懼形于色。或請收軍保營,詰旦復戰。嗣昭曰:「賊無營壘,去臨濮地遠,日已晡晚,皆有歸心,但以精騎逗撓,無令返旆,晡後追擊,破之必矣。我若收軍拔寨,賊人入臨濮,俟彼整齊復來,即勝負水決。」莊宗曰:「非兄言,幾敗吾事!」軍校王建及又陳方略,嗣昭與建及分兵于土山南北為掎角,汴軍懼,下山,因縱軍擊之,俘斬三萬級,由是莊宗之軍復振。



 十六年,嗣昭代周德威權幽州軍府事。九月,以李紹宏代,嗣昭出薊門,百姓號泣請留,截鞍惜別,
 嗣昭夜遁而歸。十七年六月,嗣昭自德勝歸籓,莊宗帳餞于戚城。莊宗酒酣,泣而言曰:「河朔生靈,十年饋挽,引領鶴望,俟破汴軍。今兵賦不充,寇孽猶在,坐食軍賦,有愧蒸民。」嗣昭曰:「臣忝急難之地,每一念此,寢不安席。大王且持重謹守,惠養士民。臣歸本籓,簡料兵賦,歲末春首,即舉眾復來。」莊宗離席拜送,如家人禮。是月,汴將劉鄩攻同州,朱友謙告急,嗣昭與李存審援之。九月,破汴軍于馮翊,乃班師。



 十九年,莊守親征張文禮于鎮州。冬,
 契丹三十萬奄至,嗣昭從莊宗擊之,敵騎圍之數十重,良久不解。嗣昭號泣赴之,引三百騎橫擊重圍,馳突出沒者數十合,契丹退,翼莊宗而還。是時,閻寶為鎮人所敗,退保趙州,莊宗命嗣昭代寶攻真定。七月二十四日,王處球之兵出自九門,嗣昭設伏于故營,賊至,伏發,擊之殆盡;餘三人匿于墻墟間,嗣昭環馬而射之,為賊矢中腦,嗣昭箙中矢盡,拔賊矢於腦射賊,一發而殪之。嗣昭日暮還營,所傷血流不止,是夜卒。



 嗣昭節制澤、潞,官
 自司徒、太保至侍中、中書令。莊宗即位,贈太師、隴西郡王。長興中,詔配饗莊宗廟庭。



 嗣昭有子七人,長曰繼儔,澤州刺史;次繼韜、繼忠、繼能、繼襲、繼遠,皆夫人楊氏所生。楊氏治家善積聚,設法販鬻,致家財百萬。



 繼韜,小字留得,少驕獪無賴。嗣昭既卒,莊宗詔諸子扶喪歸太原襄事,諸子違詔,以父牙兵數千擁喪歸潞。莊宗令李存渥馳騎追諭,兄弟俱忿,欲害存渥,存渥遁而獲免。繼韜兄繼儔,嗣昭長嫡也。當襲父爵,然柔而不武。
 方在苫廬,繼韜詐令三軍劫己為留後,囚繼儔于別室,以事奏聞。莊宗不得已,命為安義軍兵馬留後。時軍前糧餉不充,租庸計度請潞州轉米五萬貯于相州;繼韜辭以經費不足,請轉三萬。有幕客魏琢、牙將申蒙者,因入奏公事,每摭陰事報繼韜云:「朝廷無人,終為河南吞噬,止遲速間耳。」由是陰謀叛計。內官張居翰時為昭義監軍,莊宗將即位,詔赴鄴都。潞州節度判官任圜時在鎮州,亦奉詔赴鄴。魏琢、申蒙謂繼韜曰:「國家急召此二
 人,情可知矣。」弟繼遠,年十五六,謂繼韜曰:「兄有家財百萬,倉儲十年,宜自為謀,莫受人所制。」繼韜曰:「定哥以為何如?」曰:「申蒙之言是也。河北不勝河南,不如與大梁通盟,國家方事之殷,焉能討我?無如此算。」乃令繼遠將百餘騎詐云于晉、絳擒生,遂至汴。梁主見之喜,因令董璋將兵應接,營於潞州之南,加繼韜同平章事,改昭義軍為匡義軍。繼韜令其愛子二人入質於汴。



 及莊宗平河南,繼韜惶恐,計無所出,將脫身于契丹;會有詔赦之,乃
 齎銀數十萬兩,隨其母楊氏詣闕,冀以賂免。將行,其弟繼遠曰;「兄往與不往,利害一也。以反為名,何面更見天下!不如深溝峻壁,坐食積粟,尚可茍延歲月,往則亡無日矣。」或曰:「君先世有大功於國,主上季父也,宏農夫人無恙,保獲萬全。」及繼韜至,厚賂宦官、伶人,言事者翕然稱:「留後本無惡意,姦人惑之故也。嗣昭親賢,不可無嗣。」楊夫人亦于宮中哀祈劉皇后,后每于莊宗前泣言先人之功,以動聖情,由是原之。在京月餘,屢從畋遊,寵待
 如故。李存渥深訶詆之,繼韜心不自安,復賂伶閹,求歸本鎮,莊宗不聽。繼韜潛令紀綱書諭繼遠,欲軍城更變,望天子遣己安撫。事泄,斬于天津橋南。二子齠年質于汴,莊宗收城得之,撫其背曰:「爾幼如是,猶如能佐父造反,長復何為!」至是亦誅。乃遣使往潞州斬繼遠,函首赴闕,命繼儔權知軍州事,繼達充軍城巡檢。



 未幾,詔繼儔赴闕。時繼儔以繼韜所畜婢僕玩好之類悉為己有,每日料選算校,不時上路。繼達怒謂人曰:「吾仲兄被罪,父
 子誅死,大兄不仁,略無動懷,而便烝淫妻妾,詰責貨財,慚恥見人,生不如死!」繼達服縗麻,引數百騎坐于戟門,呼曰:「為我反乎!」即令人斬繼儔首,投於戟門之內。副使李繼珂聞其亂也,募市人千餘攻於城門。繼達登城樓,知事不濟,啟子城東門,至其第,盡殺其孥,得百餘騎,出潞城門,將奔契丹。行不十里,麾下奔潰,自剄于路隅。



 天成初,繼能為相州刺史,母楊氏卒於太原,繼能、繼襲奔喪行服。繼能笞掠母主藏婢,責金銀數,因笞至死。家人
 告變,言聚甲為亂,繼能、繼襲皆伏誅。嗣昭諸子自相屠害,幾于溘盡,惟繼忠一人僅保其首領焉。



 裴約,潞州之舊將也。初事李嗣昭為親信,及繼韜之叛,約方戍澤州,因召民泣而諭之曰:「餘事故使,已餘二紀,每見分財享士,志在平讎,不幸薨歿。今郎君父喪未葬,即背君親,餘可倳刃自殺,不能送死與人。」眾皆感泣。繼而梁以董璋為澤州刺史,率眾攻城,約拒久之,告急于莊宗。莊宗知其忠懇,謂諸將曰:「朕于繼韜何薄,于裴約
 何厚?裴約能分逆順,不附賊黨,先兄一何不幸,生此鴟梟!」乃顧李紹斌曰:「爾識機便,為我取裴約來,朕不藉澤州彈丸之地。」即遣紹斌率五千騎以赴之。紹斌自遼州進軍,未至,城已陷,約被害,時同光元年六月也。帝聞之,嗟痛不已。



 李嗣本,鴈門人,本姓張。父準,銅冶鎮將。嗣本少事武皇,為帳中紀綱,漸立戰功,得補軍校。乾寧中,從征李匡儔為前鋒,與燕人戰,得居庸關,以功為義兒軍使,因賜姓
 名。從討王行瑜,授檢校刑部尚書,改威遠、寧塞等軍使。五年,討羅宏信于魏州,嗣本為前鋒,師還,改馬軍都將。從李嗣昭討王暉于雲州,論功加檢校司空。汴將李思安之圍潞州也,從周德威軍于余吾,嗣本率騎軍日與汴人轉鬥,前後獻俘千計,遷代州刺史。六年,從攻晉、絳,為蕃漢副使都校。及武皇喪事有日,嗣本監護其事,改雲中防禦使、雲蔚應朔等州都知兵馬使,加特進、檢校太保。九年,周德威討劉守光,嗣本率代北諸軍、生熟吐渾,
 收山後八軍,得納降軍使盧文進、武州刺史高行珪以獻。幽州平,論功授振武節度使,號「威信可汗」。十二年,莊宗定魏博,劉鄩據莘縣,命嗣本入太原巡守都城,十三年,從破劉鄩於故元城,收洺、磁、衛三郡。六月,還鎮振武。八月,契丹安巴堅傾塞犯邊,其眾三十萬攻振武,嗣本嬰城拒戰者累日。契丹為火車地道,晝夜急攻,城中兵少,禦備罄竭,城陷,嗣本舉族入契丹。有子八人,四人陷于幕庭。嗣本性剛烈,有節義,善戰多謀,然治郡民,頗傷
 苛急,人以此少之也。



 李嗣恩,本姓駱。《歐陽史》:嗣恩本吐谷渾部人。年十五,能騎射,侍武皇于振武;及鎮太原,補鐵林軍小校。從征王行瑜,奉表獻捷,加檢校散騎常侍,漸轉突陣指揮使,賜姓名。天祐四年,逐康懷英于河西,解汾州之圍,加檢校司空,充左廂馬軍都將。戰王景仁有功,加檢校司徒。救河中府,與梁人接戰,應弦斃者甚眾,而槊中其口;及退,莊宗親視其傷,深加慰勉,轉內衙馬步都將、遼州刺史。十二年,從莊
 宗入魏,擊劉鄩有功,轉天雄軍都指揮使。劉鄩之北趣樂平也,嗣恩襲之,倍程先入晉陽。時城中無備,得嗣恩兵至,人百其勇,鄩聞其先過,乃遁。莘之戰,以功轉代州刺史,充石嶺關以北都知兵馬使,稍遷振武節度使。十五年,追赴行在,卒于太原。天成初,明宗敦念舊勳,詔贈太尉。



 有子二人,長曰武八,騎射推于軍中。嘗有時輩臂饑鷹,矜其搏擊,武八持鳴鏑一隻,賭其狩獲,暮乃多之。戰契丹于親州,歿焉。幼曰從郎,累為行軍司馬。



 史臣曰:嗣昭以精悍勤勞,佐經綸之業,終沒王事,得以為忠,然其後嗣皆不免於刑戮者,何也?蓋貨殖無窮,多財累愚故也。抑茍能以清白遺子孫,安有斯禍哉!裴約以偏裨而效忠烈,尤可貴也。嗣本、嗣恩皆以中涓之效,參再造之功,故可附於茲也。



\end{pinyinscope}