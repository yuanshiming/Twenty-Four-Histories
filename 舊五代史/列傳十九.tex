\article{列傳十九}

\begin{pinyinscope}
豆盧革,祖籍,同州刺史。父瓚,舒州刺史。
 \gezhu{
  《宣和書譜》云:失其世系。}
 革少值亂離,避地鄜、延,轉入中山,王處直禮之,闢于幕下,有奏記之譽。因牡丹會賦詩,諷處直以桑柘為意,言甚
 古雅,漸加器仰,轉節度判官。而理家無法,獨請謁處直,處直慮布政有缺,有所規諫,斂版出迎,乃為嬖人祈軍職矣。



 天祐末,莊宗將即位,講求輔相,盧質以名家子舉之,徵拜行臺左丞相。同光初,拜平章事。及登廊廟,事多錯亂,至于官階擬議,前後倒置,屢為省郎蕭希甫駿正,革改之,無難色。莊宗初定汴、洛,革引薦韋說,冀諳事體,與己同功。說既登庸,復事流品,舉止輕脫,怨歸于革。又革、說之子俱授拾遺,父子同官,為人所刺,遂改授員外
 郎。革請說之子濤為宏文館學士,說請革之子升為集賢學士,交致阿私,有同市井,識者醜之。革自作相之後,不以進賢勸能為務,唯事修煉,求長生之術;嘗服丹砂,嘔血數日,垂死而愈。



 天成初,將葬莊宗,以革為山陵使。及木主歸廟,不出私第,專俟旄鎮,數日無耗,為親友促令入朝。安重誨對眾辱之曰:「山陵使名銜尚在,不候新命,便履公朝,意謂邊人可欺也。」側目者聞之,思有所中。初,蕭希甫有正諫之望,革嘗阻之,遂上疏論革與說茍
 且自容,致君無狀。復誣其縱田客殺人,冒元亨上第。遂貶為辰州刺史,仍令所在馳驛發遣。後鄭玨、任圜等連上三章,請不行後命,乃下制曰:「豆盧革、韋說等,身為輔相,手握權衡,或端坐稱臣,或半笑奏事,于君無禮,舉世寧容。革則暫委利權,便私俸祿,文武百辟皆從五月起支,父子二人偏自正初給遣。說則自居重位,全紊大綱。敘蔭貪榮,亂兒孫于昭穆;賣官潤屋,換令錄之身名。醜行疊彰,群情共怒,雖居牧守,示塞非尤。革可責授費州
 司戶參軍,說可夷州司戶參軍,皆員外置同正員,並所在馳驛發遣。」尋貶陵州長流百姓,委長吏常知所在。天成二年夏,詔令逐處刺史監賜自盡,其骨肉並放逐便。



 子昇,官至檢校正郎,服金紫,尋亦削奪。《寶晉齋法書贊》載豆盧革《田園帖》云:大德欲要一居處,畿甸間舊無田園,鄜州雖有三兩處莊子,緣百姓租佃多年,累有令公大王書請,卻給還人戶,蓋不欲侵奪疲民,兼慮無知之輩,妄有影庇包役云云。岳珂曰:此帖乃與僧往還書,其畏強籓避罪罟,蓋慄慄淵冰,然其後卒以故縱田客貶夜郎,正坐所畏,信乎亂邦之不可居也。是時據鄜乃高萬興,官檢校太師、中書令,封北平王,即革所謂「令公大王」者。官故梁授,唐命維新,而顓面正朝者,不能致褫鞶之誅,而反竊貢
 秉旄之佞,唐之不競,有自來矣。



 韋說,福建觀察使岫之子也。案:以下有闕文。莊宗定汴、洛,說與趙光允同制拜平章事。說性謹重,奉職常不造事端。時郭崇韜秉政,說等承順而已,政事得失,無所措言。初,或有言于崇韜,銓選踰濫,選人或取他人出身銜,或取父兄資緒,與令史囊橐罔冒,崇韜乃條奏其事。其後郊天,行事官數千人,多有告敕偽濫,因定去留,塗毀告身者甚眾,選人號哭都門之外。議者亦以為積弊累年,一旦
 澄汰太細,懼失惟新含垢之意。時說與郭崇韜國列,不能執而止之,頗遭物議。說之親黨告之,說曰:「此郭漢子之意也。」及崇韜得罪,說懼流言所鐘,乃令門人左拾遺王松、吏部員外郎李慎儀等上疏,云:「崇韜往日專權,不閑故事,塞仕進之門,非獎善之道。」疏下中書,說等覆奏,深詆崇韜,識者非之。又有王傪者,能以多岐取事,納賂于說,說以其名犯祖諱,遂改之為「操」,擬官于近甸。及明宗即位,說常慮身危,每求庇于任圜,常保護之。說居有
 井,昔與鄰家共之,因嫌鄙雜,築垣于外。鄰人訟之,為希甫疏論,以為井有貨財,及案之本人,惟稱有破釜一所,反招虛妄。初貶敘州刺史,尋責授夷州司戶參軍。



 初,說在江陵,與高季興相知,及入中書,亦常通信幣。自討西蜀,季興請攻峽內,莊宗許之:「如能得三州,俾為屬郡。」西川既定,季興無尺寸之功。洎明宗纘承,季興頻請三郡,朝廷不得已而與之。革、說方在中書,亦預其議。及季興占據,獨歸其罪,流于合州。明年夏,詔曰:「陵州、合州長流
 百姓豆盧革、韋說,頃在先朝,擢居重任,欺公害物,黷貨賣官。靜惟肇亂之端,更有難容之事,且夔、忠、萬三州,地連巴蜀,路扼荊蠻,藉皇都弭難之功,徇逆帥僭求之勢,罔予視聽,率意割移。將千之土疆,開通狡穴;動兩川之兵賦,禦捍經年。致朕莫遂偃戈,猶煩運策。近者西方鄴雖復要害,高季興尚固窠巢,增吾旰食之憂,職爾朋姦之計。而又自居貶所,繼出流言。茍刑戮之稽時,處忠良于何地?宜令逐處刺史監賜自盡。」《歐陽史》:說子濤,晉天福初,為尚書膳
 部員外郎,卒。



 盧程,唐朝右族。祖懿,父蘊,歷仕通顯。程,天復末登進士第,崔魏公領鹽鐵,署為巡官。昭宗遷洛陽,柳璨陷右族,程避地河朔,客遊燕、趙,或衣道士服,干謁籓伯,人未知之。豆盧革客遊中山,依王處直,盧汝弼來太原。程與革、弼皆朝族知舊,因往來依革,處直禮遇未優,故投于太原;汝弼因為延譽,莊宗署為推官,尋改支使。程褊淺無他才,惟務恃門第,口多是非,篤厚君子尤薄之。



 初,判官
 王緘從軍掌文翰,胡柳之役,緘歿于軍。莊宗歸寧太原,置酒公宴,舉酒謂張承業曰:「予今于此會取一書記,先以卮酒辟之。」即舉酒屬巡官馮道,道以所舉非次,抗酒辭避。莊宗曰:「勿謙挹,無踰于卿也。」時以職列序遷,則程當為書記,汝弼亦左右之。程既失職,私懷憤惋,謂人曰:「主上不重人物,使田里兒居餘上。」先是,莊宗嘗于帳中召程草奏,程曰:「叨忝成名,不閑筆硯。」由是文翰之選,不及于程。時張承業專制河東留守事,人皆敬憚。舊例,支
 使監諸廩出納,程訴于承業曰:「此事非僕所長,請擇能者。」承業叱之曰:「公稱文士,即合飛文染翰,以濟霸國,嘗命草辭,自陳短拙,及留職務,又以為辭,公所能者何也?」程垂泣謝之。後歷觀察判官。



 莊宗將即位,求四鎮判官可為宰輔者。時盧汝弼、蘇循相次淪沒,當用判官盧質。質性疏放,不願重位;求留太原,乃舉定州判官豆盧革,次舉程,即詔征之,並命為平章事。程本非重器,驟歷顯位,舉止不恒。時朝廷草創,庶物未備,班列蕭然,寺署多
 缺。程、革受命之日,即乘肩輿,騶導喧沸。莊宗聞訶導之聲,詢于左右,曰:「宰相擔子入門。」莊宗駭異,登樓視之,笑曰:「所謂似是而非者也。」頃之,遣程使晉陽宮冊皇太后。山路險阻,往復綿邈,程安坐肩輿,所至州縣,驅率丁夫,長吏迎謁,拜伏輿前,少有忤意,因加笞辱。



 及汴將王彥章陷德勝南城,爭攻楊劉,莊宗御軍苦戰,臣下憂之,咸白宰臣,欲連章規諫,請不躬御士伍。豆盧革言及漢高臨廣武事,矢及于胸,紿云中足。程曰:「此劉季失策。」眾皆
 縮頸。嘗論近世士族,或曰:「員外郎孔明龜,善和宰相之令緒,宣聖之系孫,得非盛歟!」程曰:「止於孔子之後,盛則吾不知也。」親黨有假驢夫于程者,程帖府給之,府吏訴云無例,程怒鞭吏背。時任圜為興唐少尹,莊宗從姊婿也,憑其寵戚,因詣程。程方衣鶴氅、華陽巾,憑几決事,見圜怒詈曰:「是何蟲豸,恃婦力耶!宰相取給于府縣,得不識舊體!」圜不言而退,是夜,馳至博平,面訴于莊宗。莊宗怒,謂郭崇韜曰:「朕誤相此癡物,敢辱予九卿。」促令自盡。
 崇韜亦怒,事幾不測,賴盧質橫身解之,遂降為右庶子。莊宗既定河南,程隨百官從幸洛陽,沿路墜馬,因病風而卒。贈禮部尚書。



 趙鳳,幽州人也。少為儒。唐天祐中,燕帥劉守光盡率部內丁夫為軍伍,而黥其面,為儒者患之。多為僧以避之,鳳亦落髮至太原。頃之,從劉守奇奔梁,梁用守奇為博州刺史,表鳳為判官。案:下有闕文。為鄆州節度判官。唐莊宗聞鳳名,得之甚喜,以為護鑾學士。後莊宗即位,拜鳳中
 書舍人。及入汴,改授禮部員外郎。莊宗及劉皇后幸張全義第,后奏曰:「妾五六歲失父母,每見老者,思念尊親泣下,以全義年德,妾欲父事之,以慰孤女之心。」莊宗許之,命鳳作箋上全義,定往來儀注。鳳上書極諫,不納。天成初,置端明殿學士,鳳與馮道俱任其職。時任圜為宰相,為安重誨所傾,以至罷相歸磁州。及硃守殷以汴州叛,馳驛賜圜自盡。既而鳳哭謂安重誨曰:「任圜,義士也,肯造逆謀以讎君父乎?如此濫刑,何以安國!」重誨笑而
 不責。是冬,權知貢舉。



 明年春,有僧自西國取經回,得佛牙大如拳,褐漬皴裂,進于明宗。鳳揚言曰:「曾聞佛牙錘鍛不壞,請試之。」隨斧而碎。時宮中所施已踰數千緡,聞毀乃止。及車駕還洛,留知汴州事,尋授中書侍郎、平章事。李之儀《姑溪居士集》:鳳為《莊宗實錄》,將何挺論劉煦疏不載,昫既相,遂引鳳共政事。長興中,安重誨出鎮河中,人無敢言者,惟鳳極言于上前曰:「重誨是陛下家臣,其心終不背主,五年秉權,賢豪俯伏,但不周防,自貽浸潤。」明宗以為朋黨,不悅其奏。重誨獲罪,乃
 出邢州節度使。及閔帝蒙塵于衛州,鳳集賓佐軍校,垂涕曰:「主上播遷,渡河而北,吾輩安坐不赴奔問,于禮可乎?」軍校曰:「唯公所使。」將行,聞閔帝遇弒而止。清泰初,召還,授太保。既而病足,不能朝謁。疾篤,自為蓍筮,卦成,投蓍而嘆曰:「吾家世無五十者,而復窮賤;吾年已五十,又為將相,豈有遐壽哉!」清泰二年三月卒。



 鳳性豁達,輕財重義,凡士友以窮厄告者,必傾其資而餉之,人士以此多之也。



 李愚,字子晦。自稱趙郡平棘西祖之後,家世為儒。父瞻業,應進士不第,遇亂,徙家渤海之無棣,以詩書訓子孫。愚童齔時,謹重有異常兒,年長方志學,遍閱經史。慕晏嬰之為人,初名晏平。為文尚氣格,有韓、柳體。厲志端莊,風神峻整,非禮不言,行不茍且。愚初以艱貧,求為假官,滄州盧彥威署安陵簿。丁憂,服闋,隨計之長安。屬關輔亂離,頻年罷舉,客于蒲、華之間。光化中,軍容劉季述、王奉先廢昭宗,立裕王,五月餘,諸侯無奔問者。愚時在華
 陰,致書于華帥韓建,其略曰:「僕關東一布衣耳,幸讀書為文,每見君臣父子之際,有傷教害義之事,常痛心切齒,恨不得抽腸蹀血,肆之市朝。明公居近關重鎮,君父幽辱月餘,坐視凶逆,而忘勤王之舉,僕所未喻也。僕竊計中朝輔弼,雖有志而無權;外鎮諸侯,雖有權而無志。惟明公忠義,社稷是依。往年車輅播遷,號泣奉迎,累歲供饋,再復朝廟,義感人心,至今歌詠。此時事勢,尤異于前,明公地處要衝,位兼將相,自宮闈變故,已涉旬時,若
 不號令率先,以圖反正,遲疑未決,一朝山東侯伯唱義連衡,鼓行而西,明公求欲自安,如何決策!此必然之勢也。不如馳檄四方,諭以逆順,軍聲一振,則元兇破膽,浹旬之間,二豎之首傳于天下,計無便于此者。」建深禮遇之,堅辭還山。天復初,駕在鳳翔,汴軍攻蒲、華,愚避難東歸洛陽。時衛公李德裕孫道古在平泉舊墅,愚往依焉。子弟親採梠負薪,以給朝夕,未嘗干人。故少師薛廷珪掌貢籍之歲,登進士第;又登宏詞料,授河南府參軍,遂
 下居洛表白沙之別墅。



 梁有禪代之謀,柳璨希旨殺害朝士,愚以衣冠自相殘害,乃避地河朔,與宗人李延光客于山東。梁末帝嗣位,雅好儒士,延光素相款奉,得侍講禁中,屢言愚之行高學贍,有史魚、蘧瑗之風。召見,嗟賞久之,擢為左拾遺。俄充崇政院直學士,或預咨謀,而儼然正色,不畏強禦。衡王入朝,重臣李振輩皆致拜,惟愚長揖。末帝讓之曰:「衡王,朕之兄。朕猶致拜,崇政使李振等皆拜,爾何傲耶!」對曰:「陛下以家人禮兄,振等私臣
 也。臣居朝列,與王無素,安敢諂事。」其剛毅如此。晉州節度使華溫琪在任違法,籍民家財,其家訟于朝,制使劾之,伏罪。梁末帝以先朝草昧之臣,不忍加法,愚堅按其罪。梁末帝詔曰:「朕若不與鞫窮,謂予不念赤子;若或遂行典憲,謂予不念功臣。為爾君者,不亦難乎!其華溫琪所受贓,宜官給代還所訟之家。」貞明中,通事舍人李霄傭夫毆僦舍人致死,法司案律,罪在李霄。愚白:「李霄手不鬥毆。庸夫致死,安得坐其主耶!」以是忤旨。愚自拾遺
 再遷膳部員外郎,賜緋,改司勛員外郎,賜紫,至是罷職,歷許、鄧觀察判官。



 初在內職,慈州舉子張礪依焉。貞明中,礪自河陽北歸莊宗,補授太原府掾,出入崇闥之間,揄揚愚之節概,及言愚之所為文《仲尼遇》、《顏回壽》、《夷齊非餓人》等篇,北人望風稱之。洎莊宗都洛陽,鄧帥俾奏章入朝,諸貴見之,禮接如舊。尋為主客郎中,數月,召為翰林學士。三年,魏王繼岌征蜀,請為都統判官,仍帶本職從軍。時物議以蜀險阻,未可長驅,郭崇韜問計于愚,
 愚曰:「如聞蜀人厭其主荒恣,倉卒必不為用。宜乘其人二三,風馳電擊,彼必破膽,安能守險。」及前軍至固鎮,收軍食十五萬斛,崇韜喜,謂愚曰:「公能料事,吾軍濟矣!」招討判官陳乂至寶雞,稱疾乞留在後。愚厲聲曰:「陳乂見利則進,懼難則止。今大軍涉險,人心易惑,正可斬之以徇。」由是軍人無遲留者。是時,軍書羽檄,皆出其手。蜀平,就拜中書舍人。師還,明宗即位。時西征副招討使任圜為宰相,雅相欽重,屢言于安重誨,請引為同列;屬孔循
 用事,援引崔協以塞其請。俄以本職權知貢舉,改兵部侍郎,充翰林承旨。長興初,除太常卿,屬趙鳳出鎮邢臺,乃拜中書侍郎、平章事,轉集賢殿大學士。



 長興季年,秦王恣橫,權要之臣,避禍不暇,邦之存亡,無敢言者。愚性剛介,往往形言,然人無唱和者。後轉門下侍郎,監修國史,兼吏部尚書,與諸儒修成《創業功臣傳》三十卷。愚初不治第,既命為相,官借延賓館居之。嘗有疾,詔近臣宣諭,延之中堂,設席惟筦秸,使人言之,明宗特賜帷帳茵
 褥。《職官分紀》云:長興四年,愚病,明宗遣中使宣問。愚所居寢室,蕭然四壁,病榻弊氈而已。中使具言其事,帝曰:「宰相月俸幾何?而委頓如此。」詔賜絹百匹、錢百千、帷帳什物一十三事。



 閔帝嗣位,志修德政,易月之制纔除,便延訪學士讀《貞觀政要》、《太宗實錄》,有意于致理。愚私謂同列曰:「吾君延訪,少及吾輩,位高責重,事亦堪憂,奈宗社何!」皆惕息而不敢言。以恩例進位左僕射。清泰初,徽陵禮畢,馮道出鎮同州,愚加特進、太微宮使、宏文館大學士。宰相劉昫與馮道為婚家,道既出鎮,兩人在中書,或舊事不便要釐革者,對論不定。
 愚性太峻,因曰:「此事賢家翁所為,更之不亦便乎!」昫憾其言切,于是每言必相折難,或至喧呼。無幾,兩人俱罷相守本官。清泰二年秋,愚已嬰疾,率多請告,累表乞骸,不允,卒于位。



 任圜,京兆三原人。祖清,成都少尹。父茂宏,避地太原,奏授西河令;有子五人,曰圖、回、圜、團、冏,風彩俱異。武皇愛之,以宗女妻圜,歷代、憲二郡刺史。



 李嗣昭典兵于晉陽,與圜遊處甚洽,及鎮澤潞,請為觀察支使,解褐,賜硃紱。
 圜美姿容,有口辯。嗣昭為人間諜于莊宗,方有微隙,圜奉使往來,常申理之,克成友于之道,圜之力也。及丁母憂,莊宗承制起復潞州觀察判官,賜紫。常山之役,嗣昭為帥,卒于軍,圜代總其事,號令如一,敵人不知。莊宗聞之,倍加獎賞。是秋,復以上黨之師攻常山,城中萬人突出,大將孫文進死之,賊逼我軍,圜麾騎士擊之,頗有殺獲。嘗以禍福諭其城中,鎮人信之,使乞降。及城潰,誅元惡之外,官吏咸保其家屬,亦圜所庇護焉。莊宗改鎮州
 為北京,以圜為工部尚書兼真定尹、北京副留守,行留守事。明年,郭崇韜兼鎮,改行軍司馬,充北面水陸轉運使,仍知府事。同光三年,歸朝,守工部尚書。



 崇韜伐蜀,奏令從征,西蜀平,署圜黔南節度使,懇辭遂止。魏王班師,行及利州,康延孝叛,以勁兵八千回劫西川。繼岌聞之,夜半命中使李廷安召圜,圜方寢,廷安登其床以告之,圜衣不及帶,遽見繼岌。繼岌泣而言曰:「紹琛負恩,非尚書不能制。」即署圜為招討副使,與都指揮使梁漢顒等
 率兵攻延孝于漢州,擒之。旋至渭南,繼岌遇害。圜代總全師,朝于洛陽。明宗嘉其功,拜平章事,判三司。圜揀拔賢俊,杜絕倖門,百官俸入為孔謙減折。圜以廷臣為國家羽儀,故優假班行,禁其虛估,期月之內,府庫充贍,朝廷修葺,軍民咸足。雖憂國如家,而切于功名,故為安重誨所忌。嘗與重誨會于私第,有妓善歌,重誨求之不得,嫌隙自茲而深矣。先是,使人食券,皆出于戶部,重誨止之,俾須內出,爭于御前,往復數四,竟為所沮,《通鑒》:安重誨與圜爭
 于上前,往復數四,聲色俱厲。上退朝,宮人問上:「適與重誨論事為誰?」上曰:「宰相。」宮人曰:「妾在長安宮中,未嘗見宰相、樞密奏事敢如是者,蓋輕大家耳!」上愈不悅。因求罷三司。



 天成二年,除太子少保致仕,出居磁州。及朱守殷叛,重誨乘間誣其結構,立遣人稱制就害之,乃下詔曰:「太子少保致仕任圜,早推勳舊,曾委重難,既退免于劇權,俾優閑于外地,而乃不遵禮分,潛附守殷,緘題罔避于嫌疑,情旨頗彰于怨望。自收汴壘,備見蹤由,若務含宏,是孤典憲,尚全大體,止罪一身。宜令本州于私第賜自盡。」圜受命之日,聚
 族酣飲,神情不撓。清泰中,制贈太傅。



 子徹,仕皇朝,位至度支郎中,卒。



 史臣曰:革、說承舊族之胄,佐新造之邦,業雖謝於財成,罪未聞於昭著,而乃為權臣之所忌,顧後命以無逃,靜而言之,亦可憫也。盧程器狹如是,形渥攸宜。趙鳳、李愚,咸以文學之名,俱踐巖廊之位,校其貞節,愚復優焉。任圜有縱橫濟物之才,無明哲保身之道,退猶不免,籲可
 悲哉!



\end{pinyinscope}