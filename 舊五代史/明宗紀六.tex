\article{明宗紀六}

\begin{pinyinscope}

 天成四年春正月壬申朔,帝御崇元殿受朝賀,仗衛如儀。幽州節度使趙德鈞奏:「臣孫贊,年五歲,默念《論語》、《孝經》,舉童子,于汴州取解就試。」詔曰:「都尉之子,太尉之孫,
 能念儒書,備彰家訓,不勞就試,特與成名。宜賜別敕及第,附今年春榜。」戊子,放元年應欠秋稅。以左衛上將軍安崇阮為黔南節度使。壬辰,回鶻入朝使徹伯爾等五人各授懷化司戈放還。以北京副留守馮贇為宣徽使、判三司。戊戌,禁天下虛稱試攝銜。西川孟知祥奏:「支屬刺史乞臣本道自署。」



 二月乙巳,王晏球奏,此月三日收復定州,獲王都首級,生擒契丹託諾等二千餘人。百僚稱賀已畢,乃詔取今月二十四日車駕還東京。辛亥,以
 北面行營招討使、宋州節度使王晏球為鄆州節度使,加兼侍中;以北面行營副招討使、滄州節度使李從敏為定州節度使;以北面行營兵馬都監、鄭州防禦使張虔釗為滄州節度使;幽州節度使趙德鈞加兼侍中。乙卯,以樞密使趙敬怡權知汴州軍州事。丙辰,邢州奏,定州送到偽太子李繼陶,已處置訖。辛酉,帝御咸安樓受定州俘馘,百官就列,宣露布于樓前,禮畢,以王都首級獻于太社。王都男四人、弟一人,託諾父子二人,並磔于
 市。《五代會要》:尚書兵部宣露布于樓前,宣訖,尚書刑部侍郎張文寶奏曰:「逆賊王都首級請付所司。」大理卿蕭希甫受之以出,獻于郊社,其王都男并蕃將等磔于開封橋。時露布之文,類制敕之體,蓋執筆者誤,頗為識者所嗤。樞密使趙敬怡卒,贈太傅。以端明殿學士趙鳳權知汴州軍州事。甲子,車駕發汴州。丙寅,至鄭州。賜左僕射致仕鄭玨錢二十萬。丁卯,宰相崔協卒,詔贈尚書右僕射。東都留守、太子少傅李琪等奉,至偃師縣奉迎。時琪奏章中有「敗契丹之凶黨,破真定之逆城」之言。詔曰:「契丹即為兇黨,真定不是逆
 城,李琪罰一月俸。」庚午,車駕至自汴州。



 三月甲戌,馮道進表乞命相。丙戌,詔皇城使李從璨貶授房州司戶參軍,仍令盡命。從璨,帝之諸子也。先是,帝巡幸汴州,留從璨以警大內,從璨因遊會節園,酒酣,戲登御榻。安重誨奏之,故置于法焉。壬辰,中書奏:「今後群臣內有乞假覲省者,請量賜茶藥。」從之。乙未,以前鄆州節度使符習為汴州節度使。丙申,詔鄴都、幽、鎮、滄、邢、易、定等州管內百姓,除正稅外,放免諸色差配,以討王都之役,有挽運之
 勞也。



 夏四月庚子朔,禁鐵金錢。壬寅,重脩廣壽殿成,有司請以丹漆金碧飾之,帝曰:「此殿經焚,不可不修,但務宏壯,不勞華侈。」湖南奏,敗荊南賊軍于石首鎮。詔沿邊置場買馬,不許蕃部直至闕下。先是,黨項諸蕃凡將到馬,無駑良並云上進,國家雖約其價以給之,及計其館穀錫賚,所費不可勝紀。計司以為耗蠹中華,遂止之。壬子,以皇子北京留守、河東節度使從榮為河南尹,判六軍諸衛事;以皇子河南尹、判六軍諸衛事從厚為北
 京留守;以河陽節度使趙延壽為宋州節度使;以侍衛親軍都指揮使、鎮南軍節度使康義誠為河陽節度使。契丹寇雲州。癸丑,契丹遣紐赫美稜等復率其屬來朝貢,稱取托諾等骸骨,並斬于北市。甲寅,以端明殿學士趙鳳為門下侍郎兼工部尚書、平章事。丙辰,諫議大夫致仕、襲文宣公孔邈卒。庚申,以王建立、孔循帶中書直省吏歸籓,並追迴。壬戌,幽州節度使趙德鈞兼北面行營招討使,鎮州節度使范延光加檢校太傅。戊辰,中書
 奏:「五月一日,應在京九品已上官,及諸道進奉使,請準貞元七年敕,就位起居,永為恒式。」從之。



 五月己巳朔,帝御文明殿受朝。丙子,以夔州節度使西方鄴卒輟朝。丁丑,大理卿李保殷卒。己卯,以忠武軍節度使索自通為京兆尹,充西京留守;以左威衛上將軍朱漢賓為潞州節度使。乙酉,以黔州節度使安崇阮為夔州節度使,以左驍衛上將軍張溫為洋州節度使,以黔州留後楊漢賓為本州節度使。中書奏:「太常寺定少帝謚昭宣光烈
 孝皇帝,廟號景宗。伏以少帝今不入廟,難以言宗,只云昭宣光烈孝皇帝。」從之。丁亥,以鳳州武興軍留後陳皋為武興軍節度使,以新州威塞軍留後翟璋為威塞軍節度使。壬辰,以權知尚書右丞崔居儉為尚書右丞。詔葺天下廨宇。丙申,襄州奏,荊南高從誨乞歸順。雲州奏,契丹犯塞。



 六月辛丑,以左散騎常侍姚顗為兵部侍郎。壬寅,夔州節度使楊漢章移鎮雲州,以北京馬步軍都指揮使兼欽州刺史張敬達為鳳州節度使。癸卯,以前
 西京副留守事張遵誨行衛尉事,充客省使。國子博士田敏請葺四郊祠祭齋室。丙午,以沂州刺史張萬進為安北都護,充振武軍節度使。戊申,以宿州團練使康思立為利州節度使。登州刺史孫元停任,坐在任無名科率故也。詔鄴都仍舊為魏府。應魏府、汴州、益州宮殿悉去鴟尾,賜節度使為衙署。辛亥,以權知朔方軍留後、定難軍都知兵馬使韓澄為朔方留後。癸丑,以前潞州節度使符彥超為左驍衛上將軍。詔:「諸道節度使行軍司
 馬,名位雖高,或帥臣不在,其州事宜委節度副使權知。」又詔:「籓郡所請賓幕及主事親從者,悉以名聞。」丙辰,權知荊南軍府事高從誨上章首罪,乞修職貢,仍進銀三千兩贖罪。壬戌,幸至德宮。詔:「京城空地,課人蓋造。如無力者,許人請射營構。」



 秋七月庚午,以前西京留守判官張鎛為司農卿。壬申,貶前左金吾上將軍毛璋為儒州長流百姓,尋賜自盡,以其在籓鎮陰蓄奸謀故也。甲戌在此基礎上進而建立「不二論」吠檀多派學說。另有羅摩奴,御史中丞呂夢奇責授太子右贊善大夫,坐曾借毛璋
 馬故也。己卯,以工部侍郎任贊為左散騎常侍,以樞密直學士、左諫議大夫、充匭使閻至為工部侍郎充職。遂州進嘉禾,一莖九穗。壬午,以給事中、判大理卿事許光義為御史中丞。史館上言:「所編修莊宗一朝事跡,欲名為實錄,太祖、獻祖、懿祖名為紀年錄。」從之。《五代會要》:天成三年十二月,史館奏:「據左補闕張昭遠狀:『嘗讀國書,伏見懿祖昭烈皇帝自元和之初,獻祖文皇帝于太和之際,立功王室,陳力國朝。太祖武皇帝自咸通後來,勤王戮力,翦平多難,頻立大功,三換節旄,再安京國。莊宗皇帝終平大憝,奄有中原,倘闕編修,遂成湮墜。伏請與當館修撰,參序條綱,撰太祖、莊宗實錄。』」四年七月,監修國史趙鳳奏:「奉
 敕修懿祖、獻祖、太祖、莊宗四帝實錄,自今年六月一日起手,旋具進呈。伏以凡關纂述,務合品題。承乾御宇之君,行事方雲實錄;追尊冊號之帝,約文只可紀年。所修前件史書,今欲自莊宗一朝名為實錄,其太祖以上並目為紀年錄。」從之。甲申,以前荊南行軍司馬、檢校太傅高從誨起復,授檢校太傅、兼侍中,充荊南節度使。丙戌,涇州節度使李從昶移鎮華州,以冀州刺史李金全為涇州節度使。戊子,中書奏:「今後新及第舉人,有曾授正官及御署者,欲約前任資序,與除一官。」從之。壬辰,詔取來年二月二十一日有事于南郊。



 八月丁酉朔,大理正路阮奏:「
 切見春秋釋奠于文宣王,而武成王廟久曠時祭,請復常祀。」從之。戊戌,中書奏:「太子少傅李琪所撰進《霍彥威神道碑》文學說,不分真偽,是混功名,望令改撰。」從之。琪,梁之故相,私懷感遇,敘彥威在梁歷任,不欲言偽梁故也。辛丑,詔:「亂離已來,天下諸軍所掠生口,有主識認,即勒還之。」以前清河縣令、襲酅國公、食邑三千戶楊仁矩為祕書丞。御史臺奏:「主簿朱穎是前中丞奏請,合隨聽罷任。」詔曰:「主簿既為正秩,況入選門,顯自朝恩,合終考限,宜
 令仍舊守官。」甲辰,以宰臣馮道為南郊大禮使,兵部尚書盧質為禮儀使,御史中丞許光義為儀仗使,兵部侍郎姚顗為鹵簿使,河南尹從榮為橋道頓遞使,客省使、衛尉卿張遵誨為修裝法物使。乙巳,黑水朝貢使郭濟等率屬來朝,授歸德司戈,放還蕃。丁未,以翰林學士承旨、禮部侍郎、知制誥李愚為兵部侍郎,職如故。以中書舍人盧詹為禮部侍郎,以兵部侍郎裴皞為太子賓客。吐渾首領念公山來朝貢。戊申,帝服袞冕,御文明殿,追冊昭宣
 光烈孝皇帝。庚戌,以宰臣、監修國史趙鳳兼判集賢院事,以左散騎常侍任贊判大理卿事。己未,高麗王王建遣使貢方物。辛酉,詔:「準往例,節度使帶平章事、侍中、中書令,並列銜于敕牒,側書『使』字。今錢鏐是元帥、尚父,與使相名殊,馬殷守太師、尚書令,是南省官資,不合署敕尾,今後敕牒內並落下。」乙卯,黨項首領朝貢。甲子,幸金真觀,改賜建法大師,賜紫尼智願為圓惠大師,即武皇夫人陳氏也。丙寅,達靼來朝貢。京城內有南州、北州,乃
 張全義光啟中所築。至是,詔許人依街巷請射城濠,任使平填,蓋造屋宇。



 九月丁卯,中書奏:「據宗正寺申,懿祖永興陵、獻祖長寧陵、太祖建極陵並在代州鴈門縣,皇帝追尊四廟在應州金城縣。」詔:「應州升為望州,金城、鴈門並升為望縣。」辛未,太常博士段顒奏:「切見大祠則差宰相行事,中祠則卿監行事,小祠則委太祝、奉禮,並不差官,今後請差五品官行事。」從之。癸巳,制天下兵馬元帥、尚父、吳越國王錢鏐可落元帥、尚父、吳越國王,授太
 師致仕,責無禮也。先是,上將軍烏昭遇使于兩浙,以朝廷事私于吳人,仍目鏐為殿下,自稱臣,謁鏐行拜蹈之禮。及回,使副劉玫具述其事,故停削鏐官爵,令致仕。烏昭遇下御史臺,尋賜自盡。後有自浙中使還者,言昭遇無臣鏐之事,為玫所誣,人頗以為冤。乙未,詔諸道通勘兩浙綱運進奉使,並下巡獄。



 冬十月丙申朔,併吏部三銓為一銓,宜令本司官員同商量注擬,連署申奏,仍不得于私第注官。戊戌,以襄州兵馬都監、守磁州刺史康
 福為朔方、河西等節度使,靈、威、雄、警、涼等州觀察使。時朔方將吏請帥于朝廷,故命福往鎮之。庚子,以右金吾上將軍史敬熔為左金吾上將軍,以左驍衛上將軍符彥超為右金吾上將軍,以前黔州節度使李承約為右驍衛上將軍,以雲州節度使張敬詢為左驍衛上將軍,以前華州節度使王景戡為右驍衛上將軍。癸卯,太常少卿蕭願責授太子洗馬,奪緋。願南郊行事,與祠官同飲,詰旦猶醉不能行禮,為御史所劾也。詔新授朔方
 節度使康福將兵萬人赴鎮。己酉,制復故荊南節度使高季興官爵。辛亥,升閬州為保寧軍。壬子,以內客省使、左衛大將軍李仁矩為閬州節度使。幸七星亭。丙辰,夏州進白鷹,重誨奏曰:「夏州違詔進貢,臣已止約。」帝曰:「善。」朝退,帝密令左右進焉。是日,幸龍門。



 十一月丁卯,洛州水暴漲,壞居人垣舍。戊辰,以刑部侍郎張文寶為右散騎常侍。己巳,以尚書右丞李光序為刑部侍郎。癸酉,升曹州濟陰縣為次赤,以昭宣光烈孝皇帝溫陵所在故
 也。甲戌,奉國軍節度使王延稟加兼侍中,從福建節度使王延鈞請也。車駕出近郊,試夏州所進白鷹,戒左右勿令重誨知。己卯,日南至,帝御文明殿受朝賀。癸未,祕書少監于嶠配振武長流百姓,永不齒任,為宰臣趙鳳誣奏也。史官張昭遠等以新修獻祖、懿祖、太祖《紀年錄》共二十卷、《莊宗實錄》三十卷上之,賜器帛有差。《五代會要》:監修趙鳳、修撰張昭遠、呂咸休各賜繒彩、銀器等。



 十二月丁酉,靈武康福奏:「破野利、大蟲兩族三百餘帳于方渠,獲牛羊三萬。」戊戌,詔:「應
 授官及封贈官誥、舉人冬集等所費用物,一切官破。」壬戌,中書奏:「今後宰臣致齋內,不押班,不知印,不赴內殿起居。或遇國忌,行事官已受誓戒,宜不赴行香,並不奏刑殺公事。大祠致齋內,請不開宴。每遇大忌前一日,請不坐朝。」從之。



\end{pinyinscope}