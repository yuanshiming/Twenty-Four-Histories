\article{太祖紀二}

\begin{pinyinscope}

 光化元年正月,帝遣葛從周統諸將略地于山東,遂次于邢、洺。三月,昭宗以帝兼領天平軍節度使,餘如故。四月,滄州節度使盧廷彥為燕軍所攻,棄城奔于魏,魏人
 送於汴。是月,帝以大軍至鉅鹿,屯于城下,敗晉軍萬餘眾于青山口,俘馬千餘匹。丁卯,遣從周分兵攻洺州,斬刺史邢善益,擒將五十餘人。五月己巳,邢州刺史馬師素棄城遁去。辛未,磁州刺史袁奉滔自剄而死。五日之內,連下三州。因以葛從周兼邢州昭義軍節度使留後,帝遂班師。是時,襄州節度使趙匡凝聞帝軍有清口之敗,密附于淮夷。七月,帝遣氏叔琮率師伐之。未幾,泌州刺史趙璠越墉來降;隨州刺史趙匡琳臨陣就擒。



 二
 年正月,淮南楊行密舉全吳之眾,精甲五萬,以伐徐州,帝領大軍禦之。行密聞帝親征,乃收軍而退。時幽州節度使劉仁恭大舉蕃漢兵號十萬以伐魏,遂攻陷貝州,州民萬餘戶,無少長悉屠之。進攻魏州,魏人來乞師,帝遣硃友倫、張存敬、李思安等先屯于內黃,帝遂親征。三月,與燕軍戰于內黃北,燕軍大敗,殺二萬餘眾,奪馬二千餘匹,擒都將單無敵已下七十餘人。《通鑑》:單可及,幽州驍將,號單無敵。是月,葛從周自山東領其部眾,馳以救魏。翼日,乘
 勝,諸將張存敬以下連破八寨,遂逐燕軍,北至于臨清,壅其殘寇于御河,溺死者甚眾。仁恭奔于滄州。六月,帝表丁會為潞州節度使,以李罕之疾亟故也。又遣葛從周由固鎮路入于潞州,以援丁會。七月壬辰朔,海州陳漢賓擁所部三千奔于淮南。戊戌,晉人陷澤州。帝遣召葛從周于潞,留賀德倫以守之。未幾,德倫為晉人所逼,遂棄潞而歸,由是潞州復為晉人所有。十一月,陜州都將硃簡殺留後李璠,自稱留後,送款于帝。



 三年四月,遣
 葛從周以兗、鄆、滑、魏之師伐滄州。五月庚寅,攻德州,拔之,梟刺史傅公和于城上。己亥,進攻浮陽。六月,燕帥劉仁恭大舉來援,從周與諸將逆戰于乾寧軍老鴉堤,大破之,殺萬餘眾,俘其將佐馬慎交已下百餘人。既而以連雨,遂班師。八月,河東遣李進通襲陷洺州,執刺史硃紹宗。帝遣葛從周自鄴縣渡漳水,屯于黃龍鎮,親領中軍涉洺而寨;晉人懼而宵遁,洺州復平。九月,帝以仁恭、進通之入寇也,皆由鎮、定為其囊橐,即以葛從周為上
 將以伐鎮州,遂攻下臨城,渡滹沱以環其城。帝親領軍繼至,鎮帥王熔俱,納質請盟,仍獻文繒二十萬以犒戎士,帝許之。十月,晉人以帝宿兵于趙,遂南下太行,急攻河陽,留後侯言與都將閻寶力戰固守,僅而獲全。十一月,以張存敬為上將,自甘陵發軍,北侵幽、薊,連拔瀛、莫二郡,遂移軍以攻中山。定帥王郜以精甲二萬戰于懷德亭,盡殪之。郜懼,奔于太原。遲明,大軍集于城下,郜季父處直持印鑰乞降,亦以繒帛三十萬為獻,帝即以處直
 代郜領其鎮焉。是月,燕人劉守光赴援中山,寨于易水之上,繼為康懷英、張存敬等所敗,斬獲甚眾。由是河朔知懼,皆弭伏焉。



 是歲,唐左軍中尉劉季述幽昭宗于東宮內,立皇子德王裕為帝,仍遣其養子希度來言,願以唐之神器輸於帝。帝時方在河朔,聞之,遽還于汴,大計未決。會李振自長安使回,因言于帝曰:「夫豎刁、伊戾之亂,所以資霸者之事也。今閹豎幽辱天子,王不能討,無以令諸侯。」帝悟,因請振復使于長安,與時宰潛謀反正。



 天復元年正月乙酉朔,唐宰相崔允潛使人以帝密旨告于侍衛軍將孫德昭已下,令誅左右中尉劉季述、王仲先等,即時迎昭宗于東內,御樓反正。癸巳,降制進封帝為梁王,酬反正之功也。昭宗之廢也,汴之邸吏程巖牽昭宗衣下殿。帝聞之,召巖至汴,折其足,送于長安,杖殺之。是時,河中節度使王珂結援于太原,帝怒,遣大將張存敬率將涉河,由含山路鼓行而進。戊申,攻下絳州。壬子,晉州刺史張漢瑜舉郡來降,帝即以大將侯言權
 領晉州,何絪權領絳州,晉、絳平。己未,大軍至河中,存敬命繚其垣而攻之。壬戌,蒲人颺素幡以請降。庚午,帝至河中,以張存敬權領河中軍府事,河中平,帝乃東還。是月,李克用遣牙將張特來聘,請尋舊好,帝亦遣使報命。三月癸未朔,帝歸自河中。是月,遣大將賀德倫、氏叔琮領大軍以伐太原,叔琮等自太行路入,魏博都將張文恭自磁州新口入,葛從周以兗、鄆之眾自土門路入,洺州刺史張歸厚以本軍自馬嶺入,定州刺史王處直以
 本軍自飛狐入,晉州侯言自陰地入。澤州刺史李存璋棄郡奔歸太原。叔琮引軍逼潞州,節度使孟遷乞降。河東屯將李審建、王周領步軍一萬、騎二千詣叔琮歸命,乃進軍趨太原。四月乙卯,大軍出石會關,營于洞渦驛。都將白奉國自井陘入,收承天軍。張歸厚引兵至遼州,刺史張鄂迎降。氏叔琮即日與諸軍至晉陽城下,城中雖時出精騎來戰,然危蹙已甚,將謀遁矣。會叔琮以芻糧不給,遂班師。五月癸卯,昭宗以帝兼領護國軍節度使、
 河中尹。六月庚申,帝發自大梁。丁卯,視事于河中,以素服出郊,拜故節度使王重榮墓。尋辟其子瓚為節度判官,請故相張浚為重榮撰碑。帝自中和初歸唐,首依重榮,至是思其舊德,故恩禮若是。七月甲寅,帝東還梁邸。


十月戊戌,奉密詔赴長安。是時,朝廷既誅劉季述,以韓全誨、張宏彥為兩軍中尉,袁易簡、周敬容為樞密使。是時軍國大政
 \gezhu{
  今湖南道縣}
 人。曾官大理寺丞、知洪州南昌、國子博士等。,專委宰相崔允,每事裁抑宦官,宦官側目。允一日于便殿,奏欲盡去之,全誨等屬垣聞之,嘗于昭宗前祈
 哀自訴。自是昭宗敕允,每有密奏,令進囊封。全誨等乃訪京城美婦人十數以進,使求宮中陰事,昭宗不悟,允謀漸泄。中官視允皆裂,以重賂甘言誘籓臣以為城社,時因燕聚,則相向流涕。時允掌三司貨泉,全誨等教禁兵伺允出,聚而呼噪,訴以冬衣減損,又于昭宗前訴之;昭宗不得已,罷允知政事。允怒,急召帝請以兵入輔,故有是行。戊申,行次河中。同州留後司馬鄴,華之幕吏也,舉郡來降。辛亥,駐軍于渭濱,華帥韓建遣使奉箋納
 款,又以銀三萬兩助軍。是日,行次零口。癸丑,聞長安亂,昭宗為閹官韓全誨等劫遷,西幸鳳翔,蓋避帝之兵鋒也。翼日,遂命旋師,夕次于赤水。乙卯,大軍集于華州城下,韓建惶駭失措,即以城降。丙辰,帝表建權知忠武軍事,促令赴任。同、華二州平。是時,唐太子太師盧知猷等二百六十三人列狀請帝速請迎奉。己未,遂帥諸軍發自赤水。壬戌,次于咸陽。偵者云:「天子昨暮至岐山,旦日宋文通扈蹕入其闉矣。」是時,岐人遣大將符道昭領兵
 萬人屯于武功以拒帝,帝遣康懷英敗之,擄甲士六千餘眾。乙丑,次于岐山,文通遣使奉書自陳其失,請帝入覲。丙辰,及岐闉,文通渝約,閉壁不獲通,復次于岐山。是時,昭宗累遣使齎朱書御札賜帝,遣帝收軍還本道。帝診之曰:「此必文通、全誨之謀也。」皆不奉詔。癸酉,飛章奉辭,且移軍北伐。乙亥,至邠州,節度使李繼徽舉城降。繼徽因請去文通所賜李姓,復本宗楊氏,又請納其帑以為質,帝皆從之,仍易其名曰崇本。邠州平。己丑,唐丞相
 崔允、京兆尹鄭元規至華州,以速迎奉為請,許之。


二年正月,帝復次于武功,岐人堅壁不下,乃回軍于河中。二月,聞晉軍大舉南下梁丘賀西漢易學家。瑯邪諸
 \gezhu{
  今山東諸城縣}
 人。字長,聲言來援鳳翔,帝遣硃友寧帥師會晉州刺史氏叔琮以禦之,帝以大軍繼其後。三月,友寧、叔琮與晉軍戰于晉州之北,大敗之,生擒克用男廷鸞。帝喜,謂左右曰:「此岐人之所恃也,今既如此,岐之變不久矣。」四月,岐人遣符道昭領大軍屯于虢縣,康懷英帥驍騎敗之。丁酉,唐丞相崔允自華來謁帝,屢述艱運
 危急,事不可緩;又慮群閹擁昭宗幸蜀,且告帝,帝為動容。允將辭,啟宴于府署,帝舉酒,允情激于哀,因自持樂板,聲曲以侑酒。帝甚悅,座中以良馬珍玩之物賚,既行,命諸將繕戎具。



 五月丁巳,帝復西征。六月丁丑,次于虢縣。癸未,與岐軍大戰,自辰至午,殺萬餘眾,擒其將校數百人,乘勝遂逼其壘。七月丙午,岐軍復出求戰,帝軍不利。是月,遣孔勍帥師取鳳、隴、成三州,皆下之。是時,岐人相率結寨于諸山,以避帝軍;帝分兵以討,浹旬之內,并
 平之。九月甲戌,帝以岐軍諸寨連結稍盛,因親統千騎登高診之。時秋空澄霽,煙靄四絕,忽有紫雲如傘蓋,凝于龍旌之上,久之方散,觀者咸訝之。是時,帝以岐人堅壁不戰,且慮師老,思欲旋旆以歸河中,因密召上將數人語其事。時親從指揮使高季昌獨前出抗言曰:「天下雄傑,窺此舉者一歲矣;今岐人已困,願少俟之。」帝嘉其言,因曰:「兵法貴以正理,以奇勝者詐也,乘機集事,必由是乎!」乃命季昌密募人入岐以紿之,尋有騎士馬景堅
 願應命,且曰:「是行也,必無生理,願錄其孥。」帝悽然止其行,景固請,乃許之。明日軍出,《北夢瑣言》云:時因硃友倫總騎軍且至,將大出兵迓之。諸寨屏匿如無人,景因躍馬西走,直叩岐闉,詐以軍怨東遁為告,且言列寨尚留萬餘人,俟夕將遁矣,宜速掩之。李茂貞信其言,案:李茂貞即宋文通,此紀前後互異,蓋仍當時軍檄之文,未及改從畫一。遽啟二扉,悉眾來寇。時諸軍以介馬待之,中軍一鼓,百營俱進,又分遣數騎以據其闉。岐人進不能駐其趾,退不能入其壘,殺戮蹂踐,不知其數。茂貞由是喪膽,但
 閉壁而已。十一月癸卯,鄜帥李周彞《新唐書》作「李茂勛」,茂勛即周彞也。統兵萬餘人屯于岐之北原,與城中舉烽以相應。翼日,帝以周彞既離本部,鄜畤必無守備,因命孔勍乘虛襲下之。甲寅,鄜州平。周彞聞之,收軍而遁。茂貞既失鄜州之援,愕然有瓦解之懼,由是議還警蹕,誅閹寺以自贖焉。



 三年正月甲寅,岐人啟壁,唐昭宗降使宣問慰勞,兼傳密旨。尋又命翰林學士韓渥、趙國夫人寵顏齎詔押賜帝紫金酒器、御衣玉帶。丙辰,華州留後李存審遣飛
 騎來告,青州節度使王師範遣牙將張厚輦甲胄弓槊,詐言來獻,欲盜據州城,事覺,已擒之矣。是日,師範又遣其將劉鄩盜據兗州。丁巳,昭宗遣中使押送軍容使韓全誨已下三千餘人首級以示帝。甲子,昭宗發離鳳翔,幸左劍寨,權駐蹕帝營。帝素服待罪,昭宗命學士傳宣免之,帝即入見稱罪,拜伏者數四。既而促召升殿,密邇御座,且曰:「宗廟社稷是卿再造,朕與戚屬是卿再生。」因解所御玉帶面以賜帝;帝亦以玉鞍勒馬、金銀器、紋錦、
 御饌酒果等躬自拜進焉。及翠華東行,帝匹馬前導十餘里,宣令止之。己巳,昭宗至長安,謁太廟,御長樂樓。禮畢,謂帝曰:「朕生入舊京,是卿之力也。自古救君之危,曾無有如是者。況今日再及清廟,得親奉觴酒,奠于先皇帝室前,卿之德,朕知不能報矣!」即召帝執手,聲淚俱發者久之。翼日,誅宦官第五可範等五百餘人于內侍省。三月庚辰,制以帝為守太尉、兼中書令、宣武宣義天平護國等軍節度使、諸道兵馬副元師,加食邑三千戶,實
 封四百戶,仍賜回天再造竭忠守正功臣。



 戊戌,帝建旆東還,昭宗御延喜樓送之,既醉,遣內臣賜帝御制《楊柳詞》五首。三月戊午萬物從「太一」流溢而來,首先流溢出宇宙理性;然後從理,至大梁。時以青州未平,命軍士休浣以俟東征。四月丙子,巡師于臨朐,亟命逼其城,與青州兵戰于城下,大敗之。是夕,淮將王景仁以所部援軍宵遁,帝遣楊師厚追及輔唐,殺千人,乘勝攻下密州。八月戊辰,以伐叛之柄委于楊師厚,帝乃東還。九月癸卯,師厚率大軍與王師範戰于臨朐,青軍大敗,殺萬餘人,並
 擒師範弟師克,即時徙寨以逼其城。辛亥,偏將劉重霸擒棣州刺史邵播來獻。播,師範之謀主也,帝命斃之。戊午,師範舉城請降。青州平。翼日,分命將校略地于登、萊、淄、棣等州,皆下之。由是東漸至海,皆為梁土也。帝復命師範權知青州軍州事,師範乃請以錢二十萬貫犒軍,帝許之。十月辛巳,護駕都指揮使朱友倫因擊鞠墮馬,卒于長安。訃至,帝大怒,以為唐室大臣欲謀叛己,致友倫暴死。十一月丁酉,青將劉鄩舉兗州來降。鄩,王師範
 之將也,師範令竊據兗州久之,及聞師範降,鄩乃歸命。帝以鄩善事其主,待之甚優,尋署為元帥府都押牙,權知鄜州留後。



 天祐元年正月己酉,帝發自大梁,西赴河中,京師聞之,為之震懼。是時,將議迎駕東幸洛陽,慮唐室大臣異議,帝乃密令護駕都指揮使朱友諒矯昭宗命,收宰相崔允、京兆尹鄭元規等殺之。又,邠、岐兵士侵逼京畿,帝因是上表,堅請昭宗幸洛,昭宗不得已而從之。帝乃率諸道丁匠財力,同構洛陽官,不數月而成。二
 月乙亥,昭宗駐蹕于陜,帝自河中來覲,謁見行營。因灑涕而言曰:「李茂貞等竊謀禍亂,將迫乘輿,老臣無狀,請陛下東遷,為社稷大計也。」昭宗命延于寢室見何皇后,面賜酒器及衣物。何后謂帝曰:「此後大家夫婦委身于全忠矣。」因欷歔泣下。後數日,帝開宴于陜之私第,請駕臨幸。翼日,帝辭歸洛陽,昭宗開內宴,時有宮人與昭宗附耳而語。韓建躡帝之足,帝遽出,以為圖己,因連上章請車駕幸洛。《十國春秋》、《吳世家》,三月丁巳,唐帝遣間使以絹詔告難于我及西川、河東等,令糾率籓鎮,
 以圖匡復。詔有云:「朕至洛陽,則為全忠所幽閉,詔敕皆出其手,朕意不得復通矣。」



 三月丁未,昭宗制以帝兼判左右神策及六軍諸衛事。是時,昭宗累遣中使及內夫人傳宣,謂帝曰:「皇后方在草蓐,未任就路,欲以十月幸洛。」帝以陜州小籓,非萬乘久留之地,期以四月內東幸。閏月丁酉,昭宗發自陜郡。壬寅,次于穀水。是時,昭宗左右惟小黃門及打球供奉、內園小兒二百餘人,帝猶忌之。是日,密令醫官許昭遠告變,乃設饌于別幄,召而盡殺之,皆坑于幕下。先是,選二百餘人,形
 貌大小一如內園人物之狀,至是使一人擒二人,縊于坑所,即蒙其衣及戎具自飾。昭宗初不能辨,久而方察。自是,昭宗左右前後,皆梁人矣。甲辰,車駕至洛都,帝與宰相百官導駕入宮。乙卯,昭宗以帝為宣武、宣義、護國、忠武四鎮節度使。時帝請以鄆州授張全義,故有此命。五月丙寅,昭宗宴群臣,曰:「昨來御樓前一夜亡失赦書,賴梁王收得副本,不然誤事,宰執不得無過矣。」是日宴次,昭宗入內,召帝于內殿曲宴,帝不測其事,不敢奉詔。又曰:「
 卿不欲來,即令敬翔人來。」帝密遣翔出,乃止。己巳,奉辭東歸。乙亥,至大梁。六月,帝遣都將硃友裕率師討邠州,節度使楊崇本叛故也。癸丑,帝西征,遂朝于洛陽。七月甲子,昭宗宴帝于文思鞠場。乙丑,帝發東都。壬申,至河中。八月壬寅,昭宗遇弒于大內,遺制以輝王柷為嗣。乙巳,帝自河中引軍而西。癸丑,次于永壽,邠軍不出。九月辛未,班師。十月癸巳,至洛陽,詣西內,臨于梓宮前,祗見于嗣君。辛丑,制以案此下有闕文。帝至自西征。十一月辛酉,光
 州遣使來求援。時光州歸款于帝,尋為淮人所攻,故來乞師。戊寅,帝南征渡淮,次于霍丘,大掠盧、壽之境,淮人乃棄光州而去。



 二年正月庚申,進攻壽州,壽人堅壁不出。丁亥,帝自霍丘班師。二月辛卯,帝至自南征。甲午,青州節度使王師範至大梁,帝待以賓禮,尋表授河陽節度使。七月辛酉,天子賜帝迎鸞紀功碑,樹于洛陽。庚午,遣大將軍楊師厚率前軍討趙匡凝于襄州。辛未,帝南征,表趙匡凝罪狀,削奪官爵。八月,楊師厚進收唐、鄧、復、
 郢、隨、均、房等七州。帝駐軍漢江北,自循江干,經度濟師之所。九月甲子,師厚于陰谷江口造梁以濟師,趙匡凝率兵二萬振于江濱。師厚麾兵進擊,襄人大敗,殺萬餘眾。乙丑,越匡凝焚其舟,率親軍載輕舸沿漢而遁。丙寅,帝濟江,至中流,舟壞,將沒者數四,比及岸,舟沉。是日,入襄城,帝因周視府署,其帑藏悉空。惟于西廡下有一亭,窗戶儼然,扃鎖甚密,遂令破鎖啟扉,中有一大匱,緘鐍甚至;又令破其匱,內有金銀數百錠。帝因歎曰:「亂兵既
 入,公私財貨固無孑遺矣。此帑當有陰物主之,不令常人所得,俟我以有之邪!」遂以百餘錠賜楊師厚。襲荊州,留後趙匡明棄城上峽奔蜀。荊、襄二州平。帝以都將賀瑰權領荊州,楊師厚權領襄州,即表其事。



 十月丙戌朔,天子以帝為諸道兵馬元帥。辛卯,帝自襄州引軍由光州路趨淮南;將發,敬翔切諫,請班師以全軍勢傳》。,帝不聽。壬辰,次于棗陽,遇大雨,頗阻師行之勢。軍至壽春,壽春人堅壁清野以待帝。帝乃還,舍于正陽。



 十一月丙辰,大
 軍北濟。《十國春秋》:柴再用抄其後軍,斬首三千級,獲輜重萬計。帝至汝陰,深悔淮南之行,躁煩尤甚。《師友雜志》:硃全忠嘗與僚佐及游客坐于大柳之下,全忠獨言曰:「此樹宜為車轂。」眾莫應。有遊客數人起應曰:「宜為車轂。」全忠勃然厲聲曰:「書生輩好順口玩人,皆此類也。車須用夾轂,柳木豈可為之!」顧左右曰:「尚何待!」左右數十人捽言為車轂者,悉撲殺之。丁卯,帝至自南征。辛巳,天子命帝為相國,總百揆。以宣武、宣義、天平、護國、天雄、武順、祐國、河陽、義武、昭義、保義、武昭、武定、泰寧、平盧、匡國、武寧、忠義、荊南等二十一道為魏國。案《舊唐書》,尚有忠武、鎮國二道,此闕載。進封帝為魏王,入朝不趨,劍履上殿,贊拜不名,
 兼備九錫之命。癸未,唐中書門下奏:「中書印已送相國,中書公事權用中書省印。」甲申,中書門下奏:「天下州縣名與相國魏王家諱同者,請易之。」十二月乙酉朔,帝讓相國、魏王、九錫之命。丙戌,京百司各差官齎本司須知孔目並印赴魏國送納。甲午,天子以帝堅讓九錫之命。乃命宰相柳燦來使,且述揖讓之意焉。丁酉,帝又讓九錫之命,詔略曰:「但以鴻名難掩,懿實須彰,宜且徇于奏陳,未便行于典冊。」又改諸道兵馬元帥為天下兵馬元
 帥。是時,帝以唐朝百官服飾多闕,乃製造逐色衣服,請朝廷等第賜之。其所給俸錢,仍請自來年正月全支。



 三年正月,幽、滄稱兵,將寇于魏。魏人來乞師,且以牙軍驕悍,謀欲誅之,遣親吏臧延範密告于帝,帝陰許之。乙丑,北征。先是,帝之愛女適羅氏,是月卒于鄴城,因以兵仗數千事實于橐中,遣客將馬嗣勳領長直軍千人,雜以工匠、丁夫,肩其橐而入于魏,聲言為帝女設祭,魏人信而不疑。庚午夜,嗣勳率其眾與羅紹威親軍數百人同
 攻牙軍,遲明盡殺之,死者七千餘人,泊于嬰孺,亦無留者。是日,帝次于內黃,聞之,馳騎至魏。時魏之大軍方與帝軍同伐滄州,聞牙軍之死,即時奔還。帝之軍追及歷亭,殺賊幾千,餘眾乃擁大將史仁遇保于高唐,帝遣兵圍之。是月,天子詔河南尹張全義部署修制相國魏王法物。



 三月甲寅,天子命帝總判鹽鐵、度支、戶部等三司事,帝再上章切讓之,乃止。四月癸未,攻下高唐,軍民無少長皆殺之,生擒逆首史仁遇以獻,帝命支解之。未幾,
 又攻下澶、博、貝、衛等州,皆為魏軍殘黨所據故也。是時晉人圍邢州,刺史牛存節堅壁固守,帝遣符道昭帥師救之,晉人乃遁去。五月,帝略地于洺州,既而復入于魏。七月己未,自魏班師。是日,收復相州,自是魏境悉平。壬申,帝歸自魏。



 八月甲辰,以滄州未平,復命北征。九月丁卯,營于長蘆。一夕,帝夢白龍附于兩肩,左右瞻顧可畏,心兄然驚寤。十月辛巳,邠州楊崇本以鳳翔、邠、寧、涇、鄜、秦、隴之眾合五六萬來寇,屯于美原,列十五寨,其勢甚盛。帝
 命同州節度使劉知俊、都將康懷英帥師禦之。知俊等大破邠寇,殺二萬餘眾,奪馬三千餘匹,擒其列校百餘人,楊崇本、胡章僅以身免。十一月庚戌,懷英乘勝進軍,遂收鄜州。十二月乙丑,帝以文武常參官每月一、五、九日赴朝,奏請備廊餐,詔從之。遂自長蘆班師。案:以上疑有闕文。據《舊唐書·哀帝紀》:戊辰,李克用與幽州之眾同攻潞州,全忠守將丁會以澤、潞降太原,克用以其子嗣昭為留後。甲戌,全忠燒長蘆營旋軍,聞潞州陷故也。以寨內糗糧山積,帝命焚之。滄帥劉守文以城中絕食,因致書于帝,乞留餘糧以救饑民,
 帝為留十餘囷以與之。《容齋續筆》:滄州還師,悉焚諸營資糧,在舟中者鑿而沉之。守文遺全忠書曰:「城中數萬口,不食數月矣,與其焚之為煙,沉之為泥,願乞其所餘以救之。」全忠為之留數囷,滄人賴以濟。



\end{pinyinscope}