\article{列傳九}

\begin{pinyinscope}

 郭崇韜,字安時,代州雁門人也。父宏正。崇韜初為李克修帳下親信。克修鎮昭義,崇韜累典事務,以廉乾稱。克修卒,武皇用為典謁,奉使鳳翔稱旨,署教練使。崇韜臨
 事機警,應對可觀。莊宗嗣位,尤器重之。天祐十四年,用為中門副使,與孟知祥、李紹宏俱參機要。俄而紹宏出典幽州留事,知祥懇辭要職。先是,中門使吳珙、張虔厚忠而獲罪。知祥懼,求為外任,妻璚華公主泣請於貞簡太后。莊宗謂知祥曰:「公欲避路,當舉其代。」知祥因舉崇韜。乃署知祥為太原軍在城都虞候。自是崇韜專典機務,艱難戰伐,靡所不從。



 十八年,從征張文禮於鎮州。契丹引眾至新樂,王師大恐,諸將咸請退還魏州,莊宗猶
 豫未決。崇韜曰:「安巴堅只為王都所誘,本利貨財,非敦鄰好,茍前鋒小衄,遁走必矣。況我新破汴寇,威振北地,乘此驅攘,焉往不捷!且事之濟否,亦有天命。」莊宗從之,王師果捷。明年,李存審收鎮州,遣崇韜閱其府庫,或以珍貨賂遺,一無所取,但市書籍而已。



 莊宗即位於魏州,崇韜加檢校太保、守兵部尚書,充樞密使。是時,衛州陷於梁,澶、相之間,寇鈔日至,民流地削,軍儲不給,群情恟恟,以為霸業終不能就,崇韜寢不安席。俄而王彥章陷
 德勝南城,敵勢滋蔓,汴人急攻楊劉城。明宗在鄆,音驛斷絕。莊宗登城四望,計無所出。崇韜啟曰:「段凝阻絕津路,茍王師不南,鄆州安能保守!臣請於博州東岸立柵,以固通津,但慮汴人偵知,徑來薄我,請陛下募敢死之士,日以挑戰,如三四日間。賊軍未至,則柵壘成矣。」崇韜率毛璋等萬人夜趨博州,視矛戟之端有光,崇韜曰:「吾聞火出兵刃,破賊之兆也。」至博州,渡河版築,晝夜不息。崇韜於葭葦間據胡床假寢,覺褲中冷,左右視之,乃蛇
 也,其忘疲勵力也如是。居三日,梁軍果至,城壘低庳,沙土散惡,戰具不完,汴將王彥章、杜晏球率眾攻擊,軍不得休息。崇韜身先督眾,四面拒戰,有急即應,城垂陷,俄報莊宗領親軍次西岸,梁軍聞之退走,因解楊劉之圍。



 未幾,汴將康延孝來奔,崇韜延於臥內,訊其軍機。延孝曰:「汴人將四道齊舉,以困我軍。」莊宗憂之,召諸將謀進取之策。宣徽使李紹宏請棄鄆州,與汴人盟,以河為界,無相侵寇。莊宗不悅,獨臥帳中,召崇韜謂曰:「計將安出?」
 對曰:「臣不知書,不能徵比前古,請以時事言之。自陛下十五年起義圖霸,為雪家讎國恥,甲胄生蟣虱,黎人困輸挽。今纂崇大號,河朔士庶,日望蕩平,才得汶陽尺寸之地,不敢保守,況盡有中原乎!將來歲賦不充,物議咨怨,設若劃河為界,誰為陛下守之?臣自延孝言事以來,晝夜籌度,料我兵力,算賊事機,不出今年,雌雄必決。聞汴人決河,自滑至鄆,非舟楫不能濟。又聞精兵盡在段凝麾下,王彥章日寇鄆境,彼既以大軍臨我南鄙,又憑
 恃決河,謂我不能南渡,志在收復汶陽,此汴人之謀也。臣謂段凝保據河需,茍欲持我,臣但請留兵守鄴,保固楊劉;陛下親御六軍,長驅倍道,直指大梁,汴城無兵,望風自潰。若使偽主授首,賊將自然倒戈,半月之間,天下必定。如不決此計,傍採浮譚,臣恐不能濟也。今歲秋稼不登,軍糧纔支數月,決則成敗未知,不決則坐見不濟。臣聞作舍道邊,三年不成,帝王應運,必有天命,成敗天也,在陛下獨斷。」莊宗蹶然而興曰:「正合吾意。丈夫得則
 為王,失則為擄,行計決矣!」即日下令軍中,家口並還魏州。莊宗送劉皇后與興聖宮使繼岌至朝城西野亭泣別,曰:「事勢危蹙,今須一決,事茍不濟,無復相見。」乃留李紹宏及租庸使張憲守魏州,大軍自楊劉濟河。是歲,擒王彥章,誅梁氏,降段凝,皆崇韜贊成其謀也。



 莊宗至汴州,宰相豆盧革在魏州,令崇韜權行中書事。俄拜侍中兼樞密使,及郊禮畢,以崇韜兼領鎮、冀州節度使,進封趙郡公,邑二千戶,賜鐵券,恕十死。崇韜既位極人臣,權
 傾內外,謀猷獻納,必盡忠規,士族朝倫,頗亦收獎人物,內外翕然稱之。初收汴、洛,稍通賂遺,親友或規之,崇韜曰:「余備位將相,祿賜巨萬,但偽梁之日,賂遺成風,今方面籓侯,多梁之舊將,皆吾君射鉤斬祛之人也。一旦革面,化為吾人,堅拒其請,得無懼乎!藏餘私室,無異公帑。」及郊禋,崇韜悉獻家財,以助賞給。時近臣勸莊宗以貢奉物為內庫,珍貨山積,公府賞軍不足。崇韜奏請出內庫之財以助,莊宗沉吟有靳惜之意。是時天下已定,寇
 仇外息,莊宗漸務華侈,以逞己欲。洛陽大內宏敞,宮宇深邃,宦官阿意順旨,以希恩寵,聲言宮中夜見鬼物,不謀同辭。莊宗駭異其事,且問其故。宦者曰:「見本朝長安大內,六宮嬪御,殆及萬人,椒房蘭室,無不充牣。今宮室大半空閑,鬼神尚幽,亦無所怪。」由是景進、王允平等於諸道採擇宮人,不擇良賤,內之宮掖。



 三年夏,雨,河大水,壞天津橋。是時,酷暑尤甚。莊宗常擇高樓避暑,皆不稱旨。宦官曰:「今大內樓觀,不及舊時長安卿相之家,舊日
 大明、興慶兩宮,樓觀百數,皆雕楹畫栱,干雲蔽日,今官家納涼無可御者。」莊宗曰:「予富有天下,豈不能辦一樓!」即令宮苑使經營之,猶慮崇韜有所諫止,使謂崇韜曰:「今年惡熱,朕頃在河上,五六月中,與賊對壘,行宮卑濕,介馬戰賊,恆若清涼。今晏然深宮,不耐暑毒,何也?」崇韜奏:「陛下頃在河上,汴寇未平,廢寢忘食,心在戰陣,祁寒溽暑,不介聖懷。今寇既平,中原無事,縱耳目之玩,不憂戰陣,雖層臺百尺,廣殿九筵,未能忘熱於今日也。願陛
 下思艱難創業之際,則今日之暑,坐變清涼。」莊宗默然。王允平等竟加營造,崇韜復奏曰:「內中營造,日有縻費,屬當災饉,且乞權停。」不聽。



 初,崇韜與李紹宏同為內職,及莊宗即位,崇韜以紹宏素在己上,舊人難制《論理智》、《論五種原素》、《邏輯學引論》、《曲調寫作與琉特,即奏澤潞監軍張居翰同掌樞密,以紹宏為宣徽使。紹宏大失所望,泣涕憤鬱。崇韜乃置內勾使,應三司財賦,皆令勾覆,令紹宏領之,冀塞其心。紹宏怏悵不已。崇韜自以有大功,河、洛平定之後,權位熏灼,恐為人所傾奪,乃謂諸
 子曰:「吾佐主上,大事了矣,今為群邪排毀,吾欲避之,歸鎮常山,為菟裘之計。」其子廷說等曰:「大人功名及此,一失其勢,便是神龍去水,為螻蟻所制,尤宜深察。」門人故吏又謂崇韜曰:「侍中勛業第一,雖群官側目,必未能離間。宜於此時堅辭機務,上必不聽,是有辭避之名,塞其讒慝之口。魏國夫人劉氏有寵,中宮未正,宜贊成冊禮,上心必悅。內得劉氏之助,群閹其如餘何!」崇韜然之,於是三上章堅辭樞密之位,優詔不從。崇韜乃密奏請立
 魏國夫人為皇后,復奏時務利害二十五條,皆便於時,取悅人心;又請罷樞密院事,各歸本司,以輕其權,然宦官造謗不已。



 三年,堅乞罷兼領節鉞,許之。《冊府元龜》云:同光中,崇韜再表辭鎮,批答曰;「朕以卿久司樞要,常處重難。或遲疑未決之機,詢諸先見;或憂撓不定之事,訪自必成。至於贊朕丕基,登茲大寶,眾興異論,卿獨堅言,天命不可違,唐祚必須復,請納家族,明設誓文,及其密取汶陽,興師入不測之地;潛通河口,貢謀占必濟之津。人所不知,卿惟合意。迨中都嘯聚,群黨窺陵,朕決議平妖,兼收浚水,雖云先定,更審前籌,果盡贊成,悉諧沈算,斯即何須冒刃,始顯殊庸。況常山陸梁,正虞未復,卿能撫眾,共定群心,惟朕知卿,他人寧表。所以賞卿之龐,實異等倫;沃朕之心,非虛渥澤。今卿再三謙遜,重疊退辭,始納常
 陽,請歸上將,又稱梁苑,不可兼權。如此周身,貴全名節,古人操守,未可比方,既覽堅辭,難沮來表。其再讓汴州,所宜依允。」



 會客省使李嚴使西川回,言王衍可圖之狀,莊宗與崇韜議討伐之謀,方擇大將。時明宗為諸道兵馬慈管當行,崇韜自以宦者相傾育學、文獻考古及時政諸方面。反映作者早年在新文化運動,欲立大功以制之,乃奏曰:「契丹犯邊,北面須藉大臣,全倚總管鎮禦。臣伏念興聖宮使繼岌,德望日隆,大功未著,宜依故事,以親王為元帥,付以討伐之權,俾成其威望。」莊宗方愛繼岌,即曰:「小兒幼稚,安能獨行,卿當擇其副。」崇韜未奏,莊宗曰:「無踰
 於卿者。」乃以繼岌為都統,崇韜為招討使。是歲九月十八日,率親軍六萬,進討蜀川。崇韜將發,奏曰:「臣以非才,謬當戎事,仗將士之忠力,憑陛下之威靈,庶幾克捷。若西川平定,陛下擇帥,如信厚善謀,事君有節,則孟知祥有焉,望以蜀帥授之。如宰輔闕人,張憲有披榛之勞,為人謹重而多識。其次李琪、崔居儉,中朝士族,富有文學,可擇而任之。」莊宗御嘉慶殿,置酒宴征西諸將,舉酒屬崇韜曰:「繼岌未習軍政,卿久從吾戰伐,西面之事,屬之
 於卿。」



 軍發,十月十九日入大散關,崇韜以馬箠指山險謂魏王曰;「朝廷興師十萬,已入此中,儻不成功,安有歸路?今岐下飛挽,才支旬日,必須先取鳳州,收其儲積,方濟吾事。」乃令李嚴、康延孝先馳書檄,以諭偽鳳州節度使王承捷。及大軍至,承捷果以城降,得兵八千,軍儲四十萬。次至故鎮,偽命屯駐指揮使唐景思亦以城降,得兵四千。又下三泉,得軍儲三十餘萬。自是師無匱乏,軍聲大振。其招懷制置,官吏補置,師行籌畫,軍書告諭,皆
 出於崇韜,繼岌承命而已。莊宗令內官李廷安、李從襲、呂知柔為都統府紀綱,見崇韜幕府繁重,將吏輻輳,降人爭先賂遺,都統府唯大將省謁,牙門索然,由是大為詬恥。及六軍使王宗弼歸款,行賂先招討府。王衍以成都降,崇韜居王宗弼之第。宗弼選王衍之妓妾珍玩以奉崇韜,求為蜀帥,崇韜許之。又與崇韜子廷誨謀,令蜀人列狀見魏王,請奏崇韜為蜀帥。繼岌覽狀謂崇韜曰:「主上倚侍中如衡、華,安肯棄元老於蠻夷之地,況餘不
 敢議此。」《九國志·王宗弼傳》:宗弼送款於魏王,乃還成都,盡輦內藏之寶貨,歸於其家。魏王遣使徵犒軍錢數千萬,宗弼輒靳之,魏王甚怒。及王師至,令其子承班齎衍玩用直百萬,獻於魏王,並賂郭崇韜,請以己為西川節度使。魏王曰:「此吾家之物,焉用獻為!」魏王入城,翼日,數其不忠之罪,並其子斬之於市。李從襲等謂繼岌曰:「郭公收蜀部人情,意在難測,王宜自備。」由是兩相猜察。



 莊宗令中官向延嗣齎詔至蜀,促班師,詔使至,崇韜不郊迎,延嗣憤憤。從襲謂之曰:「魏王,貴太子也,主上萬福,郭公專弄威柄,旁若無人。昨令蜀人請己為帥,郭廷誨擁徒出入,貴擬王者,所與狎遊,無非軍中
 驍果,蜀中凶豪,晝夜妓樂歡宴,指天畫地,父子如此,可見其心。今諸軍將校,無非郭氏之黨,魏王懸軍孤弱,一朝班師,必恐紛亂,吾屬莫知暴骨之所!」因相向垂涕。延嗣使還具奏,皇后泣告莊宗,乞保全繼岌。莊宗復閱蜀簿曰:「人言蜀中珠玉金銀,不知其數,何如是之微也!」延嗣奏曰:「臣問蜀人,知蜀中寶貨皆入崇韜之門,言崇韜得金萬兩,銀四十萬,名馬千匹,王衍愛妓六十,樂工百,犀玉帶百。廷誨自有金銀十萬兩,犀玉帶五十,藝色絕
 妓七十,樂工七十,他財稱是。魏王府,蜀人賂不過遣匹馬而已。」莊宗初聞崇韜欲留蜀,心已不平,又聞全有蜀之妓樂珍玩,怒見顏色。即令中官馬彥珪馳入蜀視崇韜去就,如班師則已,如實遲留,則與繼岌圖之。彥珪見皇后曰:「禍機之發,間不容髮,何能數千里外復稟聖旨哉!」皇后再言之,莊宗曰:「未知事之實否,詎可便令果決?」皇后乃自為教與繼岌,令殺崇韜。時蜀土初平,山林多盜,孟知祥未至,崇韜令任圜、張筠分道招撫,慮師還後,
 部曲不寧,故歸期稍緩。



 四年正月六日,馬彥珪至軍,決取十二日發成都赴闕,令任圜權知留事,以俟知祥。諸軍部署已定,彥珪出皇后教以示繼岌,繼岌曰:「大軍將發,他無釁端,安得為此負心事!公輩勿復言。」從襲等泣曰:「聖上既有口敕,王若不行,茍中途事洩,為患轉深。」繼岌曰:「上無詔書,徒以皇后教令,安得殺招討使!」從襲等巧造事端以間之,繼岌既英斷,僶俛從之。詰旦,從襲以繼岌之命召崇韜計事,繼岌登樓避之,崇韜入,左右
 楇殺之。崇韜有子五人,廷信、廷誨隨父死於蜀,廷說誅於洛陽,廷讓誅於魏州,廷議誅於太原,家產籍沒。明宗即位,詔令歸葬,仍賜太原舊宅。延誨、廷讓各有幼子一人,姻族保之獲免,崇韜妻周氏,攜養於太原。



 崇韜服勤盡節,佐佑王家,草昧艱難,功無與比,西平巴蜀,宣暢皇威,身死之日,夷夏冤之。然議者以崇韜功烈雖多,事權太重,不能處身量力,而聽小人誤計,欲取泰山之安,如急行避跡,其禍愈速。性復剛戾,遇事便發,既不知前代
 之成敗,又未體當時之物情,以天下為己任,孟浪之甚也。及權傾四海,車騎盈門,士人諂奉,漸別流品。同列豆盧革謂崇韜曰:「汾陽王代北人,徙家華陰,侍中世在鴈門,得非祖德歟?」崇韜應曰:「經亂失譜牒,先人嘗云去汾陽王四世。」革曰:「故祖德也。」因是旌別流品,援引薄徒,委之心腹;佐命勳舊,一切鄙棄。舊僚有干進者,崇韜謂之曰:「公雖代邸之舊,然家無門閥,深知公才技,不敢驟進者,慮名流嗤餘故也。」及征蜀之行,于興平拜尚父子儀
 之墓。嘗從容白繼岌曰:「蜀平之後,王為太子,待千秋萬歲,神器在手,宜盡去宦官,優禮士族,不唯疏斥閹寺,騸馬不可復乘。」內則伶官巷伯,怒目切齒;外則舊僚宿將,戟手痛心。掇其族滅之禍,有自來矣。復以諸子驕縱不法,既定蜀川,輦運珍貨,實於洛陽之第,籍沒之日,泥封尚濕。雖莊宗季年為群小所惑,致功臣不保其終,亦崇韜自貽其災禍也。



 史臣曰:夫出身事主,得位遭時,功不可以不圖,名不可
 以不立。


\gezhu{
  以
  下缺}



\end{pinyinscope}