\article{末帝紀下}

\begin{pinyinscope}

 清泰三年春正月辛卯朔,帝御文明殿受朝賀,仗衛如式。乙未,百濟遣使獻方物。戊戌,幸龍門佛寺祈雪。癸卯,以給事中、充樞密院直學士呂琦為端明殿學士;以六
 軍諸衛判官、尚書工部郎中薛文遇為樞密院直學士。乙巳,以上元夜京城張燈,帝微行,置酒於趙延壽之第。丁未,皇子河南尹、判六軍諸衛事重美封雍王。己未,以前司農卿王彥鎔為太僕卿。


二月戊辰,吐渾寧朔兩府留後李可久加檢校司徒。可久本姓白氏,前朝賜姓。庚午,監修國史姚顗,史官張昭遠、李祥、吳承範等修撰《明宗實錄》三十卷上之。
 \gezhu{
  《五代會要》:同修撰官中書舍人張昭遠、李祥,直館左拾遺吳承範,右拾遺楊昭儉等各頒賚有差。}
 以大理卿竇維為光祿卿,以前許州節
 度判官張登為大理卿。丁丑,以太常卿李鈴為兵部尚書,以兵部尚書梁文矩為太常卿。庚辰,以前鄜州節度使皇甫立為潞州節度使。辛巳,以前均州刺史仇暉為左威衛上將軍,保順軍節度使鮑君福加檢校太尉、同平章事。丁亥,以昭義節度使安元信卒廢朝。



 三月庚子,中書門下奏:「準閣門分析內外官辭見謝規例:諸州判官、軍將進奉到闕,舊例門見門辭;今後只令朝見,依舊門辭。新除諸道判官、書記以下無例中謝,並放謝放辭,
 得替到京無例見;今後兩使判官許中謝,赴任即門辭,其書記以下並依舊例。朝臣文五品、武四品以上舊例中謝,其以下無例對謝;今請依天成四年正月敕,凡升朝官並許中謝。諸道都押衙、馬步都指揮、虞候、鎮將、諸色場院,無例謝辭,並進榜子放謝放辭,得替到闕,無例入見。在京鹽曲稅官、兩官巡即許中謝,新除令、錄並中謝,次日門辭,兼有口敕誡勵。文武兩班所差吊祭使及告廟祠祭,只正衙辭,不赴內殿。諸道進奏官到闕,見,得
 假,進榜子門辭。」從之。辛丑,權知福建節度使王昶奏,節度使王延鈞以去年十月十四日卒。是時延鈞父子雖僭竊於閩嶺,猶稱籓於朝廷,故有是奏。甲辰,以右神武統軍楊漢章為彰武軍節度使。丙午,以翰林學士、禮部侍郎馬裔孫為中書侍郎、同平章事。丁巳,以端明殿學士呂琦為御史中丞。案《通鑒》:呂琦與李崧建和親契丹之策,為薛文遇所沮,改為御史中丞,蓋疏之也。戊午,御史中丞盧損責授右贊善大夫,知雜侍御史韋稅責授太僕寺丞,侍御史魏遜責授太府寺主
 簿,侍御史王岳責授司農寺主簿。初,延州保安鎮將白文審聞兵興岐下,專殺郡人趙思謙等十餘人,已伏其罪,復下臺追系推鞫,未竟。會去年五月十二日德音,除十惡五逆、放火殺人外並放。盧損輕易即破械釋文審,帝大怒,收文審誅之。臺司稱奉德音釋放,不得追領祗證。中書詰云,德音言「不在追窮枝蔓」,無「不得追領祗證」六字,擅改敕語。大理斷以失出罪人論,故有是命。是月,有蛇鼠鬥于師子門外,鼠生而蛇死。



 夏四月己未朔,以
 左衛上將軍王景戡為左神武統軍,以右領軍上將軍李頃為華清宮使。戊辰,以太子詹事盧演為工部尚書致仕。辛未,以中書舍人、史館修撰張昭遠為禮部侍郎;以前滄州節度使李金全為右領軍上將軍。是月,有熊入京城捕人。



 五月辛卯,以河東節度使、兼大同彰國振武威塞等軍蕃漢馬步總管、檢校太師、兼中書令、駙馬都尉石敬瑭為鄆州節度使,進封趙國公。以河陽節度使、充侍衛馬步軍都指揮使宋審虔為河東節度使。甲
 午,以前晉州節度使、大同彰國振武威塞等軍蕃漢副總管張敬達充西北面蕃漢馬步都部署,落副總管。乙未,詔:「諸州兩使判官、畿赤令有闕,取省郎、遺補、丞博、少列宮僚,選擇擢任。」以忠正軍節度使、侍衛步軍都指揮使張彥琪為河陽節度使,充侍衛馬軍都指揮使;以彰聖都指揮使、饒州刺史符彥饒為忠正軍節度使,充侍衛步軍都指揮使。丙申,以雍王重美與汴州節度使範延光結婚,詔兗王從溫主之。丁酉,以國子祭酒馬縞卒
 廢朝。



 戊戌,昭義奏,河東節度使石敬瑭叛。以鴻臚卿兼通事舍人、判四方館王景崇為衛尉卿,充引進使。壬寅,削奪石敬瑭官爵,便令張敬達進軍攻討。乙卯。以晉州節度使張敬達為太原四面兵馬都部署,尋改為招討使;以河陽節度使、侍衛馬軍都指揮使張彥琪為太原四面馬步軍都指揮使;以邢州節度使安審琦為太原四面馬軍都指揮使;以陜州節度使相裏金為太原四面步軍都指揮使;以右監門上將軍武廷翰為壕寨使。
 丙辰,以定州節度使楊光遠為太原四面兵馬副部署、兼馬步都虞候,尋改為太原四面副招討使,都虞候如故。以前彰武軍節度使高行周為太原四面招撫兼排陣使。初,帝疑河東有異志,與近臣語及其事,帝曰:「石郎與朕近親,在不疑之地,流言毀譽,朕心自明,萬一失歡,如何和解?」左右皆不對。翼日,欲移石敬瑭于鄆州,房暠等堅言不可,司天監趙延乂亦言星辰失度,尤宜安靜,由是稍緩其事。會薛文遇獨宿于禁中,帝召之,諭以
 太原之事。文遇奏曰:「臣聞作舍於道,三年不成,國家利害,斷自宸旨以臣料之,石敬瑭除亦叛,不除亦叛,不如先事圖之。」帝喜曰:「聞卿此言,豁吾憤氣。」先是,有人言國家明年合得一賢佐主謀,平定天下,帝意亦疑賢佐者屬在文遇,即令手書除目,子夜下學士院草制。翼日,宣制之際,兩班失色。居六七日,敬瑭上章云:「明宗社稷,陛下纂承,未契輿情,宜推令辟。許王先朝血緒,養德皇闈,儻循當璧之言,免負鬩墻之議。」帝覽奏不悅,手攘抵地,召
 馬裔孫草詔報曰:「父有社稷,傳之於子;君有禍難,倚之於親。卿於鄂王,故非疏遠。往歲衛州之事,天下皆知;今朝許王之言,人誰肯信!英賢立事,安肯如斯」云。



 戊申,張敬達奏,西北面先鋒都指揮使安審信率雄義左第二指揮二百二十七騎,並部下共五百騎,剽劫百井,叛入太原。又奏,大軍已至太原城下。詔安審信及雄義兵士妻男並處斬,家產沒官。先是,雄義都在伏州屯戍,其指揮使安元信謀殺伏州刺史張朗,事洩,戍兵自潰,奔安
 審信軍,審信與之入太原。太常奏,於河南府東權立宣憲太后寢宮,從之。己酉,振武節度使安叔千奏,西北界巡檢使安重榮驅掠戍兵五百騎叛入太原。以新授河東節度使宋審虔為宣州節度使,充侍衛馬軍都指揮使。壬子,鄴都屯駐捧聖都虞候張令昭逐節度使劉延皓,據城叛。翼日,令昭召副使邊仁嗣已下逼令奏請節旄。



 六月辛酉,天雄軍節度使劉延皓削奪官爵,勒歸私第。癸亥,以天雄軍守禦、右捧聖第二軍都虞候張令昭
 為檢校司空,行右千牛將軍國;欲治其國者,先齊其家;欲齊其家者,先修其身。」這種,權知天雄軍府事。丙寅,御敷政殿,遣工部尚書崔居儉奉宣憲皇太后寶冊於寢宮。時陵園在河東,適會兵興,故權於京城修奉寢宮上謚焉。己巳,以西上閣門副使、少府監兼通事舍人劉頎為鴻臚卿,職如故。庚午,詔曰:「時雨稍愆,頗傷農稼,分命朝臣祈禱。」辛未,工部尚書致仕許寂卒。以權知魏府事、右千牛將軍張令昭為齊州防禦使,以捧聖右第三指揮使邢立為德州刺史,以捧聖第五指揮使康福進為
 鄚州刺史。甲戌,以汴州節度使范延光為天雄軍四面招討使,知行府事。丙子,以西京留守李周為天雄軍四面副招討使兼兵馬都監。詔河東將佐節度判官趙瑩以下十四人並籍沒家產。



 秋七月戊子,范延光奏,領軍至鄴都攻城。己丑,誅右衛上將軍石重英、皇城副使石重裔,皆敬瑭之子也。時重英等匿於民家井中,獲而誅之,並族所匿之家。奚首領達罕軍遣通事介老奏,奚王李素姑謀叛入契丹,已處斬訖,達喇罕權知本部落
 事。辛卯,沂州奏,誅都指揮使石敬德,并族其家,敬瑭之弟也。乙未,以前彰武軍節度使高行周為潞州節度使,充太原四面招撫排陣使;以潞州節度使皇甫立為華州節度使。丁酉,雲州節度使沙彥珣奏,此月二日夜,步軍指揮使桑遷作亂,以兵圍子城,彥珣突圍出城,就西山據雷公口。三日,招集兵士入城誅亂軍,軍城如故。辛丑,以將作監丞、介國公宇文頡為汝州襄城令。乙巳,以衛尉卿聶延祚為太子賓客。戊申,范延光奏,此月二十
 一日收復鄴都,群臣稱賀。己酉,以禮部侍郎張昭遠為御史中丞;以御史中丞呂琦為禮部侍郎,充端明殿學士。庚戌,中書奏:「劉延皓賓佐等,帥臣既已削奪,其行軍司馬李延筠、副使邊仁嗣以下,望命放歸田里。」奏入,帝大怒,詔大理曰:「帥臣失守,已行削奪,其僚佐合當何罪?」既而竟依中書所奏。壬子,詔范延光誅張令昭部下五指揮及忠銳、忠肅兩指揮。繼范延光奏,追兵遣襲張令昭部下敗兵至邢州沙河,斬首三百級,并獻張令昭、邢
 立、李貴等首級。又奏,獲張令昭同惡捧聖指揮使米全以下諸指揮使都頭凡十三人,並磔于府門。癸丑,左衛上將軍仇暉卒。洺州奏擒獲魏府作亂捧聖指揮使馬彥柔以下五十八人。邢、磁州相次擒獲亂兵,並送京師。彰聖指揮使張萬迪以部下五百騎叛入太原,詔誅家屬於懷州本營。



 八月戊午,契丹遣使摩哩入朝。己未,以汴州節度使范延光為天雄軍節度使、守太傅、兼中書令;以西京留守李周為汴州節度使、檢校太尉、同平章
 事。癸亥,應州奏,契丹三千騎迫城。詔端明殿學士呂琦往河東忻、代諸屯戍所犒軍。以左龍武大將軍袁義為右監門上將軍,以振武軍節度使安叔千充代北兵馬都部署。己巳,雲州沙彥珣奏,供奉官李讓勛送夏衣到州,縱酒凌轢軍都行,劫殺兵馬都監張思殷、都指揮使黨行進,其李讓勛已處斬訖。張敬達奏,造五龍橋攻太原城次。戊寅,以鎮州節度使董溫琪充東北面副招討使。己卯,洺州獻野繭二十斤。辛巳,張敬達奏,賊城內出
 騎軍三十隊、步卒三千人衝長連城,高行周襲殺入壕,溺死者大半,擒賊將安小喜以下百餘人,甲馬一百八十匹。



 九月甲辰,張敬達奏,此月十五日,與契丹戰於太原城下,王師敗績。時契丹主自率部族來援太原,高行周、符彥卿率左右廂騎軍出鬥,蕃軍引退。巳時後,蕃軍復成列,張敬達、楊光遠、安審琦等陣於賊城西北,倚山橫陣,諸將奮擊,蕃軍屢卻。至晡,我騎軍將移陣,蕃軍如山而進,王師大敗,投兵仗相藉而死者山積。是夕,收合
 餘眾,保於晉祠南晉安寨,蕃軍塹而圍之,自是音聞阻絕。朝廷大恐。是日,遣侍衛步軍都指揮使符彥饒率兵屯河陽,詔范延光率兵由青山路趨榆次,詔幽州趙德鈞由飛狐路出敵軍後,輝州防禦使潘環合防戍軍出慈、隰以援張敬達。以前絳州刺史韓彥惲為太子賓客。契丹主移帳於柳林。乙巳,詔取二十二日幸北面軍前。戊申,帝發京師,路經徽陵,帝親行謁奠。夕次河陽,召群臣議進取,盧文紀勸帝駐河橋。庚戌,樞密使趙延壽先
 赴潞州。辛亥,幸懷州。召吏部侍郎龍敏訪以機事,敏勸帝立東丹王贊華為契丹主,以兵援送入蕃,則契丹主有後顧之患,不能久駐漢地矣。帝深以為然,竟不行其謀。《遼史·義宗傳》云:「倍雖在異國,常思其親,問安之使不絕。後明宗養子從珂弒其君自立,倍密報太宗曰:「從珂弒君,盍討之!」是東丹王實啟兵端,唐君臣或知其陰謀,故龍敏之說不行。帝自是酣飲悲歌,形神慘沮。臣下勸其親征,則曰:「卿輩勿說石郎,使我心膽墮地!」其怯憊也如此。



 冬十月丁巳夜,彗星出虛危,長尺餘。壬戌,詔天下括馬,又詔民十戶出兵一人,器甲
 自備。《契丹國志》:唐發民為兵,每七戶出征夫一人,自備鎧仗,謂之「義軍」,凡得馬二千餘匹,征夫五千人,民間大擾。戊辰,代州刺史張朗超授檢校太保,以其屢殺敵眾,故以是命獎之。癸酉,幽州趙德鈞以本軍三千騎與鎮州董溫琪由吳兒谷趨潞州。



 十一月戊子,以趙德鈞為諸道行營都統,以趙延壽為河東道南面行營招討使,以劉延朗副之。庚寅,以范延光為河東道東南面行營招討使寧文集》俄文版。由部分大專院校及中央編譯局列斯室翻譯,,以李周副之。帝以呂琦嘗佐幽州幕,乃命齎都統官告以賜德鈞,兼犒軍士。琦至,從容宣帝委任之
 意,德鈞曰:「既以兵相委,焉敢惜死!」德鈞志在併范延光軍,奏請與延光會合。帝以詔諭延光,延光不從。丁酉,延州上言,節度使楊漢章為部眾所殺,以前坊州刺史劉景嚴為延州留後。庚子,趙德鈞奏,大軍至團柏谷,前鋒殺蕃軍五百騎。范延光奏,軍至榆次,蕃軍退入河東川界。潘環奏,隰州逐退蕃軍。壬寅,趙德鈞奏,軍出谷口,蕃軍漸退,契丹主見駐柳林寨。時德鈞累奏乞授延壽鎮州節制,帝怒曰:「德鈞父子堅要鎮州,茍能逐退蕃戎,要
 代予位,亦甘心矣。若玩寇要君,但恐犬兔俱斃。」德鈞聞之不悅。



 閏月丙辰,日南至,群臣稱賀于行宮,帝曰:「晉安寨內將士,應思家國矣。」因泣下久之。丁巳,以岢嵐軍為勝州。辛酉,以右龍武統軍李從昶為左龍武統軍,以前邠州節度使楊思權為右龍武統軍。壬戌,丹州刺史康承詢停任,配流鄧州。時承詢奉詔率義軍赴延州義軍亂,承詢奔鄜州,故有是責。甲子,太原行營副招討使楊光遠殺招討使張敬達于晉安寨,以兵降契丹。時契丹
 圍寨,自十一月以後芻糧乏絕,軍士毀居屋茅、淘馬糞、削松甗以供秣飼,馬尾鬣相食俱盡。楊光遠謂敬達曰:「少時人馬俱盡,不如奮命血戰,十得三四,猶勝坐受其弊。」敬達曰:「更少待之。」一日,光遠伺敬達無備,遂殺之,與諸將同降契丹。時馬猶有五千匹,契丹主以漢軍與石敬瑭,其馬及甲仗即齎驅出塞。丁卯,契丹立石敬瑭為大晉皇帝,約為父子之國,改元為天福。契丹與晉高祖南行,趙德鈞父子與諸將自團柏谷南奔,王師為蕃騎
 所蹙,投戈棄甲,自相騰踐,擠於巖谷者不可勝紀。



 己巳,帝聞晉安寨為敵所陷,詔移幸河陽,時議以魏府軍尚全,契丹必憚山東,未敢南下,車駕可幸鄴城。帝以李崧與范延光相善,召入謀之。薛文遇不知而繼至,帝變色,崧躡文遇足,乃出。帝曰:「我見此物肉顫,適擬抽刀刺之。」崧曰:「文遇小人,致誤大事,刺之益醜。」崧因請帝歸京。壬申,車駕至河陽。甲戌,晉高祖與契丹至潞州,契丹遣蕃將大詳袞率五千騎送晉高祖南行。丁丑,車駕至自河
 陽。時左右勸帝固守河陽。居數日,符彥饒、張彥琪至,奏帝不可城守。是日晚,至東上門,小黃門鳴鞘於路,索然無聲。己卯,帝遣馬軍都指揮使宋審虔率千餘騎至白馬坡,言踏陣地,時諸將謂審虔曰:「何地不堪交戰,誰人肯立於此?」審虔乃請帝還宮。庚辰,晉高祖至河陽。辛巳辰時,帝舉族與皇太后曹氏自燔於元武樓。晉高祖入洛,得帝燼骨於火中,來年三月,詔葬於徽陵之封中。帝在位共二年,年五十二。《五代史闕文》:晉高祖引契丹圍晉安寨,降楊光遠。清泰帝至自覃懷,
 京師父老迎帝於上東門外,帝垂泣不止。父老奏曰:「臣等伏聞前唐時中國有難,帝王多幸蜀以圖進取。陛下何不且入西川?」帝曰:「本朝兩川節度使皆用文臣,所以玄宗、僖宗避寇幸蜀。今孟氏已稱尊矣,吾何歸乎!」因慟哭入內,舉族自焚。



 史臣曰:末帝負神武之才,有人君之量。由尋戈而踐阼,慚德應深;及當宁以居尊,政經未失。屬天命不祐,人謀匪臧,坐俟焚如,良可悲矣!稽夫衽金甲于河需之際,斧眺樓于梁壘之時,出沒如神,何其勇也!及乎駐革輅于覃懷之日,絕羽書於汾晉之辰,涕淚霑襟,何其怯也!是
 知時之來也,雕虎可以生風;運之去也,應龍不免為醢。則項籍悲歌於帳下,信不虛矣!



\end{pinyinscope}