\article{莊宗紀六}

\begin{pinyinscope}
同光二年夏五月己亥,帝御文明殿,冊齊王張全義為太尉。禮畢,全義赴尚書省領事,左諫議大夫竇專不降階,為御史所劾,專援引舊典,宰相不能詰,寢而不行。庚
 子,太常卿李燕卒。壬寅,以教坊使陳俊為景州刺史,內園使儲德源為憲州刺史,皆梁之伶人也。初,帝平梁,俊與德源皆為寵伶周匝所薦,帝因許除郡,郭崇韜以為不可,伶官言之者眾,帝密召崇韜謂之曰:「予已許除郡,經年未行,我慚見二人,卿當屈意行之。」故有是命。
 \gezhu{
  《清異錄》:同光既即位,猶襲故態,身預俳優,尚方進御巾裹,名品日新。今伶人所預,尚有傳其遺制者。}



 甲辰,以兗州節度使李紹欽依前檢校太保、兗州節度使,進封開國侯;以邠州節度使韓恭依前檢校太保、邠州節度使,
 進封開國伯。丙午,以福建節度使、閩王王審知依前檢校太師、守中書令、福建節度使。戊申,幸郭崇韜第。己酉,詔天下收拆防城之具,不得修浚池隍。以西都留守、京兆尹張筠依前檢校太保,充西都留守。甲寅,以滄州節度使李紹斌充東北面招討使,以兗州節度李紹欽為副招討使,以宣徽使李紹宏為招討都監,率大軍渡河而北,時幽州上言契丹將寇河朔故也。



 乙卯,潞州叛將楊立遣使健步奉表乞行赦宥,帝令樞密副使宋唐玉
 齎敕書招撫。幽州上言,契丹營於州東南。丙辰,渤海國王大撰遣使貢方物。以澶州刺史李審益為幽州行軍司馬、蕃漢內外都知兵馬使。辛酉,故澤潞節度使丁會贈太師。詔割復州為荊南屬郡。壬戌,以權知鳳翔軍府事、涇州節度使李嚴為起復雲麾將軍、右金吾大將軍同正,依前檢校太尉、兼中書令,充鳳翔節度使。乙丑,以權知歸義軍留後曹義金為歸義軍節度使、沙州刺史、檢校司空。丙寅,李嗣源奏收復潞州。幽州上言,新授
 宣武軍節度使李存審卒。



 六月甲戌,中書侍郎兼吏部尚書、平章事、宏文館大學士豆盧革加右僕射,餘如故;侍中、監修國史、兼樞密使、鎮州節度使郭崇韜進爵邑,加功臣號;中書侍郎、平章事、集賢殿大學士趙光允加兼戶部尚書;禮部侍郎、平章事韋說加中書侍郎。宋州奏,節度使李紹安卒。丙子,李嗣源遣使部送潞州叛將楊立等到闕,並磔於市。潞州城峻而隍深,至是帝命刬平之,因詔諸方鎮撤防城之備焉。丁丑,有司上言:「洛陽
 已建宗廟,其北京太廟請停。」從之。


甲申,以衛國夫人韓氏為淑妃,燕國夫人伊氏為德妃,仍令所司擇日冊命。故河東節度副使、守左諫議大夫李襲吉贈禮部尚書;故河東節度副使、禮部尚書蘇循贈左僕射;故河東觀察判官、檢校右僕射司馬揆贈司空;故河東留守判官、工部尚書李敬義贈右僕射。丙戌,以順義軍節度使李令錫為許州節度使,以前保義軍留後李紹真為徐州節度使,以徐州節度使李紹榮為宋州節度使。戊子,汝
 州防禦使張繼孫賜死於本郡。繼孫即齊王張全義之假子也,本姓郝氏,為兄繼業等訟其陰事,故誅之。
 \gezhu{
  《冊府元龜》載:張繼業為河陽兩使留後。莊宗同光二年六月,繼業上疏稱:「弟繼孫,本姓郝,有母尚在,父全義養為假子,令管衙內兵士。自皇帝到京,繼孫私藏兵甲,招置部曲,欲圖不軌,兼私家淫縱,無別無義。臣若不自陳,恐累家族。」敕曰:「有善必賞,所以勸忠孝之方;有惡必誅,所以絕奸邪之跡。其或罪狀騰於眾口,醜行布於近親,須舉朝章,冀明國法。汝州防禦使張繼孫,本非張氏子孫,自小丐養,以至成立,備極顯榮,而不能酬撫育之恩,履謙恭之道,擅行威福,常恣奸兇,侵奪父權,惑亂家事,縱鳥獸之行,畜梟獍之心,有識者所不忍言,無賴者實為其黨。而又橫征暴斂,虐法峻刑,藏兵器於私家,殺平人於廣陌。罔思悛改,難議矜容,宜竄逐於遐方,仍歸還於姓氏,俾我
  勛賢之族,永除污穢之風。凡百臣僚,宜體朕命。可貶房州司戶參軍同正,兼勒復本姓。」尋賜自盡,仍籍沒資產。}



 己丑,以回鶻可汗仁美為英義可汗。詔改輝州為單州。庚寅,故左僕射裴樞,右僕射裴贄、崔遠並贈司徒;故靜海軍節度使獨孤損贈司空;故吏部尚書陸扆贈右僕射;故工部尚書王溥贈右僕射。裴樞等六人皆前朝宰輔,為梁祖所害於白馬驛,至是追贈焉。壬辰,以天平軍節度使、蕃漢總管副使、開府儀同三司、檢校太尉、兼中書令李嗣源為宣武軍節度使、蕃漢馬步總管,餘如故。
 甲午,以樞密使、特進、左領軍衛上將軍、知內侍省事張居翰為驃騎大將軍、守左驍衛上將軍,進封開國伯,賜功臣號。



 秋七月戊戌朔,故宣武軍節度使李存審男彥超進其父牙兵八千七百人。己亥,中書門下奏:「每年南郊壇四祠祭,太微宮五薦獻,並宰臣攝太尉行事,惟太廟遣庶僚行事,此後太廟祠祭,亦望差宰臣行事。」從之。乙巳,汴州雍丘縣大風,拔木傷稼。曹州大雨,平地水三尺。丙午,以襄州節度使孔勍為潞州節度使,李存霸為
 鄆州節度使。乙酉,幸龍門之雷山,祭天神,從北俗之舊事也。辛亥,以鄆州副使李紹珙為襄州留後,以前澤州刺史董璋為邠州留後。戊午,西川王衍遣偽署戶部侍郎歐陽彬來朝貢,稱「大蜀皇帝上書大唐皇帝」。庚申,以應州為雲州屬郡,升新州為威塞軍節度使,以媯、儒、武等州為屬郡。壬戌,皇子繼岌妻王氏封魏國夫人。幽州奏,契丹安巴堅東攻渤海。


八月己巳,詔洛京應有隙地,任人請射修造,有主者限半年,令本主自修蓋,如過限
 不見屋宇,許他人占射。
 \gezhu{
  《五代會要》載此詔云:籓方侯伯,內外臣僚,于京邑之中,無安居之所,亦可請射,各自修營。}
 辛未,北京副留守、太原尹孟知祥加檢校太傅,增邑,賜功臣號。帝畋于西苑。癸酉,以租庸副使、守衛尉卿孔謙為租庸使,以右威衛上將軍孔循為租庸副使。甲戌,以權知汴州軍州事、翰林學士承旨、戶部尚書盧質為兵部尚書,依前翰林學士承旨,仍賜論思匡佐功臣。丙子,以雲州刺史、鴈門以北都知兵馬使安元信為大同軍節度留後,以隰州刺史張廷裕為新州威
 塞軍節度留後。丁丑,樞密使郭崇韜上表請退,不允。戊寅,租庸使、守禮部尚書王正言罷使,守本官。辛巳,詔諸道節度、觀察、防禦、團練使、刺史,並于洛陽修宅一區。中書門下上言:「請今後諸道除節度副使、兩使判官外,其餘職員并諸州軍事判官,各任本處奏辟。」從之。
 \gezhu{
  《五代會要》:同光二年八月八日,中書門下奏:「諸道除節度副使及兩使判官除授外,其餘職員并軍事判官,伏以翹車著詠,箋帛垂文,式重弓旌,以光尊俎。由是副已知之薦,成接士之榮,必當備悉行藏,習知才行,允奉幕中之畫,以稱席上之珍。爰自偽梁,頗乖斯義,皆從除授,以佐籓宣。因緣多事之秋,慮爽得人之選,將期推擇,式示更張。今後諸道,
  除節度副使、判官兩使除授外,其餘職員并諸州軍事判官等,並任本道本州,各當辟舉,其軍事判官,仍不在奏官之限。」}
 汴州奏,大水損稼。癸未,租庸使孔謙進封會稽縣男,仍賜豐財贍國功臣。淮南楊溥遣使貢方物。宋州大水,鄆、曹等州大風雨,損稼。丁亥,中書門下侍郎奏:「請差左丞崔沂、吏部侍郎崔貽孫、給事中鄭韜光、李光序、吏部員外郎盧損等,同詳定選司長定格、循資格、十道圖。」從之。
 \gezhu{
  《五代會要》:同光二年八月,中書門下奏:「吏部三銓、下省、南曹、廢置、甲庫、格式、流外銓等司公事,並系長定格、循資格、十道圖等,前件格文,本朝創立,檢制姦濫,倫敘官資頗謂精詳,久同遵守。自亂離之後,巧偽滋多,
  兼同光元年八月,車駕在東京,權判工部員外郎盧重《本司起請》一卷,並以興復之始,務切懷來,凡有條流,多失根本,以至冬集赴選人,並南郊行事官,及陪位宗子共一千三百餘人,銓曹檢勘之時,互有援引,去留之際,不絕爭論,若又依違,必長訛濫。望差權判尚書省銓左丞崔沂、吏部侍郎崔貽孫、給事中鄭韜光、李光序、吏部員外郎盧損等,同詳定舊長定格、循資格、十道圖,務令簡要,可久施行。」從之。}
 癸巳,放朝參三日,以霖雨故也。陜州奏,河水溢岸。乙未,中書門下上言:「諸陵臺令丞請停,以本縣令知陵臺事。」從之。



 九月癸卯,畋于西北郊。幽州上言,契丹安巴堅自渤海國回軍。內園新殿成,名曰長春殿。戊申,以中書舍人、權知貢舉裴皞
 為禮部侍郎,以前鄭州防禦副使姜宏道為太僕卿。侍中郭崇韜奏:「應三銓注授官員等,內有自無出身入仕,買覓鬼名告敕;今將骨肉文書,揩改姓名;或歷任不足,妄稱失墜;或假人廕緒,托形勢論屬,安排參選,所司隨例注官。如有人陳告,特議超獎;其所犯人,檢格處分;若同保人內有偽濫者,並當駁放。應有人身死之處,今後並須申報本州,于告身上批書身死月日分明付子孫。今後銓司公事,至春末並須了畢。」從之。銓綜之司,偽濫
 日久,及崇韜條奏之後,澄汰甚嚴,放棄者十有七八,眾情亦怨之。己酉,司天臺請禁私歷日,從之。



 庚戌,有司自契丹至者,言女真、回鶻、黃頭室韋合勢侵契丹。壬子,有司上言:「八月二十二日夜,熒惑犯星二度,星周分也,請依法禳之。于京城四門懸東流水一罌,兼令都市嚴備盜火,止絕夜行。」從之。甲寅,幸郭崇韜第,置酒作樂。乙卯,以前振武節度使、安北都護馬存可依前檢校太尉、兼侍中,充寧遠軍節度、容管觀察使。存,湖南馬殷之弟也。
 丙辰,黑水國遣使朝貢。契丹寇幽州。戊午,宣宰臣于中書,磨勘吏部選人,謬濫者焚毀告敕。



 冬十月戊辰,帝畋于西北郊。己巳,故安義節度使、贈太尉、隴西郡王李嗣昭贈太師。庚午,正衙命使冊淑妃韓氏、德妃伊氏,以宰臣豆盧革、韋說充冊使。辛未,詔:「今後支郡公事,須申本道騰狀奏聞。租庸使各有徵催祇牒,觀察使貴全理體。」契丹寇易、定北鄙。壬申,故大同軍防禦使李存璋贈太尉。鄆州奏,清河泛溢,壞廬舍。癸未,畋于石橋。甲戌,河
 南尹張全義上言:「萬壽節日,請于嵩山開琉璃戒壇,度僧百人。」從之。乙亥,故守太師、尚書令、秦王李茂貞追封秦王,賜謚曰忠敬。丁丑,皇后差使賜兗州節度使李紹欽湯藥。時皇太后行誥命,皇后劉氏行教命,互遣使人宣達籓后,紊亂之弊,人不敢言。己卯,汴、鄆二州奏,大水。庚辰,以前太僕卿楊遘為大理卿。黨項進白驢,奚王李紹威進駝馬。幽州奏,契丹入寇,至近郊。辛巳,故天雄軍節度副使王緘贈司空。壬午,以天下兵馬都元帥、尚父、
 守尚書令、吳越國王錢鏐可依前天下兵馬都元帥、尚父、守尚書令,封吳越國王。癸未,幸小馬坊閱馬。甲申,以兩浙兵馬留後、清海軍節度、嶺南東道觀察等使、守太尉、兼侍中、廣州刺史錢元璙為檢校太師、兼中書令,充兩浙節度觀察留後,餘如故;以鎮東軍節度副大使、江南管內都招討使、建武軍節度、嶺南西道觀察等使、檢校太傅、守侍中、知蘇州中吳軍軍州事、行邕州刺史錢元璙為檢校太尉、兼中書令,餘如故。辛卯,天平軍監軍使
 柴重厚可特進、右領衛將軍同正,充鳳翔監軍使。甲午,以宣武軍節度押牙李從溫、李從璋、李從榮、李從厚、李從璨並銀青光祿大夫、檢校右散騎常侍兼御史大夫,宣武軍節度押牙李從臻可檢校國子祭酒兼御史中丞。自從溫而下,皆李嗣源諸子也。



 十一月丙申,靈武奏,甘州回鶻可汗仁美卒,其弟狄銀權主國事。吐渾白都督族帳移于代州東南。己亥,幸六宅宴諸弟。壬寅,尚書左丞、判吏部尚書銓事崔沂貶麟州司馬,吏部侍郎崔
 貽孫貶朔州司馬,給事中鄭韜光貶寧州司馬,吏部員外盧損貶府州司戶。時有選人吳延皓取亡叔告身故舊名求仕,事發,延皓付河南府處死,崔沂已下貶官。宰相豆盧革、趙光允、韋說詣閣門待罪,詔釋之。



 癸卯,帝畋于伊闕,侍衛金槍馬萬餘騎從,帝一發中大鹿。是日,命從官拜梁祖之陵,物議非之。其夕,宿于張全義之別墅。甲辰,宿伊闕縣。乙巳,宿椹澗。時騎士圍山,會夜,顛墜崖谷,死傷甚眾。丙午,復命衛兵分獵,殺獲萬計。是夜,方歸
 京城,六街火炬如晝。丁未,賜群臣鹿肉有差。



 庚戌,制改節將一十一人功臣號。辛亥,以兵部侍郎李德休為吏部侍郎。壬子,日南至,百官拜表稱賀。以昭儀侯氏為汧國夫人,昭容夏氏為虢國夫人,昭媛白氏為沛國夫人,出使美宣鄧氏為魏國夫人,御正楚真張氏為涼國夫人,司簿德美周氏為宋國夫人,侍真吳氏為渤海郡夫人,其餘並封郡夫人。丁巳,河中節度使、守太師、兼尚書令、西平王李繼麟可依前守太師、兼尚書令、河中護國
 軍節度使、西平王,仍賜鐵券。戊午,幸李嗣源、李紹榮之第,縱酒作樂。是日,鎮州地震;契丹寇蔚州。


十二月戊辰,幸西苑校獵。己巳,詔汴州節度使李嗣源歸鎮。
 \gezhu{
  《通鑒》云:己巳,命宣武節度使李嗣源將宿衛兵三萬七千人赴汴州,遂如幽州御契丹。}
 庚午,帝與皇后劉氏幸張全義第,酒酣,,帝命皇后拜全義為養父,全義惶恐致謝,復出珍貨貢獻。翼日,皇后傳制,命學士草謝全義書,學士趙鳳密疏,陳國后無拜人臣為父之禮,帝雖嘉之,竟不能已其事。壬申,以教坊使王承顏為興州刺
 史。丙子,詔取來年正月七日幸魏州。庚辰,畋于近郊,至夕還宮。壬午,契丹寇嵐州。黨項遣使貢方物。乙酉,幸龍門佛寺祈雪。丙戌,以徐州節度使李紹真為北面行營副招討使。戊子,李嗣源奏,部署大軍自宣武軍北征。淮南楊溥遣使貢獻。己丑,幸龍門。庚寅,詔河南尹張全義為洛京留守,判在京諸軍事。是日,日傍有背氣,凡十二。



 同光三年春正月甲午朔,帝御明堂殿受朝賀,仗衛如式。丙申,詔以昭宗、少帝山陵未備,宜令有司別選園陵
 改葬,尋以年饑財匱而止。契丹寇幽州。戊戌,詔:「起今後特恩授官及侍衛諸軍將校、內諸司等官,其告身官給,舊例朱膠錢、臺省禮錢並停,其餘合征臺省禮錢,比舊數五分中許征一分,特恩者不征。兵、吏部兩司逐月各支錢四十貫文,充吏人食直。少府監鑄錢造印文,今後不得徵納銅炭價直,其料物官給。」庚子,車駕發京師幸鄴。以前許州節度使李紹沖為太子少保;以前邠州節度使韓恭為右金吾大將軍,充兩街使;以前安州節度
 使朱漢賓為左龍武統軍。庚戌,車駕至鄴。命青州節度使符習修酸棗河隄。先是,梁末帝決河隄,引水東注至鄆、濮,以限我軍,至是方修之。丙辰,幽州上言,節度使李存賢卒。



 二月甲子朔,詔:「興唐府管內有百姓隨絲鹽錢,每兩與減五十文。逐年所人表蠶鹽,每斗與減五十文。小菉豆稅,每畝與減放三升。都城內所征稅絲,永與除放。」丙寅,定州節度使王都來朝。丁卯,畋于近郊。己巳,召從臣擊球于鞠場。辛未,許州上言:「襄城、葉縣準敕割隸汝
 州,其扶溝等縣請卻隸當州。」從之。甲戌,以滄州節度使李紹斌為幽州節度使,依前檢校太保;以大同軍留後安元信為滄州節度使。乙亥,幸王莽河射鴈。丙子,李嗣源奏,涿州東南殺敗契丹,生擒首領三十人。符習奏,修隄役夫遇雪寒逃散。樞密使郭崇韜上表辭兼鎮。時帝命李紹斌鎮幽州,以其時望未重,欲以李嗣源為鎮帥,且為紹斌聲援,移郭崇韜兼領汴州。召崇韜議之,崇韜奏以為當,因懇辭兼領。庚辰,以宣武軍節度使李嗣源
 為鎮州節度使。辛巳,以皇子繼潼、繼嵩、繼蟾、繼嶢並檢校司徒,皆沖幼,未出閣。突厥、渤海國皆遣使貢方物。帝幸近郊射鴈。甲申,以樞密使郭崇韜為依前守侍中、監修國史、兼樞密使,加食邑實封。廣南劉巖遣使奉書于帝,稱「大漢國王致書上大唐皇帝」。乙酉,帝射鴨于郭泊。丙戌,定州節度使、檢校太尉、兼侍中王都進封開國公,加食邑實封。戊子,幸近郊射鴈。工部尚書崔柅卒,贈右僕射。


三月癸巳朔,賜扈從諸軍將士優給,自二十千至
 一千。甲午,振武軍節度使、洛京內外蕃漢馬步使朱守殷奏,昨修月陂隄,至德宮南獲玉璽一紐,獻之。詔示百官,驗其文,曰「皇帝行寶」四字,方圓八寸,厚二寸,背紐交龍,光瑩精妙。守殷又于役所得古文錢四百六十六,內二十六文曰「得一元寶」,四百四十曰「順天元寶」,上之。
 \gezhu{
  龐元英《文昌雜錄》云:同光三年,洛京積善坊得古文錢,曰「得一元寶」,「順天元寶」,史不載何代鑄錢。近見錢氏《錢譜》云:史思明再陷洛陽,鑄「得一錢」,賊黨以為「得一」非佳號,乃改「順天」。蓋史思明所鑄錢也。}
 丙子,寒食節,帝與皇后出近郊,遙饗代州親廟。庚子,詔取三月十七
 日車駕歸洛京。壬寅,符習奏,修河堤畢功。



 戊申,帝召郭崇韜謂曰:「朕思在德勝寨時,霍彥威、段凝皆予之勍敵。終日格鬥,戰聲相聞,安知二年之間,在吾廡下。吾無少康、光武之才,一旦重興基構者,良由二三勳德同心輔翼故也。朕有時夢寢,如在戚城,思念曩時挑戰鏖兵,勞則勞矣,然而揚旌伐鼓,差慰人心,殘壘荒溝,依然在目。予欲按德勝故寨,與卿再陳舊事。」崇韜曰:「此去澶州不遠,陛下再觀戰地,益知王業之艱難,豈不韙哉!」己酉,車
 駕發鄴宮。辛亥,至德勝城。登城四望,指戰陣之處以諭宰臣。渡河南觀廢柵舊址,至楊村寨,沿河至戚城,置酒作樂而罷。壬子,淮南楊溥遣使朝貢。東京副留守張憲奏,諸營家口一千二百人逃亡,以艱食故也。時宮苑使王允平、伶人景進為帝廣采宮人,不擇良家委巷,殆千餘人,車駕不給,載以牛車,纍纍于路焉。庚辰,車駕至自鄴。辛酉,詔本朝以雍州為西京,洛州為東都,并州為北都。近以魏州為東京,宜依舊以洛京為東都,魏州改為
 鄴都,與北都並為次府。


夏四月癸亥朔,日有食之。以租庸副使孔循權知汴州軍州事。丙寅,淮南楊溥遣使貢方物。壬申,幸甘泉亭。癸酉,詔翰林學士承旨盧質覆試新及第進士。
 \gezhu{
  《五代會要》:時以新及第進士符蒙正等尚干浮議,故命盧質覆試。}
 租庸使奏:「時雨久愆,請下諸道州府,依法祈禱。」從之。乙亥,帝與皇后幸郭崇韜第,又幸左龍武統軍朱漢賓之第。戊寅,以耀州為團練州,其順義軍額宜停。庚辰,帝侍皇太后幸會節園,遂幸李紹榮之第。辛巳,以旱甚,詔河南府徙
 市,造五方龍,集巫禱祭。癸未,以兗州節度使李紹欽為鄧州節度使。丁亥,以鎮州節度使李嗣源兼北面水陸轉運使,以徐州節度使李紹真為副。禮部貢院新及第進士四人,其王澈改為第一,桑維翰第二,符蒙正第三,成僚第四。禮部侍郎裴皞既無黜落,特議寬容。今後新及第人,候過堂日委中書門下精加詳覆。陜州奏,木連理。庚寅,中書侍郎兼工部尚書、平章事趙光允卒,廢朝三日。



 五月壬辰朔,淮南楊溥貢端午節物。丁酉,皇太妃
 劉氏薨于晉陽,廢朝五日,帝于興安殿行服。時皇太后欲奔喪于晉陽,百官上表請留,乃止。戊戌,以鎮州行軍司馬、知軍府事任圜為工部尚書。戊申,幸龍門廣化寺祈雨。己酉,黑水、女真皆遣使朝貢。戊午,以鳳州衙內馬步軍都指揮使李繼昶為涇州節度使、檢校太傅。己未,詔天下見禁罪人,如無大過,速令疏放。幸太清宮禱雨。


六月癸亥,雲州上言,去年契丹從磧北歸帳,達靼因相掩擊,其首領裕悅族帳自磧北以部族羊馬三萬來降
 ,已到南界,今差使人來赴闕奏事。甲子,太白晝見。丁卯,以滄州節度使安元信充北面行營馬步軍都排陣使。辛未,以宗正卿李紓充昭宗、少帝改卜園陵使。壬申,京師雨足。自是大雨,至于九月,晝夜陰晦,未嘗澄霽,江河漂溢,隄防壞決,天下皆訴水災。丁丑,詔吳越王錢鏐將行冊禮,準禮文合用竹冊,宜令所司修製玉冊。時郭崇韜秉政,以為不可,樞密承旨段徊贊其事,故有是命。癸丑,以天德軍節度使、管內蕃漢都知兵馬使劉承訓為
 天德軍節度觀察留後。丙戌,詔曰:「關內諸陵,頃因喪亂,例遭穿穴,多未掩修。其下宮殿宇法物等,各令奉陵州府據所管陵園修製,仍四時各依舊例薦饗。每陵仰差近陵百姓二十戶充陵戶,以備灑掃。其壽陵等一十陵,亦一例修掩,量置陵戶。」戊子,以刑部尚書李琪充昭宗、少帝改卜園陵禮儀使。己丑,以工部郎中李途為京兆少尹,充修奉諸陵使。辛卯,詔括天下私馬,
 \gezhu{
  《五代會要》:詔下河南、河北諸州,和市戰馬,官吏除一匹外,匿者坐罪。}
 將收蜀故也。
 \gezhu{
  《三楚新錄》:莊宗謂高季興曰:「今天下負固
  不服者,惟吳、蜀耳。朕欲先有事于蜀,而蜀地險阻尤難,江南才隔荊南一水,朕欲先之,卿以為何如?」季興對曰:「臣聞蜀地富民饒,獲之可建大利;江南國貧,地狹民少,得之恐無益。臣願陛下釋吳先蜀。」時莊宗意亦欲伐蜀,及聞季興之言,果大悅。}



\end{pinyinscope}