\article{太祖紀一}

\begin{pinyinscope}

 太祖神武元聖孝皇帝,姓硃氏,諱晃,本名溫,宋州碭山人。其先,舜司徒虎之後。高祖黯,曾祖茂琳,祖信,父誠。帝即誠之第三子,母曰文惠王皇后。《五代會要》:梁肅祖宣元皇帝諱黯,舜司徒
 虎四十二代孫;開平元年七月,追尊宣元皇帝,廟號肅祖,葬興極陵。敬祖光獻皇帝諱茂琳,宣元皇帝長子,母曰宣僖皇后范氏;開平元年七月,追尊光獻皇帝,廟號敬祖,葬永安陵。憲祖昭武皇帝諱信,光獻皇帝長子,母曰光孝皇后楊氏;開平元年七月,追尊昭武皇帝,廟號憲祖,葬光天陵。烈祖文穆皇帝諱誠,昭武皇帝長子,母曰昭懿皇后劉氏;開平元年七月,追尊文穆皇帝,廟號烈祖,葬咸寧陵。以唐大中六年歲在壬申,十月二十一日夜,生于碭山縣午溝里。是夕,所居廬舍之上有赤氣上騰。里人望之,皆驚奔而來,曰:「硃家火發矣!」及至,則廬舍儼然。既入,鄰人以誕孩告,眾咸異之。昆仲三人,俱未冠而孤,母攜養寄于蕭縣人劉崇
 之家。帝既壯,不事生業,以雄勇自負,里人多厭之。崇以其慵惰,每加譴杖。唯崇母自幼憐之,親為櫛髮,嘗誡家人曰:「朱三非常人也,汝輩當善待之。」家人問其故,答曰:「我嘗見其熟寐之次,化為一赤蛇。」然眾亦未之信也。



 唐僖宗乾符中,關東薦饑,群賊嘯聚。黃巢因之,起于曹、濮,饑民願附者凡數萬。帝乃辭崇家,與仲兄存俱入巢軍,以力戰屢捷,得補為隊長。唐廣明元年十二月甲申,黃巢陷長安,遣帝領兵屯于東渭橋。是時,夏州節度使諸
 葛爽率所部屯于櫟陽,巢命帝招諭爽,爽遂降于巢。中和元年二月,巢以帝為東南面行營先鋒使,令攻南陽,下之。六月,帝歸長安,巢親勞于灞上。七月,巢遣帝西拒邠、岐、鄜、夏之師于興平,所至皆立功。



 二年二月,巢以帝為同州防禦使,使自攻取。帝乃自丹州南行,以擊左馮翊,拔之,遂據其郡。時河中節度使王重榮屯兵數萬,糾合諸侯,以圖興復。帝時與之鄰封,屢為重榮所敗,遂請濟師于巢。表章十上,為偽左軍使孟楷所蔽,不達。又聞
 巢軍勢蹙,諸校離心,帝知其必敗。九月,帝遂與左右定計,斬偽監軍使嚴實,舉郡降于重榮。重榮即日飛章上奏。時僖宗在蜀,覽表而喜曰:「是天賜予也!」乃詔授帝左金吾衛大將軍,充河中行營副招討使。仍賜名全忠。自是率所部與河中兵士偕行,所向無不克捷。三年三月,僖宗制授帝宣武軍節度使,依前充河中行營副招討使,仍令候收復京闕,即得赴鎮。四月,巢軍自藍關南走,帝與諸侯之師俱收長安,乃率部下一旅之眾,仗節東下。
 七月丁卯,入于梁苑。是時,帝年三十有二。時蔡州刺史秦宗權與黃巢餘孽合從肆虐,共圍陳州。久之,僖宗乃命帝為東北面都招討使。時汴、宋連年阻饑,公私俱困,帑廩皆虛,外為大敵所攻,內則驕軍難制,交鋒接戰,日甚一日;人皆危之,惟帝銳氣益振。是歲十二月,帝領兵于鹿邑,與巢眾相遇,縱兵擊之,斬首二千餘級,乃引兵入亳州,因是兼有譙郡之地。



 四年春,帝與許州田從異諸軍同收瓦子寨,殺賊數萬眾。是時,陳州四面,賊寨相望,驅
 擄編氓,殺以充食,號為「舂磨寨」。帝分兵翦撲,大小凡四十戰。四月丁巳,收西華寨,賊將黃鄴單騎奔陳。帝乘勝追之,鼓噪而進。會黃巢遁去,遂入陳州,刺史趙犨迎于馬前。俄聞巢黨尚在陳北故陽壘,帝遂徑歸大梁。是時,河東節度使李克用奉僖宗詔,統騎軍數千同謀破賊,與帝合勢于中牟北邀擊之,賊眾大敗于王滿渡,多束手來降。時賊將霍存、葛從周、張歸厚、張歸霸皆匍匐于馬前,悉宥而納之,遂逐殘寇,東至于冤句。



 五月甲戌,帝
 與晉軍振旅歸汴,館克用于上源驛。既而備犒宴之禮,克用乘醉任氣,帝不平之。是夜,命甲士圍而攻之。會大雨雷電,克用因得于電光中踰垣遁去,惟殺其部下數百人而已。六月,陳人感解圍之惠,為帝建生祠堂于其郡。是歲,黃巢雖歿,而蔡州秦宗權繼為巨孽,有眾數萬,攻陷鄰郡,殺掠吏民,屠害之酷,更甚巢賊,帝患之。七月,遂與陳人共攻蔡賊於溵水,殺數千人。九月己未,僖宗就加帝檢校司徒、同平章事,封沛郡侯,食邑千戶。



 光啟元年春,蔡賊掠亳、潁二郡。帝帥師以救之,遂東至於焦夷,敗賊眾數
 千,生擒賊將殷鐵林,梟首以徇軍而還。二月,僖宗自蜀還長安,改元光啟。四月戊辰,就加帝檢校太保,增食邑千五百戶。十二月,河中、太原之師逼長安,觀軍容使田令孜奉僖宗出幸鳳翔。



 二年春,蔡賊益熾。時唐室微弱,諸道州兵不為王室所用,故宗權得以縱毒,連陷汝、洛、懷、孟、唐、鄧、許、鄭,圜幅數千里,殆絕人煙,惟宋、亳、滑、潁僅能閉壘而已。帝累出兵與之交戰,然或勝或負,人甚危之。



 三月庚辰,僖宗降制就封帝為沛郡王。是月,僖宗移
 幸興元。五月,嗣襄王煴僭即帝位于長安,改元為建貞。遣使齎偽詔至汴,帝命焚之于庭。未幾,襄王果敗。七月,蔡人逼許州,節度使鹿宴宏使來求救,帝遣葛從周等率師赴援。師未至而城陷,宴宏為蔡賊所害。十一月,滑州節度使安師儒以怠于軍政,為部下所殺。帝聞之,乃遣硃珍、李唐賓襲而取之,由是遂有滑臺之地。十二月,僖宗降制就加帝檢校太傅,改封吳興郡王,食邑三千戶。



 是歲,鄭州為蔡賊所陷,刺史李璠單騎來奔,帝宥而
 納之,以為行軍司馬。宗權既得鄭,益驕,帝遣裨將邏于金隄驛,與賊相遇,因擊之,賊眾大敗,追至武陽橋,斬首千餘級。帝每與蔡人戰于四郊,既以少擊眾,常出奇以制之,但患師少,未快其旨。宗權又以己眾十倍於帝,恥于頻敗,乃誓眾堅決以攻夷門。既而獲蔡之諜者,備知其事,遂謀濟師焉。



 三年春二月乙巳,承制以朱珍為淄州刺史,俾募兵于東道,且慮蔡人暴其麥苗,期以夏首回歸。珍既至淄、棣,旬日之內,應募者萬餘人。又潛襲青
 州,獲馬千匹,鎧甲稱是,乃鼓行而歸。四月辛亥,達於夷門。帝喜曰:「吾事濟矣。」是時,賊將張晊屯于北郊,秦賢屯于版橋,各有眾數萬,樹柵相連二十餘里,其勢甚盛。帝謂諸將曰:「此賊方今息師蓄銳以俟時,必來攻我。況宗權度我兵少,又未知珍來,謂吾畏懼,止于堅守而已。今出不意,不如先擊之。」乃親引兵攻秦賢寨,將士踴躍爭先,賊果不備,連拔四寨,斬首萬餘級,時賊眾以為神助。庚午,賊將盧瑭領萬餘人于圃田北萬勝戍,夾汴水為
 營,跨河為梁,以扼運路。帝擇精銳以襲之。是日昏霧四合,兵及賊壘方覺,遂突入掩殺,赴水死者甚眾,盧瑭自投于河。河南諸賊連敗,不敢復駐,皆併在張晊寨。自是蔡寇皆懷震讋,往往軍中自相驚亂。帝旋師休息,大行犒賞,由是軍士各懷憤激,每遇敵,無不奮勇。五月丙子,出酸棗門,自卯至未,短兵相接,賊眾大敗,追斬二十餘里,僵仆就枕。宗權恥敗,益縱其虐,乃自鄭州親領突將數人,徑入張晊寨。其日晚,大星隕于賊壘,有聲如雷。辛
 巳,兗、鄆、滑軍士皆來赴援,乃陳兵于汴水之上,旌旗器甲甚盛。蔡人望之,不敢出寨。翌日,分布諸軍,齊攻賊寨,自寅至申,斬首二萬餘級。會夜收軍,獲牛馬、輜重、生口、器甲不可勝計。是夜,宗權、晊遁去,遲明追之,至陽武橋而還。宗權至鄭州,乃盡焚其廬舍,屠其郡人而去。始蔡人分兵寇陜、洛、孟、懷、許、汝,皆先據之,因是敗也,賊眾恐懼,咸棄之而遁。帝乃慎選將佐,俾完葺壁壘,為戰守之備,于是遠近流亡復歸者眾矣。是時,揚州節度使高駢
 為裨將畢師鐸所害,復有孫儒、楊行密互相攻伐,朝廷不能制,乃就加帝檢校太尉,兼領淮南節度使。



 九月,亳州裨將謝殷逐刺史宋袞,自據其郡;帝親領軍屯于太清宮,遣霍存討平之。帝之禦蔡寇也,鄆州朱瑄、兗州硃瑾皆領兵來援。及宗權既敗,帝以瑄、瑾宗人也,又有力于己,皆厚禮以歸之。瑄、瑾以帝軍士勇悍,私心愛之,乃密于曹、濮界上懸金帛以誘之,帝軍利其貨而赴者甚眾,帝乃移檄以讓之。硃瑄來詞不遜,乃命硃珍侵曹伐
 濮,以懲其姦。未幾,珍伐曹州,執刺史丘禮以獻,遂移兵圍濮。兗、鄆之釁,自茲而始矣。《通鑒考異》引高若拙《後史補》云:「梁太祖皇帝到梁園,深有大志。然兵力不足,常欲外掠;又虞四境之難,每有鬱然之狀。時有薦敬秀才于門下,乃白梁祖曰:『明公方欲圖大事,輜重必為四境所侵。但令麾下將士詐為叛者而逃,即明公奏于主上,及告四鄰,以自襲叛徒為名。』梁祖曰:『天降奇人,以佐於吾。』初從其議,一出而致眾十倍。」



 十月,僖宗命水部郎中王贊撰紀功碑以賜帝。是月,帝親帥騎數千巡師于濮上,因破硃瑄援師于范縣。丁未,攻陷濮州,刺史朱裕單騎奔鄆。尋為鄆人所敗,踰月乃還。十二月,僖宗遣使賜帝
 鐵券,又命翰林承旨劉崇望撰德政碑以賜帝。閏月甲寅。帝請行營司馬李璠權知淮南留後,乃遣大將郭言領兵援送以赴揚州。



 文德元年正月,帝率師東赴淮海,行次宋州,聞楊行密已拔揚州,遂還。是時,李璠、郭言行至淮上,為徐戎所扼,不克進而還。帝怒,遂謀伐徐。二月丙戌,僖宗制以帝為蔡州四面行營都統,由是諸鎮之師,皆受帝之節制。



 三月庚子,昭宗即位。是月,蔡人石璠領萬眾以剽陳、亳,帝遣朱珍率精騎數千擒璠以獻。四
 月戊辰,魏博樂彥禎失律,其子從訓出奔相州,使來乞師。帝遣朱珍領大軍濟河,連收黎陽、臨河二邑。既而魏軍推小校羅宏信為帥。宏信既立,遣使送款于汴,帝優而納之,遂命班師。是月,河南尹張全義襲李罕之于河陽,克之。罕之單騎出奔,因乞師于太原,李克用為發萬騎以援之。罕之遂收其眾,偕晉軍合勢,急攻河陽。全義危急,遣使求救于汴,帝遣丁會、牛存節、葛從周領兵赴之,大戰于溫縣,晉人與罕之俱敗。于是河橋解圍,全義
 歸于河陽,因以丁會為河陽留後。



 五月己亥,昭宗制以帝檢校侍中,增食邑三千戶。戊辰,昭改帝鄉衣錦,里曰沛王里。是月,帝以兼有洛、孟之地,無西顧之患,將大整師徒,畢力誅蔡。會蔡人趙德諲舉漢南之地以歸于朝廷,且遣使送款于帝,仍誓戮力同討宗權。帝表其事,朝廷因以德諲為蔡州四面副都統。又以河陽、保義、義昌三節度為帝行軍司馬,兼糧料應接。至是,帝領諸侯之師會德諲以伐蔡賊于汝水之上,遂薄其城。五日之內,
 樹二十八寨以環之,蓋象列宿之數也。時帝親臨矢石,一日,飛矢中其左腋,血漬單衣,顧謂左右曰:「勿洩。」九月,以糧運不繼,遂班師。是時,帝知宗權殘孽不足為患,遂移兵以伐徐。十月,先遣硃珍領兵與時溥戰于吳康鎮,徐人大敗,連收豐、蕭二邑;溥攜散騎馳入彭門。帝命分兵以攻宿州,刺史張友攜符印以降。既而徐人閉壁堅守,遂命龐師古屯兵守之而還。是月,蔡賊孫儒攻陷揚州,自稱淮南節度使。



 龍紀元年正月,龐師古攻下宿遷縣,
 進軍于呂梁。時溥領軍二萬,晨壓師古之軍而陣,師古促戰敗之,斬首二千餘級,溥復入于彭門。二月,蔡將申叢遣使來告,縛秦宗權于帳下,折其足而囚之矣。帝即日承制以叢為淮西留後。未幾,叢復為都將郭璠所殺。是月,璠執宗權來獻,帝遣行軍司馬李璠、牙校硃克讓檻送于長安。既至,昭宗御延喜樓受俘,即斬宗權于獨柳樹下。蔡州平。昭宗詔加帝食實封一百戶,賜莊宅各一區。三月,又加帝檢校太尉、兼中書令,進封東平王,賞
 平蔡之功也。



 大順元年四月丙辰,宿州小將張筠逐刺史張紹光,擁眾以附時溥。帝率親軍討之,殺千餘人,筠遂堅守。乙卯,時溥出兵暴碭山縣,帝遣朱友裕以兵襲之,敗徐軍三千餘眾,獲沙陀援軍石君和等三十人,斬于宿州城下。六月辛酉,淮南孫儒遣使修好于帝,帝表其事,請以淮南節度授于儒焉。辛未,昭宗命帝為宣義軍節度使,充河東東面行營招討使,時朝廷宰臣張浚將兵討太原故也。八月甲寅,昭義都將馮霸殺沙陀所
 署節度使李克恭來降,帝請河陽節度使硃崇節為潞州留後。戊辰,李克用自率蕃漢步騎數萬以圍潞州,帝遣葛從周率驍勇之士,夜中銜枚犯圍而入于潞。九月壬寅,帝至河陽,遣部將李讜引軍趨澤、潞,行至馬牢川,為晉人所敗。帝又遣硃友裕、張全義率精兵至鄆州北以為應援。既而崇節、從周棄潞來歸。戊申,帝廷責諸將敗軍之罪,斬李讜、李重允以徇,遂班師焉。十月乙酉,帝自河陽赴滑臺。時奉詔將討太原,先遣使假道於魏,魏
 人不從。先是,帝遣行人雷鄴告糴于魏,既而為牙軍所殺。羅宏信懼,故不敢從命,遂通好于太原。十二月辛丑,帝遣丁會、葛從周率眾渡河取黎陽、臨河,又令龐師古、霍存下淇門、衛縣,帝徐以大軍繼其後。



 二年春正月,魏軍屯于內黃。丙辰,帝與之接戰,自內黃至永定橋,魏軍五敗,斬首萬餘級。羅宏信懼,遣使持厚幣請和。帝命止其焚掠而歸其俘,宏信由是感悅而聽命焉。乃收軍屯于河上。八月己丑,帝遣丁會急攻宿州,刺史張筠堅守
 其壁,會乃率眾于州東築堰,壅汴水以浸其城。十月壬午,筠遂降,宿州平。十一月丁未,曹州裨將郭紹賓殺刺史郭饒,舉郡來降。是月,徐將劉知俊率眾二千來降,自是徐軍不振。十二月,兗州朱瑾領軍三萬寇單父,帝遣丁會領大軍襲敗之敗于金鄉界,殺二萬餘眾,瑾單馬遁去。



 景福元年正月,遣丁會于兗州界徙其民數千戶于許州。二月戊寅,帝親征鄆,先遣朱友裕屯軍于斗門。甲申,次衛南當前,其主,有飛鳥止于峻堞之上,鳴噪甚厲。副使李璠
 曰:「將有不如意之事。」是夜,鄆州朱瑄率步騎萬人襲朱友裕于斗門,友裕拔軍南去。乙酉,帝晨救斗門,不知友裕之退,前至斗門者皆為鄆人所殺。帝追襲鄆人至瓠河,不及,遂頓兵于村落間。時硃瑄尚在濮州。丁亥,遇朱瑄率兵將歸于鄆,遂來衝擊。帝策馬南馳,為賊所追甚急,前有浚溝,躍馬而過,張歸厚援槊力戰于其後,乃免。時李璠與部將數人皆為鄆軍所殺。五月丙午,遣朱克讓率眾暴兗、鄆之麥。十一月,遣硃友裕率兵攻濮州,下
 之,擒刺史郡儒以獻,濮州平。遂命移軍伐徐州。



 二年四月丁丑,龐師古下彭門,梟時溥首以獻。八月,帝遣龐師古移兵攻兗,駐于曲阜,與朱瑾屢戰,皆敗之。十二月,師古遣先鋒葛從周引軍以攻齊州,刺史硃威告急于兗、鄆。既而硃瑄以援兵至,遂固其壘。



 乾寧元年二月,帝親領大軍由鄆州東路北次于魚山。朱瑄覘知,即以兵徑至,且圖速戰。帝整軍出寨,時瑄、瑾已陣于前,須臾,東南風大起,我軍旌旗失次,甚有懼色,即令騎士揚鞭呼嘯。
 俄而西北風驟發,時兩軍皆在草莽中,帝因令縱火。既而煙焰亙天,乘勢以攻賊陣,瑄、瑾大敗。殺萬餘人,餘眾擁入清河,因築京觀于魚山之下,駐軍數日而還。



 二年正月癸亥,遣硃友恭帥師復伐兗,遂塹而圍之。未幾,硃瑄自鄆率步騎援糧欲入于兗,友恭設伏以敗之,盡奪其餉于高吳,因擒蕃將安福順、安福慶。二月己酉,帝領親軍屯于單父,以為友恭之援。四月,濠、壽二州復為楊行密所陷。是時,太原遣將史儼兒、李承嗣以萬騎馳入
 於鄆。朱友恭遂歸于汴。八月,帝領親軍伐鄆,至大仇,遣前軍挑戰,設伏于梁山以待之。既而獲蕃將史完府,奪馬數百匹。硃瑄脫身遁去,復入于鄆。十月,帝駐軍于鄆,齊州刺史朱瓊遣使請降,瓊即瑾之從父兄也。帝因移軍至兗,瓊果來降。未幾,瓊為朱瑾所紿,掠而殺之,帝即以其弟玭為齊州防禦使。十一月,朱瑄復遣將賀瑰、柳存及蕃將何懷寶等萬餘人以襲曹州,庶解兗州之圍也。帝知之,自兗領軍策馬先路至鉅野南,追而敗之,殺戮
 將盡,生擒賀瑰、柳存、何懷寶及賊黨三千餘人。是日申時,狂風暴起,沙塵沸湧,帝曰:「此乃殺人未足耳。」遂下令盡殺所獲囚俘,風亦止焉。翼日,縶賀瑰等以示于兗。帝素知瑰名,乃釋之,惟斬何懷寶于兗城之下,乃班師。十二月,葛從周領兵復伐兗。既至,與硃瑾戰于壘下,殺千餘眾,擒其將孫漢筠已下二十人,遂旋師。



 三年正月,河東李克用既破邠州,欲謀爭霸,乃遣蕃將張污落以萬騎寨于河北之莘縣,聲言欲救兗、鄆。魏博節度使羅宏
 信患之,使來求援。二月,帝領親軍屯于單父,會寒食,帝乃親拜文穆皇帝陵于碭日縣午溝里。四月辛酉,河東泛漲,將壞滑城。帝令決堤岸以分其勢為二河,夾滑城而東,為害滋甚。是月,帝遣許州刺史朱友恭領兵萬人渡淮,以便宜從事。時黃、鄂二州累遣使求援,故有是行。五月,命葛從周統軍屯于洹水,以備蕃軍。六月,李克用帥蕃漢諸軍營于斥丘,遣其男落落將鐵林小兒三千騎薄于洹水,從周與戰,大敗之,生擒落落以獻。克用悲
 駭,請修舊好以贖其子,帝不許,遂執落落送于羅宏信,斬之。越七日,我軍還屯陽留以伐鄆。八月,復壁于洹水。是時,昭宗幸華州,遣使就加帝檢校太師,守中書令。



 四年正月,帝以洹水之師大舉伐鄆。辛卯,營于濟水之次,龐師古令諸將撤木為橋。乙未夜,師古以中軍先濟,聲振于鄆,朱瑄聞之,棄壁夜走。葛從周逐之至中都北,擒瑄並其妻男以獻。尋斬汴橋下。鄆州平,乙亥,帝入于鄆,以硃友裕為鄆州兵馬留後。時帝聞硃瑾與史儼兒在
 豐沛間搜索糧饋,惟留康懷英以守兗州,帝因乘勝遣葛從周以大軍襲兗。懷英聞鄆失守,俄又我軍大至,乃出降;硃瑾、史儼兒遂奔淮南。兗、海、沂、密等州平。乃以葛從周為兗州留後。五月丁丑,朱友恭遣使上言,大破淮寇于武昌,收復黃、鄂二州。八月,陜州節度使王珙遣使來乞師。是時,珙弟珂實為蒲帥,迭相憤怒,日尋干戈,而珙兵寡,故來求援。帝遣張存敬、楊師厚等領兵赴陜,既而與蒲人戰于猗氏,大敗之。九月,帝以兗、鄆既平,將士
 雄勇,遂大舉南征。命龐師古以徐、宿、宋、滑之師直趨清口,葛從周以兗、鄆、曹、濮之眾徑赴安豐。淮人遣硃瑾領兵以拒師古,因決水以浸軍,遂為淮人所敗,師古沒焉。葛從周行及濠梁,聞師古之敗,亦命班師。



\end{pinyinscope}