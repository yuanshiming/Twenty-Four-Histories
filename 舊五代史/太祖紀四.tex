\article{太祖紀四}

\begin{pinyinscope}

 開平二年
 正月癸酉,帝御金祥殿,受宰臣文武百官及諸籓屏陪臣稱賀,諸道貢舉一百五十七人,見于崇元門。封從子友寧為安王,友倫為密王。幽州劉守文進海
 東鷹鶻、蕃馬、氈罽、方物。


自去冬少雪,春深農事方興,久無時雨。兼慮有災疾,帝深軫下民,二月,命庶官遍祀于群望,掩瘞暴露,令近鎮案古法以禳祈,旬日乃雨。是月弒濟陰王。帝以上黨未收,因議撫巡,便往西都赴郊禋之禮。乃下令曉告中外,取三月一日離東京,以宰臣韓建權判建昌宮事,
 \gezhu{
  《五代會要》:十月,以尚書兵部侍郎李皎為建昌宮副使。}
 兵部侍郎姚洎為鹵簿使,開封尹、博王友文為東都留守。辛未,契丹主安巴堅遣使貢良馬。



 三月壬申,帝親統六軍,巡幸澤、潞。
 是日寅時,車駕西幸,宰臣并要切司局皆扈從,晚次中牟。下詔,以去年六月後,昭義行營陣歿都將吏卒死于王事,追念忠赤,乃錄其名氏,各下本軍,令給養妻孥,三年內官給糧賜。丁丑,幸澤州。辛巳,以同州節度使劉知俊為潞州行營招討使。壬午,宴扈駕群臣并勞知俊,賜以金帶、戰袍、寶劍、茶藥。甲申,登東北隅逍遙樓搜閱騎乘,旌甲滿野。丙申,招討使劉知俊上章請車駕還東京,蓋小郡湫隘,非久駐蹕之所。達覽,帝俞其請。以鴻臚卿
 李鷿唐室宗屬,封萊國公,為二王後。有司奏:「萊國公李鷿合留三廟,于西都選地位建立廟宇,以備四仲祀祭,命度支供給,以遵彞典。」



 四月,以吏部侍郎于兢為中書侍郎、平章事,以翰林奉旨學士張策為刑部侍郎、平章事。時帝在澤州,拜二相于行在。丙午,車駕離澤州。丁未,駐蹕于懷州,宴宰臣文武百官。辛亥,至鄭州。壬子,至東京。丙寅,車駕幸繁臺觀稼。鄢陵居人程震以兩歧麥穗并畫圖來進。甲寅,淮寇侵軼潭、岳邊境,欲援朗州,以戰
 艦百餘艘揚帆西上,泊鼎口。湖南馬殷遣水軍都將黃瑀率樓船遮擊之,賊眾沿流宵遁,追至鹿角鎮。詔以戶部尚書致仕裴迪復為右僕射。迪敏事慎言,達吏治,明籌算。帝初建節旄于夷門,迪一謁見如故知,乃辟為從事。自是之後,歷三十年,委四鎮租賦、兵籍、帑廩、官吏、獄訟、賞罰、經費、運漕,事無巨細,皆得專之。帝每出師,即知軍州事,逮于二紀,不出梁之闉閎,甚有裨贊之道。禪代之歲,命為太常卿,屬年已耆耄,視聽昏塞,不任朝謁。遂
 請老,許之。期月復起,師長庶官焉。



 五月丁丑,王師圍潞州將及二年,李進通危在旦夕,不俟攻擊,當自降。太原李存勖以厚幣誘結北蕃諸部,並其境內丁壯,悉驅南征決戰,以救上黨之急。部落帳族,馳馬勵兵,數路齊進,于銅鞮樹寨,旗壘相望。王師敗于潞州。己丑,令下諸州,去年有蝗蟲下子處,蓋前冬無雪,今春亢陽,致為災沴,實傷壟畝。必慮今秋重困稼穡,自知多在荒陂榛蕪之內,所在長吏各須分配地界,精加翦撲,以絕根本。壬辰
 夜,火星犯月。太史奏,災合在荊楚。乃令設武備,寬刑罰,恤人禁暴以禳之。軍前行營都將康懷英、孫海金以下主將四十三人,于右銀臺門進狀待罪。帝以去年發軍之日不利,有違兵法,並釋放,兼各賜分物酒食勞問。制:義昌軍節度使劉守文加中書令,封大彭王;盧龍軍節度使劉守光封河間郡王;許州節度使馮行襲封長樂王。是月癸未,淮賊寇荊州石首縣,襄陽舉舟師沿瀺港襲敗之。



 六月辛亥,以亢陽,慮時政之闕,乃詔曰:「邇者下
 民喪禮,法吏舞文,銓衡既失于選求,州鎮又無其舉刺,風俗未厚,獄訟實繁,職此之由,上遭天譴。」至是,決遣囚徒及戒勵中外。丙寅,月犯角宿。帝以其分野在兗州,乃令長吏治戎事,設武備,省獄訟,恤疲病,祈福禳災,以順天戒。丙辰,邠、岐來寇,雍西編戶困于逃避,且芟害禾稼,結營自固。踰月,同州劉知俊領所部兵擊退,襲至幕谷,大破之,俘斬千計,收其器甲,宋文通僅以身免。詔曰:「敦尚儉素,抑有前聞,斥去浮華,期臻至理。如聞近日貢奉,
 競務奢淫,或奇巧蕩心,或雕鐫溢目,徒殫資用,有費工庸。此後應諸道進獻,不得以金寶裝飾戈甲劍戟。至於鞍勒,不用塗金及雕刻龍鳳。如有此色,所司不得引進。」邕州奏,鏌鎁山僧法通、道璘有道行,冬賜紫衣。是月壬戌,岳州為淮賊所據。帝以此郡五嶺、三湘水陸會合之地,委輸商賈,靡不由斯,遂令荊湘湖南北舉舟師同力致討。王師既集,淮夷毀壁焚郭而遁。



 秋七月甲戌,大霖雨,陂澤泛溢,頗傷稼穡,帝幸右天武軍河亭觀水;幸高
 僧臺閱禁衛六軍。詔曰:「車服以庸,古之制也;貴賤無別,罪莫大焉。應內外將相,許以銀飾鞍勒,其剌史都將內諸司使以降,只取用銅,冀定尊卑,永為條制。仍令執法官糾察之。」《五代會要》載七月敕曰:祭祀之儀,有國大事,如聞官吏慢于恪恭,牲具禮容有異精審,宜令御史臺疏其條件奏聞。癸巳,以禪代已來,思求賢哲,乃下令搜訪牢籠之,期以好爵,待以優榮,各隨其材,咸使登用。宜令所在長吏,切加搜訪,每得其人,則疏姓名以聞。如在下位不能自振者,有司薦導之;如任使後顯立功勞,別加
 遷陟。敕禁屠宰兩月。甲午,以高明門外繁臺為講武臺。是臺西漢梁孝王之時,嘗按歌閱樂于此,當時因名曰吹臺。其後有繁氏居于其側,里人乃以姓呼之,時代綿寢,雖官吏亦從俗焉。帝每登眺,搜乘訓戎,宰臣以是事奏而名之。


八月辛亥,敕應有暴露骸骨,各委差人埋瘞。兩浙錢鏐奏,請重鑄換諸州新印。詔禁戢諸軍節級兵士及供奉官受旨殿直以下各脩禮敬。甲寅,太史奏,壽星見于南方。兩浙錢鏐奏,改管內紫極宮為真聖觀,改
 臨安縣廣義鄉為衣錦鄉。
 \gezhu{
  《十國春秋》、《吳越世家》:八月,梁敕封唐山縣為吳昌縣,唐興縣為天台縣。又敕升杭、越等州為大都督府。復改新城縣曰新登,長城縣曰長興,樂成曰樂清,避梁諱也。}
 甲子夜,東方有大流星,光明燭地,有聲如裂帛。唐州上言,白龍見,圖形以進。



 九月丙子,太原軍出陰地關南牧,寇掠郡縣,晉、絳有備。帝慮諸將玩寇,乃下詔親議巡幸,命有司備行。丁丑,翠華西狩,宰臣、翰林學士、崇政院使、金吾仗及諸司要切官皆扈從,餘文武百官並在東京。壬午,達洛陽。帝御文思殿受朝參,許、汝、孟、懷牧守來朝,澤
 州刺史劉重霸面陳破敵之策。癸未,西幸,宿新安。丙戌,至陜州駐蹕,蒲、雍、同、華牧守皆進鎧甲、騎馬、戈戟、食味、方物。幽州都將康君紹等十人自蕃賊寨內來投,又幽州騎將高彥章八十人騎先在并州,乃于晉州軍前來降。至是到行在,皆賜分物衣服,放歸本道,以示懷服。丁亥,至陳州,賜宴扈從官。戊子,延州賊軍寇上平關,又太原軍攻平陽,烽火羽書,晝夜繼至。乙丑,六軍統軍牛存節、黃文靖各領所部將士赴行在。甲午,太原步騎數萬攻
 逼晉、絳,踰旬不克,知大軍至,乃自焚其寨,至夕而遁。福州貢玳瑁琉璃犀象器,并珍玩、香藥、奇品、海味,色類良多,價累千萬。



 十月己亥,上在陜。兩浙節度使奏,于常州東州鎮殺淮賊萬餘人,獲戰船一百二隻。以行營左廂步軍指揮使賀瑰為左龍虎統軍,以左天武軍夾馬指揮使尹皓為輝州刺史,以右天武都頭韓瑭為神捷指揮使,左天武第三都頭胡賞為右神捷指揮使,仍賜帛有差,以解晉州圍之功也。以尹皓部下五百人為神捷
 軍。乙巳,御內殿,宴宰臣扈從官共四十五人。丙午,御球場殿,宣夾馬都指揮使尹皓、韓瑭以下將士五百人,賜酒食。庚戌,至西都,御文思殿。辛亥,宰臣百僚起居于殿前,遂宣赴內宴,賜方物有差。丁巳,至東都。己未,大明節,諸道節度刺史各進獻鞍馬、銀器、綾帛以祝壽,宰臣百官設齋相國寺。壬戌,御宣和殿,宴宰臣文武百官。



 十一月辛未,御宣和殿,宴宰臣文武百官,以大駕還京故也。庚辰,御宣和殿,宴宰臣文武百官。出開明門,登高僧臺
 閱兵。諸道節度使、刺史各進賀冬田器、鞍馬、綾羅等。戊子,賜文武百官帛。乙未,又宴宰臣文武百官于宣和殿。《通鑑》,癸巳,中書侍郎、同平章事張策以刑部尚書致仕,以左僕射楊涉同平章事。


十二月,立二王三恪。南郊禮儀使狀:「伏以《詩》稱有客,《書》載虞賓,實因禪代之初,必行興繼之命。俾之助祭,式表推恩,兼垂恪敬之文,別示優崇之典。徵于歷代,襲用舊章。謹按唐朝以後魏元氏子孫韓國公為三恪,以周宇文氏子孫為介國公,隋朝楊氏子孫為酅國公,為二王後。今伏以國
 家受禪,封唐朝子孫李鷿為萊國公。今參詳合以介國公為三恪,酅國公、萊國公為二王後。」
 \gezhu{
  《五代會要》:十二月,改左右天武為龍虎軍,左右龍虎為天武軍,左右天威為羽林軍,左右羽林為天威軍,左右英武為神武軍,左右神武為英武軍。前朝置龍虎六軍謂之衛士,至是以天武、神武、英武等六軍易其軍號而任勳舊焉。}
 癸丑,獵畋于含耀門外。


開平三年正月戊辰朔,帝御金祥殿,受宰臣、翰林學士稱賀,文武百官拜表于東上閣門。己巳,奉遷太廟四室神主赴西京,太常儀仗鼓吹導引齋車,文武百官奉辭
 于開明門外。甲戌,發東都,百官扈從,次中牟縣。乙亥,次鄭州。丙子,次汜水縣,河南尹張宗奭、河陽節度使張歸霸並來朝。戊寅,次偃師縣。己卯,備法駕、六軍儀仗入西都。是日,御文明殿受朝賀。詔曰:「近年以來,風俗未泰,兵革且繁,正月燃燈,廢停已久。今屬創開鴻業,初建洛陽,方在上春,務達陽氣,宜以正月十四、十五、十六日夜,開坊市門,一任公私燃燈祈福。」庚寅,親享太廟。辛卯,祀昊天上帝于園丘。是日,降雪盈尺,帝升壇而雪霽。禮畢,御
 五鳳樓,宣制大赦天下。賜南郊行事官禮儀使趙光逢以下分物。甲午,上御文思殿宴群臣,賜金帛有差。丙申,賜文武官帛有差。命宣徽使王殷押絹一萬匹并茵褥幃帟二百六十件賜張宗奭。
 \gezhu{
  《歐陽史》:丙申,群臣上尊號曰睿文聖武廣孝皇帝。}
 改西京貞觀殿為文明殿,含元殿為朝元殿。



 二月,改思政殿為金鑾殿。敕東都曰:「自昇州作府,建邑為都,未廣邦畿,頗虧國體。其以滑州酸棗縣長垣縣、鄭州中牟縣陽武縣、宋州襄邑縣、曹州戴邑縣、許州扶溝縣鄢陵縣、
 陳州太康縣等九縣,宜並割屬開封府,仍昇為畿縣。」《輿地廣記》:朱梁時,楊氏據江、淮,于是吳越錢氏上言,以淮寇未平,恥聞逆姓,請改松陽縣為長松。丁酉,宴群臣于崇勳殿。甲辰,又宴群臣于崇勳殿,蓋籓臣進賀,勉而從之。丙午,宗正寺請修興極、永安、光天、咸寧諸陵,並令添修上下宮殿、栽植松柏。制可。癸亥,敕:「豐沛之基,寢園所在,悽愴動關于情理,充奉自繫于國章。宜設陵臺,兼升縣望。其輝州碭山縣宜為赤縣,仍以本縣令兼四陵臺令。」同州節度使劉知俊奏,延州都指揮使高萬
 興部領節級家累三十八人來降。



 三月,以萬興檢校司徒,為丹、延等州安撫、招討等使。辛未,詔曰:「同州邊隅,繼有士眾歸化。暫思巡撫,兼要指揮,今幸蒲、陜,取九日進發。」甲戌,車駕發西都,百官奉辭于師子門外。丁丑,次陜州。己卯,次解縣。河中節度使、冀王友謙來奉迎。庚辰,至河中府。幸右軍舊杏園講武。丙戌,以朔方節度使、兼中書令韓遜為潁川王。遜本靈州牙校,唐末據本鎮,朝廷因而授以節鉞。



 四月丙申朔,駐蹕河中。壬寅辰時,駕巡
 于朝邑縣界焦黎店,冀王友謙及崇政內諸司使扈從,至申時回。己亥,御前殿,宴宰臣及冀王友謙扈從官。甲寅,宴宰臣及扈從官于內殿。制:易定節度使王處直進封北平王,福建節度使王審知封閩王,廣州節度使劉隱封南平王,同州節度使劉知俊封大彭郡王,山南東道節度使楊師厚封宏農郡王。



 五月乙丑朔,朝,遂命宰臣及文武百官宴于內殿。己卯,車駕至西京。癸未,御崇勳殿,宴宰臣及文武官四品以上。己丑,復御崇勳殿,宴
 宰臣文武官四品以上。升宋州為宣武軍節鎮,仍以亳、輝、潁為屬郡。



 六月庚戌,同州節度使劉知俊據本郡反,制令削奪劉知俊在身官爵,仍徵發諸軍,速令進討。如有軍前將士,懷忠烈以知機,賊內朋徒,憤脅從而識變,便能梟夷逆豎,擒獲凶渠,務立殊功,當行厚賞。活捉得劉知俊者,賞錢一萬貫文,便授忠武軍節度使,並賜莊宅各一所;如活捉得劉知浣者,賞錢一千貫文,便與除刺史,有官者超轉三階,無官者特授兵部尚書;如活捉
 得劉知俊骨肉及近上都將並梟送闕廷者,賞賜有差。辛亥,駕至蒲、陜,文武百官于新安縣奉迎。劉知俊弟內直右保勝指揮使知浣自洛奔至潼關,右龍虎軍十將張溫以上二十二人于潼關擒獲劉知浣,送至行在。敕:「劉知浣,逆黨之中最為頭角;龍虎軍,親兵之內實冠爪牙。昨者攻取潼關,率先用命;尋則擒獲知浣,最上立功。頗壯軍威,將除國難。所懸賞格,便可支分,許賜官階,固須除授。但昨捉獲劉知浣是張溫等二十二人,一時向
 前,共立功效,其賞錢一千貫文數內,一百貫文與最先打倒劉知浣衙官李稠,四十三貫文與十將張溫,二十人各與錢四十二貫八百五十文。立功敕命便授郡府,亦緣同時立功人數不少,所除刺史,難議偏頗。宜令逐月共支給正刺史料錢二百貫文數內,十將張溫一人每月與十貫文,餘二十一人每月每人各分九貫文,仍起七月一日以後支給。人與轉官職,仍勘名銜,分析申奏,當與施行。」是月,知俊奔鳳翔,同州平。



 七月乙丑,敕行
 宮將士陣歿者,咸令所在給槥櫝,津置歸鄉里。戰卒聞之悉感涕。丙寅,命宰臣楊涉赴西都,以孟秋享太廟。改章善門為左、右銀臺門,其左、右銀臺門卻改為左、右興善門。敕:「大內皇墻使諸門,素來未得嚴謹,將令整肅,須示條章。宜令控鶴指揮,應于諸門各添差控鶴官兩人,守帖把門。其諸司使并諸司諸色人,並勒于左、右銀臺門外下馬,不得將領行官一人輒入門裏。其逐日諸道奉進,客省使于千秋門外排當抗,勒控鶴官舁抬至內
 門前,準例令黃門殿直以下舁進,輒不得令諸色一人到千秋門內。其興善門仍令長官關鎖,不用逐日開閉。」是日,又敕:「皇墻大內,本尚深嚴,宮禁諸門,豈宜輕易。未當條制,交下因循,茍出入之無常,且公私之不便。須加鈐轄,用戒門閭。宜令宣徽院使等切準此處分。」進封幽州節度使河間郡王劉守光為燕王。《通鑑》:七月癸酉,帝發陜州。乙亥,至洛陽,寢疾。己丑夕,寢殿棟折。詰旦,召近臣諸王視棟折之迹,帝慘然曰:「幾與卿等不相見。」君臣對泣久之。遂詔有司
 釋放禁人,從八月朔日後減膳,進素食,禁屠宰,避正殿,修佛事,以禳其咎。商州刺史李稠棄郡西奔,本州將吏以都牙校李玫權知州事。



 八月甲午,以秋稼將登,霖雨特甚,命宰臣以下禱于社稷諸祠。詔曰:「封嶽告功,前王重事;祭天肆覲,有國恒規。朕以眇身,恭臨大寶,既功德未敷于天下,而災祥互降于城中。慮于告謝之儀,有缺齋虔之禮,爰修昭報,用契幽通。宜令中書侍郎、平章事于兢往東嶽祭拜禱祀訖聞奏。」又敕:「朕以干戈尚熾,華
 夏未寧,宜循卑菲之言,用致雍熙之化。起八月一日,常朝不御金鑾、崇勳兩殿,只于便殿聽政。」辛亥,制諸郡如有陣歿將士,仰逐都安存家屬,如有弟兄兒侄,便給與衣糧充役。贈故山東道節度使留後王玨太保,贈故同州觀察判官盧匪躬工部尚書。玨,故河陽將,累以軍功為郡守,主留事于襄陽,為小將王求所殺。匪躬嘗為劉知俊判官,知俊反,不偕行,為亂兵所害。敕:「建國之初,用兵未罷,諸道章表,皆繫軍機,不欲滯留,用防緩急。其諸
 道所有軍事申奏,宜令至右銀臺門委客省畫時引進。諸道公事,即依前四方館準例收接。」司天臺奏:「今月二十七日平明前,東南丙上去山高三尺以來,老人星見,測在井宿十一度,其色光明闊大。」敕:「所在長吏放雜差役,兩稅外不得妄有科配。自今後州縣府鎮,凡使命經過,若不執敕文券,並不得妄差人驢及取索一物已上。又,今歲秋田,皆期大稔,仰所在切如條流本分納稅及加耗外,勿令更有科索。切戒所由人更不得于鄉村乞
 托擾人。」



 閏八月,襄陽叛將李洪差小將進表,帝示以含宏,特賜敕書慰諭。又制:「左馮背叛,元惡遁逃,如聞相濟之徒,多是脅從之輩,若能回心向國,轉禍全身,當與加恩,必不問罪。仍令同、華、雍等州切加招諭,如能梟斬溫韜,或以鎮寨歸化,必加厚賞,仍獎官班,兼委本界招復人戶,切加安存。」己卯,幸西苑觀稼。



\end{pinyinscope}