\article{太祖紀六}

\begin{pinyinscope}

 開平四年十月乙亥,東京博王友文入覲,召之也。己卯,以新修天驥院開宴落成,內外並獻馬,而魏博進絹四萬匹為駔價。壬午,以冬設禁軍,幸興安鞠場,召文武百
 官宴。幸開化,大閱軍實。


十一月丁亥朔,幸廣王第作樂。辛卯,宴文武四品已上於宣威殿。庚戌,幸左龍虎軍,宴群臣。甲寅,幸右龍虎軍,宴群臣。戊戌,詔曰:「自朔至今,暴風未息,諒惟不德,致此咎徵。皇天動威,罔敢不懼。宜遍命祈禱,副朕意焉。」差官分往祠所止風。己亥,日南至,帝被袞冕御朝元殿,列細仗,奏樂於庭,群臣稱賀。帝畋於伊水。乙巳,詔曰:「關防者,所以識異服、察異言也。況天下未息,兵民多奸,改形易衣,覘我戎事。比者有諜皆以詐
 敗,而未嘗罪所過地,叛將逃卒竊其妻孥而影附使者,亦未嘗詰其所經。今海內未同,而緩法弛禁,非所以息奸詐、止奔亡也。應在京諸司,不得擅給公驗。如有出外須執憑由者,其司門過所,先須經中書門下點檢,宜委宰臣趙光逢專判出給,俾由顯重,冀絕奸源。仍下兩京、河陽及六軍諸衛、御史臺,各加鈐轄。公私行李,復不得帶挾家口向西。其襄、鄧、鄜、延等道,並同處分。」以寧國軍節度使王景仁充北面行營都招討使,潞州副招討使
 韓勍為副,相州刺史李思安為先鋒使。時鎮州王熔、定州王處直叛,結連晉人,故遣將討之。
 \gezhu{
  《五代會要》:十一月十四日,司天奏:「月蝕,不宜用兵。」時王景仁方總大軍北伐,追之不及。至五年正月二日,果為後唐莊宗大敗於柏鄉。}



 十二月辛酉,宴文武四品已上於宣威殿。親閱禁軍,命格鬥於教馬亭。己巳,詔曰:「滑、宋、輝、亳等州,水澇敗傷,人戶愁嘆,朕為民父母,良用痛心。其令本州分等級賑貸,所在長吏監臨周給,務令存濟。」壬辰,賑貸東都畿內,如宋、滑制。


乾化元年正月丙戌朔,日有食之,帝素服避殿,百官守司以恭天事,明復而止。制曰:「兩漢以來,日食地震,百官各上封事,指陳得失。蓋欲周知時病,盡達物情,用緝國章,以奉天誡。朕每思逆耳,罔忌觸鱗,將洽政經,庶開言路。況茲謫見,當有咎徵。其在列闢群臣,危言正諫,極萬邦之利害,致六合之殷昌。毗予一人,永建皇極。」二日,日旁有祲氣,向背若環耳。崇政使敬翔望之曰:「兵可憂矣!」帝為之旰食。是日,果為晉軍及鎮、定之師所敗,都將十
 餘人被擒,餘眾奔潰。庚寅,制曰:「扈氏不恭,固難去戰,鬼方未服,尚或勞師。其蟻聚餘妖,狐鳴醜類,棄天常而拒命,據地險以偷生,言事討除,將期戡定。問罪止誅於元惡,挺災可憫於遺黎,每念傷痍,良深愧嘆。應天兵所至之地,宜令將帥節級嚴戒軍伍,不得焚燒廬舍,開發丘壟,毀廢農桑,驅掠士女。使其背叛之俗,知予吊伐之心。」又制曰:「戎機方切,國用未殷,養兵須藉於賦租,稅粟尚煩於力役。所在長吏,不得因緣徵發,自務貪求,茍有故
 違,必行重典。立法垂制,詳刑定科,傳之無窮,守而勿失。中書門下所奏新定格式律令,已頒下中外,各委所在長吏,切務遵行。盡革煩苛,皆除枉濫,用副哀矜之旨,無違欽恤之言。」詔徵陜州鎮國軍節度使楊師厚至京,見於崇勛殿。帝指授方略,依前充北面都招討使,恩賚甚厚,使督軍進發。
 \gezhu{
  《五代會要》:二月,晉師侵魏州,楊師厚帥師援邢州,晉人還師。}


二月丙辰朔,帝御文明殿,群臣入閣。以蔡州順化軍指揮使王存儼權知軍州事。蔡人久習叛逆,刺史張慎思又裒斂
 無狀,帝追慎思至京,而久未命代。右廂指揮使劉行琮乘虛作亂,因縱火驅擁,為渡淮計。存儼誅行琮而撫遏其眾,都將鄭遵與其下奉存儼為主,而以眾情馳奏。時東京留守博王友文不先請,遂討其亂。兵至鄢陵,上聞之曰:「誅行琮功也,然存儼方懼,若臨之以兵,蔡必速飛矣!」遂馳使還軍,而擢授存儼,蔡人安之。壬戌,詔曰:「東京舊邦,久不巡幸,宜以今月九日幸東都,扈從文武官委中書門下量閑劇處分。」宰臣上言曰:「龍興天府,久望法
 駕,但陛下始康愈,未宜涉寒,願少留清蹕。」從之。
 \gezhu{
  《五代會要》:二月,敕:「食人之食者,憂人之事,況丞相尊位,參決大政,而堂封未給,且無餐錢,朕甚愧之。宜令食萬錢之半。}
 甲子,幸曜村民舍閱農事。庚午,幸白馬坡。詔金吾大將軍、待制官各奏事。武安軍節度使馬殷進呈虔州刺史盧延昌箋表。虔州本支郡也,兵甚銳,自得韶州益強大,昇為百勝軍使。始洪州之陷,盧光稠願收復使府,立功自效,上因兼授江西觀察留後。光稠卒,復命延昌領州事,方伯亦頗慰薦。楊渭遣人偽署爵秩,延昌佯受官牒,禮
 遣其使,因湖南自表其事曰:「郡小寇迫,欲緩其奸謀,且開導貢路,非敢貳也。」以其偽制來自陳,上覽奏曰:「我方有北事,不可不甚加撫恤。」尋兼授鎮南將軍節度使觀察留後,命使慰勞。
 \gezhu{
  《九國志》:盧延昌歸命于吳,偽乞命于梁。}



 三月辛卯,以久旱,令宰臣分禱靈跡。翼日,大澍雨。丙申,幸甘水亭,召宰臣、翰林學士、尚書侍郎孔續已下八人扈從,宴樂甚歡。戊戌,幸右龍虎軍,召文武官四品已上宴于新殿。甲辰,幸左龍虎軍新殿,宴文武官四品已上。



 四月丁卯,幸龍
 虎門,召宰臣、學士、金吾上將軍、大將軍侍宴廣化寺。壬申,契丹遣使來貢。丁丑,幸宣威殿,宴文武官四品已上及軍使、蕃客。己卯,又幸左龍軍,宴群臣。詔曰:「邠、岐未滅,關、隴多虞,宜擇親賢,總茲戎任。應關西同、雍、華、鄜、延、夏等六道兵馬,並委冀王收掌指揮。凡有抽差,先申西面都招討使,仍別奏聞,庶合機權,以寧邊鄙。」



 五月甲申朔,帝被冕旒御朝元殿視朝,仗衛如式。制:改開平五年為乾化元年,大赦天下。詔方伯州牧,近未加恩者,並遷
 爵秩。復大賚軍旅,普宴于宣威殿,賜帛各有差。制:封延州節度使高萬興為渤海郡王。諸道節度使錢鏐、張宗奭、馬殷、王審知、劉隱各賜一子六品正員官,高季昌賜一子八品正員官,賀德倫賜一子九品正員官。癸巳,觀稼于伊水,登建春門,幸會節坊張宗奭私第,臨亭皋視物色,賞賜甚厚。詔左銀臺門,朝參諸司使庫使已下,不得帶從人入城,親王許一二人執條床手簡,餘悉止門外,闌入者抵律。閽守不禁,與所犯同。先時,門通內無門
 籍,且多勛戚,車騎眾者,尤不敢呵察。至是有以客星凌犯上言者,遂令止隔。清海軍節度使、守侍中、兼中書令劉隱薨,輟朝三日,百僚詣閣門奉慰。



 六月乙卯,命北面都招討使、鎮國軍節度使楊師厚出屯邢、洺。丁巳,鎮、定鈔我湯陰。詔曰:「常山背義,易水效尤,誘其蕃戎,動我邊鄙,南侵相、魏,東出邢、洺。是用遣將徂征,為人除害,但初頒赦令,不欲食言,宥而伐之,諒非獲已。況聞謀始,不自帥臣,致此厲階,並由姦佞。密通人使,潛結沙陀,既懼罪
 誅,乃生離叛。今雖行討伐,已舉師徒,亦開詔諭之門,不阻歸降之路。矧又王熔、處直未曾削爵除名,若翻然改圖,不遠而復,必仍舊貫,當保前功。如有率眾向明,拔州效順,亦行殊賞,冀徇來情,免令受弊于疲民,用示維新于污俗。宜令行營都招討使及陳暉軍前,準此敕文,散加招諭,將安眾懼,特舉明恩。鎮州只罪李宏規一人,其餘一切不問。」詔修天宮佛寺。又,湖南奏:「潭州僧法思、桂州僧歸真並乞賜紫衣。」從之。



 七月,帝不豫,稍厭秋暑。自
 辛丑幸會節坊張宗奭私第,宰臣視事於歸仁亭子,崇政使、內諸司及翰林院並止于河南令廨署,至甲辰,復歸大內。



 八月庚申,幸保寧殿,閱天興控鶴兵事,軍使將校各有賜。癸亥,老人星見。戊辰,幸故上陽宮,至於榆林觀稼。丙子,閱四蕃將軍、屯衛兵士於天津橋,南至龍門廣化寺。戊寅,幸興安鞠場大教閱,帝自指麾,無不踴抃,坐作進退,聲振宮掖。右神武統軍丁審衢對御,以紅帛囊劍擬乘輿物,帝曰:「宿將也。?睜恕之,以劉重霸代其任。



 九月
 辛巳朔,帝御文明殿,群臣入閣,刑法待制官各奏事。己丑,宴群臣于興安殿。庚子,親御六師,次于河陽。甲辰,至于衛州。乙巳,至于宜溝,幸民劉達墅。丙午,至相州,賞左親騎指揮使張仙、右雲騎指揮使宋鐸,嘗身先陷陣,各賜帛。



 十月辛亥朔,駐蹕于相州,宰臣洎文武從官並詣行宮起居。戶部郎中孔昌序齎留都百官冬朔起居表至自西京,諸道節度使、刺史、諸籓府留後,各以冬朔起居表來上。制以郢王友珪充控鶴指揮使,諸軍都虞候
 閻寶為御營使。有司以立冬太廟薦享上言,詔丞相杜曉赴西都攝祭行事。癸丑,閱武于州闉之南樓。左龍驤都教練使鄧季筠、魏博馬軍都指揮使何令稠、右廂馬軍都指揮使陳令勳,以部下馬瘦,並腰斬于軍門。甲寅,將以其夕幸魏縣,命閣門使李郁報宰臣,兼敕內外。是夜,車駕發軔于都署。乙卯,次洹水。丙辰,至魏縣。先鋒將黃文靖伏誅。己未,帝御朝元門,以回鶻、吐蕃二大國首領入覲故也。癸亥,令諸軍指揮使及四蕃將軍賜食于
 行宮之外廡。戊辰,幸邑西之白龍潭以觀魚焉。既而漁人獲巨魚以獻,帝命放之中流,從臣以帝有仁惻之心,皆相顧欣然。是日,名其潭曰萬歲潭。丙子,帝御城東教場閱兵,諸軍都指揮、北面招討使、太尉楊師厚總領鐵馬步甲十萬,廣亙十數里陳焉。士卒之雄銳,部隊之嚴肅,旌旗之雜遝,戈甲之照耀,屹若山岳,勢動天地,帝甚悅焉。即令丞相洎文武從臣列侍賜食,逮晚方歸。


十一月辛巳朔,上駐蹕魏縣,從官自丞相而下並詣行宮起
 居,留都文武百官及諸道節度使、防禦使、刺史、諸籓府留後,各奉表起居。壬午,帝以邊事稍息,宣命還京師。
 \gezhu{
  《通鑒》:帝以夾寨、柏鄉屢失利,故力疾北巡,思一雪其恥,意鬱鬱,多躁忿,功臣宿將往往以小過被誅,眾心益懼。既而晉、趙兵不出。十一月壬午,帝南還。}
 車駕發自行闕,夕次洹水縣。癸未,至內黃縣。甲申,至黎陽縣。乙酉,命從官丞相而下宴于行次。丁亥,次衛州。戊子晨,次新鄉,夕止獲嘉。己丑,次武陟。庚寅,次溫縣。延州節度使高萬興奏,當軍都指揮使高萬金統領兵士,今月五日收鹽州,偽刺史高行存泥首
 來降。丞相及文武百官各上表稱賀。辛卯,次孟州,命散騎常侍孫騭、右諫議大夫張衍、光祿卿李翼各齎香、祝版,告祭于孟津之望祠。留都文武官左僕射楊涉洎孟州守李周彞等皆匍匐東郊迎拜,其文武官並令先還。壬辰,詰旦離孟州,晚至都。宣宰臣各赴望祠禱雨。故事,皆以兩省無功職事為之,帝憂民重農,尤以足食足兵為念,爰自御極,每愆陽積陰,多命丞相躬其事。辛丑,大雨雪,宰臣及文武師長各奉表賀焉。


十二月,詔以時
 雪稍愆,命丞相及三省官各詣望祠祈禱。癸酉,臘假,詔諸王與河南尹、左右金吾、六統軍等較獵于近苑。命大理卿王鄯使于安南,左散騎常侍吳藹使于朗州,皆以旌節官誥錫之也。又命將作少監姜宏道為朗州旌節官使副。
 \gezhu{
  《五代會要》:舊制,巡撫、黜陟、冊命、弔贈、入蕃等使,選朝臣為之,其宣慰、加官、送旌節,即以中官為之,今以三品送旌節,新例也。}
 延州節度使高萬興奏,領軍于邠州界蒿子谷韋家寨,殺戮寧、慶兩州賊軍約二千餘人,並生擒都頭指揮使及奪馬器甲等事。其入奏軍將使宣召
 赴內殿賜對,以銀器彩物錫之,宰臣及文武官各奉表賀。是月,魏博節度上言,于涇縣北戮殺鎮州王熔兵士七千餘人,奪馬二千餘匹,戈甲未知其數,並擒都將以下四十餘人。兩浙進大方茶二萬斤,琢畫宮衣五百副。廣州貢犀象奇珍及金銀等,其估數千萬。安南兩使留後曲美,
 \gezhu{
  《通鑑》:十二月戊午,以靜海、曲美為節度使。}
 進筒中蕉五百匹,龍腦、鬱金各五瓶,他海貨等有差。又進南蠻通好金器六物、銀器十二并乾陁綾花繓棖越𣭻等雜織奇巧者各三十件。
 福建進戶部所支榷課葛三萬五千匹。



\end{pinyinscope}