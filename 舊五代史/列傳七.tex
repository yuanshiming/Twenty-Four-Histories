\article{列傳七}

\begin{pinyinscope}

 康君立,蔚州興唐人,世為邊豪。乾符中,為雲州牙校,事防禦使段文楚。時群盜起河南,天下將亂,代北仍歲阻饑,諸部豪傑,咸有嘯聚邀功之志。會文楚稍削軍人儲給,戍兵咨怨。君立與薛鐵山、程懷信、王行審、李存璋等謀曰:「段公懦人,難與共事。方今四方雲擾,武威不振,丈夫不能於此時立功立事,非人豪也。吾等雖
 權系部眾,然以雄勁聞於時者,莫若沙陀部,復又李振武父子勇冠諸軍,吾等合勢推之,則代北之地,旬月可定,功名富貴,事無不濟也。」君立等乃夜謁武皇言曰:「方今天下大亂,天子付將臣以邊事,歲偶饑荒,便削儲給,我等邊人,焉能守死!公家父子,素以威惠及五部,當共除虐帥,以謝邊人,孰敢異議者!」武
 皇
 曰:「
 明天子在上,舉事當有朝典,公等勿輕議。予家尊遠在振武,萬一相迫,俟予稟命。」君立等曰;「事機已洩,遲則變生,曷俟千里咨稟!」眾因聚噪,擁武皇。比及雲州,眾且萬人,師營鬥雞臺,城中械文楚以應武皇之軍。既收城,推武皇為大同軍防御留後。眾狀以聞,朝廷不悅,詔徵兵來討。俄而獻祖失振武,武皇失雲州。朝廷命招討使李鈞、幽州李可舉加兵于武皇,攻武皇於蔚州。君立從擊可舉之師屢捷,及獻祖入達
 靼,君立保感義軍,武皇授雁門節度,以君立為左都押牙,從入關,逐黃孽,收長安。武皇還鎮太原,授檢校工部尚書、先鋒軍使,



 文德初,李罕之既失河陽,來歸於武皇,且求援焉,乃以君立充南面招討使權出乎道」,有道法合流傾向。,李存孝副之,帥師二萬,助罕之攻取河陽。三月,與汴將丁會、牛存節戰于沇河。臨陣之次,騎將安休休叛入汴軍,君立引退。八月,授汾州刺史。大順元年,潞州小校安居受反,武皇遣君立討平之,授檢校左僕射、昭義節度使。自武皇之師連
 歲略地於邢、洺,攻孟方立,君立常率澤潞之師以為掎角。



 景福初,檢校司徒,食邑千戶。二年,李存孝據邢州叛,武皇命君立討之,以功加檢校太保。乾寧初,存孝平,班師。存孝既死,武皇深惜之,怒諸將無解慍者。初,李存信與存孝不葉,屢相傾奪,而君立素與存信善。九月,君立至太原,武皇會諸將酒博,因語及存孝事,流涕不已。時君立以一言忤旨,武皇賜鴆而殂,時年四十八。明宗即位,以念舊之故,詔贈太傅。



 薛志勤,蔚州奉誠人,小字鐵山。初為獻祖帳中親信,乾符中,與康君立共推武皇定雲州,以功授右牙都校;從入達靼,武皇授節鴈門,志勤領代北軍使;從入關,收京城,以功授檢校工部尚書、河東右都押牙、先鋒右軍使。從武皇救陳、許,平黃巢。武皇遇難于上源驛,汴將楊彥洪連車樹柵,遮絕巷陌,時騎從皆醉,宴席既闌,汴軍四面攻傳舍。志勤虓勇冠絕,復酒膽激壯,因獨登驛樓大呼曰:「朱僕射負恩無行,邀我司空圖之,吾三百人足以
 濟事!」因彎弧發射,矢無虛發,汴人斃者數十。志勤私謂武皇曰:「事急矣,如至五鼓,吾屬無遺類矣,可速行!」因扶武皇而去。雷雨暴猛,汴人扼橋,志勤以其屬血戰擊敗之,得侍武皇還營,由是恩顧益厚。大順初,張浚以天子之師來侵太原。十月,大軍入陰地,志勤與李承嗣率騎三千抗之,敗韓建之軍于蒙坑,進收晉、絳,以功授忻州刺史。二年,從討鎮州,收天長、臨城,志勤皆先登陷陣,勇敢無前。王暉據雲州叛,討平之,以志勤為大同軍防禦
 使、檢校司空。乾寧初,代康君立為昭義節度使。光化元年十二月,以疾卒于潞,時年六十二。



 史建瑭,字國寶。父敬思,雁門人,仕郡至牙校。武皇節制雁門,敬思為九府都督,從入關,定京師。及鎮太原,為裨將。中和四年,從援陳、許,為前鋒,敗黃巢於汴上,追賊至徐、兗,常將騎挺身酣戰,勇冠諸軍。是時,天下之師雲集,軍中無不推伏。六月,衛從武皇入汴州,舍於上源驛,是夕,為汴人所攻,敬思方大醉,因蹶然而興,操弓與汴人
 鬥,矢不虛發,汴人死者數百。夜分冒雨方達汴橋,左右扶武皇決圍而去。敬思後拒,血戰而歿。武皇還營,知失敬思,流涕久之。



 建瑭以父廕少仕軍門。光化中,典昭德軍。與李嗣昭攻汾州,率先登城,擒叛將李瑭以獻,授檢校工部尚書。李思安之圍上黨也,建瑭為前鋒,與總管周德威赴援。時汴人夾城深固,援路斷絕,建瑭日引精騎,設伏擒生,夜犯汴營,驅斬千計,敵人不敢芻牧。汴將王景仁營於柏鄉,建瑭與周德威先出井陘。高邑之戰,
 日已晡晚,汴軍有歸志,建瑭督部落精騎先陷其陣,夾攻魏、滑之間,遂長驅追擊。夜入柏鄉,俘斬數千計,論功加檢校左僕射。師還,留戍趙州。汴將氏延賞數犯趙之南鄙,建瑭設伏柏鄉,獲延賞,獻之。



 九年,梁祖親攻蓚縣。時王師併攻幽州,聲言汴軍五十萬,將寇鎮、定。都將符存審謂建瑭曰:「梁軍倘以五十萬來,我等何以待之?」裨將趙行實曰:「走入土門為上策。」存審曰:「事未可知,但老賊在東,別將西來,尚可徐圖。」不旬日,楊師厚圍棗強,賀
 德倫圍蓚縣,梁祖自至,攻城甚急。存審曰:「吾王方事北面,南鄙之事,付我等數人。今西道無兵,坐滋賊勢。何以為謀。老賊若不下蓚、阜,必西攻深、冀,與公等料閱騎軍,偵視賊勢。」乃選精騎八百趨信都,存審扼下博橋,建瑭與李嗣肱分道擒生。建瑭乃分麾下三百騎為五軍,自將一軍深入,各命俘掠梁軍之芻牧者還,會下博橋。翼日,諸軍皆至,獲芻牧者數百人,聚而殺之,緩數十人,令其逸去,各曰:「沙陀軍大至矣!」梁軍震恐。明日,建瑭、嗣肱
 為梁軍服色,與芻牧者相雜,晡晚,及賀德倫寨門,殺守門者,縱火大噪,俘斬而去。是夜,梁祖燒營而遁,比至貝州,迷失道咯,委棄兵仗,不可勝計。



 十二年,魏博歸款,建瑭與符存審前軍屯魏縣。十三年,敗劉鄩于元城,收澶州,以建瑭為刺史、檢校司空、外衙騎軍都將。尋歷貝、相二州刺史,屯於德勝。十八年,與閻寶討張文禮,為馬軍都將。八月,收趙州,獲刺史王金延。進逼鎮州,為流矢所中,卒於軍,時年四十六。



 李承嗣,代州雁門人。父佐方。承嗣少仕郡,補右職。中和二年,從武皇討賊關輔,為前鋒。王師之攻華陰書注釋》等。,黃巢令偽客省使王汀會軍機於黃揆,承嗣擒之以獻。賊平,以功授汾州司馬,改榆次鎮將。光啟初,從討蔡賊于陳、許。上源之難,遣承嗣奉表行在,陳訴其事。觀軍容田令孜館而慰諭,令達情於武皇,姑務葉和,仍授以左散騎常侍。朱玫之亂,遣承嗣率軍萬人援鄜州,至渭橋迎扈車駕。王行瑜既殺朱玫,承嗣會鄜、夏之師入定京城,獲偽
 相裴徹、鄭昌圖,函送朱玫、襄王首獻於行在。駕還宮,賜號迎鑾功臣、檢校工部尚書、守嵐州刺史,賜犒軍錢二萬貫。



 時車駕初還,三輔多盜,承嗣按兵警御,輦轂乂安。及還屯於鄜,留別將馬嘉福五百騎宿衛。孟方立之襲遼州也,武皇遣承嗣設伏於榆社以待之。邢人既至,承嗣發伏,擊其歸兵,大敗之,獲其將奚忠信,以功授洺州刺史。及張浚之加兵於太原也,時鳳翔軍營霍邑,承嗣率一軍攻之,岐人夜遁,追擊至趙城,合大軍攻平陽,旬
 有三日而拔。師還,改教練使、檢校司徒。


乾寧二年,兗、鄆為汴人所攻,勢漸危蹙,遣使乞師於武皇;武皇遣承嗣率三千騎假道于魏,渡河援之。時李存信屯于莘縣,既而羅宏信背盟,掩擊王師,因茲隔絕。及瑄、瑾失守,承嗣與硃瑾、史儼同入淮南。承嗣、史儼皆驍將也,淮人得之,軍聲大振。
 \gezhu{
  《十國春秋·吳列傳》:太祖署為淮南行軍副使。}
 武皇深惜之,如失左右手,乃遣趙岳間道使于淮南,請歸承嗣等;楊行密許之,遣使陳令存修好于武皇。其年九月,汴將龐師古、葛從周出
 師,將收淮南,朱瑾率淮南軍三萬,與承嗣設伏于清口,大敗汴人,生獲龐師古。行密嘉其雄才,留而不遣,仍奏授檢校太尉,領鎮海軍節度使。天綁九年,淮人聞莊宗有柏鄉之捷,乃以承嗣為楚州節度使,以張犄角。十七年七月,卒於楚州,時年五十五。


史儼,代州雁門人。以便騎射給事於武皇。為帳中親將,驍果絕眾,善擒生設伏,望塵揣敵,所向皆捷。自武皇入定三輔,誅黃巢,每出師皆從。乾寧中,從討王行瑜,師次
 渭北,遣儼率五百騎護駕石門。時京城大擾,士庶多散布南山。儼分騎警衛,比駕還京,盜賊不作,以功授檢校右散騎常侍,屯於三橋者累月,昭宗寵錫優異。明年,與李承嗣率騎渡河援兗、鄆。時汴軍雄盛,自青、徐、兗、鄆,柵壘相望。儼與騎將安福順等,每以數千騎直犯營壘,左俘右斬,汴軍為之披靡。及硃瑾失守,與李承嗣等奔淮南。淮人比善水軍,不閑騎射,既得儼等,軍聲大振,尋挫汴軍于清口。其後併鐘傳,擒杜洪,削錢鏐,成行密之
 霸跡者,皆儼與承嗣之力也。淮人館遇甚厚,妻孥第舍必推其甲,故儼等盡其死力。
 \gezhu{
  《十國春秋》云:儼累官滁州刺史。}
 天祐十三年,卒于廣陵。



 蓋寓,蔚州人。祖祚,父慶,世為州之牙將。武皇起雲中,寓與康君立等推轂佐佑之,因為腹心。武皇節制雁門,署職為都押牙,領嵐州刺史。洎移鎮太原,改左都押牙、檢校左僕射。武皇與之決事,言無不從,凡出征伐,靡不衛從。《通鑒》:光啟二年,駕幸興元,大將蓋寓說克用曰:「鑾輿播遷,天下皆歸咎于我,今不誅硃玫,黜李煴,無以自湔洗。」克用從之。又,《通鑒考異》引《紀年錄》云:偽使至太原,太祖詰其事狀,曰:「皆硃玫所為。」將斬之以徇,大將蓋寓
 等言云云。太祖燔偽詔,械其使,馳檄喻諸鎮曰:「今月二十日,得襄王偽詔及硃玫文字,云:『田令孜脅遷鑾駕,播越梁、洋,行至半塗,六軍變擾,遂至蒼皇而晏駕,不知弒逆者何人。永念丕基不可無主,昨四鎮籓後推朕纂承,已於正殿受冊畢,改元大赦者。』李煴出自贅疣,名污籓邸,智昏菽麥,識昧機權。李符擄之以塞辭,硃玫賣之以為利。呂不韋之奇貨,可見姦邪;蕭世誠之土囊,期於匪夕。近者,當道徑差健步,奉表起居,行朝現駐巴、梁,宿衛比無騷動。而朱玫脅其孤騃,自號臺衡,敢首亂階,明言晏駕,熒惑籓鎮,凌弱廟朝」雲云。乾寧二年,從入關討王行瑜,特授檢校太保、開國侯,令邑一千戶,領容管觀察經略使。光化初,車駕還京,授檢校太傅,封成陽郡公。



 寓性通黠,多智數,善揣人主情。武皇性嚴急,左
 右難事,無委遇者,小有違忤,即置於法,惟寓承顏希旨,規其趨向,婉辭順意,以盡參裨。武皇或暴怒將吏,事將不測,寓欲救止,必佯佐其怒以責之,武皇怡然釋之。有所諫諍,必徵近事以為喻,自武皇鎮撫太原,最推親信,中外將吏,無不景附,朝廷籓鄰,信使結託,先及武皇,次入寓門;既總軍中大柄,其名振主,梁祖亦使姦人離間,暴揚於天下,言蓋寓已代李,聞者寒心,武皇略無疑間。初,武皇既平王行瑜,還師渭北,暴雨六十日,諸將或請入
 覲,且云:「天顏咫尺,安得不行覲禮。」武皇意未決,寓白曰:「車駕自石門還京,寢未安席,比為行瑜兄弟驚駭乘輿,今京師未寧,奸宄流議,大王移兵渡渭,必恐復動宸情。君臣始終,不必朝覲,但歸籓守,姑務勤王,是忠臣之道也。」武皇笑曰:「蓋寓尚阻吾入覲,況天下人哉!」即日班師。天祐二年三月,寓病篤,武皇日幸其第,手賜藥餌。初,寓家每事珍膳,窮極海陸,精於府饌,武后非寓家年獻不食,每幸寓第,其往如歸,恩寵之洽,時我與比;及其卒也,哭
 之彌慟。莊宗即位,追贈太師。



 伊廣,字言,原本闕一字。元和中右僕射慎之後。廣,中和末除授忻州刺史。遇天下大亂,乃委質於武皇。廣襟情灑落,善占對,累歷右職,授汾州刺史。時武皇主盟,諸侯景附,軍機締結,聘遺旁午,廣奉使稱旨,累遷至檢校司徒。乾寧四年,從征劉仁恭,武皇之師不利于成安寨,廣歿于賊。



 有女為莊宗淑妃。子承俊,歷貝、遼二州刺史。



 李承勳者,與廣同為牙將,善于奉使,名聞軍中。承勛累
 遷至太原少尹。劉守光之僭號也,莊宗遣承勳往使,問其釁端。承勳至幽州,見守光,如籓方聘問之禮。謁者曰:「燕王為帝矣,可行朝禮。」承勳曰:「吾大國使人,太原亞尹,是唐帝除授,燕主自可臣其部人,安可臣我哉!」守光聞之不悅,拘留於獄,數日而出,詰之曰:「臣我乎?」承勳曰:「燕君能臣我王,則我臣之;吾有死而已,安敢辱命!」會王師討守光,承勳竟歿于燕。



 史敬鎔,太原人。事武皇為帳中綱紀,甚親任之。莊宗初
 嗣晉王位,李克寧陰構異圖,將害莊宗,事發有日矣,克寧密引敬鎔,以邪謀諭之。既而敬鎔白,貞簡太后惶駭,召張承業、李存璋等圖之。克寧等伏誅,以功累歷郡。同光初,為華州節度使,移鎮安州。天成中,入為金吾上將軍。期年,復授鄧州,至鎮數月卒。贈太尉。



\end{pinyinscope}