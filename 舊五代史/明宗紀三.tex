\article{明宗紀三}

\begin{pinyinscope}

 天成元年秋八月乙酉朔,日有食之。有司上言:「莊宗廟室酌獻,請奏武成之舞。」從之。鄆州節度使霍彥威移鎮青州。丁亥,莊宗神主祔廟,有司請祧懿祖室,從之。詔:「陵
 州、合州長流百姓豆盧革、韋說等,可並自長流後,縱逢恩赦,不在原宥之限。豆盧升、韋濤仍削除自前所受官秩。」壬辰,以久雨,放百僚朝參,詔天下疏理繫囚。甲午,汴州奏,舊管曹州乞卻歸當道,從之。是日,詔曰:「承前使府奏請判官,率皆隨府除移停罷。近年流例,有異前規,使府雖已除移,判官元安舊職。起今後若是朝廷除授者,即不計使府除移,如是使府奏請,即皆隨府移罷。舊例籓侯帶平章事者,所奏請判官,殿中已上許奏緋,中丞
 已上許奏紫,今不帶平章事亦許同帶平章事例處分。如防禦、團練使奏請判官,員外郎已下不在奏緋之限。其所奏判官、州縣官,並須將歷任告身隨奏至京。如未有官,假稱試攝,亦奏狀內分明署出。如籓鎮留後、權知軍州事,並不在奏請判官之限。如刺史要奏州縣官,須申本道,請發表章,不得自奏。近日州使奏請從事,本無官緒,妄結虛銜,不計職位高卑,多是請兼朱紫,不惟紊亂,實啟撓求。宜令諸道州府,切準敕命處分。」


丁酉,內出
 象笏三十四面,賜百官之無笏者。己亥,帝御文明殿,百官入閣姚
 \gezhu{
  今屬浙江}
 人。曾築室於故鄉陽明洞,世稱陽明先生。官,月望如月朔之儀,從新例也。荊南高季興上言,峽內三州,請朝廷不除刺史。幽州奏,契丹寇邊,詔齊州防禦使安審通率師禦之。辛丑,以前青州節度使符習為鄆州節度使,以前華州節度使史敬熔為安州節度使。乙巳,禁熔錢為器,仍估定生銅器價斤二百,熟銅器斤四百,如違省價買賣者,以盜鑄錢論。丁未,樞密使院條奏:「諸道節度使、刺史內,有不守詔條,公行科斂,須行
 止絕。州使所納軍糧,不得更邀加耗。節度使、刺史所置牙隊,許於軍都內抽取,便給省司衣糧,況人數已多,訪問尚有招致。諸邑人多有抵罪亡命,便於州府投名為使下元隨,邀求職務,凌壓平人;及有力戶人,于諸處行賂,希求事務。亦有州使安稱修葺城池廨宇,科賦於人,及營私宅,諸縣鎮所受州使文符,如涉科斂人戶,不得稟受。州府不得賒買行人物色,兼行科率。已前條件,州使如敢犯違,許人陳告,勘詰不虛,量行獎賞。宜令三京、
 諸道州府,準此處分。」



 新授青州節度使霍彥威奏,處斬新登州刺史王公儼,及同謀拒命指揮使李謹、王居厚等八人訖。初,同光中,符習為青州節度使,宦官楊希望為監軍,專制軍政。趙在禮之據魏州,習奉詔以本軍進討,俄而帝為亂軍所劫,習即罷歸。希望遣兵邀之,習懼而還。至滑州,帝遣人招之,習至,乃從帝入汴。希望聞魏軍亂,遣兵圍守習家,欲盡殺之。公儼素受希望獎愛,謂希望曰:「內侍宜分腹心之兵,監四面守陴者,則誰敢異
 圖。」希望從之。公儼乘其無備,圍希望之第,擒而殺之。公儼遂與州將李謹等謀據州城,以邀符節,即令軍府飛章留己,兼揚言符習在鎮,人不便其政,帝乃除公儼為登州刺史。公儼不時赴任,即以霍彥威代符習,聚兵淄州,以圖進取。彥威至淄州,會詔使至青州告諭,公儼即赴所任。彥威懲其初心,遣人擒公儼於北海縣,與同黨斬於州東。《通鑒》:彥威聚兵淄州,以圖進取,公儼懼。乙未,始之官。丁酉,彥威至青州,追擒之。有司上言:「莊宗祔廟,懿祖祧遷,準例舍故而諱新,懿祖例不
 諱,忌日不行香。」從之。壬子,襄州節度使劉訓加檢校太傅,以偽蜀右僕射、中書侍郎、平章事、趙國公張格為太子賓客,充三司副使,從任圜請也。


九月乙卯朔,詔汴州扶溝縣復隸許州。以前絳州刺史婁繼英為冀州刺史,充北面水陸轉運制置使。己未,幸至德宮,遂幸前隰州刺史袁建豐之第。帝嘗為太原內牙親將,建豐為副,至是建豐風疾沈廢,故親幸其第以撫之。庚申,以都官郎中庾傳美充三州搜訪圖籍使。傳美為蜀王衍之舊僚,
 家在成都,便於歸計,且言成都具有本朝實錄,及傳美使回,所得才九朝實錄及殘缺雜舊而已。癸亥,應聖節,百僚于敬愛寺設齋,召緇黃之眾於中興殿講論,眾近例也。戊辰,以偽蜀檢校太師、兼中書令、右金吾街使張貽範為兵部尚書致仕。都官員外郎於鄴奏請指揮不得書契券輒賣良人,從之。癸酉,天策上將軍、湖南節度使、開府儀同三司、守太師、兼尚書令、楚王馬殷加檢校太師、守尚書令。兩浙節度留後、靜海軍節度、嶺南西道
 觀察處置等使、檢校太尉、兼中書令錢元瓘加食邑。中吳建武等軍節度、嶺南東道觀察處置等使、檢校太尉、兼中書令錢元璙加開府階,進食邑。甲戌,以前代州刺史馬溉為左衛上將軍致仕。己卯,以光祿卿羅周敬為右金吾衛大將軍,充街使。辛巳,以前復州刺史袁
 \gezhu{
  山義}
 為唐州刺史。詔曰:「鳳翔節度使李嚴,世聯宗屬,任重籓宣,慶善有稱,忠勤顯著。既在維城之列,宜新定體之文。是降寵光,以隆惇敘,俾煥成家之美,貴崇猶子之親。宜于
 本名上加『從』字。」癸未,文武百僚至張全義私第柩前立班辭,以來月二日葬故也。



 冬十月甲申朔,詔賜文武百僚冬服綿帛有差。近例,十月初寒之始,天子賜近侍執政大臣冬服。帝顧謂判三司任圜曰:「百僚散未?」圜奏曰:「臣聞本朝給春冬服,遍及百僚。喪亂已來,急於軍旅,人君所賜,未能周給。今止近臣而已,外臣無所賜。」帝曰:「外臣亦吾臣也,卿宜計度。」圜遂與安重誨據品秩之差,以定春冬之賜,其後遂以為常。右拾遺曹琮上疏,內一件:「
 百僚朔望入閣,及五日內殿起居,請許三署寺監官輪次轉奏封事。」從之。刑部員外郎孔莊上言:「自兵興以來,法制不一,諸道州縣常行枷杖,多不依格律,請以舊制曉諭,改而正之。」丙戌,吏部侍郎盧文紀上言:「請內外文武臣僚,每歲有司明定考校,將相乞回御筆,以行黜陟,疏下中書門下商量,宰臣奏請施行。」從之。丁亥,雲南巂州山後兩林百蠻都鬼主、右武衛大將軍李卑晚遣大鬼主傳能、何華等來朝貢,帝御文明殿對之,百僚稱賀。
 庚寅,以客省使李嚴領泗州防御使,以河中節度副使李鈴為太子賓客。壬辰,邠州節度使毛璋移鎮潞州。巴州進嘉禾合穗。甲午,以前隰州刺史袁建豐遙領洪州節度使。



 庚子,幽州奏,契丹平州守將偽署幽州節度使盧文進,率戶口歸順,百僚稱賀。辛丑,契丹遣使來告哀,言國主安巴堅以今年七月二十七日卒。詔曰:「朕近纘皇圖,恭修帝道,務安夷夏,貴洽雍熙。契丹王世預歡盟,禮交聘問,遽聞凶訃,倍軫悲懷,可輟今月十九日朝參。」
 丙午,以巂州山後兩林、百蠻都鬼主李卑晚為寧遠將軍,大渡河山前仰川六姓都鬼主、懷安郡王勿鄧摽莎為定遠將軍。丁未,幽州奏,盧文進所率降戶孳畜人口在平州西,首尾約七十里。庚戌,以吏部侍郎盧文紀為御史中丞,時御史大夫李琪三上表求解任故也。以兵部侍郎劉岳為吏部侍郎,以戶部侍郎、充端明殿學士馮道為兵部侍郎,以中書舍人、充端明殿學士趙鳳為戶部侍郎,並依前充職。壬子,靜江軍節度使、桂州管內
 觀察使、檢校太師、兼中書令、扶風郡王馬賓加食邑實封,澧郎觀察使、檢校太傅、兼侍中馬希振加檢校太尉。盧文進至幽州,遣軍吏奉表來上。



 十一月戊午,以滄州留後王景戡為邢州節度使。青州奏,得登州狀申,契丹先攻逼渤海國,自安巴堅身死,雖已抽退,尚留兵馬在渤海扶餘城,今渤海王弟領兵馬攻圍扶餘城內契丹次。己未,以翰林學士、尚書、戶部郎中、知制誥劉句為中書舍人充職。辛酉,以前秘書少監溫輦為太子詹事。壬
 戌,以前房州刺史朱罕為潁州團練使。是日,詔曰:「應今日已前修蓋得寺院,無令毀廢;自此已後,不得輒有建造。如要願在僧門,並須官壇受戒,不得衷私剃度。」癸丑,日南至,帝御文明殿受朝賀,仗衛如式。禮部侍郎裴皞上言:「諸州刺史經三考方請替移。」詔曰:「有政聲者就加恩澤,無課最者即便替移。」密州獻芝草。庚午,河陽節度使夏魯奇移鎮許州,留後梁漢顒為邠州節度使。淮南楊溥遣使貢獻,賀登極。乙亥,以前振武留後張溫為利
 州昭武軍留後,以果州刺史孫鐸為漢州刺史,充西川馬步軍都指揮使。壬午,靜海軍節度、安南管內觀察等使、檢校太尉、兼侍中錢元球加開府階,進食邑。癸未,鎮州奏,準詔盧文進所率歸業戶口,蠲放租稅三年,仍每口給糧五斗。



 十二月戊子,盧文進及將吏四百人見,賜鞍馬、玉帶、衣被、器玩、錢帛有差。詔曰:「朕中興寶祚,復正皇綱。萬國駢羅,俱在照臨之內;八紘遼夐,咸居覆載之間。矧彼雲南,素歸正朔,洎平偽蜀,思錫舊恩,于乃眷以
 雖深,欲霈覃而未暇。百蠻都首領李卑晚、六姓蠻都首領勿鄧摽莎等,天資智勇,世稟忠勤,梯航之道路纔通,琛贐之貢輸已至。率其種落,竭乃悃誠,備傾向化之心,深獎來庭之意。今則各頒國寵,別進王封。其巂州刺史李及、大鬼主離吠等,或遙貢表函,或躬趨朝闕,亦宜特授官資,各遷階秩。勉敦信義,無墜冊書,示爾金石之堅,保我山河之誓。欽承休命,永保厥終。」壬辰,帝狩於近郊,臘故也。甲午,以契丹盧龍軍節度使盧文進為檢校太
 尉、同平章事,充滑州節度使。戊戌,詔嚴禁金錢。庚子,皇第二子金紫光祿大夫、檢校司徒從榮可檢校太保、同平章事、天雄軍節度使、鄴都留守。以武安軍馬步軍都指揮使馬希範為澧州刺史,鐵林都知兵馬馬希杲為衡州刺史。壬寅,潁州刺史孫岳加檢校太保,獎能政也。



 丙午,中書門下奏:「故事,籓鎮節度、觀察使帶平章事,于都堂上事刊石記壁,合納禮錢三十貫,以充中書及兩省公使。今欲各納禮錢五百千,于中書立石亭子,鐫勒
 宰臣使相官氏、授上年月,餘充修葺中書及兩省公署部堂什物。」從之。



 庚戌,御史臺奏:「京城坊市士庶工商之家,有婢僕自經投井,非理物故者。近者已來,凡是死亡,皆是臺司左右巡舉勘檢,施行已久,仍恐所差人吏及街市胥徒,同於民家,因事邀脅。臣詢訪故事,凡京城民庶之家,死喪委府縣檢舉,軍家委軍巡,商旅委戶部。然諸司檢舉後,具事由申臺,其間或枉濫情故,臺司訪聞,即行舉勘。如是文武兩班官吏之家,即是臺司檢舉。臣
 請自今已後,並準故事施行者。」詔曰:「今後文武兩班及諸道商旅,凡有喪亡,即準臺司所奏施行。其坊市民庶軍士之家,凡死喪及婢僕非理物故,依臺司奏,委府縣、軍巡同檢舉,仍不得縱其吏卒,于物故之家妄有邀脅。或恐暑月屍柩難停,若待申聞檢舉,縱無邀脅,亦須經時日。今後仰本家喚四鄰檢察,若無他故,逐便葬埋。如後別聞枉濫,妄有保證,官中訪知,勘詰不虛,本戶鄰保並行科罪。如聞諸道州府,坊市死喪,取分巡院檢舉,頗
 致淹停,人多流怨,亦仰約京城事例處分。」



\end{pinyinscope}