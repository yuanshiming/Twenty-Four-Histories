\article{明宗紀一}

\begin{pinyinscope}

 明宗聖德和武欽孝皇帝,諱亶,初名嗣源,及即位,改今諱,代北人也。世事武皇,及其錫姓也,遂編於屬籍。四代祖諱聿,皇贈麟州刺史。天成初,追尊為孝恭皇帝,廟號
 惠祖,陵曰遂陵;高祖妣衛國夫人崔氏,追謚為孝恭昭皇后。三代祖諱教,皇贈朔州刺史,追尊為孝質皇帝,廟號毅祖,陵曰衍陵;曾祖妣趙國夫人張氏,追謚為孝質順皇后。皇祖諱琰,皇贈尉州刺史,追尊為孝靖皇帝,廟號烈祖,陵曰奕陵;皇祖妣秦國夫人何氏,追謚為孝靖穆皇后。皇考諱霓,皇贈汾州刺史,追尊為孝成皇帝,廟號德祖,陵曰慶陵;皇妣宋國夫人劉氏,追謚為孝成懿皇后。帝即孝成之元子也。以唐咸通丁亥歲九月九日,
 懿皇后生帝於應州之金城縣。



 初,孝成事唐獻祖為愛將,獻祖之失振武,為吐渾所攻,部下離散屬,孝成獨奮忠義,解蔚州之圍。武皇之鎮雁門也,孝成厭代,帝年甫十三,善騎射,獻祖見而撫之曰:「英氣如父,可侍吾左右。」每從圍獵,仰射飛鳥,控弦必中,尋隸武皇帳下。武皇遇上源之難,將佐罹害者甚眾,帝時年十七,翼武皇逾垣脫難,于亂兵流矢之內,獨無所傷。武皇鎮河東,以帝掌親騎。時李存信為蕃漢大將,每總兵征討,師多不利,武皇遂
 選帝副之,所向克捷。



 帝嘗宿於雁門逆旅,媼方娠,不時具饌,媼聞腹中兒語云:「大家至矣,速宜進食。」媼異之,遽起,親奉庖爨甚恭;帝詰之,媼告其故。《北夢瑣言》云:帝以媼前倨後恭,詰之,曰:「公貴不可言也。」問其故,具道娠子腹語事,帝曰:「老媼遜言,懼吾辱耳。」後果如其言。帝既壯,雄武獨斷,謙和下士。每有戰功,未嘗自伐。居常惟治兵仗,持廉處靜,晏如也。武皇常試之,召於泉府,命恣其所取,帝惟持束帛數緡而出。凡所賜與,分給部下。嘗與諸將會,諸將矜衒武勇,帝徐曰:「公輩以口擊賊,吾以手擊賊。」眾慚而止。景福初,黑山戍將王弁據振武叛,帝率其屬攻之,擒弁以獻。



 乾寧三年,梁人急攻兗、鄆,鄆帥硃瑄求救於武皇。武皇先遣騎將李承嗣、史儼援之,復遣李存信將兵三萬屯於莘縣。聞汴軍益盛,攻兗甚急,存信遣帝率三百騎而往,敗汴軍於任城,遂解兗州之圍。硃瑾見帝,執手涕謝。其年,魏帥羅宏信背盟,襲破李存信於莘縣,帝奮命殿軍而還,武皇嘉其功,即以所屬五百騎號曰「橫沖都」;侍於帳下,故兩河間目帝為李橫沖。



 明年,武皇遣大將軍李嗣昭率師下馬嶺關,將復邢、洺,梁將葛從周以兵應援。嗣昭兵敗,退入青山口,梁軍扼其路,步兵不戰自潰,嗣昭不能制。會帝本軍至,謂嗣昭曰:「步兵雖散,若吾輩空回,大事去矣。為公試決一戰,不捷而死,差勝被囚。」嗣昭曰:「吾為卿副。」帝率其屬,解鞍礪鏃,憑高列陣,左右指畫,梁人莫之測,因呼曰:「吾王命我取葛司徒,他士可無並命。」即徑犯其陣,奮擊如神。嗣昭繼進,梁軍即時退去,帝與嗣昭收兵入關。帝四中流矢,血流被股,武皇解衣授藥,手賜卮酒,撫其背曰:「吾兒神人也!微吾兒,幾為從周所笑。」自青山之戰,名聞天下。



 天復中,梁祖遣氏叔琮將兵五萬,營於洞渦。是時,諸道
 之師畢萃於太原,郡縣多陷於梁,晉陽城外,營壘相望。武皇登陴號令,不遑飲食。屬大雨彌旬,城壘多壞,武皇令帝與李嗣昭分兵四出,突入諸營,梁軍由是引退;帝率偏師追襲,復諸郡邑。昭宗之幸鳳翔也,梁祖率眾攻圍岐下,
 武皇奉詔應援,遣李嗣昭、周德威出師晉、絳,營於蒲縣。嗣昭等軍,大為梁將硃友寧、氏叔琮所敗,梁之追兵直抵晉陽,營於晉祠,日以步騎環城。武皇登城督眾,憂形於色。攻城既急,武皇與大將謀,欲出奔雲中。帝曰:「攻守之謀,據城百倍,但兒等在,必能固守。」乃止。居數日,潰軍稍集,率敢死之士,日夜分出諸門掩襲梁軍,擒其驍將游昆侖等。梁軍失勢,乃燒營而退。



 天祐五年五月,莊宗親將兵以救潞州之圍,帝時領突騎左右軍與周德威分為二廣。帝晨至夾城東北
 隅,命斧其鹿角,負芻填塹,下馬乘城大噪。時德威登西北隅,亦噪以應之。帝先入夾城,大破梁軍,是日解圍,其功居最。柏鄉之役,兩軍既成列,莊宗以梁軍甚盛,慮師入
 之怯,欲
 激壯之,手持白金巨鐘賜帝酒,謂之曰:「卿見南軍白馬、赤馬都否?睹之令人膽破。」帝曰:「彼虛有其表耳,翼日當歸吾廄中。」莊宗拊髀大笑曰:「卿已氣吞之矣!」帝引鐘盡酹,即屬鞬揮弭,躍馬挺身,與其部下百人直犯白馬都,奮楇舞槊,生挾二騎校而回,飛矢麗帝甲如蝟毛焉。由是三軍增氣,自
 辰及未,騎軍百戰,帝往來衝擊,執訊獲醜,不可勝計。是日,梁軍大敗,以功授代州刺史。莊宗遣周德威伐幽州,帝分兵略定山後八軍,與劉守光愛將元行欽戰于廣邊軍,凡八戰,帝控弦發矢七中。行欽酣戰不解,矢亦中帝股,拔矢復戰。行欽窮蹙,面縛乞降,帝酌酒飲之,拊其背曰:「吾子,壯士也!」因厚遇之。



 十三年二月,莊宗與梁將劉鄩大戰于故元城北,帝以三千騎環之,鼓噪奮擊,內外合勢,鄩軍殆盡。帝徇地慈、洺。四月,相州張筠遁走,乃
 以帝為相州刺史。九月,滄州節度使戴思遠棄城歸汴,小將毛璋據州納款,莊宗命率兵慰撫。既入城,以軍府乂安報莊宗,書吏誤云:「已至滄州,禮上畢。」莊宗省狀,怒曰:「嗣源反耶!」帝聞之懼,歸罪于書吏,斬之。未幾,承制授邢州節度使。



 十四年四月,契丹安巴堅率眾三十萬攻幽州,周德威間使告急,莊宗召諸將議進取之計,諸將咸言:「敵勢不能持久,野無所掠,食盡自還,然後踵而擊之可也。」帝奏曰:「德威盡忠于家國,孤城被攻,危亡在即,
 不宜更待敵衰。願假臣突騎五千為前鋒以援之。」莊宗曰:「公言是也。」即命帝與李存審、閻寶率軍赴援,帝為前鋒,會軍于易州。帝謂諸將曰:「敵騎以馬上為生,不須營壘,況彼眾我寡,所宜銜枚箝馬,潛行溪澗,襲其不備也。」



 八月,師發上谷,陰晦而雨,帝仰天祈祝,即時晴霽,師循大房嶺,緣潤而進。翼日,敵騎大至,每遇谷口,敵騎扼其前,帝與長子從珂奮命血戰,敵即解去,我軍方得前進。距幽州兩舍,敵騎復當谷口而陣,我軍失色。帝曰:「為將
 者受命忘家,臨敵忘身,以身徇國,正在今日。諸君觀吾父子與敵周旋!」因挺身入于敵陣,以北語諭之曰:「爾輩非吾敵,吾當與天皇較力耳。」舞槌奮擊,萬眾披靡,俄挾其隊帥而還。我軍呼躍奮擊,敵眾大敗,勢如席卷,委棄鎧仗羊馬殆不勝紀。是日,解圍,大軍入幽州,周德威迎帝,執手歔欷。九月,班師于魏州,莊宗親出郊勞,進位檢校太保。



 十八年十月,從莊宗大破梁將戴思遠于戚城,斬首二萬級。莊宗以帝為蕃漢副總管,加同平章事。



 二
 十年,代李存審為滄州節度使。四月,莊宗即位于鄴宮,帝進位檢校太傅、兼侍中。尋命帝率步騎五千襲鄆州,下之,授天平軍節度使。五月,梁人陷德勝南城,圍楊劉,以扼出師之路。帝孤守汶陽,四面拒寇,久之,莊宗方解楊劉之圍。九月,梁將王彥章以步騎萬人迫鄆州,自中都渡汶。帝遣長子從珂率騎逆戰于遞坊鎮,獲梁將任釗等三百人,彥章退保中都。莊宗聞其捷,自楊劉引軍至鄆,以帝為前鋒,大破梁軍于中都,生擒王彥章等。是
 日,諸將稱賀,莊宗以酒屬帝曰:「昨朕在朝城,諸君多勸朕棄鄆州,以河為界,賴副總管禦侮于前,崇韜畫謀于內,若信李紹宏輩,大事已掃地矣。」莊宗與諸將議兵所向,諸將多云:「青、齊、徐、兗皆空城耳,王師一臨,不戰自下。」惟帝勸莊宗徑取汴州,語在《莊宗紀》中,莊宗嘉之。帝即時前進,莊宗繼發中都。十月己卯,遲明,帝先至汴州,攻封丘門,汴將王瓚開門迎降。帝至建國門,聞梁主已殂,乃號令安撫,回軍于封禪寺。辰時,莊宗至,帝迎謁路側。
 莊宗大悅,手引帝衣,以首觸帝曰:「吾有天下,由公之血戰也,當與公共之。」尋進位兼中書令。



 二年正月,契丹犯塞,帝受命北征。二月,莊宗以郊天禮畢,賜帝鐵券。四月,潞州小將楊立叛,帝受詔討之。五月,擒楊立以獻。六月,進位太尉,移鎮汴州,代李存審為蕃漢總管。十二月,契丹入寇。



 三年正月,帝領兵破契丹于涿州,移授鎮州節度使。先是,帝領兵過鄴,鄴庫素有御甲,帝取五百聯以行。是歲,莊宗幸鄴,知之,怒甚。無何,帝奏請以長子從珂
 為北京內衙都指揮使,莊宗愈不悅,曰:「軍政在吾,安得為子奏請!吾之細鎧,不奉詔旨強取,其意何也?」令留守張憲自往取之,左右說諭,乃止。帝憂恐不自安,上表申理,方解。



 十二月,帝朝于洛陽。是時,莊宗失政,四方饑饉,軍士匱乏,有賣兒貼婦者,道路怨咨。帝在京師,頗為謠言所屬,洎硃友謙、郭崇韜無名被戮,中外大臣皆懷憂懾。諸軍馬步都虞候硃守殷奉密旨伺帝起居,守殷陰謂帝曰:「德業振主者身危,功蓋天下者不賞,公可謂振
 主矣,宜自圖之,無與禍會。」帝曰:「吾心不負天地,禍福之來,吾無所避,付之于天,卿勿多談也。」



 四年二月六日,趙在禮據魏州反,莊宗遣元行欽將兵攻之;行欽不利,退保衛州。初,帝善遇樞密使李紹宏,及帝在洛陽,群小多以飛語謗毀,紹宏每為庇護。會行欽兵退,河南尹張全義密奏,請委帝北伐,紹宏贊成之,遂遣帝將兵渡河。



 三月六日,帝至鄴都,趙在禮等登城謝罪,出牲餼以勞師,帝亦慰納之,營於鄴城之西南,下令以九日攻城。八日
 夜,軍亂。從馬直軍士有張破敗者,號令諸軍,各殺都將,縱火焚營,歡噪雷動。至五鼓,亂兵逼帝營,親軍搏戰,傷痍者殆半,亂兵益盛。帝叱之,責其狂逆之狀,亂兵對曰:「昨貝州戍兵,主上不垂厚宥;又聞鄴城平定之後,欲盡坑全軍。某等初無叛志,直畏死耳。已共諸軍商量,與城中合勢,擊退諸道之師,欲主上帝河南,請令公帝河北。」帝泣而拒之,亂兵呼曰:「令公欲何之?不帝河北,則為他人所有。茍不見幾,事當不測!」抽戈露刃,環帝左右。安重
 誨、霍彥威躡帝足,請詭隨之,因為亂兵迫入鄴城。懸橋已發,共扶帝越濠而入,趙在禮等歡泣奉迎。《通鑒》:亂兵擁嗣源及李紹真等入城,城中不受外兵。皇甫暉逆擊張破敗,斬之,外兵皆潰。趙在禮等率諸校迎拜嗣源。是日,饗將士于行宮,在禮等不納外兵,軍眾流散,無所歸向。帝登南樓,謂在禮曰:「欲建大計,非兵不能集事,吾自于城外招撫諸軍。」帝乃得出。夜至魏縣,部下不滿百人。時霍彥威所將鎮州兵五千人獨不亂,聞帝既出,相率歸帝。詰朝,帝登城掩泣曰:「國家患難,一至于此!來日
 歸籓上章,徐圖再舉。」安重誨、霍彥威等曰:「此言非便也。國家付以閫外之事,不幸師徒逗撓,為賊驚奔。元行欽狂妄小人,彼在城南,未聞戰聲,無故棄甲;如朝天之日,信其奏陳,何所不至!若歸籓聽命,便是強據要君,正墮讒慝之口也。正當星行歸闕,面叩玉階,讒間沮謀,庶全功業,無便于此者也。」帝從之。十一日,發魏縣,至相州,獲官馬二千匹,始得成軍。



 元行欽退保衛州,果以飛語上奏,帝上章申理,莊宗遣帝子從審及內官白從訓齊詔
 諭帝。從審至衛州,為行欽所械,帝奏章亦不達。帝乃趨白皋渡,駐軍于河上,會山東上供綱載絹數船適至,乃取以賞軍,軍士以之增氣。及將濟,以渡船甚少,帝方憂之。忽有木伐數隻,沿流而至,即用以濟師,故無留滯焉。二十六日至汴州,莊宗領兵至滎澤,遣龍驤都校姚彥溫為前鋒。是日,彥溫率部下八百騎歸于帝,具言:「主上為行欽所惑,事勢已離,難與共事。」帝曰:「卿自不忠,言何悖也!」乃奪其兵,仍下令曰:「主上未諒吾心,遂致軍情至
 此,宜速赴京師。」既而房知溫、杜晏球自北面相繼而至。



 四月丁亥朔,至罌子谷,聞蕭牆釁作,莊宗晏駕,帝慟哭不自勝。詰旦,硃守殷遣人馳報:「京城大亂,燔剽不息,請速至京師。」己丑,帝至洛陽,止于舊宅,分命諸將止其焚掠。百官弊衣旅見,帝謝之,斂衽泣涕。時魏王繼岌征蜀未還,帝謂朱守殷曰:「公善巡撫,以待魏王。吾當奉大行梓宮山陵禮畢,即歸籓矣。」是日,群臣諸將上箋勸進,帝面諭止之。樞密使李紹宏、張居翰、宰相豆盧革、韋說、六軍
 馬步都虞候硃守殷、青州節度使符習、徐州節度使霍彥威、宋州節度使杜晏球、兗州節度使房知溫等頓首言曰:「帝王應運,蓋有天命,三靈所屬,當協冥符。福之所鐘,不可以謙遜免;道之已喪,不可以智力求。前代因敗為功,殷憂啟聖,少康重興于有夏,平王再復於宗周,其命維新,不失舊物。今日廟社無依,人神乏主,天命所屬,人何能爭!光武所謂『使成帝再生,無以讓天下』。願殿下俯徇樂推,時哉無失,軍國大事,望以教令施行。」帝優
 答不從。



 壬辰,文武百僚三拜箋,請行監國之儀,以安宗社,答旨從之。既而有司上監國儀注。甲午,幸大內興聖宮,始受百僚班見之儀。所司議即位儀注,霍彥威、孔循等言:「唐之運數已衰,不如自創新號。」因請改國號,不從土德。帝問籓邸侍臣,左右奏曰:「先帝以錫姓宗屬,為唐雪冤,以繼唐祚。今梁朝舊人,不願殿下稱唐,請更名號。」帝曰:「予年十三事獻祖,以予宗屬,愛幸不異所生。事武皇三十年,排難解紛,櫛風沐雨,冒刃血戰,體無完膚,何
 艱險之不歷!武皇功業即予功業,先帝天下即予天下也。兄亡弟紹,于義何嫌。且同宗異號,出何典禮?歷之衰隆,吾自當之,眾之莠言,吾無取也。」時群臣集議,依違不定,惟吏部尚書李琪議曰:「殿下宗室勳賢,立大功于三世,一朝雨泣赴難,安定宗社,撫事因心,不失舊物。若別新統制,則先朝便是路人,煢煢梓宮,何所歸往!不惟殿下追感舊君之義,群臣何安!請以本朝言之,則睿宗、文宗、武宗皆以弟兄相繼,即位柩前,如儲后之儀可也。」于
 是群議始定。河中軍校王舜賢奏,節度使李存霸以今月三日出奔,不知所在。乙未,敕曰:「寡人允副群情,方監國事,外安黎庶,內睦宗親,庶諧敦惇之規,永保隆平之運。昨京師變起,禍難薦臻,至于戚屬之間,不測驚奔之所,慮因藏竄,濫被傷痍,言念于茲,自然流涕。宜令河南府及諸道,應諸王眷屬等,昨因驚擾出奔,所至之處,即時津送赴闕。如不幸物故者,量事收瘞以聞。」《北夢瑣言》:莊宗諸弟存紀、存確匿於南山民家,人有以報安重誨者,重誨曰:「主上以下詔尋訪,帝之仁德,必不加害,不如密令殺之。」
 果並命于民家。後明宗聞之,切讓重誨,傷惜久之。以中門使安重誨為樞密使,以鎮州別駕張延朗為樞密副使,以客將范延光為宣徽使,進奏官馮贇為內客省使。



 丙申,下敕:「今年夏苗,委人戶自供,通頃畝五家為保,本州具帳送省,州縣不得差人檢括。如人戶隱欺,許人陳告,其田倍徵。」己亥,命石敬瑭權知陜州兵馬留後,皇子從珂權知河南府兵馬留後。庚子,淮南楊溥進新茶。以權知汴州軍州事孔循為樞密副使,以陳州刺史劉仲殷為鄧州留後,以鄭州
 防禦使王思同為同州留後。敕曰:「租庸使孔謙,濫承委寄,專掌重權,侵剝萬端,奸欺百變。遂使生靈塗炭,軍士饑寒,成天下之瘡痍,極人間之疲弊。載詳眾狀,側聽輿辭,難私降黜之文,合正誅夷之典。宜削奪在身官爵,按軍令處分。雖犯眾怒,特貸全家,所有田宅,並從籍沒。」是日,謙伏誅。敕停租庸名額,依舊為鹽鐵、戶部、度支三司,委宰臣豆盧革專判。



 中書門下上言:「請停廢諸道鹽運使、內勾司、租庸院大程官,出放豬羊柴炭戶。括田竿尺,
 一依硃梁制度,仍委節度、刺史通申三司,不得差使量檢。州使公廨錢物,先被租庸院管系,今據數卻還州府,州府不得科率百姓。百姓合散蠶鹽,每年只二月內一度人表散,依夏稅限納錢。夏秋苗稅子,除元征石斗及地頭錢,餘外不得紐配。先遇赦所放逋稅,租庸違制徵收,並與除放。今欲曉告河南府及諸道準此施行。」從之。是日,宋州節度使元行欽伏誅。壬寅,以樞密副使孔循為樞密使。



\end{pinyinscope}