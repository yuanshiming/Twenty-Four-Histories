\article{末帝紀中}

\begin{pinyinscope}
清泰二年春正月丙申朔,帝御明堂殿受朝賀,仗衛如式。乙巳,中書門下奏:「遇千春節,凡刑獄公事奏覆,候次月施行。今後請重系者即候次月,輕系者即節前奏覆
 決遣。」從之。戊申,宗正寺奏:「北京、應州、曹州諸陵,望差本州府長官朝拜。
 \gezhu{
  《五代會要》載宗正寺原奏云:北京永興、長寧、建極三陵,應州遂、衍、奕三陵,準曹州溫陵例,下本州府官朝拜。}
 雍、坤、和、徽四陵,差太常宗正卿朝拜。」從之。己酉,北京奏,光祿卿致仕周元豹卒。庚申,鄴都進天王甲。帝在籓時,有相士言帝如毗沙天王,帝知之,竊喜。及即位,選軍士之魁偉者,被以天王甲,俾居宿衛,因詔諸道造此甲而進之。三司奏,添征蠶鹽錢及增曲價。先是曲斤八十文,增至一百五十文。乙丑,雲州節度使張
 溫移鎮晉州,以西京留守安重霸為雲州節度使。



 二月庚午,定州節度使王從溫移鎮兗州;振武軍節度使楊檀移鎮定州,兼北面行營馬步都虞候。甲戌,以定州節度使李周為京兆尹,充西京留守;以樞密使、天雄軍節度使範延光為檢校太師、兼中書令,充汴州節度使;皇子鎮州節度使兼河南尹、判六軍諸衛事、左右街坊使重美加檢校太尉、同平章事,充天雄軍節度使,餘如故。辛巳,以右諫議大夫盧損為御史中丞,以御史中丞張
 鵬為刑部侍郎。壬午,寧遠軍節度使馬存加兼侍中,鎮南軍節度使馬希振加兼中書令。詔順義軍節度使姚彥璋加兼侍中。己丑,宰臣盧文紀等上皇妣魯國太夫人尊謚,曰宣憲皇太后,請擇日冊命。從之。



 三月戊戌,故太子太保趙鳳贈太傅。辛丑,以前汴州節度使趙延壽為許州節度使兼樞密使;以夏州行軍司馬李彞殷為本州節度使,兄彞超卒故也。癸卯,以靜海軍節度使、檢校太師、兼中書令、安南都護錢元球為留守太保,餘如
 故。丙午,以給事中趙光輔為右散騎常侍。戊申,皇妹魏國夫人石氏封晉國長公主,齊國公主趙氏封燕國長公主。己酉,有司上言:「宣憲皇后未及山陵,權于舊陵所建廟。」從之。辛亥,功德使奏:「每年誕節,諸州府奏薦僧道,其僧尼欲立講論科、講經科、表白科、文章應制科、持念科、禪科、聲贊科,道士欲立經法科、講論科、文章應制科、表白科、聲贊科、焚修科,以試其能否。」從之。丙辰,以右龍武統軍李德珫為涇州節度使。庚申,以鎮州節度使、知
 軍府事董溫琪為鎮州節度使、檢校太保。壬戌,以左右彰聖都指揮使、富州刺史安審琦領楚州順化軍節度使,軍職如故。審琦受閔帝命西征,至鳳翔而降,故有是命。



 是月,太常丞史在德上疏言事,其略曰:「朝廷任人,率多濫進,稱武士者,不閑計策,雖被堅執銳,戰則棄甲,窮則背軍;稱文士者,鮮有藝能,多無士行,問策謀則杜口,作文字則倩人。所謂虛設具員,枉耗國力。逢陛下惟新之運,是文明革弊之秋,臣請應內外所管軍人,凡勝衣
 甲者,請宣下本部大將一一考試武藝短長,權謀深淺。居下位有將才者便拔為大將,居上位無將略者移之下軍。其東班臣僚,請內出策題,下中書令宰臣面試。如下位有大才者便拔居大位,處大位無大才者即移之下僚。」其疏大約如此。盧文紀等見其奏不悅,班行亦多憤悱,故諫官劉濤、楊昭儉等上疏,請出在德疏,辨可否宣行,中書覆奏亦駁其錯誤。帝召學士馬裔孫謂曰:「史在德語太凶,其實難容。朕初臨天下,須開言路,若朝士
 以言獲罪,誰敢言者!爾代朕作詔,勿加在德之罪。」詔曰:



 左補闕劉濤等奏,太常丞史在德所上章疏,中書門下駁奏,未奉宣諭,乞,分明黜陟。



 朕常覽貞觀故事,見太宗之治理,以貞觀升平之運;太宗明聖之君,野無遺賢,朝無闕政,盡善盡美,無得而名。而陜縣丞皇甫德參輒上封章,恣行訕謗,人臣無禮,罪不容誅,賴文貞之彌縫,恕德參之狂瞽。魏徵奏太宗曰:「陛下思聞得失,只可恣其所陳,若所言不中,亦何損於國家。」朕每思之,誠
 要言也。遂得下情上達,德盛業隆,太宗之道彌光,文貞之節斯著。朕惟寡昧,獲奉宗祧,業業兢兢,懼不克荷,思欲率循古道,簡拔時材。懷忠抱直之人,虛心渴見,便佞詭隨之說,杜耳惡聞。史在德近所獻陳,誠無避忌,中書以文字紕繆,比類僭差,改易人名,觸犯廟諱,請歸憲法,以示戒懲。蓋以中書既委參詳,合盡事理,朕纘承前緒,誘勸將來。多言數窮,雖聖祖之所戒,千慮一得,冀愚者之可從。因覽文貞之言,遂寬在德之罪,已令停寢,不遣
 宣行。



 劉濤等官列諫垣,宜陳讜議,請定短長之理,以行黜陟之文。昔魏徵則請賞德參,今濤等請黜在德,事同言異,何相遠哉!將議允俞,恐虧開納。方朝廷粗理,俊乂畢臻,留一在德不足為多,去一在德未足為少,茍可懲勸,朕何憂焉!但緣情在傾輸,理難黜責,濤等敷奏,朕亦優容,宜體含宏,勉思竭進,凡百在下,悉聽朕言。



 夏四月辛巳,宰臣判三司張延朗奏:「州縣官徵科條格,其令錄在任徵科,依限了絕,一年加階,兩年與試銜,三年皆及
 限了絕,與服色。攝任者一年內了絕,仍攝,二年三年內皆及限,與真命。其主簿同縣令條。本判官一年加階,二年改試銜,三年轉官。本曹官省限內了絕,與試銜。諸節級三年內並了絕者,與賞錢三十貫。其責罰依天成四年五月五日敕施行。」從之。癸未,御史中丞盧損等進清泰元年以前十一年制敕,堪悠久施行者三百九十四道,編為三十卷。其不中選者,各令所司封閉,不得行用。詔其新編敕如可施行,付御史臺頒行。以宰相盧文
 紀兼太微宮使,弘文館大學士姚顗加門下侍郎,監修國史張延朗兼集賢殿大學士。以樞密使韓昭允為中書侍郎兼兵部尚書、平章事。乙酉,以前武勝軍節度使張萬進為鄜州節度使。辛卯,以宣徽南院使劉延皓為刑部尚書,充樞密使;以司天監耿瑗為太府卿;以偽蜀右衛上將軍胡杲通為司天監;以宣徽北院使房暠為左衛上將軍,充宣徽南院使;以樞密副使劉延朗為左領軍上將軍,充宣徽北院使兼樞密副使。



 五月丙申,新州、
 振武奏,契丹寇境。乙巳,詔:「天下見禁囚徒,自五月十二日以前,除十惡五逆、放火燒舍、持仗殺人、官典犯贓、偽行印信、合造毒藥并見欠省錢外,罪無輕重,一切釋放。」庚戌,詔不得貢奉寶裝龍鳳雕鏤刺作組織之物。中書奏:「準敕,凡廟諱但迴避正文,其偏旁文字不在減少點畫。今定州節度使楊檀、檀州、金壇等名,酌情制宜,並請改之。其表章文案偏旁字闕點畫,凡臣僚名涉偏旁,亦請改名。」詔曰:「偏旁文字,音韻懸殊,止避正呼,不宜全改。楊
 檀賜名光遠,餘依舊。」甲寅,以戶部侍郎楊凝式為秘書監,以尚書禮部侍郎,盧導為尚書右丞,以尚書右丞鄭韜光為尚書左丞。丙辰,以端明殿學士李專美為兵部侍郎,以端明殿學士李崧為戶部侍郎,以翰林學士馬裔孫為禮部侍郎,以禮部郎中、充樞密院直學士品琦為給事中,並充職如故。太子少保致仕任圜贈尚書右僕射,以順化軍節度使兼彰聖都指揮使、北面行營排陣使安審琦為邢州節度使。庚申,以兵部尚書李鈴為
 太常卿,以禮部尚書王權為戶部尚書,以太常卿李懌為禮部尚書。癸亥,以六軍諸衛判官、給事中張允為右散騎常侍。



 六月甲子朔,新州上言,契丹入寇。乙丑,有司上言,宣憲皇太后陵請以順從為名,從之。振武奏,契丹二萬騎在黑榆林。丁卯,以太子少保致仕朱漢賓卒廢朝。壬申,命史官修撰明宗實錄。契丹寇應州。以新州節度使楊漢賓為同州節度使,以前晉州節度使翟璋為新州節度使。庚辰,北面招討使趙德鈞奏,行營馬步軍
 都虞候、定州節度使楊光遠,行營排陣使、邢州節度使安審琦帥本軍至易州,見進軍追襲契丹次。河東節度使石敬瑭奏,邊軍乏芻糧,其安重榮巡邊兵士欲移振武就糧。從之。尋又奏,懷、孟租稅,請指揮于忻、代州輸納。朝廷以邊儲不給,詔河東戶民積粟處,量事抄借,仍于鎮州支絹五萬匹,送河東充博采之直。是月,北面轉運副使劉福配鎮州百姓車子一千五百乘,運糧至代州。時水旱民饑,河北諸州困于飛挽,逃潰者甚眾,軍前使
 者繼至,督促糧運,由是生靈咨怨。辛巳,詔諸州府署醫博士。丙戌,以前許州節度使李從昶為右龍武統軍,以前彰國軍節度使沙彥珣為右神武統軍。



 秋七月丙申,石敬瑭奏,斬挾馬都指揮使李暉等三十六人,以謀亂故也。時敬瑭以兵屯忻州,一日,軍士喧噪,遽呼萬歲,乃斬挾暉等以止之。《契丹國志》:契丹屢攻北邊,時石敬瑭將大兵屯忻州,潞王遣使賜軍士夏衣,傳詔撫諭,軍士呼萬歲者數四。敬瑭懼,幕僚段希堯請誅其倡者,敬瑭命劉知遠斬三十六人以殉。潞王聞,益疑之。御史中丞盧損奏:「準天成二年七月敕,每月首、十五
 日入閣,罷五日起居。臣以為中旬排仗,有勞聖躬,請只以月首入閣,五日起居依舊。又準天成三年五月、長興二年七月敕,許諸州節度使帶使相歲薦僚屬五人,餘薦三人,防禦、團練使薦二人,今乞行釐革。又長興二年八月敕,州縣佐官差充馬步判官,仍同一任,乞行止絕,依舊衙前選補。」詔曰:「今後籓臣帶使相許薦三人,餘薦二人,直屬京防禦、團練使薦一人,餘並從之。」丁酉,回紇可汗仁美遣使貢方物。西京弓弩指揮使任漢權奏,六
 月二十一日與川軍戰于金州之漢陰,王師不利,其部下兵士除傷痍外,已至鳳翔。先是,盩啡鎮將劉贇引軍入川界,為蜀將全師郁所敗。金州都監崔處訥重傷,諸州屯兵潰散。金州防御使馬全節收合州兵,固守獲全。以樞密使劉延皓為天雄軍節度使。甲辰,以右神武統軍沙彥珣權知雲州。乙巳,以徐州節度使張敬達充北面行營副總管。時契丹入邊,石敬瑭屢請益兵,朝廷軍士多在北鄙,俄聞忻州諸軍呼噪,帝不悅,乃命敬達為
 北軍之副,以減敬瑭之權也。丁巳,宰臣盧文紀等上疏,其略曰:



 臣近蒙召對,面奉天旨:「凡軍國庶事,利害可否,卿等合盡言者。」臣等謬處臺衡,奉行制敕,但緣事理,互有區分,軍戎不在于職司,錢穀非關于局分,茍陳異見,即類侵官。況才不濟時,職非經遠,因五日起居之例,于兩班旅見之時,略獲對揚,兼承顧問。衛士周環於階陛,庶臣羅列于殿庭,四面聚觀,十手所指,臣等茍欲各伸愚短,此時安敢敷陳。韓非昔懼於說難,孟子亦憂於言
 責。臣竊奉本朝故事,肅宗初平寇難,再復寰瀛,頗經涉于艱難,尤勤勞于委任。每正衙奏事,則泛咨訪于群臣;及便殿詢謀,則獨對揚於四輔。自上元年後,于長安東內置延英殿,宰臣如有奏議,聖旨或有特宣,皆于前一日上聞。對御之時,只奉冕旒,旁無侍衛。獻可替否,得曲盡于討論:舍短從長,故無虞于漏洩。君臣之際,情理坦然。伏望聖慈,俯循故事,或有事關軍國,謀繫否臧,未果決于聖懷,要詢訪於臣輩,則請依延英故事,前一日傳
 宣。或臣等有所聽聞,切關利害,難形文字,須面敷揚,臣等亦依故事,前一日請開延英。當君臣奏議之時,只請機要臣僚侍立左右。兼乞稍霽威嚴,恕臣荒拙,雖乏鷹鸇之效,庶盡葵藿之心。



 詔曰:「卿等濟代英才,鎮時碩德,或締構于興王之日,或經綸于纘聖之時,鹽梅之任俱崇藥石之言並切,請復延英之制,以伸議政之規。而況列聖遺芳,皇朝盛事,載詳徵引,良切歎嘉。恭惟五日起居,先皇垂範,俟百僚之俱退,召四輔以獨升,接以溫顏,
 詢其理道,計此時作事之意,亦昔日延英之流。朕叨獲嗣承,切思遵守,將成其美,不爽兼行。其五日起居,仍令仍舊,尋常公事,亦可便舉奏聞。或事屬機宜,理當秘密,量事緊慢,不限隔日,及當日便可於閣門祗候,具榜子奏聞。請面敷揚,即當盡屏侍臣,端居便殿,佇聞高議,以慰虛懷。朕或要見卿時,亦令當時宣召,但能務致理之實,何必拘延英之名。有事足可以討論,有言足可以陳述,宜以沃心為務,勿以逆耳為虞。勉罄謀猷,以裨寡昧。」
 帝性仁恕,聽納不倦,嘗因朝會謂盧文紀等曰:「朕在籓時,人說唐代為人主端拱而天下治,蓋以外恃將校,內倚謀臣,故端拱而事辦。朕荷先朝鴻業,卿等先朝舊臣,每一相見,除承奉外,略無社稷大計一言相救,坐視朕之寡昧,其如宗社何!」文紀等引咎致謝,因奏延英故事,故有是詔。



 八月庚午,滑州節度使高允韜卒。壬申,以右衛上將軍王景戡為左衛上將軍,以右神武統軍婁繼英為右衛上將軍。己卯,以西上閣門使、行少府少監兼
 通事舍人蘇繼顏為司農卿,職如故。辛巳,以權知雲州、右神武統軍沙彥珣為雲州節度使。鄴都殺人賊陳延嗣并母、妹、妻等并棄市。延嗣父子相承,與其妹、妻于諸州郡誘人殺之,而奪其財,前後被殺者數百人,至是事泄而誅之。癸未,以前潞州行軍司馬陳元為將作監,以元善醫,故有是命。丁亥,以洺州團練使李彥舜為義武軍節度使、檢校太傅。太原奏,達靼部族于靈邱安置。己丑,以太子少保致仕戴思遠卒廢朝。庚寅,以前兗州節
 度使楊漢章為左神武統軍,以前邢州節度使康思立為右神武統軍。潞州奏,前雲州節度使安重霸卒。



 九月己亥,以河陽節度使、侍衛馬軍都指揮使安從進為襄州節度使;以襄州節度使趙在禮為宋州節度使。癸卯,以忠正軍節度使、侍衛步軍都指揮使宋審虔為河陽節度使,典軍如故。己酉,禮部貢院奏:「進士請夜試,童子依舊表薦,重置明算道舉。舉人落第後,別取文解。五科試紙,不用中書印,用本司印。」並從之。以宣徽南院使房
 暠為刑部尚書,充樞密使;以宣徽北院使、充樞密副使劉延朗為宣徽南院使,充樞密副使。丙辰,以左僕射李愚卒廢朝。



 冬十月丁卯,幸崇道宮、甘泉亭。己巳,以左衛上將軍李頃為左領軍上將軍。北面行營總管石敬瑭奏,自代州歸鎮。庚午,以晉州節度使張溫卒廢朝。甲戌,幸趙延壽、張延朗第。丁丑,以端明殿學士、兵部侍郎李專美為秘書監,充宣徽北院使。庚寅,以左諫議大夫唐汭為左散騎常侍。



 十一月庚子,以左驍衛上將軍郝瓊為
 左金吾上將軍,以光祿卿王玟為太子賓客。以徐州節度使張敬達為晉州節度使,依前充大同、振武、威塞、彰國等軍兵馬副總管。丁未,以秘書少監丁濟為太子詹事。乙卯,以前金州防禦使馬全節為滄州留後。《通鑒》:劉延朗欲除全節絳州刺史,群議沸騰。帝聞之,以為橫海留後。渤海國遣使朝貢。



 十二月戊辰,禁用鉛錢。壬申,以中書侍郎兼兵部尚書、充樞密使韓昭允為檢校司空、同平章事,充河中節度使。甲戌,以宗正少卿李延祚為將作監致仕。丁丑,故武安軍州節
 度使、累贈太傅劉建峰贈太尉,從湖南之請也。戊寅,太常奏:「來年正月一日上辛,祀昊天上帝于圓丘,依禮大祠不朝。」詔曰:「祀事在質明前,儀仗在日出後,事不相妨,宜依常年受朝。」壬午,以翰林學士承旨、戶部侍郎程遂為兵部侍郎;翰林學士、工部侍郎崔棁為戶部侍郎;翰林學士、中書舍人和凝為工部侍朗,並依前充職。乙酉,以前祕書監楊凝式為兵部侍郎。己丑,以前同州節度使馮道為司空,以尚書右僕射劉煦為左僕射,以太子
 少師盧質為右僕射,以兵部侍郎馬縞兼國子祭酒。



\end{pinyinscope}