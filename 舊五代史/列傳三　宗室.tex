\article{列傳三 宗室}

\begin{pinyinscope}

 永王存霸,武皇子,莊宗第二弟,同光三年封。莊宗敗,為軍卒所殺。《歐陽史》云:存霸歷昭義、天平、河中三軍節度使,居京師食俸祿而已。趙在禮反,乃遣存霸於河中,莊宗再幸汜水,徙存霸北京留守。《通鑑》云:李紹榮欲奔河中就永王存霸,從兵稍散,存霸亦率眾千人
 棄鎮奔晉陽。又云:存霸至晉陽,從兵逃散俱盡,存霸削髮僧服謁李彥超:「願為山僧,幸垂庇護。」軍士爭欲殺之,彥超曰:「六相公來,當奏取進止。」軍士不聽,殺之於府門之碑下。


邕王存美,武皇子,莊宗第三弟,同光三年封。莊宗敗,不知所終。
 \gezhu{
  《通鑑》云:存美以病風偏枯得免,居於晉陽。}



 薛王存禮,武皇子,同光三年封。莊宗敗,不知所終。



 申王存渥,莊宗第四弟,《歐陽史》,存渥與存霸、存紀皆莊宗同母弟。同光三年封。莊宗敗,與劉皇后同奔太原,為部下所殺。《通鑒》云:存渥至晉陽,李彥超不納,走至風谷,為其下所殺。



 睦王存乂,莊宗第五弟,同光三所封。案,以下原闕。歷鄜州節度使,後以郭崇韜婿為莊宗所殺。《北夢瑣言》云:莊宗異母弟存乂,以郭崇韜女婿伏誅。先是,郭崇韜既誅之後,朝野駭惋,議論紛然。莊宗令閹人察訪外事,言存乂於諸將坐上訴郭氏之無罪,其言怨望。又於妖術人楊千郎家飲酒聚會,攘臂而泣。楊千郎者,魏州賤民,自言得墨子術于婦翁,能役使陰物,帽下召食物果實之類。又蒱爪博必勝,人有拳握之物,以法必取。又說煉丹乾汞,易人形,破扃鑰。貴要間神奇之,官至尚書郎,賜紫,其妻出入宮禁,承恩用事。皇弟存乂常朋淫於其家,至是與存乂並罹其禍。



 通王存確,莊宗第六弟,雅王存紀,莊宗第七弟,同光三年封。莊宗敗,並為霍彥威所殺。《梁紀》,太祖有子廷鸞、落落;《盧文進傳》,莊宗又有
 弟存矩。今《宗室傳》皆不載。



 魏王繼岌,莊宗子也。莊宗即位于魏州,以繼岌充北都留守;及以鎮州為北都,又命為留守。《五代會要》:三年九月二十三日,封為魏王。三年,伐蜀,以繼岌為都統,郭崇韜為招討使。十月戊寅,至鳳州,武興軍節度使王承捷以鳳、興、文、扶四州降。甲申,至故鎮,康延孝收興州。時偽蜀主王衍率親軍五萬在利州,令步騎親軍三萬逆戰于三泉,康延孝、李嚴以勁騎三千犯之,蜀軍大敗,斬首五千級,餘皆奔潰。王
 衍聞其敗也,棄利州奔歸西川,斷吉柏津,浮梁而去。己丑,繼岌至興州,偽蜀東川節度使宋光葆以梓、潼、劍、龍、普等州來降;武定軍節度使王承肇以洋、蓬、壁三州符印降;興元節度使王宗威以梁、開、通、渠、麟等五州符印送降;階州王承岳納符印;秦州節度使王承休棄城而遁。辛丑,繼岌過利州。戊申,至劍州。己酉,至綿州,王衍遣使上箋乞降。丁巳,入成都。自興師出洛至定蜀,計七十五日,走丸之勢,前代所無。師回,至渭南,聞莊宗敗。師徒
 潰散,自縊死。《太平廣記》引《王氏見聞錄》云:魏王繼岌伐蜀,回軍在道,而有鄴都之變。莊宗與劉后命內臣張漢賓齎急詔,所在催魏王歸闕。張漢賓乘驛倍道急行,至興元西縣逢魏王,宣傳詔旨。王以本軍方討漢州,康延孝相次繼來,欲候之出山,以陳凱歌,漢賓督之。有軍謀陳岷,比事梁,與漢賓熟,密問張曰:「天子改換,且是何人?」張色莊曰;「我當面奉宣詔魏王,況大軍在行,談何容易。」陳岷曰:「久忝知聞,故敢諮問,兩日來有一信風,新人已即位矣,復何形跡。」張乃說:「來時聞李嗣源過河,未知近事。」岷曰:「魏王且請盤桓,以觀其勢,未可前邁。」張以莊宗命嚴,不敢遷延,督令進發,魏王至渭南遇害矣。



 繼潼、繼嵩、繼蟾、繼嶢並莊宗子,同光三年拜光祿大夫、檢校司徒,未封。莊宗敗,並不知所終。《清異錄》:唐福慶公主下降孟知祥。長興四年,明宗晏駕,唐室亂。莊宗諸兒削發為苾
 芻,間道走蜀。時知祥新稱帝,為分主厚待猶子,賜予千計。



 從審,明宗長子,性忠勇沈厚,摧堅陷陣,人罕偕焉。從莊宗於河上,累有戰功,莊宗器賞之,用為金槍指揮使。明宗在魏府為軍士所逼,莊宗詔從審曰:「爾父于國有大功,忠孝之心,朕自明信,今為亂兵所劫,爾宜自去宣朕旨,無令有疑。」從審行至中途,為元行欽所制,復與歸洛下。莊宗改其名為繼璟,以為己子,命再往,從審固執不行,願死于御前,以明丹赤。從莊宗赴汴州,明宗之親舊
 多策馬而去,左右或勸從審令自脫,終無行意,尋為元行欽所殺。天成初,贈太保。



 秦王從榮,明宗第二子也。明宗踐阼,天成初,授鄴都留守、天雄軍節度使。三年,移北京留守,充河東節度使。四年,入為河南尹。一日,明宗謂安重誨曰:「近聞從榮左右有詐宣朕旨,令勿接儒生,儒生多懦,恐鈍志相染。朕方知之,頗駭其事。餘比以從榮方幼,出臨大籓,故先儒雅,賴其裨佐。今聞此姦憸之言,豈朕之所望屯。」鞫其言者
 將戮之,重誨曰:「若遽行刑,又慮賓從難處,且望嚴誡。」遂止。



 從榮為詩,與從事高輦等更相唱和,自謂章句獨步于一時,有詩千餘首,號曰《紫府集》。



 長興中,以本官充天下兵馬大元帥。從榮乃請以嚴衛、捧聖步騎兩指揮為秦府衙兵,每入朝,以數百騎從行,出則張弓挾矢,馳騁盈巷。既受元帥之命,即令其府屬僚佐及四方游士,各試《檄淮南書》一道,陳己將廓清宇內之意。初,言事者請為親王置師傅,明宗顧問近臣,執政以從榮名勢既隆,
 不敢忤旨,即奏云:「王官宜委。」從榮乃奏刑部侍郎劉贊為王傅,又奏翰林學士崔棁為元帥府判官。明宗曰:「學士代予詔令,不可擬議。」衣榮不悅,退謂左右曰:「既付以元帥之任,而阻予請僚佐,又未諭制旨也。」復奏刑部侍郎任贊,從之。《宋史·趙上交傳》:秦王從榮開府兼判軍衛,以上交為虞部員外郎,充六軍諸衛推官。李澣、張沆、魚崇遠皆白衣在秦府,悉與上交友善。累遷司封郎中,充判官。從榮素豪邁,不遵禮法,好暱群小,上交從容言曰:「王位尊嚴,當修令德以慰民望。王忍為此,獨不見恭世子、戾太子之事乎?」從榮怒,出之。歷涇、秦二鎮節度判官。從榮及禍,僚屬皆坐斥。上交由是知名。後舉兵犯宮室,敗死,廢為庶
 人。《通鑒·明宗紀》云:己丑,大漸,秦王從榮入問疾,帝俯首不能舉。王淑妃曰:「從榮在此。」帝不應。從榮出,聞宮中皆哭。從榮意帝已殂,明旦,稱疾不入。是夕,帝實小愈,而從榮不知。從榮自知不為時論所與,恐不得為嗣,與其黨謀,欲以兵入侍,先制權臣。壬辰,從榮自河南府常服將步騎千人陳於天津橋。孟漢瓊被甲乘馬,召馬軍都指揮使朱洪實,使將五百騎討從榮,從榮方據胡床,坐橋上,遣左右召康義誠。端門已閉,叩左掖門,從門隙窺之,見硃洪實引騎兵北來,走白從榮,從榮大驚,命取鐵掩心擐之,坐調弓矢。俄而騎兵大至,從榮走歸府,僚佐皆竄匿,牙兵掠嘉善坊潰去。從榮與妃劉氏匿床下,皇城使安從益就斬之,以其首獻。丙申,追廢從榮為庶人。《五代會要》云:清泰元年,葬以公禮。從之。《五代史補》:秦王從榮,明宗之愛子。好為詩,判河南府,辟高輦為推官。輦尤能為詩,賓主相遇甚歡。自是出入門下者,當時名士有
 若張杭、高文蔚、何仲舉之徒,莫不分庭抗禮。更唱迭和。時干戈之後,武夫用事,睹從榮所為,皆不悅。於是康知訓等竊議曰;「秦王好文,交游者多詞客,此子若一旦南面,則我等轉死溝壑,不如早圖之。」高輦知其謀,因勸秦王託疾:「此輩須來問候,請大王伏壯士,出其不意皆斬之,庶幾免禍矣。」從榮曰:「至尊在上,一旦如此,得無危乎?」輦曰:「子弄父兵,罪當笞爾;不然,則悔無及矣。」從榮猶豫不決,未幾及禍,高輦棄市。初,從榮之敗也,高輦竄於民家,且落發為僧。既擒獲,知訓以其毀形難認,復使巾幘著緋,驗其真偽,然後用刑。輦神色自若,屬聲曰:「朱衣才脫,白刃難逃。」觀者笑之。



 從璨,明宗諸子。性剛直,好客疏財,意豁如也。天成中,為右衛大將軍,時安重誨方秉事權,從璨亦不之屈,重誨嘗以此忌之。明宗幸汴,留從璨為大內皇城使。一日,召
 賓友於會節園,酒酣之後,戲登於御榻。安重誨奏請誅之。詔曰:「皇城使從璨,朕巡幸汴州,使警大內。乃全乖委任,但恣遨游,于予行從之園,頻恣歌歡之會,仍施峻法,顯辱平人,致彼喧嘩,達于聞聽。方當立法,固不黨親,宜貶授房州司戶參軍,仍令盡命。」長興中,重誨之得罪也。命復舊官,仍贈太保。



 許王從益,明宗之幼子也。宮嬪所生。明宗命王淑妃母之,嘗謂左右曰:「惟此兒生于皇宮,故尤所鐘愛。」長興末,
 封許王。晉高祖即位,以皇后即其姊也,乃養從益于宮中。晉天福中,以從益為二王後,改封郇國公,食邑三千戶。其後與母歸洛陽。及開運末,契丹主至汴,以從益遙領曹州節度使,後封許王,與王妃尋歸西京。會契丹主死,其汴州節度使蕭翰謀歸北地,慮中原無主,軍民大亂,則己亦不得按轡徐歸矣;乃詐稱契丹主命,遣人迎從益於洛陽,令知南朝軍國事。從益與王妃逃于徽陵以避之,使者至,不得已而赴焉。從益於崇元殿見群
 官。蕭翰率蕃首列拜于殿上,群官趨于殿下,乃偽署王松為左相,趙上交為右丞相。李式、翟光鄴為樞密使,王景崇為宣徽使,餘官各有署置。又以北來燕將劉祚為權侍衛使,充在京巡檢。翰北歸,從益餞于北郊。及漢高祖將離太原,從益召高行周、武行德欲拒漢高祖,行周等不從,且奏其事。漢高祖怒,車駕將至闕,從益與王妃俱賜死於私第,時年十七;時人哀之。《五代史闕文》:漢高祖自太原起軍建號,至洛陽,會郭從義先入京師,受密旨殺王淑妃與許王從益。淑妃臨刑號泣曰;「吾家子母何罪,吾
 既為契丹所立,非敢與人爭國,何不且留吾兒,每年寒食,使持一盂飯灑明宗陵寢。」聞者無不泣下。



 重吉,末帝長子,為控鶴都指揮使。閔帝嗣位,出為亳州團練使。末帝兵起,為閔帝所害。《通鑒》云:詔遣殿直楚匡祚執亳州李重吉,幽于宋州。又云:遣楚匡祚殺李重吉于宋州。匡祚榜捶重吉,責其家財。清泰元年,詔贈太尉,仍令宋州選隙地置廟。《明宗紀》:閔帝有子重哲,授銀青光祿大夫、檢校工部尚書。《歐陽史·家人傳》不載。



 雍王理美,末帝第二子,清泰三年封。晉兵入,與末帝俱自焚死。《通鑒》云:洛陽自聞兵敗,眾心大震,居人四出,逃竄山谷。門者請禁之,雍王重美曰:「國家多難,未
 能為百姓主,徒增惡名耳。不若聽其自便,事寧自還。」乃出令任從所適,眾心差安。又云:皇后積薪欲燒宮室,重美諫曰;「新天子至,必不露居,他日重勞民力,死而遺怨,將安用之。」乃止。



 史臣曰:繼岌以童騃之歲,當統帥之任,雖功成于劍外,尋求死于渭濱,蓋運盡天亡,非孺子之咎也。從審感厚遇之恩,無茍免之意,死于君側,得不謂之忠乎!從榮以狂躁之謀,賈覆亡之禍,謂為大逆,則近厚誣。從璨為權臣所忌,從益為強敵所脅,俱不得其死,亦良可傷哉!重美聽洛民之奔亡,止母后之燔爇,身雖燼于戲焰,言則
 耀乎青編。童年若斯,可謂賢矣!



\end{pinyinscope}