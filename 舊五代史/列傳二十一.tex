\article{列傳二十一}

\begin{pinyinscope}

 張
 憲,字允中,晉陽人,世以軍功為牙校。憲始童丱,喜儒學,勵志橫經,不捨晝夜。太原地雄邊服,人多尚武,恥于學業,惟憲與里人藥縱之精力遊學,弱冠盡通諸經,尤
 精《左傳》。嘗袖行所業,謁判官李襲吉,一見欣歎。既辭,謂憲曰:「子勉之,將來必成佳器。」石州刺史楊守業喜聚書,以家書示之,聞見日博。



 莊宗為行軍司馬,廣延髦俊,素知憲名,令朱守殷齎書幣延之。歲餘,釋褐交城令,秩滿,莊宗嗣世,補太原府司錄參軍。時霸府初開,幕客馬郁、王緘,燕中名士,盡與之遊。十二年,莊宗平河朔,念籓邸之舊,征赴行臺。十三年,授監察,賜緋,署魏博推官,自是恒簪筆扈從。十五年,王師戰胡柳,周德威軍不利,憲與
 同列奔馬北渡;梁軍急追,殆將不濟。至晚渡河,人皆陷水而沒,憲與從子朗履冰而行;將及岸,冰陷,朗泣,以馬箠引之,憲曰:「吾兒去矣,勿使俱陷。」朗曰:「忍季父如此,俱死無恨。」朗偃伏引箠,憲躍身而出。是夜,莊宗令於軍中求憲,或曰:「與王緘俱歿矣!」莊宗垂涕求尸,數日,聞其免也,遣使慰勞。尋改掌書記、水部郎中,賜金紫,歷魏博觀察判官。從討張文禮,鎮州平,授魏、博、鎮、冀十郡觀察判官,改考功郎中,兼御史中丞,權鎮州留事。莊宗即位,詔
 還魏都,授尚書工部侍郎,充租庸使。八月,改刑部侍郎,判吏部銓,兼太清宮副使。莊宗遷洛陽,以憲檢校吏部尚書、興唐尹、東京副留守,知留守事。憲學識優深,尤精吏道,剖析聽斷,人不敢欺。



 三年春,車駕幸鄴,時易定王都來朝,宴于行宮,將擊鞠。初,莊宗行即位之禮,卜鞠場吉,因築壇于其間,至是詔毀之。憲奏曰;「即位壇是陛下祭接天神受命之所,自風燥雨濡之外,不可輒毀,亦不可修。魏繁陽之壇,漢汜水之單,到今猶有兆象。存而不
 毀,古之道也。」即命治之于宮西。數日,未成。會憲以公事獲謫,閣門待罪,上怒,戒有司速治行宮之庭,礙事者畢去,竟毀即位壇。憲私謂郭崇韜曰;「不祥之甚,忽其本也。」



 秋,崇韜將兵征蜀,以手書告憲曰:「允中避事久矣,余受命西征,已奏還公黃閣。」憲報曰:「庖人之代尸祝,所謂非吾事也。」時樞密承旨段徊當權任事,以憲從龍舊望,不欲憲在朝廷。會孟知祥鎮蜀川,選北京留守,徊揚言曰:「北門,國家根本,非重德不可輕授;今之取才,非憲不可。」
 趨時者因附徊勢,巧中傷之。又曰:「憲有相業,然國祚中興,宰相在天子面前,得失可以改作;一方之事,制在一人,惟北面事重。」十一月,授憲銀青光祿大夫、檢校吏部尚書、太原尹、北京留守,知府事。



 四年二月,趙在禮入魏州。時憲家屬在魏,關東俶擾,在禮善待其家,遣人齎書至太原誘憲。憲斬其使,書不發函而奏。既而明宗為兵眾所劫,諸軍離散,地遠不知事實,或謂憲曰:「蜀軍未至,洛陽窘急,總管又失兵權,制在諸軍之手,又聞河朔推
 戴,事若實然,或可濟否?」憲曰:「治亂之機,間不容髮,以愚所斷,事未可知。愚聞藥縱之言,總管德量仁厚,素得士心,餘勿多言,志此而已。」四月五日,李存渥自洛陽至,口傳莊宗命,並無書詔,惟云天子授以隻箭,傳之為信。眾心惑之,時事莫測。左右獻畫曰:「存渥所乘馬,已戢其飾,復召人謀事,必行陰禍,因欲據城。寧我負人,宜早為之所,但戮呂、鄭二宦,且繫存渥,徐觀其變,事萬全矣。」憲良久曰:「吾本書生,無軍功而致身及此,一旦自布衣而紆
 金紫,向來仕宦非出他門,此畫非吾心也。事茍不濟,以身徇義。」《東都事略·張昭傳》:昭勸憲奉表明宗以勸進,憲曰:「吾書生也,天子委以保釐之任,吾豈茍生者乎!」昭曰:「此古之大節,公能行之,忠臣也。」憲既死,論者以昭能成憲之節。翌日,符彥超誅呂、鄭,軍城大亂,燔剽達曙。憲初聞有變,出奔沂州。既而有司糾其委城之罪,四月二十四日,賜死于晉陽之千佛院。幼子凝隨父走,亦為收者加害。明宗郊禮大赦,有司請昭雪,從之。憲沈靜寡慾,喜聚圖書,家書五千卷,視事之餘,手自刊校。善彈琴,不飲酒,賓僚宴語,但論文嘯詠
 而已,士友重之。



 憲長子守素,仕晉,位至尚書。



 王正言,鄆州人。父志,濟陰令。正言早孤貧,從沙門學,工詩,密州刺史賀德倫令歸俗,署郡職。德倫鎮青州,表為推官;移鎮魏州,改觀察判官。莊宗平定魏博,正言仍舊職任,小心端慎,與物無競。嘗為同職司空頲所凌,正言降心下之。頲誅,代為節度判官。同光初,守戶部尚書、興唐尹。時孔謙為租庸副使,常畏張憲挺特,不欲其領使,乃白郭崇韜留憲于魏州,請宰相豆盧革判租庸。未幾,
 復以盧質代之。孔謙白云:「錢穀重務,宰相事多,簿籍留滯。」又云:「盧質判二日,便借官錢,皆不可任。」意謂崇韜必令己代其任,時物議未允而止,謙沮喪久之。李紹宏曰:「邦計國本,時號怨府,非張憲不稱職。」即日征之。孔謙、段徊白崇韜曰:「邦計雖重,在侍中眼前,但得一人為使即可。魏博六州戶口,天下之半,王正言操守有餘,智力不足,若朝廷任使,庶幾與人共事;若專制方隅,未見其可。張憲才器兼濟,宜以委之。」崇韜即奏憲留守魏州,徵王
 正言為租庸使。正言在職,主諾而已,權柄出于孔謙。正言不耐繁浩,簿領縱橫,觸事遺忘,物論以為不可,即以孔謙代之,正言守禮部尚書。



 三年冬,代張憲為興唐尹,留守鄴都。時武德使史彥瓊,監守鄴都,廩帑出納,兵馬制置,皆出彥瓊,將佐官吏,頤指氣使,正言不能以道御之,但趑趄聽命。至是,貝州戍兵亂,入魏州,彥瓊望風敗走,亂兵剽劫坊市。正言促召書吏寫奏章,家人曰:「賊已殺人縱火,都城已陷,何奏之有。」是日,正言引諸僚佐謁
 趙在禮,《通鑒》:正言索馬,不能得,乃帥僚佐步出府門謁在禮。望塵再拜請罪。在禮曰:「尚書重德,勿自卑屈,余受國恩,與尚書共事,但思歸之眾,倉卒見迫耳。」因拜正言,厚加慰撫。明宗即位,正言求為平盧軍行軍司馬,因以授之,竟卒于任。



 胡裝,禮部尚書曾之孫。汴將楊師厚之鎮魏州,裝與副使李嗣業有舊,因往依之,薦授貴鄉令。及張彥之亂,嗣業遇害,裝罷秩,客于魏州。莊宗初至,裝謁見,求假官,司空頲以其居官貪濁,不得調者久之。十三年,莊宗還太
 原,裝候于離亭;謁者不內,乃排闥而入,曰:「臣本朝公卿子孫,從兵至此。殿下比襲唐祚,勤求英俊,以壯霸圖。臣雖不才,比于進九九,納豎刁、頭須,亦所庶幾。而羈旅累年,執事者不垂顧錄,臣不能赴海觸樹,走胡適越,今日歸死于殿下也!」莊宗愕然曰:「孤未之知,何至如是!」賜酒食慰遣之,謂郭崇韜曰:「便與擬議。」是歲,署館驛巡官。未幾,授監察御史裏行,遷節度巡官,賜緋魚袋;尋歷推官、檢校員外郎。裝學書無師法,工詩非作者,僻于題壁,所
 至宮亭寺觀,必書爵里,人或譏之,不以為愧。時四鎮幕賓皆金紫,裝獨恥銀艾。十七年,莊宗自魏州之德勝,與賓僚城樓餞別,既而群僚離席,裝獨留,獻詩三篇,意在章服。莊宗舉大鐘屬裝曰:「員外能釂此乎?」裝飲酒素少,略無難色,為之一舉而釂,莊宗即解紫袍賜之。同光初,以裝為給事中,從幸洛陽。時連年大水,百官多窘,裝求為襄州副使。四年,洛陽變擾,節度使劉訓以私忿族裝,誣奏云裝欲謀亂,人士冤之。



 崔貽孫,《新唐書·宰相世系表》:貽孫字伯垂。祖元亮,左散騎常侍。《世系表》:元亮,字晦孫,虢州刺史。父芻言,潞州判官。貽孫以門族登進士第,以監察升朝,歷清資美職。及為省郎,使于江南回,以橐裝營別墅於漢上之穀城,退居自奉。清江之上,綠竹遍野,狹徑濃密,維舟曲岸,人莫造焉,時人甚高之。及李振貶均州,貽孫曲奉之。振入朝,貽孫累遷丞郎。同光初,除吏部侍郎,銓選疏謬,貶官塞地,馳驛至潞州,致書于府帥孔勍曰:「十五年穀城山裏,自謂逸人;二千里沙塞途中,今為逐
 客。」勍以其年八十,奏留府下。明年,量移澤州司馬,遇赦還京。宰相鄭玨以姻戚之分,復擬吏部侍郎,天官任重,昏耄罔知,後遷禮部尚書,致仕而卒。《北夢瑣言》:崔貽孫年過八十,求進不休,囊橐之資,素有貯積,性好干人,喜得小惠。有子三人,自貽孫左降之後,各于舊業爭分其利,甘旨醫藥,莫有奉者。貽孫以書責之云:「生有明君宰相,死有天曹地府,吾雖考終,豈放汝耶!」



 孟鵠,魏州人。莊宗初定魏博,選幹吏以計兵賦,以鵠為度支孔目官。明宗時,為邢洺節度使,每曲意承迎,明宗
 甚德之。及孔謙專典軍賦,徵督苛急,明宗嘗切齒。及即位,鵠自租庸勾官擢為客省副使、樞密承旨,遷三司副使,出為相州刺史。會范延光再遷樞密,乃徵鵠為三司使。初,鵠有計畫之能,及專掌邦賦,操割依違,名譽頓減。期年發疾,求外任,仍授許州節度使。謝恩退,帝目送之,顧為侍臣曰:「孟鵠掌三司幾年,得至方鎮?」范延光奏曰:「鵠于同光世已為三司勾官,天成初為三司副使,出刺相州,入判三司又二年。」帝曰:「鵠以幹事,遽至方鎮,爭不
 勉旃。」鵠與延光俱魏人,厚相結托,暨延光掌樞務,援引判三司,又致節鉞,明宗知之,故以此言譏之。到任未周歲,卒。贈太傅。



 孫岳,冀州人也。強幹有才用,歷府衛右職。天成中,為潁耀二州刺史、閬州團練使,所至稱治,遷鳳州節度使。受代歸京,秦王從榮欲以岳為元帥府都押衙,事未行,馮贇舉為三司使,時預密謀。硃、馮患從榮之恣橫,岳曾極言其禍之端,康義誠聞之不悅。及從榮敗,義誠召岳同
 至河南府檢閱府藏。時紛擾未定,義誠密遣騎士射之,岳走至通利坊,為騎士所害,識與不識皆痛之。



 子璉,歷諸衛將軍、籓閫節度副使。



 張延朗,汴州開封人也。事梁,以租庸吏為鄆州糧料使。明宗克鄆州,得延朗,復以為糧料使,後徙鎮宣武、成德,以為元從孔目官。長興元年,始置三司使,拜延朗特進、工部尚書,充諸道鹽鐵轉運等使,兼判戶部度支事,詔以延朗充三司使。末帝即位,授禮部尚書,兼中書侍郎、
 平章事、判三司。延朗再上表辭曰:



 臣濫承雨露,擢處鈞衡,兼叨選部之銜,仍掌計司之重。況中省文章之地,洪爐陶鑄之門,臣自揣量,何以當處。是以繼陳章表,疊貢情誠,乞請睿恩,免貽朝論。豈謂御批累降,聖旨不移,決以此官,委臣非器,所以強收涕泗,勉遏怔忪,重思事上之門,細料盡忠之路。竊以位高則危至,寵極則謗生,君臣莫保于初終,分義難防于毀譽。臣若保茲重任,忘彼至公,徇情而以免是非,偷安而以固富貴,則內欺心腑,
 外負聖朝,何以報君父之大恩,望子孫之延慶。臣若但行王道,惟守國章,任人必取當才,決事須依正理,確違形勢,堅塞倖門,則可以振舉宏綱,彌縫大化,助陛下含容之澤,彰國家至理之風,然而讒邪者必起憾詞,憎嫉者寧無謗議,或慮至尊未悉,群謗難明,不更拔本尋源,便俟甘瑕受玷,臣心可忍,臣恥可消。只恐山林草澤之人,稱量聖制;冠履軒裳之士,輕慢朝廷。



 臣又以國計一司,掌其經費,利權二務,職在捃收。將欲養四海之貧民,
 無過薄賦;贍六軍之勁士,又藉豐儲。利害相隨,取與難酌,若使罄山採木,竭澤求魚,則地官之教化不行,國本之傷殘益甚,取怨黔首,是黷皇風。況諸道所征賦租,雖多數額,時逢水旱,或遇蟲霜,其間則有減無添,所在又申逃係欠。乃至軍儲官俸,常汲汲于供須;夏稅秋租,每懸懸于繼續。況今內外倉庫,多是罄空;遠近生民,或聞饑歉。伏惟朝廷尚添軍額,更益師徒,非時之博糴難為,異日之區分轉大。竊慮年支有闕,國計可憂。望陛下節
 例外之破除,放諸項以儉省,不添冗食,且止新兵,務急去繁,以寬經費,減奢從儉,漸俟豐盈,則屈者知恩,叛者從化,弭兵有日,富俗可期。



 臣又聞治民尚清,為政務易,易則煩苛並去,清則偏黨無施。若擇其良牧,委在正人,則境內蒸黎,必獲蘇息,官中倉庫,亦絕侵欺。伏望誡見在之處官,無乖撫俗;擇將來之蒞事,更審求賢。儻一一得人,則農無所苦;人人致理,則國復何憂。但奉公善政者,不惜重酬;昧理無功者,勿頒厚俸。益彰有道,兼絕徇
 情。伏望陛下,念臣布露之前言,閔臣驚憂于後患,察臣愚直,杜彼讒邪,臣即但副天心,不防人口,庶幾萬一,仰答聖明。



 末帝優詔答之,召于便殿,謂之曰:「卿所論奏,深中時病,形之切言,頗救朕失。國計事重,日得商量,無勞過慮也。」延朗不得已而承命。



 延朗有心計,善理繁劇。晉高祖在太原,朝廷猜忌,不欲令有積聚,係官財貨留使之外,延朗悉遣取之,晉高祖深銜其事。及晉陽起兵,末帝議親征,然亦採浮論,不能果決;延朗獨排眾議,請末帝北
 行,識者韙之。晉高祖入洛,送臺獄以誅之。其後以選求計使,難得其人,甚追悔焉。



 劉延皓,應州渾元人。祖建立,父茂成,皆以軍功推為邊將。延皓即劉后之弟也。末帝鎮鳳翔,署延皓元隨都校,奏加檢校戶部尚書。清泰元年,除宮苑使,加檢校司空。俄改宣徽南院使、檢校司徒。二年,遷樞密使、太保,出為鄴都留守、檢校太傅。延皓御軍失政,為屯將張令昭所逐,出奔相州,尋詔停所任。及晉高祖入洛,延皓逃匿龍
 門廣化寺,數日,自經而死。延皓始以后戚自籓邸出入左右,甚以溫厚見稱,故末帝嗣位之後,委居近密。及出鎮大名,而所執一變,掠人財賄,納人園宅,聚歌僮為長夜之飲,而三軍所給不時,內外怨之,因為令昭所逐。時執政以延皓失守,請舉舊章,末帝以劉后內政之故,止從罷免而已,由是清泰之政弊矣。



 劉延朗,宋州虞城人也。末帝鎮河中時,為鄆城馬步都虞候,後納為腹心。及鎮鳳翔,署為孔目吏。末帝將圖起
 義,為捍禦之備,延朗計公私粟帛,以贍其急。及西師納降,末帝赴洛,皆無所闕焉,末帝甚賞之。清泰初,除宣徽北院使,俄以劉延皓守鄴,改副樞密使,累官至檢校太傅。時房皓為樞密使,但高枕閑眠,啟奏除授,一歸延朗,由是得志。凡籓侯郡牧,自外入者,必先賂延朗,後議進貢,賂厚者先居內地,賂薄者晚出邊籓,故諸將屢有怨訕,末帝不能察之。及晉高祖入洛,延朗將竄于南山,與從者數輩,過其私第,指而嘆曰:「我有錢三十萬貫聚于
 此,不知為何人所得。」其愚暗如此。尋捕而殺之。



\end{pinyinscope}