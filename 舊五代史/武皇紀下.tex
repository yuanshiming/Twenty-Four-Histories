\article{武皇紀下}

\begin{pinyinscope}

 景福元年正月,鎮州王鎔恃燕人之援,率兵十餘萬攻邢州之堯山。武皇遣李存信將兵應援,李存孝素與存信不協,遞相猜貳,留兵不進。武皇又遣李嗣勳、李存審
 將兵援之,大破燕、趙之眾,斬首三萬,收其軍實。三月,武皇進軍渡滹沱,攻欒城,下鼓城、槁城。四月,燕軍寇雲、代,武皇班師。



 八月,赫連鐸誘幽州李匡威之眾八萬,寇天成軍,遂攻雲州,營於州北,連亙數里。武皇潛軍入於雲州,詰旦,出騎軍以擊之,斬獲數萬,李匡威燒營而遁。十月,邢州李存孝叛,納款於梁,李存信構之也。



 景福二年春,大舉以伐王鎔,以其通好於李存孝也。二月,攻天長鎮,旬日不下。王鎔出師三萬來援,武皇逆戰於叱日嶺
 下,鎮人敗,斬首萬餘級。時歲饑,軍乏食,脯屍肉而食之。進軍下井陘,李存孝將兵夜入鎮州,鎮人乞師於汴;汴帥方攻時溥,不暇應之。乃求援於幽州,李匡威率兵赴之,武皇乃班師。七月,武皇討李存孝於邢州,遂攻平山,渡滹水,攻鎮州。王鎔懼,以帛五十萬犒軍,請修舊好,仍以鎮、冀之師助擊存孝,許之。武皇進圍邢州。十二月,武皇狩於近郊,獲白兔,有角長三寸。



 乾寧元年三月,邢州李存孝出城首罪,縶歸太原,轘於市。邢、洺、磁三州平。武
 皇表馬師素為邢州節度使。



 五月,鄆州節度使朱瑄為汴軍所攻,遣使來乞師。武皇遣騎將安福順、安福應、安福遷督精騎五百,假道於魏州以應之。九月,潞州節度使康君立以鴆死。



 十月,武皇自晉陽率師伐幽州。初,李匡儔奪據兄位,燕人多不義之,安塞軍戍將劉仁恭挈族歸於武皇,武皇遇之甚厚。仁恭數進畫於蓋寓,言幽州可取之狀,願得兵一萬,指期平定。武皇方討李存孝於邢州,輟兵數千,欲納仁恭,不利而還。匡儔由是驕怠,
 數犯邊境,武皇怒,故率軍以討之。是時,雲州吐渾赫連鐸、白義誠並來歸,命皆笞而釋之。



 十一月,進攻武州。甲寅,攻新州。十二月,李匡儔命大將率步騎六萬救新州;武皇選精甲逆戰,燕軍大敗,斬首萬餘級,生獲將領百餘人,曳練徇於新州城下。是夜,新州降。辛亥,進攻媯州。壬子,燕兵復合於居庸關拒戰,武皇命精騎以疲之,令步將李存審由他道擊之,自午至晡,燕軍復敗。甲寅,李匡儔攜其族棄城而遁,將之滄州,隨行輜車,臧獲妓妾
 甚眾。滄帥盧彥威利其貨,以兵攻匡儔於景城,殺之,蓋擄其眾。丙辰,進軍幽州,其守城大將請降,武皇令李存審與劉仁恭入城撫勞,居人如故,市不改肆,封府庫以迎武皇。



 乾寧二年正月,武皇在幽州,命李存審、劉仁恭徇諸屬郡。二月,以仁恭為權幽州留後,從燕人之請也。留腹心燕留德等十餘人分典軍政,武皇遂班師,凡駐幽州四十日。



 六月,武皇率蕃漢之師自晉陽趨三輔,討鳳翔李茂貞、邠州王行瑜、華州韓建之亂。先是,三帥稱
 兵向闕,同弱王室,殺害宰輔。時河中節度使王重盈卒,重榮之子珂,即武皇之子婿也,權典軍政。其兄珙為陜州節度使,瑤為絳州刺史,與珂爭河中,遂訴於岐、邠、華三鎮,言珂本蒼頭,不當襲位。珂亦訴於武皇,武皇上表保薦珂,乞授河中旄鉞,詔可之。三帥遂以兵入覲,大掠京師,請授王珂同州節度使,王瑤河中節度使,天子亦許之。武皇遂舉兵表三帥之罪,復移檄三鎮,三鎮大懼。是月,次絳州,刺史王瑤登陴拒命,武皇攻之,旬日而拔,
 斬王瑤於軍門,誅其黨千餘人。七月,次河中,王珂迎謁於路。



 己未,同州節度使王行約棄城奔京師,與左軍兵士劫掠西市,都民大擾。行約,即行瑜弟也。庚申,樞密使駱全瓘以武皇之軍將至,請天子幸。右軍指揮使李繼鵬,茂貞假子也,本姓閻,名珪,與全瓘謀劫天子幸鳳翔。左軍指揮使王行實,亦行瑜之弟也,與劉景宣欲劫天子幸邠州。兩軍相攻,縱火燒內門,煙火蔽天。天子急詔鹽州六都兵士,令追殺亂兵,左右軍退走。王行瑜、李茂
 貞聲言自來迎駕,天子懼,出幸南山,駐蹕於莎城。是夜,熒惑犯心。壬戌,武皇進收同州,聞天子幸石門,遣判官王瑰奉表奔問,天子遣使賜詔,令與王珂同討邠、鳳。時武皇方攻華州,俄聞李茂貞領兵士三萬至盩啡,王行瑜領兵至興平,欲往石門迎駕,乃解華州之圍,進營渭橋。天子遣延王戒丕、丹王允齎詔,促武皇兵直抵邠、鳳。



 八月乙酉,供奉官張承業齎詔告諭。涇帥張鐺已領步騎三萬於京西北,扼邠、岐之路。武皇進營渭北,遣史儼
 將三千騎往石門扈駕,遣李存信、李存審會鄜延之兵攻行瑜之梨園寨。天子削奪行瑜官爵,以武皇為天下兵馬都招討使,以鄜州李思孝為北面招討使,以涇州張鐺為西南面招討使。天子又遣延王、丹王賜武皇御衣及大將茶酒、弓矢,命二王兄事武皇。延王傳天子密旨云:「日昨非卿至此,已為賊庭行酒之人矣。所慮者二凶締合,卒難翦除,且欲姑息茂貞,令與卿修好,俟梟斬行瑜,更與卿商量。」武皇上表,請駕還京。令李存節領二
 千騎於京西北,以防邠賊奔突。辛亥,天子還宮,加武皇守太師、中書令、邠寧四面行營都統。



 時王行瑜弟兄固守梨園寨,我師攻之甚急,李茂貞遣兵萬餘來援行瑜,營於龍泉鎮,茂貞自率兵三萬迫咸陽。武皇奉請詔茂貞罷兵,兼請削奪茂貞官爵。詔曰:「茂貞勒兵,蓋備非常,尋已發遣歸鎮。」又言:「茂貞已誅李繼鵬、李繼晸,卿可切戒兵甲,無犯土疆。」武皇請賜河中王珂旌節,三表許之。又表李罕之為副都統。



 十月丙戌,李存信於梨園寨北遇
 賊軍,斬首千餘級,自是賊閉壁不出。戊子,天子賜武皇內弟子四人,又降朱書御札,賜魏國夫人陳氏。是月,王行瑜因敗衄之後,閉壁自固,武皇令李罕之晝夜急攻,賊軍乏食,拔營而去。李存信與罕之等先伏軍於厄路,俟賊軍之至,縱兵擊之,殺戮萬計。是日,收梨園等三寨,生擒行瑜之子知進,並母丘氏、大將李元福等二百人,送赴闕庭。庚寅,王行約、王行實燒劫寧州遁走,寧州守將徐景乞降。武皇表蘇文建為邠州節度使,且於寧州
 為治所。十一月丁巳,收龍泉寨。時行瑜以精甲五千守之,李茂貞出兵來援,為李罕之所敗,邠賊遂棄龍泉寨而去。行瑜復入邠州,大軍進逼其城,行瑜登城號哭曰:「行瑜無罪,昨殺南北司大臣,是岐帥將兵脅制主上,請治岐州,行瑜乞束身歸朝。」武皇報曰:「王尚父何恭之甚耶!僕受命討三賊臣,公其一也。如能束身歸闕,老夫未敢專命,為公奏取進止。」行瑜懼,棄城而遁。武皇收其城,封府庫,遽以捷聞。既而慶州奏,王行瑜將家屬五百人
 到州界,為部下所殺,傳首闕下。武皇既平行瑜,還軍渭北。



 十二月,武皇營於雲陽,候討鳳翔進止。乙未,天子賜武皇為忠貞平難功臣,進封晉王,加實封二百戶。武皇復上表請討李茂貞,天子不允。武皇私謂詔使曰:「觀主上意,疑僕別有他腸,復何言哉!但禍不去胎,憂患未已。」又奏:「臣統領大軍,不敢徑赴朝覲。」遂班師。



 乾寧三年正月,汴人大舉以攻兗、鄆,朱瑄、朱瑾再乞師於武皇,假道於魏州,羅宏信許之。乃令都指揮使李存信將步騎三
 萬與李承嗣、史儼會軍,以拒汴人。存信軍於莘,與朱瑾合勢,頻挫汴軍,汴帥患之,乃間魏人。存信御兵無法,稍侵魏之芻牧者,宏信乃與汴帥通,出師三萬攻存信軍。存信揭營而退,保於洺州。三月,武皇大掠相、魏諸邑,攻李固、洹水,殺魏兵萬餘人,進攻魏州。五月,汴將葛從周、氏叔琮引兵赴援。



 六月,李茂貞舉兵犯京師。七月,車駕幸華州。是月,武皇與汴軍戰於洹水之上,鐵林指揮使落落被擒。落落,武皇之長子也。既戰,馬踣於坎,武皇馳
 騎以救之,其馬亦踣,汴之追兵將及,武皇背射一發而斃,乃退。



 九月,李存信攻魏之臨清,汴將葛從周等引軍來援,大敗於宗城北。存信進攻魏州。十月,武皇敗魏軍於白龍潭,追擊至觀音門,汴軍救至,乃退。十一月,武皇徵兵於幽、鎮、定三州,將迎駕於華下,幽州劉仁恭託以契丹入寇,俟敵退聽命。



 乾寧四年正月,汴軍陷兗、鄆,騎將李承嗣、史儼與朱瑾同奔於淮南。三月,陜帥王珙攻河中,王珂來告難;武皇遣李嗣昭率二千騎赴之,破陜
 軍於猗氏,乃解河中之圍。至是,天子遣延王戒丕至晉陽,傳宣旨於武皇:「朕不取卿言,以及於此,茍非英賢竭力,朕何由再謁廟廷!在卿表率,予所望也。」



 七月,武皇復徵兵於幽州,劉仁恭辭旨不遜,武皇以書讓之;仁恭捧書謾罵,抵之於地,仍囚武皇之行人。八月,大舉以伐仁恭。九月,師次蔚州。戊寅,晨霧晦暝,占者云不利深入。辛巳,攻安塞,俄報「燕將單可及領騎軍至矣。」武皇方置酒高會,前鋒又報「賊至矣」!武皇曰:「仁恭何在?」曰:「但見可及
 輩。」武皇張目怒曰:「可及輩何足為敵!」仍促令出師。燕軍已擊武皇軍寨,武皇乘醉擊賊,燕軍披靡。時步兵望賊而退,為燕軍所乘,大敗於木瓜澗。俄而大風雨震電,燕軍解去,武皇方醒。甲午,師次代州,劉仁恭遣使謝罪於武皇,武皇亦以書報之,自此有檄十餘返。



 光化元年春正月,鳳翔李茂貞、華州韓建皆致書於武皇,乞修和好,同獎王室,兼乞助丁匠修繕秦宮,武皇許之。



 四月,汴將葛從周寇邢、洺、磁等州,旬日之內,三州連陷。汴人以葛
 從周為邢州節度使。大將李存信收軍,自馬嶺而旋。



 八月壬戌,天子自華還宮。是時,車駕初復,而欲諸侯輯睦,賜武皇詔,令與汴帥通好。武皇不欲先下汴帥,乃致書於鎮州王鎔,令導其意。明年,汴帥遣使奉書幣來修好,武皇亦報之。自是使車交馳,朝野相賀。九月,武皇遣周德威、李嗣昭率兵三萬出青山口,以迫邢、洺。十月,遇汴將葛從周於張公橋,既戰,我軍大敗。是月,河中王珂來告急,言王珙引汴軍來寇,武皇遣李嗣昭將兵三千以
 援之,屯於胡壁堡。汴軍萬餘人來拒戰,嗣昭擊退之。



 十二月,潞州節度使薛志勤卒,澤州刺史李罕之以本軍夜入潞州,據城以叛。罕之報武皇曰:「薛鐵山新死,潞民無主,慮軍城有變,輒專命鎮撫。」武皇令人讓之,罕之乃歸於汴。武皇遣李嗣昭將兵討之,下澤州,收罕之家屬,拘送晉陽。



 光化二年春正月,李罕之陷沁州。三月,汴將葛從周、氏叔琮自土門陷承天軍,又陷遼州,進軍榆次。武皇令周德威擊之,敗汴軍於洞渦驛,叔琮棄營而遁,
 德威追擊,出石曾關,殺千餘人。汴人復陷澤州。五月,武皇令都指揮使李君慶將兵收澤、潞,為汴軍所敗而還。以李嗣昭為都指揮使,進攻潞州。八月,嗣昭營於潞州城下,前鋒下澤州。時汴將賀德倫、張歸厚等守潞州。是月,德倫等棄城而遁,潞州平。九月,武皇表汾州刺史孟遷為潞州節度使。



 光化三年,汴軍大寇河朔,幽州劉仁恭乞師,武皇遣周德威帥五千騎以援之。七月,李嗣昭攻堯山,至內丘,敗汴軍於沙河;進攻洺州,下之。九月,汴
 帥自將兵三萬圍洺州,嗣昭棄城而歸,葛從周設伏於青山口,嗣昭之軍不利。十月,汴人乘勝寇鎮、定,鎮、定懼,皆納賂於汴。是時,周德威與燕軍劉守光敗汴人二萬於望都,聞定州王郜來奔,乃班師。是月,天子加武皇實封一百戶。遣李嗣昭率步騎三萬攻懷州,下之。進攻河陽,汴將閻寶率軍來援,嗣昭退保懷州。



 天復元年正月,汴將張存敬攻陷晉、絳二州,以兵二萬屯絳州,以扼援路。二月,張存敬迫河中,王珂告急於武皇,使者相望於
 路。邠國夫人,武皇愛女也,亦以書至,懇切求援。武皇報曰:「賊阻道路,眾寡不敵,救爾即與爾兩亡,可與王郎棄城歸朝。」珂遂送款於張存敬。三月,汴帥自大梁至河中,王珂遂出迎,尋徙於汴。天子以汴帥兼鎮河中,武皇自是不復能援京師,霸業由是中否。



 四月,汴將氏叔琮率兵五萬自太行路寇澤、潞,魏博大將張文恭領軍自新口入,葛從周領兗、鄆之眾自土門入,張歸厚以邢、洺之眾自馬嶺入,定州王處直之眾自飛狐入,侯言以
 晉、絳之兵自陰地入。氏叔琮、康懷英營於澤州之昂車。武皇令李嗣昭將三千騎赴澤州援李存璋,而歸賀德倫。氏叔琮軍至潞州,孟遷開門迎,沁州刺史蔡訓亦以城降於汴,氏叔琮悉其眾趨石會關。是時,偏將李審建先統兵三千在潞州,亦與孟遷降於汴;及叔琮之入寇也,審建為其鄉導。汴人營於洞渦,別將白奉國與鎮州大將石公立自井陘入,陷承天軍。及攻壽陽,遼州刺史張鄂以城降於汴,都人大恐。時霖雨積旬,汴軍屯聚既
 眾,芻糧不給,復多痢瘧,師人多死。時大將李嗣昭、李嗣源每夜率驍騎突營掩殺,敵眾恐懼。



 五月,汴軍皆退。氏叔琮軍出石會,周德威、李嗣昭以精騎五千躡之,殺戮萬計。初,汴軍之將入寇也,汾州刺史李瑭據城叛,以連汴人,至是武皇令李嗣昭、李存審將兵討之。是歲,并、汾饑,粟暴貴,人多附瑭為亂,嗣昭悉力攻城,三日而拔,擒李瑭等斬於晉陽市。氏叔琮既旋軍,過潞州,擄孟遷以歸。汴帥以丁會為潞州節度使。


六月,遣李嗣昭、周德威
 將兵出陰地,攻慈、隰二郡,隰州刺史唐禮、慈州刺史張瑰並以城來降。武皇以汴寇方盛,難以兵服,佯降心以緩其謀,乃遣牙將張特持幣馬書檄以諭之,陳當時利害,請復舊好。十一月壬子,汴帥營於渭濱。甲寅,天子出幸鳳翔。
 \gezhu{
  《新唐書》:帝如鳳翔,李茂貞、韓全誨請召克用入衛,克用間道遣使者奔問,並詒書全忠,勸還汴,全忠不答。}
 武皇遣李嗣昭率兵三千自沁州趨平陽,遇汴軍於晉州北,斬首五百級。



 天復二年二月,李嗣昭、周德威領大軍自慈、隰進攻晉、絳,營於蒲縣。乙未,汴將朱友寧、
 氏叔琮將兵十萬,營於蒲縣之南。乙巳,汴帥自領軍至晉州,德威之軍大恐。三月丁巳,有虹貫德威之營。戊午,氏叔琮率軍來戰,德威逆擊,為汴人所敗,兵仗、輜車委棄殆盡。朱友寧長驅至汾州,慈、隰二州復為汴人所據。辛酉,汴軍營於晉陽之西北,攻城西門,周德威、李嗣昭緣山保其餘眾而旋。武皇驅丁壯登陴拒守,汴軍攻城日急;武皇召李嗣昭、周德威等謀將出奔雲州,嗣昭以為不可。李存信堅請且入北蕃,續圖進取,嗣昭等固爭
 之,太妃劉氏亦極言於內,乃止。居數日,亡散之士復集,軍城稍安。李嗣昭與李嗣源夜入汴軍,斬將搴旗,敵人捍禦不暇,自相驚擾。丁卯,朱友寧燒營而遁,周德威追至白壁關,俘斬萬計,因收復慈、隰、汾等三州。



 天復三年正月,天子自鳳翔歸京。五月,雲州都將王敬暉殺刺史劉再立,以城歸於劉仁恭。武皇遣李嗣昭討之,仁恭遣將以兵五萬來援雲州,嗣昭退保樂安,燕人擄敬暉,棄城而去。武皇怒,笞嗣昭及李存審而削其官。是時,親軍
 萬眾皆邊部人,動違紀律,人甚苦之,左右或以為言。武皇曰:「此輩膽略過人,數十年從吾征伐,比年以來,國藏空竭,諸軍之家賣馬自給。今四方諸侯皆懸重賞以募勇士,吾若束之以法,急則棄吾,吾安能獨保此乎!俟時開運泰,吾固自能處置矣。」


天祐元年閏四月,汴帥迫天子遷都於洛陽。
 \gezhu{
  《新唐書》:帝東遷,詔至太原,克用泣謂其下曰:「乘輿不復西矣!」遣使者奔問行在。}
 五月乙丑,天子制授武皇葉盟同力功臣,加食邑三千戶,實封三百戶。八月,汴帥遣朱友恭弒昭宗於洛陽宮,
 輝王即位。告哀使至晉陽,武皇南向慟哭,三軍縞素。



 天祐二年春,契丹安巴堅始盛,武皇召之,安巴堅領部族三凡十萬至雲州,與武皇會於雲州之東,握手甚歡,結為兄弟,旬日而去,留馬千匹,牛羊萬計,期以冬初大舉渡河。



 天祐三年正月,魏博既殺牙軍,魏將史仁遇據高唐以叛,遣人乞師於武皇,武皇遣李嗣昭率三千騎攻邢州以應之,遇汴將牛存節、張筠於青山口,嗣昭不利而還。九月,汴帥親率兵攻滄州,幽州劉仁恭遣
 使來乞師,武皇乃徵兵於仁恭,將攻潞州,以解滄州之圍。仁恭遣掌書記馬郁、都指揮使李溥等將兵三萬,會於晉陽,武皇遣周德威、李嗣昭合燕軍以攻澤、潞。十二月,潞州節度使丁會開門迎降,命李嗣昭為潞州節度使,以丁會歸於晉陽。



 天祐四年正月甲申,汴帥聞潞州失守,自滄州燒營而遁。四月,天子禪位於汴帥,奉天子為濟陰王。改元為開平,國號大梁。是歲,四川王建遣使至,勸武皇各王一方,俟破賊之後,訪唐朝宗室以嗣帝
 位,然後各歸籓守。武皇不從,以書報之曰:



 竊念本朝屯否,巨業淪胥,攀鼎駕以長違,撫彤弓而自咎,默默終占,悠悠彼蒼,生此厲階,永為痛毒,視橫流而莫救,徒誓楫以興言。別捧函題,過垂獎諭,省覽周既,駭惕異常。淚下霑衿,倍鬱申胥之素;汗流浹背,如聞蔣濟之言。



 僕經事兩朝,受恩三代,位叨將相,籍係宗枝,賜鈇鉞以專征,徵苞茅而問罪。鏖兵校戰,二十餘年,竟未能斬新莽之頭顱,斷蚩尤之肩髀,以至廟朝顛覆,豺虎縱橫。且授任分
 憂,叨榮冒寵,龜玉毀櫝,誰之咎歟!俯閱指陳,不勝慚恧。然則君臣無常位,陵谷有變遷,或箠塞長河,泥封函谷,時移事改,理有萬殊。即如周末虎爭,魏初鼎據。孫權父子,不顯授於漢恩,劉備君臣,自微興於涿郡。得之不謝於家世,失之無損於功名,適當逐鹿之秋,何惜華蟲之服。惟僕累朝席寵,奕世輸忠,忝佩訓詞,粗存家法。善博奕者惟先守道,治蹊田者不可奪牛。誓於此生,靡敢失節,仰憑廟勝,早殄寇讎。如其事與願違,則共臧洪遊於
 地下,亦無恨矣。



 惟公社稷元勳,嵩、衡降祉,鎮九州之上地,負一代之鴻才,合於此時,自求多福。所承良訊,非僕深心,天下其謂我何,有國非吾節也。悽悽孤懇,此不盡陳。



 五月,梁祖遣其將康懷英率兵十萬圍潞州,懷英驅率士眾,築壘環城,城中音信斷絕。武皇遣周德威將兵赴援,德威軍於余吾,率先鋒挑戰,日有俘獲,懷英不敢即戰。梁祖以懷英無功,乃以李思安代之。思安引軍將營於潞城,周德威以五千騎搏之,梁軍大敗,斬首千餘
 級。思安退保堅壁,別築外壘,謂之「夾塞」,以抗我之援軍。梁祖調發山東之民以供饋運,德威日以輕騎掩之,運路艱阻,眾心益恐。李思安乃自東南山口築夾道,連接夾寨,以通饋運,自是梁軍堅保夾塞。



 冬十月,武皇有疾,是時晉陽城無故自壞,占者惡之。



 天祐五年正月戊子朔,武皇疾革。辛卯,崩於晉陽,年五十三。遣令薄葬,發喪後二十七日除服。莊宗即位,追謚武皇帝,廟號太祖,陵在雁門。《五代史補》:太祖武皇,本朱耶赤心之後,沙陀部人也。其先生於雕窠中,酋長以其異生,諸族傳
 養之,遂以「諸爺」為氏,言非一父所養也。其後言訛,以「諸」為「朱」,以「爺」為「耶」。至太祖生,眇一目,長而驍勇,善騎射,所向無敵,時謂之「獨眼龍」,大為部落所疾。太祖恐禍及,遂舉族歸唐,授雲州刺史,賜姓李,名克用。黃巢犯長安,自北引兵赴難,功成,遂拜太原節度使,封晉王。武皇之有河東也,威聲大振。淮南楊行密常恨識其狀貌,因使畫工詐為商賈,往河東寫之。畫工到,未幾,人有知其謀者,擒之。武皇初甚怒,既而謂所親曰:「且吾素眇一目,試召之使寫,觀其所為如何。」及至,武皇按膝厲聲曰:「淮南使汝來寫吾真,必畫工之尤也,寫吾不及十分,即價下便是死汝之所矣。」畫工再拜下筆。時方盛暑,武皇執八角扇,因寫扇角半遮其面。武皇曰:「汝諂吾也。」遽使別寫之,又應聲下筆,畫其臂弓捻箭之狀,仍微合一目以觀箭之曲直,武皇大喜,因厚賂金帛遣之。《五代史闕文》:世傳武皇臨薨,以三矢付莊宗曰:「一矢討劉仁恭,汝不先下幽
 州,河南未可圖也。一矢擊契丹,且曰安巴堅與吾把臂而盟,結為兄弟,誓復唐家社稷,今背約附賊,汝必伐之。一矢滅硃溫,汝能成吾志,死無憾矣!」莊宗藏三矢於武皇廟庭。及討劉仁恭,命幕吏以少牢告廟,請一矢,盛以錦囊,使親將負之以為前驅。凱旋之日,隨俘馘納矢於太廟。伐契丹,滅硃氏亦如之。又,武皇眇一目,謂之「獨眼龍。」性喜殺,左右有小過失,必置於死。初諱眇,人無敢犯者,嘗令寫真,畫工即為捻箭之狀,微瞑一目,圖成而進,武皇大悅,賜予甚厚。



 史臣曰:武皇肇跡陰山,赴難唐室,逐豺狼於魏闕,殄氛祲於秦川,賜姓受封,奄有汾、晉,可謂有功矣。然雖茂勤王之績,而非無震主之威。及朱旗屯渭曲之師,俾翠輦有石門之幸,比夫桓、文之輔周室,無乃有所愧乎!洎失
 援於蒲、絳,久垂翅於并、汾,若非嗣子之英才,豈有興王之茂業。矧累功積德,未比於周文,創業開基,尚虧於魏祖。追謚為「武」,斯亦幸焉!



\end{pinyinscope}