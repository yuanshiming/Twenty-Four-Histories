\article{莊宗紀四}

\begin{pinyinscope}

 同光元年冬十月辛未朔,日有食之。是日,皇後
 劉
 氏、皇子繼岌歸鄴宮,帝送于離亭,歔欷而別。詔宣徽使李紹宏、宰相豆盧革、租庸使張憲、興唐尹王正言同守鄴城。
 壬申,帝御大軍自楊劉濟河。癸酉,至鄆州。是夜三鼓,渡汶。時王彥章守中都。甲戌,帝攻之,中都素無城守,師既雲合,梁眾自潰。是日,擒梁將王彥章及都監張漢傑、趙廷隱、劉嗣彬、李知節、康文通、王山興等將吏二百餘人,斬馘二萬,奪馬千匹。時既獲中都之捷,帝召諸將謀其所向,或言且徇兗州,徐圖進取,唯李嗣源曰:「宜急趨汴州。段凝方領大軍駐于河上,假如便來赴援,直路又阻決河,須自滑州濟渡,十萬之眾,舟楫焉能卒辦?此去汴
 城咫尺,若晝夜兼程,信宿即至,段凝未起河堧,夷門已為我有矣。臣請以千騎前驅,陛下御軍徐進,鮮不克矣。」帝嘉之。是夜,嗣源率前軍先進。翼日,車駕即路。丁丑,次曹州,郡將出降。



 己卯遲明,前軍至汴城,嗣源令左右捉生攻封丘門,梁開封尹王瓚請以城降。俄而帝與大軍繼至,王瓚迎帝自大梁門入。梁朝文武官屬于馬前謁見陳敘世代唐臣陷在偽廷,今日再睹中興,雖死無恨。帝諭之曰:「朕二十年血戰,蓋為卿等家門無足憂矣,各
 復乃位。」時梁末帝硃鍠已為其將皇甫麟所殺,獲其首,函之以獻。是日,賜樂工周匝幣帛。周匝者,帝之寵伶也,胡柳之役陷於梁,帝每思之,至是謁見,欣然慰接。周匝因言梁教坊使陳俊保庇之恩,垂泣推薦,請除郡守,帝亦許之。



 庚辰,帝御元德殿,梁百官于朝堂待罪,詔釋之。壬午,段凝所部馬步軍五萬解甲于封丘。凝等率大將先至請死,詔各賜錦袍、御馬、金幣。帝幸北郊,撫勞降軍,各令還本營。丙戌,詔曰:「懲惡勸善,務振紀綱;激濁揚清,
 須明真偽。蓋前王之令典,為歷代之通規,必按舊章,以令多士。而有志萌僭竊,位忝崇高,累世官而皆受唐恩,貪爵祿而但從偽命,或居台鉉,或處權衡,或列近職而預機謀,或當峻秩而掌刑憲,事分逆順,理合去留。偽宰相鄭玨等一十一人,皆本朝簪組,儒苑品流。雖博識多聞,備明今古;而修身慎行,頗負祖先。昧忠貞而不度安危,專利祿而全虧名節,合當大辟,無恕近親。朕以纘嗣丕基,初平巨憝,方務好生之道,在行含垢之恩。湯網垂
 仁,既矜全族;舜刑投裔,兼貸一身。爾宜自新,我全大體,其為顯列,不並庶僚。餘外應在周行,悉仍舊貫,凡居中外,咸體朕懷。」乃貶梁宰相鄭玨為萊州司戶,蕭頃為登州司戶,翰林學士劉岳為均州司馬,任贊房州司馬,姚顗復州司馬,封翹唐州司馬,李懌懷州司馬,竇夢徵沂州司馬,崇政院學士劉光素密州司戶,陸崇安州司戶,御史中丞王權隨州司戶,並員外置同正員。



 是日,以梁將段凝上疏奏:「梁朝權臣趙嚴等,並助成虐政,結怨于
 人,聖政惟新宜誅首惡。」乃下詔曰:朕既殄偽庭,顯平國患。好生之令,含宏雖切于予懷;懲惡之規,決斷難違于眾請。況趙嚴、趙鵠等,自朕收城數日布惠四方,尚匿迹以潛形,罔悛心而革面,須行赤族,以謝眾心。其張漢傑昨于中都與王彥章同時俘獲,此際未詳行止,偶示哀矜。今既上將陳詞,群情激怒,往日既彰于僭濫,此時難漏于網羅,宜置國刑,以塞群論。除妻兒骨肉外,其他疏屬僕使,並從釋放。敬翔、李振,首佐硃溫,共傾唐祚,屠害
 宗屬,殺戮朝臣,既寰宇以皆知,在人神而共怒。



 敬翔雖聞自盡,未豁幽冤,宜與李振並族於市。疏屬僕使,並從原宥。朱珪素聞狡蠹,唯務讒邪,鬥惑人情,枉害良善,將清內外,須切去除,況眾狀指陳,亦宜誅戮。契丹實喇鄂博,既棄其母,又背其兄。朕比重懷來,厚加恩渥,看同骨肉,錫以姓名,兼分符竹之榮,疊被頒宣之渥。而乃輒辜重惠,復背明廷,罔顧欺違,竄歸偽室,既同梟獍,難貸刑章,可並妻子同戮于市。其硃氏近親,趙鵠正身,趙嚴家
 屬,仰嚴加擒捕。其餘文武職員將校,一切不問。


是日,趙嚴、張希逸、張漢傑、張漢倫、張漢融、朱珪、敬翔、李振及契丹實喇鄂博等,并其妻孥,皆斬于汴橋下。又詔除毀硃氏宗廟神主,偽梁二主並降為庶人。天下官名府號及寺觀門額,曾經改易者,並復舊名。時帝欲發梁祖之墓,斲棺燔柩,河南尹張全義上章申理,乞存聖恩,
 \gezhu{
  《通鑒》:張全義上言:「硃溫雖國之深讎,然其人已死,刑無可加,屠滅其家,足以為報,乞免焚斫,以存聖恩。」}
 帝乃止,令刬去闕室而已。丁亥,梁百官以誅兇族,于崇元殿立班
 待罪,詔各復其位。
 \gezhu{
  《洛陽縉紳舊聞記》載張全義表云:「伏念臣誤棲惡木,曾飲盜泉,實有瑕疵,未蒙昭雪。」因下詔雪之。}
 以樞密使、檢校太保、守兵部尚書郭崇韜權行中書事。己丑,御崇元殿。制曰:



 仗順討逆,少康所以誅有窮;纘業承基,光武所以滅新莽。咸以中興景命,再造王猷,經綸于草昧之中,式遏于亂略之際。朕以欽承大寶,顯荷鴻休,雖繼前修,固慚涼德,誓平元惡,期復本朝,屬四海之阽危,允萬邦之推戴。近者親提組練,徑掃氛襖,振已墜之皇綱,殄偷安之寇孽。國讎方雪,帝道爰
 開,拯編氓覆溺之艱,救率土倒懸之苦。粵自朱溫構逆,友貞嗣兇,篡殺二君,隳殘九廟,虺毒久傷于宇宙,狼貪肆噬于華夷。剝喪元良,凌辱神主,帝里動黍離之嘆,朝廷多棟橈之危。棄德崇奸,窮兵黷武,戰士疲勞于力役,蒸民耗竭其膏腴,言念于斯,軫傷彌切。



 今則已梟逆豎,大豁群情,睹歷數之有歸,實神靈之匪昧。得不臨深表誡,馭朽為懷,將宏濟于艱難,宜特行于赦宥。應偽命流貶責授官等,已經量移者,並可復資,徒流人放歸鄉里。京
 畿及諸道見禁囚徒,大辟罪降從流,已下咸赦除之。其鄭玨等一十一人,未在移復之限。應扈從征討將校,及諸官員、職掌節級、馬步兵士及河北諸處屯駐守戍兵士等,皆情堅破敵,業茂平淮,副予戡定之謀,顯爾忠勤之節,並據等第,續議獎酬。其有歿于王事未經追贈者,各與贈官;如有子孫堪任使者,並量材錄任。應偽庭節度、觀察、防禦、團練等使及刺史、監押、行營將校等,並頒恩詔,不議改更,仍許且稱舊銜,當俟別加新命。



 理國
 之道,莫若安民;勸課之規,宜從薄賦。庶遂息肩之望,冀諧鼓腹之謠。應諸道戶口,並宜罷其差役,各務營農。所係殘欠賦稅,及諸務懸欠積年課利,及公私債負等,其汴州城內,自收復日已前,並不在征理之限;其諸道,自壬午年十二月已前,並放。北京及河北先以祆祲未平,配買征馬,如有未請卻官本錢,及買馬不迨者,可放免。應有本朝宗屬及內外文武臣僚,被硃氏無辜屠害者,並可追贈。如有子孫及本身逃難於諸處漂寓者,並令
 所在尋訪,津置赴闕。義夫節婦,孝子順孫,旌表門閭,量加賑給。或鰥寡惸獨,無所告者,仰所在各議拯救。民年過八十者,免一子從征。其有先投過偽庭將校官吏等,一切不問云。



 甲午,以樞密使、檢校太保、守兵部尚書、太原縣男郭崇韜為開府儀同三司、守侍中、監修國史、兼真定尹、成德軍節度使,依前樞密使、太原郡侯,仍賜鐵券。乙未,詔宰相豆盧革權判吏部上銓鳩魯」。,御史中丞李德休權判東西銓事。丙申,滑州留後、檢校太保段凝可依
 前滑州留後,仍賜姓,名繼欽。以金紫光祿大夫、檢校司空、守輝州刺史杜晏球為檢校司徒,依前輝州刺史,仍賜姓,名紹虔。詔處斬隨駕兵馬都監夏彥朗于和景門外。時宦官怙寵,廣侵占居人第舍,郭崇韜奏其事,乃斬彥朗以徇。



 丁酉,賜百官絹二千匹、錢二百萬,職事絹一千匹、錢百萬。戊戌,以竭忠啟運匡國功臣、天平軍節度使、開府儀同三司、檢校太傅、兼侍中、蕃漢馬步總管副使、隴西郡侯李嗣源為依前檢校太傅、兼中書令、天平
 軍節度使、特進,封開國公,加食邑實封,餘如故。以開府儀同三司、檢校太傅、北都留守、興聖宮使、判六軍諸衛事李繼岌為檢校太尉、同平章事,充東京留守。詔御史臺,班行內有欲求外職,或要分司,各許于中書投狀奏聞。



 己亥,宴勳臣于崇元殿,梁室故將咸預焉。帝酒酣,謂李嗣源曰:「今日宴客,皆吾前日之勍敵,一旦同會,皆卿前鋒之力也。」梁將霍彥威、戴思遠等皆伏陛叩頭,帝因賜御衣、酒器,盡歡而罷。齊州刺史孟璆上章請死,詔原
 之。璆初事帝為騎將,天祐十三年,帝與劉鄩莘縣對壘,璆領七百騎奔梁,至是來請罪。帝報之曰:「爾當吾急,引七百騎投賊,何面目相見!」璆惶恐請死,帝恕之。未幾,移貝州刺史。



 庚子,帝畋于汴水之陽。十一月辛丑朔,有司奏:「河南州縣見使偽印,望追毀改鑄。」從之。以光祿大夫、檢校太傅、左金吾上將軍兼領左龍武軍事、汾州刺史李存渥為滑州節度使,加特進、同平章事;以雜指揮散員都部署、特進、檢校太傅、忻州刺史李紹榮為徐州節度使;
 以滑州兵馬留後、檢校太保李紹欽為兗州節度使。壬寅,鳳翔節度使、秦王李茂貞遣使賀收復天下。癸卯,河中節度使、西平王朱友謙來朝。乙巳,賜友謙姓,改名繼麟,帝令皇子繼岌兄事之。以捧日都指揮使、博州刺史康延孝為鄭州防禦使、檢校太保,賜姓,名繼琛。以宋州節度使、檢校太尉、平章事袁象先依前為宋州節度使,仍賜姓,名紹安。以許州匡國軍節度使、檢校太尉、同平章事溫韜依前許州節度使,仍賜姓,名紹沖。



 丁未,日南
 至,帝不受朝賀。戊申,中書門下上言:「以朝廷兵革雖寧,支費猶闕,應諸寺監各請置卿、少卿監、祭酒、司業各一員,博士二員,餘官並停。唯太常寺事關大禮,大理寺事關刑法,除太常博士外,許更置丞一員,其王府及東宮官、司天五官正、奉御之屬,凡關不急司存,並請未議除授。其諸司郎中、員外應有雙曹者,且置一員。左右常侍、諫議大夫、給事中、起居郎、起居舍人、補闕、拾遺,各置一半。三院御史仍委御史中丞條理申奏。其停罷朝官,仍
 各錄名銜,具罷任時日,留在中書,候見任官滿二十五個月,並據資品,卻與除官。其西班上將軍已下,仍望宣示樞密院斟酌施行。」從之。時議者以中興之朝,事宜恢廓,驟茲自弱,頓失物情。己酉,詔:應隨處官吏、務局員僚、諸軍將校等,如聞前例,各有進獻,直貢章奏,不唯褻黷於朝廷,實且傍滋于誅斂,並宜止絕,以肅化風。又詔:左降均州司馬劉岳,有母年踰八十,近聞身故,準故事許歸,候三年喪服闋,如未量移,即卻赴貶州。



 壬子,詔取今月
 二十四日幸洛京,以十二月二十三日朝獻太微宮,二十四日朝獻太廟,二十五日有事于南郊。癸未,中書門下奏:「應隨駕及在京有帶兼官者,並望落下,只守本官。」從之。乙卯,以特進、檢校太傅、開封尹、判六軍諸衛事、充功德使王瓚為宣武軍節度副使,權知軍州事。丁巳,以銀青光祿大夫、尚書左丞趙光允為中書侍郎、平章事、集賢殿大學士;以朝散大夫、禮部侍郎韋說守本官、同平章事;以吏部侍郎、史館修撰、判館事盧文度為兵部侍
 郎,充翰林學士;以右散騎常侍、充宏文館學士、判館事馮錫嘉為戶部侍郎、知制誥,充翰林學士;以翰林學士、守尚書膳部員外郎劉昫為比部郎中、知制誥,依前充職;以扈鑾書制學士、行尚書倉部員外郎趙鳳為倉部郎中、知制誥,充翰林學士;以左拾遺于嶠守本官,充翰林學士。戊午,以中書侍郎、平章事豆盧革判租庸使,兼諸道鹽鐵、轉運等使。新羅王金朴英遣使貢方物。



 己未,以洛京留守、判六軍諸衛事、守太尉、兼中書令、河南尹、魏
 王張全義為檢校太師、守中書令,餘如故;以荊南節度使、檢校太師、守中書令、渤海王高季興依前檢校太師、守中書令,餘如故。庚申,以工部尚書、真定尹、北都副留守、知留守事任圜為檢校吏部尚書、兼御史大夫以「符命」圖讖為合法根據。東漢時達到極盛。漢末衰微。讖,充成德軍節度使行軍司馬,知軍府事。安義軍節度使李繼韜入見待罪,詔釋之。辛酉,以宣化軍留後、檢校太傅戴思遠權知青州軍州事,檢校司空、左監門上將軍安崇阮並檢校舊官,卻復本任;以鎮國軍留後、檢校太傅霍
 彥威為保義軍節度留後;以權知威化軍留後、檢校司徒高允貞權知鎮國軍留後;以權知河陽留後、檢校太保張繼業依前權知河陽留後;以鄜延兩鎮節度使、檢校太師、兼中書令、西平王高萬興依前鄜、延節度使,仍封北平王;襄州節度使、檢校太傅、平章事孔勍依前襄州節度使,餘如故。以永平軍節度使、行大安尹、檢校太保張筠為西都留守、行京兆尹;以晉州節度使、檢校太保劉,邠州節度使、檢校太保韓恭,安州節度使、檢校
 太保硃漢賓,並檢校舊官,卻復本任。壬戌,以左金吾衛大將軍史敬熔為左街使,右金吾衛大將軍李存確為右街使。


甲子,車駕發汴州。十二月庚午朔,車駕至西京。是日,有司自石橋具儀仗法物,迎引入於大內。辛未,以百官初到,放三日朝參。壬申,以租庸使、刑部侍郎、太清宮副使張憲為檢校吏部尚書、充北京副留守、知留守事、太原尹。詔改取來年二月一日行郊禮。戊寅,詔德勝寨、莘縣、楊劉口、通津鎮、胡柳陂皆戰陣之所,宜令逐處
 差人收掩戰士骸骨,量備祭奠,以慰勞魂。詔改偽梁永平軍大安府復為西京京兆府;改宋州宣武軍為歸德軍,汴州開封府復為宣武軍,華州感化軍為鎮國軍,許州匡國軍復為忠武軍,華州宣義軍復為義成軍,陜府鎮國軍復為保義軍,耀州靜勝軍復為順義軍,潞州匡義軍復為安義軍,朗州武順軍復為武貞軍,延州為彰武軍,鄧州為威勝軍,晉州為建雄軍,安州為安遠軍。淮南楊溥遣使賀登極,稱「大吳國主書上大唐皇帝」。
 \gezhu{
  《十國春秋·
  吳世家》云:唐以滅梁來告,始稱詔,我國不受,唐主隨易書,用敵國禮,曰「大唐皇帝致書于吳國主」,王遣司農卿盧蘋獻金器二百兩、銀器三千兩、羅錦一千二百疋、龍腦香五斤、龍鳳絲鞵一百事于唐。又遣使張景報聘,稱「大吳國主上書大唐皇帝」,辭禮如箋表。}
 己卯,禁屠牛馬。



 庚辰,御史臺上言:「請行用本朝律令格式,今訪聞唯定州有本朝法書,望下本州寫副本進納。」從之。辛巳,詔貶安義軍節度使李繼韜為登州長史,尋斬于天津橋下,再謀叛故也。甲申,淮南楊溥、奚首領李紹威並遣使朝貢。乙酉,以翰林學士承旨盧質權知汴州軍府事,以禮部尚書崔沂為尚
 書左丞、判吏部尚書銓事,以兵部侍郎崔協為吏部侍郎,以刑部侍郎、充集賢殿學士、判院事盧文紀為尚書兵部侍郎,依前充集賢殿學士、判院事。



 丁亥,澤州刺史董璋上言:潞州軍變,李繼達領兵出城,自刎而死,節度副使李繼珂已安撫軍城。己丑,有司上言:「上辛祈穀于上帝,請奉高祖神堯皇帝配;孟夏雩祀,請奉太宗文皇帝配;季秋大享于明堂,請奉太祖武皇帝配;冬至日祀圜丘,請奉獻祖文皇帝配;孟冬祭神州地祇,請奉懿祖昭
 聖皇帝配。」從之。


辛卯,亳州太清宮道士上言,聖祖殿前古檜萎瘁已久再生一枝,圖畫以進。詔曰:「當聖祖舊殿生枯檜新枝,應皇家再造之期,顯大國中興之運。同上林仆柳,祥既葉于漢宣;比南頓嘉禾,瑞更超於光武。宜標史冊,以示寰瀛」云。
 \gezhu{
  《五代會要》云:唐高祖神堯皇帝武德二年,枯檜重華,至安祿山僭號萎瘁。明皇自蜀歸京,枝葉復盛。至是再生一枝,長二尺餘。}
 壬辰,幸伊闕。己巳,以中書舍人崔居儉為刑部侍郎,充史館修撰、判館事。甲午,以租庸副使、光祿大夫、檢校司徒、守衛尉卿孔謙為鹽鐵
 轉運副使。



\end{pinyinscope}