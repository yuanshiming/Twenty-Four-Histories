\article{列傳二 宗室}

\begin{pinyinscope}

 克
 讓,武皇之仲弟也。少善騎射,以勇悍聞。咸通中,從討龐勛,以功為振武都校。乾符中,王仙芝陷荊、襄,朝廷徵兵,克讓率師奉詔,賊平,以功授金吾將軍,留宿衛。初,懿
 祖歸朝,憲宗賜宅于親仁坊,自長慶以來,相次一人典衛兵。武皇之起雲中,殺段文楚,朝議罪之,命加兵于我,懼,將逃歸,天子詔巡使王處存夜圍親仁坊捕克讓。詰旦兵合,克讓與紀綱何相溫、安文寬、石的歷十餘騎彎弧躍馬,突圍而出。官軍數千人追之,比至渭橋,死者數百。克讓自夏陽掠船而濟,歸于鴈門。明年,武皇昭雪,克讓復入宿衛。黃巢犯闕,僖宗幸蜀,克讓時守潼關,為賊所敗,以部下六七騎伏于南山佛寺,夜為山僧所害。



 克
 讓既死,紀綱渾進通冒刃獲免,歸于黃巢。中和二年冬,武皇入關討賊,屯沙苑。黃巢遣使米重威齎賂修好時,盡焚圖讖書籍;秘府藏本,今亦多亡佚。,因送渾進通至,兼擒送害克讓僧十人。武皇燔偽詔,還其使,盡誅諸僧,為克讓發哀行服,悲慟久之。



 克修,字崇遠,武皇從父弟也。父德成,初為天寧軍使,從獻祖討龐勛學說,以功授朔州刺史。克修少便弓馬,從父征討,所至立功,武皇節制鴈門,以克修為奉誠軍使,從入關為前鋒,破黃揆于華陰,敗尚讓于梁田坡,蹙黃巢于
 光順門,每戰皆捷,勇懾諸軍。賊平,授檢校刑部尚書,為左營軍使。其年十月,潞州牙將安居受來乞師,請復昭義軍,武皇遣大將賀公雅、李筠、安金俊等以兵從。與孟方立戰于銅鞮,不利,武皇乃令克修將兵繼進。是月,平潞州,斬其刺史李殷銳,乃表克修為昭義節度使。光啟二年九月,克修出師山東,收復邢、洺。十一月,拔故鎮。孟方立遣將呂臻來援,戰于焦崗,大敗之,擒呂臻,俘斬萬計,進拔武安、臨洺諸屬縣,乘勝進圍邢州。方立求援于鎮
 州,王鎔出師三萬援之,克修軍退。及李罕之來歸,武皇授以澤州刺史,與克修合勢進攻河陽,連歲出師,以苦懷、孟。十月,孟方立遣將奚忠信將兵三萬襲我遼州,克修設伏于遼之東山,大敗賊軍,擒忠信以獻。龍紀元年,武皇大舉以伐邢、洺,及班師,因撫封于上黨。克修性儉嗇,不事華靡,供帳饔膳,品數簡陋。武皇怒其菲薄,笞而詬之,克修慚憤發疾;明年三月,卒于潞之府第,時年三十一。莊宗即位,追贈太師。



 克修子二人,長曰嗣弼,次曰
 嗣肱。嗣弼初授澤州刺史,歷昭義、橫海節度副使,改海州刺史。天祐十九年,契丹犯燕、趙,陷涿郡,《遼史·太祖紀》:十二月癸亥,圍涿州,有白兔緣壘而上,是日破其郛。嗣弼舉家被俘,遷于幕庭。



 嗣肱,少有膽略,屢立戰功,夾城之役,從周德威為前鋒。時兄嗣弼為昭義副使,與嗣昭守城,兄弟內外奮戰,忠力威壯,感動三軍。潞圍既解,以功授檢校左僕射,入為三城巡檢,知衙內事。天祐七年,周德威援靈、夏,黨項阻道,音驛不通。嗣肱奉命自麟州渡河,應接德威,與黨項
 轉戰數十里,合德威軍。柏鄉之戰,嗣肱為馬步都虞候。明年,從莊宗會朱友謙于猗氏,改教練使,與存審援河中,敗汴軍于胡壁堡,獲將龐讓。十年,與存審屯趙州,擊汴人于觀津。時梁祖新屠棗強,其將賀德倫急攻蓚縣,率師五萬合勢營于蓚之西。嗣肱自下博率騎三百,薄晚與梁之樵芻者相雜,日既晡,入梁軍營門,諸騎相合,大噪,弧矢星發,虓闞馳突。汴人不知所為,營中擾,既暝,斂騎而退。是夜,梁祖燒營而遁,解蓚縣之圍。以功特授
 蔚州刺史、鴈門以北都知兵馬使。從平劉守光。十二年,改應州刺史,累遷澤、代二州刺史、石嶺以北都知兵馬使。十九年,新州刺史王郁叛入契丹。嗣肱進兵定媯、儒、武等三州,授山北都團練使。二十年春,卒于新州,時年四十五。



 克恭,武皇之諸弟也。龍紀中,為決勝軍使。大順初,潞帥李克修卒,克恭代為昭義節度使。性驕橫不法,未閑軍政。潞人素便克修之簡正,惡克恭之恣縱,又以克修非
 罪暴卒,人士離心。時武皇初定邢、洺三州,將有事于河朔,大搜軍實。潞州有後院軍,兵之雄勁者,克恭選其五百人獻于武皇,軍使安居受惜其兵。不悅。克恭令裨校李元審、安建、紀綱、馮霸部送太原,行次銅鞮縣,馮霸劫眾謀叛,殺都將劉杲、縣令戴勞謙,循山而南,比及沁水,有眾三千。武皇令李元審將兵擊之,與霸戰于沁水,不利,元審戰傷,收軍于潞。五月十五日,克恭視元審于孔目吏劉崇之第。是日,州將安居受引兵仗攻克恭,因風縱
 火,克恭、元審並遇害,州民推居受為留後。初,孟方立之亂,居受以澤、潞歸于武皇,至是孟遷以邢、洺納降,復任為牙將,居受懼其圖己,乃叛,殺克恭以結汴人。居受遣人召馮霸于沁水,霸不受命。居受懼,將奔歸朝廷,至長子,為野人所殺,傳首馮霸軍。霸乃引軍據潞州,自稱留後,求援于汴。武皇令康君立討之,汴將葛從周來援霸。九月,李存孝急攻潞州,汴軍夜遁,獲霸等誅之,武皇乃以康君立為昭義節度使。



 克寧,武皇之季弟也。初從起雲中,為奉誠軍使,赫連鐸之攻黃花城也,克寧奉武皇及諸弟登城,血戰三日,力盡備竭,殺賊萬計。燕軍之攻蔚州,克寧昆仲嬰城拒敵,晝夜輟寢食者旬餘。後從達靼入關,逐黃寇。凡征行無不衛從,于昆弟之間,最推仁孝,小心恭謹,武皇尤友愛之。及鎮太原,授遼州刺史,累至雲州防禦使。乾寧初,改忻州刺史,從入關討王行瑜,充馬步軍都將,以功授檢校司徒。天祐初,授內外都制置、管內蕃漢都知兵馬使、
 檢校太保,充振武節度使,凡軍政皆決于克寧。



 五年正月,武皇疾篤,克寧等侍疾,垂泣辭訣。克寧曰:「王萬一不諱,後事何屬?」因召莊宗侍側,謂克寧、張承業曰;「亞子累公等。」言終棄代。將發哀,克寧紀綱軍府,中外無嘩。初,武皇獎勵軍戎,多畜庶孽,衣服禮秩如嫡者六七輩,比之嗣王,年齒又長,各有部曲,朝夕聚謀,皆欲為亂。莊宗英察,懼及于禍,將嗣位,讓克寧曰:「兒年孤稚,未通庶政,雖承遺命,恐未能彈壓大事。季父勳德俱高,眾情推伏,且
 請制置軍府,候兒有立,聽季父處分。」克寧曰:「亡兄遺命,屬在我兒,孰敢異議者!兒但嗣世,中外之事,何憂不辦。」視事之日,率先拜賀。莊宗嗣位,軍民政事,一切委之,權柄既重,趣向者多附之。李存顥者,以陰計干克寧曰:「兄亡弟及,古今舊事,委父拜侄,理所未安,富貴功名,當宜自立,天與不取,後悔無及。」克寧曰:「公毋得不祥之言!,我家世立功三代,父慈子孝,天下知名,茍吾兄山河有託,我亦何求!公無復言,必斬爾首以徇。」克寧雖慈愛因心,
 而日為兇徒惑亂。群兇之妻復以此言乾克寧妻孟夫人,說激百端,夫人懼事泄及禍,屢讓克寧,由是愈惑。會克寧因事殺都虞候李存質,又請兼領大同節度,以蔚、朔為屬郡,又數怒監軍張承業、李存璋;由是知其有貳。近臣史敬鎔素與存顥善,盡知其事。敬鎔告貞簡太后曰:「存顥與管內太保陰圖叛亂,俟嗣王過其第即擒之,并太后子母,欲送于汴州。竊發有日矣。」莊宗召張承業、李存璋謂曰:「季父所為如此,無猶子之情,骨肉不可自
 相魚肉,吾即避路,則禍亂不作矣。」承業曰:「老夫親承遺託,言猶在耳。存顥輩欲以太原降賊,王乃何路求生?不即討除,亡無日矣。」因令吳珙、存璋為之備。二月二十日,會諸將于府第,擒存顥、克寧于坐。莊宗垂泣數之曰:「兒初以軍府讓季父,季父不忍棄先人遺命。今已事定,復欲以兒子母投畀豺虎,季父何忍此心!」克寧泣對曰:「蓋讒夫交構,吾復何言!」是日,與存顥俱伏法。克寧仁而無斷,故及於禍。案《新唐書·宰相世系表》:嗣昭,國昌有子四人:克恭、克儉、克用、克柔。是書《李嗣昭傳》云:武
 皇母弟代州刺史克柔之假子也。是克柔為武皇母弟。《新唐書·沙陀傳》:武皇有弟克勤,《通鑒》引《紀年錄》有兄克儉,而是書俱無傳,疑有闕文。



 史臣曰:昔武皇發跡于陰山,莊宗肇基于河朔,雖奄有天下,而享國日淺,眷言枝屬,空秀棣華,固未及推帝堯敦敘之恩,廣成王封建之義。自克讓而下,不獲就魯、衛之封,懋間、平之德也。況夭橫相繼,亦良可悲哉!



\end{pinyinscope}