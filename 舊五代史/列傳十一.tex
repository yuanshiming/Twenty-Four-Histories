\article{列傳十一}

\begin{pinyinscope}

 丁會,字道隱,壽州壽春人。父季。會幼放蕩縱橫,不治農產,恆隨哀挽者學紼謳,尤嗜其聲。既長,遇亂,合雄兒為盜,有志功名。黃巢渡淮,會從梁祖為部曲,梁祖鎮門,會
 歷都押衙。自梁祖誅宗權,併時溥,屠朱瑄,走朱瑾,會恆以兵從,多立奇功。文德中,表授懷州刺史,歷滑州留後、河陽節度使、檢校司徒。自河陽以疾致政於洛陽。梁祖季年猜忌,故將功大者多遭族滅,會陰有避禍之志,稱疾者累年。天復元年,梁祖奄有河中、晉、絳,乃起會為昭義節度使。昭宗幸洛陽,加同平章事。其年,昭宗遇弒,哀問至,會三軍縞素,流涕久之。時梁祖親討劉守文於滄州,駐軍於長蘆。三年十二月,王師攻會,居旬日,會以潞
 州歸於武皇。《北夢瑣言》:梁祖雄猜,疑忌功臣,忽謂敬翔曰;「吾夢丁會在前祗候,吾將乘馬欲出,圉人以馬就臺,忽為丁會跨之以出,時夢中怒,叱喝數聲,因驚覺,甚惡之。」是月,丁會舉潞州軍民歸河東矣。引見,會泣曰:「臣非不能守潞,但以汴王篡弱唐祚,猜嫌舊將,臣雖蒙保薦之恩,而不忍相從,今所謂吐盜父之食以見王也。」武皇納之,賜甲第於太原,位在諸將上。五年,汴將李思安圍潞州,以會為都招討使、檢校太尉。



 莊宗嗣王位,與會決謀,破汴軍於夾城。七年十一月,卒於大原。莊宗即位,追贈太師。有子七人,知沆為梁祖所誅,
 餘皆歷內職。



 閻寶,字瓊美,鄆州人。父佐,海州刺史。寶少事硃瑾為牙將,瑾之失守於兗也,寶與瑾將胡規、康懷英歸汴梁,皆擢任之。自梁祖陳師河朔,爭霸關西,寶與葛從周、丁會、賀德倫、李思安各為大將,統兵四出,所至立功,歷洺、隨、宿、鄭四州刺史。天祐六年,梁祖以寶為邢洺節度使、檢校太傅。莊宗定魏博,十三年,攻相、衛、洺、磁,下之,寶獨保邢州,城孤援絕。八月,寶以邢州降,莊宗嘉之,進位檢校
 太尉、同平章事,遙領天平國節度使、東南面招討等使,待以賓禮,位在諸將上,每有謀畫,與之參決。契丹之寇幽州也,周德威危急,寶與李存審從明宗擊契丹於幽州西北,解圍而還。胡柳之役,諸軍逗撓,汴軍登無石山,其勢甚盛。莊宗望之,畏其不敵,且欲保營。寶進曰:「王深入敵境,偏師不利,王彥章騎軍已入濮州,山下唯列步兵,向晚皆有歸志。我盡銳擊之,敗走必矣。今若引退,必為所乘,我軍未集,更聞賊勝,即不戰而自潰也。凡決勝
 料情,情勢已得,斷在不疑。今王之成敗,在此一戰,若不決勝,設使餘眾渡河,河朔非王有也,王其勉之!」莊宗聞之聳聽,曰:「微公幾失計。」即引騎大噪,奮槊登山,大敗汴人。十八年,張文禮殺王鎔叛,寶帥師進討。八月,收趙州,進渡滹水,擒賊黨張友順以獻。九月,進逼真定,結營西南隅。掘塹柵以環之,決大悲寺漕渠以浸其郛。十九年正月,契丹三十萬來援鎮州,前鋒至新樂,眾心憂之。寶見莊宗,指陳方略,軍情乃安。敵退,加檢校侍中。三月,城
 中饑,王處瑾之眾出城求食,寶縱其出,伏兵截擊之。饑賊大至,諸軍未集,為賊年乘;寶乃收軍退保趙州,因慚憤成疾,疽發背而卒,時年六十。同光初,追贈太師;晉天福中,追封太原郡王。



 有子八人,宏倫、宏儒皆位至郡守。



 符習,趙州昭慶縣人。少從軍,事節度使王鎔,積功至列校。自莊宗經略河朔,與鎔連衡,常令習率師從莊宗征討。鎔為張文禮所害,時習在德勝寨,文禮上書請習等歸鎮。習雨泣訴於莊宗曰:「臣本趙人,家世事王氏,故使
 嘗授臣一劍,俾臣平蕩兇寇。自聞變故,徒懷冤憤,欲以自剄,無益於營魂。且張文禮乃幽、滄叛將,趙王知人不盡,過意任使,致被反噬。臣雖不武,願在霸府血戰而死,不能委身於兇首。」莊宗曰:「爾既懷舊君之愛,可復仇乎?吾當助爾。」習等舉身投地,號慟感激,謝曰:「王必以故使輔翼之勞,雪其冤恥,臣不敢期師旅為助,但悉本軍可以誅其逆豎。」莊宗即令閻寶、史建瑭助習討文禮,乃以習為成德軍兵馬留後。及文禮誅,將正授節鉞,習不敢
 當其任,辭曰:「臣緣故使未葬,又無嗣息,臣合服斬縗,候臣禮制畢聽命。」及莊宗兼領鎮州,乃割相、衛二州置義寧軍,以習為節度使。習奏曰:「魏博六州,見係霸府,不宜遽有割隸。但授臣河南一鎮,臣自攻取。」乃授天平軍節度、東南面招討使。



 習有器度,性忠壯,自莊宗十年沿河戰守,習常以本軍從,心無顧望,諸將服其為人。同光初,以習為邢州節度。明年,移鎮青州。四年二月,趙在禮盜據魏州,習受詔以淄、青之師進討;至則會軍亂,習乃退
 軍渡河。明宗自鄴赴洛,遣使召之,習不時而至。既至,謁明宗於胙縣。霍彥威謂習曰:「主上所知者十人,公在其四,何猶豫乎!」習乃從明宗入汴。明宗即位,加兼侍中,令歸本鎮。屬青州守將王公儼拒命,復授天平軍節度使。《宋史·顏衎傳》:天成初,為鄒平令。符習初鎮天平,習武臣之廉慎者,以書告屬邑,毋聚斂為獻賀。衎未領書,以故規行之,尋為吏所訟,習遽召衎笞之,幕客軍吏,咸以為辱及正人,習甚悔焉,即表為觀察判官,且塞前事。四年,移汴州節度使。安重誨素不悅習,會汴人言習厚賦民錢,以代納槁,及納軍租,多收加耗,由是罷歸京師。《
 通鑒》:習自恃宿將,議論多抗安重誨,故重誨求其過,奏之。授太子太師致仕,求歸故里,許之,乃歸昭慶縣。明宗以其子令謙為趙州刺史。習飛揚痛飲,周游田里,不集朋徒,不過郡邑,如此累年,中風而卒。贈太師。



 子蒙嗣,位至禮部侍郎。



 烏震,冀州信都人也。少孤,自勤鄉校。弱冠從軍,初為鎮州隊長,以功漸升部將,與符習從征於河上,頗得士心。聞張文禮弒王鎔,志復主讎,雪泣請行。兵及恒陽,文禮執其母妻洎兒女十口誘之,不回,攻城日急。文禮忿
 之,咸割鼻斷腕,不絕於膚,放至軍門,觀者皆不忍正視。震一慟而止,憤激奮命,身先矢石。鎮州平,以功授震深、趙二州刺史。其性純質,以清直御下,在河北獨有政聲,移易州刺史,兼北面水陸轉運、招撫等使。契丹犯塞,漁陽路梗,震率師運糧,三入薊門,擢為河北道副招討,遙領宣州節度使,代房知溫軍於盧臺。及至軍,會戍兵龍晊所部鄴都奉節等軍數千人作亂,未及交印而遇害。明宗聞之,廢朝一日,詔贈太傅。震略涉書史,尤嗜《左
 氏傳》,好為詩,善筆札,凡郵亭佛寺,多有留題之跡。及其遇禍,燕、趙之士皆歎惜之。



 王瓚,故河中節度使重盈之諸子也。天復初,梁祖既平河中,追念王氏舊恩,辟瓚為賓佐。梁祖即位,歷諸衛大將軍、兗華兩鎮節度使、開封尹。貞明五年,代賀瑰統軍駐於河上。時李存審築壘於德勝渡。秋八月,瓚率汴軍五萬,自黎陽渡河,將掩擊魏州,明宗出師拒之。瓚至頓丘而旋,於楊村夾河築壘,架浮航,自滑饋運相繼。瓚嚴
 於軍法,令行禁止,然機略應變,則非所長。十一月,瓚率其眾觀兵於戚城,明宗以前鋒擊之,獲其將李立。十二月,邏騎報汴之饋糧千計,沿河而下,可掩而取之。莊宗遣徒兵五千,設伏以待之,使騎軍循河南岸西上,俘獲饋役數千。瓚結陣河曲,以待王師,既而兵合,一戰敗之。瓚眾走保南城,瓚以小舟北渡僅免。是日,獲馬千餘匹,俘斬萬級,王師乘勝徇地曹、濮。梁主以瓚失律,令戴思遠代還。



 及王師襲汴,時瓚為開封府尹。梁主聞王師將
 至,自登建國門樓,日夜垂泣,時持國寶謂瓚曰:「吾終保有此者,繫卿耳。」令瓚閱市人散徒,登城為備。洎明宗至封丘門,瓚開門迎降。翼日,莊宗御元德殿,瓚與百官待罪及進幣馬,詔釋之,仍令收梁主屍,備槥櫝權厝於佛寺,漆首函送於郊社。居數日,段凝上疏奏:「梁朝掌事權者趙巖等,並助成虐政,結怨於人,聖政惟新,宜誅首惡,以謝天下。」於是張漢傑、張漢融、張漢倫、張希逸、趙縠、硃珪等並族誅,家財籍沒。瓚聞諸族當法,憂悸失次,每出
 則與妻子訣別。郭崇韜遣人慰譬之,詔授宣武軍節度副使,知府事,檢校太傅如故。《歐陽史》云:瓚伏地請死,莊宗勞而起之曰:「朕與卿家世婚姻,然人臣各為主耳,復何罪邪!」因以為開封尹,遷宣武軍節度使。瓚心憂疑成疾,十二月卒。贈太子太師。



 瓚雖為治嚴肅,而慘酷有家世風。自歷守蕃鎮,頗能除盜,而明不能照下。及尹正京邑,委政於愛婿牙將辛廷蔚,曲法納賄,因緣為奸。初,汴人駐軍於河上,軍計不足,瓚請率汴之富戶,出助軍錢,賦取不均,人靡控訴,至有雉經者,又有富室致賂幸而免率者。
 及明宗即位,素知廷蔚之奸,乃勒歸田里。然瓚能優禮搢紳,抑挫豪猾,故當時士流皆稱仰焉。



 袁象先,宋州下邑人也。自稱唐中宗朝中書令、南陽郡王恕己之後。曾祖進朝,成都少尹,梁以象先貴,累贈左僕射。祖忠義,忠武軍節度判官,累贈司空。父敬初,太府卿,累贈司徒、駙馬都尉。敬初娶梁祖之妹,初封沛郡太君。開平中,追封長公主。貞明中,追封萬安大長公主。



 象先即梁祖之甥也。性寬厚,不忤於物,幼遇亂,慨然有憂
 時之意。象先嘗射一水鳥,不中,箭落水中,下貫雙鯉,見者異之。梁祖鎮夷門,象先起家授銀青光祿大夫、檢校太子賓客、兼御史中丞。景福元年,自檢校左省常侍,遷檢校工部尚書,充元從馬軍指揮使兼左靜邊都指揮使。乾寧五年,再遷檢校右僕射、左領軍衛將軍同正,充宣武軍內外馬步軍都指揮使。光化二年,權知宿州軍州事。天復元年,表授刺史,充本州團練、埇橋鎮遏都知兵馬使。會淮寇大至,圍迫州城,象先雘力御備,時援兵
 未至,頗懷憂沮。一日,登北城,憩其樓堞之上,怳然若寢,夢人告曰:「我陳璠也,嘗板築是城,舊第猶在,今為軍舍,可為我立廟,即助公陰兵。」象先納之。翼日,淮寇急攻其壘,梯沖角進,是日州城幾陷。頃之,有大風雨,居民望見城上兵甲無算,寇不能進,即時退去。象先方信鬼神之助,乃為之立祠,至今里人禱祝不輟。三年,權知洺州軍州事。天祐三年,授陳州刺史、檢校司空。是歲,陳州大水,民饑,有物生於野,形類蒲萄,其實可食,貧民賴焉。梁開
 平二年,授左英武軍使,再遷左神武、右羽林統軍。三年,轉右衛上將軍,封汝南縣男。四年,權知宋州留後,到任五月,改天平軍兩使留後。時鄆境再饑,戶民流散,象先即開倉賑恤,蒙賴者甚眾。五年,梁祖北征,以象先為鎮定東南行營都招討應接副使,進封開國伯。領兵攻蓚縣,不克而還。俄奉詔自鄆赴闕,鄆人遮留,毀石橋而不得進,乃自他門而逸。尋授左龍武統軍兼侍衛親軍都指揮使。乾化三年,與魏博節度使楊師厚合謀,誅朱友
 珪於洛陽。梁末帝即位,以功授檢校太保、同平章事,遙領洪州節度使、行開封尹、判在京馬步諸軍,進封開國公。四年,授青州節度使,加檢校太傅。未幾,移鎮宋州,加檢校太尉。象先在宋凡十年。



 初,梁祖領四鎮,擁兵十萬,威震天下,關東籓守,皆其將吏,方面補授,由其保薦,四方輿金輦璧,駿奔結轍,納賂於其庭。如是者十餘年,寢成風俗,籓侯牧守,下逮群吏,罕有廉白者,率皆掊斂剝下,以事權門。象先恃甥舅之勢,所至籓府,侵刻誅求尤
 甚,以此家財巨萬。莊宗初定河南,象先率先入覲,輦珍幣數十萬,遍賂權貴及劉皇后、伶官巷伯,居旬日,內外翕然稱之。



 初,梁將未復官資者,凡上章奏姓名而已。郭崇韜奏曰:「河南征鎮將吏,昭洗之後,未有新官,每上表章,但書名姓,未頒綸制,必負憂疑。」即日,復以象先為宋、亳、耀、輝、潁節度使,依前檢校太尉、平章事,仍賜姓,名紹安,尋令歸鎮。明年,以郊禮,象先復來朝。是時,制改宋州宣武軍為歸德軍,因侍宴,莊宗謂象先曰:「歸德之名,無
 乃著題否?」象先拜謝而退,即命歸鎮。其年夏,以疾卒於理所,年六十一。冊贈太師,周廣順中,贈中書令,追封楚國公。


象先二子,長曰正辭,歷衢、雄二州刺史。次曰
 \gezhu{
  山義}
 ,至周顯德中,終於滄州節度使。



 張溫,字德潤,魏州魏縣人也。始仕梁祖為步直小將,改崇明都校。貞明初,蔣殷以徐州叛,從劉鄩討平之,改左右捉生都指揮使。莊宗伐邢臺,獲之,用為永清都校,歷武州刺史、山後八軍都將。從莊宗襲契丹於幽州,收新州,
 歷銀槍效義都指揮使,再任武州刺史。同光初,契丹陷媯、儒、檀、順、平、薊六州,武州獨全,改授蔚州刺史。天成初,歷振武、昭武留後,尋授利州節度使,入為右衛上將軍。無幾,授洋州節度使、右龍武統軍,改雲州節制。清泰初,屯兵鴈門,逐契丹出塞,移鎮晉州,嬰疾而卒。詔贈太尉。



 李紹文,鄆州人,本姓張,名從楚。少事朱瑄為帳下,瑄敗,歸於梁祖,為四鎮牙校,累典諸軍。天祐八年,從王景仁戰,敗於柏鄉,紹文與別將曹儒收殘眾,退保相州。王師
 之攻魏州也,紹文率眾自黎陽將渡河。時汴人大恐,河無舟楫,紹文懼為王師所逼,乃剽黎陽、臨河、內黃至魏州,歸於莊宗。莊宗嘉納之,賜姓名,分其兩將三千人為左右匡霸軍旅,仍令紹文、曹儒分將之。從周德威討劉守光,進檢校司空,移將匡衛軍。十二年,授博州刺史,預破劉鄩於故元城,歷貝、隰、代三郡刺史,領天雄軍馬步副都將,屯於德勝。從閻寶討張文禮,為馬步軍都虞候。明宗收鄆州,以紹文為右都押衙、馬步軍都將,從破王
 彥章於中都。同光中,歷徐、滑二鎮副使,知府事。三年,從郭崇韜討西川,為洋州節度留後,領鎮江軍節度。天成初,為武信軍節度使,尋卒於鎮。



 史臣曰:昔丁會之事梁祖也,功既隆矣,禍將及矣,挺身北首,故亦宜然。然食人之祿,豈合如是哉!閻寶再降于人,夫何足貴焉。符習雪故主之沉冤,享通侯之貴位,乃趙之奇士也。烏震不憫其親,仁斯鮮矣,雖慕樂羊之跡,豈事文侯之宜。瓚洎象先而下,皆降將也,又何足以譏
 焉!






\end{pinyinscope}