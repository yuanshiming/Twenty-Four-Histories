\article{明宗紀五}

\begin{pinyinscope}

 天成三年春正月戊申朔,帝禦崇元殿受朝賀,仗衛如式。辛亥,前河陽節度使、檢校太傅、兼侍中孔勍以太子太師致仕。癸丑,詔取今月十七日幸鄴都。甲寅,以國子
 祭酒硃守素卒廢朝。丙辰,以鎮南軍節度使袁建豐卒廢朝,詔贈太尉。丁巳,詔曰:「朕聞堯、舜有恤刑之典,貴務好生;禹、湯申罪己之言,庶明知過。今月十七日,據巡檢軍使渾公兒口奏稱,有百姓二人,以竹竿習戰鬥之事。朕初聞奏報,實所不容,率爾傳宣,令付石敬瑭處置。今旦重誨敷奏,方知悉是幼童為戲,載聆讜議,方覺失刑,循揣再三,愧惕非一。亦以渾公兒誑誣頗甚,石敬瑭詳覆稍乖,致人枉法而殂,處朕有過之地。今減常膳十日,
 以謝幽冤。其石敬瑭是朕懿親,合施極諫,既茲錯誤,宜示省循,可罰一月俸。渾公兒決脊杖二十,仍銷在身職銜,配流登州。小兒骨肉,賜絹五十匹、粟麥各百石,便令如法埋葬。兼此後在朝及諸道州府,凡有極刑,並須子細裁遣,不得因循。」百僚進表稱賀。


己未,中書門下奏,國子祭酒,望令宰相兼判。乃詔崔協判之。
 \gezhu{
  《五代會要》載原奏云:祭酒之資,歷朝所貴,爰從近代,不重此官。況屬聖朝,方勤庶政,須宏雅道,以振時風。望令宰臣一員,兼判國子祭酒。}
 辛酉,以前潞州節度使毛璋為右金吾上將軍,以左驍衛
 上將軍華溫琪為右金吾大將軍,以春州刺史張虔釗為鄭州防禦使。契丹方陷平州。癸亥,詔應廟諱文字,只避正文,其偏旁文字,不用虧缺點畫。契丹遣使托諾巴摩哩等貢獻,帝遣指揮使奔托山押國信賜契丹王妻。戊辰,以隨駕馬軍都指揮使、富州刺史康義誠兼領鎮南軍節度使;以隨駕步軍都指揮使、潮州刺史楊漢章遙領寧國軍節度使。中書上言:「舊制,遇二月十五日為聖祖降聖節,休假三日。準會昌元年二月敕,休假
 一日,請準近敕。」從之。吐蕃伊埒雅遜等六人、回鶻米里都督等四人,並授歸德、懷遠將軍,悉放還蕃。庚午,冊贈故瀛州刺史李嗣頵為太尉。壬申,冊贈故皇子檢校司空從諲為太保。甲戌,制以楚國夫人曹氏為淑妃,以韓國夫人王氏為德妃,仍令所司擇日冊命。



 二月丁丑朔,有司上言,太陽合虧,既而有雲不見,群官表賀。詔巡幸鄴都宜停。庚辰,偽吳楊溥遣使貢獻,賀誅硃守殷。帝以荊南拒命,通連淮夷,不納其使,遣還。壬午,以光祿卿韋
 寂卒廢朝,贈禮部尚書。癸未,工部尚書盧文紀貶石州司馬,員外安置。文紀私諱「業」,時新除于鄴為工部郎中。舊例,僚屬名與長官諱同,或改其任。文紀素與宰相崔協有隙,故中書未議改官。於鄴授官之後,文紀自請連假。鄴尋就位,及差延州官告使副未行,文紀參告,且言侯鄴回日終請換曹,鄴其夕遂自經而死,故文紀貶官。以倉部郎中何澤為吏部郎中,獎伏閣諫巡幸鄴都也。丁亥,天德軍節度使郭承豐加檢校司徒。辛卯,以山南西
 道節度使張筠為左驍衛上將軍。詔中外群臣父母亡沒者,並與追封贈。癸巳,以禮部尚書崔貽孫卒輟朝。甲午,以吐渾寧朔、奉化兩府都知兵馬使李紹魯為吐渾寧朔府都督。乙未,以樞密使兼東都留守孔循為許州節度使兼東都留守,鄧州節度使高行珪移鎮安州,應州節度使李從璋移鎮滑州,滑州節度使盧文進移鎮鄧州。丁酉,以責授檀州刺史劉訓為右龍武大將軍。己亥,回鶻可汗仁喻遣都督李阿爾珊等貢獻。壬寅,以左
 金吾大將軍羅周敬為同州節度使。甲辰,以威塞軍節度使張廷裕卒廢朝,詔贈太保。以耀州團練使孫岳為閬州團練使,以左監門上將軍高允貞為右金吾衛大將軍,以右金吾衛大將軍華溫琪為左金吾衛大將軍。



 三月丁未朔,以久雨,詔文武百辟極言時政得失。丁巳,以邢州節度使王景戡為華州節度使,以前北京副留守李從溫為邢州節度使。己未,以宰臣鄭玨為開府儀同三司、左僕射致仕,加食邑五百戶。庚申,以前復
 州刺史翟章為新州威塞軍留後。中書奏:「孟夏薦饗,合宰相行事,在朝只有宰相二員,今東都留守孔循帶平章事,宜令攝太尉行事。」孔循稱:「使相有戎機,不當司祠祭重事。」癸亥,以前鎮州節度使王建立為右僕射兼中書侍郎、平章事、集賢殿大學士、判三司。西方鄴上言,收復歸州。以前鄭州刺史楊漢賓為洋州武定軍留後。戊辰,以前彰國軍節度副使陳皋為鳳州武興軍留後,以前蔡州刺史孫漢韶為應州彰國軍留後,以宣徽南院使
 范延光為樞密使,以宣徽北院使、判三司張延朗為宣徽南院使,以前冀州刺史婁繼英為耀州團練使,以懷州刺史張廷蘊為金州防禦使。己巳,命范延光權知鎮州軍府事。西方鄴奏,于歸州殺敗荊南賊軍數千人。時有太白山道士解元龜自西川至,對于便殿,稱年一百一歲。既而上表乞西都留守兼西川制置使,要修西京宮闕。帝謂侍臣曰:「此人老耄,自遠來朝,方期別有異見,反為身名,甚可笑也。」賜號為知白先生,賜紫,放歸山。甲
 戌,冊回鶻可汗仁喻為順化可汗。



 夏四月戊寅,以汴州節度使石敬瑭為鄴都留守,充天雄軍節度使,加同平章事;以樞密使、權知鎮州軍府事、檢校太保範延光為鎮州節度使兼北面水陸轉運使;以司農卿鄭繢為太僕卿。壬午,夔州節度使、東南面副招討使西方鄴加檢校太保。甲申,皇第三女石氏封永寧公主,第十三女趙氏封興平公主,仍令所司擇日冊命。幽州上言,契丹有書求樂器。乙酉,達靼遣使朝貢。以隨駕馬軍都指揮使
 康義誠為侍衛親軍馬步軍都指揮使。丙戌,樞密使安重誨兼河南尹;以皇子河南尹、判六軍諸衛事從厚為汴州節度使,判六軍如故。丁亥,復州奏,湖南大破淮賊于道人磯。以西川馬步軍都指揮使趙廷隱兼漢州刺史,從孟知祥之請也。《九國志·趙廷隱傳》:知祥至蜀,康延孝陷漢州,遣廷隱率兵擊破之,擒延孝,檻送闕下。知祥奏加檢校司空、漢州刺史,遂留屯成都。洋州上言,重開入蜀舊路三百餘里,比今官路較二十五程而近。癸巳,殿中少監石知訥貶憲州司戶,坐扇惑軍鎮也。北面副招討、宋
 州節度使王晏球以定州節度使王都反狀聞。庚子,制義武軍節度使、檢校太尉、兼中書令、太原王王都削奪官爵。壬寅,以王晏球為北面行營招討使,知定州行軍州事;以滄州節度使兼北面行營馬軍都指揮使安審通為副招討使兼諸道馬軍都指揮使;以左散騎常侍蕭希甫兼判大理卿事。西京奏,前樞密使張居翰卒。



 五月乙巳朔,回鶻可汗仁喻封順化可汗。丁未,鄴都留守、天雄軍節度使石敬瑭,河陽節度使趙延壽並加駙馬
 都尉。以右僕射李琪為太子少傅。辛亥,沙州節度使曹義金加爵邑。王晏球上言,收奪得定州北西二關城。癸丑,湖南馬殷奏,二月中,大破淮寇二萬,生擒將士五百餘人。中書上言:「諸道薦人,總與不可,全阻又難。今後節度使每年許薦二人,帶使相者許薦三人,團練、防禦使各一人,節度、觀察判官並聽旨授,書記已下即許隨府。」從之。以六軍判官、尚書司封郎中史圭為右諫議大夫,充樞密直學士。詔州縣官以三十月為考限,刺史以二十
 五月為限,以到任日為始。己未,幽州奏,契丹託諾領二千騎西南趨定州。以前同州節度使盧質行兵部尚書,判太常卿事。辛酉,以天雄軍節度副使、判興唐府事趙敬怡為樞密使。詔曰:「上柱國,勳之極也。近代已來,文臣官階稍高,便授柱國,歲月未深,便轉上柱國。武資初官,便授上柱國。今後凡加勛,先自武騎尉,十二轉方授上柱國,永作成規,不令踰越。」丁卯,鎮州奏,今月十八日,王師不利于新樂。壬申,王晏球奏,今月二十一日,大破定州
 賊軍及契丹于曲陽,斬獲數千人,王都與托諾以數十騎復入于定州。



 六月己卯,以右金吾上將軍毛璋為左金吾上將軍,以前安州節度使史敬鎔為右金吾上將軍,以前華州節度使劉彥琮為左武衛上將軍。壬午,放內園鹿七頭于深山。乙酉識而無力改造。提出人性為「善惡混」,修其善為善人,修其,皇子故金槍指揮使、檢校左僕射從璟贈太保。己丑,幽州趙德鈞奏,殺契丹千餘人於幽州東,獲馬六百匹。壬辰,宰臣馮道率百僚上表,請上尊號曰聖明神武文德恭孝皇帝,詔報不允。丙申,馮
 道等再上尊號,不允。戊戌,以西京副留守、知留守事張遵誨行京兆尹。



 秋七月乙巳,詔故偽蜀主王衍追封順正公,以諸侯禮葬。丙午,以前武信軍節度使李敬周為邠州節度使。丁未,以滄州節度使安審通卒於師輟朝。壬子,以朔方節度使韓洙卒廢朝。甲寅,王晏球奏,六月二十二日進攻逆城,將士傷者三千人。時晏球知城中有備,未欲急攻,硃宏昭、張虔釗切於立功,促攻賊壘,晏球不得已而進兵,遂致傷痍者眾。乙卯,以太子少保李
 茂勳卒輟朝。己未,詔弛曲禁,許民間自造,於秋苗上納徵曲價,畝出五錢。時孔循以曲法殺一家于洛陽,或獻此議,以為愛其人,便于國,故行之。宗正卿李紓除名,刑部侍郎馬縞貶綏州司馬,刑部員外郎李慎儀貶階州司戶。初,李紓差攝陵臺令張保嗣等各虛稱試銜,為奉先令王延朗所訟,大理寺斷以詐假官論,刑部詳覆,稱非詐假。大理執之,召兩司廷議,刑部理屈,故有是貶。紓續敕配隴州,徒一年。未幾,詔曰:「天下州府,例是攝官,皆結
 試銜,或因勘窮,便關詐假。已前或有稱試銜,一切不問,此後並宜禁止。」曹州刺史成景宏貶綏州司戶參軍,續敕長流宥州,尋賜自盡,坐受本州倉吏錢百緡也。壬戌,齊州防禦使曹廷隱以奏舉失實,配流永州,續敕賜自盡。甲子,王晏球奏,今月十九日契丹七千騎來援定州,王師逆戰于唐河北,大破之。案:《通鑒》:壬戌,王晏球破契丹於唐河。追至滿城,又破之,斬二千級,獲馬千匹。戊辰,詔福建節度使王延鈞依前檢校太師、守中書令,進封閩王。己巳,王晏球奏,此月二十一日,
 追契丹至易州,掩殺四十里,擒獲甚眾。故朔方節度使韓洙贈太尉。以兵部侍郎王權、御史中丞梁文矩並為吏部侍郎,以左諫議大夫呂夢奇為御史中丞。



 八月癸酉朔,以翰林學士守中書舍人李懌、劉煦並為戶部侍郎充職,以吏部侍郎劉岳守秘書監,以吏部侍郎韓彥惲守禮部尚書,以戶部侍郎歸藹守太子賓客,以戶部侍郎裴皞守兵部侍郎,以中書舍人張文寶守刑部侍郎。詔凡有姓犯廟諱者,以本望為姓。丁丑,以檢校尚書
 右僕射、守龍武大將軍劉訓為晉州節度使、檢校太傅。壬午,幽州趙德鈞奏,于府西邀殺契丹敗黨數千人,生擒首領特哩袞及其屬凡五十餘人。是時,官軍襲殺契丹,屬秋雨繼降,泥濘莫進,人饑馬乏,散投村落,所在村民持白梃毆殺之。德鈞出兵接於要路,惟奇峰嶺北有馬潛遁脫者數十餘,無噍類。帝致書諭其本國。辛卯,以朔方軍留後韓璞為朔方軍節度使、靈武雄警甘肅等州觀察使、檢校司徒。帝聞隨、鄧、復、郢、均、房之民,父母骨
 肉有疾,以長竿遙致粥食而餉之,出嫁女,夫家不遣來省疾,乃下詔委長吏嚴加禁察。房州奏,新開山路四百里,南通夔州,畫圖以獻。以前洋州節度使戴思遠為太子太保致仕。庚子,詔:「今後翰林學士入院,以先後為班次,承旨一員,不計官資先後,在學士之上。」



 閏月丁未,兩浙節度觀察留後、清海軍節度使、檢校太師、兼中書令錢元瓘加杭州、越州大都督府長史,充鎮東、鎮海等軍節度使。戊申,趙德鈞獻戎俘于闕下,其蕃將特哩袞
 五十人留于親衛,餘契丹六百人皆斬之。乙卯,升楚州為順化軍。以明州刺史錢元珦為本州節度使,以吏部尚書蕭頃為太子少保。契丹遣使來貢獻。契丹平州刺史張希崇上表歸順。乙丑,陜州節度使李從敏移鎮滄州。以宣徽南院使張延朗為陜州節度使。詔:「在京遇行極法日,宜不舉樂,兼減常膳。諸州遇行極法日,禁聲樂。」己巳,滑州掌書記孟升匿母服,大理寺斷處流,特敕孟升賜自盡。觀察使、觀察判官、錄事參軍硃其糾察,各行
 殿罰。襄邑縣民聞威,父為人所殺,不雪父冤,有狀和解,特敕處死。是月二十七,大水,河水溢。絳州地震。



 九月乙亥,以捧聖左右廂副都指揮使索自通為雲州節度使。丁丑,以太府卿、判四方館事李鬱為宗正卿。壬午,以晉州節度使安崇阮為左驍衛上將軍。甲申,吐蕃、回鶻各遣使貢獻。壬辰,宰臣王建立進玉杯,上有文曰「傳國萬歲杯」。乙未,詔德州流人溫韜、遼州流人段凝、嵐州司戶陶、憲州司戶石知訥、原州司馬聶嶼,並宜賜死於本處,
 暴其宿惡而誅之也。丙申,以邠州節度使梁漢顒為右威衛上將軍。丁酉,河陽節度使、駙馬都尉趙延壽為檢校司徒。己亥,詔徐州節度使房知溫兼荊南行營招討使,知荊南行府事。



 冬十月甲辰,制瓊華長公主孟氏可冊為福慶長公主。丙午,以滄州節度使李從敏兼北面招討使。戊申,帝臨軒,命禮部尚書韓彥惲、工部侍郎任贊往應州奉冊四廟。詔邠州節度使李敬周攻慶州,以刺史竇廷琬拒命故也。戊午,契丹平州刺史張希崇
 已下八十餘人見于元德殿,頒賜有差。突厥首領張慕進等來朝貢。甲子,安州節度使高行珪奏,屯駐左神捷、左懷順軍士作亂,已逐殺出城。詔升壽州為忠正軍。戊辰,以雲州節度使索自通領壽州節度使,以前雲州節度使張溫復為雲州節度使。庚午夜,西南有彗星長丈餘,在牛星五度。



 十一月癸酉,日南至,帝御崇元殿受朝賀。甲戌,捧聖指揮使何福進招收到安州作亂兵士五百人,自指揮使已下至節級四十餘人並斬,餘眾釋之。
 壬午,房知溫奏,荊南高季興卒。中書舍人劉贊奏:「請節度使及文班二品已上謝見通喚。」從之。是日,以契丹所署平州刺史、光祿大夫、檢校太保張希崇為汝州刺史,加檢校太傅。己丑,中書奏:「今後或有封冊,請御正衙。」從之。青州奏,節度使霍彥威卒,輟朝三日。詔宰臣王建立權知青州軍州事。庚寅,禮部員外郎和凝奏:「應補齋郎並須引驗正身,以防偽濫。舊例,使蔭一任官補一人,今後改官須轉品即可,如無子,許以親侄繼限,念書十卷,
 試可則補。」從之。甲午,以尚書左僕射、同平章事、集賢殿大學士、判三司王建立為青州節度使、檢校太尉、同平章事。丙申,帝謂侍臣曰:「古鐵券如何?」趙鳳對曰:「帝王誓文,許其子子孫長享爵祿。」帝曰:「先朝所賜,惟朕與郭崇韜、李繼麟三人爾,崇韜、繼麟尋已族滅,朕之危疑,慮在旦夕。」于是嗟嘆久之。趙鳳曰:「帝王執信,故不必銘金鏤石矣。」吏部郎中何澤奏:「流外官請不試書判之類。」從之。吐蕃遣使朝貢。戊戌,前安州節度副使范延榮並男皆
 斬於軍巡獄,為高行珪誣奏故也。



 十二月壬寅朔,詔真定府屬縣宜準河中、鳳翔例升為次畿,真定縣升為次赤。甲辰,邠州節度使李敬周奏,收下慶州,刺史竇廷琬族誅。



\end{pinyinscope}