\article{列傳八}

\begin{pinyinscope}

 周德威,字鎮遠,小字陽五,朔州馬邑人也。初事武皇為帳中騎督,驍勇,便騎射,膽氣智數皆過人。久在雲中,諳熟邊事,望煙塵之警,懸知兵勢。乾寧中,為鐵林軍使,
 武皇討王行瑜,以功加檢校左僕射,移內衙軍副。



 光化二年三月,汴將氏叔琮率眾逼太原。有陳章者,以虓勇知名,眾謂之「夜叉」,言于叔琮曰:「晉人所恃者周陽五,願擒之,請賞以郡。」陳章嘗乘驄馬朱甲以自異。武皇戒德威曰:「我聞陳夜叉欲取爾求郡,宜善備之。」德威曰:「陳章大言,未知鹿死誰手!」他日致師,戒部下曰:「如陣上見陳夜叉,爾等但走。」德威微服挑戰,部下偽退,陳章縱馬追之,德威背揮鐵楇擊墮馬,生獲以獻,由是知名。



 天復中,我師
 不利于潘縣,汴將朱友寧、氏叔琮來逼晉陽。時諸軍未集,城中大恐,德威與李嗣昭選募銳兵出諸門,攻其壘,擒生斬馘,汴人枝梧不暇,乃退。天祐三年,與李嗣昭合燕軍攻潞州,降丁會,以功加檢校太保、代州刺史;代嗣昭為蕃漢都將。李思安之寇潞州也,德威軍於余吾。時汴軍十萬築夾城,圍潞州,內外斷絕,德威以精騎薄之,屢敗汴人,進營高河,令遊騎邀其芻牧。汴軍閉壁不出,乃自東南山口築甬道樹柵以通夾城,德威之騎軍,
 倒牆堙塹,日數十戰,前後俘馘,不可勝紀。梁有驍將黃角鷹、方骨崙,皆生致之。



 五年正月,武皇疾篤,德威退營亂柳,武皇厭代。四月,命德威班師。時莊宗初立,德威外握兵柄,頗有浮議,內外憂之。德威既至,單騎入謁,伏靈柩哭,哀不自勝,由是群情釋然。是月二十四日,從莊宗再援潞州。二十九日,德威前軍營橫碾,距潞四十五里。五月朔,晨霧晦暝,王師伏于三垂崗下。翼日,直趨夾城,斬關破壘,梁人大敗,解潞州之圍。初,德威與李嗣昭有
 私憾,武皇臨終顧謂莊宗曰:「進通忠孝不負我,重圍累年,似與德威有隙,以吾命諭之,若不解重圍,歿有遺恨。」莊宗達遺旨,德威感泣,由是勵力堅戰,竟破強敵,與嗣昭歡愛如初。以功加檢校太保、同平章事、振武節度使。



 七年,岐人攻靈夏,遣使來求助,德威渡河以應之;師還,授蕃漢馬步總管。七年十一月,汴人據深、冀,汴將王景仁軍八萬次柏鄉,鎮州節度使王鎔來告難,帝遣德威率前軍出井陘,屯于趙州。十二月,帝親征。二十五日,進
 薄汴營,距柏鄉五里,營於野河上。汴將韓勍率精兵三萬,鎧甲皆被繒綺,金銀炫曜,望之森然,我軍懼形於色。德威謂李存璋曰:「賊結陣而來,觀其形勢,志不在戰,欲以兵甲耀威耳。我軍人乍見其來,謂其鋒不可當,此時不挫其銳,吾軍不振矣!」乃遣存璋諭諸軍曰:「爾見此賊軍否?是汴州天武健兒,皆屠沽傭販,虛有表耳,縱被精甲,十不當一,擒獲足以為資。」德威自率精騎擊其兩偏,左馳右決,出沒數四。是日,獲賊百餘人,賊渡河而退。德
 威謂莊宗曰:「賊驕氣充盛,宜按兵以待其衰。」莊宗曰:「我提孤軍,救難解紛,三鎮烏合之眾,利在速戰,卿欲持重,吾懼其不可使也。」德威曰:「鎮、定之士,長於守城,列陣野戰,素非便習。我師破賊,惟恃騎軍,平田廣野,易為施功。今壓賊營,令彼見我虛實,則勝負未可必也。」莊宗不悅,退臥帳中。德威患之,謂監軍張承業曰:「王欲速戰,將烏合之徒,欲當劇賊,所謂不量力也。去賊咫尺,限此一渠水,彼若早夜以略彴渡之,吾族其為俘矣。若退軍鄗邑,
 引賊離營,彼出則歸,復以輕騎掠其芻餉,不踰月,敗賊必矣。」承業入言,莊宗乃釋然。德威得降人問之,曰「景仁下令造浮橋數日」,果如德威所料。二十七日,乃退軍保鄗邑。



 八年正月二日,德威率騎軍致師于柏鄉,設伏于村塢間,令三百騎以壓汴營。王景仁悉其眾結陣而來,德威轉戰而退,汴軍因而乘之,至于鄗邑南。時步軍未成列,德威陣騎河上以抗之。亭午,兩軍皆陣,莊宗問戰時,德威曰:「汴軍氣盛,可以勞逸制之,造次較力,殆難與敵。
 古者師行不踰一舍,蓋慮糧餉不給,士有饑色。今賊遠來決戰,縱挾糗Я,亦不遑食。晡晚之後,饑渴內侵,戰陣外迫,士心既倦,將必求退。乘其勞弊,以生兵制之,縱不大敗,偏師必喪。以臣所籌,利在晡晚。」諸將皆然之。時汴軍以魏、博之人為右廣,宋、汴之人為左廣,自未至申,陣勢稍卻,德威麾軍呼曰;「汴軍走矣!」塵埃漲天,魏人收軍漸退。莊宗與史建瑭、安金全等因衝其陣,夾攻之,大敗汴軍,殺戮殆盡;王景仁、李思安僅以身免,獲將校二百
 八十人。



 八月,劉守光僭稱大燕皇帝。十二月,遣德威率步騎三萬出飛狐,與鎮州將王德明、定州將程嚴等軍進討。九年正月,收涿州,降刺史劉知溫。五月七日,劉守光令驍將單廷珪督精甲萬人出戰,德威遇于龍頭崗。初,廷珪謂左右曰:「今日擒周陽五。」既臨陣,見德威,廷珪單騎持槍躬追德威,垂及,德威側身避之,廷珪少退,德威奮楇南墜其馬,生獲廷珪,賊黨大敗,斬首三千級,獲大將李山海等五十二人。十二日,德威自涿州進軍良
 鄉、大城。守光既失廷珪,自是奪氣。德威之師,屢收諸郡,降者相繼。十年十一月,擒守光父子,幽州平。十二月,授德威檢校侍中、幽州盧龍等軍節度使。



 德威性忠孝,感武皇獎遇,嘗思臨難忘身。十二月,汴將劉鄩自洹水乘虛將寇太原,德威在幽州聞之性和宗教》等。,徑以五百騎馳入土門,聞鄩軍至樂平不進,德威徑至南宮以候汴軍。初,劉鄩欲據臨清以扼鎮、定轉餉之路,行次陳宋口,德威遣將擒數十人,皆倳刃于背,縶而遣之。既至,謂劉鄩曰:「周侍
 中已據宗城矣!」德威其夜急騎扼臨清,劉鄩乃入貝州。是時德威若不至,則勝負不可知也。


十四年三月,契丹寇新州,德威不利,退保範陽。
 \gezhu{
  《遼史·太祖紀》:神冊二年三月辛亥,攻幽州,節度使周德威以幽、並、鎮、定、魏五州兵拒戰于居庸關之西,戰於新州東,大破之,斬首三萬級。又,《通鑒》:契丹主帥眾三十萬,德威眾寡不敵,大為契丹所敗。}
 敵眾攻僅二百日,外援未至,德威撫循士眾,晝夜乘城,竟獲保守。十五年,我師營麻口渡,將大舉以定汴州。德威自幽州率本軍至。十二月二十三日,軍次胡柳陂。詰旦,騎報曰;「汴軍至矣!」莊宗使問戰備,
 德威奏曰:「賊倍道而來,未成營壘,我營柵已固,守備有餘,既深入賊疆,須決萬全之策。此去大梁信宿,賊之家屬,盡在其間,人之常情,孰不以家國為念?以我深入之眾,抗彼激憤之軍,不以方略制之,恐難必勝。王但按軍保柵,臣以騎軍疲之,使彼不得下營,際晚,糧餉不給,進退無據,因以乘之,破賊之道也。」莊宗曰:「河上終日挑戰,恨不遇賊,今款門不戰,非壯夫也!」乃率親軍成列而出,德威不獲已,從之。謂其子曰:「吾不知其死所矣!」莊宗
 與汴將王彥章接戰,大敗之。德威之軍在東偏,汴之游軍入我輜重,眾駭,奔入德威軍,因紛擾無行列。德威兵少,不能解,父子俱戰歿。先是,鎮星犯上將,星占者云,不利大將。是夜收軍,德威不至,莊宗慟哭謂諸將曰:「喪我良將,吾之咎也!」



 德威身長面黑,笑不改容,凡對敵列陣,凜廩然有肅殺之風。中興之朝,號為名將。及其歿也,人皆惜之。同光初,追贈太師。天成中,詔與李嗣昭、符存審配饗莊宗廟廷。晉高祖即位,追封燕王。



 子光輔,歷汾、汝
 州刺史。



 符存審,字德詳,陳州宛邱人,《歐陽史·義兒傳》,惟符存審不在其列,別自為傳。蓋存審子彥卿有女為宋太宗后,故存其本姓也。舊名存。父楚,本州牙將。存審少豪俠,多智算,言兵家事。乾符末,河南盜起,存審鳩率豪右,庇捍州里。會郡人李罕之起自群盜,授光州刺史,因往依之。中和末,罕之為蔡寇所逼,棄郡投諸葛爽;存審從至河陽,為小校,屢戰蔡賊有功。諸葛爽卒,罕之為其部將所逼,出保懷州,部下分散,存審乃歸于武皇。武皇
 署右職,令典義兒軍,賜姓名。



 存審性謹厚,寵遇日隆,自是武皇西征,存審常從,所至立功。從討赫連鐸,冒刃死戰,血流盈袖,武皇手自封瘡,日夕臨問。乾寧初,討李匡儔,存審前軍拔居庸關。明年,從討邠州,時邠之勁兵屯龍泉寨,四面懸崖,石壁險固,存審奮力拔之。師還,授檢校左僕射。副李嗣昭討李瑭于汾州,擒之,以功改左右廂步軍都指揮使。天祐三年,授蕃漢馬步副指揮使,與李嗣昭降丁會于上黨,從周德威破賊于夾城,加檢
 校司徒,授忻州刺史,領蕃漢馬步都指揮使。七年,加檢校太保,充蕃漢副總管。莊宗擊汴人于柏鄉,留存審守太原。三月,代李存璋戍趙州。九年,梁祖攻蓚縣,存審與史建瑭、李嗣肱赴援,屯下博橋,汴人驚亂,燒營而遁,以功遙領邢、洺、磁團練使。



 十二年,魏博歸款于莊宗,遣存審率前鋒據臨清,以俟進取。莊宗入魏,存審屯魏縣以抗劉鄩。六月,鄩營莘縣,存審與鎮、定之師營莘西三十里,一日數戰。八月,率師攻張源德於貝州。十三年二月,劉
 鄩自莘悉眾來襲我魏州,存審以大軍踵其後,戰於故元城,大敗汴人,從收澶、衛、磁、洺等州。秋,邢州閻寶降,授存審安國軍節度、邢洺磁等州觀察使。十月,戴思遠棄滄州,毛璋以城降,授存審檢校太傅、橫海軍節度使,兼領魏博馬步軍都指揮使。明年,就加平章事。



 十四年八月,將兵援周德威於幽州,敗契丹之眾。冬,破汴將安彥之于楊劉,諸軍進營麻口。時梁將謝彥章營行臺村,莊宗勇於接戰,每以輕騎當之,遇窘者數四。存審每俟其
 出,必叩馬諫曰:「王將復唐宗社,宜為天下自愛,搴旗挑戰,一劍之任,無益聖德,請責效於臣。古人不以賊遺君父,臣雖不武,敢不代君之憂。」莊宗及時回駕。十二月,戰于胡柳。晡晚之後,存審引所部銀槍效節軍,敗梁軍于土山下。是日辰巳間,周德威戰歿,一軍逗撓,梁軍四集,存審與其子彥圖冒刃血戰,出沒賊陣,與莊宗軍合。午後,師復集,擊敗汴人。



 十六年春,代周德威為內外蕃漢馬步總管,於德勝口築南北城以據之。七月,汴將王
 瓚自黎陽渡河寇澶州,存審拒戰,瓚退,營于楊村渡,控我上游。自是日與交鋒,對壘經年,大小凡百餘戰。



 十七年,汴將劉鄩攻同州,硃友謙求援于我,遣存審與李嗣昭將兵赴之。九月,次河中,進營朝邑。時河中久臣於梁,眾持兩端,及諸軍大集,芻粟暴貴,嗣昭懼其翻覆,將急戰以定勝負。居旬日,梁軍逼我營。會望氣者言,西南黑氣如鬥雞之狀,當有戰陣,存審曰:「我方欲決戰,而形于氣象,得非天贊歟!」是夜,閱其眾,詰旦進軍。梁軍來逆戰,大
 敗之,追斬二千餘級。自是梁軍保壘不出。存審謂嗣昭曰:「吾初懼劉鄩據渭河。偏師既敗,彼若退歸,懼我踵之;窮獸搏人,勿謂無事。可開其歸路,然後追奔。」乃令王建及牧馬於沙苑,劉鄩、尹皓知之,保眾退去,《歐陽史》:鄩以為晉軍且懈,乃夜遁去。存審追擊于渭河,又大敗之。遂解同州之圍。存審略地至奉先,謁諸帝陵,乃班師。



 十八年,王師討張文禮于鎮州,李嗣昭、李存進相次戰歿。十九年,遣存審率師進攻叛帥於城下,文禮之將李再豐陰送款於存審,我師中夜登城,
 擒文禮之子處球等,露布以獻。鎮州平,以功加檢校太傅、兼侍中。



 二十年正月,師還于魏州,莊宗出城迎勞,就第宴樂。無何,契丹犯燕薊,郭崇韜奏曰:「汴寇未平,繼韜背叛,北邊捍禦,非存審不可。」上遣中使諭之,存審臥病羸瘠,附奏曰:「臣效忠稟命,靡敢為辭,但痾恙纏綿,未堪祗役。」既而詔存審以本官充幽州盧龍節度使,自鎮州之任。同光初,加開府儀同三司、檢校太師、中書令、食邑千戶,賜號忠烈扶天啟運功臣。



 十月,平梁,遷都洛陽。存
 審以身為大將,不得預收復中原之功,舊疾愈作,堅求入覲尋醫,以情告郭崇韜。時崇韜自負一時,佐命之功,無出己右,功名事望,素在存審之下,權勢既隆,人士輻湊,不欲存審加於己上,每有章奏求覲,即陰沮之。存審妻郭氏泣訴于崇韜曰:「吾夫于國,粗效驅馳,與公鄉里親舊,公忍令死棄北荒,何無情之如是!」崇韜益慚戁。明年春,疾甚,上章懇切,乞生覲天顏,不許。存審伏枕而嘆曰:「老夫歷事二主,垂四十年,幸而遇今日天下一家,遠
 夷極塞,皆得面覲彤墀,射鉤斬祛之人,孰不奉觴丹陛,獨予壅隔,豈非命哉!」漸增危篤,崇韜奏請許存審入覲。四月,制授存審宣武軍節度使、諸道蕃漢馬步總管。詔未至,五月十五日卒于幽州官舍,時年六十三,遺命葬太原。存審遺奏陳敘不得面覲,詞旨淒惋。莊宗震悼久之,廢朝三日,贈尚書令。



 存審少在軍中,識機知變,行軍出師,法令嚴明,決策制勝,從無遺悔,功名與周德威相匹,皆近代之良將也。常戒諸子曰:「予本寒家,少小攜一
 劍而違鄉里,四十年間,位極將相。其間屯危患難,履鋒冒刃,入萬死而無一生,身方及此,前後中矢僅百餘。」乃出鏃以示諸子,因以奢侈為戒。



 存審微時,嘗為俘囚,將就戮於郊外,臨刑指危垣謂主者曰:「請就戮于此下,冀得壞垣覆尸,旅魂之幸也。」主者哀之,為移次焉。遷延之際,主將擁妓而飲,思得歌者以助歡。妓曰:「俘囚有符存審者,妾之舊識,每令擊節,以贊歌令。」主將欣然,馳騎而舍之;豈非命也!



 彥超,存審之長子也。少事武皇,累歷牙職。存審卒,莊宗以彥超為汾州刺史。同光末,魏州軍亂,詔彥超赴北京巡檢。先是,朝廷令內官呂、鄭二人在太原,一監兵,一監倉庫。及明宗入洛,皇弟存霸單騎奔河東,與呂、鄭謀殺彥超與留守張憲。彥超覺之,密與憲謀,未決,部下大噪,州兵畢集,張憲出奔。是夕,軍士殺呂、鄭、存霸于衙城。詰旦,聞洛城禍變,彥超告諭三軍。《宋史·張昭傳》云:昭為張憲推官,莊宗及難,聞鄴中兵士推戴明宗,憲部將符彥超合戍將應之。憲死,有害昭者,執之以送彥超。彥超曰:「推官正人,無得害之。」又
 逼昭為榜,安撫軍民。明宗又令其弟龍武都虞候彥卿馳騎安撫。六月,彥超入覲,明宗召見撫諭,尋授晉州留後。未行,會其弟前曹州刺史彥饒平宣武亂軍,明宗喜,召彥超謂之曰:「吾得爾兄弟力,餘更何憂,爾為我往河東撫育耆舊。」即授北京留守、太原尹。明年冬,移授昭義節度使。四年,授驍衛上將軍,改金吾上將軍。長興元年,授泰寧軍節度使,尋移鎮安州。



 彥超廝養中有王希全者,小字佛留,粗知書計,委主貨財,歲久耗失甚多,彥超止于訶譴
 而已。應順元年正月,佛留聞朝廷多事,因與任貨兒等謀亂。一夕,扣門言朝廷有急遞至,彥超出至事,佛留挾刃害之。詰旦,本州節度副使李端召州兵攻佛留等,殺之,餘眾奔淮南,擒彥超部將趙溫等二十六人誅之。彥超贈太尉。



 存審次子彥饒,《晉史》有傳。次彥卿,皇朝前鳳翔節度使、守太師、中書令,封魏王,今居於洛陽。次彥能,終於楚州防禦使。次彥琳,仕皇朝為金吾上將軍,卒於任。



\end{pinyinscope}