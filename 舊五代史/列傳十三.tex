\article{列傳十三}

\begin{pinyinscope}

 安
 金全,代北人。世為邊將,少驍果,便騎射。武皇時為騎將,屢從征討。莊宗之救潞州及平河朔特」。,皆有戰功,累為刺史,以老病退居太原。天祐中,汴將王檀率師三萬,乘
 莊宗在鄴,來襲并州。時城無備兵,敵軍奄至,監軍張承業大恐,計無所出,閱諸司丁匠,登陴禦捍。外攻甚急,金全遽出謂承業曰:「老夫退居抱病,不任軍事。然吾王家屬在此,王業本根之地,如一旦為敵所有,大事去矣。請以庫甲見授,為公備寇。」承業即時授之。金全被甲跨馬,召率子弟及退閒諸將,得數百人。夜出北門,擊賊于羊馬城內,梁人驚潰,由是退卻。俄而石君立自潞州至,汴軍退走。微金全之奮命,城幾危矣。莊宗性矜伐,凡大將
 立功,不時行賞,故金全終莊宗世,名位不進。明宗與之有舊,及登極,授金全同平章事,充振武軍節度使。在任二年,治民為政非所長,詔赴闕,俄而病卒。廢視朝二日。初,南北對壘,汴之游騎每出,必為金全所獲,故梁之偵邏者咸懼,目之為「安五道」,蓋比鬼將有五道之名也。



 子審琦等,皆位至方鎮,別有傳。



 審通,金全之猶子也。幼事莊宗,累有戰功,轉先鋒指揮使。同光初,為北京右廂馬軍都指揮使,屯奉化軍。四年
 春,赴明宗急詔,軍趨夷臒,為前鋒。天成初,授單州刺史,改齊州防禦使,兼諸道先鋒馬軍都指揮使。奉詔北征,從房知溫營于盧臺。會龍晊部下兵亂,審通脫身酒筵,奪般以濟,促騎士介馬,及亂兵南行,盡戮之,以功授檢校太傅、滄州節度使。圍王都于中山,躬冒矢石,為飛石所中而卒。贈太尉。



 安元信,字子言,代北人。父順琳,為降野軍使。元信以將族子,便騎射,幼事武皇,從平巢、蔡。光啟中,吐渾赫連鐸
 寇雲中,武皇使元信拒之,元信兵敗於居庸關。武皇性嚴急,元信不敢還,遂奔定州;王處存待之甚厚,用為突騎都校。乾寧中,處存子郜嗣位。時梁軍攻河朔三鎮,奔命不暇,梁將張存敬軍奄至城下,既無宿備,郜懼,挈其族奔太原,元信從之;武皇待之如初,用為鐵林軍使。梁將氏叔琮之攻河東也,別將葛從周自馬嶺入,元信伏于榆次,挫其前鋒。梁將李思安之攻上黨也,王師將壁高河,為梁軍所逼。別將秦武者,尤為難敵,元信與鬥,斃
 之。由是梁軍解去,城壘得立。武皇賜所乘馬及細鎧仗,遷突陣都將。莊宗嗣晉王位,元信從救上黨,破夾寨,復澤、潞,以功授檢校司空、遼州刺史,賜玉鞭名馬。柏鄉之役,日晚戰酣,元信重傷,莊宗自臨傅藥。其年,改檢校司徒、武州刺史,充內衙副都指揮使、山北諸州都團練副使。從莊宗定魏博,移為博州刺史。與梁對壘德勝渡,元信為右廂排陣使。未幾,為大同軍節度使。莊宗平定河南,移授橫海軍節度使。時契丹犯邊,元信與霍彥威從明
 宗屯常山。元信恃功,每對明宗以成敗勇怯戲侮彥威,彥威不敢答。明宗曰;「成由天地,不由於人。當氏叔琮圍太原,公有何勇!今國家運興,致我等富貴。」元興乃起謝,不復以彥威為戲。明宗即位,以元信嘗為內衙都校,尤厚待之,加同中書門下平章事。明年,移鎮徐州。王師之討高季興,襄帥劉訓逗撓軍期,移授元信山南東道節度使以代訓。歲餘,改歸德軍節度使,就加兼侍中。明宗不豫,求入。末帝即位,授潞州節度使,加檢校太尉。清泰
 三年二月,以疾卒于鎮,時年七十四。贈太師。晉高祖即位,以元信宿望,令禮官定謚曰忠懿。



 有子六人,長曰友權,歷諸衛大將軍。次曰友親,仕皇朝為滁州刺史,卒于任。



 安重霸,雲州人也。性狡譎,多智算。初,自代北與明宗俱事武皇,因負罪奔梁;在梁復以罪奔蜀,蜀以蕃人善騎射,因為親將。蜀後主王衍,幼年襲位,其政多僻。宦官王承休居中用事,與成都尹韓昭內外相結,專採擇聲色,
 以固寵幸。武臣宿將,居常切齒。重霸諂事承休,特見委信。梁末,岐下削弱,蜀人奪取秦、成、階等州,重霸說承休求鎮秦州。仍於軍中選山東驍果,得數千人,號龍武都,以承休為軍帥,重霸副焉,俱在天水。歲餘,承休欲求旄鉞,乃以隴西花木入獻,又稱秦州山水之美,人物之盛,請後主臨幸,而韓昭贊成之。《太平廣記》引《王氏見聞錄》云:承休請從諸軍揀選官健,得驍勇數千,號龍武軍,承休自為統帥,並特加衣糧,日有優給。因乞秦州節度使,且云:「願與陛下于秦州採掇美麗。」且說秦州風土,多出國色,仍請幸天水。少主甚悅,即遣仗節赴鎮,應所選龍武精銳,並充衙隊從行。



 同
 光二年十月,蜀主率眾數萬,由劍閣將出興、鳳,以遊秦州;至興州,遇魏王繼岌軍至,狼狽而旋。承休遽聞東師入討,大恐,計無從出,問于重霸。對曰;「開府何患?蜀中精兵,不下十萬,咫尺之險,安有不濟,縱東軍盡如狼虎,豈能入劍門!然國家有患,開府特受主知,不得失于奔赴,此州制置事定,無虞得失,重霸願從開府赴闕。」承休素信以為忠赤。重霸出秦州金帛以賂群羌,買由州山路歸蜀。承休擁龍武軍及招置僅萬人從行,令重霸權握部
 署,州人祖送,秦州軍亦列部隊。承休登乘,重霸馬前辭曰:「國家費盡事力,收獲隴西,若從開府南行,隴州即時疏失,請開府自行,重霸且為國守籓。」承休既去,重霸在秦州,聞明宗起河北,即時遣使以秦、成等州來降。天成初,用為閬州團練使。未幾,召還為左衛大將軍。常以姦佞揣人主意,明宗尤愛之。長興末,明宗謂侍臣曰:「安重霸朕之故人,以秦州歸國,其功不細,酬以團練防禦,恐非懷來之道。」范延光曰:「將校內有自河東、河北從陛下
 龍飛故人,尚有未及團防者,今若遽授重霸方鎮,恐為人竊議。」明宗不悅。未幾,竟以同州節鉞授之。清泰初,移授西京留守、京兆尹。先是,秦、雍之間,令長設酒食,私丐于部民者,俗謂之「搗蒜」。及重霸之鎮長安,亦為之,故秦人目重霸為「搗蒜老」。其年冬,改雲州節度。居無何,以病求代,時家寄上黨,及歸而卒。重霸善悅人,好賂遺,時人目之為俊。



 弟重進,尤凶惡。事莊宗,以試劍殺人,奔淮南。《玉堂閒話》:安重進,性凶險,莊宗潛龍時為小校,常佩劍列于翊衛。後攜劍南馳,投于梁祖,梁祖壯之,俾隸淮之
 鎮戍。復以射殺掌庾吏,逃竄江湖,淮帥得之,擢為裨將。重霸在蜀,聞之蜀主,取之于吳,用為裨將。隨重霸為龍武小將,戍長道,又殺人,奔歸洛陽。《太平廣記》云:蜀破,重進東歸,明宗補為諸州馬步軍都指揮使,後有過,鞭背卒。



 重霸之子曰懷浦,晉天福中,為禁軍指揮使。契丹寇澶州,以臨陣忸怩,為景延廣所誅。



 劉訓,字遵範,隰州永和人也。出身行間,初事武皇為馬軍隊長,漸至散將。屬河中王氏昆仲有尋戈之役,訓從史儼攻陜州。武皇討王行瑜,以訓為前鋒,後隸河中,為
 隰州防禦都將。居無何,殺陜州刺史,以郡歸莊宗,歷瀛州刺史。同光初,拜左監衛大將軍。三年,授襄州節度使。四年四月,洛陽有變,訓以私忿害節度副使胡裝,族其家,聞者冤之。天成中,荊南高季興叛,詔訓為南面行營招討使,知荊南行府事。是時,湖南馬殷請以舟師會,及王師至荊渚,殷軍方到岳州。仍傳意于訓,許助軍儲弓甲之類,久之,略無至者。案《通鑒》:劉訓至荊南,楚王殷遣都指揮使許德勳等將水軍屯岳州。高季興堅壁不戰,求救於吳,吳人遣水軍援之。荊渚地氣卑濕,漸及霖潦,糧
 運不繼,人多疾疫。訓本無將略,人咸苦之。及孔循至,得襄之小校獻竹龍之術,及造竹龍二道,傅於城下,竟無所濟。遂罷兵,令將士散略居民而回。詔訓赴闕,尋責授檀州刺史,續敕濮州安置,未幾,起為龍武大將軍,尋授建雄軍節度使,移鎮延平。卒贈太尉。



 張敬詢,勝州金河縣人,世為振武軍牙校。祖仲阮,歷勝州刺史,父漢環事武皇為牙將。敬詢當武皇時,專掌甲坊十五年,以稱職聞。復以女為武皇子存霸妻,益見親
 信。莊宗即位,以為沁州刺史,秩滿,復用為甲坊使。莊宗經略山東,敬詢從軍,歷博、澤、慈、隰四州刺史。同光末,授耀州團練使。郭崇韜之征蜀也,以敬詢善督租賦,乃表為利州留後。明宗即位,正授昭武軍節度使。天成二年,詔還京師,復授大同節度使,至鎮,招撫室韋萬餘帳。四年,徵為左驍衛上將軍。明年,授滑州節度使。以河水連年溢堤,乃自酸棗縣界至濮州,廣堤防一丈五尺,東西二百里,民甚賴之。三年,秩滿歸京,卒。輟視朝一
 日。



 劉彥琮,字比德,雲中人也。事武皇,累從征役。先是,絳州刺史王瓘叛,武皇言于彥琮,意欲致之。無幾,從略于汾、晉之郊,彥琮奔絳,瓘以為附己,待之甚厚,因命為騎將。會瓘出獵,于馳驅之際,彥琮刃瓘之首來獻,武皇甚奇之。從莊宗解上黨之圍。同光初,稍遷至鐵林指揮使、磁州刺史。後明宗赴難京師,授華州留後,尋正授節旄。天成三年,改左武衛上將軍。未幾,改陜州節度使,尋移鎮邠州,卒於鎮,時年六十四。贈太傅。



 袁建豐,武皇破巢時得于華陰,年方九歲,愛其精神爽俊,俾收養之。漸長,列于左右,復習騎射,補鐵林都虞候。從破邠州王行瑜,以功遷左親騎軍使,轉突騎指揮使。從莊宗解圍上黨,破柏鄉陣,累功遷右僕射、左廂馬軍指揮使。明宗為內衙指揮使,建豐為副。北討劉守光,常身先士伍,轉都教練使,權蕃漢副總管。莊宗入鄴,以心腹幹能,選為魏府都巡檢使。從征劉鄩,下衛、磁、洺有功,加檢校司空,授州洺州刺史。於臨洺西敗梁將王遷數千
 人,生獲將領七十餘人,俄拜相州刺史。征赴河上,豫戰于胡柳陂。建豐領相州軍士,行營在外,委州事于小人,失于撫馭,指揮使孟守謙據城以叛,建豐引兵討平之。改隰州刺史,染風痺于任。明宗嗣位,念及平昔副貳之舊,詔赴洛下,親幸其第,撫問隆厚,加檢校太傅,遙授鎮南節度使,俾請俸自給。後卒于洛陽,年五十六。廢朝一日,贈太尉。



 子可鈞,仕皇朝,位至諸衛大將軍。



 西方鄴,定州滿城人也。父再遇,為州軍校。鄴居軍中,以
 勇力聞。年二十,南渡河遊梁,不見用,復歸。莊宗以為孝義軍指揮使,累從征伐皆有功。同光中,為曹州刺史,以州兵屯汴州。明宗自魏州,南渡河,時莊宗東幸汴州。汴州節度使孔循懷二志,使北門迎明宗,西門迎莊宗,凡供帳委積悉如一,曰:「先至者入之。」鄴因責循曰:「主上破梁室于公,有不殺之恩,奈何欲納總管?」循不答。鄴度循不可理爭,以石敬瑭妻,明宗女也,時方在汴,欲殺之以堅人心。循知其謀,取之藏其家,鄴無如之何。
 乃將麾下兵五百騎西迎莊宗,見于汜水,嗚咽泣下,莊宗亦為之噓唏,使以兵為先鋒。莊宗還洛陽,遇弒。明宗入洛,鄴請死于馬前,明宗嘉歎久之。明年,荊南高季興叛,明宗遣襄州節度使劉訓等招討,而以東川董璋為西南招討使,乃拜鄴夔州刺史,副璋,以兵出三峽。已而訓等無功見黜,諸將皆罷,璋未嘗出兵,惟鄴獨取夔、忠、萬三州,乃以夔州為寧江軍,拜鄴節度使。已而又取歸州,數敗季興之兵。鄴,武人,所為多不中法度,判官譚
 善達數以諫鄴,鄴怒,遣人告善達受人金,下獄。善達素剛,辭益不遜,遂死于獄中。鄴病,見善達為祟,卒於鎮。



 張遵誨,魏州人也。父為宗城令,羅紹威殺牙軍之歲,為梁軍所害。遵誨奔太原,武皇以為牙門將。莊宗定山東,遵誨以典客從,歷幽、鎮二府馬步都虞候。同光中,為金吾大將軍。明宗即位,任圜保薦,授西都副留守,知留守事、京兆尹。天成四年,入為客省使、守衛尉卿。及將有事於南郊,為修儀仗法物使。初,遵誨自以歷位尹正,與安
 重誨素亦相款,衷心有望于節鉞。及郊禋畢,止為絳州刺史,鬱鬱不樂。離京之日,白衣乘馬于隼■之下,至郡無疾,翌日而卒。



 孫璋,齊州歷城人。出身行間,隸梁將楊師厚麾下,稍補奉化軍使。莊宗入鄴,累遷澶州都指揮使。明宗鎮常山,擢為裨校。鄴兵之變,從明宗赴難京師。天成初,歷趙、登二州刺史,齊州防禦使。王都之據中山,璋為定州行營都虞候,賊平,加檢校太保。長興初,授鄜州節度使,罷鎮,
 卒於洛陽,年六十一。贈太尉。



 史臣曰:夫天地斯晦,則帝王於是龍飛;雲雷構屯,則王侯以之蟬蛻。良以適遭亂世,得奮雄圖,故金全而下,咸以軍旅之功,坐登籓閫之位,垂名簡冊,亦可貴焉。惟重霸以奸險而仗旄鉞,蓋非數子之儔也。



\end{pinyinscope}