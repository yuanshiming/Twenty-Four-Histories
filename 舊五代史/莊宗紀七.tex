\article{莊宗紀七}

\begin{pinyinscope}
同光三年秋七月丁酉,以久雨,詔河南府依法祈晴。滑州上言,黃河決。壬寅,皇太后崩于長壽宮,帝執喪于內,出遺令以示于外。癸卯,帝于長壽宮成服,百官于長壽
 宮幕次成服後,于殿前立班奉慰。乙巳,宰臣上表請聽政,不允;表再上,敕旨宜廢朝七日。丁未,宏文館上言:「請依六典,改宏文館為崇文館。」從之。時樞密使郭崇韜亡父名宏,豆盧革希崇韜指,奏而改之。
 \gezhu{
  《五代會要》:同光三年敕云:崇文館比與宏文館並置,今請改稱,頗協舊典。蓋豆盧革曲為之說也。}
 洛水泛漲,壞天津橋,以舟濟渡,日有覆溺者。己酉,宰臣百官上表,請聽政;又請復常膳,表凡三上。以刑部尚書李琪充大行皇太后山陵禮儀使,河南尹張全義充山陵橋道排頓使,孔謙充
 監護使。壬子,河陽、陜州上言,河溢岸。以禮部尚書王正言為戶部尚書,以御史中丞崔協為禮部尚書,以刑部侍郎、史館修撰、判館事崔居儉為御史中丞,以尚書左丞歸靄為刑部侍郎。陜州上言,河漲二丈二尺,壞浮橋,入城門,居人有溺死者。乙卯,汴州上言,汴水泛漲,恐漂沒城池,于州城東西權開壕口,引水入古河。澤潞上言,自今月一日雨,至十九日未止。戊午,以刑部尚書、判太常卿兼判吏部尚書銓事李琪為吏部尚書,依前判太
 常卿;以兵部侍郎、集賢殿學士、判院事盧文紀為吏部侍郎;以給事中李光序為尚書右丞。許州、滑州奏,大水。



 八月壬戌,詔諸司人吏,不許諸處奏薦,如有勞績,只許本司奏聞。詔有司,吳越王印宜以黃金鑄成,其文曰「吳越國王之印」。丁卯,帝釋服,百官奉慰于長壽宮。戊辰,客省使李嚴使蜀回。初,帝令往市蜀中珍玩,蜀法嚴峻,不許奇貨東出,其許市者謂之「入草物」。嚴不獲珍貨,歸而奏之,帝大怒曰:「物歸中夏者命之曰『入草』,王衍寧免為
 入草之人耶!」由是伐蜀之意銳矣。庚辰,幸壽安山陵作所。鄴都大水,御河泛溢。癸未,河南縣令羅貫長流崖州,尋委河南府決痛杖一頓,處死,坐部內橋道不修故也。及死,人皆冤之。甲申,山陵禮儀使奏:「山陵封城之內,先有丘墳,合令子孫改卜。舊例給其所費,無子孫者官為瘞藏。如是五品以上官,所司仍以禮致祭。」從之。鳳翔奏,大水。己酉,中書門下上言:「據禮儀使狀,準故事,太常少卿定大行太后謚議,太常卿署定訖,告天地宗廟。伏準
 禮文:賤不得誄貴,子不得爵母,后必謚于廟者,受成于祖宗。今大行太后謚,請太常卿署定後,集百官連署謚狀訖,讀于太廟太祖皇帝室,然後差丞郎一人撰冊文,別定日,命太尉上謚冊于西宮靈座,同日差官告天地、太微宮、宗廟,如常告之儀。」從之。青州大水、蝗。己丑,以襄州留後李紹珙為襄州節度使,以邠州留後董璋為邠州節度使。



 九月辛卯朔,河陽奏,黃河漲一丈五尺。癸巳,中書上言:「大行皇太后謚議合讀于太廟太祖室,其日
 ,集兩省御史臺五品巳上、尚書省四品已上、諸司三品已上官,于太廟序立。」從之。鎮州、衛州奏,水入城,壞廬舍。乙未,制封第三子鄴都留守、興聖宮使、檢校太尉、同平章事、判六軍諸衛事繼岌為魏王。幸壽安陵。庚子,襄州奏,漢江漲溢,漂溺廬舍。是日,命大舉伐蜀,詔曰:



 朕夙荷丕基,乍平偽室,非不欲寵綏四海,協和萬邦,庶正朔以遐同,俾人倫之有序。其或地居陬裔,位極驕奢,殊乖事大之規,但蘊偷安之計,則必徵諸典訓,振以皇威,爰興
 伐罪之師,冀遏亂常之黨。蠢茲蜀主,世負唐恩,間者父總籓宣,任君統制,屬朱溫東離汴水,致昭皇西幸岐陽,不務扶持,反懷顧望,盜據劍南之土宇,全虧閫外之忳誠。先皇帝早在並門,將興霸業,彼既會馳書幣,此亦復展謝儀。後又特發使人,專持聘禮,彼則更不回一介之使,答咫尺之書,星歲俄移,歡盟頓阻。朕頃遵遺訓,嗣統列籓,追昔日之來誠,繼先皇之舊好,累馳信幣,皆絕酬還,背惠食言,棄同即異。今觀孽豎,紹據山河,委閹宦以
 持權,憑阻修而僭號。早者,曾上秦王緘札,張皇蜀地聲塵,形侮黷之言辭,謗親賢之勳德。昨朕風驅銳旅,電掃兇渠,復已墜之宗祧,纘中興之歷數。捷音旋報,復命仍稽,使來而尚抗書題,情動而先誇險固。加以宋光葆輒陳狂計,別啟姦謀,將欲北顧秦川,東窺荊渚,人而無禮,罪莫大焉。



 昨客省使李嚴奉使銅梁,近歸金闕,凡于奏對,備述端由。其宋光嗣相見之時,于坐上便有言說,先問契丹強弱,次數秦王是非,度此包藏,可見情狀。加以
 疏遠忠直,朋比姦雄。內則縱恣輕華,競貪寵位;外則滋彰法令,蠹耗生靈。既德力以不量,在神祇之共憤。今命興聖宮使、魏王繼岌充西川四面行營都統,命侍中、樞密使郭崇韜充西川東北面行營都招討制置等使,荊南節度使高季興充西川東南面行營都招討使,鳳翔節度使李嚴充供軍轉運應接等使,同州節度使李令德充行營招討副使,陜府節度使李紹琛充行營蕃漢馬步軍都排陣斬斫使,西京留守張筠充西川管內安
 撫應接使,華州節度使毛璋充行營左廂馬步都虞候,邠州節度使董璋充行營右廂馬步都虞候,客省使李嚴充西川管內招撫使,總領闕下諸軍,兼西面諸道馬步兵士,取九月十八日進發。凡爾中外,宜體朕懷。



 辛丑,授魏王繼岌諸道行營都統,餘如故。繼岌既受都統之命,以梁漢顒充中軍馬步都虞候兼馬步軍都指揮使,張廷蘊為中軍步軍都指揮使,牛景章充中軍左廂馬軍都指揮使,沈斌充中軍右廂馬軍都指揮使,卓瑰充
 中軍左廂步軍都指揮使,王贄充中軍右廂步軍都指揮使,供奉官李從襲充中軍馬步軍都監,高品李廷安、呂知柔充魏王衙通謁。詔工部尚書任圜、翰林學士李愚參魏王軍事。丁未夕,偏天陰雲,北方有聲如雷,野雉皆鳴,俗所謂「天狗落」。戊申,魏王繼岌、樞密使侍中郭崇韜進發西征。太子少師致仕薛廷珪卒,贈右僕射。甲寅,幸壽安陵。司天上言:「自七月三日大雨,至九月十八日後方晴,三辰行度不見。」丁巳,幸尖山射鴈。


冬十月庚申
 朔,宰臣及文武三品以上官赴長壽宮,上大行皇太后謚曰貞簡皇太后。辛酉,幸甘泉,遂幸壽安陵。壬戌,魏王繼岌率師至鳳翔,先遣使馳檄以諭蜀部。丁卯,奉皇太后尊謚寶冊赴西京錄座,宰臣豆盧革攝太尉讀冊文,吏部尚書李琪讀寶文,百官素服,班於長壽宮門外奉慰。淮南楊溥遣使進慰禮。己巳,中書上言:「貞簡太后陵請以坤陵為名。」從之。初卜山陵,帝欲祔于代州武皇陵,奏議:「天子以四海為家,不當分其南北。」乃于壽安縣界
 別卜是陵。
 \gezhu{
  《五代會要》載中書門下奏議云:「人君以四海為家,不當分其南北。洛陽是帝王之宅,四時朝拜,禮須便近,不能遠幸代州。今漢朝諸陵,皆近秦雍,國朝陵寢,布列京畿。後魏文帝自代遷洛之後,園陵皆在河南,兼敕功臣之家,不許北葬,今魏氏諸陵尚在京畿。祔葬代州,理未為允。」從之。}


丙子,以前翰林學士、戶部侍郎馮道依前本官充職。戊寅,西征之師入大散關,
 \gezhu{
  《九國志·趙廷隱傳》云:自入敵境,即禁兵士焚廬舍,剽財物,蜀人德之。}
 偽命鳳州節度使王承捷、故鎮屯駐指揮使唐景思次第迎降,得兵一萬二千、軍儲四十萬。又下三泉,得軍儲三十餘萬。自是師無匱乏,軍聲大振。辛巳,偽興州刺史王承鑒、成
 州刺史王承朴棄城遁去,康延孝大破蜀軍于三泉。時王衍將幸秦州,以其軍五萬屯于利州。聞我師至,遣步騎三萬逆戰于三泉,延孝與李嚴以勁騎三千擊之,蜀軍大敗,斬首五千級,餘眾奔潰。王衍聞敗,自利州奔歸成都,斷吉柏津,浮梁而去。丁亥,文武百官上表,以貞簡皇太后靈駕發引,請車駕不至山陵所。戊子,葬貞簡太后于坤陵。己丑,魏王繼岌至興州,偽東川節度使宋光葆以梓、綿、劍、龍、普五州來降;武定軍使王承肇以達、蓬、
 璧三州來降;興元節度使王宗威以梁、開、通、渠、麟五州來降;階州刺史王承岳納符印請命;秦州節度使王承休棄城自扶路奔于西川。《太平廣記》引《王氏見聞記》云:王承休握銳兵於天水,兵刃不舉。既知東軍入蜀,遂擁麾下之師及婦女孩幼萬餘口、金銀繒帛,於西蕃買路歸蜀。沿路為西蕃擄奪,凍餓相踐而死,迨至蜀,存者百餘人,唯與田宗汭等脫身而至。魏王使人問之曰:「親握重兵,何得不戰?」曰:「畏大王神武,不敢當其鋒。」曰:「何不早降?」曰:「蓋緣王師不入封部,無門納款。」曰:「初入蕃部幾許人?」曰:「萬餘口。」「今存者幾何?」曰:「纔及百數。」魏王曰:「汝可償萬人之命。」遂斬之。



 十一月庚寅朔,帝幸壽安,號慟于坤陵。戊戌,以振武節度使朱守殷為兗州節度使。
 徐州、鄴都上言,十月二十五日夜,地大震。康延孝至利州,修吉柏津浮梁。偽昭武軍節度使林思諤來降。辛丑,魏王過利州,帝賜王衍詔,諭以禍福。甲辰,魏王至劍州,偽武信軍節度使王宗壽以遂、合、渝、瀘、忠五州來降。丁未,高麗國遣使貢方物。康延孝、李嚴至漢州,王衍遣人送牛酒請降,李嚴遂先入成都。戊申,祔貞簡皇太后神主于太廟。


己酉,魏王至綿州,王衍遣使上箋歸命。庚戌,皇弟鄆州節度使存霸、滑州節度使存渥、左金吾大將
 軍晉州節度使存乂、邢州節度使存紀,並授起復雲麾將軍、右金吾大將軍同正。荊南節度使高季興奏,收復歸、夔、忠等州。辛亥,魏王至德陽。偽六軍使王宗弼報,王衍舉家遷于西宅,宗弼權稱西川兵馬留後;又報偽樞密使宋光嗣景潤澄、宣徽使李周輅歐陽晃同有異謀,惑亂蜀主,已梟斬訖。
 \gezhu{
  《九國志·王宗弼傳》:唐師陷鳳州,衍遣三招討屯三泉以拒唐師,未戰,三招討俱遁走,因令宗弼守綿谷而誅三招討,宗弼遂與三招討同送款于魏王。乃還成都,斬宋光嗣等,函首送于魏王,遷衍及母妻于西宮。}
 壬子,王衍遣使上表請降。癸丑,以吳越
 國馬步統軍使、檢校太傅錢元球為檢校太尉、守侍中,充靜海軍節度使。乙卯,魏王至西川城北。丙辰,蜀主王衍出降,語在衍傳。



 丁巳,大軍入成都,法令嚴峻,市不易肆。自興師凡七十五日,蜀平,得兵士三萬、兵仗七百萬、糧三百五十三萬、錢一百九十二萬貫、金銀共二十二萬兩、珠玉犀象二萬、紋錦綾羅五十萬,得節度州十、郡六十四、縣二百四十九。己丑,禮儀使奏:「貞簡皇太后升祔禮畢,一應宗廟伎樂及諸祀並請仍舊。」從之。十二月
 壬戌,以前雲州節度使李存敬為同州節度使;以同州節度使、檢校太保、同平章事李令德為遂州節度使;以邠州節度使、檢校太保董璋為劍南東川節度副大使、知節度事;以華州節度使毛璋為邠州節度使;以左金吾大將軍史敬熔為華州節度使。丁卯,以武寧軍節度副使李紹文為兗州觀察留後。庚午,宴諸王武臣于長春殿,始用樂。丙子,以北京副留守、太原尹孟知祥為檢校太傅、同平章事、成都尹、劍南西川節度副大使、知節度事、西山
 八國雲南都招撫等使;以戶部尚書王正言為檢校吏部尚書、守興唐尹,充鄴都副留守;以鄴都副留守、興唐尹張憲檢校吏部尚書、太原尹,充北京副留守、知留守事。



 己卯,以臘辰狩于白沙,皇后、皇子、宮人畢從。庚辰,次伊闕。辛巳,次潭泊。壬午·衛靈公》:「志士仁人,無求生以害仁,有殺身以成仁。」認,次龕澗。癸未,還宮。是時大雪苦寒,吏士有凍踣于路者。伊、汝之民,飢乏尤甚,衛兵所至,責其供餉,既不能給,因壞其什器,撤其廬舍而焚之,甚于剽劫。縣吏畏恐,竄避于山谷間。甲申,出御札示中書
 門下,以今歲水災異常,所在人戶流徙,以避征賦,關市之徵,抽納繁碎,宜令宰臣商量條奏。丙戌,第三姑宋氏封義寧大長公主,長姊孟氏封瓊華長公主,第十一妹張氏封瑤英長公主。



 閏十二月甲午,賜中書門下詔曰:



 朕聞古先哲王,臨御天下,上則以無偏無黨為至治,次則以足食足兵為遠謀,緬惟前修,誠可師範。朕纂承鳳歷,嗣守鴻圖,三載于茲,萬機是總,非不知五兵未弭,兆庶多艱,蓋賴卿等寅亮居懷,康濟為務,冀盡數輿之理,
 洞詢盍徹之規。今則潛按方區,備聆謠俗,或力役罕均其勞逸,或賦租莫辨于後先,但以督促為名,煩苛不已。被甲胄者何嘗充給,趨朝省者轉困支持,州閭之貨殖全疏,天地之災祥屢應。以至星辰越度,旱澇不時,農桑失業于丘園,道殣相望于郊野,生靈及此,寢食寧遑,豈非朕德政未孚,焦勞自拙者耶!



 朕昨親援毫翰,軫念瘡痍,一則詢爾謀猷,一則表予宵旰,未披來奏,轉撓于懷,敢不翼翼罪躬,乾乾軫慮。咨爾四岳,弼予一人,何不舉
 賢才,裨寡昧。百辟之內,群后之間,莫不有盡忠者被掩其能,抱器者艱陳其力。或草澤有遺逸之士,山林多屈滯之人,爾所不知,吾將安訪!卿等位尊調鼎,名顯代天,既逢不諱之朝,何吝由衷之說,當宜歷告中外,急訪英髦。應在仕及前資文武官已下,至草澤之士,有濟國治民、除姦革弊者,並宜各獻封章,朕當選擇施行。其近宣御札,亦告諭內外,體朕意焉。



 是時,兩河大水,戶口流亡者十四五,都下供饋不充,軍士乏食,乃有鬻子去妻,老
 弱採拾于野,殍踣于行路者。州郡飛輓,旋給京師,租庸使孔謙日于上東門外佇望其來,算而給之。加以所在泥潦,輦運艱難,愁嘆之聲,盈于道路,四方地震,天象乖越。帝深憂之,問所司濟贍之術。孔謙比以吏進,故無保邦濟民之要務,唯以急刻賦斂為事。樞密承旨段徊奏曰:「臣見本朝時或遇歲時災歉,國費不足,天子將求經濟之要,則內出朱書御札,以訪宰臣,請陛下依此故事行之。」即命學士草詞,帝親札以訪宰臣,非帝憂民之實
 也。時宰相豆盧革等依阿徇旨,竟無所陳,但云:「陛下威德冠天下,今西蜀平定,珍寶甚多,可以給軍。水旱作沴,天之常道,不足以貽聖憂。」中官李紹宏奏曰:「俟魏王旋軍之後,若兵額漸多,饋挽難給,請且幸汴州,以便漕挽。」時群臣獻議者亦多,大較詞理迂闊,不中時病。唯吏部尚書李琪引古田租之法,從權救弊之道,上疏言之,帝優詔以獎之。



 丁酉,詔偽蜀私署官員等:「惟名與器,不可假人,況是遐僻偏方,僭竊偽署,因時亂而
 濫稱名位,歸國體而悉合削除。但恐當本朝屯否之時,有歷代簪纓之士,既陷彼土,遂授偽官。又慮有曾受本朝渥恩,當時已居班秩,須為升降,不可通同。應偽署官至太師、太傅及三少,并太尉、司徒、司空、侍中、中書令、左右僕射已上,並宜降至六尚書,臨時更約偽署高低為六行次第。階至開府、特進、金紫者,宜令文班降至朝散大夫,武班降至銀青。爵偽署將相已下與開國男,餘並不得更稱封爵,其有功臣者削去。《五代會要》云:其有功臣名號,並宜削去。
 如是偽署節鎮,伐罪之初,率先向化及立功效者,宜委繼岌、崇韜臨時獎任。其刺史但許稱使君,不得更有檢校官。其偽署班行正四品已上,酌此降黜,五品已下,如不曾經本朝授官,若材智有聞,即許于府縣中量材任使;如無材智可錄,止是蜀地土人,並宜放歸田里。如是西班有稱統軍上將軍者,若是本朝功臣子孫及將相之嗣,並據人材高下,與諸衛小將軍、府率、中郎將,次第授任。如是小將軍已下,據人材堪任使者,宜委西川節
 度使衙前補押衙;不堪任使者,亦宜放歸田里。應已前降官,除軍前量事迹任使外,餘並稱前銜,候朝廷續據才行任使。」



 庚子,彰武、保大等節度使高萬興卒。甲辰,淮南楊溥遣使朝貢。乙巳,以晉州節度使李存乂為鄜州節度使,以相州刺史李存確為晉州節度使。丙子,兩省諫官上疏,請車駕不巡幸汴州,凡三上章,乃允。庚戌,魏王繼岌奏,遣秦州副史徐藹齎書招諭南詔蠻。又奏,點到兩川馬九千五百三十匹。《清異錄》:莊宗滅梁平蜀,志頗自逸,命蜀匠織十幅無
 縫錦為被材,被成,賜名「六合被」。辛亥,制皇第二弟存霸可封永王,第三弟存美可封邕王,第四弟存渥可封申王,第五弟存乂可封睦王,第六弟存確可封通王,第七弟存紀可封雅王。是歲,日傍有背氣,凡十三。



\end{pinyinscope}