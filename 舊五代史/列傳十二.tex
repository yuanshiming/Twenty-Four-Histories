\article{列傳十二}

\begin{pinyinscope}

 李襲吉,自言左相林甫之後,父圖,為洛陽令,因家焉。襲吉乾符末應進士舉,遇亂,避地河中,依節度使李都,擢為鹽鐵判官。及王重榮代,不喜文士,時喪亂之後,衣冠
 多逃難汾、晉間。襲吉訪舊至太原,武皇署為府掾,出宰榆社。光啟初,武皇遇難上源,記室歿焉,既歸鎮,辟掌奏者,多不如旨。或有薦襲吉能文,召試稱旨,即署為掌書記。襲吉博學多通,尤諳悉國朝近事,為文精意練實,動据典故,無所放縱,羽檄軍書,辭理宏健。自武皇上源之難,與梁祖不協,乾寧末,劉仁恭負恩,其間論列是非,交相聘答者數百篇,警策之句,播在人口,文士稱之。三年,遷節度副使,從討王行瑜,拜右諫議大夫。及師還渭北,
 武皇不獲入覲,為武皇作違離表,中有警句云:「穴禽有翼,聽舜樂以猶來;天路無梯,望堯雲而不到。」昭宗覽之嘉歎。洎襲吉入奏,面詔諭之,優賜特異。《北夢瑣言》:襲吉從李克用至渭南,令其入奏,帝重其文章,授諫議大夫,使上事北省以榮之。其年十二月,師還太原,王珂為浮梁於夏陽渡,襲吉從軍,時笮斷航破,武皇僅免,襲吉墜河,得大冰承足,沿流七八里,還岸而止,救之獲免。



 天復中,武皇議欲修好於梁,命襲吉為書以貽梁祖,書曰:



 一別清德,十有餘年,失意杯盤,爭鋒劍戟。山長
 水闊,難追二國之歡;鴈逝魚沉,久絕八行之賜。比者僕與公實聯宗姓,原忝恩行,投分深情,將期棲託,論交馬上,薦美朝端,傾嚮仁賢,未省疏闕。豈謂運由奇特,謗起奸邪。毒手尊拳,交相于幕夜;金戈鐵馬,蹂踐于明時。狂藥致其失歡,陳事止于堪笑。今則皆登貴位,盡及中年,蘧公亦要知非,君子何勞用壯。今公貴先列辟,名過古人。合縱連衡,本務家邦之計;拓地守境,要存子孫之基。文王貴奔走之交,仲尼譚損益之友,僕顧慚虛薄,舊忝
 眷私,一言許心,萬死不悔,壯懷忠力,猶勝他人,盟于三光,願赴湯火。公又何必終年立敵,懇意相窺,徇一時之襟靈,取四郊之倦弊,今日得其小眾,明日下其危牆,弊師無遺鏃之憂,鄰壤抱剝床之痛。又慮悠悠之黨,妄瀆聽聞,見僕韜勇枕威,戢兵守境,不量本末,誤致窺覦。



 且僕自壯歲已前,業經陷敵,以殺戮為東作,號兼并為永謀。及其首陟師壇,躬被公兗,天子命我為群后,明公許我以下交,所以斂跡愛人,蓄兵務德,收燕薊則還其故
 將,入蒲阪而不負前言。況五載休兵,三邊校士,鐵騎犀甲,雲屯谷量。馬邑兒童,皆為銳將;鷲峰宮闕,咸作京坻。問年猶少于仁明,語地幸依于險阻,有何覘睹,便誤英聰。



 況僕臨戎握兵,粗有操斷,屈伸進退,久貯心期。勝則撫三晉之民,敗則徵五部之眾,長驅席卷,反首提戈。但慮隳突中原,為公後患,四海群謗,盡歸仁明,終不能見僕一夫,得僕一馬。銳師儻失,則難整齊,請防後艱,願存前好。矧復陰山部落,是僕懿親;回紇師徒,累從外舍。文靖
 求始畢之眾,元海征五部之師,寬言虛詞,猶或得志。今僕散積財而募勇輩,輦寶貨以誘義戎,徵其密親,啗以美利,控弦跨馬,寧有數乎!但緣荷位天朝,惻心疲瘵,峨峨亭障,未忍起戎。亦望公深識鄙懷,洞回英鑒,論交釋憾,慮禍革心,不聽浮譚,以傷霸業。夫《易》惟忌滿,道貴持盈,儻恃勇以喪師,如擎盤而失水,為蛇刻鶴,幸賜徊翔,



 僕少負褊心,天與直氣,間謀詭論,誓不為之。唯將藥石之譚,願托金蘭之分。儻愚衷未豁,彼抱猶迷,假令罄三
 朝之威,窮九流之辯,遣回肝膈,如俟河清。今者執簡吐誠,願垂保鑒。



 僕自眷私睽隔,翰墨往來,或有鄙詞,稍侵英聽,亦承嘉論,每賜罵言。敘歡既罷于尋戈,焚謗幸蠲其載筆,窮因尚口,樂貴和心,願祛沉閼之嫌,以復塤篪之好。今者卜于嚬分,不欲因人,專遣使乎,直詣鈴閣。古者兵交兩地,使在其間,致命受辭,幸存前志。昔賢貴于投分,義士難于屈讎,若非仰戀恩私,安可輕露肝膈,悽悽丹愫,炳炳血情,臨紙嚮風,千萬難述。



 梁祖覽之,至「毒手
 尊拳」之句,怡然謂敬翔曰:「李公斗絕一隅,安得此文士!如吾之智算,得襲吉之筆才,虎傅翼矣!」又讀至「馬邑兒童」、「陰山部落」之句,梁祖怒謂敬翔曰:「李太原殘喘餘息,猶氣吞宇宙,可詬罵之。」及翔為報書,詞理非勝,由是襲吉之名愈重。《通鑒考異》引《唐末見聞錄》載全忠回書云:前年洹水,曾獲賢郎;去歲青山,又擒列將。蓋梁之書檄,皆此類也。



 自廣明大亂之後,諸侯割據方面,競延名士,以掌書檄。是時梁有敬翔,燕有馬郁,華州有李巨川,荊南有鄭準,《唐新纂》云:鄭準,士族,未第時,佐荊門上欲蓮幕,飛書走檄,不讓古人,秉直去邪,無慚
 往哲,考準為成汭書記,汭封上谷郡王。鳳翔有王超,《北夢瑣言》:唐末,鳳翔判官王超,推奉李茂貞,挾曹、馬之勢,箋奏文檄,恣意翱翔。後為興元留後,遇害,有《鳳鳴集》三十卷行于世。錢塘有羅隱,魏博有李山甫,皆有文稱,與襲吉齊名于時。



 襲吉在武皇幕府垂十五年,視事之暇,唯讀書業文,手不釋卷。性恬于榮利,獎誘後進,不以己能格物。參決府事,務在公平,不交賂遺,綽綽有士大夫之風概焉。天祐三年六月,以風病卒于太原。同光二年,追贈禮部尚書。



 王緘,幽州劉仁恭故吏也。少以刀筆直記室,仁恭假以
 幕職,令使鳳翔。還經太原,屬仁恭阻命,武皇留之。緘堅辭復命,書詞稍抗,武皇怒,下獄詰之,謝罪聽命,乃署為推官,歷掌書記。《契丹國志·韓延徽傳》:延徽自契丹奔晉,晉王欲置之幕府掌書記,王緘嫉之,延徽不自安,求東歸省母,遂復入契丹,寓書于晉王,敘所以北去之意。且曰:「非不戀英主,非不思故鄉,所以不留,正懼王緘之讒耳。」從莊宗經略山東,承制授檢校司空、魏博節度使。緘博學善屬文,燕薊多文士,緘後生,未知名,及在太原,名位驟達。燕人馬郁,有盛名于鄉里,而緘素以吏職事郁。及郁在太原,謂緘曰:「公在此作文士,所謂避
 風之鳥,受賜于魯人也。」每于公宴,但呼王緘而已。十年,從征幽州,既獲仁恭父子,莊宗命緘為露布,觀其旨趣。緘起草無所辭避,義士以此少之。胡柳之役,緘隨輜重前行,歿于亂兵。際晚,盧質還營,莊宗問副使所在,曰:「某醉不之知也。」既而緘凶問至,莊宗流涕久之,得其喪,歸葬太原。



 李敬義,本名延古,太尉衛公德裕之孫。初隨父煒貶連州,遇赦得還。嘗從事浙東,自言遇涿道士,謂之曰:「子方
 厄運,不宜仕進。」敬義悚然對曰;「吾終老賤哉?」涿曰:「自此四十三年,必遇聖王大任,子其志之。」敬義以為然,乃無心仕宦,退歸洛南平泉舊業。為河南尹張全義所和,歲時給遺特厚,出入其門,欲署幕職,堅辭不就。



 初,德裕之為將相也,大有勳于王室,出籓入輔,綿歷累朝;及留守洛陽,有終焉之志,於平泉置別墅,採天下奇花異竹、珍木怪石,為園池之玩。自為家戒序錄,志其草木之得處,刊于石,云:「移吾片石,折樹一枝,非子孫也。」洎巢、蔡之亂,
 洛都灰燼,全義披榛而創都邑,李氏花木,多為都下移掘,樵人鬻賣,園亭掃地矣。有醒酒石,德裕醉即踞之,最保惜者。光化初,中使有監全義軍得此石,置于家園。敬義知之,泣渭全義曰:「平泉別業,吾祖戒約甚嚴,子孫不肖,動違先旨。」因託全義請石于監軍。他日宴會,全義謂監軍曰:「李員外泣告,言內侍得衛公醒酒石,其祖戒堪哀,內侍能回遺否?」監軍忿然厲聲曰:「黃巢敗後,誰家園池完復,豈獨平泉有石哉!」全義始受黃巢偽命,以為詬
 己,大怒曰:「吾今為唐臣,非巢賊也。」即署奏笞斃之。



 昭宗遷都洛陽,以敬義為司勳員外郎。柳璨之陷裴、趙諸族,希梁祖旨奏云:「近年浮薄相扇,趨競成風,乃有臥邀軒冕,視王爵如土梗者。司空圖、李敬義三度除官,養望不至,咸宜屏黜,以勸事君者。」翌日,詔曰:「司勳史外郎李延古,世荷國恩,兩葉相位,幸從筮仕,累忝寵榮,多歷歲時,不趨班列。而自遷都卜洛,紀律載張,去明庭而非遙,處別墅而無懼,罔思報效,姑務便安。為臣之節如斯,貽厥
 之謀何在!須加懲責,以肅朝倫,九寺勾稽,尚謂寬典,可責授衛尉寺主簿。」司空圖亦追停前詔,任從閑適。圖,唐史有傳。《舊唐書·哀帝紀》:六月戊申,敕前司勳員外郎、賜緋魚袋李延古責授衛尉寺主簿。九月壬寅,敕前大中大夫、尚書兵部侍郎、賜紫金魚袋司空圖放還中條山。蓋延古與司空圖同時被劾,其降敕則有先後也。時全義既不能庇護,乃密託楊師厚,令敬義潛往依之,因挈族客居衛州者累年,師厚給遺周厚。



 十二年,莊宗定河朔,史建瑭收新鄉,敬義謁見。是歲,上遣使迎至魏州,置北京留守判官承制拜工部尚書,奉使王鎔。敬義
 以遠祖趙郡,見鎔展維桑之敬,鎔遣判官李翥送《贊皇集》三卷,令謁前代碑壟,使還,歸職太原。監軍張承業尤不悅本朝宰輔子孫,待敬義甚薄,或面折于公宴,或指言德裕過惡,敬義不得志,鬱憤而卒。同光二年,贈右僕射。《五代史闕文》:司空圖,字表聖,自言泗州人。少有俊才,威通中,一舉登進士第。雅好為文,躁于進取,頗臬矜伐,端士鄙之。初,從事使府,及登朝,驟歷清要。巢賊之亂,車駕播遷,圖有先人舊業在中條山,極林泉之美,圖自禮部員外郎,因避地焉,日以詩酒自娛。屬天下板蕩,士多往依之,互相推獎,由是聲名藉甚。昭宗反正,以戶部侍郎徵至京師。圖既負才慢世,謂己當為宰輔,時要惡之,稍抑其銳,圖憤憤謝病,復歸中條。與人書疏,不名官
 位,但稱知非子,又稱耐辱居士。其所居曰禎貽谿,溪上結茅屋,命曰休休亭,常自為記云。臣謹案:圖,河中虞鄉人,少有文彩,未為鄉里所稱。會王凝自尚書郎出為州絳刺史,圖以文謁之,大為凝所賞歎,由是知名。未幾,凝入知制誥,遷中書舍人、知貢舉。擢圖上第。頃之,凝出為宣州觀察使,闢圖為從事。既渡江,御史府奏圖監察,下詔追之。圖感知己之恩,不忍輕離幕府,滿百日不赴闕,為臺司所劾,遂以本官分司。久之,徵拜禮部員外郎,俄知制誥,故集中有文曰:戀恩稽命,點繫洛師,于今十年,方忝綸閣,此豈躁于進取者耶!舊史不詳,一至于此。圖見唐政多僻,中官用事,知天下必亂,即棄官歸中條山。尋以中書舍人征,又拜禮部、戶部侍郎,皆不起。及昭宗播遷華下,圖以密邇乘輿,即時奔問,復辭還山,故詩曰「多病形容五十三,誰憐借笏趙朝參」,此豈有意於相位耶!河中節度使王重榮請圖撰碑,得絹數千匹,圖致于虞
 鄉市心,恣鄉人所取,一日而盡。是時盜賊充斥,獨不入王官穀,河中士人依圖避難,全者甚眾。昭宗東遷,又以兵部侍郎召至洛下,為柳璨所阻,一謝而退。梁祖受禪,以禮部尚書征,辭以老疾,卒時年八十餘。臣又案:梁室大臣,如敬翔、李振、杜曉、楊涉等,皆唐朝舊族,本當忠義立身,重侯累將,三百餘年,一旦委質朱梁,其甚者贊成弒逆。惟圖以清直避世,終身不事梁祖,故《梁史》揭圖小瑕以泯大節者,良有以也。



 盧汝弼,《宣和書譜》:汝弼字子諧,祖綸,唐貞元年有詩名。父簡求,為河東節度使。汝弼少力學,不喜為世胄,篤志科舉,登進士第,文彩秀麗,一時士大夫稱之。唐昭宗景福中,擢進士第,歷臺省。昭宗自秦遷洛,時為祠部郎中、知制誥。時梁祖凌弱唐室,殄滅衣冠,懼禍渡河,由上黨歸于晉陽。初,武皇
 平王行瑜,天子許承制授將吏官秩。是時籓侯倔強者,多偽行墨制,武皇恥而不行,長吏皆表授。及莊宗嗣晉王位,承制置吏,又得汝弼,有若符契,由是除補之命,皆出汝弼之手,既而畿內官吏,考課議擬,奔走盈門,頗以賄賂聞,士論少之。洎帝平定趙、魏,汝弼每請謁迎勞,必陳說天命,顒俟中興,帝亦以宰輔期之。建國前,卒于晉。《宣和書譜》:贈兵部尚書。



 李德休,字表逸,趙郡贊皇人也。祖絳,山南西道節度使,
 唐史有傳。父璋,宣州觀察使。德休登進士第,歷鹽鐵官、渭南尉、右補闕、侍御史。天祐初,兩京喪亂,乃寓跡河朔,定州節度使王處直辟為從事。莊宗即位于魏州,徵為御史中丞,轉兵部、吏部侍郎,權知左丞,以禮部尚書致仕。卒時年七十四。贈太子少保。



 蘇循,父特,陳州刺史。循,咸通中登進士第,累歷臺閣。昭宗朝,再至禮部尚書。循性阿諛,善承順茍容,以希進取。昭宗自遷洛之後,梁祖凶勢日滋,唐室舊臣,陰懷主辱
 之憤,名族之胄,往往有違禍不仕者,唯循希旨附會。及梁祖失律于淮南,西屯于壽春,要少帝欲授九錫。朝臣或議是非,循揚言云:「梁王功業顯大,歷數有歸,朝廷速宜揖讓。」當時朝士畏梁祖如虎,罔敢違其言者。明年,梁祖逼禪,循為冊禮副使。梁祖既受命,宴于元德殿,舉酒曰:「朕夾輔日淺,代德未隆,置朕及此者,群公推崇之意也。」楊涉、張文蔚慚懼失對,致謝而已。循與張禕、薛貽矩因盛陳梁祖之德業,應天順人之美。循自以奉冊之勞,
 旦夕望居宰輔,而敬翔惡其為人,謂梁祖曰:「聖祚維新,宜選端士,以鎮風俗。如循等輩,俱無士行,實唐家之鴟梟,當今之狐魅,彼專賣國以取利,不可立維新之朝。」



 初,循子楷,乾寧二年登進士第。中使有奏御者云:「今年進士二十餘人,僥倖者半,物論以為不可。」昭宗命學士陸扆、馮渥重試于雲韶殿,及格者一十四人。詔云:「蘇楷、盧賡等四人,詩句最卑,蕪累頗甚,曾無學業,敢竊科名,浼我至公,難從濫進,宜付所司落下,不得再赴舉場。」楷以
 此慚恨,長幸國家之災。昭宗遇弒,輝王嗣位,國命出于朱氏,楷始得為起居郎。



 柳璨陷害朝臣,衣冠惕息,無敢言者。初,梁祖欲以張廷範為太常卿,裴樞以為不可。柳璨懼梁祖之毒,乃歸過于樞,故裴、趙罹白馬之禍。楷因附璨,復依廷範。時有司初定昭宗謚號,楷謂廷範曰:「謚者所以表行實,前有司之謚先帝為昭宗,所謂名實不副。司空為樂卿,餘忝史職,典章有失,安得不言。」乃上疏曰:「帝王御宇,察理亂以審污隆;祀享配天,資謚號以定
 升降。故臣下君上,皆不得而私也。先帝睿哲居尊,恭儉垂化,其于善美,孰敢蔽虧。然而否運莫興,至理猶鬱,遂致四方多事,萬乘播遷。始則宦豎凶狂,受幽辱于東內;終則嬪嬙悖亂,罹夭閼于中闈。其于易名,宜循考行。有司先定尊謚曰聖穆景文孝皇帝,廟號昭宗,敢言溢美,似異直書。今郊禋有日,祫祭惟時,將期允愜列聖之心,更在詳議新廟之稱,庶使葉先朝罪己之德,表聖上無私之明。」《舊唐書》云:蘇楷目不知書,僅能執筆,其文羅袞作也。太常卿張廷範奏議
 曰:「昭宗初實彰于聖德,後漸減于休明,致季述幽辱于前,茂貞劫幸于後,雖數拘厄運,亦道失始終。違陵寢于西京,徙兆民于東洛,軔輦輅未踰于寒暑,行大事俄起于宮闈。謹聞執事堅固之謂恭,亂而不損之謂靈,武而不遂之謂莊,在國逢難之謂閔,因事有功之謂襄。今請改謚曰恭靈莊閔皇帝,廟號襄宗。」輝王答詔曰:「勉依所奏,哀咽良深。」楷附會幸災也如是。



 及梁祖即位于汴,楷自以遭遇千載一時,敬翔深鄙其行。尋有詔云:「蘇楷、高
 貽休、蕭聞禮等,人才寢陋,不可塵穢班行,並勒歸田里。」循、楷既失所望,懼以前過獲罪,乃退歸河中依硃友謙。莊宗將即位于魏州,時百官多缺,乃求訪本朝衣冠,友謙令赴行臺。時張承業未欲莊宗即尊位,諸將賓僚無敢贊成者,及循至,入衙城見府廨即拜,謂之拜殿。時將吏未行蹈舞禮,及循朝謁,即呼萬歲舞抃,泣而稱臣,莊宗大悅。翼日,又獻大筆三十管,曰「畫日筆」,莊宗益喜。承業聞之怒,會盧汝弼卒,即令循守本官,代為副使。明年
 春,循因食蜜雪,傷寒而卒。同光二年,贈左僕射,以楷為員外郎。天成中,累歷使幕,會執政欲糾其駁謚之罪,竟以憂慚而卒。



 史臣曰:昔武皇之樹霸基,莊宗之開帝業,皆旁求多士,用佐丕圖。故數君子者,或以書檄敏才,或以搢紳舊族,咸登貴仕,諒亦宜哉!唯蘇循贊梁祖之強禪,蘇楷駁昭宗之舊謚,士風臣節,豈若是乎!斯蓋文苑之豺狼,儒林之荊棘也。



\end{pinyinscope}