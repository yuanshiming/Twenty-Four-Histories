\article{列傳五}

\begin{pinyinscope}

 李存信,本姓張,父君
 政,回鶻
 部人也。大中初,隨懷化郡王李思忠內附,因家雲中之合羅川。存信通黠多數,會四夷語,別六蕃書,善戰,識兵勢。初為獻祖親信,從武皇
 入關平賊,始補軍職,賜姓名。大順中,累遷至馬步都校,與李存孝擊張浚軍於平陽。時存孝驍勇冠絕,軍中皆下之,惟存信與爭功,由是相惡,有同水火。及平定潞州,存孝以功望領節度,既而康君立授旄鉞,存孝怒,大剽潞民,燒邑屋,言發流涕,疑存信擯己故也。明年,存孝得邢、洺,武皇與之節鉞。存孝慮存信離間,欲立大功以勝之,屢請兵於武皇,請兼并鎮、冀,存信間之,不時許。大順二年,武皇大舉略地山東,以存信為蕃漢馬步都校,存
 孝聞之怒,武皇令存質代之,存孝乃謀叛。既誅,以存信為蕃漢都校。從討李匡儔,降赫連鐸、白義誠,以功檢校右僕射,從入關討王行瑜,加檢校司空,領郴州刺史。



 乾寧三年,兗、鄆乞師于武皇,武皇遣存信營于莘縣,與硃瑄合勢以抗梁人。梁祖患之,遣使諜羅宏信曰:「河東志在吞食河朔,迴軍之日,貴道堪憂。」而存信戢兵無法,稍侵魏之芻牧,宏信怒,翻然結于梁祖,乃出兵三萬以攻存信。存信斂眾而退,為魏人所薄,委棄輜重,退保洺州,
 軍士喪失者十二三。武皇怒,大出師攻魏博,屠陷諸邑。五月,存信軍于洹水,汴將葛從周、氏叔琮來援魏人,存信與鐵林都將落落遇汴人于洹水南,汴人為陷馬坎以待之,存信戰敗,落落被擒。九月,存信敗葛從周于宗城,乘勝至魏州之北門。明年,聞兗、鄆皆陷,乃班師。八月,從討劉仁恭,師次安塞,為燕軍所敗。武皇怒謂存信曰:「昨日吾醉,不悟賊至,公不辨耶!古人三敗,公姑二矣。」存信懼,泥首謝罪,幾至不測。自光化已後,存信多稱病,武
 皇以兵柄授李嗣昭,以存信為右校而已。天復二年十月,以疾卒于晉陽,時四十一。



 李存孝,本姓安,名敬思。《新唐書》:存孝,飛狐人。少于俘囚中得隸紀綱,給事帳中。及壯法言西漢楊雄著。仿《論語》體例,共十三卷。以儒學,便騎射,驍勇冠絕,常將騎為先鋒,未嘗挫敗;從武皇救陳、許,逐黃寇,及遇難上源,每戰無不克捷。



 張浚之加兵於太原也,潞州小校馮霸殺其帥李克恭以城叛。時汴將朱崇節入潞州,梁祖令張全義攻澤州。李罕之告急于武皇,武皇遣存孝率騎五千援之。
 初,汴人攻澤州者以之泛指人倫文化。,呼罕之曰:「相公常恃太原,輕絕大國,今張相公圍太原,葛司空已入潞府,旬日之內,沙陀無穴自處,相公何路求生耶!」存孝聞其言不遜,選精騎五百,繞汴營呼曰:「我,沙陀求穴者,俟爾肉饌軍,可令肥者出鬥!」汴將有鄧季筠者,亦以驍勇聞,乃引軍出戰。存孝激勵部眾,舞槊先登,一戰敗之,獲馬千匹,生擒季筠于軍中。是夜,汴將李讜收軍而遁,存孝追擊至馬牢山,俘斬萬計,遂退攻潞州。



 時朝廷命京兆尹孫揆為昭義節度
 使,令供奉官韓歸範送旌節至平陽,揆乃仗節之潞;梁祖與揆牙兵三千為紀綱,時揆為張浚副招討,所部萬人。八月,自晉、絳踰刀黃嶺趨上黨。存孝引三百騎伏于長子西崖間。揆褒衣大蓋,擁眾而行,俟其軍前後不屬,存孝出騎橫擊之,擒揆與歸範及俘囚五百,獻于太原;存孝乃急攻潞州。九月,葛從周棄城夜遁,存孝收城,武皇乃表康君立為潞帥。存孝怒,不食者累日。十月,存孝引收潞州之師,圍張浚于平陽,營于趙城。華州韓建遣
 壯士三百夜犯其營,存孝諜知,設伏以擊之,盡殪;進壓晉州西門,獲賊三千,自是閉壁不出。存孝引軍攻絳州。十一月,刺史張行恭棄城而去,張浚、韓建亦由含口而遁。存孝收晉、絳,以功授汾州刺史。



 大順二年三月,邢州節度使安知建叛入汴軍,武皇令存孝定邢、洺,因授之節鉞。時幽州李匡威與鎮州王鎔屢弱中山,將中分其疆土。定州王處存求援於武皇;武皇命存孝侵鎮、趙之南鄙,又令李存信、李存審率師出井陘以會之,並軍攻
 臨城、柏鄉。李匡威救至,且議旋師。李存信與存孝不協,因構于武皇,言存孝望風退衄,無心擊賊,恐有私盟也。存孝知之,自恃戰功,鬱鬱不平,因致書通王鎔,又歸款于汴。明年,武皇自出井陘,將逼真定,存孝面見王鎔陳軍機。武皇暴怒,誅先獲汴將安康八方旋師。七月,復出師討存孝,自縛馬關東下,攻平山,渡滹水,擊鎮州四關城。王鎔懼,遣使乞平,請以兵三萬助擊存孝,許之。《新唐書》:王鎔失幽州助,因乞盟,進幣五十萬,歸糧二十萬,請出兵助討存孝。武皇蒐于欒城。李存
 信屯琉璃陂。九月,存孝夜犯存信營,奉誠軍使孫考老被獲,存信軍亂。武皇進攻邢州,深溝高壘以環之,旋為存孝沖突,溝塹不成。有軍校袁奉韜者,密令人謂存孝曰:「大王俟塹成即歸太原,如塹壘未成,恐無歸志。尚書所畏惟大王耳,料諸將孰出尚書右。王若西歸,雖限以黃河,亦可浮渡,況咫尺之洫,安能阻尚書鋒銳哉!」存孝然之,縱兵成塹。居旬日,深溝高壘,飛走不能及,由是存孝至敗,城中食盡。乾寧元年三月,存孝登城首罪,泣訴
 于武皇曰:「兒蒙王深恩,位至將帥,茍非讒慝離間,曷欲舍父子之深恩,附仇讎之黨!兒雖褊狹設計,實存信構陷至此,若得生見王面,一言而死,誠所甘心。」武皇愍之,遣劉太妃入城慰勞。太妃引來謁見,存孝泥首請罪曰:「兒立微勞,本無顯過,但被人中傷,申明無路,迷昧至此!」武皇叱之曰:「爾與王鎔書狀,罪我萬端,亦存信教耶!」縶歸太原,車裂于市。然武皇深惜其才。存孝每臨大敵,被重鎧橐弓坐槊,僕人以二騎從,陣中易騎,輕捷如飛,獨
 舞鐵楇,挺身陷陣,萬人辟易,蓋古張遼、甘寧之比也。存孝死,武皇不視事旬日,私憾諸將久之。


李存進,振武人,本姓孫,名重進。
 \gezhu{
  《歐陽史》:太祖破朔州得之,賜以姓名,養為子。}
 父牷,世吏單于府。重進初仕嵐州刺史湯群為部校,獻祖誅群,乃事武皇。從入關,還鎮太原,署牙職。景福中,為義兒軍使,賜姓名。從討王行瑜,以功授檢校常侍,與李嗣昭同破王珙于河中。光化三年,契丹犯塞,寇雲中,改永安軍使、鴈門以北都知兵馬使。天復初,破氏叔琮前
 軍于洞渦。三年,授石州刺史。莊宗初嗣位,入為步軍右都檢校司空,師出井陘,授行營馬軍都虞候,破汴軍于相鄉,論功授邠州刺史,轉檢校司徒。俄兼西南面行營招討使,出師收慈州,授慈、沁二州刺史。十二年,定魏博,授天雄軍都巡按使。時魏人初附,有銀槍效節都,強傑難制,專謀騷動。存進沈厚果斷,犯令者梟首屍于市,諸軍無不惕息,靡然向風。十四年,擢蕃漢馬步副總管,從攻楊劉,戰胡柳。



 十六年,以本職兼領振武節度使。時王
 師據德勝渡,汴軍據楊村渡在上流。汴人運洛陽竹木,造浮橋以濟軍。王師以船渡,緩急難濟,存進率意欲造浮橋。軍吏曰:「河橋須竹笮大め,兩岸石倉鐵牛以為固,今無竹石,竊慮難成。」存進曰:「吾成算在心,必有所立。」乃課軍造葦笮,維大艦數十艘,作土山,植巨木于岸以纜之。初,軍中以為戲,月餘橋成,制度條直,人皆服其勤智。莊宗舉酒曰:「存進,吾之杜預也。」賜寶馬御衣,進檢校太保,兼魏博馬步都將。與李存審固守德勝。



 十九年,汴將王
 瓚率眾逼北城,為地穴火車,百道進攻。存進隨機拒應,或經日不得食。汴軍退球自轉和公轉為標準,以年、月、日、時、分、秒為單位。,加檢校太傅。王師討張文禮于鎮州,閻寶、李嗣昭相次不利而歿。七月,存進代嗣昭為招討,進營東垣渡,夾滹沲為壘,沙土散惡,垣壁難成。存進斬伐林樹,版築旬日而就,賊不能寇。九月,王處球盡率其眾,乘其無備,奄至壘門。存進聞之,得部下數人出鬥,驅賊于橋下。俄而賊大至,後軍不繼,血戰而歿,時年六十六。同光時,贈太尉。存進行軍出師,雖無奇跡,然能
 以法繩其驕放,營壘守戰之備,特推精力,議者稱之。



 有子四人,長曰漢韶。



 漢韶,字享天,幼有器局,風儀峻整。初事莊宗,為定安軍使,遷河東牢城指揮使。時孟知祥權知太原軍府事,會契丹侵北鄙。表令漢韶率師進討,既而大破契丹,以功加檢校右僕射。同光中,為蔡州刺史。天成初,復姓孫氏,尋授彰國軍留後,累加檢校太保。長興中,為洋州節度使。《九國志》:閔帝嗣位,加特進,漢韶以其父名上表讓之,改檢校左僕射。制曰:「改會稽之字,抑有前聞;換瑰寶
 之文,非無故事。」末帝之起于鳳翔也,漢韶與興元張虔釗各帥部兵會王師于岐山下,及西師俱叛,漢韶逃歸本鎮。聞末帝即位,心不自安,乃與張虔釗各舉其城送款于蜀。洎至成都,孟知祥以漢韶故人,尤善待之。《九國志》:漢韶與知祥敘汾上舊事,及洛中更變,相對感泣。知祥曰:「豐沛故人,相遇于此,何樂如之!」于是賜第宅金帛,供帳什物,悉官給之。偽命永平軍節度使,孟昶嗣偽位,歷興元、遂州兩鎮連帥,累偽官至中書令,封樂安郡王。年七十餘,卒于蜀。



 李存璋,字德璜,雲中人。武皇初起雲中,存璋與康君立、
 薛志勤等為奔走交,從入關,以功授國子祭酒,累管萬勝、雄威等軍。從討李匡儔,改義兒軍使。光化二年,授澤州刺史,入為牢城使。從李嗣昭討雲州叛將王暉,平之,改教練使、檢校司空。五年,武皇疾篤,召張承業與存璋授遺顧。存璋爰立莊宗,夷內難,頗有力焉。改河東馬步都虞候,兼領鹽鐵。初,武皇稍寵軍士,籓部人多干擾廛市,肆其豪奪,法司不能禁。莊宗初嗣位,銳于求理。存璋得行其志,抑強撫弱,誅其豪首,期月之間,紀綱大振,弭
 群盜,務耕稼,去姦宄,息倖門,當時稱其材幹。從破汴軍於夾城,轉檢校司徒。柏鄉之役,為三鎮排陣使。十一年,從盟朱友謙于猗氏,授汾州刺史。汴將尹皓攻慈州,逆戰敗之。十三年,王檀逼太原,存璋率汾州之軍入城固守,授大同防禦使、應蔚朔等州都知兵馬使。秋,契丹攻蔚州,安巴堅遣使馳木書求賂,存璋斬其使。契丹逼雲州,存璋拒守,城中有古鐵車,乃熔為兵仗,以給軍士。敵退,以功加檢校太傅、大同軍節度使、應蔚等州觀察使。
 十九年四月,以疾卒于雲州府第。同光初,追贈太保、平章事。晉天福初,追贈太師。



 有子三人,彥球為裨校,戰歿于鎮州。



 李存賢,字子良,本姓王,名賢,許州人。祖啟忠,父惲。賢少遇亂,入黃巢軍;武皇破賊陳、許,存賢來歸。景福中,典義兒軍,為副兵馬使,因賜姓名。天祐三年,從周德威赴援上黨,營于交口。五年,權知蔚州刺史,以禦吐渾。六年,權沁州刺史。先是,州當賊境,不能保守,乃于州南五十里
 據險立柵為治所,已歷十餘年矣。存賢至郡,乃移復舊郡,劃闢荊棘,特立廨舍,州民完集。莊宗嘉之,轉檢校司空,真拜刺史。九年,汴人乘其無備,來攻其城,存賢擊退之。十一年,授武州刺史、山北團練使。十二年,移刺慈州。七月,汴將尹皓攻州城,存賢督軍拒戰,汴軍攻擊百端,月餘遁去。十八年,河中朱友謙來求援,命存賢率師赴之。十九年,汴將段凝軍五萬營臨晉,蒲人大恐,咸欲歸汴。或問于存賢曰:「河中將士欲拘公降於汴。」厚賢曰:「吾奉
 命河中,死王事固其所也。」汴軍退,以功加檢校司徒。同光初,授右武衛上將軍。十一月,入覲洛陽。二年三月,幽州李存審疾篤,求入覲,議擇帥代之。方內宴,莊宗曰:「吾披榛故人,零落殆盡,所殘者存審耳。今復衰疾,北門之事,知付何人!」因目存賢曰:「無易於卿。」即日授特進、檢校太保,充幽州盧龍節度使。五月,到鎮。時契丹彊盛,城門之外,烽塵交警,一日數戰。存賢性忠謹周慎,晝夜戒嚴,不遑寢食,以至憂勞成疾,卒于幽州,時年六十五。詔贈
 太傅。



 存賢少有材力,善角牴。初,莊宗在籓邸,每宴,私與王鬱角牴鬥勝,鬱頻不勝。莊宗自矜其能,謂存賢曰:「與爾一博,如勝,賞爾一郡。」即時角牴,存賢勝,得蔚州刺史。



 史臣曰:昔武皇之起並、汾也,會鹿走於中原,期龍戰於大澤,蓄驍果之士,以備鷹犬之用。故自存信而下,皆錫姓以結其心,授任以責其效。與夫董卓之畜呂布,亦何殊哉!惟存孝之勇,足以冠三軍而長萬夫,茍不為叛臣,則可謂良將矣。



\end{pinyinscope}