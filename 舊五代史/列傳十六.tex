\article{列傳十六}

\begin{pinyinscope}

 霍彥威,字子重,洺州曲周人也。梁將霍存得之於村落間,年十四,從征討。存憐其爽邁,養為己子。存,梁史有傳。彥威未弱冠,為梁祖所知,擢在左右,漸升戎秩,亟立戰
 功。嘗中流矢,眇其一目。開平二年,自開封府押衙、右親從指揮使、檢校司空授右龍驤軍使。三年,自右監門衛將軍授左天武軍使,遷右監門上將軍。乾化三年,與袁象先同誅硃友珪,梁末帝授洺州刺史,轉河陽留後。乾化末,邠州留後李保衡背李茂貞以城歸梁,梁以彥威為邠州節度使。其年五月,茂貞遣將劉知俊率大軍攻之,彥威固守逾年,竟不能下;或得其俘,悉令放之,秦人懷其惠,遂無侵擾。轉滑州節度使。移鎮鄆州,兼北面行
 營招討,總大軍於河上。師徒屢敗,降授陜州留後。莊宗入汴,彥威自陜馳至請罪,詔釋之。一日,莊宗於崇元殿宴諸將,彥威與段凝、袁象先等預會。酒酣,莊宗舉酒屬明宗曰:「此席宴客,皆吾前歲之勁敵也,一旦與吾同宴,蓋卿前鋒之效也。」彥威等伏陛謝罪,莊宗曰:「與卿話舊,無足畏也。」因賜御衣、器幣,盡歡而罷。尋放歸籓。


明年,從明宗平潞州,授徐州節度使。契丹犯塞,莊宗以明宗為北面招討使,命彥威為副。彥威善言論,頗能接奉,明宗
 尤重之。趙太叛於邢州,奉詔討平之。時趙在禮據魏州,與明宗會兵於鄴下,大軍夕亂,明宗為其所逼,彥威從入魏州。皇甫暉等尤忌彥威,欲殺之,彥威機辯開說,竟免。及出,彥威部下兵士獨全,衛護明宗至魏縣。時明宗欲北趨常山,彥威與安重誨懇請赴闕。從至洛陽,彥威首率卿相勸進於至德宮。旬日之間,內外機事,皆決於彥威。擅收段凝、溫韜下獄,將置於法。安重誨曰:「溫、段罪惡,負於梁室,眾所知矣。今主上克平內艱,冀安萬國,
 豈為公報仇耶!」至天成初,除鄆州節度使。值青州王公儼拒命,改平盧軍節度,至鎮,擒公儼,斬之。明年冬,賜覲於汴州,明宗接遇甚厚,累官至檢校太尉、兼中書令。三年冬,卒於理所,年五十七。奏至之日,明宗方出近郊。忽聞奏訃,掩泣歸宮,輟朝三日,至月終不舉樂。
 \gezhu{
  《五代會要》:天成四年六月敕:「故平盧軍節度使霍彥威,勛名顯著,宅兆已營,度遵定謚之規,俾議送終之制,宜以三公禮葬。」}
 冊贈太師、晉國公,謚曰忠武。子承訓,弟彥珂,累歷刺史。皇朝乾德中,立明宗廟於洛州,詔以彥威配饗廟庭。


王晏球,字瑩之,自言洛都人。少遇亂,為蔡賊所掠,汴人杜氏畜之為子,因冒姓杜氏。晏球少沉勇有斷,倜儻不群。梁祖之鎮汴也,選富家子有材力者,置之帳下,號曰「子都」。
 \gezhu{
  《清異錄》:宣武子都,尤勇悍,其弩張一大機,則十二小機皆發,用連珠大箭,無遠不及,晉人極畏此。}
 晏球預選,從梁祖征伐,所至立功,累遷子都指揮使。梁開平三年,自開封府押衙充直左耀武指揮使,授右千牛衛將軍,軍職如故。朱友珪之篡位也,懷州龍驤守禦軍作亂,欲入京城。已至河陽,友珪命晏球出騎迎
 戰擊亂軍,獲軍使劉重遇,以功轉左龍驤第一指揮使。梁末帝嗣位,以晏球為龍驤四軍都指揮使。



 貞明二年四月十九日夜,汴州捉生都將李霸等作亂,縱火焚剽,攻建國門,梁末帝登樓拒戰。晏球聞亂,先得龍驤馬五百屯于鞠場,俄而亂兵以竿豎麻布沃油焚建國縷,勢將危急。晏球隔門窺亂兵,見無甲胄,即出騎擊之,奮力血戰,俄而群賊散走。梁末帝見騎軍討賊,呼曰;「非吾龍驤之士乎?」晏球奏曰:「亂者惟李霸一都,陛下但守宮城,
 遲明臣必破之。」既而晏球盡戮亂軍,全營族誅,以功授單州刺史。尋領軍於河上,為行營馬軍都指揮兼諸軍排陣使。



 莊宗入汴,晏球率騎軍入援。至封丘,聞梁末帝殂,即解甲降于莊宗。明年,與霍彥威北捍契丹,授齊州防禦使、北面行營馬軍都指揮使,仍賜姓氏,名紹虔。鄴之亂,明宗入赴內難,晏球時在瓦橋,遣人招之。明宗至汴,晏球率騎從至京師,以平定功授宋州節度使,上章求還本姓名。天成二年,授北面行營副招討,以兵戍滿
 城。是歲,王都據定州,《通鑒》:遣人說北面副招討使王晏球,晏球不從,乃以金遺晏球帳下,使圖之,不克。癸巳,晏球以都反狀聞。壬寅,以王晏球為北面招討使,權知定州行州事。契丹遣托諾率騎千餘來援都,突入定州,晏球引軍保曲陽。王都、托諾出軍拒戰,晏球督厲軍士,令短兵擊賊,戒之曰:「回首者死。」符彥卿以龍武左軍攻軍其左,高行周以龍武右軍攻其右,奮劍揮楇,應手首落,賊軍大敗于嘉山之下,追襲至於城門。俄而契丹首領特哩袞率勇騎五千至唐河。是時大雨,晏球出師逆戰,特哩袞復敗,追至易州,
 河水暴漲,所在陷沒,俘獲二千騎而還。特哩袞以餘眾北走幽州,趙德鈞令牙將武從諫以騎邀擊。德鈞分扼諸要路,旬日之內,盡獲特哩袞已下酋長七百餘人,契丹遂弱。晏球圍城既久,帝遣使督攻城,晏球曰:「賊壘堅峻,但食三州租稅,撫恤黎民,愛養軍士彼自當魚潰。」帝然其言。



 晏珠能與將士同其甘苦,所得祿賜私財,盡以饗士,日具飲饌,與將校筵宴,待軍士有禮,軍中無不敬伏。其年冬,平賊。自初戰至于城拔,不戮一士,上下歡心,
 物議以為有將帥之略,以功授天平軍節度使。未幾,移鎮青州,就加兼中書令。長興三年,卒于鎮,時年六十。贈太尉。



 子徹,位至懷州刺史。



 戴思遠,本梁之故將也。初事梁祖,以武幹知名。開平元年,自右羽林統軍加檢校司徒,出為晉州刺史。二年,授右監門上將軍,尋改華州防御使。三年,自左天武使復授右羽林統軍。郢王友珪篡位,授洺州團練使。貞明中,為邢州留後,遷本州節度使。屬燕將張萬進殺滄州留
 後劉繼威,以城歸梁,末帝命思遠鎮之。莊宗平定魏博,以兵臨滄、德,思遠棄鎮渡河歸汴,累遷天平軍節度使兼北面招討使,將兵與莊宗對壘。久之,莊宗討張文禮于鎮州,契丹來援,莊宗追襲契丹至幽州。思遠聞之,總兵以襲魏州,至魏店,遇明宗騎軍適至,思遠乃涉洹水,陷成安,復歸楊村寨,盡率其眾,攻德勝北城。城中危急,符存審晝夜乘城以拒之。莊宗自薊五日馳至魏州,思遠聞之解去。及明宗襲下鄆州,思遠罷軍權,降授宣
 化軍留後。其年,莊宗入汴,思遠自鄧州入朝,復令歸鎮。明宗即位,移授洋州節度使。及西川俱叛,思遠以董璋故人,避嫌請代,徵入朝宿衛,以年告老,授太子少保致仕。清泰二年八月,卒于家。



 朱漢賓,字績臣,亳州譙縣人也。父元禮,始為郡將。梁太祖聞其名,擢為軍校,從龐師古渡淮,戰沒於淮南。漢賓少有膂力,形神壯偉,膽氣過人。梁祖以其父死王事,選置帳下,編入屬籍。梁祖之攻兗、鄆也,朱瑾募驍勇數百
 人,黥雙鴈于其頰,立為「鴈子都」。梁祖聞之,亦選數百人,別為一軍,號為「落鴈都」。署漢賓為軍使,當時目為「硃落鴈」。後與諸將破蔡賊有功,天復中,授右羽林統軍。入梁,歷天威軍使、左羽林統軍,出為磁州刺史、滑宋二州留後、亳曹二州刺史、安州節度使。莊宗至洛陽,漢賓自鎮入覲,復令還鎮。明年,授左龍武統軍。莊宗嘗幸漢賓之第,漢賓妻進酒上食,奏家樂以娛之,自是漢賓頗蒙寵待。同光四年正月,冀王朱友謙入朝,明宗居洛陽,以友
 謙故人,置酒于第。莊宗諸弟在席,友謙坐在永王存霸之上。酒酣,漢賓以大觴奉友謙曰:「公雖名位高,坐于皇弟之上,非宜也。僕與公俱在梁朝,以宗盟相厚,自公入朝,三發單函候問,略無報復,忽餘卑位,不亦甚乎!」元行欽恐其紛然,為解之方止。不數日,友謙赤族。趙在禮據魏州,元行欽率軍進討,詔漢賓權知河南府事。明宗以漢賓為右衛上將軍,樞密使安重誨方當委重,漢賓密令結託,得為婚家。天成末,為潞州節度使,移鎮晉州。重
 誨既誅,漢賓復為上將軍。明年秋,漢賓告老,授太子少保致仕。清泰二年六月卒,時年六十四。



 漢賓少勇健,及晚歲飲啖過人,其狀貌偉如也。凡所履歷,不聞踰法。梁時,嘗領軍屯魏州莘縣,適值連帥去郡,諸軍咸以利見誘,請自為留後,漢賓則斬其言者,拒而不從,聞者賞焉。在曹日,飛蝗去境,父老歌之。臨平陽遇旱,親齋潔禱龍子祠,踰日雨足,四封大稔,咸以為善政之所致也。及致仕,東還亳郡,見鄉舊親戚淪沒者,有塋兆未辦,則給以
 棺斂;有婚嫁未畢,則助以資幣,受其惠者數百家,郡人義之。尋還洛陽,有第在懷仁里,北限洛水,南枕通衢,層屋連甍,修木交乾,笙歌羅綺,日以自娛,養彼太和,保其餘齒,此乃近朝知止之良將也。晉高祖即位,贈太子少傅,謚曰貞惠。



 子四人,長曰崇勳,官至左武衛將軍。



 孔勍,字鼎文,兗州人,後徙家宿州。少便騎射,為軍中小校,事梁祖漸至郡守,累遷齊州防禦使、唐鄧節度使。梁貞明中,王球據襄州叛,勍討平之,因授山南東道節度使。莊宗至洛陽,
 勍自鎮來朝,復令歸鎮,尋移昭義節度使。同光季年,監軍楊繼源與都將謀據潞州,事泄,勍誅之。明宗即位之歲,詔還京師,授河陽節度使。未幾,以太子太師致仕,卒年七十九。贈太尉。



 劉,汴州雍邱人也。世為宣武軍牙將。少負壯節,梁祖鎮汴州,求自試,補隊長。從梁祖征伐,所至有功,遷為牙將,歷滑、徐、襄三州都指揮使。開平中,襄帥王班為帳下所害,亂軍推為留後,詭從之,翌日受賀,衙庭享士,伏甲幕下,盡斬其亂將。以功歷復、亳二州刺史,徵
 為侍衛都將,出為安州刺史。貞明中,為晉州留後。莊宗至汴,來朝。在晉州八年,日與上黨、太原之師交鬥於境上。莊宗見而勞之曰:「劉侯無恙,控我晉陽之南鄙,歲時久矣,不早相見。」頓首謝罪。復命歸鎮,正授節旄,移鎮安州。明宗即位,遷鄧州節度使。天成末,以史敬鎔代之。還京師,卒。贈侍中。



 有子師道,仕皇朝,為右贊善大夫。卒。



 周知裕,字好問,幽州人也。少事燕帥劉仁恭為騎將,表
 為媯州刺史。久之,移刺德州。天祐四年,劉守光既平滄州,乃以其幼子繼威為留後,大將張萬進與知裕佐之。繼威沖幼,宣淫于萬進之家,萬進殺之。詰旦,召知裕告其故,萬進自稱留後,署知裕為景州刺史。會萬進納款于梁,知裕先奔于汴,梁主厚待之。特置歸化軍,以知裕為指揮使。凡軍士自河朔歸梁者,皆隸于部下。梁與莊宗交戰於河,摧堅挫銳,惟恃歸化一軍,然歲將一紀,位不及郡守。同光初,莊宗入汴,知裕隨段凝軍解甲封丘。
 明宗時為總管,受降于郊外,見知裕甚喜,遙相謂曰:「周歸化今為吾人,何樂如之!」因令諸子以兄事之。莊宗撫憐尤異,而諸校心妒之。有壯士唐從益者,因獵射之,知裕遁而獲免。莊宗遂誅從益,出知裕為房州刺史。魏王繼岌伐蜀,召為前鋒騎將。明宗即位,移刺絳州,改淄州刺史、宿州團練使。知裕老于軍旅,勤于稼穡,凡為郡勸課,皆有政聲,朝廷喜之,遷安州留後。淮上之風惡病者,至于父母有疾,不親省視,甚者避于他室;或時問訊,即
 以食物揭于長竿之首,委之而去。知裕心惡之,召鄉之頑很者訶詰教導,俾知父子骨肉之恩,由是弊風稍革。長興末,入為右神武統軍。清泰初,卒于官。贈太傅。



 史臣曰:夫才之良者,在秦亦良也,在虞亦良也。故彥威而下,昔為梁臣,不虧亮節;洎歸唐祚,亦無醜聲,蓋松貞不變于四時,玉粹寧虞其烈焰故也。況彥威之輔明宗也,有翊戴之績;晏球之伐中山也,著戡定之功。方之數公,尤為優矣。



\end{pinyinscope}