\article{明宗紀八}

\begin{pinyinscope}

 長興二年春正月庚申朔,帝御明堂殿受朝賀,仗衛如儀。乙丑,詔曰:「故天策上將軍、守太師、尚書令、楚國王馬殷,品位俱高,封崇已極,無官可贈,宜賜謚及神道碑文,
 仍以王禮葬。」壬申,契丹東丹王托雲自渤海國率眾到闕,帝慰勞久之,錫賚加等,是日,百僚稱賀。丙子,以沙州節度使曹義金兼中書令。丁丑,東丹王托雲進本國印三紐。庚辰,以靜江軍節度使馬賓卒廢朝,贈尚書令。丙戌,荊南節度使高從誨落起復,加兼中書令。



 二月己丑朔,以宋州節度使趙延壽為左武衛上將軍,充宣徽北院使。癸巳,詔貢院舊以例夜試進士,今後晝試,排門齊入,即日試畢。丁酉,幸至德宮,又幸安元信、東丹王托雲之第。辛
 丑,以鴻臚卿致仕賈馥卒廢朝。以樞密院使、守太尉、兼中書令安重誨為檢校太師、兼中書令,充河中節度使,進封沂國公。己酉,以右威衛上將軍陳皋為洋州節度使。詔諸府少尹上任,以二十五日為限。諸州刺史、諸道行軍司馬、副使、兩使判官已下賓職,團防軍事判官、推官、府縣官等,並以三十日為限。凡幕職隨府者不在此例。癸丑,邠州節度使李敬周移鎮徐州。詔禁天下不得開發無主墳墓。



 三月辛酉,詔渤海國人皇王托雲宜賜姓東丹,
 名慕華,仍授檢校太保、安東都護,充懷華軍節度、瑞鎮等州觀察等使。其從慕華歸國部校,各授懷化、歸德將軍中郎將。先于定州擒獲蕃將,特哩袞宜賜姓狄,名懷惠,扎古宜賜姓列,名知恩,並授檢校右散騎常侍。錫里扎拉宜賜姓原,名知感;裕勒古宜賜姓服,名懷造;奚王副使格斯齊宜賜姓乙,名懷宥,三人並授檢校太子賓客。甲子,以前鴻臚卿王瓊為太僕卿。丙寅,以皇子從珂為左衛大將軍。從珂自河中失守,歸清化里第,至是安
 重誨出鎮河中,帝召見,泣而謂之曰:「如重誨意,爾安得更相見耶!」因有是命。壬申,以滄州節度使孔循卒廢朝。乙亥,以西京留守、權知興元軍府事王思同為山南西道節度使,充西面行營馬步軍都虞候。庚辰,以少府監聶延祚為殿中監,以前雲州節度使楊漢章為安州節度使。乙酉,太師致仕錢鏐復授天下兵馬都元帥、尚父、吳越國王,以其子兩浙節度使元瓘等上表首罪,故有是命。丁亥,以太常卿李愚為中書侍郎、平章事、集賢殿
 大學士。



 夏四月辛卯,制德妃王氏進位淑妃。詔錢鏐依舊賜不名。誅內官安希倫,以其受安重誨密指,令于內中伺帝起居故也。丁酉,幸會節園,宴群臣,因幸河南府。詔罷州縣官到任後率斂為地圖;又禁人毀廢所在碑碣,戊戌,詔今年四月禘饗太廟。故昭義節度使李嗣昭、故幽州節度使周德威、故汴州節度使符存審,並配饗莊宗廟廷。己亥,以前徐州節度使張虔釗為鳳翔節度使。癸卯,以汴州節度副使藥縱之為戶部侍郎,前宗正
 卿李諧為將作監。甲辰,以宣徽北院使、左衛上將軍趙延壽為檢校太傅、行禮部尚書,充樞密使。乙巳,潞州節度使劉仲殷移鎮秦州。帝幸龍門佛寺祈雨。己酉,天雄軍節度使石敬瑭兼六軍諸衛副使。辛亥,以前鳳翔節度使朱宏昭為左武衛上將軍,充宣徽南院使。壬子,以兵部尚書盧質為河陽節度使。甲寅,以遂州節度使夏魯奇沒于王事廢朝。詔曰:「久摐時雨,深疚予心。宜委諸州府長吏親問刑獄,省察冤濫,見禁囚徒,除死罪外,並放。」



 五月戊午朔,帝御文明殿受朝。庚申,以三司使、行工部尚書張延朗為兗州節度使。辛酉,詔:「近聞百執事等,或親居內職,或貴列廷臣,或宣達君恩,或勾當公事,經由列鎮,干撓諸侯,指射職員,安排親暱,或潛示意旨,或顯發書題。自今後一切止絕,有所犯者,發薦人貶官,求薦人流配。如逐處長吏自徇人情,只仰被替人詣闕上訴,長吏罰兩月俸,發薦人更加一等,被替人卻令依舊。」甲子,都官郎中、知制誥崔棁上言,請搜訪宣宗已來野史,
 以備編修,從之。丁卯,詔:「諸州府城郭內依舊禁曲,其曲官中自造,減舊價之半貨賣。應田畝上所征曲錢並放,鄉村人戶一任私造。」時甚便之。戊辰,中書奏,應朝臣丁憂者,望加頒賚,從之。丁丑,以祕書監劉岳為太常卿。己卯,以武德使孟漢瓊為右衛大將軍、知內侍省,充宣徽北院使。辛巳,以前相州刺史孟鵠為左驍衛大將軍,充三司使。甲申,以權知朗州軍州事、守永州刺史馬希範為洪州節度使、檢校太傅;以權知桂州軍府事、富州刺
 史馬希彞為鄂州節度使、檢校司徒。乙酉,以左金吾大將軍薄文為晉州留後。鴻臚卿柳膺將齋郎文書賣與同姓人柳居則,伏罪,大理寺斷當大辟,緣經赦減死,追奪見任官,終身不齒。詔:「應見任前資守選官等,所有本朝及梁朝出身歷任告身,並仰送納,委所在磨勘,換給公憑,只以中興已來官告,及近受文書敘理。其諸色廕補子孫,如非虛假,不計庶嫡,並宜敘錄;如實無子孫,別立人繼嗣,已補得身名者,只許敘廕一人。其不合敘使
 文書,限百日內焚毀須絕。此後更敢將合焚文書參選求仕,其所犯之人並傳者,並當極法。應合得資蔭出身人,並須依格依令施行。」



 閏月庚寅,制河中節度使、檢校太師、兼中書令安重誨可太子太師致仕。是日,重誨男崇緒等潛歸河中。以右散騎常侍張文寶為兵部侍郎。夔州節度使安崇阮棄城歸闕,待罪于閣門,詔釋之。時董璋寇峽內諸州臨川人,故名。注重訓釋《詩》、《書》、《周禮》三經義,強,崇阮望風遁走。壬辰,陜州節度使李從璋移鎮河中。癸丑,升廬州為昭順軍。甲午,以衡州刺
 史姚彥章為昭順軍節度使。丁酉,安重誨奏:「男崇贊、崇緒等到州,臣已拘送赴闕。」崇緒至陜州,詔令下獄。已亥,詔安重誨宜削奪在身官爵,並妻阿張、男崇贊崇緒等並賜死,其餘親不問。壬寅,以尚書左丞崔居儉為工部尚書,以吏部侍郎王權為尚書左丞。丙午,以隨駕馬軍都指揮使、宣州節度使安從進為陜州節度使。丁未,以前中書舍人楊凝式為左散騎常侍。戊申,以右龍武統軍王景戡為新州節度使。己酉,以右領軍上將軍李肅
 為左金吾大將軍。壬子,以隨駕步軍都指揮使藥彥稠為邠州節度使。癸丑,以邠州節度使劉行琮卒廢朝,贈太傅。詔有司及天下州縣,于律令、格式、《六典》中錄本局公事,書于壁,令其遵行。



 六月丁巳朔,復置明法科,同《開元禮》。乙丑,以皇子左衛大將軍從珂依前檢校太傅,加同平章事、行京兆尹,充西都留守。庚午,以邠州節度使張溫為右龍武統軍。甲戌,以魏徵八代孫韶為安定縣主簿。乙亥,以鎮州節度使、宋王從厚為興唐尹,以石
 敬瑭為河陽天雄軍節度使,以天雄軍節度使石敬班為河陽節度使,依前六軍諸衛副使。丙子,詔諸道觀察使均補苗稅,將有力人戶出剩田畝,補貧下不迨頃畝,有嗣者排改檢括,自今年起為定額。乙卯,定州節度使李從敏移鎮州節度使,盧質為滄州節度使。庚辰,皇孫太子舍人重美授司勳員外郎,重真已下六人並授同正將軍及檢校官。壬午,以前秦州節度使李德珫為定州節度使兼北面行營副招討使。太原地震。詔天下州
 府斷獄,先于案牘之上坐所該律令、格式及新敕,然後區分。乙酉,以前黔州節度使楊漢賓為羽林統軍。詔止絕諸射係省店宅莊園。



 秋七月庚寅,以權侍衛馬軍都指揮使、登州刺史張從實為壽州節度使兼侍衛步軍都指揮使。壬辰,福建王延鈞上言:「當境廟七所,乞封王號。」敕:「如諸史傳有名,宜封為閩越富義王,其餘任自于境內祭享。」乙未,詔:「諸道奏薦州縣官,使相先許一年薦三人,今許薦五人;不帶使相先許薦二人,今許薦三人;
 直屬京防禦、團練使先許薦一人,今許薦二人。」詔:「應州縣官內,有曾在朝行及曾佐幕府,罷任後,準前資朝官賓從別處分。其帶省銜,并內供奉裏行及諸已出選門者,或降授令錄,罷任日,並依出選門例處分,便與除官,更不在赴常調。州縣官其間書得十六考者,準格敘加朝散階,亦準出選門例處分。」三司奏:「先許百姓造曲,不來官場收買。伏恐課額不逮,請復已前曲法,鄉戶與在城條法一例指揮,仍據已造到曲納官,量支還麥本。」從
 之。甲辰,前晉州節度使朱漢賓授太子少保致仕。庚戌,大理正劇可久責授登州司戶,刑部員外郎裴選責授衛尉寺丞,刑部侍郎李光序、判大理卿事任贊各降一官,罰一季俸,坐斷罪失入也。



 八月丙寅,詔天下州府商稅務,並委逐處差人依省司年額勾當納官。以故鎮州節度使、趙王王熔男昭誨為朝議大夫、司農少卿,賜紫金魚袋,繼絕也。辛丑,升虔州為昭信軍。癸亥,以太常少卿盧文紀為祕書監,以秘書監馬縞為太子賓客,左監
 門上將軍羅周敬為右領軍上將軍,前懷州刺史婁繼英為左監門上將軍。乙丑,詔:「大理寺官員,宜同臺省官例升進,法直官比禮直官任使。仍于諸道贓罰錢內,每月支錢一百貫文,賜刑部、大理兩司,其刑部于所賜錢三分與一分。」丙寅,以武平軍節度使馬希振依前檢校太尉、兼侍中,充虔州昭信軍節度使。詔:「百官職吏,應選授外官者,考滿日,並委本州申奏,追還本司,依舊職行公事。」己巳,太傅致仕王建立、太子少保致仕朱漢賓皆
 上章求歸鄉里。詔內外致仕官,凡要出入,不在拘束之限。辛未,以翰林學士、兵部侍郎劉煦守本官,充端明殿學士;以左拾遺、直樞密院李崧充樞密直學士。壬申,以左龍武統軍李承約為潞州節度使。癸酉,詔:「文武百官,五日內殿起居仍舊,其輪次轉對若有封事,許非時上表,朔望入閣,待制候對,一依舊制。」乙亥,翰林學士、工部侍郎竇夢征卒。丁丑,以前西京副留守梁文矩為兵部尚書。己卯,詔不得薦銀青階為州縣官。壬午,詔應有朝
 臣、籓侯、郡守,凡欲營葬,未曾封贈,許追封贈。禮部尚書致仕李德休卒。



 九月丙戌,以前兗州節度使符彥超為左龍武統軍。己亥,懷化軍節度使東丹慕華賜姓李名贊華,改封隴西縣開國公。應有先配諸軍契丹並賜姓名。詔天下營田務,只許耕無主荒田各招浮客,不得留占屬縣編戶。辛丑,樞密使、檢校太傅、刑部尚書范延光加同平章事,使如故。壬寅,以中書舍人封翹為禮部侍郎,禮部侍郎盧澹為戶部侍郎。癸卯,許州節度使李從
 溫移鎮河東。詔天下州縣官,不得與部內富民于公同坐。辛亥,詔五坊見在鷹隼之類,並可就山林解放,今後不許進獻。



 冬十月戊午,以前北京留守、太原尹馮贇為許州節度使。辛酉,左補闕李詳上疏:「以北京地震多日,請遣使臣往彼慰撫,察問疾苦,祭祀山川。」從之。先是,太原留後密奏,無敢言者,及詳有是奏,帝甚嘉之,改賜章服。丙寅,詔:「應在朝臣僚、籓侯、郡守,準例合得追贈者,新授命後,便于所司投狀,旋與施行。封妻蔭子,準格合
 得者,亦與施行。外官曾任朝班,據在朝品秩格例,合得封贈敘封者,並與施行。其補廕,據資廕合得者,先受官者先與收補,後受官者據月日次第施行。」從之。



 十一月甲申朔,日有食之。己丑,日南至,帝御文明殿受賀。丁酉,以翰林學士、起居郎張歷為兵部員外郎、知制誥充職;以汝州防禦使張希崇為靈州兩使留後。庚子,以左威衛上將軍華溫琪為華州節度使。福州節度使王延鈞奏,誅建州節度使王延稟及其子繼雄。壬寅,詔今後諸
 道兩使判官罷任一年與比擬,書記、支使、防禦團練判官二年,推巡、軍事判官並三年後與比擬。仍每遇除授,量與改轉官資或階勛、職次云。以御史中丞劉贊為刑部侍郎,以鳳州節度使孫岳充西面閣道使。壬子,鄆州奏,黃河暴漲,漂溺四千餘戶。癸丑,以給事中崔衍為御史中丞。



 十二月甲寅朔,詔開鐵禁,許百姓自鑄農器、什器之屬,于夏秋田畝上,每畝輸農器錢一錢五分。乙卯又日新。」《易傳·系辭上》:「日新之謂盛德。」後世學者多以,畋于西郊。丁巳,以彰武軍節度使劉訓卒廢朝。庚午,以前
 利州節度使康思立為陜州節度使。秦州地震。丁丑,詔三司,所過西川兵士家屬,常令贍給。



\end{pinyinscope}