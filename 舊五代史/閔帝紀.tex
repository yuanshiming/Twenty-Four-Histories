\article{閔帝紀}

\begin{pinyinscope}

 閔帝,諱從厚,小字菩薩奴,明宗第三子也。母昭懿皇后夏氏,以天祐十一年歲在甲戌,十一月二十八日庚申,生帝於晉陽舊第。帝髫齔好讀《春秋》,略通大義,貌類明
 宗,尤鐘愛。



 天成元年,授金紫光祿大夫、檢校司徒。二年四月,加檢校太保、同平章事、河南尹,判六軍諸衛事。十一月,加檢校太傅。三年三月,授汴州節度使。四年,移鎮河東。長興元年,改授鎮州節度使,尋封宋王。二年,加檢校太尉、兼侍中,移鎮鄴都。三年,加中書令。秦王從榮,帝同母兄也,以帝有德望,深所猜忌。帝在鄴宮,恆憂其禍,然善於承順,竟免閑隙。



 四年十一月二十日,秦王誅。翼日,明宗遣宣徽使孟漢瓊馳驛召帝,二十六日,明宗崩,
 二十九日,帝至自鄴。十二月癸卯朔,發喪於西宮,帝於柩前即位。丁未,群臣上表請聽政,表再上,詔允。己酉,中外將士給賜有差。庚戌,帝縗服見群臣於廣壽門之東廡下,宰臣馮道進曰:「陛下久居哀毀,臣等咸願一睹聖顏。」硃宏昭前舉帽,群臣再拜而退。禦光政樓存問軍民。辛亥,賜司衣王氏死,坐秦王事也。癸丑,以前鎮州節度使、涇王從敏權知河南府事,尋以盧質代之。乙卯,賜司儀康氏死,事連王氏也。丙辰,以天雄軍節度判官唐汭
 為諫議大夫,掌書記趙彖為起居郎,元從都押衙宋令詢為磁州刺史。丁巳,以左僕射、平章事馮道為山陵使,戶部尚書韓彥惲為副,中書舍人王延為判官,禮部尚書王權為禮儀使,兵部尚書李鈴為鹵簿使,御史中丞龍敏為儀仗使,右僕射、權知河南府盧質為橋道頓遞使。庚申,以前相州刺史郝瓊為右驍衛大將軍,充宣徽北院使;以光祿卿、充三司副使王玫為三司使。癸亥,故檢校太尉、右衛上將軍、充三司使孫岳贈太尉、齊國公。
 丁卯,帝釋縗服,群臣三上表,請復常膳,御正殿,從之。辛未,帝御中興殿,群臣列位,馮道升階進酒。帝曰:「比於此物無愛,除賓友之會,不近樽斝。況在沉痛之中,安事飲啖!」命徹之。


應順元年春正月壬申朔,帝御廣壽殿視朝,百僚詣閣門奉慰。時議者云,月首以朝服臨,不視朝可也。乙亥,契丹遣使朝貢。
 \gezhu{
  《遼史·太宗紀》:天顯九年閏月戊午,唐遣使來告哀,即日遣使祭吊。}
 丁丑,以太常卿崔居儉為秘書監,以前蔡州刺史張繼祚為左
 武衛上將軍,充山陵橋道頓遞副使。戊寅,御明堂殿,仗衛如儀,宮懸樂作,群臣朝服就位,宣制大赦天下,改長興五年為應順元年。時議者以梓宮在殯,宮縣樂作,非禮也,懸而不作可也。回鶻可汗仁美遣使貢方物,故可汗仁裕進遺留馬。是日,命中使三十五人以先帝鞍馬衣帶分賜籓位。



 庚辰,宰臣馮道加司空,李愚加右僕射,劉煦加吏部尚書,餘並如故。壬午,侍衛親軍馬步軍都指揮使、河陽節度使康義誠加檢校太尉、兼侍中,判六
 軍諸衛事。甲申,以侍衛馬軍都指揮使、寧國軍節度使安彥威為河中節度使;以侍衛步軍都指揮使、忠正軍節度使張從賓為涇州節度使,並加檢校太傅;以捧聖左右廂都指揮使、欽州刺史硃洪實為寧國軍節度使,加檢校太保,充侍衛馬軍都指揮使;以嚴衛左右廂都指揮使、巖州刺史皇甫遇為中正軍節度使、檢校太保,充侍衛步軍都指揮使。戊子,樞密使、檢校太尉、同平章事硃宏昭,樞密使、檢校太尉、同中書門下二品馮贇並
 加兼中書令。北京留守、河東節度使兼大同彰國振武威塞等軍蕃漢馬步總管石敬瑭加兼中書令;幽州節度使、檢校太尉、兼中書令趙德鈞加檢校太師、兼中書令。樞密使馮贇表堅讓中書令,制改兼侍中,封邠國公。庚寅,鳳翔節度使、潞王從珂加兼侍中;青州節度使、檢校太尉、兼中書令房知溫加檢校太師。辛卯,以翰林學士承旨、尚書右丞李懌為工部尚書,以秘書監盧文紀為太常卿,充山陵禮儀使。壬辰,荊南節度使、檢校太尉、
 兼中書令高從誨封南平王;湖南節度使、檢校太尉、兼中書令馬希範封楚王。甲午,兩浙節度使、檢校太師、守中書令、吳王錢元瓘進封吳越王;前洺州團練使皇甫立加檢校太保,充鄜州節度使;前彰義軍節度使康福加檢校太傅,充邠州節度使;劍南東、西兩川節度使、檢校太尉、兼中書令、蜀王孟知祥加檢校太師。制下,知祥辭不受命。丙申,鎮州節度使、檢校太尉、兼侍中範延光,汴州節度使、檢校太尉、兼侍中趙延壽,並加檢校太師。
 戊戌,山南西道節度使、檢校太傅、同平章事張虔釗,襄州節度使趙在禮,並加檢校太尉。辛丑,以振武軍節度使、安北都護楊檀兼大同、彰國、振武、威塞等軍都虞候,充北面馬軍都指揮使。


閏月壬寅朔,群臣赴西宮臨。癸卯,御文明殿入閣。以前右僕射、權知河南府事盧質為太子少傅兼河南尹。以諫議大夫唐汭、膳部郎中知制誥陳乂並為給事中,充樞密院直學士。
 \gezhu{
  《通鑒》:汭以文學從帝,歷三鎮在幕府。及即位,將佐之有才者,朱、馮皆斥逐之。汭性迂疏,硃、馮恐帝含怒有時而發,乃引汭于密近,以其黨陳乂
  監之。}
 宣徽南院使、驃騎大將軍、左衛上將軍、知內侍省孟漢瓊加開府儀同三司,賜忠貞扶運保泰功臣。丙午,正衙命使冊皇太后曹氏。戊申,以前雄武軍節度使劉仲殷為右衛上將軍,邢州節度使趙鳳加爵邑。自是諸籓鎮文武臣僚皆次第加恩,帝嗣位覃恩澤也。以翰林學士、中書舍人崔棁為工部侍郎,依前充職。以給事中張鵬為御史中丞,以御史中丞龍敏為兵部侍郎,以太僕少卿竇維為大理卿。甲寅,正衙命使冊皇太妃王氏。集
 賢院上言:「準敕書修創凌煙閣,尋奉詔問閣高下等級。謹按凌煙閣,都長安時在西內三清殿側,畫像皆北面,閣有中隔,隔內面北寫功高宰輔,南面寫功高諸侯王,隔外面次第圖畫功臣題贊。自西京板蕩,四十餘年,舊日主掌官吏及畫像工人,並已淪喪,集賢院所管寫真官、畫真官人數不少,都洛後廢職。今將起閣,望先定佐命功臣人數,請下翰林院預令寫真本,及下將作監興功,次序間架修建。」乃詔集賢御書院復置寫真官、畫真
 官各一員,餘依所奏。丁巳,安州奏,此月七日夜,節度使符彥超為部曲王希全所害,廢朝一日。戊午,以前振武軍節度使、安北都護高行周為彰武節度使。辛酉,以前鄆州囗使范政為少府監。丙寅,幸至德宮。車駕至興教門,有飛鳶自空而墜,殭于御前。是日大風晦冥。



 二月乙亥,以前鎮州節度使、涇王從敏為宋州節度使。己卯,以前徐州節度使、檢校太傅李敬周為安州節度使。是日,宣授鳳翔節度使、潞王從珂為權北京留守;以北京留守
 石敬瑭權知鎮州軍州事;以鎮州范延光權知鄴都留守事;以前河中節度使、洋王從璋權知鳳翔軍軍府事。庚寅,幸山陵工作所。是日,西京留守王思同奏,鳳翔節度使、潞王從珂拒命。丁酉,王思同加同平章事,充西面行營都部署;以前邠州節度使藥彥稠為副部署。以河中節度使安彥威為西面兵馬都監,以前定州節度使李德珫為權北京留守。山陵使奏:「伏睹御札,皇帝親奉靈駕至園陵。伏見累朝故事,人君無親送葬之儀,請車
 駕不行。」不從。乙未,樞密使馮贇起復視事,時贇丁母憂也。己亥,以司農卿張鎛為殿中監。庚子,殿直楚匡祚上言,監取亳州團練使李重吉至宋州,系於軍院。重吉,潞王之長子,及幽於宋州,帝猶以金帛賜之,及聞西師咸叛,方遣使殺之。



 三月甲辰,以前太僕少卿魏仁鍔為太僕卿。興元節度使張虔釗奏,會合討鳳翔。丙午,以右領衛上將軍武延翰為郢州刺史。丁未,洋州孫漢韶奏,至興元與張虔釗同議進軍。己酉,以鎮州節度使范延
 光依前檢校太師、兼侍中,行興唐尹,充天雄軍節度使、北面水陸轉運制置使;以北京留守、河東節度使石敬瑭依前檢校太尉、兼中書令,其真定尹、充鎮州節度使、大同彰國振武威塞等軍蕃漢馬步總管如故。辛亥,以前定州節度使李德珫為北京留守,充河東節度使。許王從益加檢校太保,前河中節度使、洋王從璋加檢校太傅。詔:「籓侯帶平章事以上薨,許立神道碑,差官撰文。未帶平章事及刺史,準令式合立碑者,其文任自製撰,不在奏聞。」乙卯,興元張虔釗奏,自鎮將
 兵赴鳳翔,收大散關。宗正寺奏:「準故事,諸陵有令、丞各一員,近例更委本縣令兼之。緣河南洛陽是京邑,兼令、丞不便。」詔特置陵臺令、丞各一員。己未,以前金吾大將軍李肅為左衛上將軍,充山陵修奉上下宮都部署。



 庚申,西面步軍都監王景從等自軍前至,奏:「今月十五日,大軍進攻鳳翔。十六日,嚴衛右廂都指揮使尹暉引軍東面入城,右羽林都指揮使楊思權引軍西面入城,山南軍潰。」帝聞之,謂康義誠等曰:「朕幼年嗣位,委政大臣,
 兄弟之間,必無榛梗。諸公大計見告,朕獨難違,事至于此,何方轉禍?朕當與左右自往鳳翔,迎兄主社稷;朕自歸籓,于理為便。。」朱宏昭、馮贇不對,義誠曰:「西師驚潰,蓋由主將失策。今駕下兵甲尚多,臣請自往關西,振其兵威,扼其衝要。」義誠又累奏請行,帝召侍衛都將以下宣曰:「先皇帝棄萬國,朕于兄弟之中,無心爭立,一旦被召主喪,便委社稷,岐陽兄長,果致猜嫌。卿等頃從先朝千征萬戰,今日之事,寧不痛心!今據府庫,悉以頒賜,卿等
 勉之!」乃出銀絹錢厚賜于諸軍。是時方事山陵,復有此賜,府藏為之一空,軍士猶負賞物揚言于路曰:「到鳳翔更請一分。」其驕誕無畏如是。辛酉,幸左藏庫,視給將士金帛。是日,誅馬軍都指揮使硃洪實,坐與康義誠忿爭故也。



 癸亥,以康義誠為鳳翔行營都招討使,餘如故。以王思同為副招討使;以安從進為順化軍節度使,充侍衛馬軍都指揮使。詔左右羽林軍四十指揮改為嚴衛,左右龍武、神武軍改為捧聖。甲子,陜州奏,潞王至潼關,
 害西面都部署王思同。乙亥,宣諭西面行營將士,俟平鳳翔日,人賞二百千,府庫不足,以宮闈服玩增給。詔侍衛馬軍都指揮使安從進京城巡檢。是日,從進已得潞王書檄,潛布腹心矣。丁卯,潞王至陜州。戊辰,帝急召孟漢瓊,不至;召硃宏昭,宏昭懼,投於井。安從進尋殺馮贇于其第。是夜,帝以百騎出元武門,謂控鶴指揮使慕容遷曰:「爾誠有馬,控鶴從予。」及駕出,即闔門不行。遷乃帝素親信者也,臨危如是,人皆惡之。



 是月二十九日夜,帝
 至衛州東七八里,遇騎從自東來不避,左右叱之,乃曰:「鎮州節度使石敬瑭也。」帝喜,敬瑭拜舞于路,帝下馬慟哭,諭以「潞王危社稷,康義誠以下叛我,無以自庇,長公主見教,逆爾於路,謀社稷大計。」敬瑭曰:「衛州王宏贄宿舊諳事,且就宏贄圖之。」敬瑭即馳騎而前,見宏贄曰:「主上播遷,至此危迫,吾戚屬也,何以圖全?」宏贄曰:「天子避狄,古亦有之,然于奔迫之中,亦有將相、國寶、法物,所以軍長瞻奉,不覺其亡也。今宰職近臣從乎?寶玉、法物從
 乎?」詢之無有。宏贄曰:「大樹將顛,非一繩所維。今以五十騎奔竄,無將相一人擁從,安能興復大計!所謂蛟龍失雲雨者也。今六軍將士總在潞邸矣,公縱以戚籓念舊,無奈之何!」遂與宏贄同謁于驛亭,宣坐謀之。敬瑭以宏贄所陳以聞,弓箭庫使沙守榮、奔洪進前謂敬瑭曰:「主上即明宗愛子,公即明宗愛婿,富貴既同受,休戚合共之。今謀於戚籓,欲期安復,翻索從臣、國寶,欲以此為辭,為賊算天子耶!」乃抽佩刀刺敬瑭,敬瑭親將陳暉捍之,
 守榮與暉單戰而死,洪進亦自刎。是日,敬瑭盡誅帝之從騎五十餘輩,獨留帝於驛,乃馳騎趨洛。



 四月三日,潞王入洛。五日,即位。七日,廢帝為鄂王。遣弘贄子殿直王巒之衛州,時宏贄已奉帝幸州廨。九日,巒至,帝遇鴆而崩,時年二十一。是日辰時,白虹貫日。皇后孔氏在宮中,及王巒回,即日與其四子並遇害。晉高祖即位,謚曰閔,與秦王及末帝子重吉並葬於徽陵域中,封纔數尺,路人觀者悲之。



 史臣曰:閔帝爰自沖年,素有令問,及徵從代邸,入踐堯階,屬軒皇之弓劍初遺,吳王之幾杖未賜,遽生猜間,遂至奔亡。蓋輔臣無安國之謀,非少主有不君之咎。以至越在草莽,失守宗祧,斯蓋天命之難忱,土德之將謝故也。



\end{pinyinscope}