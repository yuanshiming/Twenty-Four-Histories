\article{明宗紀四}

\begin{pinyinscope}

 天成二年春正月癸丑朔,帝御明堂殿受朝賀,仗衛如常儀。制曰:「王者祗敬宗祧,統臨寰宇,必順體元之典,特新制義之文。朕以眇躬,獲承丕構,襲三百年之休運,繼
 二十聖之耿光。馭朽納隍,夕惕之心罔怠;法天師古,日躋之道惟勤。今則載戢干戈,渾同書軌,荷上穹之眷祐,契兆庶之樂推。檢玉泥金,非敢期于薄德;耕田鑿井,誠有慕于前王。將陳享謁之儀,即備郊丘之禮,宜更稱謂,永耀簡編。今改名為亶,凡在中外,宜體朕懷。」宣制訖,百僚稱賀,有司告郊廟社稷。



 丙辰,詔:「端明殿學士班位宜在翰林學士之上,今後如有轉改,只于翰林學士內選任。」先是,端明殿學士班在翰林學士之下,又如三館例,
 官在職上,趙鳳轉侍郎日,諷宰相府移之。既而禁林序列有不可之言,安重誨奏行此敕,時論便之。癸亥,宰臣鄭玨加特進、門下侍郎兼太微宮使;崇文館大學士任圜加光祿大夫、門下侍郎、監修國史;以端明殿學士、尚書兵部侍郎馮道為中書侍郎、平章事、集賢殿大學士;以太常卿崔協為中書侍郎、平章事。戊辰,以前鄧州節度使劉卒,廢朝。左拾遺李同上言:「天下繫囚,請委長吏逐旬親自引問,質其罪狀真虛,然後論之以法,庶無
 枉濫。」從之。



 辛未,皇子河中節度使從珂加同平章事。以鎮州留後、檢校司徒王建立為鎮州節度使、檢校太傅。癸酉,皇第三子金紫光祿大夫、檢校司徒從厚加檢校太保、同平章事、河南尹,判六軍諸衛事。北面副招討房知溫奏,營州界奚陀羅支內附。乙亥,以監門衛大將軍傅璉為右武衛上將軍。丙子,詔曰:「頃自本朝多難,雅道中微,皆尚浮華,罕持廉讓。其有除官蘭省,命秩柏臺,或以人事相疏,或以私讎見訝,稍乖敬奉,遽至棄捐,蓋司
 長之振威,處君恩而何地。今後應新授官朝謝後,可準例上事,司長不得輒以私事阻滯。其本官亦不得因遭抑挫,托故請假。」



 戊寅,皇子從厚領事于河南府,宰相鄭玨已下會送,非例也。己卯,樞密使、光祿大夫、檢校太保、行兵部尚書安重誨加開府儀同三司、檢校太傅、兼侍中;樞密使、檢校太保、守秘書監孔循加檢校太傅、同平章事。詔崇文館依舊為宏文館。初,同光中,宰相豆盧革以同列郭崇韜父名宏,希其意奏改之,今乃復焉。辛巳,詔
 曰:「亂離斯久,法制多隳,不有舉明,從何禁止。起今後三京及州使職員名目,是押衙兵馬使,騎馬得有暖坐。諸都軍將衙官使下繫名糧者,只得衣紫皁,庶人商旅,只著白衣,此後不得參雜。兼有富戶,或投名于勢要,以求影庇;或希假于攝貴,以免丁徭。仰所在禁勘,以肅奸欺。」



 二月壬午朔,新羅遣使朝貢。丁亥,以北京皇城使李繼朗為龍武大將軍,北京都指揮使李從臻為左衛大將軍,捧聖都指揮使李從璨為右監門衛大將軍。戊子
 ,以前北面水陸轉運招撫使、守冀州刺史烏震領宣州節度使。庚寅,陜州節度使、檢校司徒石敬瑭加檢校太傅兼六軍諸衛副使。壬辰,西川節度使孟知祥奏,泗州防禦使、充西川兵馬都監李嚴,扇搖軍眾,尋已處斬。以潁州刺史孫岳為耀州團練使。丙申,以從馬直指揮使郭從謙為景州刺史,尋令中使誅之,夷其族,以其首謀大逆以弒莊宗也。以尚書左丞崔沂為太子少保致仕。壬寅,制曰:荊南節度使、開府儀同三司、守太尉、兼尚書
 令、南平王高季興可削奪官爵,仍令襄州節度使劉訓充南面招討使、知荊南行府事,許州節度使夏魯奇為副招討使,統蕃漢馬步四萬人進討,以其叛故也。又命湖南節度使馬殷以湖南全軍會合。以東川節度使董璋充東南面招討使,新授夔州刺史西方鄴為副招討使,共領川軍下峽州,三面齊進。《通鑑考異》:梓、夔皆在荊南之西南,而云東南面者,蓋據夔、梓所向言之。



 甲辰,兗州節度使房知溫加同平章事,宋州節度使王晏球加檢校太傅。丁未,以禮部尚書蕭頃
 為太常卿。戊申,以御史大夫李琪為右僕射,以太子賓客李鈴為戶部尚書,以吏部侍郎李德休為禮部尚書,以前吏部侍即崔貽孫為吏部侍郎,以端明殿學士、戶部侍郎趙鳳為兵部侍郎,依前充職。庚戌,詔諸道節度使男及親嫡骨肉未沾恩命者,特許上聞。河南府新安縣宜為次赤,以雍陵在其界故也。辛亥,以刑部侍郎歸藹為戶部侍郎。



 三月壬子朔,以中書舍人馬縞為刑部侍郎。幸會節園,宰相、樞密使及在京節度使共進錢絹,
 請開宴。癸丑,遣供奉官賈俊使淮南。甲寅,以西川節度副使李敬周為遂州武信軍留後。乙卯,開府儀同三司、司徒致仕趙光逢可太保致仕,仍封齊國公。以武信軍節度使李紹文卒廢朝。丙辰,宰臣判三司任圜奏:「諸道籓府,請依天復三年已前許貢綾絹金銀,隨其土產折進馬之直。又請選孳生馬,分置監牧。」並從之。《五代會要》:任圜奏:三京留守、諸道節度觀察、諸州防禦使、刺史,每年應聖節及正、至等節貢奉,或討伐勝捷,各進獻馬。伏見本朝舊事,雖以獻馬為名,多將綾絹金銀折充馬價,蓋跋涉之際,護養稍難,因此群方俱為定制。自今後伏乞除蕃部
 進駝馬外,諸州所進馬,許依天復三年已前事例,隨其土產折進價直,冀貢輸之稍易,又誠敬之獲申。兼欲于諸處揀孳生馬,準舊制分置監牧,仍委三司使別具制置奏聞。太常丞段顒請國學《五經》博士各講本經,以申橫經齒胄之義,從之。庚申,以前澤潞節度使、檢校太傅、兼侍中孔勍為河陽節度使。壬戌,幸甘水亭。甲子,青州節度使霍彥威加檢校太尉、兼中書令,以大內皇城使、守饒州刺史李從璋為應州節度使。丁卯,詔:「所在府縣糾察殺牛賣肉,犯者準條科斷。其自死牛即許貨賣,肉斤不得過五錢,鄉村民家死牛,
 但報本村所由,準例輸皮入官。」癸酉,以戶部郎中、知制誥盧詹為中書舍人。



 夏四月辛巳朔,房知溫奏:「前月二十一日,盧臺戍軍亂,害副招討寧國軍節度使烏震,尋與安審通斬殺亂兵訖。」帝聞之,廢朝一日,贈震太傅。新羅國遣使貢方物。丁亥,以華州留後劉彥琮為本州節度使。是日,幸會節園宴近臣。己丑,以兵部侍郎崔居儉權知尚書左丞,以戶部侍郎王權為兵部侍郎,以禮部侍郎裴皞為戶部侍郎,以翰林承旨、守中書舍人李愚
 為禮部侍郎充職。庚寅,御史臺奏:「今月三日廊下餐,百官坐定,兩省官方來,自五品下輒起。」詔曰:「每赴廊餐,如對御宴,若行私禮,是失朝儀,各罰半月俸。」《五代會要》:長興三年五月詔:文武兩班,每遇入閣賜食,從前御史臺官及諸朝官皆在敷政門外兩廊食,惟北省官於敷政門內別坐,既為隔門,各不相見,致行坐不齊,難于肅整。今後每遇入閣賜食,北省官亦宜于敷政門外東廓下設席,以北首為上,待班齊一時就坐。



 詔:「盧臺亂軍龍晊所部鄴都奉節等九指揮三千五百人在營家口骨肉,並可全家處斬。」龍晊所部之眾,即梁故魏博節度使楊師厚之所招置也,皆天下
 雄勇之士,目其都為銀槍效節,僅八千人。師厚卒,賀德倫不能制,西迎莊宗入魏,從征河上,所向有功。莊宗一統之後,雖數頒賚,而驕縱無厭。同光末,自貝州劫趙在禮,據有魏博。及帝纘位,在禮冀脫其禍,潛奏願赴朝覲,遂除皇子從榮為帥,乃令北禦契丹。是行也,不支甲胄,惟幟于長竿表隊伍而已,故俯首遄征。在途聞李嚴為孟知祥所害,以為劍南阻絕,互相煽動。及屯于盧臺,會烏震代房知溫為帥,轉增浮說。震與房知溫博于東寨,日
 亭午,大噪于營外。知溫上馬出門,為甲士所擁,且曰:「不與兒郎為主,更何處去?」知溫紿之曰:「馬軍皆在河西,步卒獨何為也!」遂得躍馬登舟,濟于西岸。安審通戢騎軍不動,知溫與審通謀,伺便攻之,令亂兵卷甲南行。騎軍徐進,部伍嚴整。叛者相顧失色,列炬宵行,疲于荒澤。遲明,潛令外州軍別行,知溫等遂擊亂軍,橫尸于野,餘眾復趨舊寨,至則已焚之矣。翼日,盡戮之,脫于叢草溝塍者十無二三,迨夜竄于山谷,稍奔于定州。及王都之敗,
 乃無噍類矣。癸巳,兗州節度使房知溫加侍中,齊州防禦使安審通加檢校太傅,並賞盧臺之功也。



 丁酉,偽吳楊溥遣移署右威衛將軍雷現貢端午禮幣。辛丑,以前利州節度使張敬詢為雲州節度使。遣樞密使孔循赴荊南城下,時招討使劉訓有疾故也。甲辰,以戶部侍郎韓彥惲為秘書監。是日,幸石敬瑭、安重誨第。丙午,故振武節度使李嗣恩贈太尉,以司封郎中、充樞密院直學士閻至為左諫議大夫充職。右諫議大夫梁文矩上言,
 平蜀已來,軍人剽略到西川人口甚多,骨肉阻隔,恐傷和氣,請許收認。帝仁慈素深,因文矩之奏,詔河南、河北舊因兵火擄隔者,並從識認。是日,郢州進白鵲。



 五月癸丑,以福建留後、檢校太傅、舒州刺史王延鈞為檢校太師、守中書令,充福建節度使、琅琊郡王;以太常卿蕭頃為吏部尚書。是日,懷州進白鵲。戊午,以三司副使、守太子賓客張格卒廢朝。以翰林學士、駕部郎中、知制誥竇夢徵為中書舍人充職。癸亥,遣宣徽使張延朗調發
 郡縣糧運赴荊南城下,仍以軍法從事。以右龍武統軍崔公實為左龍武統軍,以前復州刺史高行周為右龍武統軍。割果州屬郡。乙丑,偽吳楊溥貢新茶。滄州進白鶴。庚午,詔罷荊南之師,既而令軍士散掠居民而回。詔:「文武臣僚及諸道節度使、刺史,有父母在者,各與恩澤。」宰臣任圜表辭三司事,乃以樞密院承旨孟鵠充三司副使權判。



 六月壬午,華州、邢州進兩歧麥,兗州進三足鳥。丙戌,宰相任圜落平章事,守太子少保。丁亥,詔天下
 除併無名額寺院。以宣徽北院使張延朗為右武衛大將軍、判三司,依前宣徽使、檢校司徒。辛卯,大理少卿王鬱上言:「凡決極刑,準敕合三覆奏。近年已來,全隳此法,伏乞今後決前一日許一覆奏。」從之。壬辰,南面招討使、知荊南行府事、襄州節度使、檢校太傅劉訓責授檢校右僕射、守澶州刺史。訓南征無功,故有是譴。詔喪葬之家,送終之禮不得過度。乙未,戶部尚書李鈴上言:「請朝班自四品已上官各許薦令錄兩人,五品官各薦簿
 尉兩人,功過賞罰,與舉者同之。」詔從之。其所舉人,仍于官告內標所舉姓名,或有不公,連坐舉主。仍令三品已上各舉堪任兩使判官者。丙申,以天策上將軍、湖南節度使、開府儀同三司、檢校太師、守尚書令、楚王馬殷為守太師、尚書令,封楚國王。庚子,幸白司馬陂,祭突厥神,從北俗之禮也。



 秋七月庚戌朔,以宋州節度使王晏球充北面行營副招討使。癸丑,以左金吾將軍烏昭遠為左衛上將軍,充入蠻國信使。中書奏:「馬殷封楚國
 王,禮文不載國王之制,請約三公之儀,用竹冊。」從之。壬戌,西川節度副大使、知節度事孟知祥加檢校太尉、兼侍中,東川董璋加爵邑。以左效義指揮使元習為資州刺史,右效義指揮使盧密為雅州刺史。癸亥,幸冷泉宮。甲子,以檢校工部尚書謝洪為宿州團練使。夔州刺史西方鄴奏,殺敗荊南賊軍,收峽內三州。丙寅,升夔州為寧江軍,以鄴為節度使。戊辰,詔曰:「頃因本朝親王,遙領方鎮,遂有副大使知節度事,年代已深,相沿未改。其東川、
 西川今後落副大使,只云節度使。」庚午,遂州留後李敬周、鄜州留後劉仲殷莘正授本州節度使。壬申,兗州節度使房知溫移鎮徐州,徐州節度使安元信移鎮襄州,滄州節度使趙在禮移鎮兗州。以齊州防禦使安審通為滄州節度使。是日,詔陵州、合州長流百姓豆盧革、韋說等,宜令逐處刺史監賜自盡,其骨肉並放逐便。是日,逐段凝于遼州,劉訓于濮州,溫韜于德州。甲戌,太子少保任圜上表乞致仕,仍于外地尋醫,詔從之。丁丑,以左
 金吾大將軍曹廷隱為齊州防禦使。



 八月己卯朔,日有食之。辛巳,以右諫議大夫孔昭序為給事中,以秘書少監崔憓為右諫議大夫。壬午,以右驍衛大將軍劉衡為左領衛上將軍;以鄴都副留守趙敬怡為右衛上將軍王弼集三國魏王弼著。據《隋書·經籍志》載,王弼著,判興唐府事。乙酉,昆明大鬼主羅殿王、普露靜王九部落,各差使隨牂家清州八郡刺史宋朝化等一百五十三人來朝,進方物,各賜官告、繒採、銀器放還蕃。丙戌,以御史中丞盧文紀為工部尚書,以左諫議大夫梁文矩
 為御史中丞。鄧州留後陶貶嵐州司馬,以其為內鄉縣令盛歸仁所訟,稅外科率故也。仍賜歸仁緋袍魚袋。癸巳,幸皇子從榮第,宣禁中伎樂觀宴;從榮進馬及器幣,帝因以伎樂賜之。華州上言,渭河泛濫害稼。丁酉,以吏部郎中、襲文宣公孔邈為左諫議大夫。史館修撰趙熙上言:「應內中公事及詔書奏對,應不到中書者,請委內臣一人抄錄,月終送史館。」詔差樞密直學士錄送。青州進芝草。新州奏,契丹乞置互市。癸卯,汴州節度使朱
 守殷加兼侍中,鄆州節度使符習加檢校太尉。甲辰,皇子從榮娶鄜州節度使劉仲殷女,是夕禮會,百僚表賀。



 九月辛亥,義武軍節度使、檢校太尉、兼中書令王都加食邑實封。幽州節度使趙德鈞加檢校太尉,鎮州節度使王建立加同平章事。偽吳楊溥遣使以應聖節貢獻。己未,以前雲州節度使高行珪為鄧州節度使。是日,出御札曰:「歷代帝王,以時巡狩,一則遵於禮制,一則按察方區。矧彼夷門,控茲東夏,當先帝戡平之始,為眇躬殿
 守之邦,俗尚貞純,兵懷忠勇。自元臣鎮靜,庶事康和,兆民咸樂于有年,闔境彌堅于望幸,事難違眾,議在省方。朕取十月七日親幸汴州。」庚申,以衛尉卿李延光為大理卿。北京留守李彥超上言:「先父存審,本姓符氏,蒙武皇賜姓,乞卻還本姓。」從之。乙丑,夏州節度使李仁福、鳳翔節度使李從嚴、朔方節度使韓洙,並加食邑,改賜功臣。以汝州防禦使趙延壽為河陽節度使,以比部郎中、知制誥劉贊為中書舍人,以河陽掌書記程遜
 為比部員外郎、知制誥,以代州刺史李德珫為蔚州刺史。



 丙寅,樞密使孔循兼東都留守。襄州夏魯奇上言,荊南高季興遣使持書乞修貢奉,詔魯奇不納。詔諸州錄事參軍,不得兼使府賓職。己巳,鄧州節度使史敬鎔加檢校太保,同州節度使盧質加檢校司徒。御史臺奏:「每遇入閣,舊例只一員侍御史在龍墀邊祗候,彈奏公事,或有南班失儀,點檢不及。今欲依常朝例,差殿中侍御史二員,押鐘鼓樓位,仍各綴供奉班出入。」從之。以青州
 節度副使淳于晏為亳州團練使。契丹遣使美棱瑪古已下朝貢。戊寅,西川奏:據黎州狀,雲南使趙和于大渡河南起舍一間,留信物十五籠,并雜箋詩一卷,遞至闕下。



 冬十月己卯朔,帝御文明殿視朝。癸未,亳州刺史李鄴貶郴州司戶,又貶崖州長流百姓,所在賜自盡。判官樂文紀配祁州,責其違法黷貨也。乙酉,駕發西京,詔留宰相崔協以奉祠祭。丁亥,帝宿于滎陽。汴州硃守殷奏,都指揮使馬彥超謀亂,已處斬訖。戊子,次京水,知硃守
 殷反,帝親統禁軍倍程前進。翼日,至汴州,攻其城,拔之,守殷伏誅。丙申,磁州刺史藥縱之上言,今月十二日,供奉官王仁鎬至,稱制殺太子少保致仕任圜。契丹遣使持書求碑石,欲為其父表其葬所。戊戌,詔曰:「諸道州府,自同光三年已前所欠秋夏稅租,並主持務局敗闕課利,並沿河舟船折欠,天成元年殘欠夏稅,並特與除放。?筆敝鞀寮裙範任圜之禍,恐人非之,思沛恩于眾以掩己過,乃奏曰:「三司積欠約二百萬貫,虛繫帳額,請並蠲放。」
 帝重違其意,故有是詔。時議者以蠲隔年之賦,猶或惠民,場院課利一概除之,得不啟奸倖之門乎!



 己亥,詔曰:「太子少保致仕任圜,早推勛舊,曾委重難,既退免于劇權,俾優閒于外地。而乃不遵禮分,潛附守殷,緘題罔避于嫌疑,情旨頗彰于怨望。自收汴壘,備見蹤由,若務含宏,是孤典憲。尚全大體,止罪一身,已令本州私第自盡,其骨肉親情僕使等並皆放罪。」辛丑,詔曰:「后來其蘇,動必從于人欲;天監厥德,靜宜布于國恩。近者言幸浚郊,
 暫離洛邑,蓋逢歲稔,共樂時康。不謂奸臣,遽彰逆狀,為厲之階既甚,覆宗之禍自貽。俾我生靈,遘茲紛擾,永言軫惻,無輟寐興。宜覃雨露之恩,式表雲雷之澤,應汴州城內百姓,既經驚動,宜放二年屋稅;諸處有曾受逆人文字者,隨處焚毀。應天下見禁囚徒,除十惡五逆、殺人放火、劫盜、合造毒藥、官典犯贓、偽行印信、屠牛外,罪無輕重,並從釋放。應有民年八十已上及家長者有廢疾者,免一丁差役」云。以山南西道節度使張筠為西京留
 守,行京兆尹。青州節度使霍彥威差人走馬進箭一對,賀誅朱守殷,帝卻賜彥威箭一對。傳箭,番家之符信也,起軍令眾則使之,彥威本非蕃將,以臣傳箭于君,非禮也。癸卯,以權知汴州事、陜州節度使石敬瑭為汴州節度使、兼六軍諸衛副使、侍衛親軍馬步都指揮使。鳳翔奏,地震。丙午,威武軍節度副使、檢校太尉、守建州刺史王延亶可同平章事、守建州刺史,充奉國軍節度副使、兼威武軍節度副使。詔割施州卻屬黔南。



 十一月己酉,
 帝祭蕃神于郊外。庚戌,以皇城使、行袁州刺史李從敏為陜州節度使。乙卯,青州霍彥威、鄆州符習來朝。以太子詹事溫輦為吏部侍郎。徐州房知溫為朝。戊午,黔南節度使李紹義加檢校太保。庚申毀滅。著作僅存一個斷片。,皇子河中節度使、檢校太保、同平章事從珂,鄴都留守、檢校太保、同平章事從榮,河南尹、判六軍諸衛事、檢校太保、同平章事從厚,並加檢校太傅,進爵邑。貝州刺史竇廷琬上言:請制置慶州青白兩池,逐年出絹十萬匹,米萬石。詔升慶州為
 防禦所,以廷琬為使。壬申,詔霍彥威等歸籓。詔太宗朝左僕射李靖可冊贈太保,鄭州僕射陂可改為太保陂。時議者以僕射陂者,後魏孝文帝賜僕射李沖,故因以為名,及是命之降以為李靖,蓋誤也。契丹遣使摩琳等率其屬來乞通和。



 十二月戊寅朔,以前鳳翔留後高允貞為右監門上將軍。詔以施州為夔州屬郡,以其便近故也。遣飛勝指揮使于契丹,賜契丹主錦綺、銀器等,兼賜其母繡被纓珞。己卯,蔚州刺史周令武得代歸闕,帝問北州
 事,令武奏曰:「山北甚安,諸蕃不相侵擾。鴈門已北,東西數千里,斗粟不過十錢。」帝悅,顧謂左右曰:「須行善事,以副天道。」居數日,帝延宰臣于元德殿,言及民事,馮道奏曰:「莊宗末年,不撫軍民,惑于聲樂,遂致人怨國亂。陛下自膺人望,歲時豐稔,亦淳化所致也。更顧居安思危。」帝然之。許州地震。庚辰,皇子鄴都留守從榮移鎮太原。以北京留守符彥超為潞州節度使。乙酉,以彰國軍節度使李從璋昧于政理,詔歸闕。敕新及第進士有聞喜宴,
 逐年賜錢四十萬。己丑,兗州節度使趙在禮來朝。詔出潛龍宅米以賑百官。壬辰,以太傅致仕齊國公趙光逢卒輟朝。丙申,許州節度使夏魯奇移鎮遂州。庚子,幸石敬瑭公署及康義誠私第。甲辰,狩于東郊,臘也。丙午,追尊四廟,以應州舊宅為廟。



\end{pinyinscope}