\article{太祖紀五}

\begin{pinyinscope}

 開平三年九月癸巳朔,御崇勛殿,宴群臣文武百官。賜張宗奭、楊師厚白綾各三百匹,銀鞍轡馬。丁酉,上幸崇政院宴內臣,賜院使敬翔、直學士李班等繒彩有差。以門下侍郎、平章事薛貽矩判建昌宮事,兼延資庫使。制:「內外使臣復命,未見便歸私第者,朝廷命使,臣下奉行,惟於辭見之儀,合守敬恭之道。近者凡差出使,往復皆越常規。或已辭而尚在本家,或未見而先歸私第,但從己便,莫稟王程。在禮敬而殊乖,置典章而私舉。宜令御史臺別具條流事件具黜罰等奏聞。」庚子,殿直王唐福自襄城走馬,以天軍勝捷逆將李洪歸降事上聞。賜唐福絹銀有加,宰臣百官上表稱賀。壬寅,開封府虞候李繼業齎襄州都指揮使程暉奏狀,以今月
 五日,殺戮逆黨千人,並生擒都指揮使傅霸以下節級共五百人,收復襄
 州人戶歸業事。《通鑒》:八月,陳暉軍至襄州,李洪逆戰,大敗,王求死。九月丁酉,拔其城,斬叛兵
 千人,執李洪、楊虔等送洛陽,斬之。癸卯,帝御文明殿,以收復襄漢,受宰臣以下稱賀。辛亥,侍中韓建罷守太保,左僕射、
 同平章事楊涉罷守本官,以太常卿趙光逢為中書侍郎、平章事,翰林學士奉旨、工部侍郎、知制誥杜曉為尚書戶部侍郎、平章事。詔曰:「秋冬之際,陰雨相仍,所司擇日拜郊,或
 慮臨時妨事,宜令別更擇日奏聞。」是月,禮儀使奏:「據所司擇,十月二日,祀圓丘。今參詳十月十七日以後入十一月節,十一月二日冬至一陽生之辰,宜行親告之禮。」從之。河中奏,準宣,詔使有銅牌者,所至即易騎以遣。



 十月癸未,大明節,帝御文明殿,設齋僧道,召宰臣、翰林學士預之,諸道節度、刺史及內外諸司使咸有進獻。詔以寇盜未平,凡諸給過所,並令司門郎中、員外郎出給,以杜姦詐。



 十一月癸已朔,帝齋於內殿,不視朝。甲午,日長至,
 五更一點自大內出,於文明殿受宰臣以下起居,自五鳳樓出南郊,左右金吾、太常、兵部等司儀仗法駕鹵簿及左右內直控鶴等引從赴壇,文武百官太保韓建以下班以候,帝升壇告謝。司天臺奏:冬至日,自夜半後,祥風微扇,帝座澄明,至曉,黃雲捧日。丙申,畋于上東門外。戊戌,制曰:



 夫嚴親報本,所以通神明;流澤覃休,所以惠黎庶。斯蓋邦家不易之道,皇王自昔之規,敢斁大猷,茲惟古義。粵朕受命,於今三年,何曾不寅畏晨興,焦勞夕
 惕。師唐、虞之典,上則于乾功,挹殷、夏之源,下涵于民極。欲使萬方有裕,六辨無愆。然而志有所未孚,理有所未達,致奸宄作釁,旱霪為災。驕將守邊,擁牙旗而背義;積陰馭氣,陵玉燭以干和。載考休徵,式昭至警。朕是以仰高俯厚,靡惜于責躬;履薄臨淵,冀昭于元覽。兢兢慄慄,夙夜匪寧。及夫動干戈而必契靈誅,陳犧齋而克章善應,茍非天垂丕佑,神贊殊休,則安可致夷兇渠,就不戰之功,變沴戾氣,作有年之慶。況靈旗北指,喪犬羊於亂
 轍之間,飛騎西臨,下鄜、翟若走丸之易。息一隅之煙燧,復千里之封疆。而又掃蕩左馮,討除峴首。故得外戎內夏,益知天命之攸歸,喙息蚑行,共識皇基之永固。仰懷昭應,欲報無階。爰因南至之辰,親展圓丘之禮。茲惟大慶,必及下民,乃宏渙汗之私,以錫疲羸之幸。所冀漸增蘇息,亟致和平。噫!朕自臨御以來,歲時尚邇,氛昏未殄,討伐猶頻。甲兵須議於饋糧,飛挽頻勞於編戶,事非獲已,慮若納隍。宜所在長吏,倍切撫綏,明加勉諭,每官中
 抽差徭役,禁猾吏廣斂貪求。免至流散靡依,凋弊不濟。宜令河南府、開封府及諸道觀察使切加鈴轄,刺史、縣令不得因緣賦斂,分外擾人。凡關庶獄,每望輕刑。只候纔罷用軍,必當便議優給。德音節文內有未該者,宜令所司類列條件奏聞。



 己亥,以羅周韓為天雄軍節度副使,知府事,從鄴王紹威病請也。辛丑,幸穀水。戊午,御文明殿,冊太傅張宗奭太保韓建受冊畢。金吾仗引昇輅車,儀仗導謁太廟訖,赴尚書省上。幸榆林坡閱兵,教諸都
 馬步兵。敕改乾文院為文思院,行殿為興安殿,球場為興安球場,又改弓箭庫殿為宣武殿。靈州奏,鳳翔賊將劉知俊率邠、岐、秦、涇之師侵迫州城。帝遣陜州康懷英、華州寇彥卿率兵攻迫邠、寧,以緩朔方之寇。《五代春秋》:十一月,秦人來侵靈州。陜州康懷英侵秦,克寧、慶、衍三州。秦人來襲,懷英兵敗於升平。



 十二月乙丑臘,較獵於甘泉驛。以蒲州肇跡之地,且因經略鄜、延,於是巡幸數月。暇日游豫至焦梨店,頗述前事,念王重榮舊功,下詔褒獎而封崇之。國子監奏:「創造文宣王廟,仍請
 率在朝及天下現任官僚俸錢,每貫每月剋一十五文,充土木之植。」允之。是歲,以所率官僚俸錢修文宣王廟。福建節度使王審知奏,舍錢造寺一所,請賜寺額。敕名大梁萬歲之寺,仍許度僧四十九人。贈牢墻使王仁嗣司空,故同州押衙史肇右僕射,押衙王彥洪、高漢詮、丘奉言、仇瓊並刑部尚書,王筠御史司憲。初,知俊將叛,謀會諸將詢所宜,仁嗣等持正不撓,悉罹其酷,至是褒贈之。劉守光上言,於薊州西與兄守文戰,擒守文。



 開平四年正月壬辰朔,帝御朝元殿,受百官稱賀,始用禮樂也。敕:「公事難于稽遲,居處悉皆遙遠。其逐日當直中書舍人及吏部司封知印郎官、少府監及篆印文兼書寫告身人吏等,並宜輪次于中書側近宿止。」乙未,帝出師子門,至榆林坡下閱教。壬寅,幸保寧球場,錫宴宰臣及文武百官。賜宰臣張宗奭已下分物有加,賜廣王分物。及湖南開元寺禪長老可復號惠光大師,仍賜紫衣。



 二月乙丑,幸甘水亭。出師子門,幸榆林東北坡,教諸軍
 兵事。賜潞州投歸軍使張行恭錦服銀帶並食。戊辰,宴於金鑾殿。甲戌,以春時無事,頻命宰臣勛戚宴于河南府池亭。辛巳,楊師厚赴鎮于陜。寒食假,諸道節度使、郡守、勳臣競以春服賀。又連清明宴,以鞍轡馬及金銀器、羅錦進者迨千萬,乃御宣威殿,宴宰臣及文武官四品已上。己丑,出光政門,至穀水觀麥。



 三月壬辰,幸崇政院宴勳臣。己亥,幸天驥院宴侍臣。壬寅,幸甘水亭宴宰臣、勳戚、翰林學士。辛亥,宴宰臣于內殿。丙辰,于興安球
 場大饗六軍,樂春時也。



 四月壬戌,詔曰:「追養以祿,王者推歸厚之恩;欲靜而風,人子抱終身之感。其以刑部尚書致仕張策及三品、四品常參官二十二人先世,各追贈一等。」乙丑,宴崇政院。帝在籓及踐阼,勵精求理,深戒逸樂,未嘗命堂上歌舞。是日,止令內妓升階,擊鼓弄曲甚懽,至午而罷。丁卯,宋州節度使、衡王友諒進瑞麥,一莖三穗。《通鑒》:友諒獻瑞麥,帝曰:「豐年為上瑞,今宋州大水,安用此為!」詔除本縣令名,遣使詰責友諒。丙戌,幸建春門閱新樓,至七里屯觀麥,召從官食於樓。
 河南張昌孫及蒲、同主事吏賜物各有差。帝過朝邑,見鎮將位在縣令上,問左右,或對曰:「宿官秩高。」帝曰:「令長字人也,鎮使捕盜耳。且鎮將多是邑民,奈何得居民父母上,是無禮也。」至是,敕天下鎮使,官秩無高卑,位在邑令下。葉縣鎮遏使馮德武於蔡州西平縣界殺戮山賊,擒首領張濆等七人以獻。鎮海軍節度使錢鏐擊高灃於湖州,大敗之,梟夷擒殺萬人,拔其郡,湖州平。先是,灃以州叛入淮南,故詔鏐討之也。



 五月己丑朔,以連雨不
 止,至壬辰,御文明殿,命宰臣分拜祠廟。自朔旦至癸巳,內外以午日奉獻巨萬,計馬三千蹄,餘稱是,復相率助修內壘。甲辰,詔曰:「奇邪亂正,假偽奪真,既刑典之不容,宜違犯而勿赦。應東、西兩京及諸道州府,創造假犀玉真珠腰帶、璧珥並諸色售用等,一切禁斷,不得更造作。如公私人家先已有者,所在送納長吏,對面毀棄;如行敕後有人故違,必當極法。仍委所在州府差人檢察收捕,明行處斷。」魏博節度使、守太師、兼中書令、鄴王羅
 紹威薨,帝哀慟曰:「天不使我一海內,何奪忠臣之速也!」詔贈尚書令。六月己未朔,詔軍鎮勿起土功。



 七月壬子,宴宰臣、河南尹、翰林學士、兩街使於甘水亭。丙辰,宴群臣於宣威殿,賜物有差。劉知俊攻逼夏州。《通鑒》:七月,岐王與邠涇二帥各遣使告晉,請合兵攻定難節度使李仁福,晉王遣振武節度使周德威將兵會之,合五萬眾,圍夏州。以宣化軍留後李思安為東北面行營都指揮使,陜州節度使楊師厚為西路行營招討使。福州貢方物,獻桐皮扇,廣州貢犀玉,獻舶上薔薇水。時陳、許、汝、蔡、潁五州境內
 有蝝為災,俄而許州上言,有野禽群飛蔽空,旬日之間,食蝝皆盡,是歲乃大有秋。



 八月,車駕西征。己巳,次陜府。是時憫雨,且命宰臣從官分禱靈跡,日中而雨,翼日止,帝大悅。辛未,老人星見。是日,宴本府節度使楊師厚及扈從官於行宮,賜師厚帛千匹,仍授西路行營招討使。丙子,宴文武從官軍使已下,設龜茲樂,賜物有差。



 九月丁亥朔,命宰臣于兢赴西都,祀昊天上帝於圓丘。甲午,至西京。下詔曰:



 朕聞歷代帝王,首推堯、舜;為人父母,
 孰比禹、湯。睿謀高出於古先,聖德普聞於天下,尚或卑躬待士,屈己求賢。俯仰星雲,慮一民之遺逸;網羅巖穴,恐片善之韜藏。延爵祿以徵求,設丹青而訪召,使其為政,樂在進賢,蓋由國有萬幾,朝稱百揆,非才不治,得士則昌。自朕光宅中區,迄今三載,宵分輟寐,日旰忘餐,思共力於廟謀,庶永清於王道。而乃朝廷之內,或未盡於昌言;軍旅之間,亦罕聞於奇策。眷言方岳,下及山林,豈無英奇,副我延佇。諸道都督、觀察防禦使等,或勳高翊
 世,或才號知人,必於途巷之賢,備察芻蕘之士。詔到,可精搜郡邑,博訪賢良,喻之以千載一時,約之以高官美秩,諒無求備,惟在得人。如有卓犖不羈,沉潛自負,通霸王之上略,達文武之大綱,究古今刑政之源,識禮樂質文之變,朕則待之不次,委以非常,用佐經綸,豈勞階級。如或一言拔俗,一事出群,亦當舍短從長,隨才授任。大小方圓之器,寧限九流,溫良恭儉之人,難誣十室。勉思薦舉,勿至因循,俟爾發揚,慰予翹渴。仍從別敕處分。



 辛
 丑,以久雨,命宰臣薛貽矩抃定鼎門,趙光逢祠嵩岳。敕:「魏博管內刺史,比來州務,並委督郵。遂使曹官擅其威權,州牧同於閑冗,俾循通制,宜塞異端。並依河南諸州例,刺史得以專達。」壬寅,頒奪馬令。先是,王師擊賊,獲馬多上獻,至是盡止之,蓋欲邀其奮擊之功也。乙巳,王師敗蕃寇於夏州。初,劉知俊誘沙陀振武賊帥周德威、涇原賊帥李繼鸞合步騎五萬大舉,欲俯拾夏臺,節度使李仁福兵力俱乏,以急來告。先是,供奉官張漢玫宣諭
 在壁,國禮使杜廷隱賜幣於夏,及石堡寨,聞賊至,以防卒三百人馳入州。既而大兵圍合,廷隱、漢玫與指揮使張初、李君用率州民防卒,與仁福部分固守,晝夜戮力踰月。及鄜、延援至,大軍奮擊,敗之。河東、邠、岐賊分路逃遁,夏州圍解。《通鑒》:甲申,遣夾馬指揮使李遇、劉綰自鄜、延趨銀、夏。李遇等至夏州,岐、晉兵皆解去。丙午,詔曰:「劉知俊貴為方伯,尊極郡王,而乃背誕朝恩,竄投賊壘,固神人之共怒,諒天地所不容。雖命討除,尚稽擒戮,宜懸爵賞,以大功名,必有忠貞,咸思憤發。有生
 擒劉知俊者,賞錢千萬,授節度使,首級次之;得孟審登者,錢百萬,除刺史;得孫亢、卓環、劉儒、張鄰等,賞有差。」乙卯,宴會群臣於宣威殿。



\end{pinyinscope}