\article{列傳二十二}

\begin{pinyinscope}

 元行欽,本幽州劉守
 光之愛將。守光之奪父位也,令行欽攻大恩山,又令殺諸兄弟。天祐九年,周德威攻圍山州我、齊是非,齊生死,齊貴賤,達到「天地與我並生,萬物,守光困蹙,令行欽于山北募兵,以應契丹。時明宗為
 將,攻行欽于山北,與之接戰,矢及明宗馬鞍,既而以勢迫來降。明宗憐其有勇,奏隸為假子,後因從征討,恩禮特隆。常臨敵擒生,必有所獲,名聞軍中。莊宗東定趙、魏,選驍健置之麾下,因索行欽,莊宗不得已而遣之。時有散指揮都頭,名為散員,命行欽為都部署,賜姓,名紹榮。莊宗好戰,勇于大敵,或臨陣有急兵,行欽必橫身解鬥翼衛之。莊宗營於德勝也,與汴軍戰于潘張,王師不利,諸軍奔亂。莊宗得三四騎而旋,中野為汴軍數百騎
 攢槊攻之,事將不測,行欽識其幟,急馳一騎,奮劍斷二矛,斬一級,汴軍乃解圍,翼莊宗還宮。莊宗因流涕言曰:「富貴與卿共之。」自是寵冠諸將,官至檢校太傅、忻州刺史。及莊宗平梁,授武寧軍節度使。嘗因內宴群臣,使相預會,行欽官為保傅,當地褥下坐。酒酣樂作,莊宗敘生平戰陣之事,因左右顧視,曰:「紹榮安在?」所司奏云:「有敕,使相預會,紹榮散官,殿上無位。」莊宗徹會不懌。翌日,以行欽為同平章事,由是不宴百官于內殿,但宴武臣而
 已,



 三年,行欽喪婦。莊宗有所愛宮人生皇子者,劉皇后心忌之,會行欽入侍,莊宗勞之曰:「紹榮喪婦復娶耶!吾給爾婚財。」皇后指所忌宮人謂莊宗曰:「皇帝憐紹榮,可使為婦。」莊宗難違所請,微許之。皇后即命紹榮謝之,未退,肩輿已出。莊宗心不懌,佯不豫者累日,業已遣去,無如之何。及貝州軍亂,趙在禮入魏州,莊宗方擇將,皇后曰:「小事不勞大將,促紹榮指揮可也。」乃以行欽為鄴都行營招撫使,領騎二千進討。洎至鄴城,攻之不能下,退
 保于澶州。未幾,諸道之師稍集,復進軍於鄴城之南。及明宗為帥,領軍至鄴,行欽來謁於軍中,拜起之際,誤呼萬歲者再,明宗驚駭,遏之方止。既而明宗營於城西,行欽營於城南。三月八日夜,明宗為亂軍所迫,惟行欽之軍不動,按甲以自固。明宗密令張虔釗至行欽營,戒之曰:「且堅壁勿動,計會同殺亂軍,莫錯疑誤。」行欽不聽,將步騎萬人棄甲而退。自知失策,且保衛州,因誣奏明宗曰:「鎮師已入賊軍,終不為國使。」明宗既劫出鄴城,令人
 走馬上章,申理其事,言:「臣且于近郡聽進止。」莊宗覽奏釋然曰:「吾知紹榮妄矣。」因令白從訓與明宗子繼璟至軍前,欲令見明宗,行欽縶繼璟于路。明宗凡奏軍機,拘留不達,故旬日之間,音驛斷絕。及莊宗出成皋,知明宗在黎陽,復令繼璟渡河召明宗,行欽即殺之,仍勸班師。四月一日,莊宗既崩,行欽引皇后、存渥,得七百騎出師子門,將之河中就存霸,沿路部下解散,從者數騎而已。四日,至平陸縣界,為百姓所擒,縣令裴進折其足,檻車
 以獻。明宗即位,詔削奪行欽在位官爵,斬于洛陽。



 夏魯奇,字邦傑,青州人也。初事宣武軍為軍校,與主將不協,遂歸于莊宗,以為護衛指揮使。從周德威攻幽州,燕將有單廷珪、元行欽,時稱驍勇,魯奇與之斗,兩不能解,將十皆釋兵縱觀。幽州平,魯奇功居多。梁將劉鄩在洹水,莊宗深入致師,鄩設伏于魏縣西南葭蘆中。莊宗不滿千騎,汴人伏兵萬餘,大噪而起,圍莊宗數重。魯奇與王門關、烏德兒等奮命決戰,自午至申,俄而李存審
 兵至方解。魯奇持槍攜劍,獨衛莊宗,手殺百餘人。烏德兒等被擒,魯奇傷痍遍體,自是莊宗尤憐之,歷磁州刺史。中都之戰,汴人大敗,魯奇見王彥章,識之,單馬追及,槍擬其頸;彥章顧曰:「爾非餘故人乎?」即擒之以獻。莊宗壯之,賞絹千匹。《九國志·趙庭隱傳》:王彥章守中都,庭隱在其軍中。及彥章敗,庭隱為莊宗所獲,將以就戮,大將夏魯奇奏曰:「此矬也,其材可用。」遂釋之。梁平,授鄭州防禦使。四年,授河陽節度使。天成初,移鎮許州,加同平章事。



 魯奇性忠義,尤通吏道,撫民有術。及移鎮許田,孟州之民,萬眾
 遮道,斷登臥轍,五日不發。父老詣闕請留,明宗令中使諭之,方得離州。明宗討荊南,魯奇為副招討使,頃之,移鎮遂州。《九國志·李仁罕傳》:夏魯奇稟朝廷之命,繕治甲兵,將圖蜀,孟知祥與董璋謀先取魯奇,令仁罕攻遂州。董璋之叛,與孟知祥攻遂州,援路斷絕,兵盡食窮,《九國志·李肇傳》:蜀師圍夏魯奇于遂州,唐師來援,劍門不守,肇領兵赴普安以拒之,唐師不得進。魯奇自刎而卒,時年四十九。帝聞其死也,慟哭之,厚給其家,贈太師、齊國公。



 姚洪,本梁之小校也。在梁時,經事董璋,長興初,率兵千
 人戍閬州。璋叛,領眾攻閬州,璋密令人誘洪,洪以大義拒之。及璋攻城,洪悉力拒守者三日,禦備既竭,城陷被擒。璋謂洪曰:「爾頃為健兒,由吾獎拔至此;吾書誘諭,投之于側,何相負耶?」洪大罵曰:「老賊,爾為天子鎮帥,何苦反耶!爾既辜恩背主,吾與爾何恩,而云相負!爾為李七郎奴,掃馬糞,得一臠殘炙,感恩無盡。今明天子付與茅土,貴為諸侯,而驅徒結黨,圖為反噬。爾本奴才,則無恥;吾忠義之士,不忍為也。吾可為天子死,不能與人奴茍
 生!」璋怒,令軍士十人,持刀刲割其膚,燃鑊于前,自取啖食,洪至死大罵不已。明宗聞之泣下,置洪二子于近衛,給賜甚厚。



 李嚴,幽州人,本名讓坤。初仕燕,為刺史,涉獵書傳,便弓馬,有口辯,多游藝,以功名自許。同光中,為客省使。奉使于蜀,及與王衍相見,陳使者之禮,因于笏記中具述莊宗興復之功,其警句云:「纔過汶水,縛王彥章于馬前;旋及夷門,斬硃友貞于樓上。」嚴復聲韻清亮,蜀人聽之愕
 然。時蜀偽樞密使宋光嗣召嚴曲宴,因以近事訊于嚴。嚴對曰:「吾皇前年四月即位于鄴宮,當月下鄆州。十月四日,親統萬騎破賊中都,乘勝鼓行,遂誅汴孽,偽梁尚有兵三十萬,謀臣猛將,解甲倒戈。西盡甘、涼,東漸海外,南踰閩、浙,北極幽陵。牧伯侯王,稱籓不暇,家財入貢,府實上供。吳國本朝舊臣,岐下先皇元老,遣子入侍,述職稱籓。淮、海之君,卑辭厚貢,湖湘、荊楚,杭越、甌閩,異貨奇珍,府無虛月。吾皇以德懷來,以威款附。順則涵之以恩
 澤,逆則問之以干戈,四海車書,大同非晚。」光嗣曰:「余所未知,惟岐下宋公,我之姻好,洞見其心,反覆多端,專謀跋扈,大不足信也。似聞契丹部族,近日稍強,大國可無慮乎?」嚴曰:「子言契丹之強盛,孰若偽梁?」曰:「比梁差劣也。」嚴曰:「吾國視契丹如蚤虱耳,以其無害,不足爬搔。吾良將勁兵布天下,彼不勞一郡之兵,一校之眾,則懸首槀街,盡為奴擄。但以天生四夷,當置度外,不在九州之本,未欲窮兵黷武也。」光嗣聞辯對,畏而奇之。時王衍失政,
 嚴知其可取,使還具奏,故平蜀之謀,始于嚴。



 郭崇韜起軍之日,以嚴為三川招撫使,嚴與先鋒使康延孝將兵五千,先驅閣道,或馳以詞說,或威以兵鋒,大軍未及,所在降下。延孝在漢州,王衍與書曰:「可請李司空先來,餘即舉城納款。」眾咸以討蜀之謀始於嚴,衍以甘言,將誘而殺之,欲不令往。嚴聞之喜,即馳騎入益州,衍見嚴于母前,以母、妻為托。即日,引蜀使歐陽彬迎謁魏王繼岌。蜀平班師,會明宗即位,遷泗州防禦使兼客省使。長興
 初,安重誨謀欲控制兩川,嚴乃求為西川兵馬都監,庶效方略。孟知祥覺之,既至,執而害之。《九國志·王彥銖傳》:李嚴之為監軍也,密懷異謀,知祥數其過,命彥銖擒斬之,嚴之左右無敢動者。贈太保。



 嚴之母,賢明婦人。初,嚴將赴蜀,母曰:「汝前啟破蜀之謀,今又入蜀,將死報蜀人矣!與汝永訣。」既而果如其言。



 李仁矩,本明宗在籓鎮時客將也。明宗即位,錄其趨走之勞,擢居內職,復為安重誨所庇,故數年之間,遷為客省使、左衛大將軍。天成中,因奉使東川,董璋張筵以召
 之,仁矩貪于館舍,與倡妓酣飲,日既中而不至,大為璋所詬辱,自是深銜之。長興初,璋既跋扈于東川,重誨奏以仁矩為閬州節度使,俾伺璋之反狀,時物議以為不可。及仁矩至鎮,偵璋所為,曲形奏報,地里遐僻,朝廷莫知事實,激成璋之逆節,由仁矩也。長興元年冬十月,璋自率兇黨,以攻其城。仁矩召軍校謀守戰利害,皆曰:「璋久圖反計,以賂誘士心,凶氣方盛,未可與戰,宜堅壁以守之。儻旬浹之間,大軍東至,即賊必退。」仁矩曰:「蜀兵懦,安
 能當我精甲!」即驅之出戰,兵未交,為賊所敗。既而城陷,仁矩被擒,舉族為璋所害。



 康思立,晉陽人也。少善騎射,事武皇為爪牙,署河東親騎軍使。莊宗嗣位,從解圍于上黨,敗梁人于柏鄉,及平薊兵,後戰于河上,皆有功,累承制加檢校戶部尚書,右突騎指揮使。莊宗即位,繼改軍帥,賜忠勇拱衛功臣,加檢校尚書右僕射。天成元年,授應州刺史,尋移嵐州,充北面諸蕃部族都監。三年,遷宿州團練使。四年,領昭武
 軍節度、利巴集等州觀察處置等使。改賜耀忠保節功臣。長興初,朝廷舉兵討東川董璋,詔監西面行營軍馬都指揮使。二年,移鎮陜州。《通鑑》:潞王至靈寶,思立謀固守陜城以俟康義誠。先是,捧聖五百騎戍陜,為潞王前鋒,至城下,呼城上人曰:「禁軍十萬已奉新帝,爾輩數人奚為!徒累一城人塗地耳!」于是捧聖卒爭出迎,思立不能禁,不得已,亦出迎。清泰初,改授邢臺,累官至檢校太傅,封會稽郡開國侯。二年,入為右神武統軍。三年,充北面行營馬軍都指揮使。是歲閏十一月,卒于軍,年六十三。



 思立本出陰山諸部,性純厚,善撫將士,明宗素重
 之,故即位之始,以應州所生之地授焉。其後歷三郡三鎮,皆得百姓之譽。末帝以其年高,徵居環衛。及出幸懷州,以北師不利,乃命思立統駕下騎軍赴團柏谷以益軍勢,俄而楊光遠以大軍降于太原,思立因憤激,疾作而卒焉。晉高祖即位,追其宿舊,為輟朝一日,贈太子少師。



 張敬達,字志通,代州人,小字生鐵。父審,素有勇,事武皇為列校,歷直軍使,同光初,卒于軍。敬達少以騎射著名,莊宗知之,召令繼父職;平河南有功,繼加檢校工部
 尚書。明宗即位,歷捧聖指揮使、檢校尚書左僕射。長興中,改河東馬步軍都指揮使,超授檢校司徒,領欽州刺史。三年,加檢校太保、應州節度使。四年,遷雲州。時以契丹率族帳自黑榆林至,云借漢界水草,敬達每聚兵塞下,以遏其衝。契丹竟不敢南牧,邊人賴之。清泰中,自彭門移鎮平陽,加檢校太傅,從石敬瑭為北面兵馬副總管,仍屯兵鴈門。未幾,晉高祖建義,末帝詔以敬達為北面行營都招討使,仍使悉引部下兵圍太原,以定州節
 度使楊光遠副焉。尋統兵三萬,營于晉安鄉。末帝自六月繼有詔促令攻取,敬達設長城連柵、雲梯飛砲,使工者運其巧思,窮土木之力。時督事者每有所構,則暴風大雨,平地水深數尺,而城柵崩墮,竟不能合其圍。九月,契丹至,敬達大敗,尋為所圍。晉高祖及蕃眾自晉安寨南門外,長百餘里,闊五十里,布以氈帳,用毛索掛鈴,而部伍多犬,以備警急。營中嘗有夜遁者,出則犬吠鈴動,跬步不能行焉。自是敬達與麾下部曲五萬人,馬萬
 匹,無由四奔,但見穹廬如崗阜相屬,諸軍相顧失色。始則削木篩糞,以飼其馬,日望朝廷救軍,及漸羸死,則與將士分食之,馬盡食殫。副將楊光遠、次將安審琦知不濟,勸敬達宜早降以求自安。敬達曰:「吾受恩于明宗,位歷方鎮,主上授我大柄,而失律如此,已有愧于心也。今救軍在近,旦暮雪恥有期,諸公何相迫耶!待勢窮,則請殺吾,攜首以降,亦未為晚。」光遠、審琦知敬達意未決,恐坐成魚肉,遂斬敬達以降。《契丹國志》:楊光遠謀害張敬達,諸將高行周陰為之備,敬
 達疏於防禦,推遠行周等。清晨,光遠上謁,見敬達左右無人,遂殺之。



 末帝聞其歿也,愴慟久之。契丹主告其部曲及漢之降者曰:「為臣當如此人!」令部人收葬之。晉高祖即位後,所有田宅,咸賜其妻子焉。時議者以敬達嘗事數帝,亟立軍功,及領籓郡,不聞其濫,繼屯守塞垣,復能撫下,而臨難固執,不求茍免,乃近代之忠臣也。晉有天下,不能追懋官封,賞其事跡,非激忠之道也。



\end{pinyinscope}