\article{末帝紀上}

\begin{pinyinscope}

 末帝,諱從珂,本姓王氏,鎮州人也。母宣憲皇後魏氏,以光啟元年歲在乙巳正月二十三日,生帝於
 平山。景福中,明宗為武皇騎將,略地至平山,遇魏氏,擄之,帝時年十餘歲,明宗養為己子。小字二十三。帝幼謹重寡言,及壯,長七尺餘,方頤大體,材貌雄偉,以驍果稱,明宗甚愛之。在太原,嘗與石敬瑭因擊球
 同
 入于趙襄子之廟,見其塑像,屹然起立,帝秘之,私心自負。及從明宗征討,以力戰知名,莊宗嘗曰:「阿三不惟與我同齒,敢戰亦相類。」莊宗與梁軍戰于胡柳陂,兩軍俱撓,帝衛莊宗奪土山,摧驍陣,其軍復振。時明宗先渡河,莊宗不悅,謂明宗曰:「公當為吾死,渡河安往?」明宗待罪,莊宗以帝從戰有功,
 由是解慍。



 天祐十八年,莊宗營于河上,議討鎮州。留守符存審在德勝寨未行,梁人謂莊宗已北,乃悉眾攻德勝,莊宗命明宗、存審為兩翼以抗之,自以中軍前進。梁軍退卻,帝以十數騎雜梁軍而退,至壘門大呼,斬首數級,斧其望櫓而還。莊宗大噱曰:「壯哉,阿三!」賜酒一器。



 同光元年四月,從明宗襲破鄆州。九月,莊宗敗梁將王彥章于中都,急趨汴州。明宗將前軍,帝率勁騎以從,晝夜兼行,率先下汴城。莊宗勞明宗曰:「復唐社稷,卿父子之
 功也。」



 二年,以帝為衛州刺史。時有王安節者,昭宗朝相杜讓能之宅吏也。安節少善賈,得相術于奇士,因事見帝于私邸,退謂人曰:「真北方天王相也,位當為天子,終則我莫知也。」



 三年,明宗奉詔北禦契丹,以家在太原,表帝為北京內衙指揮使;莊宗不悅,以帝為突騎都指揮使,遣戍石門。



 四年,魏州軍亂,明宗赴洛,時帝在橫水,率部下軍士由曲陽、孟縣趨常山,與王建立會,倍道兼行,渡河而南,由是明宗軍聲大振。



 天成初,以帝為河中節
 度使。明年二月,加檢校太保、同平章事。十一月,加檢校太傅。長興元年,加檢校太尉。先是,帝興樞密使安重誨在常山,因杯盤失意,帝以拳擊重誨腦,中其櫛,走而獲免。帝雖悔謝,然重誨終銜之。及帝鎮河中,重誨知其出入不時,因矯宣中旨,令牙將楊彥溫遇出郭則閉門勿納。是歲四月五日,帝閱馬于黃龍莊,彥溫閉城拒帝,帝聞難遽還,遣問其故,彥溫曰:「但請相公入朝,此城不可入也。」帝止虞鄉以聞,明宗詔帝歸闕。遣藥彥稠將兵討
 彥溫,令生致之,面要鞫問。十一月收城,彥溫已死,明宗以彥稠不能生致彥溫,甚怒之。後數日,安重誨以帝失守,諷宰相論奏行法,明宗不悅。重誨又自論奏,明宗曰:「朕為小將校時,家徒衣食不足,賴此兒荷石灰、收馬糞存養,以至今貴為天子,而不能庇一兒!卿欲行朝典,朕未曉其意,卿等可速退,從他私第閑坐。」遂詔歸清化里第,不預朝請。帝尚懼重誨多方危陷,但日諷佛書陰禱而已。



 二年,安重誨得罪,帝即授左衛大將軍。未幾,復檢
 校太傅、同平章事、行京兆尹,充西京留守。三年,進位太尉,移鳳翔節度使。四年五月,封潞王。



 閔帝即位,加兼侍中。既而帝子重吉出刺亳州,女尼入宮,帝方憂不測。應順元年二月,移帝鎮太原,是時不降制書,唯以宣授而已。帝聞之,召賓佐將吏以謀之,皆曰:「主上年幼,未親庶事,軍國大政悉委朱宏昭等,王必無保全之理。」判官馬裔孫曰:「君命召,不俟駕行焉。諸君凶言,非令圖也。」是夜,帝令李專美草檄求援諸道,欲誅君側之罪。朝廷命王
 思同率師來討。三月十五日,外兵大集。《九國志·李彥琦傳》:潞王守岐下,諸道將急攻其壘,彥琦時在圍中,罄家財以給軍用。十六日,大將督眾攻城,帝登城垂泣,諭于外曰:「我年未二十從先帝征伐,出生入死,金瘡滿身,樹立得社稷,軍士從我登陣者多矣。今朝廷信任賊臣,殘害骨肉,且我有何罪!」因慟哭,聞者哀之。時羽林都指揮使楊思權謂眾曰:「大相公,吾主也。」遂引軍自西門入,嚴衛都指揮使尹暉亦引軍自東門而入,外軍悉潰。十七日,率居民家財以賞軍士。是日,帝整眾而
 東。二十日,次長安,副留守劉遂雍以城降,率京兆居民家財犒軍。二十三日,次靈口,誅王思同。二十四日,次華州,收藥彥稠繫獄。二十五日,次閿鄉,王仲皋父子迎謁,命誅之。二十六日,次靈寶,河中節度使安彥威來降,待罪,宥之,遣歸鎮。陜州節度使康思立奉迎。二十七日,次陜州,下令告諭京城。二十八日,康義誠軍前兵士相繼來降,義誠詣軍門請罪,帝宥之。駕下諸軍畢至,誅宣徽南院使孟漢瓊于路左。是夜,閔帝與帳下親騎百餘出
 元武門而去。



 夏四月壬申,帝至蔣橋,文武百官立班奉迎,教旨以未拜梓宮,未可相見,俟會于至德宮,時六軍勳臣及節將內職已累表勸進。是日,帝入謁太后、太妃,至西宮,伏梓宮慟哭,宰相與百僚班見致拜,帝答拜。馮道等上箋勸進,帝立謂群臣曰:「予之此行,事非獲已,當俟主上歸闕,園陵禮終,退守籓服。諸公言遽及此,信無謂也。」衛州刺史王宏贄奏,閔帝以前月二十九日至州。癸酉,皇太后下令降閔帝為鄂王。又,太后令曰:「先皇帝
 誕膺天眷,光紹帝圖,明誠動于三靈,德澤被于四海,方期偃革,遽歎遺弓。自少主之承祧,為奸臣之擅命,離間骨肉,猜忌磐維,既輒易於籓垣,復驟興于兵甲。遂致輕離社稷,大撓軍民,萬世鴻基,將墜于地。皇長子潞王從珂,位居塚嗣,德茂沖年,乃武乃文,惟忠惟孝。前朝廓清多難,有戰伐之大功;纘紹丕圖,有夾輔之盛業。今以宗祧乏祀,園寢有期,須委親賢,俾居監撫,免萬機之壅滯,慰兆庶之推崇。可起今月四日知軍國事,權以書詔印
 施行。」是日,監國在至德宮,宰臣馮道等率百官班于宮門待罪。帝出于庭,曰:「相公諸人何罪,請復位。」乃退。甲戌,太后令曰:「先皇帝櫛風沐雨,平定華夷,嗣洪業于艱難,致蒼生于富庶。鄂王嗣位,奸臣弄權,作福作威,不誠不信,離間骨肉,猜忌磐維。鄂王輕捨宗祧,不克負荷,洪基大寶,危若綴旒,須立長君,以紹丕構。皇長子潞王從珂,日躋孝敬,天縱聰明,有神武之英姿,有寬仁之偉略。先朝經綸草昧,廓靜寰區,辛勤有百戰之勞,忠貞贊一統
 之運,臣誠子道,冠古超今。而又克己化民,推心撫士,率土之謳歌有屬,上蒼之眷命攸臨。一日萬機,不可以暫曠;九州四海,不可以無歸。況因山有期,同軌斯至,永言嗣守,屬任元良,宜即皇帝位。」



 乙亥,監國赴西宮,柩前告奠即位。攝中書令李愚宣冊書曰:



 維應順元年歲次甲午,四月庚午朔,六日乙亥,文武百僚,特進、守司空兼門下侍郎、同中書門下平章事、充太微宮使、弘文館大學士、上柱國、始平郡公、食邑二千五百戶臣馮道等九千
 五百九十三人上言:帝王興運,天地同符,河出圖而洛出書,雲從龍而風從虎。莫不恢張八表,覆育兆民,立大定之基,保無疆之祚。人謠再洽,天命顯歸,須登宸極之尊,以奉祖宗之祀。伏惟皇帝陛下,天資仁智,神助機權,奉莊宗于多難之時,從先帝于四征之際,凡當決勝,無不成功。洎正皇綱,每嚴師律,為國家之志大,守臣子之道全。自泣遺弓,常悲易月,欲期同軌,親赴因山。而自鄂王承祧,奸臣擅命,致神祇之乏饗,激朝野以歸心。使屈者
 伸,令否者泰,人情大順,天象至明。聚東井以呈祥,拱北辰而應運。由是文武百辟,岳牧群賢,至于比屋之倫,盡祝當陽之位。今則承太后慈旨,守先朝遠圖,撫四海九州,享千齡萬祀。臣等不勝大願,謹上寶冊,稟太后令,奉皇帝踐祚。臣等誠慶誠忭,謹言。



 帝就殿之東楹受群臣稱賀。先是,帝在鳳翔日,有瞽者張濛自言知術數,事太白山神,其神祠即元魏時崔浩廟也。時之否泰,人之休咎,濛告于神,即傳吉凶之言,帝親校房暠酷信之。一
 日,濛至府,聞帝語聲,駭然曰:「非人臣也。」暠詢其事,即傳神語曰:「三珠併一珠,驢馬沒人驅,歲月甲庚午,中興戊己土。」暠請解釋,曰:「神言予不知也。」長興四年五月,府廨諸門無故自動,人頗駭異。遣暠問濛,濛曰:「衙署小異勿怪,不出三日,當有恩命。」是夜報至,封潞王。及帝移鎮河東,甚懼,問濛,濛曰:「王保無患。」王思同兵至,又詰之,濛曰:「王有天下,不能獨力,朝廷兵來迎王也。王若疑臣,臣唯一子,請王致之麾下,以質臣心。」帝乃以濛攝館驛巡官。
 至是,帝受冊,冊曰:「維應順元年歲次甲午,四月庚午朔。」帝回視房暠曰:「張濛神言甲庚午,不亦異乎!」帝令暠共術士解三珠一珠事,言:「三珠,三帝也;驢馬沒人驅,失位也。」帝即位之後,以濛為將作少監同正,仍賜金紫以酬之。帝初封潞王,言事者云:「潞字一足已入洛矣。」又,帝在鳳翔日,有何叟者,年踰七十,暴卒,見陰官憑几告叟曰:「為我言于潞王,來年三月當為天子,二十三年。」叟既蘇,懼不敢言。逾月復卒,陰官見而叱之曰:「安得違吾旨,不
 達其事,再放汝還。」退見廊廡下簿書,以問主者,曰:「朝代將易,此即升降人爵之籍也。」及蘇,詣帝親校劉延朗告之。帝召而問之,叟曰:「請質之,此言無征,戮之可也。」後人云:「二十三,蓋帝之小字也。」又,石壕人胡杲通善天文,帝召問之,曰:「王貴不可言,若舉動,宜以乙未年。」及舉兵,又問之,杲通曰:「今歲篰首,王者不宜建功立事,若俟來歲入朝,則福祚永遠矣。」其後皆驗。夫如是,則大寶之位,必有冥數,可輕道哉!



 丙子,詔河南府率京城居民之財以
 助賞軍。丁丑,又詔預借居民五箇月房課,不問士庶,一概施行。帝素輕財好施,自岐下為諸軍推戴,告軍士曰:「候入洛,人賞百千。」至是,以府藏空匱,于是有配率之令,京城庶士自絕者相繼。己卯,衛州奏,此月九日鄂王薨。庚辰,以宰臣劉煦判三司。辛巳,邢州奏,磁州刺史宋令詢自經而卒。令詢,鄂王在籓時都押牙也,故至于是。甲申,帝以鄂王薨,行服于內園,群臣奉慰。癸未,太后、太妃出宮中衣服器用,以助賞軍。



 乙酉,帝服袞冕御明堂殿,
 文武百僚朝服就位,宣制改應順元年為清泰元年,大赦天下,常赦不原者咸赦除之。丁亥,以宣徽北院使郝瓊為宣徽南院使,權判樞密院;以前三司使王玫為宣徽北院使。以隨駕牙將宋審虔為皇城使,劉延朗為莊宅使。鳳翔節度判官韓昭允為左諫議大夫,充端明殿學士;觀察判官馬裔孫為翰林學士;掌書記李專美為樞密院直學士。戊子,侍衛親軍都指揮使康義誠伏誅。是日,詔曰:樞密使朱宏昭、馮贇、宣徽南院使孟漢瓊、西
 京留守王思同、前邠州節度使藥彥稠,共相朋煽,妄舉干戈,互興離間之謀,幾構傾亡之禍,宜行顯戮,以快群情,仍削奪官爵云。



 庚寅,鳳翔奏,西川孟知祥僭稱大蜀,年號明德。有司上言:「皇帝以五月朔日御明堂殿受朝,三日夏至,祀皇地祇,前二日奏告獻祖室,不坐。比正旦冬至,是日有祀事,則次日受朝。今祀在五鼓前,質明行禮華,御殿在旦後,請比例行之。」詔曰:「日出御殿,舉祀事無妨,宜依常年例。」史館奏:「凡書詔及處分公事,臣下奏
 議,望令近臣錄付當館。」詔端明殿學士韓昭允、樞密直學士李專美錄送。辛卯,以左諫議大夫盧損為右散騎常侍。壬辰,詔賜禁軍及鳳翔城下歸命將校錢帛各有差。《通鑒》云:禁軍在鳳翔歸命者,自楊思權、尹暉等各賜二馬一駝、錢七十緡,下至軍人錢二十緡,其在京者各十緡。初,帝離岐下,諸軍皆望以不次之賞,及從至京師,不滿所望,相與謠曰:「去卻生菩薩,扶起一條鐵。」其無厭如此。丙申,葬明宗皇帝于徽陵。丁酉,奉神主于太廟。戊戌,山陵使、司空兼門下侍郎、平章事馮道上表納政,不
 允。



 五月庚子朔,御文明殿受朝賀。乙巳,以左龍武指揮使安審琦為左右捧聖都指揮使,以右千牛上將軍符彥饒為左右嚴衛都指揮使。丙午,以端明殿學士韓昭允為樞密使;以莊宅使劉延朗為樞密副使;以權知樞密事房暠為宣徽北院使;以成德軍節度使、大同彰國振武威塞等軍蕃漢馬步都部署、檢校太尉、兼中書令、駙馬都尉石敬瑭為北京留守、河東節度使,加檢校太師、兼中書令,都部署如故。汴州節度使、檢校太師、兼侍中、
 駙馬都尉趙延壽進封魯國公。



 戊申,中書門下奏,太常禮院狀,明宗以此月二十日祔廟,宰臣攝太尉行事。緣馮道在假,李愚十八日私忌,在致齋,劉煦又奏判三司免祀事,《五代會要》:清泰元年五月,宰臣劉煦奏:「中書以近敕祠祭行事官致齋內,唯祀事得行,其餘悉斷。又,宰臣行事致齋內,不押班,不赴內殿起居,不知印。臣緣判三司公事,其祀事、國忌、行香,伏乞特免。」從之。詔禮官參酌。有司上言:「李愚私忌,在致齊內。諸私忌日,遇大朝會入閣宣召,皆赴朝參。今祔廟事大,忌日屬私,請比大朝會宣召例。」從之。以陜府節度使康思立為
 邢州節度使,以同州節度使安重霸為西京留守,以羽林右第一軍都指揮使、春州刺史楊思權為邠州節度使。己酉,左監門衛將軍孔知鄴、右驍衛將軍華光裔並勒停見任。時差知鄴應州告廟,稱疾辭命,改差光裔,復稱馬墜傷足,故俱罷之。



 庚戌,以司空兼門下侍郎、平章事馮道為檢校太尉、同平章事,充同州節度使;以天雄軍節度使范延光為樞密使,封齊國公;鄆州節度使李從嚴為鳳翔節度使。辛亥,以嚴衛都指揮使尹暉為齊
 州防禦使。甲寅,以侍衛馬軍都指揮、順化軍節度使安從進為河陽節度使,典軍如故。太常卿盧文紀奏:「明宗一室,酌獻舞曲,請名《雍熙之舞》。」從之。丁巳,以皇子銀青光祿大夫、檢校工部尚書重美為檢校司徒、守左衛上將軍。自是,諸道節度使、刺史、文武臣僚,相繼加檢校官,或階爵封邑,以帝登位覃慶也。戊午,以隴州防禦使相里金為陜州節度使。初,帝以檄書告籓鄰,惟金遣判官薛文遇往來計事,故以節鎮獎之。宣徽北院使、檢校工
 部尚書房暠加檢校司空,行左威衛大將軍,使如故;以樞密使、左諫議大夫韓昭允為刑部尚書,使如故。



 己未,太白晝見。以樞密副使劉延朗為左領軍大將軍,職如故。庚申,左僕射、門下侍郎、平章事、監修國史李愚加特進,充太微宮使、宏文館大學士,餘如故。中書侍郎、兼吏部尚書、同平章事、集賢院大學士、判三司劉煦加門下侍郎、兼吏部尚書、平章事、監修國史、判三司。癸亥,秦州奏,西川孟知祥出軍迫陷成州。以宣徽南院使、右驍衛大
 將軍郝瓊為左驍衛上將軍,職如故;以前義州刺史張承祐為武勝軍留後。戊辰,以前右龍武統軍王景戡為右驍衛上將軍。



 六月庚午朔,改侍衛捧聖軍為彰聖,改嚴衛軍為寧衛。壬申,封吳岳成德公為靈應王,禮秩同五岳。帝初起,遣使祭岳以求祐,及登祚,故有是報。《五代會要》載中書門下奏:天寶十載正月,封吳山成德公,與沂山、會稽、醫巫閭同封。至德二載十二月,改吳山為岳,祠享官屬一同五岳。今國家欲祈禱靈應,宜示殊禮,臣等商量,請加封為靈應王。從之。幽州節度使趙德鈞進封北平王,青州節度使房知溫進封東平王。
 癸酉,以前鄜州節度使索自通為右龍武統軍。甲戌,皇子左衛上將軍重美加檢校太保、同平章事,充鎮州節度使兼河南尹,判六軍諸衛事。丁丑,詔天下見禁罪人,委所在長吏躬親慮問,疾速疏決。庚辰,幸至德宮,因幸房知溫、安元信、范延光、索自通、李從敏第。壬午,以檢校太子太傅致仕王建立為檢校太尉、兼侍中、鄆州節度使;以前宋州節度使安元信為檢校太尉、兼侍中、潞州節度使。



 癸未,三司使劉煦奏:「天下戶民,自天成二年括
 定秋夏田稅,迨今八年。近者相次有百姓詣闕訴田不均,累行蠲放,漸失稅額,望差朝臣一概檢視。」不報。甲申,帝為故皇子亳州刺史重吉、皇長女尼惠明大師幼澄舉哀行服,群臣詣閣門奉慰。帝起兵之始,重吉、幼澄俱為閔帝所害。乙酉,以戶部侍郎韓彥惲為絳州刺史,以左武衛上將軍李肅為單州刺史。丙戌,襄州節度使趙在禮加同平章事。甲午,以武勝軍留後張承祐為華州節度使;以皇城使宋審虔為壽州節度使,充侍衛步軍
 都指揮使;以右衛上將軍劉仲殷為宋州節度使;以侍衛步軍都指揮使、壽州節度使皇甫遇為鄧州節度使;以前華州節度使華溫琪為太子太傅致仕。丁酉,左神武統軍周知裕卒,贈太傅。是月,京師大旱,熱甚,暍死者百餘人。



 秋七月庚子,太子少保致仕崔沂卒。癸卯,鳳翔進偽蜀孟知祥來書,稱「大蜀皇帝獻書于大唐皇帝」,且言「見迫群情,以今年四月十二日即皇帝位」云,帝不答。以前武州刺史鄭琮為右衛上將軍。甲辰,幸龍門佛寺
 禱雨。乙巳,皇子故亳州團練使重吉贈太尉,仍于宋州置廟。丁未,鳳翔節度使李從嚴封西平王。是日,宰臣李愚、劉煦因論公事,于政事堂相詬,辭甚鄙惡,帝令樞密副使劉延朗宣諭曰:「卿等輔弼之臣,不宜如是,今後不得更然。」辛亥,以太常卿盧文紀為中書侍郎、平章事。是日,中書門下三上章請立中宮,從之。丁巳,制立沛國夫人劉氏為皇后。庚申,太子少傅陳皋卒。乙丑,史官張昭遠以所撰莊宗朝列傳三十卷上之。



 八月庚午,詔蠲放
 長興四年十二月已前天下所欠殘稅。辛未,以前尚書左丞姚顗為中書侍郎、平章事。詔應曾受御署官逐攝同一任正官,依期限赴選。徐無黨《五代史注》云:御署官,疑是廢帝初舉兵時所置之官,以其非吏部正授,故須有旨方得選。荊南奏,偽蜀孟知祥卒,其子昶嗣偽位。壬申,以尚書禮部侍郎鄭韜光為刑部侍郎,以前工部侍郎楊凝式為禮部侍郎。甲戌,以前金州防禦使婁繼英為右神武統軍,以右神武統軍高允貞為左神武統軍。乙亥,以翰林學士承旨、工部尚書、知制誥李懌為
 太常卿;以翰林學士、戶部侍郎、知制誥程遜為學士承旨。甲申,以兵部侍郎龍敏為吏部侍郎,以秘書監崔居儉為工部尚書。乙酉,以右武衛上將軍張繼祚為右衛上將軍;以右驍衛上將軍王景戡為左衛上將軍;以右領衛上將軍劉衛為左武衛上將軍;以右千牛上將軍王陟為右領軍上將軍;以司農卿兼通事舍人,判四方館事王景崇為鴻臚卿,依前通事舍人、判四方館。丁亥,右龍武統軍索自通卒。辛卯,禮部尚書致仕李光憲卒。
 甲午,以太子少傅盧質為太子少師。乙未,以前邢州節度使趙鳳為太子太保。詔:「文武百官差使,宜令依倫次,中書置簿,不得重疊。若當使者自緣有事,或不欲行者,注簿便當一使。自長興三年正月後已曾奉使者,便為簿首;已後差者,次第注之。」有司上言:「皇后受冊,內外命婦上箋無答教。」從之。丙申,御文明殿冊皇后,命使攝太尉、宰臣盧文紀,使副攝司徒、右諫議大夫盧損指皇后宮,行禮畢,恩賜有差。



 九月己亥,以久雨,分命朝臣營都
 城門,告宗廟社稷。辛丑,夜有星如五斗器,西南流,尾跡長數丈,屈曲如龍形。又眾星亂流,不可勝數。京師大雨,雹如彈丸。曹州刺史藥縱之卒。甲辰,以霖霪甚,詔都下諸獄委御史臺憲錄問,諸州縣差判官令錄親自錄問,畫時疏理。壬子,中書門下舉行長興三年敕,常年薦送舉人,州郡行鄉飲酒之時,帖太常草定儀注奏聞。甲寅,以前潞州節度使、檢校太尉、同平章事盧文進為安州節度使。己未,雲州奏,契丹寇境。



 冬十月辛未,有雉金色,
 止于中書政事堂。中書門下奏:「請以正月二十三日皇帝誕慶日為千春節。」從之。戊寅,宰臣李愚、劉煦罷相,以愚守左僕射,煦守右僕射。契丹寇雲、應州,詔河東節度使石敬瑭率兵屯代州。戊子,宰臣姚顗奏:「吏部三銓,近年併為一司,望令依舊分銓。」從之。辛卯,以左衛上將軍李宏元卒廢朝,贈司徒。癸巳,以禮部郎中、知制誥呂琦守本官,充樞密院直學士。



 十一月辛丑,以刑部侍郎鄭韜光為尚書右丞,以光祿少卿烏昭遠為少府監。秦州
 節度使張延郎奏,率師伐蜀。中書門下奏:「二十六日明宗忌,陛下初遇忌辰,不同常歲,請于忌辰前後各一日不坐朝。」從之。御史臺奏:「前任節度使、刺史、行軍副使,雖每日于便殿起居,每遇五日起居,亦合綴班。」從之。丙午,以前興州刺史馮暉配同州衙前安置。暉為興州刺史,屯乾渠,蜀人來侵,暉自屯所奔歸鳳翔,故有是責。丁未,詔振武、新州、河東西北邊經契丹蹂踐處,放免三年兩稅差配,時契丹初退故也。癸丑,以前華州節度使王萬
 榮為左驍衛上將軍致仕。甲寅,以振武節度使楊光遠充大同、彰國、振武、威塞等軍兵馬都虞候,以前右金吾大將軍穆延暉為右武衛上將軍。壬戌,以禮部侍郎楊凝式為戶部侍郎。甲子,以中書舍人盧導為禮部侍郎。



 十二月丁卯朔,詔修奉本朝諸帝陵寢。己巳,以北面馬軍都指揮使、易州刺史安叔千為安北都護、振武節度使;以齊州防禦使尹暉為彰國軍節度使。庚午,詔葬庶人從榮。有司上言:「依貞觀中庶人承乾,以公禮葬。」從之。
 乙亥,以秦州節度使張延朗為中書侍郎、同平章事、判三司;《五代會要》:二年三月,宰臣張延朗奏:「臣判三司公事,每日內殿祗候,其合綴前班押班,伏乞特免。」從之。以中書侍郎、平章事盧文紀為門下侍郎、平章事、監修國史;以中書侍郎、平章事姚顗兼集賢院大學士;以前邠州節度使康福為秦州節度使。丙戌,夜有白氣,東西亙天。庚寅,幸龍門祈雪,自九月至是無雨雪故也。



\end{pinyinscope}