\article{莊宗紀三}

\begin{pinyinscope}

 天祐十六年春正月,李存審城德勝,夾河為柵。帝還魏州,命昭義軍節度使李嗣昭權知幽州軍府事。三月,帝兼領幽州,
 遣近臣李紹宏提舉府事。



 夏四月,梁將賀瑰圍德勝南城,百道攻擊,復以艨艟扼斷津渡。帝馳而往,陣於北岸。南城守將氏延賞告急,且言矢石將盡。帝以重賄召募能破賊艦者,於是獻技者數十,或言能吐火焚舟,或言能禁咒兵刃,悉命試之,無驗。帝憂形於色,親從都將王建及進曰:「臣請效命。」乃以巨索連舟
 十艘,選效節勇士三百人,持斧被鎧,鼓枻而進,至中流。梁樓船三層,蒙以牛革,懸板為楯。建及率持斧者入艨艟間,斬其竹笮,破其懸楯;又於上流取甕數百,用竹笮維之,積薪於上,灌以脂膏,火發亙空;又以巨艦載甲士,令乘煙鼓噪。梁之樓船斷紲而下,沈溺者殆半。軍既得渡,梁軍乃退。命騎軍追襲至濮陽,俘斬千計。賀瑰由此飲氣遘疾而卒。



 秋七月,帝歸晉陽。八月,梁將王瓚帥眾數萬自黎陽渡河,營於楊村,造舟為梁,以通津路。冬十月,帝自晉陽至魏州,發徒數萬以廣德勝北城,自是,日與
 梁軍接戰。



 十二月戊戌,帝軍於河南,夜伏步兵於潘張村梁軍寨下,以騎軍掠其餉運,擒其斥候。梁王瓚結陣以待入,帝軍以鐵騎突之,諸軍繼進,梁軍大奔,赴水死者甚眾,瓚走保北城。



 天祐十七年春,幽州民於田中得金印,文曰:「關中龜印」,李紹宏獻於行臺。



 秋七月,梁將劉鄩、尹皓寇同州。先是,河中節度使硃友謙取同州,以其子令德主留務,請梁主降節。梁主怒,不與,遂請旄節於帝。梁主乃遣劉鄩與華州節度使尹皓帥兵圍同州,友謙來告難,帝遣蕃漢總管
 李存審、昭義節度使李嗣昭、代州刺史王建及率師赴援。



 九月,師至河中,朝至夕濟,梁人不意
 王師之至,望之大駭。明日約戰,與硃友謙謀,遲明,進軍距梁壘;梁人悉眾以出,蒲人在南,王師在北。騎軍既接,蒲人小卻,李嗣昭
 以輕騎抗之,梁軍奔潰,追斬二千餘級。是夜,劉鄩收餘眾保營,自是閉壁不出。數日,鄩遂宵遁。王師追及於渭河,所棄兵仗輜重不可勝計,劉鄩、尹皓單騎獲免。未幾,鄩憂恚發病而卒。王師略地至奉先,嗣昭因謁唐帝
 諸陵而還。



 天祐十八年春正月,魏州開元寺僧傳真獲傳國寶,獻於行臺。驗其文,即「受命於天,子孫寶之」八字也,群僚稱賀。傳真師於廣明中,遇京師喪亂得之,秘藏已四十年矣。篆文古體,人不之識,至是獻之。時淮南楊溥、四川王衍皆遣使致書,勸帝嗣唐帝位,帝不從。



 二月,代州刺史王建及卒。是月,鎮州大將張文禮殺其帥
 王鎔。
 時帝方與諸將宴,酒酣樂作,聞鎔遇殺,遽投觶而泣曰:「趙王與吾把臂同盟,分如金石,何負於人,覆宗絕祀,冤哉!」先是,滹沱暴漲,漂關城之半,溺死者千計。是歲,天西北有赤昆如血,占者言趙分之災,至是
 果驗。時張文禮遣使請旄節於帝,帝曰:「文禮之罪,期於無赦,敢邀予旄節!」左右曰:「方今事繁,不欲與人生事。」帝不得已而從之,乃承制授文禮鎮州兵馬留後。



 三月,河中節度使朱友謙、昭義節度使李嗣昭、滄州節度使李存審、定州節度使王處直、邢州節度使李嗣源、成德軍兵馬留後張文禮、遙領天平軍節度使閻寶、大同軍節度使李存璋、新州節度使王郁、振武節度使李存進、同州節度使朱令德,各遣使勸進,請帝紹唐帝位,帝報書
 不允。自是,諸鎮凡三上章勸進,各獻貨幣數十萬,以助即位之費,帝左右亦勸帝早副人望,帝捴挹久之。《九國志·趙季良傳》:季良嘗蘿手扶御座,自謂輔佐之象,由是頗述天時人事以諷,莊宗深納其言。秋七月,河東節度副使盧汝弼卒。



 八月庚申,令天平節度使閻寶、成德兵馬留後符習率兵討張文禮於鎮州。初,王鎔令偏將符習以本部兵從帝屯於德勝。文禮既行弒逆,忌鎔故將,多被誅戮,因遣使聞於帝,欲以他兵代習歸鎮,習等懼,請留。帝令傳旨於習及別將趙仁貞、烏震等,明
 正文禮弒逆之罪,且言:「爾等荷戟從征,蓋君父之故,銜冤報恩,誰人無心。吾當給爾資糧,助爾兵甲,爾試思之!」於是習等率諸將三十餘人,慟哭於牙門,請討文禮。帝因授習成德軍兵馬留後,以部下鎮、冀兵致討於文禮;又遣閻寶以助之,以史建瑭為前鋒。甲子,攻趙州,刺史王鋌送符印以迎,閻寶遂引軍至鎮州城下,營於西北隅。是月,張文禮病疽而卒,其子處瑾代掌軍事。九月,前鋒將史建瑭與鎮人戰於城下,為流矢所中而卒。



 冬十月
 己未,梁將戴思遠攻德勝北城,帝命李嗣源設伏於戚城,令騎軍挑戰。梁軍大至,帝御中軍以禦之。時李從珂偽為梁幟,奔入梁壘,斧其眺樓,持級而還。梁軍愈恐,步兵漸至,李嗣源以鐵騎三千乘之,梁軍大敗,俘斬二萬計。辛酉,閻寶上言,定州節度使王處直為其子都幽於別室,都自稱留後。《歐陽史》:王處直叛附於契丹,其子都幽處直以來附。



 十一月,帝至鎮州城下,張處瑾遣弟處琪、幕客齊儉等候帝乞降,言猶不遜,帝命囚之。時王師築土山以攻其壘,城中亦
 起土山以拒之,旬日之間,機巧百變。張處瑾令韓正時以千騎夜突圍,將入定州與王處直議事,為我游軍追擊,破之;餘眾保衡唐,賊將彭贇斬正時以降。



 十二月辛未,王郁誘契丹安巴堅寇幽州,《契丹國志》:王處直在定州,以鎮、定為脣齒,恐鎮亡而定孤,乃潛使人語其子王郁,使賂契丹,令犯塞以救鎮州之圍。王鬱說太祖曰:「鎮州美女如雲,金帛似山,天皇速往,則皆為己物也;不然,則為晉王所有矣。」太祖以為然,率眾而南。遂引軍涿州,陷之。又寇定州,王都遣使告急,帝自鎮州率五千騎赴之。



 天祐十九年春正月甲午,帝至新城,契丹前鋒三千騎至
 新樂。是時,梁將戴思遠乘虛以寇魏州,軍至魏店,李嗣源自領兵馳入魏州。梁人知其有備,乃西渡洹水,陷成安而去。時契丹渡沙河,而諸將相顧失色;又聞梁人內侵,鄴城危急,皆請旋師,唯帝謂不可,乃率親騎至新城。契丹萬餘騎,遽見帝軍,惶駭而退。帝分軍為二廣,追躡數十里,獲安巴堅之子。時沙河冰薄,橋梁隘狹,敵爭踐而過,陷溺者甚眾。安巴堅方在定州,聞前軍敗,退保望都。帝至定州,王都迎謁。是夜,宿於開元寺。翼日,引軍至望都,
 契丹逆戰。帝身先士伍,馳擊數四,敵退而結陣,帝之徒兵亦陣于水次。李嗣昭躍馬奮擊,敵眾大潰,俘斬數千,追擊至易州,獲氈裘、毳幕、羊馬不可勝紀。時歲且北至,大雪平地五尺,敵乏芻糧,人馬斃踣道路,纍纍不絕,帝乘勝追襲至幽州。《契丹國志》:晉王趨望都,為契丹所圍,力戰,出入數四,不解。李嗣昭引三百騎橫擊之,晉王始得出,因縱兵奮擊,太祖兵敗,遂北至易州。會大雪彌旬,平地數尺,人馬死者相屬,太祖乃歸。是月,梁將戴思遠寇德勝北城,築壘穿塹,地道雲梯,晝夜攻擊;李存審極力拒守,城中危急。帝自幽州聞之,倍
 道兼行以赴,梁人聞帝至,燒營而遁。



 三月丙午,王師敗於鎮州城下,閻寶退保趙州。時鎮州累月受圍,城中艱食,王師築壘環之;又決滹沱水以絕城中出路。是日,城中軍出,攻其長圍,皆奮力死戰,王師不能拒,引師而退。鎮人壞其營壘,取其芻糧者累日。帝聞失律,即以昭義節度使李嗣昭為北面招討使,進攻鎮州。夏四月,嗣昭為流矢所中,卒於師。己卯,天平節度使閻寶卒。以振武節度使李存進為北面招討使。是月,大同軍節度使李存璋
 卒。


五月乙酉,李存進圍鎮州,營於東渡。八月,梁將段凝陷衛州,刺史李存儒被擒。存儒,本俳優也,帝以其有膂力,故用為衛州刺史。既而誅斂無度,人皆怨之,故為梁人所襲。
 \gezhu{
  《九國志·趙季良傳》:莊宗入鄴,時兵革屢興,屬邑租賦逋久。一日,莊宗召季良切責之,季良對曰:「殿下何時平河南?」莊宗正色曰:「爾掌輿賦而稽緩,安問我勝負乎!」季良曰:「殿下方謀攻守,復務急徵,一旦眾心有變,恐河南非殿下所有。」莊宗斂容前席曰:「微君之言,幾失吾大計!」}
 梁將戴思遠又陷共城、新鄉等邑。自是,澶淵之西,相州之南,皆為梁人所據。



 九月戊寅朔,張處球悉城中兵奄至東垣渡,急攻我之壘門。
 時騎軍已臨賊城,不覺其出,李存進惶駭,引十餘人鬥於橋上,賊退,我之騎軍前後夾擊之,賊眾大敗,步兵數千,殆無還者。是役也,李存進戰歿於師,以蕃漢馬步總管李存審為北面招討使,以攻鎮州。丙午夜,趙將李再豐之子沖投縋以接王師,諸軍登城,遲明畢入,鎮州平。獲處球、處瑾、處琪并其母,及同惡高濛李翥、齊儉等,皆折足送行臺,鎮人請醢而食之;發張文禮尸,磔於市。帝以符習為鎮州節度使,烏震為趙州刺史,趙仁貞為深
 州刺史,李再豐為冀州刺史。鎮人請帝兼領本鎮,從之,乃以符習遙領天平軍節度使。



 十一月,河東監軍張承業卒。



 十二月,以魏州觀察判官張憲權知鎮州軍州事。



 同光元年春正月丙子,五臺山僧獻銅鼎三,言於山中石崖間得之。二月,新州團練使李嗣肱卒。是時,以諸籓鎮相繼上箋勸進,乃命有司制置百官省寺仗衛法物,期以四月行即位之禮,以河東節度判官廬質為大禮使。



 三月己卯,以橫海軍節度使、內外蕃漢馬步總管李存
 審為幽州節度使。潞州留後李繼韜叛,送款於梁。是月,築即位壇於魏州牙城之南。



 夏四月己巳,帝升壇,祭告昊天上帝,遂即皇帝位,文武臣僚稱賀。禮畢,御應天門宣制:改天祐二十年為同光元年,大赦天下,自四月二十五日昧爽以前,除十惡五逆、放火行劫、持杖殺人、官典犯贓、屠牛鑄錢、合造毒藥外,罪無輕重,咸赦除之。應蕃漢馬步將校並賜功臣名號,超授檢校官,已高者與一子六品正員官,兵士並賜等第優給。其戰歿功臣各
 加追贈,仍賜謚號。民年八十已上,與免一子役。內外文武職官,並可直言極諫,無有隱諱。貢、選二司,宜令有司速商量施行。雲、應、蔚、朔、易、定、幽、燕及山後八軍,秋夏稅率量與蠲減。民有三世已上不分居者,與免雜徭。諸道應有祥瑞,不用聞奏。赦書有所未該,委所司條奏以聞云。是歲,自正月不雨,人心憂恐,宣赦之日,澍雨溥降。



 初,唐咸通中,金、水、土、火四星聚於畢、昴,太史奏:「畢、昴,趙、魏之分,其下將有王者。」懿宗乃詔令鎮州王景崇被袞冕
 攝朝三日,遣臣下備儀注、軍府稱臣以厭之。其後四十九年,帝破梁軍於柏鄉,平定趙、魏,至是即位於鄴宮。是月,以行臺左丞相豆盧革為門下侍郎、同中書門下平章事、太清宮使;以行臺右丞相盧澄為中書侍郎平章事、監修國史;以前定州掌書記李德休為御史中丞;以河東節度判官盧質為兵部尚書,充翰林學士承旨;以河東掌書記馮道為戶部侍郎,充翰林學士;以魏博、鎮冀觀察判官張憲為工部侍郎,充租庸使;以中門使郭
 崇韜、昭義監軍使張居翰並為樞密使;以權知幽州軍府事李紹宏為宣徽使;以魏博節度判官王正言為禮部尚書,行興唐尹;以河東軍城都虞候孟知祥為太原尹,充西京副留守;以澤潞節度判官任圓為工部尚書兼真定尹,充北京副留守。詔升魏州為東京興唐府,改元城縣為興唐縣,貴鄉縣為廣晉縣,以太原為西京,以鎮州為北都。是時,所管節度一十三,州五十。



 閏月丁丑,以李嗣源為檢校侍中,依前橫海軍節度使、內外蕃
 漢副總管;以幽州節度使李存審為檢校太師、兼中書令,依前蕃漢馬步總管;以河東節度使朱友謙為檢校太師、兼尚書令。安國軍節度使符習加同平章事,定州節度使王都加檢校侍中。是月,追尊曾祖蔚州太保為昭烈皇帝,廟號懿祖;夫人崔氏曰昭列皇后。追尊皇祖代州太保為文景皇帝,廟號獻祖;夫人秦氏曰文景皇后。追尊皇考河東節度使、太師、中書令、晉王為武皇帝,廟號太祖。詔於晉陽立宗廟,以高祖神堯皇帝、太宗文
 皇帝、懿宗昭聖皇帝、昭宗聖穆皇帝及懿祖以下為七廟。甲午,契丹寇幽州,至易、定而還。時有自鄆來者,言節度使戴思遠領兵在河上,州城無守兵,可襲而取之。帝召李嗣源謀曰:「昭義阻命,梁將董璋攻迫澤州,梁志在澤、潞,不虞別有事生,汶陽無備,不可失也。」嗣源以為然。壬寅,命嗣源率步騎五千,箝枚自河趨鄆。是夜陰雨,我師至城下,鄆人不覺,遂乘城而入,鄆州平。制以李嗣源為天平軍節度使。梁主聞鄆州陷,大恐,乃遣王彥章代戴
 思遠總兵以來拒。時朱守殷守德勝南城,帝懼彥章奔衝,遂幸澶州。



 五月辛酉,彥章夜率舟師自楊村浮河而下,斷德勝之浮橋,攻南城,陷之。帝令中書焦彥賓馳至楊劉,固守其城;令朱守殷撤德勝北城屋木攻具,浮河而下,以助楊劉。是時,德勝軍食芻茭薪炭數十萬計,至是令人輦負入澶州,事既倉卒,耗失殆半。朱守殷以所毀屋木編筏,置步軍于其上。王彥章以舟師沿流而下,各行一岸,每遇轉灘水匯,即中流交鬥,流矢雨集,或全
 舟覆沒,一彼一此,終日百戰,比及楊劉,殆亡其半。已巳,王彥章、段凝率大軍攻楊劉南城,焦彥賓與守城將李周極力固守。梁軍晝夜攻擊,百道齊進,竟不能下,遂結營於楊劉之南,東西延袤十數柵。



 六月己亥,帝親御軍至楊劉,登城望見梁軍,重壕復壘,以絕其路,帝乃選勇士持短兵出戰。梁軍於城門外,連延屈曲,穿掘小壕,伏甲士於中,候帝軍至,則弓弩齊發,師人多傷矢,不得進。帝患之,問計於郭崇韜;崇韜請於下流據河築壘,以救
 鄆州。又請帝日令勇士挑戰,旬日之內,寇若不至,營壘必成。帝善之,即令崇韜與毛璋率數千人中夜往博州濟河東,晝夜督役,居六日,營壘將成。戊子,梁將王彥章、杜晏球領徒數萬,晨壓帝之新壘。時板築雖畢,牆仞低庳,戰具未備,沙城散惡,王彥章列騎環城,虐用其人,使步軍堙壕登堞。又於上流下巨艦十餘艘,扼斷濟路。自旦至午,攻擊百端,城中危急。帝自楊劉引軍陣于西岸,城中望之,大呼,帝艤舟將渡,梁軍遂解圍,退保鄒家口。



 秋七月丁未,帝御軍沿河而南,梁軍棄鄒家口夜遁,委棄鍋甲芻糧千計。戊午,遣騎將李紹貽直抵梁軍壘,梁益恐。又聞李嗣源自鄆州引大軍將至,己未夜,梁軍拔營而遁,復保於楊村。帝軍屯於德勝。甲子,帝幸楊劉城,巡視梁軍故壘。



 八月壬申朔,帝遣李紹斌以甲士五千援澤州。初,李繼韜之叛也,潞之舊將裴約以兵戍澤州,不徇韜之逆。既而梁遣董璋率眾攻其城,約拒守久之,告急於帝,故遣紹斌救之。未至而城已陷,裴約被害,
 帝聞之,嗟痛不已。甲戌,帝自楊劉歸鄴。梁以段凝代王彥章為帥。戊子,凝帥眾五萬結營於王村,自高陵渡河。帝軍遇之,生擒梁前鋒軍士二百人,戮於都市。庚寅,帝御軍至朝城。戊戌,梁左右先鋒指揮使康延孝領百騎來奔,帝虛懷引見,賜御衣玉帶,屏人問之。對曰:「臣竊觀汴人兵眾不少,論其君臣將校,則終見敗亡。趙巖、趙鵠、張漢傑居中專政,締結宮掖,賄賂公行。段凝素無武略,一朝便見大用;霍彥威、王彥章皆宿將有名,翻出其下。
 自彥章獲德勝南城,梁主亦稍獎使。彥章立性剛暴,不耐凌制,梁主每一發軍,即令近臣監護,進止可否,悉取監軍處分,彥章悒悒,形於顏色。自河津失利,段凝、彥章又獻謀,欲數道舉軍,合董璋以陜虢、澤潞之眾,趨石會關以寇太原。霍彥威統關西、汝、洛之眾自相衛以寇鎮定,段凝、杜晏球領大軍以當陛下,令王彥章、張漢傑統禁軍以攻鄆州,決取十月內大舉。又自滑州南決破河堤,使水東注曹、濮之間,至於汶陽,彌漫不絕,以陷北軍。
 臣在軍側聞此議。臣惟汴人兵力,聚則不少,分則無餘。陛下但待分兵,領鐵騎五千,自鄆州兼程直抵於汴,不旬日,天下事定矣。」帝懌然壯之。



 九月壬寅朔,帝在朝城,凝兵至臨河南,與帝之騎軍接戰。是時澤潞叛,衛州、黎陽為梁人所據,州以西、相以南,寇鈔日至,編戶流亡,計其軍賦,不支半年。又王郁、盧文進召契丹南侵瀛、涿。及聞梁人將圖大舉,帝深憂之,召將吏謀其大計,或曰:「自我得汶陽以來,須大將固守,城門之外,元是賊疆,細而
 料之,得不如失。今若馳檄告諭梁人,卻衛州、黎陽以為鄆州,指河為界,約且休兵。我國力稍集,則議改圖。」帝曰:「嘻,行此謀則無葬地矣!」時郭崇韜勸帝親御六軍,直趨汴州,半月之間,天下可定。帝曰:「正合朕意。大丈夫得則為王,失則為寇,予行計決矣!」又問司天監,對曰:「今歲時不利,深入必無成功。」帝弗聽。戊辰,梁將王彥章率眾至汶河,李嗣源遣騎軍偵視,至遞公鎮,梁軍來挑戰,嗣源以精騎擊而敗之,生擒梁將任釗、田章等三百人,俘斬
 二百級,彥章引眾保於中都。嗣源飛驛告捷,帝置酒大悅,曰:「是當決行渡河之策。」己巳,下令軍中將士家屬並令歸鄴。



\end{pinyinscope}