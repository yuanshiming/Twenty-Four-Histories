\article{明宗紀二}

\begin{pinyinscope}

 天成元年夏四月丙午,帝自興聖宮赴西宮,文武百僚縞素于位,帝服斬衰,親奉攢,塗設奠,哭盡哀,乃於柩前即皇帝位。百官易吉服班于位,帝御袞冕受冊訖,百僚
 稱賀。丁未,群官縞素赴西宮臨。以樞密使安重誨為檢校司空,守左領軍大將軍,依前充樞密使。宰臣豆盧革等三上表請聽政,從之。遣使往諸道及淮南告哀。辛亥,帝始聽政于中興殿。壬子,西南面副招討使、工部尚書任圜率步騎二萬六千人入見。甲寅,帝御文明殿受朝。制改同光四年為天成元年,大赦天下。後宮內職量留一百人,內官三十人,教坊一百人,鷹坊二十人,御廚五十人,其餘任從所適。諸司使務有名無實者並停。分遣諸
 軍就食近畿,以減饋送之勞。秋夏稅子,每斗先有省耗一升,今後只納正數,其省耗宜停。天下節度、防禦使,除正、至、端午、降誕四節量事進奉,達情而已,自於州府圓融,不得科斂百姓。其刺史雖遇四節,不在貢奉。諸州雜稅,宜定合稅物色名目,不得邀難商旅。租庸司先將繫省錢物,與人迴圖,宜令盡底收納,以塞幸門云。乙卯,渤海國王大諲譔遣使朝貢。是月,北京副留守、知留守事張憲賜死,以其失守故也。



 五月丙辰朔,帝不視朝,臨
 于西宮。宰相豆盧革進位左僕射,韋說進位門下侍郎兼戶部尚書、監修國史,並依舊平章事。兗州節度使、檢校太傅朱守殷加同平章事,充河南尹,判六軍諸衛事;滄州節度使、檢校太傅安元信加同平章事,移鎮徐州;邠州節度使、檢校太保毛璋加同平章事。以太子賓客鄭玨為中書侍郎兼刑部尚書、同中書門下平章事;以工部尚書任圜為中書侍郎兼工部尚書、同中書門下平章事、判三司。徐州節度使李紹真、貝州刺史李紹英、
 齊州防禦使李紹虔、河陽節度使李紹奇、洺州刺史李紹能等上言,前朝寵賜姓名,今乞還舊。內李紹虔上言:「臣本姓王,後移杜氏,蒙前朝賜今姓名,乞復本姓。」詔並可之。李紹真復曰霍彥威,李紹英復曰房知溫,李紹虔復曰王晏球,李紹奇復曰夏魯奇,李紹能復曰米君立。青州節度使、檢校太傅、同平章事符習加兼侍中,徐州節度使、檢校太傅霍彥威加兼侍中,移鎮鄆州。丁巳,初詔文武百僚正衙常參外,五日一度內殿起居。《五代會要》:天成
 元年五月三日敕:今後宰臣文武百官,除常朝外,每五日一度入內起居。其中書非時有急切公事請開延英,不在此限。麟州奏,指揮使張延寵作亂,焚剽市民,已殺戮訖。



 戊午,河陽節度使夏魯奇加檢校太傅,以貝州刺史房知溫為兗州節度使,以齊州防御使王晏球為宋州節度使,以洺州刺史米君立為邢州節度使。己未養無害,則塞于天地之間。其為氣也,配義與道,無是餒也。,賜文武百官驢馬各一。西都知府張篯進魏王繼岌打球馬七十二匹。北京馬步都指揮使李從溫奏,準詔誅宦官。初,莊宗遇內難,宦者數百人竄匿山谷,落髮為僧,奔至
 太原七十餘人,至是盡誅于都亭驛。辛酉,詔華州放散西川宮人各歸骨肉。壬戌,以前相州刺史、北京左右廂都指揮使安金全為安北都護、振武節度使、同平章事。甲子,前西都留守、京兆尹張筠加檢校太傅,充山南西道節度使;以夔州節度使李紹文為遂州節度使;以前鄧州留後戴思遠為洋州節度使。丁卯,以金吾將軍張實為金州防禦使。戊辰,以金紫光祿大夫、檢校司空趙在禮為滑州節度使,加檢校太保。制下,在禮以軍情不
 順為辭,不之任。以許州留後陶為鄧州留後,以諸道馬步副都指揮使安審通為齊州防禦使。庚午,以權知北京軍府事、汾州刺史符彥超為晉州留後,以前陳州刺史劉伸殷為陜州留後。癸酉,以前磁州刺史劉彥琮為同州留後。甲戌,福州節度使、檢校太傅王延翰加檢校太尉、同平章事。


乙亥,翰林學士、戶部侍郎、知制誥馮道,翰林學士、中書舍人趙鳳,俱以本官充端明殿學士。端明之職,自此始也。
 \gezhu{
  《五代會要》:明宗初登位,四方書奏,多令樞密使安重誨讀之,不曉文
  義。于是孔循獻議,因唐室侍讀之號,即創端明學士之名,命馮道等為之。}
 丙子,詔:「故西道行營都招討制置等使、守侍中、監修國史、兼樞密使郭崇韜宜許歸葬,其世業田宅並還與骨肉。故萬州司戶朱友謙可復護國軍節度使、守太師、兼尚書令、河中尹、西平王,所有田宅財產,並還與骨肉。」丁丑,西都衙內指揮使張篯進納偽蜀主王衍犀玉帶各二條、馬一百五十匹。初,莊宗遣中官向延嗣就長安之殺王衍也,旋屬蕭牆之禍,延嗣藏竄,不知所之,而衍之資裝妓樂並為
 篯所有,復懼事泄,故聊有此獻。



 戊寅,以樞密使安重誨兼領襄州節度使。制下,重誨之黨謂重誨曰:「襄州地控要津,不可乏帥,無宜兼領。」重誨即自陳退,許之。以左金吾大將軍張遵誨為西京副留守、知留守事。辛巳,以衛尉卿李懌為中書舍人,充翰林學士。壬午,以前蔚州刺史張溫為振武留後,以左右廂突陣指揮使康義誠為汾州刺史,以左右廂馬軍都指揮使索自通為忻州刺史。尚父、吳越國王錢鏐遣使進金器五百兩、銀萬兩,綾
 萬匹謝恩,賜玉冊、金印。初,同光季年,鏐上疏密求玉冊、金印,郭崇韜進議以為不可,而樞密承旨段徊受其重賂,贊成其事,莊宗即允其請,至是故有貢謝。甲申,幽州節度使、檢校太保李紹斌加檢校太傅、同平章事,復姓名為趙德鈞。乙酉,詔百官朔望入閣,賜廊下食。自亂離已前,常參官每日朝退賜食于廊下,謂之「廊餐」。乾符之後,百司經費不足,無每日之賜,至是遇入閣即賜之。《五代會要》:明宗初即位,命百官五日一起居,李琪以為非故事,請罷之。惟每月朔望日合入閣賜食。至是宣旨,
 朔望入閣外,仍五日一起居,遂為定式。



 六月戊子,前襄州節度使李紹珙起復,依前襄州節度使,仍復本姓名曰劉訓。以皇子河中留後從珂為河中節度使,百僚表賀。以翰林承旨、兵部尚書、知制誥盧質為檢校司空荒,皆天也;法制與悖亂,皆人也。二之而已,其事各行不,充同州節度使。己丑,以使部尚書、判太常卿事李琪為御史大夫;以禮部尚書崔協為太常卿、判吏部尚書銓事;以御史中丞崔居儉為兵部侍郎;以太子賓客蕭頃為禮部尚書。中書奏:「請以九月九日皇帝降誕日為應聖節,休假三日。」從之。
 故忠武軍節度使、檢校太師、兼尚書令、齊王張全義贈太師,以前尚書右丞崔沂為尚書左丞。丙申,新州留後張庭裕、雲州留後高行珪並正授本軍節度使。丁酉,詔曰:「四夷來王,歷代故事,前後各因強弱,撫制互有典儀。大蕃須示于威容,即于正衙引對;小蕃但推于恩澤,仍于便殿撫懷。憲府奏論,禮院詳酌,皆徵故實,咸有明文。正衙威容,未可全廢;內殿恩澤,且可常行。若遇大蕃入朝,即準舊儀,于正殿排比鋪陳立仗,百官排班,于正門
 引入對見。」時百僚入閣班退後,卻引對朝貢蕃客,御史大夫李琪奏論之,下禮院檢討,而降是命焉。



 戊戌,樞密使安重誨加檢校太保,行兵部尚書事如故。以太子詹事劉岳為兵部侍郎,以太子右庶子王權為戶部侍郎,以太子左庶子任贊為工部侍郎。庚子《明卦適變通爻》、《明象》、《辨位》、《略例下》、《卦略》等,荊南節度使、檢校太師、兼尚書令、南平王高季興加守太尉、兼尚書令,澤潞節度使、檢校太傅、同平章事孔勍加兼侍中。汴州屯駐控鶴指揮使張諫等謀叛伏誅,以樞密使孔循權
 知汴州軍州事。甲辰,樞密使孔循加檢校太保、守秘書監,依前充使。己巳,祕書少監姚顗為左散騎常侍,以太子左諭德陸崇為右散騎常侍,以兵部郎中蕭希甫為左諫議大夫,前幽州節度判官呂夢奇為右諫議大夫,以鄴都副留守孫岳為潁州團練使。詔曰:「古者酌禮以制名,懼廢于物;取其難犯而易避,貴便於時。況『征』『在』二名,抑有前例。以太宗文皇帝自登寶位,不改舊稱,時即臣有世南,官有民部,靡聞曲避,止禁連呼。朕猥以眇躬,
 託于人上,止遵聖範,非敢自尊。應文書內所有二字,但不連稱,不得迴避。如是臣下之名,不欲與君親同字者,任自改更。」丁未,中書門下奏:「京城潛龍舊宅,望以至德宮為名。」從之。



 戊申,夏州節度使、開府儀同三司、檢校太師、兼中書令、朔方王李仁福加食邑一千戶。以延州留後高允韜為延州節度使,以利州節度觀察留後張敬詢為利州節度使。劍南西川節度副大使、知節度事孟知祥加檢校太傅、兼侍中,劍南東川節度副大使、知節
 度事董璋加檢校太傅。壬子,鳳翔節度使、檢校太尉、兼中書令李從嚴加檢校太師、兼中書令。汴州知州孔循奏,召集謀亂指揮使趙虔已下三千人並族誅訖。甲寅,以晉州留後符彥超為北京留守,以鎮州副使王建立為鎮州留後,以右龍武統軍安崇阮為晉州留後。荊南節度使高季興上言「夔、忠、萬三州,舊是當道屬郡,先被西川侵據,今乞卻割隸本管。」詔可之。其夔州,偽蜀先曾建節,宜依舊除刺史。《通鑒考異》引《十國紀年荊南史》:天成元年二月壬辰,請忠、夔、萬州及雲安
 監隸本道,莊宗許之。詔命未下,莊宗遇弒。六月壬辰,王表求三州,明宗許之。



 秋七月乙卯朔,以太原舊宅為積慶宮。庚申,契丹、渤海國俱遣使朝貢。甲子,詔割韓城、郃陽兩縣屬同州。誅滑州左右崇牙及長劍等軍士數百人,夷其族,作亂故也。其都校于可洪等相次到闕,亦斬于都市。丁卯,以偽蜀守司空、門下侍郎、平章事、晉國公王諧為檢校司空、守陵州刺史,以虢州刺史石潭為耀州團練使。辛未,詔:「諸道節度、刺史、文武將吏,舊進月旦起居表,今後除節度、留後、團
 練、防禦使,惟正、至進賀表,其四孟月並且止絕。」甲戌,中書門下上言:「宣旨令進納新授諸道判官、州縣官官告敕牒,只應宣賜。準往例,除將相外,並不賜官告,即因梁氏起例,凡宣授官,並特恩賜。臣等商量,自兩使判官令錄在京除授者,即于內殿謝恩,便辭赴任,不更進納官告,判司主簿,不合更許朝對。敕下後,望準舊例處分。」從之。



 乙亥,莊宗皇帝梓宮發引,帝縗服臨送于樓前。是日,葬莊宗于雍陵。鎮州留後王建立奏,涿州刺史劉殷肇
 不受代,謀叛,昨發兵收掩,擒劉殷肇及其黨一十三人,見折足勘詰。己卯,以比部郎中、知制誥楊凝式為給事中,充史館修撰、判館事;以偽蜀吏部尚書楊玢為給事中,充集賢殿學士、判院事。升應州為彰德軍節度,仍以興唐軍為寰州,隸彰德軍。宰相豆盧革貶辰州刺史,韋說貶漵州刺史,仍令所在馳驛發遣,為諫議大夫蕭希甫疏奏故也。制略曰:「革則縱田客以殺人,說則侵鄰家而奪井,選元亨之上第,改王參之本名。或主掌三司,委
 元隨之務局;或陶熔百里,愛長吏之桑田。咸屈塞于平人,互阿私於愛子。任官匪當,黷貨無厭,謀人之國若斯,致主之方安在!既迷理亂,又昧卷舒。而府司案牘爰來,諫署奏章疊至,備彰醜跡,深污明庭。是宜約以三章,投之四裔。其河南府文案及蕭希甫論疏,並宜宣示百僚。」庚辰,賜蕭希甫衣段二十匹、銀器五十兩,賞疏革、說之罪也。宰相鄭玨、任圜再見安重誨,求解革、說,請不復追行後命,又三上表救解,俱留中不報。



 辛巳,以捧聖嚴衛
 左廂馬步軍都指揮使李從璋領饒州刺史,充大內皇城使。中書門下奏:「條制,檢校官各納尚書省禮錢,舊例太師、太尉納四十千,後減落至二十千;太傅、太保元納三十千,減至十五千;司徒、司空元納二十千,減至一十千;僕射、尚書元納一十五千,減至七千;員外、郎中元納一十千,今納三千四百者。」詔曰:「會府華資,皇朝寵秩,凡霑新命,各納禮錢。爰自近年,多隳舊制,遂致紀綱之地,遽成廢墜之司。況累條流,就從減省,方當提舉,宜振規
 繩。但緣其間,翊衛勳庸,籓宣將佐,自軍功而遷陟,示恩澤以獎酬,須議從權,不在其例。其餘自不帶平章事節度使及防禦、團練、刺史、使府副使、行軍已下,三司職掌監務官,州縣官,凡關此例,並可徵納。其檢校官自員外郎至僕射,只初轉一任納錢,若不改呼,不在徵納。仍委尚書省部司專切檢舉,置歷逐月具數申中書門下。」



 癸未,詔辰州刺史豆盧革可責授費州司戶參軍,漵州刺史韋說可責授夷州司戶參軍,皆員外置同正員,仍令
 馳驛發遣。甲申之,不必用」。後世儒家多重義輕利,唯北宋李覯,南宋葉適、,又詔曰:「責授費州司戶參軍豆盧革、夷州司戶參軍韋說等,自居台輔,累換歲華,負先皇倚注之恩,失大國燮調之理。朕自登宸極,常委鈞衡,略無謙遜之辭,但縱貪饕之意。除官受賂,樹黨徇私,每虧敬于朕前,徒自尊于人上。道路之喧騰不已,諫臣之條疏頗多,罪狀顯彰,典刑斯舉,合從極法,以塞群情。尚緣臨御之初,含弘是務,特軫墜泉之慮,爰施解網之仁,曲示優恩,俯寬後命。革可陵州長流百姓,說可合州長流百姓,
 仍委逐處長知所在。同州長春宮判官、朝請大夫、檢校尚書、禮部郎中、賜紫金魚袋豆盧升,將仕郎、守尚書屯田員外郎、崇文館學士、賜緋魚袋韋濤等,各因權勢,驟列班行,無才業以可稱,竊寵榮而斯久。比行貶謫,以塞尤違。朕以纂襲之初,含容是務,父既寬於後命,子宜示於特恩,並停見任。」升、濤即革、說之子也。



\end{pinyinscope}