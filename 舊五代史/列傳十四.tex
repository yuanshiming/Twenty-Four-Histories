\article{列傳十四}

\begin{pinyinscope}
孟方立,
 \gezhu{
  《歐陽史》云邢州人,《通鑒》云汧州人。}
 中和二年,為澤州天井關戍將。時黃巢犯關輔,州郡易帥,有同博奕。先是,沈詢、高湜相繼為昭義節度,怠於軍政。及有歸秦、劉廣之亂,方立
 見潞帥交代之際,乘其無備,率戍扶徑入潞州,自稱留後。以邢為府,以審誨知潞州。
 \gezhu{
  案:此二句上下有脫文。今無可復考。}
 六月,李存孝下洺、磁兩郡,方立遣馬溉、袁奉韜盡率其眾,逆戰於琉璃陂。存孝擊之盡殪,生獲馬溉、奉韜。初,方立性苛急,恩不逮下,攻圍累旬,夜自巡城慰諭,守陴者皆倨。方立知其不可用,乃飲鴆而卒。


其從弟洺州刺史遷,素得士心,眾乃推為留後,求援於汴。時梁祖方攻時溥,援兵不出。
 \gezhu{
  《通鑒》云:全忠命大將王虔裕將精甲數百,間道入邢州共守。}
 大順元年,遷執王
 虔裕等乞降,武皇令安金俊代之。
 \gezhu{
  案《孟方立傳》,原本闕佚。考《新唐書》列傳云:孟方立,邢州人。始為澤州天井戍將,稍遷游奕使。中和元年,昭義節度使高鄩擊黃巢,戰石橋,不勝,保華州,為裨將成鄰所殺。還據潞州,眾怒,方立率兵攻鄰,斬之。自稱留後,擅裂邢、洺、磁為鎮,治邢為府,號昭義軍。潞人請監軍使吳全勖知兵馬留後。時王鐸領諸道行營都統,以潞未定,墨制假方立檢校左散騎常侍、兼御史大夫,知邢州事。方立不受,囚全勖,以書請鐸,願得儒臣守潞。鐸使參謀、中書舍人鄭昌圖知昭義留事,欲遂為帥。僖宗自用舊宰相王徽領節度。時天子在西,河、關雲擾,方立擅地,而李克用窺潞州,徽度朝廷未能制,乃固讓昌圖。昌圖治不三月輒去。方立更表李殷銳為刺史,謂潞險而人悍,數賊大帥為亂,欲銷懦之,乃徙治龍岡州,豪傑重遷,有懟言。會克用為河東節度使,昭義監軍祁審誨乞師求復昭義軍;克用遣賀公雅、李筠、安金俊三部
  將擊潞州,為方立所破。又使李克修攻敗之,殺殷銳,遂并潞州,表克修為節度留後。初,昭義有潞、邢、洺、磁四州。至是,方立自以山東三州為昭義,朝廷亦命克修,以潞州舊軍畀之,昭義有兩節自此始。克修,字崇遠,克用從父弟。精馳射,常從征伐,自左營軍使擢留後,進檢校司空。方立倚朱全忠為助,故無用擊邢、洺、磁無虛歲,地為鬥場,人不能稼。光啟二年,克修擊邢州,取故鎮,進攻武安,方立將呂臻、馬爽戰焦岡,為克修所破,斬首萬級,執臻等,拔武安、臨洺、邯鄲、沙河。克用以安金俊為邢州刺史招撫之。方立丐兵于王鎔,鎔以兵三萬赴之,克修還。後二年,方立督部將奚忠信兵三萬攻遼州,以金啖赫連鐸與連和。會契丹攻鐸師失期,忠信三分其兵,鼓而行,克用伏兵于險,忠信前軍沒,既戰,大敗,執忠信,餘眾走脫,歸者纔十二。龍紀元年,克用使李罕之、李存孝擊邢,攻磁、洺,方立戰琉璃陂,大敗,禽其二將,被斧金質,徇邢
  壘,呼曰:「孟公速降,有能斬其首者,假三州節度使。」方立力屈,又屬州殘墮,人心恐,性剛急,待下少恩,夜自行陴,兵皆倨告勞,自顧不可復振,乃還,引鴆自殺。從弟遷,素得士心,眾推為節度留後,請援于全忠。全忠方攻時溥,不即至,命王虔裕以精甲數百赴之,假道羅宏信,不許,乃趨間入邢州。大順元年,存孝復攻邢,遷挈邢、洺、磁三州降,執王虔裕三百人獻之;遂遷太原,表安金俊為邢、洺、磁團練使,以遷為汾州刺史。《歐陽史》云:天復元年,梁遣氏叔琮攻晉,出天井關,遷開門降,為梁兵鄉道以攻太原,不克;叔琮軍還過潞,以遷歸于梁。梁太祖惡其反覆,殺之。}



 張文禮,燕人也。初為劉仁恭裨將,性兇險,多姦謀,辭氣庸下,與人交言,癖于不遜,自少及長,專蓄異謀。及從劉守文之滄州,委將偏師。守文省父燕薊,據城為亂。及敗,
 奔于王鎔。察鎔不親政事,遂曲事當權者,以求衒達。每對鎔自言有將才,孫、吳、韓、白,莫己若也。鎔賞其言,給遺甚厚,因錄為義男,賜姓,名德明,由是每令將兵。自柏鄉戰勝之後,常從莊宗行營。素不知書,亦無方略,惟于懦兵之中萋菲上將,言甲不知進退,乙不識軍機,以此軍人推為良將。



 初,梁將楊師厚在魏州,文禮領趙兵三萬夜掠經、宗,因侵貝郡。師厚先率步騎數千人,設伏于唐店。文禮大掠而旋,士皆卷甲束兵,夜凱歌,行至唐店,師
 厚伏兵四面圍合,殺戮殆盡,文禮單騎僅免。自爾猶對諸將大言,或讓之曰:「唐店之功,不須多伐。」文禮大慚。在鎮州既久,見其政荒人僻,常蓄異圖;酒酣之後,對左右每泄惡言,聞者莫不寒心。惟王鎔略無猜間,漸為腹心,乃以符習代其行營,以文禮為防城使,自此專伺間隙。及鎔殺李宏規,委政于其子昭祚。昭祚性逼戾,未識人間情偽,素養名持重,坐作貴人,既事權在手,朝夕欲代其父,向來附勢之徒,無不族滅。



 初,李宏規、李藹持權用事,樹
 立親舊,分董要職,故奸宄之心不能搖動,文禮頗深畏憚。及宏規見殺,其部下五百人懼罪,將欲奔竄,聚泣偶語,未有所之。文禮因其離心,密以奸辭激之曰:「令公命我盡坑爾曹,我念爾十餘年荷戈隨我,為家為國,我若不即殺爾,則得罪于令公;我若不言,又負爾輩。」眾軍皆泣。是夜作亂,殺王鎔父子,舉族灰滅,惟留王昭祚妻朱氏通梁人;尋間道告于梁曰:「王氏喪于亂軍,普寧公主無恙。」文禮徇賊帥張友順所請,因為留後,于潭城視事。
 以事上聞,兼要節旄,尋亦奉箋勸進,莊宗姑示含容,乃可其請。



 文禮比廝役小人,驟居人上,行步動息,皆不自安。出則千餘人露刃相隨,日殺不辜,道路以目,常慮我師問罪,姦心百端。南通朱氏,北結契丹,往往擒獲其使,莊宗遣人送還,文禮由是愈恐。是歲八月,莊宗遣閻寶、史建瑭及趙將符習等率王鎔本軍進討。師興,文禮病疽腹,及聞史建瑭攻下趙州,驚悸而卒。其子處瑾、處球秘不發喪,軍府內外,皆不知之,每日于寢宮問安。處瑾
 與其腹心韓正時參決大事,同謀姦惡。初,文禮疽未發時,舉家咸見鬼物,昏瞑之後或歌或哭,又野河色變如血,游魚多死,浮于水上,識者知其必敗。



 十九年三月,閻寶為處瑾所敗,莊宗以李嗣昭代之。四月,嗣昭為流矢所中,尋卒于師,命李存進繼之。存進亦以戰歿,乃以符存審為北面招討使,攻鎮州。是時,處瑾危蹙日甚。昭義軍節度判官任圜馳至城下,諭以禍福,處瑾登陴以誠告,乃遣牙將張彭送款于行臺。俄而符存審師至城下。
 是夜,趙將李再豐之子沖投縋以接王師,故諸軍登城,遲明畢入,獲處瑾、處球、處琪,并其母及同惡人等,皆折足送行臺,鎮人請醢而食之。又發文禮之尸,磔之于市。



 董璋,本梁之驍將也。幼與高季興、孔循俱事豪士李七郎為童僕。李初名讓,常以厚賄奉梁祖,梁祖寵之,因畜為假子,賜姓朱,名友讓。璋既壯,得隸于梁祖帳下,後以軍功遷為列校。梁龍德末,潞州李繼韜送款于梁。時潞將裴約方領兵戍澤州,不徇繼韜之命,據城以自固。梁
 末帝遣璋攻陷澤州,遂授澤州刺史。是歲,莊宗入汴,璋來朝,莊宗素聞其名,優以待之。尋令卻赴舊任,歲餘代歸。時郭崇韜當國,待璋尤厚。同光三年夏,命為邠州留後,三年秋,正授旄鉞。九月,大舉伐蜀,以璋為行營右廂馬步都虞候。時郭崇韜為招討使,凡有軍機,皆召璋參決。是冬,蜀平,以璋為劍南東川節度副大使,知節度事。天成初,加檢校太傅。二年,加同平章事。



 是時安重誨當國,採人邪謀,言孟知祥必不為國家使,惟董璋性忠義,
 可特寵任,令圖知祥。又璋之子光業為宮苑使,在朝結託勢援,爭言璋之善,知祥之惡。恩寵既優,故璋益恣其暴戾。初,奉使東川者,皆言璋不恭于朝廷。四年夏,時明宗將議郊天,遣客省使李仁矩齎詔示諭兩川,又遣安重誨馳書于璋,以徵貢奉,約以五十萬為數。既而璋訴以地狹民貧,許貢十萬而已。翌日,璋於衙署設宴以召仁矩,日既中而不至,璋使人偵之,仁矩方擁倡婦與賓友酣飲于驛亭。璋大怒,遽領數百人,執持戈戟,驟入驛
 中,令洞開其門。仁矩惶駭,走入閣中,良久引出。璋坐,立仁矩于階下,戟手罵曰:「當我作魏博都監,爾為通引小將,其時去就,已有等威。今日我為籓侯,爾銜君命,宿張筵席,比為使臣,保敢至午不來,自共風塵耽酗,豈于王事如此不恭!只如西川解斬客省使李嚴,謂我不能斬公耶!」因目肘腋,欲令執拽仁矩,仁矩涕淚拜告,僅而獲免。璋乃馳騎入衙,竟徹饌而不召。洎仁矩復命,益言璋不法。未幾,重誨奏以仁矩為閬州團練使,尋升為節鎮。



 長興元年夏,明宗以郊禋禮畢,加璋檢校太尉。時兩川刺史嘗以兵為牙軍,小郡不下五百人,璋已疑間,及聞除仁矩鎮閬州,璋由是謀反乃決。仍先與其子光業書曰:「朝廷割吾支郡為節制,屯兵三千,是殺我必矣。爾見樞要道吾言,如朝廷更發一騎入斜谷,則吾必反,與汝決矣!」光業以書呈樞密承旨李虔徽。會朝廷再發中使荀咸乂將兵赴閬州,光業謂虔徽曰:「咸乂未至,吾父必反。吾身不足惜,慮勞朝廷徵發。請停咸乂之行,吾父必
 保常日。」重誨不從,咸乂未至,璋已擅追綿州刺史武虔裕,囚于衙署。虔裕,安重誨之心腹也,故先囚之。五月,璋傳檄于利、閬、遂等州,責以間諜朝廷。尋率其兵陷閬州,擒節度使李仁矩、軍校姚洪等害之。先是,璋欲謀叛,先遣使持厚幣于孟知祥,求為婚家。且言為朝廷猜忌,將有替移,去則喪家,住亦致討,地狹兵少,獨力不任,願以小兒結婚愛女。時知祥亦貳于朝廷,因許以為援。既而知祥出師以圍遂州,故璋攻閬州得恣其毒焉。



 其年秋,
 詔削奪璋在身官爵,命天雄軍節度使石敬瑭為東川行營招討使,率師以討之。璋之子宮苑使光業并其族,並斬于洛陽。及石敬瑭率師進討,以糧運不接,班師。明宗方務懷柔,乃放西川進奏官蘇願、東川軍將劉澄各歸本道,別無詔旨,只云「兩務求安」。時孟知祥其骨肉在京師者俱無恙焉,因遣使報璋,欲連表稱謝。璋怒曰:「西川存得弟侄,遂欲再通朝廷,璋之兒孫已入黃泉,何謝之有!」自是璋疑知祥背己,始構隙矣。三年四月,璋率所
 部兵萬餘人以襲知祥。《九國志·趙季良傳》:季良嘗與知祥從容語曰:「璋性狼戾,若堅守一城,攻之難克。」及聞璋起兵,知祥憂形於色。季良曰;「璋不守巢穴,此天以授公也。」既而璋果敗。知祥與諸將率師拒之,戰於漢州之彌牟鎮。璋軍大敗,得數十騎,復奔於東川。《九國志·趙廷隱傳》:董璋襲廣漢,將攻成都,時東川廩藏充實,部下多敢死之士,其來也,眾皆畏之。知祥親督諸將,與璋戰雞縱橋前,頗為所挫。廷隱偽遁,璋逐之,知祥與張公鐸繼進,璋軍亂不成列,廷隱整陣,與知祥合擊之,璋軍大敗。先是,前陵州刺史王暉為璋所邀,寓於東川,至是因璋之敗,率眾以害之,傳其首於西
 川。



\end{pinyinscope}