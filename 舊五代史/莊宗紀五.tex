\article{莊宗紀五}

\begin{pinyinscope}
同光二年春正月庚子朔,帝御明堂殿受朝賀,仗衛如式。壬寅,南郊禮儀使、太常卿李燕進太廟登歌酌獻樂舞名,懿祖室曰《昭德之舞》,獻祖室曰《文明之舞》,太祖室
 曰《應天之舞》,昭宗室曰《永平之舞》。甲辰,幽州上言,契丹入寇至瓦橋。
 \gezhu{
  《契丹國志》:時契丹日益強盛,遣使就唐求幽州,以處盧文進。}
 以天平軍節度使李嗣源為北面行營都招討使,陜州留後霍彥威為副,率軍援幽州。己巳,故宣武軍節度副使、權知軍州事、檢校太傅王瓚贈太子太師。丁未,詔改朝元殿復為明堂殿,又改崇勛殿為中興殿。戊申,以振武軍節度使、檢校太傅、同平章事李存霸權知潞州留後;以知保大軍軍州事高允韜為檢校太保。庚戌,以涇原節度使、
 充秦王府諸道行軍司馬、開府儀同三司、檢校太尉、兼侍中李嚴為檢校太尉、兼中書令,依前涇原軍節度使,充秦王府諸道行軍司馬。詔改應順門為永曜門,太平門為萬春門,通政門為廣政門,鳳明門為韶和門,萬春門為中興門,解卸殿為端明殿。



 是日,詔曰:「皇綱已正,紫禁方嚴,凡事內官,不合更居外地。詔諸道應有內官,不計高低,並仰逐處并家口發遣赴闕,不得輒有停滯。」帝龍潛時,寺人數已及五百,至是合諸道赴闕者,約千餘
 人,皆給賜優贍,服玩華侈,委之事務,付以腹心。唐時宦官為內諸司使務、諸鎮監軍,出納王命,造作威福,昭宗以此亡國。及帝奄有天下,當知戒彼前車,以為殷鑒,一朝復興茲弊,議者惜之。新羅王金朴英遣使朝貢。辛亥,中書門下奏:「準本朝故事,諸王、內命婦、宰臣、學士、中書舍人、諸道節度、防禦、團練使、留後官告,即中書帖官告院索綾紙褾軸,下所司書寫印署畢,進入宣賜。其文武兩班及諸道官員並奏薦將校,並合于所司送納朱膠
 綾紙價錢。伏自偽梁,不分輕重,並從官給,今後如非前件事例,請官中不給告敕,其內司大官侍衛將校轉官,即不在此限。」從之。壬子,蜀主王衍致書于帝,稱有詐為天使,馳報收復汴州者,詔捕之,不獲。癸丑,有司奏:郊祀前二日,迎祔高祖、太宗、懿祖、獻祖、太祖神主于太廟。議者以中興唐祚,不宜以追封之祖雜有國之君以為昭穆,自懿祖已下,宜別立廟于代州,如後漢南陽故事可也。幽州北面軍前奏,契丹還塞,詔李嗣源班師。鳳翔節
 度使、秦王李茂貞上表,請行籓臣之禮,帝優報之。甲寅,帝于中興殿面賜郭崇韜鐵券。有司上言:「皇太后到闕,皇帝合于銀臺門內奉迎。」詔親至懷州奉迎。中書奏:「自二十三日後散齋內,車駕不合遠出。」詔改至河陽奉迎。以禮部尚書、興唐尹王正言依前禮部尚書,充租庸使。



 乙卯,渤海國遣使貢方物。幽州奏,媯州山後十三寨百姓卻復新州。戊午,以前太子少師薛廷珪為檢校戶部尚書、太子少師致仕;以前太子賓客封舜卿為太子少
 保致仕;以前太子賓客李文規為戶部侍郎致仕。詔鹽鐵、度支、戶部並委租庸使管轄。庚申,四方館上言:「請今後除隨駕將校,及外方進奉專使文武班三品以上官,可以內殿對見,其餘並詣正衙,以申常禮。」從之。車駕幸河陽,奉迎皇太后。辛酉,帝侍皇太后至,文武百僚迎于上東門。是日,河中府上言,稷山縣割隸絳州。以太僕卿李紓為宗正卿,以衛尉卿楊遘為太僕卿。西京昭應縣華清宮道士張沖虛上言,天尊院枯檜重生枝葉。乙
 丑,有司上言:「南郊朝享太廟,舊例親王充亞獻、終獻行事。」乃以皇子繼岌為亞獻,皇弟存紀為終獻。丙寅,帝赴明堂殿致齋。丁卯,朝饗于太微宮。戊辰,饗太廟,是日赴南郊。



 二月己巳朔,親祀昊天上帝于圜丘,禮畢,宰臣率百官就次稱賀,還御五鳳樓。宣制:「大赦天下,應同光二年二月一日昧爽已前,所犯罪無輕重常赦所不原者,咸赦除之。十惡五逆、屠牛鑄錢、故意殺人、合造毒藥、持杖行劫、官典犯贓,不在此限。應自來立功將校,各與轉官,仍
 加賞給。文武常參官、節度、觀察、防禦、刺史、軍主、都虞候、指揮使,父母亡歿者,並與追贈;在者各與加爵增封。諸籓鎮各賜一子出身,仍封功臣名號。留後、刺史,官高者加階爵一級,官卑者加官一資。應本朝內外臣僚,被朱氏殺害者,特與追贈。應諸州府不得令富室分外收貯見錢,禁工人熔錢為銅器,勿令商人載錢出境。近年已來,婦女服飾,異常寬博,倍費縑綾。有力之家,不計卑賤,悉衣錦繡,宜令所在糾察。應有百姓婦女,曾經俘擄他
 處為婢妾者,一任骨肉識認。男子曾被刺面者,給與憑據,放逐營生。召天下有能以書籍進納者,各等第酬獎。仰有司速檢勘天下戶口正額,墾田實數,待憑條理,以息煩苛。」是日,風景和暢,人胥悅服。議者云,五十年來無此盛禮。然自此權臣愎戾,伶官用事,吏人孔謙酷加賦斂,赦文之所原放,謙復刻剝不行,大失人心,始于此矣。



 庚午,租庸使孔謙奏:「諸道綱運客旅,多于私路茍免商稅,請令所在關防嚴加捉搦。」從之。癸酉,宰臣豆盧革率
 百官上尊號曰昭文睿武至德光孝皇帝,凡三上表,從之。甲戌,詔曰:「汴州元管開封、浚儀、封丘、雍丘、尉氏、陳留六縣,偽庭割許州鄢陵、扶溝,陳州太康,鄭州陽武、中牟,曹州考城等縣屬焉。其陽武、匡城、扶溝、考城四縣,宜令且隸汴州,餘還本部。」丙子,以隨駕參謀耿瑗為司天監。丁丑,以光祿大夫、檢校司徒李筠為右騎衛上將軍。


戊寅,幸李嗣源第,作樂,盡歡而罷。己卯,以河中節度使、冀王李繼麟兼安邑、解縣兩池榷鹽使。辛巳,以檢校太師、
 守尚書令、河南尹、判六軍諸衛事、魏王張全義為守太尉、兼中書令、河陽節度使、河南尹,改封齊王。以開府儀同三司、守尚書令、秦王李茂貞依前封秦王,餘如故,仍賜不拜、不名。
 \gezhu{
  《五代會要》:太常禮院奏:「李茂貞封冊之命,宜準故襄州節度使趙匡凝之例施行。秦王受冊,自備革輅一乘,載冊犢車一乘,並本品鹵簿鼓吹如儀。」從之。}
 是日,帝幸左龍武軍。癸未,宰臣豆盧革率百官上表,請立中宮。制以魏國夫人劉氏為皇后,仍令所司擇日備禮冊命。



 丁亥,以天平軍節度使、蕃漢總管副使、開府儀同三司、檢校太傅、兼
 中書令李嗣源為檢校太尉,依前天平軍節度使,加實封百戶,兼賜鐵券;以前安國軍節度副使、檢校太保、左衛上將軍李存乂為晉州節度使、檢校太傅;以北京皇城留守、檢校太保、左威衛上將軍李存紀為邢州節度使,加檢校太傅;以蕃漢馬步都虞候兼東京馬步軍都指揮使、檢校太保朱守殷為振武節度使,加檢校太傅。戊子,以前右龍武軍都虞候、守左龍武大將軍李紹奇為鄭州防禦使,以楚州防禦使張繼孫為汝州防禦使。
 己丑,以振武軍節度使、權安義留後、檢校太傅、平章事李存霸為潞州節度使,以捧日都指揮使、鄭州防禦使李紹琛為陜州節度使,以成德軍馬步軍都指揮使、右監門衛大將軍毛璋為華州節度使。壬辰,樞密使郭崇韜再上表,請退樞密之職,優詔不允。



 癸巳,詔曰:「皇太后母儀天下,子視群生,當別建宮闈,顯標名號,冀因稱謂,益表尊嚴,宜以長壽宮為名。」樞密使郭崇韜奏時務利便一十五件,優詔褒美。甲午,奚王李紹威、吐渾李紹魯
 皆貢駝馬。丁酉,以武安軍衙內馬步軍都指揮使、昭州刺史馬希範為永州刺史、檢校太保。癸卯,以光祿大夫、檢校左僕射、行太常卿李燕為特進、檢校司空,依前太常卿;以御史中丞李德休為兵部侍郎;以吏部侍郎崔協為御史中丞。



 三月甲辰,故河陽節度使王師範贈太尉。乙巳,以滄州節度使、檢校太傅、同平章事符習為青州節度使,以北京衙內馬步軍都指揮使、右領軍衛大將軍李紹斌為滄州節度使。鎮州奏,契丹犯塞,詔李嗣
 源率師屯邢州。丙午,以荊南節度使、守中書令、渤海王高季興依前檢校太師、兼尚書令,封南平王;以幽州節度行軍司馬李存賢依前檢校太保,為幽州節度使。中書門下上言:「近以諸州奏薦令錄,頗亂規程,請今後節度使管三州已上,每年許奏管內官三人;如管三州已下,只奏兩人。仍須課績尤異,方得上聞。防禦使止許奏一人,刺史無奏薦之例。」從之。己酉,以太子少保李琪為刑部尚書


庚戌,幽州奏,契丹寇新城。是日,詔:「諸軍將校,
 自檢校司空已下,宜賜葉謀定亂匡國功臣。自檢校僕射、尚書、常侍及諫議大夫,並賜忠果拱衛功臣。初帶憲銜者,並賜忠烈功臣。節級長行,並賜扈蹕功臣。」中書門下上言:「州縣官在任考滿,即具關申送吏部格式,本道不得差攝官替正官。」從之。
 \gezhu{
  《五代會要》:同光二年,中書門下奏:「刺史、縣令有政績尤異,為眾所知;或招復戶口,能增加賦稅者;或辨雪冤獄,能拯人命者;或去害物之積弊,立利世之新規,有益時政,為眾所推者,即仰本處逐件分明聞奏,當議獎擢。或在任貪猥,誅戮生靈,公事不治,為政怠惰,亦加懲罰。其州縣官任滿三考,即具關申送吏部格式,候敕除銓注,其本道不得差攝官替正授者。」從之。}
 有司上言:「
 皇帝四月一日御文明殿,受冊徽號,合服袞冕,御殿前一日,散齋于內殿。」從之。是日,李嗣源上表乞退兵權,詔不允。是時伶人景進用事,閹官競進,故重臣憂懼,拜章請退。癸丑,左諫議大夫竇專上言:「請廢租庸使名目,事歸三司。」疏奏不報。唐州奏,木連理。詔:「先省員官,除已別授官外,其左散騎常侍李文矩等三十人卻復舊官,太子詹事石戩等五人宜以本官致仕,將作少監岑保嗣等十四人續敕處分。」丙辰,責授萊州司戶鄭玨等一十
 一人並量移近地。尚書戶部侍郎、知貢舉趙頎卒,以中書舍人裴皞權知貢舉。禁用鉛錫錢。


丁巳,中書門下奏:「懿祖陵請以永興為名,獻祖陵請以長寧為名,太祖陵請以建極為名。」從之。淮南楊溥遣使貢賀郊天禮物。
 \gezhu{
  《十國春秋·吳世家》:王遣右衛上將軍許確進賀郊天銀二千兩、錦綺羅一千二百匹、細茶五百斤、象牙四株、犀角十株于唐。}
 戊午,詔應南郊行事官,並付三銓磨勘,優與處分。己未,以大理卿張紹珪充制置安邑、解縣兩池榷鹽使。幸左龍武軍,以皇子繼岌代張全義判六軍諸衛事故
 也。癸亥,以彰武、保大等軍節度使、北平王高萬興可依前延州鄜州節度使、檢校太保、兼中書令、北平王。甲子,幸東宅。



 夏四月己巳朔,帝御文明殿,具袞冕,受冊尊號曰昭文睿武至德光孝皇帝。壬申,以成德軍節度行軍司馬、權知府事任圜為檢校右僕射、權北面水陸轉運制置使。甲戌,以順義軍留後華溫琪依前檢校太保,充留後。乙亥,以天策上將軍、武安等軍節度使、守太師、中書令、楚王馬殷可依前守太師,兼尚書令。詔在京諸道
 節度使、刺史、令各歸本任。丁丑,以前幽州節度使、內外蕃漢馬步總管、檢校太師、兼中書令李存審為宣武軍節度使,餘如故。



 己卯,帝御文明殿,冊魏國夫人劉氏為皇后。庚辰,賜霍彥威姓,名曰紹真。癸未,以宋州節度使李繼安依前檢校太尉、同平章事、宋州節度使;以許州節度使李繼沖依前檢校太尉、同平章事、許州節度使;以襄州節度使孔勍依前檢校太傅、同平章事、襄州節度使。甲午,以樞密副使、通議大夫、行內侍省內侍宋唐
 玉為左監門衛將軍同正,依前樞密副使;以內客省使、通議大夫、行內侍省內侍楊希朗為右監門衛將軍同正,依前內客省使:並賜推忠匡佐功臣。車駕幸龍門。丙戌,回鶻遣使貢方物。己丑,以夏州節度使李仁福依前檢校太師、兼中書令、夏州節度使,封朔方王;以朔方、河西等軍節度使韓洙依前檢校太傅、兼侍中,充朔方、河西等軍節度使,靈、鹽、威、警、雄、涼、甘、肅等州觀察使。辛卯,以宣徽南院使、判內侍省、兼內局、特進、左監門將軍同
 正李紹宏為右領軍衛上將軍。癸巳,以靜江軍節度使、扶風郡王馬賓為檢校太師、兼中書令,依前靜江軍節度使;以朗州節度使馬希振為檢校太傅、兼侍中,依前朗州節度使。鳳翔節度使、秦王李茂貞薨。



 丙申,潞州小校楊立據城叛,以李嗣源為招討使,陜州留後李紹真為副,率師以討之。



\end{pinyinscope}