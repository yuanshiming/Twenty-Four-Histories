\article{列傳十八}

\begin{pinyinscope}

 安重誨,其先本北部豪長。父福遷,為河東將,救兗、鄆而沒。重誨自明宗龍潛時得給事左右。及鎮邢州,以重誨為中門使。隨從征討,凡十餘年,委信無間,勤勞亦至。洎
 鄴城之變,佐命之功,獨居其右。明宗踐祚,領樞密使,俄遷左領軍衛大將軍充職。案:以下有闕文。明宗遣回鶻侯三馳傳至其國,侯三至醴泉縣,地素僻,無驛馬,縣令劉知章出獵,不時給馬,侯三遽以聞。明宗大怒,械知章至京師,將殺之;重誨從容為言,乃得不死。明宗幸汴州,重誨建議欲因以伐淮,而明宗難之。後李鏻得淮南諜者言:「徐知誥欲奉其國稱籓,臣願得安公一言以為信。」鏻即引諜者見重誨。重誨大喜,以為然,乃以玉帶與諜者,使遺知誥
 為信,其直千緡。



 重誨為樞密使,四五年間,獨綰大任,臧否自若,環衛、酋長、貴戚、近習,無敢干政者。弟牧鄭州,子鎮懷、孟,身為中令,任過其才,議者謂必有覆餗之禍。無何,有吏人李虔徽弟揚言于眾云:「聞相者言其貴不可言,今將統軍征淮南。」時有軍將密以是聞,頗駭上聽。明宗謂重誨曰:「聞卿樹心腹,私市兵仗,欲自討淮南,有之否?」重誨惶恐,奏曰:「興師命將,出自宸衷,必是奸人結構,臣願陛下窮詰所言者。」翌日,帝召侍衛指揮使安從進、
 藥彥稠等,謂之曰:「有人告安重誨私置兵仗,將不利于社稷,其若之何?」從進等奏曰:「此是奸人結構,離間陛下勳舊。且重誨事陛下三十年,從微至著,無不盡心,今日何苦乃圖不軌!臣等以家屬保明,必無此事。」帝意乃解。重誨三上表乞解機務,詔不允。復面奏:「乞與臣一鎮,以息謗議。」明宗不悅。重誨奏不已,明宗怒,謂曰:「放卿出,朕自有人!」即令武德使孟漢瓊至中書,與宰臣商量重誨事。馮道言曰:「諸人茍惜安令公,解樞務為便。」趙鳳曰:「大
 臣豈可輕動,公失言也。」道等因附漢瓊奏曰:「此斷自宸旨,然重臣不可輕議移改。」由是兼命范延光為樞密使,重誨如故。



 時以東川帥董璋恃險難制,乃以武虔裕為綿州刺史,董璋益懷疑忌,遂縶虔裕以叛。及石敬瑭領王師伐蜀,峽路艱阻,糧運不繼,明宗憂之,而重誨請行。翌日,領數騎而出,日馳數百里,西諸侯聞之,莫不惶駭。所在錢帛糧料,星夜輦運,人乘斃踣于山路者不可勝紀,百姓苦之。重誨至鳳翔,節度使朱宏昭延于寢室,令妻
 子奉食器,敬事尤謹。重誨坐中言及:「昨有人讒構,幾不保全,賴聖上保鑒,茍獲全族。」因泣下。重誨既辭,宏昭遣人具奏:「重誨怨望出惡言,不可令至行營,恐奪石敬瑭兵柄。」而宣徽使孟漢瓊自西回,亦奏重誨過惡。重誨已至三泉,復令歸闕。再過鳳翔,硃宏昭拒而不納,重誨懼,急騎奔程,未至京師,制授河中帥。既至鎮,心不自安,遂請致仕。制初下,其子崇贊、崇緒走歸河中。二子初至,重誨駭然曰;「渠安得來?」家人欲問故,重誨曰:「吾知之矣,此
 非渠意,是他人教來。吾但以一死報國家,餘復何言!」翌日,中使至,見重誨,號泣久之。重誨曰:「公但言其故,勿過相愍。」中使曰:「人言令公據城異志矣!」重誨曰:「吾一死未塞責,已負君親,安敢輒懷異志,遽勞朝廷興師,增聖上宵旰,則僕之罪更萬萬矣!」



 時遣翟光鄴使河中,如察重誨有異志,則誅之。既至,李從璋自率甲士圍其第,仍拜重誨于其庭,重誨下階迎拜曰:「太傅過禮。」俯首方拜,從璋以楇擊其首,其妻驚走抱之,曰:「令公死亦不遲,太傅
 何遽如此!」并擊重誨妻首碎,並剝其衣服,夫妻裸形踣于廊下,血流盈庭。翌日,副使判官白從璋,願以衣服覆其屍,堅請方許。及從璋疏重誨家財不及數千緡,議者以重誨有經綸社稷之大功,然志大才短,不能迴避權寵,親禮士大夫,求周身輔國之遠圖,而悉自恣胸襟,果貽顛覆。《五代史補》:初,知祥將據蜀也,且上表乞般家屬。時樞密使安重誨用事,拒其請,知祥曰:「吾知之矣。」因使密以金百兩為賂,重誨喜而為敷奏,詔許之。及家屬至,知祥對僚吏笑曰;「天下聞知樞密,將謂天地間未有此,誰知只銷此百金耶,亦不足畏也。」遂守險拒命。《五代史闕文》:明宗令翟光鄴、李從璋誅重誨於河中
 私第,從璋奮楇擊重誨於地,重誨曰:「某死無恨,但恨不與官家誅得潞王,他日必為朝廷之患。」言終而絕。臣謹案:《明宗實錄》是清泰帝朝修撰,潞王即清泰帝也。史臣避諱,不敢直書。嗚呼,重誨之志節泯矣!



 硃宏昭,太原人也。祖玟,父叔宗,皆為本府牙將。宏昭事明宗,在籓方為典客。天成元年,為文思使,歷東川副使,二年餘,除左衛大將軍,充內客省使。三年,轉宣徽南院使。明宗親祀南郊,宏昭為大內留守,加檢校太傅。出鎮鳳翔,會朝廷命石敬塘帥師伐蜀,久未成功,安重誨自請西行。至鳳翔,宏昭迎謁馬首,請館于府署,妻子羅拜,
 捧卮為壽。宏昭密遣人謂敬瑭曰;「安公親來勞軍,觀其舉措孟浪,儻令得至,恐士心迎合,則不戰而自潰也。可速拒之,必不敢前,則師徒萬全也。」敬瑭聞其言大懼,即日燒營遁還。重誨聞之,不敢西行,因返旆東還。復過鳳翔,宏昭拒而不納。及重誨得罪,其年宏昭入朝,授左武衛上將軍,充宣徽南院使。長興三年十二月,代康義誠為襄州節度使。四年,秦王從榮為元帥,屢宣惡言,執政大臣皆懼,謀出避之。樞密使範延光、趙延壽日夕更見,
 涕泣求去,明宗怒而不許。延壽使其妻興平公主入言于中,延光亦因孟漢瓊、王淑妃進說,故皆得免。未幾,趙延壽出鎮汴州,召宏昭于襄陽,代為樞密使,加同平章事。十月,范延光出鎮常山,以三司使馮贇與宏昭對掌樞務,與康義誠、孟漢瓊同謀以殺秦王。閔帝即位,宏昭以為由己得立,故于庶事高下在心,及赦後覃恩,宏昭首自平章事超加中書令。素猜忌潞王,致其釁隙,以致禍敗。潞王至陜,閔帝懼,欲奔,馳手詔宏昭圖之。時將
 軍穆延輝在弘昭第,曰:「急召,罪我也,其如之何?吾兒婦,君之女也,可速迎歸,無令受禍。」中使繼至,宏昭援劍大哭,至後庭欲自裁,家人力止之。使促之急,宏昭曰:「窮至此耶!」乃自投于井。安從進既殺馮贇,斷宏昭首,俱傳于陜州。及漢高祖即位,贈尚書令。



 硃洪實,不知何許人。以武勇累歷軍校,長興中,為馬軍都指揮使。秦王為元帥,以洪實驍果,尤寵待之,歲時曲遺,頗厚于諸將。及硃宏昭為樞密使,勢焰尤甚,洪實以
 宗兄事之,意頗相協。宏昭將殺秦王,以謀告之,洪實不以為辭。時康義誠以其子事于秦府,故恒持兩端。及秦王兵扣端門,洪實為孟漢瓊所使,率先領騎軍自左掖門出逐秦王,自是義誠陰銜之。閔帝嗣位,洪實自恃領軍之功,義誠每言,不為之下。應順元年三月辛酉,義誠將出征,閔帝幸左藏庫,親給軍士錢帛。是時,義誠與洪實同于庫中面論用兵利害,《歐陽史》云:洪實見軍士無鬥志,而義誠盡將以西,疑其二心。洪實言:「出軍討逆,累發兵師,今聞小衄,無一人一
 騎來者。不如以禁軍據門自固,彼安敢徑來,然後徐圖進取,全策也。」義誠怒曰:「若如此言,洪實反也。」洪實曰:「公自反,誰反!」其聲漸厲。帝聞,召而訊之,洪實猶理前謀,又曰;「義誠言臣圖反,據發兵計,義誠反必矣。」閔帝不能明辨,遂命誅洪實。既而義誠果以禁軍迎降潞王,故洪實之死,後人皆以為冤。



 康義誠,字信臣,代北三部落人也。少以騎射事武皇,從莊宗入魏博,補突騎使,累遷本軍都指揮使。同光末,從
 明宗討鄴城,軍亂,迫明宗為主,明宗不然。義誠進曰:「主上不慮社稷阽危,不思戰士勞苦,荒耽禽色,溺于酒樂。今從眾則有歸,守節則將死。」明宗納其言,由是委之心膂。明宗即位,加檢校司空,領富州刺史,總突騎如故。尋轉捧聖都指揮使,鎮邠州刺史。明宗幸汴,平朱守殷,改侍衛馬軍都指揮使,領江西節度使。車駕歸洛,授侍衛馬步軍都指揮使、河陽節度使。《太平廣記》云:長興中,侍衛使康義誠,嘗軍中差人于大宅充院子,亦曾小有笞責。忽一日,憐其老而詰其姓氏,則曰:「姓康。」別詰其鄉土、親族、息嗣,方知是父,遂相持
 而泣,聞者莫不驚異。長興末,加同平章事。



 秦王為天下兵馬元帥,氣焰熏灼,大臣皆懼,求為外任。義誠以明宗委遇,無以解退,乃令其子以弓馬事秦王冀自保全。明宗不豫,秦王諷義誠為助,義誠曲意承奉,亦非真誠。及朱宏昭、馮贇等懼禍,謀于義誠,但云:「僕為將校,不敢預議,但相公所使耳。」及秦王既誅,明宗宴駕,閔帝即位,加檢校太尉、兼侍中,判六軍諸衛事。未幾,鳳翔變起,西軍不利,義誠懼,乃請行,蓋欲盡率駕下諸軍送降於潞王求免也。會
 與硃洪實議事不葉,洪實因厲聲言義誠苞藏之志,閔帝曖昧,不能明辨,而誅洪實。及義誠率軍至新安,諸軍爭先趨陜,解甲迎降,義誠以部下數十人見潞王請罪,潞王雖罪其奸回,未欲行法。清泰元年四月,斬于興教門外,夷其族。



 藥彥稠,沙陀三部落人。幼以騎射事明宗,累遷至列校。明宗踐阼,領澄州刺史、河陽馬步都將。從王晏球討王都于定州,平之,領壽州節度使、侍衛步軍都虞候。屬
 河中指揮使楊彥溫作亂,彥稠改侍衛步軍都指揮使,充河中副招討使,將兵討平之。無幾,黨項劫回鶻入朝使,詔彥稠屯朔方,就討黨項之叛命者,搜索盜賊,盡獲回鶻所貢駝馬、寶玉,擒首領而還。尋授邠州節度使。遣會兵制置鹽州,蕃戎逃遁,獲陷蕃士庶千餘人,遣復鄉里。受詔與延州節度使,案:原本闕二字。進攻夏州,累月不克,兵罷歸鎮。閔帝嗣位,與王思同攻鳳翔,為副招討使。禁軍之潰,彥稠欲沿流而遁,為軍士所擒而獻之。時末帝已
 至華州,令拘于獄,誅之。漢高祖即位,與王思同並制贈侍中。



 宋令詢,不知何許人也。閔帝在籓時,補為客將,知書樂善,動皆由禮。長興中,閔帝連典大籓,遷為都押衙,參輔閫政,甚有時譽,閔帝深委之。及閔帝嗣位,朱、馮用事,不欲閔帝之舊臣在于左右,乃出為磁州刺史。閔帝蒙塵于衛,令詢日令人奔問。及聞帝遇害,大慟半日,自經而
 卒。



 史臣曰:夫代大匠斫者,猶傷其手,況代天子執賞罰之柄者乎!是以古之賢人,當大任、秉大政者,莫不卑以自牧,推之不有,廓自公之道,絕利己之欲,然後能保其身而脫其禍也。而重誨何人,安所逃死,古語云:「無為權首,反受其咎。」重誨之謂歟!自宏昭而下,力不能衛社稷,謀不能安國家,相踵而亡,又誰咎也。唯令詢感故君之舊恩,由大慟而自絕,以茲隕命,足以垂名。



\end{pinyinscope}