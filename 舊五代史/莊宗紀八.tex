\article{莊宗紀八}

\begin{pinyinscope}

 同光四年春正月戊午朔,帝不受朝賀,契丹寇渤海。壬戌,詔以去歲因被災沴,物價騰踴,自今月三日後避正殿,減膳撤樂,以答天譴。應去年遭水災州縣,秋夏稅賦
 並與放免。自壬午年已前所欠殘稅,及諸色課利,已有敕命放免者,尚聞所在卻有徵收,宜令租庸司切準前敕處分。應京畿內人戶,有停貯斛斗者,並令減價出糶;如不遵行,當令檢括。西川王衍父子及偽署將相官吏,除已行刑憲外,一切釋放。天下禁囚,除十惡五逆、官典犯贓、屠牛毀錢、放火劫舍、持刀殺人,準律常赦不原外,應合抵極刑者,遞降一等。其餘罪犯悉與減降;逃背軍健,並放逐便。



 癸亥,河中節度使李繼麟來朝。諸州上言,
 準宣為去年十月地震,集僧道起消災道場。甲子,魏王繼岌殺樞密使郭崇韜于西川「理」,人「只在勢之必然處見理」。著述甚豐,後人編有《船,夷其族。丙寅,百官上表,請復常膳,凡三上表,乃允之。西川行營都監李廷安進西川樂官二百九十八人。契丹寇女真、渤海。戊寅,契丹安巴堅遣使貢良馬。庚辰,帝異母弟鄜州節度使存乂伏誅。存乂,郭崇韜之子婿也,故亦及于禍。是日,以河中節度使、守太師、兼尚書令、西平王李繼麟為滑州節度使,尋令朱守殷以兵圍其第,誅之,亦夷其族。辛巳,吐渾、
 奚各遣使貢馬。鎮州上言,部民凍死者七千二百六十人。又奏,準宣進花果樹栽及抽樂人梅審譯赴京。甲申,以鄆州節度使、永王存霸為河中節度使,以滑州節度使、申王存渥為鄆州節度使。乙酉,內人景奼上言:「昭宗遇難之時,皇屬千餘人同時遇害,為三穴瘞于宮城西古龍興寺北,請改葬。」從之,仍詔河南府監護其事。丙戌,回鶻可汗阿都欲遣使貢良馬。鎮州上言,平棘等四縣部民,餓死者二千五十人。丁亥,詔硃友謙同惡人史
 武等七人,已當國法,並籍沒家產。武等友謙舊將,時皆為刺史,並以無罪族誅。《歐陽史》,丁亥,殺李繼麟之將史武、薛敬容、周唐殷、楊師太、王景、來仁、白奉國,滅其族。


二月己丑,以宣徽南院使、知內侍省兼內勾、特進、右領軍衛上將軍李紹宏為驃騎大將軍、守左武衛上將軍、知內侍省,充樞密使。甲午,以鄭州刺史李紹奇為河陽節度使,以樂人景進為銀青光祿大夫、檢校右散騎常侍、守御史大夫。進以俳優嬖幸學史大綱》
 \gezhu{
  上卷}
 、《白話文學史》
 \gezhu{
  上卷}
 、《水滸傳考證》、,善采訪閭巷鄙細事以啟奏,復密求妓媵以進,恩寵特厚。魏州錢穀
 諸務,及招兵市馬,悉委進監臨。孔謙附之以希寵,常呼為「八哥」。諸軍左右無不托附,至于士人,亦有因之而求仕進者。每入言事,左右紛然屏退,惟以陷害熒惑為意焉。是日,帝幸冷泉校獵。乙未,宰臣豆盧革上言,請支州縣官實俸,以責課效。



 丙申,武德使史彥瓊自鄴馳報稱:「今月六日,貝州屯駐兵士突入都城,剽劫坊市。」初,帝令魏博指揮使楊仁晸率兵戍瓦橋,至是代歸,有詔令駐于貝州。上歲天下大水,十月鄴地大震,自是居人或有
 亡去他郡者,每日族談巷語云:「城將亂矣!」人人恐悚,皆不自安。



 十二月,以戶部尚書王正言為興唐尹、知留守事。正言年耄風病,事多忽忘,比無經治之才。武德使史彥瓊者,以伶官得幸,帝待以腹心之任,都府之中,威福自我,正言以下,皆脅肩低首,曲事不暇。由是政無統攝,姦人得以窺圖。洎郭崇韜伏誅,人未測其禍始,皆云:「崇韜已殺繼岌,自王西川,故盡誅郭氏。」先是,有密詔令史彥瓊殺朱友謙之子澶州刺史建徽。史彥瓊夜半出城,不言
 所往。詰旦,閽報正言曰:「史武德夜半馳馬而去,不知何往。」是日人情震駭,訛言云:「劉皇后以繼岌死于蜀,已行弒逆,帝已晏駕,故急徵彥瓊。」其言播于鄴市,貝州軍士有私寧親于都下者,掠此言傳于貝州。軍士皇甫暉等因夜聚蒱博不勝,遂作亂,劫都將楊仁晸曰:「我輩十有餘年為國家效命,甲不離體,已至吞併天下,主上未垂恩澤,翻有猜嫌。防戍邊遠,經年離阻鄉國,及得代歸,去家咫尺,不令與家屬相見。今聞皇后弒逆,京邑已亂,
 將士各欲歸府寧親,請公同行。」仁晸曰:「汝等何謀之過耶!今英主在上,天下一家,從駕精兵不下百萬,西平巴、蜀,威振華夷,公等各有家族,何事如此!」軍人乃抽戈露刃環仁晸曰:「三軍怨怒,咸欲謀反,茍不聽從,須至無禮。」仁晸曰:「吾非不知此,但丈夫舉事,當計萬全。」軍人即斬仁晸。裨將趙在禮聞軍亂,衣不及帶,將踰垣而遁,亂兵追及,白刃環之曰:「公能為帥否?否則頭隨刃落!」在禮懼,即曰:「吾能為之。」眾遂呼噪,中夜燔劫貝郡。詰旦,擁在禮
 趨臨清,剽永濟、館陶。五日晚,有自貝州來者,言亂兵將犯都城,都巡檢使孫鐸等急趨史彥瓊之第,告曰:「賊將至矣,請給鎧仗,登陴拒守。」彥瓊曰:「今日賊至臨清,計程六日方至,為備未晚。」孫鐸曰:「賊來寇我,必倍道兼行,一朝失機,悔將何及!請僕射率眾登陴,鐸以勁兵千人伏于王莽河逆擊之;賊既挫勢,須至離潰,然後可以剪除。如俟其凶徒薄于城下,必慮奸人內應,則事未可測也。」彥瓊曰:「但訓士守城,何須即戰。」時彥瓊疑孫鐸等有他
 志,故拒之。是夜三更,賊果攻北門,彥瓊時以部眾在北門樓,聞賊呼噪,即時驚潰。彥瓊單騎奔京師。遲明,亂軍入城,孫鐸與之巷戰,不勝,攜其母自水門而出,獲免。晡晚,趙在禮引諸軍據宮城,署皇甫暉、趙進等為都虞候、斬斫使,諸軍大掠。興唐尹王正言謁在禮,望塵再拜。是日,眾推在禮為兵馬留后,草奏以聞。帝怒,命宋州節度使元行欽率騎三千赴鄴都招撫,詔徵諸道之師進討。



 丁酉,淮南楊溥遣使賀平蜀。己亥,魏王繼岌奏,康延孝擁
 眾反,回寇西川。遣副招討使任圜率兵追討之。庚子,福建節度副使王延翰奏,節度使王審知委權知軍府事。邢州左右步直軍四百人據城叛,推軍校趙太為留後,詔東北面副招討使李紹真率兵討之。辛丑,元行欽至鄴都,進攻南門,以詔書招諭城中,趙在禮獻羊酒勞軍,登城遙拜行欽曰:「將士經年離隔父母,不取敕旨歸寧,上貽聖憂,追悔何及!儻公善為敷奏,俾從渙汗,某等亦不敢不改過自新。」行欽曰:「上以汝輩有社稷功,必行
 赦宥。」因以詔書諭之。皇甫暉聚眾大詬,即壞詔。行欽以聞,帝怒曰:「收城之日,勿遺噍類!」壬寅,行欽自鄴退軍,保澶州。甲午,從馬直宿衛軍士王溫等五人夜半謀亂,殺本軍使,為衛兵所擒,磔于本軍之門。丙午,以右散騎常侍韓彥惲為戶部侍郎。丁未,鄴都行營招撫使元行欽率諸道之師再攻鄴都。戊申,以洋州留後李紹文為夔州節度使。詔河中節度使、永王存霸歸籓。己酉,以樞密使宋唐玉為特進、左威衛上將軍,充宣徽南院使。



 庚戌,
 諸軍大集於鄴都,進攻其城,不克。行欽又大治攻具。城中知其無赦,晝夜為備。朝廷聞之益恐,連發中使促繼岌西征之師。繼岌以康延孝據漢州,中軍之士從任圜進討,繼岌端居利州,不獲東歸。是日,飛龍使顏思威部署西川宮人至。辛亥,淮南楊溥遣使貢方物。西京上言,客省使李嚴押蜀主王衍至本府。壬子,以守太尉、中書令、河南尹兼河陽節度使、齊王張全義為檢校太師、兼尚書令,充許州節度使。東川董璋奏,準詔誅遂州節度
 使李令德於本州,夷其族。癸丑,湖南馬殷奏,福建節度使王審知疾甚,副使王延翰已權知軍府事,請降旄節。司天監上言:自二月上旬後,晝夜陰雲,不見天象,自二十六日方晴,至月終,星辰無變。以右衛上將軍朱漢賓知河南府事。



 甲辰,命蕃漢總管李嗣源統親軍赴鄴都,以討趙在禮。帝素倚愛元行欽,鄴城軍亂,即命為行營招討使,久而無功。時趙太據邢州,王景戡據滄州,自為留後,河朔郡邑多殺長吏。帝欲親征,樞密使與宰臣奏言:「
 京師者,天下根本,雖四方有變,陛下宜居中以制之,但命將出征,無煩躬御士伍。」帝曰:「紹榮討亂未有成功,繼岌之軍尚留巴、漢,餘無可將者,斷在自行。」樞密使李紹宏等奏曰:「陛下以謀臣猛將取天下,今一州之亂而云無可將者,何也?總管李嗣源是陛下宗臣,創業已來,艱難百戰,何城不下,何賊不平,威略之名,振于夷夏,以臣等籌之,若委以專征,鄴城之寇,不足平也!」帝素寬大容納,無疑于物,自誅郭崇韜、朱友謙之後,閹宦伶官交相
 讒諂,邦國大事皆聽其謀,繇是漸多猜惑,不欲大臣典兵,既聞奏議,乃曰:「予恃嗣源侍衛,卿當擇其次者。」又奏曰:「以臣等料之,非嗣源不可。」河南尹張全義亦奏云:「河朔多事,久則患生,宜令總管進兵。如倚李紹榮輩,未見其功。」帝乃命嗣源行營。是日,延州知州白彥琛奏,綏、銀兵士剽州城謀叛。魏王繼岌傳送郭崇韜父子首函至闕下,詔張全義收瘞之。乙巳,以右武衛上將軍李肅為安邑、解縣兩池榷鹽使,以吏部尚書李琪為國計使。



 三
 月丁未朔,李紹真奏,收復邢州,擒賊首趙太等二十一人,徇于鄴都城下,皆磔于軍門。庚戌,李紹真自邢州赴鄴都城下。辛亥,以威武軍節度副使、福建管內都指揮使、檢校太傅、守江州刺史王延翰為福建節度使,依前檢校太傅。壬子,李嗣源領軍至鄴都,營于西南隅。甲寅,進營于觀音門外,下令諸軍,詰旦攻城。是夜,城下軍亂,迫嗣源為帝。遲明,亂軍擁嗣源及霍彥威入于鄴城,復為皇甫暉、趙進等所脅,嗣源以詭詞得出,夜分至魏縣。
 時嗣源遙領鎮州,詰旦,議欲歸籓,上章請罪,安重誨以為不可,語在《明宗紀》中。翼日,遂次于相州。元行欽部下兵退保衛州,以飛語上奏,嗣源一日之中遣使上章申理者數四。帝遣嗣源子從審與中使白從訓齎詔以諭嗣源,行至衛州,從審為元行欽所械,不得達。是日,西面行營副招討使任圜奏,收復漢州,擒逆賊康延孝。



 丙辰,荊南高季興上言,請割峽內夔、忠、萬等三州卻歸當道,依舊管系,又請雲安監。初,將議伐蜀,詔高季興令率本
 軍上峽,自收元管屬郡。軍未進,夔、忠、萬三州已降,季興數請之,因賂劉皇后及宰臣樞密使,內外葉附,乃俞其請。戊午,詔河南府預借今年秋夏租稅。時年饑民困,百姓不勝其酷,京畿之民,多號泣於路,議者以為劉盆子復生矣。庚申,詔潞州節度使孔勍赴闕,以右龍虎統軍安崇阮權知潞州。是日,忠武軍節度使、齊王張全義薨。壬戌,宰臣豆盧革率百官上表,以魏博軍變,請出內府金帛優給將士。不報。時知星者上言:「客星犯天庫,宜散
 府藏。」又云:「流星犯天棓,主御前有急兵。」帝召宰臣于便殿,皇后出宮中妝奩銀盆各二,并皇子滿哥三人,謂宰臣曰:「外人謂內府金寶無數,向者諸侯貢獻旋供賜與,今宮中有者,妝奩、嬰孺而已,可鬻之給軍。」革等惶恐而退。癸亥,以偽置昭武軍節度使林思諤為閬州刺史。是日,出錢帛給賜諸軍,兩樞密使及宋唐玉、景進等各貢助軍錢幣。是時,軍士之家乏食,婦女掇蔬于野,及優給軍人,皆負物而詬曰:「吾妻子已殍矣,用此奚為!」甲子,元
 行欽自衛州率部下兵士歸,帝幸耀店以勞之。西川輦運金銀四十萬至闕,分給將士有差。元行欽請車駕幸汴州,帝將發京師,遣中官向延嗣馳詔所在誅蜀主王衍,仍夷其族。


乙丑,車駕發京師。戊辰,遣元行欽將騎軍沿河東向。壬申,帝至滎澤,以龍驤馬軍八百騎為前軍陸九淵集原名《象山先生集》,南宋陸九淵
 \gezhu{
  號象山}
 著。,遣姚彥溫董之。彥溫行至中牟,率所部奔于汴州。時潘瑰守王村寨,有積粟數萬,亦奔汴州。是時,李嗣源已入于汴。帝聞諸軍離散,精神沮喪,至萬勝
 鎮即命旋師。登路旁荒塚,置酒視諸將流涕。俄有野人進雉,因問塚名,對曰:「里人相傳為愁臺。」帝彌不悅,罷酒而去。是夜,次汜水。初,帝東出關,從駕兵二萬五千,及復至汜水,已失萬餘騎。乃留秦州都指揮使張塘以步騎三千守關。帝過罌子谷,道路險狹,每遇衛士執兵仗者,皆善言撫之曰:「適報魏王繼岌又進納西川金銀五十萬,到京當盡給爾等。」軍士對曰:「陛下賜與大晚,人亦不感聖恩。」帝流涕而已。又索袍帶賜從官,內庫使張容哥
 對曰:「頒給已盡。」衛士叱容哥曰:「致吾君社稷不保,是此閹豎!」抽刀逐之,或救而獲免。容哥謂同黨曰:「皇后惜物不散,軍人歸罪于吾輩,事若不測,吾輩萬段,願不見此禍。」因投河而死。
 \gezhu{
  《隆平集》:內臣李承進逮事唐莊宗,太祖嘗問莊宗時事,對曰:「莊宗好畋獵,每次近郊,衛士必控馬首曰:『兒郎輩寒冷,望陛下與救接。』莊宗隨所欲給之,如此者非一。晚年蕭墻之禍,由賞賚無節,威令不行也。」太祖嘆曰:「二十年夾河戰爭,不能以軍法約束此輩,誠兒戲。」}



 甲戌,次石橋,帝置酒野次,悲啼不樂,謂元行欽等諸將曰:「鄴下亂離,寇盜蜂起,總管迫于亂軍,存亡未測,今訛言紛擾,朕實無
 聊。卿等事予已來,富貴急難,無不共之,今茲危蹙,賴爾籌謀,而竟默默無言,坐觀成敗。予在滎澤之日,欲單騎渡河,訪求總管,面為方略,招撫亂軍,卿等各吐胸襟,共陳利害,今日俾予至此,卿等如何!」元行欽等百餘人垂泣而奏曰:「臣本小人,蒙陛下撫養,位極將相,危難之時,不能立功報主,雖死無以塞責,乞申後效,以報國恩。」于是,百餘人皆援刀截髮,置鬚于地,以斷首自誓,上下無不悲號,識者以為不祥。是日,西京留守張筠部署西征
 兵士到京,見于上東門外,晡晚,帝還宮。初,帝在汜水,衛兵散走,京師恐駭不寧,及帝至,人情稍安。乙亥,百官進名起居。安義節度使孔勍奏,點校兵士防城,準詔運糧萬石,進發次。時勍已殺監軍使據城,詭奏也。丙子,樞密使李紹宏與宰相豆盧革、韋說會于中興殿之廊下,商議軍機,因奏:「魏王西征兵士將至,車駕且宜控汜水,以俟魏王。」從之。午時,帝出上東門親閱騎軍,誡以詰旦東幸,申時還宮。



 四月丁丑朔,以永王存霸為北都留守,申
 王存渥為河中節度使。是日,車駕將發京師,從駕馬軍陳于寬仁門外,步兵陳于五鳳門外。帝內殿食次,從馬直指揮使郭從謙自本營率所部抽戈露刃,至興教門大呼,與黃甲兩軍引弓射興教門。帝聞其變,自宮中率諸王近衛禦之,逐亂兵出門。既而焚興教門,緣城而入,登宮牆歡噪,帝御親軍格鬥,殺亂兵數百。俄而帝為流矢所中,亭午,崩于絳霄殿之廡下,時年四十三。《琬琰集》載《宋實錄·王全斌傳》云:同光末,蕭墻有變,亂兵逼宮城,近臣宿將,皆釋甲潛遁,惟全斌與符彥卿等十數人居中拒戰。
 莊宗中流矢,扶掖歸絳霄殿,全斌慟哭而去。《東都事略·符彥卿傳》云:郭從謙之亂,莊宗左右皆引去,惟彥卿力戰,殺十餘人。莊宗崩,彥卿慟哭而去。是時,帝之左右例皆奔散,唯五坊人善友斂廓下樂器簇于帝屍之上,發火焚之。及明宗入洛,止得其燼骨而已。



 天成元年七月丁卯,有司上謚曰光聖神閔孝皇帝,廟號莊宗。是月丙子,葬于雍陵。《五代史補》:莊宗之嗣位也,志在渡河,但恨河東地狹兵少,思欲百練其眾,以取必勝于天下,乃下令曰:「凡出師,騎軍不見賊不許騎馬,或步騎前後已定,不得越軍分以避險惡。其分路並進,期會有處,不得違晷刻。并在路敢言病者,皆斬之。」故三軍懼法而戮力,皆一以當百,故硃梁舉天下而不能御,卒為所滅,良有以也。初,莊宗為公子時,雅
 好音律,又能自撰曲子詞。其後凡用軍,前後隊伍皆以所撰詞授之,使揭聲而唱,謂之「御制」。至于入陣,不論勝負,馬頭纔轉,則眾歌齊作。故凡所鬥戰,人忘其死,斯亦用軍之一奇也。莊宗好獵,每出,未有不蹂踐苗稼。一旦至中牟,圍合,忽有縣令,忘其姓名,犯圍諫曰:「大凡有國家者,當視民如赤子,性命所擊。陛下以一時之娛,恣其蹂踐,使比屋囂然動溝壑之慮,為民父母,豈其若是耶!」莊宗大怒,以為遭縣令所辱,遂叱退,將斬之。伶官鏡新磨者,知其不可,乃與群伶齊進,挽住令,佯為詬責曰:「汝為縣令,可以指麾百姓為兒,既天子好獵,即合多留閒地,安得縱百姓耕鋤皆遍,妨天子鷹犬飛走耶!而又不能自責,更敢咄咄,吾知汝當死罪。」諸伶亦皆嘻笑繼和,于是莊宗默然,其怒少霽,頃之,恕縣令罪。《五代史闕文》:莊宗嘗因博戲,睹骰子採有暗相輪者,心悅之,乃自置暗箭格,凡博戲並認採之在下者。及同光末,鄴都兵亂,
 從謙以兵犯興教門,莊宗御之,中流矢而崩。識者以為暗箭之應。



 史臣曰:莊宗以雄圖而起河、汾,以力戰而平汴、洛,家仇既雪,國祚中興,雖少康之嗣夏配天,光武之膺圖受命,亦無以加也。然得之孔勞,失之何速?豈不以驕于驟勝,逸于居安,忘櫛沐之艱難,徇色禽之荒樂。外則伶人亂政,內則牝雞司晨。靳吝貨財,激六師之憤怨;征搜輿賦,竭萬姓之脂膏。大臣無罪以獲誅,眾口吞聲而避禍。夫有一于此,未或不亡,矧咸有之,不亡何待!靜而思之,足
 以為萬代之炯誡也。



\end{pinyinscope}