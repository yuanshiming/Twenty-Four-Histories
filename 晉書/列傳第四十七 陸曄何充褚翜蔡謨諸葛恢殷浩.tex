\article{列傳第四十七 陸曄何充褚翜蔡謨諸葛恢殷浩}

\begin{pinyinscope}
陸曄
 \gezhu{
  弟玩玩子納何充褚翜蔡謨諸葛恢殷浩(顧悅之蔡裔}



 陸曄,字士光,吳郡吳人也。伯父喜,吳吏部尚書。父英,高平相,員外散騎常侍,曄少有雅望,從兄機每稱之曰:「我家世不乏公矣。」居喪,以孝聞。同郡顧榮與鄉人書曰:「士光氣息裁屬,慮其性命,言之傷心矣。」後察孝廉,除永世、烏江二縣令,皆不就。元帝初鎮江左,辟為祭酒,尋補振威將軍、義興太守,以疾不拜。預討華軼功,封平望亭侯,
 累遷散騎常侍、本郡大中正。太興元年,遷太子詹事。時帝以侍中皆北士,宜兼用南人,曄以清貞著稱,遂拜侍中,徙尚書,領州大中正。



 明帝即位,轉光祿勳,遷太常,代紀瞻為尚書左僕射,領太子少傅,尋加金紫光祿大夫,代卞壼為領軍將軍。以平錢鳳功,進爵江陵伯。帝不豫,曄與王導、壼、庾亮、溫嶠、郗鑒並受顧命,輔皇太子,更入殿將兵直宿。遺詔曰:「曄清操忠貞,歷職顯允,且其兄弟事君如父,憂國如家,歲寒不凋,體自門風。既委以六軍,可錄尚書事,加散騎常侍。」



 成帝踐阼,拜左光祿大夫、開府儀同三司,給親兵百人,常侍如故。蘇峻之難,曄隨
 帝左石頭,舉動方正,不以凶威變節。峻以曄吳士之望,不敢加害,使守留臺。匡術以苑城歸順,時共推曄督宮城軍事。峻平,加衛將軍。給千兵百騎,以勳進爵為公,封次子嘏新康子。



 咸和中,求歸鄉里拜墳墓。有司奏,舊制假六十日。侍中顏含、黃門侍郎馮懷駁曰:「曄內蘊至德,清一其心,受託付之重,居台司之位,既蒙詔許歸省填塋,大臣之義本在忘己,豈容有期而反,無期必遠。愚謂宜還自還,不須制日。」帝從之,曄因歸。以疾卒,時年七十四。追贈侍中、車騎大將軍,謚曰穆。子諶,散騎常侍。



 玩字士瑤。器量淹雅,弱冠有美名,賀循每稱其清允平
 當,郡檄綱紀,東海王越辟為掾,皆不就。元帝引為丞相參軍。時王導初至江左,思結人情,請婚於玩。玩對曰:「培塿無松柏,薰蕕不同器。玩雖不才,義不能為亂倫之始。」導乃止。玩嘗詣導食酪,因而得疾。與導箋曰:「僕雖吳人,幾為傖鬼。」其輕易權貴如此。



 累加奮武將軍,徵拜侍中,以疾辭。王敦請為長史,逼以軍期,不得已,乃從命。敦平,尚書令郗鑒議敦佐吏不能匡正姦惡,宜皆免官禁錮。會溫嶠上表申理,得不坐。復拜侍中,遷吏部尚書,領會稽王師,讓不拜,轉尚書左僕射,領本州大中正。及蘇峻反,遣玩與兄曄俱守宮城。玩潛說匡術歸順,以功封興
 平伯。轉尚書令。又詔曰:「玩體道清純,雅量弘遠,歷位內外,風績顯著。宜居台司,以允眾望。授左光祿大夫、開府儀同三司,加散騎常侍,餘如故。」玩頻自表,優詔褒揚。重復自陳曰:「臣實凡短,風操不立,階緣嘉會,便蕃榮顯,遂總括憲臺,豫聞政道。竟不能敷融玄風,清一朝序,咎責之來,於臣已重。誠以身許國,義忘曲讓。而慺慺所守,終於陳訴者,特以端右機要,事務殷多,臣已盈六十之年,智力有限,疾患深重,體氣日弊,朝夕自勵,非復所堪。若偃息茍免,職事並廢,則莫大之悔,天下將謂臣何!乞陛下披豁聖懷,霈然垂允。」詔不許。玩重表曰:「臣比披誠款,
 不足上暢天聰,聖恩徘徊,厲以體國。臣聞至公之道,上下玄同,用才不負其長,量力不受其短。雖加官重祿無世不有,皆庸勳親賢,時所須賴,兼統以濟世務,非優崇以榮一人。臣受遇三世,恩隆寵厚,豈敢辭職事之勞,求沖讓之譽。徒以端右要重,興替所存,久以無任,妨賢曠職。臣猶自知不可,況天下之人乎!今復外參論道,內統百揆,不堪之名,有如皎日。願陛下少垂哀矜,使四海知官不可以私於人,人不可以私取官,則天工弘坦,誰不謂允!」猶不許。尋而王導、郗鑒、庾亮相繼而薨,朝野咸以為三良既沒,國家殄瘁。以玩有德望,乃遷侍中、司空,給
 羽林四十人。玩既拜,有人詣之,索盃酒,瀉置柱梁之間,咒曰:「當今乏材,以爾為柱石,莫傾人梁棟邪!」玩笑曰:「戢卿良箴。」既而歎息,謂賓客曰:「以我為三公,是天下為無人。」談者以為知言。



 玩雖登公輔,謙讓不辟掾屬。成帝聞而勸之。玩不得已而從命,所辟皆寒素有行之士。玩翼亮累世,常以弘重為人主所貴,加性通雅,不以名位格物,誘納後進,謙若布衣,由是搢紳之徒莫不廕其德宇。後疾甚,上表曰:「臣嬰遘疾疢,沈頓歷月,不蒙痊損,而日夕漸篤,自省微綿,無復生望。荷恩不報,孤負已及,仰瞻天覆,伏枕隕涕。臣年向中壽,窮極寵榮,終身歸全,將復
 何恨!惟願陛下崇明聖德,弘敷洪化,曾構祖宗之基,道濟群生之命。臣不勝臨命遺戀之情,貪及視息,上表以聞。」薨年六十四,謚曰康,給兵千人,守冢七十家。太元中,功臣普被減削,司空何充等止得六家,以玩有佐命之勳,先陪陵而葬,由是特置興平伯官屬以衛墓。子始嗣,歷侍中、尚書。



 納字祖言。少有清操,貞厲絕俗。初辟鎮軍大將軍、武陵王掾,州舉秀才。太原王述雅敬重之,引為建威長史。累遷黃門侍郎、本州別駕、尚書吏部郎,出為吳興太守。將之郡,先至姑孰辭桓溫,因問溫曰:「公致醉可飲幾酒?食
 肉多少?」溫曰:「年大來飲三升便醉,白肉不過十臠。卿復云何?」納曰:「素不能飲,止可二升,肉亦不足言。」後伺溫閑,謂之曰:「外有微禮,方守遠郡,欲與公一醉,以展下情。」溫欣然納之。時王坦之、刁彞在坐。及受禮,唯酒一斗,鹿肉一拌,坐客愕然。納徐曰:「明公近云飲酒三升,納止可二升,今有一斗,以備杯杓餘瀝。」溫及賓客並嘆其率素,更敕中廚設精饌,酣飲極嘆而罷。納至郡,不受俸祿。頃之,徵拜左民尚書,領州大中正。將應召,外白宜裝幾船,納曰:「私奴裝糧食來,無所復須也。」臨發,止有被襆而已,其餘並封以還官。遷太常,徙吏部尚書,加奉車都尉、衛將
 軍。謝安嘗欲詣納,而納殊無供辦。其兄子俶不敢問之,乃密為之具。安既至,納所設唯茶果而已。俶遂陳盛饌,珍羞畢具。客罷,納大怒曰:「汝不能光益父叔,乃復穢我素業邪!」於是杖之四十。其舉措多此類。



 後以愛子長生有疾,求解官營視,兄子禽又犯法應刑,乞免官謝罪。詔特許輕降。頃長生小佳,喻還攝職。尋遷尚書僕射,轉左僕射,加散騎常侍。俄拜尚書令,常侍如故。恪勤貞固,始終不渝。時會稽王道子以少年專政,委任群小,納望闕而嘆曰:「好家居,纖兒欲撞壞之邪!」朝士咸服其忠亮。尋除左光祿大夫、開府儀同三司,未拜而卒,即以為贈。長
 生先卒,無子。以弟子道隆嗣,元熙中,為廷尉。



 何充,字次道,廬江灊人,魏光祿大夫禎之曾孫也。祖惲,豫州刺史。父睿,安豐太守。充風韻淹雅,文義見稱。初辟大將軍王敦掾,轉主簿。敦兄含時為廬江郡,貪汙狼藉,敦嘗於座中稱曰:「家兄在郡定佳,廬江人士咸稱之。」充正色曰:「充即廬江人,所聞異於此。」敦默然。傍人皆為之不安,充晏然自若。由是忤敦,左遷東海王文學,尋屬敦敗,累遷中書侍郎。



 充即王導妻之姊子,充妻,明穆皇后之妹也,故少與導善,早歷顯官。嘗詣導,導以麈尾反指床
 呼充共坐,曰:「此是君坐也。」導繕揚州解會,顧而言曰:」正為次道耳。」明帝亦友暱之。成帝即位,遷給事黃門侍郎。蘇峻作亂,京都傾覆,導從駕在石頭,充東奔義軍。其後導奔白石,充亦得還。賊平,封都鄉侯,拜散騎常侍,出為東陽太守,仍除建威將軍、會稽內史。在郡甚有德政,薦徵士虞喜,拔郡人謝奉、魏顗等以為佐吏。後以墓被發去郡。詔徵侍中,不拜。改葬畢,除建威將軍、丹陽尹。王導、庾亮並言於帝曰:「何充器局方概,有萬夫之望,必能總錄朝端,為老臣之副。臣死之日,願引充內侍,則外譽唯緝,社稷無虞矣。」由是加吏部尚書,進號冠軍將軍,又領
 會稽王師。及導薨,轉護軍將軍,與中書監庾冰參錄尚書事。詔充、冰各以甲杖五十人至止車門。尋遷尚書令,加左將軍。充以內外統任,宜相糾正,若使事綜一人,於課對為嫌,乃上疏固讓。許之。徙中書令,加散騎常侍,領軍如故。又領州大中正,以州有先達宿德,固讓不拜。



 庾冰兄弟以舅氏輔王室,權侔人主,慮易世之後,戚屬轉疏,將為外物所攻,謀立康帝,即帝母弟也。每說帝以國有彊敵,宜須長君,帝從之。充建議曰:「父子相傳,先王舊典,忽妄改易,懼非長計。故武王不授聖弟,即其義也。昔漢景亦欲傳祚梁王,朝臣咸以為虧亂典制,據而弗聽。
 今瑯邪踐阼,如孺子何!社稷宗廟,將其危乎!」冰等不從,既而康帝立,帝臨軒,冰、充侍坐。帝曰:「朕嗣鴻業,二君之力也。充對曰:「陛下龍飛,臣冰之力也。若如臣議,不睹升平之世。」帝有慚色。



 建元初,出為驃騎將軍、都督徐州揚州之晉陵諸軍事、假節,領徐州刺史,鎮京口,以避諸庾。頃之,庾翼將北伐,庾冰出鎮江州,充入朝,言於帝曰:「臣冰舅氏之重,宜居宰相,不應遠出。」朝議不從。於是徵充入為都督揚豫徐州之瑯邪諸軍事、假節,領揚州刺史,將軍如故。先是,翼悉發江、荊二州編戶奴以充兵役,士庶嗷然。充復欲發揚州奴以均其謗。後以中興時已發
 三吳,今不宜復發而止。



 俄而帝疾篤,冰、翼意在簡文帝,而充建議立皇太子,奏可。及帝崩,充奉遺旨,便立太子,是為穆帝,冰、翼甚恨之。獻后臨朝,詔曰:「驃騎任重,可以甲杖百人入殿。」又加中書監、錄尚書事。充自陳既錄尚書,不宜復監中書,許之。復加侍中,羽林騎十人。



 冰、翼等尋卒,充專輔幼主。翼臨終,表以後任委息爰之。於時論者並以諸庾世在西籓,人情所歸,宜依翼所請,以安物情。充曰:「不然。荊楚國之西門,戶口百萬,北帶彊胡,西鄰勁蜀,經略險阻,周旋萬里。得賢則中原可定,勢弱則社稷同憂,所謂陸抗存則吳存,抗亡則吳亡者,豈可以白
 面年少猥當此任哉!桓溫英略過人,有文武識度,西夏之任,無出溫者。」議者又曰:「庾爰之肯避溫乎?如令阻兵,恥懼不淺。」充曰:「溫足能制之,諸君勿憂。」乃使溫西。爰之果不敢爭。充以衛將軍褚裒皇太后父,宜綜朝政,上疏薦裒參錄尚書。裒以地逼,固求外出。充每曰:「桓溫、褚裒為方伯,殷浩居門下,我可無勞矣。」



 充居宰相,雖無澄正改革之能,而彊力有器局,臨朝正色,以社稷為己任,凡所選用,皆以功臣為先,不以私恩樹親戚,談者以此重之。然所暱庸雜,信任不得其人,而性好釋典,崇修佛寺,供給沙門以百數,糜費巨億而不吝也。親友至於貧乏,
 無所施遺,以此獲譏於世。阮裕嘗戲之曰:「卿志大宇宙,勇邁終古。」充問其故。裕曰:「我圖數千戶郡尚未能得,卿圖作佛,不亦大乎!」于時郗愔及弟曇奉天師道,而充與弟崇準信釋氏,謝萬譏之云:「二郗諂於道,二何佞於佛。」充能飲酒,雅為劉惔所貴。惔每云:「見次道飲,令人欲傾家釀。」言其能溫克也。



 永和二年卒,時年五十五,贈司空,謚曰文穆。無子,弟子放嗣。卒,又無子,又以兄孫松嗣,位至驃騎咨議參軍。充弟準,見《外戚傳》。



 褚翜,字謀遠,太傅裒之從父兄也。父頠,少知名,早卒。翜
 以才藝楨幹稱。襲爵關內侯,補冠軍參軍。于時長沙王乂擅權,成都、河間阻兵于外,翜知內難方作,乃棄官避地幽州。後河北有寇難,復還鄉里。河南尹舉翜行本縣事。及天下鼎沸,翜招合同志,將圖過江,先移住陽城界。潁川庾敳,即翜之舅也,亦憂世亂,以家付翜。翜道斷,不得前。東海王越以為參軍,辭疾不就。



 尋洛陽覆沒,與滎陽太守郭秀共保萬氏臺,秀不能綏眾,與將陳撫、郭重等構怨,遂相攻擊。翜懼禍及,謂撫等曰:「以諸君所以在此,謀逃難也。今宜共戮力以備賊,幸無外難,而內自相擊,是避坑落井也。郭秀誠為失理,應且容之。若遂所忿,
 城內自潰,胡賊聞之,指來掩襲,諸君雖得殺秀,無解胡虜矣,累弱非一,宜深思之。」撫等悔悟,與秀交和。時數萬口賴翜獲全。



 明年,率數千家將謀東下,遇道險,不得進,因留密縣。司隸校尉荀組以為參軍、廣威將軍,復領本縣,率邑人三千,督新城、梁、陽城三郡諸營事。頃之,遷司隸司馬,仍督營事。率眾進至汝水柴肥口,復阻賊。翜乃單馬至許昌,見司空荀籓,以為振威將軍,行梁國內史。



 建興初,復為豫州司馬,督司州軍事。太傅參軍王玄代翜為郡。時梁國部曲將耿奴甚得人情,而專勢,翜常優遇之。玄為政既急,翜知其不能容奴,因戒之曰:「卿威殺
 已多,而人情難一,宜深慎之。」玄納翜言,外羈縻奴,而內懷憤。會遷為陳留,將發,乃收奴斬之。翜奴餘黨聚眾殺玄。梁郡既有內難,而徐州賊張平等欲掩襲之。郡人遑惑,將以郡歸平。荀組遣翜往撫之,眾心乃定。頃之,組舉翜為吏部郎,不應召,遂東過江。



 元帝為晉王,以翜為散騎郎,轉太子中庶子,出為奮威將軍、淮南內史。永昌初,王敦構逆,征西將軍戴若思令翜出軍赴難,翜遣將領五百人從之。明帝即位,徵拜屯騎校尉,遷太子左衛率。成帝初,為左衛將軍。蘇峻之役,朝廷戒嚴,以翜為侍中,典征討軍事。既而王師敗績,司徒王導謂翜曰:「至尊當御
 正殿,君可啟令速出。」翜即入上大閣,躬自抱帝登太極前殿。導升御床抱帝,翜及鐘雅、劉超侍立左右。時百官奔散,殿省蕭然。峻兵既入,叱翜令下。翜正立不動,呵之曰:「蘇冠軍來覲至尊,軍人豈得侵逼!」由是兵士不敢上殿。及峻執政,猶以為侍中,從乘輿幸石頭。明年,與光祿大夫陸曄等出據苑城。蘇逸、任讓圍之,翜等固守。賊平,以功封長平縣伯,遷丹陽尹。時京邑焚蕩,人物凋殘,翜收集散亡,甚有惠政。



 代庾亮為中護軍,鎮石頭。尋為領軍,徙五兵尚書,加奉車都尉,監新宮事。遷尚書右僕射,轉左僕射,加散騎常侍。久之,代何充為護軍將軍,常侍
 如故。咸康七年卒,時年六十七,贈衛將軍,謚曰穆。子希嗣,官至豫章太守。



 蔡謨,字道明,陳留考城人也。世為著姓。曾祖睦,魏尚書。祖德,樂平太守。父克,少好學,博涉書記,為邦族所敬。性公亮守正,行不合己,雖富貴不交也。高平劉整恃才縱誕,服飾詭異,無所拘忌。嘗行造人,遇克在坐,整終席慚不自安。克時為處士,而見憚如此。後為成都王穎大將軍記室督。穎為丞相,擢為東曹掾。克素有格量,及居選官,茍進之徒,望風畏憚。初,克未仕時,河內山簡嘗與瑯
 邪王衍書曰:「蔡子尼今之正人。」衍以書示眾曰:「山子以一字拔人,然未易可稱。」後衍聞克在選官,曰:「山子正人之言,驗於今矣。」陳留時為大郡,號稱多士,瑯邪王澄行經其界,太守呂豫遣吏迎之。澄人境問吏曰:「此郡人士為誰?」吏曰:「有蔡子尼、江應元。」是時郡人多居大位者,澄以其姓名問曰:「甲乙等,非君郡人邪?」吏曰:「是也。」曰:「然則何以但稱此二人?」吏曰:「向謂君侯問人,不謂問位。」澄笑而止。到郡,以吏言謂豫曰:「舊名此郡有風俗,果然小吏亦知如此。」克以朝政日弊,遂絕不仕。東嬴公騰為車騎將軍,鎮河北,以克為從事中郎,知必不就,以軍期致之。
 克不得已,至數十日,騰為汲桑所攻,城陷,克見害。



 謨弱冠察孝廉,州辟從事,舉秀才,東海王越召為掾,皆不就。避亂渡江。時明帝為東中郎將,引為參軍。元帝拜丞相,復辟為掾,轉參軍,後為中書侍郎,歷義興太守、大將軍王敦從事中郎、司徒左長史,遷侍中。



 蘇峻構逆,吳國內史庾冰出奔會稽,乃以謨為吳國內史。謨既至,與張闓、顧眾、顧颺等共起義兵,迎冰還郡。峻平,復為侍中,遷五兵尚書,領瑯邪王師。謨上疏讓曰:「八坐之任,非賢莫居,前後所用,資名有常。孔愉、諸葛恢並以清節令才,少著名望。昔愉為御史中丞,臣尚為司徒長史;恢為會稽太
 守,臣為尚書郎;恢尹丹陽,臣守小郡。名輩不同,階級殊懸。今猥以輕鄙,超倫踰等,上亂聖朝貫魚之序,下違群士準平之論。豈惟微臣其亡之誡,實招聖政惟塵之累。且左長史一超而侍帷幄,再登而廁納言,中興已來,上德之舉所未嘗有。臣何人斯,而猥當之!是以叩心自忖,三省愚身,與其茍進以穢清塗,寧受違命狷固之罪。」疏奏,不許。轉掌吏部。以平蘇峻勳,賜爵濟陽男,又讓,不許。



 冬蒸,謨領祠部,主者忘設明帝位,與太常張泉俱免,白衣領職。頃之,遷太常,領祕書監,以疾不堪親職,上疏自解,不聽。成帝臨軒,遣使拜太傅、太尉、司空。會將作樂,宿
 縣於殿庭,門下奏,非祭祀燕饗則無設樂之制。事下太常。謨議臨軒遣使宜有金石之樂,遂從之。臨軒作樂,自此始也。彭城王紱上言,樂賢堂有先帝手畫佛象,經歷寇難,而此堂猶存,宜敕作頌。帝下其議。謨曰:「佛者,夷狄之俗,非經典之制。先帝量同天地,多才多藝,聊因臨時而畫此象,至於雅好佛道,所未承聞也。盜賊奔突,王都隳敗,而此堂塊然獨存,斯誠神靈保祚之徵,然未是大晉盛德之形容,歌頌之所先也。人臣睹物興義,私作賦頌可也。今欲發王命,敕史官,上稱先帝好佛之志,下為夷狄作一象之頌,於義有疑焉。」於是遂寢。



 時征西將軍
 庾亮以石勒新死,欲移鎮石城,為滅賊之漸。事下公卿。謨議曰:



 時有否泰,道有屈伸,暴逆之寇雖終滅亡,然當其彊盛,皆屈而避之。是以高祖受黜於巴漢,忍辱於平城也。若爭強於鴻門,則亡不終日。故蕭何曰「百戰百敗,不死何待」也。原始要終,歸於大濟而已。豈與當亡之寇爭遲速之間哉!夫惟鴻門之不爭,故垓下莫能與之爭。文王身圮於羑里,故道泰於牧野;句踐見屈於會稽,故威申於強吳。今日之事,亦由此矣。賊假息之命垂盡,而豺狼之力尚彊;宜抗威以待時。



 或曰:「抗威待時,時已可矣。」愚以為時之可否在賊之彊弱,賊之彊弱在季龍之
 能否。季龍之能否,可得而言矣。自勒初起,則季龍為爪牙,百戰百勝,遂定中國,境土所據,同於魏世。及勒死之日,將相內外欲誅季龍。季龍獨起於眾異之中,殺嗣主,誅寵臣。內難既定,千里遠出,一攻而拔金墉,再戰而斬石生,禽彭彪,殺石聰,滅郭權,還據根本,內外並定,四方鎮守,不失尺土。詳察此事,豈能乎,將不能也?假令不能者為之,其將濟乎,將不濟也?賊前襄陽而不能拔,誠有之矣。不信百戰之效,而執一攻之驗,棄多從少,於理安乎?譬若射者,百發而一不中,可謂之拙乎?且不拔襄陽者,非季龍身也。桓平北,守邊之將耳。賊前攻之,爭疆
 埸耳,得之為善,不得則止,非其所急也。今征西之往,則異於是。何者?重鎮也,名賢也,中國之人所聞而歸心也。今而西度,實有席卷河南之勢,賊所大懼,豈與桓宣同哉!季龍必率其精兵,身來距爭。若欲與戰,戰何如石生?若欲城守,守何如金墉?若欲阻沔,沔何如大江?蘇峻何如季龍?凡此數者,宜群校之。



 愚謂石生猛將,關中精兵,征西之虎不能勝也。金墉險固,劉曜十萬所不能拔,今征西之守不能勝也。又是時兗州、洛陽、關中皆舉兵擊季龍。今此三處反為其用,方之於前,倍半之覺也。若石生不能敵其半,而征西欲當其倍,愚所疑也。蘇峻之彊,
 不及季龍,沔水之險,不及大江。大江不能禦蘇峻,而以沔水禦季龍,又所疑也。昔祖士稚在譙,佃於城北,慮賊來攻,因以為資,故豫安軍屯,以禦其外。穀將熟,賊果至,丁夫戰於外,老弱獲於內,多持炬火,急則燒穀而走。如此數年,竟不得其利。是時賊唯據沔北,方之於今,四分之一耳。士稚不能捍其一,而征西欲禦其四,又所疑也。或云:「賊若多來,則必無糧。」然致糧之難,莫過崤函。而季龍昔涉此險,深入敵國,平關中而後還。今至襄陽,路既無險,又行其國內,自相供給,方之於前,難易百倍。前已經至難,而謂今不能濟其易,又所疑也。



 然此所論,但說
 征西既至之後耳,尚未論道路之慮也。自沔以西,水急岸高,魚貫溯流,首尾百里。若賊無宋襄之義,及我未陣而擊之,將如之何?今王士與賊,水陸異勢,便習不同。寇若送死,雖開江延敵,以一當千,猶吞之有餘,宜誘而致之,以保萬全。棄江遠進,以我所短擊彼所長,懼非廟勝之算。



 朝議同之,故亮不果移鎮。



 初,皇后每年拜陵,勞費甚多,謨建議曰:「古者皇后廟見而已,不拜陵也。」由是遂止。



 初,太尉郗鑒疾篤,出謨為太尉軍司,加侍中。鑒卒,即拜謨為征北將軍、都督徐兗青三州揚州之晉陵豫州之沛郡諸軍事、領徐州刺史、假節。時左衛將軍陳光上
 疏請伐胡,詔令攻壽陽,謨上疏曰:



 今壽陽城小而固。自幫陽至瑯邪,城壁相望,其間遠者裁百餘里,一城見攻,眾城必救。且王師在路五十餘日,劉仕一軍早已入淮,又遣數部北取堅壁,大軍未至,聲息久聞。而賊之郵驛,一日千里,河北之騎足以來赴,非惟鄰城相救而已。夫以白起、韓信、項籍之勇,猶發梁焚舟,背水而陣。今欲停船水渚,引兵造城,前對堅敵,顧臨歸路,此兵法之所誡也。若進攻未拔,胡騎卒至,懼桓子不知所為,而舟中之指可掬。今征軍五千,皆王都精銳之眾,又光為左衛,遠近聞之,名為殿中之軍,宜令所向有征無戰。而頓之堅
 城之下,勝之不武,不勝為笑。今以國之上駟擊寇之下邑,得之則利薄而不足損敵,失之則害重而足以益寇,懼非策之長者。臣愚以為聞寇而致討,賊退而振旅,於事無失。不勝管見,謹冒陳聞。



 季龍於青州造船數百,掠緣海諸縣,所在殺戮,朝廷以為憂。謨遣龍驤將軍徐玄等守中洲,并設募,若得賊大白船者,賞布千匹,小船百匹。是時謨所統七千餘人,所戍東至土山,西至江乘,鎮守八所,城壘凡十一處,烽火樓望三十餘處,隨宜防備,甚有算略。先是,郗鑒上部下有勳勞者凡一百八十人,帝並酬其功,未卒而鑒薨,斷不復與。謨上疏以為先已
 許鑒,今不宜斷。且鑒所上者皆積年勳效,百戰之餘,亦不可不報。詔聽之。



 康帝即位,徵拜左光祿大夫、開府儀同三司,領司徒。代殷浩為揚州刺史。又錄尚書事,領司徒如故。初,謨沖讓不辟僚佐,詔屢敦逼之,始取掾屬。



 石季龍死,中國大亂。時朝野咸謂當太平復舊,謨獨謂不然,語所親曰:「胡滅,誠大慶也,然將貽王室之憂。」或曰:「何哉?」謨曰:「夫能順天而奉時,濟六合於草昧,若非上哲,必由英豪。度德量力,非時賢所及。必將經營分表,疲人以逞志。才不副意,略不稱心,財單力竭,智勇俱屈,此韓廬、東郭所以雙斃也。」



 遷侍中、司徒。上疏讓曰:「伏自惟省,昔
 階謬恩,蒙忝非據,尸素累積而光寵更崇,謗讟彌興而榮進復加,上虧聖朝棟隆之舉,下增微臣覆餗之釁,惶懼戰灼,寄顏無所。乞垂天鑒,回恩改謬,以允群望。」皇太后詔報不許。謨猶固讓,謂所親曰:「我若為司徒,將為後代所哂,義不敢拜也。」皇太后遣使喻意,自四年冬至五年末,詔書屢下,謨固守所執。六年,復上疏,以疾病乞骸骨,上左光祿大夫、領司徒印綬。章表十餘上。穆帝臨軒,遣侍中紀璩、黃門郎丁纂征謨。謨陳疾篤,使主簿謝攸對曰:「臣謨不幸有公族穆子之疾,天威不違顏咫尺,不敢奉詔,寢伏待罪。」自旦至申,使者十餘反,而謨不至。時
 帝年八歲,甚倦,問左右曰:「所召人何以至今不來?臨軒何時當竟?」君臣俱疲弊。皇太后詔:「必不來者,宜罷朝。」中軍將軍殷浩奏免吏部尚書江[A170]官。簡文時為會稽王,命曹曰:「蔡公傲違上命,無人臣之禮。若人主卑屈於上,大義不行於下,亦不知復所以為政矣。」於是公卿奏曰:「司徒謨頃以常疾,久逋王命,皇帝臨軒,百僚齊立,俯僂之恭,有望於謨,若志存止退,自宜致辭闕庭,安有人君卑勞終日而人臣曾無一酬之禮!悖慢傲上,罪同不臣。臣等參議,宜明國憲,請送廷尉,以正刑書。」謨懼,率子弟素服詣闕稽顙,躬到廷尉待罪。皇太后詔曰:「謨先帝師
 傅,服事累世。且歸罪有司,內訟思愆。若遂致之于理,情所未忍。可依舊制免為庶人。」



 謨既被廢,杜門不出,終日講誦,教授子弟。數年,皇太后詔曰:「前司徒謨以道素著稱,軌行成名,故歷事先朝,致位台輔,以往年之失,用致黜責。自爾已來,闔門思愆,誠合大臣罪己之義。以謨為光祿大夫、開府儀同三司。」於是遣謁者僕射孟洪就加冊命。謨上疏陳謝曰:「臣以頑薄,皆忝殊寵,尸素累紀,加違慢詔命,當肆市朝。幸蒙寬宥,不悟天施復加光飾,非臣隕越所能上報。臣寢疾未損,不任詣闕。不勝仰感聖恩,謹遣拜章。」遂以疾篤,不復朝見。詔賜几杖,門施行馬。
 十二年,卒,時年七十六。賵贈之禮,一依太尉陸玩故事。詔贈侍中、司空,謚曰文穆。



 謨博學,於禮儀宗廟制度多所議定。文筆論議,有集行於世。總應劭以來注班固《漢書》者,為之集解。謨初渡江,見彭蜞,大喜曰:「蟹有八足,加以二螯。」令烹之。既食,吐下委頓,方知非蟹。後詣謝尚而說之。尚曰:「卿讀《爾雅》不熟,幾為《勸學》死。」謨性方雅。丞相王導作女伎,施設床席。謨先在坐,不悅而去,導亦不止之。性尤篤慎,每事必為過防。故時人云:「蔡公過浮航,脫帶腰舟。」長子邵,永嘉太守。少子系,有才學文義,位至撫軍長史。



 諸葛恢,字道明,瑯邪陽都人也。祖誕,魏司空,為文帝所誅。父靚,奔吳,為大司馬。吳平,逃竄不出。武帝與靚有舊,靚姊又為瑯邪王妃,帝知靚在姊間,因就見焉。靚逃于廁,帝又逼見之,謂曰:「不謂今日復得相見。」靚流涕曰:「不能漆身皮面,復睹聖顏!」詔以為侍中,固辭不拜,歸於鄉里,終身不向朝廷而坐。



 恢弱冠知名,試守即丘長,轉臨沂令,為政和平。值天下大亂,避地江左,名亞王導、庾亮。導嘗謂曰:「明府當為黑頭公。」及導拜司空,恢在從,導指冠謂曰:「君當復著此。」導嘗與恢戲爭族姓,曰:「人言王葛,
 不言葛王也。」恢曰:「不言馬驢,而言驢馬,豈驢勝馬邪!」其見親狎如此。于時潁川荀闓字道明、陳留蔡謨字道明,與恢俱有名譽,號曰「中興三明」,人為之語曰:「京都三明各有名,蔡氏儒雅荀葛清。」



 元帝為安東將軍,以恢為主簿,再遷江寧令。討周馥有功,封博陵亭侯,復為鎮東參軍。與卞壼並以時譽遷從事中郎,兼統記室。時四方多務,箋疏殷積,恢斟酌酬答,咸稱折中。于時王氏為將軍,而恢兄弟及顏含並居顯要,劉超以忠謹掌書命,時人以帝善任一國之才。愍帝即位,徵用四方賢雋,召恢為尚書郎,元帝以經緯須才,上疏留之,承制調為會稽太
 守。臨行,帝為置酒,謂曰:「今之會稽,昔之關中,足食足兵,在於良守。以君有蒞任之方,是以相屈。四方分崩,當匡振圮運。政之所先,君為言之。」恢陳謝,因對曰:「今天下喪亂,風俗陵遲,宜尊五美,屏四惡,進忠實,退浮華。」帝深納焉。太興初,以政績第一,詔曰:「自頃多難,官長數易,益有諸弊,雖聖人猶久於其道,然後化成,況其餘乎!漢宣帝稱『與我共安天下者,其惟良二千石』,斯言信矣。是以黃霸等或十年,或二十年而不徙,所以能濟其中興之勳也。賞罰黜陟,所以明政道也。會稽內史諸葛恢蒞官三年,政清人和,為諸郡首,宜進其位班,以勸風教。今增恢
 秩中二千石。」



 頃之,以母憂去官。服闋,拜中書令。王敦上恢為丹陽尹,以久疾免。明帝征敦,以恢為侍中,加奉車都尉。討王含有功,進封建安伯,以先爵賜次子為關內侯。又拜恢後將軍、會稽內史。徵為侍中,遷左民尚書、武陵王師、吏部尚書。累遷尚書右僕射,加散騎常侍、銀青光祿大夫、領選本州大中正、尚書令,常侍、吏部如故。成帝踐阼,加侍中、金紫光祿大夫。卒,年六十二。贈左光祿大夫、儀同三司。賵贈之禮,一依太尉興平伯故事,謚曰敬。祠以太牢。子甝嗣,位至散騎常侍。



 恢兄頤,字道回,亦為元帝所器重,終於太常。



 殷浩,字深源,陳郡長平人也。父羨,字洪喬,為豫章太守,都下人士因其致書者百餘函,行次石頭,皆投之水中,曰:「沈者自沈,浮者自浮,殷洪喬不為致書郵。」其資性介立如此。終於光祿勳。



 浩識度清遠,弱冠有美名,尤善玄言,與叔父融俱好《老》《易》。融與浩口談則辭屈,著篇則融勝,浩由是為風流談論者所宗。或問浩曰:「將蒞官而夢棺,將得財而夢糞,何也?」浩曰:「官本臭腐,故將得官而夢尸,錢本糞土,故將得錢而夢穢。」時人以為名言。



 三府辟,皆不就。征西將軍庾亮引為記室參軍,累遷司徒左長
 史。安西庾翼復請為司馬。除侍中、安西軍司,並稱疾不起。遂屏居墓所,幾將十年,于時擬之管、葛。王蒙、謝尚猶伺其出處,以卜江左興亡,因相與省之,知浩有確然之志。既反,相謂曰:「深源不起,當如蒼生何!」庾翼貽浩書曰:「當今江東社稷安危,內委何、褚諸君,外託庾、桓數族,恐不得百年無憂,亦朝夕而弊。足下少標令名,十餘年間,位經內外,而欲潛居利貞,斯理難全。且夫濟一時之務,須一時之勝,何必德均古人,韻齊先達邪!王夷甫,先朝風流士也,然吾薄其立名非真,而始終莫取。若以道非虞夏,自當超然獨往,而不能謀始,大合聲譽,極致名位,
 正當抑揚名教,以靜亂源。而乃高談《莊》《老》,說空終日,雖云談道,實長華競。及其末年,人望猶存,思安懼亂,寄命推務。而甫自申述,徇小好名,既身囚胡虜,棄言非所。凡明德君子,遇會處際,寧可然乎?而世皆然之。益知名實之未定,弊風之未革也。」浩固辭不起。



 建元初,庾冰兄弟及何充等相繼卒。簡文帝時在籓,始綜萬幾,衛將軍褚裒薦浩,徵為建武將軍、揚州刺史。浩上疏陳讓,并致箋於簡文,具自申敘。簡文答之曰:「屬當厄運,危弊理盡。誠賴時有其才,不復遠求版築。足下沈識淹長,思綜通練,起而明之,足以經濟。若復深存挹退,茍遂本懷,吾恐天
 下之事於此去矣,今紘領不振,晉網不綱,願蹈東海,復可得邪!由此言之,足下去就即是時之廢興,時之廢興則家國不異。足下弘思之,靜算之,亦將有以深鑒可否。望必廢本懷,率群情也。」浩頻陳讓,自三月至七月,乃受拜焉。



 時桓溫既滅蜀,威勢轉振,朝廷憚之。簡文以浩有盛名,朝野推伏,故引為心膂,以抗於溫,於是與溫頗相疑貳。會遭父憂,去職,時以蔡謨攝揚州,以俟浩,服闋,徵為尚書僕射,不拜。復為建武將軍、揚州刺史,遂參綜朝權。潁川荀羨少有令聞,浩擢為義興、吳郡,以為羽翼。王羲之密說浩、羨,令與桓溫和同,不宜內構嫌隙,浩不從。



 及石季龍死,胡中大亂,朝過欲遂蕩平關河,於是以浩為中軍將軍、假節、都督揚豫徐兗青五州軍事。浩既受命,以中原為己任,上疏北征許洛。將發,墜馬,時咸惡之。既而以淮南太守陳逵、兗州刺史蔡裔為前鋒,安西將軍謝尚、北中郎將荀羨為督統,開江西田千餘頃,以為軍儲。師次壽陽,潛誘苻健大臣梁安、雷弱兒等,使殺健,許以關右之任。初,降人魏脫卒,其弟憬代領部曲。姚襄殺憬,以并其眾,浩大惡之,使龍驤將軍劉啟守譙,遷襄於梁。既而魏氏子弟往來壽陽,襄益猜懼。俄而襄部曲有欲歸浩者,襄殺之,浩於是謀誅襄。會苻健殺其大臣,
 健兄子眉自洛陽西奔,浩以為梁安事捷,意苻健已死,請進屯洛陽,修復園陵,使襄為前驅,冠軍將軍劉洽鎮鹿臺,建武將軍劉遁據倉垣,又求解揚州,專鎮洛陽,詔不許。浩既至許昌,會張遇反,謝尚又敗績,浩還壽陽。後復進軍,次山桑,而襄反,浩懼,棄輜重退保譙城,器械軍儲皆為襄所掠,士卒多亡叛。浩遣劉啟、王彬之擊襄於山桑,並為襄所殺。



 桓溫素忌浩,及聞其敗,上疏罪浩曰:



 案中軍將軍浩過蒙朝恩,叨竊非據,寵靈超卓,再司京輦,不能恭慎所任,恪居職次,而侵官離局,高下在心。前司徒臣謨執義履素,位居台輔,師傅先帝,朝之元老,年
 登七十,以禮請退,雖臨軒固辭,不順恩旨,適足以明遜讓之風,弘優賢之禮。而浩虛生狡說,疑誤朝聽,獄之有司,將致大辟。自羯胡夭亡,群凶殄滅,而百姓塗炭,企遲拯接。浩受專征之重,無雪恥之志,坐自封植,妄生風塵,遂使寇仇稽誅,姦逆並起,華夏鼎沸,黎元殄悴。浩懼罪將及,不容於朝,外聲進討,內求茍免。出次壽陽,頓甲彌年,傾天府之資,竭五州之力,收合無賴,以自彊衛,爵命無章,猜害罔顧。故范豐之屬反叛於芍陂,奇德、龍會作變於肘腋。羌帥姚襄率眾歸化,遣其母弟入質京邑,浩不能撫而用之,陰圖殺害,再遣剌客,為襄所覺。襄遂惶
 懼,用致逆命。生長亂階,自浩始也。復不能以時掃滅,縱放小豎,鼓行毒害,身狼狽於山桑,軍破碎於梁國,舟車焚燒,輜重覆沒。三軍積實,反以資寇,精甲利器,更為賊用。神怒人怨,眾之所棄,傾危之憂,將及社稷。臣所以忘寢屏營,啟處無地。夫率正顯義,所以致訓,明罰敕法,所以齊眾,伏願陛下上追唐堯放命之刑下鑒《春秋》無君之典。若聖上含弘,末忍誅殛,且宜遐棄,擯之荒裔。雖未足以塞山海之責,粗可以宣誡於將來矣。



 竟坐廢為庶人,徙于東陽之信安縣。



 浩少與溫齊名,而每心競。溫嘗問浩:「君何如我?」浩曰:「我與君周旋久,寧作我也。」溫既以
 雄豪自許,每輕浩,浩不之憚也。至是,溫語人曰:「少時吾與浩共騎竹馬,我棄去,浩輒取之,故當出我下也。」又謂郗超曰:「浩有德有言,向使作令僕,足以儀刑百揆,朝廷用違其才耳。」



 浩雖被黜放,口無怨言,夷神委命,談詠不輟,雖家人不見其有流放之戚。但終日書空,作「咄咄怪事」四字而已。浩甥韓伯,浩素賞愛之,隨至徙所,經歲還都,浩送至渚側,詠曹顏遠詩云:「富貴他人合,貧賤親戚離。」因而泣下。後溫將以浩為尚書令,遺書告之,浩欣然許焉。將答書,慮有謬誤,開閉者數十,竟達空函,大忤溫意,由是遂絕。永和十二年卒。



 子涓,亦有美名,咸安初,桓
 溫廢太宰、武陵王晞,誣涓及庾倩與晞謀反,害之。



 浩後將改葬,其故吏顧悅之上疏訟浩曰:



 伏見故中軍將軍、揚州刺史殷浩體德沈粹,識理淹長,風流雅勝,聲蓋當時,再臨神州,萬里肅清,勳績茂著,聖朝欽嘉,遂授分陜推轂之任。戎旗既建,出鎮壽陽,驅其豺狼,翦其荊棘,收羅向義,廣開屯田,沐雨櫛風,等勤臺僕。仰憑皇威,群醜革面,進軍河洛,修復園陵。不虞之變,中路猖蹶,遂令為山之功崩於垂成,忠款之志於是而廢。既受削黜,自擯山海,杜門終身,與世兩絕,可謂克己復禮,窮而無怨者也。尋浩所犯,蓋負敗之常科,非即情之永責。論其名德
 深誠則如彼,察其補過罪己則如此,豈可棄而不恤,使法有餘冤!方今宅兆已成。埏隧已開,懸棺而窆,禮同庶人,存亡有非命之分,九泉無自訴之斯,仰感三良,昊天罔極。若使明詔爰發,旌我善人,崇復本官,遠彰幽昧,斯則國家威恩有兼濟之美,死而可作,無負心之恨。



 疏奏,詔追復浩本官。



 顧悅之,字君叔,少有義行。與簡文同年,而髮早白。帝問其故。對曰:「松柏之姿,經霜猶茂;蒲柳常質,望秋先零。」簡文悅其對。始將抗表訟浩,浩親故多謂非宜,悅之決意
 以聞,又與朝臣爭論,故眾無以奪焉。時人咸稱之。為州別駕,歷尚書右丞,卒。子凱之,別有傳。



 蔡裔者,有勇氣,聲若雷震。嘗有二偷入室,裔拊床一呼,而盜俱隕,故浩委以軍鋒焉。



 史臣曰:陸曄等並以時望國華,效彰歷試,迭居端揆,參掌機衡。然皆率由舊章,得免祗悔。而充抗言孺子,雖屈壓於權臣,翊奉儲君,竟導揚於末命,頻參大議,屢畫嘉謀,可謂忠貞在斯而已。殷浩清徽雅量,眾議攸歸,高秩厚禮,不行而至,咸謂教義由其興替,社稷俟以安危。及
 其入處國鈞,未有嘉謀善政,出總戎律,唯聞蹙國喪師,是知風流異貞固之才,談論非奇正之要。違方易任,以致播遷,悲失!蔡謨度德而處,弘斯止足,置以刑書,斯為過矣。



 贊曰:士光時望,士瑤允當。政既弟兄,任惟臺相。祖言簡率,遺風可尚。蔡葛知名,或雅或清。次道方概,謀遠忠貞。中軍鑒局,譽光雅俗。夷曠有餘,經綸不足。舍長任短,功虧名辱。



\end{pinyinscope}