\article{列傳第二十一}

\begin{pinyinscope}

 皇甫謐子方回摯虞束皙王接



 皇甫謐,字士安,幼名靜,安定朝那人,漢太尉嵩之曾孫也。出後叔父,徙居新安。年二十,不好學,游蕩無度,或以為癡。嘗得瓜果,輒進所後叔母任氏。任氏曰:「《孝經》云:『三牲之養,猶為不孝。』汝今年餘二十,目不存教,心不入道,無以慰我。」因歎曰:「昔孟母三徙以成仁,曾父烹豕以存教,豈我居不卜鄰,教有所闕,何爾魯鈍之甚也!修身篤
 學,自汝得之,於我何有!」因對之流涕。謐乃感激,就鄉人席坦受書,勤力不怠。居貧,躬自稼穡,帶經而農,遂博綜典籍百家之言。沈靜寡欲,始有高尚之志,以著述為務,自號玄晏先生。著《禮樂》、《聖真》之論。後得風痺疾,猶手不輟卷。



 或勸謐修名廣交,謐以為「非聖人孰能兼存出處,居田里之中亦可以樂堯、舜之道,何必崇接世利,事官鞅掌,然後為名乎」。作《玄守論》以答之,曰:



 或謂謐曰:「富貴人之所欲,貧賤人之所惡,何故委形待於窮而不變乎?且道之所貴者,理世也;人之所美者,及時也。先生年邁齒變,饑寒不贍,轉死溝壑,其誰知乎?」



 謐曰:「人之所至惜
 者,命也;道之所必全者,形也;性形所不可犯者,疾病也。若擾全道以損性命,安得去貧賤存所欲哉?吾聞食人之祿者懷人之憂,形強猶不堪,況吾之弱疾乎!且貧者士之常,賤者道之實,處常得實,沒齒不憂,孰與富貴擾神耗精者乎!又生為人所不知,死為人所不惜,至矣!喑聾之徒,天下之有道者也。夫一人死而天下號者,以為損也;一人生而四海笑者,以為益也。然則號笑非益死損生也。是以至道不損,至德不益。何哉?體足也。如回天下之念以追損生之禍,運四海之心以廣非益之病,豈道德之至乎!夫唯無損,則至堅矣;夫唯無益,則至厚矣。
 堅故終不損,厚故終不薄。茍能體堅厚之實,居不薄之真,立乎損益之外,游乎形骸之表,則我道全矣。」



 遂不仕。耽玩典籍,忘寢與食,時人謂之「書淫」。或有箴其過篤,將損耗精神。謐曰:「朝聞道,夕死可矣,況命之修短分定懸天乎!」



 叔父有子既冠,謐年四十喪所生後母,遂還本宗。



 城陽太守梁柳,謐從姑子也,當之官,人勸謐餞之。謐曰:「柳為布衣時過吾,吾送迎不出門,食不過鹽菜,貧者不以酒肉為禮。今作郡而送之,是貴城陽太守而賤梁柳,豈中古人之道,是非吾心所安也。」



 時魏郡召上計掾,舉孝廉;景元初,相國辟,皆不行。其後鄉親勸令應命,謐為《
 釋勸論》以通志焉。其辭曰:



 相國晉王辟餘等三十七人,及泰始登禪,同命之士莫不畢至,皆拜騎都尉,或賜爵關內侯,進奉朝請,禮如侍臣。唯餘疾困,不及國寵。宗人父兄及我僚類,咸以為天下大慶,萬姓賴之,雖未成禮,不宜安寢,縱其疾篤,猶當致身。餘唯古今明王之制,事無巨細,斷之以情,實力不堪,豈慢也哉!乃伏枕而歎曰:「夫進者,身之榮也;退者,命之實也。設餘不疾,執高箕山,尚當容之,況余實篤!故堯、舜之世,士或收迹林澤,或過門不敢入。咎繇之徒兩遂其願者,遇時也。故朝貴致功之臣,野美全志之士。彼獨何人哉!今聖帝龍興,配名前
 哲,仁道不遠,斯亦然乎!客或以常言見逼,或以逆世為慮。餘謂上有寬明之主,下必有聽意之人,天網恢恢,至否一也,何尤於出處哉!」遂究賓主之論,以解難者,名曰《釋勸》。



 客曰:「蓋聞天以懸象致明,地以含通吐靈。故黃鐘次序,律呂分形。是以春華發萼,夏繁其實,秋風逐暑,冬冰乃結。人道以之,應機乃發。三材連利,明若符契。故士或同升於唐朝,或先覺於有莘,或通夢以感主,或釋釣於渭濱,或叩角以干齊,或解褐以相秦,或冒謗以安鄭,或乘駟以救屯,或班荊以求友,或借術於黃神。故能電飛景拔,超次邁倫,騰高聲以奮遠,抗宇宙之清音。由此觀
 之,進德貴乎及時,何故屈此而不伸?今子以英茂之才,游精於六藝之府,散意於眾妙之門者有年矣。既遭皇禪之朝,又投祿利之際,委聖明之主,偶知己之會,時清道真,可以沖邁,此真吾生濯髮雲漢、鴻漸之秋也。韜光逐藪,含章未曜,龍潛九泉,堅焉執高,棄通道之遠由,守介人之局操,無乃乖於道之趣乎?



 且吾聞招搖昏迴則天位正,五教班敘則人理定。如今王命切至,委慮有司,上招迕主之累,下致駭眾之疑。達者貴同,何必獨異?群賢可從,何必守意?方今同命並臻,饑不待餐,振藻皇塗,咸秩天官。子獨栖遲衡門,放形世表,遜遁丘園,不睨華
 好,惠不加人,行不合道,身嬰大疢,性命難保。若其羲和促轡,大火西頹,臨川恨晚,將復何階!夫貴陰賤璧,聖所約也;顛倒衣裳,明所箴也。子其鑒先哲之洪範,副聖朝之虛心,沖靈翼於雲路,浴天池以濯鱗,排閶闔,步玉岑,登紫闥,侍北辰,翻然景曜,雜沓英塵。輔唐、虞之主,化堯舜、之人,宣刑錯之政,配殷、周之臣,銘功景鐘,參敘彞倫,存則鼎食,亡為貴臣,不亦茂哉!而忽金白之輝曜,忘青紫之班瞵,辭容服之光粲,抱弊褐之終年,無乃勤乎!」



 主人笑而應之曰:「吁!若賓可謂習外觀之暉暉,未睹幽人之仿佛也;見俗人之不容,未喻聖皇之兼愛也;循方圓
 於規矩,未知大形之無外也。故曰,天玄而清,地靜而寧,含羅萬類,旁薄群生,寄身聖世,託道之靈。若夫春以陽散,冬以陰凝,泰液含光,元氣混蒸,眾品仰化,誕制殊征。故進者享天祿,處者安丘陵。是以寒暑相推,四宿代中,陰陽不治,運化無窮,自然分定,兩克厥中。二物俱靈,是謂大同;彼此無怨,是謂至通。



 若乃衰周之末,貴詐賤誠,牽於權力,以利要榮。故蘇子出而六主合,張儀入而橫勢成,廉頗存而趙重,樂毅去而燕輕,公叔沒而魏敗,孫臏刖而齊寧,蠡種親而越霸,屈子疏而楚傾。是以君無常籍,臣無定名,損義放誠,一虛一盈。故馮以彈劍感主,
 女有反賜之說,項奮拔山之力,蒯陳鼎足之勢,東郭劫於田榮,顏闔恥於見逼。斯皆棄禮喪真,茍榮朝夕之急者也,豈道化之本與!



 若乃聖帝之創化也,參德乎三皇,齊風乎虞、夏,欲溫溫而和暢,不欲察察而明切也;欲混混若玄流,不欲蕩蕩而名發也;欲索索而條解,不欲契契而繩結也;欲芒芒而無垠際,不欲區區而分別也;欲闇然而內章,不欲示白若冰雪也;欲醇醇而任德,不欲瑣瑣而執法也。是以見機者以動成,好遁者無所迫。故曰,一明一昧,得道之概;一弛一張,合禮之方;一浮一沈,兼得其真。故上有勞謙之愛,下有不名之臣;朝有聘賢
 之禮,野有遁竄之人。是以支伯以幽疾距唐,李老寄迹於西鄰,顏氏安陋以成名,原思娛道於至貧,榮期以三樂感尼父,黔婁定謚於布衾,干木偃息以存魏,荊、萊志邁於江岑,君平因蓍以道著,四皓潛德於洛濱,鄭真躬耕以致譽,幼安發令乎今人。皆持難奪之節,執不迴之意,遭拔俗之主,全彼人之志。故有獨定之計者,不借謀於眾人;守不動之安者,不假慮於群賓。故能棄外親之華,通內道之真,去顯顯之明路,入昧昧之埃塵,宛轉萬情之形表,排託虛寂以寄身,居無事之宅,交釋利之人。輕若鴻毛,重若泥沈,損之不得,測之愈深。真吾徒之師
 表,餘迫疾而不能及者也。子議吾失宿而駭眾,吾亦怪子較論而不折中也。



 夫才不周用,眾所斥也;寢疾彌年,朝所棄也。是以胥克之廢,丘明列焉;伯牛有疾,孔子斯歎。若黃帝創制於九經,岐伯剖腹以蠲腸,扁鵲造虢而尸起,文摯徇命於齊王,醫和顯術於秦、晉,倉公發祕於漢皇,華佗存精於獨識,仲景垂妙於定方。徒恨生不逢乎若人,故乞命訴乎明王。求絕編於天錄,亮我躬之辛苦,冀微誠之降霜,故俟罪而窮處。



 其後武帝頻下詔敦逼不已,謐上疏自稱草莽臣曰:「臣以尪弊,迷於道趣,因疾抽簪,散髮林阜,人綱不閑,鳥獸為群。陛下披榛採蘭,
 并收蒿艾。是以皋陶振褐,不仁者遠。臣惟頑蒙,備食晉粟,猶識唐人擊壤之樂,宜赴京城,稱壽闕外。而小人無良,致災速禍,久嬰篤疾,軀半不仁,右腳偏小,十有九載。又服寒食藥,違錯節度,辛苦荼毒,于今七年。隆冬裸袒食冰,當暑煩悶,加以咳逆,或若溫虐,或類傷寒,浮氣流腫,四肢酸重。於今困劣,救命呼噏,父兄見出,妻息長訣。仰迫天威,扶輿就道,所苦加焉,不任進路,委身待罪,伏枕歎息。臣聞《韶》《衛》不並奏,《雅》《鄭》不兼御,故郤子入周,禍延王叔;虞丘稱賢,樊姬掩口。君子小人,禮不同器,況臣糠,糅之彫胡?庸夫錦衣,不稱其服也。竊聞同命之士,
 咸以畢到,唯臣疾疢,抱釁床蓐,雖貪明時,懼斃命路隅。設臣不疾,已遭堯、舜之世,執志箕山,猶當容之。臣聞上有明聖之主,下有輸實之臣;上有在寬之政,下有委情之人。唯陛下留神垂恕,更旌瑰俊,索隱於傅巖,收釣於渭濱,無令泥滓久濁清流。」謐辭切言至,遂見聽許。



 歲餘,又舉賢良方正,並不起。自表就帝借書,帝送一車書與之。謐雖羸疾,而披閱不怠。初服寒食散,而性與之忤,每委頓不倫,嘗悲恚,叩刃欲自殺,叔母諫之而止。



 濟陰太守蜀人文立,表以命士有贄為煩,請絕其禮幣,詔從之。謐聞而歎曰:「亡國之大夫不可與圖存,而以革歷代之
 制,其可乎!夫『束帛戔戔』,《易》之明義,玄纁之贄,自古之舊也。故孔子稱夙夜強學以待問,席上之珍以待聘。士於是乎三揖乃進,明致之難也;一讓而退,明去之易也。若殷湯之於伊尹,文王之於太公,或身即莘野,或就載以歸,唯恐禮之不重,豈吝其煩費哉!且一禮不備,貞女恥之,況命士乎!孔子曰:『賜也,爾愛其羊,我愛其禮。』棄之如何?政之失賢,於此乎在矣。」



 咸寧初,又詔曰:「男子皇甫謐沈靜履素,守學好古,與流俗異趣,其以謐為太子中庶子。」謐固辭篤疾。帝初雖不奪其志,尋復發詔徵為議郎,又召補著作郎。司隸校尉劉毅請為功曹,並不應。著論
 為葬送之制,名曰《篤終》,曰:



 玄晏先生以為存亡天地之定制,人理之必至也。故禮六十而制壽,至于九十,各有等差,防終以素,豈流俗之多忌者哉!吾年雖未制壽,然嬰疢彌紀,仍遭喪難,神氣損劣,困頓數矣。常懼夭隕不期,慮終無素,是以略陳至懷。



 夫人之所貪者,生也;所惡者,死也。雖貪,不得越期;雖惡,不可逃遁。人之死也,精歇形散,魂無不之,故氣屬于天;寄命終盡,窮體反真,故尸藏于地。是以神不存體,則與氣升降;尸不久寄,與地合形。形神不隔,天地之性也;尸與土並,反真之理也。今生不能保七尺之軀,死何故隔一棺之土?然則衣衾所以穢
 尸,棺槨所以隔真,故桓司馬石槨不如速朽;季孫璵璠比之暴骸;文公厚葬,《春秋》以為華元不臣;楊王孫親土,《漢書》以為賢於秦始皇。如今魂必有知,則人鬼異制,黃泉之親,死多於生,必將備其器物,用待亡者。今若以存況終,非即靈之意也。如其無知,則空奪生用,損之無益,而啟奸心,是招露形之禍,增亡者之毒也。



 夫葬者,藏也,藏也者,欲人之不得見也。而大為棺槨,備贈存物,無異於埋金路隅而書表於上也。雖甚愚之人,必將笑之。豐財厚葬以啟姦心,或剖破棺槨,或牽曳形骸,或剝臂捋金環,或捫腸求珠玉。焚如之形,不痛於是?自古及今,未
 有不死之人,又無不發之墓也。故張釋之曰:「使其中有欲,雖固南山猶有隙;使其中無欲,雖無石槨,又何戚焉!」斯言達矣,吾之師也。夫贈終加厚,非厚死也,生者自為也。遂生意於無益,棄死者之所屬,知者所不行也。《易》稱「古之葬者,衣之以薪,葬之中野,不封不樹」。是以死得歸真,亡不損生。



 故吾欲朝死夕葬,夕死朝葬,不設棺槨,不加纏斂,不修沐浴,不造新服,殯含之物,一皆絕之。吾本欲露形入坑,以身親土,或恐人情染俗來久,頓革理難,今故觕為之制,奢不石槨,儉不露形。氣絕之後,便即時服,幅巾故衣,以遽除裹尸,麻約二頭,置尸床上。擇不毛
 之地,穿坑深十尺,長一丈五尺,廣六尺,坑訖,舉床就坑,去床下尸。平生之物,皆無自隨,唯齎《孝經》一卷,示不忘孝道。遽除之外,便以親土。土與地平,還其故草,使生其上,無種樹木、削除,使生迹無處,自求不知。不見可欲,則奸不生心,終始無怵惕,千載不慮患。形骸與后土同體,魂爽與元氣合靈,真篤愛之至也。若亡有前後,不得移祔。祔葬自周公來,非古制也。舜葬蒼梧,二妃不從,以為一定,何必周禮。無問師工,無信卜筮,無拘俗言,無張神坐,無十五日朝夕上食。禮不墓祭,但月朔於家設席以祭,百日而止。臨必昏明,不得以夜。制服常居,不得墓次。
 夫古不崇墓,智也。今之封樹,愚也。若不從此,是戮尸地下,死而重傷。魂而有靈,則冤悲沒世,長為恨鬼。王孫之子,可以為誡。死誓難違,幸無改焉!



 而竟不仕。太康三年卒,時年六十八。子童靈、方回等遵其遺命。



 謐所著詩賦誄頌論難甚多,又撰《帝王世紀》、《年歷》、《高士》、《逸士》、《列女》等傳、《玄晏春秋》,並重於世。門人摯虞、張軌、牛綜、席純,皆為晉名臣。



 方回少遵父操,兼有文才。永嘉初,博士徵,不起。避亂荊州,閉戶閑居,未嘗入城府。蠶而後衣,耕而後食,先人後己,尊賢愛物,南土人士咸崇敬之。刺史陶侃禮之甚厚。
 侃每造之,著素士服,望門輒下而進。王敦遣從弟暠代侃,遷侃為廣州。侃將詣敦,方回諫曰:「吾聞敵國滅,功臣亡。足下新破杜弢,功莫與二,欲無危,其可得乎!」侃不從而行。敦果欲殺侃,賴周訪獲免。暠既至荊州,大失物情,百姓叛暠迎杜弢。暠大行誅戮以立威,以方回為侃所敬,責其不來詣己,乃收而斬之。荊土華夷,莫不流涕。



 摯虞,字仲洽,京兆長安人也。父模,魏太僕卿。虞少事皇甫謐,才學通博,著述不倦。郡檄主簿。虞嘗以死生有命,富貴在天。天之所祐者義也,人之所助者信也。履信思
 順,所以延福,違此而行,所以速禍。然道長世短,禍福舛錯,怵迫之徒,不知所守,蕩而積憤,或迷或放。故借之以身,假之以事,先陳處世不遇之難,遂棄彝倫,輕舉遠游,以極常人罔惑之情,而後引之以正,反之以義,推神明之應於視聽之表,崇否泰之運於智力之外,以明天任命之不可違,故作《思游賦》。其辭曰:



 有軒轅之遐胄兮,氏仲任之洪裔。敷華穎於末葉兮,晞靈根於上世。準乾坤以斡度兮,儀陰陽以定制。匪時運其焉行兮,乘太虛而搖曳。戴朗月之高冠兮,綴太白之明璜。製文霓以為衣兮,襲採雲以為裳。要華電之煜龠兮,佩玉衡之琳瑯。明
 景日以鑒形兮,信煥曜而重光。



 至美詭好於凡觀兮,修稀合而靡呈。燕石緹襲以華國兮,和璞遙棄於南荊。夏像韜塵于市北兮,瓶罍抗方於兩楹。鸞皇耿介而偏棲兮,蘭桂背時而獨榮。關寒暑以練真兮,豈改容而爽情。



 感昆吾之易越兮,懷暉光之速暮。羨一稔而三春兮,尚含英以容豫。悼曜靈之靡暇兮,限天晷之有度。聆鳴蜩之號節兮,恐隕葉于凝露。希前軌而增騖兮,眷後塵而旋顧。往者倏忽而不逮兮,來者冥昧而未著。二儀泊焉其無央兮,四節環轉而靡窮。星鳥逝而時反兮,夕景潛而且融。景三后之在天兮,歎聖哲之永終。諒道修而命
 微兮,孰舍盈而戢沖。握隋珠與蕙若兮,時莫悅而未遑。彼未遑其何恤兮,懼獨美之有傷。蹇委深而投奧兮,庶芬藻之不彰。芳處幽而彌馨兮,寶在夜而愈光。逼區內之迫脅兮,思攄翼乎八荒。望雲階之崇壯兮,願輕舉而高翔。



 造庖犧以問象兮,辨吉繇於姬文。將遠游於太初兮,鑒形魄之未分。四靈儼而為衛兮,六氣紛以成群。驂白獸於商風兮,御蒼龍於景雲。簡廝徒於靈圉兮,從馮夷而問津。召陵陽於游谿兮,旌王子於柏人。前祝融以掌燧兮,殿玄冥以掩塵。形飄飄而遂遐兮,氣亹癖而愈新。挹玉膏於萊嵎兮,掇紫英於瀛濱。揖太昊以假憩兮,
 聽賦政於三春。洪範翕而復張兮,百卉隕而更震。睇玉女之紛彯兮,執懿筐於扶木。覽玄象之韡曄兮,仍騰躍乎陽谷。吸朝霞以療飢兮,降廩泉而濯足。將縱轡以逍遙兮,恨東極之路促。詔纖阿而右迴兮,覿朱明之赫戲。蒞群神於夏庭兮,迴蒼梧而結知。纚焦明以承旂兮,駔天馬而高馳。讒羲和於丹丘兮,誚倒景之亂儀。尋凱風而南暨兮,謝太陽於炎離。戚溽暑之陶鬱兮,餘安能乎留斯!聞碧雞之長晨兮,吾將往乎西游。奧浮鷁於弱水兮,泊舳艫兮中流。茍精粹之攸存兮,誠沈羽以汎舟。軼望舒以陵厲兮,羌神漂而氣浮。訊碩老於金室兮,采舊
 聞於前修。譏淪陰於危山兮,問王母於椒丘。觀玄烏之參趾兮,會根壹之神籌。擾毚兔於月窟兮,詰姮娥於蓐收。爰攬轡而旋驅兮,訪北叟之倚伏。乘增冰而遂濟兮,凌固陰之所滀。探龜蛇於幽穴兮,敢罔養之潛育。哂倏忽之躁狂兮,喪中黃於耳目,偭燭龍而游衍兮,窮大明於北陸。



 攀招搖而上躋兮,忽蹈廓而凌虛。登閶闔而遺眷兮,頫玄黃於地輿。召黔雷以先導兮,覲天帝於清都。觀渾儀以寓目兮,拊造化之大爐。爰辨惑於上皇兮,稽吉凶之元符。唐則天而民咨兮,癸亂常而感虞。孔揮涕於西狩兮,臧考祥於婁句。跖肆暴而保乂兮,顏履仁而
 夙徂。何否泰之靡所兮,眩榮辱之不圖?運可期兮不可思,道可知兮不可為。求之者勞兮欲之者惑,信天任命兮理乃自得。



 且也四位為匠,乾巛為均。散而為物,結而為人。陽降陰升,一替一興。流而為川,滯而為陵。禍不可攘,福不可徵。其否兮有豫,其泰兮有數。成形兮未察,靈像兮巳固。承明訓以發蒙兮,審性命之靡求。將澄神而守一兮,奚飄飄而遐遊!



 斐陳辭以告退兮,主悖惘而永歎。惟升降之不仍兮,詠別易而會難。願大饗以致好兮,盍息駕於一飧。會司儀於有始兮,延嘉賓於九乾。陳鈞天之廣樂兮,展萬舞之至歡。枉矢鑠其在手兮,狼弧翾
 其斯彎。睨翟犬於帝側兮,殪熊羆於靈軒。



 爾乃清道夙蹕,載輪修祖。班命授號,轙輈整旅。兆司鬱以郕路兮,萬靈森而陳庭。豐隆軒其警眾兮,鉤陳帥以屬兵。堪輿竦而進時兮,文昌肅以司行。抗蚩尤之修旃兮,建雄虹之采旌。乘雲車電鞭之扶輿委移兮,駕應龍青虯之容裔陸離。俯游光逸景倏爍徽霍兮,仰流旌垂旄焱攸扦纚。前湛湛而攝進兮,後人禁僸而方馳。且啟行於重陽兮,奄稅駕乎少儀。跨列缺兮規乾巛,揮玉關兮出天門。涉漢津兮望昆侖,經赤霄兮臨玄根。觀品物兮終復魂,形已消兮氣猶存。眺懸舟之離離兮,懷舊都之藹藹。仍繁榮
 而督引兮,將遄降而速邁。華雲依霏而翼衡兮,日月炫晃而映蓋。蹈煙煴兮辭天衢,心闣兮識故居。路遂遒兮情欣欣,奄忽歸兮反常閭。修中和兮崇彞倫,大道繇兮味琴書。樂自然兮識窮達,澹無思兮心恒娛。



 舉賢良,與夏侯湛等十七人策為下第,拜中郎。武帝詔曰:「省諸賢良答策,雖所言殊塗,皆明於王義,有益政道。欲詳覽其對,究觀賢士大夫用心。」因詔諸賢良方正直言,會東堂策問,曰:「頃日食正陽,水旱為災,將何所修,以變大眚?及法令有不宜於今,為公私所患苦者,皆何事?凡平世在於得才,得才者亦借耳目以聽察。若有文武器能有
 益於時務而未見申敘者,各舉其人。及有負俗謗議,宜先洗濯者,亦各言之。」虞對曰:「臣聞古之聖明,原始以要終,體本以正末。故憂法度之不當,而不憂人物之失所;憂人物之失所,而不憂災害之流行。誠以法得於此,則物理於彼;人和於下,則災消於上。其有日月之眚,水旱之災,則反聽內視,求其所由,遠觀諸物,近驗諸身。耳目聽察,豈或有蔽其聰明者乎?動心出令,豈或有傾其常正者乎?大官大職,豈或有授非其人者乎?賞罰黜陟,豈或有不得其所者乎?河濱山巖,豈或有懷道釣築而未感於夢兆者乎?方外遐裔,豈或有命世傑出而未蒙膏
 澤者乎?推此類也,以求其故,詢事考言,以盡其實,則天人之情可得而見,咎徵之至可得而救也。若推之於物則無忤,求之於身則無尤,萬物理順,內外咸宜,祝史正辭,言不負誠,而日月錯行,夭癘不戒,此則陰陽之事,非吉凶所在也。期運度數,自然之分,固非人事所能供御,其亦振廩散滯,貶食省用而已矣。是故誠遇期運,則雖陶唐、殷湯有所不變;茍非期運,則宋、衛之君,諸侯之相,猶能有感。唯陛下審其所由,以盡其理,則天下幸甚。臣生長蓽門,不逮異物,雖有賢才,所未接識,不敢瞽言妄舉,無以疇答聖問。」擢為太子舍人,除聞喜令。



 時天子留
 心政道,又吳寇新平,天下乂安,上《太康頌》以美晉德。其辭曰:



 於休上古,人之資始。四隩咸宅,萬國同軌。有漢不競,喪亂靡紀。畿服外叛,侯衛內圮。天難既降,時惟鞠凶。龍戰獸爭,分裂遐邦。備僭岷蜀,度逆海東。權乃緣間,割據三江。明明上帝,臨下有赫。乃宣皇威,致天之辟。奮武遼隧,罪人斯獲。撫定朝鮮,奄征韓、貊。文既應期,席卷梁、益。元憝委命,九夷重譯。邛、冉、哀牢,是焉底績。我皇之登,二國既平。靡適不懷,以育群生。吳乃負固,放命南冥。聲教未暨,弗及王靈。皇震其威,赫如雷霆。截彼江、沔,荊、舒以清。邈矣聖皇,參乾兩離。陶化以正,取亂以奇。耀武六
 旬,輿徒不疲。飲至數實,干旄無虧。洋洋四海,率禮和樂。穆穆宮廟,歌雍詠鑠。光天之下,莫匪帝略。窮髮反景,承正受朔。龍馬騤騤,風于華陽。弓矢橐服,干戈戢藏。嚴嚴南金,業業餘皇。雄劍班朝,造舟為梁。聖明有造,實代天工。天地不違,黎元時邕。三務斯協,用底厥庸。既遠其迹,將明其蹤。喬山惟嶽,望帝之封。猗歟聖帝,胡不封哉!



 以母憂解職。久之,召補尚書郎。



 將作大匠陳勰掘地得古尺,尚書奏:「今尺長於古尺,宜以古為正。」潘岳以為習用已久,不宜復改。虞駮曰:「昔聖人有以見天下之賾而擬其形容,象物制器,以存時用。故參天兩地,以正算數之
 紀;依律計分,以定長短之度。其作之也有則,故用之也有征。考步兩儀,則天地無所隱其情;準正三辰,則懸象無所容其謬;施之金石,則音韻和諧;措之規矩,則器用合宜。一本不差而萬物皆正,及其差也,事皆反是。今尺長於古尺幾於半寸,樂府用之,律呂不合;史官用之,歷象失占;醫署用之,孔穴乖錯。此三者,度量之所由生,得失之所取征,皆絓閡而不得通,故宜改今而從古也。唐、虞之制,同律度量衡,仲尼之訓,謹權審度。今兩尺並用,不可謂之同;知失而行,不可謂之謹。不同不謹,是謂謬法,非所以軌物垂則,示人之極。凡物有多而易改,亦有
 少而難變,亦有改而致煩,有變而之簡。度量是人所常用,而長短非人所戀惜,是多而易改者也。正失於得,反邪於正,一時之變,永世無二,是變而之簡者也。憲章成式,不失舊物,季末茍合之制,異端雜亂之用,當以時釐改,貞夫一者也。臣以為宜如所奏。」又表論封禪,見《禮志》。



 虞以漢末喪亂,譜傳多亡失,雖其子孫不能言其先祖,撰《族姓昭穆》十卷,上疏進之,以為足以備物致用,廣多聞之益。以定品違法,為司徒所劾,詔原之。



 時太廟初建,詔普增位一等。後以主者承詔失旨,改除之。虞上表曰:「臣聞昔之聖明,不愛千乘之國而惜桐葉之信,所以重
 至尊之命而達於萬國之誠也。前《乙巳赦書》,遠稱先帝遺惠餘澤,普增位一等,以酬四海欣戴之心。驛書班下,被于遠近,莫不鳥騰魚躍,喜蒙德澤。今一旦更以主者思文不審,收既往之詔,奪已澍之施,臣之愚心竊以為不可。」詔從之。



 元康中,遷吳王友。時荀顗撰《新禮》,使虞討論得失而後施行。元皇后崩,杜預奏:「諒闇之制,乃自上古,是以高宗無服喪之文,而唯文稱不言。漢文限三十六日。魏氏以降,既虞為節。皇太子與國為體,理宜釋服,卒哭便除。」虞答預書曰:「唐稱遏密,殷云諒闇,各舉事以為名,非既葬有殊降。周室以來,謂之喪服。喪服者,以服
 表喪。今帝者一日萬機,太子監撫之重,以宜奪禮,葬訖除服,變制通理,垂典將來,何必附之於古,使老儒致爭哉!」皇太孫尚薨,有司奏「御服齊衰期」。詔令博士議。虞曰:「太子生,舉以成人之禮,則殤理除矣。太孫亦體君傳重,由位成而服全,非以年也。」從之。虞又議玉輅、兩社事,見《輿服志》。



 後歷祕書監、衛尉卿,從惠帝幸長安。及東軍來迎,百官奔散,遂流離鄠、杜之間,轉入南山中,糧絕飢甚,拾橡實而食之。後得還洛,歷光祿勳、太常卿。時懷帝親郊。自元康以來,不親郊祀,禮儀弛廢。虞考正舊典,法物粲然。及洛京荒亂,盜竊縱橫,人飢相食。虞素清貧,遂以
 餒卒。



 虞撰《文章志》四卷,注解《三輔決錄》,又撰古文章,類聚區分為三十卷,名曰《流別集》,各為之論,辭理愜當,為世所重。



 虞善觀玄象,嘗謂友人曰:「今天下方亂,避難之國,其唯涼土乎!」性愛士人,有表薦者,恒為其辭。東平太叔廣樞機清辯,廣談,虞不能對;虞筆,廣不能答;更相嗤笑,紛然於世云。



 束皙,字廣微,陽平元城人,漢太子太傅疏廣之後也。王莽末,廣曾孫孟達避難,自東海徙居沙鹿山南,因去疏之足,遂改姓焉。祖混,隴西太守。父龕,馮翊太守,並有名
 譽。皙博學多聞,與兄璆俱知名。少游國學,或問博士曹志曰:「當今好學者誰乎?」志曰:「陽平束廣微好學不倦,人莫及也。」還鄉里,察孝廉,舉茂才,皆不就。璆娶石鑒從女,棄之,鑒以為憾,諷州郡公府不得辟,故皙等久不得調。



 太康中,郡界大旱,皙為邑人請雨,三日而雨注,眾謂皙誠感,為作歌曰:「束先生,通神明,請天三日甘雨零。我黍以育,我稷以生。何以疇之?報束長生。」皙與衛恒厚善,聞恒遇禍,自本郡赴喪。



 嘗為《勸農》及《餅》諸賦,文頗鄙俗,時人薄之。而性沈退,不慕榮利,作《玄居釋》以擬《客難》,其辭曰:



 束皙閑居,門人並侍。方下帷深譚,隱幾而咍,含毫散
 藻,考撰同異,在側者進而問之曰:「蓋聞道尚變通,達者無窮。世亂則救其紛,時泰則扶其隆。振天維以贊百務,熙帝載而鼓皇風。生則率土樂其存,死則宇內哀其終。是以君子屈己伸道,不恥干時。上國有不索何獲之言,《周易》著躍以求進之辭。莘老負金鉉以陳烹割之說,齊客當康衢而詠《白水》之詩。今先生耽道修藝,嶷然山峙,潛朗通微,洽覽深識,夜兼忘寐之勤,晝騁鑽玄之思,曠年累稔,不墮其志。鱗翼成而愈伏,術業優而不試。乃欲闔櫝辭價,泥蟠深處,永戢琳瑯之耀,匿首窮魚之渚,當唐年而慕長沮,邦有道而反甯武。識彼迷此,愚竊不取。



 若乃士以援登,進必待求,附勢之黨橫擢,則林藪之彥不抽,丹墀步紈誇之童,東野遺白顛之叟。盍亦因子都而事博陸,憑鷁首以涉洪流,蹈翠雲以駭逸龍,振光耀以驚沈鰌。徒屈蟠於陷井,眄天路而不游,學既積而身困,夫何為乎祕丘。



 且歲不我與,時若奔駟,有來無反,難得易失。先生不知盱豫之讖悔遲,而忘夫朋盍之義務疾,亦豈能登海湄而抑東流之水,臨虞泉而招西歸之日?徒以曲畏為梏,儒學自桎,囚大道於環堵,苦形骸於蓬室。豈若託身權戚,憑勢假力,擇棲芳林,飛不待翼,夕宿七娥之房,朝享五鼎之食,匡三正則太階平,贊五教
 而玉繩直。孰若茹藿餐蔬,終身自匿哉!」



 束子曰:「居!吾將導爾以君子之道,諭爾以出處之事。爾其明受餘訊,謹聽餘志。



 昔元一既啟,兩儀肇立,離光夜隱,望舒晝戢,羽族翔林,蟩蛁赴濕,物從性之所安,士樂志之所執,或背豐榮以巖栖,或排蘭闥而求入,在野者龍逸,在朝者鳳集。雖其軌迹不同,而道無貴賤,必安其業,交不相羨,稷、契奮庸以宣道,巢、由洗耳以避禪,同垂不朽之稱,俱入賢者之流。參名比譽,誰劣誰優?何必貪與二八為群,而恥為七人之疇乎!且道睽而通,士不同趣,吾竊綴處者之末行,未敢聞子之高喻,將忽蒲輪而不眄,夫何權戚
 之云附哉!



 昔周、漢中衰,時難自託,福兆既開,患端亦作,朝遊巍峨之宮,夕墜崢嶸之壑,晝笑夜歎,晨華暮落,忠不足以衛己,禍不可以預度,是士諱登朝而競赴林薄。或毀名自汙,或不食其祿,比從政於匣笥之龜,譬官者於郊廟之犢,公孫泣涕而辭相,楊雄抗論於赤族。



 今大晉熙隆,六合寧靜。蜂蠆止毒,熊羆輟猛,五刑勿用,八紘備整,主無驕肆之怒,臣無犛纓之請,上下相安,率禮從道。朝養觸邪之獸,庭有指佞之草,禍戮可以忠逃,寵祿可以順保。



 且夫進無險懼,而惟寂之務者,率其性也。兩可俱是,而舍彼趣此者,從其志也。蓋無為可以解天下
 之紛,澹泊可以救國家之急,當位者事有所窮,陳策者言有不入,翟璜不能回西鄰之寇,平、勃不能正如意之立,干木臥而秦師退,四皓起而戚姬泣。夫如是何舍何執,何去何就?謂山岑之林為芳,谷底之莽為臭。守分任性,唯天所授,鳥不假甲於龜,魚不借足於獸,何必笑孤竹之貧而羨齊景之富!恥布衣以肆志,寧文裘而拖繡。且能約其躬,則儋石之畜以豐;茍肆其欲,則海陵之積不足;存道德者,則匹夫之身可榮;忘大倫者,則萬乘之主猶辱。將研六籍以訓世,守寂泊以鎮俗,偶鄭老於海隅,匹嚴叟於僻蜀。且世以太虛為輿,玄爐為肆,神游莫
 競之林,心存無營之室,榮利不擾其覺,殷憂不干其寐,捐夸者之所貪,收躁務之所棄,雉聖籍之荒蕪,總群言之一至。全素履於丘園,背纓緌而長逸,請子課吾業於千載,無聽吾言於今日也。」



 張華見而奇之。石鑒卒,王戎乃辟璆。華召皙為掾,又為司空、下邳王晃所辟。華為司空,復以為賊曹屬。



 時欲廣農,皙上議曰:


伏見詔書,以倉廩不實,關右饑窮,欲大興田農,以蕃嘉穀,此誠有虞戒大禹盡力之謂。然農穰可致,所由者三:一曰天時不諐,二曰地利無失,三曰人力咸用。若必春無
 \gezhu{
  雨脈}
 霂之潤,秋繁滂沱之患,水旱失中,雩禳有請。雖使羲和平秩,后稷
 親農,理疆甽於原隰,勤藨蓘於中田,猶不足以致倉庾盈億之積也。然地利可以計生,人力可以課致,詔書之旨,亦將欲盡此理乎?



 今天下千城,人多游食,廢業占空,無田課之實。較計九州,數過萬計。可申嚴此防,令鑒司精察,一人失課,負及郡縣,此人力之可致也。



 又州司十郡,土狹人繁,三魏尤甚,而豬羊馬牧,布其境內,宜悉破廢,以供無業。業少之人,雖頗割徙,在者猶多,田諸菀牧,不樂曠野,貪在人間。故謂北土不宜畜牧,此誠不然。案古今之語,以為馬之所生,實在冀北,大賈牂羊,取之清渤,放豕之歌,起於鉅鹿,是其效也。可悉徙諸牧,以充其
 地,使馬牛豬羊齕草於空虛之田,游食之人受業於賦給之賜,此地利之可致者也。昔騅駓在坰,史克所以頌魯僖;卻馬務田,老氏所以稱有道,豈利之所以會哉?又如汲郡之吳澤,良田數千頃,濘水停洿,人不墾植。聞其國人,皆謂通泄之功不足為難,舄鹵成原,其利甚重。而豪強大族,惜其魚捕之饒,構說官長,終於不破。此亦谷口之謠,載在史篇。謂宜復下郡縣,以詳當今之計。荊、揚、兗、豫,汙泥之土,渠塢之宜,必多此類,最是不待天時而豐年可獲者也。以其雲雨生於畚臿,多稌生於決泄,不必望朝躋而黃潦臻,禜山川而霖雨息。是故兩周爭東
 西之流,史起惜漳渠之浸,明地利之重也。宜詔四州刺史,使謹按以聞。



 又昔魏氏徙三郡人在陽平頓丘界,今者繁盛,合五六千家。二郡田地逼狹,謂可徙還西州,以充邊土,賜其十年之復,以慰重遷之情。一舉兩得,外實內寬,增廣窮人之業,以闢西郊之田,此又農事之大益也。



 轉佐著作郎,撰《晉書·帝紀》、十《志》,遷轉博士,著作如故。



 初,太康二年,汲郡人不準盜發魏襄王墓,或言安釐王冢,得竹書數十車。其《紀年》十三篇,記夏以來至周幽王為犬戎所滅,以事接之,三家分,仍述魏事至安釐王之二十年。蓋魏國之史書,大略與《春秋》皆多相應。其中經
 傳大異,則云夏年多殷;益乾啟位,啟殺之;太甲殺伊尹;文丁殺季歷;自周受命,至穆王百年,非穆王壽百歲也;幽王既亡,有共伯和者攝行天子事,非二相共和也。其《易經》二篇,與《周易》上下經同。《易繇陰陽卦》二篇,與《周易》略同,《繇辭》則異。《卦下易經》一篇,似《說卦》而異。《公孫段》二篇,公孫段與邵陟論《易》。《國語》三篇,言楚、晉事。《名》三篇,似《禮記》,又似《爾雅》、《論語》。《師春》一篇,書《左傳》諸卜筮,「師春」似是造書者姓名也。《瑣語》十一篇,諸國卜夢妖怪相書也。《梁丘藏》一篇,先敘魏之世數,次言丘藏金玉事。《繳書》二篇,論弋射法。《生封》一篇,帝王所封。《大曆》二篇,鄒子談天
 類也。《穆天子傳》五篇,言周穆王游行四海,見帝臺、西王母。《圖詩》一篇,畫贊之屬也。又雜書十九篇:《周食田法》,《周書》,《論楚事》,《周穆王美人盛姬死事》。大凡七十五篇,七篇簡書折壞,不識名題。冢中又得銅劍一枚,長二尺五寸。漆書皆科斗字。初發冢者燒策照取寶物,及官收之,多燼簡斷札,文既殘缺,不復詮次。武帝以其書付秘書校綴次第,尋考指歸,而以今文寫之。皙在著作,得觀竹書,隨疑分釋,皆有義證。遷尚書郎。



 武帝嘗問摯虞三日曲水之義,虞對曰:「漢章帝時,平原徐肇以三月初生三女,至三日俱亡,邨人以為怪,乃招攜之水濱洗祓,遂因水
 以汎觴,其義起此。」帝曰:「必如所談,便非好事。」皙進曰:「虞小生,不足以知,臣請言之。昔周公成洛邑,因流水以汎酒,故逸詩云『羽觴隨波』。又秦昭王以三日置酒河曲,見金人奉水心之劍,曰:『令君制有西夏。』乃霸諸侯,因此立為曲水。二漢相緣,皆為盛集。」帝大悅,賜皙金五十斤。



 時有人於嵩高山下得竹簡一枚,上兩行科斗書,傳以相示,莫有知者。司空張華以問皙,皙曰:「此漢明帝顯節陵中策文也。」檢驗果然,時人伏其博識。



 趙王倫為相國,請為記室。皙辭疾罷歸,教授門徒。年四十卒,元城市里為之廢業,門生故人立碑墓側。



 皙才學博通,所著《三魏人
 士傳》,《七代通記》、《晉書·紀》、《志》,遇亂亡失。其《五經通論》、《發蒙記》、《補亡詩》、文集數十篇,行於世云。



 王接,字祖游,河東猗氏人,漢京兆尹尊十世孫也。父蔚,世修儒史之學。魏中領軍曹羲作《至公論》,蔚善之,而著《至機論》,辭義甚美。官至夏陽侯相。接幼喪父,哀毀過禮,鄉親皆歎曰:「王氏有子哉!」渤海劉原為河東太守,好奇,以旌才為務。同郡馮收試經為郎,七十餘,薦接於原曰:「夫驊騮不總轡,則非造父之肆;明月不流光,則非隋侯之掌。伏惟明府苞黃中之德,耀重離之明,求賢與能,小
 無遺錯,是以鄙老思獻所知。竊見處士王接,岐嶷俊異,十三而孤,居喪盡禮,學過目而知,義觸類而長,斯玉鉉之妙味,經世之徽猷也。不患玄黎之不啟,竊樂春英之及時。」原即禮命,接不受。原乃呼見曰:「君欲慕肥遁之高邪?」對曰:「接薄祜,少孤而無兄弟,母老疾篤,故無心為吏。」及母終,柴毀骨立,居墓次積年,備覽眾書,多出異義。性簡率,不修俗操,鄉里大族多不能善之,唯裴頠雅知焉。平陽太守柳澹、散騎侍郎裴遐、尚書僕射鄧攸皆與接友善。後為郡主簿,迎太守溫宇,宇奇之,轉功曹史。州辟部平陽從事。時泰山羊亮為平陽太守,薦之於司隸校
 尉王堪,出補都官從事。



 永寧初,舉秀才。友人滎陽潘滔遺接書曰:「摯虞、卞玄仁並謂足下應和鼎味,可無以應秀才行。」接報書曰:「今世道交喪,將遂剝亂,而識智之士鉗口韜筆,禍敗日深,如火之燎原,其可救乎?非榮斯行,欲極陳所見,冀有覺悟耳。」是歲,三王義舉,惠帝復阼,以國有大慶,天下秀孝一皆不試,接以為恨。除中郎,補征虜將軍司馬。



 蕩陰之役,侍中嵇紹為亂兵所害,接議曰:「夫謀人之軍,軍敗則死之;謀人之國,國危則亡之,古之道也。蕩陰之役,百官奔北,唯嵇紹守職以遇不道,可謂臣矣,又可稱痛矣。今山東方欲大舉,宜明高節,以號令
 天下。依《春秋》褒三累之義,加紹致命之賞,則遐邇向風,莫敢不肅矣。」朝廷從之。



 河間王顒欲遷駕長安,與關東乖異,以接成都王佐,難之,表轉臨汾公相國。及東海王越率諸候討顒,尚書令王堪統行臺,上請接補尚書殿中郎,未至而卒,年三十九。



 接學雖博通,特精《禮》《傳》。常謂《左氏》辭義贍富,自是一家書,不主為經發。《公羊》附經立傳,經所不書,傳不妄起,於文為儉,通經為長。任城何休訓釋甚詳,而黜周王魯,大體乖硋,且志通《公羊》而往往還為《公羊》疾病。接乃更注《公羊春秋》,多有新義。時秘書丞衛恒考正汲冢書,未訖而遭難。佐著作郎束皙述而
 成之,事多證異義。時東萊太守陳留王庭堅難之,亦有證據。皙又釋難,而庭堅已亡。散騎侍郎潘滔謂接曰:「卿才學理議,足解二子之紛,可試論之。」接遂詳其得失。摯虞、謝衡皆博物多聞,咸以為允當。又撰《列女後傳》七十二人,雜論議、詩賦、碑頌、駁難十餘萬言,喪亂盡失。



 長子愆期,流寓江南,緣父本意,更注《公羊》,又集《列女後傳》云。



 史臣曰:皇甫謐素履幽貞,閑居養疾,留情筆削,敦悅丘墳,軒冕未足為榮,貧賤不以為恥,確乎不拔,斯固有晉之高人者歟!洎乎《篤終》立論,薄葬昭儉,既戒奢於季氏,亦無取於王孫,可謂達存亡之機矣。摯虞、束皙等並詳
 覽載籍,多識舊章,奏議可觀,文詞雅贍,可謂博聞之士也。或攝官延閣,裁成言事之書;或蒞政秩宗,參定禋郊之禮。虞既厄於從理,皙乃年位不充,天之報施,何其爽也!王接才調秀出,見賞知音,惜其夭枉,未申驥足,嗟夫!



 贊曰:士安好逸,棲心蓬蓽。屬意文雅,忘懷榮秩。遺制可稱,養生乖術。摯虞博聞,廣微絕群。財成禮度,刊緝遺文。魏篇式序,漢冊斯分。祖游後出,亦播清芬。



\end{pinyinscope}