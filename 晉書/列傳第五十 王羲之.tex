\article{列傳第五十 王羲之}

\begin{pinyinscope}

 王羲之子玄之凝之徽之徽之子楨之徽之弟操之獻之許邁



 王羲之,字逸少,司徒導之從子也,祖正,尚書郎。父曠,淮南太守。元帝之過江也,曠首創其議。羲之幼訥於言,人未之奇。年十三,嘗謁周顗,顗察而異之。時重牛心炙,坐客未啖,顗先割啖羲之,於是始知名。及長,辯贍,以骨鯁稱,尤善隸書,為古今之冠,論者稱其筆勢,以為飄若浮雲,矯若驚龍。深為從伯敦、導所器重。時陳留阮裕有重
 名,為敦主簿。敦嘗謂羲之曰:「汝是吾家佳子弟,當不減阮主簿。」裕亦目羲之與王承、王悅為王氏三少。時太尉郗鑒使門生求女婿於導,導令就東廂遍觀子弟。門生歸,謂鑒曰:「王氏諸少並佳,然聞信至,咸自矜持。惟一人在東床坦腹食,獨若不聞。」鑒曰:「正此佳婿邪!」訪之,乃羲之也,遂以女妻之。



 起家祕書郎,征西將軍庾亮請為參軍,累遷長史。亮臨薨,上疏稱羲之清貴有鑒裁。遷寧遠將軍、江州刺史。羲之既少有美譽,朝廷公卿皆愛其才器,頻召為侍中、吏部尚書,皆不就。復授護軍將軍,又推遷不拜。揚州刺史殷浩素雅重之,勸使應命,乃遺羲之
 書曰:「悠悠者以足下出處足觀政之隆替,如吾等亦謂為然。至如足下出處,正與隆替對,豈可以一世之存亡,必從足下從容之適?幸徐求眾心。卿不時起,復可以求美政不?若豁然開懷,當知萬物之情也。」羲之遂報書曰:「吾素自無廊廟志,直王丞相時果欲內吾,誓不許之,手跡猶存,由來尚矣,不於足下參政而方進退。自兒娶女嫁,便懷尚子平之志,數與親知言之,非一日也。若蒙驅使,關隴、巴蜀皆所不辭。吾雖無專對之能,直謹守時命,宣國家威德,固當不同於凡使,必令遠近咸知朝廷留心於無外,此所益殊不同居護軍也。漢末使太傅馬日
 磾慰撫關東,若不以吾輕微,無所為疑,宜及初冬以行,吾惟恭以待命。」



 羲之既拜護軍,又苦求宣城郡,不許,乃以為右軍將軍、會稽內史。時殷浩與桓溫不協,羲之以國家之安在於內外和,因以與浩書以戒之,浩不從。及浩將北伐,羲之以為必敗,以書止之,言甚切至。浩遂行果為姚襄所敗。復圖再舉,又遺浩書曰:



 知安西敗喪,公私惋怛,不能須臾去懷,以區區江左,所營綜如此,天下寒心,固以久矣,而加之敗喪,此可熟念。往事豈復可追,顧思弘將來,令天下寄命有所,自隆中興之業。政以道勝寬和為本,力爭武功,作非所當,因循所長,以固大業,
 想識其由來也。



 自寇亂以來,處內外之任者,未有深謀遠慮,括囊至計,而疲竭根本,各從所志,竟無一功可論,一事可記,忠言嘉謀棄而莫用,遂令天下將有土崩之勢,何能不痛心悲慨也。任其事者,豈得辭四海之責!追咎往事,亦何所復及,宜更虛己求賢,當與有識共之,不可復令忠允之言常屈於當權。今軍破於外,資竭於內,保淮之志非復所及,莫過還保長江,都督將各復舊鎮,自長江以外,羈縻而已。任國鈞者,引咎責躬,深自貶降以謝百姓。更與朝賢思布平政,除其煩苛,省其賦役,與百姓更始。庶可以允塞群望,救倒懸之急。



 使君起於布
 衣,任天下之重,尚德之舉,未能事事允稱。當董統之任而敗喪至此,恐闔朝群賢未有與人分其謗者。今亟修德補闕,廣延群賢,與之分任,尚未知獲濟所期。若猶以前事為未工,故復求之於分外,宇宙雖廣,自容何所!知言不必用,或取怨執政,然當情慨所在,正自不能不盡懷極言。若必親征,未達此旨,果行者,愚智所不解也。願復與眾共之。



 復被州符,增運千石,征役兼至,皆以軍期,對之喪氣,罔知所厝。自頃年割剝遺黎,刑徒竟路,殆同秦政,惟未加參夷之刑耳,恐勝廣之憂,無復日矣。



 又與會稽王箋陳浩不宜北伐,并論時事曰:



 古人恥其君不
 為堯舜,北面之道,豈不願尊其所事,比隆往代,況遇千載一時之運?顧智力屈於當年,何得不權輕重而處之也。今雖有可欣之會,內求諸己,而所憂乃重於所欣。《傳》云:「自非聖人,外寧必有內憂。」今外不寧,內憂已深。古之弘大業者,或不謀於眾,傾國以濟一時功者,亦往往而有之。誠獨運之明足以邁眾,暫勞之弊終獲永逸者可也。求之於今,可得擬議乎!



 夫廟算決勝,必宜審量彼我,萬全而後動。功就之日,便當因其眾而即其實。今功未可期,而遺黎殲盡,萬不餘一。且千里饋糧,自古為難,況今轉運供繼,西輸許洛,北入黃河。雖秦政之弊,未至於
 此,而十室之憂,便以交至。今運無還期,徵求日重,以區區吳越經緯天下十分之九,不亡何待!而不度德量力,不弊不已,此封內所痛心歎悼而莫敢吐誠。



 往者不可諫,來者猶可追,願殿下更垂三思,解而更張,令殷浩、荀羨還據合肥、廣陵,許昌、譙郡、梁、彭城諸軍皆還保淮,為不可勝之基,須根立勢舉,謀之未晚,此實當今策之上者。若不行此,社稷之憂可計日而待。安危之機,易於反掌,考之虛實,著於目前,願運獨斷之明,定之於一朝也。



 地淺而言深,豈不知其未易。然古人處閭閻行陣之間,尚或干時謀國,評裁者不以為譏,況廁大臣末行,豈可
 默而不言哉!存亡所係,決在行之,不可復持疑後機,不定之於此,後欲悔之,亦無及也。



 殿下德冠宇內,以公室輔朝,最可直道行之,致隆當年,而未允物望,受殊遇者所以寤寐長歎,實為殿下惜之。國家之慮深矣,常恐伍員之憂不獨在昔,麋鹿之游將不止林藪而已。願殿下暫廢虛遠之懷,以救倒懸之急,可謂以亡為存,轉禍為福,則宗廟之慶,四海有賴矣。



 時東土饑荒,羲之輒開倉振貸。然朝廷賦役繁重,吳會憂甚,羲之每上疏爭之,事多見從。又遺尚書僕射謝安書曰:



 頃所陳論,每蒙允納,所以令下小得蘇息,各安其業。若不耳,此一郡久以蹈
 東海矣。



 今事之大者未布,漕運是也。吾意望朝廷可申下定期,委之所司,勿復催下,但當歲終考其殿最。長吏尤殿,命檻車送詣天臺。三縣不舉,二千石必免,或可左降,令在疆塞極難之地。



 又自吾到此,從事常有四五,兼以臺司及都水御史行臺文符如雨,倒錯違背,不復可知。吾又瞑目循常推前,取重者及綱紀,輕者在五曹。主者蒞事,未嘗得十日,吏民趨走,功費萬計。卿方任其重,可徐尋所言。江左平日,揚州一良刺史便足統之,況以群才而更不理,正由為法不一,牽制者眾,思簡而易從,便足以保守成業



 倉督監耗盜官米,動以萬計,吾謂誅
 翦一人,其後便斷,而時意不同。近檢校諸縣,無不皆爾。餘姚近十萬斛,重斂以資姦吏,令國用空乏,良可歎也。



 自軍興以來,征役及充運死亡叛散不反者眾,虛耗至此,而補代循常,所在凋困,莫知所出。上命所差,上道多叛,則吏及叛者席卷同去。又有常制,輒令其家及同伍課捕。課捕不擒,家及同伍尋復亡叛。百姓流亡,戶口日減,其源在此。又有百工醫寺,死亡絕沒,家戶空盡,差代無所,上命不絕,事起成十年、十五年,彈舉獲罪無懈息而無益實事,何以堪之!謂自今諸死罪原輕者及五歲刑,可以充此,其減死者,可長充兵役,五歲者,可充雜工
 醫寺,皆令移其家以實都邑。都邑既實,是政之本,又可絕其亡叛。不移其家,逃亡之患復如初耳。今除罪而充雜役,盡移其家,小人愚迷,或以為重於殺戮,可以絕姦。刑名雖輕,懲肅實重,豈非適時之宜邪!



 羲之雅好服食養性,不樂在京師,初渡浙江,便有終焉之志。會稽有佳山水,名士多居之,謝安未仕時亦居焉。孫綽、李充、許詢、支遁等皆以文義冠世,並築室東土,與羲之同好。嘗與同志宴集於會稽山陰之蘭亭,羲之自為之序以申其志,曰:



 永和九年,歲在癸丑,暮春之初,會於會稽山陰之蘭亭,修禊事也。群賢畢至,少長咸集。此地有崇山峻嶺,
 茂林修竹,又有清流激湍,映帶左右,引以為流觴曲水,列坐其次。雖無絲竹管弦之盛,一觴一詠,亦足以暢敘幽情。



 是日也,天朗氣清,惠風和暢,仰觀宇宙之大,俯察品類之盛,所以游目騁懷,足以極視聽之娛,信可樂也。



 夫人之相與,俯仰一世,或取諸懷抱,悟言一室之內,或因寄所托,放浪形骸之外。雖趣舍萬殊,靜躁不同,當其欣於所遇,暫得於己,快然自足,不知老之將至。及其所之既倦,情隨事遷,感慨係之矣。向之所欣,俯仰之間,已為陳迹,猶不能不以之興懷。況修短隨化,終期於盡。古人云,死生亦大矣,豈不痛哉!



 每覽昔人興感之由,若合
 一契,未嘗不臨文嗟悼,不能喻之於懷。固知一死生為虛誕,齊彭殤為妄作,後之視今,亦猶今之視昔,悲夫!故列敘時人,錄其所述,雖世殊事異,所以興懷,其致一也。後之覽者,亦將有感於斯文。



 或以潘岳《金谷詩序》方其文,羲之比於石崇,聞而甚喜。



 性愛鵝,會稽有孤居姥養一鵝,善鳴,求市未能得,遂攜親友命駕就觀。姥聞羲之將至,烹以待之,羲之歎惜彌日。又山陰有一道士,養好鵝,羲之往觀焉,意甚悅,固求市之。道士云:「為寫《道德經》,當舉群相贈耳。」羲之欣然寫畢,籠鵝而歸,甚以為樂。其任率如此。嘗詣門生家,見棐几滑凈,因書之,真草相半。後
 為其父誤刮去之,門生驚懊者累日。又嘗在蕺山見一老姥,持六角竹扇賣之。羲之書其扇,各為五字。姥初有慍色。因謂姥曰:「但言是王右軍書,以求百錢邪。」姥如其言,人競買之。他日,姥又持扇來,羲之笑而不答。其書為世所重,皆此類也。每自稱「我書比鐘繇,當抗行;比張芝草,猶當鴈行也」。曾與人書云:「張芝臨池學書,池水盡黑,使人耽之若是,未必後之也。」羲之書初不勝庾翼、郗愔,及其暮年方妙。嘗以章草答庾亮,而翼深歎伏,因與羲之書云:「吾昔有伯英章草十紙,過江顛狽,遂乃亡失,常歎妙迹永絕。忽見足下答家兄書,煥若神明,頓還舊觀。」



 時驃騎將軍王述少有名譽,與羲之齊名,而羲之甚輕之,由是情好不協。述先為會稽,以母喪居郡境,羲之代述,止一弔,遂不重詣。述每聞角聲,謂羲之當候己,輒灑掃而待之。如此者累年,而羲之竟不顧,述深以為恨。及述為揚州刺史,將就徵,周行郡界,而不過羲之,臨發,一別而去。先是,羲之常謂賓友曰:「懷祖正當作尚書耳,投老可得僕射。更求會稽,便自邈然。」及述蒙顯授,羲之恥為之下,遣使詣朝廷,求分會稽為越州。行人失辭,大為時賢所笑。既而內懷愧嘆,謂其諸子曰:「吾不減懷祖,而位遇懸邈,當由汝等不及坦之故邪!」述後檢察會稽郡,
 辯其刑政,主者疲於簡對。羲之深恥之,遂稱病去郡,於父母墓前自誓曰:「維永和十一年三月癸卯朔,九日辛亥,小子羲之敢告二尊之靈。羲之不天,夙遭閔凶,不蒙過庭之訓。母兄鞠育,得漸庶幾,遂因人乏,蒙國寵榮。進無忠孝之節,退違推賢之義,每仰詠老氏、周任之誡,常恐死亡無日,憂及宗祀,豈在微身而已!是用寤寐永歎,若墜深谷。止足之分,定之於今。謹以今月吉辰肆筵設席,稽顙歸誠,告誓先靈。自今之後,敢渝此心,貪冒茍進,是有無尊之心而不子也。子而不子,天地所不覆載,名教所不得容。信誓之誠,有如皦日!」



 羲之既去官,與東土
 人士盡山水之游,弋釣為娛。又與道士許邁共修服食,採藥石不遠千里,遍游東中諸郡,窮諸名山,泛滄海,歎曰:「我卒當以樂死。」謝安嘗謂羲之曰:「中年以來,傷於哀樂,與親友別,輒作數日惡。」羲之曰:「年在桑榆,自然至此。頃正賴絲竹陶寫,恒恐兒輩覺,損其歡樂之趣。」朝廷以其誓苦,亦不復徵之。



 時劉惔為丹陽尹,許詢嘗就惔宿,床帷新麗,飲食豐甘。詢曰:「若此保全,殊勝東山。」惔曰:「卿若知吉凶由人,吾安得保此。」羲之在坐,曰:「令巢許遇稷契,當無此言。」二人並有愧色。



 初,羲之既優游無事,與吏部郎謝萬書曰:



 古之辭世者或被髮陽狂,或污身穢跡,
 可謂艱矣。今僕坐而獲逸,遂其宿心,其為慶幸,豈非天賜!違天不祥。



 頃東游還,修植桑果,今盛敷榮,率諸子,抱弱孫,游觀其間,有一味之甘,割而分之,以娛目前。雖植德無殊邈,猶欲教養子孫以敦厚退讓。或以輕薄,庶令舉策數馬,仿佛萬石之風。君謂此何如?



 比當與安石東游山海,并行田視地利,頤養閑暇。衣食之餘,欲與親知時共懽宴,雖不能興言高詠,銜杯引滿,語田里所行,故以為撫掌之資,其為得意,可勝言邪!常依陸賈、班嗣、楊王孫之處世,甚欲希風數子,老夫志願盡於此也。



 萬後為豫州都督,又遺萬書誡之曰:「以君邁往不屑之韻,而
 俯同群辟,誠難為意也。然所謂通識,正自當隨事行藏,乃為遠耳。願君每與士之下者同,則盡善矣。食不二味,居不重席,此復何有,而古人以為美談。濟否所由。實在積小以致高大,君其存之。」萬不能用,果敗。



 年五十九卒,贈金紫光祿大夫。諸子遵父先旨,固讓不受。



 有七子,知名者五人。玄之早卒。次凝之,亦工草隸,仕歷江州刺史、左將軍、會稽內史。王氏世事張氏五斗米道,凝之彌篤。孫恩之攻會稽,僚佐請為之備。凝之不從,方入靖室請禱,出語諸將佐曰:「吾已請大道,許鬼兵相助,賊自破矣。」既不設備,遂為孫所害。



 徽之字子猷。性卓犖不羈,為大司馬桓溫參軍,蓬首散帶,不綜府事。又為車騎桓沖騎兵參軍,沖問:「卿署何曹?」對曰:「似是馬曹。」又問:「管幾馬?」曰:「不知馬,何由知數!」又問:「馬比死多少?」曰:「未知生,焉知死!」嘗從沖行,值暴雨,徽之因下馬排入車中,謂曰:「公豈得獨擅一車!」沖嘗謂徽之曰:「卿在府日久,比當相料理。」徽之初不酬答,直高視,以手版柱頰云:「西山朝來致有爽氣耳。」



 時吳中一士大夫家有好竹,欲觀之,便出坐輿造竹下,諷嘯良久。主人灑掃請坐,徽之不顧。將出,主人乃閉門,徽之便以此賞之,盡嘆而去。嘗寄居空宅中,便令種竹。或問其故,徽之但
 嘯詠,指竹曰:「何可一日無此君邪!」嘗居山陰,夜雪初霽,月色清朗,四望皓然,獨酌酒詠左思《招隱詩》,忽憶戴逵。逵時在剡,便夜乘小船詣之,經宿方至,造門不前而反。人問其故,徽之曰:「本乘興而行,興盡而反,何必見安道邪!」雅性放誕,好聲色,嘗夜與弟獻之共讀《高士傳贊》,獻之賞井丹高潔,徽之曰:「未若長卿慢世也。」其傲達若此。時人皆欽其才而穢其行。



 後為黃門侍郎,棄官東歸,與獻之俱病篤,時有術人云:「人命應終,而有生人樂代者,則死者可生。」徽之謂曰:「吾才位不如弟,請以餘年代之。」術者曰:「代死者,以己年有餘,得以足亡者耳。今君與弟
 算俱盡,何代也!」未幾,獻之卒,徽之奔喪不哭,直上靈床坐,取獻之琴彈之,久而不調,歎曰:「嗚呼子敬,人琴俱亡!」因頓絕。先有背疾,遂潰裂,月餘亦卒。子楨之。



 楨之字公幹,歷位侍中、大司馬長史。桓玄為太尉,朝臣畢集,問楨之:「我何如君亡叔?」在坐咸為氣咽。楨之曰:「亡叔一時之標,公是千載之英。」一坐皆悅。



 操之字子重,歷侍中、尚書、豫章太守。



 獻之字子敬。少有盛名,而高邁不羈,雖閑居終日,容止不怠,風流為一時之冠。年數歲,嘗觀門生樗蒱,曰:「南風不競。」門生曰:「此郎亦管中窺豹,時見一斑。」獻之怒曰:「遠
 慚荀奉倩,近愧劉真長。」遂拂衣而去。嘗與兄徽之、操之俱詣謝安,二兄多言俗事,獻之寒溫而已。既出,客問安王氏兄弟優劣,安曰:「小者佳。」客問其故,安曰:「吉人之辭寡,以其少言,故知之。」嘗與徽之共在一室,忽然火發,徽之遽走,不遑取履。獻之神色恬然,徐呼左右扶出。夜臥齋中而有偷人入其室,盜物都盡。獻之徐曰:「偷兒,氈青我家舊物,可特置之。」群偷驚走。



 工草隸,善丹青。七八歲時學書,羲之密從後掣其筆不得,歎曰:「此兒後當復有大名。」嘗書壁為方丈大字,羲之甚以為能,觀者數百人。桓溫嘗使書扇,筆誤落,因畫作烏駁牸牛,甚妙。



 起家州
 主簿、秘書郎,轉丞,以選尚新安公主。嘗經吳郡,聞顧辟彊有名園。先不相識,乘平肩輿徑入。時辟彊方集賓友,而獻之遊歷既畢,傍若無人。辟彊勃然數之曰:「傲主人,非禮也。以貴驕士,非道也。失是二者,不足齒之傖耳。」便驅出門。獻之傲如也,不以屑意。



 謝安甚欽愛之,請為長史。安進號衛將軍,復為長史。太元中,新起太極殿,安欲使獻之題榜,以為萬代寶,而難言之,試謂曰:「魏時陵雲殿榜未題,而匠者誤釘之,不可下,乃使韋仲將懸橙書之。比訖,鬚鬢盡白,裁餘氣息。還語子弟,宜絕此法。」獻之揣知其旨,正色曰:「仲將,魏之大臣,寧有此事!使其若此,
 有以知魏德之不長。」安遂不之逼。安又問曰:「君書何如君家尊?」答曰:「故當不同。」安曰:「外論不爾。」答曰:「人那得知!」尋除建威將軍、吳興太守,徵拜中書令。



 及安薨,贈禮有同異之議,惟獻之、徐邈共明安之忠勛。獻之乃上疏曰:「故太傅臣安少振玄風,道譽泮溢。弱冠遐棲,則契齊箕皓;應運釋褐,而王猷允塞。及至載宣威靈,彊猾消殄。功勛既融,投AX高讓。且服事先帝,眷隆布衣。陛下踐阼,陽秋尚富,盡心竭智以輔聖明。考其潛躍始終,事情繾綣,實大晉之俊輔,義篤於曩臣矣。伏惟陛下留心宗臣,澄神於省察。」孝武帝遂加安殊禮。



 未幾,獻之遇疾,家人
 為上章,道家法應首過,問其有何得失。對曰:「不覺餘事,惟憶與郗家離婚。」獻之前妻,郗曇女也。俄而卒於官。安僖皇后立,以后父追贈侍中、特進、光祿大夫、太宰,謚曰憲。無子,以兄子靜之嗣,位至義興太守。時議者以為羲之草隸,江左中朝莫有及者,獻之骨力遠不及父,而頗有媚趣。桓玄雅愛其父子書,各為一帙,置左右以玩之。始羲之所與共游者許邁。



 許邁,字叔玄,一名映,丹陽句容人也。家世士族,而邁少恬靜,不慕仕進。未弱冠,嘗造郭璞,璞為之筮,遇《泰》之《大畜》,
 其上六爻發。璞謂曰:「君元吉自天,宜學升遐之道。」時南海太守鮑靚隱跡潛遁,人莫之知。邁乃往候之,探其至要。父母尚存,未忍違親。謂餘杭懸霤山近延陵之茅山,是洞庭西門,潛通五嶽,陳安世、茅季偉常所遊處,於是立精舍於懸霤,而往來茅嶺之洞室,放絕世務,以尋仙館,朔望時節還家定省而已。父母既終,乃遣婦孫氏還家,遂攜其同志遍游名山焉。初採藥於桐廬縣之桓山,餌術涉三年,時欲斷穀。以此山近人,不得專一,四面籓之,好道之徒欲相見者,登樓與語,以此為樂。常服氣,一氣千餘息。永和二年,移入臨安西山,登巖茹芝,眇爾自
 得,有終焉之志。乃改名玄,字遠游。與婦書告別,又著詩十二首,論神仙之事焉。羲之造之,未嘗不彌日忘歸,相與為世外之交。玄遺羲之書云:「自山陰南至臨安,多有金堂玉室,仙人芝草,左元放之徒,漢末諸得道者皆在焉。」羲之自為之傳,述靈異之跡甚多,不可詳記。玄自後莫測所終,好道者皆謂之羽化矣。



 制曰:書契之興,肇乎中古,繩文鳥跡,不足可觀。末代去朴歸華,舒箋點翰,爭相跨尚,競其工拙。伯英臨池之妙,無復餘蹤;師宜懸帳之奇,罕有遺跡。逮乎鐘王以降,略可言焉。鐘雖擅美一時,亦為回絕,論其盡善,或有所疑。
 至於布纖濃,分疏密,霞舒雲卷,無所間然。但其體則古而不今,字則長而逾制,語其大量,以此為瑕。獻之雖有父風,殊非新巧。觀其字勢疏瘦,如隆冬之枯樹;覽其筆蹤拘束,若嚴家之餓隸。其枯樹也,雖槎枿而無屈伸;其餓隸也,則羈羸而不放縱。兼斯二者,故翰墨之病歙!子雲近出,擅名江表,然僅得成書,無丈夫之氣,行行若縈春蚓,字字如綰秋蛇;臥王蒙於紙中,坐徐偃於筆下;雖禿千兔之翰,聚無一毫之筋,窮萬穀之皮,斂無半分之骨;以茲播美,非其濫名邪!此數子者,皆譽過其實。所以詳察古今,研精篆素,盡善盡美,其惟王逸少乎!觀其點曳
 之工,裁成之妙,煙霏露結,狀若斷而還連;鳳翥龍蟠,勢如斜而反直。玩之不覺為倦,覽之莫識其端,心慕手追,此人而已。其餘區區之類,何足論哉!



\end{pinyinscope}