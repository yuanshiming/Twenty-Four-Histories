\article{載記第十九 姚泓}

\begin{pinyinscope}

 姚泓



 姚泓,字元子,興之長子也。孝友寬和,而無經世之用,又多疾病,興將以為嗣而疑焉。久之,乃立為太子。興每征伐巡游,常留總後事。博學善談論,尤好詩詠。尚書王尚、黃門郎段章、尚書郎富允文以儒術侍講,胡義周、夏侯稚以文章游集。時尚書王敏、右丞郭播以刑政過寬,議欲峻制,泓曰:「人情挫辱,則壯厲之心生;政教煩苛,則茍
 免之行立。上之化下,如風靡草。君等參贊朝化,弘昭政軌,不務仁恕之道,惟欲嚴法酷刑,豈是安上馭下之理乎!」敏等遂止。泓受經於博士淳于岐。岐病,泓親詣省疾,拜于床下。自是公侯見師傅皆拜焉。



 興之如平涼也,馮翊人劉厥聚眾數千,據萬年以叛。泓遣鎮軍彭白狼率東宮禁兵討之,斬厥,赦其餘黨。諸將咸勸泓曰:「殿下神算電發,蕩平醜逆,宜露布表言,廣其首級,以慰遠近之情。」泓曰:「主上委吾後事,使式遏寇逆。吾綏御失和,以長姦寇,方當引咎責躬,歸罪行間,安敢過自矜誕,以重罪責乎!」其右僕射韋華聞而謂河南太守慕容築曰:「皇太
 子實有恭惠之德,社稷之福也。」其弟弼有奪嫡之謀,泓恩撫如初,未嘗見於色。姚紹每為弼羽翼,泓亦推心宗事,弗以為嫌。及僭立,任紹以兵權,紹亦感而歸誠,卒守其忠烈。其明識寬裕,皆此類也。



 興既死,秘不發喪。南陽公姚愔及大將軍尹元等謀為亂,泓皆誅之。命其齊公姚恢殺安定太守呂超,恢久乃誅之。泓疑恢有陰謀,恢自是懷貳,陰聚兵甲焉。泓發喪,以義熙十二年僭即帝位,大赦殊死已下,改元永和,廬于諮議堂。既葬,乃親庶政,內外百僚增位一等,令文武各盡直言,政有不便於時,事有光益宗廟者,極言勿有所諱。



 初,興徙李閏羌三
 千家於安定,尋徙新支。至是,羌酋黨容率所部叛還,遣撫軍姚贊討之。容降,徙其豪右數百戶于長安,餘遣還李閏。北地太守毛雍據趙氏塢以叛于泓,姚紹討擒之。姚宣時鎮李閏,未知雍敗,遣部將姚佛生等來衛長安。眾既發,宣參軍韋宗姦諂好亂,說宣曰:「主上初立,威化末著,勃勃彊盛,侵害必深,本朝之難未可弭也。殿下居維城之任,宜深慮之。邢望地形險固,總三方之要,若能據之,虛心撫禦,非但克固維城,亦霸王之業也。」宣乃率戶三萬八千,棄李閏,南保邢望。宣既南移,諸羌據李閏以叛,紹進討破之。宣詣紹歸罪,紹怒殺之。初,宣在邢望,
 泓遣姚佛生諭宣,佛生遂贊成宣計。紹數其罪,又戮之。



 泓下書,士卒死王事,贈以爵位,永復其家。將封宮臣十六人五等子男,姚贊諫曰:「東宮文武,自當有守忠之誠,未有赫然之效,何受封之多乎?」泓曰:「懸爵於朝,所以懲勸來效,標明盛德。元子遭家不造,與宮臣同此百憂,獨享其福,得不愧於心乎!」贊默然。姚紹進曰:「陛下不忘報德,封之是也,古者敬其事,命之以始,可須來春,然後議之。」乃止。並州、定陽、貳城胡數萬落叛泓,入于平陽,攻立義姚成都于匈奴堡,推匈奴曹弘為大單于,所在殘掠。征東姚懿自蒲阪討弘,戰于平陽,大破之,執弘,送于長
 安,徙其豪右萬五千落于雍州。



 仇池公楊盛攻陷祁山,執建節王總,遂逼秦州。泓遣後將軍姚平救之,盛引退。姚嵩與平追盛及于竹嶺,姚贊率隴西太守姚秦都、略陽太守王煥以禁兵赴之。贊至清水,嵩為盛所敗,嵩及秦都、王煥皆戰死。贊至秦州,退還仇池。先是,天水冀縣石鼓鳴,聲聞數百里,野雉皆雊。秦州地震者三十二,殷殷有聲者八,山崩舍壞,咸以為不祥。及嵩將出,群僚固諫止之。嵩曰:「若有不祥,此乃命也,安所逃乎!」遂及於難。識者以為秦州泓之故鄉,將滅之徵也。



 赫連勃勃攻陷陰密,執秦州刺史姚軍都,坑將士五千餘人。軍都真目
 厲聲數勃勃殘忍之罪,不為之屈,勃勃怒而殺之。勃勃既剋陰密,進兵侵雍,嶺北雜戶悉奔五將山。征北姚恢棄安定,率戶五千奔新平,安定人胡儼、華韜等率眾距恢,恢單騎歸長安。立節彌姐成、建武裴岐為儼所殺,鎮西姚諶委鎮東走。勃勃遂據雍,抄掠郿城。姚紹及征虜尹昭、鎮軍姚洽等率步騎五萬討勃勃,姚恢以精騎一萬繼之。軍次橫水,勃勃退保安定,胡儼閉門距之,殺鮮卑數千人,據安定以降。紹進兵躡勃勃,戰于馬鞍阪,敗之,追至朝那,不及而還。



 楊盛遣兄子倦入寇長蛇。平陽氐茍渴聚眾千餘,據五丈原以叛,遣鎮遠姚萬、恢武姚
 難討之,為渴所敗。姚諶討渴,擒之。泓使輔國斂曼嵬、前將軍姚光兒討楊倦于陳倉,倦奔于散關。勃勃遣兄子提南侵池陽,車騎姚裕、前將軍彭白狼、建義蛇玄距卻之。



 尋而晉太尉劉裕總大軍伐泓,次于彭城,遣冠軍將軍檀道濟、龍驤將軍王鎮惡入自淮、肥,攻漆丘、項城,將軍沈林子自汴入河,攻倉垣。泓將王茍生以漆丘降鎮惡,徐州刺史姚掌以項城降道濟,王師遂入潁口,所至多降服。惟新蔡太守董遵固守不降,道濟攻破之,縛遵而致諸軍門。遵厲色曰:「古之王者伐國,待士以禮。君奈何以不義行師,待國土以非禮乎。」道濟怒殺之。姚紹聞
 王師之至,還長安,言於泓曰:「晉師已過許昌,豫州、安定孤遠,卒難救衛,宜遷諸鎮戶內實京畿,可得精兵十萬,足以橫行天下。假使二寇交侵,無深害也。如其不爾,晉侵豫州,勃勃寇安定者,將若之何!事機已至,宜在速決。」其左僕射梁喜曰:「齊公恢雄勇有威名,為嶺北所憚,鎮人已與勃勃深仇,理應守死無貳,勃勃終不能棄安定遠寇京畿。若無安定,虜馬必及於郿、雍。今關中兵馬足距晉師,豈可未有憂危先自削損也。」泓從之。吏部郎懿橫密言於泓曰:「齊公恢於廣平之難有忠勳於陛下,自陛下龍飛紹統,未有殊賞以答其意。今外則致之死地,
 內則不豫朝權,安定人自以孤危逼寇,欲思南遷者十室而九,若擁精兵四萬,鼓行而向京師,得不為社稷之累乎!宜徵還朝廷,以慰其心。」泓曰:「恢若懷不逞之心,征之適所以速禍耳。」又不從。



 王師至成皋,征南姚洸時鎮洛陽,馳使請救。泓遣越騎校尉閻生率騎三千以赴之,武衛姚益男將步卒一萬助守洛陽,又遣征東、並州牧姚懿南屯陜津為之聲援。洸部將趙玄說洸曰:「今寇逼已深,百姓駭懼,眾寡勢殊,難以應敵。宜攝諸戍兵士,固守金墉,以待京師之援,不可出戰。如脫不捷,大事去矣。金墉既固,師無損敗,吳寇終不敢越金墉而西。困之於
 堅城之下,可以坐制其弊。」時洸司馬姚禹潛通于道濟,主簿閻恢、楊虔等皆禹之黨,嫉玄忠誠,咸共毀之,固勸洸出戰。洸從之,乃遣玄率精兵千餘南守柏谷塢,廣武石無諱東戍鞏城,以距王師。玄泣謂洸曰:「玄受三帝重恩,所守正死耳。但明公不用忠臣之言,為姦孽所誤,後必悔之,但無及耳。」會陽城及成皋、滎陽、武牢諸城悉降,道濟等長驅而至。無諱至石關,奔還。玄與晉將毛德祖戰于柏谷,以眾寡而敗,被瘡十餘,據地大呼,玄司馬騫鑒冒刃抱玄而泣,玄曰:「吾瘡已重,君宜速去。」鑒曰:「若將軍不濟,當與俱死,去將安之!」皆死於陣。姚禹踰城奔于
 王師。道濟進至洛陽、洸懼,遂降。時閻生至新安,益男至湖城,會洛陽已沒,遂留屯不進。



 姚懿嶮薄,惑於信受,其司馬孫暢姦巧傾佞,好亂樂禍,勸懿襲長安,誅姚紹,廢泓自立。懿納之,乃引兵至陜津,散穀以賜河北夷夏,欲虛損國儲,招引和戎諸羌,樹已私惠。懿左常侍張敞、侍郎左雅固諫懿曰:「殿下以母弟之親,居分陜之重,安危休戚,與國共之。漢有七國之難,實賴梁王。今吳寇內侵,四州傾沒,西虜擾邊,秦、涼覆敗,朝廷之危有同累卵,正是諸侯勤王之日。穀者,國之本也,而今散之。若朝廷問殿下者,將何辭以報?」懿怒,笞而殺之。泓聞之,召姚紹等
 密謀於朝堂。紹曰:「懿性識鄙近,從物推移,造成此事,惟當孫暢耳。但馳使徵暢,遣撫軍贊據陜城,臣向潼關為諸軍節度,若暢奉詔而至者,臣當遣懿率河東見兵共平吳寇。如其逆釁已成,違距詔敕者,當明其罪於天下,聲鼓以擊之。」泓曰:「叔父之言,社稷之計也。」於是遣姚贊及冠軍司馬國璠、建義蛇玄屯陜津,武衛姚驢屯潼關。



 懿遂舉兵僭號,傳檄州郡,欲運匈奴堡穀以給鎮人。寧東姚成都距之,懿乃卑辭招誘,深自結託,送佩刀為誓,成都送以呈泓。懿又遣驍騎王國率甲士數百攻成都,成都擒國,囚之,遣讓懿曰:「明公以母弟之親,受推轂之
 寄,今社稷之危若綴旒然,宜恭恪憂勤,匡輔王室。而更包藏奸宄,謀危宗廟,三祖之靈豈安公乎!此鎮之糧,一方所寄,鎮人何功,而欲給之!王國為蛇畫足,國之罪人,已就囚執,聽詔而戮之。成都方糾合義眾,以懲明公之罪,復須大兵悉集,當與明公會於河上。」乃宣告諸城,勉以忠義,厲兵秣馬,徵發義租。河東之兵無詣懿者,懿深患之。臨晉數千戶叛應懿。姚紹濟自薄津,擊臨晉叛戶,大破之,懿等震懼。鎮人安定郭純、王奴等率眾圍懿。紹入于蒲阪,執懿囚之,誅孫暢等。



 泓以內外離叛,王師漸逼,歲旦朝群臣于其前殿,悽然流涕,群臣皆泣。時征北
 姚恢率安定鎮戶三萬八千,焚燒室宇,以車為方陣,自北雍州趣長安,自稱大都督、建義大將軍,移檄州郡,欲除君側之惡。揚威姜紀率眾奔之。建節彭完都聞恢將至,棄陰密,奔還長安。恢至新支,姜紀說恢曰:「國家重將在東,京師空虛,公可輕兵徑襲,事必剋矣。」恢不從,乃南攻郿城。鎮西姚諶為恢所敗,恢軍勢彌盛,長安大震。泓馳使徵紹,遣姚裕及輔國胡翼度屯于灃西。扶風太守姚雋、安夷護軍姚墨蠡、建威姚娥都、揚威彭蠔皆懼而降恢。恢舅茍和時為立節將軍,守忠不貳,泓召而謂之曰:「眾人咸懷去就,卿何能自安邪?」和曰:「若天縱妖賊,得
 肆其逆節者,舅甥之理,不待奔馳而加親。如其罪極逆銷,天盈其罰者,守忠執志,臣之體也。違親叛君,臣之所恥。」泓善其忠恕,加金章紫綬。姚紹率輕騎先赴難,使姚洽、司馬國璠將步卒三萬赴長安。恢從曲牢進屯杜成,紹與恢相持于靈臺。姚贊聞恢漸逼,留寧朔尹雅為弘農太守,守潼關,率諸軍還長安。泓謝贊曰:「元子不能崇明德義,導率群下,致禍起蕭墻,變自同氣,既上負祖宗,亦無顏見諸父。懿始構逆滅亡,恢復擁眾內叛,將若之何?」贊曰:「懿等所以敢稱兵內侮者,諒由臣等輕弱,無防遏之方故也。」因攘袂大泣曰:「臣與大將軍不滅此賊,終
 不持面復見陛下!」泓於是班賜軍士而遣之。恢眾見諸軍悉集,咸懼而思善,其將齊黃等棄恢而降。恢進軍逼紹,贊自後要擊,大破之,殺恢及其三弟。泓哭之悲慟,葬以公禮。



 至是,王鎮惡至宜陽。毛德祖攻弘農太守尹雅于蠡城,眾潰,德祖使騎追獲之,既而殺晉守者奔固潼關。



 檀道濟、沈林子攻拔襄邑堡,建威薛帛奔河東。道濟自陜北渡,攻蒲阪,使將軍茍卓攻匈奴堡,為泓寧東姚成都所敗。泓遣姚驢救蒲阪,胡翼度據潼關。泓進紹太宰、大將軍、大都督、都督中外諸軍事、假黃鋮,改封魯公,侍中、司隸、宗正、節錄並如故,朝之大政皆往決焉。紹
 固辭,弗許。於是遣紹率武衛姚鸞等步騎五萬,距王師於潼關。姚驢與并州刺史尹昭為表裏之勢,夾攻道濟。道濟深壁不戰,沈林子說道濟曰:「今蒲阪城堅池濬,非可卒剋,攻之傷眾,守之引日,不如棄之,先事潼關。潼關天限,形勝之地,鎮惡孤軍,勢危力寡,若使姚紹據之,則難圖矣。如剋潼關,紹可不戰而服。」道濟從之,乃棄蒲阪,南向潼關。姚贊率禁兵七千,自渭北而東,進據蒲津。劉裕使沈田子及傅弘之率眾萬餘人入上洛,所在多委城鎮奔長安。田子等進及青泥,姚紹方陣而前,以距道濟。道濟固壘不戰,紹乃攻其西營,不剋,遂以大眾逼之。
 道濟率王敬、沈林子等逆衝紹軍,將士驚散,引還定城。紹留姚鸞守險,絕道濟糧道。



 時裕別將姚珍入自子午,竇霸入自洛谷,眾各數千人。泓遣姚萬距霸,姚彊距珍。姚鸞遣將尹雅與道濟司馬徐琰于潼關南,為琰所獲,送之劉裕。裕以雅前叛,欲殺之。雅曰:「前活本在望外,今死寧不甘心。明公將以大義平天下,豈可使秦無守信之臣乎!」裕嘉而免之。



 泓遣給事黃門侍郎姚和都屯於堯柳,以備田子。姚紹謂諸將曰:「道濟等遠來送死,眾旅不多,嬰壘自固者,正欲曠日持久,以待繼援耳。吾欲分軍還據閿鄉,以絕其糧運,不至一月,道濟之首可懸
 之麾下矣。濟等既沒,裕計自沮。」諸將咸以為然。其將胡翼度曰:「軍勢宜集不可以分,若偏師不利,人心駭懼,胡可以戰!」紹乃止。薛帛據河曲以叛。紹分道置諸軍為掎角之勢,遣輔國胡翼度據東原,武衛姚鸞營於大路,與晉軍相接。沈林子簡精銳銜枚夜襲之,鸞眾潰戰死,士卒死者九千餘人。



 姚贊屯于河上,遣恢武姚難運蒲阪穀以給其軍,至香城,為王師所敗。時泓遣姚諶守堯柳,姚和都討薛帛于河東,聞王師要難,乃兼道赴救,未至而難敗,固破裕裨將于河曲,遂屯蒲阪。姚贊為林子所敗,單馬奔定城。紹遣左長史姚洽及姚墨蠡等率騎三
 千屯于河北之九原,欲絕道濟諸縣租輸。洽辭曰:「夫小敵之堅,大敵之擒。今兵眾單弱,而遠在河外,雖明公神武,然鞭短勢殊,恐無所及。」紹不聽。沈林子率眾八千,耍洽于河上,洽戰死,眾皆沒。紹聞洽等敗,忿恚發病,託姚贊以後事,使姚難屯關西,紹嘔血而死。



 泓以晉師之逼,遣使乞師于魏。魏遣司徒、南平公拔拔嵩,正直將軍、安平公乙旃眷,進據河內,游擊將軍王洛生屯於河東,為泓聲援。



 劉裕次于陜城,遣沈林子率精兵萬餘,越山開道,會沈田子等于青泥,將攻堯柳。泓使姚裕率步騎八千距之,泓躬將大眾繼發。裕為田子所敗,泓退次于灞
 上,關中郡縣多潛通于王師。劉裕至潼關,遣將軍朱超石、徐猗之會薛帛于河北,以攻蒲阪。姚贊距裕于關西,姚難屯于香城。裕遣王鎮惡、王敬自秋社西渡渭,以逼難軍。鎮東姚璞及姚和都擊敗猗之等於蒲阪,猗之遇害,超石棄其眾奔于潼關。姚贊遣司馬休之及司馬國璠自軹關向河內,引魏軍以躡裕後。姚難既為鎮惡所逼,引師而西。時大霖雨,渭水泛溢,贊等不得北渡。鎮惡水陸兼進,追及姚難。泓自灞上還軍,次于石橋以援之。贊退屯鄭城。鎮北姚彊率郡人數千,與姚難陣于涇上,以距鎮惡。鎮惡遣毛德祖擊彊,大敗,彊戰死,難遁還長
 安。



 劉裕進據鄭城。泓使姚裕、尚書龐統屯兵宮中,姚洸屯于灃西,尚書姚白瓜徙四軍雜戶入長安,姚丕守渭橋,胡翼度屯石積,姚贊屯霸東,泓軍於逍遙園。鎮惡夾渭進兵,破姚丕於渭橋。泓自逍遙園赴之,逼水地狹,因丕之敗,遂相踐而退。姚諶及前軍姚烈、左衛姚寶安、散騎王帛、建武姚進、揚威姚蠔、尚書右丞孫玄等皆死於陣,泓單馬還宮。鎮惡入自平朔門,泓與姚裕等數百騎出奔于石橋。讚聞泓之敗也,召將士告之,眾皆以刀擊地,攘袂大泣。胡翼度先與劉裕陰通,是日棄眾奔裕。贊夜率諸軍,將會泓於石橋,王師已固諸門,贊軍不得入,
 眾皆驚散。



 泓計無所出,謀欲降于裕。其子佛念,年十一,謂泓曰:「晉人將逞其欲,終必不全,願自裁決。」泓憮然不答。佛念遂登宮墻自投而死。泓將妻子詣壘門而降。贊率宗室子弟百餘人亦降于裕,裕盡殺之,餘宗遷于江南。送泓于建康市斬之,時年三十,在位二年。建康百里之內,草木皆燋死焉。



 姚萇以孝武太元九年僭立,至泓三世,以安帝義熙十三年而滅,凡三十二年。



 史臣曰:自長江徙御,化龍創業,巨寇乘機而未寧,戎馬交馳而不息,晦重氛於六漠,鼓洪流於八際,天未厭亂,兇旅實繁。弋仲越自金方,言歸石氏,抗直詞於暴主,闡
 忠訓於危朝,貽厥之謀,在乎歸順,鳴哀之義,有足稱焉。景國弱歲英奇,見方孫策,詳其幹識,無忝斯言,遂踐迷途,良可悲矣!



 景茂因仲襄之緒,躡苻亡之會,嘯命群豪,恢弘霸業,假容沖之銳,俯定函、秦;挫雷惡之鋒,載寧東北。在茲姦略,實冠凶徒。列樹而表新營,雖云效績;薦棘而陵舊主,何其不仁!安枕而終,斯為幸也。



 子略剋摧勍敵,荷成先構,虛襟訪道,側席求賢,敦友弟以睦其親,明賞罰以臨其下,英髦盡節,爪牙畢命。取汾、絳,陷許、洛,款僭燕而籓偽蜀,夷隴右而靜河西,俗阜年豐,遠安邇輯,雖楚莊、秦穆何以加焉!既而逞志矜功,弗虞後患。委涼
 都於禿髮,授朔方於赫連,專己生災,邊城繼陷,距諫招禍,蕭墻屢發,戰無寧歲,人有危心。豈宜騁彼雄圖,被深恩於介士;翻崇詭說,加殊禮於桑門!當有為之時,肄無為之業,麗衣腴食,殆將萬數,析實談空,靡然成俗。夫以漢朝殷廣,猶鄙鴻都之費;況乎偽境日侵,寧堪永貴之役!儲用殫竭,山林有稅,政荒威挫,職是之由,坐致淪胥,非天喪也。



 元子以庸懦之質,屬傾擾之餘,內難方殷,外禦斯輟。王師杖順,弭節而下長安;凶嗣失圖,繫組而降軹道。物極則反,抑斯之謂歟!



 贊曰:弋仲剛烈,終表奇節。襄實英果,萇惟姦傑。興始崇
 構,泓遂摧滅。貽誡將來,無踐危轍。



\end{pinyinscope}