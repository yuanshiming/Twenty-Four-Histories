\article{列傳第三十六 劉弘陶侃}

\begin{pinyinscope}

 劉弘陶侃



 劉弘,字和季,沛國相人也。祖馥,魏揚州刺史。父靖,鎮北將軍。弘有幹略政事之才,少家洛陽,與武帝同居永安里,又同年,共研席。以舊恩起家太子門大夫,累遷率更令,轉太宰長史。張華甚重之。由是為寧朔將軍、假節、監幽州諸軍事,領烏丸校尉,甚有威惠,寇盜屏迹,為幽朔所稱。以勳德兼茂,封宣城公。太安中,張昌作亂,轉使持
 節、南蠻校尉、荊州刺史,率前將軍趙驤等討昌,自方城至宛、新野,所向皆平。及新野王歆之敗也,以弘代為鎮南將軍、都督荊州諸軍事,餘官如故。弘遣南蠻長史陶侃為大都護,參軍蒯恆為義軍督護,牙門將皮初為都戰帥,進據襄陽。張昌并軍圍宛,敗趙驤軍,弘退屯梁。侃、初等累戰破昌,前後斬首數萬級。及到官,昌懼而逃,其眾悉降,荊土平。



 初,弘之退也,范陽王虓遣長水校尉張奕領荊州。弘至,奕不受代,與兵距弘。弘遣軍討奕,斬之,表曰:「臣以凡才,謬荷國恩,作司方州,奉辭伐罪,不能奮揚雷霆,折衝萬里,軍退於宛,分受顯戮。猥蒙含宥,被遣
 之職,即進達所鎮。而范陽王虓先遣前長水校尉張奕領荊州,臣至,不受節度,擅舉兵距臣。今張昌姦黨初平,昌未梟擒,益梁流人蕭條猥集,無賴之徒易相扇動,飆風駭蕩,則滄海橫波,茍患失之,無所不至,比須表上,慮失事機,輒遣軍討奕,即梟其首。奕雖貪亂,欲為荼毒,由臣劣弱,不勝其任,令奕肆心,以勞資斧,敢引覆餗之刑,甘受專輒之罪。」詔曰:「將軍文武兼資,前委方夏,宛城不守,咎由趙驤。將軍所遣諸軍,剋滅群寇,張奕貪禍,距違詔命。將軍致討,傳首闕庭,雖有不請之嫌,古人有專之之義。其恢宏奧略,鎮綏南海,以副推轂之望焉。」張昌竄
 于下雋山,弘遣軍討昌,斬之,悉降其眾。



 時荊部守宰多闕,弘請補選,帝從之。弘迺敘功銓德,隨才補授,甚為論者所稱。乃表曰:「被中詔,敕臣隨資品選,補諸缺吏。夫慶賞刑威,非臣所專,且知人則哲,聖帝所難,非臣闇蔽所能斟酌。然萬事有機,豪釐宜慎,謹奉詔書,差所應用。蓋崇化莫若貴德,則所以濟屯,故太上立德,其次立功也。頃者多難,淳朴彌凋,臣輒以徵士伍朝補零陵太守,庶以懲波蕩之弊,養退讓之操。臣以不武,前退於宛,長史陶侃、參軍蒯恆、牙門皮初,戮力致討,蕩滅姦凶,侃恆各以始終軍事,初為都戰帥,忠勇冠軍,漢沔清肅,實初等
 之勳也。《司馬法》『賞不踰時』,欲人知為善之速福也。若不超報,無以勸徇功之士,慰熊羆之志。臣以初補襄陽太守,侃為府行司馬,使典論功事,恆為山都令。詔惟令臣以散補空缺,然沶鄉令虞潭忠誠烈正,首唱義舉,舉善以教,不能者勸,臣輒特轉潭補醴陵令。南郡廉吏仇勃,母老疾困,賊至守衛不移,以致拷掠,幾至隕命。尚書令史郭貞,張昌以為尚書郎,欲訪以朝議,遁逃不出,昌質其妻子,避之彌遠。勃孝篤著於臨危,貞忠厲於強暴,雖各四品,皆可以訓獎臣子,長益風教。臣輒以勃為歸鄉令,貞為信陵令。皆功行相參,循名校實,條列行狀,公文
 具上。」朝廷以初雖有功,襄陽又是名郡,名器宜慎,不可授初,乃以前東平太守夏侯陟為襄陽太守,餘並從之。陟,弘之婿也。弘下教曰:「夫統天下者,宜與天下一心;化一國者,宜與一國為任。若必姻親然後可用,則荊州十郡,安得十女婿然後為政哉!」乃表「陟姻親,舊制不得相監。皮初之勳宜見酬報。」詔聽之。



 弘於是勸課農桑,寬刑省賦,歲用有年,百姓愛悅。弘嘗夜起,聞城上持更者歎聲甚苦,遂呼省之。兵年過六十,羸疾無襦。弘愍之,乃謫罰主者,遂給韋袍復帽,轉以相付。舊制,峴方二山澤中不聽百姓捕魚,弘下教曰:「禮,名山大澤不封,與共其利。
 今公私并兼,百姓無復厝手地,當何謂邪!速改此法。」又「酒室中云齊中酒、聽事酒、猥酒,同用曲米,而優劣三品。投醪當與三軍同其薄厚,自今不得分別。」時益州刺史羅尚為李特所敗,遣使告急,請糧。弘移書贍給,而州府綱紀以運道懸遠,文武匱乏,欲以零陵一運米五千斛與尚。弘曰:「諸君未之思耳。天下一家,彼此無異,吾今給之,則無西顧之憂矣。」遂以零陵米三萬斛給之。尚賴以自固。於時流人在荊州十餘萬戶,羈旅貧乏,多為盜賊。弘乃給其田種糧食,擢其賢才,隨資敘用。時總章太樂伶人,避亂多至荊州,或勸可作樂者。弘曰:「昔劉景升以
 禮壞樂崩,命杜夔為天子合樂,樂成,欲庭作之。夔曰:『為天子合樂而庭作之,恐非將軍本意。』吾常為之歎息。今主上蒙塵,吾未能展效臣節,雖有家伎,猶不宜聽,況御樂哉!」乃下郡縣,使安慰之,須朝廷旋返,送還本署。論平張昌功,應封次子一人縣侯,弘上疏固讓,許之。進拜侍中、鎮南大將軍、開府儀同三司。



 惠帝幸長安,河間王顒挾天子,詔弘為劉喬繼援。弘以張方殘暴,知顒必敗,遣使受東海王越節度。時天下大亂,弘專督江漢,威行南服。前廣漢太守辛冉說弘以從橫之事,弘大怒,斬之。河間王顒使張光為順陽太守,南陽太守衛展說弘曰:「彭
 城王前東奔,有不善之言。張光,太宰腹心,宜斬光以明向背。」弘曰:「宰輔得失,豈張光之罪!危人自安,君子弗為也。」展深恨之。



 陳敏寇揚州,引兵欲西上,弘乃解南蠻,以授前北軍中候蔣超,統江夏太守陶侃、武陵太守苗光,以大眾屯於夏口。又遣治中何松領建平、宜都、襄陽三郡兵,屯巴東,為羅尚後繼。又加南平太守應詹寧遠將軍,督三郡水軍,繼蔣超。侃與敏同郡,又同歲舉吏,或有間侃者,弘不疑之。乃以侃為前鋒督護,委以討敏之任。侃遣子及兄子為質,弘遣之曰:「賢叔征行,君祖母年高,便可歸也。匹夫之交尚不負心,何況大丈夫乎!」陳敏竟
 不敢窺境。永興三年,詔進號車騎將軍,開府及餘官如故。



 弘每有興廢,手書守相,丁寧款密,所以人皆感悅,爭赴之,咸曰:「得劉公一紙書,賢於十部從事。」及東海王越奉迎大駕,弘遣參軍劉盤為督護,率諸軍會之。盤既旋,弘自以老疾,將解州及校尉,適分授所部,未及表上,卒于襄陽。士女嗟痛,若喪所親矣。



 初,成都王穎南奔,欲之本國,弘距之。及弘卒,弘司馬郭勱欲推穎為主,弘子璠追遵弘志,於是墨絰率府兵計勱,戰於濁水,斬之,襄沔肅清,初,東海王越疑弘與劉喬貳于己,雖下節度,心未能安。及弘距穎,璠又斬勵,朝廷嘉之。越手書與璠贊美
 之,表贈弘新城郡公,謚曰元。



 以高密王略代鎮,寇盜不禁,詔起璠為順陽內史,江漢之間翕然歸心。及略薨,山簡代之。簡至,知璠得眾心,恐百姓逼以為主,表陳之,由是征璠為越騎校尉。璠亦深慮逼迫,被書,便輕至洛陽,然後遣迎家累。僑人侯脫、路難等相率衛送至都,然後辭去。南夏遂亂。父老追思弘,雖《甘棠》之詠召伯,無以過也。


陶侃,字士行,本鄱陽人也。吳平,徙家廬江之尋陽。父丹,吳揚武將軍。侃早孤貧,為縣吏。鄱陽孝廉范逵嘗過侃,
 時倉卒無以待賓,其母乃截髮得雙
 \gezhu{
  髟皮}
 ,以易酒肴,樂飲極歡,雖僕從亦過所望。及逵去,侃追送百餘里。逵曰:「卿欲仕郡乎?」侃曰:「欲之,困於無津耳。」逵過廬江太守張夔,稱美之。夔召為督郵,領樅陽令。有能名,遷主簿。會州部從事之郡,欲有所按,侃閉門部勒諸吏,謂從事曰:「若鄙郡有違,自當明憲直繩,不宜相逼。若不以禮,吾能禦之。」從事即退。夔妻有疾,將迎醫於數百里。時正寒雪,諸綱紀皆難之,侃獨曰:「資於事父以事君。小君,猶母也,安有父母之疾而不盡心乎!」乃請行。眾咸服其義。長沙太守萬嗣過廬江,見侃,虛心敬悅,曰:「君終當有大名。」命其子
 與之結友而去。



 夔察侃為孝廉,至洛陽,數詣張華。華初以遠人,不甚接遇。侃每往,神無忤色。華後與語,異之。除郎中。伏波將軍孫秀以亡國支庶,府望不顯,中華人士恥為掾屬,以侃寒宦,召為舍人。時豫章國郎中令楊晫,侃州里也,為鄉論所歸。侃詣之,晫曰:「《易》稱『貞固足以幹事』,陶士行是也。」與同乘見中書郎顧榮,榮甚奇之。吏部郎溫雅謂晫曰:「奈何與小人共載?」晫曰:「此人非凡器也。」尚書樂廣欲會荊揚士人,武庫令黃慶進侃於廣。人或非之,慶曰:「此子終當遠到,復何疑也!」。慶後為吏部令史,舉侃補武岡令。與太守呂岳有嫌,棄官歸,為郡小中正。



 會劉弘為荊州刺史,將之官,辟侃為南蠻長史,遣先向襄陽討賊張昌,破之。弘既至,謂侃曰:「吾昔為羊公參軍,謂吾其後當居身處。今相觀察,必繼老夫矣。」後以軍功封東鄉侯,邑千戶。



 陳敏之亂,弘以侃為江夏太守,加鷹揚將軍。侃備威儀,迎母官舍,鄉里榮之。敏遣其弟恢來寇武昌,侃出兵禦之。隨郡內史扈瑰間侃於弘曰:「侃與敏有鄉里之舊,居大郡,統彊兵,脫有異志,則荊州無東門矣。」弘曰:「侃之忠能,吾得之已久,豈有是乎!」侃潛聞之,遽遣子洪及兄子臻詣弘以自固。弘引為參軍,資而遣之。又加侃為督護,使與諸軍並力距恢。侃乃以運船為
 戰艦,或言不可,侃曰:「用官物討官賊,但須列上有本末耳。」於是擊恢,所向必破。侃戎政齊肅,凡有虜獲,皆分士卒,身無私焉。後以母憂去職。嘗有二客來弔,不哭而退,化為雙鶴,沖天而去,時人異之。



 服闋,參東海王越軍事。江州刺史華軼表侃為揚武將軍,使屯夏口,又以臻為參軍。軼與元帝素不平,臻懼難作,託疾而歸,白侃曰:「華彥夏有憂天下之志,而才不足,且與琅邪不平,難將作矣。」侃怒,遣臻還軼。臻遂東歸於帝。帝見之,大悅,命臻為參軍,加侃奮威將軍,假赤幢曲蓋軺車、鼓吹。侃乃與華軼告絕。



 頃之,遷龍驤將軍、武昌太守。時天下饑荒,山夷
 多斷江劫掠。侃令諸將詐作商船以誘之。劫果至,生獲數人,是西陽王羕之左右。侃即遣兵逼羕,令出向賊,侃整陣於釣臺為後繼。羕縛送帳下二十人,侃斬之。自是水陸肅清,流亡者歸之盈路,侃竭資振給焉。又立夷市於郡東,大收其利。而帝使侃擊杜弢,令振威將軍周訪、廣武將軍趙誘受侃節度。侃令二將為前鋒,兄子輿為左甄,擊賊,破之。時周顗為荊州刺史,先鎮潯水城,賊掠其良口。侃使部將朱伺救之,賊退保泠口。侃謂諸將曰:「此賊必更步向武昌,吾宜還城,晝夜三日行可至。卿等認能忍饑鬥邪?」部將吳寄曰:「要欲十日忍飢,晝當擊賊,
 夜分捕魚,足以相濟。」侃曰:「卿健將也。」賊果增兵來攻,侃使朱伺等逆擊,大破之,獲其輜重,殺傷甚眾。遣參軍王貢告捷於王敦,敦曰:「若無陶侯,便失荊州矣。伯仁方入境,便為賊所破,不知那得刺史?」貢對曰:「鄙州方有事難,非陶龍驤莫可。」敦然之,即表拜侃為使持節、寧遠將軍、南蠻校尉、荊州刺史,領西陽、江夏、武昌,鎮于沌口,又移入沔江。遣朱伺等討江夏賊,殺之。賊王沖自稱荊州刺史,據江陵。王貢還,至竟陵,矯侃命,以杜曾為前鋒大督護,進軍斬沖,悉降其眾。侃召曾不到,貢又恐矯命獲罪,遂與曾舉兵反,擊侃督護鄭攀於沌陽,破之,又敗朱伺
 於沔口。侃欲退入溳中,部將張奕將貳於侃,詭說曰:「賊至而動,眾必不可。」侃惑之而不進。無何,賊至,果為所敗。賊鉤侃所乘艦,侃窘急,走入小船。朱伺力戰,僅而獲免。張奕竟奔于賊。侃坐免官。王敦表以侃白衣領職。



 侃復率周訪等進軍人湘,使都尉楊舉為先驅,擊杜弢,大破之,屯兵于城西。侃之佐史辭詣王敦曰:「州將陶使君孤根特立,從微至著,忠允之功,所在有效。出佐南夏,輔翼劉征南,前遇張昌,後屬陳敏,侃以偏旅,獨當大寇,無征不剋,群醜破滅。近者王如亂北,杜弢跨南,二征奔走,一州星馳,其餘郡縣,所在土崩。侃招攜以禮,懷遠以德,子
 來之眾,前後累至。奉承指授,獨守危阨,人往不動,人離不散。往年董督,徑造湘城,志陵雲霄,神機獨斷。徒以軍少糧懸,不果獻捷。然杜弢懾懼,來還夏口,未經信宿,建平流人迎賊俱叛。侃即回軍溯流,芟夷醜類,至使西門不鍵,華圻無虞者,侃之功也。明將軍愍此荊楚,救命塗炭,使侃統領窮殘之餘,寒者衣之,饑者食之,比屋相慶,有若挾纊。江濱孤危,地非重險,非可單軍獨能保固,故移就高莋,以避其衝。賊輕易先至,大眾在後,侃距戰經日,殺其名帥。賊尋犬羊相結,並力來攻,侃以忠臣之節,義無退顧,被堅執銳,身當戎行,將士奮擊,莫不用命。當
 時死者不可勝數。賊眾參伍,更息更戰。侃以孤軍一隊,力不獨禦,量宜取全,以俟後舉。而主者責侃,重加黜削。侃性謙沖,功成身退,今奉還所受,唯恐稽遲。然某等區區,實恐理失於內,事敗於外,豪釐之差,將致千里,使荊蠻乖離,西嵎不守,脣亡齒寒,侵逼無限也。」敦於是奏復侃官。



 弢將王貢精卒三千,出武陵江,誘五溪夷,以舟師斷官運,徑向武昌。侃使鄭攀及伏波將軍陶延夜趣巴陵,潛師掩其不備,大破之,斬千餘級,降萬餘口。貢遁還湘城。賊中離阻,杜弢遂疑張奕而殺之,眾情益懼,降者滋多。王貢復挑戰,侃遙謂之曰:「杜弢為益州吏,盜用庫
 錢,父死不奔喪。卿本佳人,何為隨之也?天下寧有白頭賊乎!」貢初橫腳馬上,侃言訖,貢斂容下腳,辭色甚順。侃知其可動,復令諭之,截髮為信,貢遂來降。而韜敗走。進剋長沙,獲其將毛寶、高寶、梁堪而還。



 王敦深忌侃功。將還江陵,欲詣敦別,皇甫方回及朱伺等諫,以為不可。侃不從。敦果留侃不遣,左轉廣州刺史、平越中郎將,以王廣為荊州。侃之佐吏將士詣敦請留侃。敦怒,不許。侃將鄭攀、蘇溫、馬俊等不欲南行,遂西迎杜曾以距暠。敦意攀承侃風旨,被甲持矛,將殺侃,出而復迴者數四。侃正色曰:「使君之雄斷,當裁天下,何此不決乎!」因起如廁。諮
 議參軍梅陶、長史陳頒言於敦曰:「周訪與侃親姻,如左右手,安有斷人左手而右手不應者乎!」敦意遂解,於是設盛饌以餞之。侃便夜發。敦引其子瞻參軍。侃既達豫章,見周訪,流涕曰:「非卿外援,我殆不免!」侃因進至始興。



 先是,廣州人背刺史郭訥,迎長沙人王機為刺史。機復遣使詣王敦,乞為交州。敦從之,而機未發。會杜弘據臨賀,因機乞降,勸弘取廣州,弘遂與溫邵及交州秀才劉沈俱謀反。或勸侃且住始興,觀察形勢。侃不聽,直至廣州。弘遣使偽降。侃知其詐,先於封口起發石車。俄而弘率輕兵而至,知侃有備,乃退。侃追擊破之,執劉沈於
 小桂。又遣部將許高討機,斬之,傳首京都。諸將皆請乘勝擊溫邵,侃笑曰:「吾威名已著,何事遣兵,但一函紙自足耳。」於是下書諭之。邵懼而走,追獲於始興。以功封柴桑侯,食邑四千戶。



 侃在州無事,輒朝運百甓於齋外,暮運於齋內。人問其故,答曰:「吾方致力中原,過爾優逸,恐不堪事。」其勵志勤力,皆此類也。



 太興初,進號平南將軍,尋加都督交州軍事。及王敦舉兵反,詔侃以本官領江州刺史,尋轉都督、湘州刺史。敦得志,上侃復本職,加散騎常侍。時交州刺史王諒為賊梁碩所陷,侃遣將高寶進擊平之。以侃領交州刺史。錄前後功,封次子夏為都
 亭侯,進號征南大將軍、開府儀同三司。及王敦平,遷都督荊、雍、益、梁州諸軍事,領護南蠻校尉、征西大將軍、荊州刺史,餘如故。楚郢士女莫不相慶。



 侃性聰敏,勤於吏職,恭而近禮,愛好人倫。終日斂膝危坐,閫外多事,千緒萬端,罔有遺漏。遠近書疏,莫不手答,筆翰如流,未嘗壅滯。引接疏遠,門無停客。常語人曰:「大禹聖者,乃惜寸陰,至於眾人,當惜分陰,豈可逸遊荒醉,生無益於時,死無聞於後,是自棄也。」諸參佐或以談戲廢事者,乃命取其酒器、蒱博之具,悉投之於江,吏將則加鞭撲,曰:「樗蒱者,牧豬奴戲耳!《老》《莊》浮華,非先王之法言,不可行也。君子
 當正其衣冠,攝其威儀,何有亂頭養望自謂宏達邪!」有奉饋者,皆問其所由。若力作所致,雖微必喜,慰賜參倍;若非理得之,則切厲訶辱,還其所饋。嘗出遊,見人持一把未熟稻,侃問:「用此何為?」人云:「行道所見,聊取之耳。」侃大怒曰:「汝既不田,而戲賊人稻!」執而鞭之。是以百姓勤於農殖,家給人足。時造船,木屑及竹頭悉令舉掌之,咸不解所以。後正會,積雪始晴,聽事前餘雪猶濕,於是以屑布地。及桓溫伐蜀,又以侃所貯竹頭作丁裝船。其綜理微密,皆此類也。



 暨蘇峻作逆,京都不守,侃子瞻為賊所害,平南將軍溫嶠要侃同赴朝廷。初,明帝崩,侃不在
 顧命之列,深以為恨,答嶠曰:「吾疆場外將,不敢越局。」嶠固請之,因推為盟主。侃乃遣督護龔登率眾赴嶠,而又追迴。嶠以峻殺其子,重遣書以激怒之。侃妻龔氏亦固勸自行。於是便戎服登舟,星言兼邁,瞻喪至不臨。五月,與溫嶠、庾亮等俱會石頭。諸軍即欲決戰,侃以賊盛,不可爭鋒,當以歲月智計擒之。累戰無功,諸將請於查浦築壘。監軍部將李根建議,請立白石壘。侃不從,曰:「若壘不成,卿當坐之。」根曰:「查浦地下,又在水南,唯白石峻極險固,可容數千人,賊來攻不便,滅賊之術也。」侃笑曰:「卿良將也。」乃從根謀,夜修曉訖。賊見壘大驚。賊攻大業壘,
 侃將救之,長史殷羨曰:「若遣救大業,步戰不如峻,則大事去矣。但當急攻石頭,峻必救之,而大業自解。」侃又從羨言。峻果棄大業而救石頭。諸軍與峻戰陳陵東,侃督護竟陵太守李陽部將彭世斬峻於陣,賊眾大潰。峻弟逸復聚眾。侃與諸軍斬逸於石頭。



 初,庾亮少有高名,以明穆皇后之兄受顧命之重,蘇峻之禍,職亮是由。及石頭平,懼侃致討,亮用溫嶠謀,詣侃拜謝。侃遽止之,曰:「庾元規乃拜陶士行邪!」王導入石頭城,令取故節,侃笑曰:「蘇武節似不如是!」導有慚色,使人屏之。侃旋江陵,尋以為侍中、太尉,加羽葆鼓吹,改封長沙郡公,邑三千戶,賜
 絹八千匹,加都督交、廣、寧七州軍事。以江陵偏遠,移鎮巴陵。遣諮議參軍張誕討五谿夷,降之。



 屬後將軍郭默矯詔襲殺平南將軍劉胤,輒領江州。侃聞之曰:「此必詐也。」遣將軍宋夏、陳脩率兵據湓口,侃以大軍繼進。默遣使送妓婢絹百匹,寫中詔呈侃。參佐多諫曰:「默不被詔,豈敢為此事。若進軍,宜待詔報。」侃厲色曰:「國家年小,不出胸懷。且劉胤為朝廷所禮,雖方任非才,何緣猥加極刑!郭默虓勇,所在暴掠,以大難新除,威網寬簡,欲因隙會騁其從橫耳。」發使上表討默。與王導書曰:「郭默殺方州,即用為方州;害宰相,便為宰相乎?」導答曰:「默居上流之
 勢,加有船艦成資,故苞含隱忍,使其有地。一月潛嚴,足下軍到,是以得風發相赴,豈非遵養時晦以定大事者邪!」侃省書笑曰:「是乃遵養時賊也。」侃既至,默將宗侯縛默父子五人及默將張丑詣侃降,侃斬默等。默在中原,數與石勒等戰,賊畏其勇,聞侃討之,兵不血刃而擒也,益畏侃。蘇峻將馮鐵殺侃子,奔于石勒,勒以為戍將。侃告勒以故,勒召而殺之。詔侃都督江州,領刺史,增置左右長史、司馬、從事中郎四人,掾屬十二人。侃旋于巴陵,因移鎮武昌。侃命張夔子隱為參軍,范達子珧為湘東太守,辟劉弘曾孫安為掾屬,表論梅陶,凡微時所荷,一
 餐咸報。



 遣子斌與南中郎將桓宣西伐樊城,走石勒將郭敬。使兄子臻、竟陵太守李陽等共破新野,遂平襄陽。拜大將軍,劍履上殿,入朝不趨,讚拜不名。上表固讓,曰:「臣非貪於疇昔,而虛讓於今日。事有合於時宜,臣豈敢與陛下有違;理有益於聖世,臣豈與朝廷作異。臣常欲除諸浮長之事,遣諸虛假之用,非獨臣身而已。若臣杖國威靈,梟雄斬勒,則又何以加!」咸和七年六月疾篤,又上表遜位曰:



 臣少長孤寒,始願有限。過蒙聖朝歷世殊恩、陛下睿鑒,寵靈彌泰。有始必終,自古而然。臣年垂八十,位極人臣,啟手啟足,當復何恨!但以陛下春秋尚
 富,餘寇不誅,山陵未反,所以憤愾兼懷,不能已已。臣雖不知命,年時已邁,國恩殊特,賜封長沙,隕越之日,當歸骨國土。臣父母舊葬,今在尋陽,緣存處亡,無心分違,已勒國臣修遷改之事,刻以來秋,奉迎窀穸,葬事訖,乃告老下籓。不圖所患,遂爾綿篤,伏枕感結,情不自勝。臣間者猶為犬馬之齒尚可小延,欲為陛下西平李雄,北吞石季龍,是以遣毌丘奧於巴東,授桓宣於襄陽。良圖未敘,於此長乖!此方之任,內外之要,願陛下速選臣代使,必得良才,奉宣王猷,遵成臣志,則臣死之日猶生之年。



 陛下雖聖姿天縱,英奇日新,方事之殷,當賴群俊。司徒導鑒
 識經遠,光輔三世;司空鑒簡素貞正,內外惟允;平西將軍亮雅量詳明,器用周時,即陛下之周召也。獻替疇諮,敷融政道,地平天成,四海幸賴。謹遣左長史殷羨奉送所假節麾、幢曲蓋、侍中貂蟬、太尉章、荊江州刺史印傳啟戟。仰戀天恩,悲酸感結。



 以後事付右司馬王愆期,加督護,統領文武。



 侃輿車出臨津就船,明日,薨于樊谿,時年七十六。成帝下詔曰:「故使持節、侍中、太尉、都督荊江雍梁交廣益寧八州諸軍事、荊江二州刺史、長沙郡公經德蘊哲,謀猷弘遠。作籓于外,八州肅清;勤王于內,皇家以寧。乃者桓文之勳,伯舅是憑。方賴大猷,俾屏予
 一人。前進位大司馬,禮秩策命,未及加崇。昊天不弔,奄忽薨殂,朕用震悼于厥心。今遣兼鴻臚追贈大司馬,假蜜章,祠以太牢。魂而有靈,喜茲寵榮。」又策謚曰桓,祠以太牢。侃遺令葬國南二十里,故吏刊石立碑畫像於武昌西。



 侃在軍四十一載,雄毅有權,明悟善決斷。自南陵迄於白帝數千里中,路不拾遺。蘇峻之役,庾亮輕進失利。亮司馬殷融詣侃謝曰:「將軍為此,非融等所裁。」將軍王章至,曰:「章自為之,將軍不知也。」侃曰:「昔殷融為君子,王章為小人;今王章為君子,殷融為小人。」侃性纖密好問,頗類趙廣漢。嘗課諸營種柳,都尉夏施盜官柳植之於
 己門。侃後見,駐車問曰:「此是武昌西門前柳,何因盜來此種?」施惶怖謝罪。時武昌號為多士,殷浩、庾翼等皆為佐吏。侃每飲酒有定限,常歡有餘而限已竭,浩等勸更少進,侃悽懷良久曰:「年少曾有酒失,亡親見約,故不敢踰。」議者以武昌北岸有邾城,宜分兵鎮之。侃每不答,而言者不已,侃乃渡水獵,引將佐語之曰:「我所以設險而禦寇,正以長江耳。邾城在江北,內無所倚,外接群夷。夷中利深,晉人貪利,夷不堪命,必引寇虜,乃致禍之由,非禦寇也。且吳時此城乃三萬兵守,今縱有兵守之,亦無益於江南。若羯虜有可乘之會,此又非所資也。」後庾
 亮戍之,果大敗。季年懷止足之分,不與朝權。未亡一年,欲遜位歸國,佐吏等苦留之。及疾篤,將歸長沙,軍資器仗牛馬舟船皆有定簿,封印倉庫,自加管鑰以付王愆期,然後登舟,朝野以為美談。將出府門,顧謂愆期曰:「老子婆娑,正坐諸君輩。」尚書梅陶與親人曹識書曰:「陶公機神明鑒似魏武,忠順勤勞似孔明,陸抗諸人不能及也。」謝安每言「陶公雖用法,而恆得法外意」。其為世所重如此。然媵妾數十,家僮千餘,珍奇寶貨富於天府。或云「侃少時漁於雷澤,網得一織梭,以挂于壁。有頃雷雨,自化為龍而去」。又夢生八翼,飛而上天,見天門九重,已登
 其八,唯一門不得入。閽者以杖擊之,因隧地,折其左翼。及寤,左腋猶痛。又嘗如廁,見一人硃衣介幘,斂板曰:「以君長者,故來相報。君後當為公,位至八州都督。」有善相者師圭謂侃曰:「君左手中指有豎理,當為公。若徹於上,貴不可言。」侃以針決之見血,灑壁而為「公」字,以紙裛,「公」字愈明。及都督八州,據上流,握彊兵,潛有窺窬之志,每思折翼之祥,自抑而止。



 侃有子十七人,唯洪、瞻、夏、琦、旗、斌、稱、範、岱見舊史,餘者並不顯。



 洪,辟丞相掾,早卒。



 瞻,字道真,少有才器,歷廣陵相,廬江、建昌二郡太守,遷
 散騎常侍、都亭侯。為蘇峻所害,追贈大鴻臚,謚愍悼世子。以夏為世子。及送侃喪還長沙,夏與斌及稱各擁兵數千以相圖。既而解散,斌先往長沙,悉取國中器仗財物。夏至,殺斌。庾亮上疏曰:「斌雖醜惡,罪在難忍,然王憲有制,骨肉至親,親運刀鋸以刑同體,傷父母之恩,無惻隱之心,應加放黜,以懲暴虐。」亮表未至都,而夏病卒。詔復以瞻息弘襲侃爵,仕至光祿勳。卒,子綽之嗣。綽之卒,子延壽嗣。宋受禪,降為吳昌侯,五百戶。



 琦,司空掾。



 旗,歷位散騎常侍、郴縣開國伯。咸和末,為散騎侍郎。性
 甚凶暴。卒,子定嗣。卒,子襲之嗣。卒,子謙之嗣。宋受禪,國除。



 斌,尚書郎。



 稱,東中郎將、南平太守、南蠻校尉、假節。性虓勇不倫,與諸弟不協。後加建威將軍。咸康五年,庾亮以稱為監江夏隨義陽三郡軍事、南中郎將、江夏相,以本所領二千人自隨。到夏口,輕將二百人下見亮。亮大會吏佐,責稱前後罪惡,稱拜謝,因罷出。亮使人於閣外收之,棄市,亮上疏曰:「案稱,大司馬侃之孽子,父亡不居喪位,荒耽于酒,昧利偷榮,擅攝五郡,自謂監軍,輒召王官,聚之軍府。
 故車騎將軍劉弘曾孫安寓居江夏,及將楊恭、趙韶,並以言色有忤,稱放聲當殺,安、恭懼,自赴水而死,韶於獄自盡。將軍郭開從稱往長沙赴喪,稱疑開附其兄弟,乃反縛懸頭於帆檣,仰而彈之,鼓棹渡江二十餘里,觀者數千,莫不震駭。又多藏匿府兵,收坐應死。臣猶未忍直上,且免其司馬。稱肆縱醜言,無所顧忌,要結諸將,欲阻兵構難。諸將惶懼,莫敢酬答,由是姦謀未即發露。臣以侃勳勞王室,是以依違容掩,故表為南中郎將,與臣相近,思欲有以匡救之。而稱豺狼愈甚,發言激切,不忠不孝,莫此之甚。茍利社稷,義有專斷,輒收稱伏法。」



 範,最知名,太元初,為光祿勳。



 岱,散騎侍郎。



 臻字彥遐,有勇略智謀,賜爵當陽亭侯。咸和中,為南郡太守、領南蠻校尉、假節。卒官,追贈平南將軍,謚曰肅。



 臻弟輿,果烈善戰,以功累遷武威將軍。初,賊張奕本中州人,元康中被差西征,遇天下亂,遂留蜀。至是,率三百餘家欲就杜弢,為侃所獲。諸將請殺其丁壯,取其妻息,輿曰:「此本官兵,數經戰陣,可赦之以為用。」侃赦之,以配輿。及侃與杜弢戰敗,賊以桔槔打沒官軍船艦,軍中失色。輿率輕舸出其上流以擊之,所向輒剋。賊又率眾將
 焚侃輜重,輿又擊破之。自是每戰輒剋,賊望見輿軍,相謂曰:「避陶武威。」無敢當者。後與杜弢戰,輿被重創,卒。侃哭之慟,曰:「喪吾家寶!」三軍皆為之垂泣。詔贈長沙太守。



 史臣曰:古者明王之建國也,下料疆宇,列為九州,輔相玄功,咨于四岳。所以仰希齊政,俯寄宣風。備連率之儀,威騰閫外;總頒條之務,禮縟區中。委稱其才,《甘棠》以之流詠;據非其德,仇餉以是興嗟。中朝叔世,要荒多阻,分符建節,並紊天綱。和季以同里之情,申盧綰之契,居方牧之地,振吳起之風。自幽徂荊,亟斂豺狼之迹;舉賢登善,窮掇孔翠之毛。由是吏民畢力,華夷順命,一州清晏,
 恬波於沸海之中;百城安堵,靜寢於稽天之際。猶獨稱善政,何其寡歟!《易》云「貞固足以幹事」,於征南見之矣。士行望非世族,俗異諸華,拔萃陬落之間,比肩髦俊之列,超居外相,宏總上流。布澤懷邊,則嚴城靜柝;釋位匡主,則淪鼎再寧。元規以戚里之崇,挹其膺而下拜;茂弘以保衡之貴,服其言而動色。望隆分陜,理則宜然。至於時屬雲屯,富逾天府,潛有包藏之志,顧思折翼之祥,悖矣!夫子曰「人無求備」,斯言之信,於是有征。



 贊曰:和季承恩,建旟南服。威靜荊塞,化揚江澳。戮力天朝,匪忘忠肅。長沙勤王,擁旆戎場。任隆三事,功宣一匡。
 繄賴之重,匪伊舟航。



\end{pinyinscope}