\article{載記第三 劉曜}

\begin{pinyinscope}

 劉曜



 劉
 曜,字永明,元海之族子也。少孤,見養於元海。幼而聰彗,有奇度。年八歲,從元海獵於西山,遇雨,止樹下,迅雷震樹,旁人莫不顛仆,曜神色自若。元海異之曰:「此吾家千里駒也,從兄為不亡矣!」身長九尺三寸,垂手過膝,生而眉白,目有赤光,鬚髯不過百餘根,而皆長五尺。性拓落高亮,與眾不群。讀書志於廣覽,不精思章句,善屬文,
 工草隸。雄武過人,鐵厚一寸,射而洞之,於時號為神射。尤好兵書,略皆闇誦。常輕侮吳、鄧,而自比樂毅、蕭、曹,時人莫之許也,惟聰每曰:「永明,世祖、魏武之流,何數公足道哉!」



 弱冠游于洛陽,坐事當誅,亡匿朝鮮,遇赦而歸。自以形質異眾,恐不容于世,隱迹管涔山,以琴書為事。嘗夜閑居,有二童子入跪曰:「管涔王使小臣奉謁趙皇帝,獻劍一口。」置前再拜而去。以燭視之,劍長二尺,光澤非常,赤玉為室,背上有銘曰:「神劍御,除眾毒。」曜遂服之。劍隨四時而變為五色。



 元海世頻歷顯職,後拜相國,都督中外諸軍事,鎮長安。靳準之難,自長安赴之。至于赤壁,
 太保呼延晏等自平陽奔之,與太傅朱紀、太尉范隆等上尊號。曜以太興元年僭即皇帝位,大赦境內,惟準一門不在赦例,改元光初。以朱紀領司徒,呼延晏領司空,范隆以下悉復本位。使征北劉雅、鎮北劉策次于汾陰,與石勒為掎角之勢。



 靳準遣侍中卜泰降于勒,勒囚泰,送之曜。謂泰曰:「先帝末年,實亂大倫,群閹撓政,誅滅忠良,誠是義士匡討之秋。司空執心忠烈,行伊霍之權,拯濟塗炭,使朕及此,勛高古人,德格天地。朕方寧濟大艱,終不以非命及君子賢人。司空若執忠誠,早迎大駕者,政由靳氏,祭則寡人,以朕此意布之司空,宣之朝士。」泰
 還平陽,具宣曜旨。準自以殺曜母兄,沈吟未從。尋而喬泰、王騰、靳康、馬忠等殺準,推尚書令靳明為盟主,遣卜泰奉傳國六璽降于曜。曜大悅,謂泰曰:「使朕獲此神璽而成帝王者,子也。」石勒聞之,怒甚,增兵攻之。明戰累敗,遣使求救於曜,曜使劉雅、劉策等迎之。明率平陽士女萬五千歸于曜,曜命誅明,靳氏男女無少長皆殺之。使劉雅迎母胡氏喪于平陽,還葬粟邑,墓號陽陵,偽謚宣明皇太后。僭尊高祖父亮為景皇帝,曾祖父廣為獻皇帝,祖防懿皇帝,考曰宣成皇帝。徙都長安,起光世殿於前,紫光殿於後。立其妻羊氏為皇后,子熙為皇太子,封
 子襲為長樂王,闡太原王,沖淮南王,敞齊王,高魯王,徽楚王,徵諸宗室皆進封郡王。繕宗廟、社稷、南北郊。以水承晉金行,國號曰趙。牲牡尚黑,旗幟尚玄,冒頓配天,元海配上帝,大赦境內殊死已下。



 黃石屠各路松多起兵於新平、扶風,聚眾數千,附于南陽王保。保以其將楊曼為雍州刺史,王連為扶風太守,據陳倉;張顗為新平太守,周庸為安定太守,據陰密。松多下草壁,秦隴氐羌多歸之。曜遣其軍騎劉雅、平西劉厚攻楊曼于陳倉,二旬不剋。曜率中外精銳以赴之,行次雍城,太史令弁廣明言於曜曰:「昨夜妖星犯月,師不宜行。」乃止。敕劉雅等攝
 圍固壘,以待大軍。



 地震,長安尤甚。時曜妻羊氏有殊寵,頗與政事,陰有餘之徵也。



 三年,曜發雍,攻陳倉,曼、連謀曰:「諜者適還,云其五牛旗建,多言胡主自來,其鋒恐不可當也。吾糧廩既少,無以支久,若頓軍城下,圍人百日,不待兵刃而吾自滅,不如率見眾以一戰。如其勝也,關中不待檄而至;如其敗也,一等死,早晚無在。」遂盡眾背城而陣,為曜所敗,王連死之,楊曼奔於南氐。曜進攻草壁,又陷之,松多奔隴城,進陷安定。保懼,遷於桑城。氐羌悉從之。曜振旅歸于長安,署劉雅為大司徒。



 晉將李矩襲金墉,剋之。曜左中郎將宋始、振威宋恕降於石勒。署
 其大將軍、廣平王岳為征東大將軍,鎮洛陽。會三軍疫甚,岳遂屯澠池。石勒遣石生馳應宋始等,軍勢甚盛。曜將尹安、趙慎等以洛陽降生,岳乃班師,鎮于陜城。



 西明門內大樹風吹折,經一宿,樹撥變為人形,髮長一尺,鬚眉長三寸,皆黃白色,有斂手之狀,亦有兩腳著裙之形,惟無目鼻,每夜有聲,十日而生柯條,遂成大樹,枝葉甚茂。



 長水校尉尹車謀反,潛結巴酋徐庫彭,曜乃誅車,囚庫彭等五十餘人于阿房,將殺之。光祿大夫游子遠固諫,曜不從。子遠叩頭流血,曜大怒,幽子遠而盡殺庫彭等,尸諸街巷之中十日,乃投之於水。於是巴氐盡叛,推
 巴歸善王句渠知為主,四山羌、氐、巴、羯應之者三十餘萬,關中大亂,城門晝閉。子遠又從獄表諫,曜怒甚,毀其表曰:「大荔奴不憂命在須臾,猶敢如此,嫌死晚邪?」叱左右速殺之。劉雅、朱紀、呼延晏等諫曰:「子遠幽而尚諫者,所謂忠於社稷,不知死之將至。陛下縱弗能用,奈何殺之!若子遠朝誅,臣等亦暮死,以彰陛下過差之咎。天下之人皆當去陛下蹈西海而死耳,陛下復與誰居乎!」曜意解,乃赦之。於是敕內外戒嚴,將親討渠知。子遠進曰:「陛下誠能納愚臣之計者,不勞大駕親動,一月之中可使清定。」曜曰:「卿試言之。」子遠曰:「彼匪有大志,希竊非望
 也,但逼於陛下峻綱耳。今死者不可追,莫若赦諸逆人之家老弱沒奚官者,使迭相撫育,聽其復業,大赦與之更始。彼生路既開,不降何待!若渠知自以罪重不即下者,願假臣弱兵五千,以為陛下梟之,不敢勞陛下之將帥也。不爾者,今賊黨既眾,彌川被谷,雖以天威臨之,恐非年歲可除。」曜大悅,以子遠為車騎大將軍、開府儀同三司、都督雍秦征討諸軍事。大赦境內。子遠次于雍城,降者十餘萬,進軍安定,氐羌悉下,惟句氏宗黨五千餘家保存于陰密,進攻平之,遂振旅循隴右,陳安郊迎。



 先是,上郡氐羌十餘萬落保險不降,酋大虛除權渠自號秦
 王。子遠進師至其壁下,權渠率眾來距,五戰敗之。權渠恐,將降,其子伊余大言於眾曰:「往劉曜自來,猶無若我何,況此偏師而欲降之!」率勁卒五萬,晨壓壘門。左右勸戰,子遠曰:「吾聞伊餘之勇,當今無敵,士馬之彊,復非其匹;又其父新敗,怒氣甚盛;且西戎剽勁,鋒銳不可擬也。不如緩之,使氣竭而擊之。」乃堅壁不戰。伊餘有驕色。子遠候其無備,夜,誓眾蓐食,晨,大風霧,子遠曰:「天贊我也!」躬先士卒,掃壁而出,遲明覆之,生擒伊餘,悉俘其眾。權渠大懼,被髮割面而降。子遠啟曜以權渠為征西將軍、西戎公,分徙伊餘兄弟及其部落二十餘萬口于長安。
 西戎之中,權渠部最強,皆稟其命而為寇暴,權渠既降,莫不歸附。



 曜大悅,宴群臣于東堂,語及平生,泫然流涕,遂下書曰:「蓋褒德惟舊,聖后之所先;念惠錄孤,明王之恒典。是以世祖草創河北,而致封於嚴尤之孫;魏武勒兵梁宋,追慟於橋公之墓。前新贈大司徒、烈愍公崔岳,中書令曹恂,晉陽太守王忠,太子洗馬劉綏等,或識朕於童齔之中,或濟朕於艱窘之極,言念君子,實傷我心。《詩》不云乎:『中心藏之,何日忘之!』岳,漢昌之初雖有褒贈,屬否運之際,禮章莫備,今可贈岳使持節、侍中、大司徒、遼東公,恂大司空、南郡公,綏左光祿大夫、平昌公,忠鎮
 軍將軍、安平侯,並加散騎常侍。但皆丘墓夷滅,申哀莫由,有司其速班訪岳等子孫,授以茅土,稱朕意焉。」初,曜之亡,與曹恂奔於劉綏,綏匿之於舊匱,載送於忠,忠送之朝鮮。歲餘,饑窘,變姓名,客為縣卒。岳為朝鮮令,見而異之,推問所由。曜叩頭自首,流涕求哀。岳曰:「卿謂崔元嵩不如孫賓碩乎,何懼之甚也!今詔捕卿甚峻,百姓間不可保也。此縣幽僻,勢能相濟,縱有大急,不過解印綬與卿俱去耳。吾既門衰,無兄弟之累,身又薄祜,未有兒子,卿猶吾子弟也,勿為過憂。大丈夫處身立世,鳥獸投人,要欲濟之,而況君子乎!」給以衣服,資供書傳。曜遂從
 岳,質通疑滯,恩顧甚厚。岳從容謂曜曰:「劉生姿宇神調,命世之才也!四海脫有微風搖之者,英雄之魁,卿其人矣。」曹恂雖於屯厄之中,事曜有君臣之禮,故皆德之。



 曜立太學於長樂宮東,小學於未央宮西,簡百姓年二十五已下十三已上,神志可教者千五百人,選朝賢宿儒明經篤學以教之。以中書監劉均領國子祭酒。置崇文祭酒,秩次國子。散騎侍郎董景道以明經擢為崇文祭酒。以游子遠為大司徒。



 曜命起酆明觀,立西宮,建陵霄臺於滈池,又將於霸陵西南營壽陵。侍中喬豫、和苞上疏諫曰:「臣聞人主之興作也,必仰準乾象,俯順人時,是
 以衛文承亂亡之後,宗廟社稷流漂無所,而猶上候營室以構楚宮。彼其急也猶尚若茲,故能興康叔、武公之迹,以延九百之慶也。奉詔書將營酆明觀,市道芻蕘咸以非之,曰一觀之功可以平涼州矣。又奉敕旨復欲擬阿房而建西宮,模瓊臺而起陵霄,此則費萬酆明,功億前役也。以此功費,亦可以吞吳蜀,翦齊魏矣。陛下何為於中興之日而蹤亡國之事!自古聖王,人誰無過!陛下此役,實為過舉。過貴在能改,終之實難。又伏聞敕旨將營建壽陵,周迴四里,下深二十五丈,以銅為棺郭,黃金飾之,恐此功費非國內所能辦也。且臣聞堯葬穀林,市
 不改肆;顓頊葬廣陽,下不及泉。聖王之於終也如是。秦皇下錮三泉,周輪七里,身亡之後,毀不旋踵,暗主之於終也如此。向魋石槨,孔子以為不如速朽;王孫惈葬,識者嘉其矯世。自古無有不亡之國,不掘之墓,故聖王知厚葬之招害也,故不為之。臣子之於君父,陵墓豈不欲高廣如山岳哉!但以保全始終,安固萬世為優耳。興亡奢儉,冏然於前,惟陛下覽之。」曜大悅,下書曰:「二侍中懇懇有古人之風烈矣,可謂社稷之臣也。非二君,朕安聞此言乎!以孝明於承平之世,四海無虞之日,尚納鐘離一言而罷北宮之役,況朕之闇眇,當今極弊,而可不敬
 從明誨乎!今敕悉停壽陵制度,一遵霸陵之法。《詩》不云乎:『無言不酬,無德不報。』其封豫安昌子,苞平輿子,並領諫議大夫。可敷告天下,使知區區之朝思聞過也。自今政法有不便於時,不利社稷者,其詣闕極言,勿有所諱。」省酆水囿以與貧戶。



 終南山崩,長安人劉終於崩所得白玉方一尺,有文字曰:「皇亡,皇亡,敗趙昌。井水竭,構五梁,咢酉小衰困囂喪。嗚呼!嗚呼!赤牛奮靷其盡乎!」時群臣咸賀,以為勒滅之徵。曜大悅,齋七日而後受之於太廟,大赦境內,以終為奉瑞大夫。中書監劉均進曰:「臣聞國主山川,故山崩川竭,君為之不舉。終南,京師之鎮,國
 之所瞻,無故而崩,其凶焉可極言!昔三代之季,其災也如是。今朝臣皆言祥瑞,臣獨言非,誠上忤聖旨,下違眾議,然臣不達大理,竊所未同。何則?玉之於山石也,猶君之於臣下。山崩石壞,象國傾人亂。『皇亡,皇亡,敗趙昌者』,此言皇室將為趙所敗,趙因之而昌。今大趙都於秦雍,而勒跨全趙之地,趙昌之應,當在石勒,不在我也。『井水竭,構五梁』者,井謂東井,秦之分也,『五謂五車』,梁謂大梁,五車、大梁,趙之分也,此言秦將竭滅,以構成趙也。『咢』者,歲之次名作咢也,言歲馭作咢酉之年,當有敗軍殺將之事。『困』謂困敦,歲在子之年名,玄囂亦在天之次,言歲
 馭於子,國當喪亡。『赤牛奮靷』謂赤奮若,在丑之歲名也。『牛』謂牽牛,東北維之宿,丑之分也,言歲在丑當滅亡,盡無復遺也。此其誡悟蒸蒸,欲陛下勤修德化以禳之。縱為嘉祥,尚願陛下夕惕以答之。《書》曰:『雖休勿休。』願陛下追蹤周旦盟津之美,捐鄙虢公夢廟之凶,謹歸沐浴以待妖言之誅。」曜憮然改容。御史劾均狂言瞽說,誣罔祥瑞,請依大不敬論。曜曰:「此之災瑞,誠不可知,深戒朕之不德,朕收其忠惠多矣,何罪之有乎!」



 曜親征氐羌,仇池楊難敵率眾來距,前鋒擊敗之,難敵退保仇池,仇池諸氐羌多降於曜。曜後復西討楊韜於南安,韜懼,與隴西
 太守梁勛等降于曜,皆封列侯。使侍中喬豫率甲士五千,遷韜等及隴右萬餘戶于長安。曜又進攻仇池。時曜寢疾,兼癘疫甚,議欲班師,恐難敵躡其後,乃以其尚書郎王獷為光國中郎將,使于仇池,以說難敵,難敵於是遣使稱籓。曜大悅,署難敵為使持節、侍中、假黃鉞、都督益寧南秦涼梁巴六州隴上西域諸軍事、上大將軍、益寧南秦三州牧、領護南氐校尉、寧羌中郎將、武都王,子弟為公侯列將二千石者十五人。



 陳安請朝,曜以疾篤不許。安怒,且以曜為死也,遂大掠而歸。曜疾甚篤,馬輿而還,使其將呼延實監輜重於後。陳安率精騎耍之于
 道。實奔戰無路,與長史魯憑俱沒于安。安囚實而謂之曰:「劉曜已死,子誰輔哉?孤當輿足下終定大業。」實叱安曰:「狗輩!汝荷人榮寵,處不疑之地,前背司馬保,今復如此。汝自視何如主上?憂汝不久梟首上邽通衢,何謂大業!可速殺我,懸我首於上邽東門,觀大軍之入城也。」安怒,遂殺之。以魯憑為參軍,又遣其弟集及將軍張明等率騎二萬追曜,曜衛軍呼延瑜逆戰,擊斬之,悉俘其眾。安懼,馳還上邽。曜至自南安。陳安使其將劉烈、趙罕襲汧城,拔之,西州氐羌悉從安。安士馬雄盛,眾十餘萬,自稱使持節、大都督、假黃鉞、大將軍、雍涼秦梁四州牧、涼
 王,以趙募為相國,領左長史。魯憑對安大哭曰:「吾不忍見陳安之死也。」安怒,命斬之。憑曰:「死自吾分,懸吾頭於秦州通衢,觀趙之斬陳安也。」遂殺之。曜聞憑死,悲慟曰:「賢人者,天下之望也。害賢人,是塞天下之情,夫承平之君猶不敢乖臣妾之心,況於四海乎!陳安今於招賢採哲之秋,而害君子,絕當時之望,吾知其無能為也。」



 休屠王石武以桑城降,曜大悅,署武為使持節、都督秦州隴上雜夷諸軍事、平西大將軍、秦州刺史,封酒泉王。



 曜后羊氏死,偽謚獻文皇后。羊氏內有特寵,外參朝政,生曜三子熙、襲、闡。



 曜始禁無官者不聽乘馬,祿八百石已上
 婦女乃得衣錦繡,自季秋農功畢,乃聽飲酒,非宗廟社稷之祭不得殺牛,犯者皆死。曜臨太學,引試學生之上第者拜郎中。



 武功男子蘇撫、陜男子伍長平並化為女子。石言於陜,若言勿東者。



 曜將葬其父及妻,親如粟邑以規度之。負土為墳,其下周迴二里,作者繼以脂燭,怨呼之聲盈于道路。游子遠諫曰:「臣聞聖主明王、忠臣孝子之於終葬也,棺足周身,槨足周棺,藏足周槨而已,不封不樹,為無窮這計。伏惟陛下聖慈幽被,神鑒洞遠,每以清儉恤下為先。社稷資儲為本。今二陵之費至以億計,計六萬夫百日作,所用六百萬功。二陵皆下錮三泉,上
 崇百尺,積石為山,增土為阜,發掘古塚以千百數,役夫呼嗟,氣塞天地,暴骸原野,哭聲盈衢,臣竊謂無益於先皇先后,而徒喪國之儲力。陛下脫仰尋堯舜之軌者,則功不盈百萬,費亦不過千計,下無怨骨,上無怨人,先帝先后有太山之安,陛下饗舜、禹、周公之美,惟陛下察焉。」曜不納,乃使其將劉岳等帥騎一萬,迎父及弟暉喪於太原。疫氣大行,死者十三四。上洛男子張盧死二十七日,有盜發其塚者,盧得蘇。曜葬其父,墓號永垣陵,葬妻羊氏,墓號顯平陵。大赦境內殊死巳下,賜人爵二級,孤老貧病不能自存者帛各有差。



 太寧元年,陳安攻曜征
 西劉貢于南安,休屠王石武自桑城將攻上邽,以解南安之圍。安聞之懼,馳歸上邽,遇於瓜田。武以眾寡不敵,奔保張春故壘。安引軍追武曰:「叛逆胡奴!要當生縛此奴,然後斬劉貢。」武閉壘距之。貢敗安後軍,俘斬萬餘。安馳還赴救,貢逆擊敗之。俄而武騎大至,安眾大潰,收騎八千,奔于隴城。貢乃留武督後眾,躬先士卒,戰輒敗之,遂圍安于隴城。



 大雨霖,震曜父墓門屋,大風飄發其父寢堂于垣外五十餘步。曜避正殿,素服哭于東堂五日,使其鎮軍劉襲、太常梁胥等繕復之。松柏眾木植已成林,至是悉枯。署其大司馬劉雅為太宰,加劍履上殿,入
 朝不趨,贊拜不名,給千兵百騎,甲仗百人入殿,增班劍六十人,前後鼓吹各二部。



 曜親征陳安,圍安于隴城。安頻出挑戰,累擊敗之,斬獲八千餘級。右軍劉乾攻平襄,剋之,隴上諸縣悉降。曲赦隴右殊死已下,惟陳安、趙募不在其例。安留楊伯支、姜沖兒等守隴城,帥騎數百突圍而出,欲引上邽、平襄之眾還解隴城之圍。安既出,知上邽被圍,平襄已敗,乃南走陜中。曜使其將軍平先、丘中伯率勁騎追安,頻戰敗之,俘斬四百餘級。安與壯士十餘騎於陜中格戰,安左手奮七尺大刀,右手執丈八蛇矛,近交則刀矛俱發,輒害五六;遠則雙帶鞬服,左右
 馳射而走。平先亦壯健絕人,勇捷如飛,與安搏戰,三交,奪其蛇矛而退。會日暮,雨甚,安棄馬,與左右五六人步踰山嶺,匿于溪澗。翌日尋之,遂不知所在。會連雨始霽,輔威呼延清尋其徑迹,斬安于澗曲。曜大悅。



 安善於撫接,吉凶夷險與眾同之,及其死,隴上歌之曰:「隴上壯士有陳安,驅乾雖小腹中寬,愛養將士同心肝。聶驄父馬鐵瑕鞍,七尺大刀奮如湍,丈八蛇矛左右盤,十盪十決無當前。戰始三交失蛇矛,棄我聶驄竄嚴幽,為我外援而懸頭。西流之水東流河,一去不還奈子何!」曜聞而嘉傷,命樂府歌之。



 楊伯支斬姜沖兒,以隴城降。宋亭斬趙
 募,以上邽降。徙秦州大姓楊、姜諸族二千餘戶於長安。氐羌悉下,並送質任。



 時劉岳與涼州刺史張茂相持於河上,曜自隴長驅至西河,戎卒二十八萬五千,臨河列營,百餘里中,鐘鼓之聲沸河動地,自古軍旅之盛未有斯比。茂臨河諸戍皆望風奔退。揚聲欲百道俱渡,直至姑臧,涼州大怖,人無固志。諸將咸欲速濟,曜曰:「吾軍旅雖盛,不踰魏武之東也。畏威而來者,三有二焉。中軍宿衛已皆疲老,不可用也。張氏以吾新平陳安,師徒殷盛,以形聲言之,非彼五郡之眾所能抗也,必怖而歸命,受制稱籓,吾復何求!卿等試之,不出中旬,張茂之表不至
 者,吾為負卿矣。」茂懼,果遣使稱籓,獻馬一千五百匹,牛三千頭,羊十萬口,黃金三百八十斤,銀七百斤,女妓二十人,及諸珍寶珠玉、方域美貨不可勝紀。曜大悅,使其大鴻臚田崧署茂使持節、假黃鉞、侍中、都督涼南北秦梁益巴漢隴右西域雜夷匈奴諸軍事、太師、領大司馬、涼州牧、領西域大都護、護氐羌校尉、涼王。曜至自河西,遣胡元增其父及妻墓高九十尺。



 楊難敵以陳安既平,內懷危懼,奔於漢中。鎮西劉厚追擊之,獲其輜重千餘兩,士女六千餘人,還之仇池。曜以大鴻臚田崧為鎮南大將軍、益州刺史,鎮仇池,以劉岳為侍中、都督中外諸
 軍事,進封中山王。



 初,靳準之亂,曜世子胤沒于黑匿郁鞠部,至是,胤自言,郁鞠大驚,資給衣馬,遣子送之。曜對胤悲慟,嘉郁鞠忠款,署使持節、散騎常侍、忠義大將軍、左賢王。胤字義孫,美姿貌,善機對,年十歲,身長七尺五寸,眉鬢如畫。聰奇之,謂曜曰:「此兒神氣豈同義真乎!固當應為卿之冢嫡,卿可思文王廢伯邑考立武王之意也。」曜曰:「臣之籓國,僅能守祭祀便足矣,不可以亂長幼之倫也。」聰曰:「卿勳格天地,國兼百城,當世祚太師,受專征之任,五侯九伯得專征之者,卿之子孫,柰何言同諸籓國也!義真既不能遠追太伯高讓之風,吾不過為卿
 封之以一國。」義真,曜子儉之字也。於是封儉為臨海王,立胤為世子。胤雖少離屯難,流躓殊荒,而風骨俊茂,爽朗卓然;身長八尺三寸,髮與身齊,多力善射,驍捷如風雲,,曜因以重之,其朝臣亦屬意焉。曜於是顧謂群下曰:「義孫可謂歲寒而不凋,涅而不淄者矣。義光雖先已樹立,然沖幼儒謹,恐難乎為今世之儲貳也,懼非所以上固社稷,下愛義光。義孫年長明德,又先世子也,朕欲遠追周文,近蹤光武,使宗廟有太山之安,義光饗無疆之福,於諸卿意如何?」其太傅呼延晏等咸曰:「陛下遠擬周漢,為國家無窮之計,豈惟臣等賴之,實亦宗廟四海之
 慶。」左光祿卜泰、太子太保韓廣等進曰:「陛下若以廢立為是也,則不應降日月之明,垂訪群下。若以為疑也,固思聞臣等異同之言,竊以誠廢太子非也。何則?昔周文以未建之前,擇聖表而超樹之可也。光武緣母色而廢立,豈足為聖朝之模範!光武誠以東海篡統,何必不如明帝!皇子胤文武才略,神度弘遠,信獨絕一時,足以擬蹤周發;然太子孝友仁慈,志尚沖雅,亦足以堂負聖基,為承平之賢主。何況儲宮者,六合人神所繫望也,不可輕以廢易。陛下誠實爾者,臣等有死而已,未敢奉詔。」曜默然。胤前泣曰:「慈父之於子也,當務存《尸鳩》之仁,何可
 替熙而立臣也!陛下謬恩乃爾者,臣請死於此,以明赤心。且陛下若愛忘其醜,以臣微堪指授,亦當能輔導義光,仰遵聖軌。」因歔欷流涕,悲感朝臣。曜亦以太子羊氏所生,羊有寵,哀之不忍廢,乃止。追謚前妻卜氏為元悼皇后,胤之母也。卜泰,胤之舅,曜嘉之,拜上光祿大夫、儀同三司、領太子太傅。封胤為永安王,署侍中、衛大將軍、都督二宮禁衛諸軍事、開府儀同三司、錄尚書事,領太子太傅,號曰皇子。命熙於胤盡家人之禮。



 時有鳳皇將五子翔於故未央殿五日,悲鳴不食皆死。曜立后劉氏。



 石勒將石他自鴈門出上郡,襲安國將軍、北羌王盆句
 除,俘三千餘落,獲牛馬羊百餘萬而歸。曜大怒,投袂而起。是日次于渭城,遣劉岳追之,曜次于富平,為岳聲援。岳及石他戰于河濱,敗之,斬他及其甲士一千五百級,赴河死者五千餘人,悉收所虜,振旅而歸。



 楊難敵自漢中還襲仇池,剋之,執田崧,立之於前。難敵左右叱崧令拜,崧瞋目叱之曰:「氐狗!安有天子牧伯而向賊拜乎!」難敵曰:「子岱,吾當與子終定大事。子謂劉氏可為盡忠,吾獨不可乎!」崧厲色大言曰:「若賊氐奴才,安敢欲希覬非分!吾寧為國家鬼,豈可為汝臣,何不速殺我!」顧排一人,取其劍,前刺難敵,不中,為難敵所殺。



 曜遣劉岳攻石生
 于洛陽,配以近郡甲士五千,宿衛精卒一萬,濟自盟津。鎮東呼延謨率荊司之眾自崤澠而東。岳攻石勒盟津、石梁二戍,剋之,斬獲五千餘級,進圍石生于金墉。石季龍率步騎四萬入自成皋關,岳陳兵以待之。戰于洛西,岳師敗績,岳中流矢,退保石梁。季龍遂塹柵列圍,遏絕內外。岳眾飢甚,殺馬食之。季龍又敗呼延謨,斬之。曜親率軍援岳,季龍率騎三萬來距。曜前軍劉黑大敗季龍將石聰于八特阪。曜次于金谷,夜無故大驚,軍中潰散,乃退如澠池。夜中又驚,士卒奔潰,遂歸長安。季龍執劉岳及其將王騰等八十餘人,並氐羌三千餘人,送于襄
 國,坑士卒一萬六千。曜至自澠池,素服郊哭,七日乃入城。



 武功豕生犬,上邽馬生牛,及諸妖變不可勝記。曜命其公卿各舉博識直言之士一人,司空劉均舉參軍臺產,曜親臨東堂,遣中黃門策問之。產極言其故,曜覽而嘉之,引見東堂,訪以政事。產流涕歔欷,具陳災變之禍,政化之闕,辭旨諒直,曜改容禮之,即拜博士祭酒、諫議大夫,領太史令。其後所言皆驗,曜彌重之,歲中三遷,歷位尚書、光祿大夫、太子少師,位特進。



 曜署劉胤為大司馬,進封南陽王,以漢陽諸郡十三為國;置單于臺于渭城,拜大單于,置左右賢王已下,皆以胡、羯、鮮卑、氐、羌豪
 桀為之。



 曜自還長安,憤恚發病,至是疾瘳,曲赦長安殊死已下。署其汝南王劉咸為太尉、錄尚書事,光祿大夫劉綏為大司徒,卜泰為大司空。



 曜妻劉氏疾甚,曜親省臨之,問其所欲言。劉泣曰:「妾叔父昶無子,妾少養於叔,恩撫甚隆,無以報德,願陛下貴之。妾叔皚女芳有德色,願備後宮。」曜許之。言終而死,偽謚獻烈皇后。以劉昶為使持節、侍中、大司徒、錄尚書事,進封河南郡公,封昶妻張氏為慈鄉君,立劉皚女芳為皇后,追念劉氏之言也。俄署驃騎劉述為大司徒,劉昶為太保。召公卿已下子弟有勇幹者為親御郎,被甲乘鎧馬,動止自隨,以充折
 衝之任。尚書郝述、都水使者支當等固諫,曜大怒,鴆而殺之。



 咸和三年,夜夢三人金面丹脣,東向逡巡,不言而退,曜拜而履其跡。旦召公卿已下議之,朝臣咸賀以為吉祥,惟太史令任義進曰:「三者,歷運統之極也。東為震位,王者之始次也。金為兌位,物衰落也。脣丹不言,事之畢也。逡巡揖讓,退舍之道也。為之拜者,屈伏於人也。履跡而行,慎不出疆也。東井,秦分也。五車,趙分也。秦兵必暴起,亡主喪師,留敗趙地。遠至三年,近七百日,其應不遠。願陛下思而防之。」曜大懼,於是躬親二郊,飾繕神祠,望秩山川,靡不周及。大赦殊死已下,復百姓租稅之半。
 長安自春不雨,至於五月。



 曜遣其武衛劉朗率騎三萬襲楊難敵于仇池,弗剋,掠三千餘戶而歸。張駿聞曜軍為石氐所敗,乃去曜官號,復稱晉大將軍、涼州牧,遣金城太守張閬及枹罕護軍辛晏、將軍韓璞等率眾數萬人,自大夏攻掠秦州諸郡。曜遣劉胤率步騎四萬擊之,夾洮相持七十餘日。冠軍呼延那雞率親御郎二千騎,絕其運路。胤濟師逼之,璞軍大潰,奔還涼州。胤追之,及于令居,斬級二萬。張閬、辛晏率眾數萬降于曜,皆拜將軍,封列侯。



 石勒遣石季龍率眾四萬,自軹關西入伐曜,河東應之者五十餘縣,進攻蒲阪。曜將東救蒲阪,懼張
 駿、楊難敵承虛襲長安,遣其河間王述發氐羌之眾屯于秦州。曜盡中外精銳水陸赴之,自衛關北濟。季龍懼,引師而退。追之,及于高候,大戰,敗之,斬其將軍石瞻,枕尸二百餘里,收其資仗億計。季龍奔于朝歌。曜遂濟自大陽,攻石生于金墉,決千金堨以灌之。曜不撫士眾,專與嬖臣飲博,左右或諫,曜怒,以為妖言,斬之。大風拔樹,昏霧四塞。聞季龍進據石門,續知勒自率大眾已濟,始議增滎陽戍,杜黃馬關。俄而洛水候者與勒前鋒交戰,擒羯,送之。曜問曰:「大胡自來邪?其眾大小復如何?」羯曰:「大胡自來,軍盛不可當也。」曜色變,使攝金墉之圍,陳于
 洛西,南北十餘里。曜少而淫酒,末年尤甚。勒至,曜將戰,飲酒數斗,常乘赤馬無故局頓,乃乘小馬。比出,復飲酒斗餘。至於西陽門,捴陣就平,勒將石堪因而乘之,師遂大潰。曜昏醉奔退,馬陷石渠,墜於冰上,被瘡十餘,通中者三,為堪所執,送於勒所。曜曰:「石王!憶重門之盟不?」勒使徐光謂曜曰:「今日之事,天使其然,復云何邪!」幽曜於河南丞廨,使金瘡醫李永療之,歸于襄國。



 曜瘡甚,勒載以馬輿,使李永與同載。北苑市三老孫機上禮求見曜,勒許之。機進酒于曜曰:「僕谷王,關右稱帝皇。當持重,保土疆。輕用兵,敗洛陽。祚運窮,天所亡。開大分,持一觴。」曜
 曰:「何以健邪!當為翁飲。」勒聞之,悽然改容曰:「亡國之人,足令老叟數之。」舍曜於襄國永豐小城,給其妓妾,嚴兵圍守。遣劉岳、劉震等乘馬,從男女,衣以見曜,曜曰:「久謂卿等為灰土,石王仁厚,全宥至今,而我殺石他,負盟之甚。今日之禍,自其分耳。」留宴終日而去。勒諭曜與其太子熙書,令速降之,曜但敕熙:「與諸大臣匡維社稷,勿以吾易意也。」勒覽而惡之,後為勒所殺。



 熙及劉胤、劉咸等議西保秦州,尚書胡勛曰:「今雖喪主,國尚全完,將士情一,未有離叛,可共并力距險,走未晚也。」胤不從,怒其沮眾,斬之,遂率百官奔于上邽,劉厚、劉策皆捐鎮奔之。
 關中擾亂,將軍蔣英、辛恕擁眾數十萬,據長安,遣使招勒,勒遣石生率洛陽之眾以赴之。胤及劉遵率眾數萬,自上邽將攻石生于長安,隴東、武都、安定、新平、北地、扶風、始平諸郡戎夏皆起兵應胤。胤次于仲橋,石生固守長安。勒使石季龍率騎二萬距胤,戰於義渠,為季龍所敗,死者五千餘人。胤奔上邽,季龍乘勝追戰,枕尸千里,上邽潰。季龍執其偽太子熙、南陽王劉胤并將相諸王等及其諸卿校公侯已下三千餘人,皆殺之。徙其臺省文武、關東流人、秦雍大族九千餘人于襄國,又坑其王公等及五郡屠各五千餘人于洛陽。曜在位十年而敗。
 始,元海以懷帝永嘉四年僭位,至曜三世,凡二十有七載,以成帝咸和四年滅。



 史臣曰:彼戎狄者,人面獸心,見利則棄君親,臨財則忘仁義者也。投之遐遠,猶懼外侵,而處以封畿,窺我中釁。昔者幽后不綱,胡塵暗於戲水;襄王失御,戎馬生於關洛。至於算強弱,妙兵權,體興衰,知利害,於我中華未可量也。況元海人傑,必致青雲之上;許以殊才,不居庸劣之下。是以策馬鴻騫,乘機豹變,五部高嘯,一旦推雄,皇枝相害,未有與之爭衡者矣。伊秩啟興王之略,骨都論剋定之秋,單于無北顧之懷,獫狁有南郊之祭,大哉天
 地,茲為不仁矣!若乃習以華風,溫乎雅度;兼其舊俗,則罕規模。雖復石勒稱籓,王彌效款,終為夷狄之邦,未辯君臣之位。至於不遠儒風,虛襟正直,則昔賢所謂並仁義而盜之者焉。



 偽主斯亡,玄明篡嗣,樹恩戎旅,既總威權,關河開曩日之疆,士馬倍前人之氣。然則信不由中,自乖弘遠,貌之為美,處事難終。縱武窮兵,殘忠害謇,佞人方轡,並后載馳,閹豎類於迴天,凝科踰於炮烙。遣豺狼之將,逐鷹犬之師,懸旌俯渭,分麾陷洛,鐵馬陵山,胡笳遵渚,粉忠貞於戎手,聚搢紳於京觀。先王井賦,乃眷維桑;舊都宮室,咸成茂草。墜露沾衣,行人灑淚。若乃上
 古敦龐,不親其子,功成高讓,歸諸有德。爰及三伐,乃用干戈,將以拯厥版蕩,恭膺天命。懿彼武王,殷之列辟,載旆乘時,興兵誓野,投焚既隕,可以絕言。而輕呂旁揮,彤弧三發,豈若響清蹕於常道之門,馳金車於山陽之館!故知黔首來蘇,居今愛古;白旗陳肆,古不如今。胡寇不仁,有同豺豕,役天子以行觴,驅乘輿以執蓋,庾氏之淚既盡,辛賓加之以血。若乃有生之貴,處死為難,弘在三之義,忘七尺之重,主憂之恨,畢命同歸,自古篡奪,於斯為甚。是以災氣呈形,賊臣苞亂,政荒民散,可以危亡。劉聰竟得壽終,非不幸也。



 曜則天資虓勇,運偶時艱,用兵
 則王翦之倫,好殺亦董公之亞。而承基醜類,或有可稱。子遠納忠,高旌暫偃;和苞獻直,酆明罷觀。而師之所處,荊棘生焉,自絕強籓,禍成勁敵。天之所厭,人事以之,駭戰士而宵奔,酌戎杯而不醒,有若假手,同乎拾芥。豈石氏之興歟,何不支之甚也!



 贊曰:惟皇不範,邇甸居穹。丹硃罕嗣,冒頓爭雄。胡旌揚月,朔馬騰風。埃塵淮浦,虓呼河宮。未央朝寂,謻門旦空。郭欽之慮,辛有知戎。



\end{pinyinscope}