\article{載記第十六 姚弋仲姚襄姚萇}

\begin{pinyinscope}

 姚弋仲姚襄姚萇



 姚弋仲,南安赤亭羌人也。其先有虞氏之苗裔。禹封舜少子于西戎,世為羌酋。其後燒當雄於洮、罕之間,七世孫填虞,漢中元末寇擾西州,為楊虛侯馬武所敗,徙出塞。虞九世孫遷那率種人內附,漢朝嘉之,假冠軍將軍、西羌校尉、歸順王,處之於南安之赤亭。那玄孫柯回為魏鎮西將軍、綏戎校尉、西羌都督。回生弋仲,少英毅,不
 營產業,唯以收恤為務,眾皆畏而親之。永嘉之亂,東徙榆眉,戎夏襁負隨之者數萬,自稱護西羌校尉、雍州刺史、扶風公。



 劉曜之平陳安也,以弋仲為平西將軍,封平襄公,邑之于隴上。及石季龍剋上邽,弋仲說之曰:「明公握兵十萬,功高一時,正是行權立策之日。隴上多豪,秦風猛勁,道隆後服,道洿先叛,宜徙隴上豪強,虛其心腹,以實畿甸。」季龍納之,啟勒以弋仲行安西將軍、六夷左都督。後晉豫州刺史祖約奔于勒,勒禮待之,弋仲上疏曰:「祖約殘賊晉朝,逼殺太后,不忠於主,而陛下寵之,臣恐姦亂之萌,此其始矣。」勒善之,後竟誅約。



 勒既死,季龍
 執權,思弋仲之言,遂徙秦、雍豪傑于關東。弋仲率部眾數萬遷于清河,拜奮武將軍、西羌大都督,封襄平縣公。及季龍廢石弘自立,弋仲稱疾不賀。季龍累召之,乃赴,正色謂季龍曰:「奈何把臂受託而反奪之乎!」季龍憚其強正而不之責。遷持節、十郡六夷大都督、冠軍大將軍。性清儉鯁直,不修威儀,屢獻讜言,無所迴避,季龍甚重之。朝之大議,靡不參決,公卿亦憚而推下之。武城左尉,季龍寵姬之弟也,曾擾其部,弋仲執尉,數以迫脅之狀,命左右斬之。尉叩頭流血,左右諫,乃止。其剛直不回,皆此類也。



 季龍末,梁犢敗李農於滎陽,季龍大懼,馳召弋
 仲。弋仲率其部眾八千餘人屯于南郊,輕騎至鄴。時季龍病,不時見弋仲,引入領軍省,賜其所食之食。弋仲怒不食,曰:「召我擊賊,豈來覓食邪!我不知上存亡,若一見,雖死無恨。」左右言之,乃引見。弋仲數季龍曰:「兒死來愁邪?乃至於疾!兒小時不能使好人輔相,至令相殺。兒自有過,責其下人太甚,故反耳。汝病久,所立兒小,若不差,天下必亂。當宜憂此,不煩憂賊也。犢等因思歸之心,共為姦盜,所行殘賊,此成擒耳。老羌請效死前鋒,使一舉而了。」弋仲性狷直,俗無尊卑皆汝之,季龍恕而不責,於坐授使持節、侍中、征西大將軍,賜以鎧馬。弋仲曰:「汝看
 老羌堪破賊以不?」於是貫鉀跨馬于庭中,策馬南馳,不辭而出,遂滅梁犢。以功加劍履上殿,入朝不趨,進封西平郡公。



 冉閔之亂,弋仲率眾討閔,次于混橋。石祗僭號于襄國,以弋仲為右丞相,待以殊禮。祗與閔相攻,弋仲遣其子襄救祗,戒襄曰:「汝才十倍於閔,若不梟擒,不須復見我也。」襄擊閔於常盧澤,大破之而歸。弋仲怒襄之不擒閔也,杖之一百。



 弋仲部曲馬何羅博學有文才,張豺之輔石世也,背弋仲歸豺,豺以為尚書郎。豺敗,復歸,咸勸殺之。弋仲曰:「今正是招才納奇之日,當收其力用,不足害也。」以為參軍。其寬恕如此。



 弋仲有子四十二人,
 常戒諸子曰:「吾本以晉室大亂,石氏待吾厚,故欲討其賊臣以報其德。今石氏已滅,中原無主,自古以來未有戎狄作天子者。我死,汝便歸晉,當竭盡臣節,無為不義之事。」乃遣使請降。永和七年,拜弋仲使持節、六夷大都督、都督江、淮諸軍事、車騎大將軍、儀同三司、大單于,封高陵郡公。八年,卒,時年七十三。



 子襄之入關也,為苻生所敗,弋仲之柩為生所得,生以王禮葬之于天水冀縣。萇僭位,追謚曰景元皇帝,廟號始祖,墓曰高陵,置園邑五百家。



 襄
 字景國,弋仲之第五子也。年十七,身長八尺五寸,臂垂過膝,雄武多才藝,明察善撫納,士眾愛敬之,咸請為嗣。弋仲弗許,百姓固請者日有千數,乃授之以兵。石祗僭號,以襄為使持節、驃騎將軍、護烏丸校尉、豫州刺史、新昌公。晉遣使拜襄持節、平北將軍、并州刺史、即丘縣公。



 弋仲死,襄祕不發喪,率戶六萬南攻陽平、元城、發干,皆破之,殺掠三千餘家,屯于碻磝津。以太原王亮為長史,天水尹赤為司馬,略陽伏子成為左部帥,南安斂岐為右部帥,略陽黑那為前部帥,強白為後部帥,太原薛贊、略陽王權翼為參軍。南至滎陽,始發喪行服。與高昌、
 李歷戰于麻田,馬中流矢死,賴其弟萇以免。晉處襄於譙城,遣五弟為任,單騎度淮,見豫州刺史謝尚于壽春。尚命去仗衛,幅巾以待之,一面交款,便若平生。



 襄少有高名,雄武冠世,好學博通,雅善談論,英濟之稱著于南夏。中軍將軍、揚州刺史殷浩憚其威名,乃因襄諸弟,頻遣刺客殺襄,刺客皆推誠告實,襄待之若舊。浩潛遣將軍魏憬率五千餘人襲襄,襄乃斬憬而并其眾。浩愈惡之,乃使將軍劉啟守譙,遷襄于梁國蠡臺,表授梁國內史。襄遣權翼詣浩,浩曰:「姚平北每舉動自由,豈所望也。」翼曰:「將軍輕納姦言,自生疑貳,愚謂猜嫌之由,不在於
 彼。」浩曰:「姚君縱放小人,盜竊吾馬,王臣之體固若是乎?」翼曰:「將軍謂姚平北以威武自強,終為難保,校兵練眾,將懲不恪,取馬者欲以自衛耳。」浩曰:「何至是也。」浩遣謝萬討襄,襄逆擊破之。浩甚怒,會聞關中有變,浩率眾北伐,襄乃要擊浩於山桑,大敗之,斬獲萬計,收其資仗。使兄益守山桑壘,復如淮南。浩遣劉啟、王彬之伐山桑,襄自淮南擊滅之,鼓行濟淮,屯于盱眙,招掠流人,眾至七萬,分置守宰,勸課農桑,遣使建鄴,罪狀殷浩,并自陳謝。



 流人郭斁等千餘人執晉堂邑內史劉仕降于襄,朝延大震,以吏部尚書周閔為中軍將軍,緣江備守。襄將佐
 部眾皆北人,咸勸襄北還。襄方軌北引,自稱大將軍、大單于,進攻外黃,為晉邊將所敗。襄收散卒而勤撫恤之,於是復振。乃據許昌,將如河東以圖關右,自許遂攻洛陽,踰月不剋。其長史王亮諫襄曰:「公英略蓋天下,士眾思效力命,不可損威勞眾,守此孤城。宜還河北,以弘遠略。」襄曰:「洛陽雖小,山河四塞之固,亦是用武之地。吾欲先據洛陽,然後開建大業。」俄而亮卒,襄哭之甚慟,曰:「天將不欲成吾事乎?王亮捨我去也!」



 晉征西大將軍桓溫自江陵伐襄,戰于伊水北,為溫所敗,率麾下數千騎奔於北山。其夜,百姓棄妻子隨襄者五千餘人,屯據陽鄉,
 赴者又四千餘戶。襄前後敗喪數矣,眾知襄所在,輒扶老攜幼奔馳而赴之。時或傳襄創重不濟,溫軍所得士女莫不北望揮涕。其得物情如此。先是,弘農楊亮歸襄,襄待以客禮。後奔桓溫,溫問襄於亮,亮曰:「神明器宇,孫策之儔,而雄武過之。」其見重如是。



 襄尋徙北屈,將圖關中,進屯杏城,遣其從兄輔國姚蘭略地鄜城,使其兄益及將軍王欽盧招集北地戎夏,歸附者五萬餘戶。苻生遣其將苻飛拒戰,蘭敗,為飛所執。襄率眾西引,生又遣苻堅、鄧羌等要之。襄將戰,沙門智通固諫襄,宜厲兵收眾,更思後舉。襄曰:「二雄不俱立,冀天不棄德以濟黎元,
 吾計決矣。」會羌師來逼,襄怒,遂長驅而進,戰于三原。襄敗,為堅所殺,時年二十七,是歲晉升平元年也。苻生以公禮葬之。萇僭號,追謚魏武王,封襄孫延定為東城侯。



 萇字景茂,弋仲第二十四子也。少聰哲,多權略,廓落任率,不修行業,諸兄皆奇之。隨襄征伐,每參大謀。襄之寇洛陽也,夢萇服袞衣,升御坐,諸酋長皆侍立,旦謂將佐曰:「吾夢如此,此兒志度不恒,或能大起吾族。」襄之敗于麻田也,馬中流矢死,萇下馬以授襄,襄曰:「汝何以自免?」萇曰:「但令兄濟,豎子安敢害萇!」會救至,俱免。



 及襄死,萇
 率諸弟降于苻生。苻堅以萇為揚武將軍。歷左衛將軍,隴東、汲郡、河東、武都、武威、巴西、扶風太守,寧、幽、兗三州刺史,復為揚武將軍,步兵校尉,封益都侯。為堅將,累有大功。



 初,萇隨楊安伐蜀,嘗晝寢水旁,上有神光煥然,左右咸異之。及苻堅寇晉,以萇為龍驤將軍、督益、梁州諸軍事,謂萇曰:「朕本以龍驤建業,龍驤之號未曾假人,今特以相授,山南之事一以委卿。」堅左將軍竇衝進曰:「王者無戲言,此將不祥之徵也,惟陛下察之。」堅默然。



 堅既敗于淮南,歸長安,慕容泓起兵叛堅。堅遣子叡討之,以萇為司馬。為泓所敗,叡死之。萇遣龍驤長史趙都詣堅
 謝罪,堅怒,殺之。萇懼,奔于渭北,遂如馬牧。西州豪族尹詳、趙曜、王欽盧、王欽盧、牛雙、狄廣、張乾等率五萬餘家,咸推萇為盟主。萇將距之,天水尹緯說萇曰:「今百六之數既臻,秦亡之兆已見,以將軍威靈命世,必能匡濟時艱,故豪傑驅馳,咸同推仰。明公宜降心從議,以副群望,不可坐觀沈溺而不拯救之。」萇乃從緯謀,以太元九年自稱大將軍、大單于、萬年秦王,大赦境內,年號白雀,稱制行事。以天水尹詳、南安龐演為左右長史,南安姚晃、尹緯為左右司馬,天水狄伯支、焦虔、梁希、龐魏、任謙為從事中郎,姜訓、閻遵為掾屬,王據、焦世、蔣秀、尹延年、牛雙、張乾
 為參軍,王欽盧、姚方成、王破虜、楊難、尹嵩、裴騎、趙曜、狄廣、黨刪等為帥。



 時慕容沖與苻堅相攻,眾甚盛。萇將西上,恐沖遏之,乃遣使通和,以子崇為質於沖,進屯北地,厲兵積粟,以觀時變。苻堅先徙晉人李祥等數千戶于敷陸,至是,降於萇,北地、新平、安定羌胡降者十餘萬戶。堅率諸將攻之,不能剋。



 萇聞容慕沖攻長安,議進趨之計,群下咸曰:「宜先據咸陽以制天下。」萇曰:「燕因懷舊之士而起兵,若功成事捷,咸有東歸之思,安能久固秦川!吾欲移兵嶺北,廣收資實,須秦弊燕迴,然後垂拱取之。兵不血刃,坐定天下,此卞莊得二之義也。」堅寧朔將軍
 宋方率騎三千從雲中將赴長安,萇自貳縣要破之,方單馬奔免,其司馬田晃率眾降萇。萇遣諸將攻新平,克之,因略地至安定,嶺北諸城盡降之。



 時苻堅為慕容沖所逼,走入五將山。沖入長安。堅司隸校尉權翼、尚書趙遷、大鴻臚皇甫覆、光祿大夫薛讚、扶風太守段鏗等文武數百人奔於萇。萇遣驍騎將軍吳忠率騎圍堅,萇如新平。俄而忠執堅,送之。



 慕容沖遣其車騎大將軍高蓋率眾五萬來伐,戰于新平南,大破之,蓋率麾下數千人來降,拜散騎常侍。



 沖既率眾東下,長安空虛。盧水郝奴稱帝于長安,渭北盡應之。扶風王驎有眾數千,保據馬
 嵬。奴遣弟多攻驎。萇伐驎,破之,驎走漢中。執多而進攻奴,降之。



 以太元十一年萇僭即皇帝位於長安,大赦,改元曰建初,國號大秦,改長安曰常安。立妻虵氏為皇后,子興為皇太子,置百官。自謂以火德承苻氏木行,服色如漢氏承周故事。徙安定五千餘戶於長安。以弟征虜緒為司隸校尉,鎮長安。



 萇如安定,擊平涼胡金熙、鮮卑沒奕於,大破之。遂如秦州,與苻堅秦州刺史王統相持,天水屠各、略陽羌胡應萇者二萬餘戶,統懼,乃降。因饗將士于上邽,南安人古成詵進曰:「臣州人殷地險,俊傑如林,用武之國也。王秦州不能收拔賢才,三分鼎足,而
 坐玩珠玉,以至于此。陛下宜散秦州金帛以施六軍,旌賢表善以副鄙州之望。」萇善之,擢為尚書郎。拜弟碩德都督隴右諸軍事、征西將軍、秦州刺史,領護東羌校尉,鎮上邽。



 萇還安定,修德政,布惠化,省非急之費,以救時弊,閭閻之士有豪介之善者,皆顯異之。



 萇復如秦州,為苻登所敗,語在《登傳》。以其太子興鎮長安,而與登相距。登馮翊太守蘭犢與苻師奴離貳,慕容永攻之,犢遣使請救。萇將赴救,尚書令姚旻、左僕射尹緯等言於萇曰:「苻登近在瓦亭,陛下未宜輕舉。」萇曰:「登遲重少決,每失時機,聞吾自行,正當廣集兵資,必不能輕軍深入。兩月
 之間,足可剋此三豎,吾事必矣。」遂師次于渥源。師奴率眾來距,大戰,敗之,盡俘其眾。又擒蘭犢,收其士馬。萇乃掘苻堅尸,鞭撻無數,裸剝衣裳,薦之以棘,坎土而埋之。慕容永征西將軍王宣率眾降萇。



 初,關西雄傑以苻氏既終,萇雄略命世,天下之事可一旦而定。萇既與苻登相持積年,數為登所敗,遠近咸懷去就之計,唯征虜齊難、冠軍徐洛生、輔國劉郭單、冠威彌姐婆觸、龍驤趙惡地、鎮北梁國兒等守忠不貳,並留子弟守營,供繼軍糧,身將精卒,隨萇征伐。時諸營既多,故號萇軍為大營,大營之號自此始也。時天大雪,萇下書深自責罰,散後宮
 文綺珍寶以供戎事,身食一味,妻不重綵。將帥死王事者,加秩二等,士卒戰沒,皆有褒贈。立太學,禮先賢之後。



 敦煌索盧曜請刺苻登,萇曰:「卿以身徇難,將為誰乎?」曜曰:「臣死之後,深以友人隴西辛暹仰託。」萇遣之。事發,為登所殺,萇以暹為騎都尉。



 登進逼安定,諸將勸萇決戰,萇曰:「與窮寇競勝,兵家之下。吾將以計取之。」於是留其尚書令姚旻守安定,夜襲登輜重於大界,剋之。諸將或欲因登駭亂擊之,萇曰:「登眾雖亂,怒氣猶盛,未可輕也。」遂止。萇以安定地狹,且逼苻登,使姚碩德鎮安定,徙安定千餘家安于陰密,遣弟征南靖鎮之。



 立社稷于長安。
 百姓年七十有德行者,拜為中大夫,歲賜牛酒。



 尹緯、姚晃謂古成詵曰:「苻登窮寇,歷年未滅,姦雄鴟峙,所在糾扇,夷夏皆貳,將若之何?」詵曰:「主上權略無方,信賞必罰,賢能之士,咸懷樂推,豈慮大業不成,氐賊不滅乎!」緯曰:「登窮寇未滅,姦雄所在扇合,吾等寧無懼乎?」詵曰:「三秦天府之國,主上十分已有其八。今所在可慮者,苻登、楊定、雷惡地耳,自餘瑣瑣,焉足論哉!然惡地地狹眾寡,不足為憂。苻登藉烏合犬羊,偷存假息,料其智勇,非至尊之匹。霸王之起,必有驅除,然後剋定大業。昔漢、魏之興也,皆十有餘年,乃能一同於海內,五六年間未為久也。
 主上神略內明,英武外發,可謂無敵於天下耳,取登有餘力。願布德行仁,招賢納士,厲兵秣馬,以候天機。如其鴻業不成者,詵請腰斬以謝明公。」緯言之於萇,萇大悅,賜詵爵關內侯。



 雷惡地率眾降萇,拜為鎮東將軍。魏褐飛自稱大將軍、衝天王,率氐胡數萬人攻安北姚當城於杏城,雷惡地應之,攻鎮東姚漢得於李潤。萇議將討之,群臣咸曰:「陛下不憂六十里苻登,乃憂六百里褐飛?」萇曰:「登非可卒殄,吾城亦非登所能卒圖。惡地多智,非常人也。南引褐飛,東結董成,甘言美說以成姦謀,若得杏城、李潤,惡地據之,控制遠近,相為羽翼,長安東北非
 復吾有。」於是潛軍赴之。萇時眾不滿二千,褐飛、惡地眾至數萬,氐胡赴之者首尾不絕。萇每見一軍至,輒有喜色。群下怪而問之,萇曰:「今同惡相濟,皆來會集,吾得乘勝席卷,一舉而覆其巢穴,東北無復餘也。」褐飛等以萇兵少,盡眾來攻。萇固壘不戰,示之以弱,潛遣子崇率騎數百,出其不意,以乘其後。褐飛兵擾亂,萇遣鎮遠王超、平遠譚亮率步騎擊之,褐飛眾大潰,斬褐飛及首級萬餘。惡地請降,萇待之如初。惡地每謂人曰:「吾自言智勇所施,足為一時之傑。校數諸雄,如吾之徒,皆應跨據一方,獸嘯千里。遇姚公智力摧屈,是吾分也。」惡地猛毅清
 肅,不可干以非義,嶺北諸豪皆敬憚之。



 萇命其將當城於營處一柵孔中蒔樹一根,以旌戰功。歲餘,問之,城曰:「營所至小,已廣之矣。」萇曰:「少來鬥戰無如此快,以千六百人破三萬眾,國之事業,由此剋舉。小乃為奇,大何足貴!」



 貳城胡曹寅、王達獻馬三千匹。以寅為鎮北將軍、并州刺史,達鎮遠將軍、金城太守。



 萇性簡率,群下有過,或面加罵辱。太常權翼言於萇曰:「陛下弘達自任,不修小節,駕馭群雄,苞羅俊異,棄嫌錄善,有高祖之量。然輕慢之風,所宜除也。」萇曰:「吾之性也。吾於舜之美,未有片焉;漢祖之短,已收其一。若不聞讜言,安知過也!」



 南羌竇鴦
 率戶五千來降,拜安西將軍。



 萇下書,有復私仇者,皆誅之。將吏亡滅者,各隨所親以立後,振給長育之。



 鎮東茍曜據逆萬堡,密引苻登。萇與登戰,敗於馬頭原,收眾復戰。姚碩德謂諸將曰:「上慎於輕戰,每欲以計取之。今戰既失利,而更逼賊者,必有由也。」萇聞而謂碩德曰:「登用兵遲緩,不識虛實,今輕兵直進,逕據吾東,必茍曜豎子與之連結也。事久變成,其禍難測。所以速戰者,欲使豎子謀之未就,好之未深,散敗其事耳。」進戰,大敗之,登退屯于郿。登將金槌以新平降萇,萇輕將數百騎入槌營。群下諫之,萇曰:「槌既去苻登,復欲圖我,將安所歸!且懷
 德初附,推款委質,吾復以不信待之,何以御物乎!」群氐果有異謀,槌不從而止。



 萇如陰密攻登,敕其太子興曰:「茍曜好姦變,將為國害,聞吾還北,必來見汝,汝便執之。」茍曜果見興於長安,興遣尹緯讓而誅之。



 萇大敗登于安定東,置酒高會,諸將咸曰:「若值魏武王,不令此賊至今,陛下將牢太過耳。」萇笑曰:「吾不如亡兄有四:身長八尺五寸,臂垂過膝,人望而畏之,一也;當十萬之眾,與天下爭衡,望麾而進,前無橫陣,二也;溫古知今,講論道藝,駕馭英雄,收羅雋異,三也;董率大眾,履險若夷,上下咸允,人盡死力,四也。所以得建立功業,策任群賢者,正望
 算略中一片耳。」群臣咸稱萬歲。



 萇下書令留臺諸鎮各置學官,勿有所廢,考試優劣,隨才擢敘。苻登驃騎將軍沒奕於率戶六千降,拜使持節、車騎將軍、高平公。



 萇寢疾,遣姚碩德鎮李潤,尹緯守長安,召其太子興詣行營。征南姚方成言於興曰:「今寇賊未滅,上復寢疾,王統、苻胤等皆有部曲,終為人害,宜盡除之。」興於是誅苻胤、王統、王廣、徐成、毛盛,乃赴召。興至,萇怒曰:「王統兄弟是吾州里,無他遠志,徐成等昔在秦朝,並為名將。天下小定,吾方任之,奈何輒便誅害,令人喪氣!」



 萇下書,兵吏從征伐,戶在大營者,世世復其家,無所豫。



 苻登與竇衝相持,
 萇議擊之,尹緯言於萇曰:「太子純厚之稱,著於遐邇,將領英略,未為遠近所知。宜遣太子親行,可以漸廣威武,防窺窬之原。」萇從之,戎興曰:「賊徒知汝轉近,必相驅入堡,聚而掩之,無不剋矣。」比至胡空堡,衝圍自解。登聞興向胡空堡,引還,興因襲平涼,大獲而歸,咸如萇策。使興還鎮長安。



 萇下書除妖謗之言及赦前姦穢,有相劾舉者,皆以其罪罪之。



 晉平遠將軍、護氐校尉楊佛嵩率胡蜀三千餘戶降于萇,晉將楊佺期、趙睦追之。遣姚崇赴救,大敗晉師,斬趙睦。以佛嵩為鎮東將軍。



 萇如長安,至於新支堡,疾篤,輿疾而進。夢苻堅將天官使者、鬼兵數百突
 入營中,萇懼,走入宮,宮人迎萇刺鬼,誤中萇陰,鬼相謂曰:「正中死處。」拔矛,出血石餘。寤而驚悸,遂患陰腫,醫刺之,出血如夢。萇遂狂言,或稱「臣萇,殺陛下者兄襄,非臣之罪,願不枉臣。」至長安,召太尉姚旻、尚書左僕射尹緯、右僕射姚晃、尚書狄伯支等入,受遺輔政。萇謂興曰:「有毀此諸人者,慎勿受之。汝撫骨肉以仁,接大臣以禮,待物以信,遇黔首以恩,四者既備,吾無憂矣。」以太元十八年死,時年六十四,在位八年。偽謚武昭皇帝,廟號太祖,墓稱原陵。



\end{pinyinscope}