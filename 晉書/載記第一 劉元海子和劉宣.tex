\article{載記第一 劉元海子和劉宣}

\begin{pinyinscope}
劉元海
 \gezhu{
  子和劉宣}



 劉元海,新興匈奴人,冒頓之後也。名犯高祖廟諱,故稱其字焉。初,漢高祖以宗女為公主,以妻冒頓,約為兄弟,故其子孫遂冒姓劉氏。建武初,烏珠留若鞮單于子右奧鞬日逐王比自立為南單于,入居西河美稷,今離石左國城即單于所徙庭也。中平中,單于羌渠使子於扶羅將兵助漢,討平黃巾。會羌渠為國人所殺,於扶羅以
 其眾留漢,自立為單于。屬董卓之亂,寇掠太原、河東,屯於河內。於扶羅死,弟呼廚泉立,以於扶羅子豹為左賢王,即元海之父也。魏武分其眾為五部,以豹為左部帥,其餘部帥皆以劉氏為之。太康中,改置都尉,左部居太原茲氏,右部居祁,南部居蒲子,北部居新興,中部居大陵。劉氏雖分居五部,然皆居於晉陽汾澗之濱。



 豹妻呼延氏,魏嘉平中祈子於龍門,俄而有一大魚,頂有二角,軒鬐躍鱗而至祭所,久之乃去。巫覡皆異之,曰:「此嘉祥也。」其夜夢旦所見魚變為人,左手把一物,大如半雞子,光景非常,授呼延氏,曰:「此是日精,服之生貴子。」寤而告
 豹,豹曰:「吉徵也。吾昔從邯鄲張冏母司徒氏相,云吾當有貴子孫,三世必大昌,仿像相符矣。」自是十三月而生元海,左手文有其名,遂以名焉。齠齔英慧,七歲遭母憂,擗踴號叫,哀感旁鄰,宗族部落咸共歎賞。時司空太原王昶聞而嘉之,並遣弔賻。幼好學,師事上黨崔游,習《毛詩》、《京氏易》、《馬氏尚書》,尤好《春秋左氏傳》、《孫吳兵法》,略皆誦之,《史》、《漢》、諸子,無不綜覽。嘗謂同門生朱紀、范隆曰:「吾每觀書傳,常鄙隨陸無武,降灌無文。道由人弘,一物之不知者,固君子之所恥也。二生遇高皇而不能建封侯之業,兩公屬太宗而不能開庠序之美,惜哉!」於是遂
 學武事,妙絕於眾,猿臂善射,膂力過人。姿儀魁偉,身長八尺四寸,鬚長三尺餘,當心有赤毫毛三根,長三尺六寸。有屯留崔懿之、襄陵公師彧等,皆善相人,及見元海,驚而相謂曰:「此人形貌非常,吾所未見也。」於是深相崇敬,推分結恩。太原王渾虛襟友之,命子濟拜焉。



 咸熙中,為任子在洛陽,文帝深待之。泰始之後,渾又屢言之於武帝。帝召與語,大悅之,謂王濟曰:「劉元海容儀機鑒,雖由余、日磾無以加也。」濟對曰:「元海儀容機鑒,實如聖旨,然其文武才幹賢於二子遠矣。陛下若任之以東南之事,吳會不足平也。」帝稱善。孔恂、楊珧進曰:「臣觀元海之
 才,當今懼無其比,陛下若輕其眾,不足以成事;若假之威權,平吳之後,恐其不復北渡也。非我族類,其心必異。任之以本部,臣竊為陛下寒心。若舉天阻之固以資之,無乃不可乎!」帝默然。



 後秦涼覆沒,帝疇咨將帥,上黨李憙曰:「陛下誠能發匈奴五部之眾,假元海一將軍之號,鼓行而西,可指期而定。」孔恂曰:「李公之言,未盡殄患之理也。」憙勃然曰:「以匈奴之勁悍,元海之曉兵,奉宣聖威,何不盡之有!」恂曰:「元海若能平涼州,斬樹機能,恐涼州方有難耳。蛟龍得雲雨,非復池中物也。」帝乃止。後王彌從洛陽東歸,元海餞彌於九曲之濱。泣謂彌曰:「王渾、李
 憙以鄉曲見知,每相稱達,讒間因之而進,深非吾願,適足為害。吾本無宦情,惟足下明之。恐死洛陽,永與子別。」因慷慨歔欷,縱酒長嘯,聲調亮然,坐者為之流涕。齊王攸時在九曲,比聞而馳遣視之,見元海在焉,言於帝曰:「陛下不除劉元海,臣恐并州不得久寧。」王渾進曰:「元海長者,渾為君王保明之。且大晉方表信殊俗,懷遠以德,如之何以無萌之疑殺人侍子,以示晉德不弘。」帝曰:「渾言是也。」



 會豹卒,以元海代為左部帥。太康末,拜北部都尉。明刑法,禁姦邪,輕財好施,推誠接物,五部俊傑無不至者。幽冀名儒,後門秀士,不遠千里,亦皆遊焉。楊駿輔
 政,以元海為建威將軍、五部大都督,封漢光鄉侯。元康末,坐部人叛出塞免官。成都王穎鎮鄴,表元海行寧朔將軍、監五部軍事。



 惠帝失馭,寇盜蜂起,元海從祖故北部都尉、左賢王劉宣等竊議曰:「昔我先人與漢約為兄弟,憂泰同之。自漢亡以來,魏晉代興,我單于雖有虛號,無復尺土之業,自諸王侯,降同編戶。今司馬氏骨肉相殘,四海鼎沸,興邦復業,此其時矣。左賢王元海姿器絕人,幹宇超世。天若不恢崇單于,終不虛生此人也。」於是密共推元海為大單于。乃使其黨呼延攸詣鄴,以謀告之。元海請歸會葬,穎弗許。乃令攸先歸,告宣等招集五
 部,引會宜陽諸胡,聲言應穎,實背之也。



 穎為皇太弟,以元海為太弟屯騎校尉。惠帝伐穎,次于蕩陰,穎假元海輔國將軍、督北城守事。及六軍敗績,穎以元海為冠軍將軍,封盧奴伯。并州刺史東嬴公騰、安北將軍王浚,起兵伐穎,元海說穎曰:「今二鎮跋扈,眾餘十萬,恐非宿衛及近都士庶所能禦之,請為殿下還說五部,以赴國難。」穎曰:「五部之眾可保發已不?縱能發之,鮮卑、烏丸勁速如風雲,何易可當邪?吾欲奉乘輿還洛陽,避其鋒銳,徐傳檄天下,以逆順制之。君意何如?」元海曰:「殿下武皇帝之子,有殊勳於王室,威恩光洽,四海欽風,孰不思為殿下
 沒命投軀者哉,何難發之有乎!王浚豎子,東嬴疏屬,豈能與殿下爭衡邪!殿下一發鄴宮,示弱於人,洛陽可復至乎?縱達洛陽,威權不復在殿下也。紙檄尺書,誰為人奉之!且東胡之悍不踰五部,願殿下勉撫士眾,靖以鎮之,當為殿下以二部摧東嬴,三部梟王浚,二豎之首可指日而懸矣。」穎悅,拜元海為北單于、參丞相軍事。元海至左國城,劉宣等上大單于之號,二旬之間,眾已五萬,都于離石。



 王浚使將軍祁弘率鮮卑攻鄴,穎敗,挾天子南奔洛陽。元海曰:「穎不用吾言,逆自奔潰,真奴才也。然吾與其有言矣,不可不救。」於是命右於陸王劉景、左獨
 鹿王劉延年等率步騎二萬,將討鮮卑。劉宣等固諫曰:「晉為無道,奴隸御我,是以右賢王猛不勝其忿。屬晉綱未馳,大事不遂,右賢塗地,單于之恥也。今司馬氏父子兄弟自相魚肉,此天厭晉德,授之於我。單于積德在躬,為晉人所服,方當興我邦族,復呼韓邪之業,鮮卑、烏丸可以為援,奈何距之而拯仇敵!今天假手於我,不可違也。違天不祥,逆眾不濟;天與不取,反受其咎。願單于勿疑。」元海曰:「善。當為崇岡峻阜,何能為培塿乎!夫帝王豈有常哉,大禹出於西戎,文王生於東夷,顧惟德所授耳。今見眾十餘萬,皆一當晉十,鼓行而摧亂晉,猶拉枯耳。
 上可成漢高之業,下不失為魏氏。雖然,晉人未必同我。漢有天下世長,恩德結於人心,是以昭烈崎嶇於一州之地,而能抗衡於天下。吾又漢氏之甥,約為兄弟,兄亡弟紹,不亦可乎?且可稱漢,追尊後主,以懷人望。」乃遷於左國城,遠人歸附者數萬。



 永興元年,元海乃為壇於南郊,僭即漢王位,下令曰:「昔我太祖高皇帝以神武應期,廓開大業。太宗孝文皇帝重以明德,升平漢道。世宗孝武皇帝拓土攘夷,地過唐日。中宗孝宣皇帝搜揚俊乂,多士盈朝。是我祖宗道邁三王,功高五帝,故卜年倍於夏商,卜世過於姬氏。而元成多僻,哀平短祚,賊臣王莽,
 滔天篡逆。我世祖光武皇帝誕資聖武,恢復鴻基,祀漢配天,不失舊物,俾三光晦而復明,神器幽而復顯。顯宗孝明皇帝、肅宗孝章皇帝累葉重暉,炎光再闡。自和安已後,皇綱漸頹,天步艱難,國統頻絕。黃巾海沸於九州,群閹毒流於四海,董卓因之肆其猖勃,曹操父子凶逆相尋。故孝愍委棄萬國,昭烈播越岷蜀,冀否終有泰,旋軫舊京。何圖天未悔禍,後帝窘辱。自社稷淪喪,宗廟之不血食四十年于茲矣。今天誘其衷,悔禍皇漢,使司馬氏父子兄弟迭相殘滅。黎庶塗炭,靡所控告。孤今猥為群公所推,紹脩三祖之業。顧茲尪闇,戰惶靡厝。但以大
 恥未雪,社稷無主,銜膽棲冰,勉從群議。」乃赦其境內,年號元熙,追尊劉禪為孝懷皇帝,立漢高祖以下三祖五宗神主而祭之。立其妻呼延氏為王后。置百官,以劉宣為丞相,崔游為御史大夫,劉宏為太尉,其餘拜授各有差。



 東嬴公騰使將軍聶玄討之,戰于大陵,玄師敗績,騰懼,率并州二萬餘戶下山東,遂所在為寇。元海遣其建武將軍劉曜寇太原、泫氏、屯留、長子、中都,皆陷之。二年,騰又遣司馬瑜、周良、石鮮等討之,次于離石汾城。元海遣其武牙將軍劉欽等六軍距瑜等,四戰,瑜皆敗,欽振旅而歸。是歲,離石大饑,遷于黎亭,以就邸閣穀,留其太
 尉劉宏、護軍馬景守離石,使大司農卜豫運糧以給之。以其前將軍劉景為使持節、征討大都督、大將軍,要擊并州刺史劉琨于版橋,為琨所敗,琨遂據晉陽。其侍中劉殷、王育進諫元海曰:「殿下自起兵以來,漸已一周,而顓守偏方,王威未震。誠能命將四出,決機一擲,梟劉琨,定河東,建帝號,鼓行而南,剋長安而都之,以關中之眾席卷洛陽,如指掌耳。此高皇帝之所以創啟鴻基,剋殄彊楚者也。」元海悅曰:「此孤心也。」遂進據河東,攻寇蒲阪、平陽,皆陷之。元海遂入都蒲子,河東、平陽屬縣壘壁盡降。時汲桑起兵趙魏,上郡四部鮮卑陸逐延、氏酋大單
 于征、東萊王彌及石勒等並相次降之,元海悉署其官爵。



 永嘉二年,元海僭即皇帝位,大赦境內,改元永鳳。以其大將軍劉和為大司馬,封梁王,尚書令劉歡樂為大司徒,封陳留王,御史大夫呼延翼為大司空,封雁州郡公,宗室以親疏為等,悉封郡縣王,異姓以勳謀為差,皆封郡縣公侯。太史令宣于脩之言於元海曰:「陛下雖龍興鳳翔。奄受大命,然遺晉未殄,皇居仄陋,紫宮之變,猶鐘晉氏,不出三年,必剋洛陽。薄子崎嶇,非可久安。平陽勢有紫氣,兼陶唐舊都,願陛下上迎乾象,下協坤祥。」於是遷都平陽。汾水中得玉璽,文曰「有新保之」,蓋王莽時
 璽也。得者因增「泉海光」三字,元海以為己瑞,大赦境內,改年河瑞。封子裕為齊王,隆為魯王。



 於是命其子聰與王彌進寇洛陽,劉曜與趙固等為之後繼。東海王越遣平北將軍曹武、將軍宋抽、彭默等距之,王師敗績。聰等長驅至宜陽,平昌公模遣將軍淳于定、呂毅等自長安討之,戰于宜陽,定等敗績。聰恃連勝,不設備,弘農太守垣延詐降。夜襲,聰軍大敗而還,元海素服迎師。



 是冬,復大發卒,遣聰、彌與劉曜、劉景等率精騎五萬寇洛陽,使呼延翼率步卒繼之,敗王師於河南。聰進屯于西明門,護軍賈胤夜薄之,戰于大夏門,斬聰將呼延顥,其眾遂
 潰。聰迴軍而南。壁於洛水,尋進屯宣陽門,曜屯上東門,彌屯廣陽門,景攻大夏門,聰親祈嵩嶽,令其將劉厲、呼延朗等督留軍。東海王越命參軍孫詢、將軍丘光、樓裒等率帳下勁卒三千,自宣陽門擊朗,斬之。聰聞而馳還。厲懼聰之罪己也,赴水而死。王彌謂聰曰:「今既失利,洛陽猶固,殿下不如還師,徐為後舉。下官當於袞豫之間收兵積穀,伏聽嚴期。」宣于脩之又言於元海曰:「歲在辛未,當得洛陽。今晉氣猶盛,大軍不歸,必敗。」元海馳遣黃門郎傅詢召聰等還師。王彌出自轘轅,越遣薄盛等追擊彌,戰于新汲,彌師敗績。於是攝薄阪之戍,還於平陽。



 以劉歡樂為太傅,劉聰為大司徒,劉延年為大司空,劉洋為大司馬,赦其境內。立其妻單氏為皇后,子和為皇太子,封子乂為北海王。



 元海寢疾,將為顧託之計,以歡樂為太宰,洋為太傅,延年為太保,聰為大司馬、大單于,並錄尚書事,置單于臺于平陽西,以其子裕為大司徒。元海疾篤,召歡樂及洋等人禁中受遺詔輔政。以永嘉四年死,在位六年,偽謚光文皇帝,廟號高祖,墓號永光陵。子和立。



 和字玄泰。身長八尺,雄毅美姿儀,好學夙成,習《毛詩》、《左
 氏春秋》、《鄭氏易》。及為儲貳,內多猜忌,馭下無恩。元海死,和嗣偽位。其衛尉西昌王劉銳、宗正呼延攸恨不參顧命也,說和曰:「先帝不惟輕重之計,而使三王總彊兵於內,大司馬握十萬勁卒居于近郊,陛下今便為寄坐耳。此之禍難,未可測也,顧陛下早為之所。」和即攸之甥也,深然之,召其領軍劉盛及劉欽、馬景等告之。盛曰:「先帝尚在殯宮,四王未有逆節,今忽一旦自相魚肉,臣恐人不食陛下之餘。四海未定,大業甫爾,願陛下以上成先帝鴻基為志,且塞耳勿聽此狂簡之言也。《詩》云:『豈無他人,不如我同父。』陛下既不信諸弟,復誰可信哉!」銳、攸怒
 曰:「今日之議,理無有二。」於是命左右刃之。景懼曰:「惟陛下詔,臣等以死奉之,蔑不濟矣。」乃相與盟于東堂,使銳、景攻聰,攸率劉安國攻裕,使侍中劉乘、武衛劉欽攻魯王隆,尚書田密、武衛劉璿攻北海王乂。密、璿等使人斬關奔于聰,聰命貫甲以待之。銳知聰之有備也,馳還,與攸、乘等會攻隆、裕。攸、乘懼安國、欽之有異志也,斬之。是日,斬裕及隆。聰攻西明門,剋之。銳等奔入南宮,前鋒隨之,斬和于光極西室。銳、攸梟首通衢。



 劉宣,字士則。朴鈍少言,好學修潔。師事樂安孫炎,沈精
 積思,不舍晝夜,好《毛詩》、《左氏傳》。炎每嘆之曰:「宣若遇漢武,當逾於金日磾也。」學成而返,不出門閭蓋數年。每讀《漢書》,至《蕭何》、《鄧禹傳》,未曾不反覆詠之,曰:「大丈夫若遭二祖,終不令二公獨擅美於前矣。」並州刺史王廣言之於武帝,帝召見,嘉其占對,因曰:「吾未見宣,謂廣言虛耳。今見其進止風儀,真所謂如圭如璋,觀其性質,足能撫集本部。」乃以宣為右部都督,特給赤幛曲蓋。蒞官清恪,所部懷之。元海即王位,宣之謀也,故特荷尊重,勛戚莫二,軍國內外靡不專之。



\end{pinyinscope}