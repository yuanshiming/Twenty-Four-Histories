\article{列傳第五十一 王遜蔡豹羊鑒劉胤桓宣族子伊硃伺毛寶子穆之劉遐鄧岳子遐硃序}

\begin{pinyinscope}
王遜蔡豹羊鑒
 劉胤桓宣
 \gezhu{
  族子伊}
 硃伺毛寶
 \gezhu{
  子穆之}
 劉遐鄧岳
 \gezhu{
  子遐}
 硃序



 王遜,字邵伯,魏興人也。仕郡察孝廉,為吏部令史,轉殿中將軍。累遷上洛太守。私牛馬在郡生駒犢者,秩滿悉以付官,云是郡中所產也。轉魏興太守。惠帝末,西南夷叛,寧州刺史李毅卒,城中百餘人奉毅女固守經年。永嘉四年,治中毛孟詣京師求刺史,不見省。孟固陳曰:「君亡親喪,幽閉窮城,萬里訴哀,不垂愍救。既慚包胥無哭
 秦之感,又愧梁妻無崩城之驗,存不若亡,乞賜臣死。」朝廷憐之,乃以遜為南夷校尉、寧州刺史,使於郡便之鎮。與孟俱行,道遇寇賊,踰年乃至。外逼李雄,內有夷寇,吏士散沒,城邑丘墟。遜披荒糾厲,收聚離散,專杖威刑,鞭撻殊俗。遜未到州,遙舉董聯為秀才,建寧功曹周悅謂聯非才,不下版檄。遜既到,收悅殺之。悅弟潛謀殺遜,以前建寧太守趙混子濤代為刺史。事覺,並誅之。又誅豪右不奉法度者數十家。征伐諸夷,俘馘千計,獲馬及牛羊數萬餘,於是莫不振服,威行寧土。又遣子澄奉表勸進於元帝,帝嘉之,累加散騎常侍、安南將軍、假節,校
 尉、刺史如故,賜爵褒中縣公。遜以地勢形便,上分牂柯為平夷郡,分朱提為南廣郡,分建寧為夜郎郡,分永昌為梁水郡,又改益州郡為晉寧郡,事皆施行。



 先是,越巂太守李釗為李雄所執,自蜀逃歸,遜復以釗為越巂太守。李雄遣李驤、任回攻釗,釗自南秦與漢嘉太守王載共距之,戰于溫水,釗敗績,載遂以二郡附雄。後驤等又渡瀘水寇寧州,遜使將軍姚崇、爨琛距之,戰于堂狼,大破驤等,崇追至瀘水,透水死者千餘人。崇以道遠不敢渡水,遜以崇不窮追也,怒囚群帥,執崇,鞭之,怒甚,髮上衝冠,冠為之裂,夜中卒。



 遜在州十四年,州人復立遜中
 子堅行州府事。詔除堅為南夷校尉、寧州刺史、假節,謚遜曰壯。陶侃懼堅不能抗對蜀人,太寧末,表以零陵太守尹奉為寧州,徵堅還京,病卒。兄澄襲爵,歷魏興太守、散騎常侍。



 蔡豹,字士宣,陳留圉城人。高祖質,漢衛尉,左中郎將邕之叔父也。祖睦,魏尚書。父宏,陰平太守。豹有氣幹,歷河南丞,長樂、清河太守。避亂南渡,元帝以為振武將軍、臨淮太守,遷建威將軍、徐州刺史。初,祖逖為徐州,豹為司馬,素易豹。至是,逖為豫州,而豹為徐州,俱受征討之寄,
 逖甚愧之。



 是時太山太守徐龕與彭城內史劉遐同討反賊周撫於寒山,龕將於藥斬撫。及論功,而遐先之。龕怒,以太山叛,自號安北將軍、兗州刺史,攻破東莞太守侯史旄而據其塢。石季龍伐之,龕懼,求降,元帝許焉。既而復叛歸石勒,勒遣其將王伏都、張景等數百騎助龕。詔征虜將軍羊鑒、武威將軍侯禮、臨淮太守劉遐、鮮卑段文鴦等與豹共討之。諸將畏懦,頓兵下邳,不敢前。豹欲進軍,鑒固不許。龕遣使請救於勒,勒辭以外難,而多求於龕。又王伏都等淫其室。龕知勒不救,且患伏都等縱暴,乃殺之,復求降。元帝惡其反覆不納,敕豹、鑒以時
 進討。鑒及劉遐等並疑憚不相聽從,互有表聞,故豹久不得進。尚書令刁協奏曰:「臣等伏思淮北征軍已失不速,今方盛暑,冒涉山險,山人便弓弩,習土俗,一人守阨,百夫不當。且運漕至難,一朝糧乏,非復智力所能防禦也。《書》云寧致人,不致於人。宜頓兵所在,深壁固壘,至秋不了,乃進大軍。」詔曰:「知難而退,誠合兵家之言。然小賊雖狡猾,故成擒耳。未戰而退,先自摧衄,亦古之所忌。且邵存已據賊壘,威勢既振,不可退一步也。」於是遣治書御史郝嘏為行臺,催攝令進討。豹欲徑進,鑒執不聽。協又奏免鑒官,委豹為前鋒,以鑒兵配之,降號折衝將軍,
 以責後效。豹進據卞城,欲以逼龕。時石季龍屯鉅平,將攻豹,豹夜遁。退守下邳。徐龕襲取豹輜重於檀丘,將軍留寵、陸黨力戰,死之。



 豹既敗,將歸謝罪,北中郎王舒止之,曰:「胡寇方至,使君且當攝職,為百姓障扞。賊退謝罪,不晚也。」豹從之。元帝聞豹退,使收之。使者至,王舒夜以兵圍豹,豹以為他難,率麾下擊之,聞有詔乃止。舒執豹,送至建康,斬之,尸于市三日,時年五十二。



 豹在徐土,內撫將士,外懷諸眾,甚得遠近情,聞其死,多悼惜之。無子,兄子裔字元子,散騎常侍、兗州刺史、高陽鄉侯。殷浩北伐,使裔率眾出彭城,卒於軍。



 羊鑒,字景期,太山人也。父濟,匈奴中郎將。兄煒,歷太僕、兗徐二州刺史。鑒為東陽太守,累遷太子左衛率。時徐龕反叛,司徒王導以鑒是龕州里冠族,必能制之,請遣北討。鑒深辭才非將帥。太尉郗鑒亦表謂鑒非才,不宜妄使。導不納,強啟授以征討都督,果敗績。導以舉鑒非才,請自貶,帝不從。有司正鑒斬刑,元帝詔以鑒太妃外屬,特免死,除名。久之,為少府。及王敦反,明帝以鑒敦舅,又素相親黨,微被嫌責。及成帝即位,豫討蘇峻,以功封豐城縣侯,徙光祿勳,卒。



 劉胤,字承胤,東萊掖人,漢齊悼惠王肥之後也,美姿容,善自任遇,交結時豪,名著海岱間,士咸慕之。舉賢良,辟司空掾,並不就。且天下大亂,攜母欲避地遼東,路經幽州,刺史王浚留胤,表為渤海太守。浚敗,轉依冀州刺史邵續。續徒眾寡弱,謀降於石勒,胤言於續曰:「夫田單、包胥,齊楚之小吏耳,猶能存已滅之邦,全喪敗之國。今將軍杖精銳之眾,居全勝之城,如何墜將登之功於一蕢,委忠信之人於豺狼乎!且項羽、袁紹非不強也,高祖縞冠,人應如響;曹公奉帝,而諸侯綏穆。何者?蓋逆順之理
 殊,自然之數定也。況夷戎醜類,屯結無賴,雖有犬羊之盛,終有庖宰之患,而欲託根結援,無乃殆哉!」續曰:「若如君言,計將安出?」胤曰:「瑯邪王以聖德欽明,創基江左,中興之隆可企踵而待。今為將軍計者,莫若抗大順以激義士之心,奉忠正以厲軍人之志。夫機事在密,時至難違,存亡廢興,在此舉矣。」續從之,乃殺異議者數人,遣使江南,朝廷嘉之。胤仍求自行,續厚遣之。



 既至,元帝命為丞相參軍,累遷尚書吏部郎。胤聞石季龍攻厭次,言於元帝曰:「北方鎮皆沒,惟餘邵續而已。如使君為季龍所制,孤義士之心,阻歸本之路。愚謂宜存救援。」元帝將
 遣救之,會續已沒而止。王敦素與胤交,甚欽貴之,請為右司馬。胤知敦有不臣心,枕疾不視事,以是忤敦意,出為豫章太守,辭以腳疾,詔就家授印綬。郡人莫鴻,南土豪族,因亂,殺本縣令,橫恣無道,百姓患之。胤至,誅鴻及諸豪右,界內肅然。咸和初,為平南軍司,加散騎常侍。蘇峻作亂,溫嶠率眾而下,留胤等守溢口。事平,以勳賜爵豐城子。俄而代嶠為平南將軍、都督江州諸軍事、領江州刺史、假節。



 胤位任轉高,矜豪日甚,縱酒耽樂,不恤政事,大殖財貨,商販百萬。初,胤之代嶠也,遠近皆為謂非選。陶侃、郗鑒咸云胤非方伯才,朝廷不從。或問王悅曰:「今
 大難之後,綱紀弛頓,自江陵至于建康三千餘里,流人萬計,布在江州。江州,國之南籓,要害之地,而胤以侈忲之性,臥而對之,不有外變,必有內患。」悅曰:「聞溫平南語家公云,連得惡夢,思見代者。尋云可用劉胤。此乃溫意,非家公也。」是時朝廷空罄,百官無祿,惟資江州運漕。而胤商旅繼路,以私廢公。有司奏免胤官。書始下,而胤為郭默所害,年四十九。



 子赤松嗣,尚南平長公主,位至黃門郎、義興太守。



 桓宣,譙國銍人也。祖詡,義陽太守。父弼,冠軍長史。宣開
 濟篤素,為元帝丞相舍人。時塢主張平自稱豫州刺史,樊雅自號譙郡太守,各據一城,眾數千人。帝以宣信厚,又與平、雅同州里,轉宣為參軍,使就平、雅。平、雅遣軍主簿隨宣詣丞相府受節度,帝皆加四品將軍,即其所部,使扞禦北方。南中郎將王含請宣為參軍。



 頃之,豫州刺史祖逖出屯蘆洲,遣參軍殷乂詣平、雅。乂意輕平,視其屋,云當持作馬廄,見大鑊,欲鑄作鐵器。平曰:「此是帝王大鑊,天下定後方當用之,奈何打破!」乂曰:「卿能保頭不?而惜大鑊邪!」平大怒,於坐斬乂,阻兵固守。歲餘,逖攻平殺之,而雅據譙城。逖以力弱,求助於含,含遣宣領兵五
 百助逖。逖謂宣曰:「卿先已說平、雅,信義大著於彼。今復為我說雅。雅若降者,方相擢用,不但免死而已。」宣復單馬從兩人詣雅,曰:「祖逖方欲平蕩二寇,每何卿為援。前殷乂輕薄,非豫州意。今若和解,則忠勳可立,富貴可保。若猶固執,東府赫然更遣猛將,以卿烏合之眾,憑阻窮城,強賊伺其北,國家攻其南,萬無一全也。願善量之。」雅與宣置酒結友,遣子隨宣詣逖。少日,雅便自詣逖,逖遣雅還撫其眾。雅僉謂前數罵辱,懼罪不敢降。雅復閉城自守。逖往攻之,復遣宣入說雅。雅即斬異己者,遂出降。未幾,石勒別將圍譙城,含又遣宣率眾救逖,未至而賊
 退。逖留宣討諸未服,皆破之。遷譙國內史。



 祖約之棄譙城也,宣以箋諫,不從,由是石勒遂有陳留。及約與蘇峻同反,宣謂祖智曰:「今強胡未滅,將戮力以討之,而與峻俱反,此安得久乎!使君若欲為雄霸,何不助國討峻,威名自舉。」智等不能用。宣欲諫約,遣其子戎白約求入。約知宣必諫,不聽。宣遂距約,不與之同。邵陵人陳光率部落數百家降宣,宣皆慰撫之。約還歷陽,宣將數千家欲南投尋陽,營於馬頭山。值祖煥欲襲湓口,陶侃使毛寶救之。煥遣眾攻宣,宣使戎求救於寶。寶擊煥,破之,宣因投溫嶠。嶠以戎為參軍。賊平,宣居于武昌,戎復為劉胤
 參軍。郭默害胤,復以戎為參軍。



 陶侃討默,默遣戎求救於宣,宣偽許之。西陽太守鄧嶽、武昌太守劉詡皆疑宣與默同。豫州西曹王隨曰:「宣尚背祖約,何緣同郭默邪!」嶽、詡乃遣隨詣宣以觀之。隨謂宣曰:「明府心雖不爾,無以自明,惟有以戎付隨耳。」宣乃遣戎與隨俱迎陶侃。辟戎為掾,上宣為武昌太守。尋遷監沔中軍事、南中郎將、江夏相。



 石勒荊州刺史郭敬戍襄陽。陶侃使其子平西參軍斌與宣俱攻樊城,拔之。竟陵太守李陽又破新野。敬懼,遁走。宣與陽遂平襄陽,侃使宣鎮之,以其淮南部曲立義成郡。宣招懷初附,勸課農桑,簡刑罰,略威儀,或
 載鉏耒於軺軒,或親芸獲於隴畝。十餘年間,石季龍再遣騎攻之,宣能得眾心,每以寡弱距守,論者以為次於祖逖、周訪。



 侃方欲使宣北事中原,會侃薨。後庾亮為荊州,將謀北伐,以宣為都督沔北前鋒征討軍事、平北將軍、司州刺史、假節,鎮襄陽。季龍使騎七千渡沔攻之,亮遣司馬王愆期、輔國將軍毛寶救宣。賊三面為地窟攻城,宣募精勇,出其不意,殺傷數百,多獲鎧馬,賊解圍退走。久之,宣遣步騎收南陽諸郡百姓沒賊者八千餘人以歸。庾翼代亮,欲傾國北討,更以宣為都督司梁雍三州荊州之南陽襄陽新野南鄉四郡軍事、梁州刺史、持
 節,將軍如故。以前後功,封竟陵縣男。



 宣久在襄陽,綏撫僑舊,甚有稱績。庾翼遷鎮襄陽,令宣進伐石季龍將李羆,軍次丹水,為賊所敗。翼怒,貶宣為建威將軍,使移戍峴山。宣望實俱喪,兼以老疾,時南蠻校尉王愆期守江陵,以疾求代,翼以宣為鎮南將軍、南郡太守,代愆期。宣不得志,未之官,發憤卒。追贈鎮南將軍。戎官至新野太守。



 伊字叔夏,父景,有當世才幹,仕至侍中、丹陽尹、中領軍、護軍將軍、長社侯,伊有武幹,標悟簡率,為王濛、劉惔所知,頻參諸府軍事,累遷大司馬參軍。時苻堅強盛,邊鄙
 多虞,朝議選能距捍疆場者,乃授伊淮南太守。以綏御有方,進督豫州之十二郡揚州之江西五郡軍事、建威將軍、歷陽太守,淮南如故。與謝玄共破賊別將王鑒、張蠔等,以功封宣城縣子,又進都督豫州諸軍事、西中郎將、豫州刺史。及苻堅南寇,伊與冠軍將軍謝玄、輔國將軍謝琰俱破堅於肥水,以功封永脩縣侯,進號右軍將軍,賜錢百萬,袍表千端。



 伊性謙素,雖有大功,而始終不替。善音樂,盡一時之妙,為江左第一。有蔡邕柯亭笛,常自吹之。王徽之赴召京師,泊舟青溪側。素不與徽之相識。伊於岸上過,船中客稱伊小字曰:「此桓野王也。」徽之
 便令人謂伊曰:「聞君善吹笛,試為我一奏。」伊是時已貴顯,素聞徽之名,便下車,踞胡床,為作三調,弄畢,便上車去,客主不交一言。



 時謝安女婿王國寶專利無檢行,安惡其為人,每抑制之。及孝武末年,嗜酒好內,而會稽王道子昏JT尤甚,惟狎暱諂邪,於是國寶讒諛之計稍行於主相之間。而好利險詖之徒,以安功名盛極,而構會之,嫌隙遂成。帝召伊飲宴,安侍坐。帝命伊吹笛。伊神色無迕,即吹為一弄,乃放笛云:「臣於箏分乃不及笛,然自足以韻合歌管,請以箏歌,并請一吹笛人。」帝善其調達,乃敕御妓奏笛。伊又云:「御府人於臣必自不合,臣有
 一奴,善相便串。」帝彌賞其放率,乃許召之。奴既吹笛,伊便撫箏而歌《怨詩》曰:「為君既不易,為臣良獨難。忠信事不顯,乃有見疑患。周旦佐文武,《金縢》功不刊。推心輔王政,二叔反流言。」聲節慷慨,俯仰可觀。安泣下沾衿,乃越席而就之,捋其須曰:「使君於此不凡!」帝甚有愧色。



 伊在州十年,綏撫荒雜,甚得物情。桓沖卒,遷都督江州荊州十郡豫州四郡軍事、江州刺史,將軍如故,假節。伊到鎮,以邊境無虞,宜以寬恤為務,乃上疏以江州虛秏,加連歲不登,今餘戶有五萬六千,宜并合小縣,除諸郡逋米,移州還鎮豫章。詔令移州尋陽,其餘皆聽之。伊隨宜拯撫,
 百姓賴焉。在任累年,徵拜護軍將軍。以右軍府千人自隨,配護軍府。卒官。贈右將軍,加散騎常侍,謚曰烈。



 初,伊有馬步鎧六百領,豫為表,令死乃上之。表曰:「臣過蒙殊寵,受任西籓。淮南之捷,逆兵奔北,人馬器鎧,隨處放散。于時收拾敗破,不足貫連。比年營繕,並已修整。今六合雖一,餘燼未滅,臣不以朽邁,猶欲輸效力命,仰報皇恩。此志永絕,銜恨泉壤。謹奉輸馬具裝百具、步鎧五百領,並在尋陽,請勒所屬領受。」詔曰:「伊忠誠不遂,益以傷懷,仍受其所上之鎧。」



 子肅之嗣。卒,子陵嗣。宋受禪,國除。伊弟不才,亦有將略。討孫恩,至冠軍將軍。



 朱伺,字仲文,安陸人。少為吳牙門將陶丹給使。吳平,內徙江夏。伺有武勇,而訥口,不知書,為郡將督,見鄉里士大夫,揖稱名而已。及為將,遂以謙恭稱。張昌之逆,太守弓欽走灄口,伺與同輩郴寶、布興合眾討之,不剋,乃與欽奔武昌。後更率部黨攻滅之。轉騎部曲督,加綏夷都尉。伺部曲等以諸縣附昌,惟本部唱義討逆,逆順有嫌,求別立縣,因此遂割安陸東界為灄陽縣而貫焉。



 其後陳敏作亂,陶侃時鎮江夏,以伺能水戰,曉作舟艦,乃遣作大艦,署為左甄,據江口,摧破敏前鋒。敏弟恢稱荊州
 刺史,在武昌,侃率伺及諸軍進討,破之。敏、恢既平,伺以功封亭侯,領騎督。時西陽夷賊抄掠江夏,太守楊氏每請督將議距賊之計,伺獨不言。氏曰:「朱將軍何以不言?」伺答曰:「諸人以舌擊賊,伺惟以力耳。」氏又問:「將軍前後擊賊,何以每得勝邪?」伺曰:「兩敵共對,惟當忍之。彼不能忍,我能忍,是以勝耳。」氏大笑。



 永嘉中,石勒破江夏,伺與楊氏走夏口。及陶侃來戍夏口,伺依之,加明威將軍。隨侃討杜弢,有殊功,語在侃傳。夏口之戰,伺用鐵面自衛,以弩的射賊大帥數人,皆殺之。賊挽船上岸,於水邊作陣。伺逐水上下以邀之,箭中其脛,氣色不變。諸軍尋至,
 賊潰,追擊之,皆棄船投水,死者太半。賊夜還長沙,伺追蒲圻,不及而反。加威遠將軍,赤幢曲蓋。



 建興中,陳聲率諸無賴二千餘家斷江抄掠,侃遣伺為督護討聲。聲眾雖少,伺容之不擊,求遣弟詣侃降,伺外許之。及聲去,伺乃遣勁勇要聲弟斬之,潛軍襲聲。聲正旦並出祭祀飲食,伺軍入其門方覺。聲將閻晉、鄭進皆死戰,伺軍人多傷,乃還營。聲東走,保董城。伺又率諸軍圍守之,遂重柴繞城,作高櫓,以勁弩下射之,又斷其水道。城中無水,殺牛飲血。閻晉,聲婦弟也,乃斬聲首出降。又以平蜀賊襲高之功,加伺廣威將軍,領竟陵內史。



 時王敦欲用從
 弟廙代侃為荊州,侃故將鄭攀、馬俊等乞侃於敦,敦不許。攀等以侃始滅大賊,人皆樂附,又以廙忌戾難事,謀共距之。遂屯結溳口,遣使告伺。伺外許之,而稱疾不赴。攀等遂進距廙。既而士眾疑阻,復散還橫桑口,欲入杜曾。時朱軌、趙誘、李桓率眾將擊之,攀等懼誅,以司馬孫景造謀距廙,因斬之,降軌等。



 廙將西出,遣長史劉浚留鎮揚口壘。時杜曾會請討第五猗於襄陽,伺謂廙曰:「曾是猾賊,外示西還,以疑眾心,欲誘引官軍使西,然後兼道襲揚口耳。宜大部分,未可便西。」廙性矜厲自用,兼以伺老怯難信,遂西行。曾等果馳還。廙乃遣伺歸,裁至壘,即
 為曾等所圍。劉浚以壘北門危,欲令伺守之。或說浚云:「伺與鄭攀同者。」乃轉守南門。賊知之,攻其北門。時鄭攀黨馬俊等亦來攻壘,俊妻子先在壘內,或請皮其面以示之。伺曰:「殺其妻子,未能解圍,但益其怒耳。」乃止。伺常所調弩忽噤不發,伺甚惡之。及賊攻陷北門,伺被傷退入船。初,浚開諸船底,以木掩之,名為船械。伺既入,賊舉鋋摘伺,伺逆接得鋋,反以摘賊。賊走上船屋,大喚云:「賊帥在此!」伺從船底沈行五十步,乃免。遇醫療,創小差。杜曾遣說伺云:「馬俊等感卿恩,妻孥得活。盡以卿家外內百口付俊,俊已盡心收視,卿可來也。」伺答曰:「賊無白首
 者,今吾年六十餘,不能復與卿作賊。吾死,當歸南,妻子付汝。」乃還甑山。時王廙與李桓、杜曾相持,累戰甑山下。軍士數驚喚云:「賊欲至!」伺驚創而卒。因葬甑山。



 毛寶,字碩真,滎陽陽武人也。王敦以為臨湘令。敦卒,為溫嶠平南參軍。蘇峻作逆,嶠將赴難,而征西將軍陶侃懷疑不從。嶠屢說不能回,更遣使順侃意曰:「仁公且守,僕宜先下。」遣信已二日,會寶別使還,聞之,說嶠曰:「凡舉大事,當與天下共同,眾克在和,不聞有異。假令可疑,猶當外示不覺,況自作疑耶!便宜急追信,改舊書,說必應
 俱徵。若不及前信,宜更遣使。」嶠意悟,即追信改書,侃果共徵峻。寶領千人為嶠前鋒,俱次茄子浦。



 初,嶠以南軍習水,峻軍便步,欲以所長制之,宜令三軍,有上岸者死。時蘇峻送米萬斛饋祖約、約遣司馬桓撫等迎之。寶告其眾曰:「兵法,軍令有所不從,豈可不上岸邪!」乃設變力戰,悉獲其米,虜殺萬計,約用大飢。嶠嘉其勳,上為廬江太守。



 約遣祖煥、桓撫等欲襲湓口,陶侃將自擊之,寶曰:「義軍恃公,公不可動,寶請討之。」侃顧謂坐客曰:「此年少言可用也。」乃使寶行。先是,桓宣背約,南屯馬頭山,為煥、撫所攻,求救於寶。寶眾以宣本是約黨,疑之。宣遣子戎
 重請,寶即隨戎赴之。未至,而賊已與宣戰。寶軍懸兵少,器杖濫惡,大為煥、撫所破。寶中箭,貫髀徹鞍,使人蹋鞍拔箭,血流滿靴,夜奔船所百餘里,望星而行。到,先哭戰亡將士,洗瘡訖,夜還救宣。寶至宣營,而煥、撫亦退。寶進攻祖約,軍次東關,破合肥,尋召歸石頭。陶侃、溫嶠未能破賊,侃欲率眾南還。寶謂嶠曰:「下官能留之。」乃往說侃曰:「公本應領蕪湖,為南北勢援,前既已下,勢不可還。且軍政有進無退,非直整齊三軍,示眾必死而已,亦謂退無所據,終至滅亡。往者杜弢非不強盛,公竟滅之,何至於峻獨不可破邪!賊亦畏死,非皆勇健,公可試與寶兵,
 使上岸斷賊資糧,出其不意,使賊困蹙。若寶不立效,然後公去,人心不恨。」侃然之,加寶督護。寶燒峻句容、湖孰積聚,峻頗乏食,侃遂留不去。



 峻既死,匡術以苑城降。侃使寶守南城,鄧嶽守西城。賊遣韓晃攻之,寶登城射殺數十人。晃問寶曰:「君是毛廬江邪?」寶曰:「是。」晃曰:「君名壯勇,何不出鬥!」寶曰:「君若健將,何不入斗!」晃笑而退。賊平,封州陵縣開國侯,千六百戶。



 庾亮西鎮,請為輔國將軍、江夏相、督隨義陽二郡,鎮上明。又進南中郎。隨亮討郭默。默平,與亮司馬王愆期救桓宣於章山,擊賊將石遇,破之,進征虜將軍。亮謀北伐,上疏解豫州,請以授寶。
 於是詔以寶監揚州之江西諸軍事、豫州刺史,將軍如故,與西陽太守樊峻以萬人守邾城。石季龍惡之,乃遣其子鑒與其將夔安、李菟等五萬人來寇,張狢渡二萬騎攻邾城。寶求救於亮,亮以城固,不時遣軍,城遂陷。寶、峻等率左右突圍出,赴江死者六千人,寶亦溺死。亮哭之慟,因發疾,遂薨。



 詔曰:「寶之傾敗,宜在貶裁。然蘇峻之難,致力王室。今咎其過,故不加贈,祭之可也。」其後公卿言寶有重勳,加死王事,不宜奪爵。升平三年,乃下詔復本封。



 初,寶在武昌,軍人有於市買得一白龜,長四五寸,養之漸大,放諸江中。邾城之敗,養龜人被鎧持刀,自投
 於水中,如覺墮一石上,視之,乃先所養白龜,長五六尺,送至東岸,遂得免焉。



 寶二子:穆之、安之。



 穆之字憲祖,小字武生,名犯王靖后諱,故行字,後又以桓溫母名憲,乃更稱小字。穆之果毅有父風,安西將軍庾翼以為參軍。襲爵州陵侯,翼等專威陜西,以子方之為建武將軍,守襄陽。方之年少,翼選武將可信杖者為輔弼,乃以穆之為建武司馬。俄而翼薨,大將乾瓚、戴羲等作亂,穆之與安西長史江[A170]、司馬朱燾等共平之。



 桓溫代翼,復取為參軍。從溫平蜀,以功賜次子都鄉侯。尋除揚威將軍、潁川太守,隨溫平洛,入關。溫將旋師,以謝
 尚未至,留穆之以二千人衛山陵。升平初,遷督寧州諸軍事、揚威將軍、寧州刺史。以桓溫封南郡,徙穆之為建安侯,復為溫太尉參軍。加冠軍將軍,以所募兵配之。溫伐慕容,使穆之監鑿鉅野百餘里,引汶會于濟川。及溫焚舟步歸,使穆之督東燕四郡軍事。領東燕太守,本官如故。袁真以壽陽叛,溫將征之。穆之以冠軍領淮南太守,守歷陽。真平,餘黨分散,乃以穆之督揚州之江西軍事,復領陳郡太守。俄而徙督揚州之義成荊州五郡雍州之兆軍事、襄陽義成河南三郡太守,將軍如故。尋進領梁州刺史。頃之,以疾解職,詔以冠軍征還。



 苻堅
 別將寇彭城,復以將軍假節、監江北軍事。鎮廣陵。遷右將軍、宣城內史、假節,鎮姑孰。穆之以為戍在近畿,無復軍警,不宜加節,上疏辭讓,許之。苻堅別將圍襄陽,詔穆之就上明受桓沖節度。沖使穆之游軍沔中。穆之始至,而朱序陷沒,引軍還郡。堅眾又寇蜀漢,梁州刺史楊亮、益州刺史周仲孫奔退,沖使穆之督梁州之三郡軍事、右將軍、西蠻校尉、益州刺史、領建平太守、假節,戍巴郡。以子球為梓潼太守。穆之與球伐堅,至于巴西郡,以糧運乏少,退屯巴東,病卒。追贈中軍將軍,謚曰烈。子珍嗣,位至天門太守。珍弟璩、球、璠、瑾、瑗,璩最知名。



 璩字叔璉。弱冠,右將軍桓豁以為參軍。尋遭父憂,服闕,為謝安衛將軍參軍,除尚書郎。安復請為參軍,轉安子琰征虜司馬。淮肥之役,苻堅迸走,璩與田次之共躡堅,至中陽,不及而歸。遷寧朔將軍、淮南太守。尋補鎮北將軍、譙王恬司馬。海陵縣界地名青蒲,四面湖澤,皆是菰葑,逃亡所聚,威令不能及。璩建議率千討人。時大旱,璩因放火,菰葑盡然,亡戶窘迫,悉出詣璩自首,近有萬戶,皆以補兵,朝廷嘉之。轉西中郎司馬、龍驤將軍、譙梁二郡內史。尋代郭銓為建威將軍、益州刺史。



 安帝初,進征虜將軍。及桓玄篡位,遣使加璩散騎常侍、左將軍。璩
 執留玄使,不受命。玄以桓希為梁州刺史,王異據涪,郭法戍宕渠,師寂戍巴郡,周道子戍白帝以防之。璩傳檄遠近,列玄罪狀,遣巴東太守柳約之、建平太守羅述、征虜司馬甄季之擊破希等,仍率眾次于白帝。武陵王令曰:「益州刺史毛璩忠誠愨亮,自桓玄萌禍,常思躡其後。今若平殄兇逆,肅清荊郢者,便當即授上流之任。」



 初,璩弟寧州刺史璠卒官,璩兄球孫祐之及參軍費恬以數百人送喪,葬江陵。會玄敗,謀奔梁州。璩弟瑾子修之時為玄屯騎校尉,誘玄使入蜀,既而脩之與祐之、費恬及漢嘉人馮遷共殺玄。約之等聞玄死,進軍到枝
 復攻沒江陵。劉毅等還尋陽,約之亦退。俄而季子、述皆病,約子詣振偽降,因欲襲振而桓振。事泄,被害。約之司馬時延祖、涪陵太守文處茂等撫其餘眾,保涪陵。振遣桓放之為益州,屯西陵。處茂距擊,破之。振死,安帝反正,詔曰:「夫貞松標於歲寒,忠臣亮於國危。益州刺史璩體識弘正,誠契義旗,受命偏師,次於近畿,匡翼之勛,實感朕心。可進征西將軍,加散騎常侍,都督益梁秦涼寧五州軍事,行宜都、寧蜀太守。文處茂宣贊蕃牧,蒙險夷難,可輔國將軍、西夷校尉、巴西梓潼二郡太守。」又詔西夷校尉瑾為持節、監梁秦二州軍事、征虜將軍、梁秦二州刺
 史、略陽武都太守。瑾弟蜀郡太守瑗為輔國將軍、寧州刺史。



 初,璩聞振陷江陵,率眾赴難,使瑾、瑗順外江而下,使參軍譙縱領巴西、梓潼二郡軍下涪水,當與璩軍會於巴郡。蜀人不樂東征,縱因人情思歸,於五城水口反,還襲涪,害瑾,瑾留府長史鄭純之自成都馳使告璩。璩時在略城,去成都四百里,遣參軍王瓊討反者,相距於廣漢。僰道令何林聚黨助縱,而璩下人受縱誘說,遂共害璩及瑗,並子姪之在蜀者,一時殄沒。璩子弘之嗣。



 義熙中,時延祖為始康太守,上疏訟璩兄弟,於是詔曰:「故益州刺史璩、西夷校尉瑾、蜀郡太守瑗勤王忠烈,事乖
 慮外。葬送日近,益懷惻愴,可皆贈先所授官,給錢三十萬、布三百匹。」論璩討桓玄功,追封歸鄉公,千五百戶。又以祐之斬玄功,封夷道縣侯。



 自寶至璩三葉,擁旄開國者四人,將帥之家,與尋陽周氏為輩,而人物不及也。



 瑾子脩之,頻歷清顯,至右衛將軍,從劉裕平姚泓。後為安西司馬,沒于魏。



 安之字仲祖,亦有武幹,累遷撫軍參軍、魏郡太守。簡文輔政,委以爪牙。及登阼,安之領兵從駕,使止宿宮中。尋拜游擊將軍。時庾希入京口,朝廷震動,命安之督城門諸軍事。孝武即位,妖賊盧悚突入殿廷。安之聞難,率眾
 直入雲龍門,手自奮擊。既而左衛將軍殷康、領軍將軍桓秘等至,與安之并力,悚因剿滅。遷右衛將軍。定后崩,領將作大匠。卒官。追贈光祿勳。



 四子:潭、泰、邃、遁。潭嗣爵,官至江夏相。泰歷太傅從事中郎、後軍諮議參軍,與邃俱為會稽王父子所暱,乃追論安之討盧悚勳,賜爵平都子,命潭襲爵。元顯嘗宴泰家,既而欲去,泰苦留之曰:「公若遂去,當取公腳。」元顯大怒,奮衣而出,遂與元顯有隙。及元顯敗,泰時為冠軍將軍、堂邑太山二郡太守。邃為游擊將軍,遁為太傅主簿,桓玄得志,使泰收元顯,遂於新亭,泰因宿恨,手加毆辱。俄並為玄所殺,惟遁被徙
 廣州。義熙初,得還,至宜都太守。



 德祖,璩宗人也。父祖並沒于賊中。德祖兄弟五人,相攜南渡,皆有武幹,荊州刺史劉道規以德祖為建武將軍、始平太守,又徙涪陵太守。盧循之役,道規又以為參軍,伐徐道覆於始興。尋遭母憂。劉裕伐司馬休之,版補太尉參軍、義陽太守,賜爵遷陵縣侯,轉南陽太守,從劉裕伐姚泓,頻攻滎陽、扶風、南安、馮詡數郡,所在剋捷。裕嘉之,以為龍驤將軍、秦州刺史。裕留第二子義真為安西將軍、雍州刺史。以德祖為中兵參軍,領天水太守,從義真還。裕以德祖督河東平陽二郡軍事、輔國將軍、河東太
 守,代劉遵考守蒲阪。及河北覆敗,德祖全軍而歸。裕方欲蕩平關洛,先以德祖督九郡軍事、冠軍將軍、滎陽京兆太守,以前後功,賜爵灌陽縣男,尋遷督司雍並三州諸軍事、冠軍將軍、司州刺史,戍武牢,為魏所沒。



 德祖次弟嶷,嶷弟辯,並有志節。嶷死於盧循之難,辯沒於魯宗之役,並奮不顧命,為世所歎。



 劉遐,字正長,廣平易陽人也。性果毅,便弓馬,開豁勇壯。值天下大亂,遐為塢主,每擊賊,率壯士陷堅摧鋒,冀方比之張飛、關羽。鄉人冀州刺史邵續深器之,以女妻焉,
 遂壁于河濟之間,賊不敢逼。遐間道遣使受元帝節度,朝廷嘉之,璽書慰勉,以為龍驤將軍、平原內史。建武初,元帝令曰:「遐忠勇果毅,義誠可嘉。以遐為下邳內史,將軍如故。」



 初,沛人周堅,一名撫,與同郡周默因天下亂,各為塢主,以寇抄為事。默降祖逖,撫怒,遂襲殺默,以彭城叛,石勒遣騎援之。詔遐領彭城內史,與徐州刺史蔡豹、太山太守徐龕共討撫,戰於寒山,撫敗走。詔徙遐為臨淮太守。徐龕復反,事平,以遐為北中郎將、兗州刺史。



 太寧初,自彭城移屯泗口。王含反,遐與蘇峻俱赴京都。含敗,隨丹陽尹溫嶠追含至于淮南,遐頗放兵虜掠。嶠曰:「
 天道助順,故王含剿絕,不可因亂為亂也。」遐深自陳而拜謝。事平,以功封泉陵公,遷散騎常侍、監淮北軍事、北中郎將、徐州刺史、假節,代王邃鎮淮陰。咸和元年卒,追贈安北將軍。



 子肇年幼,成帝以徐州授郗鑒,以郭默為北中郎將,領遐部曲。遐妹夫田防及遐故將史迭、卞咸、李龍等不樂他屬,共立肇,襲遐故位以叛。成帝遣郭默等率諸郡討之。默等始上道,而臨淮太守劉矯率將士數百掩襲遐營,迭等迸走,斬田防及督護卞咸等,追斬迭、龍於下邳,傳首詣闕。遐母妻子參佐將士悉還建康。



 遐妻驍果有父風。遐嘗為石季龍所圍,妻單將數騎,拔遐出
 於萬眾之中。及田防等欲為亂,遐妻止之,不從,乃密起火燒甲杖都盡。



 肇襲爵,官至散騎侍郎。肇卒,子舉嗣。卒,子遵之嗣。卒,子伯齡嗣。宋受禪,國除。



 鄧岳,字伯山,陳郡人也。本名岳,以犯康帝諱,改為嶽,後竟改名為岱焉。少有將帥才略,為王敦參軍。轉從事中郎、西陽太守。王含構產逆,嶽領兵隨含向京都。及含敗,嶽與周撫俱奔蠻王向蠶。後遇赦,與撫俱出。久之,司徒王導命為從事中郎,後復為西陽太守。



 及蘇峻反,平南將軍溫嶠遣嶽與督護王愆期、鄱陽太守紀睦等率舟軍
 赴難。峻平,還郡。郭默之殺劉胤也,大司馬陶侃使嶽率西陽之眾討之。默平,遷督交廣二州軍事、建武將軍、領平越中郎將、廣州刺史、假節,錄前後勳,封宜城縣伯。咸康三年,嶽遣軍伐夜郎,破之,加督寧州,進征虜將軍,遷平南將軍。卒,子遐嗣。



 遐字應遠。勇力絕人,氣蓋當時,時人方之樊噲。桓溫以為參軍,數從溫征伐,歷冠軍將軍,數郡太守,號為名將。襄陽城北沔水中有蛟,常為人害,遐遂拔劍入水,蛟繞其足,遐揮劍截蛟數段而出。枋頭之役,溫既懷恥忿,且忌憚遐之勇果,因免遐官,尋卒。寧康中,追贈廬陵太守。



 嶽弟逸,字茂山,亦有武幹。嶽卒後,以逸監交廣州、建威將軍、平越中郎將、廣州刺史、假節。



 朱序,字次倫,義陽人也。父燾,以才幹歷西蠻校尉、益州刺史。序世為名將,累遷鷹揚將軍、江夏相。興寧末,梁州刺史司馬勳反,桓溫表序為征討都護往討之,以功拜征虜將軍,封襄平子。太和中,遷兗州刺史。時長城人錢弘聚黨百餘人,藏匿原鄉山。以序為中軍司馬、吳興太守。序至郡,討擒之。事訖,還兗州。



 寧康初,拜使持節、監沔中諸軍事、南中郎將、梁州刺史,鎮襄陽。是歲,苻堅遣其
 將苻不等率眾圍序,序固守,賊糧將盡,率眾苦攻之。初,苻丕之來攻也,序母韓自登城履行,謖西北角當先受弊,遂領百餘婢并城中女子於其角斜築城二十餘丈。賊攻西北角,果潰,眾便固新築城。丕遂引退。襄陽人謂此城為夫人城。序累戰破賊,人情勞懈,又以賊退稍遠,疑未能來,守備不謹,督護李伯護密與賊相應,襄陽遂沒,序陷於苻堅。堅殺伯護徇之,以其不忠也。序欲逃歸,潛至宜陽,藏夏揆家。堅疑揆,收之,序乃詣苻暉自首,堅嘉而不問,以為尚書。



 太元中,苻堅南侵,謝石率眾距之。時堅大兵尚在項,苻融以三十萬眾先至。堅遣序說謝
 石,稱己兵威。序反謂石曰:「若堅百萬之眾悉到,莫可與敵,及其未會,擊之,可以得志。」於是石遣謝琰選勇士八千人涉肥水挑戰。堅眾小卻,序時在其軍後,唱云:「堅敗!」眾遂大奔,序乃得歸。拜龍驤將軍、瑯邪內史,轉揚州豫州五郡軍事、豫州刺史,屯洛陽。



 後丁零翟遼反,序遣將軍秦膺、童斌與淮泗諸郡共討之。又監兗青二州諸軍事、二州刺史,將軍如故,進鎮彭城。序求鎮淮陰,帝許焉。翟遼又使其子釗寇陳潁,序還遣秦膺討釗,走之,拜征虜將軍。表求運江州米十萬斛,布五千匹以資軍費,詔聽之。加都督司、雍、梁、秦四州軍事。帝遣廣威將軍、河南
 太守楊佺期,南陽太守趙睦,各領兵千人隸序。序又表求故荊州刺史桓石生府田百頃,并穀八萬斛,給之。仍戍洛陽,衛山陵也。



 其後慕容永率眾向洛陽,序自河陰北濟,與永偽將王次等相遇,乃戰於沁水,次改走,斬其支將勿支首。參軍趙睦、江夏相桓不才追永,破之于太行。永歸上黨。時楊楷聚眾數千,在湖陜,聞永敗,遣任子詣序乞降。序追永至上黨之白水,與永相持二旬。聞翟遼欲向金墉,乃還,遂攻翟釗於石門,遣參軍趙蕃破翟遼於懷縣,遼宵遁。序退次洛陽,留鷹揚將軍朱黨戍石門。序仍使子略督護洛城,趙蕃為助。序還襄陽。會稽王
 道子以序勝負相補,不加褒貶。



 其後東羌校尉竇衝欲入漢川,安定人皇甫釗、京兆人周勳等謀納之。梁州刺史周瓊失巴西三郡,眾寡力弱,告急於序,序遣將軍皇甫貞率眾赴之。衝據長安東,釗、勳散走。



 序以老病,累表解職,不許。詔斷表,遂輒去任。數旬,歸罪廷尉,詔原不問。太元十八年卒,贈左將軍、散騎常侍。



 史臣曰:晉氏淪喪,播遷江表,內難薦臻,外虞不息,經略之道,是所未弘,將帥之功,無聞焉爾。遜、豹、宣、胤服勤於太興之間,毛、鄧、劉、朱馳騖乎咸和之後。雖人不逮古,亦足列於當世焉。



 贊曰:氣分淮海,災流瀍澗。覆類玄蚖,興微《鴻雁》。鼓鞞在聽,《兔罝》有作。赳赳群英,勤茲王略。



\end{pinyinscope}