\article{列傳第二十九 汝南文成王亮子粹矩子祐羕宗熙楚隱王瑋趙王倫王冏鄭方長沙王乂成都王穎河間王顒東海孝獻王越}

\begin{pinyinscope}
汝南文成王亮
 \gezhu{
  子粹矩子祐羕宗熙}
 楚隱王瑋趙王倫王冏
 \gezhu{
  鄭方}
 長沙王乂成都王穎河間王顒
 東海孝獻王越



 自
 古帝王之臨天下也,皆欲廣樹蕃屏,崇固維城。唐、虞以前,憲章蓋闕,夏、殷以後,遺迹可知。然而玉帛會於塗山,雖云萬國,至於分疆胙土,猶或未詳。泊乎周室,粲焉可觀,封建親賢,並為列國。當其興也,周、召贊其升平;及其衰也,桓、文輔其危亂。故得卜世之祚克昌,卜年之基惟永。逮王赧即世,天祿已終,虛位無主,三十餘載。爰及暴秦,并吞天下,戒衰周之削弱,忽帝業之遠圖,謂王室
 之陵遲,由諸候之彊大。於是罷侯置守,獨尊諸己,至乎子弟,並為匹夫,惟欲肆虐陵威,莫顧謀孫翼子。枝葉微弱,宗祐孤危,內無社稷之臣,外闕籓維之助。陳、項一呼,海內沸騰,隕身於望夷,繫頸於軹道。事不師古,二世而滅。漢祖勃興,爰革斯弊。於是分王子弟,列建功臣,錫之山川,誓以帶礪。然而矯枉過直,懲羹吹齏,土地封疆,踰越往古。始則韓、彭菹醢,次乃吳、楚稱亂。然雖克滅權偪,猶足維翰王畿。洎成、哀之後,戚籓陵替,君臣乘茲間隙,竊位偷安。光武雄略緯天,慷慨下國,遂能除兇靜亂,復禹配天,休祉盛於兩京,鼎祚隆於四百,宗支繼絕之力,
 可得而言。魏武忘經國之宏規,行忌刻之小數,功臣無立錐之地,子弟君不使之人,徒分茅社,實傳虛爵,本根無所庇蔭,遂乃三葉而亡。



 有晉思改覆車,復隆盤石,或出擁旄節,蒞嶽牧之榮;入踐台階,居端揆之重。然而付托失所,授任乖方,政令不恒,賞罰斯濫。或有材而不任,或無罪而見誅,朝為伊、周,夕為莽、卓。機權失於上,禍亂作於下。楚、趙諸王,相仍構釁,徒興晉陽之甲,竟匪勤王之師。始則為身擇利,利未加而害及;初乃無心憂國,國非憂而奚拯!遂使昭陽興廢,有甚弈棋;乘輿幽縶,更同羑里。胡羯陵侮,宗廟丘墟,良可悲也。



 夫為國之有籓屏,
 猶濟川之有舟楫,安危成敗,義實相資。舟楫且完,波濤不足稱其險;籓屏式固,禍亂何以成其階!向使八王之中,一籓繄賴,如梁王之禦大故,若朱虛之除大憝,則外寇焉敢憑陵,內難奚由竊發!縱令天子暗劣,鼎臣奢放,雖或顛沛,未至土崩。何以言之?瑯邪譬彼諸王,權輕眾寡,度長絜大,不可同年。遂能匹馬濟江,奄有吳會,存重宗社,百有餘年。雖曰天時,抑亦人事。豈如趙倫、齊冏之輩,河間、東海之徒,家國俱亡,身名並滅。善惡之數,此非其效歟!西晉之政亂朝危,雖由時主,然而煽其風,速其禍者,咎在八王,故序而論之,總為其傳云耳。



 汝南文成王亮,字子翼,宣帝第四子也。少清警有才用,仕魏為散騎侍郎、萬歲亭侯,拜東中郎將,進封廣陽鄉侯。討諸葛誕於壽春,失利,免官。頃之,拜左將軍,加散騎常侍、假節,出監豫州諸軍事。五等建,改封祁陽伯,轉鎮西將軍。武帝踐阼,封扶風郡王,邑萬戶,置騎司馬,增參軍掾屬,持節、都督關中雍、涼諸軍事。會秦州刺史胡烈為羌虜所害,亮遣將軍劉旂、騎督敬琰赴救,不進,坐是貶為平西將軍。旂當斬,亮與軍司曹冏上言,節度之咎由亮而出,乞丐旂死。詔曰:「高平困急,計城中及旂足以
 相拔,就不能徑至,尚當深進。今奔突有投,而坐視覆敗,故加旂大戮。今若罪不在旂,當有所在。」有司又奏免亮官,削爵土。詔惟免官。頃之,拜撫軍將軍。是歲,吳將步闡來降,假亮節都督諸軍事以納之。尋加侍中之服。



 咸寧初,以扶風池陽四千一百戶為太妃伏氏湯沐邑,置家令丞僕,後改食南郡枝江。太妃嘗有小疾,祓於洛水,亮兄弟三人侍從,並持節鼓吹,震耀洛濱。武帝登陵雲臺望見,曰:「伏妃可謂富貴矣。」其年進號衛將軍,加侍中。時宗室殷盛,無相統攝,乃以亮為宗師,本官如故,使訓導觀察,有不遵禮法,小者正以義方,大者隨事聞奏。



 三年,
 徙封汝南,出為鎮南大將軍、都督豫州軍事,開府、假節,之國,給追鋒車、皁輪犢車,錢五十萬。頃之,徵亮為侍中、撫軍大將軍,領後軍將軍,統冠軍、步兵、射聲、長水等營,給兵五百人,騎百匹。遷太尉、錄尚書事、領太子太傅,侍中如故。



 及武帝寢疾,為楊駿所排,乃以亮為侍中、大司馬、假黃鉞、大都督、督豫州諸軍事,出鎮許昌,加軒懸之樂,六佾之舞。封子羕為西陽公。未發,帝大漸,詔留亮委以後事。楊駿聞之,從中書監華暠索詔視,遂不還。帝崩,亮懼駿疑己,辭疾不入,於大司馬門外敘哀而已,表求過葬。駿欲討亮,亮知之,問計於廷尉何勖。勖曰:「今朝
 廷皆歸心於公,公何不討人而懼為人所討!」或說亮率所領入廢駿,亮不能用,夜馳赴許昌,故得免。及駿誅,詔曰:「大司馬、汝南王亮體道沖粹,通識政理,宣翼之績,顯於本朝,《二南》之風,流于方夏,將憑遠猷,以康王化。其以亮為太宰、錄尚書事,入朝不趨,劍履上殿,增掾屬十人,給千兵百騎,與太保衛瓘對掌朝政。」亮論賞誅楊駿之功過差,欲以茍悅眾心,由是失望。



 楚王瑋有勳而好立威,亮憚之,欲奪其兵權。瑋甚憾,乃承賈后旨,誣亮與瓘有廢立之謀,矯詔遣其長史公孫宏與積弩將軍李肇夜以兵圍之。帳下督李龍白外有變,請距之,亮不聽。俄
 然楚兵登牆而呼,亮驚曰:「吾無二心,何至於是!若有詔書,其可見乎?」宏等不許,促兵攻之。長史劉準謂亮曰:「觀此必是姦謀,府中俊乂如林,猶可盡力距戰。」又弗聽,遂為肇所執,而嘆曰:「我之忠心,可破示天下也,如何無道,枉殺不辜!」是時大熱,兵人坐亮於車下,時人憐之,為之交扇。將及日中,無敢害者。瑋出令曰:「能斬亮者,賞布千匹。」遂為亂兵所害,投于北門之壁,鬢髮耳鼻皆悉毀焉。及瑋誅,追復亮爵位,給東園溫明祕器,朝服一襲,錢三百萬,布絹三百匹,喪葬之禮如安平獻王孚故事,廟設軒懸之樂。有五子:粹、矩、羕、宗、熙。



 粹字茂弘。早卒。



 矩字延
 明。拜世子,為屯騎校尉,與父亮同被害。追贈典軍將軍,謚懷王。子祐立,是為威王。



 祐字永猷。永安中,從惠帝北征。帝遷長安,祐反國。及帝還洛,以征南兵八百人給之,特置四部牙門。永興初,率眾依東海王越,討劉喬有功,拜揚武將軍,以江夏雲杜益封,并前二萬五千戶。越征汲桑,表留祐領兵三千守許昌,加鼓吹、麾旗。越還,祐歸國。永嘉末,以寇賊充斥,遂南渡江,元帝命為軍諮祭酒。建武初,為鎮軍將軍。太興末,領左軍將軍,太寧中,進號衛將軍,加散騎常侍。咸和元年,薨,贈侍中、特進。



 子恭王統立,以南頓王宗謀反,被廢。其後成帝哀亮一門殄絕,
 詔統復封,累遷祕書監、侍中。薨,追贈光祿勳。子義立,官至散騎常侍。薨,子遵之立。義熙初,梁州刺史劉稚謀反,推遵之為主,事泄,伏誅。弟楷之子蓮扶立。宋受禪,國除。



 羕字延年。太康末,封西陽縣公,拜散騎常侍。亮之被害也,羕時年八歲,鎮南將軍裴楷與之親姻,竊之以逃,一夜八遷,故得免。及瑋誅,進爵為王,歷步兵校尉、左軍驍騎將軍。元康初,進封郡王。永興初,拜侍中。以長沙王乂黨,廢為庶人。惠帝還洛,復羕封,為撫軍將軍,又以汝南期思、西陵益其國。永嘉初,拜鎮軍將軍,加散騎常侍,領後軍將軍,復以邾、蘄春益之,并前三萬五千戶。隨東海
 王越東出鄄城,遂南渡江。



 元帝承制,更拜撫軍大將軍、開府,給千兵百騎,詔與南頓王宗統流人以實中州,江西荒梗,復還。及元帝踐阼,進位侍中、太保。以羕屬尊,元會特為設床。太興初,錄尚書事,尋領大宗師,加羽葆、斧鉞,班劍六十人,進位太宰。及王敦平,領太尉。明帝即位,以羕宗室元老,特為之拜。羕放縱兵士劫鈔,所司奏免羕官,詔不問。及帝寢疾,羕與王導同受顧命輔成帝。時帝幼沖,詔羕依安平獻王孚故事,設床帳於殿上,帝親迎拜。咸和初,坐弟南頓王宗免官,降為弋陽縣王。及蘇峻作亂,羕詣峻稱述其勳,峻大悅,矯詔復羕爵位。峻平,
 賜死。世子播、播弟充及息崧並伏誅,國除。咸康初,復其屬籍,以羕孫氏為奉車都尉、奉朝請。



 宗字延祚。元康中,封南頓縣侯,尋進爵為公。討劉喬有功,進封王,增邑五千,并前萬戶,為征虜將軍。與兄羕俱過江。元帝承制,拜散騎常侍。愍帝之在西都,以宗為平東將軍。元帝即位,拜撫軍將軍,領左將軍。明帝踐阼,加長水校尉,轉左衛將軍。與虞胤俱為帝所暱,委以禁旅。



 宗與王導、庾亮志趣不同,連結輕俠,以為腹心,導、亮並以為言。帝以宗戚屬,每容之。及帝疾篤,宗、胤密謀為亂,亮排闥入,升御床,流涕言之,帝始悟。轉為驃騎將軍。胤
 為大宗正。宗遂怨望形於辭色。咸和初,御史中丞鐘雅劾宗謀反,庾亮使右衛將軍趙胤收之。宗以兵距戰,為胤所殺,貶其族為馬氏,徙妻子于晉安,既而原之。三子:綽、超、演,廢為庶人。咸康中,復其屬籍。綽為奉車都尉、奉朝請。



 熙初封汝陽公,討劉喬有功,進爵為王。永嘉末,沒於石勒。



 楚隱王瑋,字彥度,武帝第五子也。初封始平王,歷屯騎校尉。太康末,徙封於楚,出之國,都督荊州諸軍事、平南
 將軍,轉鎮南將軍。武帝崩,入為衛將軍,領北軍中候,加侍中、行太子少傅。



 楊駿之誅也,瑋屯司馬門。瑋少年果銳,多立威刑,朝廷忌之。汝南王亮、太保衛瓘以瑋性很戾,不可大任,建議使與諸王之國,瑋甚忿之。長史公孫宏、舍人岐盛並薄於行,為瑋所暱。瓘等惡其為人,慮致禍亂,將收盛。盛知之,遂與宏謀,因積弩將軍李肇矯稱瑋命,譖亮、瓘於賈后。而后不之察,使惠帝為詔曰:「太宰、太保欲為伊、霍之事,王宜宣詔,令淮南、長沙、成都王屯宮諸門,廢二公。」夜使黃門齎以授瑋。瑋欲覆奏,黃門曰:「事恐漏泄,非密詔本意也。」瑋乃止。遂勒本軍,復矯詔召
 三十六軍,手令告諸軍曰:「天禍晉室,凶亂相仍。間者楊駿之難,實賴諸君剋平禍亂。而二公潛圖不軌,欲廢陛下以絕武帝之祀。今輒奉詔,免二公官。吾今受詔都督中外諸軍。諸在直衛者皆嚴加警備,其在外營,便相率領,徑詣行府。助順討逆,天所福也。懸賞開封,以待忠效。皇天后土,實聞此言。」又矯詔使亮、瓘上太宰太保印綬、侍中貂蟬,之國,官屬皆罷遣之。又矯詔赦亮、瓘官屬曰:「二公潛謀,欲危社稷,今免還第。官屬以下,一無所問。若不奉詔,便軍法從事。能率所領先出降者,封侯受賞。朕不食言。」遂收亮、瓘,殺之。



 岐盛說瑋,可因兵勢誅賈模、郭
 彰,匡正王室,以安天下。瑋猶豫未決。會天明,帝用張華計,遣殿中將軍王宮齎騶虞幡麾眾曰:「楚王矯詔。」眾皆釋杖而走。瑋左右無復一人,窘迫不知所為,惟一奴年十四,駕牛車將赴秦王柬。帝遣謁者詔瑋還營,執之於武賁署,遂下廷尉。詔以瑋矯制害二公父子,又欲誅滅朝臣,謀圖不軌,遂斬之,時年二十一。其日大風,雷雨霹靂。詔曰:「周公決二叔之誅,漢武斷昭平之獄,所不得已者。廷尉奏瑋已伏法,情用悲痛,吾當發哀。」瑋臨死,出其懷中青紙詔,流涕以示監刑尚書劉頌曰:「受詔而行,謂為社稷,今更為罪,託體先帝,受枉如此,幸見申列。」頌亦
 歔欷不能仰視。公孫宏、岐盛並夷三族。



 瑋性開濟好施,能得眾心,及此莫不隕淚,百姓為之立祠。賈后先惡瓘、亮,又忌瑋,故以計相次誅之。永寧元年,追贈驃騎將軍,封其子範為襄陽王,拜散騎常侍,後為石勒所害。



 趙王倫,字子彞,宣帝第九子也,母曰柏夫人。魏嘉平初,封安樂亭侯。五等建,改封東安子,拜諫議大夫。武帝受禪,封瑯邪郡王。坐使散騎將劉緝買工所將盜御裘,廷尉杜友正緝棄市,倫當與緝同罪。有司奏倫爵重屬親,不可坐。諫議大夫劉毅駁曰:「王法賞罰,不阿貴賤,然後
 可以齊禮制而明典刑也。倫知裘非常,蔽不語吏,與緝同罪。當以親貴議減,不得闕而不論。宜自於一時法中,如友所正。」帝是毅駁,然以倫親親故,下詔赦之。及之國,行東中郎將、宣威將軍。咸寧中,改封於趙,遷平北將軍、督鄴城守事,進安北將軍。元康初,遷征西將軍、開府儀同三司,鎮關中。倫刑賞失中,氐、羌反叛,徵還京師。尋拜車騎將軍、太子太傅。深交賈、郭,諂事中宮,大為賈后所親信。求錄尚書,張華、裴頠固執不可。又求尚書令,華、頠復不許。



 愍懷太子廢,使倫領右軍將軍。時左衛司馬督司馬雅及常從督許超,並嘗給事東宮,二人傷太子無
 罪,與殿中中郎士猗等謀廢賈后,復太子,以華、頠不可移,難與圖權,倫執兵之要,性貪冒,可假以濟事,乃說倫嬖人孫秀曰:「中宮凶妒無道,與賈謐等共廢太子。今國無嫡嗣,社稷將危,大臣將起大事。而公名奉事中宮,與賈、郭親善,太子之廢,皆云豫知,一朝事起,禍必相及。何不先謀之乎?」秀許諾,言於倫,倫納焉。遂告通事令史張林及省事張衡、殿中侍御史殷渾、右衛司馬督路始,使為內應。事將起,而秀知太子聰明,若還東宮,將與賢人圖政,量己必不得志,乃更說倫曰:「太子為人剛猛,不可私請。明公素事賈后,時議皆以公為賈氏之黨。今雖欲
 建大功於太子,太子含宿怒,必不加賞於明公矣。當謂逼百姓之望,翻覆以免罪耳。此乃所以速禍也。今且緩其事,賈后必害太子,然後廢后,為太子報仇,亦足以立功,豈徒免禍而已。」倫從之。秀乃微泄其謀,使謐黨頗聞之。倫、秀因勸謐等早害太子,以絕眾望。



 太子既遇害,倫、秀之謀益甚,而超、雅懼後難,欲悔其謀,乃辭疾。秀復告右衛佽飛督閭和,和從之,期四月三日丙夜一籌,以鼓聲為應。至期,乃矯詔敕三部司馬曰:「中宮與賈謐等殺吾太子,今使車騎入廢中宮。汝等皆當從命,賜爵關中侯。不從,誅三族。」於是眾皆從之。倫又矯詔開門夜入,陳
 兵道南,遣翊軍校尉、齊王冏將三部司馬百人,排閣而入。華林令駱休為內應,迎帝幸東堂。遂廢賈后為庶人,幽之于建始殿。收吳太妃、趙粲及韓壽妻賈午等,付暴室考竟。詔尚書以廢后事,仍收捕賈謐等,召中書監、侍中、黃門侍郎、八坐,皆夜入殿,執張華、裴頠、解結、杜斌等,於殿前殺之。尚書始疑詔有詐,郎師景露版奏請手詔。倫等以為沮眾,斬之以徇。明日,倫坐端門,屯兵北向,遣尚書和郁持節送賈庶人于金墉。誅趙粲叔父中護軍趙浚及散騎侍郎韓豫等,內外群官多所黜免。倫尋矯詔自為使持節、大都督、督中外諸軍事、相國,侍中、王如
 故,一依宣、文輔魏故事,置左右長史、司馬、從事中郎四人、參軍十人,掾屬二十人、兵萬人。以其世子散騎常侍荂領冗從僕射;子馥前將軍,封濟陽王;虔黃門郎,封汝陰王;羽散騎侍郎,封霸城侯。孫秀等封皆大郡,並據兵權,文武官封侯者數千人,百官總己聽於倫。



 倫素庸下,無智策,復受制於秀,秀之威權振於朝廷,天下皆事秀而無求於倫。秀起自瑯邪小史,累官於趙國,以諂媚自達。既執機衡,遂恣其姦謀,多殺忠良,以逞私欲。司隸從事游顥與殷渾有隙,渾誘顥奴晉興,偽告顥有異志。秀不詳察,即收顥及襄陽中正李邁,殺之,厚待晉興,以為
 己部曲督。前衛尉石崇、黃門郎潘岳皆與秀有嫌,並見誅。於是京邑君子不樂其生矣。



 淮南王允、齊王冏以倫、秀驕僭,內懷不平。秀等亦深忌焉,乃出冏鎮許,奪允護軍。允發憤,起兵討倫。允既敗滅,倫加九錫,增封五萬戶。倫偽為飾讓,詔遣百官詣府敦勸,侍中宣詔,然後受之。加荂撫軍將軍、領軍將軍,馥鎮軍將軍、領護軍將軍,虔中軍將軍、領右衛將軍,詡為侍中。又以孫秀為侍中、輔國將軍、相國司馬,右率如故。張林等並居顯要。增相府兵為二萬人,與宿衛同,又隱匿兵士,眾過三萬。起東宮三門四角華櫓,斷宮東西道為外徼。或謂秀曰:「散騎常
 侍楊準、黃門侍郎劉逵欲奉梁王肜以誅倫。」會有星變,乃徙肜為丞相,居司徒府,轉準、逵為外官。



 倫無學,不知書;秀亦以狡黠小才,貪淫昧利。所共立事者,皆邪佞之徒,惟競榮利,無深謀遠略。荂淺薄鄙陋,馥、虔暗很彊戾,詡愚嚚輕訬,而各乖異,互相憎毀。秀子會,年二十,為射聲校尉,尚帝女河東公主。公主母喪未期,便納聘禮。會形貌短陋,奴僕之下者,初與富室兒於城西販馬,百姓忽聞其尚主,莫不駭愕。



 倫、秀並惑巫鬼,聽妖邪之說。秀使牙門趙奉詐為宣帝神語,命倫早入西宮。又言宣帝於北芒為趙王佐助,於是別立宣帝廟於芒山。謂逆謀
 可成。以太子詹事裴劭、左軍將軍卞粹等二十人為從事中郎,掾屬又二十人。秀等部分諸軍,分布腹心,使散騎常侍、義陽王威兼侍中,出納詔命,矯作禪讓之詔,使使持節、尚書令滿奮,僕射崔隨為副,奉皇帝璽綬以禪位于倫。倫偽讓不受。於是宗室諸王、群公卿士咸假稱符瑞天文以勸進,倫乃許之。左衛王輿與前軍司馬雅等率甲士入殿,譬喻三部司馬,示以威賞,皆莫敢違。其夜,使張林等屯守諸門。義陽王威及駱休等逼奪天子璽綬。夜漏未盡,內外百官以乘輿法駕迎倫。惠帝乘雲母車,鹵簿數百人,自華林西門出居金墉城。尚書和郁,
 兼侍中、散騎常侍、瑯邪王睿,中書侍郎陸機從,到城下而反。使張衡衛帝,實幽之也。



 倫從兵五千人,入自端門,登太極殿,滿奮、崔隨、樂廣進璽綬於倫,乃僭即帝位,大赦,改元建始。是歲,賢良方正、直言、秀才、孝廉、良將皆不試;計吏及四方使命之在京邑者,太學生年十六以上及在學二十年,皆署吏;郡縣二千石令長赦日在職者,皆封侯;郡綱紀並為孝廉,縣綱紀為廉史。以世子荂為太子,馥為侍中、大司農、領護軍、京兆王,虔為侍中、大將軍領軍、廣平王,詡為侍中、撫軍將軍、霸城王,孫秀為侍中、中書監、驃騎將軍、儀同三司,張林等諸黨皆登卿將,
 並列大封。其餘同謀者咸超階越次,不可勝紀,至於奴卒斯役亦加以爵位。每朝會,貂蟬盈坐,時人為之顏曰:「貂不足,狗尾續。」而以茍且之惠取悅人情,府庫之儲不充於賜,金銀冶鑄不給於印,故有白版之侯,君子恥服其章,百姓亦知其不終矣。



 倫親祠太廟,還,遇大風,飄折麾蓋。孫秀既立非常之事,倫敬重焉。秀住文帝為相國時所居內府,事無巨細,必諮而後行。倫之詔令,秀輒改革,有所與奪,自書青紙為詔,或朝行夕改者數四,百官轉易如流矣。時有雉入殿中,自太極東階上殿,驅之,更飛西鐘下,有頃,飛去。又倫於殿上得異鳥,問皆不知名,
 累日向夕,宮西有素衣小兒言是服劉鳥。倫使錄小兒并鳥閉置牢室,明旦開視,戶如故,並失人鳥所在。倫目上有瘤,時以為妖焉。



 時齊王冏、河間王顒、成都王穎並擁彊兵,各據一方。秀知冏等必有異圖,乃選親黨及倫故吏為三王參佐及郡守。



 秀本與張林有隙,雖外相推崇,內實忌之。及林為衛將軍,深怨不得開府,潛與荂箋,具說秀專權,動違眾心,而功臣皆小人,撓亂朝廷,要一時誅之。荂以書白倫,倫以示秀。秀勸倫誅林,倫從之。於是倫請宗室會於華林園,召林、秀及王輿入,因收林,殺之,誅三族。



 及三王起兵討倫檄至,倫、秀始大懼,遣其中
 堅孫輔為上軍將軍,積弩李嚴為折衝將軍,率兵七千自延壽關出,征虜張泓、左軍蔡璜、前軍閭和等率九千人自堮阪關出,鎮軍司馬雅、揚威莫原等率八千人自成皋關出。召東平王楙為使持節、衛將軍,都督諸軍以距義師。使楊珍晝夜詣宣帝別廟祈請,輒言宣帝謝陛下,某日當破賊。拜道士胡沃為太平將軍,以招福佑。秀家日為淫祀,作厭勝之文,使巫祝選擇戰日。又令近親於嵩山著羽衣,詐稱仙人王喬,作神仙書,述倫祚長久以惑眾。秀欲遣馥、虔領兵助諸軍戰,馥、虔不肯。虔素親愛劉輿,秀乃使輿說虔,虔然後率眾八千為三軍繼援。
 而泓、雅等連戰雖勝,義軍散而輒合,雅等不得前。許超等與成都王穎軍戰于黃橋,殺傷萬餘人。泓徑造陽翟,又於城南破齊王冏輜重,殺數千人,遂據城保邸閣。而冏軍已在潁陰,去陽翟四十里。冏分軍渡潁,攻泓等不利。泓乘勝至於潁上,夜臨潁而陣。冏縱輕兵擊之,諸軍不動,而孫輔、徐建軍夜亂,徑歸洛自首。輔、建之走也,不知諸軍督尚存,乃云:「齊王兵盛,不可當,泓等已沒。」倫大震,祕之,而召虔及超還。會泓敗冏露布至,倫大喜,及復遣超,而虔還已至庾倉。超還濟河,將士疑阻,銳氣內挫。泓等悉其諸軍濟潁,進攻冏營,冏出兵擊其別率孫髦、
 司馬譚、孫輔,皆破之,士卒散歸洛陽,泓等收眾還營。秀等知三方日急,詐傳破冏營,執得冏,以誑惑其眾,令百官皆賀,而士猗、伏胤、孫會皆杖節各不相從。倫復授太子詹事劉琨節,督河北將軍,率步騎千人催諸軍戰。會等與義軍戰于激水,大敗,退保河上,劉琨燒斷河橋。



 自義兵之起,百官將士咸欲誅倫、秀以謝天下。秀知眾怒難犯,不敢出省。及聞河北軍悉敗,憂懣不知所為。義陽王威勸秀至尚書省與八坐議征戰之備,秀從之。使京城四品以下子弟年十五以上,皆詣司隸,從倫出戰。內外諸軍悉欲劫殺秀,威懼,自崇禮闥走還下舍。許超、士
 猗、孫會等軍既並還,乃與秀謀,或欲收餘卒出戰,或欲焚燒宮室,誅殺不附己者,挾倫南就孫旂、孟觀等,或欲乘船東走入海,許未決。王輿反之,率營兵七百餘人自南掖門入,敕宮中兵各守衛諸門,三部司馬為應於內。輿自往攻秀,秀閉中書南門。輿放兵登墻燒屋,秀及超、猗遽走出,左衛將軍趙泉斬秀等以徇。收孫奇於右衛營,付廷尉誅之。執前將軍謝惔、黃門令駱休、司馬督王潛,皆於殿中斬之。三部司馬兵於宣化闥中斬孫弼以徇,時司馬馥在秀坐,輿使將士囚之于散騎省,以大戟守省閣。八坐皆入殿中,坐東除樹下。王輿屯雲龍門,使
 倫為詔曰:「吾為孫秀等所誤,以怒三王。今已誅秀,其迎太上復位,吾歸老于農畝。」傳詔以騶虞幡敕將士解兵。文武官皆奔走,莫敢有居者。黃門將倫自華林東門出,及荂皆還汶陽里第。於是以甲士數千迎天子于金墉,百姓咸稱萬歲。帝自端門入,升殿,御廣室,送倫及荂等付金墉城。



 初,秀懼西軍至,復召虔還。是日宿九曲,詔遣使者免虔官,虔懼,棄軍將數十人歸於汶陽里。



 梁王肜表倫父子凶逆,宜伏誅。百官會議于朝堂,皆如肜表。遣尚書袁敞持節賜倫死,飲以金屑苦酒。倫慚,以巾覆面,曰:「孫秀誤我!孫秀誤我!」於是收荂、馥、虔、詡付廷尉獄,考
 竟。馥臨死謂虔曰:「坐爾破家也!」百官是倫所用者,皆斥免之,臺省府衛僅有存者,自兵興六十餘日,戰所殺害僅十萬人。



 凡與倫為逆豫謀大事者:張林為秀所殺;許超、士猗、孫弼、謝惔、殷渾與秀為王輿所誅;張衡、閭和、孫髦、高越自陽翟還,伏胤戰敗還洛陽,皆斬于東市;蔡璜自陽翟降齊王冏,還洛自殺;王輿以功免誅,後與東萊王蕤謀殺冏,又伏法。



 齊武閔王冏,字景治,獻王攸之子也。少稱仁惠,好振施,有父風。初,攸有疾,武帝不信,遣太醫診候,皆言無病。及
 攸薨,帝往臨喪,冏號踴訴父病為醫所誣,詔即誅醫。由是見稱,遂得為嗣。元康中,拜散騎常侍,領左軍將軍、翊軍校尉。趙王倫密與相結,廢賈后,以功轉游擊將軍。冏以位不滿意,有恨色。孫秀微覺之,且憚其在內,出為平東將軍、假節,鎮許昌。倫篡,遷鎮東大將軍、開府儀同三司,欲以寵安之。



 冏因眾心怨望,潛與離狐王盛、潁川王處穆謀起兵誅倫。倫遣腹心張烏覘之,烏反,曰:「齊無異志。」冏既有成謀未發,恐或泄,乃與軍司管襲殺處穆,送首於倫,以安其意。謀定,乃收襲殺之。遂與豫州刺史何勖、龍驤將軍董艾等起軍,遣使告成都、河間、常山、新野四
 王,移檄天下征鎮、州郡縣國,咸使聞知。揚州刺史郗隆承檄,猶豫未決,參軍王邃斬之,送首于冏。冏屯軍陽翟,倫遣其將閭和、張泓、孫輔出堮阪,與冏交戰。冏軍失利,堅壘自守。會成都軍破倫眾於黃橋,冏乃出軍攻和等,大破之。及王輿廢倫,惠帝反正,冏誅討賊黨既畢,率眾入洛,頓軍通章署,甲士數十萬,旌旗器械之盛,震於京都。天子就拜大司馬,加九錫之命,備物典策,如宣、景、文、武輔魏故事。



 冏於是輔政,居攸故宮,置掾屬四十人。大築第館,北取五穀市,南開諸署,毀壞廬舍以百數,使大匠營制,與西宮等。鑿千秋門墻以通西閣,後房施鐘懸,
 前庭舞八佾,沈于酒色,不入朝見。坐拜百官,符敕三臺,選舉不均,惟寵親暱。以車騎將軍何勖領中領軍。封葛為牟平公,路秀小黃公,衛毅陰平公,劉真安鄉公,韓泰封丘公,號曰「五公」,委以心膂。殿中御史桓豹奏事,不先經冏府,即考竟之。於是朝廷側目,海內失望矣。南陽處士鄭方露版極諫,主簿王豹屢有箴規,冏並不能用,遂奏豹殺之。有白頭公入大司馬府大呼,言有兵起,不出甲子旬。即收殺之。



 冏驕恣日甚,終無悛志。前賊曹屬孫惠復上諫曰:



 惠聞天下五難,四不可,而明公皆以居之矣。捐宗廟之主,忽千乘之重,躬貫甲胄,犯冒鋒刃,此
 一難也。奮三百之卒,決全勝之策,集四方之眾,致英豪之士,此二難也。舍殿堂之尊,居單幕之陋,安囂塵之慘,同將士之勞,此三難也。驅烏合之眾,當凶彊之敵,任神武之略,無疑阻之懼,此四難也。檄六合之內,著盟信之誓,升幽宮之帝,復皇祚之業,此五難也。大名不可久荷,大功不可久任,大權不可久執,大威不可久居。未有行其五難而不以為難,遺其不可而謂之為可。惠竊所不安也。



 自永熙以來,十有一載,人不見德,惟戮是聞。公族構篡奪之禍,骨肉遭梟夷之刑,群王被囚檻之困,妃主有離絕之哀。歷觀前代,國家之禍,至親之亂,未有今日
 之甚者也。良史書過,後嗣何觀!天下所以不去於晉,符命長存於世者,主無嚴虐之暴,朝無酷烈之政,武帝餘恩,獻王遺愛,聖慈惠和,尚經人心。四海所係,實在於茲。



 今明公建不世之義,而未為不世之讓,天下惑之,思求所悟。長沙、成都,魯、衛之密,國之親親,與明公計功受賞,尚不自先。今公宜放桓、文之勳,邁臧、札之風,芻狗萬物,不仁其化,崇親推近,功遂身退,委萬機於二王,命方嶽於群后,耀義讓之旗,鳴思歸之鑾,宅大齊之墟,振泱泱之風,垂拱青、徐之域,高枕營丘之籓。金石不足以銘高,八音不足以贊美,姬文不得專聖於前,太伯不得獨賢
 於後。今明公忘亢極之悔,忽窮高之凶,棄五嶽之安,居累卵之危,外以權勢受疑,內以百揆損神。雖處高臺之上,逍遙重仞之墉,及其危亡之憂,過於潁、翟之慮。群下竦戰,莫之敢言。



 惠以衰亡之餘,遭陽九之運,甘矢石之禍,赴大王之義,脫褐冠胄,從戎于許。契闊戰陣,功無可記,當隨風塵,待罪初服。屈原放斥,心存南郢;樂毅適趙,志戀北燕。況惠受恩,偏蒙識養,雖復暫違,情隆二臣,是以披露血誠,冒昧干迕。言入身戮,義讓功舉,退就鈇鑕,此惠之死賢於生也。



 冏不納,亦不加罪。



 翊軍校尉李含奔於長安,詐云受密詔,使河間王顒誅冏,因導以利謀。
 顒從之,上表曰:



 王室多故,禍難罔已。大司馬冏雖唱義有興復皇位之功,而定都邑,克寧社稷,實成都王勳力也。而冏不能固守臣節,實協異望。在許昌營有東西掖門,官置治書侍御史,長史、司馬直立左右,如侍臣之儀。京城大清,篡逆誅夷,而率百萬之眾來繞洛城。阻兵經年,不一朝覲,百官拜伏,晏然南面。壞樂官市署,用自增廣。輒取武庫祕杖,嚴列不解。故東萊王蕤知其逆節,表陳事狀,而見誣陷,加罪黜徙。以樹私黨,僭立官屬。幸妻嬖妾,名號比之中宮。沈湎酒色,不恤群黎。董艾放縱,無所畏忌,中丞按奏,而取退免。張偉惚恫,擁停詔可,葛
 旟小豎,維持國命。操弄王爵,貨賂公行。群姦聚黨,擅斷殺生。密署腹心,實為貨謀。斥罪忠良,伺窺神器。



 臣受重任,蕃衛方嶽,見冏所行,實懷激憤。即日翊軍校尉李含乘驛密至,宣騰詔旨。臣伏讀感切,五情若灼。《春秋》之義,君親無將。冏擁彊兵,樹置私黨,權官要職,莫非腹心。雖復重責之誅,恐不義服。今輒勒兵,精卒十萬,與州征並協忠義,共會洛陽。驃騎將軍長沙王乂,同奮忠誠,廢冏還第。有不順命,軍法從事。成都王穎明德茂親,功高勳重,往歲去就,允合眾望,宜為宰輔,代冏阿衡之任。



 顒表既至,冏大懼,會百僚曰:「昔孫秀作逆,篡逼帝王,社稷傾
 覆,莫能禦難。孤糾合義眾,掃除元惡,臣子之節,信著神明。二王今日聽信讒言,造構大難,當賴忠謀以和不協耳。」司徒王戎、司空東海王越說冏委權崇讓。冏從事中郎葛旟怒曰:「趙庶人聽任孫秀,移天易日,當時喋喋,莫敢先唱。公蒙犯矢石,躬貫甲胄,攻圍陷陣,得濟今日。計功行封,事殷未遍。三臺納言,不恤王事,賞報稽緩,責不在府。讒言僭逆,當共誅討,虛承偽書,令公就第。漢、魏以來,王侯就第寧有得保妻子者乎!議者可斬。」於是百官震悚,無不失色。



 長沙王乂徑入宮,發兵攻冏府。冏遣董艾陳兵宮西。乂又遣宋洪等放火燒諸觀閣及千秋、神
 武門。冏令黃門令王湖悉盜騶虞幡,唱云:「長沙王矯詔。」乂又稱:「大司馬謀反,助者誅五族。」是夕,城內大戰,飛矢雨集,火光屬天。帝幸上東門,矢集御前。群臣救火,死者相枕。明日,冏敗,乂擒冏至殿前,帝惻然,欲活之。乂叱左右促牽出,冏猶再顧,遂斬於閶闔門外,徇首六軍。諸黨屬皆夷三族。幽其子淮陵王超、樂安王冰、濟陽王英于金墉。暴冏尸於西明亭,三日而莫敢收斂。冏故掾屬荀闓等表乞殯葬,許之。



 初,冏之盛也,有一婦人詣大司馬府求寄產。吏詰之,婦人曰:「我截齊便去耳。」識者聞而惡之。時又謠曰:「著布袙腹,為齊持服。」俄而冏誅。



 永興初,詔
 以冏輕陷重刑,前勳不宜堙沒,乃赦其三子超、冰、英還第,封超為縣王,以繼冏祀,歷員外散騎常侍。光熙初,追冊冏曰:「咨故大司馬、齊王冏:王昔以宗籓穆胤紹世,緒於東國,作翰許京,允鎮靜我王室。涎率義徒,同盟觸澤,克成元勳,大濟潁東。朕用應嘉茂績,謂篤爾勞,俾式先典,以疇茲顯懿。廓士殊分,跨兼吳楚,崇禮備物,寵侔蕭、霍,庶憑翼戴之重,永隆邦家之望。而恭德不建,取侮二方,有司過舉,致王于戮。古人有言曰:『用其法,猶思其人。』況王功濟朕身,勳存社稷,追惟既往,有悼於厥心哉!今復王本封,命嗣子還紹厥緒,禮秩典度,一如舊制。使使
 持節、大鴻臚即墓賜策,祠以太牢。魂而有靈,祗服朕命,肆寧爾心,嘉茲寵榮。」子超嗣爵。



 永嘉中,懷帝下詔,重述冏唱義元勳,還贈大司馬,加侍中、假節,追謚。及洛陽傾覆,超兄弟皆沒於劉聰,冏遂無後。太元中,詔以故南頓王宗子柔之襲封齊王,紹攸、冏之祀,歷散騎常待。元興初,會稽王道子將討桓玄,詔柔之兼侍中,以騶虞幡宣告江、荊二州,至姑孰,為玄前鋒所害。贈光祿勳。子建之立。宋受禪,國除。



 鄭方者,字子回,慷慨有志節,博涉史傳,卓犖不常,鄉閭
 有識者歎其奇,而未能薦達。及冏輔政專恣,方發憤步詣洛陽,自稱荊楚逸民,獻書於冏曰:「方聞聖明輔世,夙夜祗懼,泰而不驕,所以長守貴也。今大王安不慮危,耽於酒色,燕樂過度,其失一也。大王檄命,當使天下穆如清風,宗室骨肉永無纖介,今則不然,其失二也。四夷交侵,邊境不靜,大王自以功業興隆,不以為念,其失三也。大王興義,群庶競赴,天下雖寧,人勞窮苦,不聞大王振救之令,其失四也。又與義兵歃血而盟,事定之後,賞不踰時,自清泰已來,論功未分,此則食言,其失五也。大王建非常之功,居宰相之任,謗聲盈塗,人懷忿怨,方以狂
 愚,冒死陳誠。」冏含忍答之云:「孤不能致五闕,若無子,則不聞其過矣。」未幾而敗焉。



 長沙厲王乂,字士度,武帝第六子也。太康十年受封,拜員外散騎常侍。及武帝崩,乂時年十五,孺慕過禮。會楚王瑋奔喪,諸王皆近路迎之,乂獨至陵所,號慟以俟瑋。拜步兵校尉。及瑋之誅二公也,乂守東掖門。會騶虞幡出,乂投弓流涕曰:「楚王被詔,是以從之,安知其非!」瑋既誅,乂以同母,貶為常山王,之國。



 乂身長七尺五寸,開朗果斷,才力絕人,虛心下士,甚有名譽。三王之舉義也,乂
 率國兵應之,過趙國,房子令距守,乂殺之,進軍為成都後係。常山內史程恢將貳於乂,乂到鄴,斬恢及其五子。至洛,拜撫軍大將軍,領左軍將軍。頃之,遷驃騎將軍、開府,復本國。



 乂見齊王冏漸專權,嘗與成都王穎俱拜陵,因謂穎曰:「天下者,先帝之業也,王宜維之。」時聞其言者皆憚之。及河間王顒將誅冏,傳檄以乂為內主。冏遣其將董艾襲乂,乂將左右百餘人,手斫車幰,露乘馳赴宮,閉諸門,奉天子與冏相攻,起火燒冏府,連戰三日,冏敗,斬之,并誅諸黨與二千餘人。



 顒本以乂弱冏強,冀乂為冏所擒,然後以乂為辭,宣告四方共討之,因廢帝立成
 都王,己為宰相,專制天下。即而乂殺冏,其計不果,乃潛使侍中馮蓀、河南尹李含、中書令卞粹等襲乂。乂並誅之。顒遂與穎同伐京都。穎遣刺客圖乂,時長沙國左常侍王矩侍直,見客色動,遂殺之。詔以乂為大都督以距顒。連戰自八月至十月,朝議以乂、穎兄弟,可以辭說而釋,乃使中書令王衍行太尉,光祿勳石陋行司徒,使說穎,令與乂分陜而居,穎不從。乂因致書於穎曰:「先帝應乾撫運,統攝四海,勤身苦己,克成帝業,六合清泰,慶流子孫。孫秀作逆,反易天常,卿興義眾,還復帝位。齊王恃功,肆行非法,上無宰相之心,下無忠臣之行,遂其讒惡,
 離逖骨肉,主上怨傷,尋已蕩除。吾之與卿,友于十人,同產皇室,受封外都,各不能闡敷王教,經濟遠略。今卿復與太尉共起大眾,阻兵百萬,重圍宮城。群臣同忿,聊即命將,示宣國威,未擬摧殄。自投溝澗,蕩平山谷,死者日萬,酷痛無罪。豈國恩之不慈,則用刑之有常。卿所遣陸機不樂受卿節鉞,將其所領,私通國家。想來逆者,當前行一尺,卻行一丈,卿宜還鎮,以寧四海,令宗族無羞,子孫之福也。如其不然,念骨肉分裂之痛,故復遣書。」



 穎復書曰:「文、景受圖,武皇乘運,庶幾堯、舜,共康政道,恩隆洪業,本枝百世。豈期骨肉豫禍,后族專權,楊、賈縱毒,齊、趙
 內篡。幸以誅夷,而未靜息。每憂王室,心悸肝爛。羊玄之、皇甫商等恃寵作禍,能不興慨!於是征西羽檄,四海雲應。本謂仁兄同其所懷,便當內擒商等,收級遠送。如何迷惑,自為戎首!上矯君詔,下離愛弟,推移輦轂,妄動兵威,還任豺狼,棄戮親善。行惡求福,如何自勉!前遣陸機董督節鉞,雖黃橋之退,而溫南收勝,一彼一此,未足增慶也。今武士百萬,良將銳猛,要當與兄整頓海內。若能從太尉之命,斬商等首,投戈退讓,自求多福,穎亦自歸鄴都,與兄同之。奉覽來告,緬然慷慨。慎哉大兄,深思進退也!」



 乂前後破穎軍,斬獲六七萬人。戰久糧乏,城中大
 飢,雖曰疲弊,將士同心,皆願效死。而乂奉上之禮未有虧失,張方以為未可剋,欲還長安。而東海王越慮事不濟,潛與殿中將收乂送金墉城。乂表曰:「陛下篤睦,委臣朝事。臣小心忠孝,神祇所鑒。諸王承謬,率眾見責,朝臣無正,各慮私困,收臣別省,送臣幽宮。臣不惜軀命,但念大晉衰微,枝黨欲盡,陛下孤危。若臣死國寧,亦家之利。但恐快凶人之志:無益於陛下耳。」



 殿中左右恨乂功垂成而敗,謀劫出之,更以距穎。越懼難作,欲遂誅乂。黃門郎潘滔勸越密告張方,方遣部將郅輔勒兵三千,就金墉收乂,至營,炙而殺之。乂冤痛之聲達於左右,三軍莫
 不為之垂涕。時年二十八。



 乂將殯於城東,官屬莫敢往,故掾劉佑獨送之,步持喪車,悲號斷絕,哀感路人。張方以其義士,不之問也。初,乂執權之始,洛下謠曰:「草木萌牙殺長沙。」乂以正月二十五日廢,二十七日死,如謠言焉。永嘉中,懷帝以乂子碩嗣,拜散騎常侍,後沒于劉聰。



 成都王穎,字章度,武帝第十六子也。太康末受封,邑十萬戶。後拜越騎校尉,加散騎常侍、車騎將軍。賈謐嘗與皇太子博,爭道。穎在坐,厲聲呵謐曰:「皇太子國之儲君,賈謐何得無禮!」謐懼,由此出穎為平北將軍,鎮鄴。轉鎮
 北大將軍。



 趙王倫之篡也,進征北大將軍,加開府儀同三司。及齊王冏舉義,穎發兵應冏,以鄴令盧志為左長史,頓丘太守鄭琰為右長史,黃門郎程牧為左司馬,陽平太守和演為右司馬。使兗州刺史王彥,冀州刺史李毅,督護趙驤、石超等為前鋒。羽檄所及,莫不響應。至朝歌,眾二十餘萬。趙驤至黃橋,為倫將士猗、許超所敗,死者八千餘人,士眾震駭。穎欲退保朝歌,用盧志、王彥策,又使趙驤率眾八萬,與王彥俱進。倫復遣孫會、劉琨等率三萬人,與猗、超合兵距驤等,精甲耀日,鐵騎前驅。猗既戰勝,有輕驤之心。未及溫十餘里,復大戰,猗等奔潰。
 穎遂過河,乘勝長驅。左將軍王輿殺孫秀,幽趙王倫,迎天子反正。及穎入京都,誅倫。使趙驤、石超等助齊王冏攻張泓於陽翟,泓等遂降。冏始率眾入洛,自以首建大謀,遂擅威權。穎營于太學,及入朝,天子親勞焉。穎拜謝曰:「此大司馬臣冏之勳,臣無豫焉。」見訖,即辭出,不復還營,便謁太廟,出自東陽城門,遂歸鄴。遣信與冏別,冏大驚,馳出送穎,至七里澗及之。穎住車言別,流涕,不及時事,惟以太妃疾苦形於顏色,百姓觀者莫不傾心。



 至鄴,詔遣兼太尉王粹加九錫殊禮,進位大將軍、都督中外諸軍事、假節、加黃鉞、錄尚書事,入朝不趨,劍履上殿。穎
 拜受徽號,讓殊禮九錫,表論興義功臣盧志、和演、董洪、王彥、趙驤等五人,皆封開國公侯。又表稱:「大司馬前在陽翟,與彊賊相持既久,百姓創痍,饑餓凍餒,宜急振救。乞差發郡縣車,一時運河北邸閣米十五萬斛,以振陽翟饑人。」盧志言於穎曰:「黃橋戰亡者有八千餘人,既經夏暑,露骨中野,可為傷惻。昔周王葬枯骨,故《詩》云『行有死人,尚或墐之』。況此等致死王事乎!」穎乃造棺八千餘枚,以成都國秩為衣服,斂祭,葬於黃橋北,樹枳籬為之塋域。又立都祭堂,刊石立碑,紀其赴義之功,使亡者之家四時祭祀有所。仍表其門閭,加常戰亡二等。又命河
 內溫縣埋藏趙倫戰死士卒萬四千餘人。穎形美而神昏,不知書,然器性敦厚,委事於志,故得成其美焉。



 及齊王冏驕侈無禮,於是眾望歸之。詔遣侍中馮蓀、中書令卞粹喻穎入輔政,并使受九錫。穎猶讓不拜。尋加太子太保。穎嬖人孟玖不欲還洛,又程太妃愛戀鄴都,以此議久不決。留義募將士既久,咸怨曠思歸,或有輒去者,乃題鄴城門云:「大事解散蠶欲遽。請且歸,赴時務。昔以義來,今以義去。若復有急更相語。」穎知不可留,因遣之,百姓乃安。及冏敗,穎懸執朝政,事無巨細,皆就鄴諮之。後張昌擾亂荊土,穎拜表南征,所在響赴。既恃功驕奢,
 百度弛廢,甚於冏時。



 穎方恣其欲,而憚長沙王乂在內,遂與河間王顒表請誅后父羊玄之、左將軍皇甫商等,檄乂使就第。乃與顒將張方伐京都,以平原內史陸機為前鋒都督、前將軍、假節。穎次朝歌,每夜矛戟有光若火,其壘井中皆有龍象。進軍屯河南,阻清水為壘,造浮橋以通河北,以大木函盛石,沈之以繫橋,名曰石鱉。陸機戰敗,死者甚眾,機又為孟玖所譖,穎收機斬之,夷其三族,語在《機傳》。於是進攻京城。時常山人王輿合眾萬餘,欲襲穎,會乂被執,其黨斬輿降。穎既入京師,復旋鎮于鄴,增封二十郡,拜丞相。河間王顒表穎宜為儲副,遂
 廢太子覃,立穎為皇太弟,丞相如故,制度一依魏武故事,乘輿服御皆遷於鄴。表罷宿衛兵屬相府,更以王官宿衛。僭侈日甚,有無君之心,委任孟玖等,大失眾望。



 永興初,左衛將軍陳,殿中中郎褾苞、成輔及長沙故將上官巳等,奉大駕討穎,馳檄四方,赴者雲集。軍次安陽,眾十餘萬,鄴中震懼。穎欲走,其掾步熊有道術,曰:「勿動!南軍必敗。」穎會其眾問計,東安王繇乃曰:「天子親征,宜罷甲,縞素出迎請罪。」司馬王混、參軍崔曠勸穎距戰,穎從之,乃遣奮武將軍石超率眾五萬,次於蕩陰。二弟匡、規自鄴赴王師,云:「鄴中皆已離散。」由是不甚設備。超
 眾奄至,王師敗績,矢及乘輿,侍中嵇紹死於帝側,左右皆奔散,乃棄天子於槁中。超遂奉帝幸鄴。穎改元建武,害東安王繇,署置百官,殺生自己,立郊於鄴南。



 安北將軍王浚、寧北將軍東嬴公騰殺穎所置幽州刺史和演,穎征浚,浚屯冀州不進,與騰及烏丸、羯朱襲穎。候騎至鄴,穎遣幽州刺史王斌及石超、李毅等距浚,為羯朱等所敗。鄴中大震,百僚奔走,士卒分散。穎懼,將帳下數十騎,擁天子,與中書監慮志單車而走,五日至洛。羯朱追至朝歌,不及而還。河間王顒遣張方率甲卒二萬救穎,至洛,方乃挾帝,擁穎及豫章王并高光、慮志等歸於長
 安。顒廢穎歸籓,以豫章王為皇太弟。



 穎既廢,河北思之。鄴中故將公師籓、汲桑等起兵以迎穎,眾情翕然。顒復拜穎鎮軍大將軍、都督河北諸軍事,給兵千人,鎮鄴。穎至洛,而東海王越率眾迎大駕,所在鋒起。穎以北方盛彊,懼不可進,自洛陽奔關中。值大駕還洛,穎自華陰趨武關,出新野。帝詔鎮南將軍劉弘、南中郎將劉陶收捕穎,於是棄母妻,單車與二子廬江王普、中都王廓渡河赴朝歌,收合故將士數百人,欲就公師籓。頓丘太守馮嵩執穎及普、廓送鄴,范陽王虓幽之,而無他意。屬虓暴薨,虓長史劉輿見穎為鄴都所服,慮為後患,祕不發喪,
 偽令人為臺使,稱詔夜賜穎死。穎謂守者田徽曰:「范陽王亡乎?」徽曰:「不知。」穎曰:「卿年幾?』徽曰:「五十。」穎曰:「知天命不?」徽曰:「不知。」穎曰:「我死之後,天下安乎不安乎?我自放逐,於今三年,身體手足不見洗沐,取數斗湯來!」其二子號泣,穎敕人將去。乃散髮東首臥,命徽縊之,時年二十八。二子亦死。鄴中哀之。



 穎之敗也,官屬並奔散,惟盧志隨從不怠,論者稱之。其後汲桑害東贏公騰,稱為穎報仇,遂出穎棺,載之於軍中,每事啟靈,以行軍令。桑敗,度棺於故井中。穎故臣收之,改葬于洛陽,懷帝加以縣王禮。



 穎死後數年,開封間有傳穎子年十餘歲,流離百姓
 家,東海王越遣人殺之。永嘉中,立東萊王蕤子遵為穎嗣,封華容縣王。後沒於賊,國除。



 河間王顒,字文載,安平獻王孚孫,太原烈王瑰之子也。初襲父爵,咸寧二年就國。三年,改封河間。少有清名,輕財愛士。與諸王俱來朝,武帝歎顒可以為諸國儀表。元康初,為北中郎將,監鄴城。九年,代梁王肜為平西將軍,鎮關中。石函之制,非親親不得都督關中,顒於諸王為疏,特以賢舉。



 及趙王倫篡位,齊王冏謀討之。前安西參軍夏侯奭自稱侍御史,在始平合眾,得數千人,以應冏,
 遣信要顒。顒遣主簿房陽、河間國人張方討擒奭,及其黨十數人,於長安市腰斬之。及冏檄至,顒執冏使,送之於倫。倫徵兵於顒,顒遣方率關右健將赴之。方至華陰,顒聞二王兵盛,乃加長史李含龍驤將軍,領督護席薳等追方軍迴,以應二王。義兵至潼關,而倫、秀已誅,天子反正,含、方各率眾還。及冏論功,雖怒顒初不同,而終能濟義,進位侍中、太尉,加三賜之禮。



 後含為翊軍校尉,與冏參軍皇甫商、司馬趙驤等有憾,遂奔顒,詭稱受密詔伐冏,因說利害。顒納之,便發兵,遣使邀成都王穎。以含為都督,率諸軍屯陰盤,前鋒次於新安,去洛百二十里。
 檄長沙王乂討冏。及冏敗,顒以含為河南尹,使與馮蓀、卞粹等潛圖害乂。商知含前矯妄及與顒陰謀,具以告乂。乂乃誅含等。顒聞含死,即起兵以討商為名,使張方為都督,領精卒七萬向洛。方攻商,商距戰而潰,方遂進攻西明門。乂率中軍左右衛擊之,方眾大敗,死者五千餘人。方初於駃水橋西為營,於是築壘數重,外引廩穀,以足軍資。乂復從天子出攻方,戰輒不利。及乂死,方還長安。詔以顒為太宰、大都督、雍州牧。顒廢皇太子覃,立成都王穎為太弟,改年,大赦。



 左衛將軍陳奉天子伐穎,顒又遣方率兵二萬救鄴。天子已幸鄴。方屯兵洛陽。
 及王浚等伐穎,穎挾天子歸洛陽。方將兵入殿中,逼帝幸其壘,掠府庫,將焚宮廟以絕眾心。盧志諫,乃止。方又逼天子幸長安。顒及選置百官,改秦州為定州。及東海王越起兵徐州,西迎大駕,關中大懼,方謂顒曰:「方所領猶有十餘萬眾,奉送大駕還洛宮,使成都王反鄴,公自留鎮關中,方北討博陵。如此,天下可小安,無復舉手者。」顒慮事大難濟,不許。乃假劉喬節,進位鎮東大將軍,遣成都王穎總統樓褒、王闡等諸軍,據河橋以距越。王浚遣督護劉根,將三百騎至河上。闡出戰,為根所殺。穎頓軍張方故壘,范陽王虓遣鮮卑騎與平昌、博陵眾襲河
 橋,樓褒西走,追騎至新安,道路死者不可勝數。



 初,越以張方劫遷車駕,天下怨憤,唱義與山東諸侯剋期奉迎,先遣說顒,令送帝還都,與顒分陜而居。顒欲從之,而方不同。及東軍大捷,成都等敗,顒乃令方親信將郅輔夜斬方,送首以示東軍。尋變計,更遣刁默守潼關,乃咎輔殺方,又斬輔。顒先遣將呂朗等據滎陽,范陽王虓司馬劉琨以方首示朗,於是朗降。時東軍既盛,破刁默以入關,顒懼,又遣馬瞻、郭傳於霸水禦之,瞻等戰敗散走。顒乘單馬,逃于太白山。東軍入長安,大駕旋,以太弟太保梁柳為鎮西將軍,守關中。馬瞻等出詣柳,因共殺柳於
 城內。瞻等與始平太守梁邁合從,迎顒於南山。顒初不肯入府,長安令蘇眾、記室督朱永勸顒表稱柳病卒,輒知方事。弘農太守裴暠、秦國內史賈龕、安定太守賈疋等起義討顒,斬馬瞻、梁邁等。東海王越遣督護麋晃率國兵伐顒。至鄭,顒將牽秀距晃,晃斬秀,并其二子。義軍據有關中,顒保城而已。



 永嘉祖,詔書以顒為司徒,乃就徵。南陽王模遣將梁臣於新安雍谷車上扼殺之,並其三子。詔以彭城元王植子融為顒嗣,改封樂成縣王。薨,無子。建興中,元帝又以彭城康王釋子欽為融嗣。



 東海孝獻王越,字元超,高密王泰之次子也。少有令名,謙虛持布衣之操,為中外所宗。初以世子為騎都尉,與駙馬都尉楊邈及瑯邪王伷子繇俱侍講東宮,拜散騎侍郎,歷左衛將軍,加侍中。討楊駿有功,封五千戶侯。遷散騎常侍、輔國將軍、尚書右僕射,領游擊將軍。復為侍中,加奉車都尉,給溫信五十人,別封東海王,食六縣。永康初,為中書令,徙侍中,遷司空,領中書監。



 成都王穎攻長沙王乂,乂固守洛陽,殿中諸將及三部司馬疲於戰守,密與左衛將軍朱默夜收乂別省,逼越為主,啟惠帝免乂官。事定,越稱疾遜位。帝不許,加守尚書令。太安初,
 帝北征鄴,以越為大都督。六軍敗,越奔下邳,徐州都督、東平王楙不納,越徑還東海。成都王穎以越兄弟宗室之美,下寬令招之,越不應命。帝西幸,以越為太傅,與太宰顒夾輔朝政,讓不受。東海中尉劉洽勸越發兵以備穎,越以洽為左司馬,尚書曹馥為軍司。既起兵,楙懼,乃以州與越。越以司空領徐州都督,以楙領兗州刺史。越三弟並據方任征伐,輒選刺史守相,朝士多赴越。而河間王顒挾天子,發詔罷越等,皆令就國。越唱義奉迎大駕,還復舊都,率甲卒三萬,西次蕭縣。豫州刺史劉喬不受越命,遣子祐距之,越軍敗。范陽王虓遣督護田徽以
 突騎八百迎越,遇祐於譙,祐眾潰,越進屯陽武。山東兵盛,關中大懼,顒斬送張方首求和,尋變計距越。越率諸侯及鮮卑許扶歷、駒次宿歸等步騎迎惠帝反洛陽。詔越以太傅錄尚書,以下邳、濟陽二郡增封。



 及懷帝即位,委政於越。吏部郎周穆,清河王覃舅,越之姑子也,與其妹夫諸葛玫共說越曰:「主上之為太弟,張方意也。清河王本太子,為群凶所廢。先帝暴崩,多疑東宮。公盍思伊、霍之舉,以寧社稷乎?」言未卒,越曰:「此豈宜言邪!」遂叱左右斬之。以玫、穆世家,罪止其身,因此表除三族之法。帝始親萬機,留心庶事,越不悅,求出籓,帝不許。越遂出鎮
 許昌。



 永嘉初,自許昌率茍晞及冀州刺史丁劭討汲桑,破之。越還於許,長史潘滔說之曰:「兗州天下樞要,公宜自牧。」及轉茍晞為青州刺史,由是與晞有隙。



 尋詔越為丞相,領兗州牧,督兗、豫、司、冀、幽、並六州。越辭丞相不受,自許遷于鄄城。越恐清河王覃終為儲副,矯詔收付金墉城,尋害之。



 王彌入許,越遣左司馬王斌率甲士五千人入衛京都。鄄城自壞,越惡之,移屯濮陽,又遷于滎陽。召田甄等六率,甄不受命,越遣監軍劉望討甄。初,東嬴公騰之鎮鄴也,攜並州將田甄、甄弟蘭、任祉、祁濟、李惲、薄盛等部眾萬餘人至鄴,遣就穀冀州,號為乞活。及騰
 敗,甄等邀破汲桑於赤橋,越以甄為汲郡,蘭為鉅鹿大守。甄求魏郡,越不許,甄怒,故召不至。望既渡河,甄退。李惲、薄盛斬田蘭,率其眾降,甄、祉、濟棄軍奔上黨。



 越自滎陽還洛陽,以太學為府。疑朝臣貳己,乃誣帝舅王延等為亂,遣王景率甲士三千人入宮收延等,付廷尉殺之。越解兗州牧,領司徒。越既與茍晞構怨,又以頃興事多由殿省,乃奏宿衛有侯爵者皆罷之。時殿中武官並封侯,由是出者略盡,皆泣涕而去。乃以東海國上軍將軍何倫為右衛將軍,王景為左衛將軍,領國兵數百人宿衛。



 越自誅王延等,大失眾望,而多有猜嫌。散騎侍郎高
 韜有憂國之言,越誣以訕謗時政害之,而不自安。乃戎服入見,請討石勒,且鎮集兗、豫以援京師。帝曰:「今逆虜侵逼郊畿,王室蠢蠢,莫有固心。朝廷社稷,倚賴於公,豈可遠出以孤根本!」對曰:「臣今率眾邀賊,勢必滅之。賊滅則不逞消殄,已東諸州職貢流通。此所以宣暢國威,籓屏之宜也。若端坐京輦以失機會,則釁弊日滋,所憂逾重。」遂行。留妃裴氏,世子、鎮軍將軍毗,及龍驤將軍李惲并何倫等守衛京都。表以行臺隨軍,率甲士四萬東屯于項,王公卿士隨從者甚眾。詔加九錫。越乃羽檄四方曰:「皇綱失御,社稷多難,孤以弱才,備當大任。自頃胡寇
 內逼,偏裨失利,帝鄉便為戎州,冠帶奄成殊域,朝廷上下,以為憂懼。皆由諸侯蹉跎,遂及此難。投袂忘履,討之已晚。人情奉本,莫不義奮。當須合會之眾,以俟戰守之備。宗廟主上,相賴匡救。檄至之日,便望風奮發,忠臣戰士效誠之秋也。」所徵皆不至。而茍晞又表討越,語在《晞傳》。越以豫州刺史馮嵩為左司馬,自領豫州牧。



 越專擅威權,圖為霸業,朝賢素望,選為佐吏,名將勁卒,充于己府,不臣之迹,四海所知。而公私罄乏,所在寇亂,州郡攜貳,上下崩離,禍結釁深,遂憂懼成疾。永嘉五年,薨于項。秘不發喪。以襄陽王範為大將軍,統其眾。還葬東海。石
 勒追及於苦縣寧平城,將軍錢端出兵距勒,戰死,軍潰。勒命焚越柩曰:「此人亂天下,吾為天下報之,故燒其骨以告天地。」於是數十萬眾,勒以騎圍而射之,相踐如山。王公士庶死者十餘萬。王彌弟璋焚其餘眾,并食之。天下歸罪於越。帝發詔貶越為縣王。



 何倫、李惲聞越之死,祕不發喪,奉妃裴氏及毗出自京邑,從者傾城,所經暴掠。至洧倉,又為勒所敗,毗及宗室三十六王俱沒于賊。李惲殺妻子奔廣宗,何倫走下邳。裴妃為人所略,賣於吳氏,太興中,得渡江,欲招魂葬越。元帝詔有司詳議,博士傅純曰:「聖人制禮,以事緣情,設冢槨以藏形,而事之
 以凶;立廟祧以安神,而奉之以吉。送形而往,迎精而還。此墓廟之大分,形神之異制也。至於室廟寢廟祊祭非一處,所以廣求神之道,而獨不祭於墓,明非神之所處也。今亂形神之別,錯廟墓之宜,違禮制義,莫大於此。」於是下詔不許。裴妃不奉詔,遂葬越於廣陵。太興末,墓毀,改葬丹徒。



 初,元帝鎮建鄴,裴妃之意也,帝深德之,數幸其第,以第三子沖奉越後。薨,無子,成帝以少子奕繼之。哀帝徙奕為瑯邪王,而東海無嗣。隆安初,安帝更以會稽忠王次子彥璋為東海王,繼沖為曾孫。為桓玄所害,國除。



 史臣曰:昔高辛撫運,釁起參商;宗周嗣歷,禍纏管、蔡。祥觀曩冊,逖聽前古,亂臣賊子,昭鑒在焉。有晉鬱興,載崇籓翰,分茅錫瑞,道光恒典;儀臺飾袞,禮備彞章。汝南以純和之姿,失於無斷;楚隱習果銳之性,遂成凶很。或位居朝右,或職參近禁,俱為女子所詐,相次受誅,雖曰自貽,良可哀也!倫實庸瑣,見欺孫秀,潛構異圖,煽成姦慝。乃使元良遘怨酷,上宰陷誅夷,乾耀以之暫傾,皇綱於焉中圮。遂裂冠毀冕,幸百六之會;綰璽揚纛,窺九五之尊。夫神器焉可偷安,鴻名豈容妄假!而欲託茲淫祀,享彼天年,凶闇之極,未之有也。冏名父之子,唱義勤王,摧
 偽業於既成,拯皇輿於已墜,策勳考績,良足可稱。然而臨禍忘憂,逞心縱欲,曾不知樂不可極,盈難久持,笑古人之未工,忘己事之已拙。向若採王豹之奇策,納孫惠之嘉謀,高謝袞章,永表東海,雖古之伊、霍,何以加焉!長沙材力絕人,忠概邁俗,投弓掖門,落落標壯夫之氣;馳車魏闕,懍懍懷烈士之風。雖復陽九數屯,在三之情無奪。撫其遺節,終始可觀。穎既入總大權,出居重鎮,中臺藉以成務,東夏資其宅心,乃協契河間,共圖進取。而顒任李含之狙詐,杖張方之陵虐,遂使武閔喪元,長沙授首,逞其無君之志,矜其不義之彊。鑾駕北巡,異乎有征
 無戰;乘輿西幸,非由望秩觀風。若火燎原,猶可撲滅,矧茲安忍,能無及乎!東海糾合同盟,創為義舉,匡復之功未立,陵暴之釁已彰,罄彼車徒,固求出鎮。既而帝京寡弱,狡寇憑陵,遂令神器劫遷,宗社顛覆,數十萬眾並垂餌於豺狼,三十六王咸隕身於鋒刃。禍難之極,振古未聞。雖及焚如,猶為幸也。自惠皇失政,難起蕭墻,骨肉相殘,黎元塗炭,胡塵驚而天地閉,戎兵接而宮廟隳,支屬肇其禍端,戎羯乘其間隙,悲夫!《詩》所謂「誰生厲階,至今為梗」,其八王之謂矣。



 贊曰:亮總朝政,瑋懷職競。讒巧乘間,艷妻過聽。構怨連
 禍,遞遭非命。倫實下愚,敢竊龍圖,亂常奸位,遄及嚴誅。偉哉武閔!首創宏謨。德之不建,良可悲夫!長沙奉國,始終靡慝;功虧一簣,奄罹殘賊。章度勤王,效立名揚;合從關右,犯順爭強,事窮勢蹙,俱為亂亡。元超作輔,出征入撫,敗國喪師,無君震主。焚如之變,抑惟自取。



\end{pinyinscope}