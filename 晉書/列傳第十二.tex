\article{列傳第十二}

\begin{pinyinscope}
王渾
 \gezhu{
  子濟}
 王濬唐彬



 王渾,字玄沖,太原晉陽人也。父昶,魏司空。渾沈雅有器量。襲父爵京陵侯,辟大將軍曹爽掾。爽誅,隨例免。起為懷令,參文帝安東軍事,累遷散騎黃門侍郎、散騎常侍。咸熙中為越騎校尉。武帝受禪,加揚烈將軍,遷徐州刺史。時年荒歲饑,渾開倉振贍,百姓賴之。泰始初,增封邑千八百戶。久之,遷東中郎將,監淮北諸軍事,鎮許昌。數
 陳損益,多見納用。



 轉征虜將軍、監豫州諸軍事、假節,領豫州刺史。渾與吳接境,宣布威信,前後降附甚多。吳將薛瑩、魯淑眾號十萬,淑向弋陽,瑩向新息。時州兵並放休息,眾裁一旅,浮淮潛濟,出其不意,瑩等不虞晉師之至。渾擊破之,以功封次子尚為關內侯。遷安東將軍、都督揚州諸軍事,鎮壽春。吳人大佃皖城,圖為邊害。渾遣揚州刺史應綽督淮南諸軍攻破之,並破諸別屯,焚其積穀百八十餘萬斛、稻苗四千餘頃、船六百餘艘。渾遂陳兵東疆,視其地形險易,歷觀敵城,察攻取之勢。



 及大舉伐吳,渾率師出橫江,遣參軍陳慎、都尉張喬攻尋陽
 瀨鄉,又擊吳牙門將孔忠,皆破之,獲吳將周興等五人。又遣殄吳護軍李純據高望城,討吳將俞恭,破之,多所斬獲。吳歷武將軍陳代、平虜將軍朱明懼而來降。吳丞相張悌、大將軍孫震等率眾數萬指城陽,渾遣司馬孫疇、揚州刺史周浚擊破之,臨陣斬二將,及首虜七千八百級,吳人大震。



 孫皓司徒何植、建威將軍孫晏送印節詣渾降。既而王濬破石頭,降孫皓,威名益振。明日,渾始濟江,登建鄴宮,釃酒高會。自以先據江上,破皓中軍,案甲不進,致在王濬之後。意甚愧恨,有不平之色,頻奏濬罪狀,時人譏之。帝下詔曰:「使持節、都督揚州諸軍事、安
 東將軍、京陵侯王渾,督率所統,遂逼秣陵,令賊孫皓救死自衛,不得分兵上赴,以成西軍之功,又摧大敵,獲張悌,使皓途窮勢盡,面縛乞降。遂平定秣陵,功勛茂著。其增封八千戶,進爵為公,封子澄為亭侯、弟湛為關內侯,賜絹八千匹。」轉征東大將軍,復鎮壽陽。渾不尚刑名,處斷明允。時吳人新附,頗懷畏懼。渾撫循羈旅,虛懷綏納,座無空席,門不停賓。於是江東之士莫不悅附。



 徵拜尚書左僕射,加散騎常侍。會朝臣立議齊王攸當之籓,渾上書諫曰:「伏承聖詔,憲章古典,進齊王攸為上公,崇其禮儀,遣攸之國。昔周氏建國,大封諸姬,以籓帝室,永世
 作憲。至於公旦,武王之弟,左右王事,輔濟大業,不使歸籓。明至親義著,不可遠朝故也。是故周公得以聖德光弼幼主,忠誠著於《金縢》,光述文武仁聖之德。攸於大晉,姬旦之親也。宜贊皇朝,與聞政事,實為陛下腹心不貳之臣。且攸為人,修潔義信,加以懿親,志存忠貞。今陛下出攸之國,假以都督虛號,而無典戎幹方之實,去離天朝,不預王政。傷母弟至親之體,虧友于款篤之義,懼非陛下追述先帝、文明太后待攸之宿意也。若以攸望重,於事宜出者,今以汝南王亮代攸。亮,宣皇帝子,文皇帝弟,伷、駿各處方任,有內外之資,論以後慮,亦不為輕。攸
 今之國,適足長異同之論,以損仁慈之美耳。而令天下窺陛下有不崇親親之情,臣竊為陛下不取也。若以妃后外親,任以朝政,則有王氏傾漢之權,呂產專朝之禍。若以同姓至親,則有吳楚七國逆亂之殃。歷觀古今,茍事輕重,所在無不為害也。不可事事曲設疑防,慮方來之患者也。唯當任正道而求忠良。若以智計猜物,雖親見疑,至於疏遠者亦何能自保乎!人懷危懼,非為安之理。此最有國有家者之深忌也。愚以為太子太保缺,宜留攸居之,與太尉汝南王亮、衛將軍楊珧共為保傅,幹理朝事。三人齊位,足相持正,進有輔納廣義之益,退無
 偏重相傾之勢。令陛下有篤親親之恩,使攸蒙仁覆之惠。臣同國休戚,義在盡言,心之所見,不能默已。私慕魯女存國之志,敢陳愚見,觸犯天威。欲陛下事每盡善,冀萬分之助。臣而不言,誰當言者。」帝不納。



 太熙初,遷司徒。惠帝即位,加侍中,又京陵置士官,如睢陵比。及誅楊駿,崇重舊臣,乃加渾兵。渾以司徒文官,主史不持兵,持兵乃吏屬絳衣。自以偶因時寵,權得持兵,非是舊典,皆令皂服。論者美其謙而識體。



 楚王瑋將害汝南王亮等也。公孫宏說瑋曰:「昔宣帝廢曹爽,引太尉蔣濟參乘,以增威重。大王今舉非常事,宜得宿望,鎮厭眾心。司徒王渾
 宿有威名,為三軍所信服,可請同乘,使物情有憑也。」瑋從之。渾辭疾歸第,以家兵千餘人閉門距瑋。瑋不敢逼。俄而瑋以矯詔伏誅,渾乃率兵赴官。帝嘗訪渾元會問郡國計吏方俗之宜,渾奏曰:「陛下欽明聖哲,光于遠近,明詔沖虛,詢及芻蕘,斯乃周文疇咨之求,仲尼不恥下問也。舊三朝元會前計吏詣軒下,侍中讀詔,計吏跪受。臣以詔文相承已久,無他新聲,非陛下留心方國之意也。可令中書指宣明詔,問方土異同,賢才秀異,風俗好尚,農桑本務,刑獄得無冤濫,守長得無侵虐。其勤心政化興利除害者,授以紙筆,盡意陳聞。以明聖指垂心四
 遠,不復因循常辭。且察其答對文義,以觀計吏人才之實。又先帝時,正會後東堂見征鎮長史司馬、諸王國卿、諸州別駕。今若不能別見,可前詣軒下,使侍中宣問,以審察方國,於事為便。」帝然之。又詔渾錄尚書事。



 渾所歷之職,前後著稱,及居台輔,聲望日減。元康七年薨,時年七十五,謚曰元。長子尚早亡,次子濟嗣。



 濟字武子。少有逸才,風姿英爽,氣蓋一時,好弓馬,勇力絕人,善《易》及《莊》、《老》,文詞俊茂,伎藝過人,有名當世,與姊夫和嶠及裴楷齊名。尚常山公主。年二十,起家拜中書郎,以母憂去官。起為驍騎將軍,累遷侍中,與侍中孔恂、
 王恂、楊濟同列,為一時秀彥。武帝嘗會公卿籓牧於式乾殿,顧濟、恂而謂諸公曰:「朕左右可謂恂恂濟濟矣!」每侍見,未嘗不諮論人物及萬機得失。濟善於清言,修飾辭令,諷議將順,朝臣莫能尚焉。帝益親貴之。仕進雖速,論者不以主婿之故,咸謂才能致之。然外雖弘雅,而內多忌刻,好以言傷物,儕類以此少之。以其父之故,每排王濬,時議譏焉。



 齊王攸當之藩,濟既陳請,又累使公主與甄德妻長廣公主俱入,稽顙泣請帝留攸。帝怒謂侍中王戎曰:「兄弟至親,今出齊王,自是朕家事,而甄德、王濟連遣婦來生哭人!」以忤旨,左遷國子祭酒,常侍如故。
 數年,入為侍中。時渾為僕射,主者處事或不當,濟性峻厲,明法繩之。素與從兄佑不平,佑黨頗謂濟不能顧其父,由是長同異之言。出為河南尹,未拜,坐鞭王官吏免官。而王佑始見委任。而濟遂被斥外,於是乃移第北芒山下。



 性豪侈,麗服玉食。時洛京地甚貴,濟買地為馬埒,編錢滿之,時人謂為「金溝」。王愷以帝舅奢豪,有牛名「八百里駁」,常瑩其蹄角。濟請以錢千萬與牛對射而賭之。愷亦自恃其能,令濟先射。一發破的,因據胡床,叱左右速探牛心來,須臾而至,一割便去。和嶠性至儉,家有好李,帝求之,不過數十。濟候其上直,率少年詣園,共啖畢,
 伐樹而去。帝嘗幸其宅,供饌甚豐,悉貯琉璃器中。蒸肫甚美,帝問其故,答曰:「以人乳蒸之。」帝色甚不平,食未畢而去。



 濟善解馬性,嘗乘一馬,著連乾鄣泥,前有水,終不肯渡。濟云:「此必是惜鄣泥。」使人解去,便渡。故杜預謂濟有馬癖。



 帝嘗謂和嶠曰:「我將罵濟而後官爵之,何如?」嶠曰:「濟俊爽,恐不可屈。」帝因召濟,切讓之,既而曰:「知愧不?」濟答曰:「尺布斗粟之謠,常為陛下恥之。他人能令親疏,臣不能使親親,以此愧陛下耳。」帝默然。



 帝嘗與濟弈棋,而孫皓在側,謂皓曰:「何以好剝人面皮?」皓曰:「見無禮於君者則剝之。」濟時伸腳局下,而皓譏焉。



 尋使白衣領太
 僕。年四十六,先渾卒,追贈驃騎將軍。及其將葬,時賢無不畢至。孫楚雅敬濟,而後來,哭之甚悲,賓客莫不垂涕。哭畢,向靈床曰:「卿常好我作驢鳴,我為卿作之。」體似聲真,賓客皆笑。楚顧曰:「諸君不死,而令王濟死乎!」



 初,濟尚主,主兩目失明,而妒忌尤甚,然終無子,有庶子二人。卓字文宣,嗣渾爵,拜給事中。次聿,字茂宣,襲公主封敏陽侯。濟二弟,澄字道深,汶字茂深,皆辯慧有才藻,並歷清顯。



 王濬,字士治,弘農湖人也。家世二千石。濬博墳典,美
 姿貌,不修名行,不為鄉曲所稱。晚乃變節,疏通亮達,恢廓有大志。嘗起宅,開門前路廣數十步。人或謂之何太過,濬曰:「吾欲使容長戟幡旗。」眾咸笑之,濬曰:「陳勝有言,燕雀安知鴻鵠之志。」州郡辟河東從事。守令有不廉潔者,皆望風自引而去。刺史燕國徐邈有女才淑,擇夫未嫁。邈乃大會佐吏,令女於內觀之。女指濬告母,邈遂妻之。後參征南軍事,羊祜深知待之。祜兄子暨白祜:「濬為人志太,奢侈不節,不可專任,宜有以裁之。」祜曰:「濬有大才,將欲濟其所欲,必可用也。」轉車騎從事中郎,識者謂祜可謂能舉善焉。



 除巴郡太守。郡邊吳境,兵士苦役,生
 男多不養。濬乃嚴其科條,寬其徭課,其產育者皆與休復,所全活者數千人。轉廣漢太守,垂惠布政,百姓賴之。濬夜夢懸三刀於臥屋梁上,須臾又益一刀,濬警覺,意甚惡之。主簿李毅再拜賀曰:「三刀為州字,又益一者,明府其臨益州乎?」及賊張弘殺益州刺史皇甫晏,果遷濬為益州刺史。濬設方略,悉誅弘等,以勳封關內侯。懷輯殊俗,待以威信,蠻夷徼外,多來歸降。徵拜右衛將軍,除大司農。車騎將軍羊祜雅知濬有奇略,乃密表留濬,於是重拜益州刺史。



 武帝謀伐吳,詔濬修舟艦。濬乃作大船連舫,方百二十步,受二千餘人。以木為城,起樓櫓,開
 四出門,其上皆得馳馬來往。又畫鷁首怪獸於船首,以懼江神。舟楫之盛,自古未有。濬造船於蜀,其木柿蔽江而下。吳建平太守吾彥取流柿以呈孫皓曰:「晉必有攻吳之計,宜增建平兵。建平不下,終不敢渡。」皓不從。尋以謠言拜濬為龍驤將軍、監梁益諸軍事。語在《羊祜傳》。



 時朝議咸諫伐吳,濬乃上疏曰:「臣數參訪吳楚同異,孫皓荒淫凶逆,荊揚賢愚無不嗟怨。且觀時運,宜速征伐。若今不伐,天變難預。令皓卒死,更立賢主,文武各得其所,則強敵也。臣作船七年,日有朽敗,又臣年已七十,死亡無日。三者一乖,則難圖也,誠願陛下無失事機。」帝深納
 焉。賈充、荀勖陳諫以為不可,唯張華固勸。又杜預表請,帝乃發詔,分命諸方節度。濬於是統兵。先在巴郡之所全育者,皆堪徭役供軍,其父母戒之曰:「王府君生爾,爾必勉之,無愛死也!」



 太康元年正月,濬發自成都,率巴東監軍、廣武將軍唐彬攻吳丹楊,剋之,擒其丹楊監盛紀。吳人於江險磧要害之處,並以鐵鎖橫截之,又作鐵錐長丈餘,暗置江中,以逆距船。先是,羊祜獲吳間諜,具知情狀。濬乃作大筏數十,亦方百餘步,縛草為人,被甲持杖,令善水者以筏先行,筏遇鐵錐,錐輒著筏去。又作火炬,長十餘丈,大數十圍,灌以麻油,在船前,遇鎖,然炬燒
 之,須臾,融液斷絕,於是船無所礙。二月庚申,剋吳西陵,獲其鎮南將軍留憲、征南將軍成據、宜都太守虞忠。壬戌,剋荊門、夷道二城,獲監軍陸晏。乙丑,剋樂鄉,獲水軍督陸景。平西將軍施洪等來降。乙亥,詔進濬為平東將軍、假節、都督益梁諸軍事。



 濬自發蜀,兵不血刃,攻無堅城,夏口、武昌,無相支抗。於是順流鼓棹,徑造三山。皓遣游擊將軍張象率舟軍萬人禦濬,象軍望旗而降。皓聞濬軍旌旗器甲,屬天滿江,威勢甚盛,莫不破膽。用光祿薛瑩、中書令胡沖計,送降文於濬曰:「吳郡孫皓叩頭死罪。昔漢室失御,九州幅裂,先人因時略有江南,遂阻
 山河,與魏乖隔。大晉龍興,德覆四海,闇劣偷安,未喻天命。至於今者,猥煩六軍,衡蓋露次,還臨江渚。舉國震惶,假息漏刻,敢緣天朝,含弘光大。謹遣私署太常張夔等奉所佩璽綬,委質請命。」壬寅,濬入于石頭。皓乃備亡國之禮,素車白馬,肉袒面縛,銜璧牽羊,大夫衰服,士輿櫬,率其偽太子瑾、瑾弟魯王虔等二十一人,造于壘門。濬躬解其縛,受璧焚櫬,送于京師。收其圖籍,封其府庫,軍無私焉。帝遣使者犒濬軍。



 初,詔書使濬下建平,受杜預節度,至秣陵,受王渾節度。預至江陵,謂諸將帥曰:「若濬得下建平,則順流長驅,威名已著,不宜令受制於我。若
 不能剋,則無緣得施節度。」濬至西陵,預與之書曰:『足下既摧其西籓,便當徑取秣陵,討累世之逋寇,釋吳人於塗炭。自江入淮,逾于泗汴,溯河而上,振旅還都,亦曠世一事也。」濬大悅,表呈預書。及濬將至秣陵,王渾遣信要令暫過論事,濬舉帆直指,報曰:「風利,不得泊也。」王渾久破皓中軍,斬張悌等,頓兵不敢進。而濬乘勝納降,渾恥而且忿,乃表濬違詔不受節度,誣罪狀之。有司遂按濬檻車徵,帝弗許,詔讓濬曰:「伐國事重,宜令有一。前詔使將軍受安車將軍渾節度,渾思謀深重,案甲以待將軍。云何徑前,不從渾命,違制昧利,甚失大義。將軍功勛,簡
 在朕心,當率由詔書,崇成王法,而於事終恃功肆意,朕將何以令天下?」濬上書自理曰:



 臣前被庚戌詔書曰:「軍人乘勝,猛氣盆壯,便當順流長騖,直造秣陵。」臣被詔之日,即便東下。又前被詔書云「太尉賈充總統諸方,自鎮東大將軍伷及渾、濬、彬等皆受充節度」,無令臣別受渾節度之文。



 臣自連巴丘,所向風靡,知孫皓窮踧,勢無所至。十四日至牛渚,去秣陵二百里,宿設部分,為攻取節度。前至三山,見渾軍在北岸,遣書與臣,可暫來過,共有所議,亦不語臣當受節度之意。臣水軍風發,乘勢造賊城,加宿設部分行有次第,無緣得於長流之中回船過
 渾,令首尾斷絕。須臾之間,皓遣使歸命。臣即報渾書,並寫皓箋,具以示渾,使速來,當於石頭相待。軍以日中至秣陵,暮乃被渾所下當受節度之符,欲令臣明十六日悉將所領,還圍石頭,備皓越逸。又索蜀兵及鎮南諸軍人名定見。臣以為皓已來首都亭,無緣共合空圍。又兵人定見,不可倉卒,皆非當今之急,不可承用。中詔謂臣忽棄明制,專擅自由。伏讀嚴詔,驚怖悚心慄,不知軀命當所投厝。豈惟老臣獨懷戰灼,三軍上下咸盡喪氣。臣受國恩,任重事大,常恐託付不效,孤負聖朝,故投身死地,轉戰萬里,被蒙寬恕之恩,得從臨履之宜。是以憑賴威
 靈,幸而能濟,皆是陛下神策廟算。臣承指授,效鷹犬之用耳,有何勛勞而恃功肆意,寧敢昧利而違聖詔。



 臣以十五日至秣陵,而詔書以十六日起洛陽,其間懸闊,不相赴接,則臣之罪責宜蒙察恕。假令孫皓猶有螳螂舉斧之勢,而臣輕軍單入,有所虧喪,罪之可也。臣所統八萬餘人,乘勝席卷。皓以眾叛親離,無復羽翼,匹夫獨立,不能庇其妻子,雀鼠貪生,茍乞一活耳。而江北諸軍不知其虛實,不早縛取,自為小誤。臣至便得,更見怨恚,並云守賊百日,而令他人得之,言語噂沓,不可聽聞。



 案《春秋》之義,大夫出疆,由有專輒。臣雖愚蠢,以為事君之道,
 唯當竭節盡忠,奮不顧身,量力受任,臨事制宜,茍利社稷,死生以之。若其顧護嫌疑,以避咎責,此是人臣不忠之利,實非明主社稷之福也。臣不自料,忘其鄙劣,披布丹心,輸寫肝腦,欲竭股肱之力,加之以忠貞,庶必掃除兇逆,清一宇宙,願令聖世與唐虞比隆。陛下粗察臣之愚款,而識其欲自效之誠,是以授臣以方牧之任,委臣以征討之事。雖燕主之信樂毅,漢祖之任蕭何,無以加焉。受恩深重,死且不報,而以頑疏,舉錯失宜。陛下弘恩,財加切讓,惶怖怔營,無地自厝,願陛下明臣赤心而已。



 渾又騰周浚書,云濬軍得吳寶物。濬復表曰:



 被壬戌詔
 書,下安東將所上揚州刺史周浚書,謂臣諸軍得孫皓寶物,又謂牙門將李高放火燒皓偽宮。輒公文上尚書,具列本末。又聞渾案陷上臣。臣受性愚忠,行事舉動,信心而前,期於不負神明而已。秣陵之事,皆如前所表,而惡直醜正,實繁有徒,欲構南箕,成此貝錦,公於聖世,反白為黑。



 夫佞邪害國,自古而然。故無極破楚,宰嚭滅吳,及至石顯,傾亂漢朝,皆載在典籍,為世所戒。昔樂毅伐齊,下城七十,而卒被讒間,脫身出奔。樂羊既反,謗書盈篋。況臣頑疏,能免讒慝之口!然所望全其首領者,實賴陛下聖哲欽明,使浸潤之譖不得行焉。然臣孤根獨
 立,朝無黨援,久棄遐外,人道斷絕,而結恨彊宗,取怨豪族。以累卵之身,處雷霆之衝;繭栗之質,當豺狼之路,其見吞噬,豈抗脣齒!



 夫犯上干主,其罪可救,乖忤貴臣,則禍在不測。故朱雲折檻,嬰逆鱗之怒,慶忌救之,成帝不問。望之、周堪違忤石顯,雖闔朝嗟嘆,而死不旋踵。此臣之所大怖也。今渾之支黨姻族內外,皆根據磐LF,並處世位。聞遣人在洛中,專共交構,盜言孔甘,疑惑觀聽。夫曾參之不殺人,亦以明矣,然三人傳之,其母投杼。今臣之信行,未若曾參之著;而讒構沸騰,非徒三夫之對,外內扇助,為二五之應。夫猛獸當途,麒麟恐懼,況臣脆弱,
 敢不悚心慄。



 偽吳君臣,今皆生在,便可驗問,以明虛實。前偽中郎將孔攄說,去二月武昌失守,水軍行至。皓案行石頭還,左右人皆跳刀大呼云:「要當為陛下一死戰決之。」皓意大喜,謂必能然,便盡出金寶,以賜與之。小人無狀,得便持走,皓懼,乃圖降首。降使適去,左右劫奪財物,略取妻妾,放火燒宮。皓逃身竄首,恐不脫死,臣至,遣參軍主者救斷其火耳。周浚以十六日前入皓宮,臣時遣記室吏往視書籍,浚使收縛。若有遺寶,則浚前得,不應移蹤後人,欲求茍免也。



 臣前在三山得浚書云:「皓散寶貨以賜將士,府庫略虛。」而今復言「金銀篋笥,動有萬計」,
 疑臣軍得之。言語反覆,無復本末。臣復與軍司張牧、汝南相馮紞等共入觀皓宮,乃無席可坐。後日又與牧等共視皓舟船,渾又先臣一日上其船,船上之物,皆渾所知見。臣之案行,皆出其後,若有寶貨,渾應得之。



 又臣將軍素嚴,兵人不得妄離部陣間。在秣陵諸軍。凡二十萬眾。



 臣軍先至,為土地之主。百姓之心,皆歸仰臣,臣切敕所領,秋毫不犯。諸有市易,皆有伍任證左,明從券契,有違犯者,凡斬十三人,皆吳人所知也。餘軍縱橫,詐稱臣軍,而臣軍類皆蜀人,幸以此自別耳,豈獨浚之將士皆是夷齊,而臣諸軍悉聚盜跖耶!時有八百餘人,緣石頭城
 劫取布帛。臣牙門將軍馬潛即收得二十餘人,並疏其督將姓名,移以付浚,使得自科結,而寂無反報,疑皆縱遣,絕其端緒也。



 又聞吳人言,前張悌戰時,所殺財有二千人,而渾、浚露布言以萬計。以吳剛子為主簿,而遣剛至洛,欲令剛增斬級之數。可具問孫皓及其諸臣,則知其定審。若信如所聞,浚等虛詐,尚欺陛下,豈惜於臣!云臣屯聚蜀人,不時送皓,欲有反狀。又恐動吳人,言臣皆當誅殺,取其妻子,冀其作亂,得騁私忿。謀反大逆,尚以見加,其餘謗沓,故其宜耳。



 渾案臣「瓶磬小器,蒙國厚恩,頻繁擢敘,遂過其任」。渾此言最信,內省慚懼。今年平吳,
 誠為大慶,於臣之身,更受咎累。既無孟側策馬之好,而令濟濟之朝有讒邪之人,虧穆穆之風,損皇代之美。由臣頑疏,使致於此,拜表流汗,言不識次。



 濬至京都,有司奏,濬表既不列前後所被七詔月日,又赦後違詔不受渾節度,大不敬,付廷尉科罪。詔曰:「濬前受詔徑造秣陵,後乃下受渾節度。詔書稽留,所下不至,便令與不受詔同責,未為經通。濬不即表上被渾宣詔,此可責也。濬有征伐之勞,不足以一眚掩之。」有司又奏,濬赦後燒賊船百三十五艘,輒敕付廷尉禁推。詔曰「勿推」。拜濬輔國大將軍,領步兵校尉。舊校唯五,置此營自濬始也。有司又
 奏,輔國依比,未為達官,不置司馬,不給官騎。詔依征鎮給五百大車,增兵五百人為輔國營,給親騎百人、官騎十人,置司馬。封為襄陽縣侯,邑萬戶。封子彝楊鄉亭侯,邑千五百戶,賜絹萬匹,又賜衣一襲、錢三十萬及食物。



 濬自以功大,而為渾父子及豪強所抑,屢為有司所奏,每進見,陳其攻伐之勞,及見枉之狀,或不勝忿憤,徑出不辭。帝每容恕之。益州護軍范通,濬之外親也。謂濬曰:「卿功則美矣,然恨所以居美者,未盡善也。」濬曰:「何謂也?」通曰:「卿旋旆之日,角巾私第,口不言平吳之事。若有問者,輒曰:『聖主之德,群帥之力,老夫何力之有焉!』如斯,顏
 老之不伐,龔遂之雅對,將何以過之。藺生所以屈廉頗,王渾能無愧乎!」濬曰:「吾始懼鄧艾之事,畏禍及,不得無言,亦不能遣諸胸中,是吾偏也。」時人咸以濬功重報輕,博士秦秀、太子洗馬孟康、前溫令李密等並表訟濬之屈。帝乃遷濬鎮軍大將軍,加散騎常侍,領後軍將軍。王渾詣濬,濬嚴設備衛,然後見之,其相猜防如此。



 濬平吳之後,以勳高位重,不復素業自居,乃玉食錦服,縱奢侈以自逸。其有辟引,多是蜀人,示不遺故舊也。後又轉濬撫軍大將軍、開府儀同三司,加特進,散騎常侍、後軍將軍如故。太康六年卒,時年八十,謚曰武。葬柏谷山,大營
 塋域,葬垣周四十五里,面別開一門,松柏茂盛。子矩嗣。



 矩弟暢,散騎郎。暢子粹,太康十年,武帝詔粹尚潁川公主,仕至魏郡太守。



 濬有二孫,過江不見齒錄。安西將軍恆溫鎮江陵,表言之曰:「臣聞崇德賞功,為政之所先;興滅繼絕,百王之所務。故德參時雍,則奕世承祀;功烈一代,則永錫祚胤。案故撫軍王濬歷職內外,任兼文武,料敵制勝,明勇獨斷,義存社稷之利,不顧專輒之罪。荷戈長鶩,席卷萬里,僭號之吳,面縛象魏,今皇澤被於九州,玄風洽於區外,襄陽之封,廢而莫續;恩寵之號,墜於近嗣。遐邇酸懷,臣竊悼之。濬今有二孫,年出六十,室如懸
 磬,糊口江濱,四節蒸嘗,菜羹不給。昔漢高定業,求樂毅之嗣;世祖旌賢,建葛亮之胤。夫效忠異代,立功異國,尚通天下之善,使不泯棄,況濬建元勳於當年,著喜慶於身後,靈基託根於南垂,皇祚中興於江左,舊物克彰,神器重耀,豈不由伊人之功力也哉!誠宜加恩,少垂矜憫,追錄舊勳,纂錫茅土。則聖朝之恩,宣暢於上,忠臣之志,不墜于地矣。」卒不見省。



 唐彬,字儒宗,魯國鄒人也。父臺,太山太守。彬有經國大度,而不拘行檢。少便弓馬,好遊獵,身長八尺,走及奔鹿,
 強力兼人。晚乃敦悅經史,尤明《易經》,隨師受業,還家教授,恒數百人。初為郡門下掾,轉主簿。刺史王沈集諸參佐,盛論距吳之策,以問九郡吏。彬與譙郡主張惲俱陳吳有可兼之勢,沈善其對。又使彬難言吳未可伐者,而辭理皆屈。還遷功曹,舉孝廉,州辟主簿,累遷別駕。



 彬忠肅公亮,盡規匡救,不顯諫以自彰,又奉使詣相府計事,于時僚佐皆當世英彥,見彬莫不欽悅,稱之於文帝,薦為掾屬。帝以問其參軍孔顥,顥忌其能,良久不答。陳騫在坐,斂板而稱曰:「彬之為人,勝騫甚遠。」帝笑曰:「但能如卿,固未易得,何論於勝。」因辟彬為鎧曹屬。帝問曰:「卿何
 以致辟?」對曰:「修業陋巷,觀古人之遺迹,言滿天下無口過,行滿天下無怨惡。」帝顧四坐曰:「名不虛行。」他日,謂孔顥曰:「近見唐彬,卿受蔽賢之責矣。」



 初,鄧艾之誅也,文帝以艾久在隴右,素得士心,一旦夷滅,恐邊情搔動,使彬密察之。彬還,白帝曰:「鄧艾忌克詭狹,矜能負才,順從者謂為見事,直言者謂之觸迕。雖長史司馬,參佐牙門,答對失指,輒見罵辱。處身無禮,大失人心。又好施行事役,數勞眾力。隴右甚患苦之,喜聞其禍,不肯為用。今諸軍已至,足以鎮壓內外,願無以為慮。」



 俄除尚書水部郎。泰始初,賜爵關內侯。出補鄴令,彬道德齊禮,期月化成。遷
 弋陽太守,明設禁防,百姓安之。以母喪去官。益州東接吳寇,監軍位缺,朝議用武陵太守楊宗及彬。武帝以問散騎常侍文立,立曰:「宗、彬俱不可失。然彬多財欲,而宗好酒,惟陛下裁之。」帝曰:「財欲可足,酒者難改。」遂用彬。尋又詔彬監巴東諸軍事,加廣武將軍。上征吳之策,甚合帝意。



 後與王濬共伐吳,彬屯據衝要,為眾軍前驅。每設疑兵,應機制勝,陷西陵、樂鄉,多所擒獲。自巴陵、沔口以東,諸賊所聚,莫不震懼,倒戈肉袒。彬知賊寇已殄,孫皓將降,未至建鄴二百里,稱疾遲留,以示不競。果有先到者爭物,後到者爭功,于時有識莫不高彬此舉。吳平,詔
 曰:「廣武將軍唐彬受任方隅,東禦吳寇,南監蠻越,撫寧疆埸,有綏禦之績。又每慷慨,志在立功。頃者征討,扶疾奉命,首啟戎行,獻俘授馘,勛效顯著。其以彬為右將軍、都督巴東諸軍事。」徵拜翊軍校尉,改封上庸縣侯,食邑六千戶,賜絹六千匹。朝有疑議,每參預焉。



 北虜侵掠北平,以彬為使持節、監幽州諸軍事、領護烏丸校尉、右將軍。彬既至鎮,訓卒利兵,廣農重稼,震威耀武,宣喻國命,示以恩信。於是鮮卑二部大莫廆、擿何等並遣侍子入貢。兼脩學校,誨誘無倦,仁惠廣被。遂開拓舊境,卻地千里。復秦長城塞,自溫城洎于碣石,綿亙山谷且三千里,
 分軍屯守,烽堠相望。由是邊境獲安,無犬吠之警,自漢魏征鎮莫之比焉。鮮卑諸種畏懼,遂殺大莫廆。彬欲討之,恐列上俟報,虜必逃散,乃發幽冀車牛。參軍許祗密奏之。詔遣御史檻車徵彬付廷尉,以事直見釋。百姓追慕彬功德,生為立碑作頌。



 彬初受學於東海閻德,門徒甚多,獨目彬有廊廟才。及彬官成,而德已卒,乃為之立碑。



 元康初,拜使持節、前將軍、領西戎校尉、雍州刺史。下教曰:「此州名都,士人林藪。處士皇甫申叔、嚴舒龍、姜茂時、梁子遠等,並志節清妙,履行高潔。踐境望風,虛心飢渴,思加延致,待以不臣之典。幅巾相見,論道而已,豈以吏
 職,屈染高規。郡國備禮發遣,以副於邑之望。」於是四人皆到,彬敬而待之。元康四年卒官,時年六十,謚曰襄,賜絹二百匹,錢二十萬。長子嗣,官至廣陵太守。少子岐,征虜司馬。



 史臣曰:孫氏負江山之阻隔,恃牛斗之妖氛,奄有水鄉,抗衡上國。二王屬當戎旅,受律遄征,渾既獻捷橫江,濬亦剋清建鄴。于時討吳之役,將帥雖多,定吳之功,此焉為最。向使弘范父之不伐,慕陽夏之推功,上稟廟堂,下憑將士。豈非茂勛茂德,善始善終者歟!此而不存,彼焉是務。或矜功負氣,或恃勢驕陵,競構南箕,成茲貝錦。遂
 乃喧黷宸扆,斁亂彞倫,既為戒於功臣,亦致譏於清論,豈不惜哉!王濟遂驕父之褊心,乖爭子之明義,俊材雖多,亦奚以為也。唐彬畏避交爭,屬疾遲留,退讓之風,賢於渾濬遠矣。傳云「不拘行檢」,安得長者之行哉!



 贊曰:二王總戎,淮海攸同。渾既害善,濬亦矜功。武子豪桀,夙參朝列。逞欲牛心,紆情馬埒。儒宗知退,避名全節。



\end{pinyinscope}