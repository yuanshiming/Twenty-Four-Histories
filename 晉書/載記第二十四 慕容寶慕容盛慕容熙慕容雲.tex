\article{載記第二十四 慕容寶慕容盛慕容熙慕容雲}

\begin{pinyinscope}

 慕容
 寶慕容盛慕容熙慕容雲



 慕容寶,字道祐,垂之第四子也。少輕果無志操,好人佞己。苻堅時為太子洗馬、萬年令。堅淮肥之役,以寶為陵江將軍。及為太子,砥礪自脩,敦崇儒學,工談論,善屬文,曲事垂左右小臣,以求美譽。垂之朝士翕然稱之,垂亦以為克保家業,甚賢之。



 垂死,其年寶嗣偽位,大赦境內,改元為永康。以其太尉庫辱官偉為太師、左光祿大夫,
 段崇為太保,其餘拜授各有差。遵垂遺令,校閱戶口,罷諸軍營分屬郡縣,定士族舊籍,明其官儀,而法峻政嚴,上下離德,百姓思亂者十室而九焉。



 初,垂以寶塚嗣未建,每憂之。寶庶子清河公會多材藝,有雄略,垂深奇之。及寶之北伐,使會代攝宮事,總錄、禮遇一同太子,所以見定旨也。垂之伐魏,以龍城舊都,宗廟所在,復使會鎮幽州,委以東北之重,高選僚屬以崇威望。臨死顧命,以會為寶嗣,而寶寵愛少子濮陽公策,意不在會。寶庶長子長樂公盛自以同生年長,恥會先之,乃盛稱策宜為儲貳,而非毀會焉。寶大悅,乃訪其趙王麟、高陽王隆,麟
 等咸希旨贊成之。寶遂與麟等定計,立策母段氏為皇后,策為皇太子,盛、會進爵為王。策字道符,年十一,美姿貌,而蠢弱不慧。



 魏伐並州,驃騎農逆戰,敗績,還于晉陽,司馬慕輿嵩閉門距之。農率騎數千奔歸中山,行及潞川,為魏追軍所及,餘騎盡沒,單馬遁還。寶引群臣于東堂議之。中山尹苻謨曰:「魏軍強盛,千里轉鬥,乘勝而來,勇氣兼倍,若逸騎平原,形勢彌盛,殆難為敵,宜度險距之。」中書令晆邃曰:「魏軍多騎,師行剽銳,馬上齎糧,不過旬日。宜令郡縣聚千家為一堡,深溝高壘,清野待之。至無所掠,資食無出,不過六旬,自然窮退。」尚書封懿曰:「
 今魏師十萬,天下之勍敵也。百姓雖欲營聚,不足自固,是則聚糧集兵以資強寇,且動眾心,示之以弱,阻關距戰,計之上也。」慕容麟曰:「魏今乘勝氣銳,其鋒不可當,宜自完守設備,待其弊而乘之。」於是脩城積粟,為持久之備。



 魏攻中山不剋,進據博陵魯口,諸將望風奔退,郡縣悉降于魏,寶聞魏有內難,乃盡眾出距,步卒十二萬,騎三萬七千,次于曲陽柏肆。魏軍進至新梁。寶憚魏師之銳,乃遣征北隆夜襲魏軍,敗績而還。魏軍方軌而至,對營相持,上下兇懼,三軍奪氣。農、麟勸寶還中山,乃引歸。魏軍追擊之,寶、農等棄大軍,率騎二萬奔還。時大風雪,
 凍死者相枕於道。寶恐為魏軍所及,命去袍杖戎器,寸刃無返。



 魏軍進攻中山,屯于芳林園。其夜尚書慕容皓謀殺寶,立慕容麟。皓妻兄蘇泥告之,寶使慕容隆收皓,皓與同謀數十人斬關奔魏。麟懼不自安,以兵劫左衛將軍、北地王精,謀率禁旅弒寶。精以義距之,麟怒,殺精,出奔丁零。



 初,寶聞魏之來伐也,使慕容會率幽、並之眾赴中山,麟既叛,寶恐其逆奪會軍,將遣兵迎之。麟侍郎段平子自丁零奔還,說麟招集丁零,軍眾甚盛,謀襲會軍,東據龍城。寶與其太子策及農、隆等萬餘騎迎會于薊,以開封公慕容詳守中山。會傾身誘納,繕甲厲兵,步
 騎二萬,列陣而進,迎寶薊南。寶分其兵給農,隆,遣西河公庫辱官驥率眾三千助守中山。會以策為太子,有恨色。寶以告農、隆,俱曰:「會一年少,專任方事,習驕所致,豈有他也。臣當以禮責之。」幽平之士皆懷會威德,不樂去之,咸請曰:「清河王天資神武,權略過人,臣等與之誓同生死,感王恩澤,皆勇氣自倍。願陛下與皇太子、諸王止駕薊宮,使王統臣等進解京師之圍,然後奉迎車駕。」寶左右皆害其勇略,譖而不許,眾咸有怨言。左右勸寶殺會,侍御史仇尼歸聞而告會曰:「左右密謀如是,主上將從之。大王所恃唯父母也,父已異圖;所杖者兵也,兵已
 去手,進退路窮,恐無自全之理。盍誅二王,廢太子,大王自處東宮,兼領將相,以匡社稷。」會不從。寶謂農、隆曰:「觀會為變,事當必然,宜早殺之。不爾,恐成大禍。」農曰:「寇賊內侮,中州紛亂,會鎮撫舊都,安眾寧境,及京師有難,萬里星赴,威名之重,可以振服戎狄。又逆跡未彰,宜且隱忍。今社稷之危若綴旒然,復內相誅戮,有損威望。」寶曰:「會逆心已成,而王等仁慈,不欲去之,恐一旦釁發,必先害諸父,然後及吾。事敗之後,當思朕言。」農等固諫,乃止。會聞之彌懼,奔于廣都黃榆谷。會遣仇尼歸等率壯士二千餘人分襲農、隆,隆是夜見殺,農中重創。既而會歸
 于寶,寶意在誅會,誘而安之,潛使左衛慕輿騰斬會,不能傷。會復奔其眾,於是勒兵攻寶。寶率數百騎馳如龍城,會率眾追之,遣使請誅左右佞臣,並求太子,寶弗許。會圍龍城,侍御郎高雲夜率敢死士百餘人襲會,敗之,眾悉逃散,單馬奔還中山,乃踰圍而入,為慕容詳所殺。



 詳僭稱尊號,置百官,改年號。荒酒奢淫,殺戮無度,誅其王公以下五百餘人,內外震局,莫敢忤視。城中大飢,公卿餓死者數十人。麟率丁零之眾入中山,斬詳及其親黨三百餘人,復僭稱尊號。中山飢甚,麟出據新市,與魏師戰于義臺,麟軍敗績。魏師遂人中山,麟乃奔鄴。



 慕容
 德遣侍郎李延勸寶南伐,寶大悅,慕容盛切諫,以為兵疲師老,魏新平中原,宜養兵觀釁,更俟他年。寶將從之。撫軍慕輿騰進曰:「今眾旅已集,宜乘新定之機以成進取之功。人可使由之,而難與圖始,惟當獨決聖慮,不足廣採異同,以沮亂軍議也。」寶曰:「吾計決矣,敢諫者斬!」寶發龍城,以慕輿騰為前軍大司馬,慕容農為中軍,寶為後軍,步騎三萬,次于乙連。長上段速骨、宋赤眉因眾軍之憚役也,殺司空、樂浪王宙,逼立高陽王崇。寶單騎奔農,仍引軍討速骨。眾咸憚征幸亂,投杖奔之。騰眾亦潰,寶、農馳還龍城。蘭汗潛與速骨通謀,速骨進師攻城,農
 為蘭汗所譎,潛出赴賊,為速骨所殺。眾皆奔散,寶與慕容盛、慕輿騰等南奔。蘭汗奉太子策承制,遣使迎寶,及于薊城。寶欲還北,盛等咸以汗之忠款虛實未明,今單馬而還,汗有貳志者,悔之無及。寶從之,乃自薊而南。至黎陽,聞慕容德稱制,懼而退。遣慕輿騰招集散兵于鉅鹿,慕容盛結豪桀于冀州,段儀、段溫收部曲于內黃,眾皆響會,剋期將集。會蘭汗遣左將軍蘇超迎寶,寶以汗垂之季舅,盛又汗之壻也,必謂忠款無貳,乃還至龍城。汗引寶入于外邸,弒之,時年四十四,在位三年,即隆安三年也。汗又殺其太子策及王公卿士百餘人。汗自稱
 大都督、大將軍、大單于、昌黎王。盛僭位,偽謚寶惠愍皇帝,廟號烈宗。



 皝之遷於龍城也,植松為社主。及秦滅燕,大風吹拔之。後數年,社處忽有桑二根生焉。先是,遼川無桑,及廆通于晉,求種江南,平州桑悉由吳來。廆終而垂以吳王中興,寶之將敗,大風又拔其一。



 盛字道運,寶之庶長子也。少沈敏,多謀略。苻堅誅慕容氏,盛潛奔于沖。及沖稱尊號,有自得之志,賞罰不均,政令不明。盛年十二,謂叔父柔曰:「今中山王智不先眾,才不出下,恩未施人,先自驕大,以盛觀之,鮮不覆敗。」俄而沖為段木延所殺,盛隨慕容永東如長子,謂柔曰:「今崎
 嶇於鋒刃之間,在疑忌之際,愚則為人所猜,智則危甚巢幕,當如鴻鵠高飛,一舉萬里,不可坐待罟網也。」於是與柔及弟會間行東歸于慕容垂。遇盜陜中,盛曰:「我六尺之軀,入水不溺,在火不焦,汝欲當吾鋒乎!試豎爾手中箭百步,我若中之,宜慎爾命,如其不中,當束身相授。」盜用豎箭,盛一發中之。盜曰:「郎貴人之子,故相試耳。」資而遣之。歲餘,永誅俊、垂之子孫,男女無遺。盛既至,垂問以西事,畫地成圖。垂笑曰:「昔魏武撫明帝之首,遂乃侯之,祖之愛孫,有自來矣。」於是封長樂公。驍勇剛毅,有伯父全之風烈。



 寶即偽位,進爵為王。寶自龍城南伐,盛留
 統後事,及段速骨作亂,馳出迎衛。寶幾為速骨所獲,賴盛以免。盛屢進奇策於寶,寶不能從,是以屢敗。寶既如龍城,盛留在後。寶為蘭汗所殺,盛馳進赴哀,將軍張真固諫以為不可,盛曰:「我今投命,告以哀窮。汗性愚近,必顧念婚姻,不忍害我。旬月之間,足展吾志。」遂人赴喪。汗妻乙氏泣涕請盛,汗亦哀之,遣其子穆迎盛,舍之宮內,親敬如舊。汗兄提、弟難勸汗殺盛,汗不從。慕容奇,汗之外孫也,汗亦宥之。奇入見盛,遂相與謀。盛遣奇起兵于外,眾至數千。汗遣蘭提討奇。提驕很淫荒,事汗無禮,盛因間之於汗曰:「奇,小兒也,未能辦此,必內有應之者。提
 素驕,不可委以大眾。」汗因發怒,收提誅之,遣其撫軍仇尼慕率眾討奇。汗兄弟見提之誅,莫不危懼,皆阻兵背汗,襲敗慕軍。汗大懼,遣其子穆率眾討之。穆謂汗曰:「慕容盛,我之仇也。奇今起逆,盛必應之。兼內有蕭墻之難,不宜養心腹之疾。」汗將誅盛,引見察之。盛妻以告,於是偽稱疾篤,不復出入,汗乃止。有李旱、衛雙、劉志、張豪、張真者,皆盛之舊暱,蘭穆引為腹心。旱等屢入見盛,潛結大謀。會穆討蘭難等斬之,大饗將士,汗、穆皆醉。盛夜因如廁,袒而踰墻,入于東宮,與李旱等誅穆,眾皆踴呼,進攻汗,斬之。汗二子魯公和、陳公楊分屯令支、白狼,遣李
 旱、張真襲誅之。於是內外怗然,士女咸悅,盛謙揖自卑,不稱尊號。其年,以長樂王稱制,赦其境內,改元曰建平。諸王降爵為公,文武各復舊位。



 初,慕容奇聚眾于建安,將討蘭汗,百姓翕然從之。汗遣兄子全討奇,奇擊滅之,進屯乙連。盛既誅汗,命奇罷兵,奇遂與丁零嚴生、烏丸王龍之阻兵叛盛,引軍至橫溝,去龍城十里。盛出兵擊敗之,執奇而還,斬龍、生等百餘人。盛於是僭即尊位,大赦殊死已下,追尊伯考獻莊太子全為獻莊皇帝,尊寶后段氏為皇太后,全妃丁氏為獻莊皇后,謚太子策為獻哀太子。盛幽州刺史慕容豪、尚書左僕射張通、昌黎
 尹張順謀叛,盛皆誅之。改年為長樂。有犯罪者,十日一自決之,無撾捶之罰,而獄情多實。



 高句驪王安遣使貢方物,有雀素身綠首,集于端門,棲翔東園,二旬而去,改東園為白雀園。



 盛聽詩歌及周公之事,顧謂群臣曰:「周公之輔成王,不能以至誠感上下,誅兄弟以杜流言,猶擅美於經傳,歌德於管弦。至如我之太宰桓王,承百王之季,主在可奪之年,二寇窺窬,難過往日,臨朝輔政,群情緝穆,經略外敷,闢境千里,以禮讓維宗親,德刑制群后,敦睦雍熙,時無二論。勛道之茂,豈可與周公同日而言乎!而燕詠闕而不論,盛德掩而不述,非所謂也。」乃命
 中書更為《燕頌》以述恪之功焉。又引中書令常忠、尚書陽璆、祕書監郎敷于東堂,問曰:「古來君子皆謂周公忠聖,豈不謬哉!」璆曰:「周公居攝政之重,而能達群臣之名,及流言之謗,致烈風以悟主,道契神靈,義光萬代,故累葉稱其高,後王無以奪其美。」盛曰:「常令以為何如?」忠曰:「昔武王疾篤,周公有請令之誠,流言之際,義感天地,楚撻伯禽以訓就王德。周公為臣之忠,聖達之美,《詩》《書》已來未之有也。」盛曰:「異哉二君之言!朕見周公之詐,未見其忠聖也。昔武王得九齡之夢,白文王,文王曰:「我百,爾九十,吾與爾三焉。」及文王之終,已驗武王之壽矣。武王
 之算未盡而求代其死,是非詐乎!若惑於天命,是不聖也。據攝天位而丹誠不見,致兄弟之間有干戈之事。夫文王之化,自近及遠,故曰刑于寡妻,至于兄弟。周公親違聖父之典而蹈嫌疑之蹤,戮罰同氣以逞私忿,何忠之有乎!但時無直筆之史,後儒承其謬談故也。」忠曰:「啟金縢而返風,亦足以明其不詐。遭二叔流言之變,而能大義滅親,終安宗國,復子明辟,輔成大業,以致太平,制禮作樂,流慶無窮,亦不可謂非至德也。」盛曰:「卿徒因成文而未原大理,朕今相為論之。昔周自后稷積德累仁,至于文、武。文、武以大聖應期,遂有天下。生靈仰其德,四
 海歸其仁。成王雖幼統洪業,而卜世脩長,加呂、召、毛、畢為之師傅。若無周公攝政,王道足以成也。周公無故以安危為己任,專臨朝之權,闕北面之禮。管、蔡忠存王室,以為周公代主非人臣之道,故言公將不利於孺子。周公當明大順之節,陳誠義以曉群疑,而乃阻兵都邑,擅行誅戮。不臣之罪彰于海內,方貽王《鴟鴞》之詩,歸非於主,是何謂乎!又周公舉事,稱告二公,二公足明周公之無罪而坐觀成王之疑,此則二公之心亦有猜於周公也。但以疏不間親,故寄言於管、蔡,可謂忠不見於當時,仁不及于兄弟。知群望之有歸,天命之不在己,然後返
 政成王,以為忠耳。大風拔木之徵,乃皇天祐存周道,不忘文、武之德,是以赦周公之始愆,欲成周室之大美。考周公之心,原周公之行,乃天下之罪人,何至德之謂也!周公復位,二公所以杜口不言其本心者,以明管、蔡之忠也。」



 又謂常忠曰:「伊尹、周公孰賢?」忠曰:「伊尹非有周公之親而功濟一代,太甲亂德,放於桐宮,思愆改善,然後復之。使主無怨言,臣無流謗,道存社稷,美溢來今,臣謂伊尹之勳有高周旦。」盛曰:「伊尹以舊臣之重,顯阿衡之任,太甲嗣位,君道未洽,不能竭忠輔導。而放黜桐宮,事同夷羿,何周公之可擬乎!」郎敷曰:「伊尹處人臣之位,不
 能匡制其君,恐成、湯之道墜而莫就,是以居之桐宮,與小人從事,使知稼穡之艱難,然後返之天位,此其忠也。」盛曰:「伊尹能廢而立之,何不能輔之以至於善乎?若太甲性同桀紂,則三載之間未應便成賢后,如其性本休明,義心易發,當務盡匡規之理以弼成君德,安有人臣幽主而據其位哉!且臣之事君,惟力是視,奈何挾智藏仁以成君惡!夫太甲之事,朕已鑒之矣。太甲,至賢之主也,以伊尹歷奉三朝,績無異稱,將失顯祖委授之功,故匿其日月之明,受伊尹之黜,所以濟其忠貞之美。夫非常之人,然後能立非常之事,非常人之所見也,亦猶太
 伯之三讓,人無德而稱焉。」敷曰:「太伯三以天下讓,至仲尼而後顯其至德。太甲受謗於天下,遭陛下乃申其美。」因而談宴賦詩,賜金帛各有差。



 遼西太守李郎在郡十年,威制境內,盛疑之,累徵不赴。以母在龍城,未敢顯叛,乃陰引魏軍,將為自安之計,因表請發兵以距寇。盛曰:「此必詐也。」召其使而詰之,果驗,盡滅其族,遣輔國將軍李旱率騎討之。師次建安,召旱旋師。朗聞其家被誅也,擁三千餘戶以自固。及聞旱中路而還,謂有內變,不復為備,留其子養守令支,躬迎魏師于北平。旱候知之,襲剋令支,遣廣威孟廣平率騎追朗,及於無終,斬之。初,盛
 之追旱還也,群臣莫知其故。旱既斬朗,盛謂群臣曰:「前以追旱還者,正為此耳。朗新為叛逆,必忌官威,一則鳩合同類,劫掠良善,二則亡竄山澤,未可卒平,故非意而還,以盈怠其志,卒然掩之,必剋之理也。」群臣皆曰:「非所及也。」



 李旱自遼西還,聞盛殺其將衛雙,懼,棄軍奔走。既而歸罪,復其爵位。盛謂侍中孫勍曰:「旱總三軍之任,荷專征之重,不能杖節死綏,無故逃亡,考之軍正,不赦之罪也。然當先帝之避難,眾情離貳,骨肉忘其親,股肱失忠節,旱以刑餘之體,效力盡命,忠款之至,精貫白日。朕故錄其忘身之功,免其丘山之罪耳。」



 盛去皇帝之號,稱
 庶人大王。



 魏襲幽州,執刺史盧溥而去。遣孟廣平援之,無及。



 盛率眾三萬伐高句驪,襲其新城、南蘇,皆剋之,散其積聚,徙其五千餘戶于遼西。



 盛引見百遼于東堂,考詳器藝,超拔者十有二人。命百司舉文武之士才堪佐世者各一人。立其子遼西公定為太子,大赦殊死已下。宴其群臣于新昌殿,盛曰:「諸卿各言其志,朕將覽之。」七兵尚書丁信年十五,盛之舅子也,進曰:「在上不驕,高而不危,臣之願也。」盛笑曰:「丁尚書年少,安得長者之言乎!」盛以威嚴馭下,驕暴少親,多所猜忌,故信言及之。



 盛討庫莫奚,大虜獲而還。左將軍慕容國與殿中將軍秦輿、
 段贊等謀率禁兵襲盛,事覺,誅之,死者五百餘人。前將軍、思悔侯段璣、輿子興、贊子泰等,因眾心動搖,夜於禁中鼓躁大呼。盛聞變,率左右出戰,眾皆披潰。俄而有一賊從闇中擊傷盛,遂輦升前殿,申約禁衛,召叔父河間公熙屬以後事。熙未至而盛死,時年二十九,在位三年。偽謚昭武皇帝,墓號興平陵,廟號中宗。



 盛幼而羈賤流漂,長則遭家多難,夷險安危,備嘗之矣。懲寶暗而不斷,遂峻機威刑,織芥之嫌,莫不裁之於未萌,防之於未兆。於是上下振局,人不自安,雖忠誠親戚亦皆離貳,舊臣靡不夷滅,安忍無親,所以卒於不免。是歲隆安五年也。



 熙字道文,垂之少子也。初封河間王。段速骨之難,諸王多被其害,熙素為高陽王崇所親愛,故得免焉。蘭汗之篡也,以熙為遼東公,備宗祀之義。盛初即位,降爵為公,拜都督中外諸軍事、驃騎大將軍、尚書左僕射,領中領軍。從征高句驪、契丹,皆勇冠諸將。盛曰:「叔父雄果英壯,有世祖之風,但弘略不如耳。」



 及盛死,其太后丁氏以國多難,宜立長君。群望皆在平原公元,而丁氏意在於熙,遂廢太子定,迎熙入宮。群臣勸進,熙以讓元,元固以讓熙,熙遂僭即尊位。誅其大臣段璣、秦興等,並夷三族。元以嫌疑賜死。元字道光,寶之第四子也。赦殊死已下,改元
 曰光始,改北燕臺為大單于臺,置左右輔,位次尚書。



 初,熙烝于丁氏,故為所立。及寵幸苻貴人,丁氏怨恚咒詛,與兄子七兵尚書信謀廢熙。熙聞之,大怒,逼丁氏令自殺,葬以后禮,誅丁信。



 熙狩于北原,石城令高和殺司隸校尉張顯,閉門距熙。熙率騎馳返,和眾皆投杖,熙入誅之。於是引見州郡及單于八部耆舊於東宮,問以疾苦。



 大築龍騰苑,廣袤十餘里,役徒二萬人。起景雲山於苑內,基廣五百步,峰高十七丈。又起逍遙宮、甘露殿,連房數百,觀閣相交。鑿天河渠,引水入宮。又為其昭儀苻氏鑿曲光海、清涼池。季夏盛暑,士卒不得休息,暍死者太
 半。熙游于城南,止大柳樹下,若有人呼曰:「大王且止。。」熙惡之,伐其樹,乃有蛇長丈餘,從樹中而出。



 立其貴嬪苻氏為皇后,赦殊死已下。



 熙北襲契丹,大破之。



 昭儀苻氏死,偽謚愍皇后。贈苻謨太宰,謚文獻公。二苻並美而艷,好微行游宴,熙弗之禁也。請謁必從,刑賞大政無不由之。初,昭儀有疾,龍城人王溫稱能療之,未幾而卒,熙忿其妄也,立於公車門支解溫而焚之。其后好游田,熙從之,北登白鹿山,東過青嶺,南臨滄海,百姓苦之,士卒為豺狼所害及凍死者五千餘人矣。會高句驪寇燕郡,殺略百餘人。熙伐高句驪,以苻氏從,為衝車地道以攻遼
 東。熙曰:「待刬平寇城,朕當與后乘輦而入,不聽將士先登。」於是城內嚴備,攻之不能下。會大雨雪,士卒多死,乃引歸。



 擬鄴之鳳陽門,作弘光門,累級三層。



 熙與苻氏襲契丹,憚其眾盛,將還,苻氏弗聽,遂棄輜重,輕襲高句驪,周行三千餘里,士馬疲凍,死者屬路。攻木底城,不剋而還。



 盡殺寶諸子。大城肥如及宿軍,以仇尼倪為鎮東大將軍、營州刺史,鎮宿軍,上庸公懿為鎮西將軍、幽州刺史,鎮令支;尚書劉木為鎮南大將軍、冀州刺史,鎮肥如。



 為苻氏起承華殿,高承光一倍,負土於北門,土與穀同價。典軍杜靜載棺詣闕,上書極諫。熙大怒,斬之。苻氏
 嘗季夏思凍魚膾,仲冬須生地黃,皆下有司切責,不得,加以大辟,其虐也如此。苻氏死,熙悲號躃踴,若喪考妣,擁其尸而撫之曰:「體已就冷,命遂斷矣!」於是僵仆氣絕,久而乃蘇。大斂既訖,復啟其棺而與交接。服斬縗,食粥。制百僚於宮內哭臨,令沙門素服。使有司案檢哭者,有淚以為忠孝,無則罪之,於是群臣震懼,莫不含辛以為淚焉。慕容隆妻張氏,熙之嫂也,美姿容,有巧思。熙將以為苻氏之殉,欲以罪殺之,乃毀其禭靴,中有弊氈,遂賜死。三女叩頭求哀,熙不許。制公卿已下至於百姓,率戶營墓,費殫府藏。下錮三泉,周輸數里,內則圖畫尚書八
 坐之象。熙曰:「善為之,朕將隨后入此陵。」識者以為不祥。其右僕射韋璆等並懼為殉,沐浴而待死焉。號苻氏墓曰征平陵。熙被髮徒跣,步從苻氏喪。轜車高大,毀北門而出。長老竊相謂曰:「慕容氏自毀其門,將不久也。」



 中衛將軍馮跋、左衛將軍張興,先皆坐事亡奔,以熙政之虐也,與跋從兄萬泥等二十二人結盟,推慕容雲為主,發尚方徒五千餘人閉門距守。中黃門趙洛生奔告之,熙曰:「此鼠盜耳,朕還當誅之。」乃收髮貫甲,馳還赴難。夜至龍城,攻北門不剋,遂敗,走入龍騰宛,微服隱於林中,為人所執,雲得而弒之,及其諸子同殯城北。時年二十三,
 在位六年。雲葬之于苻氏墓,偽謚昭文皇帝。



 垂以孝武帝太元八年僭立,至熙四世,凡二十四年,以安帝義熙三年滅。初,童謠曰:「一束槁,兩頭然,禿頭小兒來滅燕。」槁字上有草,下有禾,兩頭然則禾草俱盡而成高字。雲父名拔,小字禿頭,三子,而雲季也。熙竟為雲所滅,如謠言焉。



 慕容雲,字子雨,寶之養子也。祖父和,高句驪之支庶,自云高陽氏之苗裔,故以高為氏焉。雲沈深有局量,厚重希言,時人咸以為愚,唯馮跋奇其志度而友之。寶之為太子,雲以武藝給事侍東宮,拜侍御郎,襲敗慕容會軍。
 寶子之,賜姓慕容氏,封夕陽公。



 熙之葬苻氏也,馮跋詣雲,告之以謀。雲懼曰:「吾嬰疾歷年,卿等所知,願更圖之。」跋逼曰:「慕容氏世衰,河間虐暴,惑妖淫之女而逆亂天常,百姓不堪其害,思亂者十室九焉,此天亡之時也。公自高氏名家,何能為他養子!機運難邀,千歲一時,公焉得辭也!」扶之而出。雲曰:「吾疾苦日久,廢絕世務。卿今興建大事,謬見推逼。所以徘徊,非為身也,實惟否德不足以濟元元故耳。」跋等彊之,雲遂即天王位,復姓高氏,大赦境內殊死以下,改元曰正始,國號大燕。署馮跋侍中、都督中外諸軍事、征北大將軍、開府儀同三司、錄尚書
 事、武邑公,封伯、子、男,鄉、亭侯者五十餘人,士卒賜穀帛有差。熙之群官,復其爵位。立妻李氏為天王后,子彭為太子。越騎校尉慕輿良謀叛,雲誅之。



 雲臨東堂,幸臣離班、桃仁懷劍執紙而入,稱有所啟,拔劍擊雲,雲以几距班,桃仁進而弒之。馮跋遷雲尸于東宮,偽謚惠懿皇帝。雲自以無功德而為豪桀所推,常內懷懼,故寵養壯士以為腹心。離班、桃仁等並專典禁衛,委之以爪牙之任,賞賜月至數千萬,衣食臥起皆與之同,終以此致敗云。



 史臣曰:四星東聚,金陵之氣已分;五馬南浮,玉塞之雄方擾。市朝屢改,艱虞靡息。慕容垂天資英傑,威震本朝,
 以雄略見猜而庇身寬政,永固受之而以禮,道明事之而畢力。然而隼質難羈,狼心自野。淮南失律,三甥之謀已構;河朔分麾,五木之祥云啟。斬飛龍而遐舉,踰石門而長邁,遂使翟氏景從,鄴師宵逸,收羅趙、魏,驅駕英雄。叩囊餘奇,摧五萬於河曲;浮船祕策,招七郡於黎陽。返遼陰之舊物,創中山之新社,類帝禋宗,僭擬斯備。夫以重耳歸晉,賴五臣之功;句踐紿吳,資五千之卒。惡有業殊二霸,眾微一旅,掎拔而傾山嶽,騰嘯而御風雲!雖衛人忘亡復傳於東國,任好餘裕伊媿於西鄰,信苻氏之姦回,非晉室之鯨鯢矣。



 寶以浮譽獲升,峻文御俗,蕭墻
 內憤,勍敵外陵,雖毒不被物而惡足自剿。盛則孝友冥符,文武不墜,韜光而夷仇賊,罪己而遜高危,翩翩然濁世之佳虜矣。熙乃地非奧主,舉因淫德。驪戎之態,取悅於匡床;玄妻之姿,見奇於鬒發。蕩輕舟於曲光之海,望朝涉於景雲之山,飾土木於驕心,窮怨嗟於蕞壤,宗祀夷滅,為馮氏之驅除焉。



 贊曰:戎狄憑陵,山川沸騰。天未悔禍,人非與能。疾走而捷,先鳴則興。道明烈烈,鞭笞豪傑。掃燕夷魏,釗屠永滅。大盜潛移,鴻名遂竊。寶心生亂,盛清家難。熙極驕淫,人懷憤惋。孽貽身咎,災無以逭。



\end{pinyinscope}