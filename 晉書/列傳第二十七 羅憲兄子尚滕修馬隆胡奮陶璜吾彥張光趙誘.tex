\article{列傳第二十七 羅憲兄子尚滕修馬隆胡奮陶璜吾彥張光趙誘}

\begin{pinyinscope}
羅憲
 \gezhu{
  兄子尚}
 滕修馬隆胡奮陶璜吾彥張光趙誘



 羅憲,字令則,襄陽人也。父蒙,蜀廣漢太守。憲年十三,能屬文,早知名。師事譙周,周門人稱為子貢。性方亮嚴整,待士無倦,輕財好施,不營產業。仕蜀為太子舍人、宣信校尉。再使於吳,吳人稱焉。時黃皓預政,眾多附之,憲獨介然。皓恚之,左遷巴東太守。時大將軍閻宇都督巴東,拜憲領軍,為宇副貳。魏之伐蜀,召宇西還,憲守永安城。
 及成都敗,城中擾動,邊江長吏皆棄城走,憲斬亂者一人,百姓乃安。知劉禪降,乃率所統臨于都亭三日。吳聞蜀敗,遣將軍盛憲西上,外託救援,內欲襲憲。憲曰:「本朝傾覆,吳為脣齒,不恤我難,而邀其利,吾寧當為降虜乎!」乃歸順。於是繕甲完聚,厲以節義,士皆用命。及鐘會、鄧艾死,百城無主,吳又使步協西征,憲大破其軍。孫休怒,又遣陸抗助協。憲距守經年,救援不至,城中疾疫太半。或勸南出牂柯,北奔上庸,可以保全。憲曰:「夫為人主,百姓所仰,既不能存,急而棄之,君子不為也。畢命於此矣。」會荊州刺史胡烈等救之,抗退。加陵江將軍、監巴東軍
 事、使持節,領武陵太守。泰始初入朝,詔曰:「憲忠烈果毅,有才策器幹,可給鼓吹。」又賜山玄玉佩劍。泰始六年卒,贈使持節、安南將軍、武陵太守,追封西鄂侯,謚曰烈。



 初,憲侍宴華林園,詔問蜀大臣子弟,後問先輩宜時敘用者,憲薦蜀人常忌、杜軫等,皆西國之良器,武帝並召而任之。



 子襲,歷給事中、陵江將軍,統其父部曲,至廣漢太守。兄子尚。



 尚字敬之,一名仲。父式,牂柯太守。尚少孤,依叔父憲。善屬文。荊州刺史王戎以尚及劉喬為參軍,並委任之。太康末,為梁州刺史。及趙廞反于蜀,尚表曰:「廞非雄才,必
 無所成,計日聽其敗耳。」乃假尚節為平西將軍、益州刺史、西戎校尉。性貪,少斷,蜀人言曰:「尚之所愛,非邪則佞,尚之所憎,非忠則正。富擬魯、衛,家成市里;貪如豺狼,無復極已。」又曰:「蜀賊尚可,羅尚殺我。平西將軍,反更為禍。」時李特亦起於蜀,攻蜀,殺趙廞。又攻尚於成都,尚退保江陽,初,尚乞師方嶽,荊州刺史宗岱率建平太守孫阜救之,次于江州,岱、阜兵盛,諸為寇所逼者,人有奮志。尚乃使兵曹從事任銳偽降,因出密宣告於外,剋日俱擊,遂大破之,斬李特,傳首洛陽。特子雄僭號,都於郫城。尚遣將軍隗伯攻之,不剋。俄而尚卒,雄遂據有蜀土。



 滕脩,字顯先,南陽西鄂人也。仕吳為將帥,封西鄂侯。孫皓時,代熊睦為廣州刺史,甚有威惠。徵為執金吾。廣州部曲督郭馬等為亂,皓以脩宿有威惠,為嶺表所伏,以為使持節、都督廣州軍事、鎮南將軍、廣州牧以討之。未剋而王師伐吳,脩率眾赴難。至巴丘而皓已降,乃縞素流涕而還,與廣州刺史閭豐、蒼梧太守王毅各送印綬,詔以脩為安南將軍,廣州牧、持節、都督如故,封武當侯,加鼓吹,委以南方事。脩在南積年,為邊夷所附。



 太康九年卒,請葬京師,帝嘉其意,賜墓田一頃,謚曰聲。脩之子
 並上表曰:「亡父脩羈紲吳壤,為所驅馳;幸逢開通,沐浴至化,得從俘虜握戎馬之要;未覲聖顏,委南籓之重,實由勛勞少聞天聽故也。年衰疾篤,屢乞骸骨,未蒙垂哀,奄至薨隕。臣承遺意,輿櫬還都,瞻望雲闕,實懷痛裂。竊聞博士謚脩曰聲,直彰流播,不稱行績,不勝愚情,冒昧聞訴。」帝乃賜謚曰忠。



 並子含,初為庾冰輕車長史,討蘇峻有功,封夏陽縣開國侯,邑千六百戶,授平南將軍、廣州刺史。在任積年,甚有威惠,卒謚曰戴。含弟子遁,交州刺史。



 脩曾孫恬之,龍驤將軍、魏郡太守,戍黎陽,為翟遼所執,死之。



 馬隆,字孝興,東平平陸人。少而智勇,好立名節。魏兗州刺史令狐愚坐事伏誅,舉州無敢收者。隆以武吏託稱愚客,以私財殯葬,服喪三年,列植松柏,禮畢乃還,一州以為美談。署武猛從事。泰始中,將興伐吳之役,下詔曰:「吳會未平,宜得猛士以濟武功。雖舊有薦舉之法,未足以盡殊才。其普告州郡,有壯勇秀異才力傑出者,皆以名聞,將簡其尤異,擢而用之。茍有其人,勿限所取。」兗州舉隆才堪良將。稍遷司馬督。



 初,涼州刺史楊欣失羌戎之和,隆陳其必敗。俄而欣為虜所沒,河西斷絕,帝每有
 西顧之憂,臨朝而歎曰:「誰能為我討此虜通涼州者乎?」朝臣莫對。隆進曰:「陛下若能任臣,臣能平之。」帝曰:「必能滅賊,何為不任,顧卿方略何如耳。隆曰:「陛下若能任臣,當聽臣自任。」帝曰:「云何?隆曰:「臣請募勇士三千人,無問所從來,率之鼓行而西,稟陛下威德,醜虜何足滅哉!」帝許之,乃以隆為武威太守。公卿僉曰:「六軍既眾,州郡兵多,但當用之,不宜橫設賞募以亂常典。隆小將妄說,不可從也。」帝弗納。隆募限腰引弩三十六鈞、弓四鈞,立標簡試。自旦至中,得三千五百人,隆曰:「足矣。」因請自至武庫選杖。武庫令與隆忿爭,御史中丞奏劾隆,隆曰:「臣當
 亡命戰場,以報所受,武庫令乃以魏時朽杖見給,不可復用,非陛下使臣滅賊意也。」帝從之,又給其三年軍資。隆於是西渡溫水。虜樹機能等以眾萬計,或乘險以遏隆前,或設伏以截隆後。隆依八陣圖作偏箱車,地廣則鹿角車營,路狹則為木屋施於車上,且戰且前,弓矢所及,應弦而倒。奇謀間發,出敵不意。或夾道累磁石,賊負鐵鎧,行不得前,隆卒悉被犀甲,無所留礙,賊咸以為神。轉戰千里,殺傷以千數。自隆之西,音問斷絕,朝廷憂之,或謂已沒。後隆使夜到,帝撫掌歡笑。詰朝,召群臣謂曰:「若從諸卿言,是無秦、涼也。」乃詔曰:「隆以偏師寡眾,奮不
 顧難,冒險能濟。其假節、宣威將軍,加赤幢、曲蓋、鼓吹。」隆到武威,虜大人猝跋韓、且萬能等率萬餘落歸降,前後誅殺及降附者以萬計。又率善戎沒骨能等與樹機能大戰,斬之,涼州遂平。朝議將加隆將士勳賞,有司奏隆將士皆先加顯爵,不應更授,衛將軍楊珧駁曰:「前精募將士,少加爵命者,此適所以為誘引。今隆全軍獨剋,西土獲安,不得便以前授塞此後功,宜皆聽許,以明要信。」乃從珧議,賜爵加秩各有差。



 太康初,朝廷以西平荒毀,宜時興復,以隆為平虜護軍、西平太守,將所領精兵,又給牙門一軍,屯據西平。時南虜成奚每為邊患,隆至,帥
 軍討之。虜據險距守,隆令軍士皆負農器,將若田者。虜以隆無征討意,御眾稍怠。隆因其無備,進兵擊破之。畢隆之政,不敢為寇。太熙初,封奉高縣侯,加授東羌校尉。積十餘年,威信震於隴右。時略陽太守馮翊嚴舒與楊駿通親,蜜圖代隆,毀隆年老謬耄,不宜服戎,於是徵隆,以舒代鎮。氏、羌聚結,百姓驚懼。朝廷恐關隴復擾,乃免舒,遣隆復職,竟卒於官。



 子咸嗣,亦驍勇。成都王穎攻長沙王乂,以咸為鷹揚將軍,率兵屯河橋中渚,為乂將王瑚所敗,沒於陣。



 胡奮,字玄威,安定臨涇人也,魏車騎將軍陰密侯遵之子也。奮性開朗,有籌略,少好武事。宣帝之伐遼東也,以白衣侍從左右,甚見接待。還為校尉,稍遷徐州刺史,封夏陽子。匈奴中部帥劉猛叛,使驍騎路蕃討之,以奮為監軍、假節,頓軍硜北,為蕃後繼。擊猛,破之,猛帳下將李恪斬猛而降。以功累遷征南將軍、假節、都督荊州諸軍事,遷護軍,加散騎常侍。奮家世將門,晚乃好學,有刀筆之用,所在有聲績,居邊特有威惠。



 泰始末,武帝怠政事而耽於色,大採擇公卿女以充六宮,奮女選入為貴人。奮唯有一子,為南陽王友,早亡。及聞女為貴人,哭曰:「老
 奴不死,唯有二兒,男入九地之下,女上九天之上。」奮既舊臣,兼有椒房之助,甚見寵待。遷左僕射,加鎮軍大將軍、開府儀同三司。時楊駿以后父驕傲自得,奮謂駿曰:「卿恃女更益豪邪?歷觀前代,與天家婚,未有不滅門者,但早晚事耳。觀卿舉措,適所以速禍。」駿曰:「卿女不在天家乎?」奮曰:「我女與卿女作婢耳,何能損益!」時人皆為之懼,駿雖銜之,而不能害。後卒於官,贈車騎將軍,謚曰壯。奮兄弟六人,兄廣,弟烈,並知名。



 廣字宣祖,位至散騎常侍、少府。廣子喜,字林甫,亦以開濟為稱,仕至涼州刺史、建武將軍、假節、護羌校尉。



 列字武玄,為將伐蜀。鐘會之
 反也,烈與諸將皆被閉。烈子世元,時年十八,為士卒先,攻殺會,名馳遠近。烈為秦州刺史,及涼州叛,烈屯於萬斛堆,為虜所圍,無援,遇害。



 陶璜,字世英,丹陽秣陵人也。父基,吳交州刺史。璜仕吳歷顯位。孫皓時,交阯太守孫住貪暴,為百姓所患。會察戰鄧荀至,擅調孔雀三千頭,遣送秣陵,既苦遠役,咸思為亂。郡吏呂興殺住及荀,以郡內附。武帝拜興安南將軍、交阯太守。尋為其功曹李統所殺,帝更以建寧爨谷為交阯太守,谷又死,更遣巴西馬融代之。融病卒,南中
 監軍霍弋又遣犍為楊稷代融,與將軍毛炅,九真太守董元,牙門孟幹、孟通、李松、王業、爨能等,自蜀出交阯,破吳軍于古城,斬大都督脩則、交州刺史劉俊。吳遣虞汜為監軍,薛珝為威南將軍、大都督,璜為蒼梧太守,距稷,戰于分水。璜敗,退保合浦,亡其二將。珝怒謂璜曰:「若自表討賊,而喪二帥,其責安在?」璜曰:「下官不得行意,諸軍不相順,故致敗耳。」珝怒,欲引軍還。璜夜以數百兵襲董元,獲其寶物,船載而歸,珝乃謝之,以璜領交州,為前部督。璜從海道出於不意,徑至交阯,元距之。諸將將戰,璜疑斷牆內有伏兵,列長戟於甚後。兵纔接,元偽退,璜追
 之,伏兵果出,長戟逆之,大破元等。以前所得寶船上錦物數千匹遺扶嚴賊帥梁奇,奇將萬餘人助璜。元有勇將解系同在城內,璜誘其弟象,使為書與系,又使象乘璜軺車,鼓吹導從而行。元等曰:「象尚若此,系必去志。」乃就殺之。珝、璜遂陷交阯。吳因用璜為交州刺史。



 璜有謀策,周窮好施,能得人心。滕脩數討南賊,不能制,璜曰:「南岸仰吾鹽鐵,斷勿與市,皆壞為田器。如此二年,可一戰而滅也。」脩從之,果破賊。



 初,霍弋之遣稷、炅等,與之誓曰:「若賊圍城未百日而降者,家屬誅;若過百日救兵不至,吾受其罪。」稷等守未百日,糧盡,乞降,璜不許,給其糧
 使守。諸將並諫,璜曰:「霍弋已死,不能救稷等必矣,可須其日滿,然後受降,使彼得無罪,我受有義,內訓百姓,外懷鄰國,不亦可乎!」稷等期訖糧盡,救兵不至,乃納之。脩則既為毛炅所殺,則子允隨璜南征,城既降,允求復仇,璜不許。炅密謀襲璜,事覺,收炅,呵曰:「晉賊!」炅厲聲曰:「吳狗!何等為賊?」允剖其腹,曰:「復能作賊不?」炅猶罵曰:「吾志殺汝孫皓,汝父何死狗也!」璜既擒稷等,並送之。稷至合浦,發病死。孟幹、爨能、李松等至建鄴,皓將殺之。或勸皓,乾等忠於所事,宜宥之以勸邊將,皓從其言,將徙之臨海。乾等志欲北歸,慮東徙轉遠,以吳人愛蜀側竹弩,言
 能作之,皓留付作部。後幹逃至京都,松、能為皓所殺。乾陳伐吳之計,帝乃厚加賞賜,以為日南太守。先是,以楊稷為交州刺史,毛炅為交止太守,印緩未至而敗,即贈稷交州,炅及松能子並關內侯。



 九真郡功曹李祚保郡內附,璜遣將攻之,不剋。祚舅黎晃隨軍。勸祚令降。祚答曰:「舅自吳將,祚自晉臣,唯力是視耳。」踰時乃拔。皓以璜為使持節、都督交州諸軍事、前將軍、交州牧。武平、九德、新昌土地阻險,夷獠勁悍,歷世不賓,璜征討,開置三郡,及九真屬國三十餘縣。征璜為武昌都督,以合浦太守脩允代之。交土人請留璜以千數,於是遣還。



 皓既降晉,
 手書遣璜息融敕璜歸順。璜流涕數日,遣使送印綬詣洛陽。帝詔復其本職,封宛陵侯,改為冠軍將軍。



 吳既平,普減州郡兵,璜上言曰:「交土荒裔,斗絕一方,或重譯而言,連帶山海。又南郡去州海行千有餘里,外距林邑纔七百里。夷帥范熊世為逋寇,自稱為王,數攻百姓。且連接扶南,種類猥多,朋黨相倚,負險不賓。往隸吳時,數作寇逆,攻破郡縣,殺害長吏。臣以尪駑,昔為故國所採,偏戍在南,十有餘年。雖前後征討,翦其魁桀,深山僻穴,尚有逋竄。又臣所統之卒本七千餘人,南土溫濕,多有氣毒,加累年征討,死亡減耗,其見在者二千四百二十人。
 今四海混同,無思不服,當卷甲清刃,禮樂是務。而此州之人,識義者寡,厭其安樂,好為禍亂。又廣州南岸,周旋六千餘里,不賓屬者乃五萬餘戶,及桂林不羈之輩,復當萬戶。至於服從官役,纔五千餘家。二州脣齒,唯兵是鎮。又寧州興古接據上流,去交址郡千六百里,水陸並通,互相維衛。州兵未宜約損,以示單虛。夫風塵之變,出於非常。臣亡國之餘,議不足採,聖恩廣厚,猥垂飾擢,蠲其罪釁,改授方任,去辱即寵,拭目更視,誓念投命,以報所受,臨履所見,謹冒瞽陳。」又以「合浦郡土地磽确,無有田農,百姓唯以採珠為業,商賈去來,以珠貿米。而吳時
 珠禁甚嚴,慮百姓私散好珠,禁絕來去,人以飢困。又所調猥多,限每不充。今請上珠三分輸二,次者輸一,粗者蠲除。自十月訖二月,非採上珠之時,聽商旅往來如舊」。並從之。



 在南三十年,威恩著于殊俗。及卒,舉州號哭,如喪慈親。朝廷乃以員外散騎常侍吾彥代璜。彥卒,又以員外散騎常侍顧祕代彥。秘卒,州人逼秘子參領州事。參尋卒,參弟壽求領州,州人不聽,固求之,遂領州。壽乃殺長史胡肇等,又將殺帳下督梁碩,碩走得免,起兵討壽,禽之,會壽母,令鴆殺之。碩乃迎璜子蒼梧太守威領刺史,在職甚得百姓心,三年卒。威弟淑,子綏,後並為交
 州。自基至綏四世,為交州者五人。



 璜弟浚,吳鎮南大將軍、荊州牧。濬弟抗,太子中庶子。濬子湮,字恭之;湮弟猷,字恭豫,並有名。湮至臨海太守、黃門侍郎。猷宣城內史,王導右軍長史。湮子馥,于湖令,為韓晃所殺,追贈廬江太守。抗子回,自有傳。



 吾彥,字士則,吳郡吳人也。出自寒微,有文武才幹。身長八尺,手格猛獸,旅力絕群。仕吳為通江吏。時將軍薛珝杖節南征,軍容甚盛,彥觀之,慨然而歎。有善相者劉札謂之曰:「以君之相,後當至此,不足慕也。」初為小將,給吳
 大司馬陸抗。抗奇其勇略,將拔用之,患眾情不允,乃會諸將,密使人陽狂拔刀跳躍而來,坐上諸將皆懼而走,唯彥不動,舉几禦之,眾服其勇,乃擢用焉。



 稍遷建平太守。時王濬將伐吳,造船於蜀,彥覺之,請增兵為備,皓不從,彥乃輒為鐵鎖,橫斷江路。及師臨境,緣江諸城皆望風降附,或見攻而拔,唯彥堅守,大眾攻之不能剋,乃退舍禮之。



 吳亡,彥始歸降,武帝以為金城太守。帝嘗從容問薛瑩曰:「孫皓所以亡國者何也?」瑩對曰:「歸命侯臣皓之君吳,暱近小人,刑罰妄加,大臣大將無所親信,人人憂恐,各不自安,敗亡之釁,由此而作矣。」其後帝又問彥,
 對曰:「吳主英俊,宰輔賢明。」帝笑曰:「君明臣賢,何為亡國?」彥曰:「天祿永終,歷數有屬,所以為陛下擒。此蓋天時,豈人事也!」張華時在坐,謂彥曰:「君為吳將,積有歲年,蔑爾無聞,竊所惑矣。」彥厲聲曰:「陛下知我,而卿不聞乎?」帝甚嘉之。」



 轉在敦煌,威恩甚著。遷鴈門太守。時順陽王暢驕縱,前後內史皆誣之以罪。乃彥為順陽內史,彥清身率下,威刑嚴肅,眾皆畏懼。暢不能誣,乃更薦之,冀其去職。遷員外散騎常侍。帝嘗問彥:「陸喜、陸抗二人誰多也?」彥對曰:「道德名望,抗不及喜;立功立事,喜不及抗。」



 會交州刺史陶璜卒,以彥為南中都督、交州刺史。重餉陸機兄
 弟,機將受之,雲曰:「彥本微賤,為先公所拔,而答詔不善,安可愛之!」機乃止。因此每毀之。長沙孝廉尹虞謂機等曰:「自古由賤而興者,乃有帝王,何但公卿。若何元幹、侯孝明、唐儒宗、張義允等,並起自寒役,皆內侍外鎮,人無譏者。卿以士則答詔小有不善,毀之無已,吾恐南人皆將去卿,卿便獨坐也。」於是機等意始解,毀言漸息矣。



 初,陶璜之死也,九真戍兵作亂,逐其太守,九真賊帥趙祉圍郡城,彥悉討平之。在鎮二十餘年,威恩宣著,南州寧靖。自表求代,徵為大長秋。卒於官。



 張光,字景武,江夏鐘武人也。身長八尺,明眉目,美音聲。少為郡吏,家世有部曲,以牙門將伐吳有功,遷江夏西部都尉,轉北地都尉。



 初,趙王倫為關中都督,氐、羌反叛,太守張損戰沒,郡縣吏士少有全者。光以百餘人戍馬蘭山北,賊圍之百餘日。光撫厲將士,屢出奇兵擊賊,破之。光以兵少路遠,自分敗沒。會梁王肜遣司馬索靖將兵迎光,舉軍悲泣,遂還長安。肜表光「處絕圍之地,有耿恭之忠,宜加甄賞,以明獎勸」。於是擢授新平太守,加鼓吹。



 屬雍州刺史劉沈被密詔討河間王顒,光起兵助沈。沈時委任秦州刺史皇甫重,重自以關西大族,心每輕
 光,謀多不用。及二州軍潰,為顒所擒,顒謂光曰:「前起兵欲作何策?」光正色答曰:「但劉雍州不用鄙計,故令大王得有今日也。」顒壯之,引與歡宴彌日,表為右衛司馬。



 陳敏作亂,除光順陽太守,加陵江將軍,率步騎五千詣荊州討之。刺史劉弘雅敬重光,稱為南楚之秀。時江夏太守陶侃與敏大將錢端相距於長岐,將戰,襄陽太守皮初為步軍,使光設伏以待之,武陵太守苗光為水軍,藏舟艦於沔水。皮初等與賊交戰,光發伏兵應之,水陸同奮,賊眾大敗。弘表光有殊勳,遷材官將軍,梁州刺史。先是,秦州人鄧定等二千餘家,饑餓流入漢中,保于成固,
 漸為抄盜,梁州刺史張殷遣巴西太守張燕討之。定窘急,偽乞降于燕,并餽燕金銀,燕喜,為之緩師。定密結李雄,雄遣眾救定,燕退,定遂進逼漢中。太守杜正沖東奔魏興,殷亦棄官而遁。光不得赴州,止於魏興,乃結諸郡守共謀進取。燕唱言曰:「漢中荒敗,迫近大賊,剋復之事,當俟英雄。」正沖曰:「張燕受賊金銀,不時進討,阻兵緩寇,致喪漢中,實燕之罪也。」光於是發怒,呵燕令出,斬之以徇。綏撫荒殘,百姓悅服。光於是卻鎮漢中。



 時逆賊王如餘黨李運、楊武等,自襄陽將三千餘家入漢中,光遣參軍晉邈率眾於黃金距之。邈受運重賂,勸光納運。光從
 邈言,使居成固。既而邈以運多珍貨,又欲奪之,復言於光曰:「運之徒屬不事佃農,但營器杖,意在難測,可掩而取之。」光又信焉。遣邈眾討運,不剋。光乞師於氐王楊茂搜,茂搜遣子難敵助之。難敵求貨於光,光不與。楊武乃厚賂難敵,謂之曰「流人寶物悉在光處,今伐我,不如伐光。」難敵大喜,聲言助光,內與運同,光弗之知也,遣息援率眾助邈。運與難敵夾攻邈等,援為流矢所中死,賊遂大盛。光嬰城固守,自夏迄冬,憤激成疾。佐吏及百姓咸勸光退據魏興,光按劍曰:「吾受國厚恩,不能翦除寇賊,今得自死,便如登仙,何得退還也!」聲絕而卒,時年五十
 五。百姓悲泣,遠近傷惜之。有二子炅、邁。



 炅少辟太宰掾。邁多才略,有父風。州人推邁權領州事,與賊戰沒。別駕范曠及督護王喬奉光妻息,率其遺眾,還據魏興。其後義陽太守任愔為梁州,光妻子歸本郡。南平太守應詹白都督王敦,稱「光在梁州能興微繼絕,威振巴漢。值中原傾覆,征鎮失守,外無救助,內闕資儲,以寡敵眾,經年抗禦,厲節不撓,宜應追論顯贈,以慰存亡」。敦不能從。



 趙誘,字元孫,淮南人也。世以將顯。州辟言簿。值刺史郗隆被齊王冏檄,使起兵討趙王倫,隆欲承檄舉義,而諸
 子姪並在洛陽;欲坐觀成敗,恐為冏所討,進退有疑,會群吏計議。誘說隆曰:「趙王篡逆,海內所病。今義兵飆起,其敗必矣。今為明使君計,莫若自將精兵,徑赴許昌,上策也。不然,且可留後,遣猛將將兵會盟,亦中策也。若遣小軍隨形助勝。下策耳。隆曰:「我受二帝恩,無所偏助,正欲保州而已。」誘與治中留寶、主簿張褒等諫隆:「若無所助,變難將生,州亦不可保也。」隆猶豫不決,遂為其下所害。誘還家,杜門不出。左將軍王敦以為參軍,加廣武將軍,與甘卓、周訪共討華軼,破之。又擊杜弢於西湘,太興初,復與卓攻弢,滅之。累功賜爵平阿縣侯,代陶侃為武
 昌太守。時杜曾迎第五猗於荊州作亂,敦遣誘與襄陽太守朱軌共距之。猗既愍帝所遣,加有時望,為荊楚所歸。誘等苦戰皆沒,敦甚悼惜之,表贈征虜將軍、秦州刺史,謚曰敬。



 子龔,與誘俱死。元帝為晉王,下令贈新昌太守。龔弟胤,字伯舒。王敦使周訪擊杜曾,胤請從行。訪憚曾之彊,欲先以胤餌曾,使其眾疲而後擊之。胤多梟首級。王導引為從事中郎。南頓王宗反,胤殺宗。於是王導、庾亮並倚仗之。轉冠軍將軍,遷西豫州刺史,卒於官。



 史臣曰:忠為令德,貞曰事君,徇國家而竭身,歷夷險而一節。羅憲、滕脩,濯纓入仕,指巴東而受脤,出嶺嶠而揚
 麾。屬鼎命淪胥,本朝失守,郕巴丘而流涕,集都亭而大臨。古之忠烈,罕輩子茲!孝興之智勇,玄威之武藝,滅醜虜於河西,制凶酋於硜北,審楊欣之必敗,譏楊駿之速禍。陶璜、吾彥,逸足齊驅,毛炅屈其深謀,陸抗奇其茂略。薪楢之任,清規自遠;鼙鼓之臣,厥聲彌劭。景武,南楚秀士;元孫,累葉將門,赴死喻於登仙。效誠陳於上策,竟而俱斃,貞則斯存。



 贊曰:憲居玉疊,才博流譽。脩赴石門,惠政攸著。孝興、玄威,操履無違。愚墳畢禮,楊門致譏。璜謀超絕,彥材雄傑。潛師襲董,觀兵歎薛。惟趙與張,神略多方。作尉北地,立
 功西湘。



\end{pinyinscope}