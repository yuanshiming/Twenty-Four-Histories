\article{帝紀第十}

\begin{pinyinscope}

 安
 帝恭帝



 安皇帝諱德宗,字德宗,孝武帝長子也。太元十二年八月辛巳,立為皇太子。二十一年九月庚申,孝武帝崩。辛酉,太子即皇帝位,大赦。癸亥,以司徒、會稽王道子為太傅,攝政。冬十月甲申。葬孝武皇帝于隆平陵。大雪。



 隆安元年春正月己亥朔,帝加元服,改元,增文武位一等。太傅、會稽王道子稽首歸政。以尚書右僕射王珣為尚書
 令,領軍將軍王國寶為尚書左僕射。二月,呂光將禿髮烏孤自稱大都督、大單于,國號南涼。擊光將竇茍于金昌,大破之。甲寅,尊皇太后李氏為皇太后。戊午,立皇后王氏。三月,呂光子纂為乞伏乾歸所敗。光建康太守段業自號涼州牧。慕容寶敗魏師于薊。夏四月甲戌,兗州刺史王恭,豫州刺史庾楷舉兵,以討尚書左僕射王國寶、建威將軍王緒為名。甲申,殺國寶及緒以悅於恭,恭乃罷兵。戊子,大赦。五月,前司徒長史王廞以吳郡反,王恭討平之。慕容寶將慕容詳僭即皇帝位于中山,寶奔黃龍。秋八月,呂光為其僕射楊軌、散騎常侍郭黁所
 攻,光子纂擊走之。九月,慕容寶將慕容麟斬慕容祥於中山,因僭即皇帝位。冬十月,慕容麟為魏師所敗。



 二年春三月,龍舟二災。夏五月,蘭汗弒慕容寶而自稱大將軍、昌黎王。秋七月,慕容寶子盛斬蘭汗,僭稱長樂王,攝天子位。兗州刺史王恭、豫州刺史庾楷、荊州刺史殷仲堪、廣州刺史桓玄、南蠻校尉楊佺期等舉兵反。八月,江州刺史王愉奔于臨川。丙子,寧朔將軍鄧啟方及慕容德將慕容法戰于管城,王師敗績。丙戌,慕容盛僭即皇帝位於黃龍。桓玄大敗王師于白石。九月辛卯,加太傅、會稽王道子黃鉞。遣征虜將軍會稽王世子元顯、前
 將軍王珣、右將軍謝琰討桓玄等。己亥,破庾楷于牛渚。丙午,會稽王道子屯中堂,元顯守石頭。己酉,前將軍王珣守北郊,右將軍謝琰備宣陽門。輔國將軍劉牢之次新亭,使子敬宣擊敗恭,恭奔曲阿長塘湖,湖尉收送京師,斬之。於是遣太常殷茂喻仲堪及玄,玄等走于尋陽。冬十月,新野言騶虞見。丙子,大赦。壬午,仲堪等盟于尋陽,推桓玄為盟主。十一月,以瑯邪王德文為衛將軍、開府儀同三司,領軍將軍王雅為尚書左僕射。十二月己丑,魏王珪即尊位,年號天興。京兆人韋禮帥襄陽流人叛,降于姚興。己酉,前新安太守杜炯反於京口,會稽王
 世子元顯討斬之。禿髮烏孤自稱武威王。



 三年春正月辛酉,封宗室蘊為淮陵王。二月甲辰,河間王國鎮薨。林邑范胡達陷日南、九真,遂寇交址,太守杜瑗討破之。段業自稱涼王。仇池公楊盛遣使稱籓,獻方物。三月己卯,追尊所生陳夫人為德皇太后。夏四月乙未,加尚書令王珣衛將軍,以會稽王世子元顯為揚州刺史。六月戊子,以瑯邪王德文為司徒。慕容德陷青州,害龍驤將軍辟閭渾,遂僭即皇帝位于廣固。秋八月,禿髮烏孤死,其弟利鹿孤嗣偽位。冬十月,姚興陷洛陽,執河南太守辛恭靖。十一月甲寅,妖賊孫恩陷會稽,內史王
 凝之死之,吳國內史桓謙、臨海太守新蔡王崇、義興太守魏隱並委官而遁,吳興太守謝邈、永嘉太守司馬逸皆遇害。遣衛將軍謝琰、輔國將軍劉牢之逆擊,走之。十二月,桓玄襲江陵,荊州刺史殷仲堪、南蠻校尉楊佺期並遇害。呂光立其太子紹為天王,自號太上皇。是日,光死,呂纂弒紹而自立。是歲,荊州大水,平地三丈。



 四年春正月乙亥,大赦。二月己丑,有星孛于奎婁,進至紫微。三月,彗星見於太微。夏四月,地震。孫恩寇浹口。五月丙寅,散騎常侍、衛將軍、東亭侯王珣卒。己卯,會稽內史謝琰為孫恩所敗,死之。恩轉寇臨海。六月庚辰朔,日
 有蝕之。旱。輔國司馬劉裕破恩於南山。恩將盧循陷廣陵,死者三千餘人。以瑯邪王師何澄為尚書左僕射。秋七月壬子,太皇太后李氏崩。丁卯,大赦。是月,姚興伐乞伏乾歸,降之。八月丁亥,尚書右僕射王雅卒。壬寅,葬文太后于修平陵。九月癸丑,地震。「冬十一月,寧朔將軍高雅之及孫恩戰于餘姚,王師敗績。以揚州刺史元顯為後將軍、開府儀同三司、都督揚豫徐兗青幽冀并荊江司雍梁益交廣十六州諸軍事,前將軍劉牢之為鎮北將軍,封元顯子彥璋為東海王。十二月戊寅,有星孛于天市。是歲,河右諸郡奉涼武昭王李玄盛為秦涼二州
 牧、涼公,年號庚子。



 五年春二月丙子,孫恩復寇浹口。呂超殺呂纂,以其兄隆僭即偽位。三月甲寅,眾星西流,歷太微。夏五月,孫恩寇吳國,內史袁山松死之。沮渠蒙遜殺段業,自號大都督、北涼州牧。六月甲戌,孫恩至丹徒。乙亥,內外戒嚴,百官入居于省。冠軍將軍高素、右衛將軍張崇之守石頭,輔國將軍劉襲柵斷淮口,丹陽尹司馬恢之戍南岸,冠軍將軍桓謙、輔國將軍司馬允之、游擊將軍毛邃備白石,左衛將軍王嘏、領軍將軍孔安國屯中皇堂。徵豫州刺史、譙王尚之衛京師。寧朔將軍高雅之擊孫恩於廣
 陵之郁洲,為賊所執。秋七月,段璣殺慕容盛,盛叔父熙盡誅段氏,因僭稱尊號。九月,呂隆降於姚興。冬十月,姚興帥侵魏,大敗而旋。是歲,饑,禁酒。



 元興元年春正月庚午朔,大赦,改元。以後將軍元顯為驃騎大將軍、征討大都督,鎮北將軍劉牢之為元顯前鋒,前將軍、譙王尚之為後部,以討桓玄。二月丙午,帝戎服餞元顯于西池。丁巳,遣兼侍中、齊王柔之以騶虞幡宣告荊、江二州。丁卯,桓玄敗王師于姑孰,譙王尚之、齊王柔之並死之。以右將軍吳隱之為都督交廣二州諸軍事、廣州刺史。三月己巳,劉牢之叛降于桓玄。辛未,王
 師敗績于新亭,驃騎大將軍、會稽王世子元顯,東海王彥璋,冠軍將軍毛泰,游擊將軍毛邃並遇害。壬申,桓玄自為侍中、丞相、錄尚書事,以桓謙為尚書僕射,遷太傅、會稽王道子于安城。玄俄又自稱太尉、揚州牧,總百揆,以瑯邪王德文為太宰。臨海太守辛景擊孫恩,斬之。是月,禿髮利鹿孤死,弟辱檀嗣偽位。秋七月乙亥,新蔡王崇為其奴所害。八月庚子,尚書下舍災。冬十月,冀州刺史劉軌叛奔于慕容德。十二月庚申,會稽王道子為桓玄所害。曲赦廣陵、彭城大逆以下。



 二年春二月辛丑,建威將軍劉裕破徐道覆于東陽。乙
 卯,桓玄自稱大將軍。丁巳,冀州刺史孫無終為桓玄所害。夏四月癸巳朔,日有蝕之。秋八月,玄又自號相國、楚王。九月,南陽太守庾仄起義兵,為玄所敗。冬十一月壬午,玄遷帝於永安宮。癸未,移太廟神主于瑯邪國。十二月壬辰,玄篡位,以帝為平固王。辛亥,帝蒙塵于尋陽。



 三年春二月,帝在尋陽。庚寅夜,濤水入石頭,漂殺人戶。乙卯,建武將軍劉裕帥沛國劉毅、東海何無忌等舉義兵。丙辰,斬桓玄所署徐州刺史桓修于京口,青州刺史桓弘于廣陵。丁巳,義師濟江。三月戊午,劉裕斬玄將吳甫之于江乘,斬皇甫敷於羅落。己未,玄眾潰而逃。庚申,
 劉裕置留臺,具百官。壬戌,桓玄司徒王謐推劉裕行鎮軍將軍、徐州刺史、都督揚徐兗豫青冀幽并八州諸軍事、假節。劉裕以謐領揚州刺史、錄尚書事。辛酉,劉裕誅尚書左僕射王愉、愉子荊州刺史綏、司州刺史溫詳。辛未,桓玄逼帝西上。丙戌,密詔以幽逼於玄,萬機虛曠,令武陵王遵依舊典,承制總百官行事,加侍中,餘如故。并大赦謀反大逆己下,惟桓玄一祖之後不宥。夏四月己丑,大將軍、武陵王遵稱制,總萬機。庚寅,帝至江陵。庚戌,輔國將軍何無忌、振武將軍劉道規及桓玄將庾稚、何澹之戰于湓口,大破之。玄復逼帝東下。五月癸酉,冠軍
 將軍劉毅及桓玄戰于崢嶸洲,又破之。己卯,帝復幸江陵。辛巳,荊州別駕王康產、南郡太守王騰之奉帝居于南郡。壬午,督護馮遷斬桓玄於貊盤洲。乘興反正于江陵。甲申,詔曰:「姦兇篡逆,自古有之。朕不能式遏杜漸,以致播越。賴鎮軍將軍裕英略奮發,忠勇絕世,冠軍將軍毅等誠心宿著,協助同嘉謀。義聲既振,士庶效節,社稷載安,四海齊慶。其大赦,凡諸畏逼事屈逆命者,一無所問。」戊寅,奉神主人于太廟。閏月己丑,桓玄故將揚武將軍桓振陷江陵,劉毅、何無忌退守尋陽,帝復蒙塵于賊營。六月,益州刺史毛璩討偽梁州刺史桓希,斬之。秋七月
 戊申,永安皇后何氏崩。八月癸酉,祔葬穆帝章皇后于永平陵。九月,前給事中刁騁、秘書丞王邁之謀反,伏誅。冬十月,盧循寇廣州,刺史吳隱之為循所敗。執始興相阮腆之而還。慕容德死,兄子超嗣偽位。



 義熙元年春正月,帝在江陵。南陽太守魯宗之起義兵,襲破襄陽。己丑,劉毅次于馬頭。桓振以帝屯于江津。辛卯,宗之破振將溫楷于柞溪,進次紀南,為振所敗。振武將軍劉道規擊桓謙,走之。乘輿反正,帝與瑯邪王幸道規舟。戊戌,詔曰:「朕以寡德,夙纂洪緒。不能緝熙遐邇,式遏姦宄。逆臣桓玄乘釁肆亂,乃誣罔天人,篡據極位。朕
 躬播越,淪胥荒裔,宣皇之基,眇焉以墜。賴鎮軍將軍裕忠武英斷,誠冠終古。運謀機始,貞賢協其契;抆淚誓眾,義士感其心。故霜戈一揮。巨猾奔迸,三率棱威,大憝授首。而孽振猖狂,嗣凶荊郢。幸天祚社稷,義旗載捷,狡徒沮潰,朕獲反正。斯實宗廟之靈,勤王之勳。豈朕一人獨享伊祜,思與億兆幸茲更始。其大赦,改元,唯玄振一祖及同黨不在原例。賜百官爵二級,鰥寡孤獨穀人五斛,大酺五日。」二月丁巳,留臺備乘輿法駕,迎帝於江陵。弘農太守戴寧之、建威主簿徐惠子等謀反,伏誅。平西參軍譙縱害平西將軍、益州刺史毛璩,以蜀叛。三月,桓振
 復襲江陵,荊州刺史司馬休之奔於襄陽。建威將軍劉懷肅討振,斬之。帝至自江陵。乙未,百官詣闕請罪。詔曰:「此非諸卿之過,其還率職。」戊戌,舉章皇后哀三日,臨于西堂。劉裕及何無忌等抗表遜位,不許。庚子,以瑯邪王德文為大司馬,武陵王遵為太保,加鎮軍將軍劉裕為侍中、車騎將軍、都督中外諸軍事。甲辰,詔曰:「自頃國難之後,人物凋殘,常所供奉,猶不改舊,豈所以視人如傷,禹湯歸過之誡哉!可籌量減省。」夏四月,劉裕旋鎮京口。戊辰,餞于東堂。五月癸未,禁絹扇及摴蒲。游擊將軍、章武王秀,益州刺史司馬軌之謀反,伏誅。桓玄故將桓亮、
 苻宏、刁預寇湘州,守將擊走之。秋八月甲子,封臨川王子修之為會稽王。冬十一月,乞伏乾歸伐仇池,仇池公楊盛大破之。是歲,涼武昭王玄盛遣使奉表稱籓。



 二年春正月,益州刺史司馬榮期擊譙縱將譙子明于白帝,破之。夏五月,封高密王子法蓮為高陽王。秋七月,梁州刺史楊孜敬有罪,伏誅。冬十月,論匡復之功,封車騎將軍劉裕為豫章郡公,撫軍將軍劉毅南平郡公,右將軍何無忌安成郡公,自餘封賞各有差。乙亥,以左將軍孔安國為尚書左僕射。十二月,盜殺零陵太守阮野。



 三年春二月己酉,車騎將軍劉裕來朝。誅東陽太守殷
 仲文、南蠻校尉殷叔文、晉陵太守殷道叔、永嘉太守駱球。己丑,大赦,除酒禁。夏五月,大水。六月,姚興將赫連勃勃僭稱天王于朔方,國號夏。秋七月戊戌朔,日有蝕之。汝南王遵之有罪,伏誅。八月,遣冠軍將軍劉敬宣持節監征蜀諸軍事。冬十一月,赫連勃勃大敗禿髮傉檀,傉檀奔于南山。是歲,高雲、馮跋殺慕容熙,雲僭即帝位。



 四年春正月甲辰,以瑯邪王德文領司徒,車騎將軍劉裕為揚州刺史、錄尚書事。庚申,侍中、太保、武陵王遵薨。夏四月,散騎常侍、尚書左僕射孔安國卒。甲午,加吏部尚書孟昶尚書左僕射。冬十一月辛卯,雷。梁州刺史楊
 思平有罪,棄市。癸丑,大風拔樹。是月,禿髮傉檀僭即涼王位。十二月,陳留王曹靈誕薨。



 五年春正月辛卯,大赦。庚戌,以撫軍將軍劉毅為衛將軍、開府儀同三司,加輔國將軍何無忌鎮南將軍。庚戌,尋陽地震。二月,慕容超將慕容興宗寇宿豫,陽平太守劉千載、南陽太守趙元並為賊所執。三月己亥,大雪,平地數尺。車騎將軍劉裕帥師伐慕容超。夏六月丙寅,震於太廟。劉裕大破慕容超于臨朐。秋七月,姚興將乞伏乾歸僭稱西秦王於苑川。九月戊辰,離班弒高雲,雲將馮跋攻班,殺之。跋僭即王位,仍號燕。冬十月,魏清河王
 紹弒其主圭。



 六年春正月丁亥,劉裕攻慕容超,克之,齊地悉平。是月,廣州刺史盧循反,寇江州。三月,禿髮傉檀及沮渠蒙遜戰于窮泉,傉檀敗績。壬申,鎮南將軍、江州刺史何無忌及循戰于豫章,王師敗績,無忌死之。夏四月,青州刺史諸葛長民、兗州刺史劉籓、并州刺史劉道憐乃入衛京師。五月丙子,大風,拔木。戊子,衛將軍劉毅及盧循戰於桑落洲,王師敗績。尚書左僕射孟昶懼,自殺。己未,大赦。乙丑,循至淮口,內外戒嚴。大司馬、瑯邪王德文都督宮城諸軍事,次中皇堂,太尉劉裕次石頭,梁王珍之屯南
 掖門,冠軍將軍劉敬宣屯北郊,輔國將軍孟懷玉屯南岸,建武將軍王仲德屯越城,廣武將軍劉懷默屯建陽門,淮口築柤浦、藥園、廷尉三壘以距之。丙寅,震太廟鴟尾。秋七月庚申,盧循遁走。甲子,使輔國將軍王仲德、廣川太守劉鐘、河間內史蒯恩等帥眾追之。是月,盧循寇荊州,刺史劉道規、雍州刺史魯宗之等敗之。又破徐道覆於華容,賊復走尋陽。八月,姚興將桓謙寇江陵,劉道規敗之。冬十一月,蜀賊譙縱陷巴東,守將溫祚、時延祖死之。十二月壬辰,劉裕破盧循於豫章。



 七年春二月壬午,右將軍劉籓斬徐道覆于始興,傳首
 京師。夏四月,盧循走交州,刺史杜慧度斬之。秋七月丁卯,以荊州刺史劉道規為征西大將軍、開府儀同三司。冬十月,沮渠蒙遜伐涼,涼武昭王玄盛與戰,敗之。



 八年春二月丙子,以吳興太守孔靖為尚書右僕射。三月甲寅,山陰地陷四尺,有聲如雷。夏五月,乞伏公府弒乞伏乾歸,乾歸子熾盤誅公府,僭即偽位。六月,以平北將軍魯宗之為鎮北將軍。秋七月甲午,武陵王季度薨。庚子,征西大將軍劉道規卒。八月,皇后王氏崩。辛亥,高密王純之薨。九月癸酉,葬僖皇后于休平陵。己卯,太尉劉裕害右將軍兗州刺史劉籓、尚書左僕射謝混。庚辰,
 裕矯詔曰:「劉毅苞藏禍心,構逆南夏,籓、混助亂,志肆姦宄。賴寧輔玄鑒,撫機挫銳,凶黨即戮,社稷乂安。夫好生之德,所因者本,肆眚覃仁,實資玄澤。況事興大憝,禍自元凶。其大赦天下,唯劉毅不在其例。普增文武位一等。孝順忠義,隱滯遺逸,必令聞達。」己丑,劉裕帥師討毅。裕參軍王鎮惡陷江陵城,毅自殺。冬十一月,沮渠蒙遜僭號河西王。十二月,以西陵太守朱齡石為建威將軍、益州刺史,帥師伐蜀。分荊州十郡置湘州。是歲,廬陵、南康地四震。



 九年春三月丙寅,劉裕害前將軍諸葛長民及其弟輔
 國大將軍黎民、從弟寧朔將軍秀之。戊寅,加劉裕鎮西將軍、豫州刺史。林邑范胡達寇九真,交州刺史杜慧度斬之。夏四月壬戌,罷臨沂、湖熟皇后脂澤田四十頃,以賜貧人,弛湖池之禁。封鎮北將軍魯宗之為南陽郡公。秋七月,朱齡石克成都,斬譙縱,益州平。九月,封劉裕次子義真為桂陽公。冬十二月,安平王球之薨。是歲,高句麗、倭國及西南夷銅頭大師並獻方物。



 十年春三月戊寅,地震。夏六月,乞伏熾盤帥師伐禿髮傉檀,滅之。秋七月,淮北大風,壞廬舍。九月丁巳朔,日有蝕之。林邑遣使來獻方物。是歲,城東府。



 十一年春正月,荊州刺史司馬休之、雍州刺史魯宗之並舉兵貳於劉裕,裕帥師討之。庚午,大赦。丁丑,以吏部尚書謝裕為尚書左僕射。二月丁未,姚興死,子泓嗣偽位。三月辛巳,淮陵王蘊薨。壬午,劉裕及休之戰于江津,休之敗,奔襄陽。夏四月乙卯,青、冀二州刺史劉敬宣為其參軍司馬道賜所害。五月甲申,彗星二見。甲午,休之、宗之出奔于姚泓。論平蜀功,封劉裕子義隆彭城公,朱齡石豐城公。己酉,霍山崩,出銅鐘六枚。秋七月丙戌,京師大水,壞太廟。辛亥晦,日有蝕之。八月丁未,尚書左僕射謝裕卒,以尚書右僕射劉穆之為尚書左僕射。九月己
 亥,大赦。



 十二年春正月,姚泓使其將魯軌寇襄陽,雍州刺史趙倫之擊走之。二月,加劉裕中外大都督。夏六月,赫連勃勃攻姚泓秦州,陷之。己酉,新除尚書令、都鄉亭侯劉柳卒。秋八月,劉裕及瑯邪王德文帥眾伐姚泓。丙午,大赦。冬十月丙寅,姚泓將姚光以洛陽降。己丑,遣兼司空、高密王恢之脩謁五陵。



 十三年春正月甲戌朔,日有蝕之。二月,涼武昭王李玄盛薨,世子士業嗣位為涼州牧、涼公。三月,龍驤將軍王鎮惡大破姚泓將姚紹于潼關。夏,劉裕敗魏將鵝青于
 河曲,斬青裨將阿薄干。是月,涼公李士業大敗沮渠蒙遜于鮮支澗。五月,劉裕克潼關。丁亥,會稽王脩之薨。六月癸亥,林邑獻馴象、白鸚鵡。秋七月,劉裕克長安,執姚泓,收其彞器,歸諸京師。南海賊徐道期陷廣州,始興相劉謙之討平之。冬十一月辛未,左僕射、前將軍劉穆之卒。



 十四年春正月辛巳,大赦。青州刺史沈田子害龍驤將軍王鎮惡于長安。夏六月,劉裕為相國,進封宋公。冬十月,以涼公士業為鎮西將軍,封酒泉公。十一月,赫連勃勃大敗王師于青泥北。雍州刺史朱齡石焚長安宮殿,
 奔于潼關。尋又大潰,齡石死之。十二月戊寅,帝崩于東堂,時年三十七。葬休平陵。



 帝不惠,自少及長,口不能言,雖寒暑之變,無以辯也。凡所動止,皆非己出。故桓玄之纂,因此獲全。初讖云「昌明之後有二帝」,劉裕將為禪代,故密使王韶之縊帝而立恭帝,以應二帝云。



 恭帝諱德文,字德文,安帝母弟也。初封琅邪王,歷中軍將軍、散騎常侍、衛將軍、開府儀同三司,加侍中,領司徒、錄尚書六條事。元興初,遷車騎大將軍。桓玄執政,進位太宰,加兗冕之服,綠綟綬。玄篡位,以帝為石陽縣公,與
 安帝俱居尋陽。及玄敗,隨至江陵。玄死,桓振奄至,躍馬奮戈,直至階下,瞋目謂安帝曰:「臣門戶何負國家,而屠滅若是?」帝乃下床謂振曰:「此豈我兄弟意邪!」振乃下馬致拜。振平,復為瑯邪王,又領徐州刺史,尋拜大司馬,領司徒,加殊禮。義熙二年,置左右長史、司馬、從事中郎四人,加羽葆鼓吹。十二年,詔曰:「大司馬明德懋親,大尉道勳光大,並徽序彞倫,燮和二氣,髦俊引領,思佐鼎飪。而雅尚沖挹,四門弗闢,誠合大雅謙虛之道,實違急賢贊世之務。昔蒲輪載征,異人並出,東平開府,奇士嚮臻,濟濟之盛,朕有欽焉。可敕二府,依舊辟召,必將明易又俊乂,
 嗣軌前賢矣。」於是始辟召掾屬。時太尉裕都督中外諸軍,詔曰:「大司馬地隆任重,親賢莫貳。雖府受節度,可身無致敬。」劉裕之北征也,帝上疏,請帥所蒞,啟行戎路,修敬山陵。朝廷從之,乃與裕俱發。及有司以即戎不得奉辭陵廟,帝復上疏曰:「臣推轂閫外,將革寒暑,不獲展情埏璲,私心罔極。伏願天慈,特垂聽許,使臣微誠粗申,即路無恨。」許之。及姚泓滅,歸于京都。十四年十二月戊寅,安帝崩。劉裕矯稱遺詔曰:「唯我有晉,誕膺明命,業隆九有,光宅四海。朕以不德,屬當多難,幸賴宰輔,拯厥顛覆。仍恃保祐,克黜禍亂,遂冕旒辰極,混一六合。方憑阿衡,惟
 新洪業,而遘疾大漸,將遂弗興。仰惟祖宗靈命。親賢是荷。咨爾大司馬、瑯邪王,體自先皇,明德光懋,屬惟儲貳,眾望攸集。其君臨晉邦,奉系宗祀,允執其中,燮和天下。闡揚末誥,無廢我高祖之景命。」是日,即帝位,大赦。



 元熙元年春正月壬辰朔,改元。以山陵未厝,不朝會。立皇后褚氏。甲午,征劉裕還朝。戊戌,有星孛于太微西籓。庚申,葬安皇帝于休平陵。帝受朝,懸而不樂。以驃騎將軍劉道憐為司空。秋八月,劉裕移鎮壽陽。以劉懷慎為前將軍、北徐州刺史,鎮彭城。九月,劉裕自解揚州。冬十月乙酉,裕以其子桂陽公義真為揚州刺史。十一月丁
 亥朔,日有蝕之。十二月辛卯,裕加殊禮。己卯,太史奏,黑龍四見于東方。



 二年夏六月壬戌,劉裕至於京師。傅亮承裕密旨,諷帝禪位,草詔,請帝書之。帝欣然謂左右曰:「晉氏久已失之,今復何恨。」乃書赤紙為詔。甲子,遂遜於琅邪第。劉裕以帝為零陵王,居於秣陵,行晉正朔,車旗服色一如其舊,有其文而不備其禮。帝自是之後,深慮禍機,褚后常在帝側,飲食所資,皆出褚后,故宋人莫得伺其隙,宋永初二年九月丁丑,裕使后兄叔度請后,有間,兵人踰垣而入,弒帝于內房。時年三十六。謚恭皇帝,葬沖平陵。



 帝幼
 時性頗忍急,及在籓國,曾令善射者射馬為戲。既而有人云:「馬者國姓,而自殺之,不祥之甚。」帝亦悟,甚悔之。其後復深信浮屠道,鑄貨千萬,造丈六金像,親於瓦官寺迎之,步從十許里。安帝既不惠,帝每侍左右,消息溫涼寢食之節,以恭謹聞,時人稱焉。始,元帝以丁丑歲稱晉王,置宗廟,使郭璞筮之,云「享二百年。」自丁丑至禪代之歲,年在庚申,為一百四歲。然丁丑始係西晉,庚申終入宋年,所餘惟一百有二歲耳。璞蓋以百二之期促,故婉而倒之為二百也。



 史臣曰:安帝即位之辰,鐘無妄之日,道子、元顯並傾朝
 政,主昏臣亂,未有如斯不亡者也。雖有手握戎麾,心存舊國,回首無良,忽焉蕭散。於是桓玄乘釁,勢逾飆指,六師咸泯,只馬徂遷。是以宋高非典午之臣,孫恩豈金行之寇。若乃世遇顛覆,則恭皇斯甚。於越之民,詎熏丹穴,會稽之侶,寧嘆人臣。去皇屋而歸來,灑丹書而不恨。夫五運攸革,三微數盡,猶高秋凋候,理之自然。觀其搖落,人有為之流漣者也。



 贊曰:安承流湎,大盜斯張。恭乃寓命,他人是綱。猶存周赧,始立懷王。虛尊假號,異術同亡。



\end{pinyinscope}