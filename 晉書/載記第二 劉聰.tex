\article{載記第二 劉聰}

\begin{pinyinscope}

 劉
 聰



 劉聰,字玄明,一名載,元海第四子也。母曰張夫人。初,聰之在孕也,張氏夢日入懷,寤而以告,元海曰:「此吉徵也,慎勿言。」十五月而生聰焉,夜有白光之異。形體非常,左耳有一白毫,長二尺餘,甚光澤。幼而聰悟好學,博士朱紀大奇之。年十四,究通經史,兼綜百家之言,《孫吳兵法》靡不誦之。工草隸,善屬文,著述懷詩百餘篇、賦頌五十
 餘篇。十五習擊刺,猿臂善射,彎弓三百斤,膂力驍捷,冠絕一時。太原王渾見而悅之,謂元海曰:「此兒吾所不能測也。」



 弱冠游于京師,名士莫不交結,樂廣、張華尤異之也。新興太守郭頤辟為主簿,舉良將,入為驍騎別部司馬,累遷右部都尉,善於撫接,五部豪右無不歸之。河間王顒表為赤沙中郎將。聰以元海在鄴,懼為成都王穎所害,乃亡奔成都王,拜右積弩將軍,參前鋒戰事。元海為北單于,立為右賢王,隨還右部。及即大單于位,更拜鹿蠡王。既殺其兄和,群臣勸即尊位。聰初讓其弟北海王乂,乂與公卿泣涕固請,聰久而許之,曰:「乂及群公正
 以四海未定,禍難尚殷,貪孤年長故耳。此國家之事,孤敢不祗從。今便欲遠遵魯隱,待乂年長,復子明辟。」於是以永嘉四年僭即皇帝位,大赦境內,改元光興。尊元海妻單氏曰皇太后,其母張氏為帝太后,乂為皇太弟,領大單于、大司徒,立其妻呼延氏為皇后,封其子粲為河內王,署使持節、撫軍大將軍、都督中外諸軍事,易河間王,翼彭城王,悝高平王。遣粲及其征東王彌、龍驤劉曜等率眾四萬,長驅入洛川,遂出轘轅,周旋梁、陳、汝、潁之間,陷壘壁百餘。以其司空劉景為大司馬,左光祿劉殷為大司徒,右光祿王育為大司空。偽太后單氏姿色
 絕麗,聰蒸焉。單即乂之母也,乂屢以為言,單氏慚恚而死,聰悲悼無已。後知其故,乂之寵因此漸衰,然猶追念單氏,未便黜廢。又尊母為皇太后。



 署其衛尉呼延晏為使持節、前鋒大都督、前軍大將軍。配禁兵二萬七千,自宜陽入洛川,命王彌、劉曜及鎮軍石勒進師會之。晏比及河南,王師前後十二敗,死者三萬餘人。彌等未至,晏留輜重于張方故壘,遂寇洛陽,攻陷平昌門,焚東陽、宣陽諸門及諸府寺。懷帝遣河南尹劉默距之,王師敗于社門。晏以外繼不至,出自東陽門,掠王公已下子女二百餘人而去。時帝將濟河東遁,具船于洛水,晏盡焚之,
 還于張方故壘。王彌、劉曜至,復與晏會圍洛陽。時城內飢甚,人皆相食,百官分散,莫有固志。宣陽門陷,彌、晏入於南宮,升太極前殿,縱兵大掠,悉收宮人、珍寶。曜於是害諸王公及百官已下三萬餘人,於洛水北築為京觀。遷帝及惠帝羊后、傳國六璽于平陽。聰大赦,改年嘉平,以帝為特進、左光祿大夫、平阿公。



 遣其平西趙染、安西劉雅率騎二萬攻南陽王模于長安,粲、曜率大眾繼之。染敗王師于潼關,將軍呂毅死之。軍至于下邽,模乃降染。染送模於粲,粲害模及其子范陽王黎,送衛將軍梁芬、模長史魯繇、兼散騎常侍杜驁、辛謐及北宮純等于
 平陽。聰以粲之害模也,大怒。粲曰:「臣殺模本不以其晚識天命之故,但以其晉氏肺腑,洛陽之難不能死節,天下之惡一也,故誅之。」聰曰:「雖然,吾恐汝不免誅降之殃也。夫天道至神,理無不報。」



 署劉曜為車騎大將軍、開府儀同三司、雍州牧,改封中山王,鎮長安,王彌為大將軍,封齊公。尋而石勒等殺彌於己吾而並其眾,表彌叛狀。聰大怒,遣使讓勒專害公輔,有無上之心,又恐勒之有二志也,以彌部眾配之。劉曜既據長安,安定太守賈疋及諸氐羌皆送質任,唯雍州刺史麴特、新平太守竺恢固守不降。護軍麴允、頻陽令梁肅自京兆南山將奔安
 定,遇疋任子於陰密,擁還臨涇,推疋為平南將軍,率眾五萬,攻曜於長安,扶風太守梁綜及麴特、竺恢等亦率眾十萬會之。曜遣劉雅、趙染來距,敗績而還。曜又盡長安銳卒與諸軍戰于黃丘,曜眾大敗,中流矢,退保甘渠。杜人王禿、紀特等攻劉粲于新豐,粲還平陽。曜攻陷池陽,掠萬餘人歸于長安。時閻鼎等奉秦王為皇太子,入於雍城,關中戎晉莫不響應。



 聰后呼延氏死,將納其太保劉殷女,其弟乂固諫。聰更訪之於太宰劉延年、大傅劉景,景等皆曰:「臣常聞太保自云周劉康公之後,與聖氏本源既殊,納之為允。」聰大悅,使其兼大鴻臚李弘拜
 殷二女為左右貴嬪,位在昭儀上。又納殷女孫四人為貴人,位次貴嬪。謂弘曰:「此女輩皆姿色超世,女德冠時,且太保於朕實自不同,卿意安乎?」弘曰:「太保胤自有周,與聖源實別,陛下正以姓同為恨耳。且魏司空東萊王基當世大儒,豈不達禮乎!為子納司空太原王沈女,以其姓同而源異故也。」聰大悅,賜弘黃金六十斤,曰:「卿當以此意諭吾子弟輩。」於是六劉之寵傾於後宮,聰稀復出外,事皆中黃門納奏,左貴嬪決之。



 聰假懷帝儀同三司,封會稽郡公,庾氏等以次加秩。聰引帝入宴,謂帝曰:「卿為豫章王時,朕嘗與王武子相造,武子示朕於卿,卿
 言聞其名久矣。以卿所製樂府歌示朕,謂朕曰:『聞君善為辭賦,試為看之。』朕時與武子俱為《盛德頌》,卿稱善者久之。又引朕射于皇堂,朕得十二籌,卿與武子俱得九籌,卿贈朕柘弓、銀研,卿頗憶否?」帝曰:「臣安敢忘之,但恨爾日不早識龍顏。」聰曰:「卿家骨肉相殘,何其甚也?」帝曰:「此殆非人事,皇天之意也。大漢將應乾受歷,故為陛下自相驅除。且臣家若能奉武皇之業,九族敦睦,陛下何由得之!」至日夕乃出,以小劉貴人賜帝,謂帝曰:「此名公之孫,今特以相妻,卿宜善遇之。」拜劉為會稽國夫人。



 遣其鎮北靳沖寇太原,平北卜珝率眾繼之。沖攻太原不
 剋,而歸罪於珝,輒斬之。聰聞之,大怒曰:「此人朕所不得加刑,沖何人哉!」遣其御史中丞浩衍持節斬沖。左都水使者襄陵王攄坐魚蟹不供,將作大匠望都公靳陵坐溫明、徽光二殿不成,皆斬于東市。聰游獵無度,常晨出暮歸,觀漁於汾水,以燭繼晝。中軍王彰諫曰:「今大難未夷,餘晉假息,陛下不懼白龍魚服之禍,而昏夜忘歸。陛下當思先帝創業之艱難,嗣承之不易,鴻業已爾,四海屬情,何可墜之於垂成,隳之於將就!比竊觀陛下所為,臣實痛心疾首有日矣。且愚人係漢之心未專,而思晉之懷猶盛,劉琨去此咫尺之間,狂狷刺客息頃而至。帝
 王輕出,一夫敵耳。願陛下改往修來,則憶兆幸甚。」聰大怒,命斬之。上夫人王氏叩頭乞哀,乃囚之詔獄。聰母以聰刑怒過差,三日不食,弟乂、子粲並與切諫。聰怒曰:「吾豈桀、紂、幽、厲乎,而汝等生來哭人!」其太宰劉延年及諸公卿列侯百有餘人,皆免冠涕泣固諫曰:「光文皇帝以聖武膺期,創建鴻祚,而六合未一,夙世升遐。陛下睿德自天,龍飛紹統,東平洛邑,南定長安,真可謂功高周成,德超夏啟。往也唐虞,今則陛下,歷觀書記,未有此比。而頃頻以小務不供而斬王公,直言忤旨,便囚大將,游獵無度,機管不修,臣等竊所未解,臣等所以破肝糜胃
 忘寢與食者也。」聰乃赦彰。



 麴特等圍長安,劉曜連戰敗績,乃驅掠士女八萬餘口退還平陽,因攻司徒傅祗于三渚,使其右將軍劉參攻郭默于懷城。祗病卒,城陷,遷祗孫純、粹并二萬餘戶于平陽縣。聰贈祗太保,純、粹皆給事中,謂祗子暢曰:「尊公雖不達天命,然各忠其主,吾亦有以亮之。但晉主已降,天命非人所支,而虔劉南鄙,沮亂邊萌,此其罪也。以元惡之種而贈同勳舊,逆臣之孫荷榮禁闥,卿知皇漢之德弘曠以不?」暢曰:「陛下每嘉先臣,不以小臣之故而虧其忠節,及是恩也,自是明主伐國弔人之義,臣輒同萬物,未敢謝生於自然。」



 聰遣
 劉粲、劉曜等攻劉琨於晉陽,琨使張喬距之,戰於武灌,喬敗績,死之,晉陽危懼。太原太守高喬、琨別駕郝聿以晉陽降粲。琨與左右數十騎,攜其妻子奔于趙郡之亭頭,遂如常山。粲、曜入于晉陽。先是,琨與代王猗盧結為兄弟,乃告敗於猗盧,且乞師。猗盧遣子日利孫、賓六須及將軍衛雄、姬澹等率眾數萬攻晉陽,琨收散卒千餘為之鄉導,猗盧率眾六萬至于狼猛。曜及賓六須戰于汾東,曜墜馬,中流矢,身被七創。討虜傅武以馬授曜,曜曰:「當今危亡之極,人各思免。吾創已重,自分死此矣。」武泣曰:「武小人,蒙大王識拔,以至於是,常思效命,今其時
 矣。且皇室始基,大難未弭,天下何可一日無大王也。」於是扶曜乘馬,驅令渡汾,迴而戰死。曜入晉陽,夜與劉粲等掠百姓,踰蒙山遁歸。猗盧率騎追之,戰於藍谷,粲敗績,斬其征虜邢延,獲其鎮北劉豐。琨收合離散,保于陽曲,猗盧戍之而還。



 正旦,聰宴于光極前殿,逼帝行酒,光祿大夫庾氏、王俊等起而大哭,聰惡之。會有告氏等謀以平陽應劉琨者,聰遂鴆帝而誅氏、俊,復以賜帝劉夫人為貴人,大赦境內殊死已下。立左貴嬪劉氏為皇后。聰將為劉氏起䳨儀殿於後庭,廷尉陳元達諫曰:「臣聞古之聖王愛國如家,故皇天亦祐之如子。夫天生蒸民
 而樹之君者,使為之父母以刑賞之,不欲使殿屎黎元而蕩逸一人。晉氏闇虐,視百姓如草芥,故上天剿絕其祚。乃眷皇漢,蒼生引領息肩,懷更蘇之望有日矣。我高祖光文皇帝靖言惟茲,痛心疾首,故身衣大布,居不重茵;先皇后嬪服無綺彩。重逆群臣之請,故建南北宮焉。今光極之前足以朝群后饗萬國矣,昭德、溫明已後足可以容六宮,列十二等矣。陛下龍興已來,外殄二京不世之寇,內興殿觀四十餘所,重之以饑饉疾疫,死亡相屬,兵疲於外,人怨於內,為之父母固若是乎!伏聞詔旨,將營䳨儀,中宮新立,誠臣等樂為子來者也。竊以大難
 未夷,宮宇粗給,今之所營,尤實非宜。臣聞太宗承高祖之業,惠呂息役之後,以四海之富,天下之殷,尚以百金之費而輟露臺,歷代垂美,為不朽之迹。故能斷獄四百,擬於成康。陛下之所有,不過太宗二郡地耳,戰守之備者,豈僅匈奴、南越而已哉!孝文之廣,思費如彼;陛下之狹,欲損如此。愚臣所以敢昧死犯顏色,冒不測之禍者也。」聰大怒曰:「吾為萬機主,將營一殿,豈問汝鼠子乎!不殺此奴,沮亂朕心,朕殿何當得成邪!將出斬之,并其妻子同梟東市,使群鼠共穴。」時在逍遙園李中堂,元達抱堂下樹叫曰:「臣所言者,社稷之計也,而陛下殺臣。若死
 者有知,臣要當上訴陛下於天,下訴陛下於先帝。朱雲有云:『臣得與龍逢、比干游於地下足矣。』未審陛下何如主耳!」元達先鎖腰而入,及至,即以鎖繞樹,左右曳之不能動。聰怒甚。劉氏時在後堂,聞之,密遣中常侍私敕左右停刑,於是手疏切諫,聰乃解,引元達而謝之,易逍遙園為納賢園,李中堂為愧賢堂。



 時愍帝即位于長安,聰遣劉曜及司隸喬智明、武牙李景年等寇長安,命趙染率眾赴之。時大都督麴允據黃白城,累為曜、染所敗。染謂曜曰:「麴允率大眾在外,長安可襲而取之。得長安,黃白城自服。願大王以重眾守此,染請輕騎襲之。」曜乃承
 制加染前鋒大都督、安南大將軍,以精騎五千配之而進。王師敗於渭陽,將軍王廣死之。染夜入長安外城,帝奔射鴈樓,染焚燒龍尾及諸軍營,殺掠千餘人,旦退屯逍遙園。麴允率眾襲曜,連戰敗之。曜入粟邑,遂歸平陽。



 時流星起於牽牛,入紫微,龍形委蛇,其光照地,落于平陽北十里。視之,則有肉長三十步,廣二十七步,臭聞于平陽,肉旁常有哭聲,晝夜不止。聰甚惡之,延公卿已下問曰:「朕之不德,致有斯異,其各極言,勿有所諱。」陳元達及博士張師等進對曰:「星變之異,其禍行及,臣恐後庭有三后之事,亡國喪家,靡不由此,願陛下慎之。」聰曰:「此
 陰陽之理,何關人事!」既而劉氏產一蛇一猛獸,各害人而走,尋之不得,頃之,見在隕肉之旁。俄而劉氏死,乃失此肉,哭聲亦止。自是後宮亂寵,進御無序矣。



 聰以劉易為太尉。初置相國,官上公,有殊勳德者死乃贈之。於是大定百官,置太師、丞相,自大司馬以上七公,位皆上公,綠綟綬,遠遊冠。置輔漢,都護,中軍,上軍,輔軍,鎮、衛京,前、後、左、右、上、下軍,輔國,冠軍,龍驤,武牙大將軍,營各配兵二千,皆以諸子為之。置左右司隸,各領戶二十餘萬,萬戶置一內史,凡內史四十三。單于左右輔,各主六夷十萬落,萬落置一都尉。省吏部,置左右選曹尚書。自司隸
 以下六官,皆位次僕射。置御史大夫及州牧,位皆亞公。以其子粲為丞相、領大將軍、錄尚書事,進封晉王,食五都。劉延年錄尚書六條事,劉景為太師,王育為太傅,任顗為太保,馬景為大司徒,朱紀為大司空,劉曜為大司馬。



 曜復次渭汭,趙染次新豐。索綝自長安東討染,染狃于累捷,有輕綝之色。長史魯徽曰:「今司馬鄴君臣自以逼僭王畿,雄劣不同,必致死距我,將軍宜整陣案兵以擊之,弗可輕也。困獸猶鬥,況於國乎!」染曰:「以司馬模之彊,吾取之如拉朽。索綝小豎,豈能污吾馬蹄刀刃邪!要擒之而後食。」晨率精騎數百,馳出逆之,戰于城西,敗績
 而歸,悔曰:「吾不用魯徽之言,以至於此,何面見之!」於是斬徽。徽臨刑謂染曰:「將軍愎諫違謀,戇而取敗,而復忌前害勝,誅戮忠良,以逞愚忿,亦何顏面瞬息世間哉!袁紹為之於前,將軍踵之於後,覆亡敗喪,亦當相尋,所恨不得一見大司馬而死。死者無知則已;若其有知,下見田豐為徒,要當訴將軍於黃泉,使將軍不得服床枕而死。」叱刑者曰:「令吾面東向。」大司馬曜聞之曰:「蹄涔不容尺鯉,染之謂也。」



 曜還師攻郭默于懷城,收其米粟八十萬斛,列三屯以守之。聰遣使謂曜曰:「今長安假息,劉琨游魂,此國家所尤宜先除也。郭默小醜,何足以勞公神
 略,可留征虜將軍貝丘王翼光守之,公其還也。」於是曜歸薄阪。俄而徵曜輔政。



 趙染寇北地,夢魯徽大怒,引弓射之,染驚悸而寤。旦將攻城,中弩而死。



 聰以粲為相國,總百揆,省丞相以並相國。平陽地震,烈風拔樹發屋。光義人羊充妻產子二頭,其兄竊而食之,三日而死。聰以其太廟新成,大赦境內,改年建元。雨血於其東宮延明殿,徹瓦在地者深五寸。劉乂惡之,以訪其太師盧志、太傅崔瑋、太保許遐。志等曰:「主上往以殿下為太弟者,蓋以安眾望也,志在晉王久矣,王公已下莫不希旨歸之。相國之位,自魏武已來,非復人臣之官,主上本發明詔,
 置之為贈官,今忽以晉王居之,羽儀威尊踰於東宮,萬機之事無不由之,置太宰、大將軍及諸王之營以為羽翼,此事勢去矣,殿下不得立明也。然非止不得立而已,不測之危厄在於旦夕,宜早為之所。四衛精兵不減五千,餘營諸王皆年齒尚幼,可奪而取之。相國輕佻,正可煩一刺客耳。大將軍無日不出,其營可襲而得也。殿下但當有意,二萬精兵立便可得,鼓行向雲龍門,宿衛之士孰不倒戈奉迎,大司馬不慮為異也。」乂弗從,乃止。



 聰如中護軍靳準第,納其二女為左右貴嬪,大曰月光,小曰月華,皆國色也。數月,立月光為皇后。



 東宮舍人荀裕
 告盧志等勸乂謀反,乂不從之狀。聰於是收志、瑋、遐於詔獄,假以他事殺之。使冠威卜抽監守東宮,禁乂朝賀。乂憂懼不知所為,乃上表自陳,乞為黔首,并免諸子之封,褒美晉王粲宜登儲副,抽又抑而弗通。



 其青州刺史曹嶷攻汶陽關、公丘,陷之,害齊郡太守徐浮,執建威劉宣,齊魯之間郡縣壘壁降者四十餘所。嶷遂略地,西下祝阿、平陰,眾十餘萬,臨河置戍,而歸于臨淄。嶷於是遂雄據全齊之志。石勒以嶷之懷二也,請討之。聰又憚勒之并齊,乃寢而弗許。



 劉曜濟自盟津,將攻河南,將軍魏該奔于一泉塢。曜進攻李矩於滎陽,矩遣將軍李平
 師於成皋,曜覆而滅之。矩恐,送質請降。



 時聰以其皇后靳氏為上皇后,立貴妃劉氏為左皇后,右貴嬪靳氏為右皇后。左司隸陳元達以三后之立也,極諫,聰不納,乃以元達為右光祿大夫,外示優賢,內實奪其權也。於是太尉范隆、大司馬劉丹、大司空呼延晏、尚書令王鑒等皆抗表遜位,以讓元達。聰乃以元達為御史大夫、儀同三司。



 劉曜寇長安,頻為王師所敗。曜曰:「彼猶強盛,弗可圖矣。」引師而歸。



 聰宮中鬼夜哭,三日而聲向右司隸寺,乃止。其上皇后靳氏有淫穢之行,陳元達奏之。聰廢靳,靳慚恚自殺。靳有殊寵,聰迫於元達之勢,故廢之。既而
 追念其姿色,深仇元達。



 劉曜進師上黨,將攻陽曲,聰遣使謂曜曰:「長安擅命,國家之深恥也。公宜以長安為先,陽曲一委驃騎。天時人事,其應至矣,公其亟還。」曜迴滅郭邁,朝于聰,遂如蒲阪。



 平陽地震,雨血於東宮,廣袤頃餘。



 劉曜又進軍,屯於粟邑。麴允飢甚,去黃白而軍於靈武。曜進攻上郡,太守張禹與馮翊太守梁肅奔于允吾。於是關右翕然,所在應曜。曜進據黃阜。



 聰武庫陷入地一丈五尺。時聰中常侍王沈、宣懷、俞容,中宮僕射郭猗,中黃門陵脩等皆寵幸用事。聰游宴後宮,或百日不出,群臣皆因沈等言事,多不呈聰,率以其意愛憎而決之,
 故或有勳舊功臣而弗見敘錄,姦佞小人數日而便至二千石者。軍旅無歲不興,而將士無錢帛之賞,後宮之家賜齎及於僮僕,動至數千萬。沈等車服宅宇皆踰于諸王,子弟、中表布衣為內史令長者三十餘人,皆奢僭貪殘,賊害良善。靳準合宗內外諂以事之。



 郭猗有憾於劉乂,謂劉粲曰:「太弟於主上之世猶懷不逞之志,此則殿下父子之深仇,四海蒼生之重怨也。而主上過垂寬仁,猶不替二尊之位,一旦有風塵之變,臣竊為殿下寒心。且殿下高祖之世孫,主上之嫡統,凡在含齒,孰不係仰。萬機事大,何可與人!臣昨聞太弟與大將軍相見,極
 有言矣,若事成,許以主上為在太上皇,大將軍為皇太子。乂又許衛軍為大單于,二王已許之矣。二王居不疑之地,並握重兵,以此舉事,事何不成!臣謂二王茲舉,禽獸之不若也。背父親人,人豈親之!今又茍貪其一切之力耳,事成之後,主上豈有全理!殿下兄弟故在忘言,東宮、相國、單于在武陵兄弟,何肯與人!許以三月上巳因宴作難,事淹變生,宜早為之所。《春秋傳》曰:『蔓草猶不可除,況君之寵弟乎!』臣屢啟主上,主上性敦友于,謂臣言不實。刑臣刀鋸之餘,而蒙主上、殿下成造之恩,故不慮逆鱗之誅,每所聞必言,冀垂採納。臣當入言之。願殿下不
 泄,密表其狀也。若不信臣言,可呼大將軍從事中郎王皮、衛軍司馬劉惇,假之恩顧,通其歸善之路以問之,必可知也。」粲深然之。猗密謂皮、惇曰:「二王逆狀,主、相已具知之矣,卿同之乎?」二人驚曰?:「無之。」猗曰:「此事必無疑,吾憐卿親舊并見族耳。」於是歔欷流涕。皮、惇大懼,叩頭求哀。猗曰:「吾為卿作計,卿能用不?」二人皆曰:「謹奉大人之教。」猗曰:「相國必問卿,卿但云有之。若責卿何不先啟,卿即答云:『臣誠負死罪,然仰惟主上聖性寬慈,殿下篤於骨肉,恐言成詿偽故也。』」皮、惇許諾。粲俄而召問二人,至不同時,而辭若畫一,粲以為信然。



 初,靳準從妹為乂孺
 子,淫于侍人,乂怒殺之,而屢以嘲準。準深慚恚,說粲曰:「東宮萬機之副,殿下宜自居之,以領相國,使天下知早有所系望也。」至是,準又說粲曰:「昔孝成距子政之言,使王氏卒成篡逆,可乎?」粲曰:「何可之有!」準曰:「然,誠如聖旨。下官亟欲有所言矣,但以德非更生,親非皇宗,恐忠言暫出,霜威已及,故不敢耳。」粲曰:「君但言之。」準曰:「聞風塵之言,謂大將軍、衛將軍及左右輔皆謀奉太弟,剋季春構變,殿下宜為之備。不然,恐有商臣之禍。」粲曰:「為之奈何?」準曰:「主上愛信於太弟,恐卒聞未必信也。如下官愚意,宜緩東宮之禁固,勿絕太弟賓客,使輕薄之徒得與
 交游。太弟既素好待士,必不思防此嫌,輕薄小人不能無逆意以勸太弟之心。小人有始無終,不能如貫高之流也。然後下官為殿下露表其罪,殿下與太宰拘太弟所與交通者考問之,窮其事原,主上必以無將之罪罪之。不然,今朝望多歸太弟,主上一旦晏駕,恐殿下不得立矣。」於是粲命卜抽引兵去東宮。



 聰自去冬至是,遂不復受朝賀,軍國之事一決於粲,唯發中旨殺生除授,王沈、郭猗等意所欲皆從之。又立市於後庭,與宮人宴戲,或三日不醒。聰臨上秋閣,誅其特進綦毋達,太中大夫公師彧,尚書王琰、田歆,少府陳休,左衛卜崇,大司農朱
 誕等,皆群閹所忌也。侍中卜幹泣諫聰曰:「陛下方隆武宣之化,欲使幽谷無考槃,奈何一旦先誅忠良,將何以垂之於後!昔秦愛三良而殺之,君子知其不霸。以晉厲之無道,尸三卿之後,猶有不忍之心,陛下如何忽信左右愛憎之言,欲一日尸七卿!詔尚在臣間,猶未宣露,乞垂昊天之澤,回雷霆之威。且陛下直欲誅之耳,不露其罪名,何以示四海!此豈是帝王三訊之法邪!」因叩頭流血。王沈叱幹曰:「卜侍中欲距詔乎?」聰拂衣而入,免幹為庶人。



 太宰劉易及大將軍劉敷、御史大夫陳元達、金紫光祿大夫王延等詣闕諫曰:「臣聞善人者,乾坤之紀,政
 教之本也。邪佞者,宇宙之螟螣,王化之蟊賊也。故文王以多士基周,桓靈以群閹亡漢,國之興亡,未有不由此也。自古明王之世,未嘗有宦者與政,武、元、安、順豈足為故事乎!今王沈等乃處常伯之位,握生死與奪於中,勢傾海內,愛憎任之,矯弄詔旨,欺誣日月,內諂陛下,外佞相國,威權之重,侔於人主矣。王公見之駭目,卿宰望塵下車,銓衡迫之,選舉不復以實,士以屬舉,政以賄成,多樹姦徒,殘毒忠善。知王琰等忠臣,必盡節於陛下,懼其姦萌發露,陷之極刑。陛下不垂三察,猥加誅戮,怨感穹蒼,痛入九泉,四海悲惋,賢愚傷懼。沈等皆刀鋸之餘,背
 恩忘義之類,豈能如士人君子感恩展效,以答乾澤也。陛下何故親近之?何故貴任之?昔齊桓公任易牙而亂,孝懷委黃皓而滅,此皆覆車於前,殷鑒不遠。比年地震日蝕,雨血火災,皆沈等之由。願陛下割翦凶醜與政之流,引尚書、御史朝省萬機,相國與公卿五日一入,會議政事,使大臣得極其言,忠臣得逞其意,則眾災自弭,和氣呈祥。今遺晉未殄,巴蜀未賓,石勒潛有跨趙魏之志,曹嶷密有王全齊之心,而復以沈等助亂大政,陛下心腹四支何處無患!復誅巫咸,戮扁鵲,臣恐遂成桓侯膏肓之疾,後雖欲療之,其如病何!請免沈等官,付有司定
 罪。」聰以表示沈等,笑曰:「是兒等為元達所引,遂成癡也。」寢之。沈等頓首泣曰:「臣等小人,過蒙陛下識拔,幸得備灑掃宮閣,而王公朝士疾臣等如仇讎,又深恨陛下。願收大造之恩,以臣等膏之鼎鑊,皇朝上下自然雍穆矣。」聰曰:「此等狂言恒然,卿復何足恨乎!」更以訪粲,粲盛稱沈等忠清,乃心王室。聰大悅,封沈為列侯。太宰劉易詣闕,又上疏固諫。聰大怒,手壞其表,易遂忿恚而死,元達哭之悲慟,曰:「人之云亡,邦國殄悴。吾既不復能言,安用此默默生乎!」歸而自殺。



 北地饑甚,人相食啖,羌酋大軍須運糧以給麴昌,劉雅擊敗之。麴允與劉曜戰于磻
 石谷,王師敗績,允奔靈武。平陽大饑,流叛死亡十有五六。石勒遣石越率騎二萬,屯于并州,以懷撫叛者。聰使黃門侍郎喬詩讓勒,勒不奉命,潛結曹嶷,規為鼎峙之勢。



 聰立上皇后樊氏,即張氏之侍婢也。時四后之外,佩皇后璽綬者七人,朝廷內外無復綱紀,阿諛日進,貨賄公行,軍旅在外,饑疫相仍,後宮賞賜動至千萬。劉敷屢泣言之,聰不納,怒曰:「爾欲得使汝公死乎?朝朝夕夕生來哭人!」敷憂忿發病而死。



 河東大蝗,唯不食黍豆。靳準率部人收而埋之,哭聲聞於十餘里,後乃鑽土飛出,復食黍豆。平陽饑甚,司隸部人奔于冀州二十萬戶,石越
 招之故也。犬與豕交于相國府門,又交于宮門,又交司隸、御史門。有豕著進賢冠,升聰坐。犬冠武冠,帶綬,與豕並升。俄而鬥死殿上。宿衛莫有見其入者。而聰昏虐愈甚,無誡懼之心。宴群臣于光極前殿,引見其太弟乂,容貌毀悴,鬢髮蒼然,涕泣陳謝。聰亦對之悲慟,縱酒極歡,待之如初。



 劉曜陷長安外城,愍帝使侍中宋敞送箋于曜,帝肉袒牽羊,輿櫬銜璧出降。及至平陽,聰以帝為光祿大夫、懷安侯,使粲告于太廟,大赦境內,改年麟嘉。麴允自殺。



 聰東宮四門無故自壞,後內史女人化為丈夫。時聰子約死,一指猶暖,遂不殯殮。及蘇,言見元海於不
 周山,經五日,遂復從至崑崙山,三日而復返於不周,見諸王公卿將相死者悉在,宮室甚壯麗,號曰蒙珠離國。元海謂約曰:「東北有遮須夷國,無主久,待汝父為之。汝父後三年當來,來後國中大亂相殺害,吾家死亡略盡,但可永明輩十數人在耳。汝且還,後年當來,見汝不久。」約拜辭而歸,道遇一國曰猗尼渠餘國,引約入宮,與約皮囊一枚,曰:「為吾遺漢皇帝。」約辭而歸,謂約曰:「劉郎後年來必見過,當以小女相妻。」約歸,置皮囊於機上。俄而蘇,使左右機上取皮囊開之,有一方白玉,題文曰:「猗尼渠餘國天王敬信遮須夷國天王,歲在攝提,當相見也。」
 馳使呈聰,聰曰:「若審如此,吾不懼死也。」及聰死,與此玉并葬焉。



 時東宮鬼哭;赤虹經天,南有一歧;三日並照,各有兩珥,五色甚鮮;客星歷紫宮入於天獄而滅。太史令康相言於聰曰:「蛇虹見彌天,一歧南徹;三日並照;客星入紫宮。此皆大異,其征不遠也。今虹達東西者,許洛以南不可圖也。一歧南徹者,李氏當仍跨巴蜀,司馬睿終據全吳之象,天下其三分乎!月為胡王,皇漢雖苞括二京,龍騰九五,然世雄燕代,肇基北朔,太陰之變其在漢域乎!漢既據中原,歷命所屬,紫宮之異,亦不在他,此之深重,胡可盡言。石勒鴟視趙魏,曹嶷狼顧東齊,鮮卑之
 眾星布燕代,齊、代、燕、趙皆有將大之氣。願陛下以東夏為慮,勿顧西南。吳蜀之不能北侵,猶大漢之不能南向也。今京師寡弱,勒眾精盛,若盡趙魏之銳,燕之突騎自上黨而來,曹嶷率三齊之眾以繼之,陛下將何以抗之?紫宮之變何必不在此乎!願陛下早為之所,無使兆人生心。陛下誠能發詔,外以遠追秦皇、漢武循海之事,內為高帝圖楚之計,無不剋矣。」聰覽之不悅。



 劉粲使王平謂劉乂曰:「適奉中詔,云京師將有變,敕裹甲以備之。」乂以為信然,令命宮臣裹甲以居。粲馳遣告靳準、王沈等曰:「向也王平告云東宮陰備非常,將若之何?」準白之,聰
 大驚曰:「豈有此乎!」王沈等同聲曰:「臣等久聞,但恐言之陛下弗信。」於是使粲圍東宮。粲遣沈、準收氐羌酋長十餘人,窮問之,皆懸首高格,燒鐵灼目,乃自誣與乂同造逆謀。聰謂沈等言曰:「而今而後,吾知卿等忠於朕也。當念為知無不言,勿恨往日言不用也。」於是誅乂素所親厚大臣及東宮官屬數十人,皆靳準及閹豎所怨也。廢乂為北部王,粲使準賊殺之。坑士眾萬五千餘人,平陽街巷為之空。氏羌叛者十餘萬落,以靳準行車騎大將軍以討之。時聰境內大蝗,平陽、冀、雍尤甚。靳準討之,震其二子而死。河汾大溢,漂沒千餘家。東宮災異,門閣宮
 殿蕩然。立粲為皇太子,大赦殊死已下。以粲領相國、大單于,總攝朝政如前。



 聰校獵上林,以帝行車騎將軍,戎服執戟前導,行三驅之禮。粲言於聰曰:「今司馬氏跨據江東,趙固、李矩同逆相濟,興兵聚眾者皆以子鄴為名,不如除之,以絕其望。」聰然之。



 趙固郭默攻其河東,至於絳邑,右司隸部人盜牧馬負妻子奔之者三萬餘騎。騎兵將軍劉勳追討之,殺萬餘人,固、默引歸。劉頡遮邀擊之,為固所敗。使粲及劉雅等伐趙固,次于小平津,固揚言曰:「要當生縛劉粲以贖天子。」聰聞而惡之。



 李矩使郭默、郭誦救趙固,屯于洛汭,遣耿稚、張皮潛濟,襲粲。貝丘
 王翼光自厘城覘之,以告粲。粲曰:「征北南渡,趙固望聲逃竄,彼方憂自固,何暇來邪!且聞上身在此,自當不敢北視,況敢濟乎!不須驚動將士也。」是夜,稚等襲敗粲軍,粲奔據陽鄉,稚館穀粲壘。雅聞而馳還,柵于壘外,與稚相持。聰聞粲敗,使太尉范隆率騎赴之,稚等懼,率眾五千,突圍趨北山而南。劉勳追之,戰于河陽,稚師大敗,死者三千五百人,投河死者千餘人。



 聰所居螽斯則百堂災,焚其子會稽王衷已下二十有一人。聰聞之,自投於床,哀塞氣絕,良久乃蘇。平陽西明門牡自亡,霍山崩。



 署其驃騎大將軍、濟南王劉驥為大將軍、都督中外諸軍事、
 錄尚書,衛大將軍、齊王劉勱為大司徒。



 中常侍王沈養女年十四,有妙色,聰立為左皇后。尚書令王鑒、中書監崔懿之、中書令曹恂等諫曰:「臣聞王者之立后也,將以上配乾坤之性,象二儀敷育之義,生承宗廟,母臨天下,亡配后土,執饋皇姑,必擇世德名宗,幽閑淑令,副四海之望,稱神祇之心。是故周文造舟,姒氏以興,《關雎》之化饗,則百世之祚永。孝成任心縱欲,以婢為后,使皇統亡絕,社稷淪傾。有周之隆既如彼矣,大漢之禍又如此矣。從麟嘉以來,亂淫於色,縱沈之弟女,刑餘小醜猶不可塵瓊寢,汙清廟,況其家婢邪!六宮妃嬪皆公子公孫,
 奈何一旦以婢主之,何異象榱玉簀而對腐木朽楹哉!臣恐無福於國家也。」聰覽之大怒,使宣懷謂粲曰:「鑒等小子,慢侮國家,狂言自口,無復君臣上下之禮,其速考竟。」於是收鑒等送市。金紫光祿大夫王延馳將入諫,門者弗通。鑒等臨刑,王沈以杖叩之曰:「庸奴,復能為惡乎?乃公何與汝事!」鑒瞋目叱之曰:「豎子!使皇漢滅者,坐汝鼠輩與靳準耳,要當訴汝於先帝,取汝等於地下。」懿之曰:「靳準梟聲鏡形,必為國患。汝既食人,人亦當食汝。」皆斬之。聰又立其中常侍宣懷養女為中皇后。



 鬼哭於光極殿,又哭於建始殿。雨血平陽,廣袤十里。時聰子約已
 死,至是晝見。聰甚惡之,謂粲曰:「吾寢疾惙頓,怪異特甚。往以約之言為妖,比累日見之,此兒必來迎吾也。何圖人死定有神靈,如是,吾不悲死也。今世難未夷,非諒暗之日,朝終夕殮,旬日而葬。」征劉曜為丞相、錄尚書,輔政,固辭乃止。仍以劉景為太宰,劉驥為大司馬,劉顗為太師,朱紀為太傅,呼延晏為太保,並錄尚書事;范隆守尚書令、儀同三司,靳準為大司空、領司隸校尉,皆迭決尚書奏事。



 太興元年,聰死,在位九年,偽謚曰昭武皇帝,廟號烈宗。



 粲
 字士光。少而俊傑,才兼文武。自為宰相,威福任情,疏遠忠賢,暱近姦佞,任性嚴刻無恩惠,距諫飾非。好興造宮室,相國之府仿像紫宮,在位無幾,作兼晝夜,饑困窮叛,死亡相繼,粲弗之恤也。既嗣偽位,尊聰后靳氏為皇太后,樊氏號弘道皇后,宣氏號弘德皇后,王氏號弘孝皇后。靳等年皆未滿二十,並國色也,粲晨夜蒸淫於內,志不在哀。立其妻靳氏為皇后,子元公為太子,大赦境內,改元漢昌。雨血于平陽。



 靳準將有異謀,私於粲曰:「如聞諸公將欲行伊尹、霍光之事,謀先誅太保及臣,以大司馬統萬機。陛下若不先之,臣恐禍之來也不晨則夕。」
 粲弗納。準懼其言之不從,謂聰二靳氏曰:「今諸公侯欲廢帝,立濟南王,恐吾家無復種矣。盍言之於帝。」二靳承間言之。粲誅其太宰、上洛王劉景,太師、昌國公劉顗,大司馬、濟南王劉驥,大司徒、齊王劉勱等。太傅朱紀、太尉范隆出奔長安。又誅其車騎大將軍、吳王劉逞,驥母弟也。粲大閱上林,謀討石勒。以靳準為大將軍、錄尚書事。粲荒耽酒色,游宴後庭,軍國之事一決於準。準矯粲命,以從弟明為車騎將軍,康為衛將軍。



 準將作亂,以金紫光祿大夫王延耆德時望,謀之于延。延弗從,馳將告之,遇靳康,劫延以歸。準勒兵入宮,升其光極前殿,下使甲士
 執粲,數而殺之。劉氏男女無少長皆斬于東市。發掘元海、聰墓,焚燒其宗廟。鬼大哭,聲聞百里。



 準自號大將軍、漢天王,置百官,遣使稱籓于晉。左光祿劉雅出奔西平。尚書北宮純、胡崧等招集晉人,保於東宮,靳康攻滅之。準將以王延為左光祿,延罵曰:「屠各逆奴,何不速殺我,以吾左目置西陽門,觀相國之入也,右目置建春門,觀大將軍之入也。」準怒,殺之。



 陳元達,字長宏,後部人也。本姓高,以生月妨父,故改云陳。少面孤貧,常躬耕兼誦書,樂道行詠,忻忻如也。至年
 四十,不與人交通。元海之為左賢王,聞而招之,元達不答。及元海僭號,人謂元達曰:「往劉公相屈,君蔑而不顧,今稱號龍飛,君其懼乎?」元達笑曰:「是何言邪?彼人姿度卓犖,有籠羅宇宙之志,吾固知之久矣。然往日所以不往者,以期運未至,不能無事喧喧,彼自有以亮吾矣。卿但識之,吾恐不過二三日,驛書必至。」其暮,元海果徵元達為黃門郎。人曰:「君殆聖乎!」既至,引見,元海曰:「卿若早來,豈為郎官而已。」元達曰:「臣惟性之有分,盈分者顛。臣若早叩天門者,恐大王賜處於九卿、納言之間,此則非臣之分,臣將何以堪之!是以抑情盤桓,待分而至,大王
 無過授之謗,小臣免招寇之禍,不亦可乎!」元海大悅。在位忠謇,屢進讜言,退而削草,雖子弟莫得而知也。聰每謂元達曰:「卿當畏朕,反使朕畏卿乎?」元達叩頭謝曰:「臣聞師臣者王,友臣者霸。臣誠愚暗無可採也,幸邀陛下垂齊桓納九九之義,故使微臣得盡愚忠。昔世宗遙可汲黯之奏,故能恢隆漢道;桀紂誅諫,幽厲弭謗,是以三代之亡也忽焉。陛下以大聖應期,挺不世之量,能遠捐商周覆國之弊,近模孝武光漢之美,則天下幸甚,群臣知免。」及其死也,人盡冤之。



\end{pinyinscope}