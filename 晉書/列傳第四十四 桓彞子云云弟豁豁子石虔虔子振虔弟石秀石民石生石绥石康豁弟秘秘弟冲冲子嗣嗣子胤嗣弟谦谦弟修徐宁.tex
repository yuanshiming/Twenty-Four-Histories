\article{列傳第四十四 桓彞子云云弟豁豁子石虔虔子振虔弟石秀石民石生石绥石康豁弟秘秘弟冲冲子嗣嗣子胤嗣弟谦谦弟修徐宁}

\begin{pinyinscope}
桓彞
 \gezhu{
  子雲雲弟豁豁子石虔虔子振虔弟石秀石民石生石綏石康豁弟祕秘弟沖沖子嗣嗣子胤嗣弟謙謙弟脩徐寧}



 桓彞,字茂倫,譙國龍亢人,漢五更榮之九世孫也。父顥,官至郎中。彞少孤貧,雖簞瓢,處之晏如。性通朗,早獲盛名。有人倫識鑒,拔才取士,或出於無聞,或得之孩抱,時人方之許、郭。少與庾亮深交,雅為周顗所重。顗嘗歎曰:「
 茂倫嶔崎歷落,固可笑人也。」起家州主簿。赴齊王冏義,拜騎都尉。元帝為安東將軍,版行逡遒令。尋辟丞相中兵屬,累遷中書郎、尚書吏部郎,名顯朝廷。



 于時王敦擅權,嫌忌士望,彝以疾去職。嘗過輿縣,縣宰徐寧字安期,通朗博涉,彞遇之,欣然停留累日,結交而別。先是,庾亮每屬彞覓一佳吏部,及至都,謂亮曰:「為卿得一吏部矣。」亮問所在,彞曰:「人所應有而不必有,人所應無而不必無。徐寧真海岱清士。」因為敘之,即遷吏部郎,竟歷顯職。



 明帝將伐王敦,拜彞散騎常侍,引參密謀。及敦平,以功封萬寧縣男。丹陽尹溫嶠上言:「宣城阻帶山川,頻
 經變亂,宜得望實居之,竊謂桓彝可充其選。」帝手詔曰:「適得太真表如此。今大事新定,朝廷須才,不有君子,其能國乎!方今外務差輕,欲停此事。」彝上疏深自捴挹,內外之任並非所堪,但以墳柏在此郡,欲暫結名義,遂補彞宣城內史。在郡有惠政,為百姓所懷。



 蘇峻之亂也,彞糾合義眾,欲赴朝廷。其長史裨惠以郡兵寡弱,山人易擾,可案甲以須後舉。彞厲色曰:「夫見無禮於其君者,若鷹鸇之逐鳥雀。今社稷危逼,義無晏安。」乃遣將軍朱綽討賊別帥於蕪湖,破之。彞尋出石硊。會朝廷遣將軍司馬流先據慈湖,為賊所破,遂長驅徑進。彞以郡無堅城,
 遂退據廣德。尋王師敗績,彞聞而慷慨流涕,進屯涇縣。時州郡多遣使降峻,裨惠又勸彞偽與通和,以紓交至之禍。彞曰:「吾受國厚恩,義在致死,焉能忍垢蒙辱與醜逆通問!如其不濟,此則命也。」遣將軍俞縱守蘭石。峻遣將韓晃攻之。縱將敗,左右勸縱退軍。縱曰:「吾受桓侯厚恩,本以死報。吾之不可負桓侯,猶桓侯之不負國也。」遂力戰而死。晃因進軍攻彝。彞固守經年,勢孤力屈。賊曰:「彞若降者,當待以優禮。」將士多勸彞偽降,更思後舉。彞不從,辭氣壯烈,志節不撓。城陷,為晃所害,年五十三。時賊尚未平,諸子並流迸,宣城人紀世和率義故葬之。賊
 平,追贈廷尉,謚曰簡。咸安中,改贈太常。俞縱亦以死節,追贈興古太守。



 初,彞與郭璞善,嘗令璞筮。卦成,璞以手壞之。彞問其故。曰:「卦與吾同。丈夫當此非命,如何!」竟如其言。有五子:溫、雲、豁、祕、沖。溫別有傳。



 雲字雲子。初為驃騎何充參軍、尚書郎,不拜。襲爵萬寧男,歷位建武將軍、義成太守。遭母憂去職。葬畢,起為江州刺史,稱疾,廬于墓次。詔書敦逼,固辭不行,服闋,然後蒞職。加都督司豫二州軍事、領鎮蠻護軍、西陽太守、假節。雲招集眾力,志在足兵,多所枉濫,眾皆嗟怨。時溫執權,有司不敢彈劾。升平四年卒,贈平南將軍,謚曰貞。子
 序嗣,官至宣城內史。



 豁字朗子。初辟司徒府、秘書郎,皆不就。簡文帝召為撫軍從事中郎,除吏部郎,以疾辭。遷黃門郎,未拜。時謝萬敗於梁濮,許昌、潁川諸城相次陷沒,西籓騷動。溫命豁督沔中七郡軍事、建威將軍、新野義成二郡太守,擊慕容屈塵,破之,進號右將軍。溫既內鎮,以豁監荊揚雍州軍事、領護南蠻校尉、荊州刺史、假節,將軍如故。時梁州刺史司馬勳以梁益叛,豁使其參軍桓羆討之。而南陽督護趙弘、趙憶等逐太守桓淡,據宛城以叛,豁與竟陵太守羅崇討破之。又攻偽南中郎將趙盤於宛,盤退走,
 豁追至魯陽,獲之,送于京師,置戍而旋。又監寧益軍事。溫薨,遷征西將軍,進督交廣并前五州軍事。



 苻堅寇蜀,豁遣江夏相竺瑤距之。廣漢太守趙長等戰死,瑤引軍退。頃之,堅又寇涼州,弟沖遣輔國將軍朱序與豁子江州刺史石秀溯流就路,稟節度。豁遣督護桓羆與序等游軍沔漢,為涼州聲援。俄而張天錫陷沒,詔遣中書郎王尋之詣豁,諮謀邊事。豁表以梁州刺史毛憲祖監沔北軍事,兗州刺史朱序為南中郎將、監沔中軍事,鎮襄陽,以固北鄙。



 太元初,遷征西大將軍、開府。豁上疏固讓曰:「臣聞三台麗天,辰極以之增耀;論道作弼,王猷以之
 時邕。必將仰參神契,對揚成務,弘易簡以翼化,暢玄風於宗極。故宜明揚仄陋,登庸賢俊,使版築有沖天之舉,渭濱無垂竿之逸。用乃功濟蒼生,道光千載。是以德非時望,成典所不虛授;功微賞厚,賢達不以擬心。臣實凡人,量無遠致,階藉門寵,遂叨非據。進不能闡揚皇風,贊明其政道;退不能宣力所蒞,混一華戎。尸素積載,庸績莫紀。是以敢冒成命,歸陳丹款。伏願陛下迴神玄覽,追收謬眷,則具瞻革望,臣知所免。」竟不許。及苻堅陷仇池,豁以新野太守吉挹行魏興太守、督護梁州五郡軍事,戍梁州。堅陷涪城,梁州刺史楊亮、益州刺史周仲孫並
 委戍奔潰。豁以威略不振,所在覆敗,又上疏陳謝,固辭,不拜開府。尋卒,時年五十八。贈司空,本官如故,謚曰敬。贈錢五十萬,布五百匹,使者持節監護喪事。豁時譽雖不及沖,而甚有器度。但遇彊寇,故功業不建。



 初,豁聞符堅國中有謠云:「誰謂爾堅石打碎。」有子二十人,皆以「石」為名以應之。唯石虔、石秀、石民、石生、石綏、石康知名。



 石虔小字鎮惡。有才幹,趫捷絕倫。從父在荊州,於獵圍中見猛善被數箭而伏,諸督將素知其勇,戲令拔箭。石虔因急往,拔得一箭,猛獸跳,石虔亦跳,高於獸身,猛獸伏,復拔一箭以歸。從溫入關。沖為苻健所圍,垂沒,石虔
 躍馬赴之,拔沖於數萬眾之中而還,莫敢抗者。三軍歎息,威震敵人。時有患虐疾者,謂曰「桓石虔來」以怖之,病者多愈,其見畏如此。



 初,袁真以壽陽叛,石虔以寧遠將軍、南頓太守帥諸將攻之,剋其南城。又擊苻堅將王鑒于石橋,獲馬五百匹。除竟陵太守,以父憂去職。尋而苻堅又寇淮南,詔曰:「石虔文武器幹,御戎有方。古人絕哭,金革弗避,況在餘哀,豈得辭事!可授奮威將軍、南平太守。」尋進冠軍將軍。苻堅荊州刺史梁成、襄陽太守閻震率眾入寇竟陵,石虔與弟石民距之。賊阻敖水,屯管城。石虔設計夜渡水,既濟,賊始覺,力戰破之,進剋管城,擒
 震,斬首七千級,俘獲萬人,馬數百匹,牛羊千頭,具裝鎧三百領。成以輕騎走保襄陽。石虔復領河東太守,進據樊城,逐堅兗州刺史張崇,納降二千家而還。沖卒,石虔以冠軍將軍監豫州揚州五郡軍事、豫州刺史。尋以母憂去職。服闋,復本位。久之,命移鎮馬頭,石虔求停歷陽,許之。



 太元十三年卒,追贈右將軍。追論平閻震功,進爵作塘侯。第五子誕嗣。誕長兄洪,襄城太守。洪弟振。



 振字道全。少果銳,而無行。玄為荊州,以振為揚武將軍、淮南太守。轉江夏相,以兇橫見黜。及玄之敗也,桓謙匿於沮中,振逃於華容之沮中。玄先令將軍王稚徽戍巴
 陵,稚徽遣人報振云:「桓欽已剋京邑,馮稚等復平尋陽,劉毅諸軍並敗於中路。」振大喜。時安帝在江陵,振乃聚黨數十人襲江陵。比至城,有眾二百。謙亦聚眾而出,遂陷江陵,迎帝於行宮。振聞桓昇死,大怒,將肆逆於帝,謙苦禁之,乃止。遂命群臣,辭以楚祚不終,百姓之心復歸于晉,更奉進璽綬,以瑯邪王領徐州刺史,振為都督八州、鎮西將軍、荊州刺史。帝侍御左右,皆振之腹心,既而歎曰:「公昔早不用我,遂致此敗。若使公在,我為前鋒,天下不足定。今獨作此,安歸乎!」遂肆意酒色,暴虐無道,多所殘害。



 振營於江津。南陽太守魯宗之自襄陽破振將
 溫楷于柞溪,進屯紀南。振聞楷敗,留其將馮該守營,自率眾與宗之大戰。振勇冠三軍,眾莫能禦,宗之敗績。振追奔,遇宗之單騎於道,弗之識也,乃問宗之所在。紿曰:「已前走矣。」宗之於是自後而退。尋而劉毅等破馮該,平江陵。振聞該敗,眾潰而走。後與該子宏出自溳城,復襲江陵。荊州刺史司馬休之奔襄陽,振自號荊州刺史。建威將軍劉懷肅率寧遠將軍索邈,與振戰於沙橋。振兵雖少,左右皆力戰,每一合,振輒瞋目奮擊,眾莫敢當。振時醉,且中流矢,廣武將軍唐興臨陣斬之。



 石秀,幼有令名,風韻秀徹,博涉君書尤善《老》《莊》。常獨處
 一室,簡於應接,時人方之庾純。甚為簡文帝所重。豁為荊州,請為鷹揚將軍、竟陵太守,非其好也。尋代叔父沖為寧遠將軍、江州刺史、領鎮蠻護軍、西陽太守,居尋陽。性放曠,常弋釣林澤,不以榮爵嬰心。善騎射,發則命中。嘗從沖獵,登九井山,徒旅甚盛,觀者傾坐,石秀未嘗屬目,止嘯詠而已。謝安嘗訪以世務,默然不答,安甚怪之。他日,安以語其從弟嗣,嗣以問之,石秀曰:「世事此公所諳,吾又何言哉!」在州五年,以疾去職。年四十三卒於家,朝野悼惜之。追贈後將軍,後改贈太常。子稚玉嗣。玄之篡也,以石秀一門之令,封稚玉為臨沅王。



 石民,弱冠知名,衛將軍謝安引為參軍。叔父沖上疏,版督荊江豫三州之十郡軍事、振武將軍,領襄城太守,戍夏口,與石虔攻苻堅荊州刺史梁成等於竟陵。明年,又與隨郡太守夏侯澄之破苻堅將慕容垂、姜成等於漳口。復領譙國內史、梁郡太守。沖薨,詔以石民監;荊州軍事、西中郎將、荊州刺史。桓氏世蒞荊土,石民兼以才望,甚為人情所仰。



 初,沖遣竟陵太守趙統伐襄陽。至是,石民復遣兵助之。尋而苻堅敗於淮肥,石民遣南陽太守高茂衙山陵。時堅雖破敗,而慕容垂等復盛。石民遣將軍晏謙伐弘農,賊東中郎將慕容夔降之。始置湖陜二
 戍。獲關中擔幢伎,以充太樂。時苻堅子丕僭號於河北,謀襲洛陽。石民遣將軍馮該討之,臨隈斬丕,及其左僕射王孚、吏部尚書茍操等,傳首京都。而丁零翟遼復侵逼山陵,石民使河南太守馮遵討之。時乞活黃淮自稱並州刺史,與遼共攻長社,眾數千人。石民復遣南平太守郭銓、松滋太守王遐之擊淮,斬之,遼走河北。以前後功,進左將軍。卒,無子。



 石生,隆安中以司徒左長史遷侍中,歷驃騎、太傅長史。會稽世子元顯將伐桓玄,石生馳書報玄,玄甚德之。及玄用事,以為前將軍、江州刺史。尋卒於官。



 石綏,元顯時為司徒左長史。玄用事,拜黃門郎、左衛將軍。玄敗,石綏走江西塗中,聚眾攻歷陽,後為梁州刺史傅歆之所殺。



 石康,偏為玄所親愛,玄為荊州,以為振威將軍。累遷荊州刺史。討庾仄功,封武陵王,事具玄傳。



 秘字穆子。少有才氣,不倫於俗。初拜秘書郎,兄溫抑而不用。久之,為輔國將軍、宣城內史。時梁州刺史司馬勳叛入蜀,秘以本官監梁益二州征討軍事、假節。勳平,還郡。後為散騎常侍,徙中領軍。孝武帝初即位,妖賊盧竦入宮,秘與左衛將軍殷康俱入擊之。溫入朝,竊考竦事,
 收尚書陸始等,罹罪者甚眾。祕亦免官,居于宛陵,每憤憤有不平之色。溫疾篤,祕與溫子熙、濟等謀共廢沖。沖密知之,不敢入。頃溫氣絕,先遣力士拘錄熙、濟,而後臨喪。秘於是廢棄,遂居於墓所,放志田園,好遊山水。後起為散騎常侍,凡三表自陳。詔曰:「祕受遇先朝。是以延之。而頻有讓表,以棲尚告誠,兼有疾疢,省用增嘆。可順其所執。」祕素輕沖,沖時貴盛,秘恥常侍位卑,故不應朝命,與謝安書及詩十首,辭理可觀,其文多引簡文帝之眄遇。先沖卒。長子蔚,官至散騎常侍、游擊將軍。玄篡,以為醴陵王。



 沖字幼子,溫諸弟中最淹識,有武幹,溫甚器之。弱冠,太宰、武陵王晞辟,不就。除鷹揚將軍、鎮蠻護軍、西陽太守。從溫征伐有功,遷督荊州之南陽襄陽新野義陽順陽雍州之京兆揚州之義成七郡軍事、寧朔將軍、義成新野二郡太守,鎮襄陽。又從溫破姚襄。及虜周成,進號征虜將軍,賜爵豐城公。尋遷振威將軍、江州刺史、領鎮蠻護軍、西陽譙二郡太守。溫之破姚襄也,獲襄將張駿、楊凝等,徙于尋陽。沖在江陵,未及之職,而駿率其徒五百人殺江州督護趙毗,掠武昌府庫,將妻子北叛。沖遣將討獲之,遽還所鎮。



 初,彞亡後,沖兄弟並少,家貧,母患,須
 羊以解,無由得之,溫乃以沖為質。羊主甚富,言不欲為質,幸為養買德郎,買德郎,沖小字也。及沖為江州,出射,羊主於堂邊看,沖識之,謂曰:「我買德也。」遂厚報之。頃之,進監江荊豫三州之六郡軍事、南中郎將、假節,州郡如故。



 在江州凡十三年而溫薨。孝武帝詔沖為中軍將軍、都督揚江豫三州軍事、揚豫二州刺史、假節。時詔賻溫錢布漆蠟等物,而不及大殮。沖上疏陳溫素懷每存清儉,且私物足舉凶事,求還官庫。詔不許,沖猶固執不受。初,溫執權,大辟之罪皆自己決。沖既蒞事,上疏以為生殺之重,古今所慎,凡諸死罪,先上,須報。沖既代溫居任,
 盡忠王室。或勸沖誅除時望,專執權衡,沖不從。



 謝安以時望輔政,為群情所歸,沖懼逼,寧康三年,乃解揚州,自求外出。桓氏黨與以為非計,莫不扼腕苦諫,郗超亦深止之。沖皆不納,處之澹然,不以為恨,忠言嘉謀,每盡心力。於是改授都督徐兗豫青揚五州之六郡軍事、車騎將軍、徐州刺史,以北中郎府并中軍,鎮京口,假節。又詔沖及謝安並加侍中,以甲杖五十人入殿。時丹陽尹王蘊以后父之重暱于安,安意欲出蘊為方伯,乃復解沖徐州,直以車騎將軍都督豫江二州之六郡軍事,自京口遷鎮姑熟。



 既而苻堅寇涼州,沖遣宣城內史朱序、
 豫州刺史桓伊率眾向壽陽,淮南太守劉波汎舟淮泗,乘虛致討,以救涼州,乃表曰:



 氐賊自并東胡,醜類實繁,而蜀漢寡弱,西涼無備,斯誠暴與疾顛,祇速其亡。然而天未剿絕,屢為國患。臣聞勝於無形,功立事表,伐謀之道,兵之上略。況此賊陸梁,終必越逸。北狄陵縱,常在秋冬。今日月迅邁,高風行起,臣輒較量畿甸,守衛重復,又淮泗通流,長江如海,荊楚偏遠,密邇寇仇,方城、漢水無天險之實,而過備之重勢在西門。



 臣雖凡庸,識乏武略,然猥荷重任,思在投袂。請率所統,徑進南郡,與征西將軍臣豁參同謀猷。賊若果驅犬羊,送死沔漢,庶仰憑正
 順,因致人利,一舉乘風,掃清氛穢,不復重勞王師,有事三秦,則先帝盛業永隆於聖世,宣武遺志無恨於在昔。如其懾憚皇威,窺窬計屈,則觀兵伺釁,更議進取,振旅旋旆,遲速唯宜。伏願陛下覽臣所陳,特垂聽許。



 詔答曰:「醜類違天,比年縱肆,梁益不守,河西傾喪。每惟宇內未一,憤歎盈懷。將軍經略深長,思算重復,忠國之誠,形于義旨。覽省未周,以感以慨。寇雖乘間竊利,而以無道臨之,黷武窮兇,虐用其眾,滅亡之期,勢何得久!然備豫不虞,軍之善政。輒詢于群后,敬從高算。想與征西協參令圖,嘉謀遠猷,動靜以聞。」會張天錫陷沒,於是罷兵。俄而
 豁卒,遷都督江荊梁益寧交廣七州揚州之義成雍州之京兆司州之河東軍事、領護南蠻校尉、荊州刺史、持節,將軍、侍中如故。又以其子嗣為江州刺史。沖將之鎮,帝餞於西堂,賜錢五十萬。又以酒三百四十石、牛五十頭犒賜文武。謝安送至溧洲。



 沖既到江陵,時苻堅彊盛,沖欲移阻江南,乃上疏曰:「自中興以來,荊州所鎮,隨宜迴轉。臣亡兄溫以石季龍死,經略中原,因江陵路便,即而鎮之。事與時遷,勢無常定。且兵者詭道,示之以弱,今宜全重江南,輕戍江北。南平孱陵縣界,地名上明,田土膏良,可以資業軍人。在吳時樂鄉城以上四十餘里,北
 枕大江,西接三峽。若狂狡送死,則舊郢以北堅壁不戰,接會濟江,路不云遠,乘其疲墮,撲翦為易。臣司存閫外,輒隨宜處分。」於是移鎮上明,使冠軍將軍劉波守江陵,諮議參軍楊亮守江夏。詔以荊州水旱饑荒,又沖新移草創,歲運米三十萬斛以供軍資,須年豐乃止。



 堅遣其將苻融寇樊、鄧,石越寇魯陽,姚萇寇南鄉,韋鐘寇魏興,所在陷沒。沖遣江夏相劉奭、南中郎將朱序擊之,而奭畏懦不進,序又為賊所擒。沖深自咎責,上疏送章節,請解職,不許。遣左衛將軍張玄之詣沖諮謀軍事。沖率前將軍劉波及兄子振威將軍石民、冠軍將軍石虔等伐
 苻堅,拔堅築陽。攻武當,走堅兗州刺史張崇。堅遣慕容垂、毛當寇鄧城,苻熙、石越寇新野。沖既憚堅眾,又以疾疫,還鎮上明。表以「夏口江沔衛要,密邇彊寇,兄子石民堪居此任,輒版督荊江十郡軍事、振武將軍、襄城太守。尋陽北接彊蠻,西連荊郢,亦一任之要。今府州既分,請以王薈補江州刺史」詔從之。時薈始遭兄劭喪,將葬,辭不欲出。於是衛將軍謝安更以中領軍謝輶代之。沖聞之而怒,上疏以為輶文武無堪,求自領江州,帝許之。沖使石虔伐堅襄陽太守閻震,擒之,及大小帥二十九人,送于京都,詔歸沖府。以平震功,封次子謙宜陽侯。堅使
 其將郝貴守襄陽,沖使揚威將軍朱綽討之,遂焚燒沔北田稻,拔六百餘戶而還。又遣上庸太守郭寶伐堅魏興太守褚垣、上庸太守段方,並降之。新城太守麴常遁走,三郡皆平。詔賜錢百萬,袍表千端。



 初,沖之西鎮,以賊寇方彊,故移鎮上明,謂江東力弱,正可保固封疆,自守而已。又以將相異宜,自以德望不逮謝安,故委之內相,而四方鎮扞,以為己任。又與朱序款密。俄而序沒於賊,沖深用愧惋。既而苻堅盡國內侵,沖深以根本為慮,乃遣精銳三千來赴京都。謝安謂三千人不足以為損益,而欲外示閑暇,聞軍在近,固不聽。報云:「朝廷處分已定,
 兵革無闕,西籓宜以為防。」時安已遣兄子玄及桓伊等諸軍,沖謂不足以為廢興,召佐吏,對之歎曰:「謝安乃有廟堂之量,不閑將略。今大敵垂至,方遊談不暇,雖遣諸不經事少年,眾又寡弱,天下事可知,吾其左衽矣!」俄而聞堅破,大勳克舉,又知朱序因以得還,沖本疾病,加以慚恥,發病而卒,時年五十七。贈太尉,本官如故,謚曰宣穆。賻錢五十萬,布五百匹。



 沖性儉素,而謙虛愛士。嘗浴後,其妻送以新衣,沖大怒,促令持去。其妻復送之,而謂曰:「衣不經新,何緣得故!」沖笑而服之。命處士南陽劉鄰之為長史,鄰之不屈,親往迎之,禮之甚厚。又辟處士
 長沙鄧粲為別駕,備禮盡恭。粲感其好賢,乃起應命。初,郗鑒、庾亮、庾翼臨終皆有表,樹置親戚,唯沖獨與謝安書云:「妙靈、靈寶尚小,亡兄寄託不終,以此為恨!」言不及私,論者益嘉之。及喪下江陵,士女老幼皆臨江瞻送,號哭盡哀。後玄篡位,追贈太傅、宣城王。有七子:嗣、謙、脩、崇、弘、羨、怡。



 嗣字恭祖。少有清譽,與豁子石秀並為桓氏子姪之冠。沖既代豁西鎮,詔以嗣督荊州之三郡豫州之四郡軍事、建威將軍、江州刺史。蒞事簡約,脩所住齋,應作版簷,嗣命以茅代之,版付船官。轉西陽、襄城二郡太守,鎮夏
 口。後領江夏相,卒官。追贈南中郎將,謚曰靖。子胤嗣。



 胤字茂遠。少有清操,雖奕世華貴,甚以恬退見稱。初拜秘書丞,累遷中書郎、祕書監。玄甚欽愛之,遷中書令。玄篡位,為吏部尚書,隨玄西奔。玄死,歸降。詔曰:「夫善著則祚遠,勳彰故事殊。以宣孟之忠,蒙後晉國;子文之德,世嗣獲存。故太尉沖,昔籓陜西,忠誠王室。諸子染凶,自貽罪戮。念沖遺勤,用心妻于懷。其孫胤宜見矜宥,以獎為善。可特全生命,徙于新安。」及東陽太守殷仲文、永嘉太守駱球等謀反,陰欲立胤為玄嗣,事覺,伏誅。



 謙字敬祖,詳正有器望。初以父功封宜陽縣開國侯,累
 遷輔國將軍、吳國內史。孫恩之亂,謙出奔無錫。徵拜尚書,驃騎大將軍元顯引為諮議參軍,轉司馬。元興初,朝廷將伐玄,以桓氏世在陜西,謙父沖有遺惠於荊楚,懼人情向背,乃用謙為持節、都督荊益寧梁四州諸軍事、西中郎將、荊州刺史、假節,以安荊楚。



 玄既用事,以謙為尚書左僕射,領吏部,加中軍將軍。謙兄弟顯列,玄甚倚杖之,而內不能善也。改封謙為寧都侯,拜尚書令,加散騎常侍。遷侍中、衛將軍、開府、錄尚書事。玄篡位,復領揚州刺史,本官如故,封新安王。



 及桓振作亂,謙保護乘輿,頗有功焉。然而暗懦,尤不可以造事。初,勸振率軍下戰,
 己守江陵。振既輕謙用事,故不從。及振敗,謙奔于姚興。先是,譙縱稱籓於姚興,縱與盧循通使,潛相影響,乃表興請謙共順流東下。興問謙,謙曰:「臣門著恩荊楚,從弟玄末雖篡位,皆是逼迫,人神所明。今臣與縱東下,百姓自應駭動。」興曰:「小水不容大舟,若縱才力足以濟事,亦不假君為鱗翼。宜自求多福。」遂遣之。謙至蜀,欲虛懷引士,縱疑之,乃置謙於龍格,使人守之。謙向諸弟泣曰:「姚主言神矣!」後與縱引譙道福俱下,謙於道占募,百姓感沖遺惠,投者二萬人。劉道規破謙,斬之。



 脩字承祖。尚簡文帝女武昌公主,歷吏部郎,稍遷左衛
 將軍。王恭將伐譙王尚之,先遣何澹之、孫無終向句容。脩以左衛領振武將軍,與輔國將軍陶無忌距之。脩次句容。俄而恭敗,無終遣書求降。脩既旋軍,而楊佺期已至石頭,時朝廷無備,內外崩駭。脩進說曰:「殷、桓之下,專恃王恭,恭既破滅,莫不失色。今若優詔用玄,玄必內喜,則能制仲堪、佺期,使並順命。」朝廷納之。以脩為龍驤將軍、荊州刺史、假節,權領左衛文武之鎮。又令劉牢之以千人送之。轉仲堪為廣州。脩未及發,而玄等盟於尋陽,求誅牢之。尚之并訴仲堪無罪,獨被降黜。於是詔復仲堪荊州。御史中丞江績奏脩承受楊佺期之言,交通信
 命,宣傳不盡,以為身計,疑誤朝算,請收付廷尉。特詔免官。尋代王凝之為中護軍。頃之,玄破仲堪、佺期,詔以脩為征虜將軍、江州刺史。尋復為中護軍。玄執政,以脩都督六州、右將軍、徐兗二州刺史、假節。尋進撫軍將軍,加散騎常侍。玄篡,以為撫軍大將軍,封安成王。劉裕義旗起,斬之。



 徐寧者,東海郯人也。少知名,為輿縣令。時廷尉桓彞稱有人倫鑒識,彞嘗去職,至廣陵尋親舊,還遇風,停浦中,累日憂悒,因上岸,見一室宇,有似廨署,訪之,云是輿縣。彞
 乃造之。寧清惠博涉,相遇欣然,因留數夕。彞大賞之,結交而別。至都,謂庾亮曰:「吾為卿得一佳吏部郎。」語在彞傳。即遷吏部郎、左將軍、江州刺史,卒官。



 史臣曰:醨風潛煽,醇源浸竭,遺道德於情性,顯忠信於名教。首陽高節,求仁而得仁;泗上微言,朝聞而夕死。原軫免胄,懍然於往策;季路絕纓,邈矣於前志。況交霜雪於杪歲,晦風雨於將晨,喈響或以變其音,貞柯罕能全其性。桓茂倫抱中和之氣,懷不撓之節,邁周庾之清塵,遵許郭之遐軌。懼臨危於取免,知處死之為易,揚芬千載之上,淪骨九泉之下。仁者之勇,不其然乎!至夫基構
 迭污隆,龍蛇俱山澤,沖逡巡於內輔,豁陵厲於上游,虔振北門之威,秀坦西陽之務,外有捍城之用,里無末大之嫌,求之名臣,抑亦可算。而溫為亢極之資,玄遂履霜之業,是知敬仲之美不息檀臺之亂,寧俞之忠無救弈棋之禍。子文之不血食,悲夫!



 贊曰:矯矯宣城,貞心莫陵。身隨露夭,名與雲興。虔豁重世,沖秀雙美。國賴忠臣,家推才子。振武謙文,尋邑為群。歸之篡亂,曷足以云。



\end{pinyinscope}