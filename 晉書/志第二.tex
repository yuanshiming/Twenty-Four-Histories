\article{志第二}

\begin{pinyinscope}

 天文中(七曜
 雜星氣史傳事驗}}



 ○七曜



 日為太陽之精,主生恩德,人君之象也。人君有瑕,必露其慝以告示焉。故日月行有道之國則光明,人君吉昌,百姓安寧。人君乘土而王,其政太平,則日五色無主。日變色,有軍,軍破;無軍,喪侯王。其君無德,其臣亂國,則日赤無光。日失色,所臨之國不昌。日晝昏,行人無影,到暮不止者,上刑急,下不聊生,不去一年有大水。日晝昏,烏鳥群鳴,國失政。日中
 烏
 見,主不明,為政亂,國有白衣會,將軍出,旌旗舉。日中有黑子、黑氣、黑雲,乍三乍五,臣廢其主。日蝕,陰侵陽,臣掩君之象,有亡國。



 月為太陰之精,以之配日,女主之象;以之比德,刑罰之義;列之朝廷,諸侯大臣之類。故君明,則月行依度;臣執權,則月行失道;大臣用事,兵刑失理,則月行乍南乍北;女主外戚擅權,則或進或退。月變色,將有殃。月晝明,姦邪並作,君臣爭明,女主失行,陰國兵強,中國饑,天下謀
 僭。數月重見,國以亂亡。



 歲星曰東方春木,於人,五常,仁也;五事,貌也。仁虧貌失,逆春令,傷木氣,則罰見歲星。歲星盈縮,以其舍命國。其所居久,其國有德厚,五穀豐昌,不可伐。其對為衝,歲乃有殃。歲星安靜中度,吉。盈縮失次,其國有憂,不可舉事用兵。又曰,人主之象也,色欲明,光色潤澤,德合同。又曰,進退如度,姦邪息;變色亂行,主無福。又主福,主大司農,主齊吳,主司天下諸侯人君之過,主歲五穀。赤而角,其國昌;赤黃而沈,其野大穰。



 熒惑曰南方夏火,禮也,視也。禮虧視失,逆夏令,傷火氣,
 罰見熒惑。熒惑法使行無常,出則有兵,入則兵散。以舍命國,為亂為賊,為疾為喪,為饑為兵,所居國受殃。環繞鉤已,芒角動搖,變色,乍前乍後,乍左乍右,其為殃愈甚。其南丈夫、北女子喪。周旋止息,乃為死喪;寇亂其野,亡地。其失行而速,兵聚其下,順之戰勝。又曰,熒惑主大鴻臚,主死喪,主司空。又為司馬,主楚吳越以南;又司天下群臣之過,司驕奢亡亂妖孽,主歲成敗。又曰,熒惑不動,兵不戰,有誅將。其出色赤怒,逆行成鉤已,戰凶,有圍軍;鉤已,有芒角如鋒刃,人主無出宮,下有伏兵;芒大則人眾怒。又為理,外則理兵,內則理政,為天子之理也。故曰,
 雖有明天子,必視熒惑所在。其入守犯太微、軒轅、營室、房、心,主命惡之。



 填星曰中央季夏土,信也,思心也。仁義禮智,以信為主,貌言視德,以心為正,故四星皆失,填乃為之動。動而盈,侯王不寧。縮,有軍不復。所居之宿,國吉,得地及女子,有福,不可伐;去之,失地,若有女憂。居宿久,國福厚;易則薄。失次而上二三宿曰盈,有主命不成,不乃大水。失次而下曰縮,后戚,其歲不復,不乃天裂若地動。一曰,填為黃帝之德,女主之象,主德厚安危存亡之機,司天下女主之過。又曰,天子之星也。天子失信,則填星大動。



 太白曰西方秋金,義也,言也。義虧言失,逆秋令,傷金氣,罰見太白。太白進退以候兵,高埤遲速,靜躁見伏,用兵皆象之,吉。其出西方,失行,夷狄敗;出東方,失行,中國敗。未盡期日,過參天,病其對國。若經天,天下革,民更王,是謂亂紀,人眾流亡。晝見,與日爭明,強國弱,小國強,女主昌。又曰,太白主大臣,其號上公也,大司馬位謹候此。



 辰星曰北方冬水,智也,聽也。智虧聽失,逆冬令,傷水氣,罰見辰星。辰星見,則主刑,主廷尉,主燕趙,又為燕、趙、代以北;宰相之象。亦為殺伐之氣,戰鬥之象。又曰,軍於野,辰星為偏將之象,無軍為刑事。和陰陽,應效不效,其時
 不和。出失其時,寒署失其節,邦當大饑。當出不出,是謂擊卒,兵大起。在於房心間,地動。亦曰,辰星出入躁疾,常主夷狄。又曰,蠻夷之星也,亦主刑法之得失。色黃而小,地大動。光明與月相逮,其國大水。



 凡五星有色,大小不同,各依其行而順時應節。色變有類,凡青皆比參左肩,赤比心大星,黃比參右肩,白比狼星,黑比奎大星。不失本色而應其四時者,吉;色害其行,凶。



 凡五星所出所行所直之辰,其國為得位。得位者,歲星以德,熒惑有禮,填星有福,太白兵強,辰星陰陽和。所行
 所直之辰,順其色而有角者勝,其色害者敗。居實,有德也;居虛,無德也。色勝位,行勝色,行得盡勝之。營室為清廟,歲星廟也。心為明堂,熒惑廟也。南斗為文太室,填星廟也。亢為疏廟,太白廟也。七星為員宮,辰星廟也。五星行至其廟,謹候其命。



 凡五星盈縮失位,其精降于地為人。歲星降為貴臣;熒惑降為童兒,歌謠嬉戲;填星降為老人婦女;太白降為壯夫,處於林麓;辰星降為婦人。吉凶之應,隨其象告。



 凡五星,木與土合,為內亂,饑;與水合,為變謀而更事;與火合,為饑,為旱;與金合,為白衣之會,合鬥,國有內亂,野
 有破軍,為水。太白在南,歲星在北,名曰牝牡,年穀大熟。太白在北,歲星在南,年或有或無。火與金合,為爍,為喪,不可舉事用兵。從軍,為軍憂;離之,軍卻。出太白陰,分宅;出其陽,偏將戰。與土合,為憂,主孽卿。與水合,為北軍,用兵舉事大敗。一曰,火與水合,為焠,不可舉事用兵。土與水合,為壅沮,不可舉事用兵,有覆軍下師。一曰,為變謀更事,必為旱。與金合,為疾,為白衣會,為內兵,國亡地。與木合,國饑。水與金合,為變謀,為兵憂。入太白中而上出,破軍殺將,客勝;下出,客亡地。視旗所指,以命破軍。環繞太白,若與鬥,大戰,客勝。凡木、火、土、金與水斗,皆為戰。兵不
 在外,皆為內亂。凡同舍為合,相陵為斗。二星相近,其殃大;相遠,毋傷,七寸以內必之。



 凡月蝕五星,其國皆亡。歲以飢,熒惑以亂,填以殺,太白以強國戰,辰以女亂。



 凡五星入月,歲,其野有逐相;太白,將僇。



 凡五星所聚,其國王,天下從。歲以義從,熒惑以禮從,填以重從,太白以兵從,辰以法從,各以其事致天下也。三星若合,是謂驚立絕行,其國外內有兵與喪,百姓饑乏,改立侯王。四星若合,是謂大陽,其國兵喪並起,君子憂,小人流。五星若合,是謂易行,有德承慶,改立王者,奄有
 四方,子孫蕃昌;亡德受殃,離其國家,滅其宗廟,百姓離去,被滿四方。五星皆大,其事亦大;皆小,事亦小。



 凡五星色,皆圜,白為喪,為旱;赤中不平,為兵;青為憂,為水;黑為疾疫,為多死;黃為吉。皆角,赤,犯我城;黃,地之爭;白,哭泣聲;青,有兵憂;黑,有水。五星同色,天下偃兵,百姓安寧,歌舞以行,不見災疾,五穀蕃昌。



 凡五星,歲,政緩則不行,急則過分,逆則占。熒惑,緩則不出,急則不入,違道則占。填,緩則不還,急則過舍,逆則占。太白,緩則不出,急則不入,逆則占。辰,緩則不出,急則不入,非時則占。五星不失行,則年穀豐昌。



 凡五星分天之中,積于東方,中國利;積于西方,外國用兵者利。辰星不出,太白為客;其出,太白為主。出而與太白不相從,及各出一方,為格,野雖有軍,不戰。



 凡五星見伏、留行、逆順、遲速應歷度者,為得其行,政合于常;違歷錯度,而失路盈縮者,為亂行。亂行則為天矢彗孛,而有亡國革政,兵饑喪亂之禍云。



 ○雜星氣



 圖緯舊說,及漢末劉表為荊州牧,命武陵太守劉睿集天文眾占,名《荊州占》。其雜星之體,有瑞星,有妖星,有客星,有流星,有瑞氣,有妖氣,有日月傍氣,皆略其名狀,舉
 其占驗,次之於此云。



 ○瑞星



 一曰景星,如半月,生於晦朔,助月為明。或曰,星大而中空。或曰,有三星,在赤方氣,與青方氣相連,黃星在赤方氣中,亦名德星。



 二曰周伯星,黃色,煌煌然,所見之國大昌。



 三曰含譽,光耀似慧,喜則含譽射。



 四曰格澤,如炎火,下大上銳,色黃白,起地而上。見則不種而獲,有土功,有大客。



 ○妖星



 一曰彗星,所謂掃星。本類星,末類彗,小者數寸,長或竟天。見則兵起,大水。主掃除,除舊布新。有五色,各依五行本精所主。史臣案,彗體無光,傅日而為光,故夕見
 則東指,晨見則西指。在日南北,皆隨日光而指。頓挫其芒,或長或短,光芒所及則為災。



 二曰孛星,彗之屬也。偏指曰彗,芒氣四出曰孛。孛者,孛孛然非常,惡氣之所生也。內不有大亂,則外有大兵,天下合謀,闇蔽不明,有所傷害。晏子曰:「君若不改,孛星將出,彗星何懼乎!」由是言之,災甚於彗。



 三曰天棓,一名覺星。本類星,末銳,長四丈。或出東北方西方,主奮爭。



 四曰天槍。其出,不過三月,必有破國亂君,伏死其辜。殃之不盡,當為旱飢暴疾。



 五曰天欃。石氏曰,雲如牛狀。甘氏,本類星,末銳。巫咸曰,彗星出西方,長可三丈,主捕制。



 六曰蚩尤旗,類彗而後曲,
 象旗。或曰,赤雲獨見。或曰,其色黃上白下。或曰,若植雚而長,名曰蚩尤之旗。或曰,如箕,可長二丈,末有星。主伐枉逆,主惑亂,所見之方下有兵,兵大起;不然,有喪。



 七曰天衝,出如人,蒼衣赤頭,不動。見則臣謀主,武卒發,天子亡。



 八曰國皇,大而赤,類南極老人星。或曰,去地一二丈,如炬火,主內寇內難。或曰,其下起兵,兵強。或曰,外內有兵喪。



 九曰昭明,象如太白,光芒,不行。或曰,大而白,無角,乍上乍下。一曰,赤彗分為昭明,昭明滅光,以為起霸起德之徵,所起國兵多變。一曰,大人凶,兵大起。



 十曰司危,如太白,有目。或曰,出正西,西方之野星,去地可六丈,大
 而白。或曰,大而有毛,兩角。或曰,類太白,數動,察之而赤,為乖爭之徵,主擊強兵。見則主失法,豪傑起,天子以不義失國,有聲之臣行主德。



 十一曰天讒,彗出西北,狀如劍,長四五丈。或曰,如鉤,長四丈。或曰,狀白小,數動,主殺罰。出則其國內亂,其下相讒,為饑兵,赤地千里,枯骨藉藉。



 十二曰五殘,一名五鏠,出正東,東方之星。狀類辰,可去地六七丈。或曰,蒼彗散為五殘,如辰星,出角。或曰,星表有氣如暈,有毛。或曰,大而赤,數動,察之而青。主乖亡;為五分,毀敗之徵,亦為備急兵。見則主誅,政在伯,野亂成,有急兵,有喪,不利衝。



 十三曰六賊,見出正南,南方之
 星。去地可六丈,大而赤,動有光。或曰,形如彗,五殘、六賊出,禍合天下,逆侵關樞;其下有兵,衝不利。



 十四曰獄漢,一名咸漢,出正北,北方之野星,去地可六丈,大而赤,數動,察之中青。或曰,赤表,下有三彗從橫。主遂王,主刺王。出則陰精橫,兵起其下。又為喪,動則諸侯驚。



 十五曰旬始,出北斗旁,如雄雞。其怒,有青黑,象伏鱉。或曰,怒,雌也,主爭兵。又曰,黃彗分為旬始,為立主之題,主亂,主招橫。見則臣亂兵作,諸侯虐,期十年,聖人起伐,群猾橫恣。或曰,出則諸侯雄鳴。



 十六曰天鋒,彗象矛鋒。天下從橫,則天鋒星見。



 十七曰燭星,如太白。其出也不行,見則不久
 而滅。或曰,主星上有三彗上出,所出城邑亂,有大盜不成,又以五色占。



 十八曰蓬星,大如二斗器,色白,一名王星。狀如夜火之光,多至四五,少一二。一曰,蓬星在西南,長數丈,左右兌。出而易處。星見,不出三年,有亂臣戮死。又曰,所出大水大旱,五穀不收,人相食。



 十九曰長庚,如一匹布著天。見則兵起。



 二十曰四填,星出四隅,去地六丈餘,或曰可四丈。或曰,星大而赤,去地二丈,常以夜半時出。見,十月而兵起,皆為兵起其下。



 二十一曰地維藏光,出四隅。或曰,大而赤,去地二三丈,如月始出。見則下有亂,亂者亡,有德者昌。



 《
 河圖》云:



 歲星之精,流為天棓、天槍、天猾、天衝、國皇、反登、蒼彗。



 熒惑散為昭旦、蚩尤之旗、昭明、司危、天欃、赤彗。



 填星散為五殘、獄漢、大賁、昭星、絀流、旬始、蚩尤、虹蜺、擊咎、黃彗。



 太白散為天杵、天柎、伏靈、大敗、司姦、天狗、天殘、卒起、白彗。



 辰星散為枉矢、破女、拂樞、滅寶、繞綎、驚理、大奮祀、黑彗。



 五色之彗,各有長短,曲折應象。



 漢京房著《風角書》有《集星章》,所載妖星皆見於月旁,互有五色方雲,以五寅日見,各有五星所生云:



 天槍、天根、天荊、真若、天榬,天樓、天垣,皆歲星所生也。見以甲寅,其星咸有兩青方在其旁。



 天陰、晉若、官張、天惑、天崔、赤若、蚩尤,皆熒惑之所生也。出在丙寅日,有兩赤方在其旁。



 天上、天伐、從星、天樞、天翟、天沸、荊彗,皆填星所生也。出在戊寅日,有兩黃方在其旁。



 若星、帚星、若彗、竹彗、墻星、榬星、白雚,皆太白之所生也。出在庚寅日,有兩白方在共旁。



 天美、天欃、天杜、天麻、天林、天蒿、端下,皆辰星之所生也。出以壬寅日,有兩黑方在其旁。



 已前三十五星,即五行氣所生,皆出於月左右方氣之中,各以其所生星將出不出日數期候之。當其未出之前而見,見則有水旱,兵喪,饑亂;所指亡國,失地,王死,破軍,殺將。



 ○客星



 張衡曰:「老子四星及周伯、王蓬絮、芮各一,錯乎五緯之間。其見無期。其行無度」。《荊州》占云:「老子星色淳白,然所見之國,為饑為凶,為善為惡,為喜為怒。周伯星黃色煌
 煌,所至之國大昌。蓬絮星色青而熒熒然,所至之國風雨不節,焦旱,物不生,五穀不登,多蝗蟲。」又云:「東南有三星出,名曰盜星,出則人下有大盜。西南有三星出,名曰種陵,出則天下穀貴十倍。西北三大星出而白,名曰天狗,出則人相食,大凶。東北有三大星出,名曰女帛,見則有大喪。」



 ○流星



 流星,天使也。自上而降曰流,自下而升曰飛。大者曰奔,奔亦流星也。星大者使大,星小者使小。聲隆隆者,怒之象也。行疾者期速,行遲者期遲。大而無光者。眾人之事;
 小而有光者,貴人之事;大而光者,其人貴且眾也。乍明乍滅者,賊敗成也。前大後小者,恐憂也;前小後大者,喜事也。蛇行者,姦事也;往疾者,往而不反也。長者,其事長久也;短者,事疾也。奔星所墜,其下有兵。無風雲,有流星見,良久間乃入,為大風,發屋折木。小流星百數四面行者,眾庶流移之象。



 流星之類,有音如炬火下地,野雉鳴,天保也;所墜國安,有喜。若小流星色青赤,名曰地鴈,其所墜者起兵。流星有光青赤,長二三丈,名曰天鴈,軍中之精華也;其國起兵,將軍當從星所之。流星暉然有光,光白,長竟天者,人主之星也;主相、將軍從星所之。



 飛星
 大如缶若甕,後皎然白,前卑後高,此謂頓頑,其所從者多死亡。飛星大如缶若甕,後然皎白,星滅後,白者曲環如車輪,此謂解銜,其國人相斬為爵祿。飛星大如缶若甕,其後皎然白,長數丈,星滅後,白者化為雲流下,名曰大滑,所下有流血積骨。



 枉矢,類流星,色蒼黑,蛇行,望之如有毛,目長數匹,著天,主反萌,主射愚。見則謀反之兵合射所誅,亦為以亂伐亂。



 天狗,狀如大奔星,色黃,有聲,其止地,類狗。所墜,望之如火光,炎炎衝天,其上銳,其下員,如數頃田處。或曰,星有
 毛,旁有短彗,下有狗形者。或曰,星出,其狀赤白有光,下即為天狗。一曰,流星有光,見人面,墜無音,若有足者,名曰天狗。其色白,其中黃,黃如遺火狀。主候兵討賊。見則四方相射,千里破軍殺將。或曰,五將鬥,人相食,所往之鄉有流血。其君失地,兵大起,國易政,戒守禦。



 營頭,有雲如壞山墮,所謂營頭之星,所墮,其下覆軍,流血千里。亦曰流星晝隕名營頭。



 ○雲氣



 瑞氣:一曰慶雲。若煙非煙,若雲非雲,郁郁紛紛,蕭索輪困,是謂慶雲,亦曰景雲。此喜氣也,太平之應。二曰歸邪。
 如星非星,如雲非雲。或曰,星有兩赤彗向上,有蓋,下連星。見,必有歸國者。三曰昌光,赤,如龍狀,聖人起,帝受終,則見。



 妖氣:一曰虹蜺,日旁氣也,斗之亂精。主惑心,主內淫,主臣謀君,天子詘,后妃顓,妻不一。二曰䍧雲,如狗,赤色,長尾;為亂君,為兵喪。



 ○十煇



 《周禮》眡昆氏掌十煇之法,以觀妖祥,辨吉凶。一曰昆,謂陰陽五色之氣,浸淫相侵。或曰,抱珥背璚之屬,如虹而短是也。二曰象,謂雲氣成形,象如赤烏,夾日以飛之類
 是也。三曰觿日傍氣,刺日,形如童子所佩之觿。四曰監,謂雲氣臨在日上也。五曰闇,謂日月蝕,或曰脫光也。六曰瞢,謂瞢瞢不光明也。七曰彌,謂白虹彌天而貫日也。八曰序,謂氣若山而在日上。或曰,冠珥背璚,重疊次序,在于日旁也。九曰隮,謂暈氣也。或曰,虹也,《詩》所謂「朝隮于西」者也。十曰想,謂氣五色有形想也,青饑,赤兵,白喪,黑憂,黃熟。或曰,想,思也,赤氣為人狩之形,可思而知其吉凶也。



 凡遊氣蔽天,日月失色,皆是風雨之候也,沈陰,日月俱無光,晝不見日,夜不見星,有雲障之,兩敵相當,陰相圖
 議也。日蒙蒙無光,士卒內亂。又曰,數日俱出,若鬥,天下兵起,大戰。日鬥,下有拔城。日戴者,形如直狀,其上微起,在日上為戴。戴者,德也,國有喜也。一云,立日上為戴。青赤氣抱在日上,小者為冠,國有喜事。青赤氣小而交於日下為纓,青赤氣小而員,一二在日下左右者為紐。青赤氣如小半暈狀,在日上為負,負者得地為喜。又曰,青赤氣長斜倚日旁為戟。青赤氣員而小,在日左右為珥,黃白者有喜。又曰,有軍,日有一珥為喜。在日西。西軍戰勝。在日東,東軍戰勝。南北亦如之。無軍而珥,為拜將。又日旁如半環向日為抱。青赤氣如月初生,背日者為
 背,又曰,背氣青赤而曲,外向為叛象,分為反城。璚者如帶,璚在日四方。青赤氣長而立日旁為直,日旁有一直,敵在一旁欲自立,從直所擊者勝。日旁有二直三抱,欲自立者不成,順抱擊者勝,殺將。氣形三角,在日西方為提,青赤氣橫在日上下為格。氣如半暈,在日下為承。承者,臣承君也。又曰,日下有黃氣三重若抱,名曰承福,人主有吉喜,且得地。青白氣如履,在日下者為履。日旁抱五重,戰順抱者勝。日一抱一背,為破走。抱者,順氣也;背者,逆氣也。兩軍相當,順抱擊逆者勝,故曰破走。日抱且兩珥,一虹貫抱至日,順虹擊者勝,殺將。日抱兩珥且璚,
 二虹貫抱至日,順虹擊者勝。日重抱,內有璚,順抱擊者勝。亦曰,軍內有欲反者。日重抱,左右二珥,有白虹貫抱,順抱擊勝,得二將。有三虹,得三將。日抱黃白潤澤,內赤外青,天子有喜,有和親來降者;軍不戰,敵降,軍罷。色青黃,將喜;赤,將兵爭,白,將有喪,黑,將死。日重抱且背,順抱擊者勝,得地,若有罷師。日重抱,抱內外有璚,兩珥,順抱擊者勝,破軍,軍中不和,不相信。日旁有氣,員而周匝,內赤外青,名為暈。日暈者,軍營之象。周環匝日,無厚薄,敵與軍勢齊等。若無軍在外,天子失御,民多叛。日暈有五色,有喜;不得五色者有憂。



 凡占,兩軍相當,必謹審日月暈氣,知其所起,留止遠近,應與不應,疾遲,大小,厚薄,長短,抱背為多小,有無,虛實,久亟,密疏,澤枯。相應等者勢等。近勝遠,疾勝遲,大勝小,厚勝薄,長勝短,抱勝背,多勝少,有勝無,實勝虛,久勝亟,密勝疏,澤勝枯。重背,大破;重抱為和親;抱多,親者益多;背為天下不和,分離相去,背於內者離於內,背於外者離於外也。



 ○雜氣



 天子氣,內赤外黃,四方所發之處當有王者。若天子欲有遊往處,其地亦先發此氣。或如城門隱隱在氣霧中,
 恆帶殺氣森森然。或如華蓋在霧氣中,或氣象青衣人無手,在日西,或如龍馬,或雜色鬱鬱衝天者,此皆帝王氣。



 猛將之氣,如龍,如猛獸;或如火煙之狀;或白如粉沸;或如火光之狀,夜照人;或白而赤氣繞之,或如山林竹木,或紫黑如門上樓;或上黑下赤,狀似黑旌;或如張弩;或如埃塵,頭銳而卑,本大而高。此皆猛將之氣也。氣發漸漸如雲,變作山形,將有深謀。



 凡軍勝之氣,如堤如阪,前後磨地。或如水光;將軍勇,士卒猛。或如山堤,山上若林木;將士驍勇。或如埃塵粉沸,
 其色黃白;或如人持斧向敵;或如蛇舉首向敵,或氣如覆舟,雲如牽牛;或有雲如鬥雞,赤白相隨,在氣中;或發黃氣,皆將士精勇。



 凡氣上黃下白,名曰善氣;所臨之軍,敵欲求和退。



 凡負氣,如馬肝色,或如死灰色;或類偃蓋,或類偃魚;或黑氣如壞山墜軍上者,名曰營頭之氣;或如群牛群豬,在氣中。此衰氣也。或如懸衣,如人相隨;或紛紛如轉蓬,或如揚灰;或雲如卷席,如匹布亂穰者,皆為敗徵。氣如繫牛,如人臥,如雙蛇,如飛鳥,如決隄垣,如壞屋,如驚鹿相逐,如兩雞相向,此皆為敗軍之氣。



 凡降人氣,如人十十五五,皆叉手低頭;又云,如人叉手相向。或氣如黑山,以黃為緣者,皆欲降伏之象也。



 凡堅城之上,有黑雲如星,名曰軍精。或白氣如旌旗,或青雲黃雲臨城。皆有大喜慶。或氣青色如牛頭觸人,或城上氣如煙火。如雙蛇,如杵形向外,或有雲分為兩彗狀者,皆不可攻。



 凡屠城之氣,或赤如飛鳥,或赤氣如敗車,或有赤黑氣如貍皮斑,或城中氣聚如樓,出見於外;營上有雲如眾人頭,赤色,其城營皆可屠。氣如雄雉臨城,其下必有降者。



 凡伏兵有黑氣,渾渾員長,赤氣在其中:或白氣粉沸,起如樓狀;或如幢節狀,在烏雲中;或如赤杵在烏雲中,或如烏人在赤雲中。



 凡暴兵氣,白,如瓜蔓連結,部隊相逐,須臾罷而復出;或白氣如仙人,如仙人衣,千萬連結,部隊相逐,罷而復興,當有千里兵來。或氣如人持刀楯,雲如人,色赤,所臨城邑有卒兵至。或赤氣如人持節,兵來未息。雲如方虹。此皆有暴兵之象。



 凡戰氣,青白如膏;如人無頭;如死人臥;如丹蛇,赤氣隨之,必大戰,殺將。四望無雲,見赤氣如狗入營,其下有流
 血。



 凡連陰十日,晝不見日,夜不見月,亂風四起,欲雨而無雨,名曰蒙,臣有謀。霧氣若晝若夜,其色青黃,更相奄冒,乍合乍散,亦然。視四方常有大雲五色具者,其下賢人隱也。青雲潤澤蔽日,在西北,為舉賢良。雲氣如亂穰,大風將至,視所從來。雲甚潤而厚,大雨必暴至。四始之日,有黑雲氣如陣,厚大重者,多雨。氣若霧非霧,衣冠不濡,見則其城帶甲而趣。日出沒時有霧氣橫截之,白者喪,烏者驚,三日內雨者各解。有雲如蛟龍,所見處將軍失魄。有雲如鵠尾來蔭國上,三日亡。有雲赤黃色四塞,終
 日竟夜照地者,大臣縱恣。有雲如氣,昧而濁,賢人去,小人在位。



 凡白虹者,百殃之本,眾亂所基。霧者,眾邪之氣,陰來冒陽。



 凡白虹霧,姦臣謀君,擅權立威。晝霧夜明,臣志得申。



 凡夜霧白虹見,臣有憂;晝霧白虹見,君有憂。虹頭尾至地,流血之象。



 凡霧氣不順四時,逆相交錯,微風小雨,為陰陽氣亂之象。積日不解,晝夜昏闇,天下欲分離。



 凡天地四方昏蒙若下塵,十日五日已上,或一月,或一時,雨水沾衣而有土,名曰霾。故曰,天地霾,君臣乖。



 凡海旁蜄氣象樓臺,廣野氣成宮闕,北夷之氣如牛羊群畜穹廬,南夷之氣類舟船幡旗。自華以南,氣下黑上赤;嵩高、三河之郊,氣正赤;恒山之北,氣青;勃碣海岱之間,氣皆正黑;江淮之間,氣皆白;東海氣如員簦;附漢河水,氣如引布;江漢氣勁如杼,濟水氣如黑,渭水氣如狼白尾,淮南氣如白羊,少室氣如白兔青尾,恒山氣如黑牛青尾。東夷氣如樹,西夷氣如室屋,南夷氣如闍臺,或類舟船。



 陣雲如立垣。杼軸雲類軸,搏,兩端兌。杓雲如繩,居前亙天,其半半天;其JH者類闕旗故。鉤雲句曲。諸此雲見,以五色占。而澤摶密,其見動人,乃有兵必起,合
 鬥其直。雲氣如三匹帛,廣前兌後,大軍行氣也。



 韓雲如布,趙雲如牛,楚雲如日,宋雲如車,魯雲如馬,衛雲如犬,周雲如車輪,秦雲如行人,魏雲如鼠,鄭雲如絳衣,越雲如龍,蜀雲如囷。



 車氣乍高乍下,往往而聚。騎氣卑而布。卒氣搏。前卑後高者,疾。前方而高後銳而卑者,卻。其氣平者其行徐。前高後卑者,不止而返。校騎之氣,正蒼黑,長數百丈。遊兵之氣如彗掃,一雲長數百丈,無根本。喜氣上黃下赤,怒氣上下赤,憂氣上下黑。土功氣黃白。徙氣白。



 凡候氣之法,氣初出時,若雲非雲,若霧非霧,仿佛若可見。初出森森然,在桑榆上,高五六尺者,是千五百
 里外。平視則千里,舉目望即五百里;仰瞻中天,即百里內。平望,桑榆間二千里;登高而望,下屬地者,三千里。敵在東,日出候之;在南,日中候之,在西,日入候之;在北,夜半候之。軍上氣,高勝下,厚勝薄,實勝虛,長勝短,澤勝枯。氣見以知大,占期內有大風雨,久陰,則災不成。



 ◎史傳事驗



 ○天變



 惠帝元康二年二月,天西北大裂。案劉向說:「天裂,陽不足,地動,陰有餘。」是時人主昏瞀,妃后專制。



 太安二年八月庚午,天中裂為二,有聲如雷者三。君道虧而臣下專僭之象也。是日,長沙王奉帝出距成都、河
 間二王,後成都、河間、東海又迭專威命,是其應也。



 穆帝升平五年八月已卯夜,天中裂,廣三四丈,有聲如雷,野雉皆鳴。是後哀帝荒疾,海西失德,皇太后臨朝,太宗總萬機,桓溫專權,威振內外,陰氣盛,陽氣微。



 元帝太興二年八月戊戌,天鳴東南,有聲如風水相薄。京房易妖占曰:「天有聲,人主憂。」三年十月壬辰,天又鳴,甲午止。其後王敦入石頭,王師敗績。元帝屈辱,制於強臣,即而晏駕,大恥不雪。



 安帝隆安五年閏月癸丑,天東南鳴。六年九月戊子,天東南又鳴。是後桓玄篡位,安帝播越,憂莫大焉。鳴每
 東南者,蓋中興江外,天隨之而鳴也。



 義熙元年八月,天鳴,在東南,京房《易傳》曰:「萬姓勞,厥妖天鳴。」是時安帝雖反正,而兵革歲動,眾庶勤勞也。



 ○日蝕



 魏文帝黃初二年六月戊辰晦,日有蝕之。有司奏免太尉,詔曰:「災異之作,以譴元首,而歸過股肱,豈禹湯罪已之義乎!其令百官各虔厥職。後有天地眚,勿腹劾三公。」三年正月丙寅朔,日有蝕之。十一月庚申晦,又日有蝕之。五年十一月戊申晦,日有蝕之。
 明帝太和初,太史令許芝奏,日應蝕,與太尉於靈臺祈禳。帝曰:「蓋聞人主政有不德,則天懼之以災異,所以譴告,使得自修也。故日月薄蝕,明治道有不當者。朕即位以來,即不能光明先帝聖德,而施化有不合於皇神,故天上有寤之。宜敕政自修,有以報於神明。天之於人,猶父之於子,未有父欲責其子,而可獻盛饌以求免也。今外欲譴上公與太史令俱穰祠之,於義未聞也。群公卿士大夫,其各勉修厥職。有可以補朕不逮者,各封上之。」太和五年十一月戊戌晦,日有蝕之。
 六年正月戊辰朔,日有蝕之。見吳歷。



 青龍元年閏月庚寅朔,日有蝕之。



 少帝正始元年七月戊申朔,日有蝕之。三年四月戊戌朔,日有蝕之。四年五月丁丑朔,日有蝕之。五年四月丙辰朔,日有蝕之。六年四月壬子朔,日有蝕之。十月戊申朔,又日有蝕之。八年二月庚午朔,日有蝕之。是時曹爽專政,丁謐、鄧颺等轉改法度。會有日蝕之變,詔群臣問得失。蔣濟上疏
 曰:「昔大舜佐治,戒在比周。周公輔政,慎於其朋。齊侯問災,晏子對以布惠;魯君問異,臧孫答以緩役。塞變應天,乃實人事。」濟旨譬甚切,而君臣不悟,終至敗亡。九年正月乙未朔,日有蝕之。



 嘉平元年二月二月已示朔,日有蝕之。



 高貴鄉公甘露四年七月戊子朔,日有蝕之。五年正月乙酉朔,日有蝕之。京房易占曰:「日有蝕乙酉,君弱臣強。司馬將兵,反徵其王。」五月,有成濟之變。



 元帝景元二年五月丁未朔,日有蝕之。三年十一月已亥朔,日有蝕之。



 武帝泰始二年七月丙午晦,日有蝕之。十月丙午朔,日有蝕之。七年十月丁丑朔,日有蝕之。八年十月辛未朔,日有蝕之。九年四月戊辰朔,日有蝕之。又,七月丁酉朔,日有蝕之。十年正月乙未,三月癸亥,並日有蝕之。



 咸寧元年七月甲申晦,日有蝕之。三年正月丙子朔,日有蝕之。四年正月庚午朔,日有蝕之。



 太康四年三月辛丑朔,日有蝕之。七年正月甲寅朔,日有蝕之。八年正月戊申朔,日有蝕之。九年正月壬申朔,六月庚子朔,並日有蝕之。永熙元年四月庚申,帝崩。



 惠帝元庚九年十一月甲子朔,日有蝕之。十二月,廢皇太子為庶人,尋殺之。



 永康元年正月已卯,四月辛卯朔,並日有蝕之。



 永寧元年閏月丙戌朔,日有蝕之。



 光熙元年正月戊子朔,七月乙酉朔,並日有蝕之。十一
 月,惠帝崩。十二月壬午朔,又日有蝕之。



 懷帝永嘉元年十一月戊申朔,日有蝕之。二年正月丙子朔,日有蝕之。六年二月壬子朔,日有蝕之。



 愍帝建興四年六月丁巳朔,十二月甲申朔,並日有蝕之。五年五月丙子,十一月丙子,並日有蝕之。時帝蒙塵于平陽。



 元帝太興元年四月丁丑朔,日有蝕之。



 明帝太寧三年十一月癸已朔,日有蝕之,在卯至斗。斗,
 吳分也。其後蘇峻作亂。



 成帝咸和二年五月甲申朔,日有蝕之,在井。井,主酒食,女主象也。明年,皇太后以憂崩。六年三月壬戌朔,日有蝕之。是時帝已年長,每幸司徒第,猶出入見王導夫人曹氏如子弟之禮。以入君而警敬人臣之妻,有虧君德之象也。九年十月乙未朔,日有蝕之。是時帝既冠,當親萬機,而委政大臣,著君道有虧也。



 咸康元年十月乙未朔,日有蝕之。七年二月甲子朔,日有蝕之。三月,杜皇后崩。
 八年正月乙未朔,日有蝕之。京都大雨,郡國以聞。是謂三朝,王者惡之。六月而帝崩。



 穆帝永和二年四月己酉,七年正月丁酉,八年正月辛卯,並日有蝕之。十二年十月癸巳朔,日有蝕之,在尾。燕分,北狄之象也。是時邊表姚襄、苻生互相吞噬,朝廷憂勞,征伐不止。



 升平四年八月辛丑朔,日有蝕之,幾既在角。凡蝕,淺者禍淺,深者禍大。角為天門,入主惡之。明年而帝崩。



 哀帝隆和元年三月甲寅朔,十二月戊午朔,並日有蝕之。明年而帝有疾,不識萬機。



 海西公太和三年三月丁巳朔,五年七月癸酉朔,並日有蝕之。皆海西被廢之應也。



 孝武帝寧康三年十月癸酉朔,日有蝕之。



 太元四年閏月己酉朔,日有蝕之。是時苻堅攻沒襄陽,執朱序。六年六月庚子朔,日有蝕之。九年十月辛亥朔,日有蝕之。十七年五月丁卯朔,日有蝕之。二十年三月庚辰朔,日有蝕之。明年帝崩。



 安帝隆安四年六月庚辰朔,日有蝕之。是時元顯執政。



 元興二年四月癸巳朔,日有蝕之。其冬桓玄篡位。



 義熙三年七月戊戌朔,日有蝕之。十年九月丁巳朔,日有蝕之。十一年七月辛亥晦,日有蝕之。十三年正月甲戌朔,日有蝕之。明年,帝崩。



 恭帝元熙元年十一月丁亥朔,日有蝕之。自義熙元年至是,日蝕皆從上始,皆為革命之徵。



 《周禮》只眡祲氏掌十煇之法,以觀妖祥,辯吉兇,有祲、象、鐫、監、闇、瞢、彌、序、躋、想凡十。後代名變,說者莫同。今錄其著應以次之云。



 吳孫權赤烏十一年二月,白虹貫日。權發詔戒懼。



 武帝泰始五年七月甲寅,日暈再重,白虹貫之。



 太康元年正月已丑朔,五色氣冠日,自卯至酉。占曰:「君道失明,丑為斗牛,主吳越。」是時孫皓淫暴,四月降。



 惠帝元康元年十一月甲申,日暈,再重,青赤有光。九年正月,日中有若飛燕者,數日乃消。王隱以為愍懷廢死之徵。



 永康元年正月癸亥朔,日暈,三重。十月乙未,日闇,黃霧四塞。占曰:「不及三年,下有拔城大戰。」十二月庚戌,日中有黑氣。京房《易傳》曰:「祭天不順茲謂逆,厥異日中有黑
 氣。」



 永寧元年九月甲申,月中有黑子。京房易占:「黑者陰也,臣不掩君惡,令下見,百姓惡君,則有此變。」又曰:「臣有蔽主明者。」



 太安元年十一月,日中有黑氣。



 永興元年十一月,日中有黑氣分日。



 光熙元年五月壬辰、癸巳,日光四散,赤如血流,照地皆赤。甲午又如之。占曰:「君道失明。」



 懷帝永嘉元年十一月乙亥,黃黑氣掩日,所照皆黃。案《河圖》占曰:「日薄也」。其說曰:「凡日蝕皆於朔晦,有不於晦
 朔者為日薄。雖非日月同宿,時陰氣盛,掩日光也。」占類日蝕。二年正月戊申,白虹貫日,二月癸卯,白虹貫日,青黃暈,五重。占曰:「白虹貫日,近臣為亂,不則諸侯有反者。暈五重,有國者受其祥,天下有兵,破亡其地。」明年,司馬越暴蔑人主。五年,劉聰破京都,帝蒙塵於寇庭。五年三月庚申,日散光,如血下流,所照皆赤。日中有若飛燕者。



 愍帝建興二年正月辛未辰時,日隕于地。又有三日相承,出於西方而東行。
 五年正月庚子,三日並照,虹蜺彌天。日有重暈,左右兩珥。占曰:「白虹,兵氣也。三四五六日俱出並爭,天下兵作,丁巳亦如其數。」又曰:「三日並出,不過三旬,諸侯爭為帝。日重暈,天下有立王。暈而珥,天下有立侯。」故陳卓曰:「當有大慶,天下其三分乎!」三月而江東改元為建武、劉聰、李雄亦跨曹劉疆宇,於是兵連累葉。



 元帝太興元年十一月乙卯,日夜出,高三丈,中有赤青珥。四年二月癸亥,日鬥。三月癸未,日中有黑子,辛亥,帝親錄訊囚徒。



 永昌元年十月辛卯,日中有黑子。時帝寵幸劉隗,擅威福,虧傷君道,王敦因之舉兵,逼京都,禍及忠賢。



 明帝太寧元年正月乙卯朔,日暈無光。癸巳,黃霧四塞。占曰:「君道失明,陰陽昏,臣有陰謀。」京房曰:「下專刑,茲謂分威,蒙微而日不明。」先是,王敦害尚書令刁協、僕射周顗、驃騎將軍戴若思等,是專刑之應。敦既陵上,卒伏其辜。十一月丙子,白虹貫日。史官不見,桂陽太守華包以聞。



 成帝咸和九年七月,白虹貫日。



 咸康元年七月,白虹貫日。
 二年七月,白虹貫日。自後庾氏專政,由后族而貴,蓋亦婦人擅國之義,故頻年白虹貫日。八年正月壬申,日中有黑子,丙子乃滅。夏,帝崩。



 穆帝永和八年,張重華在涼州,日暴赤如火,中有三足為烏,形見分明,五日乃止。十年十月庚辰,日中有黑子,大如雞卵。十一年三月戊申,日中有黑子,大如桃,二枚。時天子幼弱,久不親國政。



 升平三年十月丙午,日中有黑子,大如雞卵。少時而帝崩。



 海西公太和三年九月戊辰夜,二虹見東方。四年四月戊辰,日暈,厚密,白虹貫日中。十月乙未,日中有黑子。五年二月辛酉,日中有黑子,大如李。六年三月辛未,白虹貫日,日暈,五重。十一月,桓溫廢帝,即簡文咸安元年也。



 簡文咸安二年十一月丁丑,日中有黑子。



 孝武寧康元年十一月己酉,日中有黑子,大如李。二年三月庚寅,日中有黑子二枚,大如鴨卵。十一月己巳。日中有黑子,大如雞卵。時帝已長,而康獻皇后以從
 嫂臨朝,實傷君道,故日有瑕也。



 太元十三年二月庚子,日中有黑子二,大如李。十四年六月辛卯,日中又有黑子,大如李。二十年十一月辛卯,日中又有黑子。是時會稽王以母弟干政。



 安帝隆安元年十二月壬辰,日暈,有背璚。是後不親萬機,會稽王世子元顯專行威罰。四年十一月辛亥,日中有黑子。



 元興元年二月甲子,日暈,白虹貫日中。三月庚子,白虹貫日。未幾,桓玄剋京都,王師敗績。明年,玄篡位。



 義熙元年五月庚午。日有彩珥。六年五月丙子,日暈,有璚。時有廬循逼京都,內外戒嚴。七月,循走。七年七月,五虹見東方。占曰:「天子黜。」其後劉裕代晉。。十年,日在東井,有白虹十餘丈在南干日。災在秦分,秦亡之象。



 恭帝元熙二年正月壬辰,白氣貫日,東西有直珥各一丈,白氣貫之交匝。



 ○月變



 魏文帝黃初四年十一月,月暈北斗。占曰:「有大喪,赦天
 下。」七年五月,帝崩,明帝既位,大赦天下。



 孝懷帝永嘉五年三月壬申丙夜,月蝕,既。丁夜又蝕,既。占曰:「月蝕盡,大人憂。」又曰:「其國貴人死。」



 海西公太和四年閏月乙亥,月暈軫,復有白暈貫月北,暈斗柄三星。占曰:「王者惡之。」六年,桓溫廢帝。



 安帝隆安五年三月甲子,月生齒。占曰:「月生齒,天子有賊臣,群下自相殘。」桓玄篡逆之徵也。



 義熙九年十二月辛卯朔,月猶見東方。是謂之仄匿,則侯王其肅。是時劉裕輔政,威刑自己,仄匿之應云。十一年十一月乙未,月入輿鬼而暈。占曰:「主憂,財寶
 出。」一曰:「月暈,有赦。」



 ○月奄犯五緯



 凡月蝕五星,其國皆亡。五星入月,其野有逐相。



 魏明帝太和五年十二月甲辰,月犯填星。



 青龍二年十月乙丑,月又犯填星。占同上。戊寅,月犯太白,占曰:「人君死,又為兵。」景初元年七月,公孫文懿叛。二年正月,遣宣帝討之。三年正月,天子崩。四年三月已巳,太白與月俱加景晝見,月犯太白。占同上。



 景初元年十月丁未,月犯熒惑。占曰:「貴人死。」二年四月,
 司徒韓既薨。



 齊王嘉平元年正月甲午,太白襲月。宣帝奏永寧太后廢曹爽等。



 惠帝太安二年十一月庚辰,歲星入月中。占曰:「國有逐相。」十二月壬寅,太白犯月。占曰:「天下有兵。」三年正月乙卯,月犯太白,占同青龍元年。七月,左衛將軍陳等率眾奉帝伐成都王,六軍敗績,兵逼乘輿。後二年,帝崩。



 元帝太興二年十一月辛巳,月犯熒惑。占曰:「有亂臣。」三年十二月己未,太白入月,在斗。郭璞曰:「月屬《坎》,陰府
 法象也。太白金行而犯之,天意若曰,刑理失中,自毀其法。」四年十二月丁亥,月犯歲星,在房。占曰:「其國兵饑,人流亡。」永昌元年三月,王敦作亂,率江荊之眾來攻,敗京都,殺將相。又,鎮北將軍劉隗出奔,百姓並去南畝。困於兵革。四月,又殺湘州刺史、譙王司馬承,鎮南將軍甘卓。



 成帝咸康元年二月乙未,太白入月。四月甲午。月犯太白。四年四月已巳,七月乙巳,月俱奄太白。占曰:「人君死。又
 為兵,人主惡之。」明年,石季龍之眾大冠沔南,於是內外戒嚴。五年四月辛示,月犯歲星,在胃。占曰:「國饑,人流。」乙未,月犯歲星,在昴。及冬,有沔南、邾城之敗,百姓流亡萬餘家。六年二月乙未,太白入月。占曰:「人主死。」四月甲午,月犯太白。占曰:「人主惡之。」



 穆帝永和八年十二月,月在東井,犯歲星。占曰:秦饑,人流亡。」是時兵革連起。十年十一月,月奄填星,在輿鬼。占曰:「秦有兵。」時桓溫伐苻健,健堅壁長安,溫退。十二年八月,桓溫破姚襄。



 升平元年十一月壬午,月奄歲星,在房。占曰:「人飢。」一曰:「豫州有災。」二年閏三月乙亥,月犯歲星,在房。占同上。三年,豫州刺史謝萬敗。三年三月乙酉,月犯太白,在昴。占曰:「人君死。」一曰:「趙地有兵,胡不安。」四年正月,暮容俊卒。五年正月乙丑辰時,月在危宿,奄太白。占曰:「天下靡散。」三月丁未,月犯填星,在軫。占曰:「為大喪。」五月,穆帝崩。七月,慕容恪攻冀州刺史呂護於野王,拔之,護奔走。時桓溫以大眾次宛,聞護敗,乃退。



 哀帝興寧元年十月丙戌,月奄太白,在須女。占曰:「天下靡散。」一曰:「災在揚州。」三年,洛陽沒。其後桓溫傾揚州資實北討,敗績,死亡太半。及征袁真,淮南殘破。後慕容及苻堅互來侵境。三年正月乙卯,月奄歲星,在參。占曰:「參,益州分也。」六月,鎮西將軍益州刺史周撫卒。十月,梁州刺史司馬勳入益州以叛。朱序率眾助刺史周楚討平之。



 海西太和元年二月丙子,月奄熒惑,在參。占曰:「為內亂,帝不終之徵。」一曰:「參,魏地。」五年,慕容為苻堅所滅。



 孝武太元十二年二月戊寅,熒惑入月。占曰:「有亂臣死,
 若有相戮者。」一曰:「女親為政,天下亂。」是時瑯邪王輔政,王妃從兄王國寶以姻暱受寵。又陳郡人袁悅昧私茍進,交遘主相,扇揚朋黨。十三年,帝殺悅於市。於是主相有隙,亂階興矣。十三年十二月戊子,辰星入月,在危。占曰:「賊臣欲殺主,不出三年,必有內惡。」是後慕容垂、翟遼、姚萇、苻登、慕容永並阻兵爭強。十四年十二月乙未,月犯歲星。占並同上。十五年,翟遼據司兗,眾軍累討弗剋,慕容氏又跨略并冀。七月,旱。八月,諸郡大水,兗州又蝗。
 十八年正月乙酉,熒惑入月。占曰:「憂在宮中,非賊乃盜也。」一曰:「有亂臣,若有戮者。」二十一年九月,帝暴崩內殿,兆庶宣言,夫人張氏潛行大逆。又,王國寶邪狡,卒伏其辜。十九年四月已巳,月奄歲星,在尾。占曰:「為饑,燕國亡。」二十年,慕容垂遣寶伐魏,反為所破,死者數萬人。二十一年,垂死,國遂衰亡。



 安帝隆安元年六月庚午,月奄太白,在太微端門外。占曰:「國受兵。」乙酉,月奄歲星,在東壁。占曰:「為饑,衛地有兵。」二年六月,郗恢遣鄧啟方等以萬人伐慕容寶於滑臺,
 啟方敗。三年九月,桓玄等並舉兵,於是內外戒嚴。四年正月乙亥,月犯填星,在牽牛。占曰:「吳越有兵喪,女主憂。」六月乙未,月又犯填星,在牽牛。十月乙未,月奄歲星,在北河。占曰:「為饑,胡有兵。」其四年五月,孫恩破會稽,殺內史謝琰。後又破高雅之於餘姚,死者十七八。七月,太皇太后李氏崩。元興元年,孫恩寇臨海,人眾餓死,散亡殆盡。



 元興元年四月辛丑,月奄辰星。七月,大饑,人相食。二年十一月辛巳,月犯熒惑。占悉同上。二年十二月,桓玄篡位,放遷帝、后於尋陽,以永安何皇后為零陵君。三
 年二月,劉裕盡誅桓氏。三年二月甲辰,月醃歲星於左角。占曰:「天下兵起。」是年二月丙辰,劉裕起義兵,殺桓修等。明年正月,眾軍攻桓振,卒滅諸桓。



 義熙元年四月己卯,月犯填星,在東壁。占曰:「其地亡國。」一曰:「貴人死。」七月己未,月奄填星,在東壁。占曰:「其國以伐己。」一曰:「人流。」十月丁巳,月奄填星,在營室。占同上。十一月,荊州刺史魏詠之卒。二年二月,司馬國璠等攻沒弋陽。三年,恆徒揚州刺史王謐薨。四年正月,太保、武陵王遵薨。三月,左僕射孔安國薨。
 二年十二月丙午,月奄太白,在危。占曰:「齊亡國。」一曰:「彊國君死。」五年四月,劉裕大軍北討慕容超,卒滅之。七年六月庚子,月犯歲星,在畢。占曰:「有邊兵,且饑。」八月乙未,月犯歲星,在參。占曰:「益州兵飢。」七月,朱齡石剋蜀,蜀人尋反,又討之。八年正月庚戌,月犯歲星,在畢。占同上。九年七月,朱齡石滅蜀。十二年五月五月甲申,月犯歲星,在左角。占曰:「為饑。」十四年四月壬申,月犯填星於張。占曰:「天下有大喪。」其明年,帝崩。



 恭帝元熙元年七月,月犯歲星。占悉同上。十二月丁巳,月犯太白于羽林。二年六月,帝遜位,禪宋。



 ○五星聚舍



 魏明帝太和四年七月壬戌,太白犯歲星。占曰;「太白犯五星,有大兵。」五年三月,諸葛亮以大眾寇天水。時宣帝為大將軍,距退之。



 青龍二年二月己未,太白犯熒惑。占曰:「大兵起,有大戰。」是年四月,諸葛亮據渭南,吳亦起兵應之,魏東西奔命。



 惠帝元康三年,填星、歲星、太白三星聚于畢昴。占曰:「為兵喪。畢昴,趙地也。」後賈后陷殺太子,趙王廢后,又殺之,
 斬張華、裴頠,遂篡位,廢帝為太上皇,天下從此遘亂連禍。



 永寧二年十一月,熒惑、太白鬥于虛危。占曰:「大兵起,破軍殺將。虛危,又齊分也。」十二月,熒惑襲太白于營室。占曰;「天下兵起,亡君之戒。」一曰:「易相。」初,齊王冏之京都,因留輔政,遂專傲無君。是月,成都、河間檄長沙王乂討之,冏、乂交戰,攻焚宮闕,冏兵敗,夷滅。又殺其兄上軍將軍寔以下二千餘人。太安二年,成都又攻長沙,於是公私饑困,百姓力屈。



 太安三年正月,熒惑犯歲星。占曰:「有戰。」七月。左衛將軍
 陳奉帝伐成都,六軍敗績。



 光熙元年九月,填星犯歲星。占曰:「填與歲合,為內亂。」是時司馬越專權,終以無禮破滅,內亂之應也。十二月癸未,太白犯填星。占曰:「為內兵,有大戰。」是後河間王為東海王越所殺。明年正月,東海王越殺諸葛玫等。五月,汲桑破馮嵩,殺東燕王。八月,茍晞大破汲桑。



 懷帝永嘉六年七月,熒惑、歲星、太白聚牛、女之間,徘徊進退。案占曰:「牛女,揚州分」,是後兩都傾覆,而元帝中興場土。



 建武元年五月癸未,太白、熒惑合於東井。占曰:「金火合
 曰爍,為喪。」是時愍帝蒙塵於平陽,七月崩于寇庭。



 元帝太興二年七月甲午,歲星、熒惑會于東井。八月乙未,太白犯歲星,合在翼。占曰:「為兵饑。」三年六月丙辰,太白與歲星合于房。占同上。永昌元年王敦攻京師,六軍敗績。王敦尋死。



 成帝咸康三年十一月乙丑,太白犯歲星于營室。占曰:「為兵饑。」四年二月,石季龍破幽州,遷萬餘家以南。五年,季龍眾五萬寇沔南,略七千餘家而去。又騎二萬圍陷邾城,殺略五千餘人。四年十二月癸丑,太白犯填星,在箕。占曰:「王者亡地。」七
 年,慕容皝自稱燕王。七年三月,太白熒惑合于太微中,犯左執法。明年,顯宗崩。八年十二月己酉,太白犯熒惑于胃。占曰:「大兵起。」其後庾翼大發兵,謀伐石季龍,專制上流。



 康帝建元元年八月丁未,太白犯歲星,在軫。占曰:「有大兵。」是年石季龍將劉寧寇沒狄道。



 穆帝永和四年五月,熒惑入婁,犯填星。占曰:「兵大起,有喪,災在趙。」其年石季龍死,來年冉閔殺石遵及諸胡十萬餘人,其後褚裒北伐,喪眾而薨。
 六年三月戊戌,熒惑犯歲星。占曰:「為戰。」七年三月戊子,歲星、熒惑合于奎。其年劉顯殺石祗及諸胡帥,中士大亂。十二有年七月丁卯,太白犯填星,在柳。占曰:「周地有大兵。」其年八月,桓溫伐苻健,退,因破姚襄於伊水,定周地。



 升平二年八月戊午,熒惑犯填星,在張。占曰:「兵大起。」三年八月庚午,太白犯填星,在太微中。占曰:「王者惡之。」五年十月丁卯,熒惑犯歲星,在營室。占曰:「大臣有匿謀。」一曰:「衛地有兵。」時桓溫擅權,謀移晉室。



 海西公太和元年八月戊午,太白犯歲星,在太微中。
 三年六月甲寅,太白奄熒惑,在太微端門中。六年,海西公廢。



 簡文咸安二年正月己酉,歲星犯填星,在須女。占曰:「為內亂。」七月,帝崩,桓溫擅權,謀殺侍中王坦之等,內亂之應。



 孝武寧康二年十一月癸酉,太白奄熒惑,在營室。占曰:「金火合為爍,為兵喪。」太元元年七月,苻堅伐涼州,破之,虜張天錫。



 太元十一年十二月己丑,太白犯歲星。占曰:「為兵饑。」是時河朔未平,兵連在外,冬大饑。
 十七年九月丁丑,歲星、熒惑、填星同在亢、氐。十二月癸酉,填星去,熒惑、歲星猶合。占曰:「三星合,是謂驚立絕行,內外有兵喪與饑,改立王公。」十九年十月,太白、填星、熒惑辰星合于氐。十二月癸丑,太白犯歲星,在斗。占曰:「為亂饑,為內兵。斗吳越分。」至隆字元年,王恭等舉兵,顯王國寶之罪,朝廷殺之。是後連歲水旱饑。



 安帝隆安元年二月,歲星、熒惑皆入羽林。占曰:「中軍兵起。」四月,王恭等舉兵,內外戒嚴。



 元興元年八月庚子,太白犯歲星,在上將東南。占曰:「楚
 兵饑。」一曰:「災在上將。」二年,桓玄篡位。三年,劉裕盡誅桓氏。二年十月丁丑,太白犯填星,在婁。占同上。三年二月壬辰,太白、熒惑合于羽林。二年十二月,桓玄篡位,放遷帝、后。三年二月,劉裕起義兵,桓玄逼帝東下。



 義熙二年十二月丁未,熒惑、太白皆入羽林,又合于壁。三年正月,慕容超寇淮北、徐州,至下邳。八月,遣劉敬宣伐蜀。三年二月癸亥,熒惑、填星、太白、辰星聚于奎、婁、從填星也,徐州分。是時,慕容超僭號於齊,兵連徐兗,連歲寇抄,
 至於淮泗,姚興、譙縱僭號秦蜀,盧循及魏南北交侵。其五年,劉裕北殄慕容超。其六月辛卯,熒惑犯辰星,在翼。占曰:「天下兵起。」八月己卯,太白奄熒惑。占曰:「有大兵。」其四年,姚略遣眾征赫連勃勃,大為所破。五年四月甲戌,熒惑犯辰星,在東井。占曰:「皆為兵。」十二月辛丑,太白犯歲星,在奎。占曰:「大兵起,魯有兵。」是年四月,劉裕討慕容超。六年二月,滅慕容超于魯地。七年七月丁卯,歲星犯填星,在參。占曰:「歲填合,為內亂。」一曰:「益州戰,不勝,亡地。」是時朱齡石伐蜀,後竟滅之。明年,誅謝混、劉毅。
 八年七月甲申,太白犯填星,在東井。占曰:「秦有大兵。」九年二月丙午,熒惑、填星皆犯東井。占曰、:「秦有兵。」三月壬辰,歲星、熒惑、填星、太白聚於東井,從歲星也。東井,秦分。十三年,劉裕定關中,其後遂移晉祚。十四年十月癸巳,熒惑入太微,犯西蕃上將,仍順行至左掖門內,留二十日乃逆行。至恭帝元熙元年三月五日,出西蕃上將西三尺許,又順還入太微。時填星在太微,熒惑繞填星成鉤己,其年四月丙戌,從端門出。占曰:「熒惑填星鉤己天庭,天下更紀。」十二月,安帝母弟瑯邪王踐阼,是曰恭帝。來年,禪於宋。



\end{pinyinscope}