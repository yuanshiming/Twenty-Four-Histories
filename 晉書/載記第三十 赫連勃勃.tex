\article{載記第三十 赫連勃勃}

\begin{pinyinscope}

 赫連勃勃



 赫連勃勃,字屈孑,匈奴右賢王去卑之後,劉元海之族也。曾祖武,劉聰世以宗室封樓煩公,拜安北將軍、監鮮卑諸軍事、丁零中郎將,雄據肆盧川。為代王猗盧所敗,遂出塞表。祖豹子招集種落,復為諸部之雄,石季龍遣使就拜平北將軍、左賢王、丁零單于。父衛辰入居塞內,苻堅以為西單于,督攝河西諸虜,屯于代來城。及堅國
 亂,遂有朔方之地,控弦之士三萬八千。後魏師伐之,辰令其子力俟提距戰,為魏所敗。魏人乘勝濟河,剋代來,執辰殺之。勃勃乃奔於叱干部。叱干他斗伏送勃勃於魏。他斗伏兄子阿利先戍大洛川。聞將送勃勃,馳諫曰:「鳥雀投人,尚宜濟免,況勃勃國破家亡,歸命於我?縱不能容,猶宜任其所奔。今執而送之,深非仁者之舉。」他斗伏懼為魏所責,弗從。阿利潛遣勁勇篡勃勃於路,送於姚興高平公沒奕于,奕於以女妻之。



 勃勃身長八尺五寸,腰帶十圍,性辯慧,美風儀。興見而奇之,深加禮敬,拜驍騎將軍,加奉車都尉,常參軍國大議,寵遇踰於勳舊。
 興弟邕言於興曰:「勃勃天性不仁,難以親近。陛下寵遇太甚,臣竊惑之。」興曰:「勃勃有濟世之才,吾方收其藝用,與之共平天下,有何不可!」乃以勃勃為安遠將軍,封陽川侯,使助沒奕于鎮高平,以三城、朔方雜夷及衛辰部眾三萬配之,使為伐魏偵候。姚邕固諫以為不可。興曰:「卿何以知其性氣?」邕曰:「勃勃奉上慢,御眾殘,貪暴無親,輕為去就,寵之踰分,終為邊害。」興乃止。頃之,以勃勃為持節、安北將軍、五原公,配以三交五部鮮卑及雜虜二萬餘落,鎮朔方。時河西鮮卑杜崘獻馬八千匹于姚興,濟河,至大城,勃勃留之,召其眾三萬餘人偽獵高平川,
 襲殺沒奕于而並其眾,眾至數萬。



 義熙三年,僭稱天王、大單于,赦其境內,建元曰龍昇,署置百官。自以匈奴夏后氏之苗裔也,國稱大夏。以其長兄右地代為丞相、代公,次兄力俟提為大將軍、魏公,叱干阿利為御史大夫、梁公,弟阿利羅引為征南將軍、司隸校尉,若門為尚書令,叱以鞬為征西將軍、尚書左僕射,乙斗為征北將軍、尚書右僕射,自餘以次授任。



 其年,討鮮卑薛乾等三部,破之,降眾萬數千。進討姚興三城已北諸戍,斬其將楊丕、姚石生等。諸將諫固險,不從,又復言於勃勃曰:「陛下將欲經營宇內,南取長安,宜先固根本,使人心有所憑
 系,然後大業可成。高平險固,山川沃饒,可以都也。」勃勃曰:「卿徒知其一,未知其二。吾大業草創,眾旅未多,姚興亦一時之雄,關中未可圖也。且其諸鎮用命,我若專固一城,彼必并力於我,眾非其敵,亡可立待。吾以雲騎風馳,出其不意,救前則擊其後,救後則擊其前,使彼疲於奔命,我則游食自若,不及十年,嶺北、河東盡我有也。待姚興死後,徐取長安。姚泓凡弱小兒,擒之方略,已在吾計中矣。昔軒轅氏亦遷居無常二十餘年,豈獨我乎!」於是侵掠嶺北,嶺北諸城門不晝啟。興歎曰:「吾不用黃兒之言,以至於此!」黃兒,姚邕小字也。



 勃勃初僭號,求婚於
 禿髮傉檀,傉檀弗許。勃勃怒,率騎二萬伐之,自楊非至於支陽三百餘里,殺傷萬餘人,驅掠二萬七千口、牛馬羊數十萬而還。人辱檀率眾追之,其將焦朗謂傉檀曰:「勃勃天姿雄驁,御軍齊肅,未可輕也。今因抄掠之資,率思歸之士,人自為戰,難與爭鋒。不如從溫圍北渡,趣萬斛堆,阻水結營,制其咽喉,百戰百勝之術也。」傉檀將賀連怒曰:「勃勃以死亡之餘,率烏合之眾,犯順結禍,幸有大功。今牛羊塞路,財寶若山,窘弊之餘,人懷貪競,不能督厲士眾以抗我也。我以大軍臨之,必土崩魚潰。今引軍避之,示敵以弱。我眾氣銳,宜在速追。」傉檀曰:「吾追計決
 矣,敢諫者斬!」勃勃聞而大喜,乃於陽武下陜鑿凌埋車以塞路。傉檀遣善射者射之,中勃勃左臂。勃勃乃勒眾逆擊,大敗之,追奔八十餘里,殺傷萬計,斬其大將十餘人,以為京觀,號「髑髏臺」,還于嶺北。



 勃勃與姚興將張佛生戰于青石原,又敗之,俘斬五千七百人。興遣將齊難率眾二萬來伐,勃勃退如河曲。難以去勃勃既遠,縱兵掠野,勃勃潛軍覆之,俘獲七千餘人,收其戎馬兵杖。難引軍而退,勃勃復追擊于木城,拔之,擒難,俘其將士萬有三千,戎馬萬匹。嶺北夷夏降附者數萬計,勃勃於是拜置守宰以撫之。勃勃又率騎二萬入高岡,及于五井,
 掠平涼雜胡七千餘戶以配後軍,進屯依力川。



 姚興來伐,至三城,勃勃候興諸軍未集,率騎擊之。興大懼,遣其將姚文宗距戰,勃勃偽退,設伏以待之。興遣其將姚榆生等追之,伏兵夾擊,皆擒之。興將王奚聚羌胡三千餘戶于敕奇堡,勃勃進攻之。奚驍悍有膂力,短兵接戰,勃勃之眾多為所傷。於是堰斷其水,堡人窘迫,執奚出降。勃勃謂奚曰:「卿忠臣也!朕方與卿共平天下。」奚曰:「若蒙大恩,速死為惠。」乃與所親數十人自刎而死。勃勃又攻興將金洛生于黃石固,彌姐豪地于我羅城,皆拔之,徙七千餘家於大城,以其丞相右地代領幽州牧以鎮之。



 遣其尚書金纂率騎一萬攻平涼,姚興來救,纂為興所敗,死之。勃勃兄子左將軍羅提率步騎一萬攻興將姚廣都于定陽,剋之,坑將士四千餘人,以女弱為軍賞。拜廣都為太常。勃勃又攻興將姚壽都于清水城,壽都奔上邽,徙其人萬六千家于大城。是歲,齊難、姚廣都謀叛,皆誅之。



 姚興將姚詳棄三城,南奔大蘇。勃勃遣其將平東鹿奕于要擊之,執詳,盡俘其眾。詳至,勃勃數而斬之。



 其年,勃勃率騎三萬攻安定,與姚興將楊佛嵩戰于青石北原,敗之,降其眾四萬五千,獲戎馬二萬匹。進攻姚興將黨智隆于東鄉,降之,署智隆光祿勳,徙其三千餘
 戶于貳城。姚興鎮北參軍王買德來奔。勃勃謂買德曰:「朕大禹之後,世居幽、朔。祖宗重暉,常與漢、魏為敵國。中世不競,受制於人。逮朕不肖,不能紹隆先構,國破家亡,流離漂虜。今將應運而興,復大禹之業,卿以為何如?」買德曰:「自皇晉失統,神器南移,群雄嶽峙,人懷問鼎,況陛下奕葉載德,重光朔野,神武超於漢皇,聖略邁於魏祖,而不於天啟之機建成大業乎!今秦政雖衰,籓鎮猶固,深願蓄力待時,詳而後舉。」勃勃善之,拜軍師中郎將。



 乃赦其境內,改元為鳳翔,以叱干阿利領將作大匠,發嶺北夷夏十萬人,於朔方水北、黑水之南營起都城。勃勃
 自言:「朕方統一天下,君臨萬邦,可以統萬為名。」阿利性尤工巧,然殘忍刻暴,乃蒸土築城,錐入一寸,即殺作者而並築之。勃勃以為忠,故委以營繕之任。又造五兵之器,精銳尤甚。既成呈之,工匠必有死者:射甲不入,即斬弓人;如其入也,便斬鎧匠。又造百練剛刀,為龍雀大環,號曰「大夏龍雀」,銘其背曰:「古之利器,吳、楚湛盧。大夏龍雀,名冠神都。可以懷遠,可以柔逋。如風靡草,威服九區。」世甚珍之。復鑄銅為大鼓,飛廉、翁仲、銅駝、龍獸之屬,皆以黃金飾之,列於宮殿之前。凡殺工匠數千,以是器物莫不精麗。



 於是議討乞伏熾磐。王買德諫曰:「明王之行
 師也,軌物以德,不以暴。且熾磐我之與國,新遭大喪,今若伐之,豈所謂乘理而動,上感靈和之義乎!茍恃眾力,因人喪難,匹夫猶恥為之,而況萬乘哉!」勃勃曰:「甚善。微卿,朕安聞此言!」



 其年,下書曰:「朕之皇祖,自北遷幽、朔,姓改姒氏,音殊中國,故從母氏為劉。子而從母之姓,非禮也。古人氏族無常,或以因生為氏,或以王父之名。朕將以義易之。帝王者,繫天為子,是為徽赫實與天連,今改姓曰赫連氏,庶協皇天之意,永享無疆大慶。係天之尊,不可令支庶同之,其非正統,皆以鐵伐為氏,庶朕宗族子孫剛銳如鐵,皆堪伐人。」立其妻梁氏為王后,子璝為
 太子,封子延陽平公,昌太原公,倫酒泉公,定平原公,滿河南公,安中山公。



 又攻姚興將姚逵于杏城,二旬,剋之,執逵及其將姚大用、姚安和、姚利僕、尹敵等,坑戰士二萬人。



 遣其御史中丞烏洛孤盟於沮渠蒙遜曰:「自金晉數終,禍纏九服,趙、魏為長蛇之墟,秦、隴為豺狼之穴,二都神京,鞠為茂草,蠢爾群生,罔知憑賴。上天悔禍,運屬二家,封疆密邇,道會義親,宜敦和好,弘康世難。爰自終古,有國有家,非盟誓無以昭神祇之心,非斷金無以定終始之好。然晉、楚之成,吳、蜀之約,咸口血未乾,而尋背之。今我二家,契殊曩日,言未發而有篤愛之心,音一交
 而懷傾蓋之顧,息風塵之警,同克濟之誠,戮力一心,共濟六合。若天下有事,則雙振義旗;區域既清,則並敦魯、衛。夷險相赴,交易有無,爰及子孫,永崇斯好。」蒙遜遣其將沮渠漢平來盟。



 勃勃聞姚泓將姚嵩與氐王楊盛相持,率騎四萬襲上邽,未至而嵩為盛所殺。勃勃攻上邽,二旬剋之,殺泓秦州剌史姚平都及將士五千人,毀城而去。進攻陰密,又殺興將姚良子及將士萬餘人。以其子昌為使持節、前將軍、雍州刺史,鎮陰密。泓將姚恢棄安定,奔于長安,安定人胡儼、華韜率戶五萬據安定,降於勃勃。以儼為侍中,韜為尚書,留鎮東羊茍兒鎮之,配
 以鮮卑五千。進攻泓將姚諶于雍城,諶奔長安。勃勃進師次郿城,泓遣其將姚紹來距,勃勃退如安定。胡儼等襲殺茍兒,以城降泓。勃勃引歸杏城,笑謂群臣曰:「劉裕伐秦,水陸兼進,且裕有高世之略,姚泓豈能自固!吾驗以天時人事,必當剋之。又其兄弟內叛,安可以距人!裕既剋長安,利在速返,正可留子弟及諸將守關中。待裕發軫,吾取之若拾芥耳,不足復勞吾士馬。」於是秣馬厲兵,休養士卒。尋進據安定,姚泓嶺北鎮戍郡縣悉降,勃勃於是盡有嶺北之地。



 俄而劉裕滅泓,入于長安,遣使遺勃勃書,請通和好,約為兄弟。勃勃命其中書侍郎皇
 甫徽為文而陰誦之,召裕使前,口授舍人為書,封以答裕。裕覽其文而奇之,使者又言勃勃容儀瑰偉,英武絕人。裕歎曰:「吾所不如也!」既而勃勃還統萬,裕留子義真鎮長安而還。勃勃聞之,大悅,謂王買德曰:「朕將進圖長安,卿試言取之方略。」買德曰:「劉裕滅秦,所謂以亂平亂,未有德政以濟蒼生。關中形勝之地,而以弱才小兒守之,非經遠之規也。狼狽而返者,欲速成篡事耳,無暇有意於中原。陛下以順伐逆,義貫幽顯,百姓以君命望陛下義旗之至,以日為歲矣。青泥、上洛,南師之衝要,宜置游兵斷其去來之路。然後杜潼關,塞崤、陜,絕其水陸之
 道。陛下聲檄長安,申布恩澤,三輔父老皆壺漿以迎王師矣。義真獨坐空城,逃竄無所,一旬之間必面縛麾下,所謂兵不血刃,不戰而自定也。」勃勃善之,以子璝都督前鋒諸軍事,領撫軍大將軍,率騎二萬南伐長安,前將軍赫連昌屯兵潼關,以買德為撫軍右長史,南斷青泥,勃勃率大軍繼發。璝至渭陽,降者屬路。義真遣龍驤將軍沈田子率眾逆戰,不利而退,屯劉回堡。田子與義真司馬王鎮惡不平,因鎮惡出城,遂殺之。義真又殺田子。於是悉召外軍入于城中,閉門距守。關中郡縣悉降。璝夜襲長安,不剋。勃勃進據咸陽,長安樵採路絕。劉裕聞
 之,大懼,乃召義真東鎮洛陽,以朱齡石為雍州刺史,守長安。義真大掠而東,至於灞上,百姓遂逐齡石,而迎勃勃入于長安。璝率眾三萬追擊義真,王師敗績,義真單馬而遁。買德獲晉寧朔將軍傅弘之、輔國將軍蒯恩、義真司馬毛脩之於青泥,積人頭以為京觀。於是勃勃大饗將士于長安,舉觴謂王買德曰:「卿往日之言,一周而果效,可謂算無遺策矣。雖宗廟社稷之靈,亦卿謀獻之力也。此觴所集,非卿而誰!」於是拜買德都官尚書,加冠軍將軍,封河陽侯。



 赫連昌攻齡石及龍驤將軍王敬於潼關之曹公故壘,剋之,執齡石及敬送于長安。群臣乃
 勸進,勃勃曰:「朕無撥亂之才,不能弘濟兆庶,自枕戈寢甲,十有二年,而四海未同,遺寇尚熾,不知何以謝責當年,垂之來葉!將明揚仄陋,以王位讓之,然後歸老朔方,琴書卒歲。皇帝之號,豈薄德所膺!」群臣固請,乃許之。於是為壇於灞上,僭即皇帝位,赦其境內,改元為昌武。遣其將叱奴侯提率步騎二萬攻晉并州刺史毛德祖于蒲阪,德祖奔于洛陽。以侯提為并州刺史,鎮蒲阪。



 勃勃歸于長安,徵隱士京兆韋祖思。既至而恭懼過禮,勃勃怒曰:「吾以國士徵汝,柰何以非類處吾!汝昔不拜姚興,何獨拜我?我今未死,汝猶不以我為帝王,吾死之後,汝
 輩弄筆,當置吾何地!」遂殺之。



 群臣勸都長安,勃勃曰:「朕豈不知長安累帝舊都,有山河四塞之固!但荊、吳僻遠,勢不能為人之患。東魏與我同壤境,去北京裁數百餘里,若都長安,北京恐有不守之憂。朕在統萬,彼終不敢濟河,諸卿適未見此耳!」其下咸曰:「非所及也。」乃於長安置南臺,以璝領大將軍、雍州牧、錄南臺尚書事。



 勃勃還統萬,以宮殿大成,於是赦其境內,又改元曰真興。刻石都南,頌其功德,曰:



 夫庸大德盛者,必建不刊之業;道積慶隆者,必享無窮之祚。昔在陶唐,數鐘厄運,我皇祖大禹以至聖之姿,當經綸之會,鑿龍門面闢伊闕,疏三江
 而決九河,夷一元之窮災,拯六合之沈溺,鴻績侔於天地,神功邁於造化,故二儀降祉,三靈葉贊,揖讓受終,光啟有夏。傳世二十,歷載四百,賢辟相承,哲王繼軌,徽猷冠於玄古,高範煥乎疇昔。而道無常夷,數或屯險,王桀不綱,網漏殷氏,用使金暉絕于中天,神轡輟于促路。然純曜未渝,慶綿萬祀,龍飛漠南,鳳峙朔北。長轡遠馭,則西罩昆山之外;密網遐張,則東亙滄海之表。爰始逮今,二千餘載,雖三統迭制於崤、函,五德革運於伊、洛,秦、雍成篡殺之墟,周、豫為爭奪之藪,而幽朔謐爾,主有常尊於上;海代晏然,物無異望於下。故能控弦之眾百有餘
 萬,躍馬長驅,鼓行秦、趙,使中原疲於奔命,諸夏不得高枕,為日久矣。是以偏師暫擬,涇陽摧隆周之鋒;赫斯一奮,平陽挫漢祖之銳。雖霸王繼蹤,猶朝日之升扶桑;英豪接踵,若夕月之登濛汜。自開闢已來,未始聞也。非夫卜世與乾坤比長,鴻基與山嶽齊固,孰能本枝於千葉,重光於萬祀,履寒霜而踰榮,蒙重氛而彌耀者哉!



 於是玄符告徵,大猷有會,我皇誕命世之期,應天縱之運,仰協時來,俯順時望。龍升北京,則義風蓋於九區;鳳翔天域,則威聲格于八表。屬姦雄鼎峙之秋,群凶嶽立之際,昧旦臨朝,日旰忘膳,運籌命將,舉無遺策。親御六戎,則
 有征無戰。故偽秦以三世之資,喪魂於關、隴;河源望旗而委質,北虜欽風而納款。德音著於柔服,威刑彰于伐叛,文教與武功並宣,俎豆與干戈俱運。五稔之間,道風弘著,暨乎七載而王猷允洽。乃遠惟周文,啟經始之基;近詳山川,究形勝之地,遂營起都城,開建京邑。背名山而面洪流,左河津而右重塞。高隅隱日,崇墉際雲,石郭天池,周綿千里。其為獨守之形,險絕之狀,固以遠邁於咸陽,超美於周洛,若乃廣五郊之義,尊七廟之制,崇左社之規,建右稷之禮,御太一以繕明堂,模帝坐而營路寢,閶闔披霄而山亭,象魏排虛而嶽峙,華林靈沼,崇臺
 秘室,通房連閣,馳道苑園,可以陰映萬邦,光覆四海,莫不鬱然並建,森然畢備,若紫微之帶皇穹,閬風之跨后土。然宰司鼎臣,群黎士庶,僉以為重威之式,有闕前王。於是延王爾之奇工,命班輸之妙匠,搜文梓於鄧林,採繡石於恒嶽,九域貢以金銀,八方獻其瑰寶,親運神奇,參制規矩,營離宮於露寢之南,起別殿於永安之北。高構千尋,崇基萬仞。玄棟鏤榥,若騰虹之揚眉;飛簷舒咢,似翔鵬之矯翼。二序啟矣,而五時之坐開;四隅陳設,而一御之位建。溫宮膠葛,涼殿崢嶸,絡以隋珠,綷以金鏡,雖曦望互升於表,而中無晝夜之殊;陰陽迭更於外,而
 內無寒暑之別。故善目者不能為其名,博辯者不能究其稱,斯蓋神明之所規模,非人工之所經制。若乃尋名以求類,蹤狀以效真,據質以究名,形疑妙出,雖如來、須彌之寶塔,帝釋、忉利之神宮,尚未足以喻其麗,方其飾矣。



 昔周宣考室而詠於詩人,閟宮有侐而頌聲是作。況乃太微肇制,清都啟建,軌一文昌,舊章唯始,咸秩百神,賓享萬國,群生開其耳目,天下詠其來蘇,亦何得不播之管弦,刊之金石哉!乃樹銘都邑,敷贊碩美,俾皇風振於來葉,聖庸垂乎不朽。其辭曰:



 於赫靈祚,配乾比隆。巍巍大禹,堂堂聖功。仁被蒼生,德格玄穹。帝錫玄珪,揖讓
 受終。哲王繼軌,光闡徽風。道無常夷,數或不競。金精南邁,天輝北映。靈祉踰昌,世葉彌盛。惟祖惟父,克廣休命。如彼日月,連光接鏡。玄符瑞德,乾運有歸。誕鐘我后,應圖龍飛。落落神武,恢恢聖姿。名教內敷,群妖外夷。化光四表,威截九圍。封畿之制,王者常經。乃延輸、爾,肇建帝京。土苞上壤,地跨勝形。庶人子來,不日而成。崇臺霄峙,秀闕雲亭。千榭連隅,萬閣接屏。晃若晨曦,昭若列星。離宮既作,別宇云施。爰構崇明,仰準乾儀。懸薨風閱,飛軒雲垂。溫室嵯峨,層城參差。楹雕虯獸,節鏤龍螭。瑩以寶璞,飾以珍奇。稱因褒著,名由實揚。偉哉皇室,盛矣厥章!
 義高靈臺,美隆未央。邁軌三五,貽則霸王。永世垂節,億載彌光。



 其祕書監胡義周之辭也。名其南門曰朝宋門,東門曰招魏門,西門曰服涼門,北門曰平朔門。追尊其高祖訓兒曰元皇帝,曾祖武曰景皇帝,祖豹子曰宣皇帝,父衛辰曰桓皇帝,廟號太祖,母苻氏曰桓文皇后。



 勃勃性凶暴好殺,無順守之規。常居城上,置弓劍於側,有所嫌忿,便手自殺之,群臣忤視者毀其目,笑者決其脣,諫者謂之誹謗,先截其舌而後斬之。夷夏囂然,人無生賴。在位十三年而宋受禪,以宋元嘉二年死。子昌嗣偽位,尋為魏所擒。弟定僭號於平涼,遂為魏所滅。自勃勃
 至定,凡二十有六載而亡。



 史臣曰:赫連勃勃犬熏丑種類,入居邊宇,屬中壤分崩,緣間肆慝,控弦鳴鏑,據有朔方。遂乃法玄象以開宮,擬神京而建社,竊先王之徽號,備中國之禮容,驅駕英賢,窺窬天下。然其器識高爽,風骨魁奇,姚興睹之而醉心,宋祖聞之而動色。豈陰山之韞異氣,不然何以致斯乎!雖雄略過人,而兇殘未革,飾非距諫,酷害朝臣,部內囂然,忠良卷舌。滅亡之禍,宜在厥身,猶及其嗣,非不幸也。



 贊曰:淳維遠裔,名王之餘。嘯群龍漠,乘釁侵漁。爰創宮宇,易彼氈廬。雖弄神器,猶曰兇渠。



\end{pinyinscope}