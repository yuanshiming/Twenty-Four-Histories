\article{載記第二十二 呂光呂纂呂隆}

\begin{pinyinscope}

 呂光呂纂呂隆



 呂光,字世明,略陽氐人也。其先呂文和,漢文帝初,自沛避難徙焉。世為酋豪。父婆樓,佐命苻堅,官至太尉。光生於枋頭,夜有神光之異,故以光為名。年十歲,與諸童兒游戲邑里,為戰陣之法,儔類咸推為主。部分詳平,群童歎服。不樂讀書,唯好鷹馬。及長,身長八尺四寸,目重瞳子,左肘有肉印。沈毅凝重,寬簡有大量,喜怒不形于色。
 時人莫之識也,惟王猛異之,曰:「此非常人。」言之苻堅,舉賢良,除美陽令,夷夏愛服。遷鷹揚將軍。從堅征張平,戰於銅壁,刺平養子蠔,中之,自是威名大著。



 苻雙反于秦州,堅將楊成世為雙將茍興所敗,光與王鑒討之。鑒欲速戰,光曰:「興初破成世,姦氣漸張,宜持重以待其弊。興乘勝輕來,糧竭必退,退而擊之,可以破也。」二旬而興退,諸將不知所為,光曰:「揆其姦計,必攻榆眉。若得榆眉,據城斷路,資儲復贍,非國之利也,宜速進師。若興攻城,尤須赴救。如其奔也,彼糧既盡,可以滅之。」鑒從焉。果敗興軍。從王猛滅慕容,封都亭侯。



 苻重之鎮洛陽,以光為
 長史。及重謀反,苻堅聞之,曰:「呂光忠孝方正,必不同也。」馳使命光檻重送之。尋入為太子右率,甚見敬重。



 蜀人李焉聚眾二萬,攻逼益州。堅以光為破虜將軍,率兵討滅之,遷步兵校尉。苻洛反,光又擊平之,拜驍騎將軍。



 堅既平山東,士馬強盛,遂有圖西域之志,乃授光使持節、都督西討諸軍事,率將軍姜飛、彭晃、杜進、康盛等總兵七萬,鐵騎五千,以討西域,以隴西董方、馮翊郭抱、武威賈虔、弘農楊穎為四府佐將。堅太子宏執光手曰:「君器相非常,必有大福,宜深保愛。」行至高昌,聞堅寇晉,光欲更須後命。部將杜進曰:「節下受任金方,赴機宜速,有何
 不了,而更留乎!」光乃進及流沙,三百餘里無水,將士失色。光曰:「吾聞李廣利精誠玄感,飛泉湧出,吾等豈獨無感致乎!皇天必將有濟,諸君不足憂也。」俄而大雨,平地三尺。進兵至焉耆,其王泥流率其旁國請降。龜茲王帛純距光,光軍其城南,五里為一營,深溝高壘,廣設疑兵,以木為人,被之以甲,羅之壘上。帛純驅徙城外人入於城中,附庸侯王各嬰城自守。



 至是,光左臂內脈起成字,文曰「巨霸」。營外夜有一黑物,大如斷堤,搖動有頭角,目光若電,及明而雲霧四周,遂不復見。旦視其處,南北五里,東西三十餘步,鱗甲隱地之所,昭然猶在。光笑曰:「黑
 龍也。」俄而雲起西北,暴雨滅其跡。杜進言於光曰:「龍者神獸,人君利見之象。《易》曰:『見龍在田,德施普也。』斯誠明將軍道合靈和,德符幽顯。願將軍勉之,以成大慶。」光有喜色。



 又進攻龜茲城,夜夢金象飛越城外。光曰:「此謂佛神去之,胡必亡矣。」光攻城既急,帛純乃傾國財寶請救獪胡。獪胡弟吶龍、侯將馗率騎二十餘萬,並引溫宿、尉頭等國王,合七十餘萬以救之。胡便弓馬,善矛槊,鎧如連鎖,射不可入,以革索為羂,策馬擲人,多有中者。眾甚憚之。諸將咸欲每營結陣,案兵以距之。光曰:「彼眾我寡,營又相遠,勢分力散,非良策也。」於是遷營相接陣,為勾
 鎖之法,精騎為游軍,彌縫其闕。戰于城西,大敗之,斬萬餘級。,帛純收其珍寶而走,王侯降者三十餘國。光入其城,大饗將士,賦詩言志。見其宮室壯麗,命參軍京兆段業著《龜茲宮賦》以譏之。胡人奢侈,厚於養生,家有蒲桃酒,或至千斛,經十年不敗,士卒淪沒酒藏者相繼矣。諸國憚光威名,貢款屬路,乃立帛純弟震為王以安之。光撫寧西域,威恩甚著,桀黠胡王昔所未賓者,不遠萬里皆來歸附,上漢所賜節傳,光皆表而易之。



 堅聞光平西域,以為使持節、散騎常侍、都督玉門已西諸軍事,安西將軍、西域校尉,道絕不通。光既平龜茲,有留焉之志。時
 始獲鳩摩羅什,羅什勸之東還,語在《西夷傳》。光於是大饗文武,博議進止。眾咸請還,光從之,以駝二萬餘頭致外國珍寶及奇伎異戲、殊禽怪獸千有餘品,駿馬萬餘匹。而苻堅高昌太守楊翰說其涼州刺史梁熙距守高桐、伊吾二關,熙不從。光至高昌,翰以郡迎降。初,光聞翰之說,惡之,又聞苻堅喪敗,長安危逼,謀欲停師。杜進諫曰:「梁熙文雅有餘,機鑒不足,終不能納善從說也,願不足憂之。聞其上下未同,宜在速進,進而不捷,請受過言之誅。」光從之。及至玉門,梁熙傳檄責光擅命還師,遣子胤與振威姚皓、別駕衛翰率眾五萬,距光于灑泉。光報
 檄涼州,責熙無赴難之誠,數其遏歸師之罪。遣彭晃、杜進、姜飛等為前鋒,擊胤,大敗之。胤輕將麾下數百騎東奔,杜進追擒之。於是四山胡夷皆來款附。武威太守彭濟執熙請降。光入姑臧,自領涼州刺史、護羌校尉,表杜進為輔國將軍、武威太守,封武始侯,自餘封拜各有差。



 光主簿尉祐,姦佞傾薄人也,見棄前朝,與彭齊同謀執梁熙,光深見寵任,乃譖誅南安姚皓、天水尹景等名士十餘人,遠近頗以此離貳。光尋擢祐為寧遠將軍、金城太守。祐次允吾,襲據外城以叛,祐從弟隨據鸇陰以應之。光遣其將魏真討隨,隨敗,奔祐,光將姜飛又擊敗祐
 眾。祐奔據興城,扇動百姓,夷夏多從之。飛司馬張象、參軍郭雅謀殺飛應祐,發覺,逃奔。



 初,苻堅之敗,張天錫南奔,其世子大豫為長水校尉王穆所匿。及堅還長安,穆將大豫奔禿髮思復犍,思復犍送之魏安。是月,魏安人焦松、齊肅、張濟等起兵數千,迎大豫於揟次,陷昌松郡。光遣其將杜進討之,為大豫聽敗。大豫遂進逼姑臧,求決勝負,王穆諫曰:「呂光糧豐城固,甲兵精銳,逼之非利。不如席卷嶺西,厲兵積粟,東向而爭,不及期年,可以平也。」大豫不從,乃遣穆求救於嶺西諸郡,建康太守李隰、祁連都尉嚴純及閻襲起兵應之。大豫進屯城西,王穆
 率眾三萬及思復犍子奚於等陣于城南。光出擊,破之,斬奚於等二萬餘級。光謂諸將曰:「大豫若用王穆之言,恐未可平也。」諸將曰:「大豫豈不及此邪!皇天欲贊成明公八百之業,故令大豫迷於良算耳。」光大悅,賜金帛有差。大豫自西郡詣臨洮,驅略百姓五千餘戶,保據俱城。光將彭晃、徐炅攻破之,大豫奔廣武,穆奔建康。廣武人執大豫,送之,斬于姑臧市。



 光至是始聞苻堅為姚萇所害,奮怒哀號,三軍縞素,大臨于城南,偽謚堅曰文昭皇帝,長吏百石已上服斬縗三月,庶人哭泣三日。光於是大赦境內,建元曰太安,自稱使持節、侍中、中外大都督、
 督隴右河西諸軍事、大將軍、鄰護匈奴中郎將、涼州牧、酒泉公。王穆襲據酒泉,自稱大將軍、涼州牧。時穀價踴貴,斗直五百,人相食,死者太半。光西平太守康寧自稱匈奴王,阻兵以叛,光屢遣討之,不捷。



 初,光之定河西也,杜進有力焉,以為輔國將軍、武威太守。既居都尹,權高一時,出入羽儀,與光相亞。光甥石聰至自關中,光曰:「中州人言吾政化何如?」聰曰:「止知有杜進耳,實不聞有舅。」光默然,因此誅進。光後宴群僚,酒酣,語及政事。時刑法峻重,參軍段業進曰:「嚴刑重憲,非明王之義也。」光曰:「商鞅之法至峻,而兼諸侯;吳起之術無親,而荊蠻以霸,何
 也?」業曰:「明公受天眷命,方君臨四海,景行堯、舜,猶懼有弊,奈何欲以商、申之末法臨道義之神州,豈此州士女所望於明公哉!」光改容謝之,於是下令責躬,及崇寬簡之政。



 其將徐炅與張掖太守彭晃謀叛,光遣師討炅,炅奔晃。晃東結康寧,四通王穆,光議將討之,諸將咸曰:「今康寧在南,阻兵伺隙,若大駕西行,寧必乘虛出于嶺左。晃、穆未平,康寧復至,進退狼狽,勢必大危。」光曰:「事勢實如卿言。今而不往,尋坐待其來。晃、穆共相脣齒,寧又同惡相救,東西交至,城外非吾之有,若是,大事去矣。今晃叛逆始爾,寧、穆與之情契未密,及其倉卒,取之為易。且
 隆替命也,卿勿復言。」光於是自率步騎三萬,倍道兼行。既至,攻之二旬,晃將寇顗斬關納光,於是誅彭晃。王穆以其黨索嘏為敦煌太守,既而忌其威名,率眾攻嘏。光聞之,謂諸將曰:「二虜相攻,此成擒也。」光將攻之,眾咸以為不可。光曰:「取亂侮亡,武之善經,不可以累徵之勞而失永逸之舉。」率步騎二萬攻酒泉,剋之,進次涼興。穆引師東還,路中眾散,穆單騎奔騂馬,騂馬令郭文斬首送之。



 是時麟見金澤縣,百獸從之,光以為已瑞,以孝武太元十四年僭即三河王位,置百官自丞郎已下,赦其境內,年號麟嘉。光妻石氏、子紹、弟德世至自仇池,光迎于
 城東,大饗群臣。遣其子左將軍他、武賁中郎將纂討北虜匹勤于三巖山,大破之。立妻石氏為王妃,子紹為世子。宴其群臣于內苑新堂。太廟新成,追尊其高祖為敬公,曾祖為恭公,祖為宣公,父為景昭王,母曰昭烈妃。其中書侍郎楊穎上疏,請依三代故事,追尊呂望為始祖,永為不遷之廟,光從之。



 是歲,張掖督郵傅曜考核屬縣,而丘池令尹興殺之,投諸空井,曜見夢於光曰:「臣張掖郡小吏,案校諸縣,而丘池令尹興贓狀狼藉,懼臣言之,殺臣投於南亭空井中。臣衣服形狀如是。」光寤而猶見,久之乃滅。遣使覆之如夢,光怒,殺興。著作郎段業以光
 未能揚清激濁,使賢愚殊貫,因療疾于天梯山,作表志詩《九歎》、《七諷》十六篇以諷焉。光覽而悅之。



 南羌彭奚念入攻白土,都尉孫峙退奔興城。光遣其南中郎將呂方及其弟右將軍呂寶、振威楊範、強弩竇茍討乞伏乾歸于金城。方屯河北,寶進師濟河,為乾歸所敗,寶死之。武賁呂篡、強弩竇茍率步騎五千南討彭奚念,戰于盤夷,大敗而歸。光親討乾歸、奚念,遣纂及揚武楊軌、建忠沮渠羅仇、建武梁恭軍于左南。奚念大懼,於白土津累石為堤,以水自固,遣精兵一萬距守河津。光遣將軍王寶潛趣上津,夜渡湟河。光濟自石堤,攻剋枹罕,奚念單騎
 奔甘松,光振旅而旋。



 初,光徙西海郡人於諸郡,至是,謠曰:「朔馬心何悲?念舊中心勞。燕雀何徘徊?意欲還故巢。」頃之,遂相扇動,復徙之於西河樂都。



 群議以高昌雖在西垂,地居形勝,外接胡虜,易生翻覆,宜遣子弟鎮之。光以子覆為使持節、鎮西將軍、都督玉門已西諸軍事、西域大都護,鎮高昌,命大臣子弟隨之。



 光於是以太元二十一年僭即天王位,大赦境內,改年龍飛。立世子紹為太子,諸子弟為公侯者二十人。中書令王詳為尚書左僕射,段業等五人為尚書。



 乾歸從弟軻彈來奔,光下書曰:「乾歸狼子野心,前後反覆。朕方東清秦、趙,勒銘會稽,
 豈令豎子鴟峙洮南!且其兄弟內相離間,可乘之機,勿過今也。其敕中外戒嚴,朕當親討。」光於是次于長最,使呂纂率楊軌、竇茍等步騎三萬攻金城。乾歸率眾二萬救之。光遣其將王寶、徐炅率騎五千邀之,乾歸懼而不進。光又遣其將梁恭、金石生以甲卒萬餘出陽武下峽,與秦州刺史沒奕於攻其東,光弟天水公延以枹罕之眾攻臨洮、武始、河關,皆剋之。呂纂剋金城,擒乾歸金城太守衛犍,犍真目謂光曰:「我寧守節斷頭,不為降虜也。」光義而免之。乾歸因大震,泣嘆曰:「死中求生,正在今日也。」乃縱反間,稱乾歸眾潰,東奔成紀。呂延信之,引師輕
 進。延司馬耿稚諫曰:「乾歸雄勇過人,權略難測,破王廣,剋楊定,皆羸師以誘之,雖蕞爾小國,亦不可輕也。困獸猶鬥,況乾歸而可望風自散乎!且告者視高而色動,必為姦計。而今宜部陣而前,步騎相接,徐待諸軍大集,可一舉滅之。」延不從,與乾歸相遇,戰敗,死之。耿稚及將軍姜顯收集散卒,屯于枹罕。光還于姑臧。



 光荒耄信讒,殺尚書沮渠羅仇、三河太守沮渠麴粥。羅仇弟子蒙遜叛光,殺中田護軍馬邃,攻陷臨松郡,屯兵金山,大為百姓之患。蒙遜從兄男成先為將軍,守晉昌,聞蒙遜起兵,逃奔貲虜,扇動諸夷,眾至數千,進攻福祿、建安。寧戎護軍
 趙策擊敗之,男成退屯樂涫。呂纂敗蒙遜于忽谷。酒泉太守壘澄率將軍趙策、趙陵步騎萬餘討男成于樂涫,戰敗,澄、策死之。男成進攻建康,說太守段業曰:「呂氏政衰,權臣擅命,刑罰失中,人不堪役,一州之地,叛者連城,瓦解之勢,昭然在目,百姓嗷然,無所宗附。府君豈可以蓋世之才,而立忠於垂亡之世!男成等既唱大義,欲屈府君撫臨鄙州,使塗炭之餘蒙來蘇之惠。」業不從。相持二旬而外救不至,郡人高逵、史惠等言於業曰:「今孤城獨立,臺無救援,府君雖心過田單,而地非即墨,宜思高算,轉禍為福。」業先與光侍中房晷、僕射王詳不平,慮不
 自容,乃許之。男成等推業為大都督、龍驤大將軍、涼州牧、建康公。光命呂纂討業,沮渠蒙遜進屯臨洮,為業聲勢。戰于合離,纂師大敗。



 光散騎常侍、太常郭黁明天文,善占候,謂王詳曰:「於天文,涼之分野將有大兵。主上老病,太子沖暗,纂等凶武,一旦不諱,必有難作。以吾二人久居內要,常有不善之言,恐禍及人,深宜慮之。田胡王氣乞機部眾最彊,二苑之人多其故眾。吾今與公唱義,推機為主,則二苑之眾盡我有也。剋城之後,徐更圖之。」詳以為然。夜燒光洪範門,二苑之眾皆附之,詳為內應。事發,光誅之。黁遂據東苑以叛。光馳使召纂,諸將勸纂
 曰:「業聞師迴,必躡軍後。若潛師夜還,庶無後患矣。」纂曰:「業雖憑城阻眾,無雄略之才,若夜潛還,張其姦志。」乃遣使告業曰:「郭黁作亂,吾今還都。卿能決者,可出戰。」於是引還。業不敢出。纂司馬楊統謂其從兄恆曰:「郭黁明善天文,起兵其當有以。京城之外非復朝廷之有,纂今還都,復何所補!統請除纂,勒兵推兄為盟主,西襲呂弘,據張掖以號令諸郡,亦千載一時也。」桓怒曰:「吾聞臣子之事君親,有隕無二,吾未有包胥存救之效,豈可安榮其祿,亂增其難乎!呂宗若敗,吾為弘演矣。」統懼,至番禾,遂奔郭黁。黁遣軍邀纂于白石,纂大敗。光西安太守石元
 良率步騎五千赴難,與纂共擊黁軍,破之,遂入于姑臧。黁之叛也,得光孫八人于東苑。及軍敗,恚甚,悉投之于鋒刃之上,枝分節解,飲血盟眾,眾皆掩目,不忍視之,黁悠然自若。



 黁推後將軍楊軌為盟主,軌自稱大將軍、涼州牧、西平公。呂纂擊黁將王斐于城西,大破之,自是黁勢漸衰。光遺楊軌書曰:「自羌胡不靖,郭黁叛逆,南籓安否,音問兩絕。行人風傳,云卿擁逼百姓,為黁脣齒。卿雅志忠貞,有史魚之操,鑒察成敗,遠侔古人,豈宜聽納姦邪,以虧大美!陵霜不凋者松柏也,臨難不移者君子也,何圖松柏凋於微霜,雞鳴已於風雨!郭黁巫卜小數,
 時或誤中,考之大理,率多虛謬。朕宰化寡方,澤不逮遠,致世事紛紜,百城離叛。戮力一心,同濟巨海者,望之於卿也。今中倉積粟數百千萬,東人戰士一當百餘,入則言笑晏晏,出則武步涼州,吞黁咀業,綽有餘暇。但與卿形雖君臣,心過父子,欲全卿名節,不使貽笑將來。」軌不答,率步騎二萬北赴郭黁。至姑臧,壘于城北。軌以士馬之盛,議欲大決成敗,黁每以天文裁之。呂弘為段業所逼,光遣呂纂迎之。軌謀於眾曰:「呂弘精兵一萬,若與光合,則敵彊我弱。養獸不討,將為後患。」遂率兵邀纂,纂擊敗之。郭黁聞軌敗,東走魏安,遂奔于乞伏乾歸。楊軌聞
 黁走,南奔廉川。



 光疾甚,立其太子紹為天王,自號太上皇帝。以呂纂為太尉,呂弘為司徒。謂紹曰:「吾疾病唯增,恐將不濟。三寇窺窬,迭伺國隙。吾終以後,使纂統六軍,弘管朝政,汝恭己無為,委重二兄,庶可以濟。若內相猜貳,釁起蕭墻,則晉、趙之變旦夕至矣。」又謂纂、弘曰:「永業才非撥亂,直以正嫡有常,猥居元首。今外有彊寇,人心未寧,汝兄弟緝穆,則貽厥萬世。若內自相圖,則禍不旋踵。」纂、弘泣曰:「不敢有二心。」光以安帝隆安三年死,時年六十三,在位十年。偽謚懿武皇帝,廟號太祖,墓號高陵。



 纂字永緒,光之庶長子也。少便弓馬,好鷹犬。苻堅時入
 太學,不好讀書,唯以交結公侯聲樂為務。及堅亂,西奔上邽,轉至姑臧,拜武賁中郎將,封太原公。



 光死,呂紹祕不發喪,纂排閣入哭,盡哀而出。紹懼為纂所害,以位讓之,曰:「兄功高年長,宜承大統,願兄勿疑。」纂曰:「臣雖年長,陛下國家之塚嫡,不可以私愛而亂大倫。」紹固以讓纂,纂不許之。及紹嗣偽位,呂超言於紹曰:「纂統戎積年,威震內外,臨喪不哀,步高視遠,觀其舉止亂常,恐成大變,宜早除之,以安社稷。」紹曰:「先帝顧命,音猶在耳,兄弟至親,豈有此乎!吾弱年而荷大任,方賴二兄以寧家國。縱其圖我,我視死如歸,終不忍有此意也,卿懼勿過言。」超
 曰:「纂威名素盛,安忍無親,今不圖之,後必噬臍矣。」紹曰:「吾每念袁尚兄弟,未曾不痛心忘寢食,寧坐而死,豈忍行之。」超曰:「聖人稱知機其神,陛下臨機不斷,臣見大事去矣。」既而纂見紹於湛露堂,超執刀侍紹,目纂請收之,紹弗許。



 初,光欲立弘為世子,會聞紹在仇池,乃止,弘由是有憾於紹。遣尚書姜紀密告纂曰:「先帝登遐,主上闇弱,兄總攝內外,威恩被于遐邇,輒欲遠追廢昌邑之義,以兄為中宗何如?」纂於是夜率壯士數百,踰北城,攻廣夏門,弘率東苑之眾斫洪範門。左衛齊從守融明觀,逆問之曰:「誰也?」眾曰:「太原公。」從曰:「國有大故,主上新立,太
 原公行不由道,夜入禁城,將為亂邪?」因抽劍直前,斫纂中額。纂左右擒之,纂曰:「義士也,勿殺。」紹遣武賁中郎將呂開率其禁兵距戰於端門,驍騎呂超率卒二千赴之。眾素憚纂,悉皆潰散。



 纂入自青角門,升于謙光殿。紹登紫閣自殺,呂超出奔廣武。纂憚弘兵強,勸弘即位。弘曰:「自以紹弟也而承大統,眾心不順,是以違先帝遺敕,慚負黃泉。今復越兄而立,何面目以視息世間!大兄長且賢,威名振于二賊,宜速即大位,以安國家。」纂以隆安四年遂僭即天王位,大赦境內,改元為咸寧,謚紹為隱王。以弘為使持節、侍中、大都督、都督中外諸軍事、大司馬、
 車騎大將軍、司隸校尉、錄尚書事,改封番禾郡公,其餘封拜各有差。



 纂謂齊從曰:「卿前斫我,一何甚也!」從泣曰:「隱王先帝所立,陛下雖應天順時,而微心未達,惟恐陛下不死,何謂甚也。」纂嘉其忠,善遇之。纂遣使謂征東呂方曰:「超實忠臣,義勇可嘉,但不識經國大體,權變之宜。方賴其忠節,誕濟世難,可以此意諭之。」超上疏陳謝,纂復其爵位。



 呂弘自以功名崇重,恐不為纂所容,纂亦深忌之。弘遂起兵東苑,劫尹文、楊桓以為謀主,請宗燮俱行。燮曰:「老臣受先帝大恩,位為列棘,不能隕身授命,死有餘罪,而復從殿下,親為戎首者,豈天地所容乎!且智
 不能謀,眾不足恃,將焉用之!」弘曰:「君為義士,我為亂臣!」乃率兵攻纂。纂遣其將焦辨擊弘,弘眾潰,出奔廣武。纂縱兵大掠,以東苑婦女賞軍,弘之妻子亦為士卒所辱。纂笑謂群臣曰:「今日之戰何如?」其侍中房晷對曰:「天禍涼室,釁起戚籓。先帝始崩,隱王幽逼,山陵甫訖,大司馬驚疑肆逆,京邑交兵,友于接刃。雖弘自取夷滅,亦由陛下無棠棣之義。宜考已責躬,以謝百姓,而反縱兵大掠,幽辱士女。釁自由弘,百姓何罪!且弘妻,陛下之弟婦也;弘女,陛下之姪女也。奈何使無賴小人辱為婢妾。天地神明,豈忍見此!」遂歔欷悲泣。纂改容謝之,召弘妻及男
 女于東宮,厚撫之。呂方執弘繫獄,馳使告纂,纂遣力士康龍拉殺之。是月,立其妻楊氏為皇后,以楊氏父桓為散騎常侍、尚書左僕射、涼都尹,封金城侯。



 纂將伐禿髮利鹿孤,中書令楊穎諫曰:「夫起師動眾,必參之天人,茍非其時,聖賢所不為。禿髮利鹿孤上下用命,國未有釁,不可以伐。宜繕甲養銳,勸課農殖,待可乘之機,然後一舉蕩滅。比年多事,公私罄竭,不深根固本,恐為患將來,願抑赫斯之怒,思萬全之算。」纂不從。度浩亹河,為鹿弧弟傉檀所敗,遂西襲張掖。姜紀諫曰:「方今盛夏,百姓廢農,所利既少,所喪者多,若師至嶺西,虜必乘虛寇抄都
 下,宜且回師以為後圖。」纂曰:「虜無大志,聞朕西征,正可自固耳。今速襲之,可以得志。」遂圍張掖,略地建康。聞傉檀寇姑臧,乃還。



 即序胡安據盜發張駿墓,見駿貌如生,得真珠簏、琉璃榼、白玉樽、赤玉簫、紫玉笛、珊瑚鞭、馬腦鐘,水陸奇珍不可勝紀。纂誅安據黨五十餘家,遣使弔祭駿,並繕脩其墓。



 道士句摩羅耆婆言於纂曰:「潛龍屢出,豕犬見妖,將有下人謀上之禍,宜增脩德政,以答天戒。」纂納之。耆婆,即羅什之別名也。



 纂游田無度,荒耽酒色,其太常楊穎諫曰:「臣聞皇天降鑒,惟德是與。德由人弘,天應以福,故勃焉之美奄在聖躬。大業已爾,宜以道
 守之。廓靈基於日新,邀洪福於萬祀。自陛下龍飛,疆宇未闢,崎嶇二嶺之內,綱維未振於九州。當兢兢夕惕,經略四方,成先帝之遺志,拯蒼生於荼蓼。而更飲酒過度,出入無恒,宴安游盤之樂,沈湎樽酒之間,不以寇仇為慮,竊為陛下危之。糟丘酒池,洛汭不返,皆陛下之殷鑒。臣蒙先帝夷險之恩,故不敢避干將之戮。」纂曰:「朕之罪也。不有貞亮之士,誰匡邪僻之君!」然昏虐自任,終不能改,常與左右因醉馳獵於坑澗之間,殿中侍御史王回、中書侍郎王儒扣馬諫曰:「千金之子坐不垂堂,萬乘之主清道而行,奈何去輿輦之安,冒奔騎之危!銜橛之變,
 動有不測之禍。愚臣竊所不安,敢以死爭,願陛下遠思袁盎攬轡之言,不令臣等受譏千載。」纂不納。



 纂番禾太守呂超擅伐鮮卑思盤,思盤遣弟乞珍訴超於纂,纂召超將盤入朝。超至姑臧,大懼,自結於殿中監杜尚,纂見超,怒曰:「卿恃兄弟桓桓,欲欺吾也,要當斬卿,然後天下可定。」超頓首不敢。纂因引超及其諸臣宴于內殿。呂隆屢勸纂酒,已至昏醉,乘步輓車將超等游于內。至琨華堂東閤,車不得過,纂親將竇川、駱騰倚劍於壁,推車過閤。超取劍擊纂,纂下車擒超,超刺纂洞胸,奔于宣德堂。川、騰與超格戰,超殺之。纂妻楊氏命禁兵討超,杜尚約兵
 舍杖。將軍魏益多入,斬纂首以徇曰:「纂違先帝之命,殺害太子,荒耽酒獵,暱近小人,輕害忠良,以百姓為草芥。番禾太守超以骨肉之親,懼社稷顛覆,已除之矣。上以安宗廟,下為太子報仇。凡我士庶,同茲休慶。」



 偽巴西公呂他、隴西公呂緯時在北城,或說緯曰:「超陵天逆上,士眾不附。明公以懿弟之親,投戈而起,姜紀、焦辨在南城,楊桓、田誠在東苑,皆我之黨也,何慮不濟!」緯乃嚴兵謂他曰:「隆、超弒逆,所宜擊之。昔田恒之亂,孔子鄰國之臣,猶抗言於哀公,況今蕭墻有難,而可坐觀乎!」他將從之,他妻梁氏止之曰:「緯、超俱兄弟之子,何為舍超助緯而
 為禍道乎!」他謂緯曰:「超事已立,據武庫,擁精兵,圖之為難。且吾老矣,無能為也。」超聞,登城告他曰:「纂信讒言,將滅超兄弟。超以身命之切,且懼社稷覆亡,故出萬死之計,為國家唱義,叔父當有以亮之。」超弟邈有寵於緯,說緯曰:「纂殘國破家,誅戮兄弟,隆、超此舉應天人之心,正欲尊立明公耳。先帝之子,明公為長,四海顒顒,人無異議。隆、超雖不達臧否,終不以孽代宗,更圖異望也,願公勿疑。」緯信之,與隆、超結盟,單馬入城,超執而殺之。



 初,纂嘗與鳩摩羅什棋,殺羅什子,曰:「斫胡奴頭。」羅什曰:「不斫胡奴頭,胡奴斫人頭。」超小字胡奴,竟以殺纂。纂在位三
 年,以元興元年死。隆既篡位,偽謚纂靈皇帝,墓號白石陵。



 隆字永基,光弟寶之子也,美姿貌,善騎射。光末拜北部護軍,稍歷顯位,有聲稱。超既殺纂,讓位於隆,隆有難色。超曰:「今猶乘龍上天,豈可中下!」隆以安帝元興元年遂僭即天王位。超先於番禾得小鼎,以為神瑞,大赦,改元為神鼎。追尊父寶為文皇帝,母衛氏為皇太后,妻楊氏為皇后,以弟超有佐命之勳,拜使持節、侍中、都督中外諸軍事、輔國大將軍、司隸校尉、錄尚書事,封安定公。



 隆多殺豪望,以立威名,內外囂然,人不自固。魏安人焦朗
 遣使說姚興將姚碩德曰:「呂氏因秦之亂,制命此州。自武皇棄世,諸子兢尋干戈,德刑不恤,殘暴是先,饑饉流亡,死者太半,唯泣訴昊天,而精誠無感。伏惟明公道邁前賢,任尊分陜,宜兼弱攻昧,經略此方,救生靈之沈溺,布徽政于玉門。篡奪之際,為功不難。」遣妻子為質。碩德遂率眾至姑臧。其部將姚國方言於碩德曰:「今懸師三千,後無繼援,師之難也。宜曜勁鋒,示其威武。彼以我遠來,必決死距戰,可一舉而平。」碩德從之。呂超出戰,大敗,遁還。隆收集離散,嬰城固守。



 時熒惑犯帝坐,有群雀鬥于太廟,死者數萬。東人多謀外叛,將軍魏益多又唱動
 群心,乃謀殺隆、超,事發,誅之,死者三百餘家。於是群臣表求與姚興通好,隆弗許。呂超諫曰:「通塞有時,艱泰相襲,孫權屈身於魏,譙周勸主迎降,豈非大丈夫哉?勢屈故也。天錫承七世之資,樹恩百載,武旅十萬,謀臣盈朝,秦師臨境,識者導以見機,而愎諫自專,社稷為墟。前鑒不遠,我之元龜也。何惜尺書單使,不以危易安!且令卑辭以退敵,然後內脩德政,廢興由人,未損大略。」隆曰:「吾雖常人,屬當家國之重,不能嗣守成基,保安社稷,以太祖之業委之於人,何面目見先帝於地下!」超曰:「應龍以屈伸為靈,大人以知機為美。今連兵積歲,資儲內盡,強寇外
 逼,百姓嗷然無糊口之寄,假使張、陳、韓、白,亦無如之何!陛下宜思權變大綱,割區區常慮。茍卜世有期,不在和好,若天命去矣,宗族可全。」隆從之,乃請降。碩德表隆為使持節、鎮西大將軍、涼州刺史、建康公。於是遣母弟愛子文武舊臣慕容築、楊穎、史難、閻松等五十餘家質于長安,碩德乃還。姚興謀臣皆曰:「隆藉伯父餘資,制命河外。今雖飢窘,尚能自支。若將來豐贍,終非國有。涼州險絕,世難先違,道清後順,不如因其飢弊而取之。」興乃遣使來觀虛實。



 沮渠蒙遜又伐隆,隆擊敗之,蒙遜請和結盟,留穀萬餘斛以振飢人。姑臧穀價踴貴,斗直錢五千
 文,人相食,饑死者十餘萬口。城門盡閉,樵採路絕,百姓請出城乞為夷虜奴婢者日有數百。隆懼沮動人情,盡坑之,於是積尸盈于衛路。



 禿髮傉檀及蒙遜頻來伐之,隆以二寇之逼也,遣超率騎二百,多齎珍寶,請迎于姚興。興乃遣其將齊難等步騎四萬迎之。難至姑臧,隆素車白馬迎于道旁。使胤告光廟曰:「陛下往運神略,開建西夏,德被蒼生,威振遐裔。枝嗣不臧,迭相篡弒。二虜交逼,將歸東京,謹與陛下奉訣於此。」歔欷慟泣,酸感興軍。隆率戶一萬,隨難東遷,至長安,興以隆為散騎常侍,公如故;超為安定太守;文武三十餘人皆擢敘之。其後
 隆坐與子弼謀反,為興所誅。



 呂光以孝武太元十二年定涼州,十五年僭立,至隆凡十有三載,以安帝元興三年滅。



 史臣曰:自晉室不綱,中原蕩析,苻氏乘釁,竊號神州。世明委質偽朝,位居上將,爰以心膂,受脤遐征。鐵騎如雲,出玉門而長騖;雕戈耀景,捐金丘而一息。蕞爾夷陬,承風霧卷,宏圖壯節,亦足稱焉。屬永固運銷,群雄兢起,班師右地,便有覬覦。於是要結六戎,潛窺鴈鼎;并吞五郡,遂假鴻名。控黃河以設險,負玄漠而為固,自謂克昌霸業,貽厥孫謀。尋而耄及政昏,親離眾叛,瞑目甫爾,釁發
 蕭墻。紹、纂凡才,負乘致寇;弘、超兇狡,職為亂階;永基庸庸,面縛姚氏。昔竇融歸順,榮煥累葉;隗囂干紀,靡終身世。而光棄茲勝躅,遵彼覆車,十數年間,終致殘滅。向使矯邪歸正,革偽為忠,鳴檄而蕃晉朝,仗義而誅醜虜,則燕、秦之地可定,桓、文之功可立,郭黁、段業豈得肆其姦,蒙遜、烏孤無所窺其隙矣。而猥竊非據,何其謬哉!夫天地之大德曰生,聖人之大寶曰位。非其人而處其位者,其禍必速;在其位而忘其德者,其殃必至。天鑒非遠,庸可濫乎!



 贊曰:金行不兢,寶業斯屯。瓜分九寓,沴聚三秦。呂氏伺
 隙,欺我人神。天命難假,終亦傾淪。



\end{pinyinscope}