\article{載記第十 慕容儁}

\begin{pinyinscope}

 慕容俊



 慕容俊,字宣英,皝之第二子也。初,廆常言:「吾積福累仁,子孫當有中原。」既而生俊,廆曰:「此兒骨相不恒,吾家得之矣。」及長,身長八尺二寸,姿貌魁偉,博觀圖書,有文武幹略。皝為燕王,拜俊假節、安北將軍、東夷校尉、左賢王、燕王世子。皝死,永和五年,僭即燕王位,依春秋列國故事稱元年,赦于境內。是時石季龍死,趙、魏大亂,俊將圖
 兼並之計,以慕容恪為輔國將軍,慕容評為輔弼將軍,陽騖為輔義將軍,慕容垂為前鋒都督、建鋒將軍,簡精卒二十餘萬以待期。是歲,穆帝使謁者陳沈拜俊為使持節、侍中、大都督、都督河北諸軍事、幽、冀、並、平四州牧、大將軍、大單于、燕王,承制封拜一如廆、皝故事。



 明年,俊率三軍南伐,出自盧龍,次于無終。石季龍幽州刺史王午棄城走,留其將王他守薊。俊攻陷其城,斬他,因而都之。徙廣寧、上谷人于徐無,代郡人于凡城而還。



 及冉閔殺石祗,僭稱大號,遣其使人常煒聘於俊。俊引之觀下,使其記室封裕詰之曰:「冉閔養息常才,負恩篡逆,有何
 祥應而僭稱大號?」煒曰:「天之所興,其致不同,狼烏紀於三王,麟龍表于漢、魏。寡君應天馭歷,能無祥乎!且用兵殺伐,哲王盛典,湯、武親行誅放,而仲尼美之。魏武養於宦官,莫知所出,眾不盈旅,遂能終成大功。暴胡酷亂,蒼生屠膾,寡君奮劍而誅除之,黎元獲濟,可謂功格皇天,勳侔高祖。恭承乾命,有何不可?」裕曰:「石祗去歲使張舉請救,云璽在襄國,其言信不?又聞閔鑄金為己象,壞而不成,奈何言有天命?」煒曰:「誅胡之日,在鄴者略無所遺,璽何從而向襄國,此求救之辭耳。天之神璽,實在寡君。且妖孽之徒,欲假奇眩眾,或改作萬端,以神其事。寡君
 今已握乾府,類上帝,四海懸諸掌,大業集于身,何所求慮而取信此乎!鑄形之事,所未聞也。」俊既銳信舉言,又欣於閔鑄形之不成也,必欲審之,乃積薪置火於其側,命裕等以意喻之。煒神色自若,抗言曰:「結髮已來,尚不欺庸人,況千乘乎!巧詐虛言以救死者,使臣所不為也。直道受戮,死自分耳。益薪速火,君之大惠。」左右勸俊殺之,俊曰:「古者兵交,使在其間,此亦人臣常事。」遂赦之。



 遣慕容恪略地中山,慕容評攻王午于魯口。恪次唐城,冉閔將白同、中山太守侯龕固守不下。恪留其將慕容彪攻之,進討常山。評次南安,王午遣其將鄭生距評。評逆
 擊,斬之,侯龕踰城出降。恪進剋中山,斬白同。俊軍令嚴明,諸將無所犯。閔章武太守賈堅率郡兵邀評戰于高城,擒堅於陣,斬首三千餘級。



 是歲丁零翟鼠及冉閔將劉準等率其所部降于俊,封鼠歸義王,拜準左司馬。



 時鮮卑段勤初附於俊,其後復叛。俊遣慕容恪及相國封弈討冉閔於安喜,慕容垂討段勤于繹幕,俊如中山,為二軍聲勢。閔懼,奔于常山,恪追及於泒水。閔威名素振,眾咸憚之。恪謂諸將曰:「閔師老卒疲,實為難用;加其勇而無謀,一夫之敵耳。雖有甲兵,不足擊也。吾今分軍為三部,掎角以待之。閔性輕銳,又知吾軍勢非其敵,必出
 萬死衝吾中軍。吾今貫甲厚陣以俟其至,諸君但厲卒,從旁須其戰合,夾而擊之,蔑不剋也。」及戰,敗之,斬首七千餘級,擒閔,送之,斬於龍城。恪屯軍呼沲。閔將蘇亥遣其將金光率騎數千襲恪,恪逆擊,斬之,亥大懼,奔於並州。恪進據常山,段勤懼而請降,遂進攻鄴。閔將蔣幹閉城距守。俊又遣慕容評等率騎一萬會攻鄴。是時䴏巢于俊正陽殿之西椒,生三雛,項上有豎毛;凡城獻異鳥,五色成章。俊謂群僚曰:「是何祥也?」咸稱:「䴏者,燕鳥也。首有毛冠者,言大燕龍興,冠通天章甫之象也。巢正陽西椒者,言至尊臨軒朝萬國之徵也。三子者,數應三統之
 驗也。神鳥五色,言聖朝將繼五行之籙以御四海者也。」俊覽之大悅。既而蔣幹率銳卒五千出城挑戰,慕容評等擊敗之,斬首四千餘級,乾單騎還鄴。於是群臣勸俊稱尊號,俊答曰:「吾本幽漠射獵之鄉,被髮左衽之俗,歷數之籙寧有分邪!卿等茍相褒舉,以覬非望,實匪寡德所宜聞也。」慕容恪、封弈討王午于魯口,降之。尋而慕容評攻剋鄴城,送冉閔妻子僚屬及其文物于中山。



 先是,蔣幹以傳國璽送于建鄴,俊欲神其事業,言歷運在己,乃詐云閔妻得之以獻,賜號曰「奉璽君」,因以永和八年僭即皇帝位,大赦境內,建元曰元璽,署置百官。以封弈
 為太尉,慕容恪為侍中,陽騖為尚書令,皇甫真為尚書左僕射,張希為尚書右僕射,宋活為中書監,韓恒為中書令,其餘封授各有差。追尊廆為高祖武宣皇帝,皝為太祖文明皇帝。時朝廷遣使詣俊,俊謂使者曰:「汝還白汝天子,我承人之乏,為中國所推,已為帝矣。」初,石季龍使人探策于華山,得玉版,文曰:「歲在申酉,不絕如線。歲在壬子,真人乃見。」及此,燕人咸以為俊之應也。改司州為中州,置司隸校尉官。群下言:「大燕受命,上承光紀黑精之君,運歷傳屬,代金行之后,宜行夏之時,服周之冕,旗幟尚黑,牲牡尚玄。」俊從之。其從行文武、諸籓使人及登
 號之日者,悉增位三級。泒河之師,守鄴之軍,下及戰士,賜各有差。臨陣戰亡者,將士加贈二等,士卒復其子孫。殿中舊人皆隨才擢敘。立其妻可足渾氏為皇后,世子曄為皇太子。



 晉寧朔將軍榮胡以彭城、魯郡叛降于俊。



 常山人李犢聚眾數千,反於普壁壘,俊遣慕容恪率眾討降之。



 初,冉閔既敗,王午自號安國王。午既死,呂護復襲其號,保于魯口。恪進討走之,遣前軍悅綰追及于野王,悉降其眾。



 姚襄以梁國降于俊。以慕容評為都督秦、雍、益、梁、江、揚、荊、徐、袞、豫十州河南諸軍事,權鎮于洛水;慕容彊為前鋒都督、都督荊、徐二州緣淮諸軍事,進據
 河南。



 俊自和龍至薊城,幽冀之人為東遷,互相驚擾,所在屯結。其下請討之,俊曰:「群小以朕東巡,故相惑耳。今朕既至,尋當自定。然不虞之備亦不可不為。」於是令內外戒嚴。



 苻生河內太守王會、黎陽太守韓高以郡歸俊。晉蘭陵太守孫黑、濟北太守高柱、建興太守高甕各以郡叛歸于俊。初,俊車騎大將軍、范陽公劉寧屯據蕕城,降於苻氏,至此,率戶二千詣薊歸罪,拜後將軍。高句麗王釗遣使謝恩,貢其方物。俊以釗為營州諸軍事、征東大將軍、營州刺史,封樂浪公,王如故。



 俊給事黃門侍郎申胤上言曰:



 夫名尊禮重,先王之制。冠冕之式,代或
 不同。漢以蕭、曹之功,有殊群辟,故劍履上殿,入朝不趨。世無其功,則禮宜闕也。至於東宮,體此為儀,魏、晉因循,制不納舄。今皇儲過謙,準同百僚,禮卑逼下,有違朝式。太子有統天之重,而與諸王齊冠遠游,非所以辨章貴賤也。祭饗朝慶,宜正服袞衣九文,冠冕九旒。又仲冬長至,太陰數終,黃鐘產氣,綿微於下,此月閉關息旅,后不省方。《禮記》曰:「是月也,事欲靜,君子齊戒去聲色。」唯《周官》有天子之南郊從八能之說。或以有事至靈,非朝饗之節,故有樂作之理。王者慎微,禮從其重。前來二至闕鼓,不宜有設,今之鏗鏘,蓋以常儀。二至之禮、事殊餘節,猥
 動金聲,驚越神氣,施之宣養,實為未盡。又朝服雖是古禮,絳褠始於秦、漢,迄於今代,遂相仍準。朔望正旦,乃具袞舄。禮,諸侯旅見天子,不得終事者三,雨沾服失容,其在一焉。今或朝日天雨,未有定儀。禮貴適時,不在過恭。近以地濕不得納舄,而以袞襈改履。案言稱朝服,所以服之而朝,一體之間,上下二制,或廢或存,實乖禮意。大燕受命,侔蹤虞、夏,諸所施行,宜損益定之,以為皇代永制。



 俊曰:「其劍舄不趨,事下太常參議。太子服袞冕,冠九旒,超級逼上,未可行也。冠服何容一施一廢,皆可詳定。」



 初,段蘭之子龕因冉閔之亂,擁眾東屯廣固,自號齊王,
 稱籓于建鄴,遣書抗中表之儀,非俊正位。俊遣慕容恪、慕容塵討之。恪既濟河。龕弟羆驍勇有智計,言於龕曰:「慕容恪善用兵,加其眾旅既盛,恐不可抗也。若頓兵城下,雖復請降,懼終不聽。王但固守,羆請率精銳距之。若其戰捷,王可馳來追擊,使虜匹馬無反。如其敗也,遽出請降,不失千戶侯也。」龕弗從。羆固請行,龕怒斬之,率眾三萬來距恪。恪遇龕於濟水之南,與戰,大敗之,遂斬其弟欽,盡俘其眾。恪進圍廣固,諸將勸恪宜急攻之,恪曰:「軍勢有宜緩以剋敵,有宜急而取之。若彼我勢均,且有彊援,慮腹背之患者,須急攻之,以速大利。如其我彊彼
 弱,外無寇援,力足制之者,當羈縻守之,以待其斃。兵法十圍五攻,此之謂也。龕恩結賊黨,眾未離心,濟南之戰,非不銳也,但其用之無術,以致敗耳。今憑固天險,上下同心,攻守勢倍,軍之常法。若其促攻,不過數旬,剋之必矣,但恐傷吾士眾。自有事已來,卒不獲寧,吾每思之,不覺忘寢,亦何宜輕殘人命乎!當持久以取耳。」諸將皆曰:「非所及也。」乃築室反耕,嚴固圍壘。龕所署徐州刺史王騰、索頭單于薛雲降于恪。段龕之被圍也,遣使詣建鄴請救。穆帝遣北中郎將荀羨赴之,憚虜彊遷延不敢進。攻破陽都,斬王騰以歸。恪遂剋廣固,以龕為伏順將軍,
 徙鮮卑胡羯三千餘戶於薊,留慕容塵鎮廣固,恪振旅而歸。



 俊太子曄死,偽謚獻懷。升平元年,復立次子為皇太子,赦其境內,改元曰光壽。



 遣其撫軍慕容垂、中軍慕容虔與護軍平熙等率步騎八萬討丁零敕勒於塞北,大破之,俘斬十餘萬級,獲馬十三萬匹,牛羊億餘萬。



 初,廆有駿馬曰赭白,有奇相逸力。石季龍之伐棘城也,皝將出避難,欲乘之,馬悲鳴蹄齧,人莫能近。皝曰:「此馬見異先朝,孤常仗之濟難,今不欲者,蓋先君之意乎!」乃止。季龍尋退,皝益奇之。至是,四十九歲矣,而駿逸不虧,俊比之於鮑氏驄,命鑄銅以圖其象,親為銘贊,鐫勒其
 旁,置之薊城東掖門。是歲,象成而馬死。



 匈奴單于賀賴頭率部落三萬五千降于俊,拜寧西將軍、雲中郡公,處之于代郡平舒城。



 晉太山太守諸葛攸伐其東郡。俊遣慕容恪距戰,王師敗績。北中郎將謝萬先據梁、宋,懼而遁歸。恪進兵入寇河南,汝、潁、譙、沛皆陷,置守宰而還。



 俊自薊城遷于鄴,赦其境內,繕修宮殿,復銅雀臺。



 廷尉監常煒上言:「大燕雖革命創制,至於朝廷銓謨,亦多因循魏、晉,唯祖父不殮葬者,獨不聽官身清朝,斯誠王教之首,不刊之式。然禮貴適時,世或損益,是以高祖制三章之法,而秦人安之。自頃中州喪亂,連兵積年,或遇傾城
 之敗,覆軍之禍,坑師沈卒,往往而然,孤孫煢子,十室而九。兼三方岳峙,父子異邦,存亡吉凶,杳成天外。或便假一時,或依嬴博之制,孝子糜身無補,順孫心喪靡及,雖招魂虛葬以敘罔極之情,又禮無招葬之文,令不此載。若斯之流,抱琳瑯而無申,懷英才而不齒,誠可痛也。恐非明揚側陋,務盡時珍之道。吳起、二陳之疇,終將無所展其才幹。漢祖何由免於平城之圍?郅支之首何以懸於漢關?謹案《戊辰詔書》,蕩清瑕穢,與天下更始,以明惟新之慶。五六年間,尋相違伐,於則天之體,臣竊未安。」俊曰:「煒宿德碩儒,練明刑法,覽其所陳,良足採也。今六合
 未寧,喪亂未已,又正當搜奇拔異之秋,未可才行兼舉,且除此條,聽大同更議。」



 使昌黎、遼東二郡營起廆廟,范陽、燕郡構皝廟,以其護軍平熙領將作大匠,監造二廟焉。



 苻堅平州刺史劉特率戶五千降于俊。



 河間李黑聚眾千餘,攻略州郡,殺棗彊令衛顏,俊長樂太守傅顏討斬之。



 常山大樹自拔,根下得璧七十、珪七十三,光色精奇,有異常玉。俊以為嶽神之命,遣其尚書郎段勤以太宰祀之。



 初,冉閔之僭號也,石季龍將李歷、張平、高昌等並率其所部稱籓於儁,遣子入侍。既而投款建鄴,結援苻堅,並受爵位,羈縻自固,雖貢使不絕,而誠節未盡。呂
 護之走野王也,遣弟奉表謝罪於儁,拜寧南將軍、河內太守。又上黨馮鴦自稱太守,附於張平,平屢言之,儁以平故,赦其罪,以為京兆太守。護、鴦亦陰通京師。張平跨有新興、鴈門、西河、太原、上黨、上郡之地,壘壁三百餘,胡晉十餘萬戶,遂拜置徵、鎮,為鼎峙之勢。儁其司徒慕容評討平,領軍慕輿根討鴦,司空陽騖討昌,撫軍慕容臧攻歷。並州壘壁降者百餘所,以尚書右僕射悅綰為安西將軍、領護匈奴中郎將、並州刺史以撫之。平所署征西諸葛驤、鎮北蘇象、寧東喬庶、鎮南石賢等率壘壁百三十八降于儁,儁大悅,皆復其官爵。既而平率眾三
 千奔于平陽,鴦奔于野王,歷走滎陽,昌奔邵陵,悉降其眾。



 儁于是復圖入寇,兼欲經略關西,乃令州郡校閱見丁,精覆隱漏,率戶留一丁,餘悉發之,欲使步卒滿一百五十萬,期明年大集,將進臨洛陽,為三方節度。武邑劉貴上書極諫,陳百姓凋弊,召兵非法,恐人不堪命,有土崩之禍,並陳時政不便于時者十有三事。儁覽而悅之,付公卿博議,事多納用,乃改為三五占兵,寬戎備一周,悉令明年季冬赴集鄴都。



 是歲,晉將荀羨攻山茌,拔之。斬儁太山太守賈堅。儁青州刺史慕容塵遣司馬悅明救之,羨師敗績,復陷山茌。



 儁立小學于顯賢里以教胄
 子。封其子泓為濟北王,沖為中山王。宴群臣於蒲池,酒酣,賦詩,因談經史,語及周太子晉,潸然流涕,顧謂群臣曰:「昔魏武追痛倉舒,孫權悼登無已,孤常謂二主緣愛稱奇,無大雅之體。自曄亡以來,孤須髮中白,始知二主有以而然。卿等言曄定何如也?孤今悼之,得無貽怪將來乎?」其司徒左長史李績對曰:「獻懷之在東宮,臣為中庶子,既忝近侍,聖質志業,臣實不敢不知。臣聞道備無愆,其唯聖人乎。先太子大德有八,未見闕也。」儁曰:「卿言亦以過矣,然試言之。」績曰:「至孝自天,性與道合,此其一也。聰敏慧悟,機思若流,此其二也。沈毅好斷,理詣無幽,
 此其三也。疾諛亮物,雅悅直言,此其四也。好學愛賢,不恥下問,此其五也。英姿邁古,藝業超時,此其六也。虛襟恭讓,尊師重道,此其七也。輕財好施,勤恤民隱,此其八也。」儁泣曰:「卿雖褒譽,然此兒若在,吾死無憂也。吾既不能追蹤唐、虞,官天下以禪有德,近模三王,以世傳授。景茂幼沖,器藝未舉,卿以為何如?」績曰:「皇太子天資岐嶷,聖敬日躋,而八闃然,二闕未補,雅好游田,娛心絲竹,所以為損耳。」儁顧謂曰:「伯陽之言,藥石之惠,汝宜戢之。」因問高年疾苦、孤寡不能自存者,賜穀帛有差。



 儁夜夢石季龍齧其臂,寤而惡之,命發其墓,剖棺出尸,蹋而
 罵之曰:「死胡安敢夢生天子!」遣其御史中尉陽約數其殘酷之罪,鞭之,棄于漳水。



 諸葛攸又率水陸三萬討俊,入自石門,屯于河渚。攸部將匡超進據嵪敖,蕭館屯於新柵,又遣督護徐冏率水軍三千泛舟上下,為東西聲勢。儁遣慕容評、傅顏等統步騎五萬,戰于東阿,王師敗績。



 塞北七國賀蘭、涉勒等皆降。



 俄而儁寢疾,謂慕容恪曰:「吾所疾惙然,當恐不濟。修短命也,復何所恨!但二寇未除,景茂沖幼,慮其未堪多難。吾欲遠追宋宣,以社稷屬汝。」恪曰:「太子雖幼,天縱聰聖,必能勝殘刑措,不可以亂正統也。」儁怒曰:「兄弟之間豈虛飾也!」恪曰:「陛下若以
 臣堪荷天下之任者,寧不能輔少主乎!」儁曰:「若汝行周公之事,吾復何憂!李績清方忠亮,堪任大事,汝善遇之。」



 是時兵集鄴城,盜賊互起,每夜攻劫,晨昏斷行。於是寬常賦,設奇禁,賊盜有相告者賜奉車都尉,捕誅賊首木穀和等百餘人,乃止。



 升平四年,俊死,時年四十二,在位十一年。偽謚景昭皇帝,廟號烈祖,墓號龍陵。



 俊雅好文籍,自初即位至末年,講論不倦,覽政之暇,唯與侍臣錯綜義理,凡所著述四十餘篇。性嚴重,慎威儀,未曾以慢服臨朝,雖閑居宴處亦無懈怠之色云。



 韓恒,字景山,灌津人也。父默,以學行顯名。恒少能屬文,師事同郡張載,載奇之,曰:「王佐才也。」身長八尺一寸,博覽經籍,無所不通。永嘉之亂,避地遼東。廆既逐崔毖,復徙昌黎,召見,嘉之,拜參軍事。咸和中,宋該等建議以廆立功一隅,勤誠王室,位卑任重,不足以鎮華夷,宜表請大將軍、燕王之號。廆納之,命群僚博議,咸以為宜如該議。恒駁曰:「自群胡乘間,人嬰荼毒,諸夏蕭條,無復綱紀。明公忠武篤誠,憂勤社稷,抗節孤危之中,建功萬里之外,終古勤王之義,未之有也。夫立功者患信義不著,不患名位不高,故桓文有寧復一匡之功,亦不先求禮命
 以令諸侯。宜繕甲兵,候機會,除群凶,靖四海,功成之後,九錫自至。且要君以求寵爵者,非為臣之義也。」廆不平之,出為新昌令。皝為鎮軍,復參軍事。遷營丘太守,政化大行。俊為大將軍,徵拜咨議參軍,加揚烈將軍。



 俊僭位,將定五行次,眾論紛紜。恒時疾在龍城,俊召恒以決之。恆未至而群臣議以燕宜承晉為水德。既而恒至,言於俊曰:「趙有中原,非唯人事,天所命也。天實與之,而人奪之,臣竊謂不可。且大燕王迹始自於震,於《易》,震為青龍。受命之初,有龍見於都邑城,龍為木德,幽契之符也。」俊初雖難改,後終從恒議。俊秘書監清河聶熊聞恒言,乃
 歎曰:「不有君子,國何以興,其韓令君之謂乎!」後與李產俱傅東宮,從太子曄入朝,俊顧謂左右曰:「此二傅一代偉人,未易繼也。」其見重如此。



 李產,字子喬,范陽人也。少剛厲,有志格。永嘉之亂,同郡祖逖擁眾部於南土,力能自固,產遂往依之。逖素好從橫,弟約有大志,產微知其旨,乃率子弟十數人間行還鄉里,仕於石氏,為本郡太守。及慕容俊南征,前鋒達郡界,鄉人皆勸產降,產曰:「夫受人之祿,當同其安危,今若舍此節以圖存,義士將謂我何!」眾潰,始詣軍請降。俊嘲
 之曰:「卿受石氏寵任,衣錦本鄉,何故不能立功於時,而反委質乎!烈士處身於世,固當如是邪?」產泣曰:「誠知天命有歸,非微臣所抗。然犬馬為主,豈忘自效,但以孤窮勢蹙,致力無術,FC俛歸死,實非誠款。」俊嘉其慷慨,顧謂左右曰:「此真長者也。」乃擢用之,歷位尚書。性剛正,好直言,每至進見,未曾不論朝政之得失,同輩咸憚焉,俊亦敬其儒雅。前後固辭年老,不堪理劇。轉拜太子太保。謂子績曰:「以吾之才而致於此,始者之願亦已過矣,不可復以西夕之年取笑於來今也。」固辭而歸,死於家。子績。



 績字伯陽,少以風節知名,清辯有辭理。弱冠為郡功曹。
 時石季龍親征段遼,師次范陽,百姓饑儉,軍供有闕。季龍大怒,大守惶怖避匿。績進曰:「郡帶北裔,與寇接攘,疆埸之間,人懷危慮。聞輿駕親戎,將除殘賊,雖嬰兒白首,咸思效命,非唯為國,亦自求寧,雖身膏草野,猶甘為之,敢有私吝而闕軍實!但此年災儉,家有菜色,困弊力屈,無所取濟,逋廢之罪,情在可矜。」季龍見績年少有壯節,嘉而恕之,於是太守獲免。刺史王午辟為主簿。俊之南征也,隨午奔魯口。鄧恒謂午曰:「績鄉里在北,父已降燕,今雖在此,終不為用,方為人患。」午曰:「績於喪亂之中捐家立義,情節之重,有侔古烈,若懷嫌害之,必駭眾望。」恒
 乃止。午恐績終為恆所害,乃資遣之。及到,俊責其背親後至,績答曰:「臣聞豫讓報智伯仇,稱於前史。既官身所在,何事非君!陛下方弘唐、虞之化,臣實未謂歸順之晚也。」俊曰:「此亦事主之一節耳。」累遷太子中庶子。及立,慕容恪欲以績為尚書右僕射,憾績往言,不許。恪屢請,乃謂恪曰:「萬機之事委之叔父,伯陽一人,請獨裁。」績遂憂死。



\end{pinyinscope}