\article{列傳第三十一 周浚子嵩謨從父弟馥成公簡茍晞華軼劉喬孫耽耽子柳}

\begin{pinyinscope}

 周浚子嵩謨從父弟馥成公簡茍晞華軼劉喬孫耽耽子柳



 周浚,字開林,汝南安成人也。父裴,少府卿。浚性果烈。以才理見知,有人倫鑒識。鄉人史曜素微賤,眾所未知,浚獨引之為友,遂以妹妻之,曜竟有名於世。浚初不應州郡之辟,後仕魏為尚書郎。累遷御史中丞,拜折衝將軍、揚州刺史,封射陽侯。



 隨王渾伐吳,攻破江西屯戍,與孫皓中軍大戰,斬偽丞相張悌等首級數千,俘馘萬計,進
 軍屯於橫江。時聞龍驤將軍王濬既破上方,別駕何惲說浚曰:「張悌率精銳之卒,悉吳國之眾,殄滅於此,吳之朝野莫不震懾。今王龍驤既破武昌,兵威甚盛,順流而下,所向輒剋,土崩之勢見矣。竊謂宜速渡江,直指建鄴,大軍卒至,奪其膽氣,可不戰而擒。」浚善其謀,便使白渾。惲曰:「渾暗於事機,而欲慎己免咎,必不我從。」浚固使白之,渾果曰:「受詔但令江北抗衡吳軍,不使輕進。貴州雖武,豈能獨平江東!今者違命,勝不足多;若其不勝,為罪已重。且詔令龍驤受我節度,但當具君舟楫,一時俱濟耳。」惲曰:「龍驤剋萬里之寇,以既濟之功來受節度,未之
 聞也。且握兵之要,可則奪之,所謂受命不受辭也。今渡江必全剋獲,將有何慮?若疑於不濟,不可謂智;知而不行,不可謂忠,實鄙州上下所以恨恨也。」渾執不聽。居無何而濬至,渾召之不來,乃直指三山,孫皓遂降於濬。渾深恨之,而欲與浚爭功。惲箋與浚曰:「《書》貴克讓,《易》大謙光,斯古文所詠,道家所崇。前破張悌,吳人失氣,龍驤因之,陷其區宇。論其前後,我實緩師,動則為傷,事則不及。而今方競其功。彼既不吞聲,將虧雍穆之弘,興矜爭之鄙,斯愚情之所不取也。」浚得箋,即諫止渾,渾不能納,遂相表奏。



 浚既濟江,與渾共行吳城壘,綏撫新附,以功
 進封成武侯,食邑六千戶,賜絹六千匹。明年,移鎮秣陵。時吳初平,屢有逃亡者,頻討平之。賓禮故老,搜求俊乂,甚有威德,吳人悅服。



 初,吳之未平也,浚在弋陽,南北為互市,而諸將多相襲奪以為功。吳將蔡敏守于沔中,其兄珪為將在秣陵,與敏書曰:「古者兵交,使在其間,軍國固當舉信義以相高。而聞疆場之上,往往有襲奪互市,甚不可行,弟慎無為小利而忘大備也。」候者得珪書以呈浚,浚曰:「君子也。」及渡江,求珪,得之,問其本,曰;「汝南人也。」浚戲之曰:「吾固疑吳無君子,而卿果吾鄉人。」



 遷侍中。武帝問浚:「卿宗後生,稱誰為可?」答曰:「臣叔父子恢,稱重
 臣宗;從父子馥,稱清臣宗。」帝並召用。浚轉少府,以本官領將作大匠。改營宗廟訖,增邑五百戶。後代王渾為使持節、都督揚州諸軍事、安東將軍,卒於位。三子:顗、嵩、謨。顗嗣爵,別有傳云。



 嵩字仲智,狷直果俠,每以才氣陵物。元帝作相,引為參軍。及帝為晉王,又拜奉朝請。嵩上疏曰:「臣聞取天下者,常以無事。及其有事,不足以取天下。故古之王者,必應天順時,義全而後取,讓成而後得,是以享世長久,重光萬載也。今議者以殿下化流江漢,澤被六州,功濟蒼生,欲推崇尊號。臣謂今梓宮未反,舊京未清,義夫泣血,士
 女震動;宜深明周公之道,先雪社稷大恥,盡忠言嘉謀之助,以時濟弘仁之功,崇謙謙之美,推後己之誠;然後揖讓以謝天下,誰敢不應,誰敢不從!」由是忤旨,出為新安太守。



 嵩怏怏不悅,臨發,與散騎郎張嶷在侍中戴邈坐,褒貶朝士,又詆毀邈,邈密表之。帝召嵩入,面責之曰:「卿矜豪傲慢,敢輕忽朝廷,由吾不德故耳。」嵩跪謝曰:「昔唐虞至聖,四凶在朝。陛下雖聖明御世,亦安能無碌碌之臣乎!」帝怒,收付廷尉。廷尉華恆以嵩大不敬棄市論,嶷以扇和減罪除名。時顗方貴重,帝隱忍。久之,補廬陵太守,不之職,更拜御史中丞。



 是時帝以王敦勢盛,漸疏
 忌王導等。嵩上疏曰:



 臣聞明君思隆其道,故賢智之士樂在其朝;忠臣將明其節,故量時而後仕。樂在其朝,故無過任之譏;將明其節,故無過寵之謗。是以君臣並隆,功格天地。近代以來,德廢道衰,君懷術以御臣,臣挾利以事君,君臣交利而禍亂相尋,故得失之迹難可詳言。臣請較而明之。



 夫傅說之相高宗,申召之輔宣王,管仲之佐齊桓,衰范之翼晉文,或宗師其道,垂拱受成,委以權重,終至匡主,未有憂其逼己,還為國蠹者也。始田氏擅齊,王莽篡漢,皆藉封土之彊,假累世之寵,因闇弱之主,資母后之權,樹比周之黨,階絕滅之勢,然後乃能行
 其私謀,以成篡奪之禍耳。豈遇立功之主,為天人所相,而能運其姦計,以濟其不軌者哉!光武以王族奮於閭閻,因時之望,收攬英奇,遂續漢業,以美中興之功。及天下既定,頗廢黜功臣者,何哉?武力之士不達國體,以立一時之功,不可久假以權勢,其興廢之事,亦可見矣。近者三國鼎峙,並以雄略之才,命世之能,皆委賴俊哲,終成功業,貽之後嗣,未有愆失遺方來之恨者也。



 今王導、王廣等,方之前賢,猶有所後。至於忠素竭誠,義以輔上,共隆洪基,翼成大業,亦昔之亮也。雖陛下乘奕世之德,有天人之會,割據江東,奄有南極,龍飛海顒,興復舊物,
 此亦群才之明,豈獨陛下之力也。今王業雖建,羯寇未梟,天下蕩蕩,不賓者眾,公私匱竭,倉庾未充,梓宮沈淪,妃后不反,正委賢任能推轂之日也。功業垂就,晉祚方隆,而一旦聽孤臣之言,惑疑似之說,乃更以危為安,以疏易親,放逐舊德,以佞伍賢,遠虧既往之明,顧傷伊管之交,傾巍巍之望,喪如山之功,將令賢智杜心,義士喪志,近招當時之患,遠遺來世之笑。夫安危在號令,存亡在寄任,以古推今,豈可不寒心而哀歎哉!



 臣兄弟受遇,無彼此之嫌,而臣干犯時諱,觸忤龍鱗者何?誠念社稷之憂,欲報之於陛下也。古之明王,思聞其過,悟逆旅之
 言,以明成敗之由,故採納愚言,以考虛實,上為宗廟無窮之計,下收億兆元元之命。臣不勝憂憤,竭愚以聞。



 疏奏,帝感悟,故導等獲全。



 王敦既害顗而使人弔嵩,嵩曰:「亡兄天下人,為天下人所殺,復何所弔!」敦甚銜之,懼失人情,故未加害,用為從事中郎。嵩,王應嫂父也,以顗橫遇禍,意恆憤憤,嘗眾中云:「應不宜統兵。」敦密使妖人李脫誣嵩及周筵潛相署置,遂害之。嵩精於事佛,臨刑猶於市誦經云。



 謨以顗故,頻居顯職。王敦死後,詔贈戴若思、譙王承等,而未及顗。時謨為後軍將軍,上疏曰:



 臣亡兄顗,昔蒙先
 帝顧眄之施,特垂表啟,以參戎佐,顯居上列,遂管朝政,並與群后共隆中興,仍典選曹,重蒙寵授,忝位師傅,得與陛下揖讓抗禮,恩結特隆。加以鄙族結婚帝室,義深任重,庶竭股肱,以報所受。凶逆所忌,惡直醜正。身陷極禍,忠不忘君,守死善道,有隕無二。顗之云亡,誰不痛心,況臣同生,能不哀結!



 王敦無君,由來實久,元惡之甚,古今無二。幸賴陛下聖聰神武,故能摧破凶彊,撥亂反正,以寧區宇。前軍事之際,聖恩不遺,取顗息閔,得充近侍。臣時面啟,欲令閔還襲臣亡父侯爵。時卞壼、庾亮並侍御坐,壼云:「事了當論顯贈。」時未淹久,言猶在耳。至於譙
 王承、甘卓,已蒙清復,王澄久遠,猶在論議。況顗忠以衛主,身死王事,雖嵇紹之不違難,何以過之!至今不聞復封加贈褒顯之言。不知顗有餘責,獨負殊恩,為朝廷急於時務,不暇論及?此臣所以痛心疾首,重用哀歎者也。不勝辛酸,冒陳愚款。



 疏奏,不報。謨復重表,然後追贈顗官。



 謨歷少府、丹陽尹、侍中、中護軍,封西平侯。卒贈金紫光祿大夫,謚曰貞。



 馥字祖宣,浚從父弟也。父蕤,安平太守。馥少與友人成公簡齊名,俱起家為諸王文學,累遷司徒左西屬。司徒王渾表「馥理識清正,兼有才幹,主定九品,檢括精詳。臣
 委任責成,褒貶允當,請補尚書郎」。許之。稍遷司徒左長史、吏部郎,選舉精密,論望益美。轉御史中丞、侍中,拜徐州刺史,加冠軍將軍、假節。徵為廷尉。



 惠帝幸鄴,成都王穎以馥守河南尹。陳、上官已等奉清河王覃為太子,加馥衛將軍、錄尚書,馥辭不受。覃令馥與上官已合軍,馥認已小人縱暴,終為國賊,乃共司隸滿奮等謀共除之,謀泄,為已所襲,奮被害,馥走得免。及已為張方所敗,召馥還攝河南尹。暨東海王越迎大駕,以馥為中領軍,未就,遷司隸校尉,加散騎常侍、假節,都督諸軍事於澠池。帝還宮,出為平東將軍、都督揚州諸軍事,代劉準為
 鎮東將軍,與周等討陳敏,滅之,以功封永寧伯。



 馥自經世故,每欲維正朝遷,忠情懇至。以東海王越不盡臣節,每言論厲然,越深憚之。馥睹群賊孔熾,洛陽孤危,乃建策迎天子遷都壽春。永嘉四年,與長史吳思、司馬殷識上書曰:「不圖厄運遂至於此!戎狄交侵,畿甸危逼。臣輒與祖納、裴憲、華譚、孫惠等三十人伏思大計,僉以殷人有屢遷之事,周王有岐山之徙,方今王都罄乏,不可久居,河朔蕭條,崤函險澀,宛都屢敗,江漢多虞,於今平夷,東南為愈。淮揚之地,北阻塗山,南抗靈嶽,名川四帶,有重險之固。是以楚人東遷,遂宅壽春,徐邳、東海,亦足
 戍禦。且運漕四通,無患空乏。雖聖上神聰,元輔賢明,居儉守約,用保宗廟,未若相土遷宅,以享永祚。臣謹選精卒三萬,奉迎皇駕。輒檄前北中郎將裴憲行使持節、監豫州諸軍事、東中郎將,風馳即路。荊、湘、江、揚各先運四年米租十五萬斛,布絹各十四萬匹,以供大駕。令王浚、茍晞共平河朔,臣等戮力以啟南路。遷都弭寇,其計並得。皇輿來巡,臣宜轉據江州,以恢王略。知無不為,古人所務,敢竭忠誠,庶報萬分。朝遂夕隕,猶生之願。」



 越與茍晞不協,馥不先白於越,而直上書,越大怒。先是,越召馥及淮南太守裴碩,馥不肯行,而令碩率兵先進。碩貳於
 馥,乃舉兵稱馥擅命,已奉越密旨圖馥,遂襲之,為馥所敗。碩退保東城,求救於元帝。帝遣揚威將軍甘卓、建威將軍郭逸攻馥于壽春。安豐太守孫惠帥眾應之,使謝摛為檄。摛,馥之故將也。馥見檄,流涕曰:「必謝摛之辭。」摛聞之,遂毀草。旬日而馥眾潰,奔於項,為新蔡王確所拘,憂憤發病卒。



 初,華譚之失廬江也,往壽春依馥,及馥軍敗,歸于元帝。帝問曰:「周祖宣何至于反?」譚封曰:「周馥雖死,天下尚有直言之士。馥見寇賊滋蔓,王威不振,故欲移都以紓國難。方伯不同,遂致其伐。曾不踰時,而京都淪沒。若使從馥之謀,或可後亡也。原情求實,何得為反!」
 帝曰:「馥位為征鎮,握兵方隅,召而不入,危而不持,亦天下之罪人也。」譚曰:「然。馥振纓中朝,素有俊彥之稱;出據方嶽,實有偏任之重,而高略不舉,往往失和,危而不持,當與天下共受其責。然謂之反,不亦誣乎!」帝意始解。



 馥有二子:密、矯。密字泰玄,性虛簡,時人稱為清士,位至尚書郎,矯字正玄,亦有才幹。



 成公簡,字宗舒,東郡人也。家世二千石。性朴素,不求榮利,潛心味道,罔有干其志者。默識過人。張茂先每言:「簡清靜比楊子雲,默識擬張安世。」後為中書郎。時馥已為
 司隸校尉,遷鎮東將軍。簡自以才高而在馥之下,謂馥曰:「揚雄為郎,三世不徙,而王莽、董賢位列三司,古今一揆耳。」馥甚慚之。官至太子中庶子、散騎常侍。永嘉末,奔茍晞,與晞同沒。



 茍晞,字道將,河內山陽人也。少為司隸部從事,校尉石鑒深器之。東海王越為侍中,引為通事令史,累遷陽平太守。齊王冏輔政,晞參冏軍事,拜尚書右丞,轉左丞,廉察諸曹,八坐以下皆側目憚之。及冏誅,晞亦坐免。長沙王乂為驃騎將軍,以晞為從事中郎。惠帝征成都王穎,
 以為北軍中候。及帝還洛陽,晞奔范陽王虓,虓承制用晞行兗州刺史。



 汲桑之破鄴也,東海王越出次官渡以討之,命晞為前鋒。桑素憚之,於城外為柵以自守。晞將至,頓軍休士,先遣單騎示以禍福。桑眾大震,棄柵宵遁,嬰城固守。晞陷其九壘,遂定鄴而還。西討呂朗等,滅之。後高密王泰討青州賊劉根,破汲桑故將公師籓,敗石勒於河北,威名甚盛,時人擬之韓白。進位撫軍將軍、假節、都督青兗諸軍事,封東平郡侯,邑萬戶。



 晞練於官事,文簿盈積,斷決如流,人不敢欺。其從母依之,奉養甚厚。從母子求為將,晞距之曰:「吾不以王法貸人,將無後悔
 邪?」固欲之,晞乃以為督護。後犯法,晞杖節斬之,從母叩頭請救,不聽。既而素服哭之,流涕曰:「殺卿者兗州刺史,哭弟者茍道將。」其杖法如此。



 晞見朝政日亂,懼禍及己,而多所交結,每得珍物,即貽都下親貴。兗州去洛五百里,恐不鮮美,募得千里牛,每遣信,旦發暮還。



 初,東海王越以晞復其仇恥,甚德之,引升堂,結為兄弟。越司馬潘滔等說曰:「兗州要衝,魏武以之輔相漢室。茍晞有大志,非純臣,久令處之,則患生心腹矣。若遷於青州,厚其名號,晞必悅,公自牧兗州,經緯諸夏,籓衛本朝,此所謂謀之於未有,為之於未亂也。」越以為然,乃遷晞征東大將
 軍、開府儀同三司,加侍中、假節、都督青州諸軍事,領青州刺史,進為郡公。晞乃多置參佐,轉易守令,以嚴刻立功,日加斬戮,流血成川,人不堪命,號曰「屠伯」。頓丘太守魏植為流人所逼,眾五六萬,大掠兗州。晞出屯無鹽,以弟純領青州,刑殺更甚於晞,百姓號「小茍酷於大茍」。晞尋破植。



 時潘滔及尚書劉望等共誣陷晞,晞怒,表求滔等首,又請越從事中郎劉洽為軍司,越皆不許。晞於是昌言曰:「司馬元超為宰相不平,使天下淆亂,茍道將豈可以不義使之?韓信不忍衣食之惠,死於婦人之手。今將誅國賊,尊王室,桓文豈遠哉!」乃移告諸州,稱己功伐,
 陳越罪狀。



 時懷帝惡越專權,乃詔晞曰:「朕以不德,戎車屢興,上懼宗廟之累,下愍兆庶之困,當賴方嶽,為國籓翰。公威震赫然,梟斬籓、桑,走降喬、朗,魏植之徒復以誅除,豈非高識明斷,朕用委成。加王彌、石勒為社稷之憂,故有詔委統六州。而公謙分小節,稽違大命,非所謂與國同憂也。今復遣詔,便施檄六州,協同大舉,翦除國難,稱朕意焉。」晞復移諸征鎮州郡曰:「天步艱險,禍難殷流,劉元海造逆於汾陰,石世龍階亂於三魏,薦食畿甸,覆喪鄴都,結壘近郊,仍震兗豫,害三刺史,殺二都督,郡守官長,堙沒數十,百姓流離,肝腦塗地。晞以虛薄,負荷國
 重,是以弭節海隅,援枹曹衛。猥被中詔,委以關東,督統諸軍,欽承詔命。剋今月二日,當西經濟黎陽,即日得榮陽太守丁嶷白事,李惲、陳午等救懷諸軍與羯大戰,皆見破散。懷城已陷,河內太守裴整為賊所執。宿衛闕乏,天子蒙難,宗廟之危,甚於累卵。承問之日,憂嘆累息。晞以為先王選建明德,庸以服章,所以籓固王室,無俾城壞。是以舟楫不固,齊桓責楚;襄王逼狄,晉文致討。夫翼獎皇家,宣力本朝,雖陷湯火,大義所甘。加諸方牧,俱受榮寵,義同畢力,以報國恩。晞雖不武,首啟戎行,秣馬裹糧,以俟方鎮。凡我同盟,宜同赴救。顯立名節,在此行矣。」



 會王彌遣曹嶷破瑯邪,北攻齊地。茍純城守,嶷眾轉盛,連營數十里。晞還,登城望之,有懼色,與賊連戰,輒破之。後簡精銳,與賊大戰,會大風揚塵,遂敗績,棄城夜走。嶷追至東山,部眾皆降嶷。晞單騎奔高平,收邸閣,募得數千人。



 帝又密詔晞討越,晞復上表曰:「殿中校尉李初至,奉被手詔,肝心若裂。東海王越得以宗臣遂執朝政,委任邪佞,寵樹奸黨,至使前長史潘滔、從事中郎畢邈、主簿郭象等操弄天權,刑賞由己。尚書何綏、中書令繆播、太僕繆胤、黃門侍郎應紹,皆是聖詔親所抽拔,而滔等妄構,陷以重戮。帶甲臨宮,誅討后弟,翦除宿衛,私樹
 國人。崇獎魏植,招誘逋亡,覆喪州郡。王途圮隔,方貢乖絕,宗廟闕蒸嘗之饗,聖上有約食之匱。鎮東將軍周馥、豫州刺史馮嵩、前北中郎將裴憲,並以天朝空曠,權臣專制,事難之興,慮在旦夕,各率士馬,奉迎皇輿,思隆王室,以盡臣禮。而滔、邈等劫越出關,矯立行臺,逼徙公卿,擅為詔令,縱兵寇抄,茹食居人,交尸塞路,暴骨盈野。遂令方鎮失職,城邑蕭條,淮豫之萌,陷離塗炭。臣雖憤懣,守局東顒,自奉明詔,三軍奮厲,卷甲長驅,次于倉垣。即日承司空、博陵公浚書,稱殿中中郎劉權齎詔,敕浚與臣共剋大舉。輒遣前鋒征虜將軍王贊徑至項城,使越
 稽首歸政,斬送滔等。伏願陛下寬宥宗臣,聽越還國。其餘逼迫,宜蒙曠蕩。輒寫詔宣示征鎮,顯明義舉。遣揚烈將軍閻弘步騎五千,鎮衛宗廟。」



 五年,帝復詔晞曰:「太傅信用姦佞,阻兵專權,內不遵奉皇憲,外不協比方州,遂令戎狄充斥,所在犯暴。留軍何倫抄掠宮寺,劫剝公主,殺害賢士,悖亂天下,不可忍聞。雖惟親親,宜明九伐。詔至之日,其宣告天下,率齊大舉,桓文之績,一以委公。其思盡諸宜,善建弘略。道澀,故練寫副,手筆示意。」晞表曰:「奉被手詔,委臣征討,喻以桓文,紙練兼備,伏讀跪歎,五情惶怛。自頃宰臣專制,委杖佞邪,內擅朝威,外殘兆庶,
 矯詔專征,遂圖不軌,縱兵寇掠,陵踐宮寺。前司隸校尉劉暾、御史中丞溫畿、右將軍杜育,並見攻劫。廣平、武安公主,先帝遺體,咸被逼辱。逆節虐亂,莫此之甚。輒祗奉前詔,部分諸軍,遣王贊率陳午等將兵詣項,龔行天罰。」



 初,越疑晞與帝有謀,使游騎於成阜間,獲晞使,果得詔令及朝廷書,遂大構疑隙。越出牧豫州以討晞,復下檄說晞罪惡,遣從事中郎楊瑁為兗州,與徐州刺史裴盾共討晞。晞使騎收河南尹潘滔,滔夜遁,及執尚書劉會、侍中程延,斬之。會越薨,盾敗,詔晞為大將軍大都督、督青徐兗豫荊揚六州諸軍事,增邑二萬戶,加黃鉞,先官
 如故。



 晞以京邑荒饉日甚,寇難交至,表請遷都,遣從事中郎劉會領船數十艘,宿衛五百人,獻穀千斛以迎帝。朝臣多有異同。俄而京師陷,晞與王贊屯倉垣。豫章王端及和郁等東奔晞,晞群官尊端為皇太子,置行臺。端承制以晞領太子太傅、都督中外諸軍、錄尚書,自倉垣徙屯蒙城,贊屯陽夏。



 晞出於孤微,位至上將,志頗盈滿,奴婢將千人,侍妾數十,終日累夜不出戶庭,刑政苛虐,縱情肆欲。遼西閻亨以書固諫,晞怒,殺之。晞從事中郎明預有疾居家,聞之,乃舉病諫晞曰:「皇晉遭百六之數,當危難之機,明公親稟廟算,將為國家除暴。閻亨美
 士,奈何無罪一旦殺之!」晞怒白;「我自殺閻亨,何關人事,而舉病來罵我!」左右為之戰慄,預曰:「以明公以禮見進,預欲以禮自盡。今明公怒預,其若遠近怒明公何!昔堯舜之在上也,以和理而興;桀紂之在上也,以惡逆而滅。天子且猶如此,況人臣乎!願明公且置其怒而思預之言。」晞有慚色。由是眾心稍離,莫為致用,加以疾疫饑饉,其將溫畿、傅宣皆叛之。石勒攻陽夏,滅王贊,馳襲蒙城,執晞,署為司馬,月餘乃殺之。晞無子,弟純亦遇害。



 華軼,字彥夏,平原人,魏太尉歆之曾孫也。祖表,太中大
 夫。父澹,河南尹。軼少有才氣,聞於當世,汎愛博納,眾論美之。初為博士,累遷散騎常侍。東海王越牧兗州,引為留府長史。永嘉中,歷振威將軍、江州刺史。雖逢喪亂,每崇典禮,置儒林祭酒以弘道訓,乃下教曰:「今大義頹替,禮典無宗,朝廷滯議,莫能攸正,常以慨然,宜特立此官,以弘其事。軍諮祭酒杜夷,棲情玄遠,確然絕俗,才學精博,道行優備,其以為儒林祭酒。」俄被越檄使助討諸賊,軼遣前江夏太守陶侃為揚武將軍,率兵三千屯夏口,以為聲援。軼在州其有威惠,州之豪士接以友道,得江表之歡心,流亡之士赴之如歸。



 時天子孤危,四方瓦解,
 軼有匡天下之志,每遣貢獻入洛,不失臣節。謂使者曰:「若洛都道斷,可輸之瑯邪王,以明吾之為司馬氏也。」軼自以受洛京所遣,而為壽春所督,時洛京尚存,不能祗承元帝教命,郡縣多諫之,軼不納,曰:「吾欲見詔書耳。」時帝遣揚烈將軍周訪率眾屯彭澤以備軼,訪過姑孰,著作郎干寶見而問之,訪曰:「大府受分,令屯彭澤,彭澤,江州西門也。華彥夏有憂天下之誠,而不欲碌碌受人控御,頃來紛紜,粗有嫌隙。今又無故以兵守其門,將成其釁。吾當屯尋陽故縣,既在江西,可以扞禦北方,又無嫌於相逼也。」尋洛都不守,司空荀籓移檄,而以帝為盟主。
 既而帝承制改易長吏,軼又不從命,於是遣左將軍王敦都督甘卓、周訪、宋典、趙誘等討之。軼遣別駕陳雄屯彭澤以距敦,自為舟軍以為外援。武昌太守馮逸次于湓口,訪擊逸,破之。前江州刺史衛展不為軼所禮,心常怏怏。至是,與豫章太守周廣為內應,潛軍襲軼,軼眾潰,奔于安城,追斬之,及其五子,傳首建鄴。



 初,廣陵高悝寓居江州,軼避為西曹掾,尋而軼敗,悝藏匿軼二子及妻,崎嶇經年。既而遇赦,悝攜之出首,帝嘉而宥之。



 劉喬,字仲彥,南陽人也。其先漢宗室,封安眾侯,傳襲歷
 三代。祖暠,魏侍中。父阜,陳留相。喬少為祕書郎,建威將軍王戎引為參軍。伐吳之役,戎使喬與參軍羅尚濟江,破武昌,還授滎陽令,遷太子洗馬。以誅楊駿功,賜爵關中侯,拜尚書右丞。豫誅賈謐,封安眾男,累遷散騎常侍。



 齊王冏為大司馬,初,嵇紹為冏所重,每下階迎之。喬言於冏曰;「裴、張之誅,朝臣畏憚孫秀,故不敢不受財物。嵇紹今何所逼忌,故畜裴家車牛、張家奴婢邪?樂彥輔來,公未嘗下床,何獨加敬於紹?」冏乃止。紹謂喬曰:「大司馬何故不復迎客?」。喬曰:「似有正人言,以卿不足迎者。」紹曰:「正人為誰?」喬曰:「其則不遠。」紹默然。頃之,遷御史中丞。冏
 腹心董艾勢傾朝廷,百僚莫敢忤旨。喬二旬之中,奏劾艾罪釁者六。艾諷尚書右丞茍晞免喬官,復為屯騎校尉。張昌之亂,喬出為威遠將軍、豫州刺史,與荊州刺史劉弘共討昌,進左將軍。



 惠帝西幸長安,喬與諸州郡舉兵迎大駕。東海王越承制轉喬安北將軍、冀州刺史,以范陽王虓領豫州刺史。喬以虓非天子命,不受代,發兵距之。潁川太守劉輿暱於虓,喬上尚書列輿罪惡。河間王顒得喬所上,乃宣詔使鎮南將軍劉弘、征東大將軍劉準、平南將軍彭城王釋與喬并力攻虓於許昌。輿弟琨率眾救虓,未至而虓敗,虓乃與琨俱奔河北。未幾,琨
 率突騎五千濟河攻喬,喬劫琨父蕃,以檻車載之,據考城以距虓,眾不敵而潰。



 喬復收散卒,屯於平氏,河間王顒進喬鎮東將軍、假節,以其長子祐為東郡太守,又遣劉弘、劉準、彭城王釋等率兵援喬。弘與喬箋曰:「適承范陽欲代明使君。明使君受命本朝,列居方伯,當官而行,同獎王室,橫見遷代,誠為不允。然古人有言,牽牛以蹊人之田,信有罪矣,而奪之牛,罰亦重矣。明使君不忍亮直狷介之忿,甘為戎首,竊以為過。何者?至人之道,用行舍藏。跨下之辱,猶宜俯就,況於換代之嫌,纖介之釁哉!范陽國屬,使君庶姓,周之宗盟,疏不間親,曲直既均,責
 有所在。廉藺區區戰國之將,猶能升降以利社稷,況命世之士哉!今天下紛紜,主上播越,正是忠臣義士同心戮力之時。弘實暗劣,過蒙國恩,願與使君共戴盟主,鴈行下風,掃除凶寇,救蒼生之倒懸,反北辰於太極。此功未立,不宜乖離。備蒙顧遇,情隆於常,披露丹誠,不敢不盡。春秋之時,諸侯相伐,復為和親者多矣。願明使君迴既往之恨,追不二之蹤,解連環之結,修如初之好。范陽亦將悔前之失,思崇後信矣。



 東海王越將討喬,弘又與越書曰:「適聞以吾州將擅舉兵逐范陽,當討之,誠明同異、懲禍亂之宜。然吾竊謂不可。何者?今北辰遷居,元首
 移幸,群后抗義以謀王室,吾州將荷國重恩,列位方伯,亦伐鼓即戎,戮力致命之秋也。而范陽代之,吾州將不從,由代之不允,但矯枉過正,更以為罪耳。昔齊桓赦射鉤之仇而相管仲,晉文忘斬祛之怨而親勃鞮,方之於今,當何有哉!且君子躬自厚而薄責於人,今奸臣弄權,朝廷困逼,此四海之所危懼,宜釋私嫌,共存公義,含垢匿瑕,忍所難忍,以大逆為先,奉迎為急,不可思小怨忘大德也。茍崇忠恕,共明分局,連旗推鋒,各致臣節,吾州將必輸寫肝膽,以報所蒙,實不足計一朝之謬,發赫然之怒,使韓盧東郭相困而為豺狼之擒也。吾雖庶姓,負
 乘過分,實願足下率齊內外,以康王室,竊恥同儕自為蠹害。貪獻所懷,惟足下圖之。」又上表曰:「范陽王虓欲代豫州刺史喬,喬舉兵逐虓,司空、東海王越以喬不從命討之。臣以為喬忝受殊恩,顯居州司,自欲立功於時,以徇國難,無他罪闕,而范陽代之,代之為非。然喬亦不得以虓之非,專威輒討,誠應顯戮以懲不恪。然自頃兵戈紛亂,猜禍鋒生,恐疑隙構於群王,災難延于宗子,權柄隆於朝廷,逆順效於成敗,今夕為忠,明旦為逆,翩其反而,互為戎首,載籍以來,骨肉之禍未有如今者也。臣竊悲之,痛心疾首。今邊陲無備豫之儲,中華有杼軸之困,
 而股肱之臣不惟國體,職競尋常,自相楚剝,為害轉深,積毀銷骨。萬一四夷乘虛為變,此亦猛獸交鬥,自效於卞莊者矣。臣以為宜速發明詔,詔越等令兩釋猜嫌,各保分局。自今以後,其有不被詔書擅興兵馬者,天下共伐之。《詩》云:『誰能執熱,逝不以濯?』若誠濯之,必無灼爛之患,永有泰山之固矣。」



 時河間王顒方距關東,倚喬為助,不納其言。東海王越移檄天下,帥甲士三萬,將入關迎大駕,軍次於蕭,喬懼,遣子祐距越於蕭縣之靈壁。劉琨分兵向許昌,許昌人納之。琨自滎陽率兵迎越,遇祐,眾潰見殺。喬眾遂散,與五百騎奔平氏。帝還洛陽,大赦,越
 復表喬為太傅軍諮祭酒。越薨,復以喬為都督豫州諸軍事、鎮東將軍、豫州刺史。卒於官,時年六十三。愍帝末,追贈司空。子挺,潁川太守。挺子耽。



 耽字敬道。少有行檢,以義尚流稱,為宗族所推。博學,明習《詩》、《禮》、三史。歷度支尚書,加散騎常侍。在職公平廉慎,所蒞著績。桓玄,耽女婿也。及玄輔政,以耽為尚書令,加侍中,不拜,改授特進、金紫光祿大夫。尋卒,追贈左光祿大夫、開府。耽子柳。



 柳字叔惠,亦有名譽。少登清官,歷尚書左右僕射。時右丞傅迪好廣讀書而不解其義,柳唯讀《老子》而已,迪每
 輕之。柳云:「卿讀書雖多,而無所解,可謂書簏矣。」時人重其言。出為徐、兗、江三州刺史。卒,贈右光祿大夫、開府儀同三司。喬弟乂,始安太守。乂子成,丹陽尹。



 史臣曰:周浚人倫鑒悟,周馥理識精詳,華軼動顧禮經,劉喬志存諒直,用能歷官內外,咸著勳庸。而祖宣獻策遷都,乖忤於東海,彥夏係心宸極,獲罪於瑯邪,乃被以惡名,加其顯戮,豈不哀哉!向若違左衣任於伊川,建右社於淮服,據方城之險,藉全楚之資,簡練吳越之兵,漕引淮海之粟,縱未能祈天永命,猶足以紓難緩亡。嗟乎!「不用其良,覆俾我悖」,其此之謂也。茍晞擢自庸微,位居上
 將,釋位之功未立,貪暴之釁已彰,假手世龍,以至屠戮,斯所謂「殺人多矣,能無及此乎」!



 贊曰:開林才理,爰登貴仕,績著折沖,化行江汜。軼既尊主,馥亦勤王,背時獲戾,違天不祥。喬為戎首,未識行藏。道將鞠旅,威名克舉,領虐有聞,忠勤未取。



\end{pinyinscope}