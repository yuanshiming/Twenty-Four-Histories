\article{載記第四 石勒上}

\begin{pinyinscope}
石勒
 \gezhu{
  上}



 石勒字世龍,初名㔨,上黨武鄉羯人也。其先匈奴別部羌渠之胄。祖耶奕于,父周曷朱,一名乞冀加,並為部落小率。勒生時赤光滿室,白氣自天屬於中庭,見者咸異之。年十四,隨邑人行販洛陽,倚嘯上東門,王衍見而異之,顧謂左右曰:「向者胡雛,吾觀其聲視有奇志,恐將為天下之患。」馳遣收之,會勒已去。長而壯健有膽力,雄武
 好騎射。曷朱性凶粗,不為群胡所附,每使勒代己督攝,部胡愛信之。所居武鄉北原山下草木皆有鐵騎之象,家園中生人參,花葉甚茂,悉成人狀。父老及相者皆曰:「此胡狀貌奇異,志度非常,其終不可量也。」勸邑人厚遇之。時多嗤笑,唯鄔人郭敬、陽曲寧驅以為信然,並加資贍。勒亦感其恩,為之力耕。每聞鞞鐸之音,歸以告其母,母曰:「作勞耳鳴,非不祥也。」



 太安中,并州饑亂,勒與諸小胡亡散,乃自鴈門還依寧驅。北澤都尉劉監欲縛賣之,驅匿之,獲免。勒於是潛詣納降都尉李川,路逢郭敬,泣拜言饑寒。敬對之流涕,以帶貨鬻食之,并給以衣服。勒
 謂敬曰:「今者大餓,不可守窮。諸胡饑甚,宜誘將冀州就穀,因執賣之,可以兩濟。」敬深然之。會建威將軍閻粹說并州刺史、東嬴公騰執諸胡於山東賣充軍實,騰使將軍郭陽、張隆虜群胡將詣冀州,兩胡一枷。勒時年二十餘,亦在其中,數為隆所驅辱。敬先以勒屬郭陽及兄子時,陽,敬族兄也,是以陽、時每為解請,道路饑病,賴陽、時而濟。既而賣與茌平人師懽為奴。有一老父謂勒曰:「君魚龍髮際上四道已成,當貴為人主。甲戌之歲,王彭祖可圖。」勒曰:「若如公言,弗敢忘德。」忽然不見。每耕作於野,常聞鼓角之聲。勒以告諸奴,諸奴亦聞之,因曰:「吾幼來
 在家恒聞如是。」諸奴歸以告懽,懽亦奇其狀貌而免之。



 歡家鄰於馬牧,與牧率魏郡汲桑往來,勒以能相馬自託於桑。嘗傭於武安臨水,為遊軍所囚。會有群鹿旁過,軍人競逐之,勒乃獲免。俄而又見一父老,謂勒曰:「向群鹿者我也,君應為中州主,故相救爾。」勒拜而受命。遂招集王陽、夔安、支雄、冀保、吳豫、劉膺、桃豹、逯明等八騎為群盜。後郭敖、劉徵、劉寶、張曀僕、呼延莫、郭黑略、張越、孔豚、趙鹿、支屈六等又赴之,號為十八騎。復東如赤龍、驥諸苑中,乘苑馬遠掠繒寶,以賂汲桑。



 及成都王穎敗乘輿于蕩陰,逼帝如鄴宮,王浚以穎陵辱天子,使鮮卑
 擊之,穎懼,挾惠帝南奔洛陽。帝復為張方所逼,遷于長安。關東所在兵起,皆以誅穎為名。河間王顒懼東師之盛,欲輯懷東復,乃奏議廢穎。是歲,劉元海稱漢王于黎亭,穎故將陽平人公師籓等自稱將軍,起兵趙魏,眾至數萬。勒與汲桑帥牧人乘苑馬數百騎以赴之。桑始命勒以石為姓,勒為名焉。籓拜勒為前隊督,從攻平昌公模於鄴。模使將軍馮嵩逆戰,敗之。籓濟自白馬而南,濮陽太守茍晞討籓斬之。勒與桑亡潛苑中,桑以勒為伏夜牙門,帥牧人劫掠郡縣繫囚,又招山澤亡命,多附勒,勒率以應之。桑乃自號大將軍,稱為成都王穎誅東海
 王越、東嬴公騰為名。桑以勒為前驅,屢有戰功,署為掃虜將軍、忠明亭侯。桑進軍攻鄴,以勒為前鋒都督,大敗騰將馮嵩,因長驅入鄴,遂害騰,殺萬餘人,掠婦女珍寶而去。濟自延津,南擊兗州,越大懼,使茍晞、王贊等討之。



 桑、勒攻幽州刺史石鮮於樂陵,鮮死之。乞活田禋帥眾五萬救鮮,勒逆戰,敗禋,與晞等相持于平原、陽平間數月,大小三十餘戰,互有勝負。越懼,次於官渡,為晞聲援。桑、勒為晞所敗,死者萬餘人,乃收餘眾,將奔劉元海。冀州刺史丁紹要之于赤橋,又大敗之。桑奔馬牧,勒奔樂平。王師斬桑於平原。



 時胡部大張㔨督、馮莫突等擁眾
 數千,壁于上黨,勒往從之,深為所暱,因說㔨督曰:「劉單于舉兵誅晉,部大距而不從,豈能獨立乎?」曰:「不能。」勒曰:「如其不能者,兵馬當有所屬。今部落皆已被單于賞募,往往聚議欲叛部大而歸單于矣,宜早為之計。」㔨督等素無智略,懼部眾之貳己也,乃潛隨勒單騎歸元海。元海署㔨督親漢王,莫突為都督部大,以勒為輔漢將軍、平晉王以統之。勒於是命㔨督為兄,賜姓石氏,名之曰會,言其遇己也。



 烏丸張伏利度亦有眾二千,壁于樂平,元海屢招而不能致。勒偽獲罪于元海,因奔伏利度。伏利度大悅,結為兄弟,使勒率諸胡寇掠,所向無前,諸
 胡畏服。勒知眾心之附己也,乃因會執伏利度,告諸胡曰:「今起大事,我與伏利度孰堪為主?」諸胡咸以推勒。勒於是釋伏利度,率其部眾歸元海。元海加勒督山東征討諸軍事,以伏利度眾配之。



 元海使劉聰攻壺關,命勒率所統七千為前鋒都督。劉琨遣護軍黃秀等救壺關,勒敗秀於白田,秀死之,勒遂陷壺關。元海命勒與劉零、閻羆等七將率眾三萬寇魏郡、頓丘諸壘壁,多陷之,假壘主將軍、都尉,簡強壯五萬為軍士,老弱安堵如故,軍無私掠,百姓懷之。



 及元海僭號,遣使授勒持節、平東大將軍,校尉、都督、王如故。勒并軍寇鄴,鄴潰,和郁奔于衛
 國。執魏郡太守王粹于三臺。進攻趙郡,害冀州西部都尉馮沖。攻乞活赦亭、田禋於中丘,皆殺之。元海授勒安東大將軍、開府,置左右長史、司馬、從事中郎。進軍攻鉅鹿、常山,害二郡守將。陷冀州郡縣堡壁百餘,眾至十餘萬,其衣冠人物集為君子營。乃引張賓為謀主,始署軍功曹,以刁膺、張敬為股肱,夔安、孔萇為爪牙,支雄、呼延莫、王陽、桃豹、逯明、吳豫等為將率。使其將張斯率騎詣并州山北諸郡縣,說諸胡羯,曉以安危。諸胡懼勒威名,多有附者。進軍常山,分遣諸將攻中山、博陵、高陽諸縣,降之者數萬人。



 王浚使其將祁弘帥鮮卑段務塵等十
 餘萬騎討勒,大敗勒于飛龍山,死者萬餘。勒退屯黎陽,分命諸將攻諸未下及叛者,降三十餘壁,置守宰以撫之。進寇信都,害冀州刺史王斌。於是車騎將軍王堪、北中郎將裴憲自洛陽率眾討勒,勒燒營并糧,迴軍距之,次于黃牛壘。魏郡太守劉矩以郡附于勒,勒使矩統其壘眾為中軍左翼。勒至黎陽,裴憲棄其軍奔於淮南,王堪退堡倉垣。元海授勒鎮東大將軍,封汲郡公,持節、都督、王如故。勒固讓公不受。與閻羆攻者圈、苑市二壘,陷之,羆中流矢死,勒并統其眾,潛自石橋濟河,攻陷白馬,坑男女三千餘口。東襲鄄城,害兗州刺史袁孚。因攻倉
 垣,陷之,遂害堪。渡河攻廣宗、清河、平原、陽平諸縣,降勒者九萬餘口。復南濟河,滎陽太守裴純奔於建業。



 時劉聰攻河內,勒率騎會之,攻冠軍將軍梁巨于武德,懷帝遣兵救之。勒留諸將守武德,與王桑逆巨於長陵。巨請降,勒弗許,巨踰城而遁,軍人執之。勒馳如武德,坑降卒萬餘,數梁巨罪而害之。王師退還,河北諸堡壁大震,皆請降送任於勒。



 及元海死,劉聰授勒征東大將軍、并州刺史、汲郡公,持節、開府、都督、校尉、王如故。勒固辭將軍,乃止。



 劉粲率眾四萬寇洛陽,勒留輜重于重門,率騎二萬會粲於大陽,大敗王師於澠池,遂至洛川。粲出轘轅,
 勒出成皋關,圍陳留太守王讚於倉垣,為贊所敗,退屯文石津。將北攻王浚,會浚將王甲始率遼西鮮卑萬餘騎敗趙固於津北,勒乃燒船棄營,引軍向柏門,迎重門輜重,至于石門,濟河,攻襄城太守崔曠於繁昌,害之。



 先是,雍州流人王如、侯脫、嚴嶷等起兵江淮間,聞勒之來也,懼,遣眾一萬屯襄城以距,勒擊敗之,盡俘其眾。勒至南陽,屯于宛北山。如懼勒之攻襄也,使送珍寶車馬犒師,結為兄弟,勒納之。如與侯脫不平,說勒攻脫。勒夜令三軍雞鳴而駕,晨壓宛門,攻之,旬有二日而剋。嚴嶷率眾救脫,至則無及,遂降于勒。勒斬脫,囚嶷送於平陽,盡
 並其眾,軍勢彌盛。



 勒南寇襄陽,攻陷江西壘壁三十餘所,留刁膺守襄陽,躬帥精騎三萬還攻王如。憚如之盛,遂趣襄城。如知之,遣弟璃率騎二萬五千,詐言犒軍,實欲襲勒。勒逆擊,滅之,復屯江西,蓋欲有雄據江漢之志也。張賓以為不可,勸勒北還,弗從,以賓為參軍都尉,領記室,位次司馬,專居中總事。



 元帝慮勒南寇,使王導率眾討勒。勒軍糧不接,死疫太半,納張賓之策,乃焚輜重,裹糧卷甲,渡沔,寇江夏,太守楊岠棄郡而走。北寇新蔡,害新蔡王確于南頓,朗陵公何襲、廣陵公陳、上黨太守羊綜、廣平太守邵肇等率眾降于勒。勒進陷許昌,害
 平東將軍王康。



 先是,東海王越率洛陽之眾二十餘萬討勒,越薨於軍,眾推太尉王衍為主,率眾東下,勒輕騎追及之。衍遣將軍錢端與勒戰,為勒所敗,端死之,衍軍大潰,勒分騎圍而射之,相登如山,無一免者。於是執衍及襄陽王範、任城王濟、西河王喜、梁王禧、齊王超、吏部尚書劉望、豫州刺名劉喬、太傅長史庾顗等,坐之於幕下,問以晉故。衍、濟等懼死,多自陳說,惟範神色儼然,意氣自若,顧呵之曰:「今日之事,何復紛紜!」勒甚奇之。勒於是引諸王公卿士於外害之,死者甚眾。勒重衍清辨,奇範神氣,不能加之兵刃,夜使人排墻填殺之。左衛何倫、
 右衛李惲聞越薨,奉越妃裴氏及越世子毗出自洛陽。勒逆毗於洧倉,軍復大潰,執毗及諸王公卿士,皆害之,死者甚眾。因率精騎三萬,入自成皋關。會劉曜、王彌寇洛陽,洛陽既陷,勒歸功彌、曜,遂出轘轅,屯於許昌。劉聰署勒征東大將軍,勒固辭不受。



 先是,平陽人李洪有眾數千,壘於舞陽,茍晞假洪雍州刺史。勒進寇穀陽,害冠軍將軍王茲。破王贊於陽夏,獲贊,以為從事中郎。襲破大將軍茍晞于蒙城,執晞,署為左司馬。劉聰授勒征東大將軍、幽州牧,固辭將軍不受。



 先是,王彌納劉暾之說,將先誅勒,東王青州,使暾徵其將曹嶷於齊。勒遊騎獲
 暾,得彌所與嶷書,勒殺之,密有圖彌之計矣。會彌將徐邈輒引部兵去彌,彌漸削弱。及勒之獲茍晞也,彌惡之,偽卑辭使謂勒曰:「公獲茍晞而赦之,何其神也!使晞為公左,彌為公右,天下不足定。」勒謂張賓曰:「王彌位重言卑,恐其遂成前狗意也。」賓曰:「觀王公有青州之心,桑梓本邦,固人情之所樂,明公獨無并州之思乎?王公遲迴未發者,懼明公踵其後,已有規明公之志,但未獲便爾。今不圖之,恐曹嶷復至,共為羽翼,後雖欲悔,何所及邪!徐邈既去,軍勢稍弱,觀其控御之懷猶盛,可誘而滅之。」勒以為然。勒時與陳午相攻於蓬關,王彌亦與劉瑞相
 持甚急。彌請救於勒,勒未之許。張賓進曰:「明公常恐不得王公之便,今天以其便授我矣。陳午小豎,何能為寇?王彌人傑,將為我害。」勒因迴軍擊瑞,斬之。彌大悅,謂勒深心推奉,無復疑也。勒引師攻陳午于肥澤,午司馬上黨李頭說勒曰:「公天生神武,當平定四海,四海士庶皆仰屬明公,望濟於塗炭。有與公爭天下者,公不早圖之,而返攻我曹流人。我曹鄉黨,終當奉戴,何遽見逼乎!」勒心然之,詰朝引退。詭請王彌宴于已吾,彌長史張嵩諫彌勿就,恐有專諸、孫峻之禍,彌不從。既入,酒酣,勒手斬彌而并其眾,啟聰稱彌叛逆之狀。聰署勒鎮東大將軍、
 督并幽二州軍事、領并州刺史,持節、征討都督、校尉、開府、幽州牧、公如故。



 茍晞、王讚謀叛勒,勒害之。以將軍左伏肅為前鋒都尉,攻掠豫州諸郡,臨江而還,屯于葛陂,降諸夷楚,署將軍二千石以下,稅其義穀,以供軍士。



 初,勒被鬻平原,與母王相失。至是,劉琨遣張儒送王於勒,遺勒書曰:「將軍發迹河朔,席卷兗豫,飲馬江淮,折衝漢沔,雖自古名將,未足為諭。所以攻城而不有其人,略地而不有其土,翕爾雲合,忽復星散,將軍豈知其然哉?存亡決在得主,成敗要在所附;得主則為義兵,附逆則為賊眾。義兵雖敗,而功業必成;賊眾雖剋,而終歸殄滅。
 昔赤眉、黃巾橫逆宇宙,所以一旦敗亡者,正以兵出無名,聚而為亂。將軍以天挺之質,威振宇內,擇有德而推崇,隨時望而歸之,勛義堂堂,長享遐貴。背聰則禍除,向主則福至。採納往誨,翻然改圖,天下不足定,蟻寇不足掃。今相授侍中、持節、車騎大將軍、領護匈奴中郎將、襄城郡公,總內外之任,兼華戎之號,顯封大郡,以表殊能,將軍其受之,副遠近之望也。自古以來誠無戎人而為帝王者,至於名臣建功業者,則有之矣。今之遲想,蓋以天下大亂,當須雄才。遙聞將軍攻城野戰,合於機神,雖不視兵書,暗與孫吳同契,所謂生而知之者上,學而知
 之者次。但得精騎五千,以將軍之才,何向不摧!至心實事,皆張儒所具。」勒報琨曰:「事功殊途,非腐儒所聞。君當逞節本朝,吾自夷,難為效。」遺琨名馬珍寶,厚賓其使,謝歸以絕之。



 勒於葛陂繕室宇,課農造舟,將寇建鄴。會霖雨歷三月不止,元帝使諸將率江南之眾大集壽春,勒軍中飢疫死者太半。檄書朝夕繼至,勒會諸將計之。右長史刁膺諫勒先送款於帝,求掃平河朔,待軍退之後徐更計之。勒愀然長嘯。中堅夔安勸勒就高避水,勒曰:「將軍何其怯乎!」孔萇、支雄等三十餘將進曰:「及吳軍未集,萇等請各將三百步卒,乘船三十餘道,夜登其城,斬
 吳將頭,得其城,食其倉米。今年要當破丹陽,定江南,盡生縛取司馬家兒輩。」勒笑曰:「是勇將之計也。」各賜鎧馬一匹。顧問張賓曰:「於君計何如?」賓曰:「將軍攻陷帝都,囚執天子,殺害王侯,妻略妃主,擢將軍之髮不足以數將軍之罪,奈何復還相臣奉乎!去年誅王彌之後,不宜於此營建。天降霖雨方數百里中,示將軍不應留也。鄴有三臺之固,西接平陽,四塞山河,有喉衿之勢,宜北徙據之。伐叛懷服,河朔既定,莫有處將軍之右者。晉之保壽春,懼將軍之往擊爾,今卒聞迴軍,必欣於敵去,未遑奇兵掎擊也。輜重逕從北道,大軍向壽春,輜重既過,大軍
 徐迴,何懼進退無地乎!」勒攘袂鼓髯曰:「賓之計是也。」責刁膺曰:「君共相輔佐,當規成功業,如何便相勸降!此計應斬。然相明性怯,所以宥君。」於是退膺為將軍,擢賓為右長史,加中壘將軍,號曰「右侯」。



 發自葛陂,遣石季龍率騎二千距壽春。會江南運船至,獲米布數十艘,將士爭之,不設備。晉伏兵大發,敗季龍于巨靈口,赴水死者五百餘人,奔退百里,及于勒軍。軍中震擾,謂王師大至,勒陣以待之。晉懼有伏兵,退還壽春。勒所過路次,皆堅壁清野,採掠無所獲,軍中大飢,士眾相食。行達東燕,聞汲郡向冰有眾數千,壁於枋頭,勒將於棘津北渡,懼冰邀
 之,會諸將問計。張賓進曰:「如聞冰船盡在瀆中,未上枋內,可簡壯勇者千人,詭道潛渡,襲取其船,以濟大軍。大軍既濟,冰必可擒也。」勒從之,使支雄、孔萇等從文石津縛筏潛渡,勒引其眾自酸棗向棘津。冰聞勒軍至,始欲內其船。會雄等已渡,屯其壘門,下船三十餘艘以濟其軍,令主簿鮮于豐挑戰,設三伏以待之。冰怒,乃出軍,將戰,而三伏齊發,夾擊攻之,又因其資,軍遂豐振。長驅寇鄴,攻北中郎將劉演於三臺。演部將臨深、牟穆等率眾數萬降于勒。



 時諸將佐議欲攻取三臺以據之,張賓進曰:「劉演眾猶數千,三臺險固,攻守未可卒下,舍之則能
 自潰。王彭祖、劉越石大敵也,宜及其未有備,密規進據罕城,廣運糧儲,西稟平陽,掃定并薊,桓文之業可以濟也。且今天下鼎沸,戰爭方始,遊行羈旅,人無定志,難以保萬全、制天下也。夫得地者昌,失地者亡。邯鄲、襄國,趙之舊都,依山憑險,形勝之國,可擇此二邑而都之,然後命將四出,授以奇略,推亡固存,兼弱攻昧,則群凶可除,王業可圖矣。」勒曰:「右侯之計是也。」於是進據襄國。賓又言於勒曰:「今我都此,越石、彭祖深所忌也,恐及吾城池未固,資儲未廣,送死於我。聞廣平諸縣秋稼大成,可分遣諸將收掠野穀。遣使平陽,陳宜鎮此之意。」勒又然之。
 於是上表於劉聰,分命諸將攻冀州郡縣壘壁,率多降附,運糧以輸勒。劉聰署勒使持節、散騎常侍、都督冀幽並營四州雜夷、征討諸軍事、冀州牧,進封本國上黨郡公,邑五萬戶,開府、幽州牧、東夷校尉如故。



 廣平遊綸、張豺擁眾數萬,受王浚假署,保據苑鄉。勒使夔安、支雄等七將攻之,破其外壘。浚遣督護王昌及鮮卑段就六眷、末柸、匹磾等部眾五萬餘以討勒。時城隍未修,乃於襄國築隔城重柵,設鄣以待之。就六眷屯於渚陽,勒分遣諸將連出挑戰,頻為就六眷所敗,又聞其大造攻具,勒顧謂其將佐曰:「今寇來轉逼,彼眾我寡,恐攻圍不解,外
 救不至,內糧罄絕,縱孫吳重生,亦不能固也。吾將簡練將士,大陣於野以決之,何如?」諸將皆曰:「宜固守以疲寇,彼師老自退,追而擊之,蔑不剋矣。」勒顧謂張賓、孔萇曰:「君以為何如」賓、萇俱曰:「聞就六眷剋來月上旬送死北城,其大眾遠來,戰守連日,以我軍勢寡弱,謂不敢出戰,意必懈怠。今段氏種眾之悍,末柸尤最,其卒之精勇,悉在末柸所,可勿復出戰,示之以弱。速鑿北壘為突門二十餘道,候賊列守未定,出其不意,直衝末柸帳,敵必震惶,計不及設,所謂迅雷不及掩耳。末柸之眾既奔,餘自摧散。擒末柸之後,彭祖可指辰而定。」勒笑而納之,即以
 萇為攻戰都督,造突門於北城。鮮卑入屯北壘,勒候其陣未定,躬率將士鼓噪於城上。會孔萇督諸突門伏兵俱出擊之,生擒末柸,就六眷等眾遂奔散。萇乘勝追擊,枕尸三十餘里,獲鎧馬五千匹。就六眷收其遺眾,屯于渚陽,遣使求和,送鎧馬金銀,并以末柸三弟為質而請末柸。諸將並勸勒殺末柸以挫之,勒曰:「遼西鮮卑,健國也,與我素無怨讎,為王浚所使耳。今殺一人,結怨一國,非計也。放之必悅,不復為王浚用矣。」於是納其質,遣石季龍盟就六眷於渚陽,結為兄弟,就六眷等引還。使參軍閻綜獻捷於劉聰。於是游綸、張豺請降稱籓,勒將襲幽
 州,務養將士,權宜許之,皆就署將軍。於是遣眾寇信都,害冀州刺史王象。王浚復以邵舉行冀州刺史,保於信都。



 建興元年,石季龍攻鄴三臺,鄴潰,劉演奔於稟丘,將軍謝胥、田青、郎牧等率三臺流人降于勒,勒以桃豹為魏郡太守以撫之。命段末柸為子,署為使持節、安北將軍、北平公,遣還遼西。末柸感勒厚恩,在途日南面而拜者三,段氏遂專心歸附,自是王浚威勢漸衰。



 勒襲苑鄉,執游綸以為主簿。攻乞活李惲于上白,斬之,將坑其降卒,見郭敬而識之,曰:「汝郭季子乎?」敬叩頭曰:「是也。」勒下馬執其手,泣曰:「今日相遇,豈非天邪!」賜衣服車馬,署敬
 上將軍,悉免降者以配之。其將孔萇寇定陵,害兗州刺史田征。烏丸薄盛執渤海太守劉既,率戶五千降于勒。劉聰授勒侍中、征東大將軍,餘如故,拜其母王氏為上黨國太夫人,妻劉氏上黨國夫人,章綬首飾一同王妃。



 段末柸任弟亡歸遼西,勒大怒,所經令尉皆殺之。



 烏丸審廣、漸裳、郝襲背王浚,密遣使降于勒,勒厚加撫納。司冀漸寧,人始租賦。立太學,簡明經善書吏署為文學掾,選將佐子弟三百人教之。勒母王氏死,潛窆山谷,莫詳其所。既而備九命之禮,虛葬於襄國城南。



 勒謂張賓曰:「鄴,魏之舊都,吾將營建。既風俗殷雜,須賢望以綏之,誰
 可任也?」賓曰:「晉故東萊太守南陽趙彭忠亮篤敏,有佐時良幹,將軍若任之,必能允副神規。」勒於是征彭,署為魏郡太守。彭至,入泣而辭曰:「臣往策名晉室,食其祿矣。犬馬戀主,切不敢忘。誠知晉之宗廟鞠為茂草,亦猶洪川東逝,往而不還。明公應符受命,可謂攀龍之會。但受人之榮,復事二姓,臣志所不為,恐亦明公之所不許。若賜臣餘年、全臣一介之願者,明公大造之惠也。」勒默然。張賓進曰:「自將軍神旗所經,衣冠之士靡不變節,未有能以大義進退者。至如此賢,以將軍為高祖,自擬為四公,所謂君臣相知,此亦足成將軍不世之高,何必吏之。」
 勒大悅,曰:「右侯之言得孤心矣。」於是賜安車駟馬,養以卿祿,辟其子明為參軍。勒以石季龍為魏郡太守,鎮鄴三臺,季龍篡奪之萌兆于此矣。



 時王浚署置百官,奢縱淫虐,勒有吞并之意,欲先遣使以觀察之。議者僉曰:「宜如羊祜與陸抗書相聞。」時張賓有疾,勒就而謀之。賓曰:「王浚假三部之力,稱制南面,雖曰晉籓,實懷僭逆之志,必思協英雄,圖濟事業。將軍威聲震于海內,去就為存亡,所在為輕重,浚之欲將軍,猶楚之招韓信也。今權譎遣使,無誠款之形,脫生猜疑,圖之兆露,後雖奇略,無所設也。夫立大事者必先為之卑,當稱籓推奉,尚恐未信,
 羊、陸之事,臣未見其可。」勒曰:「右侯之計是也。」乃遣其舍人王子春、董肇等多齎珍寶,奉表推崇浚為天子曰:「勒本小胡,出於戎裔,值晉綱弛御,海內飢亂,流離屯厄,竄命冀州,共相帥合,以救性命。今晉祚淪夷,遠播吳會,中原無主,蒼生無繫。伏惟明公殿下,州鄉貴望,四海所宗,為帝王者,非公復誰?勒所以捐軀命、興義兵誅暴亂者,正為明公驅除爾。伏願殿下應天順時,踐登皇阼。勒奉戴明公,如天地父母,明公當察勒微心,慈眄如子也。」亦遺棗嵩書而厚賂之。浚謂子春等曰:「石公一時英武,據趙舊都,成鼎峙之勢,何為稱籓于孤,其可信乎?」子春對
 曰:「石將軍英才俊拔,士馬雄盛,實如聖旨。仰惟明公州鄉貴望,累葉重光,出鎮籓嶽,威聲播于八表,固以胡越欽風,戎夷歌德,豈唯區區小府而敢不斂衽神闕者乎!昔陳嬰豈其鄙王而不王,韓信薄帝而不帝者哉?但以知帝王不可以智力爭故也。石將軍之擬明公,猶陰精之比太陽,江河之比洪海爾。項籍、子陽覆車不遠,是石將軍之明鑒,明公亦何怪乎!且自古誠胡人而為名臣者實有之,帝王則未之有也。石將軍非所以惡帝王而讓明公也,顧取之不為天人之所許耳。願公勿疑。」浚大悅,封子春等為列侯,遣使報勒,答以方物。浚司馬游統
 時鎮范陽,陰叛浚,馳使降于勒。勒斬其使,送于浚,以表誠實。浚雖不罪統,彌信勒之忠誠,無復疑矣。



 子春等與王浚使至,勒命匿勁卒精甲,虛府羸師以示之,北面拜使而受浚書。浚遺勒麈尾,勒偽不敢執,懸之于壁,朝夕拜之,云:「我不得見王公,見王公所賜如見公也。」復遣董肇奉表于浚,期親詣幽州奉上尊號,亦修箋於棗嵩,乞並州牧、廣平公,以見必信之誠也。



 勒將圖浚,引子春問之。子春曰:「幽州自去歲大水,人不粒食,浚積粟百萬,不能贍恤,刑政苛酷,賦役殷煩,賊憲賢良,誅斥諫士,下不堪命,流叛略盡。鮮卑、烏丸離貳于外,棗嵩、田嶠貪暴于
 內,人情沮擾,甲士羸弊。而浚猶置立臺閣,布列百官,自言漢高、魏武不足並也。又幽州謠怪特甚,聞者莫不為之寒心,浚意氣自若,曾無懼容,此亡期之至也。」勒撫几笑曰:「王彭祖真可擒也。」浚使達襲幽州,具陳勒形勢寡弱,款誠無二。浚大悅,以勒為信然。



 勒纂兵戒期,將襲浚,而懼劉琨及鮮卑、烏丸為其後患,沈吟未發。張賓進曰:「夫襲敵國,當出其不意。軍嚴經日不行,豈顧有三方之慮乎?」勒曰:「然,為之奈何?」賓曰:「彭祖之據幽州,唯仗三部,今皆離叛,還為寇讎,此則外無聲援以抗我也。幽州飢儉,人皆蔬食,眾叛親離,甲旅寡弱,此則內無彊兵以禦我
 也。若大軍在郊,必土崩瓦解。今三方未靖,將軍便能懸軍千里以徵幽州也。輕軍往返,不出二旬。就使三方有動,勢足旋趾。宜應機電發,勿後時也。且劉琨、王浚雖同名晉籓,其實仇敵。若修箋于琨,送質請和,琨必欣于得我,喜于浚滅,終不救浚而襲我也。」勒曰:「吾所不了,右侯已了,復何疑哉!」



 於是輕騎襲幽州,以火宵行。至柏人,殺主簿游綸,以其兄統在范陽,懼聲軍計故也。遣張慮奉箋于劉琨,陳己過沉重,求討浚以自效。琨既素疾浚,乃檄諸州郡,說勒知命思愆,收累年之咎,求拔幽都,效善將來,今聽所請,受任通和。軍達易水,浚督護孫緯馳遣
 白浚,將引軍距勒,游統禁之。浚將佐咸請出擊勒,浚怒曰:「石公來,正欲奉戴我也,敢言擊者斬!」乃命設饗以待之。勒晨至薊,叱門者開門。疑有伏兵,先驅牛羊數千頭,聲言上禮,實欲填諸街巷,使兵不得發。浚乃懼,或坐或起。勒升其事,命甲士執浚,立之于前,使徐光讓浚曰:「君位冠元台,爵列上公,據幽都驍悍之國,跨全燕突騎之鄉,手握彊兵,坐觀京師傾覆,不救天子,而欲自尊。又專任姦暴,殺害忠良,肆情恣欲,毒遍燕壤。自貽于此,非為天也。」使其將王洛生驛送浚襄國市斬之。於是分遣流人各還桑梓,擢荀綽、裴憲,資給車服。數朱碩、棗嵩、田
 嶠等以賄亂政,責游統以不忠于浚,皆斬之。遷烏丸審廣、漸裳、郝襲、靳市等于襄國。焚燒浚宮殿。以晉尚書劉翰為寧朔將軍、行幽州刺史,戍薊,置守宰而還。遣其東曹掾傅遘兼左長史,封王浚首,獻捷于劉聰。勒既還襄國,劉翰叛勒,奔段匹磾。襄國大饑,穀二升直銀二斤,肉一斤直銀一兩。劉聰以平幽州之勳,乃遣其使人柳純持節署勒大都督陜東諸軍事、驃騎大將軍、東單于,侍中、使持節、開府、校尉、二州牧、公如故,加金鉦黃鉞,前後鼓吹二部,增封十二郡。勒固辭,受二郡而已。勒封左長史張敬等十一人為伯、子、侯,文武進位有差。



 勒將支雄
 攻劉演於廩丘,為演所敗。演遣其將韓弘、潘良襲頓丘,斬勒所署太守邵攀。支雄追擊弘等,害潘良于廩丘。劉琨遣樂平太守焦球攻勒常山,斬其太守邢泰。琨司馬溫嶠西討山胡,勒將逯明要之,敗嶠于潞城。



 勒以幽冀漸平,始下州郡閱實人戶,戶貲二匹,租二斛。



 勒將陳午以浚儀叛于勒。逯明攻寧黑於茌平,降之,因破東燕酸棗而還,徙降人二萬餘戶于襄國。勒使其將葛薄寇濮陽,陷之,害太守韓弘。



 劉聰遣其使人范龕持節策命勒,賜以弓矢,加崇為陜東伯,得專征伐,拜封刺史、將軍、守宰、列侯,歲盡集上。署其長子興為上黨國世子,加翼軍
 將軍,為驃騎副貳。



 劉琨遣王旦攻中山,逐勒所署太守秦固。勒將劉勔距旦,敗之,執旦于望都關。勒襲邵續于樂陵。續盡眾逆戰,大敗而還。



 章武人王昚起于科斗壘,擾亂勒河間、渤海諸郡。勒以揚武張夷為河間太守,參軍臨深為渤海太守,各率步騎三千以鎮靜之,使長樂太守程遐屯于昌亭為之聲勢。



 徙平原烏丸展廣、劉哆等部落三萬餘戶于襄國。



 使石季龍襲乞活王平於梁城,敗績而歸。又攻劉演于廩丘。支雄、逯明擊寧黑于東武陽,陷之,黑赴河而死,徙其眾萬餘于襄國。邵續使文鴦救演,季龍退止盧關津避之,文鴦弗能進,屯于
 景亭。兗豫豪右張平等起兵救演。季龍夜棄營設伏於外,揚聲將歸河北。平等以為信然,入于空營。季龍迴擊敗之,遂陷廩丘,演奔文鴦軍,獲演弟啟,送于襄國。演即劉琨之兄子也。勒以琨撫存其母,德之,賜啟田宅,令儒官授其經。



 時大蝗,中山、常山尤甚。中山丁零翟鼠叛勒,攻中山、常山,勒率騎討之,獲其母妻而還。鼠保于胥關,遂奔代郡。



 勒攻樂平太守韓據於坫城,劉琨遣將軍姬澹率眾十餘萬討勒,琨次廣牧,為澹聲援。勒將距之,或諫之曰:「澹兵馬精盛,其鋒不可發,宜深溝高壘以挫其銳,攻守勢異,必獲萬全。」勒曰:「澹大眾遠來,體疲力竭,犬羊烏
 合,號令不齊,可一戰而擒之,何強之有!寇已垂至,胡可捨去,大軍一動,豈易中還!若澹乘我之退,顧乃無暇,焉得深溝高壘乎!此為不戰而自滅亡之道。」立斬諫者。以孔萇為前鋒都督,令三軍後出者斬。設疑兵于山上,分為二伏。勒輕騎與澹戰,偽收眾而北。澹縱兵追之,勒前後伏發,夾擊,澹軍大敗,獲鎧馬萬匹,澹奔代郡,據奔劉琨。琨長史李弘以并州降于勒,琨遂奔于段匹磾。勒遷陽曲、樂平戶于襄國,置守宰而退。孔萇追姬澹於桑干。勒遣兼左長史張敷獻捷於劉聰。



 勒之征樂平也,其南和令趙領招合廣川、平原、渤海數千戶叛勒,奔于邵續。
 河間邢嘏累徵不至,亦聚眾數百以叛。勒巡下冀州諸縣,以右司馬程遐為寧朔將軍、監冀州七郡諸軍事。



 勒姊夫廣威張越與諸將蒱博,勒親臨觀之。越戲言忤勒,勒大怒,叱力士折其脛而殺之。



 孔萇攻代郡,澹死之。時司、冀、并、兗州流人數萬戶在于遼西,迭相招引,人不安業。孔萇等攻馬嚴、馮者,久而不剋。勒問計於張賓,賓對曰:「馮者等本非明公之深仇,遼西流人悉有戀本之思。今宜班師息甲,差選良守,任之以龔遂之事,不拘常制,奉宣仁澤,奮揚威武,幽冀之寇可翹足而靜,遼西流人可指時而至。」勒曰:「右侯之計是也。」召萇等歸,署武遂令
 李回為易北都護、振武將軍、高陽太守。馬嚴士眾多李潛軍人,回先為潛府長史,素服回威德,多叛嚴歸之。嚴以部眾離貳,懼,奔于幽州,溺水而死。馮者率眾降于勒。回移居易京,流人降者歲常數千,勒甚嘉之,封回弋陽子,邑三百戶。加賓封一千戶,進賓位前將軍,固辭不受。



 河朔大蝗,初穿地而生,二旬則化狀若蠶,七八日而臥,四日蛻而飛,彌亙百草,唯不食三豆及麻,並冀尤甚。



 石季龍濟自長壽津,寇梁國,害內史荀闔。劉琨與段匹磾、涉復辰、疾六眷,段末柸等會于固安,將謀討勒,勒使參軍王續齎金寶遺末柸以間之。末柸既思有以報勒恩,
 又忻於厚賂,乃說辰眷等引還,琨、匹磾亦退如薊城。



 邵續使兄子濟攻勒渤海,虜三千餘人而還。劉聰將趙固以洛陽歸順,恐勒襲之,遣參軍高少奉書推崇勒,請師討聰。勒以大義讓之,固深恨恚,與郭默攻掠河內、汲郡。



 段末柸殺鮮卑單于截附真,立忽跋鄰為單于。段匹磾自幽州攻末柸,末柸逆擊敗之,匹磾奔還幽州,因害太尉劉琨,琨將佐相繼降勒。末柸遣弟騎督擊匹磾於幽州,匹磾率其部眾數千,將奔邵續,勒將石越要之于鹽山,大敗之,匹磾退保幽州。越中流矢死,勒為之屏樂三月,贈平南將軍。



 初,曹嶷據有青州,既叛劉聰,南稟王命,
 以建鄴懸遠,勢援不接,懼勒襲之,故遣通和。勒授嶷東州大將軍、青州牧,封瑯邪公。



 劉聰疾甚,驛召勒為大將軍、錄尚書事,受遺詔輔政,勒固辭乃止。聰又遣其使人持節署勒大將軍、持節鉞,都督、侍中、校尉、二州牧、公如故,增封十郡,勒不受。聰死,其子粲襲偽位,其大將軍靳準殺粲於平陽,勒命張敬率騎五千為前鋒以討準,勒統精銳五萬繼之,據襄陵北原,羌羯降者四萬餘落。準數挑戰,勒堅壁以挫之。劉曜自長安屯于蒲阪,曜復僭號,署勒大司馬、大將軍,加九錫,增封十郡,并前十三郡,進爵趙公。勒攻準於平陽小城,平陽大尹周置等率雜
 戶六千降于勒。巴帥及諸羌羯降者十餘萬落,徙之司州諸縣。準使卜泰送乘輿服御請和,勒與劉曜競有招懷之計,乃送泰于曜,使知城內無歸曜之意,以挫其軍勢。曜潛與泰結盟,使還平陽宣慰諸屠各。勒疑泰與曜有謀,欲斬泰以速降之,諸將皆曰:「今斬卜泰,準必不復降,就令泰宣漢要盟于城中,使相率誅靳準,準必懼而速降矣。」勒久乃從諸將議遣之。泰入平陽,與準將喬泰、馬忠等起兵攻準,殺之,推靳明為盟主,遣泰及卜玄奉傳國六璽送于劉曜。勒大怒,遣令史羊升使平陽,責明殺準之狀。明怒,斬升。勒怒甚,進軍攻明,明出戰,勒擊敗
 之,枕尸二里。明築城門堅守,不復出戰。勒遣其左長史王脩獻捷于劉曜。晉彭城內史周堅害沛內史周默,以彭沛降于勒。石季龍率幽、冀州兵會勒攻平陽。劉曜遣征東劉暢救明。勒命舍師于蒲上。靳明率平陽之眾奔於劉曜,曜西奔粟邑。勒焚平陽宮室,使裴憲、石會脩復元海、聰二墓,收劉粲已下百餘尸葬之,徙渾儀、樂器於襄國。



 劉曜又遣其使人郭汜等持節署勒太宰,領大將軍,進爵趙王,增封七郡,并前二十郡,出入警蹕,冕十有二旒,乘金根車,駕六馬,如曹公輔漢故事,夫人為王后,世子為王太子。勒舍人曹平樂因使留仕於曜,言於曜
 曰:「大司馬遣王脩等來,外表至虔,內覘大駕彊弱,謀待脩之返,將輕襲乘輿。」時曜勢實殘弊,懼脩宣之。曜大怒,追汜等還,斬脩於粟邑,停太宰之授。劉茂逃歸,言王脩死故,勒大怒,誅平樂三族,贈脩太常。又知停殊禮之授,怒甚,下令曰:「孤兄弟之奉劉家,人臣之道過矣,若微孤兄弟,豈能南面稱朕哉!根基既立,便欲相圖。天不助惡,使假手靳準。孤惟事君之體當資舜求瞽瞍之義,故復推崇令主,齊好如初,何圖長惡不悛,殺奉誠之使。帝王之起,復何常邪!趙王、趙帝,孤自取之,名號大小,豈其所節邪!」於是置太醫、尚方、御府諸令,命參軍晁贊成正陽
 門。俄而門崩,勒大怒,斬贊。既怒刑倉卒,尋亦悔之,賜以棺服,贈大鴻臚。



 平西將軍祖逖攻陳川于蓬關,石季龍救川,逖退屯梁國,季龍使揚武左伏肅攻之。



 勒增置宣文、宣教、崇儒、崇訓十餘小學于襄國四門,簡將佐豪右子弟百餘人以教之,且備擊柝之衛。置挈壺署,鑄豐貨錢。



 河西鮮卑日六延叛於勒,石季龍討之,敗延於朔方,斬首二萬級,俘三萬餘人,獲牛馬十餘萬。孔萇討平幽州諸郡。時段匹磾部眾飢散,棄其妻子,匹磾奔邵續。曹嶷遣使來聘,獻其方物,請以河為斷。桃豹至蓬關,祖逖退如淮南。徙陳川部眾五千餘戶於廣宗。



 石季龍與張
 敬、張賓及諸將佐百餘人勸勒稱尊號,勒下書曰:「孤猥以寡德,忝荷崇寵,夙夜戰惶,如臨深薄,豈可假尊竊號,取譏四方!昔周文以三分之重,猶服事殷朝;小白居一匡之盛,而尊崇周室。況國家道隆殷周,孤德卑二伯哉!其亟止斯議,勿復紛紜。自今敢言,刑茲無赦!」乃止。



 勒又下書曰:「今大亂之後,律令滋煩,其采集律令之要,為施行條制。」於是命法曹令史貫志造《辛亥制度》五千文,施行十餘歲,乃用律令。晉太山太守徐龕叛降於勒。



 石季龍及張敬、張賓、左右司馬張屈六、程遐文武等一百二十九人上疏曰:「臣等聞有非常之度,必有非常之功;有
 非常之功,必有非常之事。是以三代陵遲,五伯迭興,靜難濟時,績侔睿后。伏惟殿下天縱聖哲,誕應符運,鞭撻宇宙,弼成皇業,普天率土,莫不來蘇,嘉瑞徵祥,日月相繼,物望去劉氏、威懷于明公者十分而九矣。今山川夷靜,星辰不孛,夏海重譯,天人繫仰,誠應升御中壇,即皇帝位,使攀附之徒蒙寸尺之潤。請依劉備在蜀、魏王在鄴故事,以河內、魏、汲、頓丘、平原、清河、鉅鹿、常山、中山、長樂、樂平十一郡,并前趙國、廣平、陽平、章武、渤海、河間、上黨、定襄、范陽、漁陽、武邑、燕國、樂陵十三郡,合二十四郡、戶二十九萬為趙國。封內依舊改為內史,準《禹貢》、魏武
 復冀州之境,南至盟津,西達龍門,東至於河,北至於塞垣。以大單于鎮撫百蠻。罷並、朔、司三州,通置部司以監之。伏願欽若昊天,垂副群望也。」勒西面而讓者五,南面而讓者四,百僚皆叩頭固請,勒乃許之。



\end{pinyinscope}