\article{列傳第二十六}

\begin{pinyinscope}
江統
 \gezhu{
  子[A170]惇}
 孫楚
 \gezhu{
  孫統}
 綽



 江統,字應元,陳留圉人也。祖蕤,以義行稱,為譙郡太守,封亢父男。父祚,南安太守。統靜默有遠志,時人為之語曰:「嶷然稀言江應元。」與鄉人蔡克俱知名。襲父爵,除山陰令。時關隴、屢為氐、羌所擾,孟觀西討,自擒氐帥齊萬年。統深惟四夷亂華,宜杜其萌,乃作《徙戎論》。其辭曰:



 夫夷蠻戎狄,謂之四夷,九服之制,地在要荒。《春秋》之義,內
 諸夏而外夷狄。以其言語不通,贄幣不同,法俗詭異,種類乖殊;或居絕域之外,山河之表,崎嶇川谷阻險之地,與中國壤斷土隔,不相侵涉,賦役不及,正朔不加,故曰「天子有道,守在四夷」。禹平九土,而西戎即敘。其性氣貪婪,凶悍不仁,四夷之中,戎狄為甚。弱則畏服,彊則侵叛。雖有賢聖之世,大德之君,咸未能以通化率導,而以恩德柔懷也。當其彊也,以殷之高宗而憊於鬼方,有周文王而患昆夷、獫狁,高祖困於白登,孝文軍於霸上。及其弱也,周公來九譯之貢,中宗納單于之朝,以元成之微,而猶四夷賓服。此其已然之效也。故匈奴求守邊塞,而
 侯應陳其不可,單于屈膝未央,望之議以不臣。是以有道之君牧夷狄也,惟以待之有備,禦之有常,雖稽顙執贄,而邊城不弛固守;為寇賊彊暴,而兵甲不加遠征,期令境內獲安,疆埸不侵而已。



 及至周室失統,諸侯專征,以大兼小,轉相殘滅,封疆不固,而利害異心。戎狄乘間,得入中國。或招誘安撫,以為己用。故申、繒之禍,顛覆宗周;襄公要秦,遽興姜戎。當春秋時,義渠、大荔居秦、晉之域,陸渾、陰戎處伊、洛之間,鄋瞞之屬害及濟東,侵入齊、宋,陵虐邢、衛,南夷與北狄交侵中國,不絕若線。齊桓攘之,存亡繼絕,北伐山戎,以開燕路。故仲尼稱管仲之力,
 嘉左衽之功。逮至春秋之末,戰國方盛,楚吞蠻氏,晉翦陸渾,趙武胡服,開榆中之地,秦雄咸陽,滅義渠之等。始皇之並天下也,南兼百越,北走匈奴,五嶺長城,戎卒億計。雖師役煩殷,寇賊橫暴,然一世之功,戎虜奔卻,當時中國無復四夷也。



 漢興而都長安,關中之郡號曰三輔,《禹貢》雍州,宗周豐、鎬之舊也。及至王莽之敗,赤眉因之,西都荒毀,百姓流亡。建武中,以馬援領隴西太守,討叛羌,徙其餘種於關中,居馮翊、河東空地,而與華人雜處。數歲之後,族類蕃息,既恃其肥彊,且苦漢人侵之。永初之元,騎都尉王弘使西域,發調羌、氏,以為行衛。於是群
 羌奔駭,互相扇動,二州之戎,一時俱發,覆沒將守,屠破城邑。鄧騭之徵,棄甲委兵,輿尸喪師,前後相繼,諸戎遂熾,至於南入蜀漢,東掠趙、魏,唐突軹關,侵及河內。及遣北軍中候朱寵將五營士於孟津距羌,十年之中,夷夏俱斃,任尚、馬賢僅乃克之。此所以為害深重、累年不定者,雖由禦者之無方,將非其才,亦豈不以寇發心腹,害起肘腋,疢篤難療,瘡大遲愈之故哉!自此之後,餘燼不盡,小有際會,輒復侵叛。馬賢忸忲,終於覆敗;段穎臨衝,自西徂樂。雍州之戎,常為國患,中世之寇,惟此為大。漢末之亂,關中殘滅。魏興之初,與蜀分隔,疆埸之戎,一彼
 一此。魏武皇帝令將軍夏侯妙才討叛氏阿貴、千萬等,後因拔棄漢中,遂徙武都之種於秦川,欲以弱寇彊國,扞禦蜀虜。此蓋權宜之計,一時之勢,非所以為萬世之利也。今者當之,已受其弊矣。」



 夫關中土沃物豐,厥田上上,加以涇、渭之流溉其舄鹵,鄭國、白渠灌浸相通,黍稷之饒,畝號一鐘,百姓謠詠其殷實,帝王之都每以為居,未聞戎狄宜在此土也。非我族類,其心必異,戎狄志態,不與華同。而因其衰弊,遷之畿服,士庶玩習,侮其輕弱,使其怨恨之氣毒於骨髓。至於蕃育眾盛,則坐生其心。以貪悍之性,挾憤怒之情,候隙乘便,輒為橫逆。而居封
 域之內,無障塞之隔,掩不備之人,收散野之積,故能為禍滋擾,暴害不測。此必然之勢,已驗之事也。當今之宜,宜及兵威方盛,眾事未罷,徙馮翊、北地、新平、安定界內諸羌,著先零、罕並、析支之地;徙扶風、始平、京兆之氐,出還隴右,著陰平、武都之界。廩其道路之糧,令足自致,各附本種,反其舊土,使屬國、撫夷就安集之。戎晉不雜,並得其所,上合往古即敘之義,下為盛世永久之規。縱有猾夏之心,風塵之警,則絕遠中國,隔閡山河,雖為寇暴,所害不廣。是以充國、子明能以數萬之眾制群羌之命,有征無戰,全軍獨剋,雖有謀謨深計,廟勝遠圖,豈不以
 華夷異處,戎夏區別,要塞易守之故,得成其功也哉!



 難者曰:方今關中之禍,暴兵二載,征戍之勞,老師十萬,水旱之害,薦饑累荒,疫癘之災,札瘥夭昏。凶逆既戮,悔惡初附,且款且畏,咸懷危懼,百姓愁苦,異人同慮,望寧息之有期,若枯旱之思雨露,誠宜鎮之以安豫。而子方欲作役起徒,興功造事,使疲悴之眾,徙自猜之寇,以無穀之人,遷乏食之虜,恐勢盡力屈,緒業不卒,羌戎離散,心不可一,前害未及弭,而後變復橫出矣。



 答曰:羌戎狡猾,擅相號署,攻城野戰,傷害牧守,連兵聚眾,載離寒暑矣。而今異類瓦解,同種土崩,老幼繫虜,丁壯降散,禽離獸
 迸,不能相一。子以此等為尚挾餘資,悔惡反善,懷我德惠而來柔附乎?將勢窮道盡,智力俱困,懼我兵誅以至於此乎?曰,無有餘力,勢窮道盡故也。然則我能制其短長之命,而令其進退由己矣。夫樂其業者不易事,安其居者無遷志。方其自疑危懼,畏怖促遽,故可制以兵威,使之左右無違也。迨其死亡散流,離逷未鳩,與關中之人,戶皆為仇,故可遐遷遠處,令其心不懷土也。夫聖賢之謀事也,為之於未有,理之於未亂,道不著而平,德不顯而成。其次則能轉禍為福,因敗為功,值困必濟,遇否能通。今子遭弊事之終而不圖更制之始,愛易轍之勤
 而得覆車之軌,何哉?且關中之人百餘萬口,率其少多,戎狄居半,處之與遷,必須口實。若有窮乏糝粒不繼者,故當傾關中之穀以全其生生之計,必無擠於溝壑而不為侵掠之害也。今我遷之,傳食而至,附其種族,自使相贍,而秦地之人得其半穀,此為濟行者以廩糧,遺居者以積倉,寬關中之逼,去盜賊之原,除旦夕之損,建終年之益。若憚暫舉之小勞,而忘永逸之弘策;惜日月之煩苦,而遺累世之寇敵,非所謂能開物成務,創業垂統,崇其拓跡,謀及子孫者也。



 並州之胡,本實匈奴桀惡之寇也。漢宣之世,凍餒殘破,國內五裂,後合為二,呼韓邪
 遂衰弱孤危,不能自存,依阻塞下,委質柔服。建武中,南單于復來降附,遂令入塞,居於漠南,數世之後,亦輒叛戾,故何熙、梁槿戎車屢征。中平中,以黃巾賊起,發調其兵,部眾不從,而殺羌渠。由是於彌扶羅求助於漢,以討其賊。仍值世喪亂,遂乘釁而作,鹵掠趙、魏,寇至河南。建安中,又使右賢王去卑誘質呼廚泉,聽其部落散居六郡。咸熙之際,以一部太彊,分為三率。泰始之初,又增為四。於是劉猛內叛,連結外虜。近者郝散之變,發於穀遠。今五部之眾,戶至數萬,人口之盛,過於西戎。然其天性驍勇,弓馬便利,倍於氐、羌。若有不虞風塵之慮,則並州
 之域可為寒心。滎陽句驪本居遼東塞外,正始中,幽州刺史毋丘儉伐其叛者,徙其餘種。始徙之時,戶落百數,子孫孳息,今以千計,數世之後,必至殷熾。今百姓失職,猶或亡叛,犬馬肥充,則有噬齧,況於夷狄,能不為變!但顧其微弱勢力不陳耳。



 夫為邦者,患不在貧而在不均,憂不在寡而在不安。以四海之廣,士庶之富,豈須夷虜在內,然後取足哉!此等皆可申諭發遣,還其本域,慰彼羈旅懷土之思,釋我華夏纖介之憂。惠此中國,以綏四方,德施永世,於計為長。



 帝不能用。未及十年,而夷狄亂華,時服其深識。



 遷中郎。選司以統叔父春為宜春令,統
 因上疏曰:「故事,父祖與官職同名,皆得改選,而未有身與官職同名,不在改選之例。臣以為父祖改選者,蓋為臣子開地,不為父祖之身也。而身名所加,亦施於臣子。佐吏係屬,朝夕從事,官位之號,發言所稱,若指實而語,則違經禮諱尊之義;若詭辭避迴,則為廢官擅犯憲制。今以四海之廣,職位之眾,名號繁多,士人殷富,至使有受寵皇朝,出身宰牧,而令佐吏不得表其官稱,子孫不得言其位號,所以上嚴君父,下為臣子,體例不通。若易私名以避官職,則違《春秋》不奪人親之義。臣以為身名與官職同者,宜與觸父祖名為比,體例既全,於義為弘。」
 朝廷從之。



 轉太子洗馬。在東宮累年,甚被親禮。太子頗闕朝覲,又奢費過度,多諸禁忌,統上書諫曰:



 臣聞古之為臣者,進思盡忠,退思補過,獻可替否,拾遺補闕。是以人主得以舉無失行,言無口過,德音發聞,揚名後世。臣等不逮,無能云補,思竭愚誠,謹陳五事如左,惟蒙一省再省,少垂察納。



 其一曰,六行之義,以孝為首,虞舜之德,以孝為稱,故太子以朝夕視君膳為職,左右就養無方。文王之為世子,可謂篤於事親者也,故能擅三代之美,為百王之宗。自頃聖體屢有疾患,數闕朝侍,遠近觀聽者不能深知其故,以致疑惑。伏願殿下雖有微苦,可堪
 扶輿,則宜自力。《易》曰:「君子終日乾乾。」蓋自勉強不息之謂也。



 其二曰,古之人君雖有聰明之姿,睿喆之質,必須輔弼之助,相導之功,故虞舜以五臣興,周文以四友隆。及成王之為太子也,則周、召為保傅,史佚昭文章,故能聞道早備,登崇大業,刑措不用,流聲洋溢。伏惟殿下天授逸才,聰鑒特達,臣謂猶宜時發聖令,宣揚德音,諮詢保傅,訪逮侍臣,覲見賓客,得令接盡,壅否之情沛然交泰,殿下之美煥然光明。如此,則高朗之風,扇於前人;弘範令軌,永為後式。



 其三曰,古之聖王莫不以儉為德,故堯稱採椽茅茨,禹稱卑宮惡服,漢文身衣弋綈,足履革
 舄,以身先物,政致太平,存為明王,沒見宗祀。及諸侯修之者,魯僖以躬儉節用,聲列《雅頌》;蚡冒以篳路藍縷,用張楚國。大夫修之者,文子相魯,妾不衣帛;晏嬰相齊,鹿裘不補,亦能匡君濟俗,興國隆家。庶人修之者,顏回以簞食瓢飲,揚其仁聲;原憲以蓬戶繩樞,邁其清德。此皆聖主明君賢臣智士之所履行也。故能懸名日月,永世不朽,蓋儉之福也。及到末世,以奢失之者,帝王則有瑤臺瓊室,玉懷象箸,肴膳之珍則熊蹯豹胎,酒池肉林。諸侯為之者,至於丹楹刻桷,餼徵百牢。大夫有瓊弁玉纓,庶人有擊鐘鼎食。亦罔不亡國喪宗,破家失身,醜名彰
 聞,以為後戒。竊聞後園鏤飾金銀,刻磨犀象,畫室之巧,課試日精。臣等以為今四海之廣,萬物之富,以今方古,不足為侈也。然上之所好,下必從之,是故居上者必慎其所好也。昔漢光武皇帝時,有獻千里馬及寶劍者,馬以駕鼓車,劍以賜騎士。世祖武皇帝有上雉頭裘者,即詔有司焚之都街。高世之主,不尚尤物,故能正天下之俗,刑四方之風。臣等以為畫室之功,可且減省,後園雜作,一皆罷遣,肅然清靜,優游道德,則日新之美光于四海矣。



 其四曰,以天下而供一人,以百里而供諸侯,故王侯食藉而衣稅,公卿大夫受爵而資祿,莫有不贍者也。
 是以士農工商四業不雜。交易而退,以通有無者,庶人之業也。《周禮》三市,旦則百族,晝則商賈,夕則販夫販婦。買賤賣貴,販鬻菜果,收十百之盈,以救旦夕之命,故為庶人之貧賤者也。樊遲匹夫,請學為圃,仲尼不答;魯大夫臧文仲使妾織蒲,又譏其不仁;公儀子相魯,則拔其園葵,言食祿者不與貧賤之人爭利也。秦、漢以來,風俗轉薄,公侯之尊,莫不殖園圃之田,而收市井之利,漸冉相放,莫以為恥,乘以古道,誠可愧也。今西園賣葵菜、藍子、雞、面之屬,虧敗國體,貶損令問。



 其五曰,竊見禁土,令不得繕修牆壁,動正屋瓦。臣以為此既違典彝舊義,且
 以拘攣小忌而廢弘廓大道,宜可蠲除,於事為宜。



 朝廷善之。



 及太子廢,徙許昌,賈后諷有司不聽宮臣追送。統與宮臣冒禁至伊水,拜辭道左,悲泣流漣。都官從事悉收統等付河南、洛陽獄。付郡者,河南尹樂廣悉散遣之,繫洛陽者猶未釋。都官從事孫琰說賈謐曰:「所以廢徙太子,以為惡故耳。東宮故臣冒罪拜辭,涕泣路次,不顧重辟,乃更彰太子之德,不如釋之。」謐語洛陽令曹攄,由是皆免。及太子薨,改葬,統作誄敘哀,為世所重。



 後為博士、尚書郎,參大司馬、齊王冏軍事。冏驕荒將敗,統切諫,文多不載。遷廷尉正,每州郡疑獄,斷處從輕。成都王穎
 請為記室,多所箴諫。申論陸雲兄弟,辭甚切至。以母憂去職。服闋,為司徒左長史。東海王越為兗州牧,以統為別駕,委以州事,與統書曰:「昔王子師為豫州,未下車,辟荀慈明;下車,辟孔文舉。貴州人士有堪應此者不?」統舉高平郗鑒為賢良,陳留阮脩為直言,濟北程收為方正,時以為知人。尋遷黃門侍郎、散騎常侍,領國子博士。永嘉四年,避難奔于成皋,病卒。凡所造賦頌表奏皆傳於後。二子:[A170],惇。



 [A170]字思玄,本州辟舉秀才,平南將軍溫嶠以為參軍。復為州別駕,辟司空郗鑒掾,除長山令。鑒又請為司馬,轉
 黃門郎。車騎將軍庾冰鎮江州,請為長史。冰薨,庾翼以為諮議參軍,俄而復補長史。翼薨,大將乾瓚作難,[A170]討平之。除尚書吏部郎,仍遷御史中丞、侍中、吏部尚書。永和中,代桓景為護軍將軍。出補會稽內史,加右軍將軍。代王彪之為尚書僕射。哀帝即位,疑周貴人名號所宜,[A170]議見《禮志》。帝欲於殿庭立鴻祀,又欲躬自藉田,[A170]並以為禮廢日久,儀注不存,中興以來所不行,謂宜停之。為僕射積年,簡文帝為相,每訪政事,[A170]多所補益,轉護軍將軍,領國子祭酒,卒官。子敳,歷瑯邪內史、驃騎諮議。敳子恒,元熙中為西中郎長史。恒弟夷,尚書。



 惇字思悛,孝友淳粹,高節邁俗。性好學,儒玄並綜。每以為君子立行,應依禮而動,雖隱顯殊途,未有不傍禮教者也。若乃放達不羈,以肆縱為貴者,非但動違禮法,亦道之所棄也。乃著《通道崇檢論》,世咸稱之。蘇峻之亂,避地東陽山,太尉郗鑒檄為兗州治中,又辟太尉掾;康帝為司徒,亦辟焉;征西將軍庾亮請為儒林參軍;徵拜博士、著作郎,皆不就。邑里宗其道,有事必諮而後行。東陽太守阮裕、長山令王濛,皆一時名士,並與惇游處,深相欽重。養志二十餘年,永和九年卒,時年四十九,友朋相與刊石立頌,以表德美云。



 孫楚,字子荊,太原中都人也。祖資,魏驃騎將軍。父宏,南陽太守。楚才藻卓絕,爽邁不群,多所陵傲,缺鄉曲之譽。年四十餘,始參鎮東軍事。文帝遣符劭、孫郁使吳,將軍石苞令楚作書遺孫皓曰:



 蓋見機而作,《周易》所貴;小不事大,《春秋》所誅。此乃吉凶之萌兆,榮辱所由生也。是故許、鄭以銜璧全國,曹譚以無禮取滅。載藉既記其成敗,古今又著其愚智,不復廣引譬類,崇飾浮辭。茍以夸大為名,更喪忠告之實。今粗論事要,以相覺悟。



 昔炎精幽昧,歷數將終,桓、靈失德,災釁並興,豺狼抗爪牙之毒,生
 靈罹塗炭之難。由是九州絕貫,王綱解紐,四海蕭條,非復漢有。太祖承運,神武應期,征討暴亂,剋寧區夏;協建靈符,天命既集,遂廓弘基,奄有魏域。土則神州中嶽,器則九鼎猶存,世載淑美,重光相襲,故知四隩之攸同,帝者之壯觀也。昔公孫氏承藉父兄,世居東裔,擁帶燕胡,憑陵險遠,講武游盤,不供職貢,內傲帝命,外通南國,乘桴滄海,交酬貨賄,葛越布于朔土,貂馬延于吳會;自以控弦十萬,奔走之力,信能右折燕、齊,左震扶桑,輮轢沙漠,南面稱王。宣王薄伐,猛銳長驅,師次遼陽,而城池不守;枹鼓暫鳴,而元凶折首。於是遠近疆埸,列郡大荒,收
 離聚散,大安其居,眾庶悅服,殊俗款附。自茲以降,九野清泰,東夷獻其樂器,肅慎貢其楛矢,曠世不羈,應化而至,巍巍蕩蕩,想所具聞也。



 吳之先祖,起自荊、楚,遭時擾攘,潛播江表。劉備震懼,亦逃巴、岷。遂因山陵積石之固,三江五湖浩汗無涯,假氣游魂,迄茲四紀。兩邦合從,東西唱和,互相扇動,距捍中國。自謂三分鼎足之勢,可與泰山共相終始也。相國晉王輔相帝室,文武桓桓,志厲秋霜,廟勝之算,應變無窮,獨見之鑒,與眾絕慮。主上欽明,委以萬機,長轡遠御,妙略潛授,偏師同心,上下用力,陵威奮伐,罙入其阻,并敵一向,奪其膽氣。小戰江由,則
 成都自潰;曜兵劍閣,則姜維面縛。開地六千,領郡三十。兵不踰時,梁、益肅清,使竊號之雄,稽顙絳闕,球琳重錦,充於府庫。夫韓并魏徙,虢滅虞亡,此皆前鑒,後事之表。又南中呂興,深睹天命蟬蛻內附,願為臣妾。外失輔車脣齒之援,內有羽毛零落之漸,而徘徊危國,冀延日月,此由魏武侯卻指山河,自以為彊,殊不知物有興亡,則所美非其地也。



 方今百僚濟濟,俊乂盈朝,武臣猛將,折衝萬里,國富兵彊,六軍精練,思復翰飛,飲馬南海。自頃國家整脩器械,興造舟楫,簡習水戰,樓船萬艘,千里相望,刳木已來,舟車之用未有如今之殷盛者也。驍勇百
 萬,畜力待時。役不再舉,今日之師也。然主相眷眷未便電發者,猶以為愛人治國,道家所尚,崇城遂卑,文王退舍,故先開大信,喻以存亡,殷勤之指,往使所究也。若能審勢安危,自求多福,蹶然改容,祗承往錫,追慕南越,嬰齊入侍,北面稱臣,伏聽告策,則世祚江表,永為魏籓,豐功顯報,隆於今日矣。若猶侮慢,未順王命,然後謀力雲合,指麾從風,雍、梁二州,順流而東,青、徐戰士,列江而西,荊、揚兗、豫,爭驅八衝,征東甲卒,武步秣陵,爾乃王輿整駕,六戎徐徵,羽校燭日,旌旗星流,龍游曜路,歌吹盈耳,士卒奔邁,其會如林,煙塵俱起,震天駭地,渴賞之士,鋒
 鏑爭先,忽然一旦,身首橫分,宗祀淪覆,取戒萬世,引領南望,良助寒心!夫療膏肓之疾者,必進苦口之藥;決狐疑之慮者,亦告逆耳之言。如其猶豫,迷而不反,恐俞附見其已死,扁鵲知其無功矣。勉思良圖,惟所去就。



 劭等至吳,不敢為通。



 楚後遷佐著作郎,復參石苞驃騎軍事。楚既負其材氣,頗侮易於苞,初至,長揖曰:「天子命我參卿軍事。」因此而嫌隙遂構。苞奏楚與吳人孫世山共訕毀時政,楚亦抗表自理,紛紜經年,事未判,又與鄉人郭奕忿爭。武帝雖不顯明其罪,然以少賤受責,遂湮廢積年。初,參軍不敬府主,楚既輕苞,遂制施敬,自楚始也。



 征
 西將軍,扶風王駿與楚舊好,起為參軍。轉梁令,遷衛將軍司馬,時龍見武庫井中,群臣將上賀,楚上言曰:「頃聞武庫井中有二龍,群臣或有謂之禎祥而稱賀者,或有謂之非祥無所賀者,可謂楚既失之,而齊亦未為得也。夫龍或俯鱗潛于重泉,或仰攀雲漢游乎蒼昊,而今蟠于坎井,同於蛙蝦者,豈獨管庫之士或有隱伏,廝役之賢沒於行伍?故龍見光景,有所感悟。願陛下赦小過,舉賢才,垂夢於傅巖,望想於渭濱,修學官,起淹滯,申命公卿,舉獨行君子可惇風厲俗者,又舉亮拔秀異之才可以撥煩理難矯世抗言者,無系世族,必先逸賤。夫戰勝
 攻取之勢,并兼混一之威,五伯之事,韓、白之功耳;至於制禮作樂,闡揚道化,甫是士人出筋力之秋也。伏願陛下擇狂夫之言。」



 惠帝初,為馮翊太守。元康三年卒。



 初,楚與同郡王濟友善,濟為本州大中正,訪問銓邑人品狀,至楚,濟曰:「此人非卿所能目,吾自為之。」乃狀楚曰:「天才英博,亮拔不群。」楚少時欲隱居,謂濟曰:「當欲枕石漱流。」誤云「漱石枕流」。濟曰:「流非可枕,石非可漱。」楚曰:「所以枕流,欲洗其耳;所以漱石,欲厲其齒。」楚少所推服,惟雅敬濟。初,楚除婦服,作詩以示濟,濟曰:「未知文生於情,情生於文,覽之悽然,增伉儷之重。」



 三子:眾、洵、纂。眾及洵俱未
 仕而早終,惟纂子統、綽並知名。



 統字承公。幼與綽及從弟盛過江。誕任不羈,而善屬文,時人以為有楚風。征北將軍褚裒聞其名,命為參軍,辭不就,家于會稽。性好山水,乃求為鄞令,轉在吳寧。居職不留心碎務,縱意游肆,名山勝川,靡不窮究。後為餘姚令,卒。



 子騰嗣,以博學著稱,位至廷尉。騰弟登,少善名理,注《老子》,行於世,仕至尚書郎,早終。



 綽字興公。博學善屬文,少與高陽許詢俱有高尚之志。居于會稽,游放山水,十有餘年,乃作《遂初賦》以致其意。嘗鄙山濤,而謂人曰:「山濤吾所不解,吏非吏,隱非隱,若
 以元禮門為龍津,則當點額暴鱗矣。」所居齋前種一株松,恒自守護,鄰人謂之曰:「樹子非不楚楚可憐,但恐永無棟梁日耳。」綽答曰:「楓柳雖復合抱,亦何所施邪!」綽與詢一時名流,或愛詢高邁,則鄙於綽,或愛綽才藻,而無取於詢。沙門支遁試問綽:「君何如許?」答曰:「高情遠致,弟子早已伏膺;然一詠一吟,許將北面矣。」絕重張衡、左思之賦,每云:「《三都》、《二京》,五經之鼓吹也。」嘗作《天台山賦》,辭致甚工,初成,以示友人范榮期,云:「卿試擲地,當作金石聲也。」榮期曰:「恐此金石非中宮商。」然每至佳句,輒云:「應是我輩語。」除著作佐郎,襲爵長樂侯。」



 綽性通率,好譏調。
 嘗與習鑿齒共行,綽在前,顧謂鑿齒曰:「沙之汰之,瓦石在後。」鑿齒曰:「簸之揚之,糠秕在前。」



 征西將軍庾亮請為參軍,補章安令,徵拜太學博士,遷尚書郎。楊州刺史殷浩以為建威長史。會稽內史王羲之引為右軍長史。轉永嘉太守,遷散騎常侍,領著作郎。



 時大司馬桓溫欲經緯中國,以河南粗平,將移都洛陽。朝廷畏溫,不敢為異,而北土蕭條,人情疑懼,雖並知不可,莫敢先諫。綽乃上疏曰:



 伏見征西大將軍臣溫表「便當躬率三軍,討除二寇,蕩滌河、渭,清灑舊京,然後神旂電舒,朝服濟江,反皇居於中土,正玉衡於天極。」斯超世之弘圖,千載之盛事。
 然臣之所懷,竊有未安,以為帝王之興,莫不藉地利人和以建功業,貴能以義平暴,因而撫之。懷愍不建,滄胥秦京,遂令胡戎交侵,神州絕綱,土崩之釁,誠由道喪。然中夏蕩蕩,一時橫流,百郡千城曾無完郛者,何哉?亦以地不可守,投奔有所故也。天祚未革,中宗龍飛,非惟信順協於天人而已,實賴萬里長江畫而守之耳。《易》稱「王公設險以守其國」,險之時義大矣哉!斯已然之明效也。今作勝談,自當任道而遺險;校實量分,不得不保小以固存。自喪亂已來六十餘年,蒼生殄滅,百不遺一,河洛丘、虛,函夏蕭條,井堙木刊,阡陌夷滅,生理茫茫,永無依
 歸。播流江表,已經數世,存者長子老孫,亡者丘隴成行。雖北風之思感其素心,目前之哀實為交切。若遷都旋軫之日,中舉五陵,即復緬成遐域。泰山之安既難以理保,烝烝之思豈不纏於聖心哉!



 溫今此舉,誠欲大覽始終,為國遠圖。向無山陵之急,亦未首決大謀,獨任天下之至難也。今發憤忘食,忠慨亮到,凡在有心,孰不致感!而百姓震駭,同懷危懼者,豈不以反舊之樂賒,而趣死之憂促哉!何者?植根於江外數十年矣,一朝拔之,頓驅踧於空荒之地,提挈萬里,踰險浮深,離墳墓,棄生業,富者無三年之糧,貧者無一餐之飯,田宅不可復售,舟車
 無從而得,捨安樂之國,適習亂之鄉,出必安之地,就累卵之危,將頓仆道塗,飄溺江川,僅有達者。夫國以人為本,疾寇所以為人,眾喪而寇除,亦安所取裁?此仁者所宜哀矜,國家所宜深慮也。自古今帝王之都,豈有常所,時隆則宅中而圖大,勢屈則遵養以待會。使德不可勝,家有三年之積,然後始可謀太平之事耳。今天時人事,有未至者矣,一朝欲一宇宙,無乃頓而難舉乎?



 臣之愚計,以為且可更遣一將有威名資實者,先鎮洛陽,於陵所築二壘以奉衛山陵,掃平梁、許,清一河南,運漕之路既通,然後盡力於開墾,廣田積穀,漸為徙者之資。如此,
 賊見亡徵,勢必遠竄。如其迷逆不化,復欲送死者,南北諸軍風馳電赴,若身手之救痛癢,率然之應首尾,山陵既固,中夏小康。陛下且端委紫極,增修德政,躬行漢文簡樸之至,去小惠,節游費,審官人,練甲兵,以養士滅寇為先。十年行之,無使隳廢,則貧者殖其財,怯者充其勇,人知天德,赴死如歸,以此致政,猶運諸掌握。何故捨百勝之長理,舉天下而一擲哉!陛下春秋方富,溫克壯其猷,君臣相與,弘養德業,括囊元吉,豈不快乎!



 今溫唱高議,聖朝互同,臣以輕微,獨獻管見。出言之難,實在今日,而臣區區必聞天聽者,竊以無諱之朝,狂瞽進說,芻蕘
 之謀,聖賢所察,所以不勝至憂,觸冒乾陳。若陛下垂神,溫少留思,豈非屈於一人而允億兆之顧哉!如以干忤罪大,欲加顯戮,使丹誠上達,退受刑誅,雖沒泉壤,尸且不朽。



 桓溫見綽表,不悅,曰:「致意興公,何不尋君《遂初賦》,知人家國事邪!」尋轉廷尉卿,領著作。綽少以文才垂稱,于時文士,綽為其冠。溫、王、郗、庾諸公之薨,必須綽為碑文,然後刊石焉。年五十八,卒。



 子嗣,有綽風,文章相亞,位至中軍參軍,早亡。



 史臣曰:江統風檢操行,良有可稱,陳留多士,斯為其冠。《徙戎》之論,實乃經國遠圖。然運距中衰,陵替有漸,假其
 言見用,恐速禍招怨,無救於將顛也。逮愍懷廢徙,冒禁拜辭,所謂命輕鴻毛,義貴熊掌。[A170]位隆端石,竭誠獻替。惇遺忽榮利,聿修天爵。雖出處異途,俱難兄弟矣。孫楚體英絢之姿,超然出類,見知武子,誠無愧色。覽其貽皓之書,諒曩代之佳筆也。而負才誕傲,蔑苞忿奕,違遜讓之道,肆陵憤之氣,丁年沈廢,諒自取矣。統、綽棣華秀發,名顯中興,可謂無忝爾祖。統竟淪跡下邑,窮觀勝地,會其心焉。綽獻直論辭,都不懾元子,有匪躬之節,豈徒文雅而已哉!



 贊曰:應元蹈義,子荊越俗。江寡悔尤,孫貽擯辱。[A170]、統昆
 弟,江左馳聲。彬彬藻思,綽冠群英。



\end{pinyinscope}