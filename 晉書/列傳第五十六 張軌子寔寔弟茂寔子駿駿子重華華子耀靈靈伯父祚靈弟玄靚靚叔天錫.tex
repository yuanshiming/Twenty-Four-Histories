\article{列傳第五十六 張軌子寔寔弟茂寔子駿駿子重華華子耀靈靈伯父祚靈弟玄靚靚叔天錫}

\begin{pinyinscope}
張軌
 \gezhu{
  子寔寔弟茂寔子駿駿子重華華子耀靈靈伯父祚靈弟玄靚靚叔天錫}



 張軌,字士彥,安定烏氏人,漢常山景王耳十七代孫也。家世孝廉,以儒學顯。父溫,為太官令。軌少明敏好學,有器望,姿儀典則,與同郡皇甫謐善,隱于宜陽女几山。泰始初,受叔父錫官五品。中書監張華與軌論經義及政事損益,甚器之,謂安定中正為蔽善抑才,乃美為之談,
 以為二品之精。衛將軍楊珧辟為掾,除太子舍人,累遷散騎常侍、征西軍司。



 軌以時方多難,陰圖據河西,筮之,遇《泰》之《觀》,乃投策喜曰:「霸者兆也。」於是求為涼州。公卿亦舉軌才堪御遠。永寧初,出為護羌校尉、涼州刺史。于時鮮卑反叛,寇盜從橫,軌到官,即討破之,斬首萬餘級,遂威著西州,化行河右。以宋配、陰充、氾瑗、陰澹為股肱謀主,徵九郡胄子五百人,立學校,始置崇文祭酒,位視別駕,春秋行鄉射之禮。祕書監繆世征、少府摯虞夜觀星象,相與言曰:「天下方亂,避難之國唯涼土耳。張涼州德量不恒,殆其人乎!」及河間、成都二王之難,遣兵三千,
 東赴京師。初,漢末金城人陽成遠殺太守以叛,郡人馮忠赴尸號哭,嘔血而死。張掖人吳詠為護羌校尉馬賢所辟,後為太尉龐參掾,參、賢相誣,罪應死,各引詠為證,詠計理無兩直,遂自刎而死。參、賢慚悔,自相和釋。軌皆祭其墓而旌其子孫。永興中,鮮卑若羅拔能皆為寇,軌遣司馬宋配擊之,斬拔能,俘十餘萬口,威名大震。惠帝遣加安西將軍,封安樂鄉侯,邑千戶。於是大城姑臧。其城本匈奴所築也,南北七里,東西三里,地有龍形,故名臥龍城。初,漢末博士敦煌侯瑾謂其門人曰:「後城西泉水當竭,有雙闕起其上,與東門相望。中有霸者出焉。」至
 魏嘉平中,郡官果起學館,築雙闕于泉上,與東門正相望矣。至是,張氏遂霸河西。



 永嘉初,會東羌校尉韓稚殺秦州刺史張輔,軌少府司馬楊胤言於軌曰:「今稚逆命,擅殺張輔,明公杖鉞一方,宜懲不恪,此亦《春秋》之義。諸侯相滅亡,桓公不能救,則恆公恥之。」軌從焉,遣中督護氾瑗率眾二萬討之。先遺稚書曰:「今天綱紛撓,牧守宜戮力勤王。適得雍州檄,云卿稱兵內侮,吾董任一方,義在伐叛,武旅三萬,駱驛繼發,伐木之感,心豈可言!古之行師,全國為上,卿若單馬軍門者,當與卿共平世難也。」稚得書而降。遣主簿令狐亞聘南陽王模,模甚悅,遺軌
 以帝所賜劍,謂軌曰:「自隴以西,征伐斷割悉以相委,如此劍矣。」俄而王彌寇洛陽,軌遣北宮純、張纂、馬魴、陰濬等率州軍擊破之,又敗劉聰于河東,京師歌之曰:「涼州大馬,橫行天下。涼州鴟苕,寇賊消;鴟苕翩翩,怖殺人。」帝嘉其忠,進封西平郡公,不受。張掖臨松山石有「金馬」字,磨滅粗可識,而「張」字分明,又有文曰:「初祚天下,西方安萬年。」姑臧又有玄石,白點成二十八宿。于時天下既亂,所在使命莫有至者,軌遣使貢獻,歲時不替。朝廷嘉之,屢降璽書慰勞。



 軌後患風,口不能言,使子茂攝州事。酒泉太守張鎮潛引秦州刺史賈龕以代軌,密使詣京師,
 請尚書侍郎曹祛為西平太守,圖為輔車之勢。軌別駕麴晁欲專威福,又遣使詣長安,告南陽王模,稱軌廢疾,以請賈龕,而龕將受之。其兄讓龕曰:「張涼州一時名士,威著西州,汝何德以代之!」龕乃止。更以侍中爰瑜為涼州刺史。治中楊澹馳詣長安,割耳盤上,訴軌之被誣,模乃表停之。



 晉昌張越,涼州大族,讖言張氏霸涼,自以才力應之。從隴西內史遷梁州刺史。越志在涼州,遂託病歸河西,陰圖代軌,乃遣兄鎮及曹祛、麴佩移檄廢軌,以軍司杜耽攝州事,使耽表越為刺史。軌令曰:「吾在州八年,不能綏靖區域,又值中州兵亂,秦隴倒懸,加以寢患
 委篤,實思斂迹避賢。但負荷任重,未便輒遂。不圖諸人橫興此變,是不明吾心也。吾視去貴州如脫屣耳!」欲遣主簿尉髦奉表詣闕,便速脂轄,將歸老宜陽。長史王融、參軍孟暢蹋折鎮檄,排闔諫曰:「晉室多故,人神塗炭,實賴明公撫寧西夏。張鎮兄弟敢肆凶逆,宜聲其罪而戮之,不可成其志也。」軌嘿然。副等出而戒嚴。武威太守張琠遣子坦馳詣京,表曰:「魏尚安邊而獲戾,充國盡忠而被譴,皆前史之所譏,今曰之明鑒也。順陽之思劉陶,守闕者千人。刺史之蒞臣州,若慈母之於赤子,百姓之愛臣軌,若旱苗之得膏雨。伏聞信惑流言,當有遷代,民
 情嗷嗷,如失父母。今戎夷猾夏,不宜騷動一方。」尋以子寔為中督護,率兵討鎮。遣鎮外甥太府主簿令狐亞前喻鎮曰:「舅何不審安危,明成敗?主公西河著德,兵馬如雲,此猶烈火已焚,待江漢之水,溺於洪流,望越人之助,其何及哉!今數萬之軍已臨近境,今唯全老親,存門戶,輸誠歸官,必保萬全之福。」鎮流涕曰:「人誤我也!」乃委罪功曹魯連而斬之,詣寔歸罪。南討曹祛,走之。張坦至自京師,帝優詔勞軌,依模所表,命誅曹祛。軌大悅,赦州內殊死已下。命寔率尹員、宋配步騎三萬討祛,別遣從事田迥、王豐率騎八百自姑臧西南出石驢,據長寧。怯遣
 麴晁距戰于黃阪。寔詭道出浩亹,戰于破羌。軌斬祛及牙門田囂。



 遣治中張閬送義兵五千及郡國秀孝貢計、器甲方物歸於京師。令有司可推詳立州已來清貞德素,嘉遁遺榮:「高才碩學,著述經史;臨危殉義,殺身為君;忠諫而嬰禍,專對而釋患;權智雄勇,為時除難;謅佞誤主,傷陷忠賢;具狀以聞。州中父老莫不相慶。光祿傅祗、太常摯虞遺軌書,告京師饑匱,軌即遣參軍杜勳獻馬五百匹、毯布三萬匹。帝遣使者進拜鎮西將軍、都督隴右諸軍事,封霸城侯,進車騎將軍、開府辟如、儀同三司。策未至,而王彌遂逼洛陽,軌遣將軍張斐、北宮純、郭敷
 等率精騎五千來衛京都。及京都陷,斐等皆沒於賊。中州避難來者日月相繼,分武威置武興郡以居之。太府主簿馬魴言於軌曰:「四海傾覆,乘輿未反,明公以全州之力徑造平陽,必當萬里風披,有征無戰。未審何憚不為此舉?」軌曰:「是孤心也。」又聞秦王入關,乃馳檄關中曰:「主上遘危,遷幸非所,普天分崩,率土喪氣。秦王天挺聖德,神武應期。世祖之孫,王今為長。凡我晉人,食土之類,龜筮克從,幽明同款。宜簡令奪奉登皇位。今遣前鋒督護宋配步騎二萬,徑至長安,翼衛乘輿,折衝左右。西中郎寔中軍三萬,武威太守張琠胡騎二萬,駱驛繼發,仲
 秋中旬會于臨晉。



 俄而秦王為皇太子,遣使拜軌為驃騎大將軍、儀同三司,固辭。秦州刺史裴苞、東羌校尉貫與據險斷使,命宋配討之。西平王叔與曹祛餘黨麴儒等劫前福祿令麴恪為主,執太守趙彞,東應裴苞。寔迴師討之,斬儒等,左督護陰預與苞戰狹西,大敗之,苞奔桑凶塢。是歲,北宮純降劉聰。皇太子遣使重申前授,固辭。左司馬竇濤言於軌曰:「曲阜周旦弗辭,營丘齊望承命,所以明國憲,厲殊勛。天下崩亂,皇輿遷幸,州雖僻遠,不忘匡衛,故朝廷傾懷,嘉命屢集。宜從朝旨,以副群心。」軌不從。



 初,寔平麴懦,徙元惡六百餘家。治中令狐瀏
 曰:「夫除惡人,猶農夫之去草,令絕其本,勿使能滋。今宜悉徙,以絕後患。」寔不納。儒黨果叛,寔進平之。



 愍帝即位,進位司空,固讓。太府參軍索輔言於軌曰:「古以金貝皮幣為貨,息穀帛量度之秏。二漢制五銖錢,通易不滯。泰始中,河西荒廢,遂不用錢。裂匹以為段數。縑布既壞,市易又難,徒壞女工,不任衣用,弊之甚也。今中州雖亂,此方主安全,宜復五銖以濟通變之會。」軌納之,立制準布用錢,錢遂大行,人賴其利。是時劉曜寇北地,軌又遣參軍麴陶領三千人衛長安。帝遣大鴻臚辛攀拜軌侍中、太尉、涼州牧、西平公,軌又固辭。



 在州十三年,寢疾,遺令曰:「
 吾無德於人,今疾病彌留,殆將命也。文武將佐咸當弘盡忠規,務安百姓,上思報國,下以寧家。素棺薄葬,無藏金玉。善相安遜,以聽朝旨。」表立子寔為世子。卒年六十。謚曰武公。



 寔字安遜,學尚明察,敬賢愛士,以秀才為郎中。永嘉初,固辭驍騎將軍,請還涼州,許之,改授議郎。及至姑臧,以討曹祛功,封建武亭侯。尋遷西中郎將,進爵福祿縣侯。建興初,除西中郎將,領護羌校尉。軌卒,州人推寔攝父位。愍帝因下策書曰:「維乃父武公,著勳西夏。頃胡賊狡猾。侵逼近甸,義兵銳卒,萬里相尋,方貢遠珍,府無虛歲。
 方委專征,蕩清九域,昊天不弔,凋餘籓后,朕用悼厥心。維爾雋劭英毅,宜世表西海。今授持節、都督涼州諸軍事、西中郎將、涼州刺史、領護羌校尉、西平公。往欽哉!其闡弘先緒,俾屏王室。」



 蘭池長趙奭上軍士張冰得璽,文曰「皇帝璽。」群僚上慶稱德,寔曰:「孤常忿袁本初擬肘,諸君何忽有此言!」因送於京師。下令國中曰:「忝紹前蹤,庶幾刑政不為百姓之患,而比年饑旱,殆由庶事有缺,竊慕箴誦之言,以補不逮。自今有面刺孤罪者,酬以束帛;翰墨陳孤過者,答以筐篚;謗言於市者,報以羊米。」賊曹佐高昌隗瑾進言曰:「聖王將舉大事,必崇三訊之法,朝
 置諫官以匡大理,疑承輔弼以補闕拾遺。今事無巨細,盡決聖慮,興軍布令,朝中不知,若有謬闕,則下無分謗。竊謂宜偃聰塞智,開納群言,政刑大小,與眾共之。若恒內斷聖心,則群僚畏威而面從矣。善惡專歸於上,雖賞千金,終無言也。」寔納之,增位三等,賜帛四十匹。遣督護王該送諸郡貢計,獻名馬方珍、經史圖籍于京師。



 會劉曜逼長安,寔遣將軍王該率眾以援京城。帝嘉之,拜都督陜西諸軍事。及帝將降于劉曜,下詔于寔曰:「天步厄運,禍降晉室,京師傾陷,先帝晏駕賊庭。朕流漂宛許,爰暨舊京。群臣以宗廟無主,歸之於朕,遂以沖眇之身託
 於王公之上。自踐寶位,四載于茲,不能翦除巨寇以救危難,元元兆庶仍遭塗炭,皆朕不明所致。羯賊劉載僭稱大號,禍加先帝,肆殺籓王,深惟仇恥,枕戈待旦。劉曜自去年九月率其蟻眾,乘虛深寇,劫質羌胡,攻沒北地。麴允總戎在外,六軍敗績,侵逼京城,矢流宮闕。胡崧等雖赴國難,殿而無效,圍塹十重,外救不至,糧盡人窮,遂為降虜。仰慚乾靈,俯痛宗廟。君世篤忠亮,勛隆西夏,四海具瞻,朕所憑賴。今進君大都督、涼州牧、侍中、司空,承制行事。瑯邪王宗室親賢,遠在江表。今朝廷播越,社稷倒懸,朕以詔王,時攝大位。君其挾贊瑯邪,共濟難運。若
 不忘主,宗廟有賴。明便出降,故夜見公卿,屬以後事,密遣黃門郎史淑、侍御史王沖齎詔假授。臨出寄命,公其勉之!」寔以天子蒙塵,沖讓不拜。



 建威將軍、西海太守張肅,寔叔父也,以京師危逼,請為先鋒擊劉曜。寔以肅年老,弗許。肅曰:「狐死首丘,心不忘本;鐘儀在晉,楚弁南音。肅受晉龍,剖符列位。羯逆滔天,朝廷傾覆,肅宴安方裔,難至不奮,何以為人臣!」寔曰:「門戶受重恩,自當闔宗效死,忠衛社稷,以申先公之志。但叔父春秋已高,氣力衰竭,軍旅之事非耆耄所堪。」乃止。既而聞京師陷沒,肅悲憤而卒。



 寔知劉曜逼遷天子,大臨三日。遣太府司馬韓
 璞、滅寇將軍田齊、撫戎將軍張閬、前鋒督護陰預步騎一萬,東赴國難。命討虜將軍陳安、故太守賈騫、隴西太守吳紹各統郡兵為璞等前驅。戒璞曰:「前遣諸將多違機信,所執不同,致有乖阻。且內不和親,焉能服物!今遣御督五將兵事,當如一體,不得令乖異之問達孤耳也。」復遺南陽王保書曰:「王室有事,不忘投軀。孤州遠域,首尾多難,是以前遣賈騫,瞻望公舉。中被符命,敕騫還軍。忽聞北地陷沒,寇逼長安,胡崧不進,麴允持金五百請救於崧,是以決遣騫等進軍度嶺。會聞朝廷傾覆,為忠不達於主,遣兵不及於難,痛慨之深,死有餘責。今更遣
 韓璞等,唯公命是從。」及璞次南安,諸羌斷軍路,相持百餘日,糧竭矢盡。璞殺駕牛饗軍,泣謂眾曰:「汝曹念父母乎?」曰:「念。」「念妻子乎?曰:「念。」「欲生還乎?」曰:「欲。」「從我令乎?」曰:「諾。」乃鼓噪進戰。會張閬率金城軍繼至,夾擊,大敗之,斬級數千。



 時焦崧、陳安寇隴石,東與劉曜相持,雍秦之人死者十八九。初,永嘉中,長安謠曰:「秦川中,血沒腕,惟有涼州倚柱觀。」至是,謠言驗矣。焦崧、陳安逼上邽,南陽王保遣使告急。以金城太守竇濤為輕車將軍。率威遠將軍宋毅及和苞、張閬、宋輯、辛韜、張選、董廣步騎二萬赴之。軍次新陽,會愍帝崩問至,素服舉哀,大臨三日。



 時南陽
 王保謀稱尊號,破羌都尉張詵言於寔曰:「南陽王忘莫大之恥,而欲自尊,天不受其圖籙,德不足以應運,終非濟時救難者也。晉王明德暱籓,先帝憑屬,宜表稱聖德,勸即尊號,傳檄諸籓,副言相府,則欲競之心息,未合之徒散矣。」從之。於是馳檄天下,推崇晉王為天子,遣牙門蔡忠奉表江南,勸即尊位。是歲,元帝即位于建鄴,改年太興,寔猶稱建興六年,不從中興之所改也。



 保聞愍帝崩,自稱晉王,建元,署置百官,遣使拜寔征西大將軍、儀同三司,增邑三千戶。俄而保為陳安所叛,氐羌皆應之。保窘迫,遂去上邽,遷祁山,寔遣將韓璞步騎五千赴難。
 陳安退保綿諸,保歸上邽。未幾,保復為安所敗,使詣寔乞師。寔遣宋毅赴之,而安退。會保為劉曜所逼,遷于桑城,將謀奔寔。寔以其宗室之望,若至河右,必動物情,遣其將陰監逆保,聲言翼衛,實禦之也。會保薨,其眾散奔涼州者萬餘人。寔自恃險遠,頗自驕恣。



 初,寔寢室梁間有人像,無頭,久而乃滅,寔甚惡之。京兆人劉弘者,挾左道,客居天梯第五山,然燈懸鏡於山穴中為光明,以惑百姓,受道者千餘人,寔左右皆事之。帳下閻沙、牙門趙仰皆弘鄉人,弘謂之曰:「天與我神璽,應王涼州。」沙、仰信之,密與寔左右十餘人謀殺寔,奉弘為主。寔潛知其謀,
 收弘殺之。沙等不之知,以其夜害寔。在位六年。私謚曰昭公,元帝賜謚曰元。子駿,年幼,弟茂攝事。



 茂字成遜,虛靖好學,不以世利嬰心。建興初,南陽王保辟從事中郎,又薦為散騎侍郎、中壘將軍,皆不就。二年,徵為侍中,以父老固辭。尋拜平西將軍、秦州刺史。太興三年,寔既遇害,州人推茂為大都督、太尉、涼州牧,茂不從,但受使持節、平西將軍、涼州牧。乃誅閻沙及黨與數百人,赦其境內。復以兄子駿為撫軍將軍、武威太守、西平公。



 歲餘,茂築靈鈞臺,周輪八十餘堵,基高九仞。武陵人閻曾夜叩門呼曰:「武公遣我來,曰:何故勞百姓而築
 臺乎?」姑臧令辛巖以曾妖妄,請殺之。茂曰:「吾信勞人。曾稱先君之令,何謂妖乎!」太府主簿馬魴諫曰:「今世駿未夷,唯當弘尚道素,不宜勞役崇飾臺榭。且比年以來,轉覺眾務日奢於往,每所經營,輕違雅度,實非士女所望於明公。」茂曰:「吾過也,吾過也!」命止作役。



 明年,劉曜遣其將劉咸攻韓璞於冀城,呼延寔攻寧羌護軍陰鑒于桑壁。臨洮人翟楷、石琮等逐令長,以縣應曜,河西大震。參軍馬岌勸茂親征,長史氾禕怒曰:「亡國之人復欲干亂大事,宜斬岌及安百姓。」岌曰:「氾公書生糟粕,刺舉近才,不惟國家大計。且朝廷旰食有年矣,今大賊自至,不煩
 遠師,遐爾之情,實繫此州,事勢不可以不出。且宜立信勇之驗,以副秦隴之望。」茂曰:「馬生之言得之矣。」乃出次石頭。茂謂參軍陳珍曰:「劉曜以乘勝之聲握三秦之銳,繕兵積年,士卒習戰,若以精騎奄剋南安,席卷河外,長驅而至者,計將何出?」珍曰:「曜雖乘威怙眾,恩德未結於下,又其關東離貳,內患未除,精卒寡少,多是氐羌烏合之眾,終不能近舍關東之難,增隴上之戍,曠日持久與我爭衡也。若二旬不退者,珍請為明公率弊卒數千以擒之。」茂大悅,以珍為平虜護軍,率卒騎一千八百救韓璞。曜陰欲引歸,聲言要先取隴西,然後迴滅桑壁。珍募
 發氐羌之眾,擊曜走之,剋復南安。茂深嘉之,拜折衝將軍。



 未幾,茂復大城姑臧,修靈鈞臺,別駕吳紹諫曰:「伏惟脩城築臺,蓋是懲既往之事。愚以為恩德未洽於近侍,雖處層樓,適所以疑諸下,徒見不安之意而失士民繫託之本心,示怯弱之形,乖匡霸之勢。遐方異境窺我之齷齱也,必有乘人之規。嘗願止役省勞,與下休息。而更興功動眾,百姓豈所望於明君哉!」茂曰:「亡兄怛然失身於物。王公設險,武夫重閉,亦達人之至戒也。且忠臣義士豈不欲盡節義於亡兄哉?直以危機密發,雖有賁育之勇,無所復施。今事未靖,不可以拘繫常言,以太平之
 理責人於迍邅之世。」紹無以對。



 茂雅有志節,能斷大事。涼州大姓賈摹,寔之妻弟也,勢傾西土。先是,謠曰:「手莫頭,圖涼州。」茂以為信,誘而殺之,於是豪右屏跡,威行涼域。永昌初,茂使將軍韓璞率眾取隴西南安之地,以置秦州。



 太寧三年卒,臨終,執駿手泣曰:「昔吾先人以孝友見稱。自漢初以來,世執忠順。今雖華夏大亂,皇輿播遷,汝當謹守人臣之節,無或失墜。吾遭擾攘之運,承先人余德,假攝此州,以全性命,上欲不負晉室,下欲保完百姓。然官非王命,位由私議,茍以集事,豈榮之哉!氣絕之日,白帢入棺,無以朝服,以彰吾志焉。」年四十八。在位五
 年。私謚曰成。茂無子,駿嗣位。



 駿字公庭,幼而奇偉。建興四年,封霸城侯。十歲能屬文,卓越不羈,而淫縱過度,常夜微行于邑里,國中化之,及統任,年十八。先是,愍帝使人黃門侍郎史淑在姑臧,左長史氾禕、右長史馬謨等諷淑,令拜駿使持節、大都督、大將軍、涼州牧、領護羌校尉西平公。赦其境內,置左右前後四率官,繕南宮。劉曜又使人拜駿涼州牧、涼王。



 時辛晏兵於枹罕,駿宴群僚於閑豫堂。命竇濤等進討辛晏。從事劉慶諫曰:「霸王不以喜怒興師,不以乾沒取勝,必須天時人事,然後起也。辛晏父子安忍凶狂,其亡
 可待,奈何以饑年大舉,猛寒攻城!昔周武迴戈以須亡殷之期,曹公緩袁氏使自斃,何獨殿下以旋兵為恥乎!」駿納之。



 遣參軍王騭聘于劉曜,曜謂之曰:「貴州必欲追蹤竇融,款誠和好,卿能保之乎?」騭曰:「不能。」曜侍中徐邈曰:「君來和同,而云不能,何也?」騭曰:「齊桓貫澤之盟,憂心兢兢,諸侯不召自至。葵丘之會,驕而矜誕,叛者九國。趙國之化,常如今日可也,若政教陵遲,尚未能察邇者之變,況鄙州乎!」曜顧謂左右曰:「此涼州高士,使乎得人。」禮而遣之。



 太寧元年,駿猶稱建興十二年,駿親耕藉田。尋承元帝崩問,駿大臨三日。會有黃龍見于胥次之嘉泉,
 右長史氾禕言於駿曰:「案建興之年,是少帝始起之號。帝以凶終,理應改易。朝廷越在江南,音問隔絕,宜因龍改號,以章休徵。」不從。初,駿之立也,姑臧謠曰:「鴻從南來雀不驚,誰謂孤雛尾翅生,高舉六翮鳳皇鳴。」至是而復收河南之地。



 咸和初,駿遣武威太守竇濤、金城太守張閬、武興太守辛巖、揚烈將軍宋輯等率眾東會韓璞,攻討秦州諸郡。曜遣其將劉胤來距,屯于狄道城。韓璞進度沃干嶺。辛巖曰:「我握眾數萬,藉氐羌之銳,宜速戰以滅之,不可以久,久則變生。」璞曰:「自夏末以來,太白犯月,辰星逆行,白虹貫日,皆變之大者,不可以輕動。輕動而
 不捷,為禍更深。吾將久而斃之。且曜與石勒相攻,胤亦不能久也。」積七十餘日,軍糧竭,遣辛巖督運於金城。胤聞之,大悅,謂其將士曰:「韓璞之眾十倍於吾,羌胡皆叛,不為之用。吾糧廩將懸,難以持久。今虜分兵運糧,可謂天授吾也。若敗辛巖,璞等自潰。彼眾我寡,宜以死戰。戰而不捷,當無匹馬得還,宜厲爾戈矛,竭汝智力。」眾咸奮。於是率騎三千,襲巖于沃干嶺,敗之,璞軍遂潰,死者二萬餘人。面縛歸罪,駿曰:「孤之罪也,將軍何辱!」皆赦之。胤乘勝追奔,濟河,攻陷令居,入據振武,河西大震。駿遣皇甫該禦之,赦其境內。



 會劉曜東討石生,長安空虛。大蒐
 講武,將襲秦雍,理曹郎中索詢諫曰:「曜雖東征,胤猶守本。險阻路遙,為主人甚易,胤若輕騎憑氐羌以距我省,則奔突難測;輟彼東合而逆戰者,則寇我未已。頃年頻出,戎馬生郊,外有饑羸,內資虛耗,豈是殿下子物之謂邪!」駿曰:「每患忠言不獻,面從背違,吾政教缺然而莫我匡者。卿盡辭規諫,深副孤之望也。」以羊酒禮之。



 西域諸國獻汗血馬、火浣布、犎牛、孔雀、巨象及諸珍異二百餘品。四域長史李柏請擊叛將趙貞,為貞所敗。議者以柏造謀致敗,請誅之。駿曰:「吾每以漢世宗之殺王恢,不如秦穆之赦孟明。」竟以減死論,群心咸悅。駿觀兵新鄉,狩
 于北野,因討軻沒虜,破之。下令境中曰:「或鯀殛而禹興,芮誅而缺進,唐帝所以殄洪災,晉侯所以成五霸。法律犯死罪,期親不得在朝。今盡聽之,唯不宜內參宿衛耳。」於是刑清國富,群僚勸駿稱涼王,領秦、涼二州牧,置公卿百官,如魏武、晉文故事。駿曰:「此非人臣所宜言也。敢有言此者,罪在不赦。」然境內皆稱之為王。群僚又請駿立世子,駿不從。中堅將軍宋輯言於駿曰:「禮急儲君者,蓋重宗廟之故。周成、漢昭立於繦褓,誠以國嗣不可曠,儲宮當素定也。昔武王始有國,元王作儲君。建興之初,先王在位,殿下正名統,況今社稷彌崇,聖躬介立,大業
 遂殷,繼貳闕然哉!臣竊以為國有累卵之危,而殿下以為安踰泰山,非所謂也。」駿納之,遂立子重華為世子。



 先是,駿遣傅穎假道于蜀,通表京師。李雄弗許。駿又遣治中從事張淳稱籓于蜀,託以假道焉。雄大悅。雄又有憾於南氐楊初,淳因說曰:「南氐無狀,屢為邊害,宜先討百頃,次平上珪。二國并勢,席卷三秦,東清許洛,掃氛燕趙,拯二帝梓宮於平陽,反皇輿於洛邑,此英霸之舉,千載一時。寡君所以遣下臣冒險通誠,不遠萬里者,以陛下義聲遠播,必能愍寡君勤王之志。天下之善一也,惟陛下圖之。」雄怒,偽許之,將覆淳於東峽。蜀人橋贊密以告
 淳。淳言於雄曰:「寡君使小臣行無迹之地、通百蠻之域、萬里表誠者,誠以陛下義矜戮力之臣,能成人之美節故也。若欲殺臣者,當顯於都市,宣示眾目,云涼州不忘舊義,通使瑯邪,為表忠誠,假途於我,主聖臣明,發覺殺之。當令義聲遠著,天下畏威。今盜殺江中,威刑不顯,何足以揚休烈,示天下也!」雄大驚曰:「安有此邪!當相放還河右耳。」雄司隸校尉景騫言於雄曰:「張淳壯士,宜留任之。」雄曰:「壯士豈為人留,且可以卿意觀之。」騫謂淳曰:「卿體大,暑熱,可且遣下吏,少住須涼。」淳曰:「寡君以皇輿幽辱,梓宮未反,天下之恥未雪,蒼生之命倒懸,故遣淳來,
 表誠大國。所論事重,非下吏能傳。若下吏所了者,則淳本亦不來,雖有火山湯海,無所辭難,豈寒暑之足避哉!」雄曰:「此人矯矯,不可得用也。」厚禮遣之。謂淳曰:「貴主英名蓋世,土險兵盛,何不稱帝,自娛一方?」淳曰:「寡君以乃祖乃父世濟忠良,未能雪天人之大恥,解眾庶之倒懸,日昃忘食,枕戈待旦。以瑯邪中興江東,故萬里翼戴,將成桓文之事,何言自娛邪!」雄有慚色,曰:「我乃祖乃父亦是晉臣,往與六郡避難此都,為同盟所推,遂有今日。瑯邪若能中興大晉於中州者,亦當率眾輔之。」淳還至龍鶴,募兵通表,後皆達京師,朝廷嘉之。



 駿議欲嚴刑峻制,
 眾咸以為宜。參軍黃斌進曰:「臣未見其可。」駿問其故。斌曰:「夫法制所以經綸邦國,篤俗齊物,既立民行,不可窪隆也。若尊者犯令,則法不行矣。」駿屏機改容曰:「夫法唯上行,制無高下。且微黃君,吾不聞過矣。黃君可謂忠之至也。」於坐擢為敦煌太守。駿有計略,於是厲操改節,勤修庶政,總御文武,咸得其用,遠近嘉詠,號曰積賢君。自軌據涼州,屬天下之亂,所在征伐,軍無寧歲。至駿,境內漸平。又使其將楊宣率眾越流沙,伐龜茲、鄯善,於是西域並降。鄯善王元孟獻女,號曰美人,立賓遐觀以處之。焉耆前部、于闐王並遣使貢方物。得玉璽於河,其文曰「
 執萬國,建無極。」



 時駿盡有隴西之地,士馬彊盛,雖稱臣於晉,而不行中興正朔。舞六佾,建豹尾,所置官僚府寺擬於王者,而微異其名。又分州西界三郡置沙州,東界六郡置河州。二府官僚莫不稱臣。又於姑臧城南築城,起謙光殿,畫以五色,飾以金玉,窮盡珍巧。殿之四面各起一殿,東曰宜陽青殿,以春三月居之,章服器物皆依方色;南曰朱陽赤殿,夏三月居之;西曰政刑白殿,秋三月居之;北曰玄武黑殿,冬三月居之。其傍皆有直省內官寺署,一同方色。及末年,任所遊處,不復依四時而居。



 咸和初,懼為劉曜所逼,使將軍宋輯、魏纂將徙隴西
 南安人二千餘家于姑臧,使聘於李雄,修鄰好。及曜工攻枹罕,護軍辛晏告急,駿使韓璞、辛巖率步騎二萬擊之,戰于臨洮,大為曜軍所敗,璞等退走,追至令居,駿遂失河南之地。初,戊己校尉趙貞不附于駿,至是,駿擊擒之,以其地為高昌郡。及石勒殺劉曜,駿因長安亂,復收河南地,至于狄道,置武衛、石門、候和、漒川、甘松五屯護軍,與勒分境。勒遣使拜駿官爵,駿不受,留其使。後懼勒強,遣使稱臣於勒,兼貢方物,遣其使歸。



 駿境內嘗大饑,穀價踴貴,市長譚詳請出倉穀與百姓,秋收三倍征之。從事陰據諫曰:「昔西門豹宰鄴,積之於人;解扁蒞東封之
 邑,計三倍。文侯以豹有罪而可賞,扁有功而可罰。今詳欲因人之饑,以要三倍,反裘傷皮,未足喻之。」駿納之。



 初,建興中,敦煌計吏耿訪到長安,既而遇賊,不得反,奔漢中,因東渡江,以太興二年至京都,屢上書,以本州未知中興,宜遣大使,乞為鄉導。時連有內難,許而未行。至是,始以訪守治書御史,拜駿鎮西大將軍,校尉、刺史、公如故,選西方人隴西賈陵等十二人配之。訪停梁州七年,以驛道不通,召還。訪以詔書付賈陵,託為賈客。到長安,不敢進,以咸和八年始達涼州。駿受詔,遣部曲督王豐等報謝,并遣陵歸,上疏稱臣,而不奉正朔,猶稱建興
 二十一年。九年,復使訪隨豐等齎印板進駿大將軍。自是每歲使命不絕。後駿遣參軍麴護上疏曰:



 東西隔塞,踰歷年載,夙承聖德,心繫本朝。而江吳寂蔑,餘波莫及,雖肆力修塗,同盟靡恤。奉詔之日,悲喜交并,天恩光被,褒崇輝渥,即以臣為大將軍、都督陜西雍秦涼州諸軍事。休寵振赫,萬里懷戴,嘉命顯至,銜感屏營。伏惟陛下天挺岐嶷,堂構晉室,遭家不造,播幸吳楚,宗廟有《黍離》之哀,園陵有殄廢之痛,普天咨嗟,含氣悲傷。臣專命一方,職在斧鉞,遐域僻陋,勢極秦隴。勒雄既死,人懷反正,謂季龍、李期之命曾不崇朝,而皆篡繼凶逆,鴟目有年。
 東西遼曠,聲援不接,遂使桃蟲鼓翼,四夷喧嘩,向義之徒更思背誕,鉛刀有干將之志,螢燭希日月之光。是以臣前章懇切,欲齊力時討。而陛下雍容江表,坐觀禍敗,懷目前之安,替四祖之業,馳檄布告,徒設空文,臣所以宵吟荒漠,痛心長路者也。且兆庶離主,漸冉經世,先老消落,後生靡識,忠良受梟懸之罰,群凶貪縱橫之利,懷君戀故,日月告流。雖時有尚義之士,畏逼首領,哀歎窮廬。臣聞少康中興,由於一旅,光武嗣漢,眾不盈百,祀夏配天,不失舊物,況以荊揚慄悍,臣州突騎,吞噬遺羯,在於掌握哉!願陛下敷弘臣慮,永念先績,敕司空鑒、征西
 亮等汎舟江沔,使首尾俱至也。



 自後駿遣使多為季龍所獲,不達。後駿又遣護羌參軍陳宇、從事徐虓、華馭等至京師,征西大將軍亮上疏言陳宇等冒險遠至,宜蒙銓敘,詔除寓西平相,虓等為縣令。永和元年,以世子重華為五官中郎將、涼州刺史。酒泉太守馬岌上言:「酒泉南山,即崑崙之體也。周穆王見西王母,樂而忘歸,即謂此山。此山有石室玉堂,珠璣鏤飾,煥若神宮。宜立西王母祠,以裨朝廷無疆之福。」駿從之。駿在位二十二年卒,時年四十,私謚曰文公,穆帝追謚曰忠成公。



 重華字泰臨,駿之第二子也。寬和懿重,沈毅少言。父卒,
 時年十六。以永和二年自稱持節、大都督、太尉、護羌校尉、涼州牧、西平公、假涼王,赦其境內。尊其母嚴氏為太王太后,居永訓宮;所生母馬氏為王太后,居永壽宮。輕賦斂,除關稅,省園囿,以恤貧窮。



 遣使奉章於石季龍。季龍使王擢、麻秋、孫伏都等侵寇不輟。金城太守張沖降于秋。於是涼州振動。重華掃境內,使其征南將軍裴恒禦之。恒壁于廣武,欲以持久弊之。牧府相司馬張耽言於重華曰:「臣聞國以兵為彊,以將為主。主將者,存亡之機,吉凶所繫。故燕任樂毅,剋平全齊,及任騎劫,喪七十城之地。是以古之明君靡不慎于將相也。今之所要,在
 於軍師。然議者舉將多推宿舊,未必妙盡精才也。且韓信之舉,非舊名也;穰宜之信,非舊將也;呂蒙之進,非舊勛也;魏延之用,非舊德也。蓋明王之舉,舉無常人,才之所能,則授以大事。今彊寇在郊,諸將不進,人情騷動,危機稍逼。主簿謝艾,兼資文武,明識兵略,若授以斧鉞,委以專征,必能折衝禦侮,殲殄凶類。」重華召艾,問以討寇方略。艾曰:「昔耿弇不欲以賊遺君父,黃權願以萬人當寇。乞假臣兵七千,為殿下吞王擢、麻秋等。」重華大悅,以艾為中堅將軍,配步騎五千擊秋。引師出振武,夜有二梟鳴於牙中,艾曰:「梟,邀也,六博得梟者勝。今梟鳴牙中,
 剋敵之兆。」於是進戰,大破之,斬首五千級。重華封艾為福祿伯,善待之。諸寵貴惡其賢,共毀譖之,乃出為酒泉太守。



 季龍又令麻秋進陷大夏,大夏護軍梁式執太守宋晏,以城應秋。秋遣晏以書誘宛戍都尉宋矩。宋矩謂秋曰:「辭父事君,當立功義;功義不立,當守名節。矩終不肯主偷生於世。」於是先殺妻子,自刎而死。



 是月,有司議遣司兵趙長迎秋西郊。謝艾以《春秋》之義,國有大喪,省蒐狩之禮,宜待踰年。別駕從事索遐議曰:「禮,天子崩,諸侯薨,末殯,五祀不行,既殯而行之。魯宣三年,天王崩,不廢郊祀。今聖上統承大位,百揆惟新,宜在璇璣玉衡以
 齊七政。立秋,萬物將成,殺氣之始,其於王事,杖麾誓眾,釁鼓禮神,所以討逆除暴,成功濟務,寧宗廟社稷,致天下之福,不可廢也。」重華從之。



 俄而麻秋進攻枹罕,時晉陽太守郎坦以城大難守,宜棄外城。武城太守張悛曰:「棄外城則大事去矣,不可以動眾心。」寧戎校尉張璩從之,固守大城。秋率眾八萬,圍塹數重,雲梯雹車,地突百道,皆通於內。城中亦應之,殺傷秋眾已數萬。季龍復遣其將劉渾等率步騎二萬會之。郎坦恨言之不從,教軍士李嘉潛與秋通,引賊千餘人上城西北隅。璩使宋脩、張弘、辛挹、郭普距之,短兵接戰,斬二百餘人,賊乃退。璩
 戮李嘉以徇,燒其攻具。秋退保大夏,謂諸將曰:「我用兵於五都之間,攻城略地,往無不捷。及登秦隴,謂有征無戰。豈悟南襲仇池,破軍殺將;築城長最,匹馬不歸;及攻此城,傷兵挫銳。殆天所贊,非人力也。」季龍聞而歎曰:「吾以偏師定九州,今以九州之力困於枹罕,真所謂彼有人焉,未可圖也。」



 重華以謝艾為使持節、軍師將軍,率步騎三萬,進軍臨河。秋以三萬眾距之。艾乘軺車,冠白,鳴鼓而行。秋望而怒曰:「艾年少書生,冠服如此,輕我也。」命黑槊龍驤三千人馳擊之。艾左右大擾。左戰帥李偉勸艾乘馬,艾不從,乃下車踞胡床,指麾處分。賊以為伏
 兵發也,懼不敢進。張瑁從左南緣河而截其後,秋軍乃退。艾乘勝奔擊,遂大敗之,斬秋將杜勳、汲魚,俘斬一萬三級,秋匹馬奔大夏。重華論功,以謝艾為太府左長史,進封福祿縣伯,邑五千戶,帛八千匹。



 麻秋又據枹罕,有眾十二萬,進屯河內,遣王擢略地晉興、廣武,越洪池嶺,至于曲柳,姑臧大震。重華議欲親出距之,謝艾固諫以為不可。別駕從事索遐進曰:「賊眾甚盛,漸逼京畿。君者,國之鎮也,不可以親動。左長史謝艾,文武兼資,國之方邵,宜委以推轂之任。殿下居中作鎮,授以算略,小賊不足平也。」重華納之,於是以艾為使持節、都督征討諸
 軍事、行衛將軍,遐為軍正將軍,率步騎二萬距之。艾建牙旗,盟將士,有西北風吹旌旗東南指。遐曰:「風為號令,今能令旗指之,天所贊也,破之必矣。」軍次神鳥,王擢與前鋒戰,敗,遁還河南。還討叛虜斯骨真萬餘落,破之,斬首千餘級,俘擒二千八百,獲牛羊十餘萬頭。



 重華自以連破勍敵,頗怠政事,希接賓客。司直索遐諫曰:「殿下承四聖之基,當升平之會,荷當今之任,憂率土之塗炭。宜躬親萬機,開延英乂,夙夜乾乾,勉於庶政。自頃內外囂然,皆云去賊投誠者應即撫慰,而彌日不接。國老朝賢,當虛己引納,詢訪政事,比多經旬積朔,不留意接之。文
 奏入內,歷月不省,廢替見務,注情於棋弈之間,繾綣左右小臣之娛,不存將相遠大之謀。至使親臣不言,朝吏杜口,愚臣所以迴惶忘寢與食也。今王室如毀,百姓倒懸,正是殿下銜膽茹辛厲心之日。深願垂心朝政,延納直言,周爰五美,以成六德,捐彼近習,弭塞外聲,修政聽朝,使下觀而化。」重華覽之大悅,優文答謝,然不之改也。



 詔遣侍御史俞歸拜重華護羌校尉、涼州刺史、假節。是時石季龍西中郎將王擢屯結隴上,為苻雄所破,奔重華。重華厚寵之,以為征虜將軍、秦州刺史、假節,使張弘、宗悠率步騎萬五千配擢,伐苻健。健遣苻碩禦之,戰于
 龍黎。擢等大敗,單騎而還,弘、悠皆沒。重華痛之,素服為戰亡吏士舉哀號慟,各遣弔問其家。復授擢兵,使攻秦州,剋之。遣使上疏曰:「季龍自斃,遺燼游魂,取亂侮亡,睹機則發。臣今遣前都鋒督裴恒步騎七萬,遙出隴上,以俟聖朝赫然之威。山東騷擾不足厝懷,長安膏腴,宜速平蕩。臣守任西荒,山川悠遠,大誓六軍,不及聽受之末;猛將鷹揚,不豫告成之次,瞻雲望日,孤憤義傷,彈劍慷慨,中情蘊結。」於是康獻皇后詔報,遣使進重華為涼州牧。



 是時御史俞歸至涼州,重華方謀為涼王,不肯受詔,使親信人沈猛謂歸曰:「我家主公奕世忠於晉室,而不
 如鮮卑矣。臺加慕容皝燕王,今甫授州主大將軍,何以加勸有功忠義之臣乎!明臺今宜移河右,共勸州主為涼王。大夫出使,茍利社稷,專之可也。」歸對曰:「王者之制,異姓不得稱王;九州之內,重爵不得過公。漢高一時王異姓,尋皆誅滅,蓋權時之宜,非舊體也。故王陵曰:『非劉氏而王,天下共伐之。』至於戎狄,不從此例。春秋時吳楚稱王,而諸侯不以為非者,蓋蠻夷畜之也。假令齊魯稱王,諸侯豈不伐之!故聖上以貴公忠賢,是以爵以上公,位以方伯,鮮卑北狄,豈足為比哉!子失問也。且吾又聞之,有殊勳絕世者亦有不世之賞,若今便以貴公為王
 者,設貴公以河右之眾南平巴蜀,東掃趙魏,修復舊都,以迎天子,天子復以何爵何位可以加賞?幸三思之。」猛具宣歸言,重華遂止。



 重華好與群小遊戲,屢出錢帛以賜左右。徵事索振諫曰:「先王寢不安席,志平天下,故繕甲兵,積資實。大業未就,懷恨九泉。殿下遭巨寇於諒闇之中,賴重餌以挫勍敵。今遺燼尚廣,倉帑虛竭,金帛之費,所宜慎之。昔世祖即位,躬親萬機,章奉詣闕,報不終日,故能隆中興之業,定萬世之功。今章奉停滯,動經時月,下情不得上達,哀窮困於囹圄,蓋非明主之事,臣竊未安。」重華善之。



 將受詔,未及而卒,時年二十七。在位十
 一年。私謚曰昭公,後改曰桓公,穆帝賜謚曰敬烈。子耀靈嗣。



 耀靈字元舒。年十歲嗣事,稱大司馬、校尉、刺史、西平公。伯父長寧侯祚性傾巧,善承內外,初與重華寵臣趙長、尉緝等結異姓兄弟。長等矯稱重華遺令,以祚為持節、督中外諸軍、撫軍將軍,輔政。長待議以耀靈沖幼,時難未夷,宜立長君。祚先烝重華母馬氏,馬氏遂從緝議,命廢耀靈為涼寧侯而立祚。祚尋使楊秋胡害耀靈於東苑,埋之於沙坑,私謚曰哀公。



 祚字太伯,博學雄武,有政事之才。既立,自稱大都督、大將軍、涼州牧、涼公。淫暴不道,又通重華妻裴氏,自閣內媵妾及駿、重華未嫁子女,無不暴亂,國人相目,咸賦《牆茨》之詩。



 永和十年。祚納尉緝、趙長等議,僭稱帝位,立宗廟,舞八佾,置百官,下書曰:「昔金行失馭,戎狄亂華,胡、羯、氐、羌咸懷竊璽。我武公以神武撥亂,保寧西夏,貢款勤王,旬朔不絕。四祖承光,忠誠彌著。往受晉禪,天下所知,謙沖遜讓,四十年于茲矣。今中原喪亂,華裔無主,群后僉以九州之望無所依歸,神祇嶽瀆罔所憑係,逼孤攝行大統,以一四海之心。辭不獲已,勉從群議。待掃穢二
 京,蕩清周魏,然後迎帝舊都,謝罪天闕,思與兆庶,同茲更始。」改建興四十二年為和平元年,赦殊死,賜鰥寡帛,加文武爵各一級,追崇曾祖軌為武王,祖寔為昭王,從祖茂為成王,父駿為文王,弟重華為明王。立妻辛氏為皇后,弟天錫為長寧王,子泰和為太子,庭堅為建康王,耀靈弟玄靚為涼武侯。其夜,天有光如車蓋,聲若雷霆,震動城邑。明日,大風拔木。災異屢見,而祚凶虐愈甚。其尚書馬岌以切諫免官。郎中丁琪又諫曰:「先公累執忠節,遠宗吳會,持盈守謙,五十作載,蒼生所以鵠企西望,四海所以注心大涼,皇天垂贊,士庶效死者,正以先公
 道高彭昆,忠踰西伯,萬里通虔,任節不貳故也。能以一州之眾抗崩天之虜,師徒歲起,人不告疲。陛下雖以大聖雄姿纂戎鴻緒,勳德未高於先公,而行革命之事,臣竊未見其可。華夷所以歸系大涼、義兵所以千里響赴者,以陛下為本朝之故。今既自尊,人斯高競,一隅之地何以當中國之師!城峻衝生,負乘致寇,惟陛下圖之。」祚大怒,斬之于闕下。遣其將和昊率眾伐麗靬戎於南山,大敗而還。



 太尉桓溫入關,王擢時鎮隴西,馳使於祚,言溫善用兵,勢在難測。祚既震懼,又慮擢反噬,即召馬岌復位而與之謀。密遣親人刺擢,事覺,不剋。祚益懼,大聚
 眾,聲言東征,實欲西保敦煌。會溫還而止。更遣其平東將軍秦州刺史牛霸、司兵張芳率三千人擊擢,破之。擢奔于苻健。其國中五月霜降,殺苗稼果實。



 祚宗人張瓘時鎮枹罕,祚惡其彊,遣其將易揣、張玲率步騎萬三千以襲之。時張掖人王鸞頗知神道,言於祚曰:「軍出不復還,涼國將有不利矣。」祚大怒,以鸞妖言沮眾,斬之以徇,三軍乃發。鸞臨刑曰:「我死不二十日,軍必敗。」時有神降於玄武殿,自稱玄冥,與人交語。祚日夜祈之,神言與之福利,祚甚信之。祚又遣張掖太守索孚代瓘鎮枹罕,為瓘所殺。玲等濟河未畢,又為瓘兵所破。仍舊單騎奔走,瓘
 軍躡之。祚眾震懼。敦煌人宋混與弟澄等聚眾以應瓘。趙長、張璹等懼罪,入閣呼重華母馬氏出殿,拜耀靈庶弟玄靚為主。揣等率眾入殿伐長,殺之。瓘弟琚及子嵩募數百市人,揚聲言:「張祚無道,我兄大軍已到城東,敢有舉手者誅三族。」祚眾披散。琚、嵩率眾入城,祚按劍殿上,大呼,令左右死戰。祚既失眾心,莫有鬥志,於是被殺。梟其首,宣示內外,暴尸道左,國內咸稱萬歲。祚篡立三年而亡。



 玄靚字元安。既立,自號大都督、大將軍、校尉、涼州牧、西平公,赦其國內,廢和平之號,復稱建興四十三年。誅祚
 二子,以張瓘為衛將軍,領兵萬人,行大將軍事,改易僚屬。



 有隴西人李儼,誅大姓彭姚,自立於隴右,奉中興年號,百姓悅之。玄靚遣牛霸率眾討之,未達,而西平人衛綝又據郡叛。霸眾潰,單騎而還。瓘先欲征綝、以兄珪在綝中為疑,綝亦以弟在瓘中,故彼我經年不相伐。西平人郭勛解天文,不應州郡之命,綝禮聘之。勛曰:「張氏應衰,衛氏當興,豈得以一弟而滅一門,宜速伐瓘。」綝將從之。瓘遣弟琚領大眾征綝敗之。西平田旋要酒泉太守馬基背瓘應綝,旋謂基曰:「綝擊其東,我等絕其西,不六旬,天下可定,斯閉口捕舌也。」基許之。瓘遣司馬張姚、王
 國將二千人伐基,敗之,斬基、旋二人之首,傳姑臧。



 瓘兄弟彊盛,負其勳力,有篡立之謀。輔國宋混與弟澄共討瓘,盡夷其屬,玄靚以混為都督中外諸軍事、車騎大將軍、假節,輔政。混卒,又以澄代之。玄靚右司馬張邕惡澄專擅,殺之。遂滅宋氏,玄靚乃以邕為中護軍,叔父天錫為中領軍,共輔政。



 邕自以功大,驕矜淫縱,又通馬氏,樹黨專權。國人患之。天錫腹心郭增、劉肅二人,並年十八九,因寢,謂天錫曰:「天下事欲未靜。」天錫曰:「何謂也?」二人曰:「今護軍出入,有似長寧。」天錫大驚曰:「我早疑之,未敢出口。計當云何?」肅曰:「政當速除之耳。」天錫曰:「安得其人?」
 肅曰:「肅即是也。」天錫曰:「汝年少,更求可與謀者。」肅曰:「趙白駒及肅二人足以辦之矣。」於是天錫從兵四百人,與邕俱入朝,肅與白駒剔刀鞘出刃,從天錫入。值邕於門下,肅斫之不中,白駒繼之,又不剋,二人與天錫俱入禁中。邕得逸走,因率甲士三百餘人反攻禁門。天錫上屋大呼,謂將士曰:「張邕凶逆,所行無道,諸宋何罪,盡誅滅之?傾覆國家,肆亂社稷。我不惜死,實懼先人廢祀,事不獲已故耳。我家門戶事,而將士豈可以干戈見向!今之所取,邕身而已。天地有靈,吾不食言。」邕眾聞之,悉散走,邕以劍自刎而死。於是悉誅邕黨。



 玄靚年既幼沖,性又
 仁弱,天錫既剋邕,專掌朝政,改建興四十九年,奉升平之號。興寧元年,駿妻馬氏卒,玄靚以其庶母郭氏為太妃。郭氏以天錫專政,與大臣張欽等謀討之。事泄,欽等伏法。是歲,天錫率眾入禁門,潛害玄靚,宣言暴薨,時年十四。在位九年。私謚曰沖公,孝武帝賜謚曰敬悼公。



 天錫字純嘏,駿少子也,小名獨活。初字公純嘏,入朝,人笑其三字,因自改焉。玄靚死,國人立之,自號大將軍、校尉、涼州牧、西平公。遣司馬綸騫奉章請命,并送御史俞歸還京都。太和初,詔以天錫為大將軍、大都督、督隴右關中諸軍事、護羌校尉、涼州刺史、西平公。



 天錫數宴園
 池,政事頗廢。盪難將軍、校書祭酒索商上疏極諫,天錫答曰:「吾非好行,行有得也。觀朝榮,則敬才秀之士;玩芝蘭,則愛德行之臣;睹松竹,則思貞操之賢;臨清流,則貴廉潔之行;覽蔓草,則賤貪穢之吏;逢飆風,則惡凶狡之徒。若引而申之,觸類而長之,庶無遺漏矣。」



 羌廉岐自稱益州刺史,率略陽四千家背苻堅就李儼。天錫自往討之,以別駕楊遹為監前鋒軍事、前將軍,趣金城。晉興相常據為使持節、征東將軍,向左南,游擊將軍張統出白土,天錫自率三萬人次倉松,伐儼。儼大敗,入城固守,遣子純求救於苻堅。堅使其將王猛救之。天錫敗績,死者
 十二三,天錫乃還。立子大懷為世子。



 自天錫之嗣事也,連年地震山崩,水泉湧出,柳化為松,火生泥中。而天錫荒于聲色,不恤政事。初,安定梁景、敦煌劉肅並以門胄,總角與天錫友暱。張邕之誅,肅、景有勳,天錫深德之賜姓張氏,又改其字,以為己子。天錫諸子皆以大為字,故景曰大奕,肅曰大誠。廢大懷為高昌公,更立嬖子大豫為世子,景、肅等俱參政事。人情怨懼,從弟從事中郎憲切諫,不納。



 時苻堅彊盛,每攻之,兵無寧歲。天賜甚懼,乃立壇刑牲,率典軍將軍張寧、中堅將軍馬芮等,遙與晉三公盟誓,獻書大司馬桓溫,剋六年夏誓同大舉。遣從
 事中郎韓博、奮節將軍康妙奉表,并送盟文。博有口才,溫甚稱之。嘗大會,溫使司馬刁彞嘲之,彞謂博曰:「君是韓盧後邪?」博曰:「卿是韓盧後。」溫笑曰:「刁以君姓韓,故相問焉。他自姓刁,那得韓盧後邪!」博曰:「明公脫未之思,短尾者則為刁也。」一坐推歎焉。



 太元元年,苻堅遣其將茍萇、毛當、梁熙、姚萇來寇,渡石城津。天錫集議,中錄事席仂曰:「先公既有故事,徐思後變,此孫仲謀屈伸之略也。」眾以仂為老怯,咸曰:「龍驤將軍馬達,精兵萬人距之,必不敢進。」廣武太守辛章保城固守。章與晉興相彭知正、西平相趙疑謀曰:「馬達出於行陣,必不為用,則秦軍深
 入。吾相與率三郡精卒,斷其糧運,決一朝命矣。」征東常據亦欲先擊姚萇,須天錫命。天錫率萬人頓金昌城。馬達萬人逆萇等,因請降,兵人散走。常據、席仂皆戰死。司兵趙充哲與萇苦戰,又死。中衛將軍史景亦沒于陣。天錫大懼,出城自戰,城內又反。天錫窘逼,降于萇等。初,天錫所居安昌門及平章殿無故而崩,旬日而國亡。即位凡十三年。自軌為涼州,至天錫,凡九世,七十六年矣。苻堅先為天錫起宅,至,以為尚書,封歸義侯。



 堅大敗于淮肥時,天錫為苻融征南司馬,於陣歸國。詔曰:「昔孟明不替,終顯厥功,豈以一眚而廢才用!其以天錫為散騎
 常侍、左員外。」又詔曰:「故太尉、西平公張軌著德遐域,世襲前勞。彊兵縱害,遂至失守。散騎常侍天錫拔迹登朝,先祀淪替,用增矜慨,可復天錫西平郡公爵。」俄拜金紫光祿大夫。



 天錫少有文才,流譽遠近。及歸朝,甚被恩遇。朝士以其國破身虜,多共毀之。會稽王道子嘗問其西土所出,天錫應聲曰:「桑葚甜甘,鴟鴞革響,乳酪養性,人無妒心。」後形神昏喪,雖處列位,不復被齒遇。隆安中,會稽世子元顯用事,常延致之,以為戲弄。以其家貧,拜廬江太守,本官如故。桓玄時,欲招懷四遠,乃用天錫為護羌校尉、涼州刺史。尋卒,年六十一。追贈金紫光祿大夫。



 史臣曰:長河外區,流沙作紀,玉關懸險,金城負固,有苗攸竄,帝舜投而不羈;渠搜是居,大禹即而方敘。世逢多難,嬰五郡以誰何;時遇兵凶,阻三邊而高視。雖非久安之地,足為茍全之所乎!周公保之而立功,士彥擁之布延世。摯虞觀象,記洪災之不流;侯瑾覘泉,知霸者之斯在。匪唯地勢,抑亦有天道歙!茂、駿、重華資忠踵武,崎嶇僻陋,無忘本朝,故能西控諸戎,東攘巨猾,綰累葉之珪組,賦絕域之琛賨,振曜遐荒,良由杖順之效矣。祚以卑孽,陰傾塚嗣,播有茨於彤管,擬宸居於黑山,丁琪以切諫遇誅夷,王鸞以讜言嬰顯戮,境內雲據,仇其竊名,卒
 致梟懸,自然之理也。純嘏微弱,竟亡其眾。奉身魏闕,齒跡朝流,再襲銀黃,祖德之延慶矣。



 贊曰:三象構氛,九土瓜分。鼎遷江介,地絕河濆。歸誠晉室,美矣張君。內撫遺黎,外攘逋寇。世既綿遠,國亦完富。杖順為基,蓋天所佑。



\end{pinyinscope}