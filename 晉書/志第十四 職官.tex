\article{志第十四 職官}

\begin{pinyinscope}

 職官



 《書》曰:「
 唐虞稽古,建官惟百。」所以獎導民萌,裁成庶政。《易》曰:「天垂象,聖人則之。」執法在南宮之右,上相處端門之外,而鳥龍居位,雲火垂名,前史詳之,其以尚矣。黃帝置三公之秩,以親黎元,少昊配九扈之名,以為農正,命重黎於天地,詔融冥於水火,則可得而言焉。伊尹曰:「三公調陰陽,九卿通寒暑,大夫知人事,列士去其私。」而成湯
 居亳,初置二相,以伊尹、仲虺為之,凡厥樞會,仰承君命。總及周武下車,成康垂則,六卿分職,二公弘化,咸樹司存,各題標準,茍非其道,人弗虛榮。貽厥孫謀,其固本也如此。及秦變周官,漢遵嬴舊,或隨時適用,或因務遷革,霸王之典,義在於斯,既獲厥安,所謂得其時制者也。四征興於漢代,四安起於魏初,四鎮通於柔遠,四平止於喪亂,其渡遼、凌江,輕車、強弩,式揚遐外,用表攻伐,興而復毀,厥號彌繁。及當塗得志,剋平諸夏,初有軍師祭酒,參掌戎律。建安十三年,罷漢台司,更置丞相,而以曹公居之,用兼端揆。孫吳、劉蜀,多依漢制,雖復臨時命氏,而
 無忝舊章。世祖武皇帝即位之初,以安平王孚為太宰,鄭沖為太傅,王祥為太保,司馬望為太尉,何曾為司徒,荀顗為司空,石苞為大司馬,陳騫為大將軍,世所謂八公同辰,攀雲附翼者也。若乃成乎棟宇,非一枝之勢;處乎經綸,稱萬夫之敵。或牽羊以葉於夢,或垂釣以申其道,或空桑以獻其術,或操版以啟其心。臥龍飛鴻,方金擬璧,秦奚、鄭產,楚材晉用,斯亦曩時之良具,其又昭彰者焉。宣王既誅曹爽,政由己出,網羅英俊,以備天官。及蘭卿受羈,貴公顯戮,雖復策名魏氏,而乃心皇晉。及文王纂業,初啟晉臺,始置二衛,有前驅養由之弩;及設三
 部,有熊渠佽飛之眾。是以武帝龍飛,乘茲奮翼,猶武王以周之十亂而理殷民者也。是以泰始盡於太康,喬柯茂葉,來居斯位;自太興訖于建元,南金北銑,用處茲秩。雖未擬乎夔拊龍言,天工人代,亦庶幾乎任官惟賢,蒞事惟能者也。



 丞相、相國,並秦官也。晉受魏禪,並不置,自惠帝之後,省置無恒。為之者,趙王倫、梁王肜、成都王穎、南陽王保、王敦、王導之徒,皆非復尋常人臣之職。



 太宰、太傅、太保,周之三公官也。魏初唯置太傅,以鐘繇為之,末年又置太保,以鄭沖為之。晉初以景帝諱故,又
 採《周官》官名,置太宰以代太師之任,秩增三司,與太傅太保皆為上公,論道經邦,燮理陰陽,無其人則闕。以安平獻王孚居之。自渡江以後,其名不替,而居之者甚寡。



 太尉、司徒、司空,並古官也。自漢歷魏,置以為三公。及晉受命,迄江左,其官相承不替。



 大司馬,古官也。漢制以冠大將軍、驃騎、車騎之上,以代太尉之職,故恒與太尉迭置,不並列。及魏有太尉,而大司馬、大將軍各自為官,位在三司上。晉受魏禪,因其制,以安平王孚為太宰,鄭沖為太傅。王祥為太保,義陽王望為太尉,何曾為司徒,荀顗為司空,石苞為大司馬,陳
 騫為大將軍,凡八公同時並置,唯無丞相焉。自義陽王望為大司馬之後,定令如舊,在三司上。



 大將軍,古官也。漢武帝置,冠以大司馬名,為崇重之職。及漢東京,大將軍不常置,為之者皆擅朝權。至景帝為大將軍,亦受非常之任。後以叔父孚為太尉,奏改大將軍在太尉下。及晉受命,猶依其制,位次三司下,後復舊,在三司上。太康元年,瑯邪王伷遷大將軍,復制在三司下,伷薨後如舊。



 開府儀同三司,漢官也。殤帝延平元年,鄭騭為車騎將軍,儀同三司;儀同之名,始自此也。及魏黃權以車騎將
 軍開府儀同三司;開府之名,起於此也。



 驃騎、車騎、衛將軍、伏波、撫軍、都護、鎮軍、中軍、四征、四鎮、龍驤、典軍、上軍、輔國等大將軍,左右光祿、光祿三大夫,開府者皆為位從公。



 太宰、太傅、太保、司徒、司空、左右光祿大夫、光祿大夫,開府位從公者為文官公,冠進賢三梁,黑介幘。



 大司馬、大將軍、太尉、驃騎、車騎、衛將軍、諸大將軍,開府位從公者為武官公,皆著武冠,平上黑幘。



 文武官公,皆假金章紫綬,著五時服。其相國、丞相,皆袞冕,綠盭綬,所以殊於常公也。



 諸公及開府位從公者,品秩第一,食奉日五斛。太康二年,又給絹,春百匹,秋絹二百匹,綿二百斤。元康元年,給菜田十頃,田騶十人,立夏後不及田者,食奉一年。置長史一人,秩一千石;西東閣祭酒、西東曹掾、戶倉賊曹令史屬各一人;御屬閣下令史、西東曹倉戶賊曹令史、門令史、記室省事令史、閣下記室書令史、西東曹學事各一人。給武賁二十人,持班劍。給朝車駕駟、安車黑耳駕三各一乘,祭酒掾屬白蓋小車七乘,軺車施耳後戶、皁輪犢車各一乘。自祭酒已下,令史已上,皆皁零辟朝服。太尉雖不加兵者,吏屬皆絳服。司徒加置左右長史各一
 人,秩千石;主簿、左西曹掾屬各一人,西曹稱右西曹,其左西曹令史已下人數如舊令。司空加置導橋掾一人。



 諸公及開府位從公加兵者,增置司馬一人,秩千石;從事中郎二人,秩比千石;主簿、記室督各一人;舍人四人;兵鎧、士曹,營軍、刺姦、帳下都督,外都督,令史各一人。主簿已下,令史已上,皆絳服。司馬給吏卒如長史,從事中郎給侍二人,主簿、記室督各給侍一人。其餘臨時增崇者,則褒加各因其時為節文,不為定制。



 諸公及開府位從公為持節都督,增參車為六人,長史、司馬、從事中郎、主簿、記室督、祭酒、掾屬、舍人如常加兵
 公制。



 特進,漢官也。二漢及魏晉以加官從本官車服。無吏卒。太僕羊琇遜位,拜特進,加散騎常侍,無餘官,故給吏卒車服。其餘加特進者,唯食其祿賜,位其班位而已,不別給特進吏卒車服,後定令。特進品秩第二,位次諸公,在開府驃騎上,冠進賢兩梁,黑介幘,五時朝服,佩水蒼玉,無章綬,食奉日四斛。太康二年,始賜春服絹五十匹,秋絹百五十匹,綿一百五十斤。元康元年,給菜田八頃,田騶八人,立夏後不及田者,食奉一年。置主簿、功曹史、門亭長、門下書佐各一人,給安車黑耳駕御一人,軺車施
 耳後戶一乘。



 左右光祿大夫,假金章紫綬。光祿大夫加金章紫綬者,品秩第二,祿賜、班位、冠幘、車服、佩玉,置吏卒羽林及卒,諸所賜給皆與特進同。其以為加官者,唯假章綬、祿賜班位而已,不別給車服吏卒也。又卒贈此位,本已有卿官者,不復重給吏卒,其餘皆給。



 光祿大夫假銀章青綬者,品秩第三,位在金紫將軍下,諸卿上。漢時所置無定員,多以為拜假賵贈之使,及監護喪事。魏氏已來,轉復優重,不復以為使命之官。其諸公告老者,皆家拜此位;及在朝顯職,復用加之,及晉受
 命,仍舊不改,復以為優崇之制。而諸公遜位,不復加之,或更拜上公,或以本封食公祿。其諸卿尹中朝大官年老致仕者,及內外之職加此者,前後甚眾。由是或因得開府,或進加金章紫綬,又復以為禮贈之位。泰始中,唯太子詹事楊珧加給事中光祿大夫。加兵之制,諸所供給依三品將軍。其餘自如舊制,終武、惠、孝懷三世。



 光祿大夫與卿同秩中二千石,著進賢兩梁冠,黑介幘,五時朝服,佩水蒼玉,食奉日三斛。太康二年,始給春賜絹五十匹,秋絹百匹,綿百斤。惠帝元康元年,始給菜田六頃,田騶六人,置主簿、功曹史、門亭長、門下書佐各一
 人。



 驃騎已下及諸大將軍不開府非持節都督者,品秩第二,其祿與特進同。置長史、司馬各一人,秩千石;主簿,功曹史,門下督,錄事,兵鎧士賊曹,營軍、刺姦、帳下都督,功曹書佐門吏,門下書吏各一人。其假節為都督者,所置與四征鎮加大將軍不開府為都督者同。



 四征鎮安平加大將軍不開府、持節都督者,品秩第二,置參佐吏卒,幕府兵騎如常都督制,唯朝會祿賜從二品將軍之例。然則持節、都督無定員,前漢遣使始有持節。光武建武初,征伐四方,始權時置督軍御史,事竟罷。建安中,魏武
 為相,始遣大將軍督之。二十一年,征孫權還,夏侯惇督二十六軍是也。魏文帝黃初三年,始置都督諸州軍事,或領刺史。又上軍大將軍曹真都督中外諸軍事、假黃鉞,則總統內外諸軍矣。魏明帝太和四年秋,宣帝征蜀,加號大都督。高貴鄉公正元二年,文帝都督中外諸軍,尋加大都督。及晉受禪,都督諸軍為上,臨諸軍次之,督諸軍為下;使持節為上,持節次之,假節為下。使持節得殺二千石以下;持節殺無官位人,若軍事,得與使持節同;假節唯軍事得殺犯軍令者。江左以來,都督中外尤重,唯王導等權重者乃居之。



 三品將軍秩中二千石者,著武冠,平上黑幘,五時朝服,佩水蒼玉,食奉、春秋賜綿絹、菜田、田騶如光祿大夫諸卿制。置長史、司馬各一人,秩千石;主簿,功曹,門下都督,錄事,兵鎧士賊曹,營軍、刺姦吏、帳下都督,功曹書佐門吏,門下書吏各一人。



 錄尚書,案漢武時,左右曹諸吏分平尚書奏事,知樞要者始領尚書事。張安世以車騎將軍,霍光以大將軍,王鳳以大司馬,師丹以左將軍並領尚書事。後漢章帝以太傅趙憙、太尉牟融並錄尚書事。尚書有錄名,蓋自憙、融始,亦西京領尚書之任,猶唐虞大麓之職也。和帝時,
 太尉鄧彪為太傅,錄尚書事,位上公,在三公上,漢制遂以為常,每少帝立則置太傅錄尚書事,猶古塚宰總己之義,薨輒罷之。自魏晉以後,亦公卿權重者為之。



 尚書令,秩千石,假銅印墨綬,冠進賢兩梁冠,納言幘,五時朝服,佩水蒼玉,食奉月五十斛。受拜則策命之,以在端右故也。太康二年,始給賜絹,春三十匹,秋七十匹。綿七十斤。元康元年,始給菜田六頃,田騶六人,立夏後不及田者,食奉一年。始賈充為尚書令,以目疾表置省事吏四人,省事蓋自此始。



 僕射,服秩印綬與令同。案漢本置一人,至漢獻帝建安
 四年,以執金吾榮郃為尚書左僕射,僕射分置左右,蓋自此始。經魏至晉,迄於江左,省置無恒,置二,則為左右僕射,或不兩置,但曰尚書僕射。令闕,則左為省主;若左右並闕,則置尚書僕射以主左事。



 列曹尚書,案尚書本漢承秦置,及武帝遊宴後庭,始用宦者主中書,以司馬遷為之,中間遂罷其官,以為中書之職。至成帝建始四年,罷中書宦者,又置尚書五人,一人為僕射,而四人分為四曹,通掌圖書秘記章奏之事,各有其任。其一曰常侍曹,主丞相御史公卿事。其二曰二千石曹,主刺史郡國事。其三曰民曹,主吏民上書事。
 其四曰主客曹,主外國夷狄事。後成帝又置三公曹,主斷獄,是為五曹。後漢光武以三公曹主歲盡考課諸州郡事,改常侍曹為吏部曹,主選舉祠祀事,民曹主繕修功作鹽池園苑事,客曹主護駕羌胡朝賀事,二千石曹主辭訟事,中都官曹主水火盜賊事,合為六曹。并令僕二人,謂之八座。尚書雖有曹名,不以為號。靈帝以侍中梁鵠為選部尚書,於此始見曹名。及魏改選部為吏部,主選部事,又有左民、客曹、五兵、度支、凡五曹尚書、二僕射、一令為八座。及晉置吏部、三公、客曹、駕部、屯田、度支六曹、而無五兵。咸寧二年,省駕部尚書。四年,省一僕射,
 又置駕部尚書。太康中,有吏部、殿中及五兵、田曹、度支、左民為六曹尚書,又無駕部、三公、客曹。惠帝世又有右民尚書,止於六曹,不知此時省何曹也。及渡江,有吏部、祠部、五兵、左民、度支五尚書。祠部尚書常與右僕射通職,不恒置,以右僕射攝之,若右僕射闕,則以祠部尚書攝知右事。



 左右丞,自漢武帝建始四年置尚書,而便置丞四人。及光武始減其二,唯置左右丞,左右丞蓋自此始也。自此至晉不改。晉左丞主臺內禁令,宗廟祠祀,朝儀禮制,選用署吏,急假;右丞掌臺內庫藏廬舍,凡諸器用之物,及
 廩振人租布,刑獄兵器,督錄遠道文書章表奏事。八座郎初拜,皆沿漢舊制,並集都座交禮,遷職又解交焉。



 尚書郎,西漢舊置四人,以分掌尚書。其一人主匈奴單于營部,一人主羌夷吏民,一人主戶口墾田,一人主財帛委輸。及光武分尚書為六曹之後,合置三十四人,秩四百石,并左右丞為三十六人。郎主作文書起草,更直五日於建禮門內。尚書郎初從三署詣臺試,守尚書郎,中歲滿稱尚書郎,三年稱侍郎,選有吏能者為之。至魏,尚書郎有殿中、吏部、駕部、金部、虞曹、比部、南主客、祠部、度支、庫部、農部、水部、儀曹、三公、倉部、民曹、二千石、中兵、
 外兵、都兵、別兵、考功、定課,凡二十三郎。青龍二年,尚書陳矯奏置都官、騎兵,合凡二十五郎。每一郎缺,白試諸孝廉能結文案者五人,謹封奏其姓名以補之。及晉受命,武帝罷農部、定課,置直事、殿中、祠部、儀曹、吏部、三公、比部、金部、倉部、度支、都官、二千石、左民、右民、虞曹、屯田、起部、水部、左右主客、駕部、車部、庫部、左右中兵、左右外兵、別兵、都兵、騎兵、左右士、北主客、南主客,為三十四曹郎。後又置運曹,凡三十五曹,置郎二十三人,更相統攝。及江左,無直事、右民、屯田、車部、別兵、都兵、騎兵、左右士、運曹十曹郎。康穆以後,又無虞曹、二千石二郎,但有殿
 中、祠部、吏部、儀曹、三公、比部、金部、倉部、度支、都官、左民、起部、水部、主客、駕部、庫部、中兵、外兵十八曹郎。後又省主客、起部、水部,餘十五曹云。



 侍中,案黃帝時風后為侍中,於周為常伯之任,秦取古名置侍中,漢因之。秦漢俱無定員,以功高者一人為僕射。魏晉以來置四人,別加官者則非數。掌儐贊威儀,大駕出則次直侍中護駕,正直侍中負璽陪乘,不帶劍,餘皆騎從。御登殿,與散騎常侍對扶,侍中居左,常侍居右。備切問近對,拾遺補闕。及江左哀帝興寧四年,桓溫奏省二人,後復舊。



 給事黃門侍郎,秦官也。漢已後並因之,與侍中俱管門下眾事,無員。及晉,置員四人。



 散騎常侍,本秦官也。秦置散騎,又置中常侍,散騎騎從乘輿車後,中常侍得入禁中,皆無員,亦以為加官。漢東京初,省散騎,而中常侍用宦者。魏文帝黃初初,置散騎,合之於中常侍,同掌規諫,不典事,貂榼插右,騎而散從,至晉不改。及元康中,惠帝始以宦者董猛為中常侍,後遂止。常為顯職。



 給事中,秦官也。所加或大夫、博士、議郎,掌顧問應對,位次中常侍。漢因之。及漢東京省,魏世復置,至晉不改。在
 散騎常侍下,給事黃門侍郎上,無員。



 通直散騎常侍,案魏末散騎常侍又有在員外者。泰始十年,武帝使二人與散騎常侍通員直,故謂之通直散騎常侍。江左置四人。



 員外散騎常侍,魏末置,無員。



 散騎侍郎四人,魏初與散騎常侍同置。自魏至晉,散騎常侍、侍郎與侍中、黃門侍郎共平尚書奏事,江左乃罷。



 通直散騎侍郎四人。初,武帝置員外散騎侍郎,及太興元年,元帝使二人與散騎侍郎通員直,故謂之通直散騎侍郎,後增為四人。



 員外散騎侍郎,武帝置,無員。



 奉朝請,本不為官,無員。漢東京罷三公、外戚、宗室、諸侯多奉朝請。奉朝請者,奉朝會請召而已。武帝亦以宗室、外戚為奉車、駙馬、騎三都尉而奉朝請焉。元帝為晉王,以參軍為奉車都尉,掾屬為駙馬都尉,行參軍舍人為騎都尉,皆奉朝請。後罷奉車、騎二都尉,唯留駙馬都尉奉朝請。諸尚公主者劉惔、桓溫皆為之。



 中書監及令,案漢武帝遊宴後庭,始使宦者典事尚書,謂之中書謁者,置令、僕射。成帝改中書謁者令曰中謁者令,罷僕射。漢東京省中謁者令,而有中官謁者令,非
 其職也。魏武帝為魏王,置秘書令,典尚書奏事。文帝黃初初改為中書,置監、令,以秘書左丞劉放為中書監,右丞孫資為中書令;監、令蓋自此始也。及晉因之,並置員一人。



 中書侍郎,魏黃初初,中書既置監、令,又置通事郎,次黃門郎。黃門郎已署事過,通事乃署名。已署,奏以入,為帝省讀,書可。及晉,改曰中書侍郎,員四人。中書侍郎蓋此始也。及江左初,改中書侍郎曰通事郎,尋復為中書侍郎。



 中書舍人,案晉初初置舍人、通事各一人,江左合舍人
 通事謂之通事舍人,掌呈奏案章。後省,而以中書侍郎一人直西省,又掌詔命。



 秘書監,案漢桓帝延熹二年置秘書監,後省。魏武為魏王,置秘書令、丞。及文帝黃初初,置中書令,典尚書奏事,而秘書改令為監。後以何禎為秘書丞,而秘書先自有丞,乃以禎為秘書右丞。及晉受命,武帝以秘書并中書省,其秘書著作之局不廢。惠帝永平中,復置秘書監,其屬官有丞,有郎,并統著作省。



 著作郎,周左史之任也。漢東京圖籍在東觀,故使名儒著作東觀,有其名,尚未有官。魏明帝太和中,詔置著作
 郎,於此始有其官,隸中書省。及晉受命,武帝以繆徵為中書著作郎。元康二年,詔曰:「著作舊屬中書,而秘書既典文籍,今改中書著作為秘書著作。」於是改隸秘書省。後別自置省而猶隸秘書。著作郎一人,謂之大著作郎,專掌史任,又置佐著作郎八人。著作郎始到職,必撰名臣傳一人。



 太常、光祿勛、衛尉、太僕、廷尉、大鴻臚、宗正、大司農、少府、將作大匠、太后三卿、大長秋,皆為列卿,各置丞、功曹、主簿、五官等員。



 太常,有博士、協律校尉員,又統太學諸博士、祭酒及太
 史、太廟、太樂、鼓吹、陵等令,太史又別置靈臺丞。



 太常博士,魏官也。魏文帝初置,晉因之。掌引導乘輿。王公已下應追謚者,則博士議定之。



 協律校尉,漢協律都尉之職也,魏杜夔為之。及晉,改為協律校尉。



 晉初承魏制,置博士十九人。及咸寧四年,武帝初立國子學,定置國子祭酒、博士各一人,助教十五人,以教生徒。博士皆取履行清淳,通明典義者,若散騎常侍、中書侍郎、太子中庶子以上,乃得召試。及江左初,減為九人。元帝末,增《儀禮》、《春秋公羊》博士各一人,合為十一人。後又
 增為十六人,不復分掌《五經》,而謂之太學博士也。孝武太元十年,損國子助教員為十人。



 光祿勳,統武賁中郎將、羽林郎將、冗從僕射、羽林左監、五官左右中郎將、東園匠、太官,御府、守宮、黃門、掖庭、清商、華林園、暴室等令。哀帝興寧二年,省光祿勳,并司徒。孝武寧康元年復置。



 衛尉,統武庫、公車、衛士、諸冶等令,左右都候,南北東西督冶掾。及渡江,省衛尉。



 太僕,統典農、典虞都尉,典虞丞,左右中典牧都尉,車府典牧,乘黃廄、驊騮廄、龍馬廄等令。典牧又別置羊牧丞。
 太僕,自元帝渡江之後或省或置。太僕省,故驊騮為門下之職。



 廷尉,主刑法獄訟,屬官有正、監、評,并有律博士員。



 大鴻臚,統大行、典客、園池、華林園、鉤盾等令,又有青宮列丞、鄴玄武苑丞。及江左,有事則權置,無事則省。



 宗正,統皇族宗人圖諜,又統太醫令史,又有司牧掾員。及渡江,哀帝省并太常,太醫以給門下省。



 大司農,統太倉、籍田、導官三令,襄國都水長,東西南北部護漕掾。及渡江,哀帝省并都水,孝武復置。



 少府,統材官校尉、中左右三尚方、中黃左右藏、左校、甄
 官、平準、奚官等令,左校坊、鄴中黃左右藏、油官等丞。及渡江,哀帝省并丹陽尹,孝武復置。自渡江唯置一尚方,又省御府。



 將作大匠,有事則置,無事則罷。



 太后三卿,衛尉、少府、太僕,漢置,皆隨太后宮為官號,在同名卿上,無太后則闕。魏改漢制,在九卿下。及晉復舊,在同號卿上。



 大長秋,皇后卿也,有后則置,無后則省。



 御史中丞,本秦官也,秦時,御史大夫有二丞,其一御史丞,其一為中丞。中丞外督部刺史,內領侍御史,受公卿
 奏事,舉劾案章。漢因之,及成帝綏和元年,更名御史大夫為大司空,置長史,而中丞官職如故。哀帝建平二年,復為御史大夫。元壽二年,又為大司空,而中丞出外為御史臺主。歷漢東京至晉因其制,以中丞為臺主。



 治書侍御史,案漢宣帝幸宣室齋居而決事,令侍御史二人治書侍側,後因別置,謂之治書侍御史,蓋其始也。及魏,又置治書執法,掌奏劾,而治書侍御史掌律令,二官俱置。及晉,唯置治書侍御史,員四人。泰始四年,又置黃沙獄治書侍御史一人,秩與中丞同,掌詔獄及廷尉不當者皆治之。後并河南,遂省黃沙治書侍御史。及太
 康中,又省治書侍御史二員。



 侍御史,案二漢所掌凡有五曹:一曰令曹,掌律令;二曰印曹,掌刻印;三曰供曹,掌齋祠;四曰尉馬曹,掌廄馬;五曰乘曹,掌護駕。魏置八人。及晉,置員九人,品同治書,而有十三曹:吏曹、課第曹、直事曹、印曹、中都督曹、外都督曹、媒曹、符節曹、水曹、中壘曹、營軍曹、法曹、算曹。及江左初,省課第曹,置庫曹,掌廄牧牛馬市租,後分曹,置外左庫、內左庫云。



 殿中侍御史,案魏蘭臺遣二御史居殿中,伺察非法,即其始也。及晉,置四人,江左置二人。又案魏晉官品令又
 有禁防御史第七品,孝武太元中有檢校御史吳琨,則此二職亦蘭臺之職也。



 符節御史,秦符璽令之職也。漢因之,位次御史中丞。至魏,別為一臺,位次御史中丞,掌授節、銅武符、竹使符。及泰始九年,武帝省并蘭臺,置符節御史掌其事焉。



 司隸校尉,案漢武初置十三州,刺史各一人,又置司隸校尉,察三輔、三河、弘農七郡,歷漢東京及魏晉,其官不替。屬官有功曹、都官從事、諸曹從事、部郡從事、主簿、錄事、門下書佐、省事、記室書佐、諸曹書佐守從事、武猛從事等員,凡吏一百人,卒三十二人。及渡江,乃罷司隸校
 尉官,其職乃揚州刺史也。



 謁者僕射,秦官也,自漢至魏因之。魏置僕射,掌大拜授及百官班次,統謁者十人。及武帝省僕射,以謁者并蘭臺。江左復置僕射,後又省。



 都水使者,漢水衡之職也。漢又有都水長丞,主陂池灌溉,保守河渠,屬太常。漢東京省都水,置河隄謁者,魏因之。及武帝省水衡,置都水使者一人,以河隄謁者為都水官屬。及江左,省河隄謁者,置謁者六人。



 中領軍將軍,魏官也。漢建安四年,魏武丞相府自置,及拔漢中,以曹休為中領軍。文帝踐阼,始置領軍將軍,以
 曹休為之,主五校、中壘、武衛等三營。武帝初省,使中軍將軍羊祜統二衛、前、後、左、右、驍衛等營,即領軍之任也。懷帝永嘉中,改中軍曰中領軍。永昌元年,改曰北軍中候,尋復為領軍。成帝世,復為中候,尋復為領軍。



 護軍將軍,案本秦護軍都尉官也。漢因之,高祖以陳平為護軍中尉,武帝復以為護軍都尉,屬大司馬。魏武為相,以韓浩為護軍,史渙為領軍,非漢官也。建安十二年,改護軍為中護軍,領軍為中領軍,置長史、司馬。魏初,因置護軍將軍,主武官選,隸領軍,晉世則不隸也。元帝永昌元年,省護軍,并領軍。明帝太寧二年,復置領、護,各領
 營兵。江左以來,領軍不復別領營,總統二衛、驍騎、材官諸營,護軍猶別有營也。資重者為領軍、護軍,資輕者為中領軍、中護軍。屬官有長史、司馬、功曹、主簿、五官,受命出征則置參軍。



 左右衛將軍,案文帝初置中衛。及武帝受命,分為左右衛,以羊琇為左、趙序為右。並置長史、司馬、功曹、主薄員,江左罷長史。



 驍騎將軍、游擊將軍,並漢雜號將軍也。魏置為中軍。及晉,以領、護、左右衛、驍騎、游擊為六軍。



 左右前後軍將軍,案魏明帝時有左軍,則左軍魏官也,
 至晉不改。武帝初又置前軍、右軍,泰始八年又置後軍,是為四軍。



 屯騎、步兵、越騎、長水、射聲等校尉,是為五校,並漢官也。魏晉逮于江左,猶領營兵,並置司馬、功曹、主簿。後省左軍、右軍、前軍、後軍為鎮衛軍,其左右營校尉自如舊,皆中領軍統之。



 二衛始制前驅、由基、彊弩為三部司馬,各置督史。左衛,熊渠武賁;右衛,佽飛武賁。二衛各五部督。其命中武賁,驍騎、遊擊各領之。又置武賁、羽林、上騎、異力四部,并命中為五督。其衛、鎮四軍如五校,各置千人。更制殿中將軍,中郎、校尉、司馬比驍騎。持椎斧武賁,分
 屬二衛。尉中武賁、持鈒冗從、羽林司馬,常從人數各有差。武帝甚重兵官,故軍校多選朝廷清望之士居之。先是,陳勰為文帝所待,特有才用,明解軍令。帝為晉王,委任使典兵事。及蜀破後,令勰受諸葛亮圍陣用兵倚伏之法,又甲乙校標幟之制,勰悉闇練之,遂以勰為殿中典兵中郎將,遷將軍。久之,武帝每出入,勰持白獸幡在乘輿左右,鹵簿陳列齊肅。太康末,武帝嘗出射雉,勰時已為都水使者,散從。車駕逼暗乃還,漏已盡,當合函,停乘輿,良久不得合,乃詔勰合之。勰舉白獸幡指麾,須臾之間而函成。皆謝勰閑解,甚為武帝所任。



 太子太傅、少傅,皆古官也。泰始三年,武帝始建官,各置一人,尚未置詹事,官事無大小,皆由二傅,並有功曹、主簿、五官。太傅中二千石,少傅二千石。其訓導者,太傅在前,少傅在後。皇太子先拜,諸傅然後答之。武帝後以儲副體尊,遂命諸公居之;以本位重,故或行或領。時侍中任愷,武帝所親敬,復使領之,蓋一時之制也。咸寧元年,以給事黃門侍郎楊珧為詹事,掌宮事,二傅不復領官屬。及楊珧為衛將軍,領少傅,省詹事,遂崇廣傅訓,命太尉賈充領太保,司空齊王攸領太傅,所置吏屬復如舊。二傅進賢兩梁冠,黑介幘,五時朝服,佩水蒼玉,食奉日
 三斛。太康二年,始給春賜絹五十匹,秋絹百匹,綿百斤。其後太尉汝南王亮、車騎將軍楊駿、司空衛瓘、石鑒皆領傅保,猶不置詹事,以終武帝之世。惠帝元康元年,復置詹事,二傅給菜田六頃,田騶六人,立夏後不及田者,食奉一年。置丞一人,秩千石;主簿、五官掾、功曹史、主記門下史、錄事、戶曹法曹倉曹賊曹功曹書佐、門下亭長、門下書佐、省事各一人,給赤耳安車一乘。及愍懷建官,乃置六傅,三太、三少,以景帝諱師,故改太師為太保,通省尚書事,詹事文書關由六傅。然自元康之後,諸傅或二或三,或四或六,及永康中復不置詹事也。自太安已
 來置詹事,終孝懷之世。渡江之後,有太傅少傅,不立師保。



 中庶子四人,職如侍中。



 中舍人四人,咸寧四年置,以舍人才學美者為之,與中庶子共掌文翰,職如黃門侍郎,在中庶子下,洗馬上。



 食官令一人,職如太官令。



 庶子四人,職比散騎常侍、中書監令。



 舍人十六人,職比散騎、中書等侍郎。



 洗馬八人,職如謁者秘書,掌圖籍。釋奠講經則掌其事,出則直者前驅,導威儀。



 率更令,主宮殿門戶及賞罰事,職如光祿勳、衛尉。



 家令,主刑獄、穀貨、飲食,職比司農、少府。漢東京主食官令,食官令及晉自為官,不復屬家令。



 僕,主車馬、親族,職如太僕、宗正。



 左右衛率,案武帝建東宮,置衛率,初曰中衛率。泰始五年,分為左右,各領一軍。惠帝時,愍懷太子在東宮,又加前後二率。及江左,省前後二率,孝武太元中又置。



 王置師、友、文學各一人,景帝諱,故改師為傅。友者因文王、仲尼四友之名號。改太守為內史,省相及僕。有郎中令、中尉、大農為三卿。大國置左右常侍各一人,省郎中,
 置侍郎二人,典書、典祠、典衛、學官令、典書丞各一人,治書四人,中尉司馬、世子庶子、陵廟牧長各一人,謁者四人,中大夫六人,舍人十人,典府各一人。



 咸寧三年,衛將軍楊珧與中書監荀勖以齊王攸有時望,懼惠帝有後難,因追故司空裴秀立五等封建之旨,從容共陳時宜於武帝,以為「古者建侯,所以籓衛王室。今吳寇未殄,方岳任大,而諸王為帥,都督封國,既各不臣其統內,於事重非宜。又異姓諸將居邊,宜參以親戚,而諸王公皆在京都,非扞城之義,萬世之固」。帝初未之察,於是下詔議其制。有司奏,從諸王公更制戶邑,皆中尉領兵。其平原、
 汝南、瑯邪、扶風、齊為大國,梁、趙、樂安、燕、安平、義陽為次國,其餘為小國,皆制所近縣益滿萬戶。又為郡公制度如小國王,亦中尉領兵。郡侯如不滿五千戶王,置一軍一千一百人,亦中尉領之。于時,唯特增魯公國戶邑,追進封故司空博陵公王沈為郡公,鉅平侯羊祜為南城郡侯。又南宮王承、隨王萬各於泰始中封為縣王,邑千戶,至是改正縣王增邑為三千戶。制度如郡侯,亦置一軍。自此非皇子不得為王,而諸王之支庶,皆皇家之近屬至親,亦各以土推恩受封。其大國次國始封王之支子為公,承封王之支子為侯,繼承封王之支子為伯。小
 國五千戶已上,始封王之支子為子,不滿五千戶始封王之支子及始封公侯之支子皆為男,非此皆不得封。其公之制度如五千戶國,侯之制度如不滿五千戶國,亦置一軍千人,中尉領之,伯子男以下各有差而不置軍。大國始封之孫罷下軍,曾孫又罷上軍,次國始封子孫亦罷下軍,其餘皆以一軍為常。大國中軍二千人,上下軍各千五百人,次國上軍二千人,下軍千人。其未之國者,大國置守土百人,次國八十人,小國六十人,郡侯縣公亦如小國制度。既行,所增徙各如本奏遣就國,而諸公皆戀京師,涕泣而去。及吳平後,齊王攸遂之國。



 中朝制,典書令在常侍下,侍郎上。及渡江,則侍郎次常侍,而典書令居三軍下。公國則無中尉、常侍、三軍,侯國又無大農、侍郎,伯子男唯典書以下,又無學官、令史職,皆以次損焉。公侯以下置官屬,隨國大小無定制,其餘官司各有差。名山大澤不以封,鹽鐵金銀銅錫,始平之竹園,別都宮室園囿,皆不為屬國。其仕在天朝者,與之國同,皆自選其文武官。諸入作卿士而其世子年已壯者,皆遣蒞國。其王公已下,茅社符璽,車旗命服,一如泰始初故事。



 州置刺史,別駕、治中從事、諸曹從事等員。所領中郡以
 上及江陽、朱提郡,郡各置部從事一人,小郡亦置一人。又有主簿,門亭長、錄事、記室書佐、諸曹佐、守從事、武猛從事等。凡吏四十一人,卒二十人。諸州邊遠,或有山險,濱近寇賊羌夷者,又置弓馬從事五十餘人。徐州又置淮海,涼州置河津,諸州置都水從事各一人。涼、益州置吏八十五人,卒二十人。荊州又置監佃督一人。



 郡皆置太守,河南郡京師所在,則曰尹。諸王國以內史掌太守之任,又置主簿、主記室、門下賊曹、議生、門下史、記室史、錄事史、書佐、循行、乾、小史、五官掾、功曹史、功曹書佐、循行小史、五官掾等員。郡國戶不滿五千者,置職吏
 五十人,散吏十三人;五千戶以上,則職吏六十三人,散吏二十一人;萬戶以上,職吏六十九人,散吏三十九人。郡國皆置文學掾一人。



 縣大者置令,小者置長。有主簿、錄事史、主記室史、門下書佐、乾、游徼、議生、循行功曹史、小史、廷掾、功曹史、小史書佐乾、戶曹掾史幹、法曹門幹、金倉賊曹掾史、兵曹史、吏曹史、獄小史、獄門亭長、都亭長、賊捕掾等員。戶不滿三百以下,職吏十八人,散吏四人;三百以上,職吏二十八人,散吏六人;五百以上,職吏四十人,散吏八人;千以上,職吏五十三人,散吏十二人;千五百以上,職吏六十
 八人,散吏一十八人;三千以上,職吏八十八人,散吏二十六人。



 郡國及縣,農月皆隨所領戶多少為差,散吏為勸農。又縣五百以上皆置鄉,三千以上置二鄉,五千以上置三鄉,萬以上置四鄉,鄉置嗇夫一人。鄉戶不滿千以下,置治書史一人;千以上置史、佐各一人,正一人;五千五百以上,置史一人,佐二人。縣率百戶置里吏一人,其土廣人稀,聽隨宜置里吏,限不得減五十戶。戶千以上,置校官掾一人。



 縣皆置方略吏四人。洛陽縣置六部尉。江左以後,建康
 亦置六部尉,餘大縣置二人,次縣、小縣各一人。鄴、長安置吏如三千戶以上之制。



 四中郎將,並後漢置,歷魏及晉,並有其職,江左彌重。



 護羌、夷、蠻等校尉,案武帝置南蠻校尉於襄陽,西戎校尉於長安,南夷校尉於寧州。元康中,護羌校尉為涼州刺史,西戎校尉為雍州刺史,南蠻校尉為荊州刺史。及江左初,省南蠻校尉,尋又置於江陵,改南夷校尉曰鎮蠻校尉。及安帝時,於襄陽置寧蠻校尉。



 護匈奴、羌、戎、蠻、夷、越中郎將,案武帝置四中郎將,或領刺史,或持節為之。武帝又置平越中郎將,居廣州,主護
 南越。



\end{pinyinscope}