\article{列傳第五}

\begin{pinyinscope}

 陳騫子輿裴秀子頠秀從弟楷楷
 子憲



 陳騫,臨淮東陽人也。父矯,魏司徒。矯本廣陵劉氏,為外祖陳氏所養,因而改焉。騫沈厚有智謀。初,矯為尚書令,侍中劉曄見幸於魏明帝,譖矯專權。矯憂懼,以問騫。騫曰:「主上明聖,大人大臣,今若不合意,不過不作公耳。」後帝意果釋,騫尚少,為夏侯玄所侮,意色自若,玄以此異之。



 起家尚書郎,遷中山、安平太守,並著稱績。徵為相國
 司馬、長史、御吏中丞,遷尚書,封安國亭侯。蜀賊寇隴右,以尚書持節行征蜀將軍,破賊而還。會諸葛誕之亂,復以尚書行安東將軍。壽春平,拜使持節、都督淮北諸軍事、安東將軍,進爵廣陵侯。轉都督豫州諸軍事、豫州刺史,持節、將軍如故。又轉都督江南諸軍事,徙都督荊州諸軍事、征南大將軍,封郯侯。武帝受禪,以佐命之勳,進車騎將軍,封高平郡公,遷侍中、大將軍,出為都督揚州諸軍事,餘如故,假黃鉞。攻拔吳枳里城,破涂中屯戍。賜騫兄子惺爵關中侯。



 咸寧初,遷太尉,轉大司馬。騫因入朝,言於帝曰:「胡烈、牽弘皆勇而無謀,彊於自用,非綏邊
 之材,將為國恥。願陛下詳之。」時弘為揚州刺史,不承順騫命。帝以為不協相構,於是徵弘,既至,尋復以為涼州刺史。騫竊歎息,以為必敗。二人後果失羌戎之和,皆被寇喪沒,征討連歲,僅而得定,帝乃悔之。



 騫少有度量,含垢匿瑕,所在有績。與賈充、石苞、裴秀等俱為心膂,而騫智度過之,充等亦自以為不及也。累處方任,為士庶所懷。既位極人臣,年踰致仕,思欲退身。咸寧三年,求入朝,因乞骸骨。賜袞冕之服,詔曰:「騫元勳舊德,統乂東夏,方弘遠績,以一吳會,而所苦未除,每表懇切,重勞以方事。今聽留京城,以前太尉府為大司馬府,增置祭酒二
 人,帳下司馬、官騎、大車、鼓吹皆如前,親兵百人,廚田十頃,廚園五十畝,廚士十人,器物經用皆留給焉。又給乘輿輦,出入殿中加鼓吹,如漢蕭何故事。」騫累稱疾辭位,詔曰:「騫履德論道,朕所諮詢。方賴謀猷,以弘庶績,宜時視事。可遣散騎常侍諭意。」騫輒歸第,詔又遣侍中敦諭還府。遂固請,許之,位同保傅,在三司之上,賜以几杖,不朝,安車駟馬,以高平公還第。帝以其勳舊耆老,禮之甚重。又以騫有疾,聽乘輿上殿。



 騫素無謇諤之風,然與帝語傲;及見皇太子加敬,時人以為諂。弟稚與其子輿忿爭,遂說騫子女穢行,騫表徙弟,以此獲譏於世。



 元康二
 年薨,年八十一,加以袞斂,贈太傅,謚曰武。及葬,帝於大司馬門臨喪,望柩流涕,禮依大司馬石苞故事。子輿嗣爵。



 輿字顯初,拜散騎侍郎、洛陽令,遷黃門侍郎,厲將校左軍、大司農、侍中。坐與叔父不睦,出為河內太守。輿雖無檢正,而有力致。尋卒,子植字弘先嗣,官至散騎常侍。卒,子粹嗣,永嘉中遇害,孝武帝以騫玄孫襲爵。卒,弟子浩之嗣。宋受禪,國除。



 裴秀,字季彥,河東聞喜人也。祖茂,漢尚書令。父潛,魏尚書令。秀少好學,有風操,八歲能屬文。叔父徽有盛名,賓
 客甚眾。秀年十餘歲,有詣徽者,出則過秀。然秀母賤,嫡母宣氏不之禮,嘗使進饌於客,見者皆為之起。秀母曰:「微賤如此,當應為小兒故也。」宣氏知之,後遂止。時人為之語曰:「後進領袖有裴秀。」



 渡遼將軍毌丘儉嘗薦秀於大將軍曹爽,曰:「生而岐嶷,長蹈自然,玄靜守真,性入道奧;博學彊記,無文不該;孝友著於鄉黨,高聲聞於遠近。誠宜弼佐謨明,助和鼎味,毗贊大府,光昭盛化。非徒子奇、甘羅之儔,兼包顏、冉、游、夏之美。」爽乃辟為掾,襲父爵清陽亭侯,遷黃門侍郎。爽誅,以故吏免。頃之,為廷尉正,歷文帝安東及衛將軍司馬,軍國之政,多見信納。遷散騎
 常侍。



 帝之討諸葛誕也,秀與尚書僕射陳泰、黃門侍郎鍾會以行臺從,豫參謀略。及誕平,轉尚書,進封魯陽鄉侯,增邑千戶。常道鄉公立,以豫議定策,進爵縣侯,增邑七百戶,遷尚書僕射。魏咸熙初,釐革憲司。時荀顗定禮儀,賈充正法律,而秀改官制焉。秀議五等之爵,自騎督已上六百餘人皆封。於是秀封濟川侯,地方六十里,邑千四百戶,以高苑縣濟川墟為侯國。



 初,文帝未定嗣,而屬意舞陽侯攸。武帝懼不得立,問秀曰:「人有相否?」因以奇表示之。秀後言於文帝曰:「中撫軍人望既茂,天表如此,固非人臣之相也。」由是世子乃定。武帝既即王位,拜
 尚書令、右光祿大夫,與御史大夫王沈、衛將軍賈充俱開府,加給事中。及帝受禪,加左光祿大夫,封鉅鹿郡公,邑三千戶。



 時安遠護軍郝詡與故人書云:「與尚書令裴秀相知,望其為益。」有司奏免秀官,詔曰:「不能使人之不加諸我,此古人所難。交關人事,詡之罪耳,豈尚書令能防乎!其勿有所問。」司隸校尉李憙復上言,騎都尉劉尚為尚書令裴秀占官稻田,求禁止秀。詔又以秀幹翼朝政,有勳績於王室,不可以小疵掩大德,使推正尚罪而解秀禁止焉。



 久之,詔曰:「夫三司之任,以翼宣皇極,弼成王事者也。故經國論道,賴之明喆,茍非其人,官不虛備。
 尚書令、左光祿大夫裴秀,雅量弘博,思心通遠,先帝登庸,贊事前朝。朕受明命,光佐大業,勳德茂著,配蹤元凱。宜正位居體,以康庶績。其以秀為司空。」



 秀儒學洽聞,且留心政事,當禪代之際,總納言之要,其所裁當,禮無違者。又以職在地官,以《禹貢》山川地名,從來久遠,多有變易。後世說者或彊牽引,漸以暗昧。於是甄摘舊文,疑者則闕,古有名而今無者,皆隨事注列,作《禹貢地域圖》十八篇,奏之,藏於秘府。其序曰:



 圖書之設,由來尚矣。自古立象垂制,而賴其用。三代置其官,國史掌厥職。暨漢屠咸陽,丞相蕭何盡收秦之圖籍。今秘書既無古之地圖,
 又無蕭何所得,惟有漢氏《輿地》及《括地》諸雜圖。各不設分率,又不考正準望,亦不備載名山大川。雖有粗形,皆不精審,不可依據。或荒外迂誕之言,不合事實,於義無取。



 大晉龍興,混一六合,以清宇宙,始於庸蜀,冞入其岨。文皇帝乃命有司,撰訪吳蜀地圖。蜀土既定,六軍所經,地域遠近,山川險易,征路迂直,校驗圖記,罔或有差。今上考《禹貢》山海川流,原隰陂澤,古之九州,及今之十六州,郡國縣邑,疆界鄉陬,及古國盟會舊名,水陸徑路,為地圖十八篇。



 制圖之體有六焉。一曰分率,所以辨廣輪之度也。二曰準望,所以正彼此之體也。三曰道里,所以
 定所由之數也。四曰高下,五曰方邪,六曰迂直,此三者各因地而制宜,所以校夷險之異也。有圖象而無分率,則無以審遠近之差;有分率而無準望,雖得之於一隅,必失之於他方;有準望而無道里,則施於山海絕隔之地,不能以相通;有道里而無高下、方邪、迂直之校,則徑路之數必與遠近之實相違,失準望之正矣,故以此六者參而攷之。然遠近之實定於分率,彼此之實定於道里,度數之實定於高下、方邪、迂直之算。故雖有峻山鉅海之隔,絕域殊方之迥,登降詭曲之因,皆可得舉而定者。準望之法既正,則曲直遠近無所隱其形也。



 秀創制
 朝儀,廣陳刑政,朝廷多遵用之,以為故事。在位四載,為當世名公。服寒食散,當飲熱酒而飲冷酒,泰始七年薨,時年四十八。詔曰:「司空經德履哲,體蹈儒雅,佐命翼世,勳業弘茂。方將宣獻敷制,為世宗範,不幸薨殂,朕甚痛之。其賜祕器、朝服一具、衣一襲、錢三十萬、布百匹。謚曰元。」



 初,秀以尚書三十六曹統事準例不明,宜使諸卿任職,未及奏而薨。其友人料其書記,得表草言平吳之事,其詞曰:「孫皓酷虐,不及聖明御世兼弱攻昧,使遺子孫,將遂不能臣;時有否泰,非萬安之勢也。臣昔雖已屢言,未有成旨。今既疾篤不起,謹重尸啟。願陛下時共施用。」
 乃封以上聞。詔報曰:「司空薨,痛悼不能去心。又得表草,雖在危困,不忘王室,盡忠憂國。省益傷切,輒當與諸賢共論也。」



 咸寧初,與石苞等並為王公,配享廟庭。有二子:濬、頠。浚嗣位,至散騎常侍,早卒。浚庶子憬不惠,別封高陽亭侯,以浚少弟頠嗣。



 頠字逸民。弘雅有遠識,博學稽古,自少知名。御史中丞周弼見而嘆曰:「頠若武庫,五兵縱橫,一時之傑也。」賈充即頠從母夫也,表「秀有佐命之勳,不幸嫡長喪亡,遺孤稚弱。頠才德英茂,足以興隆國嗣。」詔頠襲爵,頠固讓,不許。太康二年,徵為太子中庶子,遷散騎常侍。惠帝既位,轉國子祭酒,兼右軍將軍。



 初,頠
 兄子憬為白衣,頠論述世勳,賜爵高陽亭侯。楊駿將誅也,駿黨左軍將軍劉豫陳兵在門,遇頠,問太傅所在。頠紿之曰:「向於西掖門遇公乘素車,從二人西出矣。」豫曰:「吾何之?」頠曰:「宜至廷尉。」豫從頠言,遂委而去。尋而詔頠代豫領左軍將軍,屯萬春門。及駿誅,以功當封武昌侯,頠請以封憬,帝竟封頠次子該。頠苦陳憬本承嫡,宜襲鉅鹿,先帝恩旨,辭不獲命。武昌之封,己之所蒙,特請以封憬。該時尚主,故帝不聽。累遷侍中。



 時天下暫寧,頠奏修國學,刻石寫經。皇太子既講,釋奠祀孔子,飲饗射侯,甚有儀序。又令荀籓終父勖之志,鑄鐘鑿磬,以備郊廟
 朝享禮樂。頠通博多聞,兼明醫術。荀勖之修律度也,檢得古尺,短世所用四分有餘。頠上言:「宜改諸度量。若未能悉革,可先改太醫權衡。此若差違,遂失神農、岐伯之正。藥物輕重,分兩乖互,所可傷夭,為害尤深。古壽考而今短折者,未必不由此也。」卒不能用。樂廣嘗與頠清言,欲以理服之,而頠辭論豐博,廣笑而不言。時人謂頠為言談之林藪。



 頠以賈后不悅太子,抗表請增崇太子所生謝淑妃位號,仍啟增置後衛率吏,給三千兵,於是東宮宿衛萬人。遷尚書,侍中如故,加光祿大夫。每授一職,未嘗不殷勤固讓,表疏十餘上,博引古今成敗以為言,
 覽之者莫不寒心。



 頠深慮賈后亂政,與司空張華、侍中賈模議廢之而立謝淑妃。華、模皆曰:「帝自無廢黜之意,若吾等專行之,上心不以為是。且諸王方剛,朋黨異議,恐禍如發機,身死國危,無益社稷。」頠曰:「誠如公慮。但昏虐之人,無所忌憚,亂可立待,將如之何?」華曰:「卿二人猶且見信,然勤為左右陳禍福之戒,冀無大悖。幸天下尚安,庶可優游卒歲。」此謀遂寢。頠旦夕勸說從母廣城君,令戒喻賈后親待太子而已。或說頠曰:「幸與中宮內外可得盡言。言若不行,則可辭病屏退。若二者不立,雖有十表,難乎免矣。」頠慨然久之,而竟不能行。



 遷尚書左僕射,
 侍中如故。頠雖后之親屬,然雅望素隆,四海不謂之以親戚進也,惟恐其不居位。俄復使頠專任門下事,固讓,不聽。頠上言:「賈模適亡,復以臣代,崇外戚之望,彰偏私之舉。后族何常有能自保,皆知重親無脫者也。然漢二十四帝惟孝文、光武、明帝不重外戚,皆保其宗,豈將獨賢,實以安理故也。昔穆叔不拜越禮之饗,臣亦不敢聞殊常之詔。」又表云:「咎繇謨虞,伊尹相商,呂望翊周,蕭張佐漢,咸播功化,光格四極。暨于繼體,咎單、傅說,祖己、樊仲,亦隆中興。或明揚側陋,或起自庶族,豈非尚德之舉,以臻斯美哉!歷觀近世,不能慕遠,溺於近情,多任后親,
 以致不靜。昔疏廣戒太子以舅氏為官屬,前世以為知禮。況朝廷何取於外戚,正復才均,尚當先其疏者,以明至公。漢世不用馮野王,即其事也。」表上,皆優詔敦譬。



 時以陳準子匡、韓蔚子嵩並侍東宮,頠諫曰:「東宮之建,以儲皇極。其所與游接,必簡英俊,宜用成德。匡、嵩幼弱,未識人理立身之節。東宮實體夙成之表,而今有童子侍從之聲,未是光闡遐風之弘理也。」愍懷太子之廢也,頠與張華苦爭不從,語在《華傳》。



 頠深患時俗放蕩,不尊儒術,何晏、阮籍素有高名於世,口談浮虛,不遵禮法,尸祿耽寵,仕不事事;至王衍之徒,聲譽太盛,位高勢重,不以
 物務自嬰,遂相放效,風教陵遲,乃著崇有之論以釋其蔽曰:



 夫總混群本,宗極之道也。方以族異,庶類之品也。形象著分,有生之體也。化感錯綜,理迹之原也。夫品而為族,則所稟者偏,偏無自足,故憑乎外資。是以生而可尋,所謂理也。理之所體,所謂有也。有之所須,所謂資也。資有攸合,所謂宜也。擇乎厥宜,所謂情也。識智既授,雖出處異業,默語殊塗,所以寶生存宜,其情一也。眾理並而無害,故貴賤形焉。失得由乎所接,故吉凶兆焉。是以賢人君子,知欲不可絕,而交物有會。觀乎往復,稽中定務。惟夫用天之道,分地之利,躬其力任,勞而後饗。居以
 仁順,守以恭儉,率以忠信,行以敬讓,志無盈求,事無過用,乃可濟乎!故大建厥極,綏理群生,訓物垂範,於是乎在,斯則聖人為政之由也。



 若乃淫抗陵肆,則危害萌矣。故欲衍則速患,情佚則怨博,擅恣則興攻,專利則延寇,可謂以厚生而失生者也。悠悠之徒,駭乎若茲之釁,而尋艱爭所緣。察夫偏質有弊,而睹簡損之善,遂闡貴無之議,而建賤有之論。賤有則必外形,外形則必遺制,遺制則必忽防,忽防則必忘禮。禮制弗存,則無以為政矣。眾之從上,猶水之居器也。故兆庶之情,信於所習;習則心服其業,業服則謂之理然。是以君人必慎所教,班其
 政刑一切之務,分宅百姓,各授四職,能令稟命之者不肅而安,忽然忘異,莫有遷志。況於據在三之尊,懷所隆之情,敦以為訓者哉!斯乃昏明所階,不可不審。



 夫盈欲可損而未可絕有也,過用可節而未可謂無貴也。蓋有講言之具者,深列有形之故,盛稱空無之美。形器之故有征,空無之義難檢,辯巧之文可悅,似象之言足惑,眾聽眩焉,溺其成說。雖頗有異此心者,辭不獲濟,屈於所狎,因謂虛無之理,誠不可蓋。唱而有和,多往弗反,遂薄綜世之務,賤功烈之用,高浮游之業,埤經實之賢。人情所殉,篤夫名利。於是文者衍其辭,訥者贊其旨,染其眾
 也。是以立言藉於虛無,謂之玄妙;處官不親所司,謂之雅遠;奉身散其廉操,謂之曠達。故砥礪之風,彌以陵遲。放者因斯,或悖吉凶之禮,而忽容止之表,瀆棄長幼之序,混漫貴賤之級。其甚者至於裸裎,言笑忘宜,以不惜為弘,士行又虧矣。



 老子既著五千之文,表摭穢雜之弊,甄舉靜一之義,有以令人釋然自夷,合於《易》之《損》、《謙》、《艮》、《節》之旨。而靜一守本,無虛無之謂也;《損》《艮》之屬,蓋君子之一道,非《易》之所以為體守本無也。觀老子之書雖博有所經,而云「有生於無」,以虛為主,偏立一家之辭,豈有以而然哉!人之既生,以保生為全,全之所階,以順感為
 務。若味近以虧業,則沈溺之釁興;懷末以忘本,則天理之真滅。故動之所交,存亡之會也。夫有非有,於無非無;於無非無,於有非有。是以申縱播之累,而著貴無之文。將以絕所非之盈謬,存大善之中節,收流遁於既過,反澄正于胸懷。宜其以無為辭,而旨在全有,故其辭曰「以為文不足」。若斯,則是所寄之塗,一方之言也。若謂至理信以無為宗,則偏而害當矣。先賢達識,以非所滯,示之深論。惟班固著難,未足折其情。孫卿、楊雄大體抑之,猶偏有所許。而虛無之言,日以廣衍,眾家扇起,各列其說。上及造化,下被萬事,莫不貴無,所存僉同。情以眾固,乃
 號凡有之理皆義之埤者,薄而鄙焉。辯論人倫及經明之業,遂易門肆。頠用矍然,申其所懷,而攻者盈集。或以為一時口言。有客幸過,咸見命著文,擿列虛無不允之徵。若未能每事釋正,則無家之義弗可奪也。頠退而思之,雖君子宅情,無求於顯,及其立言,在乎達旨而已。然去聖久遠,異同紛糾,茍少有仿佛,可以崇濟先典,扶明大業,有益於時,則惟患言之不能,焉得靜默,及未舉一隅,略示所存而已哉!



 夫至無者無以能生,故始生者自生也。自生而必體有,則有遺而生虧矣。生以有為已分,則虛無是有之所謂遺者也。故養既化之有,非無用之
 所能全也;理既有之眾,非無為之所能循也。心非事也,而制事必由於心,然不可以制事以非事,謂心為無也。匠非器也,而制器必須於匠,然不可以制器以非器,謂匠非有也。是以欲收重泉之鱗,非偃息之所能獲也;隕高墉之禽,非靜拱之所能捷也;審投弦餌之用,非無知之所能覽也。由此而觀,濟有者皆有也,虛無奚益於已有之群生哉!



 王衍之徒攻難交至,並莫能屈。又著《辯才論》,古今精義皆辨釋焉,未成而遇禍。



 初,趙王倫諂事賈后,頠甚惡之,倫數求官,頠與張華復固執不許,由是深為倫所怨。倫又潛懷篡逆,欲先除朝望,因廢賈后之際
 遂誅之,時年三十四。二子嵩、該,倫亦欲害之。梁王肜、東海王越稱頠父秀有勳王室,配食太廟,不宜滅其後嗣,故得不死,徙帶方;惠帝反正,追復頠本官,改葬以卿禮,謚曰成。以嵩嗣爵,為中書黃門侍郎。該出後從伯凱,為散騎常侍,並為乞活賊陳午所害。



 楷字叔則。父徽,魏冀州刺史。楷明悟有識量,弱冠知名,尤精《老》、《易》,少與王戎齊名。鍾會薦之於文帝,辟相國掾,遷尚書郎。賈充改定律令,以楷為定科郎。事畢,詔楷於御前執讀,平議當否。楷善宣吐,左右屬目,聽者忘倦。武帝為撫軍,妙選僚采,以楷為參軍事。吏部郎缺,文帝問
 其人於鍾會。會曰:「裴楷清通,王戎簡要,皆其選也。」於是以楷為吏部郎。



 楷風神高邁,容儀俊爽,博涉群書,特精理義,時人謂之「玉人」,又稱「見裴叔則如近玉山,映照人也」。轉中書郎,出入宮省,見者肅然改容。武帝初登阼,探策以卜世數多少,而得一,帝不悅,群臣失色,莫有言者。楷正容儀,和其聲氣,從容進曰:「臣聞天得一以清,地得一以寧,王侯得一以為天下貞。」武帝大悅,群臣皆稱萬歲。俄拜散騎侍郎,累遷散騎常侍、河內太守,入為屯騎校尉、右軍將軍,轉侍中。



 石崇以功臣子有才氣,與楷志趣各異,不與之交。長水校尉孫季舒嘗與崇酣燕,慢傲
 過度,崇欲表免之。楷聞之,謂崇曰:「足下飲人狂藥,責人正禮,不亦乖乎!」崇乃止。



 楷性寬厚,與物無忤。不持儉素,每遊榮貴,輒取其珍玩。雖車馬器服,宿昔之間,便以施諸窮乏。嘗營別宅,其從兄衍見而悅之,即以宅與衍。梁、趙二王,國之近屬,貴重當時,楷歲請二國租錢百萬,以散親族。人或譏之,楷曰:「損有餘以補不足,天之道也。」安於毀譽,其行己任率,皆此類也。與山濤、和嶠並以盛德居位,帝嘗問曰:「朕應天順時,海內更始,天下風聲,何得何失?」楷對曰:「陛下受命,四海承風,所以未比德於堯舜者,但以賈充之徒尚在朝耳。方宜引天下賢人,與弘正
 道,不宜示人以私。」時任愷、庾純亦以充為言,帝乃出充為關中都督。充納女於太子,乃止。平吳之後,帝方修太平之化,每延公卿,與論政道。楷陳三五之風,次敘漢魏盛衰之迹。帝稱善,坐者歎服焉。



 楷子瓚娶楊駿女,然楷素輕駿,與之不平。駿既執政,乃轉為衛尉,遷太子少師,優游無事,默如也。及駿誅,楷以婚親收付廷尉,將加法。是日事倉卒,誅戮縱橫,眾人為之震恐。楷容色不變,舉動自若,索紙筆與親故書。賴侍中傅祗救護得免,猶坐去官。太保衛瓘、太宰亮稱楷貞正不阿附,宜蒙爵土,乃封臨海侯,食邑二千戶。代楚王瑋為北軍中候,加散
 騎常侍。瑋怨瓘、亮斥己任楷,楷聞之,不敢拜,轉為尚書。



 楷長子輿先娶亮女,女適衛瓘子,楷慮內難未已,求出外鎮,除安南將軍、假節、都督荊州諸軍事,垂當發而瑋果矯詔誅亮、瓘。瑋以楷前奪己中候,又與亮、瓘婚親,密遣討楷。楷素知瑋有望於己,聞有變,單車入城,匿於妻父王渾家,與亮小子一夜八徙,故得免難。瑋既伏誅,以楷為中書令,加侍中,與張華、王戎並管機要。



 楷有渴利疾,不樂處勢。王渾為楷請曰:「楷受先帝拔擢之恩,復蒙陛下寵遇,誠竭節之秋也。然楷性不競於物,昔為常侍,求出為河內太守;後為侍中,復求出為河南尹;與楊駿不
 平,求為衛尉;及轉東宮,班在時類之下,安於淡退,有識有以見其心也。楷今委頓,臣深憂之。光祿勛缺,以為可用。今張華在中書,王戎在尚書,足舉其契,無為復令楷入,名臣不多,當見將養,不違其志,要其遠濟之益。」不聽,就加光祿大夫、開府儀同三司。及疾篤,詔遣黃門郎王衍省疾,楷回眸矚之曰:「竟未相識。衍深嘆其神俊。



 楷有知人之鑒,初在河南,樂廣僑居郡界,未知名,楷見而奇之,致之於宰府。嘗目夏侯玄云「肅肅如入宗廟中,但見禮樂器」,鍾會「如觀武庫森森,但見矛戟在前」,傅嘏「汪翔靡所不見」,山濤「若登山臨下,幽然深遠」。



 初,楷家炊黍在
 甑,或變如拳,或作血,或作蕪菁子。其年而卒,時年五十五,謚曰元。有五子:輿、瓚、憲、禮、遜。



 輿字祖明。少襲父爵,官至散騎侍郎,卒謚曰簡。



 瓚字國寶,中書郎,風神高邁,見者皆敬之。特為王綏所重,每從其遊。綏父戎謂之曰:「國寶初不來,汝數往,何也?」對曰:「國寶雖不知綏,綏自知國寶。」楊駿之誅,為亂兵所害。



 憲字景思。少而穎悟,好交輕俠。及弱冠,更折節嚴重,修尚儒學,足不踰閾者數年。陳郡謝鯤、潁川庾敳皆俊郎士也,見而奇之,相謂曰:「裴憲鯁亮宏達,通機識命,不知其何如父;至於深弘保素,不以世物嬰心者,其殆過之。」



 初,侍講東宮,歷黃門吏部郎、
 侍中。東海王越以為豫州刺史、北中郎將、假節。王浚承制,以憲為尚書。永嘉末,王浚為石勒所破,棗嵩等莫不謝罪軍門,貢賂交錯,惟憲及荀綽恬然私室。勒素聞其名,召而謂之曰:「王浚虐暴幽州,人鬼同疾。孤恭行乾憲,拯茲黎元,羈舊咸歡,慶謝交路。二君齊惡傲威,誠信岨絕,防風之戮,將誰歸乎?」憲神色侃然,泣而對曰:「臣等世荷晉榮,恩遇隆重。王浚凶粗醜正,尚晉之遺籓。雖欣聖化,義岨誠心。且武王伐紂,表商容之閭,未聞商容在倒戈之例也。明公既不欲以道化厲物,必於刑忍為治者,防風之戮,臣之分也。請就辟有司。」不拜而出。勒深嘉之,
 待以賓禮。勒乃簿王浚官寮親屬,皆貲至巨萬,惟憲與荀綽家有書百餘帙,鹽米各十數斛而已。勒聞之,謂其長史張賓曰:「名不虛也。吾不喜得幽州,喜獲二子。」署從事中郎,出為長樂太守。及勒僭號,未遑制度,與王波為之撰朝儀,於是憲章文物,擬於王者。勒大悅,署太中大夫,遷司徒。



 及季龍之世,彌加禮重。憲有二子:挹、,並以文才知名。仕季龍為太子中庶子、散騎常侍。挹、俱豪俠耽酒,好臧否人物。與河間邢魚有隙,魚竊乘馬奔段遼,為人所獲,魚誣使己以季龍當襲鮮卑,告之為備。時季龍適謀伐遼,而與魚辭正合。季龍悉誅挹、,
 憲亦坐免。未幾,復以為右光祿大夫、司徒、太傅,封安定郡公。



 憲歷官無幹績之稱,然在朝玄默,未嘗以物務經懷。但以德重名高,動見尊禮。竟卒於石氏,以族人峙子邁為嗣。



 楷長兄黎,次兄康,並知名。康子盾,少歷顯位。永嘉中,為徐州刺史,委任長史司馬奧。奧勸盾刑殺立威,大發良人為兵,有不奉法者罪便至死。在任三年,百姓嗟怨。東海王越,盾妹夫也。越既薨,騎督滿衡便引所發良人東還。尋而劉元海遣將王桑、趙固向彭城,前鋒數騎至下邳,文武不堪苛政,悉皆散走,盾、奧奔淮陰,妻子為賊人所得。奧又誘盾降趙固。固妻盾女,有寵,盾向女
 涕泣,固遂殺之。



 盾弟邵,字道期。元帝為安東將軍,以邵為長史,王導為司馬,二人相與為深交。徵為太子中庶子,復轉散騎常侍,使持節、都督揚州江西淮北諸軍事、東中郎將,隨越出項,而卒於軍中。及王導為司空,既拜,嘆曰:「裴道期、劉王喬在,吾不得獨登此位。」導子仲豫與康同字,導思舊好,乃改為敬豫焉。



 楷弟綽,字季舒,器宇宏曠,官至黃門侍郎、長水校尉。綽子遐,善言玄理,音辭清暢,泠然若琴瑟。嘗與河南郭象談論,一坐嗟服。又嘗在平東將軍周馥坐,與人圍棋。馥司馬行酒,遐未即飲,司馬醉怒,因曳遐墮地。遐徐起還坐,顏色不變,復棋如
 故。其性虛和如此。東海王越引為主簿,後為越子毗所害。



 初,裴、王二族盛於魏晉之世,時人以為八裴方八王:徽比王祥,楷比王衍,康比王綏,綽比王澄,瓚比王敦,遐比王導,頠比王戎,邈比王玄云。



 史臣曰:周稱多士,漢曰得人,取類星象,頡頏符契。時乏名流,多以幹翮相許,自家光國,豈陳騫之謂歟!秀則聲蓋朋僚,稱為領袖。楷則機神幼發,目以清通。俱為晉氏名臣,良有以也。



 贊曰:世既順才,才膺世至。高平沈敏,蘊茲名器。鉅鹿自然,亦云經笥。媧皇煉石,晉圖開秘。頠有清規,承家來媚。



\end{pinyinscope}