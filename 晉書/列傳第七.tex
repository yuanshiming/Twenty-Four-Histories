\article{列傳第七}

\begin{pinyinscope}

 宗
 室安平獻王孚子邕邕弟義陽成王望望子河間平王洪洪子威洪弟隨穆王整整弟竟陵王楙望弟太原成王輔輔弟翼翼弟下邳獻王晃晃



 弟太原烈王瑰瑰弟高陽元王珪珪弟常山孝王衡衡弟沛順王景彭城穆王權曾孫紘紘子俊高密文獻王泰子孝王略略兄新蔡武哀王騰範陽康王綏子虓濟南惠王遂曾孫勛譙剛王遜子閔王承等高陽王睦任城景王陵弟順



 安平獻王孚,字叔達,宣帝次弟也。初,孚長兄朗字伯達,宣帝字仲達,孚弟馗字季達,恂字顯達,進字惠達,通字
 雅達,敏字幼達,俱知名,故時號為「八達」焉。孚溫厚廉讓,博涉經史。漢末喪亂,與兄弟處危亡之中,簞食瓢飲,而披閱不倦。性通恕,以貞白自立,未嘗有怨於人。陳留殷武有名於海內,嘗罹罪譴,孚往省之,遂與同處分食,談者稱焉。



 魏陳思王植有俊才,清選官屬,以孚為文學掾。植負才陵物,孚每切諫,初不合意,後乃謝之。遷太子中庶子。魏武帝崩,太子號哭過甚,孚諫曰:「大行晏駕,天下恃殿下為命。當上為宗廟,下為萬國,奈何效匹夫之孝乎!」太子良久乃止,曰:「卿言是也。」時群臣初聞帝崩,相聚號哭,無復行列。孚厲聲於朝曰:「今大行晏駕,天下震動,
 當早拜嗣君,以鎮海內,而但哭邪!」孚與尚書和洽罷群臣,備禁衛,具喪事,奉太子以即位,是為文帝。



 時當選侍中、常侍等官,太子左右舊人頗諷諭主者,便欲就用,不調餘人。孚曰:「雖有堯舜,必有稷契。今嗣君新立,當進用海內英賢,猶患不得,如何欲因際會自相薦舉邪!官失其任,得者亦不足貴。」遂更他選。轉孚為中書郎、給事常侍,宿省內,除黃門侍郎,加騎都尉。



 時孫權稱籓,請送任子,當遣前將軍于禁還,久而不至。天子以問孚,孚曰:「先王設九服之制,誠以要荒難以德懷,不以諸夏禮責也。陛下承緒,遠人率貢。權雖未送任子,于禁不至,猶宜以
 寬待之,畜養士馬,以觀其變。不可以嫌疑責讓,恐傷懷遠之義。自孫策至權,奕世相繼,惟彊與弱,不在一禁,禁之未至,當有他故耳。」後禁至,果以疾遲留,而任子竟不至。大軍臨江,責其違言,吳遂絕不貢獻。後出為河內典農,賜爵關內侯,轉清河太守。初,魏文帝置度支尚書,專掌軍國支計,朝議以征討未息,動須節量。及明帝嗣位,欲用孚,問左右曰:「有兄風不?」答云:「似兄。」天子曰:「吾得司馬懿二人,復何憂哉!」轉為度支尚書。



 孚以為擒敵制勝,宜有備預。每諸葛亮入寇關中,邊兵不能制敵,中軍奔赴,輒不及事機,宜預選步騎二萬,以為二部,為討賊之
 備。又以關中連遭賊寇,穀帛不足,遣冀州農丁五千屯於上邽,秋冬習戰陣,春夏修田桑。由是關中軍國有餘,待賊有備矣。後除尚書右僕射,進爵昌平亭侯,遷尚書令。及大將軍曹爽擅權,李勝、何晏、鄧颺等亂政,孚不視庶事,但正身遠害而已。及宣帝誅爽,孚與景帝屯司馬門,以功進爵長社縣侯,加侍中。



 時吳將諸葛恪圍新城,以孚進督諸軍二十萬防禦之。孚次壽春,遣毌丘儉、文欽等進討。諸將欲速擊之,孚曰:「夫攻者,借人之力以為功,且當詐巧,不可力爭也。」故稽留月餘乃進軍,吳師望風而退。



 魏明悼后崩,議書銘旌,或欲去姓而書魏,或欲
 兩書,孚以為:「經典正義,皆不應書。凡帝王皆因本國之名以為天下之號,而與往代相別耳,非為擇美名以自光也。天稱皇天,則帝稱皇帝,地稱后土,則后稱皇后。此乃所以同天地之大號,流無二之尊名,不待稱國號以自表,不俟稱氏族以自彰。是以《春秋》隱公三年《經》曰『三月庚戌天王崩』,尊而稱天,不曰周王者,所以殊乎列國之君也。『八月庚辰宋公和卒』,書國稱名,所以異乎天王也。襄公十五年《經》曰『劉夏逆王后于齊』,不云逆周王后姜氏者,所以異乎列國之夫人也。至乎列國,則曰『夫人姜氏至自齊』,又曰『紀伯姬卒』,書國稱姓,此所以異乎天
 王后也。由此考之,尊稱皇帝,赫赫無二,何待魏乎?尊稱皇后,彰以謚號,何待於姓乎?議者欲書魏者,此以為天皇之尊,同於往古列國之君也。或欲書姓者,此以為天皇之后,同於往古之夫人也。乖經典之大義,異乎聖人之明制,非所以垂訓將來,為萬世不易之式者也。」遂從孚議。



 遷司空。代王凌太尉。及蜀將姜維寇隴右,雍州刺史王經戰敗,遣孚西鎮關中,統諸軍事。征西將軍陳泰與安西將軍鄧艾進擊維,維退。孚還京師,轉太傅。



 及高貴鄉公遭害,百官莫敢奔赴,孚枕尸於股,哭之慟,曰:「殺陛下者臣之罪。」奏推主者。會太后令以庶人禮葬,
 孚與群公上表,乞以王禮葬,從之。孚性至慎。宣帝執政,常自退損。後逢廢立之際,未嘗預謀。景文二帝以孚屬尊,不敢逼。後進封長樂公。



 及武帝受禪,陳留王就金墉城,孚拜辭,執王手,流涕歔欷,不能自勝。曰:「臣死之日,固大魏之純臣也。」詔曰:「太傅勳德弘茂,朕所瞻仰,以光導弘訓,鎮靜宇內,願奉以不臣之禮。其封為安平王,邑四萬戶。進拜太宰、持節、都督中外諸軍事。」有司奏,諸王未之國者,所置官屬,權未有備。帝以孚明德屬尊,當宣化樹教,為群后作則,遂備置官屬焉。又以孚內有親戚,外有交游,惠下之費,而經用不豐,奉絹二千匹。及元會,詔
 孚輿車上殿,帝於阼階迎拜。既坐,帝親奉觴上壽,如家人禮。帝每拜,孚跪而止之。又給以雲母輦、青蓋車。



 孚雖見尊寵,不以為榮,常有憂色。臨終,遺令曰:「有魏貞士河內溫縣司馬孚,字叔達,不伊不周,不夷不惠,立身行道,終始若一,當以素棺單槨,斂以時服。」泰始八年薨,時年九十三。帝於太極東堂舉哀三日。詔曰:「王勛德超世,尊寵無二,期頤在位,朕之所倚。庶永百齡,諮仰訓導,奄忽殂隕,哀慕感切。其以東園溫明祕器、朝服一具、衣一襲、緋練百匹、絹布各五百匹、錢百萬,穀千斛以供喪事。諸所施行,皆依漢東平獻王蒼故事。」其家遵孚遣旨,所
 給器物,一不施用。帝再臨喪,親拜盡哀。及葬,又幸都亭,望柩而拜,哀動左右。給鑾輅輕車,介士武賁百人,吉凶導從二千餘人,前後鼓吹,配饗太廟。九子:邕、望、輔、翼、晃、瑰、珪、衡、景。



 邕字子魁。初為世子,拜步兵校尉、侍中。先孚卒,追贈輔國將軍,謚曰貞。邕子崇為世孫,又早夭。泰始九年,立崇弟平陽亭侯隆為安平王。立四年,咸寧二年薨,謚曰穆,無子,國絕。



 義陽成王望,字子初,出繼伯父朗,寬厚有父風。仕郡上計吏,舉孝廉,辟司徒掾,歷平陽太守、洛陽典農中郎將。
 從宣帝討王凌,以功封永安亭侯。遷護軍將軍,改封安樂鄉侯,加散騎常侍。時魏高貴鄉公好才愛士,望與裴秀、王沈、鍾會並見親待,數侍宴筵。公性急,秀等居內職,急有召便至。以望外官,特給追鋒車一乘,武賁五人。時景文相繼輔政,未嘗朝覲,權歸晉室。望雖見寵待,每不自安,由是求出,為征西將軍、持節、都督雍涼二州諸軍事。在任八年,威化明肅。先是蜀將姜維屢寇關中,及望至,廣設方略,維不得為寇,關中賴之。進封順陽侯。徵拜衛將軍,領中領軍,典禁兵。尋加驃騎將軍、開府。頃之,代何曾為司徒。



 武帝受禪,封義陽王,邑萬戶,給兵二千人。
 泰始三年,詔曰:「夫尚賢庸勛,尊宗茂親,所以體國經化,式是百辟也。且台司之重,存乎天官,故周建六職,政典為首。司徒、中領軍,以明德近屬,世濟其美;祖考創業,翼佐大命,出典方任,入贊朝政,文德既著,武功宣暢。逮朕嗣位,弼道惟明,宜登上司,兼統軍戎,內輔帝室,外隆威重,其進位太尉,中領軍如故。置太尉軍司一人,參軍事六人,騎司馬五人。又增置官騎十人,並前三十,假羽葆鼓吹。」



 吳將施績寇江夏,邊境騷動。以望統中軍步騎二萬,出屯龍陂,為二方重鎮,假節,加大都督諸軍事。會荊州刺史胡烈距績,破之,望乃班師。俄而吳將丁奉寇芍
 陂,望又率諸軍以赴之,未至而奉退。拜大司馬。孫皓率眾向壽春,詔望統中軍二萬,騎三千,據淮北。皓退,軍罷。泰始七年薨,時年六十七,賻贈有加,望性儉吝而好聚斂,身亡之後,金帛盈溢,以此獲譏。四子:弈、洪、整、楙。



 弈至黃門郎,先望卒。整亦早亡。以弈子奇襲爵。奇亦好畜聚,不知紀極,遣三部使到交廣商貨,為有司所奏,太康九年,詔貶為三縱亭侯。更以章武王威為望嗣。後威誅,復立奇為棘陽王以嗣望。



 河間平王洪,字孔業,出繼叔父昌武亭侯遺。仕魏,歷位典農中郎將、原武太守,封襄賁男。武帝受禪,封河間王。
 立十二年,咸寧二年薨。二子:威、混。威嗣,徙封章武。其後威既繼義陽王望,更立混為洪嗣。混歷位散常侍,薨。



 及洛陽陷,混諸子皆沒于胡。而小子滔初嗣新蔡王確,亦與其兄俱沒。後得南還,與新蔡太妃不協。太興二年上疏,以兄弟並沒在遼東,章武國絕,宜還所生。太妃訟之,事下太常。太常賀循議:「章武、新蔡俱承一國不絕之統,義不得替其本宗而先後傍親。按滔既已被命為人後矣,必須無復兄弟,本國永絕,然後得還所生。今兄弟在遠,不得言無,道里雖阻,復非絕域。且鮮卑恭命,信使不絕。自宜詔下遼東,依劉群、盧諶等例,發遣令還,繼嗣
 本封。謂滔今未得便委離所後也。」元帝詔曰:「滔雖出養,自有所生母。新蔡太妃相待甚薄,滔執意如此。如其不聽,終當紛紜,更為不可。今便順其所執,還襲章武。」



 滔歷位散騎常侍,薨,子休嗣。休與彭城王雄俱奔蘇峻。峻平,休已戰死。弟珍年八歲,以小弗坐。咸和六年襲爵,位至大宗正。薨,無嗣,河間王欽以子範之繼,位至游擊將軍。薨,子秀嗣。義熙元年,為桂陽太守。秀妻桓振之妹,振作逆,秀不自安,謀反,伏誅,國除。



 威字景曜,初嗣洪。咸寧三年,徙封章武。太康九年,嗣義陽王望。威凶暴無操行,諂附趙王倫。元康末,為散騎常
 侍,倫將篡,使威與黃門郎駱休逼帝奪璽綬,倫以威為中書令。倫敗,惠帝反正,曰:「阿皮捩吾指,奪吾璽綬,不可不殺。」阿皮,威小字也。於是誅威。



 隨穆王整,兄弈卒,以整為世子。歷南中郎將,封清泉侯,先父望薨,追贈冠軍將軍。武帝以義陽國一縣追封為隨縣王。子邁嗣。太康九年,以義陽之平林益邁為隨郡王。



 竟陵王楙,字孔偉,初封樂陵亭侯,起家參相國軍事。武帝受禪,封東平王,邑三千九十七戶。入為散騎常侍、尚書。



 楙善諂諛,曲事楊駿。及駿誅,依法當死,東安公繇與
 楙善,故得不坐。尋遷大鴻臚,加侍中。繇欲擅朝政,與汝南王亮不平。帝託以繇討駿顧望,免繇、楙等官,遣楙就國。楙殖財貨,奢僭踰制。趙王倫篡位,召還。及義兵起,倫以楙為衛將軍、都督諸軍事。倫敗,楙免官。齊王冏輔政,繇復為僕射,舉楙為平東將軍、都督徐州諸軍事,鎮下邳。成都王穎輔政,進楙為衛將軍。



 會惠帝北征,即以楙為車騎將軍,都督如故,使率眾赴鄴。蕩陰之役,東海王越奔于下邳,楙不納,越乃還國。帝既西幸,越總兵謀迎大駕,楙甚懼。長史王脩說曰:「東海宗室重望,今將興義,公宜舉徐州以授之,此克讓之美也。」楙從之,乃自承
 制都督兗州刺史、車騎將軍,表於天子。時帝在長安,遣使者劉虔即拜焉。



 楙慮兗州刺史茍晞不避己,乃給虔兵,使稱詔誅晞。晞時已避位,楙在州徵求不已,郡縣不堪命。范陽王虓遣晞還兗州,徙楙都督青州諸軍事。楙不受命,背山東諸侯,與豫州刺史劉喬相結。虓遣將田徽擊楙,破之,楙走還國。帝還洛陽,楙乃詣闕。



 及懷帝踐阼,改封竟陵王,拜光祿大夫。越出牧豫州,留世子毗及其黨何倫訪察宮省。楙白帝討越,乃合眾襲倫,不剋。帝委罪於楙,楙奔竄獲免。越薨,乃出。及洛陽傾覆,為亂兵所害。



 太原成王輔,魏末為野王太守。武帝受禪,封渤海王,邑五千三百七十九戶,泰始二年之國。後為衛尉,出為東中郎將,轉南中郎將,咸寧三年,徙為太原王,監并州諸軍事。太康四年入朝,五年薨,追贈鎮北將軍。永平元年,更贈衛將軍、開府儀同三司。子弘立,元康中為散騎常侍,後徙封中丘王。三年薨,子鑠立。



 翼字子世,少歷顯位,官至武賁中郎將。武帝未受禪而卒,以兄邕之支子承為嗣,封南宮縣王。薨,子祐嗣立,承遂無後。



 下邳獻王晃字子明,魏封武始亭侯,拜黃門侍郎,改封
 西安男,出為東莞太守。武帝受禪,封下邳王,邑五千一百七十六戶,泰始二年就國。



 晃孝友貞廉,謙虛下士,甚得宗室之稱。後為長水校尉、南中郎將。九年,詔曰:「南中郎將、下邳王晃清亮中正,體行明潔,才周政理,有文武策識。其以晃為使持節,都督寧益二州諸軍事、安西將軍,領益州刺史。」晃以疾不行,更拜尚書,遷右僕射。久之,出為鎮東將軍、都督青徐二州諸軍事。惠帝即位,入為車騎將軍,加散騎常侍。將誅楊駿,以晃領護軍,屯東掖門,尋守尚書令。遷司空,加侍中,令如故。,元康六年薨,追贈太傅。



 二子:裒、綽。裒早卒,綽有篤疾,別封良城縣王,以
 太原王輔第三子韡為嗣。官至侍中、尚書,早薨,子韶立。



 太原烈王瑰,字子泉,魏長樂亭侯,改封貴壽鄉侯。歷振威將軍,秘書監,封固始子。武帝受禪,封太原王,邑五千四百九十六戶,泰始二年就國。四年入朝,賜袞冕之服,遷東中郎將。十年薨,詔曰:「瑰乃心忠篤,智器雅亮。歷位文武,有幹事之績。出臨封土,夷夏懷附,鎮守許都,思謀可紀。不幸早薨,朕甚悼之。今安厝在近,其追贈前將軍。」子顒立,徙封河間王,別有傳。



 高陽元王珪,字子璋,少有才望,魏高陽鄉侯。歷河南令,進封湞陽子,拜給事黃門侍郎。武帝受禪,封高陽王,邑
 五千五百七十戶。歷北中郎將、督鄴城守諸軍事。泰始六年入朝,以父孚年高,乞留供養。拜尚書,遷右僕射。十年薨,詔遣兼大鴻臚持節監護喪事,贈車騎將軍、儀同三司。



 珪有美譽於世,而帝甚悼惜之。無子,詔以太原王輔子緝襲爵。緝立五年,咸寧四年薨,謚曰哀。無子,太康二年詔以太原王瑰世子顒子訟為緝後,封真定縣侯。



 常山孝王衡,字子平,魏封德陽鄉侯。進封汝陽子,為駙馬都尉。武帝受禪,封常山王,邑三千七百九十戶。二年薨,無子,以安平世子邕第四子敦為嗣。



 沛順王景,字子文,魏樂安亭侯。歷諫議大夫。武帝受禪,
 封沛王,邑三千四百戶。立十一年,咸寧元年薨,子韜立。



 彭城穆王權,字子輿,宣帝弟魏魯相東武城侯馗之子也。初襲封,拜冗從僕射。武帝受禪,封彭城王,邑二千九百戶。出為北中郎將、都督鄴城守諸軍事。泰始中入朝,賜袞冕之服。咸寧元年薨,子元王植立。歷位後將軍,尋拜國子祭酒、太僕卿、侍中、尚書。出為安東將軍、都督揚州諸軍事,代淮南王允鎮壽春,未發。或云植助允攻趙王倫,遂以憂薨。贈車騎將軍,增封萬五千戶。子康王釋立,官至南中郎將、持節、平南將軍,分魯國蕃、薛二縣以
 益其國,心二萬三千戶。薨,子雄立,坐奔蘇峻伏誅,更以釋子紘嗣。



 紘字偉德,初封堂邑縣公。建興末,元帝承制,以紘繼高密王據。及帝即位,拜散騎侍郎,遷翊軍校尉、前將軍。雄之誅也,紘入繼本宗。拜國子祭酒,加散騎常侍,尋遷大宗正、秘書監。有風疾,性理不恒。或欲上疏陳事,歷示公卿。又杜門讓還章印貂蟬,著《杜門賦》以顯其志。由是更拜光祿大夫,領大宗師,常侍如故。後疾甚,馳騁無度,或攻劫軍寺,或捍傷官屬,醜言悖詈,誹謗上下。又乘車突入端門,至太極殿前。於是御史中丞車灌奏劾,請免紘
 官,下其國嚴加防錄。成帝詔曰:「王以明德茂親,居宗師之重,宜敷道養德,靜一其操。而頃游行煩數,冒履風塵。宜令官屬已下,各以職奉衛,不得令王復有此勞。內外職司,各慎其局。王可解常侍、光祿、宗師,先所給車牛可錄取,賜米布床帳以養疾。」咸康八年薨,贈散騎常侍、金紫光祿大夫。二子:玄、俊。



 玄嗣立。會庚戌制不得藏戶,玄匿五戶,桓溫表玄犯禁,收付廷尉。既而宥之,位至中書侍郎。薨,子弘之立,位至散騎常侍。薨,子邵之立。薨,子崇之立。薨,子緝之立。宋受禪,國除。



 恭王俊字道度,出嗣高密王略,官至散騎常侍。薨,子敬
 王純之立,歷臨川內史、司農少府卿、太宰右長史。薨,子恢之立。義熙末,以給事中兼太尉,修謁洛陽園陵。宋受禪,國除。



 高密文獻王泰,字子舒,彭城穆王權之弟,魏陽亭侯,補陽翟令,遷扶風太守。武帝受禪,封隴西王,邑三千二百戶,拜游擊將軍。出為兗州刺史,加鷹揚將軍。遷使持節、都督寧益二州諸軍事、安西將軍,領益州刺史,稱疾不行。轉安北將軍,代兄權督鄴城守事。安西將軍、都督
 關中事。太康初,入為散騎常侍、前將軍,領鄴城門校尉,以疾去官。後代下邳王晃為尚書左僕射。出為鎮西將軍,領護西戎校尉、假節,代扶風王駿都督關中軍事,以疾還京師。永熙初,代石鑒為司空,尋領太子太保。及楊駿誅,泰領駿營,加侍中,給步兵二千五百人,騎五百匹。泰固辭,乃給千兵百騎。



 楚王瑋之被收,泰嚴兵將救之,祭酒丁綏諫曰:「公為宰相,不可輕動。且夜中倉卒,宜遣人參審定問。」泰從之。瑋既誅,乃以泰錄尚書事,遷太尉,守尚書令,改封高密王,邑萬戶。元康九年薨,追贈太傅。



 泰性廉靜,不近聲色。雖為宰輔,食大國之租,服飾肴膳
 如布衣寒士。任真簡率,每朝會,不識者不知其王公也。事視恭謹,居喪哀戚,謙虛下物,為宗室儀表。當時諸王,惟泰及下邳王晃以節制見稱。雖並不能振施,其餘莫得比焉。泰四子:越、騰、略、模。越自有傳。騰出後叔父,弟略立。



 孝王略,字元簡,孝敬慈順,小心下士,少有父風。元康初,愍懷太子在東宮,選大臣子弟有名稱者以為賓友,略與華恒等並侍左右。歷散騎黃門侍郎、散騎騎常侍、秘書監,出為安南將軍、持節、都督沔南諸軍事,遷安北將軍、都督青州諸軍事。略逼青州刺史程牧,牧避之,略自領
 州。永興初,弦令劉根起兵東萊,誑惑百姓,眾以萬數,攻略於臨淄,略不能距,走保聊城。懷帝即位,遷使持節、都督荊州諸軍事、征南大將軍、開府儀同三司。京兆流人王逌與叟人郝洛聚眾數千,屯于冠軍。略遣參軍崔曠率將軍皮初、張洛等討逌,為逌所譎,戰敗。略更遣左司馬曹攄統曠等進逼逌。將大戰,曠在後密自退走,攄軍無繼,戰敗,死之。略乃赦曠罪,復遣部將韓松又督曠攻逌,逌降。尋進開府,加散騎常侍。永嘉三年薨,追贈侍中、太尉,子據立。薨,無子,以彭城康王子紘為嗣。其後紘歸本宗,立紘子俊以奉其祀。



 新蔡武哀王騰,字元邁,少拜冗從僕射,封東嬴公,歷南陽、魏郡太守,所在稱職,徵為宗正,遷太常,轉持節、寧北將軍、都督並州諸軍事、並州刺史。惠帝討成都王穎,六軍敗績。騰與安北將軍王浚共殺穎所署幽州刺史和演,率眾討穎。穎遣北中郎將王斌距戰,浚率鮮卑騎擊斌,騰為後係,大破之。穎懼,挾帝歸洛陽,進騰位安將軍。永嘉初,遷車騎將軍,都督鄴城守諸軍事,鎮鄴。又以迎駕之勳,改封新蔡王。



 初,騰發並州,次於真定。值大雪,平地數尺,營門前方數丈雪融不積,騰怪而掘之,得玉馬,高尺許,表獻之。其後公師籓與平陽人汲桑等為群
 盜,起於清河鄃縣,眾千餘人,寇頓丘,以葬成都王穎為辭,載穎主而行,與張泓故將李豐等將攻鄴。騰曰:「孤在並州七年,胡圍城不能剋。汲桑小賊,何足憂也。」及豐等至,騰不能守,率輕騎而走,為豐所害。四子:虞、矯、紹、確。虞有勇力,騰之被害,虞逐豐,豐投水而死。是日,虞及矯、紹并鉅鹿太守崔曼、車騎長史羊恆、從事中郎蔡克等又為豐餘黨所害,及諸名家流移依鄴者,死亡並盡。初,鄴中雖府庫虛竭,而騰資用甚饒。性儉嗇,無所振惠,臨急,乃賜將士米可數升,帛各丈尺,是以人不為用,遂致於禍。及茍晞救鄴,桑還平陽。于時盛夏,尸爛壞不可復識,
 騰及三子骸骨不獲。庶子確立。



 莊王確,字嗣安,歷東中郎將、都督豫州諸軍事,鎮許昌。永嘉末,為石勒所害。無子,初以章武王混子滔奉其祀,其後復以汝南威王祐子弼為確後。太興元年薨,無子,又以弼弟邈嗣確,位至侍中。薨,子晃立,拜散騎侍郎。桓溫廢武陵王,免晃為庶人,徙衡陽。孝武帝立晃弟崇繼邈後,為奴所害,子惠立。宋受禪,國除。



 南陽王模,字元表,少好學,與元帝及范陽王虓俱有稱於宗室。初封平昌公。惠帝末,拜冗從僕射,累遷太子庶子、員外散騎常侍。成都王穎奔長安,東海王越以模為
 北中郎將,鎮鄴。永興初,成都王穎故帳下督公師籓、樓權、郝昌等攻鄴,模左右謀應之。廣平太守丁邵率眾救模,范陽王虓又遣兗州刺史茍晞援之,籓等散走。遷鎮東大將軍,鎮許昌。進爵南陽王。永嘉初,轉征西大將軍、開府、都督秦雍梁益諸軍事,代河間王顒鎮關中。模感丁邵之德,敕國人為邵生立碑。



 時關中饑荒,百姓相敢,加以疾癘,盜賊公行。模力不能制,乃鑄銅人鐘鼎為釜器以易穀,議者非之。東海王越表徵模為司空,遣中書監傅祗代之。模謀臣淳于定說模曰:「關中天府之國,霸王之地。今以不能綏撫而還,既於聲望有虧,又公兄弟
 唱起大事,而並在朝廷,若自彊則有專權之罪,弱則受制於人,非公之利也。」模納其言,不就徵。表遣世子保為西中郎將、東羌校尉,鎮上邽,秦州刺史裴苞距之。模使帳下都尉陳安率眾攻苞,苞奔安定。太守賈疋以郡迎苞,模遣軍司謝班伐疋,疋退盧水。其年,進位太尉、大都督。



 洛京傾覆,模使牙門趙染戍蒲阪,染求馮翊太守不得,怒,率眾降于劉聰。聰使其子粲及染攻長安,模使淳于定距之,為染所敗。士眾離叛,倉庫虛竭,軍祭酒韋輔曰:「事急矣,早降可以免。」模從之,遂降于染。染箕踞攘袂數模之罪,送詣粲。粲殺之,以模妃劉氏賜胡張本為
 妻。子保立。



 保字景度,少有文義,好述作。初拜南陽國世子。模遇害,保在邽上。其後賈疋死,裴苞又為張軌所殺,保全有秦州之地,自號大司馬,承制置百官。隴右氐羌並從之,涼州刺史張寔遣使貢獻。及愍帝即位,以保為右丞相,加侍中、都督陜西諸軍事。尋進位相國。



 模之敗也,都尉陳安歸於保,保命統精勇千餘人以討羌,寵遇甚厚。保將張春等疾之,譖安有異志,請除之,保不許。春等輒伏客以刺安,安被創,馳還隴城,遣使詣保,貢獻不絕。



 愍帝之蒙塵也,保自稱晉王。時上邽大饑,士眾窘困,張春奉
 保之南安。陳安自號秦州刺史,稱籓於劉曜。春復奉保奔桑城,將投于張寔。寔使兵迎保,實禦之也。是歲,保病薨,時年二十七。保體質豐偉,嘗自稱重八百斤。喜睡,痿疾,不能御婦人。無子,張春立宗室司馬瞻奉保後。陳安舉兵攻春,春走,瞻降于安,安送詣劉曜,曜殺之。安迎保喪,以天子禮葬于上邽,謚曰元。



 范陽康王綏,字子都,彭城王權季弟也,初為諫議大夫。泰始元年受封,在位十五年。咸寧五年薨,子虓立焉。



 虓字武會,少好學,馳譽,研考經記,清辯,能言論。以宗室
 選拜散騎常侍,累遷尚書。出為安南將軍、都督豫州諸軍事、持節,鎮許昌,進位征南將軍。



 河間王顒表立成都王穎為太弟,為王浚所破,挾天子還洛陽。虓與東平王楙、鎮東將軍周馥等上言曰:「自愍懷被害,皇儲不建,委重前相,輒失臣節。是以前年太宰與臣,永惟社稷之貳,不可久空,所以共啟成都王穎,以為國副。受重之後,而弗克負荷。『小人勿用』,而以為腹心。骨肉宜敦,而猜佻薦至,險言皮宜遠,而讒說殄行。此皆臣等不聰不明,失所宗賴。遂令陛下謬於降授,雖戮臣等,不足以謝天下。今大駕還宮,文武空曠,制度荒破,靡有孑遺,臣等雖劣,足匡
 王室。而道路之言,謂張方與臣等不同。既惜所在興異,又以太宰惇德允元,著於具瞻,每當義節,輒為社稷宗盟之先。張方受其指教,為國效節。昔年之舉,有死無貳。此即太宰之良將,陛下之忠臣。但以受性彊毅,不達變通,遂守前志,已致紛紜。然退思惟,既是其不易之節,且慮事翻之後,為天下所罪,故不即西還耳。原其本事,實無深責。臣聞先代明主,未嘗不全護功臣,令福流子孫。自中間以來,陛下功臣初無全者,非獨人才皆劣,其於取禍,實由朝廷策之失宜,不相容恕。以一旦之咎,喪其積年之勛,既違《周禮》議功之典,且使天下之人莫敢復
 為陛下致節者。臣等此言,豈獨為一張方,實為社稷遠計,欲令功臣長守富貴。臣愚以為宜委太宰以關右之任,一方事重,及自州郡已下,選舉授任,一皆仰成。若朝之大事,廢興損益,每輒疇諮。此則二伯述職,周召分陜之義,陛下復行於今時。遣方還郡,令群后申志,時定王室。所加方官,請悉如舊。此則忠臣義士有勸,功臣必全矣。司徒戎,異姓之賢;司空越,公族之望,並忠國愛主,小心翼翼,宜幹機事,委以朝政。安北將軍王浚佐命之胤,率身履道,忠亮清正,遠近所推。如今日之大舉,實有定社稷之勛,此是臣等所以嘆息歸高也。浚宜特崇重之,
 以副群望,遂撫幽朔,長為北籓。臣等竭力扞城,籓屏皇家,陛下垂拱,而四海自正。則四祖之業,必隆於今,日月之暉,昧而復曜。乞垂三思,察臣所言。又可以臣表西示太宰。」



 又表曰:「成都王失道,為姦邪所誤,論王之身,不宜深責。且先帝遺體,陛下群弟,自元康以來,罪戮相等,實海內所為匈匈,而臣等所以痛心。今廢成都,更封一邑,宜其必許。若廢黜尋有禍害,既傷陛下矜慈之恩,又令遠近恒謂公族無復骨肉之情,此實臣等內省悲慚,無顏於四海也。乞陛下察臣忠款。」於是虓先率眾自許屯於滎陽。



 會惠帝西遷,虓與從兄平昌公模、長史馮嵩等
 刑白馬臿血而盟,推東海王越為盟主,虓都督河北諸軍事、驃騎將軍、持節,領豫州刺史。劉喬不受越等節度,乘虛破許。虓自拔渡河,王浚表虓領翼州刺史,資以兵馬。虓入翼州發兵,又南濟河,破喬等。河間王顒聞喬敗,斬張方,傳首於越。越與虓西迎帝,而顒出奔。於是奉天子還都,拜虓為司徒。永興三年暴疾薨,時年三十七。無子,養模子黎為嗣,黎隨模就國,於長安遇害。



 濟南惠王遂,字子伯,宣帝弟魏鴻臚丞恂之子也。仕魏關內侯,進封平昌亭侯,歷典軍郎將。景元二年,轉封武
 城鄉侯、督鄴城守諸軍事、北中郎將。五等建,封祝阿伯,累遷冠軍將軍。武帝受禪,封濟南王。泰始二年薨。二子:耽、緝。耽嗣立,咸寧三年徙為中山王。是年薨,無子,緝繼。成都王穎以緝為建威將軍,與石熙等率眾距王浚,沒於陣,薨。無子,國除。



 後遂之曾孫勛字偉長,年十餘歲,愍帝末,長安陷,劉曜將令狐泥養為子。及壯,便弓馬,能左右射,咸和六年,自關右還,自列云「是大長秋恂之玄孫,冠軍將軍濟南惠王遂之曾孫,略陽太守瓘之子」,遂拜謁者僕射,以勇聞。



 庾翼之鎮襄陽,以梁州刺史援桓宣卒,請勛代之。初屯西城,退守武當。時石季龍死,中國亂,
 雍州諸豪帥馳告勳。勛率眾出駱谷,壁於懸鉤,去長安二百里,遣部將劉煥攻長安,又拔賀城。於是關中皆殺季龍太守令長以應勛。勛兵少,未能自固,復還梁州。永和中,張琚據隴東,遣使召勳,勛復入長安。初,京兆人杜洪以豪族陵琚,琚以勇俠侮洪,洪知勛憚琚兵彊,因說勛曰;「不殺張琚,關中非國家有也。」勳乃偽請琚,於坐殺之。琚弟走池陽,合眾攻勛,頻戰不利,請和,歸梁州。後桓溫伐關中,命勛出子午道,而為苻雄所敗,退屯于女媧堡。



 俄遷征虜將軍,監關中軍事,領西戎校尉,賜爵通吉亭侯。為政暴酷,至於治中別駕及州之豪右,言語忤意,
 即於坐梟斬之,或引弓自射。西土患其凶虐。在州常懷據蜀,有僭偽之意。桓溫聞之,務相綏懷,以其子康為漢中太守。勛逆謀巳成,憚益州刺史周撫,未發。及撫卒,遂擁眾人劍閣。梁州別駕雍端、西戎司馬隗粹並切諫,勳皆誅之,自號梁益二州牧、成都王。桓溫遣朱序討勛,勛兵潰,為序所獲,及息隴子、長史梁憚、司馬金壹等送于溫,並斬之,傳首京師。



 譙剛王遜,字子悌,宣帝弟魏中郎進之子也。仕魏關內
 侯,改封城陽亭侯,參鎮東軍事,拜輕車將軍、羽林左監。五等建,徙封涇陽男。武帝受禪,封譙王,邑四千四百戶。泰始二年薨。二子:隨、承。定王隨立。薨,子邃立,沒於石勒,元帝以承嗣遜。



 閔王承字敬才,少篤厚有志行。拜奉車都尉、奉朝請,稍遷廣威將軍、安夷護軍,鎮安定。從惠帝還洛陽,拜游擊將軍。永嘉中,天下漸亂,間行依征南將軍山簡,會簡卒,進至武昌。元帝初鎮揚州,承歸建康,補軍諮祭酒。愍帝徵為龍驤將軍,不行。元帝為晉王,承制更封承為譙王。太興初,拜屯騎校尉,加輔國將軍,領左軍將軍。



 承居官
 儉約,家無別室。尋加散騎常侍,輔國、左軍如故。王敦有無君之心,表疏輕慢。帝夜召承,以敦表示之,曰:「王敦頃年位任足矣,而所求不已,言至於此,將若之何?」承曰:「陛下不早裁之,難將作矣。」帝欲樹籓屏,會敦表以宣城內史沈充為湘州,帝謂承曰:「湘州南楚險固,在上流之要,控三州之會,是用武之國也。今以叔父居之,何如?」承曰:「臣幸託末屬,身當宿衛,未有驅馳之勞,頻受過厚之遇,夙夜自厲,思報天德。君之所命,惟力是視,敢有辭焉!然湘州蜀寇之餘,人物凋盡,若上憑天威,得之所蒞,比及三年,請從戎役。若未及此,雖復灰身,亦無益也。」於是詔
 曰:「夫王者體天理物,非群才不足濟其務。外建賢哲,以樹風聲,內睦親親,以廣籓屏。是以太公封齊,伯禽居魯,此先王之令典,古今之通義也。我晉開基,列國相望,乃授瑯邪武王,鎮統東夏;汝南文成,總一淮許;扶風、梁王,迭據關右;爰暨東嬴,作司並州。今公族雖寡,不逮曩時,豈得替舊章乎!散騎常侍、左將軍、譙王承貞素款亮,志存忠恪,便蕃左右,恭肅彌著。今以承監湘州諸軍事、南中郎將、湘州刺史。」



 初,劉隗以王敦威權太盛,終不可制,勸帝出諸心腹,以鎮方隅。故先以承為湘州,續用隗及戴若思等,並為州牧。承行達武昌,釋戎備見王敦。敦與
 之宴,欲觀其意,謂承曰:「大王雅素佳士,恐非將帥才也。」承曰:「公未見知耳,鉛刀豈不能一割乎!」承以敦欲測其情,故發此言。敦果謂錢鳳曰:「彼不知懼而學壯語,此之不武,何能為也。」聽承之鎮。時湘土荒殘,公私困弊,承躬自儉約,乘葦茭車,而傾心綏撫,甚有能名。敦恐其為己患,詐稱北伐,悉召承境內船乘。承知其姦計,分半與之。



 敦尋構難,遣參軍桓羆說承,以劉隗專寵,今便討擊,請承以為軍司,以軍期上道。承嘆曰:「吾其死矣!地荒人鮮,勢孤援絕。赴君難,忠也;死王事,義也。惟忠與義,夫復何求!」便欲唱義,而眾心疑惑。承曰:「吾受國恩,義無有貳。」府長
 史虞悝慷慨有志節,謂承曰:「王敦居分陜之任,而一旦作逆,天地所不容,人神所痛疾。大王宗室籓屏,寧可從其偽邪!便宜電奮,存亡以之。」於是與悝及弟前丞相掾望、建昌太守長沙王循、衡陽太守淮陵劉翼等共盟誓,囚桓羆,馳檄湘州,指期至巴陵。零陵太守尹奉首同義謀,出軍營陽,於是一州之內,皆同義舉。乃使虞望討諸不服,斬湘東太守鄭澹。澹,敦姊夫也。



 敦遣南蠻校尉魏乂、將軍李恒、田嵩等甲卒二萬以攻承。承且戰且守,待救於尹奉、虞望,而城池不固,人情震恐。或勸承南投陶侃,又云可退據零桂。承曰:「吾舉義眾,志在死節,寧偷生
 茍免,為奔敗之將乎!事之不濟,其令百姓知吾心耳。」



 初,安南將軍甘卓與承書,勸使固守,當以兵出沔口,斷敦歸路,則湘圍自解。承答書曰:「季思足下:勞於王事。天綱暫圮,中原丘墟。四海義士,方謀剋復,中興江左,草創始爾,豈圖惡逆萌自寵臣。吾以闇短,託宗皇屬。仰豫密命,作鎮南夏,親奉中詔,成規在心。伯仁諸賢,扼腕歧路,至止尚淺,凡百茫然。豺狼易驚,遂肆醜毒,聞知駭踴,神氣衝越。子來之義,人思自百,不命而至,眾過數千。誠足以決一旦之機。攄山海之憤矣。然迫於倉卒,舟楫未備,魏乂、李恒,尋見圍逼,是故事與意違,志力未展。猥辱來使,
 深同大趣;嘉謀英算,發自深衷。執讀周復,欣無以量。足下若能卷甲電赴,猶或有濟,若其狐疑,求我枯魚之肆矣。兵聞拙速,未睹工遲。季思足下,勉之勉之!書不盡意,絕筆而已。」



 卓軍次者口,聞王師敗績,停師不進,乂等攻戰日逼,敦又送所得臺中人書疏,令乂射以示承。城內知朝廷不守,莫不悵惋。劉翼戰死,相持百餘日,城遂沒。乂檻送承荊州,刺史王暠承敦旨於道中害之,時年五十九。敦平,詔贈車騎將軍。子無忌立。



 烈王無忌字公壽,承之難,以年小獲免。咸和中,拜散騎侍郎,累遷屯騎校尉、中書、黃門侍郎。江州刺史褚裒當
 之鎮,無忌及丹陽尹桓景等餞於版橋。時王暠子丹陽丞耆之在坐,無忌志欲復仇,拔刀將手刃之,裒、景命左右救捍獲免。御史中丞車灌奏無忌欲專殺人,付廷尉科罪。成帝詔曰:「王敦作亂,閔王遇禍,尋事原情,今王何責。然公私憲制,亦巳有斷,王當以體國為大,豈可尋繹由來,以亂朝憲。主者其申明法令,自今已往,有犯必誅。」於是聽以贖論。



 建元初遷散騎常侍,轉御史中丞,出為輔國將軍、長沙相,又領江夏相,尋轉南郡、河東二郡太守,將軍如故。隨桓溫伐蜀,以勛賜少子愔爵廣晉伯,進號前將軍,永和六年薨,贈衛將軍。二子:恬、愔。恬立。



 敬王恬,字元愉,少拜散騎侍郎,累遷散騎常侍、黃門郎、御史中丞。值海西廢,簡文帝登阼,未解嚴,大司馬桓溫屯中堂,吹警角,恬奏劾溫大不敬,請科罪。溫視奏嘆曰:「此兒乃敢彈我,真可畏也。」



 恬忠正有幹局,在朝憚之。遷右衛將軍、司雍秦梁四州大中正,拜尚書,轉侍中,領左衛將軍,補吳國內史,又領太子詹事。恬既宗室勛望,有才用,孝武帝時深杖之,以為都督兗、青、冀、幽並揚州之晉陵、徐州之南北郡軍事,領鎮北將軍、兗青二州刺史、假節。太元十五年薨,追贈車騎將軍。四子:尚之、恢之、允之、休之。尚之立。



 忠王尚之,字伯道,初拜祕書郎,遷散騎侍郎。恬鎮京口,尚之為振威將軍、廣陵相,父憂去職。服闋,為驃騎諮議參軍。宗室之內,世有人物。王國寶之誅也,散騎常侍劉鎮之、彭城內史劉涓子,徐州別駕徐放並以同黨被收,將加大辟。尚之言於會稽王道子曰:「刑獄不可廣,宜釋鎮之等。」道子以尚之昆季並居列職,每事仗焉,乃從之。



 兗州刺史王恭忌其盛也,與豫州刺史庾楷並稱兵,以討尚之為名,南連荊州刺史殷仲堪、南郡公桓玄等。道子命前將軍王珣、右將軍謝琰討恭,尚之距楷。允之與楷子鴻戰於當利,鴻敗走,斬楷將段方,楷單馬奔于桓
 玄。道子以尚之為建威將軍、豫州刺史、假節,一依楷故事,尋進號前將軍;允之為吳國內史;恢之驃騎司馬、丹楊尹;休之襄城太守。各擁兵馬,勢傾朝廷。後將軍元顯執政,亦倚以為援。



 元顯寵倖張法順,每宴會,坐起無別。尚之入朝,正色謂元顯曰:「張法順驅走小人,有何才異,而暴被拔擢。當今聖世,不宜如此。」元顯默然。尚之又曰:「宗室雖多,匡諫者少,王者尚納芻蕘之言,況下官與使君骨肉不遠,蒙眷累世,何可坐視得失而不盡言。」因叱法順令下。舉坐失色,尚之言笑自若,元顯深銜之。後符下西府,令出勇力二千人。尚之不與,曰:「西籓濱接荒餘,
 寇虜無常,兵止數千,不足戍衛,無復可分徹者。」元顯尤怒,會欲伐桓玄,故無他。



 及元顯稱詔西伐,命尚之為前鋒,尚之子文仲為寧遠將軍、宣城內史。桓玄至姑熟,遣馮該等攻歷陽,斷洞浦,焚尚之舟艦。尚之率步卒九千陣於浦上,先遣武都太守楊秋屯橫江。秋奔于玄軍,尚之眾潰,逃于塗中十餘日。譙國人韓連、丁元等以告玄,玄害之於建康市。玄上疏以閔王不宜絕嗣,乃更封尚之從弟康之為譙縣王。安帝反正,追贈尚之衛將軍,以休之長子文思為尚之嗣,襲封譙郡王。



 文思性凶暴,每違軌度,多殺弗辜。好田獵,燒人墳墓,數
 為有司所糾,遂與群小謀逆。劉裕聞之,誅其黨與,送文思付父休之,令自訓厲。後與休之同怨望稱兵,為裕所敗而死,國除。



 恢之字季明,歷官驃騎司馬、丹楊尹。尚之為桓玄所害,徙恢之等於廣州,而於道中害之。安帝反正,追贈撫軍將軍。



 休之字季預。少仕清塗,以平王恭、庾楷功,拜龍驤將軍、襄城太守,鎮歷陽。桓玄攻歷陽,休之嬰城固守。及尚之戰敗,休之以五百人出城力戰,不捷,乃還城,攜子姪奔於慕容超。聞義軍起,復還京師。大將軍武陵王令曰:「前
 龍驤將軍休之,才幹貞審,功業既成。歷陽之戰,事在機捷。及至勢乖力屈,奉身出奔,猶鳩集義徒,崎嶇險阻。既應親賢之舉,宜委分陜之重。可監荊益梁寧秦雍六州軍事、領護南蠻校尉、荊州刺史、假節。」到鎮無幾,桓振復襲江陵,休之戰敗,出奔襄陽。寧朔將軍張暢之、高平相劉懷肅自沔攻振,走之。休之還鎮,御史中丞王楨之奏休之失戍,免官。朝廷以豫州刺史魏詠之代之,徵休之還京師,拜後將軍、會稽內史。御史中丞阮歆之奏休之與尚書虞嘯父犯禁嬉戲,降號征虜將軍,尋復為後將軍。



 及盧循作逆,加督漸江東五郡軍事,坐公事免。劉毅
 誅,復以休之都督荊雍梁秦寧益六州軍事、平西將軍、荊州刺史、假節。以子文思為亂,上疏謝曰:「文思不能聿修,自貽罪戾,憂懼震惶,惋愧交集。臣御家無方,威訓不振,致使子姪愆法,仰負聖朝。悚赧兼懷,胡顏自處,請解所任,歸罪闕庭。」不許。



 後以文思事怨望,遂結雍州刺史魯宗之,將共誅執政。時休之次子文寶及兄子文祖並在都,收付廷尉賜死。劉裕親自征之,密使遺休之治中韓延之書曰:「文思事意,遠近所知。去秋遣康之送還司馬君者,推至公之極也。而了無愧心,久絕表疏,此是天地所不容。吾受命西征,止其父子而已。彼土僑舊,為之
 驅逼,一無所問。往年郗僧施、謝劭、任集之等交構積歲,專為劉毅規謀,所以至此。今卿諸人一時逼迫,本無纖釁。吾虛懷期物,自有由來,今在近路,是諸賢濟身之日。若大軍相臨,交鋒接刃,蘭艾雜揉,或恐不分。故白此意,並可示同懷諸人。」



 延之報曰:「聞親率戎馬,遠履西畿,闔境士庶,莫不恇駭。何者?莫知師出之名故也。辱來疏,始委以譙王前事,良增歎息。司馬平西體國忠貞,款懷待物。以君有匡復之勛,家國蒙賴,推德委誠,每事詢仰。譙王往以微事見劾,猶自遜位,況以大過,而當默然也!但康之前言,有所不盡,故重使胡道,申白所懷,道未及反,
 已表奏廢之,所不盡者命耳。推寄相與,正當如此,有何不可,便及兵戈。自義旗以來,方伯誰敢不先相諮疇,而徑表天子,可謂欲加之罪,其無辭乎!劉裕足下,海內之人,誰不見足下此心。而復欲誑國士,『天地所不容。』在彼不在此矣。來言『虛懷期物,自有由來』;今伐人之君,啖人以利,真可謂『虛懷期物,自有由來』矣!劉籓死於閶闔之門,諸葛弊於左右之手。甘言詫方伯,襲之以輕兵,遂使席上靡款懷之士,閫外無自信諸侯。以是為得算,良可恥也。吾誠鄙劣,嘗聞道於君子。以平西之至德,寧可無授命之臣乎!假令天長喪亂,九流渾濁,當與臧洪游
 於地下耳。」裕得書嘆息,以示諸佐曰:「事人當應如此!」



 宗之聞裕向荊州,自襄陽就休之共屯江陵。使文思及宗之子軌以兵距裕,戰于江津。休之大敗,遂與宗之俱奔于姚興。裕平姚泓,休之將奔于魏,未至,道死。



 允之字季度,出後叔父愔,襲爵廣晉伯,歷位輔國將軍、吳國宣城譙梁內史。王恭、庾楷、桓玄等內伐也,會稽王道子命允之兄弟距楷,破之。元興初,與兄恢之同徙廣州,於道被害。義軍起,追贈太常卿。從弟康之以子文惠襲爵。宋受禪,國除。



 韓延之,字顯宗,南陽赭陽人,魏司徒暨之後也。少以分義稱。安帝時為建威將軍、荊州治中,轉平西府錄事參軍。以劉裕父名翹字顯宗,延之遂字顯宗,名兒為翹,以示不臣劉氏。與休之俱奔姚興。劉裕入關,又奔于魏。



 愔字敬王,初封廣晉伯。早卒,無子,兄恬以子允之嗣。



 高陽王睦,字子友,譙王遜之弟也。魏安平亭侯,歷侍御史。武帝受禪,封中山王,邑五千二百戶。睦自表乞依六蓼祀皋陶,鄫杞祀相立廟。事下太常,依禮典平議。博士祭酒劉憙等議:「《禮記·王制》,諸侯五廟,二昭二穆,與太祖而五。是則立始祖之廟,謂嫡統承重,一人得立耳。假令
 支弟並為諸侯,始封之君不得立廟也。今睦非為正統,若立祖廟,中山不得並也。後世中山乃得為睦立廟,為後世子孫之始祖耳。」詔曰:「禮文不明,此制度大事,宜令詳審,可下禮官博議,乃處當之。」



 咸寧三年,睦遣使募徙國內八縣受逋逃、私占及變易姓名、詐冒復除者七百餘戶,冀州刺史杜友奏睦招誘逋亡,不宜君國。有司奏,事在赦前,應原。詔曰:「中山王所行何乃至此,覽奏甚用憮然。廣樹親戚,將以上輔王室,下惠百姓也。豈徒榮崇其身,而使民踰典憲乎!此事當大論得失,正臧否所在耳。茍不宜君國,何論於赦令之間耶。其貶睦為縣侯。」乃
 封丹水縣侯。



 及吳平,太康初詔復爵。有司奏封江陽王,帝曰:「睦退靜思愆,改修其德,今有爵土,不但以赦。江陽險遠,其以高陽郡封之。」乃封為高陽王。元康元年,為宗正。薨於位,世子蔚早卒,孫毅立。拜散騎侍郎,永嘉中沒於石勒。隆安元年,詔以譙敬王恬次子恢之子文深繼毅後。立五年,薨,無嗣,復以高密王純之子法蓮繼之。宋受禪,國除。



 任城景王陵,字子山,宣帝弟魏司隸從事安城亭侯通之子也。初拜議郎。泰始元年封北海王,邑四千七百戶。
 三年,轉封任城王,之國。咸寧五年薨,子濟立。拜散騎侍郎、給事中、散騎常侍、輔國將軍。隨東海王越在項,為石勒所害,二子俱沒。有二弟:順、斌。



 順字子思,初封習陽亭侯。及武帝受禪,順歎曰:「事乖唐虞,而假為禪名!」遂悲泣。由是廢黜,徙武威姑臧縣。雖受罪流放,守意不移而卒。



 西河繆王斌,字子政,魏中郎。武帝受禪,封陳王,邑千七百一十戶。三年,改封西河。咸寧四年薨,子隱立。薨,子LE立。



 史臣曰:泰始之初,天下少事,革魏餘弊,遵周舊典,並建
 宗室,以為籓翰。諸父同虞虢之尊,兄弟受魯衛之祉,以為歷紀長久,本支百世。安平風度宏邈,器宇高雅,內弘道義,外闡忠貞。洎高貴薨殂,則枕尸流慟;陳留就國,則拜辭隕涕。語曰『疾風彰勁草』,獻王其有焉。故能位班上列,享年眉壽,清徽至範,為晉宗英,子孫遵業,世篤其慶。高密風監清遠,簡素寡欲,孝以承親,忠以奉上,方諸枝庶,實謂國楨。新蔡、南陽,俱蒞方嶽。值王室多難,中原蕪梗,表義甄節,效績艱危。于時醜類實繁,凶威日逞,勢懸眾釁,相繼淪亡,悲夫!譙閔沈雄壯勇,作鎮南服。屬姦回肆亂,稱兵內侮。懷忠憤發,建義湘州,荊沔響應,群才致
 力。雖元勛不立,而誠節克彰,垂裕後昆,奕世貞烈,豈不休哉!勛托末屬,稟性凶暴。仍荷朝寄,推觳梁民,遂棄親背主,負恩放命。憑庸蜀之饒,苞藏不逞;恃江山之固,姦謀日深。是以搢紳切齒,攄積憤之志;義士思奮,厲忘身之節。天道禍淫,應時蕩定。昔汲黯猶在,淮南寢謀,周撫若存,凶渠未發,以邪忌正,異代同規。《詩》云「自貽伊戚」,其勛之謂矣。習陽憑慶枝葉,守約懷逸,棲情塵外,希蹤物表,顧匹夫之獨善,貴達節之弘規,言出身播,猶為幸也。



 贊曰:安平立節,雅性貞亮。高密含和,宗室之望。新蔡遇禍,忠全元喪。譙閔徇義,力屈志揚。勳自貽戚,名隕身亡。
 順不恤忌,流播遐方。



\end{pinyinscope}