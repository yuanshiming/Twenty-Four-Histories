\article{列傳第六十二}

\begin{pinyinscope}

 文苑



 應貞成公綏左思趙至鄒湛棗據褚陶王沉張翰庾闡曹毗李充袁宏伏滔羅含顧愷之郭澄之



 夫
 文以化成,惟聖之高義;行而不遠,前史之格言。是以溫洛禎圖,綠字符其丕業;苑山靈篆,金簡成其帝載。既而書契之道聿興,鐘石之文逾廣,移風俗於王化,崇孝敬於人倫,經緯乾坤,彌綸中外,故知文之時義大哉遠矣!



 洎姬歷云季,歌頌滋繁,荀宋之流,導源自遠,總金羈而齊騖,揚玉軑而並馳,言泉會於九流,交律詣於六變。
 自時已降,軌躅同趨,西都賈馬,耀靈蛇於掌握,東漢班張,發雕龍於綈槧,俱標稱首,咸推雄伯。逮乎當塗基命,文宗鬱起,三祖葉其高韻,七子分其麗則,《翰林》總其菁華,《典論》詳其澡絢,彬蔚之美,競爽當年。獨彼陳王,思風遒舉,備乎典奧,懸諸日月。



 及金行纂極,文雅斯盛,張載擅銘山之美,陸機挺焚研之奇,潘夏連輝,頡頏名輩,並綜採繁縟,杼軸清英,窮廣內之青編,緝平臺之麗曲,嘉聲茂迹,陳諸別傳。至於吉甫、太沖,江右之才傑;曹毗、庾闡,中興之時秀。信乃金相玉潤,林薈川沖,埒美前修,垂裕來葉。今撰其鴻筆之彥,著之《文苑》云。



 應貞,字吉甫,汝南南頓人,魏侍中璩之子也。自漢至魏,世以文章顯,軒冕相襲,為郡盛族。貞善談論,以才學稱。夏侯玄有盛名,貞詣玄,玄甚重之。舉高第,頻歷顯位。武帝為撫軍大將軍,以為參軍。及踐阼,遷給事中。帝於華林園宴射,貞賦詩最美。其辭曰:



 悠悠太上,人之厥初。皇極肇建,彞倫攸敷。五德更運,應錄受符。陶唐既謝,天歷在虞。於時上帝,乃顧惟眷。光我晉祚,應期納禪。位以龍飛,文以豹變。玄澤滂流,仁風潛扇。區內宅心,方隅迴面。天垂其象,地耀其文。鳳鳴朝陽,龍翔景雲。嘉禾重穎,蓂
 莢載芬。率土咸寧,人胥悅欣。



 恢恢皇度,穆穆聖容。言思其允,貌思其恭。在視斯明。在聽斯聰。登庸以德,明試以功。其恭惟何?昧旦丕顯。無義不經,無理不踐。行舍其華,言去其辯。游心至虛,同規易簡。六府孔修,九有來踐。澤罔不被,化莫不加。聲教南暨,西漸流沙。幽人肆險,遠國忘遐;越常重譯,充牣皇家。峨峨列辟,赫赫武臣。內和五品,外威四賓。順時貢職,入覲天人。備言錫命,羽蓋朱輪。



 貽宴好會,不常厥數。神心所授,不言而喻。於時肆射,弓矢斯具。發彼互的,有酒斯飫。文武之道,厥猷未墜。在昔先王,射御茲器。示武懼荒,過則有失。凡厥群后,無懈於
 位。



 初置太子中庶子官,貞與護軍長史孔恂俱為之。後遷散騎常侍,以儒學與太尉荀顗撰定新禮,未施行。泰始五年卒,文集行於世。



 弟純。純子紹,永嘉中,至黃門郎,為東海王越所害。純弟秀,秀子詹,自有傳。



 成公綏,字子安,東郡白馬人也。幼而聰敏,博涉經傳。性寡欲,不營資產,家貧歲饑,常晏如也。少有俊才,詞賦甚麗,閑默自守,不求聞達。時有孝烏,每集其廬舍,綏謂有反哺之德,以為祥禽,乃作賦美之,文多不載。又以「賦者貴能分賦物理,敷演無方,天地之盛,可以致思矣。歷觀
 古人未之有賦,豈獨以至麗無文,難以辭贊;不然,何其闕哉?」遂為《天地賦》曰:



 惟自然之初載兮,道虛無而玄清,太素紛以溷淆兮,始有物而混成,何元一之芒昧兮,廓開闢而著形。爾乃清濁剖分,玄黃判離。太極既殊,是生兩儀,星辰煥列,日月重規,天動以尊,地靜以卑,昏明迭照,或盈或虧,陰陽協氣而代謝,寒暑隨時而推移。三才殊性,五行異位,千變萬化,繁育庶類,授之以形,稟之以氣。色表文採,聲有音律,覆載無方,流形品物。鼓以雷霆,潤以慶雲,八風翱翔,六氣氤氳。蚑行蠕動,方聚類分,鱗殊族別,羽毛異群,各含精而熔冶,咸受範於陶鈞,何滋
 育之罔極兮,偉造化之至神!



 若天懸象成文,列宿有章,三辰燭耀,五緯重光,河漢委蛇而帶天,虹兒偃蹇於昊蒼,望舒彌節於九道,義和正轡於中黃,眾星回而環極,招搖運而指方,白獸峙據於參伐,青龍垂尾於心房,玄龜匿首於女虛,朱鳥奮翼於注張,帝皇正坐於紫宮,輔臣列位於文昌,垣屏駱驛而珠連,三台差池而鴈翔,軒轅華布而曲列,攝提鼎峙而相望。若乃徵瑞表祥,災變呈異,交會薄蝕,抱暈帶珥,流逆犯歷,譴悟象事,蓬容著而妖害生,老人形而主受喜,天矢黃而國吉祥,彗孛發而世所忌。



 爾乃旁觀四極,俯察地理,川瀆浩汗而分流,
 山嶽磊落而羅峙,滄海沆漭而四周,懸圃隆崇而特起,昆吾嘉於南極,燭龍曜於北址,扶桑高于萬仞,尋木長於千里,崑崙鎮於陰隅,赤縣據于辰巳。於是八十一域,區分方別;風乖俗異,險斷阻絕,萬國羅布,九州並列。青冀白壤,荊衡塗泥,海岱赤埴,華梁青黎,兗帶河洛,揚有江淮。辯方正土,經略建邦,王圻九服,列國一同,連城比邑,深池高墉,康衢交路,四達五通。東至陽谷,西極泰濛,南暨丹炮,北盡空同。遐方外區,絕域殊鄰,人首蛇軀,烏翼龍身,衣毛被羽,或介或鱗,棲林浮水,若獸若人,居于大荒之外,處於巨海之濱。



 於是六合混一而同宅,宇宙
 結體而括囊,渾元運流而無窮,陰陽循度而率常,回動糾紛而乾乾,天道不息而自彊。統群生而載育,人託命於所繫,尊太一於上皇,奉萬神於五帝,故萬物之所宗,必敬天而事地。



 若乃共工赫怒,天柱摧折,東南俄其既傾,西北豁而中裂,斷鰲足而續毀,煉玉石而補缺。豈斯事之有征,將言者之虛設?何陰陽之難測,偉二儀之奓闊!



 坤厚德以載物,乾資始而至大,俯盡鑒於有形,仰蔽視於所蓋,游萬物而極思,故一言于天外。



 綏雅好音律,嘗當暑承風而嘯,泠然成曲,因為《嘯賦》曰:



 逸群公子,體奇好異,敖世忘榮,絕棄人事,希高慕古,長想遠思,將登
 箕山以抗節,浮滄海以游志。於是延友生,集同好,精性命之至機,研道德之玄奧,愍流俗之未悟,獨超然而先覺,狹世路之阨人闢,仰天衢而高蹈,邈跨俗而遺身,乃慷慨而長嘯。于時曜靈俄景,流光濛汜,逍遙攜手,躊躇步趾,發妙聲於丹脣,激哀音於皓齒,響抑揚而潛轉,氣衝鬱而熛起,協黃宮於清角,雜商羽於流徵,飄浮雲於泰清,集長風于萬里。曲既終而響絕,餘遺玩而未已,良自然之至音,非絲竹之所擬。是故聲不假器,用不借物,近取諸身,役心御氣。動脣有曲,發口成音,觸類感物,因歌隨吟。大而不洿,細而不沈,清激切於竽笙,優潤和瑟
 琴,玄妙足以通神悟靈,精微足以窮幽測深,收激楚之哀荒,節北里之奢淫,濟洪災於炎旱,反亢陽於重陰。引唱萬變,曲用無方,和樂怡懌,悲傷摧藏。時幽散而將絕,中矯歷而慷慨,徐婉約而優游,紛繁騖而激揚。情既思而能反,心雖哀而不傷。總八音之至和,固極樂而無荒。



 若乃登高臺以臨遠,披文軒而騁望,喟仰抃而抗首,嘈長引而憀亮。或舒肆而自反,或徘徊而復放,或冉弱而柔撓,右澎濞而奔壯。橫鬱嗚而滔涸,列繚眺而清昶。逸氣奮涌,繽紛交錯,烈烈飆揚,啾啾響作。奏胡馬之長思,迴寒風乎北朔,又似鴻鴈之將雛,群鳴號乎沙漠。故能
 因形創聲,隨事造曲,應物無窮,機發響速,怫鬱衝流,參譚雲屬,若離若合,將絕復續。飛廉鼓於幽隧,猛獸應於中谷;南箕動於穹蒼,清飆振于喬木;散滯積而播揚,蕩埃靄之溷濁,變陰陽於至和,移淫風之穢俗。



 若乃游崇岡,陵景山,臨巖側,望流川,坐磐石,漱清泉,藉皋蘭之猗靡,蔭修竹之蟬蜎,乃吟詠而發歎,聲驛驛而響連,舒蓄思之悱憤,奮久結之纏綿,心滌蕩而無累,志離俗而飄然。



 若夫假象金革,擬則陶匏,眾聲繁奏,若笳若簫;磞硠震隱,訇蓋聊嘈。發徵則隆冬熙烝,騁羽則嚴霜夏凋,動商則秋霖春降,奏角則谷風鳴條。音均不恒,曲無定制,
 行而不流,止而不滯,隨口吻而發揚,假芳氣而遠逝,音要妙而流響,聲激嚁而清厲。信自然之極麗,羌殊尤而絕世,越《韶》《夏》與《咸池》,何徒取異乎《鄭》《衛》!



 于時綿駒結舌而喪精,王豹杜口而失色,虞公輟聲而止歌,寧子斂手而歎息,鐘期棄琴而改聽,尼父忘味而不食,百獸率儛而拤足,鳳皇來儀而拊翼。乃知長嘯之奇妙,此音聲之至極。



 張華雅重綏,每見其文,歎伏以為絕倫,薦之太常,徵為博士。歷秘書郎,轉丞,遷中書郎。每與華受詔並為詩賦,又與賈充等參定法律。泰始九年卒,年四十三,所著詩賦雜筆十餘卷行於世。



 左思,字太沖,齊國臨淄人也。其先齊之公族有左右公子,因為氏焉。家世儒學。父雍,起小吏,以能擢授殿中侍御史。思小學鐘、胡書及鼓琴,並不成。雍謂友人曰:「思所曉解,不及我少時。」思遂感激勤學,兼善陰陽之術。貌寢,口訥,而辭藻壯麗。不好交遊,惟以閑居為事。造《齊都賦》,一年乃成。復欲賦三都,會妹芬入宮,移家京師,乃詣著作郎張載,訪岷邛之事。遂構思十年,門庭籓溷,皆著筆紙,遇得一句,即便疏之。自以所見不博,求為秘書郎。及賦成,時人未之重。思自以其作不謝班張,恐以人廢言,
 安定皇甫謐有高譽,思造而示之。謐稱善,為其賦序。張載為注《魏都》,劉逵注《吳》《蜀》而序之曰:「觀中古以來為賦者多矣,相如《子虛》擅名於前,班固《兩都》理勝其辭,張衡《二京》文過其意。至若此賦,擬議數家,傅辭會義,抑多精致,非夫研核者不能練其旨,非夫博物者不能統其異。世咸貴遠而賤近,莫肯用心於明物。斯文吾有異焉,故聊以餘思為其引詁,亦猶胡廣之於《官箴》,蔡邕之於《典引》也。」陳留衛權又為思賦作《略解》,序曰:「余觀《三都》之賦,言不茍華,必經典要,品物殊類,稟之圖籍;辭義瑰瑋,良可貴也。有晉徵士故太子中庶子安定皇甫謐,西州之
 逸士,耽籍樂道,高尚其事,覽斯文而慷慨,為之都序。中書著作郎安平張載、中書郎濟南劉逵,並以經學洽博,才章美茂,咸皆悅玩,為之訓詁;其山川土域,草木鳥獸,奇怪珍異,僉皆研精所由,紛散其義矣。余嘉其文,不能默已,聊藉二子之遺忘,又為之《略解》,祗增煩重,覽者闕焉。」自是之後,盛重於時,文多不載。司空張華見而嘆曰:「班張之流也。使讀之者盡而有餘,久而更新。」於是豪貴之家競相傳寫,洛陽為之紙貴。初,陸機入洛,欲為此賦,聞思作之,撫掌而笑,與弟雲書曰:「此間有傖父,欲作《三都賦》,須其成,當以覆酒甕耳。」及思賦出,機絕歎伏,以為
 不能加也,遂輟筆焉。



 秘書監賈謐請講《漢書》,謐誅,退居宜春裏,專意典籍。齊王冏命為記室督,辭疾,不就。及張方縱暴都邑,舉家適冀州。數歲,以疾終。



 趙至,字景真,代郡人也。寓居洛陽。緱氏令初到官,至年十三,與母同觀。母曰:「汝先世本非微賤,世亂流離,遂為士伍耳。爾後能如此不?」至感母言,詣師受業。聞父耕叱牛聲,投書而泣。師怪問之,至曰:「我小未能榮養,使老父不免勤苦。」師甚異之。年十四,詣洛陽,游太學,遇嵇康於學寫石經,徘徊視之,不能去,而請問姓名。康曰:「年少何
 以問邪?」曰:「觀君風器非常,所以問耳。」康異而告之。後乃亡到山陽,求康不得而還。又將遠學,母禁之,至遂陽狂,走三五里,輒追得之。年十六,游鄴,復與康相遇,隨康還山陽,改名浚,字允元。康每曰:「卿頭小而銳,童子白黑分明,有白起之風矣。」及康卒,至詣魏興見太守張嗣宗,甚被優遇。嗣宗遷江夏相,隨到溳川,欲因入吳,而嗣宗卒,乃向遼西而占戶焉。



 初,至與康兄子蕃友善,及將遠適,乃與蕃書敘離,并陳其志曰:



 昔李叟入秦,及關而歎;梁生適越,登嶽長謠。夫以嘉遁之舉,猶懷戀恨,況乎不得已者哉!惟別之後,離群獨逝,背榮宴,辭倫好,經迥路,造
 沙漠。雞鳴戒旦,則飄爾晨征;日薄西山,則馬首靡託。尋歷曲阻,則沈思紆結;登高遠眺,則山川攸隔。或乃迴風狂厲,白日寢光,徙倚交錯,陵隰相望,徘徊九皋之內,慷慨重阜之顛,進無所由,退無所據,涉澤求蹊,披榛覓路,嘯詠溝渠,良不可度。斯亦行路之艱難,然非吾心之所懼也。至若蘭芷傾頓,桂林移殖,根萌未樹而牙淺弦急,每恐風波潛駭,危機密發,此所以怵惕於長衢也。又北土之性,難以託根,投人夜光,鮮不按劍。今將殖橘柚於玄朔,蒂華藕於修陵,表龍章於裸壤,奏《韶》《武》於聾俗,固難以取貴矣。夫物不我貴則莫之與,莫之與則傷之者
 至矣。飄颻遠游之士,託身無人之鄉,總轡遐路,則有前言之難;懸鞍陋宇,則有後慮之戒;朝霞啟暉,則身疲而遄征;太陽戢曜,則情劬而夕惕;肆目平隰,則寥廓而無睹;極聽修原,則掩寂而無聞。吁其悲矣!心傷瘁矣!然後知步驟之士不足為貴也。



 顧景中原,憤中雲踴,哀物悼世,激情風厲。龍嘯大野,獸睇六合,猛志紛紜,雄心四據。思躡雲梯,橫奮八極,披艱掃穢,蕩海夷嶽,蹴崑崙使西倒,蹋太山令東覆,平滌九區,恢維宇宙,斯吾之鄙願也。時不我與,垂翼遠逝,鋒距靡加,六翮摧屈,自非知命,孰能不憤悒者哉!吾子殖根芳苑,濯秀清流,晞葉華崖,飛
 藻雲肆,俯據潛龍之渚,仰蔭游鳳之林,榮曜眩其前,艷色餌其後,良疇交其左,聲名馳其右,翱翔倫黨之間,弄姿帷房之裹,從容顧眄,綽有餘裕,俯仰吟嘯,自以為得志矣,豈能與吾曹同大丈夫之憂樂哉!



 去矣嵇生,遠離隔矣!煢煢飄寄,臨沙漠矣!悠悠三千,路難涉矣!攜手之期,邈無日矣!思心彌結,誰云釋矣!無金玉爾音而有遐心。身雖胡越,意存斷金。各敬爾儀,敦履璞沈,繁華流蕩,君子弗欽。臨紙意結,知復何云。



 至身長七尺四寸,論議精辯,有從橫才氣。遼西舉郡計吏,到洛,與父相遇。時母已亡,父欲令其宦立,弗之告,仍戒以不歸,至乃還遼西。
 幽州三辟部從事,斷九獄,見稱精審。太康中,以良吏赴洛,方知母亡。初,至自恥士伍,欲以宦學立名,期於榮養。既而其志不就,號憤慟哭,歐血而卒,時年三十七。



 鄒湛,字潤甫,南陽新野人也。父軌,魏左將軍。湛少以才學知名,仕魏歷通事郎、太學博士。泰始初,轉尚書郎、廷尉平、征南從事中郎,深為羊祜所器重。入為太子中庶子。太康中,拜散騎常侍,出補渤海太守,轉太傅楊駿長史,遷侍中。駿誅,以僚佐免官。尋起為散騎常侍、國子祭酒,轉少府。元康末卒,所著詩及論事議二十五首,為時
 所重。



 初,湛嘗夢見一人,自稱甄舒仲,餘無所言,如此非一。久之,乃悟曰:「吾宅西有積土敗瓦,其中必有死人。甄舒仲者,予舍西土瓦中人也。」檢之,果然,厚加斂葬。葬畢,遂夢此人來謝。



 子捷,字太應,亦有文才。永康中,為散騎侍郎。及趙王倫篡逆,捷與陸機等俱作禪文。倫誅,坐下廷尉,遇赦免。後為太傅參軍。永嘉末,卒。



 棗據,字道彥,潁川長社人也。本姓棘,其先避仇改焉。父叔禕,魏鉅鹿太守。據美容貌,善文辭。弱冠,辟大將軍府,出為山陽令,有政績。遷尚書郎,轉右丞。賈充伐吳,請為
 從事中郎。軍還,徙黃門侍郎、冀州刺史、太子中庶子。太康中卒,時年五十餘。所著詩賦論四十五首,遇亂多亡失。



 子腆,字玄方,亦以文章顯。永嘉中為襄城太守。弟嵩,字臺產,才藝尤美,為太子中庶子、散騎常侍,為石勒所殺。



 褚陶,字季雅,吳郡錢塘人也。弱不好弄,少而聰慧,清淡閑默,以墳典自娛。年十三,作《鷗鳥》、《水磑》二賦,見者奇之。陶嘗謂所親曰:「聖賢備在黃卷中,捨此何求!」州郡辟,不就。吳平,召補尚書郎。張華見之,謂陸機曰:「君兄弟龍躍
 雲津,顧彥先鳳鳴朝陽,謂東南之寶已盡,不意復見褚生。」機曰:「公但未睹不鳴不躍者耳。」華曰:「故知延州之德不孤,川嶽之寶不匱矣。」遷九真太守,轉中尉。年五十五卒。



 王沉,字彥伯,高平人也。少有俊才,出於寒素,不能隨俗沈浮,為時豪所抑。仕郡文學掾,鬱鬱不得志,乃作《釋時論》,其辭曰:



 東野丈人觀時以居,隱耕汙腴之墟。有冰氏之子者,出自冱寒之谷,過而問塗。丈人曰:「子奚自?」曰:「自涸陰之鄉。」「奚適?」曰:「欲適煌煌之堂。」丈人曰:「入煌煌之堂
 者,必有赫赫之光。今子困於寒而欲求諸熱,無得熱之方。」冰子瞿然曰:「胡為其然也?」丈人曰:「融融者皆趣熱之士,其得爐冶之門者,惟挾炭之子。茍非斯人,不如其已。」冰子曰:「吾聞宗廟之器不要華林之木,四門之賓何必冠蓋之族。前賢有解韋索而佩朱韍舍徒擔而乘丹轂。由此言之,何恤而無祿!惟先生告我塗之速也。」



 丈人曰:「嗚呼!子聞得之若是,不知時之在彼。吾將釋子。夫道有安危,時有險易,才有所應,行有所適。英奇奮於從橫之世,賢智顯於霸王之初,當厄難則騁權譎以良圖,值制作則展儒道以暢攄,是則袞龍出於縕褐,卿相起於匹
 夫,故有朝賤而夕貴,先卷而後舒。。當斯時也,豈計門資之高卑,論勢位之輕重乎!今則不然。上聖下明,時隆道寧,群后逸豫,宴安守平。百辟君子,奕世相生,公門有公,卿門有卿。指禿腐骨,不簡蚩儜。多士豐於貴族,爵命不出閨庭。四門穆穆,綺襦是盈,仍叔之子,皆為老成。賤有常辱,貴有常榮,肉食繼踵於華屋,疏飯襲跡於耨耕。談名位者以諂媚附勢,舉高譽者因資而隨形。至乃空囂者以泓噌為雅量,瑣慧者以淺利為鎗鎗,脢胎者以無檢為弘曠,僂垢者以守意為堅貞。嘲哮者以粗發為高亮,韞蠢者以色厚為篤誠,痷婪者以博納為通濟,眂々
 者以難入為凝清,拉答者有沈重之譽,嗛閃者得清剿之聲,嗆啍怯畏於謙讓,闒茸勇敢於饕諍。斯皆寒素之死病,榮達之嘉名。凡茲流也,視其用心,察其所安,責人必急,於己恒寬。德無厚而自貴,位未高而自尊,眼罔嚮而遠視,鼻而刺天。忌惡君子,悅媚小人,敖蔑道素,懾吁權門。心以利傾,智以勢惛,姻黨相扇,毀譽交紛。當局迷於所受,聽採惑於所聞。京邑翼翼,群士千億,奔集勢門,求官買職,童僕窺其車乘,閽寺相其服飾,親客陰參於靖室,疏賓徙倚於門側。時因接見,矜歷容色,心懷內荏,外詐剛直,譚道義謂之俗生,論政刑以為鄙極。高
 會曲宴,惟言遷除消息,官無大小,問是誰力。今以子孤寒,懷真抱素,志陵雲霄,偶景獨步,直順常道,關津難渡,欲騁韓盧,時無狡兔,眾塗圮塞,投足何錯!」



 於是冰子釋然乃悟曰:「富貴人之所欲,貧賤人之所惡。僕少長於孔顏之門,久處於清寒之路,不謂熱勢自共遮錮。敬承明誨,服我初素,彈琴詠典,以保年祚。伯成、延陵,高節可慕。丹轂滅族,呂霍哀吟,朝榮夕滅,旦飛暮沈。聃周道師,巢由德林。豐屋蔀家,《易》著明箴。人薄位尊,積罰難任,三郤尸晉,宋華咎深,投扃正幅,實獲我心。」



 是時王政陵遲,官才失實,君子多退而窮處,遂終於里閭。



 元康初,松滋令
 吳郡蔡洪字叔開,有才名,作《孤奮論》,與《釋時》意同,讀之者莫不歎息焉。



 張翰,字季鷹,吳郡吳人也。父儼,吳大鴻臚。翰有清才,善屬文,而縱任不拘,時人號為「江東步兵。」會稽賀循赴命入洛,經吳閶門,於船中彈琴。翰初不相識,乃就循言譚,便大相欽悅。問循,知其入洛,翰曰:「吾亦有事北京。」便同載即去,而不告家人。齊王冏辟為大司馬東曹掾。冏時執權,翰謂同郡顧榮曰:「天下紛紛,禍難未已。夫有四海之名者,求退良難。吾本山林間人,無望於時。子善以明
 防前,以智慮後。」榮執其手,愴然曰:「吾亦與子採南山蕨,飲三江水耳。」翰因見秋風起,乃思吳中菰菜、蓴羹、鱸魚膾,曰:「人生貴得適志,何能羈宦數千里以要名爵乎!」遂命駕而歸。著《首丘賦》,文多不載。俄而冏敗,人皆謂之見機。然府以其輒去,除吏名。翰任心自適,不求當世。或謂之曰:「卿乃可縱適一時,獨不為身後名邪?」答曰:「使我有身後名,不如即時一盃酒。」時人貴其曠達。性至孝,遭母憂,哀毀過禮。年五十七卒。其文筆數十篇行於世。



 庾闡,字仲初,潁川鄢陵人也。祖輝,安北長史。父東,以勇
 力聞。武帝時,有西域健胡趫捷無敵,晉人莫敢與校。帝募勇士,惟東應選,遂撲殺之,名震殊俗。闡好學,九歲能屬文。少隨舅孫氏過江。母隨兄肇為樂安長史,在項城。永嘉末,為石勒所陷,闡母亦沒。闡不櫛沐,不婚宦,絕酒肉,垂二十年,鄉親稱之。州舉秀才,元帝為晉王,辟之,皆不行。後為太宰、西陽王羕掾,累遷尚書郎。蘇峻之難,闡出奔郗鑒,為司空參軍。峻平,以功賜爵吉陽縣男,拜彭城內史。鑒復請為從事中郎。尋召為散騎侍郎,領大著作。頃之,出補零陵太守,入湘川,弔賈誼。其辭曰:



 中興二十三載,餘忝守衡南,鼓栧三江,路次巴陵,望君山而過
 洞庭,涉湘川而觀汨水,臨賈生投書之川,慨以永懷矣。及造長沙,觀其遺象,喟然有感,乃弔之云。



 偉哉蘭生而芳,玉產而潔,陽葩熙冰,寒松負雪,莫邪挺鍔,天驥汗血,茍云其雋,誰與比傑!是以高明倬茂,獨發奇秀,道率天真,不議世疚,煥乎若望舒耀景而焯群星,矯乎若翔鸞拊翼而逸宇宙也。飛榮洛汭,擢穎山東,質清浮磬,聲若孤桐,瑯瑯其璞,巖巖其峰,信道居正,而以天下為公,方駕逸步,不以曲路期通。是以張高弘悲,聲激柱落,清唱未和,而桑濮代作,雖有惠音,莫過《韶》《濩》;雖有騰鱗,終仆一壑。嗚呼!大庭既邈,玄風悠緬,皇道不以智隆,上德不
 以仁顯。三五親譽,其輒可仰而標;霸功雖逸,其塗可翼而闡,悲矣先生,何命之蹇!懷寶如玉,而生運之淺!



 昔咎繇謨虞,呂尚歸昌,德協充符,乃應帝王。夷吾相桓,漢登蕭張;草廬三顧,臭若蘭芳。是以道隱則蠖屈,數感則鳳睹,若棲不擇木,翔非九五,雖曰玉折,雋才何補!夫心非死灰,智必存形,形託神用,故能全生。奈何蘭膏,揚芳漢庭,摧景飆風,獨喪厥明。悠悠太素,存亡一指,道來斯通,世往斯圮。吾哀其生,未見其死,敢不敬弔,寄之淥水。



 後以疾,徵拜給事中,復領著作。吳國內史虞潭為太伯立碑,闡製其文。又作《揚都賦》,為世所重。年五十四卒,謚曰
 貞,所著詩賦銘頌十卷行於世。



 子肅之,亦有文藻著稱,歷給事中、相府記室、湘東太守。太元中卒。



 曹毗,字輔佐,譙國人也。高祖休,魏大司馬。父識,右軍將軍。毗少好文籍,善屬詞賦。郡察孝廉,除郎中,蔡謨舉為佐著作郎。父憂去職。服闋,遷句章令,徵拜太學博士。時桂陽張碩為神女杜蘭香所降,毗因以二篇詩嘲之,并續蘭香歌詩十篇,甚有文彩。又著《揚都賦》,亞於庾闡。累遷尚書郎、鎮軍大將軍從事中郎、下邳太守。以名位不至,著《對儒》以自釋。其辭曰:



 或問曹子曰:「夫寶以含珍為
 貴,士以藏器為峻,麟以絕迹標奇,松以負霜稱雋,是以蘭生幽澗,玉輝於仞。故子州浮滄瀾而龍蟠,吳季忽萬乘以解印,虞公潛崇巖以頤神,梁生適南越以保慎,固能全真養和,夷跡洞潤,陵冬揚芳,披雪獨振也。



 「今少子睎冥風,弱挺秀容,奇以幼齡,翰披孺童。吐辭則藻落楊班,抗心則志擬高鴻,味道則理貫莊肆,研妙則穎奪豪鋒。固以騰廣莫而萋蒨,排素薄而青葱者矣,何必以刑禮為己任,申韓為宏通!既登東觀,染史筆;又據太學,理儒功。曾無玄韻淡泊,逸氣虛洞,養採幽翳,晦明蒙籠。不追林棲之迹,不希抱鱗之龍,不營練真之術,不慕內聽
 之聰。而處汎位以核物,扇塵教以自濛,負鹽車以顯能,飾一己以求恭。退不居漆園之場,出不躡曾城之衝,游不踐綽約之室,諆不希騄駬之蹤;徒以區區之懷而整名目之典,覆蕢之量而塞北川之洪,檢名實於俄頃之間,定得失乎一管之鋒。



 「子若謂我果是邪?則是不必以合俗。子若云俗果非邪?則俗非不可以茍從。俗我紛以交爭,利害渾而彌重,何異執朽轡以御逸駟,承勁風以握秋蓬,役恬性以充勞府,對群物以耦怨雙者乎?子不聞乎終軍之穎,賈生之才,拔奇山東,玉映漢臺,可謂響播六合,聲駭嬰孩,而見毀絳灌之口,身離狼狽之災。由
 此言之,名為實賓,福萌禍胎,朝敷榮華,夕歸塵埃,未若澄虛心於玄圃,蔭瑤林於蓬萊,絕世事而雋黃綺,鼓滄川而浪龍鰓者矣。蒙竊惑焉。」



 主人煥耳而笑,欣然而言曰:「夫兩儀既闢,陰陽汗浩,五才迭用,化生紛擾,萬類云云,孰測其兆!故不登閬風,安以瞻殊目之形?不步景宿,何以觀恢廓之表?是以迷麤者循一往之智,狷介者守一方之矯,豈知火林之蔚炎柯,冰津之擢陽草!故大人達觀,任化昏曉,出不極勞,處不巢皓,在儒亦儒,在道亦道,運屈則紆其清暉,時申則散其龍藻,此蓋員動之用舍,非尋常之所寶也。



 「今三明互照,二氣載宣,玄教夕凝,
 朗風晨鮮,道以才暢,化隨理全。故五典剋明於百揆,虞音齊響於五弦,安期解褐於秀林,漁父擺鉤於長川。如斯則化無不融,道無不延,風澄於俗,波清于川。方將舞黃虯於慶雲,招儀鳳於靈山,流玉醴乎華闥,秀朱草於庭前。何有違理之患,累真之嫌!子徒知辯其說而未測其源,明朝菌不可踰晦朔,蟪蛄無以觀大年,固非管翰之所述,聊敬對以終篇。」



 累遷至光祿勛,卒。凡所著文筆十五卷,傳於世。



 李充,字弘度,江夏人。父矩,江州刺史。充少孤,其父墓中
 柏樹嘗為盜賊所斫,充手刃之,由是知名。善楷書,妙參鐘索,世咸重之。辟丞相王導掾,轉記室參軍。幼好刑名之學,深抑虛浮之士,嘗著《學箴》,稱:



 《老子》云:「絕仁棄義,家復孝慈。」豈仁義之道絕,然後李慈乃生哉?蓋患乎情仁義者寡而利仁義者眾也。道德喪而仁義彰,仁義彰而名利作,禮教之弊,直在茲也。先王以道德之不行,故以仁義化之,行仁義之不篤,故以禮律檢之;檢之彌繁,而偽亦愈廣,老莊是乃明無為之益,塞爭欲之門。夫極靈智之妙、總會通之和者,莫尚乎聖人。革一代之弘制,垂千載之遺風,則非聖不立。然則聖人之在世,吐言則為
 訓辭,蒞事則為物軌,運通則與時隆,理喪則與世弊矣。是以大為之論以標其旨。物必有宗,事必有主,寄責於聖人而遺累乎陳迹也。故化之以絕聖棄智,鎮之以無名之樸。聖教救其末,老莊明其本,本末之塗殊而為教一也。人之迷也,其日久矣!見形者眾,及道者鮮,不覿千仞之門而遂適物之迹,逐迹逾篤,離本逾遠,遂使華端與薄俗俱興,妙緒與淳風並絕,所以聖人長潛而迹未嘗滅矣。懼後進惑其如此,將越禮棄學而希無為之風,見義教之殺而不觀其隆矣,略言所懷,以補其闕。引道家之弘旨,會世教之適當,義之違本,言不流放,庶以祛
 困蒙之蔽,悟一往之惑乎!其辭曰:



 芒芒太初,悠悠鴻荒,蚩蚩萬類,與道兼忘。聖迹未顯,賢名不彰,怡此鼓腹,率我猖狂。資生既廣,群塗思通,闇實師明,匪予求蒙,遺己濟物而天下為公。大庭唱基,義農宏贊,六位時成,離暉大觀,澤洽雨濡,化流風散,比屋同塵而人罔僭亂。爰暨中古,哲王胥承,質文代作,禮統迭興,事藉用以繁,化因阻而凝,動非性擾,靜豈神澄!名之攸彰,道之攸廢,乃損所隆,乃崇所替,刑作由於德衰,三辟興乎叔世,既敦既誘,乃矯乃厲。敦亦既備,矯亦既深,彫琢生文,抑揚成音,群能騁技,眾巧竭心,野無陸馬,山無散林。風罔不動,化
 罔不移,人之失德,反正作奇。乃放欲以越禮,不知希競之為病,違彼夷塗而遵此險徑。狡兔陵岡,游魚遁川,至賾深妙,大象幽玄,棄餌收罝而責功蹄筌,先統喪歸而寄旨忘言。政異徵辭,拔本塞源,遁迹永日,尋響窮年,刻意離性而失其自然。世有險夷,運有通圮,損益適時,升降惟理。道不可以一日廢,亦不可以一朝擬,禮不可以千載制,亦不可以當年止。非仁無以長物,非義無以齊恥,仁義固不可遠,去其害仁義者而己。力行猶懼不逮,希企邈以遠矣。室有善言,應在千里,況乎行止復禮克己。風人司箴,敬貽君子。



 征北將軍褚裒又引為參軍,充
 以家貧,苦求外出,裒將許之為縣,試問之,充曰:「窮猿投林,豈暇擇木!」乃除縣令,遭母憂。服闋,為大著作郎。



 于時典籍混亂,充刪除煩重,以類相從,分作四部,甚有條貫,秘閣以為永制。累遷中書侍郎,卒官。充注《尚書》及《周易旨》六篇、《釋莊論》上下二篇、詩賦表頌等雜文二百四十首,行於世。



 子顒,亦有文義,多所述作,郡舉孝廉。



 充從兄式,以平隱著稱,善楷隸。中興初,仕至侍中。



 袁宏,字彥伯,侍中猷之孫也。父勖,臨汝令。宏有逸才,文章絕美,曾為詠史詩,是其風情所寄。少孤貧,以運租自
 業。謝尚時鎮牛渚,秋夜乘月,率爾與左右微服泛江。會宏在舫中諷詠,聲既清會,辭又藻拔,遂駐聽久之,遣問焉。答云:「是袁臨汝郎誦詩。」即其詠史之作也。尚傾率有勝致,即迎升舟,與之譚論,申旦不寐,自此名譽日茂。尚為安西將軍、豫州刺史,引宏參其軍事。累遷大司馬桓溫府記室。溫重其文筆,專綜書記。後為《東征賦》,賦末列稱過江諸名德,而獨不載桓彞。時伏滔先在溫府,又與宏善,苦諫之。宏笑而不答。溫知之甚忿,而憚宏一時文宗,不欲令人顯問。後游青山飲歸,命宏同載,眾為之懼。行數里,問宏云:「聞君作《東征賦》,多稱先賢,何故不及家
 君?」宏答曰:「尊公稱謂非下官敢專,既未遑啟,不敢顯之耳。」溫疑不實,乃曰:「君欲為何辭?」宏即答云:「風鑒散朗,或搜或引,身雖可亡,道不可隕,宣城之節,信義為允也。」溫泫然而止。宏賦又不及陶侃,侃子胡奴嘗於曲室抽刃問宏曰:「家君勳跡如此,君賦云何相忽?」宏窘急,答曰:「我已盛述尊公,何乃言無?」因曰:「精金百汰,在割能斷,功以濟時,職思靜亂,長沙之勳,為史所贊。」胡奴乃止。



 後為《三國名臣頌》曰:



 夫百姓不能自牧,故立君以治之;明君不能獨治,則為臣以佐之。然則三五迭隆,歷代承基,揖讓之與干戈,文德之與武功,莫不宗匠陶鈞而群才緝熙,
 元首經略而股肱肆力。雖遭罹不同,迹有優劣,至於體分冥固,道契不墜,風美所扇,訓革千載,其揆一也。故二八升而唐朝盛,伊呂用而湯武寧,三賢進而小白興,五臣顯而重耳霸。中古陵遲,斯道替矣。居上者不以至公理物,為下者必以私路期榮,御員者不以信誠率眾,執方者必以權謀自顯。於是君臣離而名教薄,世多亂而時不治,故蘧寧以之卷舒,柳下以之三黜,接輿以之行歌,魯連以之赴海。衰世之中,保持名節,君臣相體,若合符契,則燕昭、樂毅古之流矣。夫未遇伯樂,則千載無一驥;時值龍顏,則當年控三傑,漢之得賢,於斯為貴。高祖
 雖不以道勝御物,群下得盡其忠;蕭曹雖不以三代事主,百姓不失其業。靜亂庇人,抑亦其次。夫時方顛沛,則顯不如隱;萬物思治,則默不如語。是以古之君子不患弘道難,患遭時難;遭時匪難,遇君難。故有道無時,孟子所以咨嗟;有時無君,賈生所以垂泣。夫萬歲一期,有生之通塗;千載一遇,賢智之嘉會。遇之不能無欣,喪之何能無慨。古人之言,信有情哉!余以暇日常覽《國志》,考其君臣,比其行事,雖道謝先代,亦異世一時也。



 文若懷獨見之照,而有救世之心,論時則人方塗炭,計能則莫出魏武,故委圖霸朝,豫謀世事。舉才不以標鑒,故人亡而
 後顯;籌畫不以要功,故事至而後定。雖亡身明順,識亦高矣。



 董卓之亂,神器遷逼,公達慨然,志在致命。由斯而譚,故以大存名節。至如身為漢隸而跡入魏幕,源流趣舍,抑亦文若之謂。所以存亡殊致,始終不同,將以文若既明且哲,名教有寄乎!夫仁義不可不明,時宗舉其致;生理不可不全,故達識攝其契。相與弘道,豈不遠哉!



 崔生高朗,折而不撓,所以策名魏武、執笏霸朝者,蓋以漢主當陽,魏后北面者哉!若乃一旦進璽,君臣易位,則崔生所以不與,魏氏所以不容。夫江湖所以濟舟,亦所以覆舟;仁義所以全身,亦所以亡身。然而先賢玉摧於
 前,來哲攘袂於後,豈天懷發中,而名教束物者乎!



 孔明盤桓,俟時而動,遐想管樂,遠明風流,治國以禮,人無怨聲,刑罰不濫,沒有餘泣,雖古之遺愛,何以加茲!及其臨終顧託,受遺作相,劉后授之無疑心,武侯受之無懼色,繼體納之無貳情,百姓信之無異辭,君臣之際,良可詠矣!



 公瑾卓爾,逸志不群,總角料主,則素契於伯符;晚節曜奇,則三分於赤壁。惜其齡促,志未可量。



 子布佐策,致延譽之美,輟哭止哀,有翼戴之功,神情所涉,豈徒謇諤而已哉!然杜門不用,登壇受譏。夫一人之身所照未異,而用舍之間俄有不同,況沈跡溝壑,遇與不遇者乎!



 夫
 詩頌之作,有自來矣。或以吟詠情性,或以紀德顯功,雖大指同歸,所託或乖。若夫出處有道,名體不滯,風軌德音,為世作範,不可廢也。復綴序所懷,以為之贊曰:



 火德既微,運纏大過。洪飆扇海,二溟揚波。虯獸雖驚,風雲未和。潛魚擇川,高鳥候柯。赫赫三雄,並迴乾軸。競收杞梓,爭採松竹。鳳不及棲,龍不暇伏。谷無幽蘭,嶺無停菊。



 英英文若,靈鑒洞照。應變知微,頤奇賞要。日月在躬,隱之彌曜。文明英心,贊之愈妙。滄海橫流,玉石俱碎。達人兼善,廢己存愛。謀解時紛,功濟宇內。始救生靈,終明風概。



 公達潛朗,思同蓍蔡。運用無方,動攝群會。爰初發迹,遘
 此顛沛。神情玄定,處之彌泰。愔愔幕裹,算無不經亹癖通韻,跡不暫停。雖懷尺璧,顧哂連城。智能極物,愚足全生。



 郎中溫雅,器識純素。貞而不諒,通而能固。恂恂德心,汪汪軌度。志成弱冠,道敷歲暮。仁者必勇,德亦有言。雖遇履尾,神氣恬然。行不修飾,名節無愆。操不激切,素風愈鮮。



 邈哉崔生,體正心直。天骨疏朗,牆岸高嶷。忠存軌跡,義形風色。思樹芳蘭,翦除荊棘。人惡其上,世不容哲。琅瑯先生,雅杖名節。雖遇塵務,猶震霜雪。運極道消,碎此明月。



 景山恢誕,韻與道合。形器不存,方寸海納。和而不同,通而不雜。遇醉忘辭,在醒貽答。



 長文通雅,義格終
 始。思戴元首,擬伊同恥。人未知德,懼若在己。嘉謀肆庭,讜言盈耳。玉生雖麗,光不踰把。德積雖微,道暎天下。



 邈哉太初,宇量高雅。器範自然。標準無假。全身由直,跡洿必偽。處死匪難,理存則易。萬物波蕩,孰任其累!六合徒廣,容身靡寄。君親自然,匪由名教。愛敬既同,情禮兼到。



 烈烈王生,知死不撓。求仁不遠,期在忠存。



 玄伯剛簡,大存名體。志在高構,增堂及陛。端委獸門,正言彌啟。臨危致命,盡其心禮。



 堂堂孔明,基宇宏邈。器同生靈,獨稟先覺。標榜風流,遠明管樂。初九龍盤,雅志彌確。百六道喪,干戈迭用。茍非命世,孰掃雰雺!宗子思寧,薄言解控。釋
 褐中林,鬱為時棟。



 士元弘長,雅性內融。崇善愛物,觀始知終。喪亂備矣。勝塗未隆。先生標之,振起清風。綢繆哲后,無妄惟時。夙夜匪懈,義在緝熙。三略既陳,霸業已基。



 公琰殖根,不忘中正。豈曰模擬,實在雅性。亦既羈勒,負荷時命。推賢恭己,久而可敬。



 公衡沖達,秉志淵塞。媚茲一人,臨難不惑。疇昔不造,假翮鄰國。進能徽音,退不失德。六合紛紜,人心將變。鳥擇高梧,臣須顧眄。



 公瑾英達,朗心獨見。披草求君,定交一面。桓桓魏武,外託霸跡。志掩衡霍,恃戰忘敵。卓卓若人,曜奇赤壁。三光參分,宇宙暫隔。



 子布擅名,遭世方擾。撫翼桑梓,息肩江表。王略威
 夷,吳魏同寶。遂贊宏謨,匡此霸道。桓王之薨,大業未純。把臂託孤,惟賢與親。轟哭止哀,臨難忘身。成此南面,實由老臣。才為世生,世亦須才。得而能任,貴在無猜。



 昂昂子敬,拔跡草萊。荷簷吐奇,乃構雲臺。



 子瑜都長,體性純懿。諫而不犯,正而不毅。將命公庭,退忘私位。豈無鶺鴒,固慎名器。



 伯言謇謇,以道佐世。出能勤功,入亦獻替。謀寧社稷,妥紛挫銳。正以招疑,忠而獲戾。



 元歎邈遠,神和形檢。如彼白珪,質無塵點。立行以恒,匡主以漸。清不增潔,濁不加染。



 仲翔高亮,性不和物。好是不群,折而不屈。屢摧逆鱗,直道受黜。歎過孫陽,放同賈屈。



 莘莘眾賢,千
 載一遇。整轡高衢,驤首天路。仰揖玄流,俯弘時務。名節殊塗,雅致同趣。日月麗天,瞻之不墜。仁義在躬,用之不匱。尚想遐風,載揖載味。後生擊節,懦夫增氣。



 從桓溫北征,作《北征賦》,皆其文之高者。嘗與王珣、伏滔同在溫坐,溫令滔讀其《北征賦》,至「聞所傳於相傳,云獲麟於此野,誕靈物以瑞德,奚授體於虞者!疚尼父之洞泣,似實慟而非假。豈一性之足傷,乃致傷於天下」,其本至此便改韻。珣云:「此賦方傳千載,無容率耳。今於『天下』之後,移韻徙事,然於寫送之致,似為未盡。」滔云:「得益寫韻一句,或為小勝。」溫曰:「卿思益之。」宏應聲答曰:「感不絕於餘心,愬
 流風而獨寫。」珣誦味久之,謂滔曰:「當今文章之美,故當共推此生。」



 性彊正亮直,雖被溫禮遇,至於辯論,每不阿屈,故榮任不至。與伏滔同在溫府,府中呼為「袁伏」。宏心恥之,每歎曰:「公之厚恩未優國士,而與滔比肩,何辱之甚。」



 謝安常賞其機對辯速。後安為揚州刺史,宏自吏部郎出為東陽郡,乃祖道於冶亭。時賢皆集,安欲以卒迫試之,臨別執其手,顧就左右取一扇而授之曰:「聊以贈行。」宏應聲答曰:「輒當奉揚仁風,慰彼黎庶。」時人歎其率而能要焉。



 宏見漢時傅毅作《顯宗頌》,辭甚典雅,乃作頌九章,頌簡文之德,上之於孝武。



 太元初,卒於東陽,時年
 四十九。撰《後漢紀》三十卷及《竹林名士傳》三卷、詩賦誄表等雜文凡三百首,傳於世。



 三子:長超子,次成子,次明子。明子有父風,最知名,官至臨賀太守。



 伏滔,字玄度,平昌安丘人也。有才學,少知名。州舉秀才,辟別駕,皆不就。大司馬桓溫引為參軍,深加禮接,每宴集之所,必命滔同游。從溫伐袁真,至壽陽,以淮南屢叛,著論二篇,名曰《正淮》。其上篇曰:



 淮南者,三代揚州之分也。當春秋時,吳、楚、陳、蔡之與地。戰國之末,楚全有之,而考烈王都焉。秦并天下,建立郡縣,是為九江。劉項之際,
 號曰東楚。爰自戰國至于晉之中興,六百有餘年,保淮南者九姓,稱兵者十一人,皆亡不旋踵,禍溢於世,而終莫戒焉。其天時歟,地勢歟,人事歟?何喪亂之若是也!試商較而論之。



 夫懸象著明,而休徵表於列宿;山河衿帶,而地險彰於丘陵;治亂推移,而興亡見於人事。由此而觀,則兼也必矣。昔妖星出於東南而弱楚以亡,飛孛橫於天漢而劉安誅絕,近則火星晨見而王凌首謀,長彗宵暎而毋丘襲亂。斯則表乎天時也。彼壽陽者,南引荊汝之利,東連三吳之富;北接梁宋,平塗不過七日;西援陳許,水陸不出千里;外有江湖之阻,內保淮肥之固。龍
 泉之陂,良疇萬頃,舒六之貢,利盡蠻越,金石皮革之具萃焉,苞木箭竹之族生焉,山湖藪澤之隈,水旱之所不害,土產草滋之實,荒年之所取給。此則係乎地利乎也。其俗尚氣力而多勇悍,其人習戰爭而貴詐偽,豪右并兼之門,十室而七;藏甲挾劍之家,比屋而發。然而仁義之化不漸,刑法之令不及,所以屢多亡國也。



 昔考烈以衰弱之楚屢遷其都,外迫彊秦之威,內遘陽申之禍,逃死劫殺,三世而滅。黥布以三雄之選,功成垓下,淮陰既囚,梁越受戮,嫌結震主之威,慮生同體之禍,遂謀圖全之計,庶幾後亡之福,眾潰於一戰,身脂於漢斧。劉長支
 庶,奄王大國,承喪亂之餘,御新化之俗,無德而寵,欲極禍發。王安內懷先父之憾,外眩姦臣之說,招引賓客,沈溺數術,藉二世之資,恃戈甲之盛,屈彊江淮之上,西向而圖宗國,言未絕口,身嗣俱滅。李憲因亡新之餘,袁術當衰漢之末,負力幸亂,遂生僭逆之計,建號九江,稱制下邑,狼狽奔亡,傾城受戮。及至彥雲、仲恭、公休之徒,或憑宿名,或怙前功,握兵淮楚,力制東夏,屬當多難之世,仍值廢興之會,謀非所議,相係禍敗。祖約助逆,身亡家族。彼十亂者,成乎人事者也。然則侵弱昏迷,以至絕滅,亡楚當之。恃彊畏逼,遂謀叛亂,黥布有焉。二王遘逆,寵
 之之過也。公路僭偽,乘釁之盜也。二將以圖功首難,士少以驕矜樂禍。本其所因,考其成跡,皆寵盛禍淫,福過災生,而制之不漸,積之有由也。



 其下篇曰:



 昔高祖之誅黥布也,撮三策之要,馳赦過之書,乘人主之威以除逆節之虜,然猶決戰陳都,暴尸橫野,僅乃剋之,害亦深矣!長安之謀,雖兵未交於山東,禍未遍於天下,而馳說之士與闔境之人幽囚誅放者,亦已眾矣。光武連兵於肥舒,魏祖馳馬於蘄苦,而廬九之間流溺兵兇者十而七八焉。夫王凌面縛,得之於砎石;仲恭接刃,成之於後覺也。而高祖以之宵征,世宗以之發疾,豈不勤哉!文皇挾
 萬乘之威,杖伊周之權,內舉京畿之眾,外征四海之銳,雲合雨集,推鋒以臨淮浦,而誕欽晏然,方嬰城自固,憑軾以觀王師。於是築長圍,起棼櫓,高壁連塹,負戈擊柝以守之。自夏及春,而後始知亡焉。然則屠城之禍,其可極言乎?約之出奔,淮左為墟,悲夫!



 信哉魯哀之言,夫生乎深宮,長於膏梁,憂懼不切於身,榮辱不交於前,則其仁義之本淺矣。奉以南面之尊,藉以列城之富,宅以制險之居,養以眾彊之盛,而無德以臨之,無制以節之,則厭溢樂禍之心生矣。夫以昏主御姦臣,利甲資堅城,偽令行於封內,邪惠結於人心,乘間幸濟之說日交於側,
 猾詐錮咎之群各馳於前,見利如歸,安在其不為亂乎!況乘舊寵,挾前功,畏逼懼亡,以謀圖身之舉者,望其俯首就羈,不亦迂哉!《易》稱「履霜堅冰,馴致之道,」蓋言漸也。嗚呼!斯所以亂臣賊子亡國覆家累世而不絕者歟!



 昔先生之宰天下也,選於有德,訪之三吏,正其分位,明其等級,畫之封疆,宣之政令,上下有序,無僭差之嫌,四人安業,無并兼之國。三載考陟,功罪不得逃其跡,九伐時修,刑賞無所謬其實。令之有漸,軌之有度,寵之有節,權不外授,威不下黷,所以杜其萌際,重其名器,深根固本,傳之百世。雖時有盛衰,弱者無所懼其亡;道有興廢,彊
 者不得資其弊。夫如是,將使天下從風,穆然軌道,慶自一人,惠流萬國,安有向時之患哉!



 壽陽平,以功封聞喜縣侯,除永世令。溫薨,征西將軍桓豁引為參軍,領華容令。太元中,拜著作郎,專掌國史,領本州大中正。孝武帝嘗會於西堂,滔豫坐,還,下車先呼子系之謂曰:「百人高會,天子先問伏滔在坐不,此故未易得。為人作父如此,定何如也?」遷游擊將軍,著作如故。卒官。



 子系之,亦有文才,歷黃門郎、侍中、尚書、光祿大夫。



 羅含,字君章,桂陽耒陽人也。曾祖彥,臨海太守。父綏,滎
 陽太守。含幼孤,為叔母朱氏所養。少有志尚,嘗晝臥,夢一鳥文彩異常,飛入口中,因驚起說之。朱氏曰:「鳥有文彩,汝後必有文章。」自此後藻思日新。弱冠,州三辟,不就。含父嘗宰新淦,新淦人楊羨後為含州將,引含為主簿,含傲然不顧,羨招致不已,辭不獲而就焉。及羨去職,含送之到縣。新淦人以含舊宰之子,咸致賂遺,含難違而受之。及歸,悉封置而去。由是遠近推服焉。後為郡功曹,刺史庾亮以為部江夏從事。太守謝尚與含為方外之好,乃稱曰:「羅君章可謂湘中之琳瑯。」尋轉州主簿。後桓溫臨州,又補征西參軍。溫嘗使含詣尚,有所檢劾。含至,
 不問郡事,與尚累日酣飲而還。溫問所劾事,含曰:「公謂尚何如人?」溫曰:「勝我也。」含曰:「豈有勝公而行非邪!故一無所問。」溫奇其意而不責焉。轉州別駕。以廨舍喧擾,於城西池小洲上立茅屋,伐木為材,織葦為席而居,布衣蔬食,晏如也。溫嘗與僚屬宴會,含後至。溫問眾坐曰:「此何如人?」或曰:「可謂荊楚之材。」溫曰:「此自江左之秀,豈惟荊楚而已。」徵為尚書郎。溫雅重其才,又表轉征西戶曹參軍。俄遷宜都太守。及溫封南郡公,引為郎中令。尋徵正員郎,累遷散騎常侍、侍中,仍轉廷尉、長沙相。年老致仕,加中散大夫,門施行馬。初,含在官舍,有一白雀妻集
 堂宇,及致仕還家,階庭忽蘭菊叢生,以為德行之感焉。年七十七卒,所著文章行於世。



 顧愷之,字長康,晉陵無錫人也。父悅之,尚書左丞。愷之博學有才氣,嘗為《箏賦》成,謂人曰:「吾賦之比嵇康琴,不賞者必以後出相遺,深識者亦當以高奇見貴。」桓溫引為大司馬參軍,甚見親暱。溫薨後,愷之拜溫墓,賦詩云:「山崩溟海竭,魚鳥將何依!」或問之曰:「卿憑重桓公乃爾,哭狀其可見乎?」答曰:「聲如震雷破山,淚如傾河注海。」愷之好諧謔,人多愛狎之。後為殷仲堪參軍,亦深被眷接。
 仲堪在荊州,愷之嘗因假還,仲堪特以布帆借之,至破塚,遭風大敗。愷之與仲堪箋曰:「地名破塚,真破冢而出。行人安穩,布帆無恙。」還至荊州,人問以會稽山川之狀。愷之云:「千巖競秀,萬壑爭流。草木蒙籠,若雲興霞蔚。」桓玄時與愷之同在仲堪坐,共作了語。愷之先曰:「火燒平原無遺燎。」玄曰:「白布纏根樹旒旐。」仲堪曰:「投魚深泉放飛鳥。」復作危語。玄曰:「矛頭淅米劍頭炊。」仲堪曰:「百歲老翁攀枯枝。」有一參軍云:「盲人騎瞎馬臨深池。」仲堪眇目,驚曰:「此太逼人!」因罷。愷之每食甘蔗,恒自尾至本。人或怪之,云:「漸入佳境。」



 尤善丹青,圖寫特妙,謝安深重之,以
 為有蒼生以來未之有也。愷之每畫人成,或數年不點目精。人問其故,答曰:「四體妍蚩,本無闕少於妙處,傳神寫照,正在阿堵中。」嘗悅一鄰女,挑之弗從,乃圖其形於壁,以棘鍼釘其心,女遂患心痛。愷之因致其情,女從之,遂密去鍼而愈。愷之每重嵇康四言詩,因為之圖,恒云:「手揮五絃易,目送歸鴻難。」每寫起人形,妙絕於時。嘗圖裴楷象,頰上加三毛,觀者覺神明殊勝。又為謝鯤象,在石巖裏,云:「此子宜置丘壑中。」欲圖殷仲堪,仲堪有目病,固辭。愷之曰:「明府正為眼耳,若明點瞳子,飛白拂上,使如輕雲之蔽月,豈不美乎!」仲堪乃從之。愷之嘗以一廚
 畫糊題其前,寄桓玄,皆其深所珍惜者。玄乃發其廚後,竊取畫,而緘閉如舊以還之,紿云未開。愷之見封題如初,但失其畫,直云妙畫通靈,變化而去,亦猶人之登仙,了無怪色。



 愷之矜伐過實,少年因相稱譽以為戲弄。又為吟詠,自謂得先賢風制。或請其作洛生詠,答曰:「何至作老婢聲!」義熙初,為散騎常侍,與謝瞻連省,夜於月下長詠,瞻每遙贊之,愷之彌自力忘倦。瞻將眠,令人代己,愷之不覺有異,遂申旦而止。尤信小術,以為求之必得。桓玄嘗以一柳葉紿之曰:「此蟬所翳葉也,取以自蔽,人不見己。」愷之喜,引葉自蔽,玄就溺焉,愷之信其不見己
 也,甚以珍之。



 初,愷之在桓溫府,常云:「愷之體中癡黠各半,合而論之,正得平耳。」故俗傳愷之有三絕:才絕,畫絕,癡絕。年六十二,卒於官,所著文集及《啟蒙記》行於世。



 郭澄之,字仲靜,太原陽曲人也。少有才思,機敏兼人。調補尚書郎,出為南康相。值盧循作逆,流離僅得還都。劉裕引為相國參軍。從裕北伐,既剋長安,裕意更欲西伐,集僚屬議之,多不同。次問澄之,澄之不答,西向誦王粲詩曰:「南登霸陵岸,回首望長安。」裕便意定,謂澄之曰:「當與卿共登霸陵岸耳。」因還。



 澄之位至裕相國從事中郎,
 封南豐侯,卒於官,所著文集行於世。



 史臣曰:夫賞好生於情,剛柔本於性,情之所適,發乎詠歌,而感召無象,風律殊製。至於應貞宴射之文,極形言之美,華林群藻罕或疇之。子安幼標明敏,少蓄清思,懷天地之寥廓,賦辭人之所遺,特構新情,豈常均之所企!太沖含豪歷載,以賦《三都》,士安見而稱善,平原睹而韜翰,匪惟高步當年,故以騰華終古。鄒湛之持論,棗據之緣情,實南陽之人傑,蓋潁川之時秀。季雅摛屬遒邁,夙備成德,稱為泉岱之珍,固其然矣。彥伯未能混迹光塵,而屈乎卑位,《釋時》宏論,亦足見其志耳。季鷹縱誕一時,
 不邀名爵,《黃花》之什,濬發神府。仲初之文,風流可尚,擢秀士林,《揚都》之美,尤重時彥。曹毗沈研秘籍,踠足下僚,綺靡降神之歌,朗暢《對儒》之論。李充之《學箴》,信清壯也。袁宏《東征》、《名臣》之作,抑潘陸之亞。玄度學藝優瞻,筆削擅奇,降帝問於西堂,故其榮觀也。君章耀湘中之寶,挺荊楚之材,夢鳥發乎精誠,豈獨日者之蛟鳳!長康矜能過實,譚諧取容,而才多逸氣,故有三絕之目。仲靜機思通敏,延譽清流,德輿西伐之計,取定於微指者矣。



 贊曰:爻彖垂法,宮征流音。美哉群彥,揚蕤翰林。俱諧振玉,各擅鏘金。子安、太沖,遒文綺爛。袁、庾、充、愷,縟藻霞煥。
 架彼辭人,共超清貫。



\end{pinyinscope}