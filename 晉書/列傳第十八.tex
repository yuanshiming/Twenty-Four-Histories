\article{列傳第十八}

\begin{pinyinscope}

 向雄段灼閻纘



 向雄,字茂伯,河內山陽人也。父韶,彭城太守。雄初仕郡為主簿,事太守王經。及經之死也,雄哭之盡哀,市人咸為之悲。後太守劉毅嘗以非罪笞雄,及吳奮代毅為太守,又以少譴繫雄於獄。司隸鐘會於獄中辟雄為都官從事,會死無人殯斂,雄迎喪而葬之。文帝召雄而責之曰:「往者王經之死,卿哭王經於東市,我不問也。今鐘會
 躬為叛逆,又輒收葬,若復相容,其如王法何!」雄曰:「昔者先王掩骼埋胔,仁流朽骨,當時豈先卜其功罪而後葬之哉!今王誅既加,於法已備。雄感義收葬,教亦無闕。法立於上,教弘於下,何必使雄違生背死以立於時!殿下仇枯骨而捐之中野,為將來仁賢之資,不亦惜乎!」帝甚悅,與談宴而遣之。



 累遷黃門侍郎。時吳奮、劉毅俱為侍中,同在門下,雄初不交言。武帝聞之,敕雄令復君臣之好。雄不得已,乃詣毅,再拜曰:「向被詔命,君臣義絕,如何?」於是即去。帝聞而大怒,問雄曰:「我令卿復君臣之好,何以故絕?」雄曰:「古之君子進人以禮,退人以禮;今之進人
 若加諸膝,退人若墜諸川。劉河內於臣不為戎首,亦已幸甚,安復為君臣之好!」帝從之。



 泰始中,累遷秦州刺史,假赤幢、曲蓋、鼓吹,賜錢二十萬。咸寧初,入為御史中丞,遷侍中,又出為征虜將軍。太康初,為河南尹,賜爵關內侯。齊王攸將歸籓,雄諫曰:「陛下子弟雖多,然有名望者少。齊王臥在京邑,所益實深,不可不思。」帝不納。雄固諫忤旨,起而徑出,遂以憤卒。



 弟匡,惠帝世為護軍將軍。



 段灼,字休然,敦煌人也。世為西土著姓,果直有才辯。少仕州郡,稍遷鄧艾鎮西司馬,從艾破蜀有功,封關內侯,
 累遷議郎。武帝即位,灼上疏追理艾曰:



 故征西將軍鄧艾,心懷至忠,而荷反逆之名;平定巴、蜀,而受三族之誅,臣竊悼之。惜哉,言艾之反也!以艾性剛急,矜功伐善,而不能協同朋類,輕犯雅俗,失君子之心,故莫肯理之。臣敢昧死言艾所以不反之狀。



 艾本屯田掌犢人,宣皇帝拔之於農吏之中,顯之於宰府之職。處內外之官,據文武之任,所在輒有名績,固足以明宣皇帝之知人矣。會值洮西之役,官兵失利,刺史王經困於圍城之中。當爾之時,二州危懼,隴右懍懍,幾非國家之有也。先帝以為深憂重慮,思惟可以安邊殺敵莫賢於艾,故授之以兵
 馬,解狄道之圍。圍解,留屯上邽。承官軍大敗之後,士卒破膽,將吏無氣,倉庫空虛,器械殫盡。艾欲積穀彊兵,以待有事。是歲少雨,又為區種之法,手執耒耜,率先將士,所統萬數,而身不離僕虜之勞,親執士卒之役。故落門、段谷之戰,能以少擊多,摧破彊賊,斬首萬計。遂委艾以廟勝成圖,指授長策。艾受命忘身,龍驤麟振,前無堅敵。蜀地阻險,山高谷深,而艾步乘不滿二萬,束馬懸車,自投死地,勇氣陵雲,將士乘勢,故能使劉禪震怖,君臣面縛。軍不踰時,而巴、蜀蕩定,此艾固足以彰先帝之善任矣。



 艾功名已成,亦當書之竹帛,傳祚萬世。七十老公,復
 何所求哉!艾以禪初降,遠郡未附,矯令承制,權安社稷。雖違常科,有合古義,原心定罪,事可詳論。故鎮西將軍鐘會,有吞天下之心,恐艾威名,知必不同,因其疑似,構成其事。艾被詔書,即遣彊兵,束身就縛,不敢顧望。誠自知奉見先帝,必無當死之理也。會受誅之後,艾參佐官屬、部曲將吏,愚戇相聚,自共追艾,破壞檻車,解其囚執。艾在困地,是以狼狽失據。夫反非小事,若懷惡心,即當謀及豪傑,然後乃能興動大眾,不聞艾有腹心一人。臨死口無惡言,獨受腹背之誅,豈不哀哉!故見之者垂涕,聞之者歎息。此賈誼所以慷慨於漢文,天下之事可為
 痛哭者,良有以也。



 陛下龍興,闡弘大度,受誅之家,不拘敘用,聽艾立後,祭祀不絕。昔秦人憐白起之無罪,吳人傷子胥之冤酷,皆為之立祠。天下之人為艾悼心痛恨,亦由是也。謂可聽艾門生故吏收艾尸柩,歸葬舊墓,還其田宅,以平蜀之功,繼封其後,使艾闔棺定謚,死無所恨。赦冤魂於黃泉,收信義於後世,則天下徇名之士,思立功之臣,必投湯火,樂為陛下死矣!



 帝省表,甚嘉其意。灼後復陳時宜曰:



 臣聞天時不如地利,地利不如人和。三里之城,五里之郭,圜圍而攻之,有不剋者,此天時不如地利。城非不高,池非不深,穀非不多,兵非不利,委而
 去之,此地利不如人和。然古之王者,非不先推恩德,結固人心。人心茍和,雖三里之城,五里之郭,不可攻也。人心不和,雖金城湯池,不能守也。臣推此以廣其義,舜彈五弦之琴,詠《南風》之詩,而天下自理,由堯人可比屋而封也。曩者多難,姦雄屢起,攪亂眾心,刀鋸相乘,流死之孤,哀聲未絕。故臣以為陛下當深思遠念,杜漸防萌,彈琴詠詩,垂拱而已。其要莫若推恩以協和黎庶,故推恩足以保四海,不推恩不足以保妻子。是故唐堯以親睦九族為先,周文以刑於寡妻為急,明王聖主莫不先親後疏,自近及遠。臣以為太宰、司徒、衛將軍三王宜留洛
 中鎮守,其餘諸王自州徵足任者,年十五以上悉遣之國。為選中郎傅相,才兼文武,以輔佐之。聽於其國繕修兵馬,廣布恩信。必撫下猶子,愛國如家,君臣分定,百世不遷,連城開地,為晉、魯、衛。所謂盤石之宗,天下服其彊矣。雖云割地,譬猶囊漏貯中,亦一家之有耳。若慮後世彊大,自可豫為制度,使得推恩以分子弟。如此則枝分葉布,稍自削小,漸使轉至萬國,亦後世之利,非所患也。



 昔在漢世,諸呂自疑,內有朱虛、東牟之親,外有諸侯九國之彊,故不敢動搖。於今之宜,諸侯彊大,是為太山之固。非我族類,其心必異。而魏法禁錮諸王,親戚隔絕,不
 祥莫大焉。間者無故又瓜分天下,立五等諸侯。上不象賢,下不議功,而是非雜糅,例受茅土。似權時之宜,非經久之制,將遂不改,此亦煩擾之人,漸亂之階也。夫國之興也,由於九族親睦,黎庶協和;其衰也,在於骨肉疏絕,百姓離心。故夏邦不安,伊尹歸殷;殷邦不和,呂氏入周。殷監在於夏后,去事之誡,誠來事之鑒也。



 又陳曰:



 昔伐蜀,募取涼州兵馬、羌胡健兒,許以重報,五千餘人,隨艾討賊,功皆第一。而《乙亥詔書》,州郡將督,不與中外軍同,雖在上功,無應封者。唯金城太守楊欣所領兵,以逼江由之勢,得封者三十人。自金城以西,非在欣部,無一人封
 者。茍在中軍之例,雖下功必侯;如在州郡,雖功高不封,非所謂近不重施,遠不遺恩之謂也。



 臣聞魚懸由於甘餌,勇夫死於重報。故荊軻慕燕丹之義,專諸感闔閭之愛,匕首振於秦庭,吳刀耀於魚腹,視死如歸,豈不有由也哉!夫功名重賞,士之所競,不平致怨,由來久矣。《詩》云:「尸鳩在桑,其子七兮。淑人君子,其儀一兮。」臣以為此等宜蒙爵封。



 灼前後陳事,輒見省覽。然身微宦孤,不見進序,乃取長假還鄉里。臨去,遣息上表曰:



 臣受恩三世,剖符守境,試用無績,沈伏數年,犬馬之力,無所復堪。陛下弘廣納之聽,採狂夫之言,原臣侵官之罪,不問干忤之
 愆,天地恩厚,於臣足矣。臣聞忠臣之於其君,猶孝子之於其親:進則有欣然之慶,非貪官也;退則有戚然之憂,非懷祿也。其意在於不忘光君榮親,情所不能已已者也。臣伏自悼,私懷至恨:生長荒裔,而久在外任,自還抱疾,未嘗覲見,陛下竟不知臣何人,此臣之恨一也。遭運會之世,值有事之時,而不能垂功名於竹帛,此臣之恨二也。逮事聖明之君,而尪悴羸劣,陳力又不能,當歸死於地下,此臣之恨三也。哀二親早亡隕,兄弟並凋喪,孝敬無復施於家門,此臣之恨四也。夏之日忽以過,冬之夜尋復來,人生百歲,尚以為不足,而臣中年嬰災,此臣
 之恨五也。慚日月之所養,愧昊蒼而無報,此臣之所以懷五恨而歎息,臨歸路而自悼者也。



 語有之曰:「華言虛也,至言實也,苦言藥也,甘言疾也。」臣欲言天下太平,而靈龜神狐未見,仙芝萐莆未生,麒麟未游乎靈禽之囿,鳳皇未儀於太極之庭,此臣之所以不敢華言而為佞者也。昔漢高祖初定天下,于時戍卒婁敬上書諫曰:「陛下取天下不與成周同,而欲比隆成周,臣竊以為不侔。」於是漢祖感悟,深納其言,賜姓為劉氏。又顧謂陸賈曰:「為我著秦所以亡,而吾所以得之者。」賈乃作《新語》之書,述敘前世成敗,以為勸戒。又田肯建一言之計,非親子
 弟莫可使王齊者,而受千金之賜。故世稱漢祖之寬明博納,所以能成帝業也。



 今之言世者,皆曰堯舜復興,天下已太平矣。臣獨以為未,亦竊有所勸焉。且百王垂制,聖賢吐言,來事之明鑒也。孟子曰:「堯不能以天下與舜,則舜之有天下也,天與之也。昔舜為相,堯崩,三年之喪畢,舜避堯之子於南河,天下諸侯朝覲者、獄訟者,不之堯之子而之舜。舜曰天也,乃之中國,踐天子位焉。若居堯之宮,逼堯之子,非天所與者也。」曩昔西有不臣之蜀,東有僭號之吳,三主鼎足,並稱天子。魏文帝率萬乘之眾,受禪於靡陂,而自以德同唐、虞,以為漢獻即是古之
 堯,自謂即是今之舜,乃謂孟柯、孫卿不通禪代之變,遂作禪代之文,刻石垂戒,班示天下,傳之後世,亦安能使將來君子皆曉然心服其義乎!然魏文徒希慕堯、舜之名,推新集之魏,欲以同於唐、虞之盛,忽骨肉之恩,忘籓屏之固,竟不能使四海賓服,混一皇化,而于時群臣莫有諫者,不其過矣哉!孫卿曰:「堯、舜禪讓,是不然矣。天下者,至重也,非至彊莫之能任;至大也,非至辯莫之能分;至眾也,非至明莫之能見。此三至者,非聖人莫之能盡。」由此言之,孫卿、孟軻亦各有所不取焉。陛下受禪,從東府入西宮,兵刃耀天,旌旗翳日。雖應天順人,同符唐、虞,
 然法度損益,則亦不異於昔魏文矣,故宜資三至以彊制之。而今諸王有立國之名,而無襟帶之實。又蜀地有自然之險,是歷世姦雄之所窺覦,逋逃之所聚也,而無親戚子弟之守,此豈深思遠慮,杜漸防萌者乎!



 昔漢文帝據已成之業,六合同風,天下一家。而賈誼上疏陳當時之勢,猶以為譬如抱火厝於積薪之下,而寢其上,火未及然,因謂之安。此言誠存不忘亡,安不忘亂者也。然臣之慺慺,亦竊願陛下居安思危,無曰高高在上,常念臨深之義,不忘履冰之戒。盡除魏世之弊法,綏以新政之大化,使萬邦欣欣,喜戴洪惠,昆蟲草木,咸蒙恩澤。朝廷詠
 康哉之歌,山藪無伐檀之人,此固天下所視望者也。陛下自初踐阼,發無諱之詔,置箴諫之官,赫然寵異諤諤之臣,以明好直言之信,恐陳事者知直言之不用,皆杜口結舌,祥瑞亦曷由來哉!



 臣無陸生之才,不在顧問之地,蓋聞主聖臣直,義在於有犯無隱。臣不惟疏遠,未信而言,敢歷論前代隆名之君及亡敗之主廢興所由,又博陳舉賢之路,廣開養老之制,崇必信之道,又張設議者之難,凡五事以聞。臣之所言,皆直陳古今已行故事,非新聲異端也。辭義實淺,不足採納。然臣私心,誠謂有可發起覺悟遺忘。願陛下察臣愚忠,愍臣狂直,無使天
 下以言者為戒。疾痛增篤,退念桑梓之詩,惟狐死之義,輒取長休,歸近墳墓。顧瞻宮闕,繫情皇極,不勝丹款,遣息穎表言。



 其一曰:臣聞善有章也,著在經典;惡有罰也,戒在刑書。上自遠古,下洎秦、漢,其明王霸主及亡國闇君,故可得而稱;至於忠蹇賢相及佞諂姦臣,亦可得而言。故朝有諤諤盡規之臣,無不昌也;任用阿諛唯唯之士,無不亡也。是有國者皆欲求忠以自輔,舉賢以自佐;而亡國破家者相繼,皆由任失其人。所謂賢者不賢,忠者不忠也。臣謹言前任賢所由興,任不肖所以亡者。堯之末年,四凶在朝而不去,八元在家而不舉,然致天平
 地寧,四門穆穆,其功固在重華之為相。夏癸放於鳴條,商辛梟於牧野,此俱萬乘之主,而國滅身擒,由不能屬任賢相,用婦人之言,荒淫無道,肆志沈宴,作靡靡之樂,長夜之飲,於是登糟丘,臨酒池,觀牛飲,望肉林,龍逢忠而被害,比干諫而剖心,天下之所以歸惡者也。太甲暴虐,顛覆湯之典制,於是伊尹放之桐宮,而能改悔反善,三年而後歸於亳。既已放而復還,殷道微而復興,諸侯咸服,號稱太宗,實賴阿衡之盡忠也。周室既衰,諸侯並爭,天王微弱,政遂陵遲。齊桓公,淫亂之主耳;然所以能九合一匡之功,有尊周之名,誠管夷吾之力。及其死也,
 蟲流出門,豈非任豎貂之過乎!且一桓公之身,得管仲,其功如彼;用豎貂,其亂如此。夫榮辱存亡,實在所任,可不審哉!秦本伯翳之後,微微小邑,至秦仲始大,有車馬禮樂侍御之好焉。自穆公至於始皇,皆能留心待賢,遠求異士,招由余於西戎,致五羖於宛市,取丕豹於晉鄉,迎蹇叔於宗里。由是四方雄俊繼踵而至,故能世為彊國,吞滅諸侯,奄有天下,兼稱皇帝,由謀臣之助也。道化未淳,崩於沙丘。胡亥乘虐,用詐自悮,不能弘濟統緒,克成堂構,而乃殘賊仁義,毒流黔首。故陳勝、吳廣,奮臂大呼,而天下響應。於是趙高逆亂,閻樂承指,二世窮迫,自
 戮望夷。子嬰雖立,去帝為王,孤危無輔,四旬而亡。此由邪臣擅命,指鹿為馬,所以速秦之禍也。秦失其鹿,豪傑競逐,項羽既得而失之,其咎在烹韓生,而范增之謀不用。假令羽既距項伯之邪說,斬沛公於鴻門,都咸陽以號令諸侯,則天下無敵矣。而羽距韓生之忠諫,背范增之深計,自謂霸王之業已定,都彭城,還故鄉,為晝被文繡,此蓋世俗兒女之情耳,而羽榮之。是故五載為漢所擒,至此尚不知覺悟,乃曰「天亡我,非戰之罪」,甚痛矣哉!且夫士之歸仁,猶水之歸下,禽之走曠野,故曰「為川驅魚者獺也,為藪驅雀者鸇也,為湯、武驅人者桀、紂也。」漢
 高祖起於布衣,提三尺之刃而取天下,用六國之資,無唐、虞之禪,豈徒賴良、平之奇謀,盡英雄之智力而已乎,亦由項氏為驅人也。子孫承基二百餘年,逮成帝委政舅家,使權勢外移。安昌侯張禹者,漢之三公,成帝保傅也,帝親幸其家,拜禹床下,深問天災人事。禹當惟大臣之節,為社稷深慮,忠言嘉謀,陳其災患,則王氏不得專權寵,王莽無緣乘勢位,遂託雲龍而登天衢,令漢祚中絕也。禹佞諂不忠,挾懷私計,徒低仰於五侯之間,茍取容媚而已。是以朱雲抗節求尚方斬馬劍,欲以斬禹,以戒其餘,可謂忠矣。而成帝尚復不寤,乃以為居下訕上,
 廷辱保傅,罪死無赦,詔御史將雲下,欲急烹之。雲攀殿折檻,幸賴左將軍辛慶忌叩頭流血,以死爭之。若不然,則雲已摧碎矣。後雖釋檻不修,欲以彰明直臣,誠足以為後世之戒,何益於漢室所由亡也哉!然世之論者以為亂臣賊子無道之甚者莫過於莽,此亦猶紂之不善不如是之甚也。傳稱莽始起外戚,折節力行,以要名譽,宗族稱孝,朋友歸仁。及其輔政成、哀之際,勤勞國家,動見稱述。然於時人士詣闕上書薦莽者不可稱紀,內外群臣莫不歸莽功德。遭遇漢室中微,國嗣三絕,而太后壽考,為之宗主,故莽得遂策命孺子而奪其位也。昔湯、
 武之興,亦逆取而順守之耳。向莽深惟殷、周取守之術,崇道德,務仁義,履信實,去華偽,施惠天下,十有八年,恩足以感百姓,義足以結英雄,人懷其德,豪傑並用,如此,宗廟社稷宜未滅也,光武雖復賢才,大業詎可冀哉!莽即位之後,自謂得天人之助,以為功廣三王,德茂唐、虞,乃自驕矜,奮其威詐,班宣符讖,震暴殘酷,窮凶極惡,人怨神怒,冬雷電以驚其耳目,夏地動以惕其心腹。而莽猶不知覺悟,方復重行不順時之令,竟連伍之刑,佞媚者親幸,忠諫者誅夷。由是天下忿憤,內外俱發,四海分崩,城池不守,身死於匹夫之手,為天下笑,豈不異哉!其
 所由然者,非取之過,而守之非道也。莽既屠肌,六合雲擾,劉聖公已立而不辨,盆子承之而覆敗,公孫述又稱帝於蜀漢。如此數子,固非所謂應天順人者,徒為光武之驅除者耳。夫天下者,蓋亦天下之天下,非一人之天下也。「殷商之旅,其會如林,矢于牧野,維予侯興。」又曰:「侯服于周,天命靡常。」由此言之,主非常人也,有德則天下歸之,無德則天下叛之。故古之明王,其勞心遠慮,常如臨川無津涯。於是法天地,象四時,隆恩德,敬大臣,近忠直,遠佞人。仁孝著乎宮牆,弘化洽乎兆庶;為平直如砥矢,信義感人神。雖有椒房外戚之寵,不受其委曲之言;
 雖有近習愛幸之豎,不聽其姑息之辭。四門穆穆,闢而不闔,待諫者而無忌。恒戰戰慄慄,不忘戒懼,所以欲永終天祿,恐為將來賢聖之驅除也。且臣聞之,懼危者,常安者也;憂亡者,恒存者也。使夫有國之君能安不忘危,則本枝百世,長保榮祚,名位與天地無窮,亦何慮乎為來者之驅除哉!傳有之曰:「狂夫之言,明主察焉。」



 其二曰:士之立業,行非一概。吳起貪官,母死不歸,殺妻求將,不孝之甚。然在魏,使秦人不敢東向;在楚,則三晉不敢南謀。曾參、閔騫,誠孝子也,不能宿夕離其親,豈肯出身致死,涉危險之地哉!今大晉應期運之所授,齊聖美於有
 虞,而吳人不臣,稱帝私附,此亦國之羞也。陛下誠欲致熊羆之士,不二心之臣,使奮威淮浦、震服蠻荊者,故宜疇咨博采,廣開貢士之路,薦巖穴,舉賢才,徵命考試,匪俊莫用。今臺閣選舉,塗塞耳目,九品訪人,唯問中正。故據上品者,非公侯之子孫,則當塗之昆弟也。二者茍然,則蓽門蓬戶之俊,安得不有陸沈者哉!



 其三曰:昔田子方養老馬,而窮士知所歸,況居天下之廣居,立天下之正位,行天下之大道乎!昔明王聖主,無不養老。老人眾多,未必皆賢,不可悉養。故父事三老,所以明孝;宗事五更,所以明敬。孟子曰:「吾老以及人之老,吾幼以及人之
 幼。」今天下雖定,而華山之陽無放馬之群,桃林之下未有休息之牛,故以吳人尚未臣服故也。夫饑者易為食,渴者易為飲,天下元元瞻望新政。願陛下思子方之仁,念犬馬之勞,思帷蓋之報,發仁惠之詔,廣開養老之制。



 其四曰:法令賞罰,莫大乎信。古人有言:「人而無信,不知其可。」況有養人以惠,使人以義,而可以不信行之哉!臣前為西郡太守,被州所下《己未詔書》:「羌胡道遠,其但募取樂行,不樂勿彊。」臣被詔書,輒宣恩廣募,示以賞信,所得人名即條言征西。其晉人自可差簡丁彊,如法調取;至於羌胡,非恩意告諭,則無欲度金城、河西者也。自往
 每興軍渡河,未曾有變,故刺史郭綏勸帥有方,深加獎厲,要許重報。是以所募感恩利賞,遂立績效,功在第一。今州郡督將,並已受封,羌胡健兒,或王或侯,不蒙論敘也。晉文猶不貪原而失信,齊桓不惜地而背盟,況聖主乎!



 其五曰:昔周、漢之興,樹親建德,周因五等之爵,漢有河山之誓。及其衰也,神器奪於重臣,國祚移於他人。故滅周者秦,非姬姓也;代漢者魏,非劉氏也。於今國家大計,使異姓無裂土專封之邑,同姓並據有連城之地,縱復令諸王後世子孫還自相并,蓋亦楚人失繁弱於雲夢,尚未為亡其弓也。其於神器不移他族,則始祖不遷
 之廟,萬年億兆不改其名矣。大晉諸王二十餘人,而公侯伯子男五百餘國,欲言其國皆小乎,則漢祖之起,俱無尺土之地,況有國者哉!將謂大晉世世賢聖,而諸侯之胤常不肖邪,則放勛欽明而有丹朱,瞽瞍頑凶面虞舜。天下有事無不由兵,而無故多樹兵本,廣開亂原,臣故曰五等不便也。臣以為可如前表,諸王宜大其國,增益其兵,悉遣守籓,使形勢足以相接,則陛下可高枕而臥耳。臣以為諸侯伯子男名號皆宜改易之,使封爵之制,祿奉禮秩,並同天下諸侯之例。



 臣聞與覆車同軌者未嘗安也,與死人同病者未嘗生也,與亡國同法者
 未嘗存也。況夫巍巍大晉,方將登太山,禪梁父,刻石書勳,垂示無窮。宜遠鑒往代興廢,深為嚴防,使著事奮筆,必有紀焉。昔伊尹恥其君不為堯、舜,此臣所以私懷慷慨,自忘輕賤者也。



 灼書奏,帝覽而異焉,擢為明威將軍、魏興太守。卒于官。



 閻纘,字續伯,巴西安漢人也。祖圃,為張魯功曹,勸魯降魏,封平樂鄉侯。父璞,嗣爵,仕吳至牂柯太守。纘僑居河南新安,少游英豪,多所交結,博覽墳典,該通物理。父卒,繼母不慈,纘恭事彌謹。而母疾之愈甚,乃誣纘盜父時
 金寶,訟於有司。遂被清議十餘年,纘無怨色,孝謹不怠。母後意解,更移中正,乃得復品。為太傅楊駿舍人,轉安復令。駿之誅也,纘棄官歸,要駿故主簿潘岳、掾崔基等共葬之。基、岳畏罪,推纘為主。墓成,當葬,駿從弟模告武陵王澹,將表殺造意者。眾咸懼,填塚而逃,纘獨以家財成墓,葬駿而去。國子祭酒鄒湛以纘才堪佐著作,薦於秘書監華嶠。嶠曰:「此職閑廩重,貴勢多爭之,不暇求其才。」遂不能用。河間王顒引為西戎校尉司馬,有功,封平樂鄉侯。



 愍懷太子之廢也,纘輿棺詣闕,上書理太子之冤曰:



 伏見赦文及榜下前太子遹手疏,以為驚愕。自古以
 來,臣子悖逆,未有如此之甚也。幸賴天慈,全其首領。臣伏念遹生於聖父而至此者,由於長養深宮,沈淪富貴,受饒先帝,父母驕之。每見選師傅下至群吏,率取膏粱擊鐘鼎食之家,希有寒門儒素如衛綰、周文、石奮、疏廣,洗馬、舍人亦無汲黯、鄭莊之比,遂使不見事父事君之道。臣案古典,太子居以士禮,與國人齒,以此明先王欲令知先賤然後乃貴。自頃東宮亦微太盛,所以致敗也。非但東宮,歷觀諸王師友文學,皆豪族力能得者,率非龔遂、王陽,能以道訓。友無亮直三益之節,官以文學為名,實不讀書,但共鮮衣好馬,縱酒高會,嬉遊博弈,豈有
 切磋,能相長益!臣常恐公族遲陵,以此歎息。今遹可以為戒,恐其被斥,棄逐遠郊,始當悔過,無所復及。



 昔戾太子無狀,稱兵距命,而壺關三老上書,有田千秋之言,猶曰:「子弄父兵,罪應笞耳!」漢武感悟之,築思子之臺。今遹無狀,言語悖逆,受罪之日,不敢失道,猶為輕於戾太子,尚可禁持,重選保傅。如司空張華,道德深遠,乃心忠誠,以為之師。光祿大夫劉寔,寒苦自立,終始不衰,年同呂望,經藉不廢,以為之保。尚書僕射裴頠,明允恭肅,體道居正,以為之友。置游談文學,皆選寒門孤宦以學行自立者,及取服勤更事、涉履艱難、事君事親、名行素聞者,
 使與共處。使嚴御史監護其家,絕貴戚子弟、輕薄賓客。如此,左右前後,莫非正人。師傅文學,可令十日一講,使共論議於前。敕使但道古今孝子慈親,忠臣事君,及思愆改過之義,皆聞善道,庶幾可全。



 昔太甲有罪,放之三年,思庸克復,為殷明王。又魏文帝懼於見廢,夙夜自祗,竟能自全。及至明帝,因母得罪,廢為平原侯,為置家臣庶子,師友文學,皆取正人,共相匡矯。兢兢慎罰,事父以孝,父沒,事母以謹,聞于天下,于今稱之。漢高皇帝數置酒於庭,欲廢太子,後四皓為師,子房為傅,竟復成就。前事不忘,後事之戒。孟軻有云,「孤臣孽子,其操心也危,慮
 患也深」,故多善功。李斯云:「慈母多敗子,嚴家無格虜。」由陛下驕遹使至於此,庶其受罪以來,足自思改。方今天下多虞,四夷未寧,將伺國隙。儲副大事,不宜空虛。宜為大計,小復停留。先加嚴誨。依平原侯故事,若不悛改,棄之未晚也。



 臣素寒門,無力仕宦,不經東宮,情不私遹。念昔楚國處女諫其王曰「有龍無尾」,言年四十,未有太子。臣嘗備近職,雖未得自結天日,情同閽寺,悾悾之誠,皆為國計。臣老母見臣為表,乃為臣卜卦,云「書御即死」。妻子守臣,涕泣見止。臣獨以為頻見拔擢,嘗為近職,此恩難忘,何以報德?唯當陳誠,以死獻忠。輒具棺絮,伏須刑
 誅。



 書御不省。



 及張華遇害,賈謐被誅,朝野震悚,纘獨撫華尸慟哭曰:「早語君遜位而不肯,今果不免,命也夫!」過叱賈謐尸曰:「小兒亂國之由,誅其晚矣!」



 皇太孫立,纘復上疏曰:



 臣前上書訟太子之枉,不見省覽。昔壺關三老陳衛太子之冤,而漢武築思子之臺。高廟令田千秋上書,不敢正言,託以鬼神之教,而孝武大感,月中三遷,位至丞相,乘車入殿,號曰車氏。恨臣精誠微薄,不能有感,竟使太子流離,沒命許昌。向令陛下即納臣言,不致此禍。天贊聖意,三公獻謀,庶人賜死,罪人斯得,太子以明,臣恨其晚,無所復及。詔書慈悼,迎喪反葬,復其禮秩,誠
 副眾望,不意呂、霍之變復生於今日!伏見詔書建立太孫,斯誠陛下上順先典以安社稷,中慰慈悼冤魂之痛,下令萬國心有所繫。追惟庶人,所為無狀,幾傾宗廟,賴相國、太宰至忠憤發,潛謀俱斷,奉贊聖意,以成神武。雖周誅二叔,漢掃諸呂,未足以喻。臣願陛下因此大更釐改,以為永制。禮置太子,居以士禮,與國人齒,為置官屬,皆如朋友,不為純臣。既使上厭至望,以崇孝道,又令不相嚴憚,易相規正。



 昔漢武既信奸讒,危害太子,復用望氣之言,欲盡誅詔獄中囚。邴吉以皇孫在焉,閉門距命,後遂擁護皇孫,督罰乳母,卒至成人,立為孝宣皇帝。茍
 志於忠,無往不可。歷觀古人雖不避死,亦由世教寬以成節。吉雖距詔書,事在於忠,故宥而不責。自晉興已來,用法太嚴,遲速之間,輒加誅斬。一身伏法,猶可彊為,今世之誅,動輒滅門。昔呂后臨朝,肆意無道。周昌相趙,三召其王而昌不遣,先徵昌入,乃後召王。此由漢制本寬,得使為快。假令如今,呂后必謂昌已反,夷其三族,則誰敢復為殺身成義者哉!此法宜改,可使經遠。又漢初廢趙王張敖,其臣貫高謀弒高祖,高祖不誅,以明臣道。田叔、孟舒十人為奴,髡鉗隨王,隱親侍養,故令平安。向使晉法得容為義,東宮之臣得如周昌,固護太子得如邴
 吉,距詔不坐,伏死諫爭,則聖意必變,太子以安。如田叔、孟舒侍從不罪者,則隱親左右,姦凶毒藥無緣得設,太子不夭也。



 臣每責東宮臣故無侍從者,後聞頗有於道路望車拜辭,而有司收付洛陽獄,奏科其罪。然臣故莫從,良有以也。又本置三率,盛其兵馬,所以宿衛防虞。而使者卒至,莫有警嚴覆請審者,此由恐畏滅族。今皇孫沖幼,去事多故。若有不虞,彊臣專制,姦邪矯詐,雖有相國保訓東宮,擁佑之恩同於邴吉,適可使玉體安全,宜開來防,可著于令:自今已後,諸有廢興倉卒,群臣皆得輒嚴,須錄詣殿前,面受口詔,然後為信,得同周昌不遣
 王節,下聽臣子隱親,得如田叔、孟舒,不加罪責,則永固儲副,以後安嗣之遠慮也。來事難知,往事可改。臣前每見詹事裴權用心懇惻,舍人秦戢數上疏啟諫;而爰倩贈以九列,權有忠意,獨不蒙賞。謂宜依倩為比,以寵其魂。推尋表疏,如秦戢輩及司隸所奏,諸敢拜辭於道路者,明詔稱揚,使微異於眾,以勸為善,以獎將來也。



 纘又陳:



 今相國雖已保傅東宮,保其安危。至於旦夕訓誨,輔導出入,動靜劬勞,宜選寒苦之士,忠貞清正,老而不衰,如城門校尉梁柳、白衣南安朱沖比者,以為師傅。其侍臣以下文武將吏,且勿復取盛戚豪門子弟,若吳太妃
 家室及賈、郭之黨。如此之輩,生而富溢,無念修己,率多輕薄浮華,相驅放縱,皆非所補益於吾少主者也。皆可擇寒門篤行、學問素士、更履險易、節義足稱者,以備群臣,可輕其禮儀,使與古同,於相切磋為益。



 昔魏文帝之在東宮,徐幹、劉楨為友,文學相接之道並如氣類。吳太子登,顧譚為友,諸葛恪為賓,臥同床帳,行則參乘,交如布衣,相呼以字,此則近代之明比也。天子之子不患不富貴,不患人不敬畏,患於驕盈,不聞其過,不知稼穡之艱難耳。至於甚者,乃不知名六畜,可不勉哉!昔周公親撻伯禽,曹參笞窋二百,聖考慈父皆不傷恩。今不忍小
 相維持,令至闕失頓相罪責,不亦誤哉!



 在禮太子朝夕視膳,昏定晨省,跪問安否,於情得盡。五日一朝,於敬既簡,於恩亦疏,易致構間。故曰「一朝不朝,其間容刀」。五日之制,起漢高祖,身為天子,父為庶人,萬機事多,故闕私敬耳。今主上臨朝,太子無事,專主孝養,宜改此俗。《文王世子》篇曰:「王季一飯亦一飯,再飯亦再飯。」安有逸豫五日一覲哉!



 纘又陳:



 今迎太子神柩,孤魂獨行,太孫幼沖,不可涉道。謂可遣妃奉迎遠路,令其父衍隨行衛護。皇太子初見誣陷,臣家門無祐,三世假親,具嘗辛苦,以家觀國,固知太子有變。臣故求副監國,欲依邴吉故事,距
 違來使,供養擁護,身親飲食醫藥,冀足救危。主者以臣名資輕淺,不肯見與。世人見笑,謂為此職進退難居,有必死憂。臣獨以為茍全儲君,賈氏所誅,甘心所願。今監國御史直副皆當三族,侍衛無狀,實自宜然。臣謂其小人,不足具責。故孔子曰:「可以託六尺之孤,臨大節而不可奪。」是以聖王慎選。故河南尹向雄,昔能犯難葬故將鐘會,文帝嘉之,始拔顯用,至於先帝,以為右率。如間之事,若得向雄之比,則豈可觸哉!此二使者,但為愚怯,亦非與謀,但可誅身,自全三族。如郭俶、郭斌,則於刑為當。



 又東宮亦宜妙選忠直亮正,如向雄比。陛下千秋萬歲
 之後,太孫幼沖,選置兵衛,宜得柱石之士如周昌者。世俗淺薄,士無廉節,賈謐小兒,恃寵恣睢,而淺中弱植之徒,更相翕習,故世號魯公二十四友。又謐前見臣表理太子,曰:「閻兒作此為健,然觀其意,欲與諸司馬家同。」皆為臣寒心。伏見詔書,稱明滿奮、樂廣。侍郎賈胤,與謐親理,而亦疏遠,往免父喪之後,停家五年,雖為小屈,有識貴之。潘岳、繆徵等皆謐父黨,共相沈浮,人士羞之,聞其晏然,莫不為怪。今詔書暴揚其罪,並皆遣出,百姓咸云清當,臣獨謂非。但岳征二十四人,宜皆齊黜,以肅風教。



 朝廷善其忠烈,擢為漢中太守。趙王倫死,既葬,纘以車
 轢其冢。時張華兄子景後徙漢中,纘又表宜還。纘不護細行,而慷慨好大節。卒於官,時年五十九。纘五子,皆開朗有才力。



 長子亨為遼西太守,屬王浚自用其人,亨不得之官。依青州刺史茍晞,刑政苛虐,亨數切諫,為晞所害。



 史臣曰:愍懷之廢也,天下稱其冤。然皆懼亂政之參夷,懾淫嬖之凶忍,遂使謀臣懷忠而結舌,義士蓄憤而吞聲。閻續伯官既微於侍郎,位不登於執戟,輕生重義,視死如歸,伏奏而待嚴誅,輿棺以趨鼎鑊,察言觀行,豈非忠直壯乎!顧視晉朝公卿,曾不得與其徒隸齒也。茂伯
 篤終,哭王經以全節。休然追遠,理鄧艾以成名。故得義感明時,仁流枯骨。雖硃勃追論新息,欒布奏事彭王,弗之尚也。



 贊曰:感義收會,篤終理艾。道既相侔,名亦俱泰。續伯區區,輿櫬陳謩。偪茲淫嬖,弗遂良圖。啜其泣矣,何嗟及乎!



\end{pinyinscope}