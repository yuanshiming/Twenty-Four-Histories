\article{列傳第十七}

\begin{pinyinscope}
傅玄
 \gezhu{
  子咸咸子敷、咸從父弟祗}



 傅玄,字休奕,北地泥陽人也。祖燮,漢漢陽太守。父干,魏扶風太守。玄少孤貧,博學善屬文,解鐘律。性剛勁亮直,不能容人之短。郡上計吏再舉孝廉,太尉辟,皆不就。州舉秀才,除郎中,與東海繆施俱以時譽選入著作,撰集魏書。後參安東、衛軍軍事,轉溫令,再遷弘農太守,領典農校尉。所居稱職,數上書陳便宜,多所匡正。五等建,封
 鶉觚男。武帝為晉王,以玄為散騎常侍。及受禪,進爵為子,加附馬都尉。



 帝初即位,廣納直言,開不諱之路,玄及散騎常侍皇甫陶共掌諫職。玄上疏曰:「臣聞先王之臨天下也,明其大教,長其義節。道化隆於上,清議行於下,上下相奉,人懷義心。亡秦蕩滅先王之制,以法術相御,而義心亡矣。近者魏武好法術,而天下貴刑名;魏文慕通達,而天下賤守節。其後綱維不攝,而虛無放誕之論盈於朝野,使天下無復清議,而亡秦之病復發於今。陛下聖德,龍興受禪,弘堯、舜之化,開正直之路,體夏禹之至儉,綜殷周之典文,臣詠歎而已,將又奚言!惟未舉清
 遠有禮之臣,以敦風節;未退虛鄙,以懲不恪,臣是以猶敢有言。」詔報曰:「舉清遠有禮之臣者,此尤今之要也。」乃使玄草詔進之。玄復上疏曰:



 臣聞舜舉五臣,無為而化,用人得其要也。天下群司猥多,不可不審得其人也。不得其人,一日則損不貲,況積日乎!典謨曰「無曠庶官」,言職之不可久廢也。諸有疾病滿百日不差,宜令去職,優其禮秩而寵存之,既差而後更用。臣不廢職於朝,國無曠官之累,此王政之急也。



 臣聞先王分士農工商以經國制事,各一其業而殊其務。自士已上子弟,為之立太學以教之,選明師以訓之,各隨其才優劣而授用之。農
 以豐其食,工以足其器,商賈以通其貨。故雖天下之大,兆庶之眾,無有一人游手。分數之法,周備如此。漢、魏不定其分,百官子弟不修經藝而務交遊,未知蒞事而坐享天祿;農工之業多廢,或逐淫利而離其事;徒繫名於太學,然不聞先王之風。今聖明之政資始,而漢、魏之失未改,散官眾而學校未設,游手多而親農者少,工器不盡其宜。臣以為亟定其制,通計天下若干人為士,足以副在官之吏;若干人為農,三年足有一年之儲;若干人為工,足其器用;若干人為商賈,足以通貨而已。尊儒尚學,貴農賤商,此皆事業之要務也。



 前皇甫陶上事,欲
 令賜拜散官皆課使親耕,天下享足食之利。禹、稷躬稼,祚流後世,是以《明堂》、《月令》著帝藉之制。伊尹古之名臣,耕於有莘;晏嬰齊之大夫,避莊公之難,亦耕於海濱。昔者聖帝明王,賢佐俊士,皆嘗從事於農矣。王人賜官,冗散無事者,不督使學,則當使耕,無緣放之使坐食百姓也。今文武之官既眾,而拜賜不在職者又多,加以服役為兵,不得耕稼,當農者之半,南面食祿者參倍於前。使冗散之官農,而收其租稅,家得其實,而天下之穀可以無乏矣。夫家足食,為子則孝,為父則慈,為兄則友,為弟則悌。天下足食,則仁義之教可不令而行也。為政之要,
 計人而置官,分人而授事,士農工商之分不可斯須廢也。若未能精其防制,計天下文武之官足為副貳者使學,其餘皆歸之於農。若百工商賈有長者,亦皆歸之於農。務農若此,何有不贍乎!《虞書》曰:「三載考績,三考黜陟幽明。」是為九年之後乃有遷敘也。故居官久,則念立慎終之化,居不見久,則競為一切之政。六年之限,日月淺近,不周黜陟。陶之所上,義合古制。



 夫儒學者,王教之首也。尊其道,貴其業,重其選,猶恐化之不崇;忽而不以為急,臣懼日有陵遲而不覺也。仲尼有言:「人能弘道,非道弘人。」然則尊其道者,非惟尊其書而已,尊其人之謂也。
 貴其業者,不妄教非其人也。重其選者,不妄用非其人也。若此,而學校之綱舉矣。



 書奏,帝下詔曰:「二常侍懇懇於所論,可謂乃心欲佐益時事者也。而主者率以常制裁之,豈得不使發憤耶!二常侍所論,或舉其大較而未備其條目,亦可便令作之,然後主者八坐廣共研精。凡關言於人主,人臣之所至難。而人主若不能虛心聽納,自古忠臣直士之所慷慨,至使杜口結舌。每念於此,未嘗不歎息也。故前詔敢有直言,勿有所距,庶幾得以發懞補過,獲保高位。茍言有偏善,情在忠益,雖文辭有謬誤,言語有失得,皆當曠然恕之。古人猶不拒誹謗,況皆
 善意在可採錄乎!近者孔晁、綦毋皆案以輕慢之罪,所以皆原,欲使四海知區區之朝無諱言之忌也。」俄遷侍中。



 初,玄進皇甫陶,及入而抵,玄以事與陶爭,言喧嘩,為有司所奏,二人竟坐免官。泰始四年,以為御史中丞。時頗有水旱之災,玄復上疏曰:



 臣聞聖帝明王受命,天時未必無災,是以堯有九年之水,湯有七年之旱,惟能濟之以人事耳。故洪水滔天而免沈溺,野無生草而不困匱。伏惟陛下聖德欽明,時小水旱,人未大飢,下祗畏之詔,求極意之言,同禹、湯之罪己,侔周文之夕惕。臣伏懽喜,上便宜五事:



 其一曰,耕夫務多種而耕不熟,徒
 喪功力而無收。又舊兵持官牛者,官得六分,士得四分;自持私牛者,與官中分,施行來久,眾心安之。今一朝減持官牛者,官得八分,士得二分;持私牛及無牛者,官得七分,士得三分,人失其所,必不懽樂。臣愚以為宜佃兵持官牛者與四分,持私牛與官中分,則天下兵作懽然悅樂,愛惜成穀,無有損棄之憂。



 其二曰,以二千石雖奉務農之詔,猶不勤心以盡地利。昔漢氏以墾田不實,徵殺二千石以十數。臣愚以為宜申漢氏舊典,以警戒天下郡縣,皆以死刑督之。



 其三曰,以魏初未留意於水事,先帝統百揆,分河堤為四部,并本凡五謁者,以水功至
 大,與農事並興,非一人所周故也。今謁者一人之力,行天下諸水,無時得遍。伏見河堤謁者車誼不知水勢,轉為他職,更選知水者代之。可分為五部,使各精其方宜。



 其四曰,古以步百為畝,今以二百四十步為一畝,所覺過倍。近魏初課田,不務多其頃畝,但務修其功力,故白田收至十餘斛,水田收數十斛。自頃以來,日增田頃畝之課,而田兵益甚,功不能修理,至畝數斛已還,或不足以償種。非與曩時異天地,橫遇災害也,其病正在於務多頃畝而功不修耳。竊見河堤謁者石恢甚精練水事及田事,知其利害,乞中書召恢,委曲問其得失,必有所
 補益。



 其五曰,臣以為胡夷獸心,不與華同,鮮卑最甚。本鄧艾茍欲取一時之利,不慮後患,使鮮卑數萬散居人間,此必為害之勢也。秦州刺史胡烈素有恩信於西方,今烈往,諸胡雖已無惡,必且消弭,然獸心難保,不必其可久安也。若後有動釁,烈計能制之。惟恐胡虜適困於討擊,便能東入安定,西赴武威,外名為降,可動復動。此二郡非烈所制,則惡胡東西有窟穴浮游之地,故復為患,無以禁之也。宜更置一郡於高平川,因安定西州都尉募樂徙民,重其復除以充之,以通北道,漸以實邊。詳議此二郡及新置郡,皆使并屬秦州,令烈得專御邊之
 宜。



 詔曰:「得所陳便宜,言農事得失及水官興廢,又安邊御胡政事寬猛之宜,申省周備,一二具之,此誠為國大本,當今急務也。如所論皆善,深知乃心,廣思諸宜,動靜以聞也。」



 五年,遷太僕。時比年不登,羌胡擾邊,詔公卿會議。玄應對所問,陳事切直,雖不盡施行,而常見優容。轉司隸校尉。



 獻皇后崩於弘訓宮,設喪位。舊制,司隸於端門外坐,在諸卿上,絕席。其入殿,按本品秩在諸卿下,以次坐,不絕席。而謁者以弘訓宮為殿內,制玄位在卿下。玄恚怒,厲聲色而責謁者。謁者妄稱尚書所處,玄對百僚而罵尚書以下。御史中丞庾純奏玄不敬,玄又自表
 不以實,坐免官。然玄天性峻急,不能有所容;每有奏劾,或值日暮,捧白簡,整簪帶,竦踴不寐,坐而待旦。於是貴游懾伏,臺閣生風。尋卒於家,時年六十二,謚曰剛。



 玄少時避難於河內,專心誦學,後雖顯貴,而著述不廢。撰論經國九流及三史故事,評斷得失,各為區例,名為《傅子》,為內、外、中篇,凡有四部、六錄,合百四十首,數十萬言,并文集百餘卷行於世。玄初作內篇成,子咸以示司空王沈。沈與玄書曰:「省足下所著書,言富理濟,經綸政體,存重儒教,足以塞楊、墨之流遁,齊孫、孟於往代。每開卷,未嘗不歎息也。『不見賈生,自以過之,乃今不及』,信矣!」



 其後
 追封清泉侯。子咸嗣。



 咸字長虞,剛簡有大節。風格峻整,識性明悟,疾惡如仇,推賢樂善,常慕季文子、仲山甫之志。好屬文論,雖綺麗不足,而言成規鑒。潁川庾純常歎曰:「長虞之文近乎詩人之作矣!」



 咸寧初,襲父爵,拜太子洗馬,累遷尚書右丞。出為冀州刺史,繼母杜氏不肯隨咸之官,自表解職。三旬之間,遷司徒左長史。時帝留心政事,詔訪朝臣政之損益。咸上言曰:「陛下處至尊之位,而修布衣之事,親覽萬機,勞心日昃。在昔帝王,躬自菲薄,以利天下,未有踰陛下也。然泰始開元以暨于今,十有五年矣。而軍國未
 豐,百姓不贍,一歲不登便有菜色者,誠由官眾事殷,復除猥濫,蠶食者多而親農者少也。臣以頑疏,謬忝近職,每見聖詔以百姓饑饉為慮,無能云補,伏用慚恧,敢不自竭,以對天問。舊都督有四,今并監軍,乃盈於十。夏禹敷土,分為九州,今之刺史,幾向一倍。戶口比漢十分之一,而置郡縣更多。空校牙門,無益宿衛,而虛立軍府,動有百數。五等諸侯,復坐置官屬。諸所寵給,皆生於百姓。一夫不農,有受其飢,今之不農,不可勝計。縱使五稼普收,僅足相接;暫有災患,便不繼贍。以為當今之急,先并官省事,靜事息役,上下用心,惟農是務也。」



 咸在位多所
 執正。豫州大中正夏侯駿上言,魯國小中正、司空司馬孔毓,四移病所,不能接賓,求以尚書郎曹馥代毓,旬日復上毓為中正。司徒三卻,駿故據正。咸以駿與奪惟意,乃奏免駿大中正。司徒魏舒,駿之姻屬,屢卻不署,咸據正甚苦。舒終不從,咸遂獨上。舒奏咸激訕不直,詔轉咸為車騎司馬。



 咸以世俗奢侈,又上書曰:「臣以為穀帛難生,而用之不節,無緣不匱。故先王之化天下,食肉衣帛,皆有其制。竊謂奢侈之費,甚於天災。古者堯有茅茨,今之百姓競豐其屋。古者臣無玉食,今之賈豎皆厭粱肉。古者后妃乃有殊飾,今之婢妾被服綾羅。古者大夫乃
 不徒行,今之賤隸乘輕驅肥。古者人稠地狹而有儲蓄,由於節也;今者土廣人稀而患不足,由於奢也。欲時之儉,當詰其奢;奢不見詰,轉相高尚。昔毛玠為吏部尚書,時無敢好衣美食者。魏武帝歎曰:『孤之法不如毛尚書。』令使諸部用心,各如毛玠,風俗之移,在不難矣。」又議移縣獄於郡及二社應立,朝廷從之。遷尚書左丞。



 惠帝即位,楊駿輔政。咸言於駿曰:「事與世變,禮隨時宜,諒闇之不行尚矣。由世道彌薄,權不可假,故雖斬焉在疚,而躬覽萬機也。逮至漢文,以天下體大,服重難久,遂制既葬而除。世祖武皇帝雖大孝蒸蒸,亦從時釋服,制心喪三
 年,至於萬機之事,則有不遑。今聖上欲委政於公,諒闇自居,此雖謙讓之心,而天下未以為善。天下未以為善者,以億兆顒顒,戴仰宸極,聽於冢宰,懼天光有蔽。人心既已若此,而明公處之固未為易也。竊謂山陵之事既畢,明公當思隆替之宜。周公聖人,猶不免謗。以此推之,周公之任既未易而處,況聖上春秋非成王之年乎!得意忘言,言未易盡。茍明公有以察其悾款,言豈在多。」時司隸荀愷從兄喪,自表赴哀,詔聽之而未下,愷乃造駿。咸因奏曰:「死喪之戚,兄弟孔懷。同堂亡隕,方在信宿,聖恩矜憫,聽使臨喪。詔未下而便以行造,急諂媚之敬,無
 友于之情。宜加顯貶,以隆風教。」帝以駿管朝政,有詔不問,駿甚憚之。咸復與駿箋諷切之,駿意稍折,漸以不平。由是欲出為京兆、弘農太守,駿甥李斌說駿,不宜斥出正人,乃止。駿弟濟素與咸善,與咸書曰:「江海之流混混,故能成其深廣也。天下大器,非可稍了,而相觀每事欲了。生子癡,了官事,官事未易了也。了事正作癡,復為快耳!左丞總司天臺,維正八坐,此未易居。以君盡性而處未易居之任,益不易也。想慮破頭,故具有白。」咸答曰:「衛公云酒色之殺人,此甚於作直。坐酒色死,人不為悔。逆畏以直致禍,此由心不直正,欲以茍且為明哲耳!自古
 以直致禍者,當自矯枉過直,或不忠允,欲以亢厲為聲,故致忿耳。安有空空為忠益,而當見疾乎!」居無何,駿誅。咸轉為太子中庶子,遷御史中丞。



 時太宰、汝南王亮輔政,咸致書曰:「咸以為太甲、成王年在蒙幼,故有伊、周之事。聖人且猶不免疑,況臣既不聖,王非孺子,而可以行伊、周之事乎!上在諒暗,聽於冢宰,而楊駿無狀,便作伊、周,自為居天下之安,所以至死。其罪既不可勝,亦是殿下所見。駿之見討,發自天聰,孟觀、李肇與知密旨耳。至於論功,當歸美於上。觀等已數千戶縣侯,聖上以駿死莫不欣悅,故論功寧厚,以敘其歡心。此群下所宜以實
 裁量,而遂扇動,東安封王,孟、李郡公,餘侯伯子男,既妄有加,復又三等超遷。此之熏赫,震動天地,自古以來,封賞未有若此者也。無功而厚賞,莫不樂國有禍,禍起當復有大功也。人而樂禍,其可極乎!作此者,皆由東安公。謂殿下至止,當有以正之。正之以道,眾亦何所怒乎!眾之所怒,在於不平耳。而今皆更倍論,莫不失望。咸之愚冗,不惟失望而已,竊以為憂。又討駿之時,殿下在外,實所不綜。今欲委重,故令殿下論功。論功之事,實未易可處,莫若坐觀得失,有居正之事宜也。」



 咸復以亮輔政專權,又諫曰:「楊駿有震主之威,委任親戚,此天下所以喧
 譁。今之處重,宜反此失。謂宜靜默頤神,有大得失,乃維持之;自非大事,一皆抑遣。比四造詣,及經過尊門,冠蓋車馬,填塞街衢,此之翕習,既宜弭息。又夏侯長容奉使為先帝請命,祈禱無感,先帝崩背,宜自咎責,而自求請命之勞,而公以為少府。私竊之論,云長容則公之姻,故至於此。一犬吠形,群犬吠聲,懼於群吠,遂至叵聽也。咸之為人,不能面從而有後言。嘗觸楊駿,幾為身禍;況於殿下,而當有惜!往從駕,殿下見語:『卿不識韓非逆鱗之言耶,而欻摩天子逆鱗!』自知所陳,誠觸猛獸之鬚耳。所以敢言,庶殿下當識其不勝區區。前摩天子逆鱗,
 欲以盡忠;今觸猛獸之鬚,非欲為惡,必將以此見恕。」亮不納。長容者,夏侯駿也。



 會丙寅,詔群僚舉郡縣之職以補內官。咸復上書曰:「臣咸以為夫興化之要,在於官人。才非一流,職有不同。譬諸林木,洪纖枉直,各有攸施。故明揚逮于仄陋,疇咨無拘內外。內外之任,出處隨宜,中間選用,惟內是隆。外舉既頹,復多節目,競內薄外,遂成風俗。此弊誠宜亟革之,當內外通塞無所偏耳。既使通塞無偏,若選用不平,有以深責,責之茍深,無憂不平也。且膠柱不可以調瑟,況乎官人而可以限乎!伏思所限者,以防選用不能出人。不能出人,當隨事而制,無須限法。法
 之有限,其於致遠,無乃泥乎!或謂不制其法,以何為貴?臣聞刑懲小人,義責君子,君子之責,在心不在限也。正始中,任何晏以選舉,內外之眾職各得其才,粲然之美於斯可觀。如此,非徒御之以限,法之所致,乃委任之由也。委任之懼,甚於限法。是法之失,非己之尤,尤不在己,責之無懼,所謂『齊之以刑,人免而無恥』者也。茍委任之,一則慮罪之及,二則懼致怨謗。己快則朝野稱詠,不善則眾惡見歸,此之戰戰,孰與倚限法以茍免乎!」



 咸再為本郡中正,遭繼母憂去官。頃之,起以議郎,長兼司隸校尉。咸前後固辭,不聽,敕使者就拜,咸復送還印綬。公車
 不通,催使攝職。咸以身無兄弟,喪祭無主,重自陳乞,乃使於官舍設靈坐。咸又上表曰:「臣既駑弱,不勝重任。加在哀疚,假息日闋,陛下過意,授非所堪。披露丹款,歸窮上聞,謬詔既往,終然無改。臣雖不能滅身以全禮教,義無靦然,虛忝隆寵。前受嚴詔,視事之日,私心自誓,隕越為報。以貨賂流行,所宜深絕,切敕都官,以此為先。而經彌日月,未有所得。斯由陛下有以獎厲,慮於愚戇,將必死系,故自掩檢以避其鋒耳。在職有日,既無赫然之舉,又不應弦垂翅,人誰復憚?故光祿大夫劉毅為司隸,聲震內外,遠近清肅。非徒毅有王臣匪躬之節,亦由所奏
 見從,威風得伸也。」詔曰:「但當思必應繩中理,威風日伸,何獨劉毅!」



 時朝廷寬弛,豪右放恣,交私請託,朝野溷淆。咸奏免河南尹澹、左將軍倩、廷尉高光、兼河南尹何攀等,京都肅然,貴戚懾伏。咸以「聖人久於其道,天下化成。是以唐、虞三載考績,九年黜陟。其在《周禮》,三年大比。孔子亦云,『三年有成』。而中間以來,長吏到官,未幾便遷,百姓困於無定,吏卒疲於送迎」。時僕射王戎兼吏部,咸奏:「戎備位台輔,兼掌選舉,不能謐靜風俗,以凝庶績,至令人心傾動,開張浮競。中郎李重、李義不相匡正。請免戎等官。」詔曰:「政道之本,誠宜久於其職,咸奏是也。戎職在
 論道,吾所崇委,其解禁止。」御史中丞解結以咸劾戎為違典制,越局侵官,乾非其分,奏免咸官。詔亦不許。



 咸上事以為「按令,御史中丞督司百僚。皇太子以下,其在行馬內,有違法憲者皆彈糾之。雖在行馬外,而監司不糾,亦得奏之。如令之文,行馬之內有違法憲,謂禁防之事耳。宮內禁防,外司不得而行,故專施中丞。今道路橋梁不修,鬥訟屠沽不絕,如此之比,中丞推責州坐,即今所謂行馬內語施於禁防。既云中丞督司百僚矣,何復說行馬之內乎!既云百僚,而不得復說行馬之內者,內外眾官謂之百僚,則通內外矣。司隸所以不復說行馬內外
 者,禁防之事已於中丞說之故也。中丞、司隸俱糾皇太子以下,則共對司內外矣,不為中丞專司內百僚,司隸專司外百僚。自有中丞、司隸以來,更互奏內外眾官,惟所糾得無內外之限也。而結一旦橫挫臣,臣前所以不羅縷者,冀因結奏得從私願也。今既所願不從,而敕云但為過耳,非所不及也,以此見原。臣忝司直之任,宜當正己率人,若其有過,不敢受原,是以申陳其愚。司隸與中丞俱共糾皇太子以下,則從皇太子以下無所不糾也。得糾皇太子而不得糾尚書,臣之闇塞既所未譬。皇太子為在行馬之內邪,皇太子在行馬之內而得糾之,
 尚書在行馬之內而不得糾,無有此理。此理灼然,而結以此挫臣。臣可無恨耳,其於觀聽,無乃有怪邪!臣識石公前在殿上脫衣,為司隸荀愷所奏,先帝不以為非,於時莫謂侵官;今臣裁糾尚書,而當有罪乎?」咸累自上稱引故事,條理灼然,朝廷無以易之。



 吳郡顧榮常與親故書曰:「傅長虞為司隸,勁直忠果,劾按驚人。雖非周才,偏亮可貴也。」元康四年卒官,時年五十六,詔贈司隸校尉,朝服一具、衣一襲、錢二十萬,謚曰貞。有三子:敷、晞、纂。長子敷嗣。



 敷字穎根,清靜有道,素解屬文。除太子舍人,轉尚書郎、
 太傅參軍,皆不起。永嘉之亂,避地會稽,元帝引為鎮東從事中郎。素有贏疾,頻見敦喻,辭不獲免,輿病到職。數月卒,時年四十六。晞亦有才思,為上虞令,甚有政績,卒於司徒西曹屬。



 祗字子莊。父嘏,魏太常。祗性至孝,早知名,以才識明練稱。武帝始建東宮,起家太子舍人,累遷散騎黃門郎,賜爵關內侯,食邑三百戶。母憂去職。及葬母,詔給太常五等吉凶導從。其後諸卿夫人葬給導從,自此始也。服終,為滎陽太守。自魏黃初大水之後,河濟汎溢,鄧艾嘗著《濟河論》,開石門而通之,至是復浸壞。祗乃造沈萊堰,至
 今兗、豫無水患,百姓為立碑頌焉。尋表兼廷尉,遷常侍、左軍將軍。



 及帝崩,梓宮在殯,而太傅楊駿輔政,欲悅眾心,議普進封爵。祗與駿書曰:「未有帝王始崩,臣下論功者也。」駿不從。入為侍中。時將誅駿,而駿不之知。祗侍駿坐,而雲龍門閉,內外不通。祗請與尚書武茂聽國家消息,揖而下階。茂猶坐,祗顧曰:「君非天子臣邪!今內外隔絕,不知國家所在,何得安坐!」茂乃驚起。駿既伏誅,裴楷息瓚,駿之婿也,為亂兵所害。尚書左僕射荀愷與楷不平,因奏楷是駿親,收付廷尉。祗證楷無罪,有詔赦之。時又收駿官屬,祗復啟曰:「昔魯芝為曹爽司馬,斬關出赴
 爽,宣帝義之,尚遷青州刺史。駿之僚佐不可加罰。」詔又赦之。祗多所維正皆如此。



 除河南尹,未拜,遷司隸校尉。以討楊駿勳,當封郡公八千戶,固讓,減半,降封靈川縣公,千八百戶,餘二千二百戶封少子暢為武鄉亭侯。又以本封賜兄子雋為東明亭侯。



 楚王瑋之矯詔也,祗以聞奏稽留,免官。期年,遷光祿勳,復以公事免。氐人齊萬年舉兵反,以祗為行安西軍司,加常侍,率安西將軍夏侯駿討平之。遷衛尉,以風疾遜位,就拜常侍,食卿祿秩,賜錢及床帳等。尋加光祿大夫,門施行馬。及趙王倫輔政,以為中書監,常侍如故,以鎮眾心。祗辭之以疾,倫遣
 御史輿祗就職。王戎、陳準等相與言曰:「傅公在事,吾屬無憂矣。」其為物所倚信如此。



 倫篡,又為右光祿、開府,加侍中。惠帝還宮,祗以經受偽職請退,不許。初,倫之篡也,孫秀與義陽王威等十餘人預撰儀式禪文。及倫敗,齊王冏收侍中劉逵、常侍騶捷、杜育、黃門郎陸機、右丞周導、王尊等付廷尉。以禪文出中書,復議處祗罪,會赦得原。後以禪文草本非祗所撰,於是詔復光祿大夫。子宣,尚弘農公主。



 尋遷太子少傅,上章遜位還第。及成都王穎為太傅,復以祗為少傅,加侍中。懷帝即位,遷光祿大夫、侍中,未拜,加右僕射、中書監。時太傅東海王越輔政,
 祗既居端右,每宣君臣謙光之道,由此上下雍穆。祗明達國體,朝廷制度多所經綜。歷左光祿、開府,行太子太傅,侍中如故。疾篤遜位,不許。遷司徒,以足疾,詔版輿上殿,不拜。



 大將軍茍晞表請遷都,使祗出詣河陰,修理舟楫,為水行之備。及洛陽陷沒,遂共建行臺,推祗為盟主,以司徒、持節、大都督諸軍事傳檄四方。遣子宣將公主與尚書令和郁赴告方伯徵義兵,祗自屯盟津小城,宣弟暢行河陰令,以待宣。祗以暴疾薨,時年六十九。祗自以義誠不終,力疾手筆敕厲其二子宣、暢,辭旨深切,覽者莫不感激慷慨。祗著文章駮論十餘萬言。



 宣字世弘。年六歲喪繼母,哭泣如成人,中表異之。及長,好學,趙王倫以為相國掾、尚書郎、太子中舍人,遷司徒西曹掾。去職,累遷為祕書丞、驃騎從事中郎。惠帝至自長安,以宣為左丞,不就,遷黃門郎。懷帝即位,轉吏部郎,又為御史中丞。卒年四十九,無子,以暢子沖為嗣。



 暢字世道。年五歲,父友見而戲之,解暢衣,取其金環與侍者,暢不之惜,以此賞之。年未弱冠,甚有重名。以選入侍講東宮,為祕書丞。尋沒於石勒,勒以為大將軍右司馬。諳識朝儀,恒居機密,勒甚重之。作《晉諸公敘贊》二十二卷,又為《公卿故事》九卷。咸和五年卒。子詠,過江為交
 州刺史、太子右率。



 史臣曰:武帝覽觀四方,平章百姓,永言啟沃,任切爭臣。傅玄體彊直之姿,懷匪躬之操,抗辭正色,補闕弼違,諤諤當朝,不忝其職者矣。及乎位居三獨,彈擊是司,遂能使臺閣生風,貴戚斂手。雖前代鮑、葛,何以加之!然而惟此褊心,乏弘雅之度,驟聞競爽,為物議所譏,惜哉!古人取戒於韋弦,良有以也。長虞風格凝峻,弗墜家聲。及其納諫汝南,獻書臨晉,居諒直之地,有先見之明矣。傅祗名父之子,早樹風猷,崎嶇危亂之朝,匡救君臣之際,卒能保全祿位,可謂有道存焉。



 贊曰:鶉觚貞諒,實惟朝望。志厲強直,性乖夷曠。長虞剛簡,無虧風尚。子莊才識,爰膺袞職。忠績未申,泉途遽逼。



\end{pinyinscope}