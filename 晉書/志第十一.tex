\article{志第十一}

\begin{pinyinscope}

 禮下



 五禮之別,三曰賓,蓋朝宗、覲遇、會同之制是也。自周以下,其禮彌繁。自秦滅學之後,舊典殘缺。漢興,始使叔孫通制禮,參用先代之儀,然亦往往改異焉。漢儀有正會禮,正旦,夜漏未盡七刻,鐘鳴受賀,公侯以下執贄夾庭,二千石以上升殿稱萬歲,然後作樂宴饗。魏武帝都鄴,正會文昌殿,用漢儀,又設百華燈。



 晉氏受命,武帝更定
 元會儀,《咸寧注》是也。傅玄《元會賦》曰:「考夏后之遺訓,綜殷周之典藝,採秦漢之舊儀,定元正之嘉會。」此則兼採眾代可知矣。



 《咸寧注》:「先正一日,有司各宿設。夜漏未盡十刻,群臣集到,庭燎起火。上賀,起,謁報,又賀皇后。還,從雲龍東中華門入,詣東閣下,便坐。漏未盡七刻,百官及受贄郎官以下至計吏皆入立其次,其陛衛者如臨軒儀。漏未盡五刻,謁者、僕射、大鴻臚各各奏群臣就位定。漏盡,侍中奏外辦。皇帝出,鐘鼓作,百官皆拜伏。太常導皇帝升御坐,鐘鼓止,百官起。大鴻臚跪奏『請朝賀』。掌禮郎讚『皇帝延
 王登』。大鴻臚跪贊『籓王臣某等奉白璧各一,再拜賀』。太常報』王悉登』。謁者引上殿,當御坐。皇帝興,王再拜。皇帝坐,復再拜。跪置璧御坐前,復再拜。成禮訖,謁者引下殿,還故位。掌禮郎贊『皇帝延太尉等』。於是公、特進、匈奴南單于、金紫將軍當大鴻臚西,中二千石、二千石、千石、六百石當大行令西,皆北面伏。鴻臚跪讚『太尉、中二千石等奉璧、皮、帛、羔、鴈、雉,再拜賀』。太常贊『皇帝延公等登』。掌禮引公至金紫將軍上殿。皇帝興,皆再拜。皇帝坐,又再拜。跪置璧皮帛御坐前,復再拜。成禮訖,謁者引下殿,還故位。公置璧成禮時,大行令並贊殿下,中二千石以下
 同。成禮訖,以贄授贄郎,郎以璧帛付謁者,羔、鴈、雉付太官。太樂令跪請奏雅樂,樂以次作。乘黃令乃出車,皇帝罷入,百官皆坐。晝漏上水六刻,諸蠻夷胡客以次入,皆再拜訖,坐。御入後三刻又出,鐘鼓作。謁者、僕射跪奏『請群臣上』。謁者引王公二千石上殿,千石、六百石停本位。謁者引王詣樽酌壽酒,跪授侍中,侍中跪置御坐前。王還。王自酌置位前。謁者跪奏『籓王臣某等奉觴,再拜上千萬歲壽』。四廂樂作,百官再拜。已飲,又再拜。謁者引王等還本位。陛下者傳就席,群臣皆跪諾。侍中、中書令、尚書令各於殿上上壽酒。登歌樂升,太官又行御酒。御酒升
 階,太官令跪授侍郎,侍郎跪進御坐前。乃行百官酒。太樂令跪奏『奏登歌』,三終乃降。太官令跪請具御飯,到階,群臣皆起。太官令持羹跪授司徒,持飯跪授大司農,尚食持案並授持節,持節跪進御坐前。群臣就席。太樂令跪奏『奏食舉樂』。太官行百官飯案遍。食畢,太樂令跪奏『請進樂』。樂以次作。鼓吹令又前跪奏『請以次進眾妓』。乃召諸郡計吏前,受敕戒於階下。宴樂畢,謁者一人跪奏『請罷退』。鐘鼓作,群臣北面再拜,出。」然則,夜漏未盡七刻謂之晨賀。晝漏上三刻更出,百官奉壽酒,謂之晝會。別置女樂三十人於黃帳外,奏房中之歌。



 江左多虞,不復
 晨賀。夜漏未盡十刻,開宣陽門,至平旦始開殿門,晝漏上五刻,皇帝乃出受賀。皇太子出會者,則在三恪下王公上。正旦元會,設白獸樽於殿庭,樽蓋上施白獸,若有能獻直言者,則發此樽飲酒。案禮,白獸樽乃杜舉之遺式也,為白獸蓋,是後代所為,示忌憚也。



 魏制,籓王不得朝覲。魏明帝時,有朝者皆由特恩,不得以為常。及泰始中,有司奏:「諸侯之國,其王公以下入朝者,四方各為二番,三歲而周,周則更始。若臨時有故,卻在明年。明年來朝之後,更滿三歲乃復朝,不得違本數。朝禮皆親執璧,如舊朝之制。不朝之歲,各遣卿奉聘。」奏
 可。江左王侯不之國,其有受任居外,則同方伯刺史二千石之禮,亦無朝聘之制,故此禮遂廢。



 漢以高帝十月定秦,且為歲首。至武帝,雖改用夏正,然每月朔朝,至於十月朔,猶常饗會。其儀,夜漏未盡七刻,受賀及贄。公侯璧,中二千石、二千石羔,千石、六百石雁,四百石以下雉。三公奉璧上殿御坐前,北面。太常讚曰『皇帝為君興』。三公伏。皇帝坐,乃前進璧。百官皆賀,二千石以上上殿稱萬歲,舉觴,御食,司徒奉羹,大司農奉飯,奏食舉之樂。百官受賜,宴饗,大作樂,如元正之儀。魏晉則冬至日受方國及百僚稱賀,因小會。其儀亞於獻歲之旦。



 古者帝王莫不巡狩。魏文帝值天下三分,方隅多事,皇輿亟動,役無寧歲,蓋應時之務,非舊章也。明帝凡三東巡狩,所過存問高年,恤疾苦,或賜穀帛,有古巡幸之風焉。齊王正始元年,巡洛陽縣,賜高年力田各有差。



 及武帝泰始四年,詔刺史二千石長吏曰:「古之王者,以歲時巡狩方岳,其次則二伯述職,不然則行人順省。故雖幽遐側微,心無壅隔,下情上通,上指遠諭,至于鰥寡,罔不得所,用垂風遺烈,休聲猶存。朕在位累載,如臨深川,夙興夕惕,明發不寢,坐而待旦,思四方水旱災眚,為之怛然。
 勤躬約己,欲令事事當宜。常恐眾吏用情,誠心未著,萬機兼猥,慮有不周,政刑失謬,而弗獲備覽。百姓有過,在予一人。惟歲之不易,未遑卜征巡省之事,下之未乂,其何以恤之。今使使持節侍中副給事黃門侍郎銜命四出,周行天下,親見刺史二千石長吏,申諭朕心,訪求得失損益諸宜,觀省政教,問人間患苦。周典有之曰:『其萬姓之利害為一書,其禮俗政事刑禁之逆順為一書,其暴亂作慝犯令為一書,其札喪凶荒厄貧為一書,其康樂和親安平為一書,每國辨異之,以返命于王。』舊章前訓,今率由之。還具條奏,俾朕昭然鑒于幽遠,若親行焉。
 大夫君子,其各悉乃心,敬乃事,嘉言令圖,苦言至戒,與使者盡之,無所隱諱。方將慮心以俟,其勉哉勖之,稱朕意焉。」



 新禮,巡狩方嶽,柴望告設壝宮如禮。諸侯之覲者,賓及執贄皆如朝儀,而不建旗。摯虞以為:「覲禮,諸侯覲天子,各建其旗。旗章所以殊爵命,示等威。《詩》稱『君子至止,言觀其旂』。宜定新禮,建旗如舊禮。」詔可其議。然終晉代,其禮不行。



 封禪之說,經典無聞。禮有因天事天,因地事地,因名山升中於天,而鳳皇降,龜龍格。天子所以巡狩,至於方嶽,燔柴祭天,以告其成功,事似而非也。讖緯諸說皆云,王
 者封泰山,禪梁甫,易姓紀號。秦漢行其典,前史各陳其制矣。



 魏文帝黃初中,護軍蔣濟奏曰:「夫帝王大禮,巡狩為先;昭祖揚禰,封禪為首。是以自古革命受符,未有不蹈梁父,登泰山,刊無竟之名,紀天人之際者也。故司馬相如謂有文以來,七十二君,或順所繇於前,謹遺教於後。太史公曰,主上有聖明而不宣布,有司之過也。然則元功懿德,不刊梁山之石,無以顯帝王之功,示兆庶不朽之觀也。語曰,『當君而歎堯舜之美,譬猶人子對厥所生而譽他人之父』。今大魏承百王之弊亂,拯流遁之艱厄,接千
 載之衰緒,繼百代之廢業。始自武文,至於聖躬,所以參成天地之道,綱維人神之化。上天報應,嘉瑞顯祥,以比往古,無所取喻。至於歷世迄今,未廢大禮。雖志在掃盡殘盜,蕩滌餘穢,未遑斯事。若爾,三苗屈彊於江海,大舜當廢東巡之儀;徐夷跳梁於淮泗,周成當止岱嶽之禮。且去歲破吳虜於江漢,今茲屠蜀賊於隴右,其震蕩內潰,在不復淹,無累於封禪之事也。此儀久廢,非倉卒所定。宜下公卿,廣撰其禮,卜年考時,昭告上帝,以副天下之望。臣待罪軍旅,不勝大願,冒死以聞。」詔曰:「聞蔣濟斯言,使吾汗出流足。自開闢以來,封禪者七十餘君耳。故太
 史公曰,雖有受命之君,而功有不洽,是以中間曠遠者千有餘年,近者數百載,其儀闕不可得記。吾何德之修,敢庶茲乎!濟豈謂世無管仲,以吾有桓公登泰山之志乎!吾不欺天也。濟之所言,華則華矣,非助我者也。公卿侍中尚書常侍省之而已,勿復有所議,亦不須答詔也。」天子雖距濟議,而實使高堂隆草封禪之儀,以天下未一,不欲便行大禮,會隆卒,不復行之。



 及武帝平吳,混一區宇,太康元年九月庚寅,尚書令衛瓘、尚書左僕射山濤、右僕射魏舒、尚書劉寔、司空張華等奏曰:「臣聞肇自生靈,則有后辟,年載之數,莫之能紀。立德濟世,揮揚仁風,
 以登封泰山者七十有四家,其謚號可知者十有四焉。沈淪寂寞,曾無遺聲者,不可勝記。大晉之德,始自重黎,實佐顓頊,至于夏商,世序天地。其在于周,不失其緒。金德將升,世濟明聖,外平蜀漢,海內歸心,武功之盛,實由文德。至于陛下,受命踐阼,弘建大業,群生仰流。惟獨江湖沅湘之表,凶桀負固,歷代不賓。神謀獨斷,命將出討,兵威暫加,數旬蕩定。羈其鯨鯢,赦其罪逆,雲覆雨施,八方來同,聲教所被,達于四極。雖黃軒遐征,大禹遠略,周之奕世,何以尚今!若夫玄石素文,底號前載,象以數表,言以事告,雖古《河圖洛書》之徵,不是過也。宜宣大典,禮
 中嶽,封泰山,禪梁父,發德號,明至尊,享天休,篤黎庶,勒千載之表,播流後之聲,俾百世之下,莫不興起。斯帝王之盛業,天人之至望也。」詔曰:「今逋寇雖殄,外則障塞有警,內則百姓未寧,此盛德之事,所未議也。」



 瓘等又奏曰:「今東漸於海,西被流沙,大漠之陰,日南北戶,莫不通屬,芒芒禹跡,今實過之。天人之道已周,巍巍之功已著,宜修禮地祗,登封泰山,致誠上帝,以答人神之願也。乞如前奏。」詔曰:「今陰陽未和,刑政未當,百姓未得其所,豈可以勒功告成邪!」詔不許。



 瓘等又奏曰:「臣聞處帝王之位者,必有歷運之期,天命之應;濟兆庶之功者,必有盛
 德之容,告成之典。無不可誣,有不敢讓,自古道也。而明詔謙沖,屢辭其禮,雖盛德攸在,推而未居。夫三公職典天地,實掌人物,國之大事,取議於此。故漢氏封禪,非是官也,不在其事。臣等前奏,蓋陳祖考之功,天命又應,陛下之德,合同四海,迹古考今,宜修此禮。至於克定歲月,須五府上議,然後奏聞。」詔曰:「雖蕩清江表,皆臨事者之勞,何足以告成。方望群后思隆大化,以寧區夏,百姓獲乂,與之休息。斯朕日夜之望,無所復下諸府矣。」



 瓘等又奏:「臣聞唐虞三代濟世弘功之君,莫不仰承天休,俯協人志,登介丘,履梁父,未有辭焉者,蓋不可讓也。今陛下
 勛高百王,德無與二,茂績宏規,巍巍之業,固非臣等所能究論。而聖旨勞謙,屢自抑損,時至弗應,推美不居,闕皇代之上儀,塞靈祗之款望,何以使大晉之典謨,同風於三五?臣等誠不敢奉詔,請如前奏施行。」詔曰:「方當共思弘道,以康庶績,且俟他年,無所復紛紜也。」



 王公有司又奏:「自古聖明,光宅四海,封禪名山,著於史籍,作者七十四君矣。舜禹之有天下也,巡狩四嶽,躬行其道。《易》著觀俗省方,《禮》有升中于天,《詩》頌陟其高山,皆載在方策。文王為西伯以服事殷,周公以魯籓列于諸侯,或享于岐山,或有事泰山,徒以聖德,猶得為其事。自是以來,功薄而僭
 其義者,不可勝數。號謚不泯,以至于今。況高祖宣皇帝肇開王業,海外有截;世宗景皇帝濟以大功,輯寧區夏;太祖文皇帝受命造晉,盪定蜀漢;陛下應期龍興,混一六合,澤被群生,威震無外。昔漢氏失統,吳蜀鼎峙,兵興以來,近將百年,地險俗殊,人望絕塞。今不羈之寇,二代而平,非聰明神武,先天弗違,孰能巍巍其有成功若茲者歟!臣等幸以千載得遭運會,親服大化,目睹太平,至公至美,誰與為讓。宜祖述先朝,憲章古昔,勒功岱嶽,登封告成,弘禮樂之制,正三雍之典,揚名萬世,以顯祖宗。是以不勝大願,敢昧死以聞。請告太常,具禮儀復上。」詔曰:「
 所議誠列代之盛事也,然方今未可以爾。」便報絕之。



 哀帝即位,欲尊崇章皇太妃。桓溫議宜稱太夫人。尚書僕射江[A170]議曰:「虞舜體仁孝之性,盡事親之禮,貴為天王,富有四海,而瞽叟無立錐之地,一級之爵。蒸蒸之心,昊天罔極,寧當忍父卑賤,不以徽號顯之,豈不以子無爵父之道,理窮義屈,靡所厝情者哉!《春秋經》曰『紀季姜歸於京師』,《傳》曰『父母之於子,雖為天王后,猶曰吾季姜』,言子尊不加父母也。或以為子尊不加父母,則武王何以追王太王、王季、文王乎?周之三王,德配天地,王跡之興,自此始也。是以武王仰尋前緒,遂奉天命,追崇祖考,
 明不以子尊加父母也。案《禮》『幼不誄長,賤不誄貴』,幼賤猶不得表彰長貴,況敢錫之以榮命邪!漢祖感家令之言而尊太公,荀悅以為孝莫大于嚴父,而以子貴加之父母,家令之言過矣。爰逮孝章,不上賈貴人以尊號,而厚其金寶幣帛,非子道之不至也,蓋聖典不可踰也。當春秋時,庶子承國,其母得為夫人。不審直子命母邪,故當告於宗祧以先君之命命之邪?竊見詔書,當臨軒拜授貴人為皇太妃。今稱皇帝策命命貴人,斯則子爵母也。貴人北面拜受,斯則母臣子也。天尊地卑,名位定矣,母貴子賤,人倫序矣。雖欲加崇貴人,而實卑之;雖顯明
 國典,而實廢之。且人主舉動,史必書之。如當載之方策,以示後世,無乃不順乎!竊謂應告顯宗之廟,稱貴人仁淑之至,宜加殊禮,以酬鞠育之惠。奉先靈之命,事不在己。妃后雖是配君之名,然自后以下有夫人九嬪,無稱妃焉。桓公謂宜進號太夫人,非不允也。如以夫人為少,可言皇太夫人。皇,君也,君太夫人於名禮順矣。」帝特下詔拜皇太妃。三月丙辰,使兼太保王恬授璽綬儀服,一如太后。又詔曰:「朝臣不為太妃敬,為合禮不?」太常江逌議:「位號不極,不應盡敬。」



 孝武追崇會稽鄭太妃為簡文太后,詔問「當開墓不」。王珣答:「據三祖追贈及中宗敬后,
 並不開墓位,更為塋域制度耳。」



 褚太后臨朝時,議褚裒進見之典。蔡謨、王彪之並以:「虞舜、漢高祖猶執子道,況后乎!王者父無拜禮。」尚書八座議以為:「純子則王道缺,純臣則孝道虧。謂公庭如臣,私覿則嚴父為允。」



 漢魏故事,皇太子稱臣。新禮以太子既以子為名,而又稱臣,臣子兼稱,於義不通,除太子稱臣之制。摯虞以為:「《孝經》『資於事父以事君』,義兼臣子,則不嫌稱臣,宜定新禮皇太子稱臣如舊。」詔從之。



 太寧三年三月戊辰,明帝立皇子衍為皇太子。癸巳,詔
 曰:「禮無生而貴者,故帝元子方之於士。而漢魏以來,尊崇儲貳,使官屬稱臣,朝臣咸拜,此甚無謂。吾昔在東宮,未及啟革。今衍幼沖之年,便臣先達,將令日習所見,謂之自然,此豈可以教之邪!主者其下公卿內外通議,使必允禮中。」尚書令卞壼議以為:「《周禮》王后太子不會,明禮同於君,皆所以重儲貳,異正嫡。茍奉之如君,不得不拜矣。太子若存謙沖,故宜答拜。臣以為皇太子之立,郊告天地,正位儲宮,豈得同之皇子揖讓而已。謂宜稽則漢魏,闔朝同拜。」從之。



 太元中,尚書符問王公已下見皇太子儀及所衣服。侍
 中領國子博士車胤議:「朝臣宜朱衣褠幘,拜敬,太子答拜。案經傳不見其文,故太傅羊祜箋慶太子,稱叩頭死罪,此則拜之證也。又太寧三年詔議其典,尚書卞壼謂宜稽則漢魏,闔朝同拜。其朱衣冠冕,惟施之天朝,宜褠幘而已。」朝議多同。



 太元十二年,議二王後與太子先後。博士庾弘之及尚書參議,並以為:「陳留,國之上賓。皇太子雖國之儲貳,猶在臣位,陳留王坐應在太子上。」陳留王勱表稱疾病積年,求放罷,詔禮官博士議之。博士曹耽云:「勱為祭主而無執祭之期,宜與穆子、孟摯事同。」王彪之云:「二王之後,
 不宜輕致廢立。記傳未見有已為君而疾病退罷者,當知古無此禮。孟縶、穆子是方應為君,非陳留之比。」



 咸康四年,成帝臨軒,遣使拜太傅、太尉、司空。《儀注》,太樂宿懸於殿庭。門下奏,非祭祀宴饗,則無設樂之制。太常蔡謨議曰:「凡敬其事則備其禮,禮備則制有樂。樂者,所以敬事而明義,非為耳目之娛,故冠亦用之,不惟宴饗。宴饗之有樂,亦所以敬賓也。故郤至使楚,楚子饗之,郤至辭曰:『不忘先君之好,貺之以大禮,重之以備樂。』尋斯辭也,則宴樂之意可知矣。公侯大臣,人君所重,故御坐為起,在輿為下,言稱伯舅。《傳》曰『國卿,君之貳也』,是以命使
 之日,御親臨軒,百僚陪列,此即敬事之意也。古者,天王饗下國之使,及命將帥,遣使臣,皆有樂。故《詩序》曰:『皇皇者華,君遣使臣也。』又曰:『《採薇》以遣之,《出車》以勞還,《杕杜》以勤歸。』皆作樂而歌之。今命大使,拜輔相,比於下國之臣,輕重殊矣。輕誠有之,重亦宜然。故謂臨軒遣使,宜有金石之樂。」議奏從焉。



 漢魏故事,王公群妾見於夫人,夫人不答拜。新禮以為禮無不答,更制妃公侯夫人答妾拜。摯虞以為:「禮,妾事女君如婦之事姑,妾服女君期,女君不報,則敬與婦同而又加賤也。名位不同,本無酬報。禮無不答,義不謂此。
 先聖殊嫡庶之別,以絕陵替之漸。峻明其防,猶有僭違。宜定新禮,自如其舊。」詔可其議。



 五禮之別,其四曰軍,所以和外寧內,保大定功者也。但兵者凶事,故因搜狩而習之。



 漢儀,立秋之日,自郊禮畢,始揚威武,斬牲於東門,以薦陵廟。其儀,乘輿御戎路,白馬朱鬣,躬執弩射牲,牲以鹿麛。太宰令謁者各一人載以獲車,馳送陵廟。還宮,遣使者齎束帛以賜武官。武官肄兵,習戰陣之儀。斬牲之禮,名曰劉。兵官皆肄孫吳兵法六十四陣。既還,公卿已下陳陽前街,乘輿到,公卿已下拜,天子下車,公卿
 親識顏色,然後還宮。古語曰在車下車,則惟此時施行。漢世率以為常。至獻帝建安二十一年,魏國有司奏:「古四時講武,皆於農隙。漢西京承秦制,三時不講,惟十月都講。今金革未偃,士眾素習,可無四時講武。但以立秋擇吉日大朝車騎,號曰閱兵,上合禮名,下承漢制。」奏可。是冬,閱兵,魏王親執金鼓以令進退。延康元年,魏文帝為魏王。是年六月立秋,閱兵于東郊,公卿相儀,王御華蓋,親令金鼓之節。魏明帝太和元年十月,又閱兵。



 武帝泰始四年九月,咸守元年,太康四年,六年冬,皆自
 臨宣武觀,大閱眾軍,然不自令進退也。自惠帝以後,其禮遂廢。元帝太興四年,詔左右衛及諸營教習,依大習儀作雁羽仗。成帝咸和中,詔內外諸軍戲兵於南郊之場,故其地因名鬥場。自後籓鎮桓、庾諸方伯往往閱習,然朝廷無事焉。



 漢魏故事,遣將出征,符節郎授節鉞於朝堂。其後荀顗等所定新禮,遣將,御臨軒,尚書受節鉞,依古兵書跪而推轂之義也。



 五禮之別,其五曰嘉,宴饗冠婚之道於是乎備。周末崩離,賓射宴饗之則罕復能行,冠婚飲食之法又多遷變。



 《周禮》雖有服冕之數,而無天子冠文。又《儀禮》云,公侯之有冠禮,夏之末造也。王、鄭皆以為夏末上下相亂,篡弒由生,故作公侯冠禮,則明無天子冠禮之審也。大夫又無冠禮,古者五十而後爵,何大夫冠禮之有。周人年五十而有賢才,則試以大夫之事,猶行士禮也。故筮日筮賓,冠於阼以著代,醮於客位,三加彌尊,皆士禮耳。



 然漢代以來,天子諸侯頗採其儀。正月甲子若丙子為吉日,可加元服,儀從冠禮是也。漢順帝冠,又兼用曹褒新禮,
 乘輿初加緇布進賢,次爵弁、武弁,次通天,皆於高廟,以禮謁見世祖廟。王公已下,初加進賢而已。案此文,始冠緇布,從古制也,冠於宗廟是也。



 魏天子冠一加。其說曰:「士禮三加,加有成也。至於天子諸侯無加數之文者,將以踐阼臨下,尊極德備,豈得與士同也。魏氏太子再加,皇子王公世子乃三加。孫毓以為一加再加,皆非也。



 《禮》醮辭曰:「令月吉曰,以歲之正,以月之令。」案魯襄公冠以冬,漢惠帝冠以三月,明無定月。而後漢以來,帝加元服咸以正月。及咸寧二年秋閏九月,遣使冠汝南王柬,此則非必歲首。



 禮冠於廟,然武、惠冠太子,太子皆即廟見,
 斯亦擬在廟之儀也。穆帝、孝武將冠,皆先以幣告廟,訖又廟見也。



 惠帝之為太子,將冠,武帝臨軒,使兼司徒高陽王珪加冠,兼光祿大夫屯騎校尉華暠贊冠。



 江左諸帝將冠,金石宿設,百僚陪位。又豫於殿上鋪大床,御府令奉冕、幘、簪導、袞服以授侍中常侍,太尉加幘,太保加冕。將加冕,太尉跪讀祝文曰:「令月吉日,始加元服。皇帝穆穆,思弘袞職。欽若昊天,六合是式。率遵祖考,永永無極。眉壽惟祺,介茲景福。」加冕訖,侍中繫玄紞,侍中脫帝絳紗服,加袞服冕冠。事畢,太保率群臣奉觴上壽,王公以下三稱萬歲乃退。案《儀注》,一加幘冕而已。



 泰始十年,南宮王承年十五,依舊應冠。有司議奏:「禮,十五成童,國君十五而生子,以明可冠之宜。又漢魏遣使冠諸王,非古典。」於是制諸王十五而冠,不復加使命。



 王彪之云,《禮》、《傳》冠皆在廟。案成帝既加元服,車駕出拜于太廟,以告成也。蓋亦猶擬在廟之儀。



 魏齊王正始四年,立皇后甄氏,其儀不存。



 武帝咸寧二年,臨軒,遣太尉賈充策立皇后楊氏,納悼后也。因大赦,賜王公以下各有差,百僚上禮。



 太康八年,有司奏:「婚禮納徵,大婚用玄纁束帛,加珪,馬二駟。王侯玄纁束帛,加璧,乘馬。大夫用玄纁束帛,加羊。古
 者以皮馬為庭實,天子加以穀珪,諸侯加大璋,可依周禮改璧用璋,其羊鴈酒米玄纁如故。諸侯婚禮,加納采、告期、親迎各帛五匹,及納徵馬四匹,皆令夫家自備。惟璋,官為具致之。」尚書朱整議:「案魏氏故事,王娶妃、公主嫁之禮,天子諸侯以皮馬為庭實,天子加以穀珪,諸侯加以大璋。漢高后制聘,后黃金二百斤,馬十二匹。夫人金五十斤,馬四匹。魏氏王娶妃、公主嫁之禮,用絹百九十匹。晉興,故事用絹三百匹。」詔曰:「公主嫁由夫氏,不宜皆為備物,賜錢使足而已。惟給璋,餘如故事。」



 成帝咸康二年,臨軒,遣使持節、兼太保、領軍將軍諸葛
 恢,兼太尉、護軍將軍孔愉,六禮備物,拜皇后杜氏。即日入宮,帝御太極殿,群臣畢賀。賀,非禮也。王者婚禮,禮無其制。《春秋》「祭公逆王后于紀」,《穀梁》、《左氏傳》說與《公羊》又不同。而自漢魏遺事,並皆闕略。武、惠納后,江左又無復《儀注》。故成帝將納杜后,太常華恆始與博士參定其儀。據杜預《左氏傳》說,主婚是供其婚禮之幣而已。又,周靈王求婚於齊,齊侯問於晏桓子,桓子對曰:「夫婦所生若如人,姑姊妹則稱先守某公之遺女若如人。」此則天子之命自得下達,臣下之答徑自上通。先儒以為丘明詳錄其事,蓋為王者婚娶之禮也。故成帝臨軒,遣使稱制
 拜后,然其《儀注》又不具存。



 康帝建元元年,納皇后褚氏,而《儀注》陛者不設旄頭。殿中御史奏:「今迎皇后,依成恭皇后入宮御物,而《儀注》至尊袞冕升殿,旄頭不設,求量處。又案,昔迎恭皇后,惟作青龍旂,其餘皆即御物。今當臨軒遣使,而立五牛旂,旄頭罼䍐並出即用,故致今闕。」詔曰:「所以正法服、升太極者,以敬其始,故備其禮也。今云何更闕所重而撤法物邪!又恭后神主入廟,先帝詔后禮宜降,不宜建五牛旗,而今猶復設之邪!既不設五牛旗,則旄頭罼䍐之物易具也。」又詔曰:「舊制既難準,且於今而備,亦非宜。府庫之
 儲,惟當以供軍國之費耳。法服儀飾粗令舉,其餘兼副雜器停之。」



 穆帝升平元年,將納皇后何氏。太常王彪之大引經傳及諸故事以定其禮,深非《公羊》婚禮不稱主人之義。又曰:『王者之於四海,無不臣妾,雖復父兄之親,師友之賢,皆純臣也。夫崇三綱之始,以定乾坤之儀,安有天父之尊,而稱臣下之命以納伉儷。安有臣下之卑,而稱天父之名以行大禮。遠尋古禮,無王者此制;近求史籍,無王者此比。於情不安,於義不通。案咸寧二年,納悼皇后時,弘訓太后母臨天下,而無命戚屬之臣為武皇父兄
 主婚之文。又考大晉已行之事,咸寧故事不稱父兄師友,則咸康華恒所上禮合於舊。臣愚謂今納后儀制。宜一依咸康故事。」於是從之。華恒所定之禮,依漢舊及晉已行之制,故彪之多從咸康,由此也。惟以娶婦之家三日不舉樂,而咸康群臣賀,為失禮。故但依咸寧上禮,不復賀。其告廟六禮版文等儀,皆彪之所定也。其納采版文璽書曰:「皇帝咨前太尉參軍何琦。渾元資始,肇經人倫,爰及夫婦,以奉天地宗廟社稷。謀于公卿,咸以宜率由舊典。今使使持節太常彪之、宗正綜以禮納采。」主人曰:「皇帝嘉命,訪婚陋族,備數采擇。臣從祖弟故散騎侍郎
 準之遺女,未閑教訓,衣履若如人。欽承舊章,肅奉典制。前太尉參軍、都鄉侯糞土臣何琦稽首頓首,再拜承詔。」次問名版文曰:「皇帝曰:咨某官某姓。兩儀配合,承天統物,正位乎內,必俟令族,重申舊典。今使使持節、太常某,宗正某,以禮問名。」主人曰:「皇帝嘉命,使者某到,重宣中詔,問臣名族。臣族女父母所生,先臣故光祿大夫、雩婁侯禎之遺玄孫,先臣故豫州刺史、關中侯惲之曾孫,先臣故安豐太守、關中侯睿之孫,先臣故散騎侍郎準之遺女。外出自先臣故尚書左丞孔胄之外曾孫,先臣故侍中、關內侯夷之外孫女,年十七。欽承舊章,肅奉典制。」次納
 吉版文曰:「皇帝曰:咨某官某姓。人謀龜從,僉曰貞吉,敬從典禮。今使使持節、太常某,宗正某以禮納吉。」主人曰:「皇帝嘉命,使者某重宣中詔,太卜元吉。臣陋族卑鄙,憂懼不堪。欽承舊章,肅奉典制。」次納徵版文曰:「皇帝曰:咨某官某姓之女,有母儀之德,窈窕之姿,如山如河,宜奉宗廟,永承天祚。以玄纁皮帛,馬羊錢璧,以章典祀。今使使侍節、司徒某,太常某,以禮納徵。」主人曰:「皇帝嘉命,降婚卑陋,崇以上公,寵以典禮,備物典策。欽承舊章,肅奉典制。」次請期版文曰:「皇帝曰:咨某官某姓。謀於公卿,泰筮元龜,罔有不臧,率遵典禮。今使使持節、太常某,宗正
 某,以禮請期。」主人曰:「皇帝嘉命,使者某重宣中詔,吉日惟某可迎。臣欽承舊章,肅奉典制。」次親迎版文曰:「皇帝曰:咨某官某姓。歲吉月令,吉日惟某,率禮以迎。今使使持節、太保某,太尉某,以禮迎。」主人曰:「皇帝嘉命,使者某重宣中詔,令月吉辰,備禮以迎。上公宗卿兼至,副介近臣百兩。臣螻蟻之族,猥承大禮,憂懼戰悸。欽承舊章,肅奉典制。」某稽首承詔,皆如初答。



 孝武納王皇后,其禮亦如之。其納採、問名、納吉、請期、親迎,皆用白鴈、白羊各一頭,酒米各十二斛。惟納徵羊一頭,玄纁用帛三匹,絳二匹,絹二百匹,獸皮二枚,錢二百萬,玉璧一枚,馬六匹,酒
 米各十二斛。鄭玄所謂五鴈六禮也。其珪馬之制,備物之數,校太康所奏又有不同云。



 古者婚冠皆有醮,鄭氏醮文三首具存。



 升平八年,臺符問「迎皇后大駕應作鼓吹不」。博士胡訥議:「臨軒《儀注》闕,無施安鼓吹處所,又無舉麾鳴鐘之條。」太常王彪之以為:「婚禮不樂。鼓吹亦樂之總名。《儀注》所以無者,依婚禮。今宜備設而不作。」時用此議。



 永和二年納后,議賀不。王述云:「婚是嘉禮。《春秋傳》曰:『娶者大吉,非常吉。』又《傳》曰:『鄭子罕如晉,賀夫人。』鄰國猶相賀,況臣下邪!如此,便應賀,但不在三日內耳。今因廟見
 成禮而賀,亦是一節也。」王彪之議云:「婚禮不樂不賀,《禮》之明文。《傳》稱子罕如晉賀夫人,既無《經》文,又《傳》不云禮也。《禮》,取婦三日不舉樂,明三日之後自當樂。至於不賀,無三日之斷,恐三日之後故無應賀之禮。」又云:「《禮記》所以言賀取妻者,是因就酒食而有慶語也。愚謂無直相賀之體,而有禮貺共慶會之義,今世所共行。」于時竟不賀。



 穆帝納后欲用九月,九月是忌月。范汪問王彪之,答云:「禮無忌月,不敢以所不見,便謂無之。」博士曹耽、荀訥等並謂無忌月之文,不應有妨。王洽曰:「若有忌月,當復有
 忌歲。」



 太元十二年,臺符問「皇太子既拜廟,朝臣奉賀,應上禮與不?國子博士車胤云:「百辟卿士,咸預盛禮,展敬拜伏,不須復上禮。惟方伯牧守,不睹大禮,自非酒牢貢羞,無以表其乃誠,故宜有上禮。猶如元正大慶,方伯莫不上禮,朝臣奉璧而已。」太學博士庾弘之議:「案咸寧三年始平、濮陽諸王新拜,有司奏依故事,聽京城近臣諸王公主應朝賀者復上禮。今皇太子國之儲副,既已崇建,普天同慶。謂應上禮奉賀。」徐邈同。又引一有元良,慶在於此。封諸王及新宮上禮,既有前事,亦皆已瞻仰致敬,而
 又奉觴上壽,應亦無疑也。



 江左以來,太子婚,納徵禮用玉璧一,獸皮二,未詳何所準況。或者獸取其威猛有班彩,玉以象德而有溫潤。尋珪璋亦玉之美者,豹皮采蔚以譬君子。王肅納徵辭云:「玄纁束帛,儷皮鴈羊。」前漢聘后,黃金二百斤,馬十二匹,亦無用羊之旨。鄭氏《婚物贊》曰「羊者祥也」,然則婚之有羊,自漢末始也。王者六禮,尚未用焉。是故太康中有司奏:「太子婚,納徵用玄纁束帛,加羊馬二駟。」



 武帝泰始十年,將聘拜三夫人、九嬪。有司奏:「禮,皇后聘以穀珪,無妾媵禮贄之制。」詔曰:「拜授可依魏氏故事。」於
 是臨軒,使使持節兼太常拜三夫人,兼御史中丞拜九嬪。



 漢魏之禮云,公主居第,尚公主者來第成婚。司空王朗以為不可,其後乃革。太元中,公主納徵以獸豹皮各一具禮,豈謂婚禮不辨王公之序,故取獸豹以尊崇其事乎!



 《禮》有三王養老膠庠之文,饗射飲酒之制,周末淪廢。漢明帝永平二年三月,帝始率群臣躬養三老五更於辟雍,行大射之禮。郡國縣道行鄉飲酒于學校,皆祠先聖先師周公孔子,牲以太牢。孟冬亦如之。及魏高貴鄉公
 甘露二年,天子親帥群司行養老之禮。於是王祥為三老,鄭小同為五更。其《儀注》不存,然漢禮猶在。



 武帝泰始六年十二月,帝臨辟雍,行鄉飲酒之禮。詔曰:「禮儀之廢久矣,乃今復講肄舊典。」賜太常絹百匹,丞、博士及學生牛酒。咸寧三年,惠帝元康九年,復行其禮。



 魏正始中,齊王每講經遍,輒使太常釋奠先聖先師於辟雍,弗躬親。及惠帝明帝之為太子,及愍懷太子講經竟,並親釋奠於太學,太子進爵於先師,中庶子進爵於顏回。成、穆、孝武三帝,亦皆親釋奠。孝武時,以太學在水南懸遠,有司議依升平元年,於中堂權立行太學。于時無復
 國子生,有司奏:「應須復二學生百二十人。太學生取見人六十,國子生權銓大臣子孫六十人,事訖罷。」奏可。釋奠禮畢,會百官六品以上。



 漢儀,季春上巳,官及百姓皆禊於東流水上,洗濯祓除去宿垢。而自魏以後,但用三日,不以上巳也。晉中朝公卿以下至于庶人,皆禊洛水之側。趙王倫篡位,三日會天泉池,誅張林。懷帝亦會天泉池,賦詩。陸機云:「天泉池南石溝引御溝水,池西積石為禊堂。」本水流杯飲酒,亦不言曲水。元帝又詔罷三日弄具。海西於鐘山立流杯曲水,延百僚,皆其事也。九月九日,馬射。或說云「秋,金之
 節,講武習射,象立秋之禮也」。



\end{pinyinscope}