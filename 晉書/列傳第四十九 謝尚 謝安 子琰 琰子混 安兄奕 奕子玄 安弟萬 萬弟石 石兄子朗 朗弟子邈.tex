\article{列傳第四十九 謝尚 謝安 子琰 琰子混 安兄奕 奕子玄 安弟萬 萬弟石 石兄子朗 朗弟子邈}

\begin{pinyinscope}

 謝尚 謝安 子琰 琰子混 安兄奕 奕子玄 安弟萬 萬弟石 石兄子朗 朗弟子邈



 謝尚,字仁祖,豫章太守鯤之子也。幼有至性。七歲喪兄,哀慟過禮,親戚異之。八歲神悟夙成。鯤嘗攜之送客,或曰:「此兒一坐之顏回也。」尚應聲答曰:「坐無尼父,焉別顏回!」席賓莫不歎異。十餘歲,遭父憂,丹陽尹溫嶠弔之,尚號咷極哀。既而收涕告訴,舉止有異常童,嶠甚奇之。及長,開率穎秀,辨悟絕倫,脫略細行,不為流俗之事。好衣
 刺文褲,諸父責之,而因自改,遂知名。善音樂,博綜眾藝。司徒王導深器之,比之王戎,常呼為「小安豐」,辟為掾。襲父爵咸亭侯。始到府通謁,導以其有勝會,謂曰:「聞君能作鴝鵒舞,一坐傾想,寧有此理不?」尚曰:「佳。」便著衣幘而舞,導令坐者撫掌擊節,尚俯仰在中,傍若無人,其率詣如此。



 轉西曹屬,時有遭亂與父母乖離,議者或以進仕理王事,婚姻繼百世,於理非嫌。尚議曰:「典禮之興,皆因循情理,開通弘勝。如運有屯夷,要當斷之以大義。夫無後之罪,三千所不過,今婚姻將以繼百世,崇宗緒,此固不可塞也。然至於天屬生離之哀,父子乖絕之痛,痛之
 深者,莫深於茲。夫以一體之小患,猶或忘思慮,損聽察,況於抱傷心之巨痛,懷忉恆之至戚,方寸既亂,豈能綜理時務哉!有心之人,決不冒榮茍進。冒榮茍進之疇,必非所求之旨,徒開偷薄之門而長流弊之路。或有執志丘園、守心不革者,猶當崇其操業以弘風尚,而況含艱履戚之人,勉之以榮貴邪?」



 遷會稽王友,入補給事黃門侍郎,出為建武將軍、歷陽太守,轉督江夏義陽隨三郡軍事、江夏相,將軍如故。時安西將軍庾翼鎮武昌,尚數詣翼咨謀軍事。嘗與翼共射,翼曰:「卿若破的,當以鼓吹相賞。」尚應聲中之,翼即以其副鼓吹給之。尚為政清簡,
 始到官,郡府以布四十匹為尚造烏布帳。尚壞之,以為軍士褚襦褲。建元二年,詔曰:「尚往以戎戍事要,故輟黃散,以授軍旅。所處險要,宜崇其威望。今以為南中郎將,餘官如故。」會庾冰薨,復以本號督豫州四郡,領江州刺史。俄而復轉西中郎將、督揚州之六郡諸軍事、豫州刺史、假節,鎮歷陽。



 大司馬桓溫欲有事中原,使尚率眾向壽春,進號安西將軍。初,苻健將張遇降尚,尚不能綏懷之。遇怒,據許昌叛。尚討之,為遇所敗,收付廷尉。時康獻皇后臨朝,即尚之甥也,特令降號為建威將軍。初,尚之行也,使建武將軍、濮陽太守戴施據枋頭。會冉閔之子智
 與其大將蔣幹來附,復遣行人劉猗詣尚請救。施止猗,求傳國璽,猗歸,以告幹。乾謂尚已敗,慮不能救己,猶豫不許。施遣參軍何融率壯士百入鄴,登三臺助戍,譎之曰:「今且可出璽付我。凶寇在外,道路梗澀,亦未敢送璽,當遣單使馳白。天子聞璽已在吾許,知卿等至誠,必遣重軍相救,並厚相餉。」幹乃出璽付融,融齎璽馳還枋頭。尚遣振武將軍胡彬率騎三百迎璽致諸京師。時苻健將楊平戍許昌,尚遣兵襲破之,徵授給事中,賜軺車、鼓吹,戍石頭。



 永和中,拜尚書僕射,出為都督江西淮南諸軍事、前將軍、豫州刺史,給事中、僕射如故,鎮歷陽,加
 都督豫州揚州之五郡軍事,在任有政績。上表求入朝,因留京師,署僕射事。尋進號鎮西將軍,鎮壽陽。尚於是採拾樂人,並制石磬,以備太樂。江表有鐘石之樂,自尚始也。



 桓溫北平洛陽,上疏請尚為都督司州諸軍事。將鎮洛陽,以疾病不行。升平初,又進都督豫、冀、幽、并四州。病篤,徵拜衛將軍,加散騎常侍,未至,卒於歷陽,時年五十。詔贈散騎常侍、衛將軍、開府儀同三司,謚曰簡。



 無子,從弟奕以子康襲爵,早卒。康弟靜復以子肅嗣,又無子。靜子虔以子靈祐繼鯤後。



 謝安,字安石,尚從弟也。父裒,太常卿。安年四歲時,譙郡桓彞見而歎曰:「此兒風神秀徹,後當不減王東海。」及總角,神識沈敏,風宇條暢,善行書。弱冠,詣王蒙,清言良久,既去,蒙子脩曰:「向客何如大人?」蒙曰:「此客亹亹,為來逼人。」王導亦深器之。由是少有重名。



 初辟司徒府,除佐著作郎,並以疾辭。寓居會稽,與王羲之及高陽許詢、桑門支遁游處,出則漁弋山水,入則言詠屬文,無處世意。揚州刺史庾冰就以安有重名,必欲致之,累下郡縣敦逼,不得已赴召,月餘告歸。復除尚書郎、琅邪王友,並不起。吏
 部尚書范汪舉安為吏部郎,安以書距絕之。有司奏安被召,歷年不至,禁錮終身,遂棲遲東土。嘗往臨安山中,坐石室,臨濬谷,悠然歎曰:「此去伯夷何遠!」嘗與孫綽等泛海,風起浪湧,諸人並懼,安吟嘯自若。舟人以安為悅,猶去不止。風轉急,安徐曰:「如此將何歸邪?」舟人承言即回。眾咸服其雅量。安雖放情丘壑,然每游賞,必以妓女從。既累辟不就,簡文帝時為相,曰:「安石既與人同樂,必不得不與人同憂,召之必至。」時安弟萬為西中郎將,總籓任之重。安雖處衡門,其名猶出萬之右,自然有公輔之望,處家常以儀範訓子弟。安妻,劉惔妹也,既見家門
 富貴,而安獨靜退,乃謂曰:「丈夫不如此也?」安掩鼻曰:「恐不免耳。」及萬黜廢,安始有仕進志,時年已四十餘矣。



 征西大將軍桓溫請為司馬,將發新亭,朝士咸送,中丞高崧戲之曰:「卿累違朝旨,高臥東山,諸人每相與言,安石不肯出,將如蒼生何!蒼生今亦將如卿何!」安甚有媿色。既到,溫甚喜,言生平,歡笑竟日。既出,溫問左右:「頗嘗見我有如此客不?」溫後詣安,值其理髮。安性遲緩,久而方罷,使取幘。溫見,留之曰:「令司馬著帽進。」其見重如此。溫當北征,會萬病卒,安投箋求歸。尋除吳興太守。在官無當時譽,去後為人所思。頃之徵拜侍中,遷吏部尚書、中
 護軍。



 簡文帝疾篤,溫上疏薦安宜受顧命。及帝崩,溫入赴山陵,止新亭,大陳兵衛,將移晉室,呼安及王坦之,欲於坐害之。坦之甚懼,問計於安。安神色不變,曰:「晉祚存亡,在此一行。」既見溫,坦之流汗沾衣,倒執手版。安從容就席,坐定,謂溫曰:「安聞諸侯有道,守在四鄰,明公何須壁後置人邪?」溫笑曰:「正自不能不爾耳。」遂笑語移日。坦之與安初齊名,至是方知坦之之劣。溫嘗以安所作簡文帝謚議以示坐賓,曰:「此謝安石碎金也。」



 時孝武帝富於春秋,政不自己,溫威振內外,人情噂沓,互生同異。安與坦之盡忠匡翼,終能輯穆。及溫病篤,諷朝廷加九錫,
 使袁宏具草。安見,輒改之,由是歷旬不就。會溫薨,錫命遂寢。



 尋為尚書僕射,領吏部,加後將軍。及中書令王坦之出為徐州刺史,詔安總關中書事。安義存輔導,雖會稽王道子亦賴弼諧之益。時彊敵寇境,邊書續至,梁益不守,樊鄧陷沒,安每鎮以和靖,御以長算。德政既行,文武用命,不存小察,弘以大綱,威懷外著,人皆比之王導,謂文雅過之。嘗與王羲之登冶城,悠然遐想,有高世之志。羲之謂曰:「夏禹勤王,手足胼胝;文王旰食,日不暇給。今四郊多壘,宜思自效,而虛談廢務,浮文妨要,恐非當今所宜。」安曰:「秦任商鞅,二世而亡,豈清言致患邪?」



 是時
 宮室毀壞,安欲繕之。尚書令王彪之等以外寇為諫,安不從,竟獨決之。宮室用成,皆仰模玄象,合體辰極,而役無勞怨。又領揚州刺史,詔以甲仗百人入殿。時帝始親萬機,進安中書監、驃騎將軍、錄尚書事,固讓軍號。於時懸象失度,亢旱彌年,安奏興滅繼絕,求晉初佐命功臣後而封之。頃之,加司徒,後軍文武盡配大府,又讓不拜。復加侍中、都督揚豫徐兗青五州幽州之燕國諸軍事、假節。



 時苻堅強盛,疆場多虞,諸將敗退相繼。安遣弟石及兄子玄等應機征討,所在剋捷。拜衛將軍、開府儀同三司,封建昌縣公。堅後率眾,號百萬,次于淮肥,京師震
 恐。加安征討大都督。玄入問計,安夷然無懼色,答曰:「已別有旨。」既而寂然。玄不敢復言,乃令張玄重請。安遂命駕出山墅,親朋畢集,方與玄圍棋賭別墅。安常棋劣於於玄,是日懼,便為敵手而又不勝。安顧謂其甥羊曇曰:「以墅乞汝。」安遂游涉,至夜乃還,指授將帥,各當其任。玄等既破堅,有驛書至,安方對客圍棋,看書既竟,便攝放床上,了無喜色,棋如故。客問之,徐答云:「小兒輩遂已破賊。」既罷,還內,過戶限,心喜甚,不覺屐齒之折,其矯情鎮物如此。以總統功,進拜太保。



 安方欲混一文軌,上疏求自北征,乃進都督揚、江、荊、司、豫、徐、兗、青、冀、幽、并、寧、益、雍、
 梁十五州軍事,加黃鉞,其本官如故,置從事中郎二人。安上疏讓太保及爵,不許。是時桓沖既卒,荊、江二州並缺,物論以玄勳望,宜以授之。安以父子皆著大勳,恐為朝廷所疑,又懼桓氏失職,桓石虔復有沔陽之功,慮其驍猛,在形勝之地,終或難制,乃以桓石民為荊州,改桓伊於中流,石虔為豫州。既以三桓據三州,彼此無恐,各得所任。其經遠無競,類皆如此。



 性好音樂,自弟萬喪,十年不聽音樂。及登台輔,期喪不廢樂。王坦之書喻之,不從,衣冠效之,遂以成俗。又於土山營墅,樓館林竹甚盛,每攜中外子姪往來游集,肴饌亦屢費百金,世頗以
 此譏焉,而安殊不以屑意。常疑劉牢之既不可獨任,又知王味之不宜專城。牢之既以亂終,而味之亦以貪敗,由是識者服其知人。



 時會稽王道子專權,而姦諂頗相扇構,安出鎮廣陵之步丘,築壘曰新城以避之。帝出祖于西池,獻觴賦詩焉。安雖受朝寄,然東山之志始末不渝,每形於言色。及鎮新城,盡室而行,造泛海之裝,欲須經略粗定,自江道還東。雅志未就,遂遇疾篤。上疏請量宜旋旆,并召子征虜將軍琰解甲息徒,命龍驤將軍朱序進據洛陽,前鋒都督玄抗威彭沛,委以董督。若二賊假延,來年水生,東西齊舉。詔遣侍中慰勞,遂還都。聞當
 輿入西州門,自以本志不遂,深自慨失,因悵然謂所親曰:「昔桓溫在時,吾常懼不全。忽夢乘溫輿行十六里,見一白雞而止。乘溫輿者,代其位也。十六里,止今十六年矣。白雞主酉,今太歲在酉,吾病殆不起乎!」乃上疏遜位,詔遣侍中、尚書喻旨。先是,安發石頭,金鼓忽破,又語未嘗謬,而忽一誤,眾亦怪異之。尋薨,時年六十六。帝三日臨於朝堂,賜東園祕器、朝服一具、衣一襲、錢百萬、布千匹、蠟五百斤,贈太傅,謚曰文靖。以無下舍,詔府中備凶儀。及葬,加殊禮,依大司馬桓溫故事。又以平苻堅勛,更封廬陵郡公。



 安少有盛名,時多愛慕。鄉人有罷中宿縣
 者,還詣安。安問其歸資,答曰:「有蒲葵扇五萬。」安乃取其中者捉之,京師士庶競市,價增數倍。安本能為洛下書生詠,有鼻疾,故其音濁,名流愛其詠而弗能及,或手掩鼻以斅之。及至新城,築埭於城北,後人追思之,名為召伯埭。



 羊曇者,太山人,知名士也,為安所愛重。安薨後,輟樂彌年,行不由西州路。嘗因石頭大醉,扶路唱樂,不覺至州門。左右白曰:「此西州門。」曇悲感不已,以馬策扣扉,誦曹子建詩曰:「生存華屋處,零落歸山丘。」慟哭而去。



 安有二子:瑤、琰。瑤襲爵,官至琅邪王友,早卒。子該嗣,終東陽太守。無子,弟光祿勳模以子承伯嗣,有罪,國除。劉裕
 以安勛德濟世,特更封該弟澹為柴桑侯,邑千戶,奉安祀。澹少歷顯位,桓玄篡位,以澹兼太尉,與王謐俱齎冊到姑孰。元熙中,為光祿大夫,復兼太保,持節奉冊禪宋。



 琰字瑗度。弱冠以貞幹稱,美風姿。與從兄護軍淡雖比居,不往來,宗中子弟惟與才令者數人相接。拜著作郎,轉祕書丞,累遷散騎常侍、侍中。苻堅之役,安以琰有軍國才用,出為輔國將軍,以精卒八千,與從兄玄俱陷陣破堅,以勳封望蔡公,尋遭父憂去官,服闋,除征虜將軍、會稽內史。頃之。徵為尚書右僕射,領太子詹事,加散騎常侍,將軍如故。又遭母憂,朝廷疑其葬禮。時議者云:「潘
 岳為賈充婦《宜城宣君誄》云:『昔在武侯,喪禮殊倫。伉儷一體,朝儀則均。』謂宜資給葬,悉依太傅故事。」先是,王珣娶萬女,珣弟氏娶安女,並不終,由是與謝氏有隙。珣時為僕射,猶以前憾緩其事。琰聞恥之,遂自造轀輬車以葬,議者譏之。



 太元末,為護軍將軍,加右將軍。會稽王道子以為司馬,右將軍如故。王恭舉兵,假琰節,都督前鋒軍事。恭平,遷衛將軍、徐州刺史、假節。孫恩作亂,加督吳興、義興二郡軍事,討恩。至義興,斬賊許允之,迎太守魏鄢還郡。進討吳興賊丘尪,破之。又詔琰與輔國將軍劉牢之俱討孫恩。恩逃於海島,朝廷憂之,以琰為會稽
 內史、都督五郡軍事,本官並如故。琰既以資望鎮越土,議者謂無復東顧之虞。及至郡,無綏撫之能,而不為武備。將帥皆諫曰:「強賊在海,伺人形便,宜振揚仁風,開其自新之路。」琰曰:「苻堅百萬,尚送死淮南,況孫恩奔衄歸海,何能復出!若其復至,正是天不養國賊,令速就戮耳。」遂不從其言。恩後果復寇浹口,入餘姚,破上虞,進及邢浦,去山陰北三十五里。琰遣參軍劉宣之距破恩。既而上黨太守張虔碩戰敗,群賊銳進,人情震駭,咸以宜持重嚴備,且列水軍於南湖,分兵設伏以待之。琰不聽。賊既至,尚未食,琰曰:「要當先滅此寇而後食也。」跨馬而出。
 廣武將軍桓寶為前鋒,摧鋒陷陣,殺賊甚多,而塘路迮狹,琰軍魚貫而前,賊於艦中傍射之,前後斷絕。琰至千秋亭,敗績。琰帳下都督張猛於後斫琰馬,琰墮地,與二子肇、峻俱被害,寶亦死之。後劉裕左里之捷,生擒猛,送琰小子混,混刳肝生食之。詔以琰父子隕於君親,忠孝萃於一門,贈琰侍中、司空,謚曰忠肅。



 三子:肇、峻、混。肇歷驃騎參軍,峻以琰勛封建昌侯。及沒於賊,詔贈肇散騎常侍,峻散騎侍郎。



 混字叔源。少有美譽,善屬文。初,孝武帝為晉陵公主求婿,謂王珣曰:「主婿但如劉真長、王子敬便足。如王處仲、
 桓元子誠可,才小富貴,便豫人家事。」珣對曰:「謝混雖不及真長,不減子敬。」帝曰:「如此便足。」未幾,帝崩,袁山松欲以女妻之,珣曰:「卿莫近禁臠。」初,元帝始鎮建業,公私窘罄,每得一,以為珍膳,項上一臠尤美,輒以薦帝,群下未嘗敢食,于時呼為「禁臠」,故珣因以為戲。混竟尚主,襲父爵。桓玄嘗欲以安宅為營,混曰:「召伯之仁,猶惠及甘棠;文靖之德,更不保五畝之宅邪?」玄聞,慚而止。歷中書令、中領軍、尚書左僕射、領選。以黨劉毅誅,國除。及宋受禪,謝晦謂劉裕曰:「陛下應天受命,登壇日恨不得謝益壽奉璽紱。」裕亦歎曰:「吾甚恨之,使後生不得見其風流!」益
 壽,混小字也。



 奕字無奕,少有名譽。初為剡令,有老人犯法,奕以醇酒飲之,醉猶未已。安時年七八歲,在奕膝邊,諫止之。奕為改容,遣之。與桓溫善。溫辟為安西司馬,猶推布衣好。在溫坐,岸幘笑詠,無異常日。桓溫曰:「我方外司馬。」奕每因酒,無復朝廷禮,嘗逼溫飲,溫走入南康主門避之。主曰:「君若無狂司馬,我何由得相見!」奕遂攜酒就聽事,引溫一兵帥共飲,曰:「失一老兵,得一老兵,亦何所怪。」溫不之責。從兄尚有德政,既卒,為西蕃所思,朝議以奕立行有素,必能嗣尚事,乃遷都督豫司冀并四州軍事、安西將
 軍、豫州刺史、假節。未幾。卒官,贈鎮西將軍。



 三子:泉、靖、玄。泉早有名譽,歷義興太守。靖官至太常。



 玄字幼度。少穎悟,與從兄朗俱為叔父安所器重。安嘗戒約子姪,因曰:「子弟亦何豫人事,而正欲使其佳?」諸人莫有言者。玄答曰:「譬如芝蘭玉樹,欲使其生於庭階耳。」安悅。玄少好佩紫羅香囊,安患之,而不欲傷其意,因戲賭取,即焚之,於此遂止。



 及長,有經國才略,屢辟不起。後與王珣俱被桓溫辟為掾,並禮重之。轉征西將軍桓豁司馬、領南郡相、監北征諸軍事。於時苻堅彊盛,邊境數被侵寇,朝廷求文武良將可以鎮禦北方者,安乃以玄
 應舉。中書郎郗超雖素與玄不善,聞而歎之,曰:「安違眾舉親,明也。玄必不負舉,才也。」時咸以為不然。超曰:「吾嘗與玄共在桓公府,見其使才,雖履屐間亦得其任,所以知之。」於是徵還,拜建武將軍、兗州刺史、領廣陵相、監江北諸軍事。



 時苻堅遣軍圍襄陽,車騎將軍桓沖禦之。詔玄發三州人丁,遣彭城內史何謙游軍淮泗,以為形援。襄陽既沒,堅將彭超攻龍驤將軍戴逯於彭城。玄率東莞太守高衡、後軍將軍何謙次于泗口,欲遣間使報逯,令知救至,其道無由。小將田泓請行,乃沒水潛行,將趣城,為賊所獲。賊厚賂泓,使云「南軍已敗」。泓偽許之。既而
 告城中曰:「南軍垂至,我單行來報,為賊所得,勉之!」遂遇害。時彭超置輜重於留城,玄乃揚聲遣謙等向留城。超聞之,還保輜重。謙馳進,解彭城圍。超復進軍南侵,堅將句難、毛當自襄陽來會。超圍幽州刺史田洛於三阿,有眾六萬。詔征虜將軍謝石率水軍次涂中,右衛將軍毛安之、游擊將軍河間王曇之、淮南太守楊廣、宣城內史丘準次堂邑。既而盱眙城陷,高密內史毛藻沒,安之等軍人相驚,遂各散退,朝廷震動。玄於是自廣陵西討難等。何謙解田洛圍,進據白馬,與賊大戰,破之,斬其偽將都顏。因復進擊,又破之。斬其偽將邵保。超、難引退。玄
 率何謙、戴逯、田洛追之,戰于君川,復大破之。玄參軍劉牢之攻破浮航及白船,督護諸葛侃、單父令李都又破其運艦。難等相率北走,僅以身免。於是罷彭城、下邳二戍。詔遣殿中將軍慰勞,進號冠軍,加領徐州刺史,還于廣陵,以功封東興縣侯。



 及苻堅自率兵次于項城,眾號百萬,而涼州之師始達咸陽,蜀漢順流,幽并係至。先遣苻融、慕容、張蠔、苻方等至潁口,梁成、王顯等屯洛澗。詔以玄為前鋒、都督徐兗青三州揚州之晉陵幽州之燕國諸軍事,與叔父征虜將軍石、從弟輔國將軍琰、西中郎將桓伊、龍驤將軍檀玄、建威將軍戴熙、揚武將軍
 陶隱等距之,眾凡八萬。玄先遣廣陵相劉牢之五千人直指洛澗,即斬梁成及成弟雲,步騎崩潰,爭赴淮水。牢之縱兵追之,生擒堅偽將梁他、王顯、梁悌、慕容屈氏等,收其軍實。堅進屯壽陽,列陣臨肥水,玄軍不得渡。玄使謂苻融曰:「君遠涉吾境,而臨水為陣,是不欲速戰。諸君稍卻,令將士得周旋,僕與諸君緩轡而觀之,不亦樂乎!」堅眾皆曰:「宜阻肥水,莫令得上。我眾彼寡,勢必萬全。」堅曰:「但卻軍,令得過,而我以鐵騎數十萬向水,逼而殺之。」融亦以為然,遂麾使卻陣,眾因亂不能止。於是玄與琰、伊等以精銳八千涉渡肥水。石軍距張蠔,小退。玄、琰仍
 進,決戰肥水南。堅中流矢,臨陣斬融。堅眾奔潰,自相蹈藉投水死者不可勝計,肥水為之不流。餘眾棄甲宵遁,聞風聲鶴唳,皆以為王師已至,草行露宿,重以飢凍,死者十七八。獲堅乘輿雲母車,儀服、器械、軍資、珍寶山積,牛馬驢騾駱駝十萬餘。詔遣殿中將軍慰勞。進號前將軍、假節,固讓不受。賜錢百萬,彩千匹。


既而安奏苻堅喪敗,宜乘其釁會,以玄為前鋒都督,率冠軍將軍桓石虔徑造渦潁,經略舊都。玄復率眾次于彭城,遣參軍劉襲攻堅兗州刺史張崇於鄄城,走之,使劉牢之守鄄城。兗州既平,玄患水道險澀,糧運艱難,用督護聞人奭謀,堰
 呂梁水,樹柵,立七埭為派,擁二岸之流,以利運漕,自此公私利便。又進伐青州,故謂之青州派。遣淮陵太守高素以三千人向廣固,降堅青州刺史苻朗。又進伐冀州,遣龍驤將軍劉牢之、濟北太守丁匡據碻磝,濟陽太守郭滿據滑臺,奮武將軍顏雄渡河立營。堅子丕遣將桑據屯黎陽。玄命劉襲夜襲據,走之。丕惶遽欲降,玄許之。丕告飢,玄饋丕米二千斛。又遣晉陵太守滕恬之渡河守黎陽,三魏皆降。以兗、青、司、豫平,加玄都督徐、兗、青、司、冀、幽、并七州軍事。玄上疏以方平河北,幽冀宜須總督,司州縣遠,應統豫州。以勳封康樂縣公。玄請以先封東
 興侯賜兄子玩,詔聽之,更封玩豫寧伯。復遣寧遠將軍
 \gezhu{
  夭曰}
 演伐申凱於魏郡,破之。玄欲令豫州刺史朱序鎮梁國,玄住彭城,北固河上,西援洛陽,內籓朝廷。朝議以征役既久,宜置戍而還,使玄還鎮淮陰,序鎮壽陽。會翟遼據黎陽反,執滕恬之,又泰山太守張願舉郡叛,河北騷動,玄自以處分失所,上疏送節,盡求解所職。詔慰勞,令且還鎮淮陰,以朱序代鎮彭城。



 玄既還,遇疾,上疏解職,詔書不許。玄又自陳,既不堪攝職,慮有曠廢,詔又使移鎮東陽城。玄即路,於道疾篤,上疏曰:



 臣以常人,才不佐世,忽蒙殊遇,不復自量,遂從戎政。驅馳十載,不辭鳴鏑
 之險,每有徵事,輒請為軍鋒,由恩厚忘軀,甘死若生也。冀有毫釐,上報榮寵。天祚大晉,王威屢舉,實由陛下神武英斷,無思不服。亡叔臣安協贊雍熙,以成天工。而雰霧尚翳,六合未朗,遺黎塗炭,巢窟宜除,復命臣荷戈前驅,董司戎首。冀仰憑皇威,宇宙寧一,陛下致太平之化,庸臣以塵露報恩,然後從亡叔臣安退身東山,以道養壽。此誠以形于文旨,達於聖聽矣。臣所以區區家國,實在於此,不謂臣愆咎夙積,罪鐘中年,上延亡叔臣安、亡兄臣靖,數月之間,相係殂背,下逮稚子,尋復夭昏。哀毒兼纏,痛百常情。臣不勝禍酷暴集,每一慟殆弊。所以含
 哀忍悲,期之必存者,雖哲輔傾落,聖明方融,伊周嗣作,人懷自厲,猶欲申臣本志,隆國保家,故能豁其情滯,同之無心耳。



 去冬奉司徒道子告括囊遠圖,逮問臣進止之宜。臣進不達事機,以蹙境為恥,退不自揆,故欲順其宿心。豈謂經略不振,自貽斯戾。是以奉送章節,待罪有司,執徇常儀,實有愧心。而聖恩赦過,黷法垂宥,使抱罪之臣復得更名於所司。木石猶感,而況臣乎!顧將身不良,動與釁會,謙德不著,害盈是荷,先疾既動,便至委篤,陛下體臣疢重,使還籓淮側。甫欲休兵靜眾,綏懷善撫,兼苦自療,冀日月漸瘳,繕甲俟會,思更奮迅。而所患沈
 頓,有增無損。今者惙惙,救命朝夕。臣之平日,率其常矩,加以匪懈,猶不能令政理弘宣,況今內外天隔,永不復接,寧可臥居重任,以招患慮。



 追尋前事,可為寒心。臣之微身,復何足惜,區區血誠,憂國實深。謹遣兼長史劉濟重奉送節蓋章傳。伏願陛下垂天地之仁,拯將絕之氣,時遣軍司鎮慰荒雜,聽臣所乞,盡醫藥消息,歸誠道門,冀神祇之祐。若此而不差,修短命也。使臣得及視息,瞻睹墳柏,以此之盡,公私真無恨矣,伏枕悲慨,不覺流涕。



 詔遣高手醫一人,令自消息,又使還京口療疾。玄奉詔便還,病久不差,又上疏曰:「臣同生七人,凋落相繼,惟臣
 一己,孑然獨存。在生荼酷,無如臣比。所以含哀忍痛,希延視息者,欲報之德,實懷罔極,庶蒙一瘳,申其此志。且臣孤遣滿目,顧之惻然,為欲極其求生之心,未能自分於灰士。慺慺之情,可哀可愍。伏願陛下矜其所訴,霈然垂恕,不令微臣銜恨泉壤。」表寢不報。前後表疏十餘上,久之。乃轉授散騎常侍、左將軍、會稽內史。時吳興太守晉寧侯張玄之亦以才學顯,自吏部尚書與玄同年之郡,而玄之名亞於玄,時人稱為「南北二玄」,論者美之。玄既輿疾之郡,十三年,卒於官,時年四十六。追贈車騎將軍、開府儀同三司,謚曰獻武。



 子瑍嗣,祕書郎,早卒。子靈
 運嗣。瑍少不惠,而靈運文藻艷逸,玄嘗稱曰:「我尚生瑍,瑍那得生靈運!」永熙中,為劉裕世子左衛率。



 始從玄征伐者,何謙字恭子,東海人,戴逯字安丘,處士逵之弟,並驍果多權略。逵厲操東山,而逯以武勇顯。謝安嘗謂逯曰:「卿兄弟志業何殊?」逯曰:「下官不堪其憂,家兄不改其樂。」逯以軍功封廣信侯,位至大司農。



 萬字萬石,才器雋秀,雖器量不及安,而善自炫曜,故早有時譽。工言論,善屬文,敘漁父、屈原、季主、賈誼、楚老、龔勝、孫登、嵇康四隱四顯為《八賢論》,其旨以處者為優,出者為劣,以示孫綽。綽與往反,以體公識遠者則出處同
 歸。嘗與蔡系送客于征虜亭,與系爭言。系推萬落床,冠帽傾脫。萬徐拂衣就席,神意自若,坐定,謂系曰:「卿幾壞我面。」系曰:「本不為卿面計。」然俱不以介意,時亦以此稱之。



 弱冠,辟司徒掾,遷右西屬,不就。簡文帝作相,聞其名,召為撫軍從事中郎。萬著白綸巾,鶴氅裘,履版而前。既見,與帝共談移日。太原王述,萬之妻父也,為揚州刺史。萬嘗衣白綸巾,乘平肩輿,徑至聽事前,謂述曰:「人言君侯癡,君侯信自癡。」述曰:「非無此論,但晚合耳。」萬再遷豫州刺史、領淮南太守、監司豫冀并四州軍事、假節。王羲之與桓溫箋曰:「謝萬才流經通,處廊廟,參諷議,故是後
 來一器。而今屈其邁往之氣,以俯順荒餘,近是違才易務矣。」溫不從。



 萬既受任北征,矜豪傲物,嘗以嘯詠自高,未嘗撫眾。兄安深憂之,自隊主將帥已下,安無不慰勉。謂萬曰:「汝為元帥,諸將宜數接對,以悅其心,豈有傲誕若斯而能濟事也!」萬乃召集諸將,都無所說,直以如意指四坐云:「諸將皆勁卒。」諸將益恨之。既而先遣征虜將軍劉建修治馬頭城池,自率眾入渦潁,以援洛陽。北中郎將郗曇以疾病退還彭城,萬以為賊盛致退,便引軍還,眾遂潰散,狼狽單歸,廢為庶人。後復以為散騎常侍,會卒,時年四十二,因以為贈。



 子韶,字穆度,少有名。時謝
 氏憂彥秀者,稱封、胡、羯、末。封謂韶,胡謂主朗,羯謂玄,末謂川,皆其小字也。韶、朗、川並早卒,惟玄以功名終,韶至車騎司馬。韶子恩,字景伯,宏達有遠略,韶為黃門郎、武昌太守。恩三子、曜、弘微,皆歷顯位。



 朗字長度。父據,早卒。朗善言玄理,文義艷發,名亞於玄。總角時,病新起,體甚贏,未堪勞,於叔父安前與沙門支遁朗論,遂至相苦。其母王氏再遣信令還,安欲留,使竟論,王氏因出云:「新婦少遭艱難,一生所寄惟在此兒。」遂流涕攜朗去。安謂坐客曰:「家嫂辭情慷慨,恨不使朝士見之。」朗終於東陽太守。



 子重,字景重,明秀有才名,為會
 稽王道子驃騎長史。嘗因侍坐,于時月夜明凈,道子歎以為佳。重率爾曰:「意謂乃不如微雲點綴。」道子因戲重曰:「卿居心不凈,乃復強欲滓穢太清邪!」



 子絢,字宣映,曾於公坐戲調,無禮於其舅袁湛。湛甚不堪之,謂曰:「汝父昔已輕舅,汝今復來加我,可謂世無渭陽情也。」絢父重,即王胡之外孫,與舅亦有不協之論,湛故有此及云。



 石字石奴。初拜祕書郎,累遷尚書僕射。征句難,以勛封興平縣伯。淮肥之役,詔石解僕射,以將軍假節征討大都督,與兄子玄、琰破苻堅。先是,童謠云:「誰謂爾堅石打碎。」故桓豁皆以「石」名子,以邀功焉。堅之敗也,雖功始牢
 之,而成于玄、琰,然石時實為都督焉。遷中軍將軍、尚書令,更封南康郡公。於時學校陵遲,石上疏請興復國學,以訓胄子,班下州郡,普修鄉校。疏奏,孝武帝納焉。



 兄安薨,石遷衛將軍,加散騎常侍。以公事與吏部郎王恭互相短長,恭甚忿恨,自陳褊阨不允,且疾源深固,乞還私門。石亦上疏遜位。有司奏,石輒去職,免官。詔曰:「石以疾求退,豈準之常制!其喻令還。」歲餘不起。表十餘上,帝不許。石乞依故尚書令王彪之例,於府綜攝,詔聽之。疾篤,進位開府儀同三司,加鼓吹,未拜,卒,時年六十二。



 石少患面創,療之莫愈,乃自匿。夜有物來舐其瘡,隨舐隨差,
 舐處甚白,故世呼為謝白面。石在職務存文刻,既無他才望,直以宰相弟兼有大才,遂居清顯,而聚斂無饜,取譏當世。追贈司空,禮官議謚,博士范弘之議謚曰襄墨公,語在弘之傳。朝議不從,單謚曰襄。



 子汪嗣,早卒。汪從兄沖以子明慧嗣,為孫恩所害。明慧從兄喻復以子暠嗣。宋受禪,國除。



 邈字茂度。父鐵,永嘉太守。邈性剛鯁,無所屈撓,頗有理識。累遷侍中。時孝武帝觴樂之後多賜侍臣文詔,辭義有不雅者,邈輒焚毀之,其他侍臣被詔者或宣揚之,故論者以此多邈。後為吳興太守。孫恩之亂,為賊胡桀、郜
 驃等所執,害之。賊逼令北面,邈厲聲曰:「我不得罪天子,何北面之有!」遂害之。邈妻郗氏,甚妒。邈先娶妾,郗氏怨懟,與邈書告絕。邈以其書非婦人詞,疑其門下生仇玄達為之作,遂斥玄達。玄達怒,遂投孫恩,並害邈兄弟,竟至滅門。



 史臣曰:建元之後,時政多虞,巨猾陸梁,權臣橫恣。其有兼將相於中外,系存亡於社稷,負扆資之以端拱,鑿井賴之以晏安者,其惟謝氏乎!簡侯任總中臺,效彰分閫;正議云唱,喪禮墮而復弘;遺音既補,雅樂缺而還備。君子哉,斯人也!文靖始居塵外,高謝人間,嘯詠山林,浮泛
 江海,當此之時,蕭然有陵霞之致。暨于褫薜蘿而襲朱組,去衡泌而踐丹墀,庶績於是用康,彞倫以之載穆。苻堅百萬之眾已瞰吳江,桓溫九五之心將移晉鼎,衣冠易慮,遠邇崩心。從容而杜姦謀,宴衎而清群寇,宸居獲太山之固,惟揚去累卵之危,斯為盛矣。然激繁會於期服之辰,敦一歡於百金之費,廢禮於偷薄之俗,崇侈於耕戰之秋,雖欲混哀樂而同歸,齊奢儉於一致,而不知頹風已扇,雅道日淪,國之儀刑,豈期若是!琰稱貞幹,卒以忠勇垂名;混曰風流,竟以文詞獲譽:並階時宰,無墮家風。奕萬以放肆為高,石奴以褊濁興累,雖曰微纇,猶
 稱名實。康樂才兼文武,志存匡濟,淮肥之役,勍寇望之而土崩;渦潁之師,中州應之而席卷。方欲西平鞏洛,北定幽燕,廟算有遺,良圖不果,降齡何促,功敗垂成,拊其遺文,經綸遠矣。



 贊曰:安西英爽,才兼辯博。宣力方鎮,流聲臺閣。太保沈浮,曠若虛舟。任高百闢,情惟一丘。琰邈忠壯,奕萬虛放。為龍為光,或卿或將。偉哉獻武,功宣授斧。克翦兇渠,幾清中宇。



\end{pinyinscope}