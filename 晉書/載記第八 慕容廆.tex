\article{載記第八 慕容廆}

\begin{pinyinscope}

 慕容廆



 慕容廆,字弈洛瑰,昌黎棘城鮮卑人也。其先有熊氏之苗裔,世居北夷,邑於紫蒙之野,號曰東胡。其後與匈奴並盛,控弦之士二十餘萬,風俗官號與匈奴略同。秦漢之際為匈奴所敗,分保鮮卑山,因以為號。曾祖莫護跋,魏初率其諸部入居遼西,從宣帝伐公孫氏有功,拜率義王,始建國於棘城之北。時燕代多冠步搖冠,莫護跋
 見而好之,乃斂髮襲冠,諸部因呼之為步搖,其後音訛,遂為慕容焉。或云慕二儀之德,繼三光之容,遂以慕容為氏。祖木延,左賢王。父涉歸,以全柳城之功,進拜鮮卑單于,遷邑於遼東北,於是漸慕諸夏之風矣。



 廆幼而魁岸,美姿貌,身長八尺,雄傑有大度。安北將軍張華雅有知人之鑒,廆童冠時往謁之,華甚歎異,謂曰:「君至長必為命世之器,匡難濟時者也。」因以所服簪幘遺廆,結殷勤而別。涉歸死,其弟耐篡位,將謀殺廆,廆亡潛以避禍。後國人殺耐,迎廆立之。



 初,涉歸有憾於宇文鮮卑,廆將脩先君之怨,表請討之。武帝弗許。廆怒,入寇遼西,殺略
 甚眾。帝遣幽州諸軍討廆,戰于肥如,廆眾大敗。自後復掠昌黎,每歲不絕。又率眾東伐扶餘,扶餘王依慮自殺,廆夷其國城,驅萬餘人而歸。東夷校尉何龕遣督護賈沈將迎立依慮之子為王,廆遣其將孫丁率騎邀之。沈力戰斬丁,遂復扶餘之國。廆謀於其眾曰:「吾先公以來世奉中國,且華裔理殊,彊弱固別,豈能與晉競乎?何為不和以害吾百姓邪!」乃遣使來降。帝嘉之,拜為鮮卑都督。廆致敬於東夷府,巾衣詣門,抗士大夫之禮。何龕嚴兵引見,廆乃改服戎衣而入。人問其故,廆曰:「主人不以禮,賓復何為哉!」龕聞而慚之,彌加敬憚。時東胡宇文鮮
 卑段部以廆威德日廣,懼有吞併之計,因為寇掠,往來不絕。廆卑辭厚幣以撫之。



 太康十年,廆又遷于徒河之青山。廆以大棘城即帝顓頊之墟也,元康四年乃移居之。教以農桑,法制同於上國。永寧中,燕垂大水,廆開倉振給,幽方獲濟。天子聞而嘉之,褒賜命服。



 太安初,宇文莫圭遣弟屈雲寇邊城,雲別帥大素延攻掠諸部,廆親擊敗之。素延怒,率眾十萬圍棘城,眾咸懼,人無距志。廆曰:「素延雖犬羊蟻聚,然軍無法制,已在吾計中矣。諸君但為力戰,無所憂也。」乃躬貫甲胄,馳出擊之,素延大敗,追奔百里,俘斬萬餘人。



 永嘉初,廆自稱鮮卑大單于。遼
 東太守龐本以私憾殺東夷校尉李臻,附塞鮮卑素連、木津等託為臻報仇,實欲因而為亂,遂攻陷諸縣,殺掠士庶。太守袁謙頻戰失利,校尉封釋懼而請和。連歲寇掠,百姓失業,流亡歸附者日月相繼。廆子翰言於廆曰:「求諸侯莫如勤王,自古有為之君靡不杖此以成事業者也。今連、津跋扈,王師覆敗,蒼生屠膾,豈甚此乎!豎子外以龐本為名,內實幸而為寇。封使君以誅本請和,而毒害滋深。遼東傾沒,垂已二周,中原兵亂,州師屢敗,勤王杖義,今其時也。單于宜明九伐之威,救倒懸之命,數連、津之罪,合義兵以誅之。上則興復遼邦,下則並吞二
 部,忠義彰於本朝,私利歸于我國,此則吾鴻漸之始也,終可以得志于諸侯。」廆從之。是日,率騎討連、津,大敗斬之,二部悉降,徙之棘城,立遼東郡而歸。



 懷帝蒙塵于平陽,王浚承制以廆為散騎常侍、冠軍將軍、前鋒大都督、大單于,廆不受。建興中,愍帝遣使拜廆鎮軍將軍、昌黎、遼東二國公。建武初,元帝承制拜廆假節、散騎常侍、都督遼左雜夷流人諸軍事、龍驤將軍、大單于、昌黎公,廆讓而不受。征虜將軍魯昌說廆曰:「今兩京傾沒,天子蒙塵,瑯邪承制江東,實人命所係。明公雄據海朔,跨總一方,而諸部猶怙眾稱兵,未遵道化者,蓋以官非王命,又
 自以為彊。今宜通使琅邪,勸承大統,然後敷宣帝命,以伐有罪,誰敢不從!」廆善之,乃遣其長史王濟浮海勸進。及帝即尊位,遣謁者陶遼重申前命,授廆將軍、單于,廆固辭公封。



 時二京傾覆,幽、冀淪陷,廆刑政脩明,虛懷引納,流亡士庶多襁負歸之。廆乃立郡以統流人,冀州人為冀陽郡,豫州人為成周郡,青州人為營丘郡,並州人為唐國郡。於是推舉賢才,委以庶政,以河東裴嶷、代郡魯昌、北平陽耽為謀主,北海逢羨、廣平游邃、北平西方虔、渤海封抽、西河宋奭、河東裴開為股肱,渤海封弈、平原宋該、安定皇甫岌、蘭陵繆愷以文章才俊任居樞要,
 會稽朱左車、太山胡毋翼、魯國孔纂以舊德清重引為賓友,平原劉贊儒學該通,引為東庠祭酒,其世子皝率國胄束脩受業焉。廆覽政之暇,親臨聽之,於是路有頌聲,禮讓興矣。



 時平州刺史、東夷校尉崔毖自以為南州士望,意存懷集,而流亡者莫有赴之。毖意廆拘留,乃陰結高句麗及宇文、段國等,謀滅廆以分其地。太興初,三國伐廆,廆曰:「彼信崔毖虛說,邀一時之利,烏合而來耳。既無統一,莫相歸伏,吾今破之必矣。然彼軍初合,其鋒甚銳,幸我速戰。若逆擊之,落其計矣。靖以待之,必懷疑貳,迭相猜防。一則疑吾與毖譎而覆之,二則自疑三國
 之中與吾有韓魏之謀者,待其人情沮惑,然後取之必矣。」於是三國攻棘城,廆閉門不戰,遣使送牛酒以犒宇文,大言於眾曰:「崔毖昨有使至。」於是二國果疑宇文同於廆也,引兵而歸。宇文悉獨官曰:「二國雖歸,吾當獨兼其國,何用人為!」盡眾逼城,連營三十里。廆簡銳士配皝,推鋒於前;翰領精騎為奇兵,從旁出,直衝其營;廆方陣而進。悉獨官自恃其眾,不設備,見廆軍之至,方率兵距之。前鋒始交,翰已入其營,縱火焚之,其眾皆震擾,不知所為,遂大敗,悉獨官僅以身免,盡俘其眾。於是營候獲皇帝玉璽三紐,遣長史裴嶷送于建鄴。崔毖懼廆之仇
 己也,使兄子燾偽賀廆。會三國使亦至請和,曰:「非我本意也,崔平州教我耳。」廆將燾示以攻圍之處,臨之以兵,曰:「汝叔父教三國滅我,何以詐來賀我乎?」燾懼,首服。廆乃遣燾歸說毖曰:「降者上策,走者下策也。」以兵隨之。毖與數十騎棄家室奔于高句麗,廆悉降其眾,徙燾及高瞻等于棘城,待以賓禮。明年,高句麗寇遼東,廆遣眾擊敗之。



 裴嶷至自建鄴,帝遣使者拜廆監平州諸軍事、安北將軍、平州刺史,增邑二千戶。尋加使持節、都督幽州東夷諸軍事、車騎將軍、平州牧,進封遼東郡公,邑一萬戶,常侍、單于並如故;丹書鐵券,承制海東,命備官司,置
 平州守宰。



 段末波初統其國,而不脩備,廆遣皝襲之,入令支,收其名馬寶物而還。



 石勒遣使通和,廆距之。送其使於建鄴。勒怒,遣宇文乞得龜擊廆,廆遣皝距之。以裴嶷為右部都督,率索頭為右翼,命其少子仁自平郭趣柏林為左翼,攻乞得龜,剋之,悉虜其眾。乘勝拔其國城,收其資用億計,徙其人數萬戶以歸。



 成帝即位,加廆侍中,位特進。咸和五年,又加開府儀同三司,固辭不受。



 廆嘗從容言曰:「獄者,人命之所懸也,不可以不慎。賢人君子,國家之基也,不可以不敬。稼穡者,國之本也,不可以不急。酒色便佞,亂德之甚也,不可以不戒。」乃著《家令》數
 千言以申其旨。



 遣使與太尉陶侃箋曰:



 明公使君轂下:振德曜威,撫寧方夏,勞心文武,士馬無恙,欽高仰止,注情彌久。王途險遠,隔以燕越,每瞻江湄,延首遐外。



 天降艱難,禍害屢臻,舊都不守,奄為虜庭,使皇輿遷幸,假勢吳、楚。大晉啟基、祚流萬節,天命未改,玄象著明,是以義烈之士深懷憤踴。猥以功薄,受國殊寵,上不能掃除群羯,下不能身赴國難,仍縱賊臣,屢逼京輦。王敦唱禍於前,蘇峻肆毒於後,凶暴過於董卓,惡逆甚於傕、汜,普天率土,誰不同忿!深怪文武之士,過荷朝榮,不能滅中原之寇,刷天下之恥。



 君侯植根江陽,發曜荊、衡,杖葉公之
 權,有包胥之志,而令白公、伍員殆得極其暴,竊為丘明恥之。區區楚國子重之徒,猶恥君弱、群臣不及先大夫,厲己戒眾,以服陳、鄭;越之種蠡尚能弼佐句踐,取威黃池;況今吳土英賢比肩,而不輔翼聖主,陵江北伐。以義聲之直,討逆暴之羯,檄命舊邦之士,招懷存本之人,豈不若因風振落,頓阪走輪哉!且孫氏之初,以長沙之眾摧破董卓,志匡漢室。雖中遇寇害,雅志不遂,原其心誠,乃忽身命。及權據揚、越,外杖周、張,內馮顧、陸,距魏亦壁,剋取襄陽。自茲以降,世主相襲,咸能侵逼徐、豫,令魏朝旰食。不知今之江表為賢俊匿智,藏其勇略邪?將呂蒙、
 凌統高蹤曠世哉?況今凶羯虐暴,中州人士逼迫勢促,其顛沛之危,甚於累卵。假號之彊,眾心所去,敵有釁矣,易可震蕩。王郎、袁術雖自詐偽,皆基淺根微,禍不旋踵,此皆君侯之所聞見者矣。



 王司徒清虛寡欲,善於全己,昔曹參亦綜此道,著畫一之稱也。庾公居元舅之尊,處申伯之任,超然高蹈,明智之權。廆於寇難之際,受大晉累世之恩,自恨絕域,無益聖朝,徒繫心萬里,望風懷憤。今海內之望,足為楚、漢輕重者,惟在君侯。若戮力盡心,悉五州之眾,據兗、豫之郊,使向義之士倒戈釋甲,則羯寇必滅,國恥必除。廆在一方,敢不竭命。孤軍輕進,不足
 使勒畏首畏尾,則懷舊之士欲為內應,無由自發故也。故遠陳寫,言不宣盡。



 廆使者遭風沒海。其後廆更寫前箋,並齎其東夷校尉封抽、行遼東相韓矯等三十餘人疏上侃府曰:



 自古有國有家,鮮不極盛而衰。自大晉龍興,剋平昬、會,神武之略,邁蹤前史。惠皇之末,后黨構難,禍結京畿,釁成公族,遂使羯寇乘虛,傾覆諸夏,舊都淪滅,山陵毀掘,人神悲悼,幽明發憤。昔獫狁之彊,匈奴之盛,未有如今日羯寇之暴,跨躡華裔,盜稱尊號者也。



 天祚有晉,挺授英傑。車騎將軍慕容廆自弱冠蒞國,忠於王室,明允恭肅,志在立勳。屬海內分崩,皇輿遷幸,元皇
 中興,初唱大業,肅祖繼統,蕩平江外。廆雖限以山海,隔以羯寇,翹首引領,繫心京師,常假寤寐,欲憂國忘身。貢篚相尋,連舟載路,戎不稅駕,動成義舉。今羯寇滔天,怙其醜類,樹基趙、魏,跨略燕、齊。廆雖率義眾,誅討大逆,然管仲相齊,猶曰寵不足以御下,況廆輔翼王室,有匡霸之功,而位卑爵輕,九命未加,非所以寵異籓翰,敦獎殊勳者也。



 方今詔命隔絕,王路險遠,貢使往來,動彌年載。今燕之舊壤,北周沙漠,東盡樂浪,西暨代山,南極冀方,而悉為虜庭,非復國家之域。將佐等以為宜遠遵周室,近準漢初,進封廆為燕王,行大將軍事,上以總統諸部,
 下以割損賊境。使冀州之人望風向化。廆得祗承詔命,率合諸國,奉辭夷逆,以成桓文之功,茍利社稷,專之可也。而廆固執謙光,守節彌高,每詔所加,讓動積年,非將佐等所能敦逼。今區區所陳,不欲茍相崇重,而愚情至心,實為國計。



 侃報抽等書,其略曰:「車騎將軍憂國忘身,貢篚載路,羯賊求和,執使送之,西討段國,北伐塞外,遠綏索頭,荒服以獻。惟北部未賓,屢遣征伐。又知東方官號,高下齊班,進無統攝之權,退無等差之降,欲進車騎為燕王,一二具之。夫功成進爵,古之成制也。車騎雖未能為官摧勒,然忠義竭誠。今騰箋上聽,可不遲速,當任
 天臺也。」朝議未定。八年,廆卒,乃止。時年六十五,在位四十九年。帝遣使者策贈大將軍、開府儀同三司,謚曰襄。及俊僭號,偽謚武宣皇帝。



 裴嶷,字文冀,河東聞喜人也。父昶,司隸校尉。嶷清方有幹略,累遷至中書侍郎,轉給事黃門郎、滎陽太守。屬天下亂,嶷兄武先為玄菟太守,嶷遂求為昌黎太守。至郡,久之,武卒,嶷被徵,乃將武子開送喪俱南。既達遼西,道路梗塞,乃與開投廆。時諸流寓之士見廆草創,並懷去就。嶷首定名分,為群士啟行。廆甚悅,以嶷為長史,委以
 軍國之謀。



 及悉獨官寇逼城下,外內騷動,廆問策於嶷,嶷曰:「悉獨官雖擁大眾,軍無號令,眾無部陣,若簡精兵,乘其無備,則成擒耳。」廆從之,遂陷寇營。廆威德於此甚振,將遣使獻捷於建鄴,妙簡行人,令嶷將命。



 初,朝廷以廆僻在荒遠,猶以邊裔之豪處之。嶷既使至,盛言廆威略,又知四海英賢並為其用,舉朝改觀焉。嶷將還,帝試留嶷以觀之,嶷辭曰:「臣世荷朝恩,濯纓華省,因事遠寄,投迹荒遐。今遭開泰,得睹朝廷,復賜恩詔,即留京輦,於臣之私,誠為厚幸。顧以皇居播遷,山陵幽辱,慕容龍驤將軍越在遐表,乃心王室,慷慨之誠,義感天地,方掃平
 中壤,奉迎皇輿,故遣使臣,萬里表誠。今若留臣,必謂國家遺其僻陋,孤其丹心,使懷義懈怠。是以微臣區區忘身為國,貪還反命耳。」帝曰:「卿言是也。」乃遣嶷還。廆後謂群僚曰:「裴長史名重中朝,而降屈於此,豈非天以授孤也。」出為遼東相,轉樂浪太守。



 高瞻,字子前,渤海蓚人也。少而英爽有俊才,身長八尺二寸。光熙中,調補尚書郎。屬永嘉之亂,還鄉里,乃與父老議曰:「今皇綱不振,兵革雲擾,此郡沃壤,憑固河海,若兵荒歲儉,必為寇庭,非謂圖安之所。王彭祖先在幽、薊,
 據燕、代之資,兵彊國富,可以託也。諸君以為何如?」眾咸善之。乃與叔父隱率數千家北徙幽州。既而以王浚政令無恒,乃依崔毖,隨毖如遼東。



 毖之與三國謀伐廆也,瞻固諫以為不可,毖不從。及毖奔敗,瞻隨眾降于廆。廆署為將軍,瞻稱疾不起。廆敬其姿器,數臨候之,撫其心曰:「君之疾在此,不在餘也。今天子播越,四海分崩,蒼生紛擾,莫知所繫,孤思與諸君匡復帝室,翦鯨豕于二京,迎天子於吳、會,廓清八表,侔勳古烈,此孤之心也,孤之願也。君中州大族,冠冕之餘,宜痛心疾首,枕戈待旦,柰何以華夷之異,有懷介然。且大禹出于西羌,文王生于
 東夷,但問志略何如耳,豈以殊俗不可降心乎!」瞻仍辭疾篤,廆深不平之。瞻又與宋該有隙,該陰勸廆除之。瞻聞其言,彌不自安,遂以憂死。



\end{pinyinscope}