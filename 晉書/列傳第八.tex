\article{列傳第八}

\begin{pinyinscope}

 宣五王平原王榦瑯邪王伷子覲澹繇漼清惠亭侯京扶風王駿子暢歆梁王肜
 文六王



 宣帝九男,穆張皇后生景帝、文帝、平原王乾,伏夫人生汝南文成王亮、琅邪武王伷、清惠亭侯京、扶風武王駿,張夫人生梁王肜,柏夫人生趙王倫。亮及倫別有傳。



 平原王幹,字子良。少以公子魏時封安陽亭侯,稍遷撫軍中郎將,進爵平陽鄉侯。五等建,改封定陶伯。武帝踐
 阼,封平原王,邑萬一千三百戶,給鼓吹、駙馬二匹,加侍中之服。咸寧初,遣諸王之國,乾有篤疾,性理不恒,而頗清虛靜退,簡於情欲,故特詔留之。太康末,拜光祿大夫,加侍中,特假金章紫綬,班次三司。惠帝即位,進左光祿大夫,侍中如故,劍履上殿,入朝不趨。



 幹雖王大國,不事其務,有所調補,必以才能。雖有爵祿,若不在己,秩奉布帛,皆露積腐爛。陰雨則出犢車而內露車,或問其故,對曰:「露者宜內也。」朝士造之,雖通姓名,必令立車馬於門外,或終夕不見。時有得觀,與人物酬接,亦恂恂恭遜,初無闕失。前後愛妾死,既斂,輒不釘棺,置後空室中,數日
 一發視,或行淫穢,須其尸壞乃葬之。



 趙王倫輔政,以幹為衛將軍。惠帝反正,復為侍中,加太保。齊王冏之平趙王倫也,宗室朝士皆以牛酒勞冏,幹獨懷百錢,見冏乂之,曰:「趙王逆亂,汝能義舉,是汝之功,今以百錢賀汝。雖然,大勢難居,不可不慎。」冏既輔政,幹詣之,冏出迎拜。幹入,踞其床,不命冏坐,語之曰:「汝勿效白女兒,」其意指倫也。及冏誅,幹哭之慟,謂左右曰:「宗室日衰,唯此兒最可,而復害之,從今殆矣!」



 東海王越興義,至洛陽,往視乾,乾閉門不通。越駐車良久,幹乃使人謝遣,而自於門間窺之。當時莫能測其意,或謂之有疾,或以為晦迹焉。永嘉
 五年薨,時年八十。會劉聰寇洛,不遑贈謚,有二子,世子廣早卒,次子永以太熙中封安德縣公,散騎常侍,皆為善士。遇難,合門堙滅。



 瑯邪武王伷,字子將,正始初封南安亭侯。早有才望,起家為寧朔將軍,監守鄴城,有綏懷之稱。累遷散騎常侍,進封東武鄉侯,拜右將軍、監兗州諸軍事、兗州刺史。五等初建,封南皮伯。轉征虜將軍、假節,。武帝踐阼,封東莞郡王,邑萬六百戶。始置二卿,特詔諸王自選令長。伷表讓,不許。入為尚書右僕射、撫軍將軍,出為鎮東大將軍、假節、徐州諸軍事,代衛瓘鎮下邳。伷鎮御有方,得將士
 死力,吳人憚之。加開府儀同三司,改封瑯邪王,以東莞益其國。



 平吳之役,率眾數萬出塗中,孫皓奉箋送璽綬,詣伷請降,詔曰:「瑯邪王伷督率所統,連據塗中,使賊不得相救。又使瑯邪相劉弘等進軍逼江,賊震懼,遣使奉偽璽綬。又使長史王恒率諸軍渡江,破賊邊守,獲督蔡機,斬道降附五六萬計,諸葛靚、孫奕皆歸命請死,功勛茂著。其封子二人為亭侯,各三千戶,賜絹六千匹。」頃之,并督青州諸軍事,加侍中之服。進拜大將軍、開府儀同三司。



 伷既戚屬尊重,加有平吳之功,克己恭儉,無矜滿之色,僚吏盡力,百姓懷化。疾篤,賜床帳、衣服、錢帛、秔
 梁等物,遣侍中問焉。太康四年薨,時年五十七。臨終表求葬母太妃陵次,并乞分國封四子,帝許之。子恭王覲立。又封次子澹為武陵王,繇為東安王,漼為淮陵王。



 覲字思祖,拜冗從僕射。太熙元年薨,時年三十五。子睿立,是為元帝。中興初,以皇子裒為瑯邪王,奉恭王祀。裒早薨,更以皇子煥為瑯邪王。其日薨,復以皇子昱為瑯邪王。咸和之初,既徙封會稽,成帝又以康帝為瑯邪王,康帝即位,封成帝長子哀帝為瑯邪王。哀帝即位,以廢帝為瑯邪王。廢帝即位,以會稽王攝行瑯邪國祀。簡文帝登阼,瑯邪王無嗣。及帝臨崩,封少子道子為瑯邪王。道子
 後為會稽王,更以恭帝為瑯邪王。帝既即位,瑯邪國除。



 武陵莊王澹字思弘。初為冗從僕射,後封東武公,邑五千二百戶。轉前將軍、中護軍。性忌害,無孝友之行。弟東安王繇有令名,為父母所愛,澹惡之如仇,遂譖繇於汝南王亮,亮素與繇有隙,奏廢徙之。趙王倫作亂,以澹為領軍將軍。澹素與河內郭俶、俶弟侃親善。酒酣,俶等言張華之冤,澹性酗酒,因並殺之,送首于倫,其酗虐如此。



 澹妻郭氏,賈后內妹也。初恃勢,無禮於澹母。齊王冏輔政,澹母諸葛太妃表澹不孝,乞還繇,由是澹與妻子徙遼東。其子禧年五歲,不肯隨去,曰:「耍當為父求還,無為
 俱徙。」陳訴歷年,太妃薨,繇被害,然後得還。拜光祿大夫、尚書、太子太傅,改封武陵王。永嘉末為石勒所害,子哀王喆立。喆字景林,拜散騎常侍,亦為勒所害。無子,其後元帝立皇子晞為武陵王,以奉澹祀焉。



 東安王繇字思玄。初拜東安公,歷散騎黃門侍郎,遷散騎常侍。美鬚髯,性剛毅,有威望,博學多才,事親孝,居喪盡禮。誅楊駿之際,繇屯雲龍門,兼統諸軍,以功拜右衛將軍,領射聲校尉,進封郡王,邑二萬戶,加侍中,兼典軍大將軍,領右衛如故。遷尚書右僕射,加散騎常侍。是日誅賞三百餘人,皆自繇出。東夷校尉文俶父欽為繇外
 祖諸葛誕所殺,繇慮俶為舅家之患,是日亦以非罪誅俶。



 繇兄澹屢構繇於汝南王亮,亮不納。至是以繇專行誅賞,澹因隙譖之,亮惑其說,遂免繇官,以公就第,坐有悖言,廢徙帶方。永康初,徵繇,復封,拜宗正卿,遷尚書,轉左僕射。惠帝之討成都王穎,時繇遭母喪在鄴,勸穎解兵而降。及王師敗績,穎怨繇,乃害之。後立瑯邪王覲子長樂亭侯渾為東安王,以奉繇祀。尋薨,國除。



 淮陵元王漼字思沖。初封廣陵公,食邑二千九百戶。歷左將軍、散騎常侍。趙王倫之篡也,三王起義,漼與左衛將軍王輿攻殺孫秀,因而廢倫。以功進封淮陵王,入為
 尚書,加侍中,轉宗正、光祿大夫。薨,子貞王融立。薨,無子,安帝時立武陵威王孫蘊為淮陵王,以奉元王之祀,位至散騎常侍。薨,無子,以臨川王寶子安之為嗣。宋受禪,國除。



 清惠亭侯京,字子佐,魏末以公子賜爵。年二十四薨,追贈射聲校尉,以文帝子機字太玄為嗣。泰始元年,封燕王,邑六千六百六十三戶。機之國,咸寧初徵為步兵校尉,以漁陽郡益其國,加侍中之服。拜青州都督、鎮東將軍、假節,以北平、上谷、廣寧郡一萬三百三十七戶增燕國為二萬戶。薨,無子,齊王冏表以子幾嗣。後冏敗,國除。



 扶風武王駿,字子臧。幼聰惠,年五六歲能書疏,諷誦經籍,見者奇之。及長,清貞守道,宗室之中最為俊望,魏景初中,封平陽亭侯。齊王芳立,駿年八歲,為散騎常侍侍講焉。尋遷步兵、屯騎校尉,常侍如故。進爵鄉侯,出為平南將軍、假節、都督淮北諸軍事,改封平壽侯,轉安東將軍。咸熙初,徙封東牟侯,轉安東大將軍,鎮許昌。



 武帝踐阼,進封汝陰王,邑萬戶,都督豫州諸軍事。吳將丁奉寇芍陂,駿督諸軍距退之。遷使持節、都督揚州諸軍事,代石苞鎮壽春。尋復都督豫州,還鎮許昌。遷鎮西大將軍、使持節、都督雍涼等州諸軍事,代汝南王亮鎮關中,加
 袞冕侍中之服。



 駿善扶御,有威恩,勸督農桑,與士卒分役,已及僚佐并將帥兵士等人限田十畝,具以表聞。詔遣普下州縣,使各務農事。



 咸寧初,羌虜樹機能等叛,遣眾討之,斬三千餘級。進位征西大將軍。開府辟召,儀同三司,持節、都督如故。又詔駿遣七千人代涼州守兵。樹機能、侯彈勃等欲先劫佃兵,駿命平虜護軍文俶督涼、秦、雍諸軍各進屯以威之。機能乃遣所領二十部彈勃面縛軍門,各遣入質子。安定、北地、金城諸胡吉軻羅、侯金多及北虜熱冏等二十萬口又來降。其年入朝,徙封扶風王,以氐戶在國界者增封,給羽葆、鼓吹。太康初,
 進拜驃騎將軍,開府、持節、都督如故。



 駿有孝行,母伏太妃隨兄亮在官,駿常涕泣思慕,若聞有疾,輒憂懼不食,或時委官定省。少好學,能著論,與荀顗論仁孝先後,文有可稱。及齊王攸出鎮,駿表諫懇切,以帝不從,遂發病薨。追贈大司馬,加侍中、假黃鉞。西土聞其薨也,泣者盈路,百姓為之樹碑,長老見碑無不下拜,其遺愛如此。有子十人,暢、歆最知名。



 暢字玄舒。改封順陽王,拜給事中、屯騎校尉、游擊將軍。永嘉末,劉聰入洛,不知所終。



 新野莊王歆字弘舒。武王薨後,兄暢推恩請分國封歆。
 太康中,詔封新野縣公,邑千八百戶,儀比縣王。歆雖少貴,而謹身履道。母臧太妃薨,居喪過禮,以孝聞。拜散騎常侍。



 趙王倫篡位,以為南中郎將。齊王冏舉義兵,移檄天下,歆未知所從。嬖人王綏曰:「趙親而強,齊疏而弱,公宜從趙。」參軍孫洵大言於眾曰:「趙王凶逆,天下當共討之,大義滅親,古之明典。」歆從之。乃使洵詣冏,冏迎執其手曰:「使我得成大節者,新野公也。」冏入洛,歆躬貫甲胄,率所領導冏。以勳進封新野郡王,邑二萬戶。遷使持節、都督荊州諸軍事、鎮南大將軍、開府儀同三司。



 歆將之鎮,與冏同乘謁陵,因說冏曰:「成都至親,同建大勛,今宜
 留之與輔政。若不能爾,當奪其兵權。」冏不從。俄而冏敗,歆懼,自結於成都王穎。



 歆為政嚴刻,蠻夷並怨。及張昌作亂於江夏,歆表請討之。時長沙王乂執政,與成都王穎有隙,疑歆與穎連謀,不聽歆出兵,昌眾日盛。時孫洵為從事中郎,謂歆曰:「古人有言,一日縱敵,數世之患。公荷籓屏之任,居推轂之重,拜表輒行,有何不可!而使姦凶滋蔓,禍釁不測,豈維翰王室,鎮靜方夏之謂乎!」歆將出軍,王綏又曰:「昌等小賊,偏裨自足制之,不煩違帝命,親矢石也!」乃止。昌至樊城,歆出距之,眾潰,為昌所害。追贈驃騎將軍。無子,以兄子劭為後,永嘉末沒於石勒。



 梁孝王肜,字子徽,清修恭慎,無他才能,以公子封平樂亭侯。及五等建,改封開平子。武帝踐阼,封梁王,邑五千三百五十八戶。及之國,遷北中郎將,督鄴城守事。



 時諸王自選官屬,肜以汝陰上計吏張蕃為中大夫。蕃素無行,本名雄,妻劉氏解音樂,為曹爽教伎,蕃又往來何晏所,而恣為姦淫。晏誅,徙河間,乃變名自結於肜。為有司所奏,詔削一縣。咸寧中,復以陳國、汝南南頓增封為次國。太康中,代孔洵監豫州軍事,加平東將軍,鎮許昌。頃之,又以本官代下邳王晃監青徐州軍事,進號安東將軍。



 元康初,轉征西將軍,代秦王柬都督關中軍事,領護
 西戎校尉。加侍中,進督梁州。尋徵為衛將軍、錄尚書事,行太子太保,給千兵百騎。久之,復為征西大將軍,代趙王倫鎮關中,都督涼、雍諸軍事,置左右長史、司馬。又領西戎校尉,屯好畤,督建威將軍周處、振威將軍盧播等伐氐賊齊萬年於六陌。肜與處有隙,促令進軍而絕其後,播又不救之,故處見害。朝廷尤之。尋徵拜大將軍、尚書令、領軍將軍、錄尚書事。



 肜嘗大會,謂參軍王銓曰:「我從兄為尚書令,不能啖大臠。大臠故難。」銓曰:「公在此獨嚼,尚難矣。」肜曰:「長史大臠為誰?」曰:「盧播是也。」肜曰:「是家吏,隱之耳。」銓曰:「天下咸是家吏,便恐王法不可復行。」肜
 又曰:「我在長安,作何等不善!」因指單衣補幰以為清。銓答曰:「朝野望公舉薦賢才,使不仁者遠。而位居公輔,以衣補幰,以此為清,無足稱也。」肜有慚色。



 永康初,共趙王倫廢賈后,詔以肜為太宰、守尚書令,增封二萬戶。趙王倫輔政,有星變,占曰「不利上相。」孫秀懼倫受災,乃省司徒為丞相,以授肜,猥加崇進,欲以應之。或曰:「肜無權,不益也。」肜固讓不受。及倫篡位,以肜為阿衡,給武賁百人,軒懸之樂十人。倫滅,詔以肜為太宰,領司徒,又代高密王泰為宗師。



 永康二年薨,喪葬依汝南文成王亮故事。博士陳留蔡克議謚曰:「肜位為宰相,責深任重,屬尊親
 近,且為宗師,朝所仰望,下所具瞻。而臨大節,無不可奪之志;當危事,不能舍生取義;愍懷之廢,不聞一言之諫;淮南之難,不能因勢輔義;趙王倫篡逆,不能引身去朝。宋有蕩氏之亂,華元自以不能居官,曰「君臣之訓,我所司也。公室卑而不正,吾罪大矣!」夫以區區之宋,猶有不素餐之臣,而況帝王之朝,而有茍容之相,此而不貶,法將何施!謹案《謚法》『不勤成名曰靈」,肜見義不為,不可謂勤,宜謚曰靈。」梁國常侍孫霖及肜親黨稱枉,臺乃下符曰:「賈氏專權,趙王倫篡逆,皆力制朝野,肜勢不得去,而責其不能引身去朝,義何所據?」克重議曰:「肜為宗臣,而
 國亂不能匡,主顛不能扶,非所以為相。故《春秋》譏華元樂舉,謂之不臣。且賈氏之酷烈,不甚於呂后,而王陵猶得杜門;趙王倫之無道,不甚於殷紂,而微子猶得去之。近者太尉陳準,異姓之人,加弟徽有射鉤之隙,亦得託疾辭位,不涉偽朝。何至於肜親倫之兄,而獨不得去乎?趙盾入諫不從,出亡不遠,猶不免於責,況肜不能去位,北面事偽主乎?宜如前議,加其貶責,以廣為臣之節,明事君之道。」於是朝廷從克議。肜故吏復追訴不已,故改焉。



 無子,以武陵王澹子禧為後,是為懷王,拜征虜將軍,與澹俱沒於石勒。元帝時,以西陽王羕子悝為肜嗣,早薨,
 是為殤王。至是懷王子翹自石氏歸國得立,是為聲王,官至散騎常侍。薨,無子,詔以武陵威王子逢為翹嗣,歷永安太僕,與父晞俱廢徙新安。薨,太元中復國,子和立。薨,子珍之立。桓玄篡位,國臣孔璞奉珍之奔于壽陽,義熙初乃歸,累遷左衛將軍、太常卿。劉裕伐姚泓,請為諮議參軍,為裕所害。國除。



 文帝九男,文明王皇后生武帝、齊獻王攸、城陽哀王兆、遼東悼惠王定國、廣漢殤王廣德,其樂安平王鑒、燕王機、皇子永祚、樂平王延祚不知母氏。燕王機繼清惠亭
 侯,別有傳。永祚早亡,無傳。



 齊獻王攸,字大猷,少而岐嶷。及長,清和平允,親賢好施,愛經籍,能屬文,善尺牘,為世所楷。才望出武帝之右,宣帝每器之。景帝無子,命攸為嗣。從征王凌,封長樂亭侯。及景帝崩,攸年十歲,哀動左右,大見稱嘆。襲封舞陽侯。奉景獻羊后於別第,事后以孝聞。復歷散騎常侍、步兵校尉,時年十八,綏撫營部,甚有威惠。五等建,改封安昌侯,遷衛將軍。居文帝喪,哀毀過禮,杖而後起。左右以稻米乾飯雜理中丸進之,攸泣而不受。太后自往勉喻曰:「若萬一加以他疾,將復如何!宜遠慮深計,不可專守一
 志。」常遣人逼進飲食,司馬嵇喜又諫曰:「毀不滅性,聖人之教。且大王地即密親,任惟元輔。匹夫猶惜其命,以為祖宗,況荷天下之大業,輔帝室之重任,而可盡無極之哀,與顏閔爭孝!不可令賢人笑,愚人幸也。」喜躬自進食,攸不得已,為之強飯。喜退,攸謂左右曰:「嵇司馬將令我不忘居喪之節,得存區區之身耳。」



 武帝踐阼,封齊王,時朝廷草創,而攸總統軍事,撫寧內外,莫不景附焉。詔議籓王令自選國內長吏,攸奏議曰;「昔聖王封建萬國,以親諸侯,軌跡相承,莫之能改。誠以君不世居,則人心偷幸;人無常主,則風俗偽薄。是以先帝深覽經遠之統,思
 復先哲之軌,分土畫疆,建爵五等,或以進德,或以酬功。伏惟陛下應期創業,樹建親戚,聽使籓國自除長吏。而今草創,制度初立,雖庸蜀順軌,吳猶未賓,宜俟清泰,乃議復古之制。」書比三上,輒報不許。其後國相上長吏缺,典書令請求差選。攸下令曰:「忝受恩禮,不稱惟憂。至於官人敘才,皆朝廷之事,非國所宜裁也。其令自上請之。」時王家人衣食皆出御府,攸表租秩足以自供,求絕之。前後十餘上,帝又不許。攸雖未之國,文武官屬,下至士卒,分租賦以給之,疾病死喪賜與之。而時有水旱,國內百姓則加振貸,須豐年乃責,十減其二,國內賴之。



 遷驃
 騎將軍,開府辟召,禮同三司。降身虛己,待物以信。常嘆公府不案吏,然以董御戎政,復有威克之宜,乃下教曰:「夫先王馭世,明罰敕法,鞭撲作教,以正逋慢。且唐虞之朝,猶須督責。前欲撰次其事,使粗有常。懼煩簡之宜,未審其要,故令劉、程二君詳定。然思惟之,鄭鑄刑書,叔向不韙;范宣議制,仲尼譏之。令皆如舊,無所增損。其常節度所不及者,隨事處決。諸吏各竭乃心,思同在公古人之節。如有所闕,以賴股肱匡救之規,庶以免負。」於是內外祗肅。時驃騎當罷營兵,兵士數千人戀攸恩德,不肯去,遮京兆主言之,帝乃還攸兵。



 攸每朝政大議,悉心陳
 之。詔以比年饑饉,議所節省,攸奏議曰:「臣聞先王之教,莫不先正其本。務農重本,國之大綱。當今方隅清穆,武夫釋甲,廣分休假,以就農業。然守相不能勤心恤公,以盡地利。昔漢宣嘆曰:『與朕理天下者,惟良二千石乎!』勤加賞罰,黜陟幽明,于時翕然,用多名守。計今地有餘羨,而不農者眾,加附業之人復有虛假,通天下謀之,則飢者必不少矣。今宜嚴敕州郡,檢諸虛詐害農之事,督實南畝,上下同奉所務。則天下之穀可復古政,豈患於暫一水旱,便憂饑餒哉!考績黜陟,畢使嚴明,畏威懷惠,莫不自厲。又都邑之內,游食滋多,巧伎末業,服飾奢麗,富
 人兼美,猶有魏之遺弊,染化日淺,靡財害穀,動復萬計。宜申明舊法,必禁絕之。使去奢即儉,不奪農時,畢力稼穡,以實倉廩。則榮辱禮節,由之而生,興化反本,於茲為盛。」



 轉鎮軍大將軍,加侍中,羽葆、鼓吹,行太子少傅。數年,授太子太傅,獻箴於太子曰:「伊昔上皇,建國立君,仰觀天文,俯察地理,創業恢道,以安人承祀,祚延統重,故援立太子。尊以弘道,固以貳己,儲德既立,邦有所恃。夫親仁者功成,邇佞者國傾,故保相之材,必擇賢明。昔在周成,旦奭作傅,外以明德自輔,內以親親立固,德以義濟,親則自然。嬴廢公族,其崩如山;劉建子弟,漢祚永傅。楚
 以無極作亂,宋以伊戾興難。張禹佞給,卒危彊漢。輔弼不忠,禍及乃躬;匪徒乃躬,乃喪乃邦。無曰父子不間,昔有江充;無曰至親匪貳,或容潘崇。諛言亂真,譖潤離親,驪姬之讒。晉侯疑申。固親以道,勿固以恩;脩身以敬,勿託以尊。自損者有餘,自益者彌昏。庶事不可以不恤,大本不可以不敦。見亡戒危,睹安思存。冢子司義,敢告在閽。」世以為工。



 咸寧二年,代賈充為司空,侍中、太傅如故。初,攸特為文帝所寵愛,每見攸,輒撫床呼其小字曰「此桃符座也」,幾為太子者數矣。及帝寢疾,慮攸不安,為武帝敘漢淮南王、魏陳思故事而泣。臨崩,執攸手以授帝。
 先是太后有疾,既瘳,帝與攸奉觴上壽,攸以太后前疾危篤,因歔欷流涕,帝有愧焉。攸嘗侍帝疾,恒有憂戚之容,時人以此稱嘆之。及太后臨崩,亦流涕謂帝曰:「桃符性急,而汝為兄不慈,我若遂不起,恐必不能相容。以是屬汝,勿忘我言。」



 及帝晚年,諸子並弱,而太子不令,朝臣內外,皆屬意於攸。中書監荀勖、侍中馮紞皆諂諛自進,攸素疾之。勖等以朝望在攸,恐其為嗣,禍必及己,乃從容言於帝曰:「陛下萬歲之後,太子不得立也。」帝曰:「何故?」勖曰:「百僚內外皆歸心於齊王,太子焉得立乎!陛下試詔齊王之國,必舉朝以為不可,則臣言有徵矣。」紞又言
 曰:「陛下遣諸侯之國,成五等之制者,宜先從親始。親莫若齊王。」帝既信勖言,又納紞說,太康三年乃下詔曰:「古者九命作伯,或入毗朝政,或出御方嶽。周之呂望,五侯九伯,實得征之,侍中、司空、齊王攸,明德清暢,忠允篤誠。以母弟之親,受台輔之任,佐命立勳,劬勞王室,宜登顯位,以稱具瞻。其以為大司馬、都督青州諸軍事,侍中如故,假節,將本營千人,親騎帳下司馬大車皆如舊,增鼓吹一部,官騎滿二十人,置騎司馬五人。餘主者詳案舊制施行。」攸不悅,主簿丁頤曰:「昔太公封齊,猶表東海;桓公九合,以長五伯。況殿下誕德欽明,恢弼大籓,穆然東
 軫,莫不得所。何必絳闕,乃弘帝載!」攸曰:「吾無匡時之用,卿言何多。」



 明年,策攸曰:「於戲!惟命不于常,天既遷有魏之祚。我有晉既受順天明命,光建群后,越造王國于東土,錫茲青社,用籓翼我邦家。茂哉無怠,以永保宗廟。」又詔下太常,議崇錫之物,以濟南郡益齊國。又以攸子寔為北海王。於是備物典策,設軒懸之樂、六佾之舞,黃鉞朝車乘輿之副從焉。



 攸知勖、紞構己,憤怨發疾,乞守先后陵,不許。帝遣御醫診視,諸醫希旨,皆言無疾。疾轉篤,猶催上道。攸自彊入辭,素持容儀,疾雖困,尚自整厲,舉止如常,帝益疑無疾。辭出信宿,歐血而薨,時年三十六。
 帝哭之慟,馮紞侍側曰:「齊王名過其實,而天下歸之。今自薨隕,社稷之福也,陛下何哀之過!」帝收淚而止。詔喪禮依安平王孚故事,廟設軒懸之樂,配饗太廟。子冏立,別有傳。



 攸以禮自拘,鮮有過事。就人借書,必手刊其謬,然後反之。加以至性過人,有觸其諱者,輒泫然流涕。雖武帝亦敬憚之,每引之同處,必擇言而後發。三子:蕤、贊、寔。



 蕤字景回,出繼遼東王定國。太康初,徙封東萊王。元康中,歷步兵、屯騎校尉。蕤性彊暴,使酒,數陵侮弟冏,冏以兄故容之。冏起義兵,趙王倫收蕤及弟北海王寔繫廷
 尉,當誅。倫太子中庶子祖納上疏諫曰:「罪不相反,惡止其身,此先哲之弘謨,百王之達制也。是故鯀既殛死,禹乃嗣興;二叔誅放,而邢衛無責。逮乎戰國,及至秦漢,明恕之道寢,猜嫌之情用,乃立質任以御眾,設從罪以發姦,其所由來,蓋三代之弊法耳。蕤、寔,獻王之子,明德之胤,宜蒙特宥,以全穆親之典。」會孫秀死,蕤等悉得免。冏擁眾入洛,蕤於路迎之。冏不即見,須符付前頓。蕤恚曰:「吾坐爾殆死,曾無友于之情!」



 及冏輔政,詔以蕤為散騎常侍,加大將軍,領後軍、侍中、特進,增邑滿二萬戶。又從冏求開府,冏曰:「武帝子吳、豫章尚未開府,宜且須後。」蕤
 以是益怨,密表冏專權,與左衛將軍王輿謀共廢冏。事覺,免為庶人。尋詔曰:「大司馬以經識明斷,高謀遠略,猥率同盟,安復社稷。自書契所載,周召之美未足比勛,故授公上宰。東萊王蕤潛懷怨妒,包藏禍心,與王輿密謀,圖欲譖害。收輿之日,蕤與青衣共載,微服奔走,經宿乃還。姦凶赫然,妖惑外內。又前表冏所言深重,雖管蔡失道,牙慶亂宗,不復過也。《春秋》之典,大義滅親,其徙蕤上庸。」後封微陽侯。永寧初,上庸內史陳鍾承冏旨害蕤。死,詔誅鍾,復蕤封,改葬以王禮。



 贊字景期,繼廣漢殤王廣德後。年六歲,太康元年薨,謚
 沖王。



 寔字景深,初為長樂亭侯。攸以贊薨,又以寔繼廣漢殤王後,改封北海王。永寧初為平東將軍、假節,加散騎常侍,代齊王冏鎮許昌。尋進安南將軍,都督豫州軍事,增邑滿二萬戶。未發,留為侍中、上軍將軍,給千兵百騎。



 城陽哀王兆,字千秋,年十歲而夭。武帝踐阼,詔曰:「亡弟千秋,少聰慧,有夙成之質,不幸早亡,先帝先后特所哀愍。先后欲紹立其後,而竟未遂,每追遺意,情懷感傷。其以皇子景度為千秋後,雖非典禮,亦近世之所行,且以述先后本旨也。」於是追加兆封謚。景度以泰始六年薨,
 復以第五子憲繼哀王後。薨,復以第六子祗為東海王,繼哀王後。薨,咸寧初又封第十三子遐為清河王,以繼兆後。



 遼東悼惠王定國,年三歲薨。咸寧初追加封謚,齊王攸以長子蕤為嗣。蕤薨,子遵嗣。



 廣漢殤王廣德,年二歲薨。咸寧初追加封謚,齊王攸以第五子贊紹封。薨,攸更以第二子寔嗣廣德。



 樂安平王鑒,字大明,初封臨泗亭侯。武帝踐阼,封樂安王。帝為鑒及燕王機高選師友,下詔曰:「樂安王鑒、燕王機並以長大,宜得輔導師友,取明經儒學,有行義節儉,
 使足嚴憚。昔韓起與田蘇游而好善,宜必得其人。」泰始中,拜越騎校尉。咸寧初,以齊之梁鄒益封,因之國,服侍中之服。元康初,徵為散騎常侍、上軍大將軍,領射聲校尉。尋遷使持節、都督豫州軍事、安南將軍,代清河王遐鎮許昌,以疾不行。七年薨,子殤王籍立。薨,無子,齊王冏以子冰紹鑒後。以濟陰萬一千二百一十九戶改為廣陽國,立冰為廣陽王。冏敗,廢。



 樂平王延祚,字大思,少有篤疾,不任封爵。太康初,詔曰:「弟祚早孤無識,情所哀愍。幼得篤疾,日冀其差,今遂廢痼,無復後望,意甚傷之。其封為樂平王,使有名號,以慰
 吾心。」尋薨,無子。



 史臣曰:平原性理不恒,世莫之測。及其處亂離之際,屬交爭之秋,而能遠害全身,享茲介福,其愚不可及已!瑯邪武功既暢,飾之以溫恭,扶風文教克宣,加之以孝行,抑宗室之可稱者也。齊王以兩獻之親,弘二南之化,道光雅俗,望重台衡,百辟具瞻,萬方屬意。既而地疑致逼,文雅見疵,紞勖陳蔓草之邪謀,武皇深翼子之滯愛。遂乃褫龍章於袞職,徙侯服於下籓,未及戒塗,終於憤恚,惜哉!若使天假之年而除其害,奉綴衣之命,膺負圖之託,光輔嗣君,允釐邦政,求諸冥兆,或廢興之有期,徵之
 人事,庶勝殘之可及,何八王之敢力爭,五胡之能競逐哉!《詩》云「人之云亡,邦國殄瘁,」攸實有之;「讒人罔極,交亂四國,」其荀馮之謂也。



 贊曰:文宣孫子,或賢或鄙。扶風遺愛,瑯邪克己。澹諂兇魁,肜參釁始。乾雖靜退,性乖恆理。彼美齊獻,卓爾不群。自家刑國,緯武經文。木摧於秀,蘭燒以薰。



\end{pinyinscope}