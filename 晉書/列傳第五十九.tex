\article{列傳第五十九}

\begin{pinyinscope}

 忠義



 嵇紹從子含王豹劉沉麴允焦嵩賈渾王育韋忠辛勉劉敏元周該桓雄
 韓階周崎易雄樂道融
 虞悝沈勁吉挹王諒宋矩車濟丁穆辛恭靖羅企生張禕



 古人有言:「君子殺身以成仁,不求生以害仁。」又云:「非死之難,處死之難。」信哉斯言也!是知隕節茍合其宜,義夫豈吝其沒;捐軀若得其所,烈士不愛其存。故能守鐵石之深衷,厲松筠之雅操,見貞心於歲暮,標勁節於嚴風,赴鼎鑊其如歸,履危亡而不顧,書名竹帛,畫象丹青,前史以為美談,後來仰其徽烈者也。



 晉自元康之後,政亂
 朝昏,禍難薦興,艱虞孔熾,遂使奸凶放命,戎狄交侵,函夏沸騰,蒼生塗炭,干戈日用,戰爭方興。雖背恩忘義之徒不可勝載,而蹈節輕生之士無乏於時。至若嵇紹之衛難乘輿,卡壼之亡軀鋒鏑,桓雄之義高田叔,周崎之節邁解揚,羅丁致命於舊君,辛吉恥臣於戎虜,張禕引鴆以全節,王諒斷臂以厲忠,莫不志烈秋霜,精貫白日,足以激清風於萬古,厲薄俗於當年者歟!所謂亂世識忠臣,斯之謂也。卡壼、劉超、鐘雅、周虓等已入列傳,其餘即敘其行事以為《忠義傳》,用旌晉氏之有人焉。



 嵇紹,字延祖,魏中散大夫康之子也。十歲而孤,事母孝謹。以父得罪,靖居私門。山濤領選,啟武帝曰:「《康誥》有言:『父子罪不相及。』嵇紹賢侔郤缺,宜加旌命,請為秘書郎。」帝謂濤曰:「如卿所言,乃堪為丞,何但郎也。」乃發詔征之,起家為秘書丞。



 紹始入洛,或謂王戎曰:「昨於稠人中始見嵇紹,昂昂然如野鶴之在雞群。」戎曰:「君復未見其父耳。」累遷汝陰太守。尚書左僕射裴頠亦深器之,每曰:「使延祖為吏部尚書,可使天下無復遺才矣。」沛國戴晞少有才智,與紹從子含相友善,時人許以遠致,紹以為必不成器。晞後為司州主簿,以無行被斥,州黨稱紹有知
 人之明。轉豫章內史,以母憂,不之官。服闋,拜徐州刺史。時石崇為都督,性雖驕暴,而紹將之以道,崇甚親敬之。後以長子喪去職。



 元康初,為給事黃門侍郎。時侍中賈謐以外戚之寵,年少居位,潘岳、杜斌等皆附託焉。謐求交於紹,紹距而不答。及謐誅,紹時在省,以不阿比凶族,封弋陽子,遷散騎常侍,領國子博士。太尉、廣陵公陳準薨,太常奏謚,紹駁曰:「謚號所以垂之不朽,大行受大名,細行受細名,文武顯於功德,靈厲表於闇蔽。自頃禮官協情,謚不依本。準謚為過,宜謚曰繆。」事下太常。時雖不從,朝廷憚焉。



 趙王倫篡位,署為侍中。惠帝復阼,遂居其
 職。司空張華為倫所誅,議者追理其事,欲復其爵,紹又駁曰:「臣之事君,當除煩去惑。華歷位內外,雖粗有善事,然闔棺之責,著于遠近,兆禍始亂,華實為之。故鄭討幽公之亂,斲子家之棺;魯戮隱罪,終篇貶翬。未忍重戮,事已弘矣,謂不宜復其爵位,理其無罪。」時帝初反正,紹又上疏曰:「臣聞改前轍者則車不傾,革往弊者則政不爽。太一統於元首,百司役於多士,故周文興於上,成康穆於下也。存不忘亡,《易》之善義;願陛下無忘金墉,大司馬無忘潁上,大將軍無忘黃橋,則禍亂之萌無由而兆矣。」



 齊王冏既輔政,大興第舍,驕奢滋甚,紹以書諫曰:「夏
 禹以卑室稱美,唐虞以茅茨顯德,豐屋蔀家,無益危亡。竊承毀敗太樂以廣第舍,興造功力為三王立宅,此豈今日之先急哉!今大事始定,萬姓顒,咸待覆潤,宜省起造之煩,深思謙損之理。復主之勳不可棄矣,矢石之殆不可忘也。」冏雖謙順以報之,而卒不能用。紹嘗詣炯諮事,遇炯宴會,召董艾、葛旗等共論時政。艾言於炯曰:「嵇侍中善於絲竹,公可令操之。」左右進琴,紹推不受。冏曰:「今日為懽,卿何吝此邪!」紹對曰:「公匡復社稷,當軌物作則,垂之於後。紹雖虛鄙,忝備常伯,腰紱冠冕,鳴玉殿省,豈可操執絲竹,以為伶人之事!若釋公服從私宴,所
 不敢辭也。」冏大慚。艾等不自得而退。頃之,以公事免,冏以為左司馬。旬日,冏被誅。初,兵交,紹奔散赴宮,有持弩在東閣下者,將射之,遇有殿中將兵蕭隆,見紹姿容長者,疑非凡人,趣前拔箭,於此得免。遂還滎陽舊宅。



 尋徵為御史中丞,未拜,復為侍中。河間王顒、成都王穎舉兵向京都,以討長沙王乂,大駕次于城東。乂言於眾曰:「今日西討,欲誰為都督乎?」六軍之士皆曰:「願嵇侍中戮力前驅,死猶生也。」遂拜紹使持節、平西將軍。屬乂被執,紹復為侍中。公王以下皆詣鄴謝罪於穎,紹等咸見廢黜,免為庶人。尋而朝廷復有北征之役,徵紹,復其爵位。
 紹以天子蒙塵,承詔馳詣行在所。值王師敗績於蕩陰,百官及侍衛莫不散潰,唯紹儼然端冕,以身捍衛,兵交御輦,飛箭雨集,紹遂被害於帝側,血濺御服,天子深哀歎之。及事定,左右欲浣衣,帝曰:「此嵇侍中血,勿去。」



 初,紹之行也,侍中秦準謂曰:「今日向難,卿有佳馬否?」紹正色曰:「大駕親征,以正伐逆,理必有征無戰。若使皇輿失守,臣節有在,駿馬何為!」聞者莫不歎息。及張方逼帝遷長安,河間王顒表贈紹司空,進爵為公。會帝還洛陽,事遂未行。東海王越屯許,路經滎陽,過紹墓,哭之悲慟,刊石立碑,又表贈官爵。帝乃遣使冊贈侍中、光祿大夫,加金
 章紫綬,進爵為侯,賜墓田一頃,客十戶,祠以少牢。元帝為左丞相,承制,以紹死節事重,而贈禮未副勳德,更表贈太尉,祠以太牢。及帝即位,賜謚曰忠穆,復加太牢之祠。



 紹誕於行己,不飾小節,然曠而有檢,通而不雜。與從子含等五人共居,撫恤如所同生。門人故吏思慕遺愛,行服墓次,畢三年者三十餘人。長子,有父風,早夭。以從孫翰襲封。成帝時追述紹忠,以翰為奉朝請。翰以無兄弟,自表還本宗。太元中,孝武帝詔曰:「褒德顯仁,哲王令典。故太尉、忠穆公執德高邈,在否彌宣,貞潔之風,義著千載。每念其事,愴然傷懷。忠貞之胤,蒸嘗宜遠,所以
 大明至節,崇獎名教。可訪其宗族,襲爵主祀。」於是復以翰孫曠為弋陽侯。



 含字君道。祖喜,徐州刺史。父蕃,太子舍人。含好學能屬文。家在鞏縣亳丘,自號亳丘子,門曰歸厚之門,室曰慎終之室。楚王瑋辟為掾。瑋誅,坐免。舉秀才,除郎中。時弘農王粹以貴公子尚主,館宇甚盛,圖莊周於室,廣集朝士,使含為之贊。含援筆為弔文,文不加點。其序曰:「帝婿王弘遠華池豐屋,廣延賢彥,圖莊生垂綸之象,記先達辭聘之事,畫真人於刻桷之室,載退士於進趣之堂,可謂託非其所,可弔不可贊也。」其辭曰:「邁矣莊周,天縱特
 放,大塊授其生,自然資其量,器虛神清,窮玄極曠。人偽俗季,真風既散,野無訟屈之聲,朝有爭寵之歎,上下相陵,長幼失貫,於是借玄虛以助溺,引道德以自獎,戶詠恬曠之辭,家畫老莊之象。今王生沈淪名利,身尚帝女,連耀三光,有出無處,池非巖石之溜,宅非茅茨之宇,馳屈產於皇衢,畫茲象其焉取!嗟乎先生,高跡何局!生處巖岫之居,死寄彫楹之屋,託非其所,沒有餘辱,悼大道之湮晦,遂含悲而吐曲。」粹有愧色。



 齊王冏辟為征西參軍,襲爵武昌鄉侯。長沙王乂召為驃騎記室督、尚書郎。乂與成都王穎交戰,穎軍轉盛,尚書郎旦出督戰,夜還
 理事。含言於乂曰:「昔魏武每有軍事,增置掾屬。青龍二年,尚書令陳矯以有軍務,亦奏增郎。今奸逆四逼,王路擁塞,倒懸之急,不復過此。但居曹理事,尚須增郎,況今都官中騎三曹晝出督戰,夜還理事,一人兩役,內外廢乏。含謂今有十萬人,都督各有主帥,推轂授綏,委付大將,不宜復令臺僚雜與其間。」乂從之,乃增郎及令史。



 懷帝為撫軍將軍,以含為從事中郎。惠帝北征,轉中書侍郎。及蕩陰之敗,含走歸滎陽。永興初,除太弟中庶子。西道阻閡,未得應召。范陽王虓為征南將軍,屯許昌,復以含為從事中郎。尋授振威將軍、襄城太守。虓為劉喬所
 破,含奔鎮南將軍劉弘於襄陽,弘待以上賓之禮。含性通敏,好薦達才賢,常欲崇趙武之謚,加臧文之罪。屬陳敏作亂,江揚震蕩,南越險遠,而廣州刺史王毅病卒,弘表含為平越中郎將、廣州刺史、假節。未發,會弘卒,時或欲留含領刑州。含性剛躁,素與弘司馬郭勱有隙,勱疑含將為己害,夜掩殺之,時年四十四。懷帝即位,謚曰憲。



 王豹,順陽人也。少而抗直。初為豫州別駕,齊王冏為大司馬,以豹為主簿。冏驕縱,失天下心,豹致箋於冏日:



 豹聞王臣蹇蹇,匪躬之故,將以安主定時,保存社稷者也。
 是以為人臣而欺其君者,刑罰不足以為誅;為人主而逆其諫者,靈厲不足以為謚。伏惟明公虛心下士,開懷納善,款誠以著,而逆耳之言未入於聽。豹伏思晉政漸缺,始自元康以來,宰相在位,未有一人獲終,乃事勢使然,未為輒有不善也。今公剋平禍亂,安國定家,故復因前傾敗之法,尋中間覆車之軌,欲冀長存,非所敢聞。今河間樹根於關右,成都盤桓於舊魏,新野大封於江漢,三面貴王,各以方剛強盛,並典戎馬,處險害之地。且明公興義討逆,功蓋天下,聖德光茂,名震當世。今以難賞之功,挾震主之威,獨據京都,專執大權,進則亢龍有悔,
 退則蒺藜生庭,冀此求安,未知其福。敢以淺見,陳寫愚情。



 昔武王伐紂,封建諸侯為二伯,自陜以東,周公主之,自陜以西,召公主之。及至其末,霸國之世,不過數州之地,四海強兵不敢入窺九鼎,所以然者,天下習於所奉故也。今誠能尊用周法,以成都為北州伯,統河北之王侯,明公為南州伯,以攝南土之官長,各因本職,出居其方,樹德於外,盡忠於內,歲終率所領而貢於朝,簡良才,命賢俊,以為天子百官,則四海長寧,萬國幸甚,明公之德當與周召同其至美,危敗路塞,社稷可保。顧明公思高祖納婁敬之策,悟張良履足之謀,遠臨深之危,保泰
 山之安。若合聖思,宛許可都也。



 書入,無報,豹重箋曰:



 豹書御已來,十有二日,而聖旨高遠,未垂採察,不賜一字之令,不敕可否之宜。蓋霸王之神寶,安危之秘術,不可須臾而忽者也。伏思明公挾大功,抱大名,懷大德,執大權,此四大者,域中所不能容,賢聖所以戰戰兢兢,日昃不暇食,雖休勿休者也。昔周公以武王為兄,成王為君,伐紂有功,以親輔政,執德弘深,聖思博遠,至忠至仁,至孝至敬。而攝事之日,四國流言,離主出奔,居東三年,賴風雨之變,成王感悟。若不遭皇天之應,神人之察,恐公旦之禍未知所限也。至於執政,猶與召公分陜為伯。今
 明公自視功德孰如周公。且元康以來,宰相之患,危機竊發,不及容思,密禍潛起,輒在呼噏,豈復晏然得全生計!前鑒不遠,公所親見也。君子不有遠慮,必有近憂,憂至乃悟,悔無所及也。



 今若從豹此策,皆遣王侯之國,北與成都分河為伯,成都在鄴,明公都宛,寬方千里,以與圻內侯伯子男小大相率,結好要盟,同獎皇家;貢御之法,一如周典。若合聖規,可先旨與成都共論。雖以小才,願備行人。昔廝養,燕趙之微者耳,百里奚,秦楚之商人也,一開其說,兩國以寧。況豹雖陋,大州之綱紀,加明公起事險難之主簿也。故身雖輕,其言未必否也。



 冏令曰:「
 得前後白事,具意,輒別思量也。」會長沙王乂至,於冏案上見豹箋,謂冏曰:「小子離間骨肉,何不銅駝下打殺!」冏既不能嘉豹之策,遂納乂言,乃奏豹曰:「臣忿奸凶肆逆,皇祚顛墜,與成都、長沙、新野共興義兵,安復社稷,唯欲戮力皇家,與親親宗室腹心從事,此臣夙夜自誓,無負神明。而主簿王豹比有白事,敢造異端,謂臣忝備宰相,必遘危害,慮在一旦,不祥之聲可蹻足而待,欲臣與成都分陜為伯,盡出籓王。上誣聖朝鑒御之威,下長妖惑,疑阻眾心,噂沓背憎,巧賣兩端,訕上謗下,讒內間外,遘惡導奸,坐生猜嫌。昔孔丘匡魯,乃誅少正;子產相鄭,先
 戮鄧析,誠以交亂名實,若趙高詭怪之類也。豹為臣不忠不順不義,輒敕都街考竟,以明邪正。」豹將死,曰:「懸吾頭大司馬門,見兵之攻齊也。」眾庶冤之。俄而冏敗。



 劉沈,字道真,燕國薊人也。世為北州名族。少仕州郡,博學好古。太保衛瓘辟為掾,領本邑大中正。敦儒道,愛賢能,進霍原為二品,及申理張華,皆辭旨明峻,為當時所稱。齊王冏輔政,引為左長史,遷侍中。于時李流亂蜀,詔沈以侍中、假節,統益州刺史羅尚、梁州刺史許雄等以討流。行次長安,河間王顒請留沉為軍司,遣席薳代之。
 後領雍州刺史。及張昌作亂,詔顒遣沉將州兵萬人並征西府五千人,自藍田關以討之,顒不奉詔。沉自領州兵至藍田,顒又逼奪其眾。長沙王乂命沉將武吏四百人還州。



 張方既逼京都,王師屢敗,王瑚、祖逖言於乂曰:「劉沈忠義果毅,雍州兵力足制河間,宜啟上詔與沈,使發兵襲顒,顒窘急,必召張方以自救,此計之良也。」乂從之。沈奉詔馳檄四境,合七郡之眾及守防諸軍、塢壁甲士萬餘人,以安定太守衛博、新平太守張光、安定功曹皇甫澹為先登,襲長安。顒時頓于鄭縣之高平亭,為東軍聲援,聞沈兵起,還鎮渭城,遣督護虞夔率步騎萬餘人
 逆沈於好畤。接戰,夔眾敗,顒大懼,退入長安,果急呼張方。沈渡渭而壘,顒每遣兵出鬥,輒不利,沈乘勝攻之,使澹、博以精甲五千,從長安門而入,力戰至顒帳下。沈軍來遲,顒軍見澹等無繼,氣益倍。馮翊太守張輔率眾救顒,橫擊之,大戰於府門,博父子皆死之,澹又被擒。顒奇澹壯勇,將活之。澹不為之屈,於是見殺。沈軍遂敗,率餘卒屯于故營。張方遣其將敦偉夜至,沈軍大驚而潰,與麾下百餘人南遁,為陳倉令所執。沈謂顒曰:「夫知己之顧輕,在三之節重,不可違君父之詔,量強弱以茍全。投袂之日,期之必死,菹醢之戮,甘之如薺。」辭義慷慨,見者
 哀之。顒怒,鞭之而後腰斬。有識者以顒干上犯順,虐害忠義,知其滅亡不久也。



 麴允,金城人也。與游氏世為豪族,西州為之語曰:「麴與游,牛羊不數頭。南開朱門,北望青樓。」洛陽傾覆,閻鼎等立秦王為皇太子於長安,鼎總攝百揆。允時為安夷護軍、始平太守,心害鼎功,且規權勢,因鼎殺京兆太守梁綜,乃與綜弟馮翊太守緯等攻鼎,走之。會雍州刺史賈疋為屠各所殺,允代其任。愍帝即尊位,以允為尚書左僕射、領軍、持節、西戎校尉、錄尚書事,雍州如故。時劉曜、
 殷凱、趙染數萬眾逼長安,允擊破之,擒凱於陣。曜復攻北地,允為太都督、驃騎將軍,次于青白城以救之。曜聞而轉寇上郡,允軍于靈武,以兵弱不敢進。曜後復圍北地,太守麴昌遣使求救,允率步騎赴之。去城數十里,群賊繞城放火,煙塵蔽天,縱反間詐允曰:「郡城已陷,焚燒向盡,無及矣。」允信之,眾懼而潰。後數日,麴昌突圍赴長安,北地遂陷。



 允性仁厚,無威斷,吳皮、王隱之徒,無賴凶人,皆加重爵,新平太守竺恢,始平太守楊像、扶風太守竺爽、安定太守焦嵩,皆征鎮杖節,加侍中、常侍,村塢主帥小者,猶假銀青、將軍之號,欲以撫結眾心。然諸將驕
 恣,恩不及下,人情頗離,由是羌胡因此跋扈,關中淆亂,劉曜復攻長安,百姓飢甚,死者太半。久之,城中窘逼,帝將出降,歎曰:「誤我事者,麴、索二公也。」帝至平陽,為劉聰所幽辱,允伏地號哭不能起。聰大怒,幽之於獄,允發憤自殺。聰嘉其忠烈,贈車騎將軍,謚節愍侯。



 焦嵩,安定人。初率眾據雍。曜之逼京都,允告難於嵩,嵩素侮允,曰:「須允困,當救之。」及京都敗,嵩亦尋為寇所滅。



 賈渾,不知何郡人也。太安中,為介休令。及劉元海作亂,遣其將喬晞攻陷之。渾抗節不降,曰:「吾為晉守,不能全
 之,豈茍求生以事賊虜,何面目以視息世間哉!」晞怒,執將殺之,晞將尹崧曰:「將軍舍之,以勸事君。」晞不聽,遂害之。



 王育,字伯春,京兆人也。少孤貧,為人傭牧羊,每過小學,必歔欷流涕。時有暇,即折蒲學書,忘而失羊,為羊主所責,育將鬻己以償之。同郡許子章,敏達之士也,聞而嘉之,代育償羊,給其衣食,使與子同學,遂博通經史。身長八尺餘,須長三尺,容貌絕異,音聲動人。子章以兄之子妻之,為立別宅,分之資業,育受之無愧色。然行己任性,
 頗不偶俗。妻喪,弔之者不過四五人,然皆鄉閭名士。太守杜宣命為主簿。俄而宣左遷萬年令,杜令王攸詣宣,宣不迎之,攸怒曰:「卿往為二千石,吾所敬也。今吾儕耳,何故不見迎?欲以小雀遇我,使我畏死鷂乎?」育執刀叱攸曰:「君辱臣死,自昔而然。我府君以非罪黜降,如日月之蝕耳,小縣令敢輕辱吾君!汝謂吾刀鈍邪,敢如是乎!」前將殺之。宣懼,跣下抱育,乃止。自此知名。司徒王渾闢為掾,除南武陽令。為政清約,宿盜逃奔他郡。遷并州督護。成都王穎在鄴,又以育為振武將軍。劉元海之為北單于,育說穎曰:「元海今去,育請為殿下促之,不然,懼不
 至也。」穎然之,以育為破虜將軍。元海遂拘之,其後以為太傅。



 韋忠字子節,平陽人也。少慷慨,有不可奪之志。好學博通,性不虛諾。閉門修己,不交當世,每至吉凶,親表贈遺,一無所受。年十二,喪父,哀慕毀悴,杖而後起。司空裴秀弔之,匍匐號訴,哀慟感人。秀出而告人曰:「此子長大必為佳器。」歸而命子頠造焉。服闋,遂廬於墓所。頠慕而造之,皆託行不見。家貧,藜藿不充,人不堪其憂,而忠不改其樂。頠為僕射,數言之於司空張華,華辟之,辭疾不起。
 人問其故,忠曰:「吾茨簷賤士,本無宦情。且茂先華而不實,裴頠慾而無厭,棄典禮而附賊后,若此,豈大丈夫之所宜行邪!裴常有心託我,常恐洪濤蕩嶽,餘波見漂,況可臨尾閭而窺沃焦哉!」太守陳楚迫為功曹。會山羌破郡,楚攜子出走,賊射之,中三創。忠冒刃伏楚。以身捍之,泣曰:「韋忠願以身代君,乞諸君哀之。」亦遭五矢。賊相謂曰:「義士也!」舍之。忠於是負楚以歸。後仕劉聰,為鎮西大將軍,平羌校尉,討叛羌,矢盡,不屈節而死。



 辛勉,字伯力,隴西狄道人也。父洪,左衛將軍。勉博學,有
 貞固之操。懷帝世,累遷為侍中。及洛陽陷,隨帝至平陽。劉聰將署為光祿大夫,勉固辭不受。聰遣其黃門侍郎喬度齎藥酒逼之,勉曰:「大丈夫豈以數年之命而虧高節,事二姓,下見武皇帝哉!」引藥將飲,度遽止之曰:「主上相試耳,君真高士也!」歎息而去。聰嘉其貞節,深敬異之,為築室於平陽西山,月致酒米,勉亦辭而不受。年八十,卒。



 勉族弟賓,愍帝時為尚書郎。及帝蒙塵於平陽,劉聰使帝行酒洗爵,欲觀晉臣在朝者意。賓起而抱帝大哭,聰曰:「前殺庾氏輩,故不足為戒邪!」引出,遂加害焉。



 劉敏元,字道光,北海人也。厲己修學,不以險難改心。好星歷陰陽術數,潛心《易》、《太玄》,不好讀史,常謂同志曰:「誦書當味義根,何為費功於浮辭之文!《易》者,義之源,《太玄》,理之門,能明此者,即吾師也。」永嘉之亂,自齊西奔。同縣管平年七十餘,隨敏元而西,行及滎陽,為盜所劫。敏元已免,乃還謂賊曰:「此公孤老,餘年無幾,敏元請以身代,願諸君舍之。」賊曰:「此公於君何親?」敏元曰:「同邑人也。窮窶無子,依敏元為命。諸君若欲役之,老不堪使,若欲食之,復不如敏元,乞諸君哀也。」有一賊真目叱敏元曰:「吾不放此公,憂不得汝乎!」敏元奮劍曰:「吾豈望生邪!當殺
 汝而後死。此公窮老,神祇尚當哀矜之。吾親非骨肉,義非師友,但以見投之故,乞以身代。諸大夫慈惠,皆有聽吾之色,汝何有靦面目而發斯言!」顧謂諸盜長曰:「夫仁義何常,寧可失諸君子!上當為高皇、光武之事,下豈失為陳項乎!當取之由道,使所過稱詠威德,柰何容畜此人以損盛美!當為諸君除此人,以成諸君霸王之業。」前將斬之。盜長遽止之,而相謂曰:「義士也!害之犯義。」乃俱免之。後仕劉曜,為中書侍郎、太尉長史。



 周該,天門人也。性果烈,以義勇稱。雖不好學,而率由名
 教。叔父級為宜都內史,亦忠節士也。聞譙王承立義湘州,甘卓又不同王敦之舉,而書檄不至,級謂該曰:「吾嘗疾王敦挾陵上之心,今稱兵構逆,有危社稷之勢。譙王宗室之望,據方州之重,建旗誓眾,圖襲武昌。甘安南少著勇名,士馬器械當今為盛,聞與譙王剋期舉義,此乃烈士急病之秋,吾致死之時也,汝其成吾之志,申款於譙王乎?」該欣然奉命,潛至湘州,與承相見,口陳至誠。承大悅。會王敦遣其將魏乂圍承甚急,該乃與湘州從事周崎間出反命,俱為乂所見,考之至死,竟不言其故,級由是獲免王敦之難。



 桓雄,長沙人也。少仕州郡。譙王承為湘州刺史,命為主簿。王敦之逆,承為敦將魏又所執,佐吏奔散,雄與西曹韓階,從事武延並毀服為僮豎,隨承向武昌。乂見雄姿貌長者,進退有禮,知非凡人,有畏憚之色,因害之。



 韓階,長沙人也。性廉謹篤慎,為閭里所敬愛。刺史、譙王承辟為議曹祭酒,轉西曹書佐。及承為魏乂所執,送武昌,階與武延等同心隨從,在承左右。桓雄被害之後,二人執志愈固。及承遇禍,階、延親營殯斂,送柩還都,朝夕
 哭奠,俱葬畢乃還。



 周崎,邵陵人也。為湘州從事。王敦之難,譙王承使崎求救于外,與周該俱為魏乂偵人所執,乂責崎辭情,臨以白刃。崎曰:「州將使求援于外,本無定指,隨時制宜耳。」又謂崎曰:「汝為我語城中,稱大將軍已破劉隗、戴若思,甘卓住襄陽,無復異議,三江州郡,萬里肅清,外援理絕。如是者,我當活汝。」崎偽許之。既到城下,大呼曰:「王敦軍敗於于湖,甘安南已剋武昌,即日分遣大眾來赴此急,努力堅守,賊今散矣!」乂於是數而殺之。



 易雄,字興長,長沙瀏陽人也。少為縣吏,自念卑賤,無由自達,乃脫幘挂縣門而去。因習律令及施行故事,交結豪右,州里稍稱之。仕郡,為主簿。張昌之亂也,執太守萬嗣,將斬之,雄與賊爭論曲直。賊怒,叱使牽雄斬之,雄趨出自若。賊又呼問之,雄對如初。如此者三,賊乃舍之。嗣由是獲免,雄遂知名。舉孝廉,為州主簿,遷別駕。自以門寒,不宜久處上綱,謝職還家。後為舂陵令。



 刺史、譙王承既距王敦,將謀起兵以赴朝廷。雄承符馳檄遠近,列敦罪惡,宣募縣境,數日之中,有眾千人,負糧荷戈而從之。
 承既固守,而湘中殘荒之後,城池不完,兵資又闕。敦遣魏乂、李恒攻之,雄勉厲所統,扞禦累旬,士卒死傷者相枕。力屈城陷,為乂所虜,意氣慷慨,神無懼色。送到武昌,敦遣人以檄示雄而數之。雄曰:「此實有之,惜雄位微力弱不能救國之難。王室如毀。雄安用生為!今日即戮,得作忠鬼,乃所願也。」敦憚其辭正,釋之。眾人皆賀,雄笑曰:「昨夜夢乘車,挂肉其傍。夫肉必有筋,筋者斤也,車傍有斤,吾其戮乎!」尋而敦遣殺之。當時見者,莫不傷惋。



 樂道融,丹陽人也。少有大志,好學不倦,與朋友信,每約
 己而務周給,有國士之風。為王敦參軍。敦將圖逆,謀害朝賢,以告甘卓。卓以為不可,遲留不赴。敦遣道融召之。道融雖為敦佐,忿其逆節,因說卓曰:「主上躬統萬機,非專任劉隗。今慮七國之禍,故割湘州以削諸侯,而王氏擅權日久,卒見分政,便謂被奪耳。王敦背恩肆逆,舉兵伐主,國家待君至厚,今若同之,豈不負義!生為逆臣,死為愚鬼,永成宗黨之恥邪!君當偽許應命,而馳襲武昌,敦眾聞之,必不戰自散,大勳可就矣。」卓大然之,乃與巴東監軍柳純等露檄陳敦過逆,率所統致討,又遣齎表詣臺。卓怍不果決,且年老多疑,遂待諸方同進,出軍稽
 遲。至豬口,敦聞卓已下兵,卓兄子仰時為敦參軍,使仰求和於卓,令其旋軍。卓信之,將旋,主簿鄧騫與道融勸卓曰:「將軍起義兵而中廢,為敗軍之將,竊為將軍不取。今將軍之下,士卒各求其利,一旦而還,恐不可得也。」卓不從。道融晝夜涕泣諫卓,憂憤而死。



 虞悝,長沙人也。弟望,字子都。並有士操,孝悌廉信為鄉黨所稱,而俱好臧否,以人倫為己任。少仕州郡,兄弟更為治中、別駕。元帝為丞相,招延四方之士,多辟府掾,時人謂之「百六掾」。望亦被召,恥而不應。



 譙王承臨州,知其
 名,檄悝為長史。未到,遭母喪。會王敦作逆,承往弔悝,因留與語曰:「吾前被詔,遣鎮此州,正以王敦專擅,防其為禍。今敦果為逆謀,吾受任一方,欲率所領馳赴朝廷,而眾少糧乏,且始到貴州,恩信未著。卿兄弟南夏之翹俊,而智勇遠聞,古人墨絰即戎,況今鯨鯢塞路,王室危急,安得遂罔極之情,忘忠義之節乎!如今起事,將士器械可以濟不?」悝、望對曰:「王敦居分陜之任,一旦構逆,圖危社稷,此天地所不容,人神所忿疾。大王不以猥劣,枉駕訪及,悝兄弟並受國恩,敢不自奮!今天朝中興,人思晉德,大王以宗子之親,奉信順而誅有罪,孰不荷戈致命!
 但鄙州荒弊,糧器空竭,舟艦寡少,難以進討。宜且收眾固守,傳檄四方,其勢必分,然後圖之,事可捷也。」承以為然,乃命悝為長史,望為司馬,督護諸軍。



 湘東太守鄭澹,敦之姊夫也,不順承旨,遣望討之。望率眾一旅,直人郡斬澹,以徇四境。及魏乂來攻,望每先登,力戰而死。城破,悝復為乂所執,將害之,子弟對之號泣,悝謂曰:「人生有死,闔門為忠義鬼,亦何恨哉!」及王敦平,贈悝襄陽太守,望滎陽太守,遣謁者至墓,祭以少牢。



 沈勁,字世堅,吳興武康人也。父充,與王敦構逆,眾敗而
 逃,為部曲將吳儒所殺。勁當坐誅,鄉人錢舉匿之得免。其後竟殺仇人。勁少有節操,哀父死於非義,志欲立勳以雪先恥。年三十餘,以刑家不得仕進。郡將王胡之深異之,及遷平北將軍、司馬刺史,將鎮洛陽,上疏曰:「臣當籓衛山陵,式遏戎狄,雖義督群心,人思自百,然方翦荊棘,奉宣國恩,艱難急病,非才不濟。吳興男子沈勁,清操著於鄉邦,貞固足以幹事。且臣今西,文武義故,吳興人最多,若令勁參臣府事者,見人既悅,義附亦眾。勁父充昔雖得罪先朝,然其門戶累蒙曠蕩,不審可得特垂沛然,許臣所上否?」詔聽之。勁既應命,胡之以疾病解職。



 升
 平中,慕容恪侵逼山陵。時冠軍將軍陳祐守洛陽,眾不過二千,勁自表求配祐效力,因以勁補冠軍長史,令自募壯士,得千餘人,以助祐擊賊,頻以寡制眾。而糧盡援絕,祐懼不能保全。會賊寇許昌,祐因以救許昌為名,興寧三年,留勁以五百人守城,祐率眾而東。會許昌已沒,祐因奔崖塢。勁志欲致命,欣獲死所。尋為恪所攻,城陷,被執,神氣自若。恪奇而將宥之,其中軍將軍慕容虔曰:「勁雖奇士,觀其志度,終不為人用。今若赦之,必為後患。」遂遇害。恪還,從容言於慕容晞曰:「前平廣固,不能濟辟閭,今定洛陽而殺沈勁,實有愧於四海。」朝廷聞而嘉之,
 贈東陽太守。子赤黔為大長秋。赤黔子叔任,義熙中為益州刺史。



 吉挹,字祖沖,馮翊蓮芍人也。祖朗,愍帝時為御史中丞。西朝不守,朗歎曰:「吾智不能謀,勇不能死,何忍君臣相隨北面事賊虜乎!」乃自殺。挹少有志節。孝武帝初,苻堅陷梁益,桓豁表挹為魏興太守,尋加輕車將軍,領晉昌太守。以距堅之功,拜員外散騎侍郎。苻堅將韋鐘攻魏興,挹遣眾距之,斬七百餘級,加督五郡軍事。鐘率眾欲趣襄陽,挹又邀擊,斬五千餘級。鐘怒,迴軍圍之,挹又屢
 挫其銳。其後賊眾繼至,挹力不能抗,城將陷,引刃欲自殺,其友止之曰:「且茍存以展他計,為計不立,死未晚也。」挹不從,友人逼奪其刀。會賊執之,挹閉口不言,不食而死。



 車騎將軍桓沖上言曰:「故輕車將軍、魏興太守吉挹祖朗,西臺傾覆,隕身守節。挹世篤忠孝,乃心本朝。臣亡兄溫昔伐咸陽,軍次灞水,挹攜將二弟,單馬來奔,錄其此誠,仍加擢授,自新野太守轉在魏興。久處兵任,委以邊戍,疆場歸懷,著稱所蒞。前年狡氏縱逸,浮河而下,挹孤城獨立,眾無一旅,外摧凶銳,內固津要,虜賊舟船,俘馘千計,而賊并力功圍,經歷時月,會襄陽失守,邊情沮
 喪,加眾寡勢殊,以至陷設。挹辭氣慷慨,志在不辱,杖刃推戈,期之以隕,將吏持守,用不即斃,遂乃杜口無言,絕粒而死。挹參軍史穎,近於賊中得還,齎挹臨終手疏,并具說意狀。挹之忠志,猶在可錄。若蒙天地垂曲宥之恩,則榮加枯朽,惠隆泉壤矣。」帝嘉之,追贈益州刺史。



 王諒,字幼成,丹陽人也。少有幹略,為王敦所擢,參其府事,稍遷武昌太守。初,新昌太守梁碩專威交土,迎立陶咸為刺史。咸卒,王敦以王機為刺史,碩發兵距機,自領交趾太守,乃迎前刺史脩則子湛行州事。永興三年,敦
 以諒為交州刺史。諒將之任,敦謂曰:「脩湛、梁碩皆國賊也,卿至,便收斬之。」諒既到境,湛退還九真。廣州刺史陶侃遣人誘湛來詣諒所,諒敕從人不得入閣,既前,執之。碩時在坐,曰:「湛故州將之子,有罪可遣,不足殺也。」諒曰:「是君義故,無豫我事。」即斬之。碩怒而出。諒陰謀誅碩,使客刺之,弗剋,遂率眾圍諒於龍編。陶侃遣軍救之,未至而諒敗。碩逼諒奪其節,諒固執不與,遂斷諒右臂。諒正色曰:「死且不畏,臂斷何有!」十餘日,憤恚而卒。碩據交州,凶暴酷虐,一境患之,竟為侃軍所滅,傳首京都。



 宋矩,字處規,敦煌人也。慷慨有志節。張重華據涼州地,以矩為宛戍都尉。石季龍遣將麻秋攻大夏,護軍梁式執太守宋晏,以城應秋。秋遣晏以書致矩。矩既至,謂秋曰:「辭父事君,當立功與義;茍功義不立,當守名節。矩終不背主覆宗,偷生於世。」先殺妻子,自刎而死。秋曰:「義士也!」命葬之。重華嘉其誠節,贈振威將軍。



 車濟,字萬度,敦煌人也。果毅有大量。張重華以為金城令,為石季龍將麻秋所陷,濟不為秋屈。秋必欲降之,乃臨之以兵。濟辭色不撓,曰:「吾雖才非龐德,而受任同之。
 身可殺,志不可移。」乃伏劍而死。秋歎其忠節,以禮葬之。後重華迎致其喪,親臨慟哭,贈宜禾都尉。



 丁穆,字彥遠,譙國人也。積功勞,封真定侯,累遷為順陽太守。太元四年,除振武將軍、梁州刺史。受詔未發,會苻堅遣眾寇順陽,穆戰敗,被執至長安,稱疾不仕偽朝。堅又傾國南寇,穆與關中人士唱義,謀襲長安,事泄,遇害,臨死作表以付其妻周。其後周得至京師,詣闕上之。孝武帝下詔曰:「故順陽太守、真定侯丁穆力屈身陷,而誠節彌固,直亮壯勁,義貫古烈。其喪柩始反,言尋傷悼。可
 贈龍驤將軍、雍州刺史,賻賜一依周虓故事。為立屋宅,並給其妻衣食,以終厥身。」



 辛恭靖,隴西狄道人也。少有器幹,才量過人。隆安中,為河南太守。會姚興來寇,恭靖固守百餘日,以無救而陷,被執至長安。興謂之曰:「朕將任卿以東南之事,可乎?」恭靖厲色曰:「我寧為國家鬼,不為羌賊臣。」興怒,幽之別室。經三年,至元興中,誑守者,乃踰垣而遁,歸于江東,安帝嘉之。桓玄請為諮議參軍,置之朝首。尋而病卒。



 羅企生,字宗伯,豫章人也。多才藝。初拜佐著作郎,以家貧親老,求補臨汝令,刺史王凝之請為別駕。殷仲堪之鎮江陵,引為功曹。累遷武陵太守。未之郡而桓玄攻仲堪,仲堪更以企生為諮議參軍。仲堪多疑少決,企生深憂之,謂弟遵生曰:「殷侯仁而無斷,事必無成。成敗,天也,吾當死生以之。」仲堪果走,文武無送者,唯企生從焉。路經家門,遵生曰:「作如此分離,何可不執手!」企生迴馬授手,遵生有勇力,便牽下之,謂曰:「家有老母,將欲何之?」企生揮淚曰:「今日之事,我必死之。汝等奉養不失子道,一門之中有忠與孝,亦復何恨!」遵生抱之愈急。仲堪於路
 待之,企生遙呼曰:「生死是同,願少見待。」仲堪見企生無脫理,策馬而去。



 玄至荊州,人士無不詣者,企生獨不往,而營理仲堪家。或謂之曰:「玄猜忍之性,未能取卿誠節,若遂不詣,禍必至矣。」企生正色曰:「我是殷侯吏,見遇以國士,為弟以力見制,遂不我從,不能共殄醜逆,致此奔敗,亦何面目復就桓求生乎!」玄聞之大怒,然素待企生厚,先遣人謂曰:「若謝我,當釋汝。」企生曰:「為殷荊州吏,荊州奔亡,存亡未判,何顏復謝!」玄即收企生,遣人問欲何言,答曰:「文帝殺嵇康,嵇紹為晉忠臣,從公乞一弟,以養老母。」玄許之。又引企生於前,謂曰:「吾相遇甚厚,何以見
 負?今者死矣!」企生對曰:「使君既興晉陽之甲,軍次尋陽,並奉王命,各還所鎮,升壇盟誓,口血未幹,而生奸計。自傷力劣,不能翦滅凶逆,恨死晚也。」玄遂害之,時年三十七,眾咸悼焉。先是,玄以羔裘遺企生母胡氏,及企生遇害,即日焚裘。



 張禕,吳郡人也。少有操行。恭帝為瑯邪王,以禕為郎中令。及帝踐阼,劉裕以禕帝之故吏,素所親信,封藥酒一罌付禕密令鴆帝。禕既受命而歎曰:「鴆君而求生,何面目視息世間哉,不如死也!」因自飲之而死。



 史臣曰:中散以膚受見誅,王儀以抗言獲戾,時皆可謂死非其罪也。偉元恥臣晉室,延祖甘赴危亡,所由之理雖同,所趣之途即異,而並見稱當世,垂芳竹帛,豈不以君父居在三之極,忠孝為百行之先者乎!且裒獨善其身,故得全其孝,而紹兼濟於物,理宜竭其忠,可謂蘭桂異質而齊芳,《韶》《武》殊音而並美。或有論紹者以死難獲譏,揚榷言之,未為篤論。夫君,天也,天可仇乎!安既享其榮,危乃違其禍,進退無據,何以立人!嵇生之隕身全節,用此道也。



 贊曰:重義輕生,亡軀殉節。勁松方操,嚴霜比烈。白刃可
 陵,貞心難折。道光振古,芳流來哲。



\end{pinyinscope}