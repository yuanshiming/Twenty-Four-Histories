\article{列傳第二十八 周處 周訪}

\begin{pinyinscope}
周處
 \gezhu{
  子玘玘子勰玘弟札札兄子筵}
 周訪
 \gezhu{
  子撫撫子楚楚子瓊瓊子虓撫弟光光子仲孫}



 周處,字子隱,義興陽羨人也。父魴,吳鄱陽太守。處少孤,未弱冠,膂力絕人,好馳騁田獵,不修細行,縱情肆慾,州曲患之。處自知為人所惡,乃慨然有改勵之志,謂父老曰:「今時和歲豐,何苦而不樂耶?」父老歎曰:「三害未除,何樂之有!」處曰:「何謂也?」答曰:「南山白額猛獸,長橋下蛟,并子為三矣。」處曰:「若此為患,吾能除之。」父老曰:「子若除之,
 則一郡之大慶,非徒去害而已。」處乃入山射殺猛獸,因投水搏蛟,蛟或沈或浮,行數十里,而處與之俱,經三日三夜,人謂死,皆相慶賀。處果殺蛟而反,聞鄉里相慶,始知人患己之甚,乃入吳尋二陸。時機不在,見雲,具以情告,曰:「欲自修而年已蹉跎,恐將無及。」雲曰:「古人貴朝聞夕改,君前途尚可,且患志之不立,何憂名之不彰!」處遂勵志好學,有文思,志存義烈,言必忠信克己。期年,州府交辟。仕吳為東觀左丞。孫皓末,為無難督。及吳平,王渾登建鄴宮釃酒,既酣,謂吳人曰:「諸君亡國之餘,得無戚乎?」處對曰:「漢末分崩,三國鼎立,魏滅於前,吳亡於後,亡
 國之戚,豈惟一人!」渾有慚色。



 入洛,稍遷新平太守。撫和戎狄,叛羌歸附,雍土美之。轉廣漢太守。郡多滯訟,有經三十年而不決者,處詳其枉直,一朝決遣。以母老罷歸。尋除楚內史,未之官,徵拜散騎常侍。處曰:「古人辭大不辭小。」乃先之楚。而郡既經喪亂,新舊雜居,風俗未一,處敦以教義,又檢尸骸無主及白骨在野收葬之,然始就徵,遠近稱歎。



 及居近侍,多所規諷。遷御史中丞,凡所糾劾,不避寵戚。梁王肜違法,處深文案之。及氐人齊萬年反,朝臣惡處彊直,皆曰:「處,吳之名將子也,忠烈果毅。」乃使隸夏侯駿西征。伏波將軍孫秀知其將死,謂之曰:「卿
 有老母,可以此辭也。」處曰:「忠孝之道,安得兩全!既辭親事君,父母復安得而子乎?今日是我死所也。」萬年聞之,曰:「周府君昔臨新平,我知其為人,才兼文武,若專斷而來,不可當也。如受制於人,此成擒耳。」既而梁王肜為征西大將軍、都督關中諸軍事。處知肜不平,必當陷己,自以人臣盡節,不宜辭憚,乃悲慨即路,志不生還。中書令陳準知肜將逞宿憾,乃言於朝曰:「駿及梁王皆是貴戚,非將率之才,進不求名,退不畏咎。周處吳人,忠勇果勁,有怨無援,將必喪身。宜詔孟觀以精兵萬人,為處前鋒,必能殄寇。不然,肜當使處先驅,其敗必也。」朝廷不從。時
 賊屯梁山,有眾七萬,而駿逼處以五千兵擊之。處曰:「軍無後繼,必至覆敗,雖在亡身,為國取恥。」肜復命處進討,乃與振威將軍盧播、雍州刺史解系攻萬年於六陌。將戰,處軍人未食,肜促令速進,而絕其後繼。處知必敗,賦詩曰:「去去世事已,策馬觀西戎。藜藿甘粱黍,期之克令終。」言畢而戰,自旦及暮,斬首萬計。弦絕矢盡,播、系不救。左右勸退,處按劍曰:「此是吾效節授命之日,何退之為!且古者良將受命,鑿凶門以出,蓋有進無退也。今諸軍負信,勢必不振。我為大臣,以身徇國,不亦可乎!」遂力戰而沒。追贈平西將軍,賜錢百萬,葬地一頃,京城地五十畝
 為第,又賜王家近田五頃。詔曰:「處母年老,加以遠人,朕每愍念,給其醫藥酒米,賜以終年。」



 處著《默語》三十篇及《風土記》,并撰集《吳書》。時潘岳奉詔作《關中詩》曰:「周徇師令,身膏齊斧。人之云亡,貞節克舉。」又西戎校尉閻纘亦上詩云:「周全其節,令問不已。身雖云沒,書名良史。」及元帝為晉王,將加處策謚,太常賀循議曰:「處履德清方,才量高出;歷守四郡,安人立政;入司百僚,貞節不撓;在戎致身,見危授命:此皆忠賢之茂實,烈士之遠節。案謚法執德不回曰孝。」遂以謚焉。有三子:、靖、札。靖早卒,、札並知名。



 字宣佩。彊毅沈斷有父風,而文學不及。閉門潔己,不妄交游,士友咸望風敬憚焉,故名重一方。弱冠,州郡命,不就。刺史初到,召為別駕從事,虛己備禮,方始應命。累薦名宰府,舉秀才,除議郎。



 太安初,妖賊張昌、丘沈等聚眾於江夏,百姓從之如歸。惠帝使監軍華宏討之,敗于障山。昌等浸盛,殺平南將軍羊伊,鎮南大將軍、新野王歆等,所在覆沒。昌別率封雲攻徐州,石冰攻揚州,刺史陳徽出奔,冰遂略有揚土。密欲討冰,潛結前南平內史王矩,共推吳興太守顧祕都督揚州九郡軍事,及江東人士同起義兵,斬冰所置吳興太守區山及諸長史。
 冰遣其將羌毒領數萬人距,玘臨陣斬毒。時右將軍陳敏自廣陵率眾助,斬冰別率趙驡於蕪湖,因與俱前攻冰於建康。冰北走投封雲,雲司馬張統斬雲、冰以降,徐、揚並平。不言功賞,散眾還家。



 陳敏反于揚州,以玘為安豐太守,加四品將軍。玘稱疾不行,密遣使告鎮東將軍劉準,令發兵臨江,己為內應,翦髮為信。準在壽春,遣督護衡彥率眾而東。時敏弟昶為廣武將軍、歷陽內史,以吳興錢廣為司馬。密諷廣殺昶。與顧榮、甘卓等以兵攻敏,敏眾奔潰,單馬北走,獲之於江乘界,斬之於建康,夷三族。東海王越聞其名,召為參軍。詔補
 尚書郎、散騎郎,並不行。元帝初鎮江左,以為倉曹屬。



 初,吳興人錢璯亦起義兵討陳敏,越命為建武將軍,使率其屬會于京都。璯至廣陵,聞劉聰逼洛陽,畏懦不敢進。帝促以軍期,璯乃謀反。時王敦遷尚書,當應徵與璯俱西。璯陰欲殺敦,藉以舉事,敦聞之,奔告帝。璯遂殺度支校尉陳豐,焚燒邸閣,自號平西大將軍、八州都督,劫孫皓子充,立為吳王,既而殺之。來寇縣。帝遣將軍郭逸、郡尉宋典等討之,並以兵少未敢前。復率合鄉里義眾,與逸等俱進,討璯,斬之,傳首於建康。



 玘三定江南,開復王略,帝嘉其勳,以行建威將軍、吳興太守,封烏
 程縣侯。吳興寇亂之後,百姓饑饉,盜賊公行,甚有威惠,百姓敬愛之,期年之間,境內寧謐。帝以頻興義兵,勳誠並茂,乃以陽羨及長城之西鄉、丹陽之永世別為義興郡,以彰其功焉。



 宗族彊盛,人情所歸,帝疑憚之。于時中州人士佐佑王業,而自以為不得調,內懷怨望,復為刁協輕之,恥恚愈甚。時鎮東將軍祭酒東萊王恢亦為周凱所侮,乃與陰謀誅諸執政,推及戴若思與諸南士共奉帝以經緯世事。先是,流人帥夏鐵等寓於淮、泗,恢陰書與鐵,令起兵,己當與以三吳應之。建興初,鐵已聚眾數百人,臨淮太守蔡豹斬鐵以聞。恢
 聞鐵死,懼罪,奔于,殺之,埋於豕牢。帝聞而祕之,召為鎮東司馬,未到,復改授建武將軍、南郡太守。既南行,至蕪湖,又下令曰:「奕世忠烈,義誠顯著,孤所欽喜。今以為軍諮祭酒,將軍如故,進爵為公,祿秩僚屬一同開國之例。」忿於回易,又知其謀泄,遂憂憤發背而卒,時年五十六。將卒,謂之勰曰:「殺我者諸傖子,能復之,乃吾子也。」吳人謂中州人曰「傖」,故云耳。贈輔國將軍,謚曰忠烈。子勰嗣。



 勰字彥和。常緘父言。時中國亡官失守之士避亂來者,多居顯位,駕御吳人,吳人頗怨。勰因之欲起兵,潛結吳
 興郡功曹徐馥。馥家有部曲,勰使馥矯稱叔父札命以合眾,豪俠樂亂者翕然附之,以討王導、刁協為名。孫皓族人弼亦起兵於廣德以應之。馥殺吳興太守袁琇,有眾數千,將奉札為主。時札以疾歸家,聞而大驚,乃告亂於義興太守孔侃。勰知札不同,不敢發兵。馥黨懼,攻馥,殺之。孫弼眾亦潰,宣城太守陶猷滅之。元帝以周氏奕世豪望,吳人所宗,故不窮治,撫之如舊。勰為札所責,失志歸家,淫侈縱恣,每謂人曰:「人生幾時,但當快意耳。」終於臨淮太守。



 勰弟彞,少知名,元帝辟為丞相掾,早亡。



 札字宣季。性矜險好利,外方內荏,少以豪右自處,州郡
 辟命皆不就。察孝廉,除郎中、大司馬齊王冏參軍。出補句容令,遷吳國上軍將軍。辟東海王越參軍,不就。以討錢璯功,賜爵漳浦亭侯。元帝為丞相,表札為寧遠將軍、歷陽內史,不之職,轉從事中郎。徐馥平,以札為奮武將軍、吳興內史,錄前後功,改封東遷縣侯,進號征虜將軍、臨揚州江北軍事、東中郎將,鎮塗中,未之職,轉右將軍、都督石頭水陸軍事。札腳疾,不堪拜,固讓經年,有司彈奏,不得已乃視職。加散騎常侍。



 王敦舉兵攻石頭,札開門應敦,故王師敗績。敦轉札為光祿勳,尋補尚書。頃之,遷右將軍、會稽內史。時札兄靖子懋晉陵太守、清流亭
 侯,懋弟筵征虜將軍,吳興內史,筵弟贊大將軍從事中郎、武康縣侯,贊弟縉太子文學、都鄉侯,次兄子勰臨淮太守、烏程公。札一門五侯,並居列位,吳士貴盛,莫與為比,王敦深忌之。後筵喪母,送者千數,敦益憚焉。及敦疾,錢風以周氏宗彊,與沈充權勢相侔,欲自託於充,謀滅周氏,使充得專威揚土,乃說敦曰:「夫有國者患於彊逼,自古釁難恆必由之。今江東之豪莫彊周、沈,公萬世之後,二族必不靜矣。周彊而多俊才,宜先為之所,後嗣可安,國家可保耳。」敦納之。時有道士李脫者,妖術惑眾,自言八百歲,故號李八百。自中州至建鄴,以鬼道療病,又
 署人官位,時人多信事之。弟子李弘養徒灊山,云應讖當王。故敦使廬江太守李恆告札及其諸兄子與脫謀圖不軌。時筵為敦諮議參軍,即營中殺筵及脫、弘,又遣參軍賀鸞就沈充盡掩殺札兄弟子,既而進軍會稽,襲札。札先不知,卒聞兵至,率麾下數百人出距之,兵散見殺。札性貪財好色,惟以業產為務。兵至之日,庫中有精杖,外白以配兵,札猶惜不與,以弊者給之,其鄙吝如此,故士卒莫為之用。



 及敦死,札、莚故吏並詣闕訟周氏之冤,宜加贈謚。事下八坐,尚書卞壺議以「札石頭之役開門延寇遂使賊敦恣亂,札之責也。追贈意所未安。懋、筵
 兄弟宜復本位。」司徒王導議以「札在石頭,忠存社稷,義在亡身。至於往年之事,自臣等有識以上,與札情豈有異!此言實貫於聖鑒,論者見姦逆既彰,便欲征往年已有不臣之漸。即復使爾,要當時眾所未悟。既悟其姦萌,札與臣等便以身許國,死而後已,札亦尋取梟夷。朝廷檄命既下,大事既定,便正以為逆黨。邪正失所,進退無據,誠國體所宜深惜。臣謂宜與周顗、戴若思等同例。」尚書令希鑒議曰:「夫褒貶臧否,宜令體明例通。今周、戴以死節復位,周札以開門同例,事異賞均,意所疑惑。如司徒議,謂往年之事自有識以上皆與札不異,此為邪正
 坦然有在。昔宋文失禮,華樂荷不臣之罰;齊靈嬖孽,高厚有從昏之戮。以古況今,譙王、周、戴宜受若此之責,何加贈復位之有乎!今據已顯復,則札宜貶責明矣。」導重議曰:「省令君議,必札之開門與譙王、周、戴異。今札開門,直出風言,竟實事邪?便以風言定褒貶,意莫若原情考徵也。論者謂札知隗、協亂政,信敦匡救,茍匡救信,姦佞除,即所謂流四凶族以隆人主巍巍之功耳。如此,札所以忠於社稷也。後敦悖謬出所不圖,札亦闔門不同,以此滅族,是其死於為義也。夫信敦當時之匡救,不圖將來之大逆,惡隗、協之亂政,不失為臣之貞節者,於時朝
 士豈惟周、札邪!若盡謂不忠,懼有誣乎譙王、周、戴。各以死衛國,斯亦人臣之節也。但所見有同異,然期之於必忠,故宜申明耳。即如今君議,宋華、齊高其在隗、協矣。昔子糾之難,召忽死之,管仲不死。若以死為賢,則管仲當貶;若以不死為賢,則召忽死為失。先典何以兩通之?明為忠之情同也。死雖是忠之一目,亦不必為忠皆當死也。漢祖遺約,非劉氏不王,非功臣不侯,違命天下共誅之。後呂后王諸呂,周勃從之,王陵廷爭,可不謂忠乎?周勃誅呂尊文,安漢社稷,忠莫尚焉,則王陵又何足言,而前史兩為美談。固知死與不死,爭與不爭,茍原情盡意,
 不可定於一概也。且札闔棺定謚,違逆黨順,受戮凶邪,不負忠義明矣。」鑒又駁不同,而朝廷竟從導議,追贈札衛尉,遣使者祠以少牢。



 札長子澹,太宰府掾。次子稚,察孝廉,不行。



 筵卓犖有才幹,拜征虜將軍、吳興太守,遷黃門侍郎。徐馥之役,筵族兄續亦聚眾應之。元帝議欲討之,王導以為「兵少則不足制寇,多遣則根本空虛。黃門侍郎周筵忠烈至到,為一郡所敬。意謂直遣筵,足能殺續」。於是詔以力士百人給筵,使輕騎還陽羨。筵即日取道,晝夜兼行。既至郡,將入,遇續於門,筵謂續曰:「宜與君共詣孔府
 君,有所論。」續不肯入,筵逼牽與俱。坐定,筵謂太守孔侃曰:「府君何以置賊在坐?」續衣裏帶小刀,便操刃逼筵,筵叱郡傳教吳曾:「何不舉手!」曾有膽力,便以刀環築續,殺之。筵因欲誅勰,札拒不許,委罪於從兄邵,誅之。筵不歸家省母,遂長驅而去,母狼狽追之。其忠公如此。



 遷太子右衛率。及王敦作難,加冠軍將軍、都督會稽、吳興、義興、晉陵、東陽軍事,率水軍三千人討沈充,未發而王師敗績。筵聞札開城納敦,憤吒慷慨形于辭色。尋遇害。敦平後,與札同被復官。



 初,筵於姑孰立屋五間,而六梁一時躍出墮地,衡獨立柱頭零節之上,甚危,雖以人功,不能
 然也。後竟覆族。



 筵弟縉,少無行檢,嘗在建康、烏衣道中逢孔氏婢,時與同僚二人共載,便令左右捉婢上車,其彊暴若此。



 周訪,字士達,本汝南安城人也。漢末避地江南,至訪四世。吳平,因家廬江尋陽焉。祖纂,吳威遠將軍。父敏,左中郎將。訪少沈毅,謙而能讓,果於斷割,周窮振乏,家無餘財。為縣功曹,時陶侃為散吏,訪薦為主簿,相與結友,以女妻侃子瞻。訪察孝廉,除郎中、上甲令,皆不之官。鄉人盜訪牛於冢間殺之,訪得之,密埋其肉,不使人知。



 及元
 帝渡江,命參鎮東軍事。時有與訪同姓名者,罪當死,吏誤收訪,訪奮擊收者,數十人皆散走,而自歸於帝,帝不之罪。尋以為揚烈將軍,領兵一千二百,屯尋陽鄂陵,與甘卓、趙誘討華軼。所統厲武將軍丁乾與軼所統武昌太守馮逸交通,訪收斬之。逸來攻訪,訪率眾擊破之。逸遁保柴桑,訪乘勝進討。軼遣其黨王約、傅札等萬餘人助逸,大戰於湓口,約等又敗。訪與甘卓等會於彭澤,與軼水軍將朱矩等戰,又敗之。軼將周廣燒城以應訪,軼眾潰,訪執軼,斬之,遂平江州。



 帝以訪為振武將軍、尋陽太守,加鼓吹、曲蓋。復命訪與諸軍共征杜弢。弢作桔槔
 打官軍船艦,訪作長岐棖以距之,桔槔不得為害。而賊從青草湖密抄官軍,又遣其將張彥陷豫章,焚燒城邑。王敦時鎮湓口,遣督護繆蕤、李恆受訪節度,共擊彥。蕤於豫章、石頭,與彥交戰,彥軍退走,訪率悵下將李午等追彥,破之,臨陣斬彥。時訪為流矢所中,折前兩齒,形色不變。及暮,訪與賊隔水,賊眾數倍,自知力不能敵,乃密遣人如樵採者而出,於是結陣鳴鼓而來,大呼曰:「左軍至!」士卒皆稱萬歲。至夜,令軍中多布火而食,賊謂官軍益至,未曉而退。訪謂諸將曰:「賊必引退,然終知我無救軍,當還掩人,宜促渡水北。」既渡,斷橋訖,而賊果至,隔水
 不得進,於是遂歸湘州。訪復以舟師造湘城,軍達富口,而弢遣杜弘出海昏。時湓口騷動,訪步上柴桑,偷渡,與賊戰,斬首數百。賊退保廬陵,訪追擊敗之,賊嬰城處自守。尋而軍糧為賊所掠,退住巴丘。糧廩既至,復圍弘於廬陵。弘大擲寶物於城外,軍人競拾之,弘因陣亂突圍而出。訪率軍追之,獲鞍馬鎧杖不可勝數。弘入南康,太守將率兵逆擊,又破之,奔於臨賀。帝又進訪龍驤將軍。王敦表為豫章太守。加征討都督,賜爵尋陽縣侯。



 時梁州刺史張光卒,愍帝以侍中第五猗為征南大將軍,監荊、梁、益、寧四州,出自武關。賊率杜曾、摯瞻、胡混等並迎猗,
 奉之,聚兵數萬,破陶侃於石城,攻平南將軍荀崧於宛,不剋,引兵向江陵。王敦以從弟廙為荊州刺史,令督護征虜將軍趙誘、襄陽太守朱軌、陵江將軍黃峻等討曾,而大敗於女觀湖,誘、軌並遇害。曾遂逐廙,徑造沔口,大為寇害,威震江、沔。元帝命訪擊之。訪有眾八千,進至沌陽。曾等銳氣甚盛,訪曰:「先人有奪人之心,軍之善謀也。」使將軍李恆督左甄,許朝督右甄,訪自領中軍,高張旗幟。曾果畏訪,先攻左右甄。曾勇冠三軍,訪甚惡之,自於陣後射雉以安眾心。令其眾曰:「一甄敗,鳴三鼓;兩甄敗,鳴六鼓。」趙胤領其父餘兵屬左甄,力戰,敗而復合。胤馳
 馬告訪,訪怒,叱令更進。胤號哭還戰,自旦至申,兩甄皆敗。訪聞鼓音,選精銳八百人,自行酒飲之,敕不得妄動,聞鼓音乃進。賊未至三十步,訪親鳴鼓,將士皆騰躍奔赴,曾遂大潰,殺千餘人。訪夜追之,諸將請待明日,訪曰:「曾驍勇能戰,向之敗也,彼勞我逸,是以剋之。宜及其衰乘之,可滅。」鼓行而進,遂定漢、沔。曾等走固武當。訪以功遷南中郎將、督梁州諸軍、梁州刺史,屯襄陽。訪謂其僚佐曰:「昔城濮之役,晉文以得臣不死而有憂色,今不斬曾,禍難未已。」於是出其不意,又擊破之,曾遁走。訪部將蘇溫收曾詣軍,并獲第五猗、胡混、摯瞻等,送於王敦。又
 白敦,說猗逼於曾,不宜殺。敦不從而斬之。進位安南將軍、持節,都督、刺史如故。



 初,王敦懼杜曾之難,謂訪曰:「擒曾,當相論為荊州刺史。」及是而敦不用。至王廙去職,詔以訪為荊州。敦以訪名將,勳業隆重,有疑色。其從事中郎郭舒說敦曰:「鄙州雖遇寇難荒弊,實為用武之國,若以假人,將有尾大之患,公宜自領,訪為梁州足矣。」敦從之,訪大怒。敦手書譬釋,並遺玉環玉碗以申厚意。訪投碗于地曰:「吾豈賈豎,可以寶悅乎!」陰欲圖之。即在襄陽,務農訓卒,勤於採納,守宰有缺輒補,然後言上。敦患之,而憚其彊,不敢有異。訪威風既著,遠近悅服,智勇過人,
 為中興名將。性謙虛,未嘗論功伐。或問訪曰:「人有小善,鮮不自稱。卿功勳如此,初無一言何也?」訪曰:「朝廷威靈,將士用命,訪何功之有!」士以此重之。訪練兵簡卒,欲宣力中原,與李矩、郭默相結,慨然有平河、洛之志。善於撫納,士眾皆為致死。聞敦有不臣之心,訪恆切齒。敦雖懷逆謀,故終訪之世未敢為非。



 初,訪少時遇善相者廬江陳訓,謂訪與陶侃曰:「二君皆位至方嶽,功名略同,但陶得上壽,周當下壽,優劣更由年耳。」訪小侃一歲,太興三年卒,時年六十一。帝哭之甚慟,詔贈征西將軍,謚曰壯,立碑於本郡。二子:撫、光。



 撫字道和。彊毅有父風,而將御
 不及。元帝辟為丞相掾,父喪去官。服闋,襲爵,除鷹揚將軍、武昌太守。王敦命為從事中郎,與鄧嶽俱為敦爪牙。甘卓遇害。敦以撫為沔北諸軍事、南中郎將,鎮沔中。及敦作逆,撫領二千人從之。敦敗,撫與嶽俱亡走。撫弟光將資遺其兄,而陰欲取嶽。撫怒曰:「我與伯山同亡,何不先斬我!」會嶽至,撫出門遙謂之曰:「何不速去!今骨肉尚欲相危,況他人乎!」嶽迴船而走,撫遂共入西陽蠻中,蠻酋向蠶納之。初,嶽為西陽,欲伐諸蠻,及是諸蠻皆怨,將殺之。蠶不聽,曰:「鄧府君窮來歸我,我何忍殺之!」由是俱得免。明年,詔原敦黨,嶽、撫詣闕請罪,有詔禁錮之。



 咸和
 初,司徒王導以撫為從事中郎,出為寧遠將軍、江夏相。蘇峻作逆,率所領從溫嶠討之。峻平,遷監沔北軍事、南中郎將,鎮襄陽。石勒將郭敬率騎攻撫,撫不能守,率所領奔於武昌,坐免官。尋遷振威將軍、豫章太守,後代毌丘奧監巴東諸軍事、益州刺史、假節,將軍如故。尋進征虜將軍,加督寧州諸軍事。永和初,桓溫征蜀,進撫督梁州之漢中巴西梓潼陰平四郡軍事,鎮彭模。撫擊破蜀餘寇隗文、鄧定等,斬偽尚書僕射王誓、平南將軍王潤,以功遷平西將軍。隗文、鄧定等復反,立范賢子賁為帝。初,賢為李雄國師,以左道惑百姓,人多事之,賁遂有眾
 一萬。撫與龍驤將軍朱燾擊破斬之,以功進爵建城縣公。征西督護蕭敬文作亂,殺征虜將軍楊謹,據涪城,自號益州牧。恆溫使督護鄧遐助撫討之,不能拔,引退。溫又令梁州刺史司馬勳等會撫伐之。敬文固守,自二月至於八月,乃出降,撫斬之,傳首京師。升平中,進鎮西將軍。在州三十餘年,興寧三年卒,贈征西將軍,謚曰襄。子楚嗣。



 楚字元孫。起家參征西軍事,從父入蜀,拜鷹揚將軍、犍為太守。父卒,以楚監梁、益二州、假節,襲爵建城公。世在梁、益,甚得物情。時梁州刺史司馬勳作逆,楚與朱序討
 平之,進冠軍將軍。太和中,蜀盜李金銀、廣漢妖賊李弘並聚眾為寇,偽稱李勢子,當以聖道王,年號鳳皇。又隴西人李高詐稱李雄子,破涪城。梁州刺史楊亮失守,楚遣其子詩平之。是歲,楚卒,謚曰定。子瓊嗣。



 瓊勁烈有將略,歷數郡,代楊亮為梁州刺史、建武將軍,領西戎校尉。初,氏人竇衝求降,朝廷以為東羌校尉。後衝反,欲入漢中,安定人皇甫釗、京兆人周勳等謀納衝,瓊密知之,收釗、勳等斬之。尋卒。子虓嗣。



 虓字孟威。少有節操。州召為祭酒,後歷位至西夷校尉,領梓潼太守。寧康初,苻堅將揚安寇梓潼,虓固守涪城,
 遣步騎數千,送母妻從漢水將抵江陵,為堅將朱肜邀而獲之,虓遂降於安。堅欲以為尚書郎,虓曰:「蒙國厚恩,以至今日。但老母見獲,失節於此。母子獲全,秦之惠也。雖公侯之貴,不以為榮,況郎任乎!」堅乃止。自是每入見堅,輒箕鋸而坐,呼之為氐賊。堅不悅。屬元會,威儀甚整,堅因謂虓曰:「晉家元會何如此?」虓攘袂厲聲曰:「戎狄集聚,譬猶犬羊相群,何敢比天子!」及呂光征西域,堅出餞之,戎士二十萬,旌旗數百里,又問虓曰:「朕眾力何如?」虓曰:「戎狄已來,未之有也。」堅黨以虓不遜,屢請除之。堅待之彌厚。虓乃密書與桓沖,說賊姦計。太元三年,虓潛至
 漢中,堅追得之。後又與堅兄子苞謀襲堅,事泄,堅引虓問其狀,虓曰:「昔漸離、豫讓,燕、智之微臣,猶漆身吞炭,不忘忠節。況虓世荷晉恩,豈敢忘也。生為晉臣,死為晉鬼,復何問乎!」堅曰:「今殺之,適成其名矣。」遂撻之,徙于太原。後堅復陷順陽、魏興,獲二守,皆執節不撓,堅歎曰:「周孟威不屈於前,丁彥遠潔己於後,吉祖沖不食而死,皆忠臣也。」



 虓竟以病卒於太原。其子興迎致其喪,冠軍將軍謝玄親臨哭之,因上疏曰:「臣聞旌善表功,崇義明節,所以振揚聲教,垂美來葉。故西夷校尉、梓潼太守周虓,執心忠烈,厲節寇庭,遂嬰禍荒裔,痛窴泉壤。臣每悲其志,
 以為蘇武之賢,不復過也。前宣告并州,訪求虓喪,并索其家。負荷數千,始得來至。即以資送,還其舊隴。伏願聖朝追其志心,表其殊節,使負霜之志不墜於地,則榮慰存亡,惠被幽顯矣。」孝武帝詔曰:「虓厲志貞亮,無愧古烈。未及拔身,奄隕厥命。甄表義節,國之典也。贈龍驤將軍、益州刺史,賻錢二十萬,布百匹。」又贍賜其家。



 光少有父風,年十一,見王敦,敦謂曰:「貴郡未有將,誰可用者?」光曰:「明公不恥下問,竊謂無復見勝。」敦笑以為寧遠將軍、尋陽太守。及敦舉兵,光率千餘人赴之。既至,敦已死,光未之知,求見敦。王應祕不言,以疾告。光退曰:「今
 我遠來而不得見王公,公其死乎?」遽見其兄撫曰:「王公已死,兄何為與錢鳳作賊?」眾並愕然。其夕,眾散,錢鳳走出,至闔廬洲,光捕鳳,詣闕贖罪,故得不廢。蘇峻作逆,隨溫嶠力戰有功。峻平,賜爵曲江男,卒官。



 子仲孫,興寧初督寧州軍事、振武將軍、寧州刺史。在州貪暴,人不堪命。桓溫以梁、益多寇,周氏世有威稱,復除仲孫監益、豫、梁州之三郡。寧康初,楊安寇蜀,仲孫失守,免官。後徵為光祿勳,卒。



 初,陶侃微時,丁艱,將葬,家中忽失牛而不知所在。遇一老父,謂曰:「前崗見一牛眠山汙中,其地若葬,位極人臣矣。」又指一山云:「此亦其次,當世出二千石。」言訖
 不見。侃尋牛得之,因葬其處,以所指別山與訪。訪父死,葬焉,果為刺史,著稱寧、益,自訪以下,三世為益州四十一年,如其所言云。



 史臣曰:夫仁義豈有常,蹈之即君子,背之即小人。周子隱以跅弛之材,負不羈之行,比凶蛟猛獸,縱毒鄉閭,終能克己厲精,朝聞夕改,輕生重義,徇國亡軀,可謂志節之士也。宣佩奮茲忠勇,屢殄妖氛,威略冠於本朝,庸績書於王府。既而結憾朝宰,潛構異圖,忿不思難,斯為隘矣。終於憤恚,豈不惜哉!札、筵等負俊逸之材,以雄豪自許,始見疑於朝廷,終獲戾於權右,彊弗如弱,信有徵矣。
 而札受委捍城,乃開門揖盜,去順效逆,彼實有之。後雖假手兇徒,可謂罪人斯得。朝廷議加榮贈,不其僭乎!有晉之刑政陵夷,用此道也。周訪器兼文武,任在折沖,戡定湘、羅,克清江、漢,謀孫翼子,杖節擁旄,西蜀仰其威風,中興推為名將,功成名立,不亦美乎!孟威陷跡虜廷,抗辭偽主,雖圖史所載,何以加焉!



 贊曰:平西果勁,始邪末正。勇足除殘,忠能致命。宣佩懋功,三定江東。札雖啟敵,筵實懷忠。尋陽緯武,擁旄持斧。曰子曰孫,重規疊矩。孟威抗烈,心存舊主。



\end{pinyinscope}