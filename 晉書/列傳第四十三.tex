\article{列傳第四十三}

\begin{pinyinscope}
庾亮
 \gezhu{
  子彬羲龢弟懌冰條翼}



 庾亮,字元規,明穆皇后之兄也。父琛,在《外戚傳》。亮美姿容,善談論,性好《莊》《老》,風格峻整,動由禮節,閨門之內,不肅而成,時人或以為夏侯太初、陳長文之倫也。年十六,東海王越辟為掾,不就,隨父在會稽,嶷然自守。時人皆憚其方儼,莫敢造之。



 元帝為鎮東時,聞其名,辟西曹掾。及引見,風情都雅,過於所望,甚器重之,由是聘亮妹為
 皇太子妃。亮固讓,不許。轉丞相參軍。預討華軼功,封都亭侯,轉參丞相軍事,掌書記。中興初,拜中書郎,領著作,侍講東宮。其所論釋,多見稱述。與溫嶠俱為太子布衣之好。時帝方任刑法,以《韓子》賜皇太子,亮諫以申韓刻薄傷化,不足留聖心,太子甚納焉。累遷給事中、黃門侍郎、散騎常侍。時王敦在蕪湖,帝使亮詣敦籌事。敦與亮談論,不覺改席而前,退而嘆曰:「庾元規賢於裴顧遠矣!」因表為中領軍。



 明帝即位,以為中書監,亮上書讓曰:



 臣凡庸固陋,少無殊操,昔以中州多故,舊邦喪亂,隨侍先臣,遠庇有道,爰容逃難,求食而已。不悟徼時之福,遭遇
 嘉運。先帝龍興,垂異常之顧,既眷同國士,又申以婚姻,遂階親寵,累忝非服。弱冠濯纓,沐浴芳風,頻煩省闥,出總六軍,十餘年間,位超先達。無勞受遇,無與臣比。小人祿薄,福過災生,止足之分,臣所宜守。而偷榮昧進,日爾一日,謗讟既集,上塵聖朝。始欲自聞,而先帝登遐,區區微誠,竟未上達。



 陛下踐阼,聖政惟新,宰輔賢明,庶僚咸允,康哉之歌,實存于至公。而國恩不已,復以臣領中書。臣領中書,則示天下以私矣。何者?臣於陛下,后之兄也。姻婭之嫌,與骨肉中表不同。雖太上至公,聖德無私,然世之喪道,有自來矣。悠悠六合,皆私其姻,人皆有私,則
 天下無公矣。是以前後二漢,咸以抑后黨安,進婚族危。向使西京七族、東京六姓皆非姻族,各以平進,縱不悉全,決不盡敗。今之盡敗,更由姻暱。



 臣歷觀庶姓在世,無黨於朝,無援於時,植根之本輕也薄也。茍無大瑕,猶或見容。至於外戚,憑託天地,連勢四時,根援扶疏,重矣大矣。而或居權寵,四海側目,事有不允,罪不容誅。身既招殃,國為之弊。其故何邪?由姻媾之私群情之所不能免,是以疏附則信,姻進則疑。疑積於百姓之心,則禍成於重閨之內矣。此皆往代成鑒,可為寒心者也。夫萬物之所不通,聖賢因而不奪。冒親以求一寸之用,未若防嫌以明
 至公。今以臣之才,兼如此之嫌,而使內處心膂,外總兵權,以此求治,未之聞也;以此招禍,可立待也。雖陛下二相明其愚款,朝士百僚頗識其情,天下之人安可門到戶說使皆坦然邪!



 夫富貴榮寵,臣所不能忘也;刑罰貧賤,臣所不能甘也。今恭命則愈,違命則苦,臣雖不達,何事背時違上,自貽患責邪?實仰覽殷鑒,量己知弊,身不足惜,為國取悔,是以悾悾屢陳丹款。而微誠淺薄,未垂察諒,憂惶屏營不知所措。願陛下垂天地之鑒,察臣之愚,則臣雖死之日,猶生之年矣。



 疏奏,帝納其言而止。



 王敦既有異志,內深忌亮,而外崇重之。亮憂懼,以疾去官。
 復代王導為中書監。及敦舉兵,加亮左衛將軍,與諸將距錢鳳。及沈充之走吳興也,又假亮節、都督東征諸軍事,追充。事平,以功封永昌縣開國公,賜絹五千四百匹,固讓不受。轉護軍將軍。



 及帝疾篤,不欲見人,群臣無得進者。撫軍將軍、南頓王宗,右衛將軍虞胤等,素被親愛,與西陽王羕將有異謀。亮直入臥內見帝,流涕不自勝。既而正色陳羕與宗等謀廢大臣,規共輔政,社稷安否,將在今日,辭旨切至。帝深感悟,引亮升御座,遂與司徒王導受遺詔輔幼主。加亮給事中,徙中書令。太后臨朝,政事一決於亮。



 先是,王導輔政,以寬和得眾,亮任法裁
 物,頗以此失人心。又先帝遺詔褒進大臣,而陶侃、祖約不在其例,侃、約疑亮刪除遺詔,並流怨言。亮懼亂,於是出溫嶠為江州以廣聲援,修石頭以備之。會南頓王宗復謀廢執政,亮殺宗而廢宗兄羕。宗,帝室近屬,羕,國族元老,又先帝保傅,天下咸以亮翦削宗室。



 瑯邪人卞咸,宗之黨也,與宗俱誅。咸兄闡亡奔蘇峻,亮符峻送闡,而峻保匿之。峻又多納亡命,專用威刑,亮知峻必為禍亂,徵為大司農。舉朝謂之不可,平南將軍溫嶠亦累書止之,皆不納。峻遂與祖約俱舉兵反。溫嶠聞峻不受詔,便欲下衛京都,三吳又欲起義兵,亮並不聽,而報嶠書曰:「
 吾憂西陲過於歷陽,足下無過雷池一步也。」既而峻將韓晃寇宣城,亮遣距之,不能制,峻乘勝至于京都。詔假亮節、都督征討諸軍事,戰于建陽門外。軍未及陣,士眾棄甲而走。亮乘小船西奔,亂兵相剝掠,亮左右射賊,誤中柂工,應弦而倒,船上咸失色欲散。亮不動容,徐曰:「此手何可使著賊!」眾心乃安。



 亮攜其三弟懌、條、翼南奔溫嶠,嶠素欽重亮,雖在奔敗,猶欲推為都統。亮固辭,乃與嶠推陶侃為盟主。侃至尋陽,既有憾於亮,議者咸謂侃欲誅執政以謝天下。亮甚懼,及見侃,引咎自責,風止可觀。侃不覺釋然,乃謂亮曰:「君侯修石頭以擬老子,今日
 反見求耶!」便談宴終日。亮啖薤,因留白。侃問曰:「安用此為?」亮云:「故可以種。」侃於是尤相稱歎云:「非惟風流,兼有為政之實。」



 既至石頭,亮遣督護王彰討峻黨張曜,反為所敗。亮送節傳以謝侃,侃答曰:「古人三敗,君侯始二。當今事急,不宜數耳。」又曰:「朝政多門,用生國禍。喪亂之來,豈獨由峻也!」亮時以二千人守白石壘,峻步兵萬餘,四面來攻,眾皆震懼。亮激厲將士,並殊死戰,峻軍乃退,追斬數百級。



 峻平,帝幸溫嶠舟,亮得進見,稽顙鯁噎,詔群臣與亮俱升御坐。亮明日又泥首謝罪,乞骸骨,欲闔門投竄山海。帝遣尚書、侍中手詔慰喻:「此社稷之難,非
 舅之責也。」亮上疏曰:



 臣凡鄙小人,才不經世,階緣戚屬,累忝非服,叨竊彌重,謗議彌興。皇家多難,未敢告退,遂隨牒展轉,便煩顯任。先帝不豫,臣參侍醫藥,登遐顧命,又豫聞後事,豈云德授,蓋以親也。臣知其不可,而不敢逃命,實以田夫之交猶有寄託,況君臣之義,道貫自然,哀悲眷戀,不敢違距。且先帝謬顧,情同布衣,既今恩重命輕,遂感遇忘身。加以陛下初在諒闇,先后親覽萬機,宣通外內,臣當其地,是以激節驅馳,不敢依違。雖知無補,志以死報。而才下位高,知進忘退,乘寵驕盈,漸不自覺。進不能撫寧外內,退不能推賢宗長,遂使四海側心,
 謗議沸騰。



 祖約、蘇峻不堪其憤,縱肆兇逆,事由臣發。社稷傾覆,宗廟虛廢,先后以憂逼登遐,陛下旰食踰年,四海哀惶,肝腦塗地,臣之招也,臣之罪也。朝廷寸斬之,屠戮之,不足以謝祖宗七廟之靈;臣灰身滅族,不足以塞四海之責。臣負國家,其罪莫大,實天所不覆,地所不載。陛下矜而不誅,有司縱而不戮。自古及今,豈有不忠不孝如臣之甚!不能伏劍北闕,偷存視息,雖生之日,亦猶死之年,朝廷復何理齒臣於人次,臣亦何顏自次於人理!



 臣欲自投草澤,思愆之心也,而明詔謂之獨善其身。聖旨不垂矜察,所以重其罪也。願陛下覽先朝謬授之
 失,雖垂寬宥,全其首領,猶宜棄之,任其自存自沒,則天下粗知勸戒之綱矣。



 疏奏,詔曰:



 省告懇惻,執以感歎,誠是仁舅處物宗之責,理亦盡矣。若大義既不開塞,舅所執理勝,何必區區其相易奪!



 賊峻姦逆,書契所未有也。是天地所不容,人神所不宥。今年不反,明年當反,愚智所見也。舅與諸公勃然而召,正是不忍見無禮於君者也。論情與義,何得謂之不忠乎!若以己總率征討,事至敗喪,有司宜明直繩,以肅國體,誠則然矣。且舅遂上告方伯,席卷來下,舅躬貫甲胄,賊峻梟懸。大事既平,天下開泰,衍得反正,社稷乂安,宗廟有奉,豈非舅二三方伯
 忘身陳力之勛邪!方當策勛行賞,豈復議既往之咎乎!



 且天下大弊,死者萬計,而與桀寇對岸。舅且當上奉先帝顧託之旨,弘濟艱難,使衍沖人永有憑賴,則天下幸甚。



 亮欲遁逃山海,自暨陽東出。詔有司錄奪舟船。亮乃求外鎮自效,出為持節、都督豫州揚州之江西宣城諸軍事、平西將軍、假節、豫州刺史,領宣城內史。亮遂受命,鎮蕪湖。



 頃之,後將軍郭默據湓口以叛,亮表求親征,於是以本官加征討都督,率將軍路永、毛寶、趙胤、匡術、劉仕等步騎二萬,會太尉陶侃俱討破之。亮還蕪湖,不受爵賞。侃移書曰:「夫賞罰黜陟,國之大信,竊怪矯然,獨為
 君子。」亮曰:「元帥指捴,武臣效命,亮何功之有!」遂苦辭不受。進號鎮西將軍,又固讓。初,以誅王敦功,封永昌縣公。亮比陳讓,疏數十上,至是許之。陶侃薨,遷亮都督江、荊、豫、益、梁、雍六州諸軍事,領江、荊、豫三州刺史,進號征西將軍、開府儀同三司、假節。亮固讓開府,乃遷鎮武昌。



 時王導輔政,主幼時艱,務存大綱,不拘細目,委任趙胤、賈寧等諸將,並不奉法,大臣患之。陶侃嘗欲起兵廢導,而郗鑒不從,乃止。至是,亮又欲率眾黜導,又以諮鑒,而鑒又不許。亮與鑒箋曰:



 昔於蕪湖反覆謂彼罪雖重,而時弊國危,且令方嶽道勝,亦足有所鎮壓,故共隱忍,解釋
 陶公。自茲迄今,曾無悛改。



 主上自八九歲以及成人,入則在宮入之手,出則唯武官小人,讀書無從受音句,顧問未嘗遇君子。侍臣雖非俊士,皆時之良也,知今古顧問,豈與殿中將軍、司馬督同年而語哉!不云當高選侍臣,而云高選將軍、司馬督,豈合賈生願人主之美,習以成德之意乎!秦政欲愚其黔首,天下猶知不可,況乃欲愚其主哉!主之少也,不登進賢哲以輔導聖躬。春秋既盛,宜復子明辟。不稽首歸政,甫居師傅之尊;成人之主,方受師臣之悖。主上知君臣之道不可以然,而不得不行殊禮之事。萬乘之君,寄坐上九,亢龍之爻,有位無人。
 挾震主之威以臨制百官,百官莫之敢忤。是先帝無顧命之臣,勢屈於驕姦而遵養之也。趙賈之徒有無君之心,是而可忍,孰不可忍!



 且往日之事,含容隱忍,謂其罪可宥,良以時弊國危,兵甲不可屢動,又冀其當謝往釁,懼而修己。如頃日之縱,是上無所忌,下無所憚,謂多養無賴足以維持天下。公與下官並蒙先朝厚顧,荷託付之重,大姦不掃,何以見先帝於地下!願公深惟安國家、固社稷之遠算,次計公之與下官負荷輕重,量其所宜。



 鑒又不許,故其事得息。



 時石勒新死,亮有開復中原之謀,乃解豫州授輔國將軍毛寶,使與西陽太守樊峻精
 兵一萬,俱戍邾城。又以陶稱為南中郎將、江夏相,率部曲五千人入沔中。亮弟翼為南蠻校尉、南郡太守,鎮江陵。以武昌太守陳囂為輔國將軍、梁州刺史,趣子午。又遣偏軍伐蜀,至江陽,執偽荊州刺史李閎、巴郡太守黃植,送于京都。亮當率大眾十萬,據石城,為諸軍聲援,乃上疏曰:「蜀胡二寇凶虐滋甚,內相誅鋤,眾叛親離。蜀甚弱而胡尚彊,並佃並守,修進取之備。襄陽北接宛許,南阻漢水,其險足固,其土足食。臣宜移鎮襄陽之石城下,并遣諸軍羅布江沔。比及數年,戎士習練,乘釁齊進,以臨河洛。大勢一舉,眾知存亡,開反善之路,宥逼協之
 罪,因天時,順人情,誅逋逆,雪大恥,實聖朝之所先務也。願陛下許其所陳,濟其此舉。淮泗壽陽所宜進據,臣輒簡練部分。乞槐棘參議,以定經略。」帝下其議。時王導與亮意同,郗鑒議以資用未備,不可大舉。亮又上疏,便欲遷鎮。會寇陷邾城,毛寶赴水而死。亮陳謝,自貶三等,行安西將軍。有詔復位。尋拜司空,餘官如故,固讓不拜。



 亮自邾城陷沒,憂慨發疾。會王導薨,徵亮為司徒、揚州刺史、錄尚書事,又固辭,帝許之。咸康六年薨,時年五十二。追贈太尉,謚曰文康。喪至,車駕親臨。及葬,又贈永昌公印綬。亮弟冰上疏曰:「臣謹詳先事,亦會聞臣亮對臣等
 之言,懇懇於斯事。是以屢自陳請,將迄十年。豈直好讓而不肅恭,顧曩時之釁近出宇下,加先帝神武,算略兼該,是以役不踰時,而凶彊馘滅。計之以事,則功歸聖主,推之於運,則勝非人力。至如亮等,因聖略之弘,得效所職,事將何論!功將何賞!及後傷蹶,責踰先功,是以陛下優詔聽許。亮實思自效以報天德,何悟身潛聖世,微志長絕,存亡哀恨,痛貫心膂。願陛下發明詔,遂先恩,則臣亮死且不朽。」帝從之。亮將葬,何充會之,歎曰:「埋玉樹於土中,使人情何能已!」



 初,亮所乘馬有的顱,殷浩以為不利於主,勸亮賣之。亮曰:「曷有己之不安而移之於人!」
 浩慚而退。亮在武昌,諸佐吏殷浩之徒,乘秋夜往共登南樓,俄而不覺亮至,諸人將起避之。亮徐曰:「諸君少住,老子於此處興復不淺。」便據胡床與浩等談詠竟坐。其坦率行己,多此類也。三子彬、羲、龢。



 彬年數歲,雅量過人。溫嶠嘗隱暗怛之,彬神色恬如也,乃徐跪謂嶠曰:「君侯何至於此!」論者謂不減於亮。蘇峻之亂,遇害。



 羲少有時譽,初為吳國內史。時穆帝頗愛文義,羲至郡獻詩,頗存諷諫。因上表曰:「陛下以聖明之德,方隆唐虞之化,而事役殷曠,百姓凋殘。以數州之資,經瞻四海之
 務,其為勞弊,豈可具言!昔漢文居隆盛之世,躬自儉約,斷獄四百,殆致刑厝。賈誼歎息,猶有積薪之言。以古況今,所以益其憂懼。陛下明鑒天挺,無幽不燭,弘濟之道,豈待瞽言。臣受恩奕世,思盡絲髮。受任到東,親臨所見,敢緣弘政,獻其丹愚。伏願聽斷之暇,少垂察覽。。」其詩文多不載。羲方見授用而卒。子準,太元中,自侍中代桓石虔為豫州刺史、西中郎將,鎮歷陽,卒官。準子悅,義熙中江州刺史。準弟楷,自有傳。



 龢字道季,好學,有文章。叔父翼將遷襄陽,龢年十五,以書諫曰:「承進據襄陽,耀威荊楚,且田且戍,漸臨河洛,使
 向化之萌懷德而附,凶愚之徒畏威反善,太平之基,便在於旦夕。昔殷伐鬼方,三年而剋;樂生守齊,遂至歷載。今皇朝雖隆,無有殷之盛;凶羯雖衰,猶醜類有徒。而沔漢之水,無萬仞之固;方城雖峻,無千尋之險。加以運漕供繼有溯流之艱,征夫勤役有勞來之歎。若窮寇慮逼,送死一決,東西互出,道尾俱進,則廩糧有抄截之患,遠略乏率然之勢。進退惟思,不見其可。此明闇所共見,賢愚所共聞,況於臨事者乎!願迴師反旆,詳擇全勝,修城池,立壘壁,勤耕農,練兵甲。若凶運有極,天亡此虜,則可泛舟北濟,方軌齊進,水陸騁邁,亦不踰旬朔矣。願詳思
 遠猷,算其可者。」翼甚奇之。升平中,代孔巖為丹陽尹,表除重役六十餘事。太和初,代王恪為中領軍,卒於官。子恆,尚書僕射,贈光祿大夫。



 懌字叔預,少以通簡為兄亮所稱。弱冠,西陽王羕闢,不就。東海王沖為長水校尉,清選綱紀,以懌為功曹,除暨陽令,又為沖中軍司馬,轉散騎侍郎,遷左衛將軍。以討蘇峻功,封廣饒男,出補臨川太守,歷監梁、雍二州軍事,轉輔國將軍、梁州刺史、假節,鎮魏興。時兄亮總統六州,以懌寬厚容眾,故授以遠任,為東西勢援。尋進監秦州氐羌諸軍事。懌遣牙門霍佐迎將士妻子,佐驅三百餘
 口亡入石季龍。亮表上,貶懌為建威將軍。朝議欲召還,亮上疏曰:「懌御眾簡而有惠,州戶雖小,賴其寬政。佐等同惡,大數不多。且懌名號大,不可以小故輕議進退。其文武之心轉已安定,賊帥艾秀遣使歸誠,上洛附賊降者五百餘口,冀一安隱,無復怵惕。」從之。後以所鎮險遠,糧運不繼,詔懌以將軍率所領還屯半洲。尋遷輔國將軍、豫州刺史,進號西中郎將、監宣城廬江歷陽安豐四郡軍事、假節,鎮蕪湖。



 懌嘗以白羽扇獻成帝,帝嫌其非新,反之。侍中劉劭曰:「柏梁雲構,大匠先居其下;管弦繁奏,夔牙先聆其音。懌之上扇,以好不以新。」後懌聞之,曰:「
 此人宜在帝之左右。」又嘗以毒酒餉江州刺史王允之。王允之覺其有毒,飲犬,犬斃,乃密奏之。帝曰:「大舅已亂天下,小舅復欲爾邪!」懌聞,遂飲鴆而卒,時年五十。贈侍中、衛將軍,謚曰簡。子統嗣。



 統字長仁,少有令名,司空、太尉辟,皆不就。調補撫軍、會稽王司馬,出為建威將軍、寧夷護軍、尋陽太守。年二十九,卒,時人稱其才器,甚痛惜之。子玄之,官至宣城內史。



 冰字季堅。兄亮以名德流訓,冰以雅素垂風,諸弟相率莫不好禮,為世論所重,亮常以為庾氏之寶。司徒辟,不就,徵祕書郎。預討華軼功,封都鄉侯。王導請為司徒右
 長史,出補吳興內史。



 會蘇峻作逆,遣兵攻冰,冰不能禦,便棄郡奔會稽。會稽內史王舒以冰行奮武將軍,距峻別率張健於吳中。時健黨甚眾,諸將莫敢先進。冰率眾擊健走之,於是乘勝西進,赴于京都。又遣司馬滕含攻賊石頭城,拔之。冰勛為多,封新吳縣侯,固辭不受。遷給事黃門侍郎,又讓不拜。司空郗鑒請為長史,不就。出補振威將軍、會稽內史。徵為領軍將軍,又辭。尋入為中書監、揚州刺史、都督揚豫兗三州軍事、征虜將軍、假節。



 是時王導新喪,人情恇然。冰兄亮既固辭不入,眾望歸冰。既當重任,經綸時務,不捨夙夜,賓禮朝賢,升擢後進,由
 是朝野注心,咸曰賢相。初,導輔政,每從寬惠,冰頗任威刑。殷融諫之,冰曰:「前相之賢,猶不堪其弘,況吾者哉!」范汪謂冰曰:「頃天文錯度,足下宜盡消禦之道。」冰曰:「玄象豈吾所測,正當勤盡人事耳。」又隱實戶口,料出無名萬餘人,以充軍實。詔復論前功,冰上疏曰:「臣門戶不幸,以短才贊務,釁及天庭,殃流邦族,若晉典休明,夷戮久矣。而于時顛沛,刑憲暫墜,遂令臣等復得為時陳力。徇國之臣,因之而奮,立功於大罪之後,建義於顛覆之餘,此是臣等所以復得視息於天壤,王憲不復必明於往愆也。此之厚幸,可謂弘矣,豈復得計勞納封,受賞司勛哉!
 願陛下曲降靈澤,哀恕由中,申命有司,惠臣所乞,則愚臣之願於此畢矣。」許之。



 成帝疾篤,時有妄為中書符,敕宮門宰相不得前,左右皆失色。冰神氣自若,曰:「是必虛妄。」推問,果詐,眾心乃定。進號左將軍。康帝即位,又進車騎將軍。冰懼權盛,乃求外出。會弟翼當伐石季龍,於是以本號除都督江荊寧益梁交廣七州豫州之四郡軍事、領江州刺史、假節,鎮武昌,以為翼援。冰臨發,上疏曰:



 臣因循家寵,冠冕當世,而志無殊操,量不及遠。頃皇家多難,釁故頻仍,朝望國器,與時殲落,遂令天眷下墜,降及臣身。俯仰伏事,於今五年。上不能光贊聖猷,下不能
 緝熙政道,而陛下遇之過分,求之不已,復策敗駕之駟,以冀萬里之功,非天眷之隆,將何以至此!是以敢竭狂瞽,以獻血誠,願陛下暫屏旒纊,以弘聽納。



 今彊寇未殄,戎車未戢,兵弱於郊,人疲於內,寇之侵逸,未可量也;黎庶之困,未之安也;群才之用,未之盡也。而陛下崇高,事與下隔,視聽察覽,必寄之群下。群下宜忠,不引不進;百司宜勤,不督不勸。是以古之帝王勤於降納,雖日總萬機,猶兼聽將相;或借訟輿人,或求謗芻蕘,良有以也。況今日之弊,開闢之極,而陛下歷數屬當其運,否剝之難嬰之聖躬,普天所以痛心於既往而傾首於將來者也。實
 冀否終而泰,屬運在今。誠願陛下弘天覆之量,深地載之厚,宅沖虛以為本,勤訓督以為務。廣引時彥,詢于政道,朝之得失必關聖聽,人之情偽必達天聰。然後覽其大當,以總國綱,躬儉節用,堯舜豈遠!大布之衣,衛文何人!是以古人有云:「非知之難,行之難;非行之難,安之難也。」願陛下既思日側於勞謙,納其起予之情,則天下幸甚矣。臣朝夕伏膺,猶不能暢,臨疏徘徊,不覺辭盡。



 頃之,獻皇后臨朝,征冰輔政,冰辭以疾篤。尋而卒,時年四十九。冊贈侍中、司空,謚曰忠成,祠以太牢。



 冰天性清慎,常以儉約自居。中子襲嘗貸官絹十匹,冰怒,捶之,市絹還
 官。臨卒,謂長史江[A170]曰:「吾將逝矣,恨報國之志不展,命也如何!死之日,斂以時服,無以官物也。」及卒,無絹為衾。又室無妾媵,家無私積,世以此稱之。冰七子:希、襲、友、蘊、倩、邈、柔。



 希字始彥。初拜秘書郎,累遷司徒右長史、黃門侍郎、建安太守,未拜,復為長史兼右衛將軍,遷侍中,出為輔國將軍、吳國內史。希既后之戚屬,冰女又為海西公妃,故希兄弟並顯貴。太和中,希為北中郎將、徐兗二州刺史,蘊為廣州刺史,並假節,友東陽太守,倩太宰長史,邈會稽王參軍,柔散騎常侍。倩最有才器,桓溫深忌之。



 初,慕
 容厲圍梁父,斷澗水,太山太守諸葛攸奔鄒山,魯、高平等數郡皆沒,希坐免官。頃之,徵為護軍將軍。希怒,固辭。希初免時,多盜北府軍資,溫諷有司劾之,復以罪免,遂客于晉陵之暨陽。初,郭璞筮冰云:「子孫必有大禍,唯用三陽可以有後。」故希求鎮山陽,友為東陽,家于暨陽。



 及海西公廢,桓溫陷倩及柔以武陵王黨,殺之。希聞難,便與弟邈及子攸之逃於海陵陂澤中。蘊於廣州飲鴆而死。及友當伏誅,友子婦,桓秘女也,請溫,故得免。故青州刺史武沈,希之從母兄也,潛餉給希經年。溫後知逾之,遣兵捕希。武沈之子遵與希聚眾於海濱,略漁人船,夜人
 京口城。平北司馬卞耽踰城奔曲阿,吏士皆散走。希放城內囚徒數百人,配以器杖,遵於外聚眾,宣令云逆賊醒溫廢帝殺王,稱海西公密旨,誅除凶逆。京都震擾,內外戒嚴,屯備六門。平北參軍劉奭與高平太守郗逸之、遊軍督護郭龍等集眾距之。卞耽又與典阿人弘戎發諸縣兵二千,并力屯新城以擊希。希戰敗,閉城自守。溫遣東海太守周少孫討之,城陷,被擒。希、邈及子姪五人斬於建康市,遵及黨與並伏誅,唯友及蘊諸子獲全。



 友子叔宣,右衛將軍。蘊子廓之,東陽太守。



 條字幼序。初避太宰府,累遷黃門郎、豫章太守。徵拜秘
 書監,賜爵鄉亭侯,出為冠軍將軍、臨川太守。豫章黃韜自稱孝神皇帝,臨川人李高為相,聚黨數百人,乘犢車,衣皁袍,攻郡縣,條討平之。條於兄弟最凡劣,故祿位不至。卒官,贈左將軍。



 翼字稚恭。風儀秀偉,少有經綸大略。京兆杜乂、陳郡殷浩並才名冠世,而翼弗之重也,每語人曰:「此輩宜束之高閣,俟天下太平,然後議其任耳。」見桓溫總角之中,便期之以遠略,因言於成帝曰:「桓溫有英雄之才,願陛下勿以常人遇之,常婿畜之,宜委以方邵之任,必有弘濟艱難之勛。」



 蘇峻作逆,翼時年二十二,兄亮使白衣領數
 百人,備石頭。高敗,與翼俱奔。事平,始辟太尉陶侃府,轉參軍,累遷從事中郎。在公府,雍容諷議。頃之,除振威將軍、鄱陽太守。轉建威將軍、西陽太守。撫和百姓,甚得歡心。遷南蠻校尉,領南郡太守,加輔國將軍、假節。及邾城失守,石城被圍,翼屢設奇兵,潛致糧杖。石城得全,翼之勛也。賜爵都亭侯。



 及亮卒,授都督江荊司雍梁益六州諸軍事、安西將軍、荊州刺史、假節,代亮鎮武昌。翼以帝舅,年少超居大任,遐邇屬目,慮其不稱。翼每竭志能,勞謙匪懈,戎政嚴明,經略深遠,數年之中,公私充實,人情翕然,稱其才幹。由是自河以南皆懷歸附,石季龍汝南
 太守戴開率數千人詣翼降。又遣使東至遼東,西到涼州,要給二方,欲同大舉。慕容皝、張駿並報使請期。翼雅有大志,欲以滅胡平蜀為己任,言論慷慨,形于辭色。將兵都尉錢頎陳事合旨,翼拔為五呂將軍,賜穀二百斛。時東土多賦役,百姓乃從海道人廣州,刺史鄧嶽大開鼓鑄,諸夷因此知造兵器。翼表陳東境國家所資,侵擾不已,逃逸漸多,夷人常伺隙,若知造鑄之利,將不可禁。



 時殷浩徵命無所就,而翼請為司馬及軍司,並不肯赴。翼遺浩書,因致其意。先是,浩父羨為長沙,在郡貪殘,,兄冰與翼書屬之。翼報曰:「殷君始往,雖多驕豪,實有風力
 之益,亦似由有佳兒、弟,故不令物情難之。自頃以來,奉公更退,私累日滋,亦不稍以此寥蕭之也。既雅敬洪遠,又與浩親善,其父兄得失,豈以小小計之。大較江東政,以傴儛豪彊,以為民蠹,時有行法,輒施之寒劣。如往年偷石頭倉米一百萬斛,皆是豪將輩,而直打殺倉督監以塞責。山遐作餘姚斗年,而為官出二千戶,政雖不倫,公彊官長也,而群共驅之,,不得安席。紀睦、徐寧奉王使糾罪人,船頭到渚,桓逸還復,而二使免官。雖皆前宰之惛謬,江東事去,實此之由也。兄弟不幸,橫陷此中,自不能拔腳於風塵之外,當共明目而治之。荊州所統一二十
 郡,唯長沙最惡。惡而不黜,與殺督監者復何異耶!」翼有風力格裁,發言立論皆如此。



 康帝即位,翼欲率眾北伐,上疏曰:「賊季龍年已六十,奢淫理盡,醜類怨叛,又欲決死遼東。皝雖驍果,未必能固。若北無掣手之虜,則江南將不異遼左矣。臣所以輒發良人,不顧忿咎。然東西形援未必齊舉,且欲北進,移鎮安陸,人沔五百,溳水通流。輒率南郡太守王愆期、江夏相謝尚、尋陽太守袁真、西陽太守曹據等精銳三萬,風馳上道,并勒平北將軍桓宣撲取黃季,欲并丹水,搖蕩秦雍。御以長轡,用逸待勞,比及數年,興復可冀。臣既臨許洛,竊謂恆溫可渡戍
 廣陵,何充可移據淮灑赭圻,路永進屯合肥。伏願表御之日便決聖聽,不可廣詢同異,以乖事會。兵聞拙速,不聞工之久也。」於是並發所統六州奴及車牛驢馬,百姓嗟怨。時欲向襄陽,慮朝遷不許,故以安陸為辭。帝及朝士皆遣使譬止,車騎參軍孫綽亦致書諫。翼不從,遂違如輒行。至夏口,復上表曰:



 臣近以胡寇有弊亡之勢,暫率所統,致討山北,並分見眾,略復江夏數城。臣等以九月十九日發武昌,以二十四日達夏口,輒簡卒搜乘停當上道。而所調借牛馬,來處皆遠,百姓所蓄,穀草不充,並多羸瘠,難以涉路。加以向冬,野草漸枯,往反二千,或
 容躓頓,輒便隨事籌量,權停此舉。又山南諸城,每至秋冬,水多燥涸,運漕用功,實為艱阻。



 計襄陽,荊楚之舊,西接益梁,與關隴咫尺,北去洛河,不盈千里,土沃田良,方城險峻,水路流通,轉運無滯,進可以掃盪秦趙,退可以保據上流。臣雖不武,意略淺短,荷國重恩,志存立效。是以受任四年,唯以習戎為務,實欲上憑聖朝威靈高略,下藉士民義慨之誠,因寇衰弊,漸臨逼之。而八年春上表請據樂鄉,廣農蓄穀,以伺二寇之釁,而值天高聽邈,未垂察照,朝議紛紜,遂令微誠不暢。



 自爾以來,上參天人之徵,下採降俘之言,胡寇衰滅,其日不遠。臣雖未獲
 長驅中原,馘截凶醜,亦不可以不進據要害,思攻取之宜。是以輒量宜入沔,徙鎮襄陽。其謝尚、王愆期等,悉令還據本戍,須到所在,馳遣啟聞。



 翼時有眾四萬,詔加都督征討軍事。師次襄陽,大會僚佐,陳旌甲,親授弧矢,曰:「我之行也,若此射矣。」遂三起三疊,徒眾屬目,其氣十倍。初,翼遷襄陽,舉朝謂之不可,議者或謂避衰,唯兄冰意同,桓溫及譙王無忌贊成其計。至是,冰求鎮武昌,為翼繼援。朝議謂冰不宜出,冰乃止。又進翼征西將軍,領南蠻校尉。胡賊五六百騎出樊城,翼遣冠軍將軍曹據追擊於撓溝北,破之,死者近半,獲馬百匹。翼綏來荒遠,務
 盡招納之宜,立客館,置典賓參軍。桓宣卒,翼以長子方之為義成太守,代領宣眾,司馬應誕為龍驤將軍、襄陽太守,參軍司勛為建威將軍、梁州刺史,戍西城。康帝崩,兄冰卒,以家國情事,留方之戍襄陽,還鎮夏口,悉取冰所領兵自配,以兄子統為尋陽太守。詔使翼還督江州,又領豫州刺史,辭豫州。復欲移鎮樂鄉,詔不許。繕修軍器,大佃積穀,欲圖後舉。遣益州刺史周撫、西陽太守曹據伐蜀,破蜀將李桓於江陽。



 翼如廁,見一物如方相,俄而疽發背。疾篤,表第二子爰之行輔國將軍、荊州刺史,司馬朱燾為南蠻校尉,以千人守巴陵。永和元年卒,
 時年四十一。追贈車騎將軍,謚曰肅。翼卒未幾,部將乾瓚、戴羲等作亂,殺將軍曹據。翼長史江[A170]、司馬朱燾、將軍袁真等共誅之。



 爰之有翼風,尋為桓溫所廢。溫既廢爰之,又以征虜將軍劉惔監沔中軍事,領義成太守,代方之。而方之。而方之、爰之並遷徙于豫章。



 史臣曰:外戚之家,連輝椒掖,舅氏之族,同氣蘭閨,靡不憑藉寵私,階緣險謁。門藏金穴,地使其驕;馬控龍媒,勢成其逼。古者右賢左戚,用杜溺私之路,愛而知惡,深慎滿覆之災,是以厚贈瓊瑰,罕升津要。塗山在夏,靡與禼稷同驅;姒氏居周,不預燕齊等列。聖人慮遠,殊有旨哉!
 搢暱元規,參聞顧命。然其筆敷華藻,吻縱濤波,方駕搢紳,足為翹楚。而智小謀大,昧經邦之遠圖;才高識寡,闕安國之長算。璇萼見誅,物議稱其拔本;牙尺垂訓,帝念深於負芒。是使蘇祖尋戈,宗祧殆覆。已而猜嫌上宰,謀黜負圖。向使郗鑒協從,必且戎車犯順,則與夫臺、產、安、桀,亦何以異哉!幸漏吞舟,免淪昭憲,是庾宗之大福,非晉政之不綱明矣。懌恣凶懷,鴆加連率,再世之後,三陽存僅,餘殃所及,蓋其宜也。



 贊曰:元規矯迹,寵階椒掖。識暗釐道,亂由乘隙。下拜長沙,有慚忠益。季堅清貞,毓德馳名。處泰逾約,居權戒盈。
 稚恭慷慨,亦擅雄聲。



\end{pinyinscope}