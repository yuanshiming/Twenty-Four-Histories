\article{載記第二十八}

\begin{pinyinscope}

 慕
 容超



 慕容超字祖明,德兄北海王納之子。苻堅破鄴,以納為廣武太守,數歲去官,家於張掖。德之南征,留金刀而去。及垂起兵山東,苻昌收納及德諸子,皆誅之。納母公孫氏以耄獲免,納妻段氏方娠,未決,囚之於郡獄。獄掾呼延平,德之故吏也,嘗有死罪,德免之。至是,將公孫及段氏逃于羌中,而生超焉。年十歲而公孫氏卒,臨終授超
 以金刀,曰:「若天下太平,汝得東歸,可以此刀還汝叔也。」平又將超母子奔于呂光。及呂隆降于姚興,超又隨涼州人徙于長安。超母謂超曰:「吾母子全濟,呼延氏之力。平今雖死,吾欲為汝納其女以答厚惠。」於是娶之。超自以諸父在東,恐為姚氏所錄,乃陽狂行乞。秦人賤之,惟姚紹見而異焉,勸興拘以爵位。召見與語,超深自晦匿,興大鄙之,謂紹曰:「諺云『妍皮不裹癡骨』,妄語耳。」由是得去來無禁。德遣使迎之,超不告母妻乃歸。及至廣固,呈以金刀,具宣祖母臨終之言,德撫之號慟。



 超身長八尺,腰帶九圍,精彩秀發,容止可觀。德甚加禮遇,始名之曰
 超,封北海王,拜侍中、驃騎大將軍、司隸校尉,開府,置佐吏。德無子,欲以超為嗣,故為超起第於萬春門內,朝夕觀之。超亦深達德旨,入則盡歡承奉,出則傾身下士,於是內外稱美焉。頃之,立為太子。



 及德死,以義熙元年僭嗣偽位,大赦境內,改元曰太上。尊德妻段氏為皇太后。以慕容鐘都督中外諸軍、錄尚書事,慕容法為征南、都督徐、兗、揚、南兗四州諸軍事,慕容鎮加開府儀同三司、尚書令,封孚為太尉,鞠仲為司空,潘聰為左光祿大夫,封嵩為尚書左僕射,自餘封拜各有差。後又以鐘為青州牧,段宏為徐州剌史,公孫五樓為武衛將軍、領屯騎
 校尉,內參政事。封孚言於超曰:「臣聞五大不在邊,五細不在庭。鐘,國之宗臣,社稷所賴;宏,外戚懿望,親賢具瞻。正應參翼百揆,不宜遠鎮方外。今鍾等出籓,五樓內輔,臣竊未安。」超新即位,害鐘等權逼,以問五樓。五樓欲專斷朝政,不欲鐘等在內,屢有間言,孚說竟不行。鐘、宏俱有不平之色,相謂曰:「黃犬之皮恐當終補狐裘也。」五樓聞之,嫌隙漸遘。



 初,超自長安行至梁父,慕容法時為兗州,鎮南長史悅壽還謂法曰:「向見北海王子,天資弘雅,神爽高邁,始知天族多奇,玉林皆寶。」法曰:「昔成方遂詐稱衛太子,人莫辯之,此復天族乎?」超聞而恚恨,形于言
 色。法亦怒,處之外館,由是結憾。及德死,法又不奔喪,超遣使讓焉。法常懼禍至,因此遂與慕容鐘、段宏等謀反。超知而征之,鐘稱疾不赴,於是收其黨侍中慕容統、右衛慕容根、散騎常侍段封誅之,車裂僕射封嵩於東門之外。西中郎將封融奔于魏。



 超尋遣慕容鎮等攻青州,慕容昱等攻徐州,慕容凝、韓範攻梁父。昱等攻莒城,拔之,徐州刺史段宏奔於魏。封融又集群盜襲石塞城,殺鎮西大將軍餘鬱,青土振恐,人懷異議。慕容凝謀殺韓範,將襲廣固。範知而攻之,凝奔梁父。範并其眾,攻梁父剋之,凝奔姚興,慕容法出奔於魏。慕容鎮剋青州,鐘殺
 其妻子,為地道而出,單馬奔姚興。



 于時超不恤政事,畋游是好,百姓苦之。其僕射韓言卓切諫,不納。超議復肉刑、九等之選,乃下書於境內曰:



 陽九數纏,永康多難。自北都傾陷,典章淪滅,律令法憲,靡有存者。綱理天下,此焉為本,既不能導之以德,必須齊之以刑。且虞舜大聖,猶命咎繇作士,刑之不可已已也如是!先帝季興,大業草創,兵革尚繁,未遑脩制。朕猥以不德,嗣承大統,撫御寡方,至蕭牆釁發,遂戎馬生郊,典儀寢廢。今四境無虞,所宜脩定,尚書可召集公卿。至如不忠不孝若封嵩之輩,梟斬不足以痛之,宜致烹轘之法,亦可附之律條,納以
 大辟之科。肉刑者,乃先聖之經,不刊之典,漢文易之,輕重乖度。今犯罪彌多,死者稍眾。肉刑之於化也,濟育既廣,懲慘尤深,光壽、建興中二祖已議復之,未及而晏駕。其令博士已上參考舊事,依《呂刑》及漢、魏、晉律令,消息增損,議成燕律。五刑之屬三千,而罪莫大於不孝。孔子曰:「非聖人者無法,非孝者無親,此大亂之道也。」轘裂之刑,烹煮之戮,雖不在五品之例,然亦行之自古。渠彌之轘,著之《春秋》;哀公之烹,爰自中代。世宗都齊,亦愍刑罰失中,咨嗟寢食。王者之有刑糾,猶人之左右手焉。故孔子曰:「刑罰不中,則人無所措手足。」是以蕭何定法令
 而受封,叔孫通以制儀為奉常。立功立事,古之所重。其明議損益,以成一代準式。周、漢有貢士之條,魏立九品之選,二者孰愈,亦可詳聞。



 群下議多不同,乃止。



 超母妻既先在長安,為姚興所拘,責超稱籓,求太樂諸伎,若不可,使送吳口千人。超下書遣群臣詳議。左僕射段暉議曰:「太上囚楚,高祖不迴。今陛下嗣守社稷,不宜以私親之故而降統天之尊。又太樂諸伎,皆是前世伶人,不可與彼,使移風易俗,宜掠吳口與之。」尚書張華曰:「若侵掠吳邊,必成鄰怨。此既能往,彼亦能來,兵連禍結,非國之福也。昔孫權重黎庶之命,屈己以臣魏;惠施惜愛子
 之頭,舍志以尊齊。況陛下慈德在秦,方寸崩亂,宜暫降大號,以申至孝之情。權變之道,典謨所許。韓範智能迴物,辯足傾人,昔與姚興俱為秦太子中舍人,可遣將命,降號脩和。所謂屈於一人之下,申於萬人之上也。」超大悅曰:「張尚書得吾心矣。」使範聘于興。及至長安,興謂範曰:「封愷前來,燕王與朕抗禮。及卿至也,款然而附。為依春秋以小事大之義?為當專以孝敬為母屈也?」範曰:「周爵五等,公侯異品,小大之禮,因而生焉。今陛下命世龍興,光宅西秦,本朝主上承祖宗遺烈,定鼎東齊,中分天曜,南面並帝。通聘結好,義尚廉沖,便至矜誕,茍折行
 人,殊似吳、晉爭盟,滕、薛競長,恐傷大秦堂堂之盛,有損皇燕巍巍之美,彼我俱失,竊未安之。」興怒曰:「若如卿言,便是非為大小而來。」範曰:「雖由大小之義,亦緣寡君純孝過於重華,願陛下體敬親之道,霈然垂愍。」興曰:「吾久不見賈生,自謂過之,今不及矣。」於是為範設舊交之禮,申敘平生,謂範曰:「燕王在此,朕亦見之,風表乃可,於機辯未也。」範曰:「大辯若訥,聖人美之,況爾日龍潛鳳戢,和光同塵,若使負日月而行,則無繼天之業矣。」興笑曰:「可謂使乎延譽者也。」範承間逞說,姚興大悅,賜範千金,許以超母妻還之。慕容凝自梁父奔于姚興,言於興曰:「燕
 王稱籓,本非推德,權為母屈耳。古之帝王尚興師徵質,豈可虛還其母乎!母若一還,必不復臣也。宜先制其送伎,然後歸之。」興意乃變,遣使聘於超。超遣其僕射張華、給事中宗正元入長安,送太樂伎一百二十人於姚興。興大悅,延華入宴。酒酣,樂作,興黃門侍郎尹雅謂華曰:「昔殷之將亡,樂師歸周;今皇秦道盛,燕樂來庭。廢興之兆,見於此矣。」華曰:「自古帝王,為道不同,權譎之理,會於功成。故老子曰:『將欲取之,必先與之。』今總章西入,必由余東歸,禍福之驗,此其兆乎!」興怒曰:「昔齊、楚競辯,二國連師。卿小國之臣,何敢抗衡朝士!」華遜辭曰:「奉使之始,
 實願交歡上國,上國既遺小國之臣,辱及寡君社稷,臣亦何心,而不仰酬!」興善之,於是還超母妻。



 義熙三年,追尊其父為穆皇帝,立其母段氏為皇太后,妻呼延氏為皇后。祀南郊,將登壇,有獸大如馬,狀類鼠而色赤,集於圓丘之側,俄而不知所在。須臾大風暴起,天地晝昏,其行宮習儀皆振裂。超懼,密問其太史令成公綏,對曰:「陛下信用姦臣,誅戮賢良,賦斂繁多,事役殷苦所致也。」超懼而大赦,譴責公孫五樓等。俄而復之。是歲廣固地震,天齊水湧,井水溢,女水竭,河、濟凍合,而澠水不冰。



 超正旦朝群臣於東陽殿,聞樂作,歎音佾不備,悔送伎於姚
 興,遂議入寇。其領軍韓言卓諫曰:「先帝以舊京傾沒,輯翼三齊,茍時運未可,上智輟謀。今陛下嗣守成規,宜閉關養士,以待賦釁,不可結怨南鄰,廣樹仇隙。」超曰:「我計已定,不與卿言。」於是遣其將斛穀提、公孫歸等率騎寇宿豫,陷之,執陽平太守劉千載、濟陰太守徐阮,大掠而去。簡男女二千五百,付太樂教之。



 時公孫五樓為侍中、尚書,領左衛將軍,專總朝政,兄歸為冠軍、常山公,叔父頹為武衛、興樂公。五樓宗親皆夾輔左右,王公內外無不憚之。



 超論宿豫之功,封斛穀提等並為郡、縣公。慕容鎮諫曰:「臣聞縣賞待勳,非功不侯,今公孫歸結禍延兵,殘
 賊百姓,陛下封之,得無不可乎!夫忠言逆耳,非親不發。臣雖庸朽,忝國戚籓,輒盡愚款,惟陛下圖之。」超怒,不答,自是百僚杜口,莫敢開言。



 尚書都令史王儼諂事五樓,遷尚書郎,出為濟南太守,入為尚書左丞,時人為之語曰:「欲得侯,事五樓。」



 又遣公孫歸等率騎三千入寇濟南,執太守趙元,略男女千餘人而去。劉裕率師將討之,超引見群臣于節陽殿,議距王師。公孫五樓曰:「吳兵輕果,所利在戰,初鋒勇銳,不可爭也。宜據大峴,使不得入,曠日延時,沮其銳氣。可徐簡精騎二千,循海而南。絕其糧運,別敕段暉率兗州之軍,緣山東下。腹背擊之,上策也。
 各命守宰,依險自固,校其資儲之外,餘悉焚蕩,芟除粟苗,使敵無所資。堅壁清野,以待其釁,中策也。縱賊入峴,出城逆戰,下策也。」超曰:「京都殷盛,戶口眾多,非可一時入守。青苗布野,非可卒芟。設使芟苗城守,以全性命,朕所不能。今據五州之彊,帶山河之固,戰車萬乘,鐵馬萬群,縱令過峴,至於平地,徐以精騎踐之,此成擒也。」賀賴盧苦諫,不從,退謂五樓曰:「上不用吾計,亡無日矣。」慕容鎮曰:「若如聖旨,必須平原用馬為便,宜出峴逆戰,戰而不勝,猶可退守。不宜縱敵入峴,自貽窘逼。昔成安君不守井陘之關,終屈於韓信;諸葛瞻不據束馬之險,卒擒
 於鄧艾。臣以為天時不如地利,阻守大峴,策之上也。」超不從。鎮出,謂韓言卓曰:「主上既不能芟苗守險,又不肯徙人逃寇,酷似劉璋矣。今年國滅,吾必死之,卿等中華之士,復為文身矣。」超聞而大怒,收鎮下獄。乃攝莒、梁父二戍,脩城隍,簡士馬,畜銳以待之。



 其夏,王師次東莞,超遣其左軍段暉、輔國賀賴盧等六將步騎五萬,進據臨朐。俄而王師度峴,超懼,率卒四萬就暉等于臨朐,謂公孫五樓曰:「宜進據川源,晉軍至而失水,亦不能戰矣。」五樓馳騎據之。劉裕前驅將軍孟龍符已至川源,五樓戰敗而返。裕遣諮議參軍檀韶率銳卒攻破臨朐,超大懼,單
 騎奔段暉於城南。暉眾又戰敗,裕軍人斬暉。超又奔還廣固,徙郭內人入保小城,使其尚書郎張綱乞師于姚興。赦慕容鎮,進錄尚書、都督中外諸軍事。引見群臣,謝之曰:「朕嗣奉成業,不能委賢任善,而專固自由,覆水不收,悔將何及!智士逞謀,必在事危,忠臣立節,亦在臨難,諸君其勉思六奇,共濟艱運。」鎮進曰:「百姓之心,係於一人。陛下既躬率六軍,身先奔敗,群臣解心,士庶喪氣,內外之情,不可復恃。如聞西秦自有內難,恐不暇分兵救人,正當更決一戰,以爭天命。今散卒還者,猶有數萬,可悉出金帛、宮女,餌令一戰。天若相我,足以破賊。如其不
 濟,死尚為美,不可閉門坐受圍擊。」司徒慕容惠曰:「不然。今晉軍乘勝,有陵人之氣,敗軍之將,何以禦之!秦雖與勃勃相持,不足為患。且二國連橫,勢成脣齒,今有寇難,秦必救我。但自古乞援,不遣大臣則不致重兵,是以趙隸三請,楚師不出;平原一使,援至從成。尚書令韓範德望具瞻,燕秦所重,宜遣乞援,以濟時難。」於是遣範與王蒲乞師于姚興。



 未幾,裕師圍城,四面皆合。人有竊告裕軍曰:「若得張綱為攻具者,城乃可得耳。」是月,綱自長安歸,遂奔于裕。裕令綱周城大呼曰:「勃勃大破秦軍,無兵相救。」超怒,伏弩射之,乃退。右僕射張華、中丞封愷並
 為裕軍所獲。裕令華、愷與超書,勸令早降。超乃遺裕書,請為籓臣,以大峴為界,并獻馬千區,以通和好,裕弗許。江南繼兵相尋而至。尚書張俊自長安還,又降于裕,說容曰:「今燕人所以固守者,外杖韓範,冀得秦援。範既時望,又與姚興舊暱,若勃勃敗後,秦必救燕,宜密信誘範,啖以重利,範來則燕人絕望,自然降矣。」裕從之,表範為散騎常侍,遺範書以招之。時姚興乃遣其將姚彊率步騎一萬,隨範就其將姚紹于洛陽,並兵來援。會赫連勃勃大破秦軍,興追彊還長安。範歎曰:「天其滅燕乎!」會得裕書,遂降於裕。裕謂範曰:「卿欲立申包胥之功,何以虛
 還也?」範曰:「自亡祖司空世荷燕寵,故泣血秦庭,冀匡禍難。屬西朝多故,丹誠無效,可謂天喪弊邑而贊明公。智者見機而作,敢不至乎!」翌日,裕將範循城,由是人情離駭,無復固志,裕謂範曰:「卿宜至城下,告以禍福。」範曰:「雖蒙殊寵,猶未忍謀燕。」裕嘉而不彊,左右勸超誅範家,以止後叛。超知敗在旦夕,又弟言卓盡忠無貳,故不罪焉。是歲東萊雨血,廣固城門鬼夜哭。



 明年朔旦,超登天門,朝群臣于城上,殺馬以饗將士,文武皆有遷授。超幸姬魏夫人從超登城,見王師之盛,握超手而相對泣,韓言卓諫曰:「陛下遭百六之會,正是勉彊之秋,而反對女子悲泣,
 何其鄙也!」超拭目謝之。其尚書令董銳勸超出降,超大怒,繫之於獄。於是賀賴盧、公孫五樓為地道出戰王師,不利。河間人玄文說裕曰:「昔趙攻曹嶷,望氣者以為澠水帶城,非可攻拔,若塞五龍口,城必自陷。石季龍從之,而嶷請降。後慕容恪之圍段龕,亦如之,而龕降。降後無幾,又震開之。今舊基猶在,可塞之。」裕從其言。至是,城中男女患腳弱病者太半。超輦而升城,尚書悅壽言於超曰:「天地不仁,助寇為虐,戰士尪病,日就凋隕,守困窮城,息望外援,天時人事,亦可知矣。茍歷運有終,堯、舜降位,轉禍為福,聖達以先。宜追許、鄭之蹤,以全宗廟之重。」超
 歎曰:「廢興,命也。吾寧奮劍決死,不能銜璧求生。」於是張綱為裕造衝車,覆以版屋,蒙之以皮,並設諸奇巧,城上火石弓矢無所施用;又為飛樓、懸梯、木幔之屬,遙臨城上。超大怒,懸其母而支解之。城中出降者相繼。裕四面進攻,殺傷其眾,悅壽遂開門以納王師。超與左右數十騎出亡,為裕軍所執。裕數之以不降之狀,超神色自若,一無所言,惟以母託劉敬宣而已。送建康市斬之,時年二十六。在位六年。



 德以安帝隆安四年僭位,至超二世,凡十一年,以義熙六年滅。



 慕容鐘,字道明,德從弟也。少有識量,喜怒不形於色,機
 神秀發,言論清辯。至於臨難對敵,智勇兼濟,累進奇策,德用之頗中。由是政無大小,皆以委之,遂為佐命無勳。後公孫五樓規挾威權,慮鐘抑己,因勸超誅之,鐘遂謀反。事敗,奔于姚興,興拜始平太守、歸義侯。



 封孚,字處道,渤海蓚人也。祖悛,振威將軍。父放,慕容之世吏部尚書。孚幼而聰敏和裕,有士君子之稱。寶僭位,累遷吏部尚書。及蘭汗之篡,南奔辟閭渾,渾表為渤海太守。德至莒城,孚出降,德曰:「朕平青州,不以為慶,喜於得卿也。」常外總機事,內參密謀,雖位任崇重,謙虛博
 納,甚有大臣之體。及超嗣位,政出權嬖,多違舊章,軌憲日頹,殘虐滋甚,孚屢盡匡救,超不能納也。後臨軒謂孚曰:「朕於百王可方誰?」孚對曰:「桀紂之主。」超大慚怒。孚徐步而出,不為改容。司空鞠仲失色,謂孚曰:「與天子言,何其亢厲,宜應還謝。」孚曰:「行年七十,墓木已拱,惟求死所耳。」竟不謝。以超三年死于家,時年七十一。文筆多傳于世。



 史臣曰:慕容德以季父之親,居鄴中之重,朝危未聞其節,君存遽踐其位,豈人理哉!然稟倜儻之雄姿,韞縱橫之遠略,屬分崩之運,成角逐之資,跨有全齊,竊弄神器,
 撫劍而爭衡秦、魏,練甲而志靜荊、吳,崇儒術以弘風,延讜言而勵己,觀其為國,有足稱焉。



 超繼已成之基,居霸者之業,政刑莫恤,畋游是好,杜忠良而讒佞進,暗聽受而勛戚離,先緒俄頹,家聲莫振,陷宿豫而貽禍,啟大峴而延敵,君臣就虜,宗廟為墟。跡其人謀,非不幸也。



 贊曰:德實奸雄,轉敗為功。奄有青土,淫名域中。超承偽祚,撓其國步。廟失良籌,庭悲沾露。



\end{pinyinscope}