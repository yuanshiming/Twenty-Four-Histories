\article{志第七}

\begin{pinyinscope}

 律歷中



 昔
 者聖人擬宸極以運璿璣,揆天行而序景曜,分辰野,辨躔歷,敬農時,興物利,皆以繫順兩儀,紀綱萬物者也。然則觀象設卦,扐閏成爻,歷數之原,存乎此也。逮乎炎帝,分八節以始農功,軒轅紀三綱而闡書契,乃使羲和占日,常儀占月,臾區占星氣,伶倫造律呂,大撓造甲子,隸首作算數。容成綜斯六術,考定氣象,建五行,察發斂,
 起消息,正閏餘,述而著焉,謂之《調歷》。洎于少昊則鳳鳥司歷,顓頊則南正司天,陶唐則分命羲和,虞舜則因循堯法。及夏殷承運,周氏應期,正朔既殊,創法斯異。《傳》曰:「火出,於夏為三月,於商為四月,於周為五月。」是故天子置日官,諸侯有日御,以和萬國,以協三辰。至乎寒暑晦明之徵,陰陽生殺之數,啟閉升降之紀,消息盈虛之節,皆應躔次而無淫流,故能該浹生靈,堪輿天地。周德既衰,史官失職,疇人分散,禨祥不理。秦并天下,頗推五勝,自以獲水德之瑞,用十月為正。漢氏初興,多所未暇,百有餘載,襲秦正朔。爰及武帝,始詔司馬遷等議造《漢歷》,
 乃行夏正。其後劉歆更造夏《三統》,以說《左傳》,辯而非實,班固惑之,采以為志。逮光武中興,太僕朱浮數言歷有乖謬,於時天下初定,未能詳考。至永平之末,改行《四分》,七十餘年,儀式乃備。及光和中,乃命劉洪、蔡邕共修律歷,其後司馬彪因之,以繼班史。今採魏文黃初已後言歷數行事者,以續司馬彪云。



 漢靈帝時,會稽東部尉劉洪,考史官自古迄今歷注,原其進退之行,察其出入之驗,視其往來,度其終始,始悟《四分》於天疏闊,皆斗分太多故也。更以五百八十九為紀法,百四十五為斗分,作《乾象法》,冬至日日在斗二十
 二度,以術追日、月、五星之行,推而上則合於古,引而下則應於今。其為之也,依《易》立數,遁行相號,潛處相求,名為《乾象歷》。又創制日行遲速,兼考月行,陰陽交錯於黃道表裏,日行黃道,於赤道宿度復有進退。方於前法,轉為精密矣。獻帝建安元年,鄭玄受其法,以為窮幽極微,又加注釋焉。



 魏文帝黃初中,太史令高堂隆復詳議歷數,更有改革。太史丞韓翊以為《乾象》減斗分太過,後當先天,造《黃初歷》,以四千八百八十三為紀法,千二百五為斗分。



 其後尚書令陳群奏,以為:「歷數難明,前代通儒多共紛爭。《
 黃初》之元以《四分歷》久遠疏闊,大魏受命,宜改歷明時,韓翊首建,猶鞏不審,故以《乾象》互相參校。其所校日月行度,弦望朔晦,歷三年,更相是非,無時而決。案三公議皆綜盡典理,殊塗同歸,欲使效之璿璣,各盡其法,一年之間,得失足定。」奏可。



 太史令許芝云:「劉洪月行術用以來且四十餘年,以復覺失一辰有奇。」



 孫飲議:「史遷造《太初》,其後劉歆以為疏,復為《三統》。章和中,改為《四分》,以儀天度,考合符應,時有差跌,日蝕覺過半日。至熹平中,劉洪改為《乾象》,推天七曜之符,與天地合其敘。」



 董巴議云:「聖人迹太陽於晷景,效太陰於弦望,明五星於見伏,正
 是非於晦朔。弦望伏見者,歷數之綱紀,檢驗之明者也。」



 徐岳議:「劉洪以歷後天,潛精內思二十餘載,參校漢家《太初》、《三統》、《四分》歷術,課弦望於兩儀郭間。而月行九歲一終,謂之九道;九章,百七十一歲,九道小終;九九八十一章,五百六十七分而九終,進退牛前四度五分。學者務追合《四分》,但減一道六十三分,分不下通,是以疏闊,皆由斗分多故也。課弦望當以昏明度月所在,則知加時先後之意,不宜用兩儀郭間。洪加《太初》元十二紀,減十斗下分,元起己丑,又為月行遲疾交會及黃道去極度、五星術,理實粹密,信可長行。今韓翊所造,皆用洪法,
 小益斗下分,所錯無幾。翊所增減,致亦留思,然十術新立,猶未就悉,至於日蝕,有不盡效。效歷之要,要在日蝕。熹平之際,時洪為郎,欲改《四分》,先上驗日蝕:日蝕在晏,加時在辰,蝕從下上,三分侵二。事御之後如洪言,海內識真,莫不聞見,劉歆以來,未有洪比。夫以黃初二年六月二十九日戊辰加時未日蝕,《乾象術》加時申半強,於消息就加未,《黃初》以為加辛強,《乾象》後天一辰半強為近,《黃初》二辰半為遠,消息與天近。三年正月丙寅朔加時申北日蝕,《黃初》加酉弱,《乾象》加午少,消息加未,《黃初》後天半辰近,《乾象》先天二辰少弱,於消息先天一辰強,
 為遠天。三年十一月二十九日庚申加時西南維日蝕,《乾象》加未初,消息加申,《黃初》加未強,《乾象》先天一辰遠,《黃初》先天半辰近,消息《乾象》近中天。二年七月十五日癸未,日加壬月加丙蝕,《乾象》月加申,消息加未,《黃初》月加子強,入甲申日,《乾象》後天二辰,消息後一辰為近,《黃初》後天六辰遠。三年十一月十五日乙巳,日加丑月加未蝕,《乾象》月加巳半,於消息加午,《黃初》以丙午月加酉強,《乾象》先天二辰近,《黃初》後天二辰強為遠,於消息於《乾象》先一辰。凡課日月蝕五事,《乾象》四遠,《黃初》一近。」



 翊於課難徐岳:「《乾象》消息但可減,不可加。加之無可說,不可用。」
 嶽云:本術自有消息,受師法,以消息為奇,辭不能改,故列之正法消息。翊術自疏。



 木以三年五月二十四日丁亥晨見;《黃初》五月十七日庚辰見,先七日;《乾象》五月十五日戊寅見,先九日。



 土以二年十一月二十六日壬辰見;《乾象》十一月二十一日丁亥見,先五日;《黃初》十一月十八日甲申見,先八日。



 土以三年十月十一日壬申伏;《乾象》同,壬申伏;《黃初》已下十月七日戊辰伏,先四日。



 土以三年十一月二十二日壬子見;《乾象》十一月十五日乙巳見,先七日;《黃初》十一月十二日壬寅見,先十日。



 金以三年閏六月十五日丁丑晨伏;《乾象》六月二十五日戊午伏,先十九日;《黃初》六月二十二日乙卯伏,先二十三日。



 金以三年九月十一日壬寅見;《乾象》以八月十八日庚辰見,先二十三日;《黃初》八月十五日丁丑見,先二十五日。



 水以二年十一月十七日癸未晨見;《乾象》十一月十三日己卯見,先四日;《黃初》十一月十二日戊寅見,先五日。



 水以二年十二月十三日己酉晨伏;《乾象》十二月十五日辛亥伏,後二日;《黃初》十二月十四日庚戌伏,後一日。



 水以三年五月十八日辛巳夕見;《乾象》亦以五月十八日見;《黃初》五月十七日庚辰見,先一日。



 水以三年六月十三日丙午伏;《乾象》六月二十日癸丑伏,後七日;《黃初》六月十九日壬子伏,後六日。



 水以三年閏六月二十五日丁亥晨見;《乾象》以閏月九日辛未見,先十六日;《黃初》閏月八日庚午見,先十七日。



 水以三年七月七日己亥伏;《乾象》七月十一日癸卯伏,後四日;《黃初》以七月十日壬寅伏,後三日。



 水以三年十一月日於晷度十四日甲辰伏;《乾象》以十一月九日己亥伏,先五日;《黃初》十一月八日戊戌伏,先六日。



 水以三年十二月二十八日戊子夕見;二歷同以十二月壬申見,俱先十六日。



 凡四星見伏十五;《乾象》七近二中,《黃初》五近一中。



 郎中李恩議:「以太史天度與相覆校,二年七月、三年十一月望與天度日皆差異,月蝕加時乃後天六時半,非從三度之謂,定為後天過半日也。」



 董巴議曰:「昔伏羲始造八卦,作三畫,以象二十四氣。黃帝因之,初作《調歷》。歷代十一,更年五千,凡有七歷。顓頊以今之孟春正月為元,其時正月朔旦立春,五星會于廟,營室也,冰凍始泮,蟄蟲始發,雞始三號,天曰作時,地曰作昌,人曰作樂,鳥獸萬物莫不應和,故顓頊聖人為歷宗也。湯作《殷歷》弗復以正月朔旦立春為節也,更以十一月朔旦冬至
 為元首,下至周魯及漢,皆從其節,據正四時。夏為得天,以承堯舜,從顓頊故也。《禮記》大戴曰虞夏之歷,建正於孟春,此之謂也。」



 楊偉請:「六十日中疏密可知,不待十年。若不從法,是校方員棄規矩,考輕重背權衡,課長短廢尺寸,論是非違分理。若不先定校歷之本法,而懸聽棄法之末爭,則孟軻所謂『方寸之基,可使高於岑樓』者也。今韓翊據劉洪術者,知貴其術,珍其法。而棄其論,背其術,廢其言,違其事,是非必使洪奇妙之式不傳來世。若知而違之,是挾故而背師也;若不知而據之,是為挾不知而罔知也。」校議未定,會帝崩而寢。



 至明帝景初元年,尚
 書郎楊偉造《景初歷》。表上,帝遂改正朔,施行偉歷,以建丑之月為正,改其年三月為孟夏,其孟、仲、季月雖與夏正不同,至於郊祀蒐狩,班宣時令,皆以建寅為正。三年正月帝崩,復用夏正。



 其劉氏在蜀,仍漢《四分歷》。吳中書令闞澤受劉洪《乾象法》於東萊徐岳,又加解注。中常待王蕃以洪術精妙,用推渾天之理,以制儀象及論,故孫氏用《乾象歷》,至吳亡。



 武帝踐阼,泰始元年,因魏之《景初歷》,改名《泰始歷》。楊偉推五星尤疏闊,故元帝渡江左以後,更以《乾象》五星法代偉歷。自黃初已後,改作歷術,皆斟酌《乾象》所減斗分、朔餘、月行陰陽遲疾,以求折衷。洪
 術為後代推步之師表,故先列之云。



 乾象歷



 上元己丑以來,至建安十一年丙戌,歲積七千三百七十八年。



 乾法,千一百七十八。



 會通,七千一百七十一。



 紀法,五百八十九。



 周天,二十一萬五千一百三十。



 通法,四萬三千二十六。



 通數,三十一。



 日法,千四百五十七。



 歲中,十二。



 餘數,三千九十。



 章歲,十九。



 沒法,百三。



 章閏,七。



 會數,四十七。



 會歲,八百九十三。



 章月,二百三十五。



 會率,千八百八十二。



 朔望合數,九百四十一。



 會月,萬一千四十五。



 紀月,七千二百八十五。



 元月,一萬四千五百七十。



 月周,七千八百七十四。



 小周,二百五十四。



 推入紀



 置上元盡所求年,以乾法除之,不滿乾法,以紀法除之,餘不滿紀法者,入內紀甲子年也。滿法去之,入外紀甲午年也。



 推朔



 置入紀年,外所求,以章月乘之,章歲而一,所得為定積月,不盡為閏餘。閏餘十二以上,歲有閏。以通法乘定積月,為假積日,滿日法為定積日,不盡為小餘。以六旬去積日為大餘,命以所入紀,算外,所求年天正十一月朔日也。



 求次月,加大餘二十九,小餘七百七十三,小餘滿日法從大餘。小餘六百八十四已上,其月大。



 推冬至



 置入紀年,外所求,以餘數乘之,滿紀法為大餘,不盡為小
 餘。以六旬去之,命以紀,算外,天正冬至日也。



 求二十四氣



 置冬至小餘,加大餘十五,小餘五百一十五,滿二千三百五十六從大餘,命如法。



 推閏月



 以閏餘減章歲,餘以歲中乘之,滿章閏為一月。不盡,半法己上亦一,有進退,以無中月。



 推弦望



 加大餘七,小餘五百五十七半,小餘如日法從大餘,餘命如前,得上弦。又加得望,又加得下弦,又加得後月朔。
 其弦望定小餘四百一以下,以百刻乘之,滿日法得一刻,不盡什之,求分,以課所近節氣夜漏未盡,以算上為日。



 推沒



 置入紀年,外所求,以餘數乘之,滿紀法為積沒,有餘加盡積為一。以會通乘之,滿沒法為大餘,不盡為小餘。大餘命以紀,算外,冬至後沒日。



 求次沒,加大餘六十九,小餘六十四,滿其法從大餘,無分為滅。



 推日度



 以紀法乘積日,滿周天去之,餘以紀法除之,所得為度。
 命度以牛前五度起,宿次除之,不滿宿,即天正朔夜半日所在。



 求次日,加一度,經斗除分;分少,損一度為紀法,加焉。



 推月度



 以月周乘積日,滿周天去之,餘滿紀法為度,不盡為分,命如上,則天正朔夜半月所在度。



 求次月,小月加度二十二,分二百五十八。大月又加一日,度十三,分二百一十七,滿法得一度。其冬下旬,月在張、心署之。



 推合朔度



 以章歲乘朔小餘,滿會數為大分;不盡,小分。以大分從朔夜半日分,滿紀法從度,命如前,天正合朔日月所共會也。



 求次月,加度二十九,大分三百一十二,小分滿會數從大分,大分滿紀法後度,經斗除大分。



 求弦望日所在度,加合朔度七,分二百二十五,小分十七半,大小分及度命如前,則上弦日所在度。又加得望、下弦、後月合。



 求弦望月行所在度,加合朔度九十八,大分四百八,小分四十一,大小分及度命如前合朔,則上弦月所在。又
 加得望、下弦、後月合。



 求日月昏明度,日以紀法,月以月周,乘所近節氣夜漏,二百而一為明分。日以減紀法,月以減月周,餘為昏分。各以加夜半,如法為度。



 推月蝕



 置上元年,外所求,以會歲去之,其餘年以會率乘之,如會歲為積蝕,有餘加積一。會月乘之,如會率為積月,不盡為月餘。以章閏乘餘年,滿章歲為積閏,以減積月,餘以歲中去之,不盡,數起天正。



 求次蝕,加五月,月餘千六百三十五,滿會率得一月,
 月以望。



 推卦用事日



 因冬至大餘,倍其小餘,坎用事日也。加小餘千七十五,滿乾法從大餘,中孚用事日也。



 求次卦,各加大餘六,小餘百三。其四正各因共中日,而倍其小餘。



 推五行用事



 置冬至大小餘,加大餘二十七,小餘九百二十七,滿二千三百五十六從大餘,得土用事日也。加大餘十八,小餘六百一十八,得立春木用事日。加大餘七十三,小餘
 百一十六,復得土。又加土如得其火,金、水放此。



 推加時



 以十二乘小餘,滿其法得一辰,數從子起,算外,朔、弦、望以定小餘。



 推漏刻



 以百乘小餘,滿其法得一刻,不盡什之,求分,課所近節氣,起夜分盡;夜上水未盡,以所近言之。



 推有進退,進加退減所得也。進退有差,起二分度後,率四度轉增少,少每半者,三而轉之,差滿三止,歷五度而減如初。



 月行三道術



 月行遲疾,周進有恆。會數從天地凡數,乘餘率自乘,如會數而一,為過周分。以從周天,月周除之,歷日數也。遲疾有衰,其變者勢也。以衰減加月行率,為日轉度分。衰左右相加,為損益率。益轉相益,損轉相損,盈縮積也。半小周乘通法,如通數而一,以歷周減焉,為朔行分也。日轉度分列衰損益率盈縮積月行分一日十四度十分一退減益二十二盈初二百七十六二日十四度九分二退減益二十一
 盈二十二二百七十五三日十四度七分三退減益十九盈四十三二百七十三四日十四度四分四退減益十六盈六十二二百七十五日十四度四退減益十二盈七十八二百六十六六日十三度十五分四退減益八盈九十二百六十二七日十三度十一分四退減益四
 盈九十八二百五十八八日十三度七分四退減損盈百二二百五十四九日十三度三分四退加損四盈百二二百五十十日十二度十八分三退加損八盈九十八二百四十六十一日十二度十五分四退加損十一盈九十二百四十三十二日十二度十一分三退加損十五
 盈七十九二百三十九十三日十二度八分二退加損十八盈六十四二百三十六十四日十二度六分一退加損二十盈四十六二百三十四十五日十二度五分一進減損二十一盈二十六二百三十三十六日十二度六分二進減損二十損不足反減五為益,盈有五謂益



 而損縮初二十,故不足。



 盈五縮初二百三十四十七日十二度八分三進減益十八
 縮十五二百三十六十八日十二度十一分四進減益十五縮二十三二百三十九十九日十二度十五分三進減益十一縮四十八二百四十三二十日十二度十八分四進減益八縮五十九二百四十六二十一日十三度三分四進減益四縮六十七二百五十二十二日十三度七分四進加損
 縮七十一二百五十四二十三日十三度十一分四進加損四縮七十一二百五十八二十四日十三度十五分四進加損八縮六十七二百六十二二十五日十四度四進加損十二縮五十九二百六十六二十六日十四度四分三進加損十六縮四十七二百七十二十七日十四度七分三歷初進加損十九
 縮三十一二百七十三



 三大周日周日十四度九分少進加損二十一縮十二二百七十五



 周日分,三千三百三。



 周虛,二千六百六十六。



 周日法,五千九百六十九。



 通周,十八萬五千三十九。



 歷周,十六萬四千四百六十六。



 少大法,一千一百一。



 朔行大分,萬一千八百一。



 小分,二十五。



 周半,一百二十七。



 推合朔入歷



 以上元積月乘朔行大小分,小分滿通數三十一從大分,大分滿歷周去之,餘滿周法得一日,不盡為日餘。日餘命算外,所求合朔入歷也。



 求次月,加一日,日餘五千八百三十二,小分二十五。



 求弦望,各加七日,日餘二千二百八十三,小分二十九半,分各如法成日,日滿二十七日去之。餘如周分。不足除,減一日,加周虛。



 求弦望定大小餘



 置所入歷盈縮積,以通周乘之為實。令通數乘日餘分,以乘損益率,以損益實,為加時盈縮也。章歲減月行分,乘周半為差法,以除之,所得盈減縮加大小餘,如日法盈不足,朔加時在前後日。弦望進退大餘,為定小餘。



 求朔弦望加時定度



 以章歲乘加時盈縮,差法除之,所得滿會數為盈縮大小分,以盈減縮加本日月所在,盈不足,以紀法進退度,為日月所在定度分。



 推月行夜半入歷



 以周半乘朔小餘,如通數而一,以減入歷日餘。餘不足,
 加周法而減焉,卻一日。卻得周日加其分,即得夜半入歷。



 求次日,轉一日,因日餘到二十七日,日餘滿周日分去之,不直周日也。其不滿直之,加周虛於餘,餘皆次日入歷日餘也。



 求月夜半定度



 以夜半入歷日餘,乘損益率,如周法得一,不盡為餘,以損益縮積,餘無所損,破全為法損之,為夜半盈縮也。滿章歲為度,不盡為分。通數乘分及餘,餘如周法從分,分滿紀法從度,以盈加縮減本夜半度及餘,為定度。



 求變衰法



 以入歷日餘乘列衰,如周法得一,不盡為餘,即穀知其日變衰也。



 求次歷



 以周虛乘列衰,如周法為常數,歷竟,輒以加變衰,滿列衰去之,轉為次歷變衰也。



 求次日夜半定度



 以變衰進加退減歷日轉分,分盈不足,章歲出入度也。通數乘分及餘,而日轉加夜定度,為次日也。竟歷不直周日,減餘三十八,乃以通數乘之,直周日者加餘八
 百三十七,又以少大分八百九十九,加次歷變衰,轉求如前。



 求次日夜半盈縮



 以變衰減加損益率,為變損益率,而以轉損益夜半盈縮。歷竟損不足,反減為入次歷,減加餘如上數。



 求昏明月度



 以歷月行分乘所近節氣夜漏,二百而一為明分。以減月行分為昏分。分如章歲為度,以通數乘分,以加夜半定度,為昏明定度。餘分半法以上成,不滿廢之。



 求月行遲疾



 月經四表,出入三道,交錯分天,以月率除之,為歷之日。周天乘朔望合,如會月而一,朔合分也。通數乘合數,餘如會數而一,退分也。以從月周,為日進分。會數而一,為差率也。



 陰陽歷衰損益率兼數



 一日一減益十七初



 二日限餘千二百九十微分四百五十七



 一減益十六十七



 此為前限



 三日三減益十五三十三



 四日四減益十二四十八



 五日四減益八六十



 六日三減益四六十八



 七日三減減不足,反損為加,謂益有一,當減三,為不足



 益一七十二



 八日四加損二過極損之,謂月行半周,



 度已過極,則當損之。



 九日四加損六七十一



 十日三加損十六十五



 十一日二加損十三五十五



 十二日一加損十五四十二



 十三日限餘三千九百一十二,微分一千七百五十二。



 此為後限



 一加歷初大,分日。損十六二十七



 分日五千二百而三少加少者損十六大十一



 少大法,四百七十三。



 歷周,十萬七千五百六十五。



 差率,萬一千九百八十六。



 朔合分,萬八千三百二十八。



 微分,九百一十四。



 微分法,二千二百九。



 推朔入陰陽歷



 以會月去上元積月,餘以朔合分及微分各乘之,微分滿其法從合分,合分滿周天去之,其餘不滿歷周者,為入陽歷;滿去之,餘為入陰歷。餘皆如月周得一日,算外,所求月合朔入歷,不盡為日餘。



 求次月



 加二日,日餘二千五百八十,微分九百一十四,如法成日,滿十三去之,除餘如分日。陰陽歷竟互入端,入歷在前限餘前,後限餘後者月行中道也。



 求朔望定數



 各置入遲疾歷盈縮大小分,會數乘小分為微分,盈減縮加陰陽日餘,日餘盈不足,進退日而定。以定日餘乘損益率,如月周得一,以損益兼數,為加時定數。



 推夜半入歷



 以差率乘朔小餘,如微分法得一,以減入歷日餘,不足,
 加月周而減之,卻一日。卻得分日加其分,以會數約微分為小分,即朔日夜半入歷。



 求次日,加一日,日餘三十一,小分三十一,小分如會數從餘,餘滿月周去之,又加一日,歷竟下,日餘滿分日去之,為入歷初也。不滿分日者直之,加餘二千七百二,小分三十一,為入次歷。



 求夜半定日



 以通數乘入遲疾歷夜半盈縮及餘,餘滿周半為小分,以盈加縮減入陰陽日餘,日餘盈不足,以月周進退日而定也。以定日餘乘損益率,如月周得一,以損益兼數,為夜半定數也。



 求昏明數



 以損益率乘所近節氣夜漏,二百而一為明,以減損益率為昏,而以損益夜半數為昏明定數。



 求月去極度



 置加時若昏明定數,以十二除之為度,其餘三而一為少,不盡一為強,二少弱也,所得為月去黃道度也。其陽歷以加日所在黃道歷去極度,陰歷以減之,則月去極度。強正弱負,強弱相并,同名相從,異名相消。其相減也,同名相消,異名相從,無對互之,二強進少而弱。



 上元己丑以來,至建安十一年丙戌,歲積七千三百七十八。



 己丑戊寅丁卯丙辰乙巳甲午癸未



 壬申辛酉庚戌己亥戊子丁丑丙寅



 推五星



 五行:木,歲星;火,熒惑;土,填星;金,太白;水,辰星。各以終日與天度相約,為周率、日率。章歲乘周,為月法。章月乘日,為月分。分如法,為月數。通數乘月法,日度法也。斗分乘周率,為斗分。日度法用紀法乘周率,故此同以分乘之。



 五星朔大餘、小餘。以通法各乘月數,日法各除之,為大餘,不盡為小餘。以六十去大餘。



 五星入月日、日餘。各以通法乘月餘,以合月法乘朔小餘,並之,會數約之,所得各以日度法除
 之,則皆是。



 五星度數、度餘。減多為度餘分,以周天乘之,以日度法約之,所得為度,不盡為度餘,過周天去之及斗分。



 紀月,七千二百八十五。



 章閏,七。



 章月,二百三十五。



 歲中,十二。



 通法,四萬三千二十六。



 日法,千四百五十七。



 會數,四十七。



 周天,二十一萬五千一百三十。



 斗分,一百四十五。



 木:周率,六千七百二十二。



 日率,七千三百四十一。



 合月數,十三。



 月餘,六萬四千八百一。



 合月法,十二萬七千七百一十八。



 日度法,三百九十五萬九千二百五十八。



 朔大餘,二十三。



 朔小餘,一千三百七。



 入月日,十五。



 日餘,三百四十八萬四千六百四十六。



 朔虛分,一百五十。



 斗分,九十七萬四千六百九十。



 度數,三十三。



 度餘,二百五十萬九千九百五十六。



 火:周率,三千四百七。



 日率,七千二百七十一。



 合月數,二十六。



 月餘,二萬五千六百二十七。



 合月法,六萬四千七百三十三。



 日度法,二百萬六千七百二十三。



 朔大餘,四十七。



 朔小餘,一千一百五十七。



 入月日,十二。



 日餘,九十七萬三千一十三。



 朔虛分,三百。



 斗分,四十九萬四千一十五。



 度數,四十八。



 度餘,一百九十九萬一千七百六。



 土:周度,三千五百二十九。



 日率,三千六百五十三。



 合月數,十二。



 月餘,五萬三千八百四十三。



 合月法,六萬七千五十一。



 日度法,二百七萬八千五百八十一。



 朔大餘,五十四。



 朔小餘,五百三十四。



 入月日,二十四。



 日餘,十六萬六千二百七十二。



 朔虛分,九百二十三。



 斗分,五十一萬一千七百五。



 度數,十二。



 度餘,一百七十三萬三千一百四十八。



 金:周率,九千二十二。



 日率,七千二百一十三。



 合月數,九。



 月餘,十五萬二千二百九十三。



 合月法,十七萬一千四百一十八。



 日度法,五百三十一萬三千九百五十八。



 朔大餘,二十五。



 朔小餘,一千一百二十九。



 入月日,二十七。



 日餘,五萬六千九百五十四。



 朔虛分,三百二十八。



 斗分,一百三十萬八千一百九十。



 度數,二百九十二。



 度餘,五萬六千九百五十四。



 水:周率,一萬一千五百六十一。



 日率,一千八百三十四。



 合月數,一。



 月餘,二十一萬一千三百三十一。



 合月法,二十一萬九千六百五十九。



 日度法,六百八十萬九千四百二十九。



 朔大餘,二十九。



 朔小餘,七百七十三。



 入月日,二十八。



 日餘,六百四十一萬九百六十七。



 朔虛分,六百八十四。



 斗分,一百六十七萬六千三百四十五。



 度數,五十七。



 度餘,六百四十一萬九百六十七。



 推五星



 置上元盡所求年,以周率乘之,滿日率得一,名積合,不盡為合餘。以周率除之,得一,星合往年。二,合前往年。無所得,合其年。合餘減周率為度分。金、水積合,奇為晨,耦
 為夕。



 推星合月



 以月數、月餘各乘積合,滿合月法從月,不盡為月餘。以紀月去積月,餘為入紀月。副以章閏乘之,滿章月得一閏,以減入紀月,餘以歲中去之,命以天正算外,合月也。其在閏交際,以朔御之。



 推入月日



 以通法乘月餘,合月法乘朔小餘,并以會數約之,所得滿日度法得一,則星合入月日也。不滿為日餘,命以朔算外。



 推星合度



 以周天乘度分,滿日度法得一度,不盡為餘,命度以牛前五起。



 右求星合。



 求後合月



 以月數加月數,以月餘加月餘,滿合月法得一月,不滿歲中,即合其年,滿去之,有閏計焉,餘為後年;再滿,在後二年。金、水加晨得夕,加夕得晨。



 求後合朔日



 以朔大小餘,加合月大小餘,上成月者,又加大餘二十九,小餘七百七十三,小餘滿日法從大餘,命如前。



 求後入月日術



 以入月日、日餘,加合入月日及餘,餘滿日度法得一日,其前合朔小餘滿其虛分者,減一日。。後小餘滿七百七十三以上者,去二十九日,不滿,去三十日,其餘則後合,入月日也。



 求後度



 以度加度,度餘加度餘,滿日度法得一度。



 木:



 伏三十二日。三百四十八萬四千六百四十六分。



 見三百六十六日。



 伏行五度。
 二百五十萬九千九百五十六分。



 見行四十度。除逆退十二度,定行二十八度。



 火:伏百四十三日。九十七萬三千一十三分。



 見六百三十六日。



 伏行一百一十度。四十七萬八千九百九十八分。



 見行三百二十度。除逆十七度,定行三百三度。



 土:伏三十三日。十六萬六千二百七十二分。



 見三百四十五日。



 伏行三度。一百七十三萬三千一百四十八分。



 見行十五度。除逆六度,定行九度。



 金:晨伏東方八十二日。十一萬三千九百八分。



 見西方。二百四十六日。除逆六度,定行二百四十六度。



 晨伏行百度。十一萬三千九百八分。



 見東方。日度加西。
 伏十日,退八度。



 水:晨伏三十三日。六百一萬二千五百五分。



 見西方。三十二日。除逆一度,定行三十二度。



 伏行六十五度。六百一萬二千五百五分。



 見東方。日度如西,伏十八日,退十四度



 五星歷步術



 以法伏日度及餘,加星合日度餘,餘滿日度法得一,從全命之如前,得星見日及度也。以星行分母乘見度,餘如日度法得一,分不盡半法以上亦得一;而日加所
 行分,分滿其母得一度,逆順母不同,以當行之母乘故分,如故母而一,當行分也。留者承前,遞則減之,伏不盡度,經斗除分,以行母為率,分有損益,前後相御。凡言如盈約滿,皆求實之除也;去及除之,取盡之除也。



 木:晨與日合,伏,順,十六日百七十四萬二千三百二十三分,行星二度三百二十三萬四千六百七分,而晨見東方,在日後。順,疾,日行五十八分之十一,五十八日行十一度。更順,遲,日行九分,五十八日行九度。留,不行二十五日而旋。逆,日行七分之一,八十四日退十二度。復留,二十五日而順,日行五十八分之九,五十八日行
 九度。順,疾,日行十一分,五十八日行十一度,在日前,夕伏西方。十六日百七十四萬二千三百二十三分,行星二度三百二十三萬四千六百七分,而與日合。凡一終,三百九十八日三百四十八萬四千六百四十六分,行星四十三度二百五十萬九千九百五十六分。



 火:晨與日合,伏,順,七十一日百四十八萬九千八百六十八分,行星五十五度百二十四萬二千八百六十分半,而晨見東方,在日後。順,日行二十三分之十四,百八十四日行一百一十二度。更順,遲,日行二十三分之十二,九十二日行四十八度。留,不行十一日。旋,逆,日行六
 十二分之十七,六十二日退十七度。復留,十一日而順,日行十二分,九十二日行四十八度。復順,疾,日行十四分,百八十四日行百一十二度,在日前,夕伏西方。七十一日百四十八萬九千八百六十八分,行星五十五度百二十四萬二千八百六十分半,而與日合。凡一終,七百七十九日九十七萬三千一十三分,行星四百一十四度四十七萬八千九百九十八分。



 土:晨與日合,伏,順,十六日百一十二萬二千四百二十六分半,行星一度百九十九萬五千八百六十四分半,而晨見東方,在日後。順,日行三十五分之三,八十七
 日半行七度半。留,不行三十四日。旋,逆,日行十七分之一,百二日退六度。復三十四日而順,日行三分,八十七日行七度半,在日前,夕伏西方。十六日百一十二萬二千四百二十六分半,行星一度百九十萬五千八百六十四分半,而與日合也。凡一終,三百七十八日十六萬六千二百七十二分,行星十二度百七十三萬三千一百四十八分。



 金:晨與日合,伏,逆,五日退四度,而晨見東方,在日後。逆,日行五分度之三,十日退六度。留,不行八日。旋,順,遲,日行四十六分之三十三,四十六日行三十三度而順。疾,
 日行一度九十一分之十五,九十一日行一百六度。更順,益疾,日行一度九十一分之二十二,九十一日行百一十三度,在日後,晨伏東方。順,四十一日五萬六千九百五十四分,行星五十度五萬六千九百五十四分,而與日合。一合,二百九十二日五萬六千九百五十四分,行星亦如之。



 金:夕與日合,伏,順,四十一日五萬六千九百五十四分,行星五十度五萬九千九百五十四分,而夕見西方,在日前。順,疾,日行一度九十一分之二十二,九十一日行百一十三度。更順,減疾,日行一度十五分,九十一日行百六度而順。遲,日行四十六分之三十三,四十六日行
 三十三度。留,不行八日。旋,逆,日行五分之三,十日退六度,在日前,夕伏西方,逆,疾,五日退四度,而與日合。凡再合一終,五百八十四日十一萬三千九百八分,行星亦如之。



 水:晨與日合,伏,逆,九日退七度,而晨見東方,在日後。更逆,疾,一日退一度。留,不行二日。旋,順,遲,日行九分之八,九日行八度而順。疾,日行一度四分之一,二十日行二十五度,在日後。晨伏東方,順,十六日六百四十一萬九百六十七分,行星三十二度六百四十一萬九百六十七分,而與日合,一合,五十七日六百四十一萬九百六十七分,行星亦如之。



 水:夕與日合,伏,順,十六日六百四十一萬九百六十七分行星三十二度六百四十一萬九百六十七分,而夕見西方,在日前。順,疾,日行一度四分之一,二十日行二十五度而順。遲,日行九分之八,九日行八度。留,不行二日。旋,逆,一日退一度,在日前,夕伏西方。逆,遲,九日退七度,與日合。凡再合一終,一百一十五日六百一萬二千五百五分,行星亦如之。



\end{pinyinscope}