\article{列傳第四十五 王湛袁悅之祖台之荀崧範汪劉惔張憑韓伯}

\begin{pinyinscope}
王湛
 \gezhu{
  子承承子述述子坦之禕之坦之子愷愉國寶忱愉子綏承族子嶠}
 袁悅之祖台之荀崧
 \gezhu{
  子蕤羨}
 範汪
 \gezhu{
  子寧叔堅}
 劉惔張憑韓伯



 王湛,字處沖,司徒渾之弟也。少有識度。身長七尺八寸,龍顙大鼻,少言語。初有隱德,人莫能知,兄弟宗族皆以為癡,其父昶獨異焉。遭父喪,居于墓次。服闋,闔門守靜,不交當世,沖素簡淡,器量隤然,有公輔之望。



 兄子濟輕之,所食方丈盈前,不以及淇。湛命取菜蔬,對而食之。濟
 嘗詣湛,見床頭有《周易》,問曰:「叔父何用此為?」湛曰:「體中不佳時,脫復看耳。」濟請言之。湛因剖析玄理,微妙有奇趣,皆濟所未聞也。濟才氣抗邁,於湛略無子姪之敬。既聞其言,不覺慄然,心形俱肅。遂留連彌日累夜,自視缺然,乃歎曰:「家有名士,三十年而不知,濟之罪也。」既而辭去,湛送至門。濟有從馬絕難乘,濟問湛曰:「叔頗好騎不?」湛曰:「亦好之。」因騎此馬,姿容既妙,迴策如縈,善騎者無以過之。又濟所乘馬,甚愛之,湛曰:「此馬雖快,然力薄不堪苦行。近見督郵馬當勝,但芻秣不至耳。」濟試養之,而與己馬等。湛又曰:「此馬任重方知之,平路無以別也。」於
 是當蟻封內試之,濟馬果躓,而督郵馬如常。濟益歎,還白其父,曰:「濟始得一叔,乃濟以上人也。」武帝亦以湛為癡,每見濟,輒調之曰:「卿家癡叔死未?」濟常無以答。及是,帝又問如初,濟曰:「臣叔殊不癡。」因稱其美。帝曰:「誰比?」濟曰:「山濤以下,魏舒以上。」時人謂湛上方山濤不足,下比魏舒有餘。湛聞曰:「欲處我於季孟之間乎?」



 湛少仕歷秦王文學、太子洗馬、尚書郎、太子中庶子,出為汝南內史。元康五年卒,年四十七。子承嗣。



 承字安期。清虛寡欲,無所修尚。言理辯物,但明其指要而不飾文辭,有識者服其約而能通。弱冠知名。太尉王
 衍雅貴異之,比南陽樂廣焉。永寧初,為驃騎參軍。值天下將亂,乃避難南下。遷司空從事中郎。豫迎大駕,賜爵藍田縣侯。遷尚書郎,不就。東海王越鎮許,以為記室參軍。雅相知重,敕其子毗曰:「夫學之所益者淺,體之所安者深。閑習禮度,不如式瞻儀形;諷味遺言,不若親承音旨。王參軍人倫之表,汝其師之。」在府數年,見朝政漸替,辭以母老,求出。越不許。久之,遷東海太守,政尚清凈,不為細察。小吏有盜池不魚者,綱紀推之,承曰:「文王之囿與眾共之,池魚復何足惜耶!」有犯夜者,為吏所拘,承問其故,答曰:「從師受書,不覺日暮。」承曰:「鞭撻寧越以立威
 名,非政化之本。」使吏送,令歸家。其從容寬恕若此。



 尋去官,東渡江。是時道路梗澀,人懷危懼,承每遇艱險,處之夷然,雖家人近習,不見其憂喜之色。既至下邳,登山北望,歎曰:「人言愁,我始欲愁矣。」及至建鄴,為元帝鎮東府從事中郎,甚見優禮。承少有重譽,而推誠接物,盡弘恕之理,故眾咸親愛焉。渡江名臣王導、衛玠、周顗、庾亮之徒皆出其下,為中興第一。年四十六卒,朝野痛惜之。自昶至承,世有高名,論者以為祖不及孫,孫不及父。子述嗣。



 述字懷祖。少孤,事母以孝聞。安貧守約,不求聞達。性沈
 靜,每坐客馳辨,異端競起,而述處之恬如也。少襲父爵。年三十,尚未知名,人或謂之癡。司徒王導以門地辟為中兵屬。既見,無他言,惟問以江東米價。述但張目不答。導曰:「王掾不癡,人何言癡也?」嘗見導每發言,一坐莫不贊美,述正色曰:「人非堯舜,何得每事盡善!」導改容謝之,庾亮曰:「懷祖清貞簡貴,不減祖、父,但曠淡微不及耳。」



 康帝為驃騎將軍,召補功曹,出為宛陵令。太尉、司空頻辟,又除尚書吏部郎,並不行。歷庾冰征虜長史。時庾翼鎮武昌,以累有妖怪,又猛獸入府,欲移鎮避之。述與冰箋曰:



 竊聞安西欲移鎮樂鄉,不審此為算邪,將為情邪?
 若謂為算,則彼去武昌千有餘里,數萬之眾造創移徒,方當興立城壁,公私勞擾。若信要害之地,所宜進據,猶當計移徙之煩,權二者輕重,況此非今日之要邪!方今彊胡陸梁,當畜力養銳,而無故遷動,自取非算。又江州當溯流數千,供繼軍府,力役增倍,疲曳道路。且武昌實是江東鎮戍之中,非但扞禦上流而已。急緩赴告,駿奔不難。若移樂鄉,遠在西陲,一朝江渚有虞,不相接救。方岳取重將,故當居要害之地,為內外形勢。使窺窬之心不知所向。若是情邪,則天道玄遠,鬼神難言,妖祥吉凶,誰知其故!是以達人君子直道而行,不以情失。昔秦忌:「
 亡胡」之讖,卒為劉項之資;周惡檿弧之謠,而成褒姒之亂。此既然矣。歷觀古今,鑒其遺事,妖異速禍敗者,蓋不少矣,禳避之道,茍非所審,且當擇人事之勝理,思社稷之長計,斯則天下幸甚,令名可保矣。



 若安西盛意已耳,不能安於武昌,但得近移夏口,則其次也。樂鄉之舉,咸謂不可。願將軍體國為家,固審此舉。



 時朝議亦不允,翼遂不移鎮。



 述出補臨海太守,遷建威將軍、會稽內史。蒞政清肅,終日無事。母憂去職。服闋,代殷浩為揚州刺史,加征虜將軍。初至,主簿請諱。報曰:「亡祖先君,名播海內,遠近所知;內諱不出門,餘無所諱。」尋加中書監,固讓,經
 年不拜。復加征虜將軍,進都督揚州徐州之瑯邪諸軍事、衛將軍、並冀幽平四州大中正,刺史如故。尋遷散騎常侍、尚書令,將軍如故。述每受職,不為虛讓,其有所辭,必於不受。至是,子坦之諫,以為故事應讓。述曰:「汝謂我不堪邪?」坦之曰:「非也。但克讓自美事耳。」述曰:「既云堪,何為復讓!人言汝勝我,定不及也。」坦之為桓溫長史。溫欲為子求婚於坦之。及還家省父,而述愛坦之。雖長大,猶抱置膝上。坦之因言溫意。述大怒,遽排下,曰:「汝竟癡邪!詎可畏溫面而以女妻兵也。」坦之乃辭以他故。溫曰:「此尊君不肯耳。」遂止。簡文帝每言述才既不長,直以真率
 便敵人耳。謝安亦歎美之。



 初,述家貧。求試宛陵令。頗受贈遺。而修家具,為州司所檢,有一千三百條。王導使謂之曰:「名父之子不患無祿,屈臨小縣,甚不宜耳。」述答曰:「足自當止。時人未之達也。」比後屢居州郡,清潔絕倫,祿賜皆散之親故,宅宇舊物不革於昔,始為當時所嘆。但性急為累。嘗食雞子,以箸刺之,不得,便大怒擲地。雞子圓轉不止,便下床以屐齒踏之,又不得。瞋甚,掇內口中,嚙破而吐之。既躋重位,每以柔克為用。謝奕性麤,嘗忿述,極言罵之。述無所應,面壁而已,居半日,奕去,始復坐。人以此稱之。



 太和二年,以年迫懸車,上疏乞骸骨,曰:「臣
 曾祖父魏司空昶白箋於文皇帝曰:『昔與南陽宗世林共為東宮官屬。世林少得好名,州里瞻敬。及其年老,汲汲自勵,恐見廢棄,時人咸共笑之。若天假其壽,致仕之年,不為此公婆娑之事。』情旨慷慨,深所鄙薄。雖是箋書,乃實訓誡。臣忝端右,而以疾患,禮敬廢替。猶謂可有差理,日復一日,而年衰疾痼,永無復瞻華幄之期。乞奉先誡,歸老丘園。」不許。述竟不起。三年卒,時年六十六。



 初,桓溫平洛陽,議欲遷都,朝廷憂懼,將遣侍中止之。述曰:「溫欲以虛聲威朝廷,非事實也。但從之,自無所至。」事果不行。又議欲移洛陽鐘虡,述曰:「永嘉不競,暫都江左。方當
 蕩平區宇,旋軫舊京。若其不耳,宜改遷園陵。不應先事鐘虡。」溫竟無以奪之。追贈侍中、驃騎將軍、開府,謚曰穆,以避穆帝,改曰簡。子坦之嗣。



 坦之字文度。弱冠與郗超俱有重名,時人為之語曰:「盛德絕倫郗嘉賓,江東獨步王文度。」嘉賓,超小字也。僕射江[A170]領選,將擬為尚書郎。坦之聞曰:「自過江來,尚書郎正用第二人,何得以此見擬!」[A170]遂止。簡文帝為撫軍將軍,辟為掾。累遷參軍、從事中郎,仍為司馬,加散騎常侍。出為大司馬桓溫長史。尋以父憂去職,服闋。徵拜侍中,襲父爵。時卒士韓悵逃之歸首,云「失牛故叛。」有司劾悵
 偷牛,考掠服罪。坦之以為悵束身自歸,而法外加罪,懈怠失牛,事或可恕,加之木石,理有自誣,宜附罪疑從輕之例,遂以見原。海西公廢,領左衛將軍。



 坦之有風格,尤非時俗放蕩,不敦儒教,頗尚刑名學,著《廢莊論》曰:



 荀卿稱莊子「蔽於天而不知人」,揚雄亦曰「莊周放蕩而不法」,何晏云「鬻莊軀,放玄虛,而不周乎時變」。三賢之言,遠有當乎!夫獨構之唱,唱虛而莫和;無感之作,義偏而用寡。動人由於兼忘,應物在乎無心。孔父非不體遠,以體遠故用近;顏子豈不具德,以德備故膺教。胡為其然哉?不獲已而然也。



 夫自足者寡,故理懸於羲農;徇教者眾,故
 義申於三代。道心惟微,人心惟危,吹萬不同,孰知正是!雖首陽之情,三黜之智,摩頂之甘,落毛之愛,枯槁之生,負石之死,格諸中庸,未入乎道,而況下斯者乎!先王知人情之難肆,懼違行以致訟,悼司徹之貽悔,審褫帶之所緣,故陶鑄群生,謀之未兆,每攝其契,而為節焉。使夫敦禮以崇化,日用以成俗,誠存而邪忘,利損而競息,成功遂事,百姓皆曰我自然。蓋善闇者無怪,故所遇而無滯,執道以離俗,孰踰於不達!語道而失其為者,非其道也;辯德而有其位者,非其德也。言默所未究,況揚之以為風乎!且即濠以尋魚,想彼之我同;推顯以求隱,理得
 而情昧。若夫莊生者,望大庭而撫契,仰彌高於不足,寄積想於三篇,恨我懷之未盡,其言詭譎,其義恢誕。君子內應。從我游方之外,眾人因藉之,以為弊薄之資。然則天下之善人少,不善人多,莊子之利天下也少,害天下也多。故曰魯酒薄而邯鄲圍,莊生作而風俗頹。禮與浮雲俱徵,偽與利蕩並肆,人以克己為恥,士以無措為通,時無履德之譽,俗有蹈義之愆。驟語賞罰不可以造次,屢稱無為不可與適變。雖可用於天下,不足以用天下人。



 昔漢陰丈人修渾沌之術,孔子以為識其一不識其二。莊生之道,無乃類乎!與夫如愚之契,何殊間哉!若夫
 利而不害,天之道也;為而不爭,聖之德也。群方所資而莫知誰氏,在儒而非儒,非道而有道。彌貫九流,玄同彼我,萬物用之而不既,亹癖日新而不朽,昔吾孔老固已言之矣。



 又領本州大中正。簡文帝臨崩,詔大司馬溫依周公居攝故事。坦之自持詔入,於帝前毀之。帝曰:「天下,儻來之運,卿何所嫌!」坦之曰:「天下,宣元之天下,陛下何得專之!」帝乃使坦之改詔焉。



 溫薨,坦之與謝安共輔幼主,遷中書令,領丹陽尹。俄授都督徐兗青三州諸軍事、北中郎將、徐兗二州刺史,鎮廣陵。將之鎮,上表曰:



 臣聞人君之道以孝敬為本,臨御四海以委任為貴。恭順無
 為,則盛德日新;親杖賢能,則政道邕睦。昔周成、漢昭,並以幼年纂承大統。當時天下未為無難,終能顯揚祖考,保安社稷,蓋尊尊親親,信納大臣之所致也。



 伏維陛下誕奇秀之姿,稟生知之量,春秋尚富,涉道未廣,方須訓導以成天德。皇太后仁淑之體,過於三母,先帝奉事積年,每稱聖明。臣願奉事之心,便當自同孝宗;太后慈愛之隆,亦不必異所生。瑯邪王、餘姚主及諸皇女,宜朝夕定省,承受教誨,導習儀刑,以成景仰恭敬之美,不可以屬非至親,自為疏疑。昔肅祖崩殂,成康幼沖,事無大小,必諮丞相導,所以克就聖德,實此之由,今僕射臣安、中
 軍臣沖,人望具瞻,社稷之臣。且受遇先帝,綢繆繾綣,並志竭忠貞,盡心盡力,歸誠陛下,以報先帝。愚謂周旋舉動。皆應諮此二臣。二臣之於陛下,則周之旦奭,漢之霍光,顯宗之於王導。沖雖在外,路不云遠,事容信宿,必宜參詳,然後情聽獲盡,庶事可畢。



 又天聽雖聰,不啟不廣;群情雖忠,不引不盡。宜數引侍臣,詢求讜言。平易之世,有道之主猶尚誡懼,日昃不倦;況今艱難理盡,慮經安危,祖宗之基系之陛下,不可不精心務道,以申先帝堯舜之風。可不敬修至德,以保宣元天地之祚?



 表奏,帝納之。



 初,謝安愛好聲律,期功之慘,示廢妓樂頗,以成俗。坦
 之非而苦諫之。安遺坦之書曰:「知君思相愛惜之至。僕所求者聲,謂稱情義,無所不可為,卿復以自娛耳。若絜軌跡,崇世教,非所擬議,亦非所屑。常謂君粗得鄙趣者,猶未悟之濠上邪!故知莫逆,未易為人。」坦之答曰:「具君雅旨,此是誠心而行,獨往之美,然恐非大雅中庸之謂。意者以為人之體韻猶器之方圓,方圓不可錯用,體韻豈可易處!各順其方,以弘其業,則歲寒之功必有成矣。實吾子少立德行,體議淹允,加以令地,優游自居,僉曰之談,咸以請遠相許,至於此事,實有疑焉。公私二三,莫見其可。以此為濠上,悟之者得無鮮乎!且天下之寶,故為
 天下所惜,天下之所非,何為不可以天下為心乎?想君幸復三思。」書往反數四,安竟不從。



 坦之又嘗與殷康子書論公謙之義曰:



 夫天道以無私成名,二儀以至公立德。立德存乎至公,故無親而非理;成名在乎無私,故在當而忘我。此天地所以成功,聖人所以濟化,由斯論之,公道體於自然,故理泰而愈降;謙義生於不足,故時弊而義著。故大禹、咎繇稱功言惠而成名於彼,孟反、范燮殿軍後入而全身於此。從此觀之,則謙公之義固以殊矣。



 夫物之所美,己不可收;人之所貴,我不可取。誠患人惡其上,眾不可蓋,故君子居之,而每加損焉。隆名在於
 矯伐,而不在於期當,匿迹在於違顯,而不在於求是。於是謙光之義與矜競而俱生,卑挹之義與夸伐而並進。由親譽生於不足,未若不知之有餘;良藥效於瘳疾,未若無病之為貴也。



 夫乾道確然,示人易矣;坤道貴然,示人簡矣。二象顯於萬物,兩德彰於群生,豈矯枉過直而失其所哉!由此觀之,則大通之道公坦於天地,謙伐之議險崨於人事。今存公而廢謙,則自伐者託至公以生嫌,自美者因存黨以致惑。此王生所謂同貌而實異,不可不察者也,然理必有根,教亦有主。茍探其根,則玄指自顯;若尋其末,弊無不至。豈可以嫌似而疑至公,弊貪
 而忘於諒哉!



 康子及袁宏並有疑難,坦之標章擿句,一一申而釋之,莫不厭服。又孔嚴著《通葛論》,坦之與書贊美之。其忠公慷慨,標明賢勝,皆此類也。



 初,坦之與沙門竺法師甚厚,每共論幽明報應。便要先死者當報其事。後經年,師忽來云:「貧道已死,罪福皆不虛。惟當勤修道德,以升濟神明耳。」言訖不見。坦之尋亦卒,時年四十六。臨終,與謝安、桓沖書,言不及私,惟憂國家之事,朝野甚痛惜之。追贈安北將軍,謚曰獻。



 禕之字文邵。少知名,尚尋陽公主,歷中書侍郎。年未三十而卒,贈散騎常侍。



 坦之四子:愷、愉、國寶、忱。



 愷字茂仁,愉字茂和,並少踐清階。愷襲父爵,愉稍遷驃騎司馬,加輔國將軍。愷太元末為侍中,領右衛將軍,多所獻替。兄弟貴盛,當時莫比。



 及王恭等討國寶,愷、愉並請解職。以與國寶異生,又素不協,故得免禍。國寶既死,出愷為吳郡內史,愉為江州刺史、都督豫州四郡、輔國將軍、假節。未幾,徵愷為丹陽尹。及桓玄等至江寧,愷令兵守石頭。俄而玄等走,復為吳郡。病卒,追贈太常。



 愉至鎮,未幾,殷仲堪、桓玄、楊佺期舉兵應王恭,乘流奄至。愉既無備,惶遽奔臨川,為玄所得。玄盟于尋陽,以愉置壇所,愉甚恥之。及事解,除會稽內史。玄篡位,以為尚書僕
 射。」劉裕義旗建,加前將軍。愉既桓氏婿,父子寵貴,又嘗輕侮劉裕,心不自安,潛結司州刺史溫詳,謀作亂,事泄,被誅,子孫十餘人皆伏法。



 國寶少無士操,不修廉隅。婦父謝安惡其傾側,每抑而不用。除尚書郎。國寶以中興膏腴之族,惟作吏部,不為餘曹郎,甚怨望,固辭不拜。從妹為會稽王道子妃,由是與道子遊處,遂間毀安焉。



 及道子輔政,以為秘書丞。俄遷瑯邪內史,領堂邑太守,加輔國將軍。人補侍中,遷中書令、中領軍,與道子持威權,扇動內外。中書郎范寧,國寶舅也,儒雅方直,疾其阿諛,勸孝武帝黜之。國寶乃使
 陳郡袁悅之因尼支妙音致書與太子母陳淑媛,說國寶忠謹,宜見親信。帝知之,託以他罪殺悅之。國寶大懼,遂因道子譖毀寧,寧由是出為豫章太守。及弟忱卒,國寶自表求解職迎母。并奔忱喪。詔特賜假,而盤桓不時進發,為御史中丞褚粲所奏。國寶懼罪,衣女子衣,託為王家婢,詣道子告其事。道子言之於帝,故得原。後驃騎參軍王徽請國寶同宴,國寶素驕貴使酒,怒尚書左丞祖台之,攘袂大呼,以盤盞樂器擲台之,台之不敢言,復為粲所彈。詔以國寶縱肆情性,甚不可長,台之懦弱,非監司體,並坐免官。頃之,復職,愈驕蹇不遵法度。起齋侔
 清暑殿,帝惡其僭侈。國寶懼,遂諂媚於帝,而頗疏道子。道子大怒,嘗於內省面責國寶,以劍擲之,舊好盡矣。



 是時王雅亦有寵,薦王珣於帝。帝夜與國寶及雅宴,帝微有酒,令召珣,將至,國寶自知才出珣下,恐至,傾其寵,因曰:「王珣當今名流,不可以酒色見。」帝遂止,而以國寶為忠。將納國寶女為瑯邪王妃,未婚,而帝崩。



 安帝即位,國寶復事道子,進從祖弟緒為瑯邪內史,亦以佞邪見知。道子復惑之,倚為心腹,並為時之所疾。國寶遂參管朝權,威震內外。遷尚書左僕射。領選,加後將軍、丹陽尹,道子悉以東宮兵配之。



 時王恭與殷仲堪並以才器,各居
 名籓。恭惡道子、國寶亂政,屢有憂國之言。道子等亦深忌憚之,將謀去其兵。未及行,而恭檄至,以討國寶為名,國寶惶遽不知所為。緒說國寶,令矯道子命,召王珣、車胤殺之,以除群望,因挾主相以討諸侯。國寶許之。珣、胤既至,而不敢害,反問計於珣。珣勸國寶放兵權以迎恭,國寶信之。語在《珣傳》。又問計於胤,胤曰:「南北同舉,而荊州未至,若朝廷遣軍,恭必城守。昔桓公圍壽陽,彌時乃剋。若京城未拔,而上流奄至,君將何以待之?」國寶尤懼,遂上疏解職,詣闕待罪。既而悔之,祚稱詔復其本官,欲收其兵距王恭。



 道子既不能距諸侯,欲委罪國寶,乃遣
 譙王尚之收國寶,付廷尉,賜死,並斬緒於市。以謝王恭。國寶貪縱聚斂,不知紀極,後房伎妾以百數,天下珍玩充滿其室。及王恭伏法,詔追復國寶本官。元興初,桓玄得志,表徙其家屬於交州。



 忱字元達。弱冠知名,與王恭、王珣俱流譽一時。歷位驃騎長史。嘗造其舅范寧,與張玄相遇,寧使與玄語。玄正坐斂衽,待其有發,忱竟不與言,玄失望便去。寧讓忱曰:「張玄,吳中之秀,何不與語?」忱笑曰:「張祖希欲相識,自可見詣。」寧謂曰:「卿風流雋望,真後來之秀。」忱曰:「不有此舅,焉有此甥!」既而寧使報玄,玄束帶造之,始為賓主。



 太元
 中,出為荊州刺史、都督荊益寧三州軍事、建武將軍、假節。忱自恃才氣,放酒誕節,慕王澄之為人,又年少居方伯之任,談者憂之。及鎮荊州,威風肅然,殊得物和。桓玄時在江陵,既其本國。且奕葉故義,常以才雄駕物。忱每裁抑之。玄嘗詣忱,通人未出,乘轝直進。忱對玄鞭門幹,玄怒,去之,忱亦不留。嘗朔日見客,仗衛甚盛,玄言欲獵,借數百人,忱悉給之。玄憚而服焉。



 性任達不拘,末年尤嗜酒,一飲連月不醒,或裸體而游,每歡三日不嘆,便覺形神不相親。婦父嘗有慘,忱乘醉吊之,婦父慟哭,忱與賓客十許人,連臂被髮裸身而入,繞之三幣百而出。其所
 行多此類。數年卒官,追贈右將軍,謚曰穆。



 綏字彥猷。少有美稱,厚自矜邁,實鄙而無行。愉為殷、桓所捕,綏未測存亡,在都有憂色,居處飲食,每事貶降,時人每謂為「試守孝子」。桓玄之為太尉,綏以桓氏甥甚見寵待,為太尉右長史。及玄篡,遷中書令。劉裕建義,以為冠軍將軍。其家夜中梁上無故有人頭墮於床,而流血滂沲。俄拜荊州刺史、假節。坐父愉之謀,與弟納並被誅。



 初,綏與王謐、桓胤齊名,為後進之秀。謐位官既極,保身而終。胤以從坐誅,聲稱猶全。綏身死,名論殆盡,亦以薄行矜峭而尚人故也。自昶父漢鴈門太守澤已有名稱,
 忱又秀出,綏亦著稱,八葉繼軌,軒冕莫與為比焉。



 嶠字開山。祖默,魏尚書。父佑,以才智稱,為楊駿腹心。駿之排汝南王亮,退衛瓘,皆佑之謀也。位至北軍中候。嶠少有風尚,并、司二州交辟,不就。永嘉末,攜其二弟避亂渡江。時元帝鎮建鄴,教曰:「王佑三息始至,名德之胄,並有操行,宜蒙飾敘。且可給錢三十萬,帛三百匹,米五十斛,親兵二十人。」尋以嶠參世子東中郎軍事。不就。愍帝徵拜著作郎,右丞相南陽王保辟,皆以道險不行。元帝作相,以為水曹屬,除長山令,遷太子中舍人以疾不拜。王敦請為參軍,爵九原縣公。



 敦在石頭,欲禁私伐蔡洲
 荻,以問群下。時王師新敗,士庶震懼,莫敢異議。嶠獨曰:「中原有菽,庶人採之。百姓不足,君孰與足!若禁人樵伐,未知其可。」敦不悅。敦將殺周顗、戴若思,嶠於坐諫曰:「濟濟多士,交王以寧。安可戮諸名士,以自全生!」敦大怒,欲斬嶠,賴謝鯤以免。敦猶銜之,出為領軍長史。敦平後,除中書侍郎,兼大著作,固辭。轉越騎校尉,頻遷吏部郎、御史中丞、祕書監,領本州大中正。咸和初,朝議欲以嶠為丹陽尹。嶠以京尹望重,不宜以疾居之,求補廬陵郡,乃拜嶠廬陵太守。以嶠家貧,無以上道,賜布百匹。錢十萬。尋卒官,謚曰穆。子淡嗣,歷位右衛將軍、侍中、中護軍、尚
 書、廣州刺史。淡子度世,驍騎將軍。



 袁悅之,字元禮,陳郡陽夏人也。父朗,給事中。悅之能長短說,甚有精理。始為謝玄參軍,為玄所遇,丁憂去職。服闋還都,止齎《戰國策》,言天下要惟此書。後甚為會稽王道子所親愛,每勸道子專覽朝權,道子頗納其說。俄而見誅。



 祖台之,字元辰,范陽人也。官至侍中、光祿大夫。撰志怪,書行於世。



 荀崧,字景猷,潁川臨潁人,魏太尉彧之玄孫也。父頵,羽林右監、安陵鄉侯,與王濟、何劭為拜親之友。崧志操清純,雅好文學。齠齔時,族曾祖顗見而奇之,以為必興頵門。弱冠,太原王濟甚相器重,以方其外祖陳郡袁侃,謂侃弟奧曰:「近見荀監子,清虛名理,當不及父,德性純粹,是賢兄輩人也。」其為名流所賞如此。泰始中,詔以崧代兄襲父爵,補濮陽王允文學。與王敦、顧榮、陸機等友善,趙王倫引為相國參軍。倫篡,轉護軍司馬、給事中,稍遷尚書吏部郎、太弟中庶子,累遷侍中、中護軍。



 王彌入洛,
 崧與百官奔於密,未至而母亡。賊追將及,同旅散走,崧被髮從車,守喪號泣。賊至,棄其母尸于地,奪車而去。崧被四創,氣絕,至夜方蘇。葬母于密山。服闋,族父籓承制,以崧監江北軍事、南中郎將、後將軍、假節、襄城太守。時山陵發掘,崧遣主簿石覽將兵入洛,修復山陵。以勳進爵舞陽縣公,遷都督荊州江北諸軍事、平南將軍,鎮宛,改封曲陵公。為賊杜曾所圍。石覽時為襄城太守,崧力弱食盡,使其小女灌求救於覽及南中郎將周訪。訪即遣子撫率兵三千人會石覽,俱救崧。賊聞兵至,散走。崧既得免,乃遣南陽中部尉王國、劉願等潛軍襲穰縣,獲
 曾從兄偽新野太守保,斬之。



 元帝踐阼,徵拜尚書僕射,使崧與協共定中興禮儀。從弟馗早亡,二息序、廞,年各數歲,崧迎與共居,恩同其子。太尉、臨淮公荀顗國胤廢絕,朝庭以崧屬近,欲以崧子襲封。崧哀序孤微,乃讓封與序,論者稱焉。轉太常。時方修學校,簡省博士,置《周易》王氏、《尚書》鄭氏、《古文尚書》孔氏、《毛詩》鄭氏、《周官禮記》鄭氏、《春秋左傳》杜氏服氏、《論語》《孝經》鄭氏博士各一人,凡九人,其《儀禮》、《公羊》、《穀梁》及鄭《易》皆省不置。崧以為不可,乃上疏曰:



 自喪亂以來,儒學尤寡,今處學則闕明廷之秀,仕朝則廢儒學之俊。昔咸寧、太康、永嘉之中,侍中、
 常侍、黃門通洽古今、行為世表者,領國子博士。一則應對殿堂,奉酬顧問;二則參訓國子,以弘儒訓;三則祠、儀二曹及太常之職,以得質疑。今皇朝中興,美隆往初,宜憲章令軌,祖述前典。世祖武皇帝應運登禪,崇儒興學。經始明堂,營建辟雍,告朔班政,鄉飲大射。西閣東序,河圖秘書禁籍。臺省有宗廟太府金墉故事,太學有石經古文先儒典訓。賈、馬、鄭、杜、服、孔、王、何、顏、尹之徒,章句傳注眾家之學,置博士十九人。九州之中,師徒相傳,學士如林,猶選張華、劉寔居太常之官,以重儒教。



 傳稱「孔子沒而微言絕,七十二子終而大義乖」。自頃中夏殄瘁,講
 誦遏密,斯文之道,將墮于地。陛下聖哲龍飛,恢崇道教,樂正雅頌,於是乎在。江、揚二州,先漸聲教,學士遺文,於今為盛。然方疇昔,猶千之一。臣學不章句,才不弘通,方之華實,儒風殊邈。思竭駑駘,庶增萬分。願斯道隆於百世之上,搢紳詠於千載之下。



 伏聞節省之制,皆三分置二。博士舊置十九人,今五經合九人,準古計今,猶未能半,宜及節省之制,以時施行。今九人以外,猶宜增四。願陛下萬機餘暇,時垂省覽。宜為鄭《易》置博士一人,鄭《儀禮》博士一人,《春秋公羊》博士一人,《穀梁》博士一人。



 昔周之衰,下陵上替,上無天子,下無方伯,善者誰賞,惡者誰
 罰,孔子懼而作《春秋》。諸侯諱妒,懼犯時禁,是以微辭妙旨,義不顯明,故曰「知我者其惟《春秋》,罪我者其惟《春秋》」。時左丘明、子夏造膝親受,無不精究。孔子既沒,微言將絕,於是丘明退撰所聞,而為之傳。其書善禮,多膏腴美辭,張本繼末,以發明經意,信多奇偉,學者好之。稱公羊高親受子夏,立於漢朝,辭義清雋,斷決明審,董仲舒之所善也。穀梁赤師徒相傳,暫立於漢世。向、歆,漢之碩儒,猶父子各執一家,莫肯相從。其書文清義約,諸所發明,或是《左氏》、《公羊》所不載,亦足有所訂正。是以三傳並行於先代,通才未能孤廢。今去聖久遠,其文將墮,與其過
 廢,寧與過立。臣以為三傳雖同曰《春秋》,而發端異趣,案如三家異同之說,此乃義則戰爭之場,辭亦劍戟之鋒,於理不可得共。博士宜各置一人,以博其學。



 元帝詔曰:「崧表如此,皆經國之務。為政所由。息馬投戈,猶可講藝,今雖日不暇給,豈忘本而遺存邪!可共博議者詳之。」議者多請從崧所奏。詔曰:「《穀梁》膚淺,不足置博士,餘如奏。」會王敦之難,不行。



 敦表以崧為尚書左僕射。及帝崩,群臣議廟號,王敦遣使謂曰:「豺狼當路,梓宮未反,祖宗之號,宜別思詳。」崧議以為:「禮,祖有功,宗有德。元皇帝天縱聖哲,光啟中興,德澤侔於太戊,功惠邁于漢宣,臣敢依
 前典,上號曰中宗。」既而與敦書曰:「承以長蛇未翦,別詳祖宗。先帝應天受命,以隆中興;中興之主,寧可隨世數而遷毀!敢率丹直。詢之朝野,上號中宗。卜日有期,不及重請,專輒之愆,所不敢辭。」初,敦待崧甚厚,欲以為司空,於此銜之而止。



 太寧初,加散騎常侍,後領太子太傅。以平王敦功,更封平樂伯。坐使威儀為猛獸所食,免職。後拜金紫光祿大夫、錄尚書事,散騎常侍如故。遷右光祿大夫、開府儀同三司,錄尚書如故。又領祕書監,給親兵百二十人。年雖衰老,而孜孜典籍,世以此嘉之。



 蘇峻之役,崧與王導、陸曄共登御床擁衛帝,及帝被逼幸石頭,
 崧亦侍從不離帝側。賊平,帝幸溫嶠舟,崧時年老病篤,猶力步而從。咸和三年薨,時年六十七。贈侍中,謚曰敬。



 其後著作郎虞預與丞相王導箋曰:「伏見前秘書、光祿大夫荀公,生於積德之族,少有儒雅之稱,歷位內外,在貴能降。蘇峻肆虐,乘輿失幸,公處嫌忌之地,有累卵之危,朝士為之寒心,論者謂之不免。而公將之以智,險而不懾,扶侍至尊,繾綣不離。雖無扶迎之勛,宜蒙守節之報。且其宣慈之美,早彰遠近,朝野之望,許以台司,雖未正位,已加儀同。至守終純固,名定闔棺,而薨卒之日,直加侍中。生有三槐之望,沒無鼎足之名,寵不增於前秩,
 榮不副於本望,此一時愚智所慷慨也。今承大弊之後,淳風頹散,茍有一介之善,宜在旌表之例,而況國之元老,志節若斯者乎!」不從。升平四年,崧改葬,詔賜錢百萬,布五千匹。有二子:蕤、羨。蕤嗣。



 蕤字令遠。起家秘書郎,稍遷尚書左丞。蕤有儀操風望,雅為簡文帝所重。時桓溫平蜀,朝廷欲以豫章郡封溫。蕤言於帝曰:「若溫復假王威,北平河洛,修復園陵,將何以加此!」於是乃止。轉散騎常侍、少府,不拜,出補東陽太守。除建威將軍、吳國內史。卒官。籍嗣位,至散騎常侍、大長秋。



 羨字令則。清和有準。纔年七歲,遇蘇峻難,隨父在石頭,峻甚愛之,恒置膝上。羨陰白其母,曰:「得一利刀子,足以殺賊。」母掩其口,曰:「無妄言!」年十五,將尚尋陽公主,羨不欲連婚帝室,仍遠遁去。監司追,不獲已,乃出尚公主,拜駙馬都尉。弱冠,與琅邪王洽齊名,沛國劉惔、太原王濛、陳郡殷浩並與交好。



 驃騎將軍何充出鎮京口,請為參軍。穆帝又以為撫軍參軍,徵補太常博士,皆不就。後拜秘書丞、義興太守。征北將軍褚裒以為長史。既到,裒謂佐吏曰:「荀生資逸群之氣,將有沖天之舉,諸君宜善事之。」尋遷建威將軍、吳國內史。除北中郎將、徐州刺史、監
 徐兗二州揚州之晉陵諸軍事、假節。殷浩以羨在事有能名,故居以重任。時年二十八,中興方伯,未有如羨之少者。羨至鎮,發二州兵,使參軍鄭襲戍準陰。羨尋北鎮淮陽,屯田于東陽之石鱉。尋加監青州諸軍事,又領兗州刺史,鎮下邳。羨自鎮來朝,時蔡謨固讓司徒,不起,中軍將軍殷浩欲加大辟,以問於羨。羨曰:「蔡公今日事危,明日必有桓文之舉。」浩乃止。



 及慕容俊攻段蘭於青州,詔使羨救之。俊將王騰、趙盤寇瑯邪、鄄城,北境騷動。羨討之,擒騰,盤迸走。軍次瑯邪,而蘭已沒,羨退還下邳,留將軍諸葛攸、高平太守劉莊等三千人守瑯邪,參軍戴
 逯、蕭鎋二千人守泰山。是時,慕容蘭以數萬眾屯汴城,甚為邊害。羨自光水引汶通渠,至于東阿以征之。臨陣,斬蘭。帝將封之,羨固辭不受。



 先是,石季龍死,胡中大亂,羨撫納降附,甚得眾心。以疾篤解職。後除右軍將軍,加散騎常侍,讓不拜。升平二年卒,時年三十八。帝聞之,歎曰:「荀令則、王敬和相繼凋落,股肱腹心將復誰寄乎!」追贈驃騎將軍。



 范汪,字玄平,雍州刺史晷之孫也。父稚,蚤卒。汪少孤貧,六歲過江,依外家新野庾氏。荊州刺史王澄見而奇之,
 曰:「興范族者,必是子也。」年十三,喪母,居喪盡禮,親鄰哀之。及長,好學。外氏家貧,無以資給,汪乃廬于園中,布衣蔬食,然薪寫書,寫畢,誦讀亦遍,遂博學多通,善談名理。弱冠,至京師,屬蘇峻作難。王師敗績,汪乃遁逃西歸。庾亮、溫嶠屯兵尋陽,時行李斷絕,莫知峻之虛實,咸恐賊彊,未敢輕進。及汪至,嶠等訪之,汪曰:「賊政令不一,貪暴縱橫,滅亡已兆,雖彊易弱。朝廷有倒懸之急,宜時進討。」嶠深納之。是日,護軍、平南二府禮命交至,始解褐,參護軍事。賊平,賜爵都鄉侯。復為庾亮平西參軍、從討郭默,進爵亭侯。辟司空郗鑒掾,除宛陵令。復參亮征西軍事,
 轉州別駕。汪為亮佐使十有餘年,甚相欽待。轉鷹揚將軍、安遠護軍、武陵內史,徵拜中書侍郎。



 時庾翼將悉郢漢之眾以事中原,軍次安陸,尋轉屯襄陽。汪上疏曰:



 臣伏思安西將軍翼今至襄陽,倉卒攻討,凡百草創,安陸之調,不復為襄陽之用。而玄冬之月,沔漢乾涸,皆當魚貫百行,排推而進。設一處有急,勢不相救。臣所至慮一也。又既至之後,桓宣當出。宣往實翦豺狼之林,招攜貳之眾,待之以至寬,御之以無法。田疇墾闢,生產始立,而當移之,必有嗷然,悔吝難測。臣所至慮二也。襄陽頓益數萬口,奉師之費,皆當出於江南。運漕之難,船人之力,
 不可不熟計。臣之所至慮三也。且申伯之尊,而與邊將並驅。又東軍不進,殊為孤懸。兵書云:「知彼知此,百戰不殆。知彼不知此,一勝一負。」賊誠衰弊,然得臣猶在;我雖方隆,今實未暇。而連兵不解,患難將起,臣所至慮四也。



 翼豈不知兵家所患常在於此,顧以門戶事任,憂責莫大,晏然終年,憂心情所安,是以抗表輒行,畢命原野。以翼宏規經略,文武用命,忽遇釁會,大事便濟。然國家之慮,常以萬全,非至安至審,王者不舉。臣謂宜嚴詔諭翼,還鎮養銳,以為後圖。若少合聖聽,乞密出臣表,與車騎臣冰等詳共集議。



 尋而驃騎將軍何充輔政,請為長史。
 桓溫代翼為荊州,復以汪為安西長史。溫西征蜀,委以留府。蜀平,進爵武興縣侯。而溫頻請為長史、江州刺史,皆不就。自請還京,求為東陽太守。溫甚恨焉。在郡大興學校,甚有惠政。頃之,召入,頻遷中領軍、本州大中正。時簡文帝作相,甚相親暱,除都督徐兗青冀四州揚州之晉陵諸軍事、安北將軍、徐兗二州刺史、假節。



 既而桓溫北伐,令汪率文武出梁國,以失期,免為庶人。朝廷憚溫不敢執,談者為之歎恨。汪屏居吳郡,從容講肆,不言枉直。後至姑孰,見溫。溫時方起屈滯以傾朝廷,謂汪遠來詣己,傾身引望,謂袁宏曰:「范公來,可作太常邪?」汪既至,
 纔坐,溫謝其遠來意。汪實來造溫,恐以趨時致損,乃曰:「亡兒瘞此,故來視之。」溫殊失望而止。時年六十五,卒於家。贈散騎常侍,謚曰穆。長子康嗣,早卒。康弟寧,最知名。



 寧字武子。少篤學,多所通覽。簡文帝為相,將辟之,為桓溫所諷,遂寢不行。故終溫之世,兄弟無在列位者。時以浮虛相扇,儒雅日替,寧以為其源始於王弼、何晏,二人之罪深於桀紂,乃著論曰:



 或曰:「黃唐緬邈,至道淪翳,濠濮輟詠,風流靡託,爭奪兆於仁義,是非成於儒墨。平叔神懷超絕,輔嗣妙思通微,振千載之頹綱,落周孔之塵網。斯蓋軒冕之龍門,濠梁之宗匠。嘗聞夫子之論,以為
 罪過桀紂,何哉?」



 答曰:「子信有聖人之言乎?夫聖人者,德侔二儀,道冠三才,雖帝皇殊號,質文異制,而統天成務,曠代齊趣。王何蔑棄典文,不遵禮度,游辭浮說,波蕩後生,飾華言以翳實,騁繁文以惑世。搢紳之徒,翻然改轍,洙泗之風,緬焉將墮。遂令仁義幽淪,儒雅蒙塵,禮壞樂崩,中原傾覆。古之所謂言偽而辯、行僻而堅者,其斯人之徒歟!昔夫子斬少正於魯,太公戮華士於齊,豈非曠世而同誅乎!桀紂暴虐,正足以滅身覆國,為後世鑒誡耳,豈能回百姓之視聽載!王何叨海內之浮譽,資膏粱之傲誕,畫螭魅以為巧,扇無檢以為俗。鄭聲之亂樂,利
 口之覆邦,信矣哉!吾固以為一世之禍輕,歷代之罪重,自喪之釁小,迷眾之愆大也。」



 寧崇儒抑俗,率皆如此。



 溫薨之後,始解褐為餘杭令,在縣興學校,養生徒,潔己脩禮,志行之士莫不宗之。期年之後,風化大行。自中興已來,崇學敦教,未有如寧者也。在職六年,遷臨淮太守,封陽遂鄉侯。頃之,徵拜中書侍郎。在職多所獻替,有益政道。時更營新廟,博求辟雍、明堂之制,寧據經傳奏上,皆有典證。孝武帝雅好文學,甚被親愛,朝廷疑議,輒諮訪之。寧指斥朝士,直言無諱。



 王國寶,寧之甥也,以諂媚事會稽王道子,懼為寧所不容,乃相驅扇,因被疏隔。求補
 豫章太守,帝曰:「豫章不宜太守,何急以身試死邪?」寧不信卜占,固請行,臨發,上疏曰:「臣聞道尚虛簡,政貴平靜,坦公亮於幽顯,流子愛於百姓,然後可以經夷險而不憂,乘休否而常夷。先王所以致太平,如此而已。今四境晏如,烽燧不舉,而倉庾虛秏,帑藏空匱。古者使人,歲不過三日,今之勞擾,殆無三日休停,至有殘刑翦髮,要求復除,生兒不復舉養,鰥寡不敢妻娶。豈不怨結人鬼,感傷和氣。臣恐社稷之憂,積薪不足以為喻。臣久欲粗啟所懷,日復一日。今當永離左右,不欲令心有餘恨。請出臣啟事,付外詳擇。」帝詔公卿牧守普議得失,寧又陳時
 政曰:



 古者分土割境,以益百姓之心;聖王作制,籍無黃白之別。昔中原喪亂,流寓江左,庶有旋反之期,故許其挾注本郡。自爾漸久,人安其業,丘壟墳柏,皆已成行,雖無本邦之名,而有安土之實。今宜正其封疆,以土斷人戶,明考課之科,修閭伍之法。難者必曰:「人各有桑梓,俗自有南北。一朝屬戶,長為人隸,君子則有土風之慨,小人則懷下役之慮。」斯誠并兼者之所執,而非通理者之篤論也。古者失地之君,猶臣所寓之主,列國之臣,亦有違適之禮。隨會仕秦,致稱《春秋》;樂毅宦燕,見褒良史。且今普天之人,原其氏出,皆隨世遷移,何至於今而獨不
 可?



 凡荒郡之人,星居東西,遠者千餘,近者數百,而舉召役調,皆相資須,期會差違,輒致嚴坐,人不堪命,叛為盜賊。是以山湖日積,刑獄愈滋。今荒小郡縣,皆宜并合,不滿五千戶,不得為郡,不滿千戶,不得為縣。守宰之任,宜得清平之人。頃者選舉,惟以恤貧為先,雖制有六年,而富足便退。又郡守找吏,牽置無常,或兼臺職,或帶府官。夫府以統州,州以監郡,郡以蒞縣,如令互相領帖,則是下官反為上司,賦調役使無復節限。且牽曳百姓,營起廨舍,東西流遷,人人易處,文書簿籍,少有存者。先之室宇,皆為私家,後來新官,復應修立。其為弊也,胡可勝言!



 又方鎮去官,皆割精兵器杖以為送故,米布之屬不可稱計。監司相容,初無彈糾。其中或有清白,亦復不見甄異。送兵多者至有千餘家,少者數十戶。既力人私門,復資官廩布。兵役既竭,枉服良人,牽引無端,以相充補。若是功勛之臣,則已享裂土之祚,豈應封外復置吏兵乎!謂送故之格宜為節制,以三年為斷,夫人性無涯,奢儉由勢。今並兼之士亦多不瞻,非力不足以厚身,非祿不足以富家,是得之有由,而用之無節。蒱酒永日,馳騖卒年,一宴之饌,費過十金,麗服之美,不可貲算,盛狗馬之飾,營鄭衛之音,南畝廢而不墾,講誦闕而無聞,凡庸競
 馳,傲誕成俗。謂宜驗其鄉黨,考其業尚,試其能否,然後升進。如此,匪惟家給人足,賢人豈不繼踵而至哉!



 官制謫兵,不相襲代,頃者小事,便從補役,一愆之違,辱及累世,親戚傍支,罹其禍毒,戶口減秏,亦由於此。皆宜料遣,以全國信,禮,十九為長殤,以其未成人也。十五為中殤,以為尚童幼也。今以十六為全丁,則備成人之役矣。以十三為半丁,所任非復童幼之事矣。豈可傷天理,遠經典,困苦萬姓,乃至此乎!今宜修禮文,以二十為全丁,十六至十九為半丁,則人無夭折,生長滋繁矣。



 帝善之。



 初,寧之出,非帝本意,故所啟多合旨。寧在郡又大設庠序,
 遣人往交州採磬石,以供學用,改革舊制,不拘常憲。遠近至者千餘人,資給眾費,一出私祿。并取郡四姓子弟,皆充學生,課續五經。又起學臺,功用彌廣,江州刺史王凝之上言曰:「豫章郡居此州之半。太守臣寧入參機省,出宰名郡,而肆其奢濁,所為狼籍。郡城先有六門,寧悉改作重樓,復更開二門,合前為八。私立下舍七所。臣伏尋宗廟之設,各有品秩,而寧自置家廟。又下十五縣,皆使左宗廟,右社稷,準之太廟,皆資人力,又奪人居宅,工夫萬計。寧若以古制宜崇,自當列上,而敢專輒,惟在任心。州既聞知,既符從事,制不復聽。而寧嚴威屬縣,惟令
 速立。願出臣表下太常,議之禮典。」詔曰:「漢宣云:可與共治天下者,良二千石也!若范寧果如凝之所表者,豈可復宰郡乎!」以此抵罪。子泰時為天門太守,棄官稱訴。帝以寧所務惟學,事久不判。會赦,免。



 初,寧嘗患目痛就中書侍郎張湛求方,湛因嘲之曰:「古方,宋陽里子少得其術,以授魯東門伯,魯東門伯以授左丘明,遂世也上傳。及漢杜子夏鄭康成、魏高堂隆、晉左太沖,凡此諸賢,並有目疾,得此方云:用損讀書一,減思慮二,專內視三,簡外觀四,旦晚起五,夜早眠六。凡六物熬以神火,下以氣簁,蘊於胸中七日,然後納諸方寸。修之一時,近能數其
 目睫,遠視尺捶之餘。長服不已,洞見牆壁之外。非但明目,乃亦延年。」既免官,家于丹陽,猶勤經學,終年不輟。年六十三,卒于家。



 初,寧以《春秋穀梁氏》未有善釋,遂沈思積年,為之集解。其義精審,為世所重。既而徐邈復為之注,世亦稱之。



 子泰,元熙中,為護軍將軍。



 堅字子常。博學善屬文。永嘉中,避亂江東,拜佐著作郎、撫軍參軍。討蘇峻,賜爵都亭侯。累遷尚書右丞。時廷尉奏殿中帳吏邵廣盜官幔三張,合布三十匹,有司正刑棄市。廣二子,宗年十三,雲年十一,黃幡撾登聞鼓乞恩,辭求自沒為奚官奴,以贖父命。尚書郎朱暎議以為天
 下之人父,無子者少,一事遂行,便成永制,懼死罪之刑,於此而弛。堅亦同暎議。時議者以廣為鉗徒,二兒沒入,既足以懲,又使百姓知父子道,聖朝有垂恩之仁。可特聽減廣死罪為五歲刑,宗等付奚官為奴,而不為永制。堅駁之曰:「自淳朴澆散,刑辟仍作,刑之所以止刑,殺之所以止殺。雖時有赦過宥罪,議獄緩死,未有行小不忍而輕易典刑也。且既許宗等,宥廣以死,若復有宗比而不求贖父者,豈得不擯絕人倫,同之禽獸邪!案主者今奏云,惟特聽宗等而不為永制。臣以為王者之作,動關盛衰,嚬笑之間,尚慎所加,況於國典,可以徒虧!今
 之所以宥廣,正以宗等耳。人之愛父,誰不如宗?今既居然許宗之請,將來訴者,何獨匪民!特聽之意,未見其益;不以為例,交興怨讟。此為施一恩于今,而開萬怨於後也。」成帝從之,正廣死刑。後遷護軍長史,卒官。



 子啟,字榮期,雖經學不及堅,而以才義顯於當世。于時清談之士庾龢、韓伯、袁宏等,並相知友。為祕書郎,累居顯職,終於黃門侍郎。父子並有文筆傳於世。



 劉惔,字真長,沛國相人也。祖宏,字終嘏,光祿勳。宏兄粹,字純嘏,侍中。宏弟潢,字沖嘏,吏部尚書。並有名中朝。時
 人語曰:「洛中雅雅有三嘏。」父耽,晉陵太守,亦知名。惔少清遠,有標奇,與母任氏寓居京口,家貧,織芒屩以為養,雖蓽門陋巷,晏如也。人未之識,惟王導深器之。後稍知名,論者比之袁羊。惔喜,還告其母。其母,聰明婦人也,謂之曰:「此非汝比,勿受之。」又有方之範汪者。惔復喜,母又不聽。及惔年德轉升,論者遂比之荀粲。尚明帝女廬陵公主。以惔雅善言理,簡文帝初作相,與王濛並為談客,俱蒙上賓禮。時孫盛作《易象妙於見形論》,帝使殷浩難之,不能屈。帝曰:「使真長來,故應有以制之。」乃命迎惔。盛素敬服惔,及至,便與抗答,辭甚簡至,盛理遂屈。一坐撫
 掌大笑,咸稱美之。



 累遷丹陽尹。為政清整,門無雜賓。時百姓頗有訟官長者,諸郡往往有相舉正,惔歎曰:「夫居下訕上,此弊道也。古之善政,司契而已,豈不以其敦本正源,鎮靜流末乎!君雖不君,下安可以失禮。若此風不革,百姓將往而不反。」遂寢而不問。



 性簡貴,與王羲之雅相友善。郗愔有傖奴善知文章,羲之愛之,每稱奴於心炎。惔曰:「何如方回邪?」羲之曰:「小人耳,何比郗公!」惔曰:「若不如方回,故常奴耳。」桓溫嘗問惔:「會稽王談更進邪?」惔曰:「極進,然故第二流耳。」溫曰:「第一復誰?」惔曰:「故在我輩。」其高自標置如此。



 惔每奇溫才,而知其有不臣之迹。及溫
 為荊州,惔言於帝曰:「溫不可使居形勝地,其位號常宜抑之。」勸帝自鎮上流,而己為軍司,帝不納。又請自行,復不聽。及溫伐蜀,時咸謂未易可制,惟惔以為必剋。或問其故,云:「以蒱博驗之,其不必得,則不為也。恐溫終專制朝廷。」及後竟如其言。嘗薦吳郡張憑,憑卒為美士,眾以此服其知人。



 尤好《老莊》,任自然趣。疾篤,百姓欲為之祈禱,家人又請祭神,惔曰:「丘之禱久矣。」年三十六,卒官。孫綽為之誄云:「居官無官官之事,處事無事事之心。」時人以為名言。後綽嘗詣褚裒,言及惔,流涕曰:「可謂人之云亡,邦國殄瘁。」裒大怒曰:「真長生平何嘗相比數,而卿今
 日作此面向人邪!」其為名流所敬重如此。



 張憑,字長宗。祖鎮,蒼梧太守。憑年數歲。鎮謂其父曰:「我不如汝有佳兒。」憑曰:「阿翁豈宜以子戲父邪!」及長,有志氣,為鄉閭所稱。舉孝廉,負其才,自謂必參時彥。初,欲詣惔,鄉里及同舉者共笑之。既至,惔處之下坐,神意不接,憑欲自發而無端。會王就蒙惔清言,有所不通,憑於末坐判之,言旨深遠,足暢彼我之懷,一坐皆驚。惔延之上坐,清言彌日,留宿至旦遣之。憑既還船,須臾,惔遣傳教覓張孝廉船,便召與同載,遂言之於簡文帝。帝召與語,
 嘆曰:「張憑勃窣為理窟。」官至吏部郎、御史中丞。



 韓伯,字康伯,潁川長社人也。母殷氏,高明有行。家貧窶,伯年數歲,至大寒,母方為作襦,令伯捉熨斗,而謂之曰:「且著襦,尋當作復衣軍。」伯曰:「不復須。」母問其故,對曰:「火在斗中,而柄尚熱,今既著襦,下亦當暖。」母甚異之。及長,清和有思理,留心文藝。舅殷浩稱之曰:「康伯能自標置,居然是出群之器。」潁川庾龢名重一時,少所推服,常稱伯及王坦之曰:「思理倫和,我敬韓康伯;志力彊正,吾愧王文度。自此以還,吾皆百之矣。」



 舉秀才,徵佐著作郎,並不
 就。簡文帝居籓,引為談客,自司徒左西屬轉撫軍掾、中書郎、散騎常侍、豫章太守,入為侍中。陳郡周勰為謝安主簿,居喪廢禮,崇尚莊老,脫落名教。伯領中正,不通勰,議曰:「拜下之敬,猶違眾從禮。情理之極,不宜以多比為通。」時人憚焉。」識者謂伯可謂澄世所不能澄,而裁世所不能裁者矣,與夫容己順眾者,豈得同時而共稱哉!



 王坦之又嘗著《公謙論》,袁宏作論以難之。伯覽而美其辭旨,以為是非既辯,誰與正之,遂作《辯謙》以折中曰:



 夫尋理辯疑,必先定其名分所存。所存既明,則彼我之趣可得而詳也。夫謙之為義,存乎降己者也。以高從卑,以賢
 同鄙,故謙名生焉。孤寡不穀,人之所惡,而侯王以自稱,降其貴者也。執御執射,眾之所賤,而君子以自目,降其賢才也。與夫山在地中之象,其致豈殊哉!捨此二者,而更求其義,雖南轅求冥,終莫近也。



 夫有所貴,故有降焉;夫有所美,故有謙焉。譬影響之與形聲,相與而立。道足者,忘貴賤而一賢愚;體公者,乘理當而均彼我。降挹之義,於何而生!則謙之為美,固不可以語至足之道,涉乎大方之家矣。然君子之行己,必尚於至當,而必造乎匿善。至理在乎無私,而動之於降己者何?誠由未能一觀於能鄙,則貴賤之情立;非忘懷於彼我,則私己之累存。
 當其所貴在我則矜,值其所賢能之則伐。處貴非矜,而矜己者常有其貴;言善非伐,而伐善者驟稱其能。是以知矜貴之傷德者,故宅心於卑素;悟驟稱之虧理者,故情存於不言。情存於不言,則善斯匿矣;宅心於卑素,則貴斯降矣。夫所況君子之流,茍理有未盡,情有未夷,存我之理未冥於內,豈不同心於降挹洗之所滯哉!體有而擬無者,聖人之德;有累而存理者,君子之情。雖所滯不同,其於遣情之累緣有弊而用,降己之道由私我而存,一也。故懲忿窒欲,著於《損》象;卑以自牧,實繫《謙》爻。皆所以存其所不足,拂其所有餘者也。



 王生之談,以至理
 無謙,近得之矣。云人有爭心,善不可收,假後物之迹,以逃動者之患,以語聖賢則可,施之於下斯者,豈惟逃患於外。亦所以洗心於內也。



 轉丹陽尹、吏部尚書、領軍將軍。既疾病,占候者云:「不宜此官。」朝廷改授太常,未拜,卒,時年四十九,即贈太常。子璯,官至衡陽太守。



 史臣曰:王湛門資台鉉,地處膏腴,識表鄰機,才惟王佐。旪宣尼之遠契,玩道韋編;遵伯陽之幽旨,含虛牝谷。所謂天質不雕,合於大朴者也。安期英姿挺秀,籍甚一時,朝野挹其風流,人倫推其表燭。雖崇勳懋績有闕於旂常,素德清規足傳於汗簡矣。懷祖鑒局夷遠,沖衿玉粹。
 坦之牆宇疑曠,逸操金貞。騰諷庾之良箋,情嗤語怪;演《廢莊》之宏論,道煥崇儒。或寄重文昌,允釐於袞職;或任華綸閣,密勿於王言。咸能克著徽音,保其榮秩,美矣!國寶檢行無聞,坐升彼相,混暗識於心鏡,開險路於情田。於時疆埸多虞,憲章罕備,天子居綴旒之連,人臣微覆餗之憂。於是竊勢擁權,黷明王之彞典;窮奢縱侈,假凶豎之餘威。繡桷雕楹,陵跨於宸極;麗珍冶質,充牣於帷房。亦猶犬彘腴肥,不知禍之將及。告盡私室,固其宜哉!荀景猷履孝居忠,無慚往烈。范玄平陳謀獻策,有會時機。崧則思業該通,緝遺經於已紊。汪則風飆直亮,抗高
 節於將顛,揚榷而言,俱為雅士。劉韓俊爽,標置軼群,勝氣籠霄,飛談卷霧,並蘭芬菊耀,無絕於終古矣。



 贊曰:處沖純懿,是稱奇器。養素虛庭,同塵下位。雅道雖屈,高風不墜。猗歟後胤,世傳清德。帝室馳芬,士林揚則。國寶庸暗,托意驕奢。既豐其屋,終蔀其家。荀範令望,金聲遠暢。劉韓秀士,珠談間起。異術同華,葳蕤青史。



\end{pinyinscope}