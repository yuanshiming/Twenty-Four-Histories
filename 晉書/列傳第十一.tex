\article{列傳第十一}

\begin{pinyinscope}

 魏舒李憙劉寔高光



 魏舒,字陽元,任城樊人也。少孤,為外家寧氏所養。寧氏起宅,相宅者云:「當出貴甥。」外祖母以魏氏甥小而慧,意謂應之。舒曰:「當為外氏成此宅相。」久乃別居。身長八尺二寸,姿望秀偉,飲酒石餘,而遲鈍質朴,不為鄉親所重。從叔父吏部郎衡,有名當世,亦不之知,使守水碓,每歎曰:「舒堪數百戶長,我願畢矣!」舒亦不以介意。不修常人
 之節,不為皎歷之事,每欲容才長物,終不顯人之短。性好騎射,著韋衣。入山澤,以漁獵為事。唯太原王乂謂舒曰:「卿終當為台輔,然今未能令妻子免飢寒,吾當助卿營之。」常振其匱乏,舒受而不辭。舒嘗詣野王,主人妻夜產,俄而聞車馬之聲,相問曰:「男也,女也?」曰:「男,書之,十五以兵死。」復問:「寢者為誰?」曰:「魏公舒。」後十五載,詣主人,問所生兒何在,曰:「因條桑為斧傷而死。」舒自知當為公矣。



 年四十餘,郡上計掾察孝廉。宗黨以舒無學業,勸令不就,可以為高耳。舒曰:「若試而不中,其負在我,安可虛竊不就之高以為己榮乎!」於是自課。百日習一經,因而對
 策升第。除澠池長,遷浚儀令,入為尚書郎。時欲沙汰郎官。非其才者罷之。舒曰:「吾即其人也。」襆被而出。同僚素無清論者咸有愧色,談者稱之。



 累遷後將軍鐘毓長史,毓每與參佐射,舒常為畫籌而已。後遇朋人不足,以舒滿數。毓初不知其善射。舒容範閑雅,發無不中,舉坐愕然。莫有敵者。毓嘆而謝曰:「吾之不足以盡卿才,有如此射矣,豈一事哉!」轉相國參軍,封劇陽子。府朝碎務,未嘗見是非;至於廢興大事,眾人莫能斷者,舒徐為籌之,多出眾議之表。文帝深器重之,每朝會坐罷,目送之曰:「魏舒堂堂,人之領袖也。」遷宜陽、滎陽二郡太守,甚有聲稱。
 徵拜散騎常侍。出為冀州刺史,在州三年,以簡惠稱。入為侍中。武帝以舒清素,特賜絹百匹。遷尚書,以公事當免官,詔以贖論。舒三娶妻皆亡,是歲自表乞假還本郡葬妻,詔賜葬地一頃,錢五十萬。



 太康初,拜右僕射。舒與衛瓘、山濤、張華等以六合混一,宜用古典封禪東嶽,前後累陳其事,帝謙讓不許。以舒為左僕射,領吏部。舒上言:「今選六宮,聘以玉帛,而舊使御府丞奉聘,宣成嘉禮,贄重使輕。以為拜三夫人宜使卿,九嬪使五官中郎將,美人、良人使謁者,於典制為弘。」有詔詳之,眾議異同,遂寢。加右光祿大夫、儀同三司。



 及山濤薨,以舒領司徒,有頃
 即真。舒有威重德望,祿賜散之九族,家無餘財。陳留周震累為諸府所辟,辟書既下,公輒喪亡,僉號震為殺公掾,莫有辟者。舒乃命之,而竟無患,識者以此稱其達命。以年老,每稱疾遜位。中復暫起,署兗州中正,尋又稱疾。尚書左丞郤詵與舒書曰:「公久疾小差,視事是也,唯上所念。何竟起訖還臥,曲身迴法,甚失具瞻之望。公少立巍巍,一旦棄之,可不惜哉!」舒稱疾如初。後以災異遜位,帝不聽。後因正旦朝罷還第,表送章綬。帝手詔敦勉。而舒執意彌固,乃下詔曰:「司徒、劇陽子舒,體道弘粹,思量經遠,忠肅居正,在公盡規。入管銓衡,官人允敘;出贊袞
 職,敷弘五教。惠訓播流,德聲茂著,可謂朝之俊乂者也。而屢執沖讓,辭旨懇誠,申覽反覆,省用憮然。蓋成人之美,先典所與,難違至情。今聽其所執,以劇陽子就第,位同三司,祿賜如前。几杖不朝,賜錢百萬,床帳簟褥自副。以舍人四人為劇陽子舍人,置官騎十人。使光祿勳奉策,主者詳案典禮,令皆如舊制。」於是賜安車駟馬,門施行馬。舒為事必先行而後言,遜位之際,莫有知者。時論以為晉興以來,三公能辭榮善終者,未之有也。司空衛瓘與舒書曰:「每與足下共論此事,日日未果,可謂瞻之在前,忽焉在後矣。」太熙元年薨,時年八十二。帝甚傷悼,
 賵賻優厚,謚曰康。



 子混,字延廣,清惠有才行,為太子舍人。年二十七,先舒卒,朝野咸為舒悲惜。舒每哀慟,退而歎曰:「吾不及莊生遠矣,豈以無益自損乎!」於是終服不復哭。詔曰:「舒惟一子,薄命短折。舒告老之年,處窮獨之苦,每念怛然,為之嗟悼。思所以散愁養氣,可更增滋味品物。仍給賜陽燧四望繐窗戶皁輪車牛一乘,庶出入觀望,或足散憂也。」以庶孫融嗣。又早卒,從孫晃嗣。



 李憙,字季和,上黨銅鞮人也。父牷,漢大鴻臚。憙少有高行,博學研精,與北海管寧以賢良徵,不行。累辟三府,不
 就。宣帝復辟憙為太傅屬,固辭疾,郡縣扶輿上道,時憙母疾篤,乃竊踰泫氏城而徒還,遂遭母喪,論者嘉其志節。後為并州別駕,時驍騎將軍秦朗過并州,州將畢軌敬焉。令乘車至閣。憙固諫以為不可,軌不得已從之。



 景帝輔政,命憙為大將軍從事中郎,憙到,引見,謂憙曰:「昔先公辟君而君不應,今孤命君而君至,何也?」對曰:「先君以禮見待,憙得以禮進退。明公以法見繩,憙畏法而至。」帝甚重之。轉司馬,尋拜右長史。從討毌丘儉還,遷御史中丞。當官正色,不憚強御,百僚震肅焉。薦樂安孫璞,亦以道德顯,時人稱為知人。尋遷大司馬,以公事免。



 司馬
 伷為寧北將軍,鎮鄴,以憙為軍司。頃之,除涼州刺史,加揚威將軍、假節,領護羌校尉,綏御華夷,甚有聲績。羌虜犯塞,憙因其隙會,不及啟聞,輒以便宜出軍深入,遂大剋獲,以功重免譴,時人比之漢朝馮、甘焉。於是請還,許之。居家月餘,拜冀州刺史,累遷司隸校尉。及魏帝告禪于晉,憙以本官行司徒事,副太尉鄭沖奉策。泰始初,封祁侯。



 憙上言:「故立進令劉友、前尚書山濤、中山王睦、故尚書僕射武陔各占官三更稻田,請免濤、睦等官。陔已亡,請貶謚。」詔曰:「法者,天下取正,不避親貴,然後行耳,吾豈將枉縱其間哉!然案此事皆是友所作,侵剝百姓,以
 繆惑朝士。姦吏乃敢作此,其考竟友以懲邪佞。濤等不貳其過者,皆勿有所問。《易》稱『王臣蹇蹇,匪躬之故』。今憙亢志在公,當官而行,可謂『邦之司直』者矣。光武有云:『貴戚且斂手以避二鮑』。豈其然乎!其申敕群僚,各慎所司,寬宥之恩,不可數遇也。」憙為二代司隸,朝野稱之。以公事免。



 其年,皇太子立,以憙為太子太傅。自魏明帝以後,久曠東宮,制度廢闕,官司不具,詹事、左右率、庶子、中舍人諸官並未置,唯置衛率令典兵,二傅并攝眾事。憙在位累年,訓道盡規。遷尚書僕射,拜特進、光祿大夫,以年老遜位。詔曰:「光祿大夫、特進李憙,杖德居義,當升台司。
 毗亮朕躬,而以年尊致仕。雖優游無為,可以頤神,而虛心之望,能不憮然!其因光祿之號,改假金紫,置官騎十人,賜錢五十萬,祿賜班禮,一如三司,門施行馬。」



 初,憙為僕射時,涼州虜寇邊,憙唱義遣軍討之。朝士謂出兵不易,虜未足為患,竟不從之。後虜果大縱逸,涼州覆沒,朝廷深悔焉。以憙清素貧儉,賜絹百匹。及齊王攸出鎮,憙上疏諫爭,辭甚懇切。憙自歷仕,雖清非異眾,而家無儲積,親舊故人乃至分衣共食,未嘗私以王官。及卒,追贈太保,謚曰成。子贊嗣。



 少子儉,字仲約,歷左積弩將軍、屯騎校尉。儉子弘字世彥,少有清節,永嘉末,歷給事黃門
 侍郎、散騎常侍。



 劉寔,字子真,平原高唐人也。漢濟北惠王壽之後也,父廣,斥丘令。寔少貧苦,賣牛衣以自給。然好學,手約繩,口誦書,博通古今。清身潔己,行無瑕玷。郡察孝廉,州舉秀才,皆不行。以計吏入洛,調為河南尹丞,遷尚書郎、廷尉正。後歷吏部郎,參文帝相國軍事,封循陽子。



 鐘會、鄧艾之伐蜀也,有客問寔曰:「二將其平蜀乎?」寔曰:「破蜀必矣,而皆不還。」客問其故,笑而不答,竟如其言。寔之先見,皆此類也。



 以世多進趣,廉遜道闕,乃著《崇讓論》以矯之。其
 辭曰:



 古之聖王之化天下,所以貴讓者,欲以出賢才,息爭競也。夫人情莫不欲已之賢也,故勸令讓賢以自明賢也,豈假讓不賢哉!故讓道興,賢能之人不求而自出矣,至公之舉自立矣,百官之副亦豫具矣。一官缺,擇眾官所讓最多者而用之,審之道也。在朝之士相讓於上,草廬之人咸皆化之,推賢讓能之風從此生矣。為一國所讓,則一國士也;天下所共推,則天下士也。推讓之風行,則賢與不肖灼然殊矣。此道之行,在上者無所用其心,因成清議,隨之而已。故曰,蕩蕩乎堯之為君,莫之能名。言天下自安矣,不見堯所以化之,故不能名也。又曰,
 舜禹之有天下而不與焉,無為而化者其舜也歟。賢人相讓於朝,大才之人恆在大官,小人不爭於野,天下無事矣。以賢才化無事,至道興矣。已仰其成,復何與焉!故可以歌《南風》之詩,彈五弦之琴也。成此功者非有他,崇讓之所致耳。孔子曰,能以禮讓為國,則不難也。



 在朝之人不務相讓久矣,天下化之。自魏代以來,登進辟命之士,及在職之吏,臨見受敘,雖自辭不能,終莫肯讓有勝己者。夫推讓之風息,爭競之心生。孔子曰,上興讓則下不爭,明讓不興下必爭也。推讓之道興,則賢能之人日見推舉;爭競之心生,則賢能之人日見謗毀。夫爭者之
 欲自先,甚惡能者之先,不能無毀也。故孔墨不能免世之謗己,況不及孔墨者乎!議者僉然言,世少高名之才,朝廷不有大才之人可以為大官者。山澤人小官吏亦復云,朝廷之士雖有大官名德,皆不及往時人也。餘以為此二言皆失之矣。非時獨乏賢也,時不貴讓。一人有先眾之譽,毀必隨之,名不得成使之然也。雖令稷契復存,亦不復能全其名矣。能否混雜,優劣不分,士無素定之價,官職有缺,主選之吏不知所用,但案官次而舉之。同才之人先用者,非勢家之子,則必為有勢者之所念也。非能獨賢,因其先用之資,而復遷之無已。遷之無已,不
 勝其任之病發矣。觀在官之人,政績無聞,自非勢家之子,率多因資次而進也。



 向令天下貴讓,士必由於見讓而後名成,名成而官乃得用之。諸名行不立之人,在官無政績之稱,讓之者必矣,官無因得而用之也。所以見用不息者,由讓道廢,因資用人之有失久矣。故自漢魏以來,時開大舉,令眾官各舉所知,唯才所任,不限階次,如此者甚數矣。其所舉必有當者,不聞時有擢用,不知何誰最賢故也。所舉必有不當,而罪不加,不知何誰最不肖也。所以不可得知,由當時之人莫肯相推,賢愚之名不別,令其如此。舉者知在上者察不能審,故敢漫舉
 而進之。或舉所賢,因及所念,一頓而至,人數猥多,各言所舉者賢,加之高狀,相似如一,難得而分矣。參錯相亂,真偽同貫,更復由此而甚。雖舉者不能盡忠之罪,亦由上開聽察之路濫,令其爾也。昔齊王好聽竽聲,必令三百人合吹而後聽之,廩以數人之俸。南郭先生不知吹竽者也,以三百人合吹可以容其不知,因請為王吹竽,虛食數人之俸。嗣王覺而改之,難彰先王之過。乃下令曰:「吾之好聞竽聲有甚於先王,欲一一列而聽之。」先生於此逃矣。推賢之風不立,濫舉之法不改,則南郭先生之徒盈於朝矣。才高守道之士日退,馳走有勢之門日
 多矣。雖國有典刑,弗能禁矣。



 夫讓道不興之弊,非徒賢人在下位,不得時進也,國之良臣荷重任者,亦將以漸受罪退矣。何以知其然也?孔子以為顏氏之子不貳過耳,明非聖人皆有過。寵貴之地欲之者多矣,惡賢能者塞其路,其過而毀之者亦多矣。夫謗毀之生,非徒空設,必因人之微過而甚之者也。毀謗之言數聞,在上者雖欲弗納,不能不杖所聞,因事之來而微察之也,無以,其驗至矣。得其驗,安得不理其罪。若知而縱之,王之威日衰,令之不行自此始矣。知而皆理之,受罪退者稍多,大臣有不自固之心。夫賢才不進,貴臣日疏,此有國者之
 深憂也。《詩》曰:「受祿不讓,至于已斯亡。」不讓之人憂亡不暇,而望其益國朝,不亦難乎!



 竊以為改此俗甚易耳。何以知之?夫一時在官之人,雖雜有凡猥之才,其中賢明者亦多矣,豈可謂皆不知讓賢為貴邪!直以其時皆不讓,習以成俗,故遂不為耳。人臣初除,皆通表上聞,名之謝章,所由來尚矣。原謝章之本意,欲進賢能以謝國恩也。昔舜以禹為司空,禹拜稽首,讓于稷契及咎繇。使益為虞官,讓于朱虎、熊、羆。使伯夷典三禮,讓于夔龍。唐虞之時,眾官初除,莫不皆讓也。謝章之義,蓋取於此。《書》記之者,欲以永世作則。季世所用,不賢不能讓賢,虛謝見
 用之恩而已。相承不變,習俗之失也。



 夫敘用之官得通章表者,其讓賢推能乃通,其不能有所讓徒費簡紙者,皆絕不通。人臣初除,各思推賢能而讓之矣,讓之文付主者掌之。三司有缺,擇三司所讓最多者而用之。此為一公缺,三公已豫選之矣。且主選之吏,不必任公而選三公,不如令三公自共選一公為詳也。四征缺,擇四征所讓最多而用之,此為一征缺,四征已豫選之矣,必詳於停缺而令主者選四征也。尚書缺,擇尚書所讓最多者而用之,此為八尚書共選一尚書,詳於臨缺令主者選八尚書也。郡守缺,擇眾郡所讓最多者而用之,詳於
 任主者令選百郡守也。



 夫以眾官百郡之讓,與主者共相比,不可同歲而論也。雖復令三府參舉官,本不委以舉選之任,各不能以根其心也。其所用心者裁之不二三,但令主者案官次而舉之,不用精也。賢愚皆讓,百姓耳目盡為國耳目。夫人情爭則欲毀己所不知,讓則競推於勝己。故世爭則毀譽交錯,優劣不分,難得而讓也。時讓則賢智顯出,能否之美歷歷相次,不可得而亂也。當此時也,能退身修已者,讓之者多矣。雖欲守貧賤,不可得也。馳騖進趣而欲人見讓,猶卻行而求前也。夫如此,愚智咸知進身求通,非修之於己則無由矣。游外求
 者,於此相隨而歸矣。浮聲虛論,不禁而自息矣。人人無所用其心,任眾人之議,而天下自化矣。不言之化行,巍巍之美於此著矣。讓可以致此,豈可不務之哉!



 《春秋傳》曰:「范宣子之讓,其下皆讓。欒黶雖汰,弗敢違也。晉國以平,數世賴之。」上世之化也,君子尚能而讓其下,小人力農以事其上,上下有禮,讒慝遠黜,由不爭也。及其亂也,國家之弊,恒必由之。篤論了了如此。在朝君子典選大官,能不以人廢言,舉而行之,各以讓賢舉能為先務,則群才猥出,能否殊別,蓋世之功,莫大於此。



 泰始初,進爵為伯,累遷少府。咸寧中為太常。轉尚書。杜預之伐吳也,
 寔以本官行鎮南軍司。



 初,寔妻盧氏生子躋而卒,華氏將以女妻之。寔弟智諫曰:「華家類貪,必破門戶。」辭之不得,竟婚華氏而生子夏。寔竟坐夏受賂,免官。頃之為大司農,又以夏罪免。



 寔每還州里,鄉人載酒肉以候之。寔難逆其意,輒共啖而返其餘。或謂寔曰:「君行高一世,而諸子不能遵。何不旦夕切磋,使知過而自改邪!」寔曰:「吾之所行,是所聞見,不相祖習,豈復教誨之所得乎!」世以寔言為當。



 後起為國子祭酒、散騎常侍。愍懷太子初封廣陵王,高選師友,以寔為師。元康初,進爵為侯,累遷太子太保,加侍中、特進、右光祿大夫、開府儀同三司,領冀
 州都督。九年,策拜司空,遷太保,轉太傅。太安初,寔以老病遜位,賜安車駟馬、錢百萬,以侯就第。及長沙成都之相攻也,寔為軍人所掠,潛歸鄉里。



 惠帝崩,寔赴山陵。懷帝即位,復授太尉。寔自陳年老,固辭,不許。左丞劉坦上言曰:「夫堂高級遠,主尊相貴。是以古之哲王莫不師其元臣,崇養老之教,訓示四海,使少長有禮。七十致仕,亦所以優異舊德,厲廉高之風。太尉寔體清素之操,執不渝之潔,懸車告老,二十餘年,浩然之志,老而彌篤。可謂國之碩老,邦之宗模。臣聞老者不以筋力為禮,寔年踰九十,命在日制,遂自扶輿,冒險而至,展哀山陵,致敬闕
 庭,大臣之節備矣。聖詔殷勤,必使寔正位上台,光飪鼎實,斷章敦喻,經涉二年。而寔頻上露板,辭旨懇誠。臣以為古之養老,以不事為優,不以吏之為重,謂宜聽寔所守。」



 三年,詔曰:「昔虞任五臣,致垂拱之化,漢相蕭何,興寧一之譽,故能光隆於當時,垂裕于百代。朕紹天明命,臨御萬邦,所以崇顯政道者,亦賴之於元臣庶尹,畢力股肱,以副至望。而君年耆告老,確然難違。今聽君以侯就第,位居三司之上,秩祿準舊,賜几杖不朝及宅一區。國之大政,將就諮于君,副朕意焉。」歲餘薨,時年九十一,謚曰元。



 寔少貧窶,杖策徒行,每所憩止,不累主人,薪水之
 事,皆自營給。及位望通顯,每崇儉素,不尚華麗。嘗詣石崇家,如廁,見有絳紋帳,裀褥甚麗,兩婢持香囊。寔便退,笑謂崇曰:「誤入卿內。」崇曰:「是廁耳。」寔曰:「貧士未嘗得此。」乃更如他廁。雖處榮寵,居無第宅,所得俸祿,贍恤親故。雖禮教陵遲,而行己以正。喪妻為廬杖之制,終喪不御內。輕薄者笑之,寔不以介意。自少及老,篤學不倦,雖居職務,卷弗離手。尤精《三傳》,辨正《公羊》,以為衛輒不應辭以王父命,祭仲失為臣之節,舉此二端以明臣子之體,遂行於世。又撰《春秋條例》二十卷。



 有二子,躋、夏。躋字景雲,官至散騎常侍。夏以貪污棄放於世。



 弟智,字子房,貞素有兄風。少貧窶,每負薪自給,讀誦不輟,竟以儒行稱。歷中書黃門吏部郎,出為潁川太守。平原管輅嘗謂人曰:「吾與劉潁川兄弟語,使人神思清發,昏不假寐。自此之外,殆白日欲寢矣。」入為秘書監,領南陽王師,加散騎常侍,遷侍中、尚書、太常。著《喪服釋疑論》,多所辨明。太康末卒,謚曰成。



 高光,字宣茂,陳留圉城人,魏太尉柔之子也。光少習家業,明練刑理。初以太子舍人累遷尚書郎,出為幽州刺史、潁州太守。是時武帝置黃沙獄,以典詔囚。以光歷世
 明法,用為黃沙御史,秩與中丞同,遷廷尉。元康中,拜尚書,典三公曹。時趙王倫篡逆,光於其際,守道全貞。及倫賜死,齊王冏輔政,復以光為廷尉,遷尚書,加奉車都尉。後從駕討成都王穎有勳,封延陵縣公,邑千八百戶。於時朝廷咸推光明於用法,故頻典理官。惠帝為張方所逼,幸長安,朝臣奔散,莫有從者,光獨侍帝而西。遷尚書左僕射,加散騎常侍。光兄誕為上官巳等所用,歷徐、雍二州刺史。誕性任放無倫次,而決烈過人,與光異操。常謂光小節,恒輕侮之,光事誕愈謹。帝既還洛陽,時太弟新立,重選傅訓,以光為少傅,加光祿大夫,常侍如故。及
 懷帝即位,加光祿大夫金章紫綬,與傅祗並見推崇。尋為尚書令,本官如故。以疾卒,贈司空、侍中。屬京洛傾覆,竟未加謚。



 子韜字子遠,放佚無檢。光為廷尉時,韜受貨賕,有司奏案之,而光不知。時人雖非光不能防閑其子,以其用心有素,不以為累。初,光詣長安留臺,以韜兼右衛將軍。韜與殿省小人交通,及光卒,仍於喪中往來不絕。時東海王越輔政,不朝觀。韜知人心有望,密與太傅參軍姜賾、京兆杜概等謀討越,事泄伏誅。



 史臣曰:下士競而文,中庸靜而質,不若進不足而退有餘也。魏舒、劉寔發慮精華,結綬登槐,覽止成務。季和切
 問近對,當官正色。詩云「貪人敗類」,豈劉夏之謂歟!



 贊曰:舒言不矜,憙對千乘。子真、宣茂,雅志難陵。進忠能舉,退讓攸興。皎皎瑚器,來光玉繩。



\end{pinyinscope}