\article{志第十六}

\begin{pinyinscope}

 食貨



 昔者先王量地以制邑,度地以居民,因三才以節其務,敬四序以成其業,觀其謠俗而正其紀綱。勖農桑之本,通魚鹽之利,登良山而採符玉,泛瀛海而罩珠璣。日中為市,總天下之隸,先諸布帛,繼以貨泉,貿遷有無,各得其所。《周禮》,正月始和,乃布教于象魏。若乃一夫之士,十畝之宅,三日之徭,九均之賦,施陽禮以興其讓,命春社
 以勖其耕。天之所貴者人也,明之所求者學也,治《經》入官,則君子之道焉。《詩》曰:「三之日于,四之日舉趾。」是以農官澤虞,各有攸次,父兄之習,不玩而成,十五從務,始勝衣服,鄉無遊手,邑不廢時,所謂厥初生民,各從其事者也。是以太公通市井之貨,以致齊國之強;鴟夷善發斂之居,以盛中陶之業。昔在金天,勤於民事,命春鳸以耕稼,召夏鳸以耘鋤,秋鳸所以收斂,冬鳸於焉蓋藏。《書》曰:「歷象日月星辰,敬授民時。」傳曰:「禹稷躬稼而有天下。」若乃九土既敷,四民承範,東吳有齒角之饒,西蜀有丹沙之富,兗豫漆絲之BW,燕齊怪石之府,秦邠旄羽,迥帶
 瑯玕,荊郢桂林,旁通竹箭,江幹橘柚,河外舟車,遼西旃罽之鄉,葱右蒲梢之駿,殖物怪錯,于何不有。若乃上法星象,下料無外,因天地之利,而總山海之饒,百畝之田,十一而稅,九年躬稼,而有三年之蓄,可以長孺齒,可以養耆年。因乎人民,用之邦國,宮室有度,旗章有序。朝聘自其儀,宴饗由其制,家殷國阜,遠至邇安。救水旱之災,恤寰瀛之弊,然後王之常膳,乃間笙鏞。商周之興,用此道也。辛紂暴虐,玩其經費,金鏤傾宮,廣延百里,玉飾鹿臺,崇高千仞,宮中九市,各有女司。厚賦以實鹿臺之錢,大斂以增鉅橋之粟,多發妖冶以充傾宮之麗,廣收珍
 玩以備沙丘之遊。懸肉成林,積醪為沼,使男女裸體相逐於其間,伏詣酒池中牛飲者三千餘人,宮中以錦綺為席,綾紈為薦。及周王誅紂,肅拜殷墟,乃盡振鹿財,並頒橋粟,上天降休,殷人大喜。王赧云季,徙都西周,九鼎淪沒,二南堙盡,貸於百姓,無以償之,乃上層臺以避其責,周人謂王所居為逃責臺者也。昔周姬公制以六典,職方陳其九貢,頒財內府,永為不刊。及刑政陵夷,菁茅罕至,魯侯初踐畝之稅,秦君收太半之入,前王之範,靡有孑遺。史臣曰:班固為《殖貨志》,自三代至王莽之誅,網羅前載,其文詳悉。



 光武寬仁,龔行天討,王莽之後,赤眉
 新敗,雖復三暉乃眷,而九服蕭條,及得隴望蜀,黎民安堵,自此始行五銖之錢,田租三十稅一,民有產子者復以三年之算。顯宗即位,天下安寧,民無橫徭,歲比登稔。永平五年作常滿倉,立粟市於城東,粟斛直錢二十。草樹殷阜,牛羊彌望,作貢尤輕,府廩還積,姦回不用,禮義專行。于時東方既明,百官詣闕,戚里侯家,自相馳騖,車如流水,馬若飛龍,照映軒廡,光華前載。傳曰:「三統之元,有陰陽之九焉」,蓋天地之恒數也。安帝永初三年,天下水旱,人民相食。帝以鴻陂之地假與貧民。以用度不足,三公又奏請令吏民入錢穀得為關內侯云。桓帝永興
 元年,郡國少半遭蝗,河泛數千里,流人十餘萬戶,所在廩給。迨建寧永和之初,西羌反叛,二十餘年兵連師老,軍旅之費三百二十餘億,府帑空虛,延及內郡。沖質短祚,桓靈不軌。中平二年,南宮災,延及北闕。於是復收天下田畝十錢,用營宮宇。帝出自侯門,居貧即位,常曰:「桓帝不能作家,曾無私蓄。」故於西園造萬金堂,以為私藏。復寄小黃門私錢,家至巨億。於是懸鴻都之,開賣官之路,公卿以降,悉有等差。廷尉崔烈入錢五百萬以買司徒,刺史二千石遷除,皆責助治宮室錢,大郡至二千萬錢,不畢者或至自殺。獻帝作五銖錢,而有四道連於
 邊緣。有識者尤之曰:「豈京師破壞,此錢四出也。」



 及董卓尋戈,火焚宮室,乃劫鸞駕,西幸長安,悉壞五銖錢,更鑄小錢,盡收長安及洛陽銅人飛廉之屬,以充鼓鑄。又錢無輪郭,文章不便。時人以為秦始皇見長人於臨洮,乃鑄銅人。卓,臨洮人也,興毀不同,凶訛相類。及卓誅死,李傕、郭汜自相攻伐,於長安城中以為戰地。是時穀一斛五十萬,豆麥二十萬,人相食啖,白骨盈積,殘骸餘肉,臭穢道路。帝使侍御史侯汶出太倉米豆,為飢民作糜,經日頒布而死者愈多。帝於是始疑有司盜其糧廩,乃親於御前自加臨給,飢者人皆泣曰:「今始得耳!」帝東歸也,
 李傕、郭汜等追敗乘輿於曹陽,夜潛渡河,六宮皆步。初出營欄,后手持縑數匹,董承使符節令孫徽以刃脅奪之,殺旁侍者,血濺后服。既至安邑,御衣穿敗,唯以野棗園菜以為餱糧。自此長安城中盡空,並皆四散,二三年間,關中無復行人。建安元年,車駕至洛陽,宮闈蕩滌,百官披荊棘而居焉。州郡各擁強兵,而委輸不至,尚書郎官自出採穭,或不能自反,死於墟巷。



 魏武之初,九州雲擾,攻城掠地,保此懷民,軍旅之資,權時調給。于時袁紹軍人皆資椹棗,袁術戰士取給蠃蒲。魏武于是乃募良民屯田許下,又於州郡列置田官,歲有數千萬斛,以充
 兵戎之用。及初平袁氏,以定鄴都,令收田租畝粟四升,戶絹二匹而綿二斤,餘皆不得擅興,藏強賦弱。文帝黃初二年,以穀貴,始罷五銖錢。于時天下未並,戎車歲動,孔子曰,「加之以師旅,因之以饑饉」,此言兵凶之謀而沴氣應之也。于時三方之人,志相吞滅,戰勝攻取,耕夫釋耒,江淮之鄉,尤缺儲峙。吳上大將軍陸遜抗疏,請令諸將各廣其田。權報曰:「甚善。今孤父子親自受田,車中八牛,以為四耦。雖未及古人,亦欲與眾均其勞也。」有吳之務農重穀,始於此焉。魏明帝不恭,淫於宮籞,百僚編於手役,天下失其躬稼。此後關東遇水,民亡產業,而興師
 遼陽,坐甲江甸,皆以國乏經用,胡可勝言。



 世祖武皇帝太康元年,既平孫皓,納百萬而罄三吳之資,接千年而總西蜀之用,韜干戈於府庫,破舟船於江壑,河濱海岸,三丘八藪,耒耨之所不至者,人皆受焉。農祥晨正,平秩東作,荷鍤贏糧,有同雲布。若夫因天而資五緯,因地而興五材,世屬升平,物流倉府,宮闈增飾,服玩相輝。於是王君夫、武子、石崇等更相誇尚,輿服鼎俎之盛,連衡帝室,布金埒之泉,粉珊瑚之樹,物盛則衰,固其宜也。永寧之初,洛中尚有錦帛四百萬,珠寶金銀百餘斛。惠后北征,蕩陰反駕,寒桃在御,隻雞以給,其布衾兩幅,囊錢三
 千,以為車駕之資焉。懷帝為劉曜所圍,王師累敗,府帑既竭,百官飢甚,比屋不見火煙,飢人自相啖食。愍皇西宅,餒饉弘多,斗米二金,死者太半。劉曜陳兵,內外斷絕,十之曲,屑而供帝,君臣相顧,莫不揮涕。元后渡江,軍事草創,蠻陬賧布,不有恒準,中府所儲,數四千匹。于時石勒勇銳,挻亂淮南,帝懼其侵逼,甚患之,乃詔方鎮云,有斬石勒首者,賞布千匹云。



 漢自董卓之亂,百姓流離,穀石至五十餘萬,人多相食。魏武既破黃巾,欲經略四方,而苦軍食不足,羽林監潁川棗祗建置屯田議。魏武乃令曰:「夫定國之術在於強
 兵足食,秦人以急農兼天下,孝武以屯田定西域,此先世之良式也。」於是以任峻為典農中郎將,募百姓屯田許下,得穀百萬斛。郡國列置田官,數年之中,所在積粟,倉廩皆滿。祗死,魏武後追思其功,封爵其子。建安初,關中百姓流入荊州者十餘萬家,及聞本土安寧,皆企望思歸,而無以自業。於是衛覬議為「鹽者國之大寶,自喪亂以來放散,今宜如舊置使者監賣,以其直益市犁牛,百姓歸者以供給之。勤耕積粟,以豐殖關中,遠者聞之,必多競還。」於是魏武遣謁者僕射監鹽官,移司隸校尉居弘農。流人果還,關中豐實。既而又以沛國劉馥為揚
 州刺史,鎮合肥,廣屯田,修芍陂、茹陂、七門、吳塘諸堨,以溉稻田,公私有蓄,歷代為利。賈逵之為豫州,南與吳接,修守戰之具,堨汝水,造新陂,又通運渠二百餘里,所謂賈侯渠者也。當黃初中,四方郡守懇田又加,以故國用不匱。時濟北顏斐為京兆太守,京兆自馬超之亂,百姓不專農殖,乃無車牛。斐又課百姓,令閑月取車材,轉相教匠。其無牛者令養豬,投貴賣以買牛。始者皆以為煩,一二年中編戶皆有車牛,於田役省贍,京兆遂以豐沃。鄭渾為沛郡太守,郡居下濕,水澇為患,百姓飢乏。渾於蕭、相二縣興陂堨,開稻田,郡人皆不以為便。渾以為終
 有經久之利,遂躬率百姓興功,一冬皆成。比年大收,頃畝歲增,租入倍常,郡中賴其利,刻石頌之,號曰鄭陂。魏明帝世徐邈為涼州,土地少雨,常苦乏穀。邈上修武威、酒泉鹽池,以收虜穀。又廣開水田,募貧民佃之,家家豐足,倉庫盈溢。及度支州界軍用之餘,以市金錦犬馬,通供中國之費,西域人入貢,財貨流通,皆邈之功也。其後皇甫隆為敦煌太守,敦煌俗不作耬犁,及不知用水,人牛功力既費,而收穀更少。隆到,乃教作耬犁,又教使灌溉。歲終率計,所省庸力過半,得穀加五,西方以豐。



 嘉平四年,關中饑,宣帝表徙冀州農夫五千人佃上邽,興京
 兆、天水、南安鹽池,以益軍實。青龍元年,開成國渠自陳倉至槐里;築臨晉陂,引汧洛溉舄鹵之地三千餘頃,國以充實焉。正始四年,宣帝又督諸軍伐吳將諸葛恪,焚其積聚,恪棄城遁走。帝因欲廣田積穀,為兼并之計,乃使鄧艾行陳、項以東,至壽春地。艾以為田良水少,不足以盡地利,宜開河渠,可以大積軍糧,又通運漕之道。乃著《濟河論》以喻其指。又以為昔破黃巾,因為屯田,積穀許都,以制四方。今三隅已定,事在淮南。每大軍征舉,運兵過半,功費巨億,以為大役。陳蔡之間,土下田良,可省許昌左右諸稻田,並水東下。令淮北二萬人、淮南三萬
 人分休,且佃且守。水豐,常收三倍於西,計除眾費,歲完五百萬斛以為軍資。六七年間,可積三千萬餘斛於淮北,此則十萬之眾五年食也。以此乘敵,無不剋矣。宣帝善之,皆如艾計施行。遂北臨淮水,自鐘離而南橫石以西,盡沘水四百餘里,五里置一營,營六十人,且佃且守。兼修廣淮陽、百尺二渠,上引河流,下通淮潁,大治諸陂於潁南、潁北,穿渠三百餘里,溉田二萬頃,淮南、淮北皆相連接。自壽春到京師,農官兵田,雞犬之聲,阡陌相屬。每東南有事,大軍出征,泛舟而下,達于江淮,資食有儲,而無水害,艾所建也。



 及晉受命,武帝欲平一江表。時穀
 賤而布帛貴,帝欲立平糴法,用布帛市穀,以為糧儲。議者謂軍資尚少,不宜以貴易賤。泰始二年,帝乃下詔曰:「夫百姓年豐則用奢,凶荒則窮匱,是相報之理也。故古人權量國用,取贏散滯,有輕重平糴之法。理財鈞施,惠而不費,政之善者也。然此事廢久,天下希習其宜。加以官蓄未廣,言者異同,財貨未能達通其制。更令國寶散於穰歲而上不收,貧弱困於荒年而國無備。豪人富商,挾輕資,蘊重積,以管其利。故農夫苦其業,而末作不可禁也。今者省徭務本,并力墾殖,欲令農功益登,耕者益勸,而猶或騰踴,至於農人並傷。今宜通糴,以充儉乏。主
 者平議,具為條制。」然事竟未行。是時江南未平,朝廷厲精於稼墻。四年正月丁亥,帝親耕藉田。庚寅,詔曰:「使四海之內,棄末反本,競農務功,能奉宣朕志,令百姓勸事樂業者,其唯郡縣長吏乎!先之勞之,在於不倦。每念其經營職事,亦為勤矣。其以中左典牧種草馬,賜縣令長相及郡國丞各一匹。」是歲,乃立常平倉,豐則糴,儉則糶,以利百姓。五年正月癸巳,敕戒郡國計吏、諸郡國守相令長,務盡地利,禁遊食商販。其休假者令與父兄同其勤勞,豪勢不得侵役寡弱,私相置名。十月,詔以「司隸校尉石鑒所上汲郡太守王宏勤恤百姓,導化有方,督勸
 開荒五千餘頃,遇年普饑而郡界獨無匱乏,可謂能以勸教,時同功異者矣。其賜穀千斛,布告天下」。八年,司徒石苞奏:「州郡農桑未有殿最之制,宜增掾屬令史,有所循行。」帝從之。事見《石苞傳》。苞既明於勸課,百姓安之。十年,光祿勛夏侯和上修新渠、富壽、遊陂三渠,凡溉田千五百頃。



 咸寧元年十二月,詔曰:「出戰入耕,雖自古之常,然事力未息,未嘗不以戰士為念也。今以鄴奚官奴婢著新城,代田兵種稻,奴婢各五十人為一屯,屯置司馬,使皆如屯田法。」三年,又詔曰:「今年霖雨過差,又有蟲災。潁川、襄城自春以來,略不下種,深以為慮。主者何以為
 百姓計,促處當之。」杜預上疏曰:



 臣輒思惟,今者水災東南特劇,非但五稼不收,居業并損,下田所在停污,高地皆多磽脊,此即百姓困窮方在來年。雖詔書切告長吏二千石為之設計,而不廓開大制,定其趣舍之宜,恐徒文具,所益蓋薄。當今秋夏蔬食之時,而百姓已有不贍,前至冬春,野無青草,則必指仰官穀,以為生命。此乃一方之大事,不可不豫為思慮者也。



 臣愚謂既以水為困,當恃魚菜螺蜯,而洪波泛濫,貧弱者終不能得。今者宜大壞兗、豫州東界諸陂,隨其所歸而宣導之。交令饑者盡得水產之饒,百姓不出境界之內,旦暮野食,此目下
 日給之益也。水去之後,填淤之田,畝收數鐘。至春大種五穀,五穀必豐,此又明年益也。



 臣前啟,典牧種牛不供耕駕,至於老不穿鼻者,無益於用,而徒有吏士穀草之費,歲送任駕者甚少,尚復不調習,宜大出賣,以易穀及為賞直。



 詔曰:「孳育之物,不宜減散。」事遂停寢。問主者,今典虞右典牧種產牛,大小相通,有四萬五千餘頭。茍不益世用,頭數雖多,其費日廣。古者匹馬丘牛,居則以耕,出則以戰,非如豬羊類也。今徒養宜用之牛,終為無用之費,甚失事宜。東南以水田為業,人無牛犢。今既壞陂,可分種牛三萬五千頭,以付二州將吏士庶,使及春耕。
 穀登之後,頭責三百斛。是為化無用之費,得運水次成穀七百萬斛,此又數年後之益也。加以百姓降丘宅土,將來公私之饒乃不可計。其所留好種萬頭,可即令右典牧都尉官屬養之。人多畜少,可並佃牧地,明其考課。此又三魏近甸,歲當復入數十萬斛穀,牛又皆當調習,動可駕用,皆今日之可全者也。」



 預又言:



 諸欲修水田者,皆以火耕水耨為便。非不爾也,然此事施於新田草萊,與百姓居相絕離者耳。往者東南草創人稀,故得火田之利。自頃戶口日增,而陂堨歲決,良田變生蒲葦,人居沮澤之際,水陸失宜,放牧絕種,樹木立枯,皆陂之害也。
 陂多則土薄水淺,潦不下潤。故每有水雨,輒復橫流,延及陸田。言者不思其故,因云此土不可陸種。臣計漢之戶口,以驗今之陂處,皆陸業也。其或有舊陂舊堨,則堅完脩固,非今所謂當為人害者也。臣前見尚書胡威啟宜壞陂,其言懇至。臣中者又見宋侯相應遵上便宜,求壞泗陂,徙運道。時下都督度支共處當,各據所見,不從遵言。臣案遵上事,運道東詣壽春,有舊渠,可不由泗陂。泗陂在遵地界壞地凡萬三千餘頃,傷敗成業。遵縣領應佃二千六百口,可謂至少,而猶患地狹,不足肆力,此皆水之為害也。當所共恤,而都督度支方復執異,非所
 見之難,直以不同害理也。人心所見既不同,利害之情又有異。軍家之與郡縣,士大夫之與百姓,其意莫有同者,此皆偏其利以忘其害者也。此理之所以未盡,而事之所以多患也。



 臣又案,豫州界二度支所領佃者,州郡大軍雜士,凡用水田七千五百餘頃耳,計三年之儲,不過二萬餘頃。以常理言之,無為多積無用之水,況於今者水澇湓溢,大為災害。臣以為與其失當,寧瀉之不滀。宜發明詔,敕刺史二千石,其漢氏舊陂舊堨及山谷私家小陂,皆當修繕以積水。其諸魏氏以來所造立,及諸因雨決溢蒲葦馬腸陂之類,皆決瀝之。長吏二千石躬
 親勸功,諸食力之人並一時附功令,比及水凍,得粗枯涸,其所修功實之人皆以俾之。其舊陂堨溝渠當有所補塞者,皆尋求微跡,一如漢時故事,豫為部分列上,須冬,東南休兵交代,各留一月以佐之。夫川瀆有常流,地形有定體,漢氏居人眾多,猶以無患,今因其所患而宣寫之,跡古事以明近,大理顯然,可坐論而得。臣不勝愚意,竊謂最是今日之實益也。



 朝廷從之。



 及平吳之後,有司又奏:「詔書『王公以國為家,京城不宜復有田宅。今未暇作諸國邸,當使城中有往來處,近郊有芻槁之田』。今可限之,國王公侯,京城得有一宅之處。近郊田,大國田
 十五頃,次國十頃,小國七頃。城內無宅城外有者,皆聽留之。」



 又制戶調之式:丁男之戶,歲輸絹三匹,綿三斤,女及次丁男為戶者半輸。其諸邊郡或三分之二,遠者三分之一。夷人輸賨布,戶一匹,遠者或一丈。男子一人占田七十畝,女子三十畝。其外丁男課田五十畝,丁女二十畝,次丁男半之,女則不課。男女年十六已上至六十為正丁,十五已下至十三、六十一已上至六十五為次丁,十二已下六十六已上為老小,不事。遠夷不課田者輸義米,戶三斛,遠者五斗,極遠者輸算錢,人二十八文。其官品第一至于第九,各以貴賤占田,品第一者占五
 十頃,第二品四十五頃,第三品四十頃,第四品三十五頃,第五品三十頃,第六品二十五頃,第七品二十頃,第八品十五頃,第九品十頃。而又各以品之高卑蔭其親屬,多者及九族,少者三世。宗室、國賓、先賢之後及士人子孫亦如之。而又得蔭人以為衣食客及佃客,品第六已上得衣食客三人,第七第八品二人,第九品及舉輦、跡禽、前驅、由基、強弩、司馬、羽林郎、殿中冗從武賁、殿中武賁、持椎斧武騎武賁、持鈒冗從武賁、命中武賁武騎一人。其應有佃客者,官品第一第二者佃客無過五十戶,第三品十戶,第四品七戶,第五品五戶,第六品三戶,
 第七品二戶,第八品第九品一戶。



 是時天下無事,賦稅平均,人咸安其業而樂其事。及惠帝之後,政教陵夷,至於永嘉,喪亂彌甚。雍州以東,人多饑乏,更相鬻賣,奔迸流移,不可勝數。幽、并、司、冀、秦、雍六州大蝗,草木及牛馬毛皆盡。又大疾疫,兼以飢饉。百姓又為寇賊所殺,流尸滿河,白骨蔽野。劉曜之逼,朝廷議欲遷都倉垣。人多相食,飢疫總至,百官流亡者十八九。



 元帝為晉王,課督農功,詔二千石長吏以入穀多少為殿最。其非宿衛要任,皆宜赴農,使軍各自佃作,即以為廩。太興元年,詔曰:「徐、揚二州土宜三麥,可督令地,投秋下種,至夏而熟,繼
 新故之交,於以周濟,所益甚大。昔漢遣輕車使者氾勝之督三輔種麥,而關中遂穰。勿令後晚。」其後頻年麥雖有旱蝗,而為益猶多。二年,三吳大飢,死者以百數,吳郡太守鄧攸輒開倉廩賑之。元帝時使黃門侍郎虞斐、桓彞開倉廩振給,并省眾役。百官各上封事,後軍將軍應詹表曰:「夫一人不耕,天下必有受其飢者。而軍興以來,征戰運漕,朝廷宗廟,百官用度,既已殷廣,下及工商流寓僮僕不親農桑而遊食者,以十萬計。不思開立美利,而望國足人給,豈不難哉!古人言曰,飢寒並至,雖堯舜不能使野無寇盜;貧富并兼,雖皋陶不能使強不陵弱。
 故有國有家者,何嘗不務農重穀。近魏武皇帝用棗祗、韓浩之議,廣建屯田,又於征伐之中,分帶甲之士,隨宜開墾,故下不甚勞,而大功克舉也。間者流人奔東吳,東吳今儉,皆已還反。江西良田,曠廢未久,火耕水耨,為功差易。宜簡流人,興復農官,功勞報賞,皆如魏氏故事。一年中與百姓,二年分稅,三年計賦稅以使之,公私兼濟,則倉盈庾億,可計日而待也。」又曰:「昔高祖使蕭何鎮關中,光武令寇恂守河內,魏武委鐘繇以西事,故能使八表夷蕩,區內輯寧。今中州蕭條,未蒙疆理,此兆庶所以企望。壽春一方之會,去此不遠,宜選都督有文武經略
 者,遠以振河洛之形勢,近以為徐豫之籓鎮,綏集流散,使人有攸依,專委農功,令事有所局。趙充國農於金城,以平西零;諸葛亮耕於渭濱,規抗上國。今諸軍自不對敵,皆宜齊課。



 咸和五年,成帝始度百姓田,取十分之一,率畝稅米三升。六年,以海賊寇抄,運漕不繼,發王公以下餘丁,各運米六斛。是後頻年水災旱蝗,田收不至。咸康初,算度田稅米,空懸五十餘萬斛,尚書褚裒以下免官。穆帝之世,頻有大軍,糧運不繼,制王公以下十三戶共借一人,助度支運。升平初,荀羨為北府都督,鎮下邳,起田于東陽之石鱉,公私利之。哀帝即位,乃減田租,畝
 收二升。孝武太元二年,除度田收租之制,王公以下口稅三斛,唯蠲在役之身。八年,又增稅米,口五石。至於末年,天下無事,時和年豐,百姓樂業,穀帛殷阜,幾乎家給人足矣。



 漢錢舊用五銖,自王莽改革,百姓皆不便之。及公孫述僭號於蜀,童謠曰:「黃牛白腹,五銖當復。」好事者竊言,王莽稱黃,述欲繼之,故稱白帝。五銖漢貨,言漢當復併天下也。至光武中興,除莽貨泉。建武十六年,馬援又上書曰:「富國之本,在於食貨,宜如舊鑄五銖錢。」帝從之。於是復鑄五銖錢,天下以為便。及章帝時,穀帛價貴,縣官經
 用不足,朝廷憂之。尚書張林言:「今非但穀貴也,百物皆貴,此錢賤故爾。宜令天下悉以布帛為租,市買皆用之,封錢勿出,如此則錢少物皆賤矣。又,鹽者食之急也,縣官可自賣鹽,武帝時施行之,名曰均輸。」於是事下尚書通議。尚書朱暉議曰:「王制,天子不言有無,諸侯不言多少,食祿者不與百姓爭利。均輸之法,與賈販無異。以布帛為租,則吏多姦。官自賣鹽,與下爭利,非明王所宜行。」帝本以林言為是,得暉議,因發怒,遂用林言,少時復止。



 桓帝時有上書言:「人以貨輕錢薄,故致貧困,宜改鑄大錢。」事下四府群僚及太學能言之士。孝廉劉陶上議曰:



 臣伏讀鑄錢之詔,平輕重之義,訪覃幽微,不遺窮賤,是以藿食之人,謬延逮及。



 蓋以當今之憂,不在於貨,在乎人飢。是以先王觀象育物,敬授民時,使男不逋畝,女不下機,故君臣之道行,王路之教通。由是言之,食者乃有國之所寶,百姓之至貴也。竊以比年已來,良苗盡於蝗螟之口,杼柚空於公私之求。所急朝夕之食,所患靡盬之事,豈謂錢之厚薄,銖兩之輕重哉!就使當今沙礫化為南金,瓦石變為和玉,使百姓渴無所飲,飢無所食,雖皇羲之純德,唐虞之文明,猶不能以保蕭墻之內也。蓋百姓可百年無貨,不可以一朝有飢,故食為至急也。



 議
 者不達農殖之本,多言鑄冶之便,或欲因緣行詐,以賈國利。國利將盡,取者爭競,造鑄之端,於是乎生。蓋萬人鑄之,一人奪之,猶不能給,況今一人鑄之則萬人奪之乎!雖以陰陽為炭,萬物為銅,役不食之民,使不飢之士,猶不能足無厭之求也。



 夫欲民財殷阜,要在止役禁奪,則百姓不勞而足。陛下聖德,愍海內之憂戚,傷天下之艱難,欲鑄錢齊貨,以救其弊,此猶養魚沸鼎之中,棲鳥列火之上。木水,本魚鳥之所生也,用之不時,必至焦爛。願陛下寬鍥薄之禁,後冶鑄之議也。



 帝竟不鑄錢。



 及獻帝初平中,董卓乃更鑄小錢,由是貨輕而物貴,穀一
 斛至錢數百萬。至魏武為相,於是罷之,還用五銖。是時不鑄錢既久,貨本不多,又更無增益,故穀賤無已。及黃初二年,魏文帝罷五銖錢,使百姓以穀帛為市。至明帝世,錢廢穀用既久,人間巧偽漸多,競濕穀以要利,作薄絹以為市,雖處以嚴刑而不能禁也。司馬芝等舉朝大議,以為用錢非徒豐國,亦所以省刑。今若更鑄五銖錢,則國豐刑省,於事為便。魏明帝乃更立五銖錢,至晉用之,不聞有所改創。孫權嘉禾五年,鑄大錢一當五百。赤烏元年,又鑄當千錢。故呂蒙定荊州,孫權賜錢一億。錢既太貴,但有空名,人間患之。權聞百姓不以為便,省息
 之,鑄為器物,官勿復出也。私家有者,並以輸藏,平卑其直,勿有所枉。



 晉自中原喪亂,元帝過江,用孫氏舊錢,輕重雜行,大者謂之比輪,中者謂之四文。吳興沈充又鑄小錢,謂之沈郎錢。錢既不多,由是稍貴。孝武太元三年,詔曰:「錢,國之重寶,小人貪利,銷壞無已,監司當以為意。廣州夷人寶貴銅鼓,而州境素不出銅,聞官私賈人皆於此下貪比輪錢斤兩差重,以入廣州,貨與夷人,鑄敗作鼓。其重為禁制,得者科罪。」安帝元興中,桓玄輔政,立議欲廢錢用穀帛。孔琳之議曰:



 《洪範》八政,貨為食次,豈不以交易所資,為用之至要者乎!若使百姓用力於為
 錢,則是妨為生之業,禁之可也。今農自務穀,工自務器,各隸其業,何嘗致勤於錢。故聖王制無用之貨,以通有用之財,既無毀敗之費,又省難運之苦,此錢所以嗣功龜貝,歷代不廢者也。穀帛為寶,本充衣食,分以為貨,則致損甚多。又勞毀於商販之手,秏棄於割截之用,此之為弊,著自於曩。故鐘繇曰,巧偽之人,競濕穀以要利,制薄絹以充資。魏世制以嚴刑,弗能禁也。是以司馬芝以為用錢非徒豐國,亦所以省刑。錢之不用,由於兵亂積久,自致於廢,有由而然,漢末是也。今既用而廢之,則百姓頓亡其利。今括囊天下之穀,以周天下之食,或倉廩
 充溢,或糧靡並儲,以相資通,則貧者仰富。致富之道,實假於錢,一朝斷之,便為棄物。是有錢無糧之人,皆坐而飢困,以此斷之,又立弊也。



 且據今用錢之處,不以為貧,用穀之處,不以為富。又人習來久,革之必惑。語曰,利不百,不易業,況又錢便于穀邪!魏明帝時錢廢,穀用既久,不以便於人,乃舉朝大議。精才達政之士莫不以宜復用錢,下無異情,朝無異論。彼尚舍穀帛而用錢,足以明穀帛之弊著於已誡也。



 世或謂魏氏不用錢久,積累巨萬,故欲行之,利公富國,斯殆不然。晉文後舅犯之謀,而先成季之信,以為雖有一時之勛,不如萬世之益。于時
 名賢在列,君子盈朝,大謀天下之利害,將定經國之要術。若穀實便錢,義不昧當時之近利,而廢永用之通業,斷可知矣。斯實由困而思革,改而更張耳。近孝武之末,天下無事,時和年豐,百姓樂業,穀帛殷阜,幾乎家給人足,驗之實事,錢又不妨人也。



 頃兵革屢興,荒饉薦及,饑寒未振,實此之由。公既援而拯之,大革視聽,弘敦本之教,明廣農之科,敬授人時,各從其業,游蕩知反,務末自休,同以南畝競力,野無遺壤矣。於此以往,將升平必至,何衣食之足恤!愚謂救弊之術,無取於廢錢。



 朝議多同琳之,故玄議不行。



\end{pinyinscope}