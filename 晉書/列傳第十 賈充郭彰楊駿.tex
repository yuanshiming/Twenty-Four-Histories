\article{列傳第十 賈充郭彰楊駿}

\begin{pinyinscope}

 賈充郭彰楊駿



 賈充,字公閭,平陽襄陵人也。父逵,魏豫州刺史、陽里亭侯。逵晚始生充,言後當有充閭之慶,故以為名字焉。充少孤,居喪以孝聞。襲父爵為侯。拜尚書郎,典定科令,兼度支考課。辯章節度,事皆施用。累遷黃門侍郎、汲郡典農中郎將。參大將軍軍事,從景帝討毌丘儉、文欽於樂嘉。帝疾篤,還許昌,留充監諸軍事,以勞增邑三百五十
 戶。



 後為文帝大將軍司馬,轉右長史。帝新執朝權,恐方鎮有異議,使充詣諸葛誕,圖欲伐吳,陰察其變。充既論說時事,因謂誕曰:「天下皆願禪代,君以為如何?」誕歷聲曰:「卿非賈豫州子乎,世受魏恩,豈可欲以社稷輸人乎!若洛中有難,吾當死之。」充默然。及還,白帝曰:「誕在再揚州,威名夙著,能得人死力。觀其規略,為反必也。今征之,反速而事小;不征,事遲而禍大。」帝乃徵誕為司空,而誕果叛。復從征誕,充進計曰:「楚兵輕而銳,若深溝高壘以逼賊城,可不戰而剋也。」帝從之。城陷,帝登壘以勞充。帝先歸洛陽,使充統後事,進爵宣陽鄉侯,增邑千戶。遷廷尉,
 充雅長法理,有平反之稱。



 轉中護軍,高貴鄉公之攻相府也,充率眾距戰於南闕。軍將敗,騎督成倅弟太子舍人濟謂充曰:「今日之事如何?」充曰:「公等養汝,正擬今日,復何疑!」濟於是抽戈犯蹕。及常道鄉公即位,進封安陽鄉侯,增邑千二百戶,統城外諸軍,加散騎常侍。



 鐘會謀反於蜀,帝假充節,以本官都督關中、隴右諸軍事,西據漢中,未至而會死。時軍國多事,朝廷機密,皆與籌之。帝甚信重充,與裴秀、王沈、羊祜、荀勖同受腹心之任。帝又命充定法律。假金章,賜甲第一區。五等初建,封臨沂侯,為晉元勛,深見寵異,祿賜常優於群官。



 充有刀筆才,能
 觀察上旨。初,文帝以景帝恢贊王業,方傳位於舞陽侯攸。充稱武帝寬仁,且又居長,有人君之德,宜奉社稷。及文帝寢疾,武帝請問後事。文帝曰:「知汝者賈公閭也。」帝襲王位,拜充晉國衛將軍、儀同三司、給事中,改封臨潁侯。及受禪,充以建明大命,轉車騎將軍、散騎常侍、尚書僕射,更封魯郡公,母柳氏為魯國太夫人。



 充所定新律既班于天下,百姓便之。詔曰:「漢氏以來,法令嚴峻。故自元成之世,及建安、嘉平之間,咸欲辯章舊典,刪革刑書。述作體大,歷年無成。先帝愍元元之命陷於密網,親發德音,釐正名實。車騎將軍賈充,獎明聖意,諮詢善道。太
 傅鄭沖,又與司空荀顗、中書監荀勖、中軍將軍羊祜、中護軍王業,及廷尉杜友、守河南尹杜預、散騎侍郎裴楷、潁川太守周雄、齊相郭頎、騎都尉成公綏荀煇、尚書郎柳軌等,典正其事。朕每鑒其用心,常慨然嘉之。今法律既成,始班天下,刑寬禁簡,足以克當先旨。昔蕭何以定律受封,叔孫通以制儀為奉常,賜金五百斤,弟子皆為郎。夫立功立事,古之所重。自太傅、車騎以下,皆加祿賞。其詳依故典。」於是賜充子弟一人關內侯,絹五百匹。固讓,不許。



 後代裴秀為尚書令,常侍、車騎將軍如故。尋改常侍為侍中,賜絹七百匹。以母憂去職,詔遣黃門侍郎
 慰問。又以東南有事,遣典軍將軍楊囂宣諭,使六旬還內。



 充為政,務農節用,并官省職,帝善之,又以文武異容,求罷所領兵。及羊祜等出鎮,充復上表欲立勳邊境,帝並不許。從容任職,褒貶在已,頗好進士,每有所薦達,必終始經緯之,是以士多歸焉。帝舅王恂嘗毀充,而充更進恂。或有背充以要權貴者,充皆陽以素意待之。而充無公方之操,不能正身率下,專以諂媚取容。



 侍中任愷、中書令庾純等剛直守正,咸共疾之。又以充女為齊王妃,懼後益盛。及氐羌反叛,時帝深以為慮,愷因進說,請充鎮關中。乃下詔曰:「秦涼二境,比年屢敗,胡虜縱暴,百
 姓荼毒。遂使異類扇動,害及中州。雖復吳蜀之寇,未嘗至此。誠由所任不足以內撫夷夏,外鎮醜逆,輕用其眾而不能盡其力。非得腹心之重,推轂委成,大匡其弊,恐為患未已。每慮斯難,忘寢與食。侍中、守尚書令、車騎將軍賈充,雅量弘高,達見明遠,武有折衝之威,文懷經國之慮,信結人心,名震域外。使權統方任,綏靜西夏,則吾無西顧之念,而遠近獲安矣。其以充為使持節、都督秦涼二州諸軍事,侍中、車騎將軍如故,假羽葆、鼓吹,給第一駙馬。」朝之賢良欲進忠規獻替者,皆幸充此舉,望隆惟新之化。



 充既外出,自以為失職,深銜任愷,計無所從。
 將之鎮,百僚餞于夕陽亭,荀勖私焉。充以憂告,勖曰:「公,國之宰輔,而為一夫所制,不亦鄙乎!然是行也,辭之實難,獨有結婚太子,不頓駕而自留矣。」充曰:「然。孰可寄懷?」對曰:「勖請行之。」俄而侍宴,論太子婚姻事,勖因言充女才質令淑,宜配儲宮。而楊皇后及荀顗亦並稱之。帝納其言。會京師大雪,平地二尺,軍不得發。既而皇儲當婚,遂不西行。詔充居本職。先是羊祜密啟留充,及是,帝以語充。充謝祜曰:「始知君長者。」



 時吳將孫秀降,拜為驃騎大將軍。帝以充舊臣。欲改班,使車騎居驃騎之右。充固讓,見聽。尋遷司空,侍中、尚書令、領兵如故。



 會帝寢疾,充
 及齊王攸、荀勖參醫藥。及疾愈,賜絹各五百匹。初,帝疾篤,朝廷屬意於攸。河南尹夏侯和謂充曰:「卿二女婿,親疏等耳,立人當立德。」充不答。及是,帝聞之,徙和光祿勛,乃奪充兵權,而位遇無替。尋轉太尉、行太子太保、錄尚書事。咸寧三年,日蝕於三朝,充請遜位,不許。更以沛國之公丘益其封,寵倖愈甚,朝臣咸側目焉。



 河南尹王恂上言:「弘訓太后入廟,合食於景皇帝,齊王攸不得行其子禮。」充議以為:「禮,諸侯不得祖天子,公子不得禰先君,皆謂奉統承祀,非謂不得復其父祖也。攸身宜服三年喪事,自如臣制。」有司奏:「若如充議,服子服,行臣制,未有
 前比。宜如恂表,攸喪服從諸侯之例。」帝從充議。



 伐吳之役,詔充為使持節、假黃鉞、大都督,總統六師,給羽葆、鼓吹、緹幢、兵萬人、騎二千,置左右長史、司馬、從事中郎,增參軍、騎司馬各十人,帳下司馬二十人,大車、官騎各三十人。充慮大功不捷,表陳「西有昆夷之患,北有幽并之戍,天下勞擾,年穀不登,興軍致討,懼非其時。又臣老邁,非所克堪。」詔曰:「君不行,吾便自出。」充不得已,乃受節鉞,將中軍,為諸軍節度,以冠軍將軍楊濟為副,南屯襄陽。吳江陵諸守皆降,充乃徙屯項。



 王濬之剋武昌也,充遣使表曰:「吳未可悉定,方夏,江淮下濕,疾疫必起,宜召諸
 軍,以為後圖。雖腰斬張華,不足以謝天下。」華豫平吳之策,故充以為言。中書監荀勖奏,宜如充表。帝不從。杜預聞充有奏,馳表固爭,言平在旦夕。使及至轘轅,而孫皓已降。吳平,軍罷。帝遣侍中程咸犒勞,賜充帛八千匹,增邑八千戶;分封從孫暢新城亭侯,蓋安陽亭侯;弟陽里亭侯混、從孫關內侯眾增戶邑。充本無南伐之謀,固諫不見用。及師出而吳平,大慚懼,議欲請罪。帝聞充當詣闕,豫幸東堂以待之。罷節鉞、僚佐,仍假鼓吹、麾幢。充與群臣上告成之禮,請有司具其事。帝謙讓不許。



 及疾篤,上印綬遜位。帝遣侍臣諭旨問疾,殿中太醫致湯藥,賜
 床帳錢帛,自皇太子宗室躬省起居。太康三年四月薨,時年六十六。帝為之慟,使使持節、太常奉策追贈太宰,加袞冕之服、綠綟綬、御劍,賜東園祕器、朝服一具、衣一襲,大鴻臚護喪事,假節鉞、前後部羽葆、鼓吹、緹麾,大路、鑾路、轀輬車、帳下司馬大車,椎斧文衣武賁、輕車介士。葬禮依霍光及安平獻王故事,給塋田一頃。與石苞等為王功配饗廟庭,謚曰武。追贈充子黎民為魯殤公。



 充婦廣城君郭槐,性妒忌。初,黎民年三歲,乳母抱之當閣。黎民見充入,喜笑,充就而拊之。槐望見,謂充私乳母,即鞭殺之。黎民戀念,發病而死。後又生男,過期,復為乳母
 所抱,充以手摩其頭。郭疑乳母,又殺之,兒亦思慕而死。充遂無胤嗣。及薨,槐輒以外孫韓謐為黎民子,奉充後。郎中令韓咸、中尉曹軫諫槐曰:「禮,大宗無後,以小宗支子後之,無異姓為後之文。無令先公懷腆后土,良史書過,豈不痛心。」槐不從。咸等上書求改立嗣,事寢不報。槐遂表陳是充遺意。帝乃詔曰:「太宰、魯公充,崇德立勛,勤勞佐命,背世殂隕,每用悼心。又胤子早終,世嗣未立。古者列國無嗣,取始封支庶,以紹其統,而近代更除其國。至於周之公旦,漢之蕭何,或豫建元子,或封爵元妃,蓋尊顯勳庸,不同常例。太宰素取外孫韓謐為世子黎民
 後。吾退而斷之,外孫骨肉至近,推恩計情,合於人心。其以謐為魯公世孫,以嗣其國。自非功如太宰,始封無後如太宰,所取必以己自出不如太宰,皆不得以為比。」及下禮官議充謚,博士秦秀議謚曰荒,帝不納。博士段暢希旨,建議謚曰武,帝乃從之。自充薨至葬,賻賜二千萬。惠帝即位,賈后擅權,加充廟備六佾之樂,母郭為宜城君。及郭氏亡,謚曰宣,特加殊禮。時人譏之,而莫敢言者。



 初,充前妻李氏淑美有才行,生二女褒、裕,褒一名荃,裕一名浚。父豐誅,李氏坐流徙。後娶城陽太守郭配女,即廣城君也。武帝踐阼,李以大赦得還,帝特詔充置左右
 夫人,充母亦敕充迎李氏。郭槐怒,攘袂數充曰:「刊定律令,為佐命之功,我有其分。李那得與我並!」充乃答詔,託以謙沖,不敢當兩夫人盛禮,實畏槐也。而荃為齊王攸妃,欲令充遣郭而還其母。時沛國劉含母,及帝舅羽林監王虔前妻,皆毌丘儉孫女。此例既多,質之禮官,俱不能決。雖不遣後妻,多異居私通。充自以宰相為海內準則,乃為李築室於永年里而不往來。荃、浚每號泣請充,充竟不往。會充當鎮關右,公卿供帳祖道,荃、浚懼充遂去,乃排幔出於坐中,叩頭流血,向充及群僚陳母應還之意。眾以荃王妃,皆驚起而散。充甚愧愕,遣黃門將宮
 人扶去。既而郭槐女為皇太子妃,帝乃下詔斷如李比皆不得還,後荃恚憤而薨。初,槐欲省李氏,充曰:「彼有才氣,卿往不如不往。」及女為妃,槐乃盛威儀而去。既入戶,李氏出迎,槐不覺腳屈,因遂再拜。自是充每出行,槐輒使人尋之,恐其過李也。初,充母柳見古今重節義,竟不知充與成濟事,以濟不忠,數追罵之。侍者聞之,無不竊笑。及將亡,充問所欲言,柳曰:「我教汝迎李新婦尚不肯,安問他事!」遂無言。及充薨後,李氏二女乃欲令其母祔葬,賈后弗之許也。及后廢,李氏乃得合葬。李氏作《女訓》行於世。



 謐字長深。母賈午,充少女也。父韓壽,字德真,南
 陽堵陽人,魏司徒暨曾孫。美姿貌,善容止,賈充辟為司空掾。充每宴賓僚,其女輒於青金巢中窺之,見壽而悅焉。問其左右識此人不,有一婢說壽姓字,云是故主人。女大感想,發於寤寐。婢後往壽家,具說女意,并言其女光麗艷逸,端美絕倫。壽聞而心動,便令為通殷勤。婢以白女,女遂潛修音好,厚相贈結,呼壽夕入。壽勁捷過人,踰垣而至,家中莫知,惟充覺其女悅暢異於常日。時西域有貢奇香,一著人則經月不歇,帝甚貴之,惟以賜充及大司馬陳騫。其女密盜以遺壽,充僚屬與壽燕處,聞其芬馥,稱之於充。自是充意知女與壽通,而其門閣嚴峻,
 不知所由得入。乃夜中陽驚,託言有盜,因使循牆以觀其變。左右白曰:「無餘異,惟東北角如狐狸行處。」充乃考問女之左右,具以狀對。充秘之,遂以女妻壽。壽官至散騎常侍、河南尹。元康初卒,贈驃騎將軍。



 謐好學,有才思。既為充嗣,繼佐命之後,又賈后專恣,謐權過人主,至乃鎖系黃門侍郎,其為威福如此。負其驕寵,奢侈踰度,室宇崇僭,器服珍麗,歌僮舞女,選極一時。開閣延賓。海內輻湊,貴遊豪戚及浮競之徙,莫不盡禮事之。或著文章稱美謐,以方賈誼。渤海石崇歐陽建、滎陽潘岳、吳國陸機陸雲、蘭陵繆徵、京兆杜斌摯虞、瑯邪諸葛詮、弘農王
 粹、襄城杜育、南陽鄒捷、齊國左思、清河崔基、沛國劉瑰、汝南和郁周恢、安平牽秀、潁川陳眕、太原郭彰、高陽許猛、彭城劉訥、中山劉輿劉琨皆傅會於謐,號曰二十四友,其餘不得預焉。



 歷位散騎常侍、後軍將軍。廣城君薨,去職。喪未終。起為祕書監,掌國史。先是,朝廷議立晉書限斷,中書監荀勖謂宜以魏正始起年,著作郎王瓚欲引嘉平已下朝臣盡入晉史,于時依違未有所決。惠帝立,更使議之。謐上議,請從泰始為斷。於是事下三府,司徒王戎、司空張華、領軍將軍王衍、侍中樂廣、黃門侍郎嵇紹、國子博士謝衡皆從謐議。騎都尉濟北侯荀畯、侍
 中荀籓、黃門侍郎華混以為宜用正始開元。博士荀熙、刁協謂宜嘉平起年。謐重執奏戎、華之議,事遂施行。



 尋轉侍中。領秘書監如故。謐時從帝幸宣武觀校獵,諷尚書於會中召謐受拜,誡左右勿使人知,於是眾疑其有異志矣。謐既親貴,數入二宮,共愍懷太子遊處,無屈降心。常與太子弈棋爭道,成都王穎在坐,正色曰:「皇太子國之儲君,賈謐何得無禮!」謐懼,言之於后,遂出穎為平北將軍,鎮鄴。



 及為常侍,侍講東宮,太子意有不悅,謐患之。而其家數有妖異,飄風吹其朝服飛上數百丈,墜于中丞臺,又蛇出其被中,夜暴雷震其室,柱陷入地,壓毀
 床帳,謐益恐。及遷侍中,專掌禁內,遂與后成謀,誣陷太子。及趙王倫廢后,以詔召謐於殿前,將戮之。走入西鐘下,呼曰:「阿后救我!」乃就斬之。韓壽少弟蔚有器望,及壽兄鞏令保、弟散騎侍郎預、吳王友鑒、謐母賈午皆伏誅。



 初,充伐吳時,嘗屯項城,軍中忽失充所在。充帳下都督周勤時晝寢,夢見百餘人錄充,引入一逕。勤驚覺,聞失充,乃出尋索,忽睹所夢之道。遂往求之。果見充行至一府舍,侍衛甚盛。府公南面坐,聲色甚厲,謂充曰:「將亂吾家事,必爾與荀勖,既惑吾子,又亂吾孫。間使任愷黜汝而不去,又使庾純詈汝而不改。今吳寇當平,汝方表斬
 張華。汝之闇戇,皆此類也。若不悛慎,當旦夕加罪。」充因叩頭流血,公曰:「汝所以延日月而名器如此者,是衛府之勳耳。終當使係嗣死於鐘虡之間,大子斃於金酒之中,小子困於枯木之下。荀勖亦宜同,然其先德小濃。故在汝後,數世之外,國嗣亦替。」言畢,命去。充忽然得還營,顏色憔悴,性理昏喪,經日乃復。及是,謐死於鐘下,賈后服金酒而死,賈午考竟用大杖。終皆如所言。



 趙王倫之敗,朝廷追述充勛,議立其後。欲以充從孫散騎侍郎眾為嗣,眾陽狂自免。以子禿後充,封魯公,又病死。永興中,立充從曾孫湛為魯公,奉充後,遭亂死,國除。泰始中,人
 為充等謠曰:「賈、裴、王,亂紀綱。王、裴、賈,濟天下。」言亡魏而成晉也。



 充弟混,字宮奇,篤厚自守,無殊才能。太康中,為宗正卿。歷鎮軍將軍,領城門校尉,加侍中,封永平侯。卒,贈中軍大將軍、儀同三司。



 充從子彞、遵並有鑒裁,俱為黃門郎。遵弟模最知名。



 模字思範,少有志尚。頗覽載籍,而沈深有智算,確然難奪。深為充所信愛,每事籌之焉。充年衰疾劇,恒憂己謚傳,模曰:「是非久自見,不可掩也。」起家為邵陵令,遂歷事二宮尚書吏部郎,以公事免,起為車騎司馬。豫誅楊駿,
 封平陽鄉侯,邑千戶。及楚王瑋矯詔害汝南王亮、太保衛瓘,詔使模將中騶二百人救之。



 是時賈后既豫朝政,欲委信親黨,拜模散騎常侍,二日擢為侍中。模乃盡心匡弼,推張華、裴顗同心輔政。數年之中,朝野寧靜,模之力也。乃加授光祿大夫。然模潛執權勢,外形欲遠之,每有啟奏賈后事,入輒取急,或託疾以避之。至於素有嫌忿,多所中陷,朝廷甚憚之。加貪冒聚斂,富擬王公。但賈后性甚強暴,模每盡言為陳禍福,后不能從,反謂模毀己。於是委任之情日衰,而讒間之徒遂進。模不得志,憂憤成疾。卒,追贈車騎將軍、開府儀同三司,謚曰成。子遊
 字彥將嗣,歷官太子侍講、員外散騎侍郎。



 郭彰,字叔武,太原人,賈后從舅也。與賈充素相親遇,充妻待彰若同生。歷散騎常侍、尚書、衛將軍,封冠軍縣侯。及賈后專朝,彰豫參權勢,物情歸附,賓客盈門。世人稱為「賈郭」,謂謐及彰也。卒,謚曰烈。



 楊駿,字文長,弘農華陰人也。少以王官為高陸令,驍騎、鎮軍二府司馬。後以后父超居重位,自鎮軍將軍遷車騎將軍,封臨晉侯。識者議之曰:「夫封建諸侯,所以籓屏
 王室也。后妃,所以供粢盛,弘內教也。后父始封而以臨晉為侯,兆於亂矣。」尚書褚、郭奕並表駿小器,不可以任社稷之重。武帝不從。帝自太康以後,天下無事,不復留心萬機,惟耽酒色,始寵后黨,請謁公行。而駿及珧、濟勢傾天下,時人有「三楊」之號。



 及帝疾篤,未有顧命,佐命功臣,皆已沒矣,朝臣惶惑,計無所從。而駿盡斥群公,親侍左右。因輒改易公卿,樹其心腹。會帝小間,見所用者非,乃正色謂駿曰:「何得便爾!」乃詔中書,以汝南王亮與駿夾輔王室。駿恐失權寵,從中書借詔觀之,得便藏匿。中書監華廙恐懼,自往索之,終不肯與。信宿之間,上疾
 遂篤,后乃奏帝以駿輔政,帝頷之。便召中書監華暠、令何劭,口宣帝旨使作遺詔,曰:「昔伊望作佐,勛垂不朽;周霍拜命,名冠往代。侍中、車騎將軍、行太子太保,領前將軍楊駿,經德履吉,鑒識明遠,毗翼二宮,忠肅茂著,宜正位上台,擬跡阿衡。其以駿為太尉、太子太傅、假節、都督中外諸軍事,侍中、錄尚書、領前將軍如故。置參軍六人、步兵三千人、騎千人,移止前衛將軍珧故府。若止宿殿中宜有翼衛,其差左右衛三部司馬各二十人、殿中都尉司馬十人給駿,令得持兵仗出入。」詔成,后對暠、劭以呈帝,帝親視而無言。自是二日而崩,駿遂當寄託之重,
 居太極殿。梓宮將殯,六宮出辭,而駿不下殿,以武賁百人自衛。不恭之迹,自此而始。



 惠帝即位,進駿為太傅、大都督、假黃鉞,錄朝政,百官總己。慮左右間己,乃以其甥段廣、張劭為近侍之職。凡有詔命,帝省訖,入呈太后,然後乃出。駿知賈后情性難制,甚畏憚之。又多樹親黨,皆領禁兵。於是公室怨望,天下憤然矣。駿弟珧、濟並有俊才,數相諫止,駿不能用,因廢於家。駿闇於古義,動違舊典。武帝崩未踰年而改元,議者咸以為違《春秋》踰年書即位之義。朝廷惜於前失,令史官沒之,故明年正月復改年焉。



 駿自知素無美望,懼不能輯和遠近,乃依魏明
 帝即位故事,遂大開封賞,欲以悅眾,為政嚴碎,愎諫自用,不允眾心。馮翊太守孫楚素與駿厚,說之曰:「公以外戚,居伊霍之重,握大權,輔弱主。當仰思古人至公至誠謙順之道。於周則周召為宰,在漢則朱虛、東牟,未有庶姓專朝,而克終慶祚者也。今宗室親重,籓王方壯,而公不與共參萬機,內懷猜忌,外樹私暱,禍至無日矣。」駿不能從。弘訓少府蒯欽,駿之姑子。少而相暱,直亮不回,屢以正言犯駿,珧、濟為之寒心。欽曰:「楊文長雖闇,猶知人之無罪不可妄殺,必當疏我。我得疏外,可以不與俱死。不然,傾宗覆族,其能久乎!」



 殿中中郎孟觀、李肇,素不為
 駿所禮,陰構駿將圖社稷。賈后欲預政事,而憚駿未得逞其所欲,又不肯以婦道事皇太后。黃門董猛,始自帝之為太子即為寺人監,在東宮給事於賈后。後密通消息於猛,謀廢太后。猛乃與肇、觀潛相結托。賈后又令肇報大司馬、汝南王亮,使連兵討駿。亮曰:「駿之凶暴,死亡無日,不足憂也。」肇報楚王瑋,瑋然之。於是求入朝,駿素憚瑋,先欲召入,防其為變,因遂聽之。及瑋至,觀、肇乃啟帝,夜作詔,中外戒嚴,遣使奉詔廢駿,以侯就第。東安公繇率殿中四百人隨其後以討駿。段廣跪而言於帝曰:「楊駿受恩先帝,竭心輔政。且孤公無子,豈有反理?願陛下
 審之。」帝不答。



 時駿居曹爽故府,在武庫南,聞內有變,召眾官議之。太傅主簿朱振說駿曰:「今內有變,其趣可知,必是閹豎為賈后設謀,不利於公。宜燒雲龍門以示威,索造事都首,開萬春門,引東宮及外營兵,公自擁翼皇太子,入宮取姦人。殿內震懼,必斬送之,可以免難。」駿素怯懦,不決,乃曰:「魏明帝造此大功,奈何燒之!」侍中傅祗夜白駿,請與武茂俱入雲龍門觀察事勢。祗因謂群僚「宮中不宜空」,便起揖,於是皆走。



 尋而殿中兵出,燒駿府,又令弩士於閣上臨駿府而射之,駿兵皆不得出。駿逃于馬廄,以戟殺之。觀等受賈后密旨,誅駿親黨,皆夷三
 族,死者數千人。又令李肇焚駿家私書,賈后不欲令武帝顧命手詔聞于四海也。駿既誅,莫敢收者,惟太傅舍人巴西閻纂殯斂之。



 初,駿征高士孫登,遺以布被。登截被於門,大呼曰:「斫斫刺刺!」旬日託疾詐死,及是,其言果驗。永熙中,溫縣有人如狂,造書曰:「光光文長,大戟為牆。毒藥雖行,戟還自傷。」及駿居內府,以戟為衛焉。



 永寧初,詔曰:「舅氏失道,宗族隕墜,渭陽之思,孔懷感傷。其以蓩亭侯楊超為奉朝請、騎都尉,以慰《蓼莪》之思焉。」



 珧字文琚,歷位尚書令、衛將軍。素有名稱,得幸於武帝,時望在駿前。以兄貴盛,知權寵不可居,自乞遜位,前後
 懇至,終不獲許。初,聘后,珧表曰:「歷觀古今,一族二后,未嘗以全,而受覆宗之禍。乞以表事藏之宗廟,若如臣之言,得以免禍。」從之。右軍督趙休上書陳:「王莽五公,兄弟相代。今楊氏三公,並在大位,而天變屢見,臣竊為陛下憂之。」由此珧益懼。固求遜位,聽之,賜錢百萬、絹五千匹。



 珧初以退讓稱,晚乃合朋黨,構出齊王攸。中護軍羊琇與北軍中侯成粲謀欲因見珧而手刃之。珧知而辭疾不出。諷有司奏琇,轉為太僕。自是舉朝莫敢枝梧,而素論盡矣。珧臨刑稱冤,云:「事在石函,可問張華。」當時皆謂宜為申理,合依鐘毓事例。而賈氏族黨待諸楊如仇,促
 行刑者遂斬之。時人莫不嗟歎焉。



 濟字文通,歷位鎮南、征北將軍,遷太子太傅。濟有才藝,嘗從武帝校獵北芒下,與侍中王濟俱著布褲褶,騎馬執角弓在輦前。猛獸突出,帝命王濟射之,應弦而倒。須臾復一出,濟受詔又射殺之,六軍大叫稱快。帝重兵官,多授貴戚清望,濟以武藝號為稱職。與兄珧深慮盛滿,乃與諸甥李斌等共切諫。駿斥出王佑為河東太守,建立皇儲,皆濟謀也。



 初,駿忌大司馬汝南王亮,催使之籓。濟與斌數諫止之,駿遂疏濟。濟謂傅咸曰:「若家兄征大司馬入,退身避之,門戶可得免耳。不爾,行當赤族。」咸曰:「
 但征還,共崇至公,便立太平,無為避也。夫人臣不可有專,豈獨外戚!今宗室疏,因外戚之親以得安,外戚危,倚宗室之重以為援,所謂脣齒相依,計之善者。」濟益懼而問石崇曰:「人心云何?」崇曰:「賢兄執政,疏外宗室,宜與四海共之。」濟曰:「見兄,可及此。」崇見駿,及焉,駿不納。後與諸兄俱見害。難發之夕,東宮召濟。濟謂裴楷曰:「吾將何之?」楷曰:「子為保傅,當至東宮。」濟好施,久典兵馬,所從四百餘人皆秦中壯士,射則命中,皆欲救濟。濟已入宮,莫不歎恨。



 史臣曰:賈充以諂諛陋質,刀筆常材,幸屬昌辰,濫叨非
 據。抽戈犯順,曾無猜憚之心;杖鉞推亡,遽有知難之請,非惟魏朝之悖逆,抑亦晉室之罪人者歟!然猶身極寵光,任兼文武,存荷台衡之寄,沒有從享之榮,可謂無德而祿,殃將及矣。逮乎貽厥,乃乞丐之徒,嗣惡稔之餘基,縱姦邪之凶德。煽茲哲婦,索彼惟家,雖及誅夷,曷云塞責。昔當塗闕翦,公閭實肆其勞,典午分崩,南風亦盡其力,可謂「君以此始,必以此終」,信乎其然矣。楊駿階緣寵幸,遂荷棟梁之任,敬之猶恐弗逮,驕奢淫泆,庸可免乎?括母以明智全身,會昆以先言獲宥,文琚識同曩烈,而罰異昔人,裴夫!



 贊曰:公閭便佞,心乖雅正。邀遇時來,遂階榮命。乞丐承緒,兇家亂政。瑣瑣文長,遂居棟梁。據非其位,乃底滅亡。珧雖先覺,亦罹禍殃。



\end{pinyinscope}