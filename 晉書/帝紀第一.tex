\article{帝紀第一}

\begin{pinyinscope}

 宣帝



 宣皇帝諱懿,字仲達,河內溫縣孝敬里人,姓司馬氏。其先出自帝高陽之子重黎,為夏官祝融,歷唐、虞、夏、商,世序其職。及周,以夏官為司馬。其後程柏休父,周宣王時,以世官克平徐方,錫以官族,因而為氏。楚漢間,司馬仰為趙將,與諸侯伐秦。秦亡,立為殷王,都河內。漢以其地為郡,子孫遂家焉。自仰八世,生征西將軍鈞,字叔平。鈞
 生豫章太守量,字公度。量生潁川太守俊,字元異。俊生京兆尹防,字建公。帝即防之第二子也。少有奇節,聰明多大略,博學洽聞,伏膺儒教。漢末大亂,常慨然有憂天下心。南陽太守同郡楊俊名知人,見帝,未弱冠,以為非常之器。尚書清河崔琰與帝兄朗善,亦謂朗曰:「君弟聰亮明允,剛斷英特,非子所及也。」



 漢建安六年,郡舉上計掾。魏武帝為司空,聞而辟之。帝知漢運方微,不欲屈節曹氏,辭以風痺,不能起居。魏武使人夜往密刺之,帝堅臥不動。及魏武為丞相,又辟為文學掾,敕行者曰:「若復盤桓,便收之。」帝懼而就職。於是,使與太子游處,遷黃門
 侍郎,轉議郎、丞相東曹屬,尋轉主簿。從討張魯,言於魏武曰:「劉備以詐力虜劉璋,蜀人未附而遠爭江陵,此機不可失也。今若曜威漢中,益州震動,進兵臨之,勢必瓦解。因此之勢,易為功力。聖人不能違時,亦不失時矣。」魏武曰:「人苦無足,既得隴右,復欲得蜀!」言竟不從。既而從討孫權,破之。軍還,權遣使乞降,上表稱臣,陳說天命。魏武帝曰:「此兒欲踞吾著爐炭上邪!」答曰:「漢運垂終,殿下十分天下而有其九,以服事之。權之稱臣,天人之意也。虞、夏、殷、周不以謙讓者,畏天知命也。」



 魏國既建,遷太子中庶子。每與大謀,輒有奇策,為太子
 所信重,與陳群、吳質、朱樂號曰四友。遷為軍司馬,言於魏武曰:「昔箕子陳謀,以食為首。今天下不耕者蓋二十餘萬,非經國遠籌也。雖戎甲未卷,自宜且耕且守。」魏武納之,於是務農積穀,國用豐贍。帝又言荊州刺史胡脩粗暴,南鄉太守傅方驕奢,並不可居邊。魏武不之察。及蜀將羽圍曹仁於樊,于禁等七軍皆沒,脩、方果降羽,而仁圍甚急焉。是時漢帝都許昌,魏武以為近賊,欲徙河北。帝諫曰:「禁等為水所沒,非戰守之所失,於國家大計未有所損,而便遷都,既示敵以弱,又淮沔之人大不安矣。孫權、劉備,外親內疏,羽之得意,權所不願也。可喻
 權所,令掎其後,則樊圍自解。」魏武從之。權果遣將呂蒙西襲公安,拔之,羽遂為蒙所獲。



 魏武以荊州遺黎及屯田在潁川者逼近南寇,皆欲徙之。帝曰:「荊楚輕脫,易動難安。關羽新破,諸為惡者藏竄觀望。今徙其善者,既傷其意,將令去者不敢復還。」從之。其後諸亡者悉復業。及魏武薨於洛陽,朝野危懼。帝綱紀喪事,內外肅然。乃奉梓宮還鄴。



 魏文帝即位,封河津亭侯,轉丞相長史。會孫權帥兵西過,朝議以樊、襄陽無穀,不可以禦寇。時曹仁鎮襄陽,請召仁還宛。帝曰:「孫權新破關羽,此其欲自結之時也,必
 不敢為患。襄陽水陸之衝,禦寇要害,不可棄也。」言竟不從。仁遂焚棄二城,權果不為寇,魏文悔之。及魏受漢禪,以帝為尚書。頃之,轉督軍、御史中丞,封安國鄉侯。



 黃初二年,督軍官罷,遷侍中、尚書右僕射。



 五年,天子南巡,觀兵吳疆。帝留鎮許昌,改封向鄉侯,轉撫軍、假節,領兵五千,加給事中、錄尚書事。帝固辭。天子曰:「吾於庶事,以夜繼晝,無須臾寧息。此非以為榮,乃分憂耳。」



 六年,天子復大興舟師征吳,復命帝居守,內鎮百姓,外供軍資。臨行,詔曰:「吾深以後事為念,故以委卿。曹參雖
 有戰功,而蕭何為重。使吾無西顧之憂,不亦可乎!」天子自廣陵還洛陽,詔帝曰:「吾東,撫軍當總西事;吾西,撫軍當總東事。」於是帝留鎮許昌。及天子疾篤,帝與曹真、陳群等見於崇華殿之南堂,並受顧命輔政。詔太子曰:「有間此三公者,慎勿疑之。」明帝即位,改封舞陽侯。及孫權圍江夏,遣其將諸葛瑾、張霸並攻襄陽,帝督諸軍討權,走之。進擊,敗瑾,斬霸,並首級千餘。遷驃騎將軍。



 太和元年六月,天子詔帝屯於宛,加督荊、豫二州諸軍事。初,蜀將孟達之降也,魏朝遇之甚厚。帝以達言行傾巧,不可任,驟諫,不見聽,乃以達領新城太守,封侯,假節。
 達於是連吳固蜀,潛圖中國。蜀相諸葛亮惡其反覆,又慮其為患。達與魏興太守申儀有隙,亮欲促其事,乃遣郭模詐降,過儀,因漏泄其謀。達聞其謀漏泄,將舉兵。帝恐達速發,以書喻之曰:「將軍昔棄劉備,託身國家,國家委將軍以疆埸之任,任將軍以圖蜀之事,可謂心貫白日。蜀人愚智,莫不切齒於將軍。諸葛亮欲相破,惟苦無路耳。模之所言,非小事也,亮豈輕之而令宣露,此殆易知耳。」達得書大喜,猶與不決。帝乃潛軍進討。諸將言達與二賊交構,宜觀望而後動。帝曰:「達無信義,此其相疑之時也,當及其未定促決之。」乃倍道兼行,八日到其城下。吳
 蜀各遣其將向西城安橋、木闌塞以救達,帝分諸將距之。初,達與亮書曰:「宛去洛八百里,去吾一千二百里,聞吾舉事,當表上天子,比相反覆,一月間也,則吾城已固,諸軍足辦。則吾所在深險,司馬公必不自來;諸將來,吾無患矣。」及兵到,達又告亮曰:「吾舉事,八日而兵至城下,何其神速也!」上庸城三面阻水,達於城外為木柵以自固。帝渡水,破其柵,直造城下。八道攻之,旬有六日,達甥鄧賢、將李輔等開門出降。斬達,傳首京師。俘獲萬餘人,振旅還於宛。乃勸農桑,禁浮費,南土悅附焉。初,申儀久在魏興,專威疆埸,輒承制刻印,多所假授。達既誅,有
 自疑心。時諸郡守以帝新克捷,奉禮求賀,皆聽之。帝使人諷儀,儀至,問承制狀,執之,歸于京師。又徙孟達餘眾七千餘家于幽州。蜀將姚靜、鄭他等帥其屬七千餘人來降。時邊郡新附,多無戶名,魏朝欲加隱實。屬帝朝于京師,天子訪之於帝。帝對曰:「賊以密網束下,故下棄之。宜弘以大綱,則自然安樂。」又問二虜宜討,何者為先?對曰:「吳以中國不習水戰,故敢散居東關。凡攻敵,必扼其喉而摏其心。夏口、東關,賊之心喉。若為陸軍以向皖城,引權東下,為水戰軍向夏口,乘其虛而擊之,此神兵從天而墜,破之必矣。」天子並然之,復命屯于宛。



 四年,遷大將軍,加大都督、假黃鉞,與曹真伐蜀。帝自西城斫山開道,水陸並進,溯沔而上,至于朐,拔其新豐縣。軍次丹口,遇雨,班師。明年,諸葛亮寇天水,圍將軍賈嗣、魏平於祁山。天子曰:「西方有事,非君莫可付者。」乃使帝西屯長安,都督雍、梁二州諸軍事,統車騎將軍張郃、後將軍費曜、征蜀護軍戴凌、雍州刺史郭淮等討亮。張郃勸帝分軍往雍、郿為後鎮,帝曰:「料前軍獨能當之者,將軍言是也。若不能當,而分為前後,此楚之三軍所以為黥布禽也。」遂進軍隃麋。亮聞大軍且至,乃自帥眾將芟上邽之麥。諸將皆懼,帝曰:「亮慮多決少,必安營自固,
 然後芟麥。吾得二日兼行足矣。」於是卷甲晨夜赴之。亮望塵而遁。帝曰:「吾倍道疲勞,此曉兵者之所貪也。亮不敢據渭水,此易與耳。」進次漢陽,與亮相遇,帝列陣以待之。使將牛金輕騎餌之,兵才接而亮退,追至祁山。亮屯鹵城,據南北二山,斷水為重圍。帝攻拔其圍,亮宵遁。追擊,破之,俘斬萬計。天子使使者勞軍,增封邑。時軍師杜襲、督軍薛悌皆言,明年麥熟,亮必為寇,隴右無穀,宜及冬豫運。帝曰:「亮再出祁山,一攻陳倉,挫衄而反。縱其後出,不復攻城,當求野戰,必在隴東,不在西也。亮每以糧少為恨,歸必積穀,以吾料之,非三稔不能動矣。」於是表
 徙冀州農夫佃上邽,興京兆、天水、南安監冶。



 青龍元年,穿成國渠,築臨晉陂,溉田數千頃,國以充實。



 二年,亮又率眾十餘萬出斜谷,壘于郿之渭水南原。天子憂之,遣征蜀護軍秦朗督步騎二萬,受帝節度。諸將欲住渭北以待之,帝曰:「百姓積聚皆在渭南,此必爭之地也。」遂引軍而濟,背水為壘。因謂諸將曰:「亮若勇者,當出武功依山而東,若西上五丈原,則諸軍無事矣。」亮果上原,將北渡渭,帝遣將軍周當屯陽遂以餌之。數日,亮不動。帝曰:「亮欲爭原而不向陽遂,此意可知也。」遣將軍胡遵、雍州刺史郭淮共備陽遂,與亮會于積石,臨原而
 戰,亮不得進,還于五丈原。會有長星墜亮之壘,帝知其必敗,遣奇兵掎亮之後,斬五百餘級,獲生口千餘,降者六百餘人。時朝廷以亮僑軍遠寇,利在急戰,每命帝持重,以候其變。亮數挑戰,帝不出,因遺帝巾幗婦人之飾。帝怒,表請決戰,天子不許,乃遣骨鯁臣衛尉辛毗杖節為軍師以制之。後亮復來挑戰,帝將出兵以應之,毗杖節立軍門,帝乃止。初,蜀將姜維聞毗來,謂亮曰:「辛毗杖節而至,賊不復出矣。」亮曰:「彼本無戰心,所以固請者,以示武於其眾耳。將在軍,君命有所不受,茍能制吾,豈千里而請戰邪!」帝弟孚書問軍事,帝復書曰:「亮志大而不
 見機,多謀而少決,好兵而無權,雖提卒十萬,已墮吾畫中,破之必矣。」與之對壘百餘日,會亮病卒,諸將燒營遁走,百姓奔告,帝出兵追之。亮長史楊儀反旗鳴鼓,若將距帝者。帝以窮寇不之逼,於是楊儀結陣而去。經日,乃行其營壘,觀其遺事,獲其圖書、糧穀甚眾。帝審其必死,曰:「天下奇才也。」辛毗以為尚未可知。帝曰:「軍家所重,軍書密計、兵馬糧穀,今皆棄之,豈有人捐其五藏而可以生乎?宜急追之。」關中多蒺藜,帝使軍士二千人著軟材平底木屐前行,蒺藜悉著屐,然後馬步俱進。追到赤岸,乃知亮死。審問,時百姓為之諺曰:「死諸葛走生仲達。」帝
 聞而笑曰:「吾便料生,不便料死故也。」先是,亮使至,帝問曰:「諸葛公起居何如,食可幾米?」對曰:「三四升。」次問政事,曰:「二十罰已上皆自省覽。」帝既而告人曰:「諸葛孔明其能久乎!」竟如其言。亮部將楊儀、魏延爭權,儀斬延,并其眾。帝欲乘隙而進,有詔不許。



 三年,遷太尉,累增封邑。蜀將馬岱入寇,帝遣將軍牛金擊走之,斬千餘級。武都氐王苻雙、強端帥其屬六千餘人來降。關東飢,帝運長安粟五百萬斛于京師。



 四年,獲白鹿,獻之。天子曰:「昔周公旦輔成王,有素雉之貢。今君受陜西之任,有白鹿之獻,豈非忠誠協符,千載
 同契,俾乂邦家,以永厥休邪!」及遼東太守公孫文懿反,徵帝詣京師。天子曰:「此不足以勞君,事欲必克,故以相煩耳。君度其行何計?」對曰:「棄城預走,上計也。據遼水以距大軍,次計也。坐守襄平,此成擒耳。」天子曰:「其計將安出?」對曰:「惟明者能深度彼己,豫有所棄,此非其所及也。今懸軍遠征,將謂不能持久,必先距遼水而後守,此中下計也。」天子曰:「往還幾時?」對曰:「往百日,還百日,攻百日,以六十日為休息,一年足矣。」是時大脩宮室,加之以軍旅,百姓飢弊。帝將即戎,乃諫曰:「昔周公營洛邑,蕭何造未央,今宮室未備,臣之責也。然自河以北,百姓困窮,外
 內有役,勢不並興,宜假絕內務,以救時急。」



 景初二年,帥牛金、胡遵等步騎四萬發自京都。車駕送出西明門。詔弟孚、子師送過溫,賜以穀帛牛酒,敕郡守典農以下皆往會焉。見父老故舊,宴飲累日。帝歎息,悵然有感,為歌曰:「天地開闢,日月重光。遭遇際會,畢力遐方。將掃群穢,還過故鄉。肅清萬里,總齊八荒。告成歸老,待罪舞陽。」遂進師,經孤竹,越碣石,次于遼水。文懿果遣步騎數萬,阻遼隧,堅壁而守,南北六七十里,以距帝。帝盛兵多張旗幟,出其南,賊盡銳赴之。乃泛舟潛濟以出其北,與賊營相逼,沈舟焚梁,傍遼水作長圍,棄賊而向
 襄平。諸將言曰:「不攻賊而作圍,非所以示眾也。」帝曰:「賊堅營高壘,欲以老吾兵也。攻之,正入其計,此王邑所以恥過昆陽也。古人曰,敵雖高壘,不得不與我戰者,攻其所必救也。賊大眾在此,則巢窟虛矣。我直指襄平,則人懷內懼,懼而求戰,破之必矣。」遂整陣而過。賊見兵出其後,果邀之。帝謂諸將曰:「所以不攻其營,正欲致此,不可失也。」乃縱兵逆擊,大破之,三戰皆捷。賊保襄平,進軍圍之。初,文懿聞魏師之出也,請救於孫權。權亦出兵遙為之聲援,遺文懿書曰:「司馬公善用兵,變化若神,所向無前,深為弟憂之。」會霖潦,大水,平地數尺,三軍恐,欲移營。
 帝令軍中敢有言徙者斬。都督令史張靜犯令,斬之,軍中乃定。賊恃水,樵牧自若。諸將欲取之,皆不聽。司馬陳珪曰:「昔攻上庸,八部並進,晝夜不息,故能一旬之半,拔堅城,斬孟達。今者遠來而更安緩,愚竊惑焉。」帝曰:「孟達眾少而食支一年,吾將士四倍於達而糧不淹月,以一月圖一年,安可不速?以四擊一,正令半解,猶當為之。是以不計死傷,與糧競也。今賊眾我寡,賊飢我飽,水雨乃爾,功力不設,雖當促之,亦何所為。自發京師,不憂賊攻,但恐賊走。今賊糧垂盡,而圍落未合,掠其牛馬,抄其樵采,此故驅之走也。夫兵者詭道,善因事變。賊憑眾恃雨,故
 雖飢困,未肯束手,當示無能以安之。取小利以驚之,非計也。」朝廷聞師遇雨,咸請召還。天子曰:「司馬公臨危制變,計日擒之矣。」既而雨止,遂合圍。起土山地道,楯櫓鉤橦,發矢石雨下,晝夜攻之。時有長星,色白,有芒鬣,自襄平城西南流于東北,墜于梁水,城中震懾。文懿大懼,乃使其所署相國王建、御史大夫柳甫乞降,請解圍而縛。不許,執建等,皆斬之。檄告文懿曰:「昔楚鄭列國,而鄭伯猶肉袒牽羊而迎之。孤為王人,位則上公,而建等欲孤解圍退舍,豈楚鄭之謂邪!二人老耄,必傳言失旨,已相為斬之。若意有未已,可更遣年少有明決者來。」文懿復
 遣侍中衛演乞剋日送任。帝謂演曰:「軍事大耍有五,能戰當戰,不能戰當守,不能守當走,餘二事惟有降與死耳。汝不肯面縛,此為決就死也,不須送任。」文懿攻南圍突出,帝縱兵擊敗之,斬于梁水之上星墜之所。既入城,立兩標以別新舊焉。男子年十五已上七千餘人皆殺之,以為京觀。偽公卿已下皆伏誅,戮其將軍畢盛等二千餘人。收戶四萬,口三十餘萬。初,文懿篡其叔父恭位而囚之。及將反,將軍綸直、賈範等苦諫,文懿皆殺之。帝乃釋恭之囚,封直等之墓,顯其遺嗣。令曰:「古之伐國,誅其鯨鯢而已,諸為文懿所詿誤者,皆原之。中國人欲還
 舊鄉,恣聽之。」時有兵士寒凍,乞襦,帝弗之與。或曰:「幸多故襦,可以賜之。」帝曰:「襦者官物,人臣無私施也。」乃奏軍人年六十已上者罷遣千餘人,將吏從軍死亡者致喪還家。遂班師。天子遣使者勞軍于薊,增封食昆陽,並前二縣。初,帝至襄平,夢天子枕其膝,曰:「視吾面。」俯視有異於常,心惡之。先是,詔帝便道鎮關中;及次白屋,有詔召帝,三日之間,詔書五至。手詔曰:「間側息望到,到便直排閣入,視吾面。」帝大遽,乃乘追鋒車晝夜兼行,自白屋四百餘里,一宿而至。引入嘉福殿臥內,升御床。帝流涕問疾,天子執帝手,目齊王曰:「以後事相託。死乃復可忍,吾
 忍死待君,得相見,無所復恨矣。」與大將軍曹爽並受遺詔輔少主。及齊王即帝位,遷侍中、持節、都督中外諸軍、錄尚書事,與爽各統兵三千人,共執朝政,更直殿中,乘輿入殿。爽欲使尚書奏事先由己,乃言於天子,徙帝為大司馬。朝議以為前後大司馬累薨於位,乃以帝為太傅。入殿不趨,贊拜不名,劍履上殿,如漢蕭何故事。嫁娶喪葬取給於官,以世子師為散騎常侍,子弟三人為列侯,四人為騎都尉。帝固讓子弟官不受。



 魏正始元年春正月,東倭重譯納貢,焉耆、危須諸國,弱水以南,鮮卑名王,皆遣使來獻。天子歸美宰輔,又增帝
 封邑。初,魏明帝好脩宮室,制度靡麗,百姓苦之。帝自遼東還,役者猶萬餘人,雕玩之物動以千計。至是皆奏罷之,節用務農,天下欣賴焉。



 二年夏五月,吳將全琮寇芍陂,朱然、孫倫圍樊城,諸葛瑾、步騭掠柤中,帝請自討之。議者咸言,賊遠來圍樊,不可卒拔。挫於堅城之下,有自破之勢,宜長策以御之。帝曰:「邊城受敵而安坐廟堂,疆場騷動,眾心疑惑,是社稷之大憂也。」六月,乃督諸軍南征,車駕送出津陽門。帝以南方暑濕,不宜持久,使輕騎挑之,然不敢動。於是休戰士,簡精銳,募先登,申號令,示必攻之勢。吳軍夜遁走,追
 至三州口,斬獲萬餘人,收其舟船軍資而還。天子遣侍中常侍勞軍于宛。秋七月,增封食郾、臨潁,并前四縣,邑萬戶,子弟十一人皆為列侯。帝勳德日盛,而謙恭愈甚。以太常常林鄉邑舊齒,見之每拜。恒戒子弟曰:「盛滿者道家之所忌,四時猶有推移,吾何德以堪之。損之又損之,庶可以免乎?」



 三年春,天子追封,謚皇考京兆尹為舞陽成侯。三月,奏穿廣漕渠,引河入汴,溉東南諸陂,始大佃於淮北。先是,吳遣將諸葛恪屯皖,邊鄙苦之,帝欲自擊恪。議者多以賊據堅城,積穀,欲引致官兵,今懸軍遠攻,其救必至,進
 退不易,未見其便。帝曰:「賊之所長者水也,今攻其城,以觀其變。若用其所長,棄城奔走,此為廟勝也。若敢固守,湖水冬淺,船不得行,勢必棄水相救,由其所短,亦吾利也。」



 四年秋九月,帝督諸軍擊諸葛恪,車駕送出津陽門。軍次于舒,恪焚燒積聚,棄城而遁。帝以滅賊之耍,在於積穀,乃大興屯守,廣開淮陽、百尺二渠,又修諸陂於潁之南北,萬餘頃。自是淮北倉庾相望,壽陽至于京師,農官屯兵連屬焉。



 五年春正月,帝至自淮南,天子使持節勞軍。尚書鄧揚、
 李勝等欲令曹爽建立功名,勸使伐蜀。帝止之,不可,爽果無功而還。



 六年秋八月,曹爽毀中壘中堅營,以兵屬其弟中領軍羲,帝以先帝舊制禁之不可。冬十二月,天子詔帝朝會乘輿升殿。



 七年春正月,吳寇柤中,夷夏萬餘家避寇北渡沔。帝以沔南近賊,若百姓奔還,必復致寇,宜權留之。曹爽曰:「今不能修守沔南而留百姓,非長策也。」帝曰:「不然。凡物致之安地則安。危地則危。故兵書曰『成敗,形也;安危,勢也』。形勢,御眾之耍,不可以不審。設令賊以二萬人斷沔水,
 三萬人與沔南諸軍相持,萬人陸梁柤中,將何以救之?」爽不從,卒令還南。賊果襲破柤中,所失萬計。



 八年夏四月,夫人張氏薨。曹爽用何晏、鄧揚、丁謐之謀,遷太后於永寧宮,專擅朝政,兄弟並典禁兵,多樹親黨,屢改制度。帝不能禁,於是與爽有隙。五月,帝稱疾不與政事。時人為之謠曰:「何、鄧、丁,亂京城。」



 九年春三月,黃門張當私出掖庭才人石英等十一人,與曹爽為伎人。爽、晏謂帝疾篤,遂有無君之心,與當密謀,圖危社稷,期有日矣。帝亦潛為之備,爽之徒屬亦頗疑帝。會河南尹李勝將蒞荊州,來候帝。帝詐疾篤,使兩
 婢侍,持衣衣落,指口言渴,婢進粥,帝不持杯飲,粥皆流出霑胸。勝曰:「眾情謂明公舊風發動,何意尊體乃爾!」帝使聲氣纔屬,說「年老枕疾,死在旦夕。君當屈並州,並州近胡,善為之備。恐不復相見,以子師、昭兄弟為託。」勝曰:「當還忝本州,非並州。」帝乃錯亂其辭曰:「君方到并州。」勝復曰:「當忝荊州。」帝曰:「年老意荒,不解君言。今還為本州,盛德壯烈,好建功勳!」勝退告爽曰:「司馬公尸居餘氣,形神已離,不足慮矣。」他日,又言曰:「太傅不可復濟,令人愴然。」故爽等不復設備。



 嘉平元年春正月甲午,天子謁高平陵,爽兄弟皆從。是
 日,太白襲月。帝於是奏永寧太后,廢爽兄弟。時景帝為中護軍,將兵屯司馬門。帝列陣闕下,經爽門。爽帳下督嚴世上樓,引弩將射帝,孫謙止之曰:「事未可知。」三注三止,皆引其肘不得發。大司農桓範出赴爽,蔣濟言於帝曰:「智囊往矣。」帝曰:「爽與範內疏而智不及,駑馬戀棧豆,必不能用也。」於是假司徒高柔節,行大將軍事,領爽營,謂柔曰:「君為周勃矣。」命太僕王觀行中領軍,攝羲營。帝親帥太尉蔣濟等勒兵出迎天子,屯於洛水浮橋,上奏曰:「先帝詔陛下、秦王及臣升于御床,握臣臂曰『深以後事為念』。今大將軍爽背棄顧命,敗亂國典,內則僭擬,外
 專威權。群官耍職,皆置所親;宿衛舊人,並見斥黜。根據槃牙,縱恣日甚。又以黃門張當為都監,專共交關,伺候神器。天下洶洶,人懷危懼。陛下便為寄坐,豈得久安?此非先帝詔陛下及臣升御床之本意也。臣雖朽邁,敢忘前言。昔趙高極意,秦是以亡;呂霍早斷,漢祚永延。此乃陛下之殷鑒,臣授命之秋也。公卿群臣皆以爽有無君之心,兄弟不宜典兵宿衛;奏皇太后,皇太后敕如奏施行。臣輒敕主者及黃門令罷爽、羲,訓吏兵各以本官侯就第,若稽留車駕,以軍法從事。臣輒力疾將兵詣洛水浮橋,伺察非常。」爽不通奏,留車駕宿伊水南,伐樹為鹿
 角,發屯兵數千人以守。桓範果勸爽奉天子幸許昌,移檄徵天下兵。爽不能用,而夜遣侍中許允、尚書陳泰詣帝,觀望風旨。帝數其過失,事止免官。泰還以報爽勸之通奏。帝又遣爽所信殿中校尉尹大目諭爽,指洛水為誓,爽意信之。桓範等援引古今,諫說萬端,終不能從。乃曰:「司馬公正當欲奪吾權耳。吾得以侯還第,不失為富家翁。」範拊膺曰:「坐卿。滅吾族矣!」遂通帝奏。既而有司劾黃門張當,並發爽與何晏等反事,乃收爽兄弟及其黨與何晏、丁謐、鄧揚、畢軌、李勝、桓範等誅之。蔣濟曰:「曹真之勳,不可以不祀。」帝不聽。初,爽司馬魯芝、主簿楊綜斬
 關奔爽。及爽之將歸罪也,芝、綜泣諫曰:「公居伊周之任,挾天子,杖天威,孰敢不從?舍此而欲就東市,豈不痛哉!」有司奏收芝、綜科罪,帝赦之,曰:「以勸事君者。」二月,天子以帝為丞相,增封潁川之繁昌、鄢陵、新汲、父城,並前八縣,邑二萬戶,奏事不名。固讓丞相。冬十二月,加九錫之禮,朝會不拜。固讓九錫。



 二年春正月,天子命帝立廟于洛陽,置左右長史,增掾屬、舍人滿十人,歲舉掾屬任御史、秀才各一人,增官騎百人,鼓吹十四人,封子肜平樂亭侯,倫安樂亭侯。帝以久疾不任朝請,每有大事,天子親幸第以諮訪焉。兗州
 刺史令狐愚、太尉王凌貳於帝,謀立楚王彪。



 三年春正月,王凌詐言吳人塞塗水,請發兵以討之。帝潛知其計,不聽。夏四月,帝自帥中軍,汎舟沿流,九日而到甘城。凌計無所出,乃迎於武丘,面縛水次,曰:「凌若有罪,公當折簡召凌,何苦自來邪!」帝曰:「以君非折簡之客故耳。」即以凌歸于京師。道經賈逵廟,凌呼曰:「賈梁道!王凌是大魏之忠臣,惟爾有神知之。」至項,仰鴆而死。收其餘黨,皆夷三族,並殺彪。悉錄魏諸王公置于鄴,命有司監察,不得交關。天子遣侍中韋誕持節勞軍于五池。帝至自甘城,天子又使兼大鴻臚、太僕庾嶷持節,策命帝
 為相國,封安平郡公,孫及兄子各一人為列侯,前後食邑五萬戶,侯者十九人。固讓相國、郡公不受。六月,帝寢疾,夢賈逵、王凌為祟,甚惡之。秋八月戊寅,崩於京師,時年七十三。天子素服臨弔,喪葬威儀依漢霍光故事,追贈相國、郡公。弟孚表陳先志,辭郡公及韞輬車。九月庚申,葬于河陰,謚曰文貞,後改謚文宣。先是,預作終制,於首陽山為土藏,不墳不樹;作顧命三篇,斂以時服,不設明器,後終者不得合葬。一如遺命。晉國初建,追尊曰宣王。武帝受禪,上尊號曰宣皇帝,陵曰高原,廟稱高祖。



 帝內忌而外寬,猜忌多權變。魏武察帝有雄豪志,聞有狼顧
 相。欲驗之。乃召使前行,令反顧,面正向後而身不動。又嘗夢三馬同食一槽,甚惡焉。因謂太子丕曰:「司馬懿非人臣也,必預汝家事。」太子素與帝善,每相全佑,故免。帝於是勤於吏職,夜以忘寢,至於芻牧之間,悉皆臨履,由是魏武意遂安。及平公孫文懿,大行殺戮。誅曹爽之際,支黨皆夷及三族,男女無少長,姑姊妹女子之適人者皆殺之,既而竟遷魏鼎云。明帝時,王導侍坐。帝問前世所以得天下,導乃陳帝創業之始,用文帝末高貴鄉公事。明帝以面覆床曰:「若如公言,晉祚復安得長遠!」迹其猜忍,蓋有符於狼顧也。



 制曰:夫天地之大,黎元為本。邦國之貴,元首為先。治亂無常,興亡有運。是故五帝之上,居萬乘以為憂;三王已來,處其憂而為樂。競智力,爭利害,大小相吞,強弱相襲。逮乎魏室,三方鼎峙,干戈不息,氛霧交飛。宣皇以天挺之姿,應期佐命,文以纘治,武以棱威。用人如在己,求賢若不及;情深阻而莫測,性寬綽而能容,和光同塵,與時舒卷,戢鱗潛翼,思屬風雲。飾忠於已詐之心,延安於將危之命。觀其雄略內斷,英猷外決,殄公孫於百日,擒孟達於盈旬,自以兵動若神,謀無再計矣。既而擁眾西舉,與諸葛相持。抑其甲兵,本無鬥志,遺其巾幗,方發憤心。
 杖節當門,雄圖頓屈,請戰千里,詐欲示威。且秦蜀之人,勇懦非敵,夷險之路,勞逸不同,以此爭功,其利可見。而返閉軍固壘,莫敢爭鋒,生怯實而未前,死疑虛而猶遁,良將之道,失在斯乎!文帝之世,輔翼權重,許昌同蕭何之委,崇華甚霍光之寄。當謂竭誠盡節,伊傅可齊。及明帝將終,棟梁是屬,受遺二主,佐命三朝,既承忍死之託,曾無殉生之報。天子在外,內起甲兵,陵土未乾,遽相誅戮,貞臣之體,寧若此乎!盡善之方,以斯為惑。夫征討之策,豈東智而西愚?輔佐之心,何前忠而後亂?故晉明掩面,恥欺偽以成功;石勒肆言,笑姦回以定業。古人有云:「
 積善三年,知之者少,為惡一日,聞於天下。」可不謂然乎!雖自隱過當年,而終見嗤後代。亦猶竊鐘掩耳,以眾人為不聞;銳意盜金,謂市中為莫睹。故知貪於近者則遺遠,溺於利者則傷名;若不損己以益人,則當禍人而福己。順理而舉易為力,背時而動難為功。況以未成之晉基,逼有餘之魏祚?雖復道格區宇,德被蒼生,而天未啟時,寶位猶阻,非可以智競,不可以力爭,雖則慶流後昆,而身終於北面矣。



\end{pinyinscope}