\article{列傳第六十五}

\begin{pinyinscope}

 藝術



 陳訓戴洋韓友淳於智步熊杜不愆嚴卿隗炤卜珝鮑靚吳猛幸靈佛圖澄麻襦單道開黃泓索紞孟欽王嘉僧涉郭黁鳩摩羅什曇霍臺產



 藝
 術之興,由來尚矣。先王以是決猶豫,定吉凶,審存亡,省禍福。曰神與智,藏往知來;幽贊冥符,弼成人事;既興利而除害,亦威眾以立權,所謂神道設教,率由於此。然而詭託近於妖妄,迂誕難可根源,法術紛以多端,變態諒非一緒,真雖存矣,偽亦憑焉。聖人不語怪力亂神,良有以也。逮丘明首唱,敘妖夢以垂文,子長繼作,援龜策
 以立傳,自茲厥後,史不絕書。漢武雅好神仙,世祖尤耽讖術,遂使文成、五利,逞詭詐而取寵榮,尹敏、桓譚,由忤時而嬰罪戾,斯固通人之所蔽,千慮之一失者乎!詳觀眾術,抑惟小道,棄之如或可惜,存之又恐不經。載籍既務在博聞,筆削則理宜詳備,晉謂之《乘》,義在於斯。今錄其推步尤精、伎能可紀者,以為《藝術傳》,式備前史云。



 陳訓,字道元,歷陽人。少好祕學,天文、算歷、陰陽、占候無不畢綜,尤善風角。孫晧以為奉禁都尉,使其占侯。晧政嚴酷,訓知其必敗而不敢言。時錢唐湖開,或言天下當
 太平,青蓋入洛陽。晧以問訓,訓曰:「臣止能望氣,不能達湖之開塞。」退而告其友曰:「青蓋入洛,將有輿櫬銜璧之事,非吉祥也。」尋而吳亡。訓隨例內徙,拜諫義大夫。俄而去職還鄉。



 及陳敏作亂,遣弟宏為歷陽太守,訓謂邑人曰:「陳家無王氣,不久當滅。」宏聞,將斬之。訓鄉人秦琚為宏參軍,乃說訓曰:「訓善風角,可試之。如不中,徐斬未晚也。」乃赦之。時宏攻征東參軍衡彥於歷陽,乃問訓曰:「城中有幾千人?攻之可拔不?」訓登牛渚山望氣,曰:「不過五百人。然不可攻,攻之必敗。」宏復大怒曰:「何有五千人攻五百人而有不得理?」命將士攻之,果為彥所敗,方信訓
 有道術,乃優遇之。



 都水參軍淮南周亢嘗問訓以官位,訓曰:「君至卯年當剖符近郡,酉年當有曲蓋。」亢曰:「脫如來言,當相薦拔。」訓曰:「性不好官,惟欲得米耳。」後亢果為義興太守、金紫將軍。時劉聰、王彌寇洛陽,歷陽太守武瑕問訓曰:「國家人事如何?」訓曰:「胡賊三逼,國家當敗,天子野死。今尚未也。」其後懷愍二帝果有平陽之酷焉。或問其以明年吉凶者,訓曰:「揚州刺史當死,武昌大火,上方節將亦當死。」至時,劉陶、周訪皆卒,武昌大火,燒數千家。時甘卓為歷陽太守,訓私謂所親曰:「甘侯頭低而視仰,相法名為眄刀,又目有赤脈,自外而入,不出十年,必
 以兵死,不領兵則可以免。」卓果為王敦所害。丞相王導多病,每自憂慮,以問訓。訓曰:「公耳豎垂肩,必壽,亦大貴,子孫當興於江東。」咸如其言。訓年八十餘卒。



 戴洋,字國流,吳興長城人也。年十二,遇病死,五日而蘇。說死時天使其為酒藏吏,授符錄,給吏從幡麾,將上蓬萊、崑崙、積石、太室、恒、廬、衡等諸山。既而遣歸,逢一老父,謂之曰:「汝後當得道,為貴人所識。」及長,遂善風角。



 為人短陋,無風望,然好道術,妙解占侯卜數。吳末為臺吏,知吳將亡,託病不仕。及吳平,還鄉里。後行至瀨鄉,經老子
 祠,皆是洋昔死時所見使處,但不復見昔物耳。因問守藏應鳳曰:「去二十餘年,嘗有人乘馬東行,過老君而不下馬,未達橋,墜馬死者不?」鳳言有之。所問之事,多與洋同。



 揚州刺史嘗問吉凶於洋,答曰:「熒惑入南斗,八月有暴水,九月當有客軍西南來。」如期果大水,而石冰作亂。冰既據揚州,洋謂人曰:「視賊雲氣,四月當破。」果如其言。時陳敏為右將軍,堂邑令孫混見而羨之。洋曰:「敏當作賊族滅,何足願也!」未幾,敏果反而誅焉。初,混欲迎其家累,洋曰:「此地當敗,得臘不得正,豈可移家於賊中乎!」混便止。歲末,敏弟昶攻堂邑,混遂以單身走免。其後都水
 馬武舉洋為都水令史,洋請急還鄉。將赴洛,夢神人謂之曰:「洛中當敗,人盡南渡,後五年揚州必有天子。」洋信之,遂不去。既而皆如其夢。



 廬江太守華譚問洋曰:「天下誰當復作賊者?」洋曰:「王機。」尋而機反。陳問洋曰:「人言江南當有貴人,顧彥先、周宣珮當是不?」洋曰:「顧不及臘,周不見來年八月。」榮果以十二月十七日卒,十九日臘,以明年七月晦亡。王導遇病,召洋問之。洋曰:「君侯本命在申,金為土使之主,而於申上石頭立冶,火光照天,此為金火相爍,水火相煎,以故受害耳。」導即移居東府,病遂差。



 鎮東從事中郎張闓舉洋為丞相令史。時司馬
 颺為烏程令,將赴職,洋曰:「君宜深慎下吏。」揚後果坐吏免官。洋又謂曰:「卿雖免官,十一月當作郡,加將軍。」至期,為太山太守、鎮武將軍。颺賣宅將行,洋止之曰:「君不得至,當還,不可無宅。」颺果為徐龕所逼,不得之郡。元帝增颺眾二千,使助祖逖。洋勸颺不行,颺乃稱疾。收付廷尉,俄而因赦得出。



 元帝將登阼,使洋擇日,洋以為宜用三月二十四日丙午。太史令陳卓奏用二十二日,言:「昔越王用甲辰三月反國,范蠡稱在陽之前,當主盡出,上下盡空,德將出游,刑入中宮,今與此同。」洋曰:「越王為吳所囚,雖當時遜媚,實懷怨憤,蠡故用甲辰,乘德而歸,留刑
 吳宮。今大王內無含咎,外無怨憤,當承天洪命,納祚無窮,何為追越王去國留殃故事邪!」乃從之。



 及祖約代兄鎮譙,請洋為中典軍,遷督護。永昌元年四月庚辰,禺中時有大風,起自東南,折木。洋謂約曰:「十月必有賊到譙城東,至歷陽,南方有反者。」主簿王振以洋為妖,白約收洋,付刺奸而絕其食五十日,言語如故。約知其有神術,乃赦之而讓振。振後有罪被收,洋救之。約曰:「振往日相繫,今何以救之?」洋曰:「振不識風角,非有宿嫌。振往時垂餓死,洋養活之,振猶尚遺忘。夫處富貴而不棄貧賤甚難。」約義之,即原振,賜洋米三十石。至十月三日,石勒騎
 果到譙城東。洋言於約曰:「賊必向城父,可遣騎水南追之,步軍於水北斷要路,賊必敗。」約竟不追,賊乃掠城父婦女輜重而去。約將魯延求追賊,洋曰:「不可。」約不從,使兄子智與延追之。賊偽棄婦女輜重走,智與延等爭物,賊還掩之,智、延僅以身免,士卒皆死。約表洋為下邑長。時梁國人反,逐太守袁晏。梁城峻險,約欲討之而未決,洋曰:「賊以八月辛酉日反,日辰俱王,辛德在南方,酉受自刑,梁在譙北,乘德伐刑,賊必破亡。又甲子日東風而雷西行,譙在東南,雷在軍前,為軍驅除。昔吳伐關羽,天雷在前,周瑜拜賀。今與往同,故知必剋。」約從之,果平梁
 城。



 太寧三年正月,有大流星東南行,洋曰:「至秋,府當移壽陽。」及王敦作逆,約問其勝敗,洋曰:「太白在東方,辰星不出。兵法先起為主,應者為客。辰星若出,太白為主,辰星為客。辰星不出,太白為客,先起兵者敗。今有客無主,有前無後,宜傳檄所部,應詔伐之。」約乃率眾向合肥。俄而敦死眾敗,遂住壽陽。洋又曰:「江淮之間當有軍事,譙城虛曠,宜還固守。不者,雍丘、沛皆非官有也。」約不從,豫土遂陷於賊。



 咸和元年春,約南行佃,遇大雷雨西南來,洋曰:「甲子西南天雷,其夏必失大將。」至夏,汝南人反,執約兄子濟,送于石勒。約府內地忽赤如丹,洋曰:「案《河圖
 征》云:『地赤如丹血丸丸,當有下反上者。』恐十月二十七日胡馬當來飲淮水。」至時,石勒騎大至,攻城大戰。其日西風,兵火俱發,約大懼。會風回,賊退。時傳言勒遣騎向壽陽,約欲送其家還江東,洋曰:「必無此事。」尋而傳言果妄。



 咸和初,月暈左角,有赤白珥。約問洋,洋曰:「角為天門,開布陽道,官門當有大戰。」俄而蘇峻遣使招約俱反,洋謂約曰:「蘇峻必敗,然其初起,兵鋒不可當,可外和內嚴,以待其變。」約不從,遂與峻反。至三年五月,大風雷雨西北來,城內晦螟,洋謂約曰:「雷鳴人上,明使君當遠佞近直,愛下振貧。昔秦有此變,卒致亂亡。」約大怒,收洋系之。
 遣部將李概將兵到盧江,其眾盡散。約召洋出,問之曰:「吾還東何如留壽陽?若留壽陽,何如入胡?」洋曰:「東入失半,入胡滅門,留壽陽尚可。」約欲東向歷陽,其眾不樂東下,皆叛約,劫約姊及嫂奔于石勒。約到歷陽,祖煥問洋曰:「君昔言平西在壽陽可得五年,果如君言。今在歷陽,可得幾時?」洋曰:「得六月耳。」約問洋:「臺下及此氣侯何如?」洋曰:「此當復有反者。臺下來年三月當太平,江州當大喪。後南方復有軍事,去此千里。」尋而牽騰叛約,約率所親將家屬奔于石勒。二月而天子反正,四月而溫嶠卒,郭默據湓口以叛。後勒誅約及親屬並盡,皆如洋言。



 約
 既敗,洋往尋陽。時劉胤鎮尋陽,胤問洋曰:「我病當差不?」洋曰:「不憂使君不差,憂使君今年有大厄。使君年四十七,行年入庚寅。《太公陰謀》曰:『六庚為白獸,在上為客星,在下為害氣。』年與命并,必凶當忌。十二月二十二日庚寅勿見客。」胤曰:「我當解職,將君還野中治病。」洋曰:「使君當作江州,不得解職。」胤曰:「溫公不復還邪?」洋曰:「溫公雖還,使君故作江州。」俄如其言。九月甲寅申時,迴風從東來,入胤兒船中,西過,狀如匹練,長五六丈。洋曰:「風從咸池下來,攝提下去,咸池為刀兵,大殺為死喪。到甲子日申時,府內大聚骨理之。胤問在何處,洋曰:「不出州府門
 也。」胤架府東門。洋又曰:「東為天牢,牢下開門,憂天獄至。」十二月十七日,洋又曰:「臘近可閉門,以五十人備守,并以百人備東北寅上,以卻害氣。」胤不從。二十四日壬辰,胤遂為郭默所害。



 南中郎將桓宣以洋為參軍,將隨宣往襄陽,太尉陶侃留之住武昌。時侃謀北伐,洋曰:「前年十一月熒惑守胃昴,至今年四月,積五百餘日。昴,趙之分野,石勒遂死。熒惑以七月退,從畢右順行入黃道,未及天關,以八月二十二日復逆行還鉤,繞畢向昴。昴畢為邊兵,主胡夷,故置天弓以射之。熒惑逆行,司無德之國,石勒死是也。勒之餘燼,以自殘害。今年官與太歲、太
 陰三合癸巳,癸為北方,北方當受災。歲鎮二星共合翼軫,從子及巳,徘徊六年。荊楚之分,歲鎮所守,其下國昌,豈非功德之徵也!今年六月,鎮星前角亢。角亢,鄭之分。歲星移入房,太白在心。心房,宋分。順之者昌,逆之者亡。石季龍若興兵東南,此其死會也。官若應天伐刑,徑據宋鄭,則無敵矣。若天與不取,反受其咎。」侃志在中原,聞而大喜。會病篤,不果行。



 侃薨,征西將軍庾亮代鎮武昌,復引洋問氣侯。洋曰:「天有白氣,喪必東行,不過數年必應。」尋有大鹿向西城門,洋曰:「野獸向城,主人將去。」城東家夜半望見城內有數炬火,從城上出,如大車狀,白布
 幔覆,與火俱出城東北行,至江乃滅。洋聞而歎曰:「此與前白氣同。」時亮欲西鎮石城,或問洋:「此西足當欲東不?」洋曰:「不當也。」咸康三年,洋言於亮曰:「武昌土地有山無林,政可圖始,不可居終。山作八字,數不及九。昔吳用壬寅來上;創立宮城,至己酉,還下秣陵。陶公亦涉八年。土地盛衰有數,人心去就有期,不可移也。公宜更擇吉處,武昌不可久住。」五年,亮令毛寶屯邾城。九月,洋言於亮曰:「毛豫州今年受死問。昨朝大霧晏風,當有怨賊報仇,攻圍諸侯,誠宜遠偵邏。」寶問當在何時,答曰:「五十日內。」其夕,又曰:「九月建戌,朱雀飛驚,征軍還歸,乘戴火光,天
 示有信,災發東房,葉落歸本,慮有後患。」明日,又曰:「昨夜火殃,非國福,今年架屋,致使君病,可因燒屋,移家南渡,無嫌也。」寶即遣兒婦還武昌。尋傳賊當來攻城,洋曰:「十月丁亥夜半時得賊問,乾為君,支為臣,丁為征西府,亥為邾城,功曹為賊神,加子時十月水王木相,王相氣合,賊必來。寅數七,子數九,賊高可九千人,下可七千人。從魁為貴人加丁,下剋上,有空亡之事,不敢進武昌也。」賊果陷邾城而去。亮問洋曰:「故當不失石城否?」洋曰:「賊從安陸向石城,逆太白,當伐身,無所慮。」亮曰:「天何以利胡而病我?」洋曰:「天符有吉凶,土地有盛衰,今年害氣三合
 己亥,己為天下,亥為戎胡,季龍亦當受死。今乃不憂賊,但憂公病耳。」亮曰:「何方救我疾?」洋曰:「荊州受兵,江州受災,公可去此二州。」亮曰:「如此,當有解不」?」洋曰:「恨晚,猶差不也。」亮竟不能解二州,遂至大困。洋曰:「昔蘇峻時,公於白石祠中祈福,許賽其牛,至今未解,故為此鬼所考。」亮曰:「有之,君是神人也。」或問洋曰:「庾公可得幾時?」洋曰:「見明年。」時亮已不識人,咸以為妄,果至正月一日而薨。



 庾翼代亮,洋復為占侯。少時卒,年八十餘。所占驗者不可勝紀。



 韓友,字景先,廬江舒人也。為書生,受《易》於會稽伍振,善占卜,能圖宅相冢,亦行京費厭勝之術。龍舒長鄧林婦病積年,垂死,醫巫皆息意。友為筮之,使畫作野豬著臥處屏風上,一宿覺佳,於是遂差。舒縣廷掾王睦病死,已復魄。友為筮之,令以丹畫版作日月置床頭,又以豹皮馬鄣泥臥上,立愈。劉世則女病魅積年,巫為攻禱,伐空冢故城間,得狸鼉數十,病猶不差。友筮之,命作布囊,依女發時,張囊著窗牖間,友閉戶作氣,若有所驅。斯須之間,見囊大脹如吹,因決敗之,女仍大發。友乃更作皮囊二枚,沓張之,施張如前,囊復脹滿。因急縛囊口,懸著樹
 二十許日,漸消,開視有二斤狐毛,女遂差。



 宣城邊洪以四月中就友卜家中安否,友曰:「卿家有兵殃,其禍甚重。可伐七十束柴,積於庚地,至七月丁酉放火燒之,咎可消也。不爾,其凶難言。」洪即聚柴。至日,大風,不敢發火。洪後為廣陽領校,遭母喪歸家,友來投之,時日已暮,出告從者,速裝束,吾當夜去。從者曰:「今日已暝,數十里草行,何急復去?」友曰:「非汝所知也。此間血覆地,寧可復住!」苦留之,不待食而去。其夜洪欻發狂,絞殺兩子,並殺婦,又斫父妾二人,皆被創,因出亡走。明日,其宗族往收殯亡者,尋索洪,數日,於宅前林中得之,已自經死。



 宣城太守
 殷祐有病,友筮之,曰:「七月晦日,將有大鸜鳥來集事上,宜勤伺取,若獲者為善,不獲將成禍。」祐乃謹為其備。至日,果有大鸜垂尾九尺,來集事上,掩捕得之,祐乃遷石頭督護,後為吳郡太守。



 友卜占神效甚多,而消殃轉禍,無不皆驗。于寶問其故,友曰:「筮封用五行相生殺,如案方投藥治病,以冷熱相救。其差與不差,不可必也。」友以元康六年舉賢良,元帝渡江,以為廣武將軍,永嘉末卒。



 淳于智字叔平,濟北盧人也。有思義,能《易》筮,善厭勝之
 術。高平劉柔夜臥,鼠齧其左手中指,以問智。智曰:「是欲殺君而不能,當為君使其反死。」乃以硃書手腕橫文後三寸作田字,辟方一寸二分,使露手以臥。明旦,有大鼠伏死手前。譙人夏侯藻母病困,詣智卜,忽有一狐當門向之嗥。藻怖愕,馳見智。智曰:「其禍甚急,君速歸,在狐嗥處拊心啼哭,令家人驚怪,大小必出,一人不出,哭勿止,然後其禍可救也。」藻還,如其言,母亦扶病而出。家人既集,堂屋五間拉然而崩。護軍張劭母病篤,智筮之,使西出市沐猴,繫母臂,令傍人捶拍,恒使作聲,三日放去。劭從之。其猴出門即為犬所咋死,母病遂差。上黨鮑瑗家
 多喪病貧苦,或謂之曰:「淳于叔平神人也,君何不試就卜,知禍所在?」瑗性質直,不信卜筮,曰:「人生有命,豈卜筮所移!」會智來,應詹謂曰:「此君寒士,每多屯虞,君有通靈之思,可為一卦。」智乃為卦,卦成,謂瑗曰:「君安宅失宜,故令君困。君舍東北有大桑樹,君徑至市,入門數十步,當有一人持荊馬鞭者,便就買以懸此樹,三年當暴得財。」瑗承言詣市,果得馬鞭,懸之三年,浚井,得錢數十萬,銅鐵器復二十餘萬,於是致贍,疾者亦愈。其消災轉禍,不可勝紀,而卜筮所占,千百皆中。應詹少亦多病,智乃為符使詹佩之,誦其文,既而皆驗,莫能學也。



 性深沈,常自
 言短命,曰:「辛亥歲天下有事,當有巫醫挾道術者死。吾守《易》義以行之,猶當不應此乎!」太康末,為司馬督,有寵於楊駿,故見殺。



 步熊,字叔羆,陽平發干人也。少好卜筮數術,門徒甚盛。熊學舍側有一人燒死,吏持熊諸生,謂為失火。熊曰:「已為卿卜得其人矣。使從道南行,當有一人來問得火主未者,便縛之。」吏如熊言,果是耕人,自言草惡難耕,故燒之,忽風起延燒遠近,實不知草中有人。又鄰人兒遠行,或告已死,其父母號哭制服,熊為之卜,剋日當還,如期
 果至。趙王倫聞其名,召之。熊謂諸生曰:「倫死不久,不足應也。」倫怒,遣兵圍之數重。熊乃使諸生著其裘南走,倫兵悉赴捉之,熊密從北出,得脫。後為成都王穎所辟,穎使熊射覆,物無所失。後穎奔關中,平昌公模鎮鄴,以熊穎黨,誅之。



 杜不愆,廬江人也。少就外祖郭璞學《易》卜。屢有驗。高平郗超年二十餘,得重疾,試令筮之。不愆曰:「案卦言之,卿所苦尋除。然宜於東北三十里上宮姓家索其所養雄雉,籠盛置東簷下,卻後九日丙午日午時,必當有雌雉
 飛來與交,既而雙去。若如此,不出二十日病都除,又是休應,年將八十,位極人臣。若但雌逝雄留者,病一周方差,年半八十,名位亦失。」超時正羸篤,慮命在旦夕,笑而答曰:「若保八十之半,便有餘矣。一周病差,何足為淹!」然未之信。或勸依其言,索雉果得。至丙午日,超臥南軒之下觀之,至日晏,果有雌雉飛入籠,與雄雉交而去,雄雉不動。超歎曰:「雖管郭之奇,何以尚此!」超病彌年乃起,至四十,卒於中書郎。不愆後占筮轉疏,無復此類。後為桓嗣建威參軍。



 嚴卿,會稽人也。善卜筮。鄉人魏序欲暫東行,荒年多抄盜,令卿筮之。卿筮曰:「君慎不可東行,必遭暴害之氣,而非劫也。」序不之信。卿曰:「既必不停,宜以禳之,可索西郭外獨母家白雄狗繫著船前。」求索止得駮狗,無白者。卿曰:「駮者亦足,然猶恨其色不純,當餘小毒,正及六畜輩耳,無所復憂。」序行半路,狗忽然作聲甚急,有如人打之者。比視,已死,吐黑血斗餘。其夕,序墅上白鵝數頭無故自死,而序家無恙。



 隗炤,汝陰人也。善於《易》。臨終,書版授其妻曰:「吾亡後當
 大荒窮,雖爾慎莫賣宅也。卻後五年春,當有詔使來頓此亭,姓龔,此人負吾金,即以此版往責之,勿違言也。」炤亡後,其家大困乏,欲賣宅,憶夫言輒止。期日,有龔使者止亭中,妻遂齎版往責之。使者執版惘然,不知所以。妻曰:「夫臨亡,手書版見命如此,不敢妄也。」使者沈吟良久而悟,謂曰:「賢夫何善?」妻曰:「夫善於《易》,而未會為人卜也。」使者曰:「噫,可知矣!」乃命取蓍筮之,卦成,撫掌而歎曰:「妙哉隗生!含明隱跡,可謂鏡窮達而洞吉凶者也。」於是告炤妻曰:「吾不相負金也,賢夫自有金耳,知亡後當暫窮,故藏金以待太平,所以不告兒婦者,恐金盡而困無已也。
 知吾善《易》,故書版以寄意耳。金有五百斤,盛以青甕,覆以銅柈,埋在堂屋東頭,去壁一丈,入地九尺。」妻還掘之,皆如卜焉。



 卜珝,字子玉,匈奴後部人也。少好讀《易》,郭璞見而歎曰:「吾所弗如也,柰何不免兵厄!」珝曰:「然。吾大厄在四十一,位為卿將,當受禍耳。不爾者,亦為猛獸所害。吾亦未見子之令終也。」璞曰:「吾禍在江南,甚營之,未見免兆。雖然,在南猶可延期,住此不過時月。」珝曰:「子勿為公吏,可以免諸。」璞曰:「吾不能免公吏,猶子之不能免卿將也。」珝曰:「
 吾此雖當有帝王子,終不復奉二京矣。瑯邪可奉,卿謹奉之,主晉記者必雌也。」珝遂隱于龍門山。劉元海僭號,徵為大司農、侍中,固以疾辭。元海曰:「人各有心,卜珝不欲在吾朝,何異高祖四公哉!可遂其高志。」後復徵為光祿大夫,珝謂使者曰:「非吾死所也。」及劉聰嗣偽位,徵為太常。時劉琨據並州,聰問何時可平,珝答曰:「并州陛下之分,今茲剋之必矣。」聰戲曰:「朕欲勞先生一行可乎?」珝曰:「臣所以來不及裝者,正為是行也。」聰大悅,署珝使持節、平北將軍。將行,謂其妹曰:「此行也,死自吾分,後慎勿紛紜。」及攻晉陽,為琨所敗,珝卒先奔,為其元帥所
 殺。



 鮑靚,字太玄,東海人也。年五歲,語父母云:「本是曲陽李家兒,九歲墜井死。」其父母尋訪得李氏,推問皆符驗。靚學兼內外,明天文河洛書,稍遷南陽中部都尉,為南海太守。嘗行部入海,遇風,飢甚,取白石煮食之以自濟。王機時為廣州刺史,入廁,忽見二人著烏衣,與機相捍,良久擒之,得二物似烏鴨。靚曰:「此物不祥。」機焚之,徑飛上天,機尋誅死。靚嘗見仙人陰君,授道訣,百餘歲卒。



 吳猛,豫章人也。少有孝行,夏日常手不驅蚊,懼其去己而噬親也。年四十,邑人丁義始授其神方。因還豫章,江波甚急,猛不假舟楫,以白羽扇畫水而渡,觀者異之。庾亮為江州刺史,嘗遇疾,聞猛神異,乃迎之,問己疾何如。猛辭以算盡,請具棺服。旬日而死,形狀如生。未及大斂,遂失其尸。識者以為亮不祥之徵。亮疾果不起。



 幸靈者,豫章建昌人也。性少言,與小人群居,見侵辱而無慍色,邑里號之癡,雖其父母兄弟亦以為癡也。嘗使守稻,群牛食之,靈見而不驅,待牛去乃往理其殘亂者。
 其父母見而怒之,靈曰:「夫萬物生天地之間,各欲得食。牛方食,柰何驅之!」其父愈怒曰:「即如汝言,復用理壞者何為?」靈曰:「此稻又欲得終其性,牛自犯之,靈可以不收乎!」



 時順陽樊長賓為建昌令,發百姓作官船於建城山中,吏令人各作箸一雙。靈作而未輸,或竊之焉。俄而竊者心痛欲死,靈謂之曰:「爾得無竊我箸乎?」竊者不應。有頃,愈急,靈曰:「若爾不以情告我者,今真死矣。」竊者急遽,乃首出之。靈於是飲之以水,病即立愈。行人由此敬畏之。船成,當下,吏以二百人引一艘,不能動,方請益人。靈曰:「此以過足,但部分未至耳。靈請自牽之。」乃手執箸,惟
 用百人,而船去如流。眾大驚怪,咸稱其神,於是知名。



 有龔仲儒女病積年,氣息財屬,靈使以水含之,已而強起,應時大愈。又呂猗母皇氏得痿痺病,十有餘年,靈療之,去皇氏數尺而坐,冥目寂然,有頃,顧謂猗曰:「扶夫人令起。」猗曰:「老人得病累年,奈何可倉卒起邪?」靈曰:「但試扶起。」於是兩人夾扶以立。少選,靈又令去扶,即能自行,由此遂愈。於是百姓奔趣,水陸輻輳,從之如雲。皇氏自以病久,懼有發動,靈乃留水一器令食之,每取水,輒以新水補處,二十餘年水清如新,塵垢不能加焉。



 時高悝家有鬼怪,言語訶叱,投擲內外,不見人形,或器物自行,再
 三發火,巫祝厭劾而不能絕。適值靈,乃要之。靈於陌頭望其屋,謂悝曰:「此君之家邪?」悝曰:「是也。」靈曰:「知之足矣。」悝固請之,靈不得已,至門,見符索甚多,謂悝曰:「當以正止邪,而以邪救邪,惡得已乎!」並使焚之,惟據軒小坐而去,其夕鬼怪即絕。



 靈所救愈多此類,然不取報謝。行不騎乘,長不娶妻,性至恭,見人即先拜,言輒自名。凡草木之夭傷於山林者,必起理之,器物之傾覆於途路者,必舉正之。周旋江州間,謂其士人曰:「天地之於人物,一也,咸欲不失其情性,奈何制服人以為奴婢乎!諸君若欲享多福以保性命,可悉免遣之。」十餘年間,賴其術以濟
 者極多。後乃娶妻,畜車以奴婢,受貨賂致遺,於是其術稍衰,所療得失相半焉。



 佛圖澄,天竺人也。本姓帛氏。少學道,妙通玄術。永嘉四年,來適洛陽,自云百有餘歲,常服氣自養,能積日不食。善誦神咒,能役使鬼神。腹旁有一孔,常以絮塞之,每夜讀書,則拔絮,孔中出光,照于一室。又嘗齋時,平旦至流水側,從腹旁孔中引出五藏六府洗之,訖,還內腹中。又能聽鈴音以言吉凶,莫不懸驗。



 及洛中寇亂,乃潛草野以觀變。石勒屯兵葛陂,專行殺戮,沙門遇害者其眾。澄
 投勒大將軍郭黑略家,黑略每從勒征伐,輒豫剋勝負,勒疑而問曰:「孤不覺卿有出眾智謀,而每知軍行吉凶何也?」黑略曰:「將軍天挺神武,幽靈所助,有一沙門智術非常,云將軍當略有區夏,己應為師。臣前後所白,皆其言也。」勒召澄,試以道術。澄即取缽盛水,燒香咒之,須臾缽中生青蓮花,光色曜日,勒由此信之。



 勒自葛陂還河北,過枋頭,枋頭人夜欲斫營,澄謂黑略曰:「須臾賊至,可令公知。」果如其言,有備,故不敗。勒欲試澄,夜冠胄衣甲,執刀而坐,遣人告澄云:「夜來不知大將軍何所在。」使人始至,未及有言,澄逆問曰:「平居無寇,何故夜嚴?」勒益信
 之。勒後因忿,欲害諸道士,并欲苦澄。澄乃潛避至黑略舍,語弟子曰:「若將軍信至,問吾所在者,報云不知所之。」既而勒使至,覓澄不得。使還報勒,勒驚曰:「吾有惡意向澄,澄捨我去矣。」通夜不寢,思欲見澄。澄知勒意悔,明旦造勒。勒曰:「昨夜何行?」澄曰:「公有怒心,昨故權避公。今改意,是以敢來。勒大笑曰:「道人謬矣。」



 襄國城塹水源在城西北五里,其水源暴竭,勒問澄何以致水。澄曰:「今當敕龍取水。」乃與弟子法首等數人至故泉源上,坐繩床,燒安息香,咒願數百言。如此三日,水泫然微流,有一小龍長五六寸許,隨水而來,諸道士競往觀之。有頃,水大至,
 隍塹皆滿。



 鮮卑段末波攻勒,眾甚盛。勒懼,問澄。澄曰:「昨日寺鈴鳴云,明旦食時,當擒段末波。」勒登城望末波軍,不見前後,失色曰:「末波如此,豈可獲乎!」更遣夔安問澄。澄曰:「已獲末波矣。」時城北伏兵出,遇末波,執之。澄勸勒宥末波,遣還本國,勒從之,卒獲其用。



 劉曜遣從弟岳攻勒,勒遣石季龍距之。岳敗,退保石梁塢,季龍堅柵守之。澄在襄國,忽歎曰:「劉岳可憫!」弟子法祚問其故,澄曰「昨日亥時,岳已敗被執。」果如所言。



 及曜自攻洛陽,勒將救之,其群下咸諫以為不可。勒以訪澄,澄曰:「相輪鈴音云:『秀支替戾岡,僕谷劬禿當。」此羯語也,秀支,軍也。替戾岡,
 出也。僕谷,劉曜胡位也。劬禿當,捉也。此言軍出捉得曜也。」又令一童子潔齋七日,取麻油合胭脂,躬自研於掌中,舉手示童子,粲然有輝。童子驚曰:「有軍馬甚眾,見一人長大白晳,以朱絲縛其肘。」澄曰:「此即曜也。」勒其悅,遂赴洛距曜,生擒之。



 勒僭稱趙天王,行皇帝事,敬澄彌篤。時石葱將叛,澄誡勒曰:「今年葱中有蟲,食必害人,可令百姓無食蔥也。」勒班告境內,慎無食葱。俄而石葱果走。勒益重之,事必諮而後行,號曰大和尚。



 勒愛子斌暴病死,將殯,勒歎曰:「朕聞虢太子死,扁鵲能生之,今可得效乎?」乃令告澄。澄取楊枝沾水,灑而咒之。就執斌手曰:「可
 起矣!」因此遂蘇,有頃,平復。自是勒諸子多在澄寺中養之。勒死之年,天靜無風,而塔上一鈴獨鳴,澄謂眾曰:「鈴音云,國有大喪,不出今年矣。」既而勒果死。



 及季龍僭位,遷都于鄴,傾心事澄,有重於勒。下書衣澄以綾錦,乘以彫輦,朝會之日,引之升殿,常侍以下悉助舉輿,太子諸公扶翼而上,主者唱大和尚,眾坐皆起,以彰其尊。又使司空李農旦夕親問,其太子諸公五日一朝,尊敬莫與為比。支道林在京師,聞澄與諸石游,乃曰:「澄公其以季龍為海鷗鳥也。百姓因澄故多奉佛,皆營造寺廟,相競出家,真偽混淆,多生愆過。季龍下書料簡,其著作郎王
 度奏曰:「佛,外國之神,非諸華所應祠奉。漢代初傳其道,惟聽西域人得立寺都邑,以奉其神,漢人皆不出家。魏承漢制,亦循前軌。今可斷趙人悉不聽詣寺燒香禮拜,以遵典禮,其百辟卿士下逮眾隸,例皆禁之,其有犯者,與淫祀同罪。其趙人為沙門者,還服百姓。」朝士多同度所奏。季龍以澄故,下書曰:「朕出自邊戎,忝君諸夏,至於饗祀,應從本俗。佛是戎神,所應兼奉,其夷趙百姓有樂事佛者,特聽之。」



 澄時止鄴城寺中,弟子遍於郡國。嘗遣弟子法常北至襄國,弟子法佐從襄國還,相遇於梁基城下,對車夜談,言及和尚,比旦各去。佐始入,澄逆笑曰:「昨
 夜爾與法常交車共說汝師邪?」佐愕然愧懺。於是國人每相語:「莫起惡心,和尚知汝。」及澄之所在,無敢向其方面涕唾者。



 季龍太子邃有二字,在襄國,澄語邃曰:「小阿彌比當得疾,可往看之。」邃即馳信往視,果已得疾。太醫殷騰及外國道士自言能療之。澄告弟子法牙曰:「正使聖人復出,不愈此疾,況此等乎!」後三日果死。邃將圖為逆,謂內豎曰:「和尚神通,儻發吾謀。明日來者,當先除之。」澄月望將入覲季龍,謂弟子僧慧曰:「昨夜天神呼我曰:『明日若入,還勿過人。」我儻有所過,汝當止我。」澄常入,必過邃。邃知澄入,要侯甚苦。澄將上南臺,僧慧引衣,澄
 曰:「事不得止。」坐未安便起,邃固留不住,所謀遂差。還寺,嘆曰:「太子作亂,其形將成,欲言難言,欲忍難忍。」乃因事從容箴季龍,季龍終不能解。俄而事發,方悟澄言。



 後郭黑略將兵征長安北山羌,墮羌伏中。時澄在堂上坐,慘然改容曰:「郭公今有厄。」乃唱云:「眾僧祝願。」澄又自祝願。須臾,更曰:若東南出者活,餘向者則困。」復更祝願。有頃,曰:「脫矣。」後月餘,黑略還,自說墜羌圍中,東南走,馬乏,正遇帳下人,推馬與之曰:「公乘此馬,小人乘公馬,濟與不濟,命也。」略得其馬,故獲免。推檢時日,正是澄祝願時也。



 時天旱,季龍遣其太子詣臨漳西滏口祈雨,久而不降,
 乃令澄自行,即有白龍二頭降於祠所,其日大雨方數千里。澄嘗遣弟子向西域市香,既行,澄告余弟子曰:「掌中見買香弟子在某處被劫垂死。」因燒香祝願,遙救護之。弟子後還,云某月某日某處為賊所劫,垂當見殺,忽聞香氣,賊無故自驚曰:「救兵已至。」棄之而走。黃河中舊不生黿,時有得者,以獻季龍。澄見而之曰:「桓溫入河,其不久乎!」溫字元子,後果如其言也。季龍嘗晝寢,夢見群羊負魚從東北來,寤以訪澄。澄曰:「不祥也,鮮卑其有中原乎!」後亦皆驗。澄嘗與季龍升中臺,澄忽驚曰:「變,變,幽州當火災。」乃取酒噀之,久而笑曰:「救已得矣。」季龍遣
 驗幽州,云爾日火從四門起,西南有黑雲來,驟雨滅之,雨亦頗有酒氣。



 石宣將殺石韜,宣先到寺與澄同坐,浮屠一鈴獨鳴,澄謂曰:「解鈴音乎?鈴云胡子洛度。」宣變色曰:「是何言歟?」澄謬曰:「老胡為道,不能山居無言,重茵美服,豈非洛度乎!」石韜後至,澄孰視良久。韜懼而問澄,澄曰:「怪公血臭,故相視耳。」季龍夢龍飛西南,自天而落,旦而問澄,澄曰:「禍將作矣,宜父子慈和,深以慎之。」季龍引澄入東閣,與其后杜氏問訊之。澄曰:「脅下有賊,不出十日,自浮圖以西,此殿以東,當有血流,慎勿東也。」杜后曰:「和尚耄邪!何處有賊?」澄即易語云:「六情所受,皆悉是賊。老
 自應耄,但使少者不昏即好耳。」遂便寓言,不復彰的。後二日,宣果遣人害韜於佛寺中,欲因季龍臨喪殺之。季龍以澄先誡,故獲免。及宣被收,澄諫季龍曰:「皆陛下之子也,何為重禍邪!陛下若含怒加慈者,尚有六十餘歲。如必誅之,宣當為彗星下掃鄴宮。」季龍不從。後月餘,有一妖馬,髦尾皆有燒狀,入中陽門,出顯陽門,東首東宮,皆不得入,走向東北,俄爾不見。澄聞而歎曰:「災其及矣!」季龍大享群臣於太武前殿,澄吟曰:「殿乎,殿乎!棘子成林,將壞人衣。」季龍令發殿石下視之,有棘生焉。冉閔小字棘奴。


季龍造太武殿初成,圖畫自古賢聖、忠臣、孝子、
 烈士、貞女,皆變為胡狀,旬餘,頭悉縮入肩中,惟冠
 \gezhu{
  髟介}
 仿佛微出,季龍大惡之,祕而不言也。澄對之流涕,乃自啟塋墓於鄴西紫陌,還寺,獨語曰:「得三年乎?」自答:「不得。」又曰:「得二年、一年、百日、一月乎?」自答:「不得。」遂無復言。謂弟子法祚曰:「戊申歲禍亂漸萌,己酉石氏當滅。吾及其未亂,先從化矣。」卒於鄴宮寺。後有沙門從雍州來,稱見澄西入關,季龍掘而視之,惟有一石無尸。季龍惡之曰:「石者,朕也,葬我而去,吾將死矣。」因而遇疾。明年,季龍死,遂大亂。



 麻襦者,不知何許人也,莫得其姓名。石季龍時,在魏縣市中乞丐,恒著麻襦布裳,故時人謂之麻襦。言語卓越,狀如狂者,乞得米穀不食,輒散置大路,云飴天馬。趙興太守籍狀收送詣季龍。



 先是,佛圖澄謂季龍曰:「國東二百里某月日當送一非常人,勿殺之也。」如期果至。季龍與共語,了無異言,惟道:「陛下當終一柱殿下。」季龍不解,送以詣澄。麻襦謂澄曰:「昔在光和中會,奄至今日。酉戎受玄命,絕歷終有期。金離消于壞,邊荒不能遵,驅除靈期迹,莫已已之懿。裔苗葉繁,其來方積。休期於何期,永以歎之。」澄曰:「天回運極,否將不支,九木水為難,無可以
 術寧。玄哲雖存世,莫能基必莫能基必頹。久游閻浮利,擾擾多此患。行登陵雲宇,會於虛游間。」其所言人莫能曉。季龍遣驛馬送還本縣,既出城,請步,云:「我當有所過,君至合口橋見待。」使人如言而馳,至橋,麻襦已先至。



 後慕容俊投季龍尸於漳水,倚橋柱不流,時人以為「一柱殿下」即謂此也。及元帝嗣位江左,亦以為「天馬」之應云。



 單道開,敦煌人也。常衣麤褐,或贈以繒服,皆不著,不畏寒暑,晝夜不臥。恒服細石子,一吞數枚,日一服,或多或少。好山居,而山樹諸神見異形試之,初無懼色。石季龍
 時,從西平來,一日行七百里,其一沙彌年十四,行亦及之。至秦州,表送到鄴,季龍令佛圖澄與語,不能屈也。初止鄴城西沙門法綝祠中,後徙臨漳昭德寺。於房內造重閣,高八九尺,於上編管為禪室,常坐其中。季龍資給甚厚,道開皆以施人。人或來諮問者,道開都不答。日服鎮守藥數丸,大如梧子,藥有松蜜姜桂伏苓之氣,時復飲荼蘇一二升而已。自云能療目疾,就療者頗驗。視其行動,狀若有神。佛圖澄曰:「此道士觀國興衰,若去者,當有大亂。」及季龍末,道開南渡許昌,尋而鄴中大亂。



 升平三年至京師,後至南海,入羅浮山,獨處茅茨,蕭然物外。
 年百餘歲,卒于山舍,敕弟子以尸置石穴中,弟子乃移入石室。陳郡袁宏為南海太守,與弟穎叔及沙門支法防共登羅浮山,至石室口,見道開形骸如生,香火瓦器猶存。宏曰:「法師業行殊群,正當如蟬蛻耳。」乃為之贊云。



 黃泓,字始長,魏郡斥丘人也。父沈,善天文祕術。泓從父受業,精妙踰深,兼博覽經史,尤明《禮》《易》。性忠勤,非禮不動。永嘉之亂,與渤海高瞻避地幽州,說瞻曰:「王浚昏暴,終必無成,宜思去就以圖久安。慕容廆法政修明,虛懷引納,且讖言真人出東北,儻或是乎?宜相與歸之,同建
 事業。」瞻不從。泓乃率宗族歸廆,廆待以客禮,引為參軍,軍國之務動輒訪之。泓指說成敗,事皆如言。廆常曰:「黃參軍,孤之仲翔也。」及皝嗣位,遷左常侍,領史官,甚重之。石季龍攻皝,皝將走遼東,泓曰:「賊有敗氣,無可憂也,不過二日,必當奔潰。宜嚴勒士馬,為追擊之備。」皝曰:「今寇盛如此,卿言必走,孤未敢信。」泓曰:「殿下言盛者,人事耳,臣言必走者,天時也,胡足為疑!」及期,季龍果退,皝益奇之。



 及慕容俊即王位,遷從事中郎,人雋聞冉閔亂,將圖中原,訪之於泓,泓勸行,人雋從之。及僭號,署為進謀將軍、太史令、關內侯,尋加奉車都尉、西海太守、領太史令、開陽
 亭侯,又封平舒縣五等伯,常從左右,諮決大事,靈臺令許敦害其寵,諂事慕容評,設異議以毀之,及以泓為太史靈臺諸署統,加給事中。泓待敦彌厚,不以毀己易心。慕容敗,以老歸家,歎曰:「燕必中興,其在吳王,恨吾年過不見耳。」年九十七卒。卒後三年,偽吳王慕容垂興焉。



 索紞,字叔徹,敦煌人也。少游京師,受業太學,博綜經籍,遂為通儒。明陰陽天文,善術數占侯。司徒辟,除郎中,知中國將亂,避世而歸。鄉人從紞占問吉凶,門中如市,紞曰:「攻乎異端,戒在害己;無為多事,多事多患。」遂詭言虛
 說,無驗乃止。惟以占夢為無悔吝,乃不逆問者。



 孝廉令狐策夢立冰上,與冰下人語。紞曰:「冰上為陽,冰下為陰,陰陽事也。士如歸妻,迨冰未泮,婚姻事也。君在冰上與冰下人語,為陽語陰,媒介事也。君當為人作媒,冰泮而婚成。」策曰:「老夫耄矣,不為媒也。」會太守田豹因策為子求鄉人張公征女,仲春而成婚焉。郡主簿張宅夢走馬上山,還繞舍三周,但見松柏,不知門處。紞曰:「馬屬離,離為火。火,禍也。人上山,為凶字。但見松伯,墓門象也。不知門處,為無門也。三周,三期也。後三年必有大禍。」宅果以謀反伏誅。索充初夢天上有二棺落充前,紞曰:「棺者,職
 也,當有京師貴人舉君。二官者,頻再遷。」俄而司徒王戎書屬太守使舉充,太守先署充功曹而舉孝廉。充後夢見一虜,脫上衣來詣充。紞曰:「虜去上中,下半男字,夷狄陰類,君婦當生男。」終如其言。宋桷夢內中有一人著赤衣,桷手把兩杖,極打之。紞曰:「內中有人,肉字也。肉色,赤也。兩杖,箸象也。極打之,飽肉食也。」俄而亦驗焉。黃平問紞曰:「我昨夜夢舍中馬舞,數十人向馬拍手,此何祥也?」紞曰:「馬者,火也,舞為火起。向馬拍手,救火人也。」平未歸而火作。索綏夢東有二角書詣綏,大角朽敗,小角有題韋囊角佩,一在前,一在後。紞曰:「大角朽敗,腐棺木。小角
 有題,題所詣。一在前,前凶也。一在後,後背也。當有凶背之問。」時綏父在東,居三日而凶問至。郡功曹張邈嘗奉使詣州,夜夢狼啖一腳。紞曰:「腳肉被啖,為卻字。」會東虜反,遂不行。凡所占莫不驗。



 太守陰澹從求占書,紞曰:「昔入太學,因一父老為主人,其人無所不知,又匿姓名,有似隱者,紞因從父老問占夢之術,審測而說,實無書也。」澹命為西閣祭酒,紞辭曰:「少無山林之操,游學京師,交結時賢,希申鄙藝。會中國不靖,欲養志終年。老亦至矣,不求聞達。又少不習勤,老無吏乾,濛汜之年,弗敢聞命。」澹以束帛禮之,月致羊酒。年七十五,卒于家。



 孟欽,洛陽人也。有左慈、劉根之術,百姓惑而赴之。苻堅召詣長安,惡其惑眾,命苻融誅之。俄而欽至,融留之,遂大宴郡僚,酒酣,目左右收欽。欽化為旋風,飛出第外。頃之,有告在城東者,融遣騎追之,垂及,忽然已遠,或有兵眾距戰,或前有谿澗,騎不得過,遂不知所在。堅未,復見於青州。苻朗尋之,入于海島。



 王嘉,字子年,隴西安陽人也。輕舉止,醜形貌,外若不足,而聰睿內明。滑稽好語笑,不食五穀,不衣美麗,清虛服
 氣,不與世人交游。隱于東陽谷,鑿崖穴居,弟子受業者數百人,亦皆穴處。石季龍之末,棄其徒眾,至長安,潛隱於終南山,結庵廬而止。門人聞而復隨之,乃遷于倒獸山。苻堅累徵不起,公侯已下咸躬往參詣,好尚之士無不師宗之。問其當世事者,皆隨問而對。好為譬喻,狀如戲調;言未然之事,辭如讖記,當時鮮能曉之,事過皆驗。



 堅將南征,遣使者問之。嘉曰:「金剛火彊。」乃乘使者馬,正衣冠,徐徐東行數百步,而策馬馳反,脫衣服,棄冠履而歸,下馬踞床,一無所言。使者還告,堅不語,復遣問之,曰:「吾世祚云何?」嘉曰:「未央。」咸以為吉。明年癸未,敗于淮南,
 所謂未年而有殃也。人侯之者,至心則見之,不至心則隱形不見。衣服在架,履杖猶存,或欲取其衣者,終不及,企而取之,衣架踰高,而屋亦不大,覆杖諸物亦如之。



 姚萇之入長安,禮嘉如苻堅故事,逼以自隨,每事諮之。萇既與苻登相持,問嘉曰:「吾得殺苻登定天下不?」嘉曰:「略得之。」萇怒曰:「得當云得,何略之有!」遂斬之。先此,釋道安謂嘉曰:「世故方殷,可以行矣。」嘉答曰:「卿其先行,吾負債未果去。」俄而道安亡,至是而嘉戮死,所謂「負債」者也。苻登聞嘉死,設壇哭之,贈太師,謚曰文。及萇死,萇子興字子略方殺登,「略得」之謂也。嘉之死日,人有隴上見之。其
 所造《牽三歌讖》,事過皆驗,累世猶傳之。又著《拾遺錄》十卷,其記事多詭怪,今行於世。



 僧涉者,西域人也,不知何姓。少為沙門,苻堅時入長安。虛靜服氣,不食五穀,日能行五百里,言未然之事,驗若指掌。能以祕祝下神龍,每旱,堅常使之咒龍請雨。俄而龍下缽中,天輒大雨,堅及群臣親就缽觀之。卒于長安。後大旱移時,苻堅嘆曰:「涉公若在,豈憂此乎!」



 郭黁,西平人也。少明《老》《易》,仕郡主簿。張天錫末年,苻氏
 每有西伐之問,太守趙凝使黁筮之,黁曰:「若郡內二月十五日失囚者,東軍當至,涼祚必終。」凝乃申約屬縣。至十五日,鮮卑折掘送馬於凝,凝怒其非駿,幽之內廄,鮮卑懼而夜遁。凝以告黁,黁曰:「是也。國家將亡,不可復振。」



 苻堅末,當陽門震,刺史梁熙問黁曰:「其祥安在?」黁曰:「為四夷之事也。當有外國二王來朝主上,一當反國,一死此城。」歲餘而鄯善及前部王朝于苻堅,西歸,鄯善王死於姑臧。



 呂光之王河西也,西海太守王楨叛,黁勸光襲之。光之左丞呂寶曰:「千里襲人,自昔所難,況王者之師天下所聞,何可僥倖以邀成功!黁不可從,誤人大事。」黁
 曰:「若其不捷,黁自伏鈇鉞之誅。如其剋也,左丞為無謀矣。」光從而剋之。光比之京管,常參帷幄密謀。



 光將伐乞伏乾歸,黁諫曰:「今太白未出,不宜行師,往必無功,終當覆敗。」太史令賈曜以為必有秦隴之地。及剋金城,光使曜詰黁,黁密謂光曰:「昨有流星東墮,當有伏尸死將,雖得此城,憂在不守。正月上旬,河冰將解,若不早渡,恐有大變。」後二日而敗問至,光引軍渡河訖,冰泮。時人服其神驗。光以黁為散騎常侍、太常。



 黁後以光年老,知其將敗,遂與光僕射王祥起兵作亂。百姓聞黁起兵,咸以聖人起事,事無不成,故相率從之如不及。黁以為代呂者
 王,乃推王乞基為主。後呂隆降姚興,興以王尚為涼州刺史,終如黁言。黁之與光相持也,逃人稱呂統病死,黁曰:「未也,光、統之命盡在一時。」黁後統死三日而光死。黁嘗曰:「涼州謙光殿後當有索頭鮮卑居之。」終於禿髮傉檀、沮渠蒙遜迭據姑臧。黁性褊酷,不為士庶所附。戰敗,奔乞伏乾歸。乾歸敗,入姚興。黁以滅姚者晉,遂將妻子南奔,為追兵所殺也。



 鳩摩羅什,天竺人也。世為國相。父鳩摩羅炎,聰懿有大節,將嗣相位,乃辭避出家,東渡葱嶺。龜茲王聞其名,郊
 迎之,請為國師。王有妹,年二十,才悟明敏,諸國交娉,並不許,及見炎,心欲當之,王乃逼以妻焉。既而羅什在胎,其母慧解倍常。及年七歲,母遂與俱出家。



 羅什從師受經,日誦千偈,偈有三十二字,凡三萬二千言,義亦自通。年十二,其母攜到沙勒,國王甚重之,遂停沙勒一年。博覽五明諸論及陰陽星算,莫不必盡,妙達吉凶,言若符契。為性率達,不拘小檢,修行者頗共疑之。然羅什自得於心,未嘗介意,專以大乘為化,諸學者皆共師焉。年二十,龜茲王迎之還國,廣說諸經,四遠學徒莫之能抗。



 有頃,羅什母辭龜茲王往天竺,留羅什住,謂之曰:「方等深
 教,不可思議,傳之東土,惟爾之力。但於汝無利,其可如何?」什曰:「必使大化流傳,雖苦而無恨。」母至天竺,道成,進登第三果。西域諸國咸伏羅什神俊,每至講說,諸王皆長跪坐側,令羅什踐而登焉。苻堅聞之,密有迎羅什之意。會太史奏云:「有星見外國分野,當有大智入輔中國。」堅曰:「朕聞西域有鳩摩羅什,將非此邪?」乃遣驍騎將軍呂光等率兵七萬,西伐龜茲,謂光曰:「若獲羅什,即馳驛送之。」光軍未至,羅什謂龜茲王白純曰:「國運衰矣,當有勍敵從日下來,宜恭承之,勿抗其鋒。」純不從,出兵距戰,光遂破之,乃獲羅什。光見其年齒尚少,以凡人戲之,強
 妻以龜茲王女,羅什距而不受,辭甚苦至。光曰:「道士之操不踰先父,何所固辭?」乃飲以醇酒,同閉密室。羅什被逼,遂妻之。光還,中路置軍於山下,將士已休,羅什曰:「在此必狼狽,宜徙軍隴上。」光不納。至夜,果大雨,洪潦暴起,水深數丈,死者數千人,光密異之。光欲留王西國,羅什謂光曰:「此凶亡之地,不宜淹留,中路自有福地可居。」光還至涼州,聞苻堅已為姚萇所害,於是竊號河右。屬姑臧大風,羅什曰:「不祥之風當有奸叛,然不勞自定也。」俄而有叛者,尋皆殄滅。



 沮渠蒙遜先推建康太守段業為主,光遣其子纂率眾討之。時論謂業等烏合,纂有威聲,
 勢必全剋。光以訪羅什,答曰:「此行未見其利。」既而纂敗於合黎,俄又郭黁起兵,纂棄大軍輕還,復為黁所敗,僅以身免。



 中書監張資病,光博營救療。有外國道人羅叉,云能差資病。光喜,給賜甚重。羅什知叉誑詐,告資曰:「叉不能為益,徒煩費耳。冥運雖隱,可以事試也。」乃以五色絲作繩結之,燒為灰末,投水中,灰若出水還成繩者,病不可愈。須臾,灰聚浮出,復為繩,叉療果無效,少日資亡。



 頃之,光死,纂立。有豬生子,一身三頭。龍出東箱井中,於殿前蟠臥,比旦失之。纂以為美瑞,號其殿為龍翔殿。俄而有黑龍升於當陽九宮門,纂改九宮門為龍興門。羅
 什曰:「比日潛龍出游,豕妖表異,龍者陰類,出入有時,而今屢見,則為災眚,必有下人謀上之變。宜剋己脩德,以答天戒。」纂不納,後果為呂超所殺。



 羅什之在涼州積年,呂光父子既不弘道,故蘊其深解,無所宣化。姚興遣姚碩德西伐,破呂隆,乃迎羅什,待以國師之禮,仍使入西明閣及逍遙園,譯出眾經。羅什多所暗誦,無不究其義旨,既覽舊經多有紕繆,於是興使沙門僧睿、僧肇等八百餘人傳受其旨,更出經論,凡三百餘卷。沙門慧睿才識高明,常隨羅什傳寫,羅什每為慧睿論西方辭體,商略同異,云:「天竺國俗甚重文制,其宮商體韻,經入管弦
 為善。凡覲國王,必有贊德,經中偈頌,皆其式也。」羅什雅好大乘,志在敷演,常歎曰:「吾若著筆作大乘阿毗曇,非迦旃子比也。今深識者既寡,將何所論!」惟為姚興著《實相論》二卷,興奉之若神。



 嘗講經于草堂寺,興及朝臣、大德沙門千有餘人肅容觀聽,羅什忽下高坐,謂興曰:「有二小兒登吾肩,慾鄣須婦人。」興乃召宮女進之,一交而生二子焉。興嘗謂羅什曰:「大師聽明超悟,天下莫二,何可使法種少嗣。」遂以伎女十人,逼令受之。爾後不住僧坊,別立解舍。諸僧多效之。什乃聚針盈缽,引諸僧謂之曰:「若能見效食此者,乃可畜室耳。」因舉匕進針,與常食
 不別,諸僧愧服乃止。



 杯渡比丘在彭城,聞羅什在長安,乃歎曰:「吾與此子戲,別三百餘年,相見杳然未期,遲有遇於來生耳。」羅什未終少日,覺四大不愈,乃口出三番神咒,令外國弟子誦之以自救,未及致力,轉覺危殆,於是力疾與眾僧告別曰:「因法相遇,殊未盡心,方復後世,惻愴可言。」死於長安。姚興於逍遙園依外國法以火焚尸,薪滅形碎,惟舌不爛。



 沙門曇霍者,不知何許人也。禿髮傉檀時從河南來,持一錫杖,令人跪曰:「此是般若眼,奉之可以得道。」時人咸
 異之。或遺以衣服,受而投之於河,後日以還其本主,衣無所污。行步如風雲,言人死生貴賤無毫釐之差。人或藏其錫杖,曇霍大哭數聲,閉目須臾,起而取之,咸奇其神異,莫能測也。每謂傉檀曰:「若能安坐無為,則天下可定,祚胤克昌,如其窮兵好殺,禍將及己。」傉檀不能從。傉檀女病甚,請救療,曇霍曰:「人之生死自有定期,聖人亦不能轉禍為福,曇霍安能延命邪!正可知早晚耳。」傉檀固請之。時後宮門閉,曇霍曰:急開後門,及開門則生,不及則死。」傉檀命開之,不及而死。後兵亂,不知所在也。



 臺產,
 字國俊,上洛人,漢侍中崇之後也。少專京氏《易》,善圖讖、秘緯、天文、洛書、風角、星算、六日七分之學,尤善望氣、占候、推步之術。隱居商洛南山,兼善經學,泛情教授,不交當世。劉曜時,災異特甚,命公卿各舉博識直言之士一人。其大司空劉均舉產。曜親臨東堂,遣中黃門策問之,產極言其故。曜覽而嘉之,引見,訪以政事。產流涕歔欷,具陳災變之禍,政化之闕,辭甚懇至。曜改容禮之,署為博士祭酒、諫議大夫,領太史令。至明年而其言皆驗,曜彌重之,轉太中大夫,歲中三遷。歷位尚書、光祿大夫、太子少師,位特進,金章紫綬,爵關中侯。



 史臣曰:陳戴等諸子並該洽墳典,研精數術,究推步之幽微,窮陰陽之祕奧,雖前代京管,何以加之!郭黁知有晉之亡姚,去姚以歸晉,追兵奄及,致斃中途,斯則遠見秋毫,不能近知目睫。澄什爰自遐裔,來游諸夏。什既兆見星象,澄乃驅役鬼神,並通幽洞冥,垂文闡教,諒見珍於道藝,非取貴於他山,姚石奉之若神,良有以也。鮑、吳、王、幸等或假靈道訣,或受教神方,遂能厭勝禳災,隱文彰義,雖獲譏於妖妄,頗有益於世用者焉。然而碩學通人,未宜枉轡。



 贊曰:《傳》敘災祥,《書》稱龜筮。應如影響,葉若符契。怪力亂
 神,詭時惑世。崇尚弗已,必致流弊。



\end{pinyinscope}