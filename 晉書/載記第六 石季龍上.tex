\article{載記第六 石季龍上}

\begin{pinyinscope}

 石季龍上



 石季龍,勒之從子也,名犯太祖廟諱,故稱字焉。祖曰[C111]邪,父曰寇覓。勒父朱幼而子季龍,故或稱勒弟焉。年六七歲,有善相者曰:「此兒貌奇有壯骨,貴不可言。」永興中,與勒相失。後劉琨送勒母王及季龍于葛陂,時年十七矣。性殘忍,好馳獵,游蕩無度,尤善彈,數彈人,軍中以為毒患。勒白王將殺之,王曰:「快牛為犢子時,多能破車,汝
 當小忍之。」年十八,稍折節。身長七尺五寸,趫捷便弓馬,勇冠當時,將佐親戚莫不敬憚,勒深嘉之,拜征虜將軍。為娉將軍郭榮妹為妻。季龍寵惑優僮鄭櫻桃而殺郭氏,更納清河崔氏女,櫻桃又譖而殺之。所為酷虐。軍中有勇乾策略與己俟者,輒方便害之,前後所殺甚眾。至於降城陷壘,不復斷別善惡,坑斬士女,鮮有遺類。勒雖屢加責誘,而行意自若。然御眾嚴而不煩,莫敢犯者,指授攻討,所向無前,故勒寵之,信任彌隆,仗以專征之任。



 勒之居襄國,署為魏郡太守,鎮鄴三臺,後封繁陽侯。勒即大單于、趙王位,署為單于元輔、都督禁衛諸軍事,遷
 侍中、開府,進封中山公。及勒僭號,授太尉、守尚書令,進封為王,邑萬戶。季龍自以勳高一時,謂勒即位之後,大單于必在己,而更以授其子弘。季龍深恨之,私謂其子邃曰:「主上自都襄國以來,端拱指授,而以吾躬當矢石。二十餘年,南擒劉岳,北走索頭,東平齊、魯,西定秦、雍,克殄十有三州。成大趙之業者,我也。大單于之望實在于我,而授黃吻婢兒,每一憶此,令人不復能寢食。待主上晏駕之後,不足復留種也。」



 咸康元年,季龍廢勒子弘,群臣已下勸其稱尊號。季龍下書曰:「王室多難,海陽自棄,四海業重,故免從推逼。朕聞道合乾坤者稱皇,德協人
 神者稱帝,皇帝之號非所敢聞,且可稱居攝趙天王,以副天人之望。」於是赦其境內,改年曰建武。以夔安為侍中、太尉、守尚書令,郭殷為司空,韓晞為尚書左僕射,魏概、馮莫、張崇、曹顯為尚書,申鐘為侍中,郎闓為光祿大夫,王波為中書令,文武封拜各有差。立其子邃為太子。季龍以讖文天子當從東北來,於是備法駕行自信都而還以應之。分癭陶之柳鄉立停駕縣。



 季龍徐州從事硃縱殺刺史郭祥,以彭城歸順。季龍遣將王朗擊之,縱奔淮南。



 季龍荒游廢政,多所營繕,使邃省可尚書奏事,選牧守,祀郊廟;惟征伐刑斷乃親覽之。觀雀臺崩,殺典
 匠少府任汪。復使脩之,倍於常度。



 季龍自率眾南寇歷陽,臨江而旋,京師大震。遣其征虜石遇寇中廬,遂圍平北將軍桓宣於襄陽。輔國將軍毛寶、南中郎將王國、征西司馬王愆期等率荊州之眾救之,屯於章山。遇攻守二旬,軍中饑疫而還。



 季龍以租入殷廣,轉輸勞煩,令中倉歲入百萬斛,餘皆儲之水次。



 晉將軍淳于安攻其瑯邪費縣,俘獲而歸。



 石邃保母劉芝初以巫術進,既養邃,遂有深寵,通賄賂,豫言論,權傾朝廷,親貴多出其門,遂封芝為宜城君。



 季龍下書令刑贖之家得以錢代財帛,無錢聽以穀麥,皆隨時價輸水次倉。冀州八郡雨雹,大
 傷秋稼,下書深自咎責。遣御史所在發水次倉麥,以給秋種,尤甚之處差復一年。



 季龍將遷于鄴,尚書請太常告廟,季龍曰:「古者將有大事,必告宗廟,而不列社稷。尚書可詳議以聞。」公卿乃請使太尉告社稷,從之。及入鄴宮,澍雨周洽,季龍大悅,赦殊死已下。尚方令解飛作司南車成,季龍以其構思精微,賜爵關內侯,賞賜甚厚。始制散騎常侍已上得乘軺軒,王公郊祀乘副車,駕四馬,龍旂八旒,塑望朝會即乘軺軒。



 時羌薄句大猶保險未賓,遣其子章武王斌帥精騎二萬,并秦、雍二州兵以討之。



 季龍如長樂、衛國,有田疇不闢、桑業不脩者,貶其守
 宰而還。



 咸康二年,使牙門將張彌徙洛陽鐘虡、九龍、翁仲、銅駝、飛廉于鄴。鐘一沒於河,募浮沒三百人入河,繫以竹絙,牛百頭,鹿櫨引之乃出。造萬斛舟以渡之,以四輪纏輞車,轍廣四尺,深二尺,運至鄴。季龍大悅,赦二歲刑,賚百官穀帛,百姓爵一級。



 下書曰:「三載考績,黜陟幽明,斯則先王之令典,政道之通塞。魏始建九品之制,三年一清定之,雖未盡弘美,亦縉紳之清律,人倫之明鏡。從爾以來,遵用無改。先帝創臨天下,黃紙再定。至於選舉,銓為首格。自不清定,三載于茲。主者其更銓論,務揚清激濁,使九流咸允也。吏部選舉,可依晉氏九班選制,
 永為揆法。選畢,經中書、門下宣示三省,然後行之。其著此詔書於令。銓衡不奉行者,御史彈坐以聞。」



 索頭郁鞠率眾三萬降于季龍,署鞠等一十三人親通趙王,皆封列侯,散其部眾于冀、青等六州。



 時眾役煩興,軍旅不息,加以久旱穀貴,金一斤直米二斗,百姓嗷然無生賴矣。又納解飛之說,於鄴正南投石於河,以起飛橋,功費數千億萬,橋竟不成,役夫飢甚,乃止。使令長率丁壯隨山津采橡捕魚以濟老弱,而復為權豪所奪,人無所得焉。又料殷富之家,配飢人以食之,公卿已下出穀以助振給,姦吏因之侵割無已,雖有貸贍之名而無其實。



 改直
 盪為龍騰,冠以絳幘。



 於襄國起太武殿,於鄴造東西宮,至是皆就。太武殿基高二丈八尺,以文石碎之,下穿伏室,置衛士五百人於其中。東西七十五步,南北六十五步。皆漆瓦、金鐺、銀楹、金柱、珠簾、玉壁,窮極枝巧。又起靈風臺九殿於顯陽殿後,選士庶之女以充之。後庭服綺縠、玩珍奇者萬餘人,內置女官十有八等,教宮人星占及馬步射。置女太史于靈臺,仰觀災祥,以考外太史之虛實。又置女鼓吹羽儀,雜伎工巧,皆與外侔。禁郡國不得私學星讖,敢有犯者誅。



 左校令成公段造庭燎于崇杠之末,高十餘丈,上盤置燎,下盤置人,絙繳上下。季龍
 試而悅之。其太保夔安等文武五百九人勸季龍稱尊號,安等方入而庭燎油灌下盤,死者七人。季龍惡之,大怒,斬成公段於閶闔門。



 於是依殷周之制,以咸康三年僭稱大趙天王,即位於南郊,大赦殊死已下。追尊祖[C111]邪為武皇帝,父寇覓為太宗孝皇帝。立其鄭氏為天王皇后,以子邃為天王皇太子。親王皆貶封郡公,籓王為縣侯,百官封署各有差。



 太原徙人五百餘戶叛入黑羌。



 武鄉長城徙人韓彊獲玄玉璽,方四寸七分,龜紐金文,詣鄴獻之。拜彊騎都尉,復其一門。夔安等又勸進曰:「臣等謹案大趙水德,玄龜者,水之精也;玉者,石之寶
 也;分之數以象七政,寸之紀以準四極。昊天成命,不可久違。輒下史官擇吉日,具禮儀,謹昧死上皇帝尊號。」季龍下書曰:「過相褒美,猥見推逼,覽增恧然,非所望也,其亟止茲議。今東作告始,自非京城內外,皆不得表慶。」中書令王波上《玄璽頌》以美之。季龍以石弘時造此璽,彊遇而獻之。



 邃自總百揆之後,荒酒淫色,驕恣無道,或盤游于田,懸管而入,或夜出于宮臣家,淫其妻妾。妝飾宮人美淑者,斬首洗血,置於盤上,傳共視之。又內諸比丘尼有姿色者,與其交褻而殺之,合牛羊肉煮而食之,亦賜左右,欲以識其味也。河間公宣、樂安公韜有寵於季
 龍,邃疾之如仇。季龍荒耽內游,威刑失度,邃以事為可呈呈之,季龍恚曰:「此小事,何足呈也。」時有所不聞,復怒曰:「何以不呈?」誚責杖捶,月至再三。邃甚恨,私謂常從無窮、長生、中庶子李顏等曰:「官家難稱,吾欲行冒頓之事,卿從我乎?」顏等伏不敢對。邃稱疾不省事,率宮臣文武五百餘騎宴于李顏別舍,謂顏等曰:「我欲至冀州殺石宣,有不從者斬!」行數里,騎皆逃散,李顏叩頭固諫,邃亦昏醉而歸。邃母鄭氏聞之,私遣中人責邃。邃怒,殺其使。季龍聞邃有疾,遣所親任女尚書察之。邃呼前與語,抽劍擊之。季龍大怒,收李顏等詰問,顏具言始末,誅顏等
 三十餘人。幽邃于東宮,既而赦之,引見太武東堂。邃朝而不謝,俄而便出。季龍遣使謂邃曰:「太子應入朝中宮,何以便去?」邃逕出不顧。季龍大怒,廢邃為庶人。其夜,殺邃及妻張氏并男女二十六人,同埋於一棺之中。誅其宮臣支黨二百餘人。廢鄭氏為東海太妃。立其子宣為天王皇太子,宣母杜昭儀為天王皇后。



 安定人侯子光,弱冠美姿儀,自稱佛太子,從大秦國來,當王小秦國。易姓名為李子楊,游于鄠縣爰赤眉家,頗見其妖狀,事微有驗。赤眉信敬之,妻以二女,轉相扇惑。京兆樊經、竺龍、嚴諶、謝樂子等聚眾數千人於杜南山,子楊稱大黃帝,
 建元曰龍興。赤眉與經為左右丞相,龍、諶為左右大司馬,樂子為大將軍。鎮西石廣擊斬之。子楊頸無血,十餘日而面色無異於生。



 季龍將伐遼西鮮卑段遼,募有勇力者三萬人,皆拜龍騰中郎。遼遣從弟屈雲襲幽州,刺史李孟退奔易京。季龍以桃豹為橫海將軍,王華為渡遼將軍,統舟師十萬出漂渝津,支雄為龍驤大將軍,姚弋仲為冠軍將軍,統步騎十萬為前鋒,以伐段遼。季龍眾次金臺,支雄長驅入薊,遼漁陽太守馬鮑、代相張牧、北平相陽裕、上谷相侯龕等四十餘城並率眾降于季龍。支雄攻安次,斬其部大夫那樓奇。遼懼,棄令支,奔于
 密雲山。遼右左長史劉群、盧諶、司馬崔悅等封其府庫,遣使請降。季龍遣將軍郭太、麻秋等輕騎二萬追遼,及之,戰於密雲,獲其母妻,斬級三千。遼單馬竄險,遣子乞特真送表及名馬,季龍納之。乃遷其戶二萬餘于雍、司、兗、豫四州之地,諸有才行者皆擢敘之。先是,北單于乙回為鮮卑敦那所逐,既平遼西,遣其將李穆擊那破之,復立乙回而還。季龍入遼宮,論功封賞各有差。



 初,慕容皝與段遼有隙,遣使稱籓于季龍,陳遼宜伐,請盡眾來會。及軍至令支,皝師不出,季龍將伐之。天竺佛圖澄進曰:「燕福德之國,未可加兵。」季龍作色曰:「以此攻城,何城
 不剋?以此眾戰,誰能禦之?區區小豎,何所逃也?」太史令趙攬固諫曰:「燕城歲星所守,行師無功,必受其禍。」季龍怒,鞭之,黜為肥如長。進師攻棘城,旬餘不剋。皝遣子恪帥胡騎二千,晨出挑戰,諸門皆若有師出者,四面如雲,季龍大驚,棄甲而遁。於是召趙攬復為太史令。季龍旋自令支,過易京,惡其固而毀之。還謁石勒墓,朝其群臣于襄國建德前殷,復從徵文武有差。至鄴,設飲至之禮,賜俘偏於丞郎。



 季龍謀伐昌黎,遣渡遼曹伏將青州之眾渡海,戍蹋頓城,無水而還,因戍於海島,運穀三百萬斛以給之。又以船三百艘運穀三十萬斛詣高句麗,使
 典農中郎將王典率眾萬餘屯田於海濱。又令青州造船千艘。使石宣率步騎二萬擊朔方鮮卑斛摩頭破之,斬首四萬餘級。



 冀州八郡大蝗,司隸請坐守宰,季龍曰:「此政之失和,朕之不德,而欲委咎守宰,豈禹、湯罪己之義邪!司隸不進讜言,佐朕不逮,而歸咎無辜,所以重吾之責,可白衣領司隸。」



 加其子司徒韜金鉦黃鉞,鑾輅九旒。



 先是,使襄城公涉歸、上庸公日歸率眾戍長安,二歸告鎮西石廣私樹恩澤,潛謀不軌。季龍大怒,追廣至鄴,殺之。



 段遼於密雲山遣使詐降,季龍信之,使征東麻秋百里郊迎,敕秋曰:「受降如待敵,將軍慎之。」遼又遣使降
 于慕容皝曰:「胡貪而無謀,吾今請降求迎,彼終不疑也。若伏重軍以要之,可以得志。」皝遣子恪伏兵于密雲。麻秋統眾三萬迎遼,為恪所襲,死者十六七,秋步遁而歸。季龍聞之驚怒,方食吐餔,乃削秋官爵。



 下書令諸郡國立五經博士。初,勒置大小學博士,至是復置國子博士、助教。季龍以吏部選舉斥外耆德,而勢門童幼多為美官,免郎中魏KJ為庶人。以其太子宣為大單于,建天子旌旗。



 以夔安為征討大都督,統五將步騎七萬寇荊揚北鄙。石閔敗王師于沔陰,將軍蔡懷死之。宣將朱保又敗王師于白石,將軍鄭豹、談玄、郝莊、隨相、蔡熊皆遇害。
 季龍將張賀度攻陷邾城,敗晉將毛寶于邾西,死者萬餘人。夔安進據胡亭,晉將軍黃沖、歷陽太守鄭進皆降之。安於是掠七萬戶而還。



 時豪戚侵恣,賄託公行,季龍患之,擢殿中御史李矩為御史中丞,特親任之。自此百僚震懾,州郡肅然。季龍曰:「朕聞良臣如猛獸,高步通衢而豺狼避路,信矣哉!」



 鎮遠王擢表雍、秦二州望族,自東徙已來,遂在戍役之例,既衣冠華胄,宜蒙優免,從之。自是皇甫、胡、梁、韋、杜、牛、辛等十有七姓蠲其兵貫,一同舊族,隨才銓敘,思欲分還桑梓者聽之;其非此等,不得為例。



 以其撫軍李農為使持節、監遼西北平諸軍事、征東
 將軍、營州牧,鎮令支。



 于時大旱,白虹經天,季龍下書曰:「朕在位六載,不能上和乾象,下濟黎元,以致星虹之變。其令百僚各上封事,解西山之禁,蒲葦魚鹽除歲供之外,皆無所固。公侯卿牧不得規占山澤,奪百姓之利。」又下書曰:「前以豐國、澠池二冶初建,徙刑徒配之,權救時務。而主者循為恒法,致起怨聲。自今罪犯流徒,皆當申奏,不得輒配也。京獄見囚,非手殺人,一皆原遣。」其日澍雨。



 季龍將討慕容皝,令司、冀、青、徐、幽、、并、雍兼復之家五丁取三。四丁取二,合鄴城舊軍滿五十萬,具船萬艘,自河通海,運穀豆千一百萬斛于安樂城,以備征軍之調。
 徙遼西、北平、漁陽萬戶于兗、豫、雍、洛四州之地。



 季龍僭位之後,有所調用,皆選司擬官,經令僕而後奏行。不得其人,案以為令僕之負,尚書及郎不坐。至是,吏部尚書劉真以為失銓考之體而言之,季龍責怒主者,加真光祿大夫,金章紫綬。



 季龍如宛陽,大閱於曜武場。



 慕容皝襲幽、冀,略三萬餘家而去。幽州刺史石光坐懦弱徵還。



 賜征士辛謐几杖衣服,穀五百斛,敕平原為起甲第。



 先是,李壽將李宏自晉奔于季龍,壽致書請之,題曰趙王石君。季龍不悅,付外議之,多有異同。中書監王波議曰:「今李宏以死自誓,若得反魂蜀漢,當鳩率宗族,混同
 王化。若遣而果也,則不煩一旅之師而坐定梁、益,就有進退,豈在逃命一夫。壽既號並日月,跨僭一方,今若制詔,或敢酬反,則取誚戎裔。宜書答之,并贈以楛矢,使壽知我遐荒必臻也。」於是遣宏,備物以酬之。



 以石韜為太尉,與太子宣迭日省可尚書奏事。自幽州東至白狼,大興屯田。



 張駿憚季龍之盛,遣其別駕馬詵朝之。季龍初大悅,及覽其表,辭頗蹇傲,季龍大怒,將斬詵。侍中石璞進曰:「為陛下之患者,丹陽也。區區河右,焉能為有無!今斬馬詵,必征張駿,則南討之師勢分為二,建鄴君臣延其數年之命矣。勝之不為武,弗剋為四夷所笑,不如因
 而厚之。彼若改圖謝罪,率其臣職者,則我又何求!迷而不悟,討之未後也。」季龍乃止。



 李宏既至蜀漢,李壽欲誇其境內,下令云:「羯使來庭,獻其楛矢。」季龍聞之怒甚,黜王波以白衣守中書監。



 季龍志在窮兵,以其國內少馬,乃禁畜私馬,匿者腰斬,收百姓馬四萬餘匹以入于公。兼盛興宮室於鄴,起臺觀四十餘所,營長安、洛陽二宮,作者四十餘萬人。又敕河南四州具南師之備,并、朔、秦、雍嚴四討之資,青、冀、幽州三五發卒,諸州造甲者五十萬人。兼公侯牧宰競興私利,百姓失業,十室而七。船夫十七萬人為水所沒、猛獸所害,三分而一。貝丘人李弘因
 眾心之怨,自言姓名應讖,遂連結姦黨,署置百僚。事發,誅之,連坐者數千家。



 季龍畋獵無度,晨出夜歸,又多微行,躬察作役之所。侍中韋謏諫曰:「臣聞千金之子坐不垂堂,萬乘之主行不履危。陛下雖天生神武,雄據四海,乾坤冥贊,萬無所慮。然白龍魚服,有豫且之禍;海若潛游,罹葛陂之酷,深願陛下清宮蹕路,思二神為元鑒,不可忽天下之重,輕行斤斧之間。一旦有狂夫之變,龍騰之勇不暇施也,智士之計豈及設哉!又自古聖王之營建宮室,未始不於三農之隙,所以不奪農時也。今或盛功于耘藝之辰,或煩役于收獲之月,頓斃屬途,怨聲塞
 路,誠非聖君仁后所忍為也。昔漢明賢君也,鐘離一言而德陽役止。臣誠識慚昔士,言無可採,陛下道越前王,所宜哀覽。」季龍省而善之,賜以穀帛,而興繕滋繁,游察自若。



 右僕射張離領五兵尚書,專總兵要,而欲求媚於石宣,因說之曰:「今諸公侯吏兵過限,宜漸削弱,以盛儲威。」宣素疾石韜之寵,甚說其言,乃使離奏奪諸公府吏,秦、燕、義陽、樂平四公聽置吏一百九十七人,帳下兵二百人,自此以下,三分置一,餘兵五萬,悉配東宮。於是諸公咸怨,為大釁之漸矣。



 遣征北張舉自雁門討索頭郁鞠,剋之。



 制:「徵士五人車一乘,牛二頭,米各十五斛,絹十
 匹,調不辦者以斬論。」將以圖江表。於是百姓窮窘,鬻子以充軍制,猶不能赴,自經于道路死者相望,而求發無已。會青州言濟南平陵城北石獸,一夜中忽移在城東南善石溝,上有狼狐千餘迹隨之,迹皆成路。季龍大悅曰:「獸者,朕也。自平陵城北而東南者,天意將使朕平蕩江南之徵也。天命不可違,其敕諸州兵明年悉集。朕當親董六軍,以副成路之祥。」群臣皆賀,上《皇德頌》者一百七人。時妖怪尤多,石然于泰山,八日而滅。東海有大石自立,旁有血流。鄴西山石間血流出,長十餘步,廣二尺餘。太武殿畫古賢悉變為胡,旬餘,頭悉縮入肩中。季龍
 大惡之,佛圖澄對之流涕。



 寧遠劉寧攻武都狄道,陷之。使石宣討鮮卑斛穀提,大破之,斬首三萬級。



 中謁者令申扁有寵於季龍,而宣亦暱之。扁聰辯明斷,專綜機密之任。季龍既不省奏案,宣荒酒內游,石韜沈湎好獵,生殺除拜皆扁所決。於是權傾內外,刺史二千石多出其門,九卿已下望塵而拜,唯侍中鄭系、王謨、常侍盧諶、崔約等十餘人與之抗禮。



 季龍又取州郡吏馬一萬四千餘匹,以配曜武關將,馬主皆復一年。



 鎮北宇文歸執送段遼之子蘭降于季龍,獻駿馬萬匹。



 季龍以平西張伏都為使持節、都督征討諸軍事,帥步騎三萬擊涼州。既
 濟河,與張駿將謝艾大戰于河西,伏都敗績。



 季龍雖昏虐無道,而頗慕經學,遣國子博士詣洛陽寫石經,校中經於祕書。國子祭酒聶熊注《穀梁春秋》,列于學官。



 燕公石斌淫酒荒獵,常懸管而入。征北張賀度以邊防宜警,每裁諫之。斌怒,辱賀度。季龍聞之大怒,杖斌一百,遣主書禮儀持節監之。斌行意自若,儀持法呵禁,斌怒殺之。欲殺賀度,賀度嚴衛馳白之,季龍遣尚書張離持節帥騎追斌,鞭之三百,免官歸第,誅其親任十餘人。



 建元初,季龍饗群臣于太武前殿,有白鴈百餘集于馬道南。季龍命射之,無所獲。既將討三方,諸州兵至者百餘萬。太
 史令趙攬私於季龍曰:「白鴈集殿庭,宮室將空,不宜行也。」季龍納之,臨宣武觀大閱而解嚴。



 以燕公斌為使持節、侍中、大司馬、錄尚書事。置左右戎昭、曜武將軍,位在左右衛上。東宮置左右統將軍,位在四率上。置上、中光祿大夫,在左右光祿上。置鎮衛將軍,在車騎將軍上。



 時石宣淫虐日甚,而莫敢以告。領軍王朗言之於季龍曰:「今隆冬雪寒,而皇太子使人斫伐宮材,引於漳水,功役數萬,士眾吁嗟。陛下宜因游觀而罷之也。」季龍如其言。既而宣知朗所為,怒欲殺之而無因。會熒惑守房,趙攬承宣旨言於季龍曰:「昴者,趙之分也,熒惑所在,其主惡
 之。房為天子,此殃不小。宜貴臣姓王者當之。」季龍曰:「誰可當者?」攬久而對曰:「無復貴於王領軍也。」季龍既惜朗,且猜之,曰:「更言其次。」攬曰:「其次唯中書監王波耳。」季龍乃下書追波前議遣李宏及答楛矢之愆,腰斬之,及其四子投于漳水,以厭熒惑之變。尋愍波之無罪,追贈司空,封其孫為侯。



 平北尹農攻慕容皝凡城,不剋而還。黜農為庶人。



 時白虹出自太社,經鳳陽門,東南連天,十餘刻乃滅。季龍下書曰:「蓋古明王之理天下也,政以均平為首,化以仁惠為本,故能允協人和,絹熙神物。朕以眇薄,君臨萬邦,夕惕乾乾,思遵古烈,是以每下書蠲除徭
 賦,休息黎元,庶俯懷百姓,仰稟三光。而中年已來變眚彌顯,天文錯亂,時氣不應,斯由人怨于下,譴感皇天。雖朕之不明,亦群后不能翼獎之所致也。昔楚相脩政,洪災旋弭;鄭卿厲道,氛祲自消,皆服肱之良,用康群變,而群公卿士各懷道迷邦,拱默成敗,豈所望於台輔百司哉!其各上封事,極言無隱。」於是閉鳳陽門,唯元日乃開。立二畤于靈昌津,祠天及五郊。



 李壽以建寧、上庸、漢固、巴徵、梓潼五郡降于季龍。



 先是,季龍起河橋於靈昌津,採石為中濟,石無大小,下輒隨流,用功五百餘萬而不成。季龍遣使致祭,沈璧于河。俄而所沈譬流于渚上,地
 震,水波騰上,津所殿觀莫不傾壞,壓死者百餘人。季龍恚甚,斬工匠而止作焉。



 命石宣、石韜,生殺拜除皆迭日省決,不復啟也。司徒申鐘諫曰:「度賞刑威,后皇攸執,名器至重,不可以假人,皆以防姦杜漸,以示軌儀。太子國之儲貳,朝夕視膳而不及政也。庶人邃往以聞政致敗,殷鑒不遠,宜革而弗遵。且二政分權,鮮不及禍。周有子頹之釁,鄭有叔段之難,此皆由寵之不道,所以亂國害親,惟陛下覽之。」季龍不從。太子詹事孫珍問侍中崔約曰:「吾患目疾,何方療之?」約素狎珍,戲之曰:「溺中則愈。」珍曰:「目何可溺?」約曰:「卿目睕々,正耐溺中。」珍恨之,以白宣。
 宣諸子中最胡狀,目深,聞之大怒,誅約父子。珍有寵于宣,頗預朝政,自誅約之後,公卿已下憚之側目。



 季龍子義陽公鑒時鎮關中,役煩賦重,失關右之和。其友李松勸鑒,文武有長髮者,拔為冠纓,餘以給宮人。長史取髮白之,季龍大怒,以其右僕射張離為征西左長史、龍驤將軍、雍州刺史以察之,信然,徵鑒還鄴,收松下廷尉,以石苞代鎮長安。發雍、洛、秦、並州十六萬人城長安未央宮。



 季龍性既好獵,其後體重,不能跨鞍,乃造獵車千乘,轅長三丈,高一丈八尺,罝高一丈七尺,格獸車四十乘,立三級行樓二層於其上,剋期將校獵。自靈昌津南至
 滎陽,東極陽都,使御史監察,其中禽獸有犯者罪至大辟。御史因之擅作威福,百姓有美女好牛馬者,求之不得,便誣以犯獸論,死者百餘家,海岱、河濟間人無寧志矣。



 又發諸州二十六萬人脩洛陽宮。發百姓牛二萬餘頭配朔州牧官。



 增置女官二十四等,東宮十有二等,諸公侯七十餘國皆為置女官九等。先是,大發百姓女二十已下十三已上三萬餘人,為三等之第以分配之。郡縣要媚其旨,務於美淑,奪人婦者九千餘人。百姓妻有美色,豪勢因而脅之,率多自殺。石宣及諸公又私令采發者,亦垂一萬。總會鄴宮。季龍臨軒簡第諸女,大悅,封
 使者十二人皆為列侯。自初發至鄴,諸殺其夫及奪而遣之縊死者三千餘人。荊、楚、揚、徐間流叛略盡,宰守坐不能綏懷,下獄誅者五十餘人。金紫光祿大夫逯明因侍切諫,季龍大怒,遣龍騰拉而殺之。自是朝臣杜口,相招為祿仕而已。季龍常以女騎一千為鹵簿,皆著紫綸巾、熟錦褲、金銀鏤帶、五文織成靴,游于戲馬觀。觀上安詔書五色紙,在木鳳之口,鹿盧迴轉,狀若飛翔焉。



 遣涼州刺史麻秋等伐張重華。



 尚書朱軌與中黃門嚴生不協,會大雨霖,道路陷滯不通,生因而譖軌不脩道,又訕謗朝政,季龍遂殺之。於是立私論之條,偶語之律,聽吏
 告其君,奴告其主,威刑日濫,公卿已下,朝會以目,吉凶之問,自此而絕。軌之囚也,冠軍苻洪諫曰:「臣聞聖主之馭天下也,土階三尺,茅茨不翦,食不累味,刑措而不用。亡君之馭海內也,傾宮瓊榭,象箸玉杯,截脛剖心,脯賢刳孕,故其亡也忽焉。今襄國、鄴宮足康帝宇,長安、洛陽何為者哉?盤于游田,耽於女德,三代之亡恒必由此。而忽為獵車千乘,養獸萬里,奪人妻女,十萬盈宮。尚書朱軌,納言大臣,以道路不修,將加酷法,此自陛下政之失和,陰陽災沴,暴降霖雨七旬,霽方二日,縱有鬼兵百萬,尚未及脩之,而況人乎!刑政如此,其如史筆何!其如四
 海何!特願止作徒,休宮女,赦硃軌,允眾望。」季龍省之不悅,憚其強,但寢而不納,弗之罪也。乃停二京作役焉。



\end{pinyinscope}