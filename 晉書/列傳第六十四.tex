\article{列傳第六十四}

\begin{pinyinscope}

 若夫穹昊垂景,少微以躔其次;《文》《繫》探幽,貞遁以成其象。故有避於言色,其道聞乎孔公;驕乎富貴,厥義詳於孫子。是以處柔伊存,有生之恒性;在盈斯害,惟神之常道。古先智士體其若茲,介焉超俗,浩然養素,藏聲江海之上,卷迹囂氛之表,漱流而激其清,寢巢而韜其耀,良畫以符其志,絕機以虛其心。玉輝冰潔,川渟嶽峙,修至
 樂之道,固無疆之休,長往邈而不追,安排窅而無悶,修身自保,悔吝弗生,詩人《考槃》之歌,抑在茲矣。至於體天作制之后,訟息刑清之時,尚乃仄席幽貞以康神化,徵聘之禮賁於巖穴,玉帛之贄委於窒衡,故《月令》曰:「季春之月聘名士,禮賢者」,斯之謂歟!



 自典午運開,旁求隱逸,譙元彥之杜絕人事,江思悛之嘯詠林藪,峻其貞白之軌,成其出塵之迹,雖不應其嘉招,亦足激其貪競。今美其高尚之德,綴集于篇。



 孫登,字公和,汲郡共人也。無家屬,於郡北山為土窟居
 之,夏則編草為裳,冬則被髮自覆。好讀《易》,撫一絃琴,見者皆親樂之。性無恚怒,人或投諸水中,欲觀其怒,登既出,便大笑。時時游人間,所經家或設衣食者,一無所辭,去皆捨棄。嘗住宜陽山,有作炭人見之,知非常人,與語,登亦不應。文帝聞之,使阮籍往觀,既見,與語,亦不應。嵇康又從之游三年,問其所圖,終不答,康每歎息。將別,謂曰:「先生竟無言乎?」登乃曰:「子識火乎?火生而有光,而不用其光,果在於用光。人生而有才,而不用其才,而果在於用才。故用光在乎得薪,所以保其耀;用才在乎識真,所以全其年。今子才多識寡,難乎免於今之世矣!子
 無求乎?」康不能用,果遭非命,乃作《幽憤詩》曰:「昔慚柳下,今愧孫登。」或謂登以魏晉去就,易生嫌疑,故或嘿者也。竟不知所終。



 董京,字威輦,不知何郡人也。初與隴西計吏俱至洛陽,被髮而行,逍遙吟詠,常宿白社中。時乞於市,得殘碎繒絮,結以自覆,全帛佳綿則不肯受。或見推排罵辱,曾無怒色。孫楚時為著作郎,數就社中與語,遂載與俱歸,京不肯坐。楚乃貽之書,勸以今堯舜之世,胡為懷道迷邦。京答之以詩曰:「周道斁兮頌聲沒,夏政衰兮五常汨。便
 便君子,顧望而逝,洋洋乎滿目,而作者七。豈不樂天地之化也?哀哉乎時之不可與,對之以獨處。無娛我以為歡,清流可飲,至道可餐,何為棲棲,自使疲單?魚懸獸檻,鄙夫知之。夫古之至人,藏器於靈,縕袍不能令暖,軒冕不能令榮;動如川之流,靜如川之渟。鸚鵡能言,泗濱浮磬,眾人所玩,豈合物情!玄鳥紆幕,而不被害?尺隼遠巢,咸以欲死。眄彼梁魚,逡巡倒尾,沈吟不決,忽焉失水。嗟呼!魚鳥相與,萬世而不悟;以我觀之,乃明其故。焉知不有達人,深穆其度,亦將窺我,顰而去。萬物皆賤,惟人為貴,動以九州為狹,靜以環堵為大。」後數年,遁去,莫知
 所之,於其所寢處惟有一石竹子及詩二篇。其一曰:「乾道剛簡,坤體敦密,茫茫太素,是則是述。末世流奔,以文代質,悠悠世目,孰知其實!逝將去此至虛,歸我自然之室。」又曰:「孔子不遇,時彼感麟。麟乎麟!胡不遁世以存真?」



 夏統,字仲御,會稽永興人也。幼孤貧,養親以孝聞,睦於兄弟,每採梠求食,星行夜歸,或至海邊,拘螊越以資養。雅善談論。宗族勸之仕,謂之曰:「卿清亮質直,可作郡綱紀,與府朝接,自當顯至,如何甘辛苦於山林,畢性命於海濱也!」統悖然作色曰:「諸君待我乃至此乎!使統屬太
 平之時,當與元凱評議出處,遇濁代,念與屈生同汙共泥;若汙隆之間,自當耦耕沮溺,豈有辱身曲意於郡府之間乎!聞君之談,不覺寒毛盡戴,白汗四匝,顏如渥丹,心熱如炭,舌縮口張,兩耳壁塞也。」言者大慚。統自此遂不與宗族相見。



 會母疾,統侍醫藥,宗親因得見之。其從父敬寧祠先人,迎女巫章丹、陳珠二人,並有國色,莊服甚麗,善歌儛,又能隱形匿影。甲夜之初,撞鐘擊鼓,間以絲竹,丹、珠乃拔刀破舌,吞刀吐火,雲霧杳冥,流光電發。統諸從兄弟欲往觀之,難統,於是共紿之曰:「從父間疾病得瘳,大小以為喜慶,欲因其祭祀,並往賀之,卿可俱
 行乎?」統從之。入門,忽見丹、珠在中庭,輕步佪舞,靈談鬼笑,飛觸挑柈,酬酢翩翻。統驚愕而走,不由門,破籓直出。歸責諸人曰:「昔淫亂之俗興,衛文公為之悲惋;蝀蝀之氣見,君子尚不敢指;季桓納齊女,仲尼載馳而退;子路見夏南,憤恚而忼愾。吾常恨不得頓叔向之頭,陷華父之眼。奈何諸君迎此妖物,夜與游戲,放傲逸之情,縱奢淫之行,亂男女之禮,破貞高之節,何也?」遂隱床上,被髮而臥,不復言。眾親踧,即退遣丹、珠,各各分散。



 後其母病篤,乃詣洛市藥。會三月上巳,洛中王公已下並至浮橋,士女駢填,車服燭路。統時在船中曝所市藥,諸貴人
 車乘來者如雲,統並不之顧。太尉賈充怪而問之,統初不應,重問,乃徐答曰:「會稽夏仲御也。」充使問其土地風俗,統曰:「其人循循,猶有大禹之遺風,大伯之義讓,嚴遵之抗志,黃公之高節。」又問「卿居海濱,頗能隨水戲乎?」答曰:「可。」統乃操柂正櫓,折旋中流,初作鯔鳥躍,後作鯆孚引,飛鷁首,掇獸尾,奪長梢而船直逝者三焉。於是風波振駭,雲霧杳冥,俄而白魚跳入船者有八九。觀者皆悚遽,充心尤異之,乃更就船與語,其應如響,欲使之仕,即俯而不答。充又謂曰:「昔堯亦歌,舜亦歌,子與人歌而善,必反而後和之,明先聖前哲無不盡歌。卿頗能作卿土
 地間曲乎?」統曰:「先公惟寓稽山,朝會萬國,授化鄙邦,崩殂而葬。恩澤雲布,聖化猶存,百姓感詠,遂作《慕歌》。又孝女曹娥,年甫十四,貞順之德過越梁宋,其父墮江不得戶,娥仰天哀號,中流悲歎,便投水而死,父子喪尸,後乃俱出,國人哀其孝義,為歌《河女》之章。伍子胥諫吳王,言不納用,見戮投海,國人痛其忠烈,為作《小海唱》。今欲歌之。」眾人僉曰:「善。」統於是以足叩船,引聲喉囀,清激慷慨,大風應至,含水敕天,雲雨響集,叱吒讙呼,雷電晝冥,集氣長嘯,沙塵煙起。王公已下皆恐,止之乃已。諸人顧相謂曰:「若不游洛水,安見是人!聽《慕歌》之聲,便仿佛見大
 禹之容。聞《河女》之音,不覺涕淚交流,即謂伯姬高行在目前也。聆《小海》之唱,謂子胥、屈平立吾左右矣。」充欲耀以文武鹵簿,覬其來觀,因而謝之,遂命建朱旗,舉幡校,分羽騎為隊,軍伍肅然。須臾,鼓吹亂作,胡葭長鳴,車乘紛錯,縱橫馳道,又使妓女之徒服袿襡,炫金翠,繞其船三匝。統危坐如故,若無所聞。充等各散曰:「此吳兒是木人石心也。」統歸會稽,竟不知所終。



 朱沖,字巨容,南安人也。少有至行,閑靜寡欲,好學而貧,常以耕藝為事。鄰人失犢,認沖犢以歸,後得犢於林下,
 大慚,以犢還沖,沖竟不受。有牛犯其禾稼,沖屢持芻送牛而無恨色。主愧之,乃不復為暴。咸寧四年,詔補博士,沖稱疾不應。尋又詔曰:「東宮官屬亦宜得履蹈至行、敦悅典籍者,其以沖為太子右庶子。」沖每聞征書至,輒逃入深山,時人以為梁管之流。沖居近夷俗,羌戎奉之若君,沖亦以禮讓為訓,邑里化之,路不拾遺,村無凶人,毒蟲猛獸皆不為害。卒以壽終。



 范粲,字承明,陳留外黃人,漢萊蕪長丹之孫也。粲高亮貞正,有丹風,而博涉強記,學皆可師,遠近請益者甚眾,
 性不矜莊,而見之皆肅如也。魏時州府交辟,皆無所就。久之,乃應命為治中,轉別駕,辟太尉掾、尚書郎,出為征西司馬,所歷職皆有聲稱。及宣帝輔政,遷武威太守。到郡,選良吏,立學校,勸農桑。是時戎夷頗侵疆場,粲明設防備,敵不敢犯,西域流通,無烽燧之警。又郡壤富實,珍玩充積,粲檢制之,息其華侈。以母老罷官。郡既接近寇戎,粲又重鎮輒去職,朝廷尤之,左遷樂涫令。



 頃之,轉太宰從事中郎。遭母憂,以至孝稱。服闕,復為太宰中郎。齊王芳被廢,遷于金墉城,粲素服拜送,哀慟左右。時景帝輔政,召群官會議,粲又不到,朝廷以其時望,優容之。粲
 又稱疾,闔門不出。於是特詔為侍中,持節使于雍州。粲因陽狂不言,寢所乘車,足不履地。子孫恆侍左右,至有婚宦大事,輒密諮焉。合者則色無變,不合則眠寢不安,妻子以此知其旨。



 武帝踐阼,泰始中,粲同郡孫和時為太子中庶子,表薦粲,稱其操行高潔,久嬰疾病,可使郡縣輿致京師,加以聖恩,賜其醫藥,若遂瘳除,必有益於政。乃詔郡縣給醫藥,又以二千石祿養病,歲以為常,加賜帛百匹。子喬以父疾篤,辭不敢受,詔不許。以太康六年卒,時年八十四,不言三十六載,終於所寢之車。長子
 喬。



 喬字伯孫。年二歲時,祖馨臨終,撫喬首曰:「恨不見汝成人!」因以所用硯與之。至五歲,祖母以告喬,喬便執硯涕泣。九歲請學,在同輩之中,言無媟辭。弱冠,受業於樂安蔣國明。濟陰劉公榮有知人之鑒,見喬,深相器重。友人劉彥秋夙有聲譽,嘗謂人曰:「范伯孫體應純和,理思周密,吾每欲錯其一事而終不能。」光祿大夫李銓嘗論楊雄才學優於劉向,喬以為向定一代之書,正群籍之篇,使雄當之,故非所長,遂著《劉楊優劣論》,文多不載。



 喬好學不倦。父粲陽狂不言,喬與二弟並棄學業,絕人事,侍疾家庭,至粲沒,足不出邑里。司隸校尉劉毅嘗抗論於
 朝廷曰:「使范武威疾若不篤,是為伯夷、叔齊復存於今。如其信篤,益是聖主所宜哀矜。其子久侍父疾,名德著茂,不加敘用,深為朝廷惜遺賢之譏也。」元康中,詔求廉讓沖退覆道寒素者,不計資,以參選敘。尚書郎王琨乃薦喬曰:「喬稟德真粹,立操高潔,儒學精深,含章內奧,安貧樂道,棲志窮巷,簞瓢詠業,長而彌堅,誠當今之寒素,著厲俗之清彥。」時張華領司徒,天下所舉凡十七人,於喬特發優論。又吏部郎郗隆亦思求海內幽遁之士,喬供養衡門,至于白首,於是除樂安令。辭疾不拜。喬凡一舉孝廉,八薦公府,再舉清白異行,又舉寒素,一無所就。



 初,喬邑人臘夕盜斫其樹,人有告者,喬陽不聞,邑人愧而歸之。喬往喻曰:「卿節日取柴,欲與父母相歡娛耳,何以愧為!」其通物善導,皆此類也。外黃令高頵歎曰:「諸士大夫未有不及私者,而范伯孫恂恂率道,名諱未嘗經於官曹,士之貴異,於今而見。大道廢而有仁義,信矣!」其行身不穢,為物所歎服如此。以元康八年卒,年七十八。



 魯勝,字叔時,代郡人也。少有才操,為佐著作郎。元康初,遷建康令。到官,著《正天論》云:「以冬至之後立晷測影,準度日月星。臣案日月裁徑百里,無千里;星十里,不百里。」
 遂表上求下群公卿士考論。「若臣言合理,當得改先代之失,而正天地之紀。如無據驗,甘即刑戮,以彰虛妄之罪。」事遂不報。嘗歲日望氣,知將來多故,便稱疾去官。中書令張華遣子勸其更仕,再徵博士,舉中書郎,皆不就。



 其著述為世所稱,遭亂遺失,惟注《墨辯》,存其敘曰:



 名者所以別同異,明是非,道義之門,政化之準繩也。孔子曰:「必也正名,名不正則事不成。」墨子著書,作《辯經》以立名本,惠施、公孫龍祖述其學,以正別名顯於世。孟子非墨子,其辯言正辭則與墨同。荀卿、莊周等皆非毀名家,而不能易其論也。



 名必有形,察形莫如別色,故有堅白之
 辯。名必有分明,分明莫如有無,故有無序之辯。是有不是,可有不可,是名兩可。同而有異,異而有同,是之謂辯同異。至同無不同,至異無不異,是謂辯同辯異。同異生是非,是非生吉凶,取辯於一物而原極天下之汙隆,名之至也。



 自鄧析至秦時名家者,世有篇籍,率頗難知,後學莫復傳習,於今五百餘歲,遂亡絕,《墨辯》有上下《經》,《經》各有《說》,凡四篇,與其書眾篇連第,故獨存。今引說就經,各附其章,疑者闕之。又采諸眾雜集為《刑》《名》二篇,略解指歸,以俟君子。其或興微繼絕者,亦有樂乎此也!



 董養,
 字仲道,陳留浚儀人也。泰始初,到洛下,不干祿求榮。及楊后廢,養因游太學,升堂歎曰:「建斯堂也,將何為乎?每覽國家赦書,謀反大逆皆赦,至於殺祖父母、父母不赦者,以為王法所不容也。奈何公卿處議,文飾禮典,以至此乎!天人之理既滅,大亂作矣。」因著《無化論》以非之。永嘉中,洛城東北步廣里中地陷,有二鵝出焉,其蒼者飛去,白者不能飛。養聞歎曰:「昔周時所盟會狄泉,即此地也。今有二鵝,蒼者胡象,白者國家之象,其可盡言乎!顧謂謝鯤、阮孚曰:「《易》稱知機其神乎,君等可深藏矣。」乃與妻荷擔入蜀,莫知所終。



 霍原,字休明,燕國廣陽人也。少有志力,叔父坐法當死,原入獄訟之,楚毒備加,終免叔父。年十八,觀太學行禮,因留習之。貴游子弟聞而重之,欲與相見,以其名微,不欲晝往,乃夜共造焉。父友同郡劉岱將舉之,未果而病篤,臨終,敕其子沈曰:「霍原慕道清虛,方成奇器,汝後必薦之。」後歸鄉里。高陽許猛素服其名,會為幽州刺史,將詣之,主簿當車諫不可出界,猛歎恨而止。原山居積年,門徒百數,燕王月致羊酒。及劉沈為國大中正,元康中,進原為二品,司徒不過,沈乃上表理之。詔下司徒參論,
 中書監張華令陳準奏為上品,詔可。元康末,原與王褒等俱以賢良徵,累下州郡,以禮發遣,皆不到。後王浚稱制謀僭,使人問之,原不答,浚心銜之。又有遼東囚徒三百餘人,依山為賊,意欲劫原為主事,亦未行。時有謠曰:「天子在何許?近在豆田中。」浚以豆為霍,收原斬之,懸其首。諸生悲哭,夜竊尸共埋殯之。遠近駭愕,莫不冤痛之。



 郭琦,字公偉,太原晉陽人也。少方直,有雅量,博學,善五行,作《天文志》、《五行傳》,注《穀梁》、《京氏易》百卷。鄉人王游等皆就琦學。武帝欲以琦為佐著作郎,問琦族人尚書郭
 彰。彰素疾琦,答云:「不識」。帝曰:「若如卿言,烏丸家兒能事卿,即堪為郎矣。」遂決意用之。及趙王倫篡位,又欲用琦,琦曰:「我已為武帝吏,不容復為今世吏。」終身處於家。



 伍朝,字世明,武陵漢壽人也。少有雅操,閑居樂道,不修世事。性好學,以博士徵,不就。刺史劉弘薦朝為零陵太守,主者以非選例,不聽。尚書郎胡濟奏曰:「臣以為當今資喪亂之餘運,承百王之遺弊,進趨者乘國故以僥倖,守道者懷蘊櫝以終身,故令敦褒之化虧,退讓之風薄。案朝游心物外,不屑時務,守靜衡門,志道日新,年過耳
 順而所尚無虧,誠江南之奇才,丘園之逸老也。不加飾進,何以勸善!且白衣為郡,前漢有舊,宜聽光顯,以獎風尚。」奏可,而朝不就,終于家。



 魯褒,字元道,南陽人也。好學多聞,以貧素自立。元康之後,綱紀大壞,褒傷時之貪鄙,乃隱姓名,而著《錢神論》以刺之。其略曰:



 錢之為體,有乾坤之象,內則其方,外則其圓。其積如山,其流如川。動靜有時,行藏有節,市井便易,不患耗折。難折象壽,不匱象道,故能長久,為世神寶。親之如兄,字曰孔方,失之則貧弱,得之則富昌。無翼而飛,
 無足而走,解嚴毅之顏,開難發之口。錢多者處前,錢少者居後。處前者為君長,在後者為臣僕。君長者豐衍而有餘,臣僕者窮竭而不足。《詩》云:「哿矣富人,哀此煢獨。」



 錢之為言泉也,無遠不往,無幽不至。京邑衣冠,疲勞講肄,厭聞清談,對之睡寐,見我家兄,莫不驚視。錢之所祐,吉無不利,何必讀書,然後富貴!昔呂公欣悅於空版,漢祖克之於贏二,文君解布裳而被錦繡,相如乘高蓋而解犢鼻,官尊名顯,皆錢所致。空版至虛,而況有實;贏二雖少,以致親密。由此論之,謂為神物。無德而尊,無勢而熱,排金門而入紫闥。危可使安,死可使活,貴可使賤,生可
 使殺。是故忿爭非錢不勝,幽滯非錢不拔,怨仇非錢不解,令問非錢不發。



 洛中朱衣,當途之士,愛我家兄,皆我已已。執我之手,抱我終始,不計優劣,不論年紀,賓客輻輳,門常如市。諺曰:「錢無耳,可使鬼。」凡今之人,惟錢而已。故曰軍無財,士不來;軍無賞,士不往。仕無中人,不如歸田。雖有中人,而無家兄,不異無翼而欲飛,無足而欲行。



 蓋疾時者共傳其文。褒不仕,莫知其所終。



 氾騰,字無忌,敦煌人也。舉孝廉,除郎中。屬天下兵亂,去官還家。太守張閟造之,閉門不見,禮遺一無所受。歎曰:「
 生於亂世,貴而能貧,乃可以免。」散家財五十萬,以施宗族,柴門灌園,琴書自適。張軌征之為府司馬,騰曰:「門一杜,其可開乎!」固辭。病兩月餘而卒。



 任旭,字次龍,臨海章安人也。父訪,吳南海太守。旭幼孤弱,兒童時勤於學。及長,立操清修,不染流俗,鄉曲推而愛之。郡將蔣秀嘉其名,請為功曹。秀居官貪穢,每不奉法,旭正色苦諫。秀既不納,旭謝去,閉門講習,養志而已。久之,秀坐事被收,旭狼狽營送,秀慨然歎曰:「任功曹真人也。吾違其讜言,以至於此,復何言哉!」尋察孝廉,除郎
 中,州郡仍舉為郡中正,固辭歸家。永康初,惠帝博求清節俊異之士,太守仇馥薦旭清貞潔素,學識通博,詔下州郡以禮發遣。旭以朝廷多故,志尚隱遁,辭疾不行。尋天下大亂,陳敏作逆,江東名豪並見羈縶,惟旭與賀循守死不迴。敏卒不能屈。



 元帝初鎮江東,聞其名,召為參軍,手書與旭,欲使必到,旭固辭以疾。後帝進位鎮東大將軍,復召之;及為左丞相,辟為祭酒,並不就。中興建,公車徵,會遭母憂。于時司空王導啟立學校,選天下明經之士,旭與會稽虞喜俱以隱學被召。事未行,會有王敦之難,尋而帝崩,事遂寢。明帝即位,又徵拜給事中,旭稱
 疾篤,經年不到,尚書以稽留除名,僕射荀崧議以為不可。太寧末,明帝復下詔備禮徵旭,始下而帝崩。咸和二年卒,太守馮懷上疏謂宜贈九列值蘇峻作亂,事竟不行。



 子琚,位至大宗正,終于家。



 郭文,字文舉,河內軹人也。少愛山水,尚嘉遁。年三十,每游山林,彌旬忘反。父母終,服畢,不娶,辭家游名山,歷華陰之崖,以觀石室之石函。洛陽陷,乃步擔入吳興餘杭大闢山中窮谷無人之地,倚木於樹,苫覆其上而居焉,亦無壁障。時猛獸為暴,入屋害人,而文獨宿十餘年,卒
 無患害。恒著鹿裘葛巾,不飲酒食肉,區種菽麥,採竹葉木實,貿鹽以自供。人或酬下價者,亦即與之。後人識文,不復賤酬。食有餘穀,輒恤窮匱。人有臻遺,取其粗者,示不逆而已。有猛獸殺大麀鹿於庵側,文以語人,人取賣之,分錢與文。文曰:「我若須此,自當賣之。所以相語,正以不須故也。」聞者皆嗟嘆之。嘗有猛獸忽張口向文,文視其口中有橫骨,乃以手探去之,猛獸明旦致一鹿於其室前。獵者時往寄宿,文夜為擔水而無倦色。餘杭令顧颺與葛洪共造之,而攜與俱歸。颺以文山行或須皮衣,贈以韋褲褶一具,文不納,辭歸山中。颺追遣使者置衣
 室中而去,文亦無言,韋衣乃至爛於戶內,竟不服用。



 王導聞其名,遣人迎之,文不肯就船車,荷擔徒行。既至,導置之西園,園中果木成林,又有鳥獸麋鹿,因以居文焉。於是朝士咸共觀之,文頹然踑踞,傍若無人。溫嶠嘗問文曰:「人皆有六親相娛,先生棄之何樂?」文曰:「本行學道,不謂遭世亂,欲歸無路,是以來也。」又問曰:「饑而思食,壯而思室,自然之性,先生安獨無情乎?」文曰:「情由憶生,不憶故無情。」又問曰:「先生獨處窮山,若疾病遭命,則為烏鳥所食,顧不酷乎?」文曰:「藏埋者亦為螻蟻所食,復何異乎!」又問曰:「猛獸害人,人之所畏,而先生獨不畏邪?」文曰:「
 人無害獸之心,則獸亦不害人。」又問曰:「茍世不寧,身不得安。今將用先生以濟時,若何?」文曰:「山草之人,安能佐世!」導嘗眾客共集,絲竹並奏,試使呼之。文瞪眸不轉,跨躡華堂如行林野。于時坐者咸有鉤深味遠之言,文常稱不達來語。天機鏗宏,莫有窺其門者。溫嶠嘗稱曰:「文有賢人之性,而無賢人之才,柳下、梁踦之亞乎!」永昌中,大疫,文病亦殆。王導遺藥,文曰:「命在天,不在藥也。夭壽長短,時也。」



 居導園七年,未嘗出入。一旦忽求還山,導不聽。後逃歸臨安,結廬舍於山中。臨安令萬寵迎置縣中。及蘇峻反,破餘杭,而臨安獨全,人皆異之,以為知機。自
 後不復語,但舉手指麾,以宣其意。病甚,求還山,欲枕石安尸,不令人殯葬,寵不聽。不食二十餘日,亦不瘦。寵問曰:「先生復可得幾日?」文三舉手,果以十五日終。寵葬之於所居之處而祭哭之,葛洪、庾闡並為作傳,贊頌其美云。



 龔壯,字子瑋,巴西人也。潔己自守,與鄉人譙秀齊名。父叔為李特所害,壯積年不除喪,力弱不能復仇。及李壽戍漢中,與李期有嫌,期,特孫也,壯欲假壽以報,乃說壽曰:「節下若能並有西土,稱籓於晉,人必樂從。且捨小就
 大,以危易安,莫大之策也。」壽然之,遂率眾討期,果剋之。壽猶襲偽號,欲官之,壯誓不仕,賂遺一無所取。會天久雨,百姓饑墊,壯上書說壽以歸順,允天心,應人望,永為國籓,福流子孫。壽省書內愧,祕而不宣。乃遣使入胡,壯又諫之,壽又不納。壯謂百行之本莫大忠孝,即假壽殺期,私仇以雪,又欲使其歸朝,以明臣節。壽既不從,壯遂稱聾,又云手不制物,終身不復至成都,惟研考經典,譚思文章,至李勢時卒。



 初,壯每歎中夏多經學,而巴蜀鄙陋,兼遭李氏之難,無復學徒,乃著《邁德論》,文多不載。



 孟陋,
 字少孤,武昌人也。吳司空宗之曾孫也。兄嘉,桓溫征西長史。陋少而貞立,清操絕倫,布衣蔬食,以文籍自娛。口不及世事,未曾交游,時或弋釣,孤興獨往,雖家人亦不知其所之也。喪母,毀瘠殆於滅性,不飲酒食肉十有餘年。親族迭謂之曰:「少孤!誰無父母?誰有父母!聖人制禮,令賢者俯就,不肖企及。若使毀性無嗣,更為不孝也。陋感此言,然後從吉。由是名著海內。簡文帝輔政,命為參軍,稱疾不起。桓溫躬往造焉。或謂溫曰:「孟陋高行,學為儒宗,宜引在府,以和鼎味。」溫歎曰:「會稽王尚不能屈,非敢擬議也。」陋聞之曰:「桓公正當以我不往故耳。億
 兆之人,無官者十居其九,豈皆高士哉!我疾病不堪恭相王之命,非敢為高也。」由是名稱益重。博學多通,長於《三禮》。註《論語》,行於世。卒以壽終。



 韓績,字興齊,廣陵人也。其先避亂,居於吳之嘉興。父建,仕吳至大鴻臚。績少好文學,以潛退為操,布衣蔬食,不交當世,由是東土並宗敬焉。司徒王導聞其名,辟以為掾,不就。咸康末,會稽內史孔愉上疏薦之,詔以安車束帛征之。尚書令諸葛恢奏績名望猶輕,未宜備禮,於是召拜博士。稱老病不起,卒於家。



 於時高密劉鮞字長魚、
 城陽邴郁字弘文,並有高名。鮞幼不慕俗,長而希古,篤學厲行,化流邦邑。郁,魏徵士原之曾孫,少有原風,敕身謹潔,口不妄說,耳不妄聽,端拱恂恂,舉動有禮。咸康中,成帝博求異行之士,鮞、郁並被公卿薦舉,於是依績及翟湯等例,以博士徵之。郁辭以疾,鮞隨使者到京師,自陳年老,不拜。各以壽終。



 譙秀,字元彥,巴西人也。祖周,以儒學著稱,顯明蜀朝。秀少而靜默,不交於世,知天下將亂,預絕人事,雖內外宗親,不與相見。郡察孝廉,州舉秀才,皆不就。及李雄據蜀,
 略有巴西,雄叔父驤、驤子壽皆慕秀名,具束帛安車徵之,皆不應。常冠皮弁,弊衣,躬耕山藪。龔壯常歎服焉。桓溫滅蜀,上疏薦之,朝廷以秀年在篤老,兼道遠,故不征,遣使敕所在四時存問。尋而范賁、蕭敬相繼作亂,秀避難宕渠,鄉里宗族依憑之者以百數。秀年出八十,眾人欲代之負擔,秀曰:「各有老弱,當先營護。吾氣力猶足自堪,豈以垂朽之年累諸君也!」年九十餘卒。



 翟湯,字道深,尋陽人。篤行純素,仁讓廉潔,不屑世事,耕而後食,人有餽贈,雖釜庾一無所受。永嘉末,寇害相繼,
 聞湯名德,皆不敢犯,鄉人賴之。司徒王導辟,不就,隱於縣界南山。始安太守干寶與湯通家,遣船餉之,敕吏云:「翟公廉讓,卿致書訖,便委船還。」湯無人反致,乃貨易絹物,因寄還寶。寶本以為惠,而更煩之,益愧歎焉。咸康中,征西大將軍庾亮上疏薦之,成帝徵為國子博士,湯不起。建元初,安西將軍庾翼北征石季龍,大發僮客以充戎役,敕有司特蠲湯所調。湯悉推僕使委之鄉吏,吏奉旨一無所受,湯依所調限,放免其僕,使令編戶為百姓。康帝復以散騎常侍徵湯,固辭老疾,不至。年七十三,卒於家。



 子
 莊,字祖休。少以孝友著名,遵湯之操,不交人物,耕而後食,語不及俗,惟以弋釣為事。及長,不復獵。或問:「漁獵同是害生之事,而先生止去其一,何哉?」莊曰:「獵自我,釣自物,未能頓盡,故先節其甚者。且夫貪餌吞鉤,豈我哉!」時人以為知言。晚節亦不復釣,端居篳門,歠菽飲水。州府禮命,及公車徵,並不就。年五十六,卒。子矯,亦有高操,屢辭辟命。矯子法賜,孝武帝以散騎郎征,亦不至。世有隱行云。



 郭翻,字長翔,武昌人也。伯父訥,廣州刺史。父察,安城太
 守。翻少有志操,辭州郡辟及賢良之舉。家于臨川,不交世事,惟以漁釣射獵為娛。居貧無業,欲墾荒田,先立表題,經年無主,然後乃作。稻將熟,有認之者,悉推與之。縣令聞而詰之,以稻還翻,翻遂不受。嘗以車獵,去家百餘里,道中逢病人,以車送之,徒步而歸。其漁獵所得,或從買者,便與之而不取直,亦不告姓名。由是士庶咸敬貴焉。與翟湯俱為庾亮所薦,公車博士徵,不就。咸康末,乘小船暫歸武昌省墳墓,安西將軍庾翼以帝舅之重,躬往造翻,欲強起之。翻曰:「人性各有所短,焉可彊逼!」翼又以其船小狹,欲引就大船。翻曰:「使君不以鄙賤而辱臨
 之,此固野人之舟也。」翼俯屈入其船中,終日而去。嘗墜刀於水,路人有為取者,因與之。路人不取,固辭,翻曰:「爾向不取,我豈能得!」路人曰:「我若取此,將為天地鬼神所責矣。」翻知其終不受,復沈刀於水。路人悵焉,乃復沈沒取之。翻於是不逆其意,乃以十倍刀價與之。其廉不受惠,皆此類也。卒于家。



 辛謐,字叔重,隴西狄道人也。父怡,幽州刺史,世稱冠族。謐少有志尚,博學善屬文,工草隸書,為時楷法。性恬靜,不妄交游。召拜太子舍人、諸王文學,累徵不起。永嘉末,
 以謐兼散騎常侍,慰撫關中。謐以洛陽將敗,故應之。及長安陷沒于劉聰,聰拜太中大夫,固辭不受。又歷石勒、季龍之世,並不應辟命。雖處喪亂之中,頹然高邁,視榮利蔑如也。及冉閔僭號,復備禮徵為太常,謐遺閔書曰:「昔許由辭堯,以天下讓之,全其清高之節。伯夷去國,子推逃賞,皆顯史牒,傳之無窮。此往而不反者也。然賢人君子雖居廟堂之上,無異於山林之中,期窮理盡性之妙,豈有識之者邪!是故不嬰於禍難者,非為避之,但冥心至趣而與吉會耳。謐聞物極則變,冬夏是也;致高則危,累棋是也。君王功以成矣,而久處之,非所以顧萬全
 遠危亡之禍也。宜因茲大捷,歸身本朝,必有許由、伯夷之廉,享松喬之壽,永為世輔,豈不美哉!」因不食而卒。



 劉驎之,字子驥,南陽人,光祿大夫耽之族也。驎之少尚質素,虛退寡欲,不修儀操,人莫之知。好游山澤,志存遁逸。嘗採藥至衡山,深入忘反,見有一澗水,水南有二石囷,一囷閉,一囷開,水深廣不得過。欲還,失道,遇伐弓人,問徑,僅得還家。或說囷中皆仙靈方藥諸雜物,驎之欲更尋索,終不復知處也。車騎將軍桓沖聞其名,請為長史,驎之固辭不受。沖嘗到其家,驎之於樹條桑,使者致
 命,驎之曰:「使君既枉駕光臨,宜先詣家君。」沖聞大愧,於是乃造其父。父命驎之,然後方還,拂短褐與沖言話。父使驎之於內自持濁酒蔬菜供賓,沖敕人代驎之斟酌,父辭曰:「若使從者,非野人之意也。」沖慨然,至昏乃退。驎之雖冠冕之族,信儀著於群小,凡廝伍之家婚娶葬送,無不躬自造焉。居於陽岐,在官道之側,人物來往,莫不投之。驎之躬自供給,士君子頗以勞累,更憚過焉。凡人致贈,一無所受。去驎之家百餘里,有一孤姥,病將死,歎息謂人曰:「誰當埋我,惟有劉長史耳!何由令知。」驎之先聞其有患,故往侯之,值其命終,乃身為營棺殯送之。其
 仁愛隱惻若此。卒以壽終。



 索襲,字偉祖,敦煌人也。虛靖好學,不應州郡之命,舉孝廉、賢良方正,皆以疾辭。游思於陰陽之術,著天文地理十餘篇,多所啟發。不與當世交通,或獨語獨笑,或長歎涕泣,或請問不言。張茂時,敦煌太守陰澹奇而造焉,經日忘反,出而歎曰:「索先生碩德名儒,真可以諮大義。」澹欲行鄉射之禮,請襲為三老,曰:「今四表輯寧,將行鄉射之禮,先生年耆望重,道冠一時,養老之義,實繫儒賢。既樹非梧桐,而希鸞鳳降翼;器謝曹公,而冀蓋公枉駕,誠
 非所謂也。然夫子至聖,有召赴焉;孟軻大德,無聘不至,蓋欲弘闡大猷,敷明道化故也。今之相屈,遵道崇教,非有爵位,意者或可然乎!」會病卒,時年七十九。澹素服會葬,贈賤二萬。澹曰:「世人之所有餘者,富貴也;目之所好者,五色也;耳之所玩者,五音也。而先生棄眾人之所收,收眾人之所棄,味無味於慌惚之際,兼重玄於眾妙之內。宅不彌畝而志忽九州,形居塵俗而棲心天外,雖黔婁之高遠,莊生之不願,蔑以過也。」乃謚曰玄居先生。



 楊軻,天水人也。少好《易》,長而不娶,學業精微,養徒數百,
 常食粗飲水,衣褐縕袍,人不堪其憂,而軻悠然自得,疏賓異客,音旨未曾交也。雖受業門徒,非入室弟子,莫昨親言。欲所論授,須旁無雜人,授入室弟子,令遞相宣授。劉曜僭號,徵拜太常,軻固辭不起,曜亦敬而不逼,遂隱于隴山。曜後為石勒所擒,秦人東徙,軻留長安。及石季龍嗣偽位。備玄纁束帛安車徵之,軻以疾辭。迫之,乃發。既見季龍,不拜,與語,不言,命舍之於永昌乙第。其有司以軻倨傲,請從大不敬論,季龍不從,下書任軻所尚。軻在永昌,季龍每有饋餼,輒口授弟子,使為表謝,其文甚美,覽者歎有深致。季龍欲觀其真趣,乃密令美女夜以
 動之,軻蕭然不顧。又使人將其弟子盡行,遣魁壯羯士衣甲持刀,臨之以兵,并竊其所賜衣服而去,軻視而不信,了無懼色。常臥土床,覆以布被,惈寢其中,下無茵褥。潁川荀鋪,好奇之士也,造而談經,軻瞑目不答。鋪發軻被露其形,大笑之。軻神體頹然,無驚怒之狀。于時咸以為焦先之徒,未有能量其深淺也。後上疏陳鄉思,求還,季龍送以安車蒲輪,蠲十戶供之。自歸秦州,仍教授不絕。其後秦人西奔涼州,軻弟子以牛負之,為戍軍追擒,並為所害。



 公孫鳳,字子鸞,上谷人也。隱於昌黎之九城山谷,冬衣單布,寢處士床,夏則並食於器,停令臭敗,然後食之。彈琴吟詠,陶然自得,人咸異之,莫能測也。慕容以安車徵至鄴,及見,不言不拜,衣食舉動如在九城。賓客造請,鮮得與言。數年病卒。



 公孫永,字子陽,襄平人也。少而好學恬虛,隱于平郭南山,不娶妻妾,非身所墾植,則不衣食之,吟詠巖間,欣然自得,年餘九十,操尚不虧。與公孫鳳俱被容徵至鄴,及見,不拜,王公以下造之,皆不與言,雖經隆冬盛
 暑,端然自若。一歲餘,詐狂,送之平郭。後苻堅又將備禮徵之,難其年耆路遠,乃遣使者致問。未至而永亡,堅深悼之,謚曰崇虛先生。



 張忠,字巨和,中山人也。永嘉之亂,隱于泰山。恬靜寡欲,清虛服氣,餐芝餌石,修導養之法。冬則縕袍,夏則帶索,端拱若尸。無琴書之適,不修經典,勸教但以至道虛無為宗。其居依崇巖幽谷,鑿地為窟室。弟子亦以窟居,去忠六十餘步,五日一朝。其教以形不以言,弟子受業,觀形而退。立道壇於窟上,每旦朝拜之。食用瓦器,鑿石為
 釜。左右居人饋之衣食,一無所受。好事少年頗或問以水旱之祥,忠曰:「天不言而四時行焉,萬物生焉,陰陽之事非窮山野叟所能知之。」其遣諸外物,皆此類也。年在期頤,而視聽無爽。苻堅遣使徵之。使者至,忠沐浴而起,謂弟子曰:「吾餘年無幾,不可以逆時主之意。」浴訖就車。及至長安,堅賜以冠衣,辭曰:「年朽髮落,不堪衣冠,請以野服入覲。」從之。及見,堅謂之曰:「先生考磐山林,研精道素,獨善之美有餘,兼濟之功未也。故遠屈先生,將任齊尚父。」忠曰:「昔因喪亂,避地泰山,與鳥獸為侶,以全朝夕之命。屬堯舜之世,思一奉聖顏。年衰志謝,不堪展效,尚
 父之況,非敢竊擬。山棲之性,情存巖岫,乞還餘齒,歸死岱宗。堅以安車送之。行達華山。歎曰:「我東嶽道士,沒於西嶽,命也,奈何!行五十里,及關而死。使者馳驛白之,堅遣黃門郎韋華持節策弔,祀以太牢,褒賜命服,謚曰安道先生。



 石垣,字洪孫,自云北海劇人。居無定所,不娶妻妾,不營產業,食不求美,衣必粗弊。或有遺其衣服,受而施人。人有喪葬,輒杖策弔之。路無遠近,時有寒暑,必在其中;或同日共時,咸皆見焉。又能闇中取物,如晝無差。姚萇之
 亂,莫知所終。



 宋纖,字令艾,敦煌效穀人也。少有遠操,沈靖不與世交,隱居于酒泉南山。明究經緯,弟子受業三千餘人。不應州郡辟命,惟與陰顒、齊好友善。張祚時,太守楊宣畫其象於閣上;出入視之,作頌曰:「為枕何石?為瀨何流?身不可見,名不可求。」酒泉太守馬岌,高尚之士也,具威儀,鳴鐃鼓,造焉。纖高樓重閣,距而不見。岌歎曰:「名可聞而身不可見,德可仰而形不可睹,吾而今而後知先生人中之龍也。」銘詩於石壁曰:「丹崖百丈,青壁萬尋。奇木蓊鬱,
 蔚若鄧林。其人如玉,維國之琛。室邇人遐,實勞我心。」



 纖注《論語》,及為詩頌數萬言。年八十,篤學不倦。張祚後遣使者張興備禮徵為太子友,興逼喻甚切,纖喟然歎曰:「德非莊生,才非干木,何取稽停明命!」遂隨興至姑臧。祚遣其太子太和以執友禮造之,纖稱疾不見,贈遺一皆不受。尋遷太子太傅。頃之,上疏曰:「臣受生方外,心慕太古。生不喜存,死不悲沒。素有遺屬,屬諸知識,在山投山,臨水投水,處澤露形,在人親土。聲聞書疏,勿告我家。今當命終,乞如素願。」遂不食而卒,時年八十二,謚曰玄虛先生。



 郭荷,字承休,略陽人也。六世祖整,漢安順之世,公府八辟,公車五徵,皆不就。自整及荷,世以經學致位。荷明究群籍,特善史書。不應州郡之命。張祚遣使者以安車束帛徵為博士祭酒,使者迫而致之。及至,署太子友。荷上疏乞還,祚許之,遣以安車蒲輪送還張掖東山。年八十四卒,謚曰玄德先生。



 郭瑀字元瑜,敦煌人也。少有超俗之操,東游張掖,師事郭荷,盡傳其業。精通經義,雅辯談論,多才藝,善屬文。荷
 卒,瑀以為父生之,師成之,君爵之,而五服之制,師不服重,蓋聖人謙也,遂服斬衰,廬墓三年。禮畢,隱于臨松薤谷,鑿石窟而居,服柏實以輕身,作《春秋墨說》、《孝經錯緯》,弟子著錄千餘人。



 張天賜遣使者孟公明持節,以浦輪玄纁備禮徵之,遺瑀書曰:「先生潛光九皋,懷真獨遠,心與至境冥符,志與四時消息,豈知蒼生倒懸,四海待拯者乎!孤忝承時運,負荷大業,思與賢明同贊帝道。昔傳說龍翔殷朝,尚父鷹揚周室,孔聖車不停軌,墨子駕不俟旦,皆以黔首之禍不可以不救,君不獨立,道由人弘故也。況今九服分為狄場,二都盡為戎穴,天子僻陋江
 東,名教淪於左衽,創毒之甚,開避未聞。先生懷濟世之才,坐觀而不救,其於仁智,孤竊惑焉。故遣使者虛左授綏,鶴企先生,乃眷下國。」公明至山,瑀指翔鴻以示之曰:「此鳥也,安可籠哉!」遂深逃絕迹。公明拘其門人,瑀歎曰:「吾逃祿,非避罪也,豈得隱居行義,害及門人!」乃出而就徵。及至姑臧,值天賜母卒,瑀括髮入弔,三踴而出,還于南山。



 及天錫滅,苻堅又以安車徵瑀定禮儀,會父喪而止,太守辛章遣書生三百人就受業焉。及苻氏之末,略陽王穆起兵酒泉,以應張大豫,遣使招瑀。瑀歎曰:「臨河救溺,不卜命之短長;脈病三年,不豫絕其餐饋;魯連在
 趙,義不結舌,況人將左衽而不救之!」乃與敦煌索嘏起兵五千,運粟三萬石,東應王穆。穆以瑀為太府左長史、軍師將軍。雖居元佐,而口詠黃老,冀功成世定,追伯成之蹤。



 穆惑於讒間,西伐索嘏,瑀諫曰:「昔漢定天下,然後誅功臣。今事業未建而誅之,立見麋鹿游于此庭矣。」穆不從。瑀出城大哭,舉手謝城曰:「吾不復見汝矣!」還而引被覆面,不與人言,不食七日,與疾而歸,旦夕祈死。夜夢乘青龍上天,至屋而止,寤而嘆曰:「龍飛在天,今止於屋。屋之為字,尸下至也。龍飛至尸,吾其死也。古之君子不卒內寢,況吾正士乎!」遂還酒泉南山赤崖閣,飲氣而卒。



 祈嘉,字孔賓,酒泉人也。少清貧,好學。年二十餘,夜忽窗中有聲呼曰:「祈孔賓,祈孔賓!隱去來,隱去來!修飾人世,甚苦不可諧。所得未毛銖,所喪如山崖。」旦而逃去,西至敦煌,依學官誦書,貧無衣食,為書生都養以自給,遂博通經傳,精究大義。西游海渚,教授門生百餘人。張重華徵為儒林祭酒。性和裕,教授不倦,依《孝經》作《二九神經》。在朝卿士、郡縣守令彭和正等受業獨拜床下者二千餘人,天錫謂為先生而不名之。竟以壽終。



 瞿硎先生者,不得姓名,亦不知何許人也。太和末,常居宣城郡界文脊山中,山有瞿硎,因以為名焉。大司馬桓溫嘗往造之。既至,見先生被鹿裘,坐于石室,神無忤色,溫及僚佐數十人皆莫測之,乃命伏滔為之銘贊。竟卒於山中。



 謝敷,字慶緒,會稽人也。性澄靖寡欲,入太平山十餘年。鎮軍郗愔召為主簿,臺徵博士,皆不就。初,月犯少微,少微一名處士星,占者以陷士當之。譙國戴逵有美才,人或憂之。俄而敷死,故會稽人士以嘲吳人云:「吳中高士,
 便是求死不得死。」



 戴逵,字安道,譙國人也。少博學,好談論,善屬文,能鼓琴,工書畫,其餘巧藝靡不畢綜。總角時,以雞卵汁溲白瓦屑作《鄭玄碑》,又為文而自鐫之,詞麗器妙,時人莫不驚歎。性不樂當世,常以琴書自娛。師事術士范宣於豫章,宣異之,以兄女妻焉。太宰、武陵王晞聞其善鼓琴,使人召之,逵對使者破琴曰:「戴安道不為王門伶人!」晞怒,乃更引其兄述。述聞命欣然,擁琴而往。



 逵後徙居會稽之剡縣。性高潔,常以禮度自處,深以放達為非道,乃著論
 曰:



 夫親沒而採藥不反者,不仁之子也;君危而屢出近關者,茍免之臣也。而古之人未始以彼害名教之體者何?達其旨故也。達其旨,故不惑其迹。若元康之人,可謂好遁跡而不求其本,故有捐本徇末之弊,舍實逐聲之行,是猶美西施而學其顰眉,慕有道而折其巾角,所以為慕者,非其所以為美,徒貴貌似而已矣。夫紫之亂朱,以其似朱也。故鄉原似中和,所以亂德;放者似達,所以亂道。然竹林之為放,有疾而為顰者也,元康之為放,無德而折巾者也,可無察乎!



 且儒家尚譽者,本以興賢也,既失其本,則有色取之行。懷情喪真,以容貌相欺,其
 弊必至於末偽。道家去名者,欲以篤實也,茍失其本,又有越檢之行。情禮俱虧,則仰詠兼忘,其弊必至於本薄。夫偽薄者,非二本之失,而為弊者必託二本以自通。夫道有常經而弊無常情,是以六經有失,王政有弊,茍乖其本,固聖賢所無奈何也。



 嗟夫!行道之人自非性足體備、闇蹈而當者,亦曷能不棲情古烈,擬規前修。茍迷擬之然後動,議之然後言,固當先辯其趣舍之極,求其用心之本,識其枉尺直尋之旨,採其被褐懷玉之由。若斯,途雖殊,而其歸可觀也;跡雖亂,而其契不乖也。不然,則流遁忘反,為風波之行,自驅以物,自誑以偽,外眩囂華,
 內喪道實,以矜尚奪其真主,以塵垢翳其天正,貽笑千載,可不慎歟!



 孝武帝時,以散騎常侍、國子博士累徵,辭父疾不就。郡縣敦逼不已,乃逃于吳。吳國內史王珣有別館在武丘山,逵潛詣之,與珣游處積旬。會稽內史謝玄慮逵遠遁不反,乃上疏曰:「伏見譙國戴逵希心俗表,不嬰世務,棲遲衡門,與琴書為友。雖策命屢加,幽操不回,超然絕跡,自求其志。且年垂耳順,常抱羸疾,時或失適,轉至委篤。今王命未回,將離風霜之患。陛下既已愛而器之,亦宜使其身名並存,請絕其召命。」疏奏,帝許之,逵復還剡。



 後王珣為尚書僕射,上疏復請徵為國子祭
 酒,加散騎常侍,征之,復不至。太元二十年,皇太子始出東宮,太子太傳會稽王道子、少傅王雅、詹事王珣又上疏曰:「逵執操貞厲,含味獨游,年在耆老,清風彌劭。東宮虛德,式延事外,宜加旌命,以參僚侍。逵既重幽居之操,必以難進為美,宜下所在備禮發遣。」會病卒。



 長子勃,有父風。義熙初,以散騎侍郎征,不起,尋卒。



 龔玄之,字道玄,武陵漢壽人也。父登,歷長沙相、散騎常侍。玄之好學潛默,安於陋巷。州舉秀才,公府辟,不就。孝武帝下詔曰:「夫哲王御世,必搜揚幽隱,故空谷流縶維
 之詠,丘園旅束帛之觀。譙國戴逵、武陵龔玄之並高尚其操,依仁游藝,潔己貞鮮,學弘儒業,朕虛懷久矣。二三君子,豈其戢賢於懷抱哉!思挹雅言,虛誠諷議,可並以為散騎常侍,領國子博士,指下所在備禮發遣,不得循常,以稽側席之望。」郡縣敦逼,苦辭疾篤,不行。尋卒,時年五十八。



 弟子元壽,亦有德操,高尚不仕,舉秀才及州辟召,並稱疾不就。孝武帝以太學博士、散騎侍郎、給事中累徵,遂不起。卒於家。



 陶淡,字處靜,太尉侃之孫也。父夏,以無行被廢。淡幼孤,
 好導養之術,謂仙道可祈。年十五六,便服食絕穀,不婚娶。家累千金,僮客百數,淡終日端拱,曾不營問。頗好讀《易》善卜筮。於長沙臨湘山中結廬居之,養一白鹿以自偶。親故有候之者,輒移渡澗水,莫得近之。州舉秀才,淡聞,遂轉逃羅縣埤山中,終身不返,莫知所終。



 陶潛,字元亮,大司馬侃之曾孫也。祖茂,武昌太守。潛少懷高尚,博學善屬文,穎脫不羈,任真自得,為鄉鄰之所貴。嘗著《五柳先生傳》以自況曰:「先生不知何許人,不詳姓字,宅邊有五柳樹,因以為號焉。閑靜少言,不慕榮利。
 好讀書,不求甚解,每有會意,欣然忘食。性嗜酒,而家貧不能恒得。親舊知其如此,或置酒招之,造飲必盡,期在必醉。既醉而退,曾不吝情。環堵蕭然,不蔽風日,短褐穿結,簞瓢屢空,晏如也。常著文章自娛,頗示己志,忘懷得失,以此自終。」其自序如此,時人謂之實錄。



 以親老家貧,起為州祭酒,不堪吏職,少日自解歸。州召主簿,不就,躬耕自資,遂抱羸疾。復為鎮軍、建威參軍,謂親朋曰:「聊欲絃歌,以為三徑之資可乎?」執事者聞之,以為彭澤令。在縣,公田悉令種秫穀,曰:「令吾常醉於酒足矣。」妻子固請種粳。乃使一頃五十畝種秫,五十畝種粳。素簡貴,不私
 事上官。郡遣督郵至縣,吏白應束帶見之,潛歎曰:「吾不能為五斗米折腰,拳拳事鄉里小人邪!」義熙二年,解印去縣,乃賦《歸去來》。其辭曰:



 歸去來兮,田園將蕪胡不歸?既自以心為形役,奚惆悵而獨悲?悟已往之不諫,知來者之可追。實迷途其未遠,覺今是而昨非。舟遙遙以輕颺,風飄飄而吹衣,問征夫以前路,恨晨光之希微。乃瞻衡宇,載欣載奔。僮僕來迎,稚子侯門。三徑就荒,松菊猶存。攜幼入室,有酒盈樽。引壺觚以自酌,眄庭柯以怡顏,倚南窗以寄傲,審容膝之易安。園日涉而成趣,門雖設而常關;策扶老而流憩,時翹首而遐觀。雲無心而出岫,
 鳥倦飛而知還;景翳翳其將入,撫孤松而盤桓。



 歸去來兮,請息交以絕游,世與我而相遺,復駕言兮焉求!悅親戚之情話,樂琴書以消憂。農人告余以春暮,將有事乎西疇。或命巾車,或棹孤舟,既窈窕以尋壑,亦崎嶇而經丘。木欣欣以向榮,泉涓涓而始流,善萬物之得時,感吾生之行休。



 已矣乎!寓形宇內復幾時,曷不委心任去留,胡為乎遑遑欲何之?富貴非吾願,帝鄉不可期。懷良晨以孤往,或植杖而芸秄,登東皋以舒嘯,臨清流而賦詩;聊乘化而歸盡,樂夫天命復奚疑!



 頃之,徵著作郎,不就。既絕州郡覲謁,其鄉親張野及周旋人羊松齡、寵遵等
 或有酒要之,或要之共至酒坐,雖不識主人,亦欣然無忤,酣醉便反。未嘗有所造詣,所之唯至田舍及廬山游觀而已。



 刺史王弘以元熙中臨州,甚欽遲之,後自造焉。潛稱疾不見,既而語人云:「我性不狎世,因疾守閑,幸非潔志慕聲,豈敢以王公紆軫為榮邪!夫謬以不賢,此劉公幹所以招謗君子,其罪不細也。」弘每令人候之,密知當往廬山,乃遣其故人龐通之等齎酒,先於半道要之。潛既遇酒,便引酌野亭,欣然忘進。弘乃出與相見,遂歡宴窮日。潛無履,弘顧左右為之造履。左右請履度,潛便於坐申腳令度焉。弘要之還州,問其所乘,答云:「素有腳
 疾,向乘藍輿,亦足自反。」乃令一門生二兒共轝之至州,而言笑賞適,不覺其有羨於華軒也。弘後欲見,輒於林澤間候之。至於酒米乏絕,亦時相贍。



 其親朋好事,或載酒肴而往,潛亦無所辭焉。每一醉,則大適融然。又不營生業,家務悉委之兒僕。未嘗有喜慍之色,惟遇酒則飲,時或無酒,亦雅詠不輟。嘗言夏月虛閒,高臥北窗之下,清風颯至,自謂羲皇上人。性不解音,而畜素琴一張,絃徽不具,每朋酒之會,則撫而和之,曰:「但識琴中趣,何勞絃上聲!」以宋元嘉中卒,時年六十三,所有文集並行於世。



 史臣曰:君子之行殊途,顯晦之謂也。出則允釐庶政,以
 道濟時;處則振拔囂埃,以卑自牧。詳求厥義,其來夐矣。公和之居窟室,裳唯編草,誡叔夜而凝神鑒;威輦之處叢祠,衣無全帛,對子荊而陳貞則:並滅景而弗追,柳禽、尚平之流亞。夏統遠邇稱其孝友,宗黨高其諒直,歌《小海》之曲。則伍胥猶存;固貞石之心,則公閭尤愧,時幸洛濱之觀,信乎茲言。宋纖幼懷遠操,清規映拔,楊宣頌其畫象,馬岌歎其人龍,玄虛之號,實期為美。餘之數子,或移病而去官,或著論而矯俗,或箕踞而對時人,或弋釣而棲衡泌,含和隱璞,乘道匿輝,不屈其志,激清風於來葉者矣。



 贊曰:厚秩招累,修名順欲。確乎群士,超然絕俗。養粹巖阿,銷聲林曲。激貪止競,永垂高躅。



\end{pinyinscope}