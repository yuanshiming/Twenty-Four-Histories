\article{志第十九}

\begin{pinyinscope}

 五行下



 《傳》曰:「聽之不聰,是謂不謀,厥咎急,厥罰恒寒,厥極貧。時則有鼓妖,時則有魚孽,時則有豕禍,時則有耳痾,時則有黑眚黑祥。惟火沴水。」聽之不聰,是謂不謀,言上偏聽不聰,下情隔塞,則謀慮利害,失在嚴急,故其咎急也。盛冬日短,寒以殺物,政促迫,故其罰常寒也。寒則不生百穀,上下俱貧,故其極貧也。君嚴猛而閉下,臣戰慄而塞
 耳,則妄聞之氣發於音聲,故有鼓妖。寒氣動,故有魚孽。而龜能為孽,龜能陸處,非極陰也,魚去水而死,極陰之薛也。於《易》,《坎》為水,為豕,豕大耳而不聰察,聽氣毀,故有豕禍也。一曰,寒歲豕多死及為怪,亦是也。及人,則多病耳者,故有耳痾。水色黑,故有黑眚黑祥。凡聽傷者,病水敢;水氣病,則火沴之。其極貧者,順之,其福曰富。劉歆《聽傳》曰有介蟲之孽也。



 庶徵之恒寒,劉歆以為大雨雪,及未當雨雪而雨雪,及大雨雹,隕霜殺菽草,皆恒寒之罰也。京房《易傳》曰:「有德遭險茲謂逆命,厥異寒。誅罰過深,當燠而寒,盡六日,亦為雹。害正不誅茲謂養賊,寒七十
 二日,殺飛禽。道人始去茲謂傷,其寒,物無霜而死,涌水而出。戰不量敵幫茲謂辱命,其寒,雖雨物不茂。聞善不予,厥咎聾。」



 吳孫權嘉禾三年九月朔,隕霜傷穀。案劉向說,誅罰不由君出,在臣下之象也」。是時,校事呂壹專作威福,與漢元帝時石顯用事隕霜同應。班固書九月二日,陳壽言朔,皆明未可以傷穀也。壹後亦伏誅。京房《易傳》曰:「興兵妄誅茲謂亡法,厥災霜,夏殺五穀,冬殺麥。誅不原情茲謂不仁,其霜,夏先大雷風,冬先雨,乃隕霜,有芒角。賢聖遭害,其霜附木不下地。佞人依刑茲謂私賊,其霜在草根
 土隟間。不教而誅茲謂虐,其霜反在草下。



 四年七月,雨雹,又隕霜。案劉向說,「雹者,陰協陽也」。是時,呂壹作威用事,詆毀重臣,排陷無辜。自太子登以下咸患毒之,而壹反獲封侯寵異,與春秋時公子遂專任雨雹同應也。漢安帝信讒,多殺無辜,亦雨雹。董仲舒曰:「凡雹皆為有脅,行專一之政故也。」



 赤烏四年正月,大雪,平地深三尺,鳥獸死者太半。是年夏,全琮等四將軍攻略淮南、襄陽,戰死者千餘人。其後,權以讒邪數責讓陸議,議憤恚致卒,與漢景武大雪同事。



 十一年四月,雨雹。是時權聽讒,將危太子。其後,朱據、屈晃以迕意黜辱,陳正、陳象以忠諫族誅,而太子終廢。此有德遭險,誅罰過深之應也。



 武帝泰始六年冬,大雪。七年十二月,又大雪。明年,有步闡、楊肇之敗,死傷甚眾,不聰之罰也。



 九年四月辛未,隕霜。是時,賈充親黨比周用事,與魯定公、漢元帝時隕霜同應也。



 咸寧三年八月,平原、安平、上黨、泰山四郡霜,害三豆。是月,河間暴風寒冰,郡國五隕霜傷穀。是後大舉征吳,馬隆又帥精勇討涼州。



 五年五月丁亥,鉅鹿、魏郡雨雹,傷禾麥。辛卯,鴈門雨雹,傷秋稼。六月庚戌,汲郡、廣平、陳留、滎陽雨雹。丙辰,又雨雹,隕霜,傷秋麥千三百餘頃,壞屋百二十餘間。癸亥,安定雨雹。七月丙申,魏郡又雨雹。閏月壬子,新興又雨雹。八月庚子,河南、河東、弘農又雨雹,兼傷秋稼三豆。



 太康元年三月,河東、高平霜雹,傷桑麥。四月,河南、河內、河東、魏郡、弘農雨雹,傷麥豆。是月庚午,畿內縣二及東平、范陽雨雹。癸酉,畿內縣五又雨雹。五月,東平、平陽、上黨、鴈門、濟南雨雹,傷禾麥三豆。是時王濬有大功,而權戚互加陷抑,帝從容不斷,陰脅陽之應也。



 二年二月辛酉,隕霜于濟南、瑯邪,傷麥。壬申,瑯邪雨雹,傷麥。三月甲午,河東隕霜,害桑。五月丙戌,城陽、章武、瑯邪傷麥。庚寅,河東、樂安、東平、濟陰、弘農、濮陽、齊國、頓丘、魏郡、河內、汲郡、上黨雨雹,傷禾稼。六月,郡國十七雨雹。七月,上黨雨雹。三年十二月,大雪。



 五年七月乙卯,中山、東平雨雹,傷秋稼。甲辰,中山雨雹。九月,南安大雪,折木。



 六年二月,東海隕霜,傷桑麥。三月戊辰,齊郡臨淄、長廣不其等四縣,樂安梁鄒等八縣,瑯邪臨沂等八縣,河間
 易城等六縣,高陽北新城等四縣隕霜,傷桑麥。六月,榮陽、汲郡、鴈門雨雹。



 八年四月,齊國、天水二郡隕霜。十二月,大雪。九年正月,京都大風雨雹,發屋拔木。四月,隴西隕霜。十年四月,郡國八隕霜。



 惠帝元康二年八月,沛及蕩陰雨雹。三年四月,滎陽雨雹。六月,弘農湖、華陰又雨雹,深三尺。是時,賈后凶淫專恣,與春秋魯桓夫人同事,陰氣盛也。



 五年六月,東海雨雹,深五寸。十二月,丹陽建鄴雨雹。是月,
 丹陽建鄴大雪。六年三月,東海隕雪,殺桑麥。七年五月,魯國雨雹。七月,秦、雍二州隕霜,殺稼也。



 九年三月旬有八日,河南、滎陽、潁川隕霜,傷禾。五月、雨雹。是時,賈后凶躁滋甚,及冬,遂廢愍懷。



 永寧元年七月,襄城、河南雨雹。十月,襄城、河南、高平、平陽又風雹,折木傷稼。



 光熙元年閏八月甲申朔,霰雪。劉向曰:「盛陽雨水,傷熱,陰氣脅之,則轉而為雹。盛陰雨雪,凝滯,陽氣薄之,則散而為霰。今雪非其時,此聽不聰之應。」是年,帝崩。



 孝懷帝永嘉元年十二月冬,雪,平地三尺。七年十月庚午,大雪。



 元帝太興二年三月丁未,成都風雹,殺人。三年三月,海鹽雨雹。是時,王敦陵上。



 永昌二年十二月,幽、冀、并三州大雨。



 明帝太寧元年十二月,幽、冀、并三州大雪。二年四月庚子,京都雨雹,燕雀死。三年三月丁丑,雨雪。癸巳,隕霜。四月,大雨雹。是年,帝崩,尋有蘇峻之亂。



 成帝咸和六年三月癸未,雨雹。是時,帝幼弱,政在大臣。
 九年八月,成都大雪。是歲,李雄死。



 咸康二年正月丁巳,皇后見于太廟,其夕雨雹。



 康帝建元元年八月,大雪。是時,政在將相,陰氣盛也。劉向曰:「凡雨陰也,雪又雨之陰也。出非其時,迫近象也。」



 穆帝永和二年八月,冀方大雪,人馬多凍死。五年六月,臨漳暴風震電,雨雹,大如升。



 十年五月,涼州雪。明年八月,張祚枹罕護軍張瓘率宋混等攻滅祚,更立張耀靈弟玄靚。京房《易傳》曰:「夏雪,戒臣為亂。」此其亂之應也。



 十一年四月壬申朔,霜。十二月戊午,雷。己未,雪。是時帝
 幼,母后稱制,政在大臣,陰盛故也。



 升平二年正月,大雪。



 海西太和三年四月,雨雹,折木。



 孝武太元二年四月己酉,雨雹。十二月,大雪。是時帝幼,政在將相,陰之盛也。



 十二年四月己丑,雨雹。二十年五月癸卯,上虞雨雹。



 二十一年四月丁亥,雨雹。是時,張夫人專寵,及帝暴崩,兆庶尤之。十二月,雨雪二十三日。是時嗣主幼沖,塚宰專政。



 安帝隆安二年三月乙卯,雨雹。是秋,王恭、殷仲堪稱兵內侮,終皆誅之也。



 元興二年十二月,酷寒過甚。是時,桓玄篡位,政事煩苛。識者以為朝政失在舒緩,玄則反之以酷。案劉向曰:「周衰無寒歲,秦滅無燠年。」此之謂也。



 三年正月甲申,霰雪又雷。雷霰同時,皆失節之應也。四月丙午,江陵雨雹。是時,安帝蒙塵。



 義熙元年四月壬申,雨雹。是時,四方未一,鉦鼓日戒。



 五年三月己亥,雪,深數尺。五月癸巳,溧陽雨雹。九月己丑,廣陵雨雹。明年,盧循至蔡洲。



 六年正月丙寅,雪又雷。五月壬申,雨雹。八年四月辛未朔,雨雹。六月癸亥,雨雹,大風發屋。是秋,誅劉蕃等。



 十年四月辛卯,雨雹。



 雷震



 魏明帝景初中,洛陽城東橋、城西洛水浮橋桓楹同日三處俱時震。尋又震西城上候風木飛鳥。時勞役大起,帝尋晏駕。



 吳孫權赤烏八年夏,震宮門柱,又擊南津大橋桓楹。



 孫亮建興元年十二月朔,大風震電。是月,又雷雨。義同
 前說,亮終廢。



 武帝太康六年十二月甲申朔,淮南郡震電。七年十二月己亥,毗陵雷電,南沙司鹽都尉戴亮以聞。十年十二月癸卯,廬江、建安雷電大雨。



 惠帝永康元年六月癸卯,震崇陽陵標,西南五百步標破為七十片。是時,賈后陷害鼎輔,寵樹私戚,與漢桓帝時震憲陵寢同事也。后終誅滅。



 永興二年十月丁丑,雷震。



 懷帝永嘉四年十月,震電。



 愍帝建興元年十一月戊午,會稽大雨震電。己巳夜,赤
 氣曜於西北。是夕,大雨震電。庚午,大雪。案劉同說,「雷以二月出,八月入。」今此月震電者,陽不閉藏也。既發泄而明日便大雪,皆失節之異也。是時,劉聰僭號平陽,李雄稱制於蜀,九州幅裂,西京孤微,為君失時之象也。赤氣,赤祥也。



 元帝太興元年十一月乙卯,暴雨雷電。



 永昌二年七月庚子朔,雷震太極殿柱。十二月,會稽、吳郡雷震電。



 成帝咸和元年十月己巳,會稽郡大雨震電。三年六月辛卯,臨海大雷,破郡府內小屋柱十枚,殺人。
 九月二日壬午立冬,會稽雷電。四年十一月,吳郡、會稽大震電。



 穆帝永和七年十月壬午,雷雨震電。升平元年十一月庚戌,雷。乙丑,又雷。



 五年十月庚午,雷發東南方。



 孝武帝太元五年六月甲寅,雷震含章殿四柱,并殺內侍二人。十年十二月,雷聲在南方。十四年七月甲寅,雷震,燒宣陽門西柱。



 安帝隆安二年九月壬辰,雷雨。



 元興三年,永安皇后至自巴陵,將設儀導入宮,天雷震,人馬各一俱殪焉。



 義熙四年十一月辛卯朔,西北方疾風發。癸丑,雷。五年六月丙寅,雷震太廟,破東鴟尾,徹柱,又震太子西池合堂。是時,帝不親蒸嘗,故天震之,明簡宗廟也。西池是明帝為太子時所造次,故號太子池。及安帝多病,患無嗣,故天震之,明無後也。



 六年正月丙寅,雷,又雪。十二月壬辰,大雷。九年十一月甲戌,雷。乙亥,又雷。



 鼓妖



 惠帝元康九年三月,有聲若牛,出許昌城。十二月,廢愍懷太子,幽于許宮。明年,賈后遣黃門孫慮殺太子,擊以藥杵,聲聞于外,是其應也。



 蘇峻在歷陽外營,將軍鼓自鳴,如人弄鼓者。峻手自破之,曰:「我鄉土時有此,則城空矣。」俄而作亂夷滅,此聽不聰之罰也。



 石季龍末,洛陽城西北九里,石牛在青石趺上,忽鳴,聲聞四十里。季龍遣人打落兩耳及尾,鐵釘釘四腳。尋而季龍死。



 孝武太元十五年三月己酉朔,東北方有聲如雷。案劉
 向說,以為「雷當託於雲,猶君託於臣。無雲而雷,此君不恤於下,下人將叛之象也。」及帝崩而天下漸亂,孫恩、桓玄交陵京邑。



 吳興長城夏架山有石鼓,長丈餘,面逕三尺許,下有盤石為足,鳴則聲如金鼓,三吳有兵。至安帝隆安中大鳴,後有孫恩之亂。



 魚孽



 魏齊王嘉平四年五月,有二魚集于武庫屋上,此魚孽也。王肅曰:「魚生於水,而亢於屋,介鱗之物,失其所也。邊將其殆有棄甲之變乎!」後果有東關之敗。干寶又以為
 高貴鄉公兵禍之應。二說皆與班固旨同。



 武帝太康中,有鯉魚二見武庫屋上。干寶以為:「武庫兵府,魚有鱗甲,亦兵類也。魚既極陰,屋上太陽,魚見屋上,象至陰以兵革之禍干太陽也。至惠帝初,誅楊駿,廢太后,矢交館閣。元康末,賈后謗殺太子,尋亦誅廢。十年之間,母后之難再興,是其應也,自是禍亂構矣。」京房《易傳》曰:「魚去水,飛入道路,兵且作。」



 蝗蟲



 《春秋》,螽。劉歆從介蟲之孽,與魚同占。



 魏文帝黃初三年七月,冀州大蝗,人飢。案蔡邕說,「蝗者,
 在上貪苛之所致也」。是時,孫權歸順,帝因其有西陵之役,舉大眾襲之,權遂背叛也。



 武帝泰始十年六月,蝗。是時,荀、賈任政,疾害公直。



 惠帝永寧元年,郡國六蝗。



 懷帝永嘉四年五月,大蝗,自幽、并、司、冀至于秦雍,草木牛馬毛鬣皆盡。是時,天下兵亂,漁獵黔黎,存亡所繼,惟司馬越、茍晞而已。競為暴刻,經略無章,故有此孽。



 愍帝建興四年六月,大蝗。去歲劉曜頻攻北地、馮翊,麴允等悉眾御之,卒為劉曜所破,西京遂潰。五年,帝在平陽,司、冀、青、雍螽。



 元帝太興元年六月,蘭陵合鄉蝗,害禾稼。乙未,東莞蝗蟲縱廣三百里,害苗稼。七月,東海、彭城、下邳、臨淮四郡蝗蟲害禾豆。八月,冀、青、徐三州蝗,食生草盡,至于二年。是時,中州淪喪,暴亂滋甚也。



 二年五月,淮陵、臨淮、淮南、安豐、廬江等五郡蝗蟲食秋麥。是月癸丑,徐州及揚州江西諸郡蝗,吳郡百姓多餓死。是年,王敦并領荊州,苛暴之釁自此興矣。



 孝武帝太元十五年八月,兗州蝗。是時,慕容氏逼河南,征戍不已,故有斯孽。十六年五月,飛蝗從南來,集堂邑縣界,害苗稼。是年春,
 發江州兵營甲士二千人,家口六七千,配護軍及東宮,後尋散亡殆盡。又邊將連有征役,故有斯孽。



 豕禍



 吳孫皓寶鼎元年,野豕入右大司馬丁奉營,此豕禍也。後奉見遣攻穀陽,無功而反。皓怒,斬其導軍。及舉大眾北出,奉及萬彧等相謂曰:「若至華里,不得不各自還也。」此謀泄,奉時雖已死,皓追討穀陽事,殺其子溫,家屬皆遠徙,豕禍之應也。龔遂曰,「山野之獸,來入宮室,宮室將空」,又其象也。



 懷帝永嘉中,壽春城內有豕生兩頭而不活。周馥取而
 觀之,時識者云:「豕,北方畜,胡狄象。兩頭者,無上也。生而死,不遂也。天戒若曰,勿生專利之謀,將自致傾覆也。」周馥不寤,遂欲迎天子令諸侯,俄為元帝所敗,是其應也。石勒亦尋渡淮,百姓死者十有其九。



 元帝建武元年,有豕生八足,此聽不聰之罰,又所任邪也。是後有劉隗之變。



 成帝咸和六年六月,錢唐人家豭豕產兩子,而皆人面,如胡人狀,其身猶豕。京房《易妖》曰:「豕生人頭豕身者,危且亂。今此豭豕而產,異之甚者也。」



 孝武帝太元十年四月,京都有豚一頭二脊八足。
 十三年,京都人家豕產子,一頭二身八足,並與建武同妖也。是後,宰相沈酗,不恤朝政,近習用事,漸亂國綱,至於大壞也。



 黑眚黑祥



 孝懷帝永嘉五年十二月,黑氣四塞,近黑祥也。帝尋淪陷,王室丘墟,是其應也。



 愍帝建興二年正月己已朔,黑霧著人如墨,連夜,五日乃止,此近黑祥也。其四年,帝降劉曜。



 元帝永昌元年十月,京師大霧,黑氣蔽天,日月無光。十一月,帝崩。



 火沴水



 武帝太康五年六月,任城、魯國池水皆赤如血。案劉向說,近水沴水,聽之不聰之罰也。京房《易傳》曰:「君淫於色,賢人潛,國家危,厥異水流赤。」



 穆帝升平三年二月,涼州城東池中有火。四年四月,姑臧澤水中又有火。此火沴水之妖也。明年,張天錫殺中護軍張邕。邕,執政之人也。



 安帝元興二年十月,錢唐臨平湖水赤,桓玄諷吳郡使言開除以為己瑞,俄而桓玄敗。


《傳》曰:「思心之不容,是謂不聖,厥咎
 \gezhu{
  雨瞀}
 ,厥罰恒風,厥極凶
 短折。時則有脂夜之妖,時則有華孽,時則有牛禍,時則有心腹之痾,時則有黃眚黃祥,時則有金木水火沴土。」思心不容,是謂不聖。思心者,心思慮也。容,寬也。孔子曰:「居上不寬,吾何以觀之哉!」言上不寬大包容,臣下則不能居聖位。貌言視聽,以心為主,四進皆失,則區
 \gezhu{
  雨瞀}
 無識,故其咎
 \gezhu{
  雨瞀}
 也。雨旱寒燠,亦以風為本,四氣皆亂,故其罰恒風也。恒風傷物,故其極凶短折也。傷人曰凶,禽獸曰短,草木曰折。一曰,凶,夭也;兄喪弟曰短,父喪子曰折。在人,腹中肥而包裹心者,脂也。心區
 \gezhu{
  雨瞀}
 則冥晦,故有脂夜之妖。一曰,有脂物而夜為妖,若脂夜汙人衣,淫之象也。
 一曰,夜妖者,雲風並起而杳冥,故與常風同象也。溫而風則生螟螣,有裸蟲之孽。劉向以為:「於《易》,《巽》為風,為木。卦在三月四月,繼陽而治,主木之華實。風氣盛至,秋冬木復華,故有華孽。」一曰,地氣盛同秋冬復華。一曰,華者色也,土為內事,謂女孽也。於《易》,《坤》為土,為牛。牛大心而不能思慮,心氣毀,故有牛禍。一曰,牛多死及為怪,亦是也。及人,則多病心腹者,故有心腹之痾。土色黃,故有黃眚黃祥。凡思心傷者,病土氣;土氣病,則金木水火沴之,故曰時則有金木水火沴土。不言「惟」而獨曰「時則有」者,非一沖氣所沴,明其異大也。其極凶短折者,順之,其福
 曰考終命。劉歆《思心傳》曰:「時有臝蟲之孽,謂螟螣之屬也。」



 庶徵恒風



 魏齊王正始九年十一月,大風數十日,發屋折樹。十二月戊午晦尤甚,動太極東閣。


嘉平元年正月壬辰朔,西北大風,發屋折樹木,昏塵蔽天。案管輅說,此為時刑大臣,執政之憂也。是時,曹爽區
 \gezhu{
  雨瞀}
 自專,驕僭過度,天戒數見,終不改革,此思心不睿,恒風之罰也。後踰旬而爽等誅滅。京房《易傳》曰:「眾逆同志,至德乃潛,厥異風。其風也,行不解,物不長,雨小而傷。政
 悖德隱茲謂亂,厥風先風不雨,大風暴起,發屋折木。守義不進茲謂眊,厥風與雲俱起,折五穀莖。臣易上政茲謂不順,厥風大飆發屋。賦斂不理茲謂禍,厥風絕經紀,止即溫,溫即蟲。侯專封茲謂不統,厥風疾而樹不搖,穀不成。辟不思道利茲謂無澤,厥風不搖木,旱無雲,傷禾。公常於利茲謂亂,厥風微而溫,生蟲蝗,害五穀。棄政作淫茲謂惑,厥風溫,螟蟲起,害有益人之物。諸侯不朝茲謂畔,厥風無恒,地變赤,雨殺人。」


吳孫權太元元年八月朔,大風,江海涌溢,平地水深八尺,拔高陵樹二千株,石碑蹉動,吳城兩門飛落。案華核
 對,役繁賦重,區
 \gezhu{
  雨瞀}
 不容之罰也。明年,權薨。



 孫亮建興元年十二月丙申,大風震電。是歲,魏遣大眾三道來攻,諸葛恪破其東興軍,二軍亦退。明年,恪又攻新城,喪眾太半,還,伏誅。



 孫休永安元年十一月甲午,風四轉五復,蒙霧連日。是時,孫綝一門五侯,權傾吳主,風霧之災,與漢五侯、丁、傅同應也。十二月丁卯夜,有大風,發木揚沙。明日,綝誅。



 武帝泰始五年五月辛卯朔,廣平大風,折木。



 咸寧元年五月,下邳、廣陵大風,壞千餘家,折樹木。其月甲申,廣陵、司吾、下邳大風,折木。
 三年八月,河間大風,折木。



 太康二年五月,濟南暴風,折木,傷麥。六月,高平大風,折木,發壞邸閣四十餘區。七月,上黨又大風,傷秋稼。八年六月,郡國八大風。九年正月,京都風雹,發屋拔樹。後二年,宮車晏駕。



 惠帝元康四年六月,大風雨,拔木。五年四月庚寅夜,暴風,城東渠波浪殺人。七月,下邳大風,壞廬舍。九月,鴈門、新興、太原、上黨災風傷稼。明年,氐羌反叛,大兵西討。



 九年六月,飆風吹賈謐朝服飛數百丈。明年,謐誅。十一
 月甲子朔,京都連大風,發屋折木。十二月,愍懷太子廢,幽于許昌。



 永康元年二月,大風拔木。三月,愍懷被害。己卯,喪柩發許昌還洛。是日,又大風雷電,幃蓋飛裂。四月,張華第舍飆風起,折木飛繒,折軸六七。是月,華遇害。十一月戊午朔,大風從西北來,折木飛沙石,六日止。明年正月,趙王倫篡位。



 永寧元年八月,郡國三大風。



 永興元年正月乙丑,西北大風。趙王倫建始元年正月癸酉,趙土倫祠太廟,災風暴起,
 塵四合。其年四月,倫伏辜。



 元帝永昌元年七月丙寅,大風拔木,屋瓦皆飛。八月,暴風壞屋,拔御道柳樹百餘株。其風縱橫無常,若風自八方來者。是時,王敦專權,害尚書令刁協、僕射周顗等,故風縱橫若非一處也。此臣易上政,諸侯不朝之罰也。十一月,宮車晏駕。



 成帝咸康四年三月壬辰,成都大風,發屋折木。四月,李壽襲殺李期,自立。



 穆帝升平元年八月丁未,策立皇后何氏。是日,疾風。後
 桓玄篡位,乃降后為零陵縣君,不睿之罰也。五年正月戊戌朔,疾風。



 海西公太和六年二月,大風迅急,是年被廢。



 孝武帝寧康元年三月,京都大風,火大起。是時,桓溫入朝,志在陵上,帝又幼少,人懷憂恐,斯不睿之徵也。三年三月戊申朔,暴風迅起,從丑上來,須臾逆轉,從子上來,飛沙揚礫。



 太元二年二月乙丑朔,暴風折木。閏三月甲子朔,暴風疾雨俱至,發屋折木。三年六月,長安大風,拔苻堅宮中樹。其後,堅再南伐,遂
 有淝水之敗,身戮國亡。四年八月乙未,暴風揚沙石。



 十二年正月壬子夜,暴風。七月甲辰,大風折木。十三年十二月乙未,大風,晝晦。其後帝崩而諸侯違命,權奪於元顯,禍成於桓玄,是其應也。十七年六月乙卯,大風折木。



 安帝元興二年二月甲辰夜,大風雨,大航門屋瓦飛落。明年,桓玄篡位,由此門入。



 三年正月,桓玄出遊大航南,飄風飛其卑輗蓋,經三月而玄敗歸江陵。五月,江陵又大風折木。是月,桓玄敗於
 崢嶸洲,身亦屠裂。十一月丁酉,大風,江陵多死者。



 義熙四年十一月辛卯朔,西北疾風起。五年閏十月丁亥,大風發屋。明年,盧循至蔡洲。六年五月壬申,大風拔北郊樹,樹幾百年也。并吹瑯邪、揚州二射堂倒壞。是日,盧循大艦漂沒。甲戌,又風,發屋折木。是冬,王師南討。九年正月,大風,白馬寺浮圖剎柱折壞。十年四月己丑朔,大風拔木。六月辛亥,大風拔木。七月,淮北大風,壞廬舍。明年,西討司馬休之應。



 夜妖



 魏高貴鄉公正元二年正月戊戌,景帝討毌丘儉,大風晦暝,行者皆頓伏,近夜妖也。劉向曰:「正晝而暝,陰為陽,臣制君也。」



 元帝景元三年十月,京都大震,晝晦,此夜妖也。班固曰:「夜妖者,雲風並起而杳冥,故與常風同象也。」劉向《春秋說》云:「天戒若曰,勿使大夫世官,將令專事。暝晦,公室卑矣。」魏見此妖,晉有天下之應也。



 懷帝永嘉四年十月辛卯,晝昏,至于庚子,此夜妖也。後年,劉曜寇洛川,王師頻為賊所敗,帝蒙塵于平陽。



 孝武帝太元十三年十二月乙未,大風晦暝。其後帝崩,
 而諸侯違命,干戈內侮,權奪於元顯,禍成於桓玄。



 臝蟲之孽



 京房《易傳》曰:「臣安祿位茲謂貪,厥災蟲食根。德無常茲謂煩,蟲食葉。不絀無德,蟲食本。與東作爭茲謂不時,蟲食莖。蔽惡生孽,蟲食心。」



 武帝咸寧元年七月,郡國螟。九月。青州又螟。是月,郡國有青蟲食其禾稼。四年,司、冀、兗、豫、荊、揚郡國二十螟。



 太康四年,會稽彭蜞及蟹皆化為鼠,甚眾,復大食稻為災。
 九年八月,郡國二十四螟。九月,蟲又傷秋稼。是時,帝聽讒諛,寵任賈充、楊駿,故有蟲蝗之災,不絀無德之罰。



 惠帝元康三年九月,帶方等六縣螟,食禾葉盡。



 永寧元年七月,梁、益、涼三州螟。是時,齊王冏執政,貪苛之應也。十月,南安、巴西、江陽、太原、新興、北海青蟲食禾葉,甚者十傷五六。十二月,郡國六螟。



 牛禍



 武帝太康九年,幽州塞北有死牛頭語,近牛禍也。是時,帝多疾病,深以後事為念,而託付不以至公,思瞀亂之
 應也。案師曠曰:「怨讟動於人。則有非言之物而言。」又其義也。京房《易傳》曰:「殺無罪,牛生妖。」



 惠帝太安中,江夏張騁所乘牛言曰:「天下亂,乘我何之!」騁懼而還,犬又言曰:「歸何早也?」尋後牛又人立而行。騁使善卜者卦之,謂曰:「天下將有兵亂,為禍非止一家。」其年,張昌反,先略江夏,騁為將帥,於是五州殘亂,騁亦族滅。京房「易傳」曰:「牛能言,如其言占吉凶。」《易萌氣樞》曰:「人君不好士,走馬被文繡,犬狼食人食,則有六畜談言。」時天子諸侯不以惠下為務,又其應也。



 元帝建武元年七月,晉陵陳門才牛生犢,一體兩頭。案
 京房《易傳》言:「牛生子二首一身,天下將分之象也。」是時,愍帝蒙塵於平陽,尋為逆胡所殺。元帝即位江東,天下分為二,是其應也。



 太興元年,武昌太守王諒牛生子,兩頭八足,兩尾共一腹,三年後死。又有牛一足三尾,皆生而死。案司馬彪說,「兩頭者,政在私門,上下無別之象也。」京房《易傳》曰:「足多者,所任邪也;足少者,不勝任也。」其後王敦等亂政,此其祥也。


四年十二月,郊牛死。案劉向說《春秋》效牛死曰:「宣公區
 \gezhu{
  雨瞀}
 昏亂,故天不饗其祀。」今元帝中興之業,實王導之謀
 也。劉隗探會上意,以得親幸,導見疏外,此區
 \gezhu{
  雨瞀}
 不睿之禍。



 成帝咸和二年五月,護軍牛生犢,兩頭六足。是冬,蘇峻作亂。七年,九德人袁榮家牛產犢,兩頭八足,二尾共身。



 桓玄之國,在荊州詣刺史殷仲堪,行至鶴穴,逢一老公驅青牛,形色瑰異,桓玄即以所乘牛易取。乘至零陵涇溪,駿駛非常,息駕飲牛,牛逕入江水不出。玄遣人覘守,經日無所見。於後玄敗被誅。



 黃眚黃祥



 蜀劉備章武二年,東伐。二月,自秭歸進屯夷道。六月,秭歸有黃氣見,長十餘里,廣數十丈。後踰旬,備為陸議所破,近黃祥也。



 魏齊王正始中,中山王周南為襄邑長。有鼠從穴出,語曰:「王周南,爾以某日死。」周南不應,鼠還穴。後至期,更冠幘皂衣出,語曰:「周南,汝日中當死。」又不應,鼠復入穴。斯須更出,語如向。日適欲中,鼠入須臾復出,出復入,轉更數,語如前。日適中,鼠曰:「周南,汝不應,我復何道!」言絕,顛蹶而死,即失衣冠。取視,俱如常鼠。案班固說,此黃祥也。是時,曹爽專政,競為比周,故鼠作變也。


惠帝元康四年十二月,大霧。帝時昏眊,政非已出,故有區
 \gezhu{
  雨瞀}
 之妖。



 元帝太興四年八月,黃霧四寒,埃氛蔽天。



 永昌元年十月,京師大霧,黑氣貫天,日無光。



 明帝太守元年正月癸巳,黃霧四塞。二月,又黃霧四塞。是時王敦擅權,謀逆愈甚。



 穆帝永和七年三月,涼州大風拔木,黃霧下塵。是時,張重華納譖,出謝艾為酒泉太守,而所任非其人,至九年死,嗣子見殺,是其應也。京房《易傳》曰:「聞善不予茲謂不知,厥異黃,厥咎聾,厥災不嗣。黃者,有黃濁氣四塞天下。
 蔽賢絕道,災至絕世也。」



 孝武太元八年二月癸未,黃霧四塞。是時,道子專政,親近佞人,朝綱方替。



 安帝元興元年十月丙申朔,黃霧昏濁不雨。是時桓玄謀逆之應。



 義熙五年十一月,大霧。十年十一月,又大霧。是時,帝室衰微,臣下權盛,兵及土地,略非君有,此其應也。



 地震



 劉向曰:「地震,金木水火沴土者也。伯陽甫曰:「天地之氣,
 不過其序;若過其序,人之亂也。陽伏而不能出,陰迫而不能升,於是有地震。」



 吳孫權黃武四年,江東地連震。是時,權受魏爵命為大將軍、吳王,改元專制,不修臣跡。京房《易傳》曰:「臣事雖正,專必震。其震,於水則波,於木則搖,於屋則瓦落。大經在辟而易臣茲謂陰動,厥震搖政宮。大經搖政茲謂不陰,厥震搖山,出涌水。嗣子無德專祿茲謂不順,厥震動丘陵,涌水出。」劉向並云;「臣下強盛,將動而為害之應也。」



 魏明帝青龍二年十一月,京都地震,從東來,隱隱有聲,搖屋瓦。



 景初元年六月戊申,京都地震。是秋,吳將朱然圍江夏,荊州刺史胡質擊退之。又,公孫文懿叛,自立為燕王,改年,置百官。明年,討平之。



 吳孫權嘉禾六年五月,江東地震。



 赤烏二年正月,地再震。是時,呂壹專事,步騭上疏曰:「伏聞校事吹毛求瑕,趣欲陷人,成其威福,無罪無辜,橫受重刑,雖有大臣,不見信任,如此,天地焉得無變!故地連震動,臣下專政之應也。冀所以警悟人主,可不深思其意哉!」壹後卒敗。



 魏齊王正始二年十一月,南安郡地震。
 三年七月甲申,南安郡地震。十二月,魏郡地震。六年二月丁卯,南安郡地震。是時,曹爽專政,遷太后於永寧宮,太后與帝相泣而別。連年地震,是其應也。



 吳孫權赤烏十一年二月,江東地仍震。是時,權聽讒,尋黜朱據,廢太子。



 蜀劉禪炎興元年,蜀地震。是時宦人黃皓專權。案司馬彪說,「閹官無陽施,猶婦人也」。皓見任之應,與漢和帝時同事也。是冬,蜀亡。



 武帝泰始五年四月辛酉,地震。是年冬,新平氐羌叛。明年,孫皓遣大眾入渦口。
 七年六月丙申,地震。



 咸寧二年八月庚辰,河南、河東、平陽地震。四年六月丁未,陰平廣武地震,甲子又震。



 太康二年二月庚申,淮南、丹陽地震。五年正月朔壬辰,京師地震。六年七月己丑,地震。七年七月,南安、犍為地震。八月,京兆地震。八年五月壬子,建安地震。七月,陰平地震。八月,丹陽地震。九年正月,會稽、丹陽、吳興地震。四月辛酉,長沙、南海等
 郡國八地震。七月至于八月,地又四震,其三有聲如雷。九月,臨賀地震,十二月又震。十年十二月己亥,丹楊地震。



 太熙元年正月,地又震,武帝世,始於賈充,終於楊駿,阿黨昧利,茍竊朝權。至于末年,所任轉弊,故頻年地震,過其序也,終喪天下。



 惠帝元康元年十二月辛酉,京都地震。此夏,賈后使楚王瑋殺汝南王亮及太保衛瓘,此陰道盛、陽道微故也。



 四年二月,上谷、上庸、遼東地震。五月,蜀郡山移;淮南壽春洪水出,山崩地陷,壞城府。八月,上谷地震,水出,殺百
 餘人。十月,京都地震。十一月,滎陽、襄城、汝陰、梁國、南陽地皆震。十二月,京都又震。是時,賈后亂朝,終至禍敗之應也。漢鄧太后攝政時,郡國地震。李固以為:「地,陰也,法當安靜。今乃越陰之職,專陽之政,故應以震。」此同事也。京房《易傳》曰:「小人剝廬,厥妖山崩,茲謂陰乘陽,弱勝彊。」又曰:「陰背陽則地裂,父子分離,夷羌叛去。」



 五年五月丁丑,地震。六月,金城地震。六年正月丁丑,地震。八年正月丙辰,地震。



 太安元年十月,地震。時齊王冏專政之應。
 二年十二月丙辰,地震。是時,長沙王乂專政之應也。



 孝杯帝永嘉三年十月,荊、湘二州地震。時司馬越專政。四年四月,兗州地震。五月,石勒寇汲郡,執太守胡寵,遂南濟河,是其應也。



 愍帝建興二年四月甲辰,地震。三年六月丁卯,長安又地震。是時主幼,權傾於下,四方雲擾,兵亂不息之應也。



 元帝太興元年四月,西平地震,湧水出。十二月,廬陵、豫章、武昌、西陵地震,湧水出,山崩。干寶以為王敦陵上之應也。



 二年五月己丑,祁山地震,山崩,殺人。是時,相國南陽王保在祁山,稱晉王不終之象也。三年五月庚寅,丹陽、吳郡、晉陵又地震。



 成帝咸和二年二月,江陵地震。三月,益州地震。四月己未,豫章地震。是年,蘇峻作亂。九年三月丁酉,會稽地震。



 穆帝永和元年六月癸亥,地震。是時,嗣主幼沖,母后稱制,政在臣下,所以連年地震。二年十月,地震。三年正月丙辰,地震。九月,地又震。
 四年十月己未,地震。



 五年正月庚寅,地震。是時,石季龍僭即皇帝位,亦過其序也。



 九年八月丁酉,京都地震,有聲如雷。十年正月丁卯,地震,聲如雷,雞雉皆鳴句。十一年四月乙酉,地震。五月丁未,地震。



 升平二年十一月辛酉,地震。五年八月,涼州地震。



 哀帝隆和元年四月甲戌,地震。是時,政在將相,人主南面而已。



 興寧元年四月甲戌,揚州地震,湖瀆溢。二年二月庚寅,江陵地震。是時,桓溫專政。



 海西公太和元年二月,涼州地震,水涌。是海西將廢之應也。



 簡文帝咸安二年十月辛未,安成地震。是年帝崩。



 孝武帝寧康元年十月辛未,地震。二年二月丁巳,地震。七月甲午,涼州地又震,山崩。是時,嗣主幼沖,權在將相,陰盛之應也。



 太元二年閏三月壬午,地震。五月丁丑,地震。十一年六月己卯,地震。是後緣河諸將連歲兵役,人勞
 之應也。十五年二月己酉朔夜,地震。八月,京都地震。十二月己未,地震。十七年六月癸卯,地震。十二月己未,地又震。是時,群小弄權,天下側目。十八年正月癸亥朔,地震。二月乙未夜,地震。



 安帝隆安四年四月乙未,地震。九月癸丑,地震。是時,幼主沖昧,政在臣下。



 義熙四年正月壬子夜,地震有聲。十月癸亥,地震。五年正月戊戌夜,尋陽地震,有聲如雷。明年,盧循下。
 八年,自正月至四月,南康、廬陵地四震。明年,王旅西討荊益。十年三月戊寅,地震。



 山崩地陷裂



 吳孫權赤烏十三年八月,丹陽、句容及故鄣、寧國諸山崩,鴻水溢。案劉向說,「山,陽,君也。水,陰,百姓也。天戒若曰,君道崩壞,百姓將失其所與」!春秋梁山崩,漢齊、楚眾山發水,同事也。夫三代命祀,祭不越望,吉凶禍福,不是過也。吳雖稱帝,其實列國,災發丹陽,其天意矣。劉歆以為:「國主山川,山崩川竭,亡之徵也。」後二年而權薨,又二十
 六年而吳亡。



 魏元帝咸熙二年二月,太行山崩,此魏亡之徵也。其冬,晉有天下。



 武帝泰始三年三月戊午,大石山崩。四年七月,泰山崩墜三里。京房《易傳》曰:「自上下者為崩,厥應泰山之石顛而下,聖王受命人君虜。」及帝晏駕,而祿去王室,惠皇懦弱,懷愍二帝俱辱虜庭,淪胥於北,元帝中興於南,此其應也。



 太康五年五月丙午,宣帝廟地陷。六年十月,南安新興山崩,涌水出。
 七年二月,朱提之大瀘山崩,震壞郡舍,陰平之仇池崖隕。八年七月,大雨,殿前地陷,方五尺,深數丈,中有破船。



 惠帝元康四年,蜀郡山崩,殺人。五月壬子,壽春山崩,洪水出,城壞,地陷方三十丈,殺人。六月,壽春大雷,山崩地坼,人家陷死,上庸亦如之。八月,居庸地裂,廣三十六丈,長八十四丈,水出,大飢。上庸四處山崩,地墜廣三十丈,長百三十丈,水出殺人。皆賈后亂朝之應也。



 太安元年四月,西墉崩。



 懷帝永嘉元年三月,洛陽東北步廣里地陷。二
 年八月乙亥,鄄城城無故自壞七十餘丈,司馬越惡之,遷于濮陽,此見沴之異也。越卒以陵上受禍。三年七月戊辰,當陽地裂三所,廣三丈,長三百餘步。京房《易傳》曰:「地坼裂者,臣下分離,不肯相從也。」其後司馬越茍晞交惡,四方牧伯莫不離散,王室遂亡。三年十月,宜都夷道山崩。四年四月,湘東需阜黑石山崩。



 元帝太興元年二月,廬陵、豫章、武昌,西陽地震山崩。二年五月,祁山地震,山崩,殺人。三年,南平郡山崩,出雄黃數千斤。時王敦陵傲,帝優容
 之,示含養禍萌也。四年八月,常山崩,水出,滹沲盈溢,大木傾拔。



 成帝咸和四年十月,柴桑廬山西北崖崩。十二月,劉胤為郭默所殺。



 穆帝永和七年九月,峻平、崇陽二陵崩。十二年十一月,遣散騎常侍車灌修峻平陵,開埏道,崩壓,殺數十人。



 升平五年二月,南掖門馬足陷地,得鐘一,有文四字。



 哀帝隆和元年四月丁丑,浩亹山崩,張天錫亡徵也。



 安帝義熙八年三月壬寅,山陰地陷,方四丈,有聲如雷。
 十年五月戊寅,西明門地穿,湧水出,毀門扇及限,此水沴土也。十一年五月,霍山崩,出銅鐘六枚。十三年七月,漢中成固縣水涯有聲若雷,既而岸崩,出銅鐘十有二枚。



 惠帝元康九年六月夜,暴雷雨,賈謐齋屋柱陷入地,壓謐床帳,此木沴土,土失其性,不能載也。明年,謐誅焉。



 光熙元年五月,范陽國地燃,可以爨,此火沴土也。是時,禮樂征伐自諸候出。



 《傳》曰:「皇之不極,是謂不建,厥咎眊,厥罰恒陰,厥極弱。時
 則有射妖,時則有龍蛇之孽,時則有馬禍,時則有下人伐上之痾,時則有日月亂行,星辰逆行。」皇之不極,是謂不建。皇,君;極,中;建,立也。人君貌言視聽思心五事皆失,不得其中,不能立萬事,失在眊悖,故其咎眊也。王者自下承天理物。雲起於山,而彌於天;天氣亂,故其罰恒陰,一曰:「上失中,則下彊盛而蔽君明也。」《易》曰:「亢龍有悔,貴而亡位,高而亡民,賢人在下位而亡輔。」如此,則君有南面之尊,而亡一人之助,故其極弱也。盛陽動進輕疾。禮,春而大射,以順陽氣。上微弱則下奮驚動,故有射妖。《易》曰:「雲從龍。」又曰:「龍蛇之蟄,以存身也。」陰氣動,故有龍蛇
 之孽。於《易》,《乾》為君,為馬。任用而強力,君氣毀,故有馬禍。一曰,馬多死及為怪,亦是也。君亂且弱,人之所叛,天之所去,不有明王之誅,則有篡殺之禍,故有下人伐上之痾。凡君道傷者,病天氣。不言五行沴天,而曰「日月亂行,星辰逆行」者,為若下不敢沴天,猶《春秋》曰「王師敗績于貿戎」,不言敗之者,以自敗為文,尊尊之意也。劉歆《皇極傳》曰有下體生於上之痾。說以為下人伐上,天誅已成,不得復為痾云。



 恒陰



 吳孫亮太平三年,自八月沈陰不雨,四十餘日。是時,將
 誅孫綝,謀泄。九月戊午,綝以兵圍宮,廢亮為會稽王,此恆陰之罰也。



 吳孫皓寶鼎元年十二月,太史奏久陰不雨,將有陰謀。孫皓驚懼。時陸凱等謀因其謁廟廢之。及出,留平領兵前驅,凱先語平,平不許,是以不果。皓既肆虐,群下多懷異圖,終至降亡



 射妖



 蜀車騎將軍鄧芝征涪陵,見玄猨緣山,手射中之。猨拔其箭,卷木葉塞其創。芝曰:「嘻!吾違物之性,其將死矣!」俄而卒,此射妖也。一曰,猨母抱子,芝射中之,子為拔箭,取
 木葉塞創。芝歎息,投弩水中,自知當死。



 恭帝為瑯邪王,好奇戲,嘗閑一馬於門內,令人射之,欲觀幾箭死,左右有諫者曰:「馬,國姓也。今射之,不祥。」於是乃止,而馬已被十許箭矣。此蓋射妖也。俄而禪位於宋焉。



 龍蛇之孽



 魏明帝青龍元年正月甲申,青龍見郟之摩陂井中。凡瑞興非時,則為妖孽,況困于井,非嘉祥矣。魏以改年,非也。干寶曰:「自明帝,終魏世,青龍、黃龍見者,皆其主興廢之應也。魏土運,青木色,而不勝于金。黃得位,青失位之
 象也。青能多見者,君德國運內相剋伐也。故高貴鄉公卒敗于兵。」案劉向說,龍貴象而困井中,諸侯將有幽執之禍也。魏世,龍莫不在井,此居上者逼制之應。高貴鄉公著《潛龍詩》,即此旨也。



 高貴鄉公正元元年十月戊戌,黃龍見于鄴井中。



 甘露元年正月辛丑,青龍見軹縣井中。六月乙丑,青龍見元城縣界井中。二年二月,青龍見溫縣井中。三年,黃龍、青龍俱見頓丘、冠軍、陽夏縣界井中。四年正月,黃龍二見寧陵縣界井中。



 元帝景元元年十二月甲申,黃龍見華陰縣井中。三年二月,龍見軹縣井中。



 吳孫皓天冊中,龍乳於長沙人家,啖雞雛。京房《易妖》曰:「龍乳人家,王者為庶人。」其後皓降晉。



 武帝咸寧二年六月丙午,白龍二見于九原井中。



 太康五年正月癸卯,二龍見武庫井中。帝觀之,有喜色。百僚將賀,劉毅獨表曰:「昔龍漦夏庭,禍發周室。龍見鄭門,子產不賀。」帝答曰:「朕德政未修,未有以應受嘉祥。」遂不賀也。孫盛曰:「龍,水物也,何與於人!子產言之當矣。但非其所處,實為妖災。夫龍以飛翔顯見為瑞,今則潛伏
 幽處,非休祥也。」漢惠帝二年,兩龍見蘭陵井中,本志以為其後趙王幽死之象。武庫者,帝王威御之器所寶藏也,屋宇邃密,非龍所處。是後七年,籓王相害,二十八年,果有二胡僭竊神器,二逆皆字曰龍,此之表異,為有證矣。



 愍帝建興二年十一月,枹罕羌妓產一龍子,色似錦,文常就母乳,遙見神光,少得就視。此亦皇之不建,於是帝竟淪沒。



 呂纂末,龍出東廂井中,到其殿前蟠臥,比旦失之。俄又有黑龍升其宮門。纂咸以為美瑞。或曰:「龍者陰類,出入
 有時,今而屢見,必有下人謀上之變。」後纂果為呂超所殺。



 武帝咸寧中,司徒府有二大蛇,長十許丈,居聽事平尞上而人不知,但數年怪府中數失小兒及豬犬之屬。後有一蛇夜出,被刃傷不能去,乃覺之,發徒攻擊,移時乃死。夫司徒,五教之府;此皇極不建,故蛇孽見之。漢靈帝時,蛇見御座,楊賜云為帝溺於色之應也。魏代宮人猥多,晉又過之,燕游是湎,此其孽也。《詩》云「惟虺惟蛇,女子之祥」也。



 惠帝元康五年三月癸巳,臨淄有大蛇,長十餘丈,負二
 小蛇入城北門,逕從市入漢城陽景王祠中,不見。天戒若曰,昔漢景王有定傾之功,而不厲節忠慎,以至失職奪功之辱。今齊王冏不寤,雖建興復之功,而驕陵取禍,此其徵也。



 明帝太寧初,武昌有大蛇,常居故神祠空樹中,每出頭從人受食。京房《易妖》曰:「蛇見於邑,不出三年有大兵,國有大憂。」尋有王敦之逆。



 馬禍



 武帝太熙元年,遼東有馬生角,在兩耳下,長三寸。案劉向說曰,「此兵象也」。及帝晏駕之後,王室毒於兵禍,是其
 應也。京房《易傳》曰:「臣易上,政不順,厥妖馬生角,茲謂賢士不足。」又曰:「天子親伐,馬生角。」《呂氏春秋》曰:「人君失道,馬有生角。」及惠帝踐阼,昏愚失道,又親征伐成都,是其應也。



 惠帝元康八年十二月,皇太子將釋奠,太傅趙王倫驂乘,至南城門,馬止,力士推之不能動。倫入軺車,乃進。此馬禍也。天戒若曰,倫不知義方,終為亂逆,非傅導行禮之人也。



 九年十一月戊寅,忽有牡騮馬驚奔至廷尉訊堂,悲鳴而死。天戒若曰,愍懷冤死之象也。見廷尉訊堂,其天意
 乎!



 懷帝永嘉六年二月,神馬鳴南城門。



 愍帝建興二年九月,蒲子縣馬生人。京房《易傳》曰:「上亡天子,諸侯相伐,厥妖馬生人。」是時,帝室衰微,不絕如線,胡狄交侵,兵戈日逼,尋而帝亦淪陷,故此妖見也。



 元帝太興二年,丹陽郡吏濮陽演馬生駒,兩頭,自項前別,生而死。司馬彪說曰:「此政在私門,二頭之象也。」其後王敦陵上。



 成帝咸康八年五月甲戌,有馬色赤如血,自宣陽門直走入于殿前,盤旋走出,尋逐,莫知所在。己卯,帝不豫。六
 月,崩。此馬禍,又赤祥也。是年,張重華在涼州,將誅其西河相張祚,廄馬數十匹,同時悉無後尾也。



 安帝隆安四年十月,梁州有馬生角,刺史郭銓送示桓玄。案劉向說曰,馬不當生角,猶玄不當舉兵向上也。玄不寤,以至夷滅。



 石季龍在鄴,有一馬尾有燒狀,入其中陽門,出顯陽門,東宮皆不得入,走向東北,俄爾不見。術者佛圖澄歎曰:「災其及矣!」逾年季龍死,其國遂滅。



 人痾



 魏文帝黃初初,清河宋士宗母化為鱉,入水。



 明帝太和三年,曹休部曲丘奚農女死復生。時又有開周世塚,得殉葬女子,數日而有氣,數月而不能言,郭太后愛養之。又,太原人發塚破棺,棺中有一生婦人,問其本事,不知也,視其墓木,可三十歲。案京房《易傳》曰:「至陰為陽,下人為上。」宣帝起之象也。漢平帝、獻帝並有此異,占以為王莽、曹操之徵。



 孫休永安四年,安吳民陳焦死七日復生,穿塚出。干寶曰:「此與漢宣帝同事,烏程侯皓承廢故之家,得位之祥也。」



 孫皓寶鼎元年,丹陽宣騫母年八十,因浴化為黿,兄弟
 閉戶衛之。掘堂上作大坎,實水其中,黿入坎遊戲,一二日恒延頸外望。伺戶小開,便輪轉自躍,入于遠潭,遂不復還。與漢靈帝時黃氏母同事,吳亡之象也。



 魏元帝咸熙二年八月,襄武縣言有大人見,長三丈餘,跡長三尺二寸,髮白,著黃巾黃單衣,柱杖呼王始語曰:「今當太平。」晉尋代魏。



 武帝泰始五年,元城人年七十生角。殆趙王倫篡亂之象也。



 咸寧二年十二月,琅邪人顏畿病死,棺斂已久,家人咸夢畿謂己曰;「我當復生,可急開棺。」遂出之,漸能飲食屈
 伸視瞻,不能行語,二年復死。京房《易傳》曰:「至陰為陽,下人為上,厥妖人死復生。」其後劉元海、石勒僭逆,遂亡晉室,下為上之應也。



 惠帝元康中,安豐有女子周世寧,年八歲,漸化為男,至十七八而氣性成。京房《易傳》曰:「女子化為丈夫,茲謂陰昌,賤人為王。」此亦劉元海、石勒蕩覆天下之妖也。



 永寧初,齊王冏唱義兵,誅除亂逆,乘輿反正。忽有婦人詣大司馬門求寄產,門者詰之,婦曰;「我截臍便去耳。」是時,齊王冏匡復王室,天下歸功,識者為其惡之,後果斬
 戮。



 永寧元年十二月甲子,有白頭公入齊王冏大司馬府,大呼曰:「有大兵起,不出甲子旬。」冏殺之。明年十二月戊辰,冏敗,即甲子旬也。



 太安元年四月癸酉,有人自雲龍門入殿前,北面再拜曰:「我當作中書監。」即收斬之。干寶以為「禁庭尊祕之處,今賤人徑入而門衛不覺者,宮室將虛而下人踰上之妖也」。是後帝北遷鄴,又遷長安,宮闕遂空焉。



 元康中,梁國女子許嫁,已受禮娉,尋而其夫戍長安,經年不歸,女家更以適人。女不樂行,其父母逼彊,不得已而去,尋得病亡。後其夫還,問其女所在,其家具說之。其
 夫逕至女墓,不勝哀情,便發冢開棺,女遂活,因與俱歸。後婿聞知,詣官爭之,所在不能決。秘書郎王導議曰:「此是非常事,不得以常理斷之,宜還前夫。」朝廷從其議。



 惠帝世,杜錫家葬而婢誤不得出,後十年開冢祔葬而婢尚生。始如瞑,有頃漸覺,問之,自謂再宿耳。初,婢之埋年十五六,及開冢更生,猶十五六也,嫁之有子。



 光熙元年,會稽謝真生子,頭大而有髮,兩蹠反向上,有男女兩體,生便作丈夫聲,經一日死。此皇之不極,下人伐上之痾,於是諸王有僭亂之象也。



 惠帝之世,京洛有人兼男女體,亦能兩用人道,而性尤
 淫,此亂氣所生。自咸寧、太康之後,男寵大興,甚於女色,士大夫莫不尚之,天下相仿傚,或至夫婦離絕,多生怨曠,故男女之氣亂而妖形作也。



 懷帝永嘉元年,吳郡吳縣萬詳婢生子,鳥頭,兩足馬蹄,一手,無毛,尾黃色,大如枕。此亦人妖,亂之象也。



 五年五月,枹罕令嚴根妓產一龍,一女,一鵝。京房《易傳》曰:「人生他物,非人所見者,皆為天下大兵。」是時,帝承惠皇之後,四海沸騰,尋而陷於平陽,為逆胡所害,此其徵也。



 愍帝建興四年,新蔡縣吏任僑妻產二女,腹與心相合,
 自胸以上、臍以下各分,此蓋天下未一之妖也。時內史呂會上言:「案《瑞應圖》,異根同體謂之連理,異畝同穎謂之嘉禾。草木之異猶以為瑞,今二人同心,《易》稱『二人同心,其利斷金』,蓋四海同心之瑞也。」時皆哂之。俄而四海分崩,帝亦淪沒。



 元帝太興初,有女子其陰在腹,當臍下,自中國來至江東,其性淫而不產。又有女子陰在首,渡在揚州,性亦淫。京房《易妖》曰;「人生子,陰在首,天下大亂;在腹,天下有事;在背,天下無後。」于時王敦據上流,將欲為亂,是其征。



 三年十二月,尚書騶謝平妻生女,墮地濞濞有聲,須臾
 便死。鼻目皆在頂上,面處如項,口有齒,都連為一,胸如鱉,手足爪如鳥爪,皆下勾。此亦人生他物,非人所見者。後二年,有石頭之敗。



 明帝太寧二年七月,丹陽江寧侯紀妻死,經三日復生。



 成帝咸康五年四月,下邳民王和僑居暨陽,息女可年二十,自云上天來還,得征瑞印綬,當母天下。晉陵太守以為妖,收付獄。至十一月,有人持柘杖絳衣詣止車門,口列為聖人使求見天子。門侯受辭,辭稱姓呂名賜,其言王和女可右足下有七星,星皆有毛,長七寸,天今命可為天下母。奏聞,即伏誅,并下晉陵誅可。



 康帝建元二年十月,衛將軍營督過望所領兵陳瀆女臺有文在其足,曰「天下之母」,灸之愈明。京都喧譁,有司收繫以聞。俄自建康縣獄亡去。明年,帝崩,獻后臨朝,此其祥也。



 孝武帝寧康初,南郡州陵女唐氏漸化為丈夫。



 安帝義熙七年,無錫人趙未年八歲,一旦暴長八尺,髭鬚蔚然,三日而死。



 義熙中,東陽人莫氏生女不養,埋之數日,於土中啼,取養遂活。



 義熙末,吳豫章人有二陽道,重累生。



 恭帝元熙元年,建安人陽道無頭,正平,本下作女人形體。



\end{pinyinscope}