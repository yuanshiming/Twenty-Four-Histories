\article{列傳第三十}

\begin{pinyinscope}
解系
 \gezhu{
  弟結結弟育}
 孫旂孟觀牽秀繆播
 \gezhu{
  從弟胤}
 皇甫重張輔李含張方閻鼎索靖
 \gezhu{
  子綝}
 賈疋



 解系,字少連,濟南著人也。父脩,魏瑯邪太守、梁州刺史,考績為天下第一。武帝受禪,封梁鄒侯。系及二弟結、育並清身潔己,甚得聲譽。時荀勖門宗彊盛,朝野畏憚之。勖諸子謂系等曰:「我與卿為友,應向我公拜。」勖又曰:「我與尊先使君親厚。」系曰:「不奉先君遺教。公若與先君厚,往日哀頓,當垂書問。親厚之誨,非所敢承。」勖父子大慚,
 當世壯之。後辟公府掾,歷中書黃門侍郎、散騎常侍、豫州刺史,遷尚書,出為雍州刺史、揚烈將軍、西戎校尉、假節。會氐羌叛,與征西將軍趙王倫討之。倫信用佞人孫秀,與系爭軍事,更相表奏。朝廷知系守正不撓,而召倫還。系表殺秀以謝氐羌,不從。倫、秀譖之,系坐免官,以白衣還第,闔門自守。及張華、裴頠之被誅也,倫、秀以宿憾收系兄弟。梁王肜救系等,倫怒曰:「我於水中見蟹且惡之,況此人兄弟輕我邪!此而可忍,孰不可忍!」肜苦爭之不得,遂害之,並戮其妻子。



 後齊王冏起義時,以裴、解為冤首。倫、秀既誅,冏乃奏曰:「臣聞興微繼絕,聖主之高政;
 貶惡嘉善,《春秋》之美談。是以武王封比干之墓,表商容之閭,誠幽明之故有以相通也。孫秀逆亂,滅佐命之國,誅骨鯁之臣,以斲喪王室,肆其虐戾,功臣之後,多見泯滅。至如張華、裴頠,各以見憚取誅於時,系、結同以羔羊被害,歐陽建等無罪而死,百姓憐之。陛下更日月之光照,布惟新之明命,然此等未蒙恩理。昔欒郤降在皁隸,而《春秋》傳其人;幽王絕功臣之後,棄賢者子孫,而詩人以為刺。臣備忝右職,思竭股肱,獻納愚誠。若合聖意,可群官通議。」八坐議以「系等清公正直,為奸邪所疾,無罪橫戮,冤痛已甚。如大司馬所啟,彰明枉直,顯宣當否,使
 冤魂無愧無恨,為恩大矣。」永寧二年,追贈光祿大夫,改葬,加弔祭焉。



 結字叔連,少與系齊名。辟公府掾,累遷黃門侍郎,歷散騎常恃、豫州刺史、魏郡太守、御史中丞。時孫秀亂關中,結在都,坐議秀罪應誅,秀由是致憾。及系被害,結亦同戮。女適裴氏,明日當嫁,而禍起,裴氏欲認活之,女曰:「家既若此,我何活為!」亦坐死。朝廷遂議革舊制,女不從坐,由結女始也。後贈結光祿大夫,改葬,加弔祭。



 結弟育,字稚連,名亞二兄。歷公府掾、太子洗馬、尚書郎、衛軍長史、弘農太守,與二兄俱被害,妻子徙邊。



 孫旂,字伯旗,樂安人也。父歷,魏晉際為幽州刺史、右將軍。旂潔靜,少自脩立。察孝廉,累遷黃門侍郎,出為荊州刺史,名位與二解相亞。永熙中,徵拜太子詹事,轉衛尉,坐武庫火,免官。歲餘,出為兗州刺史,遷平南將軍、假節。旂子弼及弟子髦、輔、琰四人,並有吏材,稱於當世,遂與孫秀合族。及趙王倫起事,夜從秀開神武門下觀閱器械。兄弟旬月相次為公府掾、尚書郎。弼又為中堅將軍,領尚書左丞,轉為上將軍,領射聲校尉。髦為武衛將軍,領太子詹事。琰為武威將軍,領太子左率。皆賜爵開國
 郡侯。推崇旂為車騎將軍、開府。初,旂以弼等受署偽朝,遣小息回責讓弼等,以過差之事,必為家禍。弼等終不從,旂制之不可,但慟哭而已。及齊王冏起義,四子皆伏誅。襄陽太守宗岱承冏檄斬旂,夷三族。



 弟尹,字文旗,歷陳留、陽平太守,早卒。



 孟觀,字叔時,渤海東光人也。少好讀書,解天文。惠帝即位,稍遷殿中中郎。賈后悖婦姑之禮,陰欲誅楊駿而廢太后,因駿專權,數言之於帝,又使人諷觀。會楚王瑋將討駿,觀受賈后旨宣詔,頗加誣其事。及駿誅,以觀為黃
 門侍郎,特給親信四十人。遷積弩將軍,封上谷郡公。氐帥齊萬年反於關中,眾數十萬,諸將覆敗相繼。中書令陳準、監張華,以趙、梁諸王在關中,雍容貴戚,進不貪功,退不懼罪,士卒雖眾,不為之用,周處喪敗,職此之由,上下離心,難以勝敵。以觀沈毅,有文武材用,乃啟觀討之。觀所領宿衛兵,皆趫捷勇悍,並統關中士卒,身當矢石,大戰十數,皆破之,生擒萬年,威懾氐羌。轉東羌校尉,徵拜右將軍。



 趙王倫篡位,以觀所在著績,署為安南將軍、監河北諸軍事、假節,屯宛。觀子平為淮南王允前鋒將軍,討倫,戰死。孫秀以觀杖兵在外,假言平為允兵所害,
 贈積弩將軍以安觀。義軍既起,多勸觀應齊王冏,觀以紫宮帝坐無他變,謂倫應之,遂不從眾議而為倫守。及帝反正,永饒冶令空桐機斬觀首,傳于洛陽,遂夷三族。



 牽秀,字成叔,武邑觀津人也。祖招,魏鴈門太守。秀博辯有文才,性豪俠,弱冠得美名,為太保衛瓘、尚書崔洪所知。太康中,調補新安令,累遷司空從事中郎。與帝舅王愷素相輕侮,愷諷司隸荀愷奏秀夜在道中載高平國守士田興妻。秀即表訴被誣,論愷穢行,文辭亢厲,以譏抵外戚。于時朝臣雖多證明其行,而秀盛名美譽由是
 而損,遂坐免官。後司空張華請為長史。



 秀任氣,好為將帥。張昌作亂,長沙王乂遣秀討昌,秀出關,因奔成都王穎。穎伐乂,以秀為冠軍將軍,與陸機、王粹等共為河橋之役。機戰敗,秀證成其罪,又諂事黃門孟玖,故見親於穎。惠帝西幸長安,以秀為尚書。秀少在京輦,見司隸劉毅奏事而扼腕慷慨,自謂居司直之任,當能激濁揚清;處鼓鞞之間,必建將帥之勛。及在常伯納言,亦未曾有規獻弼違之奇也。



 河間王顒甚親任之。關東諸軍奉迎大駕,以秀為平北將軍,鎮馮翊。秀與顒將馬瞻等將輔顒以守關中,顒密遣使就東海王越求迎,越遣將麋晃
 等迎顒。時秀擁眾在馮翊,晃不敢進。顒長史楊騰前不應越軍,懼越討之,欲取秀以自效,與馮翊大姓諸嚴詐稱顒命,使秀罷兵,秀信之,騰遂殺秀於萬年。



 繆播,字宣則,蘭陵人也。父悅,光祿大夫。播才思清辯,有意義。高密王泰為司空,以播為祭酒,累遷太弟中庶子。



 惠帝幸長安,河間王顒欲挾天子令諸侯。東海王越將起兵奉迎天子,以播父時故吏,委以心膂。播從弟右衛率胤,顒前妃之弟也。越遣播、胤詣長安說顒,令奉帝還洛,約與顒分陜為伯。播、胤素為顒所敬信,既相見,虛懷
 從之。顒將張方自以罪重,懼為誅首,謂顒曰:「今據形勝之地,國富兵彊,奉天子以號令,誰敢不服!」顒惑方所謀,猶豫不決。方惡播、胤為越游說,陰欲殺之。播等亦慮方為難,不敢復言。時越兵鋒甚盛,顒深憂之,播、胤乃復說顒,急斬方以謝,可不勞而安。顒從之,於是斬方以謝山東諸侯。顒後悔之,又以兵距越,屢為越所敗。帝反舊都,播亦從太弟還洛,契闊艱難,深相親狎。



 及帝崩,太弟即帝位,是為懷帝,以播為給事黃門侍郎。俄轉侍中,徙中書令,任遇日隆,專管詔命。時越威權自己,帝力不能討,心甚惡之。以播、胤等有公輔之量,又盡忠於國,故委以
 心膂。越懼為己害,因入朝,以兵入宮,執播等於帝側。帝歎曰:「姦臣賊子無世無之,不自我先,不自我後,哀哉!」起執播等手,涕泗歔欷,不能自禁。越遂害之。朝野憤惋,咸曰:「善人,國之紀也,而加虐焉,其能終乎!」及越薨,帝贈播衛尉,祠以少牢。



 胤字休祖,安平獻王外孫也,與播名譽略齊。初為尚書郎,後遷太弟左衛率,轉魏郡太守。及王浚軍逼鄴,石超等大敗,胤奔東海王越於徐州,越使胤與播俱入關,而所說得行,大駕東還。越以胤為冠軍將軍、南陽太守。胤從藍田出武關,之南陽,前守衛展距胤不受,胤乃還洛。
 懷帝即位,拜胤左衛將軍,轉散騎常侍、太僕卿。既而與播及帝舅王延、尚書何綏、太史令高堂沖並參機密,為東海王越所害。



 皇甫重,字倫叔,安定朝那人也。性沈果,有才用,為司空張華所知,稍遷新平太守。元康中,華版為秦州刺史。齊王冏輔政,以重弟商為參軍。冏誅,長沙王乂又以為參軍。時河間王顒鎮關中,其將李含先與商、重有隙,每銜之,及此,說顒曰:「商為乂所任,重終不為人用,宜急除之,以去一方之患。可表遷重為內職,因其經長安,乃執之。」
 重知其謀,乃露檄上尚書,以顒信任李含,將欲為亂,召集隴上士眾,以討含為名。乂以兵革累興,今始寧息,表請遣使詔重罷兵,徵含為河南尹。含既就徵,重不奉詔,顒遣金城太守游楷、隴西太守韓稚等四郡兵攻之。



 頃之,成都王穎與顒起兵共攻乂,以討后父尚書僕射羊玄之及商為名。乂以商為左將軍、河東太守,領萬餘人於關門距張方,為方所破,顒軍遂進。乂既屢敗,乃使商間行齎帝手詔,使游楷盡罷兵,令重進軍討顒。商行過長安,至新平,遇其從甥,從甥素憎商,以告顒,顒捕得商,殺之。乂既敗,重猶堅守,閉塞外門,城內莫知,而四郡兵
 築土山攻城,重輒以連弩射之。所在為地窟以防外攻,權變百端,外軍不得近城,將士為之死戰。顒知不可拔,乃上表求遣御史宣詔喻之令降。重知非朝廷本意,不奉詔。獲御史騶人問曰:「我弟將兵來,欲至未?」騶云:「已為河間王所害。」重失色,立殺騶。於是城內知無外救,遂共殺重。



 先是,重被圍急,遣養子昌請救於東海王越,越以顒新廢成都王穎,與山東連和,不肯出兵。昌乃與故殿中人楊篇詐稱越命,迎羊后於金墉城入宮,以后令發兵討張方,奉迎大駕。事起倉卒,百官初皆從之,俄而又共誅昌。



 張輔,字世偉,南陽西鄂人,漢河間相衡之後也。少有幹局,與從母兄劉喬齊名。初補藍田令,不為豪彊所屈。時彊弩將軍龐宗,西州大姓,護軍趙浚,宗婦族也,故僮僕放縱,為百姓所患。輔繩之,殺其二奴,又奪宗田二百餘頃以給貧戶,一縣稱之。轉山陽令,太尉陳準家僮亦暴橫,輔復擊殺之。累遷尚書郎,封宜昌亭侯。



 轉御史中丞。時積弩將軍孟觀與明威將軍郝彥不協,而觀因軍事害彥,又賈謐、潘岳、石崇等共相引重,乃義陽王威有詐冒事,輔並糾劾之。梁州刺史楊欣有姊喪,未經旬,車騎
 長史韓預彊聘其女為妻。輔為中正,貶預以清風俗,論者稱之。用孫秀執權,威構輔於秀,秀惑之,將繩輔以法。輔與秀箋曰:「輔徒知希慕古人,當官而行,不復自知小為身計。今義陽王誠弘恕,不以介意。然輔母年七十六,常見憂慮,恐輔將以怨疾獲罪。願明公留神省察輔前後行事,是國之愚臣而已。「秀雖凶狡,知輔雅正,為威所誣,乃止。



 後遷馮翊太守。是時長沙王乂以河間王顒專制關中,有不臣之跡,言於惠帝,密詔雍州刺史劉沈、秦州刺史皇甫重使討顒。於是沈等與顒戰於長安,輔遂將兵救顒,沈等敗績。顒德之,乃以輔代重為秦州刺史。
 當赴顒之難,金城太守游楷亦皆有功,轉梁州刺史,不之官。楷聞輔之還,不時迎輔,陰圖之。又殺天水太守封尚,欲揚威西土。召隴西太守韓稚會議,未決。稚子朴有武幹,斬異議者,即收兵伐輔。輔與稚戰於遮多谷口,輔軍敗績,為天水故帳下督富整所殺。



 初,輔嘗著論云:「管仲不若鮑叔,鮑叔知所奉,知所投。管仲奉主而不能濟,所奔又非濟事之國,三歸反坫,皆鮑不為。」又論班固、司馬遷云:「遷之著述,辭約而事舉,敘三千年事唯五十萬言;班固敘二百年事乃八十萬言,煩省不同,不如遷一也。良史述事,善足以獎勸,惡足以監誡,人道之常。中流小事,
 亦無取焉,而班皆書之,不如二也。毀貶晁錯,傷忠臣之道,不如三也。遷既造創,固又因循,難易益不同矣。又遷為蘇秦、張儀、范睢、蔡澤作傳,逞辭流離,亦足以明其大才。故述辯士則辭藻華靡,敘實錄則隱核名檢,此所以遷稱良史也。」又論魏武帝不及劉備,樂毅減於諸葛亮,詞多不載。



 李含,字世容,隴西狄道人也。僑居始平。少有才幹,兩郡並舉孝廉。安定皇甫商州里年少,少恃豪族,以含門寒微,欲與結交,含距而不納,商恨焉,遂諷州以短檄召含
 為門亭長。會州刺史郭奕素聞其賢,下車擢含為別駕,遂處群僚之右。尋舉秀才,薦之公府,自太保掾轉秦國郎中令。司徒遷含領始平中正。秦王柬薨,含依臺儀,葬訖除喪。尚書趙浚有內寵,疾含不事己,遂奏含不應除喪。本州大中正傅祗以名義貶含。中丞傅咸上表理含曰:



 臣州秦國郎中令始平李含,忠公清正,才經世務,實有史魚秉直之風。雖以此不能協和流俗,然其名行峻厲,不可得掩,二郡並舉孝廉異行。尚書郭奕臨州,含寒門少年,而奕超為別駕。太保衛瓘辟含為掾,每語臣曰:「李世容當為晉匪躬之臣。」



 秦王之薨,悲慟感人,百僚會
 喪,皆所目見。而今以含俯就王制,謂之背戚居榮,奪其中正。天王之朝,既葬不除,籓國之喪,既葬而除。籓國欲同不除,乃當責引尊準卑,非所宜言耳。今天朝告于上,欲令籓國服于下,此為籓國之義隆,而天朝之禮薄也。又云諸王公皆終喪,禮寧盡乃敘,明以喪制宜隆,務在敦重也。夫寧盡乃敘,明以哀其病耳。異於天朝,制使終喪,未見斯文。國制既葬而除,既除而祔。爰自漢魏迄于聖晉,文皇升遐,武帝崩殂,世祖過哀,陛下毀頓,銜疚諒闇,以終三年,率土臣妾豈無攀慕遂服之心,實以國制不可而踰,故於既葬不敢不除。天王之喪,釋除於上,籓
 國之臣,獨遂於下,此不可安。復以秦王無後,含應為喪主,而王喪既除而附,則應吉祭。因曰王未有廟,主不應除服。秦王始封,無所連祔,靈主所居,即便為廟。不問國制云何,而以無廟為貶。以含今日之所行,移博士使案禮文,必也放勛之殂,遏密三載,世祖之崩,數旬即吉,引古繩今,闔世有貶,何但李含不應除服。今也無貶,王制故也。聖上諒闇,哀聲不輟,股肱近侍,猶宜心喪,不宜便行婚娶歡樂之事,而莫云者,豈不以大制不可而曲邪?且前以含有王喪,上為差代。尚書敕王葬日在近,葬訖,含應攝職,不聽差代。葬訖,含猶躊躇,司徒屢罰訪問,踧
 含攝職,而隨擊之,此為臺敕府符陷含於惡。若謂臺府為傷教義,則當據正,不正符敕,唯含是貶,含之困躓尚足惜乎!國制不可偏耳。



 又含自以隴西人,雖戶屬始平,非所綜悉。自初見使為中正,反復言辭,說非始平國人,不宜為中正。後為郎中令,又自以選官引臺府為比,以讓常山太守蘇韶,辭意懇切,形于文墨。含之固讓,乃在王未薨之前,葬後躊躇,窮於對罰而攝職耳。臣從弟祗為州都,意在欲隆風教,議含已過,不良之人遂相扇動,冀挾名義,法外致案,足有所邀,中正龐騰便割含品。臣雖無祁大夫之德,見含為騰所侮,謹表以聞,乞朝廷
 以時博議,無令騰得妄弄刀尺。



 帝不從,含遂被貶,退割為五品。歸長安,歲餘,光祿差含為壽城邸閣督。司徒王戎表含曾為大臣,雖見割削,不應降為此職。詔停。後為始平令。



 及趙王倫篡位,或謂孫秀曰:「李含有文武大才,無以資人。」秀以為東武陽令。河間王顒表請含為征西司馬,甚見信任。頃之,轉為長史。顒誅夏侯奭,送齊王冏使與趙王倫,遣張方率眾赴倫,皆含謀也。後顒聞三王兵盛,乃加含龍驤將軍,統席薳等鐵騎,回遣張方軍以應義師。天子反正,含至潼關而還。



 初,梁州刺史皇甫商為趙王倫所任,倫敗,去職詣顒,顒慰撫之甚厚。含諫顒
 曰:「商,倫之信臣,懼罪至此,不宜數與相見。」商知而恨之。及商當還都,顒置酒餞行,商因與含忿爭,顒和釋之。後含被徵為翊軍校尉。時商參齊王冏軍事,而夏侯奭兄在冏府,稱奭立義,被西籓枉害。含心不自安。冏右司馬趙驤又與含有隙,冏將閱武,含懼驤因兵討之,乃單馬出奔于顒,矯稱受密詔。顒即夜見之,乃說顒曰:「成都王至親,有大功,還籓,甚得眾心。齊王越親而專執威權,朝廷側目。今檄長沙王令討齊,使先聞於齊,齊必誅長沙,因傳檄以加齊罪,則冏可擒也。既去齊,立成都,除逼建親,以安社稷,大勳也。」顒從之,遂表請討冏,拜含為都督,
 統張方等率諸軍以向洛陽。含屯陰盤,而長沙王乂誅冏,含等旋師。



 初,含之本謀欲并去乂、冏,使權歸於顒,含因得肆其宿志。既長沙勝齊,顒、穎猶各守籓,志望未允。顒表含為河南尹。時商復被乂任遇,商兄重時為秦州刺史,含疾商滋甚,復與重構隙。顒自含奔還之後,委以心膂,復慮重襲己,乃使兵圍之,更相表罪。侍中馮蓀黨顒,請召重還。商說乂曰:「河間之奏,皆李含所交構也。若不早圖,禍將至矣。且河間前舉,由含之謀。」乂乃殺含。



 張方,河間人也。世貧賤,以材勇得幸於河間王顒,累遷
 兼振武將軍。永寧中,顒表討齊王冏,遣方領兵二萬為前鋒。及冏被長沙王乂所殺,顒及成都王穎復表討乂,遣方率眾自函谷人屯河南。惠帝遣左將軍皇甫商距之,方以潛軍破商之眾,遂入城。乂奉帝討方于城內,方軍望見乘輿,於是小退,方止之不得,眾遂大敗,殺傷滿於衢巷。方退壁于十三里橋,人情挫衄,無復固志,多勸方夜遁。方曰:「兵之利鈍是常,貴因敗以為成耳。我更前作壘,出其不意,此用兵之奇也。」乃夜潛進逼洛城七里。乂既新捷,不以為意,忽聞方壘成,乃出戰,敗績。東海王越等執乂,送于金墉城。方使郅輔取乂還營,炙殺之。於
 是大掠洛中官私奴婢萬餘人,而西還長安。顒加方右將軍、馮翊太守。



 蕩陰之役,顒又遣方鎮洛陽,上官已、苗願等距之,大敗而退。清河王覃夜襲已、願,已、願出奔,方乃入洛陽。覃於廣陽門迎方而拜,方馳下車扶止之。於是復廢皇后羊氏。及帝自鄴還洛,方遣息羆以三千騎奉迎。將渡河橋,方又以所乘陽燧車、青蓋素升三百人為小鹵簿,迎帝至芒山下。方自帥萬餘騎奉雲母輿及旌旗之飾,衛帝而進。初,方見帝將拜,帝下車自止之。



 方在洛既久,兵士暴掠,發哀獻皇女墓。軍人喧喧,無復留意,議欲西遷,尚匿其跡,欲須天子出,因劫移都。乃請帝
 謁廟,帝不許。方遂悉引兵入殿迎帝,帝見兵至,避之於竹林中,軍人引帝出,方於馬上稽首曰:「胡賊縱逸,宿衛單少,陛下今日幸臣壘,臣當捍禦寇難,致死無二。」於是軍人便亂入宮閣,爭割流蘇武帳而為馬帴。方奉帝至弘農,顒遣司馬周弼報方,欲廢太弟,方以為不可。



 帝至長安,以方為中領軍、錄尚書事,領京兆太守。時豫州刺史劉喬檄稱潁川太守劉輿迫脅范陽王虓距逆詔命,及東海王越等起兵於山東,乃遣方率步騎十萬往討之。方屯兵霸上,而劉喬為虓等所破。顒聞喬敗,大懼,將罷兵,恐方不從,遲疑未決。



 初,方從山東來,甚微賤,長安
 富人郅輔厚相供給。及貴,以輔為帳下督,甚暱之。顒參軍畢垣,河間冠族,為方所侮,忿而說顒曰:「張方久屯霸上,聞山東賊盛,盤桓不進,宜防其未萌。其親信郅輔具知其謀矣。」而繆播等先亦構之,顒因使召輔,垣迎說輔曰:「張方欲反,人謂卿知之。王若問卿,何辭以對?」輔驚曰:「實不聞方反,為之若何?」垣曰:「王若問卿,但言爾爾。不然,必不免禍。」輔既入,顒問之曰:「張方反,卿知之乎?」輔曰:「爾。」顒曰:「遣卿取之可乎?」又曰:「爾。」顒於是使輔送書於方,因令殺之。輔既暱於方,持刀而入,守閣者不疑,因火下發函,便斬方頭。顒以輔為安定太守。初繆播等議斬方,送
 首與越,冀東軍可罷。及聞方死,更爭入關,顒頗恨之,又使人殺輔。



 史臣曰:晉氏之禍難薦臻,實始籓翰。解系等以干時之用,處危亂之辰,並託迹府朝,參謀王室。或抗忠盡節,或飾詐懷姦。雖邪正殊途,而咸至誅戮,豈非時艱政紊,利深禍速者乎!古人所以危邦不入,亂邦不居,戒懼於此也。



 閻鼎,字台臣,天水人也。初為太傅東海王越參軍,轉卷令,行豫州刺史事,屯許昌。遭母喪,乃於密縣間鳩聚西
 州流人數千,欲還鄉里。值京師失守,秦王出奔密中,司空荀籓、籓弟司隸校尉組,及中領軍華恒、河南尹華薈,在密縣建立行臺,以密近賊,南趣許潁。司徒左長史劉疇在密為塢主,中書令李恆、太傅參軍騶捷劉蔚、鎮軍長史周顗、司馬李述皆來赴疇。僉以鼎有才用,且手握強兵,勸籓假鼎冠軍將軍、豫州刺史,蔚等為參佐。



 鼎少有大志,因西土人思歸,欲立功鄉里,乃與撫軍長史王毗、司馬傳遜懷翼戴秦王之計,謂疇、捷等曰:「山東非霸王處,不如關中。」河陽令傅暢遺鼎書,勸奉秦王過洛陽,謁拜山陵,徑據長安,綏合夷晉,興起義眾,剋復宗廟,雪
 社稷之恥。鼎得書,便欲詣洛,流人謂北道近河,懼有抄截,欲南自武關向長安。疇等皆山東人,咸不願西入,荀籓及疇、捷等並逃散。鼎追籓不及,恆等見殺,唯顗、述走得免。遂奉秦王行,止上洛,為山賊所襲,殺百餘人,率餘眾西至藍田。時劉聰向長安,為雍州刺史賈疋所逐,走還平陽。疋遣人奉迎秦王,遂至長安,而與大司馬南陽王保、衛將軍梁芬、京兆尹梁綜等並同心推戴,立王為皇太子,登壇告天,立社稷宗廟,以鼎為太子詹事,總攝百揆。



 梁綜與鼎爭權,鼎殺綜,以王毗為京兆尹。鼎首建大謀,立功天下。始平太守曲允、撫夷護軍索綝並害其
 功,且欲專權,馮翊太守梁緯、北地太守梁肅,並綜母弟,綝之姻也,謀欲除鼎,乃證其有無君之心,專戮大臣,請討之,遂攻鼎。鼎出奔雍,為氐竇首所殺,傳首長安。



 索靖,字幼安,敦煌人也。累世官族,父湛,北地太守。靖少有逸群之量,與鄉人氾衷、張甝、索糸、索永俱詣太學,馳名海內,號稱「敦煌五龍」。四人並早亡,唯靖該博經史,兼通內緯。州辟別駕,郡舉賢良方正,對策高第。傅玄、張華與靖一面,皆厚與之相結。拜駙馬都尉,出為西域戊己校尉長史。太子僕同郡張勃特表,以靖才藝絕人,宜在
 臺閣,不宜遠出邊塞。武帝納之,擢為尚書郎。與襄陽羅尚、河南潘岳、吳郡顧榮同官,咸器服焉。靖與尚書令衛瓘俱以善草書知名,帝愛之。瓘筆勝靖,然有楷法,遠不能及靖。



 靖在臺積年,除雁門太守,遷魯相,又拜酒泉太守。惠帝即位,賜爵關內侯。



 靖有先識遠量,知天下將亂,指洛陽宮門銅駝,歎曰:「會見汝在荊棘中耳!」



 元康中,西戎反叛,拜靖大將軍梁王肜左司馬,加蕩寇將軍,屯兵粟邑,擊賊,敗之。遷始平內史。及趙王倫篡位,靖應三王義舉,以左衛將軍討孫秀有功,加散騎常侍,遷後將軍。太安末,河間王顒舉兵向洛陽,拜靖使持節、監洛城諸
 軍事、游擊將軍,領雍、秦、涼義兵,與賊戰,大破之,靖亦被傷而卒,追贈太常,時年六十五。後又贈司空,進封安樂亭侯,謚曰莊。



 靖著《五行三統正驗論》,辯理陰陽氣運。又撰《索子》、《晉詩》各二十卷。又作《草書狀》,其辭曰:



 聖皇御世,隨時之宜。倉頡既生,書契是為。科斗烏篆,類物象形。睿哲變通,意巧茲生。損之隸草,以崇簡易。百官畢脩,事業並麗。蓋草書之為狀也,婉若銀鉤,漂若驚鸞。舒翼未發,若舉復安;蟲蛇虯蟉,或往或還。類阿那以羸形,欻奮釁而桓桓。及其逸遊肸向,乍正乍邪。騏驥暴怒逼其轡,海水窊隆揚其波。芝草蒲陶還相繼,棠棣融融載其華。玄
 熊對踞于山嶽,飛燕相追而差池。舉而察之,又似乎和風吹林,偃草扇樹。枝條順氣,轉相比附,窈嬈廉苫,隨體散布。紛擾擾以猗靡,中持疑而猶豫。玄螭狡獸嬉其間,騰猿飛猿相奔趣。凌魚奮尾,蛟龍反據。投空自竄,張設牙距。或若登高望其類,或若既往而中顧,或若俶儻而不群,或若自檢於常度。於是多才之英,篤藝之彥,役心精微,耽此文憲。守道兼權,觸類生變。離析八體,靡形不判。去繁存微,大象未亂。上理開元,下周謹案。騁辭放手,雨行冰散。高音翰厲,溢越流漫。忽班班而成章,信奇妙之煥爛。體磥落而壯麗,姿光潤以粲粲。命杜度運其指,
 使伯英回其腕。著絕勢於紈素,垂百世之殊觀。



 先時,靖行見姑臧城南石地,曰:「此後當起宮殿。」至張駿,於其地立南城,起宗廟,建宮殿焉。



 靖有五子:鯁、綣、璆、聿、綝,皆舉秀才。聿,安昌鄉侯,卒。少子綝最知名。



 綝字巨秀,少有逸群之量,靖每曰;「綝廊廟之才,非簡札之用,州郡吏不足汗吾兒也。」舉秀才,除郎中。嘗報兄仇,手殺三十七人,時人壯之。俄轉太宰參軍,除好畤令,人為黃門侍郎,出參征西軍事,轉長安令,在官有稱。



 及成都王穎劫遷惠帝幸鄴,穎為王浚所破,帝遂播越。河間王顒使張方及綝東迎乘輿,以功拜鷹楊將軍,轉南陽
 王模從事中郎。劉聰侵掠關東,以綝為奮威將軍以禦之,斬聰將呂逸,又破聰黨劉豐,遷新平太守。聰將蘇鐵、劉五斗等劫掠三輔,除綝安西將軍、馮翊太守。綝有威恩,華夷嚮服,賊不敢犯。



 及懷帝蒙塵,長安又陷,模被害,綝泣曰:「與其俱死,寧為伍子胥。」乃赴安定,與雍州刺史賈疋、扶風太守梁綜、安夷護軍麴允等糾合義眾,頻破賊黨,脩復舊館,遷定宗廟。進救新平,小大百戰,綝手擒賊帥李羌,與閻鼎立秦王為皇太子,及即尊位,是為愍帝。綝遷侍中、太僕,以首迎大駕、升壇授璽之功,封弋居伯。又遷前將軍、尚書右僕射、領吏部、京兆尹,加平東將
 軍,進號征東。尋又詔曰:「朕昔遇厄運,遭家不造,播越宛楚,爰失舊京。幸宗廟寵靈,百辟宣力,得從籓衛,託乎群公之上。社稷之不隕,實公是賴,宜贊百揆,傅弼朕躬。其授衛將軍,領太尉,位特進,軍國之事悉以委之。」



 及劉曜侵逼王城,以綝為都督征東大將軍,持節討之。破曜呼日逐王呼延莫,以功封上洛郡公,食邑萬戶,拜夫人荀氏為新豐君,子石元為世子,賜子弟二人鄉亭侯。劉曜入關芟麥苗,綝又擊破之。自長安伐劉聰,聰將趙染杖其累捷,有自矜之色,帥精騎數百與綝戰,大敗之,染單馬而走。轉驃騎大將軍、尚書左僕射、錄尚書,承制行
 事。



 劉曜復率眾人馮翊,帝累徵兵於南陽王保,保左右議曰;「蝮蛇在手,壯士解其腕。且斷隴道,以觀其變。」從事中郎裴詵曰:「蛇已螫頭,頭可截不?」保以胡崧行前鋒都督,須諸軍集,乃當發。麴允欲挾天子趣保,綝以保必逞私欲,乃止。自長安以西,不復奉朝廷。百官飢乏,採穭自存。時三秦人尹桓、解武等數千家,盜發漢霸、杜二陵,多獲珍寶。帝問綝曰:「漢陵中物何乃多邪?」綝對曰:「漢天子即位一年而為陵,天下貢賦三分之,一供宗廟,一供賓客,一充山陵。漢武帝饗年久長,比崩而茂陵不復容物,其樹皆已可拱。赤眉取陵中物不能減半,于今猶有朽
 帛委積,珠玉未盡。此二陵是儉者耳,亦百世之誡也。」



 後劉曜又率眾圍京城、綝與麴允固守長安小城。胡崧承檄奔命,破曜于靈臺。崧慮國家威舉,則麴、索功盛,乃案兵渭北,遂還槐里。城中飢窘,人相食,死亡逃奔不可制,唯涼州義眾千人守死不移。帝使侍中宋敞送箋降於曜。綝潛留敞,使其子說曜曰:「今城中食猶足支一歲,未易可剋也。若許綝以車騎、儀同、萬戶郡公者,請以城降。」曜斬而送之曰:「帝王之師,以義行也。孤將軍十五年,未嘗以譎詭敗人,必窮兵極勢,然後取之。今索綝所說如是,天下之惡一也,輒相為戮之。若審兵食未盡者,便可
 勉強固守。如其糧竭兵微,亦宜早悟天命。孤恐霜威一震,玉石俱摧。」及帝出降,綝隨帝至平陽,劉聰以其不忠於本朝,戮之於東市。



 賈疋,字彥度,武威人,魏太尉詡之曾孫也。少有志略,器望甚偉,見之者莫不悅附,特為武夫之所瞻仰,願為致命。初辟公府,遂歷顯職,遷安定太守。雍州刺史丁綽,貪橫失百姓心,乃譖疋于南陽王模,模以軍司謝班伐之。疋奔瀘水,與胡彭蕩仲及氐竇首結為兄弟,聚眾攻班。綽奔武都,疋復入安定,殺班。愍帝以疋為驃騎將軍、雍
 州刺史,封酒泉公。時諸郡百姓饑饉,白骨蔽野,百無一存。疋帥戎晉二萬餘人,將伐長安,西平太守竺恢亦固守,劉粲聞之,使劉曜、劉雅及趙染距疋,先攻恢,不剋,疋邀擊,大敗之,曜中流矢,退走。疋追之,至于甘泉。旋自渭橋襲蕩仲,殺之。遂迎秦王,奉為皇太子。後蕩仲子夫保持帥群胡攻之,疋敗走,夜墮於澗,為夫護所害。疋勇略有志節,以匡復晉室為己任,不幸顛墮,時人咸痛惜之。



 史臣曰:自永嘉蕩覆,宇內橫流,億兆靡依,人神乏主。于時武皇之胤,惟有建興,眾望攸歸,曾無與二。閻鼎等忠存社稷,志在經綸,乃契闊艱難,扶持幼孺,遂得纂堯承
 緒,祀夏配天,校績論功,有足稱矣。然而抗滔天之巨寇,接凋弊之餘基,威略未申,尋至傾覆。昔宗周遭犬戎而東徙,有晉違獷狄而西遷,彼既靈慶悠長,此則禍難遄及,豈愍皇地非奧主,將綝允材謝輔臣,何修短之殊途,而成敗之異數者也?



 贊曰:懷惠不競,戚籓力爭。狙詐參謀,憑兇亂政。為惡不已,並羅非命。解繆忠肅,無聞餘慶。愍皇纂戎,實賴群公。鼎圖福始,綝遂兇終。



\end{pinyinscope}