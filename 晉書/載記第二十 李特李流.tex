\article{載記第二十 李特李流}

\begin{pinyinscope}

 李特李流



 李特,字玄休,巴西宕渠人,其先廩君之苗裔也。昔武落鐘離山崩,有石穴二所,其一赤如丹,一黑如漆。有人出於赤穴者,名曰務相,姓巴氏。有出於黑穴者,凡四姓:曰KL氏、樊氏、柏氏、鄭氏。五姓俱出,皆爭為神,於是相與以劍刺穴屋,能著者以為廩君。四姓莫著,而務相之劍懸焉。又以土為船,彫畫之而浮水中,曰:「若其船浮存者,以
 為廩君。」務相船又獨浮。於是遂稱廩君,乘其土船,將其徒卒,當夷水而下,至於鹽陽。鹽陽水神女子止廩君曰:「此魚鹽所有,地又廣大,與君俱生,可止無行。」廩君曰:「我當為君求廩地,不能止也。」鹽神夜從廩君宿,旦輒去為飛蟲,諸神皆從其飛,蔽日晝昏。廩君欲殺之不可,別又不知天地東西。如此者十日,廩君乃以青縷遺鹽神曰:「嬰此,即宜之,與汝俱生。弗宜,將去汝。」鹽神受而嬰之。廩君立碭石之上,望膺有青縷者,跪而射之,中鹽神。鹽神死,群神與俱飛者皆去,天乃開朗。廩君復乘土船,下及夷城。夷城石岸曲,泉水亦曲。廩君望如穴狀,歎曰:「我新
 從穴中出,今又入此,奈何!」岸即為崩,廣三丈餘,而階陛相乘,廩君登之。岸上有平石方一丈,長五尺,廩君休其上,投策計算,皆著石焉,因立城其旁而居之。其後種類遂繁。秦并天下,以為黔中郡,薄賦斂之,口歲出錢四十。巴人呼賦為賨,因謂之賨人焉。及漢高祖為漢王,募賨人平定三秦。既而求還鄉里,高祖以其功,復同豐、沛,不供賦稅,更名其地為巴郡。土有鹽鐵丹漆之饒,俗性剽勇,又善歌舞。高祖愛其舞,詔樂府習之,今《巴渝舞》是也。漢末,張魯居漢中,以鬼道教百姓,賨人敬信巫覡,多往奉之。值天下大亂,自巴西之宕渠遷於漢中楊車阪,抄
 掠行旅,百姓患之,號為楊車巴。魏武帝剋漢中,特祖將五百餘家歸之,魏武帝拜為將軍,遷于略陽,北土復號之為巴氐。特父慕,為東羌獵將。



 特少仕州郡,見異當時,身長八尺,雄武善騎射,沈毅有大度。元康中,氐齊萬年反,關西擾亂,頻歲大饑,百姓乃流移就穀,相與入漢川者數萬家。特隨流人將入于蜀,至劍閣,箕踞太息,顧眄險阻曰:「劉禪有如此之地而面縛於人,豈非庸才邪!」同移者閻式、趙肅、李遠、任回等咸歎異之。



 初,流人既至漢中,上書求寄食巴、蜀,朝議不許,遣侍御史李苾持節慰勞,且監察之,不令入劍閣。苾至漢中,受流人貨賂,反為
 表曰:「流人十萬餘口,非漢中一郡所能振贍,東下荊州,水湍迅險,又無舟船。蜀有倉儲,人復豐稔,宜令就食。」朝廷從之,由是散在益、梁,不可禁止。



 永康元年,詔徵益州刺史趙廞為大長秋,以成都內史耿滕代廞。廞遂謀叛,潛有劉氏割據之志,乃傾倉廩,振施流人,以收眾心。特之黨類皆巴西人,與廞同郡,率多勇壯,廞厚遇之,以為爪牙,故特等聚眾,專為寇盜,蜀人患之。滕密上表,以為流人剛剽而蜀人懦弱,客主不能相制,必為亂階,宜使移還其本。若致之險地,將恐秦雍之禍萃於梁益,必貽聖朝西顧之憂。廞聞而惡之。時益州文武千餘人已往
 迎滕,滕率眾入州,廞遣眾逆滕,戰於西門,滕敗,死之。



 廞自稱大都督、大將軍、益州牧。特弟庠與兄弟及妹夫李含、任回、上官惇、扶風李攀、始平費佗、氐苻成、隗伯等以四千騎歸廞。廞以庠為威寇將軍,使斷北道。庠素東羌良將,曉軍法,不用麾幟,舉矛為行伍,斬部下不用命者三人,部陣肅然。廞惡其齊整,欲殺之而未言。長史杜淑、司馬張粲言於廞曰:「傳云五大不在邊,將軍起兵始爾,便遣李庠握強兵於外,愚竊惑焉。且非我族類,其心必異,倒戈授人,竊以為不可,願將軍圖之。」廞斂容曰:「卿言正當吾意,可謂起予者商,此天使卿等成吾事也。」會庠
 在門,請見廞,廞大悅,引庠見之。庠欲觀廞意旨,再拜進曰:「今中國大亂,無復綱維,晉室當不可復興也。明公道格天地,德被區宇,湯、武之事,實在於今。宜應天時,順人心,拯百姓於塗炭,使物情知所歸,則天下可定,非但庸、蜀而已。,」廞怒曰:「此豈人臣所宜言!」令淑等議之。於是淑等上庠大逆不道,廞乃殺之,及其子姪宗族三十餘人。廞慮特等為難,遣人喻之曰:「庠非所宜言,罪應至死,不及兄弟。」以庠尸還特,復以特兄弟為督將,以安其眾。牙門將許弇求為巴東監軍,杜淑、張粲固執不許。弇怒,於廞閣下手刃殺淑、粲,淑、粲左右又殺弇,皆廞腹心也。



 特兄弟
 既以怨廞,引兵歸綿竹。廞恐朝廷討己,遣長史費遠、犍為太守李苾、督護常俊督萬餘人斷北道,次綿竹之石亭。特密收合得七千餘人,夜襲遠軍,遠大潰,因放火燒之,死者十八九。進攻成都。廞聞兵至,驚懼不知所為。李苾、張征等夜斬關走出,文武盡散。廞獨與妻子乘小船走至廣都,為下人朱竺所殺。特至成都,縱兵大掠,害西夷護軍姜發,殺廞長史袁治及廞所置守長,遣其牙門王角、李基詣洛陽陳廞之罪狀。



 先是,惠帝以梁州刺史羅尚為平西將軍、領護西夷校尉、益州刺史,督牙門將王敦、上庸都尉義歆、蜀郡太守徐儉、廣漢太守辛冉等
 凡七千餘人入蜀。特等聞尚來,甚懼,使其弟驤於道奉迎,並貢寶物。尚甚悅,以驤為騎督。特及弟流復以牛酒勞尚於綿竹。王敦、辛冉並說尚曰:「特等流人,專為盜賊,急宜梟除,可因會斬之。」尚不納。冉先與特有舊,因謂特曰:「故人相逢,不吉當凶矣。」特深自猜懼。



 尋有符下秦、雍州,凡流人入漢川者,皆下所在召還。特兄輔素留鄉里,託言迎家,既至蜀,謂特曰:「中國方亂,不足復還,」特以為然,乃有雄據巴、蜀之意。朝廷以討趙廞功,拜特宣威將軍,封長樂鄉侯,流為奮威將軍、武陽侯。璽書下益州,條列六郡流人與特協同討廞者,將加封賞。會辛冉以非
 次見征,不顧應召,又欲以滅廞為己功,乃寢朝命,不以實上。眾咸怨之。羅尚遣從事催遣流人,限七月上道,辛冉性貪暴,欲殺流人首領,取其資貨,乃移檄發遣。又令梓潼太守張演於諸要施關,搜索寶貨。特等固請,求至秋收。流人布在梁、益,為人傭力,及聞州郡逼遣,人人愁怨,不知所為。又知特兄弟頻請求停,皆感而恃之。且水雨將降,年穀未登,流人無以為行資,遂相與詣特。特乃結大營於綿竹,以處流人,移冉求自寬。冉大怒,遣人分榜通逵,購募特兄弟,許以重賞。特見,大懼,悉取以歸,與驤改其購云:「能送六郡之豪李、任、閻、趙、楊、上官及氐、叟
 侯王一首,賞百匹。」流人既不樂移,咸往歸特,騁馬屬鞬,同聲雲集,旬月間眾過二萬。流亦聚眾數千。物乃分為二營,特居北營,流居東營。



 特遣閻式詣羅尚,求申期。式既至,見冉營柵衝要,謀掩流人,嘆曰:「無寇而城,仇必保焉。今而速之,亂將作矣!」又知冉及李苾意不可迴,乃辭尚還綿竹。尚謂式曰:「子且以吾意告諸流人,今聽寬矣。」式曰:「明公惑於姦說,恐無寬理。弱而不可輕者百姓也,今促之不以理,眾怒難犯,恐為禍不淺。」尚曰:「然。吾不欺子,子其行矣。」式至綿竹,言於特曰:「尚雖云爾,然未可必信也。何者?尚威刑不立,冉等各擁彊兵,一旦為變,亦非
 尚所能制,深宜為備。」特納之。冉、苾相與謀曰:「羅侯貪而無斷,日復一日,流人得展姦計。李特兄弟並有雄才,吾屬將為豎子虜矣。宜為決計,不足復問之。乃遣廣漢都尉曾元、牙門張顯、劉並等潛率步騎三萬襲特營。羅尚聞之,亦遣督護田佐助元。特素知之,乃繕甲厲兵,戒嚴以待之。元等至,特安臥不動,待其眾半入,發伏擊之,殺傷者甚眾,害田佐、曾元、張顯,傳首以示尚、冉。尚謂將佐曰:「此虜成去矣,而廣漢不用吾言,以張賊勢,今將若之何!」



 於是六郡流人推特為主。特命六郡人部曲督李含、上邽令任臧、始昌令閻式、諫議大夫李攀、陳倉令李武、
 陰平令李遠、將兵都尉楊褒等上書,請依梁統奉竇融故事,推特行鎮北大將軍,承制封拜,其弟流行鎮東將軍,以相鎮統。於是進兵攻冉於廣漢。冉眾出戰,特每破之。尚遣李苾及費遠率眾救冉,憚特不敢進。冉智力既窘,出奔江陽。特入據廣漢,以李超為太守,進兵攻尚於成都。閻式遺尚書,責其信用讒構,欲討流人,又陳特兄弟立功王室,以寧益土。尚覽書,知特等將有大志,嬰城固守,求救於梁、寧二州。於是特自稱使持節、大都督、鎮北大將軍,承制封拜一依竇融在河西故事。兄輔為驃騎將軍,弟驤為驍騎將軍,長子始為武威將軍,次子蕩
 為鎮軍將軍,少子雄為前將軍,李含為西夷校尉,含子國離、任回、李恭、上官晶、李攀、費佗等為將帥,任臧、上官惇、楊褒、楊珪、王達、麴歆等為爪牙,李遠、李博、夕斌、嚴檉、上官琦、李濤、王懷等為僚屬,閻式為謀主,何世、趙肅為腹心。時羅尚貪殘,為百姓患,而特與蜀人約法三章,施捨振貸,禮賢拔滯,軍政肅然。百姓為之謠曰:「李特尚可,羅尚殺我。」尚頻為特所敗,乃阻長圍,緣水作營,自都安至犍為七百里,與特相距。



 河間王顒遣督護衙博、廣漢太守張征討特,南夷校尉李毅又遣兵五千助尚,尚遣督護張龜軍繁城,三道攻特。特命蕩、雄襲博。特躬擊張
 龜,龜眾大敗。蕩又與博接戰連日,博亦敗績,死者太半。蕩追博至漢德,博走葭萌。蕩進寇巴西,巴西郡丞毛植、五官襄珍以郡降蕩。蕩撫恤初附,百姓安之。蕩進攻葭萌,博又遠遁,其眾盡降于蕩。



 太安元年,特自稱益州牧、都督梁、益二州諸軍事、大將軍、大都督,改年建初,赦其境內。於是進攻張征。征依高據險,與特相持連日。時特與蕩分為二營,徵候特營空虛,遣步兵循山攻之,特逆戰不利,山險窘逼,眾不知所為。羅準、任道皆勸引退,特量蕩必來,故不許。徵眾至稍多,山道至狹,唯可一二人行,蕩軍不得前,謂其司馬王辛曰:「父在深寇之中,是我
 死日也。」乃衣重鎧,持長矛,大呼直前,推鋒必死,殺十餘人。徵眾來相救,蕩軍皆殊死戰,征軍遂潰。特議欲釋徵還涪,蕩與王辛進曰:「征軍連戰,士卒傷殘,智勇俱竭,宜因其弊遂擒之。若舍而寬之,徵養病收亡,餘眾更合,圖之未易也。」特從之,復進攻征,徵潰圍走。蕩水陸追之,遂害征,生擒徵子存,以征喪還之。



 以騫碩為德陽太守,碩略地至巴郡之墊江。



 特之攻張征也,使李驤與李攀、任回、李恭屯軍毗橋,以備羅尚。尚遣軍挑戰,驤等破之。尚又遣數千人出戰,驤又陷破之,大獲器甲,攻燒其門。流進次成都之北。尚遣將張興偽降於驤,以觀虛實。時驤
 軍不過二千人,興夜歸白尚,尚遣精勇萬人銜枚隨興夜襲驤營。李攀逆戰死,驤及將士奔於流柵,與流並力迴攻尚軍。尚軍亂,敗還者十一二。晉梁州刺史許雄遣軍攻特,特陷破之,進擊,破尚水上軍,遂寇成都。蜀郡太守徐儉以小城降,特以李瑾為蜀郡太守以撫之。羅尚據大城自守。流進屯江西,尚懼,遣使求和。



 是時蜀人危懼,並結村堡,請命于特,特遣人安撫之。益州從事任明說尚曰:「特既凶逆,侵暴百姓,又分人散眾,在諸村堡,驕怠無備,是天亡之也。可告諸村,密剋期日,內外擊之,破之必矣。」尚從之。明先偽降特,特問城中虛實,明曰:「米穀
 已欲盡,但有貨帛耳。」因求省家,特許之。明潛說諸村,諸村悉聽命。還報尚,尚許如期出軍,諸村亦許一時赴會。



 二年,惠帝遣荊州刺史宋岱、建平太守孫阜救尚。阜已次德陽,特遣蕩督李璜助任臧距阜。尚遣大眾奄襲特營,連戰二日,眾少不敵,特軍大敗,收合餘卒,引趣新繁。尚軍引還,特復追之,轉戰三十餘里,尚出大軍逆戰,特軍敗績,斬特及李輔、李遠,皆焚尸,傳首洛陽。在位二年。其子雄僭稱王,追謚特景王,及僭號,追尊曰景皇帝,廟號始祖。



 李流,字玄通,特第四弟也。少好學,便弓馬,東羌校尉何
 攀稱流有賁育之勇,舉為東羌督。及避地益州,刺史趙廞器異之。廞之使庠合部眾也,流亦招鄉里子弟得數千人。庠為廞所殺,流從特安慰流人,破常俊於綿竹,平趙廞於成都。朝廷論功,拜奮威將軍,封武陽侯。



 特之承制也,以流為鎮東將軍,居東營,號為東督護。特常使流督銳眾,與羅尚相持。特之陷成都小城,使六郡流人分口入城,壯勇督領村堡。流言於特曰:「殿下神武,已剋小城,然山藪未集,糧仗不多,宜錄州郡大姓子弟以為質任,送付廣漢,縶之二營,收集猛銳,嚴為防衛。」又書與特司馬上官惇,深陳納降若待敵之義。特不納。



 特既死,蜀
 人多叛,流人大懼。流與兄子蕩、雄收遺眾,還赤祖,流保東營,蕩、雄保北營。流自稱大將軍、大都督、益州牧。



 時宋岱水軍三萬,次于墊江,前鋒孫阜破德陽,獲特所置守將騫碩,太守任臧等退屯涪陵縣。羅尚遣督護常深軍毗橋,牙門左氾、黃訇、何沖三道攻北營。流身率蕩、雄攻深柵,剋之,深士眾星散。追至成都,尚閉門自守,蕩馳馬追擊,觸倚矛被傷死。流以特、蕩並死,而岱、阜又至,甚懼。太守李含又勸流降,流將從之。雄與李驤迭諫,不納,流遣子世及含子胡質於阜軍。胡兄含子離聞父欲降,自梓潼馳還,欲諫不及,退與雄謀襲阜軍,曰:「若功成事濟,
 約與君三年迭為主。」雄曰:「今計可定,二翁不從,將若之何?」離曰:「今當制之,若不可制,便行大事。翁雖是君叔,勢不得已,老父在君,夫復何言!」雄大喜,乃攻尚軍。尚保大城。雄渡江害汶山太守陳圖,遂入郫城,流移營據之。三蜀百姓並保險結塢,城邑皆空,流野無所略,士眾飢困。涪陵人范長生率千餘家依青城山,尚參軍涪陵徐轝求為汶山太守,欲要結長生等,與尚掎角討流。尚不許,轝怨之,求使江西,遂降于流,說長生等使資給流軍糧。長生從之,故流軍復振。



 流素重雄有長者之德,每云:「興吾家者,必此人也。」敕諸子尊奉之。流疾篤,謂諸將曰:「驍
 騎高明仁愛,識斷多奇,固足以濟大事,然前軍英武,殆天所相,可共受事於前軍,以為成都王。」遂死,時年五十六。諸將共立雄為主。雄僭號,追謚流秦文王。



 李庠,字玄序,特第三弟也。少以烈氣聞。仕郡督郵、主簿,皆有當官之稱。元康四年,察孝廉,不就。後以善騎射,舉良將,亦不就。州以庠才兼文武,舉秀異,固以疾辭。州郡不聽,以其名上聞,中護軍切徵,不得已而應之,拜中軍騎督。弓馬便捷,膂力過人,時論方之文鴦。以洛陽方亂,稱疾去官。性在任俠,好濟人之難,州黨爭附之。與六郡流人避難梁、益,道路有飢病者,庠常營護隱恤,振施窮
 乏,大收眾心。至蜀,趙廞深器之,與論兵法,無不稱善,每謂所親曰:「李玄序蓋亦一時之關、張也。」及將有異志,委以心膂之任,乃表庠為部曲督,使招合六郡壯勇,至萬餘人。以討叛羌功,表庠為威寇將軍,假赤幢曲蓋,封陽泉亭侯,賜錢百萬,馬五十匹。被誅之日,六郡士庶莫不
 流涕,時年五十五。



\end{pinyinscope}