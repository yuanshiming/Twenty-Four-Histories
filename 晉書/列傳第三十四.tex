\article{列傳第三十四}

\begin{pinyinscope}

 武十三王元
 四王簡文三子



 武帝二十六男:楊元后生毗陵悼王軌、惠帝、秦獻王柬。審美人生城陽懷王景、楚隱王瑋、長沙厲王乂。徐才人生城陽殤王憲。匱才人生東海沖王祗。趙才人生始平哀王裕。趙美人生代哀王演。李夫人生淮南忠壯王允、吳孝王晏。莊保林生新都懷王該。陳美人生清河康王遐。諸姬生汝陰哀王謨。程才人生成都王穎。王才人生
 孝懷帝。楊悼后生渤海殤王恢。餘八子不顯母氏,並早夭,又無封國及追謚,今並略之。其瑋、乂、穎自有傳。



 毗陵悼王軌,字正則,初拜騎都尉,年二歲而夭。太康十年,追加封謚,以楚王瑋子義嗣。



 秦獻王柬,字弘度,沈敏有識量。泰始六年,封汝南王。咸寧初,徙封南陽王,拜左將軍、領右軍將軍、散騎常侍。武帝嘗幸宣武場,以三十六軍兵簿令不料校之,東一省便擿脫謬,帝異之,於諸子中尤見寵愛。以左將軍居齊獻王故府,甚貴寵,為天下所屬目。性仁訥,無機辯之譽。太康十年,徙封於秦,邑八萬戶。于時諸王封中土者皆
 五萬戶,以柬與太子同產,故特加之。轉鎮西將軍、西戎校尉、假節,與楚、淮南王俱之國。



 及惠帝即位,來朝,拜驃騎將軍、開府儀同三司,加侍中、錄尚書事,進位大將軍。時楊駿伏誅,柬既痛舅氏覆滅,甚有憂危之慮,屢述武帝旨,請還籓,而汝南王亮留柬輔政。及亮與楚王瑋被誅,時人謂柬有先識。



 元康元年薨,時年三十,朝野痛惜之。葬禮如齊獻文王攸故事,廟設軒懸之樂。無子,以淮南王允子郁為嗣,與允俱被害。永寧二年,追謚曰悼。又以吳王晏子鄴嗣。懷帝崩,鄴入纂帝位,國絕。



 城陽懷王景,字景度,出繼叔父城陽哀王兆後。泰始五
 年受封,六年薨。



 東海沖王祗,字敬度,泰始九年五月受封。殤王薨,復以祗繼兆,其年薨,時年三歲。



 始平哀王裕,字濬度,咸寧三年受封,其年薨,年七歲。無子,以淮南王允子迪為嗣。太康十年,改封漢王,為趙王倫所害。



 淮南忠壯王允,字欽度,咸寧三年,封濮陽王,拜越騎校尉。太康十年,徙封淮南,仍之國,都督揚江二州諸軍事、鎮東大將軍、假節。元康九年入朝。



 初,愍懷之廢,議者將立允為太弟。會趙王倫廢賈后,詔遂以允為驃騎將軍、
 開府儀同三司、侍中,都督如故,領中護軍。允性沈毅,宿衛將士皆敬服之。



 倫既有篡逆志,允陰知之,稱疾不朝,密養死士,潛謀誅倫。倫甚憚之,轉為太尉,外示優祟,實奪其兵也。允稱疾不拜。倫遣御史逼允,收官屬以下,劾以太逆。允恚,視詔,乃孫秀手書也。大怒,便收御史,將斬之,御史走而獲免,斬其令史二人。厲色謂左右曰:「趙王欲破我家!」遂率國兵及帳下七百人直出,大呼曰:「趙王反,我將攻之,佐淮南王者左袒。」於是歸之者甚眾。允將赴宮,尚書左丞輿閉東掖門,允不得人,遂圍相府。允所將兵,皆淮南奇才劍客也。與戰,頻敗之,倫兵死者千
 餘人。太子左率陳徽勒東宮兵鼓噪於內以應,允結陳於承華門前,弓弩齊發,射倫,飛矢雨下。主書司馬畦祕以身蔽倫,箭中其背而死。倫官屬皆隱樹而立,每樹輒中數百箭,自辰至未。徽兄淮時為中書令,遣麾騶虞以解鬥。倫子虔為侍中,在門下省,密要壯士,約以富貴。於是遣司馬督護伏胤領騎四百從宮中出,舉空版,詐言有詔助淮南王允。允不之覺,開陳納之,下車受詔,為胤所害,時年二十九。初,倫兵敗,皆相傳:「已擒倫矣。」百姓大悅。既而聞允死,莫不歎息。允三子皆被害,坐允夷滅者數千人。



 及倫誅,齊王冏上表理允曰:「故淮南王允
 忠孝篤誠,憂國忘身,討亂奮發,幾於剋捷。遭天凶運,奄至隕沒,逆黨進惡,並害三子,冤魂酷毒,莫不悲酸。洎興義兵,淮南國人自相率領,眾過萬人,人懷慷愾,愍國統滅絕,發言流涕。臣輒以息超繼允後,以尉存亡。」有詔改葬,賜以殊禮,追贈司徒。冏敗,超被幽金墉城。後更以吳王晏子祥為嗣,拜散騎常侍洛京傾覆,為劉聰所害。



 代哀王演,字宏度,太康十年受封。少有廢疾,不之國,演常止于宮中。薨,無子,以成都王穎子廓為嗣,改封中都王,後與穎俱死。



 新都王該,字玄度,咸寧三年受封,太康四年薨,時年十
 二。無子,國除。



 清河康王遐,字深度,美容儀,有精彩,武帝愛之。既受封,出繼叔父城陽哀王兆。太康十年,封渤海郡,歷右將軍、散騎常侍、前將軍。元康初,進撫軍將軍,加侍中,遐長而懦弱,無所是非。性好內,不能接士大夫。及楚王瑋之舉兵也,使遐收衛瓘,而瓘故吏榮晦遂盡殺瓘子孫,遐不能禁,為世所尤。永康元年薨,時年二十八。四子:覃、籥、銓、端。覃嗣立。



 及沖太孫薨,齊王冏表曰:「東宮曠然,冢嗣莫繼。天下大業,帝王神器,必建儲副,以固洪基。今者後宮未有孕育,不可庶幸將來而虛天緒,非祖宗之遺志,
 社稷之長計也。禮,兄弟之子猶子,故漢成無嗣,繼由定陶;孝和之絕,安以紹興。此先王之令典,往代之成式也。清河王覃神姿岐嶷,慧智早成,康王正妃周氏所生,先帝眾孫之中,於今為嫡。昔薄姬賢明,文則承位。覃外祖恢世載名德,覃宜奉宗廟之重,統無窮之祚,以寧四海顒顒之望。覃兄弟雖並出紹,可簡令淑還為國胤,不替其嗣。輒諮大將軍穎及群公卿士,咸同大願。請具禮儀,擇日迎拜。」遂立覃為皇太子。既而河間王顒協遷大駕,表成都王穎為皇太弟,廢覃復為清河王。初,覃為清河世子,所佩金鈴欻生隱起如麻粟,祖母陳太妃以為不
 祥,毀而賣之。占者以金是晉行大興之祥,覃為皇胤,是其瑞也。毀而賣之,象覃見廢不終之驗也。永嘉初,前北軍中候任城呂雍、度支校尉陳顏等謀立覃為太子,事覺,幽於金墉城。未幾,被害,時年十四,葬以庶人禮。



 籥初封新蔡王,覃薨,還封清河王。



 銓初封上庸王,懷帝即位,更封豫章王。二年,立為皇太子。洛京傾覆,沒於劉聰。



 端初封廣川王,銓之為皇太子也,轉封豫章,禮秩如皇子,拜散騎常侍、平南將軍、都督江州諸軍事、假節。當之國,會洛陽陷沒,端東奔茍晞於蒙。晞立為皇太子,七十日,為石勒所沒。



 汝陰哀王謨,字令度,太康七年薨,時年十一。無後,國除。



 吳敬王晏,字平度,太康十年受封,食丹陽、吳興並吳三郡,歷射聲校尉、後軍將軍。與兄淮南王允共攻趙王倫,允敗,收晏付廷尉,欲殺之。傅祗於朝堂正色而爭,於是群官並諫,倫乃貶為賓徒縣王。後徙封代王。倫誅,詔復晏本封,拜上軍大將軍、開府,加侍中。長沙王乂、成都王穎之相攻也,乂以晏為前鋒都督,數交戰。永嘉中,為太尉、太將軍。晏為人恭愿,才不及中人,於武帝諸子中最劣。又少有風疾,視瞻不端,後轉增劇,不堪朝覲。及洛京傾覆,晏亦遇害,時年三十一。愍帝即位,追贈太保。五子,
 長子不顯名,與晏同沒。餘四子:祥、鄴、固、衍。祥嗣淮南王允。鄴即愍帝。固初封漢王,改封濟南。衍初封新都王,改封濟陰,為散騎常侍。皆沒于賊。



 渤海殤王恢,字思度,太康五年薨,時年二歲,追加封謚。



 元帝六男:宮人荀氏生明帝及瑯邪孝王裒。石婕妤生東海哀王沖。王才人生武陵威王晞。鄭夫人生瑯邪悼王煥及簡文帝。



 瑯邪孝王裒字道成,母荀氏,以微賤入宮,元帝命虞妃養之。裒初繼叔父長樂亭侯渾,後徙封宣城郡公,拜後
 將軍。及帝為晉王,有司奏立太子,帝以裒有成人之量,過於明帝,從容謂王導曰:「立子以德不以年。」導曰:「世子、宣城俱有朗雋之目,固當以年。」於是太子位遂定。更封裒瑯邪,嗣恭王後,改食會稽、宣城邑五萬二千戶,拜散騎常侍、使持節、都督青徐兗三州諸軍事、車騎將軍,徵還京師。建武元年薨,年十八,贈車騎大將軍,加侍中。及妃山氏薨,祔葬,穆帝更贈裒太保。子哀王安國立,未踰年薨。



 東海哀王沖,字道讓。元帝以東海王越世子毗沒于石勒,不知存亡,乃以沖繼毗後,稱東海世子,以毗陵郡增
 本封邑萬戶,又改食下邳、蘭陵,以越妃裴氏為太妃,拜長水校尉。高選僚佐,以沛國劉耽為司馬,潁川庾懌為功曹,吳郡顧和為主簿。永昌初,遷中軍將軍,加散騎常侍。及東海太妃薨,因發毗喪。沖即王位,以滎陽益東海國,轉車騎將軍,徙驃騎將軍。咸康七年薨,年三十一,贈侍中、驃騎大將軍、儀同三司,無子。



 成帝臨崩,詔曰:「哀王無嗣,國統將絕,朕所哀怛。其以小晚生奕繼哀王為東海王。」以道遠,罷滎陽,更以臨川郡益東海。及哀帝以瑯邪王即尊位,徙奕為瑯邪王,東海國闕,無嗣。奕後入纂大業,桓溫廢之,復為東海王,既而貶為海西公,東海國
 又闕嗣。隆安三年,安帝詔以會稽忠王次子彥璋為東海王,繼哀王為曾孫,改食吳興郡。為桓玄所害,國除。



 武陵威王晞,字道叔,出繼武陵王喆後,太興元年受封。咸和初,拜散騎常侍。後以湘東增武陵國,除左將軍,遷鎮軍將軍,加散騎常侍。康帝即位,加侍中、特進。建元初,領祕書監。穆帝即位,轉鎮軍大將軍,遷太宰。太和初,加羽葆鼓吹,入朝不趨,贊拜不名,劍履上殿。固讓。



 晞無學術而有武幹,為桓溫所忌。及簡文帝即位,溫乃表晞曰:「晞體自皇極,故寵靈光世,不能率由王度,修己慎行,而聚納輕剽,苞藏亡命。又息綜矜忍,虐加于人。袁真叛逆,
 事相連染。頃自猜懼,將成亂階。請免晞官,以王歸籓,免其世子綜官,解子逢散騎常侍。」逢以梁王隨晞,晞既見黜,送馬八十五匹、三百人杖以歸溫。溫又逼新蔡王晁使自誣與晞、綜及著作郎殷涓、太宰長史庾倩、掾曹秀、舍人劉彊等謀逆,遂收付廷尉,請誅之。簡文帝不許,溫於是奏徙新安郡,家屬悉從之,而族誅殷涓等,廢晃徙沖陽郡。



 太元六年,晞卒於新安,時年六十六。孝武帝三日臨于西堂,詔曰:「感惟摧慟,便奉迎靈柩,并改移妃應氏及故世子梁王諸喪,家屬悉還。」復下詔曰:「故前武陵王體自皇極,剋己思愆。仰惟先朝仁宥之旨,豈可情禮
 靡寄!其追封新寧郡王,邑千戶。」晞三子:綜、逢、遵。以遵嗣。追贈綜給事中,逢散騎郎。十二年,追復晞武陵國,綜、逢各復先官,逢還繼梁國。



 梁王逢,字賢明,出繼梁王翹,官至永安太僕,與父晞俱廢。薨,子和嗣。太元中復國。薨,子珍之嗣。桓玄篡位,國人孔樸奉珍之奔于壽陽。桓玄敗,珍之歸朝廷。太將軍武陵王令曰:「梁王珍之理悟貞立,蒙險違難,撫義懷順,載奔闕庭。值壽陽擾亂,在危克固,且可通直散騎郎。」累遷游擊將軍、左衛、太常。劉裕伐姚泓,請為諮議參軍。裕將弱王室,誣其罪害之。



 忠敬王遵,字茂遠。初襲封新寧,時年十二,受拜流涕,哀感左右。右將軍桓伊嘗詣遵,遵曰:「門何為通桓氏?」左右曰:「伊與桓溫疏宗,相見無嫌。」遵曰:「我聞人姓木邊,便欲殺之,況諸桓乎!」由是少稱聰慧。及晞追復封武陵王,以遵嗣,歷位散騎常侍、秘書監、太常、中領軍。桓玄用事,拜金紫光祿大夫。玄篡,貶為彭澤侯,遣之國。行次石頭,夜濤水入淮,船破,未得發。會義旗興,復還國第。朝廷稱受密詔,使遵總攝萬機,加侍中、大將軍,移入東宮,內外畢敬。遷轉百官,稱制書;又教稱令書。安帝反正,更拜太保,加班劍二十人。義熙四年薨,時年三十五,詔賜東園溫
 明神器,朝服一具,衣一襲,錢百萬,布千匹,策贈太傳,葬加殊禮。子定王季度立,拜散騎侍郎。薨,子球之立。宋興,國除。



 瑯邪悼王煥,字耀祖。母有寵,元帝特所鐘愛。初繼帝弟長樂亭侯渾,後封顯義亭侯。尚書令刁協奏:「昔魏臨淄侯以邢顒為家丞,劉楨為庶子。今侯幼弱,宜選明德。」帝令曰:「臨淄萬戶封,又植少有美才,能同遊田蘇者。今晚生蒙弱,何論於此!間封此兒,不以寵稚子也。亡弟當應繼嗣,不獲已耳。家丞、庶子,足以攝祠祭而已,豈宜屈賢才以受無用乎!」及煥疾篤,帝為之撤膳,乃下詔封為瑯
 邪王,嗣恭王後,俄而薨,年二歲。



 帝悼念無已,將葬,以煥既封列國,加以成人之禮,詔立凶門柏歷,備吉凶儀服,營起陵園,功役甚眾。瑯邪國右常侍會稽孫霄上疏諫曰:



 臣聞法度典制,先王所重,吉凶之禮,事貴不過。是以世豐不使奢放,凶荒必務約殺。朝聘嘉會,足以展庠序之儀;殯葬送終,務以稱哀榮之情。上無奢泰之謬,下無匱竭之困。故華元厚葬,君子謂之不臣;嬴博至儉,仲尼稱其合。禮明傷財害時,古人之所譏;節省簡約,聖賢之所嘉也。語曰,上之化下,如風靡草。京邑翼翼,四方所則,明教化法制,不可不慎也。陛下龍飛踐阼,興微濟弊,聖
 懷勞謙,務從簡儉,憲章舊制,猶欲節省,禮典所無,而反尚飾,此臣愚情竊所不安也。棺槨輿服旒之屬,禮典舊制,不可廢闕。凶門柏歷,禮典所無,天晴可不用,遇雨則無益,此至宜節省者也。若瑯邪一國一時所用,不為大費,臣在機近,義所不言。今天臺所居,王公百僚聚在都輦,凡有喪事,皆當供給材木百數、竹薄千計,凶門兩表,衣以細竹及材,價直既貴,又非表凶哀之宜,如此過飾,宜從粗簡。



 又案《禮記》,國君之葬,棺槨之間容柷,大夫容壼,士容甒。以壼甒為差,則柷財大於壼明矣,槨周於棺,槨不甚大也。語曰,葬者藏也,藏欲其深而固也。槨大
 則難為堅固,無益於送終,而有損於財力。凶荒殺禮,經國常典,既減殺而猶過舊,此為國之所厚惜也。又禮,將葬,遷柩于廟祖而行,及墓即窆,葬之日即反哭而虞。如此,則柩不宿於墓上也。聖人非不哀親之在土而無情於丘墓,蓋以墓非安神之所,故修虞於殯宮。始則營草宮於山陵,遷神柩於墓側,又非典也。非禮之事,不可以訓萬國。



 臣至愚至賤,忽求革前之非,可謂狂瞽不知忌諱。然今天下至弊,自古所希,宗廟社稷,遠託江表半州之地,凋殘以甚。加之荒旱,百姓困瘁,非但不足,死亡是懼。此乃陛下至仁之所矜愍,可憂之至重也。正是匡矯
 末俗,改張易調之時,而猶當竭已罷之人,營無益之事,殫已困之財,修無用之費,此固臣之所不敢安也。今瑯邪之於天下,國之最大,若割損非禮之事,務遵古典,上以彰聖朝簡易之至化,下以表萬世無窮之規則,此芻蕘之言有補萬一,塵露之微有增山海。



 表寢不報。



 永昌元年,立煥母弟昱為瑯邪王,即簡文帝也。咸和二年,徙封會稽,以康帝為瑯邪王。康帝即位,哀帝為瑯邪王。哀帝即位,廢帝為瑯邪王。廢帝即位,又以簡文帝攝行瑯邪王國祀。簡文登阼,國遂無嗣。帝臨崩,封少子道子為瑯邪王。太元十七年,道子為會稽王,更以恭帝為瑯邪
 王。恭帝即位,於是瑯邪國除。



 簡文帝七子:王皇后生會稽思世子道生、皇子俞生。胡淑儀生臨川獻王鬱、皇子硃生。王淑儀生皇子天流。李夫人生孝武帝、會稽文孝王道子。俞生、硃生、天流並早夭,今並略之。



 會稽思世子道生,字延長。帝為會稽王,立道生為世子,拜散騎侍郎、給事中。性疏躁,不脩行業,多失禮度,竟以幽廢而卒,時年二十四,無後。及孝武帝即位,嘗晝日見道生及臨川獻王郁,郁曰:「大郎飢乏辛苦。」言竟不見。帝
 傷感,因以西陽王錄玄孫珣之為後。珣之歷吳興太守。劉裕之伐關中,以為諮議參軍。時帝道方謝,珣之為宗室之美,與梁王珍之俱被害。



 臨川獻王郁,字深仁,幼而敏慧。道生初以無禮失旨,郁數勸以敬慎之道。道生不納,郁為之流涕,簡文帝深器異之。年十七而薨。久之,追謚獻世子。寧康初,贈左將軍,加散騎常侍,追封郡王,以武陵威王曾孫寶為嗣,追尊其母胡淑儀為臨川太妃。



 寶字弘文,歷秘書監、太常、左將軍、散騎常侍、護軍將軍。宋興,以為金紫光祿大夫,降為西豐侯,食邑千戶。



 會稽文孝王道子,字道子。出後瑯邪孝王,少以清澹為謝安所稱。年十歲,封瑯邪王,食邑一萬七千六百五十一戶,攝會稽國五萬九千一百四十戶。太元初,拜散騎常侍、中軍將軍,進驃騎將軍。後公卿奏:「道子親賢莫二,宜正位司徒。」固讓不拜。使隸尚書六條事,尋加開府,領司徒。及謝安薨,詔曰:「新喪哲輔,華戎未一,自非明賢懋德,莫能綏御內外。司徒、瑯邪王道子體道自然,神識穎遠,實當旦奭之重,宜總二南之任,可領揚州刺史、錄尚書、假節、都督中外諸軍事。衛府文武,一以配驃騎府。」讓不受。數年,領徐州刺史、太子太傅。公卿又奏:「宜進位丞
 相、揚州牧、假黃鉞,羽葆鼓吹。」並讓不受。



 於時孝武帝不親萬機,但與道子酣歌為務,姏姆尼僧,尤為親暱,並竊弄其權。凡所幸接,皆出自小豎。郡守長吏,多為道子所樹立。既為揚州總錄,勢傾天下,由是朝野奔湊。中書令王國寶性卑佞,特為道子所寵暱。官以賄遷,政刑謬亂。又崇信浮屠之學,用度奢侈,下不堪命。太元以後,為長夜之宴,蓬首昏目,政事多闕。桓玄嘗候道子,正遇其醉,賓客滿坐,道子張目謂人曰:「桓溫晚途欲作賊,云何?」玄伏地流汗不得起。長史謝重舉板答曰:「故宣武公黜昏登聖,功超伊霍,紛紜之議,宜裁之聽覽。」道子頷曰:「儂知
 儂知。」因舉酒屬玄,玄乃得起。由是玄益不自安,切齒於道子。



 于時朝政既紊,左衛領營將軍會稽許榮上疏曰:「今臺府局吏、直衛武官及僕隸婢兒取母之姓者,本臧獲之徒,無鄉邑品第,皆得命議,用為郡守縣令,並帶職在內,委事於小吏手中;僧尼乳母,競進親黨,又受貨賂,輒臨官領眾。無衛霍之才,而比方古人,為患一也。臣聞佛者清遠玄虛之神,以五誡為教,絕酒不淫。而今之奉者,穢慢阿尼,酒色是耽,其違二矣。夫致人於死,未必手刃害之。若政教不均,暴濫無罪,必夭天命,其違三矣。盜者未必躬竊人財,江乙母失布,罪由令尹。今禁令不明,
 劫盜公行,其違四矣。在上化下,必信為本。昔年下書,敕使盡規,而眾議兼集,無所採用,其違五矣。尼僧成群,依傍法服。誡粗法,尚不能遵,況精妙乎!而流惑之徒,競加敬事,又侵漁百姓,取財為惠,亦未合布施之道也。」又陳「太子宜出臨東宮,剋獎德業」。疏奏,並不省。中書郎范寧亦深陳得失,帝由是漸不平於道子,然外每優崇之。國寶即寧之甥,以諂事道子,寧奏請黜之。國寶懼,使陳郡袁悅之因尼妙音致書與太子母陳淑媛,說國寶忠謹,宜見親信。帝因發怒,斬悅之。國寶甚懼,復潛寧於帝。帝不獲已,流涕出寧為豫章太守。道子由是專恣。



 嬖人
 趙牙出自優倡,茹千秋本錢塘捕賊吏,因賂諂進,道子以牙為魏郡太守,千秋驃騎諮議參軍。牙為道子開東第,築山穿池,列樹竹木,功用鉅萬。道子使宮人為酒肆,沽賣於水側,與親暱乘船就之飲宴,以為笑樂。帝嘗幸其宅,謂道子曰:「府內有山,因得遊矚,甚善也。然修飾太過,非示天下以儉。」道子無以對,唯唯而已,左右侍臣莫敢有言。帝還宮,道子謂牙曰:「上若知山是板築所作,爾必死矣。」牙曰:「公在,牙何敢死!」營造彌甚。千秋賣官販爵,聚資貨累億。



 又道子既為皇太妃所愛,親遇同家人之禮,遂恃寵乘酒,時失禮敬。帝益不能平,然以太妃之故,
 加崇禮秩。博平令吳興聞人奭上疏曰:「驃騎諮議參軍茹千秋協輔宰相,起自微賤,竊弄威權,衒賣天官。其子壽齡為樂安令,贓私狼藉,畏法奔逃,竟無罪罰,傲然還縣。又尼姏屬類,傾動亂時。穀賤人飢,流殣不絕,由百姓單貧,役調深刻。又振武將軍庾恆鳴角京邑,主簿戴良夫苦諫被囚,殆至沒命。而恒以醉酒見怒,良夫以執忠廢棄。又權寵之臣,各開小府,施置吏佐,無益於官,有損於國。」疏奏,帝益不平,而逼於太妃,無所廢黜,乃出王恭為兗州,殷仲堪為荊州,王珣為僕射,王雅為太子少傳,以張王室,而潛制道子也。道子復委任王緒,由是朋黨
 競扇,友愛道盡。太妃每和解之,而道子不能改。



 中書郎徐邈以國之至親,唯道子而已,宜在敦穆,從容言於帝曰:「昔漢文明主,猶悔淮南;世祖聰達,負愧齊王。兄弟之際,實宜深慎。」帝納之,復委任道子如初。



 時有人為《雲中詩》以指斥朝廷曰:「相王沈醉,輕出教命。捕賊千秋,干豫朝政。王愷守常,國寶馳競。荊州大度,散誕難名;盛德之流,法護、王寧;仲堪、仙民,特有言詠,東山安道,執操高抗,何不征之,以為朝匠?」荊州,謂王忱也;法護,即王殉;寧,即王恭;仙民,即徐邈字;安道,戴逵字也。



 及恭帝為瑯邪王,道子受封會稽國,並宣城為五萬九千戶。安帝踐阼,有
 司奏:「道子宜進位太傅、揚州牧、中書監,假黃鉞,備殊禮。」固辭不拜,又解徐州。詔內外眾事,動靜諮之。帝既冠,道子稽首歸政,王國寶始總國權,勢傾朝廷。王恭乃舉兵討之。道子懼,收國實付廷尉,并其徒弟瑯邪內史緒悉斬之,以謝於恭,恭即罷兵。道子乞解中外都督、錄尚書以謝方岳,詔不許。



 道子世子元顯,時年十六,為侍中,心惡恭,請道子討之。乃拜元顯為征虜將軍,其先衛府及徐州文武悉配之。屬道子妃薨,帝下詔曰:「會稽王妃尊賢莫二,朕義同所親。今葬加殊禮,一依瑯邪穆太妃故事。元顯夙令光懋,乃心所寄,誠孝性蒸蒸,至痛難奪。然
 不以家事辭王事,《陽秋》之明義;不以私限違公制,中代之變禮。故閔子腰絰,山王逼屈。良以至戚由中,軌容著外,有禮無時,賢哲斯順。須妃葬畢,可居職如故。」



 于時王恭威振內外,道子甚懼,復引譙王尚之以為腹心。尚之說道子曰:「籓伯彊盛,宰相權輕,宜密樹置,以自籓衛。」道子深以為然,乃以其司馬王愉為江州刺史以備恭,與尚之等日夜謀議,以伺四方之隙。王恭知之,復舉兵,以討尚之為名。荊州刺史殷仲堪、豫州刺史庾楷、廣州刺史桓玄並應之。道子使人說楷曰:「本情相與,可謂斷金。往年帳中之飲,結帶之言,寧可忘邪!卿今棄舊交,結新
 援,忘王恭疇昔陵侮之恥乎,若乃欲委體而臣之。若恭得志,以卿為反覆之人,必不相信,何富貴可保,禍敗亦旋及矣!」楷怒曰:「王恭昔赴山陵,相王憂懼無計,我知事急,即勒兵而至。去年之事,亦俟命而奮。我事相王,無相負者。既不能距恭,反殺國寶。自爾已來,誰復敢攘袂於君之事乎!庾楷實不能以百口助人屠滅,當與天下同舉,誅鉏姦臣,何憂府不開,爵不至乎!」時楷已應恭檄,正徵士馬。信反,朝廷憂懼,於是內外戒嚴。元顯攘袂慷慨謂道子曰:「去年不討王恭,致有今役。今若復從其欲,則太宰之禍至矣。」道子日飲醇酒,而委事於元顯。元顯雖
 年少,而聰明多涉,志氣果銳,以安危為己任。尚之為之羽翼。時相傅會者,皆謂元顯有明帝神武之風。於是以為征討都督、假節,統前將軍王珣、左將軍謝琰及將軍桓之才、毛泰、高素等伐恭,滅之。



 既而楊佺期、桓玄、殷仲堪等復至石頭,元顯於竹里馳還京師,遣丹陽尹王愷、鄱陽太守桓放之、新蔡內史何嗣、潁川太守溫詳、新安太守孫泰等,發京邑士庶數萬人,據石頭以距之。道子將出頓中堂,忽有驚馬蹂藉軍中,因而擾亂,赴江而死者甚眾。仲堪既知王恭敗死,狼狽西走,與桓玄屯于尋陽。朝廷嚴兵相距,內外騷然。詔元顯甲杖百人入殿,尋
 加散騎常侍、中書令,又領中領軍,持節、都督如故。



 會道子有疾,加以昏醉,元顯知朝望去之,謀奪其權,諷天子解道子揚州、司徒,而道子不之覺元顯自以少年頓居權重,慮有譏議,於是以瑯邪王領司徒,元顯自為揚州刺史。既而道子酒醒,方知去職,於是大怒,而無如之何。廬江太守會稽張法順以刀筆之才,為元顯謀主,交結朋援,多樹親黨,自桓謙以下,諸貴遊皆斂衽請交。元顯性苛刻,生殺自己,法順屢諫,不納。又發東土諸郡免奴為客者,號曰「樂屬」,移置京師,以充兵役,東土囂然,人不堪命,天下苦之矣。既而孫恩乘釁作亂,加道子黃鉞,元
 顯為中軍以討之。又加元顯錄尚書事。然道子更為長夜之飲,政無大小,一委元顯。時謂道子為東錄,元顯為西錄。西府車騎填湊,東第門下可設雀羅矣。元顯無良師友,正言弗聞,諂譽日至,或以為一時英傑,或謂為風流名士,由是自謂無敵天下,故驕侈日增。帝又以元顯有翼亮之功,加其所生母劉氏為會稽王夫人,金章紫綬。會洛陽覆沒,道子以山陵幽辱,上疏送章綬,請歸籓,不許。及太皇太后崩,詔道子乘輿入殿。元顯因諷禮官下議,稱己德隆望重,既錄百揆,內外群僚皆應盡敬。於是公卿皆拜。於時軍旅薦興,國用虛竭,自司徒已下,日
 廩七升,而元顯聚斂不已,富過帝室。及謝琰為孫恩所害,元顯求領徐州刺史,加侍中、後將軍、開府儀同三司、都督十六州諸軍事,封其子彥璋為東海王。尋以星變,元顯解錄,復加尚書令。



 會孫恩至京口,元顯柵斷石頭,率兵距戰,頻不利。道子無他謀略,唯日禱蔣侯廟為厭勝之術。既而孫恩遁于北海,桓玄復據上流,致箋於道子曰:「賊造近郊,以風不得進,以雨不致火,食盡故去耳,非力屈也。昔國寶卒後,王恭不乘此威入統朝政,足見其心非侮於明公也,而謂之非忠。今之貴要腹心,有時流清望者誰乎?豈可云無佳勝,直是不能信之耳。用理
 之人,然後可以信義相期;求利之徒,豈有所惜而更委信邪?爾來一朝一夕,遂成今日之禍矣。阿衡之重,言何容易,求福則立至,干忤或致禍。在朝君子,豈不有懷,但懼害及身耳。玄忝任在遠,是以披寫事實。」元顯覽而大懼。張法順謂之曰:「桓玄承籍門資,素有豪氣,既並殷、楊,專有荊楚。然桓氏世在西籓,人或為用,而第下之所控引,止三吳耳。孫恩為亂,東土塗地,編戶饑饉,公私不贍,玄必乘此縱其姦兇,竊用憂之。」元顯曰:「為之奈何?」法順曰:「玄始據荊州,人情未輯,方就綏撫,未遑他計。及其如此,發兵誅之,使劉牢之為前鋒,而第下以大軍繼進,桓
 玄之首必懸於麾下矣。」元顯以為然,遣法順至京口,謀於牢之,而牢之有疑色。法順還,說元顯曰:「觀牢之顏色,必貳於我,未若召入殺之。不爾,敗人大事。」元顯不從。



 道子尋拜侍中、太傅,置左右長史、司馬、從事中郎四人,崇異之儀,備盡盛典。其驃騎將軍僚佐文武,即配太傅府。加元顯侍中、驃騎大將軍、開府、征討大都督、十八州諸軍事、儀同三司,加黃鉞,班劍二十人,以伐桓玄,竟以牢之為前鋒。法順又言於元顯曰:「自舉大事,未有威斷,桓謙兄弟每為上流耳目,斬之,以孤荊楚之望。且事之濟不,繼在前軍,而牢之反覆,萬一有變,則禍敗立至。可令
 牢之殺謙兄弟,以示不貳。若不受命,當逆為其所。」元顯曰:「非牢之無以當桓玄。且始事而誅大將,人情必動,二三不可。」于時揚土饑虛,運漕不繼,玄斷江路,商旅遂絕。於是公私匱乏,士卒唯給粰橡。



 大軍將發,玄從兄驃騎長史石生馳使告玄。玄進次尋陽,傳檄京師,罪狀元顯。俄而玄至西陽,帝戎服餞元顯于西池,始登舟而玄至新亭。元顯棄船退屯國子學堂。明日,列陣於宣陽門外,元顯佐吏多散走。或言玄已至大桁,劉牢之遂降于玄。元顯回入宣陽門,牢之參軍張暢之率眾遂之,眾潰。元顯奔入相府,唯張法順隨之。問計於道子,道子對之泣。
 玄遣太傅從事中郎毛泰收元顯送于新亭,縛於舫前而數之。元顯答曰:「為王誕、張法順所誤。」於是送付廷尉,並其六子皆害之。玄又奏:「道子酣縱不孝,當棄市。」詔徒安成郡,使御史杜竹林防衛,竟承玄旨殺之,時年三十九。帝三日哭於西堂。



 及玄敗,大將軍、武陵王遵承旨下令曰:「故太傅公阿衡二世,契闊皇家,親賢之重,地無與二。驃騎大將軍內總朝維,外宣威略,志蕩世難,以寧國祚。天未靜亂,禍酷備鐘,悲動區宇,痛貫人鬼,感惟永往,心情崩隕。今皇祚反正,幽顯式敘,宜崇明國體,以述舊典。便可追崇太傅為丞相,加殊禮,一依安平獻王故
 事。追贈驃騎為太尉,加羽葆鼓吹。丞相填塋翳然,飄薄非所,須南道清通,便奉迎神柩。太尉宜便遷改。可下太史祥吉日,定宅兆。」於是遣通直常侍司馬珣之迎道子柩于安成。時寇賊未平,喪不時達。義熙元年,合葬于王妃陵。追謚元顯曰忠。以臨川王寶子修之為道子嗣,尊妃王氏為太妃。義熙中,有稱元顯子秀熙避難蠻中而至者,太妃請以為嗣,於是脩之歸于別第。劉裕意其詐而案驗之,果散騎郎滕羨奴勺藥也,竟坐棄市。太妃不悟,哭之甚慟。脩之復為嗣。薨,謚悼王,無子,國除。



 史臣曰:泰始之受終也,乃憲章往昔,稽古前王,廣誓山
 河,大開籓屏,文昭武穆,方駕於魯、衛、應、韓;磐石犬牙,連衡於吳、楚、齊、代。然而作法於亂,付托非才,何曾嘆經國之無謀,郭欽識危亡之有兆。及宮車晏駕,填土未幹,國難薦臻,朝章馳廢。重以八王繼亂,九服沸騰,戎羯交馳,乘輿幽逼,瑤枝瓊萼,鋒鏑而消亡;硃芾綠車,與波塵而殄瘁。遂使茫茫禹跡,咸窟穴於豺狼;惵惵周餘,竟沈淪於塗炭。嗚呼!運極數窮,一至於此!詳觀載籍,未或前聞。道子地則親賢,任惟元輔,耽荒曲蘗,信惑讒諛。遂使尼媼竊朝權,奸邪制國命,始則彞倫攸斁,終則宗社淪亡。元顯以童丱之年,受棟梁之寄,專制朝廷,陵蔑君親,
 奮庸瑣之常材,抗奸兇之臣寇,喪師殄國。不亦宜乎!斯則元顯為安帝之孫強,道子實晉朝之宰嚭者也。列代之崇建維城,用籓王室;有晉之分封子弟,實樹亂階。《詩》云:「懷德惟寧,宗子維成。無俾城壞,無獨期畏。」城既壞矣,畏也宜哉!典午之喪亂弘多,實此之由矣。



 贊曰:帝子分封,嬰此鞠兇。札瘥繼及,禍難仍鐘。秦獻聰悟,清河內顧。淮南忠勇,宣城識度。道子昏兇,遂傾國祚。



\end{pinyinscope}