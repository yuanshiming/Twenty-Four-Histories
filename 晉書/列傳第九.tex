\article{列傳第九}

\begin{pinyinscope}
王沈
 \gezhu{
  子浚}
 荀顗荀勖
 \gezhu{
  子籓籓子邃闓籓弟組組子奕}
 馮紞



 王沈,字處道,太原晉陽人也。祖柔,漢匈奴中郎將。父機,魏東郡太守。沈少孤,養於從叔司空昶,事昶如父。奉繼母寡嫂以孝義稱。好書,善屬文。大將軍曹爽辟為掾,累遷中書門下侍郎。及爽誅,以故吏免。後起為治書侍御史,轉秘書監。正元中,遷散騎常侍、侍中,典著作。與荀顗、阮籍共撰《魏書》,多為時諱,未若陳壽之實錄也。



 時魏高
 貴鄉公好學有文才,引沈及裴秀數於東堂講宴屬文,號沈為文籍先生,秀為儒林丈人。及高貴鄉公將攻文帝,召沈及王業告之,沈、業馳白帝,以功封安平侯,邑二千戶。沈既不忠於主,甚為眾論所非。



 尋遷尚書,出監豫州諸軍事、奮武將軍、豫州刺史。至鎮,乃下教曰:「自古賢聖,樂聞誹謗之言,聽輿人之論,芻蕘有可錄之事,負薪有廊廟之語故也。自至鎮日,未聞逆耳之言,豈未明虛心,故令言者有疑。其宣下屬城及士庶,若能舉遺逸於林藪,黜姦佞於州國,陳長吏之可否,說百姓之所患,興利除害,損益昭然者,給穀五百斛。若達一至之言,說刺
 史得失,朝政寬猛,令剛柔得適者,給穀千斛。謂餘不信,明如皎日。」主簿陳廞、褚曰:「奉省教旨,伏用感歎。勞謙日昃,思聞苦言。愚謂上之所好,下無不應。而近未有極諫之辭,遠無傳言之箴者,誠得失之事將未有也。今使教命班下,示以賞勸,將恐拘介之士,或憚賞而不言;貪賕之人,將慕利而妄舉。茍不合宜,賞不虛行,則遠聽者未知當否之所在,徒見言之不用,謂設有而不行。愚以告下之事,可小須後。」



 沈又教曰:「夫德薄而位厚,功輕而祿重,貪夫之所徇,高士之所不處也。若陳至言於刺史,興益於本州,達幽隱之賢,去祝鮀之佞,立德於上,受分
 於下,斯乃君子之操,何不言之有!直言至理,忠也。惠加一州,仁也。功成辭賞,廉也。兼斯而行,仁智之事,何故懷其道而迷其國哉!」褚復白曰:「堯、舜、周公所以能致忠諫者,以其款誠之心著也。冰炭不言,而冷熱之質自明者,以其有實也。若好忠直,如冰炭之自然,則諤諤之臣,將濟濟而盈庭;逆耳之言,不求而自至。若德不足以配唐虞,明不足以並周公,實不可以同冰炭,雖懸重賞,忠諫之言未可致也。昔魏絳由和戎之功,蒙女樂之賜,管仲有興齊之勳,而加上卿之禮,功勳明著,然後賞勸隨之。未聞張重賞以待諫臣,懸穀帛以求盡言也。」沈無以
 奪之,遂從議。



 沈探尋善政,案賈逵以來法制禁令,諸所施行,擇善者而從之。又教曰:「後生不聞先王之教,而望政道日興,不可得也。文武並用,長久之道也。俗化陵遲,不可不革。革俗之要,實在敦學。昔原伯魯不悅學,閔馬父知其必亡。將吏子弟,優閑家門,若不教之,必致游戲,傷毀風俗矣。」於是九郡之士,咸悅道教,移風易俗。



 遷征虜將軍、持節、都督江北諸軍事。五等初建,封博陵侯,班在次國。平蜀之役,吳人大出,聲為救蜀,振蕩邊境,沈鎮御有方,寇聞而退。轉鎮南將軍。武帝即王位,拜御史大夫,守尚書令,加給事中。沈以才望,顯名當世,是以創
 業之事,羊祜、荀勖、裴秀、賈充等,皆與沈諮謀焉。



 及帝受禪,以佐命之勳,轉驃騎將軍、錄尚書事,加散騎常侍,統城外諸軍事。封博陵郡公,固讓不受,乃進爵為縣公,邑千八百戶。帝方欲委以萬機,泰始二年薨。帝素服舉哀,賜祕器朝服一具、衣一襲、錢三十萬、布百匹、葬田一頃,謚曰元。明年,帝追思沈勳,詔曰:「夫表揚往行,所以崇賢垂訓,慎終紀遠,厚德興教也。故散騎常侍、驃騎將軍、博陵元公沈蹈禮居正,執心清粹,經綸墳典,才識通洽。入歷常伯納言之位,出乾監牧方嶽之任,內著謀猷,外宣威略。建國設官,首登公輔,兼統中朝,出納大命,實有翼
 亮佐世之勳。其贈沈司空公,以寵靈既往,使沒而不朽。又前以翼贊之勳,當受郡公之封,而固辭懇至,嘉其讓德,不奪其志。可以郡公官屬送葬。沈素清儉,不營產業。其使所領兵作屋五十間。」子浚嗣。後沈夫人荀氏卒,將合葬,沈棺櫬已毀,更賜東園秘器。咸寧中,復追封沈為郡公。



 浚字彭祖。母趙氏婦,良家女也,貧賤,出入沈家,遂生浚,沈初不齒之。年十五,沈薨,無子,親戚共立浚為嗣,拜駙馬都尉。太康初,與諸王侯俱就國。三年來朝,除員外散騎侍郎。元康初,轉員外常侍,遷越騎校尉、右軍將軍。出
 補河內太守,以郡公不得為二千石,轉東中郎將,鎮許昌。



 及愍懷太子幽于許昌,浚承賈后旨,與黃門孫慮共害太子。遷寧北將軍、青州刺史。尋徙寧朔將軍、持節、都督幽州諸軍事。于時朝廷昏亂,盜賊蜂起,浚為自安之計,結好夷狄,以女妻鮮卑務勿塵,又以一女妻蘇恕延。



 及趙王倫篡位,三王起義兵,浚擁眾挾兩端,遏絕檄書,使其境內士庶不得赴義,成都王穎欲討之而未暇也。倫誅,進號安北將軍。及河間王顒、成都王穎興兵內向,害長沙王乂,而浚有不平之心。穎表請幽州刺史石堪為右司馬,以右司馬和演代堪,密使演殺浚,並其眾。演
 與烏丸單于審登謀之,於是與浚期游薊城南清泉水上。薊城內西行有二道,演浚各從一道。演與浚欲合鹵簿,因而圖之。值天暴雨,兵器霑濕,不果而還。單于由是與其種人謀曰:「演圖殺浚,事垂克而天卒雨,使不得果,是天助浚也。違天不祥,我不可久與演同。」乃以謀告浚。浚密嚴兵,與單于圍演。演持白幡詣浚降,遂斬之,自領幽州。大營器械,召務勿塵,率胡晉合二萬人,進軍討穎。以主溥祁弘為前鋒,遇穎將石超於平棘,擊敗之。浚乘勝遂克鄴城,士眾暴掠,死者甚多。鮮卑大略婦女,浚命敢有挾藏者斬,於是沉於易水者八千人。黔庶荼毒,自
 此始也。



 浚還薊,聲實益盛。東海王越將迎大駕,浚遣祁弘率烏丸突騎為先驅。惠帝旋洛陽,轉浚驃騎大將軍、都督東夷河北諸軍事,領幽州刺史,以燕國增博陵之封。懷帝即位,以浚為司空,領烏丸校尉,務勿塵為大單于。浚又表封務勿塵遼西郡公,其別部大飄滑及其弟渴末別部大屠甕等皆為親晉王。



 永嘉中,石勒寇冀州,浚遣鮮卑文鴦討勒,勒走南陽。明年,勒復寇冀州,刺史王斌為勒所害,浚又領冀州。詔進浚為大司馬,加侍中、大都督、督幽冀諸軍事。使者未及發,會洛京傾覆,浚大樹威令,專征伐,遣督護王昌、中山太守阮豹等,率諸軍
 及務勿塵世子疾陸眷,并弟文鴦、從弟末柸,攻石勒於襄國,勒率眾來距,昌逆擊敗之。末柸逐北入其壘門,為勒所獲。勒質末柸,遣間使來和,疾陸眷遂以鎧馬二百五十匹、金銀各一簏贖末柸,結盟而退。



 其後浚布告天下,稱受中詔承制,乃以司空荀籓為太尉,光祿大夫荀組為司隸,大司農華薈為太常,中書令李絙為河南尹。又遣祁弘討勒,及於廣宗。時大霧,弘引軍就道,卒與勒遇,為勒所殺。由是劉琨與浚爭冀州。琨使宗人劉希還中山合眾,代郡、上谷、廣寧三郡人皆歸于琨。浚患之,遂輟討勒之師,而與琨相距。浚遣燕相胡矩督護諸軍,與
 疾陸眷並力攻破希。驅略三郡士女出塞,琨不復能爭。



 浚還,欲討勒,使棗嵩督諸軍屯易水,召疾陸眷,將與之俱攻襄國。浚為政苛暴,將吏又貪殘,並廣占山澤,引水灌田,漬陷冢墓,調發殷煩,下不堪命,多叛入鮮卑。從事韓咸切諫,浚怒,殺之。疾陸眷自以前後違命,恐浚誅之。勒亦遣使厚賂,疾陸眷等由是不應召。浚怒,以重幣誘單于猗盧子右賢王日律孫,令攻疾陸眷,反為所破。



 時劉琨大為劉聰所迫,諸避亂游士多歸於浚。浚日以彊盛,乃設壇告類,建立皇太子,備置眾官。浚自領尚書令,以棗嵩、裴憲並為尚書,使其子居王宮,持節,領護匈奴
 中郎將,以妻舅崔毖為東夷校尉。又使嵩監司冀並兗諸軍事、行安北將軍,以田徽為兗州,李惲為青州。惲為石勒所殺,以薄盛代之。



 浚以父字處道,為「當塗高」應王者之讖,謀將僭號。胡矩諫浚,盛陳其不可。浚忿之,出矩為魏郡守。前渤海太守劉亮、從子北海太守搏、司空掾高柔並切諫,浚怒,誅之。浚素不平長史燕國王悌,遂因他事殺之。時童謠曰:「十囊五囊入棗郎。」棗嵩,浚之子婿也。浚聞,責嵩而不能罪之也。又謠曰:「幽州城門似藏戶,中有伏尸王彭祖。」有狐踞府門,翟雉入聽事。時燕國霍原,北州名賢,浚以僭位事示之,原不答,浚遂害之。由是士
 人憤怨,內外無親。以矜豪日甚,不親為政,所任多苛刻;加亢旱災蝗,士卒衰弱。



 浚之承制也,參佐皆內敘,唯司馬游統外出。統怨,密與石勒通謀。勒乃詐降於浚,許奉浚為主。時百姓內叛,疾陸眷等侵逼。浚喜勒之附己,勒遂為卑辭以事之。獻遺珍寶,使驛相繼。浚以勒為誠,不復設備。勒乃遣使剋日上尊號於浚,浚許之。



 勒屯兵易水,督護孫緯疑其詐,馳白浚,而引軍逆勒。浚不聽,使勒直前。眾議皆曰:「胡貪而無信,必有詐,請距之。」浚怒,欲斬諸言者,眾遂不敢復諫。盛張設以待勒。勒至城,便縱兵大掠。浚左右復請討之,不許。及勒登聽事,浚乃走出堂
 皇,勒眾執以見勒。勒遂與浚妻並坐,立浚于前。浚罵曰:「胡奴調汝公,何凶逆如此!」勒數浚不忠於晉,并責以百姓餒乏,積粟五十萬斛而不振給。遂遣五百騎先送浚于襄國,收浚麾下精兵萬人,盡殺之。停二日而還,孫緯遮擊之,勒僅而得免。勒至襄國,斬浚,而浚竟不為之屈,大罵而死。無子。



 太元二年,詔興滅繼絕,封沈從孫道素為博陵公。卒,子崇之嗣。義熙十一年,改封東莞郡公。宋受禪,國除。



 荀顗,字景倩,潁川人,魏太尉彧之第六子也。幼為姊婿
 陳群所賞。性至孝,總角知名,博學洽聞,理思周密。魏時以父勛除中郎。宣帝輔政,見顗奇之,曰:「荀令君之子也。」擢拜散騎侍郎,累遷侍中。為魏少帝執經,拜騎都尉,賜爵關內侯。難鐘會《易》無互體,又與扶風王駿論仁孝孰先,見稱於世。



 時曹爽專權,何晏等欲害太常傅嘏,顗營救得免。及高貴鄉公立,顗言於景帝曰:「今上踐阼,權道非常,宜速遣使宣德四方,且察外志。」毌丘儉、文欽果不服,舉兵反。顗預討儉等有功,進爵萬歲亭侯,邑四百戶。文帝輔政,遷尚書。帝徵諸葛誕,留顗鎮守。顗甥陳泰卒,顗代泰為僕射,領吏部,四辭而後就職。顗承泰後,加之
 淑慎,綜核名實,風俗澄正。咸熙中,遷司空,進爵鄉侯。



 顗年踰耳順,孝養蒸蒸,以母憂去職,毀幾滅性,海內稱之。文帝奏,宜依漢太傅胡廣喪母故事,給司空吉凶導從。及蜀平,興復五等,命顗定禮儀。顗上請羊祜、任愷、庚峻、應貞、孔顥共刪改舊文,撰定晉禮。



 咸熙初,封臨淮侯。武帝踐阼,進爵為公,食邑一千八百戶。又詔曰:「昔禹命九官,契敷五教,所以弘崇王化,示人軌儀也。朕承洪業,昧于大道,思訓五品,以康四海。侍中、司空顗,明允篤誠,思心通遠,翼亮先皇,遂輔朕躬,實有佐命弼導之勛。宜掌教典,以隆時雍。其以顗為司徒。」尋加侍中,遷太尉、都督
 城外牙門諸軍事,置司馬親兵百人。頃之,又詔曰:「侍中、太尉顗,溫恭忠允,至行純備,博古洽聞,耆艾不殆。其以公行太子太傅,侍中、太尉如故。」



 時以《正德》、《大豫》雅頌未合,命顗定樂。事未終,以泰始十年薨。帝為舉哀,皇太子臨喪,二宮賻贈,禮秩有加。詔曰:「侍中、太尉、行太子太傅、臨淮公顗,清純體道,忠允立朝,歷司外內,茂績既崇,訓傅東宮,徽猷弘著,可謂行歸于周,有始有卒者矣。不幸薨殂,朕甚痛之。其賜溫明祕器、朝服一具,衣一襲。謚曰康。」又詔曰:「太尉不恤私門,居無館宇,素絲之志,沒而彌顯。其賜家錢二百萬,使立宅舍。」咸寧初,詔論次功臣,將
 配饗宗廟。所司奏顗等十二人銘功太常,配饗清廟。



 顗明《三禮》,知朝廷大儀,而無質直之操,唯阿意茍合於荀勖、賈充之間。初,皇太子將納妃,顗上言賈充女姿德淑茂,可以參選,以此獲譏於世。



 顗無子,以從孫徽嗣。中興初,以顗兄玄孫序為顗後,封臨淮公。序卒,又絕,孝武帝又封序子恆繼顗後。恒卒,子龍符嗣。宋受禪,國除。



 荀勖,字公曾,潁川潁陰人,漢司空爽曾孫也。祖棐,射聲校尉。父肸,早亡。勖依于舅氏。岐嶷夙成,年十餘歲能屬文。從外祖魏太傅鐘繇曰:「此兒當及其曾祖。」既長,遂博
 學,達於從政。仕魏,辟大將軍曹爽掾,遷中書通事郎。爽誅,門生故吏無敢往者,勖獨臨赴,眾乃從之。為安陽令,轉驃騎從事中郎。勖有遺愛,安陽生為立祠。遷廷尉正,參文帝大將軍軍事,賜爵關內侯,轉從事中郎,領記室。



 高貴鄉公欲為變時,大將軍掾孫佑等守閶闔門。帝弟安陽侯幹聞難欲入,佑謂乾曰:「未有入者,可從東掖門。」及幹至,帝遲之,乾以狀白,帝欲族誅佑。勖諫曰:「孫佑不納安陽,誠宜深責。然事有逆順,用刑不可以喜怒為輕重。今成倅刑止其身,佑乃族誅,恐義士私議。」乃免佑為庶人。時官騎路遺求為刺客入蜀,勖言於帝曰:「明公以
 至公宰天下,宜杖正義以伐違貳。而名以刺客除賊,非所謂刑于四海,以德服遠也。」帝稱善。



 及鐘會謀反,審問未至,而外人先告之。帝待會素厚,未之信也。勖曰:「會雖受恩,然其性未可許以見得思義,不可不速為之備。」帝即出鎮長安,主簿郭奕、參軍王深以勖是會從甥,少長舅氏,勸帝斥出之。帝不納,而使勖陪乘,待之如初。先是,勖啟「伐蜀,宜以衛瓘為監軍」。及蜀中亂,賴瓘以濟。會平,還洛,與裴秀、羊祜共管機密。



 時將發使聘吳,並遣當時文士作書與孫皓,帝用勖所作。皓既報命和親,帝謂勖曰:「君前作書,使吳思順,勝十萬之眾也。」帝即晉王位,以
 勖為侍中,封安陽子,邑千戶。武帝受禪,改封濟北郡公。勖以羊祜讓,乃固辭為侯。拜中書監,加侍中,領著作,與賈充共定律令。



 充將鎮關右也,勖謂馮紞曰:「賈公遠放,吾等失勢。太子婚尚未定,若使充女得為妃,則不留而自停矣。」勖與紞伺帝間並稱「充女才色絕世,若納東宮,必能輔佐君子,有《關雎》后妃之德。」遂成婚。當時甚為正直者所疾,而獲佞媚之譏焉。久之,進位光祿大夫。既掌樂事,又修律呂,並行於世。初,勖於路逢趙賈人牛鐸,識其聲。及掌樂,音韻未調,乃曰:「得趙之牛鐸則諧矣。」遂下郡國,悉送牛鐸,果得諧者。又嘗在帝坐進飯,謂在坐人
 曰:「此是勞薪所炊。」咸未之信。帝遣問膳夫,乃云:「實用故車腳。」舉世伏其明識。俄領祕書監,與中書令張華依劉向《別錄》,整理記籍。又立書博士,置弟子教習,以鐘、胡為法。



 咸寧初,與石苞等並為佐命功臣,列於銘饗。及王濬表請伐吳,勖與賈充固諫不可,帝不從,而吳果滅。以專典詔命,論功封子一人為亭侯,邑一千戶,賜絹千匹。又封孫顯為潁陽亭侯。



 及得汲郡冢中古文竹書,詔勖撰次之,以為《中經》,列在祕書。



 時議遣王公之國,帝以問勖,勖對曰:「諸王公已為都督,而使之國,則廢方任。又分割郡縣,人心戀本,必用嗷嗷。國皆置軍,官兵還當給國,而
 闕邊守。」帝重使勖思之,勖又陳曰:「如詔準古方伯選才,使軍國各隨方面為都督,誠如明旨。至於割正封疆。使親疏不同誠為佳矣。然分裂舊土,猶懼多所搖動,必使人心聰擾,思惟竊宜如前。若於事不得不時有所轉封,而不至分割土域,有所損奪者,可隨宜節度。其五等體國經遠,實不成制度。然但虛名,其於實事,略與舊郡縣鄉亭無異。若造次改奪,恐不能不以為恨。今方了其大者,以為五等可須後裁度。凡事雖有久而益善者,若臨時或有不解,亦不可忽。」帝以勖言為允,多從其意。



 時又議省州郡縣半吏以赴農功,勖議以為:「省吏不如省官,
 省官不如省事,省事不如清心。昔蕭曹相漢,載其清靜,致畫一之歌,此清心之本也。漢文垂拱,幾致刑措,此省事也。光武並合吏員,縣官國邑裁置十一,此省官也。魏太和中,遣王人四出,減天下吏員,正始中亦並合郡縣,此省吏也。今必欲求之於本,則宜以省事為先。凡居位者,使務思蕭曹之心,以翼佐大化。篤義行,崇敦睦,使昧寵忘本者不得容,而偽行自息,浮華者懼矣。重敬讓,尚止足,令賤不妨貴,少不陵長,遠不間親,新不間舊,小不加大,淫不破義,則上下相安,遠近相信矣。位不可以進趣得,譽不可以朋黨求,則是非不妄而明,官人不惑於
 聽矣。去奇技,抑異說,好變舊以徼非常之利者必加其誅,則官業有常,人心不遷矣。事留則政稽,政稽則功廢。處位者而孜孜不怠,奉職司者而夙夜不懈,則雖在挈瓶而守不假器矣。使信若金石,小失不害大政,忍忿悁以容之。簡文案,略細苛,令之所施,必使人易視聽,願之如陽春,畏之如雷震。勿使微文煩撓,為百吏所黷,二三之命,為百姓所饜,則吏竭其誠,下悅上命矣。設官分職,委事責成。君子心競而不力爭,量能受任,思不出位,則官無異業,政典不奸矣。凡此皆愚心謂省事之本也。茍無此愆,雖不省吏,天下必謂之省矣。若欲省官,私謂九
 寺可并於尚書,蘭臺宜省付三府。然施行歷代,世之所習,是以久抱愚懷而不敢言。至於省事,實以為善。若直作大例,皆減其半,恐文武眾官郡國職業,及事之興廢,不得皆同。凡發號施令,典而當則安,儻有駁者,或致壅否。凡職所臨履,先精其得失。使忠信之官,明察之長,各裁其中,先條上言之。然後混齊大體,詳宜所省,則令下必行,不可搖動。如其不爾,恐適惑人聽,比前行所省,皆須臾輒復,或激而滋繁,亦不可不重。」勖論議損益多此類。



 太康中詔曰:「勖明哲聰達,經識天序,有佐命之功,兼博洽之才。久典內任,著勳弘茂,詢事考言,謀猷允誠。宜
 登大位,毗贊朝政。今以勖為光祿大夫、儀同三司、開府辟召,守中書監、侍中、侯如故。」時太尉賈充、司徒李胤並薨,太子太傅又缺,勖表陳:「三公保傅,宜得其人。若使楊珧參輔東宮,必當仰稱聖意。尚書令衛瓘、吏部尚書山濤皆可為司徒。若以瓘新為令未出者,濤即其人。」帝並從之。



 明年秋,諸州郡大水,兗土尤甚。勖陳宜立都水使者。其後門下啟通事令史伊羨、趙咸為舍人,對掌文法。詔以問勖,勖曰:今天下幸賴陛下聖德,六合為一,望道化隆洽,垂之將來。而門下上稱程咸、張惲,下稱此等,欲以文法為政,皆愚臣所未達者。昔張釋之諫漢文,謂獸
 圈嗇夫不宜見用;邴吉住車,明調和陰陽之本。此二人豈不知小吏之惠,誠重惜大化也。昔魏武帝使中軍司荀攸典刑獄,明帝時猶以付內常侍。以臣所聞,明帝時唯有通事劉泰等官,不過與殿中同號耳。又頃言論者皆云省官減事,而求益吏者相尋矣。多云尚書郎太令史不親文書,乃委付書令史及幹,誠吏多則相倚也。增置文法之職,適恐更耗擾臺閣,臣竊謂不可。」



 時帝素知太子闇弱,恐後亂國,遣勖及和嶠往觀之。勖還盛稱太子之德,而嶠云太子如初。於是天下貴嶠而賤勖。帝將廢賈妃,勖與馮紞等諫請,故得不廢。時議以勖傾國害
 時,孫資、劉放之匹。然性慎密,每有詔令大事,雖已宣布,然終不言,不欲使人知己豫聞也。族弟良曾勸勖曰:「公大失物情,有所進益者自可語之,則懷恩多矣。」其婿武統亦說勖「宜有所營置,令有歸戴者」。勖並默然不應,退而語諸子曰:「人臣不密則失身,樹私則背公,是大戒也。汝等亦當宦達人間,宜識吾此意。」久之,以勖守尚書令。



 勖久在中書,專管機事。及失之,甚罔罔悵恨。或有賀之者,勖曰:「奪我鳳皇池,諸君賀我邪!」及在尚書,課試令史以下,核其才能,有闇於文法,不能決疑處事者,即時遣出。帝嘗謂曰:「魏武帝言『荀文若之進善,不進不止;荀公
 達之退惡,不退不休』。二令君之美,亦望於君也。」居職月餘,以母憂上還印綬,帝不許。遣常侍周恢喻旨,勖乃奉詔視職。



 勖久管機密,有才思,探得人主微旨,不犯顏忤爭,故得始終全其寵祿。太康十年卒,詔贈司徒,賜東園祕器、朝服一具、錢五十萬、布百匹。遣兼御史持節護喪,謚曰成。勖有十子,其達者輯、籓、組。



 輯嗣,官至衛尉。卒,謚曰簡。子畯嗣。卒,謚曰烈。無嫡子,以弟息識為嗣。輯子綽。



 綽字彥舒,博學有才能,撰《晉後書》十五篇,傳於世。永嘉末,為司空從事中郎,沒於石勒,為勒參軍。



 籓字大堅。元康中,為黃門侍郎,受詔成父所治鐘磬。以
 從駕討齊王冏勛,封西華縣公。累遷尚書令。永嘉末,轉司空,未拜而洛陽陷沒,籓出奔密。王浚承制,奉籓為留臺太尉。及愍帝為太子,委籓督攝遠近。建興元年薨於開封,年六十九,因葬亡所。謚曰成,追贈太保。籓二子:邃、闓。



 邃字道玄,解音樂,善談論。弱冠辟趙王倫相國掾,遷太子洗馬。長沙王乂以為參軍。乂敗,成都王為皇太弟,精選僚屬,以邃為中舍人。鄴城不守,隨籓在密。元帝召為丞相從事中郎,以道險不就。愍帝就加左將軍、陳留相。父憂去職,服闋,襲封。愍帝欲納邃女,先徵為散騎常侍。
 邃懼西都危逼,故不應命,而東渡江,元帝以為軍諮祭酒。太興初,拜侍中。邃與刁協婚親,時協執權,欲以邃為吏部尚書,邃深距之。尋而王敦討協,協黨與並及於難,唯邃以疏協獲免。敦表為廷尉,以疾不拜。遷太常,轉尚書。蘇峻作亂,邃與王導、荀崧並侍天子於石頭。峻平後卒,贈金紫光祿大夫,謚曰靖。子汪嗣。



 闓字道明,亦有名稱,京都為之語曰:「洛中英英荀道明。」大司馬、齊王冏辟為掾。冏敗,暴尸已三日,莫敢收葬。闓與冏故吏李述、嵇含等露板請葬,朝議聽之,論者稱焉。為太傅主簿、中書郎。與邃俱渡江,拜丞相軍諮祭酒。中
 興建,遷右軍將軍,轉少府。明帝嘗從容問王暠曰:「二荀兄弟孰賢?」暠答以闓才明過邃。帝以語庾亮,亮曰:「邃真粹之地,亦闓所不及。」由是議者莫能定其兄弟優劣。歷御史中丞、侍中、尚書,封射陽公。太寧二年卒,追贈衛尉,謚曰定。子達嗣。



 組字大章。弱冠,太尉王衍見而稱之曰:「夷雅有才識。」初為司徒左西屬,補太子舍人。司徒王渾請為從事中郎,轉左長史,歷太子中庶子、滎陽太守。



 趙王倫為相國,欲收大名,選海內德望之士,以江夏李重及組為左右長史,東平王堪沛國劉謨為左右司馬。倫篡,以組為侍中。
 及長沙王乂敗,惠帝遣組及散騎常侍閭丘沖詣成都王穎,慰勞其軍。帝西幸長安,以組為河南尹。遷尚書,轉衛尉,賜爵成陽縣男,加散騎常侍、中書監。轉司隸校尉,加特進、光祿大夫,常侍如故。于時天下已亂,組兄弟貴盛,懼不容於世,雖居大官,並諷議而已。



 永嘉末,復以組為侍中,領太子太保。未拜,會劉曜、王彌逼洛陽,組與籓俱出奔。懷帝蒙塵,司空王浚以組為司隸校尉。組與籓移檄天下,以瑯邪王為盟主。



 愍帝稱皇太子,組即太子之舅,又領司隸校尉,行豫州刺史事,與籓並保滎陽之開封。建興初,詔籓行留臺事。俄而籓薨,帝更以組為司
 空,領尚書左僕射,又兼司隸,復行留臺事,州征郡守皆承制行焉。進封臨潁縣公,加太夫人、世子印綬。明年,進位太尉,領豫州牧、假節。



 元帝承制,以組都督司州諸軍,加散騎常侍,餘如故。頃之,又除尚書令,表讓不拜。及西都不守,組乃遣使移檄天下共勸進。帝欲以組為司徒,以問太常賀循。循曰:「組舊望清重,忠勤顯著,遷訓五品,實允眾望。」於是拜組為司徒。



 組逼於石勒,不能自立。太興初,自許昌率其屬數百人渡江,給千兵百騎,組先所領仍皆統攝。頃之,詔組與太保、西陽王羕並錄尚書事,各加班劍六十人。永昌初,遷太尉,領太子太保。未拜,薨,
 年六十五。謚曰元。子奕嗣。



 奕字玄欣。少拜太子舍人、駙馬都尉,侍講東宮。出為鎮東參軍,行揚武將軍、新汲令。愍帝為皇太子,召為中舍人,尋拜散騎侍郎,皆不就。隨父渡江。元帝踐阼,拜中庶子,遷給事黃門郎。父憂去職,服闋,補散騎常侍、侍中。



 時將繕宮城,尚書符下陳留王,使出城夫。奕駁曰:「昔虞賓在位,《書》稱其美;《詩》詠《有客》,載在《雅》《頌》。今陳留王位在三公之上,坐在太子右,故答表曰書,賜物曰與。此古今之所崇,體國之高義也。謂宜除夫役。」時尚書張闓、僕射孔愉難奕,以為:「昔宋不城周,《陽秋》所譏。特蠲非體,宜應
 減夫。」奕重駁,以為:「《陽秋》之末,文武之道將墜于地,新有子朝之亂,于時諸侯逋替,莫肯率職。宋之于周,實有列國之權。且同巳勤王而主之者晉,客而辭役,責之可也。今之陳留,無列國之勢,此之作否,何益有無!臣以為宜除,於國職為全。」詔從之。



 時又通議元會日帝應敬司徒王導不。博士郭熙、杜援等以為禮無拜臣之文,謂宜除敬。侍中馮懷議曰:「天子修禮,莫盛於辟雍。當爾之日,猶拜三老,況今先帝師傅。謂宜盡敬。」事下門下,奕議曰:「三朝之首,宜明君臣之體,則不應敬。若他日小會,自可盡禮。又至尊與公書手詔則曰『頓首言』,中書為詔則云『敬
 問』,散騎優冊則曰:『制命』。今詔文尚異,況大會之與小會,理豈得同!」詔從之。



 咸和七年卒,追贈太僕,謚曰定。



 馮紞,字少胄,安平人也。祖浮,魏司隸校尉。父員,汲郡太守。紞少博涉經史,識悟機辯。歷仕為魏郡太守,轉步兵校尉,徙越騎。得幸於武帝,稍遷左衛將軍。承顏悅色,寵愛日隆。賈充、荀勖並與之親善。充女之為皇太子妃也,紞有力焉。及妃之將廢,紞、勖乾沒救請,故得不廢。伐吳之役,紞領汝南太守,以郡兵隨王浚入秣陵。遷御史中丞,轉侍中。



 帝病篤得愈,紞與勖見朝野之望,屬在齊王
 攸。攸素薄勖。勖以太子愚劣,恐攸得立,有害於己,乃使紞言於帝曰:「陛下前者疾若不差,太子其廢矣。齊王為百姓所歸,公卿所仰,雖欲高讓,其得免乎!宜遣還籓,以安社稷。」帝納之。及攸薨,朝野悲恨。初,帝友于之情甚篤,既納紞、勖邪說,遂為身後之慮,以固儲位。既聞攸殞,哀慟特深。紞侍立,因言曰:「齊王名過於實,今得自終,此乃大晉之福。陛下何乃過哀!」帝收淚而止。



 初謀伐吳,紞與賈充、荀勖同共苦諫不可。吳平,紞內懷慚懼,疾張華如讎。及華外鎮,威德大著,朝論當徵為尚書令。紞從容侍帝,論晉魏故事,因諷帝,言華不可授以重任,帝默然而
 止。事具《華傳》。



 太康七年,紞疾,詔以紞為散騎常侍,賜錢二十萬、床帳一具。尋卒。二子:播、熊。播,大長秋。熊字文羆,中書郎。紞兄恢,自有傳。


史臣曰:夫立身之道,曰仁與義。動靜既形,悔吝斯及。有莘之媵,殊《北門》之情;渭濱之叟,匪西山之節。湯武有以濟其功,夏殷不能譏其志。王沈才經文武,早尸人爵,在魏參席上之珍,居晉為幄中之士,桐宮之謀遽泄,武闈之禍遂臻。是知田光之口,豈燕丹之可絕;豫讓之形,非智氏之能變。動靜之際,有據蒺藜,仁義之方,求之彌遠矣。彭祖謁由捧雉,孕本貿絲,因家乏主,遂登顯秩。擁北
 州之士馬,偶東京之糜沸,自可感召諸侯,宣力王室。而乘間伺隙,潛圖不軌,放肆獯虜,遷播乘輿。遂使漳滏蕭然,黎元塗地。縱貪夫於藏戶,戮高士於燕垂,阻越石之內難,邀世龍之外府。惡稔毒
 \gezhu{
  疒甫}
 ,坐致焚燎,假手仇敵,方申凶獷,慶封之戮,慢罵何補哉!公曾,慈明之孫;景倩,文若之子,踐隆堂而高視,齊逸軌而長騖。孝敬足以承親,周慎足以事主,刊姬公之舊典,採蕭相之遺法。然而援朱均以貳極,煽褒閻而偶震。雖廢興有在,隆替靡常,稽之人事,乃二荀之力也。至於斗粟興謠,踰里成詠,勖之階禍,又已甚焉。馮紞外騁戚施,內窮狙詐,斃攸安賈,交
 勖仇張,心滔楚費,過逾晉伍。爰絲獻壽,空取慰於仁心,紞之陳說,幸收哀於迷慮,投畀之罰無聞,《青蠅》之詩不作矣。



 贊曰:處道文林,胡貳爾心?彭祖兇孽,自貽伊戚。臨淮翼翼,孝形於色。安陽英英,匪懈其職。傾齊附魯,是為蝥賊。紞之不臧,交亂罔極。



\end{pinyinscope}