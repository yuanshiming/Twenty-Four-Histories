\article{帝紀第二}

\begin{pinyinscope}

 景帝
 文
 帝



 景皇帝諱師,字子元,宣帝長子也。雅有風彩,沈毅多大略。少流美譽,與夏侯玄、何晏齊名。晏常稱曰:「惟幾也能成天下之務,司馬子元是也。」魏景初中,拜散騎常侍,累遷中護軍。為選用之法,舉不越功,吏無私焉。宣穆皇后崩,居喪以至孝聞。宣帝之將誅曹爽,深謀祕策,獨與帝潛畫,文帝弗之知也。將發夕乃告之,既而使人覘之,帝
 寢如常,而文帝不能安席。晨會兵司馬門,鎮靜內外,置陣甚整。宣帝曰:「此子竟可也。」初,帝陰養死士三千,散在人間,至是一朝而集,眾莫知所出也。事平,以功封長平鄉侯,食邑千戶,尋加衛將軍。及宣帝薨,議者咸云「伊尹既卒,伊陟嗣事」,天子命帝以撫軍大將軍輔政。魏嘉平四年春正月,遷大將軍,加侍中,持節、都督中外諸軍、錄尚書事。命百官舉賢才,明少長,恤窮獨,理廢滯。諸葛誕、毌丘儉、王昶、陳泰、胡遵都督四方,王基、州泰、鄧艾、石苞典州郡,盧毓、李豐裳選舉,傅嘏、虞松參計謀,鐘會、夏侯玄、王肅、陳本、孟康、趙酆、張緝預朝議,四海傾注,
 朝野肅然。或有請改易制度者,帝曰:「『不識不知,順帝之則』,詩人之美也。三祖典制,所宜遵奉;自非軍事,不得妄有改革。」



 五年夏五月,吳太傅諸葛恪圍新城,朝議慮其分兵以寇淮泗,欲戍諸水口。帝曰:「諸葛恪新得政於吳,欲徼一時之利,并兵合肥,以冀萬一,不暇復為青徐患也。且水口非一,多戍則用兵眾,少戍則不足以禦寇。」恪果并力合肥,卒如所度。帝於是使鎮東將軍毌丘儉、揚州刺史文欽等距之。儉、欽請戰,帝曰:「恪卷甲深入,投兵死地,其鋒未易當。且新城小而固,攻之未可拔。」遂命諸將高壘
 以弊之。相持數月,恪攻城力屈,死傷太半。帝乃敕欽督銳卒趨合榆,要其歸路,儉帥諸將以為後繼。恪懼而遁,欽逆擊,大破之,斬首萬餘級。



 正元元年春正月,天子與中書令李豐、后父光祿大夫張緝、黃門監蘇鑠、永寧署令樂敦、冗從僕射劉寶賢等謀以太常夏侯玄代帝輔政。帝密知之,使舍人王羨以車迎豐。豐見迫,隨羨而至,帝數之。豐知禍及,因肆惡言。帝怒,遣勇士以刀鐶築殺之。逮捕玄、緝等,皆夷三族。三月,乃諷天子廢皇后張氏,因下詔曰:「姦臣李豐等靖譖庸回,陰構凶慝。大將軍糾虔天刑,致之誅辟。周勃之克
 呂氏,霍光之擒上官,曷以過之。其增邑九千戶,并前四萬。」帝讓不受。天子以玄、緝之誅,深不自安。而帝亦慮難作,潛謀廢立,乃密諷魏永寧太后。秋九月甲戌,太后下令曰:「皇帝春秋已長,不親萬機,耽淫內寵,沈嫚女德,日近倡優,縱其醜虐,迎六宮家人留止內房,毀人倫之敘,亂男女之節。又為群小所迫,將危社稷,不可承奉宗廟。」帝召群臣會議,流涕曰:「太后令如是,諸君其如王室何?」咸曰:「伊尹放太甲以寧殷,霍光廢昌邑以安漢,權定社稷,以清四海。二代行之於古,明公當之於今,今日之事,惟命是從。」帝曰:「諸君見望者重,安敢避之?」乃與群公卿
 士共奏太后曰:「臣聞天子者,所以濟育群生,永安萬國。皇帝春秋已長,未親萬機,日使小優郭懷、袁信等裸袒淫戲。又於廣望觀下作遼東妖婦,道路行人莫不掩目。清商令令狐景諫帝,帝燒鐵炙之。太后遭合陽君喪,帝嬉樂自若。清商丞龐熙諫帝,帝弗聽。太后還北宮,殺張美人,帝甚恚望。熙諫,帝怒,復以彈彈熙。每文書入,帝不省視。太后令帝在式乾殿講學,帝又不從。不可以承天序。臣請依漢霍光故事,收皇帝璽綬,以齊王歸籓。」奏可,於是有司以太牢策告宗廟,王就乘輿副車,群臣從至西掖門。帝泣曰:「先臣受歷世殊遇,先帝臨崩,託以遣
 詔。臣復忝重任,不能獻可替否。群公卿士,遠翟舊典,為社稷深計,寧負聖躬,使宗廟血食。」於是使使者持節衛送,舍河內之重門,誅郭懷、袁信等。是日,與群臣議所立。帝曰:「方今宇宙未清,二虜爭衡,四海之主,惟在賢哲。彭城王據,太祖之子,以賢,則仁聖明允;以年,則皇室之長。天位至重,不得其才,不足以寧濟六合。」乃興群公奏太后。太后以彭城王先帝諸父,於昭穆之序為不次,則烈祖之世永無承嗣。東海定王,明帝之弟,欲立其子高貴鄉公髦。帝固爭不獲,乃從太后令,遣使迎高貴鄉公於元城而立之,改元曰正元。天子受璽惰,舉趾高,帝聞而
 憂之。及將大會,帝訓於天了曰:「夫聖王重始,正本敬初,古人所慎也。明當大會,萬眾瞻穆穆之容,公卿聽玉振之音。詩云:『示人不佻,是則是效。』易曰:『出其言善,則千里之外應之』。雖禮儀周備,猶宜加之以祗恪,以副四海顒顒式仰。」癸巳,天子詔曰:「朕聞創業之君,必須股肱之臣;守文之主,亦賴匡佐之輔。是故文武以呂召彰受命之功,宣王倚山甫享中興之業。大將軍世載明德,應期作輔。遭天降險,帝室多難,齊王蒞政,不迪率典。公履義執忠,以寧區夏,式是百辟,總齊庶事。內摧寇虐,外靜姦宄,日昃憂勤,劬勞夙夜。德聲光于上下,勳烈施于四方。深
 惟大議,首建明策,權定社稷,援立朕躬,宗廟獲安,億兆慶賴。伊摯之保乂殷邦,公旦之綏寧周室,蔑以尚焉。朕甚嘉之。夫德茂者位尊,庸大者祿厚,古今之通義也。其登位相國,增邑九千,并前四萬戶;進號大都督、假黃鉞,入朝不趨,奏事不名,劍履上殿;賜錢五百萬,帛五千匹,以彰元勳。」帝固辭相國。又上書訓于天子曰:「荊山之璞雖美,不琢不成其寶;顏冉之才雖茂,不學不弘其量。仲尼有云:『予非生而知之者,好古敏以求之者也。』仰觀黃軒五代之主,莫不有所稟則,顓頊受學於綠圖,高辛問道於柏招。逮至周成,旦望作輔,故能離經辯志,安道樂
 業。夫然,故君道明於上,兆庶順於下。刑措之隆,實由於此。宜遵先王下問之義,使講誦之業屢聞於聽,典謨之言日陳於側也。」時天子頗修華飾,帝又諫曰:「履端初政,宜崇玄樸。」並敬納焉。十一月,有白氣經天。



 二年春正月,有彗星見于吳楚之分,西北竟天。鎮東大將軍毌丘儉、揚州刺史文欽舉兵作亂,矯太后令移檄郡國,為壇盟于西門之外,各遣子四人質于吳以請救。二月,儉、欽帥眾六萬,渡淮而西。帝會公卿謀征討計,朝議多謂可遣諸將擊之,王肅及尚書傅嘏、中書侍郎鐘會勸帝自行。戊午,帝統中軍步騎十餘萬以征之。倍道
 兼行,召三方兵,大會于陳許之郊。甲申,次于隱橋,儉將史招、李績相次來降。儉、欽移入項城,帝遣荊州刺史王基進據南頓以逼儉。帝深壁高壘,以待東軍之集。諸將請進軍攻其城,帝曰:「諸君得其一,未知其二。淮南將士本無反志。且儉、欽欲蹈縱橫之迹,習儀秦之說,謂遠近必應。而事起之日,淮北不從,史招、李績前後瓦解。內乖外叛,自知必敗,困獸思鬥,速戰更合其志。雖云必克,傷人亦多。且儉等欺誑將士,詭變萬端,小與持久,詐情自露,此不戰而克之也。」乃遣諸葛誕督豫州諸軍自安風向壽春,征東將軍胡遵督青、徐諸軍出譙宋之間,絕其
 歸路。帝屯汝陽,遣競州刺史鄧艾督太山諸軍進屯樂嘉,示弱以誘之。欽進軍將攻艾,帝潛軍銜枚,輕造樂嘉,與欽相遇。欽子鴦,年十八,勇冠三軍,謂欽曰:「及其未定,請登城鼓噪,擊之可破也。」既謀而行,三噪而欽不能應,鴦退,相與引而東。帝謂諸將曰:「欽走矣。」命發銳軍以追之。諸將皆曰:「欽舊將,鴦少而銳,引軍內入,未有失利,必不走也。」帝曰:「一鼓作氣,再而衰,三而竭。鴦三鼓,欽不應,其勢已屈,不走何待?」欽將遁,鴦曰:「不先折其勢,不得去也。」乃與驍騎十餘摧鋒陷陣,所向皆披靡,遂引去。帝遣左長史司馬璉督驍騎八千翼而追之,使將軍樂林等
 督步兵繼其後。比至沙陽,頻陷欽陣,弩矢雨下,欽蒙盾而馳。大破其軍。眾皆投戈而降,欽父子與麾下走保項。儉聞欽敗,棄眾宵遁淮南。安風津都尉追儉,斬之,傳首京都。欽遂奔吳,淮南平。



 初,帝目有瘤疾,使醫割之。鴦之來攻也,驚而目出。懼六軍之恐,蒙之以被,痛甚,齧被敗而左右莫知焉。閏月疾篤,使文帝總統諸軍。辛亥,崩于許昌,時年四十八。二月,帝之喪至自許昌,天子素服臨弔,詔曰:「公有濟世寧國之勳,剋定禍亂之功,重之以死王事,宜加殊禮。其令公卿議制。」有司議以為忠安社稷,功濟宇內,宜依霍光故事,追加大司馬之號以冠大
 將軍,增邑五萬戶,謚曰武公。文帝表讓曰:「臣亡父不敢受丞相相國九命之禮,亡兄不敢受相國之位,誠以太祖常所階歷也。今謚與二祖同,必所祗懼。昔蕭何、張良、霍光咸有匡佐之功,何謚文終,良謚文成,光謚宣成。。必以文武為謚,請依何等就加。」詔許之,謚曰忠武。晉國既建,追尊曰景王。武帝受禪,上尊號曰景皇帝,陵曰峻平,廟稱世宗。



 文皇帝諱昭,字子上,景帝之母弟也。魏景初二年,封新城鄉侯。正始初,為洛陽典農中郎將。值魏明奢侈之後,
 帝蠲除苛碎,不奪農時,百姓大悅。轉散騎常侍。大將軍曹爽之伐蜀也,以帝為征蜀將軍,副夏侯玄出駱谷,次於興勢。蜀將王林夜襲帝營,帝堅臥不動。林退,帝謂玄曰:「費禕以據險距守,進不獲戰,攻之不可,宜亟旋軍,以為後圖。」爽等引旋,禕果馳兵趣三嶺,爭險乃得過。遂還,拜議郎。及誅曹爽,帥眾衛二宮,以功增邑千戶。蜀將姜維之寇隴右也,征西將軍郭淮自長安距之。進帝位安西將軍、持節,屯關中,為諸軍節度。淮攻維別將句安於麴,久而不決。帝乃進據長城,南趣駱谷以疑之。維懼,退保南鄭,安軍絕援,帥眾來降。轉安東將軍、持節,鎮許昌。
 及大軍討王凌,帝督淮北諸軍事,帥師會于項。增邑三百戶,假金印紫綬。尋進號都督,統征東將軍胡遵、鎮東將軍諸葛誕伐吳,戰于東關。二軍敗績,坐失侯。蜀將姜維又寇隴右,揚聲欲攻狄道。以帝行征西將軍,次長安。雍州刺史陳泰欲先賊據狄道,帝曰:「姜維攻羌,收其質任,聚穀作邸閣訖,而復轉行至此,正欲了塞外諸羌,為後年之資耳。若實向狄道,安肯宣露,令外人知?今揚聲言出,此欲歸也。」維果燒營而去。會新平羌胡叛,帝擊破之,遂耀兵靈州,北虜震讋,叛者悉降。以功復封新城鄉侯。高貴鄉公之立也,以參定策,進封高都侯,增封二千
 戶。毌丘儉、文欽之亂,大軍東征,帝兼中領軍,留鎮洛陽。及景帝疾篤,帝自京都省疾,拜衛將軍。景帝崩,天子命帝鎮許昌,尚書傅嘏帥六軍還京師。帝用嘏及鐘會策,自帥軍而還。至洛陽,進位大將軍加侍中,都督中外諸軍、錄尚書事,輔政,劍履上殿。帝固辭不受。



 甘露元年春正月,加大都督,奏事不名。夏六月,進封高都公,地方七百里,加之九錫,假斧鉞,進號大都督,劍履上殿。又固辭不受。秋八月庚申,加假黃鉞,增封三縣。



 二年夏五月辛未,鎮東大將軍諸葛誕殺揚州刺史樂綝,以淮南作亂,遣子靚為質於吳以請救。議者請速伐
 之,帝曰:「誕以毌丘儉輕疾傾覆,今必外連吳寇,此為變大而遲。吾當與四方同力,以全勝制之。」乃表曰:「昔黥布叛逆,漢祖親征;隗囂違戾,光武西伐;烈祖明皇帝乘輿仍出:皆所以奮揚赫斯,震耀威武也。陛下宜暫臨戎,使將士得憑天威。今諸軍可五十萬,以眾擊寡,蔑不剋矣。」秋七月,奉天子及皇太后東征,徵兵青、徐、荊、豫,分取關中遊軍,皆會淮北。師次于項,假廷尉何楨節,使淮南,宣慰將士,申明逆順,示以誅賞。甲戌,帝進軍丘頭。吳使文欽、唐咨、全端、全懌等三萬餘人來救誕,諸將逆擊,不能禦。將軍李廣臨敵不進,泰山太守常時稱疾不出,並斬
 之以徇。八月,吳將朱異帥兵萬餘人,留輜重於都陸,輕兵至黎漿。監軍石苞、袞州刺史州泰禦之,異退。泰山太守胡烈以奇兵襲都陸,焚其糧運。苞、泰復進擊異,大破之。異之餘卒餒甚,食葛葉而遁,吳人殺異。帝曰:「異不得至壽春,非其罪也,而吳人殺之,適以謝壽春而堅誕意,使其猶望救耳。若其不爾,彼當突圍,決一旦之命。或謂大軍不能久,省食減口,冀有他變。料賊之情,不出此三者。今當多方以亂之,備其越逸,此勝計也。」因命合圍,分遣羸疾就穀淮北,稟軍士大豆,人三升。欽聞之,果喜。帝愈羸形以示之,多縱反間,揚言吳救方至。誕等益寬恣
 食,俄而城中乏糧。石苞、王基並請攻之,帝曰:「誕之逆謀,非一朝一夕也,聚糧完守,外結吳人,自謂足據淮南。欽既同惡相濟,必不便走。今若急攻之,損游軍之力。外寇卒至,表裏受敵,此危道也。今三叛相聚於孤城之中,天其或者將使同戮。吾當以長策縻之,但堅守三面。若賊陸道而來,軍糧必少,吾以游兵輕騎絕其轉輸,可不戰而破外賊。外賊破,欽等必成擒矣。」全懌母,孫權女也,得罪於吳,全端兄子禕及儀奉其母來奔。儀兄靜時在壽春,用鐘會計,作禕、儀書以譎靜。靜兄弟五人帥其眾來降,城中大駭。



 三年春正月壬寅,誕、欽等出攻長圍,諸軍逆擊,走之。初,誕、欽內不相協,及至窮蹙,轉相疑貳。會欽計事與誕忤,誕手刃殺欽。欽子鴦攻誕,不克,踰城降。以為將軍,封侯,使鴦巡城而呼。帝見城上持弓者不發,謂諸將曰:「可攻矣!」二月乙酉,攻而拔之,斬誕,夷三族。吳將唐咨、孫彌、徐韶等帥其屬皆降,表加爵位,稟其餒疾。或言吳兵必不為用,請坑之。帝曰:「就令亡還,適見中國之弘耳。」於是徙之三河。夏四月,歸于京師,魏帝命改丘頭曰武丘,以旌武功。五月,天子以并州之太原上黨西河樂平新興鴈門、司州之河東平陽八郡,地方七百里,封帝為晉
 公,加九錫,進位相國,晉國置官司焉。九讓,乃止。於是增邑萬戶,食三縣,諸子之無爵者皆封列侯。秋七月,奏錄先世名臣元功大勳之子了,隨才敘用。



 四年夏六月,分荊州置二都督,王基鎮新野,州泰鎮襄陽。使石苞都督揚州,陳騫都督豫州,鐘毓都督徐州,宋鈞監青州諸軍事。



 景元元年夏四月,天子復命帝爵秩如前,又讓不受。天子既以帝三世宰輔,政非己出,情不能安,又慮廢辱,將臨軒召百僚而行放黜。五月戊子夜,使冗從僕射李昭等發甲於陵雲臺,召侍中王沈、散騎常侍王業、尚書王
 經,出懷中黃素詔示之,戒嚴俟旦。沈、業馳告于帝,帝召護軍賈充等為之備。天子知事泄,帥左右攻相府,稱有所討,敢有動者族誅。相府兵將止不敢戰,賈充叱諸將曰:「公畜養汝輩,正為今日耳!」太子舍人成濟抽戈犯蹕,刺之,刃出於背,天子崩于車中。帝召百僚謀其故,僕射陳泰不至。帝遣其舅荀顗輿致之,延於曲室,謂曰:「玄伯,天下其如我何?」泰曰:「惟腰斬賈充,微以謝天下。」帝曰:「卿更思其次。」泰曰:「但見其上。不見其次。」於是歸罪成濟而斬之。太后令曰:「昔漢昌邑王以罪發為庶人,此兒亦宜以庶人禮葬之,使外內咸知其所行也。」殺尚書王經,貳
 於我也。庚寅,帝奏曰:「故高貴鄉公帥從駕人兵,拔刃鳴鼓向臣所,臣懼兵刃相接,即敕將士不得有所傷害,違令者以軍法從事。騎督成倅弟太子舍人濟入兵陣,傷公至隕。臣聞人臣之節,有死無貳,事上之義,不敢逃難。前者變故卒至,禍同發機,誠欲委身守死,惟命所裁。然惟本謀,乃欲上危皇太后,傾覆宗廟。臣忝當元輔,義在安國,即駱驛申敕,不得迫近輿輦。而濟妄入陣間,以致大變,哀怛痛恨,五內摧裂。濟干國亂紀,罪不容誅,輒收濟家屬,付廷尉。」太后從之,夷濟三族。與公卿議,立燕王宇之子常道鄉公璜為帝。六月,改元。丙辰,天子進帝為
 相國,封晉公,增十郡,加九錫如初,群從子弟未侯者封亭侯,賜錢千萬,帛萬匹。固讓,乃止。冬十一月,吳吉陽督蕭慎以書詣鎮東將軍石苞偽降,求迎。帝知其詐也,使苞外示迎之,而內為之備。



 二年秋八月甲寅,天子使太尉高柔授帝相國印綬,司空鄭沖致晉公茅土九錫,固辭。



 三年夏四月,肅慎來獻楛矢、石砮、弓甲、貂皮等,天子命歸於大將軍府。



 四年春二月丁丑,天子復命帝如前,又固讓。三月,詔大將軍府增置司馬一人,從事中郎二人,舍人十人。夏,帝
 將伐蜀,乃謀眾曰:「自定壽春已來,息役六年,治兵繕甲,以擬二虜。略計取吳,作戰船,通水道,當用千餘萬功,此十萬人百數十日事也。又南土下溼,必生疾疫。今宜先取蜀,三年之後,在巴蜀順流之勢,水陸並進,此滅虞定虢,吞韓並魏之勢也。計蜀戰士九萬,居守成都及備他郡不下四萬,然則餘眾不過五萬。今絆姜維於沓中,使不得東顧,直指駱谷,出其空虛之地,以襲漢中。彼若嬰城守險,兵勢必散,首尾離絕。舉大眾以屠城,散銳卒以略野,劍閣不暇守險,關頭不能自存。以劉禪之闇,而邊城外破,士女內震,其亡可知也。」征西將軍鄧艾以為未
 有釁,屢陳異議。帝患之,使主簿師纂為艾司馬以喻之,艾乃奉命。於是征四方之兵十八萬,使鄧艾自狄道攻姜維於沓中,雍州刺史諸葛緒自祁山軍于武街,絕維歸路,鎮西將軍鐘會帥前將軍李輔、征蜀護軍胡烈等自駱谷襲漢中。秋八月,軍發洛陽,大賚將士,陳師誓眾。將軍鄧敦謂蜀未可討,帝斬以徇。九月,又使天水太守王頎攻維營,隴西太守牽弘邀其前,金城太守楊頎趣甘松。鐘會分為二隊,入自斜谷,使李輔圍王含於樂城,又使步將易愷攻蔣斌於漢城。會直指陽安,護軍胡烈攻陷關城。姜維聞之,引還,王頎追敗維於彊川。維與張
 翼、廖化合軍守劍閣,鐘會攻之。冬十月,天子以諸侯獻捷交至,乃申前命曰:



 朕以寡德,獲承天序,嗣我祖宗之洪烈。遭家多難,不明于訓。曩者奸逆屢興,方寇內侮,大懼淪喪四海,以墮三祖之弘業。惟公經德履哲,明允廣深,迪宣武文,世作保傅,以輔乂皇家。櫛風沐雨,周旋征伐,劬勞王室,二十有餘載。毗翼前人,乃斷大政,克厭不端,維安社稷。暨儉、欽之亂,公綏援有眾,分命興師,統紀有方,用緝寧淮浦。其後巴蜀屢侵,西土不靖,公奇畫指授,制勝千里。是以段谷之戰,乘釁大捷,斬將搴旗,效首萬計。孫峻猾夏,致寇徐方,戎車首路,威靈先邁,黃鉞未
 啟,鯨鯢竄迹。孫壹構隙,自相疑阻,幽鑒遠照,奇策洞微,遠人歸命,作籓南夏,爰授銳卒,畢力戎行。暨諸葛誕,滔天作逆,稱兵揚楚,欽、咨逋罪,同惡相濟,帥其蝥賊,以入壽春,憑阻淮山,敢距王命。公躬擐甲胄,龔行在罰,玄謀廟算,遵養時晦。奇兵震擊,而朱異摧破;神變應機,而全琮稽服;取亂攻昧,而高墉不守。兼九伐之弘略,究五兵之正度,用能戰不窮武,而大敵殲潰;旗不再麾,而元憝授首。收勍吳之雋臣,係亡命之逋虜。交臂屈膝,委命下吏,俘馘十萬積尸成京。雪宗廟之滯恥,拯兆庶之艱難。掃平區域,信威吳會,遂戢干戈,靖我疆土,天地鬼神,罔
 不獲乂。乃者王室之難,變起蕭墻,賴公之靈,弘濟艱險。宗廟危而獲安,社稷墜而復寧。忠格皇天,功濟六合。是用疇咨古訓,稽諸典籍,命公崇位相國,加於群后,啟土參墟,封以晉域。所以方軌齊魯,翰屏帝室。而公遠蹈謙損,深履沖讓,固辭策命,至於八九。朕重違讓德,抑禮虧制,以彰公志,於今四載。上闕在昔建侯之典,下違兆庶具瞻之望。



 惟公嚴虔王度,闡濟大猷,敦尚純樸,省繇節用,務穡勸分,九野康乂。耆叟荷崇養之德,鰥寡蒙矜恤之施,仁風興於中夏,流澤布於遐荒。是以東夷西戎,南蠻北狄,狂狡貪悍,世為寇讎者,皆感義懷惠,款塞內附,
 或委命納貢,或求置官司。九服之外,絕域之氓,曠世所希至者,咸浮海來享,鼓舞王德,前後至者八百七十餘萬口。海隅幽裔,無思不服;雖西旅遠貢,越裳九譯,義無以踰。維翼朕躬,下匡萬國,思靖殊方,寧濟八極。以庸蜀未賓,蠻荊作猾,潛謀獨斷,整軍經武。簡練將帥,授以成策,始踐賊境,應時摧陷。狂狡奔北,首尾震潰,禽其戎帥,屠其城邑。巴漢震疊,江源雲徹,地平天成,誠在斯舉。公有濟六合之勳,加以茂德,實總百揆,允釐庶政。敦五品以崇仁,恢六典以敷訓。而靖恭夙夜,勞謙昧旦,雖尚父之左右文武,周公之勤勞王家,罔以加焉。



 昔先王選建
 明德,光啟諸侯,體國經野,方制五等。所以籓翼王畿,垂祚百世也。故齊魯之封,於周為弘,山川土田,邦畿七百,官司典策,制殊群后。惠襄之難,桓文以翼戴之勞,猶受錫命之禮,咸用光疇大德,作範于後。惟公功邁於前烈,而賞闕於舊式,百辟於邑,人神同恨焉,豈可以公謙沖而久淹弘典哉?今以并州之太原上黨西河樂平新興雁門、司州之河東平陽弘農、雍州之馮翊凡十郡,南至於華,北至于陘,東至于壺口,西踰于河,提封之數,方七百里,皆晉之故壤,唐叔受之,世作盟主,實紀綱諸夏,用率舊職。爰胙茲土,封公為晉公。命使持節、兼司徒、司隸
 校尉陔即授印綬策書,金獸符第一至第五,竹使符第一至第十。錫茲玄土,苴以白茅,建爾國家,以永籓魏室。



 昔在周召,並以公侯,入作保傅。其在近代,酂侯蕭何,實以相國,光尹漢朝。隨時之制,禮亦宜之。今進公位為相國,加綠綟綬。又加公九錫,其敬聽後命。以公思弘大猷,崇正典禮,儀刑作範,旁訓四方,是用錫公大輅、戎輅各一,玄牡二駟。公道和陰陽,敬授人時,嗇夫反本,農殖維豐,是用錫公袞冕之服,赤舄副焉。公光敷顯德,惠下以和,敬信思順,庶尹允諧,是用錫公軒懸之樂、六佾之舞。公鎮靖宇宙,翼播聲教,海外懷服,荒裔款附,殊方馳義,
 諸夏順軌,是用錫公朱戶以居。公簡賢料材,營求俊逸,爰升多士,置彼周行,是用錫公納陛以登。公嚴恭寅畏,底平四國,式遏寇虐,苛厲不作,是用錫公武賁之士三百人。公明慎用刑,簡恤大中,章厥天威,以糾不虔,是用錫公鈇鉞各一。公爰整六軍,典司征伐,犯命凌正,乃維誅殛,是用錫公彤弓一、彤矢百,JF弓十、JF矢千。公饗祀蒸蒸,孝思維則,篤誠之至,通于神明,是用錫公秬鬯一卣,珪瓚副焉。晉國置官司以下,率由舊式。



 往欽哉!祗服朕命,弘敷訓典,光澤庶方,永終爾明德,丕顯餘一人之休命。



 公卿將校皆詣府喻旨,帝以禮辭讓。司空鄭沖率
 群官勸進曰:「伏見嘉命顯至,竊聞明公固讓,沖等眷眷,實有愚心。以為聖王作制,百代同風,褒德賞功,有自來矣。昔伊尹,有莘氏之媵臣耳,一佐成湯,遂荷阿衡之號。周公藉已成之勢,據既安之業,光宅曲阜,奄有龜蒙。呂尚,磻溪之漁者也,一朝指麾,乃封營丘。自是以來,功薄而賞厚者,不可勝數,然賢哲之士,猶以為美談。況自先相國以來,世有明德,翼輔魏室,以綏天下,朝無秕政,人無謗言。前者明公西征靈州,北臨沙漠,榆中以西,望風震服,羌戎來馳,回首內向,東誅叛逆,全軍獨克。禽闔閭之將,虜輕銳之卒以萬萬計,威加南海,名懾三越,宇內
 康寧,苛慝不作。是以時俗畏懷,東夷獻舞。故聖上覽乃昔以來禮典舊章,開國光宅,顯茲太原。明公宜承奉聖旨,受茲介福,允當天人。元功盛勛,光光如彼;國土嘉祚,巍巍如此。內外協同,靡愆靡違。由斯征伐,則可朝服濟江,掃除吳會,西塞江源,望祀岷山。迴戈弭節,以麾天下,遠無不服,邇無不肅。令大魏之德,光于唐虞;明公盛勳,超於桓文。然後臨滄海而謝支伯,登箕山而揖許由,豈不盛乎!至公至平,誰與為鄰,何必勤勤小讓也哉。」帝乃受命。十一月,鄧艾帥萬餘人自陰平踰絕險至江由,破蜀將諸葛瞻於綿竹,斬瞻,傳首。進軍雒縣,劉禪降。天子
 命晉公以相國總百揆,於是上節傳,去侍中、大都督、錄尚書之號焉。表鄧艾為太尉,鐘會為司徒。會潛謀叛逆,因密使譖艾。



 咸熙元年春正月,檻車徵艾。乙丑,帝奉天子西征,次於長安。是時魏諸王侯悉在鄴城,命從事中郎山濤行軍司事,鎮於鄴,遣護軍賈充持節、督諸軍,據漢中。鐘會遂反於蜀,監軍衛瓘、右將軍胡烈攻會,斬之。初,會之伐蜀也,西曹屬邵悌言於帝曰:「鐘會難信,不可令行。」帝笑曰:「取蜀如指掌,而眾人皆言不可,唯會與吾意同。滅蜀之後,中國將士,人自思歸,蜀之遺黎,猶懷震恐,縱有異志,
 無能為也。」卒如所量。丙辰,帝至自長安。三月己卯,進帝爵為王,增封并前二十郡。夏五月癸未,天子追加舞陽宣文侯為晉宣王,舞陽忠武侯為晉景王。秋七月,帝奏司空荀顗定禮儀,中護軍賈充正法律,尚書僕射裴秀議官制,太保鄭沖總而裁焉。始建五等爵。冬十月丁亥,奏遣吳人相國參軍徐劭、散騎常侍水曹屬孫彧使吳,喻孫皓以平蜀之事,致馬錦等物,以示威懷。丙午,天子命中撫軍新昌鄉侯炎為晉世子。



 二年春二月甲辰,朐縣獻靈龜,歸於相府。夏四月,孫皓使紀陟來聘,且獻方物。五月,天子命帝冕十有二旒,
 建天子旌旗,出警入蹕,乘金根車,駕六馬,備五時副車,置旄頭雲罕,樂舞八佾,設鐘虡宮懸,位在燕王上。進王妃為王后,世子為太子,王女王孫爵命之號皆如帝者之儀。諸禁網煩苛及法式不便於時者,帝皆奏除之。晉國置御史大夫、侍中、常侍、尚書、中領軍、衛將軍官。



 秋八月辛卯,帝崩于露寢,時年五十五。九月癸酉,葬崇陽陵,謚曰文王。武帝受禪,追尊號曰文皇帝,廟稱太祖。



 史臣曰:世宗以睿略創基,太祖以雄才成務。事殷之跡空存,翦商之志彌遠,三分天下,功業在焉。及踰劍銷氛,浮淮靜亂,桐宮胥怨,或所不堪。若乃體以名臣,格之端
 揆,周公流連於此歲,魏武得意於茲日。軒懸之樂,大啟南陽,師摯之圖,於焉北面。壯矣哉,包舉天人者也!為帝之主,不亦難乎。



 贊曰:世宗繼文,邦權未分。三千之士,其從如雲。世祖無外,靈關靜氛。反雖討賊,終為弒君。



\end{pinyinscope}