\article{列傳第四十一}

\begin{pinyinscope}

 孫惠熊遠王鑒陳頵高崧



 孫惠,字德施,吳國富陽人,吳豫章太守賁曾孫也。父祖並仕吳。惠口訥,好學有才識,州辟不就,寓居蕭沛之間。永寧初,赴齊王冏義,討趙王倫,以功封晉興縣侯,辟大司馬戶曹掾,轉東曹屬。冏驕矜僭侈,天下失望。惠獻言於冏,諷以五難、四不可,勸令歸籓,辭甚切至。冏不納。惠懼罪,辭疾去。頃之,冏果敗。成都王穎薦惠為大將軍參
 軍、領奮威將軍、白沙督。是時,穎將征長沙王乂,以陸機為前鋒都督。惠與機同鄉里,憂其致禍,勸機讓都督於王粹。及機兄弟被戮,惠甚傷恨之。時惠又擅殺穎牙門將梁俊,懼罪,因改姓名以遁。



 後東海王越舉兵下邳,惠乃詭稱南嶽逸士秦祕之,以書干越曰:



 天禍晉國,遘茲厄運。歷觀危亡,其萌有漸,枝葉先零,根株乃斃。伏惟明公資睿哲之才,應神武之略,承衰亂之餘,當傾險之運,側身昏讒之俗,局蹐凶諂之間。執夷正立,則取疾姦佞;抱忠懷直,則見害賊臣。餔糟非聖性所堪,茍免非英雄之節,是以感激於世,發憤忘身。抗辭金門,則謇諤之言
 顯;扶翼皇家,則匡主之功著。事雖未集,大命有在。夫以漢祖之賢,猶有彭城之恥;魏武之能,亦有濮陽之失。孟明三退,終於致果;勾踐喪眾,期於擒吳。今明公名著天下,聲振九域,公族歸美,萬國宗賢。加以四王齊聖,仁明篤友,急難之感,同獎王室,股肱爪牙,足相維持。皇穹無親,惟德是輔,惡盈福謙,鬼神所贊。以明公達存亡之符,察成敗之變,審所履之運,思天人之功,武視東夏之籓,龍躍海嵎之野。西諮河間,南結征鎮,東命勁吳銳卒之富,北有幽并率義之旅,宣喻青徐,啟示群王,旁收雄俊,廣延秀傑,糾合攜貳,明其賞信。仰惟天子蒙塵鄴宮,外
 矯詔命,擅誅無辜,豺狼篡噬,其事無遠。夫心火傾移,喪亂可必,太白橫流,兵家攸杖,歲鎮所去,天厭其德。玄象著明,謫譴彰見。違天不祥,奉時必剋。明公思安危人神之應,慮禍敗前後之徵,弘勞謙日昃之德,躬吐握求賢之義,傾府竭庫以振貧乏,將有濟世之才,渭濱之士,含奇謨於朱脣,握神策於玉掌,逍遙川嶽之上,以俟真人之求。目想不世之佐,耳聽非常之輔,舉而任之,則元勳建矣。



 祕之不天,值此衰運,竊慕墨翟、申包之誠,跋涉荊棘,重繭而至,櫛風沐雨,來承禍難。思以管穴毗佐大猷,道險時吝,未敢自顯。伏在川泥,繫情宸極,謹先白箋,以
 啟天慮。若猶沈吟際會,徘徊二端,徼倖在險,請從恕宥之例。



 明公今旋軫臣子之邦,宛轉名義之國,指麾則五嶽可傾,呼噏則江湖可竭。況履順討逆,執正伐邪,是烏獲摧冰,賁育拉朽,猛獸吞狐,泰山壓卵,因風燎原,未足方也。今時至運集,天與神助,復不能鵲起於慶命之會,拔劍於時哉之機,恐流濫之禍不在一人。自先帝公王,海內名士,近者死亡,皆如蟲獸,尸元曳於糞壞,形骸捐於溝澗,非其口無忠貞之辭,心無義正之節,皆希目下之小生而惑終焉之大死。凡人知友,猶有刎頸之報,朝廷之內,而無死命之臣。非獨秘之所恥,惜乎晉世之無
 人久矣。今天下喁喁,四海注目。社稷危而復安,宗廟替而復紹,惟明公兄弟能弘濟皇猷。國之存亡,在斯舉矣。



 祕之以下才之姿,而值危亂之運,竭其狗馬之節,加之忠貞之心,左屬平亂之鞬,右握滅逆之矢,控馬鵠立,計日俟命。時難獲而易失,機速變而成禍,介如石焉,實無終日,自求多福,惟君裁之!



 越省書,榜道以求之,惠乃出見。越即以為記室參軍,專職文疏,豫參謀議。除散騎郎、太子中庶子,復請補司空從事中郎。越誅周穆等,夜召參軍王廙造表,廙戰懼,壞數紙不成。時惠不在,越歎曰:「孫中郎在,表久就矣。」越遷太傅,以惠為軍諮祭酒,數諮
 訪得失。每造書檄,越或驛馬催之,應命立成,皆有文採。除秘書監,不拜。轉彭城內史、廣陵相,遷廣武將軍、安豐內史。以迎大駕之功,封臨湘縣公。



 元帝遣甘卓討周馥於壽陽,惠乃率眾應卓,馥敗走。廬江何銳為安豐太守,惠權留郡境。銳以他事收惠下人推之,惠既非南朝所授,常慮讒間,因此大懼,遂攻殺銳,奔入蠻中。尋病卒,時年四十七。喪還鄉里,朝廷明其本心,追加弔賻。



 熊遠,字孝文,豫章南昌人也。祖翹,嘗為石崇蒼頭,而性廉直,有士風。黃門郎潘岳見而稱異,勸崇免之,乃還鄉
 里。遠有志尚,縣召為功曹,不起,強與衣幘,扶之使謁。十餘日薦於郡,由是辟為文學掾。遠曰:「辭大不辭小也。」固請留縣。太守察遠孝廉。屬太守討氐羌,遠遂不行,送至隴右而還。後太守會稽夏靜辟為功曹。及靜去職,遠送至會稽以歸。州辟主簿、別駕,舉秀才,除監軍華軼司馬、領武昌太守、寧遠護軍。



 元帝作相,引為主簿。時傳北陵被發,帝將舉哀,遠上疏曰:「園陵既不親行,承傳言之者未可為定。且園陵非一,而直言侵犯,遠近弔問,答之宜當有主。謂應更遣使攝河南尹案行,得審問,然後可發哀。即宜命將至洛,修復園陵,討除逆類。昔宋殺無畏,莊
 王奮袂而起,衣冠相追於道,軍成宋城之下。況此酷辱之大恥,臣子奔馳之日!夫修園陵,至孝也;討逆叛,至順也;救社稷,至義也;恤遺黎,至仁也。若修此四道,則天下響應,無思不服矣。昔項羽殺義帝以為罪,漢祖哭之以為義,劉項存亡,在此一舉。群賊豺狼,弱於往日;惡逆之甚,重於丘山。大晉受命,未改於上;兆庶謳吟,思德於下。今順天下之心,命貔貅之士,鳴檄前驅,大軍後至,威風赫然,聲振朔野,則上副西土義士之情,下允海內延頸之望矣。」屬有杜弢之難,不能從。



 時江東草創,農桑弛廢,遠建議曰:「立春之日,天子祈穀于上帝,乃擇元辰,載耒
 耜,帥三公、九卿、諸侯、大夫,躬耕帝藉,以勸農功。《詩》云:『弗躬弗親,庶人不信。』自喪亂以來,農桑不修,遊食者多,皆由去本逐末故也。」時議美之。



 建興初,正旦將作樂,遠諫曰;「謹案《尚書》,堯崩,四海遏密八音。《禮》云,凶年,天子撤樂減膳。孝懷皇帝梓宮未反,豺狼當途,人神同忿。公明德茂親,社稷是賴。今杜弢蟻聚湘川,比歲征行,百姓疲弊,故使義眾奉迎未舉。履端元日,正始之初,貢士鱗萃,南北雲集,有識之士於是觀禮。公與國同體,憂容未歇。昔齊桓貫澤之會,有憂中國之心,不召而至者數國。及葵丘自矜,叛者九國。人心所歸,惟道與義。將紹皇綱於既
 往,恢霸業於來今,表道德之軌,闡忠孝之儀,明仁義之統,弘禮樂之本,使四方之士退懷嘉則。今榮耳目之觀,崇戲弄之好,懼違《雲》、《韶》、《雅》、《頌》之美,非納軌物,有塵大教。謂宜設饌以賜群下而已。」元帝納之。



 轉丞相參軍。是時瑯邪國侍郎王鑒勸帝親征杜弢,遠又上疏曰:「皇綱失統,中夏多故,聖主肇祚,遠奉西都。梓宮外次,未反園陵,逆寇遊魂,國賊未夷。明公憂勞,乃心王室,伏讀聖教,人懷慷慨。杜弢小豎,寇抄湘川,比年征討,經載不夷。昔高宗伐鬼方,三年乃剋,用兵之難,非獨在今。伏以古今之霸王遭時艱難,亦有親征以隆大勛,亦有遣將以平小
 寇。今公親征,文武將吏、度支籌量、舟輿器械所出若足用者,然後可徵。愚謂宜如前遣五千人,徑與水軍進徵,既可得速,必不後時。昔齊用穰苴,燕晉退軍;秦用王翦,剋平南荊。必使督護得才,即賊不足慮也。」會弢已平,轉從事中郎,累遷太子中庶子、尚書左丞、散騎常侍。帝每歎其忠公,謂曰:「卿在朝正色,不茹柔吐剛,忠亮至到,可為王臣也。吾所欣賴,卿其勉之!」



 及中興建,帝欲賜諸吏投刺勸進者加位一等,百姓投刺者賜司徒吏,凡二十餘萬。遠以為「秦漢因赦賜爵,非長制也。今案投刺者不獨近者情重,遠者情輕,可依漢法例,賜天下爵,於恩為
 普,無偏頗之失。可以息檢核之煩,塞巧偽之端。」帝不從。



 轉御史中丞。時尚書刁協用事,眾皆憚之。尚書郎盧綝將入直,遇協於大司馬門外。協醉,使綝避之,綝不回。協令威儀牽捽綝墮馬,至協車前而後釋。遠奏免協官。



 時冬雷電,且大雨,帝下書責躬引過,遠復上疏曰:



 被庚午詔書,以雷電震,暴雨非時,深自剋責。雖禹湯罪己,未足以喻。臣闇於天道,竊以人事論之。陛下節儉敦朴,愷悌流惠,而王化未興者,皆群公卿士不能夙夜在公,以益大化,素餐負乘,秕穢明時之責也。



 今逆賊猾夏,暴虐滋甚,二帝幽殯,梓宮未反,四海延頸,莫不東望。而未能遣
 軍北討,仇賊未報,此一失也。昔齊侯既敗,七年不飲酒食肉,況此恥尤大。臣子之責,宜在枕戈為王前驅。若此志未果者,當上下克儉,恤人養士,撤樂減膳,惟修戎事。陛下憂勞於上,而群官未同戚容於下,每有會同,務在調戲酒食而已,此二失也。選官用人,不料實德,惟在白望,不求才幹,鄉舉道廢,請託交行。有德而無力者退,修望而有助者進;稱職以違俗見譏,虛資以從容見貴。是故公正道虧,私途日開,彊弱相陵,冤枉不理。今當官者以理事為俗吏,奉法為苛刻,盡禮為諂諛,從容為高妙,放蕩為達士,驕蹇為簡雅,此三失也。



 世所謂三失者,公
 法加其身;私議貶其非;轉見排退,陸沈泥滓。時所謂三善者,王法所不加;清論美其賢;漸相登進,仕不輟官,攀龍附鳳,翱翔雲霄。遂使世人削方為圓,撓直為曲,豈待顧道德之清塗,踐仁義之區域乎!是以萬機未整,風俗偽薄,皆此之由。不明其黜陟,以審能否,此則俗未可得而變也。



 今朝廷群司以從順為善,相違見貶,不復論才之曲直,言之得失也。時有言者,或不見用,是以朝少辯爭之臣,士有祿仕之志焉。郭翼上書,武帝擢為屯留令,又置諫官,所以容受直言,誘進將來,故人得自盡,言無隱諱。任官然後爵之,位定然後祿之。敷奏以言,明試以
 功,車服以庸。舜猶歷試諸難,而今先祿不試,甚違古義,亂之所由也。求才急於疏賤,用刑先於親貴,然後令行禁止,野無遺滯。堯取舜於仄陋,舜拔賢於巖穴,姬公不曲繩於天倫,叔向不虧法於孔懷。今朝廷法吏多出於寒賤,是以章書日奏而不足以懲物,官人選才而不足以濟事。宜招賢良於屠釣,聘耿介於丘園。若此道不改,雖并官省職,無救弊亂也。能哲而惠,何憂乎歡兜,何遷乎有苗,何畏乎巧言令色孔壬!此官得其人之益也。



 累遷侍中,出補會稽內史。時王敦作逆,沈充舉兵應之,加遠將軍,距而不受,不輸軍資於充,保境安眾為務。敦至
 石頭,諷朝廷徵遠,乃拜太常卿,加散騎常侍。敦深憚其正而有謀,引為長史。數月病卒。



 遠弟縉,名亞於遠,為王敦主簿,終於鄱陽太守。縉子鳴鵠,位至武昌太守。



 王鑒,字茂高,堂邑人也。父濬,御史中丞。鑒少以文筆著稱,初為元帝琅邪國侍郎。時杜弢作逆,江湘流弊,王敦不能制,朝廷深以為憂。鑒上疏勸帝征之,曰:



 天禍晉室,四海顛覆,喪亂之極,開闢未有。明公遭歷運之厄,當陽九之會,聖躬負伊周之重,朝廷延匡合之望。方將振長轡而御八荒,掃河漢而清天途。所藉之資,江南之地,蓋
 九州之隅角,垂盡之餘人耳。而百越鴟視於五嶺,蠻蜀狼顧於湘漢,江州蕭條,白骨塗地,豫章一郡,十殘其八。繼以荒年,公私虛匱,倉庫無旬月之儲,三軍有絕乏之色。賦斂搜奪,周而復始,卒散人流,相望於道。殘弱之源日深,全勝之勢未舉。鑒懼雲旗反旆,元戎凱入,未在旦夕也。昔齊旅未期而申侯懼其老,況暴甲三年,介胄生蟣虱,而可不深慮者哉!江揚本六郡之地,一州封域耳。若兵不時戢,人不堪命,三江受敵,彭蠡振搖,是賊踰我垣牆之內,窺我室家之好。黷武之眾易動,驚弓之鳥難安,鑒之所甚懼也。去年已來,累喪偏將,軍師屢失,送死
 之寇,兵厭奔命,賊量我力矣。雖繼遣偏裨,懼未足成功也。愚謂尊駕宜親幸江州,然後方召之臣,其力可得而宣;熊羆之士,其銳可得而奮。進左軍於武昌,為陶侃之重;建名將於安成,連甘卓之壘。南望交廣,西撫蠻夷。要害之地,勒勁卒以保之;深溝堅壁,按精甲而守之。六軍既贍,戰士思奮,爾乃乘隙騁奇,擾其窟穴,顯示大信,開以生途,杜弢之頸固已鎖於麾下矣。



 議者將以大舉役重,人不可擾。鑒謂暫擾以制敵,愈於放敵而常擾也。夫四體者,人之所甚愛,茍宜伐病,則削肌刮骨矣。然守不可虛,鑒謂王導可委以蕭何之任。或以小賊方斃,不足
 動千乘之重。鑒見王彌之初,亦小寇也,官軍不重其威,狡逆得肆其變,卒令溫懷不守,三河傾覆,致有今日之弊,此已然之明驗也。蔓草猶不可長,況狼兕之寇乎!當五霸之世,將非不良,士非不勇,征伐之役,君必親之,故齊桓免胄於邵陵,晉文擐甲於城濮。昔漢高、光武二帝,征無遠近,敵無大小,必手振金鼓,身當矢石,櫛風沐雨,壺漿不贍,馳騖四方,匪遑寧處,然後皇基克構,元勳以融。今大弊之極,劇於曩代,崇替之命,繫我而已。欲使鑾旂無野次之役,聖躬遠風塵之勞,而大功坐就,鑒未見其易也。魏武既定中國,親征柳城,揚旍盧龍之嶺,頓轡
 重塞之表,非有當時烽燧之虞,蓋一日縱敵,終己之患,雖戎略蒙險,不以為勞,況急於此者乎!劉玄德躬登漢山,而夏侯之鋒摧;吳偽祖親水斥長江,而關羽之首懸;袁紹猶豫後機,挫衄三分之勢;劉表臥守其眾,卒亡全楚之地。歷觀古今撥亂之主,雖聖賢,未有高拱閒居不勞而濟者也。前鑒不遠,可謂蓍龜。



 議者或以當今暑夏,非出軍之時。鑒謂今宜嚴戒,須秋而動。高風啟途,龍舟電舉,曾不十日,可到豫章。豫章去賊尚有千里之限,但臨之以威靈,則百勝之理濟矣。既掃清湘野,滌蕩楚郢,然後班爵序功,酬將士之勞;卷甲韜旗,廣農桑之務,播愷
 悌之惠,除煩苛之賦。比及數年,國富兵彊,龍驤虎步,以威天下,何思而不服,何往而不濟,桓文之功不難懋也。今惜一舉之勞,而緩垂死之寇,誠國家之大恥,臣子之深憂也。



 鑒以凡瑣,謬蒙獎育,思竭遇忠以補萬一。芻蕘之言,聖王不棄,戍卒之謀,先后採之。乞留神鑒,思其所陳。



 疏奏,帝深納之,即命中外戒嚴,將自征弢。會弢已平,故止。



 中興建,拜駙馬都尉、奉朝請,出補永興令。大將軍王敦請為記室參軍,未就而卒,時年四十一。文集傳於世。



 鑒弟濤及弟子ρ,並有才筆。濤字茂略,歷著作郎、無錫令。ρ字庭堅,亦為著作。並早卒。



 陳頵,字延思,陳國苦人也。少好學,有文義。父立宅起門,頵曰:「當使容馬車。」笑而從之。仕為郡督郵,檢獲隱匿者三千人,為一州尤最。太守劉享拔為主簿,州辟部從事,乘馬車還家,宗黨榮之。



 劾案沛王韜獄,未竟,會解結代楊準為刺史,韜因河間王顒屬結。結至大會,問主簿史鳳曰:「沛王貴籓,州據何法而擅拘邪?」時頵在坐,對曰:「甲午詔書,刺史銜命,國之外臺,其非所部而在境者,刺史並糾。事徵文墨,前後列上,七被詔書。如州所劾,無有違謬。」結曰:「眾人之言不可妄聽,宜依法窮竟。」又問僚
 佐曰:「河北白壤膏粱,何故少人士,每以三品為中正?」答曰:「《詩》稱『維嶽降神,生甫及申』。夫英偉大賢多出於山澤,河北土平氣均,蓬蒿裁高三尺,不足成林故也。」結曰:「張彥真以為汝潁巧辯,恐不及青徐儒雅也。」頵曰:「彥真與元禮不協,故設過言。老子、莊周生陳梁,伏羲、傅說、師曠、大項出陽夏,漢魏二祖起於沛譙,準之眾州,莫之與比。」結甚異之,曰:「豫州人士常半天下,此言非虛。」會結遷尚書,結恨不得盡其才用。



 元康中,舉孝廉,而州將留之。頵薦同縣焦保曰:「保出自寒素,稟質清沖,若得參嘉命,必能光贊大猷,允清朝望,使黃憲之徒不乏於豫土,令頵
 庶免臧文之責。」州乃辟保。



 齊王冏起義,州遣頵將兵赴之,拜駙馬都尉。遭賊避難于江西。歷陽內史朱彥引為參軍。鎮東從事中郎袁琇薦頵於元帝,遷鎮東行參軍事,典法兵二曹。頵與王導書曰:「中華所以傾弊,四海所以土崩者,正以取才失所,先白望而後實事,浮競驅馳,互相貢薦,言重者先顯,言輕者後敘,遂相波扇,乃至陵遲。加有莊老之俗傾惑朝廷,養望者為弘雅,政事者為俗人,王職不恤,法物墜喪。夫欲制遠,先由近始。故出其言善,千里應之。今宜改張,明賞信罰,拔卓茂於密縣,顯朱邑於桐鄉,然後大業可舉,中興可冀耳。」



 建興初制,版
 補錄事參軍。參佐掾屬多設解故以避事任。頵議:「諸僚屬乘昔西臺養望餘弊,小心恭肅,更以為俗,偃蹇倨慢,以為優雅。至今朝士縱誕,臨事遊行,漸弊不革,以至傾國。故百尋之屋突直而燎焚,千里之隄蟻垤而穿敗,古人防小以全大,慎微以杜萌。自今臨使稱疾,須催乃行者,皆免官。」



 初,趙王倫篡位,三王起義,制《己亥格》,其後論功雖小,亦皆依用。頵意謂不宜以為常式,駁之曰:「聖王懸爵賞功,制罰糾違,斯道茍明,人赴水火。且名器之實,不可妄假,非才謂之致寇,寵厚戒在斯亡。昔孫秀口唱篡逆,手弄天機,惠皇失御,九服無戴。三王建議,席卷四
 海,合起義之眾,結天下之心,故設《己亥義格》以權濟難。此自一切之法,非常倫之格也。其起義以來,依格雜猥,遭人為侯,或加兵伍,或出皁僕,金紫佩士卒之身,符策委庸隸之門,使天官降辱,王爵黷賤,非所以正皇綱重名器之謂也。請自今以後宜停之。」頵以孤寒,數有奏議,朝士多惡之,出除譙郡太守。



 大興初,以疾徵。久之,白衣兼尚書,因陳時務,以為「昔江外初平,中州荒亂,故貢舉不試。宜漸循舊,搜揚隱逸,試以經策。又馬隆、孟觀雖出貧賤,勳濟甚大,以所不習,而統戎事,鮮能以濟。宜開舉武略任將率者,言問核試,盡其所能,然後隨才授任。舉
 十得一,猶勝不舉,況或十得二三。日磾降虜,七世內侍;由餘戎狄,入為秦相。豈藉華宗之族,見齒於奔競之流乎!宜引幽滯之雋,抑華校實,則天清地平,人神感應。」



 後拜天門太守,殊俗安之。選腹心之吏為荊州參軍,若有調發,動靜馳白,故恒得宿辦。陶侃徵還,頵先至巴陵上禮。侃以為能,表為梁州刺史。綏懷荒弊,甚有威惠。梁州大姓互相嫉妒,說頵年老耳聾,侃召頵還,以西陽太守蔣巽代之。年六十九卒。



 高崧,字茂琰,廣陵人也。父悝,少孤,事母以孝聞。年十三,
 值歲饑,悝菜蔬不饜,每致甘肥於母。撫幼弟以友愛稱。寓居江州,刺史華軼辟為西曹書佐。及軼敗,悝藏匿軼子經年,會赦乃出。元帝嘉而宥之,以為參軍,遂歷顯位,至丹陽尹、光祿大夫,封建昌伯。



 崧少好學,善史書。總角時,司空何充稱其明惠。充為揚州,引崧為主簿,益相欽重。轉驃騎主簿,舉州秀才,除太學博士,父艱去職。初,悝以納妾致訟被黜,及終,崧乃自繫廷尉訟冤,遂停喪五年不葬,表疏數十上。帝哀之,乃下詔曰:「悝備位大臣,違憲被黜,事已久判。其子崧求直無已。今特聽傳侯爵。」由是見稱。拜中書郎、黃門侍郎。



 簡文帝輔政,引為撫
 軍司馬。時桓溫擅威,率眾北伐,軍次武昌,簡文患之。崧曰:「宜致書喻以禍福,自當反旆。如其不爾,便六軍整駕,逆順於茲判矣。若有異計,請先釁鼓。」便於坐為簡文書草曰:「寇難宜平,時會宜接,此實為國遠圖,經略大算。能弘斯會,非足下而誰!但以此興師動眾,要當以資實為本。運轉之艱,古人之所難,不可易之於始而不熟慮,須所以深用惟疑,在乎此耳。然異常之舉,眾之所駭,遊聲噂沓,想足下亦少聞之。茍患失之,無所不至。或能望風振擾,一時崩散。如其不然者,則望實並喪,社稷之事去矣。皆由吾闇弱,德信不著,不能鎮靜群庶,保固維城,所
 以內愧于心,外慚良友。吾與足下雖職有內外,安社稷,保家國,其致一也。天下安危,繫之明德。先存寧國,而後圖其外,使王基克隆,大義弘著,所望於足下。區區誠懷,豈可復顧嫌而不盡哉!」溫得書,還鎮。



 崧累遷侍中。是時謝萬為豫州都督,疲於親賓相送,方臥在室。崧徑造之,謂曰:「卿令疆理西籓,何以為政?」萬粗陳其意。崧便為敘刑政之要數百言。萬遂起坐,呼崧小字曰:「阿酃!故有才具邪!」哀帝雅好服食,崧諫以為「非萬乘所宜。陛下此事,實日月之一食也」。後以公事免,卒於家。子耆,官至散騎常侍。



 史臣曰:昔張良拙說項氏,巧謀於沛公;孫惠沮計齊王,耀奇於東海,終而誓甘之旅炎運載昌,稱狩之師金行不競。豈遭時之會斯蹇,將謀國之道未通?迷於委質之貞,暗於所修之慮,本既顛矣,何以能終!熊遠、王鑒有毗濟之道,比之大廈,其榱桷之佐乎!崧之詆溫,頵之距結,挫其勞役之策,申其汝潁之論,採郭嘉之風旨,挹硃育之餘波,故桓溫輟許攸之謀,解結欽王朗之跡。緝之時典,用此道歟!



 贊曰:臨湘游藝,才識英發。詭名違穎,陳書干越。孝文忠謇,嘉言斯踐。茂高器鑒,雕章尤善。侯爵崧傳,高門頵顯。



\end{pinyinscope}