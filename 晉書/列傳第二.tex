\article{列傳第二}

\begin{pinyinscope}

 后妃下元敬虞皇后豫章君明穆庾皇后成恭杜皇後章太妃康獻褚皇后穆章何皇后哀靖王皇后廢帝孝庾皇後簡文宣鄭太后簡文順王皇后孝武文李太后孝武定王皇后安德陳太后安僖王皇后恭思褚皇后



 元敬虞皇后,諱孟母,濟陽外黃人也。父豫,見《外戚傳》。帝為瑯邪王,納后為妃,無子。永嘉六年薨,時年三十五。



 帝為晉王,追尊為王后。有司奏王后應別立廟。令曰:「今宗廟未成,不宜更興作,便修飾陵上屋以為廟。」太興三年,冊曰:「皇帝咨前瑯邪王妃虞氏:朕祗順昊天成命,用陟
 帝位,悼妃夙徂,徽音潛翳,御于家邦,靡所儀刑,陰教有虧,用傷于懷。追號制謚,先王之典。今遣使持節兼太尉萬勝奉冊贈皇后璽綬,祀以太牢。魂而有靈,嘉茲寵榮。」乃祔于太廟,葬建平陵。太寧初,明帝追懷母養之恩,贈豫妻王氏為雲阜陽縣君,從母散騎常侍新野王罕妻為平陽鄉君。



 豫章君荀氏,元帝宮人也。初有寵,生明帝及瑯邪王裒,由是為虞后所忌。自以位卑,每懷怨望,為帝所譴,漸見疏薄。及明帝即位,封建安君,別立第宅。太寧元年,帝迎
 還臺內,供奉隆厚。及成帝立,尊重同於太后。咸康元年薨。詔曰:「朕少遭憫凶,慈訓無稟,撫育之勤,建安君之仁也。一旦薨殂,實思報復,永惟平昔,感痛哀摧。其贈豫章郡君,別立廟于京都。」



 明穆庾皇后,諱文君,潁川鄢陵人也。父琛,見《外戚傳》。后性仁慈,美姿儀。元帝聞之,聘為太子妃,以德行見重。明帝即位,立為皇后。冊曰:「妃庾氏昔承明命,作嬪東宮,虔恭中饋,思媚軌則。履信思順,以成肅雍之道;正位閨房,以著協德之美。朕夙罹不造,煢煢在疚。群公卿士,稽之
 往代,僉以崇嫡明統,載在典謨,宜建長秋,以奉宗廟。是以追述先志,不替舊命,使使持節兼太尉授皇后璽綬。夫坤德尚柔,婦道承姑,崇粢盛之禮,敦螽斯之義,是以利在永貞,克隆堂基,母儀天下,潛暢陰教。鑒于六列,考之篇籍,禍福無門,盛衰由人,雖休勿休。其敬之哉,可不慎歟!」



 及成帝即位,尊后曰皇太后。群臣奏:天子幼沖,宜依漢和熹皇后故事。辭讓數四,不得已而臨朝攝萬機。后兄中書令亮管詔命,公卿奏事稱皇太后陛下。咸和元年,有司奏請追贈后父及夫人毌丘氏,后陳讓不許,三請不從。及蘇峻作逆,京都傾覆,后見逼辱,遂以憂崩,
 時年三十二。后即位凡六年。其後帝孝思罔極,贈琛驃騎大將軍、儀同三司,毌丘氏安陵縣君。從母荀氏永寧縣君,何氏建安縣君。亮表陳先志,讓而不受。



 成恭杜皇后,諱陵陽,京兆人,鎮南將軍預之曾孫也。父乂,見《外戚傳》。成帝以后奕世名德,咸康二年備禮拜為皇后,即日入宮。帝御太極前殿,群臣畢賀,晝漏盡,懸籥,百官乃罷。,后少有姿色,然長猶無齒,有來求婚者輒中止。及帝納采之日,一夜齒盡生。改宣城陵陽縣為廣陽縣。七年三月,后崩,年二十一。外官五日一臨,內官旦一
 入,葬訖止。后在位六年,無子。



 先是,三吳女子相與簪白花,望之如素柰,傳言天公織女死,為之著服,至是而后崩。帝下詔曰:「吉凶典儀,誠宜備設。然豐約之度,亦當隨時,況重壤之下,而崇飾無用邪!今山陵之事,一從節儉,陵中唯潔掃而已,不得施塗車芻靈。」有司奏造凶門柏歷及調挽郎,皆不許,又禁遠近遣使,明年元會,有司奏廢樂。詔廢管絃,奏金石如故。



 孝武帝立,寧康二年,以后母裴氏為廣德縣君。裴氏名穆,長水校尉綽孫,太傅主簿遐女,太尉王夷甫外孫。中表之美,高於當世。遐隨東海王越遇害,無子,唯穆渡江,遂享榮慶,立第南掖門外,
 世所謂杜姥宅云。



 章太妃周氏以選入成帝宮,有寵,生哀帝及海西公。始拜為貴人。哀帝即位,詔有司議貴人位號,太尉桓溫議宜稱夫人,尚書僕射江[A170]議應曰太夫人,詔崇為皇太妃,儀服與太后同。又詔「朝臣不為太妃敬,合禮典不。」太常江逌議「位號不極,不應盡敬」。興寧元年薨。帝欲服重,江[A170]啟應緦麻三月。詔欲降為期年,[A170]又啟「厭屈私情,所以上嚴祖考」,帝從之。



 康獻褚皇后,諱蒜子,河南陽翟人也。父裒,見《外戚傳》。后聰明有器識,少以名家入為瑯邪王妃。及康帝即位,立為皇后,封母謝氏為尋陽鄉君。及穆帝即位,尊后曰皇太后。時帝幼沖,未親國政。領司徒蔡謨等上奏曰:「嗣皇誕哲岐嶷,繼承天統,率土宅心,兆庶蒙賴。陛下體茲坤道,訓隆文母。昔塗山光夏,簡狄熙殷,實由宣哲,以隆休祚。伏惟陛下德侔二媯,淑美《關雎》,臨朝攝政,以寧天下。今社稷危急,兆庶懸命,臣等章惶,一日萬機,事運之期,天祿所鐘,非復沖虛高讓之日。漢和熹、順烈,並亦臨朝,近明穆故事,以為先制。臣等不勝悲怖,謹伏地上請。乞
 陛下上順祖宗,下念臣吏,推公弘道,以協天人,則萬邦承慶,群黎更生。」太后詔曰:「帝幼沖,當賴群公卿土將順匡救,以酬先帝禮賢之意,且是舊德世濟之美,則莫重之命不墜,祖宗之基有奉,是其所以欲正位於內而已。所奏懇到,形於翰墨,執省未究,以悲以懼。先后允恭謙抑,思順坤道,所以不距群情,固為國計。豈敢執守沖暗,以違先旨。輒敬從所奏。」於是臨朝稱制。



 有司奏,謝夫人既封,荀、卞二夫人亦應追贈,皆后之前母也。太后不許。太常殷融議依鄭玄義,衛將軍裒在宮庭則盡臣敬,太后歸寧之日自如家人之禮。太后詔曰:「典禮誠所未詳,
 如所奏,是情所不能安也,更詳之。」征西將軍翼、南中郎尚議謂「父尊盡於一家,君敬重於天下,鄭玄義合情禮之中」。太后從之。自後朝臣皆敬裒焉。



 帝既冠,太后詔曰:「昔遭不造,帝在幼沖,皇緒之微,眇若贅旒。百辟卿士率遵前朝,勸喻攝政。以社稷之重,先代成義,僶俛敬從,弗遑固守。仰憑七廟之靈,俯仗群后之力,帝加元服,禮成德備,當陽親覽,臨御萬國。今歸事反政,一依舊典。」於是居崇德宮,手詔群公曰:「昔以皇帝幼沖,從群后之議,既以闇弱,又頻丁極艱,銜恤歷祀,沈憂在疚。司徒親尊德重,訓救其弊,王室之不壞,實公是恁。帝既備茲冠禮,而
 四海未一,五胡叛逆,豺狼當路,費役日興,百姓困苦。願諸君子思量遠算,戮力一心,輔翼幼主,匡救不逮。未亡人永歸別宮,以終餘齒。仰惟家國,故以一言託懷。」



 及哀帝、海西公之世,太后復臨朝稱制。桓溫之廢海西公也,太后方在佛屋燒香,內侍啟云:「外有急奏」,太后乃出。尚倚戶前視奏數行,乃曰「我本自疑此」,至半便止,索筆答奏云:「未亡人罹此百憂,感念存沒,心焉如割。」溫始呈詔草,慮太后意異,悚動流汗,見於顏色。及詔出,溫大喜。



 簡文帝即位,尊后為崇德太后。及帝崩,孝武帝幼沖,桓溫又薨。群臣啟曰:「王室多故,禍艱仍臻,國憂始周,復喪元輔,
 天下惘然,若無攸濟。主上雖聖資奇茂,固天誕縱。而春秋尚富,如在諒闇,蒸蒸之思,未遑庶事。伏惟陛下德應坤厚,宣慈聖善,遭家多艱,臨朝親覽。光大之美,化洽在昔,謳歌流詠,播溢無外。雖有莘熙殷,妊姒隆周,未足以喻,是以五謀克從,人鬼同心,仰望來蘇,懸心日月。夫隨時之義,《周易》所尚,寧固社稷,大人之任。伏願陛下撫綜萬機,釐和政道,以慰祖宗,以安兆庶。不勝憂國喁喁至誠。」太后詔曰:「王室不幸,仍有艱屯。覽省啟事,感增悲嘆。內外諸君並以主上春秋沖富,加蒸蒸之慕,未能親覽,號令宜有所由。茍可安社稷,利天下,亦豈有所執,輒敬
 從所啟。但暗昧之闕,望盡弼諧之道。」於是太后復臨朝。帝既冠,乃詔曰:「皇帝婚冠禮備,遐邇宅心,宜當陽親覽,緝熙惟始。今歸政事,率由舊典。」於是復稱崇德太后。



 太元九年,崩於顯陽殿,年六十一,在位凡四十年。太后於帝為從嫂,朝議疑其服。太學博士徐藻議曰:「資父事君而敬同。又《禮》云『其夫屬父道者,妻皆母道也』,則夫屬君道,妻亦后道矣。服后以齊,母之義也。魯譏逆祀,以明尊卑。今上躬奉康、穆、哀皇及靖后之祀,致敬同於所天,豈可敬之以君道,而服廢於本親。謂應齊衰期。」從之。



 穆章何皇后,諱法倪,廬江灊人也。父準,見《外戚傳》。以名家膺選。升平元年八月,下璽書曰:「皇帝咨前太尉參軍何琦:混元資始,肇經人倫,爰及夫婦,以奉天地宗廟社稷。謀於公卿,咸以宜率由舊典。今使使持節太常彪之、宗正綜,以禮納采。」琦答曰:「前太尉參軍、都鄉侯糞土臣何琦稽首頓首再拜。皇帝嘉命,訪婚陋族,備數採擇。臣從祖弟故散騎侍郎準之遺女,未閑教訓,衣履若如人。欽承舊章,肅奉典制。」又使兼太保、武陵王晞,兼太尉、中領軍洽,持節奉冊立為皇后。



 后無子。哀帝即位,稱穆皇后,居永安宮。桓玄篡位,移后入司徒府。路經太廟,后停
 輿慟哭,哀感路人。玄聞而怒曰:「天下禪代常理,何預何氏女子事耶!」乃降后為零陵縣君,與安帝俱西,至巴陵。及劉裕建義,殷仲文奉后還京都,下令曰:「戎車屢警,黎元阻饑。而饍御豐靡,豈與百姓同其儉約。減損供給,勿令游過。」后時以遠還,欲奉拜陵廟。有司以寇難未平,奏停。元興三年崩,年六十六,在位凡四十八年。



 哀靖王皇后,諱穆之,太原晉陽人也。司徒左長史濛之女也。后初為瑯邪王妃。哀帝即位,立為皇后,追贈母爰氏為安國鄉君。后在位三年,無子。興寧二年崩。



 廢帝孝庾皇后,諱道憐,潁川焉陵人也。父冰,自有傳。初為東海王妃。及帝即位,立為皇后。太和六年崩,葬于敬平陵。帝廢為海西公,追貶后曰海西公夫人。太元十一年,海西公薨于吳,又以后合葬于吳陵。



 簡文宣鄭太后,諱阿春,河南滎陽人也。世為冠族。祖合,臨濟令。父愷,字祖元,安豐太守。后少孤,無兄弟,唯姊妹四人,后最長。先適渤海田氏,生一男而寡,依于舅濮陽吳氏。元帝為丞相,敬后先崩,將納吳氏女為夫人。后及
 吳氏女並游後園,或見之,言於帝曰:「鄭氏女雖嫠,賢於吳氏遠矣。」建武元年,納為瑯邪王夫人。甚有寵。后雖貴幸,而恒有憂色。帝問其故,對曰:「妾有妹,中者已適長沙王褒,餘二妹未有所適,恐姊為人妾,無復求者。」帝因從容謂劉隗曰:「鄭氏二妹,卿可為求佳對,使不失舊。」隗舉其從子傭娶第三者,以小者適漢中李氏,皆得舊門。帝召王褒為尚書郎,以悅后意。后生瑯邪悼王、簡文帝、尋陽公主。帝稱尊號,后雖為夫人,詔太子及東海、武陵王皆母事之。帝崩,后稱建平國夫人。



 咸和元年薨,簡文帝時為瑯邪王,制服重。有司以王出繼,宜降所生,國臣不
 能匡正,奏免國相諸葛頤。王上疏曰:「亡母生臨臣國,沒留國第,臣雖出後,亦無所厭,則私情得敘。昔敬后崩,孝王已出繼,亦還服重。此則明比,臣所憲章也。」明穆皇后不奪其志,乃徙瑯邪王為會稽王,追號后曰會稽太妃。及簡文帝即位,未及追尊。臨崩,封皇子道子為瑯邪王,領會稽國,奉太妃祀。



 太元十九年,孝武帝下詔曰:「會稽太妃文母之德,徽音有融,誕載聖明,光延于晉。先帝追尊聖善,朝議不一,道以疑屈。朕述遵先志,常惕于心。今仰奉遺旨,依《陽秋》二漢孝懷皇帝故事,上太妃尊號曰簡文太后。」於是立廟於太廟路西,陵曰嘉平。時群臣希
 旨,多謂鄭太后應配食於元帝者。帝以問太子前率徐邈,邈曰:「臣案《陽秋》之義,母以子貴。魯隱尊桓母,別考仲子之宮而不配食於惠廟。又平素之時,不伉儷於先帝,至於子孫,豈可為祖考立配?其崇尊盡禮,由於臣子,故得稱太后,陵廟備典。若乃祔葬配食,則義所不可。」從之。



 簡文順王皇后,諱簡姬,太原晉陽人也。父遐,見《外戚傳》。后以冠族,初為會稽王妃,生子道生,為世子。永和四年,母子並失帝意,俱被幽廢,后遂以憂薨。咸安二年,孝武帝即位,追尊曰順皇后,合葬高平陵,追贈后父遐特進、
 光祿大夫,加散騎常侍。



 孝武文李太后,諱陵容,本出微賤。始簡文帝為會稽王,有三子,俱夭。自道生廢黜,獻王早世,其後諸姬絕孕將十年。帝令卜者扈謙筮之,曰:「後房中有一女,當育二貴男,其一終盛晉室。」時徐貴人生新安公主,以德美見寵。帝常冀之有娠,而彌年無子,會有道士許邁者,朝臣時望多稱其得道。帝從容問焉,答曰:「邁是好山水人,本無道術,斯事豈所能判!但殿下德厚慶深,宜隆奕世之緒,當從扈謙之言,以存廣接之道。」帝然之,更加採納。又數
 年無子,乃令善相者召諸愛妾而示之,皆云非其人,又悉以諸婢媵示焉。時后為宮人,在織坊中,形長而色黑,宮人皆謂之昆侖。既至,相者驚云:「此其人也。」帝以大計,召之侍寢。后數夢兩龍枕膝,日月入懷,意以為吉祥,向儕類說之,帝聞而異焉,遂生孝武帝及會稽文孝王、鄱陽長公主。



 及孝武帝初即位,尊為淑妃。太元三年,進為貴人,九年,又進為夫人。十二年,加為皇太妃,儀服一同太后。十九年,會稽王道子啟:「母以子貴,慶厚禮崇。伏惟皇太妃純德光大,休祐攸鐘,啟嘉祚於聖明,嗣徽音於上列。雖幽顯同謀,而稱謂未盡,非所以仰述聖心,允答
 天人。宜崇正名號,詳案舊典。」八月辛巳,帝臨軒,遣兼太保劉耽尊為皇太后,稱崇訓宮。安帝即位,尊為太皇太后。



 隆安四年,崩于含章殿。朝議疑其服制,左僕射何澄、右僕射王雅、尚書車胤、孔安國、祠部郎徐廣等議曰:「太皇太后名位允正,體同皇極,理制備盡,情禮兼申。《陽秋》之義,母以子貴,既稱夫人,禮服從正。故成風顯夫人之號,文公服三年之喪。子於父母之所生,體尊義重。且禮祖不厭孫,固宜追服無屈,而緣情立制。若嫌明文不存,則疑斯從重,謂應同於為祖母後齊衰三年。」從之。皇后及百官皆服齊衰期,永安皇后一舉哀。於是設廬於西
 堂,凶儀施于神獸門,葬修平陵,神主祔于宣太后廟。



 孝武定王皇后,諱法慧,哀靖皇后之姪也。父蘊,見《外戚傳》。初,帝將納后,訪于公卿。于時蘊子恭以弱冠見僕射謝安,安深敬重之。既而謂人曰:「昔毛嘉恥於魏朝,楊駿幾傾晉室。若帝納后,有父者,唯蔭望如王蘊乃可。」既而訪蘊女,容德淑令,乃舉以應選。寧康三年,中軍將軍桓沖等奏曰:「臣聞天地之道,蓋相須而化成;帝后之德,必相協而政隆。然後品物流形,彞倫攸敘,靈根長固,本枝百世。天人同致,莫不由此。是以塗山作儷,而夏族以熙;
 妊姒配周,而姬祚以昌。今長秋將建,宜時簡擇。伏聞試守晉陵太守王蘊女,天性柔順,四業允備。且盛德之胄,美善先積。臣等參議,可以配德乾元,恭承宗廟,徽音六宮,母儀天下。」於是帝始納焉。封蘊妻劉氏為樂平鄉君。后性嗜酒驕妒,帝深患之。乃召蘊於東堂,具說后過狀,令加訓誡。蘊免冠謝焉。后於是少自改飾。太元五年崩,年二十一,葬隆平陵。



 安德陳太后,諱歸女,松滋潯陽人也。父廣,以倡進,仕至平昌太守。后以美色能歌彈,入宮為淑媛,生安、恭二帝。
 太元十五年薨,贈夫人。追崇曰皇太后,神主祔于宣太后廟,陵曰熙平。



 安僖王皇后,諱神愛,瑯邪臨沂人也。父獻之,見別傳。母新安愍公主。后以太元二十一年納為太子妃。及安帝即位,立為皇后。無子。義熙八年崩於徽音殿,時年二十九,葬休平陵。



 恭思褚皇后,諱靈媛,河南陽翟人,義興太守爽之女也。后初為瑯邪王妃。元熙元年,立為皇后,生海鹽、富陽公
 主。及帝禪位于宋,降為零陵王妃。宋元嘉十三年崩,時年五十三,祔葬沖平陵。



 史臣曰:方祇體安,儷乾儀而合德;圓舒循晷,配羲曜以齊明。故知陽爍陰凝,萬物假其陶鑄;火炎水潤,六氣由其調理。取譬賢淑,作伉文思,靈根式固,實資於此。宣穆閱禮,偶德潛鱗,翊天造之艱虞,嗣塗山之逸響,寶運歸其後胤,蓋有母儀之助焉。武元楊氏預聞朝政,明不逮遠,愛弱私情,深杜衛瓘之言,不曉張泓之詐,運其陰沴,韜映乾明,晉道中微,基於是矣。惠皇稟質,天縱其嚚,識暗鳴蛙,智昏文蛤。南風肆狡,扇禍稽天。初踐椒宮,逞梟
 心於長樂;方觀梓樹,頒鴆羽於離明。褒后滅周,方之蓋小;妹妃傾夏,曾何足喻。中原陷於鳴鏑,其兆彰於此焉。昔者高宗諒闇,總百官於元老;成王沖眇,託萬機于上公。太后御宸,諒知非古。而明穆、康獻,仍世臨朝,時屬委裘,躬行負扆。各免華陽之釁,竟躡和熹之蹤,保陵遲以克終,所幸實為多矣。



 贊曰:二妃光舜,三母翼周。末升夷癸,褒進亡幽。家邦興滅,職此之由。穆后沈斷,忘情執爨。故劍辭恩,池蒲起嘆。崇化繁祉,肇基商亂。二楊繼寵,福極災生。南風熾虐,國喪身傾。獻容幸亂,居辱疑榮。援筆廢主,持尺威帝。契闊
 終罹,殷憂以斃。芬實窈窕,芳菲婉[A148]。呂妾變嬴,黃姬化羋。石文遠著,金行潛徙。婦德傾城,迷朱奪紫。



\end{pinyinscope}