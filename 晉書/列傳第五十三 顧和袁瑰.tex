\article{列傳第五十三 顧和袁瑰}

\begin{pinyinscope}
顧和袁
 瑰
 \gezhu{
  子喬喬孫松
  瑰弟猷從祖準準孫耽耽子質質子湛豹}
 江逌
 \gezhu{
  從弟灌灌子績車胤殷顗王雅}



 顧和,字君孝,侍中眾之族子也。曾祖容,吳荊州刺史。祖相,臨海太守。和二歲喪父,總角便有清操,族叔榮雅重之,曰:「此吾家麒麟,興吾宗者,必此子也。」時宗人球亦有令聞,為州別駕,榮謂之曰:「卿速步,君孝超卿矣!」



 王導為揚州,辟從事。月旦當朝,未入,停車門外。周顗遇之,和方擇虱,夷然不動。顗既過,顧指和心曰:「此中何所有?」和徐
 應曰:「此中最是難測地。」顗入,謂導曰:「卿州吏中有一令僕才。」導亦以為然。和嘗詣導,導小極,對之疲睡。和欲叩會之,因謂同坐曰:「昔每聞族叔元公道葉贊中宗,保全江表。體小不安,令人喘息。」導覺之,謂和曰:「卿珪璋特達,機警有鋒,不徒東南之美,實為海內之俊。」由是遂知名。既而導遣八部從事之部,和為下傳還,同時俱見,諸從事各言二千石官長得失,和獨無言。導問和:「卿何所聞?」答曰:「明公作輔,寧使網漏吞舟,何緣採聽風聞,以察察為政。」導咨嗟稱善。



 累遷司徒左曹掾。時東海王沖為長水校尉,妙選僚屬,以沛國劉耽為司馬,和為主簿。永昌初,
 除司徒掾。太寧初,王敦請為主簿,遷太子舍人、車騎參軍、護軍長史。王導為揚州,請為別駕,所歷皆著稱。遷散騎侍郎、尚書吏部。司空郗鑒請為長史,領晉陵太守。咸康初,拜御史中丞,劾奏尚書左丞戴抗臟汙百萬,付法議罪,並免尚書傅玩、郎劉傭官,百僚憚之。遷侍中。初,中興東遷,舊章多闕,而冕旒飾以翡翠珊瑚及雜珠等。和奏:「舊冕十有二旒,皆用玉珠,今用雜珠等,非禮。若不能用玉,可用白旋珠。」成帝於是始下太常改之。先是,帝以保母周氏有阿保之勞,欲假其名號,內外皆奉詔。和獨上疏以為「周保祐聖躬,不遺其勳,第舍供給擬於戚
 屬,恩澤所加已為過隆。若假名號,記籍未見明比,惟漢靈帝以乳母趙嬈為平氏君,此末代之私恩,非先代之令典。且君舉必書,將軌物垂則。書而不法,後嗣何觀!」帝從之。轉吏部尚書,頻徙領軍將軍、太常卿、國子祭酒。



 康帝即位,將祀南北郊,和議以為車駕宜親行。帝從之,皆躬親行禮。遷尚書僕射,以母老固辭,詔書敕喻,物聽暮出朝還,其見優遇如此。尋朝議以端右之副不宜處外,更拜銀青光祿大夫,領國子祭酒。頃之,母憂去職,居喪以孝聞。既練,衛將軍褚裒上疏薦和,起為尚書令,遣散騎郎喻旨。和每見逼促,輒號兆慟絕,謂所親曰:「古人或
 有釋其憂服以祗王命,蓋以才足乾時,故不得不體國徇義。吾在常日猶不如人,況今中心荒亂,將何以補於萬分,只足以示輕忘孝道,貽素冠之義耳。」帝又下詔曰:「百揆務殷,端右總要,而曠職經久,甚以悒然。昔先朝政道休明,中夏隆盛,山賈諸公皆釋服從時,不獲遂其情禮。況今日艱難百王之弊,尚書令禮已過祥練,豈得聽不赴急疾而遂罔極之情乎!」和表疏十餘上,遂不起,服闋,然後視職。



 時南中郎將謝尚領宣城內史,收涇令陳幹殺之,有司以尚違法糾黜,詔原之。和重奏曰:「尚先劾姦臟罪,入甲戍赦,聽自首減死。而尚近表雲幹包藏姦
 猾,輒收行刑。幹事狀自郡,非犯軍戎,不由都督。案尚蒙親賢之舉,荷文武之任,不能為國惜體,平心聽斷,內挾小憾,肆其威虐,遠近怪愕,莫不解體。尚忝外屬,宥之有典,至於下吏,宜正刑辟。」尚,皇太后舅,故寢其奏。時汝南王統、江夏公衛崇並為庶母制服三年,和乃奏曰:』禮所以軌物成教,故有國家者莫不崇正明本,以一其統,斯人倫之紀,不二之道也。為人後者,降其所出,奪天屬之性,顯至公之義,降殺節文,著于周典。案汝南王統為庶母居廬服重,江夏公衛崇本由疏屬,開國之緒,近喪所生,復行重制,違冒禮度,肆其私情。閭閻許其過厚,談者
 莫以為非,則政道陵遲由乎禮廢,憲章頹替始於容違。若弗糾正,無以齊物。皆可下太常奪服。若不祗王命,應加貶黜。」詔從之。和居任多所獻納,雖權臣不茍阿撓。



 永和七年,以疾篤辭位,拜左光祿大夫、儀同三司,加散騎常侍,尚書令如故。其年卒,年六十四。追贈侍中、司空,謚曰穆。



 子淳,歷尚書吏部郎、給事黃門侍郎、左衛將軍。



 袁瑰,字山甫,陳郡陽夏人,魏郎中令渙之曾孫也。祖、父並早卒。瑰與弟猷欲奉母避亂,求為江淮間縣,拜呂令,轉江都,因南渡。元帝以為丹陽令。中興建,拜奉朝請,遷
 治書御史。時東海王越尸既為石勒所焚,妃裴氏求招魂葬越,朝廷疑之。瑰與博士傅純議,以為招魂葬是謂埋神,不可從也。帝然之,雖許裴氏招魂葬越,遂下詔禁之。尋除廬江太守。大將軍王敦引為諮議參軍。俄為臨川太守。敦平,為鎮南將軍卡敦軍司。尋自解還都,游于會稽。蘇峻之難,與王舒共起義軍,以功封長合鄉侯,徵補散騎常侍,徙大司農尋除國子祭酒。頃之,加散騎常侍。



 于時喪亂之後,禮教陵遲,瑰上疏曰:



 臣聞先王之教也,崇典訓以弘遠代,明禮樂以流後生,所以導萬物之性,暢為善之道也。宗周既興,文史載煥,端委垂於南蠻,
 頌聲溢於四海,故延州聘魯,聞《雅》而歎;韓起適魯,觀《易》而美。何者?立人之道,於斯為首。孔子恂恂以教洙泗,孟軻係之,誨誘無倦,是以仁義之聲于今猶存,禮讓之節時或有之。



 疇昔皇運陵替,喪亂屢臻,儒林之教漸頹,庠序之禮有闕,國學索然,墳籍莫啟,有心之徒抱志無由。昔魏武帝身親介胄,務在武功,猶尚廢鞍覽卷,投戈吟詠,況今陛下以聖明臨朝,百官以虔恭蒞事,朝野無虞,江外謐靜,如之何泱泱之風漠然無聞,洋洋之美墜於聖世乎!古人有言:「《詩》《書》義之府,禮樂德之則。」實宜留心經籍,闡明學義,使諷誦之音盈於京室,味道之賢是則
 是詠,豈不盛哉!若得給其宅地,備其學徒,博士僚屬粗有其官,則臣之願也。



 疏奏,成帝從之。國學之興,自瑰始也。以年在懸車,上疏告老,尋卒,追贈光祿大夫,謚曰恭。子喬嗣。



 喬字彥叔。初拜佐著作郎。輔國將軍桓溫請為司馬,除司徒左西屬,不就,拜尚書郎。桓溫鎮京口,復引為司馬,領廣陵相。初,喬與褚裒友善,及康獻皇后臨朝,喬與裒書曰:「皇太后踐登正阼,臨御皇朝,將軍之於國,外姓之太上皇也。至於皇子近屬,咸有揖讓之禮,而況策名人臣,而交媟人父,天性攸尊,亦宜體國而重矣。故友之好,
 請於此辭。染絲之變,墨翟致懷,岐路之感,楊朱興歎,況於將軍游處少長,雖世譽先後而臭味同歸也。平昔之交,與禮數而降,箕踞之嘆,隨時事而替,雖欲虛詠濠肆,脫落儀制,其能得乎!來物無停,變化遷代,豈惟寸晷,事亦有之。夫御器者神,制眾以約,願將軍貽情無事,以理勝為任,親杖賢達,以納善為大。執筆惆悵,不能自盡。」論者以為得禮。



 遷安西諮議參軍、長沙相,不拜。尋督沔中諸戍江夏隨義陽三郡軍事、建武將軍、江夏相。時桓溫謀伐蜀,眾以為不可,喬勸溫曰:「夫經略大事,故非常情所具,智者了於胸心,然後舉無遺算耳。今天下之難,二
 寇而已。蜀雖險固,方胡為弱,將欲除之,先從易者。今溯流萬里,經歷天險,彼或有備,不必可剋。然蜀人自以斗絕一方,恃其完固,不修攻戰之具,若以精卒一萬,輕軍速進,比彼聞之,我已入其險要,李勢君臣不過自力一戰,擒之必矣。論者恐大軍既西,胡必窺覦,此又似是而非。何者?胡聞萬里征伐,以為內有重備,必不敢動。縱復越逸江渚,諸軍足以守境,此無憂矣。蜀土富實,號稱天府,昔諸葛武侯欲以抗衡中國。今誠不能為害,然勢據上流,易為寇盜。若襲而取之者,有其人眾,此國之大利也。」溫從之,使喬以江夏相領二千人為軍鋒。師次彭模,
 去賊已近,議者欲兩道並進,以分賊勢。喬曰:「今深入萬里,置之死地,士無反顧之心,所謂人自為戰者也。今分為兩軍,軍力不一,萬一偏敗,則大事去矣。不如全軍而進,棄去釜甑,齎三日糧,勝可必矣。」溫以為然,即一時俱進。去成都十里,與賊大戰,前鋒失利,喬軍亦退,矢及馬首,左右失色。喬因麾而進,聲氣愈厲,遂大破之,長驅至成都。李勢既降,勢將鄧定、隗文以其屬反,眾各萬餘。溫自擊定,喬擊文,破之。進號龍驤將軍,封湘西伯。尋卒,年三十六,溫甚悼惜之。追贈益州刺史,謚曰簡。



 喬博學有文才,注《論語》及《詩》,並諸文筆皆行於世。



 子方平嗣,亦以
 軌素自立,辟大司馬掾,歷義興、瑯邪太守。卒,子山松嗣。



 山松少有才名,博學有文章,著《後漢書》百篇。衿情秀遠,善音樂。舊歌有《行路難》曲,辭頗疏質,山松好之,乃文其辭句,婉其節制,每因酣醉縱歌之。聽者莫不流涕。初羊曇善唱樂,桓伊能挽歌,及山松《行路難》繼之,時人謂之「三絕」。時張湛好於齋前種松柏,而山松每出游,好令左右作挽歌,人謂「湛屋下陳尸,山松道上行殯。」



 山松歷顯位,為吳郡太守。孫恩作亂,山松守滬瀆,城陷被害。



 猷字申甫,少與瑰齊名。代瑰為呂令,復相繼為江都,由是俱渡江。瑰為丹陽,猷為武康,兄弟列宰名邑,論者美
 之。歷位侍中、衛尉卿。猷孫宏,見《文苑傳》。



 準字孝尼,以儒學知名,注《喪服經》。官至給事中。準子沖,字景玄,光祿勳。沖子耽。



 耽字彥道,少有才氣,倜儻不羈,為士類所稱。桓溫少時游于博徒,資產俱盡,尚有負進,思自振之方,莫知所出,欲求濟於耽,而耽在艱,試以告焉。耽略無難色,遂變服懷布帽,隨溫與債主戲。耽素有藝名,債者聞之而不相識,謂之曰:「卿當不辦作袁彥道也。」遂就局十萬一擲,直上百萬。耽投馬絕叫,探布帽擲地,曰:「竟識袁彥道不?」其通脫若此。蘇峻之役,王導引為參軍,隨導在石頭。初,路
 永、匡術、寧等皆峻心腹,聞祖約奔敗,懼事不立,迭說峻誅大臣。峻既不納,永等慮必敗,陰結於導。導使耽潛說路永,使歸順。峻平,封秭歸男,拜建威將軍、歷陽太守。咸康初,石季龍游騎十餘匹至歷陽,耽上列不言騎少。時胡寇強盛,朝野危懼,王導以宰輔之重請自討之。既而賊騎不多,又已退散,導止不行。朝廷以耽失於輕妄,黜之。尋復為導從事中郎,方加大任,會卒,時年二十五。子質。



 質字道和。自渙至質五世,並以道素繼業,惟其父耽以雄豪著。及質又以孝行稱。官歷瑯邪內史、東陽太守。質
 子湛。



 湛字士深。少有操植,以沖粹自立,而無文華,故不為流俗所重。時謝混為僕射,范泰贈湛及混詩云:「亦有後出雋,離群頗騫翥。」湛恨而不答。自中書令為僕射、左光祿大夫、晉寧男,卒於官。湛弟豹。



 豹字士蔚,博學善文辭,有經國材,為劉裕所知。後為太尉長史、丹陽尹,卒。



 江逌,字道載,陳留圉人也。曾祖蕤,譙郡太守。祖允,蕪湖令。父濟,安東參軍。逌少孤,與從弟灌共居,甚相友悌,由
 是獲當時之譽。避蘇峻之亂,屏居臨海,絕棄人事,翦茅結宇,耽玩載籍,有終焉之志。本州辟從事,除佐著作郎,並不就。征北將軍蔡謨命為參軍,何充復引為驃騎功曹。以家貧,求試守,為太末令。縣界深山中,有亡命數百家,恃險為阻,前後守宰莫能平。逌到官,召其魁帥,厚加撫接,諭以禍福,旬月之間,襁負而至,朝廷嘉之。州檄為治中,轉別駕,遷吳令。



 中軍將軍殷浩將謀北伐,請為諮議參軍。浩甚重之,遷長史。浩方脩復洛陽,經營荒梗,逌為上佐,甚有匡弼之益,軍中書檄皆以委逌。時羌及丁零叛,浩軍震懼。姚襄去浩十里結營以逼浩,浩令逌擊
 之。逌進兵至襄營,謂將校曰:「今兵非不精,而眾少於羌,且其塹柵甚固,難與校力,吾當以計破之。乃取數百雞以長繩連之,系火於足。群雞駭散,飛集襄營。襄營火發,其亂,隨而擊之,襄遂小敗。及桓溫奏廢浩佐吏,遂免。頃之,除中書郎。升平中,遷吏部郎,長兼侍中。



 穆帝將修後池,起閣道,逌上疏曰:



 臣聞王者處萬乘之極,享富有之大,必顯明制度以表崇高,盛其文物以殊貴賤。建靈臺,浚辟雍,立宮館,設苑囿,所以弘於皇之尊,彰臨下之義。前聖創其禮,後代遵其矩,當代之君咸營斯事。周宣興百堵之作,《鴻鴈》歌安宅之歡;魯僖修泮水之營,採
 芹有思樂之頌。蓋上之有為非予欲是盈,下之奉上不以劬勞為勤,此自古之令典,軌儀之大式也。



 夫理無常然,三正相詭,司牧之體,與世而移。致飾則素,故《賁》返於《剝》;有大必盈,則受之以《謙》。損上益下,順兆庶之悅;享以二簋,用至約之義。是以唐虞流化於茅茨,夏禹垂美於卑室。過儉之陋,非中庸之制,然三聖行之以致至道。漢高祖當營建之始,怒宮庫之壯;孝文處既富之世,愛十家之產,亦以播惠當時,著稱來葉。



 今者二虜未殄,神州荒蕪,舉江左之眾,經略艱難,漕揚越之粟,北餽河洛,兵不獲戢,運戍悠遠,倉庫內罄,百姓力竭。加春夏以來,水
 旱為害,遠近之收普減常年,財傷人困,大役未已,軍國之用無所取給。方之往代,豐弊相懸,損之又損,實在今日。伏惟陛下聖質天縱,凝曠清虛,闡日新之盛,茂欽明之量,無欲體於自然,沖素刑乎萬國。《韶》既盡美,則必盡善。宜養以玄虛,守以無為,登覽不以臺觀。游豫不以苑沼,偃息畢於仁義,馳騁極於六藝,觀巍巍之隆,鑒二代之文,仰味羲農,俯尋周孔。其為逍遙,足以尊道德之輔,親搢紳之秀。疇咨以時,顧問不倦,獻替諷諫,日月而聞,則庶績惟凝,六合咸熙,中興之盛邁於殷宗,休嘉之慶流乎無窮。昔漢起德陽,鐘離抗言;魏營宮殿,陳群正辭。
 臣雖才非若人,然職忝近侍,言不足採,而義在以聞。



 帝嘉其言而止。復領本州大中正。升平末,遷太常,逌累讓,不許。



 穆帝崩,山陵將用寶器,諫曰:「以宣皇顧命終制,山陵不設明器,以貽後則。景帝奉遵遺制。逮文明皇后崩,武皇帝亦亦承前制,無所施設,惟脯Я之奠,瓦器而已。昔康皇帝玄宮始用寶劍金舄,此蓋太妃罔已之情,實違先旨累世之法。今外欲以為故事,臣請述先旨,停此二物。」書奏,從之。



 哀帝以天文失度,欲依《尚書》洪祀之制,於太極前殿親執虔肅,冀以免咎,使太常集博士草其制。逌上疏諫曰:



 臣尋《史》《漢》舊事,《藝文志》劉向《五行傳》,洪
 祀出於其中。然自前代以來,莫有用者。又其文惟說為祀,而不載儀注。此蓋久遠不行之事,非常人所參校。案《漢儀》,天子所親之祠,惟宗廟而已。祭天於雲陽,祭地於汾陰,在於別宮遙拜,不詣壇所。其餘群祀之所,必在幽靜,是以圓丘方澤列於郊野。今若於承明之庭,正殿之前,設群神之坐,行躬親之禮,準之舊典,有乖常式。



 臣聞妖眚之發,所以鑒悟時主,故夤畏上通,則宋災退度;德禮增修,則殷道以隆。此往代之成驗,不易之定理。頃者星辰頗有變異,陛下祗戒之誠達於天人,在予之懼,忘寢與食,仰虔玄象,俯疑庶政,嘉祥之應,實在今日。而猶
 乾乾夕惕,思廣茲道,誠實聖懷殷勤之至。然洪祀有書無儀,不行於世,詢訪時學,莫識其禮。且其文曰:「洪祀,大祀也。陽曰神,陰曰靈。舉國相率而行祀,順四時之序,無令過差。」今案文而言,皆漫而無適,不可得詳。若不詳而修,其失不小。



 帝不納,逌又上疏曰:



 臣謹更思尋,參之時事。今強戎據於關雍,桀狄縱於河朔,封豕四逸,虔劉神州,長旌不卷,鉦鼓日戒,兵疲人困,歲無休已。人事弊於下,則七曜錯於上,災沴之作,固其宜然。又頃者以來,無乃大異。彼月之蝕,義見詩人,星辰莫同,載於《五行》,故《洪範》不以為沴。



 陛下今以晷度之失同之六沴,引其輕變
 方之重眚,求己篤於禹湯,憂勤踰乎日昃,將修大祀,以禮神祇。傳曰:「外順天地時氣而祭其鬼神。」然則神必有號,祀必有義。案洪祀之文,惟神靈大略而無所祭之名,稱舉國行祀而無貴賤之阻,有赤黍之盛而無牲醴之奠,儀法所用,闕略非一。若率文而行,則舉義皆閡;有所施補,則不統其源。漢侍中盧植,時之達學,愛法不究,則不敢厝心。誠以五行深遠,神道幽昧,探賾之求難以常思,錯綜之理不可一數。臣非至精,孰能與此!



 帝猶敕撰定,逌又陳古義,帝乃止。逌在職多所匡諫。著《阮籍序贊》、《逸士箴》及詩賦奏議數十篇行於世。病卒,時年五十八。
 子蔚,吳興太守。



 灌字道群。父瞢,尚書郎。灌少知名,才識亞於逌。州辟主簿,舉秀才,為治中,轉別駕,歷司徒屬、北中郎中長史,領晉陵太守。簡文帝引為撫軍從事中郎,後遷吏部郎。時謝奕為尚書,銓敘不允,灌每執正不從,奕託以他事免之,受黜無怨色。頃之,簡文帝又以為撫軍司馬,甚相賓禮。遷御史中丞,轉吳興太守。灌性方正,視權貴蔑如也,為大司馬桓溫所惡。溫欲中傷之,徵拜侍中,以在郡時公事有失,追免之。後為祕書監,尋復解職。時溫方執權,朝廷希旨,故灌積年不調。溫末年,以為諮議參軍。會溫薨,
 遷尚書、中護軍,復出為吳郡太守,加秩中二千石,未拜,卒。子績。



 績字仲元,有志氣,除祕書郎。以父與謝氏不穆,故謝安之世辟召無所從,論者多之。安薨,始為會稽王道子驃騎主簿,多所規諫。歷諮議參軍,出為南郡相。會荊州刺史殷仲堪舉兵以應王恭,仲堪要績與南蠻校尉殷顗同行,並不從。仲堪等屢以為言,績終不為之屈。顗慮績及禍,乃於仲堪坐和解之。績曰:「大丈夫何至以死相脅!江仲元行年六十,但未知獲死所耳。」一坐為之懼。仲堪憚其堅正,以楊佺期代之。朝廷聞而征績為御史中
 丞,奏劾無所屈撓。會稽世子元顯專政,夜開六門,績密啟會稽王道子,欲以奏聞,道子不許。車胤亦曰:「元顯驕縱,宜禁制之。」道子默然。元顯聞而謂眾曰:「江績、車胤間我父子。」遣人密讓之。俄而績卒,朝野悼之。



 車胤,字武子,南平人也。曾祖浚,吳會稽太守。父育,郡主簿。太守王胡之名知人,見胤於童幼之中,謂胤父曰:「此兒當大興卿門,可使專學。」胤恭勤不倦,博學多通。家貧不常得油,夏月則練囊盛數十螢火以照書,以夜繼日焉。及長,風姿美劭,機悟敏速,甚有鄉曲之譽。桓溫在荊
 州,辟為從事,以辯識義理深重之。引為主簿,稍遷別駕、征西長史,遂顯於朝廷。時惟胤與吳隱之以寒素博學知名於世。又善於賞會,當時每有盛坐而胤不在,皆云:「無車公不樂。」謝安游集之日,輒開筵待之。



 寧康初,以胤為中書侍郎、關內侯。孝武帝嘗講《孝經》,僕射謝安侍坐,尚書陸納侍講,侍中卞眈執讀,黃門侍郎謝石、吏部郎袁宏執經,胤與丹陽尹王混擿句,時論榮之。累遷侍中。太元中,增置太學生百人,以胤領國子博士。其後年,議郊廟明堂之事,胤以「明堂之制既甚難詳,且樂主於和,禮主於敬,故質文不同,音器亦殊。既茅茨廣廈不一其
 度,何必守其形範而不弘本順時乎!九服咸寧,四野無塵,然後明堂辟雍可光而修之。」時從其議。又遷驃騎長史、太常,進爵臨湘侯,以疾去職。俄為護軍將軍。時王國寶諂於會稽王道子,諷八坐啟以道子為丞相,加殊禮。胤曰:「此乃成王所以尊周公也。今主上當陽,非成王之地,相王在位,豈得為周公乎!望實二三,並不宜爾,必大忤上意。」乃稱疾不署其事。疏奏,帝大怒,而甚嘉胤。



 隆安初,為吳興太守,秩中二千石,辭疾不拜。加輔國將軍、丹陽尹。頃之,遷吏部尚書。元顯有過,胤與江績密言於道子,將奏之,事泄,元顯逼令自裁。俄而胤卒,朝廷傷之。



 殷顗,字伯通,陳郡人也。祖融,太常卿。父康,吳興太守。顗性通率,有才氣,少與從弟仲堪俱知名。太元中,以中書郎擢為南蠻校尉。蒞職清明,政績肅舉。及仲堪得王恭書,將興兵內伐,告顗,欲同舉。顗不平之,曰:「夫人臣之義,慎保所守。朝廷是非,宰輔之務,豈籓屏之所圖也。晉陽之事,宜所不豫。」仲堪要之轉切,顗怒曰:「吾進不敢同,退不敢異。」仲堪以為恨。猶密諫仲堪,辭甚切至。仲堪既貴,素情亦殊,而志望無厭,謂顗言為非。顗見江績亦以正直為仲堪所斥,知仲堪當逐異己,樹置所親,因出行
 散,託疾不還。仲堪聞其病,出省之,謂顗曰:「兄病殊為可憂。」顗曰:「我病不過身死,但汝病在滅門,幸熟為慮,勿以我為念也。」仲堪不從,卒與楊佺期、桓玄同下。顗遂以憂卒。隆安中,詔曰:「故南蠻校尉殷顗忠績未融,奄焉隕喪,可贈冠軍將軍。」弟仲文、叔獻別有傳。



 王雅,字茂達,東海郯人,魏衛將軍肅之曾孫也。祖隆,後將軍。父景,大鴻臚。雅少知名,州檄主簿,舉秀才,除郎中,出補永興令,以幹理著稱。累遷尚書左右丞,歷廷尉、侍中、左衛將軍、丹陽尹,領太子左衛率。雅性好接下,敬慎
 奉公,孝武帝深加禮遇,雖在外職,侍見甚數,朝廷大事多參謀議。帝每置酒宴集,雅未至,不先舉觴,其見重如此。然任遇有過其才,時人被以佞幸之目。帝起清暑殿於後宮,開北上閣,出華林園,與美人張氏同游止,惟雅與焉。



 會稽王道子領太子太傅,以雅為太子少傅。時王珣兒婚,賓客車騎甚眾,會聞雅拜少傅,回詣雅者過半。時風俗頹弊,無復廉恥。然少傅之任,朝望屬珣,珣亦頗以自幸。及中詔用雅,眾遂赴雅焉。將拜,遇雨,請以傘入。王珣不許之,因冒雨而拜。雅既貴倖,威權甚震,門下車騎常數百,而善應接,傾心禮之。



 帝以道子無社稷器幹,
 慮晏駕之後皇室傾危,乃選時望以為籓屏,將擢王恭、殷仲堪等,先以訪雅。雅以恭等無當世之才,不可大任,從從容曰:「王恭風神簡貴,志氣方嚴,既居外戚之重,當親賢之寄,然其稟性峻隘,無所苞容,執自是之操,無守節之志。仲堪雖謹於細行,以文義著稱,亦無弘量,且幹略不長。若委以連率之重,據形勝之地,今四海無事,足能守職,若道不常隆,必為亂階矣。」帝以恭等為當時秀望,謂雅疾其勝己,故不從。二人皆被升用,其後竟敗,有識之士稱其知人。



 遷領軍、尚書、散騎常侍,方大崇進之,將參副相之重,而帝崩,倉卒不獲顧命。雅素被優遇,一
 旦失權,又以朝廷方亂,內外攜離,但慎默而已,無所辯正。雖在孝武世,亦不能犯顏廷爭,凡所謀謨,唯唯而已。尋遷左僕射。隆安四年卒,時年六十七。追贈光祿大夫、儀同三司。



 長子準之,散騎侍郎。次協之,黃門。次少卿,侍中。並有士操,立名於世云。



 史臣曰:爰在中興,玄風滋扇,溺王綱於拱默,撓國步於清虛,骨鯁蹇諤之風蓋亦微矣。而君孝固情禮而違顯命,山甫獻誠讜而振頹風,彥叔之兵謀,道載之正諫,洋洋盈耳,有足可稱。灌不屈節於權臣,績敢危言於賊將,道子殊物之禮,車胤沮之無懼心,仲堪反常之舉,殷顗
 折之以正色,求諸古烈,何以加焉!山松悅哀挽於軒冕之辰,彥道歡博徒於衰絰之日,天心已喪,其能濟乎!旋及於促齡,俄致於非命,宜哉!



 贊曰:顧生軌物,屢申誠讜。袁子崇儒,拯斯頹喪。逌績剛蹇,車殷忠壯。睠言遺直,莫之能尚。



\end{pinyinscope}