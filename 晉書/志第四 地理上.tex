\article{志第四 地理上}

\begin{pinyinscope}

 地理上


昔者元胎無象,太素流形,對越在天,以為元首,則《記》所謂冬居營窟,夏居橧巢,飲血茹毛,未有麻絲者也。及燧人鑽火,庖犧出震,風宗下武,炎胤昌基,畫野無聞,其歸一揆。黃帝則東海南江,登空躡岱,至於崑峰振轡,崆山防道,存諸汗竹,不可厚誣。高陽任地依神,帝嚳順天行
 義。東逾蟠木,西濟流沙,北至幽陵,南撫交址,日月所經,舟車所至,莫匪王臣,不踰茲域。帝堯時,禹平水土,以為九州。虞舜登庸,厥功彌劭,表提類而分區宇,判山河而考疆域,冀北創並部之名,燕齊起幽營之號,則《書》所謂肇十有二州,封十有二山者也。夏功在于唐堯,殷因無所損益。周武克商,自豐徂鎬。至成王時,改作《禹貢》,徐梁入於青雍,冀野析於幽并。職方掌天下之土,以周厥利;保章辯九州之野,皆有分星。東南曰揚州,正南曰荊州,河南曰豫州,正東曰青州,河東曰兗州,正西曰雍州,東北曰幽州,河內曰冀州,正北曰並州。始皇初并天下,懲
 \gezhu{
  乂心}
 戰國,削罷列侯,分天下為三十六郡。三川、河東、南陽、南郡、九江、鄣郡、會稽、潁川、碭郡、泗水、薛郡、東郡、瑯邪、齊郡、上谷、漁陽、右北平、遼西、遼東、代郡、鉅鹿、邯鄲、上黨、太原、雲中、九原、鴈門、上郡、隴西、北地、漢中、巴郡、蜀郡、黔中、長沙、凡三十五郡,與內史為三十六郡也。於是興師踰江,平取百越,又置閩中、南海、桂林、象郡,凡四十郡,郡一守焉。其他則西臨洮而北沙漠,東縈西帶,皆臨大海。漢祖龍興,革秦之弊,分內史為三部,更置郡國二十有三,桂陽、江夏、豫章、河內、魏郡、東海、楚國、平原、梁國、定襄、泰山、汝南、淮陽、千乘、東萊、燕國、清河、信都、常山、中山、渤海、廣漢、涿郡,合二十三也。三內史者,河上、渭南、中地也。《地理志》曰:高祖增二十六,武帝改河上、渭南、中地以為京兆、馮翊、扶風是為三輔也。文增厥九,廣平、城陽、淄川、濟南、膠西、膠東、河間、廬江、衡山、武帝改衡山曰六安。景加其四。濟北、濟陰、山陽、北海也。宣改濟北曰東平。武帝開越攘胡,初置
 十七,南海、蒼梧、鬱林、合浦、交恥、九真、日南、珠崖、儋耳九郡,平西南夷置牂柯、越嵩、沈黎、汶山、犍為、益州六郡,西置武都郡,又分立零陵郡,合十七郡。拓土分疆,又增十四。弘農、臨淮、西河、朔方、酒泉、陳留、安定、天水、玄菟、樂浪、廣陵、敦煌、武威、張掖。昭帝少事,又增其一。金城也。至平帝元始二年,凡新置郡國七十有一,與秦四十,合一百一十有一。改雍曰涼,改梁曰益,又置徐州,復夏舊號,南置交址,北有朔方,凡為十三部。涼、益、荊、揚、青、豫、兗、徐、幽、並、冀十一州,交址、朔方二刺史,合十三部。光武投戈之歲,在彫秏之辰,郡國蕭條,並省者八。城陽、淄川、高密、膠東、六安、真定、泗水、廣陽。建武十一年,省州牧,復為刺史,員十三人,各掌一州。明帝置一,永昌也。章帝置二,任城、吳郡。和順改作,其名有九。和置濟北、廣陽,順改淮陽為陳,改楚為彭城,濟東為東平,臨淮為下
 邳,千乘為六安,信都為安平,天水為漢陽。省朔方刺史,合之於司隸,凡十三部,其與西漢不同者,司隸校尉部郡治河南,朔方隸於並部。而郡國百有八焉。省前漢八,分置五,改舊名七,因舊九十六,少前漢三也。桓靈頗增於前,復置六郡。桓,高陽、高涼、博陵;靈,南安、鄱陽、廬陵。魏武定霸,三方鼎立,生寧版蕩,關洛荒蕪,所置者十二,新興、樂平、西平、新平、略陽、陰平、帶方、譙、樂陵、章武、南鄉、襄陽。所省者七,上郡、朔方、五原、雲中、定襄、漁陽、廬江。而文帝置七,朝歌、陽平、弋陽、魏興、新城、義陽、安豐。明及少帝增二,明,上庸也;少,平陽也。得漢郡者五十四焉。蜀先主於漢建安之間初置郡九,巴東、巴西、梓潼、江陽、汶山、漢嘉、朱提、宕渠、涪陵。後主增二,雲南、興古。得漢郡者十有一焉。吳主大皇帝初置郡五,臨賀、武昌、珠崖、新安、廬陵南部。少帝、景帝各四,少,臨川、臨海、衡陽、湘東。景,天門、建安、建平、合浦北部。歸
 命侯亦置十有二郡,始安、始興、邵陵、安成、新昌、武平、九德、吳興、東陽、桂林、滎陽、宜都。得漢郡者十有八焉。



 晉武帝太康元年,既平孫氏,凡增置郡國二十有三,滎陽、上洛、頓丘、臨淮、東莞、襄城、汝陰、長廣、廣寧、昌黎、新野、隨郡、陰平、義陽、毗陵、宣城、南康、晉安、寧浦、始平、咯陽、樂平、南平。省司隸置司州,別立梁、秦、寧、平四州,仍吳之廣州,凡十九州,司、冀、兗、豫、荊、徐、揚、青、幽、平、並、雍、涼、秦、梁、益、寧、交、廣州。郡國一百七十三,仍吳所置二十五,仍蜀新置十一,仍魏所置二十一,仍漢舊九十三,置二十三。以為冠帶之國,盡有殷周之土。若乃敦龐於天地之始,昭晰於犧農之世,用長黎元,未爭疆埸。而玉環楛矢,夷裘風駕,南翬表貺,東風入律,光乎上德,奚遠弗臻。然則星象麗天,山河紀地,端掖裁其弘敞,崤函判其都邑,仰
 觀俯察,萬物攸歸。是以洛沚咸陽,宛然秦漢,晉濱河西,同知堯禹,於茲新邑,宅是鎬京,五尺童子皆能口誦者,史官弗之書也。



 昔庖犧氏生于成紀,而為天子,都於陳。神農氏都陳,而別營于曲阜。黃帝生於壽丘。而都於涿鹿。少昊始自窮桑,而遷都曲阜。顓頊始自窮桑,而徙邑商丘。高辛即號,建都于亳。孫卿子曰:「不登高山,不知天之高;不臨深谿,不知地之厚也。」大哉坤象,萬物資生,載昆華而不墜,傾河海而寧泄。考卜惟王,乘飛駐軫,睨弇山而鐫勒,覽曾城以為玩。時逢稽浸,道接陵夷,平王東遷,星離豆剖,當塗馭寓,瓜分鼎立。世祖武皇帝接千祀
 之餘,當八堯之禪,先王桑梓,罄宇來歸,斯固可得而言者矣。惠皇不虞,中州盡棄,永嘉南度,綸行建鄴,九分天下而有二焉。



 昔大禹觀於濁河而受綠字,寰瀛之內可得而言也。天有七星,地有七表;天有四維,地有四瀆。八紘之外,名為八極。地不足東南,天不足西北。八極之廣,東西二億三萬一千三百里,南北二億三萬一千三百里。自地至天,半八極之數,自下亦如之。昔黃帝令豎亥步自東極,至于西極,五億十萬九千八百八步。史臣案,凡周天積百七萬九百一十三里,徑三十五萬六千九百七十里。所
 謂南北為經,東西為緯。天有十二次,日月之所躔;地有十二辰,王侯之所國也。或因生得姓,因功命土,祁、酉、燕、齊,在乎茲域。



 昔黃帝旁行天下,方制萬里,得百里之國萬區,則《周易》所謂「首出庶物,萬國咸寧」者也。昔在帝堯,葉和萬邦,制八家為鄰,三鄰為朋,三朋為里,五里為邑,十邑為都,十都為師,州十有二師焉。夏后氏東漸于海,西被于流沙,南浮于江,而朔南暨聲教,窮豎亥所步,莫不率俾,會群臣於塗山,執玉帛者萬國。於是九州之內,作為五服。天子之國,內五百里甸服,百里賦納總,二百里納銍,三百里納秸服,四百里粟,五百里米。甸服外五
 百里侯服,百里采,二百里任,三百里候。侯服外五百里綏服,三百里揆文教,二百里奮武衛。綏服外五百里要服,三百里夷,二百里蔡。要服外五百里荒服,三百里蠻,二百里流。訖于四海,弼成五服,五服至于五千里。夏德中微,遇有窮之亂。少康中興,不失舊物。自孔甲之後,以至於桀,諸侯相兼,其能存者三千餘國,方於塗山,十損其七矣。成湯敗桀於焦,遷鼎于亳,伊摯、仲虺之徒,大明憲典。王者之制爵祿,公侯伯子男凡五等。天子之田方千里,公侯田方百里,伯七十里,子男五十里。不能五十里者,不達於天子,附於諸侯,曰附庸。凡四海之內九州,
 州方千里。州建百里之國三十,七十里之國六十,五十里之國百有二十,凡二百一十國。名山大澤不以封,其餘以為附庸間田。八州,州二百一十國。天子之縣內,百里之國九。七十里之國二十有一,五十里之國六十有三,凡九十三國。名山大澤不以班,其餘以祿士,以為間田。凡九州,千七百七十三國。天子之元士,諸侯之附庸,不與。天子百里之內以供官,千里之內以為御,千里之外設方伯。五國以為屬,屬有長;十國以為連,連有帥;三十國以為卒,卒有正;二百一十國以為州,州有伯。八州,八伯,五十六正,百六十八帥,三百三十六長。八伯各以
 其屬屬於天子之老二人,分天下為左右,曰二伯。千里之內曰甸,千里之外曰采,曰流。天子使其大夫為三監,監於方伯之國,國三人。天子之縣,內,諸侯祿也;外,諸侯嗣也。武王歸豐,監於二代,設爵惟五,分土惟三。封同姓五十餘國,周公、康叔建于魯衛,各數百里。太公封於齊,表東海者也。凡一千八百國,布列於五千里內。而太昊、黃帝之後,唐虞侯伯猶存。大司徒以諸公之地封疆方五百里,其食者半;諸侯之地方四百里,其食者參之一;諸伯之地方三百里,其食者參之一;諸子之地方二百里,其
 食者四之一;諸男之地方百里,其食者四之一。不易之地家百畝,一易之地家二百畝,再易之地家三百畝。五家為比,使之相保;五比為閭,使之相受;四閭為族,使之相葬;五族為黨,使之相救;五黨為州,使之相賙;五州為鄉,使之相賓。小司徒以五人為伍,五伍為兩,四兩為卒,五卒為旅,五旅為師,五師為軍。以起軍旅,以作田役,以比追胥,以令貢賦。乃經土地而井牧其田野,九夫為井,四井為邑,四邑為丘,四丘為甸,四甸為縣,四縣為都。遺人則十里有廬,廬有飲食。三十里有宿,宿有路室,路室有委。五十里有市,市有候,候有館,館有積。遂人則五家
 為鄰,五鄰為里,四里為酂,五酂為鄙,五鄙為縣,五縣為遂。大司馬以九畿之籍,施邦國之政。方千里曰國畿,其外方五百里曰侯畿,又其外方五百里曰甸畿,又其外方五百里曰男畿,又其外方五百里曰採畿,又其外方五百里曰衛畿,又其外方五百里曰蠻畿,又其外方五百里曰夷畿,又其外方五百里曰鎮畿,又其外方五百里曰籓畿。畿,田限也。自王城以外,面五千里為界,有分限者九也。于時治致太平,政稱刑措,民口千三百七十一萬四千九百三十三,蓋周之盛者也。其衰也,則禮樂征伐出自諸侯,彊吞弱而眾暴寡。春秋之初,尚有千二百國;迄獲麟之末,二百四
 十二年,弒君三十六,亡國五十二,諸侯奔走不得保其社稷者不可勝數,而見於《春秋》經傳者百有七十國焉。百三十九知其所居,魯、邾、鄭、宋、紀、衛、西、莒、齊、陳、杞、蔡、邢、郕、晉、薛、許、鄧、秦、曹、楚、隨、黃、梁、虞、鄖、小邾、徐、燕、鄀、麋、舒、庸、郯、萊、吳、越、有窮、三苗、瓜州、有虞、東、共、宿、申、夷、向、南燕、滕、凡、戴、息、郜、芮、魏、淳于、穀、巴、州、蓼、羅、賴、牟、葛、譚、蕭、遂、滑、權、鄣、霍、耿、江、冀、弦、道、柏、微、鄫、厲、項、密、任、須句、顓臾、頓、管、雍、畢、豐、邘、應、蔣、茅、胙、夔、介、焦、沈、六、巢、根牟、唐、黎、郇瑕、寒、有鬲、斟灌、斟尋、過、有過、戈、偪陽、邿、鑄、豕韋、唐杜、楊、豳、鄶、觀、扈、邳、胡、黎、大庭、駘、岐、邶、鍾吾、浦姑、昆吾、房、密須、甲父、鄅、桐、亳、韓、趙。三十一國盡亡其處,祭、極、荀、賈、貳、軫、絞、於餘丘、陽、箕、英氏、毛、聃、莘、偪、封父、仍、有仍、崇、鄟、庸、姺、奄、商奄、褒姒、蓐、有緡、闕鞏、飂、鬷、窮桑。蠻夷戎狄不在其間。五伯迭興,總其盟會。陵夷至于戰國,遂有七王,韓、魏、趙、燕、齊、秦、楚。又有宋、衛、中山,不斷如線,如三晉篡奪,亦稱孤
 也。



 《司馬法》廣陳三代,曰:古者六尺為步,步百為畝,畝百為夫,夫三為屋,屋三為井。井方一里,是為九夫,八家共之。一夫一婦受私田百畝,公田十畝,是為八百八十畝,餘二十畝為廬舍,出入相友,守望相助,疾病相救。民受田,上田夫百畝,中田夫二百畝,下田夫三百畝,歲受耕之,爰自其處。其家眾男為餘夫,亦以口受田如此。士工商家受田,五口乃當農夫一口。有賦有稅,稅謂公田什一及工商衡虞之入也,賦供車馬甲兵士從之役。民年二十受田,六十歸田。種穀必雜五種,以備災旱。田中不得有樹,以妨五穀。環廬種桑柘,菜茹有畦,瓜瓠果蓏植
 於疆埸,雞狗豕無失其時。閭有序,鄉有庠,序有明教,庠以行禮。司馬之法,官設六軍之眾,因井田而制軍。令地方一里為井,井十為通,通十為成,成方十里。成十為終,終十為同,同方百里。同十為封,封十為畿,畿方千里。故井四為邑,邑四為丘,丘十六井,有戎馬一區,牛三頭。四丘為甸,甸六十四井也,有戎馬四匹,兵車一乘,牛十二頭,甲士三人,卒七十二人。是謂乘車之制。一同百里,提封萬井,除山川、坑岸、城池、邑居、園囿、街路三千六百井,定出賦六千四百井,戎馬四百匹,兵車百乘,此卿大夫采地之大者也,是謂百乘之家。一封三百六十六里,
 提封十萬井,定出賦六萬四千井,戎馬四千匹,兵車千乘,此謂諸侯之大者也,謂之千乘之國。天子畿內方千里,提封百萬井,定出賦六十四萬井,戎馬四萬匹,兵車萬乘,戎卒七十二萬人,故天子稱萬乘之主焉。



 秦始皇既得志於天下,訪周之敗,以為處士橫議,諸侯尋戈,四夷交侵,以弱見奪,於是削去五等焉。漢興,創艾亡秦孤立而敗,於是割裂封疆,立爵二等,功臣侯者百有餘邑。于時民罹秦項,戶口凋弊,大侯不過萬家,小者五六百戶,而尊王子弟,大啟九國。古者有分土而無分民,若乃大者跨州連郡,小則十有餘城,以戶
 口為差降,略封疆之遠近,所謂分民自漢始也。起鴈門以東,盡遼陽,為燕代。常山以南,太行左轉,渡河濟,漸于海,為齊趙。穀泗以注,奄有龜蒙,為梁楚。東帶江湖,薄會稽,為荊吳。北界淮瀕,略廬衡,為淮南。波漢之陽,亙九疑,為長沙。諸侯比境,周匝三垂,外接胡越。天子自有三河、東郡、潁川、南陽,自江陵以不西至巴蜀,北至雲中,西至隴西,與京師內史,凡十五郡。文帝采賈生之議分齊趙,景帝用朝錯之計削吳楚。武帝施主父之冊,下推恩之令,使諸侯王得分戶邑以封子弟,不行黜陟,而籓國自析。自此以來,齊分為七,趙分為六,梁分為五,淮南分為三。
 皇子始立者大國不過十餘城,長沙、燕、代雖有舊名,皆亡南北邊矣。自文景與民休息,至平帝元始二年,民戶千二百二十三萬三千六十二,口五千九百五十九萬四千九百七十八,其地東西九千三百二里,南北萬三千三百六十八里。大率十里一亭,亭有長。十亭一鄉,鄉有三老,有秩嗇夫、游徼各一人。縣大率方百里,民稠則減,稀則曠,鄉、亭亦如之。皆秦制也。光武中興,不踰前制,東海王疆以去就有禮,故優以大封,兼食魯郡二十九縣,其餘稱為寵錫者,兼一郡而已。至桓帝永壽三年,戶千六十七萬七千九百六十,口五
 千六百四十八萬六千八百五十六,斯亦戶口之滋殖者也。獻帝建安元年拜曹操為鎮東將軍,封費亭侯。魏文帝黃初三年,初制封王之庶子為鄉公,嗣王之庶子為亭侯,公侯之庶子為亭伯。劉備章武元年,亦以郡國封建諸王,或遙採嘉名,不由檢土地所出。其戶二十萬九十萬。孫叔赤烏五年,亦取中州嘉號封建諸王。其戶五十二萬三千,男女口二百四十萬。晉文帝為晉王,命裴秀等建立五等之制,惟安平郡公
 孚邑萬戶,制度如魏諸王。其餘縣公邑千八百戶,地方七十五里;大國侯邑千六百戶,地方七十里;次國侯邑千四百戶,地方六十五里;大國伯邑千二百戶,地方六十里;次國伯邑千戶,地方五十五里;大國子邑八百戶,地方五十里;次國子邑六百戶,地方四十五里;男邑四百戶,地方四十里。武帝泰始元年,封諸王以郡為國。邑二萬戶為大國,置上下中下三軍,兵五千人;邑萬戶為次國,置上軍下軍,兵三千人;五千戶為小國,置一軍,兵千五百人。王不之國,官於京師。罷五等之制,公侯邑萬戶以上為大國,五千
 戶以上為次國,不滿五千戶為小國。太康元年,平吳,大凡戶二百四十五萬九千八百四十,口一千六百一十六萬三千八百六十三。而江左諸國並三分食一,元帝渡江,太興元年,始制九分食一。



 司州。案《禹貢》豫州之地。及漢武帝,初置司隸校尉,所部三輔、三河諸郡。其界西得雍州之京兆、馮翊、扶風三郡,北得冀州分河東、河內二郡,東得豫州之弘農、河南二郡,郡凡七。位望降于牧伯,銀印青綬。及光武都洛陽,司隸所部與前漢不異。魏氏受禪,即都漢宮,司隸所部河南、河東、河內、弘農並冀州之平陽,合五郡,置司州。晉仍
 居魏都,乃以三輔還屬雍州,分河南立滎陽,分雍州之京兆立上洛,廢東郡立頓丘,遂定名司州,以司隸校尉統之。州統郡一十二,縣一百,戶四十七萬五千七百。



 河南郡漢置。統縣十二,戶一十一萬四千四百。置尹。



 洛陽置尉。五部、三市。東西七里,南北九里。東有建春、東陽、清明三門,南有開陽、平昌、宣陽、建陽四門,西有廣陽、西明、閶闔三門,北有大夏、廣莫二門。司隸校尉、河南尹及百官列城內也。河南周東都王城郟鄏也。鞏周孝王封周桓公孫惠公於鞏,號東周,故戰國時有東、西周號。芒山、首陽其界也。河陰新安函谷關所居。成皋有關,鄭之武牢。緱氏有劉聚,周大夫劉子邑。有延壽城、仙人祠。陽城有鄂阪關。此邑是為地中,夏至景尺五寸。有陽城山、箕山,許由墓在焉。新城有延壽關。故戎蠻子之國。陸渾故蠻子國,楚壯王伐陸渾是也。梁戰國時謂為南梁,別少梁也。陽翟



 滎陽郡泰始二年置。統縣八,戶三萬四千。



 滎陽地名敖,秦置敖倉者。京鄭太叔段所居。密故周畿內。卷有博浪長沙,張良擊秦始皇處。陽武苑陵中牟六國時,趙獻侯都。開封宋蓬池在東北,或曰蓬澤。



 弘農郡漢置。統縣六,戶一萬四千。



 弘農本函谷關。漢武帝遷於新安縣。湖故曰胡,漢武更名湖。陜故虢國,周分陜東西,二相主之。宜陽黽池華陰華山在縣南。



 上洛郡泰始二年,分京兆南郡置。統縣三,戶萬七千。



 上洛嶢關在縣西北。商秦相衛商鞅邑。廬氏熊耳山在東,伊水所出。



 平陽郡故屬河東,魏分立。統縣十二,戶四萬二千。



 平陽舊堯都。侯國。楊故楊侯國。端氏韓、魏。趙既為諸侯,以端氏封晉君也。永安故霍伯國。
 霍山在東。浦子狐讘襄陵公國相。絳邑晉武公自曲沃徙此。濩澤析城山在西南。臨汾公國相。北屈壺口山在東南。有南屈,故稱北。皮氏故耿國。



 河東郡秦置。統縣九,戶四萬二千五百。



 安邑舊舜都。聞喜故曲沃。晉武公自晉陽徙此。垣王屋山在東北,沇水所出。汾陽公國相。大陽吳山在西。



 周武王封西周太伯後於此猗氏古猗頓城。解有鹽池。蒲阪有歷山,舜所耕也。有雷首山,夷齊居其陽,所謂首陽山。河北



 汲郡泰始二年置。統縣六,戶三萬七千。



 汲有銅關。朝歌紂所都。共故國。北山,淇水所出。林慮



 獲嘉故汲新中鄉。漢武帝行過時,獲呂嘉首,因改名。脩武晉所啟南陽,秦改名修武。



 河內郡漢置。統縣九,戶五
 萬二千。



 野王太行山在西北。州故晉邑。懷平皋邢侯自襄國徙此。河陽



 沁水軹故周原邑。山陽溫故國也,蘇忿生封。



 廣平郡魏置。統縣十五,戶三萬五千二百。



 廣平邯鄲秦置為郡。易陽武安涉襄國故邢侯國都。南和任曲梁列人肥鄉臨水廣年侯相。斥漳平恩



 陽平郡魏置。統縣七,戶五萬一千。



 元城漢元后生邑。館陶清泉發干東武陽陽平樂平



 魏郡漢置。統縣
 八,戶四萬七百。



 鄴魏武受封居此。長樂魏斥丘安陽蕩陰內黃黃池在西。黎陽故黎侯國。



 頓丘郡泰始二年置。統縣四,戶六千三百。



 頓丘繁陽陰安衛



 永嘉之後,司州淪沒劉聰。聰以洛陽為荊州,及石勒,復以為司州。石季龍又分司州之河南、河東、弘農、滎陽,兗州之陳留、東燕為洛州。元帝渡江,亦僑置司州於徐,非本所也。後以弘農人流寓尋陽者僑立為弘農郡。又以河東人南寓者,於漢武陵郡孱陵縣界上明地僑立河東郡,統安邑、聞喜、永安、臨汾、弘農、譙、松滋、大戚八縣。並
 寄居焉。永和五年,桓溫入洛,復置河南郡,屬司州。



 兗州。案《禹貢》濟河之地,舜置十二牧,則其一也。《周禮》:「河東曰兗州。」《春秋元命包》云命:「五星流為兗州。兗,瑞也,信也。」又云:「蓋取兗水以名焉。」漢武帝置十三州,以舊名為兗州,自此不改。州統郡國八,縣五十六,戶八萬三千三百。



 陳留國漢置。統縣十,戶三萬。魏武帝封。



 小黃浚儀有洪溝,漢高祖項羽欲分處。封丘酸棗烏巢地在東南。濟陽長垣故匡城,孔子所厄也。雍丘故杞國。尉氏襄邑外黃



 濮陽國故屬東郡,晉初分東郡置。統縣四,戶二萬一千。



 濮陽古昆吾國。師延為紂作靡靡之樂,即而投此水。公國相。廩丘公國相。有羊角城。白馬有瓠子堤。鄄城公國相。



 濟陰郡漢置。統縣九,戶七千六百。



 定陶漢高祖封彭越為梁王,都此。乘氏故侯國。句陽離狐宛句己氏成武有楚丘亭。單父故侯國。城陽舜所漁,堯塚在西。



 高平國。故屬梁國,晉初分山陽置。統縣七,戶三千八百。



 昌邑侯相。有甲父亭。鉅野魯獲麟所。方與金鄉湖陸高平侯國。南平陽侯國。有漆亭。



 任城國漢置。統縣三,戶一千七百。



 任城古任國。亢父樊



 東平國漢置。統縣七,戶六千四百。



 須昌壽張有蚩尤祠。范無鹽富城東平陸剛平



 濟北國漢置。統縣五,戶三千五百。



 廬扁鵲所生。縣西有石門。臨邑東阿穀城有烏下聚。蛇丘有下灌亭。



 泰山郡漢置。統縣十一,戶九千三百。



 奉高西南有明堂。博有龜山。贏南武城梁父侯國。有菟裘聚。山茌茌山在東北。新泰故曰平陽。南武陽有顓臾城。萊蕪有原山。東牟故牟國。鉅平有陽關亭。



 惠帝之末,兗州闔境淪沒石勒。後石季龍改陳留郡為建昌郡,屬洛州。是時遺黎南渡,元帝僑置兗州,寄居京口。明帝以郗鑒為刺史,寄居廣陵,置濮陽、濟陰、高平、太
 山等郡。後改為南兗州,或還江南,或居盱眙,或居山陽。後始割地為境,常居廣陵,南與京口對岸。咸康四年,於北譙界立陳留郡。安帝分廣陵郡之建陵、臨江、如皋、寧海、蒲濤五縣置山陽郡,屬南兗州。



 豫州。案《禹貢》為荊河之地。《周禮》:「河南曰豫州。」豫者舒也,言稟中和之氣,性理安舒也。《春秋元命包》云:「鉤鈐星別為豫州。」地界,西自華山,東至于淮,北自濟,南界荊山。秦兼天下,以為三川、河東、南陽、潁川、碭、泗水、薛七郡。漢改三川為河南郡,武帝置十三州,豫州舊名不改,以河南、河東二郡屬司隸,又以南陽屬荊州。先是,改泗水曰
 沛郡,改碭郡曰梁,改薛曰魯,分梁沛立汝南郡,分潁川立淮陽郡。後漢章帝改淮陽曰陳郡。魏武分沛立譙郡,魏文分汝南立弋陽郡。及武帝受命,又分潁川立襄城都,分汝南立汝陰郡,合陳郡於梁國。州統郡國十,縣八十五,戶十一萬六千七百九十六。



 潁川郡秦置。統縣九,戶二萬八千三百。



 許昌漢獻帝都許。魏禪,徙都洛陽,許宮室武庫存焉,改為許昌。長社



 潁陰臨潁公國相。郾邵陵公國相。鄢陵公國相。新汲



 長平



 汝南郡漢置。統縣十五,戶二萬一千五百。



 新息南安陽安成侯相。慎陽北宜春朗陵陽
 安故江國。有江亭。上蔡平輿故沈子國。有沈亭。定潁灈陽南頓汝陽吳房故房子國。西平故柏國。有龍泉,水可用淬刀劍。



 襄城郡泰始二年置。統縣七,戶一萬八千。



 襄城侯相。有西不羹城。繁昌魏文受禪於此。郟定陵侯相。父城侯相。昆陽公國相。舞陽宣帝始封此邑。



 汝陰郡魏置郡,後廢,泰始二年復置。統縣八,戶八千五百。



 汝陰故胡子國。慎故楚邑。原鹿固始鮦陽新蔡宋侯相褒信



 梁國漢置。統縣十二,戶一萬三千。



 睢陽春秋時宋都。蒙虞下邑有陽山,山有文石。寧陵故葛伯國。穀熟
 陳項長平陽夏武平苦東有賴鄉祠,老子所生地。



 沛國漢置。統縣九,戶五千九十六。



 相沛漢高祖所起處。豐竺邑符離杼秋汶虹蕭



 譙郡魏置。統縣七,戶一千。



 譙城父酂山桑龍亢蘄銍



 魯郡漢置。統縣七,戶三千五百。



 魯曲阜之地,魯侯伯禽所居。汶陽卞鄒有繹山。番故小邾之國。薛奚仲所封。公丘



 弋陽郡魏置。統縣七,戶一萬六千七百。



 西陽故弦子國。軑蘄春邾西陵期思弋陽



 安豐郡魏置。統縣五,戶一千二百。



 安風雩婁安豐侯相。蓼松滋侯相。



 惠帝分汝陰立新蔡,分梁國立陳郡,分汝南立南頓。永嘉之亂,豫州淪沒石氏。元帝渡江,以春穀縣僑立襄城郡及繁昌縣。成帝乃僑立豫州於江淮之間,居蕪湖。時淮南入北,乃分丹陽僑立淮南郡,居于湖。又以舊當塗縣流人渡江,僑立為縣,并淮南、廬江、安豐並屬豫州。寧康元年,移鎮姑孰。孝武改蘄春縣為蘄陽縣,因新蔡縣人於漢九江王黥布舊城置南新蔡郡,屬南豫州。又於
 漢廬江郡之南部置晉熙郡。



 冀州。案《禹貢》、《周禮》並為河內之地,舜置十二牧,則其一也。《春秋元命包》云:「昴畢散為冀州,分為趙國。」其地有險有易,帝王所都,則冀安,弱則冀彊,荒則冀豐。舜以冀州南北闊大,分衛以西為並州,燕以北為幽州,周人因焉。及漢武置十三州,以其地依舊名焉冀州,歷後漢至晉不改。州統郡國十三,縣八十三,戶三十二萬六千。



 趙國。漢置。統縣九,戶四萬二千。



 房子元氏平棘高邑。公國相。中丘柏人平鄉下曲陽故鼓子國。鄔



 鉅鹿國秦置。統縣二,戶一萬四十。



 黽陶鉅鹿



 安平國漢置。統縣八,戶二萬一千。



 信都下博武邑武遂觀津侯相。扶柳廣宗侯國。



 經



 平原國漢置。統縣五。戶五萬一千。



 平原高唐茌平博平聊城安德西平昌般鬲



 樂陵國漢置。統縣五,戶三萬三千。



 厭次陽信漯沃新樂樂陵有
 都尉居。



 勃海郡漢置。統縣十,戶四萬。



 南皮東光浮陽饒安高城重合東安陵脩廣川侯相。阜城



 章武國泰始元年置。統縣四,戶一萬三千。



 東平舒文安章武束州



 河間國漢置。統縣六,戶二萬七千。



 樂城侯相。武垣鄚侯相。易城中水成平



 高陽國泰始元年置。統縣四,戶七千。



 博陸高陽北新城侯相蠡吾



 博陵郡漢置。統縣四,戶一
 萬。



 安平饒陽南深澤安國



 清河國漢置。統縣六,戶二萬二千。



 清河東武城繹幕侯相。貝丘靈



 中山國漢置。統縣八,戶三萬三千。



 盧奴魏昌新市安喜蒲陰望都唐北平



 常山郡漢置。統縣八,戶二萬四千。



 真定石邑井陘上曲陽恆山在縣西北,有阪號飛狐口。蒲吾



 南行唐靈壽九門侯相。



 惠帝之後,冀州淪沒於石勒。勒以太興二年僭號於襄
 國,稱趙。後為慕容俊所滅,慕容氏又為苻堅所滅。孝武太元八年,堅敗,其地入慕容垂。垂僭號於中山,是為後燕。後燕卒滅於魏。



 幽州。案《禹貢》冀州之域,舜置十二牧,則其一也。《周禮》「東北曰幽州。」《春秋元命包》云:「箕星散為幽州,分為燕國。」言北方太陰,故以幽冥為號。武王定殷,封召公於燕,其後與六國俱稱王。及秦滅燕,以為漁陽、上谷、右北平、遼西、遼東五郡。漢高祖分上谷置涿郡。武帝置十三州,幽州依舊名不改。其後開東邊,置玄菟、樂浪等郡,亦皆屬焉。元鳳元年,改燕曰廣陽郡。幽州所部凡九郡,至晉不
 改。幽州統都國七,縣三十四,戶五萬九千二十。



 范陽國漢置涿郡。魏文更名範陽郡。武帝置國,封宣帝弟子綏為王。統縣八,戶一萬一千。



 涿良鄉方城長鄉遒故安范陽容城侯相。



 燕國漢置,孝昭改為廣陽郡。統縣十,戶二萬九千。



 薊安次侯相。昌平軍都有關。廣陽潞安樂國相。蜀主劉禪封此縣公。泉州侯相。雍奴狐奴



 北平郡秦置。統縣四,戶五千。



 徐無土垠俊靡無終



 上谷郡秦置,郡在谷之上頭,故因名焉。統縣二,戶四千七十。



 沮陽居庸



 廣寧郡故屬上谷,太康中置郡,都尉居。統縣三,戶三千九百五十。



 下洛潘涿鹿



 代郡。秦置。統縣四,戶三千四百。



 代廣昌平舒當城



 遼西郡秦置。統縣三,戶二千八百。



 陽樂肥如海陽



 惠帝之後,幽州沒於石勒。及穆帝永和五年,慕容俊僭早於薊,是為前燕。七年,俊移都於鄴。俊死,子為苻堅所滅。堅敗,地復入慕容垂,是為後燕。垂死,寶遷于和龍。



 平州。案《禹貢》冀州之域,於周為幽州界,漢屬右北平郡。後漢末,公孫度自號平州牧。及其子康、康子文懿並擅據遼東,東夷九種皆服事焉。魏置東夷校尉,居襄平,而分遼東、昌黎、玄菟、帶方、樂浪五郡為平州,後還合為幽州。及文懿滅後,有護東夷校尉,居襄平。咸寧二年十月,分昌黎、遼東、玄菟、帶方、樂浪等郡國五置平州。統縣二十六,戶一萬八千一百。



 昌黎郡漢屬遼東屬國都尉,魏置郡。統縣二,戶九百。



 昌黎賓徒



 遼東國秦立為郡。漢光武以遼東等屬青州,後還幽州。統縣八,戶五千四百。



 襄平東夷校尉所居。汶居就樂就安市西安平新昌



 力城



 樂浪郡漢置。統縣六,戶三千七百。



 朝鮮周封箕子地。屯有渾彌遂城秦築長城之所起。鏤方



 駟望



 玄菟郡漢置。統縣三,戶三千二百。



 高句麗望平高顯



 帶方郡公孫度置。統縣七,戶四千九百。



 帶方列口南新長岑提奚含資海冥



 平州初置,以慕容廆為刺史,遂屬永嘉之亂,廆為眾所
 推。及其孫俊移都于薊。其後慕容垂子寶又遷于和龍,自幽州至子廬溥鎮以南地入於魏。慕容熙以幽州刺史鎮令支,青州刺史鎮新城,並州刺史鎮凡城,營州刺史鎮宿軍,冀州刺史鎮肥如。高雲以幽、冀二州牧鎮肥如,並州刺史鎮白狼。後為馮跋所篡,跋僭號於和龍,是為後燕,卒滅於魏。



 並州。案《禹貢》蓋冀州之域,舜置十二牧,則其一也。《周禮》:正北曰并州,其鎮曰恒山。《春秋元命包》云:「營室流為並州,分為衛國。」州不以衛水為號,又不以恒山為稱,而云並者,蓋以其在兩谷之間也。漢武帝置十三州,並州
 依舊名不改,統上黨、太原、雲中、上郡、鴈門、代郡、定襄、五原、西河、朔方十郡,又別置朔方刺史。後漢建武十一年,省朔方入並州。靈帝末,羌胡大擾,定襄、雲中、五原、朔方、上郡等五郡並流徙分散。建安十八年,省入冀州。二十年,始集塞下荒地立新興郡,後又分上黨立樂平郡。魏黃初元年,復置并州,自陘嶺以北並棄之,至晉因而不改。並州統郡國六,縣四十五,戶五萬九千二百。



 太原國秦置。統縣十三,戶一萬四千。



 晉陽侯相。陽曲榆次於離盂狼孟陽邑大陵祁平陶京陵中都鄔



 上黨郡秦置。統縣十,戶一萬三千。



 潞屯留壺關長子泫氏高都銅鞮涅襄垣武鄉



 西河國漢置。統縣四,戶六千三百。



 離石隰城中陽介休



 樂平郡泰始中置。統縣五,戶四千三百。



 沾上艾壽陽尞陽樂平



 雁門郡秦置。統縣八,戶一萬二千七百。



 廣武崞水枉陶平城葰人繁畤原平馬邑



 新興郡魏置。統縣五,戶九千。



 九原定襄雲中廣牧晉昌



 惠帝改新興為晉昌郡。及永興元年,劉元海僭號於平陽,稱漢,於是並州之地皆為元海所有。元海乃以雍州刺史鎮平陽,幽州刺史鎮離石。及劉聰攻陷洛陽,置左右司隸,各領戶二十餘萬,萬戶置一內史,凡內史四十三人,單于左右輔各主六夷。又置殷、衛、東梁、西河陽、北兗五州,以懷安新附。劉曜徙都長安,其平陽以東地入石勒。勒平朔方,又置朔州。自惠懷之間,離石縣荒廢,勒於其處置永石郡,又別置武鄉郡。及苻堅、姚興、赫連
 勃勃,并州並徙置河東,又姚興以河東為並、冀二州云。



 雍州。案《禹貢》黑水、西河之地,舜置十二牧,則其一也。以其四山之地,故以雍名焉。亦謂西北之位,陽所不及,陰氣雍閼也。《周禮》:西曰雍州。蓋並禹梁州之地。周自武王剋殷,都於酆鎬,雍州為王畿。及平王東遷洛邑,以岐酆之地賜秦襄公,則為秦地,累世都之,至始皇遂平六國。秦滅,漢又都之。及武帝置十三州,其地以西偏為涼州,其餘並屬司隸,不統於州。後漢光武都洛陽,關中復置雍州。後罷,復置司隸校尉,統三輔如舊。獻帝時又置雍州,自三輔距西域皆屬焉。魏文帝即位,分河西為
 涼州,分隴右為秦州,改京兆尹為太守,馮翊、扶風各除左右,仍以三輔屬司隸。晉初於長安置雍州,統郡國七,縣三十九,戶九萬九千五百。



 京兆郡漢置。統縣九,戶四萬。



 長安杜陵霸城藍田高陸萬年故櫟陽縣。新豐陰般鄭周宣王弟鄭桓公邑。



 馮翊郡漢置,名左馮翊。統縣八,戶七千七百。



 臨晉。故大荔,秦獲之,更名。有河水祠,祠臨晉水,故名。下邽秦武公伐邽戎,置有上邽,故加下。重泉頻陽秦厲公置,在頻水之陽。粟邑蓮芍郃陽夏陽故少梁,秦惠文王更名。梁山在西北。



 扶風郡漢武帝以為主爵都尉,太初中更名右扶風。統縣六,戶二萬三千。


池陽漢惠帝置。有嶻
 \gezhu{
  山闢}
 山。郿成國渠首受渭。雍侯相。有五畤、太昊、黃帝以下祠三百三所。汧吳山在西,古文以為汧山。陳倉



 美陽岐山在西北,周太王所邑。



 安定郡漢置。統縣七,戶五千五百。



 臨涇朝那烏氏都盧鶉觚陰密殷時密國。西川



 北地郡秦置。統縣二,戶二千六百。



 泥陽富平



 始平郡泰始二年置。統縣五,戶一萬八千。



 槐里秦曰廢丘,漢高帝更名。有黃山宮。始平武功太一山在東,古文以為終南。鄠古國,夏啟所伐。蒯城



 新平郡漢置。統縣二,戶二千七百。



 漆漆水在西。汾邑



 惠帝即位,改扶風國為秦國。徙都。建興之後,雍州沒於劉聰。及劉曜徙都長安,改號曰趙,以秦、涼二州牧鎮上邽,朔州牧鎮高平,幽州刺史鎮北地,并州牧鎮蒲阪。石勒剋長安,復置雍州。石氏既敗,苻健僭據關中,又都長安,是為前秦。於是乃於雍州置司隸校尉,以豫州刺史鎮許昌,秦州刺史鎮上邽,荊州刺史鎮豐陽,洛州刺史鎮宜陽,並州刺史鎮蒲阪。苻堅時,分司隸為雍州,分京兆為咸陽郡,洛州刺史鎮陜城。滅燕之後,分幽州置平
 州,鎮龍城,幽州刺史鎮薊城,河州刺史鎮桴罕,並州刺史鎮晉陽,豫州刺史鎮洛陽,兗州刺史鎮倉垣,雍州刺史鎮蒲阪。於是移洛州居豐陽,以許昌置東豫州,以荊州刺史鎮襄陽,徐州刺史鎮彭城。即而姚萇滅苻氏,是為後秦。及萇子興剋洛陽,以並、冀二州牧鎮蒲阪,豫州牧鎮洛陽,兗州刺史鎮倉垣,分司隸領北五郡,置幽州刺史鎮安定。及姚泓為劉裕所滅,其地尋入赫連勃勃。勃勃僭號於統萬,是為夏。置幽州牧於大城,又平劉義真於長安,遣子璝鎮焉,號曰南臺。以朔州牧鎮三城,秦州刺史鎮杏城,雍州刺史鎮陰密,並州刺史鎮蒲阪,梁
 州牧鎮安定,北秦州刺史鎮武功,豫州牧鎮李閏,荊州刺史鎮陜,其州郡之名並不可知也。然自元帝渡江,所置州亦皆遙領。初以魏該為雍州刺史,鎮酂城,尋省,僑立始平郡,寄居武當城。有秦國流人至江南,改堂邑為秦郡,僑立尉氏縣屬焉。康帝時,庾翼為荊州刺史,遷鎮襄陽。其後秦雍流人多南出樊沔,孝武始於襄陽僑立雍州,仍立京兆、始平、扶風、河南、廣平、義成、北河南七郡,並屬襄陽。襄陽故屬荊州。



 涼州。案《禹貢》雍州之西界,周衰,其地為狄。秦興美陽甘泉宮,本匈奴鑄金人祭天之處。匈奴既失甘泉,又使
 休屠、渾邪王等居涼州之地。二王後以地降漢,漢置張掖、酒泉、敦煌、武威郡。其後又置金城郡,謂之河西五郡。漢改周之雍州為涼州,蓋以地處西方,常寒涼也。地勢西北邪出,在南山之間,南隔西羌,西通西域,于時號為斷匈奴右臂。獻帝時,涼州數有亂,河西五郡去州隔遠,於是乃別以為雍州。末又依古典定九州,乃合關右以為雍州。魏時復分以為涼州,刺史領戊己校尉,護西域,如漢故事,至晉不改。統郡八,縣四十六,戶三萬七百。



 金城郡漢置。統縣五,戶二千。



 榆中允街金城白土浩亹



 西
 平郡漢置。統縣四,戶四千



 西都臨羌長寧安夷



 武威郡漢置。統縣七,戶五千九百。



 姑臧宣威揖次倉松顯美驪靬番禾



 張掖郡漢置。統縣三,戶三千七百。



 永平臨澤漢昭武縣,避文帝諱改也。屋蘭漢因屋蘭名焉。



 西郡漢置。統縣五,戶一千九百。



 日勒刪丹仙提萬歲蘭池一云蘭絕池。



 酒泉郡漢置。統縣九,戶四千四百。



 福祿會水安彌騂馬樂涫表氏延壽
 玉門沙頭



 敦煌郡漢置。統縣十二,戶六千三百。



 昌蒲敦煌龍勒陽關效穀廣至宜禾宜安深泉伊吾新鄉乾齊



 西海郡故屬張掖,漢獻帝興平二年,武威太守張雅請置。統縣一,戶二千五百。



 居延澤在東南,《尚書》所謂流沙也。



 元康五年,惠帝分敦煌郡之宜禾、伊吾、宜安、深泉、廣至等五縣,分酒泉之沙頭縣,又別立會稽、新鄉,凡八縣為晉昌郡。永寧中,張軌為涼州刺史,鎮武威,上表請合秦雍流移人於姑臧西北,置武興郡,統武興、大城、烏支、襄
 武、晏然、新鄣、平狄、司監等縣。又分西平界置晉興郡,統晉興、枹罕、永固、臨津、臨鄣、廣昌、大夏、遂興、罕唐、左南等縣。是時中原淪沒,元帝徙居江左,軌乃控據河西,稱晉正朔,是為前涼。及張寔,分金城之令居、枝陽二縣,又立永登縣,合三縣立廣武郡。張茂分武興、金城、西平、安故為定州。張駿分武威、武興、西平、張掖、酒泉、建康、西海、西郡、湟河、晉興、廣武合十一郡為涼州,興晉、金城、武始、南安、永晉、大夏、武成、漢中為河州,敦煌、晉昌、高昌、西域都護、戊己校尉、玉門大護軍三郡三營為沙州。張駿假涼州都督,攝三州。張祚又以敦煌郡為商州。永興中,置漢陽縣以
 守牧地,張玄靚改為祁連郡。張天錫又別置臨松郡。天錫降於苻氏,其地尋為呂光所據。呂光都於姑臧後,以郭黁言讖,改昌松為東張掖郡。及呂隆降於姚興,其地三分。武昭王為西涼,建號於敦煌,禿髮烏孤為南涼,建號於樂都。。沮渠蒙遜為北涼,建號於張掖。而分據河西五郡。



 秦州。案《禹貢》本雍州之域,魏始分隴右置焉,刺史領護羌校尉,中間暫廢。及泰始五年,又以雍州隴右五郡及涼州之金城、梁州之陰平,合七郡置秦州,鎮冀城。太康三年,罷秦州,并雍州。七年,復立,鎮上邽。統郡六,縣二
 十四,戶三萬二千一百。



 隴西郡秦置。統縣四,戶三千。



 襄武首陽烏鼠山在東。臨洮狄道



 南安郡漢置。統縣三,戶四千三百。



 獂道新興中陶



 天水郡漢武置,孝明改為漢陽,晉復為天水。統縣六,戶八千五百。



 上邽冀秦州故居。始昌新陽顯新漢顯親縣。成紀



 略陽郡本名廣魏,泰始中更名焉。統縣四,戶九千三百二十。



 臨渭平襄略陽清水



 武都郡漢置。統縣五,戶三千。



 下辯河池沮武都故道



 陰平郡泰始中置。統縣二,戶三千。



 陰平平廣



 惠帝分隴西之狄道、臨洮、河關,又立洮陽、遂平、武街、始興、第五、真仇六縣,合九縣置狄道郡,屬秦州。張駿分屬涼州,又以狄道縣立武始郡。江左分梁為秦,寄居梁州,又立氐池為北秦州。



 梁州。案《禹貢》華陽黑水之地,舜置十二牧,則其一也。梁者,言西方金剛之氣彊梁,故因名焉。《周禮》職方氏以梁並雍。漢不立州名,以其地為益州。及獻帝初平六年,
 以臨江縣屬永寧郡。建安六年,劉璋改永寧為巴東郡,分巴郡墊江置巴西郡。劉備據蜀,又分廣漢之葭萌,涪城、梓潼、白水四縣,改葭萌曰漢壽,又立漢德縣,以為梓潼郡;割巴郡之宕渠、宣漢、漢昌三縣宕渠郡,尋省,以縣並屬巴西郡。泰始三年,分益州,立梁州於漢中,改漢壽為晉壽,又分廣漢置新都郡。梁州統郡八,縣四十四,戶七萬六千三百。



 漢中郡秦置,統縣八,戶一萬五千。



 南鄭蒲池褒中沔陽成固西鄉黃金興道



 梓潼郡蜀置。統縣八,戶一萬二百。



 梓潼涪城武連黃安漢德晉壽劍閣白水



 廣漢郡漢置。統縣三,戶五千一百。



 廣漢德陽五城



 新都郡泰始二年置。統縣四,戶二萬四千五百。



 雒什方綿竹新都



 涪陵郡蜀置。統縣五,戶四千二百。



 漢復涪陵漢平漢葭萬寧



 巴郡秦置。統縣四,戶三
 千三百。



 江州墊江臨江枳



 巴西郡蜀置。統縣九,戶一萬二千。



 閬中西充國蒼溪岐愜南充國漢昌宕渠安漢



 平州



 巴東郡漢置。統縣三,戶六千五百。



 魚復朐南浦



 太康六年九月,罷新都郡並廣漢郡。惠帝復分巴西置宕渠郡,統宕渠、漢昌,宣漢三縣,並以新城、魏興、上庸合四郡以屬梁州。尋而梁州郡縣沒于李特,永嘉中又分屬楊茂搜,其晉人流寓於梁益者,仍於二州立南北二
 陰平郡。及桓溫平蜀之後,以巴漢流人立晉昌郡,領長樂、安晉、延壽、安樂、宣漢、寧都、新興、吉陽、東關、永安十縣;又置益昌、晉興二縣,屬巴西郡;於德陽界東南置遂寧郡;又於晉壽置劍閣縣,屬梁州。後孝武分梓潼北界立晉壽郡,統晉壽、白水、邵歡、興安四縣;梓潼郡徙居梓潼,罷劍閣縣;又別置南漢中郡,分巴西、梓潼為金山郡。及安帝時,又立新巴、汶陽二郡,又有北新巴、華陽、南陰平、北陰平四郡,其後又立巴渠、懷安、宋熙、白水、上洛、北上洛、南宕渠、懷漢、新興、安康等十郡。



 益州。案《禹貢》及舜十二牧俱為梁州之域,周合梁於
 雍,則又為雍州之地。《春秋元命包》云:「參伐流為益州,益之為言阨也。」言其所在之地險阨也,亦曰疆壤益大,故以名焉。始秦惠王滅蜀,置郡,以張若為蜀守。及始皇置三十六郡,蜀郡之名不改。漢初有漢中、巴蜀。高祖六年,分蜀置廣漢,凡為四郡。武帝開西南夷,更置犍為、牂柯、越巂、益州四郡,凡八郡,遂置益州統焉,益州始此也。及後漢,明帝以新附置永昌郡,安帝又以諸道置蜀、廣漢、犍為三郡屬國都尉,及靈帝又以汶江、蠶陵、廣柔三縣立汶山郡。獻帝初平元年,劉璋分巴郡立永寧郡。建安六年,改永寧為巴東,以巴郡為巴西,又立涪陵郡。二十一
 年,劉備分巴郡立固陵郡。蜀章武元年又改固陵為巴東郡,巴西郡為巴郡,又分廣漢立梓潼郡,分犍為立江陽郡,以蜀郡屬國為漢嘉郡,以犍為屬國為朱提郡。劉禪建興二年,改益州郡為建寧郡,廣漢屬國為陰平郡,分建寧永昌立雲南郡,分建寧柯立興古郡,分廣漢立東廣漢郡。魏景元中,蜀平,省東廣漢郡。及武帝泰始二年,分益州置梁州,以漢中屬焉。七年,又分益州置寧州。益州統郡八,縣四十四,戶十四萬九千三百。



 蜀郡秦置。統縣六,戶五萬。



 成都廣都繁江原臨邛郫



 犍為郡漢置。統縣五,戶一萬。



 武陽南安僰道資中牛鞞



 汶山郡漢置。統縣八,戶一萬六千。



 汶山升遷都安廣陽興樂平康蠶陵廣柔



 漢嘉郡蜀置。統縣四,戶一萬三千。



 漢嘉徙陽嚴道旄牛



 江陽郡蜀置。統縣三,戶三千一百。



 江陽符漢安



 朱提郡蜀置。統縣五,戶二千六百。



 朱提南廣漢陽南秦堂狼



 越巂郡漢置。統縣五,戶五萬三千四百。



 會無邛都卑水定苲臺登



 牂柯郡漢置。統縣八,戶一千二百。



 萬壽且蘭談指夜郎毋斂並渠鄨平夷



 惠帝之後,李特僭號於蜀,稱漢,益州郡縣皆沒于特。李雄又分漢嘉、蜀二郡立沈黎、漢原二郡。是時益州郡縣雖沒李氏,江左並遙置之。桓溫滅蜀,其地復為晉有,省漢原、沈黎而立南陰平、晉原、寧蜀、始寧四郡焉。咸安二
 年,益州復沒於苻氏。太元八年,復為晉有。隆安二年,又立晉熙、遂寧、晉寧三郡云。



 寧州。於漢魏為益州之域。泰始七年,武帝以益州地廣,分益州之建寧、興古、雲南,交州之永昌,合四郡為寧州,統縣四十五,戶八萬三千。



 雲南郡蜀置。統縣九,戶九千二百。



 雲平雲南梇棟青蛉姑復邪龍榆遂久永寧



 興古郡蜀置。統縣十一,戶六千二百。



 律高句町宛溫漏臥毋掇賁古滕休
 鐔封漢興進乘都篖



 建寧郡蜀置。統縣十七,戶二萬九千。



 味昆澤存邑新定談槁母單同瀨漏江牧麻穀昌連然秦臧雙柏



 俞元脩雲泠丘滇池



 永昌郡漢置。統縣八,戶三萬八千。



 不韋永壽比蘇雍鄉南涪巂唐哀牢博南



 太康三年,武帝又廢寧州入益州,立南夷校尉以護之。太安二年,惠帝復置寧州,又分建寧以西七縣別立為
 益州郡。永嘉二年,改益州郡曰晉寧,分牂柯立平夷、夜郎二郡,然是時其地再為李特所有。其後李壽分寧州興古、永昌、雲南、硃提、越巂、河陽六郡為漢州。咸康四年,分牂柯、夜郎、硃提、越巂四郡置安州。八年,又罷並寧州。以越巂還屬益州,省永昌郡焉。



\end{pinyinscope}