\article{載記第十一 慕容瑋慕容恪陽騖皇甫真}

\begin{pinyinscope}

 慕容瑋慕容恪陽騖皇甫真



 慕容瑋,字景茂,俊第三子也。初封中山王,尋立為太子。及俊死,群臣欲立慕容恪,恪辭曰:「國有儲君,非吾節也。」於是立瑋。升平四年,僭即皇帝位,大赦境內,改元曰建熙,立其母可足渾氏為皇太后。以慕容恪為太宰、錄尚書,行周公事;慕容評為太傅,副贊朝政;慕輿根為太師;慕容垂為河南大都督、征南將軍、兗州牧、荊州刺史,領
 護南蠻校尉,鎮梁國;孫希為安西將軍、并州刺史;傅顏為護軍將軍;其餘封授各有差。



 瑋既庸弱,國事緣委之於恪。慕輿根自恃勛舊,驕傲有無上之心,忌恪之總朝權,將伺隙為亂,乃言於恪曰:「今主上幼沖,母后干政,殿下宜慮楊駿、諸葛元遜之變,思有以自全。且定天下者,殿下之功也,兄亡弟及,先王之成制,過山陵之後,可廢主上為一國王,殿下踐尊位,以建大燕無窮之慶。」恪曰:「公醉乎?何言之勃也!昔曹臧、吳札並於家難之際,猶曰為君非吾節,況今儲君嗣統,四海無虞,宰輔受遺,奈何便有私議!公忘先帝之言乎?」根大懼,陳謝而退。恪以告
 慕容垂,垂勸恪誅之。恪曰:「今新遭大凶,二虜伺隙,山陵未建,而宰輔自相誅滅,恐乖遠近之望,且可容忍之。」根與左衛慕輿乾潛謀誅恪及評,因而纂位。入白可足渾氏及瑋曰:「太宰、太傅將謀為亂,臣請率禁兵誅之,以安社稷。」可足渾氏將從之,瑋曰:「二公國之親穆,先帝所託,終應無此,未必非太師將為亂也。」於是使其侍中皇甫真、護軍傅顏收根等,於禁中斬之,大赦境內。遣傅顏率騎二萬觀兵河南,臨淮而還,軍威甚盛。



 初,俊所署寧南將軍呂護據野王,陰通京師,穆帝以護為前將軍、冀州刺史。俊死,謀引王師襲鄴,事覺,瑋使慕容恪等率眾五
 萬討之。傅顏言於恪曰:「護窮寇假合,王師既臨,則上下喪氣,曾不敢規兵中路,展其螗良之心。此則士卒懾魂,敗亡之驗也。殿下前以廣固天險,守易攻難,故為長久之策。今賊形便不與往同,宜急攻之,以省千金之費。」恪曰:「護老賊,經變多矣。觀其為備之道,未易卒平。今圈之窮城,樵採路絕,內無蓄積,外無彊援,不過十旬,其斃必矣,何必遽殘士卒之命而趣一時之利哉!吾嚴濬圍壘,休養將卒,以重官美貨間而離之。事淹勢窮,其釁易動;我則未勞,而寇已斃。此為兵不血刃,坐以制勝也。」遂列長圍守之。護遣其將張興率勁卒七千出戰,傅顏擊斬
 之。自三月至八月而野王潰,護南奔于晉,悉降其眾。尋復叛歸于瑋,瑋待之如初。因遣傅顏與護率眾據河陰。顏北襲敕勒,大獲而還。護攻洛陽,中流矢而死。將軍段崇收軍北渡,屯于野王。



 瑋遣其寧東慕容忠攻陷滎陽,又遣鎮南慕容塵寇長平。時晉冠軍將軍陳祐戍洛陽,遣使請救,帝遣桓溫援之。



 興寧初,瑋復使慕容評寇許昌、懸瓠、陳城,並陷之,遂略汝南諸郡,徙萬餘戶于幽、冀。瑋豫州刺史孫興上疏,請步卒五千先圖洛陽。瑋納之,遣其太宰司馬悅希軍于盟津,孫興分戍成皋,以為之聲援。尋而陳祐率眾奔陸渾,河南諸壘悉陷于希。慕容恪
 攻陷金墉,害揚威將軍沈勁。以其左中郎將慕容築為假節、征虜將軍、洛州刺史,鎮金墉,慕容垂為都督荊、揚、洛、徐、兗、豫、雍、益、梁、秦等十州諸軍事、征南大將軍、荊州牧,配兵一萬,鎮魯陽。



 時瑋境內多水旱,慕容恪、慕容評並稽首歸政,請遜位還第,曰:「臣以朽闇,器非經國,過荷先帝拔擢之恩,又蒙陛下殊常之遇,猥以輕才,竊位宰錄,不能上諧陰陽,下釐庶政,致使水旱愆和,彞倫失序,轅弱任重,夕惕唯憂。臣聞王者則天建國,辨方正位,司必量才,官惟德舉。台傅之重,參理三光,茍非其人,則靈曜為虧。尸祿貽殃,負乘招悔,由來常道,未之或差。以
 姬旦之勛聖,猶近則二公不悅,遠則管、蔡流言,況臣等寵緣戚來,榮非才授,而可久點天官,塵蔽賢路!是以中年拜表,披陳丹款。聖恩齒舊,未忍遐棄,奄冉偷榮,愆責彌厚。自待罪鼎司,歲餘辰紀;忝冒宰衡,七載于茲。雖乃心經略,而思不周務,至令二方干紀,跋扈未庭,同文之詠,有慚盛漢,深乖先帝託付之規,甚違陛下垂拱之義。臣雖不敏,竊聞君子之言,敢忘虞丘避賢之美,輒循兩疏知止之分,謹送太宰、大司馬、太傅、司徒章綬,惟垂昭許。」瑋曰:「朕以不天,早傾乾覆,先帝所託,唯在二公。二公懿親碩德,勳高魯、衛,翼贊王室,輔導朕躬,宣慈惠和,坐
 而待旦,虔誠夕惕,美亦至矣。故能外掃群凶,內清九土,四海晏如,政和時洽。雖宗廟社稷之靈,抑亦公之力也。今關右有未賓之氐,江、吳有遺燼之虜,方賴謀猷,混寧六合,豈宜虛己謙沖,以違委任之重!王其割二疏獨善之小,以成公旦復袞之大。」恪、評等固請致政,瑋曰:「夫建德者必以終善為名,佐命者則以功成為效。公與先帝開構洪基,膺天明命,將廓夷群醜,紹復隆周之跡。災眚橫流,乾光墜曜。朕以眇躬,猥荷大業,不能上成先帝遺志,致使二虜遊魂,所以功未成也,豈宜沖退。且古之王者,不以天下為榮,憂四海若荷擔,然後仁讓之風行,則
 比屋而可封。今道化未純,鯨鯢未殄,宗社之重,非唯朕身,公所憂也。當思所以寧濟兆庶,靖難敦風,垂美將來,侔蹤周、漢,不宜崇飾常節,以違至公。」遂斷其讓表,恪、評等乃止。



 瑋鐘律郎郭欽奏議以瑋承石季龍水為木德,瑋從之。



 太和元年,瑋遣撫軍慕容厲攻晉太山太守諸葛攸。攸奔于淮南,厲悉陷兗州諸郡,置守宰而還。



 慕容恪有疾,深慮瑋政不在己,慕容評性多猜忌,大司馬之位不能允授人望,乃召瑋兄樂安王臧謂之曰:「今勁秦跋扈,彊吳未賓,二寇並懷進取,但患事之無由耳。夫安危在得人,國興在賢輔,若能推才任忠,和同宗盟,則四
 海不足圖,二虜豈能為難哉!吾以常才,受先帝顧託之重,每欲掃平關、隴,蕩一甌、吳,庶嗣成先帝遺志,謝憂責于當年。而疾固彌留,恐此志不遂,所以沒有餘恨也。吳王天資英傑,經略超時,司馬職統兵權,不可以失人,吾終之後,必以授之。若以親疏次第,不以授汝,當以授沖。汝等雖才識明敏,然未堪多難,國家安危,實在于此,不可昧利忘憂,以致大悔也。」又以告評。月餘而死,其國中皆痛惜之。



 先是,晉南陽督護趙弘以宛降于瑋,瑋遣其南中郎將趙盤自魯陽戍宛。至此,晉右將軍桓豁攻宛,拔之,趙盤退奔魯陽。豁遣輕騎追盤,及於雉城,大戰敗
 之,執盤,戍宛而歸。



 苻堅將苻謏據陜,降于瑋。時有圖書云:「燕馬當飲渭水。」堅恐瑋乘釁入關,大懼,乃盡精銳以備華陰。瑋群下議欲遣兵救謏,因圖關右。慕容評素無經略,又受苻堅間貨,沮議曰:「秦雖有難,未易可圖。朝廷雖明,豈如先帝,吾等經略,又非太宰之匹,終不能平秦也。但可閉關息旅,保寧疆埸足矣。」瑋魏尹慕容德上疏曰:「先帝應天順時,受命革代,方以文德懷遠,以一六合。神功未就,奄忽升遐。昔周文既沒,武王嗣興,伏惟陛下則天比德,揆聖齊功,方闡崇乾基,纂成先志。逆氐僭據關、隴,號同王者,惡積禍盈,自相疑戮,釁起蕭牆,勢分四
 國,投城請援,旬日相尋,豈非凶運將終,數歸有道。兼弱攻昧,取亂侮亡,機之上也。今秦土四分,可謂弱矣。時來運集,天贊我也。天與不取,反受其殃。吳、越之鑒,我之師也。宜應天人之會,建牧野之旗。命皇甫真引并、冀之眾,徑趣蒲阪;臣垂引許、洛之兵,馳解謏圍;太傅總京都武旅,為二軍後繼。飛檄三輔,仁聲先路,獲城即侯,微功必賞,此則鬱概待時之雄,抱志未申之傑,必嶽峙灞上,雲屯隴下。天羅既張,內外勢合,區區僭豎,不走則降,大同之舉,今其時也。願陛下獨斷聖慮,無訪仁人。」瑋覽表大悅,將從之。評固執不許,乃止。苻謏知評、瑋之無遠略,恐
 救師弗至,乃箋於慕容垂、皇甫真曰:「苻堅、王猛皆人傑也,謀為燕患,為日久矣。今若乘機不赴,恐燕之君臣將有甬東之悔。」垂得書,私於真曰:「方為人患者必在於秦,主上富於春秋,未能留心政事,觀太傅度略,豈能抗苻堅、王猛乎?」真曰:「然,繞朝有云,謀之不從可如何!」



 瑋僕射悅綰言於瑋曰:「太宰政尚寬和,百姓多有隱附。《傳》曰,唯有德者可以寬臨眾,其次莫如猛。今諸軍營戶,三分共貫,風教陵弊,威綱不舉,宜悉罷軍封,以實天府之饒,肅明法令,以清四海。」瑋納之。綰既定制,朝野震驚,出戶二十餘萬。慕容評大不平,尋賊綰,殺之。



 晉大司馬桓溫、江
 州刺史桓沖、豫州刺史袁真率眾五萬伐瑋,前兗州刺史孫元起兵應之。溫部將檀玄攻胡陸,執瑋寧東慕容忠。瑋遣其將慕容厲與溫戰于黃墟,厲師大敗,單馬奔還。高平太守徐翻以郡歸順。溫前鋒朱序又破瑋將傅顏于林渚,溫軍大振,次於枋頭。瑋懼,謀奔和龍。慕容垂曰:「不然。臣請擊之,若戰不捷,走未晚也。」乃以垂為使持節、南討大都督,慕容德為征南將軍,率眾五萬距溫,使其散騎侍郎樂嵩乞師於苻堅。堅遣將軍茍池率眾二萬,出自洛陽,師于潁川,外為赴援,內實觀隙,有兼并之志矣。慕容德屯于石門,絕溫糧漕。豫州刺史李邦率州
 兵五千斷溫饋運。溫頻戰不利,糧運復絕,及聞堅師之至,乃焚舟棄甲而退。德率勁騎四千,先溫至襄邑東,伏於澗中,與垂前後夾擊,王師大敗,死者三萬餘人。茍池聞溫班師,邀擊于譙,溫眾又敗,死者萬計。



 垂既有大功,威德彌振,慕容評素不平之。垂又言其將孫蓋等摧鋒陷銳,宜論功超授,評寢而不錄。垂數以為言,頗與評廷爭。可足渾氏素惡垂,毀其戰功,遂與評謀殺垂。垂懼,奔於苻堅。



 先是,瑋使其黃門侍郎梁琛聘于堅。琛還,言於評曰:「秦揚兵講武,運粟陜東,以琛觀之,無久和之理。兼吳王西奔,必有觀釁之計,深宜備之。」評曰:「不然。秦豈可
 受吾叛臣而不懷和好哉!」琛曰:「鄰國相并,有自來矣。況今並稱大號,理無俱存。苻堅機明好斷,納善如流。王猛有王佐之才,銳於進取。觀其君臣相得,自謂千載一時。桓溫不足為慮,終為人患者,其唯王猛乎?。瑋、評不以為虞。皇甫真又陳其事曰:「苻堅雖聘使相尋,託輔車為諭,然抗均鄰敵,勢同戰國,明其甘於取利,無慕善之心,終不能守信存和,以崇久要也。頃來行人累續,兼師出洛川,夷險要害,具之耳目。觀虛實以措奸圖,聽風塵而伺國隙者,寇之常也。又吳王外奔,為之謀主,伍員之禍,不可不慮。洛陽、并州、壺關諸城,並宜增兵益守,以防未兆。」
 瑋召評而謀之。評曰:「秦國小力弱,杖我為援,且苻堅庶幾善道,終不納叛臣之言。不宜輕自擾懼,以動寇心也。」瑋從之。



 俄而堅遣其將王猛率眾伐瑋,攻慕容築於金墉。瑋遣慕容臧率眾救之。臧次滎陽,猛部將梁成、洛州刺史鄧羌與臧戰于石門,臧師敗績,死者萬餘,遂相持於石門。築以救兵不至,以金墉降于猛。梁成又敗慕容臧,斬首三千餘級,獲其將軍楊璩,臧遂城新樂而還。



 桓溫之敗也,歸罪于豫州刺史袁真。真怒,以壽陽降瑋,瑋遣其大鴻臚溫統署真為使持節、散騎常侍、都督淮南諸軍事、征南大將軍、領護南蠻校尉、揚州刺史,封宣城
 公,未至而真、統俱卒。真黨硃輔立真子瑾為建威將軍、豫州刺史,以固壽陽。



 時外則王師及苻堅交侵,兵革不息;內則瑋母亂政,評等貪冒,政以賄成,官非才舉,群下切齒焉。其尚書左承申紹上疏曰:



 臣聞漢宣有言:「與朕共治天下者,其唯良二千石乎!」是以特重此選,必妙盡英才,莫不拔自貢士,歷資內外,用能仁感猛獸,惠致群祥。今者守宰或擢自匹夫兵將之間,或因寵戚,藉緣時會,非但無聞於州閭,亦不經于朝廷。又無考績,黜陟幽明。貪惰為惡,無刑戮之懼;清勤奉法,無爵賞之勤。百姓窮弊,侵賕無已,兵士逋逃,乃相招為賊盜。風頹化替,莫
 相糾攝。且吏多則政煩,由來常患。今之見戶,不過漢之一大郡,而備置百官,加之新立軍號,兼重有過往時。虛假名位,廢棄農業,公私驅擾,人無聊生。宜并官省職,務勸農桑。秦、吳二虜僻僭一時,尚能任道捐情,肅諧偽部,況大燕累聖重光,君臨四海,而可美政或虧,取陵奸寇哉!鄰之有善,眾之所望,我之不修,彼之願也。



 秦、吳狡猾,地居形勝,非唯守境而已,乃有吞噬之心。中州豐實,戶兼二寇,弓馬之勁,秦、晉所憚,雲騎風馳,國之常也,而比赴敵後機,兵不速濟者何也?皆由賦法靡恒,役之非道。郡縣守宰每於差調之際,無不舍越殷彊,首先貧弱,行
 留俱窘,資贍無所,人懷嗟怨,遂致奔亡,進闕供國之饒,退離蠶農之要。兵豈在多,貴於用命。宜嚴制軍科,務先饒復,習兵教戰,使偏伍有常,從戎之外,足營私業,父兄有陟岵之觀,子弟懷孔爾之顧,雖赴水火,何所不從!



 節儉約費,先王格謨;去華敦僕,哲后恒憲。故周公戒成王以嗇財為本,漢文以皂幃變俗,孝景宮人弗過千餘,魏武寵賜不盈十萬,薄葬不墳,儉以率下,所以割肌膚之惠,全百姓之力。謹案後宮四千有餘,僮侍廝養通兼十倍,日費之重,價盈萬金,綺縠羅紈,歲增常調,戎器弗營,奢玩是務。今帑藏虛竭,軍士無襜褕之齎,宰相侯王迭
 以侈麗相尚,風靡之化,積習成俗,臥新之諭,未足甚焉。宜罷浮華非要之設,峻明婚姻喪葬之條,禁絕奢靡浮煩之事,出傾宮之女,均商農之賦。公卿以下以四海為家,信賞必罰,綱維肅舉者,溫、猛之首可懸之白旗,秦、吳二主可以禮之歸命,豈唯不復侵寇而巳哉!陛下若不遠追漢宗弋綈之模,近崇先帝補衣之美,臣恐頹風弊俗亦革變靡途,中興之歌無以軫之糸玄詠。



 又拓宇兼并,不在一城之地;控制戎夷者,懷之以德。令魯陽、上郡重山之外,雲陰之北,四百有餘,而未可以羈服塞表,為平寇之基,徒孤危託落,令善附內駭。宜攝就并、豫,以臨二
 河,通接漕轂,擬之丘後;重晉陽之戍,增南籓之兵,戰守之備,炫以千金之餌,蓄力待時,可一舉而滅。如其虔劉送死,俟入境而斷之,可令匹馬不反。非唯絕二賊窺窬,乃是戡殄之要,惟陛下覽焉。



 瑋不納。



 苻堅又使王猛、楊安率眾伐瑋,猛攻壺關,安攻晉陽。瑋使慕容評等率中外精卒四十餘萬距之。猛、安進師潞川。州郡盜賊大起,鄴中多怪異,瑋憂懼不知所為,乃召其使而問曰:「秦眾何如?今大師既出,猛等能戰不?」或對曰:「秦國小兵弱,豈王師之敵,景略常才,又非太傅之匹,不足憂也。」黃門待郎梁琛、中書侍郎樂嵩進曰:「不然。兵書之義,計敵能鬥,當
 以算取之。若冀敵不鬥,非萬全之道也。慶鄭有云:『秦眾雖少,戰士倍我。』眾之多少,非可問也。且秦行師千里,固戰是求,何不戰之有乎!」瑋不悅。



 猛與評等相持。評以猛懸軍遠入,利在速戰,議以持久制之。猛乃遣其將郭慶率騎五千,夜從間道起火高山,燒評輜重,火見鄴中。評性貪鄙,鄣固山泉,賣樵鬻水,積錢絹如丘陵,三軍莫有鬥志。瑋遣其侍中蘭伊讓評曰:「王,高祖之子也,宜以宗廟社稷為憂,奈何不務撫養勳勞,專以聚斂為心乎!府藏之珍貨,朕豈與王愛之!若寇軍冒進,王持錢帛安所置也!皮之不存,毛將安傅!錢帛可散之三軍,以平寇凱
 旋為先也。」評懼而與猛戰于潞川,評師大敗,死者五萬餘人,評等單騎遁還。猛遂長驅至鄴,堅復率眾十萬會猛攻瑋。



 先是,慕容桓以眾萬餘屯于沙亭,為評等後繼。聞評敗,引屯內黃。堅遣將鄧羌攻信都,桓率鮮卑五千退保和龍。散騎侍郎徐蔚等率扶餘、高句麗及上黨質子五百餘人,夜開城門以納堅軍。瑋與評等數十騎奔于昌黎。堅遣郭慶追及瑋于高陽,堅將巨武執瑋,將縛之,瑋曰:「汝何小人而縛天子!」武曰:「我梁山巨武,受詔縛賊,何謂天子邪!」遂送瑋于堅。堅詰其奔狀,瑋曰:「狐死首丘,欲歸死於先人墳墓耳!」堅哀而釋之,令還宮率文武
 出降。郭慶遂追評、桓子和龍。桓殺其鎮東慕容亮而并其眾,攻其遼東太守韓稠于平川。郭慶遣將軍朱嶷擊桓,執而送之。



 堅徙瑋及其王公已下并鮮卑四萬餘戶于長安,封瑋新興侯,署為尚書。堅征壽春,以瑋為平南將軍、別部都督。淮南之敗,隨堅還長安。既而慕容垂攻苻丕于鄴,慕容沖起兵關中,瑋謀殺堅以應之,事發,為堅所誅,時年三十五。及德僭稱尊號,偽謚幽皇帝。



 始廆以武帝太康六年稱公,至瑋四世。瑋在位一十一年,以海西公太和五年滅,通廆、皝凡八十五年。



 慕容恪,
 字玄恭,皝之第四子也。幼而謹厚,沈深有大度。母高氏無寵,皝未之奇也。年十五,身長八尺七寸,容貌魁傑,雄毅嚴重,每所言及,輒經綸世務,皝始異焉,乃授之以兵。數從皝征伐,臨機多奇策。使鎮遼東,甚有威惠。高句麗憚之,不敢為寇。皝使恪與俊俱伐夫餘,俊居中指授而已,恪身當矢石,推鋒而進,所嚮輒潰。



 皝將終,謂俊曰:「今中原未一,方建大事,恪智勇俱濟,汝其委之。」及俊嗣位,彌加親任。累戰有大功,封太原王,拜侍中、假節、大都督、錄尚書。俊寢疾,引恪與慕容評屬以後事。及瑋之世,總攝朝權。初,建鄴聞俊死,曰:「中原可圖矣。」桓溫曰:「
 慕容恪尚存,所憂方為大耳。」



 慕輿根之就誅也,內外危懼。恪容止如常,神色自若,出入往還,一人步從。或有諫之者,恪曰:「人情懷懼,且當自安以靖之。吾復不安,則眾何瞻仰哉!」於是人心稍定。恪虛襟待物,咨詢善道,量才處任,使人不踰位。朝廷謹肅,進止有常度,雖執權政,每事必咨之於評。罷朝歸第,則盡心色養,手不釋卷。其百僚有過,未嘗顯之,自是庶僚化德,稀有犯者。



 恪之圖洛陽也,秦中大震,苻堅親將以備潼關,軍迴乃定。恪為將不尚威嚴,專以恩信御物,務於大略,不以小令勞眾。軍士有犯法,密縱舍之,捕斬賊首以令軍。營內不整似可
 犯,而防禦甚嚴,終無喪敗。



 臨終,瑋親臨問以後事,恪曰:「臣聞報恩莫大薦士,板築猶可,而況國之懿籓!吳王文武兼才,管、蕭之亞,陛下若任之以政,國其少安。不然,臣恐二寇必有窺窬之計。」言終而死。



 陽騖,字士秋,右北平無終人也。父耽,仕廆,官至東夷校尉。騖少清素好學,器識沈遠。起家為平州別駕,屢獻安時彊國之術,事多納用,廆甚奇之。皝即王位,遷左長史。東西征伐,參謀幃幄。皝臨終謂俊曰:「陽士秋忠幹貞固,可託付大事,汝善待之。」俊之將圖中原也,騖制勝之功
 亞於慕容恪。瑋既嗣偽位,申以師傅之禮,親遇日隆。及為太尉,慨然而歎曰:「昔常林、徐邈先代名臣,猶以鼎足任重而終辭三事。以吾虛薄,何德以堪之!固求罷職,言甚墾至,瑋優答不許。騖清貞謙謹,老而彌篤,既以宿望舊齒,自慕容恪已下莫不畢拜。性儉約,常乘弊車瘠馬,及死,無斂財。



 皇甫真,字楚季,安定朝那人也。弱冠,以高才,廆拜為遼東國侍郎。皝嗣位,遷平州別駕。時內難連年,百姓勞瘁,真議欲寬減歲賦,休息力役。不合旨,免官。後以破麻秋
 之功,拜奉車都督,守遼東、營丘二郡太守,皆有善政。及俊僭位,入為典書令。後從慕容評攻拔鄴都,珍貨充溢,真一無所取,唯存恤人物,收圖籍而已。俊臨終,與慕容恪等俱受顧託。



 慕輿根將謀為亂,真陰察知之,乃言於恪,請除之。恪未忍顯其事。俄而根謀發伏誅,恪謝真曰:「不從君言,幾成禍敗。」呂護之叛,恪謀於朝曰:「遠人不服,修文德以來之。今護宜以恩詔降乎,不宜以兵戈取也?」真曰:「護九年之間三背王命,揆其姦心,凶勃未已。明公方飲馬江、湘,勒銘劍閣,況護蕞爾近幾而不梟戮,宜以兵算取之,不可復以文檄喻也。」恪從之。以真為冠軍將
 軍、別部都督。師還,拜鎮西將軍、并州刺史,領護匈奴中郎將。徵還,拜侍中、光祿大夫,累遷太尉、侍中。



 苻堅密謀兼并,欲觀審釁隙,乃遣其西戎主簿郭辯潛結匈奴左賢王曹轂,令轂遣使詣鄴,辯因從之。真兄典仕苻堅為散騎常侍,從子奮、覆並顯關西。辯既至鄴,歷造公卿,言於真曰:「辯家為秦所誅,故寄命曹王,貴兄常侍及奮、覆兄弟並相知在素。」真怒曰:「臣無境外之交,斯言何以及我!君似姦人,得無因緣假託乎!」乃白瑋請窮詰之,瑋、評不許。辯還謂堅曰:「燕朝無綱紀,實可圖之。鑒機識變,唯皇甫真耳。」堅曰:「以六州之地,豈無智識士一人哉!真亦
 秦人,而燕用之,固知關西多君子矣。」



 真性清儉寡慾,不營產業,飲酒至石餘不亂,雅好屬文,凡著詩賦四十餘篇。



 王猛入鄴,真望馬首拜之。明日更見,語乃卿猛。猛曰:「昨拜今卿,何恭慢之相違也?」真答曰:「卿昨為賊,朝是國士,吾拜賊而卿國士,何所怪也?」猛大嘉之,謂權翼曰:「皇甫真故大器也。」從堅入關,為奉車都尉,數歲而死。



 史臣曰:觀夫北陰衍氣,醜虜匯生,隔閡諸華,聲教莫之漸,雄據殊壤,貪悍成其俗,先叛後服,蓋常性也。自當塗紊紀,典午握符,推亡之功,掩岷、吳而可錄,御遠之策,懷戎狄而猶漏。慕容廆英姿偉量,是曰邊豪,釁迹姦圖,實
 惟亂首。何者?無名而舉,表深譏於魯冊;象龔致罰,昭大訓於姚典。況乎放命挻禍,距戰發其狼心;剽邑屠城,略地騁其蝥賊。既而二帝遘平陽之酷,按兵窺運;五鐸啟金陵之祚,率禮稱籓。勤王之誠,當君危而未立;匡主之節,俟國泰而將徇。適所謂相時而動,豈素蓄之款戰!然其制敵多權,臨下以惠,勸農桑,敦地利,任賢士,該時傑,故能恢一方之業,創累葉之基焉。



 元真體貌不恒,暗符天表,沈毅自處,頗懷奇略。于時群雄角立,爭奪在辰,顯宗主祭於沖年,庾亮竊政於元舅,朝綱不振,天步孔艱,遂得據已成之資,乘土崩之會。揚兵南矛騖,則烏丸卷甲;
 建旆東征,則宇文摧陣。乃負險自固,恃勝而驕,端拱稱王,不待朝命,昔鄭武職居三事,爵不改伯;齊桓績宣九合,位止為侯。瞻曩烈而功微,徵前經而禮縟,谿壑難滿,此之謂乎?



 宣英文武兼優,加之以機斷,因石氏之釁,首圖中原,燕士協其籌,冀馬為其用,一戰而平巨寇,再舉而拔堅城,氣讋傍鄰,威加邊服。便謂深功被物,天數在躬,遽竊鴻名,偷安寶錄。猶將席卷京洛,肆其蟻聚之徒;宰割黎元,縱其鯨吞之勢。使江左疲於奔命,職此之由。非夫天厭素靈而啟異類,不然者,其鋒何以若斯!



 景茂庸材,不親厥務,賢輔攸賴,逆臣挫謀,於是陷金墉而款
 河南,包銅城而臨漠北,西秦勁卒頓函關而不進,東夏遺黎企鄴宮而授首。當此之時也,兇威轉熾。及玄恭即世,虐媼亂朝。垂以勛德不容,評以黷貨干政,志士絕忠貞之路,讒人襲交亂之風。輕鄰反速其咎,禦敵罕修其備,以攜離之眾,抗敢死之師。鋒鏑未交,白溝淪境;沖輣暫擬,紫陌成墟。是知由余出而戎亡,子常升而郢覆,終於身死異域,智不自全,吉兇惟人,良所謂也。



 贊曰:青山徙構,玄塞分疆。蠢茲雜種,奕世彌昌。角端掩月,步搖翻霜。乘危猥起,怙險鴟張。假竊神器,憑陵帝鄉。守不以德,終致餘殃。



\end{pinyinscope}