\article{列傳第二十}

\begin{pinyinscope}
曹志庾峻
 \gezhu{
  子氏敳}
 郭象庾純
 \gezhu{
  子旉}
 秦秀



 曹志,字允恭,譙國譙人,魏陳思王植之孽子也。少好學,以才行稱,夷簡有大度,兼善騎射。植曰:「此保家主也。」立以為嗣。後改封濟北王。武帝為撫軍將軍,迎陳留王于鄴,志夜謁見,帝與語,自暮達旦,甚奇之。及帝受禪,降為鄄城縣公。詔曰:「昔在前世,雖歷運迭興,至於先代苗裔,傳祚不替,或列籓九服,式序王官。選眾命賢,惟德是與,
 蓋至公之道也。魏氏諸王公養德藏器,壅滯曠久,前雖有詔,當須簡授,而自頃眾職少缺,未得式敘。前濟北王曹志履德清純,才高行潔,好古博物,為魏宗英,朕甚嘉之。其以志為樂平太守。」志在郡上書,以為宜尊儒重道,請為博士置吏卒。遷章武、趙郡太守。雖累郡職,不以政事為意,晝則游獵,夜誦《詩》《書》,以聲色自娛,當時見者未能審其量也。



 咸寧初,詔曰:「鄄城公曹志,篤行履素,達學通識,宜在儒林,以弘胄子之教。其以志為散騎常侍、國子博士。」帝嘗閱《六代論》,問志曰:「是卿先王所作邪?」志對曰:「先王有手所作目錄,請歸尋按。」還奏曰:「按錄無此。」帝曰:「
 誰作?」志曰:「以臣所聞,是臣族父冏所作。以先王文高名著,欲令書傳於後,是以假託。」帝曰:「古來亦多有是。」顧謂公卿曰:「父子證明,足以為審。自今已後,可無復疑。」



 後遷祭酒。齊王攸將之國,下太常議崇錫文物。時博士秦秀等以為齊王宜內匡朝政,不可之籓。志又常恨其父不得志於魏,因愴然歎曰:「安有如此之才,如此之親,不得樹本助化,而遠出海隅?晉朝之隆,其殆乎哉!」乃奏議曰:「伏聞大司馬齊王當出籓東夏,備物盡禮,同之二伯。今陛下為聖君,稷、契為賢臣,內有魯、衛之親,外有齊、晉之輔,坐而守安,此萬世之基也。古之夾輔王室,同姓則周
 公其人也,異姓則太公其人也,皆身在內,五世反葬。後雖有五霸代興,桓、文譎主,下有請隧之僭,上有九錫之禮,終於譎而不正,驗於尾大不掉,豈與召公之歌《棠棣》,周詩之詠《鴟鴞》同日論哉!今聖朝創業之始,始之不諒,後事難工。乾植不彊,枝葉不茂;骨鯁不存,皮膚不充。自羲皇以來,豈是一姓之獨有!欲結其心者,當有磐石之固。夫欲享萬世之利者,當與天下議之。故天之聰明,自我人之聰明。秦、魏欲獨擅其威,而財得沒其身;周、漢能分其利,而親疏為之用。此自聖主之深慮,日月之所照。事雖淺,當深謀之;言雖輕,當重思之。志備位儒官,若言
 不及禮,是志寇竊。知忠不言,議所不敢。志以為當如博士等議。」議成當上,見其從弟高邑公嘉。嘉曰:「兄議甚切,百年之後必書晉史,目下將見責邪。」帝覽議,大怒曰:「曹志尚不明吾心,況四海乎!」以議者不指答所問,橫造異論,策免太常鄭默。於是有司奏收志等結罪,詔惟免志官,以公還第,其餘皆付廷尉。



 頃之,志復為散騎常侍。遭母憂,居喪過禮,因此篤病,喜怒失常。九年卒,太常奏以惡謚。崔褒歎曰:「魏顆不從亂,以病為亂故也。今謚曹志而謚其病,豈謂其病不為亂乎!」於是謚為定。



 庾峻,字山甫,潁川鄢陵人也。祖乘,才學洽聞,漢司徒辟,有道征,皆不就。伯父嶷,中正簡素,仕魏為太僕。父道,廉退貞固,養志不仕。牛馬有踶齧者,恐傷人,不貨於市。及諸子貴,賜拜太中大夫。峻少好學,有才思。嘗游京師,聞魏散騎常侍蘇林老疾在家,往候之。林嘗就乘學,見峻流涕,良久曰:「尊祖高才而性退讓,慈和汎愛,清靜寡欲,不營當世,惟修德行而已。鄢陵舊五六萬戶,聞今裁有數百。君二父孩抱經亂,獨至今日,尊伯為當世令器,君兄弟復俊茂,此尊祖積德之所由也。」



 歷郡功曹,舉計掾,州辟從事。太常鄭袤見峻,大奇之,舉為博士。時重《莊》《老》
 而輕經史,駿懼雅道陵遲,乃潛心儒典。屬高貴鄉公幸太學,問《尚書》義於峻,峻援引師說,發明經旨,申暢疑滯,對答詳悉。遷秘書丞。長安有大獄,久不決,拜峻侍御史,往斷之,朝野稱允。武帝踐阼,賜爵關中侯,遷司空長史,轉秘書監、御史中丞,拜侍中,加諫議大夫。常侍帝講《詩》,中庶子何劭論《風》《雅》正變之義,峻起難往反,四坐莫能屈之。



 是時風俗趣競,禮讓陵遲。峻上疏曰:



 臣聞黎庶之性,人眾而賢寡;設官分職,則官寡而賢眾。為賢眾而多官,則妨化;以無官而棄賢,則廢道。是故聖王之御世也,因人之性,或出或處,故有朝廷之士,又有山林之士。朝
 廷之士,佐主成化,猶人之有股肱心膂,共為一體也。山林之士,被褐懷玉,太上棲於丘園,高節出於眾庶。其次輕爵服,遠恥辱以全志。最下就列位,惟無功而能知止。彼其清劭足以抑貪汙,退讓足以息鄙事。故在朝之士聞其風而悅之,將受爵者皆恥躬之不逮。斯山林之士、避寵之臣所以為美也,先王嘉之。節雖離世,而德合于主;行雖詭朝,而功同于政。故大者有玉帛之命,其次有几杖之禮,以厚德載物,出處有地。既廊廟多賢才,而野人亦不失為君子,此先王之弘也。



 秦塞斯路,利出一官。雖有處士之名,而無爵列於朝者,商君謂之六蠍,韓非
 謂之五蠹。時不知德,惟爵是聞。故閭閻以公乘侮其鄉人,郎中以上爵傲其父兄。漢祖反之,大暢斯否。任蕭、曹以天下,重四皓於南山。以張良之勳,而班在叔孫之後;蓋公之賤,而曹相諮之以政。帝王貴德於上,俗亦反本於下。故田叔等十人,漢廷臣無能出其右者,而未嘗干祿於時。以釋之之貴,結王生之襪於朝,而其名愈重。自非主臣尚德兼愛,孰能通天下之志,如此其大者乎!



 夫不革百王之弊,徒務救世之政,文士競智而務入,武夫恃力而爭先。官高矣,而意未滿;功報矣,其求不已。又國無隨才任官之制,俗無難進易退之恥。位一高,雖無功
 而不見下,已負敗而後見用。故因前而升,則處士之路塞矣。又仕者黜陟無章,是以普天之下,先競而後讓,舉世之士,有進而無退。大人溺於動俗,執政撓於群言,衡石為之失平,清濁安可復分?昔者先王患向之所以取天下者,今之為弊,是故功成必改其物,業定必易其教。雖以爵祿使下,臣無貪陵之行;雖以甲兵定功,主無窮武之悔也。



 臣愚以為古者大夫七十懸車,今自非元功國老,三司上才,可聽七十致仕,則士無懷祿之嫌矣。其父母八十,可聽終養,則孝莫大於事親矣。吏歷試無績,依古終身不仕,則官無秕政矣。能小而不能大,可降還
 涖小,則使人以器矣。人主進人以禮,退人以禮,人臣亦量能受爵矣。其有孝如王陽,臨九折而去官,潔如貢禹,冠一免而不著,及知止如王孫,知足如疏廣,雖去列位而居東野,與人父言,依於慈,與人子言,依於孝。此其出言合於國檢,危行彰於本朝。去勢如脫屣,路人為之隕涕;辭寵如金石,庸夫為之興行。是故先王許之,而聖人貴之。



 夫人之性陵上,猶水之趣下也,益而不已必決,升而不已必困。始於匹夫行義不敦,終於皇輿為之敗績,固不可不慎也。下人并心進趣,上宜以退讓去其甚者。退讓不可以刑罰使,莫若聽朝士時時從志,山林往往
 間出。無使入者不能復出,往者不能復反。然後出處交泰,提衡而立,時靡有爭,天下可得而化矣。



 又疾世浮華,不修名實,著論以非之,文繁不載。九年卒,詔賜朝服一具、衣一襲、錢三十萬。臨終,敕子氏朝卒夕殯,幅巾布衣,葬勿擇日。氏奉遵遺命,斂以時服。二子:氏、敳。



 氏字子琚。性淳和好學,行己忠恕。少歷散騎常侍、本國中正、侍中,封長岑男。懷帝之沒劉元海也,氏從在平陽。元海大會,因使帝行酒,氏不勝悲憤,再拜上酒,因大號哭,賊惡之。會有告氏及王人雋等謀應劉琨者,元海因圖弒逆,氏等並遇害。初,洛陽之未陷也,氏為侍中,直于省
 內,謂同僚許遐曰:「世路如此,禍難將及,吾當死乎此屋耳!」及是,竟不免焉。太元末,追謚曰貞。



 敳字子嵩。長不滿七尺,而腰帶十圍,雅有遠韻。為陳留相,未嘗以事嬰心,從容酣暢,寄通而已。處眾人中,居然獨立。嘗讀《老》《莊》,曰:「正與人意闇同。」太尉王衍雅重之。



 敳見王室多難,終知嬰禍,乃著《意賦》以豁情,猶賈誼之《服鳥》也。其詞曰:「至理歸於渾一兮,榮辱固亦同貫。存亡既已均齊兮,正盡死復何歎。物咸定於無初兮,俟時至而後驗。若四節之素代兮,豈當今之得遠?且安有壽之與夭兮,或者情橫多戀。宗統竟初不別兮,大德亡其情願。
 蠢動皆神之為兮,癡聖惟質所建。真人都遣穢累兮,性茫蕩而無岸。縱驅於遼廓之庭兮,委體乎寂寥之館。天地短於朝生兮,億代促於始旦。顧瞻宇宙微細兮,眇若豪鋒之半。飄搖玄曠之域兮,深漠暢而靡玩。兀與自然并體兮,融液忽而四散。」從子亮見賦,問曰:「若有意也,非賦所盡;若無意也,復何所賦?」答曰:「在有無之間耳!」



 遷吏部郎。是時天下多故,機變屢起,敳常靜默無為。參東海王越太傅軍事,轉軍諮祭酒。時越府多雋異,敳在其中,常自袖手。豫州牧長史河南郭象善《老》《莊》,時人以為王弼之亞。敳甚知之,每曰:「郭子玄何必減瘐子嵩」。象後為
 太傅主簿,任事專勢。敳謂象曰:「卿自是當世大才,我疇昔之意都已盡矣。」



 敳有重名,為搢紳所推,而聚斂積實,談者譏之。都官從事溫嶠奏之,豈又更器嶠,目嶠森森如千丈松,雖礧砢多節,施之大廈,有棟梁之用。時劉輿見任於越,人士多為所構,惟敳縱心事外,無迹可間。後以其性儉家富,說越令就換錢千萬,冀其有吝,因此可乘。越於眾坐中問於敳,而敳乃頹然已醉,幘墮機上,以頭就穿取,徐答云:「下官家有二千萬,隨公所取矣。」輿於是乃服。越甚悅,因曰:「不可以小人之慮度君子之心。」王衍不與敳交,敳卿之不置。衍曰:「君不得為耳。」敳曰:「卿自君
 我,我自卿卿。我自用我家法,卿自用卿家法。」衍甚奇之。石勒之亂,與衍俱被害,時年五十。



 郭象,字子玄,少有才理,好《老》《莊》,能清言。太尉王衍每云:「聽象語,如懸河瀉水,注而不竭。」州郡辟召,不就。常閑居,以文論自娛。後辟司徒掾,稍至黃門侍郎。東海王越引為太傅主簿,甚見親委,遂任職當權,熏灼內外,由是素論去之。永嘉末病卒,著碑論十二篇。



 先是,註《莊子》者數十家,莫能究其旨統。向秀於舊註外而為解義,妙演奇致,大暢玄風,惟《秋水》、《至樂》二篇未竟而秀卒。秀子幼,其
 義零落,然頗有別本遷流。象為人行薄,以秀義不傳于世,遂竊以為己注,乃自註《秋水》、《至樂》二篇,又易《馬蹄》一篇,其餘眾篇或點定文句而巳。其後秀義別本出,故今有向、郭二《莊》,其義一也。



 庾純,字謀甫,博學有才義,為世儒宗。郡補主簿,仍參征南府,累遷黃門侍郎,封關內侯,歷中書令、河南尹。初,純以賈充姦佞,與任愷共舉充西鎮關中,充由是不平。充嘗宴朝士,而純後至,充謂曰:「君行常居人前,今何以在後?」純曰:「旦有小市井事不了,是以來後。」世言純之先嘗
 有伍伯者,充之先有市魁者,充、純以此相譏焉。充自以位隆望重,意殊不平。及純行酒,充不時飲。純曰:「長者為壽,何敢爾乎!」充曰:「父老不歸供養,將何言也!」純因發怒曰:「賈充!天下兇兇,由爾一人。」充曰:「充輔佐二世,蕩平巴、蜀,有何罪而天下為之兇兇?」純曰:「高貴鄉公何在?」眾坐因罷。充左右欲執純,中護軍羊琇、侍中王濟佑之,因得出。充慚怒,上表解職。純懼,上河南尹、關內侯印綬,上表自劾曰:「司空公賈充請諸卿校并及臣。臣不自量,飲酒過多。醉亂行酒,重酌於公,公不肯飲,言語往來,公遂訶臣父老不歸供養,卿為無天地。臣不服罪自引,而更忿
 怒,厲聲名公,臨時喧饒,遂至荒越。禮,『八十月制』,誠以衰老之年,變難無常也。臣不惟生育之恩,求養老父,而懷祿貪榮,烏鳥之不若。充為三公,論道興化,以教義責臣,是也。而以枉錯直,居下犯上,醉酒迷荒,昏亂儀度。臣得以凡才,擢授顯任。《易》戒濡首,《論》誨酒困,而臣聞義不服,過言盈庭,黷幔台司,違犯憲度,不可以訓。請臺免臣官,廷尉結罪,大鴻臚削爵土。敕身不謹,伏須罪誅。」御史中丞孔恂劾純,請免官。詔曰:「先王崇尊卑之禮,明貴賤之序,著溫克之德,記沈酗之禍,所以光宣道化,示人軌儀也。昔廣漢陵慢宰相,獲犯上之刑;灌夫託醉肆忿,致誅
 斃之罪。純以凡才,備位卿尹,不惟謙敬之節,不忌覆車之戒,陵上無禮,悖言自口,宜加顯黜,以肅朝倫。」遂免純官。



 又以純父老不求供養,使據禮典正其臧否。太傅何曾、太尉荀顗、驃騎將軍齊王攸議曰:「凡斷正臧否,宜先稽之禮、律。八十者,一子不從政;九十者,其家不從政。新令亦如之。按純父年八十一,兄弟六人,三人在家,不廢侍養。純不求供養,其於禮、律未有違也。司空公以純備位卿尹,望其有加於人。而純荒醉,肆其忿怒。臣以為純不遠布孝至之行,而近習常人之失,應在譏貶。」司徒石苞議:「純榮官忘親,惡聞格言,不忠不孝,宜除名削爵土。」
 司徒西曹掾劉斌議以為:「敦敘風俗,以人倫為先;人倫之教,以忠孝為主。忠故不忘其君,孝故不忘其親。若孝必專心於色養,則明君不得而臣;忠必不顧其親,則父母不得而子也。是以為臣者,必以義斷其恩;為子也,必以情割其義。在朝則從君之命,在家則隨父之制。然後君父兩濟,忠孝各序。純兄峻以父老求歸,峻若得歸,純無不歸之勢;峻不得歸,純無得歸之理。純雖自聞,同不見聽。近遼東太守孫和、廣漢太守鄧良皆有老母,良無兄弟,授之遠郡,辛苦自歸,皆不見聽。且純近為京尹,父在界內,時得自啟定省,獨於禮法外處其貶黜,斌愚以
 為非理也。禮,年八十,一子不從政。純有二弟在家,不為違禮。又令,年九十,乃聽悉歸。今純父實未九十,不為犯令。罵辱宰相,宜加放斥,以明國典。聖恩愷悌,示加貶退,臣愚無所清議。」河南功曹史龐札等表曰:



 臣郡前尹關內侯純,醉酒失常,《戊申詔書》既免尹官,以父篤老不求供養,下五府依禮典正其臧否。臣謹按三王養老之制,八十,一子不從政;九十,其家不從政,斯誠使人無闕孝養之道,為臣不違在公之節也。先王制禮垂訓,莫尚於周。當其時也,姬公留周,伯禽之魯,孝子不匱,典禮無愆。今公府議,七十時制,八十月制,欲以駮奪從政之限,削
 除爵土。是為公旦立法,還自越之,魯侯為子,即為罰首也。石奮期頤,四子列郡。近太宰獻王諸子,亦有籓外。古今同符,忠孝並濟。



 臣聞悔吝之疵,君子有之。尹性少飲多,遂至沈醉。尹醒聞知,悼恨前失,執謙引罪,深自奏劾,求入重法。今公府不原所由,而謂傲很,是為重罪過醉之言,而沒迷復之義也。臣聞父子天性,愛由自然,君臣之交,出自義合,而求忠臣必於孝子。是以先王立禮,敬同於父,原始要終,齊於所生,如此猶患人臣罕能致身。今公府議云,禮律雖有常限,至於疾病歸養,不奪其志。如此則為禮禁正直,而陷人以詐,違越王制,開其殆原。
 尹少履清苦,事親色養,歷職內外,公廉無私,此陛下之所以屢發明詔,而尹之所以仍見擢授也。尹行己也恭,率下也敬,先眾後己,實是宿心。一旦由醉,責以暴慢。按奏狀不忠不孝,群公建議削除爵土,此愚臣所以自悲自悼,拊心泣血也。



 按今父母年過八十,聽令其子不給限外職,誠以得有歸來之緣。今尹居在郡內,前每表屢蒙定省。尹昆弟六人,三人在家,孝養不廢。兄侍中峻,家之嫡長,往比自表,求歸供養,詔喻不聽。國體法同,兄弟無異,而虛責尹不求供養如斯,臣懼長假飾之名,而損忠誠之實也。夫禮者,所以經國家,定社稷也。故陶唐之
 隆,順考古典;周成之美,率由舊章。伏惟陛下聖德欽明,敦禮崇教,疇諮四嶽,以詳典制。尹以犯違受黜,而所由者醉。公以教義見責,而所因者忿。積忿以立義,由醉以得罪,禮律不復為斷,文致欲以成法。是以愚臣敢冒死亡之誅,而恥不伸於盛明之世。惟蒙哀察。



 帝復下詔曰:「自中世以來,多為貴重順意,賤者生情,故令釋之、定國得揚名於前世。今議責庾純,不惟溫克,醉酒沈湎,此責人以齊聖也。疑賈公亦醉,若其不醉,終不於百客之中責以不去官供養也。大晉依聖人典禮,制臣子出處之宜,若有八十,皆當歸養,亦不獨純也。古人云:『由醉之言,
 俾出童羖。』明不責醉,恐失度也。所以免純者,當為將來之醉戒耳。齊王、劉掾議當矣。」復以純為國子祭酒,加散騎常侍。後將軍荀眅於朝會中奏純以前坐不孝免黜,不宜升進。侍中甄德進曰:「孝以顯親為大,祿養為榮。詔赦純前愆,擢為近侍,兼掌教官,此純召不俟駕之日。而後將軍眅敢以私議貶奪公論,抗言矯情,誣罔朝廷,宜加貶黜。」眅坐免官。



 初,眅與純俱為大將軍所辟,眅整麗車服,純率素而已,眅以為愧恨。至是,毀純。眅既免黜,純更以此愧之,亟往慰勉之,時人稱純通恕。



 遷侍中,以父憂去官。起為御史中丞,轉尚書。除魏郡太守,不之官,拜
 少府。年六十四卒。子旉。



 旉字允臧。少有清節,歷位博士。齊王攸之就國也,下禮官議崇錫之物。旉與博士太叔廣、劉暾、繆蔚、郭頤、秦秀、傅珍等上表諫曰:



 《書》稱帝堯「克明俊德,以親九族」。武王光有天下,兄弟之國十有六人,同姓之國四十人,元勛睦親,顯以殊禮,而魯、衛、齊、晉大啟土宇,並受分器。所謂惟善所在,親疏一也。大晉龍興,隆唐、周之遠迹,王室親屬,佐命功臣,咸受爵土,而四海乂安。今吳、會已平,詔大司馬齊王出統方嶽,當遂撫其國家,將準古典,以垂永制。



 昔周之選建明德以左右王室也,則周公為太宰,康
 叔為司寇,聃季為司空。及召、芮、畢、毛諸國,皆入居公卿大夫之位,明股肱之任重,守地之位輕也,未聞古典以三事之重出之國者。漢氏諸侯王位尊勢重,在丞相三公上。其入讚朝政者,乃有兼官,其出之國,亦不復假台司虛名為隆寵也。



 昔申無宇曰「五大不在邊」,先儒以為貴寵公子公孫,累世正卿也。又曰「五細不在庭」,先儒以為賤妨貴,少陵長,遠間親,新間舊,小加大也。不在庭,不在朝廷為政也。又曰:「親不在外,羈不在內。今棄疾在外,鄭丹在內,君其少戒之。」叔向有言:「公室將卑,其枝葉先落。」公族,公室之本,而去之,諺所謂芘焉而縱尋斧柯者
 也。



 今使齊王賢邪,則不宜以母弟之親尊,居魯、衛之常職;不賢邪,不宜大啟土宇,表建東海也。古禮,三公無職,坐而論道,不聞以方任嬰之。惟周室大壞,宣王中興,四夷交侵,救急朝夕,然後命召穆公征淮夷。故其詩曰「徐方不回,王曰旋歸」,宰相不得久在外也。今天下已定,六合為家,將數延三事,與論太平之基,而更出之,去王城二千里,違舊章矣。



 旉草議,先以呈父純,純不禁。太常鄭默、博士祭酒曹志並過其事。武帝以博士不答所問,答所不問,大怒,事下有司。尚書朱整、褚等奏:「旉等侵官離局,迷罔朝廷,崇飾惡言,假託無諱,請收旉等八人付
 廷尉科罪。」旉父純詣廷尉自首:「旉以議草見示,愚淺聽之。」詔免純罪。



 廷尉劉頌又奏旉等大不敬,棄市論,求平議。尚書又奏請報聽廷尉行刑。尚書夏侯駿謂硃整曰:「國家乃欲誅諫臣!官立八座,正為此時,卿可共駁正之。」整不從,駿怒起,曰:「非所望也!」乃獨為駁議。左僕射魏舒、右僕射下邳王晃等從駿議。奏留中七日,乃詔曰:「旉等備為儒官,不念奉憲制,不指答所問,敢肆其誣罔之言,以干亂視聽。而旉是議主,應為戮首。但旉及家人並自首,大信不可奪。秦秀、傅珍前者虛妄,幸而得免,復不以為懼,當加罪戮,以彰兇慝。猶復不忍,皆丐其死命。秀、珍、
 旉等並除名。」後數歲,復起為散騎侍郎。終於國子祭酒。



 秦秀,字玄良,新興雲中人也。父朗,魏驍騎將軍。秀少敦學行,以忠直知名。咸寧中,為博士。何曾卒,下禮官議謚。秀議曰:



 故太宰何曾,雖階世族之胤,而少以高亮嚴肅,顯登王朝。事親有色養之名,在官奏科尹模,此二者實得臣子事上之概。然資性驕奢,不循軌則。《詩》云:「節彼南山,惟石巖巖,赫赫師尹,人具爾瞻。」言其德行高峻,動必以禮耳。丘明有言:「儉,德之恭;侈,惡之大也。」大晉受命,勞廉隱約,曾受寵二代,顯赫累世。暨乎耳順之年,身兼三
 公之位,食大國之租,荷保傅之貴,執司徒之均。二子皆金貂卿校,列于帝側。方之古人,責深負重,雖舉門盡死,猶不稱位。而乃驕奢過度,名被九域,行不履道,而享位非常。以古義言之,非惟失輔相之宜,違斷金之利也。穢皇代之美,壞人倫之教,生天下之醜,示後生之傲,莫大於此。自近世以來,宰臣輔相,未有受垢辱之聲,被有司之劾,父子塵累而蒙恩貸若曾者也。



 周公弔二季之陵遲,哀大教之不行,於是作謚以紀其終。曾參奉之,啟手歸全,易簀而沒,蓋明慎終,死而後已。齊之史氏,亂世陪臣耳,猶書君賊,累死不懲。況於皇代守典之官,敢畏彊
 盛,而不盡禮。管子有言:「禮義廉恥,是謂四維,四維不張,國乃滅亡。」宰相大臣,人之表儀,若生極其情,死又無貶,是則帝室無正刑也。王公貴人,復何畏哉!所謂四維,復何寄乎!謹按《謚法》:「名與實爽曰繆,怙亂肆行曰醜。」曾之行己,皆與此同,宜謚繆醜公。



 時雖不同秀議,而聞者懼焉。



 秀性忌讒佞,疾之如仇,素輕鄙賈充,及伐吳之役,聞其為大都督,謂所親者曰:「充文案小才,乃居伐國大任,吾將哭以送師。」或止秀曰:「昔蹇叔知秦軍必敗,故哭送其子耳。今吳君無道,國有自亡之形,群率踐境,將不戰而潰。子之哭也,既為不智,乃不赦之罪。」於是乃止。及孫
 皓降于王濬,充未之知,方以吳未可平,抗表請班師。充表與告捷同至,朝野以充位居人上,智出人下,僉以秀為知言。



 及充薨,秀議曰:「充舍宗族弗授,而以異姓為後,悖禮溺情,以亂大倫。昔鄫養外孫莒公子為後,《春秋》書『莒人滅鄫』。聖人豈不知外孫親邪!但以義推之,則無父子耳。又案詔書『自非功如太宰,始封無後如太宰,所取必己自出如太宰,不得以為比』。然則以外孫為後,自非元功顯德,不之得也。天子之禮,蓋可然乎?絕父祖之血食,開朝廷之禍門。《謚法》『昏亂紀度曰荒』,請謚荒公。」不從。



 王濬有平吳之勳,而為王渾所譖毀。帝雖不從,無明賞
 罰,以濬為輔國大將軍,天下咸為之怨。秀乃上言曰:「自大晉啟祚,輔國之號,率以舊恩。此為王濬無功之時,受九列之顯位,立功之後更得寵人之辱號也。四海視之,孰不失望!蜀小吳大,平蜀之後,二將皆就加三事,今濬還而降等,天下安得不惑乎!吳之未亡也,雖以三祖之神武,猶躬受其屈。以孫皓之虛名,足以驚動諸夏,每一小出,雖聖心知其垂亡,然中國輒懷惶怖。當爾時,有能借天子百萬之眾,平而有之,與國家結兄弟之交,臣恐朝野實皆甘之耳。今濬舉蜀、漢之卒,數旬而平吳,雖舉吳人之財寶以與之,本非己分有焉,而遽與計校乎?」



 後與
 劉暾等同議齊王攸事,忤旨,除名。尋復起為博士。秀性悻直,與物多忤。為博士前後垂二十年,卒於官。



 史臣曰:齊獻王以明德茂親,經邦論道,允釐庶績,式敘彞倫。武帝納姦諂之邪謀,懷紹終之遠慮,遂乃君茲青土,作牧東籓。遠邇驚嗟,朝野失望。曹志等服膺教義,方軌儒門,蹇蹇匪躬,慺慺體國。故能抗言鳳闕,忤犯龍鱗,身雖暫屈,道亦弘矣!庾氏世載清德,見稱於世,汝潁之多奇士,斯焉取斯。謀甫素疾佞邪,而發因醉飽,投鼠忌器,豈易由言。竊人之財,猶謂之盜,子玄假譽攘善,將非盜乎!



 贊曰:魏氏維城,濟北知名。潁川多士,峻亦飛英。長岑徇義,祭酒遺榮。謀甫三爵,酗斯作。象既攘善,秀惟癉惡。旉獻嘉謀,幾趨鼎鑊。



\end{pinyinscope}