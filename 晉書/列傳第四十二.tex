\article{列傳第四十二}

\begin{pinyinscope}

 郭璞葛洪



 郭璞,字景純,河東聞喜人也。父瑗,尚書都令史。時尚書杜預有所增損,瑗多駮正之,以公方著稱。終於建平太守。璞好經術,博學有高才,而訥於言論,詞賦為中興之冠。好古文奇字,妙於陰陽算歷。有郭公者,客居河東,精於卜筮,璞從之受業。公以《青囊中書》九卷與之,由是遂洞五行、天文、卜筮之術,攘災轉禍,通致無方,雖京房、管
 輅不能過也。璞門人趙載嘗竊《青襄書》,未及讀,而為火所焚。



 惠懷之際,河東先擾。璞筮之,投策而歎曰:「嗟乎!黔黎將湮於異類,桑梓其翦為龍荒乎!」於是潛結姻暱及交遊數十家,欲避地東南。抵將軍趙固,會固所乘良馬死,固惜之,不接賓客。璞至,門吏不為通。璞曰:「吾能活馬。」吏驚入白固。固趨出,曰:「君能活吾馬乎?」璞曰:「得健夫二三十人,皆持長竿,東行三十里,有丘林社廟者,便以竿打拍,當得一物,宜急持歸。得此,馬活矣。」固如其言,果得一物似猴,持歸。此物見死馬,便噓吸其鼻。頃之馬起,奮迅嘶鳴,食如常,不復見向物。固奇之,厚加資給。



 行至廬
 江,太守胡孟康被丞相召為軍諮祭酒。時江淮清宴,孟康安之,無心南渡。璞為占曰「敗」。康不之信。璞將促裝去之,愛主人婢,無由而得,乃取小豆三斗,繞主人宅散之。主人晨見赤衣人數千圍其家,就視則滅,甚惡之,請璞為卦。璞曰:「君家不宜畜此婢,可於東南二十里賣之,慎勿爭價,則此妖可除也。」主人從之。璞陰令人賤買此婢。復為符投於井中,數千赤衣人皆反縛,一一自投于並,主人大悅。璞攜婢去。後數旬而廬江陷。



 璞既過江,宣城太守殷祐引為參軍。時有物大如水牛,灰色卑腳,腳類象,胸前尾上皆白,大力而遲鈍,來到城下,眾咸異焉。祐
 使人伏而取之,令璞作卦,遇《遁》之《蠱》,其卦曰:「《艮》體連《乾》,其物壯巨。山潛之畜,匪兕匪武。身與鬼並,精見二午。法當為禽,兩靈不許。遂被一創,還其本墅。按卦名之,是為驢鼠。」卜適了,伏者以戟刺之,深尺餘,遂去不復見。郡綱紀上祠,請殺之。巫云:「廟神不悅,曰:『此是共阜亭驢山君鼠,使詣荊山,暫來過我,不須觸之。』」其精妙如此。祐遷石頭督護,璞復隨之。時有鼯鼠出延陵,璞占之曰:「此郡東當有妖人欲稱制者,尋亦自死矣。後當有妖樹生,然若瑞而非瑞,辛螫之木也。儻有此者,東南數百里必有作逆者,期明年矣。」無錫縣欻有茱萸四株交枝而生,若連理
 者,其年盜殺吳興太守袁琇。或以問璞,璞曰:「卯爻發而沴金,此木不曲直而成災也。」王導深重之,引參己軍事。嘗令作卦,璞言:「公有震厄,可命駕西出數十里,得一柏樹,截斷如身長,置常寢處,災當可消矣。」導從其言。數日果震,柏樹粉碎。



 時元帝初鎮鄴,導令璞筮之,遇《咸》之《井》,璞曰:「東北郡縣有『武』名者,當出鐸,以著受命之符。西南郡縣有『陽』名者,井當沸。」其後晉陵武進縣人於田中得銅鐸五枚,歷陽縣中井沸,經日乃止。及帝為晉王,又使璞筮,遇《豫》之《睽》,璞曰:「會稽當出鐘,以告成功,上有勒銘,應在人家井泥中得之。繇辭所謂『先王以作樂崇德,
 殷薦之上帝』者也。」及帝即位,太興初,會稽剡縣人果於井中得一鐘,長七寸二分,口徑四寸半,上有古文奇書十八字,云「會稽嶽命」,餘字時人莫識之。璞曰:「蓋王者之作,必有靈符,塞天人之心,與神物合契,然後可以言受命矣。觀五鐸啟號於晉陵,棧鐘告成於會稽,瑞不失類,出皆以方,豈不偉哉!若夫鐸發其響,鐘徵其象,器以數臻,事以實應,天人之際不可不察。」帝甚重之。



 璞著《江賦》,其辭甚偉,為世所稱。後復作《南郊賦》,帝見而嘉之,以為著作佐郎。於時陰陽錯繆,而刑獄繁興,璞上疏曰:



 臣聞《春秋》之義,貴元慎始,故分至啟閉以觀雲物,所以顯天
 人之統,存休咎之徵。臣不揆淺見,輒依歲首粗有所占,卦得《解》之《既濟》。案爻論思,方涉春木王龍德之時,而為廢水之氣來見乘,加升陽未布,隆陰仍積,《坎》為法象,刑獄所麗,變《坎》加《離》,厥象不燭。以義推之,皆為刑獄殷繁,理有壅濫。又去年十二月二十九日,太白蝕月。月者屬《坎》,群陰之府,所以照察幽情,以佐太陽者也。太白,金行之星,而來犯之,天意若曰刑理失中,自壞其所以為法者也。臣術學庸近,不練內事,卦理所及,敢不盡言。又去秋以來,沈雨跨年,雖為金家涉火之祥,然亦是刑獄充溢,怨歎之氣所致。往建興四年十二月中,行丞相
 令史淳于伯刑於市,而血逆流長標。伯者小人,雖罪在未允,何足感動靈變,致若斯之怪邪!明皇天所以保祐金家,子愛陛下,屢見災異,殷勤無已。陛下宜側身思懼,以應靈譴。皇極之謫,事不虛降。不然,恐將來必有愆陽苦雨之災,崩震薄蝕之變,狂狡蠢戾之妖,以益陛下旰食之勞也。



 臣謹尋按舊經,《尚書》有五事供禦之術,京房易傳有消復之救,所以緣咎而致慶,因異而邁政。故木不生庭,太戊無以隆;雉不鳴鼎,武丁不為宗。夫寅畏者所以饗福,怠傲者所以招患,此自然之符應,不可不察也。案《解卦》繇云:「君子以赦過宥罪。」《既濟》云;「思患而豫防
 之。」臣愚以為宜發哀矜之詔,引在予之責,蕩除瑕釁,贊陽布惠,使幽斃之人應蒼生以悅育,否滯之氣隨谷風而紓散。此亦寄時事以制用,藉開塞而曲成者也。



 臣竊觀陛下貞明仁恕,體之自然,天假其祚,奄有區夏,啟重光於已昧,廓四祖之遐武,祥靈表瑞,人鬼獻謀,應天順時,殆不尚此。然陛下即位以來,中興之化未闡,雖躬綜萬機,勞逾日昃,玄澤未加於群生,聲教未被乎宇宙,臣主未寧於上,黔細未輯於下,《鴻鴈》之詠不興,康衢之歌不作者,何也?杖道之情未著,而任刑之風先彰,經國之略未震,而軌物之迹屢遷。夫法令不一則人情惑,職次
 數改則覬覦生,官方不審則秕政作,懲勸不明則善惡渾,此有國者之所慎也。臣竊為陛下惜之。夫以區區之曹參,猶能遵蓋公之一言,倚清靖以鎮俗,寄市獄以容非,德音不忘,流詠于今。漢之中宗,聰悟獨斷,可謂令主,然厲意刑名,用虧純德。《老子》以禮為忠信之薄,況刑又是禮之糟粕者乎!夫無為而為之,不宰以宰之,固陛下之所體者也。恥其君不為堯舜者,亦豈惟古人!是以敢肆狂瞽,不隱其懷。若臣言可採,或所以為塵露之益;若不足採,所以廣聽納之門。願陛下少留神鑒,賜察臣言。



 疏奏,優詔報之。



 其後日有黑氣,璞復上疏曰:



 臣以頑昧,
 近者冒陳所見,陛下不遺狂言,事蒙御省。伏讀聖詔,歡懼交戰。臣前云升陽未布,隆陰仍積,《坎》為法象,刑獄所麗,變《坎》加《離》,厥象不燭,疑將來必有薄蝕之變也。此月四日,日出山六七丈,精光潛昧,而色都赤,中有異物大如雞子,又有青黑之氣共相薄擊,良久方解。案時在歲首純陽之月,日在癸亥全陰之位,而有此異,殆元首供禦之義不顯,消復之理不著之所致也。計去微臣所陳,未及一月,而便有此變,益明皇天留情陛下懇懇之至也。



 往年歲末,太白蝕月,今在歲始,日有咎謫。會未數旬,大眚再見。日月告釁,見懼詩人,無曰天高,其鑒不遠。故
 宋景言善,熒惑退次;光武寧亂,呼沲結冰。此明天人之懸符,有若形影之相應。應之以德,則休祥臻;酬之以怠,則咎徵作。陛下宜恭承靈譴,敬天之怒,施沛然之恩,諧玄同之化,上所以允塞天意,下所以弭息群謗。



 臣聞人之多幸,國之不幸。赦不宜數,實如聖旨。臣愚以為子產之鑄刑書,非政事之善,然不得不作者,須以救弊故也。今之宜赦,理亦如之。隨時之宜,亦聖人所善者。此國家大信之要,誠非微臣所得干豫。今聖朝明哲,思弘謀猷,方闢四門以亮采,訪輿誦於群心,況臣蒙珥筆朝末,而可不竭誠盡規哉!



 頃之遷尚書郎。數言便宜,多研匡益。
 明帝之在東宮,與溫嶠、庾亮並有布衣之好,璞亦以才學見重,埒於嶠、亮,論者美之。然性輕易,不修威儀,嗜酒好色,時或過度。著作郎干寶常誡之曰:「此非適性之道也。」璞曰:「吾所受有本限,用之恒恐不得盡,卿乃憂酒色之為患乎!」



 璞既好卜筮,縉紳多笑之。又自以才高位卑,乃著《客傲》,其辭曰:



 客傲郭生曰:「玉以兼城為寶,士以知名為賢。明月不妄映,蘭葩豈虛鮮。今足下既以拔文秀於叢薈,蔭弱根於慶雲,陵扶搖而竦翮,揮清瀾以濯鱗,而響不徹於一皋,價不登乎千金。傲岸榮悴之際,頡頏龍魚之間,進不為諧隱,退不為放言,無沈冥之韻,而希
 風乎嚴先,徒費思於贊味,摹《洞林》乎《連山》,尚何名乎!夫攀驪龍之髯,撫翠禽之毛,而不得絕霞肆、跨天津者,未之前聞也。」



 郭生粲然而笑曰:「鷦鷯不可與論雲翼,井蛙難與量海鰲。雖然,將祛子之惑,訊以未悟,其可乎?



 「乃者地維中絕,乾光墜采,皇運暫迴,廓祚淮海。龍德時乘,群才雲駭,藹若鄧林之會逸翰,爛若溟海之納奔濤,不煩咨嗟之訪,不假蒲帛之招,羈九有之奇駿,咸總之於一朝,豈惟豐沛之英,南陽之豪!昆吾挺鋒,驌驦軒髦,杞梓競敷,蘭荑爭翹,嚶聲冠於伐木,援類繁乎拔茅。是以水無浪士,巖無幽人,刈蘭不暇,爨桂不給,安事錯薪乎!



 「且
 夫窟泉之潛不思雲翬,熙冰之采不羨旭晞,混光耀於埃藹者,亦曷願滄浪之深,秋陽之映乎!登降紛於九五,淪湧懸乎龍津。蚓蛾以不才陸槁,蟒蛇以騰騖暴鱗。連城之寶,藏於褐裏,三秀雖艷,糜于麗采。香惡乎芬?賈惡乎在?是以不塵不冥,不驪不騂,支離其神,蕭悴其形。形廢則神王,跡粗而名生。體全者為犧,至獨者不孤,傲俗者不得以自得,默覺者不足以涉無。故不恢心而形遺,不外累而智喪,無巖穴而冥寂,無江湖而放浪。玄悟不以應機,洞鑒不以昭曠。不物物我我,不是是非非。忘意非我意,意得非我懷。寄群籟乎無象,域萬殊於一歸。不
 壽殤子,不夭彭涓,不壯秋豪,不小太山。蚊淚與天地齊流,蜉蝣與大椿齒年。然一闔一開,兩儀之跡,一沖一溢,懸象之節,渙互期於寒暑,凋蔚要乎春秋。青陽之翠秀,龍豹之委穎,駿狼之長暉,玄陸之短景。故皋壤為悲欣之府,胡蝶為物化之器矣。



 「夫欣黎黃之音者,不顰蟪蛄之吟;豁雲臺之觀者,必閟帶索之歡。縱蹈而詠採薺,擁璧而歎抱關。戰機心以外物,不能得意於一弦。悟往復於嗟歎,安可與言樂天者乎!若乃莊周偃蹇於漆園,老萊婆娑於林窟,嚴平澄漠於塵肆,梅真隱淪乎市卒,梁生吟嘯而矯跡,焦先混沌而槁杌,阮公昏酣而賣傲,翟
 叟遁形以倏忽。吾不能歲韻於數賢,故寂然玩此員策與智骨。」



 永昌元年,皇孫生,璞上疏曰:



 有道之君未嘗不以危自持,亂世之主未嘗不以安自居。故存而不忘亡者,三代之所以興也;亡而自以為存者,三季之所以廢也。是以古之令主開納忠讜,以弼其違;標顯切直,用攻其失。至乃聞一善則拜,見規誡則懼。何者?蓋不私其身,處天下以至公也。臣竊惟陛下符運至著,勛業至大,而中興之祚不隆、聖敬之風未躋者,殆由法令太明,刑教太峻。故水至清則無魚,政至察則眾乖,此自然之勢也。



 臣去春啟事,以囹圄充斥,陰陽不和,推之卦理,宜因郊
 祀作赦,以蕩滌瑕穢。不然,將來必有愆陽苦雨之災,崩震薄蝕之變,狂狡蠢戾之妖。其後月餘,日果薄斗。去秋以來,諸郡並有暴雨,水皆洪潦,歲用無年。適聞吳興復欲有構妄者,咎徵漸成,臣甚惡之。頃者以來,役賦轉重,獄犴日結,百姓困擾,甘亂者多,小人愚險,共相扇惑。雖勢無所至,然不可不虞。案《洪範傳》,君道虧則日蝕,人憤怨則水涌益,陰氣積則下代上。此微理潛應已著實於事者也。假令臣遂不幸謬中,必貽陛下側席之憂。



 今皇孫載育,天固靈基,黔首顒顒,實望惠潤。又歲涉午位,金家所忌。宜於此時崇恩布澤,則火氣潛消,災譴不生矣。
 陛下上承天意,下順物情,可因皇孫之慶大赦天下。然後明罰敕法,以肅理官,克厭天心,慰塞人事,兆庶幸甚,禎祥必臻矣。



 臣今所陳,暫而省之,或未允聖旨,久而尋之,終亮臣誠。若所啟上合,願陛下勿以臣身廢臣之言。臣言無隱,而陛下納之,適所以顯君明臣直之義耳。



 疏奏,納焉,即大赦改年。



 時暨陽人任谷因耕息於樹下,忽有一人著羽衣就淫之,既而不知所在,谷遂有娠。積月將產,羽衣人復來,以刀穿其陰下,出一蛇子便去。谷遂成宦者。後詣闕上書,自云有道術。帝留谷于宮中。璞復上疏曰:「任谷所為妖異,無有因由。陛下玄鑒廣覽,欲知
 其情狀,引之禁內,供給安處。臣聞為國以禮正,不聞以奇邪。所聽惟人,故神降之吉。陛下簡默居正,動遵典刑。案《周禮》,奇服怪人不入宮,況谷妖詭怪人之甚者,而登講肆之堂,密邇殿省之側,塵點日月,穢亂天聽,臣之私情竊所以不取也。陛下若以谷信為神靈所憑者,則應敬而遠之。夫神,聰明正直,接以人事。若以谷為妖蠱詐妄者,則當投畀裔土,不宜令褻近紫闈。若以谷或是神祇告譴、為國作眚者,則當克己修禮以弭其妖,不宜令谷安然自容,肆其邪變也。臣愚以為陰陽陶烝,變化萬端,亦是狐狸魍魎憑假作慝。願陛下採臣愚懷,特遣谷
 出。臣以人乏,忝荷史任,敢忘直筆,惟義是規。」其後元帝崩,谷因亡走。



 璞以母憂去職,卜葬地於暨陽,去水百步許。人以近水為言,璞曰:「當即為陸矣。」其後沙漲,去墓數十里皆為桑田。未期,王敦起璞為記室參軍。是時潁川陳述為大將軍掾,有美名,為敦所重,未幾而沒。璞哭之哀甚,呼曰:「嗣祖,嗣祖,焉知非福!」夫幾而敦作難。時明帝即位踰年,未改號,而熒惑守房。璞時休歸,帝乃遣使齎手詔問璞。會暨陽縣復上言曰赤烏見。璞乃上疏請改年肆赦,文多不載。璞嘗為人葬,帝微服往觀之,因問主人何以葬龍角,此法當滅族。主人曰:「郭璞云此葬龍耳,
 不出三年當致天子也。」帝曰:「出天子邪?」答曰:「能致天子問耳。」帝甚異之。璞素與桓彞友善,彞每造之,或值璞在婦間,便入。璞曰:「卿來,他處自可徑前,但不可廁上相尋耳。必客主有殃。」彞後因醉詣璞,正逢在廁,掩而觀之,見璞裸身被髮,銜刀設醊。璞見彞,撫心大驚曰:「吾每屬卿勿來,反更如是!非但禍吾,卿亦不免矣。天實為之,將以誰咎!」璞終嬰王敦之禍,彞亦死蘇峻之難。



 王敦之謀逆也,溫嶠、庾亮使璞筮之,璞對不決。嶠、亮復令占己之吉凶,璞曰:「大吉。」嶠等退,相謂曰:「璞對不了,是不敢有言,或天奪敦魄。今吾等與國家共舉大事,而璞云大吉,是為
 舉事必有成也。」於是勸帝討敦。初,璞每言「殺我者山宗」,至是果有姓崇者構璞於敦。敦將舉兵,又使璞筮。璞曰:「無成。」敦固疑璞之勸嶠、亮,又聞卦凶,乃問璞曰;「卿更筮吾壽幾何?」答曰:「思向卦,明公起事,必禍不久。若住武昌,壽不可測。」敦大怒曰:「卿壽幾何?」曰:「命盡今日日中。」敦怒,收璞,詣南岡斬之。璞臨出,謂行刑者欲何之。曰:「南岡頭。」璞曰:「必在雙柏樹下。」既至,果然。復云:「此樹應有大鵲巢。」眾索之不得。璞更令尋覓,果於枝間得一大鵲巢,密葉蔽之。初,璞中興初行經越城,間遇一人,呼其姓名,因以褲褶遺之。其人辭不受,璞曰:「但取,後自當知。」其人遂受
 而去。至是,果此人行刑。時年四十九。及王敦平,追贈弘農太守。



 初,庾翼幼時嘗令璞筮公家及身,卦成,曰:「建元之末丘山傾,長順之初子凋零。」及康帝即位,將改元為建元,或謂庾冰曰:「子忘郭生之言邪?丘山上名,此號不宜用。」冰撫心歎恨。及帝崩,何充改元為永和,庾翼歎曰:「天道精微,乃當如是。長順者,永和也,吾庸得免乎!」其年翼卒。冰又令筮其後嗣,卦成,曰:「卿諸子並當貴盛,然有白龍者,凶徵至矣。若墓碑生金,庾氏之大忌也。」後冰子蘊為廣州刺史,妾房內忽有一新生白狗子,莫知所由來,其妾祕愛之,不令蘊知。狗轉長大,蘊入,是狗眉眼分
 明,又身至長而弱,異於常狗,蘊甚怪之。將出,共視在眾人前,忽失所在。蘊慨然曰:「殆白龍乎!庾氏禍至矣。」又墓碑生金。俄而為桓溫所滅,終如其言。璞之占驗,皆如此類也。



 璞撰前後筮驗六十餘事,名為《洞林》。又抄京、費諸家要最,更撰《新林》十篇、《卜韻》一篇。注釋《爾雅》,別為《音義》、《圖譜》。又注《三蒼》、《方言》、《穆天子傳》、《山海經》及《楚辭》、《子虛》、《上林賦》數十萬言,皆傳於世。所作詩賦誄頌亦數萬言。子驁,官至臨賀太守。



 葛洪,字稚川,丹陽句容人也。祖系,吳大鴻臚。父悌,吳平
 後入晉,為邵陵太守。洪少好學,家貧,躬自伐薪以貿紙筆,夜輒寫書誦習,遂以儒學知名。性寡欲,無所愛玩,不知棋局幾道,摴蒱齒名。為人木訥,不好榮利,閉門卻掃,未嘗交游。於餘杭山見何幼道、郭文舉,目擊而已,各無所言。時或尋書問義,不遠數千里崎嶇冒涉,期於必得,遂究覽典籍,尤好神仙導養之法。從祖玄,吳時學道得仙,號曰葛仙公,以其練丹祕術授弟子鄭隱。洪就隱學,悉得其法焉。後師事南海太守上黨鮑玄。玄亦內學,逆占將來,見洪深重之,以女妻洪。洪傳玄業,兼綜練醫術,凡所著撰,皆精核是非,而才章富贍。



 太安中,石冰作亂,吳
 興太守顧秘為義軍都督,與周等起兵討之,秘檄洪為將兵都尉,攻冰別率,破之,遷伏波將軍。冰平,洪不論功賞,徑至洛陽,欲搜求異書以廣其學。



 洪見天下已亂,欲避地南土,乃參廣州刺史嵇含軍事。及含遇害,遂停南土多年,征鎮檄命一無所就。後還鄉里,禮辟皆不赴。元帝為丞相,辟為掾。以平賊功,賜爵關內侯。咸和初,司徒導召補州主簿,轉司徒掾,遷諮議參軍。干寶深相親友,薦洪才堪國史,選為散騎常侍,領大著作,洪固辭不就。以年老,欲練丹以祈遐壽,聞交阯出丹,求為句漏令。帝以洪資高,不許。洪曰:「非欲為榮,以有丹耳。」帝從之。洪
 遂將子姪俱行。至廣州,刺史鄧嶽留不聽去,洪乃止羅浮山煉丹。嶽表補東官太守,又辭不就。嶽乃以洪兄子望為記室參軍。在山積年,優游閑養,著述不輟。其自序曰:



 洪體乏進趣之才,偶好無為之業。假令奮翅則能陵厲玄霄,騁足則能追風躡景,猶欲戢勁翮於於鷦鷃之群,藏逸迹於跛驢之伍,豈況大塊稟我以尋常之短羽,造化假我以至駑之蹇足?自卜者審,不能者止,又豈敢力蒼蠅而慕沖天之舉,策跛鱉而追飛兔之軌;飾嫫母之篤陋,求媒陽之美談;推沙礫之賤質,索千金於和肆哉!夫僬僥之步而企及夸父之蹤,近才所以躓礙也;要離
 之羸而強赴扛鼎之勢,秦人所以斷筋也。是以望絕於榮華之途,而志安乎窮圮之域;藜藿有八珍之甘,蓬蓽有藻棁之樂也。故權貴之家,雖咫尺弗從也;知道之士,雖艱遠必造也。考覽奇書,既不少矣,率多隱語,難可卒解,自非至精不能尋究,自非篤勤不能悉見也。



 道士弘博洽聞者寡,而意斷妄說者眾。至於時有好事者,欲有所修為,倉卒不知所從,而意之所疑又無足諮。今為此書,粗舉長生之理。其至妙者不得宣之於翰墨,蓋粗言較略以示一隅,冀悱憤之徒省之可以思過半矣。豈謂闇塞必能窮微暢遠乎,聊論其所先覺者耳。世儒徒知服
 膺周孔,莫信神仙之書,不但大而笑之,又將謗毀真正。故予所著子言黃白之事,名曰《內篇》,其餘駮難通釋,名曰《外篇》,大凡內外一百一十六篇。雖不足藏諸名山,且欲緘之金匱,以示識者。



 自號抱朴子,因以名書。其餘所著碑誄詩賦百卷,移檄章表三十卷,神仙、良吏、隱逸、集異等傳各十卷,又抄《五經》、《史》、《漢》、百家之言、方技雜事三百一十卷,《金匱藥方》一百卷,《肘後要急方》四卷。



 洪博聞深洽,江左絕倫。著述篇章富於班馬,又精辯玄賾,析理入微。後忽與嶽疏云:「當遠行尋師,剋期便發。」嶽得疏,狼狽往別。而洪坐至日中,兀然若睡而卒,嶽至,遂不及見。
 時年八十一。視其顏色如生,體亦柔軟,舉尸入棺,甚輕,如空衣,世以為尸解得仙云。



 史臣曰:景純篤志綈緗,洽聞彊記,在異書而畢綜,瞻往滯而咸釋;情源秀逸,思業高奇;襲文雅於西朝,振辭鋒於南夏,為中興才學之宗矣。夫語怪徵神,伎成則賤,前修貽訓,鄙乎茲道。景純之探策定數,考往知來,邁京管於前圖,軼梓窀於遐篆。而宦微於世,禮薄於時,區區然寄《客傲》以申懷,斯亦伎成之累也。若乃大塊流形,玄天賦命,吉凶修短,定乎自然。雖稽象或通,而厭勝難恃,稟之有在,必也無差,自可居常待終,頹心委運,何至銜刀
 被發,遑遑於穢向之間哉!晚抗忠言,無救王敦之逆;初慚智免,竟斃「山宗」之謀。仲尼所謂攻乎異端,斯害也已,悲夫!稚川束發從師,老而忘倦。紬奇冊府,總百代之遺編;紀化仙都,窮九丹之秘術。謝浮榮而捐雜藝,賤尺寶而貴分陰,游德棲真,超然事外。全生之道,其最優乎!



 贊曰:景純通秀,夙振宏材。沈研鳥冊,洞曉龜枚。匪寧國釁,坐致身災。稚川優洽,貧而樂道。載範斯文,永傳洪藻。



\end{pinyinscope}