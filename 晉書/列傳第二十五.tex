\article{列傳第二十五}

\begin{pinyinscope}
夏侯湛
 \gezhu{
  弟淳淳子承}
 潘岳
 \gezhu{
  從子尼}
 張載
 \gezhu{
  弟協協弟亢}



 夏侯湛,字孝若,譙國譙人也。祖威,魏兗州刺史。父莊,淮南太守。湛幼有盛才,文章宏富,善構新詞,而美容觀,與潘岳友善,每行止同輿接茵,京都謂之「連璧」。



 少為太尉掾。泰始中,舉賢良,對策中第,拜郎中,累年不調,乃作《抵疑》以自廣。其辭曰:



 當路子有疑夏侯湛者而謂之曰:「吾聞有其才而不遇者,時也;有其時而不遇者,命也。吾子
 童幼而岐立,弱冠而著德,少而流聲,長而垂名。拔萃始立,而登宰相之朝;揮翼初儀,而受卿尹之舉。盪典籍之華,談先王之言。入閶闔,躡丹墀,染彤管,吐洪煇,乾當世之務,觸人主之威,有效矣。而官不過散郎,舉不過賢良。鳳棲五期,龍蟠六年,英耀禿落,羽儀摧殘。而獨雍容藝文,蕩駘儒林,志不轟著述之業,口不釋《雅》《頌》之音,徒費情而耗力,勞神而苦心,此術亦以薄矣。而終莫之辯,宜吾子之陸沈也。且以言乎才,則吾子優矣。以言乎時,則子之所與二三公者,義則骨肉之固,交則明道之觀也。富於德,貴於官,其所發明,雖叩牛操築之客,傭賃拘關
 之隸,負俗懷譏之士,猶將登為大夫,顯為卿尹。於何有寶咳唾之音,愛錙銖之力?向若垂一鱗,迴一翼,令吾子攀其飛騰之勢,挂其羽翼之末,猶奮迅於雲霄之際,騰驤於四極之外。今乃金口玉音,漠然沈默。使吾子棲遲窮巷,守此困極,心有窮志,貌有饑色。吝江河之流,不以濯舟船之畔;惜東壁之光,不以寓貧婦之目。抑非二三公之蔽賢也,實吾子之拙惑也。」



 夏侯子曰:「噫!湛也幸,有過,人必知之矣。吾子所以褒飾之太矣。斟酌之喻,非小醜之所堪也。然過承古人之誨,抑因子大夫之忝在弊室也,敢布其腹心,豈能隱几以覽其概乎!」



 客曰:「敢祗以
 聽。」



 夏侯子曰:「吾聞先大夫孔聖之言:『德之不修,學之不講,聞義不能徙,不善不能改,是吾憂也。』四德具而名位不至者,非吾任也。是以君子求諸己,小人求諸人。僕也承門戶之業,受過庭之訓,是以得接冠帶之末,充乎士大夫之列,頗窺《六經》之文,覽百家之學。弱年而入公朝,蒙蔽而當顯舉,進不能拔群出萃,卻不能抗排當世,志則乍顯乍昧,文則乍幽乍蔚。知之者則謂之欲逍遙以養生,不知之者則謂之欲遑遑以求達,此皆未是僕之所匱也。



 僕又聞,世有道,則士無所執其節;黜陟明,則下不在量其力。是以當舉而不辭,入朝而酬問。僕,東野之
 鄙人,頑直之陋生也。不識當世之便,不達朝廷之情,不能倚靡容悅,出入崎傾,逐巧點妍,嘔喁辯佞。隨群班之次,伏簡墨之後。當此之時,若失水之魚,喪家之狗,行不勝衣,言不出口,安能幹當世之務,觸人主之威,適足以露狂簡而增塵垢。縱使心有至言,言有偏直,此委巷之誠,非朝廷之欲也。



 今天子以茂德臨天下,以八方六合為四境,海內無虞,萬國玄靜,九夷之從王化,猶洪聲之收清響;黎苗之樂函夏,若遊形之招惠景。鄉曲之徒,一介之士,曾諷《急就》、習甲子者,皆奮筆揚文,議制論道。出草苗,起林藪,御青瑣,入金墉者,無日不有。充三臺之寺,
 盈中書之閣。有司不能竟其文,當年不能編其籍,此執政之所厭聞也。若乃群公百辟,卿士常伯,被朱佩紫,耀金帶白,坐而論道者,又充路盈寢,黃幄玉階之內,飽其尺牘矣。若僕之言,皆糞土之說,消磨灰爛,垢辱招穢,適可充衛士之爨,盈掃除之器。譬猶投盈寸之膠,而欲使江海易色;燒一羽之毛,而欲令大爐增勢。若燎原之煙,彌天之雲,噓之不益其熱,翕之不減其氣。今子見僕入朝暫對,便欲坐望高位,吐言數百,謂陵曾一世,何吾子之失評也!僕固脂車以須放,秣馬以待卻,反耕於枳落,歸志乎渦瀨,從容乎農夫,優游乎卒歲矣。



 古者天子畫土
 以封群后,群后受國以臨其邦,懸大賞以樂其成,列九伐以討其違,興衰相形,安危相傾。故在位者以求賢為務,受任者以進才為急。今也則九州為一家,萬國為百郡,政有常道,法有恒訓,因循而禮樂自定,揖讓而天下大順。夫道學之貴游,閭邑之搢紳,皆高門之子,世臣之胤,弘風長譽,推成而進,悠悠者皆天下之彥也。諷詁訓,傳《詩》《書》,講儒墨,說玄虛,僕皆不如也。二三公之簡僕於凡庸之肆,顯僕於細猥之中,則為功也重矣;時而清談,則為親也周矣。且古之君子,不知士,則不明不安。是以居逸而思危,對食而肴乾。今也則否。居位者以善身為
 靜,以寡交為慎,以弱斷為重,以怯言為信。不知士者無公誹,不得士者不私愧。彼在位者皆稷、契、咎、益、伊、呂、周、召之倫,叔豹、仲熊之儔,稽古則踰黃、唐,經緯則越虞、夏,蔑昆吾之功,嗤桓文之勳,抵管仲,蹉雹晏嬰。其遠則欲升鼎湖,近則欲超太平。方將保保重嗇神,獨善其身,玄白沖虛,仡爾養真。雖力挾太山,將不舉一羽;揚波萬里,將不濯一鱗。咳唾成珠玉,揮袂出風雲。豈肯𧾷敝𧾷薜鄙事,取才進人,此又吾子之失言也。子獨不聞夫神人乎!噏風飲露,不食五穀。登太清,遊山嶽,靡芝草,弄白玉。不因而獨備,無假而自足。不與人路同嗜欲,不與世務齊榮辱。故能入無
 窮之門,享不死之年。以此言之,何待進賢!」



 客曰:「聖人有言曰:『邦有道,貧且賤焉,恥也。』今子值有道之世,當太平之會,不攘袂奮氣,發謀出奇。使鳴鶴受和,好爵見縻。抑乃沈身郎署,約志勤卑,不亦羸哉!且伊尹之干成湯,寧戚之迕桓公,或投己鼎俎,或庸身飯牛,明廢興之機,歌《白水》之流,德入殷王,義感齊侯。故伊尹起庖廚而登阿衡,寧戚出車下而階大夫。外無微介,內無請謁,矯身擢手,徑躡名位。吾子亦何不慕賢以自厲,希古以慷慨乎!」



 夏侯子曰:「嗚呼!是何言歟!富與貴是人之所欲,非僕之所惡也。夫干將之劍,陸斷狗馬,水截蛟龍,而金公刀不能
 入泥。騏驥驊騮之乘,一日而致千里,而駑蹇不能邁畝。百煉之監,別鬚眉之數,而壁土不見泰山。鴻鵠一舉,橫四海之區,出青雲之外,而尺鷃不陵桑榆。此利鈍之覺,優劣之決也,夫欲進其身者,不過千萬乘,而僕以上朝堂,答世問,不過顯所知。僕以竭心思,盡才學,意無雅正可準,論無片言可採,是以頓於鄙劣而莫之能起也。以此言之,僕何為其不自衒哉!子不嫌僕德之不劭,而疑其位之不到,是猶反鏡而索照,登木而下釣,僕未以此為不肖也。



 若乃伊尹負鼎以干湯,呂尚隱遊以徼文,傅說操築以寤主,寧戚擊角以要君,此非僕所能也。莊周
 駘蕩以放言,君平賣卜以自賢,接輿陽狂以蔽身,梅福棄家以求仙,此又非僕之所安也。若乃季札抗節於延陵,楊雄覃思於《太玄》,伯玉和柔於人懷,柳惠三絀於士官,僕雖不敏,竊頗仿佛其清塵。」



 後選補太子舍人,轉尚書郎,出為野王令。以血阜隱為急,而緩於公調。政清務閑,優游多暇,乃作《昆弟誥》。其辭曰:



 惟正月才生魄,湛若曰:「咨爾弟淳、琬、瑫、謨、總、瞻:古人有言,『孝乎惟孝,友于兄弟。』『死喪之戚,兄弟孔懷。』又曰,『周之有至德也,莫如兄弟。』於戲!古之載于訓籍,傳于《詩》《書》者,厥乃不思,不可不行。爾其專乃心,一乃聽,砥礪乃性,以聽我之格言。」淳等拜
 手稽首。



 湛若曰:「嗚呼!惟我皇乃祖滕公,肇釐厥德厥功,以左右漢祖,弘濟于嗣君,用垂祚于後。世世增敷前軌,濟其好行美德。明允相繼,冠冕胥及。以逮于皇曾祖愍侯,寅亮魏祖,用康乂厥世,遂啟土宇,以大綜厥勳于家。我皇祖穆侯,崇厥基以允釐顯志,用恢闡我令業。維我后府君侯,祗服哲命,欽明文思,以熙柔我家道,丕隆我先緒。欽若稽古訓,用敷訓典籍,乃綜其微言。嗚呼!自三墳、五典、八索、九丘,圖緯六藝,及百家眾流,罔不探賾索隱,鉤深致遠。《洪範》九疇,彝倫攸敘。乃命世立言,越用繼尼父之大業,斯文在茲。且九齡而我王母薛妃登遐,我
 后孝思罔極,惟以奉于穆侯之繼室蔡姬,以致其子道。蔡姬登遐,隘于穆侯之命,厥禮乃不得成,用不祔于祖姑。惟乃用騁其永慕,厥乃以疾辭位,用遜于厥家,布衣席稿,以終于三載。厥乃古訓無文,我后丕孝其心,用假于厥制,以穆于世父使君侯。惟伯后聰明睿智,奕世載德,用慈友于我后。我惟烝烝是虔,罔不克承厥誨,用增茂我敦篤,以播休美于一世,厥乃可不遵。惟我用夙夜匪懈,日鑽其道,而仰之彌高,鑽之彌堅,我用欲罷不敢。豈唯予躬是懼,實令跡是奉。厥乃晝分而食,夜分而寢。豈唯令跡是畏,實爾猶是儀。嗚呼,予其敬哉!俞!予聞之,
 周之有至德,有婦人焉。我母氏羊姬,宣慈愷悌,明粹篤誠,以撫訓群子。厥乃我齔齒,則受厥教於書學,不遑惟寧。敦《詩》《書》禮樂,孳孳弗倦。我有識惟與汝服厥誨,惟仁義惟孝友是尚,憂深思遠,祗以防於微。翳義形於色,厚愛平恕,以濟其寬裕。用緝和我七子,訓諧我五妹。惟我兄弟姊妹束修慎行,用不辱於冠帶,實母氏是憑。予其為政蕞爾,惟母氏仁之不行是戚,予其望色思寬。獄之不情,教之不泰是訓,予其納戒思詳。嗚呼!惟母氏信著于不言,行感于神明。若夫恭事于蔡姬,敦穆于九族,乃高于古之人。古之人厥乃千里承師,矧我惟父惟母世
 德之餘烈,服膺之弗可及,景仰之弗可階。汝其念哉!俾群弟天祚于我家,俾爾咸休明是履。淳英哉文明柔順,琬乃沈毅篤固,惟瑫厥清粹平理,謨茂哉人雋哲寅亮,總其弘肅簡雅,瞻乃純鑠惠和。惟我蒙蔽,極否于義訓。嗟爾六弟,汝其滋義洗心,以補予之尤。予乃亦不敢忘汝之闕。嗚呼!小子瞻,汝其見予之長於仁,未見予之長於義也。」



 瞻曰:「俞!以如何?」湛若曰:「我之肇于總角,以逮于弱冠,暨于今之二毛,受學于先載,納誨于嚴父慈母。予其敬忌于厥身,而匡予之纖介,翼予之小疵,使予有過未曾不知,予知之逌改,惟沖子是賴。予親于心,愛于中,敬
 于貌。厥乃口無擇言,柔惠且直,廉而不劌,肅而不厲,厥其成予哉。用集我父母之訓,庶明厲翼,邇可遠在茲。」瞻拜手稽首曰:「俞!」湛曰:「都!在修身,在愛人。」瞻曰:「吁!惟聖其難之。」湛曰:「都!厥不行惟難,厥行惟易。」



 淳曰:「俞!明而昧,崇而卑,沖而恒,顯而賢,同而疑,厲而柔,和而矜。」湛曰:「俞!乃言厥有道。」淳曰:「俞!祗服訓。」湛曰:「來!琬,汝亦昌言。」琬曰:「俞!身不及於人,不敢墮於勤,厥故維新。」湛曰:「俞!瑫亦昌言。」瑫曰:「俞!滋敬于己,不滋敬于己,惟敬乃恃,無忘有恥。」湛曰:「俞!謨亦昌言。」謨曰:「俞!無忘於不可不虞,形貌以心,訪心於虞。」湛曰:「俞!總亦昌言。」總曰:「俞!若憂厥憂以休。」湛曰:「
 俞!瞻亦昌言。」瞻曰:「俞!復外惟內,取諸內,不忘諸外。」湛曰:「俞!休哉」淳等拜手稽首,湛亦拜手稽首。乃歌曰:「明德復哉,家道休哉,世祚悠哉,百祿周哉!」又作歌曰:「訊德恭哉,訓翼從哉,內外康哉!」皆拜曰:「欽哉!」



 居邑累年,朝野多歎其屈。除中書侍郎,出補南陽相。遷太子僕,未就命,而武帝崩。惠帝即位,以為散騎常侍。元康初,卒,年四十九。著論三十餘篇,別為一家之言。



 初,湛作《周詩》成,以示潘岳。岳曰:「此文非徒溫雅,乃別見孝弟之性。」岳因此遂作《家風詩》。



 湛族為盛門,性頗豪侈,侯服玉食,窮滋極珍。及將沒,遺命小棺薄斂,不修封樹。論者謂湛雖生不砥礪名
 節,死則儉約令終,是深達存亡之理。



 淳字孝沖。亦有文藻,與湛俱知名。官至弋陽太守。遭中原傾覆,子姪多沒胡寇,唯息承渡江。



 承字文子。參安東軍事,稍遷南平太守。太興末,王敦舉兵內向,承與梁州刺史甘卓、巴東監軍柳純、宜都太守譚該等,並露檄遠近,列敦罪狀。會甘卓懷疑不進,王師敗績,敦悉誅滅異己者,收承,欲殺之,承外兄王暠苦請得免。尋為散騎常侍。



 潘岳,字安仁,滎陽中牟人也。祖瑾,安平太守。父芘,瑯邪
 內史。岳少以才穎見稱,鄉邑號為奇童,謂終賈之儔也。早辟司空太尉府,舉秀才。



 泰始中,武帝躬耕藉田,岳作賦以美其事,曰:


伊晉之四年正月丁未,皇帝親率群后藉于千畝之甸,禮也。於是乃使甸師清畿,野廬掃路,封人壝宮,掌舍設枑。青壇鬱其嶽立兮,翠幕黕以雲布。結崇基之靈阯兮,啟四塗之廣阼。沃野墳腴,膏壤平砥。清洛濁渠,引流激水。遐阡繩直,邇陌如矢。
 \gezhu{
  艸}
 犗服于縹軛兮,紺轅綴於黛耜。儼儲駕於廛左兮,俟萬乘之躬履。百僚先置,位以職分,自上下下,具惟命臣。襲春服之萋萋兮,接游車之轔轔。微風生於輕幰兮,纖埃起乎朱輪。森
 奉璋以階列兮,望皇軒而肅震。若湛露之晞朝陽兮,眾星之拱北辰也。



 於是前驅魚麗,屬車鱗萃,閶闔洞啟,參途方駟,常伯陪乘,太僕執轡。后妃獻穜[QXDV]之種,司農撰播殖之器,挈壺掌升降之節,宮正設門閭之蹕。天子乃御玉輦,蔭華蓋,衝牙錚鎗,綃紈綷糸蔡。金根照耀以烱晃兮,龍驥騰驤而沛艾。表朱玄於離坎兮,飛青縞於震兌。中黃曄以發輝兮,方彩紛其繁會。五路嗚鑾,九旗揚旆,瓊鈒入蘂,雲罕晻藹。簫管嘲哲以啾嘈兮,鼓鼙硡急以砰蓋,筍虡嶷以軒翥兮,洪鐘越乎區外。震震填填,塵霧連天,以幸乎藉田。蟬冕熲以灼灼兮,碧色肅其芊芊。似
 夜光之剖荊璞兮,若茂松之依山顛也。



 於是我皇乃降靈壇,撫御耦,游場染屨,洪縻在手。三推而舍,庶人終畝。貴賤以班,或五或九。于斯時也,居靡都鄙,人無華裔,長幼雜遝以交集,士女頒斌而咸戾。被褐振裾,垂髫總髻,躡踵側肩,掎裳連襼。黃塵為之四合兮,陽光為之潛翳。動容發音而觀者,莫不抃舞乎康衢,謳吟乎聖世。情欣樂乎昏作兮,慮盡力乎樹藝。靡誰督而常勤兮,莫之課而自厲。躬先勞而悅使兮,豈嚴刑而猛制哉!



 有邑老田父,或進而稱曰:「蓋損益隨時,理有常然。高以下為基,人以食為天。正其末者端其本,善其後者慎其先。夫九土之
 宜弗任,四業之務不壹,野有菜蔬之色,朝乏代耕之秩。無儲蓄以虞災,徒望歲以自必。三代之衰,皆此物也。今聖上昧旦丕顯,夕惕若慄,圖匱於豐,防儉於逸,欽哉欽哉,惟穀之恤。展三時之弘務,致倉廩於盈溢,固堯、湯之用心,而存救之要術也。」若乃廟祧有事,祝宗諏日,簠簋普淖,則此之自實,縮鬯蕭茅,又於是乎出。黍稷馨香,旨酒嘉栗。宜其時和年登,而神降之吉也。古人有言曰:「聖人之德,無以加於孝乎!」夫孝者,天之性、人之所由靈也。昔者明王以孝治天下,其或繼之者,鮮哉希矣!逮我皇晉,實光斯道,儀刑孚于萬國,愛敬盡於祖考。故躬稼以
 供粢盛,所以致孝也;勸穡以足百姓,所以固本也。能本而孝,盛德大業至矣哉!此一役也,二美顯焉,不亦遠乎,不亦重平!敢作頌曰:



 「思樂甸畿,薄採其芳。大君戾止,言藉其農。其農三推,萬國以祗。耨我公田,遂及我私。我簠斯盛,我簋斯齊。我倉如陵,我庾如坻。念茲在茲,永言孝思。人力普存,祝史正辭。神只攸歆,逸豫無期。一人有慶,兆民賴之。」



 岳才名冠世,為眾所疾,遂棲遲十年。出為河陽令,負其才而鬱鬱不得志。時尚書僕射山濤、領吏部王濟、裴楷等並為帝所親遇,岳內非之,乃題閣道為謠曰:「閣道東,有大牛。王濟鞅,裴楷鞧,和嶠刺促不得休。」



 轉
 懷令。時以逆旅逐末廢農,奸淫亡命,多所依湊,敗亂法度,敕當除之。十里一官樆,使老小貧戶守之,又差吏掌主,依客舍收錢。岳議曰:



 「謹案:逆旅,久矣其所由來也。行者賴以頓止,居者薄收其直,交易貿遷,各得其所。官無役賦,因人成利,惠加百姓而公無末費。語曰:『許由辭帝堯之命,而舍於逆旅。』《外傳》曰:『晉陽處父過寧,舍於逆旅。』魏武皇帝亦以為宜,其詩曰:『逆旅整設,以通商賈。』然則自堯到今,未有不得客舍之法。唯商鞅尤之,固非聖世之所言也。方今四海會同,九服納貢,八方翼翼,公私滿路。近畿輻輳,客舍亦稠。冬有溫廬,夏有涼蔭,芻秣成行,器用
 取給。疲牛必投,乘涼近進,發槅寫鞍,皆有所憩。



 又諸劫盜皆起於迥絕,止乎人眾。十里蕭條,則奸軌生心;連陌接館,則寇情震懾。且聞聲有救,已發有追,不救有罪,不追有戮,禁暴捕亡,恒有司存。凡此皆客舍之益,而官樆之所乏也。又行者貪路,告糴炊爨,皆以昏晨。盛夏晝熱,又兼星夜,既限早閉,不及樆門。或避晚關,迸逐路隅,祇是慢藏誨盜之原。茍以客舍多敗法教,官守棘樆,獨復何人?彼河橋、孟津,解券輸錢,高第督察,數入校出,品郎兩岸相檢,猶懼或失之。故懸以祿利,許以功報。今賤吏疲人,獨專樆稅,管開閉之權,藉不校之勢,此道路之蠹,
 奸利所殖也。率歷代之舊俗,獲行留之懽心,使客舍灑掃,以待征旅擇家而息,豈非眾庶顒顒之望。」



 請曹列上,朝廷從之。



 岳頻宰二邑,勤於政績。調補尚書度支郎,遷廷尉評,以公事免。楊駿輔政,高選吏佐,引岳為太傅主簿。駿誅,除名。初,譙人公孫宏少孤貧,客田於河陽,善鼓琴,頗能屬文。岳之為河陽令,愛其才藝,待之甚厚。至是,宏為楚王瑋長史,專殺生之政。時駿綱紀皆當從坐,同署主簿朱振已就戮。岳其夕取急在外,宏言之瑋,謂之假吏,故得免。未幾,選為長安令,作《西征賦》,述所經人物山水,文清旨詣,辭多不錄。徵補博士,未召,以母疾輒去,官
 免。尋為著作郎,轉散騎侍郎,遷給事黃門侍郎。



 岳性輕躁,趨世利,與石崇等諂事賈謐,每候其出,與崇輒望塵而拜。構愍懷之文,岳之辭也。謐二十四友,岳為其首。謐《晉書》限斷,亦岳之辭也。其母數誚之曰:「爾當知足,而乾沒不已乎?」而岳終不能改。



 既仕宦不達,乃作《閑居賦》曰:



 岳讀《汲黯傳》至司馬安四至九卿,而良史書之,題以巧宦之目,未曾不慨然廢書而歎也。曰:嗟乎!巧誠有之,拙亦宜然。顧常以為士之生也,非至聖無軌微妙玄通者,則必立功立事,效當年之用。是以資忠履信以進德,修辭立誠以居業。僕少竊鄉曲之譽,忝司空太尉之命,所奉之主,即太宰魯
 武公其人也。舉秀才為郎。逮事世祖武皇帝,為河陽、懷令,尚書郎,廷尉評。今天子諒暗之際,領太傅主簿。府主誅,除名為民。俄而復官,除長安令。遷博士,未召拜,親疾輒去,官免。自弱冠涉于知命之年,八徙官而一進階,再免,一除名,一不拜職,遷者三而已矣。雖通塞有遇,抑亦拙之效也。昔通人和長輿之論餘也,固曰「拙於用多」。稱多者,吾豈敢;言拙,則信而有征。方今俊乂在官,百工惟時,拙者可以絕意乎寵榮之事矣。太夫人在堂,有羸老之疾,尚何能違膝下色養,而屑屑從斗筲之役?於是覽止足之分,庶浮雲之志,築室種樹,逍遙自得。池沼足以漁
 釣,舂稅足以代耕。灌園鬻蔬,供朝夕之膳;牧羊酤酪,俟伏臘之費。孝乎惟孝,友于兄弟,此亦拙者之為政也。乃作《閑居賦》以歌事遂情焉。其辭曰:



 遨墳素之長圃,步先哲之高衢。雖吾顏之云厚,猶內愧於寧蘧。有道餘不仕,無道吾不愚。何巧智之不足,而拙艱之有餘也!於是退而閑居,于洛之涘。身齊逸民,名綴下士。背京溯伊,面郊後市。浮梁黝以逕度,靈臺傑其高峙。窺天文之秘奧,睹人事之終始。其西則有元戎禁營,玄幕綠徽,谿子巨黍,異絭同歸,炮石雷駭,激矢虻飛,以先啟行,耀我皇威。其東則有明堂辟雍,清穆敞閑,環林縈映,圓海回泉,聿追
 孝以嚴父,宗文考以配天,祗聖敬以明順,養更老以崇年。若乃背冬涉春,陰謝陽施,天子有事于柴燎,以郊祖而展義,張鈞天之廣樂,備千乘之萬騎,服棖棖以齊玄,管啾啾而並吹,煌煌乎,隱隱乎,茲禮容之壯觀,而王制之巨麗也。兩學齊列,雙宇如一,右延國胄,左納良逸。祁祁生徒,濟濟儒術,或升之堂,或入之室。教無常師,道在則是。故髦士投紱,名王懷璽,訓若風行,應猶草靡。此里仁所以為美,孟母所以三徙也。



 爰定我居,築室穿池,長楊映沼,芳枳樹樆,遊鱗瀺灂,菡萏敷披,竹木蓊藹,靈果參差。張公大谷之梨,梁侯烏椑之柿,周文弱枝之棗,
 房陵朱仲之李,靡不畢植。三桃表櫻胡之別,二奈耀丹白之色,石榴蒲桃之珍,磊落蔓延乎其側。梅杏郁棣之屬,繁榮藻麗之飾,華實照爛,言所不能極也。菜則蔥韭蒜芋,青筍紫姜,堇薺甘旨,蓼荾芬芳,蘘荷依陰,時藿向陽,綠葵含露,白薤負霜。



 於是凜秋暑退,熙春寒往,微雨新晴,六合清朗。太夫人乃御版輿,升輕軒,遠覽王畿,近周家園,體以行和,藥以勞宣,常膳載加,舊痾有痊。於是席長筵,列孫子柳垂蔭,車結軌,陸摘紫房,水掛赬鯉,或宴于林,或禊于汜。昆弟斑白,兒童稚齒,稱萬壽以獻觴,咸一懼而一喜。壽觴舉,慈顏和,浮盃樂飲,絲竹駢羅,頓
 足起舞,抗音高歌,人生安樂,孰知其他。退求已而自省,信用薄而才劣。奉周任之格言,敢陳力而就列。幾陋身之不保,而奚擬乎明哲,仰眾妙而絕思,終優游以養拙。



 初,芘為瑯邪內史,孫秀為小史給岳,而狡黠自喜。岳惡其為人,數撻辱之,秀常銜忿。及趙王倫輔政,秀為中書令。岳於省內謂秀曰:「孫令猶憶疇昔周旋不?」答曰:「中心藏之,何日忘之!」岳於是自知不免。俄而秀遂誣岳及石崇、歐陽建謀奉淮南王允、齊王冏為亂,誅之,夷三族。岳將詣市,與母別曰:「負阿母!」初被收,俱不相知,石崇已送在市,岳後至,崇謂之曰:「安仁,卿亦復爾邪!」岳曰:「可謂白
 首同所歸。」岳《金谷詩》云:「投分寄石友,白首同所歸。」乃成其讖。岳母及兄侍御史釋、弟燕令豹、司徒掾據、據弟詵,兄弟之子,己出之女,無長幼一時被害,唯釋子伯武逃難得免。而豹女與其母相抱號呼不可解,會詔原之。



 岳美姿儀,辭藻絕麗,尤善為哀誄之文。少時常挾彈出洛陽道,婦人遇之者,皆連手縈繞,投之以果,遂滿車而歸。時張載甚醜,每行,小兒以瓦石擲之,委頓而反。岳從子尼。



 尼字正叔。祖勖,漢東海相。父滿,平原內史。並以學行稱。尼少有清才,與岳俱以文章見知。性靜退不競,唯以勤
 學著述為事。著《安身論》以明所守,其辭曰:



 蓋崇德莫大乎安身,安身莫尚乎存正,存正莫重乎無私,無私莫深乎寡欲。是以君子安其身而後動,易其心而後語,定其交而後求,篤其志而後行。然則動者,吉凶之端也;語者,榮辱之主也;求者,利病之幾也;行者,安危之決也。故君子不妄動也,動必適其道;不徒語也,語必經於理;不茍求也,求必造於義;不虛行也,行必由於正。夫然,用能免或繫之凶,享自天之祐。故身不安則殆,言不從則悖,交不審則惑,行不篤則危。四者行乎中,則憂患接乎外矣。憂患之接,必生於自私,而興於有欲。自私者不能成其
 私,有欲者不能濟其欲,理之至也。欲茍不濟,能無爭乎?私茍不從,能無伐乎?人人自私,家家有欲,眾欲並爭,群私交伐。爭,則亂之萌也;伐,則怨之府也。怨亂既構,危害及之,得不懼乎?



 然棄本要末之徒,知進忘退之士,莫不飾才銳智,抽鋒擢穎,傾側乎勢利之交,馳騁乎當塗之務。朝有彈冠之朋,野有結綬之友,黨與熾於前,榮名扇其後。握權,則赴者鱗集;失寵,則散者瓦解;求利,則託刎頸之歡;爭路,則構刻骨之隙。於是浮偽波騰,曲辯雲沸,寒暑殊聲,朝夕異價,駑蹇希奔放之跡,鉛刀競一割之用。至於愛惡相攻,與奪交戰,誹謗噂沓,毀譽縱橫,君子
 務能,小人伐技,風頹於上,俗弊於下。禍結而恨爭也不彊,患至而悔伐之未辯,大者傾國喪家,次則覆身滅祀。其故何邪?豈不始於私欲而終於爭伐哉?



 君子則不然。知自私之害公也,然後外其身;知有欲之傷德也,故遠絕榮利;知爭競之遘災也,故犯而不校;知好伐之招怨也,故有功而不德。安身而不為私,故身正而私全;慎言而不適欲,故言濟而欲從;定交而不求益,故交立而益厚;謹行而不求名,故行成而名美。止則立乎無私之域,行則由乎不爭之塗,必將通天下之理,而濟萬物之性。天下猶我,故與天下同其欲;己猶萬物,故與萬物同其
 利。



 夫能保其安者,非謂崇生生之厚而耽逸豫之樂也,不忘危而已。有期進者,非謂窮貴寵之榮而藉名位之重也,不忘退而已。存其治者,非謂嚴刑政之威而明司察之禁也,不忘亂而已。故寢蓬室,隱陋巷,披短褐,茹藜藿,環堵而居,易衣而出,茍存乎道,非不安也。雖坐華殿,載文軒,服黼繡,御方丈,重門而處,成列而行,不得與之齊榮。用天時,分地利,甘布衣,安藪澤,沾體塗足,耕而後食,茍崇乎德,非不進也。雖居高位,饗重祿,執權衡,握機秘,功蓋當時,勢侔人主,不得與之比逸。遺意慮,沒才智,忘肝膽,棄形器,貌若無能,志若不及,茍正乎心,非不治
 也。雖繁計策,廣術藝,審刑名,峻法制,文辯流離,論議絕世,不得與之爭功。故安也者,安乎道者也。進也者,進乎德者也。治也者,治乎心者也。未有安身而不能保國家,進德而不能處富貴,治心而不能治萬物者也。



 然思危所以求安,慮退所以能進,懼亂所以保治,戒亡所以獲存也。若乃弱志虛心,曠神遠致,徙倚乎不拔之根,浮遊乎無垠之外,不自貴於物而物宗焉,不自重於人而人敬焉。可親而不可慢也,可尊而不可遠也。親之如不足,天下莫之能狎也;舉之如易勝,而當世莫之能困也。達則濟其道而不榮也,窮則善其身而不悶也,用則立於
 上而非爭也,舍則藏於下而非讓也。夫榮之所不能動者,則辱之所不能加也;利之所不能勸者,則害之所不能嬰也。譽之所不能益者,則毀之所不能損也。



 今之學者誠能釋自私之心,塞有欲之求,杜交爭之原,去矜伐之態,動則行乎至通之路,靜則入乎大順之門,泰則翔乎寥廓之宇,否則淪乎渾冥之泉,邪氣不能干其度,外物不能擾其神,哀樂不能盪其守,死生不能易其真,而以造化為工匠,天地為陶鈞,名位為糟粕,勢利為埃塵,治其內而不飾其外,求諸己而不假諸人,忠肅以奉上,愛敬以事親,可以御一體,可以牧萬民,可以處富貴,可
 以安賤貧,經盛衰而不改,則庶幾乎能安身矣。



 初應州辟,後以父老,辭位致養。太康中,舉秀才,為太常博士。歷高陸令、淮南王允鎮東參軍。元康初,拜太子舍人,上《釋奠頌》。其辭曰:



 元康元年冬十二月,上以皇太子富於春秋,而人道之始莫先於孝悌,初命講《孝經》於崇正殿。實應天縱生知之量,微言奧義,發自聖問,業終而體達。三年春閏月,將有事於上庠,釋奠于先師,禮也。越二十四日丙申,侍祠者既齊,輿駕次于太學。太傅在前,少傅在後,恂恂乎弘保訓之道;宮臣畢從,三率備衛,濟濟乎肅翼贊之敬。乃掃壇為殿,懸幕為宮。夫子位於西序,顏回
 侍于北墉。宗伯掌禮,司儀辯位。二學儒官,搢紳先生之徒,垂纓佩玉,規行矩步者,皆端委而陪於堂下,以待執事之命。設樽篚於兩楹之間,陳罍洗於阼階之左。几筵既布,鐘懸既列,我后乃躬拜俯之勤,資在三之義。謙光之美彌劭,闕里之教克崇,穆穆焉,邕邕焉,真先王之徽典,不刊之美業,允不可替已。於是牲饋之事既終,享獻之禮已畢,釋玄衣,御春服,馳齋禁,反故式。天子乃命內外群司,百辟卿士,蕃王三事,至于學徒國子,咸來觀禮,我后皆延而與之燕。金石簫管之音,八佾六代之舞,鏗鏘闛閤,般辟俯仰,可以澄神滌欲,移風易俗者,罔不畢
 奏。抑淫哇,屏《鄭》《衛》,遠佞邪,釋巧辯。是日也,人無愚智,路無遠邇,離鄉越國,扶老攜幼,不期而俱萃。皆延頸以視,傾耳以聽,希道慕業,洗心革志,想洙、泗之風,歌來蘇之惠。然後知居室之善,著應乎千里之外;不言之化,洋溢于九有之內。於熙乎若典,固皇代之壯觀,萬載之一會也。尼昔忝禮官,嘗聞俎豆。今廁末列,親睹盛美,瀸漬徽猷,沐浴芳潤,不知手舞口詠,竊作頌一篇。義近辭陋,不足測盛德之形容,光聖明之遐度。其辭曰:



 三元迭運,五德代微。黃精既亢,素靈乃暉。有皇承天,造我晉畿。祚以大寶,登以龍飛。宣基誕命,景熙遐緒,三分自文,受終惟
 武。席卷要蠻,蕩定荒阻;道濟群生,化流率土。後帝承哉,丕隆曾構。奄有萬方,光宅宇宙。



 篤生上嗣,繼期挺秀。聖敬日躋,浚哲閎茂。留精儒術,敦閱古訓。遵道讓齒,降心下問。鋪以金聲,光以玉潤。如日之升,如乾之運。乃延台保,乃命學臣。聖容穆穆,侍講訚訚。抽演微言,啟發道真。探幽窮賾,溫故知新。講業既終,精義既研。崇聖重師,卜日告奠。陳其三牢,引其四縣。既戒既式,乃盥乃薦。



 恂恂孔聖,百王攸希。亹亹顏生,好學無違。曰皇儲后,體神合幾。兆吉先見,知來洞微。濟濟二宮,藹藹庶僚。俊乂鱗萃,髦士盈朝。如彼和肆,莫匪瓊瑤;如彼儀鳳,樂我《雲》《韶》。瓊
 瑤誰剖?四門洞開;《雲》《韶》奚樂?神人允諧。蟬冕耀庭。細珮振階。德以謙光,仁以恩懷。我酒惟清,我肴惟馨。舞以六代,歌以九成。



 莘莘胄子,祁祁學生。洗心自百,觀國之榮。學猶蒔苗,化若偃草。博我以文,弘我以道。萬邦蟬蛻,矧乃俊造。鑽蚌瑩珠,剖石摛藻。絲匪玄黃,水罔方圓。引之斯流,染之斯鮮。若金受範,若埴在甄。上好如雲,下效如川。



 昔在周興,王化之始。曰文曰武,時惟世子。今我皇儲,齊聖通理。緝熙重光,於穆不已。於穆伊何?思文哲后。媚茲一人,實副元首。孝洽家邦,光照九有。純嘏自晉,永世昌阜。微微下臣,過充近侍。猥躡風雲,鸞龍是廁。身澡芳
 流,目玩盛事。竭誠作頌,祗詠聖志。



 出為宛令,在任寬而不縱,恤隱勤政,厲公平而遺人事。入補尚書郎,俄轉著作郎。為《乘輿箴》,其辭曰:



 《易》稱「有天地然後有人倫,有父子然後有君臣」。傳曰:「大者天地,其次君臣。」然君臣父子之道,天地人倫之本,未有以先之者也。故天生蒸人而樹之君,使司牧之,將以導群生之性,而理萬物之情。豈以寵一人之身,極無量之欲,如斯而已哉!夫古之為君者,無欲而至公,故有茅茨土階之儉;而後之為君,有欲而自利,故有瑤臺瓊室之侈。無欲者,天下共推之;有欲者,天下共爭之。推之之極,雖禪代猶脫屣;爭之之極,雖
 劫殺而不避。故曰「天下非一人之天下,乃天下之天下」,安可求而得,辭而已者乎!



 夫修諸己而化諸人,出乎邇而見乎遠者,言行之謂也。故人主所患,莫甚於不知其過;而所美,莫美於好聞其過。若有君於此,而曰予必無過,唯其言而莫之違,斯孔子所謂其庶幾乎一言而喪國者也。蓋君子之過,如日月之蝕:過也,人皆見之,更也,人皆仰之。雖以堯、舜、湯、武之盛,必有誹謗之木,敢諫之鼓,盤杅之銘,無諱之史,所以閑其邪僻而納諸正道,其自維持如此之備。故箴規之興,將以救過補闕,然猶依違諷喻,使言之者無罪,聞之者足以自誡。先儒既援古
 義,舉內外之殊,而高祖亦序六官,論成敗之要,義正辭約,又盡善矣。自《虞人箴》以至于《百官》,非唯規其所司,誠欲人主斟酌其得失焉。《春秋傳》曰「命百官箴王闕」,則亦天子之事也。



 尼以為王者膺受命之期,當神器之運,總萬機而撫四海,簡群才而審所授,孜孜於得人,汲汲於聞過,雖廷爭面折,猶將祈請而求焉。至於箴規,諫之順者,曷為獨闕之哉?是以不量其學陋思淺,因負擔之餘,嘗試撰而述之。不敢斥至尊之號,故以「乘輿」目篇。蓋帝王之事至大,而古今之變至眾,文繁而義詭,意局而辭野,將欲希企前賢,仿佛崇軌,譬猶丘坻之望華岱,恒星
 之繫日月也,其不逮明矣。頌曰:



 元元遂初,芒芒太始。清濁同流,玄黃錯歭。上下弗形,尊卑靡紀。赫胥悠哉,大庭尚矣。皇極啟建,兩儀既分。彞倫需永序,萬邦已紛。國事明王,家奉嚴君。各有攸尊,德用不勤。羲、農已降,暨於夏、殷。或禪或傳,乃質乃文。



 太上無名,下知有之。仁義不存,而人歸孝慈。無為無執,何欲何思。忠信之薄,禮刑實滋。既譽既畏,以侮以欺。作誓作盟,而人始叛疑。煌煌四海,藹藹萬乘,菲誓焉憑?左輔右弼,前疑後丞。一日萬機,業業兢兢。夫出其言善,則千里是應;而莫余違,亦喪邦有征。樞機之動,式以廢興。殷監不遠,若之何勿懲!



 且厚味臘
 毒,豐屋生災。辛作FM室,而夏興瑤臺。糟丘酒池,象箸玉杯。厥肴伊何?龍肝豹胎。惟此哲婦,職為亂階。殷用喪師,夏亦不恢。是以帝堯在位,茅茨不翦。周文日昃,昧旦丕顯。夫德輶如毛,而或舉之者鮮。故《濩》有慚德,《武》未盡善。下世道衰,末俗化淺。耽樂逸遊,荒淫沈湎。不式古訓,而好是佞辯;不遵王路,而覆車是踐。成敗之效,載在先典。匪唯陵夷,厥世用殄。故曰樹君如之何?將人是司牧。視之猶傷,而知其寒奧。故能撫之斯柔,而敦之斯睦;無遠不懷,靡思不服。夫豈厭縱一人,而玩其耳目;內迷聲色,外荒弛逐;不修政事,而終於顛覆?



 昔唐氏授舜,舜亦命
 禹。受終納祖,丕承天序。放桀惟湯,剋殷伊武。故禪代非一姓,社稷無常主。四嶽三塗,九州之阻。彭蠡、洞庭,殷商之旅。虞夏之隆,非由尺土。而紂之百剋,卒於絕緒。故王者無親,唯在擇人。傾蓋惟舊,白首乃新。望由釣夫,伊起有莘。負鼎鼓刀,而謀合聖神。夫豈借官左右,而取介近臣。蓋有國有家者,莫云我聰,或此面從;莫謂我智,聽受未易。甘言美疾,鮮不為累。由夷逃寵,遠於脫屣。奈何人主,位極則侈?



 知人則哲,惟帝所難。唐朝既泰,四族作奸。周室既隆,而管、蔡不虔。匪我二聖,孰弭斯患?若九德咸受,俊乂在官,君非臣莫治,臣非君莫安。故《書》美康哉,而《
 易》貴金蘭。有皇司國,敢告納言。



 及趙王倫篡位,孫秀專政,忠良之士皆罹禍酷。尼遂疾篤,取假拜掃墳墓。聞齊王冏起義,乃赴許昌。冏引為參軍,與謀時務,兼管書記。事平,封安昌公。歷黃門侍郎、散騎常侍、侍中、秘書監。永興末,為中書令。時三王戰爭,皇家多故,尼職居顯要,從容而已。雖憂虞不及,而備嘗艱難。永嘉中,遷太常卿。洛陽將沒,攜家屬東出成皋,欲還鄉里。道遇賊,不得前,病卒於塢壁,年六十餘。



 張載,字孟陽,安平人也。父收,蜀郡太守。載性閑雅,博學
 有文章。太康初,至蜀省父,道經劍閣。載以蜀人恃險好亂,因著銘以作誡曰:



 巖巖梁山,積石峨峨。遠屬荊、衡,近綴岷、嶓。南通邛、僰,北達褒斜。狹過彭、碣,高踰嵩、華。惟蜀之門,作固作鎮。是曰劍閣,壁立千仞。窮地之險,極路之峻。世濁則逆,道清斯順。閉由往漢,開自有晉。秦得百二,並吞諸侯。齊得十二,田生獻籌。矧茲狹隘,土之外區。一人荷戟,萬夫趄。形勝之地,非親勿居。昔在武侯,中流而喜。河山之固,見屈吳起。洞庭孟門,二國不祀。興實由德,險亦難恃。自古及今,天命不易。憑阻作昏,鮮不敗績。公孫既沒,劉氏銜壁。覆車之軌,無或重跡。勒銘山阿,敢
 告梁益。



 益州刺史張敏見而奇之,乃表上其文,武帝遣使鐫之於劍閣山焉。



 載又為《榷論》曰:



 夫賢人君子將立天下之功,成天下之名,非遇其時,曷由致之哉!故嘗試論之:殷湯無鳴條之事,則伊尹,有莘之匹夫也;周武無牧野之陣,則呂牙,渭濱之釣翁也。若茲之類,不可勝紀。蓋聲發響應,形動影從,時平則才伏,世亂則奇用,豈不信歟!設使秦、莽修三王之法,時致隆平,則漢祖,泗上之健吏;光武,舂陵之俠客耳,況乎附麗者哉!故當其有事也,則足非千里,不入於輿;刃非斬鴻,不韜於鞘。是以駑蹇望風而退,頑鈍未試而廢。及其無事也,則牛驥共牢,
 利鈍齊列,而無長塗犀革以決之,此離朱與瞽者同眼之說也。處守平之世,而欲建殊常之勳,居太平之際,而吐違俗之謀,此猶卻步而登山,鬻章甫於越也。漢文帝見李廣而嘆曰:「惜子不遇,當高帝時,萬戶侯豈足道哉!」故智無所運其籌,勇無所奮其氣,則勇怯一也;才無所騁其能,辯無所展其說,則頑慧均也。是以吳榜越船,不能無水而浮;青虯赤螭,不能無雲而飛。故和璧之在荊山,隋珠之潛重川,非遇其人,焉有連城之價,照車之名乎!青骹繁霜,縶於籠中,何以效其撮東郭於韝下也?白猨玄豹,藏於欞檻,何以知其接垂條於千仞也?孱夫與
 烏獲訟力,非龍文赤鼎,無以明之;蓋聶政與荊卿爭勇,非彊秦之威,孰能辨之?故餓夫庸隸,抱關屠釣之倫,一旦而都卿相之位,建金石之號者,或有懷顏、孟之術,抱伊、管之略,沒世而不齒者,此言有事之世易為功,無為之時難為名也。若斯湮滅而不稱,曾不足以多說。



 況夫庸庸之徒,少有不得意者,則自以為枉伏。莫不飾小辯、立小善以偶時,結朋黨、聚虛譽以驅俗。進之無補於時,退之無損於化。而世主相與雷同齊口,吹而煦之,豈不哀哉!今士循常習故,規行矩步,積階級,累閥閱,碌碌然以取世資。若夫魁梧俊傑,卓躒俶儻之徒,直將伏死嶔
 岑之下,安能與步驟共爭道里乎!至如軒冕黻班之士,茍不能匡化輔政,佐時益世,而徒俯仰取容,要榮求利,厚自封之資,豐私家之積,此沐猴而冠耳,尚焉足道哉!



 載又為《蒙汜賦》,司隸校尉傅玄見而嗟歎,以車迎之,言談盡日,為之延譽,遂知名。起家佐著作郎,出補肥鄉令。復為著作郎,轉太子中舍人,遷樂安相、弘農太守。長沙王乂請為記室督。拜中書侍郎,復領著作。載見世方亂,無復進仕意,遂稱疾篤告歸,卒於家。



 協字景陽,少有俊才,與載齊名。辟公府掾,轉祕書郎,補華陰令、征北大將軍從事中郎,遷中書侍郎。轉河間內
 史,在郡清簡寡欲。



 于時天下已亂,所在寇盜,協遂棄絕人事,屏居草澤,守道不競,以屬詠自娛。擬諸文士作《七命》。其辭曰:



 沖漠公子,含華隱曜,嘉遯龍蟠,超世高蹈,遊心於浩然,玩志乎眾妙,絕景乎大荒之遐阻,吞響乎幽山之窮奧。於是徇華大夫聞而造焉。乃整雲輅,驂飛黃,越奔沙,輾流霜,陵扶搖之風,躡堅冰之津,旌拂霄崿,軌出蒼垠,天清泠而無霞,野曠朗而無塵,臨重岫而攬轡,顧石室而迴輪。遂適沖漠公子之所居。其居也,崢嶸幽藹,蕭瑟虛玄,溟海渾濩涌其後,嶰谷嶆張其前,尋竹竦莖蔭其壑,百籟群鳴籠其山,衝飆發而回日,飛礫起
 而灑天。於是登絕巘,逆長風,陳辨惑之辭,命公子於巖中。曰:「蓋聞聖人不卷道而背時,智士不遺身而匿跡,生必耀華名於玉牒,沒則勒鴻伐於金冊。今公子違世陸沈,避地獨竄,有生之懽滅,資父之義廢。愁洽百年,苦溢千載,何異促鱗之遊汀濘,短羽之栖翳薈!今將榮子以天人之大寶,悅子以縱性之至娛,窮地而遊,中天而居,傾四海之歡,殫九州之腴,鑽屈穀之瓠,解疏屬之拘,子欲之乎?」公子曰:大夫不遺,來萃荒外,雖在不敏,敬聽嘉話。」



 大夫曰:「寒山之桐,出自太冥,含黃鐘以吐幹,據蒼岑而孤生。既乃瓊巘層崚,金岸崥崹,右當風谷,左臨雲谿,
 上無陵虛之巢,下無跖實之蹊,搖刖峻挺,茗邈嶕嶢,晞三春之溢露,溯九秋之鳴飆,零雪寫其根,霏霜封其條,木既繁而後綠,草未素而先凋。於是構雲梯,陟崢嶸,翦蕤賓之陽柯,剖大呂之陰莖。營匠斲其樸,伶倫均其聲。器舉樂奏,促調高張,音朗號鐘,韻清繞梁。追逸響於八風,採奇律於歸昌,啟中黃之妙宮,發蓐收之變商。若乃龍火西頹,暄氣初收,飛霜迎節,高風送秋,羈旅懷土之徒,流宕百罹之儔,撫促柱則酸鼻,揮危弦則涕流。若乃追清哇,赴嚴節,奏《淥水》,吐《白雪》,激楚迴,流風結,悲蓂莢之朝落,悼望舒之夕缺。煢嫠為之擗摽,孀老為之嗚咽,
 王子拂纓而傾耳,六馬噓天而仰秣。此蓋音曲之至妙,子豈能從我而聽之乎?」公子曰:「餘病未能也。」



 大夫曰:「蘭宮秘宇,雕堂綺櫳,雲屏爛旰,瓊壁青蔥,應門八襲,FM臺九重,表以百常之闕,圜以萬雉之墉。爾乃嶢榭迎風,秀出中天,翠觀岑青,彤閣霞連,長翼臨雲,飛陛陵山,望玉繩而結極,承倒景而開軒。赬素煥爛,枌栱嵯峨,陰虯負簷,陽馬承阿。錯以瑤英,鏤以金華,方疏含秀,圓井吐葩。重殿疊起,交綺對榥。幽堂晝密,明室夜朗。焦冥飛而風生,尺蠖動而成響。若乃目厭常玩,體倦帷幄,攜公子而雙遊,時娛觀於林麓。登翠阜,臨丹谷,華草錦繁,飛采星
 燭,陽葉春青,陰條秋綠,華實代新,承意恣觀。仰折神[B241],俯採朝蘭,愬惠風於蘅薄,眷椒塗於瑤壇。爾乃浮三翼,戲中沚,潛鰓駭,驚翰起,沈絲結,飛矰理,掛歸翮於赤霄之表,出華鱗於紫潭之裏。然後縱棹隨風,弭楫乘波,吹孤竹,撫雲和,川客唱淮南之曲,榜人奏《採菱》之歌。歌曰:『乘鷁舟兮為水嬉,臨芳洲兮拔靈芝。』樂以忘戚,遊以卒時,窮夜為日,畢歲為期。此蓋宴居之浩麗,子豈能從我而處之乎?」公子曰:「餘病未能也。」



 大夫曰:「若乃白商素節,月既授衣,天凝地閉,風厲霜飛,柔條夕勁,密葉晨稀,將因氣以效殺,臨金郊而講師。爾乃列輕武,整戎剛,建雲
 髦,啟雄芒。駕紅陽之飛燕,驂唐公之驌驦,屯羽隊於外林,縱輕翼於中荒。爾乃張脩罠,布飛羅,陵黃岑,挂青巒,畫長壑以為限,帶流谿以為關。既乃內無疏蹊,外無漏跡,叩鉦散校,舉麾贊獲,彀金機,馳鳴鏑,翦剛豪,落勁翮,連騎競騖,駢武齊轍,翕忽揮霍,雲迴風烈,聲動響飛,形移影發,舉戈林聳,揮鋒電滅,仰傾雲巢,俯殫地穴。乃有圓文之豜,斑題之豵,彭鬣風生,怒目電瞛,口咬霜刃,足撥飛鋒,齀林蹶石,扣跋幽叢。於是飛、黃奮銳,賁、育逞伎。𧾷戚封犬希,手費馮豕,拉甝,挫解,鉤爪摧,踞牙擺。瀾漫狼藉,傾榛倒壑,隕胔挂山,僵踣掩澤,藪為毛林,隰為丹薄。
 於是徹圍頓網,卷旆收鳶,虞人數獸,林衡計鮮;論最犒勤,息馬韜弦;肴駟連麃,酒駕方軒,千鐘電釂,萬燧星繁,陵阜沾流膏,谿谷厭芳煙。歡極樂殫,迴節而旋。此亦畋遊之壯觀,子豈能從我而為之乎?」公子曰:「餘病未能也。」



 大夫曰:「楚之陽劍,歐冶所營,邪谿之鋌,赤山之精,銷踰羊頭,鍱越鍛成。乃煉乃鑠,萬辟千灌。豐隆奮椎,飛廉扇炭,神器化成,陽文陰漫。既乃流綺星連,浮采艷發,光如散電,質如耀雪,霜鍔水凝,冰刃露潔,形冠豪曹,名珍巨闕,指鄭則三軍白首,麾晉則千里流血。豈徒水截蛟鴻,陸灑奔駟,斷浮翮以為工,絕重甲而稱利云爾而已哉!
 若其靈寶,則舒辟無方,奇鋒異模,形震薛燭,光駭風胡,價兼三鄉,聲貴二都,或馳名傾秦,或夜飛去吳。是以功冠萬載,威曜無窮,揮之者無前,擁之者身雄,可以從服九國,橫制八戎,爪牙景附,函夏承風。此蓋希世之神兵,子豈能從我而服之乎?」公子曰:「餘病未能也。」



 大夫曰:「天驥之駿,逸態超越,稟氣靈川,受精皎月,眸瞷黑照,玄採紺發,沫如揮紅,汗如振血,秦青不能識其眾尺,方堙不能睹其若滅。爾乃巾雲軒,踐朝霧,赴春衢,整秋御,虯踴螭騰,麟超龍翥,望山載奔,視林載赴。氣盛怒發,星飛電駭,志陵九州,勢越四海。影不及形,塵不暇起,浮箭未移,
 再踐千里。爾乃踰天根,越地隔,過汗漫之所下遊,躡章、亥之所未跡,陽烏為之頓羽,夸父為之投策。斯蓋天下之俊乘,子豈能從我而御之乎?」公子曰:「餘病未能也。」



 大夫曰:「大梁之黍,瓊山之禾,唐、稷播其根,農帝嘗其華。爾乃六禽殊珍,四膳異肴,窮海之錯,極陸之毛,伊公爨鼎,庖丁揮刀。味重九沸,和兼勺藥,晨鳧露鵠,霜黃雀,圓案星亂,方丈華錯。封熊之蹯,翰音之跖,燕髀猩脣,髦殘象白,靈川之龜,萊黃之鮐,丹穴之鷚,玄豹之胎,燀以秋橙,酤以春梅,接以商王之箸,承以帝辛之懷。范公之鱗,出自九谿,赬尾丹腮,紫翼青鬐。爾乃命支離,飛霜鍔,紅
 肌綺散,素膚雪落,婁子之豪不能廁其細,秋蟬之翼不足擬其薄。繁肴既闋,亦有嘉羞。商山之果,漢皋之楱,析龍眼之房,剖椰子之殼。芳旨萬選,承意代奏。乃有荊南烏程、豫北竹葉,浮蟣星沸,飛華萍接,玄石嘗其味,儀氏進其法,傾罍一朝,可以流湎千日,單醪投川,可使三軍告捷。斯人神之所歆羨,觀聽之所煒曄也,子豈能強起而御之乎?」公子曰:「耽爽口之饌,甘臘毒之味,服腐腸之藥,御亡國之器,雖子大夫之所榮,顧亦吾人之所畏,餘病未能也。」



 大夫曰:「蓋有晉之融皇風也,金華啟徵,大人有作,繼明代照,配天光宅。其基德也,隆於姬公之處岐;
 其垂仁也,富乎有殷之在亳。南箕之風不能暢其化,離畢之雲無以豐其澤。皇道昭煥,帝載緝熙。導氣以樂,宣德以詩,教清乎雲官之世,政穆乎鳥紀之時。玉猷四塞,函夏謐靜,丹冥投鋒,青徼釋警,卻馬於糞車之轅,銘德於昆吾之鼎。群萌反素,時文載郁,耕父推畔,漁豎讓陸,樵夫恥危冠之飾,輿臺笑短後之服。六合時雍,巍巍蕩蕩,玄髫巷歌,黃髮擊壤,解羲皇之繩,錯陶唐之象。若乃華裔之夷,流荒之貊,語不傳於輶軒,地未被乎正朔,莫不駿奔稽顙,委質重譯。於時昆蚑感惠,無思不擾。苑戲九尾之禽,囿棲三足之鳥,鳴鳳在林,夥於黃帝之園;有
 龍遊川,盈於孔甲之沼。萬物煙煴,天地交泰,義懷靡內,化感無外,林無被褐,山無韋帶。皆象刻於百工,兆發乎靈蔡,搢紳濟濟,軒冕藹藹,功與造化爭流,德與二儀比大。」言未終,公子蹶然而興曰:「鄙夫固陋,守茲狂狷。蓋理有毀之,而爭寶之訟解;言有怒之,而齊王之疾痊。向子誘我以聾耳之樂,棲我以蔀家之屋,田遊馳蕩,利刃駿足,既老氏之攸戒,非吾人之所欲,故靡得而應子。至聞皇風載韙,時聖道醇,舉實為秋,摛藻為春,下有可封之人,上有大哉之君,餘雖不敏,請從後塵。」



 世以為工。



 永嘉初,復徵為黃門侍郎,託疾不就,終於家。



 亢字季陽。才藻不逮二昆,亦有屬綴,又解音樂伎術。時人謂載、協、亢、陸機、雲曰:「二陸」「三張」。中興初過江,拜散騎侍郎。祕書監荀崧舉亢領佐著作郎,出補烏程令,入為散騎常侍,復領佐著作。述《歷贊》一篇,見《律歷志》。



 史臣曰:孝若掞蔚春華,時標麗藻。睹其《抵疑》詮理,本窮通於自天;作誥敷文,流英聲於孝悌,旨深致遠,殊有大雅之風烈焉。安仁思緒雲騫,詞鋒景煥,前史儔於賈誼,先達方之士衡。賈論政範,源王化之幽賾;潘著哀詞,貫人靈之情性。機文喻海,韞蓬山而育蕪;岳藻如江,濯美錦而增絢。混三家以通校,為二賢之亞匹矣。然其挾彈
 盈果,拜塵趨貴,蔑棄倚門之訓,乾沒不逞之間,斯才也而有斯行也,天之所賦,何其駁歟!正叔含咀藝文,履危居正,安其身而後動,契其心而後言,著論究人道之綱,裁箴懸乘輿之鑒,可謂玉質而金相者矣。孟陽鏤石之文,見奇於張敏;《蒙汜》之詠,取重於傅玄,為名流之所挹,亦當代之文宗矣。景陽摛光王府,棣萼相輝。洎乎二陸入洛,三張減價。考核遺文,非徒語也。



 贊曰:湛稱弄翰,縟彩雕煥。才高位卑,往哲攸嘆。岳實含章,藻思抑揚。趨權冒勢,終亦罹殃。尼標雅性,夙聞詞令。載協飛芳,棣華增映。



\end{pinyinscope}