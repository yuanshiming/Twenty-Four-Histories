\article{列傳第六十八 王敦桓溫}

\begin{pinyinscope}

 王敦桓溫



 王
 敦,字處仲,司徒導之從父兄也。父基,治書侍御史。敦少有奇人之目,尚武帝女襄城公主,拜駙馬都尉,除太子舍人。時王愷、石崇以豪侈相尚,愷嘗置酒,敦與導俱在坐,有女伎吹笛小失聲韻,愷便驅殺之,一坐改容,敦神色自若。他日,又造愷,愷使美人行酒,以客飲不盡,輒殺之。酒至敦、導所,敦故不肯持,美人悲懼失色,而敦傲
 然不視。導素不能飲,恐行酒者得罪,遂勉強盡觴。導還,歎曰:「處仲若當世,心懷剛忍,非令終也。」洗馬潘滔見敦而目之曰:「處仲蜂目已露,但豺聲未振,若不噬人,亦當為人所噬。」及太子遷許昌,詔東宮官屬不得送。敦及洗馬江統、潘滔,舍人杜蕤、魯瑤等,冒禁於路側望拜流涕,時論稱之。遷給事黃門侍郎。



 趙王倫篡位,敦叔父彥為兗州刺史,倫遣敦慰勞之。會諸王起義兵;彥被齊王冏檄,懼倫兵強,不敢應命,敦勸彥起兵應諸王,故彥遂立勛績。惠帝反正,敦遷散騎常侍、左衛將軍、大鴻臚、侍中,出除廣武將軍、青州刺史。永嘉初,徵為中書監。于時天
 下大亂,敦悉以公主時侍婢百餘人配給將士,金銀寶物散之於眾,單車還洛。東海王越自滎陽來朝,敦謂所親曰:「今威權悉在太傅,而選用表情,尚書猶以舊制裁之,太傅今至,必有誅罰。」俄而越收中書令繆播等十餘人殺之。越以敦為揚州刺史,潘滔說越曰:「今樹處仲於江外,使其肆豪彊之心,是見賊也。」越不從。其後徵拜尚書,不就。元帝召為安東軍諮祭酒。會揚州刺史劉陶卒,帝復以敦為揚州刺史,加廣武將軍。尋進左將軍、都督征討諸軍事、假節。帝初鎮江東,威名未著,敦與從弟導等同心翼戴,以隆中興,時人為之語曰:「王與馬,共天下。」
 尋與甘卓等討江州刺史華軼,斬之。



 蜀賊杜弢作亂,荊州刺史周顗退走,敦遣武昌太守陶侃、豫章太守周訪等討韜,而敦進住豫章,為諸軍繼援。及侃破弢,敦上侃為荊州刺史。既而侃為弢將杜曾所敗,敦以處分失所,自貶為廣武將軍,帝不許。侃之滅弢也,敦以元帥進鎮東大將軍、開府儀同三司,加都督江揚荊湘交廣六州諸軍事、江州刺史,封漢安侯。敦始自選置,兼統州郡焉。頃之,杜弢將杜弘南走廣州,求討桂林賊自效,敦許之。陶侃距弘不得進,乃詣零陵太守尹奉降,奉送弘與敦,敦以為將,遂見寵待。南康人何欽所居險固,聚黨數千
 人,敦就加四品將軍,於是專擅之迹漸彰矣。



 建武初,又遷征南大將軍,開府如故。中興建,拜侍中、大將軍、江州牧。遣部將朱軌、趙誘伐杜曾,為曾所殺,敦自貶,免侍中,并辭牧不拜。尋加荊州牧,敦上疏曰:



 昔漢祖以神武革命,開建帝業,繼以文帝之賢,纂承洪緒,清虛玄默,擬跡成康。賈誼歎息,以為天下倒懸,雖言有抑揚,不失事體。今聖朝肇建,漸振宏綱,往段匹磾遣使求效忠節,尚未有勞,便以方州與之。今靳明等為國雪恥,欲除大逆,此之志望,皆欲附翼天飛。雖功大宜報,亦宜有以裁之,當杜漸防萌,慎之在始。中間不逞,互生事變,皆非忠義,率
 以一朝之榮。天下漸弊,實由於此。春秋之時,天子微弱,諸侯奢侈,晉文思崇周室,至有求隧之請,襄王讓之以禮,聞義而服,自爾諸侯莫敢越度。臣謂前者賊寇未殄,茍以濟事,朝廷諸所加授,頗多爵位兼重。今自臣以下,宜皆除之,且以塞群小矜功之望,夷狄無懨之求。若復遷延,顧望流俗,使姦狡生心,遂相怨謗,指摘朝廷,讒諛蜂起,臣有以知陛下無以正之。此安危之機,天下之望。



 臣門戶特受榮任,備兼權重,渥恩偏隆,寵過公族。行路廝賤猶謂不可,臣獨何心可以安之。臣一宗誤陛下,傾覆亦將尋至;雖復灰身剖心,陛下追悔將何所及!伏願
 諒臣至款,及今際會,小解散之,並授賢俊,少慰有識,各得盡其所懷,則人思競勸矣。州牧之號,所不敢當,輒送所假侍中貂蟬。又宜并官省職,以塞群小覬覦之望。



 帝優詔不許。又固辭州牧,聽為刺史。時劉隗用事,頗疏間王氏,導等甚不平之。敦上疏曰:



 導昔蒙殊寵,委以事機,虛己求賢,竭誠奉國,遂藉恩私,居輔政之重。帝王體遠,事義不同,雖皇極初建,道教方闡,惟新之美,猶有所闕。臣每慷慨於遐遠,愧憤於門宗,是以前後表疏,何嘗不寄言及此。陛下未能少垂顧眄,暢臣微懷,云導頃見疏外,所陳如昨,而其萌已著,其為咎責,豈惟導身而已。群
 從所蒙,並過才分。導誠不能自量,陛下亦愛忘其短。常人近情,恃恩昧進,獨犯龍鱗,迷不自了。臣竊所自憂慮,未詳所由,惶愧踧躇,情如灰土。天下事大,盡理實難,導雖凡近,未有穢濁之累;既往之勳,疇昔之顧,情好綢繆,足以歷薄俗,明君臣,合德義,同古賢。昔臣親受嘉命,云:「吾與卿及茂弘當管鮑之交。」臣忝外任,漸冉十載,訓誘之誨,日有所忘;至於斯命,銘之於心,竊猶眷眷,謂前恩不得一朝而盡。



 伏惟陛下聖哲日新,廣延俊乂,臨之以政,齊之以禮。頃者令導內綜機密,出錄尚書,杖節京都,並統六軍,既為刺史,兼居重號,殊非人臣之體。流俗好
 評,必有譏謗,宜省錄尚書、杖節及都督。且王佐之器,當得宏達遠識、高正明斷、道德優備者,以臣闇識,未見其才。然於見人,未踰於導;加輔翼積年,實盡心力。霸王之主,何嘗不任賢使能,共相終始!管仲有三歸反坫之識,子犯有臨河要君之責,蕭何、周勃得罪囹圄,然終為良佐。以導之才,何能無失,!當令任不過分,役其所長,以功補過,要之將來。導性慎密,尤能忍事,善於斟酌,有文章才義,動靜顧問,起予聖懷,外無過寵,公私得所。今皇祚肇建,八表承風;聖恩不終,則遐邇失望。天下荒弊,人心易動;物聽一移,將致疑惑。臣非敢茍私親親,惟欲忠於
 社稷。



 表至,導封以還敦,敦復遣奏之。



 初,敦務自矯厲,雅尚清談,口不言財色。既素有重名,又立大功於江左,專任閫外,手控彊兵,群從貴顯,威權莫貳,遂欲專制朝廷,有問鼎之心。帝畏而惡之,遂引劉隗、刁協等以為心膂。敦益不能平,於是嫌隙始構矣。每酒後輒詠魏武帝樂府歌曰:「老驥伏櫪,志在千里。烈士暮年,壯心不已。」以如意打唾壺為節,壺邊盡缺。及湘州刺史甘卓遷梁州,敦欲以從事中郎陳頒代卓,帝不從,更以譙王承鎮湘州。敦復上表陳古今忠臣見疑於君,而蒼蠅之人交構其間,欲以感動天子。帝愈忌憚之。俄加敦羽葆鼓吹,增從
 事中郎、掾屬、舍人各二人。帝以劉隗為鎮北將軍,戴若思為征西將軍,悉發揚州奴為兵,外以討胡,實禦敦也。永昌元年,敦率眾內向,以誅隗為名,上疏曰:



 劉隗前在門下,邪佞諂媚,譖毀忠良,疑惑聖聽,遂居權寵,撓亂天機,威福自由,有識杜口。大起事役,勞擾士庶,外託舉義,內自封植;奢僭過制,乃以黃散為參軍,晉魏已來,未有此比。傾盡帑藏,以自資奉;賦役不均,百姓嗟怨;免良人奴,自為惠澤。自可使其大田以充倉廩,今便割配,皆充隗軍。臣前求迎諸將妻息,聖恩聽許,而隗絕之,使三軍之士莫不怨憤。又徐州流人辛苦經載,家計始立,隗悉
 驅逼,以實己府。當陛下踐阼之始,投刺王官,本以非常之慶使豫蒙榮分。而更充征役,復依舊名,普取出客,從來久遠,經涉年載,或死亡滅絕,或自贖得免,或見放遣,或父兄時事身所不及,有所不得,輒罪本主,百姓哀憤,怨聲盈路。身欲北渡,以遠朝廷為名,而密知機要,潛行險慝,進人退士,高下任心,姦狡饕餮,未有隗比,雖無忌、宰嚭、弘恭、石顯未足為喻。是以遐邇憤慨,群后失望。



 臣備位宰輔,與國存亡,誠乏平勃濟時之略,然自忘駑駘,志存社稷,豈忍坐視成敗,以虧聖美。事不獲已,今輒進軍,同討姦孽,願陛下深垂省察,速斬隗首,則眾望厭服,
 皇祚復隆。隗首朝懸,諸軍夕退。昔太甲不能遵明湯典,顛覆厥度,幸納伊尹之勳,殷道復昌。漢武雄略,亦惑江充讒佞邪說,至乃父子相屠,流血丹地,終能剋悟,不失大綱。今日之事,有逾於此,願陛下深垂三思,諮詢善道,則四海乂安,社稷永固矣。



 又曰:



 陛下昔鎮揚州,虛心下士,優賢任能,寬以得眾,故君子盡心,小人畢力。臣以闇蔽,豫奉徽猷,是以遐邇望風,有識自竭,王業遂隆,惟新克建,四海延頸,咸望太平。



 自從信隗已來,刑罰不中,街談巷議,皆云如吳之將亡。聞之惶惑,精魂飛散,不覺胸臆摧破,泣血橫流。陛下當全祖宗之業,存神器之重,察
 臣前後所啟,奈何棄忽忠言,遂信姦佞,誰不痛心!願出臣表,諮之朝臣,介石之幾,不俟終日,令諸軍早還,不至虛擾。



 敦黨吳興人沈充起兵應敦。敦至蕪湖,又上表罪狀刁協。帝大怒,下詔曰:「王敦憑恃寵靈,敢肆狂逆,方朕太甲,欲見幽囚。是可忍也,孰不可忍也!今親率六軍,以誅大逆,有殺敦者,封五千戶侯。」召戴若思、劉隗並會京師。敦兄含時為光祿勛,叛奔于敦。



 敦至石頭,欲攻劉隗,其將杜弘曰:「劉隗死士眾多,未易可剋,不如攻石頭。周札少恩,兵不為用,攻之必敗。札敗,則隗自走。」敦從之。札果開城門納弘。諸將與敦戰,王師敗績。既入石頭,擁兵
 不朝,放肆兵士劫掠內外。官省奔散,惟有侍中二人侍帝。帝脫戎衣,著朝服,顧而言曰:「欲得我處,但當早道,我自還瑯邪,何至困百姓如此!」敦收周顗、戴若思害之。以敦為丞相、江州牧,進爵武昌郡公,邑萬戶,使太常荀崧就拜,又加羽葆鼓吹,並偽讓不受。還屯武昌,多害忠良,寵樹親戚,以兄含為衛將軍、都督沔南軍事、領南蠻校尉、荊州刺史,以義陽太守任愔督河北諸軍事、南中郎將,敦又自督寧、益二州。



 及帝崩,太寧元年,敦諷朝廷徵己,明帝乃手詔征之,語在《明帝紀》。又使兼太常應詹拜授加黃金戊,班劍武賁二十人,奏事不名,入朝不趨,劍覆
 上殿。敦移鎮姑孰,帝使侍中阮孚齎牛酒犒勞,敦稱疾不見,使主簿受詔。以王導為司徒,敦自為揚州牧。



 敦既得志,暴慢愈甚,四方貢獻多入己府,將相嶽牧悉出其門。徙含為征東將軍、都督揚州江西諸軍事,從弟舒為荊州,彬為江州,邃為徐州。含字處弘,凶頑剛暴,時所不齒,以敦貴重,故歷顯位。敦以沈充、錢鳳為謀主,諸葛瑤、鄧嶽、周撫、李恆、謝雍為爪牙。充等並凶險驕恣,共相驅扇,殺戮自己;又大起營府,侵人田宅,發掘古墓,剽掠市道,士庶解體,咸知其禍敗焉。敦從弟豫章太守棱日夜切諫,敦怒,陰殺之。敦無子,養含子應。及敦病甚,拜應為
 武衛將軍以自副。錢鳳謂敦曰:「脫其不諱,便當以後事付應。」敦曰:「非常之事,豈常人所能!且應年少,安可當大事。我死之後,莫若解眾放兵,歸身朝廷,保全門戶,此計之上也。退還武昌,收兵自守,貢獻不廢,亦中計也。及吾尚存,悉眾而下,萬一僥倖,計之下也。」鳳謂其黨曰:「公之下計,乃上策也。」遂與沈充定謀,須敦死後作難。



 敦又忌周札,殺之而盡滅其族。常從督冉曾、公乘雄等為元帝腹心,敦又害之。以宿衛尚多,奏令三番休二。及敦病篤,詔遣侍中陳晷、散騎常侍虞斐問疾。時帝將討敦,微服至蕪湖,察其營壘,又屢遣大臣訊問其起居。遷含驃騎
 大將軍、開府儀同三司,含子瑜散騎常侍。



 敦以溫嶠為丹陽尹,欲使覘伺朝廷。嶠至,具言敦逆謀。帝欲討之,知其為物情所畏服,乃偽言敦死,於是下詔曰:



 先帝以聖德應運,創業江東,司徒導首居心膂,以道翼言贊。故大將軍敦參處股肱,或內或外,夾輔之勛,與有力焉。階緣際會,遂據上宰,杖節專征,委以五州。刁協、劉隗立朝不允,敦抗義致討,情希鬻拳,兵雖犯順,猶嘉乃誠,禮秩優崇,人臣無貳。事解之後,劫掠城邑,放恣兵人,侵及宮省;背違赦信,誅戮大臣;縱凶極逆,不朝而退。六合阻心,人情同憤。先帝含垢忍恥,容而不責,委任如舊,禮秩有加。朕
 以不天,尋丁酷罰,煢煢在疚,哀悼靡寄。而敦曾無臣子追遠之誠,又無輔孤同獎之操,繕甲聚兵,盛夏來至,輒以天官假授私屬,將以威脅朝廷,傾危宗社。朕愍其狂戾,冀其覺悟,故且含隱以觀其終。而敦矜其不義之強,有侮弱朝廷之志,棄親用羈,背賢任惡。錢鳳豎子,專為謀主,逞其凶慝,誣罔忠良。周嵩亮直,讜言致禍;周札、周莚累世忠義,聽受讒構,殘夷其宗。秦人之酷,刑不過五。敦之誅戮,傍濫無辜,滅人之族,莫知其罪。天下駭心,道路以目。神怒人怨,篤疾所嬰,昏荒悖逆,日以滋其,輒立兄息以自承代,多樹私黨,莫非同惡,未有宰相繼體而
 不由王命者也。頑凶相獎,無所顧忌,擅錄冶工,輒割運漕,志騁凶醜,以窺神器。社稷之危,匪夕則旦。天下長奸,敦以隕斃。鳳承凶宄,彌復煽逆。是可忍也,孰不可忍也!



 今遣司徒導,鎮南將軍、丹陽尹嶠,建威將軍趙胤武旅三萬,十道並進;平西將軍邃率兗州刺史遐、奮武將軍峻、奮威將軍贍精銳三萬,水陸齊勢;朕親御六軍,左衛將軍亮,右衛將軍胤,護軍將軍詹,領軍將軍瞻,中軍將軍壺,驍騎將軍艾,驃騎將軍、南頓王宗,鎮軍將軍、汝南王祐,太宰、西陽王羕被練三千,組甲三萬,總統諸軍,討鳳之罪。罪止一人,朕不濫刑。有能殺鳳送首,封五千戶
 侯,賞布五千匹。



 冠軍將軍鄧嶽志氣平厚,識經邪正;前將軍周撫質性詳簡,義誠素著;功臣之胄,情義兼常,往年從敦,情節不展,畏逼首領,不得相違,論其乃心,無貳王室,朕嘉其誠,方任之以事。其餘文武,諸為敦所授用者,一無所問,刺史二千石不得輒離所職。書到奉承,自求多福,無或猜嫌,以取誅滅。敦之將士,從敦彌所,怨曠日久,或父母隕沒,或妻子喪亡,不得奔赴,銜哀從役,朕甚愍之,希不心妻愴。其單丁在軍無有兼重者,皆遣歸家,終身不調,其餘皆與假三年,休訖還臺,當與宿衛同例三番。明承詔書,朕不負信。



 又詔曰:「敢有捨王敦姓名而
 稱大將軍者,軍法從事。」



 敦病轉篤,不能御眾,使錢鳳、鄧岳、周撫等率眾三萬向京師。含謂敦曰:「此家事,吾便當行。」於是以含為元帥。鳳等問敦曰:「事剋之日,天子云何?」敦曰:「尚未南郊,何得稱天子!便盡卿兵勢,保護東海王及裴妃而已。」乃上疏罪狀溫嶠,以誅奸臣為名。



 含至江寧,司徒導遺含書曰:



 近承大將軍困篤綿綿,或云已有不諱,悲怛之情,不能自勝。尋知錢鳳大嚴,欲肆奸逆,朝士忿憤,莫不扼腕。去月二十三日,得征北告,劉遐、陶瞻、蘇峻等深懷憂慮,不謀同辭。都邑大小及二宮宿衛咸懼有往年之掠,不復保其妻孥,是以聖主發赫斯之命,
 具如檄旨。近有嘉詔,崇兄八命,望兄獎群賢忠義之心,抑奸細不逞之計,當還武昌,盡力籓任。卒奉來告,乃承與犬羊俱下,雖當逼近,猶以罔然。兄立身率素,見信明於門宗,年踰耳順,位極人臣,仲玉、安期亦不足作佳少年,本來門戶,良可惜也!



 兄之此舉,謂可得如大將軍昔年之事乎?昔年佞臣亂朝,人懷不寧,如導之徒,心思外濟。今則不然。大將軍來屯于湖,漸失人心,君子危怖,百姓勞弊。將終之日,委重安期,安期斷乳未幾日,又乏時望,便可襲宰相之迹邪?自開闢以來,頗有宰相孺子者不?諸有耳者皆是將禪代意,非人臣之事也。先帝中興,
 遺愛在人。聖主聰明,德洽朝野,思與賢哲弘濟艱難。不北面而執臣節,乃私相樹建,肆行威福,凡在人臣,誰不憤歎!此直錢鳳不良之心聞於遠近,自知無地,遂唱姦逆。至如鄧伯山、周道和恆有好情,往來人士咸皆明之,方欲委任,與共戮力,非徒無慮而已也。



 導門戶小大受國厚恩,兄弟顯寵,可謂隆矣。導雖不武,情在寧國。今日之事,明目張膽為六軍之首,寧忠臣而死,不無賴而生矣。但恨大將軍桓文之勳不遂,而兄一旦為逆節之臣,負先人平素之志,既沒之日,何顏見諸父於黃泉,謁先帝於地下邪?執省來告,為兄羞之,且悲且慚。願速建大
 計,惟取錢鳳一人,使天下獲安,家國有福,故是竹素之事,非惟免禍而已。



 夫福如反手,用之即是。導所統六軍,石頭萬五千人,宮內後苑二萬人,護軍屯金城六千人,劉遐已至,征北昨已濟江萬五千人。以天子之威,文武畢力,豈可當乎!事猶可追,兄早思之。大兵一奪,導以為灼炟也。



 含不答。帝遣中軍司馬曹渾等擊含于越城,含軍敗,敦聞,怒曰:「我兄老婢耳,門戶衰矣!兄弟才兼文武者,世將、處季皆早死,今世事去矣。」語參軍呂寶曰:「我當力行。」因作勢而起,困乏復臥。



 鳳等至京師,屯于水南。帝親率六軍以禦鳳,頻戰破之。敦謂羊鑒及子應曰:「我亡
 後,應便即位,先立朝廷百官,然後乃營葬事。」初,敦始病,夢白犬自天而下嚙之,又見刁協乘軺車導從,瞋目令左右執之。俄而敦死,時年五十九。應祕不發喪,裹尸以席,蠟塗其外,埋於廳事中,與諸葛瑤等恆縱酒淫樂。



 沈充自吳率眾萬餘人至,與含等合。充司馬顧揚說充曰:「今舉大事,而天子已扼其喉,情離眾沮,鋒摧勢挫,持疑猶豫,必致禍敗。今若決破柵塘,因湖水灌京邑,肆舟檻之勢,極水軍之用,此所謂不戰而屈人之兵,上策也。籍初至之銳,並東南眾軍之力,十道俱進,眾寡過倍,理必摧陷,中策也。轉禍為福,因敗為成,召錢鳳計事,因斬之
 以降,下策也。」充不能用,揚逃歸于吳。含復率眾渡淮,蘇峻等逆擊,大敗之,充亦燒營而退。



 既而周光斬錢鳳,吳儒斬沈充,並傳首京師。有司議曰:「王敦滔天作逆,有無君之心,宜依崔杼、王浚故事,剖棺戮尸,以彰元惡。」於是發瘞出尸,焚其衣冠,跽而刑之。敦、充首同日懸于南桁,觀者莫不稱慶。敦首既懸,莫敢收葬者。尚書令郗鑒言於帝曰:「昔王莽漆頭以輗車,董卓然腹以照市,王凌儭土,徐馥焚首。前朝誅楊駿等,皆先極官刑,後聽私殯。然《春秋》許齊襄之葬紀侯,魏武義王脩之哭袁譚。由斯言之,王誅加於上,私義行於下。臣以為可聽私葬,於義為
 弘。」昭許之,於是敦家收葬焉。含父子乘單船奔荊州刺史王舒,舒使人沈之于江,餘黨悉平。



 敦眉目疏朗,性簡脫,有鑒裁,學通《左氏》,口不言財利,尤好清談,時人莫知,惟族兄戎異之。經略指麾,千里之外肅然,而麾下擾而不能整。武帝嘗召時賢共言伎藝之事,人人皆有所說,惟敦都無所關,意色殊惡。自言知擊鼓,因振袖揚枹,音節諧韻,神氣自得,傍若無人,舉坐歎其雄爽。石崇以奢豪矜物,廁上常有十餘婢侍列,皆有容色,置甲煎粉、沈香汁,有如廁者,皆易新衣而出。客多羞脫衣,而敦脫故著新,意色無怍。群婢相謂曰:「此客必能作賊。」又嘗荒恣
 於色,體為之弊,左右諫之,敦曰:「此甚易耳。」乃開後閣,驅諸婢妾數十人並放之,時人歎異焉。



 沈充,字士居。少好兵書,頗以雄豪聞於鄉里。敦引為參軍,充因薦同郡錢鳳。鳳字世儀,敦以為鎧曹參軍,數得進見。知敦有不臣之心,因進邪說,遂相朋構,專弄威權,言成禍福。遭父喪,外託還葬,而密為敦使,與充交構。



 初,敦參軍熊甫見敦委任鳳,將有異圖,因酒酣謂敦曰:「開國承家,小人勿用,佞倖在位,鮮不敗業。」敦作色曰:「小人阿誰?」甫無懼容,因此告歸。臨與敦別,因歌曰:「徂風飆起
 蓋山陵,氛霧蔽日玉石焚。往事既去可長歎,念別惆悵復會難。」敦知其諷己而不納。



 明帝將伐敦,遣其鄉人沈禎諭充,許以為司空。充謂禎曰:「三司具瞻之重,豈吾所任!幣厚言甘,古人所畏。且丈夫共事,終始當同,寧可中道改易,人誰容我!」禎曰:「不然。舍忠與順,未有不亡者也。大將軍阻兵不朝,爵賞自己,五尺之童知其異志。今此之舉,將行篡弒耳,豈同於往年乎?是以疆場諸將莫不歸赴本朝,內外之士咸願致死,正以移國易主,義不北面以事之也,奈何協同逆圖,當不義之責乎!朝廷坦誠,禎所知也。賊之黨類,猶宥其罪,與之更始,況見機而作
 邪!」充不納。率兵臨發,謂其妻子曰:「男兒不豎豹尾,終不還也。」及敗歸吳興,亡失道,誤入其故將吳儒家。儒誘充內重壁中,因笑謂充曰:「三千戶侯也。」充曰:「封侯不足貪也。爾以大義存我,我宗族必厚報汝。若必殺我,汝族滅矣。」儒遂殺之。充子勁竟滅吳氏。勁見《忠義傳》。



 史臣曰:瑯邪之初鎮建鄴,龍德猶潛,雖當璧膺圖預定於冥兆,豐功厚利未被於黎氓。王敦歷官中朝,威名夙著,作牧淮海,望實逾隆,遂能託魚水之深期,定金蘭之密契,弼成王度,光佐中興,卜世延百二之期,論都創三分之業,此功固不細也。既而負勛高而圖非望,恃勢逼
 而肆驕陵。釁隙起自刁劉,禍難成於錢沈。興晉陽之甲,纏象魏之兵。蜂目既露,豺聲又發,擅竊國命,殺害忠良,遂欲篡盜乘輿,逼遷龜鼎。賴嗣君英略,晉祚靈長,諸侯釋位,股肱戮力,用能運茲廟算,殄彼凶徒,克固鴻圖,載清天步者矣。



 桓溫,字元子,宣城太守彝之子也。生未期而太原溫嶠見之,曰:「此兒有奇骨,可試使啼。」及聞其聲,曰:「真英物也!」以嶠所賞,故遂名之曰溫。嶠笑曰:「果爾,後將易吾姓也。」彝為韓晃所害,涇令江播豫焉。溫時年十五,枕戈泣
 血,志在復仇。至年十八,會播已終,子彪兄弟三人居喪,置刃杖中,以為溫備。溫詭稱弔賓,得進,刃彪於廬中,並追二弟殺之,時人稱焉。



 溫豪爽有風概,姿貌甚偉,面有七星。少與沛國劉惔善,惔嘗稱之曰:「溫眼如紫石棱,鬚作猥毛磔,孫仲謀、晉宣王之流亞也。」選尚南康長公主,拜駙馬都尉,襲爵萬寧男,除瑯邪太守,累遷徐州刺史。



 溫與庾翼友善,恆相期以寧濟之事。翼嘗薦溫於明帝曰;「桓溫少有雄略,願陛下勿以常人遇之,常婿畜之,宜委以方召之任,託其弘濟艱難之勛。」翼卒,以溫為都督荊梁四州諸軍事、安西將軍、荊州刺史、領護南蠻校尉、
 假節。



 時李勢微弱,溫志在立勛于蜀,永和二年,率眾西伐。時康獻太后臨朝,溫將發,上疏而行。朝廷以蜀險遠,而溫兵寡少,深入敵場,甚以為憂。初,諸葛亮造八陣圖於魚復平沙之上,壘石為八行,行相去二丈。溫見之,謂「此常山蛇勢也。」文武皆莫能識之。及軍次彭模,乃命參軍周楚、孫盛守輜重,自將步卒直指成都。勢使其叔父福及從兄權等攻彭模,楚等禦之,福退走。溫又擊權等,三戰三捷,賊眾散,自間道歸成都。勢於是悉眾與溫戰于笮橋,參軍龔護戰沒,眾懼欲退,而鼓吏誤鳴進鼓,於是攻之,勢眾大潰。溫乘勝直進,焚其小城,勢遂夜遁九
 十里,至晉壽葭萌城,其將鄧嵩、昝堅勸勢降,乃面縛輿親請命。溫解縛焚親,送于京師。溫停蜀三旬,舉賢旌善,偽尚書僕射王誓、中書監王瑜、鎮東將軍鄧定、散騎常侍常璩等,皆蜀之良也,並以為參軍,百姓咸悅。軍未旋而王誓、鄧定、隗文等反,溫復討平之。振旅還江陵,進位征西大將軍、開府,封臨賀郡公。



 及石季龍死,溫欲率眾北征,先上疏求朝廷議水陸之宜,久不報。時知朝廷杖殷浩等以抗己,溫甚忿之,然素知浩,弗之憚也。以國無他釁,遂得相持彌年,雖有君臣之跡,亦相羈縻而已,八州士眾資調,殆不為國家用。聲言北伐,拜表便行,順流
 而下,行達武昌,眾四五萬。殷浩慮為溫所廢,將謀避之,又欲以騶虞幡住溫軍,內外噂沓,人情震駭。簡文帝時為撫軍,與溫書明社稷大計,疑惑所由。溫即迴軍還鎮,上疏曰:



 臣近親率所統,欲北掃趙魏,軍次武昌,獲撫軍大將軍、會稽王昱書,說風塵紛紜,妄生疑惑,辭旨危急,憂及社稷。省之惋愕,不解所由,形影相顧,隕越無地。臣以闇蔽,忝荷重任,雖才非其人,職在靜亂。寇仇不滅,國恥未雪,幸因開泰之期,遇可乘之會,匹夫有志,猶懷憤慨,臣亦何心,坐觀其弊!故荷戈驅馳,不遑寧處,前後表陳,于今歷年矣。丹誠坦然,公私所察,有何纖介,容此嫌
 忌?豈醜正之徒心懷怵惕,操弄虛說,以惑朝聽?



 昔樂毅謁誠,垂涕流奔,霍光盡忠,上官告變。讒說殄行,姦邪亂德,及歷代之常患,存亡之所由也。今主上富於陽秋,陛下以聖淑臨朝,恭己委任,責成群下,方寄會通於群才,布德信於遐荒。況臣世蒙殊恩,服事三朝,身非羈旅之賓,跡無韓彭之釁,而反間起於胸心,交亂過於四國,此古賢所以歎息於既往,而臣亦大懼於當年也。今橫議妄生,成此貝錦,使垂滅之賊復獲蘇息,所以痛心絕氣,悲慨彌深。臣雖所存者
 公,所務者國;然外難未弭,而內弊交興,則臣本心陳力之志也。



 進位太尉,固讓不拜。時殷浩至洛陽脩復園陵,經涉數年,屢戰屢敗,器械都盡。溫復進督司州,因朝野之怨,乃奏廢浩,自此內外大權一歸溫矣。溫遂統步騎四萬發江陵,水軍自襄陽入均口。至南鄉,步自淅川以征關中,命梁州刺史司馬勛出子午道。別軍攻上洛,獲苻健荊州刺史郭敬,進擊青泥,破之。健又遣子生、弟雄眾數萬屯嶢柳、愁思塠以距溫,遂大戰,生親自陷陣,殺溫將應庭、劉泓,死傷千數。溫軍力戰,生眾乃散。雄又與將軍桓沖戰白鹿原,又為沖所破。雄遂馳襲司馬勳,勳
 退次女媧堡。溫進至霸上,健以五千人深溝自固,居人皆安堵復業,持牛酒迎溫於路者十八九,耆老感泣曰:「不圖今日復見官軍!」初,溫恃麥熟,取以為軍資。而健芟苗清野,軍糧不屬,收三千餘口而還。帝使侍中黃門勞溫于襄陽。



 初,溫自以雄姿風氣是宣帝、劉琨之儔,有以其比王敦者,意甚不平。及是徵還,於北方得一巧作老婢,訪之,乃琨伎女也,一見溫,便潸然而泣。溫問其故,答曰:「公甚似劉司空。」溫大悅,出外整理衣冠,又呼婢問。婢云:「面甚似,恨薄;眼甚似,恨小;鬚甚似,恨赤;形甚似,恨短;聲甚似,恨雌。」溫於是褫冠解帶,昏然而睡,不怡者數日。



 母孔氏卒,上疏解職,欲送葬宛陵,詔不許。贈臨賀太夫人印綬,謚曰敬,遣侍中弔祭,謁者監護喪事,旬月之中,使者八至,軺軒相望於道。溫葬畢視事,欲脩復園陵,移都洛陽,表疏十餘上,不許。進溫征討大都督、督司冀二州諸軍事,委以專征之任。



 溫遣督護高武據魯陽,輔國將軍戴施屯河上,勒舟師以逼許洛,以譙梁水道既通,請徐豫兵乘淮泗入河。溫自江陵北伐,行經金城,見少為瑯邪時所種柳皆已十圍,慨然曰:「木猶如此,人何以堪!」攀枝執條,泫然流涕。於是過淮泗,踐北境,與諸僚屬登平乘樓,眺矚中原,慨然曰:「遂使神州陸沈,百年丘墟,
 王夷甫諸人不得不任其責!」袁宏曰:「運有興廢,豈必諸人之過!」溫作色謂四座曰:「頗聞劉景升有千斤大牛,啖芻豆十倍於常牛,負重致遠,曾不若一羸牸,魏武入荊州,以享軍士。」意以況宏,坐中皆失色。師次伊水,姚襄屯水北,距水而戰。溫結陣而前,親被甲督弟沖及諸將奮擊,襄大敗,自相殺死者數千人,越北芒而西走,追之不及,遂奔平陽。溫屯故太極殿前,徙入金墉城,謁先帝諸陵,陵被侵毀者皆繕復之,兼置陵令。遂旋軍,執降賊周成以歸,遷降人三千餘家於江漢之間。遣西陽太守滕畯出黃城,討蠻賊文盧等,又遣江夏相劉岵、義陽太守
 胡驥討妖賊李弘,皆破之,傳首京都。溫還軍之後,司、豫、青、兗復陷于賊。升平中,改封南郡公,降臨賀為縣公,以封其次子濟。



 隆和初,寇逼河南,太守戴施出奔,冠軍將軍陳祐告急,溫使竟陵太守鄧遐率三千人助祐,并欲還都洛陽,上疏曰:



 巴蜀既平,逆胡消滅,時來之會既至,休泰之慶顯著。而人事乖違,屢喪王略,復使二賊雙起,海內崩裂,河洛蕭條,山陵危逼,所以遐邇悲惶,痛心於既往者也。伏惟陛下稟乾坤自然之姿,挺羲皇玄朗之德,鳳妻外籓,龍飛皇極,時務陵替,備徹天聽,人之情偽,盡知之矣。是以九域宅心,幽遐企踵,思佇雲羅,混網四
 裔。誠宜遠圖廟算,大存經略,光復舊京,疆理華夏,使惠風陽澤洽被八表,霜威寒飆陵振無外,豈不允應靈休,天人齊契!今江河悠闊,風馬殊邈,故向義之徒履亡相尋,而建節之士猶繼踵無悔。況辰極既迴,眾星斯仰,本源既運,枝泒自遷;則晉之餘黎欣皇德之攸憑,群凶妖逆知滅亡之無日,騁思順之心,鼓雷霆之勢,則二豎之命不誅而自絕矣。故員通貴於無滯,明哲尚於應機,砎如石焉,所以成務。若乃海運既徒,而鵬翼不舉,永結根於南垂,廢神州於龍漠,令五尺之童掩口而歎息。



 夫先王經始,玄聖宅心,畫為九州,制為九服,貴中區而內諸
 夏,誠以晷度自中,霜露惟均,冠冕萬國,朝宗四海故也。自彊胡陵暴,中華蕩覆,狼狽失據,權幸揚越,蠖屈以待龍伸之會,潛蟠之俟風雲之期,蓋屯圮所鐘,非理勝而然也。而喪亂緬邈,五十餘載,先舊徂沒,後來童幼,班荊輟音,積習成俗,遂望絕於本邦,宴安於所託。眷言悼之,不覺悲歎!臣雖庸劣,才不周務,然攝官承乏,屬當重任,願竭筋骨,宣力先鋒,翦除荊棘,驅諸豺狼。自永嘉之亂,播流江表者,請一切北徙,以實河南,資其舊業,反其土宇,勸農桑之務,盡三時之利,導之以義,齊之以禮,使文武兼宣,信順交暢,井邑既脩,綱維粗舉。然後陛下建三
 辰之章,振旂旗之旌,冕旒錫鑾,朝服濟江,則宇宙之內誰不幸甚!



 夫人情昧安,難與圖始;非常之事,眾人所疑。伏願陛下決玄照之明,斷常均之外,責臣以興復之效,委臣以終濟之功。此事既就,此功既成,則陛下盛勳比隆前代,周宣之詠復興當年。如其不效,臣之罪也,褰裳赴鑊,其甘如薺。



 詔曰:「在昔喪亂,忽涉五紀,戎狄肆暴,繼襲凶跡,眷言西顧,慨歎盈懷!知欲躬率三軍,蕩滌氛穢,廓清中畿,光復舊京,非夫外身殉國,孰能若此者哉!諸所處分,委之高算。但河洛丘墟,所營者廣,經始之勤,致勞懷也。」於是改授並、司、冀三州,以交廣遼遠,罷都督,溫
 表辭不受。又加侍中、大司馬、都督中外諸軍事、假黃鉞。溫以既總督內外,不宜在遠,又上疏陳便宜七事:其一,朋黨雷同,私議沸騰,宜抑杜浮競,莫使能植。其二,戶口凋寡,不當漢之一郡,宜並官省職,令久於其事。其三,機務不可停廢,常行文案宜為限日。其四,宜明長幼之禮,獎忠公之吏。其五,褒貶賞罰,宜允其實。其六,宜述遵前典,敦明學業。其七,宜選建史官,以成晉書。有司皆奏行之。尋加羽葆鼓吹,置左右長史、司馬、從事中郎四人。受鼓吹,餘皆辭。復率舟軍進合肥。加揚州牧、錄尚書事,使侍中顏旄宣旨,召溫入參朝政。溫上疏曰:



 方攘除群凶,
 掃平禍亂,當竭天下智力,與眾共濟,而朝議咸疑,聖詔彌固,事異本圖,豈敢執遂!至於入參朝政,非所敢聞。臣違離宮省二十餘載,鞸奉戎務,役勤思苦,若得解帶逍遙,鳴玉闕廷,參贊無為之契,豫聞曲成之化,雖實不敏,豈不是願!但顧以江漢艱難,不同曩日,而益梁新平,寧州始服,懸兵漢川,戍禦彌廣,加彊蠻盤牙,勢處上流,江湖悠遠,當制命侯伯,自非望實重威,無以鎮御遐外。臣知捨此之艱危,敢背之而無怨,願奮臂投身造事中原者,實恥帝道皇居仄陋於東南,痛神華桑梓遂埋於戎狄。若憑宗廟之靈,則雲徹席卷,呼吸蕩清。如當假息
 游魂,則臣據河洛,親臨二寇,廣宣皇靈,襟帶秦趙,遠不五載,大事必定。



 今臣昱以親賢贊國,光輔二世,即無煩以臣疏鈍,並是機務。且不有行者,誰扞牧圉?表裏相濟,實深實重。伏願陛下察臣所陳,兼訪內外,乞時還屯,撫寧方隅。



 詔不許,復徵溫。溫至赭圻,詔又使尚書車灌止之,溫遂城赭圻,固讓內錄,遙領揚州牧。屬鮮卑攻洛陽,陳祐出奔,簡文帝時輔政,會溫于洌洲,議征討事,溫移鎮姑孰。會哀帝崩,事遂寢。



 溫性儉,每燕惟下七奠柈茶果而已。然以雄武專朝,窺覦非望,或臥對親僚曰:「為爾寂寂,將為文景所笑。」眾莫敢對。既而撫枕起曰:「既不能
 流芳後世,不足復遺臭萬載邪!」嘗行經王敦墓,望之曰:「可人,可人!」其心迹若是。時有遠方比丘尼名有道術,於別室浴,溫竊窺之。尼惈身先以刀自破腹,次斷兩足。浴竟出,溫問吉凶,尼云:「公若作天子,亦當如是。」



 太和四年,又上疏悉眾北伐。平北將軍郗愔以疾解職,又以溫領平北將軍、徐兗二州刺史,率弟南中郎沖、西中郎袁真步騎五萬北伐。百官皆於南州祖道,都邑盡傾。軍次湖陸,攻慕容將慕容忠,獲之,進次金鄉。時亢旱,水道不通,乃鑿鉅野三百餘里以通舟運,自清水入河。將慕容垂、傅末波等率眾八萬距溫,戰于林渚。溫擊破之,遂
 至枋頭。先使袁真伐譙梁,開石門以通運。真討譙梁皆平之,而不能開石門,軍糧竭盡。溫焚舟步退,自東燕出倉垣,經陳留,鑿井而飲,行七百餘里。垂以八千騎追之,戰于襄邑,溫軍敗績,死者三萬人。溫甚恥之,歸罪於真,表廢為庶人。真怨溫誣己,據壽陽以自固,潛通苻堅、慕容。



 帝遣侍中羅含以牛酒犒溫於山陽,使會稽王昱會溫于途中,詔以溫世子給事熙為征虜將軍、豫州刺史、假節。及南康公主薨,詔賻布千匹,錢百萬,溫辭不受。又陳息熙三年之孤,且年少未宜使居偏任,詔不許。發州人築廣陵城,移鎮之。時溫行役既久,又兼疾癘,死者
 十四五,百姓嗟怨。



 袁真病死,其將朱輔立其子瑾以嗣事。慕容、苻堅並遣軍授瑾,溫使督護竺瑤、矯陽之等與水軍擊之。時軍已至,瑤等與戰於武丘,破之。溫率二萬人自廣陵又至,瑾嬰城固守,溫築長圍守之。苻堅乃使其將王鑒、張蠔等率兵以救瑾,屯洛澗,先遣精騎五千次于肥水北。溫遣桓伊及弟子石虔等逆擊,大破之,瑾眾遂潰,生擒之,并其宗族數十人及朱輔送于京都而斬之,所侍養乞活數百人悉坑之,以妻子為賞。溫以功,詔加班劍十人,犒軍於路次,文武論功賞賜各有差。



 溫既負其才力,久懷異志,欲先立功河朔,還受九
 錫。既逢覆敗,名實頓減,於是參軍郗超進廢立之計,溫乃廢帝而立簡文帝。詔溫依諸葛亮故事,甲仗百人入殿,賜錢五千萬,絹二萬匹,布十萬匹。溫多所廢徒,誅庾倩、殷涓、曹秀等。是時溫威勢翕赫,侍中謝安見而遙拜,溫驚曰:「安石,卿何事乃爾!」安曰:「未有君拜於前,臣揖於後。」時溫有腳疾,詔乘輿入朝,既見,欲陳廢立本意,帝便泣下數十行,溫兢懼,不得一言而出。



 初,元明世,郭璞為讖曰:「君非無嗣,兄弟代禪。」謂成帝有子,而以國祚傳弟。又曰:「有人姓李,兒專征戰。譬如車軸,脫在一面。」兒者,子也;李去子木存,車去軸為亙,合成「桓」字也。又曰:「爾來,爾
 來,河內大縣。」爾來謂自爾已來為元始,溫字元子也;故河內大縣,溫也。成康既崩,桓氏始大,故連言之。又曰:「賴子之薨,延我國祚。痛子之隕,皇運其暮。」二子者,元子、道子也。溫志在篡奪,事未成而死,幸之也。會稽王道子雖首亂晉國,而其死亦晉衰之由也,故云痛也。



 溫復還白石,上疏求歸姑孰。詔曰:「夫乾坤體合,而化成萬物;二人同心,則不言所利。古之哲王咸賴元輔,姬旦光于四表,而周道以隆;伊尹格於皇天,而殷化以洽。大司馬明德應期,光大深遠,上合天心,含章時發,用集大命,在予一人,功美博陸,道固萬世。今進公丞相,其大司馬本官皆
 如故,留公京都,以鎮社稷。」溫固辭,仍請還鎮。遣侍中王坦之徵溫人相,增邑為萬戶,又辭。詔以西府經袁真事故,軍用不足,給世子熙布三萬匹,米六萬斛,又以熙弟濟為給事中。



 及帝不豫,詔溫曰:「吾遂委篤,足下便入,冀得相見。便來,便來!」於是一日一夜頻有四詔。溫上疏曰:「聖體不和,以經積日,愚心惶恐,無所寄情。夫盛衰常理,過備無害,故漢高枕疾,呂后問相,孝武不豫,霍光啟嗣。嗚噎以問身後,蓋所存者大也。今皇子幼稚,而朝賢時譽惟謝安、王坦之才識智皆簡在聖鑒。內輔幼君,外禦彊寇,實群情之大懼,然理盡於此。陛下便宜崇授,使
 群下知所寄,而安等奉命陳力,公私為宜。至如臣溫位兼將相,加陛下垂布衣之顧,但朽邁疾病,懼不支久,無所復堪託以後事。」疏未及奏而帝崩,遺詔家國事一稟之於公,如諸葛武侯、王丞相故事。溫初望簡文臨終禪位於己,不爾便為周公居攝。事既不副所望,故甚憤怨,與弟沖書曰:「遺詔使吾依武侯、王公故事耳。」王、謝處大事之際,日憤憤少懷。



 及孝武即位,詔曰:「先帝遺敕云:『事大司馬如事吾。』令答表便可盡敬。」又詔:「大司馬社稷所寄,先帝託以家國,內外眾事便就關公施行。」復遣謝安征溫入輔,加前部羽葆鼓吹,武賁六十人,溫讓不受。及
 溫入朝,赴山陵,詔曰:「公勛德尊重,師保朕躬,兼有風患,其無敬。」又敕尚書安等於新亭奉迎,百僚皆拜于道側。當時豫有位望者咸戰懾失色,或云因此殺王、謝,內外懷懼。溫既至,以盧悚入宮,乃收尚書陸始付廷尉,責替慢罪也。於是拜高平陵,左右覺其有異,既登車,謂從者曰:「先帝向遂靈見。」既不述帝所言,故眾莫之知,但見將拜時頻言「臣不敢」而已。又問左右殷涓形狀,答者言肥短,溫云:「向亦見在帝側。」初,殷浩既為溫所廢死,涓頗有氣尚,遂不詣溫,而與武陵王晞游,故溫疑而害之,竟不識也。及是,亦見涓為祟,因而遇疾。凡停京師十有四日,
 歸于姑孰,遂寢疾不起。諷朝廷加己九錫,累相催促。謝安、王坦之聞其病篤,密緩其事。錫文未及成而薨,時年六十二。皇太后與帝臨于朝堂三日,詔賜九命袞冕之服,又朝服一具,衣一襲,東園秘器,錢二百萬,布二千匹,臘五百斤,以供喪事。及葬,一依太宰安平獻王、漢大將軍霍光故事,賜九旒鸞輅,黃屋左纛,縕輬車,挽歌二部,羽葆鼓吹,武賁班劍百人,優冊即前南郡公增七千五百戶,進地方三百里,賜錢五千萬,絹二萬匹,布十萬匹,追贈丞相。



 初,沖問溫以謝安、王坦之所任,溫曰:「伊等不為汝所處分。」溫知己存彼不敢異,害之無益於沖,更失
 時望,所以息謀。



 溫六子:熙、濟、歆、禕、偉、玄。熙字伯道,初為世子,後以才弱,使沖領其眾。及溫病,熙與叔祕謀殺沖,沖知之,徙于長沙。濟字仲道,與熙同謀,俱徙長沙。歆字叔道,賜爵臨賀公。禕最愚,不辨菽麥。偉字幼道,平厚篤實,居籓為士庶所懷。歷使持節、督荊益寧秦梁五州諸軍事、安西將軍、領南蠻校尉、荊州刺史、西昌侯,贈驃騎將軍、開府儀同三司。玄嗣爵,別有傳。



 孟嘉字萬年,江夏鄳人,吳司空宗曾孫也。嘉少知名,太尉庾亮領江州,辟部廬陵從事。嘉還都,亮引問風俗得
 失,對曰:「還傳當問吏。」亮舉麈尾掩口而笑,謂弟翼曰:「孟嘉故是盛德人。」轉勸學從事。褚裒時為豫章太守,正旦朝亮,裒有器識,亮大會州府人士,嘉坐次甚遠。裒問亮:「聞江州有孟嘉,其人何在?」亮曰:「在坐,卿但自覓。」裒歷觀,指嘉謂亮曰:「此君小異,將無是乎?」亮欣然而笑,喜裒得嘉,奇嘉為裒所得,乃益器焉。



 後為征西桓溫參軍,溫甚重之。九月九日,溫燕龍山,僚佐畢集。時佐吏並著戎服,有風至,吹嘉帽墮落,嘉不之覺。溫使左右勿言,欲觀其舉止。嘉良久如廁,溫令取還之,命孫盛作文嘲嘉,著嘉坐處。嘉還見,即答之,其文甚美,四坐嗟歎。



 嘉好酣飲,愈
 多不亂。溫問嘉:「酒有何好?而卿嗜之?」嘉曰:「公未得酒中趣耳。」又問:「聽妓,絲不如竹,竹不如肉,何謂也?」嘉答曰:「漸近使之然。」一坐咨嗟。轉從事中郎,遷長史。年五十三卒於家。



 史臣曰:桓溫挺雄豪之逸氣,韞文武之奇才,見賞通人,夙標令譽。時既豺狼孔熾,疆場多虞,受寄扞城,用恢威略,乃踰越險阻,戡定岷峨,獨剋之功,有可稱矣。及觀兵洛汭,脩復五陵,引旆秦郊,威懷三輔,雖未能梟除兇逆,亦足以宣暢王靈。既而總戎馬之權,居形勝之地,自謂英猷不世,勳績冠時。挾震主之威,蓄無君之志,企景文
 而慨息,想處仲而思齊,睥睨漢廷,窺覦周鼎。復欲立奇功於趙魏,允歸望於天人;然後步驟前王,憲章虞夏。逮乎石門路阻,襄邑兵摧,懟謀略之乖違,恥師徒之撓敗,遷怒於朝廷,委罪於偏裨,廢主以立威,殺人以逞欲,曾弗知寶命不可以求得,神器不可以力征。豈不悖哉!豈不悖哉!斯寶斧鋮之所宜加,人神之所同棄。然猶存極光寵,沒享哀榮,是知朝政之無章,主威之不立也。



 贊曰:播越江濆,政弱權分。元子悖力,處仲矜勛。跡既陵上,志亦無君。罪浮浞,心窺舜禹。樹威外略,稱兵內侮。惟身與嗣,竟罹齊斧。



\end{pinyinscope}