\article{帝紀第七}

\begin{pinyinscope}

 成帝康
 帝



 成皇帝諱衍,字世根,明帝長子也。太寧三年三月戊辰,立為皇太子。閏月戊子,明帝崩。己丑,太子即皇帝位,大赦,增文武位二等,賜鰥寡孤老帛,人二匹,尊皇后庾氏為皇太后。秋九月癸卯,皇太后臨朝稱制。司徒王導錄尚書事,與中書令庾亮參輔朝政。以撫軍將軍、南頓王宗為驃騎將軍,領軍將軍、汝南王祐為衛將軍。辛丑,葬
 明帝於武平陵。冬十一月癸巳朔,日有蝕之。廣陵相曹渾有罪,下獄死。



 咸和元年春二月丁亥,大赦,改元,大酺五日,賜鰥寡孤老米,二人斛,京師百里內復一年。夏四月,石勒遣其將石生寇汝南,汝南人執內史祖濟以叛。甲子,尚書左僕射鄧攸卒。五月,大水。六月癸亥,使持節、散騎常侍、監淮北諸軍事、北中郎將、徐州刺史、泉陵公劉遐卒。癸酉,以車騎將軍郗鑒領徐州刺史,征虜將軍郭默為北中郎將、假節、監淮北諸軍。劉遐部曲將李龍、史迭奉遐子肇代遐位以距默,臨淮太守劉矯擊破之,斬龍,傳首京師。
 秋七月癸丑,使持節、都督江州諸軍事、江州刺史、平南將軍、觀陽伯應詹卒。八月,以給事中、前將軍、丹陽尹溫嶠為平南將軍、假節、都督,江州刺史。九月,旱。李雄將張龍寇涪陵,執太守謝俊。冬十月,封魏武帝玄孫曹勵為陳留王,以紹魏。丙寅,衛將軍、汝南王祐薨。己巳,封皇弟岳為吳王。車騎將軍、南頓王宗有罪,伏誅,貶其族為馬氏。免太宰、西陽王羕,降為弋陽縣王。庚辰,赦百里內五歲以下刑。是月,劉曜將黃秀、帛成寇酂,平北將軍魏該帥眾奔襄陽。十一月壬子,大閱於南郊。改定王侯國秩,九分食一。石勒將石聰攻壽陽,不剋,遂侵逡遒、阜陵。加
 司徒王導大司馬、假黃鉞、都督中外征討諸軍事以禦之。歷陽太守蘇峻遣其將韓晃討石聰,走之。時大旱,自六月不雨,至於是月。十二月,濟岷太守劉闓殺下邳內史夏侯嘉,叛降石勒。梁王翹薨。



 二年春正月,寧州秀才龐遺起義兵,攻李雄將任回、李謙等,雄遣其將羅恒、費黑救之。寧州刺史尹奉遣裨將姚岳、朱提太守楊術援遺,戰于臺登,岳等敗績,術死之。三月,益州地震。夏四月,旱。己未,豫章地震。五月甲申朔,日有蝕之。丙戌,加豫州刺史祖約為鎮西將軍。戊子,京師大水。冬十月,劉曜使其子胤侵枹罕,遂略河南地。十
 一月,豫州刺史祖約、歷陽太守蘇峻等反。十二月辛亥,蘇峻使其將韓晃入姑孰,屠于湖。壬子,彭城王雄、章武王休叛,奔峻。庚申,京師戒嚴。假護軍將軍庾亮節為征討都督,以右衛將軍趙胤為冠軍將軍、歷陽太守,使與左將軍司馬流帥師距峻,戰於慈湖,流敗,死之。假驍騎將軍鐘雅節,帥舟軍,與趙胤為前鋒,以距峻。丙寅,徒封瑯邪王昱為會稽王,吳王岳為瑯邪王。辛未,宣城內桓彞及峻戰於蕪湖,彞軍敗績。軍騎將軍郗鑒遣廣陵相劉矩帥師赴京師。



 三年春正月,平南將軍溫嶠帥師救京師,次於尋陽,遣
 督護王愆期、西陽太守鄧嶽、鄱陽太守紀睦為前鋒。征西大將軍陶侃遣督護龔登受嶠節度。鐘雅、趙胤等次慈湖,王愆期、鄧嶽等次直瀆。丁未,峻濟自橫江,登牛渚。二月庚戌,峻至于蔣山。假領軍將軍卞壼節,帥六軍,及峻戰於西陵,王師敗績。丙辰,峻攻青溪柵,因風縱火,王師又大敗。尚書令、領軍將軍卞壼,丹陽尹羊曼,黃門侍郎周導,廬江太守陶瞻並遇害,死者數千人。庾亮又敗于宣陽門內,遂攜其諸弟與郭默、趙胤奔尋陽。於是司徒王導、右光祿大夫陸曄、荀崧等衛帝於太極殿,太常孔愉守宗廟。賊乘勝麾戈接於帝座,突入太后後宮,左
 右侍人皆見掠奪。是時太官唯有燒餘米數石,以供御膳。百姓號泣,響震都邑。丁巳,峻矯詔大赦,又以祖約為侍中、太尉、尚書令,自為驃騎將軍、錄尚書事。吳郡太守庾冰奔于會稽。三月丙子,皇太后庾氏崩。夏四月,石勒攻宛,南陽太守王國叛,降于勒。壬申,葬明穆皇后于武平陵。五月乙未,峻逼遷天子于石頭,帝哀泣升車,宮中慟哭。峻以倉屋為宮,遣管商、張瑾、弘徽寇晉陵,韓晃寇義興。吳興太守虞潭與庚冰、王舒等起義兵於三吳。丙午,征西大將軍陶侃、平南將軍溫嶠、護軍將軍庾亮、平北將軍魏該舟軍四萬,次于蔡洲。六月,韓晃攻宣城,內
 史桓彞力戰,死之。壬辰,平北將軍、雍州刺史魏該卒于師。廬江太守毛寶攻賊合肥戍,拔之。秋七月,祖約為石勒將石聰所攻,眾潰,奔于歷陽。石勒將石季龍攻劉曜於蒲阪。八月,曜及石季龍戰于高候,季龍敗績,曜遂圍石生於洛陽。九月戊申,司徒王導奔于白石。庚午,陶侃使督護楊謙攻峻于石頭。溫嶠、庾亮陣于白石,竟陵太守李陽距賊南偏。峻輕騎出戰,墜馬,斬之,眾遂大潰。賊黨復立峻弟逸為帥。前交州刺史張璉據始興反,進攻廣州,鎮南司馬曾勰等擊破之。冬十月,李雄將張龍寇涪陵,太守趙弼沒於賊。十二月乙未,石勒敗劉曜於洛
 陽,獲之。是歲,石勒將石季龍攻氐帥蒲洪於隴山,降之。



 四年春正月,帝在石頭,賊將匡術以苑城歸順,百官赴焉。侍中鐘雅、右衛將軍劉超謀奉帝出,為賊所害。戊辰,冠軍將軍趙胤遣將甘苗討祖約于歷陽,敗之,約奔於石勒,其將牽騰帥眾降。峻子碩攻臺城,又焚太極東堂、秘閣,皆盡。城中大饑,米斗萬錢。二月,大雨霖。丙戌,諸軍攻石頭。李陽與蘇逸戰於柤浦,陽軍敗。建威長史滕含以銳卒擊之,逸等大敗。含奉帝御於溫嶠舟,群臣頓首號泣請罪。弋陽王羕有罪,伏誅。丁亥,大赦。時兵火之後,宮闕灰燼,以建平園為宮。甲午,蘇逸以萬餘人自延陵
 湖將入吳興。乙未,將軍王允之及逸戰于溧陽,獲之。壬寅,以湘州并荊州。劉曜太子毗與其大司馬劉胤帥百官奔于上邽,關中大亂。三月壬子,以征西大將軍陶侃為太尉,封長沙郡公;車騎將軍郗鑒為司空,封南昌縣公;平南將軍溫嶠為驃騎將軍、開府儀同三司,封始安郡公。其餘封拜各有差。庚午,以右光祿大夫陸曄為衛將軍、開府儀同三司,復封高密王紘為彭城王。以護軍將軍庾亮為平西將軍、都督揚州之宣城江西諸軍事、假節,領豫州刺史,鎮蕪湖。夏四月乙未,驃騎將軍、始安公溫嶠卒。秋七月,有星孛於西北。會稽、吳興、宣城、丹陽
 大水。詔復遭賊郡縣租稅三年。八月,利曜將劉胤等帥眾侵石生,次於雍。九月,石勒將石季龍擊胤,斬之,進屠上邽,盡滅劉氏,坑其黨三千餘人。冬十月,廬山崩。十二月壬辰,右將軍郭默害平南將軍、江州刺史劉胤,太尉陶侃帥眾討默。是歲,天裂西北。



 五年春正月己亥,大赦。癸亥,詔除諸將任子。二月,以尚書陸玩為尚書左僕射,孔愉為右僕射。夏五月,旱,且饑疫。乙卯,太尉陶侃擒郭默於尋陽,斬之。石勒將劉徵寇南沙,都尉許儒遇害,進入海虞。六月癸巳,初稅田,畝三升。秋八月,石勒僭即皇帝位,使其將郭敬寇襄陽。南中
 郎將周撫退歸武昌,中州流人悉降于勒。郭敬遂寇襄陽,屯于樊城。九月,造新宮,始繕苑城。甲辰,徒樂成王欽為河間王,封彭城王紘子浚為高密王。冬十月丁丑,幸司徒王導第,置酒大會。李雄將李壽寇巴東、建平,監軍毌丘奧、太守楊謙退歸宜都。十二月,張駿稱臣於石勒。



 六年春正月癸巳,劉徵復寇婁縣,遂掠武進。乙未,進司空郗鑒都督吳國諸軍事。戊午,以運漕不繼,發王公已下千餘丁,各運米六斛。二月己丑,以幽州刺史、大單于段遼為驃騎將軍。三月壬戌朔,日有蝕之。癸未,詔舉賢良直言之士。夏四月,旱。六月丙申,復故河間王顒爵位,
 封彭城王植子融為樂成王,章武王混子珍為章武王。秋七月,李雄將李壽侵陰平,武都氐帥楊難敵降之。八月庚子,以左僕射陸玩為尚書令。



 七年春正月辛未,大赦。三月,西中郎將趙胤、司徒中郎匡術攻石勒馬頭塢,剋之。勒將韓雍寇南沙及海虞。夏四月,勒將郭敬陷襄陽。五月,大水。秋七月丙辰,詔諸養獸之屬,損費者多,一切除之。太尉陶侃遣子平西參軍斌與南中郎將桓宣攻石勒將郭敬,破之,剋樊城。竟陵太守李陽拔新野、襄陽,因而戍之。冬十一月壬子朔,進太尉陶侃為大將軍。詔舉賢良。十二月庚戌,帝遷于新
 宮。



 八年春正月辛亥朔,詔曰:「昔犬賊縱暴,宮室焚蕩,元惡雖翦,未暇營築。有司屢陳,朝會逼狹,遂作斯宮,子來之勞,不日而成。既獲臨御,大饗群后,九賓充誕,百官象物。知君子勤禮,小人盡力矣。思蠲密綱,咸同斯惠,其赦五歲刑以下。」令諸郡舉力人能舉千五百斤以上者。丙寅,李雄將李壽陷寧州,刺史尹奉及建寧太守霍彪並降之。癸酉,以張駿為鎮西大將軍。丙子,石勒遣使致賂,詔焚之。夏四月,詔封故新蔡王弼弟邈為新蔡王。以束帛徵處士尋陽翟湯、會稽虞喜。五月,有星隕于肥鄉。麒麟、
 騶虞見于遼東。乙未,車騎將軍、遼東公慕容廆卒,子皝嗣位。六月甲辰,撫軍將軍王舒卒。秋七月戊辰,石勒死,子弘嗣偽位,其將石聰以譙來降。冬十月,石弘將石生起兵於關中,稱秦州刺史,遣使來降。石弘將石季龍攻石朗於洛陽,因進擊石生,俱滅之。十二月,石生故部將郭權遣使請降。



 九年春正月,隕石於涼州二。以郭權為鎮西將軍、雍州刺史。二月丁卯,加鎮西大將軍張駿為大將軍。三月丁酉,會稽地震。夏四月,石弘將石季龍使石斌攻郭權於郿,陷之。六月,李雄死,其兄子班嗣偽位。乙卯,太尉、長沙
 公陶侃薨。大旱,詔太官撤膳;省刑,恤孤寡,貶費節用。辛末,加平西將軍庾亮都督江、荊、豫、益、梁、雍六州諸軍事。秋八月,大雩。自五月不雨,至于是月。九月戊寅,散騎常侍,衛將軍、江陵公陸曄卒。冬十月,李雄子期弒李班而自立,班弟玝與其將焦會、羅凱等並來降。十一月,石季龍弒石弘,自立為天王。十二月丁卯,以東海王沖為車騎將軍,琅邪王岳為驃騎將軍。蘭陵人朱縱斬石季龍將郭祥,以彭城來降。



 咸康元年春正月庚午朔,帝加元服,大赦,改元,增文武位一等,大酺三日,賜鰥寡孤獨不能自存者米,人五斛。
 二月甲子,帝親釋奠。揚州諸郡饑,遣使振給。三月乙酉,幸司徒府。夏四月癸卯,石季龍寇歷陽,加司徒王導大司馬、假黃鉞、都督征討諸軍事,以禦之。癸丑,帝觀兵于廣莫門,分命諸將,遣將軍劉仕救歷陽,平西將軍趙胤屯慈湖,龍驤將軍路永戍牛渚,建武將軍王允之戍蕪湖。司空郗鑒使廣陵相陳光帥眾衛京師,賊退向襄陽。戊午,解嚴。石季龍將石遇寇中廬,南中郎將王國退保襄陽。秋八月,長沙、武陵大水。束帛徵處士翟湯、郭翻。冬十月乙未朔,日有蝕之。是歲,大旱,會稽餘姚尤甚,米斗五百價,人相賣。



 二年春正月辛巳,彗星見于奎。以吳國內史虞潭為衛將軍。二月,算軍用稅米,空懸五十餘萬石,尚書謝褒巳下免官。辛亥,立皇后杜氏,大赦,增文武位一等。庚申,高句驪遣使貢方物。三月,旱,詔太官減膳,免所旱郡縣繇役。戊寅,大雩。夏四月丁巳,皇后見于太廟。雨雹。秋七月,揚州會稽饑,開倉振給。冬十月,廣州刺史鄧嶽遣督護王隨擊夜郎,新昌太守陶協擊興古,並剋之。詔曰:「歷觀先代,莫不褒崇明祀,賓禮三恪。故杞宋啟土,光于周典;宗姬侯衛,垂美漢冊。自頃喪亂,庶邦殄悴,周漢之後,絕而莫繼。其祥求衛公、山陽公近屬,有履行修明,可以繼
 承其祀者,依舊典施行。」新作朱雀浮桁。十一月,遣建威將軍司馬勛安集漢中,為李期將李壽所敗。



 三年春正月辛卯,立太學。夏六月,旱。冬十一月丁卯,慕容皝自立為燕王。



 四年春二月,石季龍帥眾七萬,擊段遼于遼西,遼奔于平崗。夏四月,李壽弒李期。僭即偽位,國號漢。石季龍為慕容皝所敗,癸丑,加皝征北大將軍。五月乙未,以司徒王導為太傅、都督中外諸軍事,司空郗鑒為太尉,征西將軍庾亮為司空。六月,改司徒為丞相,以太傅王導為之。秋八月丙午,分寧州置安州。



 五年春正月辛丑,大赦。三月乙丑,廣州刺史鄧嶽伐蜀,建寧人孟彥執李壽將霍彪以降。夏四月辛未,征西將軍庾亮遣參軍趙松擊巴郡、江陽,獲石季龍將李閎、黃桓等。秋七月庚申,使持節、侍中、丞相、領揚州刺史、始興公王導薨。辛酉,以護軍將軍何充錄尚書事。八月壬午,復改丞相為司徒。辛酉,太尉、南昌公郗鑒薨。九月,石季龍將夔安、李農陷沔南,張貉陷邾城,因寇江夏、義陽,征虜將軍毛寶、西陽太守樊俊、義陽太守鄭進並死之。夔安等進圍石城,竟陵太守李陽距戰,破之,斬首五千餘級。安乃退,遂略漢東,擁七千餘家遷于幽冀。冬十二月
 丙戌,以驃騎將軍、瑯邪王岳為司徒。李壽將李奕寇巴東,守將勞揚戰敗,死之。



 六年春正月庚子,使持節、都督江豫益梁雍交廣七州諸軍事、司空、都亭侯庾亮薨。辛亥,以左光祿大夫陸玩為司空。二月,慕容皝及石季龍將石成戰于遼西,敗之,獻捷于京師。庚辰,有星孛于太微。三月丁卯,大赦。以車騎將軍、東海王沖為驃騎將軍。李壽陷丹川,守將孟彥、劉齊、李秋皆死之。秋七月乙卯,初依中興故事,朔望聽政於東堂。冬十月,林邑獻馴象。十一月癸卯,復瑯邪,比漢豐沛。



 七年春二月甲子朔,日有蝕之,己卯,慕容皝遣使求假燕王章璽,許之。三月戊戌,杜皇后崩。夏四月丁卯。葬恭皇后于興平陵。實編戶,王公已下皆正土斷白籍。秋八月辛酉,驃騎將軍、東海王沖薨。九月,罷太僕官。冬十二月癸酉,司空、興平伯陸玩薨。除樂府雜伎。罷安州。



 八年春正月己未朔,日有蝕之。乙丑,大赦。三月,初以武悼楊皇后配饗武帝廟。夏六月庚寅,帝不豫,詔曰:「朕以眇年,獲嗣洪緒,託於王公之上,於茲十有八年。未能闡融政道。翦除逋昆,夙夜戰兢,匪遑寧處。今遘疾殆不興,是用震悼于厥心。千齡眇眇,未堪艱難。司徒、瑯邪王岳,
 親則母弟,體則仁長,君人之風,允塞時望。肆爾王公卿士,其輔之!以祗奉祖宗明祀,協和內外,允執其中。嗚呼,敬之哉!無墜祖宗之顯命。」壬辰,引武陵王晞、會稽王昱、中書監庾冰、中書令何充、尚書令諸葛恢並受顧命。癸巳,帝崩于西堂,時年二十二,葬興平陵,廟號顯宗。



 帝少而聰敏,有成人之量。南頓王宗之誅也,帝不之知,及蘇峻平,問庾亮曰:「常日白頭公何在?」亮對以謀反伏誅,帝泣謂亮曰:「舅言人作賊,便殺之,人言舅作賊,復若何?」亮懼,變色。庾懌嘗送酒於江州刺史王允之,允之與犬,犬斃,懼而表之。帝怒曰:「大舅已亂天下,小舅復欲爾邪?」懌
 聞,飲藥而死。然少為舅氏所制,不親庶政。及長,頗留心萬機,務在簡約,常欲於後園作射堂,計用四十金,以勞費乃止。雄武之度,雖有愧於前王;恭儉之德,足追蹤於住烈矣。



 康皇帝諱岳,字世同,成帝母弟也。咸和元年封吳王,二年徙封瑯邪王;九年拜散騎常侍,加驃騎將軍,咸康五年遷侍中、司徒。八年六月庚寅,成帝不豫,詔以瑯邪王為嗣。癸巳,成帝崩。甲午,即皇帝位,大赦。諸屯戍文武及二千石官長,不得輒離所局而來奔赴。己亥,封成帝子
 丕為瑯邪王,奕為東海王。時帝諒陰不言,委政于庾冰、何充。秋七月丙辰,葬成皇帝于興平陵。帝親奉奠於西階,既發引,徒行至閶闔門,升素輿,至於陵所。己未,以中書令何充為驃騎將軍。八月辛丑,彭城王紘薨。以江州刺史王允之為衛將軍。九月,詔瑯邪國及府史進位各有差。冬十月甲午,衛將軍王允之卒。十二月,增文武位二等。壬子,立皇后褚氏。



 建元元年春正月,改元,振恤鰥寡孤獨。三月,以中書監庾冰為車騎將軍。夏四月,益州刺史周撫、西陽太守曹據伐李壽,敗其將恒于江陽。五月,旱。六月壬午,又以
 束帛徵處士尋陽翟湯、會稽虞喜。有司奏,成帝崩一周,請改素服,御進膳如舊。壬寅,詔曰:「禮之降殺,因時而寢興,誠無常矣。至于君親相準,名教之重,莫之改也。權制之作,蓋出近代,雖曰適事,實弊薄之始。先王崇之,後世猶怠,而況因循,又從輕降,義弗可矣。」石季龍帥眾伐慕容皝,皝大敗之。秋七月,石季龍將戴開帥眾來降。丁巳,詔曰:「慕容皝摧殄羯寇,乃云死沒八萬餘人,將是其天亡之始也。中原之事,宜加籌量。且戴開已帥部黨歸順,宜見慰勞。其遣使詣安西、驃騎,咨謀諸軍事。」以輔國將軍、琅邪內史桓溫為前鋒小督、假節,帥眾入臨淮,安西
 將軍庾翼為征討大都督,遷鎮襄陽。庚申,晉陵、吳郡災。八月,李壽死,子勢嗣偽位。石季龍使其將劉寧攻陷狄道。冬十月辛巳,以車騎將軍庾冰都督荊江司雍益梁六州諸軍事、江州刺史,以驃騎將軍何充為中書監、都督揚豫二州諸軍事、揚州刺史、錄尚書事,輔政。以瑯邪內史桓溫都督青徐兗三州諸軍事、徐州刺史,褚裒為衛將軍、領中書令。十一月己巳,大赦。十二月,石季龍侵張駿,駿使其將軍謝艾拒之,大戰于河西,季龍敗績。十二月,高句驪遣使朝獻。



 二年春正月,張駿遣其將和驎、謝艾討南羌于闐和,大
 破之。二月,慕容皝及鮮卑帥宇文歸戰于昌黎,歸眾大敗,奔于漠北。四月,張駿將張瓘敗石季龍將王擢于三交城。秋八月丙子,進安西將軍庾翼為征西將軍。庚辰,持節、都督司雍梁三州諸軍事、梁州刺史、平北將軍、竟陵公桓宣卒。丁巳,以衛將軍褚裒為特進、都督徐兗二州諸軍事、兗州刺史,鎮金城。九月,巴東太守楊謙擊李勢將申陽,走之,獲其將樂高。丙申,立皇子聃為皇太子。戊戌,帝崩于式乾殿。時年二十三,葬崇平陵。



 初,成帝有疾,中書令庾冰自以舅氏當朝,權侔人主,恐異世之後,戚屬將疏,乃言國有彊敵,宜立長君,遂以帝為嗣。制度
 年號,再興中朝,因改元曰建元。或謂冰曰:「郭璞讖云『立始之際丘山傾』,立者,建也;始者,元也;丘山,諱也。」冰瞿然,既而歎曰:「如有吉凶,豈改易所能救乎?」至是果驗云。



 史臣曰:肆虐滔天,豈伊朝夕。若乃詳刑不怨,庶情猶仰,又可以見逆順之機焉。成帝因削弱之資,守江淮之地,政出渭陽,聲乖威服。凶徒既縱,神器阽危,京華元敖之資,宮室類咸陽之火。桀犬吠堯。封狐嗣亂,方諸后羿,曷若斯之甚也。反我皇駕,不有晉文之師,繫于苞桑,且賴陶公之力。古之侯服,不幸臣家,天子宣遊,則避宮北面,聞諸遺策,用為恒範。顯宗於王導之門,斂衣前拜,豈
 魯公受玉之卑乎!帝亦克儉於躬,庶能激揚流弊者也。



 贊曰:惟皇夙表,餘舅為毗。勤於致寇,拙於行師。火及君屋,兵纏帝帷。石頭之駕,海內含悲。康後天資,居哀禮縟。墜典方興,降齡奚促。



\end{pinyinscope}