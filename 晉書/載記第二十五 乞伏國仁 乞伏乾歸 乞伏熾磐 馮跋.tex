\article{載記第二十五 乞伏國仁 乞伏乾歸 乞伏熾磐 馮跋}

\begin{pinyinscope}
乞伏國仁乞伏乾歸乞伏熾磐馮跋
 \gezhu{
  馮素弗}



 乞伏國仁,隴西鮮卑人也。在昔有如弗斯、出連、叱盧三部,自漠北南出大陰山,遇一巨蟲於路,狀若神龜,大如陵阜,乃殺馬而祭之,祝曰:「若善神也,便開路;惡神也,遂塞不通。」俄而不見,乃有一小兒在焉。時又有乞伏部有老父無子者,請養為子,眾咸許之。老父欣然自以有所
 依憑,字之曰紇干。紇干者,夏言依倚也。年十歲,驍勇善騎射,彎弓五百斤。四部服其雄武,推為統主,號之曰乞伏可汗託鐸莫何。託鐸者,言非神非人之稱也。其後有祐鄰者,即國仁五世祖也。泰始初,率戶五千遷于夏緣,部眾稍盛。鮮卑鹿結七萬餘落,屯于高平川,與祐鄰迭相攻擊。鹿結敗,南奔略陽,祐鄰盡並其眾,固居高平川。祐鄰死,子結權立,徙于牽屯。結權死,子利那立,擊鮮卑吐賴于烏樹山,討尉遲渴權于大非川,收眾三萬餘落。利那死,弟祁埿立。祁埿死,利那子述延立。討鮮卑莫侯于苑川,大破之,降其眾二萬餘落,固居苑川。以叔父軻
 埿為師傅,委以國政,斯引烏埿為左輔將軍,鎮蔡園川,出連高胡為右輔將軍,鎮至便川,叱盧那胡為率義將軍,鎮牽屯山。述延死,子傉大寒立。會石勒滅劉曜,懼而遷于麥田無孤山。大寒死,子司繁立,始遷于度堅山。尋為苻堅將王統所襲,部眾叛降于統。司繁歎謂左右曰:「智不距敵,德不撫眾,劍騎未交而本根已敗,見眾分散,勢亦難全。若奔諸部,必不我容,吾將為呼韓邪之計矣。」乃詣統降于堅。堅大悅,署為南單于,留之長安。以司繁叔父吐雷為勇士護軍,撫其部眾。俄而鮮卑勃寒侵斥隴右,堅以司繁為使持節、都督討西胡諸軍事、鎮西將
 軍以討之。勃寒懼而請降,司繁遂鎮勇士川,甚有威惠。



 司繁卒,國仁代鎮,及堅興壽春之役,徵為前將軍,領先鋒騎。會國仁叔父步頹叛於隴西,堅遣國仁還討之。步頹聞而大悅,迎國仁於路。國仁置酒高會,攘袂大言曰:「苻氏往因趙石之亂,遂妄竊名號,窮兵極武,跨僭八州。疆宇既寧,宜綏以德,方虛廣威聲,勤心遠略,騷動蒼生,疲弊中國,違天怒人,將何以濟!且物極則虧、禍盈而覆者,天之道也。以吾量之,是役也,難以免矣。當與諸君成一方之業。」及堅敗歸,乃招集諸部,有不附者,討而並之,眾至十餘萬。及堅為姚萇所殺,國仁謂其豪帥曰:「苻氏
 以高世之姿而困於烏合之眾,可謂天也。夫守常迷運,先達恥之;見機而作,英豪之舉。吾雖薄德,藉累世之資,豈可睹時來之運而不作乎!」以孝武太元十年自稱大都督、大將軍、大單于、領秦、河二州牧,建元曰建義。以其將乙旃音埿為左相,屋引出支為右相,獨孤匹蹄為左輔,武群勇士為右輔,弟乾歸為上將軍,自餘拜授各有差。置武城、武陽、安固、武始、漢陽、天水、略陽、漒川、甘松、匡朋、白馬、苑川十二郡,築勇士城以居之。



 鮮卑匹蘭率眾五千降。明年,南安秘宜及諸羌虜來擊國仁,四面而至。國仁謂諸將曰:「先人有奪人之心,不可坐待其至。宜抑
 威餌敵,羸師以張之,軍法所謂怒我而怠寇也。」於是勒眾五千,襲其不意,大敗之。秘宜奔還南安,尋與其弟莫侯悌率眾三萬餘戶降於國仁,各拜將軍、刺史。



 苻登遣使者署國仁使持節、大都督、都督雜夷諸軍事、大將軍、大單于、苑川王。國仁率騎三萬襲鮮卑大人密貴、裕茍、提倫等三部於六泉。高平鮮卑沒奕于、東胡金熙連兵來襲,相遇于渴渾川,大戰敗之,斬級三千,獲馬五千匹。沒奕于及熙奔還,三部震懼,率眾迎降。署密貴建義將軍、六泉侯,裕茍建忠將軍、蘭泉侯,提倫建節將軍、鳴泉侯。



 國仁建威將軍叱盧烏孤跋擁眾叛,保牽屯山。國仁
 率騎七千討之,斬其部將叱羅侯,降者千餘戶。跋大懼,遂降,復其官位。因討鮮卑越質叱黎于平襄,大破之,獲其子詰歸、弟子復半及部落五千餘人而還。



 太元十三年,國仁死,在位四年,偽謚宣烈王,廟號烈祖。



 乾歸,國仁弟也。雄武英傑,沈雅有度量。國仁之死也,其群臣咸以國仁子公府沖幼,宜立長君,乃推乾歸為大都督、大將軍、大單于、河南王,赦其境內,改元曰太初。立其妻邊氏為王后,以出連乞都為丞相,鎮南將軍、南梁州刺史悌眷為御史大夫,自餘封拜各有差。遂遷於金城。



 太元十四年,苻登遣使署乾歸大將軍、大單于、金
 城王。南羌獨如率眾七千降之。休官阿敦、侯年二部各擁五千餘落,據牽屯山,為其邊害。乾歸討破之,悉降其眾,於是聲振邊服。吐谷渾大人視連遣使貢方物。鮮卑豆留奇、叱豆渾及南丘鹿結並休官曷呼奴、盧水尉地跋並率眾降于乾歸,皆署其官爵。隴西太守越質詰歸以平襄叛,自稱建國將軍、右賢王。乾歸擊敗之,詰歸東奔隴山。既而擁眾來降,乾歸妻以宗女,署立義將軍。



 苻登將沒奕于遣使結好,以二子為質,請討鮮卑大兜國。乾歸乃與沒奕於攻大兜於安陽城,大兜退固鳴蟬堡,乾歸攻陷之,遂還金城。為呂光弟寶所攻,敗於鳴雀峽,
 退屯青岸。寶進追乾歸,乾歸使其將彭奚念斷其歸路,躬貫甲胄,連戰敗之,寶及將士投河死者萬餘人。



 苻登遣使署乾歸假黃鉞、大都督隴右河西諸軍事、左丞相、大將軍、河南王,領秦、梁、益、涼、沙五州牧,加九錫之禮。時登為姚興所逼,遣使請兵,進封乾歸梁王,命置官司,納其妹東平長公主為梁王后。乾歸遣其前將軍乞伏益州、冠軍翟瑥率騎二萬救之。會登為興所殺,乃還師。



 氐王楊定率步騎四萬伐之。乾歸謂諸將曰:「楊定以勇虐聚眾,窮兵逞欲。兵猶火也,不戢,將自焚。定之此役,殆天以之資我也。」於是遣其涼州牧乞伏軻殫、秦州牧乞伏
 益州、立義將軍詰歸距之。定敗益州於平川,軻殫、詰歸引眾而退。翟瑥奮劍諫曰:「吾王以神武之姿,開基隴右,東征西討,靡不席卷,威震秦、梁,聲光巴、漢。將軍以維城之重,受閫外之寄,宜宣力致命,輔寧家國。秦州雖敗,二軍猶全,奈何不思直救,便逆奔敗,何面目以見王乎!昔項羽斬慶子以寧楚,胡建戮監軍以成功,將軍之所聞也。瑥誠才非古人,敢忘項氏之義乎!」軻殫曰:「向所以未赴秦州者,未知眾心何如耳。敗不相救,軍罰所先,敢自寧乎!」乃率騎赴之。益州、詰歸亦勒眾而進,大敗定,斬定及首虜萬七千級。於是盡有隴西、巴西之地。



 太元十七
 年,赦其境內殊死以下,署其長子熾磐領尚書令,左長史邊芮為尚書左僕射,右長史秘宜為右僕射,翟瑥為吏部尚書,翟勍為主客尚書,杜宣為兵部尚書,王松壽為民部尚書,樊謙為三公尚書,方弘、麴景為侍中,自餘拜授一如魏武、晉文故事。猶稱大單于、大將軍。



 楊定之死也,天水姜乳襲據上邽。至是,遣乞伏益州討之。邊芮、王松壽言於乾歸曰:「益州以懿弟之親,屢有戰功,狃於累勝,常有驕色。若其遇寇,必將易之。且未宜專任,示有所先。」乾歸曰:「益州驍勇,善御眾,諸將莫有及之者,但恐其專擅耳。若以重佐輔之,當無慮也。」於是以平北韋虔
 為長史、散騎常侍務和為司馬。至大寒嶺,益州恃勝自矜,不為部陣,命將士解甲游畋縱飲,令曰:「敢言軍事者斬!」虔等諫曰:「王以將軍親重,故委以專征之任,庶能摧彼凶醜,以副具瞻。賊已垂逼,奈何解甲自寬,宴安耽毒,竊為將軍危之。」益州曰:「乳以烏合之眾,聞吾至,理應遠竄。今乃與吾決戰者,斯成擒也。吾自揣之有方,卿等不足慮也。」乳率眾距戰,益州果敗。乾歸曰:「孤違蹇叔,以至於此。將士何為,孤之罪也。」皆赦之。



 索虜禿髮如茍,率戶二萬降之,乾歸妻以宗女。



 呂光率眾十萬將伐乾歸,左輔密貴周、左衛莫者羖羝言於乾歸曰:「光旦夕將至。陛
 下以命世雄姿,開業洮罕,剋翦群光,威振遐邇,將鼓淳風於東夏,建八百之鴻慶。不忍小下屈,與姦豎兢於一時,若機事不捷,非國家利也。宜遣愛子以退之。」乾歸乃稱籓於光,遣子敕勃為質。既而悔之,遂誅周等。



 乞伏軻殫與乞伏益州不平,奔于呂光。光又伐之,咸勸其東奔成紀,乾歸不從,謂諸將曰:「昔曹孟德敗袁本初於官渡,陸伯言摧劉玄德於白帝,皆以權略取之,豈在眾乎!光雖舉全州之軍,而無經遠之算,不足憚也。且其精卒盡在呂延,延雖勇而愚,易以奇策制之。延軍若敗,光亦遁還,乘勝追奔,可以得志。」眾咸曰:「非所及也。」隆安元年,光遣
 其子纂伐乾歸,使呂延為前鋒。乾歸泣謂眾曰:「今事勢窮踧,逃命無所,死中求生,正在今日。涼軍雖四面而至,然相去遼遠,山河既阻,力不周接,敗其一軍而眾軍自退。」乃縱反間,稱秦王乾歸眾潰,東奔成紀。延信之,引師輕進,果為乾歸所敗,遂斬之。



 禿髮烏孤遣使來結和親。使乞伏益州攻剋支陽、鸇武、允吾三城,俘獲萬餘人而還。又遣益州與武衛慕容允、冠軍翟瑥率騎二萬伐吐谷渾視羆,至于度周川,大破之。視羆遁保白蘭山,遣使謝罪,貢其方物,以子宕豈為質。鮮卑疊掘河內率尸五千,自魏降乾歸。



 乾歸所居南景門崩,惡之,遂遷于苑川。姚
 興將姚碩德率眾五萬伐之,入自南安峽。乾歸次于隴西以距碩德。興潛師繼發。乾歸聞興將到,謂諸將曰:「吾自開建以來,屢摧勍敵,乘機籍算,舉無遺策。今姚興盡中國之師,軍勢甚盛。山川阻狹,無從騎之地,宜引師平川,伺其怠而擊之。存亡之機,在斯一舉,卿等戮力勉之。若梟翦姚興,關中之地盡吾有也。」於是遣其衛軍慕容允率中軍二萬遷于柏陽,鎮軍羅敦將外軍四萬遷于侯辰谷,乾歸自率輕騎數千候興軍勢。俄而大風昏霧,遂與中軍相失,為興追騎所逼,入于外軍。旦而交戰,為興所敗。乾歸遁還苑川,遂走金城,謂諸豪帥曰:「吾才非
 命世,謬為諸君所推,心存撥亂,而德非時雄,叨竊名器,年踰一紀,負乘致寇,傾喪若斯!今人眾已散,勢不得安,吾欲西保允吾,以避其鋒。若方軌西邁,理難俱濟,卿等宜安土降秦,保全妻子。」群下咸曰:「昔古公杖策,豳人歸懷;玄德南奔,荊、楚襁負。分岐之感,古人所悲,況臣等義深父子,而有心離背!請死生與陛下俱。」乾歸曰:「自古無不亡之國,廢興命也。茍天未亡我,冀興復有期。德之不建,何為俱死!公等自愛,吾將寄食以終餘年。」於是大哭而別,乃率騎數百馳至允吾,禿髮利鹿孤遣弟傉檀迎乾歸,處之於晉興。



 南羌梁戈等遣使招之。乾歸將叛,謀
 洩,利鹿孤遣弟吐雷屯于捫天嶺。乾歸懼為利鹿孤所害,謂其子熾磐曰:「吾不能負荷大業,致茲顛覆。以利鹿孤義兼姻好,冀存脣齒之援,方乃忘義背親,謀人父子,忌吾威名,勢不全立。姚興方盛,吾將歸之。若其俱去,必為追騎所及。今送汝兄弟及汝母為質,彼必不疑。吾既在秦,終不害汝。」於是送熾磐兄弟於西平,乾歸遂奔長安。姚興見而大悅,署乾歸持節、都督河南諸軍事、鎮遠將軍、河州刺史、歸義侯,遣乾歸還鎮苑川,盡以部眾配之。乾歸既至苑川,以邊芮為長史,王松壽為司馬,公卿大將已下悉降號為偏裨。



 元興元年,熾磐自西平奔長
 安,姚興以為振忠將軍、興晉太守。尋遣使者加乾歸散騎常侍、左賢王。遣隨興將齊難迎呂隆于河西,討叛羌黨龍頭于滋川,攻楊盛將苻帛于皮氏堡,並剋之。又破吐谷渾將大孩,俘獲萬餘人而還。尋復率眾攻楊盛將楊玉于西陽堡,剋之。既而苑川地震裂生毛,狐雉入于寢內,乾歸甚惡之。姚興慮乾歸終為西州之患,因其朝也,興留為主客尚書,以熾磐為建武將軍、行西夷校尉,監撫其眾。



 熾磐以長安兵亂將始,乃招結諸部二萬七千,築城于嵻良山以據之。熾磐攻剋枹罕,遣使告之,乾歸奔還苑川。鮮卑悅大堅有眾五千,自龍馬苑降乾歸。
 乾歸遂如枹罕,留熾磐鎮之。乾歸收眾三萬,遷于度堅山。群下勸乾歸稱王,乾歸以寡弱弗許。固請曰:「夫道應符歷,雖廢必興;圖籙所棄,雖成必敗。本初之眾,非不多也,魏武運籌,四州瓦解。尋、邑之兵,非不盛也,世祖龍申,亡新鳥散。固天命不可虛邀,符籙不可妄冀。姚數將終,否極斯泰,乘機撫運,實繫聖人。今見眾三萬,足可以疆理秦、隴,清蕩洮河。陛下應運再興,四海鵠望,豈宜固守謙沖,不以社稷為本!願時即大位,允副群心。」乾歸從之。義熙三年,僭稱秦王,赦其境內,改元更始,置百官,公卿已下皆復本位。



 遣熾磐討諭薄地延,師次煩于,地延率
 眾出降,署為尚書,徙其部落于苑川。又遣隴西羌昌何攻剋姚興金城郡,以其驍騎乞伏務和為東金城太守。乾歸復都苑川,又攻剋興略陽、南安、隴西諸郡,徙二萬五千戶於苑川、枹罕。姚興力未能西討,恐更為邊害,遣使署乾歸使持節、散騎常侍、都督隴西嶺北匈奴雜胡諸軍事、征西大將軍、河州牧、大單于、河南王。乾歸方圖河右,權宜受之,遂稱籓於興。



 遣熾磐與其次子中軍審虔率步騎一萬伐禿髮傉檀,師濟河,敗傉檀太子武臺於嶺南,獲牛馬十餘萬而還。又攻剋興別將姚龍于伯陽堡,王憬于永洛城,徙四千餘戶於苑川,三千餘戶于譚
 郊。乾歸率步騎三萬征西羌彭利髮于枹罕,師次于奴葵谷,利髮棄其部眾南奔。乾歸遣其將公府追及于清水,斬之。乾歸入枹罕,收羌戶一萬三千。因率騎二萬討吐谷渾支統阿若干于赤水,大破降之。



 乾歸畋于五溪,有梟集于其手,甚惡之。六年,為兄子公府所弒,并其諸子十餘人。公府奔固大夏,熾磐與乾歸弟廣武智達、揚武木奕于討之。公府走,達等追擒於嵻良南山,并其四子,轘之於譚郊。葬乾歸于枹罕,偽謚武元王,在位二十四年。



 熾磐,乾歸長子也。性勇果英毅,臨機能斷,權略過人。初,
 乾歸為姚興所敗,熾磐質於禿髮利鹿孤。後自西平逃而降興,興以為振忠將軍、興晉太守,又拜建武將軍、行西夷校尉,留其眾鎮苑川。及乾歸返政,復立熾磐為太子,領冠軍大將軍、都督中外諸軍、錄尚書事。後乾歸稱籓于姚興,興遣使署熾磐假節、鎮西將軍、左賢王、平昌公,尋進號撫軍大將軍。



 乾歸死,義熙六年,熾磐襲偽位,大赦,改元曰永康。署翟勍為相國,麴景為御史大夫,段暉為中尉,弟延祚為禁中錄事,樊謙為司直。罷尚書令、僕射、尚書、六卿、侍中、散騎常侍、黃門郎官,置中左右常侍、侍郎各三人。



 義熙九年,遣其龍驤乞伏智達、平東王
 松壽討吐谷渾樹洛干於澆河,大破之,獲其將呼那烏提,虜三千餘戶而還。又遣其鎮東曇達與松壽率騎一萬,東討破休官權小郎、呂破胡于白石川,虜其男女萬餘口,進據白石城,休官降者萬餘人。後顯親休官權小成、呂奴迦等叛保白坑,曇達謂將士曰:「昔伯珪憑險,卒有滅宗之禍;韓約肆暴,終受覆族之誅。今小成等逆命白坑,宜在除滅。王者之師,有征無戰,粵爾輿人,戮力勉之!」眾咸拔劍大呼,於是進攻白坑,斬小成、奴迦及首級四千七百,隴右休官悉降。遣安北烏地延、冠軍翟紹討吐谷渾別統句旁于泣勤川,大破之,俘獲甚眾。熾磐率
 諸將討吐谷渾別統支旁于長柳川,掘達于渴渾川,皆破之,前後俘獲男女二萬八千。



 僭立十年,有雲五色,起於南山,熾磐以為己瑞,大悅,謂群臣曰:「吾今年應有所定,王業成矣!」於是繕甲整兵,以待四方之隙。聞禿髮辱檀西征乙弗,投劍而起曰:「可以行矣!」率步騎二萬襲樂都。禿髮武臺憑城距守,熾磐攻之,一旬而剋。遂入樂都,論功行賞各有差。遣平遠犍虔率騎五千追傉檀,徙武臺與其文武及百姓萬餘戶于枹罕。傉檀遂降,署為驃騎大將軍、左南公。隨傉檀文武,依才銓擢之。熾磐既兼傉檀,兵強地廣,置百官,立其妻禿髮氏為王后。



 十一年,
 熾磐攻剋沮渠蒙遜河湟太守沮渠漢平,以其左衛匹逵為河湟太守,因討降乙弗窟乾而還。遣其將曇達、王松壽等討南羌彌姐康薄于赤水,降之。



 熾磐攻漒川,師次沓中,沮渠蒙遜率眾攻石泉以救之。熾磐聞而引還,遣曇達與其將出連虔率騎五千赴之。蒙遜聞曇達至,引歸,遣使聘於熾磐,遂結和親。又遣曇達、王松壽等率騎一萬伐姚艾于上邽。曇達進據蒲水,艾距戰,大敗之,艾奔上邽。曇達進屯大利,破黃石、大羌二戍,徙五千餘戶於枹罕。



 令其安東木奕于率騎七千討吐谷渾樹洛干於塞上,破其弟阿柴於堯扞川,俘獲五千餘口而還,
 洛干奔保白蘭山而死。熾磐聞而喜曰:「此虜矯矯,所謂有豕白蹢。往歲曇達東征,姚艾敗走;今木奕于西討,黠虜遠逃。境宇稍清,姦凶方殄,股肱惟良,吾無患矣。」於是以曇達為左丞相,其子元基為右丞相,麴景為尚書令,翟紹為左僕射。遣曇達、元基東討姚艾,降之。



 至是,乙弗鮮卑烏地延率戶二萬降于熾磐,署為建義將軍。地延尋死,弟他子立,以子軻蘭質于西平。他子從弟提孤等率戶五千以西遷,叛于熾磐。涼州刺史出連虔遣使喻之,提孤等歸降。熾磐以提孤姦猾,終為邊患,稅其部中戎馬六萬匹。後二歲而提孤等扇動部落,西奔出塞。他
 子率戶五千入居西平。



 先是,姚艾叛降蒙遜,蒙遜率眾迎之。艾叔父俊言於眾曰:「秦王寬仁有雅度,自可安土事之,何為從涼主西遷?」眾咸以為然,相率逐艾,推俊為主,遣使請降。熾磐大悅,征俊為侍中、中書監、征南將軍,封隴西公,邑一千戶。



 使征西孔子討吐谷渾覓地于弱水南,大破之。覓地率眾六千降於熾磐,署為弱水護軍。遣其左衛匹逵,建威梯君等討彭利和于漒川,大破之,利和單騎奔仇池,獲其妻子。徙羌豪三千戶于枹罕,漒川羌三萬餘戶皆安堵如故。



 元熙元年,立其第二子慕末為太子,領撫軍大將軍、都督中外諸軍事,大赦境內,
 改元曰建弘,其臣佐等多所封授。熾磐在位七年而宋氏受禪,以宋元嘉四年死。子慕末嗣偽位,在位四年,為赫連定所殺。



 始國仁以孝武太元十年僭位,至慕末四世,凡四十有六載而滅。



 史臣曰:夫天地閉,大昆生;雲雷屯,群凶作。自晉室遘孽,胡兵肆禍,封域無紀,干戈是務。國仁陰山遺噍,難以義服,伺我阽危,長其陵暴。向使偶欽明之運,遭雄略之主,已當褫魂沙漠,請命槁街,豈暇竊據近郊,經綸王業者也。



 乾歸智不及遠而以力詐自矜。陷呂延之師,姦謀潛斷;俘視羆之眾,威策遐舉。便欲誓湃、隴之餘卒,窺崤、函
 之奧區,秣疲馬而宵征,翦勍敵而朝食。既而控弦嗚鏑,厥志未逞,沮岸崩山,其功已喪。履重氛於外難,幸以計全;貽巨釁於蕭墻,終成凶禍,宜哉!



 熾磐叱吒風雲,見機而動,牢籠俊傑,決勝多奇,故能命將掩澆河之酋,臨戎襲樂都之地,不盈數載,遂隆偽業。覽其遺跡,盜亦有道乎!



 馮跋,字文起,長樂信都人也,小字乞直伐,其先畢萬之後也。萬之子孫有食采馮鄉者,因以氏焉。永嘉之亂,跋祖父和避地上黨。父安,雄武有器量,慕容永時為將軍。
 永滅,跋東徙和龍,家于長谷。幼而懿重少言,寬仁有大度,飲酒一石不亂。三弟皆任俠,不脩行業,惟跋恭慎,勤於家產,父母器之。所居上每有雲氣若樓閣,時咸異之。嘗夜見天門開,神光赫然燭於庭內。及慕容寶僭號,署中衛將軍。



 初,跋弟素弗與從兄萬泥及諸少年游於水濱,有一金龍浮水而下,素弗謂萬泥曰:「頗有見否?」萬泥等皆曰:「無所見也。」乃取龍而示之,咸以為非常之瑞。慕容熙聞而求焉,素弗秘之,熙怒。及即偽位,密欲誅跋兄弟。其後跋又犯熙禁,懼禍,乃與其諸弟逃于山澤。每夜獨行,猛獸常為避路。時賦役繁數,人不堪命,跋兄弟謀
 曰:「熙今昏虐,兼忌吾兄弟,既還首無路,不可坐受誅滅。當及時而起,立公侯之業。事若不成,死其晚乎!」遂與萬泥等二十二人結謀。跋與二弟乘車,使婦人御,潛入龍城,匿于北部司馬孫護之室。遂殺熙,立高雲為主。雲署跋為使持節、侍中、都督中外諸軍事、征北大將軍、開府儀同三司、錄尚書事、武邑公。



 跋宴群僚,忽有血流其左臂,跋惡之。從事中郎王垂因說符命之應,跋戒其勿言。雲為其幸臣離班、桃仁所殺,跋升洪光門以觀變。帳下督張泰、李桑謂跋曰:「此豎勢何所至!請為公斬之。」於是奮劍而下,桑斬班于西門,泰殺仁于庭中。眾推跋為主,
 跋曰:「范陽公素弗才略不恒,志於靖亂,掃清凶桀,皆公勳也。」素弗辭曰:「臣聞父兄之有天下,傳之於子弟,未聞子弟籍父兄之業而先之。今鴻基未建,危甚綴旒,天工無曠,業繫大兄。願上順皇天之命,下副元元之心。」群臣固請,乃許之,於是以太元二十年乃僭稱天王于昌黎,而不徙舊號,即國曰燕,赦其境內,建元曰太平。分遣使者巡行郡國,觀察風俗。追尊祖和為元皇帝,父安為宣皇帝,尊母張氏為太后,立妻孫氏為王后,子永為太子。署弟素弗為侍中、車騎大將軍、錄尚書事,弘為侍中、征東大將軍、尚書右僕射、汲郡公,從兄萬泥為驃騎大將
 軍、幽平二州牧,務銀提為上大將軍、遼東太守,孫護為侍中、尚書令、陽平公,張興為衛將軍、尚書左僕射、永寧公,郭生為鎮東大將軍、領右衛將軍、陳留公,從兄子乳陳為征西大將軍、並青二州牧、上谷公,姚昭為鎮南大將軍、司隸校尉、上黨公,馬弗勤為吏部尚書、廣宗公,王難為侍中、撫軍將軍、潁川公,自餘拜授,文武進位各有差。尋而萬泥抗表請代,跋曰:「猥以不德,謬為群賢所推,思與兄弟同茲休戚。今方難未寧,維城任重,非明德懿親,孰克居也!且折衝禦侮,為國籓屏,雖有他人,不如我弟兄,豈得如所陳也。」於是加開府儀同三司。



 義熙六年,
 跋下書曰:「昔高祖為義帝舉哀,天下歸其仁。吾與高雲義則君臣,恩踰兄弟。其以禮葬雲及其妻子,立雲廟於韭町,置園邑二十家,四時供薦。」



 初,跋之立也,萬泥、乳陳自以親而有大功,謂當入為公輔,跋以二籓任重,因而弗征,並有憾焉。乳陳性粗獷,勇氣過人,密遣告萬泥曰:「乳陳有至謀,顧與叔父圍之。」萬泥遂奔白狼,阻兵以叛。跋遣馮弘與將軍張興將步騎二萬討之。弘遣使喻之曰:「昔者兄弟乘風雲之運,撫翼而起。群公以天命所鍾,人望攸繫,推逼主上光踐寶位。裂土疏爵,當與兄弟共之,奈何欲尋干戈於蕭墻,棄友于而為閼伯!過貴能改,
 善莫大焉。宜舍茲嫌,同獎王室。」萬泥欲降,乳陳按劍怒曰:「大丈夫死生有命,決之于今,何謂降也。」遂剋期出戰。興謂弘曰:「賊明日出戰,今夜必來驚我營,宜命三軍以備不虞。」弘乃密嚴人課草十束,畜火伏兵以待之。是夜,乳陳果遣壯士千餘人來斫營。眾火俱起,伏兵邀擊,俘斬無遺。乳陳等懼而出降,弘皆斬之。



 署素弗為大司馬,改封遼西公,馮弘為驃騎大將軍,改封中山公。



 跋下書曰:「自頃多故,事難相尋,賦役系苦,百姓困窮。宜加寬宥,務從簡易,前朝苛政,皆悉除之。守宰當垂仁惠,無得侵害百姓,蘭臺都官明加澄察。」初,慕容熙之敗也,工人李
 訓竊寶而逃,貲至巨萬,行貨於馬弗勤,弗勤以訓為方略令。既而失志之士書之於闕下碑,馮素弗言之於跋,請免弗勤官,仍推罪之。跋曰:「大臣無忠清之節,貨財公行於朝,雖由吾不明所致,弗勤宜肆諸市朝,以正刑憲。但大業草創,彞倫未敘,弗勤拔自寒微,未有君子之志,其特原之。李訓小人,汙辱朝士,可東市考竟。」於是上下肅然,請賕路絕。



 蝚蠕勇斛律遣使求跋女偽樂浪公主,獻馬三千匹,跋命其群下議之。素弗等議曰:「前代舊事,皆以宗女妻六夷,宜許以妃嬪之女,樂浪公主不宜下降非類。」跋曰:「女生從夫,千里豈遠!朕方崇信殊俗,奈何
 欺之!」乃許焉。遣其游擊秦都率騎二千,送其女婦于蝚蠕。庫莫奚虞出庫真率三千餘落請交市,獻馬千匹,許之,處之於營丘。



 分遣使者巡行郡國,孤老久疾不能自存者,振穀帛有差,孝悌力田閨門和順者,皆褒顯之。昌黎郝越、營丘張買成、周刁、溫建德、何纂以賢良皆擢敘之。遣其太常丞劉軒徙北部人五百戶于長谷,為祖父園邑。以其太子永領大單于,置四輔。跋勵意農桑,勤心政事,乃下書省徭薄賦,墮農者戮之,力田者褒賞,命尚書紀達為之條制。每遣守宰,必親見東堂,問為政事之要,令極言無隱,以觀其志,於是朝野競勸焉。



 先是,河間
 人褚匡言於跋曰:「陛下至德應期,龍飛東夏,舊邦宗族,傾首朝陽,以日為歲。若聽臣往迎,致之不遠。」跋曰:「隔絕殊域,阻回數千,將何可致也?」匡曰:「章武郡臨海,船路甚通,出於遼西臨渝,不為難也。」跋許之,署匡游擊將軍、中書侍郎,厚加資遣。匡尋與跋從兄買、從弟睹自長樂率五千餘戶來奔,署買為衛尉,封城陽伯,睹為太常、高城伯。



 契丹庫莫奚降,署其大人為歸善王。



 跋又下書曰:「今疆宇無虞,百姓寧業,而田畝荒穢,有司不隨時督察,欲今家給人足,不亦難乎!桑柘之益,有生之本。此土少桑,人未見其利,可令百姓人殖桑一百根,柘二十根。」又下
 書曰:「聖人制禮,送終有度。重其衣衾,厚其棺槨,將何用乎?人之亡也,精魂上歸於天,骨肉下歸於地,朝終夕壞,無寒暖之期,衣以錦繡,服以羅紈,寧有知哉!厚於送終,貴而改葬,皆無益亡者,有損於生。是以祖考因舊立廟,皆不改營陵寢。申下境內,自今皆令奉之。」



 魏使耿貳至其國,跋遣其黃門郎常陋迎之於道。跋為不稱臣,怒而不見。及至,跋又遣陋勞之。貳忿而不謝。跋散騎常侍申秀言於跋曰:「陛下接貳以禮,而敢驕蹇若斯,不可容也。,」中給事馮懿以傾佞有幸,又盛稱貳之陵慠以激跋。跋曰:「亦各其志也。匹夫尚不可屈,況一方之主乎!」請幽而
 降之,跋乃留貳不遣。



 是時井竭三日而復。其尚書令孫護里有犬與豕交,護見而惡之,召太史令閔尚筮之。尚曰:「犬豕異類而交,違性失本,其於《洪範》為犬禍,將勃亂失眾,以至敗亡。明公位極冢宰,遐邇具瞻,諸弟並封列侯,貴傾王室,妖見里庭,不為他也。願公戒滿盈之失,脩尚恭儉,則妖怪可消,永享元吉。」護默然不悅。



 昌黎尹孫伯仁、護弟叱支、叱支弟乙拔等俱有才力,以驍勇聞。跋之立也,並冀開府,而跋未之許,由是有怨言。每於朝饗之際,常拔劍擊柱曰:「興建大業,有功力焉,而滯於散將,豈是漢祖河山之義乎!」跋怒,誅之。進護左光祿大夫、開
 府儀同三司、錄尚書事以慰之。護自三弟誅後,常怏怏有不悅之色,跋怒,CG之。尋而遼東太守務銀提自以功在孫護、張興之右,而出為邊郡,抗表有恨言,密謀外叛。跋怒,殺之。



 跋下書曰:「武以平亂,文以經務,寧國濟俗,實所憑焉。自頃喪難,禮崩樂壞,閭閻絕諷誦之音,後行無庠序之教,子衿之歎復興于今,豈所以穆章風化,崇闡斯文!可營建太學,以長樂劉軒、營丘張熾、成周翟崇為博士郎中,簡二千石已下子弟年十五已上教之。」



 跋弟丕,先是因亂投於高句麗,跋迎致之,至龍城,以為左僕射、常山公。



 蝚蠕斛律為其弟大但所逐,盡室奔跋,乃館
 之於遼東郡,待之以客禮。跋納其女為昭儀。時三月不雨,至于夏五月。斛律上書請還塞北,跋曰:「棄國萬里,又無內應。若以彊兵相送,糧運難繼;少也,勢不能固。且千里襲國,古人為難,況數千里乎!」斛律固請曰:「不煩大眾,願給騎三百足矣。得達敕勒國,人必欣而來迎。」乃許之,遣單于前輔萬陵率騎三百送之。陵憚遠役,至黑山,殺斛律而還。



 晉青州刺史申永遣使浮海來聘,跋乃使其中書郎李扶報之。蝚蠕大但遣使獻馬三千匹,羊萬口。



 有赤氣四塞,太史令張穆言於跋曰:「兵氣也。今大魏威制六合,而聘使斷絕。自古未有鄰接境,不通和好。違
 義怒鄰,取亡之道。宜還前使,修和結盟。」跋曰:「吾當思之。」尋而魏軍大至,遣單于右輔古泥率騎候之。去城十五里,遇軍奔還。又遣其將姚昭、皇甫軌等距戰,軌中流矢死。魏以有備,引還。



 跋境地震山崩,洪光門鸛雀折。又地震,右寢壞。跋問閔尚曰:「比年屢有地動之變,卿可明言其故。」尚曰:「地,陰也,主百姓。震有左右,此震皆向右,臣懼百姓將西移。」跋曰:「吾亦甚慮之。」分遣使者巡行郡國,問所疾苦,孤老不能自存者,賜以穀帛有差。



 跋立十一年,至是,元熙元年也,此後事入于宋。至元嘉七年死。弟弘殺跋子翼自立,後為魏所伐,東奔高句麗。居二年,高句麗
 殺之。



 始,跋以孝武太元二十年僭號,至弘二世,凡二十有八載。



 馮素弗,跋之長弟也。慷慨有大志,姿貌魁偉,雄傑不群,任俠放蕩,不修小節,故時人未之奇,惟王齊異焉,曰:「撥亂才也。」惟交結時豪為務,不以產業經懷。弱冠,自詣慕容熙尚書左丞韓業請婚,業怒而距之。復求尚書郎高邵女,邵亦弗許。南宮令成藻,豪俊有高名,素弗造焉,藻命門者勿納。素弗逕入,與藻對坐,旁若無人。談飲連日。藻始奇之,曰:「吾遠求騏驥,不知近在東鄰,何識子之晚也!」當世俠士莫不歸之。及熙僭號,為侍御郎、小帳下督。



 跋之偽業,素弗所建也。及為宰輔,謙虛恭慎,非禮不動,雖廝養之賤,皆與之抗禮。車服屋宇,務於儉約,脩己率下,百僚憚之。初為京尹。及鎮營丘,百姓歌之。嘗謂韓業曰:「君前既不顧,今將自取,何如?」業拜而陳謝。素弗曰:「既往之事,豈復與君計之!」然待業彌厚。好存亡繼絕,申拔舊門,問侍中陽哲曰:「秦、趙勳臣子弟今何在乎?」哲曰:「皆在中州,惟桃豹孫鮮在焉。」素弗召為左常侍,論者歸其有宰衡之度。



 跋之七年死,跋哭之哀慟。比葬,七臨之。



 史臣曰:自五胡縱慝,九域淪胥,帝里神州,遂混之於荒裔,鴻名寶位,咸假之於雜種。嘗謂戎狄凶囂,未窺道德,
 欺天擅命,抑乃其常。而馮跋出自中州,有殊醜類,因鮮卑之昏虐,亦盜名於海隅。然其遷徙之餘,少非雄傑,幸以寬厚為眾所推。初雖砥礪,終罕成德,舊史稱其信惑妖祀,斥黜諫臣,無開馭之才,異經決之士,信矣。速禍致寇,良謂在茲。猶能撫育黎萌,保守疆宇,發號施令,二十餘年,豈天意乎,非人事也!



 贊曰:國仁驍武,乾歸勇悍。矯矯熾磐,臨機能斷。孰謂獯虜,亦懷沈算。文起常才,憑時叛換。咸竊大寶,為我多難。



\end{pinyinscope}