\article{列傳第七十 王彌張昌陳敏王如杜曾杜弢王機祖約蘇峻孫恩盧循譙縱載記序}

\begin{pinyinscope}

 王彌張昌陳敏王如杜曾杜弢王機祖約蘇峻孫恩盧循譙縱載記序



 王彌,東萊人也。家世二千石。祖頎,魏玄菟太守,武帝時,至汝南太守。彌有才幹,博涉書記。少游俠京都,隱者董仲道見而謂之曰:「君豺聲豹視,好亂樂禍,若天下騷擾,不作士大夫矣。」惠帝末,妖賊劉柏根起於東萊之弦縣,彌率家僮從之,柏根以為長史。柏根死,聚徒海渚,為茍純所敗,亡入長廣山為群賊。彌多權略,凡有所掠,必豫
 圖成敗,舉無遺策,弓馬迅捷,膂力過人,青土號為「飛豹」。後引兵入寇青徐,兗州刺史茍晞逆擊,大破之。彌退集亡散,眾復大振,晞與之連戰,不能克。彌進兵寇泰山、魯國、譙、梁、陳、汝南、潁川、襄城諸郡,入許昌,開府庫,取器杖,所在陷沒,多殺守令,有眾數萬,朝廷不能制。



 會天下大亂,進逼洛陽,京邑大震,宮城門晝閉。司徒王衍等率百官距守,彌屯七里澗,王師進擊,大破之。彌謂其黨劉靈曰:「晉兵尚彊,歸無所厝。劉元海昔為質子,我與之周旋京師,深有分契,今稱漢王,將歸之,可乎?」靈然之。乃渡河歸元海。元海聞而大悅,遣其侍中兼御史大夫郊迎,致
 書於彌曰:「以將軍有不世之功,超時之德,故有此迎耳。遲望將軍之至,孤今親行將軍之館,輒拂席洗爵,敬待將軍。」及彌見元海,勸稱尊號,元海謂彌曰:「孤本謂將軍如竇周公耳,今真吾孔明、仲華也。烈祖有云:『吾之有將軍,如魚之有水。』」於是署彌司隸校尉,加侍中、特進,彌固辭。使隨劉曜寇河內,又與石勒攻臨漳。



 永嘉初,寇上黨,圍壺關,東海王越遣淮南內史王曠、安豐太守衛乾等討之,及彌戰于高都、長平間,大敗之,死者十六七。元海進彌征東大將軍,封東萊公。與劉曜、石勒等攻魏郡、汲郡、頓丘,陷五十餘壁,皆調為軍士。又與勒攻鄴,安北將
 軍和郁棄城而走。懷帝遣北中郎將裴憲次白馬討彌,車騎將軍王堪次東燕討勒,平北將軍曹武次大陽討元海。武部將軍彭默為劉聰所敗,見害,眾軍皆退。聰渡黃河,帝遣司隸校尉劉暾、將軍宋抽等距之,皆不能抗。彌、聰以萬騎至京城,焚二學。東海王越距戰於西明門,彌等敗走。彌復以二千騎寇襄城諸縣,河東、平陽、弘農、上黨諸流人之在潁川、襄城、汝南、南陽、河南者數萬家,為舊居人所不禮,皆焚燒城邑,殺二千石長吏以應彌。彌又以二萬人會石勒寇陳郡、潁川,屯陽曜,遣弟璋與石勒共寇徐兗,因破越軍。



 彌後與曜寇襄城,遂逼京師。
 時京邑大饑,人相食,百姓流亡,公卿奔河陰。曜、彌等遂陷宮城,至太極前殿,縱兵大掠。幽帝於端門,逼辱羊皇后,殺皇太子詮,發掘陵墓,焚燒宮廟,城府蕩盡,百官及男女遇害者三萬餘人,遂遷帝于平陽。



 彌之掠也,曜禁之,彌不從。曜斬其牙門王延以徇,彌怒,與曜阻兵相攻,死者千餘人。彌長史張嵩諫曰:「明公與國家共興大事,事業甫耳,便相攻討,何面見主上乎!平洛之功誠在將軍,然劉曜皇族,宜小下之。晉二王平吳之鑒,其則不遠,願明將軍以為慮。縱將軍阻兵不還,其若子弟宗族何!」彌曰:「善,微子,吾不聞此過也。」於是詣曜謝,結分如初。彌
 曰:「下官聞過,乃是張長史之功。」曜謂嵩曰:「君為朱建矣,豈況范生乎!」各賜嵩金百斤。彌謂曜曰:「洛陽天下之中,山河四險之固,城池宮室無假營造,可徙平陽都之。」曜不從,焚燒而去。彌怒曰:「屠各子,豈有帝王之意乎!汝柰天下何!」遂引眾東屯項關。



 初,曜以彌先入洛,不待己,怨之,至是嫌隙遂構。劉暾說彌還據青州,彌然之,乃以左長史曹嶷為鎮東將軍,給兵五千,多齎寶物還鄉里,招誘亡命,且迎其室。彌將徐邈、高梁輒率部曲數千人隨嶷去,彌益衰弱。



 初,石勒惡彌驍勇,常密為之備。彌之破洛陽也,多遺勒美女寶貨以結之。時勒擒茍晞,以為左
 司馬,彌謂勒曰:「公獲茍晞而用之,何其神妙!使晞為公左,彌為公右,天下不足定也!」勒愈忌彌,陰圖之。劉暾又勸彌征曹嶷,藉其眾以誅勒。於是彌使暾詣青州,令曹嶷引兵會己,而詐要勒共向青州。暾至東阿,為勒游騎所獲。勒見彌與嶷書,大怒,乃殺暾。彌未之知,勒伏兵襲彌,殺之,并其眾。



 張昌,本義陽蠻也。少為平氏縣吏,武力過人,每自占卜,言應當富貴。好論攻戰,儕類咸共笑之。及李流寇蜀,昌潛遁半年,聚黨數千人,盜得幢麾,詐言臺遣其募人討
 流。會《壬午詔書》發武勇以赴益土,號曰「壬午兵」。自天下多難,數術者云當有帝王興於江左,及此調發,人咸不樂西征,昌黨因之誑惑,百姓各不肯去。而詔書催遣嚴速,所經之界停留五日者,二千石免。由是郡縣官長皆躬出驅逐,展轉不遠,屯聚而為劫掠。是歲江夏大稔,流人就食者數千口。



 太安二年,昌於安陸縣石巖山屯聚,去郡八十里,諸流人及避戍役者多往從之。昌乃易姓名為李辰。太守弓欽遣軍就討,輒為所破。昌徒眾日多,遂來攻郡。欽出戰,大敗,乃將家南奔沔口。鎮南大將軍、新野王歆遣騎督靳滿討昌於隨郡西,大戰,滿敗走,昌
 得其器杖,據有江夏,即其府庫。造妖言云:「當有聖人出。」山都縣吏丘沈遇於江夏,昌名之為聖人,盛車服出迎之,立為天子,置百官。沈易姓名為劉尼,稱漢後,以昌為相國,昌兄味為車騎將軍,弟放廣武將軍,各領兵。於石巖中作宮殿,又於巖上織竹為鳥形,衣以五彩,聚肉於其傍,眾鳥群集,詐云鳳皇降,又言珠袍、玉璽、鐵券、金鼓自然而至。乃下赦書,建元神鳳,郊祀、服色依漢故事。其有不應其募者,族誅。又流訛言云:「江淮已南當圖反逆,官軍大起,悉誅討之。」群小互相扇動,人情惶懼,江沔間一時猋起,豎牙旗,鳴鼓角,以應昌,旬月之間,眾至三萬,
 皆以絳科頭,手替之以毛。江夏、義陽士庶莫不從之,惟江夏舊姓江安令王傴、秀才呂蕤不從。昌以三公位征之,傴、蕤密將宗室並奔汝南,投豫州刺史劉喬。鄉人期思令李權、常安令吳鳳、孝廉吳暢糾合善土,得五百餘家,追隨傴等,不豫妖逆。



 新野王歆上言:「妖賊張昌、劉尼妄稱神聖,犬羊萬計,絳頭毛面,挑刀走戟,其鋒不可當。請臺敕諸軍,三道救助。」於是劉喬率諸軍據汝南以禦賊,前將軍趙驤領精卒八千據宛,助平南將軍羊伊距守。昌遣其將軍黃林為大都督,率二萬人向豫州,前驅李宮欲掠取汝水居人,喬遣將軍李楊逆擊,大破之。林等
 東攻弋陽,太守梁桓嬰城固守。又遣其將馬武破武昌,害太守,昌自領其眾。西攻宛,破趙驤,害羊伊。進攻襄陽,害新野王歆。昌別率石冰東破江、揚二州,偽置守長。當時五州之境皆畏逼從逆。又遣其將陳貞、陳蘭、張甫等攻長沙、湘東、零陵諸郡。昌雖跨帶五州,樹立牧守,皆桀盜小人而無禁制,但以劫掠為務,人情漸離。



 是歲,詔以寧朔將軍、領南蠻校尉劉弘鎮宛,弘遣司馬陶侃、參軍蒯桓、皮初等率眾討昌於竟陵,劉喬又遣將軍李楊、督護尹奉總兵向江夏。侃等與昌苦戰累日,大破之,納降萬計,昌乃沈竄於下俊山。明年秋,乃擒之,傳首京師,同
 黨並夷三族。



 陳敏,字令通,廬江人也。少有幹能,以郡廉吏補尚書倉部令史。及趙王篡逆,三王起義兵,久屯不散,京師倉廩空虛,敏建議曰:「南方米穀皆積數十年,時將欲腐敗,而不漕運以濟中州,非所以救患周急也。」朝廷從之,以敏為合肥度支,遷廣陵度支。



 張昌之亂,遣其將石冰等趣壽春,都督劉準憂惶計無所出。時敏統大軍在壽春,謂準曰:「此等本不樂遠戍,故逼迫成賊。烏合之眾,其勢易離。敏請合率運兵,公分配眾力,破之必矣。」準乃益敏
 兵擊之,破吳弘、石冰等,敏遂乘勝逐北,戰數十合。時冰眾十倍,敏以少擊眾,每戰皆剋,遂至揚州。迴討徐州賊封雲,雲將張統斬雲降。敏以功為廣陵相。時惠帝幸長安,四方交爭,敏遂有割據江東之志。其父聞之,怒曰:「滅我門者,必此兒也!」父亡,去職。東海王越當西迎大駕,承制起敏為右將軍、假節、前鋒都督,致書於敏曰:



 將軍建謀富國,則有大漕之勳。及遭冰昌之亂,則首率義徒,以寡敵眾。外無彊兵之援,內無運籌之侶,隻身挺立,雄略從橫,擢奇謀於馬首,奪靈計於臨危,金聲振於江外,精光赫于揚楚。攻堅陷險,三十餘戰,師徒無虧,勍敵自滅。
 五州復全,苞茅入貢,豈非將軍之功力哉!



 今羯賊屯結,遊魂河濟,鼠伏雉竄,藏匿陳留,始欲姦盜,終圖不軌。將軍孫吳之術既明,已試之功先著,孤與將軍情分特隆,想割草土之哀,抑難居之思,舍絰執戈,來恤國難。天子遠巡,鑾輿未反,引領東眷,有懷山陵。當憑將軍戮力,王輅有旋。將軍率將所領,承書風發,米布軍資,惟將軍所運。



 時越討豫州刺史劉喬,敏引兵會之,與越俱敗於蕭。敏因中國大亂,遂請東歸,收兵據歷陽。會吳王常侍甘卓自洛至,教卓假稱皇太弟命,拜敏為揚州刺史,并假江東首望顧榮等四十餘人為將軍、郡守,榮並偽從之。
 敏為息娶卓女,遂相為表裏。揚州刺史劉機、丹陽太守王廣等皆棄官奔走。敏弟昶知顧榮等有貳心,勸敏殺之,敏不從。昶將精兵數萬據烏江,弟恢率錢端等南寇江州,刺史應邈奔走,弟斌東略諸郡,遂據有吳越之地。敏命寮佐以己為都督江東軍事、大司馬、楚公,封十郡,加九錫,列上尚書,稱自江入河,奉迎鑾駕。



 東海王軍諮祭酒華譚聞敏自相署置,而顧榮等並江東首望,悉受敏官爵,乃遺榮等書曰:



 石冰之亂,朝廷錄敏微功,故加越次之禮,授以上將之任,庶有韓盧一噬之效。而本性凶狡,素無識達,貪榮干運,逆天而動,阻兵作威,盜據吳
 會,內用凶弟,外委軍吏,上負朝廷寵授之榮,下孤宰輔過禮之惠。天道伐惡,人神所不祐。雖阻長江,命危朝露。忠節令圖,君子高行,屈節附逆,義士所恥。王蜀匹夫,志不可屈;於期慕義,隕首燕庭。況吳會仁人並受國寵,或剖符名郡,或列為近臣,而便辱身姦人之朝,降節逆叛之黨,稽顙屈膝,不亦羞乎!昔龔勝絕粒,不食莽朝;魯連赴海,恥為秦臣。君子義行,同符千載,遙度雅量,豈獨是安!



 昔吳之武烈,稱美一代,雖奮奇宛葉,亦受折襄陽。討逆雄氣,志存中夏,臨江發怒,命訖丹徒。賴先主承運,雄謀天挺,尚內倚慈母仁明之教,外杖子布廷爭之忠,又
 有諸葛、顧、步、張、朱、陸、全之族,故能鞭笞百越,稱制南州。然兵家之興,不出三世,運未盈百,歸命入臣。今以陳敏倉部令史,七第頑冗,六品下才,欲躡桓王之高蹤,蹈大皇之絕軌,遠度諸賢,猶當未許也。諸君垂頭,不能建翟義之謀;而顧生俯眉,已受羈絆之辱。皇輿東軒,行即紫館,百僚垂纓,雲翔鳳闕,廟勝之謨,潛運帷幄。然後發荊州武旅,順流東下,徐州銳鋒,南據堂邑;征東勁卒,耀威歷陽;飛橋越橫江之津,泛舟涉瓜步之渚;威震丹陽,擒寇建鄴,而諸賢何顏見中州之士邪!



 小寇隔津,音符道闊,引領南望,情存舊懷。忠義之人,何世蔑有!夫危而不
 能安,亡而不能存,將何貴乎!永長宿德,情所素重;彥先垂髮,分著金石;公胄早交,恩紀特隆;令伯義聲,親好密結。上欲與諸賢效冀紫宸,建功帝籍。如其不爾,亦可泛舟河渭,擊楫清歌。何為辱身小寇之手,以蹈逆亂之禍乎!昔為同志,今已殊域;往為一體,今成異身。瞻江長嘆,非子誰思!願圖良策,以存嘉謀也。



 敏凡才無遠略,一旦據有江東,刑政無章,不為英俊所服,且子弟凶暴,所在為患。周、顧榮之徒常懼禍敗,又得譚書,皆有慚色。、榮遣使密報征東大將軍劉準遣兵臨江,己為內應。準遣揚州刺史劉機、寧遠將軍衡彥等出歷陽,敏使弟昶
 及將軍錢廣次烏江以距之,又遣弟閎為歷陽太守,戍牛渚。錢廣家在長城,鄉人也,潛使圖昶。廣遣其屬何康、錢象投募送白事於昶,昶俯頭視書,康揮刀斬之,稱州下已殺敏,敢有動者誅三族,吹角為內應。廣先勒兵在朱雀橋,陳兵水南、、榮又說甘卓,卓遂背敏。敏率萬餘人將與卓戰,未獲濟,榮以白羽扇麾之,敏眾潰散。敏單騎東奔至江乘,為義兵所斬,母及妻子皆伏誅,於是會稽諸郡並殺敏諸弟無遺焉。



 王如,京兆新豐人也。初為州武吏,遇亂流移至宛。時諸
 流人有詔並遣還鄉里,如以關中荒殘,不願歸。征南將軍山簡、南中郎將杜蕤各遣兵送之,而促期令發。如遂潛結諸無賴少年,夜襲二軍,破之。杜蕤悉眾擊如,戰於涅陽,蕤軍大敗。山簡不能禦,移屯夏口,如又破襄城。於是南安龐實、馮翊嚴嶷、長安侯脫等各帥其黨攻諸城鎮,多殺令長以應之。未幾,眾至四五萬,自號大將軍,領司、雍二州牧。



 如懼石勒之攻己也,乃厚賄於勒,結為兄弟,勒亦假其彊而納之。時侯脫據宛,與如不協,如說勒曰:「侯脫雖名漢臣,其實漢賊。如常恐其來襲,兄宜備之。」勒素怒脫貳己,憚如脣齒,故不攻之。及聞如言,甚悅,遂
 夜令三軍蓐食待命,雞鳴而駕,後出者斬,晨壓宛門攻之,旬有二日而剋之,勒遂斬脫。如於是大掠沔漢,進逼襄陽。征南山簡使將趙同帥師擊之,經年不能剋,智力並屈,遂嬰城自守。王澄帥軍赴京都,如邀擊破之。



 如連年種穀皆化為莠,軍中大饑,其黨互相攻劫,官軍進討,各相率來降。如計無所出,歸于王敦。敦從弟棱愛如驍武,請敦配己麾下。敦曰:「此輩虓險難蓄,汝性忌急,不能容養,更成禍端。」棱固請,與之。棱置諸左右,甚加寵遇。如數與敦諸將角射,屢鬥爭為過失,棱果不容而杖之,如甚以為恥。初,敦有不臣之迹,棱每諫之,敦常怒其異己。
 及敦聞如為棱所辱,密使人激怒之,勸令殺棱。如詣棱,因閑宴,請劍舞為歡,棱從之。如於是舞刀為戲,漸漸來前。棱惡而呵之不止,叱左右使牽去,如直前害棱。敦聞而陽驚,亦捕如誅之。



 杜曾,新野人,南中郎將蕤之從祖弟也。少驍勇絕人,能被甲游於水中。始為新野王歆鎮南參軍,歷華容令,至南蠻司馬。凡有戰陣,勇冠三軍。會永嘉之亂,荊州荒梗,故牙門將胡亢聚眾於竟陵,自號楚公,假曾竟陵太守。亢後與其黨自相猜貳,誅其驍將數十人,曾心不自安,
 潛謀圖之,乃卑身屈節以事於亢,亢弗之覺,甚信任之。會荊州賊王沖自號荊州刺史,部眾亦盛,屢遣兵抄亢所統,亢患之,問計於曾,曾勸令擊之,亢以為然。曾白亢取帳下刀戟付工磨之,因潛引王沖之兵。亢遣精騎出距沖,城中空虛,曾因斬亢而并其眾,自號南中郎將、領竟陵太守。曾求南郡太守劉務女不得,盡滅其家。會愍帝遣第五猗為安南將軍、荊州刺史,曾迎猗於襄陽,為兄子娶猗女,遂分據沔漢。



 時陶侃新破杜弢,乘勝擊曾,有輕曾之色。侃司馬魯恬言於侃曰:「古人爭戰,先料其將,今使君諸將無及曾者,未易可逼也。」侃不從,進軍圍
 之於石城。時曾軍多騎,而侃兵無馬,曾密開門,突侃陣,出其後,反擊其背,侃師遂敗,投水死者數百人。曾將趨順陽,下馬拜侃,告辭而去。既而致箋於平南將軍荀崧,求討丹水賊以自效,崧納之。侃遺崧書曰:「杜曾凶狡,所將之卒皆豺狼也,可謂䲭梟食母之物。此人不死,州土未寧,足下當識吾言。」崧以宛中兵少,藉曾為外援,不從侃言。曾復率流亡二千餘人圍襄陽,數日不下而還。



 及王廙為荊州刺史,曾距之,廙使將未軌、趙誘擊曾,皆為曾所殺。王敦遣周訪討之,屢戰不能克,訪潛遣人緣山開道,出曾不意以襲之,曾眾潰,其將馬俊、蘇溫等執曾
 詣訪降。訪欲生致武昌,而朱軌息昌、趙誘息胤皆乞曾以復冤,於是斬曾,而昌、胤臠其肉而啖之。



 杜弢,字景文,蜀郡成都人也。祖植,有名蜀土,武帝時為符節令。父,略陽護軍。弢初以才學著稱,州舉秀才。遭李庠之亂,避地南平,太守應詹愛其才而禮之。後為醴陵令。時巴蜀流人汝班、蹇碩等數萬家,布在荊湘間,而為舊百姓之所侵苦,並懷怨恨。會屬賊李驤殺縣令,屯聚樂鄉,眾數百人,弢與應詹擊驤,破之。蜀人杜疇、蹇撫等復擾湘州,參軍馮素與汝班不協,言於刺史荀眺曰:「
 流人皆欲反。」眺以為然,欲盡誅流人。班等懼死,聚眾以應疇。時弢在湘中,賊眾共推弢為主,弢自稱梁益二州牧、平難將軍、湘州刺史,攻破郡縣,眺委城走廣州。廣州刺史郭訥遣始興太守嚴佐率眾攻弢,弢逆擊破之。荊州刺史王澄復遣王機擊弢,敗於巴陵。弢遂縱兵肆暴,偽降於山簡,簡以為廣漢太守。



 眺之走也,州人推安成太守郭察領州事,因率眾討弢,反為所敗,察死之。弢遂南破零陵,東侵武昌,害長沙太守崔敷、宜都太守杜鑒、邵陵太守鄭融等。元帝命征南將軍王敦、荊州刺史陶侃等討之,前後數十戰,弢將士多物故,於是請降。帝不
 許。弢乃遺應詹書曰:



 天步艱難,始自吾州;州黨流移,在於荊土。其所遇值,蔑之如遺,頓伏死亡者略復過平,備嘗荼毒,足下之所鑒也。客主難久,嫌隙易構,不謂樂鄉起變出於不意,時與足下思散疑結,求擒其黨帥,惟患算不經遠,力不陷堅耳。及在湘中,懼死求生,遂相結聚,欲守善自衛,天下小定,然後輸誠盟府。尋山公鎮夏口,即具陳之。此公鑒開塞之會,察窮通之運,納吾於眾疑之中,非高識玄睹,孰能若此!西州人士得沐浴於清流,豈惟滌蕩瑕穢,乃骨肉之施。此公薨逝,斯事中廢,賢愚痛毒,竊心自悼。欲遣滕永文、張休豫詣大府備列起事
 以來本末,但恐貪功殉名之徒將讒間於聖主之聽,戮吾使於市朝以彰叛逆之罪,故未敢遣之。而甘陶卒至,水陸十萬,旌旗曜於山澤,舟艦有盈於三江,威則威矣,然吾眾竊未以為懼。晉文伐原,以全信為本,故能使諸侯歸之。陶侃宣赦書而繼之以進討,豈所以崇奉明詔,示軌憲於四海!逼向義之夫以為叛逆之虜,踧思善之眾以極不赦之責,非不戰而屈人之算也。驅略烏合,欲與必死者求一戰,未見爭衡之機權也。吾之赤心,貫於神明,西州人士,卿粗悉之耳。寧當令抱枉於時,不證於大府邪!



 昔虞卿不榮大國之相,與魏齊同其安危;司馬遷
 明言於李陵,雖刑殘而無慨。足下抗威千里,聲播汶衡,進宜為國思靜難之略,退與舊交措枉直之正,不亦綽然有餘裕乎!望卿騰吾箋令,時達盟府,遣大使光臨,使吾得披露肝膽,沒身何恨哉!伏想盟府必結紐於紀綱,為一匡於聖世,使吾廁列義徒,負戈前驅,迎皇輿於閶闔,掃長蛇於荒裔,雖死之日,猶生之年也。若然,先清方夏,卻定中原,吾得一所之糧,使水斥流西歸,夷李雄之逋寇,脩《禹貢》之舊獻,展微勞以補往愆,復州邦以謝鄰國,亦其志也,惟所裁處耳。



 吾遠州寒士,與足下出處殊倫,誠不足感神交而濟其傾危。但顯吾忠誠,則汶嶽荷忠
 順之恕,衡湘無伐叛之虞,隆足下宏納之望,拯吾徒陷溺之艱,焉可金玉其音哉!然顒顒十餘萬口,亦勞瘁於警備,思放逸於南畝矣。衡獄、江、湘列吾左右,若往言有貳,血誠不亮,益梁受殃,不惟鄙門而已。



 詹甚哀之,乃啟呈弢書,并上言曰:「弢益州秀才,素有清望,文理既優,幹事兼美。往因使流寓,居詹郡界,其貞心堅白:詹所委究。李驤為變樂鄉,劫略良善,弢時出家財,招募忠勇,登壇歃血,義誠慷慨。會驤攻燒南平,弢遂東下巴漢,與湘中鄉人相遇,推其素望,遂相憑結。論弢本情,非首作亂階者也。然破湘川,實弢之罪,亦由兵交其間,遂使滋蔓。按
 弢今書,血誠亦至矣。昔朱鮪自疑於洛陽,光武指河水以明心,鮪感義歸誠,終展力報施,受封侯之寵,由恕過以錄功也。詹竊謂今者當圮運之會,思弘遠猷,故齊赦射鉤之誅,晉貰斬袪之戮,用能濟冀戴之高勳,隆一匡之美譽,況弢等素無斯愆而稽顙投命邪!以為可遣大使宣揚聖旨,雲澤沾之於上,百姓沐浴於下,則上下交泰,江左無風塵之虞矣。」帝乃使前南海太守王運受弢降,宣詔書大赦,凡諸反逆一皆除之,加韜巴東監軍。



 弢受命後,諸將殉功者攻擊之不已,弢不勝憤怒,遂殺運而使其將王真領精卒三千為奇兵,出江南,向武陵,斷
 官軍運路。陶侃使伏波將軍鄭攀邀擊,大破之,真步走湘城。於是侃等諸軍齊進,真遂降侃,眾黨散潰。弢乃逃遁,不知所在。



 王機,字令明,長沙人也。父毅,廣州刺史,甚得南越之情。機美姿儀,人周儻有度量。陳恢之亂,機年十七,率眾擊破之。嘗慕王澄為人,澄亦雅知之,以為己亞,遂與友善,內綜心膂,外為牙爪。尋用為成都內史。機終日醉酒,不存政事,由是百姓怨之,人情騷動。



 會澄遇害,機懼禍及,又屬杜弢所在發墓,而獨為機守塚,機益自疑。就王敦求
 廣州,敦不許。會廣州人背刺史郭納,迎機為刺史,機遂將奴客門生千餘人入廣州,州部將溫邵率眾迎機。郭遣參軍葛幽追之,及於廬陵,機叱幽曰:「何以敢來?欲取死邪?」幽不敢逼而歸。郭訥聞邵之納機也,乃遣兵擊邵,反為所破。訥又遣機父兄時吏距之,咸倒戈迎機,訥眾皆散,乃握節而避機。機遂入城就訥求節,訥歎曰:「昔蘇武不失其節,前史以為美談。此節天朝所假,義不相與,自可遣兵來取之。」機慚而止。



 機自以篡州,懼為王敦所討,乃更來交州。時杜弢餘黨杜弘奔臨賀,送金數千兩與機,求討桂林賊以自效。機為列上,朝廷許之。王敦以
 機難制,又欲因機討梁碩,故以降杜弘之勳轉為交州刺史。碩聞而遣子侯侯機於鬱林,機怒其迎遲,責云:「須至州當相收拷。」碩子馳使報碩,碩曰:「王郎已壞廣州,何可復來破交州也!」乃禁州人不許迎之。府司馬杜贊以碩不迎機,率兵討碩,為碩所敗。碩恐諸僑人為機,於是悉殺其良者,乃自領交址太守。機既為碩所距,遂住鬱林。時杜弘大破桂林賊還,遇機於道,機勸弘取交州。弘素有意,乃執機節曰:「當相與迭持,何可獨捉!」機遂以節與之。於是機與弘及溫邵、劉沈等並反。



 尋而陶侃為廣州,到始興,州人皆諫不可輕進,侃不聽。及至州,諸郡縣
 皆已迎機矣。侃先討溫邵、劉沈,皆殺之。機遣牙門屈藍還州,詐言增糧,密招誘所部,欲以距侃。侃即收藍斬之,遣督護許高討機走之,病死于道。高掘出其尸斬首,并殺其二子焉。



 機兄矩,字令式。美姿容,每出游,觀者盈路。初為南平太守,豫討陳恢有功,遷廣州刺史。將赴職,忽見一人持奏謁矩,自云京兆杜靈之。矩問之,答稱:「天上京兆,被使召君為主簿。」矩意甚惡之。至州月餘卒。



 祖約,字士少,豫州刺史逖之弟也。初以孝廉為成皋令,
 與逖甚相友愛。永嘉末,隨逖過江。元帝稱制,引為掾屬,與陳留阮孚齊名。後轉從事中郎,典選舉。



 約妻無男而性妒,約亦不敢違忤。嘗夜寢於外,忽為人所傷,疑其妻所為,約求去職,帝不聽,約便從右司馬營東門私出。司直劉隗劾之曰:「約幸荷殊寵,顯位選曹,銓衡人物,眾所具瞻。當敬以直內,義以方外,杜漸防萌,式遏寇害。而乃變起蕭牆,患生婢妾,身被刑傷,虧其膚髮。群小噂沓,囂聲遠被,塵穢清化,垢累明時。天恩含垢,猶復慰喻,而約違命輕出,既無明智以保其身,又孤恩廢命,宜加貶黜,以塞眾謗。」帝不之罪。隗重加執據,終不許。



 及逖有功於譙
 沛,約漸見任遇。逖卒,自侍中代逖為平西將軍、豫州刺史,領逖之眾。約異母兄光祿大夫納密言於帝曰:「約內懷陵上之心,抑而使之可也。今顯侍左右,假其權勢,將為亂階矣。」帝不納。時人亦謂納與約異生,忌其寵貴,故有此言。而約竟無綏馭之才,不為士卒所附。



 及王敦舉兵,約歸衛京都,率眾次壽陽,逐敦所署淮南太守任台,以功封五等侯,進號鎮西將軍,使屯壽陽,為北境籓扞。自以名輩不後郗、卞,而不豫明帝顧命,又望開府,及諸所表請多不見許,遂懷怨望。石聰嘗以眾逼之,約屢表請救,而官軍不至。聰既退,朝議又欲作塗塘以遏胡寇,
 約謂為棄己,彌懷憤恚。先是,太后使蔡謨勞之,約見謨,瞋目攘袂,非毀朝政。及蘇峻舉兵,遂推崇約而罪執政,約聞而大喜。從子智及衍並傾險好亂,又贊成其事,於是命逖子沛內史渙,女婿淮南太守許柳以兵會峻。逖妻,柳之姊也,固諫不從。及峻剋京都,矯詔以約為侍中、太尉、尚書令。穎川人陳光率其屬攻之,約左右閻禿貌類約,光謂為約而擒之,約踰垣護免。光奔於石勒,而約之諸將復陰結於勒,請為內應。勒遣石聰來攻之,約眾潰,奔歷陽。遣兄子渙攻桓宣于皖城,會毛寶援宣,擊渙,敗之。趙胤復遣將軍甘苗從三焦上歷陽,約懼而夜遁,
 其將牽騰率眾出降。



 約以左右數百人奔於石勒,勒薄其為人,不見者久之。勒將程遐說勒曰:「天下粗定,當顯明逆順,此漢高祖所以斬丁公也。今忠於事君者莫不顯擢,背叛不臣者無不夷戮,此天下所以歸伏大王也。祖約猶存,臣切惑之。且約大引賓客,又占奪鄉里先人田地,地主多怨。」於是勒乃詐約曰:「祖侯遠來,未得喜歡,可集子弟一時俱會。」至日,勒辭之以疾,令遐請約及其宗室。約知禍及,大飲致醉。既至于市,抱其外孫而泣。遂殺之,并其親屬中外百餘人悉滅之,婦女伎妾班賜諸胡。



 初,逖有胡奴曰王安,待之甚厚。及在雍丘,告之曰:「石
 勒是汝種類,吾亦不在爾一人。」乃厚資遣之,遂為勒將。祖氏之誅也,安多將從人於市觀省,潛取逖庶子道重,藏之為沙門,時年十歲。石氏滅後來歸。



 蘇峻,字子高,長廣掖人也。父模,安樂相。峻少為書生,有才學,仕郡主簿。年十八,舉孝廉。永嘉之亂,百姓流亡,所在屯聚,峻糾合得數千家,結壘于本縣。于時豪傑所在屯聚,而峻最彊。遣長史徐瑋宣檄諸屯,示以王化,又收枯骨而葬之,遠近感其恩義,推峻為主。遂射獵於海邊青山中。元帝聞之,假峻安集將軍。時曹嶷領青州刺史,
 表峻為掖令,峻辭疾不受。嶷惡其得眾,恐必為患,將討之。峻懼,率其所部數百家汎海南渡。既到廣陵,朝廷嘉其遠至,轉鷹揚將軍。會周堅反於彭城,峻助討之,有功,除淮陵內史,遷蘭陵相。



 王敦作逆,詔峻討敦。卜之不吉,遲迴不進。及王師敗績,峻退保盱眙。淮陵故吏徐深、艾毅重請峻為內史,詔聽之,加奮威將軍。太寧初,更除臨淮內史。王敦復肆逆,尚書令郗鑒議召峻及劉遐援京都,敦遣峻兄說峻曰:「富貴可坐取,何為自來送死?」峻不從,遂率眾赴京師,頓于司徒故府。道遠行速,軍人疲困。沈充、錢鳳謀曰:「北軍新到,未堪攻戰,擊之必剋。若復猶
 豫,後難犯也」賊於其夜度竹格渚,拔柵將戰,峻率其將韓晃於南塘橫截,大破之。又隨庾亮追破沈充。進使持節、冠軍將軍、歷陽內史,加散騎常侍,封邵陵公,食邑一千八百戶。



 峻本以單家聚眾於擾攘之際,歸順之後,志在立功,既有功於國,威望漸著。至是有銳卒萬人,器械甚精,朝廷以江外寄之。而峻頗懷驕溢,自負其眾,潛有異志,撫納亡命,得罪之家有逃死者,峻輒蔽匿之。眾力日多,皆仰食縣官,運漕者相屬,稍有不如意,便肆忿言。



 時明帝初崩,委政宰輔,護軍庾亮欲征之。峻聞將征,遣司馬何仍詣亮曰:「討賊外任,遠近從命,至於內輔,實非
 所堪。」不從,遂下優詔徵峻為大司農,加散騎常侍,位特進,以弟逸代領部曲。峻素疑帝欲害己,表曰:「昔明皇帝親執臣手,使臣北討胡寇。今中原未靖,無用家為,乞補青州界一荒郡,以展鷹犬之用。」復不許。峻嚴裝將赴召,而猶豫未決,參軍任讓謂峻曰:「將軍求處荒郡而不見許,事勢如此,恐無生路,不如勒兵自守。」峻從之,遂不應命。朝廷遣使諷諭之,峻曰:「臺下云我欲反,豈得活邪!我寧山頭望廷尉,不能廷尉望山頭。往者國危累卵,非我不濟,狡兔既死,獵犬理自應烹,但當死報造謀者耳。」於是遣參軍徐會結祖約,謀為亂,而以討亮為名。約遣祖
 渙、許柳率眾助峻,峻遣將韓晃、張健等襲姑孰,進逼慈湖,殺于湖令陶馥及振威將軍司馬流。峻自率渙、柳眾萬人,乘風濟自橫江,次于陵口,與王師戰,頻捷,遂據蔣陵覆舟山,率眾因風放火,臺省及諸營寺署一時蕩盡。遂陷宮城,縱兵大掠,侵逼六宮,窮凶極暴,殘酷無道。驅役百官,光祿勳王彬等皆被捶撻,逼令擔負登蔣山。裸剝士女,皆以壞席苫草自鄣,無草者坐地以土自覆,哀號之聲震動內外。時官有布二十萬匹,金銀五千斤,錢億萬,絹數萬匹,他物稱是,峻盡費之。矯詔大赦,惟庾亮兄弟不在原例。自為驃騎領軍將軍、錄尚書事,許柳丹
 陽尹,加前將軍馬雄左衛將軍,祖渙驍騎將軍,復弋陽王羕為西陽王、太宰、錄尚書事,羕息播亦復本官。於是改易官司,置其親黨,朝廷政事一皆由之。又遣韓晃入義興,張健、管商、弘徽等入晉陵。



 時溫嶠、陶侃已唱義於武昌,峻聞兵起,用參軍賈寧計,還據石頭,更分兵距諸義軍,所過無不殘滅。嶠等將至,峻遂遷天子於石頭,逼迫居人,盡聚之後苑,使懷德令匡術守苑城。嶠等既到,乃築壘於白石,峻率眾攻之,幾至陷沒。東西抄掠,多所擒虜,兵威日盛,戰無不剋,由是義眾沮衄,人懷異計。朝士之奔義軍者,皆云:「峻狡黠有智力,其徒黨驍勇,所向
 無敵。惟當以天討有罪,誅滅不久;若以人事言之,未易除也。」溫嶠怒曰:「諸君怯懦,乃是譽賊。」及後累戰不捷,嶠亦深憚之。管商等進攻吳郡,焚吳縣、海監、嘉興,敗諸義軍。韓晃又攻宣城,害太守桓彞。商等又焚餘杭,而大敗於武康,退還義興。嶠與趙胤率步兵萬人,從白石南上,欲以臨之。峻與匡孝將八千人逆戰,峻遣子碩與孝以數十騎先薄趙胤,敗之。峻望見胤走,曰:「孝能破賊,我更不如乎!」因舍其眾,與數騎北下突陣,不得入,將迴趨白木陂,牙門彭世、李千等投之以矛,墜馬,斬首臠割之,焚其骨,三軍皆稱萬歲。峻司馬任讓等共立峻弟逸為主。
 求峻尸不獲,碩乃發庾亮父母墓,剖棺焚尸。逸閉城自守。韓晃聞峻死,引兵赴石頭。管商及弘徽進攻庱亭壘,督護李閎及輕車長史滕含擊破之,斬首千級。商率眾走延陵,李閎與庱亭諸軍追之,斬獲數千級。商詣庾亮降,匡術舉苑城降。韓晃與蘇逸等并力攻術,不能陷。溫嶠等選精銳將攻賊營,碩率驍勇數百渡淮而戰,於陣斬碩。晃等震懼,以其眾奔張健於曲阿,門阨不得出,更相蹈藉,死者萬數。逸為李湯所執,斬于車騎府。



 管商之降也,餘眾並歸張健。健又疑弘徽等不與己同,盡殺之,更以舟軍自延陵向長塘,小大二萬餘口,金銀寶物不
 可勝數。揚烈將軍王允之與吳興諸軍擊健,大破之,獲男女萬餘口。健復與馬雄、韓晃等輕軍俱走,閎率銳兵追之,及於巖山,攻之甚急。健等不敢下山,惟晃獨出,帶兩步靫箭,卻據胡床,彎弓射之,傷殺甚眾。箭盡,乃斬之。健等遂降,並梟其首。



 孫恩,字靈秀,瑯邪人,孫秀之族也。世奉五斗米道。恩叔父泰,字敬遠,師事錢唐杜子恭。而子恭有祕術,嘗就人借瓜刀,其主求之,子恭曰:「當即相還耳。」既而刀主行至嘉興,有魚躍入船中,破魚得瓜刀。其為神效往往如此。
 子恭死,泰傳其術。然浮狡有小才,誑誘百姓,愚者敬之如神,皆竭財產,進子女,以祈福慶。王珣言於會稽王道子,流之於廣州。廣州刺史王懷之以泰行鬱林太守,南越亦歸之。太子少傅王雅先與泰善,言於孝武帝,以泰知養性之方,因召還。道子以為徐州主簿,猶以道術眩惑士庶。稍遷輔國將軍、新安太守。王恭之役,泰私合義兵,得數千人,為國討恭。黃門郎孔道、鄱陽太守桓放之、驃騎諮議周勰等皆敬事之,會稽世子元顯亦數詣泰求其秘術。泰見天下兵起,以為晉祚將終,乃扇動百姓,私集徒眾,三吳士庶多從之。於時朝士皆懼泰為
 亂,以其與元顯交厚,咸莫敢言。會稽內史謝輶發其謀,道子誅之。恩逃于海。眾聞泰死,惑之,皆謂蟬蛻登仙,故就海中資給。恩聚合亡命得百餘人,志欲復仇。



 及元顯縱暴吳會,百姓不安,恩因其騷動,自海攻上虞,殺縣令,因襲會稽,害內史王凝之,有眾數萬。於是會稽謝金咸、吳郡陸瑰、吳興丘尪、義興許允之、臨海周胄、永嘉張永及東陽、新安等凡八郡,一時俱起,殺長史以應之,旬日之中,眾數十萬。於是吳興太守謝邈,永嘉太守謝逸,嘉興公顧胤,南康公謝明慧,黃門郎謝沖、張琨,中書郎孔道,太子洗馬孔福,烏程令夏侯愔等皆遇害。吳國內史桓
 謙,義興太守魏傿,臨海太守、新蔡王崇等並出奔。於是恩據會稽,自號征東將軍,號其黨曰「長生人」,宣語令誅殺異己,有不同者戮及嬰孩,由是死者十七八。畿內諸縣處處蜂起,朝廷震懼,內外戒嚴。遣衛將軍謝琰、鎮北將軍劉牢之討之,並轉鬥而前。吳會承平日久,人不習戰,又無器械,故所在多被破亡。諸賊皆燒倉廩,焚邑屋,刊木堙井,虜掠財貨,相率聚于會稽。其婦女有嬰累不能去者,囊簏盛嬰兒投於水,而告之曰:「賀汝先登仙堂,我尋後就汝。」



 初,恩聞八郡響應,告其屬曰:「天下無復事矣,當與諸君朝服而至建康。」既聞牢之臨江,復曰:「我割
 浙江,不失作句踐也。」尋知牢之已濟江,乃曰:「孤不羞走矣。」乃虜男女二十餘萬口,一時逃入海。懼官軍之躡,乃緣道多棄寶物子女。時東土殷實,莫不粲麗盈目,牢之等遽於收斂,故恩復得逃海。朝廷以謝琰為會稽,率徐州文武戍海浦。



 隆安四年,恩復入餘姚,破上虞,進至刑浦。琰遣參軍劉宣之距破之,恩退縮。少日,復寇刑浦,害謝琰。朝廷大震,遣冠軍將軍桓不才、輔國將軍孫無終、寧朔將軍高雅之擊之,恩復還于海。於是復遣牢之東屯會稽,吳國內史袁山松築扈瀆壘,緣海備恩。明年,恩復入浹口,雅之敗績。牢之進擊,恩復還于海。轉寇扈瀆,
 害袁山松,仍浮海向京口。牢之率眾西擊,未達,而恩已至,劉裕乃總兵緣海距之。及戰,恩眾大敗,狼狽赴船。尋又集眾,欲向京都,朝廷駭懼,陳兵以待之。恩至新州,不敢進而退,北寇廣陵,陷之,乃浮海而北。劉裕與劉敬宣並軍躡之於郁洲,累戰,恩復大敗,由是漸衰弱,復沿海還南。裕亦尋海要截,復大破恩於扈瀆,恩遂遠迸海中。



 及桓玄用事,恩復寇臨海,臨海太守辛景討破之。恩窮戚,乃赴海自沈,妖黨及妓妾謂之水仙,投水從死者百數。餘眾復推恩妹夫盧循為主。自恩初入海,所虜男女之口,其後戰死及自溺并流離被傳賣者,至恩死時裁
 數千人存,而恩攻沒謝琰、袁山松,陷廣陵,前後數十戰,亦殺百姓數萬人。



 盧循,字于先,小名元龍,司空從事中郎諶之曾孫也。雙眸冏徹,瞳子四轉,善草隸弈棋之藝。沙門慧遠有鑒裁,見而謂之曰:「君雖體涉風素,而志存不軌。」循娶孫恩妹。及恩作亂,與循通謀。恩性酷忍,循每諫止之,人士多賴以濟免。恩亡,餘眾推循為主。元興二年正月,寇東陽,八月,攻永嘉。劉裕討循至晉安,循窘急,泛海到番禺,寇廣州,逐刺史吳隱之,自攝州事,號平南將軍,遣使獻貢。時
 朝廷新誅桓氏,中外多虞,乃權假循征虜將軍、廣州刺史、平越中郎將。



 義熙中,劉裕伐慕容超,循所署始興太守徐道覆,循之姊夫也,使人勸循乘虛而出,循不從。道覆乃至番禺,說循曰:「朝廷恒以君為腹心之疾,劉公未有旋日,不乘此機而保一日之安,若平齊之後,劉公自率眾至豫章,遣銳師過嶺,雖復君之神武,必不能當也。今日之機,萬不可失。既剋都邑,劉裕雖還,無能為也。君若不同,便當率始興之眾直指尋陽。」循甚不樂此舉,無以奪其計,乃從之。



 初,道覆密欲裝舟艦,乃使人伐船材於南康山,偽云將下都貨之。後稱力少不能得致,即於
 郡賤賣之,價減數倍,居人貪賤,賣衣物而市之。贛石水急,出船甚難,皆儲之。如是者數四,故船版大積,而百姓弗之疑。及道覆舉兵,案賣券而取之,無得隱匿者,乃并力裝之,旬日而辦。遂舉眾寇南康、廬陵、豫章諸郡,守相皆委任奔走。鎮南將軍何無忌率眾距之,兵敗被害。



 循遣道覆寇江陵,未至,為官軍所敗,馳走告循曰:「請并力攻京都,若剋之,江陵非所憂也。」乃連旗而下,戎卒十萬,舳艫千計,敗衛將軍劉毅於桑落洲,逕至江寧。道覆素有膽決,知劉裕已還,欲乾沒一戰,請於新亭至白石,焚舟而上,數道攻之。循多謀少決,欲以萬全之計,固不聽。
 道覆以循無斷,乃歎曰:「我終為盧公所誤,事必無成。使我得為英雄驅馳,天下不足定也!」裕懼其侵軼,乃柵石頭,斷柤浦,以距之。循攻柵不利,船艦為暴風所傾,人有死者。列陣南岸,戰又敗績。乃進攻京口,寇掠諸縣,無所得。循謂道覆曰:「師老矣!弗能復振。可據尋陽,并力取荊州,徐更與都下爭衡,猶可以濟。」因自蔡洲南走,復據尋陽。裕先遣群率追討,自統大眾繼進,又敗循於雷池。循欲遁還豫章,乃悉力柵斷左里。裕命眾攻柵,循眾雖死戰,猶不能抗。裕乘勝擊之,循單舸而走,收散卒得千餘人,還保廣州。裕先遣孫處從海道據番禺城,循攻之不
 下。道覆保始興,因險自固。循乃襲合浦,剋之,進攻交州。至龍編,刺史杜慧度譎而敗之。



 循勢屈,知不免,先鴆妻子十餘人,又召妓妾問曰:「我今將自殺,誰能同者?」多云:「雀鼠貪生,就死實人情所難。」有云:「官尚當死,某豈願生!」於是悉鴆諸辭死者,因自投於水。慧度取其尸斬之,及其父嘏;同黨盡獲,傳首京都。



 譙縱,巴西南充人也。祖獻之,有重名於西土。縱少而謹慎,蜀人愛之。為安西府參軍。義熙元年,刺史遣縱及侯暉等領諸縣氐進兵東下。暉有貳志,因梁州人不樂東
 也,將圖益州刺史毛璩,與巴西陽昧結謀於五城水口,共逼縱為主。縱懼而不當,走投于水,暉引出而請之,至於再三,遂以兵逼縱於輿上。攻璩弟西夷校尉瑾於涪城,城陷,瑾死之,縱乃自號梁、秦二州刺史。璩聞縱反,自略城步還成都,遣參軍王瓊率三千人討縱,又遣弟瑗領四千兵繼瓊後進。縱遣弟明子及暉距瓊於廣漢,瓊擊破暉等,追至綿竹。明子設二伏以待之,大敗瓊眾,死者十八九。益州營戶李騰開城以納縱。



 毛璩既死,縱以從弟洪為益州刺史,明子為鎮東將軍、巴州刺史,率其眾五千人屯白帝,自稱成都王。明年,遣使稱籓於姚興,
 將順流東寇,以討車騎將軍劉裕為名,乞師於姚興,且請桓謙為助,興遣之。



 九年,劉裕以西陽太守朱齡石為益州刺史,寧朔將軍臧喜、下邳太守劉鐘,蘭陵太守蒯恩等率眾二萬,自江陵討縱。初謀元率,僉難其人,齡石資名素淺,裕違眾拔之,授以麾下之半。藏喜,裕妻弟也,位出其右,又隸焉。齡石次于白帝,縱遣譙道福重兵守涪。齡石師次平模,去成都二百里,縱遣其大將軍侯暉、尚書僕射譙詵屯平模,夾岸連城,層樓重柵,眾未能攻。齡石謂劉鐘曰:「天方暑熱,賊今固險,攻之難拔,只困我師。吾欲蓄銳息兵,伺隙而進,卿以為何如?」鐘曰:「不然。前
 揚聲言大將由內水,故道福不敢捨涪,今重軍逼之,出其不意,侯暉之徒已破膽矣。正可因其兇而攻之,勢當必剋。剋平模之後,自可鼓行而前,成都必不能守。若綏兵相持,虛實相見,涪軍復來,難為敵也。進不能戰,退無所資,二萬餘人因為蜀子虜耳。」從之。翌日,進攻皆剋,斬侯暉等,於是遂進。縱之城守者相次瓦解,縱乃出奔。其尚書令馬耽封倉庫以待王師。及齡石入成都,誅縱同祖之親,餘皆安堵,使復其業。



 縱之走也,先如其墓,縱女謂縱曰:「走必不免,只取辱焉。等死,死於先人之墓可也。」縱不從,投道福于涪。道福怒謂縱曰:「大丈夫居如斯功
 業,安可棄哉!今欲為降虜,豈可而得!人誰不死,何懼之甚!」因投縱以劍,中其馬鞍。縱去之,乃自縊。道福謂其徒曰:「吾養爾等,正為今日。蜀之存亡,實係在我,不在譙王。我尚在,猶足一戰。」士咸許諾。乃散金帛以賜其眾,眾受之而走。道福獨奔廣漢,廣漢人杜瑾執之。朱齡石徙馬耽於越巂,追殺之。耽之徙也,謂其徒曰:「朱侯不送我京師,滅眾口也,吾必不免。」乃盥洗而臥,引繩而死。須臾,齡石師至,遂戮尸焉。



 史臣曰:惠皇失御,政紊朝危,難起蕭牆,毒痡函夏,九州波駭,五嶽塵飛,干戈日尋,戎車競逐。王彌好亂樂禍,挾
 詐懷姦,命儔嘯侶,伺間侯隙,助悖逆於平陽,肆殘忍於都邑,遂使生靈塗炭,神器流離,邦國軫《麥秀》之哀,宮廟興《黍離》之痛,豈天意乎?豈人事乎?何醜虜之猖狂而亂離之斯瘼者也!張昌等或䲭張淮浦,或蟻聚荊衡,招烏合之凶徒,逞豺狼之貪暴,憑陵險隘,倔強江湖,未淹歲稔,咸至誅戮,實自取之,非為不幸。峻約同惡相濟,生此亂階,孫盧同類相求,嗣成妖逆。至乃干戈掃地,災沴滔天,雖樊謝之毒被含靈,李郭之禍延宮闕,方凶比暴,弗是加也。譙縱乘茲釁隙,肆彼姦謀,旋踵而亡,無足論矣。



 贊曰:中朝隳政,王彌肇亂。神器流離,生靈塗炭。群妖伺
 隙,構茲多難。薦食荊衡,陵虐江漢。孫盧奸慝,約峻殘賊。窮兇極暴,為鬼為蜮。縱竊岷峨,旋至顛踣。



 



 載記序



 古者帝王,乃生奇類、淳維,伯禹之苗裔,豈異類哉?反首衣皮,餐膻飲水重,而震驚中域,其來自遠。天未悔禍,種落彌繁。其風俗險詖,性靈馳突,前史載之,亦以詳備。軒帝患其干紀,所以徂征;武王竄以荒服,同乎禽獸。而於露寒之野,候月覘風,睹隙揚埃,乘間騁暴,邊城不得緩帶,百姓靡有室家。孔子曰:「微管仲,吾其被髮左衽矣。」此言能教訓卒伍,整齊車甲,邊埸既伏,境內以安。然則燕築造陽之郊,秦塹臨洮之險,登天山,絕地脈,苞玄菟,款黃河,所以防夷狄之亂中華,其備豫如此。



 漢宣帝初
 納呼韓,居之亭鄣,委以候望,始寬戎狄。光武亦以南庭數萬徙入西河,後亦轉至五原,連延七郡。董卓之亂,則汾晉之郊蕭然矣。郭欽騰箋於武帝,江統獻策於惠皇,皆以為魏處戎夷,繡居都鄙,請移沙塞之表,定一殷周之服。統則憂諸并部,欽則慮在盟津。言猶自口,元海已至。語曰「失以豪釐」,晉卿大夫之辱也。聰之誓兵,東兼齊地;曜之馳旆,西踰隴山,覆沒兩京,蒸徒百萬。天子陵江御物,分據地險,迴首中原,力不能救,劃長淮以北,大抵棄之。胡人利我艱虞,分鑣起亂;晉臣或阻兵遐遠,接武效尤。



 大凡劉元海以惠帝永興元年據離石稱漢。後九
 年,石勒據襄國稱趙。張氏先據河西,是歲,自石勒後三十六年也,重華自稱涼王。後一年,冉閔據鄴稱魏。後一年,苻健據長安稱秦。慕容氏先據遼東稱燕,是歲,自苻健後一年也,俊始僭號。後三十一年,後燕慕容垂據鄴。後二年,西燕慕容沖據阿房。是歲也,乞伏國仁據桴罕稱秦。後一年,慕容永據上黨。是歲也,呂光據姑臧稱涼。後十二年,慕容德據滑臺稱南燕。是歲也,禿髮烏孤據廉川稱南涼,段業據張掖稱北涼。後三年,李玄盛據敦煌稱西涼。後一年,沮渠蒙遜殺段業,自稱涼。後四年,譙縱據蜀稱成都王。後二年,赫連勃勃據朔方稱大夏。後
 二年,馮跋殺離班,據和龍稱北燕。提封天下,十喪其八,莫不龍旌帝服,建社開祊,華夷咸暨,人物斯在。或篡通都之鄉,或擁數州之地,雄圖內卷,師旅外并,窮兵凶於勝負,盡人命於鋒鏑,其為戰國者一百三十六載,抑元海為之禍首云。



\end{pinyinscope}