\article{列傳第五十七 涼武昭王子士業}

\begin{pinyinscope}
涼武昭王
 \gezhu{
  子士業}



 武昭王諱暠,字玄盛,小字長生,隴西成紀人,姓李氏,漢前將軍廣之十六世孫也。廣曾祖仲翔,漢初為將軍,討叛羌于素昌,素昌即狄道也,眾寡不敵,死之。仲翔子伯考奔喪,因葬于狄道之東川,遂家焉,世為西州右姓。高祖雍,曾祖柔,仕晉並歷位郡守。祖弇,仕張軌為武衛將軍、安世亭侯。父昶,幼有令名,早卒,遺腹生玄盛。少而好
 學,性沈敏寬和,美器度,通涉經史,尤善文義。及長,頗習武藝,誦孫吳兵法。嘗與呂光太史令郭黁及其同母弟宋繇同宿,黁起謂繇曰:「君當位極人臣,李君有國土之分,家有騧草馬生白額駒,此其時也。」



 呂光末,京兆段業自稱涼州牧,以敦煌太守趙郡孟敏為沙州刺史,署玄盛效穀令。敏尋卒,敦煌護軍馮翊郭謙、沙州治中敦煌索仙等以玄盛溫毅有惠政,推為寧朔將軍、敦煌太守。玄盛初難之,會宋繇仕於業,告歸敦煌,言於玄盛曰:「兄忘郭黁之言邪?白額駒今已生矣。」玄盛乃從之。尋進號冠軍,稱籓於業。業以玄盛為安西將軍、敦煌太守,領護
 西胡校尉。



 及業僭稱涼王,其右衛將軍索嗣構玄盛於業,乃以嗣為敦煌太守,率騎五百而西,未至二十里,移玄盛使迫己。玄盛驚疑,將出迎之,效穀令經邈及宋繇止之曰:「呂氏政衰,段業闇弱,正是英豪有為之日,將軍處一國成資,奈何束手於人!索嗣自以本邦,謂人情附己,不虞將軍卒能距之,可一戰而擒矣。」宋繇亦曰:「大丈夫已為世所推,今日便授首於嗣,豈不為天下笑乎!大兄英姿挺傑,有雄霸之風,張王之業不足繼也。」玄盛曰:「吾少無風雲之志,因官至此,不圖此郡士人忽爾見推。向言出迎者,未知士大夫之意故也。」因遣繇覘嗣。繇見
 嗣,啖以甘言,還謂玄盛曰:「嗣志驕兵弱,易擒耳。」於是遣其二子士業、讓與邈、繇及以司馬尹建興等逆戰,破之,嗣奔還張掖。玄盛素與嗣善,結為刎頸交,反為所構,故深恨之,乃罪狀嗣於段業。業將且渠男又惡嗣,至是,因勸除之。業乃殺嗣,遣使謝玄盛,分敦煌之涼興、烏澤、晉昌之宜禾三縣為涼興郡,進玄盛持節、都督涼興已西諸軍事、鎮西將軍,領護西夷校尉。時有赤氣起于玄盛後園,龍跡見于小城。



 隆安四年,晉昌太守唐瑤移檄六郡,推玄盛為大都督、大將軍、涼公、領秦涼二州牧、護羌校尉。玄盛乃赦其境內,建年為庚子,追尊祖弇曰涼景公,
 父昶涼簡公。以唐瑤為征東將軍,郭謙為軍諮祭酒,索仙為左長史,張邈為右長史,尹建興為左司馬,張體順為右司馬,張條為牧府左長史,令狐溢為右長史,張林為太府主簿,宋繇、張謖為從事中郎,繇加折衝將軍,謖加揚武將軍,索承明為牧府右司馬,令狐遷為武衛將軍、晉興太守,氾德瑜為寧遠將軍、西郡太守,張靖為折衝將軍、河湟太守,索訓為威遠將軍,西平太守,趙開為騂馬護軍、大夏太守,索慈為廣武太守,陰亮為西安太守,令狐赫為武威太守,索術為武興太守,以招懷東夏。又遣宋繇東伐涼興,並擊玉門已西諸城,皆下之,遂屯
 玉門、陽關,廣田積穀,為東伐之資。



 初,呂光之稱王也,遣使市六璽玉於于闐,至是,玉至敦煌,納之郡府。仍於南門外臨水起堂,名曰靖恭之堂,以議朝政,閱武事。圖贊自古聖帝明王、忠臣孝子、烈士貞女,玄盛親為序頌,以明鑒戒之義,當時文武群僚亦皆圖焉。有白雀翔於靖恭堂,玄盛觀之大悅。又立泮宮,增高門學生五百人。起嘉納堂於後園,以圖贊所志。



 義熙元年,玄盛改元為建初,遣舍人黃始、梁興間行奉表詣闕曰:



 昔漢運將終,三國鼎峙,鈞天之歷,數鐘皇晉。高祖闡鴻基,景文弘帝業,嗣武受終,要荒率服,六合同風,宇宙齊貫。而惠皇失馭,
 權臣亂紀,懷愍屯邅,蒙塵于外,懸象上分,九眼下裂,眷言顧之,普天同憾。伏惟中宗元皇帝基天紹命,遷幸江表,荊揚蒙弘覆之矜,五都為荒榛之藪。故太尉、西平武公軌當元康之初,屬擾攘之際,受命典方,出撫此州,威略所振,聲蓋海內。明盛繼統,不損前志,長旌所指,仍闢三秦,義立兵強,拓境萬里。文桓嗣位,奕葉載德,囊括關西,化被崐裔,遐邇款籓,世修職貢。晉德之遠揚,翳此州是賴。大都督、大將軍天錫以英挺之姿,承七世之業,志匡時難,剋隆先勳,而中年降災,兵寇侵境,皇威遐邈,同獎弗及,以一方之師抗七州之眾,兵孤力屈,社稷以喪。



 臣聞歷數相推,歸餘於終,帝王之興,必有閏位。是以共工亂象於黃農之間,秦項篡竊於周漢之際,皆機不轉踵,覆束成凶。自戎狄陵華,已涉百齡,五胡僭襲,期運將杪,四海顒顒,懸心象魏。故師次東關,趙魏莫不企踵;淮南大捷,三方欣然引領。伏惟陛下道協少康,德侔光武,繼天統位,志清函夏。至如此州,世篤忠義,臣之群僚以臣高祖東莞太守雍、曾祖北地太守柔荷寵前朝,參忝時務,伯祖龍驤將軍、廣晉太守、長寧侯卓,亡祖武衛將軍、天水太守、安世亭侯弇毗佐涼州,著功秦隴,殊寵之隆,勒於天府,妄臣無庸,輒依竇融故事,迫臣以義,上臣
 大都督、大將軍、涼公、領秦涼二州牧、護羌校尉。臣以為荊楚替貢。齊桓興召陵之師,諸侯不恭,晉文起城濮之役,用能勛光踐土,業隆一匡,九域賴其弘猷,《春為》恕其專命。功冠當時,美垂幹祀。況今帝居未復,諸夏昏墊,大禹所經,奄為戎墟,五嶽神山,狄汙其三,九州名都,夷穢其七,辛有所言,於茲而驗。微臣所以叩心絕氣,忘寢與食,彫肝焦慮,不遑寧息者也。江涼雖遼,義誠密邇,風雲茍通,實如脣齒。臣雖名未結於天臺,量未著於海內,然憑賴累祖寵光餘烈,義不細辭,以稽大務,輒順群議,亡身即事。轅弱任重,懼忝威命。昔在春秋,諸侯宗周,國皆
 稱元,以布時令。今天臺邈遠,正朔未加,發號旋令,無以紀數。輒年冠建初,以崇國憲。冀杖寵靈,全制一方,使義誠著於所天,玄風扇于九壤,殉命灰身,隕越慷慨。



 玄盛謂群僚曰:「昔河右分崩,群豪競起,吾以寡德為眾賢所推,何嘗不忘寢與食,思濟黎庶。故前遣母弟繇董率雲騎,東殄不庭,軍之所至,莫不賓下。今惟蒙遜鴟跱一城。自張掖已東,晉之遺黎雖為戎虜所制,至於向義思風,過於殷人之望西伯。大業須定,不可安寢,吾將遷都酒泉,漸逼寇穴,諸君以為何如?」張邈贊成其議,玄盛大悅曰:「二人同心,其利斷金。張長史與孤同矣,夫復何疑!」乃
 以張體順為寧遠將軍、建康太守,鎮樂涫,徵宋繇為右將軍,領敦煌護軍,與其子敦煌太守讓鎮敦煌,遂遷居于酒泉。手令誡其諸子曰:



 吾自立身,不營世利;經涉累朝,通否任時;初不役智,有所要求,今日之舉,非本願也。然事會相驅,遂荷州土,憂責不輕,門戶事重。雖詳人事,未知天心,登車理轡,百慮填胸。後事付汝等,粗舉旦夕近事數條,遭意便言,不能次比。至於杜漸防萌,深識情變,此當任汝所見深淺,非吾敕誡所益也。汝等雖年未至大,若能剋己纂修,比之古人,亦可以當事業矣。茍其不然,雖至白首,亦復何成!汝等其戒之慎之。



 節酒慎言,
 喜怒必思,愛而知惡,憎而知善,動念寬恕,審而後舉。眾之所惡,勿輕承信,詳審人,核真偽,遠佞諛,近忠正。蠲刑獄,忍煩擾,存高年,恤喪病,勤省案,聽訟訴。刑法所應,和顏任理,慎勿以情輕加聲色。賞勿漏疏,罰勿容親。耳目人間,知外患苦。禁禦左右,無作威福。勿伐善施勞,逆詐億必,以示己明。廣加諮詢,無自專用,從善如順流,去惡如探湯。富貴而不驕者至難也,念此貫心,勿忘須臾。僚佐邑宿,盡禮承敬,宴饗饌食,事事留懷。古今成敗,不可不知,退朝之暇,念觀典籍,面牆而立,不成人也。



 此郡世篤忠厚,人物郭雅,天下全盛時,海內猶稱之,況復今日,
 實是名邦,正為五百年鄉黨婚親相連,至於公理,時有小小頗迴,為當隨宜斟酌。吾臨蒞五年,兵難騷動,未得休眾息役,惠康士庶。至於掩瑕藏疾,滌除疵垢,朝為寇仇,夕委心膂,雖未足希準古人,粗亦無負於新舊。事任公平,坦然無類,初不容懷,有所損益,計近便為少,經遠如有餘,亦無愧於前志也。



 初,玄盛之西也,留女敬愛養於外祖尹文。文既東遷,玄盛從姑梁褒之母養之。其後禿髮傉檀假道於北山。鮮卑遣褒送敬愛于酒泉,并通和好。玄盛遣使報聘,贈以方物。玄盛親率騎二萬,略地至於建東,鄯善前部王遣使貢其方物,且渠蒙遜來侵,
 至於建康,掠三千餘戶而歸。玄盛大怒,率騎追之,及于彌安,大敗之,盡收所掠之戶。



 初,苻堅建元之末,徙江漢之人萬餘戶于郭煌,中州之人有田疇不闢者,亦徙七千餘戶。郭黁之寇武威,武威、張掖已東人西奔敦煌、晉昌者數千戶。及玄盛東遷。皆徙之於酒泉,分南人五千戶置會稽郡,中州人五千戶置廣夏郡,餘萬三千戶分置武威、武興、張掖三郡,築城於敦煌南子亭,以威南虜,又以前表未報,復遣沙門法泉間行奉表,曰:



 江山悠隔,朝宗無階,延首雲極,翹企遐方。伏惟陛下應期踐位,景福自天,臣去乙巳歲順從群議,假統方城,時遣舍人黃
 始奉表通誠,遙途險曠,未知達不?吳涼懸邈,蜂蠆充衢,方珍貢使,無由展御,謹副寫前章,或希簡達。



 臣以其歲進師酒泉,戒戎廣平,庶攘茨穢,而黠虜恣睢,未率威教,憑守巢穴,阻臣前路。竊以諸事草創,倉帑未盈,故息兵按甲,務農養士。時移節邁,荏苒三年,撫劍歎憤,以日成歲。今資儲已足,器械已充,西招城郭之兵,北引丁零之眾,冀憑國威席卷河隴,揚旌秦川,承望詔旨,盡節竭誠,隕越為效。



 又臣州界回遠,勍寇未除,當順鎮副為行留部分,輒假臣世子士業監前鋒諸軍事、撫軍將軍、護羌校尉,督攝前軍,為臣先驅。又敦煌郡大眾殷,制御西域,
 管轄萬里,為軍國之本,輒以次子讓為寧朔將軍、西夷校尉、敦煌太守,統攝昆裔,輯寧殊方。自餘諸子,皆在戎間,率先士伍,臣總督大綱,畢在輸力,臨機制命,動靖續聞。



 玄盛既遷酒泉,乃敦勸稼穡。郡僚以年穀頻登,百姓樂業,請勒銘酒泉,玄盛許之。於是使儒林祭酒劉彥明為文,刻石頌德。既而蒙遜每年侵寇不止,玄盛志在以德撫其境內,但與通和立盟,弗之校也。是時白狼、白兔、白雀、白雉、白鳩皆棲其園囿,其群下以為白祥金精所誕,皆應時邕而至,又有神光、甘露、連理、嘉禾眾瑞,請史官記其事,玄盛從之。尋而蒙遜背盟來侵,玄盛遣世子
 士業要擊敗之,獲其將且渠百年。



 玄盛上巳日宴于曲水,命群僚賦詩。而親為之序。於是寫諸葛亮訓誡以勖諸子曰:「吾負荷艱難,寧濟之勳未建,雖外總良能,憑股肱之力,而戎務孔殷,坐而待旦。以維城之固,宜兼親賢,故使汝等未及師保之訓,皆弱年受任。常懼弗剋,以貽咎悔。古今之事不可以不知,茍近而可師,何必遠也。覽諸葛亮訓勵,應璩奏諫,尋其終始,周孔之教盡在中矣。為國足以致安,立身足以成名,質略易通,寓目則了,雖言發往人,道師於此。且經史道德如採菽中原,勤之者則功多,汝等可不勉哉!」玄盛乃修敦煌舊塞東西二圍,
 以防北虜之患,築敦煌舊塞西南二圍,以威南虜。



 玄盛以緯世之量,當呂氏之末,為群雄所奉,遂啟霸圖,兵無血刃,坐定千里,謂張氏之業指期而成,河西十郡歲月而一。既而禿髮傉檀入據姑臧,且渠蒙遜基宇稍廣,於是慨然著《述志賦》焉,其辭曰:



 涉至虛以誕駕,乘有輿於本無,稟玄元而陶衍,承景靈之冥符。蔭朝雲之庵藹,仰朗日之照煦。既敷既載,以育以成。幼希顏子曲肱之榮,游心上典,玩禮敦經。蔑玄冕於朱門,羨漆園之傲生;尚漁父於滄浪,善沮溺之耦耕,穢鵄鳶之籠哧,欽飛鳳于太清;杜世競於方寸,絕時譽之嘉聲。超霄吟於崇嶺。奇
 秀木之陵霜;挺修幹之青葱,經歲寒而彌芳。情遙遙以遠寄,想四老軍光;將戢繁榮於常衢,控雲轡而高驤;攀瓊枝於玄圃,漱華泉之淥漿;和吟鳳之逸響,應鳴鸞于南岡。



 時弗獲青彡,心往形留,眷駕陽林,宛首一丘;衝風沐雨,載沈載浮。利害繽紛以交錯,嘆感循環而相求。乾扉奄寂以重閉,天地絕津而無舟;悼貞信之道薄,謝慚德於圜流。遂乃去玄覽,慶世賓,肇弱巾於東宮,並羽儀於英倫,踐宣德之秘庭,翼明后於紫宸。赫赫謙光,崇明奕奕,岌岌王居,詵詵百辟,君希虞夏,臣庶夔益。



 張王頹巖,梁后墜壑,淳風杪莽以永喪,搢紳淪胥而覆溺。呂發
 釁於閨牆,厥構摧以傾顛;疾風飄於高木,回湯沸於重泉;飛塵翕以蔽日,大火炎其燎原;名都幽然影絕,千邑闃而無煙。斯乃百六之恒數,起滅相因而迭然。於是人希逐鹿之圖,家有雄霸之想,闇王命而不尋,邀非分於無象。故覆車接路而繼軌,膏生靈於土壤。哀餘類之忪懞,邈靡依而靡仰;求欲專而失逾遠,寄玄珠於罔象。



 悠悠涼道。鞠焉荒凶,杪杪余躬,迢迢西邦,非相期之所會,諒冥契而來同。跨弱水以建基,躡昆墟以為墉,總奔駟之駭轡,接摧轅於峻峰。崇崖崨嶪,重險萬尋,玄邃窈窕,磐紆嶔岑,榛棘交橫,河廣水深,狐狸夾路,鴞鵄群吟,挺
 非我以為用,任至當如影響;執同心以御物,懷自彼於握掌;匪矯情而任荒,乃冥合而一往,華德是用來庭,野逸所以就鞅。



 休矣時英,茂哉雋哲,庶罩網以遠籠,豈徒射鉤與斬袂!或脫梏而纓蕤,或後至而先列,採殊才於巖陸,拔翹彥於無際。思留侯之神遇,振高浪以蕩穢;想孔明於草廬,運玄籌之罔滯;洪操槃而慷慨,起三軍以激銳。詠群豪之高軌,嘉關張之飄傑,誓報曹而歸劉,何義勇之超出!據斷橋而橫矛,亦雄姿之壯發。輝輝南珍,英英周魯,挺奇荊吳,昭文烈武,建策烏林,龍驤江浦。摧堂堂之勁陣,鬱風翔而雲舉,紹攀韓之遠蹤,侔徽猷於
 召武,非劉孫之鴻度,孰能臻茲大祜!信乾坤之相成,庶物希風而潤雨。



 崏益既蕩,三江已清,穆穆盛勳,濟濟隆平,御群龍而奮策,彌萬載以飛榮,仰遺塵於絕代,企高山而景行。將建朱旗以啟路,驅長轂而迅征,靡商風以抗旆,拂招搖之華旌,資神兆於皇極,協五緯之所寧。赳赳干城,翼翼上粥,恣馘奔鯨,截彼醜類。且灑遊塵於當陽,拯涼德於已墜。間昌寓之驂乘,暨襄城而按轡。知去害之在茲,體牧童之所述,審機動之至微,思遺餐而忘寐,表略韻於紈素,託精誠于白日。



 玄盛寢疾,顧命宋繇曰:「吾少離荼毒,百艱備嘗,於喪亂之際,遂為此方所推,
 才弱智淺,不能一同河右。今氣力惙然,當不復起矣。死者大理,吾不悲之,所恨志不申耳。居元首之位者,宜深誡危殆之機。吾終之後,世子猶卿子也,善相輔導,述吾平生,勿令居人之上,專驕自任。軍國之宜,委之於卿,無使籌略乖衷,失成敗之要。」十三年,薨,時年六十七。國人上謚曰武昭王,墓曰建世陵,廟號太祖。



 先是,河右不生楸、槐、柏、漆,張駿之世,取於秦隴而植之,終於皆死,而酒泉宮之西北隅有槐樹生焉,玄盛又著《槐樹賦》以寄情,蓋歎僻陋遐方,立功非所也。亦命主簿梁中庸及劉彥明等並作文。感兵難繁興,時俗喧競,乃著《大酒容賦》以
 表恬豁之懷。與辛景、辛恭靖同志友善,景等歸晉,遇害江南,玄盛聞而弔之。玄盛前妻,同郡辛納女,貞順有婦儀,先卒,玄盛親為之誄。自餘詩賦數十篇。世子譚早卒,第二子士業嗣。



 涼後主諱歆,字士業。玄盛薨時,府僚奉為大都督、大將軍、涼公、領涼州牧、護羌校尉,大赦境內,改年為嘉興。尊母尹氏為太后,以宋繇為武衛將軍、廣夏太守、軍諮祭酒、錄三府事,索仙為征虜將軍、張掖太守。



 且渠蒙遜遣其張掖太守且渠廣宗祚降誘士業,士業遣武衛溫宜等赴之,親勒大軍為之後繼。蒙遜率眾三萬,設伏于蓼
 泉。士業聞,引兵還,為遜所逼。士業親貫甲先登,大敗之,追奔百餘里,俘斬七千餘級。明年,蒙遜又伐士業,士業將出距之,左長史張體順固諫,乃止。蒙遜大芟秋稼而還。是歲,朝廷以士業為持節、都督七郡諸軍事、鎮西大將軍、護羌校尉、酒泉公。



 士業用刑頗嚴,又繕築不止,從事中郎張顯上疏諫曰:「入歲已來,陰陽失序,屢有賊風暴雨,犯傷和氣。今區域三分,勢不久並,並兼之本,實在農戰,懷遠之略,事歸寬簡。而更繁刑峻法,宮室是務,人力凋殘,百姓愁悴。致災之咎,實此之由。」主簿氾稱又上疏諫曰:



 臣聞天之子愛人后,殷勤至矣。故政之不修,則
 垂災譴以戒之。改者雖危必昌,宋景是也;其不改者,雖安必亡,虢公是也。元年三月癸卯,敦煌謙德堂陷;八月,效穀地烈;二年元日,昏霧四塞;四月,日赤無光,二旬乃復;十一月,狐上南門;今茲春夏地頗五震;六月,隕星於建康。臣雖學不稽古,敏謝仲舒,頗亦聞道于先師,且行年五十有九,請為殿下略言耳目之所聞見,不復能遠論書傳之事也。



 乃者咸安之初,西平地烈,狐入謙光殿前,俄而秦師奄至,都城不守。梁熙既為涼州,藉秦氏兵亂,規有全涼之地,外有撫百姓,內多聚斂,建元十九年姑臧南門崩,隕石於閑豫堂,二十年而呂光東反,子敗
 於前,身戮於後。段業因群胡創亂,遂稱制此方,三年之中,地震五十餘所,既而先王龍興瓜州,蒙遜殺之張掖。此皆目前之成事,亦殿下之所聞知。效穀,先王鴻漸之始,謙德,即尊之室,基陷地裂,大凶之徵也。日者太陽之精,中國之象,赤而無光,中國將為胡夷之所陵滅。諺曰:「野獸入家,主人將去。」今狐上南門,亦災之大也。又狐者胡也,天意若曰將有胡人居於此城,南面而居者也。昔春秋之世,星隕于宋,襄公卒為楚所擒。地者至陰,胡夷之象,當靜而動,反亂天常,天意若曰胡夷將震動中國,中國若不修德,將有宋襄之禍。



 臣蒙先朝布衣之眷,輒
 自同子弟之親,是以不避忤上之誅,昧死而進愚款。願殿下親仁善鄰,養威觀釁,罷宮室之務,止游畋之娛。後宮嬪妃、諸夷子女,躬受分田,身勸蠶績,以清儉素德為榮,息茲奢靡之費,百姓租稅,專擬軍國。虛衿下士,廣招英雋,修秦氏之術,以強國富俗。待國有數年之積,庭盈文武之士,然後命韓白為前驅,納子房之妙算,一鼓而姑臧可平,長驅可以飲馬涇渭,方江面而爭天下,豈蒙遜之足憂!不然,臣恐宗廟之危必不出紀。



 士業並不納。



 士業立四年而宋受禪,士業將謀東伐,張體順切諫,乃止。士業聞蒙遜南伐禿髮傉檀,命中外戒嚴,將攻張掖,尹
 氏固諫,不聽,宋繇又固諫,士業並不從。繇退而歎曰:「大事去矣,吾見師之出,不見師之還也!」士業遂率步騎三萬東伐,鎰于都瀆澗。蒙遜自浩亹來,距戰于懷城,為蒙遜所敗。左右勸士業還酒泉,士業曰:「吾違太后明誨,遠取敗辱,不殺此胡,復何面目以見母也!」勒眾復戰,敗于蓼泉,為蒙遜所害。士業諸弟酒泉太守翻、新城太守預、領羽林右監密、左將軍眺、右將軍亮等西奔敦煌,蒙遜遂入于酒泉。士業之未敗也,有大蛇從南門而入,至于恭德殿前;有雙雉飛出宮內;通街大樹上有烏鵲爭巢,鵲為烏所殺。又有敦煌父老令狐熾夢白頭公衣帢而
 謂熾曰:「南風動,吹長木,胡桐椎,不中轂。」言訖忽然不見。士業小字桐椎,至是而亡。



 翻及弟敦煌太守恂與諸子等棄敦煌,奔於北山,蒙遜以索嗣子遠緒行敦煌太守。元緒粗險好殺,大失人和。郡人宋承、張弘以恂在郡有惠政。密信招恂。恂率數十騎入于敦煌,元緒東奔涼興,宋承等推恂為冠軍將軍、涼州刺史。蒙遜遣世子德政率眾攻恂,恂閉門不戰,蒙遜自率眾二萬攻這,三面起堤,以水灌城。恂遣壯士一千,連版為橋,潛欲決堤,蒙遜勒兵逆戰,屠其城。士業子重耳,脫身奔于江左,仕于宋。後歸魏,為恒農太守。蒙遜徙翻子寶等于姑臧,歲餘,北
 奔伊吾,後歸于魏,獨尹氏及諸女死於伊吾。



 玄盛以安帝隆安四年立,至宋少帝景平元年滅,據河右凡二十四年。



 史臣曰:王者受圖,咸資世德,猶混成之先大帝,若一氣之生兩儀。是以中陽勃興,資豢龍之構趾;景亳垂統,本吞燕之開基。涼武昭王英姿傑出,運陰陽而緯武,應變之道如神;吞日月以經天,成物之功若歲。故能懷茺弭暴,開國化家,宅五郡以稱籓,屈三分而奉順。若乃《詩》褒秦仲,後嗣建削平之業;頌美公劉,末孫興配天之祚。或發迹於汧渭,或布化於邠岐,覆簣創元天之基,疏涓開
 環海之宅。彼既有漸,此亦同符,是知景命攸歸,非一朝之可致,累功積慶,其所由來遠矣。



 贊曰:武昭英睿,忠勇霸世。王室雖微,乃誠無替。遺黎飲德,絕壤霑惠。積祉丕基,克昌來裔。



\end{pinyinscope}