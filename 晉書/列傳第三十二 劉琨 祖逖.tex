\article{列傳第三十二 劉琨 祖逖}

\begin{pinyinscope}

 劉琨子群琨兄輿輿子演祖逖兄納



 劉琨,字越石,中山魏昌人,漢中山靖王勝之後也。祖邁,有經國之才,為相國參軍、散騎常侍。父蕃,清高沖儉,位至光祿大夫。琨少得俊朗之目,與范陽祖納俱以雄豪著名。年二十六,為司隸從事。時征虜將軍石崇河南金谷澗中有別廬,冠絕時輩,引致賓客,日以賦詩。琨預其間,文詠頗為當時所許。祕書監賈謐參管朝政,京師人
 士無不傾心。石崇、歐陽建、陸機、陸雲之徒,並以文才降節事謐,琨兄弟亦在其間,號曰「二十四友』。太尉高密王泰辟為掾,頻遷著作郎、太學博士、尚書郎。



 趙王倫執政,以琨為記室督,轉從事中郎。倫子荂,即琨姊婿也,故琨父子兄弟並為倫所委任。及篡,荂為皇太子,琨為荂詹事。三王之討倫也,以琨為冠軍、假節,與孫秀子會率宿衛兵三萬距成都王穎,戰於黃橋,琨大敗而還,焚河橋以自固。及齊王冏輔政,以其父兄皆有當世之望,故特宥之,拜兄輿為中書郎,琨為尚書左丞,轉司徒左長史。冏敗,范陽王虓鎮許昌,引為司馬。



 及惠帝幸長安,東海
 王越謀迎大駕,以琨父蕃為淮北護軍、豫州刺史。劉喬攻范陽王虓於許昌也,琨輿汝南太守杜育等率兵救之,未至而虓敗,琨輿虓俱奔河北,琨之父母遂為劉喬所執。琨乃說冀州刺史溫羨,使讓位於虓。及虓領冀州,遺琨詣幽州,乞師於王浚,得突騎八百人,與虓濟河,共破東平王懋於廩丘,南走劉喬,始得其父母。又斬石超,降呂朗,因統諸軍奉迎大駕於長安。以動封廣武侯,邑二千戶。



 永嘉元年,為并州刺史,加振威將軍,領匈奴中郎將。琨在路上表曰:「臣以頑蔽,志望有限,因緣際會,遂忝過任。九月末得發,道險山峻,胡寇塞路,輒以少擊眾,
 冒險而進,頓伏艱危,辛苦備嘗,即日達壺口關。臣自涉州疆,目睹困乏,流移四散,十不存二,攜老扶弱,不絕於路。及其在者,鬻賣妻子,生相捐棄,死亡委危,白骨橫野,哀呼之聲,感傷和氣。群胡數萬,周匝四山,動足遇掠,開目睹寇。唯有壺關,可得告糴。而此二道,九州之陰,數人當路,則百夫不敢進,公私往反,沒喪者多。嬰守窮城,不得薪采,耕牛既盡,又乏田器。以臣愚短,當此至難,憂如循環,不遑寢食。臣伏思此州雖去邊朔,實邇皇畿,南通河內,東連司冀,北捍殊俗,西禦彊虜,是勁弓良馬勇士精銳之所出也。當須委輸,乃全其命。今上尚書,請此州
 穀五百萬斛,絹五百萬匹,綿五百萬斤。願陛下時出臣表,速見聽處。」朝廷許之。



 時東嬴公騰自晉陽鎮鄴,並土饑荒,百姓隨騰南下,餘戶不滿二萬,寇賊繼橫,道路斷塞。琨募得千餘人,轉鬥至晉陽。府寺焚毀,僵尸蔽地,其有存者,饑羸無復人色,荊棘成林,豺狼滿道。琨翦除荊棘,收葬枯骸,造府朝,建市獄。寇盜互來掩襲,恒以城門為戰場,百姓負楯以耕,屬鞬而耨。琨撫循勞徠,甚得物情。劉元海時在離石,相去三百許里。琨密遣離間其部雜虜,降者萬餘落。元海甚懼,遂城蒲子而居之。在官未期,流人稍復,雞犬之音復相接矣。琨父蕃自洛赴之。人
 士奔迸者多歸於琨,琨善於懷撫,而短於控御。一日之中,雖歸者數千,去者亦以相繼。然素奢豪,嗜聲色,雖暫自矯勵,而輒復縱逸。



 河南徐潤者,以音律自通,遊於貴勢,琨甚愛之,署為晉陽令。潤恃寵驕恣,干預琨政。奮威護軍令狐盛性亢直,數以此為諫,并勸琨除潤,琨不納。初,單于猗以救東嬴公騰之功,琨表其弟猗盧為代郡公,與劉希合眾於中山。王浚以琨侵己之地,數來擊琨,琨不能抗,由是聲實稍損。徐潤又譖令狐盛於琨曰:「盛將勸公稱帝矣。」琨不之察,便殺之。琨母曰:「汝不能弘經略,駕豪傑,專欲除勝己以自安,當何以得濟!如是,禍
 必及我。」不從。盛子泥奔於劉聰,具言虛實。聰大喜,以泥為鄉導。屬上黨太守襲醇降於聰,雁門烏丸復反,琨親率精兵出御之。聰遣子粲及令狐泥乘虛襲晉陽,太原太守高喬以郡降聰,琨父母並遇害。琨引猗盧並力攻粲,大敗之,死者十五六。琨乘勝追之,更不能克。猗盧以為聰未可滅,遺琨牛羊車馬而去,留其將箕澹、段繁等戍晉陽。琨志在復仇,而屈於力弱,泣血尸立,撫慰傷痍,移居陽邑城,以招集亡散。



 愍帝即位,拜大將軍、都督並州諸軍事,加散騎常侍、假節。琨上疏謝曰:



 陛下略臣大愆,錄臣小善,猥蒙天恩,光授殊寵,顯以蟬冕之榮,崇以
 上將之位。伏省詔書,五情飛越。



 臣聞晉文以郤縠為元帥而定霸功,高祖以韓信為大將而成王業,咸有敦詩閱禮之德,戎昭果毅之威,故能振豐功於荊南,拓洪基於河北。況臣凡陋,擬蹤前哲,俯懼折鼎,慮在覆餗。昔曹沫三北,而收功於柯盟;馮異垂翅,而奮翼於澠池,皆能因敗為成,以功補過。陛下宥過之恩已隆,而臣自新之善不立。臣雖不逮,預聞前訓,恭讓之節,臣猶庶幾。所以冒承寵命者,實欲沒身報國,輒死自效,要以致命寇場,盡其臣節。至於寵榮之施,非言辭所謝。又謁者史蘭、殿中中郎王春等繼至,奉詔,臣俯尋聖旨,伏紙飲淚。



 臣聞夷險流行,古今代有,靈厭皇德,曾未悔禍。蟻狄續毒於神州,夷裔肆虐於上國,七廟闕禋祀之饗,百官喪彞倫之序,梓宮淪辱,山陵未兆,率土永慕,思同考妣。陛下龍姿日茂,睿質彌光,升區宇於既頹,
 崇社稷於已替,四海之內,肇有上下,九服之萌,復睹典制。伏惟陛下蒙塵於外,越在秦郊,蒸嘗之敬在心,桑梓之思未克。臣備位歷年,才質駑下,丘山之釁已彰,毫厘之效未著。頃以時宜,權假位號,竟無殪戎之績,而有負乘之累,當肆刑書,以明黜陟。是以臣前表上聞,敢緣愚款,乞奉先朝之班,茍存偏師之職,赦其三敗之愆,必其一功之用,得騁志虜場,快意大逆,雖身膏野草,無恨黃墟。陛下偏恩過隆,曲蒙擢拔,遂授上將,位兼常伯,征討之務,得從事宜。拜命驚惶,五情戰悸,懼於隕越,以為朝羞。昔申胥不徇伯舉,而成公壻之勛;伍員不從城父,而濟入郢之庸。臣雖頑兇,無覬古人,其於被堅執銳,致身寇仇,所謂天地之施,群生莫謝不勝。受恩至深,謹拜表陳聞。



 及麴允敗劉曜,斬趙冉,琨又表曰:



 逆胡劉聰,敢率犬羊,馮陵輦轂,人神發憤,遐邇奮怒。伏省詔書,相國、南陽王保,太尉、涼州刺史軌,糾合二州,同恤王室,冠軍將軍允、護軍將軍綝,總齊六軍,戮力國難,王旅大捷,俘馘千計,旌旗首於晉路,金鼓振於河曲,崤函無虔劉之警,汧隴有安業之慶,斯誠宗廟社稷陛下神武之所致。含氣之
 類,莫不引領,況臣之心,能無踴躍。



 臣前表當與鮮卑猗盧克今年三月都會平陽,會匈羯石勒以三月三日徑掩薊城,大司馬、博陵公浚受其偽和,為勒所虜,勒勢轉盛,欲來襲臣。城塢駭懼,志在自守。又猗盧國內欲生奸謀,幸盧警慮,尋皆誅滅。遂使南北顧慮,用愆成舉,臣所以泣血宵吟,扼腕長嘆者也。勒據襄國,與臣隔山,寇騎朝發,夕及臣城,同惡相求,其徒實繁。自東北八州,勒滅其七,先朝所授,存者唯臣。是以勒朝夕謀慮,以圖臣為計,窺伺間隙,寇抄相尋,戎士不得解甲,百姓不得在野。天網雖張,靈澤未及,唯臣孑然與寇為伍。自守則稽聰之誅,進討則勒襲其後,進退唯谷,首尾狼狽。徒懷憤踴,力不從願,慚怖征營,痛心疾首,形留所在,神馳寇庭。秋穀既登,胡馬已肥,前鋒諸軍並有至者,臣當首啟戎行,身先士卒。臣與二虜,勢不並立,聰、勒不梟,臣無歸志,庶憑陛下威靈,使微意獲展,然後隕首謝國,沒而無恨。



 三年,帝遣兼大鴻臚趙廉持節拜琨為司空、都督并冀幽三州諸軍事。琨上表讓司空,受都督,剋期與猗盧討劉聰。尋猗盧父子相圖,盧及兄子根皆病死,部落四散。琨子遵
 先質於盧,眾皆附之。及是,遵與箕澹等帥盧眾三萬人,馬牛羊十萬,悉來歸琨,琨由是復振,率數百騎自平城撫納之。屬石勒攻樂平,太守韓據請救於琨,而琨自以士眾新合,欲因其銳以威勒。箕澹諫曰:「此雖晉人,久在荒裔,未習恩信,難以法御。今內收鮮卑之餘穀,外抄殘胡之牛羊,且閉關守險,務農息士,既服化感義,然後用之,則功可立也。」琨不從,悉發其眾,命澹領步騎二萬為前驅,琨自為後繼。勒先據險要,設伏以擊澹,大敗之,一軍皆沒,并土震駭。尋又炎旱,琨窮蹙不能復守。幽州刺史鮮卑段匹磾數遣信要琨,欲與同獎王室。琨由是率眾赴之,從
 飛狐人薊。匹磾見之,甚相崇重,與琨結婚,約為兄弟。



 是時西都不守,元帝稱制江左,琨乃令長史溫嶠勸進,於是河朔征鎮夷夏一百八十人連名上表,語在《元紀》。令報曰:「豺狼肆毒,薦覆社稷,億兆顒顒,延首罔繫。是以居於王位,以答天下,庶以剋復聖主,掃蕩讎恥,豈可猥當隆極,此孤之至誠著於遐邇者也。公受奕世之寵,極人臣之位,忠允義誠,精感天地。實賴遠謀,共濟艱難。南北迥邈,同契一致,萬里之外,心存咫尺。公其撫寧華戎,致罰醜類。動靜以聞。」



 建武元年,琨與匹磾期討石勒,匹磾推琨為大都督,臿血載書,檄諸方守,俱集襄國。琨、匹磾
 進屯固安,以俟眾軍。匹磾從弟末波納勒厚賂,獨不進,乃沮其計。琨、匹磾以勢弱而退。是歲,元帝轉琨為侍中、太尉,其餘如故,并贈名刀。琨答曰:「謹當躬自執佩,馘截二虜。」



 匹磾奔其兄喪,琨遣世子群送之,而末波率眾要擊匹磾而敗走之,群為末波所得。末波厚禮之,許以琨為幽州刺史,共結盟而襲匹磾,密遣使齎群書請琨為內應,而為匹磾邏騎所得。時琨別屯故征北府小城,不之知也。因來見匹磾,匹磾以群書示琨曰:「意亦不疑公,是以白公耳。」琨曰:「與公同盟,志獎王室,仰憑威力,庶雪國家之恥。若兒書密達,亦終不以一子之故負公忘義
 也。」匹磾雅重琨,初無害琨志,將聽還屯。其中弟叔軍好學有智謀,為匹磾所信,謂匹磾曰:「吾胡夷耳,所以能服晉人者,畏吾眾也。今我骨肉構禍,是其良圖之日,若有奉琨以起,吾族盡矣。」匹磾遂留琨。琨之庶長子遵懼誅,與琨左長史楊橋、並州治中如綏閉門自守。匹磾諭之不得,因縱兵攻之。琨將龍季猛迫於乏食,遂斬橋、綏而降。



 初,琨之去晉陽也,慮及危亡而大恥不雪,亦知夷狄難以義伏,冀輸寫至誠,僥倖萬一。每見將佐,發言慷慨,悲其道窮,欲率部曲列於賊壘。斯謀未果,竟為匹磾所拘。自知必死,神色怡如也。為五言詩贈其別駕盧諶曰:



 握中有懸璧,本是荊山球。惟彼太公望,昔是渭濱叟。鄧生何感激,千里來相求。白登幸曲逆,鴻門賴留侯。重耳憑五賢,小白相射鉤。能隆二伯主,安問黨與仇!中夜撫枕歎,想與數子遊。吾衰久矣夫,何其不夢周?誰云聖達節,知命故無憂。宣尼悲獲麟,西狩泣孔丘。功業未及建,夕陽忽西流。時哉不我與,去矣如雲浮。朱實隕勁風,繁英落素秋。狹路頌華蓋,駭駟摧雙輈。何意百煉剛,化為繞指柔。



 琨詩託意非常,攄暢幽憤,遠想張陳,感鴻門、白登之事,用以激諶。諶素無奇略,以常詞酬和,殊乖琨心,重以詩贈之,乃謂琨曰:「前篇帝王大志,非人臣所言矣。」



 然琨既忠於晉室,素有重望,被拘經月,遠近憤歎。匹磾所署代郡太守辟閭嵩,與琨所署鴈門太守王據、後將軍韓據連謀,密作攻具,欲以襲匹磾。而韓據女為匹磾兒妾,聞其謀而告之匹磾,於是執王據、辟閭嵩及其徒黨悉誅之。會王敦密使匹磾殺琨,匹磾又懼眾反己,遂稱有詔收琨。初,琨聞敦使到,謂其子曰:「處仲使來而不我告,是殺我也。死生有命,但恨仇恥不雪,無以下見二親耳。」因歔欷不能自勝。匹磾遂縊之,時年四十八。子姪四人俱被害。朝廷以匹磾尚彊,當為國討石勒,不舉琨哀。



 三年,琨故從事中郎盧諶、崔悅等上表理琨曰:



 臣聞
 經國之體,在於崇明典刑;立政之務,在於固慎關塞。況方岳之臣,殺生之柄,而可不正其枉直,以杜其姦邪哉!竊見故司空、廣武侯琨,在惠帝擾攘之際,值群后鼎沸之難,戮力皇家,義誠彌厲,躬統華夷,親受矢石,石超授首,呂朗面縛,社稷克寧,鑾輿反駕,奉迎之勳,琨實為隆,此琨效忠之一驗也。其後并州刺史、東贏公騰以晉川荒匱,移鎮臨漳,太原、西河盡徙三魏。琨受任并州,屬承其弊,到官之日,遺戶無幾,當易危之勢,處難濟之土,鳩集傷痍,撫和戎狄,數年之間,公私漸振。會京都失守,群逆縱逸,邊萌頓仆,茍懷宴安,咸以為并州之地四塞為
 困,且可閉關守險,畜資養徒,抗辭厲聲,忠亮奮發,以為天子沈辱而不隕身死節,情非所安,遂乃跋履山川,東西征討。屠各乘虛,晉陽沮潰,琨父母罹屠戮之殃,門族受殲夷之禍。向使琨從州人之心,為自守之計,則聖朝未必加誅,而族黨可以不喪。及猗盧敗亂,晉人歸奔,琨於平城納其初附。將軍箕澹又以為此雖晉人,久在荒裔,難以法整,不可便用。琨又讓之,義形於色。假從澹議,偷於茍存,則晏然於並土,必不亡身於燕薊也。琨自以備位方嶽,綱維不舉,無緣虛荷大任,坐居三司,是以陛下登阼,使引衍告遜,前後章表,具陳誠款。尋令從事中
 郎臣續澹以章綬節傳奉還本朝,與匹磾使榮邵期一時俱發。又匹磾以琨王室大臣,懼奪己威重,忌琨之形,漸彰於外。琨知其如此,慮不可久,欲遣妻息大小盡詣京城,以其門室一委陛下。有徵舉之會,則身充一卒;若匹磾縱凶慝,則妻息可免。具令臣澹密宣此旨,求詔敕路次,令相迎衛。會王成從平陽逃來,說南陽王保稱號隴右,士眾甚盛,當移關中。匹磾聞此,私懷顧望,留停榮邵,欲遣前兼鴻臚邊邈奉使詣保,懼澹獨南,言其此事,遂不許引路。丹誠赤心,卒不上達。匹磾兄眷喪亡,嗣子幼弱,欲因奔喪奪取其國。又自以欺國陵家,懷邪樂禍,
 恐父母宗黨不容其罪,是以卷甲櫜弓,陰圖作亂,欲害其從叔驎、從弟末波等,以取其國。匹磾親信密告驎、波,驎、波乃遣人距之,匹磾僅以身免。百姓謂匹磾已沒,皆憑向琨。若琨于時有害匹磾之情,則居然可擒,不復營於人力。自此之後,上下並離,匹磾遂欲盡勒胡晉,徙居上谷。琨深不然之,勸移厭次,南憑朝廷。匹磾不能納,反禍害父息四人,從兄二息同時并命。琨未遇害,知匹磾必有禍心,語臣等云:「受國厚恩,不能克報,雖才略不及,亦由遇此厄運。人誰不死,死生命也。唯恨下不能效節於一方,上不得歸誠於陛下。」辭旨慷慨,動於左右。匹磾
 既害琨,橫加誣謗,言琨欲窺神器,謀圖不軌。琨免述囂頑凶之思,又無信布懼誅之情,崎嶇亂亡之際,夾肩異類之間,而有如此之心哉!雖臧獲之愚,廝養之智,猶不為之,況在國士之列,忠節先著者乎!



 匹磾之害琨,稱陛下密詔。琨信有罪,陛下加誅,自當肆諸市朝,與眾棄之,不令殊俗之豎戮台輔之臣,亦已明矣。然則擅詔有罪,雖小必誅;矯制有功,雖大不論,正以興替之根咸在於此,開塞之由不可不閉故也。而匹磾無所顧忌,怙亂專殺,虛假王命,虐害鼎臣,辱諸夏之望,敗王室之法,是可忍也,孰不可忍!若聖朝猶加隱忍,未明大體,則不逞之
 人襲匹磾之跡,殺生自由,好惡任意,陛下將何以誅之哉!折衝厭難,唯存戰勝之將;除暴討亂,必須知略之臣。故古語云「山有猛獸,藜藿為之不採」,非虛言矣。自河以北,幽并以南,醜類有所顧憚者,唯琨而已。琨受害之後,群凶欣欣,莫不得意,鼓行中州,曾無纖介,此又華夷小大所以長歎者也。



 伏惟陛下睿聖之隆,中興之緒,方將平章典刑,以經序萬國。而琨受害非所,冤痛已甚,未聞朝廷有以甄論。昔壺關三老訟衛太子之罪,谷永、劉向辨陳湯之功,下足以明功罪之分,上足以悟聖主之懷。臣等祖考以來,世受殊遇,人侍翠幄,出簪彤管,弗克負
 荷,播越遐荒,與琨周旋,接事終始,是以仰慕三臣在昔之義,謹陳本末,冒以上聞,仰希聖朝曲賜哀察。



 太子中庶子溫嶠又上疏理之,帝乃下詔曰:「故太尉、廣武侯劉琨忠亮開濟,乃誠王家,不幸遭難,志節不遂,朕甚悼之。往以戎事,未加弔祭。其下幽州,便依舊弔祭。」贈侍中、太尉,謚曰愍。



 琨少負志氣,有縱橫之才,善交勝己,而頗浮誇。與范陽祖逖為友,聞逖被用,與親故書曰:「吾枕戈待旦,志梟逆虜,常恐祖生先吾著鞭。」其意氣相期如此。在晉陽,常為胡騎所圍數重,城中窘迫無計,琨乃乘月登樓清嘯,賊聞之,皆悽然長歎。中夜奏胡笳,賊又流涕歔
 欷,有懷土之切。向曉復吹之,賊並棄圍而走。子群嗣。



 群字公度,少拜廣武侯世子。隨父在晉陽,遭逢寇亂,數領偏軍征討。性清慎,有裁斷,得士類懽心。及琨為匹磾所害,琨從事中郎盧諶等率餘眾奉群依末波。溫嶠前後表稱:「姨弟劉群,內弟崔悅、盧諶等,皆在末波中,翹首南望。愚謂此等並有文思,於人之中少可愍惜。如蒙錄召,繼絕興亡,則陛下更生之恩,望古無二。」咸康二年,成帝詔徵群等,為末波兄弟愛其才,託以道險不遣。



 石季龍滅遼西,群及諶、悅同沒胡中,季龍皆優禮之,以群為中書令。至冉閔敗後,群遇害。時勒及季龍得公卿人士
 多殺之,其見擢用,終至大官者,唯有河東裴憲,渤海石璞,滎陽鄭系,潁川荀綽,北地傅暢及群、悅、諶等十餘人而已。



 輿字慶孫。雋朗有才局,與琨並尚書郭奕之甥,名著當時。京都為之語曰:「洛中奕奕,慶孫,越石。」辟宰府尚書郎。兄弟素侮孫秀,及趙王倫輔政,孫秀執權,並免其官。妹適倫世子荂,荂與秀不協,復以輿為散騎侍郎。齊王冏輔政,以輿為中書侍郎。東海王越、范陽王虓之舉兵也,以輿為潁川太守。及河間王顒檄劉喬討虓於許昌,矯詔曰:「潁川太守劉輿迫協范王虓,距逆詔命,多樹私
 黨,擅劫郡縣,合聚兵眾。輿兄弟昔因趙王婚親,擅弄權勢,凶狡無道,久應誅夷,以遇赦令,得全首領。小人不忌,為惡日滋,輒用茍晞為兗州,斷截王命。鎮南大將軍弘,平南將軍、彭城王釋,征東大將軍準,各勒所領,徑會許昌,與喬并力。今遣右將這張方為大都督,督建威將軍呂朗、陽平太守刁默,率步騎十萬,同會許昌,以除輿兄弟。敢有舉兵距違王命,誅及五族。能殺輿兄弟送首者,封三千戶縣侯,賜絹五千匹。」虓之敗,輿與之俱奔河北。虓既鎮鄴,以輿為征虜將軍、魏郡太守。



 虓薨,東海王越將召之,或曰:「輿猶膩也,近則污人。」及至,越疑而御之。輿
 密視天下兵簿及倉庫、牛馬、器械、水陸之形,皆默識之。是時軍國多事,每會議,自潘滔以下,莫知所對。輿既見越,應機辯畫,越傾膝酬接,即以為左長史。越既總錄,以輿為上佐,賓客滿筵,文案盈機,遠近書記日有數千,終日不倦,或以夜繼之,皆人人懽暢,莫不悅附。命議如流,酬對款備,時人服其能,比之陳遵。時稱越府有三才:潘滔大才,劉輿長才,裴邈清才。越誅繆播、王延等,皆輿謀也。延愛妾荊氏有音伎,延尚未殮,輿便娉之。未及迎,又為太傅從事中郎王俊所爭奪。御史中取丞傅宣劾奏,越不問輿,而免俊官。輿乃說越,遣琨鎮并州,為越北面之
 重。洛陽未敗,病指疽卒,時年四十七。追贈驃騎將軍。先有功封定襄侯,謚曰貞。子演嗣。



 演字始仁。初辟太尉掾,除尚書郎,以父憂去職。服闋,襲爵,太傅、東海王越引為主簿。遷太子中庶子,出為陽平太守。自洛奔琨,琨以為輔國將軍、魏郡太守。琨將討石勒,以演領勇士千人,行北中郎將、兗州刺史,鎮廩丘。演斬王桑,走趙固,得眾七千人。為石勒所攻,演距戰,勒退。元帝拜為都督、後將軍,假節。後為石季龍所圍,求救於邵續、段鴦,鴦騎救之,季龍走,隨鴦屯厭次,被害。



 弟胤為琨引兵,路逢烏桓賊,戰沒。胤弟挹初為太傅、東海王越
 掾,與琨俱被害。挹弟啟,啟弟述,與琨子群俱在末波中,後並入石季龍。啟為季龍尚書僕射,後歸國,穆帝拜為前將軍,加給事中。永和九年,隨中軍將軍殷浩北伐,為姚襄所敗,啟戰沒。述為季龍侍中,隨啟歸國,拜驍騎將軍。



 祖逖,字士稚,范陽遒人也。世吏二千石,為北州舊姓。父武,晉王掾、上谷太守。逖少孤,兄弟六人。兄該、納等並開爽有才幹。逖性豁蕩,不修儀檢,年十四五猶未知書,諸兄每憂之。然輕財好俠,慷慨有節尚,每至田舍,輒稱兄
 意,散穀帛以周貧乏,鄉黨宗族以是重之。後乃博覽書記,該涉古今,往來京師,見者謂逖有贊世才具。僑居陽平。年二十四,陽平辟察孝廉,司隸再辟舉秀才,皆不行。與司空劉琨俱為司州主簿,情好綢繆,共被同寢。中夜聞荒雞鳴,蹴琨覺曰:「此非惡聲也。」因起舞。逖、琨並有英氣,每語世事,或中宵起坐,相謂曰:「若四海鼎沸,豪傑並起,吾與足下當相避於中原耳。」



 辟齊王冏大司馬掾、長沙王乂驃騎祭酒,轉主簿,累遷太子中舍人、豫章王從事中郎。從惠帝北伐,王師敗績於蕩陰,遂退還洛。大駕西幸長安,關東諸侯范陽王虓、高密王略、平昌公模等
 競召之,皆不就。東海王越以逖為典兵參軍、濟陰太守,母喪不之官。及京師大亂,逖率親黨數百家避地淮泗,以所乘車馬載同行老疾,躬自徒步,藥物衣糧與眾共之,又多權略,是以少長咸宗之,推逖為行主。達泗口,元帝逆用為徐州刺史,尋征軍諮祭酒,居丹徒之京口。



 逖以社稷傾覆,常懷振復之志。賓客義徒皆暴傑勇士,逖遇之如子弟。時揚土大饑,此輩多為盜竊,攻剽富室,逖撫慰問之曰:「比復南塘一出不?」或為吏所繩,逖輒擁護救解之。談者以此少逖,然自若也。時帝方拓定江南,未遑北伐,逖進說曰:「晉室之亂,非上無道而下怨叛也。由
 籓王爭權,自相誅滅,遂使戎狄乘隙,毒流中原。今遺黎既被殘酷,人有奮擊之志。大王誠能發威命將,使若逖等為之統主,則郡國豪傑必因風向赴,沈弱之士欣於來蘇,庶幾國恥可雪,願大王圖之。」帝乃以逖為奮威將軍、豫州刺史,給千人稟,布三千匹,不給鎧仗,使自招募。仍將本流徙部曲百餘家渡江,中流擊楫而誓曰:「祖逖不能清中原而復濟者,有如大江!」辭色壯烈,眾皆慨歎。屯于江陰,起冶鑄兵器,得二千餘人而後進。



 初,北中郎將劉演距于石勒也,流人塢主張平、樊雅等在譙,演署平為豫州刺史,雅為譙郡太守。又有董瞻、于武、謝浮等
 十餘部,眾各數百,皆統屬平。逖誘浮使取平,浮譎平與會,遂斬以獻逖。帝嘉逖勳,使運糧給之,而道遠不至,軍中大飢。進據太丘。樊雅遣眾夜襲逖,遂入壘,拔戟大呼,直趣逖幕,軍土大亂。逖命左右距之,督護董昭與賊戰,走之。逖率眾追討,而張平餘眾助雅攻逖。蓬陂塢主陳川,自號寧朔將軍、陳留太守。逖遣使求救於川,川遣將李頭率眾援之,逖遂剋譙城。



 初,樊雅之據譙也,逖以力弱,求助於南中郎將王含,含遣桓宣領兵助逖。逖既剋譙,宣等乃去。石季龍聞而引眾圍譙,含又遣宣救逖,季龍聞宣至而退。宣遂留,助逖討諸屯塢未附者。



 李頭之
 討樊雅也,力戰有勳。逖時獲雅駿馬,頭甚欲之而不敢言,逖知其意,遂與之。頭感逖恩遇,每歎曰:「若得此人為主,吾死無恨。」川聞而怒,遂殺頭。頭親黨馮寵率其屬四百入歸于逖,川益怒,遣將魏碩掠豫州諸郡,大獲子女車馬。逖遣將軍衛策邀擊於谷水,盡獲所掠者,皆令歸本,軍無私焉。川大懼,遂以眾附石勒。逖率眾伐川,石季龍領兵五萬救川,逖設奇以擊之,季龍大敗,收兵掠豫州,徙陳川還襄國,留桃豹等守川故城,住西臺。逖遣將韓潛等鎮東臺。同一大城,賊從南門出入放牧,逖軍開東門,相守四旬。逖以布囊盛土如米狀,使千餘人運上
 臺,又令數人擔米,偽為疲極而息於道,賊果逐之,皆棄擔而走。賊既獲米,謂逖士眾豐飽,而胡戍飢久,益懼,無復膽氣。石勒將劉夜堂以驢千頭運糧以饋桃豹,逖遣韓潛、馮鐵等追擊於汴水,盡獲之。豹宵遁,退據東燕城,逖使潛進屯封丘以逼之。馮鐵據二臺,逖鎮雍丘,數遣軍要截石勒,勒屯戍漸蹙。候騎常獲濮陽人,逖厚待遣歸。咸感逖恩德,率鄉里五百家降逖。勒又遣精騎萬人距逖,復為逖所破,勒鎮戍歸附者甚多。時趙固、上官巳、李矩、郭默等各以詐力相攻擊,逖遣使和解之,示以禍福,遂受逖節度。逖愛人下士,雖疏交賤隸,皆恩禮遇之,
 由是黃河以南盡為晉土。河上堡固先有任子在胡者,皆聽兩屬,時遣游軍偽抄之,明其未附。諸塢主感戴,胡中有異謀,輒密以聞。前後剋獲,亦由此也。其有微功,賞不踰日。躬自儉約,勸督農桑,剋己務施,不畜資產,子弟耕耘,負擔樵薪,又收葬枯骨,為之祭醊,百姓感悅。嘗置酒大會,耆老中坐流涕曰:「吾等老矣!更得父母,死將何恨!」乃歌曰:「幸哉遺黎免俘虜,三辰既朗遇慈父,玄酒忘勞甘瓠脯,何以詠恩歌且舞。」其得人心如此。故劉琨與親故書,盛贊逖威德。詔進逖為鎮西將軍。



 石勒不敢窺兵河南,使成皋縣修逖母墓,因與逖書,求通使交市,逖
 不報書,而聽互市,收利十倍,於是公私豐贍,士馬日滋。方當推鋒越河,掃清冀朔,會朝廷將遣戴若思為都督,逖以若思是吳人,雖有才望,無弘致遠識,且已翦荊棘,收河南地,而若思雍容,一旦來統之,意甚怏怏。且聞王敦與劉隗等構隙,慮有內難,大功不遂。感激發病,乃致妻孥汝南大木山下。時中原士庶咸謂逖當進據武牢,而反置家險阨,或諫之,不納。逖雖內懷憂憤,而圖進取不輟,營繕武牢城,城北臨黃河,西接成皋,四望甚遠。逖恐南無堅壘,必為賊所襲,乃使從子汝南太守濟率汝陽太守張敞、新蔡內史周閎率眾築壘。未成,而逖病甚。
 先是,華譚、庾闡問術人戴洋,洋曰:「祖豫州九月當死。」初有妖星見于豫州之分,歷陽陳訓又謂人曰:「今年西北大將當死。」逖亦見星,曰:「為我矣!方平河北,而天欲殺我,此乃不祐國也。」俄卒於雍丘,時年五十六。豫州士女若喪考妣,譙梁百姓為之立祠。冊贈車騎將軍。王敦久懷逆亂,畏逖不敢發,至是始得肆意焉。尋以逖弟約代領其眾。約別有傳。逖兄納。



 納字士言,最有操行,能清言,文義可觀。性至孝,少孤貧,常自炊釁以養母,平北將軍王敦聞之,遺其二婢,辟為從事中郎。有戲之曰:「奴價倍婢。」納曰:「百里奚何必輕於
 五羖皮邪!」轉尚書三公郎,累遷太子中庶子。歷官多所駮正,有補於時。



 齊王冏建義,越王倫收冏弟北海王實及前前黃門郎弘農董祚弟艾,與冏俱起,皆將害之,納上疏救焉,並見宥。後為中護軍、太子詹事,封晉昌公。以洛下將亂,乃避地東南。元帝作相,引為軍諮祭酒。納好奕棋,王隱謂之曰:「禹惜寸陰,不聞數棋。」對曰:「我奕忘憂耳。」隱曰;「蓋聞古人遭逢,則以功達其道,若其不遇,則以言達其道。古必有之,今亦宜然。當晉未有書,而天下大亂,舊事蕩滅,君少長五都,遊臣四方,華裔成敗,皆當聞見,何不記述而有裁成?應仲遠作《風俗通》,崔子真作《政論》,
 蔡伯喈作《勸學篇》,史游作《急就章》,猶皆行於世,便成沒而不朽。僕雖無才,非志不立,故疾沒世而無聞焉,所以自彊不息也。況國史明乎得失之跡,俱取散悉,此可兼濟,何必圍棋然後忘憂也!」納喟然歎曰:「非不悅子之道,力不足耳。」乃言之於帝曰:「自古小國猶有史官,況於大府,安可不置。」因舉隱,稱「清純亮直,學思沈敏,五經、群史多所綜悉,且好學不倦,從善如流。若使修著一代之典,褒貶與奪,誠一時之俊也。」帝以問記室參軍鐘雅,雅曰:「納所舉雖有史才,而今未能立也。」事遂停。然史官之立,自納始也。



 初,弟約與逖同母,偏相親愛,納與約異母,頗
 有不平,乃密以啟帝,稱:「約懷陵上之性,抑而使之可也。今顯侍左右,假其權勢,將為亂階」。人謂納與約異母,忌其寵貴,乃露其表以示約,約憎納如仇,朝廷因此棄納。納既閑居,但清談、披閱文史而已。及約為逆,朝野歎納有鑒裁焉。溫嶠以納州里父黨,敬而拜之。嶠既為時用,盛言納有名理,除光祿大夫。



 納嘗問梅陶曰:「君鄉里立月旦評,何如?」陶曰:「善褒惡貶,則佳法也。」納曰:「未益。」時王隱在坐,因曰:「《尚書》稱『三載考績,三考黜陟幽明』,何得一月便行褒貶!」陶曰:「此官法也。月旦,私法也。」隱曰:「《易》稱『積善之家必有餘慶,積不善之家必有餘殃。』稱家者豈不
 是官?必須積久,善惡乃著,公私何異!古人有言,貞良而亡,先人之殃;酷烈而存,先人之勳。累世乃著,豈但一月!若必月旦,則顏回食埃,不免貪污;盜跖引少,則為清廉。朝種暮獲,善惡未定矣。」時梅陶及鐘雅數說餘事,納輒困之,因曰:「君汝潁之士,利如錐;我幽冀之士,鈍如槌。持我鈍槌,捶君利錐,皆當摧矣。」陶、雅並稱「有神錐,不可得槌」。納曰:「假有神錐,必有神槌。」雅無以對。卒於家。



 史臣曰:劉琨弱齡,本無異操,飛纓賈謐之館,借箸馬倫之幕,當于是日,實佻巧之徒歟!祖逖散穀周貧,聞雞暗舞,思中原之燎火,幸天步之多艱,原其素懷,抑為貪亂
 者矣。及金行中毀,乾維失統,三后流亡,遞縈居彘之禍,六戎橫噬,交肆長蛇之毒,於是素絲改色,跅弛易情,各運奇才,並騰英氣,遇時屯而感激,因世亂以驅馳,陳力危邦,犯疾風而表勁,勵其貞操,契寒松而立節,咸能自致三鉉,成名一時。古人有言曰:「世亂識忠良。」益斯之謂矣。天不祚晉,方啟戎心,越石區區,獨禦鯨鯢之銳,推心異類,竟終幽圄,痛哉!士稚葉迹中興,剋復九州之半,而災星告釁,笠轂徒招,惜矣!



 贊曰:越石才雄,臨危效忠,枕戈長息,投袂徼功,崎嶇汾晉,契闊獯戎。見欺段氏,于嗟道窮!祖生烈烈,夙懷奇節。
 扣楫中流,誓清兇孽。鄰醜景附,遺萌載悅。天妖是征,國恥奚雪!



\end{pinyinscope}