\article{列傳第四}

\begin{pinyinscope}
羊祜杜預
 \gezhu{
  杜錫}



 羊祜,字叔子,泰山南城人也。世吏二千石,至祜九世,並以清德聞。祖續,仕漢南陽太守。父衜,上黨太守。祜,蔡邕外孫,景獻皇后同產弟。祜年十二喪父,孝思過禮,事叔父耽甚謹。嘗遊汶水之濱,遇父老謂之曰:「孺子有好相,年未六十,必建大功於天下。」既而去,莫知所在。及長,博學能屬文,身長七尺三寸,美鬚眉,善談論。郡將夏侯威
 異之,以兄霸之子妻之。舉上計吏,州四辟從事、秀才,五府交命,皆不就。太原郭奕見之曰:「此今日之顏子也。」與王沈俱被曹爽辟。沈勸就徵,祜曰:「委質事人,復何容易。」及爽敗,沈以故吏免,因謂祜曰:「常識卿前語。」祜曰:「此非始慮所及。」其先識不伐如此。



 夏侯霸之降蜀也,姻親多告絕,祜獨安其室,恩禮有加焉。尋遭母憂,長兄發又卒,毀慕寢頓十餘年,以道素自居,恂恂若儒者。



 文帝為大將軍,辟祜,未就,公車徵拜中書侍郎,俄遷給事中、黃門郎。時高貴鄉公好屬文,在位者多獻詩賦,汝南和逌以忤意見斥,祜在其間,不得而親疏,有識尚焉。陳留王立,
 賜爵關中侯,邑百戶。以少帝不願為侍臣,求出補吏,徙秘書監。及五等建,封鉅平子,邑六百戶。鐘會有寵而忌,祜亦憚之。及會誅,拜相國從事中郎,與荀勖共掌機密。遷中領軍,悉統宿衛,入直殿中,執兵之耍,事兼內外。



 武帝受禪,以佐命之勳,進號中軍將軍,加散騎常侍,改封郡公,邑三千戶。固讓封不受,乃進本爵為侯,置郎中令,備九官之職,加夫人印綬。泰始初,詔曰:「夫總齊機衡,允釐六職,朝政之本也。祜執德清劭,忠亮純茂,經緯文武,謇謇正直,雖處腹心之任,而不總樞機之重,非垂拱無為委任責成之意也。其以祜為尚書右僕射、衛將軍,給
 本營兵。」時王佑、賈充、裴秀皆前朝名望,祜每讓,不處其右。



 帝將有滅吳之志,以祜為都督荊州諸軍事、假節,散騎常侍、衛將軍如故。祜率營兵出鎮南夏,開設庠序,綏懷遠近,甚得江漢之心。與吳人開布大信,降者欲去皆聽之。時長吏喪官,後人惡之,多毀壞舊府,祜以死生有命,非由居室,書下征鎮,普加禁斷。吳石城守去襄陽七百餘里,每為邊害,祜患之,竟以詭計令吳罷守。於是戍邏減半,分以墾田八百餘頃,大獲其利。祜之始至也,軍無百日之糧,及至季年,有十年之積。詔罷江北都督,置南中郎將,以所統諸軍在漢東江夏者皆以益祜。在軍
 常輕裘緩帶,身不被甲,鈴閤之下,侍衛者不過十數人,而頗以畋漁廢政。嘗欲夜出,軍司徐胤執棨當營門曰:「將軍都督萬里,安可輕脫!將軍之安危,亦國家之安危也。胤今日若死,此門乃開耳。」祜改容謝之,此後稀出矣。



 後加車騎將軍,開府如三司之儀。祜上表固讓曰:「臣伏聞恩詔,拔臣使同台司。臣自出身以來,適十數年,受任外內,每極顯重之任。常以智力不可頓進,恩寵不可久謬,夙夜戰悚,以榮為憂。臣聞古人之言,德未為人所服而受高爵,則使才臣不進;功未為人所歸而荷厚祿,則使勞臣不勸。今臣身託外戚,事連運會,誡在過寵,不患
 見遺。而猥降發中之詔,加非次之榮。臣有何功可以堪之,何心可以安之。身辱高位,傾覆尋至,願守先人弊廬,豈可得哉!違命誠忤天威,曲從即復若此。蓋聞古人申於見知,大臣之節,不可則止。臣雖小人,敢緣所蒙,念存斯義。今天下自服化以來,方漸八年,雖側席求賢,不遺幽賤,然臣不爾推有德,達有功,使聖聽知勝臣者多,未達者不少。假令有遺德於版築之下,有隱才於屠釣之間,而朝議用臣不以為非,臣處之不以為愧,所失豈不大哉!臣忝竊雖久,未若今日兼文武之極寵,等宰輔之高位也。且臣雖所見者狹,據今光祿大夫李憙執節高
 亮,在公正色;光祿大夫魯芝潔身寡欲,和而不同;光祿大夫李胤清亮簡素,立身在朝,皆服事華髮,以禮終始。雖歷位外內之寵,不異寒賤之家,而猶未蒙此選,臣更越之,何以塞天下之望,少益日月!是以誓心守節,無茍進之志。今道路行通,方隅多事,乞留前恩,使臣得速還屯。不爾留連,必於外虞有闕。匹夫之志,有不可奪。」不聽。



 及還鎮,吳西陵督步闡舉城來降。吳將陸抗攻之甚急,詔祜迎闡。祜率兵五萬出江陵,遣荊州刺史楊肇攻抗,不剋,闡竟為抗所擒。有司奏:「祜所統八萬餘人,賊眾不過三萬。祜頓兵江陵,使賊備得設。乃遣楊肇偏軍入險,
 兵少糧懸,軍人挫衄。背違詔命,無大臣節。可免官,以侯就第。」竟坐貶為平南將軍,而免楊肇為庶人。



 祜以孟獻營武牢而鄭人懼,晏弱城東陽而萊子服,乃進據險耍,開建五城,收膏腴之地,奪吳人之資,石城以西,盡為晉有。自是前後降者不絕,乃增修德信,以懷柔初附,慨然有吞並之心。每與吳人交兵,剋日方戰,不為掩襲之計。將帥有欲進譎詐之策者,輒飲以醇酒,使不得言。人有略吳二兒為俘者,祜遣送還其家。後吳將夏詳、邵顗等來降,二兒之父亦率其屬與俱。吳將陳尚、潘景來寇,祜追斬之,美其死節而厚加殯斂。景、尚子弟迎喪,祜以禮
 遣還。吳將鄧香掠夏口,祜募生縛香,既至,宥之。香感其恩甚,率部曲而降。祜出軍行吳境,刈穀為糧,皆計所侵,送絹償之。每會眾江沔游獵,常止晉地。若禽獸先為吳人所傷而為晉兵所得者,皆封還之。於是吳人翕然悅服,稱為羊公,不之名也。



 祜與陸抗相對,使命交通,抗稱祜之德量,雖樂毅、諸葛孔明不能過也。抗嘗病,祜饋之藥,抗服之無疑心。人多諫抗,抗曰:「羊祜豈鴆人者!」時談以為華元、子反復見於今日。抗每告其戍曰:「彼專為德,我專為暴,是不戰而自服也。各保分界而已,無求細利。」孫皓聞二境交和,以詰抗。抗曰:「一邑一鄉,不可以無信
 義,況大國乎!臣不如此,正是彰其德,於祜無傷也。」



 祜貞愨無私,疾惡邪佞,旬勖、馮紞之徒甚忌之。從甥王衍嘗詣祜陳事,辭甚俊辨,祜不然之,衍拂衣而起。祜顧謂賓客曰:「王夷甫方以盛名處大位,然敗俗傷化,必此人也。」步闡之役,祜以軍法將斬王戎,故戎、衍並憾之,每言論多毀祜。時人為之語曰:「二王當國,羊公無德。」



 咸寧初,除征南大將軍、開府儀同三司,得專辟召。初,祐以伐吳必藉上流之勢。又時吳有童謠曰:「阿童復阿童,銜刀浮渡江。不畏岸上獸,但畏水中龍。」祜聞之曰:「此必水軍有功,但當思應其名者耳。」會益州刺史王浚徵為大司農,祜
 知其可任,浚又小字阿童,因表留浚監益州諸軍事,加龍驤將軍,密令修舟楫,為順流之計。



 祜繕甲訓卒,廣為戎備。至是上疏曰:「先帝順天應時,西平巴蜀,南和吳會,海內得以休息,兆庶有樂安之心。而吳復背信,使邊事更興。夫期運雖天所授,而功業必由人而成,不一大舉掃滅,則眾役無時得安。亦所以隆先帝之勳,成無為之化也。故堯有丹水之伐,舜有三苗之徵,咸以寧靜宇宙,戢兵和眾者也。蜀平之時,天下皆謂吳當并亡,自此來十三年,是謂一周,平定之期復在今日矣。議者常言吳楚有道後服,無禮先彊,此乃謂侯之時耳。當今一統,不
 得與古同諭。夫適道之論,皆未應權,是故謀之雖多,而決之欲獨。凡以險阻得存者,謂所敵者同,力足自固。茍其輕重不齊,彊弱異勢,則智士不能謀,而險阻不可保也。蜀之為國,非不險也,高山尋雲霓,深谷肆無景,束馬懸車,然後得濟,皆言一夫荷戟,千人莫當。及進兵之日,曾無籓籬之限,斬將搴旗,伏尸數萬,乘勝席卷,徑至成都,漢中諸城,皆鳥棲而不敢出。非皆無戰心,誠力不足相抗。至劉禪降服,諸營堡者索然俱散。今江淮之難,不過劍閣;山川之險,不過岷漢;孫皓之暴,侈於劉禪;吳人之困,甚於巴蜀。而大晉兵眾,多於前世;資儲器械,盛於
 往時;今不於此平吳,而更阻兵相守,征夫苦役,日尋干戈,經歷盛衰,不可長久,宜當時定,以一四海。今若引梁益之兵水陸俱下,荊楚之眾進臨江陵,平南、豫州,直指夏口,徐、揚、青、兗並向秣陵,鼓旆以疑之,多方以誤之,以一隅之吳,當天下之眾,勢分形散,所備皆急,巴漢奇兵出其空虛,一處傾壞,則上下震蕩。吳緣江為國,無有內外,東西數千里,以籓籬自持,所敵者大,無有寧息。孫皓孫恣情任意,與下多忌,名臣重將不復自信,是以孫秀之徒皆畏逼而至。將疑於朝,士困於野,無有保世之計,一定之心。平常之日,猶懷去就,兵臨之際,必有應者,終不
 能齊力致死,已可知也。其俗急速,不能持久,弓弩戟盾不如中國,唯有水戰是其所便。一入其境,則長江非復所固,還保城池,則去長入短。而官軍懸進,人有致節之志,吳人戰於其內,有憑城之心。如此,軍不踰時,剋可必矣。」帝深納之。



 會秦涼屢敗,祜復表曰:「吳平則胡自定,但當速濟大功耳。」而議者多不同,祜嘆曰:「天下不如意,恒十居七八,故有當斷不斷。天與不取,豈非更事者恨於後時哉!」



 其後,詔以泰山之南武陽、牟、南城、梁父、平陽五縣為南城郡,封祜為南城侯,置相,與郡公同。祜讓曰:「昔張良請受留萬戶,漢祖不奪其志。臣受鉅平於先帝,敢
 辱重爵,以速官謗!」固執不拜,帝許之。祜每被登進,常守沖退,至心素著,故特見申於分列之外。是以名德遠播,朝野具瞻,搢紳僉議,當居台輔。帝方有兼并之志,仗祜以東南之任,故寢之。祜歷職二朝,任典樞要,政事損益,皆諮訪焉,勢利之求,無所關與。其嘉謀讜議,皆焚其草,故世莫聞。凡所進達,人皆不知所由。或謂祜慎密太過者,祜曰:「是何言歟!夫入則造膝,出則詭辭,君臣不密之誡,吾惟懼其不及。不能舉賢取異,豈得不愧知人之難哉!且拜爵公朝,謝恩私門,吾所不取。」



 祜女夫嘗勸祜「有所營置,令有歸戴者,可不美乎?」祜默然不應,退告諸子
 曰:「此可謂知其一不知其二。人臣樹私則背公,是大惑也。汝宜識吾此意。」嘗與從弟琇書曰:「既定邊事,當角巾東路,歸故里,為容棺之墟。以白士而居重位,何能不以盛滿受責乎!疏廣是吾師也。」



 祜樂山水,每風景,必造峴山,置酒言詠,終日不倦。嘗慨然歎息,顧謂從事中郎鄒湛等曰:「自有宇宙,便有此山。由來賢達勝士,登此遠望,如我與卿者多矣!皆湮滅無聞,使人悲傷。如百歲後有知,魂魄猶應登此也。」湛曰:「公德冠四海,道嗣前哲,令聞令望,必與此山俱傳。至若湛輩,乃當如公言耳。」



 祜當討吳賊功,將進爵土,乞以賜舅子蔡襲。詔封襲關內侯,
 邑三百戶。



 會吳人寇弋陽、江夏,略戶口,詔遣侍臣移書詰祐不追討之意,并欲移州復舊之宜。祜曰:「江夏去襄陽八百里,比知賊問,賊去亦已經日矣。步軍方往,安能救之哉!勞師以免責,恐非事宜也。昔魏武帝置都督,類皆與州相近,以兵勢好合惡離。疆埸之間,一彼一此,慎守而已,古之善教也。若輒徙州,賊出無常,亦未知州之所宜據也。」使者不能詰。



 祜寢疾,求入朝。既至洛陽,會景獻宮車在殯,哀慟至篤。中詔申諭,扶疾引見,命乘輦入殿,無下拜,甚見優禮。及侍坐,面陳伐吳之計。帝以其病,不宜常入,遣中書令張華問其籌策。祜曰:「今主上有禪
 代之美,而功德未著。吳人虐政已甚,可不戰而剋。混一六合,以興文教,則主齊堯舜,臣同稷契,為百代之盛軌。如舍之,若孫皓不幸而沒,吳人更立令主,雖百萬之眾,長江未可而越也,將為後患乎!」華深贊成其計。祜謂華曰:「成吾志者,子也。」帝欲使祜臥護諸將,祜曰:「取吳不必須臣自行,但既平之後,當勞聖慮耳。功名之際,臣所不敢居。若事了,當有所付授,願審擇其人。」



 疾漸篤,乃舉杜預自代。尋卒,時年五十八。帝素服哭之,甚哀。是日大寒,帝涕淚霑鬚鬢,皆為冰焉。南州人征市日聞祜喪,莫不號慟,罷市,巷哭者聲相接。吳守邊將士亦為之泣。其仁
 千所感如此。賜以東園祕器,朝服一襲,錢三十萬,布百匹。詔曰:「征南大將軍南城侯祜,蹈德沖素,思心清遠。始在內職,值登大命,乃心篤誠,左右王事,入綜機密,出統方岳。當終顯烈,永輔朕躬,而奄忽殂隕,悼之傷懷。其追贈侍中、太傅,持節如故。」



 祜立身清儉,被服率素,祿俸所資,皆以贍給九族,賞賜軍士,家無餘財。遺令不得以南城侯印入柩。從弟琇等述祜素志,求葬於先人墓次。帝不許,賜去城十里外近陵葬地一頃,謚曰成。祜喪既引,帝於大司馬門南臨送。祜甥齊王攸表祜妻不以侯斂之意,帝乃詔曰:「祜固讓歷年,志不可奪。身沒讓存,遺操益
 厲,此夷叔所以稱賢,季子所以全節也。今聽復本封,以彰高美。」



 初,文帝崩,祜謂傅玄曰:「三年之喪,雖貴遂服,自天子達;而漢文除之,毀禮傷義,常以歎息。今主上天縱至孝,有曾閔之性,雖奪其服,實行喪禮。喪禮實行,除服何為邪!若因此革漢魏之薄,而興先王之法,以敦風俗,垂美百代,不亦善乎!」玄曰:「漢文以末世淺薄,不能行國君之喪,故因而除之。除之數百年,一旦復古,難行也。」祜曰:「不能使天下如禮,且使主上遂服,不猶善乎!」玄曰:「主上不除而天下除,此為但有有父子,無復君臣,三綱之道虧矣。」祜乃止。



 祜所著文章及為《老子傳》並行於世。襄陽
 百姓於峴山祜平生游憩之所建碑立廟,歲時饗祭焉。望其碑者莫不流涕,杜預因名為墮淚碑。荊州人為祜諱名,屋室皆以門為稱,改戶曹為辭曹焉。



 祜開府累年,謙讓不辟士,始有所命,會卒,不得除署。故參佐劉儈、趙寅、劉彌、孫勃等箋詣預曰:「昔以謬選,忝備官屬,各得與前征南大將軍祜參同庶事。祜執德沖虛,操尚清遠,德高而體卑,位優而行恭。前膺顯命,來撫南夏,既有三司之儀,復加大將軍之號。雖居其位,不行其制。至今海內渴佇,群俊望風。涉其門者,貪夫反廉,懦夫立志,雖夷惠之操,無以尚也。自鎮此境,政化被乎江漢,潛謀遠計,闢
 國開疆,諸所規摹,皆有軌量。志存公家,以死勤事,始辟四掾,未至而隕。夫舉賢報國,台輔之遠任也;搜揚側陋,亦台輔之宿心也;中道而廢,亦台輔之私恨也。履謙積稔,晚節不遂,此遠近所以為之感痛者也。昔召伯所憩,愛流甘棠;宣子所游,封殖其樹。夫思其人,尚及其樹,況生存所辟之士,便當隨例放棄者乎!乞蒙列上,得依已至掾屬。」預表曰:「祜雖開府而不備僚屬,引謙之至,宜見顯明。及扶疾辟士,未到而沒,家無胤嗣,官無命士,此方之望,隱憂載懷。夫篤終追遠,人德歸厚,漢祖不惜四千戶之封,以慰趙子弟心。請議之。」詔不許。



 祜卒二歲而吳
 平,群臣上壽,帝執爵流涕曰:「此羊太傅之功也。」因以克定之功,策告祜廟,仍依蕭何故事,封其夫人。策曰:「皇帝使謁者杜宏告故侍中、太傅鉅平成侯祜:昔吳為不恭,負險稱號,郊境不闢,多歷年所。祜受任南夏,思靜其難,外揚王化,內經廟略,著德推誠,江漢歸心,舉有成資,謀有全策。昊天不弔,所志不卒,朕用悼恨于厥心。乃班命群帥,致天之討,兵不踰時,一征而滅,疇昔之規,若合符契。夫賞不失勞,國有彞典,宜增啟土宇,以崇前命,而重違公高讓之素。今封夫人夏侯氏萬歲鄉君,食邑五千戶,又賜帛萬匹,穀萬斛。」



 祜年五歲,時令乳母取所弄金
 環。乳母曰:「汝先無此物。」祜即詣鄰人李氏東垣桑樹中探得之。主人驚曰:「此吾亡兒所失物也,云何持去!」乳母具言之,李氏悲惋。時人異之,謂李氏子則祜之前身也。又有善相墓者,言祜祖墓所有帝王氣,若鑿之則無後,祜遂鑿之。相者見曰「猶出折臂三公」,而祜竟墮馬折臂,位至公而無子。



 帝以祜兄子暨為嗣,暨以父沒不得為人後。帝又令暨弟伊為祜後,又不奉詔。帝怒,並收免之。太康二年,以伊弟篇為鉅平侯,奉祜嗣。篇歷官清慎,有私牛於官舍產犢,及遷而留之,位至散騎常侍,早卒。



 孝武太元中,封祜兄玄孫之子法興為鉅平侯,邑五千戶。
 以桓玄黨誅,國除。尚書祠部郎荀伯子上表訟之曰:「臣聞咎繇亡嗣,臧文以為深嘆;伯氏奪邑,管仲所以稱仁。功高可百世不泯,濫賞無得崇朝。故太傅、鉅平侯羊祜明德通賢,國之宗主,勳參佐命,功成平吳,而後嗣闕然,烝嘗莫寄。漢以蕭何元功,故絕世輒繼,愚謂鉅平封宜同酂國。故太尉廣陵公準黨翼賊倫,禍加淮南,因逆為利,竊饗大邦。值西朝政刑失裁,中興因而不奪。今王道維新,豈可不大判臧否,謂廣陵國宜在削除。故太保衛瓘本爵菑陽縣公,既被橫害,乃進茅土,始贈蘭陵,又轉江夏。中朝名臣,多非理終,瓘功德無殊,而獨受偏賞,謂
 宜罷其郡封,復邑菑陽,則與奪有倫,善惡分矣。」竟寢不報。



 祜前母,孔融女,生兄發,官至都督淮北護軍。初,發與祜同母兄承俱得病,祜母度不能兩存,乃專心養發,故得濟,而承竟死。



 發長子倫,高陽相。倫弟暨,陽平太守。暨弟伊,初為車騎賈充掾,後歷平南將軍、都督江北諸軍事,鎮宛,為張昌所殺,追贈鎮南將軍。祜伯父秘,官至京兆太守。子祉,魏郡太守。秘孫亮,字長玄,有才能,多計數。與之交者,必偽盡款誠,人皆謂得其心,而殊非其實也。初為太傅楊駿參軍,時京兆多盜竊。駿欲更重其法,盜百錢加大辟,請官屬會議,亮曰:「昔楚江乙母失布,以為
 盜由令尹。公若無欲,盜宜自止,何重法為?」駿慚而止。累轉大鴻臚。時惠帝在長安,亮與關東連謀,內不自安,奔於並州,為劉元海所害。亮弟陶,為徐州刺史。



 杜預,字元凱,京兆杜陵人也。祖畿,魏尚書僕射。父恕,幽州刺史。預博學多通,明於興廢之道,常言:「德不可以企及,立功立言可庶幾也。」初,其父與宣帝不相能,遂以幽死,故預久不得調。文帝嗣立,預尚帝妹高陸公主,起家拜尚書郎,襲祖爵豐樂亭侯。在職四年,轉參相府軍事。鐘會伐蜀,以預為鎮西長史。及會反,僚佐並遇害,唯預
 以智獲免,增邑千一百五十戶。



 與車騎將軍賈充等定律令,既成,預為之注解,乃奏之曰:「法者,蓋繩墨之斷例,非窮理盡性之書也。故文約而例直,聽省而禁簡。例直易見,禁簡難犯。易見則人知所避,難犯則幾於刑厝。刑之本在於簡直,故必審名分。審名分者,必忍小理。古之刑書,銘之鐘鼎,鑄之金石,所以遠塞異端,使無淫巧也。今所注皆綱羅法意,格之以名分。使用之者執名例以審趣舍,伸繩墨之直,去析薪之理也。」詔班于天下。



 泰始中,守河南尹。預以京師王化之始,自近及遠,凡所施論,務崇大體。受詔為黜陟之課,其略曰;「臣聞上古之政,
 因循自然,虛己委誠,而信順之道應,神感心通,而天下之理得。逮至淳樸漸散,彰美顯惡,設官分職,以頒爵祿,弘宣六典,以詳考察。然猶倚明哲之輔,建忠貞之司,使名不得越功而獨美,功不得後名而獨隱,皆疇咨博詢,敷納以言。及至末世,不能紀遠而求於密微,疑諸心而信耳目,疑耳目而信簡書。簡書愈繁,官方愈偽,法令滋章,巧飾彌多。昔漢之刺史,亦歲終奏事,不制算課,而清濁粗舉。魏氏考課,即京房之遺意,其文可謂至密。然由於累細以違其體,故歷代不能通也。豈若申唐堯之舊,去密就簡,則簡而易從也。夫宣盡物理,神而明之,存乎
 其人。去人而任法,則以傷理。今科舉優劣,莫若委任達官,各考所統。在官一年以後,每歲言優者一人為上第,劣者一人為下第,因計偕以名聞。如此六載,主者總集採案,其六歲處優舉者超用之,六歲處劣舉者奏免之,其優多劣少者敘用之,劣多優少者左遷之。今考課之品,所對不鈞,誠有難易。若以難取優,以易而否,主者固當準量輕重,微加降殺,不足復曲以法盡也。《己丑詔書》以考課難成,聽通薦例。薦例之理,即亦取於風聲。六年頓薦,黜陟無漸,又非古者三考之意也。今每歲一考,則積優以成陟,累劣以取黜。以士君子之心相處,未有官
 故六年六黜清能,六進否劣者也。監司將亦隨而彈之。若令上下公相容過,此為清議大頹,亦無取於黜陟也。」



 司隸校尉石鑒以宿憾奏預,免職。時虜寇隴石,以預為安西軍司,給兵三百人,騎百匹。到長安,更除秦州刺史,領東羌校尉、輕車將軍、假節。屬虜兵彊盛,石鑒時為安西將軍,使預出兵擊之。預以虜乘勝馬肥,而官軍懸乏,宜并力大運,須春進討,陳五不可、四不須。鑒大怒,復奏預擅飾城門官舍,稽乏軍興,遣御史檻車徵詣廷尉。以預尚主,在八議,以侯贖論,。其後隴右之事卒如預策。



 是時朝廷皆以預明於籌略,會匈奴帥劉猛舉兵反,自并
 州西及河東、平陽,詔預以散侯定計省闥,俄拜度支尚書。預乃奏立藉田,建安邊,論處軍國之要。又作人排新器,興常平倉,定穀價,較鹽運,制課調,內以利國外以救邊者五十餘條,皆納焉。石鑒自軍還,論功不實,為預所糾,遂相仇恨,言論喧譁,並坐免官,以侯兼本職。數年,復拜度支尚書。



 元皇后梓宮將遷於峻陽陵。舊制,既葬,帝及群臣即吉。尚書奏,皇太子亦宜釋服。預議「皇太子宜復古典,以諒闇終制」,從之。



 預以時歷差舛,不應晷度,奏上《二元乾度歷》,行於世。預又以孟津渡險,有覆沒之患,請建河橋于富平津。議者以為殷周所都,歷聖賢而不
 作者,必不可立故也。預曰:「『造舟為梁』,則河橋之謂也。」及橋成,帝從百僚臨會,舉觴屬預曰:「非君,此橋不立也。」對曰:「非陛下之明,臣亦不得施其微巧。」周廟欹器,至漢東京猶在御坐。漢末喪亂,不復存,形制遂絕。預創意造成,奏上之,帝甚嘉歎焉。咸寧四年秋,大霖雨,蝗蟲起。預上疏多陳農要,事在《食貨志》。預在內七年,損益萬機,不可勝數,朝野稱美,號曰「杜武庫」,言其無所不有也。



 時帝密有滅吳之計,而朝議多違,唯預、羊祜、張華與帝意合。祜病,舉預自代,因以本官假節行平東將軍,領征南軍司。及祜卒,拜鎮南大將軍、都督荊州諸軍事,給追鋒車,第
 二駙馬。預既至鎮,繕甲兵,耀威武,乃簡精銳,襲吳西陵督張政,大破之,以功增封三百六十五戶。政,吳之名將也,據要害之地,恥以無備取敗,不以所喪之實告于孫皓。預欲間吳邊將,乃表還其所獲之眾於皓。皓果召政,遣武昌監劉憲代之。故大軍臨至,使其將帥移易,以成傾蕩之勢。



 預處分既定,乃啟請伐吳之期。帝報待明年方欲大舉,預表陳至計曰:「自閏月以來,賊但敕嚴,下無兵上。以理勢推之,賊之窮計,力不兩完,必先護上流,勤保夏口以東,以延視息,無緣多兵西上,空其國都。而陛下過聽,便用委棄大計,縱敵患生。此誠國之遠圖,使舉
 而有敗,勿舉可也。事為之制,務從完牢。若或有成,則開太平之基;不成,不過費損日月之間,何惜而不一試之!若當須後年,天時人事不得如常,臣恐其更難也。陛下宿議,分命臣等隨界分進,其所禁持,東西同符,萬安之舉,未有傾敗之慮。臣心實了,不敢以曖昧之見自取後累。惟陛下察之。」預旬月之中又上表曰:「羊祜與朝臣多不同,不先博畫而密與陛下共施此計,故益令多異。凡事當以利害相較,今此舉十有八九利,其一二止於無功耳。其言破敗之形亦不可得,直是計不出已,功不在身,各恥其前言,故守之也。自頃朝廷事無大小,異意鋒
 起,雖人心不同,亦由恃恩不慮後難,故輕相同異也。昔漢宣帝議趙充國所上,事效之後,詰責諸議者,皆叩頭而謝,以塞異端也。自秋已來,討賊之形頗露。若今中止,孫皓怖而生計,或徙都武昌,更完修江南諸城,遠其居人,城不可攻,野無所掠,積大船於夏口,則明年之計或無所及。」時帝與中書令張華圍棋,而預表適至。華推枰斂手曰:「陛下聖明神武,朝野清晏,國富兵彊,號令如一,吳主荒淫驕虐,誅殺賢能,當今討之,可不勞而定。」帝乃許之。



 預以太康元年正月,陳兵于江陵,遣參軍樊顯、尹林、鄧圭、襄陽太守周奇等率眾循江西上,授以節度,旬
 日之間,累剋城邑,皆如預策焉。又遣牙門管定、周旨、伍巢等率奇兵八百,泛舟夜渡,以襲樂鄉,多張旗幟,起火巴山,出於要害之地,以奪賊心。吳都督孫歆震恐,與伍延書曰:「北來諸軍,乃飛渡江也。」吳之男女降者者萬餘口,旨、巢等伏兵樂鄉城外。歆遣軍出距王浚,大敗而還。旨等發伏兵,隨歆軍而入,歆不覺,直至帳下,虜歆而還。故軍中為之謠曰:「以計代戰一當萬。」於是進逼江陵。吳督將伍延偽請降而列兵登陴,預攻剋之。既平上流,於是沅湘以南,至于交廣,吳之州郡皆望風歸命,奉送印綬,預仗節稱詔而綏撫之。凡所斬及生獲吳都督、監軍十
 四,牙門、郡守百二十餘人。又因兵威,徙將士屯戍之家以實江北,南郡故地各樹之長吏,荊土肅然,吳人赴者如歸矣。



 王浚先列上得孫歆頭,預後生送歆,洛中以為大笑。時眾軍會議,或曰:「百年之寇,未可盡剋。今向暑,水潦方降,疾疫將起,宜俟來冬,更為大舉。」預曰:「昔樂毅藉濟西一戰以并彊齊,今兵威已振,譬如破竹,數節之後,皆迎刃而解,無復著手處也。」遂指授群帥,徑造秣陵。所過城邑,莫不束手。議者乃以書謝之。



 孫皓既平,振旅凱入,以功進爵當陽縣侯,增邑并前九千六百戶,封子耽為亭侯,千戶,賜絹八千匹。



 初,攻江陵,吳人知預病癭,憚
 其智計,以瓠繫狗頸示之,每大樹似癭,輒斫使白,題曰:「杜預頸。」及城平,盡捕殺之。



 預既還鎮,累陳家世吏職,武非其功,請退。不許。



 預以天下雖安,忘戰必危,勤於講武,修立泮宮,江漢懷德,化被萬里。攻破山夷,錯置屯營,分據要害之地,以固維持之勢。又修邵信臣遺跡,激用JC淯諸水以浸原田萬餘頃,分疆刊石,使有定分,公私同利。眾庶賴之,號曰「杜父」。舊水道唯沔漢達江陵千數百里,北無通路。又巴丘湖,沅湘之會,表裏山川,實為險固,荊蠻之所恃也。預乃開楊口,起夏水達巴陵千餘里,內瀉長江之險,外通零桂之漕。南土歌之曰:「後世無叛由
 杜翁,孰識智名與勇功。」預公家之事,知無不為。凡所興造,必考度始終,鮮有敗事。或譏其意碎者,預曰:「禹稷之功,期於濟世,所庶幾也。」



 預好為後世名,常言「高岸為谷,深谷為陵」,刻石為二碑,紀其勳績,一沈萬山之下,一立峴山之上,曰:「焉知此後不為陵谷乎!」



 預身不跨馬,射不穿札,而每任大事,輒居將率之列。結交接物,恭而有禮,問無所隱,誨人不倦,敏於事而慎於言。既立功之後,從容無事,乃耽思經籍,為《春秋左氏經傳集解》。又參考眾家譜第,謂之《釋例》。又作《盟會圖》、《春秋長歷》,備成一家之學,比老乃成。又撰《女記贊》。當時論者謂預文義質直,世
 人未之重,唯祕書監摯虞賞之,曰:「左丘明本為《春秋》作傳,而《左傳》遂自孤行,《釋例》本為《傳》設,而所發明何但《左傳》,故亦孤行。」時王濟解相馬,又甚愛之,而和嶠頗聚斂,預常稱「濟有馬癖,嶠有錢癖」。武帝聞之,謂預曰:「卿有何癖?」對曰:「臣有《左傳》癖。」



 預在鎮,數餉遺洛中貴要。或問其故,預曰:「吾但恐為害,不求益也。」



 預初在荊州,因宴集,醉臥齋中。外人聞嘔吐聲,竊窺於戶,止見一大蛇垂頭而吐。聞者異之。其後徵為司隸校尉,加位特進,行次鄧縣而卒,時年六十三。帝甚嗟悼,追贈征南大將軍、開府儀同三司,謚曰成。預先為遺令曰:「古不合葬,明於終始之
 理,同於無有也。中古聖人改而合之,蓋以別合無在,更緣生以示教也。自此以來,大人君子或合或否,未能知生,安能知死,故各以己意所欲也。吾往為臺郎,嘗以公事使過密縣之邢山。山上有塚,問耕父,云是鄭大夫祭仲,或云子產之塚也,遂率從者祭而觀焉。其造塚居山之頂,四望周達,連山體南北之正而邪東北,向新鄭城,意不忘本也。其隧道唯塞其後而空其前,不填之,示藏無珍寶,不取於重深也。山多美石不用,必集洧水自然之石以為塚藏,貴不勞工巧,而此石不入世用也。君子尚其有情,小人無利可動,歷千載無毀,儉之致也。吾去
 春入朝,因郭氏喪亡,緣陪陵舊義,自表營洛陽城東首陽之南為將來兆域。而所得地中有小山,上無舊冢。其高顯雖未足比邢山,然東奉二陵,西瞻宮闕,南觀伊洛,北望夷叔,曠然遠覽,情之所安也。故遂表樹開道,為一定之制,至時皆用洛水圓石,開遂道南向,儀制取法於鄭大夫,欲以儉自完耳。棺器小斂之事,皆當稱此。」子孫一以遵之。子錫嗣。



 錫字世嘏。少有盛名,起家長沙王乂文學,累遷太子中舍人。性亮直忠烈,屢諫愍懷太子,言辭懇切,太子患之。後置針著錫常所坐處氈中,刺之流血。他日,太子問錫:「向著何事?」錫對:「醉不知。」太子詰之曰:「
 君喜責人,何自作過也。」後轉衛將軍長史。趙王倫篡位,以為治書御史。孫秀求交於錫,而錫拒之,秀雖銜之,憚其名高,不敢害也。惠帝反政,遷吏部郎、城陽太守,不拜,仍遷尚書左丞。年四十八卒,贈散騎常侍。子乂嗣,在《外戚傳》。



 史臣曰:泰始之際,人祇呈貺,羊公起平吳之策,其見天地之心焉。昔齊有黔夫,燕人祭北門之鬼;趙有李牧,秦王罷東并之勢。桑枝不競,瓜潤空慚。垂大信於南服,傾吳人於漢渚,江衢如砥,襁袂同歸。而在乎成功弗居,幅巾窮巷,落落焉其有風飆者也。杜預不有生知,用之則
 習,振長策而攻取,兼儒風而轉戰。孔門稱四,則仰止其三;《春秋》有五,而獨擅其一,不其優歟!夫三年之喪,雲無貴賤。輕纖奪於在位,可以興嗟;既葬釋於儲君,何其斯酷。徇以茍合,不求其正,以當代之元良,為諸侯之庶子,檀弓習於變禮者也,杜預其有焉。



 贊曰:漢池西險,吳江左回。羊公恩信,百萬歸來。昔之誓旅,懷經罕素。元凱文場,稱為武庫。



\end{pinyinscope}