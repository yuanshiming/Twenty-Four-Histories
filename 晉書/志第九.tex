\article{志第九}

\begin{pinyinscope}

 禮上



 夫人含
 天地陰陽之靈,有哀樂喜怒之情。乃聖垂範,以為民極,節其驕淫,以防其暴亂;崇高天地,虔敬鬼神,列尊卑之序,成夫婦之義,然後為國為家,可得而治也。《傳》曰:「一日克己復禮,天下歸仁。」若乃太一初分,燧人鉆火,志有暢於恭儉,情不由乎玉帛,而酌玄流於春澗之右,焚封豕於秋林之外,亦無得而闕焉。軒頊依神,唐虞稽古,逮乎隆周,其文大備。或垂百官之範,置不刊之法;或禮經三百,威儀三千,皆所以弘宣天意,雕刻人理。叔代澆訛,王風陵謝,事睽光國,禮亦愆家。趙簡子問太叔以揖讓周旋之禮,對曰:「蓋所謂儀而非禮也。」天經地義之道,自茲尤缺。哀公十一年,孔子自衛反魯,跡三代之典,垂百王之訓,時無明後,道噎不行。



 若夫情尚分流,堤防之仁是棄;澆訛異術,洙泗之風斯泯。是以漢文罷再期之喪,中興為一郊之祭,隨時之義,不其然歟!而西京元鼎之辰,
 中興永平之日,疏璧流而延冠帶,啟儒門而引諸生,兩京之盛,於斯為美。及山魚登俎,澤豕睽經,禮樂恆委,浮華相尚,而郊禋之制,綱紀或存。魏氏光宅,憲章斯美。王肅、高堂隆之徒,博通前載,三千條之禮,十七篇之學,各以舊文增損當世,豈所謂致君於堯舜之道焉。世屬雕墻,時逢秕政,周因之典,務多違俗,而遺編殘冊,猶有可觀者也。景初元年,營洛陽南委粟山以為圓丘,祀之日以始祖帝舜配,房俎生魚,陶樽玄酒,非搢紳為之綱紀,其孰能興於此者哉!



 宣景戎旅,未遑伊制。太康平吳,
 九州共一,禮經咸至,樂器同歸,於是齊魯諸生,各攜緗素。武皇帝亦初平寇亂,意先儀範。其吉禮也,則三茅不翦,日觀停瑄;其兇禮也,則深衣布冠,降席撤膳。明乎一謙三益之義,而教化行
 焉。元皇中興,事多權道,遺文舊典,不斷如發。是以常侍戴邈詣闕上疏云:「方今天地更始,萬物權輿,蕩近世之流弊,創千齡之英範。是故雙劍之節崇,而飛白之俗成;挾琴之容飾,而赴曲之和作。」
 其所以興起禮文,勸帝身先之也。穆哀之後,王猷漸替,桓溫居揆,政由己出,而有司或曜斯文,增暉執事,主威長謝,臣道專行。《記》曰,「茍無其位,不可以作禮樂」,豈斯之謂歟!



 晉始則有荀顗、鄭沖裁成國典,江左則有荀崧、刁協損益朝儀。《周官》五禮,吉兇軍賓嘉,而吉禮之大,莫過祭祀,故《洪範》八政,三日祀。祀者,所以昭孝事祖,通於神明者也。漢興,承秦滅學之後,制度多未能復古。歷東、西京四百餘年,故往往改變。
 魏氏承漢末大亂,舊章殄滅,命侍中王粲、尚書衛顗草創朝儀。及晉國建,文帝又命荀顗因魏代前事,撰為新禮,參考今古,更其節文,羊祜、任愷、庾峻、應貞並共刊定,成百六十五篇,奏之。太康初,尚書僕射硃整奏付尚書郎摯虞討論之。虞表所宜損增曰:



 臣典校故太尉顗所撰《五禮》,臣以為夫革命以垂統,帝王之美事也,隆禮以率教,邦國之大務也,
 是以臣
 前表禮事稽留,求速訖施行。又以《喪服》最多疑闕,宜見補定。又以今禮篇卷煩重,宜隨類通合。事久不出,懼
 見寢嘿。



 蓋冠婚祭會諸吉禮,其制少變;至於《喪服》,世之要用,而特易失旨。故子張疑高宗諒陰三年,子思不聽其子服出母,子游謂異父昆弟大功,而子夏謂之齊衰,及孔子沒而門人疑於所服。此等皆明達習禮,仰讀周典,俯師仲尼,漸漬聖訓,講肄積年,及遇喪事,尤尚若此,明喪禮易惑,不可不詳也。況自此已來,篇章焚散,去聖彌遠,喪制詭謬,固其宜矣。是以《喪服》一卷,卷不盈握,而爭說紛然。三年之喪,鄭云二十七月,王云二十五月。改葬之服,鄭云服緦三月,王云葬訖而除。繼母出嫁,鄭云皆服,王云從乎繼寄育乃為之服。無服之殤,鄭云子生
 一月哭之一日,王云以哭之日易服之月。如此者甚眾。《喪服》本文省略,必待注解事義乃彰;其傳說差詳,世稱子夏所作。鄭王祖《經》宗《傳》,而各有異同,天下並疑,莫知所定,而顗直書古《經》文而已,盡除子夏《傳》及先儒注說,其事不可得行。及其行事,故當還頒異說,一彼一此,非所以定制也。臣以為今宜參采《禮記》,略取《傳》說,補其未備,一其殊義。可依準王景侯所撰《喪服變除》,使類統明正,以斷疑爭,然後制無二門,咸同所由。



 又此禮當班於天下,不宜繁多。顗為百六十五篇,篇為一卷,合十五餘萬言,臣猶謂卷多文煩,類皆重出。案《尚書·堯典》祀山
 川之禮,惟於東嶽備稱牲幣之數,陳所用之儀,其餘則但曰「如初」。《周禮》祀天地五帝享先王,其事同者皆曰「亦如之」,文約而義舉。今禮儀事同而名異者,輒別為篇,卷煩而不典。皆宜省文通事,隨類合之,事有不同,乃列其異。如此,所減三分之一。



 虞討論新禮訖,以元康元年上之。所陳惟明堂五帝、二社六宗及吉凶王公制度,凡十五篇。有詔可其議。後虞與傅咸纘續其事,竟未成功。中原覆沒,虞之《決疑注》,是其遺事也。逮于江左,僕射刁協、太常荀崧補緝舊文,光祿大夫蔡謨又踵脩其事云。



 魏明帝太和元年正月丁未,郊祀武帝以配天,宗祀文
 帝於明堂以配上帝。於是時,二漢郊禋之制具存,魏所損益可知。四年八月,天子東巡,過繁昌,使執金吾臧霸行太尉事,以特牛祠受禪壇。景初元年十月乙卯,始營洛陽南委粟山為圜丘。詔曰:「昔漢氏之初,承秦滅學之後,採摭殘缺,以備郊祀。自甘泉后土,雍宮五畤,神祗兆位,多不經見,並以興廢無常,一彼一此,四百餘年,廢無禘禮,古代之所更立者,遂有闕焉。曹氏世係,出自有虞氏。今祀圜丘以始祖帝舜配,號圜丘曰皇皇帝天。方丘所祭曰皇皇后地,以舜妃伊
 氏配。天郊所祭曰皇天之神,以太祖武皇帝配。地郊所祭曰皇地之祗,以武宣皇后配。宗祀皇考高祖文皇帝於明堂,以配上帝。」十二月壬子冬至,始祀皇皇帝天于圜丘,以始祖有虞帝舜配。自正始以後,終魏世不復郊祀。



 魏元帝咸熙二年十二月甲子,使持節侍中太保鄭沖、兼太尉司隸校尉李憙奉皇帝璽綬策書,禪位于晉。丙寅,武皇帝設壇場于南郊,柴燎告類于上帝,是時尚未有祖配。泰始二年正月,詔曰:「有司前奏郊祀權用魏禮,朕不慮改作之難,令便為永制,眾議紛互,遂不時定,不得
 以時供饗神祗,配以祖考。日夕難企,貶食忘安,其便郊祀。」時群臣又議,五帝即天也,王氣時異,故殊其號,雖名有五,其實一神。明堂南郊,宜除五帝之坐,五郊改五精之號,皆同稱昊天上帝,各設一坐而已。地郊又除先后配祀。帝悉從之。二月丁丑,郊禮宣皇帝以配天,宗祀文皇帝於明堂以配上帝。是年十一月,有司又議奏,古者丘郊不異,宜並圓丘方丘於南北郊,更脩立壇兆,其二至之祀合於二郊。帝又從之,一如宣帝所用王肅議也。是月庚寅冬至,帝親祠圓丘於南郊。自是後,圓丘方澤不別立。



 太康三年正月,帝親郊祀,皇太子、皇子悉侍祠。十年十月,又詔曰:「《孝經》『郊祀后稷以配天,宗祀文王於明堂以配上帝。』而《周官》云『祀天旅上帝』,又曰『祀地旅四望』。望非地,則明堂上帝不得為天也。往者眾議除明堂五帝位,考之禮文不正。且《詩序》曰『文武之功,起於后稷』,故推以配天焉。宣帝以神武創業,既已配天,復以先帝配天,於義亦所不安。其復明堂及南郊五帝位。愍帝都長安,未及立郊廟而敗。



 元帝渡江,太興二年始議立郊祀儀。尚書令刁協、國子祭酒杜夷議,宜須旋都洛邑乃脩之。司徒荀組據漢獻帝都許即便立郊,自宜於此脩奉。驃騎
 王導、僕射荀崧、太常華恒、中書侍郎庾亮皆同組議,事遂施行,立南郊於已地。其制度皆太常賀循所定,多依漢及晉初之儀。三月辛卯,帝親郊祀,饗配之禮一依武帝始郊故事。是時尚未立北壇,地祗眾神共在天郊。



 明帝太寧三年七月,始詔立北郊,未及建而帝崩。及成帝咸和八年正月,追述前旨,於覆舟山南立之。天郊則五帝之佐、日月、五星、二十八宿、文昌、北斗、三台、司命、軒轅、后土、太一、天一、太微、句陳、北極、雨師、雷電、司空、風伯、老人,凡六十二神也。地郊則五嶽、四望、四海、四瀆、五湖、五帝之佐、沂山、嶽山、白山、霍山、醫無閭山、蔣山、松江、會
 稽山、錢唐江、先農,凡四十四神也。江南諸小山,蓋江左所立,猶如漢西京關中小水皆有祭秩也。是月辛未,祀北郊,始以宣穆張皇后配,此魏氏故事,非晉舊也。



 康帝建元元年正月,將北郊,有疑議。太常顧和表:「泰始中,合二至之禮於二郊。北郊之月,古無明文,或以夏至,或同用陽月。漢光武正月辛未,始建北郊,此則與南郊同月。及中興草創,百度從簡,合七郊於一丘,憲章未備,權用斯禮,蓋時宜也。至咸和中,議別立北郊,同用正月。魏承後漢,正月祭天以地配。時高堂隆等以為禮祭天不以地配,而稱《周禮》三王之郊一用夏正。」於是從和議。
 是月辛未南郊,辛已北郊,帝皆親奉。



 安帝元興三年,劉裕討桓玄,走之。已卯,告義功于南郊。是年,帝蒙塵江陵未反。其明年應郊,朝議以為宜依《周禮》,宗伯攝職,三公行事。尚書左丞王納之獨曰:「既殯郊祀,自是天子當陽,有君存焉,稟命而行,何所辯也。郊之興否,豈如今日之比乎!」議者又云:「今宜郊,故是承制所得令三公行事。」又「郊天極尊,惟一而已,故非天子不祀也。庶人以上,莫不蒸嘗,嫡子居外,介子執事,未有不親受命而可祭天者。」納之又曰:「武皇受禪,用二月郊,元帝中興,以三月郊。今郊時未過,日望輿駕,無為欲速,而
 使皇輿旋反,更不得親奉也。」於是從納之議。



 郊廟牲幣璧玉之色,雖有成文,秦世多以騮駒,漢則但云犢,未辯其色。江左南北郊同用玄牲,明堂廟社同以赤牲。



 禮,有事告祖禰宜社之文,未有告郊之典也。漢儀,天子之喪,使太尉告謚于南郊,他無聞焉。魏文帝黃初四年七月,帝將東巡,以大軍當出,使太常以一特牛告祠南郊。及文帝崩,太尉鐘繇告謚南郊,皆是有事於郊也。江左則廢。



 禮,春分祀朝日於東,秋分祀夕月於西。漢武帝郊泰畤,平旦出竹宮,東向揖日,其夕西向揖月。既郊明,又不
 在東西郊也。後遂旦夕常拜。故魏文帝詔曰:「漢氏不拜日於東郊,而旦夕常於殿下東西拜日月,煩褻似家人之事,非事天神之道也。」黃初二年正月乙亥,祀朝日于東門之外,又違禮二分之義。魏明帝太和元年二月丁亥,祀朝日于東郊,八月己丑,祀夕月于西郊,始得古禮。及武帝太康二年,有司奏,春分依舊請車駕祀朝日,寒溫未適,可不親出。詔曰:「禮儀宜有常,若如所奏,與故太尉所撰不同,復為無定制也。間者方難未平,故每從所奏,今戎事弭息,惟此為大。」案
 此詔,帝復為親祀朝日也。此後廢。



 禮,「郊祀后稷以配天,宗祀文王於明堂以配上帝。」魏文帝即位,用漢明堂而未有配。明帝太和元年,始宗祀文帝於明堂,齊王亦行其禮。



 晉初以文帝配,後復以宣帝,尋復還以文帝配,其餘無所變革。是則郊與明堂,同配異配,參差不同矣。摯虞議以為:「漢魏故事,明堂祀五帝之神。新禮,五帝即上帝,即天帝也。明堂除五帝之位,惟祭上帝。案仲尼稱『郊祀后稷以配天,宗祀文王於明堂以配上帝。』《周禮》,祀天旅上帝,祀地旅四望。望非地,則上帝非天,斷可識矣。郊丘之祀,
 掃地而祭,牲用繭栗,器用陶匏,事反其始,故配以遠祖。明堂之祭,備物以薦,玉牲並陳,籩豆成列,禮同人鬼,故配以近考。郊堂兆位,居然異體,牲牢品物,質文殊趣。且祖考同配,非謂尊嚴之美,三日再祀,非謂不黷之義,其非一神,亦足明矣。昔在上古,生為明王,沒則配五行,故太昊配木,神農配火,少昊配金,顓頊配水,黃帝配土。此五帝者,配天之神,同兆之於四郊,報之於明堂。祀天,大裘而冕,祀五帝亦如之。或以為五精之帝,佐天育物者也。前代相因,莫之或廢,晉初始從異議。《庚午詔書》,明堂及南郊除五帝之位,惟祀天神,新禮奉而用之。前太醫令
 韓楊上書,宜如舊祀五帝。太康十年,詔已施用。宜定新禮,明堂及郊祀五帝如舊。」詔從之。江左以後,未遑脩建。



 漢儀,太史每歲上其年歷,先立春、立夏、大暑、立秋、立冬常讀五時令,皇帝所服,各隨五時之色。帝升御坐,尚書令以下就席位,尚書三公郎以令置案上,奉以入,就席伏讀訖,賜酒一卮。魏氏常行其禮。魏明帝景初元年,通事白曰:「前後但見讀春夏秋冬四時令,至於服黃之時,獨闕不讀,今不解其故。」散騎常侍領太史令高堂隆以為「黃於五行,中央土也,王四季各十八日。土生於火,故於火用事之末服黃,三季則否。其令則隨四時,不以五行
 為令也,是以服黃無令。」斯則魏氏不讀大暑令也。



 及晉受命,亦有其制。傅咸云:「立秋一日,白路光於紫庭,白旂陳於玉階。」然則其日旂路皆白也。成帝咸和五年六月丁未,有司奏讀秋令。兼侍中散騎常侍荀奕、兼黃門侍郎散騎侍郎曹宇駮曰:「尚書三公曹奏讀秋令,儀注舊典未備。臣等參議光祿大夫臣華恒議,武皇帝以秋夏盛暑,常闕不讀令,在春冬不廢也。夫先王所以順時讀令者,蓋後天而奉天時,正服尊嚴之所重。今服章多闕,加比熱隆赫,臣等謂可如恆議,依故事闕而不讀。」詔可。六年三月,有司奏「今月十六日立
 夏。今正服漸備,四時讀令,是祗述天和隆殺之道,謂今故宜讀夏令。」奏可。



 《禮》,孟春之月,「乃擇元辰,天子親載耒耜,措之于參保介之御間,帥三公九卿諸侯大夫躬耕帝藉」。至秦滅學,其禮久廢。漢文帝之後,始行斯典。魏之三祖,亦皆親耕藉田。



 及武帝泰始四年,有司奏耕祠先農,可,令有司行事。詔曰:「夫國之大事,在祀與農。是以古之聖王,躬耕帝藉,以供郊廟之粢盛,且以訓化天下。近世以來,耕藉止於數步之中,空有慕古之名,曾無供祀訓農之實,而有百官車徒之費。今脩千畝之制,當與群公卿士躬稼穡之
 艱難,以率先天下。主者詳具其制,下河南,處田地於東郊之南,洛水之北。若無官田,隨宜使換,而不得侵人也。」於是乘輿御木輅以耕,以太牢祀先農。自惠帝之後,其事便廢。



 江左元帝將脩耕藉,尚書符問「藉田至尊應躬祠先農不」?賀循答:「漢儀無,止有至尊應自祭之文。然則《周禮》王者祭四望則毳冕,祭社稷五祀則絺冕,以此不為無親祭之義也。宜立兩儀注。」賀循等所上儀注又未詳允,事竟不行。後哀帝復欲行其典,亦不能遂。



 漢儀,縣邑常以乙未日祠先農,乃耕於乙地,以丙戌日祠風伯於戌地,以
 已丑日祠雨師於丑地,牲用羊豕。立春之日,皆青幡幘迎春于東郊外野中。迎春至自野中出,則迎拜之而還,弗祭。三時不迎。



 魏氏雖天子耕藉,籓鎮闕諸侯百畝之禮。及武帝末,有司奏:「古諸侯耕藉田百畝,躬執耒以奉社稷宗廟,以勸率農功。今諸王臨國,宜依脩耕藉之義。」然竟未施行。



 《周禮》,王后帥內外命婦享先蠶於北郊。漢儀,皇后親桑東郊苑中,蠶室祭蠶神,曰苑窳婦人、寓氏公主,祠用少牢。魏文帝黃初七年正月,命中宮蠶于北郊,依周典也。



 及
 武帝太康六年,散騎常侍華嶠奏:「先王之制,天子諸侯親耕藉田千畝,后夫人躬蠶桑。今陛下以聖明至仁,脩先王之緒,皇后體資生之德,合配乾之義,而坤道未光,蠶禮尚缺。以為宜依古式,備斯盛典。」詔曰:「昔天子親藉,以供粢盛,后夫人躬蠶,以備祭服,所以聿遵孝敬,明教示訓也。今藉田有制,而蠶禮不脩,由中間務多,未暇崇備。今天下無事,宜脩禮以示四海。其詳依古典,及近代故事,以參今宜,明年施行。」於是蠶於西郊,蓋與藉田對其方也。乃使侍中成粲草定其儀。先蠶壇高一丈,方二丈,為四出陛,陛廣五尺,在皇后採桑壇東南帷宮外
 門之外,而東南去帷宮十丈,在蠶室西南,桑林在其東。取列侯妻六人為蠶母。蠶將生,擇吉日,皇后著十二笄步搖,依漢魏故事,衣青衣,乘油畫雲母安車,駕六騩馬。女尚書著貂蟬佩璽陪乘,載筐鉤。公主、三夫人、九嬪、世婦、諸太妃、太夫人及縣鄉君、郡公侯特進夫人、外世婦、命婦皆步搖、衣青,各載筐鉤從蠶。先桑二日,蠶室生蠶著薄上。桑日,皇后未到,太祝令質明以一太牢告祠,謁者一人監祠。祠畢撤饌,班餘胙於從桑及奉祠者。皇后至西郊升壇,公主以下陪列壇東。皇后東面躬桑,採三條,諸妃公主各採五條,縣鄉君以下各採九條,悉以桑
 授蠶母,還蠶室。事訖,皇后還便坐,公主以下乃就位,設饗宴,賜絹各有差。



 前漢但置官社而無官稷,王莽置官稷,後復省。故漢至魏但太社有稷,而官社無稷,故常二社一稷也。



 晉初仍魏,無所增損。至太康九年,改建宗廟,而社稷壇一廟俱徙。乃詔曰:「社實一神,其並二社之祀。」於是車騎司馬傅咸表曰:



 《祭法》王社太社,各有其義。天子尊事郊廟,故冕而躬耕。躬耕也者,所以重孝享之粢盛。親耕故自報,自為立社者,為藉田而報者也。國以人為本,人以穀為命,故又為百姓立社而祈報焉。事異報殊,此社
 之所以有二也。



 王景侯之論王社,亦謂春祈藉田,秋而報之也。其論太社,則曰王者布下圻內,為百姓立之,謂之大社,不自立之於京都也。景侯此論據《祭法》。《祭法》:「大夫以下成群立社,曰置社。」景侯解曰,「今之里社是也」。景侯解《祭法》,則以置社為人間之社矣。而別論復以太社為人間之社,未曉此旨也。太社,天子為百姓而祀,故稱天子社。《郊特牲》曰:「天子太社,必受霜露風雨。」以群姓之眾,王者通為立社,故稱太社也。若夫置社,其數不一,蓋以里所為名,《左氏傳》盟于清丘之社是也。眾庶之社,既已不稱太矣,若復不立之京都,當安所立乎!



 《祭法》又曰,王
 為群姓立七祀,王自為立七祀。言自為者,自為而祀也;為群姓者,為群姓而祀也。太社與七祀其文正等。說者窮此,因云墳籍但有五祀,無七祀也。案祭,五祀國之大祀,七者小祀。《周禮》所云祭凡小祀,則墨冕之屬也。景侯解大厲曰,「如周杜伯,鬼有所歸,乃不為厲」。今云無二社者稱景侯,《祭法》不謂無二,則曰「口傳無其文也」。夫以景侯之明,擬議而後為解,而欲以口論除明文,如此非但二社當見思惟,景侯之後解亦未易除也。



 前被敕,《尚書·召告》乃社于新邑,惟一太牢,不二社之明義也。案《郊特牲》曰社稷太牢,必援一牢之文以明社之無二,則稷無牲矣。
 說者曰,舉社則稷可知。茍可舉社以明稷,何獨不舉一以明二?國之大事,在祀與戎。若有二而除之,不若過而存之。況存之有義,而除之無據乎?



 《周禮》封人掌設社壝,無稷字。今帝社無稷,蓋出於此。然國主社稷,故經傳動稱社稷。《周禮》王祭社稷則絺冕,此王社有稷之交也。封人所掌社壝之無稷字,說者以為略文,從可知也。謂宜仍舊立二社,而加立帝社之稷。



 時成粲義稱景侯論太社不立京都,欲破鄭氏學。咸重表以為:「如粲之論,景侯之解文以此壞。《大雅》云『乃立冢土』,毛公解曰,『冢土,大社也。』景侯解《詩》,即用此說。《禹貢》『惟土五色』,景侯解曰,『王者
 取五色土為太社,封四方諸侯,各割其方色土者覆四方也』。如此,太社復為立京都也。不知此論何從而出,而與解乖,上違經記明文,下壞景侯之解。臣雖頑蔽,少長學門,不能默已,謹復續上。」劉寔與咸議同。詔曰:「社實一神,而相襲二位,眾議不同,何必改作!其便仍舊,一如魏制。」



 其後摯虞奏,以為:「臣案《祭法》『王為群姓立社曰太社,王自『為立社曰王社。』《周禮》大司徒『設其社稷之壝』,又曰『以血祭祭社稷』,則太社也。又曰『封人掌設王之社壝』,又有軍旅宜乎社,則王社也。太社為群姓祈報,祈報有時,主不可廢。故凡祓社釁鼓,主奉以從是也。此皆二社之明
 文,前代之所尊。以《尚書·召告》社于新邑三牲各文,《詩》稱『乃立冢土』,無兩社之交,故廢帝社,惟立太社。《詩書》所稱,各指一事,又皆在公旦制作之前,未可以易《周禮》之明典,《祭法》之正義。前改建廟社,營一社之處,朝議斐然,執古匡今。世祖武皇帝躬發明詔,定二社之義,以為永制。宜定新禮,從二社。」詔從之。



 至元帝建武元年,又依洛京立二社一稷。其太社之祝曰:「地德普施,惠存無疆。乃建太社,保祐萬邦。悠悠四海,咸賴嘉祥。」其帝社之祝曰:「坤德厚載,邦畿是保。乃建帝社,以神地道。明祀惟辰,景福來造。」



 漢儀,每月旦,太史上其月歷,有司侍郎尚書見讀其令,奉行其正。朔前後二日,牽牛酒至社下以祭日。日有變,割羊以祠社,用救日變。執事者長冠,衣絳領袖緣中衣、絳緣以行禮,如故事。自晉受命,日月將交會,太史乃上合朔,尚書先事三日,宣攝內外戒嚴。摯虞《決疑》曰:「凡救日蝕者,著赤幘,以助陽也。日將蝕,天子素服避正殿,內外嚴警。太史登靈臺,伺侯日變,便伐鼓於門。聞鼓音,侍臣皆著赤幘,帶劍入侍。三臺令史以上皆各持劍,立其戶前。衛尉卿驅馳繞宮,伺察守備。周而復始,亦伐鼓於社,用周禮也。又以赤絲為繩以繫社,祝史陳辭以責
 之。社,勾龍之神,天子之上公,故陳辭以責之。日復常,乃罷。」



 漢建安中,將正會,而太史上言,正旦當日蝕。朝士疑會否,共諮尚書令荀彧。時廣平計吏劉邵在坐,曰:「梓慎、裨灶,古之良史,尤占水火,錯失天時。《禮》,諸侯旅見天子,入門不得終禮者四,日蝕在一。然則聖人垂制,不為變異豫廢朝禮者,或災消異伏,或推術謬誤也。」彧及眾人咸善而從之,遂朝會如舊,日亦不蝕,邵由此顯名。



 至武帝咸寧三年、四年,並以正旦合朔卻元會,改魏故事也。元帝太興元年四月,合朔,中書侍郎孔愉奏曰:「《春秋》,日有蝕之,天子伐鼓于社,攻諸陰也;諸侯伐鼓于朝,臣自
 攻也。案尚書符,若日有變,便擊鼓於諸門,有違舊典。」詔曰:「所陳有正義,輒敕外改之。」



 至康帝建元元年,太史上元日合朔,後復疑應卻會與否。庾冰輔政,寫劉邵議以示八坐。于時有謂邵為不得禮意,荀彧從之,是勝人之一失。故蔡謨遂著議非之,曰:「邵論災消異伏,又以梓慎、裨灶猶有錯失,太史上言,亦不必審,其理誠然也。而云聖人垂制,不為變異豫廢朝禮,此則謬矣。災祥之發,所以譴告人君,王者之所重誡,故素服廢樂,退避正寢,百官降物,用幣伐鼓,躬親而救之。夫敬誡之事,與其疑而廢之,寧慎而行之。故孔子、老聃助葬於巷黨,以喪不
 見星而行,故日蝕而止柩,曰安知其不見星也。而邵廢之,是棄聖賢之成規也。魯桓公壬申有災,而以乙亥嘗祭,《春秋》譏之。災事既過,猶追懼未已,故廢宗廟之察,況聞天眚將至,行慶樂之會,於禮乖矣。《禮記》所云諸侯入門不得終禮者,謂日官不豫言,諸侯既入,見蝕乃知耳,非先聞當蝕而朝會不廢也。引此,可謂失其義旨。劉邵所執者《禮記》也,夫子、老聃巷黨之事,亦《禮記》所言,復違而反之,進退無據。然荀令所善,漢朝所從,遂使此言至今見稱,莫知其誤矣,後來君子將擬以為式,故正之云爾。」於是冰從眾議,遂以卻會。



 至永和中,殷浩輔政,又欲從劉
 邵議不卻會。王彪之據咸寧、建元故事,又曰:「《禮》云諸侯旅見天子,不得終禮而廢者四,自謂卒暴有之,非為先存其事,而僥倖史官推術繆錯,故不豫廢朝禮也。」於是又從彪之議。



 《尚書》「禋于六宗」,諸儒互說,往往不同。王莽以《易》六子,遂立六宗祠。魏明帝時疑其事,以問王肅,亦以為易六子,故不廢。及晉受命,司馬彪等表六宗之祀不應特立新禮,於是遂罷其祀。其後摯虞奏之,又以為:「案舜受終,『類于上帝,系于六宗,望于山川』,則六宗非上帝之神,又非山川之靈也。《周禮》肆師職曰:『用牲于社宗。』黨正職曰:『春
 秋祭禜亦如之。』肆師之宗,與社並列,則班與社同也。黨正之禜,文不繫社,則神與社異也。周之命祀,莫重郊社,宗同於社,則貴神明矣。又,《月令》孟冬祈于天宗,則《周禮》祭禜,《月令》天宗,六宗之神也。漢光武即位高邑,依《虞書》禋于六宗。安帝元初中,立祀乾位,禮同太社。魏氏因之,至景初二年,大議其神,朝士紛紜,各有所執。惟散騎常侍劉邵以為萬物負陰而抱陽,沖氣以為和。六宗者,太極沖和之氣,為六氣之宗者也。《虞書》謂之六宗,《周書》謂之天宗。是時考論異同,而從其議。漢魏相仍,著為貴祀。凡崇祀百神,放而不至,有其興之,則莫敢廢之。宜定新
 禮,祀六宗如舊。」詔從之。



 《禮》,王為群姓立七祀,曰司命、中霤、國門、國行、大厲、戶、灶。仲春玄鳥至之日,以太牢祀高禖。《毛詩》《絲衣篇》,高子曰靈星之尸。漢興,高帝亦立靈星祠。及武帝,以李少君故,始祠灶;及生戾太子,始立高禖。《漢儀》云,國家亦有五祀,有司行事,其禮頗輕於社稷,則亦存其典矣。又云,常以仲春之月,立高禖祠于城南,祀以特牲。又,是月也,祠老人星于國都南郊老人星廟。立夏祭灶,季秋祠心星于城南壇心星廟。元康時,洛陽猶有高禖壇,百姓祠其旁,或謂之落星。是後諸祀無聞,江左以來,不立七祀,靈
 星則配饗南郊,不復特置焉。



 左氏傳「龍見而雩」,經典尚矣。漢儀,自立春到立夏,盡立秋,郡國尚旱,郡縣各掃除社稷。其旱也,公卿官長以次行雩禮求雨,閉諸陽,衣皁,興土龍,立土人,舞僮二佾,七日一變,如故事。武帝咸寧二年,春久旱。四月丁已,詔曰「諸旱處廣加祈請」。五月庚午,始祈雨于社稷山川。六月戊子,獲澍雨。此雩之舊典也。太康三年四月,十年二月,又如之。其雨多則禜祭,赤幘朱衣,閉諸陰,朱索縈社,伐朱鼓焉。



 《周禮》,王者祭昊天上帝、日月星辰、司中司命、風伯雨師、
 社稷、五土、五嶽、山林川澤、四方百物,兆四類四望,亦如之。魏文帝黃初二年六月庚子,初禮五嶽四瀆,咸秩群祀,瘞沈珪璧。六年七月,帝以舟軍入淮。九月壬戌,遣使者沈璧于淮。魏明帝太和四年八月,帝東巡,遣使者以特牛祠中嶽。魏元帝咸熙元年,行幸長安,使使者以璧幣禮祠華山。



 及穆帝升平中,何琦論修五嶽祠曰:「唐虞之制,天子五載一巡狩,順時之方,柴燎五嶽,望于山川,遍於群神,故曰,因名山升中于天,所以昭告神祗,饗報功德。是以災
 厲不作,而風雨寒暑以時。降及三代,年數雖殊,而其禮不易,五嶽視三公,四瀆視諸侯,著在經紀,所謂『有其舉之,莫敢廢也。』及秦漢都西京,涇、渭、長水,雖不在祀典,以近咸陽故,盡得比大川之祠,而正立之祀可以闕哉!自永嘉之亂,神州傾覆,茲事替矣。惟灊之天柱,在王略之內也,舊臺選百戶吏卒,以奉其職。中興之際,未有官守,廬江郡常遣大吏兼假四時禱賽,春釋寒而冬請冰。咸和迄今,又復隳替。計今非典之祠,可謂非一。考其正名,則淫昏之鬼;推其糜費,則百姓之蠹。而山川大神更為簡缺,禮俗頹紊,人神雜擾,公私奔蹙,漸以繁滋。良由頃
 國家多難,日不暇給,草建廢滯,事有未遑。今元憝已殲,宜脩舊典。嶽瀆之域,風教所被,來蘇之眾,咸蒙德澤。而神明禋祀,未之或甄,巡狩柴燎,其廢尚矣。崇明前典,將俟皇輿北旋,稽古憲章,大釐制度。俎豆牲牢,祝嘏文辭,舊章靡記,可令禮官作式,歸諸誠簡,以達明德馨香,如斯而已。其諸襖孽,可粗依法令,先去其甚,俾邪正不黷。」時不見省。



 昔武王入殷,未及下車而封先代之後,蓋追思其德也。孔子以大聖而終於陪臣,未有封爵。至漢元帝,孔霸以帝師賜爵,號褒成君,奉孔子後。
 魏文帝黃初二年正月,詔以議郎孔羨為宗聖侯,邑百戶,奉孔子祀,令魯郡脩舊廟,置百戶吏卒以守衛之。及武帝泰始三年十一月,改封宗聖侯孔震為奉聖亭侯。又詔太學及魯國,四時備三牲以祀孔子。明帝太寧三年,詔給奉聖亭侯孔亭四時祠孔子祭直,如泰始故事。



 禮,始立學必先釋奠于先聖先師,及行事必用幣。漢世雖立學,斯禮無聞。魏齊王正始二年二月,帝講論語通,五年五月,講《尚書》通,七年十二月,講《禮記》通,並使太常釋奠,以太牢祠孔子於辟雍,以顏回配。
 武帝泰始七年,皇太子講《孝經》通。咸寧三年,講《詩》通,太康三年,講《禮記》通。惠帝元康三年,皇太子講《論語》通。元帝太興二年,皇太子講《論語》通。太子並親釋奠,以太牢祠孔子,以顏回配。成帝咸康元年,帝講《詩》通。穆帝升平元年三月,帝講《孝經》通。孝武寧康三年七月,帝講《孝經》通。並釋奠如故事。穆帝、孝武並權以中堂為太學。



 故事,祀皋陶於廷尉寺,新禮移祀於律署,以同祭先聖於太學也。故事,祀以社日,新禮改以孟秋之月,以應秋政。摯虞以為:「案《虞書》,皋陶作士師,惟明克允,國重其功,人思其當,是以獄官禮其神,繫者致其祭,功在斷獄之
 成,不在律令之始也。大學之設,義重太常,故祭于太學,是崇聖而從重也。律署之置,卑於廷尉,移祀於署,是去重而就輕也。律非正署,廢興無常,宜如舊祀於廷尉。又,祭用仲春,義取重生,改用孟秋,以應刑殺,理未足以相易。宜定新禮,皆如舊。」制:「可。」



 歲旦常設葦茭桃梗,磔雞於宮及百寺之門,以禳惡氣。案漢儀則仲夏設之,有桃印,無磔雞。及魏明帝大脩禳禮,故何晏禳祭議雞特牲供禳釁之事。磔雞宜起於魏,桃印本漢制,所以輔卯金,又宜魏所除也。但未詳改仲夏在歲旦之所起耳。
 魏明帝青龍元年,詔郡國,山川不在祀典者勿祠。



 武帝泰始元年十二月,詔曰:「昔聖帝明王脩五嶽四瀆,名山川澤,各有定制,所以報陰陽之功故也。然以道蒞天下者,其鬼不神,其神不傷人,故祝史薦而無媿辭,是以其人敬慎幽冥而淫祀不作。末世信道不篤,僭禮瀆神,縱欲祈請,曾不敬而遠之,徒偷以求幸,襖妄相煽,舍正為邪,故魏朝疾之。其案舊禮具為之制,使功著於人者必有其報,而襖淫之鬼不亂其間。」二年正月,有司奏春分祠厲殃及禳祠,詔曰:「不在祀典,除之。」



 《王制》,天子七廟,諸侯以下各有等差,禮文詳矣。漢獻帝
 建安十八年五月,以河北十郡封魏武帝為魏公。是年七月,始建宗廟于鄴,自以諸侯禮立五廟也。後雖進爵為王,無所改易。延康元年,文帝繼王位,七月,追尊皇祖為大王,丁夫人曰大王后。黃初元年十一月受禪,又追尊大王曰大皇帝,皇考武王曰武皇帝。二年六月,以洛京宗廟未成,乃祠武帝於建始殿,親執饋奠,如家人禮。案《禮》將營宮室,宗廟為先,庶人無廟,故祭於寢,帝者行之非禮甚矣。



 明帝太和三年六月,又追尊高祖大長秋曰高皇,夫人吳氏曰高皇后,並在鄴廟。廟所祠,則文帝之高祖處士、
 曾祖高皇、祖大皇帝共一廟,考太祖武皇帝特一廟,百世不毀,然則所祠止於親廟四室也。其年十一月,洛京廟成,則以親盡遷處士主置園邑,使行太傅太常韓暨、行太常宗正曹恪持節迎高皇以下神主,共一廟,猶為四室而已。至景初元年六月,群公有司始更奏定七廟之制,曰:「大魏三聖相承,以成帝業。武皇帝肇建洪基,撥亂夷險,為魏太祖。文皇帝繼天革命,應期受禪,為魏高祖。上集成大命,清定華夏,興制禮樂,宜為魏烈祖。於太祖廟北為二祧,其左為文帝廟,號曰高祖昭祧,其右擬明帝,號曰烈祖穆祧。三祖之廟,萬世不毀。其餘四廟,親盡迭
 遷,一如周后稷、文武廟祧之禮。」



 文帝甄后賜死,故不列廟。明帝即位,有司奏請追謚曰文昭皇后,使司空王朗持節奉策告祠于陵。三公又奏曰:「自古周人歸祖后稷,又特立廟以祀姜嫄。今文昭皇后之於後嗣,聖德至化,豈有量哉!夫以皇家世妃之尊,神靈遷化,而無寢廟以承享祀,非以報顯德,昭孝敬也。稽之古制,宜依周禮,別立寢廟。」奏可。太和元年二月,立廟於鄴。四月,洛邑初營宗廟,掘地得玉璽,方一寸九分,其文曰「天子羨思慈親。」明帝為之改容,以太牢告廟。至景初元年十二月己未,有司又奏文
 昭皇后立廟京師,永傳享祀,樂舞與祖廟同,廢鄴廟。



 魏元帝咸熙元年,進文帝爵為王,追命舞陽宣文侯為宣王,忠武侯為景王。是年八月,文帝崩,謚曰文王。



 武帝泰始元年十二月丙寅,受禪,丁卯,追尊皇祖宣王為宣皇帝,伯考景王為景皇帝,考文王為文皇帝,宣王妃張氏為宣穆皇后,景王夫人羊氏為景皇后。二年正月,有司奏置七廟。帝重其役,詔宜權立一廟。於是群臣議奏:「上古清廟一宮,尊遠神祗。逮至周室,制為七廟,以辯宗祧。聖旨深弘,遠跡上世,敦崇唐虞,舍七廟之繁華,遵一宮之遠旨。昔舜承堯禪,受終文祖,遂陟帝位,蓋三十
 載,月正元日,又格于文祖,遂陟帝位,此則虞氏不改唐廟,因仍舊宮。可依有虞氏故事,即用魏廟。」奏可。於是追祭征西將軍、豫章府君、潁川府君、京兆府君,與宣皇帝、景皇帝、文皇帝為三昭三穆。是時宣皇未升,太祖虛位,所以祠六世,與景帝為七廟,其禮則據王肅說也。七月,又詔曰:「主者前奏,就魏舊廟,誠亦有準。然於祗奉神明,情猶未安,宜更營造。」於是改創宗廟。十一月,追尊景帝夫人夏侯氏為景懷皇后。任茂議以為夏侯初嬪之時,未有王業。帝不從。太康元年,靈壽公主脩麗祔于太廟,周漢未有其準。魏明帝則別立平原主廟,晉又異魏也。六
 年,因廟陷,當改脩創,群臣又議奏曰:「古者七廟異所,自宜如禮。」詔又曰:「古雖七廟,自近代以來皆一廟七室,於禮無廢,於情為敘,亦隨時之宜也。其便仍舊。」至十年,乃更改築於宣陽門內,窮極壯麗,然坎位之制猶如初爾。廟成,帝用摯虞議,率百官遷神主于新廟,自征西以下,車服導從皆如帝者之儀。及武帝崩則遷征西,及惠帝崩又遷豫章。而惠帝世愍懷太子、太子二子哀太孫臧、沖太孫尚並祔廟,元帝世,懷帝殤太子又祔廟,號為陰室四殤。懷帝初,又策謚武帝楊后曰武悼皇后,改葬峻陽陵側,別祠弘訓宮,不列於廟。



 元帝既即尊位,上繼武帝,於元為禰,
 如漢光武上繼元帝故事也。是時,西京神主,堙滅虜庭,江左建廟,皆更新造。尋以登懷帝之主,又遷潁川,位雖七室,其實五世,蓋從刁協以兄弟為世數故也。于時百度草創,舊禮未備,毀主權居別室。至太興三年正月乙卯,詔曰:「吾雖上繼世祖,然於懷、愍皇帝皆北面稱臣。今祠太廟,不親執觴酌,而令有司行事,於情禮不安。可依禮更處。」太常恒議:「今聖上繼武皇帝,宜準漢世祖故事,不親執觴爵。」又曰:「今上承繼武帝,而廟之昭穆,四世而已,前太常賀循、博士傅純,並以為惠、懷及愍,宜別立廟。然臣愚謂廟室當以容主為限,無拘常數。殷世有二祖
 三宗,若拘七室,則當祭禰而已。推此論之,宜還復豫章、潁川,全祠七廟之禮。」驃騎長史溫嶠議:「凡言兄弟不相入廟,既非禮文,且光武奮劍振起,不策名於孝平,務神其事,以應九世之讖,又古不共廟,故別立焉。今上以策名而言,殊於光武之事,躬奉蒸嘗,於經既正,於情又安矣。太常恒欲還二府君,以全七世,嶠謂是宜。」驃騎將軍王導從嶠議。嶠又曰:「其非子者,可直言皇帝敢告某皇帝,又若以一帝為一世,則不祭禰,反不及庶人。」帝從嶠議,悉施用之。於是乃更定制,還復豫章、潁川于昭穆之位,以同惠帝嗣武故事,而惠、懷、愍三帝自從《春秋》尊尊
 之義,在廟不替也。



 及元帝崩,則豫章復遷。然元帝神位猶在愍帝之下,故有坎室者十也。至明帝崩,而穎川又遷,猶十室也。於時續廣太廟,故三遷主並還西儲,名之曰祧,以準遠廟。成帝咸康七年五月,始作武悼皇后神主,祔于廟,配饗世祖。成帝崩而康帝承統,以兄弟一世,故不遷京兆,始十一室也。



 至康帝崩,穆帝立,永和二年七月,有司奏:「十月殷祭,京兆府君當遷祧室。昔征西、豫章、潁川三府君毀主,中興之初權居天府,在廟門之西。咸康中,太常馮懷表續奉還於西儲夾室,謂之為祧,疑亦非禮。今京兆遷入,是為四世遠祖,長在太祖之上。昔
 周室太祖世遠,故遷有所歸。今晉廟宣皇為主。而四祖居之,是屈祖就孫也;殷祫在上,是代太祖也。」領司徒蔡謨議:「四府君宜改築別室,若未展者,當入就太廟之室,人莫敢卑其祖,文武不先不窋。殷祭之日,征西東面,處宣皇之上。其後遷廟之主,藏於征西之祧,祭薦不絕。」護軍將軍馮懷議:「禮,無廟者為壇以祭,可立別室藏之,至殷禘則祭于壇也。」輔國將軍譙王司馬無忌等議:「諸儒謂太王、王季遷主,藏於文武之祧。如此,府君遷主宜在宣帝廟中。然今無寢室,宜變通而改築。又殷祫太廟,征西東面。」尚書郎孫綽與無忌議同,曰:「太祖雖位始九
 五,而道以從暢,替人爵之尊,篤天倫之道,所以成教本而光百代也。」尚書郎徐禪議:「《禮》『去祧為壇,去壇為墠』,歲祫則祭之。今四祖遷主,可藏之石室,有禱則祭於壇墠。」又遣禪至會稽,訪處士虞喜。喜答曰:「漢世韋玄成等以毀主瘞於園,魏朝議者云應埋兩階之間。且神主本在太廟,若今別室而祭,則不如永藏。又四君無追號之禮,益明應毀而無祭。」是時簡文為撫軍、與尚書郎劉邵等奏:「四祖同居西祧,藏主石室,禘祫及祭,如先朝舊儀。」時陳留范宣兄子問此禮,宣答曰:「舜廟所祭,皆是庶人,其後世遠而毀,不居舜上,不序昭穆。今四君號猶依
 本,非以功德致祀也。若依虞主之瘞,則猶藏子孫之所;若依夏主之埋,則又非本廟之階。宜思其變,則築一室,親未盡則禘祫處宣帝之上,親盡則無緣下就子孫之列。」其後太常劉遐等同蔡謨議。博士張憑議:「或疑陳於太祖者,皆其後之毀主,憑案古義無別前後之文也。禹不先鯀,則遷主居太祖之上,亦何疑也。」於是京兆遷入西儲,同謂之祧,如前三祖遷主之禮,故正室猶十一也。穆帝崩而哀帝、海西並為兄弟,無所登除。咸安之初,簡文皇帝上繼元皇,世秩登進,於是潁川、京兆二主復還昭穆之位。至簡文崩,潁川又遷。



 孝武帝太元十二年五月壬戌,
 詔曰:「昔建太廟,每事從儉,太祖虛位,明堂未建。郊祀國之大事,而稽古之制闕然,便可詳議。」祠部郎中徐邈議:「圓丘郊祀,經典無二,宣皇帝嘗辯斯義,而檢以聖典。爰及中興,備加研極,以定南北二郊,誠非異學所可輕改也。謂仍舊為安。武皇帝建廟六世,祖三昭三穆。宣皇帝創基之主,實惟太祖,親則王考。四廟在上,未及遷世,故權虛東向之位也。兄弟相及,義非二世。故當今廟祀,世數未足,而欲太祖正位,則違事七之義矣。又《禮》曰庶子王亦禘祖立廟,蓋謂支胤援立,則親近必復。京兆府君於今六世,宜復立此室,則宣皇未在六世之上,須前世
 既遷,乃太祖位定耳。京兆遷毀宜藏主於石室,雖禘祫猶弗及。何者?傳稱毀主升合乎太祖,升者自下之名,不謂可降尊就卑也。太子太孫,陰室四主,儲嗣之重,升祔皇祖,所配之廟,世遠應遷,然後從食之孫,與之俱毀。明堂方圓之制,綱領已舉,不宜闕配帝之祀。且王者以天下為家,未必一邦,故周平、光武無廢於二京也。明堂所配之神,積疑莫辯。案《易》『殷薦上帝,以配祖考』,祖考同配,則上帝亦為天,而嚴父之義顯。《周禮》旅上帝者,有故告天,與郊祀常禮同用四圭,故並言之。若上帝是五帝,《經》文何不言祀天旅五帝,祀地旅四望乎?」侍中車胤議同。又
 曰:「明堂之制,既其難詳,且樂主於和,禮主於敬,故質文不同,音器亦殊。既茅茨廣夏,不一其度,何必守其形範,而不弘本從俗乎?九服咸寧,河朔無塵,然後明堂辟雍可崇而脩之。」時朝議多同,於是奉行,一無所改。十六年,始改作太廟殿,正室十四間,東西儲各一間,合十六間,棟高八丈四尺。備法駕遷神主于行廟,征西至京兆四主及太子太孫各用其位之儀服。四主不從帝者之儀,是與太康異也。諸主既入廟,設脯醢之奠。及新廟成,神主還室,又設脯醢之奠。十九年二月,追尊簡文母會稽太妃鄭氏為簡文皇帝宣太后,立廟太廟道西。及孝武崩,京兆又遷,
 如穆帝之世四祧故事。



 義熙九年四月,將殷祠,詔博議遷毀之禮。大司馬瑯邪王德文議:「泰始之初,虛太祖之位,而緣情流遠,上及征西,故世盡則宜毀,而宣帝正太祖之位。又漢光武移十一帝主於洛邑,則毀主不設,理可推矣。宜築別室,以居四府君之主,永藏而弗祀也。」大司農徐廣議:「四府君嘗處廟堂之首,歆率土之祭,若埋之幽壤,於情理未必咸盡。謂可遷藏西儲,以為遠祧,而禘饗永絕也。」太尉諮議參軍袁豹議:「仍舊無革,殷祠猶及四府君,情理為允。」時劉裕作輔,意與大司馬議同,須後殷祠行事改制。會安帝崩,未及禘而天祿終焉。



 武帝咸寧五年十一月己酉,弘訓羊太后崩,宗廟廢一時之祀,天地明堂去樂,且不上胙。穆帝升平五年十月己卯,殷祀,以帝崩後不作樂。孝武太元十一年九月,皇女亡,及應烝祠,中書侍郎范1111奏:「案《喪服傳》有死宮中者三月不舉祭,不別長幼之與貴賤也。皇女雖在嬰孩,臣竊以為疑。」於是尚書奏使三公行事。



 武帝泰始七年四月,帝將親祠,車駕夕牲,而儀注還不拜。詔問其故,博士奏歷代相承如此。帝曰:「非致敬宗廟之禮也。」於是實拜而還,遂以為制,夕牲必躬臨拜,而江
 左以來復止。



 魏故事,天子為次殿於廟殿之北東,天子入自北門。新禮,設次殿於南門中門外之右,天子入自南門。摯虞以為:「次殿所以為解息之處,凡適尊以不顯為恭,以由隱為順,而設之於上位,入自南門,非謙厭之義。宜定新禮,皆如舊說。」從之。



 禮,大事則告祖禰,小事則特告禰,秦漢久廢。魏文帝黃初四年七月,將東巡,以大軍當出,使太常以特牛告南郊。及文帝崩,又使太尉告謚策於南郊。自是迄晉相承,告郊之後仍以告廟,至江左其禮廢。至成帝
 咸和三年,蘇峻覆亂京都,溫嶠等立行廟於白石,復行其典。告先君及后曰:「逆臣蘇峻,傾覆社稷,毀棄三正。汙辱海內。臣侃、臣嶠、臣亮等手刃戎首,龔行天罰。惟中宗元皇帝、肅祖明皇帝、明穆皇后之靈,降鑒有罪,剿絕其命,翦此群凶,以安宗廟。臣等雖隕首摧軀,猶生之年。」



 魏明帝太和三年,詔曰:「禮,王后無嗣,擇建支子,以繼大宗,則當纂正統而奉公義,何得復顧私親哉!漢宣繼昭帝後,加悼考以皇號。哀帝以外籓援立,而董宏等稱引亡秦,惑誤朝議,遂尊恭皇,立廟京師。又寵籓妾,使比長信,僭差無禮,人神弗佑。非罪師丹忠正之諫,用致丁傅
 焚如之禍。自是之後,相踵行之。其令公卿有司,深以前世為戒。後嗣萬一有由諸侯入奉大統,則當明為人後之義。敢為佞邪導諛君上,妄建非正之號,謂考為皇,稱妣為後,則股肱大臣誅之無赦。其書之金策,藏之宗廟。」是後高貴、常道援立,皆不外尊。及愍帝建興四年,司徒梁芬議追尊之禮,帝既不從,而左僕射索綝等亦稱引魏制,以為不可,故追贈吳王為太保而已。元帝太興二年,有司言瑯邪恭王宜稱皇考。賀循議云:「禮典之義,子不敢以己爵加其父號。」帝又從之。



\end{pinyinscope}