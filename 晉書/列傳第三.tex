\article{列傳第三}

\begin{pinyinscope}
王祥
 \gezhu{
  王覽}
 鄭沖何曾
 \gezhu{
  何劭何遵}
 石苞
 \gezhu{
  石崇歐陽健孫鑠}



 王祥,字休徵,瑯邪臨沂人,漢諫議大夫吉之後也。祖仁,青州刺史。父融,公府辟不就。



 祥性至孝。早喪親,繼母朱氏不慈,數譖之,由是失愛於父。每使掃除牛下,祥愈恭謹。父母有疾,衣不解帶,湯藥必親嘗。母常欲生魚,時天寒冰凍,祥解衣將剖冰求之,冰忽自解,雙鯉躍出,持之而歸。母又思黃雀灸,復有黃雀數十飛入其幕,復以供
 母。鄉里驚嘆,以為孝感所致焉。有丹柰結實,母命守之,每風雨,祥輒抱樹而泣。其篤孝純至如此。



 漢未遭亂,扶母攜弟覽避地廬江,隱居三十餘年,不應州郡之命。母終,居喪毀瘁,杖而後起。徐州刺史呂虔檄為別駕,祥年垂耳順,固辭不受。覽勸之,為具車牛,祥乃應召,虔委以州事。于時寇盜充斥,祥率勵兵士,頻討破之。州界清靜,政化大行。時人歌之曰:「海沂之康,實賴王祥。邦國不空,別駕之功。」



 舉秀才,除溫令,累遷大司農。高貴鄉公即位,與定策功,封關內侯,拜光祿勳,轉司隸校尉。從討毌丘儉,增邑四百戶,遷太常,封萬歲亭侯。天子幸太學,命祥
 為三老。祥南面几杖,以師道自居。天子北面乞言,祥陳明王聖帝君臣政化之要以訓之,聞者莫不砥礪。



 及高貴鄉公之弒也,朝臣舉哀,祥號哭曰「老臣無狀」,涕淚交流,眾有愧色。頃之,拜司空,轉太尉,加侍中。五等建,封睢陵侯,邑一千六百戶。



 及武帝為晉王,祥與荀顗往謁,顗謂祥曰:「相王尊重,何侯既已盡敬,今便當拜也。」祥曰:「相國誠為尊貴,然是魏之宰相。吾等魏之三公,公王相去,一階而已,班例大同,安有天子三司而輒拜人者!損魏朝之望,虧晉王之德,君子愛人以禮,吾不為也。」及入,顗遂拜,而祥獨長揖。帝曰:「今日方知君見顧之重矣!」



 武帝
 踐阼,拜太保,進爵為公,加置七官之職。帝新愛命,虛己以求讜言。祥與何曾、鄭沖等耆艾篤老,希復朝見,帝遣侍中任愷諮問得失,及政化所先。祥以年老疲耄,累乞遜位,帝不許。御史中丞侯史光以祥久疾,闕朝會禮,請免祥官。詔曰:「太保元老高行,朕所毗倚以隆政道者也。前後遜讓,不從所執,此非有司所得議也。」遂寢光奏。祥固乞骸骨,詔聽以睢陵公就第,位同保傅,在三司之右,祿賜如前。詔曰:「古之致仕,不事王侯。今雖以國公留居京邑,不宜復苦以朝請。其賜几杖,不朝,大事皆諮訪之。賜安車駟馬,第一區,錢百萬,絹五百匹,床帳簟褥,以舍
 人六人為睢陵公舍人,置官騎二十人。以公子騎都尉肇為給事中,使常優游定省。又以太保高潔清素,家無宅宇,其權留本府,須所賜第成乃出。」



 及疾篤,著遺令訓子孫曰:「夫生之有死,自然之理。吾年八十有五,啟手何恨。不有遺言,使爾無述。吾生值季末,登庸歷試,無毗佐之勳,沒無以報。氣絕但洗手足,不須沐浴,勿纏尸,皆浣故衣,隨時所服。所賜山玄玉佩、衛氏玉玦、綬笥皆勿以斂。西芒上土自堅貞,勿用甓石,勿起墳隴。穿深二丈,槨取容棺。勿作前堂、布几筵、置書箱鏡奩之具,棺前但可施床榻而已。Я脯各一盤,玄酒一杯,為朝夕奠。家人大
 小不須送喪,大小祥乃設特牲。無違餘命!高柴泣血三年,夫子謂之愚。閔子除喪出見。援琴切切而哀,仲尼謂之孝。故哭泣之哀,日月降殺,飲食之宜,自有制度。夫言行可覆,信之至也;推美引過,德之至也;揚名顯親,孝之至也;兄弟怡怡,宗族欣欣,悌之至也;臨財莫過乎讓:此五者,立身之本。顏子所以為命,未之思也,夫何遠之有!」其子皆奉而行之。



 泰始五年薨,詔賜東園秘器,朝服一具,衣一襲,錢三十萬,布帛百匹。時文明皇太后崩始踰月,其後詔曰:「為睢陵公發哀,事乃至今。雖每為之感傷,要未得特敘哀情。今便哭之。」明年,策謚曰元。



 祥之薨,奔
 赴者非朝廷之賢,則親親故吏而已,門無雜弔之賓。族孫戎嘆曰:「太保可謂清達矣!」又稱:「祥在正始,不在能言之流。及與之言,理致清遠,將非以德掩其言乎!」祥有五子:肇、夏、馥、烈、芬。



 肇孽庶,夏早卒,馥嗣爵。咸寧初,以祥家甚貧儉,賜絹三百匹,拜馥上洛太守,卒謚曰孝。子根嗣,散騎郎。肇仕至始平太守。肇子俊,守太子舍人,封永世侯。俊子遐,鬱林太守。烈、芬並幼知名,為祥所愛。二子亦同時而亡。將死,烈欲還葬舊土,芬欲留葬京邑。祥流涕曰:「不忘故鄉,仁也;不戀本土,達也。惟仁與達,吾二子有焉。」



 覽字玄通。母朱,遇祥無道。覽年數歲,見祥被楚撻,輒涕泣抱持。至于成童,每諫其母,其母少止凶虐。朱屢以非理使祥,覽輒與祥俱。又虐使祥妻,覽妻亦趨而共之。朱患之,乃止。祥喪父之後,漸有時譽。朱深疾之,密使鴆祥。覽知之,徑起取酒。祥疑其有毒,爭而不與,朱遽奪反之。自後朱賜祥饌,覽輒先嘗。朱懼覽致斃,遂止。



 覽孝友恭恪,名亞於祥。及祥仕進,覽亦應本郡之召,稍遷司徒西曹掾、清河太守。五等建,封即丘子,邑六百戶。泰始末,除弘訓少府。職省,轉太中大夫,祿賜與卿同。咸寧初,詔曰:「
 覽少篤至行,服仁履義,貞素之操,長而彌固。其以覽為宗正卿。」頃之,以疾上疏乞骸骨。詔聽之,以太中大夫歸老,賜錢二十萬,床帳薦褥,遣殿中醫療疾給藥。後轉光祿大夫,門施行馬。



 咸寧四年卒,時年七十三,謚曰貞。有六子:裁、基、會、正、彥、琛。



 裁字士初,撫軍長史。基字士先,治書御史。會字士和,侍御史。正字士則,尚書郎。彥字士治,中護軍。琛字士瑋,國子祭酒。



 初,呂虔有佩刀,工相之,以為必登三公,可服此刀。虔謂祥曰:「茍非其人,刀或為害。卿有公輔之量,故以相與。」祥固辭,彊之乃受。祥臨薨,以刀授覽,曰:「汝後必興,足稱此刀。」覽後奕世多賢才,興於
 江左矣。裁子導,別有傳。



 鄭沖,字文和,滎陽開封人也。起自寒微,卓爾立操,清恬寡欲,耽玩經史,遂博究儒術及百家之言。有姿望,動必循禮,任真自守,不要鄉曲之譽,由是州郡久不加禮。及魏文帝為太子,搜揚側陋,命沖為文學,累遷尚書郎,出補陳留太守。沖以儒雅為德,蒞職無幹局之譽,簞食縕袍,不營資產,世以此重之。大將軍曹爽引為從事中郎,轉散騎常侍、光祿勳。嘉平三年,拜司空。及高貴鄉公講《尚書》,沖執經親授,與侍中鄭小同俱被賞賜。俄轉司徒。
 常道鄉公即位,拜太保,位在三司之上,封壽光侯。沖雖位階台輔,而不預世事。時文帝輔政,平蜀之後,命賈充、羊祜等分定禮儀、律令,皆先諮於沖,然後施行。



 及魏帝告禪,使沖奉策。武帝踐阼,拜太傅,進爵為公。頃之,司隸李憙、中丞侯史光奏沖及何曾,荀顗等各以疾病,俱應免官。帝不許。沖遂不視事,表乞骸骨。優詔不許,遣使申喻。沖固辭,上貂蟬印綬,詔又不許。泰始六年,詔曰:「昔漢祖以知人善任,克平宇宙,推述勳勞,歸美三俊。遂與功臣剖符作誓,藏之宗廟,副在有司,所以明德庸勳,籓翼王室者也。昔我祖考,遭世多難,攬授英俊,與之斷金,遂
 濟時務,克定大業。太傅壽光公鄭沖、太保郎陵公何曾、太尉臨淮公荀顗各尚德依仁,明允篤誠,翼亮先皇,光濟帝業。故司空博陵元公王沈、衛將軍鉅平侯羊祜才兼文武,忠肅居正,朕甚嘉之。《書》不云乎:『天秩有禮,五服五章哉!』其為壽光、郎陵、臨淮、博陵、鉅平國置郎中令,假夫人、世子印綬,食本秩三分之一,皆如郡公侯比。」



 九年,沖又抗表致仕。詔曰:「太傅韞德深粹,履行高潔,恬遠清虛,確然絕世。艾服王事,六十餘載,忠肅在公,慮不及私。遂應眾舉,歷登三事。仍荷保傅之重,綢繆論道之任,光輔奕世,亮茲天工,迪宣謀猷,弘濟大烈,可謂朝之俊老,
 眾所具瞻者也。朕昧于政道,庶事未康,挹仰耆訓,導揚厥蒙,庶賴顯德,緝熙有成。而公屢以年高疾篤,致仕告退。惟從公志,則朕孰與諮謀?譬彼涉川,罔知攸濟。是用未許,迄于累載。而高讓彌篤,至意難違,覽其盛指,俾朕憮然。夫功成弗有,上德所隆,成人之美,君子與焉。豈必遂朕憑賴之心,以枉大雅進止之度哉!今聽其所執,以壽光公就第,位同保傅,在三司之右。公宜頤精養神,保衛太和,以究遐福。其賜几杖,不朝。古之哲王,欽祗國老,憲行乞言,以彌縫其闕。若朝有大政,皆就諮之。又賜安車駟馬,第一區,錢百萬,絹五百匹,床帷簟褥,置舍人六
 人,官騎二十人,以世子徽為散騎常侍,使常優游定省。祿賜所供,策命儀制,一如舊典而有加焉。」



 明年薨。帝於朝堂發哀,追贈太傅,賜祕器,朝服,衣一襲,錢三十萬,布百匹。謚曰成。咸寧初,有司奏,沖與安平王孚等十二人皆存銘太常,配食於廟。



 初,沖與孫邕、曹羲、荀顗、何晏共集《論語》諸家訓註之善者,記其姓名,因從其義,有不安者輒改易之,名曰《論語集解》。成,奏之魏朝,于今傳焉。



 沖無子,以從子徽為嗣,位至平原內史。徽卒,子簡嗣。



 何曾,字穎考,陳國陽夏人也。父夔,魏太僕、陽武亭侯。曾
 少襲爵,好學博聞,與同郡袁侃齊名。魏明帝初為平原侯,曾為文學。及即位,累遷散騎侍郎、汲郡典農中郎將、給事黃門侍郎。上疏曰:「臣聞為國者以清靜為基,而百姓以良吏為本。今海內虛耗,事役眾多,誠宜恤養黎元,悅以使人。郡守之權雖輕,猶專任千里,比之於古,則列國之君也。上當奉宣朝恩,以致惠和,下當興利而除其害。得其人則可安,非其人則為患。故漢宣稱曰:「百姓所以安其田里,而無歎息愁恨之心者,政平訟理也。與我共此者,其惟良二千石乎!」此誠可謂知政之本也。方今國家大舉,新有發調,軍師遠征,上下劬勞。夫百姓可與
 樂成,難與慮始。愚惑之人,能厭目前之小勤,而忘為亂之大禍者,是以郡守益不可不得其人。才雖難備,猶宜粗有威恩,為百姓所信憚者。臣聞諸郡守,有年老或疾病,皆委政丞掾,不恤庶事。或體性疏怠,不以政理為意。在官積年,惠澤不加於人。然於考課之限,罪亦不至詘免。故得經延歲月,而無斥罷之期。臣愚以為可密詔主者,使隱核參訪郡守,其有老病不隱親人物,及宰牧少恩,好修人事,煩撓百姓者,皆可徵還,為更選代。」頃之,遷散騎常侍。



 及宣帝將伐遼東,曾上疏魏帝曰:「臣聞先王制法,必全於慎。故建官受任,則置副佐;陳師命將,則立
 監貳;宣命遣使,則設介副;臨敵交刃,又參御右,蓋以盡思謀之功,防安危之變也。是以在險當難,則權足相濟;隕缺不豫,則才足相代。其為國防,至深至遠。及至漢氏,亦循舊章,韓信伐趙,張耳為貳;馬援討越,劉隆副軍。前世之迹,著在篇志。今太尉奉辭誅罪,精甲銳鋒,步騎數萬,道路迥阻,且四千里。雖假天威,有征無戰,寇或潛遁,消引日月。命無常期,人非金石,遠慮詳備,誠宜有副。今北軍諸將及太尉所督,皆為僚屬,名位不殊,素無定分統御之尊,卒有變急,不相鎮攝。存不忘亡,聖達所裁。臣愚以為宜選大臣名將威重宿著者,成其禮秩,遣詣北
 軍,進同謀略,退為副佐。雖有萬一不虞之變,軍主有儲,則無患矣。」帝不從。出補河內太守,在任有威嚴之稱。徵拜侍中,母憂去官。



 嘉平中,為司隸校尉。撫軍校事尹模憑寵作威,姦利盈積,朝野畏憚,莫敢言者。曾奏劾之,朝廷稱焉。時曹爽專權,宣帝稱疾,曾亦謝病。爽誅,乃起視事,魏帝之廢也,曾預其謀焉。



 時步兵校尉阮籍負才放誕,居喪無禮。曾面質籍於文帝座曰:「卿縱情背禮,敗俗之人,今忠賢執政,綜核名實,若卿之曹,不可長也。」因言於帝曰:「公方以孝治天下,而聽阮籍以重哀飲酒食肉於公座。宜擯四裔,無令汙染華夏。」帝曰:「此子羸病若此,
 君不能為吾忍邪!」曾重引據,辭理甚切。帝雖不從,時人敬憚之。



 毌丘儉誅,子甸、妻荀應坐死。其族兄顗、族父虞並景帝姻通,共表魏帝以丐其命。詔聽離婚,荀所生女芝為潁川太守劉子元妻,亦坐死,以懷妊繫獄。荀辭詣曾乞恩曰:「芝繫在廷尉,顧影知命,計日備法。乞沒為官婢,以贖芝命。」曾哀之,騰辭上議。朝廷僉以為當,遂改法。語在《刑法志》。



 曾在司隸積年,遷尚書,正元年中為鎮北將軍、都督河北諸軍事、假節。將之鎮,文帝使武帝、齊王攸辭送數十里。曾盛為賓主,備太牢之饌。侍從吏騶,莫不醉飽。帝既出,又過其子劭。曾先敕劭曰:「客必過汝,汝
 當豫嚴。」劭不冠帶,停帝良久,曾深以譴劭。曾見崇重如此。遷征北將軍,進封潁昌鄉侯。咸熙初,拜司徒,改封郎陵侯。文帝為晉王,曾與高柔、鄭沖俱為三公,將入見,曾獨致拜盡敬,二人猶揖而已。



 武帝襲王位,以曾為晉丞相,加侍中。與裴秀、王沈等勸進。踐阼,拜太尉,進爵為公,食邑千八百戶。泰始初,詔曰:「蓋謨明弼諧,王躬是保,所以宣崇大訓,克咸四海也。侍中、太尉何曾,立德高峻,執心忠亮,博物洽聞,明識弘達,翼佐先皇,勳庸顯著。朕纂洪業,首相王室。迪惟前人,施于朕躬。實佐命興化,光贊政道。夫三司之任,雖左右王事,若乃予違汝弼,匡獎不
 逮,則存平保傅。故將明袞職,未如用乂厥辟之重。其以曾為太保,侍中如故。」久之,以本官領司徒。曾固讓,不許。遣散騎常侍諭旨,乃視事。進位太傅。曾以老年,屢乞遜位。詔曰:「太傅明朗高亮,執心弘毅,可謂舊德老成,國之宗臣者也。而高尚其事,屢辭祿位。朕以寡德,憑賴保佑,省覽章表,實用憮然。雖欲成人之美,豈得遂其雅志,而忘翼佐之益哉!又司徒所掌務煩,不可久勞耆艾。其進太宰,侍中如故。朝會劍履乘輿上殿,如漢相國蕭何、田千秋、魏太傅鍾繇故事。賜錢百萬,絹五百匹及八尺床帳簟褥自副。置長史掾屬祭酒及員吏,一依舊制。所給
 親兵官騎如前。主者依次按禮典,務使優備。」後每召見,敕以常所飲食服物自隨,令二子侍從。



 咸寧四年薨,時年八十。帝於朝堂素服舉哀,賜東園秘器,朝服一具,衣一襲,錢三十萬,布百匹。將葬,下禮官議謚。博士秦秀謚為「繆醜」,帝不從,策謚曰孝。太康末,子劭自表改謚為元。



 曾性至孝,閨門整肅,自少及長,無聲樂嬖幸之好。年老之後,與妻相見,皆正衣冠,相待如賓。己南向,妻北面,再拜上酒,酬酢既畢便出。一歲如此者不過再三焉。初,司隸校尉傅玄著論稱曾及荀顗曰:「以文王之道事其親者,其潁昌何侯乎,其荀侯乎!古稱曾、閔,今日荀、何。內盡
 其心以事其親,外崇禮讓以接天下。孝子,百世之宗;仁人,天下之命。有能行孝之道,君子之儀表也。《詩》云:「高山仰止,景行行止。」令德不遵二夫子之景行者,非樂中正之道也。」又曰:「荀、何,君子之宗也。」又曰:「潁昌侯之事親,其盡孝子之道乎!存盡其和,事盡其敬,亡盡其哀,予於潁昌侯見之矣。」又曰:「見其親之黨,如見其親,六十而孺慕,予於潁昌侯見之矣。」然性奢豪,務在華侈。帷帳車服,窮極綺麗,廚膳滋味,過於王者。每燕見,不食太官所設,帝輒命取其食。蒸餅上不坼作十字不食。食日萬錢,猶曰無下箸處。人以小紙為書者,敕記室勿報。劉毅等數劾
 奏曾侈忲無度,帝以其重臣,一無所問。



 都官從事劉享嘗奏曾華侈,以銅鉤紖車,瑩牛蹄角。後曾辟享為掾,或勸勿應,享謂至公之體,不以私憾,遂應辟。曾常因小事加享杖罰。其外寬內忌,亦此類也。時司空賈充權擬人主,曾卑充而附之。及充與庾純因酒相競,曾議黨充而抑純,以此為正直所非。二子:遵、劭。劭嗣。



 劭字敬祖,少與武帝同年,有總角之好。帝為王太子,以劭為中庶子。及即位,轉散騎常侍,甚見親待。劭雅有姿望,遠客朝見,必以劭侍直。每諸方貢獻,帝輒賜之,而觀其占謝焉。咸寧初,有司奏劭及兄遵等受故鬲令袁毅貨,雖經赦宥,
 宜皆禁止。事下廷尉。詔曰:「太保與毅有累世之交,遵等所取差薄,一皆置之。」遷侍中尚書。



 惠帝即位,初建東宮,太子年幼,欲令親萬機,故盛選六傅,以劭為太子太師,通省尚書事。後轉特進,累遷尚書左僕射。



 劭博學,善屬文,陳說近代事,若指諸掌。永康初,遷司徒。趙王倫篡位,以劭為太宰。及三王交爭,劭以軒冕而游其間,無怨之者。而驕奢簡貴,亦有父風。衣裘服玩,新故巨積。食必盡四方珍異,一日之供以錢二萬為限。時論以為太官御膳,無以加之。然優游自足,不貪權勢。嘗語鄉人王詮曰:「僕雖名位過幸,少無可書之事,惟與夏侯長容諫授博
 士,可傳史冊耳。」所撰《荀粲》、《王弼傳》及諸奏議文章並行於世。永寧元年薨,贈司徒,謚曰康。子岐嗣。



 劭初亡,袁粲弔岐,岐辭以疾。粲獨哭而出曰:「今年決下婢子品。」王詮謂之曰:「知死弔死,何必見生!岐前多罪,爾時不下,何公新亡,便下岐品。人謂中正畏彊易弱。」粲乃止。



 遵字思祖,劭庶兄也。少有幹能。起家散騎黃門郎、散騎常侍、侍中,累轉大鴻臚。性亦奢忲,役使御府工匠作禁物,又鬻行器,為司隸劉毅所奏,免官。太康初,起為魏郡太守,遷太僕卿,又免官,卒於家,四子,嵩、綏、機、羨。



 嵩字泰基,寬弘愛士,博觀墳籍,尤善《史》、《漢》。少歷清官,領著作郎。



 綏字伯蔚,
 位至侍中尚書。自以繼世名貴,奢侈過度,性既輕物,翰札簡傲。城陽王尼見綏書疏,謂人曰:「伯蔚居亂而矜豪乃爾,豈其免乎!」劉輿、潘滔譖之於東海王越,越遂誅綏。初,曾侍武帝宴,退而告遵等曰:「國家應天受禪,創業垂統。吾每宴見,未嘗聞經國遠圖,惟說平生常事,非貽厥孫謀之兆也。及身而已,後嗣其殆乎!此子孫之憂也。汝等猶可獲沒。」指諸孫曰:「此等必遇亂亡也。」及綏死,嵩哭之曰;「我祖其大聖乎!」



 機為鄒平令。性亦矜傲,責鄉里謝鯤等拜。或戒之曰:「禮敬年爵,以德為主。令鯤拜勢,懼傷風俗。」機不以為慚。



 羨為離狐令。既驕且吝,陵駕人物,鄉
 閭疾之如仇。永嘉之末,何氏滅亡無遺焉。



 石苞,字仲容,渤海南皮人也。雅曠有智局,容儀偉麗,不修小節。故時人為之語曰:「石仲容,姣無雙。」縣召為吏,給農司馬。會謁者陽翟郭玄信奉使,求人為御,司馬以苞及鄧艾給之。行十餘里,玄信謂二人曰:「子後並當至卿相。」苞曰:「御隸也,何卿相乎?」既而又被使到鄴,事久不決,乃販鐵於鄴市。市長沛國趙元儒名知人,見苞,異之,因與結交。歎苞遠量,當至公輔,由是知名,見吏部郎許允,求為小縣。允謂苞曰;「卿是我輩人,當相引在朝廷,何欲
 小縣乎?」苞還嘆息,不意允之知己乃如此也。



 稍遷景帝中護軍司馬。宣帝聞苞好色薄行,以讓景帝。帝答曰:「苞雖細行不足,而有經國才略。夫貞廉之士,未必能經濟世務。是以齊桓忘管仲之奢僭,而錄其匡合之大謀;漢高捨陳平之汙行,而取其六奇之妙算。苞雖未可以上儔二子,亦今日之選也。」意乃釋。徙鄴典農中郎將。時魏世王侯多居鄴下,尚書丁謐貴傾一時,並較時利。苞奏列其事,由是益見稱。歷東萊、瑯邪太守,所在皆有威惠。遷徐州刺史。



 文帝之敗於東關也,苞獨全軍而退。帝指所持節謂苞曰:「恨不以此授卿,以究大事。」乃遷苞為奮
 武將軍、假節、監青州諸軍事。及諸葛誕舉兵淮南,苞統青州諸軍,督兗州刺史州泰、徐州刺史胡質,簡銳卒為游軍,以備外寇。吳遣大將朱異、丁奉等來迎,誕等留輜重於都陸,輕兵渡黎水。苞等逆擊,大破之。泰山太守胡烈以奇兵詭道襲都陸,盡焚其委輸。異等收餘眾而退,壽春平。拜苞鎮東將軍,封東光侯、假節。頃之,代王基都督揚州諸軍事。苞因入朝。當還,辭高貴鄉公,留語盡日。既出,白文帝曰:「非常主也。」數日而有成濟之事。後進位征東大將軍,俄遷驃騎將軍。



 文帝崩,賈充、荀勖議葬禮未定。苞時奔喪,慟哭曰:「基業如此,而以人臣終乎!」葬禮
 乃定。後每與陳騫諷魏帝以歷數已終,天命有在。及禪位,苞有力焉。武帝踐阼,遷大司馬,進封樂陵郡公,加侍中,羽葆鼓吹。



 自諸葛破滅,苞便鎮撫淮南,士馬彊盛,邊境多務,苞既勤庶事,又以威德服物。淮北監軍王琛輕苞素微,又聞童謠曰:「宮中大馬幾作驢,大石壓之不得舒。」因是密表苞與吳人交通。先時望氣者云「東南有大兵起」。及琛表至,武帝甚疑之。會荊州刺史胡烈表吳人欲大出為寇,苞亦聞吳師將入,乃築壘遏水以自固。帝聞之,謂羊祜曰:「吳人每來,常東西相應,無緣偏爾,豈石苞果有不順乎?」祜深明之,而帝猶疑焉。會苞子喬為
 尚書郎,上召之,經日不至。帝謂為必叛,欲討苞而隱其事。遂下詔以苞不料賊勢,築壘遏水,勞擾百姓,策免其官。遣太尉義陽王望率大軍征之,以備非常。又敕鎮東將軍、瑯邪王伷自下邳會壽春。苞用掾孫鑠計,放兵步出,住都亭待罪。帝聞之,意解。及苞詣闕,以公還第。苞自恥受任無效而無怨色。



 時鄴奚官督郭暠上書理苞。帝詔曰:「前大司馬苞忠允清亮,才經世務,乾用之績,所歷可紀。宜掌教典,以贊時政。其以苞為司徒。」有司奏:「苞前有折撓,不堪其任。以公還第,已為弘厚,不宜擢用。」詔曰:「吳人輕脆,終無能為。故疆埸之事,但欲完固守備,使不
 得越逸而已。以苞計畫不同,慮敵過甚,故徵還更授。昔鄧禹撓於關中,而終輔漢室,豈以一眚而掩大德哉!」於是就位。



 苞奏:「州郡農桑未有賞罰之制,宜遣掾屬循行,皆當均其土宜,舉其殿最,然後黜陟焉。」詔曰:「農殖者,為政之本,有國之大務也。雖欲安時興化,不先富而教之,其道無由。而至今四海多事,軍國用廣,加承征伐之後,屢有水旱之事,倉庫不充,百姓無積。古道稼穡樹藝,司徒掌之。今雖登論道,然經國立政,惟時所急,故陶唐之世,稷官為重。今司徒位當其任,乃心王事,有毀家紓國,乾乾匪躬之志。其使司徒督察州郡播殖,將委事任成,
 垂拱仰辦。若宜有所循行者,其增置掾屬十人,聽取王官更練事業者。」苞在位稱為忠勤,帝每委任焉。



 泰始八年薨。帝發哀於朝堂,賜祕器,朝服一具,衣一襲,錢三十萬,布百匹。及葬,給節、幢、麾、曲蓋、追鋒車、鼓吹、介士、大車,皆如魏司空陳泰故事。車駕臨送於東掖門外。策謚曰武。咸寧初,詔苞等並為王功,列於銘饗。



 苞豫為《終制》曰:「延陵薄葬,孔子以為達禮;華元厚葬,《春秋》以為不臣,古之明義也。自今死亡者,皆斂以時服,不得兼重。又不得飯含,為愚俗所為。又不得設床帳明器也。定窆之後,復土滿坎,一不得起墳種樹。昔王孫裸葬矯時,其子奉命,
 君子不譏,況於合禮典者耶?」諸子皆奉遵遺令,又斷親戚故吏設祭。有六子:越、喬、統、浚、俊、崇。以統為嗣。



 統字弘緒,歷位射聲校尉、大鴻臚。子順,為尚書郎。



 越字弘倫,早卒。



 喬字弘祖,歷尚書郎、散騎侍郎。帝既召喬不得,深疑苞反。及苞至,有慚色,謂之曰「卿子幾破卿門」。苞遂廢之,終身不聽仕。又以有穢行,徙頓丘,與弟崇同被害。二子超、熙亡走得免。成都王穎之起義也,以超為折衝將軍,討孫秀,以功封侯。又為振武將軍,征荊州賊李辰。穎與長沙王乂相攻,超常為前鋒,遷中護軍。陳等挾惠帝北伐,超走還鄴。穎使超距帝於蕩陰,王師敗績,超逼帝
 幸鄴宮。會王浚攻穎於鄴,穎以超為右將軍以距浚,大敗而歸。從駕之洛陽,西遷長安。河間王顒以超領北中郎將,使與穎共距東海王越。超於滎陽募兵,右將軍王闡與典兵中郎趙則並受超節度,為豫州刺史劉喬繼援。范陽王虓逆擊斬超,而熙得走免。永嘉中,為太傅越參軍。



 浚字景倫,清儉有鑒識,敬愛人物。位至黃門侍郎,為當世名士,早卒。



 俊字彥倫,少有名譽,議者稱為令器。官至陽平太守,早卒。



 崇字季倫,生於青州,故小名齊奴。少敏惠,勇而有謀。苞臨終,分財物與諸子,獨不及崇。其母以為言,苞曰:「此兒雖小,後自能得。」年二十餘,為修武
 令,有能名。入為散騎郎,遷城陽太守。伐吳有功,封安陽鄉侯。在郡雖有職務,好學不倦,以疾自解。頃之,拜黃門郎。



 兄統忤扶風王駿,有司承旨奏統,將加重罰,既而見原。以崇不詣闕謝恩,有司欲復加統罪。崇自表曰:「臣兄統以先父之恩,早被優遇,出入清顯,歷位盡勤。伏度聖心,有以垂察。近為扶風王駿橫所誣謗,司隸中丞等飛筆重奏,劾案深文,累塵天聽。臣兄弟跼蹐,憂心如悸。駿戚屬尊重,權要赫奕。內外有司,望風承旨。茍有所惡,易於投卵。自統枉劾以來,臣兄弟不敢一言稍自申理。戢舌鉗口,惟須刑書。古人稱「榮華於順旨,枯槁於逆違」,誠
 哉斯言,於今信矣。是以雖董司直繩,不能不深其文,抱枉含謗,不得不輸其理。幸賴陛下天聽四達,靈鑒昭遠,存先父勳德之重,察臣等勉勵之志。中詔申料,罪譴澄雪。臣等刻肌碎首,未足上報。臣即以今月十四日,與兄統、浚等詣公車門拜表謝恩。伏度奏御之日,暫經天聽。此月二十日,忽被蘭臺禁止符,以統蒙宥,恩出非常,臣晏然私門,曾不陳謝,復見彈奏,訕辱理盡。臣始聞此,惶懼狼狽,靜而思之,固無怪也。茍尊勢所驅,何所不至,望奉法之直繩,不可得也。臣以凡才,累荷顯重,不能負載析薪,以答萬分。一月之中,奏劾頻加,曲之與直,非臣所
 計。所愧不能承奉戚屬,自陷於此。不媚於灶,實愧王孫,《隨巢子》稱「明君之德,察情為上,察事次之」。所懷具經聖聽,伏待罪黜,無所多言。」由是事解。累遷散騎常侍、侍中。



 武帝以崇功臣子,有幹局,深器重之。元康初,楊駿輔政,大開封賞,多樹黨援。崇與散騎郎蜀郡何攀共立議,奏於惠帝曰:「陛下聖德光被,皇靈啟祚,正位東宮,二十餘年,道化宣流,萬國歸心。今承洪基,此乃天授。至於班賞行爵,優於泰始革命之初。不安一也。吳會僭逆,幾於百年,邊境被其荼毒,朝廷為之旰食。先帝決獨斷之聰,奮神武之略,蕩滅逋寇,易於摧枯。然謀臣猛將,猶有致思
 竭力之效。而今恩澤之封,優於滅吳之功。不安二也。上天眷祐,實在大晉,卜世之數,未知其紀。今之開制,當垂於後。若尊卑無差,有爵必進,數世之後,莫非公侯。不安三也。臣等敢冒陳聞。竊謂泰始之初,及平吳論功,制度名牒,皆悉具存。縱不能遠遵古典,尚當依準舊事。」書奏,弗納。出為南中郎將、荊州刺史,領南蠻校尉,加鷹揚將軍。崇在南中,得鴆鳥雛,以與後軍將軍王愷。時制,鴆鳥不得過江,為司隸校尉傅祗所糾,詔原之,燒鴆於都街。



 崇穎悟有才氣,而任俠無行檢。在荊州,劫遠使商客,致富不貲。徵為大司農,以征書未至擅去官免。頃之,拜太僕,
 出為征虜將軍,假節、監徐州諸軍事,鎮下邳。崇有別館在河陽之金谷,一名梓澤,送者傾都,帳飲於此焉。至鎮,與徐州刺史高誕爭酒相侮,為軍司所奏,免官。復拜衛尉,與潘岳諂事賈謐。謐與之親善,號曰「二十四友」。廣城君每出,崇降車路左,望塵而拜,其卑佞如此。



 財產豐積,室宇宏麗。後房百數,皆曳紈繡,珥金翠。絲竹盡當時之選,庖膳窮水陸之珍。與貴戚王愷、羊琇之徒以奢靡相尚。愷以臺澳釜,崇以蠟代薪。愷作紫絲布步障四十里,崇作錦步障五十里以敵之。崇塗屋以椒,愷用赤石脂。崇、愷爭豪如此。武帝每助愷,嘗以珊瑚樹賜之,高二尺
 許,枝柯扶疏,世所罕比。愷以示崇,崇便以鐵如意擊之,應手而碎。愷既惋惜,又以為嫉己之寶,聲色方厲。崇曰:「不足多恨,今還卿。」乃命左右悉取珊瑚樹,有高三四尺者六七株,條幹絕俗,光彩曜日,如愷比者甚眾。愷心兄然自失矣。



 崇為客作豆粥,咄嗟便辦。每冬,得韭萍齏。嘗與愷出遊,爭入洛城,崇牛迅若飛禽,愷絕不能及。愷每以此三事為根,乃密貨崇帳下問其所以。答云:「豆至難煮,豫作熟末,客來,但作白粥以投之耳。韭萍齏是搗韭根雜以麥苗耳。牛奔不遲,良由馭者逐不及反制之,可聽蹁轅則駃矣。」於是悉從之,遂爭長焉。崇後知之,因殺所
 告者。



 嘗與王敦入太學,見顏回、原憲之象,顧而歎曰:「若與之同升孔堂,去人何必有間。」敦曰:「不知餘人云何,子貢去卿差近。」崇正色曰:「士當身名俱泰,何至甕牖哉!」其立意類此。



 劉輿兄弟少時為王愷所嫉,愷召之宿,因欲坑之。崇素與輿等善,聞當有變,夜馳詣愷,問二劉所在,愷迫卒不得隱。崇徑進於後齋索出,同車而去。語曰:「年少何以輕就人宿!」輿深德之。



 及賈謐誅,崇以黨與免官。時趙王倫專權,崇甥歐陽建與倫有隙。崇有妓曰綠珠,美而艷,善吹笛。孫秀使人求之。崇時在金谷別館,方登涼臺,臨清流,婦人侍側。使者以告。崇盡出其婢妾數十
 人以示之,皆蘊蘭麝,被羅縠,曰:「在所擇。」使者曰:「君侯服御麗則麗矣,然本受命指索綠珠,不識孰是?」崇勃然曰:「綠珠吾所愛,不可得也。」使者曰:「君侯博古通今,察遠照邇,願加三思。」崇曰:「不然。」使者出而又反,崇竟不許。秀怒,乃勸倫誅崇、建。崇、建亦潛知其計,乃與黃門郎潘岳陰勸淮南王允、齊王冏以圖倫、秀。秀覺之,遂矯詔收崇及潘岳、歐陽建等。崇正宴於樓上,介士到門。崇謂綠珠曰:「我今為爾得罪。」綠珠泣曰:「當效死於官前。」因自投於樓下而死。崇曰:「吾不過流徙交、廣耳。」及車載詣東市,崇乃歎曰:「奴輩利吾家財。」收者答曰:「知財致害,何不早散之?」
 崇不能答。崇母兄妻子無少長皆被害,死者十五人,崇時年五十二。



 初,崇家稻米飯在地,經宿皆化為螺,時人以為族滅之應。有司簿閱崇水碓三餘區,蒼頭八百餘人,他珍寶貨賄田宅稱是。及惠帝復阼,詔以卿禮葬之。封崇從孫演為樂陵公。



 苞曾孫樸字玄真,為人謹厚,無他材藝,沒於胡。石勒以與樸同姓,俱出河北,引樸為宗室,特加優寵,位至司徒。



 歐陽建字堅石,世為冀方右族。雅有理思,才藻美贍,擅名北州。時人為之語曰:「渤海赫赫,歐陽堅石。」辟公府,歷山陽令、尚書郎、馮翊太守,甚得時譽。及遇禍,莫不悼惜
 之,年三十餘。臨命作詩,文甚哀楚。



 孫鑠字巨鄴,河內懷人也。少樂為縣吏,太守吳奮轉以為主簿。鑠自微賤登綱紀,時僚大姓猶不與鑠同坐。奮大怒,遂薦鑠為司隸都官從事。司隸校尉劉訥甚知賞之。時奮又薦鑠於大司馬石苞,苞辟為掾。鑠將應命,行達許昌,會臺已密遣輕軍襲苞。于時汝陰王鎮許,鑠過謁之。王先識鑠,以鄉里之情私告鑠曰:「無與禍。」鑠即出,即馳詣壽春,為苞畫計,苞賴而獲免。遷尚書郎,在職駁議十有餘事,為當時所稱。



 史臣曰:若夫經為帝師,鄭沖於焉無愧;孝為德本,王祥
 所以當仁;何曾善其親而及其親之黨者也。夏禹恭儉,殷因損益。牲牢服用,各有品章,諸侯不恆牛,命士不恆豕。御而驕奢,其關乎治政。乘時立制,莫不由之。石崇學乃多聞,情乖寡悔,超四豪而取富,喻五侯而競爽。春畦𧆑靡,列於凝沍之晨;錦障逶迤,亙以山川之外。撞鐘舞女,流宕忘歸,至於金穀含悲,吹樓將墜,所謂高蟬處乎輕陰,不知螳良襲其後也。



 贊曰:鄭沖含素,王祥遲暮。百行斯融,雙飛天路。何石殊操,芳飪標奇。帝風流靡,崇心載馳。矜奢不極,寇害成貲。邦分身墜,樂往哀隨。



\end{pinyinscope}