\article{列傳第五十四}

\begin{pinyinscope}
王恭庾楷劉牢之
 \gezhu{
  子敬宣}
 殷仲堪楊佺期



 王恭,字孝伯,光祿大夫蘊子,定皇后之兄也。少有美譽,清操過人,自負才地高華,恆有宰輔之望。與王忱齊名友善,慕劉惔之為人。謝安常曰:「王恭人地可以為將來伯舅。」嘗從其父自會稽至都,忱訪之,見恭所坐六尺簟,忱謂其有餘,因求之。恭輒以送焉,遂坐薦上。忱聞而大驚,恭曰:「吾平生無長物。」其簡率如此。



 起家為佐著作郎,
 歎曰:「仕宦不為宰相,才志何足以騁!」因以疾辭。俄為祕書丞,轉中書郎,未拜,遭父憂。服闋,除吏部郎,歷建威將軍。太元中,代沈嘉為丹陽尹,遷中書令,領太子詹事。



 孝武帝以恭后兄,深相欽重。時陳郡袁悅之以傾巧事會稽王道子,恭言之於帝,遂誅之。道子嘗集朝士,置酒於東府,尚書令謝石因醉為委巷之歌,恭正色曰:「居端右之重,集籓王之第,而肆淫聲,欲令群下何所取則!」石深銜之。淮陵內史虞珧子妻裴氏有服食之術,常衣黃衣,狀如天師,道子甚悅之,令與賓客談論,時人皆為降節。恭抗言曰:「未聞宰相之坐有失行婦人。」坐賓莫不反側,道
 子甚愧之。其後帝將擢時望以為籓屏,乃以恭為都督兗青冀幽並徐州晉陵諸軍事、平北將軍、兗青二州刺史、假節,鎮京口。初,都督以「北」為號者,累有不祥,故桓沖、王坦之、刁彝之徒不受鎮北之號。恭表讓軍號,以超受為辭,而實惡其名,於是改號前將軍。慕容垂入青州,恭遣偏師禦之,失利,降號輔國將軍。



 及帝崩,會稽王道子執政,寵暱王國寶,委以機權。恭每正色直言,道子深憚而忿之。及赴山陵,罷朝,歎曰:「榱棟雖新,便有《黍離》之歎矣。」時國寶從弟緒說國寶,因恭入覲相王,伏兵殺之,國寶不許。而道子亦欲輯和內外,深布腹心於恭,冀除舊
 惡。恭多不順,每言及時政,輒厲聲色。道子知恭不可和協,王緒之說遂行,於是國難始結。或勸恭因人朝以兵誅國寶,而庾楷黨於國寶,士馬甚盛,恭憚之,不敢發,遂還鎮。臨別,謂道子曰:「主上諒闇,塚宰之任,伊周所難,願大王親萬機,納直言,遠鄭聲,放佞人。」辭色甚厲,故國寶等愈懼。以恭為安北將軍,不拜。乃謀誅國寶,遣使與殷仲堪、桓玄相結,仲堪偽許之。恭得書,大喜,乃抗表京師曰:「後將軍國寶得以姻戚頻登顯列,不能感恩效力,以報時施,而專寵肆威,將危社稷。先帝登遐,夜乃犯闔叩扉,欲矯遺詔。賴皇太后聰明,相王神武,故逆謀不果。又
 割東宮見兵以為己府,讒疾二昆甚於仇敵。與其從弟緒同黨兇狡,共相扇動。此不忠不義之明白也。以臣忠誠,必亡身殉國,是以譖臣非一。賴先帝明鑒,浸潤不行。昔趙鞅興甲,誅君側之惡,臣雖駑劣,敢忘斯義!」表至,內外戒嚴。國寶及緒惶懼不知所為,用王珣計,請解職。道子收國寶,賜死,斬緒于市,深謝愆失,恭乃還京口。



 恭之初抗表也,慮事不捷,乃版前司徒左長史王廞為吳國內史,令起兵於東。會國寶死,令廞解軍去職。廞怒,以兵伐恭。恭遣劉牢之擊滅之,上疏自貶,詔不許。譙王尚之復說道子以籓伯強盛,宰相權弱,宜多樹置以自衛。道
 子然之,乃以其司馬王愉為江州刺史,割庾楷豫州四郡使愉督之。由是楷怒,遣子鴻說恭曰:「尚之兄弟專弄相權,欲假朝威貶削方鎮,懲警前事,勢轉難測。及其議未成,宜早圖之。」恭以為然,復以謀告殷仲堪、桓玄。玄等從之,推恭為盟主,剋期同赴京師。



 時內外疑阻,津邏嚴急,仲堪之信因庾楷達之,以斜絹為書,內箭稈中,合鏑漆之,楷送於恭。恭發書,絹文角戾,不復可識,謂楷為詐。又料仲堪去年已不赴盟,今無動理,乃先期舉兵。司馬劉牢之諫曰:「將軍今動以伯舅之重,執忠貞之節,相王以姬旦之尊,時望所係,昔年已戮寶、緒,送王廞書,是深
 伏將軍也。頃所授用,雖非皆允,未為大失。割庾楷四郡以配王愉,於將軍何損!晉陽之師,其可再乎!」恭不從,乃上表以封王愉、司馬尚之兄弟為辭。朝廷使元顯及王珣、謝琰等距之。



 恭夢牢之坐其處,旦謂牢之曰:「事剋,即以卿為北府。」遣牢之率帳下督顏延先據竹里。元顯使說牢之,啖以重利,牢之乃斬顏延以降。是日,牢之遣其婿高雅之、子敬宣,因恭曜軍。輕騎擊恭。恭敗,將還,雅之已閉城門,恭遂與弟履單騎奔曲阿。恭久不騎乘,髀生瘡,不復能去。曲阿人殷確,恭故參軍也,以船載之,藏於葦席之下,將奔桓玄。至長塘湖,遇商人錢強。強宿憾於
 確,以告湖浦尉。尉收之,以送京師。道子聞其將至,欲出與語,面折之,而未之殺也。時桓玄等已至石頭,懼其有變,即於建康之倪塘斬之。恭五男及弟爽、爽兄子祕書郎和及其黨孟璞、張恪等皆殺之。



 恭性抗直。深存節義,讀《左傳》至「奉王命討不庭」,每輟卷而嘆。為性不弘,以暗於機會,自在北府,雖以簡惠為政,然自矜貴,與下殊隔。不閑用兵,尤信佛道,調役百姓,修營佛寺,務在壯麗,士庶怨嗟。臨刑,猶誦佛經,自理鬚鬢,神無懼容,謂監刑者曰:「我暗於信人,所以致此,原其本心,豈不忠於社稷!但令百代之下知有王恭耳。」家無財帛,唯書籍而已,為識
 者所傷。



 恭美姿儀,人多愛悅,或目之云「濯濯如春月柳」。嘗被鶴氅裘,涉雪而行,孟昶窺見之,歎曰:「此真神仙中人也!」初見執,遇故吏戴耆之為湖孰令,恭私告之曰:「我有庶兒未舉,在乳母家,卿為我送寄桓南郡。」耆之遂送之於夏口。桓玄撫養之,為立喪庭弔祭焉。及玄執政,上表理恭,詔贈侍中、太保,謚曰忠簡。爽贈太常,和及子簡並通直散騎郎,殷確散騎侍郎。腰斬湖浦尉及錢強等。恭庶子曇亨,義熙中為給事中。



 庾楷,征西將軍亮之孫,會稽內史羲小子也。初拜侍中,
 代兄準為西中郎將、豫州刺史、假節,鎮歷陽。隆安初,進號左將軍。時會稽王道子憚王恭、殷仲堪等擅兵,故出王愉為江州,督豫州四郡,以為形援。楷上疏以江州非險塞之地,而西府北帶寇戎,不應使愉分督,詔不許。時楷懷恨,使子鴻說王恭,以譙王尚之兄弟復握機權,勢過國寶。恭亦素忌尚之。遂連謀舉兵。事在恭傳。詔使尚之討楷。楷遣汝南太守段方逆尚之,戰于慈湖,方大敗,被殺,楷奔于桓玄。及玄等盟于柴桑,連名上疏自理,詔赦玄等而不赦恭、楷,楷遂依玄,玄用為武昌太守。楷後懼玄必敗,密遣使結會稽世子元顯:「若朝廷討玄,當為
 內應。」及玄得志,楷以謀泄,為玄所誅。



 劉牢之,字道堅,彭城人也。曾祖羲,以善射事武帝,歷北地、鴈門太守。父建,有武幹,為征虜將軍。世以壯勇稱。牢之面紫赤色,鬚目驚人,而沈毅多計畫。太元初,謝玄北鎮廣陵,時苻堅方盛,玄多募勁勇,牢之與東海何謙、瑯邪諸葛侃、樂安高衡、東平劉軌、西河田洛及晉陵孫無終等以驍猛應選。玄以牢之為參軍,領精銳為前鋒,百戰百勝,號為「北府兵」,敵人畏之。及堅將句難南侵,玄率何謙等距之。牢之破難輜重於盱眙,獲其運船,遷鷹揚
 將軍、廣陵相。



 時車騎將軍桓沖擊襄陽,宣城內史胡彬率眾向壽陽,以為沖聲援。牢之領卒二千,為彬後繼。淮肥之役,苻堅遣其弟融及驍將張蠔攻陷壽陽,謝玄使彬與牢之距之。師次硤石,不敢進。堅將梁成又以二萬人屯洛澗,玄遣牢之以精卒五千距之。去賊十里,成阻澗列陣。牢之率參軍劉襲、諸葛求等直進渡水,臨陣斬成及其弟雲,又分兵斷其歸津。賊步騎崩潰,爭赴淮水,殺獲萬餘人,盡收其器械。堅尋亦大敗,歸長安,餘黨所在屯結。牢之進平譙城,使安豐太守戴寶戍之。遷龍驤將軍、彭城內史,以功賜爵武岡縣男,食邑五百戶。牢之
 進屯鄄城,討諸未服,河南城堡承風歸順者甚眾。



 時苻堅子丕據鄴,為慕容垂所逼,請降,牢之引兵救之。垂聞軍至,出新城北走。牢之與沛郡太守田次之追之,行二百里,至五橋澤中,爭趣輜重,稍亂,為垂所擊,牢之敗績,士卒殲焉。牢之策馬跳五丈澗,得脫。會丕救至,因入臨漳,集亡散,兵復少振。牢之以軍敗徵還。頃之,復為龍驤將軍,守淮陰。後進戍彭城,復領太守。祅賊劉黎僭尊號於皇丘,牢之討滅之。苻堅將張遇遣兵擊破金鄉。圍太山太守羊邁,牢之遣參軍向欽之擊走之。會慕容垂叛將翟釗救遇,牢之引還。釗還,牢之進平太山,追釗於
 鄄城,釗走河北,因獲張遇以歸之彭城。襖賊司馬徽聚黨馬頭山,牢之遣參軍竺朗之討滅之。時慕容氏掠廩丘,高平太守徐含遠告急,牢之不能救,坐畏懦免。



 及王恭將討王國寶,引牢之為府司馬,領南彭城內史,加輔國將軍。恭使牢之討破王廞,以牢之領晉陵太守。恭本以才地陵物,及檄至京師,朝廷戮國寶、王緒,自謂威德已著,雖杖牢之為爪牙,但以行陣武將相遇,禮之甚薄。牢之負其才能,深懷恥恨。及恭之後舉,元顯遣廬江太守高素說牢之使叛恭,事成,當即其位號,牢之許焉。恭參軍何澹之以其謀告恭。牢之與澹之有隙,故恭疑而
 不納。乃置酒請牢之於眾中,拜牢之為兄,精兵利器悉以配之,使為前鋒。行至竹里,牢之背恭歸朝廷。恭既死,遂代恭為都督兗、青、冀、幽、並、徐、揚州、晉陵軍事。牢之本自小將,一朝據恭位,眾情不悅,乃樹用腹心徐謙之等以自強。時楊佺期、桓玄將兵上表理王恭,求誅牢之。牢之率北府之眾馳赴京師,次于新亭。玄等受詔退兵,牢之還鎮京口。



 及孫恩攻陷會稽,牢之遣將桓寶率師救三吳,復遣子敬宣為寶後繼。比至曲阿,吳郡內史桓謙已棄郡走,牢之乃率眾東討,拜表輒行。至吳,與衛將軍謝琰擊賊,屢勝,殺傷甚眾,徑臨浙江。進拜前將軍、
 都督吳郡諸軍事。時謝琰屯烏程,遣司馬高素助牢之。牢之率眾軍濟浙江,恩懼,逃于海。牢之還鎮,恩復入會稽,害謝琰。牢之進號鎮北將軍、都督會稽五郡,率眾東征,屯上虞,分軍戍諸縣。恩復攻破吳國,殺內史袁山松。牢之使參軍劉裕討之,恩復入海。頃之。恩浮海奄至京口,戰士十萬,樓船千餘。牢之在山陰,使劉裕自海鹽赴難,牢之率大眾而還。裕兵不滿千人,與賊戰,破之。恩聞牢之已還京口,乃走郁洲,又為敬宣、劉裕等所破。及恩死,牢之威名轉振。



 元興初,朝廷將討桓玄,以牢之為前鋒都督、征西將軍,領江州事。元顯遣使以討玄事諮牢
 之。牢之以玄少有雄名,杖全楚之眾,懼不能制,又慮平玄之後功蓋天下,必不為元顯所容,深懷疑貳,不得已率北府文武屯洌洲。桓玄遣何穆說牢之曰:「自古亂世君臣相信者有燕昭樂毅、玄德孔明,然皆勳業未卒而二主早世,設使功成事遂,未保二臣之禍也。鄙語有之:『高鳥盡,良弓藏;狡兔殫,獵犬烹。』故文種誅於句踐,韓白戮於秦漢。彼皆英雄霸王之主,猶不敢信其功臣,況凶愚凡庸之流乎!自開闢以來,戴震主之威,挾不賞之功,以見容於闇世者而誰?至如管仲相齊,雍齒侯漢,則往往有之,況君見與無射鉤屢逼之仇邪!今君戰敗則傾
 宗,戰勝亦覆族,欲以安歸乎?孰若翻然改圖,保其富貴,則身與金石等固,名與天壤無窮,孰與頭足異處,身名俱滅,為天下笑哉!惟君圖之。」牢之自謂握強兵,才能算略足以經綸江表,時譙王尚之已敗,人情轉沮,乃頗納穆說,遣使與玄交通。其甥何無忌與劉裕固諫之,並不從。俄令葆宣降玄。玄大喜,與敬宣置酒宴集,陰謀誅之,陳法書畫圖與敬宣共觀,以安悅其志。敬宣不之覺,玄佐吏莫不相視而笑。



 元顯既敗,玄以牢之為征東將軍、會稽太守,牢之乃歎曰:「始爾,便奪我兵,禍將至矣!」時玄屯相府,敬宣勸牢之襲玄,猶豫不決,移屯班瀆,將北奔
 廣陵相高雅之,欲據江北以距玄,集眾大議。參軍劉襲曰:「事不可者莫大於反,而將軍往年反王兗州,近日反司馬郎君,今復欲反桓公。一人而三反,豈得立也。」語畢,趨出,佐吏多散走。而敬宣先還京口拔其家,失期不到。牢之謂其為劉襲所殺,乃自縊而死。俄而敬宣至,不遑哭,奔于高雅之。將吏共殯斂牢之,喪歸丹徒。桓玄令斲棺斬首,暴尸於市,及劉裕建義,追理牢之,乃復本官。



 敬宣,牢之長子也。智略不及父,而技藝過之。孫恩之亂,隨父征討,所向有功。為元顯從事中郎,又為桓玄諮議參軍。牢之敗,與廣陵相高雅之俱奔慕容超,夢丸土而服
 之,既覺,喜曰:「丸者桓也,丸既吞矣,我當復土也。」旬日而玄敗,遂與司馬休之還京師。拜輔國將軍、晉陵太守。與諸葛長民破桓歆於芍陂,遷建威將軍、江州刺史,鎮尋陽。又擊桓亮、苻宏於湘中,所在有功。安帝反政,徵拜冠軍將軍、宣城內史,領襄城太守。譙縱反,以敬宣督征蜀軍事、假節,與寧朔將軍臧喜西伐。敬宣人自白帝,所攻皆剋。軍次黃獸,與偽將譙道福相持六十餘日,遇癘疫,又以食盡,班師,為有司所劾,免官。頃之,為中軍諮議,加冠軍將軍,尋遷鎮蠻護軍、安豐太守、梁國內史。會盧循反,以冠軍將軍從大軍南討。循平,遷左衛將軍、散
 騎常侍,又遷征虜將軍、青州刺史。尋改鎮冀州,為其參軍司馬道賜所害。



 殷仲堪,陳郡人也。祖融,太常、吏部尚書。父師,驃騎諮議參軍、晉陵太守、沙陽男。仲堪能清言,善屬文,每云三日不讀《道德論》,便覺舌本間強。其談理與韓康伯齊名,士咸愛慕之。調補佐著作郎。冠軍謝玄鎮京口,請為參軍。除尚書郎,不拜。玄以為長史,厚任遇之。仲堪致書於玄曰:



 胡亡之後,中原子女鬻於江東者不可勝數,骨肉星離,荼毒終年,怨苦之氣,感傷和理,誠喪亂之常,足以懲
 戒,復非王澤廣潤,愛育蒼生之意也。當世大人既慨然經略,將以救其塗炭,而使理至於此,良可嘆息!願節下弘之以道德,運之以神明,隱心以及物,垂理以禁暴,使足踐晉境者必無懷戚之心,枯槁之類莫不同漸天潤,仁義與干戈並運,德心與功業俱隆,實所期於明德也。



 頃聞抄掠所得,多皆採梠飢人,壯者欲以救子,少者志在存親,行者傾筐以顧念,居者吁嗟以待延。而一旦幽縶,生離死絕,求之於情,可傷之甚。昔孟孫獵而得麑,使秦西以之歸,其母隨而悲鳴,不忍而放之,孟孫赦其罪以傅其子。禽獸猶不可離,況於人乎!夫飛鴞,惡鳥也,食
 桑葚,猶懷好音。雖曰戎狄,其無情乎!茍感之有物,非難化也。必使邊界無貪小利,強弱不得相陵,德音一發,必聲振沙漠,二寇之黨,將靡然向風,何憂黃河之不濟,函谷之不開哉!



 玄深然之。



 領晉陵太守,居郡禁產子不舉,久喪不葬,錄父母以質亡叛者,所下條教甚有義理。父病積年,仲堪衣不解帶,躬學醫術,究其精妙,執藥揮淚,遂眇一目。居喪哀毀,以孝聞。服闋,孝武帝召為太子中庶子,甚相親愛。仲堪父嘗患耳聰,聞床下蟻動,謂之牛斗。帝素聞之而不知其人。至是,從容問仲堪曰:「患此者為誰?」仲堪流涕而起曰:「臣進退惟谷。」帝有愧焉。復領黃
 門郎,寵任轉隆。帝嘗示仲堪侍,乃曰:「勿以己才而笑不才。」帝以會稽王非社稷之臣,擢所親幸以為籓捍,乃授伸堪都督荊益寧三州軍事、振威將軍、荊州刺史、假節,鎮江陵。將之任,又詔曰:「卿去有日,使人酸然。常謂永為廊廟之寶,而忽為荊楚之珍,良以慨恨!」其恩狎如此。



 仲堪雖有英譽,議者未以分陜許之。既受腹心之任,居上流之重,朝野屬想,謂有異政。及在州,綱目不舉,而好行小惠,夷夏頗安附之。先是,仲堪游於江濱,見流棺,接而葬焉。旬日間,門前之溝忽起為岸。其夕,有人通仲堪,自稱徐伯玄,云:「感君之惠,無以報也。」仲堪因問:「門前之岸
 是何祥乎?」對曰:「水中有岸,其名為洲,君將為州。」言終而沒。至是,果臨荊州。桂陽人黃欽生父沒已久,詐服衰麻,言迎父喪。府曹先依律詐取父母卒棄市,仲堪乃曰:「律詐取父母寧依驅詈法棄市。原此之旨,當以二親生存而橫言死沒,情事悖逆,忍所不當,故同之驅詈之科,正以大辟之刑。今欽生父實終沒,墓在舊邦,積年久遠,方詐服迎喪,以此為大妄耳。比之於父存言亡,相殊遠矣。」遂活之。又以異姓相養,禮律所不許,子孫繼親族無後者,唯令主其蒸嘗,不聽別籍以避役也。佐史咸服之。



 時朝廷徵益州刺史郭銓,犍為太守卞苞於坐勸銓以蜀
 反,仲堪斬之以聞。朝廷以仲堪事不預察,降號鷹揚將軍。尚書下以益州所統梁州三郡人丁一千番戍漢中,益州未肯承遣。仲堪乃奏之曰:



 夫制險分國,各有攸宜,劍閣之隘,實蜀之關鍵。巴西、梓潼、宕渠三郡去漢中遼遠,在劍閣之內,成敗與蜀為一,而統屬梁州,蓋定鼎中華,慮在後伏,所以分斗絕之勢,開荷戟之路。自皇居南遷,守在岷邛,衿帶之形,事異曩昔。是以李勢初平,割此三郡配隸益州,將欲重復上流為習坎之防。事經英略,歷年數紀。梁州以統接曠遠,求還得三郡,忘王侯設險之義,背地勢內外之實,盛陳事力之寡弱,飾哀矜之苦
 言。今華陽乂清,隴順軌,關中餘燼,自相魚肉,梁州以論求三郡,益州以本統有定,更相牽制,莫知所從。致令巴、宕二郡為群獠所覆,城邑空虛,士庶流亡,要害膏腴皆為獠有。今遠慮長規,宜保全險塞。又蠻獠熾盛,兵力寡弱,如遂經理乖謬,號令不一,則劍閣非我保,醜類轉難制。此乃籓扞之大機,上流之至要。



 昔三郡全實,正差文武三百,以助梁州。今俘沒蠻獠,十不遺二,加逐食鳥散,資生未立,茍順符指以副梁州,恐公私困弊,無以堪命,則劍閣之守無擊柝之儲,號令選用不專於益州,虛有監統之名,而無制御之用,懼非分位之本旨,經國之
 遠術。謂今正可更加梁州文武五百,合前為一千五百,自此之外,一仍舊貫。設梁州有急,蜀當傾力救之。



 書奏,朝廷許焉。



 桓玄在南郡,論四皓來儀漢庭,孝惠以立,而惠帝柔弱,呂后凶忌,此數公者,觸彼埃塵,欲以救弊。二家之中,各有其黨,奪彼與此,其仇必興。不知匹夫之志,四公何以逃其患?素履終吉,隱以保生者,其若是乎!以其文贈仲堪。仲堪乃答之曰:



 隱顯默語,非賢達之心,蓋所遇之時不同,故所乘之途必異。道無所屈而天下以之獲寧,仁者之心未能無感。若夫四公者,養志巖阿,道高天下,秦網雖虐,游之而莫懼,漢祖雖雄,請之而弗顧,
 徒以一理有感,泛然而應,事同賓客之禮,言無是非之對,孝惠以之獲安,莫由報其德,如意以之定籓,無所容其怨。且爭奪滋生,主非一姓,則百姓生心,祚無常人,則人皆自賢,況夫漢以劍起,人未知義,式遏姦邪,特宜以正順為寶。天下,大器也,茍亂亡見懼,則滄海橫流。原夫若人之振策,豈為一人之廢興哉!茍可以暢其仁義,與夫伏節委質可榮可辱者,道迹懸殊,理勢不同,君何疑之哉!



 又謂諸呂強盛,幾危劉氏,如意若立,必無此患。夫禍福同門,倚伏萬端,又未可斷也。于時天下新定,權由上制,高祖分王子弟,有磐石之固,社稷深謀之臣,森然
 比肩,豈瑣瑣之祿產所能傾奪之哉!此或四公所預,于今亦無以辯之,但求古賢之心,宜存之遠大耳。端本正源者,雖不能無危,其危易持。茍啟競津,雖未必不安,而其安難保。此最有國之要道。古今賢哲所同惜也。



 玄屈之。



 仲堪自在荊州,連年水旱,百姓饑饉,仲堪食常五碗,盤無餘肴,飯粒落席間,輒拾以啖之,雖欲率物,亦緣其性真素也。每語子弟云:「人物見我受任方州,謂我豁平昔時意,今吾處之不易。貧者士之常,焉得登枝而捐其本?爾其存之!」其後蜀水大出,漂浮江陵數千家。以隄防不嚴,復降為寧遠將軍。安帝即位,進號冠軍將軍,固讓
 不受。



 初,桓玄將應王恭,乃說仲堪,推恭為盟主,共興晉陽之舉,立桓文之功,仲堪然之。仲堪以王恭在京口,去都不盈二百,自荊州道遠連兵,勢不相及,乃偽許恭,而實不欲下。聞恭已誅王國寶等,始抗表興師,遣龍驤將軍楊佺期次巴陵。會稽王道子遣書止之,仲堪乃還。



 初,桓玄棄官歸國,仲堪憚其才地,深相交結。玄亦欲假其兵勢,誘而悅之。國寶之役,仲堪既納玄之誘,乃外結雍州刺史郗恢,內要從兄南蠻校尉顗、南郡相江績等。恢、顗、績並不同之,乃以楊佺期代績,顗自遜位。



 會王恭復與豫州刺史庾楷舉兵討江州刺史王愉及譙王尚之
 等,仲堪因集議,以為朝廷去年自戮國寶,王恭威名已震,今其重舉,勢無不剋。而我去年緩師,已失信於彼,今可整棹晨征,參其霸功。於是使佺期舟師五千為前鋒,桓玄次之。仲堪率兵二萬,相繼而下。佺期、玄至湓口,王愉奔于臨川,玄遣偏軍追獲之。佺期等進至橫江,庾楷敗奔於玄,譙王尚之等退走,尚之弟恢之所領水軍皆沒。玄等至石頭,仲堪至蕪湖,忽聞王恭已死,劉牢之反恭,領北府兵在新亭,玄等三軍失色,無復固志,乃迴師屯于蔡洲。



 時朝廷新平恭、楷,且不測西方人心,仲堪等擁眾數萬,充斥郊畿,內外憂逼。玄從兄修告會稽王道
 子曰:「西軍可說而解也。修知其情矣。若許佺期以重利,無不倒戈於仲堪者。」道子納之,乃以玄為江州,佺期為雍州,黜仲堪為廣州,以桓修為荊州,遣仲堪叔父太常茂宣詔回軍。仲堪恚被貶退,以王恭雖敗,己眾亦足以立事,令玄等急進軍。玄等喜於寵授,並欲順朝命,猶豫未決。會仲堪弟遹為佺期司馬,夜奔仲堪,說佺期受朝命,納桓修。仲堪遑遽,即於蕪湖南歸,使徇於玄等軍曰:「若不各散而歸,大軍至江陵,當悉戮餘口。」仲堪將劉系先領二千人隸于佺期,輒率眾而歸。玄等大懼,狼狽追仲堪,至尋陽,及之。於是仲堪失職,倚玄為援,玄等又資
 仲堪之兵,雖互相疑阻,亦不得異。仲堪與佺期以子弟交質,遂於尋陽結盟,玄為盟主,臨壇歃血,並不受詔,申理王恭,求誅劉牢之、譙王尚之等。朝廷深憚之。於是詔仲堪曰:「間以以將軍憑寄失所,朝野懷憂。然既往之事,宜其兩忘,用乃班師回旆,祗順朝旨,所以改授方任,蓋隨時之宜。將軍大義,誠感朕心,今還復本位,即撫所鎮,釋甲休兵,則內外寧一,故遣太常茂具宣乃懷。」仲堪等並奉詔,各旋所鎮。



 頃之。桓玄將討佺期,先告仲堪云:「今當人沔討除佺期,已頓兵江口。若見與無貳,可殺楊廣;若其不然,便當率軍入江。」仲堪乃執玄兄偉,遣從弟遹等
 水軍七千至江西口。玄使郭銓、苻宏擊之,遹等敗走。玄頓巴陵,而館其穀。玄又破楊廣於夏口。仲堪既失巴陵之積,又諸將皆敗,江陵震駭。城內大飢,以胡麻為廩。仲堪急召佺期。佺期率眾赴之,直濟江擊玄,為玄所敗,走還襄陽。仲堪出奔酂城,為玄追兵所獲,逼令自殺,死于柞溪,弟子道獲、參軍羅企生等並被殺。仲堪少奉天師道,又精心事神,不吝財賄,而怠行仁義,嗇於周急,及玄來攻,猶勤請禱。然善取人情,病者自為診脈分藥,而用計倚伏煩密,少於鑒略,以至於敗。



 子簡之,載喪下都,葬于丹徒,遂居墓側。義旗建,率私僮客隨義軍躡桓玄。玄
 死,簡之食其肉。桓振之役,義軍失利,簡之沒陣。弟曠之,有父風,仕至剡令。



 楊佺期,弘農華陰人,漢太尉震之後也。曾祖準,太常。自震至準,七世有名德。祖林,少有才望,值亂沒胡。父亮,少仕偽朝,後歸國,終於梁州刺史,以貞幹知名。佺期沈勇果勁,而兄廣及弟思平等皆強獷粗暴。自云門戶承籍,江表莫比,有以其門地比王珣者,猶恚恨,而時人以其晚過江,婚宦失類,每排抑之,恒慷慨切齒,欲因事際以逞其志。



 佺期少仕軍府。咸康中,領眾屯成固。苻堅將潘
 猛距守康回壘,佺期擊走之,其眾悉降,拜廣威將軍、河南太守,戍洛陽。苻堅將竇衝率眾攻平陽太守張元熙於皇天塢,佺期擊走之。佺期自湖城入潼關,累戰皆捷,斬獲千計,降九百餘家,歸於洛陽,進號龍驤將軍。以病,改為新野太守,領建威司馬。遷唐邑太守,督石頭軍事,以疾去職。荊州刺史殷仲堪引為司馬,代江績為南郡相。



 仲堪與恆玄舉眾應王恭、庾楷,仲堪素無戎略,軍旅之事一委佺期兄弟,以兵五千人為前鋒,與桓玄相次而下。至石頭,恭死,楷敗,朝廷未測玄軍,乃以佺期代郗恢為都督梁雍秦三州諸軍事、雍州刺史。仲堪、玄皆有
 遷換,於是俱還尋陽,結盟不奉詔。俄而朝廷復仲堪本職,乃各還鎮。



 初,玄未奉詔,欲自為雍州,以郗恢為廣州。恢懼玄之來,問於眾,咸曰:「佺期來者,誰不戮力!若桓玄來,恐難與為敵。」既知佺期代己,乃謀於南陽太守閭丘羨,稱兵距守。佺期慮事不濟,乃聲言玄來入沔,而佺期為前驅。恢眾信之,無復固志。恢軍散請降,佺期入府斬閭丘羨,放恢還都,撫將士,恤百姓,繕修城池,簡練甲卒,甚得人情。



 佺期、仲堪與桓玄素不穆,佺期屢欲相攻,仲堪每抑止之。玄以是告執政,求廣其所統。朝廷亦欲成其釁隙,故以桓偉為南蠻校尉。佺期內懷忿懼,勒兵建
 牙,聲云援洛,欲與仲堪襲玄。仲堪雖外結佺期,內疑其心,苦止之,又遣從弟遹屯北塞以駐之。佺期勢不獨舉,乃解兵。



 隆安三年,桓玄遂舉兵討佺期,先攻仲堪。初,仲堪得玄書,急召佺期。佺期曰:「江陵無食,當何以待敵?可來見就,共守襄陽。」仲堪自以保境全軍,無緣棄城逆走,憂佺期不赴,乃紿之曰:「比來收集,已有儲矣。」佺期信之,乃率眾赴焉。步騎八千,精甲耀日。既至,仲堪唯以飯餉其軍。佺期大怒曰:「今茲敗矣!」乃不見仲堪。時玄在零口,佺期與兄廣擊玄。玄畏佺期之銳,乃渡軍馬頭。明日,佺期率殷道護等精銳萬人乘艦出戰,玄距之,不得進。佺期
 乃率其麾下數十艦,直濟江,徑向玄船。俄而迴擊郭銓,殆獲銓,會玄諸軍至,佺期退走,餘眾盡沒,單馬奔襄陽。玄追軍至,佺期與兄廣俱死之,傳首京都,梟於朱雀門。弟思平,從弟尚保、孜敬,俱逃于蠻。劉裕起義,始歸國,歷位州郡。



 孜敬為人剽銳,果於行事。昔與佺期勸殷仲堪殺殷顗,仲堪不從,孜敬拔刃而起,欲自己出取之,仲堪苦禁乃止。及為梁州刺史,常怏怏不滿其志。經襄陽,見魯宗之侍衛皆佺期之舊也,孜敬愈憤,見於辭色。宗之參軍劉千期於座面折之,因大發怒,抽劍刺千期立死。宗之表而斬之。思平、尚保後亦以罪誅,楊氏遂滅。



 史臣曰:生靈道斷,忠貞路絕,棄彼弊冠,崇茲新履。牢之事非其主,抑亦不臣,功多見疑,勢陵難信,而投兵散地,二三之甚。若夫司牧居愆,方隅作戾,口順勤王,心乖抗節。王恭鯁言時政,有昔賢之風。國寶就誅,而晉陽猶起。是以仲堪僥幸,佺期無狀,雅志多隙,佳兵不和,足以亡身,不足以靜亂也。



 贊曰:孝伯懷功,牢之總戎。王因起釁,劉亦慚忠。殷楊乃武,抽旆爭雄。庾君含怨,交鬥其中。猗歟群採,道睽心異。是曰亂階,非關臣事。



\end{pinyinscope}