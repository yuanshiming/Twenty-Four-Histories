\article{列傳第六十三}

\begin{pinyinscope}

 外戚



 羊琇王恂楊文宗羊玄之虞豫庾琛杜
 乂褚裒何準王濛王遐王蘊褚爽



 詳觀往誥,逖聽前聞,階緣外戚以致顯榮者,其所由來尚矣。而多至禍敗,鮮克令終者,何哉?豈不由祿以恩升,位非德舉;識慚明悊。材謝經通;假椒房之寵靈,總軍國之樞要。或威權震主,或勢力傾朝;居安而不慮危,務進而不知退;驕奢既至,釁隙隨之者乎!是以呂霍之家,誅夷於西漢,梁鄧之族,剿絕於東都,其餘干紀亂常、害時
 蠹政者,不可勝載。至若樊靡卿之父子,竇廣國之弟兄,陰興之守約戒奢,史丹之掩惡揚善,斯並后族之所美者也。由此觀之,乾時縱溢者必以凶終,守道謙沖者永保貞吉,古人所謂禍福無門,惟人所召,此非其效歟!



 逮于晉難,始自宮掖。楊駿藉武帝之寵私,叨竊非據,賈謐乘惠皇之蒙昧,成此厲階,遂使悼后遇雲林之災,愍懷濫湖城之酷。天人道盡,喪亂弘多,宗廟以之顛覆,黎庶於焉殄瘁。《詩》云:「赫赫宗周,褒姒滅之。」其此之謂也。爰及江左,未改覆車。庾亮世族羽儀,王恭高門領袖,既而職兼出納,任切股肱。孝伯竟以亡身,元規幾於敗國,豈不
 哀哉!若褚季野之畏避朝權,王叔仁之固求出鎮,用能全身遠害,有可稱焉。賈充、楊駿、庾亮、王獻之、王恭等已入列傳,其餘既敘其成敗,以為《外戚篇》云。



 羊琇,字稚舒,景獻皇后之從父弟也。父耽,官至太常。兄瑾,尚書右僕射。琇少舉郡計,參鎮西鐘會軍事,從平蜀。及會謀反,琇正言苦諫,還,賜爵關內侯。琇涉學有智算,少與武帝通門,甚相親狎,每接筵同席,嘗謂帝曰:「若富貴見用,任領護各十年。」帝戲而許之。初,帝未立為太子,而聲論不及弟攸,文帝素意重攸,恒有代宗之議。琇密
 為武帝畫策,甚有匡救。又觀察文帝為政損益,揆度應所顧問之事,皆令武帝默而識之。其後文帝與武帝論當世之務及人間可否,武帝答無不允,由是儲位遂定。及帝為撫軍,命琇參軍事。帝即王位後,擢琇為左衛將軍,封甘露亭侯。帝踐阼,累遷中護軍,加散騎常侍。琇在職十三年,典禁兵,豫機密,寵遇甚厚。



 初,杜預拜鎮南將軍,朝士畢賀,皆連榻而坐。琇與裴楷後至,曰:「杜元凱乃復以連榻而坐客邪?」遂不坐而去。



 琇性豪侈,費用無復齊限,而屑炭和作獸形以溫酒,洛下豪貴咸競效之。又喜遊燕,以夜續晝,中外五親無男女之別,時人譏之。然
 黨慕勝己,其所推舉,便盡心無二。窮窘之徒,特能振恤。選用多以得意者居先,不盡銓次之理。將士有冒官位者,為其致節,不惜軀命。然放恣犯法,每為有司所貸。其後司隸校尉劉毅劾之,應至重刑,武帝以舊恩,直免官而已。尋以侯白衣領護軍。頃之,復職。及齊王攸出鎮也,琇以切諫忤旨,左遷太僕。既失寵憤怨,遂發病,以疾篤求退。拜特進,加散騎常侍,還第,卒。帝手詔曰:「琇與朕有先后之親,少小之恩,歷位外內,忠允茂著。不幸早薨,朕甚悼之。其追贈輔國大將軍、開府儀同三司,賜東園祕器,朝服一襲,錢三十萬,布百匹。」謚曰威。



 王恂,字良夫,文明皇后之弟也。父肅,魏蘭陵侯。恂文義通博,在朝忠正,累遷河南尹,建立二學,崇明《五經》。鬲令袁毅嘗饋以駿馬,恂不受。及毅敗,受貨者皆被廢黜焉。魏氏給公卿已下租牛客戶數各有差,自後小人憚役,多樂為之,貴勢之門動有百數。又太原諸部亦以匈奴胡人為田客,多者數千。武帝踐位,詔禁募客,恂明峻其防,所部莫敢犯者。咸寧四年卒,贈車騎將軍。恂弟虔、愷。



 虔字恭祖。以功幹見稱,累遷衛尉,封安壽亭侯,拜平東將軍、假節、監青州諸軍事。徵為光祿勳,轉尚書,卒。子士
 文嗣,歷右衛將軍、南中郎將,鎮許昌,為劉聰所害。



 愷字君夫。少有才力,歷位清顯,雖無細行,有在公之稱。以討楊駿勳,封山都縣公,邑千八百戶。遷龍驤將軍,領驍騎將軍,加散騎常侍,尋坐事免官。起為射聲校尉,久之,轉後將軍。愷既世族國戚,性復豪侈,用赤石脂泥壁。石崇與愷將為鴆毒之事,司隸校尉傅祗劾之,有司皆論正重罪,詔特原之。由是眾人僉畏愷,故敢肆其意,所欲之事無所顧憚焉。及卒,謚曰醜。



 楊文宗,武元皇后父也。其先事漢,四世為三公。文宗為
 魏通事郎,襲封蓩亭侯。早卒,以后父,追贈車騎將軍,謚曰穆。



 羊玄之,惠皇后父,尚書右僕射瑾之子也。玄之初為尚書郎,以后父,拜光祿大夫、特進、散騎常侍,更封興晉侯。遷尚書右僕射,加侍中,進爵為公。成都王穎之攻長沙王乂也,以討玄之為名,遂憂懼而卒。追贈車騎將軍、開府儀同三司。



 虞豫,元敬皇后父也。少有美稱,州郡禮辟,並不就。拜南
 陽王文學。早卒。明帝即位,追贈散騎常侍、驃騎大將軍、開府儀同三司、平山縣侯。子胤嗣。



 胤,敬后弟也。初拜散騎常侍,遷步兵校尉。太寧末,追贈豫官,以胤襲侯爵,轉右衛將軍。與南頓王宗俱為明帝所暱,並典禁兵。及帝不豫,宗以陰謀發覺,事連胤,帝隱忍不問,徙胤為宗正卿,加散騎常侍。咸和二年,宗伏誅,左遷胤為桂陽太守,秩中二千石。頻徙瑯邪、盧陵太守。咸康元所卒,追贈衛將軍,加散騎常侍。子洪襲爵。



 庾琛,字子美,明穆皇后父也。兄袞,在《孝友傳》。琛永嘉初
 為建威將軍,過江,為會稽太守,徵為丞相軍諮祭酒。卒官,以後父追贈左將軍,妻毌丘氏追封鄉君,子亮陳先志不受。咸和中,成帝又下詔追贈琛驃騎將軍、儀同三司,亮又辭焉。亮在列傳。



 杜乂,字弘理,成恭皇后父,鎮南將軍預孫,尚書左丞錫之
 子也。性純和,美姿容,有盛名於江左。王羲之見而目之曰:「膚若凝脂,眼如點漆,此神仙人也。」桓彞亦曰:「衛玠神清,杜乂形清。」襲封當陽侯,闢公府掾,為丹陽丞。早卒,無男,生後而乂終,妻裴氏嫠居養後,以禮自防,甚有德音。咸康初,追贈金紫光祿大夫,謚曰穆。封裴氏為高安鄉君,邑五百戶。至孝武帝時,崇進為廣德縣君。裴氏壽考,百姓號曰杜姥。初,司徒蔡謨甚器重乂,嘗言於朝曰:「恨諸君不見杜乂也。」其為名流所重如此。



 褚裒,字季野,康獻皇后父也。祖,有局量,以干用稱。嘗為縣吏,事有不合,令欲鞭之,曰:「物各有所施,
 榱椽之材不合以為籓落也,願明府垂察。」乃舍之。家貧,辭吏。年垂五十,鎮南將軍羊祜與有舊,言於武帝,始被升用,官至安東將軍。父洽,武昌太守。



 裒少有簡貴之風,與京兆杜乂俱有盛名,冠於中興。譙國桓彞見而目之曰:「季野有皮裏春秋。」言其外無臧否,而內有所褒貶也。謝安亦雅重之,恆云:「裒雖不言,而四時之氣亦備矣。」初闢西陽王掾、吳王文學。蘇峻之構逆也,車騎將軍郗鑒以裒為參軍。峻平,以功封都鄉亭侯,稍遷司徒從事中郎,除給事黃門侍郎。康帝為瑯邪王時,將納妃,妙選素望,詔娉裒女為妃,於是出為豫章太守。及康帝即位,徵拜侍中,遷尚書。以後父,苦求外出,除建威將軍、江州刺
 史,鎮半洲。在官清約,雖居方伯,恆使私童樵採。頃之,徵為衛將軍,領中書令。裒以中書銓管詔命,不宜以姻戚居之,固讓,詔以為左將卦、兗州刺史、都督兗州徐州之瑯邪諸軍事、假節,鎮金城,又領瑯邪內史。



 初,裒總角詣庾亮,亮使郭璞筮之。卦成,璞駭然,亮曰:「有不祥乎?」璞曰:「此非人臣卦,不知此年少何以乃表斯祥?二十年外,吾言方驗。」及此二十九年而康獻皇太后臨朝,有司以裒皇太后父,議加不臣之禮,拜侍中、衛將軍、錄尚書事,持節、都督、刺史如故。裒以近戚,懼獲譏嫌,上疏固請居籓,曰:「臣以虛鄙,才不周用,過蒙國恩,累忝非據。無勞受寵,負愧實深,豈可復加殊特之命,顯號重疊!臣有何勛可以克堪?何顏可以冒進?委身聖世,豈復遺力,實懼顛墜,所誤者大。今王略未振,萬機至殷,陛下
 宜委誠宰輔,一遵先帝任賢之道,虛己受成,坦平心於天下,無宜內示私親之舉,朝野失望,所損豈少!」於是改授都督徐兗青揚州之晉陵吳國諸軍事、衛將軍、徐兗二州刺史、假節、鎮京口。



 永和初,復徵裒,將以為揚州、錄尚書事。吏部尚書劉遐說裒曰:「會稽王令德,國之周公也,足下宜以大政付之。」裒長史王胡之亦勸焉,於是固辭歸籓,朝野咸嘆服之。進號征北大將軍、開府儀同三司,固辭開府。裒又以政道在於得才,宜委賢任能,升敬舊齒,乃薦前光祿大夫顧和、侍中殷浩。疏奏,即以和為尚書令,
 浩為揚州刺史。



 及石季龍死,裒上表請伐之,即日戒嚴,直指泗口。朝議以裒事任貴重,不宜深入,可先遣偏師。裒重陳前所遣前鋒督護王頤之等徑造彭城,示以威信,後遣督護麋嶷進軍下邳,賊即奔潰,嶷率所領據其城池,今宜速發,以成聲勢,於是除征討大都督青、揚、徐、兗、豫五州諸軍事。裒率眾三萬徑進彭城,河朔士庶歸降者日以千計,裒撫納之,甚得其歡心。先遣督護徐龕伐沛,獲偽相支重,郡中二千餘人歸降。魯郡山有五百餘家,亦建義請援,裒遣龕領銳卒三千迎之。龕違裒節度,軍次代陂,
 為石遵將李菟所敗,死傷太半,龕執節不撓,為賊所害。裒以《春秋》責帥,授任失所,威略虧損,上疏自貶,以征北將軍行事,求留鎮廣陵。詔以偏帥之責,不應引咎,逋寇未殄,方鎮任重,不宜貶降,使還鎮京口,解征討都督。



 時石季龍新死,其國大亂,遺戶二十萬口渡河,將歸順,乞師救援。會裒已旋,威勢不接,莫能自拔,皆為慕容皝及苻健之眾所掠,死亡咸盡。裒以遠圖不就,憂慨發病。及至京口,聞哭聲甚眾,裒問:「何哭之多?」左右曰:「代陂之役也。」裒益慚恨。永和五年卒,年四十七,遠近嗟悼,吏士哀慕之。贈侍中、太傅,本官如故,謚曰元穆。子歆,字幼安,以學行知名,歷散騎常侍、秘書監。



 何準,字幼道,穆章皇后父也。高尚寡欲,弱冠知名,州府交闢,並不就。兄充為驃騎將軍,勸其令仕,準曰:「第五之名何滅驃騎?」準兄弟中第五,故有
 此言。充居宰輔之重,權傾一時,而準散帶衡門,不及人事,唯誦佛經,修營塔廟而已。徵拜散騎郎,不起。年四十七卒。升平元年,追贈金紫光祿大夫,封晉興縣侯。子惔以父素行高潔,表讓不受。三子:放、惔、澄。



 放繼充。



 惔官至南康太守,早卒。惔子元度,西陽太守;次叔度,太常卿、尚書。



 澄字季玄,起家秘書郎,轉丞,清正有器望,累遷秘書監、太常、中護軍。孝武帝深愛之,以為冠軍將軍、吳國內史。太元末,瑯邪王出居外第,妙選師傅,徵拜尚書,領瑯邪王師。安帝即位,遷尚書左僕射,典選、王師如故。時澄腳疾,固讓,特聽不朝,坐家視事。又領本州大中正。及桓玄執政,以疾奏免,卒於家。安帝反正,追贈金紫光祿大夫。長子籍,早卒。次子融,元熙中,為大司農。



 王濛,字仲祖,哀靖皇后父也。曾祖黯,歷位尚書。祖佑,北軍中候。父訥,新淦令。濛少時放縱不羈,不為鄉曲所齒,晚節始克己勵行,有風流美譽,虛己應物,恕而後行,莫不敬愛焉。事諸母甚謹,奉祿資產常推厚居薄,喜慍不形於色,不修小潔,而以清約見稱。善隸書。美姿容,嘗覽鏡自照,稱其父字曰:「王文開生如此兒邪!」居貧,帽敗,自入市買之,嫗悅其
 貌,遺以新帽,時人以為達。與沛國劉惔齊名友善,惔常稱濛性至通,而自然有節,濛每云:「劉君知我,勝我自知。」時人以惔方荀奉倩,濛比袁曜卿,凡稱風流者,舉濛、惔為宗焉。



 司徒王導闢為掾。導復引匡術弟孝,濛致箋于導曰:「開國承家,小人勿用。杖德義以尹天下,方將澄清彞倫,崇重名器。夫軍國殊用,文武異容,豈可令涇渭混流,虧清穆之風,以允答具瞻,儀形海內!」導不答。後出補長山令,復為司徒左西屬。蒙以此職有譴則應受杖,固辭。詔為停罰,猶不就。徙中書郎。



 簡文帝之為會稽王也,嘗與孫綽商略諸風流人,綽言曰:「劉惔清蔚簡令,王濛溫潤恬和,桓溫高爽邁出。謝尚清易令達,而濛性和暢,能言理,辭簡而有會。」及簡文帝輔政,益貴幸之,與劉惔號為入室之賓。轉司徒左長史。晚求為東陽,不許。及蒙病,乃恨不用之。濛聞之曰:「人言會稽王癡,竟癡也!」疾漸篤,於燈下轉麈尾視之,歎曰:「如此人曾不得四十也!」年三十九卒。臨殯,劉惔以犀杷麈尾置棺中,因慟絕久之。謝安亦常稱濛云:「王長史語甚不
 多,可謂有令音。」有二子:脩、蘊。



 脩字敬仁,小字茍子。明秀有美稱,善隸書,號曰流奕清舉。年十二,作《賢全論》。蒙以示劉惔曰:「敬仁此論,便足以參微言。」起家著作郎、瑯邪王文學,轉中軍司馬,未拜而卒,年二十四。臨終,歎曰:「無愧古人,年與之齊矣。」



 王遐,字桓子,簡順皇后父,驃騎將軍述之從叔也。少以華族,仕至光祿勛。寧康初,追贈特進、光祿大夫,加散騎常侍,謚曰靖。



 長子恪,領軍將軍。恪子欣之,豫章太守,秩中二千石。欣之弟歡之,廣州刺史。遐少子臻,崇德衛尉。



 王蘊,字叔仁,孝武定皇后父,司徒左長史濛之子也。起家佐著作郎,累遷尚書吏部郎。性平和,不抑寒素,每一官缺,求者十輩,蘊無所是非。時簡文帝為會稽王,輔政,蘊輒連狀白之,曰:「某人有地,某人有才。」務存進達,各隨其方,故不得者無怨焉。補吳興太守,甚有德政。屬郡荒人飢,輒開倉贍恤。主簿執諫,請先列表上待報,蘊曰:「今百姓嗷然,路有饑饉,若表上須報,何以救將死之命乎!專輒之愆,罪在太守,且行仁義而敗,無所恨也。」於是大振貸之,賴蘊全者十七八焉。朝廷以違科免蘊官,士庶
 詣闕訟之,詔特左降晉陵太守。復有惠化,百姓歌之。



 定后立,以后父,遷光祿大夫,領五兵尚書、本州大中正,封建昌縣侯。蘊以恩澤賜爵,非三代令典,固辭不受。朝廷敦勸,終不肯拜,乃授都督京口諸軍事、左將軍、徐州刺史、假節,復固讓。謝安謂蘊曰:「卿居后父之重,不應妄自菲薄,以虧時遇,宜依褚公故事,但令在貴權於事不事耳。可暫臨此任,以紓國姻之重。」於是乃受命,鎮於京口。頃之,徵拜尚書左僕射,將軍如故,遷丹陽尹,即本軍號加散騎常侍。蘊以姻戚,不欲在內,苦求外出,復以為都督浙江東五郡、鎮軍將軍、會稽內史,常侍如故。



 蘊素嗜
 酒,末年尤甚。及在會稽,略少醒日,然猶以和簡為百姓所悅。時王悅來拜墓,蘊子恭往省之,素相善,遂留十餘日方還。蘊問其故,恭曰:「與阿太語,蟬連不得歸。」蘊曰:「恐阿太非爾之友。」阿太,悅小字也。後竟乖初好,時以為知人。太元九年卒,年五十五,追贈左光祿大夫、開府儀同三司。



 長子華,早卒。次恭,在列傳。恭弟爽,字季明,彊正有志力,歷給事黃門侍郎、侍中。孝武帝崩,王國寶夜欲開門入為遺詔,爽距之,曰:「大行晏駕,皇太子未至,敢入者斬!」乃止。爽嘗與會稽王道子飲,道子醉呼爽為小子,爽曰:「亡祖長史與簡文皇帝為布衣之交。亡姑、亡姊伉儷
 二宮,何小子之有!」及國寶執權,免爽官。後兄恭再起事,並以爽為寧朔將軍,參預軍事。恭敗,被誅。



 褚爽,字弘茂,小字斯生,恭思皇后父也。祖裒,父歆。爽少有令稱,謝安甚重之,嘗曰:「若期生不佳,我不復論士矣。」為義興太守,早卒,以后父,追贈金紫光祿大夫。爽子秀之、炎之、喻之,義熙中,並歷大官。



 史臣曰:羊琇託肺腑之親,處多聞之益,遭逢潛躍之際,預參經始之謀,故得繾綣恩私,便蕃任遇。憑寵靈而逞欲,恃勢位而驕陵,屢犯憲章,頻干國紀,幸逢寬政,得免
 刑書。王愷地即渭陽,家承世祿,曾弗聞於恭儉,但崇縱於奢淫,競爽於季倫,爭先於武子,既塵清論,有斁王猷,雖復議行易名,未足懲惡勸善。弘理儀形外朗,季野神鑒內融,仲祖溫潤風流,幼道清虛寡慾,皆擅名江表,見重當時,豈惟后族之英華,抑亦搢紳之令望者也。



 贊曰:託屬丹掖,承輝紫宸。地既權寵,任惟執鈞。約乃寡失,驕則陵人。覆車遺戒,諒足書紳。



\end{pinyinscope}