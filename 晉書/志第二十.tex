\article{志第二十}

\begin{pinyinscope}

 刑法


傳曰:「
 齊之以禮,有恥且格。」刑之不可犯,不若禮之不可踰,則昊歲比於犧年,宜有降矣。若夫穹圓肇判,宵貌攸分,流形播其喜怒,稟氣彰其善惡,則有自然之理焉。念室後刑,衢樽先惠,將以屏除災害,引導休和,取譬琴瑟,不忘銜策,擬陽秋之成化,若堯舜之為心也。效原布肅,軒皇有轡野之師;雷電揚威,高辛有觸山之務。陳乎兵
 甲而肆諸市朝,具嚴天刑,以懲亂首,論其本意,蓋有不得已而用之者焉。是以丹浦興仁,羽山咸服。而世屬僥倖,事關攸蠹,政失禮微,獄成刑起,則孔子曰:「聽訟吾猶人也,必也使無訟乎!」及周氏龔行,卻收鋒刃,祖述生成,憲章堯禹,政有膏露,威兼禮樂,或觀辭以明其趣,或傾耳以照其微,或彰善以激其情,或除惡以崇其本。至夫取威定霸,一匡九合,寓言成康,不由凝網,此所謂酌其遺美,而愛民治國者焉。若乃化蔑彞倫,道睽明慎,則夏癸之虔劉百姓,商辛之毒
 \gezhu{
  疒甫}
 四海,衛鞅之無所自容,韓非之不勝其虐,與夫《甘棠》流詠,未或同歸。秦文初造參
 夷,始皇加之抽協,囹圄如市,悲哀盈路。漢王以三章之法以弔之,文帝以刑厝之道以臨之,于時百姓欣然,將逢交泰。而犴逐情遷,科隨意往,獻瓊杯於闕下,徙青衣於蜀路,覆醢裁刑,傾宗致獄。況乃數囚於京兆之夜,五日於長安之市,北闕相引、中都繼及者,亦往往而有焉。而將亡之國,典刑咸棄,刊章以急其憲,適意以寬其網,桓靈之季,不其然歟!魏明帝時,宮室盛興,而期會迫急,有稽限者,帝親召問,言猶在口,身首已分。王肅抗疏曰:「陛下之所行刑,皆宜死之人也。然眾庶不知,將為倉卒,願陛下下之於吏而暴其罪。均其死也,不汙宮掖,不為
 搢紳驚惋,不為遠近所疑。人命至重,難生易殺,氣絕而不續者也,是以聖王重之。孟軻云:『殺一不辜而取天下者,仁者不為也。』」



 世祖武皇帝接三統之微,酌千年之範,乃命有司,大明刑憲。于時詔書頒新法於天下,海內同軌,人甚安之。條綱雖設,稱為簡惠,仰昭天眷,下濟民心,道有法而無敗,德俟刑而久立。及晉圖南徙,百有二年,仰止前規,挹其流潤,江左無外,蠻陬來格。孝武時,會稽王道子傾弄朝權,其所樹之黨,貨官私獄,烈祖惛迷,不聞司敗,晉之綱紀大亂焉。



 傳曰「三皇設言而民不違,五帝畫象而民知禁」,則《書》所
 謂「象以典刑,流宥五刑,鞭作官刑,撲作教刑」者也。然則犯黥者皁其巾,犯劓者丹其服,犯臏者墨其體,犯宮者雜其屢,大辟之罪,殊刑之極,布其衣裾而無領緣,投之於市,與眾棄之。舜命皋陶曰;「五刑有服,五服三就,五流有宅,五宅三居。」方乎前載,事既參倍。夏后氏之王天下也,則五刑之屬三千。殷因於夏,有所損益。周人以三典刑邦國,以五聽察民情,左嘉右肺,事均熔造,而五刑之屬猶有二千五百焉。乃置三刺、三宥、三赦之法:一刺曰訊群臣,再刺曰訊群吏,三刺曰訊萬民;一宥曰不識,再宥曰過失,三宥曰遺忘;一赦曰幼弱,再赦曰老旄,三赦
 曰蠢愚。《司馬法》:或起甲兵以征不義,廢貢職則討,不朝會則誅,亂嫡庶則縶,變禮刑則放。



 傳曰:「殷周之質,不勝其文。」及昭后徂征,穆王斯耄,爰制刑辟,以詰四方,奸宄弘多,亂離斯永,則所謂「夏有亂政而作《禹刑》,商有亂政而作《湯刑》,周有亂政而作《九刑》」者也。古者大刑用甲兵,中刑用刀鋸,薄刑用鞭撲。自茲厥後,狙詐彌繁。武皇帝並以為往憲猶疑,不可經國,乃命車騎將軍、守尚書令、魯公徵求英俊,刊律定篇云爾。



 漢自王莽篡位之後,舊章不存。光武中興,留心庶獄,常臨朝聽訟,躬決疑事。是時承離亂之後,法網弛縱,罪名既
 輕,無以懲肅。梁統乃上疏曰:



 臣竊見元帝初元五年,輕殊刑三十四事,哀帝建平元年盡四年,輕殊死者刑八十一事,其四十二事,手殺人皆減死罪一等,著為常法。自是以後,人輕犯法,吏易殺人,吏民俱失,至於不羈。



 臣愚以為刑罰不茍務輕,務其中也。君人之道,仁義為主,仁者愛人,義者理務。愛人故當為除害,理務亦當為去亂。是以五帝有流殛放殺之誅,三王有大辟刻肌之刑,所以為除殘去亂也。故孔子稱「仁者必有勇」,又曰「理財正辭,禁人為非曰義」。高帝受命,制約令,定法律,傳之後世,可常施行。文帝寬惠溫克,遭世康平,因時施恩,省去
 肉刑,除相坐之法,他皆率由舊章,天下幾致升平。武帝值中國隆盛,財力有餘,出兵命將,征伐遠方,軍役數興,百姓罷弊,豪傑犯禁,姦吏弄法,故設遁匿之科,著知縱之律。宣帝聰明正直,履道握要,以御海內,臣下奉憲,不失繩墨。元帝法律,少所改更,天下稱安。孝成、孝哀,承平繼體,即位日淺,聽斷尚寡。丞相王嘉等猥以數年之間,虧除先帝舊約,穿令斷律,凡百餘事,或不便於政,或不厭人心。臣謹表取其尤妨政事、害善良者,傅奏如左。



 伏惟陛下苞五常,履九德,推時撥亂,博施濟時,而反因循季世末節,衰微軌迹,誠非所以還初反本,據元更始也。
 願陛下宣詔有司,悉舉初元、建平之所穿鑿,考其輕重,察其化俗,足以知政教所處,擇其善者而從之,其不善者而改之,定不易之典,施之無窮,天下幸甚。



 事下三公、廷尉議,以為隆刑峻法,非明王急務,不可開許。統復上言曰:「有司猥以臣所上不可施行。今臣所言,非曰嚴刑。竊謂高帝以後,至于宣帝,其所施行,考合經傳,此方今事,非隆刑峻法。不勝至願,願得召見,若對尚書近臣,口陳其意。」帝令尚書問狀,統又對,極言政刑宜改。議竟不從。及明帝即位,常臨聽訟觀錄洛陽諸獄。帝性既明察,能得下姦,故尚書奏決罰近於苛碎。



 至章帝時,尚書陳
 寵上疏曰:「先王之政,賞不僭,刑不濫,與其不得已,寧僭不濫。故唐堯著典曰『流宥五刑,眚災肆赦』。帝舜命皋陶以『五宅三居,惟明克允』。文王重《易》六爻,而列叢棘之聽;周公作《立政》,戒成王勿誤乎庶獄。陛下即位,率由此義,而有司執事,未悉奉承。斷獄者急於榜格酷烈之痛,執憲者繁於詐欺放濫之文,違本離實,箠楚為姦,或因公行私,以逞威福。夫為政也,猶張琴瑟,大弦急者小弦絕,故子貢非臧孫之猛法,而美鄭僑之仁政。方今聖德充塞,假於上下,宜因此時,隆先聖之務,蕩滌煩苛,輕薄箠楚,以濟群生,廣至德也。」帝納寵言,決罪行刑,務於寬厚。
 其後遂詔有司,禁絕鑽金贊諸酷痛舊制,解祅惡之禁,除文致之請,讞五十餘事,定著于令。是後獄法和平。



 永元六年,寵又代郭躬為廷尉,復校律令,刑法溢於《甫刑》者,奏除之,曰:「臣聞禮經三百,威儀三千,故《甫刑》大辟二百,五刑之屬三千。禮之所去,刑之所取,失禮即入刑,相為表裏者也。今律令,犯罪應死刑者六百一十,耐罪千六百九十八,贖罪以下二千六百八十一,溢於《甫刑》千九百八十九,其四百一十大辟,千五百耐罪,七十九贖罪。《春秋保乾圖》曰:『王者三百年一蠲法。』漢興以來,三百二年,憲令稍增,科條無限。又律有三家,說各駁異。刑法繁多,
 宜令三公、廷尉集平律令,應經合義可施行者,大辟二百,耐罪、贖罪二千八百,合為三千,與禮相應。其餘千九百八十九事,悉可詳除。使百姓改易視聽,以成大化,致刑措之美,傳之無窮。」未及施行,會寵抵罪,遂寢。寵子忠。忠後復為尚書,略依寵意,奏上三十三條,為《決事比》,以省請讞之弊。又上除蠶室刑,解贓吏三世禁錮,狂易殺人得減重論,母子兄弟相代死聽赦所代者,事皆施行。雖時有蠲革,而舊律繁蕪,未經纂集。



 獻帝建安元年,應劭又刪定律令,以為《漢議》,表奏之曰:「夫國之大事,莫尚載籍。載籍也者,決嫌疑,明是非,賞刑之宜,允執厥中,俾
 後之人永有鑒焉。故膠東相董仲舒老病致仕,朝廷每有政議,數遣廷尉張湯親至陋巷,問其得失,於是作《春秋折獄》二百三十二事,動以《經》對,言之詳矣。逆臣董卓,蕩覆王室,典憲焚燎,靡有孑遺,開闢以來,莫或茲酷。今大駕東邁,巡省許都,拔出險難,其命惟新。臣竊不自揆,輒撰具《律本章句》、《尚書舊事》、《廷尉板令》、《決事比例》、《司徒都目》、《五曹詔書》及《春秋折獄》,凡二百五十篇,蠲去復重,為之節文。又集《議駁》三十篇,以類相從,凡八十二事。其見《漢書》二十五,《漢記》四,皆刪敘潤色,以全本體。其二十六,博採古今瑰瑋之士,德義可觀。其二十七,臣所創造。《
 左氏》云:『雖有姬姜,不棄憔悴;雖有絲麻,不棄菅蒯。』蓋所以代匱也。是用敢露頑才,廁于明哲之末,雖未足綱紀國體,宣洽時雍。庶幾觀察,增闡聖德。惟因萬機之餘暇,遊意省覽。」獻帝善之,於是舊事存焉。是時天下將亂,百姓有土崩之勢,刑罰不足以懲惡,於是名儒大才故遼東太守崔實、大司農鄭玄、大鴻臚陳紀之徒,咸以為宜復行肉刑。漢朝既不議其事,故無所用矣。



 及魏武帝匡輔漢室,尚書令荀彧博訪百官,復欲申之,而少府孔融議以為:「古者敦厖,善否區別,吏端刑清政簡,一無過失,百姓有罪,皆自取之。末世陵遲,風化壞亂,政撓其俗,法害其教。
 故曰『上失其道,人散久矣』。而欲繩之以古刑,投之以殘棄,非所謂與時消息也。紂斮朝涉之脛,天下謂為無道。夫九牧之地,千八百君,若各刖一人,是天下常有千八百紂也,求世休和,弗可得已。且被刑之人,慮不念生,志在思死,類多趨惡,莫復歸正。夙沙亂齊,伊戾禍宋,趙高、英布,為世大患。不能止人遂為非也,適足絕人還為善耳。雖忠如鬻拳,信如卞和,智如孫臏,冤如巷伯,才如史遷,達如子政,一罹刀鋸,沒世不齒。是太甲之思庸,穆公之霸秦,陳湯之都賴,魏尚之臨邊,無所復施也。漢開改惡之路,凡為此也。故明德之君,遠度深惟,棄短就長,不
 茍革其政者也。」朝廷善之,卒不改焉。



 及魏國建,陳紀子群時為御史中丞,魏武帝下令又欲復之,使群申其父論。群深陳其便。時鐘繇為相國,亦贊成之,而奉常王修不同其議。魏武帝亦難以籓國改漢朝之制,遂寢不行。於是乃定甲子科,犯釱左右趾者易以木械,是時乏鐵,故易以木焉。又嫌漢律太重,故令依律論者聽得科半,使從半減也。



 魏文帝受禪,又議肉刑。詳議未定,會有軍事,復寢。時有大女劉朱,撾子婦酷暴,前後三婦自殺,論朱減死輸作尚方,因是下怨毒殺人減死之令。魏明帝改士庶罰金之令,男聽以罰金,婦人加笞還從鞭督之
 例,以其形體裸露故也。



 是時承用秦漢舊律,其文起自魏文侯師李悝。悝撰次諸國法,著《法經》。以為王者之政,莫急於盜賊,故其律始於《盜賊》。盜賊須劾捕,故著《網捕》二篇。其輕狡、越城、博戲、借假不廉、淫侈踰制以為《雜律》一篇,又以《具律》具其加減。是故所著六篇而已,然皆罪名之制也。商君受之以相秦。漢承秦制,蕭何定律,除參夷連坐之罪,增部主見知之條,益事律《興》、《廄》、《戶》三篇,合為九篇。叔孫通益律所不及,傍章十八篇。張湯《越宮律》二十七篇。趙禹《朝律》六篇。合六十篇。又漢時決事,集為《令甲》以下三百餘篇,及司徒鮑公撰嫁娶辭訟決為《法
 比都目》,凡九百六卷。世有增損,率皆集類為篇,結事為章。一章之中或事過數十,事類雖同,輕重乖異。而通條連句,上下相蒙,雖大體異篇,實相採入。《盜律》有賊傷之例,《賊律》有盜章之文,《興律》有上獄之法,《廄律》有逮捕之事,若此之比,錯糅無常。後人生意,各為章句。叔孫宣、郭令卿、馬融、鄭玄諸儒章句十有餘家,家數十萬言。凡斷罪所當由用者,合二萬六千二百七十二條,七百七十三萬二千二百餘言,言數益繁,覽者益難。天子於是下詔,但用鄭氏章句,不得雜用餘家。



 衛覬又奏曰:「刑法者,國家之所貴重,而私議之所輕賤;獄吏者,百姓之所懸命,而
 選用者之所卑下。王政之弊,未必不由此也。請置律博士,轉相教授。」事遂施行。然而律文煩廣,事比眾多,離本依末,決獄之吏如廷尉獄吏范洪受囚絹二丈,附輕法論之,獄吏劉象受屬偏考囚張茂物故,附重法論之。洪、象雖皆棄市,而輕枉者相繼。是時太傅鐘繇又上疏求復肉刑,詔下其奏,司徒王朗議又不同。時議者百餘人,與朗同者多。帝以吳蜀未平,又寢。其後,天子又下詔改定刑制,命司空陳群、散騎常侍劉邵、給事黃門侍郎韓遜、議郎庾嶷、中郎黃休、荀詵等刪約舊科,傍采漢律,定為魏法,制《新律》十八篇,《州郡令》四十五篇,《尚書官令》、《軍中
 令》,合百八十餘篇。其序略曰:



 舊律所難知者,由於六篇篇少故也。篇少則文荒,文荒則事寡,事寡則罪漏。是以後人稍增,更與本體相離。今制新律,宜都總事類,多其篇條。



 舊律因秦《法經》,就增三篇,而《具律》不移,因在第六。罪條例既不在始,又不在終,非篇章之義。故集罪例以為《刑名》,冠於律首。



 《盜律》有劫略、恐猲、和賣買人,科有持質,皆非盜事,故分以為《劫略律》。《賊律》有欺謾、詐偽、踰封、矯制、《囚律》有詐偽生死,《令丙》有詐自復免,事類眾多,故分為《詐律》。《賊律》有賊伐樹木、殺傷人畜產及諸亡印,《金布律》有毀傷亡失縣官財物,故分為《毀亡律》。《囚律》有告
 劾、傳覆,《廄律》有告反逮受,科有登聞道辭,故分為《告劾律》。《囚律》有繫囚、鞫獄、斷獄之法,《興律》有上獄之事,科有考事報讞,宜別為篇,故分為《繫訊》、《斷獄律》。《盜律》有受所監受財枉法,《雜律》有假借不廉,《令乙》有呵人受錢,科有使者驗賂,其事相類,故分為《請賕律》。《盜律》有勃辱強賊,《興律》有擅興徭役,《具律》有出賣呈,科有擅作修舍事,故分為《興擅律》。《興律》有乏徭稽留,《賊律》有儲峙不辨,《廄律》有乏軍之興,及舊典有奉詔不謹、不承用詔書,漢氏施行有小愆之反不如令,輒劾以不承用詔書乏軍要斬,又減以《丁酉詔書》,《丁酉詔書》,漢文所下,不宜復以為法,
 故別為之《留律》。秦世舊有廄置、乘傳、副車、食廚,漢初承秦不改,後以費廣稍省,故後漢但設騎置而無車馬,則律猶著其文,則為虛設,故除《廄律》,取其可用合科者,以為《郵驛令》。其告反逮驗,別入《告劾律》。上言變事,以為《變事令》,以驚事告急,與《興律》烽燧及科令者,以為《驚事律》。《盜律》有還贓畀主,《金布律》有罰贖入責以呈黃金為價,科有平庸坐贓事,以為《償贓律》。律之初制,無免坐之文,張湯、趙禹始作監臨部主、見知故縱之例。其見知而故不舉劾,各與同罪,失不舉劾,各以贖論,其不見不知,不坐也,是以文約而例通。科之為制,每條有違科,不覺不知,
 從坐之免,不復分別,而免坐繁多,宜總為免例,以省科文,故更制定其由例,以為《免坐律》。諸律令中有其教制,本條無從坐之文者,皆從此取法也。凡所定增十三篇,就故五篇,合十八篇,於正律九篇為增,於旁章科令為省矣。



 改漢舊律不行於魏者皆除之,更依古義制為五刑。其死刑有三,髡刑有四,完刑、作刑各三,贖刑十一,罰金六,雜抵罪七,凡三十七名,以為律首。又改《賊律》,但以言語及犯宗廟園陵,謂之大逆無道,要斬,家屬從坐,不及祖父母、孫。至於謀反大逆,臨時捕之,或汙瀦,或梟菹,夷其三族,不在律令,所以嚴絕惡跡也。賊鬥殺人,以劾
 而亡,許依古義,聽子弟得追殺之。會赦及過誤相殺,不得報仇,所以止殺害也。正殺繼母,與親母同,防繼假之隙也。除異子之科,使父子無異財也。歐兄姊加至五歲刑,以明教化也。囚徒誣告人反,罪及親屬,異於善人,所以累之使省刑息誣也。改投書棄市之科,所以輕刑也。正篡囚棄市之罪,斷凶強為義之蹤也。二歲刑以上,除以家人乞鞫之制,省所煩獄也。改諸郡不得自擇伏日,所以齊風俗也。



 斯皆魏世所改,其大略如是。其後正始之間,天下無事,於是征西將軍夏侯玄、河南尹李勝、中領軍曹羲、尚書丁謐又追議肉刑,卒不能決。其文甚多,
 不載。



 及景帝輔政,是時魏法,犯大逆者誅及已出之女。毋丘儉之誅,其子甸妻荀氏應坐死,其族兄顗與景帝姻,通表魏帝,以匄其命。詔聽離婚。荀氏所生女芝,為潁川太守劉子元妻,亦坐死,以懷妊繫獄。荀氏辭詣司隸校尉何曾乞恩,求沒為官婢,以贖芝命。曾哀之,使主簿程咸上議曰:「夫司寇作典,建三等之制;甫侯修刑,通輕重之法。叔世多變,秦立重辟,漢又修之。大魏承秦漢之弊,未及革制,所以追戮已出之女,誠欲殄醜類之族也。然則法貴得中,刑慎過制。臣以為女人有三從之義,無自專之道,出適他族,還喪父母,降其服紀,所以明外成
 之節,異在室之恩。而父母有罪,追刑已出之女;夫黨見誅,又有隨姓之戮。一人之身,內外受辟。今女既嫁,則為異姓之妻;如或產育,則為他族之母,此為元惡之所忽。戮無辜之所重,於防則不足懲奸亂之源,於情則傷孝子之心。男不得罪於他族,而女獨嬰戮於二門,非所以哀矜女弱,蠲明法制之本分也。臣以為在室之女,從父母之誅;既醮之婦,從夫家之罰。宜改舊科,以為永制。」於是有詔改定律令。



 文帝為晉王,患前代律令本注煩雜,陳群、劉邵雖經改革,而科網本密,又叔孫、郭、馬、杜諸儒章句,但取鄭氏,又為偏黨,未可承用。於是令賈充定法
 律,令與太傅鄭沖、司徒荀顗、中書監荀勖、中軍將軍羊祜、中護軍王業、廷尉杜友、守河南尹杜預、散騎侍郎裴楷、潁川太守周雄、齊相郭頎、騎都尉成公綏、尚書郎柳軌及吏部令史榮邵等十四人典其事,就漢九章增十一篇,仍其族類,正其體號,改舊律為《刑名》、《法例》,辨《囚律》為《告劾》、《繫訊》、《斷獄》,分《盜律》為《請賕》、《詐偽》、《水火》、《毀亡》,因事類為《衛宮》、《違制》,撰《周官》為《諸侯律》,合二十篇,六百二十條,二萬七千六百五十七言。蠲其苛穢,存其清約,事從中典,歸於益時。其餘未宜除者,若軍事、田農、酤酒,未得皆從人心,權設其法,太平當除,故不入律,悉以為令。施行
 制度,以此設教,違令有罪則入律。其常事品式章程,各還其府,為故事。減梟斬族誅從坐之條,除謀反適養母出女嫁皆不復還坐父母棄市,省禁固相告之條,去捕亡、亡沒為官奴婢之制。輕過誤老少女人當罰金杖罰者,皆令半之。重奸伯叔母之令,棄市。淫寡女,三歲刑。崇嫁娶之要,一以下娉為正,不理私約。峻禮教之防,準五服以制罪也。凡律令合二千九百二十六條,十二萬六千三百言,六十卷,故事三十卷。泰始三年,事畢,表上。武帝詔曰:「昔蕭何以定律令受封,叔孫通制儀為奉常,賜金五百斤,弟子百人皆為郎。夫立功立事,古今之所
 重,宜加祿賞,其詳考差敘。輒如詔簡異弟子百人,隨才品用,賞帛萬餘匹。」武帝親自臨講,使裴楷執讀。四年正月,大赦天下,乃班新律。



 其後,明法掾張裴又注律,表上之,其要曰:



 律始於《刑名》者,所以定罪制也;終於《諸侯》者,所以畢其政也。王政布於上,諸侯奉於下,禮樂撫於中,故有三才之義焉,其相須而成,若一體焉。



 《刑名》所以經略罪法之輕重,正加減之等差,明發眾篇之多義,補其章條之不足,較舉上下綱領。其犯盜賊、詐偽、請賕者,則求罪於此,作役、水火、畜養、守備之細事,皆求之作本名。告訊為之心舌,捕繫為之手足,斷獄為之定罪,名例齊
 其制。自始及終,往而不窮,變動無常,周流四極,上下無方,不離于法律之中也。



 其知而犯之謂之故,意以為然謂之失,違忠欺上謂之謾,背信藏巧謂之詐,虧禮廢節謂之不敬,兩訟相趣謂之斗,兩和相害謂之戲,無變斬擊謂之賊,不意誤犯謂之過失,逆節絕理謂之不道,陵上僭貴謂之惡逆,將害未發謂之戕,唱首先言謂之造意,二人對議謂之謀,制眾建計謂之率,不和謂之強,攻惡謂之略,三人謂之群,取非其物謂之盜,貨財之利謂之贓:凡二十者,律義之較名也。



 夫律者,當慎其變,審其理。若不承用詔書,無故失之刑,當從贖。謀反之同伍,實
 不知情,當從刑。此故失之變也。卑與尊鬥,皆為賊。斗之加兵刃水火中,不得為戲,戲之重也。向人室廬道徑射,不得為過,失之禁也。都城人眾中走馬殺人,當為賊,賊之似也。過失似賊,戲似鬥,鬥而殺傷傍人,又似誤,盜傷縛守似強盜,呵人取財似受賕,囚辭所連似告劾,諸勿聽理似故縱,持質似恐猲。如此之比,皆為無常之格也。



 五刑不簡,正于五罰,五罰不服,正于五過,意善功惡,以金贖之。故律制,生罪不過十四等,死刑不過三,徒加不過六,囚加不過五,累作不過十一歲,累笞不過千二百,刑等不過一歲,金等不過四兩。月贖不計日,日作不拘
 月,歲數不疑閏。不以加至死,並死不復加。不可累者,故有並數;不可並數,乃累其加。以加論者,但得其加;與加同者,連得其本。不在次者,不以通論。以人得罪與人同,以法得罪與法同。侵生害死,不可齊其防;親疏公私,不可常其教。禮樂崇於上,故降其刑;刑法閑於下,故全其法。是故尊卑敘,仁義明,九族親,王道平也。



 律有事狀相似而罪名相涉者,若加威勢下手取財為強盜,不自知亡為縛守,將中有惡言為恐猲,不以罪名呵為呵人,以罪名呵為受賕,劫召其財為持質。此六者,以威勢得財而名殊者也。即不求自與為受求,所監求而後取為盜
 贓,輸入呵受為留難,斂人財物積藏於官為擅賦,加歐擊之為戮辱。諸如此類,皆為以威勢得財而罪相似者也。



 夫刑者,司理之官;理者,求情之機,情者,心神之使。心感則情動於中,而形於言?暢於四支,發於事業。是故奸人心愧而面赤,內怖而色奪。論罪者務本其心,審其情,精其事,近取諸身,遠取諸物,然後乃可以正刑。仰手似乞,俯手似奪,捧手似謝,擬手似訴,拱臂似自首,攘臂似格鬥,矜莊似威,怡悅似福,喜怒憂懽,貌在聲色。奸真猛弱,候在視息。出口有言當為告,下手有禁當為賊,喜子殺怒子當為戲,怒子殺喜子當為賊。諸如此類,自非至
 精不能極其理也。



 律之名例,非正文而分明也。若八十,非殺傷人,他皆勿論,即誣告謀反者反坐。十歲,不得告言人;即奴婢捍主,主得謁殺之。賊燔人廬舍積聚,盜贓五匹以上,棄市;即燔官府積聚盜,亦當與同。歐人教令者與同罪,即令人歐其父母,不可與行者同得重也。若得遺物強取強乞之類,無還贓法隨例畀之文。法律中諸不敬,違儀失式,及犯罪為公為私,贓入身不入身,皆隨事輕重取法,以例求其名也。



 夫理者,精玄之妙,不可以一方行也;律者,幽理之奧,不可以一體守也。或計過以配罪,或化略以循常,或隨事以盡情,或趣舍以從
 時,或推重以立防,或引輕而就下。公私廢避之宜,除削重輕之變,皆所以臨時觀釁,使用法執詮者幽於未制之中,采其根牙之微,致之於機格之上,稱輕重於豪銖,考輩類於參伍,然後乃可以理直刑正。



 夫奉聖典者若操刀執繩,刀妄加則傷物,繩妄彈則侵直。梟首者惡之長,斬刑者罪之大,棄市者死之下,髡作者刑之威,贖罰者誤之誡。王者立此五刑,所以寶君子而逼小人,故為敕慎之經,皆擬《周易》有變通之體焉。欲令提綱而大道清,舉略而王法齊,其旨遠,其辭文,其言曲而中,其事肆而隱。通天下之志唯忠也,斷天下之疑唯文也,切天下
 之情唯遠也,彌天下之務唯大也,變無常體唯理也,非天下之賢聖,孰能與於斯!



 夫刑而上者謂之道,刑而下者謂之器,化而裁之謂之格。刑殺者是冬震曜之象,髡罪者似秋彫落之變,贖失者是春陽悔吝之疵之。五刑成章,輒相依準,法律之義焉。



 是時侍中盧珽、中書侍郎張華又表:「抄《新律》諸死罪條目,懸之亭傳,以示兆庶。」有詔從之。



 及劉頌為廷尉,頻表宜復肉刑,不見省,又上言曰:



 臣昔上行肉刑,從來積年,遂寢不論。臣竊以為議者拘孝文之小仁,而輕違聖王之典刑,未詳之甚,莫過於此。



 今死刑重,故非命者眾;生刑輕,故罪不禁奸。所以然
 者,肉刑不用之所致也。今為徒者,類性元惡不軌之族也,去家懸遠,作役山谷,飢寒切身,志不聊生,雖有廉士介者,茍慮不首死,則皆為盜賊,豈況本性奸凶無賴之徒乎!又令徒富者輸財,解日歸家,乃無役之人也。貧者起為奸盜,又不制之虜也。不刑,則罪無所禁;不制,則群惡橫肆。為法若此,近不盡善也。是以徒亡日屬,賊盜日煩,亡之數者至有十數,得輒加刑,日益一歲,此為終身之徒也。自顧反善無期,而災困逼身,其志亡思盜,勢不得息,事使之然也。



 古者用刑以止刑,今反於此。諸重犯亡者,髮過三寸輒重髡之,此以刑生刑;加作一歲,此以
 徒生徒也。亡者積多,繫囚猥畜。議者曰囚不可不赦,復從而赦之,此為刑不制罪,法不勝奸。下知法之不勝,相聚而謀為不軌,月異而歲不同。故自頃年以來,奸惡陵暴,所在充斥。議者不深思此故,而曰肉刑於名忤聽,忤聽孰與賊盜不禁?



 聖王之制肉刑,遠有深理,其事可得而言,非徒懲其畏剝割之痛而不為也,乃去其為惡之具,使夫奸人無用復肆其志,止奸絕本,理之盡也。亡者刖足,無所用復亡。盜者截手,無所用復盜。淫者割其勢,理亦如之。除惡塞源,莫善於此,非徒然也。此等已刑之後,便各歸家,父母妻子,共相養恤,不流離於塗路。有今
 之困,創愈可役,上準古制,隨宜業作,雖已刑殘,不為虛棄,而所患都塞,又生育繁阜之道自若也。



 今宜取死刑之限輕,及三犯逃亡淫盜,悉以肉刑代之。其三歲刑以下,已自杖罰遣,又宜制其罰數,使有常限,不得減此。其有宜重者,又任之官長。應四五歲刑者,皆髡笞,笞至一百,稍行,使各有差,悉不復居作。然後刑不復生刑,徒不復生徒,而殘體為戳,終身作誡。人見其痛,畏而不犯,必數倍於今。且為惡者隨發被刑,去其為惡之具,此為諸已刑者皆良士也,豈與全其為奸之手足,而蹴居必死之窮地同哉!而猶曰肉刑不可用,臣竊以為不識務之
 甚也。



 臣昔常侍左右,數聞明詔,謂肉刑宜用,事便於政。願陛下信獨見之斷,使夫能者得奉聖慮,行之於今。比填溝壑,冀見太平。《周禮》三赦三宥,施於老幼悼耄,黔黎不屬逮者,此非為惡之所出,故刑法逆舍而宥之。至於自非此族,犯罪則必刑而無赦,此政之理也。暨至後世,以時嶮多難,因赦解結,權以行之,又不以寬罪人也。至今恒以罪積獄繁,赦以散之,是以赦愈數而獄愈塞,如此不已,將至不勝。原其所由,內刑不用之故也。今行肉刑,非徒不積,且為惡無具則奸息。去此二端,獄不得繁,故無取於數赦,於政體勝矣。



 疏上,又不見省。



 至惠帝之
 世,政出群下,每有疑獄,各立私情,刑法不定,獄訟繁滋。尚書裴頠表陳之曰:



 夫天下之事多塗,非一司之所管;中才之情易擾,賴恒制而後定。先王知其所以然也,是以辨方分職,為之準局。準局既立,各掌其務,刑賞相稱,輕重無二,故下聽有常,群吏安業也。舊宮掖陵廟有水火毀傷之變,然後尚書乃躬自奔赴,其非此也,皆止於郎令史而已。刑罰所加,各有常刑。



 去元康四年,大風之後,廟闕屋瓦有數枚傾落,免太常荀寓。于時以嚴詔所譴,莫敢據正。然內外之意,僉謂事輕責重,有違於常。會五年二月有大風,主者懲懼前事。臣新拜尚書始三日,
 本曹尚書有疾,權令兼出,按行蘭臺。主者乃瞻望阿棟之間,求索瓦之不正者,得棟上瓦小邪十五處。或是始瓦時邪,蓋不足言,風起倉卒,臺官更往,太常按行,不及得周,文書未至之頃,便競相禁止。臣以權兼暫出,出還便罷,不復得窮其事。而本曹據執,卻問無已。臣時具加解遣,而主者畏咎,不從臣言,禁止太常,復興刑獄。



 昔漢氏有盜廟玉環者,文帝欲族誅,釋之但處以死刑,曰:「若侵長陵一抔土,何以復加?」文帝從之。大晉垂制,深惟經遠,山陵不封,園邑不飾,墓而不墳,同乎山壤,是以丘阪存其陳草,使齊乎中原矣。雖陵兆尊嚴,唯毀發然後族
 之,此古典也。若登踐犯損,失盡敬之道,事止刑罪可也。



 去八年,奴聽教加誣周龍燒草,廷尉遂奏族龍,一門八口并命。會龍獄翻,然後得免。考之情理,準之前訓,所處實重。今年八月,陵上荊一枝圍七寸二分者被斫,司徒太常,奔走道路,雖知事小,而案劾難測,搔擾驅馳,各競免負,于今太常禁止未解。近日太祝署失火,燒屋三間半。署在廟北,隔道在重墻之內,又即已滅,頻為詔旨所問。主者以詔旨使問頻繁,便責尚書不即案行,輒禁止,尚書免,皆在法外。



 刑書之文有限,而舛違之故無方,故有臨時議處之制,誠不能皆得循常也。至於此等,皆為
 過當,每相逼迫,不得以理,上替聖朝畫一之德,下損崇禮大臣之望。臣愚以為犯陵上草木,不應乃用同產異刑之制。按行奏劾,應有定準,相承務重,體例遂虧。或因餘事,得容淺深。



 頠雖有此表,曲議猶不止。時劉頌為三公尚書,又上疏曰:



 自近世以來,法漸多門,令甚不一。臣今備掌刑斷,職思其憂,謹具啟聞。



 臣竊伏惟陛下為政,每盡善,故事求曲當,則例不得直;盡善,故法不得全。何則?夫法者,固以盡理為法,而上求盡善,則諸下牽文就意,以赴主之所許,是以法不得全。刑書徵文,徵文必有乖於情聽之斷,而上安於曲當,故執平者因文可引,
 則生二端。是法多門,令不一,則吏不知所守,下不知所避。姦偽者因法之多門,以售其情,所欲淺深,茍斷不一,則居上者難以檢下,於是事同議異,獄犴不平,有傷於法。



 古人有言:「人主詳,其政荒;人主期,其事理。」詳匪他,盡善則法傷,故其政荒也。期者輕重之當,雖不厭情,茍入於文,則循而行之,故其事理也。夫善用法者,忍違情不厭聽之斷,輕重雖不允人心,經於凡覽,若不可行,法乃得直。又君臣之分,各有所司。法欲必奉,故令主者守文;理有窮塞,故使大臣釋滯;事有時宜,故人主權斷。主者守文,若釋之執犯蹕之平也;大臣釋滯,若公孫弘斷郭
 解之獄也;人主權斷,若漢祖戮丁公之為也。天下萬事,自非斯格重為,故不近似此類,不得出以意妄議,其餘皆以律令從事。然後法信於下,人聽不惑,吏不容奸,可以言政。人主軌斯格以責群下,大臣小吏各守其局,則法一矣。



 古人有言:「善為政者,看人設教。」看人設教,制法之謂也。又曰:「隨時之宜」,當務之謂也。然則看人隨時,在大量也,而制其法。法軌既定則行之,行之信如四時,執之堅如金石,群吏豈得在成制之內,復稱隨時之宜,傍引看人設教,以亂政典哉!何則?始制之初,固已看人而隨時矣。今若設法未盡當,則宜改之。若謂已善,不得盡
 以為制,而使奉用之司公得出入以差輕重也。夫人君所與天下共者,法也。已令四海,不可以不信以為教,方求天下之不慢,不可繩以不信之法。且先識有言,人至遇而不可欺也。不謂平時背法意斷,不勝百姓願也。



 上古議事以制,不為刑辟。夏殷及周,書法象魏。三代之君齊聖,然咸棄曲當之妙鑒,而任徵文之直準,非聖有殊,所遇異也。今論時敦樸,不及中古,而執平者欲適情之所安,自託於議事以制。臣竊以為聽言則美,論理則違。然天下至大,事務眾雜,時有不得悉循文如令。故臣謂宜立格為限,使主者守文,死生以之,不敢錯思於成制
 之外,以差輕重,則法恒全。事無正據,名例不及,大臣論當,以釋不滯,則事無閡。至如非常之斷,出法賞罰,若漢祖戮楚臣之私己,封趙氏之無功,唯人主專之,非奉職之臣所得擬議。然後情求傍請之跡絕,似是而非之奏塞,此蓋齊法之大準也。主者小吏,處事無常。何則?無情則法徒克,有情則撓法。積克似無私,然乃所以得其私,又恒所岨以衛其身。斷當恒克,世謂盡公,時一曲法,迺所不疑。故人君不善倚深似公之斷,而責守文如令之奏,然後得為有檢,此又平法之一端也。



 夫出法權制,指施一事,厭情合聽,可適耳目,誠有臨時當意之快,勝於
 徵文不允人心也。然起為經制,經年施用,恒得一而失十。故小有所得者,必大有所失;近有所漏者,必遠有所苞。故諳事識體者,善權輕重,不以小害大,不以近妨遠。忍曲當之近適,以全簡直之大準。不牽於凡聽之所安,必守徵文以正例。每臨其事,恒御此心以決斷,此又法之大概也。



 又律法斷罪,皆當以法律令正文,若無正文,依附名例斷之,其正文名例所不及,皆勿論。法吏以上,所執不同,得為異議。如律之文,守法之官,唯當奉用律令。至於法律之內,所見不同,乃得為異議也。今限法曹郎令史,意有不同為駁,唯得論釋法律,以正所斷,不得
 援求諸外,論隨時之宜,以明法官守局之分。



 詔下其事。侍中、太宰、汝南王亮奏以為:「夫禮以訓世,而法以整俗,理化之本,事實由之。若斷不斷,常輕重隨意,則王憲不一,人無所錯矣。故觀人設教,在上之舉;守文直法,臣吏之節也。臣以去太康八年,隨事異議。周懸象魏之書,漢詠畫一之法,誠以法與時共,義不可二。今法素定,而法為議,則有所開長,以為宜如頌所啟,為永久之制。」於是門下屬三公曰:「昔先王議事以制,自中古以來,執法斷事,既以立法,誠不宜復求法外小善也。若常以善奪法,則人逐善而不忌法,其害甚於無法也。案啟事,欲令法
 令斷一,事無二門,郎令史已下,應復出法駁案,隨事以聞也。」



 及於江左,元帝為丞相時,朝廷草創,議斷不循法律,人立異議,高下無狀。主簿熊遠奏曰:「禮以崇善,法以閑非,故禮有常典,法有常防,人知惡而無邪心。是以周建象魏之制,漢創畫一之法,故能闡弘大道,以至刑厝。律令之作,由來尚矣。經賢智,歷夷險,隨時斟酌,最為周備。自軍興以來,法度陵替,至於處事不用律令,競作屬命,人立異議,曲適物情,虧傷大例。府立節度,復不奉用,臨事改制,朝作夕改,至於主者不敢任法,每輒關咨,委之大官,非為政之體。若本曹處事不合法令,監司當以法
 彈違,不得動用開塞,以壞成事。按法蓋粗術,非妙道也,矯割物情,以成法耳。若每隨物情,輒改法制,此為以情壞法。法之不一,是謂多門,開人事之路,廣私請之端,非先王立法之本意也。凡為駁議者,若違律令節度,當合經傳及前比故事,不得任情以破成法。愚謂宜令錄事更立條制,諸立議者皆當引律令經傳,不得直以情言,無所依準,以虧舊典也。若開塞隨宜,權道制物,此是人君之所得行,非臣子所宜專用。主者唯當徵文據法,以事為斷耳。」



 是時帝以權宜從事,尚未能從。而河東衛展為晉王大理,考擿故事有不合情者,又上書曰:「今施行
 詔書,有考子正父死刑,或鞭父母問子所在。近主者所稱《庚寅詔書》,舉家逃亡家長斬。若長是逃亡之主,斬之雖重猶可。設子孫犯事,將考祖父逃亡,逃亡是子孫,而父祖嬰其酷。傷順破教,如此者眾。相隱之道離,則君臣之義廢。君臣之義廢,則犯上之奸生矣。秦網密文峻,漢興,掃除煩苛,風移俗易,幾於刑厝。大人革命,不得不蕩其穢匿,通其圮滯。今詔書宜除者多,有便於當今,著為正條,則法差簡易。」元帝令曰:「禮樂不興,則刑罰不中,是以明罰敕法,先王所慎。自元康已來,事故薦臻,法禁滋漫。大理所上,宜朝堂會議,蠲除詔書不可用者,此孤所
 虛心者也。」



 及帝即位,展為廷尉,又上言:「古者肉刑,事經前聖,漢文除之,增加大辟。今人戶凋荒,百不遺一,而刑法峻重,非句踐養胎之義也。愚謂宜復古施行,以隆太平之化。」詔內外通議。於是驃騎將軍王導、太常賀循、侍中紀瞻、中書郎庾亮、大將軍咨議參軍梅陶、散騎郎張嶷等議,以:「肉刑之典,由來尚矣。肇自古先,以及三代,聖哲明王所未曾改也。豈是漢文常主所能易者乎!時蕭曹已沒,絳灌之徒不能正其義。逮班固深論其事,以為外有輕刑之名,內實殺人。又死刑太重,生刑太輕,生刑縱於上,死刑怨於下,輕重失當,故刑政不中也。且原先
 王之造刑也,非以過怒也,非以殘人也,所以救奸,所以當罪。今盜者竊人之財,淫者好人之色,亡者避叛之役,皆無殺害也,則加之以刑。刑之則止,而加之斬戮,戮過其罪,死不可生,縱虐於此,歲以巨計。此乃仁人君子所不忍聞,而況行之於政乎!若乃惑其名而不練其實,惡其生而趣其死,此畏水投舟,避坎蹈井,愚夫之不若,何取於政哉!今大晉中興,遵復古典,率由舊章,起千載之滯義,拯百殘之遺黎,使皇典廢而復存,黔首死而更生,至義暢于三代之際,遺風播乎百世之後,生肉枯骨,惠侔造化,豈不休哉!惑者乃曰,死猶不懲,而況於刑?然人
 者冥也,其至愚矣,雖加斬戮,忽為灰土,死事日往,生欲日存,未以為改。若刑諸市朝,朝夕鑒戒,刑者詠為惡之永痛,惡者睹殘刖之長廢,故足懼也。然後知先王之輕刑以御物,顯誡以懲愚,其理遠矣。」



 尚書令刁協、尚書薛兼等議,以為:「聖上悼殘荒之遺黎,傷犯死之繁眾,欲行刖以代死刑,使犯死之徒得存性命,則率土蒙更生之澤,兆庶必懷恩以反化也。今中興祚隆,大命惟新,誠宜設寬法以育人。然懼群小愚蔽,習玩所見而忽異聞,或未能咸服。愚謂行刑之時,先明申法令,樂刑者刖,甘死者殺,則心必服矣。古典刑不上大夫,今士人有犯者,謂
 宜如舊,不在刑例,則進退為允。」



 尚書顗、郎曹彥、中書郎桓彞等議,以為:「復肉刑以代死,誠是聖王之至德,哀矜之弘私。然竊以為刑罰輕重,隨時而作。時人少罪而易威,則從輕而寬之;時人多罪而難威,則宜化刑而濟之。肉刑平世所應立,非救弊之宜也。方今聖化草創,人有餘奸,習惡之徒,為非未已,截頭絞頸,尚不能禁,而乃更斷足劓鼻,輕其刑罰,使欲為惡者輕犯寬刑,蹈罪更眾,是為輕其刑以誘人於罪,殘其身以加楚酷也。昔之畏死刑以為善人者,今皆犯輕刑而殘其身,畏重之常人,反為犯輕而致囚,此則何異斷刖常人以為恩仁邪!
 受刑者轉廣,而為非者日多,踴貴屨賤,有鼻者醜也。徒有輕刑之名,而實開長惡之源。不如以殺止殺,重以全輕,權小停之。須聖化漸著,兆庶易威之日,徐施行也。」



 議奏,元帝猶欲從展所上。大將軍王敦以為:「百姓習俗日久,忽復肉刑,必駭遠近。且逆寇未殄,不宜有慘酷之聲,以聞天下。」於是乃止。



 咸康之世,庾冰好為糾察,近於繁細,後益矯違,復存寬縱,疏密自由,律令無用矣。



 至安帝元興末,桓玄輔政,又議欲復肉刑斬左右趾之法,以輕死刑,命百官議。蔡廓上議曰:「建邦立法,弘教穆化,必隨時置制,德刑兼施。長貞一以閑其邪,教禁以檢其慢,灑
 湛露以流潤,厲嚴霜以肅威,雖復質文迭用,而斯道莫革。肉刑之設,肇自哲王。蓋由曩世風淳,人多惇謹,圖像既陳,則機心直戢,刑人在塗,則不逞改操,故能勝殘去殺,化隆無為。季末澆偽,設網彌密,利巧之懷日滋,恥畏之情轉寡。終身劇役,不足止其奸,況乎黥劓,豈能反於善。徒有酸慘之聲,而無濟俗之益。至於棄市之條,實非不赦之罪,事非手殺,考律同歸,輕重均科,減降路塞,鐘陳以之抗言,元皇所為留愍。今英輔翼贊,道邈伊周,誠宜明慎用刑,愛人弘育,申哀矜以革濫,移大辟於支體,全性命之至重,恢繁息於將來。」而孔琳之議不同,用王
 朗、夏侯玄之旨。時論多與琳之同,故遂不行。



\end{pinyinscope}