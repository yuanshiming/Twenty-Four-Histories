\article{志第十八}

\begin{pinyinscope}

 五行中



 《傳》曰:「言之不從,是謂不乂,厥咎僭,厥罰恒陽,厥極憂。時則有詩妖,時則有介蟲之孽,時則有犬禍,時則有口舌之痾,時則有白眚白祥。惟木沴金。」言之不從,從,順也。是謂不乂,乂,治也。孔子曰:「君子居其室,出其言不善,則千里之外違之,況其邇者乎!」《詩》曰:「如蜩如螗,如沸如羹。」言上號令不順人心,虛譁憒亂,則不能治海內。失在過差,
 故其咎僭差也。刑罰妄加,群陰不附,則陽氣勝,故其罰常陽也。旱傷百穀,則有寇難,上下俱憂,故其極憂也。君炕陽而暴虐,臣畏刑而箝口,則怨謗之氣發於歌謠,故有詩妖。介蟲孽者,謂小蟲有甲飛揚之類,陽氣所生也,於《春秋》為螽,今謂之蝗,皆其類也。於《易》,《兌》為口,犬以吠守而不可信,言氣毀,故有犬禍。一曰,旱歲犬多狂死及為怪,亦是也。及人,則多病口喉咳嗽者,故有口舌痾。金色白,故有白眚白祥。凡言傷者,病金氣;金氣病,則木沴之。其極憂者,順之,其福曰康寧。劉歆《言傳》曰時則有毛蟲之孽。說以為於天文西方參為獸星,故為毛蟲。



 魏齊王嘉平初,東郡有訛言,云白馬河出妖馬,夜過官牧邊鳴呼,眾馬皆應,明日見其跡,大如斛,行數里,還入河。楚王彪本封白馬,兗州刺史令狐愚以彪有智勇,及聞此言,遂與王凌謀共立之。事泄,凌、愚被誅,彪賜死。此言不從之罰也。《詩》云:「人之訛言,寧莫之懲。」



 蜀劉禪嗣位,譙周曰:「先主諱備,其訓具也,後主諱禪,其訓授也。若言劉已具矣,當授與人,甚於晉穆侯、漢靈帝命子之祥也。」蜀果亡,此言之不從也。劉備卒,劉禪即位,未葬,亦未踰月,而改元為建興,此言之不從也。禮,國君即位踰年而後改元者,緣臣子之心不忍一年而有二君。今可謂
 亟而不知禮義矣。後遂降焉。



 魏明帝太和中,姜維歸蜀,失其母。魏人使其母手書呼維令反,并送當歸以譬之。維報書曰:「良田百頃,不計一畝,但見遠志,無有當歸。」維卒不免。



 景初元年,有司奏,帝為烈祖,與太祖、高祖並為不毀之廟,從之。案宗廟之制,祖宗之號,皆身沒名成乃正其禮。故雖功赫天壤,德邁前王,未有豫定之典。此蓋言之不從失之甚者也。後二年而宮車晏駕,於是統微政逸。



 吳孫休時,烏程人有得困病,及差,能以響言者,言於此而聞於彼。自其所聽之,不覺其聲之大也。自遠聽之,如
 人對言,不覺聲之自遠來也。聲之所往,隨其所向,遠者所過十數里。其鄰人有責息於外,歷年不還,乃假之使為責讓,懼以禍福。負物者以為鬼神,即傎倒畀之,其人亦不自知所以然也。言不從之咎也。



 魏時起安世殿,武帝後居之。安世,武帝字也。武帝每延群臣,多說平生常事,未嘗及經國遠圖。此言之不從也。何曾謂子遵曰:「國家無貽厥之謀,及身而已,後嗣其殆乎!此子孫之憂也。」自永熙後王室漸亂,永嘉中天下大壞,及何綏以非辜被殺,皆如曾言。



 趙王倫廢惠帝於金墉城,改號金墉城為永安宮。帝尋
 復位而倫誅。



 惠帝永興元年,詔廢太子覃還為清河王,立成都王潁為皇太弟,猶加侍中、大都督,領丞相,備九錫,封二十郡,如魏王故事。案周禮傳國以胤不以勳,故雖公旦之聖不易成王之嗣,所以遠絕覬覦,永一宗祧。後代遵履,改之則亂。今擬非其實,僭差已甚。且既為國嗣,則不應復開封土,兼領庶職。此言之不從,進退乖爽,故帝既播越,穎亦不終,是其咎僭也。後猶不悟,又立懷帝為皇太弟。懷終流弒,不永厥祚,又其應也。語曰,「變古易常,不亂則亡」,此之謂乎。



 元帝永昌二年,大將軍王敦下據姑孰。百姓訛言行蟲病,食人大孔,數日入腹,入腹則死;療之有方,當得白犬膽以為藥。自淮泗遂及京都,數日之間,百姓驚擾,人人皆自云已得蟲病。又云,始在外時,當燒鐵以灼之。於是翕然,被燒灼者十七八矣。而白犬暴貴,至相請奪,其價十倍。或有自云能行燒鐵灼者,賃灼百姓,日得五六萬,憊而後已。四五日漸靜。說曰:「夫裸蟲人類,而人為之主。今云蟲食人,言本同臭類而相殘賊也。自下而上,明其逆也。必入腹者,言害由中不由外也。犬有守衛之性,白者金色,而膽,用武之主也。帝王之運,王霸會於戌。戌主
 用兵,金者晉行,火燒鐵以療疾者,言必去其類而來火與金合德,共除蟲害也。」案中興之際,大將軍本以腹心受伊呂之任,而元帝末年,遂攻京邑,明帝諒闇,又有異謀,是以下逆上,腹心內爛也。及錢鳳、沈充等逆兵四合,而為王師所挫,踰月而不能濟水,北中郎劉遐及淮陵內史蘇峻率淮泗之眾以救朝廷,故其謠言首作於淮泗也。朝廷卒以弱制強,罪人授首,是用白犬膽可救之效也。



 海西公時,庾晞四五年中喜為挽歌,自搖大鈴為唱,使左右齊和。又宴會輒令倡妓作新安人歌舞離別之辭,
 其聲悲切。時人怪之,後亦果敗。



 太元中,小兒以兩鐵相打於土中,名曰斗族。後王國寶、王孝伯一姓之中自相攻擊。



 桓玄初改年為大亨,遐邇讙言曰「二月了」,故義謀以仲春發也。玄篡立,又改年為建始,以與趙王倫同,又易為永始,永始復是王莽受封之年也。始徙司馬道子于安成。安帝遜位,出永安宮,封為平固王,瑯邪王德文為石陽公,並使住尋陽城。識者皆以為言不從之妖僭也。



 武帝初,何曾薄太官御膳,自取私食,子劭又過之,而王愷又過劭。王愷、羊琇之儔,盛致聲色,窮珍極麗。至元康
 中,夸恣成俗,轉相高尚,石崇之侈,遂兼王、何,而儷人主矣。崇既誅死,天下尋亦淪喪。僭踰之咎也。



 庶徵恒陽,劉向以為《春秋》大旱也。其夏旱,雩,《禮》謂之大雩。不傷二穀謂之不雨。京房《易傳》曰:「欲德不用茲謂張,厥災荒,旱也。其旱陰雲不雨,變而赤,因四際。師出過時茲謂廣,其旱不生。上下皆蔽茲謂隔,其旱天赤三月,時有雹殺飛禽。上緣求妃茲謂僭,其旱三月大溫亡雲。君高臺府茲謂犯陰侵陽,其旱萬物根死,數有火災。庶位踰節茲為僭,其旱澤物枯,為火所傷。」



 魏明帝太和二年五月,大旱。元年以來崇廣宮府之應
 也。又,是春宣帝南擒孟達,置二郡,張郃西破諸葛亮,斃馬謖。亢陽自大,又其應也。



 太和五年三月,自去冬十月至此月不雨。辛已,大雩。



 齊王正始元年二月,自去冬十二月至此月不雨。去歲正月,明帝崩。二月,曹爽白嗣主,轉宣帝為太傅,外示尊崇,內實欲令事先由已。是時宣帝功蓋魏朝,欲德不用之應也。



 高貴鄉公甘露三年正月,自去秋至此月旱。是時文帝圍諸葛誕,眾出過時之應也。初,壽春秋夏常雨淹城,而此旱踰年,城陷,乃大雨。咸以誕為天亡。



 吳孫亮五鳳二年,大旱,百姓飢。是歲征役煩興,軍士怨叛。此亢陽自大,勞役失眾之罰也。其役彌歲,故旱亦竟年。



 孫皓寶鼎元年,春夏旱。時孫皓遷都武昌,勞役動眾之應也。



 武帝泰始七年五月閏月旱,大雩。八年五月,旱。是時帝納荀勖邪說,留賈充不復西鎮,而任愷漸疏,上下皆蔽之應也。及李憙、魯芝、李胤等並在散職,近厥德不用之謂也。



 九年,自正月旱,至于六月,祈宗廟社稷山川。癸未,雨。
 十年四月,旱。去年秋冬,採擇卿校諸葛沖等女。是春,五十餘人入殿簡選。又取小將吏女數十人,母子號哭於宮中,聲聞于外,行人悲酸。是殆積陰生陽,上緣求妃之應也。



 咸寧二年五月旱,大雩。至六月,乃澍雨。



 太康二年旱,自去冬旱至此春。三年四月旱,乙酉詔司空齊王攸與尚書、廷尉、河南尹錄訊繫囚,事從蠲宥。



 五年六月,旱。此年正月天陰,解而復合。劉毅上疏曰:「必有阿黨之臣姦以事君者,當誅而不赦也。」帝不答。是時
 荀勖、馮紞僭作威福,亂朝尤甚。



 六年三月,青、梁、幽、冀郡國旱。六月,濟陰、武陵旱,傷麥。七年夏,郡國十三大旱。八年四月,冀州旱。九年夏,郡國三十三旱,扶風、始平、京兆、安定旱,傷麥。十年二月,旱。



 太熙元年二月,旱。自太康已後,雖正人滿朝,不被親仗,而賈充、荀勖、楊駿、馮紞等迭居要重,所以無年不旱者,欲德不用,上下皆蔽,庶位踰節之罰也。



 惠帝元康七年七月,秦、雍二州大旱,疾疫,關中饑,米斛
 萬錢。因此氐羌反叛,雍州刺史解系敗績。而饑疫薦臻,戎晉並困,朝廷不能振,詔聽相賣鬻。其九月,郡國五旱。



 永寧元年,自夏及秋,青、徐、幽、并四州旱。十二月,又郡國十二旱。是年春,三王討趙王倫,六旬之中數十戰,死者十餘萬人。



 懷帝永嘉三年五月,大旱,襄平縣梁水淡池竭,河、洛、江、漢皆可涉。是年三月,司馬越歸京都,遣兵入宮,收中書令繆播等九人殺之,皆僭踰之罰也。又四方諸侯多懷無君之心,劉元海、石勒、王彌、李雄之徒賊害百姓,流血成泥,又其應也。五年,自去冬旱至此春。去歲十一月,司
 馬越以行臺自隨,斥黜宮衛,無君臣之節。



 元帝建武元年六月,揚州旱。去年十二月,淳于伯冤死,其年即旱,而太興元年六月又旱。干寶曰「殺淳于伯之後旱三年」是也。刑罰妄加,群陰不附,則陽氣勝之罰也。



 元帝太興四年五月,旱。是時王敦陵僭已著。



 永昌元年夏,大旱。是年三月,王敦有石頭之變,二宮陵辱,大臣誅死,僭踰無上,故旱尤甚也。其閏十一月,京都大旱,川谷并竭。



 明帝太寧三年,自春不雨,至于六月。



 成帝咸和元年,夏秋旱。是時庾太后臨朝稱制,言不從
 而僭踰之罰也。



 二年夏,旱。五年五月,大旱。六年四月,大旱。八年秋七月,旱。九年,自四月不雨,至于八月。



 咸康元年六月,旱。是時成帝沖弱,未親萬機,內外之政,決之將相。此僭踰之罰,連歲旱也。至四年,王導固讓太傅,復子明辟。是後不旱,殆其應也。時天下普旱,會稽、餘姚特甚,米斗直五百,人有相鬻者。
 二年三月,旱。三年六月,旱。時王導以天下新定,務在遵養,不任刑罰,遂盜賊公行,頻五年亢旱,亦舒綬之應也。



 康帝建元元年五月,旱。



 穆帝永和元年五月,旱。是時帝在衣強褓,褚太后臨朝,如明穆太后故事。五年七月不雨,至於十月。六年夏,旱。八年夏,旱。九年春,旱。



 升平三年冬,大旱。四年冬,大旱。



 哀帝隆和元年夏,旱。是時桓溫強恣,權制朝廷,僭踰之罰也。



 海西公太和元年夏,旱。四年冬,旱。涼州春旱至夏。



 簡文帝咸安二年十月,大旱,饑。自永和至是,嗣主幼沖,桓溫陵僭,用兵征伐,百姓怨苦。



 孝武帝寧康元年三月,旱。是時桓溫入覲高平陵,闔朝致拜,踰僭之應也。
 三年冬,旱。



 太元四年夏,大旱。八年六月,旱。十年七月,旱,饑。初,八年破苻堅,九年諸將略地,有事徐豫,楊亮、趙統攻討巴沔。是年正月,謝安又出鎮廣陵,使子琰進次彭城,頻有軍役。



 十三年六月,旱。去歲北府遣戍胡陸,荊州經略河南。是年夏,郭銓置戍野王,又遣軍破黃淮。



 十五年七月,旱。十七年,秋旱至冬。是時烈宗仁恕,信任會稽王道子,政
 事舒緩。又茹千秋為驃騎諮議,竊弄主相威福。又比丘尼乳母親黨及婢僕之子階緣近習,臨部領眾。又所在多上春竟囚,不以其辜,建康獄吏,枉暴既甚。此又僭踰不從冤濫之罰。



 安帝隆安二年冬,旱,寒甚。四年五月,旱。五年,夏秋大旱。十二月,不雨。時孫恩作亂,桓玄疑貳,迫殺殷仲堪,而朝廷即授以荊州之任,司馬元顯又諷百僚悉使敬己,內外騷動,兵革煩興。此皆陵僭憂愁之應
 也。



 元興元年七月,大饑。九月、十月不雨,泉水涸。二年六月,不雨。冬,又旱。時桓玄奢僭,十二月遂篡位。三年八月,不雨。



 義熙四年冬,不雨。六年九月,不雨。八年十月,不雨。九年,秋冬不雨。十年九月,旱。十二月又旱,井瀆多竭。是時軍役煩興。



 詩妖



 魏明帝太和中,京師歌《兜鈴曹子》,其唱曰「其柰汝曹何」,
 此詩妖也。其後曹爽見誅,曹氏遂廢。



 景初初,童謠曰:「阿公阿公駕馬車,不意阿公東渡河,阿公來還當柰何!」及宣帝遼東歸,至白屋,當還鎮長安。會帝疾篤,急召之,乃乘追鋒車東渡河,終如童謠之言。



 齊王嘉平中,有謠曰:「白馬素羈西南馳,其誰乘者朱虎騎。」朱虎者,楚王小字也。王凌、令狐愚聞此謠,謀立彪。事發,凌等伏誅,彪賜死。



 吳孫亮初,童謠曰;「吁汝恪,何若若,蘆葦單衣篾鉤絡,於何相求常子閣。」「常子閣」者,反語石子堈也。鉤絡,鉤帶也。及諸葛恪死,果以葦席裹身,篾束其要,投之石子堈。後
 聽恪故吏收斂,求之此堈云。



 孫亮初,公安有白鼉鳴。童謠曰:「白鼉鳴,龜背平。南郡城中可長生,守死不去義無成。」「南郡城中可長生」者,有急易以逃也。明年,諸葛恪敗,弟融鎮公安,亦見襲,融刮金印龜服之而死。鼉有鱗介,甲兵之象。又曰,白祥也。



 孫休永安二年,將守質子群聚嬉戲,有異小兒忽來言曰:「三公鋤,司馬如。」又曰:「我非人,熒惑星也。」言畢上昇,仰視若曳一匹練,有頃沒。干寶曰:「後四年而蜀亡,六年而魏廢,二十一年而吳平。」於是九服歸晉。魏與吳蜀並戰國,「三公鋤,司馬如」之謂也。



 孫皓遣使者祭石印山下妖祠,使者因以丹書巖曰:「楚九州渚,吳九州都。揚州士,作天子。四世治,太平矣。」皓聞之,意益張,曰:「從大皇帝至朕四世,太平之主非朕復誰!」恣虐踰甚,尋以降亡,近詩妖也。



 孫皓天紀中,童謠曰:「阿童復阿童,銜刀游渡江。不畏岸上獸,但畏水中龍。」武帝聞之,加王浚龍驤將軍。及征吳,江西眾軍無過者,而王浚先定秣陵。



 武帝太康三年平吳後,江南童謠曰:「局縮肉,數橫目,中國當敗吳當復。」又曰:「宮門柱,且當朽,吳當復,在三十年後。」又曰:「雞鳴不拊翼,吳復不用力。」于時吳人皆謂在孫
 氏子孫,故竊發為亂者相繼。案「橫目」者四字,自吳亡至元帝興幾四十年,元帝興於江東,皆如童謠之言焉。元帝心而而少斷,「局縮肉」者,有所斥也。



 太康末,京洛為《折楊柳》之歌,其曲始有兵革苦辛之辭,終以擒獲斬截之事。是時三楊貴盛而被族滅,太后廢黜,幽死中宮,「折楊柳」之應也。



 惠帝永熙中,河內溫縣有人如狂,造書曰:「光光文長,大戟為牆。毒藥雖行,戟還自傷。」又曰:「兩火沒地,哀哉秋蘭。歸形街郵,終為人嘆。」及楊駿居內府,以戟為衛,死時又為戟所害傷。楊后被廢,賈后絕其膳八日而崩,葬街郵
 亭北,百姓哀之也。雨火,武帝諱,蘭,楊后字也。其時又有童謠曰:「二月末,三月初,荊筆楊板行詔書,宮中大馬幾作驢。」此時楊駿專權,楚王用事,故言「荊筆楊板」。二人不誅,則君臣禮悖,故云「幾作驢」也。



 元康中,京洛童謠曰:「南風起,吹白沙,遙望魯國何嵯峨,千歲髑髏生齒牙。」又曰:「城東馬子莫嚨哅,此至來年纏女閤。」南風,賈后字也。白,晉行也。沙門,太子小名也。魯,賈謐國也。言賈后將與謐為亂,以危太子,而趙王因釁咀嚼豪賢,以成篡奪,不得其死之應也。



 元康中,天下商農通著大鄣日。時童謠曰:「屠蘇鄣日覆
 兩耳,當見瞎兒作天子。」及趙王倫篡位,其目實眇焉。趙王倫既篡,洛中童謠曰:「獸從北來鼻頭汗,龍從南來登城看,水從西來河灌灌。」數月而齊王、成都、河間義兵同會誅倫。案成都西籓而在鄴,故曰「獸從北來。」齊東籓而在許,故曰「龍從南來。」河間水源而在關中,故曰「水從西來」。齊留輔政,居于宮西,又有無君之心,故言「登城看」也。



 太安中,童謠曰:「五馬游渡江,一馬化為龍。」後中原大亂,宗籓多絕,唯瑯邪、汝南、西陽、南頓、彭城同至江東,而元帝嗣統矣。



 司馬越還洛,有童謠曰:「洛中大鼠長尺二,若不早去大狗至。」及茍晞將破汲桑,又謠曰:「元超兄弟大
 落度,上桑打椹為茍作。」由是越惡晞,奪其兗州,隙難遂構焉。



 愍帝初,有童謠曰:「天子何在豆田中。」至建興四年,帝降劉曜,在城東豆田壁中。



 建興中,江南謠歌曰:訇如白坑破,合集持作。揚州破換敗,吳興覆瓿甊。」案白者,晉行。坑器有口屬甕,瓦甕質剛,亦金之類也。「訇如白坑破」者,言二都傾覆,王室大壞也。「合集持作」者,元帝鳩集遺餘,以主社稷,未能剋復中原,但偏王江南,故其喻也。及石頭之事,六軍大潰,兵人抄掠京邑,爰及二宮。其後三年,錢鳳復攻京邑,阻水
 而守,相持月餘日,焚燒城邑,井堙木刊矣。鳳等敗退,沈充將其黨還吳興,官軍踵之,蹈藉郡縣,充父子授首,黨與誅者以百數。所謂「揚州破換敗,吳興覆瓿甊」,瓿甊瓦器,又小於也。



 明帝太寧初,童謠曰:「惻惻力力,放馬山側。大馬死,小馬餓。高山崩,石自破。」及明帝崩,成帝幼,為蘇峻所逼,遷于石頭,御膳不足,此「大馬死,小馬餓」也。高山,峻也,又言峻尋死。石,峻弟蘇石也。峻死後,石據石頭,尋為諸公所破,復是崩山石破之應也。



 成帝之末,又有童謠曰:「蓋蓋何隆隆,駕車入梓宮。」少日
 而宮車晏駕。



 咸康二年十二月,河北謠云:「麥入土,殺石武。」後如謠言。



 庾亮初鎮武昌,出至石頭,百姓於岸上歌曰:「庾公上武昌,翩翩如飛鳥。庾公還揚州,白馬牽旒旐。」又曰:「庾公初上時,翩翩如飛烏。庾公還揚州,白馬牽流蘇。」後連徵不入,及薨於鎮,以喪還都葬,皆如謠言。



 穆帝升平中,童兒輩忽歌於道曰《阿子聞》,曲終輒云「阿子汝聞不」?無幾而帝崩,太后哭之曰:「阿子汝聞不?」



 升平末,俗間忽作《廉歌》,有扈謙者聞之曰:「廉者,臨也。歌云『白門廉,宮庭廉』,內外悉臨,國家其大諱乎!」少時而穆
 帝晏駕。



 哀帝隆和初,童謠曰:「升平不滿斗,隆和那得久!桓公入石頭,陛下徒跣走。」朝廷聞而惡之,改年曰興寧。人復歌曰:「雖復改興寧,亦復無聊生。」哀旁壽崩。升平五年而穆帝崩,「不滿斗」,升平不至十年也。



 海西公太和中,百姓歌曰:「青青御路楊,白馬紫遊韁。汝非皇太子,那得甘露漿?」識者曰:「白者,金行,馬者,國族。紫為奪正之色,明以紫間朱也。」海西公尋廢,其三子並非海西公之子,縊以馬韁。死之明日,南方獻甘露馬。



 太和末,童謠曰:「犁牛耕御路,白門種小麥。」及海西公被
 廢,百姓耕其門以種小麥,遂如謠言。



 海西公初生皇子,百姓歌云:「鳳皇生一雛,天下莫不喜。本言是馬駒,今定成龍子。」其歌甚美,其旨甚微。海西公不男,使左右向龍與內侍接,生子,以為己子。



 桓石民為荊州,鎮上明,百姓忽歌曰「黃曇子」。曲中又曰:「黃曇英,揚州大佛來上明。」頃之而桓石民死,王忱為荊州。黃曇子乃是王忱字也。忱小字佛大,是「大佛來上明」也。



 孝武帝太元末,京口謠曰:「黃雌雞,莫作雄父啼。一旦去毛衣,衣被拉颯棲。」尋而王恭起兵誅王國寶,旋為劉牢之
 所敗,故言「拉颯棲」也。



 會稽王道子於東府造土山,名曰靈秀山。無幾而孫恩作亂,再踐會稽。會稽,道子所封;靈秀,孫恩之字也



 庾楷鎮歷陽,百姓歌曰:「重羅黎,重羅黎,使君南上無還時。」後楷南奔桓玄,為玄所誅。



 殷仲堪在荊州,童謠曰:「芒籠目,繩縛腹。殷當敗,桓當復。」未幾而仲堪敗,桓玄遂有荊州。



 王恭鎮京口,舉兵誅王國寶。百姓謠云:「昔年食白飯,今年食麥麩。天公誅謫汝,教汝捻嚨喉。嚨喉喝復喝,京口敗復敗。」識者曰:「昔年食白飯,言得志也。今年食麥麩,麩
 粗穢,其精已去,明將敗也,天公將加譴謫而誅之也。捻嚨喉,氣不通,死之祥也。敗復敗,丁寧之辭也。」恭尋死,京都又大行亥疾,而喉並喝焉。



 王恭在京口,百姓間忽云;「黃頭小兒欲作賊,阿公在城,下指縛得。」又云:「黃頭小人欲作亂,賴得金刀作籓扞。」黃字上恭字頭也,小人恭字下也,尋如謠言者焉。



 安帝隆安中,百姓忽作《懊憹》之歌,其曲曰:「草生可攬結,女兒可攬擷。」尋而桓玄篡位,義旗以三月二日掃定京都,誅之。玄之宮女及逆黨之家子女妓妾悉為軍賞,東及甌越,北流淮泗,皆人有所獲。故言時則草可結,事則
 女可擷也。



 桓玄既篡,童謠曰:「草生及馬腹,烏啄桓玄目。」及玄敗,走至江陵,時正五月中,誅如其期焉。



 安帝義熙初,童謠曰:「官家養蘆化成荻,蘆生不止自成積。」其時官養盧龍,寵以金紫,奉以名州,養之極也。而龍不能懷我好音,舉兵內伐,遂成仇敵也。「蘆生不止自成積」,及盧龍之敗,斬伐其黨,猶如草木以成積也。



 盧龍據廣州,人為之謠曰:「蘆生漫漫竟天半。」後擠上流數州之地,內逼京輦,應「天半」之言。



 義熙二年,小兒相逢於道,輒舉其兩手曰「盧健健」,次曰「
 斗歎斗歎」,末曰「翁年老翁年老」。當時莫知所謂。其後盧龍內逼,舟艦蓋川,「健健」之謂也。既至查浦,屢剋期欲與官斗,「斗歎」之應也。「翁年老」,群公有期頤之慶,知妖逆之徒自然消殄也。其時復有謠言曰;「盧橙橙,逐水流,東風忽如起,那得入石頭!」盧龍果敗,不得入石頭也。



 昔溫嶠令郭景純卜己與庾亮吉兇,景純云:「元吉。」嶠語亮曰:「景純每筮是,不敢盡言。吾等與國家同安危,而曰『元吉』,是事有成也。」於是協同討滅王敦。



 苻堅初,童謠云:「阿堅連牽三十年,後若欲敗時,當在江湖邊。」及堅在位凡三十年,敗於淝水,是其應也。又謠語
 云:「河水清復清,苻堅死新城。」及堅為姚萇所殺,死於新城。復謠歌云:「魚羊田升當滅秦。」識者以為「魚羊,鮮也;田升,卑也,堅自號秦,言滅之者鮮卑也。」其群臣諫堅,令盡誅鮮卑,堅不從。及淮南敗還,初為慕容沖所攻,又為姚萇所殺,身死國滅。



 毛蟲之孽



 武帝太康六年,南陽獻兩足猛獸,此毛蟲之孽也。識者為其文曰:「武形有虧,金獸失儀,聖主應天,期異何為!」言兆亂也。京房《易傳》曰:「足少者,下不勝任也。」干寶以為:「獸者陰精,居于陽,金獸也。南陽,火名也。金精入火而失其
 形,王室亂之妖也。」六,水數,言水數既極,火慝得作,而金受其敗也。至元康九年,始殺太子,距此十四年。二七十四,火始終相乘之數也。自帝受命,至愍懷之廢,凡三十五年焉。



 太康七年十一月丙辰,四角獸見于河間,河間王顒獲以獻。天戒若曰,角,兵象也,四者,四方之象,當有兵亂起於四方。後河間王遂連四方之兵,作為亂階,殆其應也。



 懷帝永嘉五年,蝘鼠出延陵。郭景純筮之曰:「此郡東之縣,當有妖人欲稱制者,亦尋自死矣。」其後吳興徐馥作亂,殺太守袁琇,馥亦時滅,是其應也。



 成帝咸和六年正月丁巳,會州郡秀孝於樂賢堂,有麏見於前,獲之。孫盛以為吉祥。夫秀孝,天下之彥士;樂賢堂,所以樂養賢也。自喪亂以後,風教陵夷,秀孝策試,乏四科之實。麏興於前,或斯故乎?



 哀帝隆和元年十月甲申,有麈入東海第。百姓讙言曰:「麈入東海第」,識者怪之。及海西廢為東海王,乃入其第。



 孝武太元十三年四月癸巳,祠廟畢,有兔行廟堂上。天戒若曰,兔,野物也,而集宗廟之堂,不祥莫之甚焉。



 犬禍



 公孫文懿家有犬,冠幘絳衣上屋,此犬禍也。屋上,亢陽
 高危之地。天戒若曰,亢陽無上,偷自尊高,狗而冠者也。及文懿自立為燕王,果為魏所滅。京房《易傳》曰:「君不正,臣欲篡,厥妖狗出朝門。」



 魏侍中應璩在直廬,欻見一白狗出門,問眾人,無見者。踰年卒,近犬禍也。



 吳諸葛恪征淮南歸,將朝會,犬銜引其衣。恪曰:「犬不欲我行乎?」還坐。有頃復起,犬又銜衣,乃令逐犬,遂升車,入而被害。



 武帝太康九年,幽州有犬,鼻行地三百餘步。天戒若曰,是時帝不思和嶠之言,卒立惠帝,以致衰亂,是言不從之罰也。



 惠帝元康中,吳郡婁縣人家聞地中有犬子聲,掘之,得雌雄各一。還置窟中,覆以磨石,經宿失所在。天戒若曰,帝既衰弱,籓王相譖,故有犬禍。



 永興元年,丹陽內史朱逵家犬生三子,皆無頭。後逵為揚州刺史曹武所殺。



 孝懷帝永嘉五年,吳郡嘉興張林家狗人言云:「天下人餓死。」於是果有二胡之亂,天下饑荒焉。



 愍帝建興元年,狗與豬交。案《漢書》,景帝時有此,以為悖亂之氣,亦犬豕禍也。犬,兵革之占也。豕,北方匈奴之象。逆言失聽,異類相交,必生害也。餓而帝沒于胡,是其應
 也。



 元帝太興中,吳郡太守張懋聞齋內床下犬聲,求而不得。既而地自坼,見有二犬子,取而養之,皆死。尋而懋為沈充所害。京房《易傳》曰:「讒臣在側,則犬生妖。」



 太興四年,廬江灊縣何旭家忽聞地中有犬子聲,掘之得一母犬,青釐色,狀甚羸瘦,走入草中,不知所在。視其處有二犬子,一雄一雌,哺而養之,雌死雄活。及長為犬,善噬獸。其後旭里中為蠻所沒。



 安帝隆安初,吳郡治下狗恒夜吠,聚高橋上,人家狗有限而吠聲甚眾。或有夜覘視之云:「一狗假有兩三頭,皆
 前向亂吠。」無幾,孫恩亂於吳會焉。是時輔國將軍孫無終家于既陽,地中聞犬子聲,尋而地斥,有二犬子,皆白色,一雄一雌,取而養之,皆死。後無終為桓玄所誅滅。案《尸子》曰:「地中有犬,名曰地狼。」《夏鼎志》曰;「掘地得犬,名曰賈。」此蓋自然之物,不應出而出,為犬禍也。



 桓玄將拜楚王,已設拜席,群官陪位。玄未及出,有狗來便其席,莫不驚怪。玄性猜暴,竟無言者,逐狗改席而已。天戒若曰,桓玄無德而叨竊大位,故犬便其席,示其妄據之甚也。八十日玄敗亡焉。



 白眚白祥



 魏明帝青龍三年正月乙亥,隕石于壽光。案《左氏傳》「隕石,星也」,劉歆說曰:「庶眾惟星隕於宋者,象宋襄公將得諸侯而不終也。」秦始皇時有隕石,班固以為:「石,陰類也。又白祥,臣將危君。」是後宣帝得政云。



 武帝太康五年五月丁巳,隕石於溫及河陽各二。六年正月,隕石於溫,三。



 成帝咸和八年五月,星隕于肥鄉,一。九年正月,隕石于涼州,二。



 吳孫亮五鳳二年五月,陽羨縣離里山大石自立。案京房《易傳》曰「庶士為天子之祥也」,其說曰:「石立於山同姓,
 平地異姓。」干寶以為「孫皓承廢故之家得位,其應也。」或曰孫休見立之祥也。



 武帝太康十年,洛陽宮西宜秋里石生地中,始高三尺,如香鈩形,後如傴人,槃薄不可掘。案劉向說,此白眚也。明年宮車晏駕,王室始騷,卒以亂亡。京房《易傳》曰:「石立如人,庶士為天下雄。」此近之矣。



 惠帝元康五年十二月,有石生于宜年里。永康元年,襄陽郡上言,得鳴石,撞之,聲聞七八里。太安元年,丹陽湖熟縣夏架湖有大石,浮二百步而登岸,民驚噪相告曰:「石來。」干寶曰:「尋有石冰入建鄴。」



 車騎大將軍、東嬴王騰自并州遷鎮鄴,行次真定。時久積雪,而當門前方數丈獨消釋,騰怪而掘之,得玉馬,高尺許,口齒缺。騰以馬者國姓,上送之,以為瑞。然馬無齒則不得食,妖祥之兆,衰亡之徵。案占,此白祥也。是後騰為汲桑所殺,而天下遂亂。



 武帝泰始八年五月,蜀地雨白毛,此白祥也。時益州刺史皇甫晏伐汶山胡,從事何旅固諫,不從,牙門張弘等困眾之怨,誣晏謀逆,害之。京房《易傳》曰:「前樂後憂,厥妖天雨羽。」又曰:「邪人進,賢人逃,天雨毛。」其《易妖》曰:「天雨毛羽,貴人出走。」三占皆應。



 惠帝永寧元軍,齊王冏舉義軍。軍中有小兒,出於襄城繁昌縣,年八歲,髮體悉白,頗能卜,於《洪範》,白祥也。



 成帝咸康初,地生毛,近白祥也。孫盛以為人勞之異也。是後石季龍滅而中原向化,將相皆甘心焉。於是方鎮屢革,邊戍仍遷,皆擁帶部曲,動有萬數。其間征伐征賦,役無寧歲,天下勞擾,百姓疲怨。



 咸康三年六月,地生毛。



 孝武太元二年五月,京都地生毛,至四年而氐賊次襄國,圍彭城,向廣陵,征戍仍出,兵連年不解。



 太元十四年四月,京都地生毛。是時苻堅滅後,經略多
 事,人勞之應也。十七年四月,地生毛。



 安帝隆安四年四月乙未,地生毛,或白或黑。元興三年五月,江陵地生毛。是後江陵見襲,交戰者數矣。



 義熙三年三月,地生白毛。十年三月地生毛。明年,王旅西討司馬休之。又明年,北掃關洛。



 木沴金



 魏齊王正始末,河南尹李勝治聽事,有小材激墮,楇受
 符吏石彪頭,斷之,此木沴金也。勝後旬日而敗。



 惠帝元康八年五月,郊禖壇石中破為二,此木沴金也。郊禖壇者,求子之神位,無故自毀,太子將危之象也。明年愍懷廢死。



 孝武帝太元十年四月,謝安出鎮廣陵,始發石頭,金鼓無故自破。此木沴金之異也,天意也。天戒若曰,安徒揚經略之聲,終無其實,鉦鼓不用之象也。月餘,以疾還而薨。



 《傳》曰:「視之不明,是謂不哲,厥咎舒,厥罰恒燠,厥極疾。時則有草妖,時則有蠃蟲之孽,時則有羊禍,時則有目痾,
 時則有赤眚赤祥。惟水沴火。」視之不明,是謂不哲。哲,知也。《詩》云:「爾德不明,以亡陪亡卿。不明爾德,以亡背亡側。」言上不明,暗昧蔽惑,則不能知善惡,親近習,長同類,亡功者受賞,有罪者不殺,百官廢亂,失在舒緩,故其咎舒也。盛夏日長,暑以養物,政弛緩,故其罰常燠也。燠則冬溫,春夏不和,傷病疾人,其極疾也。誅不行則霜不殺草,繇臣下則殺不以時,故有草妖。凡妖,貌則以服,言則以詩,聽則以聲。視不以色者,五色,物之大分也,在於眚祥,故聖人以為草妖,失物柄之明者也。溫燠生蟲,故有蠃蟲之孽,謂螟螣之類當死不死,當生而不生,或多於
 故而為災也。劉歆以為屬思心不容。於《易》,剛而苞柔為《離》,《離》為火,為目。羊上角下蹄,剛而苞柔,羊大目而不精明,視氣毀,故有羊禍。一日,暑歲羊多疫死,及為怪,亦是也。及人,則多病目者,故有目痾。火色赤,故有赤眚赤祥。凡視傷者,病火氣;火氣傷,則水沴之。其極疾者順之,其福曰壽。劉歆《視傳》曰有羽蟲之孽,雞禍。說以為於天文南方朱張為鳥星,故為羽蟲。禍亦從羽,故為雞。雞於《易》自在《巽》,說非是。



 庶徵之恒燠,劉向以為《春秋》無冰也。小燠不書,無冰然後書,舉其大者也。京房《易傳》曰:「祿不遂行茲謂欺,廝咎燠。其燠,雨雲四至而溫。臣安祿樂逸茲謂亂,
 燠而生蟲。知罪不誅茲謂舒,其燠,夏則暑殺人,冬則物華實。重過不誅茲謂亡徵,其咎當寒而燠盡六日也。」



 吳孫亮建興元年九月,桃李華,孫權世政煩賦重,人凋於役。是時諸葛恪始輔政,息校官,原逋責,除關梁,崇寬厚,此舒緩之應也。一說桃李寒華為草妖,或屬華孽。



 魏少帝景元三年十月,桃李華。時少帝深樹恩德,事崇優緩,此其應也。



 惠帝元康二年二月,巴西郡界草皆生華,結子如麥,可食。時帝初即位,楚王瑋矯詔誅汝南王亮及太保衛瓘,帝不能察。今非時草結實,此恒燠寬舒之罰。



 穆帝永和九年十二月,桃李華,是時簡文輔政,事多馳略,舒緩之應也。



 草妖



 漢獻帝建安二十五年春正月,魏武帝在洛陽起建始殿,伐濯龍樹而血出,又掘徙梨,根傷亦血出。帝惡之,遂寢疾,是月崩。蓋草妖,又赤祥,是歲魏文帝黃初元年也。



 吳孫亮五鳳元年六月,交止稗草化為稻。昔三苗將亡,五穀變種,此草妖也。其後亮廢。



 蜀劉禪景耀五年,宮中大樹無故自折。譙周憂之,無所與言,乃書柱曰:「眾而大,其之會。具而授,若何復。」言曹者
 眾也,魏者大也,眾而大,天下其當會也。具而授,如何復有立者乎?蜀果亡,如周言,此草妖也。



 吳孫皓天璽元年,吳郡臨平湖自漢末穢塞,是時一夕忽開除無草。長老相傳:此湖塞,天下亂;此湖開,天下平。吳尋亡而九服為一。



 天紀三年八月,建鄴有鬼目菜於工黃狗家生,依緣棗樹,長丈餘,莖廣四寸,厚二分。又有賣菜生工吳平家,高四尺,如枇杷形,上圓,徑一尺八寸,莖廣五寸,兩邊生葉,綠色。東觀案圖,名鬼目作芝萆,賣菜作平慮,遂以狗為侍芝郎,平為平慮郎,皆銀印青綬。干寶曰:明年平吳,王
 浚止船正得平渚,姓名顯然,指事之徵也。黃狗者,吳以土運承漢,故初有黃龍之瑞。及其季年,而有鬼目之妖託黃狗之家。黃稱不改,而貴賤大殊,天道精微之應敢也。



 惠帝元康二年春,巴西郡界竹生花,紫色,結實如麥,外皮青,中赤白,味甘。



 元康九年六月庚子,有桑生東宮西廂,日長尺餘,甲辰枯死。此與殷太戊同妖,太子不能悟,故至廢戮也。班固稱「野木生朝而暴長,小人將暴居大臣之位,危國亡家之象,朝將為墟也。」是後孫秀、張林用事,遂至大亂。



 永康元年四月,立皇孫臧為皇太孫。五月甲子,就東宮,
 桑又生於西廂。明年,趙王倫篡位,鴆殺臧,此與愍懷同妖也。是月,壯武國有桑化為柏,而張華遇害。壯武,華之封邑也。



 孝懷帝永嘉二年冬,項縣桑樹有聲如解材,人謂之桑樹哭。案劉向說,「桑者喪也」,又為哭聲,不祥之甚。是時京師虛弱,胡寇交侵,東海王越無衛國之心,四年冬季而南出,五年春薨于此城。石勒邀其眾,圍而射之,王公以下至眾庶,死者十餘萬人。又剖越棺,焚其屍。是敗也,中原無所請命,洛京亦尋覆沒,桑哭之應也。



 六年五月,無錫縣有四株茱萸樹,相樛而生,狀若連理。
 先是,郭景純筮延陵蝘鼠,遇《臨》之《益》,曰:「後當復有妖樹生,若瑞而非,辛螫之木也,儻有此,東西數百里必有作逆者。」及此木生,其後徐馥果作亂,亦草妖也。郭又以為「木不曲直」。其七月,豫章郡有樟樹久枯,是月忽更榮茂,與漢昌邑枯社復生同占。是懷愍淪陷之徵,元帝中興之應也。



 明帝太寧元年九月,會稽剡縣木生如人面。是後王敦稱兵作逆,禍敗無成。昔漢哀成之世並有此妖,而人貌備具,故春禍亦大。今此但如人面而已,故其變也輕矣。



 成帝咸和六年五月癸亥,曲阿有柳樹枯倒六載,是日
 忽復起生,至九年五月甲戌,吳縣吳雄家有死榆樹,是日因風雨起生,與漢上林斷柳起生同象。初,康帝為吳王,于時雖改封瑯邪,而猶食吳郡為邑,是帝越正體饗國之象也。曲阿先亦吳地,象見吳邑雄之舍,又天意乎!



 哀帝興寧三年五月癸卯,廬陵西昌縣修明家有僵栗樹,是日忽復起生。時孝武年始四歲,俄而哀帝崩,海西即位,未幾而廢,簡文越自籓王,入纂大業,登阼享國,又不踰二年,而孝武嗣統。帝諱昌明,識者竊謂西昌修明之祥,帝諱實應焉。是亦與漢宣帝同象也。



 海西太和元年,涼州楊樹生松。天戒若曰,松者不改柯
 易葉,楊者柔脆之木,今松生於楊,豈非永久之業將集危亡之地邪?是時張天錫稱雄於涼州,尋而降苻堅。



 孝武太元十四年六月,建寧郡銅樂縣枯樹斷折,忽然自立相屬。京房《易傳》曰:「棄正作淫,厥妖木斷自屬。妃后有專,木仆反立。」是時正道多僻,其後張夫人專寵,及旁崩,兆庶歸咎張氏焉。



 安帝元興三年,荊、江二州界竹生實,如麥。



 義熙二年九月,揚武將軍營士陳蓋家有苦賣菜,莖高四尺六寸,廣三尺二寸,厚三寸,亦草妖也。此殆與吳終同象。識者以為苦賣者,買勤苦也。自後歲歲征討,百姓
 勞苦,是買苦也。十餘年中,姚泓滅,兵始戢,是苦賣之應也。



 義熙中,宮城上及御道左右皆生蒺藜,亦草妖也。蒺藜有刺,不可踐而行。生宮墻及馳道,天戒若曰,人君不聽政,雖有宮室馳道,若空廢也,故生蒺藜。



 羽蟲之孽



 魏文帝黃初四年五月,有鵜鶘鳥集靈芝池。案劉向說,此羽蟲之孽,又青祥也。詔曰:「此詩人所謂污澤者也。《曹詩》『刺共公遠君子近小人』,今豈有賢智之士處于下位,否則斯鳥何為而至哉!其博舉天下俊德茂才獨行君
 子,以答曹人之刺。」於是楊彪、管寧之徒咸見薦舉,些所謂睹妖知懼者也。然猶不能優容亮直而多溺偏私矣。京房《易傳》曰「辟退有德,厥妖水鳥集於國中」。



 黃初元年,未央宮中又有燕生鷹,口爪俱赤,此與商紂、宋隱同象。



 景初元年,又有燕生巨鷇於衛國李蓋家,形若鷹,吻似燕,此羽蟲之孽,又赤眚也。高堂隆曰:「此魏室之大異,宜防鷹揚之臣於蕭墻之內。」其後宣帝起誅曹爽,遂有魏室。



 漢獻帝建安二十三年,禿鶖鳥集鄴宮文昌殿後池。明
 年,魏武王薨。魏文帝黃初三年,又集雒陽芳林園池。七年,又集。其夏,文帝崩。景初末,又集芳林園池。已前再至,輒有大喪,帝惡之。其年,明帝崩。



 蜀劉禪建興九年十月,江陽至江州有鳥從江南飛渡江北,不能達,墮水死者以千數。是時諸葛亮連年動眾,志吞中夏,而終死渭南,所圖不遂。又諸將分爭,頗喪徒旅,鳥北飛不能達墮水死者,皆有其象也。亮竟不能過渭,又其應乎!此與漢時楚國烏鬥墮泗水粗類矣。



 景初元年,陵霄闕始構,有鵲巢其上。鵲體白黑雜色,此羽蟲之孽,又白黑祥也。帝以問高堂隆,對曰:「《詩》云『惟鵲有巢,惟鳩居之』,今興起宮室而鵲來巢,此宮室未成身不得居之象也。天戒若曰,宮室未成,將有他姓制御之,不可不深慮。」於是帝改顏動色。



 吳孫權赤烏十二年四月,有兩烏銜鵲墮東館,權使領丞相朱據燎鵲以祭。案劉歆說,此羽蟲之孽,又黑祥也。視不明、聽不聰之罰也。是時權意溢德衰,信讒好殺,二子將危,將相俱殆,睹妖不悟,加之以燎,昧道之甚者也。明年,太子和廢,魯王霸賜死,朱據左遷,陸議憂卒,是其
 應也。東館,典教之府;鵲墮東館,又天意乎?



 吳孫權太元二年正月,封前太子和為南陽王,遣之長沙,有鵲巢其帆檣。和故宮僚聞之,皆憂慘,以為檣末傾危,非久安之象。是後果不得其死。



 孫亮建興二年十一月,有大鳥五見於春申,吳人以為鳳皇。明年,改元為五鳳。漢桓帝時有五色大鳥,司馬彪云:「政道衰缺,無以致鳳,乃羽蟲孽耳。」孫亮未有德政,孫峻驕暴方甚,此與桓帝同事也。案《瑞應圖》,大鳥似鳳而為孽者非一,宜皆是也。



 孫皓建衡三年,西苑言鳳皇集,以之改元,義同於亮。



 武帝泰始四年八月,有翟雉飛上閶闔門。天戒若曰,閶闔門非雉所止,猶殷宗雉登鼎耳之戒也。



 惠帝永康元年,趙王倫既篡,京師得異鳥,莫能名。倫使人持出,周旋城邑市以問人。積日,宮西有小兒見之,遂自言曰:「服留鳥翳。」持者即還白倫,倫使更求,又見之,乃將入宮,密籠鳥,并閉小兒戶中,明日視之,悉不見。此羽蟲之孽。時趙王倫有目瘤之疾,言服留者,謂倫留將服其罪也。尋而倫誅。



 趙王倫篡位,有鶉入太極殿,雉集東堂。天戒若曰,太極東堂皆朝享聽政之所,而鶉雉同日集之者,趙王倫不
 當居此位也。《詩》云:「鵲之強強,鶉之奔奔,人之無良,我以為君。」其此之謂乎!尋而倫誅。



 孝懷帝永嘉元年二月,洛陽東北步廣里地陷,有蒼白二色鵝出,蒼者飛翔沖天,白者止焉。此羽蟲之孽,又黑白祥也。陳留董養曰:「步廣,周之狄泉,盟會地也。白者,金色,國之行也。蒼為胡象,其可盡言乎?」是後,劉元海、石勒相繼亂華。



 明帝太寧三年八月庚戌,有大鳥二,蒼黑色,翼廣一丈四尺,其一集司徒府,射而殺之,其一集市北家人舍,亦獲焉。此羽蟲之孽,又黑祥也。及閏月戊子而帝崩,後
 遂有蘇峻、祖約之亂。



 成帝咸和二年正月,有五鷗鳥集殿庭,此又白祥也。是時庾亮茍違眾謀,將召蘇峻,有言不從之咎,故白祥先見也。三年二月,峻果作亂,宮掖焚毀,化為汙萊,此其應也。



 咸康八年七月,有白鷺集殿屋。是時康帝初即位,不永之祥也。後涉再期而帝崩。案劉向曰:「野鳥入處,宮室將空。」此其應也。



 海西初以興守三年二月即位,有野雉集于相風。此羽蟲之孽也。尋為桓溫所廢也。



 孝武帝太元十六年六月,鵲巢太極東頭鴟尾,又巢國子學堂西頭。十八年東宮始成,十九年正月鵲又巢其西門。此殆與魏景初同占。學堂,風教所聚;西頭,又金行之祥。及帝崩後,安皇嗣位,桓玄遂篡,風教乃頹,金行不競之象也。



 安帝義熙三年,龍驤將軍朱猗戍壽陽。婢炊飯,忽有群烏集灶,競來啄敢,婢驅遂不去。有獵狗咋殺兩烏,餘烏因共啄殺狗,又敢其肉,唯餘骨存。此亦羽蟲之孽,又黑祥也。明年六月,猗死,此其應也。



 羊禍



 成帝咸和二年五月,司徒王導廄羊生無後足,此羊禍也。京房《易傳》曰:「足少者,下不勝任也。」明年,蘇峻破京都,導與帝俱幽石頭,僅乃得免,是其應也。



 赤眚赤祥



 公孫文懿時,襄平北市生肉,長圍各數尺,有頭目口喙,無手足而動搖,此赤祥也。占曰:「有形不成,有體不聲,其國滅亡。」文懿尋為魏所誅。



 吳戍將鄧喜殺豬祠神,治畢懸之,忽見一人頭往食肉,喜引弓射中之,咋咋作聲,繞屋三日,近赤祥也。後人白喜謀北叛,闔門被誅。京房《易傳》曰:「山見葆,江于邑,邑有
 兵,狀如人頭,赤色。」



 武帝太康五年四月壬子,魯國池水變赤如血。



 七年十月,河陰有赤雪二頃。此赤祥也。是後四載而帝崩,王室遂亂。



 惠帝元康五年三月,呂縣有流血,東西百餘步,此赤祥也。至元康末,窮凶極亂,僵屍流血之應也。干寶以為「後八載而封雲亂徐州,殺傷數萬人」,是其應也。



 永康元年三月,尉氏雨血。夫政刑舒緩,則有常燠赤祥之妖。此歲正月,送愍懷太子,幽于許宮。天戒若曰,不宜緩恣姦人,將使太子冤死。惠帝愚眊不寤,是月愍懷遂
 斃。於是王室成釁,禍流天下。淖齒殺齊湣王日,天雨血霑衣。天以告也,此之謂乎?京房、《易傳》曰:「歸獄不解,茲謂追非,厥咎天雨血。茲謂不親,下有惡心,不出三年,無其宗。」又曰:「佞人祿,功臣戮,天雨血也。」



 愍帝建興元年十二月,河東地震,雨肉。四年十二月丙寅,丞相府斬督運令史淳于伯,血逆流上柱二丈三尺,此赤祥也。是時,後將軍褚裒鎮廣陵,丞相揚聲北伐,伯以督運稽留及役使贓罪,依軍法戮之。其息訴稱:「督運事訖,無所稽乏,受賕役使,罪不及死。兵家之勢,先聲後實,實是屯戍,非為征軍。自四年已來,運
 漕稽停,皆不以軍興法論。」僚佐莫之理。及有變,司直彈劾眾官,元帝不問,遂頻旱三年。干寶以為冤氣之應也。郭景純曰:「血者水類,同屬於《坎》。《坎》為法象,水平潤下,不宜逆流。此政有咎失之徵也。」



 劉聰偽建元元年正月,平陽地震,其崇明觀陷為池,水赤如血,赤氣至天,有赤龍奮迅而去。流星起于牽牛,入紫微,龍形委蛇,其光照地,落于平陽北十里。視之則肉,臭聞于平陽。長三十步,廣二十七步。肉旁常有哭聲,晝夜不止。數日,聰后劉氏產一蛇一獸,各害人而走。尋之不得,頃之見於隕肉之旁。是時,劉聰納劉殷三女,並為
 其後。天戒若曰,聰既自稱劉姓,三後又俱劉氏,逆骨肉之綱,成人倫之則。隕肉諸妖,其眚亦大。俄而劉氏死,哭聲自絕矣。



\end{pinyinscope}