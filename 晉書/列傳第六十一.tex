\article{列傳第六十一}

\begin{pinyinscope}

 儒林



 範平文立陳邵
 虞喜劉兆氾毓徐苗崔游範隆杜夷董景道續咸徐邈孔衍範宣韋謏範弘之王歡



 昔周德既衰,諸侯力政,禮經廢缺,雅頌陵夷。夫子將聖多能,固天攸縱,歎鳳鳥之不至,傷麟出之非時,於是乃刪《詩》《書》,定禮樂,贊《易》道,修《春秋》,載籍逸而復存,風雅變而還正。其後卜商、衛賜、田、吳、孫、孟之儔,或親稟微言,或傳聞大義,猶能彊晉存魯,籓魏卻秦,既抗禮於邦君,亦馳聲於海內。及嬴氏慘虐,棄德任刑,煬墳籍於埃塵,填
 儒林於坑阱,嚴是古之法,抵挾書之罪,先王徽烈,靡有孑遺。漢祖勃興,救焚拯溺,粗修禮律,未遑俎豆。逮於孝武,崇尚文儒。爰及東京,斯風不墜。於是傍求蠹簡,博訪遺書,創甲乙之科,擢賢良之舉,莫不紆青拖紫,服冕乘軒,或徒步而取公卿,或累旬以膺台鼎。故晉紳之士,靡然嚮風,餘芳遺烈,煥乎可紀者也。洎當塗草創,深務兵權,而主好斯文,朝多君子,鴻儒碩學,無乏於時。



 武帝受終,憂勞軍國,時既初並庸蜀,方事江湖,訓卒厲兵,務農積穀,猶復修立學校,臨幸辟雍。而荀顗以制度贊惟新,鄭沖以儒宗登保傅,茂先以博物參朝政,子真以好禮
 居秩宗,雖愧明揚,亦非遐棄。既而荊揚底定,區寓乂安,群公草封禪之儀,天子發謙沖之詔,未足比隆三代,固亦擅美一時。惠帝纘戎,朝昏政弛,釁起宮掖,禍成籓翰。惟懷逮愍,喪亂弘多,衣冠禮樂,掃地俱盡。元帝運鐘百六,光啟中興,賀、荀、刁、杜諸賢並稽古博文,財成禮度。雖尊儒勸學,亟降於綸言,東序西膠,未聞於弦誦。明皇聰睿,雅愛流略,簡文玄嘿,敦悅丘墳,乃招集學徒,弘獎風烈,並時艱祚促,未能詳備。有晉始自中朝,迄於江左,莫不崇飾華競,祖述虛玄,擯闕里之典經,習正始之餘論,指禮法為流俗,目縱誕以清高,遂使憲章弛廢,名教頹
 毀,五胡乘間而競逐,二京繼踵以淪胥,運極道消,可為長歎息者矣。鄭沖等名位既隆,自有列傳,其餘編之于左,以續前史《儒林》云。



 范平,字子安,吳郡錢塘人也。其先銍侯馥,避王莽之亂適吳,因家焉。平研覽墳素,遍該百氏,姚信、賀邵之徒皆從受業。吳時舉茂才,累遷臨海太守,政有異能。孫晞初,謝病還家,敦悅儒學。吳平,太康中,頻征不起,年六十九卒。有詔追加謚號曰文貞先生,賀循勒碑紀其德行。



 三子:奭、咸、泉,並以儒學至大官。泉子蔚,關內侯。家世好學,
 有書七千餘卷。遠近來讀者恒有百餘人,蔚為辦衣食。蔚子文才,亦幼知名。



 文立,字廣休,巴郡臨江人也。蜀時游太學,專《毛詩》、《三禮》,師事譙周,門人以立為顏回,陳壽、李虔為游夏,羅憲為子貢。仕至尚書。蜀平,舉秀才,除郎中。泰始初,拜濟陰太守,入為太子中庶子。上表請以諸葛亮、蔣琬、費禕等子孫流徙中畿,宜見敘用,一以慰巴蜀之心,其次傾吳人之望,事皆施行。詔曰:「太子中庶子文立忠貞清實,有思理器幹。前濟在陰,政事修明。後事東宮,盡輔導之節。昔
 光武平隴蜀,皆收其賢才以敘之,蓋所以拔幽滯而濟殊方也。其以立為散騎常侍。」蜀故尚書犍為程瓊雅有德業,與立深交。武帝聞其名,以問立,對曰:「臣至知其人,但年垂八十,稟姓謙退,無復當時之望,不以上聞耳。」瓊聞之曰:「廣休可謂不黨矣,故吾善夫人也。」時西域獻馬,帝問立:「馬何如?」對曰:「乞問太僕。」帝善之。遷衛尉。咸寧末,卒。所著章奏詩賦數十篇行於世。



 陳邵,字節良,東海襄賁人也。郡察孝廉,不就。以儒學徵為陳留內史,累遷燕王師。撰《周禮評》,甚有條貫,行於世。
 泰始中,詔曰:「燕王師陳邵清貞潔靜,行著邦族,篤志好古,博通六籍,耽悅典誥,老而不倦,宜在左右以篤儒教。可為給事中。」卒於官。



 虞喜,字仲寧,會稽餘姚人,光祿潭之族也。父察,吳征虜將軍。喜少立操行,博學好古。諸葛恢臨郡,屈為功曹。察孝廉,州舉秀才,司徒辟,皆不就。元帝初鎮江左,上疏薦喜。懷帝即位,公車徵拜博士,不就。喜邑人賀循為司空,先達貴顯,每詣喜,信宿忘歸,自云不能測也。



 太寧中,與臨海任旭俱以博士徵,不就。復下詔曰:「夫興化致政,莫
 尚乎崇道教,明退素也。喪亂以來,儒雅陵夷,每覽《子衿》之詩,未嘗不慨然。臨海任旭、會稽虞喜並潔靜其操,歲寒不移,研精墳典,居今行古,志操足以勵俗,博學足以明道,前雖不至,其更以博士徵之。」喜辭疾不赴。咸和末,詔公卿舉賢良方正直言之士,太常華恒舉喜為賢良。會國有軍事,不行。咸康初,內史何充上疏曰:「臣聞二八舉而四門穆,十亂用而天下安,徽猷克闡,有自來矣。方今聖德欽明,思恢遐烈,旌輿整駕,俟賢而動。伏見前賢良虞喜天挺貞素,高尚邈世,束修立德,皓首不倦,加以傍綜廣深,博聞彊識,鑽堅研微有弗及之勤,處靜味道
 無風塵之志,高枕柴門,怡然自足。宜使蒲輪紆衡,以旌殊操,一則翼贊大化,二則敦勵薄俗。」疏奏,詔曰:「尋陽翟湯、會稽虞喜並守道清貞,不營世務,耽學高尚,操擬古人。往雖徵命而不降屈,豈素絲難染而搜引禮簡乎!政道須賢,宜納諸廊廟,其並以散騎常侍征之。」又不起。



 永和初,有司奏稱十月殷祭,京兆府君當遷祧室,征西、豫章、潁川三府君初毀主,內外博議不能決。時喜在會稽,朝廷遣就喜諮訪焉。其見重如此。



 喜專心經傳,兼覽讖緯,乃著《安天論》以難渾、蓋,又釋《毛詩略》,注《孝經》,為《志林》三十篇。凡所注述數十萬言,行於世。年七十六卒,無子。
 弟豫,自有傳。



 劉兆,字延世,濟南東平人,漢廣川惠王之後也。兆博學洽聞,溫篤善誘,從受業者數千人。武帝時五辟公府,三徵博士,皆不就。安貧樂道,潛心著述,不出門庭數十年。以《春秋》一經而三家殊塗,諸儒是非之議紛然,互為仇敵,乃思三家之異,合而通之。《周禮》有調人之官,作《春秋調人》七萬餘言,皆論其首尾,使大義無乖,時有不合者,舉其長短以通之。又為《春秋左氏》解,名曰《全綜》,《公羊》、《穀梁》,解詁皆納經傳中,朱書以別之。又撰《周易訓註》,以正
 動二體互通其文。凡所贊述百餘萬言。



 嘗有人著靴騎驢至兆門外,曰:「吾欲見劉延世。」兆儒德道素,青州無稱其字者,門人大怒。兆曰:「聽前。」既進,踞床問兆曰:「聞君大學,比何所作?」兆答如上事,末云:「多有所疑。」客問之。兆說疑畢,客曰:「此易解耳。」因為辯釋疑者是非耳。兆別更立意,客一難,兆不能對。客去,已出門,兆欲留之,使人重呼還。客曰:「親親在此營葬,宜赴之,後當更來也。」既去,兆令人視葬家,不見此客,竟不知姓名。兆年六十六卒。有五子:卓、炤、耀、育、臍。



 氾毓,
 字稚春,濟北盧人也。奕世儒素,敦睦九族,客居青州,逮毓七世,時人號其家「兒無常父,衣無常主,」毓少履高操,安貧有志業。父終,居于墓所三十餘載,至晦朔,躬掃墳壟,循行封樹,還家則不出門庭。或薦之武帝,召補南陽王文學、祕書郎、太傅參軍,並不就。于時青土隱逸之士劉兆、徐苗等皆務教授,惟毓不蓄門人,清靜自守。時有好古慕德者諮詢,亦傾懷開誘,以一隅示之。合《三傳》為之解注,撰《春秋釋疑》、《肉刑論》,凡是述造七萬餘言。年七十一卒。



 徐苗,字叔胄,高密淳于人也。累世相承,皆以博士為郡守。曾祖華,有至行。嘗宿亭舍,夜有神人告之「亭欲崩」,遽出,得免。祖邵,為魏尚書郎,以廉直見稱。苗少家貧,晝執鉏耒,夜則吟誦。弱冠,與弟賈就博士濟南宋鈞受業,遂為儒宗。作《五經同異評》,又依道家著《玄微論》,前後所造數萬言,皆有義味。性抗烈,輕財貴義,兼有知人之鑒。弟患口癰,膿潰,苗為吮之。其兄弟皆早亡,撫養孤遺,慈愛聞于州里,田宅奴婢盡推與之。鄉鄰有死者,便輟耕助營棺郭,門生亡於家,即斂於講堂。其行己純至,類皆如此。遠近咸歸其義,師其行焉。郡察孝廉,州辟從事、治中、
 別駕、舉異行,公府五辟博士,再徵,並不就。武惠時計吏至臺,帝輒訪其安不。永寧二年卒,遺命濯巾浣衣,榆棺雜磚,露車載尸,葦席瓦器而已。



 崔遊,字子相,上黨人也。少好學,儒術甄明,恬靖謙退,自少及長,口未嘗語及財利。魏末,察孝廉,除相府舍人,出為氐池長,甚有惠政。以病免,遂為廢疾。泰始初,武帝祿敘文帝故府僚屬,就家拜郎中。年七十餘,猶敦學不倦,撰《喪服圖》,行於世。及劉元海僭位,命為御史大夫,固辭不就。卒於家,時年九十三。



 范隆,字玄嵩,鴈門人。父方,魏鴈門太守。隆在孕十五月,生而父亡。年四歲,又喪母,哀號之聲,感慟行路。單孤無緦功之親,疏族范廣愍而養之,迎歸教書,為立祠堂。隆好學修謹,奉廣如父。博通經籍,無所不覽,著《春秋三傳》,撰《三禮吉凶宗紀》,甚有條義。惠帝時,天下將亂,隆隱迹不應州郡之命,晝勤耕稼,夜誦書典。頗習祕歷陰陽之學,知并州將有氛祲之祥,故彌不復出仕。與上黨硃紀友善,嘗共紀遊山,見一父老於窮澗之濱。父老曰:「二公何為在此?」隆等拜之,仰視則不見。後與紀依于劉元海,
 元海以隆為大鴻臚,紀為太常,並封公。隆死于劉聰之世,聰贈太師。



 杜夷,字行齊,廬江灊人也。世以儒學稱,為郡著姓。夷少而恬泊,操尚貞素,居甚貧窘,不營產業,博覽經籍百家之書,算歷圖緯靡不畢究。寓居汝潁之間,十載足不出門。年四十餘,始還鄉里,閉門教授,生徒千人。惠帝時三察孝廉,州命別駕,永嘉初,公車徵拜博士,太傅、東海王越辟,並不就。懷帝詔王公舉賢良方正,刺史王敦以賀循為賢良,夷為方正,乃上疏曰:「臣聞有唐疇咨,元凱時
 登;漢武欽賢,俊彥響應,故能允協時雍,敷崇盛化。伏見太孫舍人會稽賀循、處士盧江杜夷履道彌高,清操絕俗,思學融通,才經王務。循宰二縣,皆有名績,備僚東宮,忠恪允著。夷清虛沖淡,與俗異軌,考盤空谷,肥遁匿跡。蓋經國之良寶,聘命之所急。若得待詔公車,承對冊問,必有忠讜良謨,弘益政道矣。」敦於是逼夷赴洛。夷遁於壽陽。鎮東將軍周馥,傾心禮接,引為參軍,夷辭之以疾。馥知不可屈,乃自詣夷,為起宅宇,供其醫藥。馥敗,夷歸舊居,道遇兵寇。刺史劉陶告盧江郡曰:「昔魏文侯軾干木之閭,齊相曹參尊崇蓋公,皆所以優賢表德,敦勵末
 俗。徵士杜君德懋行潔,高尚其志,頃流離道路,聞其頓躓,刺史忝任,不能崇飾有道,而使高操之士有此艱屯。今遣吏宣慰,郡可遣一吏,縣五吏,恒營恤之,常以市租供給家人糧廩,勿令闕乏。」尋以胡寇,又移渡江,王導遣吏周贍之。元帝為丞相,教曰:「今大義頹替,禮典無宗,朝廷滯義莫能攸正,宜特立儒林祭酒官,以弘其事。處士杜夷棲情遺遠,確然絕俗,才學精博,道行優備,其以夷為祭酒。」夷辭疾,未嘗朝會。帝常欲詣夷,夷陳萬乘之主不宜往庶人之家。帝乃與夷書曰:「吾與足下雖情在忘言,然虛心歷載。正以足下羸疾,故欲相省,寧論常儀也!」
 又除國子祭酒。建武中,令曰:「國子祭酒杜夷安貧樂道,靜志衡門,日不暇給,雖原憲無以加也。其賜穀二百斛。」皇太子三至夷第,執經問義。夷雖逼時命,亦未嘗朝謁,國有大政,恒就夷諮訪焉。明帝即位,夷自表請退。詔曰:「先王之道將墜於地,君下帷研思,今之劉、楊。搢紳之徒景仰軌訓,豈得高退,而朕靡所取則焉!」太寧元年卒,年六十六。贈大鴻臚,謚曰貞子。夷臨終,遺命子晏曰:「吾少不出身,頃雖見羈錄,冠舄之飾,未嘗加體,其角巾素衣,斂以時服,殯葬之事,務從簡儉,亦不須茍取矯異也。」夷所著《幽求子》二十篇行於世。



 晏仕至蒼梧太守。夷兄弟三人。
 兄崧,字行高,亦有志節。惠帝時,俗多浮偽,著《任子春秋》以刺之。弟援,高平相。援子潛,右衛將軍。



 董景道,字文博,弘農人也。少而好學,千里追師,所在惟晝夜讀誦,略不與人交通。明《春秋三傳》、《京氏易》、《馬氏尚書》、《韓詩》,皆精究大義。《三禮》之義,專遵鄭氏,著《禮通論》非駁諸儒,演廣鄭旨。永平中,知天下將亂,隱於商洛山,衣木葉,食樹果,彈琴歌笑以自娛,毒蟲猛獸皆繞其傍,是以劉元海及聰屢征,皆礙而不達。至劉曜時出山,廬於渭汭。曜徵為太子少傅、散騎常侍,並固辭,竟以壽終。



 續咸,字孝宗,上黨人也。性孝謹敦重,履道貞素。好學,師事京兆杜預,專《春秋》、《鄭氏易》、教授常數十人,博覽群言,高才善文論。又修陳杜律,明達刑書。永嘉中,歷廷尉平、東安太守。劉琨承制于并州,以為從事中郎。後遂沒石勒,勒以為理曹參軍。持法平詳,當時稱其清裕,比之于公。著《遠游志》、《異物志》、《汲塚古文釋》皆十卷,行於世。年九十七,死於石季龍之世,季龍贈儀同三司。



 徐邈,東莞姑幕人也。祖澄之為州治中,屬永嘉之亂,遂
 與鄉人臧琨等率子弟並閭里士庶千餘家,南渡江,家于京口。父藻,都水使者。邈姿性端雅,勤行勵學,博涉多聞,以慎密自居。少與鄉人臧壽齊名,下帷讀書,不游城邑。及孝武帝始覽典籍,招延儒學之士,邈既東州儒素,太傅謝安舉以應選。年四十四,始補中書舍人,在西省侍帝。雖不口傳章句,然開釋文義,標明指趣,撰正五經音訓,學者宗之。遷散騎常侍,猶處西省,前後十年,每被顧問,輒有獻替,多所匡益,甚見寵待。帝宴集酣樂之後,好為手詔詩章以賜侍臣,或文詞率爾,所言穢雜,邈每應時收斂,還省刊削,皆使可觀,經帝重覽,然後出之。是
 時侍臣被詔者,或宣揚之,故時議以此多邈。及謝安薨,論者或有異同,邈固勸中書令王獻之奏加殊禮,仍崇進謝石為尚書令,玄為徐州。邈轉祠部郎,上南北郊宗廟迭毀禮,皆有證據。



 豫章太守范寧欲遣十五議曹下屬城採求風政,并使假還,訊問官長得失。邈與寧書曰:



 知足下遣十五議曹各之一縣,又吏假歸,白所聞見,誠是足下留意百姓,故廣其視聽。吾謂勸導以實不以文,十五議曹欲何所敷宣邪?庶事辭訟,足下聽斷允塞,則物理足矣。上有理務之心,則物理足矣。上有理務之心,則下之求理者至矣。日昃省覽,庶事無滯,則吏慎其負而入聽不惑,豈須邑至里詣,
 飾其游聲哉!非徒不足致益,乃是蠶漁之所資,又不可縱小吏為耳目也。豈有善人君子而干非其事,多所告白者乎!君子之心,誰毀誰譽?如有所譽,必由歷試;如有所毀,必以著明。託社之鼠,政之其害。自古以來,欲為左右耳目者,無非小人,皆先因小忠而成其大不忠,先藉小信而成其大不信,遂使君子道消,善人輿尸,前史所書,可為深鑒。



 足下選綱紀必得國士,足以攝諸曹;諸曹皆是良吏,則足以掌文案;又擇公方之人以為監司,則清濁能否,與事而明。足下但平心居宗,何取於耳目哉!昔明德馬后未嘗顧與左右言,可謂遠識,況大丈夫而
 不能免此乎!



 遷中書侍郎,專掌綸詔,帝甚親暱之。



 初,范寧與邈皆為帝所任使,共補朝廷之闕。寧才素高而措心正直,遂為王國寶所讒,出守遠郡。邈孤宦易危,而無敢排彊族,乃為自安之計。會帝頗疏會稽王道子,邈欲和協之,因從容言於帝曰:「昔淮南、齊王,漢晉成戒。會稽王雖有酣媟之累,而奉上純一,宜加弘貸,消散紛議,外為國家之計,內慰太后之心。」帝納焉。邈嘗詣東府,遇眾賓沈湎,引滿喧嘩。道子曰:「君時有暢不?」邈對曰:「邈陋巷書生,惟以節儉清修為暢耳。」道子以邈業尚道素,笑而不以為忤也。道子將用為吏部郎,邈以波競成俗,非己所
 能節制,苦辭乃止。



 時皇太子尚幼,帝甚鐘心,文武之選皆一時之後。以邈為前衛率,領本郡大中正,授太子經。帝謂邈曰:「雖未敕以師禮相待,然不以博士相遇也。」古之帝王,受經必敬,自魏晉以來,多使微人教授,號為博士,不復尊以為師,故帝有云。邈雖在東宮,猶朝夕入見,參綜朝政,修飾文詔,拾遺補闕,劬勞左右。帝嘉其謹密,方之於金霍,有託重之意,將進顯位,未及行而帝暴崩。安帝即位,拜驍騎將軍。隆安元年,遭父憂。邈先疾患,因哀毀增篤,不踰年而卒,年五十四,州里傷悼,識者悲之。



 邈蒞官簡惠,達於從政,論議精密,當時多諮稟之,觸類
 辯釋,問則有對。舊疑歲辰在卯,此宅之左則彼宅之右,何得俱忌於東。邈以為太歲之屬,自是遊神,譬如日出之時,向東皆逆,非為藏體地中也。所注《穀梁傳》,見重於時。



 邈長子豁,有父風,以孝聞,為太常博士、祕書郎。豁弟浩,散騎侍郎。鎮南將軍何無忌請為功曹,出補西陽太守,與無忌俱為盧循所害。邈弟廣,別有傳。



 孔衍,字舒元,魯國人,孔子二十二世孫也。祖文,魏大鴻臚。父毓,征南軍司。衍少好學,年十二,能通《詩》《書》。弱冠,公府辟,本州舉異行直言,皆不就。避地江東,元帝引為安
 東參軍,專掌記室。書令殷積,而衍每以稱職見知。中興初,與庚亮俱補中書郎。明帝之在東宮,領太子中庶子。於時庶事草創,衍經學深博,又練識舊典,朝儀軌制多取正焉。由是元明二帝並親愛之。王敦專權,衍私於太子曰:「殿下宜博延朝彥,搜揚才俊,詢謀時政,以廣聖聰。」敦聞而惡之,乃啟出衍為廣陵郡。時人為之寒心,而衍不形于色。雖郡鄰接西賊,猶教誘後進,不以戎務廢業。石勒嘗騎至山陽,敕其黨以衍儒雅之士,不得妄入郡境。視職期月,以太興三年卒於官,年五十三。



 衍雖不以文才著稱,而博覽過於賀循,凡所撰述,百餘萬言。



 子啟,
 盧陵太守。



 宗人夷吾,有美名,博學不及衍,涉世聲譽過之。元帝以為主簿,轉參軍,稍遷侍中,徙太子左衛率,卒,追贈太僕。



 范宣,字宣子,陳留人也。年十歲,能誦《詩》《書》。嘗以刀傷手,捧手改容。人問痛邪,答曰:「不足為痛,但受全之體而致毀傷,不可處耳。」家人以其年幼而異焉。少尚隱遁,加以好學,手不釋卷,以夜繼日,遂博綜眾書,尤善《三禮》。家至貧儉,躬耕供養。親沒,負土成墳,廬于墓側。太尉郗鑒命為主簿,詔徵太學博士、散騎郎,並不就。家于豫章,太守
 殷羨見宣茅茨不完,欲為改宅,宣固辭之。庾爰之以宣素貧,加年荒疾疫,厚餉給之,宣又不受。爰之問宣曰:「君博學通綜,何以太儒?」宣曰:「漢興,貴經術,至於石渠之論,實以儒為弊。正始以來,世尚老莊。逮晉之初,競以裸裎為高。僕誠太儒,然『丘不與易』。」宣言談未嘗及《老》《莊》。客有問人生與憂俱生,不知此語何出。宣云:「出《莊子·至樂篇》。」客曰:「君言不讀《老》《莊》,何由識此?」宣笑曰:「小時嘗一覽。」時人莫之測也。



 宣雖閑居屢空,常以講誦為業,譙國戴逵等皆聞風宗仰,自遠而至,諷誦之聲,有若齊、魯。太元中,順陽范寧為豫章太守,寧亦儒博通綜,在郡立鄉校,教
 授恒數百人。由是江州人士並好經學,化二范之風也。年五十四卒。著《禮》《易論難》皆行於世。



 子輯,歷郡守、國子博士、大將軍從事中郎。自免歸,亦以講授為事。義熙中,連徵不至。



 韋謏,字憲道,京兆人也。雅好儒學,善著述,於群言祕要之義,無不綜覽。仕於劉曜,為黃門郎。後又入石季龍,署為散騎常侍,歷守七郡,咸以清化著名。又徵為廷尉,識者擬之於、張。前後四登九列,六在尚書,二為侍中,再為太子太傅,封京兆公。好直諫,陳軍國之宜,多見允納。著《
 伏林》三千餘言,遂演為《典林》二十三篇。凡所述作及集記世事數十萬言,皆深博有才義。



 至冉閔,又署為光祿大夫。時閔拜其子胤為大單于,而以降胡一千處之麾下。謏諫曰:「今降胡數千,接之如舊,誠是招誘之恩。然胡羯本為仇敵,今之款附,茍全性命耳。或有刺客,變起須臾,敗而悔之,何所及也!古人有言,一夫不可狃,而況千乎!願誅屏降胡,去單于之號,深思聖五苞桑之誡也。」閔志在綏撫,銳於澄定,聞其言,大怒,遂誅之,並殺其子伯陽。



 謏性不嚴重,好徇己之功,論者亦以是少之。嘗謂伯陽曰:「我高我曾重光累徽,我祖我考父父子子,汝為我
 對,正值惡抵。」伯陽曰:「伯陽之不肖,誠如尊教,尊亦正值軟抵耳。」謏慚無言。時人傳之,以為嗤笑。



 范弘之,字長文,安北將軍汪之孫也。襲爵武興侯。雅正好學,以儒術該明,為太學博士。時衛將軍謝石薨,請謚,下禮官議。弘之議曰:



 石階藉門蔭,屢登崇顯,總司百揆,翼贊三臺,閑練庶事,勤勞匪懈,內外僉議,皆曰與能。當淮肥之捷,勳拯危墜,雖皇威遐震,狡寇天亡,因時立功,石亦與焉。又開建學校,以延胄子,雖盛化未洽,亦愛禮存羊。然古之賢輔,大則以道事君,侃侃終日;次則厲身
 奉國,夙夜無怠;下則愛人惜力,以濟時務。此數者,然後可以免惟塵之議,塞素餐之責矣。今石位居朝端,任則論道,唱言無忠國之謀,守職則容身而已,不可謂事君;貨黷京邑,聚斂無厭,不可謂厲身;坐擁大眾,侵食百姓,《大東》流於遠近,怨毒結於眾心,不可謂愛人;工徒勞於土木,思慮殫於機巧,紈綺盡於婢妾,財用縻於絲桐,不可謂惜力。此人臣之大害,有國之所去也。



 先王所以正風俗,理人倫者,莫尚乎節儉,故夷吾受謗乎三歸,平仲流美於約己。自頃風軌陵遲,奢僭無度,廉恥不興,利競交馳,不可不深防原本,以絕其流。漢文襲弋綈之服,諸
 侯猶侈;武帝焚雉頭之裘,靡麗不息。良由儉德雖彰,而威禁不肅;道自我建,而刑不及物。若存罰其違,亡貶其惡,則四維必張,禮義行矣。



 案謚法,因事有功曰「襄」,貪以敗官曰「墨」,宜謚曰襄墨公。



 又論殷浩宜加贈謚,不得因桓溫之黜以為國典,仍多敘溫移鼎之迹。時謝族方顯,桓宗猶盛,尚書僕射王珣,溫故吏也,素為溫所寵,三怨交集,乃出弘之為餘杭令。將行,與會稽王道子箋曰:



 下官輕微寒士,謬得廁在俎豆,實懼辱累清流,惟塵聖世。竊以人君居廟堂之上、智周四海之外者,非徒聰明內照,亦賴群言之助也。是以舜之佐堯,以啟闢為首;咎繇
 謨禹,以侃侃為先,故下無隱情之責,上收神明之功。敢緣斯義,志在輸盡。常以謝石黷累,應被清澄,殷浩忠貞,宜蒙褒顯,是以不量輕弱,先眾言之。而惡直醜正。其徒實繁,雖仰恃聖主欽明之度,俯賴明公愛物之隆,而交至之患,實有無賴。下官與石本無怨忌,生不相識,事無相干,正以國體宜明,不應稍計彊弱。與浩年時邈絕,世不相及,無復藉聞,故老語其遺事耳,於下官之身有何痛癢,而當為之犯時幹主邪!



 每觀載籍,志士仁人有發中心任直道而行者,有懷知陽愚負情曲從者,所用雖異,而並傳後世。故比干處三仁之中,箕子為名賢之首。
 後人用舍,參差不同,各信所見,率應而至,或榮名顯赫,或禍敗係踵,此皆不量時趣,以身嘗禍,雖有硜硜之稱,而非大雅之致,此亦下官所不為也。世人乃云下官正直,能犯艱難,斯談實過。下官知主上聖明,明公虛己,思求格言,必不使盡忠之臣屈於邪枉之門也。是以敢獻愚誠,布之執事,豈與昔人擬其輕重邪!亦以臣之事君,惟思盡忠而已,不應復計利鈍,事不允心則讜言悟主,義感於情則陳辭靡悔。若懷情藏意,蘊而不言,此乃古人所以得罪於明君,明君所以致法於群下者也。



 桓溫事跡,布在天朝,逆順之情,暴之四海。在三者臣子,情豈
 或異!凡厥黔首,誰獨無心!舉朝嘿嘿,未有唱言者,是以頓筆按氣,不敢多云。桓溫於亡祖,雖其意難測,求之於事,止免黜耳,非有至怨也。亡父昔為溫吏,推之情禮,義兼他人。所以每懷憤發,痛若身首者,明公有以尋之。王珣以下官議殷浩謚,不宜暴揚桓溫之惡。珣感其提拔之恩,懷其入幕之遇,託以廢黜昏闇,建立聖明,自謂此事足以明其忠貞之節。明公試復以一事觀之。昔周公居攝,道致升平,禮樂刑政皆自己出。以德言之,周公大聖,以年言之,成王幼弱,猶復遽避君位,復子明辭。漢之霍光,大勳赫然,孝宣年未二十,亦反萬機。故能君臣俱
 隆,道邁千歲。若溫忠為社稷,誠存本朝,便當仰遵二公,式是令矩,何不奉還萬機,退守籓屏?方提勒公王,匡總朝廷,豈為先帝幼弱,未可親政邪?將德桓溫,不能聽政邪?又逼脅袁宏,使作九錫,備物光赫,其文具存,朝廷畏怖,莫不景從,惟謝安、王坦之以死守之,故得稽留耳。會上天降怒,姦惡自亡,社稷危而復安,靈命墜而復構。



 晉自中興以來,號令威權多出彊臣,中宗、肅祖斂衽於王敦,先皇受屈於桓氏。今主上親覽萬機,明公光贊百揆,政出王室,人無異望,復不於今大明國典,作制百代,不審復欲待誰?先王統物,必明其典誥,貽厥孫謀,故令問
 休嘉,千歲承風。願明公遠覽殷周,近察漢魏,慮其所以危,求其所以安,如此而已。



 又與王珣書曰:



 見足下答仲堪書,深具義發之懷。夫人道所重,莫過君親,君親所係,忠孝而已。孝以揚親為主,忠以節義為先。殷侯忠貞居正,心貫人神,加與先帝隆布衣之好,著莫逆之契,契闊艱難,夷嶮以之,雖受屈姦雄,志達千載,此忠貞之徒所以義干其心不獲以已者也。既當時貞烈之徒所究見,亦後生所備聞,吾亦何敢茍避狂狡,以欺聖明。足下不推居正之大致,而懷知己之小惠,欲以幕府之小節奪名教之重義,於君臣之階既以虧矣。尊大君以殷侯協
 契忠規,同戴王室,志厲秋霜,誠貫一時,殷侯所以得宣其義聲,實尊大君協贊之力也。足下不能光大君此之直志,乃感溫小顧,懷其曲澤,公在聖世,欺罔天下,使丞相之德不及三葉,領軍之基一構而傾,此忠臣所以解心,孝子所以喪氣,父子之道固若是乎?足下言臣則非忠,語子則非孝。二者既亡,吾誰畏哉!



 吾少嘗過庭,備聞祖考之言,未嘗不發憤衝冠,情見乎辭。當爾之時,惟覆亡是懼,豈暇謀及國家。不圖今日得操筆斯事,是以上憤國朝無正義之臣,次惟祖考有沒身之恨,豈得與足下同其肝膽邪!先君往亦嘗為其吏,於時危懼,恒不自
 保,仰首聖朝,心口憤歎,豈復得計策名昔日,自同在三邪!昔子政以五世純臣,子駿以下委質王莽,先典既已正其逆順,後人亦已鑒其成敗。每讀其事,未嘗不臨文痛歎,憤愾交懷。以今況古,乃知一揆耳。



 弘之詞雖亮直,終以桓、謝之故不調,卒於餘杭令,年四十七。



 王歡,字君厚,樂陵人也。安貧樂道,專精耽學,不營產業,常丐食誦《詩》,雖家無斗儲,意怡如也。其妻患之,或焚毀其書而求改嫁,歡笑而謂之曰:「卿不聞朱買臣妻邪?」時聞者多哂之。歡守志彌固,遂為通儒。至慕容晞襲偽號,
 署為國子博士,親就受經。遷祭酒。及晞為苻堅所滅,歡死於長安。



 史臣曰:范平等學府儒宗,譽隆望重,或質疑是屬,或師範攸歸,雖為未及古人,故亦一時之俊。若仲寧之清貞守道,抗志柴門;行齊之居室屢空,棲心陋巷;文博之漱流枕石,鏟跡銷聲;宣子之樂道安貧,弘風闡教:斯並通儒之高尚者也。而邈協和主相,刊削繁辭,可謂將順其美,匡救其惡。舒元入參機務,明主賞其博聞;出蒞邊隅,獷狄欽其明德。弘之抗言立論,不避朝權,貶石抵溫,斯為當矣,遂乃厄三怨,以至陵遲,悲夫!



 贊曰:鬱鬱周文,洋洋漢典。炙輠流譽,解頤飛辯。雅誥弗淪,微言復顯。爰及晉代,斯風逾闡。



\end{pinyinscope}