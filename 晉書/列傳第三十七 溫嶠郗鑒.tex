\article{列傳第三十七 溫嶠郗鑒}

\begin{pinyinscope}
溫嶠郗鑒
 \gezhu{
  子愔愔子超愔弟曇鑒叔父隆}



 溫嶠,字太真,司徒羨弟之子也。父憺,河東太守。嶠性聰敏,有識量,博學能屬文,少以孝悌稱於邦族。風儀秀整,美於談論,見者皆愛悅之。年十七,州郡辟召,皆不就。司隸命為都官從事。散騎常侍庾敳有重名,而頗聚斂,嶠舉奏之,京都振肅。後舉秀才、灼然。司徒辟東閣祭酒,補上黨潞令。



 平北大將軍劉琨妻,嶠之從母也。琨深
 禮之,請為參軍。琨遷大將軍,嶠為從事中郎、上黨太守,加建威將軍、督護前鋒軍事。將兵討石勒,屢有戰功。琨遷司空,以嶠為右司馬。于時并土荒殘,寇盜群起,石勒、劉聰跨帶疆埸,嶠為之謀主,琨所憑恃焉。



 屬二都傾覆,社稷絕祀,元帝初鎮江左,琨誠繫王室,謂嶠曰:「昔班彪識劉氏之復興,馬援知漢光之可輔。今晉祚雖衰,天命未改,吾欲立功河朔,使卿延譽江南,子其行乎?」對曰:「嶠雖無管張之才,而明公有桓文之志,欲建匡合之功,豈敢辭命。」乃以為左長史,檄告華夷,奉表勸進。嶠既至,引見,具陳琨忠誠,志在效節,因說社稷無主,天人係望,辭
 旨慷慨。舉朝屬目,帝器而喜焉。王導、周顗、謝鯤、庾亮、桓彞等並與親善。于時江左草創,綱維未舉,嶠殊以為憂。及見王導共談,歡然曰;「江左自有管夷吾,吾復何慮!」屢求反命,不許。會琨為段匹磾所害,嶠表琨忠誠,雖勳業不遂,然家破身亡,宜在褒崇,以慰海內之望。帝然之。



 除散騎侍郎。初,嶠欲將命,其母崔氏固止之,嶠絕裾而去。其後母亡,嶠阻亂不獲歸葬,由是固讓不拜,苦請北歸。詔三司、八坐議其事,皆曰:「昔伍員志復私仇,先假諸侯之力,東奔闔閭,位為上將,然後鞭荊王之尸。若嶠以母未葬沒在胡虜者,乃應竭其智謀,仰憑皇靈,使逆寇冰
 消,反哀墓次,豈可稍以乖嫌,廢其遠圖哉!」嶠不得已,乃受命。



 後歷驃騎王導長史,遷太子中庶子。及在東宮,深見寵遇,太子與為布衣之交。數陳規諷,又獻《侍臣箴》,甚有弘益。時太子起西池樓觀,頗為勞費,嶠上疏以為朝廷草創,巨寇未滅,宜應儉以率下,務農重兵,太子納焉。王敦舉兵內向,六軍敗績,太子將自出戰,嶠執鞚諫曰:「臣聞善戰者不怒,善勝者不武,如何萬乘儲副而以身輕天下!」太子乃止。



 明帝即位,拜侍中,機密大謀皆所參綜,詔命文翰亦悉豫焉。俄轉中書令。嶠有棟梁之任,帝親而倚之,甚為王敦所忌,因請為左司馬。敦阻兵不朝,
 多行陵縱,嶠諫敦曰:「昔周公之相成王,勞謙吐握,豈好勤而惡逸哉!誠由處大任者不可不爾。而公自還輦轂,入輔朝政,闕拜覲之禮,簡人臣之儀,不達聖心者莫不於邑。昔帝舜服事唐堯,伯禹竭身虞庭,文王雖盛,臣節不愆。故有庇人之大德,必有事君之小心,俾方烈奮乎百世,休風流乎萬祀。至聖遺軌,所不宜忽。願思舜、禹、文王服事之勤,惟公旦吐握之事,則天下幸甚。」敦不納。嶠知其終不悟,於是謬為設敬,綜其府事,乾說密謀,以附其欲。深結錢鳳,為之聲譽,每曰:「錢世儀精神滿腹。」嶠素有知人之稱,鳳聞而悅之,深結好於嶠。會丹陽尹缺,嶠
 說敦曰:「京尹輦轂喉舌,宜得文武兼能,公宜自選其才。若朝廷用人,或不盡理。」敦然之,問嶠誰可作者。嶠曰:「愚謂錢鳳可用。」鳳亦推嶠,嶠偽辭之。敦不從,表補丹陽尹。嶠猶懼錢鳳為之姦謀,因敦餞別,嶠起行酒,至鳳前,鳳未及飲,嶠因偽醉,以手版擊鳳幘墜,作色曰:「錢鳳何人,溫太真行酒而敢不飲!」敦以為醉,兩釋之。臨去言別,涕泗橫流,出閣復入,如是再三,然後即路。及發後,鳳入說敦曰:「嶠於朝廷甚密,而與庾亮深交,未必可信。」敦曰:「太真昨醉,小加聲色,豈得以此便相讒貳。」由是鳳謀不行,而嶠得還都,乃具奏敦之逆謀,請先為之備。



 及敦構逆,
 加嶠中壘將軍、持節、都督東安北部諸軍事。敦與王導書曰:「太真別來幾日,作如此事!」表誅姦臣,以嶠為首。募生得嶠者,當自拔其舌。及王含、錢鳳奄至都下,嶠燒朱雀桁以挫其鋒,帝怒之,嶠曰:「今宿衛寡弱,徵兵未至,若賊豕突,危及社稷,陛下何惜一橋。」賊果不得渡。嶠自率眾與賊夾水戰,擊王含,敗之,復督劉遐追錢鳳於江寧。事平,封建寧縣開國公,賜絹五千四百匹,進號前將軍。



 時制王敦綱紀除名,參佐禁固,嶠上疏曰:「王敦剛愎不仁,忍行殺戮,親任小人,疏遠君子,朝廷所不能抑,骨肉所不能間。處其朝者恒懼危亡,故人士結舌,道路以目,
 誠賢人君子道窮數盡,遵養時晦之辰也。且敦為大逆之日,拘錄人士,自免無路,原其私心,豈遑晏處,如陸玩、羊曼、劉胤、蔡謨、郭璞常與臣言,備知之矣。必其凶悖,自可罪人斯得;如其枉入姦黨,宜施之以寬。加以玩等之誠,聞於聖聽,當受同賊之責,實負其心。陛下仁聖含弘,思求允中;臣階緣博納,乾非其事,誠在愛才,不忘忠益。」帝從之。



 是時天下凋弊,國用不足,詔公卿以下詣都坐論時政之所先,嶠因奏軍國要務。其一曰:「祖約退舍壽陽,有將來之難。今二方守禦,為功尚易。淮泗都督,宜竭力以資之。選名重之士,配徵兵五千人,又擇一偏將,將
 二千兵,以益壽陽,可以保固徐豫,援助司土。」其二曰:「一夫不耕,必有受其飢者。今不耕之夫,動有萬計。春廢勸課之制,冬峻出租之令,下未見施,惟賦是聞。賦不可以己,當思令百姓有以殷實。司徒置田曹掾,州一人,勸課農桑,察吏能否,今宜依舊置之。必得清恪奉公,足以宣示惠化者,則所益實弘矣。」其三曰:「諸外州郡將兵者及都督府非臨敵之軍,且田且守。又先朝使五校出田,今四軍五校有兵者,及護軍所統外軍,可分遣二軍出,并屯要處。緣江上下,皆有良田,開荒須一年之後即易。且軍人累重者在外,有樵採蔬食之人,於事為便。」其四曰:「
 建官以理世,不以私人也。如此則官寡而材精。周制六卿蒞事,春秋之時,入作卿輔,出將三軍。後代建官漸多,誠由事有煩簡耳。然今江南六州之土,尚又荒殘,方之平日,數十分之一耳。三省軍校無兵者,九府寺署可有并相領者,可有省半者,粗計閑劇,隨事減之。荒殘之縣,或同在一城,可并合之。如此選既可精,祿俸可優,令足代耕,然後可責以清公耳。」其五曰:「古者親耕藉田以供粢盛,舊置藉田、廩犧之官。今臨時市求,既上黷至敬,下費生靈,非所以虔奉宗廟蒸嘗之旨。宜如舊制,立此二官。」其六曰:「使命愈遠,益宜得才,宣揚王化,延譽四方。人
 情不樂,遂取卑品之人,虧辱國命,生長患害。故宜重其選,不可減二千石見居二品者,」其七曰:「罪不相及,古之制也。近者大逆,誠由凶戾。凶戾之甚,一時權用。今遂施行,非聖朝之令典,宜如先朝除三族之制。」議奏,多納之。



 帝疾篤,嶠與王導、郗鑒、庚亮、陸曄、卞壼等同受顧命。時歷陽太守蘇峻藏匿亡命,朝廷疑之。征西將軍陶侃有威名於荊楚,又以西夏為虞,故使嶠為上流形援。咸和初,代應詹為江州刺史、持節、都督、平南將軍,鎮武昌,甚有惠政,甄異行能,親祭徐孺子之墓。又陳豫章十郡之要,宜以刺史居之。尋陽濱江,都督應鎮其地。今以州貼
 府,進退不便。且古鎮將多不領州,皆以文武形勢不同故也。宜選單車刺史別撫豫章,專理黎庶。」詔不許。在鎮見王敦畫像,曰:「敦大逆,宜加斲棺之戮,受崔杼之刑。古人闔棺而定謚,《春秋》大居正,崇王父之命,未有受戮於天子而圖形於群下。」命削去之。



 嶠聞蘇峻之徵也,慮必有變,求還朝以備不虞,不聽。未幾而蘇峻果反。嶠屯尋陽,遣督護王愆期、西陽太守鄧嶽、鄱陽內史紀瞻等率舟師赴難。及京師傾覆,嶠聞之號慟。人有候之者,悲哭相對。俄而庾亮來奔,宣太后詔,進嶠驃騎將軍、開府儀同三司。嶠曰:「今日之急,殄寇為先,未效勳庸而逆受榮
 寵,非所聞也,何以示天下乎!」固辭不受。時亮雖奔敗,嶠每推崇之,分兵給亮。遣王愆期等要陶侃同赴國難,侃恨不受顧命,不許。嶠初從之,後用其部將毛寶說,復固請侃行,語在寶傳。初,嶠與庾亮相推為盟主,嶠從弟充言於嶠曰:「征西位重兵彊,宜共推之。」嶠於是遣王愆期奉侃為盟主。侃許之,遣督護襲登率兵詣嶠。嶠於是列上尚書,陳峻罪狀,有眾七千,灑泣登舟,移告四方征鎮曰:



 賊臣祖約、蘇峻同惡相濟,用生邪心。天奪其魄,死期將至。譴負天地,自絕人倫。寇不可縱,宜增軍討撲,輒屯次湓口。即日護軍庾亮至,宣太后詔,寇逼宮城,王旅撓
 敗,出告籓臣,謀寧社稷。後將軍郭默、冠軍將軍趙胤、奮宗將軍襲保與嶠督護王愆期、西陽太守鄧嶽、鄱陽內史紀瞻,率其所領,相尋而至。逆賊肆凶,陵蹈宗廟,火延宮掖,矢流太極,二御幽逼,宰相困迫,殘虐朝士,劫辱子女。承問悲惶,精魂飛散。嶠闇弱不武,不能徇難,哀恨自咎,五情摧隕,慚負先帝託寄之重,義在畢力,死而後已。今躬率所統,為士卒先,催進諸軍,一時電擊。西陽太守鄧嶽、尋陽太守褚誕等連旗相繼,宣城內史桓彞已勒所屬屯濱江之要,江夏相周撫乃心求徵,軍已向路。



 昔包胥楚國之微臣,重趼致誠,義感諸侯。藺相如趙邦之
 陪隸,恥君之辱,按劍秦庭。皇漢之季,董卓作亂,劫遷獻帝,虐害忠良,蘭東州郡相率同盟。廣陵功曹臧洪,郡之小吏耳,登壇臿血,涕淚橫流,慷慨之節,實厲群后。況今居台鼎,據方州,列名邦,受國恩者哉!不期而會,不謀而同,不亦宜乎!



 二賊合眾,不盈五千,且外畏胡寇,城內饑乏,後將軍郭默即於戰陣俘殺賊千人。賊今雖殘破都邑,其宿衛兵人即時出散,不為賊用。且祖約情性褊阨,忌剋不仁,蘇峻小子,惟利是視,殘酷驕猜,權相假合。江表興義,以抗其前,彊胡外寇,以躡其後,運漕隔絕,資食空懸,內乏外孤,勢何得久!



 群公征鎮,職在禦侮。征西陶
 公,國之耆德,忠肅義正,勳庸弘著。諸方鎮州郡咸齊斷金,同稟規略,以雪國恥,茍利社稷,死生以之。嶠雖怯劣,忝據一方,賴忠賢之規,文武之助,君子竭誠,小人盡力,高操之士被褐而從戎,負薪之徒匍匐而赴命,率其私僕,致其私杖,人士之誠,竹帛不能載也。豈嶠無德而致之哉?士稟義風,人感皇澤。且護軍庾公,帝之元舅,德望隆重,率郭後軍、趙、龔三將,與嶠戮力,得有資憑,且悲且慶,若朝廷之不泯也。其各明率所統,無後事機。賞募之信,明如日月。有能斬約峻者,封五等侯,賞布萬匹。夫忠為令德,為仁由己,萬里一契,義不在言也。



 時陶侃雖許
 自下而未發,復追其督護龔登。嶠重與侃書曰:



 僕謂軍有進而無退,宜增而不可減。近已移檄遠近,言於盟府,剋後月半大舉。南康、建安、晉安三郡軍並在路次,同赴此會,惟須仁公所統至,便齊進耳。仁公今召軍還,疑惑遠近,成敗之由,將在於此。



 僕才輕任重,實憑仁公篤愛,遠稟成規。至於首啟戎行,不敢有辭,僕與仁公當如常山之蛇,首尾相衛,又脣齒之喻也。恐惑者不達高旨,將謂仁公緩於討賊,此聲難追。僕與仁公並受方嶽之任,安危休戚,理既同之。且自頃之顧,綢繆往來,情深義重,著於人士之口,一旦有急,亦望仁公悉眾見救,況社稷
 之難!



 惟僕偏當一州,州之文武莫不翹企。假令此州不守,約峻樹置官長於此,荊楚西逼彊胡,東接逆賊,因之以饑饉,將來之危乃當甚於此州之今日也。以大義言之,則社稷顛覆,主辱臣死,公進當為大晉之忠臣,參桓文之義,開國承家,銘之天府;退當以慈父雪愛子之痛。



 約峻凶逆無道,囚制人士,裸其五形。近日來者,不可忍見。骨肉生離,痛感天地,人心齊一,咸皆切齒。今之進討,若以石投卵耳!今出軍既緩,復召兵還,人心乖離,是為敗於幾成也。願深察所陳,以副三軍之望。



 峻時殺侃子瞻,由是侃激勵,遂率所統與嶠、亮同赴京師,戎卒六萬,
 旌旗七百餘里,鉦鼓之聲震於百里,直指石頭,次于蔡洲。侃屯查浦,嶠屯沙門浦。時祖約據歷陽,與峻為首尾,見嶠等軍盛,謂其黨曰:「吾本知嶠能為四公子之事,今果然矣。」



 峻聞嶠將至,逼大駕幸石頭。時峻軍多馬,南軍杖舟楫,不敢輕與交鋒。用將軍李根計,據白石築壘以自固,使庾亮守之。賊步騎萬餘來攻,不下而退,追斬二百餘級。嶠又於四望磯築壘以逼賊,曰:「賊必爭之,設伏以逸待勞,是制賊之一奇也。」是時義軍屢戰失利,嶠軍食盡,陶侃怒曰:「使君前云不憂無將士,惟得老僕為主耳。今數戰皆北,良將安在?荊州接胡蜀二虜,倉廩當備
 不虞,若復無食,僕便欲西歸,更思良算。但今歲計,殄賊不為晚也。」嶠曰:「不然。自古成監,師克在和。光武之濟昆陽,曹公之拔官渡,以寡敵眾,杖義故也。峻、約小豎,為海內所患,今日之舉,決在一戰。峻勇而無謀,藉驕勝之勢,自謂無前,今挑之戰,可一鼓則擒也。奈何捨垂立之功,設進退之計!且天子幽逼,社稷危殆,四海臣子,肝腦塗地,嶠等與公並受國恩,是臻命之日,事若克濟,則臣主同祚,如其不捷,身雖灰滅,不足以謝責於先帝。今之事勢,義無旋踵,騎猛獸,安可中下哉!公若違眾獨反,人心必沮。沮眾敗事,義旗將迴指於公矣。」侃無以對,遂留不
 去。



 嶠於是創建行廟,廣設壇場,告皇天后土祖宗之靈,親讀祝文,聲氣激揚,流涕覆面,三軍莫能仰視。其日侃督水軍向石頭,亮、嶠等率精勇一萬從白石以挑戰。時峻勞其將士,因醉,突陣馬躓,為侃將所斬,峻弟逸及子碩嬰城自固。嶠乃立行臺,布告天下,凡故吏二千石、臺郎御史以下,皆令赴臺。於是至者雲集。司徒王導因奏嶠、侃錄尚書,遣間使宣旨,並讓不受。賊將匡術以臺城來降,為逸所擊,求救於嶠。江州別駕羅洞曰:「今水暴長,救之不便,不如攻榻杭。榻杭軍若敗,術圍自解。」嶠從之,遂破賊石頭軍。奮威長史滕含抱天子奔于嶠船。時陶
 侃雖為盟主,而處分規略一出於嶠,及賊滅,拜驃騎將軍、開府儀同三司,加散騎常侍,封始安郡公,邑三千戶。



 初,峻黨路永、匡術、賈寧中途悉以眾歸順,王導將褒顯之,嶠曰:「術輩首亂,罪莫大焉。晚雖改悟,未足以補前失。全其首領,為幸已過,何可復寵授哉!」導無以奪。



 朝議將留輔政,嶠以導先帝所任,固辭還籓。復以京邑荒殘,資用不給,嶠借資蓄,具器用,而後旋于武昌,至牛渚磯,水深不可測,世云其下多怪物,嶠遂毀犀角而照之。須臾,見水族覆火,奇形異狀,或乘馬車著赤衣者。嶠其夜夢人謂己曰:「與君幽明道別,何意相照也?」意甚惡之。嶠先
 有齒疾,至是拔之,因中風,至鎮未旬而卒,時年四十二。江州士庶聞之,莫不相顧而泣。帝下冊書曰:「朕以眇身,纂承洪緒,不能光闡大道,化洽時雍,至乃狂狡滔天,社稷危逼。惟公明鑒特達,識心經遠,懼皇綱之不維,忿凶寇之縱暴,唱率群后,五州響應,首啟戎行,元惡授馘。王室危而復安,三光幽而復明,功格宇宙,勳著八表。方賴大獻以拯區夏,天不憖遺,早世薨殂,朕用痛悼于厥心。夫褒德銘動,先王之明典,今追贈公侍中、大將軍、持節、都督、刺史,公如故,賜錢百萬,布千匹,謚曰忠武,祠以太牢。」



 初葬于豫章,後朝廷追嶠勳德,將為造大墓於元明
 二帝陵之北,陶侃上表曰:「故大將軍嶠忠誠著於聖世,勳義感于人神,非臣筆墨所能稱陳。臨卒之際,與臣書別,臣藏之篋笥,時時省視,每一思述,未嘗不中夜撫膺,臨飯酸噎。『人之云亡』,嶠實當之。謹寫嶠書上呈,伏惟陛下既垂御省,傷其情旨,死不忘忠,身沒黃泉,追恨國恥,將臣戮力,救濟艱難,使亡而有知,抱恨結草,豈樂今日勞費之事。願陛下慈恩,停其移葬,使嶠棺柩無風波之危,魂靈安於后土。」詔從之。其後嶠後妻何氏卒,子放之便載喪還都。詔葬建平陵北,並贈嶠前妻王氏及何氏始安夫人印綬。



 放之嗣爵,少歷清官,累至給事黃門侍
 郎。以貧,求為交州,朝廷許之。王述與會稽王箋曰:「放之溫嶠之子,宜見優異,而投之嶺外,竊用愕然。願遠存周禮,近參人情,則望實惟允。」時竟不納。放之既至南海,甚有威惠。將征林邑,交阯太守杜寶、別駕阮朗並不從,放之以其沮眾,誅之,勒兵進,遂破林邑而還。卒于官。



 弟式之,新建縣侯,位至散騎常侍。



 郗鑒,字道徽,高平金鄉人,漢御史大夫慮之玄孫也。少孤貧,博覽經籍,躬耕隴畝,吟詠不倦。以儒雅著名,不應
 州命。趙王倫辟為掾,知倫有不臣之迹,稱疾去職。及倫篡,其黨皆至大官,而鑒閉門自守,不染逆節。惠帝反正,參司空軍事,累遷太子中舍人、中書侍郎。東海王越辟為主簿,舉賢良,不行。征東大將軍茍晞檄為從事中郎。晞與越方以力爭,鑒不應其召。從兄旭,晞之別駕,恐禍及己,勸之赴召,鑒終不迴,晞亦不之逼也。及京師不守,寇難鋒起,鑒遂陷於陳午賊中。邑人張實先求交於鑒,鑒不許。至是,實於午營來省鑒疾,既而卿鑒。鑒謂實曰:「相與邦壤,義不及通,何可怙亂至此邪!」實大慚而退。午以鑒有名於世,將逼為主,鑒逃而獲免。午尋潰散,鑒得
 歸鄉里。于時所在饑荒,州中之士素有感其恩義者,相與資贍。鑒復分所得,以恤宗族及鄉曲孤老,賴而全濟者甚多,咸相謂曰:「今天子播越,中原無伯,當歸依仁德,可以後亡。」遂共推鑒為主,舉千餘家俱避難於魯之嶧山。



 元帝初鎮江左,承制假鑒龍驤將軍、兗州刺史,鎮鄒山。時荀籓用李述,劉琨用兄子演,並為兗州,各屯一郡,以力相傾,闔州編戶,莫知所適。又徐龕、石勒左右交侵,日尋干戈,外無救援,百姓饑饉,或掘野鼠蟄燕而食之,終無叛者。三年間,眾至數萬。帝就加輔國將軍、都督兗州諸軍事。



 永昌初,徵拜領軍將軍,既至,轉尚書,以疾不
 拜。時明帝初即位,王敦專制,內外危逼,謀杖鑒為外援,由是拜安西將軍、兗州刺史、都督揚州江西諸軍、假節,鎮合肥。敦忌之,表為尚書令,徵還。道經姑孰,與敦相見,敦謂曰:「樂彥輔短才耳。後生流宕,言違名檢,考之以實,豈勝滿武秋邪?」鑒曰:「擬人必于其倫。彥輔道韻平淡,體識沖粹,處傾危之朝,不可得而親疏。及愍懷太子之廢,可謂柔而有正。武秋失節之士,何可同日而言!」敦曰:「愍懷廢徙之際,交有危機之急,人何能以死守之乎!以此相方,其不減明矣。」鑒曰:「丈夫既潔身北面,義同在三,豈可偷生屈節,靦顏天壤邪!茍道數終極,固當存亡以之
 耳。」敦素懷無君之心,聞鑒言,大忿之,遂不復相見,拘留不遣。敦之黨與譖毀日至,鑒舉止自若,初無懼心。敦謂錢鳳曰:「郗道徽儒雅之士,名位既重,何得害之!」乃放還臺。鑒遂與帝謀滅敦。



 既而錢鳳攻逼京都,假鑒節,加衛將軍、都督從駕諸軍事。鑒以無益事實,固辭不受軍號。時議者以王含、錢鳳眾力百倍,苑城小而不固,宜及軍勢未成,大駕自出距戰。鑒曰:「群逆縱逸,其勢不可當,可以算屈,難以力競。且含等號令不一,抄盜相尋,百姓懲往年之暴,皆人自為守。乘逆順之勢,何往不剋!且賊無經略遠圖,惟恃豕突一戰,曠日持久,必啟義士之心,令
 謀猷得展。今以此弱力敵彼彊寇,決勝負於一朝,定成敗於呼吸,雖有申胥之徒,義存投袂,何補於既往哉!」帝從之。鑒以尚書令領諸屯營。



 及鳳等平,溫嶠上議,請宥敦佐吏,鑒以為先王崇君臣之教,故貴伏死之節;昏亡之主,故開待放之門。王敦佐吏雖多逼迫,然居逆亂之朝,無出關之操,準之前訓,宜加義責。又奏錢鳳母年八十,宜蒙全宥。乃從之。封高平侯,賜絹四千八百匹。帝以其有器望,萬機動靜輒問之,乃詔鑒特草上表疏,以從簡易。王導議欲贈周札官,鑒以為不合,語在札傳。導不從。鑒於是駁之曰:「敦之逆謀,履霜日久,緣札開門,令王
 師不振。若敦前者之舉,義同桓文,則先帝可為幽厲邪?」朝臣雖無以難,而不能從。俄而遷車騎將軍、都督徐兗青三州軍事、兗州刺史、假節,鎮廣陵。尋而帝崩,鑒與王導、卞壼、溫嶠、庾亮、陸曄等並受遺詔,輔少主,進位車騎大將軍、開府儀同三司,加散騎常侍。



 咸和初,領徐州刺史。及祖約、蘇峻反,鑒聞難,便欲率所領東赴。詔以北寇不許。於是遣司馬劉矩領三千人宿衛京都。尋而王師敗績,矩遂退還。中書令庾亮宣太后口詔,進鑒為司空。鑒去賊密邇,城孤糧絕,人情業業,莫有固志,奉詔流涕,設壇場,刑白馬,大誓三軍曰:「賊臣祖約、蘇峻不恭天命,
 不畏王誅,凶戾肆逆,干國之紀,陵汨五常,侮弄神器,遂制脅幽主,拔本塞原,殘害忠良,禍虐黎庶,使天地神祇靡所依歸。是以率土怨酷,兆庶泣血,咸願奉辭罰罪,以除元惡。昔戎狄泯周,齊桓糾盟;董卓陵漢,群后致討。義存君親,古今一也。今主上幽危,百姓倒懸,忠臣正士志存報國。凡我同盟,既盟之後,戮力一心,以救社稷。若二寇不梟,義無偷安。有渝此盟,明神殘之!」鑒登壇慷慨,三軍爭為用命。乃遣將軍夏侯長等間行,謂平南將軍溫嶠曰:「今賊謀欲挾天子東入會稽,宜先立營壘,屯據要害,既防其越逸,又斷賊糧運,然後靜鎮京口,清壁以待
 賊。賊攻城不拔,野無所掠,東道既斷,糧運自絕,不過百日,必自潰矣。」嶠深以為然。



 及陶侃為盟主,進鑒都督揚州八郡軍事。時撫軍將軍王舒、輔軍將軍虞潭皆受鑒節度,率眾渡江,與侃會于茄子浦。鑒築白石壘而據之。會舒、潭戰不利,鑒與後將軍郭默還丹徒,立大業、曲阿、庱亭三壘以距賊。而賊將張健來攻大業,城中乏水,郭默窘迫,遂突圍而出,三軍失色。參軍曹納以為大業京口之扞,一旦不守,賊方軌而前,勸鑒退還廣陵以俟後舉。鑒乃大會僚佐,責納曰:「吾蒙先帝厚顧,荷託付之重,正復捐軀九泉不足以報。今彊寇在郊,眾心危迫,君腹
 心之佐,而生長異端,當何以率先義眾,鎮一三軍邪!」將斬之,久而乃釋。會峻死,大業圍解。及蘇逸等走吳興,鑒遣參軍李閎追斬之,降男女萬餘口。拜司空,加侍中,解八郡都督,更封南昌縣公,以先爵封其子曇。



 時賊帥劉徵聚眾數千,浮海抄東南諸縣。鑒遂城京口,加都督揚州之晉陵吳郡諸軍事,率眾討平之。進位太尉。後以寢疾,上疏遜位曰:「臣疾彌留,遂至沈篤,自忖氣力,差理難冀。有生有死,自然之分。但忝位過才,會無以報,上慚先帝,下愧日月。伏枕哀歎,抱恨黃泉。臣今虛乏,救命朝夕,輒以府事付長史劉遐,乞骸骨歸丘園。惟願陛下崇山
 海之量,弘濟大猷,任賢使能,事從簡易,使康哉之歌復興於今,則臣雖死,猶生之日耳。臣所統錯雜,率多北人,或逼遷徙,或是新附,百姓懷土,皆有歸本之心。臣宣國恩,示以好惡,處興田宅,漸得少安。聞臣疾篤,眾情駭動,若當北渡,必啟寇心。太常臣謨,平簡貞正,素望所歸,謂可以為都督、徐州刺史。臣亡兄息晉陵內史邁,謙愛養士,甚為流亡所宗,又是臣門戶子弟,堪任兗州刺史。公家之事,知無不為,是以敢希祁奚之舉。」疏奏,以蔡謨為鑒軍司。鑒尋薨,時年七十一。帝朝晡哭于朝堂,遣御史持節護喪事,贈一依溫嶠故事。冊曰:「惟公道德沖邃,體
 識弘遠,忠亮雅正,行為世表,歷位內外,勳庸彌著。乃者約峻狂狡,毒流朝廷,社稷之危,賴公以寧。功侔古烈,勳邁桓文。方倚大猷,籓翼時難,昊天不弔,奄忽薨殂,朕用震悼于厥心。夫爵以顯德,謚以表行,所以崇明軌跡,丕揚徽劭。今贈太宰,謚曰文成,祠以太牢。魂而有靈,嘉茲寵榮。」



 初,鑒值永嘉喪亂,在鄉里甚窮餒,鄉人以鑒名德,傳共飴之。時兄子邁、外甥周翼並小,常攜之就食。鄉人曰:「各自饑困,以君賢,欲共相濟耳,恐不能兼有所存。」鑒於是獨往,食訖,以飯著兩頰邊,還吐與二兒,後並得存,同過江。邁位至護軍,翼為剡縣令。鑒之薨也,翼追撫育
 之恩,解職而歸,席苫心喪三年。二子:愔、曇。



 愔字方回。少不交競,弱冠,除散騎侍郎,不拜。性至孝,居父母憂,殆將滅性。服闋,襲爵南昌公,徵拜中書侍郎。驃騎何充輔政,征北將軍褚裒鎮京口,皆以愔為長史。再遷黃門侍郎。時吳郡守闋,欲以愔為太守。愔自以資望少,不宜超蒞大郡,朝議嘉之。轉為臨海太守。會弟曇卒,益無處世意,在郡優游,頗稱簡默,與姊夫王羲之、高士許詢並有邁世之風,俱棲心絕穀,修黃老之術。後以疾去職,乃築宅章安,有終焉之志。十許年間,人事頓絕。



 簡文帝輔政,與尚書僕射江[A170]等薦愔,以為執德存正,識
 懷沈敏,而辭職遺榮,有不拔之操,成務須才,豈得遂其獨善,宜見徵引,以參政術。於是徵為光祿大夫,加散騎常侍。既到,更除太常,固讓不拜。深抱沖退,樂補遠郡,從之,出為輔國將軍、會稽內史。大司馬桓溫以愔與徐兗有故義,乃遷愔都督徐兗青幽揚州之晉陵諸軍事、領徐兗二州刺史、假節。雖居籓鎮,非其好也。



 俄屬桓溫北伐,愔請督所部出河上,用其子超計,以己非將帥才,不堪軍旅,又固辭解職,勸溫並領己所統。轉冠軍將軍、會稽內史。



 及帝踐阼,就加鎮軍、都督浙江東五郡軍事。久之,以年老乞骸骨,因居會稽。徵拜司空,詔書優美,敦獎
 殷勤,固辭不起。太元九年卒,時年七十二。追贈侍中、司空,謚曰文穆。三子。超、融、沖。超最知名。



 超字景興,一字嘉賓。少卓犖不羈,有曠世之度,交游士林,每存勝拔,善談論,義理精微。愔事天師道,而超奉佛。愔又好聚斂,積錢數千萬,嘗開庫,任超所取。超性好施,一日中散與親故都盡。其任心獨詣,皆此類也。



 桓溫辟為征西大將軍掾。溫遷大司馬,又轉為參軍。溫英氣高邁,罕有所推,與超言,常謂不能測,遂傾意禮待。超亦深自結納。時王珣為溫主簿,亦為溫所重。府中語曰:「髯參軍,短主簿,能令公喜,能令公怒。」超髯,珣短故也。尋除散
 騎侍郎。時愔在北府,徐州人多勁悍,溫恒云「京口酒可飲,兵可用」,深不欲愔居之。而愔暗於事機,遣箋詣溫,欲共獎王室,修復園陵。超取視,寸寸毀裂,乃更作箋,自陳老病,甚不堪人間,乞閑地自養。溫得箋大喜,即轉愔為會稽太守。溫懷不軌,欲立霸王之基,超為之謀。謝安與王坦之嘗詣溫論事,溫令超帳中臥聽之,風動帳開,安笑曰:「郗生可謂入幕之賓矣。」



 太和中,溫將伐慕容氏於臨漳,超諫以道遠,汴水又淺,運道不通。溫不從,遂引軍自濟入河,超又進策於溫曰:「清水入河,無通運理。若寇不戰,運道又難,因資無所,實為深慮也。今盛夏,悉力徑
 造鄴城,彼伏公威略,必望陣而走,退還幽朔矣。若能決戰,呼吸可定。設欲城鄴,難為功力。百姓布野,盡為官有。易水以南,必交臂請命。但恐此計輕決,公必務其持重耳。若此計不從,便當頓兵河濟,控引糧運,令資儲充備,足及來夏,雖如賒遲,終亦濟剋。若舍此二策而連軍西進,進不速決,退必愆乏,賊因此勢,日月相引,僶俛秋冬,船道澀滯,且北土早寒,三軍裘褐者少,恐不可以涉冬。此大限閡,非惟無食而已。」溫不從,果有枋頭之敗,溫深慚之。尋而有壽陽之捷,問超曰:「此足以雪枋頭之恥乎?」超曰:「未厭有識之情也。」既而超就溫宿,中夜謂溫曰:「明
 公都有慮不?」溫曰:「卿欲有所言邪?」超曰:「明公既居重任,天下之責將歸於公矣。若不能行廢立大事、為伊霍之舉者,不足鎮壓四海,震服宇內,豈可不深思哉!」溫既素有此計,深納其言,遂定廢立,超始謀也。



 遷中書侍郎。謝安嘗與王文度共詣超,日旰未得前,文度便欲去,安曰:「不能為性命忍俄頃邪!」其權重當時如此。轉司徒左長史,母喪去職。常謂其父名公之子,位遇應在謝安右,而安入掌機權,愔優游而已,恒懷憤憤,發言慷慨,由是與謝氏不穆。安亦深恨之。服闋,除散騎常侍,不起。以為臨海太守,加宣威將軍,不拜。年四十二,先愔卒。



 初,超雖實
 黨桓氏,以愔忠於王室,不令知之。將亡,出一箱書,付門生曰:「本欲焚之,恐公年尊,必以傷愍為弊。我亡後,若大損眠食,可呈此箱。不爾,便燒之。」愔後果哀悼成疾,門生依旨呈之,則悉與溫往反密計。愔於是大怒曰:「小子死恨晚矣!」更不復哭。凡超所交友,皆一時秀美,雖寒門後進,亦拔而友之。及死之日,貴賤操筆而為誄者四十餘人,其為眾所宗貴如此。王獻之兄弟,自超未亡,見愔,常躡履問訊,甚修舅甥之禮。及超死,見愔慢怠,屐而候之,命席便遷延辭避。愔每慨然曰:「使嘉賓不死,鼠子敢爾邪!」性好聞人棲遁,有能辭榮拂衣者,超為之起屋宇,作
 器服,畜僕豎,費百金而不吝。又沙門支遁以清談著名於時,風流勝貴,莫不崇敬,以為造微之功,足參諸正始。而遁常重超,以為一時之俊,甚相知賞。超無子,從弟儉之以子僧施嗣。



 僧施字惠脫,襲爵南昌公。弱冠,與王綏、桓胤齊名,累居清顯,領宣城內史,入補丹陽尹。劉毅鎮江陵,請為南蠻校尉、假節。與毅俱誅,國除。



 曇字重熙,少賜爵東安縣開國伯。司徒王導辟祕書郎。朝論以曇名臣之子,每逼以憲制,年三十,始拜通直散騎侍郎,遷中書侍郎。簡文帝為撫軍,引為司馬。尋除尚
 書吏部郎,拜御史中丞。時北中郎荀羨有疾,朝廷以曇為羨軍司,加散騎常侍。頃之,羨徵還,仍除北中郎將、都督徐兗青幽揚州之晉陵諸軍事、領徐兗二州刺史、假節,鎮下邳,後與賊帥傅末波等戰失利,降號建威將軍。尋卒,年四十二。追贈北中郎,謚曰簡。子恢嗣。



 恢字道胤,少襲父爵,散騎侍郎,累遷給事黃門侍郎,領太子右衛率。恢身長八尺,美鬢髯,孝武帝深器之,以為有籓伯之望。會朱序自表去職,擢恢為梁秦雍司荊揚並等州諸軍事、建威將軍、雍州刺史、假節,鎮襄陽。恢甚得關隴之和,降附者動有千計。



 初,姚萇將竇衝來降,拜
 東羌校尉。衝後舉兵反,入漢川,襲梁州。時關中有巴蜀之眾,皆背萇,據弘農以結苻登。而登署衝為左丞相,徙屯華陰。河南太守楊佺期遣上黨太守荀靜戍皇天塢以距之。衝數來攻,恢遣將軍趙睦守金墉城,而佺期率眾次湖城,討衝,走之。



 尋而慕容垂圍慕容永於潞川,永窮蹙,遣其子弘求救於恢,並獻玉璽一紐,恢獻璽於臺,又陳「垂若並永,其勢難測。今於國計,謂宜救永。永垂並存,自為仇讎,連雞不棲,無能為患。然後乘機雙斃,則河北可平」。孝武帝以為然,詔王恭、庾楷救之,未及發而永沒。楊佺期以疾去職。



 恢以隨郡太守夏侯宗之為河南
 太守,戍洛陽。姚萇遣其子略攻湖城及上洛,又使其將楊佛嵩圍洛陽。恢遣建武將軍辛恭靖救洛陽,梁州刺史王正胤率眾出子午谷,以為聲援。略懼而退。恢以功進征虜將軍,又領秦州刺史,加督隴上軍。



 時魏氏彊盛,山陵危逼,恢遣江夏相鄧啟方等以萬人距之,與魏主拓跋珪戰于滎陽,大敗而還。



 及王恭計王國寶,桓玄、殷仲堪皆舉兵應恭,恢與朝廷掎角玄等。襄陽太守夏侯宗之、府司馬郭毗並以為不可,恢皆殺之。既而玄等退守尋陽。以恢為尚書,將家還都,至楊口,仲堪陰使人於道殺之,及其四子,託以群蠻所殺。喪還京師,贈鎮軍將
 軍。子循嗣。



 隆字弘始,蹇亮有匪躬之節。初為尚書郎,轉左丞,在朝為百僚所憚,坐漏洩事免。頃之,為吏部郎,復免。補東郡太守。



 隆少為趙王倫所善,及倫專擅,召為散騎常侍。倫之篡也,以為揚州刺史。僚屬有犯,輒依臺閣峻制繩之,遠近咸怨。尋加寧東將軍,未拜,而齊王冏檄至,中州人在軍者皆欲赴義,隆以兄子鑒為趙王掾,諸子悉在京洛,故猶豫未決。主簿趙誘、前秀才虞潭白隆曰:「當今上計,明使君自將精兵徑赴齊王;中計,明使君可留督攝,速遣猛將率精兵疾赴;下計,示遣兵將助,而稱背倫。」隆
 素敬別駕顧彥,密與謀之。彥曰:「趙誘下計,乃上策也。」西曹留承聞彥言,請見,曰:「不審明使君當今何施?」隆曰:「我俱受二帝恩,無所偏助,惟欲守州而已。」承曰:「天下者,世祖皇帝之天下也。太上承代已積十年,今上取四海不平,齊王應天順時,成敗之事可見。使君若顧二帝,自可不行,宜急下檄文,速遣精兵猛將。若其疑惑,此州豈可得保也!」隆無所言,而停檄六日。時寧遠將軍陳留王邃領東海都尉,鎮石頭,隆軍人西赴邃甚眾。隆遣從事於牛渚禁之,不得止。將士憤怒,夜扶邃為主而攻之,隆父子皆死,顧彥亦被害,誣隆聚合遠近,圖為不軌。隆之死
 也,時議莫不痛惜焉。



 史臣曰:忠臣本乎孝子,奉上資乎愛親,自家刑國,於期極矣。太真性履純深,譽流邦族,始則承顏候色,老萊弗之加也;既而辭親蹈義,申胥何以尚焉!封狐萬里,投軀而弗顧;猰窳千群,探穴而忘死。竟能宣力王室,揚名本朝,負荷受遺,繼之全節。言念主辱,義聲動於天地;祗赴國屯,信誓明於日月。枕戈雨泣,若雪分天之仇;皇輿旋軫,卒復夷庚之躅。微夫人之誠懇,大盜幾移國乎!道徽儒雅,柔而有正,協德始安,頗均連璧。方回踵武,奕世登台。露冕為飾,援高人以同志,抑惟大隱者獻!愛子云亡,
 省遺文而輟泣,殊有大義之風矣。



 贊曰:太真懷貞,勤宣乃誠。謀敦翦峻,奮節摛名。道徽忠勁,高芬遠映。愔克負荷,超慚雅正。



\end{pinyinscope}