\article{列傳第一}

\begin{pinyinscope}

 後妃上宣穆張皇後景懷夏侯皇後景獻羊皇后文明王皇后武元楊皇后武悼楊皇后左
 貴嬪胡貴嬪諸葛夫人惠賈皇后惠羊皇后謝夫人懷王皇太后元夏侯太妃



 夫乾坤定位,男女流形,伉儷之義同歸,貴賤之名異等。若乃作配皇極,齊體紫宸,象玉床之連後星,喻金波之合羲璧。爰自夐古,是謂元妃;降及中年,乃稱王后。四人並列,光于帝嚳之宮;二妃同降,著彼有虞之典。夏商以上,六宮之制,其詳靡得而聞焉。姬劉以降,五翟之規,其事可略而言矣。周禮,天子立一后、三夫人、九嬪、二十七
 世婦、八十一御妻,以聽王者內政。故《婚義》曰:「天子之與后,如日之與月,陰之與陽。」由斯而談,其所從來遠矣。故能母儀天,助宣王化,德均載物,比大坤維,宗廟歆其薦羞,穹壤俟其交泰。是以哲王垂憲,尤重造舟之禮;詩人立言,先獎《葛覃》之訓。後燭流景,所以裁其宴私,房樂希聲,是用節其容止。履端正本,抑斯之謂歟!若乃娉納有方,防閑有禮,肅尊儀而修四德,體柔範而弘六義,陰教洽於宮闈,淑譽騰於區域。則玄雲入戶,上帝錫母萌之符;黃神降徵,坤靈贊壽丘之道,終能鼎祚惟永,胤嗣克昌。至若儷極虧閑,憑天作孽,倒裳衣於衽席,感朓側
 於弦望。則龍漦結釁,宗周鞠為黍苗。燕尾挻災,隆漢墜其枌擱社矣。自曹劉內主,位以色登,甄衛之家,榮非德舉。淫荒挺性,蔑西郊之禮容;婉孌含辭,作南國之奇態。詖謁由斯外入,穢德於是內宣。椒掖播晨牝之風,蘭殿絕河雎之響。永言彤史,大練之範逾微;緬視青蒲,脫珥之猷替矣。晉承其末,與世污隆,宣皇創基,功弘而道屈;穆后一善,績侔於十亂。洎乎世祖,始親選良家,既而帝掩紈扇,躬行請託。后採長白,實彰妒忌之情;賈納短青,竟踐覆亡之轍。得失遺跡,煥在綈緗,興滅所由,義同畫一。故列其本事,以為后妃傳云。



 宣穆張皇后,諱春華,河內平皋人也。父汪,魏粟邑令。母河內山氏,司徒濤之從祖姑也。后少有德行,智識過人,生景帝、文帝、平原王乾、南陽公主。宣帝初辭魏武之命,託以風痺,嘗暴書,遇暴雨,不覺自起收之。家惟有一婢見之,后乃恐事泄致禍,遂手殺之以滅口,而親自執爨。帝由是重之。其後柏夫人有寵,后罕得進見。帝嘗臥疾,后往省病。帝曰:「老物可憎,何煩出也!」后慚恚不食,將自殺,諸子亦不食。帝驚而致謝,后乃止。帝退而謂人曰:「老物不足惜,慮困我好兒耳!」魏正始八年崩,時年五十九,
 葬洛陽高原陵,追贈廣平縣君。咸熙元年,追號宣穆妃。及武帝受禪,追尊為皇后。



 景懷夏侯皇后,諱徽,字媛容,沛國譙人也。父尚,魏征南大將軍。母曹氏,魏德陽鄉主。后雅有識度,帝每有所為,必豫籌畫。魏明帝世,宣帝居上將之重,諸子並有雄才大略。后知帝非魏之純臣,而后既魏氏之甥,帝深忌之。青龍二年,遂以鴆崩,時年二十四,葬峻平陵。武帝登阼,初未追崇,弘訓太后每以為言,泰始二年始加號謚。后無男,生五女。



 景獻羊皇后,諱徽瑜,泰山南城人。父[A155],上黨太守。后母陳留蔡氏,漢左中郎將邕之女也。后聰敏有才行。景懷皇后崩,景帝更娶鎮北將軍濮陽吳質女,見黜,復納后,無子。武帝受禪,居弘訓宮,號弘訓太后。泰始九年,追贈蔡氏濟陽縣君,謚曰穆。咸寧四年,太后崩,時年六十五,祔葬峻平陵。



 文明王皇后,諱元姬,東海郯人也。父肅,魏中領軍、蘭陵侯。后年八歲,誦《詩》《論》,尤善喪服。茍有文義,目所一見,必
 貫於心。年九歲,遇母疾,扶侍不捨左右,衣不解帶者久之。每先意候指,動中所適,由是父母令攝家事,每盡其理。祖郎甚愛異之,曰:「興吾家者,必此女也,惜不為男矣!」年十二,郎薨。后哀戚哭泣,發於自然,其父益加敬異。既笄,歸于文帝,生武帝及遼東悼王定國、齊獻王攸、城陽哀王兆、廣漢殤王廣德、京兆公主。后事舅姑盡婦道,謙沖接下,嬪御有序。及居父喪,身不勝衣,言與淚俱。時鐘會以才能見任,后每言於帝曰:「會見利忘義,好為事端,寵過必亂,不可大任。」會後果反。



 武帝受禪,尊為皇太后,宮曰崇化。初置宮卿,重選其職,以太常諸葛緒為衛尉,
 太僕劉原為太僕,宗正曹楷為少府。后雖處尊位,不忘素業,躬執紡績,器服無文,御浣濯之衣,食不參味。而敦睦九族,垂心萬物,言必典禮,浸潤不行。



 帝以后母羊氏未崇謚號,泰始三年下詔曰:「昔漢文追崇靈文之號,武、宣有平原、博平之封,咸所以奉尊尊之敬,廣親親之恩也。故衛將軍、蘭陵景侯夫人羊氏,含章體順,仁德醇備,內承世胄,出嬪大國,三從之行,率禮無違。仍遭不造,頻喪統嗣,撫育眾胤,克成家道。母儀之教,光于邦族,誕啟聖明,祚流萬國,而早世殂隕,不遇休寵。皇太后孝思蒸蒸,永慕罔極。朕感存遺訓,追遠傷懷。其封夫人為縣君,
 依德紀謚,主者詳如舊典」於是使使持節、謁者何融追謚為平陽靖君。



 四年,后崩,時年五十二,合葬崇陽陵。將遷祔,帝手疏后德行,命史官為哀策曰:



 明明先后,興我晉道。暉章淑問,以翼皇考。邁德宣猷,大業有造,貽慶孤矇,堂構是保。庶資復顧,永享難老。奄然登遐,棄我何早!沈哀罔訴,如何穹昊。嗚呼哀哉!



 厥初生民,樹之惠康。帝遷明德,顧予先皇。天立厥配,我皇是光。作邦作對,德音無疆。愍予不弔,天篤降殃。日沒《明夷》,中年隕喪。煢煢在疚,永懷摧傷。尋惟景行,於穆不已。海岱降靈,世荷繁祉。永錫祚胤,篤生文母。誕膺純和,淑慎容止。質直不渝,體
 茲孝友。《詩》《書》是悅,禮籍是紀。三從無違,中饋允理。追惟先后,勞謙是尚。爰初在室,竭力致養。嬪于大邦,皇基是相。謐靜隆化,帝業以創。內敘嬪御,外協時望。履信居順,德行洽暢。密勿無荒,劬勞克讓。崇儉抑華,沖素是放。雖享崇高,歡嘉未饗。胡寧棄之,我將曷仰?咨餘不造,大罰薦臻。皇考背世,始踰三年。仰奉慈親,冀無後艱。凶災仍集,何辜于天。嗚呼哀哉!



 靈轜夙駕,設祖中闈。轀輬動軫,既往不追。哀哀皇妣,永潛靈暉。進攀梓宮,顧援素旂。屏營窮痛,誰告誰依?訴情贈策,以舒傷悲。尚或有聞,顧予孤遺。嗚呼哀哉!



 其後帝追慕不已,復下詔曰:「外曾祖母
 故司徒王郎夫人楊氏,舅氏尊屬,鄭、劉二從母,先后至愛。每惟聖善,敦睦遺旨,渭陽之感,永懷靡及。其封楊夫人及從母為鄉君,邑各五百戶。」太康七年,追贈繼祖母夏侯氏為滎陽鄉君。



 武元楊皇后,諱艷,字瓊芝,弘農華陰人也。父文宗,見《外戚傳》。母天水趙氏,早卒。后依舅家,舅妻仁愛,親乳養后,遣他人乳其子。及長,又隨後母段氏,依其家。后少聰慧,善書,姿質美麗,閑於女工。有善相者嘗相后,當極貴,文帝聞而為世子聘焉。甚被寵遇,生毗陵悼王軌、惠帝、秦
 獻王柬,平陽、新豐、陽平公主。武帝即位,立為皇后。有司奏依漢故事,皇后、太子各食湯沐邑四十縣,而帝以非古典,不許。后追懷舅氏之恩,顯官趙俊,納俊兄虞女粲於後宮為夫人。



 帝以皇太子不堪奉大統,密以語后。后曰:「立嫡以長不以賢,豈可動乎?」初,賈充妻郭氏使賂后,求以女為太子妃。及議太子婚,帝欲娶衛瓘女。然后盛稱賈后有淑德,又密使太子太傅荀顗進言,上乃聽之。泰始中,帝博選良家以充後宮,先下書禁天下嫁娶,使宦者乘使車,給騶騎,馳傳州郡,召充選者使后揀擇。后性妒,惟取潔白長大,其端正美麗者並不見留。時卞籓
 女有美色,帝掩扇謂后曰:「卞氏女佳。」后曰:「籓三世后族,其女不可枉以卑位。」帝乃止。司徒李胤、鎮軍大將軍胡奮、廷尉諸葛沖、太僕臧權、侍中馮蓀、秘書郎左思及世族子女並充三夫人九嬪之列。司、冀、兗、豫四州二千石將吏家,補良人以下。名家盛族子女,多敗衣瘁貌以避之。



 及后有疾,見帝素幸胡夫人,恐後立之,慮太子不安。臨終,枕帝膝曰:「叔父駿女男胤有德色,願陛下以備六宮。」因悲泣,帝流涕許之。泰始十年,崩于明光殿,絕子帝膝,時年三十七。詔曰:「皇后逮事先后,常冀能終始永奉宗廟,一旦殂隕,痛悼傷懷。每自以夙喪二親,於家門之
 情特隆。又有心欲改葬父祖,以頃者務崇儉約,初不有言,近垂困,說此意,情亦愍之。其使領前軍將軍駿等自克改葬之宜,至時,主者供給葬事。賜謚母趙氏為縣君,以繼母段氏為鄉君。傳不云乎,『慎終追遠,民德歸厚。』且使亡者有知,尚或嘉之。」於是有司卜吉,窀穸有期,乃命史臣作哀策敘懷。其詞曰:



 天地配序,成化兩儀。王假有家,道在伉儷。姜嫄佐嚳,二妃興媯。仰希古昔,冀亦同規。今胡不然,景命夙虧。嗚呼哀哉!



 我應圖LB,統臨萬方。正位于內,實在嬪嬙。天作之合,駿發之祥。河嶽降靈,啟祚華陽。奕世豐衍,朱紼斯煌。纘女惟行,受命溥將。來翼家
 邦,憲度是常。緝熙陰教,德聲顯揚。昔我先妣,暉曜休光。后承前訓,奉述遺芳。宜嗣徽音,繼序無荒。如何不弔,背世隕喪。望齊無主,長去烝嘗。追懷永悼,率土摧傷。嗚呼哀哉!



 陵兆既窆,將遷幽都,宵陳夙駕,元妃其徂。宮闈遏密,階庭空虛。設祖布紼,告駕啟塗。服翬褕狄,寄象容車。金路晻藹,裳帳不舒。千乘動軫,六驥躊躇。銘旌樹表,翣柳雲敷。祁祁同軌,岌岌烝徒。孰不云懷,哀感萬夫。寧神虞卜,安體玄廬。土房陶簋,齊制遂初。依行紀謚,聲被八區。雖背明光,亦歸皇姑。沒而不朽,世德作謨。嗚呼哀哉!



 乃葬于峻陽陵。



 武悼楊皇后,諱芷,字季蘭,小字男胤,元后從妹。父駿,別有傳。以咸寧二年立為皇后。婉[A148]有婦德,美映椒房,甚有寵。生渤海殤王,早薨,遂無子。太康九年,后率內外夫人命婦躬桑于西郊,賜帛各有差。



 太子妃賈氏妒忌,帝將廢之。后言于帝曰:「賈公閭有勛社稷,猶當數世宥之,賈妃親是其女,正復妒忌之間,不足以一眚掩其大德。」后又數誡厲妃,妃不知后之助己,因以致恨,謂后構之於帝,忿怨彌深。及帝崩,尊為皇太后。賈后凶悖,忌后父駿執權,遂誣駿為亂,使楚王瑋與東安王繇稱詔誅駿。
 內外隔塞,后題帛為書,射之城外,曰「救太傅者有賞,」賈后因宣言太后同逆。



 駿既死,詔使後軍將軍荀悝送后於永寧宮。特全后母高都君龐氏之命,聽就后居止。賈后諷群公有司奏曰:「皇太后陰漸姦謀,圖危社稷,飛箭系書,要募將士,同惡相濟,自絕于天。魯侯絕文姜,《春秋》所許,蓋以奉順祖宗,任至公於天下。陛下雖懷無已之情,臣下不敢奉詔。可宣敕王公於朝堂會議。」詔曰:「此大事,更詳之。」有司又奏:「駿藉外戚之資,居冢宰之任,陛下既居諒闇,委以重權,至乃陰圖凶逆,布樹私黨。皇太后內為脣齒,協同逆謀,禍釁既彰,背捍詔命,阻兵負眾,血
 刃宮省,而復流書募眾,以獎凶黨,上背祖宗之靈,下絕億兆之望。昔文姜與亂,《春秋》所貶,呂宗叛戾,高后降配,宜廢皇太后為峻陽庶人。」中書監張華等以為「太后非得罪於先帝者也,今黨惡所親,為不母於聖世。宜依孝成趙皇后故事,曰武帝皇后,處之離宮,以全貴終之恩」。尚書令、下邳王晃等議曰:「皇太后與駿潛謀,欲危社稷,不可復奉承宗廟,配合先帝。宜貶尊號,廢詣金墉城。」於是有司奏:「請從晃等議,廢太后為庶人。遣使者以太牢告于郊廟,以奉承祖宗之命,稱萬國之望。至於諸所供奉,可順聖恩,務從豐厚。」詔不許。有司又固請,乃可之。又
 奏:「楊駿造亂,家屬應誅,詔原其妻龐命,以慰太后之心。今太后廢為庶人,請以龐付廷尉行刑。」詔曰:「聽龐與庶人相隨。」有司希賈后旨,固請,乃從之。龐臨刑,太后抱持號叫,截髮稽顙,上表詣賈后稱妾,請全母命,不見省。初,太后尚有侍御十餘人,賈后奪之,絕膳而崩,時年三十四,在位十五年。賈后又信妖巫,謂太后必訴冤先帝,乃覆而殯之,施諸厭劾符書藥物。



 永嘉元年,追復尊號,別立廟,神主不配武帝。至成帝咸康七年,下詔使內外詳議。衛將軍虞潭議曰:「世祖武皇帝光有四海,元皇后應乾作配。元后既崩,悼后繼作,至楊駿肆逆,禍延天母。孝
 懷皇帝追復號謚,豈不以鯀殛禹興,義在不替者乎!又太寧二年,臣LC宗正,帝譜泯棄,罔所循按。時博諮舊齒,以定昭穆,與故驃騎將軍華恒、尚書荀崧、侍中荀邃因舊譜參論撰次,尊號之重,一無改替。今聖上孝思,祗肅禋祀,詢及群司,將以恢定大禮。臣輒思詳,伏見惠皇帝《起居注》、群臣議奏,列駿作逆謀,危社稷,引魯之文姜,漢之呂后。臣竊以文姜雖莊公之母,實為父仇;呂后寵樹私戚,幾危劉氏,按此二事異於今日,昔漢章帝竇后殺和帝之母,和帝即位盡誅諸竇。當時議者欲貶竇后,及后之亡,欲不以禮葬。和帝以奉事十年,義不可違,臣子
 之道,務從豐厚,仁明之稱,表於往代。又見故尚書僕射裴頠議悼后故事,稱繼母雖出,追服無改。是以孝懷皇帝尊崇號謚,還葬峻陵。此則母子道全,而廢事蕩革也。于時祭於弘訓之宮,未入太廟。蓋是事之未盡,非義典也。若以悼后復位為宜,則應配食世祖;若以復之為非,則譜謚宜闕,未有位號居正,而偏祠別室者也。若以孝懷皇帝私隆母子之道,特為立廟者,此茍崇私情,有虧國典,則國譜帝諱,皆宜除棄,匪徒不得同祀於世祖之廟也。」會稽王昱、中書監庾冰、中書令何充、尚書令諸葛恢、尚書謝廣、光祿勳留擢、丹楊尹殷融、護軍將軍馮懷、
 散騎常侍鄧逸等咸從潭議,由是太后配食武帝。



 左貴嬪,名芬。兄思,別有傳。芬少好學,善綴文,名亞於思,武帝聞而納之。泰始八年,拜修儀。受詔作愁思之文,因為《離思賦》曰:



 生蓬戶之側陋兮,不閑習於文符。不見圖畫之妙像兮,不聞先哲之典謨。既愚陋而寡識兮,謬LC廁於紫廬。非草苗之所處兮,恒怵惕以憂懼。懷思慕之忉怛兮,兼始終之萬慮。嗟隱憂之沈積兮,獨鬱結而靡訴。意慘憒而無聊兮,思纏綿以增慕。夜耿耿而不寐兮,魂憧憧而至曙。風騷騷而四起兮,霜皚皚而依庭。日晻
 曖而無光兮,氣懰慄以冽清。懷愁戚之多感兮,患涕淚之自零。



 昔伯瑜之婉孌兮,每彩衣以娛親。悼今日之乖隔兮,奄與家為參辰。豈相去之云遠兮,曾不盈乎數尋。何宮禁之清切兮,欲瞻睹而莫因。仰行雲以歔欷兮,涕流射而沾巾。惟屈原之哀感兮,嗟悲傷於離別。彼城闕之作詩兮,亦以日而喻月。況骨肉之相於兮,永緬邈而兩絕。長含哀而抱戚兮,仰蒼天而泣血。



 亂曰:骨肉至親,化為他人,永長辭兮。慘愴愁悲,夢想魂歸,見所思兮。驚寤號咷,心不自聊,泣漣水而兮。援筆舒情,涕淚增零,訴斯詩兮。



 後為貴嬪,姿陋無寵,以才德見禮。體羸多患,常居
 薄室,帝每遊華林,輒回輦過之。言及文義,辭對清華,左右侍聽,莫不稱美。



 及元楊皇后崩,芬獻誄曰:



 惟泰始十年秋七月丙寅,晉元皇后楊氏崩,嗚呼哀哉!昔有莘適殷,姜姒歸周,宣德中闈,徽音永流。樊衛二姬,匡齊翼楚;馬鄧兩妃,亦毗漢主。峨峨元后,光嬪晉宇。伉儷聖皇,比蹤往古。遭命不永,背陽即陰。六宮號咷,四海慟心。嗟余鄙妾,銜恩特深。追慕三良,甘心自沈。何用存思?不忘德音。何用紀述?託辭翰林。乃作誄曰:



 赫赫元后,出自有楊。奕世朱輪,耀彼華陽。惟嶽降神,顯茲禎祥。篤生英媛,休有烈光。含靈握文,異于庶姜。和暢春日,操厲秋霜。疾彼
 攸遂,敦此義方。率由四教,匪怠匪荒。行周六親,徽音顯揚。顯揚伊何?京室是臧。乃娉乃納,聿嬪聖皇。正位閨閾,惟德是將。鳴珮有節,發言有章。仰觀列圖,俯覽篇籍。顧問女史,咨詢竹帛。思媚皇姑,虔恭朝夕。允釐中饋,執事有恪。



 于禮斯勞,于敬斯勤。雖曰齊聖,邁德日新。日新伊何,克廣弘仁。終溫且惠,帝妹是親。經緯六宮,罔不彌綸。群妾惟仰,譬彼北辰。亦既青陽,鳴鳩告時,躬執桑曲,率導媵姬。修成蠶蔟,分繭理絲。女工是察,祭服是治。祗奉宗廟,永言孝思。于彼六行,靡不蹈之。皇英佐舜,塗山翼禹。惟衛惟樊,二霸是輔。明明我后,異世同矩。亦能有亂,
 謀及天府。內敷陰教,外毗陽化。綢繆庶正,密勿夙夜。恩從風翔,澤隨雨播。中外禔福,遐邇詠歌。



 天祚貞吉,克昌克繁。則百斯慶,育聖育賢。教踰妊姒,訓邁姜嫄。堂堂太子,惟國之元。濟濟南陽,為屏為籓。本支菴藹,四海蔭焉。微斯皇妣,孰茲克臻。曰乾蓋聰,曰聖允誠。積善之堂,五福所並。宜享高年,匪隕匪傾。如彭之齒,如聃之齡。云胡不造,于茲禍殃。寢疾彌留,寤寐不康。巫咸騁術,和鵲奏方。祈禱無應,嘗藥無良。形神將離,載昏載荒。奄忽崩殂,湮精滅光。哀哀太子,南陽繁昌。攀援不寐,擗踴摧傷。嗚呼哀哉!闔宮號咷,宇內震驚。奔者填衢,赴者塞庭,哀慟
 雷駭,流淚雨零。歔欷不已,若喪所生。



 惟帝與后,契闊在昔。比翼白屋,雙飛紫閣。悼后傷后,早即窀穸。言斯既及,涕泗隕落。追惟我后,實聰實哲。通于性命,達于儉節。送終之禮,比素上世。襚無珍寶,唅無明月。潛輝梓宮,永背昭晰。臣妾哀號,同此斷絕。庭宇遏密,幽室增陰。空設幃帳,虛置衣衾。人亦有言,神道難尋。悠悠精爽,豈浮豈沈。豐奠日陳,冀魂之臨。孰云元后,不聞其音。



 乃議景行,景行已溢。乃考龜筮,龜筮襲吉。爰定宅兆,克成玄室。魂之往矣,于以令日。仲秋之晨,啟明始出。星陳夙駕,靈輿結駟。其輿伊何?金根玉箱。其駟伊何?二駱雙黃。習習容車,
 朱服丹章。隱隱轜軒,弁絰繐裳。華轂曜野,素蓋被原。方相仡仡,旌旐翻翻。挽童引歌,白驥鳴轅。觀者夾塗,士女涕漣。千乘萬騎,迄彼峻山。峻山峨峨,曾阜重阿。弘高顯敞,據洛背河。左瞻皇姑,右睇帝家。推存揆亡,明神所嘉。諸姑姊妹,娣以媵御。追送塵軌,號光衢路。王侯卿士,雲會星布。群官庶僚,縞蓋無數。咨嗟通夜,東方云曙。百祗奉迎,我后安厝。中外俱臨,同哀並慕。涕如連雲,淚如湛露。扃闓既闔,窈窈冥冥。有夜無晝,曷用其明。不封不樹,山阪同形。



 昔后之崩,大火西流。寒往暑過,今亦孟秋。自我銜恤,倏忽一周。衣服將變,痛心若抽。逼彼禮制,惟以
 增憂。去此素衣,結戀靈丘。有始有終,天地之經。自非三光,誰能不零。存播令德,沒圖丹青。先哲之志,以此為榮。溫溫元后,實宣慈焉。撫育群生,恩惠滋焉。遺愛不已,永見思焉。懸名日月,垂萬春焉。嗚呼庶妾,感四時焉。言思言慕,涕漣水而焉。



 咸寧二年,納悼后,芬於座受詔作頌,其辭曰:



 峨峨華獄,峻極泰清。巨靈導流,河瀆是經。惟瀆之神,惟嶽之靈。鐘於楊族,載育盛明。穆穆我后,應期挺生。含聰履喆,岐嶷夙成。如蘭之茂,如玉之榮。越在幼沖,休有令名。飛聲八極,翕習紫庭。超妊邈姒,比德皇英。京室是嘉,備禮致娉。令月吉辰,百僚奉迎。周生歸韓,詩人是
 詠。我后戾止,車服暉映。登位太微,明德日盛。群黎欣戴,函夏同慶。



 翼翼聖皇,睿喆孔純。愍茲狂戾,闡惠播仁。蠲釁滌穢,與時惟新。沛然洪赦,恩詔遐震。后之踐阼,囹圄虛陳。萬國齊歡,六合同欣。坤神抃舞,天人載悅。興瑞降祥,表精日月。和氣煙煴,三光朗烈。既獲嘉時,尋播甘雪。玄雲晻藹,靈液霏霏,既儲既積,待陽而晞。曣晛沾濡,柔潤中畿。長享豐年,福祿永綏。



 及帝女萬年公主薨,帝痛悼不已,詔芬為誄,其文甚麗。帝重芬詞藻,每有方物異寶,必詔為賦頌,以是屢獲恩賜焉。答兄思詩、書及雜賦頌數十篇,並行於世。



 胡貴嬪名芳。父奮,別有傳。泰始九年,帝多簡良家子女以充內職,自擇其美者以絳紗繫臂。而芳既入選,下殿號泣。左右止之曰:「陛下聞聲。」芳曰:「死且不畏,何畏陛下!」帝遣洛陽令司馬肇策拜芳為貴嬪。帝每有顧問,不飾言辭,率爾而答,進退方雅。時帝多內寵,平吳之後復納孫皓宮人數千,自此掖庭殆將萬人,而並寵者甚眾,帝莫知所適,常乘羊車,恣其所之,至便宴寢。官人乃取竹葉插戶,以鹽汁灑地,而引帝車。然芳最蒙愛幸,殆有專房之寵焉,侍御服飾亞于皇后。帝嘗與之摴蒱,爭矢,遂
 傷上指。帝怒曰:「此固將種也!」芳對曰:「北伐公孫,西距諸葛,非將種而何?」帝甚有慚色。芳生武安公主。



 諸葛夫人,名婉,琅邪陽都人也。父沖,字茂長,廷尉卿。婉以泰始九年春入宮,帝臨軒,使使持節、洛陽令司馬肇拜為夫人。兄銓,字德林,散騎常侍。銓弟玫,字仁林,侍中、御史中丞。玫婦弟周穆,清河王覃之舅也。永嘉初,穆與玫勸東海王越廢懷帝,立覃,越不許。重言之,越怒,遂斬玫及穆。臨刑,玫謂穆曰:「我語卿何道?」穆曰:「今日復何所說。」時人方知謀出于穆,非玫之意。



 惠賈皇后,諱南風,平陽人也,小名LD。父充,別有傳。初,武帝欲為太子取衛瓘女,元后納賈郭親黨之說,欲婚賈氏。帝曰:「衛公女有五可,賈公女有五不可。衛家種賢而多子,美而長白;賈家種妒而少子,醜而短黑。」元后固請,荀顗、荀勖並稱充女之賢,乃定婚。始欲聘后妹午,午年十二,小太子一歲,短小未勝衣。更娶南風,時年十五,大太子二歲。泰始八年二月辛卯,冊拜太子妃。妒忌多權詐,太子畏而惑之,嬪御罕有進幸者。



 帝常疑太子不慧,且朝臣和嶠等多以為言,故欲試之。盡召東宮大小官
 屬,為設宴會,而密封疑事,使太子決之,停信待反。妃大懼,倩外人作答。答者多引古義。給使張泓曰:「太子不學,而答詔引義,必責作草主,更益譴負。不如直以意對。」妃大喜,語泓:「便為我好答,富貴與汝共之。」泓素有小才,具草,令太子自寫。帝省之,甚悅。先示太子少傅衛瓘,瓘大踧,眾人乃知瓘先有毀言,殿上皆稱萬歲。充密遣語妃云:「衛瓘老奴,幾破汝家。」



 妃性酷虐,嘗手殺數人。或以戟擲孕妾,子隨刃墮地。帝聞之,大怒,已修金墉城,將廢之。充華趙粲從容言曰:「賈妃年少,妒是婦人之情耳,長自當差。願陛下察之。」其後楊珧亦為之言曰:「陛下忘賈
 公閭耶?」荀勖深救之,故得不廢。惠帝即位,立為皇后,生河東、臨海、始平公主、哀獻皇女。



 后暴戾日甚。侍中賈模,后之族兄,右衛郭彰,后之從舅,並以才望居位,與楚王瑋、東安公繇分掌朝政。后母廣城君養孫賈謐干預國事,權侔人主。繇密欲廢后,賈氏憚之。及太宰亮、衛瓘等表繇徙帶方,奪楚王中候,后知瑋怨之,乃使帝作密詔令瑋誅瓘、亮,以報宿憾。模知后凶暴,恐禍及己,乃與裴頠、王衍謀廢之,衍悔而謀寢。



 后遂荒淫放恣,與太醫令程據等亂彰內外。洛南有盜尉部小吏,端麗美容止,既給廝役,忽有非常衣服,眾咸疑其竊盜,尉嫌而辯之。賈
 后疏親欲求盜物,往聽對辭。小吏云:「先行逢一老嫗,說家有疾病,師卜云宜得城南少年厭之,欲暫相煩,必有重報。於是隨去,上車下帷,內簏箱中,行可十餘里,過六七門限,開簏箱,忽見樓闕好屋。問此是何處,云是天上,即以香湯見浴,好衣美食將入。見一婦人,年可三十五六,短形青黑色,眉後有疵。見留數夕,共寢歡宴。臨出贈此眾物。」聽者聞其形狀,知是賈后,慚笑而去,尉亦解意。時他人入者多死,惟此小吏,以后愛之,得全而出。及河東公主有疾,師巫以為宜施寬令,乃稱詔大赦天下。



 初,后詐有身,內稿物為產具,遂取妹夫韓壽子慰祖養之,
 託諒闇所生,故弗顯。遂謀廢太子,以所養代立。時洛中謠曰:「南風烈烈吹黃沙,遙望魯國鬱嵯峨,前至三月滅汝家。」后母廣城君以后無子,甚敬重愍懷,每勸厲后,使加慈愛。賈謐恃貴驕縱,不能推崇太子,廣城君恒切責之,及廣城君病篤,占術謂不宜封廣城,乃改封宜城。后出侍疾十餘日,太子常往宜城第,將醫出入,恂恂盡禮。宜城臨終執后手,令盡意於太子,言甚切至,又曰:「趙粲及午必亂汝事,我死後,勿復聽入,深憶吾言。」后不能遵之,遂專制天下,威服內外。更與粲、午專為姦謀,誣害太子,眾惡彰著。初,誅楊駿及汝南王亮、太保衛瓘、楚王瑋
 等,皆臨機專斷。宦人董猛參預其事。猛,武帝時為寺人監,侍東宮,得親信於后,預誅楊駿,封武安侯,猛三兄皆為亭侯,天下咸怨。



 及太子廢黜,趙王倫、孫秀等因眾怨謀欲廢后。后數遣宮婢微服於人間視聽,其謀頗泄。后甚懼,遂害太子,以絕眾望。趙王倫乃率兵入宮,使翊軍校尉齊王冏入殿廢后。后與冏母有隙,故倫使之。后驚曰:「卿何為來!」冏曰:「有詔收后。」后曰:「詔當從我出,何詔也?」后至上閤,遙呼帝曰:「陛下有婦,使人廢之,亦行自廢。」又問冏曰:「起事者誰?」冏曰:「梁、趙。」后曰:「繫狗當繫頸,今反繫其尾,何得不然!」至宮西,見謐尸,再舉聲而哭遽止。倫乃
 矯詔遣尚書劉弘等持節齎金屑酒賜后死。后在位十一年。趙粲、賈午、韓壽、董猛等皆伏誅。



 臨海公主先封清河,洛陽之亂,為人所略,傳賣吳興錢溫。溫以送女,女遇主甚酷。元帝鎮建鄴,主詣縣自言。元帝誅溫及女,改封臨海,宗正曹統尚之。



 惠羊皇后,諱獻容,泰山南城人。祖瑾,父玄之,並見《外戚傳》。賈后既廢,孫秀議立后。后外祖孫旂與秀合族,又諸子自結於秀,故以太安元年立為皇后。將入宮,衣中有火。



 成都王穎伐長沙王乂,以討玄之為名。乂敗,穎奏廢
 后為庶人,處金墉城。陳等唱伐成都王,大赦,復后位。張方入洛,又廢后。方逼遷大駕幸長安,留臺復后位。永興初,張方又廢后。河間王顒矯詔,以后屢為姦人所立,遣尚書田淑敕留臺賜后死。詔書累至,司隸校尉劉暾與尚書僕射荀籓、河南尹周馥馳上奏曰:「奉被手詔,伏讀惶悴。臣按古今書籍,亡國破家,毀喪宗祊,皆由犯眾違人之所致也。陛下遷幸,舊京廓然,眾庶悠悠,罔所依倚。家有跂踵之心,人想鑾輿之聲,思望大德,釋兵歸農。而兵纏不解,處處互起,豈非善者不至,人情猜隔故耶!今上官巳犯闕稱兵,焚燒宮省,百姓喧駭,宜鎮之以靜。
 而大使卒至,赫然執藥,當詣金墉,內外震動,謂非聖意。羊庶人門戶殘破,廢放空宮,門禁峻密,若絕天地,無緣得與姦人構亂。眾無智愚,皆謂不然,刑書猥至,罪不值辜,人心一憤,易致興動。夫殺一人而天下喜悅者,宗廟社稷之福也。今殺一枯窮之人而令天下傷慘,臣懼凶豎乘間,妄生變故。臣LC司京輦,觀察眾心,實以深憂,宜當含忍。不勝所見,謹密啟聞。願陛下更深與太宰參詳,勿令遠近疑惑,取謗天下。」顒見表大怒,乃遣陳顏、呂朗東收暾。暾奔青州,后遂得免,帝還洛,迎后復位。後洛陽令何喬又廢后。及張方首至,其日復后位。



 會帝崩,后慮
 太弟立為嫂叔,不得稱太后,催前太子清河王覃入,將立之,不果。懷帝即位,尊后為惠帝皇后,居弘訓宮。洛陽敗,沒于劉曜。曜僭位,以為皇后。因問曰:「吾何如司馬家兒?」后曰:「胡可並言?陛下開基之聖主,彼亡國之暗夫,有一婦一子及身三耳,不能庇之,貴為帝王,而妻子辱於凡庶之手。遣妾爾時實不思生,何圖復有今日。妾生於高門,常謂世間男子皆然。自奉巾櫛以來,始知天下有丈夫耳。」曜甚愛寵之,生曜二子而死,偽謚獻文皇后。



 謝夫人,名玖。家本貧賤,父以屠羊為業,玖清惠貞正而
 有淑姿,選入後庭為才人。惠帝在東宮,將納妃。武帝慮太子尚幼,未知帷房之事,乃遣往東宮侍寢,由是得幸有身。賈后妒忌之,玖求還西宮,遂生愍懷太子,年三四歲,惠帝不知也。入朝,見愍懷與諸皇子共戲,執其手,武帝曰:「是汝兒也。」及立為太子,拜玖為淑媛。賈后不聽太子與玖相見,處之一室。及愍懷遇酷,玖亦被害焉。永康初,詔改葬太子,因贈玖夫人印綬,葬顯平陵。



 懷王皇太后,諱媛姬,不知所出。初入武帝宮,拜中才人,早卒。懷帝即位,追尊曰皇太后。



 元夏侯太妃,名光姬,沛國譙人也。祖威,兗州刺史。父莊,字仲容,淮南太守、清明亭侯。妃生自華宗,幼而明慧。瑯邪武王為世子覲納焉,生元帝。及恭王薨,元帝嗣立,稱王太妃。永嘉元年,薨于江左,葬瑯邪國。初有讖云「銅馬入海建鄴期,」太妃小字銅環,而元帝中興於江左焉。



\end{pinyinscope}