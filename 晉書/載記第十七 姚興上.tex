\article{載記第十七 姚興上}

\begin{pinyinscope}
姚興
 \gezhu{
  上}



 姚
 興,字子略,萇之長子也。苻堅時為太子舍人。萇之在馬牧,興自長安冒難奔萇,萇立為皇太子。萇出征討,常留統後事。及鎮長安,甚有威惠。與其中舍人梁喜、洗馬范勖等講論經籍,不以兵難廢業,時人咸化之。



 萇死,興秘不發喪,以其叔父緒鎮安定,碩德鎮陰密,弟崇守長安。碩德將佐言於碩德曰:「公威名宿重,部曲最強,今喪
 代之際,朝廷必相猜忌,非永安之道也。宜奔秦州,觀望事勢。」碩德曰:「太子志度寬明,必無疑阻。今苻登未滅而自尋干戈,所謂追二袁之蹤,授首與人。吾死而已,終不若斯。」及至,興優禮而遣之。



 興自稱大將軍,以尹緯為長史,狄伯支為司馬,率眾伐苻登。咸陽太守劉忌奴據避世堡以叛,興襲忌奴,擒之。苻登自六陌向廢橋,始平太守姚詳據馬嵬堡以距登。登眾甚盛,興慮詳不能遏,乃自將精騎以迫登,遣尹緯領步卒赴詳。緯用詳計,據廢橋以抗登。登因急攻緯,緯將出戰,興馳遣狄伯支謂緯曰:「兵法不戰而制人者,蓋為此也。苻登窮寇,宜持重,不
 可輕戰。」緯曰:「先帝登遐,人情擾懼,今不因思奮之力,梟殄逆豎,大事去矣。緯敢以死爭。」遂與登戰,大破之,登眾渴死者十二三,其夜大潰,登奔雍。興乃發喪行服。太元十九年,僭即帝位于槐里,大赦境內,改元曰皇初,遂如安定。



 先是,苻登使弟廣守雍,子崇屯胡空堡,聞登敗,各棄守走。登無所投據,遂奔平涼,率其餘眾入馬毛山。興自安定如涇陽,與登戰于山南,斬登。散其部眾,歸復農業。徙陰密三萬戶於長安,分大營戶為四,置四軍以領之。



 安南強熙、鎮遠楊多叛,推竇衝為盟主,所在擾亂。興率諸將討之,軍次武功,多兄子良國殺多而降。衝弟彰
 武與衝離貳,衝奔強熙。熙聞興將至,率戶二千奔秦州。竇衝走汧川,汧川氐仇高執送之。衝從弟統率其眾降于興。



 封征虜緒為晉王,征西碩德為隴西王,征南靖等及功臣尹緯、齊難、楊佛嵩等並為公侯,其餘封爵各有差。



 鮮卑薛勃於貳城為魏軍所伐,遣使請救,使姚崇赴救。魏師既還,薛勃復叛,崇伐而執之,大收其士馬而還。



 興追尊其庶母孫氏為皇太后,配饗太廟。



 楊盛保仇池,遣使請命,拜使持節、鎮南將軍、仇池公。鮮卑越質詰歸率戶二萬叛乞伏乾歸,降于興,興處之于成紀,拜使持節、鎮西將軍、平襄公。



 姚碩德討平涼胡金豹于洛城,剋
 之。初,上邽姜乳據本縣以叛,自稱秦州刺史。碩德進討之,乳率眾降。以碩德為秦州牧,領護東羌校尉,鎮上邽。征乳為尚書。強熙及略陽豪族權干城率眾三萬圍上邽,碩德擊破之。熙南奔仇池,遂假道歸晉。碩德西討干城,干城降。



 興令郡國各歲貢清行孝廉一人。



 慕容永既為慕容垂所滅,河東太守柳恭等各阻兵自守,興遣姚緒討之。恭等依河距守,緒不得濟。鎮東薛強先據楊氏壁,引緒從龍門濟河,遂入蒲阪。恭勢屈,請降。徙新平、安定新戶六千于蒲阪。



 興母虵氏死,興哀毀過禮,不親庶政。群臣議請依漢、魏故事,既葬即吉。興尚書郎李嵩上
 疏曰「三王異制,五帝殊禮。孝治天下,先王之高事也,宜遵聖性,以光道訓。既葬之後,應素服臨朝,率先天下,仁孝之舉也。」尹緯駁曰:「帝王喪制,漢、魏為準。嵩矯常越禮,愆于軌度,請付有司,以專擅論。既葬即吉,乞依前議。」興曰:「嵩忠臣孝子,有何咎乎?尹僕射棄先王之典,而欲遵漢、魏之權制,豈所望於朝賢哉!其一依嵩議。」



 鮮卑薛勃叛奔嶺北,上郡、貳川雜胡皆應之,遂圍安遠將軍姚詳於金城。遣姚崇、尹緯討之。勃自三交趣金城,崇列營掎之,而租運不繼,三軍大飢。緯言於崇曰:「輔國彌姐高地、建節杜成等皆諸部之豪,位班三品,督運稽留,令三軍
 乏絕,宜明置刑書,以懲不肅。」遂斬之。諸部大震,租入者五十餘萬。興率步騎二萬親討之,勃懼,棄其眾奔于高平公沒奕于,於執而送之。



 泫氏男姚買得欲因興葬母虵氏殺興,會有告之者,興未之信,遣李嵩詐往。買得具以告嵩,嵩還,以聞,興乃賜買得死,誅其黨與。



 興下書禁百姓造錦繡及淫祀。



 興率眾寇湖城,晉弘農太守陶仲山、華山太守董邁皆降于興。遂如陜城,進寇上洛,陷之。遣姚崇寇洛陽,晉河南太守夏侯宗之固守金墉,崇攻之不剋,乃陷柏谷,徙流人西河嚴彥、河東裴岐、韓襲等二萬餘戶而還。



 興下書,令士卒戰亡者守宰所在埋藏
 之,求其近親為之立後。



 武都氐屠飛、啖鐵等殺隴東太守姚迴,略三千餘家,據方山以叛。興遣姚紹等討之,斬飛、鐵。遣狄伯支迎流人曹會、牛壽萬餘戶于漢中。



 興留心政事,苞容廣納,一言之善,咸見禮異。京兆杜瑾、馮翊吉默、始平周寶等上陳時事,皆擢處美官。天水姜龕、東平淳于岐、馮翊郭高等皆耆儒碩德,經明行修,各門徒數百,教授長安,諸生自遠而至者萬數千人。興每於聽政之暇,引龕等于東堂,講論道藝,錯綜名理。涼州胡辯,苻堅之末,東徙洛陽,講授弟子千有餘人,關中後進多赴之請業。興敕關尉曰:「諸生諮訪道藝,修己厲身,往來
 出入,勿拘常限。」於是學者咸勸,儒風盛焉。給事黃門侍郎古成詵、中書侍郎王尚、尚書郎馬岱等,以文章雅正,參管機密。詵風韻秀舉,確然不群,每以天下是非為己任。時京兆韋高慕阮籍之為人,居母喪,彈琴飲酒。詵聞而泣曰:「吾當私刃斬之,以崇風教。」遂持劍求高。高懼,逃匿,終身不敢見詵。



 興遣將鎮東楊佛嵩攻陷洛陽。



 班命郡國,百姓因荒自賣為奴婢者,悉免為良人。興以日月薄蝕,災眚屢見,降號稱王,下書令群公卿士將牧守宰各降一等。於是其太尉趙公旻等五十三人上疏諫曰:「伏惟陛下勛格皇天,功濟四海,威靈振於殊域,聲教暨
 於遐方,雖成湯之隆殷基,武王之崇周業,未足比喻。方當廓靖江、吳,告成中岳,豈宜過垂沖損,違皇天之眷命乎!」興曰:「殷湯、夏禹德冠百王,然猶順守謙沖,未居崇極,況朕寡昧,安可以處之哉!」乃遣旻告于社稷宗廟,大赦,改元弘始。賜孤獨鰥寡慄帛有差,年七十已上加衣杖。始平太守周班、槐里令李青彡皆以黷貨誅,於是郡國肅然矣。洛陽既陷,自淮、漢已北諸城,多請降送任。



 興下書聽祖父母昆弟得相容隱。姚緒、姚碩德以興降號,固讓王爵,興弗許。



 京兆韋華、譙郡夏侯軌、始平龐眺等率襄陽流人一萬叛晉,奔于興。興引見東堂,謂華曰:「晉自南
 遷,承平已久,今政化風俗何如?」華曰:「晉主雖有南面之尊,無總御之實,宰輔執政,政出多門,權去公家,遂成習俗,刑網峻急,風俗奢宕。自桓溫、謝安已後,未見寬猛之中。」興大悅,拜華中書令。



 興如河東。時姚緒鎮河東,興待以家人之禮。下書封其先朝舊臣姚驢磑、趙惡地、王平、馬萬載、黃世等子為五等子男。命百僚舉殊才異行之士,刑政有不便於時者,皆除之。兵部郎金城邊熙上陳軍令煩苛,宜遵簡約。興覽而善之,乃依孫吳誓眾之法以損益之。興立律學于長安,召郡縣散吏以授之。其通明者還之郡縣,論決刑獄。若州郡縣所不能決者,讞之
 廷尉。興常臨諮議堂聽斷疑獄,于時號無冤滯。



 姚緒、姚碩德固讓王爵,許之。緒、碩德威權日盛,興恐姦佞小人沮惑之,乃簡清正君子為之輔佐。



 興以司隸校尉郭撫、扶風太守強超、長安令魚佩、槐里令彭明、倉部郎王年等清勤貞白,下書褒美,增撫邑一百戶,賜超爵關內侯,佩等進位一級。



 使碩德率隴右諸軍伐乞伏乾歸,興潛軍赴之,乾歸敗走,降其部眾三萬六千,收鎧馬六萬匹。軍無私掠,百姓懷之。興進如枹罕,班賜王公以下,遍于卒伍。



 興之西也,沒奕于密欲乘虛襲安定,長史皇甫序切諫乃止。于自恨失言,陰欲殺序。



 乞伏乾歸以窮蹙來
 降,拜鎮遠將軍、河州刺史、歸義侯,復以其部眾配之。



 興下書,將帥遭大喪,非在疆埸險要之所,皆聽奔赴,及期,乃從王役。臨戎遭喪,聽假百日。若身為邊將,家有大變,交代未至,敢輒去者,以擅去官罪罪之。遣晉將軍劉嵩等二百三十七人歸于建鄴。



 魏人襲沒奕于,于棄其部眾,率數千騎與赫連勃勃奔于秦州。魏軍進次瓦亭,長安大震,諸城閉門固守。魏平陽太守貳塵入侵河東。興於是練兵講武,大閱于城西,於勇壯異者召入殿中。引見群臣于東堂,大議伐魏。群臣咸諫以為不可,興不從。司隸姚顯進曰:「陛下天下之鎮,不宜親行,可使諸將分
 討,授以廟勝之策。」興曰:「王者正以廓土靖亂為務,吾焉得而辭之!」



 興立其子泓為皇太子,大赦境內,賜男子為父後者爵一級。



 遣姚平、狄伯支等率步騎四萬伐魏,姚碩德、姚穆率步騎六萬伐呂隆。平等軍次河東,興遣其光遠黨娥、立節雷星、建忠王多等率杏城及嶺北突騎自和寧赴援,越騎校尉唐小方、積弩姚良國率關中勁卒為平後繼,姚緒統河東見兵為前軍節度,姚紹率洛東之兵,姚詳率朔方見騎,並集平望,以會于興。使沒奕于權鎮上邽,中軍、廣陵公斂權鎮洛陽,姚顯及尚書令姚晃輔其太子泓,入直西宮。



 碩德至姑臧,大敗呂隆之眾,
 俘斬一萬。隆將呂他等率眾二萬五千,以東苑來降。先是,禿髮利鹿孤據西平,沮渠蒙遜據張掖,李玄盛據敦煌,與呂隆相持。至是,皆遣使降。



 興率戎卒四萬七千,自長安赴姚平。平攻魏乾城,陷之,逐據柴壁。魏軍大至,攻平,截汾水以守之。興至蒲阪,憚而不進。



 時碩德攻呂隆,撫納夷夏,分置守宰,節糧積粟,為持久之計。隆懼,遂降。碩德軍令齊整,秋毫無犯,祭先賢,禮儒哲,西土悅之。



 姚平糧竭矢盡,將麾下三十騎赴汾水而死,狄伯支等十將四萬餘人,皆為魏所擒。興下書,軍士戰沒者,皆厚加褒贈。魏軍乘勝進攻蒲阪,姚緒固守不戰,魏乃引還。



 興
 徙河西豪右萬餘戶于長安。



 晉輔國將軍袁虔之、寧朔將軍劉壽、冠軍將軍高長慶、龍驤將軍郭恭等貳于桓玄,懼而奔興。興臨東堂引見,謂虔之等曰:「桓玄雖名晉臣,其實晉賊,其才度定何如父也?能辦成大事以不?」虔之曰:「玄籍世資,雄據荊、楚,屬晉朝失政,遂偷竊宰衡。安忍無親,多忌好殺,位不才授,爵以愛加,無公平之度,不如其父遠矣。今既握朝權,必行篡奪,既非命世之才,正可為他人驅除耳。此天以機便授之陛下,願速加經略,廓清吳、楚。」興大悅,以虔之為大司農,餘皆有拜授。虔之固讓,請疆埸自效,改授假節、寧南將軍、廣州刺史。



 興立
 其昭儀張氏為皇后,封子懿、弼、洸、宣、諶、愔、璞、質、逵、裕、國兒皆為公。遣其兼大鴻臚梁斐,以新平張構為副,拜禿髮傉檀車騎將軍、廣武公,沮渠蒙遜鎮西將軍、沙州刺史、西海侯,李玄盛安西將軍、高昌侯。



 興遣鎮遠趙曜率眾二萬西屯金城,建節王松忿率騎助呂隆等守姑臧。松忿至魏安,為傉檀弟文真所圍,眾潰,執松忿,送于傉檀。傉檀大怒,送松忿還長安,歸罪文真,深自陳謝。



 興下書,錄馬嵬戰時將吏,盡擢敘之,其堡戶給復二十年。



 興性儉約,車馬無金玉之飾,自下化之,莫不敦尚清素。然好游田,頗損農要。京兆杜挻以僕射齊難無匡輔之益,
 著《豐草詩》以箴之,馮翊相雲作《德獵賦》以諷焉。興皆覽而善之,賜以金帛,然終弗能改。



 晉順陽太守彭泉以郡降興,興遣楊佛嵩率騎五千,與其荊州刺史趙曜迎之,遂寇陷南鄉,擒建威將軍劉嵩,略地至于梁國而歸。又遣其兼散騎常侍席確詣涼州,征呂隆弟超入侍,隆遣之。呂隆懼禿髮傉檀之逼,表請內徙。興遣齊難及鎮西姚詰、鎮遠乞伏乾歸、鎮遠趙曜等步騎四萬,迎隆于河西。難至姑臧,以其司馬王尚行涼州刺史,配兵三千鎮姑臧,以將軍閻松為倉松太守,郭將為番禾太守,分戍二城,徙隆及其宗室僚屬于長安。沮渠蒙遜遣弟如子
 貢其方物。王尚綏撫遺黎,導以信義,百姓懷其惠化,翕然歸之。北部鮮卑並遣使貢款。



 桓玄遣使來聘,請辛恭靖、何澹之。興留恭靖而遣澹之,謂曰:「桓玄不推計歷運,將圖篡逆,天未忘晉,必將有義舉,以吾觀之,終當傾覆。卿今馳往,必逢其敗,相見之期,遲不云遠。」初,恭靖至長安,引見興而不拜,興曰:「朕將任卿以東南之事。」靖曰:「我寧為國家鬼,不為羌賊臣。」興怒,幽之別室。至是,恭靖亦踰墻遁歸。



 興遣其將姚碩德、姚斂成、姚壽都等率眾三萬,伐楊盛于仇池。壽都等入自宕昌,斂成從下辯而進。盛遣其弟壽距成,從子斌距都。都逆擊擒之,盡俘其眾。
 楊壽等懼,率眾請降。碩德還師。



 晉汝南太守趙策委守奔于興。



 興如趙逍園,引諸沙門于澄玄堂聽鳩摩羅什演說佛經。羅什通辯夏言,尋覽舊經,多有乘謬,不與胡本相應。興與羅什及沙門僧略、僧遷、道樹、僧睿、道坦、僧肇、曇順等八百餘人,更出大品,羅什持胡本,興執舊經,以相考校,其新文異舊者皆會於理義。續出諸經并諸論三百餘卷。今之新經皆羅什所譯。興既託意於佛道,公卿已下莫不欽附,沙門自遠而至者五千餘人。起浮圖于永貴里,立波若臺于中宮,沙門坐禪者恒有千數。州郡化之,事佛者十室而九矣。



 使姚碩德及冠軍徐洛
 生等伐仇池,又遣建武趙琨自宕昌而進,遣其將斂俱寇漢中。



 時劉裕誅桓玄,迎復安帝,玄衛將軍、新安王桓謙,臨原王桓怡,雍州刺史桓蔚,左衛將軍桓謐,中書令桓胤,將軍何澹之等奔于興。劉裕遣大參軍衡凱之詣姚顯,請通和,顯遣吉默報之,自是聘使不絕。晉求南鄉諸郡,興許之。群臣咸諫以為不可,興曰:「天下之善一也,劉裕拔萃起微,匡輔晉室,吾何惜數郡而不成其美乎!」遂割南鄉、順陽、新野、舞陰等十二郡歸于晉。



 姚碩德等頻敗楊盛,盛懼,請降,遣子難當及僚佐子弟數十人為質,碩德等引還。署盛為使持節、散騎常侍、都督益、寧州
 諸軍事、征南大將軍、開府、益州牧、武都侯。斂俱陷城固,徙漢中流人郭陶等三千餘家於關中。



 興班告境內及在朝文武,立名不得犯叔父緒及碩德之名,以彰殊禮。興謙恭孝友,每見緒及碩德,如家人之禮,整服傾悚,言則稱字,車馬服玩,必先二叔,然後服其次者,朝廷大政,必諮之而後行。



 太史令郭黁言於興曰:「戌亥之歲,當有孤寇起於西北,宜慎其鋒。起兵如流沙,死者如亂麻,戎馬悠悠會隴頭,鮮卑、烏丸居不安,國朝疲於奔命矣。」時所在有泉水涌出,傳云飲則愈病,後多無驗。屢有妖人自稱神女,戮之乃止。



 興大閱,自杜郵至于羊牧。興以姚
 碩德來朝,大赦其境內。及碩德歸于秦州,興送之,及雍乃還。



 禿髮傉檀獻興馬三千匹,羊三萬頭。興以為忠於己,乃署傉檀為涼州刺史,征涼州刺史王尚還長安。涼州人申屠英等二百餘人,遣主簿胡威詣興,請留尚,興弗許。引威見之,威流涕謂興曰:「臣州奉國五年,王威不接,銜膽棲冰、孤城獨守者,仰恃陛下威靈,俯杖良牧惠化。忽違天人之心,以華土資狄。若人辱檀才望應代,臣豈敢言。竊聞乃以臣等貿馬三千匹,羊三萬口,如所傳實者,是為棄人貴畜。茍以馬供軍國,直煩尚書一符,三千餘家戶輸一匹,朝下夕辦,何故以一方委此姦胡!昔漢
 武傾天下之資,開建河西,隔絕諸戎,斷匈奴右臂,所以終能屠大宛王毋寡。今陛下方布政玉門,流化西域,奈何以五郡之地資之犬嚴狁,忠誠華族棄之虐虜!非但臣州里塗炭,懼方為聖朝旰食之憂。」興乃遣西平人車普馳止王尚,又遣使喻傉檀。會傉檀已至姑臧,普以狀先告之。傉檀懼,脅遣王尚,遂入姑臧。



 尚既至長安,坐匿呂氏宮人,擅殺逃人薄禾等,禁止南臺。涼州別駕宗敞、治中張穆、主簿邊憲、胡威等上疏理尚曰:



 臣州荒裔,鄰帶寇仇,居泰無垂拱之安,運否離傾覆之難。自張氏頹基,德風絕而莫扇;呂數將終,梟鶚以之翻翔。群生嬰罔極
 之痛,西夏有焚如之禍。幸皇鑒降眷,純風遠被。刺史王尚受任垂滅之州,策成難全之際,輕身率下,躬儉節用,勞逸豐約,與眾同之,勸課農桑,時無廢業。然後振王威以掃不庭,迴天波以蕩氛穢。則群逆冰摧,不俟朱陽之曜;若秋霜隕籜,豈待勁風之威。何定遠之足高,營平之獨美!經始甫爾,會朝算改授,使希世之功不終於必成,易失之機踐之而莫展。當其時而明其事者,誰不慨然!



 既遠役遐方,劬勞于外,雖效未酬恩,而在公無闕。自至京師,二旬于今,出車之命莫逮,萋斐之責惟深。以取呂氏宮人裴氏及殺逃人薄禾等為南臺所禁,天鑒玄鏡,
 暫免圇圄,譏繩之文,未離簡墨。裴氏年垂知命,首髮二毛,嫠居本家,不在尚室,年邁姿陋,何用送為!邊籓耍捍,眾力是寄,禾等私逃,罪應憲墨,以殺止殺,安邊之義也。假若以不送裴氏為罪者,正闕奚官之一女子耳。論勳則功重,言瑕則過微。而執憲吹毛求疵,忘勞記過,斯先哲所以泣血於當年,微臣所以仰天而灑淚。



 且尚之奉國,歷事二朝,能否效於既往,優劣簡在聖心,就有微過,功足相補,宜弘罔極之施,以彰覆載之恩。



 臣等生自西州,無翰飛之翼,久沈偽政,絕進趣之途。及皇化既沾,投竿之心冥發,遂策名委質,位忝吏端。主辱臣憂,故重繭
 披款,惟陛下亮之。



 興覽之大悅,謂其黃門侍郎姚文祖曰:「卿知宗敞乎?」文祖曰:「與臣州里,西方之英雋。」興曰:「有表理王尚,文義甚佳,當王尚研思耳。」文祖曰:「尚在南臺,禁止不與賓客交通,敞寓於楊桓,非尚明矣。」興曰:「若爾,桓為措思乎?」文祖曰:「西方評敞甚重,優於楊桓。敞昔與呂超周旋,陛下試可問之。」興因謂超曰:「宗敞文才何如?可是誰輩?」超曰:「敞在西土,時論甚美,方敞魏之陳、徐,晉之潘、陸。」即以表示超曰:「涼州小地,寧有此才乎?」超曰:「臣以敞餘文比之,未足稱多。琳瑯出于崑嶺,明珠生于海濱,若必以地求人,則文命大夏之棄夫,姬昌東夷之擯
 士。但當問其文彩何如,不可以區宇格物。」興悅,赦尚之罪,以為尚書。



\end{pinyinscope}