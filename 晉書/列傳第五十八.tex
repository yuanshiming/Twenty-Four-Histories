\article{列傳第五十八}

\begin{pinyinscope}

 孝友



 李密盛彥夏方王裒許孜庾袞孫晷顏含劉殷王延王談桑虞何琦吳逵



 大矣哉,孝之為德也!分渾元而立體,道貫三靈;資品匯以順名,功苞萬象。用之于國,動天地而降休徵;行之于家,感鬼神而昭景福。若乃博施備物,尊仁安義,柔色承顏,怡怡盡樂,擊鮮就養,亹癖忘劬,集包思藝黍之勤,循陔有採蘭之詠,事親之道也。屬屬如在,哀哀罔極,聚薪流慟,銜索興嗟,曬風樹以隤心,頫寒泉而沫泣,追遠之
 情也。審德筮仕,正務移官,居高匪危,在醜無爭,協修升以匡化,懷履冰而砥節,立身之行也。是以閔曾翼翼,遵六教而緝貞規;蔡董烝烝,弘七體而垂令迹。亦有至誠上感,明祗下贊,郭巨致錫金之慶。陽雍標蒔玉之祉;烏馴丹羽,巢叔和之室,鹿呈白毳,擾功文之廬。然則因彼孝慈而生友悌,理在兼綜,義歸一揆。夫天倫之重,共氣分形,心睽則葉悴荊權,性合則華承棣萼。乃有推代瘦,徇急難之情;讓果同衾,盡懽愉之致:緬窺緗素,載流塵躅者歟!



 晉氏始自中朝,逮于江左,雖百六之災遄及,而君子之道未消,孝悌名流,猶為繼踵。王偉元之行己,
 許季義之立節,夏方、盛彥體至性以馳芬,庾袞、顏含篤友于而宣範,自餘群士,咸標懿德。採其遺絢,足厲澆風,故著《孝友篇》以續前史云耳。



 李密,字令伯,犍為武陽人也,一名虔。父早亡,母何氏醮。密時年數歲,感戀彌至,烝烝之性,遂以成疾。祖母劉氏,躬自撫養,密奉事以孝謹聞。劉氏有疾,則涕泣側息,未嘗解衣,飲膳湯藥必先嘗後進。有暇則講學忘疲,而師事譙周,周門人方之游夏。



 少仕蜀,為郎。數使吳,有才辯,吳人稱之。蜀平,泰始初,詔徵為太子洗馬。密以祖母
 年高,無人奉養,遂不應命。乃上疏曰:



 臣以險釁,夙遭閔凶,生孩六月,慈父見背,行年四歲,舅奪母志。祖母劉愍臣孤弱,躬親撫養。臣少多疾病,九歲不行,零丁辛苦,至于成立。既無伯叔,終鮮兄弟,門衰祚薄,晚有兒息。外無期功彊近之親,內無應門五尺之童,煢煢孑立,形影相弔。而劉早嬰疾病,常在床蓐。臣侍湯藥,未嘗廢離。



 自奉聖朝,沐浴清化,前太守臣逵,察臣孝廉,後刺史臣榮,舉臣秀才。臣以供養無主,辭不赴命。明詔特下,拜臣郎中,尋蒙國恩,除臣洗馬。猥以微賤,當侍東宮,非臣隕首所能上報。臣具以表聞,辭不就職。詔書切峻,責臣逋慢,郡
 縣逼迫,催臣上道,州司臨門,急於星火。臣欲奉詔奔馳,則劉病日篤;茍徇私情,則告訴不許。臣之進退,實為狼狽。



 伏惟聖朝以孝治天下,凡在故老,猶蒙矜恤,況臣孤苦尪羸之極。且臣少仕偽朝,歷職郎署,本圖宦達,不矜名節。今臣亡國賤俘,至微至陋,猥蒙拔擢,寵命殊私,豈敢盤桓,有所希冀!但以劉日薄西山,氣息奄奄,人命危淺,朝不慮夕。臣無祖母,無以至今日;祖母無臣,無以終餘年。母孫二人,更相為命,是以私情區區不敢棄遠。臣密今年四十有四,祖母劉今年九十有六,是臣盡節於陛下之日長,而報養劉之日短也。烏鳥私情,願乞終養。



 臣之辛苦,非但蜀之人士及二州牧伯之所明知,皇天后土,實所鑒見。伏願陛下矜愍愚誠,聽臣微志,庶劉僥倖,保卒餘年。臣生當隕身,死當結草。



 帝覽之曰:「士之有名,不虛然哉!」乃停召。後劉終,服闋,復以洗馬徵至洛。司空張華問之曰:「安樂公何如?」密曰:「可次齊桓。」華問其故,對曰:「齊桓得管仲而霸,用豎刁而蟲流。安樂公得諸葛亮而抗魏,任黃皓而喪國,是知成敗一也。」次問:「孔明言教何碎?」密曰:「昔舜、禹、皋陶相與語,故得簡雅;《大誥》與凡人言,宜碎。孔明與言者無己敵,言教是以碎耳。」華善之。



 出為溫令,而憎疾從事,嘗與人書曰:「慶父不死,魯難未
 已。」從事白其書司隸,司隸以密在縣清慎,弗之劾也。密有才能,常望內轉,而朝廷無援,乃遷漢中太守,自以失分懷怨。及賜餞東堂,詔密令賦詩,末章曰:「人亦有言,有因有緣。官無中人,不如歸田。明明在上,斯語豈然!」武帝忿之,於是都官從事奏免密官。後卒於家。二子:賜、興。



 賜字宗石,少能屬文,嘗為《玄鳥賦》,詞甚美。州辟別駕,舉秀才,未行而終。興字雋石,亦有文才,刺史羅尚辟別駕。尚為李雄所攻,使興詣鎮南將軍劉弘求救,興因願留,為弘參軍而不還。尚白弘,弘即奪其手版而遣之。興之在弘府,弘立諸葛孔明、羊叔子碣,使興俱為之文,甚有辭理。



 盛彥,字翁子,廣陵人也。少有異才。年八歲,詣吳太尉戴昌,昌贈詩以觀之,彥於坐答之。辭甚康慨。母王氏因疾失明,彥每言及,未嘗不流涕。於是不應辟召,躬自侍養,母食必自哺之。母既疾久,至於婢使數見捶撻。婢忿恨,伺彥暫行,取蠐螬灸飴之。母食以為美,然疑是異物,密藏以示彥。彥見之,抱母慟哭,絕而復蘇。母目豁然即開,從此遂愈。彥仕吳,至中書侍郎,吳平,陸雲薦之於刺史周浚,本邑大中正劉頌又舉彥為小中正。太康中卒。



 夏方,字文正,會稽永興人也。家遭疫癘,父母伯叔群從死者十三人。方年十四,夜則號哭,晝則負土,十有七載,葬送得畢,因廬于墓側,種植松柏,烏鳥猛獸馴擾其旁。吳時拜仁義都尉,累遷五官中郎將。朝會未嘗乘車,行必讓路。吳平,除高山令。百姓有罪應加捶撻者,方向之涕泣而不加罪,大小莫敢犯焉。在官三年,州舉秀才,還家,卒,年八十七。



 王裒,字偉元,城陽營陵人也。祖修,有名魏世。父儀,高亮雅直,為文帝司馬。東關之役,帝問於眾曰:「近曰之事,誰
 任其咎?」儀對曰:「責在元帥。」帝怒曰:「司馬欲委罪於孤邪!」遂引出斬之。



 裒少立操尚,行己以禮,身長八尺四寸,容貌絕異,音聲清亮,辭氣雅正,博學多能,痛父非命,未嘗西向而坐。示不臣朝廷也。於是隱居教授,三徵七辟皆不就。廬于墓側,旦夕常至墓所拜跪,攀柏悲號,涕淚著樹,樹為之枯。母性畏雷,母沒,每雷,輒到墓曰:「裒在此。」及讀《詩》至「哀哀父母,生我劬勞」,未嘗不三復流涕,門人受業者並廢《蓼莪》之篇。



 家貧,躬耕,計口而田,度身而蠶。或有助之者,不聽。諸生密為刈麥,裒遂棄之。知舊有致遺者,皆不受。門人為本縣所役,告裒求屬令,良曰:「卿學不
 足以庇身,吾德薄不足以蔭卿,屬之何益!且吾不執筆已四十年矣。」乃步擔乾飯,兒負鹽豉草屐,送所役生到縣,門徒隨從者千餘人。安丘令以為詣己,整衣出迎之。裒乃下道至土牛旁,磬折而立,云:「門生為縣所役,故來送別。」因執手涕泣而去。令即放之,一縣以為恥。



 鄉人管彥少有才而未知名,裒獨以為必當自達,拔而友之,男女各始生,便共許為婚。彥後為西夷校尉,卒而葬於洛陽,裒後更嫁其女。彥弟馥問裒,裒曰:「吾薄志畢願山藪,昔嫁姊妹皆遠,吉凶斷絕,每以此自誓。今賢兄子葬父子洛陽。此則京邑之人也,由吾結好之本意哉!」馥曰:「嫂,
 齊人也,當還臨淄。」裒曰:「安有葬父河南而隨母還齊!用意如此,何婚之有!」



 北海邴春少立志操,寒苦自居,負笈游學,鄉邑僉以為邴原復出。裒以春性險狹慕名,終必不成。其後春果無行,學業不終,有識以此歸之。裒常以為人之所行期於當歸善道,何必以所能而責人所不能。



 及洛京傾覆,寇資蜂起,親族悉欲移渡江東,裒戀墳壟不去。賊大盛,方行,猶思慕不能進,遂為賊所害。



 許孜,字季義,東陽吳寧人也。孝友恭讓,敏而好學。年二十,師事豫章太守會稽孔沖,受《詩》、《書》、《禮》、《易》及《孝經》、《論語》。
 學竟,還鄉里。沖在郡喪亡,孜聞問盡哀,負擔奔赴,送喪還會稽,蔬食執役,制服三年。俄而二親沒,柴毀骨立,杖而能起,建墓於縣之東山,躬自負土,不受鄉人之助。或愍孜羸憊,苦求來助,孜晝助不逆,夜便除之。每一悲號,鳥獸翔集。孜以方營大功,乃棄其妻,鎮宿墓所,列植松柏亙五六里。時有鹿犯其松栽,改悲歎曰:「鹿獨不念我乎!」明日,忽見鹿為猛獸所殺,置於所犯栽下。孜悵惋不已,乃為作塚,埋於隧側。猛獸即於孜前自撲而死,孜益嘆息,又取埋之。自後樹木滋茂,而無犯者。積二十餘年孜乃更娶妻,立宅墓次,烝烝朝夕,奉亡如存,鷹雉棲其
 梁,簷鹿與猛獸擾其庭圃,交頸同游,不相搏噬。元康中,郡察孝廉,不起,巾褐終身。年八十餘,卒于家。邑人號其居為孝順里。



 咸康中,太守張虞上疏曰:「臣聞聖賢明訓存乎舉善,褒貶所興,不遠千載。謹案所領吳寧縣物故人許孜,至性孝友,立節清峻,與物恭讓,言行不貳。當其奉師,則在三之義盡;及其喪親,實古今之所難。咸稱殊類致感,猛獸弭害。雖臣不及見,然備聞斯語,竊謂蔡順、董黯無以過之。孜沒積年,其子尚在,性行純愨,今亦家於墓側。臣以為孜之履操,世所希逮,宜標其令跡,甄其後嗣,以酬既往,以獎方來。《陽秋傳》曰:『善善及其子孫』。臣
 不達大體,請臺量議。」疏奏,詔旌表門閭。蠲復子孫。其子生亦有孝行。圖孜像於堂,朝夕拜焉。



 庾袞,字叔褒,明穆皇后伯父也。少履勤儉,篤學好問,事親以孝稱。咸寧中,大疫,二兄俱亡,次兄毗復殆,癘氣方熾,父母諸弟皆出次於外,袞獨留不去。諸父兄強之,乃曰:「袞性不畏病。」遂親自扶持,晝夜不眠,其間復撫柩哀臨不輟。如此十有餘旬,疫勢既歇,家人乃反,毗病得差,袞亦無恙。父老咸曰:「異哉此子!守人所不能守,行人所不能行,歲寒然後知松柏之後凋,始疑疫癘之不相染
 也。」



 初,袞諸父並貴盛,惟父獨守貧約。袞躬親稼穡,以給供養,而執事勤恪,與弟子樹籬,跪以授條。或曰:「今在隱屏,先生何恭之過?」袞曰:「幽顯易操,非君子之志也。」父亡,作筥賣以養母。母見其勤,曰:「我無所食。」對曰:「母食不甘,袞將何居!」母感而安之。袞前妻荀氏,繼妻樂氏,皆官族富室,及適袞,俱棄華麗,散資財,與袞共安貧苦,相敬如賓。母終,服喪居於墓側。



 歲大饑,藜羹不糝,門人欲進其飯者,而袞每曰已食,莫敢為設。及麥熟,獲者已畢,而採捃尚多,袞乃引其群子以退,曰「待其間。」及其捃也,不曲行,不旁掇,跪而把之,則亦大獲,又與邑人入山拾橡,分
 夷險,序長幼,推易居難,禮無違者。或有斬其墓柏,莫知其誰,乃召鄰人集于墓而自責焉,因叩頭泣涕,謝祖禰曰:「德之不修,不能庇先人之樹,袞之罪也。」父老咸亦為之垂泣,自後人莫之犯。撫諸孤以慈,奉諸寡以仁,事加於厚而教之義方,使長者體其行,幼者忘其孤。孤甥郭秀,比諸子姪,衣食而每先之。孤兄女曰芳,將嫁,美服既具,袞乃刈荊苕為箕帚,召諸子集之于堂,男女以班,命芳曰:「芳乎!汝少孤,汝逸汝豫,不汝疵瑕。今汝適人,將事舅姑,灑掃庭內,婦之道也,故賜汝此。匪器之為美,欲溫恭朝夕,雖休勿休也。」而以舊宅與其長兄子賡、翕。及翕
 卒,袞哀其早孤,痛其成人而未娶,乃撫柩長號,哀感行路,聞者莫不垂涕。



 初,袞父誡袞以酒,每醉,輒自責曰:「餘廢先父之誡,其何以訓人!」乃於父墓前自杖三十。鄰人褚德逸者,善事其親,老而不倦,袞每拜之。嘗與諸兄過邑人陳準兄弟,諸兄友之,皆拜其母,袞獨不拜。準弟徽曰:「子不拜吾親何?」袞曰:「未知所以拜也。夫拜人之親者,將自同於人之子也,其義至重,袞敢輕之乎?」遂不拜。準、徽歎曰:「古有亮直之士,君近之矣。君若當朝,則社稷之臣歟!君若握兵,臨大節,孰能奪之!方今徵聘,君實宜之。」於是鄉黨薦之,州郡交命,察孝廉,舉秀才、清白異行,皆
 不降志,世遂號之為異行。



 元康末,潁川太守召為功曹,袞服造役之衣,杖鍤荷斧,不俟駕而行,曰:「請受下夫之役。」太守飾車而迎,袞逡巡辭退,請徒行入郡,將命者遂逼扶升車,納於功曹舍。既而袞自取己車而寢處焉,形雖恭而神有不可動之色。太守知其不屈,乃歎曰:「非常士也,吾何以降之!」厚為之禮而遣焉。



 齊王冏之唱義也,張泓等肆掠於陽翟,袞乃率其同族及庶姓保于禹山。是時百姓安寧,未知戰守之事,袞曰:「孔子云:『不教而戰,是謂棄之。』」乃集諸如士而謀曰:「二三君子相與處於險,將以安保親尊,全妻孥也。古人有言:『千人聚而不以一
 人為主,不散則亂矣。』將若之何!」眾曰:「善。今日之主,非君而誰!」袞默然有間,乃言曰:「古人急病讓夷,不敢逃難,然人之立主,貴從其命也。」乃誓之曰:「無恃險,無怙亂,無暴鄰,無抽屋,無樵採人所植,無謀非德,無犯非義,戮力一心,同恤危難。」眾咸從之。於是峻險阨,杜蹊徑,修壁塢,樹蕃障,考功庸,計丈尺,均勞逸,通有無,繕完器備,量力任能,物應其宜,使邑推其長,里推其賢,而身率之。分數既明,號令不二,上下有禮,少長有儀,將順其美,匡救其惡。及賊至,袞乃勒部曲,整行伍,皆持滿而勿發。賊挑戰,晏然不動,且辭焉。賊服其慎而畏其整,是以皆退,如是者
 三。時人語曰:「所謂臨事而懼、好謀而成者,其庾異行乎!」



 及冏歸于京師,踰年不朝,袞曰:「晉室卑矣,寇難方興!」乃攜其妻適林慮山,事其新鄉如其故鄉,言忠信,行篤敬。經及期年,而林慮之人歸之,咸曰庾賢。及石勒攻林慮,父老謀曰:「此有大頭山,九州之絕險也。上有古人遺迹,可共保之。」惠帝遷于長安,袞乃相與登于大頭山而田於其下。年穀未熟,食木實,餌石蕊,同保安之,有終焉之志。及將收獲,命子怞與之下山,中途目眩瞀,墜崖而卒。同保赴哭曰:「天乎!獨不可舍我賢乎!」時人傷之曰:「庾賢絕塵避地,超然遠迹,固窮安陋,木食山棲,不與世同
 榮,不與人爭利,不免遭命,悲夫!」



 袞學通《詩》《書》,非法不言,非道不行,尊事耆老,惠訓蒙幼,臨人之喪必盡哀,會人之葬必躬築,勞則先之,逸則後之,言必行之,行必安之。是以宗族鄉黨莫不崇仰,門人感慕,為人樹碑焉。



 有四子:怞、蔑、澤、捃。在澤生,故名澤,因捃生,故曰捃。蔑後南渡江,中興初,為侍中。蔑生願,安成太守。



 孫晷,字文度,吳國富春人,吳伏波將軍秀之曾孫也。晷為兒童,未嘗被呵怒。顧榮見而稱之,謂其外祖薛兼曰:「此兒神明清審,志氣貞立,非常童也。」及長,恭孝清約,學
 識有理義,每獨處幽闇之中,容止瞻望未嘗傾邪。雖侯家豐厚,而晷常布衣蔬食,躬親壟畝,誦詠不廢,欣然獨得。父母愍其如此,欲加優饒,而夙興夜寐,無暫懈也。父母起居嘗饌,雖諸兄親饋,而晷不離左右。富春車道既少,動經江川,父難於風波,每行乘籃輿,晷躬自扶侍,所詣之處,則於門外樹下籓屏之間隱息,初不令主人知之。兄嘗篤疾經年,晷躬自扶侍,藥石甘苦,必經心目,跋涉山水,祈求懇至。而聞人之善,欣若有得;聞人之惡,慘若有失。見人饑寒,並周贍之,鄉里贈遺,一無所受。親故有窮老者數人,恒往來告索,人多厭慢之,而晷見之。欣
 敬逾甚,寒則與同衾,食則與同器,或解衣推被以恤之。時年饑穀貴,人有生刈其稻者,晷見而避之,須去而出,既而自刈送與之。鄉鄰感愧,莫敢侵犯。



 會稽虞喜隱居海嵎,有高世之風。晷欽其德,聘喜弟預女為妻。喜戒女棄華尚素,與晷同志。時人號為梁鴻夫婦。濟陽江淳少有高操,聞晷學行過人,自東陽往候之,始面,便終日譚宴,結歡而別。



 司空何充為揚州,檄晷為主簿,司徒蔡謨辟為掾屬,並不就。尚書經國明,州土之望,表薦晷,公車徑征。會卒,時年三十八,朝野嗟痛之。晷未及大斂,有一老父縕袍草屨,不通姓名,徑入撫柩而哭,哀聲慷慨,感
 於左右。哭止便出,容貌甚清,眼瞳又方,門者告之喪主,怪而追焉。直去不顧。同郡顧和等百餘人歎其神貌有異,而莫之測也。



 顏含,字弘都,瑯邪莘人也。祖欽,給事中。父默,汝陰太守。含少有操行,以孝聞。兄畿,咸寧中得疾,就醫自療,遂死於醫家。家人迎喪,旐每繞樹而不可解,引喪者顛仆,稱畿言曰:「我壽命未死,但服藥太多,傷我五藏耳。今當復活,慎無葬也。」其父祝之曰:「若爾有命復生,豈非骨肉所願!今但欲還家,不爾葬也。」旐乃解。及還,其婦夢之曰:「吾
 當復生,可急開棺。」婦頗說之。其夕,母及家人又夢之,即欲開棺,而父不聽。含時尚少,乃慨然曰:「非常之事,古則有之,今靈異至此,開棺之痛,孰與不開相負?」父母從之,乃共發棺果有生驗,以手刮棺,指爪盡傷,然氣息甚微,存亡不分矣。飲哺將護,累月猶不能語,飲食所須,託之以夢。闔家營視,頓廢生業,雖在母妻,不能無倦矣。含乃絕棄人事,躬親侍養,足不出戶者十有三年。石崇重含淳行,贈以甘旨,含謝而不受。或問其故,答曰:「病者綿昧,生理未全,既不能進敢,又未識人惠,若當謬留,豈施者之意也!」畿竟不起。



 含二親既終,兩兄繼沒,次嫂樊氏因
 疾失明,含課勵家人,盡心奉養,每日自嘗省藥饌,察問息耗,必簪屨束帶。醫人疏方,應須髯蛇膽,而尋求備至,無由得之,含憂歎累時。嘗晝獨坐,忽有一青衣童子年可十三四,持一青囊授含,含開視,乃蛇膽也。童子逡巡出戶,化成青鳥飛去。得膽,藥成,嫂病即愈。由是著名。



 本州辟,不就。東海王趙以為太傅參軍,出補闓陽令。元帝初鎮下邳,復命為參軍。過江,以含為上虞令。轉王國郎中、丞相東閣祭酒,出為東陽太守。東宮初建,含以儒素篤行補太子中庶子,遷黃門侍郎、本州大中正,歷散騎常侍、大司農。豫討蘇峻功,封西平縣侯,拜侍中,除吳郡
 太守。王導問含曰:「卿今蒞名郡,政將何先?」答曰:「王師歲動,編戶虛耗,南北權豪競招游食,國弊家豐,執事之憂。且當征之勢門,使反田桑,數年之間,欲令戶給人足,如其禮樂,俟之明宰。」含所歷簡而有恩,明而能斷,然以威御下。導歎曰:「顏公在事,吳人斂手矣。」未之官,復為侍中。尋除國子祭酒,加散騎常侍,遷光祿勳,以年老遜位。成帝美其素行,就加右光祿大夫,門施行馬,賜床帳被褥,敕太官四時致膳,固辭不受。



 于時論者以王導帝之師傅,名位隆重,百僚宜為降禮。太常馮懷以問於含,含曰:「王公雖重,理無偏敬,降禮之言,或是諸君事宜。鄙人老
 矣,不識時務。」既而告人曰:「吾聞伐國不問仁人。向馮祖思問佞於我,我有邪德乎?」人嘗論少正卯、盜跖其惡孰深。或曰:「正卯雖姦,不至剖人棄膳,盜跖為甚。」含曰:「為惡彰露,人思加戮;隱伏之姦,非聖不誅。由此言之,少正為甚。」眾咸服焉。郭璞嘗遇含,欲為之筮。含曰:「年在天,位在人,修己而天不與者,命也;守道而人不知者,性也。自有性命,無勞蓍龜。」桓溫求婚於含,含以其盛滿,不許。惟與鄧攸深交。或問江左群士優劣,答曰:「周伯仁之正,鄧伯道之清,卞望之之節,餘則吾不知也。」其雅重行實,抑絕浮偽如此。



 致仕二十餘年,年九十三卒。遺命素棺薄斂。
 謚曰靖。喪在殯而鄰家失火,移棺紼斷,火將至而滅,僉以為淳誠所感也。



 三子:髦、謙、約。髦歷黃門郎、侍中、光祿勳,謙至安成太守,約零陵太守,並有聲譽。



 劉殷,字長盛,新興人也。高祖陵,漢光祿大夫。殷七歲喪父,哀毀過禮,服喪三年,未曾見齒。曾祖母王氏,盛冬思堇而不言,食不飽者一旬矣。殷怪而問之,王言其故。殷時年九歲,乃於澤中慟哭,曰:「殷罪釁深重,幼丁艱罰,王母在堂,無旬月之養。殷為人子,而所思無獲,皇天后土,願垂哀愍。」聲不絕者半日,於是忽若有人云:「止,止聲。」殷
 收淚視地,便有堇生焉,因得斛餘而歸,食而不減,至時,堇生乃盡。又嘗夜夢人謂之曰:「西籬下有粟。」寤而掘之,得粟十五鐘,銘曰「七年粟百石,以賜孝子劉殷。」自是食之,七載方盡。時人嘉其至性通感,競以穀帛遺之。殷受而不謝,直云待後貴當相酬耳。



 弱冠,博通經史,綜核群言,文章詩賦靡不該覽,性倜儻,有濟世之志,儉而不陋,清而不介,望之頹然而不可侵也。鄉黨親族莫不稱之。郡命主簿,州辟從事,皆以供養無主,辭不赴命。司空、齊王攸辟為掾,征南將軍羊祜召參軍事,皆以疾辭。同郡張宣子,識達之士也,勸殷就徵。殷曰:「當今二公,有晉之
 棟楹也。吾方希達如榱椽耳,不憑之,豈能立乎!吾今王母在堂,既應他命,無容不竭盡臣禮,使不得就養。子輿所以辭齊大夫,良以色養無主故耳。」宣子曰:「如子所言,豈庸人所識哉!而今而後,吾子當為吾師矣。」遂以女妻之。宣子者,并州豪族也,家富於財,其妻怒曰:「我女年始十四。姿識如此,何慮不得為公侯妃,而遽以妻劉殷乎!」宣子曰:「非爾所及也。」誡其女曰:「劉殷至孝冥感,兼才識超世,此人終當遠達,為世名公,汝其謹事之。」張氏性亦婉順,事王母以孝聞,奉殷如君父焉。及王氏卒,殷夫婦毀瘠,幾至滅性,時柩在殯而西鄰失火,風勢甚盛,殷夫
 婦叩殯號哭,火遂越燒東家。後有二白鳩巢其庭樹,自是名譽彌顯。



 太傅楊駿輔政,備禮聘殷,殷以母老固辭。駿於是表之,優詔遂其高志,聽終色養,敕所在供其衣食,蠲其徭賦,賜帛二百匹,穀五百斛。趙王倫纂位,孫秀重殷名,以散騎常侍征之,殷逃奔鴈門。及齊王冏輔政,辟為大司馬軍諮祭酒。既至,謂殷曰;「先王虛心召君,君不至。今孤辟君,君何能屈也?」殷曰:「世祖以大聖應期,先王以至德輔世,既堯舜為君,稷契為佐,故殷希以一夫而距千乘,為不可迴之圖,幸邀唐虞之世,是以不懼斧鉞之戮耳。今殿下以神武睿姿,除殘反政,然聖跡稍
 粗,嚴威滋肅,殷若復爾,恐招華士之誅,故不敢不至也。」冏奇之,轉拜新興太守,明刑旌善,甚有政能。



 屬永嘉之亂,沒於劉聰。聰奇其才而擢任之,累至侍中、太保、錄尚書事。殷恒戒子孫曰:「事君之法,當務幾諫,凡人尚不可面斥其過,而況萬乘乎!夫犯顏之禍,將彰君過,宜上思召公咨商之義,下念鮑勛觸鱗之誅也。」在聰之朝,與公卿恂恂然,常有後己之色。士不修操行者,無得入其門,然滯理不申,藉殷而濟者,亦已百數。



 有七子,五子各授一經。一子授《太史公》,一子授《漢書》,一門之內,七業俱興,北州之學,殷門為盛。竟以壽終。



 王延,字延元。西河人也。九歲喪母,泣血三年,幾至滅性。每至忌日,則悲啼至旬。繼母卜氏遇之無道,恒以薄穰及敗麻頭與延貯衣。其姑聞而問之,延知而不言,事母彌謹。卜氏嘗盛冬思生魚,敕延求而不獲,杖之流血。延尋汾叩凌而哭,忽有一魚長五尺,踴出水上,延取以進母。卜氏食之,積日不盡,於是心悟,撫延如己生。延事親色養,夏則扇枕席,冬則以身溫被,隆立盛寒,體無全衣,而親極滋味。晝則傭賃,夜則誦書,遂究覽經史,皆通大義。州郡禮辟,貪供養不起。父母終後,廬於墓側,非其蠶
 不衣,非其耕不食。屬天下喪亂,隨劉元海遷于平陽,農蠶之暇,訓誘宗族,侃侃不倦。家牛一生犢,他人認之,延牽而授與,初無吝色。其人後自知妄認,送犢還延,叩頭謝罪,延仍以與之,不復取也。年六十,方仕於劉聰,稍遷尚書左丞,至金紫光祿大夫。聰死後,靳準將作亂,謀之于延,延不從。準既誅劉氏,自號漢天王,以延為左光祿大夫,延又大罵不受,準遂殺之。



 王談,吳興烏程人也。年十歲,父為鄰人竇度所殺。談陰有復仇志,而懼為度所疑,寸刃不畜,日夜伺度,未得。至
 年十八,乃密市利鍤,陽若耕鉏者。度常乘船出入,經一橋下,談伺度行還,伏草中,度既過,談於橋上以鍤斬之,應手而死。既而歸罪有司,太守孔巖義其孝勇,列上宥之。巖諸子為孫恩所害,無嗣,談乃移居會稽,修理巖父子墳墓,盡其心力。後太守孔廞究其義行,元興三年,舉談為孝廉,時稱其得人。談不應召,終于家。



 桑虞,字子深,魏郡黎陽人也。父沖,有深識遠量,惠帝時為黃門郎。河間王顒執權,引為司馬。沖知顒必敗,就職一旬,便稱疾求退。虞仁孝自天至,年十四喪父,毀瘠過
 禮,日以米百粒用糝藜藿,其姊諭之曰:「汝毀瘠如此,必至滅性,滅性不孝,宜自抑割。」虞曰:「藜藿雜米,足以勝哀。」虞有園在宅北數里,瓜果初熟,有人踰垣盜之。虞以園援多棘刺,恐偷見人驚走而致傷損,乃使奴為之開道。及偷負瓜將出,見道通利,知虞使除之,乃送所盜瓜,叩頭請罪。虞乃歡然,盡以瓜與之。嘗行,寄宿逆旅,同宿客失脯,疑虞為盜。虞默然無言,便解衣償之。主人曰:「此舍數失魚肉雞鴨,多是狐狸偷去,君何以疑人?」乃將脯主至山冢間尋求,果得之。客求還衣,虞投之不顧。



 虞諸兄仕於石勒之世,咸登顯位,惟虞恥臣非類,陰欲避地海
 東,會丁母憂,遂止。哀毀骨立,廬於墓側。五年後,石勒以為武城令。虞以密邇黃河,去海微近,將申前志,欣然就職。石季龍太守劉徵甚器重之,徵遷青州刺史,請虞長史,帶祝阿郡。徵遇疾還鄴,令虞監行州府屬。季龍死,國中大亂,朝廷以虞名父之子,必能立功海岱,潛遣東莞人華挺授虞寧朔將軍、青州刺史。虞曰:「功名非吾志也。」乃附使者啟,讓刺史,靖居海右,不交境外。雖歷偽朝,而不豫亂,世以此高之。卒於官。



 何
 琦,字萬倫,司空充之從兄也。祖父龕,後將軍。父阜,淮南內史。琦年十四喪父,哀毀過禮。性沈敏有識度,好古博學,居于宣城陽穀縣,事母孜,朝夕色養。常患甘鮮不贍,乃為郡主簿,察孝廉,除郎中,以選補宣城涇縣令。司徒王導引為參軍,不就。及丁母憂,居喪泣血,杖而後起,停柩在殯,為鄰火所逼,煙焰已交,家乏僮使,計無從出,乃匍匐撫棺號哭。俄而風止火息,堂屋一間免燒,其精誠所感如此。服闋,乃慨然歎曰:「所以出身仕者,非謂有尺寸之能以效智力,實利微祿,私展供養。一旦煢然,
 無復恃怙,豈可復以朽鈍之質塵默清朝哉!」於是養志衡門,不交人事,耽玩典籍,以琴書自娛。不營產業,節儉寡欲,豐約與鄉鄰共之。鄉里遭亂,姊沒人家,琦惟有一婢,便為購贖。然不為小謙,凡有贈遺,亦不茍讓,但於己有餘,輒復隨而散之。任心而行,率意而動,不占卜,無所事。司空陸玩、太尉桓溫並辟命,皆不就。詔徵博士,又不起。簡文帝時為撫軍,欽其名行,召為參軍,固辭以疾。公車再徵通直散騎侍郎、散騎常侍,不行。由是君子仰德,莫能屈也。桓溫嘗登琦縣界山,喟然嘆曰:「此山南有人焉,何公真止足者也!」琦善養性,老而不衰,布褐蔬食,恒
 以述作為事,著《三國評論》,凡所撰錄百許篇,皆行於世。年八十二卒。



 吳逵,吳興人也。經荒饑疾病,合門死者十有三人,逵時亦病篤,其喪皆鄰里以葦席裹而埋之。逵夫妻既存,家極貧窘,冬無衣被,晝則傭賃,夜燒磚甓,晝夜在山,未嘗休止,遇毒蟲猛獸,輒為之下道。期年,成七墓、十三棺。時有賻贈,一無所受。太守張崇義之,以羔雁之禮禮焉。卒於家。



 史臣曰:尊親之道,禮經之明訓;孝友之義,詩人之美談,
 是知人倫之本,罔茲攸尚。盛翁子立行淳至,素蓄異才,流慟致其感通,含哺申其就養,戴昌賞其清韻,陸雲嘉其茂德。王裒隱居不從其辟,行己莫逾其禮,枯柏以應其誠,驚雷以危其慮。永言董蔡,異時均美。許孜少而敏學,禮備在三,馴雉棲其梁棟,猛獸擾其庭圃,居喪之禮,實古今之所難焉。庾叔褒不匱表於執勤,則裕存乎敬業,幽顯不易其操,疫癘不駭其心,急病讓夷之規,有古人之風烈矣。孫晷之匪懈,王談之復仇,神人惜其亡,良守宥其罪。劉殷幼丁艱酷,柴毀逾制,發三冬之堇,賜七年之粟,至誠之契,義形于茲。王延叩冰而召鱗,扇席而
 清暑,雖黃香、孟宗,抑為倫輩。其餘群子,並孝養可崇,清風素範,高山景行,會其宗流,同斯志也。



 贊曰:德之所屆,有感和征。孝哉王許,永慕烝烝。揮泗凋柏,對榥巢鷹。密、彥、夏、庾,夙標至性。文度、弘都,勤修懿行。敦彼孝友,載光謠詠。鳩馴長盛,魚薦延元。談桑義闡,琦吳道存。專洞之德,咸摛左言。



\end{pinyinscope}