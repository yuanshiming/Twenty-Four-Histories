\article{載記第二十一 李雄李班李期李壽李勢}

\begin{pinyinscope}

 李
 雄李班李期李壽李勢



 李雄,字仲俊,特第三子也。母羅氏,夢雙虹自門升天,一虹中斷,既而生蕩。後羅氏因汲水,忽然如寐,又夢大蛇繞其身,遂有孕,十四月而生雄。常言吾二子若有先亡,在者必大貴。蕩竟前死。雄身長八尺三寸,美容貌。少以烈氣聞,每周旋鄉里,識達之士皆器重之。有劉化者,道術士也,每謂人曰:「關、隴之士皆當南移,李氏子中惟仲
 俊有奇表,終為人主。」



 特起兵於蜀,承制,以雄為前將軍。流死,雄自稱大都督、大將軍、益州牧,都於郫城。羅尚遣將攻雄,雄擊走之。李驤攻犍為,斷尚運道,尚軍大餒,攻之又急,遂留牙門羅特固守,尚委城夜遁。特開門內雄,遂剋成都。於時雄軍飢甚,乃率眾就穀於郪,掘野芋而食之。蜀人流散,東下江陽,南入七郡。雄以西山范長生巖居穴處,求道養志,欲迎立為君而臣之。長生固辭。雄乃深自挹損,不敢稱制,事無巨細,皆決于李國、李離兄弟。國等事雄彌謹。



 諸將固請雄即尊位,以永興元年僭稱成都王,赦其境內,建元為建興,除晉法,約法七章。以
 其叔父驤為太傅,兄始為太保,折衝李離為太尉,建威李雲為司徙,翊軍李璜為司空,材官李國為太宰,其餘拜授各有差。追尊其曾祖武曰巴郡桓公,祖慕隴西襄王,父特成都景王,母羅氏曰王太后。范長生自西山乘素輿詣成都,雄迎之於門,執版延坐,拜丞相,尊曰范賢。長生勸雄稱尊號,雄於是僭即帝位,赦其境內,改年曰太武。追尊父特曰景帝,廟號始祖,母羅氏為太后。加范長生為天地太師,封西山侯,復其部曲不豫軍征,租稅一入其家。雄時建國草創,素無法式,諸將恃恩,各爭班位。其尚書令閻式上疏曰:「夫為國制法,勳尚仍舊。漢、晉
 故事,惟太尉、大司馬執兵,太傅、太保父兄之官,論道之職,司徙、司空掌五教九土之差。秦置丞相,總領萬機。漢武之末,越以大將軍統政。今國業初建,凡百末備,諸公大將班位有差,降而兢請施置,不與典故相應,宜立制度以為楷式。」雄從之。



 遣李國、李雲等率眾二萬寇漢中,梁州刺史張殷奔于長安。國等陷南鄭,盡徙漢中人於蜀。



 先是,南土頻歲饑疫,死者十萬計。南夷校尉李毅固守不降,雄誘建寧夷使討之。毅病卒,城陷,殺壯士三千餘人,送婦女千口於成都。



 時李離據梓潼,其部將羅羕、張金茍等殺離及閻式,以梓潼歸于羅尚。尚遣其將向
 奮屯安漢之宜福以逼雄,雄率眾攻奮,不剋。時李國鎮巴西,其帳下文碩又殺國,以巴西降尚。雄乃引還,遣其將張寶襲梓潼,陷之。會羅尚卒,巴郡亂,李驤攻涪,又陷之,執梓潼太守譙登,遂乘勝進軍討文碩,害之。雄大悅,赦其境內,改元曰玉衡。



 雄母羅氏死,雄信巫覡者之言,多有忌諱,至欲不葬。其司空趙肅諫,雄乃從之。雄欲申三年之禮,群臣固諫,雄弗許。李驤謂司空上官惇曰:「今方難未弭,吾欲固諫,不聽主上終諒闇,君以為何如?」惇曰:「三年之喪,自天子達於庶人,故孔子曰:『何必高宗,古之人皆然。』但漢、魏以來,天下多難,宗廟至重,不可久曠,
 故釋衰絰,至哀而已。」驤曰:「任回方至,此人決於行事,且上常難達違言,待其至,當與俱請。」及回至,驤與回俱見雄。驤免冠流涕,固請公除。雄號泣不許。回跪而進曰:「今王業初建,凡百草創,一日無主,天下惶惶。昔武王素甲觀兵,晉襄墨絰從戎,豈所願哉?為天下屈己故也。願陛下割情從權,永隆天保。」遂彊扶雄起,釋服親政。



 是時南得漢嘉、涪陵,遠人繼至,雄於是下寬大之令,降附者皆假復除。虛己愛人,授用皆得其才,益州遂定。偽立其妻任氏為皇后。氐王楊難敵兄弟為劉曜所破,奔葭萌,遣子入質。隴西賊帥陳安又附之。



 遣李驤征越巂,太守李
 釗降。驤進軍由小會攻寧州刺史王遜,遜使其將姚岳悉眾距戰。驤軍不利,又遇霖雨,驤引軍還,爭濟瀘水,士眾多死。釗到成都,雄待遇甚厚,朝遷儀式,喪紀之禮,皆決於釗。



 楊難敵之奔葭萌也,雄安北李稚厚撫之,縱其兄弟還武都,難敵遂恃險多為不法,稚請討之。雄遣中領軍琀及將軍樂次、費他、李乾等由白水橋攻下辯,征東李壽督琀弟玝攻陰平。難敵遣軍距之,壽不得進,而琀、稚長驅至武街。難敵遣兵斷其歸道,四面攻之,獲琀、稚,死者數千人。琀、稚,雄兄蕩之子也。雄深悼之,不食者數日,言則流涕,深自咎責焉。



 其後將立蕩子班為太子。
 雄有子十餘人,群臣咸欲立雄所生。雄曰:「起兵之初,舉手扞頭,本不希帝王之業也。值天下喪亂,晉氏播蕩,群情義舉,志濟塗炭,而諸君遂見推逼,處王公之上。本之基業,功由先帝。吾兄嫡統,丕祚所歸,恢懿明睿,殆天報命,大事垂剋,薨于戎戰。班姿性仁孝,好學夙成,必為名器。」李驤與司徒王達諫曰:「先王樹冢嫡者,所以防篡奪之萌,不可不慎。吳子捨其子而立其弟,所以有專諸之禍;宋宣不立與夷而立穆公,卒有宋督之變。猶子之言,豈若子也?深願陛下思之。」雄不從,竟立班,驤退而流涕曰:「亂自此始矣!」



 張駿遣使遺雄書,勸去尊號,稱籓於晉。
 雄復書曰:「吾過為士大夫所推,然本無心於帝王也,進思為晉室元功之臣,退思共為守籓之將,掃除氛埃,以康帝宇。而晉室陵遲,德聲不振,引領東望,有年月矣。會獲來貺,情在闇室,有何已已。知欲遠遵楚、漢,尊崇義帝,《春秋》之義,於斯莫大。」駿重其言,使聘相繼。巴郡嘗告急,云有東軍。雄曰:「吾嘗慮石勒跋扈,侵逼瑯邪,以為耿耿。不圖乃能舉兵,使人欣然。」雄之雅譚,多如此類。



 雄以中原喪亂,乃頻遣使朝貢,與晉穆帝分天下。張駿領秦、梁,先是,遣傅穎假道于蜀,通表京師,雄弗許。駿又遣治中從事張淳稱籓于蜀,託以假道。雄大悅,謂淳曰:「貴主英
 名蓋世,土險兵彊,何不自稱帝一方?」淳曰:「寡君以乃祖世濟忠良,未能雪天下之恥,解眾人之倒懸,日昃忘食,枕戈待旦。以瑯邪中興江東,故萬里翼戴,將成桓文之事,何言自取邪!」雄有慚色,曰:「我乃祖乃父亦是晉臣,往與六郡避難此地,為同盟所推,遂有今日。瑯邪若能中興大晉於中夏,亦當率眾輔之。」淳還,通表京師,天子嘉之。



 時李驤死,以其子壽為大將軍、西夷校尉,督征南費黑、征東任巳攻陷巴東,太守楊謙退保建平。壽別遣費黑寇建平,晉巴東監軍毌丘奧退保宜都。雄遣李壽攻朱提,以費黑、仰攀為前鋒,又遣鎮南任回征木落,分寧
 州之援。寧州刺史尹奉降,遂有南中之地。雄於是赦其境內,使班討平寧州夷,以班為撫軍。



 咸和八年,雄生瘍於頭,六日死,時年六十一,在位三十年。偽謚武帝,廟曰太宗,墓號安都陵。



 雄性寬厚,簡刑約法,甚有名稱。氐苻成、隗文既降復叛,手傷雄母,及其來也,咸釋其罪,厚加待納。由是夷夏安之,威震四土。時海內大亂,而蜀獨無事,故歸之者相尋。雄乃興學校,置史官,聽覽之暇,手不釋卷。其賦男丁歲穀三斛,女丁半之,戶調絹不過數丈,綿數兩。事少役稀,百姓富貴,閭門不閉,無相侵盜。然雄意在招致遠方,國用不足,故諸將每進金銀珍寶,多有
 以得官者。丞相楊褒諫曰:「陛下為天下主,當網羅四海,何有以官買金邪!」雄遜辭謝之。後雄嘗酒醉而推中書令,杖太官令,褒進曰:「天子穆穆,諸侯皇皇,安有天子而為酗也!」雄即捨之。雄無事小出,褒於後持矛馳馬過雄。雄怪問之,對曰:「夫統天下之重,如臣乘惡馬而持矛也,急之則慮自傷,緩之則懼其失,是以馬馳而不制也。」雄寤,即還。雄為國無威儀,官無祿秩,班序不別,君子小人服章不殊;行軍無號令,用兵無部隊,戰勝不相讓,敗不相救,攻城破邑動以虜獲為先。此其所以失也。



 班字世文。初署平南將軍,後立為太子。班謙虛博納,敬
 愛儒賢,自何點、李釗,班皆師之,又引名士王嘏及隴西董融、天水文夔等以為賓友。每謂融等曰:「觀周景王太子晉、魏太子丕、吳太子孫登,文章鑒識,超然卓絕,未嘗不有慚色。何古賢之高朗,後人之莫逮也!」為性汎愛,動脩軌度。時諸李子弟皆尚奢靡,而班常戒厲之。每朝有大議,雄輒令豫之。班以古者墾田均平,貧富獲所,今貴者廣占荒田,貧者種殖無地,富者以己所餘而賣之,此豈王者大均之義乎!雄納之。及雄寢疾,班晝夜侍側。雄少數攻戰,多被傷夷,至是疾甚,痕皆膿潰,雄子越等惡而遠之。班為吮膿,殊無難色,每嘗藥流涕,不脫衣冠,其
 孝誠如此。



 雄死,嗣偽位,以李壽錄尚書事輔政。班居中執喪禮,政事皆委壽及司徒何點、尚書令王瑰等。越時鎮江陽,以班非雄所生,意甚不平。至此,奔喪,與其弟期密計圖之。李玝勸班遣越還江陽,以期為梁州刺史,鎮葭萌。班以未葬,不忍遣,推誠居厚,心無纖芥。時有白氣二道帶天,太史令韓豹奏:「宮中有陰謀兵氣,戒在親戚。」班不悟。咸和九年,班因夜哭,越殺班於殯宮,時年四十七,在位一年,遂立雄之子期嗣位焉。



 期字世運,雄第四子也。聰慧好學,弱冠能屬文,輕財好施,虛心招納。初為建威將軍,雄令諸子及宗室子弟以
 恩信合眾,多者不至數百,而期獨致千餘人。其所表薦,雄多納之,故長史列署頗出其門。



 既殺班,欲立越為主,越以期雄妻任氏所養,又多才藝,乃讓位於期。於是僭即皇帝位,大赦境內,改元玉恒。誅班弟都。使李壽伐都弟玝於涪,玝棄城降晉。封壽漢王,拜梁州刺史、東羌校尉、中護軍、錄尚書事;封兄越建寧王,拜相國、大將軍、錄尚書事。立妻閻氏為皇后。以其衛將軍尹奉為右丞相、驃騎將軍、尚書令,王瑰為司徒。期自以謀大事既果,輕諸舊臣,外則信任尚書令景騫、尚書姚華、田褒。褒無他才藝,雄時勸立期,故寵待甚厚。內則信宦豎許涪等。國
 之刑政,希復關之卿相,慶賞威刑,皆決數人而已,於是綱維紊矣。乃誣其尚書僕射、武陵公李載謀反,下獄死。



 先是,晉建威將軍司馬勳屯漢中,期遣李壽攻而陷之,遂置守宰,戍南鄭。



 雄子霸、保並不病而死,皆云期鴆殺之,於是大臣懷懼,人不自安。天雨大魚於宮中,其色黃。又宮中豕犬交。期多所誅夷,籍沒婦女資財以實後庭,內外兇兇,道路以目,諫者獲罪,人懷茍免。期又鴆殺其安北李攸。攸,壽之養弟也。於是與越及景騫、田褒、姚華謀襲壽等,欲因燒市橋而發兵。期又累遣中常侍許涪至壽所,伺其動靜。及殺攸,壽大懼,又疑許涪往來之數
 也,乃率步騎一萬,自涪向成都,表稱景騫、田褒亂政,興晉陽之甲,以除君側之惡。以李奕為先登。壽到成都,期、越不虞其至,素不備設,壽遂取其城,屯兵至門。期遣侍中勞壽,壽奏相國、建寧王越,尚書令、河南公景騫,尚書田褒、姚華,中常侍許涪,征西將軍李遐及將軍李西等,皆懷姦亂政,謀傾社稷,大逆不道,罪合夷滅。期從之,於是殺越、騫等。壽矯任氏令,廢期為邛都縣公,幽之別宮。期歎曰:「天下主乃當為小縣公,不如死也!」咸康三年,自縊而死,時年二十五,在位三年。謚曰幽公。及葬,賜鸞輅九旒,餘如王禮。雄之子皆為壽所殺。



 壽字武考,驤之子也。敏而好學,雅量豁然,少尚禮容,異於李氏諸子。雄奇其才,以為足荷重任,拜前將軍、督巴西軍事,遷征東將軍。時年十九,聘處士譙秀以為賓客,盡其讜言,在巴西威惠甚著。驤死,遷大將軍、大都督、侍中,封扶風公,錄尚書事。徵寧州,攻圍百餘日,悉平諸郡,雄大悅,封建寧王。雄死,受遺輔政。期立,改封漢王,食梁州五郡,領梁州刺史。



 壽威名遠振,深為李越、景騫等所憚,壽深憂之。代李玝屯涪,每應期朝覲,常自陳邊疆寇警,不可曠鎮,故得不朝。壽又見期、越兄弟十餘人年方壯大,而並有彊兵,懼不自全,乃數聘禮巴西龔壯。壯雖
 不應聘,數往見壽。時岷山崩,江水竭,壽惡之,每問壯以自安之術。壯以特殺其父及叔,欲假手報仇,未有其由,因說壽曰:「節下若能捨小從大,以危易安,則開國裂土,長為諸侯,名高桓文,勳流百代矣。」壽從之,陰與長史略陽羅恒、巴西解思明共謀據成都,稱籓歸順。乃誓文武,得數千人,襲成都,剋之,縱兵虜掠,至乃姦略雄女及李氏諸婦,多所殘害,數日乃定。



 恆與思明及李奕、王利等勸壽稱鎮西將軍、益州牧、成都王,稱籓於晉,而任調與司馬蔡興、侍中李艷及張烈等勸壽自立。壽命筮之,占者曰:「可數年天子。」調喜曰:「一日尚為足,而況數年乎!」思
 明曰:「數年天子,孰與百世諸侯!」壽曰:「朝聞道,夕死可矣。任侯之言,策之上也。」遂以咸康四年僭即偽位,赦其境內,改元為漢興。以董皎為相國,羅恒、馬當為股肱,李奕、任調、李閎為爪牙,解思明為謀主。以安車束帛聘龔壯為太師,壯固辭,特聽縞巾素帶,居師友之位。拔擢幽滯,處之顯列。追尊父驤為獻帝,母昝氏為太后,立妻閻氏為皇后,世子勢為太子。



 有告廣漢太守李乾與大臣通謀,欲廢壽者。壽令其子廣與大臣盟于前殿,徙乾漢嘉太守。大風暴雨,震其端門。壽深自悔責,命群臣極盡忠言,勿拘忌諱。



 遣其散騎常侍王嘏、中常侍王廣聘於石
 季龍。先是,季龍遺壽書,欲連橫入寇,約分天下。壽大悅,乃大修船艦,嚴兵繕甲,吏卒皆備候糧。以其尚書令馬當為六軍都督,假節鉞,營東場大閱,軍士七萬餘人,舟師溯江而上。過成都,鼓噪盈江,壽登城觀之。其群臣咸曰:「我國小眾寡,吳、會險遠,圖之未易。」解思明又切諫懇至,壽於是命群臣陳其利害。龔壯諫曰:「陛下與胡通,孰如與晉通?胡,豺狼國也。晉既滅,不得不北面事之。若與之爭天下,則彊弱勢異。此虞、虢之成範,已然之明戒,願陛下熟慮之。」群臣以壯之言為然,叩頭泣諫,壽乃止,士眾咸稱萬歲。



 遣其鎮東大將軍李奕征牂柯,太守謝恕
 保城距守者積日,不拔。會奕糧盡,引還。



 壽以其太子勢領大將軍、錄尚書事。



 壽承雄寬儉,新行篡奪,因循雄政,未逞其志欲。會李閎、王嘏從鄴還,盛稱季龍威強,宮觀美麗,鄴中殷實。壽又聞季龍虐用刑法,王遜亦以殺罰御下,並能控制邦域,壽心欣慕,人有小過,輒殺以立威。又以郊甸未實,都邑空虛,工匠器械,事未充盈,乃徙旁郡戶三丁已上以實成都,興尚方御府,發州郡工巧以充之,廣修宮室,引水入城,務於奢侈。又廣太學,起宴殿。百姓疲於使役,呼嗟滿道,思亂者十室而九矣。其左僕射蔡興切諫,壽以為誹謗,誅之。右僕射李嶷數以直言
 懺旨,壽積忿非一,託以他罪,下獄殺之。



 壽疾篤,常見李期、蔡興為祟。八年,壽死,時年四十四,在位五年。偽謚昭文帝,廟曰中宗,墓曰安昌陵。



 壽初為王,好學愛士,庶幾善道,每覽良將賢相建功立事者,未嘗不反覆誦之,故能征伐四剋,闢國千里。雄既垂心於上,壽亦盡誠於下,號為賢相。及即偽位之後,改立宗廟,以父驤為漢始祖廟,特、雄為大成廟,又下書言與期、越別族,凡諸制度,皆有改易。公卿以下,率用己之僚佐,雄時舊臣及六郡士人,皆見廢黜。壽初病,思明等復議奉王室,壽不從。李演自越巂上書,勸壽歸正返本,釋帝稱王,壽怒殺之,以威
 龔壯、思明等。壯作詩七篇,託言應璩以諷壽。壽報曰:「省詩知意,若今人所作,賢哲之話言也。古人所作,死鬼之常辭耳!」動慕漢武、魏明之所為,恥聞父兄時事,上書者不得言先世政化,自以己勝之也。



 勢字子仁,壽之長子也。初,壽妻閻氏無子,驤殺李鳳,為壽納鳳女,生勢。期愛勢姿貌,拜翊軍將軍、漢王世子。勢身長七尺九寸,腰帶十四圍,善於俯仰,時人異之。壽死,勢嗣偽位,赦其境內,改元曰太和。尊母閻氏為太后,妻李氏為皇后。



 太史令韓皓奏熒惑守心,以過廟禮廢,勢命群臣議之。其相國董皎、侍中王嘏等以為景武昌業,
 獻文承基,至親不遠,無宜疏絕。勢更令祭特、雄,同號曰漢王。



 勢弟大將軍、漢王廣以勢無子,求為太弟,勢弗許。馬當、解思明以勢兄弟不多,若有所廢,則益孤危,固勸許之。勢疑當等與廣有謀,遣其太保李奕襲廣於涪城,命董皎收馬當、思明斬之,夷其三族。貶廣為臨邛侯,廣自殺。思明有計謀,彊諫諍,馬當甚得人心。自此之後,無復紀綱及諫諍者。



 李奕自晉壽舉兵反之,蜀人多有從奕者,眾至數萬。勢登城距戰。奕單騎突門,門者射而殺之,眾乃潰散。勢既誅奕,大赦境內,改年嘉寧。



 初,蜀土無獠,至此,始從山而出,北至犍為,梓潼,布在山谷,十餘萬
 落,不可禁制,大為百姓之患。勢既驕吝,而性愛財色,常殺人而取其妻,荒淫不恤國事。夷獠叛亂,軍守離缺,境宇日蹙。加之荒儉,性多忌害,誅殘大臣,刑獄濫加,人懷危懼。斥外父祖臣佐,親任左右小人,群小因行威福。又常居內,少見公卿。史官屢陳災譴,乃加董皎太師,以名位優之,實欲與分災眚。



 大司馬桓溫率水軍伐勢。溫次青衣,勢大發軍距守,又遣李福與昝堅等數千人從山陽趣合水距溫。謂溫從步道而上,諸將皆欲設伏於江南以待王師,昝堅不從,率諸軍從江北鴛鴦碕渡向犍為,而溫從山陽出江南,昝堅到犍為,方知與溫異道,乃
 迴從沙頭津北渡。及堅至,溫已造成都之十里陌,昝堅眾自潰。溫至城下,縱火燒其大城諸門。勢眾惶懼,無復固志,其中書監王嘏、散騎常侍常璩等勸勢降。勢以問侍中馮孚,孚言:「昔吳漢征蜀,盡誅公孫氏。今晉下書,不赦諸李,雖降,恐無全理。」勢乃夜出東門,與昝堅走至晉壽,然後送降文於溫曰:「偽嘉寧二年三月十七日,略陽李勢叩頭死罪。伏惟大將軍節下,先人播流,恃險因釁,竊自汶、蜀。勢以闇弱,復統未緒,偷安荏苒,未能改圖。猥煩朱軒,踐冒險阻。將士狂愚,干犯天威。仰慚俯愧,精魂飛散,甘受斧鑕,以釁軍鼓。伏惟大晉,天網恢弘,澤及四
 海,恩過陽日。逼迫倉卒,自投草野。即日到白水城,謹遣私署散騎常侍王幼奉箋以聞,并敕州郡投戈釋杖。窮池之魚,待命漏刻。」勢尋輿櫬面縛軍門,溫解其縛,焚其櫬,遷勢及弟福、從兄權親族十餘人于健康,封勢歸義侯。升平五年,死于建康。在位五年而敗。



 始,李特以惠帝太安元年起兵,至此六世,凡四十六年,以穆帝永和三年滅。



 史臣曰:昔周德方隆,古公切踰梁之患;漢祚斯永,宣后興渡湟之師。是知戎狄亂華,釁深自古,況乎巴、濮雜種,厥類實繁,資剽竊以全生,習獷悍而成俗。李特世傳兇
 狡,早擅梟雄,太息劍門,志吞井絡。屬晉綱之落紐,乘羅侯之無斷,騁馬屬犍,同聲雲集,殲殄蜀、漢,薦食巴、梁,沃野無半菽之資,華陽有析骸之釁。蓋上失其道,覆敗之至於斯!



 仲俊天挺英姿,見稱奇偉,摧鋒累載,克隆霸業。蹈玄德之前基,掩子陽之故地,薄賦而綏弊俗,約法而悅新邦,擬於其倫,實孫權之亞也。若夫立子以嫡,往哲通訓,繼體承基,前脩茂範。而雄闇經國之遠圖,蹈匹夫之小節,傳大統於猶子。託彊兵於厥胤。遺骸莫斂,尋戈之釁已深;星紀未周,傾巢之釁便及。雖云天道,抑亦人謀。



 班以寬愛罹災,期以暴戾速禍,殊塗並失,異術同亡。
 武考憑藉世資,窮兵竊位,罪百周帶,毒甚楚圍,獲保歸全,何其幸也!子仁承緒,繼傳昏虐,驅率餘燼,敢距大邦。授甲晨征,則理均於困獸;斬關宵遁,則義殊於前禽。宜其懸首國門,以明大戮,遂得禮同劉禪,不亦優乎!



 贊曰:晉圖馳馭,百六斯鐘。天垂伏鱉,野戰群龍。李特窺釁,盜我巴、庸。世歷五朝,年將四紀。篡殺移國,昏狂繼軌。德之不修,險亦難恃。



\end{pinyinscope}