\article{載記第二十三 慕容垂}

\begin{pinyinscope}

 慕容垂



 慕容垂,字道明,皝之第五子也。少岐嶷有器度,身長七尺七寸,手垂過膝。皝甚寵之,常目而謂諸弟曰:「此兒闊達好奇,終能破人家,或能成人家。」故名霸,字道業,恩遇踰于世子俊,故俊不能平之。以滅宇文之功,封都鄉侯。石季龍來伐,既還,猶有兼并之志,遣將鄧恒率眾數萬屯于樂安,營攻取之備。垂戍徒河,與恒相持,恒憚而不
 敢侵。垂少好畋游,因獵墜馬折齒,慕容俊僭即王位,改名,外以慕郤為名,內實惡而改之。尋以讖記之文,乃去「夬」,以「垂」為名焉,



 石季龍之死也,趙魏亂,垂謂俊曰:「時來易失,赴機在速,兼弱攻昧,今其時矣。」俊以新遭大喪,不許。慕輿根言於俊曰:「王子之言,千載一時,不可失也。」俊乃從之,以垂為前鋒都督。俊既剋幽州,將坑降卒,垂諫曰:「弔伐之義,先代常典。今方平中原,宜綏懷以德,坑戮之刑不可為王師之先聲。」俊從之。及俊僭稱尊號,封垂吳王,徙鎮信都,以侍中、右禁將軍錄留臺事,大收東北之利。又為征南將軍、荊、兗二州牧,有聲于梁、楚之
 南。再為司隸,偽王公已下莫不累迹。時莫容嗣偽位,慕容恪為太宰。恪甚重垂,常謂曰:「吳王將相之才十倍於臣,先帝以長幼之次,以臣先之,臣死之後,願陛下委政吳王,可謂親賢兼舉。」及敗桓溫于枋頭,威名大振。慕容評深忌惡之,乃謀誅垂。垂懼禍及己,與世子全奔於苻堅。



 自恪卒後,堅密有圖之謀,憚垂威名而未發。及聞其至,堅大悅,郊迎執手,禮之甚重。堅相王猛惡垂雄略,勸堅殺之。堅不從,以為冠軍將軍,封賓都侯,食華陰之五百戶。王猛伐洛,引全為參軍。猛乃令人詭傳垂語於全曰:「吾已東還,汝可為計也。」全信之,乃奔。猛表
 全叛狀,垂懼而東奔,及藍田,為追騎所獲。堅引見東堂,慰勉之曰:「卿家國失和,委身投朕。賢子志不忘本,猶懷首丘。《書》不云乎:「父父子子,無相及也。」卿何為過懼而狼狽若斯也!」於是復垂爵位,恩待如初。



 及堅擒,垂隨堅入鄴,收集諸子,對之悲慟,見其故吏,有不悅之色。前郎中令高弼私於垂曰:「大王以命世之姿,遭無妄之運,迍邅妻伏,艱亦至矣。天啟嘉會,靈命暫遷,此乃鴻漸之始,龍變之初,深願仁慈有以慰之。且夫高世之略必懷遺俗之規,方當網漏吞舟,以弘苞養之義;收納舊臣之胄,以成為山之功,奈何以一怒捐之?竊為大王不取。」垂深
 納之。垂在堅朝,歷位京兆尹,進封泉州侯,所在征伐,皆在大功。



 堅之敗於淮南也,垂軍獨全,堅以千餘騎奔垂。垂世子寶言於垂曰:「家國傾喪,皇綱廢馳,至尊明命著之圖籙,當隆中興之業,建少康之功。但時來之運未至,故韜光俟奮耳。今天厭亂德,凶眾土崩,可謂乾啟神機,授之于我。千載一時,今其會也,宜恭承皇天之意,因而取之。且夫立大功者不顧小節,行大仁者不念小惠。秦既蕩覆二京,空辱神器,仇恥之深,莫甚於此,願不以意氣微恩而忘社稷之重。五木之祥,今其至矣。」垂曰:「汝言是也。然彼以赤心投命,若何害之!茍天所棄,圖之多便。
 且縱令北還,更待其釁,既不負宿心,可以義取天下。」垂弟德進曰:「夫鄰國相吞,有自來矣。秦強而并燕,秦弱而圖之,此為報仇雪辱,豈所謂負宿心也!昔鄧祁侯不納三甥之言,終為楚所滅;吳王夫差違子胥之諫,取禍句踐。前事之不忘,後事之師表也。願不棄湯、武之成蹤,追韓信之敗迹,乘彼土崩,恭行天罰,斬逆氐,復宗祀,建中興,繼洪烈,天下大機,弗宜失也。若釋數萬之眾,授干將之柄,是郤天時而待後害,非至計也。語曰:『當斷不斷,反受其亂。』願兄無疑。」垂曰:「吾昔為太傅所不容,投身於秦主,又為王猛所譖,復見昭亮,國士之禮每深,報德之分
 未一。如使秦運必窮,歷數歸我者,授首之便,何慮無之。關西之地,會非吾有,自當有擾之者,吾可端拱而定關東。君子不怙亂,不為禍先,且可觀之。」乃以兵屬堅。初,寶在長安,與韓黃、李根等因讌摴蒱,寶危坐整容,誓之曰:「世云摴蒱有神,豈虛也哉!若富貴可期,頻得三盧。」於是三擲盡盧,寶拜而受賜,故云五木之祥。



 堅至澠池,垂請至鄴展拜陵墓,因張國威刑,以安戎狄。堅許之,權翼諫曰:「垂爪牙名將,所謂今之韓、白,世豪東夏,志不為人用。頃以避禍歸誠,非慕德而至,列土干城未可以滿其志,冠軍之號豈足以稱其心!且垂猶鷹也,飢則附人,飽便
 高颺,遇風塵之會,必有陵霄之志。惟宜急其羈靽,不可任其所欲。」堅不從,遣其將李蠻、閔亮、尹國率眾三千送垂,又遣石越戍鄴,張蠔戍并州。



 時堅子丕先在鄴,及垂至,丕館之于鄴西,垂具說淮南敗狀。會堅將苻暉告丁零翟斌聚眾謀逼洛陽,歪謂垂曰:「惟斌兄弟因王師小失,敢肆凶勃,子母之軍,殆難為敵,非冠軍英略,莫可以滅也。欲相煩一行可乎?」垂曰:「下官殿下之鷹犬,敢不惟命是聽。」於是大賜金帛,一無所受,惟請舊田園。丕許之,配垂兵二千,遣其將苻飛龍率氐騎一千為垂之副。丕戒飛龍曰:「卿王室肺腑,年秩雖卑,其實帥也。垂為三軍
 之統,卿為謀垂之主,用兵制勝之權,防微杜貳之略,委之於卿,卿其勉之。」垂請入鄴城拜廟,丕不許。乃潛服而入,亭吏禁之,垂怒,斬吏燒亭而去。石越言於丕曰:「垂之在燕,破國亂家,及投命聖朝,蒙超常之遇,忽敢輕侮方鎮,殺吏焚亭,反形已露,終為亂階。將老兵疲,可襲而取之矣。」歪曰:「淮南之敗,眾散親離,而垂侍衛聖躬,誠不可忘。」越曰:「垂既不忠於燕,其肯盡忠於我乎!且其亡虜也,主上寵同功舊,不能銘澤誓忠,而首謀為亂,今不擊之,必為後害。」丕不從。越退而告人曰:「公父子好存小仁,不顧天下大計,吾屬終當為鮮卑虜矣。」



 垂至河內,殺飛龍,
 悉誅氐兵,召募遠近,眾至三萬,濟河焚橋,令曰:「吾本外假秦聲,內規興復。亂法者軍有常刑,奉命者賞不踰日,天下既定,封爵有差,不相負也。」



 翟斌聞垂之將濟河也,遣使推垂為盟主。垂距之曰:「吾父子寄命秦朝,危而獲濟,荷主上不世之恩,蒙更生之惠,雖曰君臣,義深父子,豈可因其小隙,便懷二三。吾本救豫州,不赴君等,何為斯議而及於我!」垂進欲襲據洛陽,故見苻暉以臣節,退又未審斌之誠款,故以此言距之。垂至洛陽,暉閉門距守,不與垂通。斌又遣長史河南郭通說垂,乃許之。斌率眾會垂,勸稱尊號,垂曰:「新興侯,國之正統,孤之君也。若
 以諸君之力,得平關東,當以大義喻秦,奉迎反正。無上自尊,非孤心也。」謀于眾曰:「洛陽四面受敵,北阻大河,至於控馭燕、趙,非形勝之便,不如北取鄴都,據之而制天下。」眾咸以為然。乃引師而東,遣建威將軍王騰起浮橋於石門。



 初,垂之發鄴中,子農及兄子楷、紹,北子宙,為苻丕所留。及誅飛龍,遣田生密告農等,使起兵趙、魏以相應。於是農、宙奔列人,楷、紹奔辟陽,眾咸應之。農西招庫辱官偉于上黨,東引乞特歸于東阿,各率眾數萬赴之,眾至十餘萬。丕遣石越討農,為農所敗,斬越于陳。



 垂引兵至滎陽,以太元八年自稱大將軍、大都督、燕王,承制
 行事,建元曰燕元。令稱統府,府置四佐,王公已下稱臣,凡所封拜,一如王者,以翟斌為建義大將軍,封河南王;翟檀為柱國大將軍、弘農王;弟德為車騎大將軍、范陽王;兄子楷征西大將軍、太原王。眾至二十餘萬,濟自石門,長驅攻鄴。農、楷、紹、宙等率眾會垂。立子寶為燕王太子,封功臣為公侯伯子男者百餘人。



 苻丕乃遣侍郎姜讓謂垂曰:「往歲大駕失據,君保衛鑾輿,勤王誠義,邁蹤前烈。宜述修前規,終忠貞之節,奈何棄崇山之功,為此過舉!過貴能改,先賢之嘉事也。深宜詳思,悟猶未晚。」垂謂讓曰:「孤受主上不世之恩,故欲安全長樂公,使盡眾
 赴京師,然後脩復家國之業,與秦永為鄰好。何故闇於機運,不以鄴見歸也?大義滅親,況於意氣之顧!公若迷而不返者,孤亦欲竊兵勢耳。今事已然,恐單馬乞命不可得也。」讓厲色責垂曰:「將軍不容於家國,投命於聖朝,燕之尺土,將軍豈有分乎!主上與將軍風殊類別,臭味不同,奇將軍於一見,託將軍以斷金,寵踰宗舊,任齊懿籓,自古君臣冥契之重,豈甚此邪!方付將軍以六尺之孤,萬里之命,奈何王師小敗,便有二圖!夫師起無名,終則弗成,天之所廢,人不能支。將軍起無名之師,而欲興天所廢,竊未見其可。長樂公主上之元子,聲德邁於唐、
 衛,居陜東之任,為朝廷維城,其可束手輸將軍以百城之地!大夫死王事,國君死社稷,將軍欲裂冠毀冕,拔本塞源者,自可任將軍兵勢,何復多云。但念將軍以七十之年,懸首白旗,高世之忠,忽為逆鬼,竊為將軍痛之。」垂默然。左右勸垂殺之,垂曰:「古者兵交,使在其間,犬各吠非其主,何所問也!」乃遣讓歸。



 垂上表於苻堅曰:「臣才非古人,致禍起蕭墻,身嬰時難,歸命聖朝。陛下恩深周、漢,猥叨微顧之遇,位為列將,爵忝通侯,誓在戮力輸誠,常懼不及。去夏桓沖送死,一擬雲消,回討鄖城,俘馘萬計,斯誠陛下神算之奇,頗亦愚臣忘死之效。方將飲馬桂
 州,懸旌閩會,不圖天助亂德,大駕班師。陛下單馬奔臣,臣奉衛匪貳,豈陛下聖明鑒臣單心,皇天后土實亦知之。臣奉詔北巡,受制長樂。然丕外失眾心,內多猜忌,今臣野次外庭,不聽謁廟。丁零逆豎寇逼豫州,丕迫臣單赴,限以師程,惟給弊卒二千,盡無兵杖,復令飛龍潛為刺客。及至洛陽,平原公暉復不信納。臣竊惟進無淮陰功高之慮,退無李廣失利之愆,懼有青蠅,交亂白黑,丁零夷夏以臣忠而見疑,乃推臣為盟主。臣受託善始,不遂令終,泣望西京,揮涕即邁。軍次石門,所在雲赴,雖復周武之會於孟津,漢祖之集於垓下,不期之眾,實有甚
 焉。欲令長樂公盡眾赴難,以禮發遣,而丕固守匹夫之志,不達變通之理。臣息農收集故營,以備不虞,而石越傾鄴城之眾,輕相掩襲,兵陣未交,越已隕首。臣既單車懸軫,歸者如雲,斯實天符,非臣之力。且鄴者臣國舊都,應即惠及,然後西面受制,永守東籓,上成陛下遇臣之意,下全愚臣感報之誠。今進師圍鄴,并喻丕以天時人事。而丕不察機運,杜門自守,時出挑戰,鋒戈屢交,恒恐飛矢誤中,以傷陛下天性之念。臣之此誠,未簡神聽,輒遏兵止銳,不敢竊攻。夫運有推移,去來常事,惟陛下察之。」



 堅報曰:「朕以不德,忝承靈命,君臨萬邦,三十年矣。遐
 方幽裔,莫不來庭,惟東南一隅,敢違王命。朕爰奮六師,恭行天罰,而玄機不弔,王師敗績。賴卿忠誠之至,輔翼朕躬,社稷之不隕,卿之力也。《詩》云:『中心藏之,何日忘之。』方任卿以元相,爵卿以郡侯,庶弘濟艱難,敬酬勛烈,何圖伯夷忽毀冰操,柳惠倏為淫夫!覽表惋然,有慚朝士。卿既不容於本朝,匹馬而投命,朕則寵卿以將位,禮卿以上賓,任同舊臣,爵齊勛輔,歃血斷金,披心相付。謂卿食椹懷音,保之偕老。豈意畜水覆舟,養獸反害,悔之噬臍,將何所及!誕言駭眾,誇擬非常,周武之事,豈卿庸人所可論哉!失籠之鳥,非羅所羈;脫網之鯨,豈罟所制!翹
 陸任懷,何須聞也。念卿垂老,老而為賊,生為叛臣,死為逆鬼,侏張幽顯,布毒存亡,中原士女,何痛如之!朕之歷運興喪,豈復由卿!但長樂、平原以未立之年,遇卿於兩都,慮其經略未稱朕心,所恨者此焉而已。」



 垂攻拔鄴郛,丕固守中城,垂塹而圍之,分遣老弱於魏郡、肥鄉,築新興城以置輜重,擁漳水以灌之。



 翟斌潛諷丁零及西人,請斌為尚書令。垂訪之群僚,其安東將軍封衡厲色曰:「馬能千里,不免羈靽,明畜生不可以人御也。斌戎狄小人,遭時際會,兄弟封王,自驩兜已來,未有此福。忽履盈忘止,復有斯求,魂爽錯亂,必死不出年也。」垂猶隱忍容
 之,令曰:「翟王之功宜居上輔,但臺既未建,此官不可便置。待六合廓清,更當議之。」斌怒,密應苻丕,潛使丁零決防潰水。事洩,垂誅之。斌兄子真率其部眾北走邯鄲,引兵向鄴,欲與丕為內外之勢,垂令其太子寶、冠軍慕容隆擊破之。真自邯鄲北走,又使慕容楷率騎追之,戰于下邑,為真所敗,真遂屯于承營。垂謂諸將曰:「苻丕窮寇,必守死不降。丁零叛擾,乃我腹心之患。吾欲遷師新城,開其逸路,進以謝秦主疇昔之恩,退以嚴擊真之備。」於是引師去鄴,北屯新城。慕容農進攻翟嵩于黃泥,破之。垂謂其范陽王德曰:「苻丕吾縱之不能去,方引晉師規
 固鄴都,不可置也。」進師又攻鄴,開其西奔之路。



 垂將有北都中山之意,農率眾數萬迎之。群僚聞慕容為苻堅所殺,勸垂僭位。垂以慕容沖稱號關中,不許。



 晉龍驤將軍劉牢之率眾救苻丕,至鄴,垂逆戰,敗績,遂撤鄴圍,退屯新城。垂自新城北走,牢之追垂,連戰皆敗。又戰于五橋澤,王師敗績,德及隆引兵要之于五丈橋,牢之馳馬跳五丈澗,會苻丕救至而免。



 翟真去承營,徙屯行唐,真司馬鮮于乞殺真,盡誅翟氏,自立為趙王。營人攻殺乞,迎立真從弟成為主,真子遼奔黎陽。



 高句驪寇遼東,垂平北慕容佐遣司馬郝景率眾救之,為高句驪所敗,
 遼東、玄菟遂沒。



 建節將軍徐巖叛于武邑,驅掠四千餘人,北走幽州。垂馳敕其將平規曰:「但固守勿戰,比破丁零,吾當自討之。」規違命距戰,為巖所敗。巖乘勝入薊,掠千餘戶而去,所過寇暴,遂據令支。



 翟成長史鮮於得斬成而降,垂入行唐,悉坑其眾。



 苻丕棄鄴城,奔於并州。



 慕容農攻剋令支,斬徐巖兄弟。時伐高句驪,復遼東、玄菟二郡,還屯龍城。



 垂定都中山,群僚勸即尊號,具典儀,修郊燎之禮。垂從之,以太元十一年僭即位。赦其境內,改元曰建興,置百官,繕宗廟社稷,立寶為太子。以其左長史庫辱官偉、右長史段崇、龍驤張崇,中山尹封衡為吏
 部尚書,慕容德為侍中、都督中外諸軍事、領司隸校尉,撫軍慕容麟為衛大將軍,其餘拜授有差。追尊母蘭氏為文昭皇后,遷皝后段氏,以蘭氏配饗。博士劉詳、董謐議以堯母妃位第三,不以貴陵姜嫄,明聖王之道以至公為先。垂不從。



 遣其征西慕容楷、衛軍慕容麟、鎮南慕容紹、征虜慕容宙等攻苻堅冀州牧苻定、鎮東苻紹、幽州牧苻謨、鎮北苻亮。楷與定等書,喻以禍福,定等悉降。



 垂留其太子寶守中山,率諸將南攻翟遼,以楷為前鋒都督。遼之部眾皆燕、趙人也,咸曰:「太原王之子,吾之父母。」相率歸附。遼懼,遣使請降。垂至黎陽,遼肉袒謝罪,垂
 厚撫之。



 為其太子寶起承華觀,以寶錄尚書政事,巨細皆委之,重總大綱而已。立其夫人段氏為皇后。又以寶領侍中、大單于、驃騎大將軍、幽州牧。建留臺于龍城,以高陽王慕容隆錄留臺尚書事。時慕容及諸宗室為苻堅所害者,並招魂葬之。



 清河太守賀耕聚眾定陵以叛,南應翟遼,慕容農討斬之,毀定陵城。進師入鄴,以鄴城廣難固,築鳳陽門大道之東為隔城。



 其尚書郎婁會上疏曰:「三年之喪,天下之達制,兵荒殺禮,遂以一切取士。人心奔兢,茍求榮進,至乃身冒縗絰,以赴時役,豈必殉忠於國家,亦昧利於其間也。聖王設教,不以顛沛而
 虧其道,不以喪亂而變其化,故能杜豪兢之門,塞奔波之路。陛下鐘百王之季,廓中興之業,天下漸平,兵革方偃,誠宜蠲蕩瑕穢,率由舊章。吏遭大喪,聽終三年之禮,則四方知化,人斯服禮。」垂不從。



 翟遼死,子釗代立,攻逼鄴城,慕容農擊走之。垂引師伐釗于滑臺,次于黎陽津,釗於南岸距守,諸將惡其兵精,咸諫不宜濟河。垂笑曰:「堅子何能為,吾今為鯽等殺之。」遂徙營就西津,為牛皮船百餘艘,載疑兵列杖,溯流而上。釗先以大眾備黎陽,見垂向西津,乃棄營西距。垂潛遣其桂林王慕容鎮、驃騎慕容國於黎陽津夜濟,壁于河南。釗聞而奔還,士眾
 疲渴,走歸滑臺,釗攜妻子率數百騎北趣白鹿山。農追擊,盡擒其眾,釗單騎奔長子。釗所統七郡戶三萬八千皆安堵如故。徙徐州流人七千餘戶于黎陽。



 於是議征長子。諸將咸諫,以慕容永未有釁,連歲征役,士卒疲怠,請俟他年。垂將從之,及聞慕容德之策,笑曰:「吾計決矣。且吾投老,扣囊底智,足以剋之,不復留逆賊以累子孫也。」乃發步騎七萬,遣其丹陽王慕容贊、龍驤張崇攻永弟支于晉陽。永遣其將刁雲、慕容鐘率眾五萬屯潞川。垂遣慕容楷出自滏口,慕容農入自壺關,垂頓于鄴之西南,月餘不進。永謂垂詭道伐之,乃攝諸軍還杜太行
 軹關。垂進師入自天井關,至于壺壁。永率精卒五萬來距,阻河曲以自固,馳使請戰。垂列陣于壺避之南,農、楷分為二翼,慕容國伏千兵於深澗,與永大戰。垂引軍偽退,永追奔數里,國發伏兵馳斷其後,楷、農夾擊之,永師大敗,斬首八千餘級,永奔還長子。慕容贊攻剋晉陽。垂進圍長子,永將賈韜潛為內應。垂進軍入城,永奔北門,為前驅所獲,於是數而戮之,并其所署公卿刁雲等三十餘人。永所統新舊八郡戶七萬六千八百及乘輿、服御、伎樂、珍寶悉獲之,於是品物具矣。



 使慕容農略地河南,攻廩丘、陽城,皆剋之,太山、瑯邪諸郡皆委城奔潰,
 農進師臨海,置守宰而還。垂告捷于龍城之廟。



 遣其太子寶及農與慕容麟等率眾八萬伐魏,慕容德、慕容紹以步騎一萬八千為寶後繼。魏聞寶將至,徙往河西。寶進師臨河,懼不敢濟。還次參合,忽有大風黑氣,狀若隄防,或高或下,臨覆軍上。沙門支曇猛言於寶曰:「風氣暴迅,魏軍將至之候,宜遣兵禦之。」寶笑而不納。曇猛固以為言,乃遣麟率騎三萬為後殿,以禦非常。麟以曇猛言為虛,縱騎遊獵。俄而黃霧四塞,日月晦冥,是夜魏師大至,三軍奔潰,寶與德等數千騎奔免,士眾還者十一二,紹死之。初,寶至幽州,所乘車軸無故自折。術士靳安以
 為大凶,固勸寶還,寶怒不從,故及於敗。



 寶恨參合之敗,屢言魏有可乘之機。慕容德亦曰:「魏人狃於參合之役,有陵太子之心,宜及聖略,摧其銳志。」垂從之,留德守中山,自率大眾出參合,鑿山開道,次於獵嶺。遣寶與農出天門,征北慕容隆、征西慕容盛踰青山,襲魏陳留公泥於平城,陷之,收其眾三萬餘人而還。



 垂至參合,見往年戰處積骸如山,設弔祭之禮,死者父兄一時號哭,軍中皆慟。垂慚憤歐血,因而寢疾,乘馬輿而進。過平城北三十里,疾篤,築燕昌城而還。寶等至雲中,聞垂疾,皆引歸。及垂至于平城,或有叛者奔告魏曰:「垂病已亡,輿尸在
 軍。」魏又聞參合大哭,以為信然,乃進兵追之,知平城已陷而退,還館陰山。垂至上谷之沮陽,以太元二十一年死,時年七十一,凡在位十三年。遺令曰:「方今禍難尚殷,喪禮一從簡易,朝終夕殯,事訖成服,三日之後,釋服從政。強寇伺隙,秘勿發喪,至京然後舉哀行服。」寶等遵行之。偽謚成武皇帝,廟號世祖,墓曰宣平陵。



\end{pinyinscope}