\article{志第六}

\begin{pinyinscope}

 律歷上



 《易》曰:「形而上者謂之道,形而下者謂之器。」夫神道廣大,妙本於陰陽;形器精微,義先於律呂。聖人觀四時之變,刻玉紀其盈虛,察五行之聲,鑄金均其清濁,所以遂八風而宣九德,和大樂而成政道。然金質從革,侈弇無方;竹體圓虛,脩短利制。是以神瞽作律,用寫鐘聲,乃紀之以三,平之以六,成於十二,天之道也。又葉時日於晷度,
 效地氣於灰管,故陰陽和則景至,律氣應則灰飛。灰飛律通,吹而命之,則天地之中聲也。故可以範圍百度,化成萬品,則《虞書》所謂「葉時月正日,同律度量衡」者也。中聲節以成文,德音章而和備,則可以動天地,感鬼神,道性情,移風俗。葉言志於詠歌,鑒盛衰於治亂,故君子審聲以知音,審音以知樂,審樂以知政,蓋由茲道。太史公律書云:「王者制事立物,法度軌則,一稟於六律。六律為萬事之本,其於兵械尤所重焉。故云望敵知吉凶,聞聲效勝負,百王不易之道也。」



 及秦氏滅學,其道浸微。漢室初興,丞相張蒼首言律,未能審備。孝武帝創置協律
 之官,司馬遷言律呂相生之次詳矣。及王莽際,考論音律,劉歆條奏,大率有五:一曰備數,一、十、百、千、萬也;二曰和聲,宮、商、角、徵、羽也;三曰審度,分、寸、尺、丈、引也;四曰嘉量,龠、合、升、斗、斛也;五曰權衡,銖、兩、斤、鈞、石也。班固因而志之。蔡邕又記建武已後言律呂者,至司馬紹統採而續之。漢末天下大亂,樂工散亡,器法堙滅,魏武始獲杜夔,使定樂器聲調。夔依當時尺度,權備典章。及武帝受命,遵而不革。至泰始十年,光祿大夫荀勖奏造新度,更鑄律呂。元康中,勖子籓嗣其事,未及成功,屬永嘉之亂,中朝典章,咸沒於石勒。及元帝南遷,皇度草昧,禮容
 樂器,掃地皆盡,雖稍加採掇,而多所淪胥,終于恭、安,竟不能備。今考古律相生之次,及魏武已後言音律度量者,以聲明於篇云。



 《傳》云:「十二律,黃帝之所作也。使伶倫自大夏之西,乃之崑崙之陰,取竹之嶰谷生,其竅厚均者,斷雨節間長三寸九分而吹之,以為黃鐘之宮,曰含少。次制十二竹筒,寫鳳之鳴,雄鳴為六,雌鳴亦六,以比黃鐘之宮,皆可以生之以定律呂。則律之始造,以竹為管,取其自然圓虛也。」又云「黃帝作律,以玉為管,長尺,六孔,這二十月音。至舜時,西王母獻昭華之琯,以玉為之。」及漢章帝時,零陵文學奚景於泠道舜祠下得白玉琯。又
 武帝太康元年,汲郡盜發六國時魏襄王冢,亦得玉律。則古者又以玉為管矣。以玉者,取其體含廉潤也。而漢平帝時,王莽又以銅為之。銅者,自名也,所以同天下,齊風俗也。為物至精,不為燥濕寒暑改節,介然有常,似士君子之行,故用焉。



 《周禮》太師掌六律、六呂,以合陰陽之聲。六律陽聲,黃鐘、太蔟、姑洗、蕤賓、夷則、無射也;六呂陰聲,大呂、應鐘、南呂、林鐘、仲呂、夾鐘也。又有太師則執同律以聽軍聲,而詔以吉凶。其典同掌六律之和,以辯天地四方陰陽之聲,以為樂器,皆以十有二律而為之數度,以十有二
 聲而為之齊量焉。



 及周景王將鑄無射,問律於泠州鳩,對曰:「夫六,中之色,故名之曰黃鐘,所以宣養六氣九德也。由是第之。二曰太蔟,所以金奏贊陽出滯也。三曰姑洗,所以羞潔百物,考神納賓也。四曰蕤賓,所以安靜神人,獻酬交酢也。五曰夷則,所以詠歌九德,平人無貳也。六曰無射,所以宣布哲人之令德,示人軌儀也。為之六間,以揚沈伏而黜散越也。元間大呂,助宣物也,二間夾鐘,出四隙之細也。三間中呂,宣中氣也。四間林鐘,和展百事,俾莫不任肅純恪也。五間南呂,贊陽秀也。六間應鐘,均利器用,俾應復也。」此皆所以律述時氣效節
 物也。



 及秦始皇焚書蕩覆,典策缺亡,諸子瑣言時有遺記。呂不韋《春秋》言:黃鐘之宮,律之本也,下生林鐘,林鐘上生太蔟,太蔟下生南呂,南呂上生姑洗,姑洗下生應鐘,應鐘上生蕤賓,蕤賓下生大呂,大呂下生夷則,夷則上生夾鐘,夾鐘下生無射,無射上生中呂。三分所生,益其一分以上生;三分所生,去其一分以下生。後代之言音律者多宗此說。



 及漢興,承秦之弊,張蒼首治律歷,頗未能詳。故孝武帝正樂,乃置協律之官,雖律呂清濁之體粗正,金石高下之音有準,然徒捃採遺存,以成一時之制,而數猶用五。



 時淮南王安延致儒博,亦為律
 呂。云黃鐘之律九寸而宮音調,因而九之,九九八十一,故黃鐘之數立焉,位在子。林鐘位在未,其數五十四。太蔟其數七十二,南呂之數四十八,姑洗之數六十四,應鐘之數四十二,蕤賓之數五十七,大呂之數七十六,夷則之數五十一,夾鐘之數六十八,無射之數四十五,中呂之數六十,極不生。以黃鐘為宮,太蔟為商,姑洗為角,林鐘為徽,南呂為羽。宮生徵,徵生商,商生羽,羽生角,角生應鐘,不比正音,故為和;應鐘生蕤賓,不比正音,故為繆。日冬至,音比林鐘浸以濁。日夏至,音比黃鐘浸以清。十二律應二十四時之變。甲子,中呂之徵也。丙子,夾鐘
 之羽也。戊子,黃鐘之宮也。庚子,無射之商也。壬子,夷則之角也。其為音也,一律而生五音,十二律而為六十音。因而六之,六六三十六,故三百六十音以當一歲之日。故律歷之數,天地之道也。



 司馬遷八書言律呂,粗舉大經,著於前史。則以太極元氣函三為一,而始動於子,十二律之生,必所起焉。於是參一於丑得三,因而九三之,舉本位合十辰,得一萬九千六百八十三,謂之成數,以為黃鐘之法。又參之律於十二辰,得十七萬七千一百四十七,謂之該數,以為黃鐘之實。實如法而一,得黃鐘之律長九寸,十一月冬至之氣應焉。蓋陰陽合德,氣鐘
 於子,而化生萬物,則物之生莫不函三。故十二律空徑三分,而上下相生,皆損益以三。其術則因黃鐘之長九寸,以下生者倍其實,三其法:以上生者,四其實,三其法。所以明陽下生陰,陰上生陽。



 起子,為黃鐘九寸,一。



 丑,三分之二。



 寅,九分之八。



 卯,二十七分之十六。



 辰,八十一分之六十四。



 巳,二百四十三分之一百二十八。



 午,七百二十九分之五百一十二。



 未,二千一百八十七分之一千二十四。



 申,六千五百六十一分之四千九十六。



 酉,一萬九千六百八十二分之八千一百九十二。



 戌,五萬九千四十九分之三萬二千七百六十八。



 亥,十七萬七千一百四十七分之六萬五千五百三十六。



 如是周十二辰,在六律為陽,則當位自得而下生陰,在六呂為陰,則得其所衡而上生於陽,推算之術無重上生之法也。所謂律取妻,呂生子,陰陽升降,律呂之大經
 也。而遷又言十二律之長,今依淮南九九之數,則蕤賓為重上。又言五音相生,而以宮生角,角生商,商生徵,徵生羽,羽生宮。求其理用,罔見通途。



 及元始中,王莽輔政,博徵通知鐘律者,考其音義,使羲和劉歆典領調奏。班固《漢書》採而志之,其序論雖博,而言十二律損益次第,自黃鐘長九寸,三分損一,下生林鐘,長六寸。三分益一,上生太蔟而左旋,八八為位。一上一下,終於無射,下生中呂。校其相生所得,與司馬遷正同。班固採以為志。



 元帝時,郎中京房知五音六十律之數,上使太子傅玄成、諫議大夫章雜試問房於樂府,房對:「受學於故小黃
 令焦延壽。六十律相生之法:以上生下,皆三生二;以下生上,皆三生四。陽下生陰,陰上生陽,終於中呂,而十二律畢矣。中呂上生執始,執始下生去滅。上下相生,終於南事,而六十律畢矣。夫十二律之變至於六十,猶八卦之變至於六十四也。宓犧作《易》,紀陽氣之初以為律法。建日冬至之聲,以黃鐘為宮,太蔟為商,姑洗為角,林鐘為徽,南呂為羽,應鐘為變宮,蕤賓為變徵,此聲氣之元,五音之正也。故各統一日,其餘以次運行,當日者各自為宮,而商角徽羽以類從焉。《禮運》曰「五聲、六律、十二管還相為宮」,此之謂也。以六十律分期之日,黃鐘自冬至始,及冬
 至而復,陰陽、寒燠、風雨之占生焉。於以檢攝群音,考其高下,茍非革木之聲,則無不有所合。《虞書》曰「律和聲,此之謂也。」



 京房又曰:「竹聲不可以度調,故作準以定數。準之狀如瑟,而長丈,十三絃,隱間九尺,以應黃鐘之律九寸。中央一絃,下有畫分寸,以為六十律清濁之節。」房言律詳於歆所奏,其術施行於史官,候部用之,文多不悉載。截管為律,吹以考聲,列以效氣,道之本也。術家以其聲微而體難知,其分數不明,故作準以代之。準之聲明暢易達,分寸又粗,然絃以緩急清濁,非管無以正也。均其中弦,令與黃鐘相得,案畫以求諸律,則無不如數而
 應者矣。《續漢志》具載其六十律準度數,其相生之次與《呂覽》、《淮南》同。



 漢章帝元和元年,待詔候鐘律殷肜上言:「官無曉六十律以準調音者。故待詔嚴崇具以準法教子男宣,原召宣補學官,主調樂器。」詔曰:「崇子學審曉律,別其族,協其聲者,審試。不得依託父學,以聾為聰。聲微妙,獨非莫知,獨是莫曉。以律錯吹,能知命十二律不失一,乃為能傳崇學耳。」試宣十二律,其二中,其四不中,其六不知何律,宣遂罷。自此律家莫能為準。



 靈帝熹平六年,東觀召典律者太子舍人張光等問準
 意,光等不知,歸閱舊藏,乃得其器。形制如房書,猶不能定其絃緩急。音,不可書以曉人,知之者欲教而無從,心達者體知而無師,故史官能辨清濁者遂絕。其可以相傳者,唯候氣而已。



 漢末紛亂,亡失雅樂。魏武時,河南杜夔精識音韻,為雅樂郎中,令鑄銅工柴玉鑄鐘,其聲均清濁多不如法,數毀改作,玉甚厭之,謂夔清濁任意,更相訴白於魏武王。魏武王取玉所鑄鐘雜錯更試,然後知夔為精,於是罪玉。



 泰始十年,中書監荀勖、中書令張華出御府銅竹律二十五具,部太樂郎劉秀等校試,其三具與杜夔及左延
 年律法同,其二十二具,視其銘題尺寸,是笛律也。問協律中郎將列和,辭:「昔魏明帝時,令和承受笛聲以作此律,欲使學者別居一坊,歌詠講習,依此律調。至於都合樂時,但識其尺寸之名,則絲竹歌詠,皆得均合。歌聲濁者用長笛長律,歌聲清者用短笛短律。凡弦歌調張清濁之制,不依笛尺寸名之,則不可知也。」



 勖等奏:「昔先王之作樂也,以振風蕩俗,饗神祐賢,必協律呂之和,以節八音之中。是故郊祀朝宴,用之有制,歌奏分獻,清濁有宜。故曰「五聲、十二律還相為宮」,此經傳記籍可得知者也。如和對辭,笛之長短無所象則,率意而
 作,不由曲度。考以正律,皆不相應;吹其聲均,多不諧合。又辭『先師傳笛,別其清濁,直以長短。工人裁制,舊不依律。』是為作笛無法。而和寫笛造律,又令琴瑟歌詠,從之為正,非所以稽古先哲,垂憲于後者也。謹條牒諸律,問和意狀如左。及依典制,用十二律造笛象十二枚,聲均調和,器用便利。講肄彈擊,必合律呂,況乎宴饗萬國,奏之廟堂者哉?雖伶夔曠遠,至音難精,猶宜儀形古昔,以求厥衷,合乎經禮,於制為詳。若可施用,請更部笛工選竹造作,下太樂樂府施行。平議諸杜夔、左延年律可皆留,其御府笛正聲、下徽各一具,皆銘題作者姓名,其餘
 無所施用,還付御府毀。」奏可。



 勖又問和:「作笛為可依十二律作十二笛,令一孔依一律,然後乃以為樂不?」和辭:「太樂東廂長笛正聲已長四尺二寸,今當復取其下徵之聲。於法,聲濁者笛當長,計其尺寸乃五尺有餘,和昔日作之,不可吹也。又,笛諸孔雖不校試,意謂不能得一孔輒應一律也。」案太樂四尺二寸笛正聲均應蕤賓,以十二律還相為宮,推法下徵之孔當應律大呂。大呂笛長二尺六寸有奇,不得長五尺餘。輒令太樂郎劉秀、鄧昊等依律作大呂笛以示和,又吹七律,一孔一校,聲皆相應。然後令郝生鼓箏,宋同吹
 笛,以為雜引、《相和》諸曲。和乃辭曰:「自和父祖漢世以來,笛家相傳,不知此法,而令調均與律相應,實非所及也。」郝生、魯基、種整、朱夏皆與和同。



 又問和:「笛有六孔,及其體中之空為七,和為能盡名其宮商角徵不?孔調與不調,以何檢知?」和辭:「先師相傳,吹笛但以作曲,相語為某曲當舉某指,初不知七孔盡應何聲也。若當作笛,其仰尚方笛工依案舊像訖,但吹取鳴者,初不復校其諸孔調與不調也。」案《周禮》調樂金石,有一定之聲,是故造鐘磬者先依律調之,然後施於廂懸。作樂之時,諸音皆受鐘磬之均,即
 為悉應律也。至於饗宴殿堂之上,無廂懸鐘磬,以笛有一定調,故諸絃歌皆從笛為正,是為笛猶鐘磬,宜必合於律呂。如和所對,直以意造,率短一寸,七孔聲均,不知其皆應何律,調與不調,無以檢正,唯取竹之鳴者,為無法制。輒部郎劉秀、鄧昊、王艷、魏邵等與笛工參共作笛,工人造其形,律者定其聲,然後器象有制,音均和協。



 又問和:「若不知律呂之義作樂,音均高下清濁之調,當以何名之?」和辭:「每合樂時,隨歌者聲之清濁,用笛有長短。假令聲濁者用三尺二笛,因名曰此三尺二調也;聲清者用二尺九笛,因名曰此二尺九調也。漢魏相傳,施行
 皆然。」案《周禮》奏六樂,乃奏黃鐘,歌大呂;乃奏太蔟,歌應鐘,皆以律呂之義,紀歌奏清濁。而和所稱以二尺,三尺為名,雖漢魏用之,俗而不典。部郎劉秀、鄧昊等以律作笛,三尺二寸者應無射之律,若宜用長笛,執樂者曰請奏無射;二尺八寸四分四厘應黃鐘之律,若宜用短笛,執樂者曰請奏黃鐘。則歌奏之義,若合經禮,考之古典,於制為雅。



 《書》曰:「予欲聞六律、五聲、八音,在治忽。」《周禮》、《國語》載六律六同,《禮記》又曰「五聲、十二律還相為宮」。劉歆、班固撰《律歷志》亦紀十二律,惟京房始創六十律。至章帝時,其法己
 絕,蔡邕追紀其言,亦曰今無能為者。依案古典及今音家所用,六十律者無施於樂。謹依典記,以五聲、十二律還相為宮之法,制十二笛象,記注圖側,如別,省圖,不如視笛之孔,故復重作蕤賓伏孔笛。其制云:



 黃鐘之笛,正聲應黃鐘,下徵應林鐘,長二尺八寸四分四厘有奇。正聲調法,以黃鐘為宮,則姑洗為角,翕笛之聲應姑洗,故以四角之長為黃鐘之笛也。其宮聲正而不倍,故曰正聲。



 正聲調法:黃鐘為宮,第一孔也。應鐘為變宮,第二孔也。南呂為羽,第三孔也。林鐘為徵,第四孔也。蕤賓為變徵,第五附孔也。姑洗為角,笛體中聲。太蔟為商。笛後出孔也。商聲濁於角,當在角下,而角聲以在體中,故上其商孔,令在宮上,清於宮也。然則宮商正也,餘聲皆倍也;是故從宮以下,孔轉下轉濁也。此章記笛孔上下次第之
 名也。下章說律呂相生,笛之制也。正聲調法,黃鐘為宮。作黃鐘之笛,將求宮孔,以始洗及黃鐘律,從笛首下度之,盡二律之長而為孔,則得宮聲也。宮生徵,黃鐘生林鐘也。以林鐘之律從宮孔下度之。盡律作孔,則得徵聲也。徵生商,林鐘生太蔟也。以太蔟律從徵孔上度之,盡律以為孔,則得商聲也。商生羽,太蔟生南呂也。以南呂律從商孔下度之,盡律為孔,則得羽聲也。羽生角,南呂生姑洗也。以姑洗律從羽孔上行度之,盡律而為孔,則得角聲也。然則出於商孔之上,吹笛者左手所不及也。從羽孔下行度之,盡律而為孔,亦得角聲,出於商附孔之下,則吹者右手所不逮也,故不作角孔。推而下之,復倍其均,是以角聲在笛體中,古之制也。音家舊法,雖一倍再倍,但令均同,適足為唱和之聲,無害於曲均故也。《國語》曰,匏竹利制,議宜,謂便於事用從宜者也。角生變宮,姑洗生應鐘也。上句所謂當為角孔而出於商上者,墨點識之,以應鐘律。從此點下行度之,盡律為孔,則得變宮之聲也。變宮生變徵,應鐘生蕤賓也。
 以蕤賓律從變宮下度之,盡律為孔,則得變徵之聲。十二笛之制,各以其宮為主,相生之法,或倍或半,其便事用,例皆一也。



 下徵調法:林鐘為宮,第四孔也。本正聲黃鐘之徵。徵清,當在宮上,用笛之宜,倍令濁下,故曰下徵。下徵更為宮者,《記》所謂「五聲,十二律還相為宮」也。然則正聲清,下徵為濁也。南呂為商,第三孔也。本正聲黃鐘之羽,今為下徵之商也。應鐘為角,第二孔也。本正聲黃鐘之變宮,今為下徵之角也。黃鐘為變徵,下徵之調,林鐘為宮,大呂當為變徵,而黃鐘笛本無大呂之聲,故假用黃鐘以為變徵也。假用之法,當為變徵之聲,則俱發黃鐘及太蔟、應鐘三孔。黃鐘應濁而太蔟清,大呂律在二律之間,俱發三孔而徵磑蒦之,則得大呂變徵之聲矣。諸笛下徵調求變徵之法,皆如此也。太蔟為徵,笛後出孔。本正聲之商,今為下徵之徵也。姑洗為羽,笛體中翕聲。本正聲之角,今為下徵之羽。蕤賓為變宮。附孔是也。本正聲之變徵也,今為下徵之變宮也。然則正聲之調,孔轉下轉濁,下徵之調,孔轉上轉清也。



 清角之調:以姑洗為宮,即是笛體中翕聲。於正聲為
 角,於下徵為羽。清角之調乃以為宮,而哨吹令清,故曰清角。惟得為宛詩謠俗之曲,不合雅樂也。蕤賓為商,正也。林鐘為角,非正也。南呂為變徵,非正也。應鐘為徵,正也。黃鐘為羽,非正也。太蔟為變宮。非正也。清角之調,唯宮、商及徵與律相應,餘四聲非正者皆濁,一律哨吹令清,假而用之,其例一也。



 凡笛體用律,長者八之,蕤賓、林鐘也。短者四之。其餘十笛,皆四角也。空中實容,長者十六。短笛竹宜受八律之黍也。若長短大小不合於此,或器用不便聲均法度之齊等也。然笛竹率上大下小,不能均齊,必不得已,取其聲均合。三宮,一曰正聲,二曰下徵,三曰清角也。二十一變也。宮有七聲,錯綜用之,故二十一變也。諸笛例皆一也。伏孔四,所以便事用也。一曰正角,出於商上者也,二曰倍角,近笛下者也,三曰變宮,近於宮孔,倍令下者也;四曰變徵,遠於徵孔,倍令高者也。或倍或半,或四分一,取則於琴徽也。四者皆不作其孔,而取
 其度,以應退上下之法,所以協聲均,便事用也。其本孔隱而不見,故曰伏孔也。



 大呂之笛,正聲應大呂,下徵應夷則,長二尺六寸六分三釐有奇。



 太蔟之笛,正聲應太蔟,下徵應南呂,長二尺五寸三分一釐有奇。



 夾鐘之笛,正聲應夾鐘,下徵應無射,長二尺四寸。



 姑洗之笛,正聲應姑洗,下徵應應鐘,長二尺二寸三分三釐有奇。



 蕤賓之笛,正聲應蕤賓,下徵應大呂,長三尺九寸九分五釐有奇。變宮近宮孔,故倍半令下,便於用也。林鐘亦如之
 一。林鐘之笛,正聲應林鐘,下徵應太蔟,長三尺七寸九分七釐有奇。



 夷則之笛,正聲應夷則,下徵應夾鐘,長三盡六寸。變宮之法,亦如蕤賓,體用四角,故四分益一也。



 南呂之笛,正聲應南呂,下徵姑洗,長三尺三寸七分有奇。



 無射之笛,正聲應無射,下徵應中呂,長三尺二寸。



 應鐘之笛,正聲應應鐘,下徵應蕤賓,長二尺九寸九分六釐有奇。



 五音十二律



 土音宮,數八十一,為聲之始。屬土者,以其最濁,君之象也。季夏之氣和,則宮聲調。宮亂則荒,其君驕。黃鐘之宮,律最長也。



 火音徵,三分宮去一以生,其數五十四。屬火者,以其徵清,事之象也。夏氣和,則徵聲調。徵亂則哀,其事勤也。



 金音商,三分徵益一以生,其數七十二。屬金者,以其濁次宮,臣之象也。秋氣和,則商聲調。商亂則詖,其官壞也。



 水音羽,三分商去一以生,其數四十八。屬水者,以為最清,物之象也。冬氣和,則羽聲調。羽亂則危,其財匱也。



 木音角,三分羽益一以生,其數六十四。屬木者,以其清
 濁中,人之象也。春氣和,則角聲調。角亂則憂,其人怨也。



 凡聲尊卑,取象五行,數多者濁,數少者清;大不過宮,細不過羽。



 十一月,律中黃鐘,律之始也,長九寸。仲冬氣至,則其律應,所以宣養六氣九德也。班固三分損一,下生林鐘。



 十二月,律中大呂,司馬遷未下生之律,長四寸二百四十三分寸之五十二,倍之為八寸二百四十三分寸之一百四。季冬氣至,則其律應,所以助宣物也。三分益一,上生夷則;京房三分損一,下生夷則。



 正月,律中太蔟,未上生之律,長八寸。孟春氣至,則其律
 應,所以贊陽出滯也。三分損一,下生南呂。



 二月,律中夾鐘,酉下生之律,長三寸二千一百八十七分寸之一千六百三十一,倍之為七寸二千一百八十七分寸之一千七十五。仲春氣至,則其律應,所以出四隙之細也。三分益一,上生無射;京房三分損一,下生無射。



 三月,律中姑洗,酉上生之律,長七寸九分寸之一。季春氣至,則其律應,所以修絜百物,考神納賓也。三分損一,下生應鐘。



 四月,律中中呂,亥下生之律,長三寸萬九千六百八十三分寸之六千四百八十七,倍之為六寸萬九千六百八十三分寸之萬二
 千九百七十四。孟夏氣至,則其律應,所以宣中氣也。



 五月,律中蕤賓,亥上生之律,長六寸八十一分寸之二十六。仲夏氣至,則其律應,所以安靜人神,獻酬交酢也。三分損一,下生大呂;京房三分益一,上生大呂。



 六月,律中林鐘,丑下生之律,長六寸。季夏氣至,則其律應,所以和展百物,俾莫不任肅純恪也。三分益一,上生太蔟。



 七月,律中夷則,丑上生之律,長五寸七百二十九分寸之四百五十一。孟秋氣至,則其律應,所以詠歌九則,平百姓而無貸也。三分損一,下生夾鐘;京房三分益一,上
 生夾鐘。



 八月,律中南呂,卯下生之律,長五寸三分寸之一。仲秋氣至,則其律應,所以贊陽秀也。三分益一,上生姑洗。



 九月,律中無射,卯上生之律,長四寸六千五百六十一分寸之六千五百二十四。季秋氣至,則其律應,所以宣布哲人之令德,示人軌儀也。三分損一,下生中呂;京房三分益一,上生中呂。



 十月,律中應鐘,巳下生之律,長四寸二十七分寸之二十。孟冬氣至,則其律應,所以均利器用,俾應復也。三分益一,上生蕤賓。



 淮南、京房、鄭玄諸儒言律歷,皆上下相生,至蕤賓又重上生大呂,長八寸二百四十三分寸之百四;夷則上生夾鐘,長七寸千一百八十七分寸之千七十五;無射上生中呂,長六寸萬九千六百八十三分寸之萬二千九百七十四;此三品於司馬遷、班固所生之寸數及分皆倍焉,餘則並同。斯則泠州鳩所謂六間之道,揚沈伏,黜散越,假之為用者也。變通相半,隨事之宜,贊助之法也。凡音聲之體,務在和均,益則加倍,損則減半,其於本音恒為無爽。然則言一上一下者,相生之道;言重上生者,吹候之用也。於蕤賓重上生者,適會為用之數,故言律
 者因焉,非相生之正也。



 楊子雲曰:「聲生於日,謂甲己為角,乙庚為商,丙辛為徵,丁壬為羽,戊癸為宮也。律生於辰,謂子為黃鐘,醜為大呂之屬也。聲以情質,質,正也。各以其行本情為正也。律以和聲,當以律管鐘均和其清濁之聲。聲律相協而八音生。協,和也。」宮、商、角、徵、羽,謂之五聲。金、石、匏、革、絲、竹、土、木,謂之八音。聲和音諧,是謂五樂。



 夫陰陽和則景至,律氣應則灰除。是故天子常以冬夏至日御前殿,合八能之士,陳八音,聽樂均,度晷景,候鐘律,權土灰,效陰陽,冬至陽氣應則灰除,是故樂均清,景長極,黃鐘通,土灰輕而衡仰。夏至陰氣應則樂均濁,景短極,蕤賓通,土灰重而衡低。進退於先後五
 日之中,八能各以候狀聞,太史令封上。效則和,否則占。



 候氣之法,為室三重,戶閉,塗釁周密,布緹幔。室中以木為案,每律各一,內房中外高,從其方位,加律其上,以葭莩灰抑其內端,案歷而候之:氣至者灰去;其為氣所動者,其灰散;人及風所動者,其灰聚。殿中候用玉律十二,惟二至乃候。靈臺用竹律。楊泉記云:「取弘農宜陽縣金門山竹為管,河內葭莩為灰。」或云以律著室中,隨十二辰埋之,上與地平,以竹莩灰實律中,以羅縠覆律呂,氣至吹灰動縠。小動為和,大動,君弱臣強;不動,君嚴暴之應也。



 審度



 起度之正,《漢志》言之詳矣。武帝泰始九年,中書監荀勖校太樂,八音不和,始知後漢至魏,尺長於古四分有餘。勖乃部著作郎劉恭依《周禮》制尺,所謂古尺也。依古尺更鑄銅律呂,以調聲韻。以尺量古器,與本銘尺寸無差。又,汲郡盜發六國時魏襄王塚,得古周時玉律及鐘、磬,與新律聲韻闇同。于時郡國或得漢時故鐘,吹律命之皆應。勖銘其尺曰:「晉泰始十年,中書考古器,揆校今尺,長四分半。所校古法有七品:一曰姑洗玉律,二曰小呂玉律,三曰西京銅望臬,四曰金錯望臬,五曰銅斛,六曰
 古錢,七曰建武銅尺。姑洗微彊,西京望臬微弱,其餘與此尺同。」銘八十二字。此尺者勖新尺也,今尺者杜夔尺也。



 荀勖造新鐘律,與古器諧韻,時人稱其精密,惟散騎侍郎陳留阮咸譏其聲高,聲高則悲,非興國之音,亡國之音。亡國之音哀以思,其人困。今聲不合雅,懼非德正至和之音,必古今尺有長短所致也。會咸病卒,武帝以勖律與周漢器合,故施用之。後始平掘地得古銅尺,歲久欲腐,不知所出何代,果長勖尺四分,時人服咸之妙,而莫能厝意焉。



 史臣案:「勖於千載之外,推百代之法,度數既宜,聲韻又
 契,可謂切密,信而有徵也。而時人寡識,據無聞之一尺,忽周漢之兩器,雷同臧否,何其謬哉!《世說》稱「有田父於野地中得周時玉尺,便是天下正尺,荀勖試以校己所治金石絲竹,皆短校一米」。又,漢章帝時,零陵文學史奚景於泠道舜祠下得玉律,度以為尺,相傳謂之漢官尺。以校荀勖尺,勖尺短四分;漢官、始平兩尺,長短度同。又,杜夔所用調律尺,比勖新尺,得一尺四分七厘。魏景元四年,劉徽注《九章》云:王莽時劉歆斛尺弱於今尺四分五厘,比魏尺其斛深九寸五分釐;即荀勖所謂今尺長四分半是也。元帝後,江東所用尺,比荀勖尺一尺六
 分二釐。趙劉曜光初四年鑄渾儀,八年鑄土圭,其尺比荀勖尺一尺五分。荀勖新尺惟以調音律,至於人間未甚流布,故江左及劉曜儀表,並與魏尺略相依準。



 嘉量



 《周禮》:「栗氏為量,鬴深尺,內方尺而圓其外,其實一鬴。其臀一寸,其實一豆。其耳三寸,其實一升。重一鈞,其聲中黃鐘。概而不稅。其銘曰:『時文思索,允臻其極。嘉量既成,以觀四國。永啟厥後,茲器維則。』」《春秋左氏傳》曰:「齊舊四量,豆、區、釜、鐘。四升曰豆,各自其四,以登於釜。」四豆為區,區斗六升也。四區為釜,六斗四升也。釜十則鐘,六十四
 斗也。鄭玄以為釜方尺,積千寸,比《九章粟米法》少二升八十一分升之二十二。以算術考之,古斛之積凡一千五百六十二寸半,方尺而圓其外,減傍一釐八毫,其徑一尺四寸一分四毫七秒二忽有奇,而深尺,即古斛之制也。



 《九章商功法》程粟一斛,積二千七百寸;米一斛,積一千六百二十七寸;菽荅麻麥一斛,積二千四百三十寸。此據精麤為率,使價齊,而不等其器之積寸也。以米斛為正,則同于《漢志》。魏陳留王景元四年,劉徽注《九章商功》曰:「當今大司農
 斛,圓徑一尺三寸五分五釐,深一尺,積一千四百四十一寸十分寸之三。王莽銅斛,於今尺為深九寸五分五釐,徑一尺三寸六分八釐七毫。以徽術計之,於今斛為容九斗七升四合有奇。」魏斛大而尺長,王莽斛小而尺短也。



 衡權



 衡權者,衡,平也;權,重也。衡所以任權而均物,平輕重也。古有黍、壘、錘、錙、鐶、鈞、鋝、鎰之目,歷代參差。《漢志》言衡權名理甚備,自後變更,其詳未聞。元康中,裴頠以為醫方人命之急,而稱兩不與古同,為害特重,宜因此改治權
 衡,不見省。趙石勒十八年七月,造建德殿,得圓石,狀如水碓,銘曰:「律權石,重四鈞,同律度量衡。有辛氏造。」續咸議,是王莽時物。



\end{pinyinscope}