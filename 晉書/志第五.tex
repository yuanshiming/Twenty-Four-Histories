\article{志第五}

\begin{pinyinscope}
地理下
 \gezhu{
  青州徐州荊州揚州交州
  廣州}



 青州。案《禹貢》為海岱之地,舜置十二牧,則其一也。舜以青州越海,又分為營州,則遼東本為青州矣。《周禮》:「正東曰青州。」蓋取土居少陽,其色為青,故以名也。《春秋元命包》云:「虛危流為青州。」漢武帝置十三州,因舊名,歷後漢至晉不改。州統郡國六,縣三十七,戶五萬三千。



 齊國秦置郡,漢以為國。景帝以為北海郡。統縣五,戶一萬四千。



 臨淄西安有棘里亭。東安平汝水出東北。廣饒昌國樂毅所封。



 濟南郡漢置。統縣五,戶五千。或云魏平蜀,徙其豪將家於濟河北,故改為濟岷郡。而《太康地理志》無此郡名,未之詳。



 平壽古國。寒浞封此。下密有三石祠。膠東侯國。即墨有天山祠。祝阿



 樂安國漢置。統縣八,戶一萬一千。



 高苑臨濟有蚩尤祠。博昌有薄姑祠。利益侯相。蓼城侯國。



 鄒壽光古斟灌氏所封國。東朝陽



 城陽郡漢置,屬北海,自魏至晉,分北海而立焉。郡統縣十,戶一萬二千。



 莒故莒子國。姑幕古薄姑氏國。諸淳于故淳于公國。東武



 高密漢改為郡。壯武黔陬平昌昌安



 東萊國漢置郡。統縣六,戶六千五百。



 掖侯相。當利侯國。廬鄉曲城黃有萊山、松林萊君祠弦侯國。有百支萊王祠。



 長廣郡咸寧三年置。統縣三,戶四千五百。



 不其侯國。長廣挺



 惠帝元康十年,又置平昌郡。又分城陽之黔陬、壯武、淳于、昌安、高密、平昌、營陵、安丘、大、劇、臨朐十一縣為高密國。自永嘉喪亂,青州淪沒石氏。東萊人曹嶷為刺史。造廣固城,後為石季龍所滅。季龍末,遼西段龕自號齊王,據青州。慕容恪滅趙,剋青州。苻氏平燕,盡有其地。及苻氏敗後,刺史苻朗以州降。朝廷置幽州,以別駕辟閭渾
 為刺史,鎮廣固。隆安四年,為慕容德所滅,遂都之,是為南燕,復改為青州。德以并州牧鎮陰平,幽州刺史鎮發干,徐州刺史鎮莒城,青州刺史鎮東萊,兗州刺史鎮梁父。慕容超移青州於東萊郡,後為劉裕所滅,留長史羊穆之為青州刺史,築東陽城而居之。自元帝渡江,於廣陵僑置青州。至是始置北青州,鎮東陽城,以僑立州為南青州。而後省南青州,而北青州直曰青州。



 徐州。案《禹貢》海岱及淮之地,舜十二牧,則其一也。於周入青州之域。《春秋元命包》云:「天氐流為徐州。」蓋取舒緩之義,或云因徐丘以立名。秦兼天下,以置泗水、薛、瑯
 邪三郡。楚漢之際,分置東陽郡。漢又分置東海郡,改泗水為沛,改薛為魯,分沛置楚國,以東陽屬吳國。景帝改吳為江都,武帝分沛、東陽置臨淮郡,改江都為廣陵。及置十三州,以其地為徐州,統楚國及東海、瑯邪、臨淮、廣陵四郡。宣帝改楚為彭城郡,後漢改為彭城國,以沛郡之廣戚縣來屬,改臨淮為下邳國。及太康元年,復分下邳屬縣在淮南者置臨淮郡,分瑯邪置東莞郡。州凡領郡國七,縣六十一,戶八萬一千二十一。



 彭城國漢以為郡。統縣七,戶四千一百二十一。



 彭城故殷伯太彭國。留張良所封。廣戚傅陽武原呂梧



 下邳國漢置為臨淮郡。統縣七,戶七千五百。



 下邳葛嶧山在西,古嶧陽也。韓信為楚王,都之。凌良城侯相。睢陵夏丘取慮僮



 東海郡漢置。統縣十二,戶一萬一千一百。



 郯故郯子國。祝其羽山在縣之西。朐襄賁利城贛榆厚丘蘭陵承昌慮合鄉戚



 琅邪國秦置郡。統縣九,戶二萬九千五百。



 開陽侯相。臨沂陽都繒即丘華費魯季氏邑。東安蒙陰山在西南。



 東莞郡太康中置。統縣八,戶一萬。



 東莞故魯鄆邑。朱虛營陵尚父呂望所封。安丘故莒渠丘父封邑。



 蓋臨朐有海水祠。劇廣



 廣陵郡漢置。統縣八,戶八千八百。



 淮陰射陽輿海陵有江海會祠。廣陵鹽瀆淮浦江都有江水祠。



 臨淮郡漢置,章帝以合下邳,太康元年復立。統縣十,戶一萬。



 盱眙東陽高山贅其潘旌高郵淮陵司吾下相徐



 太康十年,以青州城陽郡之莒、姑幕、諸、東武四縣屬東莞。元康元年,分東海置蘭陵郡。七年,又分東莞置東安
 郡,分臨淮置淮陵郡,以堂邑置堂邑郡。永嘉之亂,臨淮、淮陵並淪沒石氏。元帝渡江之後,徐州所得惟半,乃僑置淮陽、陽平、濟陰、北濟陰四郡。又瑯邪國人隨帝過江者,遂置懷德縣及琅邪郡以統之。是時,幽、冀、青、並、兗五州及徐州之淮北流人相帥過江淮,帝並僑立郡縣以司牧之。割吳郡之海虞北境,立郯、朐、利城、祝其、厚丘、西隰、襄賁七縣,寄居曲阿,以江乘置南東海、南瑯邪、南東平、南蘭陵等郡,分武進立臨淮、淮陵、南彭城等郡,屬南徐州,又置頓丘郡屬北徐州。明帝又立南沛、南清河、南下邳、南東莞、南平昌、南濟陰、南濮陽、南太平、南泰山、南
 濟陽、南魯等郡,以屬徐、兗二州,初或居江南,或居江北,或以兗州領州。郗鑒都督青兗二州諸軍事、兗州刺史,加領徐州刺史,鎮廣陵。蘇峻平後,自廣陵還鎮京口。又於漢故九江郡界置鐘離郡,屬南徐州,江北又僑立幽、冀、青、並四州。穆帝時,移南東海七縣出居京口。義熙七年,始分淮北為北徐州,淮南但為徐州,統彭城、沛、下邳、蘭陵、東莞、東安、瑯邪、淮陽、陽平、濟陰、北濟陰十一郡,以盱眙立盱眙郡,統考城、直瀆、陽城三縣,又分廣陵界置海陵、山陽二郡。後又以幽冀合徐州,青並合兗州。



 荊州。案《禹貢》荊及衡陽之地,舜置十二牧,則其一也。《
 周禮》:「正南曰荊州。」《春秋元命包》云:「軫星散為荊州。」荊,強也,言其氣躁強。亦曰警也,言南蠻數為寇逆,其人有道後服,無道先強,常警備也。又云取名於荊山。六國時,其地為楚。及秦,取楚鄢郢為南郡,又取巫中地為黔中郡,以楚之漢北立南陽郡,滅楚之後,分黔中為長沙郡。漢高祖分長沙為桂陽郡,改黔中為武陵郡,分南郡為江夏郡。武帝又分長沙為零陵郡。及置十三州,因舊名為荊州,統南郡、南陽、零陵、桂陽、武陵、長沙、江夏七郡。後漢獻帝建安十三年,魏武盡得荊州之地,分南郡以北立襄陽郡,又分南陽西界立南鄉郡,分枝江以西立臨江
 郡。及敗於赤壁,南郡以南屬吳,吳後遂與蜀分荊州。於是南郡、零陵、武陵以西為蜀,江夏、桂陽、長沙三郡為吳,南陽、襄陽、南鄉三郡為魏。而荊州之名,南北雙立。蜀分南郡,立宜都郡,劉備沒後,宜都、武陵、零陵、南郡四郡之地悉復屬吳。魏文帝以漢中遺黎立魏興、新城二郡,明帝分新城立上庸郡。孫權分江夏立武昌郡,又分蒼梧立臨賀郡,分長沙立衡陽、湘東二郡。孫休分武陵立天門郡,分宜都立建平郡。孫皓分零陵立始安郡,分桂陽立始興郡,又分零陵立邵陵郡,分長沙立安成郡。荊州統南郡、武昌、武陵、宜都、建平、天門、長沙、零陵、桂陽、衡陽、
 湘東、邵陵、臨賀、始興、始安十五郡,其南陽、江夏、襄陽、南鄉、魏興、新城、上庸七郡屬魏之荊州。及武帝平吳,分南郡為南平郡,分南陽立義陽郡,改南鄉順陽郡,又以始興、始安、臨賀三郡屬廣州,以揚州之安成郡來屬。州統郡二十二,縣一百六十九,戶三十五萬七千五百四十八。



 江夏郡漢置。統縣七,戶二萬四千。



 安陸橫尾山在東北,古之陪尾山。雲杜故雲子國。曲陵平春



 邑阜竟陵章山在東北,古之方山。南新市



 南郡漢置。統縣十一,戶五萬
 五千。



 江陵故楚都。編有雲夢官。當陽華容鄀故鄀子國。枝江故羅國。旌陽州陵楚嬖人州侯所邑。監利松滋石首



 襄陽郡魏置。統縣八,戶二萬二千七百。



 宜城故鄢也。中廬臨沮荊山在東北。巳阜襄陽侯相。山都鄧城JM



 南陽國秦置郡。統縣十四,戶二萬四千四百。



 宛西鄂侯相。雉魯陽公國相。犨淯陽公國相。博望公國相。堵陽葉侯相。有長城山,號曰方城。



 舞陰公國相。比陽公國相。涅陽冠軍酈



 順陽郡太康中置。統八縣,戶二萬一
 百。



 酂順陽南鄉丹水武當侯相。陰築陽析



 義陽郡太康中置。統縣十二,戶一萬九千。



 新野侯相。穰鄧故鄧侯國。蔡陽隨故隨國。安昌棘陽厥西平氏桐柏山在南。義陽平林朝陽



 新城郡魏置。統縣四,戶一萬五千二百。



 房陵綏陽昌魏沶鄉



 魏興郡魏置。統縣六,戶一萬二千。



 晉興安康西城錫長利洵陽



 上庸郡魏置。統縣六,戶一萬一千四百四十八。



 上庸侯相。安富北巫武陵上廉微陽



 建平郡吳、晉各有建平郡,太康元年合。統縣八,戶一萬三千二百。



 巫北井秦昌信陵興山建始秭歸故楚子國。沙渠



 宜都郡吳置。統縣三,戶八千七百。



 夷陵夷道佷山



 南平郡吳置,以為南郡,太康元年改曰南平。統縣四,戶七千。



 作唐孱陵南安江安



 武陵郡漢置。統縣十,戶一萬四千。



 臨沅龍陽漢壽沅陵黚陽酉陽鐔城沅南遷陵



 舞陽



 天門郡吳置。統縣五,戶三千一百。



 零陽漊中袞臨澧澧陽



 長沙郡漢置。統縣十,戶三萬三千。



 臨湘攸下雋醴陵劉陽建寧吳昌羅蒲沂巴陵



 衡陽郡吳置,故屬長沙。統縣九,戶二萬一千。



 湘鄉重安湘南湘西烝陽衡山連道新康益陽



 湘東郡吳置,故屬長沙。統縣七,戶一萬九千五百。



 酃茶陵臨烝利陽陰山新平新寧



 零陵郡漢置。統縣十一,戶二萬五千一百。



 泉陵有香茅,云古貢之以縮酒。祁陽零陵營浦洮陽永昌觀陽營道春陵泠道應陽東界有鼻墟,云象所封。



 邵陵郡吳置。統縣六。戶一萬二千。



 邵陵都梁夫夷建興邵陽高平



 桂陽郡漢置。統縣六,戶一萬一千三百。



 郴項羽義帝之邑。耒陽便臨武晉寧南平



 武昌郡吳置。統縣七,戶一萬四千八百。



 武昌故東鄂也。楚子熊渠封中子紅於此。柴桑。有湓口關。陽新沙羨有夏口,對沔口,有津。沙陽鄂有新興、馬頭鐵官官陵



 安成郡吳置。統縣七,戶三千。



 平都宜春新諭永新安復萍鄉廣興



 惠帝分桂陽、武昌、安成三郡立江州,以新城、魏興、上庸三郡屬梁州,又分義陽立隨郡,分南陽立新野郡,分江夏立竟陵郡。懷帝又分長沙、衡陽、湘東、零陵、邵陵、桂陽及廣州之始安、始興、臨賀九郡置湘州。時蜀亂,又割南郡之華容、州陵、監利三縣別立豐都,合四縣置成都郡,為成都王穎國,居華容縣。愍帝建興中,併還南郡,亦併豐都於監利。元帝渡江,又僑立新興、南河東二郡。穆帝時,又分零陵立營陽郡,以義陽流人在南郡者立為義
 陽郡。又以廣州之臨賀、始興、始安三郡及江州之桂陽,益州之巴東,合五郡來屬,以長沙、衡陽、湘東、零陵、邵陵、營陽六郡屬湘州。桓溫又分南郡立武寧郡。安帝又僑立南義陽、東義陽、長寧三郡。義熙十三年,省湘州,長沙、衡陽、湘東、零陵、邵陵、營陽還入荊州。



 揚州。案《禹貢》淮海之地,舜置十二牧,則其一也。《周禮》:「東南曰揚州。」《春秋元命包》云:「牽牛流為揚州,分為越國。」以為江南之氣躁勁,厥性輕揚。亦曰,州界多水,水波揚也。於古則荒服之國,戰國時其地為楚分。秦始皇並天下,以置鄣、會稽、九江三郡。項羽封英布為九江王,盡有
 其地。漢改九江曰淮南,即封布為淮南王。六年,分淮南置豫章郡。十一年,布誅,立皇子長為淮南王,封劉濞為吳王,二國盡得揚州之地。文帝十六年,分淮南立廬江、衡山二郡。景帝四年,封皇子非為江都王,並得鄣、會稽郡,而不得豫章。武帝改江都曰廣陵,封皇子胥為王而以屬徐州。元封二年,改鄣曰丹楊,改淮南復為九江。後漢順帝分會稽立吳郡,揚州統會稽、丹楊、吳、豫章、九江、廬江六郡,省六安並廬江郡。獻帝興平中,孫策分豫章立廬陵郡。孫權又分豫章立鄱陽郡,分丹楊立新都郡。孫亮又分豫章立臨川郡,分會稽立臨海郡。孫休又分
 會稽立建安郡。孫皓分會稽立東陽郡,分吳立吳興郡,分豫章、廬陵、長沙立安成郡,分廬陵立廬陵南部都尉,揚州統丹楊、吳、會稽、吳興、新都、東陽、臨海、建安、豫章、鄱陽、臨川、安成、廬陵南部十四郡。江西廬江、九江之地,自合肥之北至壽春悉屬魏。及晉平吳,以安成屬荊州,分丹楊之宣城、宛陵、陵陽、安吳、涇、廣德、寧國、懷安、石城、臨城、春穀十一縣立宣城郡,理宛陵,改新都曰新安郡,改廬陵南部為南康郡,分建安立晉安郡,又分丹楊立毗陵郡。揚州合統郡十八,縣一百七十三,戶三十一萬一千四百。



 丹陽郡漢置。統縣十一,戶五萬一千五百。



 建鄴本秣陵,孫氏改為建業。武帝平吳,以為秣陵。太康三年,分秣陵北為建鄴,改業為鄴。江寧太康二年,分建鄴置。丹楊丹楊山多赤柳,在西也。于湖蕪湖永世溧陽溧水所出。江乘句容有茅山。



 湖熟秣陵



 宣城郡太康二年置。統縣十一,戶二萬三千五百。



 宛陵侯相。彭澤聚在西南。宣城陵陽淮水出東北入江。仙人陵陽子明所居。安吳臨城石城涇春穀孝武改春為陽。



 廣德寧國懷安



 淮南郡秦置九江郡。漢以為淮南國,漢武帝置為九江郡。武帝改為淮南郡。統縣十六,戶三萬三千四百。



 壽春成德下蔡義城西曲陽平阿有塗
 山。歷陽全椒阜陵漢明帝時淪為麻湖。鐘離故州來邑。合肥逡遒陰陵當塗古塗山國。東城烏江



 廬江郡漢置。統縣十,戶四千二百。



 陽泉舒故國,有桐鄉。灊天柱山在南,有祠。皖尋陽居巢桀死於此。臨湖襄安龍舒六故六國。



 毗陵郡吳分會稽無錫已西為屯田,置典農校尉。太康二年,省校尉為毗陵郡。統縣七,戶一萬二千。



 丹徒故朱方。曲阿故雲陽。武進延陵毗陵既陽無錫有磨山、春申君祠。



 吳郡漢置。統縣十一,戶二萬五千。



 吳故國。具區在西。嘉興海鹽鹽官錢唐武林山、武林水所出。富陽
 桐廬建德壽昌海廬婁



 吳興郡吳置。統縣十,戶二萬四千。



 烏程臨安餘杭武康。故防風氏國。東遷於潛有潛水。



 故鄣安吉原鄉長城



 會稽郡秦置。統縣十,戶三萬。



 山陰會稽山在南,上有禹冢。上虞有仇亭,舜避丹朱於此地。餘姚有句餘山在南。句章鄞有鮚埼亭。鄮始寧剡永興諸暨



 東陽郡吳置。統縣九,戶一萬二千。



 長山有赤松子廟。永康烏傷吳寧太末信安豐安定陽遂昌



 新安郡吳置。統縣六,戶五千。



 始新遂安黝歙海寧黎陽



 臨海郡吳置。統縣八,戶一萬八千。



 章安臨海始豐永寧寧海松陽安固橫陽



 建安郡故秦閩中郡,漢高帝五年以立閩越王。及武帝滅之,徙其人,名為東冶,又更名東城。後漢改為候官都尉,及吳置建安郡。統縣七,戶四千三百。



 建安吳興東平建陽將樂邵武延平



 晉安郡太康三年置。統縣八,戶四千三百。



 原豐新羅宛平同安候官羅江晉安
 溫麻



 豫章郡漢置。統縣十六,戶三萬五千。



 南昌海昏新淦建城望蔡永脩建昌吳平豫章彭澤艾康樂豐城新吳宜豐鐘陵



 臨川郡吳置。統縣十,戶八千五百。



 臨汝西豐南城東興南豐永成宜黃安浦西寧新建



 鄱陽郡吳置。統縣八,戶六千一百。



 廣晉鄱陽樂安餘汗鄡陽歷陵葛陽
 晉興



 廬陵郡吳置。統縣十,戶一萬二千二百。



 西昌高昌石陽巴丘南野東昌遂興吉陽興平陽豐



 南康郡太康三年置。統縣五,戶一千四百。



 贛雩都平固南康揭陽



 惠帝元康元年,有司奏,荊、揭二州疆土廣遠,統理尤難,於是割揚州之豫章、鄱陽、廬陵、臨川、南康、建安、晉安、荊州之武昌、桂陽、安成,合十郡,因江水之名而置江州。永興元年,分廬江之尋陽、武昌之柴桑二縣置尋陽郡,屬
 江州,分淮南之烏江、歷陽二縣置歷陽郡。又以周創義討石冰,割吳興之陽羨並長城縣之北鄉置義鄉、國山、臨津並陽羨四縣,又分丹陽之永世置平陵及永世,凡六縣,立義興郡,以表紀之功,並屬揚州。又以毗陵郡封東海王世子毗,避毗諱,改為晉陵。懷帝永嘉元年,又以豫章之彭澤縣屬尋陽郡。愍帝立,避帝諱改建鄴為建康。元帝渡江,建都揚州,改丹陽太守為尹,江州又置新蔡郡。尋陽郡又置九江、上甲二縣,尋又省九江縣入尋陽。是時司、冀、雍、涼、青、並、兗、豫、幽、平諸州皆淪沒,江南所得但有揚、荊、湘、江、梁、益、交、廣,其徐州則有過半,豫州
 惟得譙城而已。明帝太寧元年,分臨海立永嘉郡,流永寧、安固、松陽、橫陽等四縣,而揚州統丹陽、吳郡、吳興、新安、東陽、臨海、永嘉、宣城、義興、晉陵十一郡。



 自中原亂離,遺黎南渡,並僑置牧司在廣陵,丹徒南城,非舊土也。及胡寇南侵,淮南百姓皆渡江。成帝初,蘇峻、祖約為亂於江淮,胡寇又大至,百姓南渡者轉多,乃於江南僑立淮南郡及諸縣,又於尋陽僑置松滋郡,遙隸揚州。咸康四年,僑置魏郡、廣川、高陽、堂邑等諸郡,并所統縣并寄居京邑,改陵陽為廣陽。孝武寧康二年,又分永嘉郡之永寧縣置樂成縣。是時上黨百姓南渡,僑立上黨郡為四
 縣,寄居蕪湖。尋又省上黨郡為縣,又罷襄城郡為繁昌縣,並以屬淮南。安帝義熙八年,省尋陽縣入柴桑縣,柴桑仍為郡,後又省上甲縣入彭澤縣。舊江州督荊州之竟陵郡,及何無忌為刺史,表以竟陵去州遼遠,去江陵三百里,荊州所立綏安郡人戶入境,欲資此郡助江濱戍防,以竟陵郡還荊州。又司州之弘農、揚州之松滋二郡寄在尋陽,人戶難居,並宜建督。安帝從之。後又省松滋郡為松滋縣,弘農郡為弘農縣,並屬尋陽郡。



 交州。案《禹貢》揚州之域,是為南越之土。秦始皇即略定揚越,以謫戍卒五十萬人守五嶺。自北徂南,入越之
 道,必由嶺嶠,時有五處,故曰五嶺。後使任囂、趙他攻越,略取陸梁地,遂定南越,以為桂林、南海、象等三郡,非三十六郡之限,乃置南海尉以典之,所謂東南一尉也。漢初,以嶺南三郡及長沙、豫章封吳芮為長沙王。十一年,以南武侯織為南海王。陸賈使還,拜趙他為南越王,割長沙之南三郡以封之。武帝元鼎六年,討平呂嘉,以其地為南海、蒼梧、鬱林、合浦、日南、九真、交趾七郡,蓋秦時三郡之地。元封中,又置儋耳、珠崖二郡,置交趾刺史以督之。昭帝始元五年,罷儋耳并珠崖。元帝初元三年,又罷珠崖郡。後漢馬援平定交部,始調立城郭置井邑。順
 帝永和九年,交趾太守周敞求立為州,朝議不許,即拜敞為交趾刺史。桓帝分立高興郡,靈帝改曰高涼。建安八年,張津為刺史,土燮交趾太守,共表立為州,乃拜津為交州牧。十五年,移居番禺,詔以邊州使持節,郡給鼓吹,以重城鎮,加以九錫六佾之舞。吳黃武五年,割南海、蒼梧、鬱林三郡立廣州,交趾、日南、九真、合浦四郡為交州。戴良為刺史,值亂不得入,呂岱擊平之,復還並交部。赤烏五年,復置珠崖部。永安七年,復以前三郡立廣州。及孫皓,又立新昌、武平、九德三郡。蜀以李恢為建寧太守,遙領交州刺史。晉平蜀,以蜀建寧太守霍弋遙領
 交州,得以便宜選用長吏。平吳後,省珠崖入合浦。交州統郡七,縣五十三,戶二萬五千六百。



 合浦郡漢置。統縣六,戶二千。



 合浦南平蕩昌徐聞毒質珠官



 交趾郡漢置。統縣十四,戶一萬二千。



 龍編茍漏望海[B231]婁西于武寧朱鳶曲易



 交興北帶稽徐安定南定海平



 新昌郡吳置。統縣六,戶三千。



 麋泠婦人徵側為主處,馬援平之。嘉寧吳定封山臨西西道



 武平郡吳置。統縣七,戶五千。



 武寧武興進山根寧安武扶安封溪



 九真郡漢置。統縣七,戶三千。



 胥浦移風津梧建初常樂扶樂松原



 九德郡吳置,周時越常氏地。統縣八,無戶。



 九德咸驩南陵陽遂扶苓曲胥浦陽都洨



 日南郡秦置象郡,漢武帝改名焉。統縣五,戶六百。



 象林自此南有四國,其人皆云漢人子孫,今有銅柱,亦是漢置此為界。貢金供稅也。盧容象郡所居。朱吾西卷比景



 廣州。案《禹貢》揚州之域,秦末趙他所據之地。及漢武帝,以其地為交址郡。至吳黃武五年,分交州之南海、蒼梧、鬱林、高梁四郡立為廣州,俄復舊。永安六年,復分交州置廣州,分合浦立合浦北部,以都尉領之。孫皓分鬱林立桂林郡。及太康中,吳平,遂以荊州始安、始興、臨賀三郡來屬。合統郡十,縣六十八,戶四萬三千一百二十。



 南海郡秦置。統縣六,戶九千五百。



 番禺四會增城博羅龍川平夷



 臨賀郡吳置。統縣六,戶二千五百。



 臨賀謝沐馮乘封陽興安富
 川



 始安郡吳置。統縣七,戶六千。



 始安始陽平樂荔浦常安熙平永豐



 始興郡吳置。統縣七,戶五千。



 曲江桂陽始興含洭湞陽中宿陽山



 蒼梧郡漢置。統縣十二,戶七千七百。



 廣信端溪高要建陵新寧猛陵鄣平農城元谿臨允都羅武城



 鬱林郡秦置桂林郡,武帝更名。統縣九,戶六千。



 布山阿林新邑晉平始建鬱平領方武熙安廣



 桂林郡吳置。統縣八,戶二千。



 潭中武豐粟平羊平龍剛夾陽武城軍騰



 高涼郡吳置。統縣三,戶二千。



 安寧高涼思平



 高興郡吳置。統縣五,戶一千二百。



 廣化海安化平黃陽西平



 寧浦郡吳置。統縣五,戶一千二百二十。



 寧浦連道吳安昌平平山



 武帝後省高興郡。懷帝永嘉元年,又以臨賀、始興、始安
 三郡凡二十縣為湘州。元帝分鬱林立晉興郡。成帝分南海立東官郡,以始興、臨賀二郡還蜀荊州。穆帝分蒼梧立晉康、新寧、永平三郡。哀帝太和中置新安郡,安帝分東官立義安郡,恭帝分南海立新會郡。



\end{pinyinscope}