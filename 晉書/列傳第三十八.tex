\article{列傳第三十八}

\begin{pinyinscope}

 顧榮紀瞻賀循楊方薛兼



 顧榮,字彥先,吳國吳人也,為南土著姓。祖雍,吳丞相。父穆,宜都太守。榮機神朗悟,弱冠仕吳,為黃門侍郎、太子輔義都尉。吳平,與陸機兄弟同入洛,時人號為「三俊。」例拜為郎中,歷尚書郎、太子中舍人、廷尉正。恒縱酒酣暢,謂友人張翰曰:「惟酒可以忘憂,但無如作病何耳。」



 會趙王倫誅淮南王允,收允僚屬付廷尉,皆欲誅之,榮平心
 處當,多所全宥。及倫篡位,倫子虔為大將軍,以榮為長史。初,榮與同僚宴飲,見執炙者貌狀不凡,有欲炙之色,榮割炙啖之。坐者問其故,榮曰:「豈有終日執之而不知其味!」及倫敗,榮被執,將誅,而執炙者為督率,遂救之,得免。



 齊王冏召為大司馬主簿。冏擅權驕恣,榮懼及禍,終日昏酣,不綜府事,以情告友人長樂馮熊。熊謂冏長史葛旟曰:「以顧榮為主簿,所以甄拔才望,委以事機,不復計南北親疏,欲平海內之心也。今府大事殷,非酒客之政。」旟曰:「榮江南望士,且居職日淺,不宜輕代易之。」熊曰:「可轉為中書侍郎,榮不失清顯,而府更收實才。」旟然之,
 白冏,以為中書侍郎。在職不復飲酒。人或問之曰:「何前醉而後醒邪?」榮懼罪,乃復更飲。與州里楊彥明書曰:「吾為齊王主簿,恒慮禍及,見刀與繩,每欲自殺,但人不知耳。」及旟誅,榮以討葛旟功,封喜興伯,轉太子中庶子。



 長沙王乂為驃騎,復以榮為長史。乂敗,轉成都王穎丞相從事中郎。惠帝幸臨漳,以榮兼侍中,遣行園陵。會張方據洛,不得進,避之陳留。及帝西遷長安,徵為散騎常侍,以世亂不應,遂還吳。東海王越聚兵於徐州,以榮為軍諮祭酒。



 屬廣陵相陳敏反,南渡江,逐揚州刺史劉機、丹陽內史王曠,阻兵據州,分置子弟為列郡,收禮豪桀,有
 孫氏鼎峙之計。假榮右將軍、丹陽內史。榮數踐危亡之際,恒以恭遜自勉。會敏欲誅諸士人,榮說之曰:「中國喪亂,胡夷內侮,觀太傅今日不能復振華夏,百姓無復遺種。江南雖有石冰之寇,人物尚全。榮常憂無竇氏、孫、劉之策,有以存之耳。今將軍懷神武之略,有孫吳之能,功勳效於已著,勇略冠於當世,帶甲數萬,舳艫山積,上方雖有數州,亦可傳檄而定也。若能委信君子,各得盡懷,散蒂芥之恨,塞讒諂之口,則大事可圖也。」敏納其言,悉引諸豪族委任之。敏仍遣甘卓出橫江,堅甲利器,盡以委之。榮私於卓曰:「若江東之事可濟,當共成之。然卿觀
 事勢當有濟理不?敏既常才,本無大略,政令反覆,計無所定,然其子弟各已驕矜,其敗必矣。而吾等安然受其官祿,事敗之日,使江西諸軍函首送洛,題曰逆賊顧榮、甘卓之首,豈惟一身顛覆,辱及萬世,可不圖之!」卓從之。明年,周與榮及甘卓、紀瞻潛謀起兵攻敏。榮廢橋斂舟於南岸,敏率萬餘人出,不獲濟,榮麾以羽扇,其眾潰散。事平,還吳。永嘉初,徵拜侍中,行至彭城,見禍難方作,遂輕舟而還,語在《紀瞻傳》。



 元帝鎮江東,以榮為軍司,加散騎常侍,凡所謀畫,皆以諮焉。榮既南州望士,躬處右職,朝野甚推敬之。時帝所幸鄭貴嬪有疾,以祈禱頗廢
 萬機,榮上箋諫曰:「昔文王父子兄弟乃有三聖,可謂窮理者也。而文王日昃不暇食,周公一沐三握髮,何哉?誠以一日萬機,不可不理;一言蹉跌,患必及之故也。當今衰季之末,屬亂離之運,而天子流播,豺狼塞路,公宜露營野次,星言夙駕,伏軾怒蛙以募勇士,懸膽於庭以表辛苦。貴嬪未安,藥石實急;禱祀之事,誠復可修;豈有便塞參佐白事,斷賓客問訊?今彊賊臨境,流言滿國,人心萬端,去就紛紜。願沖虛納下,廣延俊彥,思畫今日之要,塞鬼道淫祀,弘九合之勤,雪天下之恥,則群生有賴,開泰有期矣。」



 時南土之士未盡才用,榮又言:「陸士光貞正
 清貴,金玉其質;甘季思忠款盡誠,膽幹殊快;殷慶元質略有明規,文武可施用;榮族兄公讓明亮守節,困不易操;會稽楊彥明、謝行言皆服膺儒教,足為公望;賀生沈潛,青雲之士;陶恭兄弟才幹雖少,實事極佳。凡此諸人,皆南金也。」書奏,皆納之。



 六年,卒官。帝臨喪盡哀,欲表贈榮,依齊王功臣格。吳郡內史殷祐箋曰:



 昔賊臣陳敏憑寵藉權,滔天作亂,兄弟姻婭盤固州郡,威逼士庶以為臣僕,于時賢愚計無所出。故散騎常侍、安東軍司、嘉興伯顧榮經德體道,謀猷弘遠,忠貞之節,在困彌厲。崎嶇艱險之中,逼迫姦逆之下,每惟社稷,發憤慷愾。密結腹
 心,同謀致討。信著群士,名冠東夏,德聲所振,莫不響應,荷戈駿奔,其會如林。榮躬當矢石,為眾率先,忠義奮發,忘家為國,歷年逋寇,一朝土崩,兵不血刃,蕩平六州,勳茂上代,義彰天下。



 伏聞論功依故大司馬齊王格,不在帷幕密謀參議之例,下附州徵野戰之比,不得進爵拓土,賜拜子弟,遐邇同歎,江表失望。齊王親則近屬,位為方嶽,杖節握兵,都督近畿,外有五國之援,內有宗室之助,稱兵彌時,役連天下,元功雖建,所喪亦多。榮眾無一旅,任非籓翰,孤絕江外,王命不通,臨危獨斷,以身徇國,官無一金之費,人無終朝之勞。元惡既殄,高尚成功,封
 閉倉廩,以俟大軍,故國安物阜,以義成俗,今日匡霸事舉,未必不由此而隆也。方之於齊,彊弱不同,優劣亦異。至於齊府參佐,扶義助彊,非創謀之主,皆錫珪受瑞,或公或侯。榮首建密謀,為方面盟主,功高元帥,賞卑下佐,上虧經國紀功之班,下孤忠義授命之士。



 夫考績幽明,王教所崇,況若榮者,濟難寧國,應天先事,歷觀古今,未有立功若彼,酬報如此者也。



 由是贈榮侍中、驃騎將軍、開府儀同三司,謚曰元。及帝為晉王,追封為公,開國,食邑。



 榮素好琴,及卒,家人常置琴於靈座。吳郡張翰哭之慟,既而上床鼓琴數曲,撫琴而歎曰:「顧彥先復能賞此
 不?」因又慟哭,不弔喪主而去。子毗嗣,官至散騎侍郎。



 紀瞻,字思遠,丹陽秣陵人也。祖亮,吳尚書令。父陟,光祿大夫。瞻少以方直知名。吳平,徙家歷陽郡。察孝廉,不行。



 後舉秀才,尚書郎陸機策之曰:「昔三代明王,啟建洪業,文質殊制,而令名一致。然夏人尚忠,忠之弊也朴,救朴莫若敬。殷人革而修焉,敬之弊也鬼,救鬼莫若文。周人矯而變焉,文之弊也薄,救薄則又反之於忠。然則王道之反覆其無一定邪,亦所祖之不同而功業各異也?自無聖王,人散久矣。三代之損益,百姓之變遷,其故可得
 而聞邪?今將反古以救其弊,明風以蕩其穢,三代之制將何所從?太古之化有何異道?」瞻對曰:「瞻聞有國有家者,皆欲邁化隆政,以康庶績,垂歌億載,永傳於後。然而俗變事弊,得不隨時,雖經聖哲,無以易也。故忠弊質野,敬失多儀。周鑒二王之弊,崇文以辯等差,而流遁者歸薄而無款誠,款誠之薄,則又反之於忠。三代相循,如水濟火,所謂隨時之義,救弊之術也。羲皇簡朴,無為而化;後聖因承,所務或異。非賢聖之不同,世變使之然耳。今大晉闡元,聖功日隮,承天順時,九有一貫,荒服之君,莫不來同。然而大道既往,人變由久,謂當今之政宜去文
 存朴,以反其本,則兆庶漸化,太和可致也。」



 又問:「在昔哲王象事備物,明堂所以崇上帝,清廟所以寧祖考,辟雍所以班禮教,太學所以講藝文,此蓋有國之盛典,為幫之大司。亡秦廢學,制度荒闕。諸儒之論,損益異物。漢氏遺作,居為異事,而蔡邕《月令》謂之一物。將何所從?」對曰:「周制明堂,所以宗其祖以配上帝,敬恭明祀,永光孝道也。其大數有六。古者聖帝明王南面而聽政,其六則以明堂為主。又其正中,皆云太廟,以順天時,施行法令,宗祀養老,訓學講肄,朝諸侯而選造士,備禮辯物,一教化之由也。故取其宗祀之類,則曰清廟;取其正室之貌,則
 曰太廟;取其室,則曰太室;取其堂,則曰明堂;取其四門之學,則曰太學;取其周水圜如璧,則白璧雍。異名同事,其實一也。是以蔡邕謂之一物。」



 又問:「庶明亮採,故時雍穆唐;有命既集,而多士隆周。故《書》稱明良之歌,《易》貴金蘭之美。此長世所以廢興,有邦所以崇替。夫成功之君勤於求才,立名之士急於招世,理無世不對,而事千載恆背。古之興王何道而如彼?後之衰世何闕而如此?」對曰:「興隆之政務在得賢,清平之化急於拔才,故二八登庸,則百揆序;有亂十人,而天下泰。武丁擢傅巖之徒,周文攜渭濱之士,居之上司,委之國政,故能龍奮天衢,垂
 勛百代。先王身下白屋,搜揚仄陋,使山無扶蘇之才,野無《伐檀》之詠。是以化厚物感,神祇來應,翔鳳飄颻,甘露豐墜,醴泉吐液,朱草自生,萬物滋茂,日月重光,和氣四塞,大道以成;序君臣之義,敦父子之親,明夫婦之道,別長幼之宜,自九州,被八荒,海外移心,重譯入貢,頌聲穆穆,南面垂拱也。今貢賢之途已闓,而教學之務未廣,是以進競之志恒銳,而務學之心不修。若闢四門以延造士,宣五教以明令德,考績殿最,審其優劣,厝之百僚,置之群司,使調物度宜,節宣國典,必協濟康哉,符契往代,明良來應,金蘭復存也。」



 又問:「昔唐虞垂五刑之教,周公
 明四罪之制,故世歎清問而時歌緝熙。姦宄既殷,法物滋有。叔世崇三辟之文,暴秦加族誅之律,淫刑淪胥,虐濫已甚。漢魏遵承,因而弗革。亦由險泰不同,而救世異術,不得已而用之故也。寬剋之中,將何立而可?族誅之法足為永制與不?」對曰:「二儀分則兆庶生,兆庶生則利害作。利害之作,有由而然也。太古之時,化道德之教,賤勇力而貴仁義。仁義貴則彊不陵弱,眾不暴寡。三皇結繩而天下泰,非惟象刑緝熙而已也。且太古知法,所以遠獄。及其末,不失有罪,是以獄用彌繁,而人彌暴,法令滋章,盜賊多有。《書》曰:『惟敬五刑,以成三德。』叔世道衰,既
 興三辟,而文公之弊,又加族誅,淫刑淪胥,感傷和氣,化染後代,不能變改。故漢祖指麾而六合響應,魏承漢末,因而未革,將以俗變由久,權時之宜也。今四海一統,人思反本,漸尚簡樸,則貪夫不競;尊賢黜否,則不仁者遠。爾則斟參夷之刑,除族誅之律,品物各順其生,緝熙異世而偕也。」



 又問曰:「夫五行迭代,陰陽相須,二儀所以隗育,四時所以化生。《易》稱『在天成象,在地成形』。形象之作,相須之道也。若陰陽不調,則大數不得不否;一氣偏廢,則萬物不得獨成。此應同之至驗,不偏之明證也。今有溫泉而無寒火,其故何也?思聞辯之,以釋不同之理。」對
 曰:「蓋聞陰陽升降,山澤通氣,初九純卦,潛龍勿用,泉源所託,其溫宜也。若夫水潤下,火炎上,剛柔燥濕,自然之性,故陽動而外,陰靜而內。內性柔弱,以含容為質;外動剛直,以外接為用。是以金水之明內鑒,火日之光外輝,剛施柔受,陽勝陰伏。水之受溫,含容之性也。」



 又問曰:「夫窮神知化,才之盡稱;備物致用,功之極目。以之為政,則黃羲之規可踵;以之革亂,則玄古之風可紹。然而唐虞密皇人之闊綱,夏殷繁帝者之約法,機心起而日進,淳德往而莫返。豈太樸一離,理不可振,將聖人之道稍有降殺邪?」對曰:「政因時以興,機隨物而動,故聖王究窮通
 之源,審始終之理,適時之宜,期於濟世。皇代質朴,禍難不作,結繩為信,人知所守。大道既離,智惠擾物,夷險不同,否泰異數,故唐虞密皇人之綱,夏殷繁帝者之法,皆廢興有由,輕重以節,此窮神之道,知化之術,隨時之宜,非有降殺也。」



 永康初,州又舉寒素,大司馬辟東閣祭酒。其年,除鄢陵公國相,不之官。明年,左降松滋侯相。太安中,棄官歸家,與顧榮等共誅陳敏,語在榮傳。



 召拜尚書郎,與榮同赴洛,在途共論《易》太極。榮曰:「太極者,蓋謂混沌之時曚昧未分,日月含其輝,八卦隱其神,天地混其體,聖人藏其身。然後廓然既變,清濁乃陳,二儀著象,陰
 陽交泰,萬物始萌,六合闓拓。《老子》云『有物混成,先天地生』,誠《易》之太極也。而王氏云『太極天地』,愚謂末當。夫兩儀之謂,以體為稱,則是天地;以氣為名,則名陰陽。今若謂太極為天地,則是天地自生,無生天地者也。《老子》又云『天地所以能長且久者,以其不自生,故能長久』『一生二,二生三,三生萬物』,以資始沖氣以為和。原元氣之本,求天地之根,恐宜以此為準也。」瞻曰:「昔皰犧畫八卦,陰陽之理盡矣。文王、仲尼係其遺業,三聖相承,共同一致,稱《易》準天,無復其餘也。夫天清地平,兩儀交泰,四時推移,日月輝其間,自然之數,雖經諸聖,孰知其始。吾子云『
 曚昧未分』分,豈其然乎!聖人,人也,安得混沌之初能藏其身於未分之內!老氏先天之言,此蓋虛誕之說,非《易》者之意也。亦謂吾子神通體解,所不應疑。意者直謂太極極盡之稱,言其理極,無復外形;外形既極,而生兩儀。王氏指向可謂近之。古人舉至極以為驗,謂二儀生於此,非復謂有父母。若必有父母,非天地其孰在?」榮遂止。至徐州,聞亂日甚,將不行。會刺史裴盾得東海王越書,若榮等顧望,以軍禮發遣,乃與榮及陸玩等各解船棄車牛,一日一夜行三百里,得還揚州。



 元帝為安東將軍,引為軍諮祭酒,轉鎮東長史。帝親幸瞻宅,與之同乘而歸。
 以討周馥、華軼功,封都鄉侯。石勒入寇,加揚威將軍、都督京口以南至蕪湖諸軍事,以距勒。勒退,除會稽內史。時有詐作大將軍府符收諸暨令,令已受拘,瞻覺其詐,便破檻出之,訊問使者,果伏詐妄。尋遷丞相軍諮祭酒。論討陳敏功,封臨湘縣侯。西臺除侍中,不就。



 及長安不守,與王導俱入勸進。帝不許。瞻曰:「陛下性與天道,猶復役機神於史籍,觀古人之成敗,今世事舉目可知,不為難見。二帝失御,宗廟虛廢,神器去晉,于今二載,梓宮未殯,人神失御。陛下膺錄受圖,特天所授。使六合革面,遐荒來庭,宗廟既建,神主復安,億兆向風,殊俗畢至,若列
 宿之綰北極,百川之歸巨海,而猶欲守匹夫之謙,非所以闡七廟,隆中興也。但國賊宜誅,當以此屈己謝天下耳。而欲逆天時,違人事,失地利,三者一去,雖復傾匡於將來,豈得救祖宗之危急哉!適時之宜萬端,其可綱維大業者,惟理與當。晉祚屯否,理盡於今。促之則得,可以隆中興之祚;縱之則失,所以資姦寇之權:此所謂理也。陛下身當厄運,纂承帝緒,顧望宗室,誰復與讓!當承大位,此所謂當也。四祖廓開宇宙,大業如此。今五都燔爇,宗廟無主,劉載竊弄神器於西北,陛下方欲高讓於東南,此所謂揖讓而救火也。臣等區區,尚所不許,況大人
 與天地合德,日月並明,而可以失機後時哉!」帝猶不許,使殿中將軍韓績撤去御坐。瞻叱績曰:「帝坐上應星宿,敢有動者斬!」帝為之改容。



 及帝踐位,拜侍中,轉尚書,上疏諫諍,多所匡益,帝甚嘉其忠烈。會久疾,不堪朝請,上疏曰:



 臣疾疢不痊,曠廢轉久,比陳誠款,未見哀察。重以尸素,抱罪枕席,憂責之重,不知垂沒之餘當所投厝。臣聞易失者時,不再者年,故古之志士義人負鼎趣走,商歌於市,誠欲及時效其忠規,名傳不朽也。然失之者億萬,得之者一兩耳。常人之情,貪求榮利。臣以凡庸,邂逅遭遇,勞無負鼎,口不商歌,橫逢大運,頻煩饕竊。雖思慕
 古人自效之志,竟無毫釐報塞之效,而犬馬齒衰,眾疾廢頓,僵臥救命,百有餘日,叩棺曳衾,日頓一日。如復天假之年,蒙陛下行葦之惠,適可薄存性命,枕息陋巷,亦無由復廁八坐,升降臺閣也。臣目冥齒墮,胸腹冰冷,創既不差,足復偏跛,為病受困,既以荼毒。七十之年,禮典所遺,衰老之徵,皎然露見。臣雖欲勤自藏護,隱伏何地!



 臣之職掌,戶口租稅,國之所重。方今六合波盪,人未安居,始被大化,百度草創,發卒轉運,皆須人力。以臣平彊,兼以晨夜,尚不及事,今俟命漏刻,而當久停機職,使王事有廢。若朝廷以之廣恩,則憂責日重;以之序官,則官
 廢事弊;須臣差,則臣日月衰退。今以天慈,使官曠事滯,臣受偏私之宥,於大望亦有虧損。今萬國革面,賢俊比跡,而當虛停好爵,不以縻賢,以臣穢病之餘,妨官固職,誠非古今黜進之急。惟陛下割不已之仁,賜以敝帷,隕仆之日,得以藉尸;時銓俊乂,使官修事舉,臣免罪戮,死生厚幸!



 因以疾免。尋除尚書右僕射,屢辭不聽,遂稱病篤,還第,不許。



 時郗鑒據鄒山,屢為石勒等所侵逼。瞻以鑒有將相之材,恐朝廷棄而不恤,上疏請徵之,曰:「臣聞皇代之興,必有爪牙之佐,捍城之用,帝王之利器也。故虞舜舉十六相而南面垂拱。伏見前輔國將軍郗鑒,少
 立高操,體清望峻,文武之略,時之良幹。昔與戴若思同辟,推放荒地,所在孤特,眾無一旅,救援不至。然能綏集殘餘,據險歷載,遂使凶寇不敢南侵。但士眾單寡,無以立功,既統名州,又為常伯。若使鑒從容臺闥,出內王命,必能盡抗直之規,補袞職之闕。自先朝以來,諸所授用,已有成比。戴若思以尚書為六州都督、征西將軍,復加常侍,劉隗鎮北,陳鎮東。以鑒年時,則與若思同;以資,則俱八坐。況鑒雅望清重,一代名器。聖朝以至公臨天下,惟平是與,是以臣寢頓陋巷,思盡聞見,惟開聖懷,垂問臣導,冀有毫釐萬分之一。」



 明帝嘗獨引瞻於廣室,慨
 然憂天下,曰:「社稷之臣,欲無復十人,如何?」因屈指曰:「君便其一。」瞻辭讓。帝曰:「方欲與君善語,復云何崇謙讓邪!」瞻才兼文武,朝廷稱其忠亮雅正。俄轉領軍將軍,當時服其嚴毅。雖恒疾病,六軍敬憚之。瞻以久病,請去官,不聽,復加散騎常侍。及王敦之逆,帝使謂瞻曰:「卿雖病,但為朕臥護六軍,所益多矣。」乃賜布千匹。瞻不以歸家,分賞將士。賊平,復自表還家,帝不許,固辭不起。詔曰:「瞻忠亮雅正,識局經濟,屢以年耆病久,逡巡告誠。朕深明此操,重違高志,今聽所執,其以為驃騎將軍,常侍如故。服物制度,一按舊典。」遣使就拜,止家為府。尋卒,時年七十
 二。冊贈本官、開府儀同三司,謚曰穆,遣御史持節監護喪事。論討王含功,追封華容子,降先爵二等,封次子一人亭侯。



 瞻性靜默,少交遊,好讀書,或手自抄寫,凡所著述,詩賦箋表數十篇。兼解音樂,殆盡其妙。厚自奉養,立宅於烏衣巷,館宇崇麗,園池竹木,有足賞玩焉。慎行愛士,老而彌篤。尚書閔鴻、太常薛兼、廣川太守河南褚沈、給事中宣城章遼、歷陽太守沛國武嘏,並與瞻素疏,咸藉其高義,臨終託後於瞻。瞻悉營護其家,為起居宅,同於骨肉焉。少與陸機兄弟親善,及機被誅,贍恤其家周至,及嫁機女,資送同於所生。長子景早卒。景子友嗣,官
 至廷尉。景弟鑒,太子庶子、大將軍從事中郎,先瞻卒。



 賀循,字彥先,會稽山陰人也。其先慶普,漢世傳《禮》,世所謂慶氏學。族高祖純,博學有重名,漢安帝時為侍中,避安帝父諱,改為賀氏。曾祖齊,仕吳為名將。祖景,滅賊校尉。父邵,中書令,為孫皓所殺,徙家屬邊郡。循少嬰家難,流放海隅,吳平,乃還本郡。操尚高厲,童齔不群,言行進止,必以禮讓,國相丁乂請為五官掾。刺史嵇喜舉秀才,除陽羨令,以寬惠為本,不求課最。後為武康令,俗多厚葬,及有拘忌迴避歲月,停喪不葬者,循皆禁焉。政教大
 行,鄰城宗之。然無援於朝,久不進序。著作郎陸機上疏薦循曰:「伏見武康令賀循德量邃茂,才鑒清遠,服膺道素,風操凝峻,歷試二城,刑政肅穆。前蒸陽令郭訥風度簡曠,器識朗拔,通濟敏悟,才足幹事。循守下縣,編名凡悴;訥歸家巷,棲遲有年。皆出自新邦,朝無知己,居在遐外,志不自營,年時倏忽,而邈無階緒,實州黨愚智所為恨恨。臣等伏思臺郎所以使州,州有人,非徒以均分顯路,惠及外州而已。誠以庶士殊風,四方異俗,壅隔之害,遠國益甚。至於荊、揚二州,戶各數十萬,今揚州無郎,而荊州江南乃無一人為京城職者,誠非聖朝待四方之
 本心。至於才望資品,循可尚書郎,訥可太子洗馬、舍人。此乃眾望所積,非但企及清途,茍充方選也。謹條資品,乞蒙簡察。」久之,召補太子舍人。



 趙王倫篡位,轉侍御史,辭疾去職。後除南中郎長史,不就,會逆賊李辰起兵江夏,征鎮不能討,皆望塵奔走。辰別帥石冰略有揚州,逐會稽相張景,以前寧遠護軍程超代之,以其長史宰與領山陰令。前南平內史王矩、吳興內史顧祕、前秀才周等唱義,傳檄州郡以討之,循亦合眾應之。冰大將抗寵有眾數千,屯郡講堂。循移檄於寵,為陳逆順,寵遂遁走,超、與皆降,一郡悉平。循迎景還郡,即謝遣兵士,杜門
 不出,論功報賞,一無豫焉。



 及陳敏之亂,詐稱詔書,以循為丹陽內史。循辭以腳疾,手不制筆,又服寒食散,露髮袒身,示不可用,敏竟不敢逼。是時州內豪傑皆見維縶,或有老疾,就加秩命,惟循與吳郡朱誕不豫其事。及敏破,征東將軍周馥上循領會稽相,尋除吳國內史,公車徵賢良,皆不就。



 元帝為安東將軍,復上循為吳國內史,與循言及吳時事,因問曰:「孫皓嘗燒鋸截一賀頭,是誰邪?」循未及言,帝悟曰:「是賀邵也。」循流涕曰:「先父遭遇無道,循創巨痛深,無以上答。」帝甚愧之,三日不出。東海王越命為參軍,徵拜博士,並不起。



 及帝遷鎮東大將軍,以
 軍司顧榮卒,引循代之。循稱疾篤,箋疏十餘上。帝遺之書曰:



 夫百行不同,故出處道殊,因性而用,各任其真耳。當宇宙清泰,彞倫攸序,隨運所遇,動默在己。或有遐棲高蹈,輕舉絕俗,逍遙養和,恬神自足,斯蓋道隆人逸,勢使其然。若乃時運屯弊,主危國急,義士救時,驅馳拯世,燭之武乘縋以入秦,園綺彈冠而匡漢,豈非大雅君子卷舒合道乎!虛薄寡德,忝備近親,謬荷寵位,受任方鎮,餐服玄風,景羨高矩,常願棄結駟之軒軌,策柴篳而造門,徒有其懷,而無從賢之實者何?良以寇逆殷擾,諸夏分崩,皇居失御,黎元荼毒,是以日夜憂懷,慷慨發憤,志
 在竭節耳。前者顧公臨朝,深賴高算。元凱既登,巢許獲逸。至於今日,所謂道之云亡,邦國殄悴,群望顒顒,實在君侯。茍義之所在,豈得讓勞居逸!想達者亦一以貫之也。庶稟徽猷,以弘遠規。今上尚書,屈德為軍司,謹遣參軍沈禎銜命奉授,望必屈臨,以副傾遲。



 循猶不起。



 及帝承制,復以為軍諮祭酒。循稱疾,敦逼不得已,乃轝疾至。帝親幸其舟,因諮以政道。循羸疾不拜謁,乃就加朝服,賜第一區,車馬床帳衣褥等物。循辭讓,一無所受。



 廷尉張闓住在小市,將奪左右近宅以廣其居,乃私作都門,早閉晏開,人多患之,論於州府,皆不見省。會循出,至
 破岡,連名詣循質之。循曰:「見張廷尉,當為言及之。」闓聞而遽毀其門,詣循致謝。其為世所敬服如此。



 時江東草創,盜賊多有,帝思所以防之,以問於循。循答曰:「江道萬里,通涉五州,朝貢商旅之所來往也。今議者欲出宣城以鎮江渚,或使諸縣領兵。愚謂令長威弱,而兼才難備,發憚役之人,而御之不肅,恐未必為用。以循所聞,江中劇地惟有闔廬一處,地勢險奧,亡逃所聚。特宜以重兵備戍,隨勢討除,絕其根帶。沿江諸縣各有分界,分界之內,官長所任,自可度土分力,多置亭行,恒使徼行,峻其綱目,嚴其刑賞,使越常科,勤則有殊榮之報,墮則有
 一身之罪,謂於大理不得不肅。所給人以時番休,役不至困,代易有期。案漢制十里一亭,亦以防禁切密故也。當今縱不能爾,要宜籌量,使力足相周。若寇劫彊多,不能獨制者,可指其縱跡,言所在都督尋當致討。今不明部分,使所在百姓與軍家雜其徼備,兩情俱墮,莫適任負,故所以徒有備名而不能為益者也。」帝從之。



 及愍帝即位,徵為宗正,元帝在鎮,又表為侍中,道險不行。以討華軼功,將封鄉侯,循自以臥疾私門,固讓不受。建武初,為中書令,加散騎常侍,又以老疾固辭。帝下令曰:「孤以寡德,忝當大位,若涉巨川,罔知所憑。循言行以禮,乃時
 之望,俗之表也。實賴其謀猷,以康萬機。疾患有素,猶望臥相規輔,而固守捴謙,自陳懇至,此賢履信思順,茍以讓為高者也。今從其所執。」於是改拜太常,常侍如故。循以九卿舊不加官,今又疾患,不宜兼處此職,惟拜太常而已。



 時宗廟始建,舊儀多闕,或以惠懷二帝應各為世,則潁川世數過七,宜在迭毀。事下太常。循議以為:



 禮,兄弟不相為後,不得以承代為世。殷之盤庚不序陽甲,漢之光武不繼成帝,別立廟寢,使臣下祭之,此前代之明典,而承繼之著義也。惠帝無後,懷帝承統,弟不後兄,則懷帝自上繼世祖,不繼惠帝,當同殷之陽甲,漢之成帝。
 議者以聖德沖遠,未便改舊。諸如此禮,通所未論。是以惠帝尚在太廟,而懷帝復人,數則盈八。盈八之理,由惠帝不出,非上祖宜遷也。下世既升,上世乃遷,遷毀對代,不得相通,未有下升一世而上毀二世者也。惠懷二帝俱繼世祖,兄弟旁親,同為一世,而上毀二為一世。今以惠帝之崩已毀豫章,懷帝之入復毀潁川,如此則一世再遷,祖位橫析。求之古義,未見此例。惠帝宜出,尚未輕論,況可輕毀一祖而無義例乎?潁川既無可毀之理,則見神之數居然自八,此盡有由而然,非謂數之常也。既有八神,則不得不於七室之外權安一位也。至尊於惠
 懷俱是兄弟,自上後世祖,不繼二帝,則二帝之神行應別出,不為廟中恒有八室也。又武帝初成太廟時,正神止七,而楊元后之神亦權立一室。永熙元年,告世祖謚於太廟八室,此是茍有八神,不拘於七之舊例也。



 又議者以景帝俱已在廟,則惠懷一例。景帝盛德元功,王基之本,義著祖宗,百世不毀,故所以特在本廟,且亦世代尚近,數得相容,安神而已,無逼上祖,如王氏昭穆既滿,終應別廟也。以今方之,既輕重義異,又七廟七世之親;昭穆,父子位也。若當兄弟旁滿,輒毀上祖,則祖位空懸,世數不足,何取於三昭三穆與太祖之廟然後成七哉!
 今七廟之義,出於王氏。從禰以上至於高祖,親廟四世,高祖以上復有五世六世無服之祖,故為三昭三穆並太祖而七也。故世祖郊定廟禮,京兆、潁川會、高之親,豫章五世,征西六世,以應此義。今至尊繼統,亦宜有五六世之祖,豫章六世,潁川五世,俱不應毀。今既云豫章先毀,又當重毀潁川,此為廟中之親惟從高祖已下,無復高祖以上二世之祖,於王氏之義,三昭三穆廢闕其二,其非宗廟之本所據承,又違世祖祭征西、豫章之意,於一王定禮所闕不少。



 時尚書僕射刁協與循異議,循答義深備,辭多不載,竟從循議焉。朝廷疑滯皆諮之於循,
 循輒依經禮而對,為當世儒宗。



 其後帝以循清貧,下令曰:「循冰清玉潔,行為俗表,位處上卿,而居身服物蓋周形而已,屋室財庇風雨。孤近造其廬,以為慨然。其賜六尺床薦席褥並錢二十萬,以表至德,暢孤意焉。」循又讓,不許,不得已留之,初不服用。及帝踐位,有司奏瑯邪恭王宜稱皇考,循又議曰:「案禮子不敢以己爵加父。」帝納之。俄以循行太子太傅,太常如故。



 循自以枕疾廢頓,臣節不修,上隆降尊之義,不替交敘之敬,懼非垂典之教也,累表固讓。帝以循體德率物,有不言之益,敦厲備至,期於不許,命皇太子親往拜焉。循有羸疾,而恭於接對;
 詔斷賓客,其崇遇如此。疾漸篤,表乞骸骨,上還印綬,改授左光祿大夫、開府儀同三司。帝臨軒,遣使持節,加印綬。循雖口不能言,指麾左右,推去章服。車駕親幸,執手流涕。太子親臨者三焉,往還皆拜,儒者以為榮。太興二年卒,時年六十。帝素服舉哀,哭之甚慟。贈司空,謚曰穆。將葬,帝又出臨其柩,哭之盡哀,遣兼侍御史持節監護。皇太子追送近途,望船流涕。



 循少玩篇籍,善屬文,博覽眾書,尤精禮傳。雅有知人之鑒,拔同郡楊方於卑陋,卒成名於世。子隰,康帝時官至臨海太守。



 楊方,字公回。少好學,有異才。初為郡鈴下威儀,公事之暇,輒讀《五經》,鄉邑未之知。內史諸葛恢見而奇之,待以門人之禮,由是始得周旋貴人間。時虞喜兄弟以儒學立名,雅愛方,為之延譽。恢嘗遣方為文,薦郡功曹主簿。虞預稱美之,送以示循。循報書曰:「此子開拔有志,意只言異於凡猥耳,不圖偉才如此。其文甚有奇分,若出其胸臆,乃是一國所推,豈但牧豎中逸群邪!聞處舊黨之中,好有謙沖之行,此亦立身之一隅。然世衰道喪,人物凋弊,每聞一介之徒有向道之志,冀之願之。如方者乃荒萊之特苗,鹵田之善秀,姿質已良,但沾染未足耳;移
 植豐壤,必成嘉豎。足下才為世英,位為朝右,道隆化立,然後為貴。昔許子將拔樊仲昭於賈堅,郭林宗成魏德公於畎畝。足下志隆此業,二賢之功不為難及也。」循遂稱方於京師。司徒王導辟為掾,轉東安太守,遷司徒參軍事。方在都邑,搢紳之士咸厚遇之,自以地寒,不願久留京華,求補遠郡,欲閑居著述。導從之,上補高梁太守。在郡積年,著《五經鉤沈》,更撰《吳越春秋》,並雜文筆,皆行於世。以年老,棄郡歸。導將進之臺閣,固辭還鄉里,終於家。



 薛兼,字令長,丹陽人也。祖綜,仕吳為尚書僕射。父瑩,有名吳朝。吳平,為散騎常侍。兼清素有器宇,少與同郡紀瞻、廣陵閔鴻、吳郡顧榮、會稽賀循齊名,號為「五俊」。初入洛,司空張華見而奇之,曰:「皆南金也。」察河南孝謙,辟公府,除比陽相,蒞任有能名。歷太子洗馬、散騎常侍、懷令。司空、東海王越引為參軍,轉祭酒,賜爵安陽亭侯。元帝為安東將軍,以為軍諮祭酒,稍遷丞相長史。甚勤王事,以上佐祿優,每自約損,取周而已。進爵安陽鄉侯,拜丹陽太守。中興建,轉尹,加秩中二千石,遷尚書,領太子少傅。自綜至兼,三世傅東宮,談者美之。



 永昌初,王敦表兼
 為太常。明帝即位,加散騎常侍。帝以東宮時師傅,猶宜盡敬,乃下詔曰:「朕以不德,夙遭閔凶。猥以眇身,託于王公之上。哀煢在疚,靡所諮仰,憂懷惴惴,如臨于谷。孔子有云:『故雖天子,必有尊也。』朕將祗奉先師之禮,以諮有德。太宰西陽王秩尊望重,在貴思降。丞相武昌公、司空即丘子體道高邈,勳德兼備,先帝執友,朕之師傅。太常安陽鄉侯訓保朕躬,忠肅篤誠。夫崇親尊賢,先帝所重,朕見四君及書疏儀體,一如東宮故事。」是歲,卒。詔曰:「太常、安陽鄉侯兼履德沖素,盡忠恪己。方賴德訓,弘濟政道,不幸殂殞,痛于厥心。今遣持節侍御史贈左光祿大
 夫、開府儀同三司。魂而有靈,嘉茲榮寵。」及葬,屬王敦作逆,朝廷多故,不得議謚,直遣使者祭以太牢。子顒,先兼卒,無後。



 史臣曰:元帝樹基淮海,百度權輿,夢想群材,共康庶績。顧、紀、賀、薛等並南金東箭,世胄高門,委質霸朝,豫聞邦政;典憲資其刊輯,帷幄佇其謀猷;望重搢紳,任惟元凱,官成名立,光國榮家。非惟感會所鐘,抑亦材能斯至。而循位登保傅,朝望特隆,遂使鑾蹕降臨,承明下拜。雖西漢之恩崇張禹,東都之禮重桓榮,弗是過也。



 贊曰:彥先通識,思遠方直。薛既清貞,賀惟學植。逢時遇
 主,摶風矯翼。



\end{pinyinscope}