\article{志第一}

\begin{pinyinscope}

 天文上{{天體儀象天文經星二十八舍二十八宿外星天漢起沒十二次度數州郡躔次



 昔在庖犧,觀象察法,以通神明之德,以類天地之情,可以藏往知來,開物成務。故《易》曰:「天垂象,見吉凶,聖人象之。」此則觀乎天文以示變者也。《尚書》曰:「天聰明自我民聰明。」此則觀乎人文以成化者也。是故政教兆於人理,
 祥變應乎天文,得失雖微,罔不昭著。然則三皇邁德,七曜順軌,日月無薄蝕之變,星辰靡錯亂之妖。黃帝創受《河圖》,始明休咎,故其《星傳》尚有存焉。降在高陽,乃命南正重司天,北正黎司地。爰洎帝嚳,亦式序三辰。唐虞則羲和繼軌,有夏則昆吾紹德。年代綿邈,文籍靡傳。至于殷之巫咸,周之史佚,格言遺記,于今不朽。其諸侯之史,則魯有梓慎,晉有卜偃,鄭有裨灶,宋有子韋,齊有甘德,楚有唐昧,趙有尹皋,魏有石申夫,皆掌著天文,各論圖驗。其巫咸、甘、石之說,後代所宗。暴秦燔書,六經殘滅,天官星占,存而不毀。及漢景武之際,司馬談父子繼為史
 官,著《天官書》,以明天人之道。其後中壘校尉劉向,廣《洪範》災條,作《皇極論》,以參往之行事。及班固敘漢史,馬續述《天文》,而蔡邕、譙周各有撰錄,司馬彪採之,以繼前志。今詳眾說,以著於篇。



 古言天者有三家,一曰蓋天,二曰宣夜,三曰渾天。漢靈帝時,蔡邕於朔方上書,言「宣夜之學,絕無師法。《周髀》術數具存,考驗天狀,多所違失。惟渾天近得其情,今史官候臺所用銅儀則其法也。立八尺圓體而具天地之形,以正黃道,占察發斂,以行日月,以步五緯,精微深妙,百代不易之道也。官有其器而無本書,前志亦闕」。



 蔡邕所
 謂《周髀》者,即蓋天之說也。其本庖犧氏立周天歷度,其所傳則周公受於殷商,周人志之,故曰《周髀》。髀,股也;股者,表也。其言天似蓋笠,地法覆槃,天地各中高外下。北極之下為天地之中,其地最高,而滂沲四隤,三光隱映,以為晝夜。天中高於外衡冬至日之所在六萬里,北極下地高於外衡下地亦六萬里,外衡高於北極下地二萬里。天地隆高相從,日去地恒八萬里。日麗天而平轉,分冬夏之間日所行道為七衡六間。每衡周徑里數,各依算術,用句股重差推晷影極游,以為遠近之數,皆得於表股者也。故曰《周髀》。



 又《周髀》家云:「天圓如張蓋,地方
 如棋局。天旁轉如推磨而左行,日月右行,隨天左轉,故日月實東行,而天牽之以西沒。譬之於蟻行磨石之上,磨左旋而蟻右去,磨疾而蟻遲,故不得不隨磨以左迴焉。天形南高而北下,日出高,故見;日入下,故不見。天之居如倚蓋,故極在人北,是其證也。極在天之中,而今在人北,所以知天之形如倚蓋也。日朝出陽中,暮入陰中,陰氣暗冥,故沒不見也。夏時陽氣多,陰氣少,陽氣光明,與日同輝,故日出即見,無蔽之者,故夏日長也。冬天陰氣多,陽氣少,陰氣暗冥,掩日之光,雖出猶隱不見,故冬日短也。」



 宣夜之書亡,惟漢秘書郎郗萌記先師相傳云:「
 天了無質,仰而瞻之,高遠無極,眼瞀精絕,故蒼蒼然也。譬之旁望遠道之黃山而皆青,俯察千仞之深谷而窈黑,夫青非真色,而黑非有體也。日月眾星,自然浮生虛空之中,其行其止皆須氣焉。是以七曜或逝或住,或順或逆,伏見無常,進退不同,由乎無所根繫,故各異也。故辰極常居其所,而北斗不與眾星西沒也。攝提、填星皆東行,日行一度,月行十三度,遲疾任情,其無所繫著可知矣。若綴附天體,不得爾也。



 成帝咸康中,會稽虞喜因宣夜之說作《安天論》,以為「天高窮於無窮,地深測於不測。天確乎在上,有常安之形;地塊焉在下,有居靜之體。
 當相覆冒,方則俱方,圓則俱圓,無方圓不同之義也。其光曜布列,各自運行,猶江海之有潮汐,萬品之有行藏也」。葛洪聞而譏之曰:「茍辰宿不麗於天,天為無用,便可言無,何必復云有之而不動乎?」由此而談,稚川可謂知言之選也。



 虞喜族祖河間相聳又立穹天論云:「天形穹隆如雞子,幕其際,周接四海之表,浮於元氣之上。譬如覆奩以抑水,而不沒者,氣充其中故也。日繞辰極,沒西而還東,不出入地中。天之有極,猶蓋之有斗也。天北下於地三十度,極之傾在地卯酉之北亦三十度,人在卯酉之南十餘萬里,故斗極之下不為地中,當對天地卯
 酉之位耳。日行黃道繞極,極北去黃道百一十五度,南去黃道六十七度,二至之所舍以為長短也。」



 吳太常姚信造昕天論云:「人為靈蟲,形最似天。今人頤前侈臨胸,而項不能覆背。近取諸身,故知天之體南低入地,北則偏高。又冬至極低,而天運近南,故日去人遠,而斗去人近,北天氣至,故冰寒也。夏至極起,而天運近北,故斗去人遠,日去人近,南天氣至,故蒸熱也。極之立時,日行地中淺,故夜短;天去地高,故晝長也。極之低時,日行地中深,故夜長;天去地下,故晝短也。」



 自虞喜、虞聳、姚信皆好奇徇異之說,非極數談天者也。至於渾天理妙,學者
 多疑。漢王仲任據蓋天之說,以駮渾儀云:「舊說天轉從地下過。今掘地一丈輒有水,天何得從水中行乎?甚不然也。日隨天而轉,非入地。夫有目所望,不過十里,天地合矣;實非合也,遠使然耳。今視日入,非入也,亦遠耳。當日入西方之時,其下之人亦將謂之為中也。四方之人,各以其所近者為出,遠者為入矣。何以明之?今試使一人把大炬火,夜行於平地,去人十里,火光滅矣;非滅也。遠使然耳。今日西轉不復見,是火滅之類也。日月不員也,望視之所從員者,去人遠也。夫日,火之精也;月,水之精也。水火在地下員,在天何故員?」故丹陽葛洪釋之曰:《
 渾天儀注》云:「天如雞子,地如雞中黃,孤居於天內,天大而地小。天表裏有水,天地各乘氣而立,載水而行。周天三百六十五度四分度之一,又中分之,則半覆地上,半繞地下,故二十八宿半見半隱,天轉如車轂之運也。」諸論天者雖多,然精於陰陽者少。張平子、陸公紀之徒,咸以為推步七曜之道,以度歷象昏明之證候,校以四八之氣,考以漏刻之分,占晷景之往來,求形驗於事情,莫密於渾象者也。



 張平子既作銅渾天儀,於密室中以漏水轉之,令伺之者閉戶而唱之。其伺之者以告靈臺之觀天者曰:「璇璣所加,某星始見,某星已中,某星今沒」,皆如合
 符也。崔子玉為其碑銘曰:「數術窮天地,制作侔造化,高才偉藝,與神合契。」蓋由於平子渾儀及地動儀之有驗故也。



 若天果如渾者,則天之出入行於水中,為的然矣。故黃帝書曰,「天在地外,水在天外」,水浮天而載地者也。又《易》曰:「時乘六龍。」夫陽爻稱龍,龍者居水之物,以喻天。天,陽物也,又出入水中,與龍相似,故以比龍也。聖人仰觀俯察,審其如此,故《晉》卦《坤》下《離》上,以證日出於地也。又《明夷》之卦《離》下《坤》上,以證日入於地也。《需》卦《乾》下《坎》上,此亦天入水中之象也。天為金,金水相生之物也。天出入水中,當有何損,而謂為不可乎?故桓君山曰:「春分
 日出卯入酉,此乃人之卯酉。天之卯酉,常值斗極為天中。今視之乃在北,不正在人上。而春秋分時,日出入乃在斗極之南。若如磨右轉,則北方道遠而南方道近,晝夜漏刻之數不應等也。」後奏事待報,坐西廊廡下,以寒故暴背。有頃,日光出去,不復暴背。君山乃告信蓋天者曰:「天若如推磨右轉而日西行者,其可知矣。」然則天出入水中,無復疑矣。



 又今視諸星出於東者,初但去地小許耳。漸而西行,先經人上,從遂西轉而下焉,不旁旋也。其先在西之星,亦稍下
 而沒,無北轉者。日之出入亦然。若謂天磨右轉者,日之出入亦然,眾日月宜隨夫而迴,初在於東,次經於南,次到於西,次及於北,而復還於東,不應橫過去也。今日出於東,冉冉轉上,及其入西,亦復漸漸稍下,都不繞邊北去。了了如此,王生必固謂為不然者,疏矣。



 今日徑千里,圍周三千里,中足以當小星之數十也。若日以轉遠之故,但當光曜不能復來照及人耳,宜猶望見其體,不應都失其所在也。日光既盛,其體又大於星多矣。今見極北之小星,而不見日之在北者,明其不北行也。若日以轉還之故,不復可見,其北入之間,應當稍小,而日方
 入之時乃更大,此非轉遠之徵也。王生以火炬喻日,吾亦將借子之矛以刺子之楯焉。把火之人去人轉遠,其光轉微,而日月自出至入,不漸小也。王生以火喻之,謬矣。



 又日之入西方,視之稍稍去,初尚有半,如橫破鏡之狀,須臾淪沒矣。若如王生之言,日轉北去有半者,其北都沒之頃,宜先如豎破鏡之狀,不應如橫破鏡也。如此言之,日入西方,不亦孤於乎?又月之光微,不及日遠矣。月盛之時,雖有重雲敝之,不見月體,而夕猶朗然,是光猶從雲中而照外也。日若繞西及北者,其光故應如月在雲中之狀,不得夜便大暗也。又日入則星月出焉。明知
 天以日月分主晝夜,相代而照也。若日常出者,不應日亦入而星月亦出也。



 又案《河》、《洛》之文,皆云水火者,陰陽之餘氣也。夫言餘氣,則不能生日月可知也,顧當言日精生火者可耳。若水火是日月所生,則亦何得盡如日月之員乎?今火出於陽燧,陽燧員而火不員也;水出於方諸,方諸方而水不方也。又陽燧可以取火於日,而無取日於火之理,此則日精之生火明矣,方諸可以取水於月,而無取月於水之道,此則月精之生水了矣。王生又云遠故視之員。若審然者,月初生之時及既虧之後,何以視之不員乎?而日食或上或下,從側而起,或
 如鉤至盡。若遠視見員,不宜見其殘缺左右所起也。此則渾天之理,信而有徵矣。



 ◎儀象



 《虞書》曰:「在旋璣玉衡,以齊七政。」《考靈曜》云:「分寸之咎,代天氣生,以制方員。方員以成,參以規矩。昏明主時,乃命中星觀玉儀之游。」鄭玄謂以玉為渾儀也。《春秋文曜鉤》云:「唐堯既位,羲和立渾儀。」此則儀象之設,其來遠矣。綿代相傳,史官禁密,學者不睹,故宣、蓋沸騰。



 暨漢太初,落下閎、鮮于妄人、耿壽昌等造員儀以考歷度。後至和帝時,賈選逵繫作,又加黃道。至順帝時,張衡又制渾象,具內
 外規、南北極、黃赤道,列二十四氣、二十八宿中外星官及日月五緯,以漏水轉之於殿上室內,星中出沒與天相應。因其關戾,又轉瑞輪蓂莢於階下,隨月虛盈,依歷開落。



 其後陸績亦造渾象。至吳時,中常侍廬江王蕃善數術,傳劉洪《乾象歷》,依其法而制渾儀,立論考度曰:



 前儒舊說天地之體,狀如鳥卵,天包地外,猶殼之果黃也;周旋無端,其形渾渾然,故曰渾天也。周天三百六十五度五百八十九分度之百四十五,半覆地上,半在地下。其二端謂之南極、北極。北極出地三十六度,南極入地三十六度,兩極相去一百八十二度半彊。繞北極徑七
 十二度,常見不隱,謂之上規。繞南極七十二度,常隱不見,謂之下規。赤道帶天之紘,去兩極各九十一度少彊。



 黃道,日之所行也,半在赤道外,半在赤道內,與赤道東交於角五少弱,西交於奎十四少彊。其出赤道外極遠者,去赤道二十四度,斗二十一度是也。其入赤道內極遠者,亦二十四度,井二十五度是也。



 日南至在斗二十一度,去極百一十五度少彊。是也日最南,去極最遠,故景最長。黃道斗二十一度,出辰入申,故日亦出辰入申。日晝行地上百四十六度彊,故日短;夜行地下二百一十九度少弱,故夜長。自南至之後,日去極稍近,故景稍短。
 日晝行地上度稍多,故日稍長;夜行地下度稍少,故夜稍短。日所在度稍北,故日稍北,以至於夏至,日在井二十五度,去極六十七度少彊,是日最北,去極最近,景最短。黃道井二十五度,出寅入戌,故日亦出寅入戌。日晝行地上二百一十九度少弱,故日長;夜行地下百四十六度彊,故夜短。自夏至之後,日去極稍遠,故景稍長。日晝行地上度稍少,故日稍短;夜行地下度稍多,故夜稍長。日所在度稍南,故日出入稍南,以至於南至而復初焉。斗二十一,井二十五,南北相應四十八度。



 春分日在奎十四少彊,秋分日在角五少弱,此黃赤二道之交中
 也。去極俱九十一度少彊。南北處斗二十一,井二十五之中,故景居二至長短之中。奎十四,角五,出卯入酉,故日亦出卯入酉。日晝行地上,夜行地下,俱百八十二度半彊,故日見之漏五十刻,不見之漏五十刻,謂之晝夜同。夫天之晝夜以日出沒為分,人之晝夜以昏明為限。日未出二刻半而明,日入二刻半而昏,故損夜五刻以益晝,是以春秋分漏晝五十五刻。



 三光之行,不必有常,術家以算求之,各有同異,故諸家歷法參差不齊。《洛書甄曜度》、《春秋考異郵》皆云:「周天一百七萬一千里,一度為二千九百三十二里七十一步二尺七寸四分四
 百八十七分分之三百六十二。」陸績云:「天東西南北徑三十五萬七千里。」此言周三徑一也。考之徑一不啻周三,率周百四十二而徑四十五,則天徑三十二萬九千四百一里一百二十二步二尺二寸一分七十一分分之十。



 《周禮》:「日至之景尺有五寸,謂之地中。」鄭眾說:「土圭之長尺有五寸,以夏至之日立八尺之表,其景與土圭等,謂之地中,今潁川陽城地也。」鄭玄云:「凡日景於地,千里而差一寸,景尺有五寸者,南戴日下萬五千里也。」以此推之,日當去其下地八萬里矣。日邪射陽城,則天徑之半也。天體員如彈丸,地處天之半,而陽城為中,則日春
 秋冬夏,昏明晝夜,去陽城皆等,無盈縮矣。故知從日邪射陽城,為天徑之半也。



 以句股法言之,旁萬五千里,句也;立八萬里,股也;從日邪射陽城,弦也。以句股求弦法入之,得八萬一千三百九十四里三十步五尺三寸六分,天徑之半而地上去天之數也。倍之,得十六萬二千七百八十八里六十一步四尺七寸二分,天徑之數也。以周率乘之,徑率約之,得五十一萬三千六百八十七里六十八步一尺八寸二分,周天之數也。減《甄曜度》、《考異郵》五十五萬七千三百一十七里有奇。一度凡千四百六里百二十四步六寸四分十萬七千五百六十五
 分分之萬九千四十九,減舊度千五百二十五里二百五十六步三尺三寸二十一萬五千一百三十分分之十六萬七百三十。



 分黃赤二道,相興交錯,其間相去二十四度。以兩儀推之,二道俱三百六十五度有奇,是以知天體員如彈丸也。而陸績造渾象,其形如鳥卵,然則黃道應長於赤道矣。績云「天東西南北徑三十五萬七千里」,然則績亦以天形正員也,而渾象為鳥卵,則為自相違背。



 古舊渾象以二分為一度,凡周七尺三寸半分。張衡更制,以四分為一度,凡周一丈四尺六寸一分。蕃以古制局小,星辰稠穊,衡器傷大,難可轉移,更制渾象,以三
 分為一度,凡周天一丈九寸五分四分分之三也。



 ◎天文經星



 《洪範傳曰》:「清而明者,天之體也。天忽變色,是謂易常。天裂,陽不足,是謂臣彊。天裂見人,兵起國亡。天鳴有聲,至尊憂且驚。皆亂國之所生也。」



 馬續云:「天文在圖籍昭昭可知者,經星常宿中外官凡一百一十八名,積數七百八十三,皆有州國官宮物類之象。」



 張衡云:「文曜麗乎天,其動者有七,日月五星是也。日者,陽精之宗;月者,陰精之宗;五星,五行之精。眾星列布,體生於地,精成於天,列居錯峙,各有攸屬。在野象物,在朝象官,在人象神。其以
 神差,有五列焉,是為三十五名。一居中央,謂之北斗。四布於方各七,為二十八舍。日月運行,歷示吉凶,五緯躔次,用告禍福。中外之官,常明者百有二十四,可名者三百二十,為星二千五百,微星之數蓋萬有一千五百二十。庶物蠢蠢,咸得繫命。不然,何得總而理諸?」後武帝時,太史令陳卓總甘、石、巫咸三家所著星圖,大凡二百八十三官,一千四百六十四星,以為定紀。今略其昭昭者,以備天官云。



 ◎中宮



 北極五星,鉤陳六星,皆在紫宮中。北極,北辰最尊者也,
 其紐星,天之樞也。天運無窮,三光迭耀,而極星不移,故曰「居其所而眾星共之」。第一星主月,太子也。第二星主日,帝王也;亦太乙之坐,謂最赤明者也。第三星主五星,庶子也。中星不明,主不用事;右星不明,太子憂。鉤陳,後宮也,大帝之正妃也,大帝之常居也。北四星曰女御宮,八十一御妻之象也。鉤陳口中一星曰天皇大帝,其神曰耀魄寶,主御群靈,執萬神圖。抱北極四星曰四輔,所以輔佐北極而出度授政也。大帝上九星曰華蓋,所以覆蔽大帝之坐也。蓋下九星曰杠,蓋之柄也。華蓋下五星曰五帝內坐,設敘順帝所居也。客星犯紫宮中坐,大
 臣犯主。華蓋杠旁六星曰六甲,可以分陰陽而配節候,故在帝旁,所以布政教而授農時也。極東一星曰柱下史,主記過;左右史,此之象也。柱史北一星曰女史,婦人之微者,主傳漏,故漢有侍史。傳舍九星在華蓋上,近河,賓客之館,主胡人入中國。客星守之,備姦使,亦曰胡兵起。傳舍南河中五星曰造父,御官也,一曰司馬,或曰伯樂。星亡,馬大貴。其西河中九星如鉤狀,曰鉤星,直則地動。天一星在紫宮門右星南,天帝之神也,主戰鬥,知人吉凶者也。太一星在天一南,相近,亦天帝神也,主使十六神,知風雨水旱、兵革飢謹、疾疫災害所在之國也。



 紫
 宮垣十五星,其西番七,東番八,在北斗北。一曰紫微,大帝之坐也,天子之常居也,主命主度也。一曰長垣,一曰天營,一曰旗星,為番衛,備番臣也。宮闕兵起,旗星直,天子出,自將宮中兵。東垣下五星曰天柱,建政教,懸圖法。門內東南維五星曰尚書,主納言,夙夜諮謀;龍作納言,此之象也。尚書西二星曰陰德、陽德,主周急振撫。宮門左星內二星曰大埋,主平刑斷獄也。門外六星曰天床,主寢舍,解息燕休。西南角外二星曰內廚,主六宮之內飲食,主平刑斷獄也。門外六星曰天床,主寢舍,解息燕休。西南角外二星曰內廚,主六宮之內飲食,主后妃夫人與太子宴飲。東北維外六星曰天廚,主盛饌。



 北斗七星在太微北,七政之樞機,陰陽之元本
 也。故運乎天中,而臨制四方,以建四時,而均五行也。魁四星為旋璣,杓三星為玉衡。又曰,斗為人君之象,號令之主也。又為帝車,取乎運動之義也。又魁第一星曰天樞,二曰璇,三曰璣,四曰權,五曰玉衡,六曰開陽,七曰搖光,一至四為魁,五至七為杓。樞為天,璇為地,璣為人,權為時,玉衡為音,開陽為律,搖光為星。石氏云:「第一曰正星,主陽德,天子之象也。二曰法星,主陰刑,女主之位也。三曰令星,主中禍。四曰伐星,主天理,伐無道。五曰殺星,主中央,助四旁,殺有罪。六曰危星,主天倉五穀。七曰部星,亦曰應星,主兵。」又云:「一主天,二主地,三主火,四主水,五
 主土,六主木,七主金。」又曰:「一主秦,二主楚,三主梁,四主吳,五主燕,六主趙,七主齊。」



 魁中四星為貴人之牢,曰天理也。輔星傅乎開陽,所以佐斗成功,丞相之象也。七政星明,其國昌;輔星明,則臣彊。杓南三星及魁第一星西三星皆曰三公,主宜德化,調七政,和陰陽之官也。



 文昌六星,在北斗魁前,天之六府也,主集計天道。一曰上將,大將軍建威武。二曰次將,尚書正左右。三曰貴相,太常理文緒。四曰司祿、司中,司隸賞功進。五曰司命、司怪,太史主滅咎。六曰司冠,大理佐理寶。所謂一者,起北斗魁前近內階者也。明潤,大小齊,天瑞臻。



 文昌北六星
 曰內階,天皇之階也。相一星在北斗南。相者,總領百司而掌邦教,以佐帝王安邦國,集眾事也。其星明,吉。太陽守一星,在相西,大將大臣之象也,主戒不虞,設武備。西北四星曰勢。勢,腐刑人也。天牢六星,在北斗魁下,貴人之牢也。



 太微,天子庭也,五帝之坐也,十二諸侯府也。其外蕃,九卿也。一曰太微為衡。衡,主平也。又為天庭,理法平辭,監升授德,列宿受符,諸神考節,舒情稽疑也。南蕃中二星間曰端門。東曰左執法,廷尉之象也。西曰右執法,御史大夫之象也。執法,所以舉刺凶姦者也。左執法之東,左掖門也。右執法之西,右掖門也。東蕃四星,南第一星曰
 上相,其北,東太陽門;第二星曰次相,其北,中華東門也;第三星曰次將,其北,東太陰門也;第四星曰上將:所謂四輔也。西蕃四星,南第一星曰上將,其北,西太陽門也;第二星曰次將,其北,中華西門也;第三星曰次相,其北,西太陰門也;第四曰上相:亦曰四輔也。東西蕃有芒及動搖者,諸侯謀。執法移,刑罰尤急。月、五星入太微,軌道,吉。其所犯中坐,成刑。



 其西南角外三星曰明堂,天子布政之宮。明堂西三星曰靈臺,觀臺也,主觀雲物,察符瑞,候災變也。左執法東北一星曰謁者,主贊賓客也。謁者東北三星曰三公內坐,朝會之所居也。三公北
 三星曰九卿內坐,治萬事。九卿西五星曰內五諸侯,內待天子,不之國也。辟雍之禮得,則太微、諸侯明。



 黃帝坐在太微中,含樞紐之神也。天子動得天度,止得地意,從容中道,則太微五帝坐明以光。黃帝坐不明,人主求賢士以輔法,不然則奪勢。四帝星夾黃帝坐,東方蒼帝,靈威仰之神也;南方赤帝,赤熛怒之神也;西方白帝,白招矩之神也;北方黑帝,葉光紀之神也。



 五帝坐北一星曰太子,帝儲也。太子北一星曰從官,侍臣也。帝坐東北一星曰幸臣。屏四星在端門之內,近右執法。屏,所以雍蔽學帝也。執法主刺舉;臣尊敬君上,則星光明潤
 澤。郎位十五星在帝坐東北。一曰依烏郎府也。周官之元士,漢官之光祿、中散、諫議、議郎、三署郎中,是其職也。郎,主守衛也。其星不具,后妃死,幸臣誅。星明大及客星入之,大臣為亂。郎將在郎位北,主閱具,所以為武備也。武賁一星,在太微西蕃北,下臺南,靜室旄頭之騎官也。常陳七星,如畢狀,在帝坐北,天子宿衛武賁之士,以設彊禦也。星搖動,天子自出,明則武兵用,微則兵弱。



 三台六星,兩兩而居,起文昌,列抵太微。一曰天住,三公之位也。在人曰三公,在天曰三台,主開德宜符也。西近文昌二星曰上台,為司命,主壽。次二星曰中台,為司中,主宗
 室。東二星曰下台,為司祿,主兵,所以昭德塞違也。又曰三台為天階,太一躡以上下。一曰泰階。上階,上星為天子,下星為女主;中階,上星為諸侯三公,下星為卿大夫;下階,上星為士,下星為庶人:所以和陰陽而理萬物也。君臣和集,如其常度,有變則占其人。



 南四星曰內平,近職執法平罪之官也。中台之北一星曰太尊,貴戚也。



 攝提六星,直斗杓之南,主建時節,伺禨祥。攝提為楯,以夾擁帝座也,主九卿。明大,三公恣。客星入之,聖人受制。西三星曰周鼎,主流亡。大角在攝提間。大角者,天王座也。又為天棟,正經紀也。北三星曰帝席,主宴獻酬酢。北
 三星曰梗河,天矛也。一曰天鋒,主胡兵。又為喪,故其變動應以兵喪也。星亡,其國有兵謀。其北一星曰招搖,一曰矛楯,其北一星曰玄戈,皆主胡兵,占與梗河略相類也。招搖與北斗杓間曰天庫。星去其所,則有庫開之祥也。招搖欲與棟星、梗河、北斗相應,則胡兵當來受命於中國。玄戈又主北夷,客星守之,胡大敗。天槍三星,在北斗杓東,一曰天鉞,天之武備也。故在紫宮之左,所以禦難也。女床三星,在紀星北,後宮御也,主女事。天棓五星,在女床北,天子先驅也,主分爭與刑罰,藏兵亦所以禦難也。槍、棓,皆以備非常也;一星不具,
 其國兵起。東七星曰扶筐,盛桑之器,主勸蠶也。七公七星,在招搖東,天之相也,三公之象也,主七政。貫索九星在其前,賤人之牢也。一曰連索,一曰連營,一曰天牢,主法律,禁暴彊也。牢口一星為門,欲其開也。九星皆明,天下獄煩;七星見,小赦;六星、五星,大赦。動則斧金質用,中空則更元。《漢志》云十五星。天紀九星,在貫索東,九卿也,主萬事之紀,理怨訟也。明則天下多辭訟;亡則政理壞,國紀亂;散絕則地震山崩。織女三星,在天紀東端,天女也,主果蓏絲帛珍寶也。王者至孝,神祗咸喜,則織女星俱明,天下和平。大星怒角,布帛貴。東足四星曰漸臺,臨水
 之臺也,主晷漏律呂之事。西足五星曰輦道,王者得嬉游之道也,漢輦道通南北宮,其象也。



 左右角間二星曰平道之官。平道西一星曰進賢,主卿相舉逸才。亢、東咸、西咸各四星,在房心北,日月五星之道也。房之戶,所以防淫佚也。星明則吉;月、五星犯守之,有陰謀。鍵閉一星,在房東北,近鉤鈐,主關籥。



 天市垣二十二星,在房心東北,主權衡,主聚眾。一曰天旗庭,主斬戮之事也。市中星眾潤澤,則歲實。熒惑守之,戮不忠之臣。彗星除之,為徙市易都。客星入之,兵大起;出之,有貴喪。



 帝坐一星,在天市中候星西,天庭也。光而潤則天子吉,威令行。候一星,在帝
 坐東北,主伺陰陽也。明大,輔臣彊,四夷開;候細微,則國安;亡則主失位;移則不安。宦者四星,在帝坐西南,侍主刑餘之人也。星微,吉;非其常,宦者有夏。宗正二星,在帝坐東南,宗大夫也。彗星守之,若失色,宗正有事;客星守之,更號令也。宗人四星,在宗正東,主錄親疏享祀。族人有序,則如綺文而明正。動則天子親屬有變;客星守之,貴人死。宗星二,在候星東,宗室之象,帝輔血脈之臣也。客星守之,宗支不和。



 天江四星,在尾北,主太陰。江星不具,天下津河關道不通。明若動搖,大水出,大兵起;參差則馬貴。熒惑守之,有立主。客星入之,河津絕。



 天籥八星
 在南斗柄西,主關閉。建星六星在南斗北,亦曰天旗,天之都關也。為謀事,為天鼓,為天馬。南二星,天庫也。中央二星,市也,鈇金質也。上二星,旗跗也。斗建之間,三光道也。星動則眾勞。月暈之,蛟龍見,牛馬疫。月、五星犯之,大臣相譖有謀,亦為關梁不通,有大水。東南四星曰狗國,主鮮卑、鳥丸、沃且。熒惑守之,外夷為變。狗國北二星曰天雞,主候時。天弁九星,在建星北,市官之長也,以知市珍也。星欲明,吉。彗星犯守之,糴貴,囚徒起兵。



 河鼓三星,旗九星,在牽牛北,天鼓也,主軍鼓,主鈇鉞。一曰三武,主天子三將軍;中央大星為大將軍,左星為左將軍,右星為
 右將軍。左星,南星也,所以備關梁而距難也,設守陰險,知謀徽也。旗即天鼓之旗,所以為旌表也。左旗九星,在鼓左旁。鼓欲正直而明,色黃光澤,將吉;不正,為兵憂也。星怒,馬貴。動則兵起,曲則將失計奪勢。旗星差戾,亂相陵。旗端四星南北列,曰天桴,鼓桴也。星不明,漏刻失時。前近河鼓,若桴鼓相直,皆為桴鼓用。



 離珠五星,在須女北,須女之藏府,女子之星也。天津九星,橫河中,一曰天漢,一曰天江,主四瀆津梁,所以度神通四方也。一星不備,津關道不通。



 騰蛇二十二星,在營室北,天蛇也,主水蟲。王良五星,在奎北,居河中,天子奉車御官也。其四星
 曰天駟,旁一星曰王良,亦曰天馬。其星動,為策馬,車騎滿野。亦曰梁,為天橋,主禦風雨水道,故或占車騎,或占津梁。客星守之,橋不通道。前一星曰策星,王良之御策也,主天子之僕,在王良旁。若移在馬後,是謂策馬,則車騎滿野。閣道六星,在王良前,飛道也。從紫宮至河,神所乘也,一曰,閣道星,天子游別宮之道也。傅路一星,在閣道南,旁別道也。東壁北十星曰天廄,主馬之官,若今驛亭也,主傳令置驛,逐漏馳騖,謂其行急疾,興晷漏競馳也。



 天將軍十二星,在婁北,主武兵。中央大星,天之大將也。南一星曰軍南門,主誰何出入。太陵八
 星在胃北,亦曰積京,主大喪也。積京中星眾,則諸侯有喪,民多疾,兵起。太陵中一星曰積尸,明則死人如山。北九星曰天船,一曰舟星,所以濟不通也。中一星曰積水,候水災。昴西二星曰天街,三光之道,主伺候關梁中外之境。卷舌六星,在昴北,主口語,以知侫讒也。曲,吉;直而動,天下有口舌之害。中一星曰天讒,主巫醫。



 五車五星,三柱九星,在畢北。五車者,五帝車舍也,五帝坐也,主天子五兵,一曰主五穀豐耗。西北大星曰天庫,主太白,主秦。次東北星曰獄,主辰星,主燕趙。次東星曰天倉,主歲星,主魯衛。次東南星曰司空,主填星,主楚。次西南星曰卿
 星,主熒惑,主魏。五星有變,皆以其所主占之。三柱一曰三泉。天子得靈臺之禮,則五車、三柱均明有常。其中五星曰天潢。天潢南三星曰咸池,魚囿也。月、五星入天潢,兵起,道不通,天下亂。五車南六星曰諸王,察諸侯存亡。其西八星曰八穀,主候歲。八穀一星亡,一穀不登。天關一星,在五車南,亦曰天門,日月之所行也,主邊事,主關閉。芒角,有兵。五星守之,貴人多死。



 東井鉞前四星曰司怪,主候天地日月星辰變異及鳥獸草木之妖,明主聞災,修德保福也。司怪西北九星曰坐旗,君臣設位之表也。坐旗西四星曰天高,臺謝之高,主遠望氣象。天高西
 一星曰天河,主察山林妖變。南河、北河各三星,夾東井。一曰天高,天之關門也,主關梁。南河曰南戍,一曰南宮,一曰陽門,一曰越門,一曰權星,主火。北河曰北戍,一曰北宮,一曰陰門,一曰胡門,一曰衡星,主水。兩河戍間,日月五星之常道也。河戍動搖,中國兵起。南河南三星曰闕丘,主宮門外象魏也。五諸侯五星,在東井北,主刺舉,戒不虞。又曰理陰陽,察得失。亦曰主帝心。一曰帝師,二曰帝友,三曰三公,四曰博士,五曰太史,此五者常為帝定疑議。星明大潤澤,則天下大治;芒角,則禍在中。五諸侯南三星曰天尊,主盛饘粥以給貪餒。積水一星,在北河
 西北,水河也,所以供酒食之正也。積薪一星在積水束北,供庖廚之正也。水位四星,在積薪柬,主水衡。客星若水火守犯之,百川流溢。



 軒轅十七星,在七星北。軒轅,黃帝之神,黃龍之體也;后妃之主,土職也。一曰東陵,一曰權星,主雷雨之神。南大星,女主也。次北一星,夫人也,屏也,上將也。次北一星,妃也,次將也。其次諸星,皆次妃之屬也。女主南小星,女御也。左一星少民,后宗也。右一星大民,太后宗也。欲其色黃小而明也。軒轅右角南三星曰酒旗,酒官之旗也,主宴饗飲食。五星守酒旗,天下大酺,有酒肉財物,賜若爵宗室。酒旗南三星曰天相,丞相
 之象也。軒轅西四星曰爟,爟者,烽火之爟也,邊亭之警候。



 爟北四星曰內平,平罪之官,明刑罰。少微四星在太微西,士大夫之位也。一名處士,亦天子副主,或曰博士官,一曰主衛掖門。南第一星處士,第二星議士,第三星博士,第四星大夫。明大而黃,則賢士舉也。月、五星犯守之,處士、女主憂,宰相易。南四星曰長垣,主界域及胡夷。熒惑入之,胡入中國;太白入之,九卿謀。



 ◎二十八舍



 東方。角二星為天關,其間天門也,其內天庭也。故黃道經其中,七曜之所行也。左角為天田,為理,主刑;其南為
 太陽道。右角為將,主兵;其北為太陰道。蓋天之三門,猶房之四表。其星明大,王道太平,賢者在朝;動搖移徙,王者行。



 亢四星,天子之內朝也,總攝天下奏事,聽訟理獄錄功者也。一曰疏廟,主疾疫。星明大,輔納忠,天下寧。



 氐四星,王者之宿宮,后妃之府,休解之房。前二星,適也,後二星,妾也。後二星大,則臣奉度。



 房四星,為明堂,天子布政之宮也,亦四輔也。下第一星,上將也;次,次將也;次,次相也;上星,上相也。南二星君位,北二星夫人位。又為四表,中間為天衢,為天關,黃道之所經也。南間曰陽環,其南曰太陽;北間曰陰間,其北曰太陰。七曜由乎天衢,則
 天下平和;由陽道則旱喪;由陰道則水兵。亦曰天駟,為天馬,主車駕。南星曰左驂,次左服,次右服;次右驂。亦曰天廄,又主開閉,為畜藏之所由也。房星明,則王者明;驂星大,則兵起;星離,民流。又北二小星曰鉤鈐,房之鈐鍵,天之管籥,主閉鍵天心也。明而近房,天下同心。鉤鈐間有星及疏坼,則地動河清。



 心三星,天王正位也。中星曰明堂,天子位,為大辰,主天下之賞罰。天下變動,心星見祥。星明大,天下同。前星為太子,後星為庶子。心星直,則王失勢。



 尾九星,後宮之場,妃后之府。上第一星,后也;次三星,夫人;次星,嬪妾。第三星傍一星名曰神宮,解衣
 之內室。尾亦為九子,星色欲均明,大小相承,則後宮有敘,多子孫。



 箕四星,亦後宮妃后之府。亦曰天津,一曰天雞,主八風。凡日月宿在箕、東壁、翼、軫者風起。又主口舌,主客蠻夷胡貉;故蠻胡將動,先表箕焉。



 北方。南斗六星,天廟也,丞相太宰之位,主褒賢進士,稟授爵祿。又主兵,一曰天機。南二星魁,天梁也。中央二星,天相也。北二星,天府庭也,亦為壽命之期也。將有天子之事,占於斗。斗星盛明,王道平和,爵祿行。



 牽牛六星,天之關梁,主犧牲事。其北二星,一曰即路,一曰聚火。又曰,上一星主道路,次二星主關梁,次三星主南越。搖動變色則占之。星明
 大,王道昌,關梁通。



 須女四星,天少府也。須,賤妾之稱,婦職之卑者也,主布帛裁製嫁娶。



 虛二星,塚宰之官也,主北方邑居廟堂祭祀祝禱事,又主死喪哭泣。



 危三星,主天府天市架屋;餘同虛占。墳墓四星,屬危之下,主死喪哭泣,為墳墓也。



 營室二星,天子之宮也。一曰玄宮,一曰清廟,又為軍糧之府及土功事。星明,國昌;小不明,祠祀鬼神不享。離宮六星,天子之別宮,主隱藏休息之所。



 東壁二星,主文章,天下圖書之秘府也。星明,王者興,道術行,國多君子;星失色,大小不同,王者好武,經士不用,圖書隱;星動,則有土功。



 西方。奎十六星,天之武庫也。一曰天豕,亦曰封豕。主以兵禁暴,又主溝瀆。西南大星,所謂天豕目,亦曰大將,欲其明。



 婁三星,為天獄,主苑牧犧牲,供給郊祀。



 胃三星,天之廚藏,主食廩,五穀府也,明則和平。



 昴七星,天之耳目也,主西方,主獄事。又為旄頭,胡星也。昴、畢間為天街,天子出,旄頭罕畢以前驅,此其義也。黃道之所經也。昴明,則天下牢獄平。昴六星皆明,與大星等,大水。七星皆黃,兵大起。一星亡,為兵喪;搖動,有大臣下獄,及有白衣之會。大而數盡動若跳躍者,胡兵大起。



 畢八星,主邊兵,主弋獵。其大星曰天高,一曰邊將,主四夷之尉也。星明大,則
 遠夷來貢,天下安;失色,則邊兵亂。附耳一星,在畢下,主聽得失,伺愆邪,察不祥。星盛,則中國微,有盜賊,邊候驚,外國反;移動,佞讒行。月入畢,多雨。



 觜觿三星,為三軍之候,行軍之藏府,主葆旅,收斂萬物。明則軍儲盈,將得勢。



 參十星,一曰參伐,一曰大辰,一曰天市,一曰鈇鋮,主斬刈。又為天獄,主殺伐。又主權衡。所以平理也。又主邊城,為九譯,故不欲其動也。參,白獸之體。其中三星橫列,三將也。東北曰左肩,主左將;西北曰右肩,主右將;東南曰左足,主後將軍;西南曰右足,主偏將軍。故《黃帝占》參應七將。中央三小星曰伐,天之都尉也,主胡、鮮卑、戎、狄之國,
 故不欲明。七將皆明大,天下兵精也。王道缺則芒角張。伐星明與參等,大臣皆謀,兵起。參星失色,軍散敗。參芒角動搖,邊候有急,兵起,有斬伐之事。參星移,客伐主。參左足入玉井中,兵大起,秦大水,若有喪,山石為怪。參星差戾,王臣貳。



 南方。東井八星,天之南門,黃道所經,天之亭候,主水衡事,法令所取平也。王者用法平,則井星明而端列。鉞一星,附井之前,主伺淫奢而斬之。故不欲其明,明與井齊,則用鉞於大臣。月宿井,有風雨。



 輿鬼五星,天目也,主視,明察姦謀。東北星主積馬,東南星主積兵,西南星主積
 布帛,西北星主積金玉,隨變占之。中央星為積尸,主死喪祠祀。一曰鈇鑕,主誅斬。鬼星明,大穀成;不明,百姓散。鑕欲其忽忽不明,明則兵起,大臣誅。



 柳八星,天之廚宰也,主尚食,和滋味,又主雷雨。



 七星七星,一名天都,主衣裳文繡,又主急兵盜賊。故星明王道昌;暗則賢良不處,天下空。



 張六星,主珍寶、宗廟所用及衣服,又主天廚飲食賞齎之事。星明則王者行五禮,得天之中。



 翼二十二星,天之樂府俳倡,又主夷狄遠客、負海之賓。星明大,禮樂興,四夷寶。動則蠻夷使來,離徙則天子舉兵。



 軫四星,主冥宰,輔臣也;主車騎,主載任。有軍出入,皆占於軫。
 又主風,主死喪。軫星明,則車駕備;動則車駕用。轄星傅軫兩傍,主王侯,左轄為王者同姓,右轄為異姓,星明,兵大起。遠軫,凶。轄舉,南蠻侵。長沙一星,在軫之中,主壽命。明則主壽長,子孫昌。又曰,車無轄,國有憂;軫就聚,兵大起。



 星官在二十八宿之外者



 庫樓十星,六大星為庫,南四星為樓,在角南。一曰天庫,兵車之府也。旁十五星三三而聚者,柱也。中央四小星,衡也,主陳兵。東北二星曰陽門,主守隘塞也。南門二星,在庫樓南,天之外門也,主守兵。平星二星,在庫樓北,平
 天下之法獄事,廷尉之象也。天門二星,在平星北。



 亢南七星曰折威,主斬殺。頓頑二星,在折威東南,主考囚情狀,察詐偽也。



 騎官二十七星,在氐南,若天子武賁,主宿衛。東端一星騎陣將軍,騎將也。南三星車騎,車騎之將也。陣車三星,在騎官東北,革車也。



 積卒十二星,在房心南,主為衛也。他星守之,近臣誅。從官二星,在積卒西北。



 龜五星,在尾南,主卜以占吉凶。傅說一星,在尾後。傅說主章祝,巫官也。魚一星,在尾後河中,主陰事,知雲雨之期也。



 杵三星,在箕南,杵給庖舂。客星入杵臼,天下有急。穅星在箕舌前杵西北。



 鱉十四星,在南斗南。鱉為水蟲,歸太
 陰。有星守之,白衣會,主有水令。農丈人一星,在南斗西南,老農主穡也。狗二星,在南斗魁前,主吠守。



 天田九星,在牛南。羅堰九星,在牽牛東,岠馬也,以壅蓄水潦,灌溉溝渠也。九坎九星,在牽牛南。坎,溝渠也,所以導達泉源,疏盈瀉溢,通溝洫也。九坎間十星曰天池。一曰三池,一曰天海,主灌溉田疇事。



 虛南二星曰哭,哭東二星曰泣,泣、哭皆近墳墓。泣南十三星曰天壘城,如貫索狀,主北夷於丁零、匈奴。南二星曰蓋屋,治宮室之官也。其南四星曰虛梁,園陵寢廟之所也。



 羽林四十五星,在營室南,一曰天軍,主軍騎,又主翼王也。壘壁陣十二星,在羽林北,羽
 林之垣壘也,主軍衛為營壅也。五星有在天軍中者,皆為兵起,熒惑、太白、辰星尤甚。北落師門一星,在羽林西南。北者,宿在北方也;落,天之籓落也;師,眾也;師門,猶軍門也。長安城北門曰北落門,以象此也。主非常以候兵。有星守之,虜入塞中,兵起。其西北有十星,曰天錢。北落西南一星曰天綱,主武帳。北落東南九星曰八魁,主張禽獸。



 天倉六星,在婁南,倉穀所藏也。南四星曰天庾,積廚粟之所也。



 天囷十三星,在胃南。囷,倉廩之屬也,主給御糧也。



 天廩四星在昴南,一曰天BW,主蓄黍稷以供饗祀;《春秋》所謂御廩,此之象也。天苑十六星,在昴畢南,天子
 之苑囿,養獸之所也。苑南十三星曰天園,植果菜之所也。



 畢附耳南八星曰天節,主使臣之所持者也。天節下九星曰九州殊口,曉方俗之官,通重譯者也。



 參旗九星在參西,一曰天旗,一曰天弓,主司弓弩之張,候變禦難。玉井四星,在參左足下,主水漿以給廚。西南九星曰九游,天子之旗也。玉井東南四星曰軍井,行軍之井也。軍井未達,將不言渴,名取此也。軍市十三星在參東南,天軍貿易之市,使有無通也。野雞一星,主變怪,在軍市中。軍市西南二星曰丈人,丈人東二星曰子,子東二星曰孫。



 東井西南四星曰水府,主水之官也。東井南垣之東
 四星曰四瀆,江、河、淮、濟之精也。狼一星,在東井東南。狼為野將,主侵掠。色有常,不欲動也。北七星曰天狗,主守財。弧九星在狼東南,天弓也,主備盜賊,常向於狼。弧矢動移不如常者,多盜賊,胡兵大起。狼弧張,害及胡,天下乖亂。又曰,天弓張,天下盡兵。弧南六星為天社,昔共工氏之子句龍,能平水土,故祀以配社,其精為星。老人一星,在弧南,一曰南極,常以秋分之旦見于丙,春分之夕而沒于丁。見則治平,主壽昌,常以秋分候之南郊。



 柳南六星曰外廚。廚南一星曰天紀,主禽獸之齒。



 稷五星,在七星南。稷,農正也,取乎百穀之長以為號也。



 張南十四星
 曰天廟,天子之祖廟也。客星守之,祠官有憂。



 翼南五星曰東區,蠻夷星也。



 軫南三十二星曰器府,樂器之府也。青丘七星,在軫東南,蠻夷之國號也。青丘西四星曰土司空,主界域,亦曰司徒。土司空北二星曰軍門,主營候彪尾威旗。



 天漢起沒



 天漢起東方,經尾箕之間,謂之漢津。乃分為二道,其南經傅說、魚、天籥、天弁、河鼓,其北經龜,貫箕下,次絡南斗魁、左旗,至天津下而合南道。乃西南行,又分夾匏瓜,絡人星、杵、造父、騰蛇、王良、傅路、閣道北端、太陵、天船、卷舌
 而南行,絡五車,經北河之南,入東井水位而東南行,絡南河、闕丘、天狗、天紀、天稷,在七星南而沒。



 ○十二次度數



 十二次。班固取《三統歷》十二次配十二野,其言最詳。又有費直說《周易》、蔡邕《月令章句》,所言頗有先後。魏太史令陳卓更言郡國所入宿度,今附而次之。



 自軫十二度至氐四度為壽星,於辰在辰,鄭之分野,屬兗州。費直《周易分野》,壽星起軫七度。蔡邕《月令章句》,壽星起軫六度。



 自氐五度至尾九度為大火,於辰在卯,宋之分野,屬豫州。費直,起氐十一度。蔡邕,起亢八度。



 自
 尾十度至南斗十一度為析木,於辰在寅,燕之分野,屬幽州。費直,起尾九度。蔡邕,起尾四度。



 自南斗十二度至須女七度為星紀,於辰在丑,吳越之分野,屬揚州。費直,起斗十度。蔡邕,起斗六度。



 自須女八度至危十五度為玄枵,於辰在子,齊之分野,屬青州。費直,起女六度。蔡邕,起女十度。



 自危十六度至奎四度為諏訾,於辰在亥,衛之分野,屬並州。費直,起危十四度。蔡邕,起危十度。



 自奎五度至胃六度為降婁,於辰在戌,魯之分野,屬徐州。費直,起奎二度。蔡邕,起奎八度。



 自胃七度至畢十一度為大梁,於辰在酉,趙之分野,屬冀州。費直,起婁十度。蔡邕,起胃一度。



 自畢十二度至東井十五度為實沈,於辰在申,魏之分野,屬益州。費直,起畢九度。蔡邕,起畢六度。



 自東井十六度至柳八度為鶉首,於辰在未,秦之分野,屬雍州。費直,起井十二度。蔡邕,起井十度。



 自柳九度至張十六度為鶉火,於辰在午,周之分野,屬三河。費直,起柳五度。蔡邕,起柳三度。



 自張十七度至軫十一度為鶉尾,於辰在已,楚之分野,屬荊州。費直,起張十三度。蔡邕,起張十二度。



 州郡躔次



 陳卓、范蠡、鬼谷先生、張良、諸葛亮、譙周、京房、張衡並云:



 角、亢、氐,鄭,兗州:



 東郡入角一度東平、任城、山陽入角六度



 泰山入角十二度濟北陳留入亢五度



 濟陰入氐二度東平入氐七度



 房、心,宋,豫州:



 潁川入房一度汝南入房二度



 沛郡入房四度梁國入房五度



 淮陽入心一度魯國入心三度,



 楚國入房四度。



 尾、箕,燕,幽州:



 涼州入箕中十度上谷入尾一度



 漁陽入尾三度右北平入尾七度



 西河、上郡、北地、遼西東入尾十度涿郡入尾十六度



 渤海入箕一度樂浪入箕三度



 玄菟入箕六度廣陽入箕九度。



 斗、牽牛、須女,吳、越,揚州:



 九江入斗一度廬江入斗六度



 豫章入斗十度丹陽入斗十六度



 會稽入牛一度臨淮入牛四度



 廣陵入牛八度泗水入女一度



 六安入女六度



 虛、危,齊,青州:



 齊國入虛六度北海入虛九度



 濟南入危一度樂安入危四度



 東萊入危九度平原入危十一度



 菑川入危十四度



 營室、東壁,衛,并州:



 安定入營室一度天水入營室八度



 隴西入營室四度酒泉入營室十一度



 張掖入營室十二度武都入東壁一度



 金城入東壁四度武威入東壁六度



 敦煌入東壁八度。



 奎、婁、胃,魯,徐州:



 東海入奎一度瑯邪入奎六度



 高密入婁一度城陽入婁九度



 膠東入胃一度



 昴、畢,趙、冀州:



 魏郡入昴一度鉅鹿入昴三度



 常山入昴五度廣平入昴七度



 中山入昴一度清河入昴九度



 信都入畢三度趙郡入畢八度



 安平入畢四度河間入畢十度



 真定入畢十三度



 觜、參,魏,益州:



 廣漢入觜一度越巂入觜三度



 蜀郡入參一度犍為人參三度



 牂柯入參五度巴郡入參八度



 漢中入參九度益州入參七度



 東井、輿鬼,秦,雍州:



 雲中入東井一度定襄入東井八度



 鴈門入東井十六度代郡入東井二十八度



 太原入東井二十九度上黨入輿鬼二度。



 柳、七星、張,周,三輔:



 弘農入柳一度河南入七星三度



 河東入張一度河內入張九度



 翼、軫,楚。荊州:



 南陽入翼六度南郡入翼十度



 江夏入翼十二度零陵入軫十一度



 桂陽入軫六度武陵入軫十度



 長沙入軫十六度



\end{pinyinscope}