\article{列傳第三十五}

\begin{pinyinscope}
王導
 \gezhu{
  子悅恬洽協劭薈洽子殉氏劭子謐}



 王導,字茂弘,光祿大夫覽之孫也。父裁,鎮軍司馬。導少有風鑒,識量清遠。年十四,陳留高士張公見而奇之,謂其從兄敦曰:「此兒容貌志氣,將相之器也。」初襲祖爵即丘子。司空劉實尋引為東閣祭酒,遷祕書郎、太子舍人、尚書郎,並不行。後參東海王越軍事。



 時元帝為瑯邪王,與導素相親善。導知天下已亂,遂傾心推奉,潛有興復
 之志。帝亦雅相器重,契同友執。帝之在洛陽也,導每勸令之國。會帝出鎮下邳,請導為安東司馬,軍謀密策,知無不為。及徙鎮建康,吳人不附,居月餘,士庶莫有至者,導患之。會敦來朝,導謂之曰:「瑯邪王仁德雖厚,而名論猶輕。兄威風已振,宜有以匡濟者。」會三月上巳,帝親觀禊,乘肩輿,具威儀,敦、導及諸名勝皆騎從。吳人紀瞻、顧榮,皆江南之望,竊覘之,見其如此,咸驚懼,乃相率拜於道左。導因進計曰:「古之王者,莫不賓禮故老,存問風俗,虛己傾心,以招俊乂。況天下喪亂,九州分裂,大業草創,急於得人者乎!顧榮、賀循,此土之望,未若引之以結人
 心。二子既至,則無不來矣。」帝乃使導躬造循、榮,二人皆應命而至,由是吳會風靡,百姓歸心焉。自此之後,漸相崇奉,君臣之禮始定。



 俄而洛京傾覆,中州士女避亂江左者十六七,導勸帝收其賢人君子,與之圖事。時荊揚晏安,戶口殷實,導為政務在清靜,每勸帝剋己勵節,匡主寧邦。於是尤見委杖,情好日隆,朝野傾心,號為「仲父」。帝嘗從容謂導曰:「卿,吾之蕭何也。」對曰:「昔秦為無道,百姓厭亂,巨猾陵暴,人懷漢德,革命反正,易以為功。自魏氏以來,迄于太康之際,公卿世族,豪侈相高,政教陵遲,不遵法度,群公卿士,皆饜於安息,遂使人乘釁,有虧
 至道。然否終斯泰,天道之常。大王方立命世之勳,一匡九合,管仲、樂毅,於是乎在,豈區區國臣所可擬議!願深弘神慮,廣擇良能。顧榮、賀循、紀贍、周皆南土之秀,願盡優禮,則天下安矣。」帝納焉。



 永嘉末,遷丹陽太守,加輔國將軍。導上箋曰:「昔魏武,達政之主也;荀文若,功臣之最也,封不過亭侯。倉舒,愛子之寵,贈不過別部司馬。以此格萬物,得不局跡乎!今者臨郡,不問賢愚豪賤,皆加重號,輒有鼓蓋,動見相準。時有不得者,或為恥辱。天官混雜,朝望頹毀。導忝荷重任,不能崇浚山海,而開導亂源,饕竊名位,取紊彞典,謹送鼓蓋加崇之物,請從導始。
 庶令雅俗區別,群望無惑。」帝下令曰:「導德重勳高,孤所深倚,誠宜表彰殊禮。而更約己沖心,進思盡誠,以身率眾,宜順其雅志,式允開塞之機。」拜寧遠將軍,尋加振威將軍。愍帝即位,徵吏部郎,不拜。



 晉國既建,以導為丞相軍諮祭酒。桓彞初過江,見朝廷微弱,謂周顗曰:「我以中州多故,來此欲求全活,而寡弱如此,將何以濟!」憂懼不樂。往見導,極談世事,還,謂顗曰:「向見管夷吾,無復憂矣。」過江人士,每至暇日,相要出新亭飲宴。周顗中坐而歎曰:「風景不殊,舉目有江河之異。」皆相視流涕。惟導愀然變色曰:「當共戮力王室,剋復神州,何至作楚囚相對泣
 邪!」眾收淚而謝之。俄拜右將軍、揚州刺史、監江南諸軍事,遷驃騎將軍,加散騎常侍、都督中外諸軍、領中書監、錄尚書事、假節,刺史如故。導以敦統六州,固辭中外都督。後坐事除節。



 于時軍旅不息,學校未修,導上書曰:



 夫風化之本在於正人倫,人倫之正存乎設庠序。庠序設,五教明,德禮洽通,彞倫攸敘,而有恥且格,父子兄弟夫婦長幼之序順,而君臣之義固矣。《易》所謂「正家而天下定」者也。故聖王蒙以養正,少而教之,使化霑肌骨,習以成性,遷善遠罪而不自知,行成德立,然後裁之以位。雖王之世子,猶與國子齒,使知道而後貴。其取才用士,咸
 先本之於學。故《周禮》,卿大夫獻賢能之書于王,王拜而受之,所以尊道而貴士也。人知士之貴由道存,則退而修其身以及家,正其家以及鄉,學於鄉以登朝,反本復始,各求諸己,敦樸之業著,浮偽之競息,教使然也。故以之事君則忠,用之蒞下則仁。孟軻所謂「未有仁而遺其親,義而後其君者也」。



 自頃皇綱失統,頌聲不興,于今將二紀矣。《傳》曰:「三年不為禮,禮必壞;三年不為樂,樂必崩。」而況如此之久乎!先進忘揖讓之容,後生惟金鼓是聞,干戈日尋,俎豆不設,先王之道彌遠,華偽之俗遂滋,非所以端本靖末之謂也。殿下以命世之資,屬陽九之運,
 禮樂征伐,翼成中興。誠宜經綸稽古,建明學業,以訓後生,漸之教義,使文武之道墜而復興,俎豆之儀幽而更彰。方今戎虜扇熾,國恥未雪,忠臣義夫所以扼腕拊心。茍禮儀膠固,淳風漸著,則化之所感者深而德之所被者大。使帝典闕而復補,皇綱弛而更張,獸心革面,饕餮檢情,揖讓而服四夷,緩帶而天下從。得乎其道,豈難也哉!故有虞舞干戚而化三苗,魯僖作泮宮而服淮夷。桓文之霸,皆先教而後戰。今若聿遵前典,興復道教,擇朝之子弟並入于學,選明博修禮之士而為之師,化成俗定,莫尚於斯。



 帝甚納之。



 及帝登尊號,百官陪列,命導升
 御床共坐。導固辭,至于三四,曰:「若太陽下同萬物,蒼生何由仰照!」帝乃止。進驃騎大將軍、儀同三司。以討華軼功,封武岡侯。進位侍中、司空、假節、錄尚書,領中書監。會太山太守徐龕反,帝訪可以鎮撫河南者,導舉太子左衛率羊鑒。既而鑒敗,抵罪。導上疏曰:「徐龕叛戾,久稽天誅,臣創議征討,調舉羊鑒。鑒闇懦覆師,有司極法。聖恩降天地之施,全其首領。然臣受重任,總錄機衡,使三軍挫衄,臣之責也。乞自貶黜,以穆朝倫。」詔不許。尋代賀循領太子太傅。時中興草創,未置史官,導始啟立,於是典籍頗具。時孝懷太子為胡所害,始奉諱,有司奏天子三
 朝舉哀,群臣一哭而已。導以為皇太子副貳宸極,普天有情,宜同三朝之哀。從之。及劉隗用事,導漸見疏遠,任真推分,澹如也。有識咸稱導善處興廢焉。



 王敦之反也,劉隗勸帝悉誅王氏,論者為之危心。導率群從昆弟子姪二十餘人,每旦詣臺待罪。帝以導忠節有素,特還朝服,召見之。導稽首謝曰:「逆臣賊子,何世無之,豈意今者近出臣族!」帝跣而執之曰:「茂弘,方託百里之命於卿,是何言邪!」乃詔曰:「導以大義滅親,可以吾為安東時節假之。」及敦得志,加導守尚書令。初,西都覆沒,海內思主,群臣及四方並勸進於帝。時王氏彊盛,有專天下之心,敦
 憚帝賢明,欲更議所立,導固爭乃止。及此役也,敦謂導曰:「不從吾言,幾致覆族。」導猶執正議,敦無以能奪。



 自漢魏已來,賜謚多由封爵,雖位通德重,先無爵者,例不加謚。導乃上疏,稱「武官有爵必謚,卿校常伯無爵不謚,甚失制度之本意也」。從之。自後公卿無爵而謚,導所議也。



 初,帝愛瑯邪王裒,將有奪嫡之議,以問導。導曰:「夫立子以長,且紹又賢,不宜改革。」帝猶疑之。導日夕陳諫,故太子卒定。及明帝即位,導受遺詔輔政,解揚州,遷司徒,一依陳群輔魏故事。王敦又舉兵內向。時敦始寢疾,導便率子弟發哀,眾聞,謂敦死,咸有奮志。及帝伐敦,假導節,
 都督諸軍,領揚州刺史。敦平,進封始興郡公,邑三千戶,賜絹九千匹,進位太保,司徒如故,劍履上殿,入朝不趨,贊拜不名。固讓。帝崩,導復與庾亮等同受遺詔,共輔幼主,是為成帝。加羽葆鼓吹,班劍二十人。及石勒侵阜陵,詔加導大司馬、假黃鉞,出討之。軍次江寧,帝親餞于郊。俄而賊退,解大司馬。



 庾亮將征蘇峻,訪之於導。導曰:「峻猜阻,必不奉詔。且山藪藏疾,宜包容之。」固爭不從,亮遂召峻。既而難作,六軍敗績,導入宮侍帝。峻以導德望,不敢加害,猶以本官居己之右。峻又逼乘輿幸石頭,導爭之不得。峻日來帝前肆醜言,導深懼有不測之禍。時路
 永、匡術、賈寧並說峻,令殺導,盡誅大臣,更樹腹心。峻敬導,不納,故永等貳於峻。導使參軍袁耽潛諷誘永等,謀奉帝出奔義軍。而峻衙禦甚嚴,事遂不果。導乃攜二子隨永奔于白石。



 及賊平,宗廟宮室並為灰燼,溫嶠議遷都豫章,三吳之豪請都會稽,二論紛紜,未有所適。導曰:「建康,古之金陵,舊為帝里,又孫仲謀、劉玄德俱言王者之宅。古之帝王不必以豐儉移都,茍弘衛文大帛之冠,則無往不可。若不績其麻,則樂土為虛矣。且北寇游魂,伺我之隙,一旦示弱,竄於蠻越,求之望實,懼非良計。今特宜鎮之以靜,群情自安。」由是嶠等謀並不行。



 導善於
 因事,雖無日用之益,而歲計有餘。時帑藏空竭,庫中惟有練數千端,鬻之不售,而國用不給。導患之,乃與朝賢俱制練布單衣,於是士人翕然競服之,練遂踴貴。乃令主者出賣,端至一金。其為時所慕如此



 六年冬,烝,詔歸胙於導,曰:「無下拜。」導辭疾不敢當。初,帝幼沖,見導,每拜。又嘗與導書手詔,則云「惶恐言」,中書作詔,則曰「敬問」,於是以為定制。自後元正,導入,帝猶為之興焉。



 時大旱,導上疏遜位。詔曰:「夫聖王御世,動合至道,運無不周,故能人倫攸敘,萬物獲宜。朕荷祖宗之重,託於王公之上,不能仰陶玄風,俯洽宇宙,亢陽踰時,兆庶胥怨,邦之不臧,
 惟予一人。公體道明哲,弘猶深遠,勳格四海,翼亮三世,國典之不墜,實仲山甫補之。而猥崇謙光,引咎克讓,元道之愆,寄責宰輔,只增其闕。博綜萬機,不可一日有曠。公宜遺履謙之近節,遵經國之遠略。門下速遣侍中以下敦喻。」導固讓。詔累逼之,然後視事。



 導簡素寡欲,倉無儲穀,衣不重帛。帝知之,給布萬匹,以供私費。導有羸疾,不堪朝會,帝幸其府,縱酒作樂,後令輿車入殿,其見敬如此。



 石季龍掠騎至歷陽,導請出討之。加大司馬、假黃鉞、中外諸軍事,置左右長史、司馬,給布萬匹。俄而賊退,解大司馬,復轉中外大都督,進位太傅,又拜丞相,依漢
 制罷司徒官以並之。冊曰:「朕夙罹不造,肆陟帝位,未堪多難,禍亂旁興。公文貫九功,武經七德,外緝四海,內齊八政,天地以平,人神以和,業同伊尹,道隆姬旦。仰思唐虞,登庸雋乂,申命群官,允釐庶績。朕思憑高謨,弘濟遠獻,維稽古建爾于上公,永為晉輔。往踐厥職,敬敷道訓,以亮天工。不亦休哉!公其戒之!」



 是歲,妻曹氏卒,贈金章紫綬。初,曹氏性妒,導甚憚之,乃密營別館,以處眾妾。曹氏知,將往焉。導恐妾被辱,遽令命駕,猶恐遲之,以所執麈尾柄驅牛而進。司徒蔡謨聞之,戲導曰:「朝廷欲加公九錫。」導弗之覺,但謙退而已。謨曰:「不聞餘物,惟有短轅
 犢車,長柄麈尾。」導大怒,謂人曰:「吾往與群賢共游洛中,何曾聞有蔡克兒也。」



 于時庾亮以望重地逼,出鎮於外。南蠻校尉陶稱間說亮當舉兵內向,或勸導密為之防。導曰:「吾與元規休戚是同,悠悠之談,宜絕智者之口。則如君言,元規若來,吾便角巾還第,復何懼哉!」又與稱書,以為庾公帝之元舅,宜善事之。於是讒間遂息。時亮雖居外鎮,而執朝廷之權,既據上流,擁彊兵,趣向者多歸之。導內不能平,常遇西風塵起,舉扇自蔽,徐曰:「元規塵污人。」



 自漢魏以來,群臣不拜山陵。導以元帝睠同布衣,匪惟君臣而已,每一崇進,皆就拜,不勝哀戚。由是詔百
 官拜陵,自導始也。



 咸康五年薨,時年六十四。帝舉哀於朝堂三日,遣大鴻臚持節監護喪事,賵襚之禮,一依漢博陸侯及安平獻王故事。及葬,給九游轀輬車、黃屋左纛、前後羽葆鼓吹、武賁班劍百人,中興名臣莫與為比。冊曰:「蓋高位以酬明德,厚爵以答懋勳;至乎闔棺標跡,莫尚號謚,風流百代,於是乎在。惟公邁達沖虛,玄鑒劭邈;夷淡以約其心,體仁以流其惠;棲遲務外,則名雋中夏,應期濯纓,則潛算獨運。昔我中宗、肅祖之基中興也,下帷委誠而策定江左,拱己宅心而庶績咸熙。故能威之所振,寇虐改心,化之所鼓,檮杌易質;調陰陽之和,通
 彞倫之紀,遼隴承風,丹穴景附。隆高世之功,復宣武之績,舊物不失,公協其猷。若乃荷負顧命,保朕沖人,遭遇艱圮,夷險委順;拯其淪墜而濟之以道,扶其頹傾而弘之以仁,經緯三朝而蘊道彌曠。方賴高謨,以穆四海,昊天不弔,奄忽薨殂,朕用震慟于心。雖有殷之殞保衡,有周之喪二南,曷諭茲懷!今遣使持節、謁者僕射任瞻錫謚曰文獻,祠以太牢。魂而有靈,嘉茲榮寵!」



 二弟:穎、敞,少與導俱知名,時人以穎方溫太真,以敞比鄧伯道,並早卒。導六子:悅、恬、洽、協、邵、薈。



 悅字長豫,弱冠有高名,事親色養,導甚愛之。導嘗共悅
 奕棋,爭道,導笑曰:「相與有瓜葛,那得為爾邪!」導性儉節,帳下甘果爛敗,令棄之,云:「勿使大郎知。」悅少侍講東宮,歷吳王友、中書侍郎,先導卒,謚貞世子。先是,導夢人以百萬錢買悅,潛為祈禱者備矣。尋掘地,得錢百萬,意甚惡之,一皆藏閉。及悅疾篤,導憂念特至,不食積日。忽見一人形狀甚偉,被甲持刀,導問:「君是何人?」曰:「僕是蔣侯也。公兒不佳,欲為請命,故來耳。公勿復憂。」因求食,遂啖數升。食畢,勃然謂導曰:「中書患,非可救者。」言訖不見,悅亦殞絕。悅與導語,恆以慎密為端。導還臺,及行,悅未嘗不送至車後,又恒為母曹氏襞斂箱篋中物。悅亡後,導
 還臺,自悅常所送處哭至臺門,其母長封作篋,不忍復開。



 悅無子,以弟恬子琨為嗣,襲導爵丹陽尹,卒,贈太常。子嘏嗣,尚鄱陽公主,歷中領軍、尚書。卒,子恢嗣,義熙末,為游擊將軍。



 恬字敬豫。少好武,不為公門所重。導見悅輒喜,見恬便有怒色。州辟別駕,不行,襲爵即丘子。性傲誕,不拘禮法。謝萬嘗造恬,既坐,少頃,恬便入內。萬以為必厚待己,殊有喜色。恬久之乃沐頭散髮而出,據胡床於庭中曬髮,神氣傲邁,竟無賓主之禮。萬悵然而歸。晚節更好士,多技藝,善奕棋,為中興第一。遷中書郎。帝欲以為中書令,
 導固讓,從之。除後將軍、魏郡太守,加給事中,領兵鎮石頭。導薨,去官。俄起為後將軍,復鎮石頭。轉吳國、會稽內史,加散騎常侍。卒,贈中軍將軍,謚曰憲。



 洽字敬和,導諸子中最知名,與荀羨俱有美稱。弱冠,歷散騎、中書郎、中軍長史、司徒左長史、建武將軍、吳郡內史。徵拜領軍,尋加中書令,固讓,表疏十上。穆帝詔曰:「敬和清裁貴令,昔為中書郎,吾時尚小,數呼見,意甚親之。今所以用為令,既機任須才,且欲時時相見,共講文章,待以友臣之義。而累表固讓,甚違本懷。其催洽令拜。」苦讓,遂不受。升平二年卒於官,年三十六。二子:珣、氏。



 珣字元琳。弱冠與陳郡謝玄為桓溫掾,俱為溫所敬重,嘗謂之曰:「謝掾年四十,必擁旄杖節。王掾當作黑頭公。皆未易才也。」珣轉主簿。時溫經略中夏,竟無寧歲,軍中機務並委珣焉。文武數萬人,悉識其面。從討袁真,封東亭侯,轉大司馬參軍、瑯邪王友、中軍長史、給事黃門侍郎。



 珣兄弟皆謝氏婿,以猜嫌致隙。太傅安既與珣絕婚,又離氏妻,由是二族遂成仇釁。時希安旨,乃出珣為豫章太守,不之官。除散騎常侍,不拜。遷祕書監。安卒後,遷侍中,孝武深杖之。轉輔國將軍、吳國內史,在郡為士庶所悅。徵為尚書右僕射,領吏部,轉左僕射,加征虜將軍,
 復領太子詹事。



 時帝雅好典籍,珣與殷仲堪、徐邈、王恭、郗恢等並以才學文章見暱於帝。及王國寶自媚於會稽王道子,而與珣等不協,帝慮晏駕後怨隙必生,故出恭、恢為方伯,而委珣端右。珣夢人以大筆如椽與之,既覺,語人云:「此當有大手筆事。」俄而帝崩,哀冊謚議,皆珣所草。



 隆安初,國寶用事,謀黜舊臣,遷珣尚書令。王恭赴山陵,欲殺國寶,珣止之曰:「國寶雖終為禍亂,要罪逆未彰,今便先事而發,必大失朝野之望。況擁強兵,竊發於京輦,誰謂非逆!國寶若遂不改,惡布天下,然後順時望除之,亦無尤不濟也。」恭乃止。既而謂珣曰:「比來視君,一
 似胡廣。」旬曰:「王陵廷爭,陳平慎默,但問歲終何如耳。」恭尋起兵,國寶將殺珣等,僅而得免,語在國寶傳。二年,恭復舉兵,假珣節,進衛將軍、都督瑯邪水陸軍事。事平,上所假節,加散騎常侍。



 四年,以疾解職。歲餘,卒,時年五十二。追贈車騎將軍、開府,謚曰獻穆。桓玄與會稽王道子書曰:「珣神情朗悟,經史明徹,風流之美,公私所寄。雖逼嫌謗,才用不盡;然君子在朝,弘益自多。時事艱難,忽爾喪失,歎懼之深,豈但風流相悼而已!其崎嶇九折,風霜備經,雖賴明公神鑒,亦識會居之故也。卒以壽終,殆無所哀。但情發去來,置之未易耳。」玄輔政,改贈司徒。



 初,珣
 既與謝安有隙,在東聞安薨,便出京師,詣族弟獻之,曰:「吾欲哭謝公。」獻之驚曰:「所望於法護。」於是直前哭之甚慟。法護,珣小字也。珣五子:弘、虞、柳、孺、曇首,宋世並有高名。



 氏字季琰。少有才藝,善行書,名出珣右。時人為之語曰:「法護非不佳,僧彌難為兄。」僧彌,氏小字也。時有外國沙門,名提婆,妙解法理,為珣兄弟講《毗曇經》。氏時尚幼,講未半,便云已解,即於別室與沙門法綱等數人自講。法綱歎曰:「大義皆是,但小未精耳。」辟州主簿,舉秀才,不行。後歷著作、散騎郎、國子博士、黃門侍郎、侍中,代王獻之
 為長兼中書令。二人素齊名,世謂獻之為「大令」,氏為「小令」。太元十三年卒,時年三十八,追贈太常。二子:朗、練。義熙中,並歷侍中。



 協字敬祖,元帝撫軍參軍,襲爵武岡侯,早卒,無子,以弟劭子謐為嗣。



 謐字稚遠。少有美譽,與譙國桓胤、太原王綏齊名。拜秘書郎,襲父爵,遷祕書丞,歷中軍長史、黃門郎、侍中。及桓玄舉兵,詔謐銜命詣玄,玄深敬暱焉。拜建威將軍、吳國內史,未至郡,玄以為中書令、領軍將軍、吏部尚書,遷中書監,加散騎常侍,領司徒。及玄將篡,以謐兼太保,奉璽
 冊詣玄。玄篡,封武昌縣開國公,加班劍二十人。



 初,劉裕為布衣,眾未之識也,惟謐獨奇貴之,嘗謂裕曰:「卿當為一代英雄。」及裕破恆玄,謐以本官加侍中,領揚州刺史、錄尚書事。謐既受寵桓氏,常不自安。護軍將軍劉毅嘗問謐曰:「璽綬何在?」謐益懼。會王綏以桓氏甥自疑,謀反,父子兄弟皆伏誅。謐從弟諶,少驍果輕俠,欲誘謐還吳,起兵為亂,乃說謐曰:「王綏無罪,而義旗誅之,是除時望也。兄少立名譽,加位地如此,欲不危,得乎!」謐懼而出奔。劉裕箋詣大將軍、武陵王遵,遣人追躡,謐既還,委任如先,加謐班劍二十人。義熙三年卒,時年四十八。追贈侍
 中、司徒,謚曰文恭。三子:瓘、球、琇。入宋,皆至大官。



 劭字敬倫,歷東陽太守、吏部郎、司徒左長史、丹陽尹。劭美姿容,有風操,雖家人近習,未嘗見其墜替之容。桓溫甚器之。遷吏部尚書、尚書僕射,領中領軍,出為建威將軍、吳國內史。卒,贈車騎將軍,謚曰簡。三子:穆、默、恢。穆,臨海太守。默,吳國內史,加二千石。恢,右衛將軍。穆三子:簡、智、超。默二子:鑒、惠。義熙中,並歷顯職。



 薈字敬文。恬虛守靖,不競榮利,少歷清官,除吏部郎、侍中、建威將軍、吳國內史。時年饑粟貴,人多餓死,薈以私米作饘粥,以飴餓者,所濟活甚眾。徵補中領軍,不拜。徙
 尚書,領中護軍,復為征虜將軍、吳國內史。頃之,桓沖表請薈為江州刺史,固辭不拜。轉督浙江東五郡、左將軍、會稽內史,進號鎮軍將軍,加散騎常侍。卒於官,贈衛將軍。



 子廞,歷太子中庶子、司徒左長史。以母喪,居於吳。王恭舉兵,假廞建武將軍、吳國內史,令起軍,助為聲援。廞即墨絰合眾,誅殺異己,仍遣前吳國內史虞嘯父等入吳興、義興聚兵,輕俠赴者萬計。廞自謂義兵一動,勢必未寧,可乘間而取富貴。而曾不旬日,國寶賜死,恭罷兵符,廞去職。廞大怒,迴眾討恭。恭遣司馬劉牢之距戰于曲阿,廞眾潰奔走,遂不知所在。長子泰為恭所殺,少子
 華以不知廞存亡,憂毀布衣蔬食。後從兄謐言其死所,華始發喪,入仕。



 初,導渡淮,使郭璞筮之,卦成,璞曰:「吉,無不利。淮水絕,王氏滅。」其後子孫繁衍,竟如璞言。



 史臣曰:飛龍御天,故資雲雨之勢;帝王興運,必俟股肱之力。軒轅,聖人也,杖師臣而授圖;商湯,哲后也,託負鼎而成業。自斯已降,罔不由之。原夫典午發蹤,本于陵寡,金行撫運,無德在時。九土未宅其心,四夷已承其弊。既而中原蕩覆,江左嗣興,兆著玄石之圖,乖少康之祀夏;時無思晉之士,異文叔之興劉;輔佐中宗,艱哉甚矣!茂弘策名枝屏,葉情交好,負其才智,恃彼江湖,思建剋復
 之功,用成翌宣之道。於是王敦內侮,憑天邑而狼顧;蘇峻連兵,指宸居而隼擊。實賴元宰,固懷匪石之心;潛運忠謨,竟翦吞沙之寇。乃誠貫日,主垂餌以終全;貞志陵霜,國綴旒而不滅。觀其開設學校,存乎沸鼎之中,爰立章程,在乎櫛風之際;雖則世道多故,而規模弘遠矣。比夫蕭曹弼漢,六合為家;奭望匡周,萬方同軌,功未半古,不足為儔。至若夷吾體仁,能相小國;孔明踐義,善翊新邦,撫事論情,抑斯之類也。提挈三世,終始一心,稱為「仲父」,蓋其宜矣。恬珣踵德,副呂虔之贈刀;謐乃聵聲,慚劉毅之徵璽。語曰:「深山大澤,有龍有蛇。」實斯之謂也。



 贊曰:虎嘯猋馳,龍升雲映。武岡矯矯,匡時輯政。懿績克宣,忠規靡競。契葉三主,榮逾九命。貽刀表祥,巫水流慶。赫矣門族,重光斯盛。



\end{pinyinscope}