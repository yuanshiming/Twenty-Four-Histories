\article{列傳第五十五}

\begin{pinyinscope}
劉毅
 \gezhu{
  兄邁}
 諸葛長民何無忌檀憑之魏詠之



 劉毅,字希樂,彭城沛人也。曾祖距,廣陵相。叔父鎮,左光祿大夫。毅少有大志,不修家人產業,仕為州從事,桓弘以為中兵參軍屬。桓玄篡位,毅與劉裕、何無忌、魏詠之等起義兵,密謀討玄,毅討徐州刺史桓修於京口、青州刺史桓弘於廣陵。裕率毅等至竹里,玄使其將皇甫敷、吳甫之北距義軍,遇之於江乘,臨陣斬甫之,進至羅落
 橋,又斬敷首。玄大懼,使桓謙、何澹之屯覆舟山。毅等軍至蔣山,裕使羸弱登山,多張旗幟,玄不之測,益以危懼。謙等士卒多北府人,素懾伏裕,莫敢出鬥。裕與毅等分為數隊,進突謙陣,皆殊死戰,無不一當百。時東北風急,義軍放火,煙塵張天,鼓噪之音震駭京邑,謙等諸軍一時奔散。玄既西走,裕以毅為冠軍將軍、青州刺史,與何無忌、劉道規躡玄。玄逼帝及瑯邪王西上,毅與道規及下邳太守孟懷玉等追及玄,戰於崢嶸洲。毅乘風縱火。盡銳爭先,玄眾大潰,燒輜重夜走。玄將郭銓、劉雅等襲陷尋陽,毅遣武威將軍劉懷肅討平之。



 及玄死,桓振、桓
 謙復聚眾距毅於靈溪。玄將馮該以兵會振,毅進擊,為振所敗,退次尋陽,坐免官,尋原之。劉裕命何無忌受毅節度,無忌以督攝為煩,輒便解統。毅疾無忌專擅,免其瑯邪內史,以輔國將軍攝軍事,無忌遂與毅不平。毅唯自引咎,時論韙之。毅復與道規發尋陽。桓亮自號江州刺史,遣劉敬宣擊走之。毅軍次夏口。時振黨馮該戍大岸,孟山圖據魯城,桓山客守偃月壘,眾合萬人,連艦二岸,水陸相援。毅督眾軍進討,未至復口,遇風飄沒千餘人。毅與劉懷肅、索邈等攻魯城,道規攻偃月壘,何無忌與檀祗列艦於中流,以防越逸。毅躬貫甲胄,陵城半
 日而二壘俱潰,生擒山客,而馮該遁走。毅進平巴陵。以毅為使持節、兗州刺史,將軍如故。毅號令嚴整,所經墟邑,百姓安悅。南陽太守魯宗之起義,襲襄陽,破桓蔚。毅等諸軍次江陵之馬頭。振擁乘輿,出營江津。宗之又破偽將溫楷,振自擊宗之。毅因率無忌、道規等諸軍破馮該於豫章口,推鋒而進,遂入江陵。振聞城陷,與謙北走,乘輿反正。毅執玄黨卞範之、羊僧壽、夏侯崇之、桓道恭等,皆斬之。桓振復與苻宏自鄖城襲陷江陵,與劉懷肅相持。毅遣部將擊振,殺之,并斬偽輔將軍桓珍。毅又攻拔遷陵,斬玄太守劉叔祖於臨幛。其餘擁眾假號以
 十數,皆討平之。二州既平,以毅為撫軍將軍。時刁預等作亂,屯於湘中,毅遣將分討,皆滅之。



 初,毅丁憂在家,及義旗初興,遂墨絰從事。至是,軍役漸寧,上表乞還京口,以終喪禮,曰:「弘道為國者,理盡於仁孝。訴窮歸天者,莫甚於喪親。但臣凡庸,本無感概,不能隕越,故其宜耳。往年國難滔天,故志竭愚忠,靦然茍存。去春鸞駕迴軫,而狂狡未滅,雖姦凶時梟,餘燼竄伏,威懷寡方,文武勞弊,微情未申,顧景悲憤。今皇威遐肅,海內清蕩,臣窮毒艱穢,亦已具於聖聽。兼羸患滋甚,眾疾互動,如今寢頓無復人理。臣之情也,本不甘生;語其事也,亦可以沒。乞賜
 餘骸,終其丘墳,庶幾忠孝之道獲宥於聖世。」不許。詔以毅為都督豫州揚州之淮南歷陽廬江安豐堂邑五郡諸軍事、豫州刺史,持節、將軍、常侍如故,本府文武悉令西屬。以匡復功,封南平郡開國公,兼都督宣城軍事,給鼓吹一部。梁州刺史劉稚反,毅遣將討擒之。初,桓玄於南州起齋,悉畫盤龍於其上,號為盤龍齋。毅小字盤龍,至是,遂居之。俄進拜衛將軍、開府儀同三司。



 及何無忌為盧循所敗,賊軍乘勝而進,朝廷震駭。毅具舟船討之,將發,而疾篤,內外失色。朝議欲奉乘輿北就中軍劉裕,會毅疾瘳,將率軍南征,裕與毅書曰:「吾往與妖賊戰,曉其變
 態。今修船垂畢,將居前撲之。剋平之日,上流之任皆以相委。」又遣毅從弟籓往止之。毅大怒,謂籓曰:「我以一時之功相推耳,汝便謂我不及劉裕也!」投書於地。遂以舟師二萬發姑孰。徐道覆聞毅將至建鄴,報盧循曰:「劉毅兵重,成敗擊此一戰,宜併力距之。」循乃引兵發巴陵,與道覆連旗而下。毅次於桑落洲,與賊戰,敗績,棄船,以數百人步走,餘眾皆為賊所虜,輜重盈積,皆棄之。毅走,經涉蠻晉,飢困死亡,至得十二三。參軍羊邃竭力營護之,僅而獲免。劉裕深慰勉之,復其本職。毅乃以邃為諮議參軍。



 及裕討循,詔毅知內外留事。毅以喪師,乞解任,降
 為後將軍。尋轉衛將軍、開府儀同三司、江州都督。毅上表曰:



 臣聞天以盈虛為運,政以損益為道。時否而政不革,人凋而事不損,則無以救急病於已危,拯塗炭於將絕。自頃戎車屢駭,干戈溢境,所統江州,以一隅之地當逆順之衝,自桓玄以來,驅蹙殘敗,至乃男不被養,女無匹對,逃亡去就,不避幽深,自非財殫力竭,無以至此。若不曲心矜理,有所釐改,則靡遺之歎奄焉必及。



 夫設官分職,軍國殊用,牧養以息務為大,武略以濟事為先。兼而領之,蓋出於權事,因藉既久,遂似常體。江州在腹心之內,憑接揚豫,籓屏所倚,實為重復。昔胡寇縱逸。朔馬
 臨江,抗禦之宜,蓋權爾耳。今江左區區,戶不盈數十萬,地不踰數千里,而統旅鱗次,未獲減息,大而言之,足為國恥。況乃地在無虞,而猶置軍府文武將佐,資費非要,豈所謂經國大情,揚湯去火者哉!自州郡邊江,百姓遼落,加郵亭險閡,畏阻風波,轉輸往復,恒有淹廢,又非所謂因其所利以濟其弊者也,愚謂宜解軍府,移鎮豫章,處十郡之中,厲簡惠之政,比及數年,可有生氣。且屬縣凋散,示有所存,而役調送迎不得止息,亦謂應隨宜并合以簡眾費。刺史庾悅,自臨蒞以來,甚有恤隱之誠,但綱維不革,自非綱目所理。尋陽接蠻,宜示有遏防,可即
 州府千兵以助郡戍。



 於是解悅,毅移鎮豫章,遣其親將趙恢領千兵守尋陽。俄進毅為都督荊寧秦雍四州之河東河南廣平揚州之義成四郡諸軍事、衛將軍、開府儀同三司、荊州刺史,持節、公如故。毅表荊州編戶不盈十萬,器械索然。廣州雖凋殘,猶出丹漆之用,請依先準。於是加督交、廣二州。



 毅至江陵,乃輒取江州兵及豫州西府文武萬餘,留而不遣,又告疾困,請籓為副。劉裕以毅貳於己,乃奏之。安帝下詔曰:「劉毅傲很凶戾,履霜日久,中間覆敗,宜即顯戮。晉法含弘,復蒙寵授。曾不思愆內訟,怨望滋甚。賴宰輔藏疾,特加遵養,遂復推轂陜西,
 寵榮隆泰,庶能洗心感遇,革音改意,而長惡不悛,志為姦宄,陵上虐下,縱逸無度。既解督任,江州非復所統,而輒徙兵眾,略取軍資,驅斥舊戍,厚樹親黨。西府二局,文武盈萬,悉皆割留,曾無片言。肆心恣欲,罔顧天朝。又與從弟籓遠相影響,招聚剽狡,繕甲阻兵,外託省疾,實規伺隙,同惡想濟,圖會荊郢。尚書左僕射謝混憑藉世資,超蒙殊遇,而輕佻躁脫,職為亂階,扇動內外,連謀萬里。是而可忍,孰不可懷!」乃誅籓、混。



 劉裕自率眾討毅,命王弘、王鎮惡、蒯恩等率軍至豫章口,於江津燔舟而進。毅參軍朱顯之逢鎮惡,以所統千人赴毅。鎮惡等攻陷外
 城,毅守內城,精銳尚數千人,戰至日昃,鎮惡以裕書示城內,毅怒,不發書而焚之。毅冀有外救,督士卒力戰。眾知裕至,莫有鬥心。既暮,鎮惡焚諸門,齊力攻之,毅眾乃散,毅自北門單騎而走,去江陵二十里而縊。經宿,居人以告,乃斬於市,子侄皆伏誅。毅兄模奔於襄陽,魯宗之斬送之。



 毅剛猛沈斷,而專肆很愎,與劉裕協成大業,而功居其次,深自矜伐,不相推伏。及居方嶽,常怏怏不得志,裕每柔而順之。毅驕縱滋甚,每覽史籍,至藺相如降屈於廉頗,輒絕歎以為不可能也。嘗云:「恨不遇劉項,與之爭中原。」又謂郗僧施曰:「昔劉備之有孔明,猶魚之有
 水。今吾與足下雖才非古賢,而事同斯言。」眾咸惡其陵傲不遜。及敗於桑落,知物情去己,彌復憤激。初,裕征盧循,凱歸,帝大宴於西池,有詔賦詩。毅詩云:「六國多雄士,正始出風流。」自知武功不競,故示文雅有餘也。後於東府聚樗蒱大擲,一判應至數百萬,餘人並黑犢以還,唯劉裕及毅在後。毅次擲得雉,大喜,褰衣繞床,叫謂同坐曰:「非不能盧,不事此耳。」裕惡之,因挼五木久之,曰:「老兄試為卿答。」既而四子俱黑,其一子轉躍未定,裕厲聲喝之,即成盧焉。毅意殊不快,然素黑,其面如鐵色焉,而乃和言曰:「亦知公不能以此見借!」既出西籓,雖上流分
 陜,而頓失內權,又頗自嫌事計,故欲擅其威彊,伺隙圖裕,以至於敗。



 初,江州刺史庾悅,隆安中為司徒長史,曾至京口。毅時甚屯窶,先就府借東堂與親故出射。而悅後與僚佐徑來詣堂,毅告之曰:「毅輩屯否之人,合一射甚難。君於諸堂並可,望以今日見讓。」悅不許。射者皆散,唯毅留射如故。既而悅食鵝,毅求其餘,悅又不答,毅常銜之。義熙中,故奪悅豫章,解其軍府,使人微示其旨,悅忿懼而死。毅之褊躁如此。



 邁字伯群。少有才幹,為殷仲堪中兵參軍。桓玄之在江陵,甚豪橫,士庶畏之過於仲堪。玄曾於仲堪事前戲
 馬,以槊擬仲堪。邁時在坐,謂玄曰:「馬槊有餘,精理不足。」玄自以才雄冠世,而心知外物不許之。仲堪為之失色,玄出,仲堪謂邁曰:「卿乃狂人也!玄夜遣殺卿,我豈能相救!」邁以正辭折仲堪,而不以為悔。仲堪使邁下都以避之。玄果令追之,邁僅而免禍。後玄得志,邁詣門稱謁,玄謂邁曰:「安知不死而敢相見?」邁對曰:「射鉤、斬袪,與邁為三,故知不死。」玄甚喜,以為刑獄參軍。後為竟陵太守。及毅與劉裕等同謀起義,邁將應之,事泄,為玄所害。



 諸葛長民,瑯邪陽都人也。有文武幹用,然不持行檢,無
 鄉曲之譽。桓玄引為參軍平西軍事,尋以貪刻免。及劉裕建義,與之定謀,為揚武將軍。從裕討桓玄,以功拜輔國將軍、宣城內史。于時桓歆聚眾向歷陽,長民擊走之,又與劉敬宣破歆于芍陂,封新淦縣公,食邑二千五百戶,以本官督淮北諸軍事,鎮山陽。義熙初,慕容超寇下邳,長民遣部將徐琰擊走之,進位使持節、督青揚二州諸軍事、青州刺史,領晉陵太守,鎮丹徒,本號及公如故。



 及何無忌為徐道覆所害,賊乘勝逼京師,朝廷震駭,長民率眾人衛京都,因表曰:「妖賊集船伐木,而南康相郭澄之隱蔽經年,又深相保明,屢欺無忌,罪合斬刑。」詔原
 澄之。及盧循之敗劉毅也,循與道覆連旗而下,京都危懼,長民勸劉裕權移天子過江。裕不聽,令長民與劉毅屯於北陵,以備石頭。事平,轉督豫州揚州之六郡諸軍事、豫州刺史,領淮南太守。



 及裕討毅,以長民監太尉留府事,詔以甲杖五十人入殿。長民驕縱貪侈,不恤政事,多聚珍寶美色,營建第宅,不知紀極,所在殘虐,為百姓所苦。自以多行無禮,恒懼國憲。及劉毅被誅,長民謂所親曰:「昔年醢彭越,前年殺韓信,禍其至矣!」謀欲為亂,問劉穆之曰:「人間論者謂太尉與我不平,其故何也?」穆之曰:「相公西征,老母弱弟委之將軍,何謂不平!」長民弟黎
 民輕狡好利,固勸之曰:「黥彭異體而勢不偏全,劉毅之誅,亦諸葛氏之懼,可因裕未還以圖之。」長民猶豫未發,既而歎曰:「貧賤常思富貴,富貴必履危機。今日欲為丹徒布衣,豈可得也!」裕深疑之,駱驛繼遣輜重兼行而下,前剋至日,百司於道候之,輒差其期。既而輕舟徑進,潛入東府。明旦,長民聞之,驚而至門,裕伏壯士丁旿於幕中,引長民進語,素所未盡皆說焉。長民悅,旿自後拉而殺之,輿尸付廷尉。使收黎民,黎民驍勇絕人,與捕者苦戰而死。小弟幼民為大司馬參軍,逃于山中,追擒戮之。諸葛氏之誅也,士庶咸恨正刑之晚,若釋桎梏焉。



 初,長
 民富貴之後,常一月中輒十數夜眠中驚起,跳踉,如與人相打。毛修之嘗與同宿,見之駭愕,問其故,長民答曰:「正見一物,甚黑而有毛,腳不分明,奇健,非我無以制之。」其後來轉數。屋中柱及椽桷間,悉見有蛇頭,令人以刀懸斫,應刃隱藏,去輒復出。又搗衣杵相與語如人聲,不可解。於壁見有巨手,長七八尺,臂大數圍,令斫之,豁然不見。未幾伏誅。



 何無忌,東海郯人也。少有大志,忠亮任氣,人有不稱其心者,輒形於言色。州辟從事,轉太學博士。鎮北將軍劉
 牢之,即其舅也,時鎮京口,每有大事,常與參議之。會稽世子元顯子彥章封東海王,以無忌為國中尉,加廣武將軍。及桓玄害彥章於市,無忌入市慟哭而出,時人義焉。隨牢之南征桓玄,牢之將降於玄也,無忌屢諫,辭旨甚切,牢之不從。及玄篡位,無忌與玄吏部郎曹靖之有舊,請蒞小縣。靖之白玄,玄不許,無忌乃還京口。



 初,劉裕嘗為劉牢之參軍,與無忌素相親結。至是,因密共圖玄。劉毅家在京口,與無忌素善,言及興復之事,無忌曰:「桓氏強盛,其可圖乎?」毅曰:「天下自有彊弱,雖彊易弱,正患事主難得耳!」無忌曰:「天下草澤之中非無英雄也。」毅曰:「
 所見唯有劉下邳。」無忌笑而不答,還以告裕,因共要毅,與相推結,遂共舉義兵,襲京口。無忌偽著傳詔服,稱敕使,城中無敢動者。



 初,桓玄聞裕等及無忌之起兵也,甚懼。其黨曰:「劉裕烏合之眾,勢必無成,願不以為慮。」玄曰:「劉裕勇冠三軍,當今無敵。劉毅家無儋石之儲,樗蒱一擲百萬。何無忌,劉牢之之甥,酷似其舅。共舉大事,何謂無成!」其見憚如此。及玄敗走,武陵王遵承制以無忌為輔國將軍、瑯邪內史,以會稽王道子所部精兵悉配之,南追桓玄,與振武將軍劉道規俱受冠軍將軍劉毅節度。玄留其龍驤將軍何澹之、前將軍郭銓、江州刺史郭
 昶之守湓口。無忌等次桑落洲,澹之等率軍來戰。澹之常所乘舫旌旗甚盛,無忌曰:「賊帥必不居此,欲詐我耳,宜亟攻之。」眾咸曰:「澹之不在其中,其徒得之無益。」無忌謂道規曰:「今眾寡不敵,戰無全勝。澹之雖不居此舫,取則易獲,因縱兵騰之,可以一鼓而敗也。」道規從之,遂獲賊舫,因傳呼曰:「已得何澹之矣!」賊中驚擾,無忌之眾亦謂為然。道規乘勝徑進,無忌又鼓噪赴之,澹之遂潰。進據尋陽,遣使奉送宗廟主祏及武康公主、瑯邪王妃還京都。又與毅、道規破走玄於崢嶸洲。無忌進據巴陵。玄從兄謙、從子振乘間陷江陵,無忌、道規進攻謙於馬頭,
 攻桓蔚於龍泉,皆破之。既而為桓振所敗,退還尋陽。無忌與毅、道規復進討振,剋夏口三城,遂平巴陵,進次馬頭。桓謙請割荊、江二州,奉送天子,無忌不許。進軍破江陵,謙等敗走。無忌侍衛安帝還京師,以無忌督豫州揚州淮南廬江安豐歷陽堂邑五郡軍事、右將軍、豫州刺史、加節,甲仗五十人入殿,未之職。遷會稽內史、督江東五郡軍事,持節、將軍如故,給鼓吹一部。義熙二年,遷都督江荊二州江夏隨義陽綏安豫州西陽新蔡汝南潁川八郡軍事、江州刺史,將軍、持節如故。以興復之功,封安成郡開國公,食邑三千戶,增督司州之弘農揚州之
 松滋,加散騎侍郎,進鎮南將軍。



 盧循遣別帥徐道覆順流而下,舟艦皆重樓。無忌將率眾距之,長史鄧潛之諫曰:「今以神武之師抗彼逆眾,迴山壓卵,未足為譬。然國家之計在此一舉。聞其舟艦大盛。勢居上流。蜂蠆之毒,邾魯成鑒。宜決破南塘,守二城以待之,其必不敢捨我遠下。蓄力俟其疲老,然後擊之。若棄萬全之長策,而決成敗於一戰,如其失利,悔無及矣。」無忌不從,遂以舟師距之。既及,賊令強弩數百登西岸小山以邀射之,而薄於山側。俄而西風暴急,無忌所乘小艦被飄東岸,賊乘風以大艦逼之,眾遂奔敗,無忌尚厲聲曰:「取我蘇武節
 來!」節至,乃躬執以督戰。賊眾雲集,登艦者數十人。無忌辭色無撓,遂握節死之。詔曰:「無忌秉哲履正,忠亮明允,亡身殉國,則契協英謨;經綸屯昧,則重氛載廓。及敷政方夏,實播風惠。妖寇構亂,侵擾邦畿,投袂致討,志清王略。而事出慮外,臨危彌厲,握節隕難,誠貫古賢,朕用傷慟於厥懷。其贈侍中、司空,本官如故,謚曰忠肅。」子邕嗣。



 初,桓玄剋京邑,劉裕東征,無忌密至裕軍所,潛謀舉義,勸裕於山陰起兵。裕以玄大逆未彰,恐在遠舉事,剋濟為難。若玄遂竊天位,然後於京口圖之,事未晚也。無忌乃還。及義師之舉,參贊大勳,皆以算略攻取為效,而此
 舉敗於輕脫,朝野痛之。



 檀憑之,字慶子,高平人也。少有志力。閨門邕肅,為世所稱。從兄子韶兄弟五人,皆稚弱而孤,憑之撫養若己所生。初為會稽王驃騎行參軍,轉桓修長流參軍,領東莞太守,加寧遠將軍。與劉裕有州閭之舊,又數同東討,情好甚密。義旗之建,憑之與劉毅俱以私艱,墨絰而赴。雖才望居毅之後,而官次及威聲過之,故裕以為建武將軍。裕將義舉也,嘗與何無忌、魏詠之同會憑之所。會善相者晉陵韋叟見憑之,大驚曰:「卿有急兵之厄,其候不
 過三四日耳。且深藏以避之,不可輕出。」及桓玄將皇甫敷之至羅落橋也,憑之與裕各領一隊而戰,軍敗,為敷軍所害。贈冀州刺史。義熙初,詔曰:「夫旌善紀功,有國之通典,沒而不朽,節義之篤行。故冀州刺史檀憑之忠烈果毅,亡身為國。既義敦其情,故臨危授命。考諸心迹,古人無以遠過,近者之贈,意猶恨焉。可加贈散騎常侍,本官如故。既隕身王事,亦宜追論封賞。可封曲阿縣公,邑三千戶。」



 魏詠之,字長道,任城人也。家世貧素,而躬耕為事,好學
 不倦。生而兔缺。有善相者謂之曰:「卿當富貴。」年十八,聞荊州刺史殷仲堪帳下有名醫能療之,貧無行裝,謂家人曰:「殘醜如此,用活何為!」遂齎數斛米西上,以投仲堪。既至,造門自通。仲堪與語,嘉其盛意,召醫視之。醫曰:「可割而補之,但須百日進粥,不得語笑。」詠之曰:「半生不語,而有半生,亦當療之,況百日邪!」仲堪於是處之別屋,令醫善療之。詠之遂閉口不語,唯食薄粥,其厲志如此。及差,仲堪厚資遣之。



 初為州主簿,嘗見桓玄。既出,玄鄙其精神不雋,謂坐客曰:「庸神而宅偉幹,不成令器。」竟不調而遣之。詠之早與劉裕游款,及玄篡位,協贊義謀。玄敗,
 授建威將軍、豫州刺史。桓歆寇歷陽,詠之率眾擊走之。義熙初,進征虜將軍、吳國內史,尋轉荊州刺史、持節、都督六州,領南蠻校尉。詠之初在布衣,不以貧賤為恥;及居顯位,亦不以富貴驕人。始為殷仲堪之客,未幾竟踐其位,論者稱之。尋卒於官。詔曰:「魏詠之器宇弘劭,識局貞隱,同獎之誠,實銘王府;敷績之效,垂惠在人。奄致隕喪,惻愴于心。可贈太常,加散騎常侍。」其後錄其贊義之功,追封江陵縣公,食邑二千五百戶,謚曰桓。弟順之至瑯邪內史。



 史臣曰:臣觀自古承平之化,必杖正人:非常之業,莫先
 奇士。當衰晉陵夷之際,逆玄僭擅之秋,外乏桓文,內無平勃,不有雄傑,安能濟之哉!此數子者,氣足以冠時,才足以經世,屬大亨數窮之運,乘義熙天啟之資,建大功若轉圜,翦群凶如拉朽,勢傾百辟,祿極萬鐘,斯亦丈夫之盛也。然希樂陵傲而速禍,諸葛驕淫以成釁,造宋而乖同德,復晉而異純臣,謀之不臧,自取夷滅。無忌挾功名之大志,挺文武之良才,追舊而慟感時人,率義而響震勍敵,因機效捷,處死不懦,比乎向時之輩,豈同日而言歟!



 贊曰:劉生剛愎,葛侯凶恣。患結滿盈,禍生疑貳。安成英
 武,體茲忠烈。舍家殉義,忘生存節。檀實棱威,身隕名飛。魏終協契,效績揚輝。



\end{pinyinscope}