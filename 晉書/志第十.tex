\article{志第十}

\begin{pinyinscope}

 禮
 中



 五禮之別,二曰兇。自天子至于庶人,身體髮膚,受之父母,其理既均,其情亦等,生則養,死則哀,故曰三年之喪,天下之達禮者也。漢禮,天子崩,自不豫至於登遐及葬,喪紀之制,與夫三代變易。魏晉以來,大體同漢。然自漢文革喪禮之制,後代遵之,無復三年之禮。及魏武臨終,遺令「天下尚未安定,未得遵古。百官當臨中者,十五舉
 音,葬畢便除。其將兵屯戍者,不得離部。」魏武以正月庚子崩,辛丑即殯,是月丁卯葬,是為不踰月也。



 及宣帝、景帝之崩,並從權制。文帝之崩,國內服三日。武帝亦遵漢魏之典,既葬除喪,然猶深衣素冠,降席撤膳。太宰司馬孚、太傅鄭沖、太保王祥、太尉何曾、司徒領中領軍司馬望、司空荀顗、車騎將軍賈充、尚書令裴秀、尚書僕射武陔、都護大將軍郭建、侍中郭綏、中書監荀勖、中軍將軍羊祜等奏曰:「臣聞禮典軌度,豐殺隨時,虞夏商周,咸不相襲,蓋有由也。大晉紹承漢魏,有革有因,期於足以興化而已,故未得皆返太素,同規上古也。陛下既以俯
 遵漢魏降喪之典,以濟時務,而躬蹈大孝,情過乎哀,素冠深衣,降席撤膳,雖武丁行之於殷世,曾閔履之於布衣,未足以踰。方今荊蠻未夷,庶政未乂,萬機事殷,動勞神慮,豈遑全遂聖旨,以從至情。臣等以為陛下宜割情以康時濟俗,輒敕御府易服,內省改坐,太官復膳,諸所施行,皆如舊制。」詔曰:「每感念幽冥,而不得終苴絰於草土,以存此痛,況當食稻衣錦,誠詭然激切其心,非所以相解也。吾本諸生家,傳禮來久,何心一旦便易此情於所天!相從已多,可試省孔子答宰我之言,無事紛紜也。言及悲剝,柰何!柰何!」孚等重奏:「伏讀聖詔,感以悲懷,輒思
 仲尼所以抑宰我之問,聖思所以不能已已,甚深甚篤。然今者干戈未戢,武事未偃,萬機至重,天下至眾。陛下以萬乘之尊,履布衣之禮,服粗席稿,水飲疏食,殷憂內盈,毀悴外表。而躬勤萬機,坐而待旦,降心接下,仄不遑食,所以勞力者如斯之甚。是以臣等悚息不寧,誠懼神氣用損,以疚大事。輒敕有司,改坐復常,率由舊典。惟陛下察納愚款,以慰皇太后之心。」又詔曰:「重覽奏議,益以悲剝,不能自勝,柰何!柰何!三年之喪,自古達禮,誠聖人稱情立衷,明恕而行也。神靈日遠,無所訴告,雖薄於情,食旨服美,所不堪也。不宜反覆,重傷其心,言用斷絕,
 柰何!柰何!」帝遂以此禮終三年。後居太后之喪亦如之。



 泰始二年八月,詔曰:「此上旬,先帝棄天下日也,便以周年。吾煢煢,當復何時一得敘人子之情邪!思慕煩毒,欲詔陵瞻侍,以盡哀憤。主者具行備。」太宰安平王孚、尚書令裴秀、尚書僕射武陔等奏:「陛下至孝蒸蒸,哀思罔極。衰麻雖除,哀毀疏食,有損神和。今雖秋節,尚有餘暑,謁見山陵,悲感摧傷,群下竊用竦息,以為宜降抑聖情,以慰萬國。」詔曰:「孤煢忽爾,日月已周,痛慕摧感,永無逮及。欲瞻奉山陵,以敘哀憤,體氣自佳耳。又已涼,便當行,不得如所奏也。主者便具行備。」又詔曰:「漢文不使天下盡哀,亦
 帝王至謙之志。當見山陵,何心而無服,其以衰絰行。」孚等重奏曰:「臣聞上古喪期無數,後世乃有年月之漸。漢文帝隨時之義,制為短喪,傳之于後。陛下以社稷宗廟之重,萬方億兆之故,既從權制,釋除衰麻,群臣百姓吉服,今者謁陵,以敘哀慕,若加衰絰,進退無當。不敢奉詔。」詔曰:「亦知不在此麻布耳。然人子情思,為欲令哀喪之物在身,蓋近情也。群臣自當案舊制。」孚等又奏曰:「臣聞聖人制作,必從時宜。故五帝殊樂,三王異禮,此古今所以不同,質文所以迭用也。陛下隨時之宜,既降心克己,俯就權制,既除衰麻,而行心喪之禮,今復制服,義無所依。
 若君服而臣不服,亦未之敢安也。參議宜如前奏。」詔曰:「患情不能跂及耳,衣服何在。諸君勤勤之至,豈茍相違。」



 泰始四年,皇太后崩。有司奏:「前代故事,倚廬中施白縑帳、蓐、素床,以布巾裹塊草,軺輦、版輿、細犢車皆施縑里。」詔不聽,但令以布衣車而已,其餘居喪之制,不改禮文。有司又奏:「大行皇太后當以四月二十五日安厝。故事,虞著衰服,既虞而除。其內外官僚皆就朝晡臨位,御除服訖,各還所次除衰服。」詔曰:「夫三年之喪,天下之達禮也。受終身之愛,而無數年之報,柰何葬而便即吉,情所不忍也。」有司又奏:「世有險易,道有洿隆,所遇之時異,誠有由
 然,非忽禮也。方今戎馬未散,王事至殷,更須聽斷,以熙庶績。昔周康王始登翌室,猶戴冕臨朝。降于漢魏,既葬除釋,諒闇之禮,自遠代而廢矣。惟陛下割高宗之制,從當時之宜。」詔曰:「夫三年之喪,所以盡情致禮,葬已便除,所不堪也。當敘吾哀懷,言用斷絕,柰何!柰何!」有司又固請。詔曰:「不能篤孝,勿以毀傷為憂也。誠知衣服末事耳,然今思存草土,率當以吉物奪之,迺所以重傷至心,非見念也。每代禮典質文皆不同耳,何為限以近制,使達喪闕然乎!」群臣又固請,帝流涕久之迺許。文明皇后崩及武元楊后崩,天下將吏發哀三日止。



 穆帝崩,哀帝立。帝於穆帝為從父昆弟,穆帝舅褚歆有表,中書答表朝廷無其儀,詔下議。尚書僕射江[A170]等四人並云,閔僖兄弟也,而為父子,則哀帝應為帝嗣。衛軍王述等二十五人云「成帝不私親愛,越授天倫,康帝受命顯宗。社稷之重,已移所授,纂承之序,宜繼康皇。」尚書謝奉等六人云:「繼體之正,宜本天屬,考之人情,宜繼顯宗也。」詔從述等議,上繼顯宗。



 寧康二年七月,簡文帝崩再周而遇閏。博士謝攸、孔粲議:「魯襄二十八年十二月乙未,楚子卒,實閏月而言十二月者,附正於前月也。喪事先遠,則應用博士吳商之
 言,以閏月祥。」尚書僕射謝安、中領軍王劭、散騎常侍鄭襲、右衛將軍殷康、驍騎將軍袁宏、散騎侍郎殷茂、中書郎車胤、左丞劉遵、吏部郎劉耽意皆同。康曰:「過七月而未及八月,豈可謂之踰期。必所不了,則當從其重者。」宏曰:「假值閏十二月而不取者,此則歲未終,固不可得矣。《漢書》以閏為後九月,明其同體也。」襲曰:「中宗、肅祖皆以閏月崩,祥除之變皆用閏之後月。先朝尚用閏之後月,今閏附七月,取之何疑,亦合遠日申情之言。又閏是後七而非八也,豈踰月之嫌乎!」尚書令王彪之、侍中王混、中丞譙王恬、右丞戴謐等議異,彪之曰:「吳商中才小官,
 非名賢碩儒、公輔重臣、為時所準則者。又取閏無證據,直攬遠日之義,越祥忌,限外取,不合卜遠之理。又丞相桓公嘗論云,《禮》二十五月大祥。何緣越期取閏,乃二十六月乎?」於是啟曰:「或以閏附七月,宜用閏月除者。或以閏名雖除七月,而實以三旬別為一月,故應以七月除者。臣等與中軍將軍沖參詳,一代大禮,宜準經典。三年之喪,十三月而練,二十五月而畢,《禮》之明文也。《陽秋》之義,閏在年內,則略而不數。明閏在年外,則不應取之以越期忌之重,禮制祥除必正期月故也。」己酉晦,帝除縞即吉。徐廣論曰:「凡辨義詳理,無顯據明文可以折中奪
 易,則非疑如何。禮疑從重,喪易寧戚,順情通物,固有成言矣。彪之不能徵援正義,有以相屈,但以名位格人,君子虛受,心無適莫,豈其然哉!執政從而行之,其殆過矣。」



 魏武以正月崩,魏文以其年七月設妓樂百戲,是則魏不以喪廢樂也。武帝以來,國有大喪,輒廢樂終三年。惠帝太安元年,太子喪未除,及元會亦廢樂。穆帝永和中,為中原山陵未脩復,頻年元會廢樂。是時太后臨朝,后父褚裒薨,元會又廢樂也。孝武太元六年,為皇后王氏喪,亦廢樂。孝武崩,太傅錄尚書會稽王道子議:「山陵之後,通婚嫁不得作樂,以一期為斷。」



 漢儀,太皇太后、皇太后崩,長樂太僕、少府大長秋典喪事,三公奉制度,他皆如禮。魏晉亦同天子之儀。



 泰始十年,武元楊皇后崩,及將遷于峻陽陵,依舊制,既葬,帝及群臣除喪即吉。先是,尚書祠部奏從博士張靖議,皇太子亦從制俱釋服。博士陳逵議,以為「今制所依,蓋漢帝權制,興於有事,非禮之正。皇太子無有國事,自宜終服。」有詔更詳議。尚書杜預以為:「古者天子諸侯三年之喪始同齊斬,既葬除喪服,諒闇以居,心喪終制,不與士庶同禮。漢氏承秦,率天下為天子脩服三年。漢文帝見其下不可久行,而不知古制,更以意制祥禫,除喪即
 吉。魏氏直以訖葬為節,嗣君皆不復諒闇終制。學者非之久矣,然竟不推究經傳,考其行事,專謂王者三年之喪,當以衰麻終二十五月。嗣君茍若此,則天子群臣皆不得除喪。雖志在居篤,更逼而不行。至今世主皆從漢文輕典,由處制者非制也。今皇太子與尊同體,宜復古典,卒哭除衰麻,以諒闇終制。於義既不應不除,又無取於漢文,乃所以篤喪禮也。」於是尚書僕射盧飲、尚書魏舒問杜預證據所依。預云:「傳稱三年之喪自天子達,此謂天子絕期,唯有三年喪也。非謂居喪衰服三年,與士庶同也。故后、世子之喪,而叔嚮稱有三年之喪二也。周
 公不言高宗服喪三年,而云諒闇三年,此釋服心喪之文也。叔嚮不譏景王除喪,而譏其燕樂已早,明既葬應除,而違諒闇之節也。《春秋》,晉侯享諸侯,子產相鄭伯,時簡公未葬,請免喪以聽命,君子謂之得禮。宰咺來歸惠公仲子之賵,傳曰『弔生不及哀』。此皆既葬除服諒闇之證,先儒舊說,往往亦見,學者來之思耳。《喪服》,諸侯為天子亦斬衰,豈可謂終服三年邪!上考七代,未知王者君臣上下衰麻三年者誰;下推將來,恐百世之主其理一也。非必不能,乃事勢不得,故知聖人不虛設不行之制。仲尼曰『禮所損益雖百世可知』,此之謂也。」於是飲、舒從
 之,遂命預造議,奏曰:



 侍中尚書令司空魯公臣賈充、侍中尚書僕射奉車都尉大梁侯臣盧欽、尚書新沓伯臣山濤、尚書奉車都尉平春侯臣胡威、尚書劇陽子臣魏舒、尚書堂陽子臣石鑒、尚書豐樂亭侯臣杜預稽首言:禮官參議博士張靖等議,以為「孝文權制三十六日之服,以日易月,道有污隆,禮不得全,皇太子亦宜割情除服」。博士陳逵等議,以為「三年之喪,人子所以自盡,故聖人制禮,自上達下。是以今制,將吏諸遭父母喪,皆假寧二十五月。敦崇孝道,所以風化天下。皇太子至孝著於內,而衰服除于外,非禮所謂稱情者也。宜其不除。」



 臣欽、臣舒、臣預謹案靖、逵等議,各見所學之一端,未曉帝者居喪古今之通禮也。自上及下,尊卑貴賤,物有其宜。故禮有以多為貴者,有以少為貴者,有以高為貴者,有以下為貴者,唯其稱也。不然,則本末不經,行之不遠。天子之與群臣,雖哀樂之情若一,而所居之宜實異,故禮不得同。《易》曰「上古之世喪期無數」,《虞書》稱「三載四海遏密八音」,其後無文。至周公旦,乃稱「殷之高宗諒闇三年不言」。其傳曰「諒,信也;闇,默也」。下逮五百餘歲,而子張疑之,以問仲尼。仲尼答云:「何必高宗,古之人皆然,君薨,百官總己以聽於冢宰三年。」周景王有后、世子之喪,既
 葬除喪而樂。晉叔向譏之曰:「三年之喪,雖貴遂服,禮也。王雖弗遂,宴樂已早,亦非禮也。」此皆天子喪事見於古文者也。稱高宗不云服喪三年,而云諒闇三年,此釋服心喪之文也。譏景王不譏其除喪,而譏其宴樂已早,明既葬應除,而違諒闇之節也。堯崩,舜諒闇三年,故稱遏密八音。由此言之,天子居喪,齊斬之制,菲杖絰帶,當遂其服。既葬而除,諒闇以終之,三年無改父之道,故百官總已聽於冢宰。喪服已除,故稱不言之美,明不復寢苫枕塊,以荒大政也。《禮記》:「三年之喪,自天子達。」又云:「父母之喪,無貴賤一也。」又云:「端衰喪車皆無等。」此通謂天子
 居喪,衣服之節同於凡人,心喪之禮終於三年,亦無服喪三年之文。然繼體之君,猶多荒寧。自從廢諒闇之制,至令高宗擅名於往代,子張致疑於當時,此乃賢聖所以為譏,非譏天子不以服終喪也。



 秦燔書籍,率意而行,亢上抑下。漢祖草創,因而不革。乃至率天下皆終重服,旦夕哀臨,經罹寒暑,禁塞嫁娶飲酒食肉,制不稱情。是以孝文遺詔,斂畢便葬,葬畢制紅禫之除。雖不合高宗諒闇之義,近於古典,故傳之後嗣。于時預脩陵廟,故斂葬得在浹辰之內,因以定制。近至明帝,存無陵寢,五旬乃葬,安在三十六日。此當時經學疏略,不師前聖之病
 也。魏氏革命,以既葬為節,合於古典,然不垂心諒闇,同譏前代。自泰始開元,陛下追尊諒闇之禮,慎終居篤,允臻古制,超絕於殷宗,天下歌德,誠非靖等所能原本也。



 天子諸侯之禮,當以具矣。諸侯惡其害己而削其籍,今其存者唯《士喪》一篇,戴聖之記雜錯其間,亦難以取正。天子之位至尊,萬機之政至大,群臣之眾至廣,不同之於凡人。故大行既葬,祔祭于廟,則因疏而除之。己不除則群臣莫敢除,故屈己以除之。而諒闇以終制,天下之人皆曰我王之仁也。屈己以從宜,皆曰我王之孝也。既除而心喪,我王猶若此之篤也。凡等臣子,亦焉得不自
 勉以崇禮。此乃聖制移風易俗之本,高宗所以致雍熙,豈惟衰裳而已哉!



 若如難者,更以權制自居,疑於屈伸厭降,欲以職事為斷,則父在為母期,父卒三年,此以至親屈於至尊之義也。出母之喪,以至親為屬,而長子不得有制,體尊之義,升降皆從,不敢獨也。《禮》:諸子之職,掌國子之倅。國有事則帥國子而致之太子,唯所用之。《傳》曰,「君行則守,有守則從,從曰撫軍,守曰監國」,不無事矣。《喪服》母為長子,妻為夫,妾為主,皆三年。內宮之主,可謂無事、揆度漢制,孝文之喪,紅禫既畢,孝景即吉於未央,薄后、竇后必不得齊斬於別宮,此可知也。況皇太子配
 貳至尊,與國為體,固宜遠遵古禮,近同時制,屈除以寬諸下,協一代之成典。



 君子之於禮,有直而行,曲而殺;有經而等,有順而去之,存諸內而已。禮云非玉帛之謂,喪云唯衰麻之謂乎?此既臣等所謂經制大義,且即實近言,亦有不安。今皇太子至孝蒸蒸,發於自然,號咷之慕,匍匐殯宮,大行既奠,往而不反,必想像平故,彳旁徨寢殿。若不變從諒闇,則東宮臣僕,義不釋服。此為永福官屬,當獨衰麻從事,出入殿省,亦難以繼。今將吏雖蒙同二十五月之寧,至於大臣,亦奪其制。昔翟方進自以身為漢相,居喪三十六日,不敢踰國典,而況於皇太子?
 臣等以為皇太子宜如前奏,除服諒闇終制。



 於是太子遂以厭降之議,從國制除衰麻,諒闇終制。



 于時外內卒聞預異議,多怪之。或者乃謂其違禮以合時。時預亦不自解說,退使博士段暢博採典籍,為之證據,令大義著明,足以垂示將來。暢承預旨,遂撰集書傳舊文,條諸實事成言,以為定證,以弘指趣。其傳記有與今議同者,亦具列之,博舉二隅,明其會歸,以證斯事。文多不載。



 武帝楊悼皇后既母養懷帝,后遇難時,懷帝尚幼,及即位,中詔述后恩愛。及后祖載,群官議帝應為追制服,或以庶母慈己,依禮制小功五月,或以謂慈母服如母服齊
 衰者,眾議不同。閭丘沖議云:「楊后母養聖上,蓋以曲情。今以恩禮追崇,不配世祖廟。王者無慈養之服,謂宜祖載之日,可三朝素服發哀而已。」於是從之。



 康帝建元元年正月晦,成恭杜皇后周忌,有司奏,至尊期年應改服。詔曰:「君親,名教之重也,權制出於近代耳。」於是素服如舊,固非漢魏之典也。



 興寧元年,哀帝章皇太妃薨,帝欲服重。江[A170]啟:「先王制禮,應在緦服。」詔欲降期,[A170]又啟:「厭屈私情,所以上嚴祖考。」於是制緦麻三月。



 孝武寧康中,崇德太后褚氏崩。后於帝為從嫂,或疑其
 服。博士徐藻議,以為:「資父事君而敬同。又,禮,其夫屬父道者,其妻皆母道也。則夫屬君道,妻亦后道矣。服后宜以資母之義。魯譏逆祀,以明尊尊。今上躬奉康、穆、哀皇及靖后之祀,致敬同於所天。豈可敬之以君道,而服廢於本親。謂應服齊衰期。」於是帝制期服。



 隆安四年,孝武太皇太后李氏崩,疑所服。尚書左僕射何澄、右僕射王雅、尚書車胤、孔安國、祠部郎徐廣議、太皇太后名位允正,體同皇極,理制備盡,情禮彌申。《陽秋》之義,母以子貴,既稱夫人,禮服從正。故成風顯夫人之號,文公服三年之喪。子於父之所生,體尊義重。且禮,祖
 不厭孫,固宜遂服無屈,而緣情立制。若嫌明文不存,則疑斯從重,謂應同於為祖母後齊衰期。永安皇后無服,但一舉哀,百官亦一期。」詔可。



 孝武帝太元十五年,淑媛陳氏卒,皇太子所生也。有司參詳母以子貴,贈淑媛為夫人,置家令典喪事。太子前衛率徐邈議:「《喪服傳》稱與尊者為體,則不服其私親。又,君父所不服,子亦不敢服。故王公妾子服其所生母練冠麻衣,既葬而除,非五服之常,則謂之無服。」從之。



 太元二十一年,孝武帝崩,孝武太后制三年之服。



 惠帝太安元年三月,皇太孫尚薨。有司奏,御服齊衰期。
 詔下通議。散騎常侍謝衡以為:「諸侯之太子,誓與未誓,尊卑體殊。《喪服》云為嫡子長殤,謂未誓也,已誓則不殤也。」中書令卞粹曰:「太子始生,故已尊重,不待命誓。若衡議已誓不殤,則無服之子當斬衰三年;未誓而殤,則雖十九當大功九月。誓與未誓,其為升降也微;斬衰與大功,其為輕重也遠。而今注云『諸侯不降嫡殤重』。嫌於無服,以大功為重嫡之服,則雖誓,無復有三年之理明矣。男能衛社稷,女能奉婦道,以可成之年而有已成之事,故可無殤,非孩齔之謂也。為殤後者尊之如父,猶無所加而止殤服,況以天子之尊,而為無服之殤行成人之制
 邪!凡諸宜重之殤,皆士大夫不加服,而令至尊獨居其重,未之前聞也。」博士蔡克同粹。秘書監摯虞云:「太子初生,舉以成人之禮,則殤理除矣。太孫亦體君傳重,由位成而服,全非以年也。天子無服殤之義,絕期故也。」於是從之。



 魏氏故事,國有大喪,群臣凶服,以帛為綬囊,以布為劍衣。新禮,以傳稱「去喪無所不佩」,明在喪則無佩也,更制齊斬之喪不佩劍綬。摯虞以為「《周禮》武賁氏,士大夫之職也,皆以兵守王宮,國有喪故,則衰葛執戈楯守門,葬則從車而哭。又,成王崩,太保命諸大夫以干戈內外警
 設。明喪故之際,蓋重宿衛之防。去喪無所不佩,謂服飾之事,不謂防禦之用。宜定新禮布衣劍如舊,其餘如新制。」詔叢之。



 漢魏故事,將葬,設吉凶鹵簿,皆以鼓吹。新禮以禮無吉駕導從之文,臣子不宜釋其衰麻以服玄黃,除吉駕鹵簿。又,凶事無樂,遏密八音,除凶服之鼓吹。摯虞以為:「葬有祥車曠左,則今之容車也。既葬,日中反虞,逆神而還。《春秋傳》,鄭大夫公孫蠆卒,天子追賜大路,使以行。《士喪禮》,葬有稿車乘車,以載生之服。此皆不唯載柩,兼有吉駕之明文也。既設吉駕,則宜有導從,以象平生之容,明不
 致死之義。臣子衰麻不得為身而釋,以為君父則無不可。《顧命》之篇足以明之。宜定新禮設吉服導從如舊,其凶服鼓吹宜除。」詔從之。



 漢魏故事,大喪及大臣之喪,執紼者挽歌。新禮以為挽歌出於漢武帝役人之勞歌,聲哀切,遂以為送終之禮。雖音曲摧愴,非經典所制,違禮設銜枚之義。方在號慕,不宜以歌為名。除,不挽歌。摯虞以為:「挽歌因倡和而為摧愴之聲,銜枚所以全哀,此亦以感眾。雖非經典所載,是歷代故事。《詩》稱『君子作歌,惟以告哀』,以歌為名,亦無所嫌。宜定新禮如舊。」詔從之。



 咸寧二年,安平穆王薨,無嗣,以母弟敦上繼獻王後,移太常問應何服。博士張靖答,宜依魯僖服閔三年例。尚書符詰靖:「穆王不臣敦,敦不繼穆,與閔僖不同。」孫毓、宋昌議,以穆王不之國,敦不仕諸侯,不應三年。以義處之,敦宜服本服,一期而除,主穆王喪祭三年畢,乃吉祭獻王。毓云:「《禮》,君之子孫所以臣諸兄者,以臨國故也。《禮》又與諸侯為兄弟服斬者,謂鄰國之臣於鄰國之君,有猶君之義故也。今穆王既不之國,不臣兄弟,敦不仕諸侯,無鄰臣之義,異於閔僖,如符旨也。但喪無主,敦既奉詔紹國,受重主喪,典其祭祀。『大功者主人之喪,有三年者
 則必為之再祭』。鄭氏《注》云,『謂死者之從父昆弟來為喪主也。」有三年者,謂妻若子幼少也』。『再祭,謂大小祥也』。穆妃及國臣於禮皆當三年,此為有三年者,敦當為之主大小兩祥祭也。且哀樂不相雜,吉凶不相干。凶服在宮,哭泣未絕。敦遽主穆王之喪,而國制未除,則不得以己本親服除而吉祭獻王也。」



 咸寧四年,陳留國上,燕公是王之父,王出奉明帝祀,今於王為從父,有司奏應服期,不以親疏尊卑為降。詔曰:「王奉魏氏,所承者重,不得服其私親。」穆帝時,東海國言,哀王薨踰年,嗣王乃來繼,不復追服,群臣皆已反吉,國
 妃亦宜同除。詔曰:「朝廷所以從權制者,以王事奪之,非為變禮也。婦人傳重義大,若從權制,義將安託!」於是國妃終三年之禮。孫盛以為:「廢三年之禮,開偷薄之源,漢魏失之大者也。今若以大夫宜奪以王事。婦人可終本服,是吉凶之儀雜陳於宮寢,彩素之制乖異於內外,無乃情禮俱違,哀樂失所乎!」



 太元十七年,太常車胤上言:「謹案《喪服禮經》,庶子為母緦麻三月。《傳》曰:『何以緦麻?以尊者為體,不敢服其私親也。』此《經》《傳》之明文,聖賢之格言。而自頃開國公侯,至于卿士,庶子為後,各肆私情,服其庶母,同之於嫡。此末俗之弊,溺情傷教,縱而不革,則流
 遁忘返矣。且夫尊尊親親,雖禮之大本,然厭親於尊,由來尚矣。《禮記》曰,『為父後,出母無服也者,不祭故也』。又,禮,天子父母之喪,未葬,越紼而祭天地社稷。斯皆崇嚴至敬,不敢以私廢尊也。今身承祖宗之重,而以庶母之私,廢烝嘗之事。五廟闕祀,由一妾之終,求之情禮,失莫大焉。舉世皆然,莫之裁貶。就心不同,而事不敢異。故正禮遂頹,而習非成俗。此《國風》所以思古,《小雅》所以悲歎。當今九服漸寧,王化惟新,誠宜崇明禮訓,以一風俗。請臺省考脩經典,式明王度。」不答。



 十八年,胤又上言:「去年上,自頃開國公侯,至于卿士,庶
 子為後者,服其庶母,同之於嫡,違禮犯制,宜加裁抑。事上經年,未被告報,未審朝議以何為疑。若以所陳或謬,則經有文;若以古今不同,則晉有成典。升平四年,故太宰武陵王所生母喪,表求齊衰三年,詔聽依昔樂安王故事,制大功九月。興寧三年,故梁王逢又所生母喪,亦求三年。《庚子詔書》依太宰故事,同服大功。若謹案周禮,則緦麻三月;若奉晉制,則大功九月。古禮今制,並無居廬三年之文,而頃年已來,各申私情,更相擬襲,漸以成俗。縱而不禁,則聖典滅矣。夫尊尊親親,立人之本,王化所由,二端而已。故先王設教,務弘其極,尊郊社之敬,制
 越紼之禮,嚴宗廟之祀,厭庶子之服,所以經緯人文,化成天下。夫屈家事於王道,厭私恩於祖宗,豈非上行乎下,父行乎子!若尊尊之心有時而替,宜厭之情觸事而申,祖宗之敬微,而君臣之禮虧矣。嚴恪微於祖宗,致敬虧於事上,而欲俗安化隆,不亦難乎!區區所惜,實在於斯。職之所司,不敢不言。請臺參詳。」尚書奏:「案如辭輒下主者詳尋。依禮,庶子與尊者為體,不敢服其私親,此尊祖敬宗之義。自頃陵遲,斯禮遂廢。封國之君廢五廟之重,士庶匹夫闕烝嘗之禮,習成頹俗,宜被革正。輒內外參詳,謂宜聽胤所上,可依樂安王大功為正。請為告書
 如左,班下內外,以定永制,普令依承,事可奉行。」詔可。



 《禮》,王為三公六卿錫衰,為大夫士疑衰,首服弁絰。天子諸侯皆為貴臣貴妾服三月。漢為大臣制服無聞焉。漢明帝時,東海恭王薨,帝出幸津門亭發哀。



 及武帝咸寧二年十一月,詔「諸王公大臣薨,應三朝發哀者,踰月不舉樂,其一朝發哀者,三日不舉樂也」。



 元帝姨廣昌鄉君喪,未葬,中丞熊遠表云:「案《禮》『君於卿大夫,比葬不食肉,比卒哭不舉樂』,惻隱之心未忍行吉事故也。被尚書符,冬至後二日小會。臣以為廣昌鄉君喪殯日,聖恩垂悼。禮,大夫死,廢一時之祭。祭猶可廢,而況餘事。冬至唯
 可群下奉賀而已,未便小會。」詔以遠表示賀循,又曰:「咸寧二年武皇帝故事云『王公大臣薨,三朝發哀,踰月不舉樂,其一朝發哀,三日不舉樂』,此舊事明文。」賀循答曰:「案《禮·雜記》,『君於卿大夫之喪,比葬不食肉,比卒哭不舉樂』。古者君臣義重,雖以至尊之義,降而無服,三月之內,猶錫衰以居,不接吉事。故春秋晉大夫智悼子未葬,平公作樂,為屠蒯所譏。如遠所答,合於古義。咸寧詔書雖不會經典,然隨時立宜,以為定制,誠非群下所得稱論。」升平元年,帝姑廬陵公主未葬,符問太常,冬至小會應作樂不。博士胡訥議云:「君於卿大夫,比卒哭不舉樂。公
 主有骨肉之親,宜闕樂。」太常王彪之云:「案武帝詔,三朝舉哀,三旬乃舉樂;其一朝舉哀者,三日則舉樂。泰始十年春,長樂長公主薨,太康七年秋,扶風王駿薨,武帝並舉哀三日而已。中興已後,更參論不改此制。今小會宜作樂。」二議竟不知所取。



 《喪服記》,公為所寓,齊衰三月。新禮以今無此事,除此一章。摯虞以為:「《周禮》作於刑厝之時,而著荒政十二。禮備制待物,不以時衰而除盛典,世隆而闕衰教也。曩者王司徒失守播越,自稱寄公。是時天下又多此比,皆禮之所及。宜定新禮自如舊經。」詔從之。



 漢魏故事無五等諸侯之制,公卿朝士服喪,親疏各如其親。新禮王公五等諸侯成國置卿者,及朝廷公孤之爵,皆傍親絕期,而旁親為之服斬衰,卿校位從大夫者皆絕緦。摯虞以為:「古者諸侯君臨其國,臣諸父兄,今之諸侯未同於古。未同于古,則其尊未全,不宜便從絕期之制,而令傍親服斬衰之重也。諸侯既然,則公孤之爵亦宜如舊。昔魏武帝建安中已曾表上,漢朝依古為制,事與古異,皆不施行,施行者著在魏科。大晉采以著令,宜定新禮皆如舊。」詔從之。



 《喪服》無弟子為師服之制,新禮弟子為師齊衰三月。摯
 虞以為:「自古無師服之制,故仲尼之喪,門人疑於所服。子貢曰:『昔夫子之喪顏回,若喪子而無服,請喪夫子若喪父而無服。』遂心喪三年。此則懷三年之哀,而無齊衰之制也。群居,入則絰,出則否,所謂弔服加麻也。先聖為禮,必易從而可傳。師徒義誠重,而服制不著,歷代相襲,不以為缺。且尋師者以彌高為得,故屢遷而不嫌;脩業者以日新為益,故舍舊而不疑。仲尼稱『三人行,必有我師焉』。子貢云,『夫何常師之有』。淺學之師,暫學之師,不可皆為之服。義有輕重,服有廢興,則臧否由之而起,是非因之而爭,愛惡相攻,悔吝生焉。宜定新禮無服如舊。」詔
 從之。



 古者天子諸侯葬禮粗備,漢世又多變革,魏晉以下世有改變,大體同漢之制。而魏武以禮送終之制,襲稱之數,繁而無益,俗又過之,豫自制送終衣服四篋,題識其上,春秋冬夏,日有不諱,隨時以斂。金珥珠玉銅鐵之物,一不得送。文帝遵奉,無所增加。及受禪,刻金璽,追加尊號,不敢開埏,乃為石室,藏璽埏首,以示陵中無金銀諸物也。漢禮明器甚多,自是皆省之矣。魏文帝黃初三年,又自作終制曰:「禮,國君即位為椑,存不忘亡也。壽陵因山為體,無封樹,無立寢殿,造園邑,通
 神道。夫葬者藏也,欲人之不得見也。禮不墓祭,欲存亡不黷也。皇后及貴人以下不隨王之國者,有終沒,皆葬澗西,前又已表其處矣。」此詔藏之宗廟,副在尚書、秘書、三府。明帝亦遵奉之。明帝性雖崇奢,然未遽營陵墓之制也。



 宣帝豫自於首陽山為土藏,不填不樹,作《顧命終制》,斂以時服,不設明器。景、文皆謹奉成命,無所加焉。景帝崩,喪事制度又依宣帝故事。武帝泰始四年,文明王皇后崩,將合葬,開崇陽陵,使太尉司馬望奉祭,進皇帝密璽綬於便房神坐。魏氏金璽,
 此又儉矣。江左初,元、明崇儉,且百度草創,山陵奉終,省約備矣。成帝咸康七年,皇后杜氏崩。詔外官五日一入臨,內官旦一入而已,過葬虞祭禮畢止。有司奏,大行皇后陵所作凶門柏歷門,號顯陽端門。詔曰:「門如所處。凶門柏歷,大為煩費,停之。」案蔡謨說,以二瓦器盛始死之祭,繫於木,裹以葦席,置庭中,近南,名為重,今之凶門是其象也。禮,既虞而作主,今未葬,未有主,故以重當之。禮稱為主道,此其義也。范堅又曰:「凶門非禮,禮有懸重,形似凶門。後人出之門外以表喪,俗遂行之。薄帳,即古弔幕之類
 也。」是時,又詔曰:「重壤之下,豈宜崇飾無用,陵中唯潔掃而已。」有司又奏,依舊選公卿以下六品子弟六十人為挽郎,詔又停之。孝武帝太元四年九月,皇后王氏崩。詔曰:「終事唯從儉速。」又詔:「遠近不得遣山陵使。」有司奏選挽郎二十四人,詔停之。



 古無墓祭之禮。漢承秦,皆有園寢。正月上丁,祠南郊禮畢,次北郊、明堂、高廟、世祖廟,謂之五供。



 魏武葬高陵,有司依漢立陵上祭殿。至文帝黃初三年,乃詔曰:「先帝躬履節儉,遺詔省約。子以述父為孝,臣以
 系事為忠。古不墓祭,皆設於廟。高陵上殿皆毀壞,車馬還廄,衣服藏府,以從先帝儉德之志。」文帝自作終制,又曰「壽陵無立寢殿,造園邑」,自後園邑寢殿遂絕。齊王在位九年,始一謁高平陵而曹爽誅,其後遂廢,終於魏世。



 及宣帝,遺詔「子弟群官皆不得謁陵」。於是景、文遵旨。至武帝,猶再謁崇陽陵,一謁峻平陵,然遂不敢謁高原陵,至惠帝復止也。



 逮于江左,元帝崩後,諸公始有謁陵辭告之事。蓋由眷同友執,率情而舉,非洛京之舊也。成帝時,中宮亦年年拜陵,議者以為非禮,於是遂止,以為永制。至穆帝時,褚太后臨朝,又拜陵,帝幼故也。至孝武崩,
 驃騎將軍司馬道子曰:「今雖權制釋服,至於朔望諸節,自應展情陵所,以一周為斷。」於是至陵,變服單衣,煩黷無準,非禮意也。及安帝元興元年,尚書左僕射桓謙奏:「百僚拜陵,起於中興,非晉舊典,積習生常,遂為近法。尋武皇帝詔,乃不使人主諸王拜陵,豈唯百僚!謂宜遵奉。」於是施行。及義熙初,又復江左之舊。



 太康七年,大鴻臚鄭默母喪,既葬,當依舊攝職,固陳不起,於是始制大臣得終喪三年。然元康中,陳準、傅咸之徒,猶以權奪,不得終禮,自茲已往,以為成比也。



 太康元年,東平王楙上言,相王昌父毖,本居長沙,有妻
 息,漢末使入中國,值吳叛,仕魏為黃門郎,與前妻息死生隔絕,更娶昌母。今江表一統,昌聞前母久喪,言疾求平議。



 守博士謝衡議曰:「雖有二妻,蓋有故而然,不為害於道,議宜更相為服。」守博士許猛以為「地絕,又無前母之制,正以在前非沒則絕故也。前母雖在,猶不應服。」段暢、秦秀、騶沖從猛。散騎常侍劉智安議:「禮為常事制,不為非常設也。亡父母不知其死生者,不著於禮。平生不相見,去其加隆,以期為斷。」都令史虞溥議曰:「臣以為禮不二嫡,所以重正,非徒如前議者防妒忌而已。故曰『一與之齊,終身不改』,未有遭變而二嫡。茍不二,則昌
 父更娶之辰,是前妻義絕之日也。使昌父尚存,二妻俱在,必不使二嫡專堂,兩婦執祭,同為之齊也。」秦秀議:「二妾之子,父命令相慈養,而便有三年之恩,便同所生。昌父何義不命二嫡依此禮乎!父之執友有如子之禮,況事兄之母乎!」許猛又議:「夫少婦稚,則不可許以改娶更適矣。今妻在許以更聘,夫存而妻得改醮者,非絕而何。」侍中領博士張惲議:「昔舜不告而娶,婚禮蓋闕,故《堯典》以釐降二女為文,不殊嫡媵。傳記以妃夫人稱之,明不立正后也。夫以聖人之弘,帝者嫡子,猶權事而變,以定典禮。黃昌之告新妻使避正室,時論許之。推姬氏之讓,執
 黃卿之決,宜使各自服其母。」黃門侍郎崔諒、荀悝、中書監荀勖、領中書令和嶠、侍郎夏侯湛皆如溥議。侍郎山雄、兼侍郎著作陳壽以為:「溥駁一與之齊,非大夫也,禮無二嫡,不可以並耳。若昌父及二母於今各存者,則前母不廢,已有明徵也。設令昌父將前母之子來入中國尚在者,當從出母之服。茍昌父無棄前妻之命,昌兄有服母之理,則昌無疑於不服。」賊曹屬卞粹議:「昌父當莫審之時而娶後妻,則前妻同之於死而義不絕。若生相及而後妻不去,則妾列於前志矣。死而會乎,則同祔於葬,無並嫡之實。必欲使子孫於沒世之後,追計二母
 隔絕之時,以為並嫡,則背違死父,追出亡母。議者以為禮無前母之服者,可謂以文害意。愚以為母之不親而服三年,非一無異於前母也。倉曹屬衛恒議:「或云,嫡不可二,前妻宜絕。此為奪舊與新,違母從子,禮律所不許,人情所未安也。或云,絕與死同,無嫌二嫡,據其相及,欲令有服。此為論嫡則死,議服則生,還自相伐,理又不通。愚以為地絕死絕,誠無異也,宜一如前母,不復追服。」主薄劉卞議:「毖在南為邦族,於北為羈旅,以此名分言之,前妻為元妃,後婦為繼室。何至王路既通,更當逐其今妻,廢其嫡子!不書姜氏,絕不為親,以其犯至惡也。趙姬雖
 貴,必推叔隗;原同雖寵,必嫡宣孟。若違禮茍讓,何則《春秋》所當善也!論者謂地絕,其情終已不得往來。今地既通,何為故當追而絕之邪!黃昌見美,斯又近世之明比。」司空齊王攸議:「《禮記》『生不及祖父母、諸父昆弟,而父稅喪,己則否』,諸儒皆以為父以他故子生異域,不及此親存時歸見之,父雖追服,子不從稅,不責非時之恩也。但不相見,尚不服其先終,而況前母非親所生,義不踰祖,莫往莫來,恩絕殊隔,而令追服,殆非稱情立文之謂也。以為昌不宜追服。」司徒李胤議:「毖為黃門侍郎,江南已叛。石厚與焉,大義滅親,況於毖之義,可得以為妻乎!」大
 司馬騫不議,太尉充、撫軍大將軍妝南王亮皆從主者。溥又駁粹曰:「喪從寧戚,謂喪事尚哀耳,不使服非其親也。夫死者終也,終事已故無絕道。分居兩存,則離否由人。夫婦以判合為義,今土隔人殊,則配合理絕。彼已更娶代己,安得自同於死婦哉!伯夷讓孤竹,不可以為後王法也。且既已為嫡後服,復云為妾,生則或貶或離,死則同祔於葬,妻專一以事夫,夫懷貳以接己,開偽薄之風,傷貞信之教,於以純化篤俗,不亦難乎!今昌二母雖土地殊隔,據同時並存,何得為前母後母乎!設使昌母先亡,以嫡合葬,而前母不絕,遠聞喪問,當復相為制何
 服邪!夫制不應禮,動而愈失。夫孝子不納親於不義,貞婦不昧進而茍容。今同前嫡於死婦,使後妻居正而或廢,於二子之心,曾無恧乎!而云誣父棄母,恐此文致之言,難以定臧否也。禮,違諸侯適天子,不服舊君,然則昌父絕前君矣,更納後室,廢舊妻矣,又何取於宜誅宜撫乎!且婦人之有惡疾,乃慈夫之所愍也,而在七出,誠以在人理應絕故也。今夫婦殊域,與無妻同,方之惡疾,理無以異。據己更娶,有絕前之證。而云應服,於義何居!」尚書八座以為「設令有人於此,父為敦煌太守,而子後任於洛,若父娶妻,非徒不見,乃可不知,及其死亡,不得不服。
 但鞠養已者情哀,而不相見名制,雖戚念之心殊,而為之服一也。又,兩后匹嫡,自謂違禮,不謂非常之事而以常禮處之也。昔子思哭出母於廟,其門人曰:『庶氏之女死,何為哭於孔氏之廟!』子思懼,改哭於他室。若昌不制服,不得不告其父祖,掘其前母之尸,徙之他地。若其不徙,昌為罪人。何則?異族之女不得祔于先姑,藏其墓次故也。且夫婦人牽夫,猶有所尊,趙姬之舉,禮得權通,故先史詳之,不譏其事耳。今昌之二母,各已終亡,尚無並主輕重之事也。昌之前母,宜依叔隗為比。若亡在昌未生之前者,則昌不應復服。生及母存,自應如禮以名服
 三年。輒正定為文,章下太常報楙奉行。」



 制曰:「凡事有非常,當依準舊典,為之立斷。今議此事,稱引趙姬、叔隗者粗是也。然後狄與晉和,故姬氏得迎叔隗而下之。吳寇隔塞,毖與前妻,終始永絕。必義無兩嫡,則趙衰可以專制隗氏。昌為人子,豈得擅替其母。且毖二妻並以絕亡,其子猶後母之子耳,昌故不應制服也。」



 太興初,著作郎干寶論之曰:「禮有經有變有權,王毖之事,有為為之也。有不可責以始終之義,不可求以循常之文,何群議之紛錯!同產者無嫡側之別,而先生為兄;諸侯同爵無等級之差,而先封為長。今二妻之入,無貴賤之禮,則宜
 以先後為秩,順序義也。今生而同室者寡,死而同廟者眾,及其神位,固有上下也。故《春秋》賢趙姬遭禮之變而得禮情也。且夫吉凶哀樂,動乎情者也,五禮之制,所以敘情而即事也。今二母者,本他人也,以名來親,而恩否於時,敬不及生,愛不及喪,夫何追服之道哉!張惲、劉卞,得其先後之節,齊王、衛恒,通於服絕之制,可以斷矣。朝廷於此,宜導之以趙姬,齊之以詔命,使先妻恢含容之德,後妻崇卑讓之道,室人達長少之序,百姓見變禮之中。若此,可以居生,又況於死乎!古之王者,有以師友之禮待其臣,而臣不敢自尊。今令先妻以一體接後,而後妻不
 敢抗,及其子孫交相為服,禮之善物也。然則王昌兄弟相得之日,蓋宜祫祭二母,等其禮饋,序其先後,配以左右,兄弟肅雍,交酬奏獻,上以恕先父之志,中以高二母之德,下以齊兄弟之好,使義風弘于王教,慈讓洽乎急難,不亦得禮之本乎!」



 是時,沛國劉仲武先娶毌丘氏,生子正舒、正則二人。毌丘儉反敗,仲武出其妻,娶王氏,生陶,仲武為毌丘氏別舍而不告絕。及毌丘氏卒,正舒求祔葬焉,而陶不許。舒不釋服,訟於上下,泣血露骨,縗裳綴絡,數十年弗得從,以至死亡。



 時吳國朱某娶妻陳氏,生子東伯。入晉,晉賜妻某氏,生子綏伯。太康之中,某已
 亡,綏伯將母以歸邦族,兄弟交愛敬之道,二母篤先後之序,雍雍人無間焉。及其終也,二子交相為服,君子以為賢。



 安豐太守程諒先已有妻,後又娶,遂立二嫡。前妻亡,後妻子勳疑所服。中書令張華造甲乙之問曰:「甲娶乙為妻,後又娶丙,匿不說有乙,居家如二嫡,無有貴賤之差。乙亡,丙之子當何服?本實並列,嫡庶不殊,雖二嫡非正,此失在先人,人子何得專制析其親也。若為庶母服,又不成為庶。進退不知所從。」太傅鄭沖議曰:「甲失禮於家,二嫡並在,誠非人子所得正。則乙丙之子並當三年,禮疑從重。」車騎賈充、侍中少傳任愷議略與鄭同。太
 尉荀顗議曰:「《春秋》並后匹嫡,古之明典也。今不可以犯禮並立二妻,不別尊卑而遂其失也。故當斷之以禮,先至為嫡,後至為庶。丙子宜以嫡母服乙,乙子宜以庶母事丙。昔屈建去芰,古人以為違禮而得禮。丙子非為抑其親,斯自奉禮先後貴賤順敘之義也。」中書監荀勖議曰:「昔鄉里鄭子群娶陳司空從妹,後隔呂布之亂,不復相知存亡,更娶鄉里蔡氏女。徐州平定,陳氏得還,遂二妃並存。蔡氏之子字元釁,為陳氏服嫡母之服,事陳公以從舅之禮。族兄宗伯曾責元釁,謂抑其親,鄉里先達以元釁為合宜。不審此事粗相似否。」



 建武元年,以溫嶠為散騎侍郎,嶠以母亡值寇,不臨殯葬,欲營改葬,固讓不拜。元帝詔曰:「溫嶠不拜,以未得改卜葬送,朝議又頗有異同。為審由此邪?天下有闕塞,行禮制物者當使理可經通。古人之制三年,非情之所盡,蓋存亡有斷,不以死傷生耳。要絰而服金革之役者,豈營官邪?隨王事之緩急也。今桀逆未梟,平陽道斷,奉迎諸軍猶未得徑進,嶠特一身,於何濟其私艱,而以理閡自疑,不服王命邪!其令三司八座、門下三省、外內群臣,詳共通議如嶠比,吾將親裁其中。」於是太宰、西陽王羕,司徒臨潁公組,驃騎將軍、即丘子導,侍中紀瞻,尚書周
 顗,散騎常侍荀邃等議,以「昔伍員挾弓去楚,為吳行人以謀楚,誠志在報仇,不茍滅身也。溫嶠遭難,昔在河朔,日尋干戈,志刷讎惡,萬里投身,歸赴朝廷,將欲因時竭力,憑賴王威,以展其情,此乃嶠之志也。無緣道路未通,師旅未進,而更中辭王事,留志家巷也。以為誠宜如明詔。」於是有司奏曰:「案如眾議,去建武元年九月下辛未令書,依禮文,父喪未葬,唯喪主不除。以他故未葬,人子之情,不可居殯而除,故期於畢葬,無遠近之斷也。若亡遇賊難,喪靈無處,求索理絕,固應三年而除,不得故從未葬之例也。若骨肉殲於寇害,死亡漫於中原,而繼以
 遺賊未滅,亡者無收殯之實,存者又闕於奔赴之禮,而人子之情,哀痛無斷,輒依未葬之義,久而不除,若遂其情,則人居無限之喪,非有禮無時不得之義也。諸如此,皆依東關故事,限行三年之禮畢而除也。唯二親生離,吉凶未分,服喪則凶事未據,從吉則疑於不存,心憂居素,出自人情,有如此者,非官制之所裁。今嶠以未得改卜奔赴,累設疾辭。案辛未之制,已有成斷,皆不得復遂其私情,不服王命,以虧法憲。參議可如前詔嶠受拜,重告以中丞司徒,諸如嶠比者,依東關故事辛未令書之制。」嶠不得已,乃拜。



 是時中原喪亂,室家離析,朝廷議二
 親陷沒寇難,應制服不。太常賀循曰:二親生離,吉凶未分,服喪則凶事未據,從吉則疑於不存,心憂居素,允當人情。」元帝令以循議為然。太興二年,司徒荀組云:「二親陷沒寇難,萬無一冀者,宜使依王法,隨例行喪。」庾蔚之云:「二親為戎狄所破,存亡未可知者,宜盡尋求之理。尋求之理絕,三年之外,便宜婚宦,胤嗣不可絕,王政不可廢故也。猶宜以哀素自居,不豫吉慶之事,待中壽而服之也。若境內賊亂清平,肆眚之後,尋覺無蹤跡者,便宜制服。」



 咸康二年,零陵李繁姊先適南平郡陳詵為妻,產四子
 而遭賊。姊投身於賊,請活姑命,賊略將姊去。詵更娶嚴氏,生三子。繁後得姊消息,往迎還詵,詵籍注領二妻。及李亡,詵疑制服,以事言征西大將軍庾亮府平議,時議亦往往異同。司馬王愆期議曰:「案禮不二嫡,故惠公元妃孟子,孟子卒,繼室以聲子。諸侯猶爾,況庶人乎!《士喪禮》曰,繼母本實繼室,故稱繼母,事之如嫡,故曰如母也。詵不能遠慮避難,以亡其妻,非犯七出見絕於詵。始不見絕,終又見迎,養姑於堂,子為首嫡,列名黃籍,則詵之妻也。為詵也妻,則為暉也母,暉之制服無所疑矣。禮為繼母服而不為前母服者,如李比類,曠世所希。前母既終,乃有繼
 母,後子不及前母,故無制服之文。然礿祠蒸嘗,未有不以前母為母者,亡猶母之,況其存乎!詵有老母,不可以莫之養,妻無歸期,納妾可也。李雖沒賊,尚有生冀,詵尋求之理不盡,而便娶妻,誠詵之短也。然隴畝之夫,不達禮義,考之傳記不勝。有施孝叔之妻失身於郤犨而不棄者,以非其罪也。詵有兩妻,非故犯法。李鄙野人,而能臨危請活姑命,險不忘順,可謂孝婦矣。議者欲令在沒略之中,必全苦操,有隕無二,是望凡人皆為宋伯姬也。詵雖不應娶妻,耍以嚴為妻,妻則繼室,本非嫡也。雖云非嫡,義在始終,寧可以詵不應二妻而己涉二庭乎!若能
 下之,則趙姬之義。若云不能,官當有制。先嫡後繼,有自來矣。眾議貶譏太峻,故略序異懷。」亮從愆期議定。



 《五經通義》以為有德則謚善,無德則謚惡,故雖君臣可同。魏朝初謚宣帝為文侯,景王為武侯,文王表不宜與二祖同,於是改謚宣文、忠武。至文王受晉王之號,魏帝又追命宣文為宣王,忠武為景王。太康八年十月,太常上謚故太常平陵男郭奕為景侯。有司奏云:「晉受命以來,祖宗號謚群下未有同者,故郭奕為景,與景皇同,不可聽,宜謚曰穆。」王濟、羊璞等並云:「夫無窮之祚,名謚不一,若皆相避,於制難全。如悉不避,
 復非推崇事尊之禮。宜依諱名之義,但及七廟祖宗而已,不及遷毀之廟。」成粲、武茂、劉訥並云:「同謚非嫌。號謚者,國之大典,所以厲時作教,經天人之遠旨也。固雖君父,義有所不隆,及在臣子,或以行顯。故能使上下邁德,罔有怠荒。臣願聖世同符堯舜,行周同謚之禮,舍漢魏近制相避之議。」又引周公父子同謚曰文。武帝詔曰:「非言君臣不可同,正以奕謚景不相當耳,宜謚曰簡。」及太元四年,侍中王欣之表君臣之嫌同謚,尚書奏以欣之言為然。詔可。



 驃騎將軍溫嶠前妻李氏,在嶠微時便卒。又娶王氏、何
 氏,並在嶠前死。及嶠薨,朝廷以問陳舒:「三人並得為夫人不?」舒云:「《禮記》『其妻為夫人而卒,而後其夫不為大夫,而祔於其妻,則不易牲。妻卒。而後夫為大夫,而祔於其妻,則以大夫牲』。然則夫榮於朝,妻貴於室,雖先夫沒,榮辱常隨於夫也。《禮記》曰『妻祔於祖姑,祖姑有三人,則祔其親者』。如禮,則三人皆為夫人也。自秦漢已來,廢一娶九女之制,近世無復繼室之禮,先妻卒則更娶。茍生加禮,則亡不應貶。」庾蔚之云:「賤時之妻不得並為夫人,若有追贈之命則不論耳。」《嶠傳》,贈王、何二人夫人印綬,不及李氏。



 永和十一年,彭城國為李太妃求謚。博士曹耽之議:「夫婦行不必同,不得以夫謚謚婦。《春秋》婦人有謚甚多,經無譏文,知禮得謚也。」胡訥云:「禮,婦人生以夫爵,死以夫謚。《春秋》夫人有謚,不復依禮耳。安平獻王李妃、琅邪武王諸葛妃,太傅東海王裴妃並無謚,今宜率舊典。」王彪之云:「婦人有謚,禮壞故耳。聲子為謚,服虔諸儒以為非。杜預亦云『禮,婦人無謚』。《春秋》無譏之文,所謂不待貶絕自明者也。近世惟后乃有謚耳。」



 太尉荀顗上謚法云:「若賜謚而道遠不及葬者,皆封策下屬,遣所承長吏奉策即冢祭賜謚。」



 太元十三年,召孔安國為侍中。安國表以黃門郎王愉名犯私諱,不得連署,求解。有司議云:「名終諱之,有心所同,聞名心瞿,亦明前誥。而《禮》復云『君所無私諱,大夫之所有公諱』,無私諱。又云『詩書不諱,臨文不諱』。豈非公義奪私情,王制屈家禮哉!尚書安眾男臣先表中兵曹郎王祐名犯父諱,求解職,明詔爰發,聽許換曹,蓋是恩出制外耳。而頃者互相瞻式,源流既啟,莫知其極。夫皇朝禮大,百僚備職,編官列署,動相經涉。若以私諱,人遂其心,則移官易職,遷流莫已,既違典法,有虧政體。請一斷之。」從之。



\end{pinyinscope}