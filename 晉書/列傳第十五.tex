\article{列傳第十五}

\begin{pinyinscope}
劉毅
 \gezhu{
  子暾}
 程衛和嶠武陔任愷崔洪郭奕侯史光何攀



 劉毅,字仲雄,東萊掖人。漢城陽景王章之後。父喈,丞相屬。毅幼有孝行,少厲清節,然好臧否人物,王公貴人望風憚之。僑居平陽,太守杜恕請為功曹,沙汰郡吏百餘人,三魏稱焉。為之語曰:「但聞劉功曹,不聞杜府君。」魏末,本郡察孝廉,辟司隸都官從事,京邑肅然。毅將彈河南尹,司隸不許,曰:「攫獸之犬,鼷鼠蹈其背。」毅曰:「既能攫獸,
 又能殺鼠,何損於犬!」投傳而去。同郡王基薦毅於公府,曰:「毅方正亮直,介然不群,言不茍合,行不茍容。往日僑仕平陽,為郡股肱,正色立朝,舉綱引墨,朱紫有分,《鄭》、《衛》不雜,孝弟著於邦族,忠貞效於三魏。昔孫陽取騏驥於吳阪,秦穆拔百里於商旅。毅未遇知己,無所自呈。前已口白,謹復申請。」太常鄭袤舉博士,文帝辟為相國掾,辭疾,積年不就。時人謂毅忠於魏氏,而帝怒其顧望,將加重辟。毅懼,應命,轉主薄。



 武帝受禪,為尚書郎、駙馬都尉,遷散騎常侍、國子祭酒。帝以毅忠蹇正直,使掌諫官。轉城門校尉,遷太僕,拜尚書,坐事免官。咸寧初,復為散騎
 常侍、博士祭酒。轉司隸校尉,糾正豪右,京師肅然。司部守令望風投印綬者甚眾,時人以毅方之諸葛豐、蓋寬饒。皇太子朝,鼓吹將入東掖門,毅以為不敬,止之於門外,奏劾保傅以下。詔赦之,然後得入。



 帝嘗南郊,禮畢,喟然問毅曰:「卿以朕方漢何帝也?」對曰:「可方桓、靈。」帝曰:「吾雖德不及古人,猶克己為政。又平吳會,混一天下。方之桓、靈,其已甚乎!」對曰:「桓、靈賣官,錢入官庫;陛下賣官,錢入私門。以此言之,殆不如也。」帝大笑曰:「桓靈之世,不聞此言。今有直臣,故不同也。」散騎常侍鄒湛進曰:「世談以陛下比漢文帝,人心猶不多同。昔馮唐答文帝,云不能用
 頗牧而文帝怒,今劉毅言犯順而陛下歡。然以此相校,聖德乃過之矣。」帝曰:「我平天下而不封禪,焚雉頭裘,行布衣禮,卿初無言。今於小事,何見褒之甚?」湛曰:「臣聞猛獸在田,荷戈而出,凡人能之。蜂蠆作於懷袖,勇夫為之驚駭,出於意外故也。夫君臣有自然之尊卑,言語有自然之逆順。向劉毅始言,臣等莫不變色。陛下發不世之詔,出思慮之表,臣之喜慶,不亦宜乎!」



 在職六年,遷尚書左僕射。時龍見武庫井中,帝親觀之,有喜色。百官將賀,毅獨表曰:「昔龍降鄭時門之外,子產不賀。龍降夏庭,沫流不禁,卜藏其漦,至周幽王,禍釁乃發。《易》稱『潛龍勿用,
 陽在下也。』證據舊典,無賀龍之禮。」詔報曰:「正德未修,誠未有以膺受嘉祥。省來示,以為瞿然。賀慶之事,宜詳依典義,動靜數示。」尚書郎劉漢等議,以為:「龍體既蒼,雜以素文,意者大晉之行,戢武興文之應也。而毅乃引衰世妖異,以疑今之吉祥。又以龍在井為潛,皆失其意。潛之為言,隱而不見。今龍彩質明煥,示人以物,非潛之謂也。毅應推處。」詔不聽。後陰氣解而復合,毅上言:「必有阿黨之臣,姦以事君者,當誅而不誅故也。



 毅以魏立九品,權時之制,未見得人,而有八損,乃上疏曰:



 臣聞:立政者,以官才為本,官才有三難,而興替之所由也。人物難知,一
 也;愛憎難防,二也;情偽難明,三也。今立中正,定九品,高下任意,榮辱在手。操人主之威福,奪天朝之權勢。愛憎決於心,情偽由於己。公無考校之負,私無告訐之忌。用心百態,求者萬端。廉讓之風滅,茍且之欲成。天下訩訩,但爭品位,不聞推讓,竊為聖朝恥之。



 夫名狀以當才為清,品輩以得實為平,安危之要,不可不明。清平者,政化之美也;枉濫者,亂敗之惡也,不可不察。然人才異能,備體者釁。器有大小,達有早晚。前鄙後修,宜受日新之報;抱正違時,宜有質直之稱;度遠闕小,宜得殊俗之狀;任直不飾,宜得清實之譽;行寡才優,宜獲器任之用。是以
 三仁殊途而同歸,四子異行而均義。陳平、韓信笑侮於邑里,而收功於帝王;屈原、伍胥不容於人主,而顯名於竹帛,是篤論之所明也。



 今之中正,不精才實,務依黨利,不均稱尺,備隨愛憎。所欲與者,獲虛以成譽;所欲下者,吹毛以求疵。高下逐強弱,是非由愛憎。隨世興衰,不顧才實,衰則削下,興則扶上,一人之身,旬日異狀。或以貨賂自通,或以計協登進,附託者必達,守道者困悴。無報於身,必見割奪。有私於己,必得其欲。是以上品無寒門,下品無勢族。暨時有之,皆曲有故。慢主罔時,實為亂源。損政之道一也。



 置州都者,取州里清議,咸所歸服,將以
 鎮異同,一言議。不謂一人之身,了一州之才,一人不審便坐之。若然,自仲尼以上,至于庖犧,莫不有失,則皆不堪,何獨責於中人者哉!若殊不修,自可更選。今重其任而輕其人,所立品格,還訪刁攸。攸非州里之所歸,非職分之所置。今訪之,歸正於所不服,決事於所不職,以長讒構之源,以生乖爭之兆,似非立都之本旨,理俗之深防也。主者既善刁攸,攸之所下而復選以二千石,已有數人。劉良上攸之所下,石公罪攸之所行,駮違之論橫於州里,嫌讎之隙結於大臣。夫桑妾之訟,禍及吳、楚;鬥雞之變,難興魯邦。況乃人倫交爭而部黨興,刑獄滋生
 而禍根結。損政之道二也。



 本立格之體,將謂人倫有序,若貫魚成次也。為九品者,取下者為格,謂才德有優劣,倫輩有首尾。今之中正,務自遠者,則抑割一國,使無上人;穢劣下比,則拔舉非次,并容其身。公以為格,坐成其私。君子無大小之怨,官政無繩姦之防。使得上欺明主,下亂人倫。乃使優劣易地,首尾倒錯。推貴異之器,使在凡品之下,負戴不肖,越在成人之首。損政之道三也。



 陛下踐阼,開天地之德,弘不諱之詔,納忠直之言,以覽天下之情,太平之基,不世之法也。然嘗罰,自王公以至于庶人,無不加法。置中正,委以一國之重,無嘗罰之防。人
 心多故,清平者寡,故怨訟者眾。聽之則告訐無已,禁絕則侵枉無極,與其理訟之煩,猶愈侵枉之害。今禁訟訴,則杜一國之口,培一人之勢,使得縱橫,無所顧憚。諸受枉者抱怨積直,獨不蒙天地無私之德,而長壅蔽於邪人之銓。使上明不下照,下情不上聞。損政之道四也。



 昔在前聖之世,欲敦風俗,鎮靜百姓,隆鄉黨之義,崇六親之行,禮教庠序以相率,賢不肖於是見矣。然鄉老書其善以獻天子,司馬論其能以官於職,有司考績以明黜陟。故天下之人退而修本,州黨有德義,朝廷有公正,浮華邪佞無所容厝。今一國之士多者千數,或流徙異邦,
 或取給殊方,面猶不識,況盡其才力!而中正知與不知,其當品狀,采譽於臺府,納毀於流言。任己則有不識之蔽,聽受則有彼此之偏。所知者以愛憎奪其平,所不知者以人事亂其度;既無鄉老紀行之譽,又非朝廷考績之課;遂使進宮之人,棄近求遠,背本逐末。位以求成,不由行立,品不校功,黨譽虛妄。損政五也。



 凡所以立品設狀者,求人才以理物也,非虛飾名譽,相為好醜。雖孝悌之行,不施朝廷,故門外之事,以義斷恩。既以在官,職有大小,事有劇易,各有功報,此人才之實效,功分之所得也。今則反之,於限當報,雖職之高,還附卑品,無績於官,
 而獲高敘,是為抑功實而隆虛名也。上奪天朝考績之分,下長浮華朋黨之士。損政六也。



 凡官不同事,人不同能,得其能則成,失其能則敗。今品不狀才能之所宜,而以九等為例。以品取人,或非才能之所長;以狀取人,則為本品之所限。若狀得其實,猶品狀相妨,繫縶選舉,使不得精於才宜。況今九品,所疏則削其長,所親則飾其短。徒結白論,以為虛譽,則品不料能,百揆何以得理,萬機何以得修?損政七也。



 前九品詔書,善惡必書,以為褒貶,當時天下,少有所忌。今之九品,所下不彰其罪,所上不列其善,廢褒貶之義,任愛憎之斷,清濁同流,以植其
 私。故反違前品,大其形勢,以驅動眾人,使必歸己。進者無功以表勸,退者無惡以成懲。懲勸不明,則風俗汙濁,天下人焉得不解德行而銳人事?損政八也。



 由此論之,選中正而非其人,授權勢而無嘗罰,或缺中正而無禁檢,故邪黨得肆,枉濫縱橫。雖職名中正,實為姦府;事名九品,而有八損。或恨結於親親,猜生於骨肉,當身困於敵讎,子孫離其殃咎。斯乃歷世之患,非徒當今之害也。是以時主觀時立法,防姦消亂,靡有常制,故周因於殷,有所損益。至于中正九品,上聖古賢皆所不為,豈蔽於此事而有不周哉,將以政化之宜無取於此也。自魏立
 以來,未見其得人之功,而生讎薄之累。毀風敗俗,無益於化,古今之失,莫大於此。愚臣以為宜罷中正,除九品,棄魏氏之弊法,立一代之美制。



 疏奏,優詔答之。後司空衛瓘等亦共表宜省九品,復古鄉議里選。帝竟不施行。



 毅夙夜在公,坐而待旦,言議切直,無所曲撓,為朝野之所式瞻。嘗散齋而疾,其妻省之,毅便奏加妻罪而請解齋。妻子有過,立加杖捶,其公正如此。然以峭直,故不至公輔。帝以毅清貧,賜錢三十萬,日給米肉。年七十,告老。久之,見許,以光祿大夫歸第,門施行馬,復賜錢百萬。



 後司徒舉毅為青州大中正,尚書以毅懸車致仕,不宜勞
 以碎務。陳留相樂安孫尹表曰:「禮,凡卑者執勞,尊得居逸,是順敘之宜也。司徒魏舒、司隸校尉嚴詢與毅年齒相近,往者同為散騎常侍,後分授外內之職,資途所經,出處一致。今詢管四十萬戶州,兼董司百僚,總攝機要,舒所統殷廣,兼執九品,銓十六州論議,主者不以為劇。毅但以知一州,便謂不宜累以碎事,於毅太優,詢、舒太劣。若以前聽致仕,不宜復與遷授位者,故光祿大夫鄭袤為司空是也。夫知人則哲,惟帝難之。尚可復委以宰輔之任,不可諮以人倫之論,臣竊所未安。昔鄭武公年過八十,入為周司徒,雖過懸車之年,必有可用。毅前為
 司隸,直法不撓,當朝之臣,多所按劾。諺曰:『受堯之誅,不能稱堯。』直臣無黨,古今所悉。是以汲黯死於淮陽,董仲舒裁為諸侯之相。而毅獨遭聖明,不離輦轂,當世之士咸以為榮。毅雖身偏有風疾,而志氣聰明,一州品第,不足勞其思慮。毅疾惡之心小過,主者必疑其論議傷物,故高其優禮,令去事實,此為機閣毅,使絕人倫之路也。臣州茂德惟毅,越毅不用,則清談倒錯矣。」



 於是青州自二品已上憑毅取正。光祿勳石鑒等共奏曰:「謹按陳留相孫尹表及與臣等書如左。臣州履境海岱,而參風齊、魯,故人俗務本,而世敦德讓,今雖不充於舊,而遺訓猶存,
 是以人倫歸行,士識所守也。前被司徒符,當參舉州大中正。僉以光祿大夫毅,純孝至素,著在鄉閭。忠允亮直,竭於事上,仕不為榮,惟期盡節。正身率道,崇公忘私,行高義明,出處同揆。故能令義士宗其風景,州閭歸其清流。雖年耆偏疾,而神明克壯,實臣州人士所思準繫者矣。誠以毅之明格,能不言而信,風之所動,清濁必偃,以稱一州咸同之望故也。竊以為禮賢尚德,教之大典,王制奪與,動為開塞,而士之所歸,人倫為大。臣等虛劣,雖言廢於前,今承尹書,敢不列啟。按尹所執,非惟惜名議於毅之身,亦通陳朝宜奪與大準。以為尹言當否,應蒙評議。」



 由
 是毅遂為州都,銓正人流,清濁區別,其所彈貶,自親貴者始。太康六年卒,武帝撫几驚曰:「失吾名臣,不得生作三公!」即贈儀同三司,使者監護喪事。羽林左監北海王宮上疏曰:「中詔以毅忠允匪躬,贈班台司,斯誠聖朝考績以毅著勛之美事也。臣謹按,謚者行之迹,而號者功之表。今毅功德並立,而有號無謚,於義不體。臣竊以《春秋》之事求之,謚法主於行而不繫爵。然漢、魏相承,爵非列侯,則皆沒而高行,不加之謚,至使三事之賢臣,不如野戰之將。銘跡所殊,臣願聖世舉《春秋》之遠制,改列爵之舊限,使夫功行之實不相掩替,則莫不率賴。若以革
 舊毀制,非所倉卒,則毅之忠益,雖不攻城略地,論德進爵,亦應在例。臣敢惟行甫請周之義,謹牒毅功行如石。」帝出其表使八坐議之,多同宮議。奏寢不報。二子:暾、總。



 暾字長升,正直有父風。太康初為博士,會議齊王攸之國,加崇典禮,暾與諸博士坐議迕旨。武帝大怒,收暾等付廷尉。會赦得出,免官。初,暾父毅疾馮紞姦佞,欲奏其罪,未果而卒。至是,紞位宦日隆,暾慨然曰:「使先人在,不令紞得無患。」



 後為酸棗令,轉侍御史。會司徒王渾主簿劉輿獄辭連暾,將收付廷尉。渾不欲使府有過,欲距劾自舉之。與暾更相曲直,渾怒,便遜位還第。暾乃奏渾曰:「
 謹按司徒王渾,蒙國厚恩,備位鼎司,不能上佐天子,調和陰陽,下遂萬物之宜,使卿大夫各得其所。敢因劉輿拒扞詔使,私欲大府興長獄訟。昔陳平不答漢文之問,邴吉不問死人之變,誠得宰相之體也。既興刑獄,怨懟而退,舉動輕速,無大臣之節,請免渾官。右長史、楊丘亭侯劉肇,便辟善柔,茍於阿順,請大鴻臚削爵土。」諸聞暾此奏者,皆歎美之。



 其後武庫火,尚書郭彰率百人自衛而不救火,暾正色詰之。彰怒曰:「我能截君角也。」暾勃然謂彰曰:「君何敢恃寵作威作福,天子法冠而欲截角乎!」求紙筆奏之,彰伏不敢言,眾人解釋,乃止。彰久貴豪侈,
 每出輒眾百餘人。自此之後,務從簡素。



 暾遷太原內史,趙王倫篡位,假征虜將軍,不受,與三王共舉義。惠帝復阼,暾為左丞,正色立朝,三臺清肅。尋兼御史中丞,奏免尚書僕射、東安公繇及王粹、董艾等十餘人。朝廷嘉之,遂即真。遷中庶子、左衛將軍、司隸校尉,奏免武陵王澹及何綏、劉坦、溫畿、李晅等。長沙王乂討齊王冏,暾豫謀,封朱虛縣公,千八百戶。乂死,坐免。頃之,復為司隸。



 及惠帝之幸長安也,留暾守洛陽。河間王顒遣使鴆羊皇后,暾乃與留臺僕射荀籓、河南尹周馥等上表,理后無罪。語在《后傳》。顒見表,大怒,遣陳顏、呂朗率騎五千收暾,暾
 東奔高密王略。會劉根作逆,略以暾為大都督,加鎮軍將軍討根。暾戰失利,還洛。至酸棗,值東海王越奉迎大駕。及帝還洛,羊后反宮。后遣使謝暾曰:「賴劉司隸忠誠之志,得有今日。」以舊勛復封爵,加光祿大夫。



 暾妻前卒,先陪陵葬。子更生初婚,家法,婦當拜墓,攜賓客親屬數十乘,載酒食而行。先是,洛陽令王棱為越所信,而輕暾,暾每欲繩之,棱以為怨。時劉聰、王彌屯河北,京邑危懼。棱告越,云暾與彌鄉親而欲投之。越嚴騎將追暾,右長史傅宣明暾不然。暾聞之,未至墓而反,以正義責越,越甚慚。



 及劉曜寇京師,以暾為撫軍將軍、假節、都督城守
 諸軍事。曜退,遷尚書僕射。越憚暾久居監司,又為眾情所歸,乃以為右光祿大夫,領太子少傅,加散騎常侍。外示崇進,實奪其權。懷帝又詔暾領衛尉,加特進。後復以暾為司隸,加侍中。暾五為司隸,允協物情故也。



 王彌入洛,百官殲焉。彌以暾鄉里宿望,故免於難。暾因說彌曰:「今英雄競起,九州幅裂,有不世之功者,宇內不容。將軍自興兵已來,何攻不剋,何戰不勝,而復與劉曜不協,宜思文種之禍,以范蠡為師。且將軍可無帝王之意,東王本州,以觀時勢,上可以混一天下,下可以成鼎峙之事,豈失孫、劉乎!蒯通有言,將軍宜圖之。」彌以為然,使暾于
 青州,與曹嶷謀,且徵之。暾至東阿,為石勒游騎所獲,見彌與嶷書而大怒,乃殺之。暾有二子:佑、白。



 佑為太傅屬,白太子舍人。白果烈有才用,東海王越忌之,竊遣上軍何倫率百餘人入暾第,為劫取財物,殺白而去。



 總字弘紀,好學直亮,後叔父彪,位至北軍中候。



 程衛,字長玄,廣平曲周人也。少立操行,彊正方嚴。劉毅聞其名,辟為都官從事。毅奏中護軍羊琇犯憲應死。武帝與琇有舊,乃遣齊王攸喻毅,毅許之。衛正色以為不可,徑自馳車入護軍營,收琇屬吏,考問陰私,先奏琇所
 犯狼藉,然後言於毅。由是名振遐邇,百官厲行。遂辟公府掾,遷尚事郎、侍御史,在職皆以事幹顯。補洛陽令,歷安定、頓丘太守,所蒞著績。卒於官。



 和嶠,字長輿,汝南西平人也。祖洽,魏尚書令。父逌,魏吏部尚書。嶠少有風格,慕舅夏侯玄之為人,厚自崇重。有盛名於世,朝野許其能風俗,理人倫。襲父爵上蔡伯,起家太子舍人。累遷潁川太守,為政清簡,甚得百姓歡心。太傅從事中郎庾顗見而歎曰:「嶠森森如千丈松,雖磥可多節目,施之大廈,有棟梁之用。」賈充亦重之,稱於
 武帝,入為給事黃門侍郎,遷中書令,帝深器遇之。舊監令共車入朝,時荀勖為監,嶠鄙勖為人,以意氣加之,每同乘,高抗專車而坐。乃使監令異車,自嶠始也。



 吳平,以參謀議功,賜弟郁爵汝南亭侯。嶠轉侍中,愈被親禮,與任愷、張華相善。嶠見太子不令,因侍坐曰:「皇太子有淳古之風,而季世多偽,恐不了陛下家事。」帝默然不答。後與荀顗、荀勖同侍,帝曰:「太子近入朝,差長進,卿可俱詣之,粗及世事。」即奉詔而還。顗、勖並稱太子明識弘雅,誠如明詔。嶠曰:「聖質如初耳!」帝不悅而起。嶠退居,恒懷慨歎,知不見用,猶不能已。在御坐言及社稷,未嘗不以儲
 君為憂。帝知其言忠,每不酬和。後與嶠語,不及來事。或以告賈妃,妃銜之。太康末,為尚書,以母憂去職。



 及惠帝即位,拜太子少傅,加散騎常侍、光祿大夫。太子朝西宮,嶠從入。賈后使帝問嶠曰:「卿昔謂我不了家事,今日定云何?」嶠曰:「臣昔事先帝,曾有斯言。言之不效,國之福也。臣敢逃其罪乎!」元康二年卒,贈金紫光祿大夫,加金章紫綬,本位如前。永平初,策謚曰簡。嶠家產豐富,擬於王者,然性至吝,以是獲譏於世,杜預以為嶠有錢癖。以弟郁子濟嗣,位至中書郎。



 郁字仲輿,才望不及嶠,而以清乾稱,歷尚書左右僕射、中書令、尚書令。洛陽傾沒,奔于
 茍晞,疾卒。



 武陔,字元夏,沛國竹邑人也。父周,魏衛尉。陔沈敏有器量,早獲時譽,與二弟韶叔夏、茂季夏並總角知名,雖諸父兄弟及鄉閭宿望,莫能覺其優劣。同郡劉公榮有知人之鑒,常造周,周見其三子焉。公榮曰:「皆國士也。元夏最優,有輔佐之才,陳力就列,可為亞公。叔夏、季夏不減常伯、納言也。」



 陔少好人倫,與潁川陳泰友善。魏明帝世,累遷下邳太守。景帝為大將軍,引為從事中郎,累遷司隸校尉,轉太僕卿。初封亭侯,五等建,改封薛縣侯。文帝
 甚親重之,數與詮論時人。嘗問陳泰孰若其父群,陔各稱其所長,以為群、泰略無優劣,帝然之。泰始初,拜尚書,掌吏部,遷左僕射、左光祿大夫、開府儀同三司。陔以宿齒舊臣,名位隆重,自以無佐命之功,又在魏已為大臣,不得已而居位,深懷遜讓,終始全潔,當世以為美談。卒於位,謚曰定。子輔嗣。



 韶歷吏部郎、太子右衛率、散騎常侍。



 茂以德素稱,名亞於陔,為上洛太守、散騎常侍、侍中、尚書。潁川荀愷年少於茂,即武帝姑子,自負貴戚,欲與茂交,距而不答,由是致怨。及楊駿誅,愷時為僕射,以茂駿之姨弟,陷為逆黨,遂見害。茂清正方直,聞於朝野,
 一旦枉酷,天下傷焉。侍中傅祗上申明之,後追贈光祿勛。



 任愷,字元褒,樂安博昌人也。父昊,魏太常。愷少有識量,尚魏明帝女,累遷中書侍郎、員外散騎常侍。晉國建,為侍中,封昌國縣侯。



 愷有經國之乾,萬機大小多管綜之。性忠正,以社稷為己任,帝器而暱之,政事多諮焉。泰始初,鄭沖、王祥、何曾、荀顗、裴秀等各以老疾歸第。帝優寵大臣,不欲勞以筋力,數遣愷諭旨於諸公,諮以當世大政,參議得失。愷惡賈充之為人也,不欲令久執朝政,每
 裁抑焉。充病之,不知所為。後承間言愷忠貞局正,宜在東宮,使護太子。帝從之,以為太子少傅,而侍中如故,充計畫不行。會秦、雍寇擾,天子以為憂。愷因曰:「秦、涼覆敗,關右騷動,此誠國家之所深慮。宜速鎮撫,使人心有庇。自非威望重臣有計略者,無以康西土也。」帝曰:「誰可任者?」愷曰:「賈充其人也。」中書令庾純亦言之,於是詔充西鎮長安。充用荀勖計得留。



 充既為帝所遇,欲專名勢,而庾純、張華、溫顒、向秀、和嶠之徒皆與愷善,楊珧、王恂、、華暠等充所親敬,於是朋黨紛然。帝知之,召充、愷宴於式乾殿,而謂充等曰:「朝廷宜一,大臣當和。」充、愷各拜謝而
 罷。既而充、愷等以帝已知之而不責,結怨愈深,外相崇重,內甚不平。或為充謀曰:「愷總門下樞耍,得與上親接,宜啟令典選,便得漸疏,此一都令史事耳。且九流難精,間隙易乘。」充因稱愷才能,宜在官人之職。帝不之疑,謂充舉得其才。即日以愷為吏部尚書,加奉車都尉。



 愷既在尚書,選舉公平,盡心所職,然侍覲轉希。充與荀勖、馮紞承間浸潤,謂愷豪侈,用御食器。充遣尚書右僕射、高陽王珪奏愷,遂免官。有司收太官宰人檢核,是愷妻齊長公主得賜魏時御器也。愷既免而毀謗益至,帝漸薄之。然山濤明愷為人通敏有智局,舉為河南尹。坐賊發
 不獲,又免官。復遷光祿勳。



 愷素有識鑒,加以在公勤恪,甚得朝野稱譽。而賈充朋黨又諷有司奏愷與立進令劉友交關。事下尚書,愷對不伏。尚書杜友、廷尉劉良並忠公士也,知愷為充所抑,欲申理之,故遲留而未斷,以是愷及友、良皆免官。愷既失職,乃縱酒耽樂,極滋味以自奉養。初,何劭以公子奢侈,每食必盡四方珍饌,愷乃踰之,一食萬錢,猶云無可下箸處。愷時因朝請,帝或慰諭之,愷初無復言,惟泣而已。後起為太僕,轉太常。



 初,魏舒雖歷位郡守,而未被任遇,愷為侍中,薦舒為散騎常侍。至是舒為右光祿、開府,領司徒,帝臨軒使愷拜授。舒
 雖以弘量寬簡為稱,時以愷有佐世器局,而舒登三公,愷止守散卿,莫不為之憤歎也。愷不得志,竟以憂卒,時年六十一,謚曰元,子罕嗣。



 罕字子倫,幼有門風,才望不及愷,以淑行致稱,為清平佳士。歷黃門侍郎、散騎常侍、兗州刺史、大鴻臚。



 崔洪,字良伯,博陵安平人也。高祖寔,著名漢代。父讚,魏吏部尚書、左僕射,以雅量見稱。洪少以清厲顯名,骨鯁不同於物,人之有過,輒面折之,而退無後言。武帝世,為御史治書。時長樂馮恢父為弘農太守,愛少子淑,欲以
 爵傳之。恢父終,服闋,乃還鄉里,結草為廬,陽喑不能言,淑得襲爵。恢始仕為博士祭酒,散騎常侍翟嬰薦恢高行邁俗,侔繼古烈。洪奏恢不敦儒素,令學生番直左右,雖有讓侯微善,不得稱無倫輩,嬰為浮華之目。遂免嬰官,朝廷憚之。尋為尚書左丞,時人為之語曰:「叢生棘刺,來自博陵。在南為鷂,在北為鷹。」選吏部尚書,舉用甄明,門無私謁。薦雍州刺史郤詵代己為左丞。詵後糾洪,洪謂人曰:「我舉郤丞而還奏我,是挽弩自射也。」詵聞曰:「昔趙宣子任韓厥為司馬,以軍法戮宣子之僕。宣子謂諸大夫曰:『可賀我矣,我選厥也任其事。』崔侯為國舉才,我
 以才見舉,惟官是視,各明至公,何故私言乃至此!」洪聞其言而重之。



 洪口不言貨財,手不執珠玉。汝南王亮常晏公卿,以琉璃鐘行酒。酒及洪,洪不執。亮問其故,對曰:「慮有執玉不趨之義故爾」。然實乖其常性,故為詭說。楊駿誅,洪與都水使者王佑親,坐見黜。後為大司農,卒於官。子廓,散騎侍郎,亦以正直稱。



 郭奕,字大業,太原陽曲人也。少有重名,山濤稱其高簡有雅量。初為野王令,羊祜常過之,奕歎曰:「羊叔子何必減郭大業!」少選復往,又歎曰:「羊叔子去人遠矣。」遂送祜
 出界數百里,坐此免官。咸熙末,為文帝相國主薄。時鐘會反於蜀,荀勖即會之從甥,少長會家,勖為文帝掾,奕啟出之。帝雖不用,然知其雅正。武帝踐阼,初建東宮,以奕及鄭默並為中庶子。遷右衛率、驍騎將軍,封平陵男。咸寧初,遷雍州刺史、鷹揚將軍,尋假赤幢曲蓋、鼓吹。奕有寡姊,隨奕之官,姊下僮僕多有姦犯,而為人所糾。奕省按畢,曰:「大丈夫豈當以老姊求名?」遂遣而不問。時亭長李含有俊才,而門寒為豪族所排,奕用為別駕,含後果有名位,時以奕為知人。



 太康中,徵為尚書。奕有重名,當世朝臣皆出其下。時帝委任楊駿,奕表駿小器,不可
 任以社稷。帝不聽,駿後果誅。及奕疾病,詔賜錢二十萬,日給酒米。太康八年卒,太常上謚為景。有司議以貴賤不同號,謚與景皇同,不可,請謚曰穆。詔曰:「謚所以旌德表行,按謚法一德不懈為簡。奕忠毅清直,立德不渝。」於是遂賜謚曰簡。



 侯史光,字孝明,東萊掖人也。幼有才悟,受學於同縣劉夏。舉孝廉,州辟別駕。咸熙初,為洛陽典農中郎將,封關中侯。泰始初,拜散騎常侍,尋兼侍中。與皇甫陶、荀暠持節循省風俗,及還,奏事稱旨,轉城門校尉,進爵臨海侯。
 其年詔曰:「光忠亮篤素,有居正執義之心,歷職內外,恪勤在公,其以光為御史中丞。雖屈其列校之位,亦所以伸其司直之才。」光在職寬而不縱。太保王祥久疾廢朝,光奏請免之,詔優祥而寢光奏。後遷少府,卒官,詔賜朝服一具、衣一襲、錢三十萬、布百匹。及葬,又詔曰:「光厲志守約,有清忠之節。家極貧儉,其賜錢五十萬。」光儒學博古,歷官著績,文筆奏議皆有條理。長子玄嗣,官至玄菟太守。卒,子施嗣,東莞太守。



 何攀,字惠興,蜀郡郫人也。仕州為主薄。屬刺史皇甫晏
 為牙門張弘所害,誣以大逆。時攀適丁母喪,遂詣梁州拜表,證晏不反,故晏冤理得申。王濬為益州,辟為別駕。濬謀伐吳,遣攀奉表詣臺,口陳事機,詔再引見,乃令張華與攀籌量進時討之宜。濬兼遣攀過羊祜,面陳伐吳之策。攀善於將命,帝善之,詔攀參濬軍事。及孫皓降於濬,而王渾恚於後機,欲攻濬,攀勸濬送皓與渾,由是事解。以攀為濬輔國司馬,封關內侯。轉滎陽令,上便宜十事,甚得名稱。除廷尉平。時廷尉卿諸葛沖以攀蜀士,輕之,及共斷疑獄,沖始歎服。遷宣城太守,不行,轉散騎侍郎。楊駿執政,多樹親屬,大開封嘗,欲以恩澤自衛。攀以為
 非,乃與石崇共立議奏之。語在崇傳。帝不納。以豫誅駿功,封西城侯,邑萬戶,賜絹萬匹,弟逢平卿侯,兄子逵關中侯。攀固讓所封戶及絹之半,餘所受者分給中外宗親,略不入己。遷翊軍校尉,頃之,出為東羌校尉。徵為揚州刺史,在任三年,遷大司農。轉兗州刺史,加鷹揚將軍,固讓不就。太常成粲、左將軍卞粹勸攀涖職,中詔又加切厲,攀竟稱疾不起。及趙王倫篡位,遣使召攀,更稱疾篤。倫怒,將誅之,攀不得已,扶疾赴召。卒于洛陽,時年五十八。攀居心平允,水位官整肅,愛樂人物,敦儒貴才。為梁、益二州中正,引致遺滯。巴西陳壽、閻乂、犍為費立皆西
 州名士,並被鄉閭所謗,清議十餘年。攀申明曲直,咸免冤濫。攀雖居顯職,家甚貧素,無妾媵伎樂,惟以周窮濟乏為事。子璋嗣,亦有父風。



 史臣曰:幽厲不君,上德猶懷進善;共驩在位,大聖之所不堪。況乎志士仁人,寧求茍合!懷其寵秩,所以繫其存亡者也。雖復自口銷金,投光撫劍,馳書北闕,敗車猶踐,而諫主不易,譏臣實難。劉毅一遇寬容,任和兩遭膚受,詳觀餘烈,亦各其心焉。若夫武陔懷魏臣之志,崔洪愛郤詵之道,長升勸王彌之尊,何攀從趙倫之命,君子之人,觀乎臨事者也。



 贊曰:仲雄初令,忠謇揚庭。身方諸葛,帝擬桓、靈。大業非楊,元褒誚賈。和氏條暢,堪施大廈。崔門不謁,聲飛朝野。侯史、武陔,輔佐之才。何攀平允,冤濫多回。



\end{pinyinscope}