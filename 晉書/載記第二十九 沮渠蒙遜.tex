\article{載記第二十九 沮渠蒙遜}

\begin{pinyinscope}

 沮渠蒙遜



 沮渠蒙遜,臨松盧水胡人也。其先世為匈奴左沮渠,遂以官為氏焉。蒙遜博涉群史,頗曉天文,雄傑有英略,滑稽善權變,梁熙、呂光皆奇而憚之,故常游飲自晦。會伯父羅仇、麴粥從呂光征河南,光前軍大敗,麴粥言於兄羅仇曰:「主上荒耄驕縱,諸子朋黨相傾,讒人側目。今軍敗將死,正是智勇見猜之日,可不懼乎!吾兄弟素為所
 憚,與其經死溝瀆,豈若勒眾向西平,出苕藋,奮臂大呼,涼州不足定也。」羅仇曰:「理如汝言,但吾家累世忠孝,為一方所歸,寧人負我,無我負人。」俄而皆為光所殺。宗姻諸部會葬者萬餘人,蒙遜哭謂眾曰:「昔漢祚中微,吾之乃祖翼獎竇融,保寧河右。呂王昏耄,荒虐無道,豈可不上繼先祖安時之志,使二父有恨黃泉!」眾咸稱萬歲。遂斬光中田護軍馬邃、臨松令井祥以盟,一旬之間,眾至萬餘。屯據金山,與從兄男成推光建康太守段業為使持節、大都督、龍驤大將軍、涼州牧、建康公,改呂光龍飛二年為神璽元年。業以蒙遜為張掖太守,男成為輔國
 將軍,委以軍國之任。



 業將使蒙遜攻西郡,眾咸疑之。蒙遜曰:「此郡據嶺之要,不可不取。」業曰:「卿言是也。」遂遣之。蒙遜引水灌城,城潰,執太守呂純以歸。於是王德以晉昌,孟敏以敦煌降業。業封蒙遜臨池侯。呂弘去張掖,將東走,業議欲擊之。蒙遜諫曰:「歸師勿遏,窮寇弗追,此兵家之戎也。不如縱之,以為後圖。」業曰:「一日縱敵,悔將無及。」遂率眾追之,為弘所敗。業賴蒙遜而免,歎曰:「孤不能用子房之言,以至於此!」業築西安城,以其將臧莫孩為太守。蒙遜曰:「莫孩勇而無謀,知進忘退,所謂為之築冢,非築城也。」業不從。俄而為呂纂所敗。蒙遜懼業不能容
 己,每匿智以避之。



 業僭稱涼王,以蒙遜為尚書左丞,梁中庸為右丞。



 呂光遣其二子紹、纂伐業,業請救於禿髮烏孤,烏孤遣其弟鹿孤及楊軌救業。紹以業等軍盛,欲從三門關挾山而東。纂曰:「挾山示弱,取敗之道,不如結陣衛之,彼必憚我而不戰也。」紹乃引軍而南。業將擊之,蒙遜諫曰:「楊軌恃虜騎之強,有窺覦之志。紹、纂兵在死地,必決戰求生。不戰則有太山之安,戰則有累卵之危。」業曰:「卿言是也。」乃按兵不戰。紹亦難之,各引兵歸。



 業憚蒙遜雄武,微欲遠之,乃以蒙遜從叔益生為酒泉太守,蒙遜為臨池太守。業門下侍郎馬權雋爽有逸氣,武
 略過人。業以權代蒙遜為張掖太守,甚見親重,每輕陵蒙遜。蒙遜亦憚而怨之,乃譖之於業曰:「天下不足慮,惟當憂馬權耳。」業遂殺之。蒙遜謂男成曰:「段業愚闇,非濟亂之才,信讒愛佞,無鑒斷之明。所憚惟索嗣、馬權,今皆死矣,蒙遜欲除業以奉兄何如?」男成曰:「業羈旅孤飄,我所建立,有吾兄弟,猶魚之有水,人既親我,背之不祥。」乃止。蒙遜既為業所憚,內不自安,請為西安太守。業亦以蒙遜有大志,懼為朝夕之變,乃許焉。



 蒙遜期與男成同祭蘭門山,密遣司馬許咸告業曰:「男成欲謀叛,許以取假日作逆。若求祭蘭門山,臣言驗矣。」至期日,果然。業收
 男成,令自殺。男成曰:「蒙遜欲謀叛,先已告臣,臣以兄弟之故,隱忍不言。以臣今在,恐部人不從,與臣剋期祭山,返相誣告。臣若朝死,蒙遜必夕發。乞詐言臣死,說臣罪惡,蒙遜必作逆,臣投袂討之,事無不捷。」業不從。蒙遜聞男成死,泣告眾曰:「男成忠於段公,枉見屠害,諸君能為報仇乎?且州土兵亂,似非業所能濟。吾所以初奉之者,以之為陳、吳耳,而信讒多忌,枉害忠良,豈可安枕臥觀,使百姓離於塗炭。」男成素有恩信,眾皆憤泣而從之。比至氐池,眾逾一萬。鎮軍臧莫孩率部眾附之,羌胡多起兵響應。蒙遜壁于侯塢。



 業先疑其右將軍田昂,幽之于
 內,至是,謝而赦之,使與武衛梁中庸等攻蒙遜。業將王豐孫言於業曰:「西平諸田,世有反者,昂貌恭而心很,志大而情險,不可信也。」業曰:「吾疑之久矣,但非昂無可以討蒙遜。」豐孫言既不從,昂至侯塢,率騎五百歸于蒙遜。蒙遜至張掖,昂兄子承愛斬關內之,業左右皆散。蒙遜大呼曰:「鎮西何在?」軍人曰:「在此。」業曰:「孤單飄一己,為貴門所推,可見丐餘命,投身嶺南,庶得東還,與妻子相見。」蒙遜遂斬之。



 業,京兆人也。博涉史傳,有尺牘之才,為杜進記室,從征塞表。儒素長者,無他權略,威禁不行,群下擅命,尤信卜筮、讖記、巫覡、徵祥,故為姦佞所誤。



 隆安五
 年,梁中庸、房晷、田昂等推蒙遜為使持節、大都督、大將軍、涼州牧、張掖公,赦其境內,改元永安。署從兄伏奴為鎮軍將軍、張掖太守、和平侯,弟挐為建忠將軍、都谷侯,田昂為鎮南將軍、西郡太守,臧莫孩為輔國將軍,房晷、梁中庸為左右長史,張騭、謝正禮為左右司馬。擢任賢才,文武咸悅。



 時姚興遣將姚碩德攻呂隆于姑臧,蒙遜遣從事中郎李典聘于興,以通和好。蒙遜以呂隆既降于興,酒泉、涼寧二郡叛降李玄盛,乃遣建忠挐、牧府長史張潛見碩德于姑臧,請軍迎接,率郡人東遷。碩德大悅,拜潛張掖太守,挐建康太守。潛勸蒙遜東遷。挐私於
 蒙遜曰:「呂氏猶存,姑臧未拔,碩德糧竭將遠,不能久也。何故違離桑梓,受制於人!」輔國莫孩曰:「建忠之言是也。」蒙遜乃斬張潛,因下書曰:「孤以虛薄,猥忝時運。未能弘闡大獻,戡蕩群孽,使桃蟲鼓翼東京,封豕烝涉西裔,戎車屢動,干戈未戢,農失三時之業,百姓戶不粒食。可蠲省百徭,專功南畝,明設科條,務盡地利。」



 時梁中庸為西郡太守,西奔李玄盛。蒙遜聞之,笑曰:「吾與中庸義深一體,而不信我,但自負耳,孤豈怪之!」乃盡歸其妻孥。



 蒙遜下令曰:「養老乞言,晉文納輿人之誦,所以能招禮英奇,致時邕之美。況孤寡德,智不經遠,而可不思聞讜言以
 自鏡哉!內外群僚,其各搜揚賢雋,廣進芻蕘,以匡孤不逮。」



 遣輔國臧莫孩襲山北虜,大破之。姚興遣將齊難率眾四萬迎呂隆,隆勸難伐蒙遜,難從之。莫孩敗其前軍,難乃結盟而還。



 蒙遜伯父中田護軍親信、臨松太守孔篤並驕奢侵害,百姓苦之。蒙遜曰:「亂吾國者,二伯父也,何以綱紀百姓乎!」皆令自殺。



 蒙遜襲狄洛磐於番禾,不剋,遷其五百餘戶而還。



 姚興遣使人梁斐、張構等拜蒙遜鎮西大將軍、沙州刺史、西海侯。時興亦拜禿髮傉檀為車騎將軍,封廣武公。蒙遜聞之,不悅,謂斐等曰:「傉檀上公之位,而身為侯者何也!」構對曰:「傉檀輕狡不仁,款
 誠未著,聖朝所以加其重爵者,褒其歸善即敘之義耳。將軍忠貫白日,勛高一時,當入諧鼎味,匡贊帝室,安可以不信待也。聖朝爵必稱功,官不越德,如尹緯、姚晁佐命初基,齊難、徐洛元勳驍將,並位纔二品,爵止侯伯。將軍何以先之乎?竇融殷勤固讓,不欲居舊臣之右,未解將軍忽有此問!」蒙遜曰:「朝廷何不即以張掖見封,乃更遠封西海邪?」構曰:「張掖,規畫之內,將軍已自有之。所以遠授西海者,蓋欲廣大將軍之國耳。」蒙遜大悅,乃受拜。



 時地震,山崩折木。太史令劉梁言於蒙遜曰:「辛酉,金也。地動於金,金動刻木,大軍東行無前之徵。」時張掖城每
 有光色,蒙遜曰:「王氣將成,百戰百勝之象也。」遂攻禿髮西郡太守楊統於日勒。統降,拜為右長史,寵踰功舊。



 張掖太守句呼勒出奔西涼。以從弟成都為金山太守,羅仇子也;鄯為西郡太守,麴粥子也。句呼勒自西涼奔還,待之如初。



 蒙遜率騎二萬東征,次于丹嶺,北虜大人思盤率部落三千降之。



 時木連理,生于永安,永安令張披上書曰:「異枝同乾,遐方有齊化之應;殊本共心,上下有莫二之固。蓋至道之嘉祥,大同之美征。」蒙遜曰:「此皆二千石令長匪躬濟時所致,豈吾薄德所能感之!」



 蒙遜率步騎三萬伐禿髮傉檀,次于西郡。大風從西北來,氣
 有五色,俄而晝昏。至顯美,徙數千戶而還。傉檀追及蒙遜于窮泉,蒙遜將擊之。諸將皆曰:「賊已安營,弗可犯也。」蒙遜曰:「傉檀謂吾遠來疲弊,必輕而無備,及其壘壁未成,可以一鼓而滅。」進擊,敗之,乘勝至于姑臧,夷夏降者萬數千戶。傉檀懼,請和,許之而歸。及傉檀南奔樂都,魏安人焦朗據姑臧自立,蒙遜率步騎三萬攻朗,剋而宥之。饗文武將士于謙光殿,班賜金馬有差。以敦煌張穆博通經史,才藻清贍,擢拜中書侍郎,委以機密之任。以其弟挐為護羌校尉、秦州刺史,封安平侯,鎮姑臧。旬餘而挐死,又以從祖益子為鎮京將軍、護羌校尉、秦州剌
 史,鎮姑臧。



 俄而蒙遜遷于姑臧,以義熙八年僭即河西王位,大赦境內,改元玄始。置官僚,如呂光為三河王故事。繕宮殿,起城門諸觀。立其子政德為世子,加鎮衛大將軍、錄尚書事。



 傉檀來伐,蒙遜敗之於若厚塢。傉檀湟河太守文支據湟川,護軍成宜侯率眾降之。署文支鎮東大將軍、廣武太守、振武侯,成宜侯為振威將軍、湟川太守,以殿中將軍王建為湟河太守。蒙遜下書曰:「古先哲王應期撥亂者,莫不經略八表,然後光闡純風。孤雖智非靖難,職在濟時,而狡虜傉檀鴟峙舊京,毒加夷夏。東苑之戮,酷甚長平,邊城之禍,害深獫狁。每念蒼生之
 無辜,是以不遑啟處,身疲甲胄,體倦風塵。雖傾其巢穴,傉檀猶未授首。傉檀弟文支追項伯歸漢之義,據彼重籓,請為臣妾。自西平已南,連城繼順。惟傉檀窮獸,守死樂都。四支既落,命豈久全!五緯之會已應,清一之期無賒,方散馬金山,黎元永逸。可露布遠近,咸使聞知。」



 蒙遜西如苕藋,遣冠軍伏恩率騎一萬襲卑和、烏啼二虜,大破之,俘二千餘落而還。



 蒙遜寢于新臺,閹人王懷祖擊蒙遜,傷足,其妻孟氏擒斬之,夷其三族。



 蒙遜母車氏疾篤,蒙遜升南景門,散錢以賜百姓。下書曰:「孤庶憑宗廟之靈,乾坤之祐,濟否剝之運會,拯遺黎之荼蓼,上望掃
 清氣穢,下冀保寧家福。而太后不豫,涉歲彌增,將刑獄枉濫,眾有怨乎?賦役繁重,時不堪乎?群望不絜,神所譴乎?內省諸身,未知罪之攸在。可大赦殊死已下。」俄而車氏死。



 蒙遜遣其將運糧于湟河,自率眾攻剋乞伏熾磐廣武郡。以運糧不繼,自廣武如湟河,度浩亹。熾磐遣將乞伏魋尼寅距蒙遜,蒙遜擊斬之。熾磐又遣將王衡、折斐、麴景等率騎一萬據勒姐嶺,蒙遜且戰且前,大破之,擒折斐等七百餘人,麴景奔還。蒙遜以弟漢平為折衝將軍、湟河太守,乃引還。



 晉益州刺史朱齡石遣使來聘。蒙遜遣舍人黃迅報聘益州,因表曰:「上天降禍,四海分
 崩,靈耀擁於南裔,蒼生沒于醜虜。陛下累聖重光,道邁周、漢,純風所被,八表宅心。臣雖被髮邊徼,才非時雋,謬為河右遺黎推為盟主。臣之先人,世荷恩寵,雖歷夷險,執義不回,傾首陽,乃心王室。去冬益州刺史朱齡石遣使詣臣,始具朝廷休問。承車騎將軍劉裕秣馬揮戈,以中原為事,可謂天贊大晉,篤生英輔。臣聞少康之興大夏,光武之復漢業,皆奮劍而起,眾無一旅,猶能成配天之功,著《車攻》之詠。陛下據全楚之地,擁荊、揚之銳,而可垂拱晏然,棄二京以資戎虜!若六軍北軫,剋復有期,臣請率河西戎為晉右翼前驅。」



 熾磐率眾三萬襲湟河,
 漢平力戰固守,遣司馬隗仁夜出擊熾磐,斬級數百。熾磐將引退,先遣老弱。漢平長史焦昶、將軍段景密信招熾磐,熾磐復進攻漢平。漢平納昶、景之說,而縛出降。仁勒壯士百餘據南門樓上,三日不下,眾寡不敵,為熾磐所擒。熾磐怒,命斬之。段暉諫曰:「仁臨難履危,奮不顧命,忠也。宜宥之,以厲事君。」熾磐乃執之而歸。在熾磐所五年,暉又為之固請,乃得還姑臧。及至,蒙遜執其手曰:「卿孤之蘇武也!」以為高昌太守。為政有威惠之稱,然頗以愛財為失。



 蒙遜西祀金山,遣沮渠廣宗率騎一萬襲烏啼虜,大捷而還。蒙遜西至苕藋,遣前將軍沮渠成都將
 騎五千襲卑和虜,蒙遜率中軍三萬繼之,卑和虜率眾迎降。遂循海而西,至鹽池,祀西王母寺。寺中有《玄石神圖》,命其中書侍郎張穆賦焉,銘之於寺前,遂如金山而歸。



 蒙遜下書曰:「頃自春炎旱,害及時苗,碧原青野,倏為枯壤。將刑政失中,下有冤獄乎?役繁賦重,上天所譴乎?內省多缺,孤之罪也。《書》不云乎:『百姓有過,罪予一人。』可大赦殊死已下。」翌日而澍雨大降。



 蒙遜聞劉裕滅姚泓,怒甚。門下校郎劉祥言事於蒙遜,蒙遜曰:「汝聞劉裕入關,敢研研然也!」遂殺之。其峻暴如是。顧謂左右曰:「古之行師,不犯歲鎮所在。姚氏舜後,軒轅之苗裔也。今鎮星
 在軒轅,而裕滅之,亦不能久守關中。」



 蒙遜為李士業敗於解支澗,復收散卒欲戰。前將軍成都諫曰:「臣聞高祖有彭城之敗,終成大漢,宜旋師以為後圖。」蒙遜從之,城建康而歸。



 其群下上書曰:「設官分職,所以經國濟時;恪動官次,所以緝熙庶政。當官者以匪躬為務,受任者以忘身為效。自皇綱初震,戎馬生郊,公私草創,未遑舊式。而朝士多違憲制,不遵典章;或公文御案,在家臥署;或事無可否,望空而過。至今黜陟絕於皇朝,駮議寢於聖世,清濁共流,能否相雜,人無勸競之心,茍為度日之事。豈憂公忘私,奉上之道也!今皇化曰隆,遐邇寧泰,宜
 肅振綱維,申修舊則。」蒙遜納之,命征南姚艾、尚書左丞房晷撰朝堂制。行之旬日,百僚振肅。



 太史令張衍言於蒙遜曰:「今歲臨澤城西當有破兵。」蒙遜乃遣其世子政德屯兵若厚塢。蒙遜西至白岸,謂張衍曰:「吾今年當有所定,但太歲在申,月又建申,未可西行。且當南巡,要其歸會,主而勿客,以順天心。計在臨機,慎勿露也。」遂攻浩亹,而蛇盤於帳前。蒙遜笑曰:「前一為騰蛇,今盤在吾帳,天意欲吾回師先定酒泉。」燒攻具而還,次于川巖。聞李士業徵兵欲攻張掖,蒙遜曰:「入吾計矣。但恐聞吾迴軍,不敢前也。兵事尚權。」乃露布西境,稱得浩亹,將進軍黃
 谷。士業聞而大悅,進入都瀆澗。蒙遜潛軍逆之,敗士業於壞城,遂進剋酒泉。百姓安堵如故,軍無私焉。以子茂虔為酒泉太守,士業舊臣皆隨才擢敘。



 蒙遜以安帝隆安五年自稱州牧,義熙八年僭立,後八年而宋氏受禪,以元嘉十年死,時年六十六,在偽位三十三年。子茂虔立,六年,為魏所擒,合三十九載而滅。



 史臣曰:蒙遜出自夷陬,擅雄邊塞。屬呂光之悖德,深懷仇粥之冤;推段業以濟時,假以陳、吳之事。稱兵白澗,南涼請和;出師丹嶺,北寇賓服。然而見利忘義,苞禍滅親,雖能制命一隅,抑亦備諸凶德者矣。



 贊曰:光猜人傑,業忌時賢。游飲自晦,匿智圖全。兇心既逞,偽績攸宣。挺茲奸數,馳競當年。



\end{pinyinscope}