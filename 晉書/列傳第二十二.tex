\article{列傳第二十二}

\begin{pinyinscope}

 郤詵阮種華譚袁甫



 郤詵,字廣基,濟陰單父人也。父晞,尚書左丞。詵博學多才,瑰偉倜儻,不拘細行,州郡禮命並不應。泰始中,詔天下舉賢良直言之士,太守文立舉詵應選。



 詔曰:「蓋太上以德撫時,易簡無文。至于三代,禮樂大備,制度彌繁。文質之變,其理何由?虞、夏之際,聖明係踵,而損益不同。周道既衰,仲尼猶曰從周。因革之宜,又何殊也?聖王既沒,
 遺制猶存,霸者迭興而翼輔之,王道之缺,其無補乎?何陵遲之不反也?豈霸德之淺歟?期運不可致歟?且夷吾之智,而功止於霸,何哉?夫昔人之為政,革亂亡之弊,建不刊之統,移風易俗,刑措不用,豈非化之盛歟?何修而嚮茲?朕獲承祖宗之休烈,于茲七載,而人未服訓,政道罔述。以古況今,何不相逮之遠也?雖明之弗及,猶思與群賢慮之,將何以辨所聞之疑昧,獲至論於讜言乎?加自頃戎狄內侵,災害屢作,邊氓流離,征夫苦役,豈政刑之謬,將有司非其任歟?各悉乃心,究而論之。上明古制,下切當今。朕之失德,所宜振補。其正議無隱,將敬聽之。」



 詵對曰:



 伏惟陛下以聖德君臨,猶垂意於博採,故招賢正之士,而臣等薄陋,不足以降大問也。是以竊有自疑之心,雖致身於闕庭,亦FC俯矣。伏讀聖策,乃知下問之旨篤焉。臣聞上古推賢讓位,教同德一,故易簡而人化;三代世及,季末相承,故文繁而後整。虞、夏之相因,而損益不同,非帝王之道異,救弊之路殊也。周當二代之流,承凋偽之極,盡禮樂之致,窮制度之理,其文詳備,仲尼因時宜而曰從周,非殊論也。臣聞聖王之化先禮樂,五霸之興勤政刑。禮樂之化深,政刑之用淺。勤之則可以小安,墮之則遂陵遲。所由之路本近,故所補之功不侔
 也。而齊桓失之葵丘,夷吾淪于小器,功止於霸,不亦宜乎!



 策曰:「建不刊之統,移風易俗,使天下洽和,何修而嚮茲?」臣以為莫大於擇人而官之也。今之典刑,匪無一統,宰牧之才,優劣異績,或以之興,或以之替,此蓋人能弘政非政弘人也。舍人務政,雖勤何益?臣竊觀乎古今,而考其美惡:古人相與求賢,今人相與求爵。古之官人,君責之於上,臣舉之於下,得其人有賞,失其人有罰,安得不求賢乎!今之官者,父兄營之,親戚助之,有人事則通,無人事則塞,安得不求爵乎!賢茍求達,達在修道,窮在失義,故靜以待之也。爵茍可求,得在進取,失在後時,故
 動以要之也。動則爭競,爭競則朋黨,朋黨則誣誷,誣誷則臧否失實,真偽相冒,主聽用惑,姦之所會也。靜則貞固,貞固則正直,正直則信讓,信讓則推賢,推賢不伐,相下無厭,主聽用察,德之所趣也。故能使之靜,雖日高枕而人自正;不能禁動,雖復夙夜,俗不一也。且人無愚智,咸慕名宦,莫不飾正於外,藏邪於內,故邪正之人難得而知也。任得其正,則眾正益至;若得其邪,則眾邪亦集。物繁其類,誰能止之!故亡國失世者,未嘗不為眾邪所積也。方其初作,必始於微,微而不絕,其終乃著。天地不能頓為寒暑,人主亦不能頓為隆替。故寒暑漸於春秋,
 隆替起於得失。當今之世,宦者無關梁,邪門啟矣;朝廷不責賢,正路塞矣。得失之源,何以甚此!所謂責賢,使之相舉也;所謂關梁,使之相保也。賢不舉則有咎,保不信則有罰。故古者諸侯必貢士,不貢者削,貢而不適亦削。夫士者,難知也;不適者,薄過也。不得不責,彊其所不知也;罰其所不適,深其薄過,非恕也。且天子於諸侯,有不純臣之義,斯責之矣。施行之道,寧縱不濫之矣。今皆反是,何也?夫賢者天地之紀,品物之宗,其急之也,故寧濫以得之,無縱以失之也。今則不然,世之悠悠者,各自取辨耳。故其材行並不可必,於公則政事紛亂。於私則污
 穢狼籍。自頃長吏特多此累,有亡命而被購懸者矣,有縛束而絞戮者矣。貪鄙竊位,不知誰升之者?獸兕出檻,不知誰可咎者?漏網吞舟,何以過此!人之於利,如蹈水火焉。前人雖敗,後人復起,如彼此無已,誰止之者?風流日競,誰憂之者?雖今聖思勞於夙夜,所使為政,恒得此屬,欲聖世化美俗平,亦俟河之清耳。若欲善之,宜創舉賢之典,峻關梁之防。其制既立,則人慎其舉而不茍,則賢者可知。知賢而試,則官得其人矣。官得其人,則事得其序;事得其序,則物得其宜;物得其宜,則生生豐植,人用資給,和樂興焉。是故寡過而遠刑,知恥以近禮,此
 所以建不刊之統,移風易俗,刑措而不用也。



 策曰:「自頃夷狄內侵,災眚屢降,將所任非其人乎?何由而至此?」臣聞蠻夷猾夏,則皋陶作士,此欲善其末,則先其本也。夫任賢則政惠,使能則刑恕。政惠則下仰其施,刑恕則人懷其勇。施以殖其財,勇以結其心。故人居則資贍而知方,動則親上而志勇。茍思其利而除其害,以生道利之者,雖死不貳;以逸道勞之者,雖勤不怨。故其命可授,其力可竭,以戰則剋,以攻則拔。是以善者慕德而安服,惡者畏懼而削迹。止戈而武,義實在文,唯任賢然後無患耳。若夫水旱之災,自然理也。故古者三十年耕必有十
 年之儲,堯、湯遭之而人不困,有備故也。自頃風雨雖頗不時,考之萬國,或境土相接,而豐約不同;或頃畝相連,而成敗異流,固非天之必害於人,人實不能均其勞苦。失之於人,而求之於天,則有司惰職而不勸,百姓殆業而咎時,非所以定人志,致豐年也。宜勤人事而已。



 臣誠愚鄙不足以奉對聖朝,猶進之于廷者,將使取諸其懷而獻之乎!臣懼不足也。若收不知言以致知言,臣則可矣,是以辭鄙不隱也。



 以對策上第,拜議郎。母憂去職。



 詵母病,苦無車,及亡,不欲車載柩,家貧無以市馬,乃於所住堂北壁外假葬,開戶,朝夕拜哭。養雞種蒜,竭其方術。
 喪過三年,得馬八匹,輿柩至冢,負土成墳。未畢,召為征東參軍。徙尚書郎,轉車騎從事中郎。



 吏部尚書崔洪薦詵為左丞。及在職,嘗以事劾洪,洪怨詵,詵以公正距之,語在《洪傳》。洪聞而慚服。



 累遷雍州刺史。武帝於東堂會送,問詵曰:「卿自以為何如?」詵對曰:「臣舉賢良對策,為天下第一,猶桂林之一枝,崑山之片玉。」帝笑。侍中奏免詵官,帝曰:「吾與之戲耳,不足怪也。」詵在任威嚴明斷,甚得四方聲譽。卒於官。子延登為州別駕。



 阮種,字德猷,陳留尉氏人,漢侍中胥卿八世孫也。弱冠
 有殊操,為嵇康所重。康著《養生論》,所稱阮生,即種也。察孝廉,為公府掾。是時西虜內侵,災眚屢見,百姓饑饉,詔三公、卿尹、常伯、牧守各舉賢良方正直言之士。於是太保何曾舉種賢良。



 策曰:「在昔哲王,承天之序,光宅宇宙,咸用規矩乾坤,惠康品類,休風流衍,彌于千載。朕應踐洪運統位,七載於今矣。惟德弗嗣,不明于政,宵興惕厲,未燭厥猷。子大夫韞韥道術,儼然而進,朕甚嘉焉。其各悉乃心,以闡喻朕志,深陳王道之本,勿有所隱,朕虛心以覽焉。」種對曰:「夫天地設位,聖人成能,王道至深,所以行化至遠。故能開物成務,而功業不匱,近無不聽,遠無
 不服,德逮群生,澤被區宇,聲施無窮,而典垂百代。故《經》曰:『聖人久於其道,而天下化成。』宜師蹤往代,襲迹三五,矯世更俗,以從人望。令率士遷義,下知所適,播醇美之化,杜邪枉之路,斯誠群黎之所欣想盛德而幸望休風也。」



 又問政刑不宣,禮樂不立。對曰:「政刑之宣,故由乎禮樂之用。昔之明王,唯此之務,所以防遏暴慢,感動心術,制節生靈,而陶化萬姓也。禮以體德,樂以詠功,樂本於和,而禮師於敬矣。」



 又問戎蠻猾夏。對曰:「戎蠻猾夏,侵敗王略,雖古盛世,猶有此虞。故《詩》稱『獫狁孔熾』,《書》歎『蠻夷帥服』。自魏氏以來,夷虜內附,鮮有桀悍侵漁之患。由是
 邊守遂怠,鄣塞不設。而今醜虜內居,與百姓雜處,邊吏擾習,人又忘戰。受方任者,又非其材,或以狙詐,侵侮邊夷;或干賞啗利,妄加討戮。夫以微羈而御悍馬,又乃操以煩策,其不制者,固其理也。是以群醜蕩駭,緣間而動。雖三州覆敗,牧守不反,此非胡虜之甚勁,蓋用之者過也。臣聞王者之伐,有征無戰,懷遠以德,不聞以兵。夫兵凶器,而戰危事也。兵興則傷農,眾集則費積;農傷則人匱,積費則國虛。昔漢武之世,承文帝之業,資海內之富,役其材臣,以甘心匈奴,競戰勝之功,貪攻取之利,良將勁卒,屈於沙漠,勝敗相若,克不過當,夭百姓之命,填餓
 狼之口。及其以眾制寡,令匈奴遠迹,收功祁連,飲馬瀚海,天下之耗,已過太半矣。夫虛中國以事夷狄,誠非計之得者也。是以盜賊蜂起,山東不振。暨宣元之時,趙充國征西零,馮奉世征南羌,皆兵不血刃,摧抑彊暴,擒其首惡,此則折衝厭難,勝敗相辨,中世之明效也。」



 又問咎徵作見。對曰:「陰陽否泰,六沴之災,則人主修政以禦之,思患而防之,建皇極之首,詳庶徵之用。《詩》曰『敬之敬之,天惟顯思』,天聰明自我人聰明,是以人主祖承天命,日慎一日也。故能應受多福而永世克祚,此先王之所以退災消眚也。」



 又問經化之務。對曰:「夫王道之本,經國之
 務,必先之以禮義,而致人於廉恥。禮義立,則君子軌道而讓於善;廉恥立,則小人謹行而不淫於制度。賞以勸其能,威以懲其廢。此先王所以保乂定功,化洽黎元,而勛業長世也。故上有克讓之風,則下有不爭之俗;朝有矜節之士,則野無貪冒之人。夫廉恥之於政,猶樹藝之有豐壤,良歲之有膏澤,其生物必油然茂矣。若廉恥不存,而惟刑是御,則風俗凋弊,人失其性,錐刀之末,皆有爭心,雖峻刑嚴辟,猶不勝矣。其於政也,如農者之殖磽野,旱年之望豐穡,必不幾矣。此三代所以享德長久,風醇俗美,皆數百年保天之祿。而秦二世而弊者,蓋其所
 由之塗殊也。」



 又問:「將使武成七德,文濟九功,何路而臻于茲?凡厥庶事,曷後曷先?」對曰:「夫文武經德,所以成功丕業,咸熙庶績者,莫先於選建明哲,授方任能。令才當其官而功稱其職,則萬機咸理,庶僚不曠。《書》曰:『天工人其代之。』然則繼天理物,寧國安家,非賢無以成也。夫賢才之畜於國,由良工之須利器,巧匠之待繩墨也。器用利,則斲削易而材不病;繩墨設,則曲直正而眾形得矣。是以人主必勤求賢,而佚以任之也。賢臣之於主,進則忠國愛人,退則砥節潔志,營職不乾私義,出心必由公途,明度量以呈其能,審經制以效其功。此昔之聖王所
 以恭己南面而化於陶鈞之上者,以其所任之賢與所賢之信也。方今海內之士皆傾望休光,希心紫極,唯明主之所趣舍。若開四聰之聽,廣疇咨之求,抽群英,延俊乂,考工授職,呈能制官,朝無素餐之士,如此化流罔極,樹功不朽矣。」



 時種與郤詵及東平王康俱居上第,即除尚書郎。然毀譽之徒,或言對者因緣假託,帝乃更延群士,庭以問之。詔曰:「前者對策各指答所問,未盡子大夫所欲言,故復延見,其具陳所懷。又比年連有水旱災眚,雖戰戰兢兢,未能究天人之理,當何修以應其變?人遇水旱饑饉者,何以救之?中間多事,未得寧靜,思以省息
 煩務,令百姓不失其所。若人有所患苦者,有宜損益,使公私兩濟者,委曲陳之。又政在得人,而知之至難,唯有因人視聽耳。若有文武隱逸之士,各舉所知,雖幽賤負俗,勿有所限。故虛心思聞事實,勿務華辭,莫有所諱也。」



 種對曰:「伏惟陛下以聖哲玄覽,降血阜黎蒸,將濟元元,同之三代,旁求俊乂,以輔至化,此誠堯、舜之用心也。臣猥以頑魯之質,應清明之舉,前者對策,不足以疇塞聖詔,所陳不究,臣誠蒙昧,所以為罪。臣聞天生蒸庶,樹君以司牧之,人君道洽,則彞倫攸序,五福來備。若政有愆失,刑理頗僻,則庶徵不應,而淫亢為災。此則天人之理,而
 興廢之由也。昔之聖王,政道備而制先具,軌人以務,致之於本,是以雖有水旱之眚,而無饑饉之患也。自頃陰陽隔并,水旱為災,亦猶期運之致。不然,則亦有司之不帥,不能宣承聖德,以贊揚大化,故和氣未降而人事未敘也。方今百姓凋弊,公私無儲,誠在於休役靜人,勸嗇務分,此其救也。人之所患,由於役煩網密而信道未孚也。役煩則百姓失業,網密則下背其誠,信道未孚則人無固志。此則損益之至務,安危之大端也。傳曰:『始與善,善進,則不善蔑由至。』孔子曰:『視其所以,觀其所由,人焉廋哉!』若夫文武隱逸之士,幽賤負俗之才,故非愚臣之
 所能識。謹竭愚以對。」



 策奏,帝親覽焉,又擢為第一。轉中書郎。進止有方,正已率下,朝廷咸憚其威容。每為駁議,事皆施用,遂為楷則。



 遷平原相。時襄邑衛京自南陽太守遷于河內,與種俱拜,帝望而歎曰:「二千石皆若此,朕何憂乎!」種為政簡惠,百姓稱之,卒於郡。



 華譚,字令思,廣陵人也。祖融,吳左將軍、錄尚書事。父住,吳黃門郎。譚期歲而孤,母年十八,便守節鞠養,動勞備至。及長,好學不倦,爽慧有口辯,為鄰里所重。揚州刺史周浚引為從事史,愛其才器,待以賓友之禮。



 太康中,刺
 史嵇紹舉譚秀才,將行,別駕陳總餞之,因問曰:「思賢之主以求才為務,進取之士以功名為先,何仲舒不仕武帝之朝,賈誼失分漢文之時?此吳、晉之滯論,可辨此理而後別。」譚曰:「夫聖人在上,物無不理,百揆之職,非賢不居。故山林無匿景,衡門不棲遲。至承統之王,或是中才,或復凡人,居聖人之器,處兆庶之上,是以其教日頹,風俗漸弊。又中才之君,所資者偏,物以類感,必於其黨,黨言雖非,彼以為是。以所授有顏、冉之賢,所用有廊廟之器,居官者日冀元凱之功,在上者日庶堯、舜之義,彼豈知其政漸毀哉!朝雖有求賢之名,而無知才之實。言雖
 當,彼以為誣;策雖奇,彼以為妄。誣則毀己之言入,妄則不忠之責生,豈故為哉?淺明不見深理,近才不睹遠體也。是以言不用,計不施,恐死亡之不暇,何論功名之立哉!故上官暱而屈原放,宰嚭寵而伍員戮,豈不哀哉!若仲舒抑於孝武,賈誼失於漢文,蓋復是其輕者耳。故白起有云:『非得賢之難,用之難。非用之難,信之難。』得賢而不能用,用而不能信,功業豈可得而成哉!」



 譚至洛陽,武帝親策之曰:「今四海一統,萬里同風,天下有道,莫斯之盛。然北有未羈之虜,西有醜施之氐,故謀夫未得高枕,邊人未獲晏然,將何以長弭斯患,混清六合?」對曰:「臣聞
 聖人之臨天下也,祖乾綱以流化,順谷風以興仁,兼三才以御物,開四聰以招賢。故勞謙日昃,務在擇才,宣明巖穴,垂光隱滯。俊乂龍躍,帝道以光;清德風翔,王化克舉。是以皋陶見舉,不仁者遠;陸賈重漢,遠夷折節。今聖朝德音發於帷幄,清風翔乎無外,戎旗南指,江、漢席卷;干戈西征,羌蠻慕化,誠闡四門之秋,興禮教之日也。故髦俊聞聲而響赴,殊才望險而雲集。虛高館以俟賢,設重爵以待士,急善過於饑渴,用人疾於影響,杜佞諂之門,廢鄭聲之樂,混清六合,實由乎此。雖西北有未羈之寇,殊漠有不朝之虜,征之則勞師,得之則無益,故班固
 云:『有其地不可耕而食,得其人不可臣而畜,來則懲而禦之,去則備而守之。』蓋安邊之術也。」



 又策曰:「吳、蜀恃險,今既蕩平。蜀人服化,無攜貳之心;而吳人趑雎,屢作妖寇。豈蜀人敦樸,易可化誘;吳人輕銳,難安易動乎?今將欲綏靜新附,何以為先?」對曰:「臣聞漢末分崩,英雄鼎峙,蜀棲岷隴,吳據江表。至大晉龍興,應期受命,文皇運籌,安樂順軌;聖上潛謀,歸命向化。蜀染化日久,風教遂成;吳始初附,未改其化,非為蜀人敦愨而吳人易動也。然殊俗遠境,風土不同,吳阻長江,舊俗輕悍。所安之計,當先籌其人士,使雲翔閶闔,進其賢才,待以異禮;明選牧
 伯,致以威風;輕其賦斂,將順咸悅,可以永保無窮,長為人臣者也。」



 又策曰:「聖人稱如有王者,必世而後仁。今天成地平,大化無外,雖匈奴未羈,羌、氐驕黠,將修文德以綏之,舞干戚以來之,故兵戈載戢,武夫寢息。如此,已可消鋒刃為佃器,罷尚方武庫之用未邪?」對曰:「夫唐堯歷載,頌聲乃作;文、武相承,禮樂大同。清一八紘,綏盪無外,萬國順軌,海內斐然。雖復被髮之鄉,徒跣之國,皆習章甫而入朝,要衣裳以磬折。夫大舜之德,猶有三苗之徵;以周之盛,獫狁為寇。雖有文德,又須武備。備預不虞,古之善教;安不忘危,聖人常誡。無為罷武庫之常職,鑠鋒
 刃為佃器。自可倒戢干戈,苞以獸皮,將帥之士,使為諸侯,於散樂休風,未為不泰也。」



 又策曰:「夫法令之設,所以隨時制也。時險則峻法以取平,時泰則寬網以將化。今天下太平,四方無事,百姓承德,將就無為而乂。至於律令,應有所損益不?」對曰:「臣聞五帝殊禮,三王異教,故或禪讓以光政,或干戈以攻取。至於興禮樂以和人,流清風以寧俗,其歸一也。今誠風教大同,四海無虞,人皆感化,去邪從正。夫以堯、舜之盛,而猶設象刑;殷、周之隆,而甫侯制律。律令之存,何妨於政。若乃大道四達,禮樂交通,凡人修行,黎庶勵節,刑罰懸而不用,律令存而無施,
 適足以隆太平之雅化,飛仁風乎無外矣。」



 又策曰:「昔帝舜以二八成功,文王以多士興周。夫制化在於得人,而賢才難得。今大統始同,宜搜才實。州郡有貢薦之舉,猶未獲出群卓越之倫。將時無其人?有而致之未得其理也?」對曰:「臣聞興化立法,非賢無以光其道;平世理亂,非才無以宣其業。上自皇羲,下及帝王,莫不張皇綱以羅遠,飛仁風以被物。故得賢則教興,失人則政廢。今四海一統,萬里同風,州郡貢秀孝,臺府簡良才,以八紘之廣,兆庶之眾,豈當無卓越俊逸之才乎!譬猶南海不少明月之寶,大宛不乏千里之駒也。異哲難見,遠數難睹,故
 堯、舜太平之化,二八由舜而甫顯,殷湯革王之命,伊尹負鼎而方用。當今聖朝禮亡國之士,接遐裔之人,或貂蟬於帷幄,或剖符於千里,巡狩必有呂公之遇,宵夢必有巖穴之感。賢俊之出,可企踵而待也。」



 時九州秀孝策無逮譚者。譚素以才學為東土所推。同郡劉頌時為廷尉,見之歎息曰:「不悟鄉里乃有如此才也!」博士王濟於眾中嘲之曰:「五府初開,群公辟命,採英奇於仄陋,拔賢俊於巖穴。君吳、楚之人,亡國之餘,有何秀異而應斯舉?」譚答曰:「秀異固產於方外,不出於中域也。是以明珠文貝,生於江、鬱之濱;夜光之璞,出乎荊、藍之下。故以人求
 之,文王生於東夷,大禹生於西羌。子弗聞乎?昔武王剋商,遷殷頑民于洛邑,諸君得非其苗裔乎?」濟又曰:「夫危而不持,顛而不扶,至於君臣失位,國亡無主,凡在冠帶,將何所取哉!」答曰:「吁!存亡有運,興衰有期,天之所廢,人不能支。徐偃修仁義而失國,仲尼逐魯而逼齊,段幹偃息而成名,諒否泰有時,曷人力之所能哉!」濟甚禮之。



 尋除郎中,遷太子舍人、本國中正。以母憂去職。服闋,為鄄城令,過濮水,作《莊子贊》以示功曹。而廷掾張延為作答教,其文甚美。譚異而薦之,遂見升擢。及譚為廬江,延已為淮陵太守。又舉寒族周訪為孝廉,訪果立功名,時以
 譚為知人。以父墓毀去官。尋除尚書郎。



 永寧初,出為郟令。于時兵亂之後,境內饑饉,譚傾心撫血阜。司徒王戎聞而善之,出穀三百斛以助之。譚甚有政績,再遷廬江內史,加綏遠將軍。時石冰之黨陸珪等屯據諸縣,譚遣司馬褚敦討平之。又遣別軍擊冰都督孟徐,獲其驍率。以功封都亭侯,食邑千戶,賜絹千匹。



 陳敏之亂,吳士多為其所逼。顧榮先受敏官,而潛謀圖之。譚不悟榮旨,露檄遠近,極言其非,由此為榮所怨。又在郡政嚴,而與上司多忤。揚州刺史劉陶素與譚不善,因法收譚,下壽陽獄。鎮東將軍周馥與譚素相親善,理而出之。及甘卓討馥,
 百姓奔散,馥謂譚已去,遣人視之,而更移近馥。馥歎曰:「吾嘗謂華令思是臧子源之疇,今果效矣。」甘卓嘗為東海王越所捕,下令敢有匿者誅之,卓投譚而免。及此役也,卓遣人求之曰:「華侯安在?吾甘揚威使也。」譚答不知,遺絹二匹以遣之。使反,告卓。卓曰:「此華侯也。」復求之,譚已亡矣。後為紀瞻所薦,而為顧榮所止遏,遂數年不得調。



 建興初,元帝命為鎮東軍諮祭酒。譚博學多通,在府無事,乃著書三十卷,名曰《辨道》,上箋進之,帝親自覽焉。轉丞相軍諮祭酒,領郡大中正。譚薦干寶、范珧於朝,乃上箋求退曰:「譚聞霸主遠聽,以求才為務;僚屬量身,以審
 己為分。故疏廣告老,漢宣不違其志;干木偃息,文侯就式其廬。譚無古人之賢,竊有懷遠之慕。自登清顯,出入二載,執筆無贊事之功,拾遺無補闕之績;過在納言,闇於舉善;狂寇未賓,復乏謀策。年向七十,志力日衰,素餐無勞,實宜辭退。謹奉還所假左丞相軍諮祭酒版。」不聽。



 建武初,授祕書監,固讓不拜。太興初,拜前軍,以疾復轉祕書監。自負宿名,恒怏怏不得志。時晉陵朱鳳、吳郡吳震並學行清修,老而未調,譚皆薦為著作佐郎。



 或問譚曰:「諺言人之相去,如九牛毛,寧有此理乎?」譚對曰:「昔許由、巢父讓天子之貴,市道小人爭半錢之利,此之相去,
 何啻九牛毛也!」聞者稱善。



 戴若思弟邈,則譚女婿也。譚平生時常抑若思而進邈,若思每銜之。殆用事,恒毀譚於帝,由是官塗不至。譚每懷觖望,嘗從容言於帝曰:「臣已老矣,將待死秘閣。汲黯之言,復存於今。」帝不懌。久之,加散騎常侍,屢以疾辭。及王敦作逆,譚疾甚,不能入省,坐免。卒於家。贈光祿大夫,金章紫綬,加散騎常侍,謚曰胡。二子:化、茂。



 化字長風,為征虜司馬,討汲桑,戰沒。茂嗣爵。



 淮南袁甫,字公胄,亦好學,與譚齊名,以詞辯稱。嘗詣中
 領軍何勖,自言能為劇縣。勖曰:「唯欲宰縣,不為臺閣職,何也?」甫曰:「人各有能有不能。譬繒中之好莫過錦,錦不可以為;穀中之美莫過稻,稻不可以為贇。是以聖王使人,必先以器,茍非周材,何能悉長!黃霸馳名於州郡,而息譽於京邑。廷尉之材,不為三公,自昔然也。」勖善之,除松滋令。轉淮南國大農、郎中令。石珩問甫曰:「卿名能辯,豈知壽陽已西何以恒旱?壽陽已東何以恒水?」甫曰:「壽陽已東皆是吳人,夫亡國之音哀以思,鼎足強邦,一朝失職,憤歎甚積,積憂成陰,陰積成雨,雨久成水,故其域恒澇也。壽陽已西皆是中國,新平彊吳,美寶皆入,志
 盈心滿,用長歡娛。《公羊》有言,魯僖甚悅,故致旱京師。若能抑彊扶弱,先疏後親,則天下和平,災害不生矣。」觀者歎其敏捷。年八十餘,卒於家。



 史臣曰:夫緝政釐俗,拔群才以成務;振景觀光,俟明主而宣績。武皇之世,天下乂安,朝廷屬意於求賢,軸有懷於干祿。郤詵等並韞價州里,裒然應召,對揚天問,高步雲衢,求之前哲,亦足稱矣。令思行己徇義,志篤周、甘,仁者必通,抑斯之謂!雖才行夙章,而待終祕閣,積薪之恨,豈獨古人乎!



 贊曰:郤、阮洽聞,含章體政。華生毓德,褫巾應命。鳥路曾
 飛,龍津派泳。素業可久,高芬斯盛。



\end{pinyinscope}