\article{志第三}

\begin{pinyinscope}
天文下
 \gezhu{
  月五星犯列舍經星變附見妖星客星星流隕雲氣}



 ○月五星犯列舍經星變附見



 魏文帝黃初四年三月癸卯,月犯心大星。占曰:「心為天王位,王者惡之。」六月甲申,太白晝見。案劉向《五紀論》曰:「太白少陰,弱,不得專行,故以己未為界,不得經天而行。經天則晝見,其占為兵喪,為不臣,為更王;彊國弱,小國彊。」
 是時孫權受魏爵號,而稱兵距守。其十二月丙子,月犯心大星。占同上。五年十月乙卯,太白晝見。占同上。又歲星入太微逆行,積百四十九日乃出。占曰:「五星入太微,從右入三十日以上,人主有大憂。」一曰:「有赦至。」七年五月,帝崩,明帝即位,大赦天下。六年五月壬戌,熒惑入太微,至壬申,興歲星相及,俱犯右執法,至癸酉乃出。占曰:「從右入三十日以上,人主有大憂。」又曰:「月、五星犯左右執法,大臣有憂。」一曰:「執法者誅,金、火尤甚。」十一月,皇子東武陽王鑒薨。七年正月,驃
 騎將軍曹洪免為庶人。四月,征南大將軍夏侯尚薨。五月,帝崩。《蜀記》稱明帝問黃權曰:「天下鼎立,何地為正?」對曰:「當驗天文。往者熒惑守心而文帝崩,吳、蜀無事,此其徵也。」案三國史並無熒惑守心之文,疑是入太微。八月,吳遂圍江夏,寇襄陽,大將軍宣帝救襄陽,斬吳將張霸等,兵喪更王之應也。



 明帝太和五年五月,熒惑犯房。占曰:「房四星,股肱臣將相位也,月、五星犯守之,將相有憂。」其七月,車騎將軍張郃追諸葛亮,為亮所害。十二月,太尉華歆薨。其十一月乙酉,月犯軒轅大星。占曰:「女主憂。」
 六年三月乙亥,月又犯軒轅大星。十一月寅,太白晝見南斗,遂歷八十餘日,恒見。占曰:「吳有兵。」明年,孫權遣張彌等將兵萬人,錫授公孫文懿為燕王,文懿斬彌等,虜其眾。青龍三年正月,太后郭氏崩。



 青龍三年三月辛卯,月犯輿鬼。輿鬼主斬殺。占曰:「人多病,國有憂。」又曰:「大臣憂。」是年夏及冬,大疫。四年五月,司徒董昭薨。其五月丁亥,太白晝見,積三十餘日。以晷度推之,非秦魏,則楚也。是時,諸葛亮據渭南,宣帝與相持;孫權寇合肥,又遣陛議、孫韶等人淮沔,天子親東征。蜀本秦地,則為秦魏及楚兵悉起矣。其七月己巳,月犯楗
 閉。占曰:「有火災。」三年七月,崇華殿災。三年六月丁未,填星犯井鉞。戊戌,太白又犯之。占曰:「凡月、五星犯井鉞,悉為兵災。」一曰:「斧鉞用,大臣誅。」七月己丑,填星犯東井距星。占曰:「填星入井,大人憂。」行近距,為行陰。其占曰:「大水,五穀不成。」景初元年夏,大水,傷五穀。其年十月壬申,太白晝見,在尾,歷二百餘日,恒晝見。占曰:「尾為燕,有兵。」十二月戊辰,月犯鉤鈐。占曰:「王者憂。」四年閏正月己巳,填星犯井鉞。三月癸卯,填星犯東井。己巳,太白與月加景晝見。五月壬寅,太白犯畢左股第一星。占曰:「畢為邊兵,又主刑罰。」九月,涼州塞外胡阿畢
 師使侵犯諸國,西域校尉張就討之,斬首捕虜萬計。其年七月甲寅,太白犯軒轅大星。占曰:「女主憂。」景初元年,皇后毛氏崩。



 景初元年二月乙酉,月犯房第二星。占曰:「將軍有憂。」其七月,司徒陳矯薨。二年四月,司徒韓暨薨。其七月辛卯,太白晝見,積二百八十餘日。時公孫文懿自立為燕王,署置百官,發兵距守,宣帝討滅之。二年二月己丑,月犯心距星,又犯中央大星。五月乙亥,月又犯心距星及中央大星。案占曰:「王者惡之。犯前星,太子有憂。」三年正月,帝崩。太子立,卒見廢。其年十月甲
 午,月犯箕。占曰:「將軍死。」正始元年四月,車騎將軍黃權薨。其閏十一月癸丑,月犯心中央大星。



 少帝正始元年四月戊午,月犯昴東頭第一星。十月庚寅,月又犯昴北斗四星。占曰:「月犯昴,胡不安。」二年六月,鮮卑阿妙兒等寇西方,敦煌太守王延破之,斬二萬餘級。三年,又斬鮮卑大師及千餘級。二年九月癸酉,月犯輿鬼西北星。三年二月丁未,又犯西南星。占曰:「有錢令。」一曰:「大臣憂。」三年三月,太尉滿寵薨。四年正月,帝加元服,賜群臣錢各有差。四年十月、十一月,月再犯井鉞。是月,宣帝討諸葛恪,恪
 棄城走。五年二月,曹爽征蜀。五年十一月癸巳,填星犯亢距星。占曰:「諸侯有失國者。」七年七月丁丑,月犯左角。占曰:「天下有兵,左將軍死。」七月乙亥,熒惑犯畢距星。占曰;「有邊兵。」一曰:「刑罰用。」九年正月辛亥,月犯亢南星。占曰:「兵起。」一曰:「將軍死。」七月癸丑,填星犯楗閉。占曰:「王者不宜出宮下殿。」嘉平元年,天子謁陵,宣帝奏誅曹爽等。天子野宿,於是失勢。



 嘉平元年六月壬戌,太白犯東井距星。占曰:「國失政,大臣為亂。」四月辛巳,太白犯輿鬼。占曰:「大臣誅。」一曰:「兵起。」二年三月己未,太白又犯井距星。三年七月,王凌與楚
 王彪有謀,皆伏誅,人主遂卑。



 吳孫權赤烏十三年夏五月,日北至,熒惑逆行,入南斗。秋七月,犯魁第三星而東。《漢晉春秋》云「逆行」。案占:「熒惑入南斗,三月吳王死。」一曰:「熒惑逆行,其地有死君。」太元二年,權薨,是其應也,故《國志》書於吳。是時,王凌謀立楚王彪,謂「斗中有星,當有暴貴者」,以問知星人浩詳。詳疑有故,欲悅其意,不言吳有死喪,而言「淮南楚分,吳楚同占,當有王者興」,故凌計遂定。



 嘉平二年十二月丙申,月犯輿鬼。三年四月戊寅,月犯東井。五月甲寅,月犯亢距星。占曰:「
 將軍死。」一曰:「為兵。」是月,王凌、楚王彪等誅。七月,皇后甄氏崩。四年三月,吳將為寇,鎮東將軍諸葛誕破走之。其年七月己巳,月犯輿鬼。九月乙巳,又犯之。十月癸未,熒惑犯亢南星。占曰:「臣有亂。」四年十一月丁未,月又犯鬼積尸。五年六月戊午,太白犯角。占曰:「群臣有謀,不成。」庚辰,月犯箕星。占曰:「將軍死。」七月,月犯井鉞。丙午,月又犯鬼西北星。占曰:「國有憂。」十一月癸酉,月犯東井距星。占曰:「將軍死。」正元元年正月,鎮東將軍毋丘儉、揚州刺史文欽反,兵俱敗,誅死。二月,李豐及弟翼、后父張緝等謀亂,事泄,悉誅,皇后張氏廢。九月,帝廢為齊王。蜀將姜維
 攻隴西,車騎將軍郭淮討破之。



 高貴鄉公正元二年二月戊午,熒惑犯東井北轅西頭第一星。甘露元年七月乙卯,熒惑犯東井鉞星。壬戌,月又犯鉞星。八月辛亥,月犯箕。



 吳廢孫亮太平元年九月壬辰,太白犯南斗,《吳志》所書也。占曰:「太白犯斗,國有兵,大臣有反者。」其明年,諸葛誕反。又明年,孫綝廢亮。吳魏並有兵事也。



 甘露元年九月丁巳,月犯東井。二年六月己酉,月犯心中央大星。八月壬子,歲星犯井鉞。九月庚寅,歲星逆行,乘井鉞。十月丙寅,太白犯亢距星。占曰:「逆臣為亂,人君
 憂。」景元元年五月,有成濟之變及諸葛誕誅,皆其應也。二年三月庚子,太白犯東井。占曰:「國失政,大臣為亂。」是夜,歲星又犯東井。占曰:「兵起。」至景元元年,高貴鄉公敗。三年八月壬辰,歲星犯輿鬼鑕星。占曰:「斧鑕用,大臣誅。」四年四月甲申,歲星又犯輿鬼東南星。占曰:「鬼東南星主兵,木入鬼,大臣誅。」景元元年,殺尚書王經。



 元帝景元元年二月,月犯建星。案占:「月五星犯建星,大臣相譖。」是後鐘會、鄧艾破蜀,會譖艾。二年四月,熒惑入太微,犯右執法。占曰:「人主有大憂。」一云:「大臣憂。」
 四年十月,歲星守房。占曰:「將相憂。」一云:「有大赦。」明年,鄧艾、鐘會皆夷滅,赦蜀土。五年,帝遜位。



 武帝咸寧四年九月,太白當見不見。占曰:「是謂失舍,不有破軍,必有亡國。」是時羊祜表求伐吳,上許之。五年十一月,兵出,太白始夕見西方。太康元年三月,大破吳軍,孫皓面縛請罪,吳國遂亡。



 太康八年三月,熒惑守心。占曰:「王者惡之。」太熙元年四月乙酉,帝崩。



 惠帝元康三年四月,熒惑守太微六十日。占曰:「諸侯三公謀其上,必有斬臣。」一曰:「天子亡國。」是春太白守畢,至
 是百餘日。占曰:「有急令之憂。」一曰:「相死。」又為邊境不安。後賈后陷殺太子。六年十月乙未,太白晝見。九年六月,熒惑守心。占曰:「王者惡之。」八月,熒惑入羽林。占曰:「禁兵大起。」其後,帝見廢為太上皇,俄而三王起兵討趙王倫,倫悉遣中軍兵相距累月。



 永康元年三月,中台星坼,太白晝見。占曰:「台星失常,三公憂。太白晝見,為不臣。」是月,賈后殺太子,趙王倫尋廢殺后,斬司空張華。其五月,熒惑入南斗。占曰:「宰相死,兵大起。斗,又吳分野。」是時,趙王倫為相,明年,篡位,三王興
 師誅之。太安二年,石冰破揚州。其八月,熒惑入箕。占曰:「人主失位,兵起。」明年趙王倫篡位,改元。二年二月,太白出西方,逆行入東井。占曰:「國失政,大臣為亂。」是時,齊王冏起兵討趙王倫,倫滅,冏擁兵不朝,專權淫奢,明年,誅死。



 永寧元年,自正月至于閏月,五星互經天,縱橫無常。《星傳》曰:「日陽,君道也;星陰,臣道也。日出則星亡,臣不得專也。晝而星見午上者為經天,其占『為不臣,為更王』。」今五星悉經天,天變所未有也。石氏說曰:「辰星晝見,其國不亡則大亂。」是後,台鼎方伯,互執大權,二帝流亡,遂至六
 夷更王,迭據華夏,亦載籍所未有也。其四月,歲星晝見。五月,太白晝見。占同前。七月,歲星守虛危。占曰:「木守虛危,有兵憂。虛危,齊分。」一曰:「守虛,饑;守危,徭役煩多,下屈竭。」辰星入太微,占曰「為內亂」,一曰「群臣相殺」。太白守右掖門,占曰:「為兵,為亂,為賊。」八月戊午,填星犯左執法,又犯上相,占曰「上相憂」。熒惑守昴,占曰「趙魏有災」。辰星守輿鬼,占曰「秦有災」。九月丁未,月犯左角。占曰:「人主憂。」一曰:「左衛將軍死,天下有兵。」二年四月癸酉,歲星晝見。占曰:「為臣彊。」初,齊王冏定京都,因留輔政,遂專慠無君。是月,成都、河間檄長沙王乂
 討之,冏乂交戰,攻焚宮闕,冏兵敗,夷滅。又殺其兄上軍將軍寔以下二十餘人。太安二年,成都攻長沙,於是公私饑困,百姓力屈。



 太安二年二月,太白入昴。占曰:「天下擾,兵大起。」七月,熒惑入東井。占曰:「兵起,國亂。」是秋,太白守太微上將。占曰:「上將以兵亡。」是年冬,成都、河間攻洛陽。八月,長沙王奉帝出距二王。三年正月,東海王越執長沙王乂,張方又殺之。三年正月,熒惑入南斗,占同永康。七月,左衛將軍陳率眾奉帝伐成都,六軍敗績,兵偪乘輿。是時,天下盜賊
 群起,張昌尤盛。



 永興元年七月庚申,太白犯角、亢,經房、心,歷尾、箕。九月,入南斗。占曰:「犯角,天下大戰;犯亢,有大兵,人君憂;入房心,為兵喪;犯尾箕,女主憂。」一曰:「天下大亂。入南斗,有兵喪。」一曰:「將軍為亂。其所犯守,又兗、豫、幽、冀、揚州之分野。」是年七月,有蕩陰之役。九月,王浚殺幽州刺史和演,攻鄴,鄴潰,於是兗豫為天下兵衝。陳敏又亂揚土。劉元海、石勒、李雄等並起微賤,跨有州郡。皇后羊氏數被幽廢。皆其應也。二年四月丙子,太白犯狼星。占曰:「大兵起。」九月,歲星守
 東井。占曰:「有兵,井又秦分野。」是年,茍晞破公師籓,張方破范陽王猇,關西諸將攻河間王顒,顒奔走,東海王迎殺之。



 光熙元年四月,太白失行,自翼入尾、箕。占曰:「太白失行而北,是謂反生。不有破軍,必有屠城。」五月,汲桑攻鄴,魏郡太守馮嵩出戰,大敗,桑遂害東燕王騰,殺萬餘人,焚燒魏時宮室皆盡。其九月丁未,熒惑守心。占曰:「王者惡之。」己亥,填星守房、心。占曰:「填守房,多禍喪;守心,國內亂,天下赦。」是時,司馬越專權,終以無禮破滅,內亂之應也。十一月,帝崩,懷帝即位,大赦天下。



 懷帝永嘉元年十二月丁亥,星流震散。按劉向說,天官列宿,在位之象;其眾小星無名者,眾庶之類。此百官眾庶將流散之象也。是後天下大亂,百官萬姓,流移轉死矣。二年正月庚午,太白伏不見,二月庚子,始晨見東方,是謂當見不見,占同上條。其後破軍殺將,不可勝數,帝崩虜庭,中夏淪覆。三年正月庚子,熒惑犯紫微。占曰:「當有野死之王,又為火燒宮。」是時太史令高堂沖奏,乘興宜遷幸,不然必無洛陽。五年六月,劉曜、王彌入京都,焚燒宮廊,執帝歸平
 陽。三年,填星久守南斗。占曰:「填星所居久者,其國有福。」是時,安東將軍、瑯邪王始有揚土。其年十一月,地動,陳卓以為是地動應也。五年十月,熒惑守心。六年六月丁卯,太白犯太微。占曰:「兵入天子庭,王者惡之。」七月,帝崩於寇庭,天下行服大臨。



 元帝太興元年七月,太白犯南斗。占曰:「吳越有兵,大人憂。」二年二月甲申,熒惑犯東井。占曰:「兵起,貴臣相戮。」八月
 己卯,太白犯軒轅大星。占曰:「後宮憂。」三年五月己戊子,太白入太微,又犯上將星。占曰:「天子自將,上將誅。」九月,太白犯南斗。十月己亥,熒惑在東井,居五諸侯南,踟躕留積三十日。占曰:「熒惑守井二十日以上,大人憂。守五諸侯,諸侯有誅者。」永昌元年三月,王敦率江荊之眾來攻京都,六軍距戰,敗績,人主謝過而已。於是殺護軍將軍周顗、尚書令刁協、驃騎將軍戴若思。又,鎮北將軍劉隗出奔。四月,又殺湘州刺史譙王司馬承、鎮南將軍甘卓。閏十二月,帝崩。



 明帝太寧三年正月,熒惑逆行,入太微。占曰:「為兵喪,王
 者惡之。」閏八月,帝崩。後二年,蘇峻反,攻焚宮室,太后以憂偪崩,天子幽劫于石頭城,遠近兵亂,至四年乃息。



 成帝咸和六年正月丙辰,月入南斗。占曰:「有兵。」是月,石勒殺略婁、武進二縣人。明年,石勒眾又抄略南沙、海虞。其十一月,熒惑守胃昴。占曰:趙魏有兵。」八年七月,石勒死,石季龍自立。是時,雖二石僭號,而其彊弱常占於昴,不關太微、紫宮也。八年三月己巳,月入南斗。與六年占同。其年七月,石勒死,彭彪以譙,石生以長安,郭權以秦州並歸順。於是遣督護喬球率眾救彪,彪敗,球退。又,石季龍、石斌攻滅生、
 權。其七月,熒惑入昴。占曰:「胡王死。」一曰:「趙地有兵。」是月,石勒死,石季龍多所攻沒。八月,月又犯昴。占曰:「胡不安。」九年三月己亥,熒惑入輿鬼,犯積尸。占曰:「兵在西北,有沒軍死將。」六月、八月,月又犯昴。是時,石弘雖襲勒位,而石季龍擅威橫暴,十一月廢弘自立,遂幽殺之。



 咸康元年二月己亥,太白犯昴。占曰:「兵起,歲中旱。」四月,石季龍略騎至歷陽,加司徒王道大司馬,治兵列戍衝要。是時,石季龍又圍襄陽。六月,旱。其年三月丙戌,月入昴。占曰:「胡王死。」八月戊戌,熒惑入東井。占曰:「無兵,兵起;有兵,兵止。」十一月,月犯昴。
 二年正月辛亥,月犯房南第二星。八月,月又犯昴。九月庚寅,太白犯南斗,因晝見。占曰:「斗為宰相,又揚州分,金犯之,死喪之象。晝見,為不臣,又為兵喪。」其後,石季龍僭稱天王,發眾七萬;四年二月自隴西攻殺段遼于薊,又襲慕容皝於棘城,不克,皝擊破其將麻秋,并虜段遼殺之。三年七月己酉,月犯房上星。八月,熒惑入輿鬼,犯積尸。甲戌,月犯東井距星。九月戊子,月犯建星。四年四月己巳,太白晝見,在柳。占曰:「為兵,為不臣。」明年,石季龍大寇沔南,於是內外戒嚴。其五月戊戌,熒惑犯右執法。占曰:「大臣死,執政者憂。」九月,太白又犯右執法。
 案占:「五星災同,金火尤甚。」十一月戊子,太白犯房上星。占曰:「上相憂。」五年四月乙未,月犯畢距星。占曰:「兵起。」七月己酉,月犯房上星。占曰:「將相憂。」是月庚申,丞相王導薨,庾冰代輔政。八月,太尉郗鑒薨。又有沔南邾城之敗,百姓流亡萬餘家。六年正月,征西大將軍庾亮薨。六年三月甲辰,熒惑犯太微外將星。占曰:「上將憂。」四月丁丑,熒惑犯右執法。占曰:「執政者憂。」六月乙亥,月犯牽牛中央星。占曰:「大將憂。」是時,尚書今何充為執法,有譴,欲避其咎,明年求為中書今。其四月丙午,太白犯畢距
 星。占曰:「兵革起。」一曰:「女主憂。」六月乙卯,太白犯軒轅大星。占曰:「女主憂。」七年三月,皇后杜氏崩。七年三月壬午,月犯房。四月己丑,太白入輿鬼。五月,太白晝見。八月辛丑,月犯輿鬼。八年六月,熒惑犯房上第二星。占曰:「次相憂。」八月壬寅,月犯畢。占曰:「下犯上,兵革起。」十月,月又掩畢大星。占同上。其建元二年,車騎將軍庾冰薨。庾翼大發兵,謀伐石季龍,專制上流,朝廷憚之。



 康帝建元元年正月壬午,太白入昴。占曰:「趙地有兵。」又曰:「天下兵起。」四月乙酉,太白晝見。是年,石季龍殺其子
 邃,又遣將寇沒狄道,及屯薊東,謀慕容皝。二年,歲星犯天關。安西將軍庾翼與兄冰書曰:「歲星犯天關,占云關梁當分。比來江東無他故,江道亦不艱難,而石季龍頻年再閉關,不通信使,此復是天公憒憒,無皂白之徵也。」其閏月乙酉,太白犯斗。占曰:「為喪,天下受爵祿。」九月,帝崩,太子立,大赦,賜爵。



 穆帝永和元年正月丁丑,月入畢。占曰:「兵大起。」戊寅,月犯天關。占曰:「有亂臣更天子之法。」五月辛巳,太白晝見,在東井。占曰:「為臣彊,秦有兵。」六月辛丑,月入太微,犯屏西南星。占曰:「輔臣有免罷者。」七月、八月,月皆犯畢。占同
 上。己未,月犯輿鬼。占曰:「大臣有誅。」九月庚戌,月又犯畢。是年初,庾翼在襄陽。七月,翼疾將終,輒以子爰之為荊州刺史,代己任。爰之尋被廢。明年,桓溫又輒率眾伐蜀,執李勢,送至京都。蜀本秦地也。二年二月壬子,月犯房上星。四月丙戎,月又犯房上星。八月壬申,太白犯左執法。三年正月壬午,月犯南斗第五星。占曰:「將軍死,近臣去。」五月壬申,月犯南斗第四星,因入魁。占曰:「有兵。」一曰:「有大赦。」六月,月犯東井距星。占曰:「將軍死,國有憂。」戍戌,月犯五諸侯。占曰:「諸侯有誅。」九月庚寅,太白犯南斗第五
 星。占曰:「為喪,為兵。」四年七月丙申,太白犯左執法。甲寅,月犯房。丁巳,月入南斗,犯第二星。乙丑,太白犯左執法。占悉同上。十月甲辰,月犯亢。占曰:「兵起,將軍死。」十一月戊戌,月犯上將星。三年六月,大赦。是月,陳逵征壽春,敗而還。七月,氐蜀餘寇反,亂益土。九月,石季龍伐涼州。五年,征北大將軍褚裒卒。四年四月,太白入昴,是時,戎晉相侵,趙地連兵尤甚。七月,太白犯軒轅。占曰:「在趙,及為兵喪。」甲寅,月犯房。十月甲戊戌,月犯亢。占曰:「兵起,將軍死。」八月,石季龍太子宣殺
 第韜,宣亦死。其十一月戊戌,月犯上將星。五年正月,石季龍僭號稱皇帝,尋死。五年四月丁未,太白犯東井。占曰:「秦有兵。」九月戊戌,太白犯左角。占曰:「為兵。」十月,月犯昴。占曰:「胡有憂,將軍死。」是年八月,褚裒北征兵敗。十月,關中二十餘舉兵內附。石遵攻沒南陽。十一月,冉閔殺石遵,又盡殺胡十餘萬人,於是趙魏大亂。十二月,褚裒薨。八年,劉顯、苻健、慕容俊並僭號。殷浩北伐,敗績,見廢。六年二月辛酉,月犯心大星。占曰:「大人憂,又豫州分野也。」丁丑,月犯房。占曰:「將相憂。」六月已丑月犯昴。占同上。
 乙未,月犯五諸侯。占同上。七月壬寅,月始出西方,犯左角。占曰:「大將軍死。」一曰:「天下有兵。」丁未,月犯箕。占曰:「將軍死。」丙寅,熒惑犯鉞星。占曰:「大臣有誅。」八月辛卯,月犯左角。太白晝見,在南斗。月犯右執法。占並同上。是歲,司徒蔡謨免為庶人。七年二月,太白犯昴。占同上。三月乙卯,熒惑入輿鬼,犯積尸。占曰:「貴人有憂。」五月乙未,熒惑犯軒轅大星。占曰:「女主憂。」太白入畢口,犯左股。占曰:「將相當之。」六月乙亥,月犯箕。占曰:「國有兵。」丙子,月犯斗。丁丑,熒惑入太微,犯右執法。八月庚午,太白犯軒轅。戊子,太白犯右執法。占
 悉同上。七年,劉顯殺石祗及諸將帥,山東大亂,疾疫死亡。八年三月戊戌,月犯軒轅大星。癸丑,月入南斗,犯第二星。五月,月犯心星。六月癸酉,月犯房。七月壬子,歲星犯東井距星。占曰:「內亂兵起。」八月戊戌,熒惑入輿鬼。占曰:忠臣戮死。」丙辰,太白入南斗,犯第四星。占曰:「將為亂。」一曰:「丞相免。」九年二月乙巳,月入南斗,犯第三星。三月戊辰,月犯房。八月,歲星犯輿鬼東南星。占曰:「兵起。」是時,帝幼沖,母后稱制,將相有隙,兵革連起,慕容俊僭號稱燕王,攻伐不
 休。十年正月乙卯,月蝕昴星。占曰:「趙魏有兵。」癸酉,填星奄鉞星。占曰:「斧鉞用。」二月甲申,月犯心大星。占曰:「王者惡之。」七月庚午,太白晝見。晷度推之,災在秦鄭。九月辛酉,太白犯左執法。是時,桓溫擅命,朝臣多見迫脅。四月,溫伐苻健,破其嶢柳軍。十二月,慕容恪攻齊。十一年三月辛亥,月奄軒轅。占同上。四月庚寅,月犯牛宿南星。占曰:「國有憂。」八月己未,太白犯天江。占曰:「河津不通。」十二年六月庚子,太白晝見,在東井。占如上。己未,月犯
 鉞星。八月癸酉,月奄建星。九月戊寅,熒惑入太微,犯西蕃上將星。十一月丁丑,熒惑犯太微東蕃上相星。十二年十一月,齊城陷,執段龕,殺三千餘人。永和三年,鮮卑侵略河、冀。升平元年,慕容俊遂據臨漳,盡有幽、並、青、冀之地。緣河諸將奔散,河津隔絕。時權在方伯,九服交兵。



 升平元年四月壬子,太白入輿鬼。丁亥,月奄井南轅西頭第二星。占曰:「秦地有兵。」一曰:「將死。」六月戊戌,太白晝見,在軫。占同上。軫是楚分野。壬子,月犯畢。占曰:「為邊兵。」七月辛巳,熒惑犯天江。占曰:「河津不通。」十一月,歲星犯房。占曰:「豫州有災。」其年五月,苻堅殺苻生而立。十二月,
 慕容JI入屯鄴。二年八月,豫州刺史謝奕薨。二年二月辛卯,填星犯軒轅大星。占曰:「人主惡之。」甲午,月犯東井。六月辛酉,月犯房。十月己未,太白犯哭星。占曰:「有大哭泣。」三年正月壬辰,熒惑犯楗閉星。案占曰:「人主憂。」三月乙酉,熒惑逆行犯鉤鈐。案占:「王者惡之。」六月,太白犯東井。七月乙酉,熒惑犯天江。丙戌,太白犯輿鬼。占悉同上。戊子,月犯牽牛中央大星。占曰:「牽牛,天將也。犯中央大星,將軍死。」八月丁未,太白犯軒轅大星。甲子,月犯畢大星。占曰:「為邊兵。」一曰:下犯上。三年十月,諸葛攸舟軍入河,
 敗績。豫州刺史謝萬入潁,眾潰而歸,萬除名。十一月,司徒會稽王以郗曇、謝萬二鎮敗,求自敗,求自貶三等。四年正月,慕容俊死,子代立。慕容恪殺其尚書令陽騖等。四年正月乙亥,月犯牽牛中央大星。六月辛亥,辰星犯軒轅。占曰:「女主憂。」己未,太白入太微右掖門,從端門出。占曰:「貴奪勢。」一曰:「有兵。」又曰:「出端門,臣不臣。」八月戊申,太白犯氐。占曰:「國有憂。」丙辰,熒惑犯太微西蕃上將星。九月壬午,太白入南斗口,犯第四星。占曰:「為喪,有赦,天下受爵祿。」十二月甲寅,熒惑犯房。丙寅,太白晝見。庚寅,月犯健閉,占曰:「人君惡之。」
 五年正月乙巳,填星逆行,犯太微。五月壬寅,月犯太微。庚戌,月犯建星。占曰:「大臣相謀。」是時,殷浩敗績,卒致遷徙。其月辛亥,月犯牽牛宿。占曰:「國有憂。」六月癸亥,月犯氐東北星。占曰:「大將當之。」五年正月,北中郎將郗曇薨。五月,帝崩,哀帝立,大赦,賜爵,褚后失勢。七月,慕容恪攻冀州刺史呂護於野王,護奔滎陽。是時,桓溫以大眾次宛,聞護敗,乃退。五年六月癸酉,月奄氐東北星。占曰:「大將軍當之。」九月乙酉,月奄畢。占曰:「有邊兵。」十月丁未,月犯畢大星。占曰:「下犯上。」又曰:「有邊兵。」八月,范汪廢。隆和元年,慕容遣
 將寇河陰。



 哀帝興寧三年七月庚戌,月犯南斗。占曰:「女主憂。」歲星犯輿鬼。占曰:「人君憂。」十月,太白晝見,在亢。占曰:「亢為朝廷,有兵喪,為臣彊。」明年五月,皇后庾氏崩。



 海西太和二年正月,太白入昴。五年,慕容為苻堅所滅,又據司、冀、幽、并四州。六年閏月,熒惑守太微端門。占曰:「天子亡國。」又曰:「諸侯三公謀其上。」一曰:「有斬臣。」辛卯,月犯心大星。占曰:「王者惡之。」十一月,桓溫廢帝,並奏誅武陵王,簡文不許,溫乃徙之新安,皆臣彊之應也。



 簡文咸安元年十二月辛卯,熒惑逆行入太微,二年三月猶不退。占曰:「國不安,有憂。」是時,帝有桓溫之逼。二年五月丁未,太白犯天關。占曰:「兵起。」歲星形色如太白。占曰:「進退如度,姦邪息;變色亂行,主無福。歲星於仲夏當細小而不明,此其失常也。又為臣彊。」六月,太白晝見,在七星。乙酉,太白犯輿鬼。占曰:「國有憂。」七月,帝崩,桓溫以兵威擅權,將誅王坦之等,內外迫脅。又,庾希入京城,盧悚入宮,並誅滅之。



 孝武寧康元年正月戊申,月奄心大星。案上曰:「災不在王者,則在豫州。」一曰:「主命惡之。」三月丙午,月奄南斗第
 五星。占曰:「大臣憂,有死亡。」一曰:「將軍死。」七月,桓溫薨。九月癸巳,熒惑入太微。是時,女主臨朝,政事多缺。二年閏月己未,月奄牽牛南星。占曰:「左將軍死。」十二月甲申,太白晝見,在氐。氐,兗州分野。三年五月丙午,北中郎將五坦之薨。三年六月辛卯,太白犯東井。占曰:「秦地有兵。」九月戊申,熒惑奄左執法。占曰:「執法者死。」太元元年,苻堅破涼州。二年十月,尚書令王彪之卒。太元元年四月丙戌,熒惑犯南斗第三星。丙申,又奄第四星。占曰:「兵大起,中國饑。」一曰:「有赦。」八月癸酉,太白晝
 見,在氐。氐,兗州分野。九月,熒惑犯哭泣星,遂入羽林。占曰:「天子有哭泣事,中軍兵起。」十一月己未,月奄氐角。占曰:「天下有兵。」一曰:「國有憂。」二年二月,熒惑守羽林。占曰:「禁兵大起。」九月壬午,太白晝見,在角。角,兗州分野。升平元年五月,大赦。三年八月,秦人寇樊、鄧、襄陽、彭城。四年二月,襄陽陷,朱序沒。四月,魏興陷,賊聚廣陵、三河,眾五六萬。於是諸軍外次衝要,丹陽尹屯衛京都。六月,兗州刺史謝玄討賊,大破之。是時,中外連兵,比年荒儉。四年十一月丁巳,太白犯哭星。占曰:「天子有哭泣事。」
 五年七月丙子,辰星犯軒轅。占曰:「女主當之。」九月癸未,皇后王氏崩。六年九月丙子。太白晝見。七年十一月,太白又晝見,在斗。占曰:「吳有兵喪。」八年四月甲子,太白又晝見,在參。占曰:「魏有兵喪。」是月,桓沖徵沔漢,楊亮伐蜀,並拔城略地。八月,苻堅自將,號百萬,九月,攻沒壽陽。十月,劉牢之破苻堅將梁成,斬之,殺獲萬餘人。謝玄等又破苻堅於淝水,斬其弟融,堅大眾奔潰。九年六月,皇太后褚氏崩。八月,謝玄出屯彭城,經略中州矣。
 九年七月丙戌,太白晝見。十一月丁巳,又晝見。十年四月乙亥,又晝見於畢昴。占曰:「魏國有兵喪。」是時苻堅大眾奔潰,趙魏連兵相攻,堅為姚萇所殺。十一年三月戊申,太白晝見,在東井。占曰:「秦有兵,臣彊。」六月甲申,又晝見於輿鬼。占曰:「秦有兵。」時魏、姚萇、苻登連兵,相征不息。甲午,歲星晝見,在胃。占曰:「魯有兵,臣彊。」十二年,慕容垂寇東阿,翟遼寇河上,姚萇假號安定,苻登自立隴上,呂光竊據涼土。十二年六月癸卯,太白晝見,在柳。十月庚午,太白晝見,在斗。
 十三年正月丙戌,又晝見。十二月,熒惑在角亢,形色猛盛。占曰:「熒惑失其常,吏且棄其法,諸侯亂其政。」自是後,慕容垂、翟遼、姚萇、苻登、慕容永並阻兵爭彊。十四年正月,彭城妖賊又稱號於皇丘,劉牢之破滅之。三月,張道破合鄉,圍泰山,向欽之擊走之。是年,翟遼又攻沒滎陽,侵略陳項。于時政事多弊,君道陵遲矣。十四年四月乙巳,太白晝見於柳。六月辛卯,又晝見于翼。九月丙寅,又晝見於軫。十二月,熒惑入羽林。占並同上。十五年,翟遼掠司兗,眾軍累討不剋,慕容垂又跨略并、冀等州。七月,旱。八月,諸郡大水,兗州又蝗。
 十五年九月癸未,熒惑入太微。十月,太白入羽林。十六年四月癸卯朔,太白晝見。十一月癸巳,月奄心前星。占曰:「太子憂。」是時,太子常有篤疾。十七年七月丁丑,太白晝見。十月丁酉,又晝見。十八年六月,又晝見。十九年五月,又晝見于柳。六月辛酉,又晝見于輿鬼。九月,又見于軫。二十年六年,熒惑入天囷。占曰:「大饑。」七月丁亥,太白晝見,在太微。占曰:「太白入太微,國有憂。晝見為兵喪。」十二月己巳,月犯楗閉及東西咸。占曰:「楗閉司心腹喉舌,東
 西咸主陰謀。」二十一年二月壬申,太白晝見。三月癸卯,太白連晝見,在羽林。占曰:「有彊臣,有兵喪,中軍兵起。」三月,太白晝見于胃。占曰:「中軍兵起。」四月壬午,太白入天囷。占曰:「為饑。」六月,歲星犯哭泣星。占曰:「有哭泣事。」是年九月,帝崩。隆安元年,王恭等舉兵脅朝廷,於是內外戒嚴,殺王國寶以謝之。又連歲水旱,三方動,眾人飢。



 安帝隆安元年正月癸亥,熒惑犯哭泣星。占曰:「有哭泣事。」四月丁丑,太白晝見,在東井。占曰:「秦有兵喪。」六月,姚興攻洛陽,郗恢遣兵救之。冬姚萇死,子略代立。魏王圭
 即位於中山。其八月,熒惑守井鉞。占曰:「大臣有誅。」二年六月戊辰,攝提移度失常。歲星晝見,在胃,兗州分野。是年六月,郗恢遣鄭啟方等以萬人伐慕容寶於滑臺,敗而還。閏月,太白晝見,在羽林。丁丑,月犯東上相。三年五月辛酉,月又奄東上相。辛未,辰星犯軒轅大星。占悉同上。二年九月,庾楷等舉兵,表誅王愉等,於是內外戒嚴。三年六月,洛陽沒於冠。桓玄破荊州,雍州殺殷仲堪等。孫恩聚眾攻沒會稽,殺內史。四年六月辛酉,月犯哭泣星。五年正月,太白晝見。自去年十二月在斗晝見,至于是
 月乙卯。案占:「災在吳越。」七月癸亥,大角星散搖五色。占曰:「王者流散。」丁卯,月犯天關。占曰:「王者憂。」九月庚子,熒惑犯少微,又守之。占曰:「處士誅。」十月甲子,月犯東次相。其年七月,太皇太后李氏崩。十月,妖賊大破高雅之於餘姚,死者十七八。五年,孫恩攻侵郡縣,殺內史,至京口,進軍蒲洲,於是內外戒嚴。恩遣別將攻廣陵,殺三千餘人,退據郁洲,是時劉裕又追破之。九月,桓玄表至,逆旨陵上。十月,司馬元顯大治水軍,將以伐玄。元興元年正月,盧循自稱征虜將軍,領孫恩餘眾,略有永嘉、晉安之地。二月,帝戎服遣西軍。三月,桓玄,剋京都,殺司馬元顯,
 放太傅會稽王道子。



 元興元年三月戊子,太白犯五諸侯,因晝見。占曰:「諸侯有誅。」七月戊寅,熒惑在東井。熒惑犯輿鬼、積尸。占並同上。八月丙寅,太白奄右執法。九月癸未,太白犯進賢。占曰:「進賢者誅。」二年二月,歲星犯西上將。六月甲辰,月奄斗第四星。占曰:「大臣誅,不出三年。」八月癸丑,太白犯房北第二星。九月己丑,歲星犯進賢,熒惑犯西上將。十月甲戌,太白犯泣星。十一月丁酉,熒惑犯東上相。十二月乙巳,月奄軒轅第二星。占悉同上。元年冬,魏破姚興軍。二年十
 二月,桓玄篡位,放遷帝、后於尋陽,以永安何皇后為零陵君。三年二月,劉裕盡誅桓氏。三年正月戊戌,熒惑逆行,犯太微西上相。占曰:「天子戰於野,上相死。」二月丙辰,熒惑逆行,在左執法西北。占曰:「執法者誅。」四月甲午,月奄軒轅第二星。五月壬申,月奄斗第二星,填星入羽林。占並同上。是年二月丙辰,劉裕殺桓修等。三月己末,破走桓玄,遣軍西討。辛巳,誅左僕射王愉,桓玄劫天子如江陵。五月,玄下至崢嶸洲,義軍破滅之。桓振又攻沒江陵,幽劫天子。七月,永安何皇后崩。



 義熙元年三月壬辰,月奄左執法。占同上。丁酉,月奄
 心前星。占曰:「豫州有災」。太白犯東井。占曰:「秦有兵。」七月庚辰,太白晝見,在翼、軫。占曰:「為臣彊,荊州有兵喪。」八月丁巳,月犯斗第一星。占曰:「天下有兵。」一曰:「大臣憂。」九月甲子,熒惑犯少微。占曰:「處士誅。」庚寅,熒惑犯右執法。癸卯,熒惑犯左執法。占並同上。十一月丙戌,太白犯鉤鈐。占曰:「喉舌憂。」十二月己卯,歲星犯天江。占曰:「有兵亂,河津不通。」十一月,荊州刺史魏詠之薨。二年二月,司馬國璠等攻沒弋陽。四月,姚興伐仇池公楊盛,擊走之。九月,益州刺史司馬榮期為其參軍楊承祖所害。三年十二月,司徒揚州刺史王謐薨。四年正月,太保武陵王遵薨。三
 月,左僕射孔安國卒。自後政在劉裕,人主端拱而已。二年二月,太白犯南斗。占曰:「兵起。己丑,月犯心後星。占曰:「豫州有災。」四月癸丑,月犯太微西上將。己未,月犯房南第二星。乙丑,歲星犯天江。占曰:「有兵亂,河津不通。」五月癸未,月犯左角。占曰:「左將軍死,天下有兵。」壬寅,熒惑犯氐。占曰:「氐為宿宮,人主憂。」六月庚午,熒惑犯房北第二星。八月癸亥,熒惑犯南斗第五星。丁巳,犯建星。占曰:「為兵。」九月壬午,熒惑犯哭星,又犯泣星。是年二月甲戌,司馬國璠等攻沒弋陽。又,慕容超侵略徐、兗,三年正月,又寇北徐州,至下邳。十二月,司徒王謐薨。四年正月,武
 陵王遵薨。五年,慕容超復寇淮北。四月,劉裕大軍討之,拔臨朐。又圍廣固拔之。三年正月丙子,太白晝見,在奎。二月庚申,月奄心後星。占同上。五月癸未,月犯左角。己丑,太白晝見,在參。占曰:「益州有兵喪,臣彊。」八月己卯,太白犯左執法。辛卯,熒惑犯左執法。九月壬子,熒惑犯進賢星。是年八月,劉敬宣伐蜀,不剋而旋。四年三月,左僕射孔安國卒。七月,司馬叔璠等攻沒鄒山,魯郡太守徐邕破走之。姚略遣眾征赫連勃勃,大為破所。五年,劉裕討慕容超,滅之。四年正月庚子,熒惑犯天關。五月丁未,月奄斗第二星。
 壬子,填星犯天廩。占曰:「天下饑,倉粟少。」六月己丑,太白犯太微西上將。乙卯又犯左執法。十月戊子,熒惑入羽林。占悉同上。五年,劉裕討慕容超,後南北軍旅運轉不息。五年二月甲子,月犯昴。占曰:「胡不安,天子破匈奴。」五月戊戌,歲星入羽林。九月壬寅,月犯昴。十月,熒惑犯氐。閏月丁酉,月犯昴。辛亥,熒惑犯鉤鈐。己巳,月奄心大星。占曰:「王者惡之。」是年四月,劉裕討慕容超。十月,魏王圭遇弒殂。六年五月,盧循逼郊甸,宮衛被甲。六年三月丁卯,月奄房南第二星。災在次相。己巳,又奄
 斗第五星。占曰:「斗主吳,吳地兵起。」太白犯五諸侯。占曰:「諸侯有誅。」五月甲子,月奄斗第五星。己亥,月奄昴第三星。占曰:「國有憂。」一曰:「有白衣之會。」六月己丑,月犯房南第二星。甲午,太白晝見。七月己亥,月犯輿鬼。占曰:「國有憂。」一曰:「秦有兵。八月壬午,太白犯軒轅大星。甲申,月犯心前星。災在豫州。丙戌,月犯斗第五星。占同上。丁亥,月奄牛宿南星。占曰:「天下有大誅。」乙未,太白犯少微。丙午,太白在少微而晝見。九月甲寅,太白犯左執法。丁丑,填星犯畢。占曰:「有邊兵。」是年三月,始興太守徐道覆反。四月,盧循寇湘中,沒巴陵,率眾逼京畿。是月,左僕射孟昶
 懼王威不振,仰藥自殺。七年十二月,劉蕃梟徐道覆首,杜慧度斬盧循,並傳首京都。八年六月,劉道規卒,時為豫州刺史。八月,皇后王氏崩。九月,兗州刺史劉蕃、尚書左僕射謝混伏誅。劉裕西討劉毅,斬首徇之。十二月,遣益州刺史朱齡石伐蜀。七年四月辛丑,熒惑入輿鬼。占曰:「秦有兵。」一曰:「雍州有災。」六月,太白晝見,在翼。己亥,填星犯天關。占曰:「臣謀主。」八月,太白犯房南第二星。十一月丙子,太白犯哭星。其七月,朱齡石剋蜀。蜀又反,討滅之。八年七月癸亥,月奄房北第二星。己未,月犯井鉞。八月
 戊申,月犯泣星。十月辛亥,月奄天關。占曰:「有兵。」十一月丁丑,填星犯東井。占曰:「大人憂。」十二月癸卯,填星犯井鉞。是年八月,皇后王氏崩。九月,誅劉蕃、謝混,討滅劉毅。十二月,朱齡石滅蜀。九年二月,熒惑入輿鬼。占曰:「有兵喪。」太白犯南河。占曰:「兵起。」五月壬辰,太白犯右執法,晝見。七月庚午,月奄鉤鈐。占曰:「喉舌臣憂。」九月庚午,歲星犯軒轅大星。己丑,月犯左角。時劉裕擅命,兵革不休。十年,裕討司馬休之。王師不利,休之等奔長安。十年正月丁卯,月犯畢。占曰:「將相有以象坐罪者。」二月
 己酉,月犯房北星。五月壬寅,月犯牽牛南星。乙丑,歲星犯軒轅大星。占悉同上。六月丙申,月奄氐。占曰:「將死之,國有誅者。」七月庚辰,月犯天關。占曰:曰:「兵起。」熒惑犯井鉞。填星犯輿鬼,遂守之。占曰:「大人憂,宗廟改。」八月丁酉,月奄牽牛南星。占同上。九月,填星犯輿鬼。占曰:「人主憂。」丁巳,太白入羽林。十二月己酉,月犯西咸。占曰:「有陰謀。」十一年,林邑冠交州,距敗之。十一年三月丁巳,月入畢。占曰:「天下兵起。」一曰:「有邊兵。」己卯,熒惑入輿鬼。閏月丙午,填星又入輿鬼。占曰:「為旱,大疫,為亂臣。」五月癸卯,熒惑入太微。甲辰,犯右執法。六
 月己未,太白犯東井。占曰:「秦有兵。」戊寅,犯輿鬼。占曰:「國有憂。」七月辛丑,月犯畢。占同上。八月壬子,月犯氐。占同上。庚申,太白順行,從右掖門入太微。丁卯,奄左執法。十一月癸亥,月入畢。占同上。乙未,月入輿鬼而暈。十二年五月甲申,歲星留房心之間,宋之分野。始封劉裕為宋公。六月壬子,太白順行入太微右掖門。己巳,月犯畢。占同上。七月,月犯牛宿。十月丙戌,月入畢。十三年五月丙子,月犯軒轅。丁亥,犯牽牛。癸巳,熒惑犯右執法。八月己酉,月犯牽牛。丁卯,月犯太微。占曰:「人君憂。」九月壬辰,熒惑犯軒轅。十月戊申,月犯畢。占悉同上。
 月犯箕。占曰:「國有憂。」甲寅,月犯畢。占同上。乙卯,填星犯太微,留積七十餘日。占曰:「亡君之戒。」壬戌,月犯太微。十四年三月癸巳,太白犯五諸侯。五月庚子,月犯太微。七月甲辰,熒惑犯輿鬼。占曰:「秦有兵,又為旱,為兵喪。」亦曰:「大人憂,宗廟改,亦為亂臣。」時劉裕擅命,軍旅數興,饑旱相屬,其後卒移晉室。丁巳,月犯東井。占曰:「軍將死。」八月甲子,太白犯軒轅。癸酉,填星入太微,犯右執法,因留太微中,積二百餘日乃去。占曰:「填星守太微,亡君之戒,有徙王。九月乙未,太白入太微,犯左執法。丁巳,月入太微。占曰:「大人憂。」十月甲申,月入太微。癸巳,熒惑入太微,犯
 西蕃上將,仍順行,至左掖門內,留二十日,乃逆行。義熙十二年七月,劉裕伐姚泓。十三年八月,禽姚泓,司、兗、秦、雍悉平。十四年,劉裕還彭城,受宋公。十一月,左僕射前將軍劉穆之卒。明年,西虜寇長安,雍州刺史朱齡石諸軍陷沒,官軍捨而東。十二月,帝崩。



 恭帝元熙元年正月丙午,三月壬寅,五月丙申,月皆犯太微,占悉同上。乙卯,辰星犯軒轅。六月庚辰,太白犯太微。七月己卯,月犯太微,太白晝見。自義熙元年至是,太白經天者九,日蝕者四,皆從上始,革代更王,臣失君之象也。是夜,太白犯哭星。十二月丁巳,月、太白俱入羽
 林。二年二月庚午,填星犯太微。占悉同上。元年七月,劉裕受宋王。是年六月,帝遜位于宋。



 ○妖星客星



 魏文帝黃初三年九月甲辰,客星見太微左掖門內。占曰:「客星出太微,國有兵喪。」十月,帝南征孫權。是後,累有征役。六年十月乙未,有星孛於少微,歷軒轅,占:「為兵喪,除舊布新之象。」時帝軍廣陵,辛丑,親御甲胃觀兵。明年五月,帝崩。



 明帝太和六年十一月丙寅,有星孛于翼,近太微上將星。占曰:「為兵喪。」甘氏曰:「孛彗所當之國,是受其殃。翼又楚分野,孫權封略也。」明年,權有遼東之敗。又明年,諸葛亮入秦川。孫權發兵,緣江淮屯要衝,權自圍新城以應亮,天子東征權。



 青龍四年十月甲申,有星孛于大辰,長三尺。乙酉,又孛于東方。十一月己亥,彗星見,犯宦者天紀星。占曰:「大辰為天王,天下有喪。」劉向《五紀論》曰:「《春秋》,星孛于東方,不言宿者,不加宿也。宦者在天市,為中外有兵。天紀為地震,孛彗主兵喪。」景初元年六月,地震。九月,吳將朱然圍
 江夏。皇后毛氏崩。二年正月,討公孫文懿。三年正月,明帝崩。



 景初二年八月,彗星見張,長三尺,逆西行,四十一日滅。占同上。張,周分野。十月癸巳,客星見危,逆行,在離宮北、騰蛇南。甲辰,犯宗星。己酉,滅。占曰:「客星所出有兵喪。虛危為宗廟,又為墳墓。客星近離宮,則宮中將有大喪,就先君於宗廟之象也。三年正月,帝崩。



 少帝正始元年十月乙酉,彗星見西方,在尾,長三丈,拂牽牛,犯太白。十一月甲子,進犯羽林。占曰:「尾為燕,又為吳,牛亦吳越之分。太白為上將,羽林中軍兵。為吳越有
 喪,中軍兵動。」二年五月,吳遣三將寇邊。吳太子登卒。六月,宣帝討諸葛恪於皖。太尉滿寵薨。六年八月戊午,彗星見七星,長二尺,色白,進至張,積二十三日滅。七年十一月癸亥,又見軫,長一尺,積百五十六日滅。九年三月,又見昴,長六尺,色青白,芒西南指。七月,又見翼,長二尺,進至軫,積四十二日滅。案占曰:「七星張為周分野,翼軫為楚,昴為趙魏。彗所以除舊布新,主兵喪也。」嘉平元年,宣帝誅曹爽兄弟及其黨與,皆夷三族,京師嚴兵。三年,誅楚王彪,又襲王凌於淮南。淮南,東楚也。魏諸王幽於鄴。



 嘉平三年十一月癸亥,有星孛于營室,西行,積九十日滅。占曰:「有兵喪。室為後宮,後宮且有亂。」四年二月丁酉,彗星見西方,在胃,長五六丈,色白,芒南指,貫參,積二十日滅。五年十一月,彗星又見軫,長五丈,在太微左執法西,東南指,積百九十日滅。案占:「胃,兗州之分野。參,主兵。太微,天子庭。執法,為執政。孛彗為兵喪,除舊布新之象。」正元元年二月,李豐、豐弟翼、后父張緝等謀亂,皆誅,皇后亦廢。九月,帝廢為齊王。



 高貴鄉公正元元年十一月,白氣出南斗側,廣數丈,長竟天。王肅曰:「蚩尤之旗也,東南其有亂乎!」二年正月,有
 彗星見于吳楚分,西北竟天。鎮東大將軍毋丘儉等據淮南叛,景帝討平之。案占:「蚩尤旗見,王者征伐四方。」自後又征淮南,西平巴蜀。是歲,吳主孫亮五鳳元年也。斗牛,吳越分。案占:「吳有兵喪,除舊布新之象也。」太平三年,孫綝盛兵圍宮,廢亮為會稽王,故《國志》又書於吳也。淮南江東同揚州地,故于時變見吳、楚。楚之分則魏之淮南,多與吳同災。是以毋丘儉以孛為己應,遂起兵而敗。後三年,即魏甘露二年,諸葛誕又反淮南,吳遣將救之。及城陷,誕眾與吳兵死沒各數萬人,猶前長星之應也。



 甘露二年十一月,彗星見角,色白。占曰:「彗星見兩角間
 色白者,軍起不戰,邦有大喪。」景元元年,高貴鄉公為成濟所害。四年十月丁丑,客星見太微中,轉東南行,歷軫宿,積七日滅。占曰:「客星出太微,有兵喪。」景元元年,高貴鄉公被害。



 元帝景元三年十一月壬寅,彗星見亢,色白,長五寸,轉北行,積四十五日滅。占曰:「為兵喪。」一曰:「彗星見亢,天子失德。』四年,鐘會、鄧艾伐蜀,剋之。二將反亂,皆誅。



 咸熙二年五月,彗星見王良,長丈餘,色白,東南指,積十二日滅。占曰:「王良,天子御駟。彗星掃之,禪代之表,除舊
 布新之象也。白色為喪。王良在東壁宿,又并州之分野。」八月,文帝崩。十二月,武帝受魏禪。



 武帝泰始四年正月丙戌,彗星見軫,青白色,西北行,又轉東行。占曰:「為兵喪,軫又楚分野。」三月,皇太后王氏崩。十月,吳寇江夏、襄陽。五年九月,星孛于紫宮。占如上。紫宮,天子內宮。十年,武元楊皇后崩。十年十二月,有星孛于軫。占曰:「天下兵起,軫又楚分野。」



 咸寧二年六月甲戌,星孛于氐。占曰:「天子失德易政。氐,又兗州分。」七月,星孛大角。大角為帝坐。八月,星孛太微,
 至翼、北斗、三台。占曰:「太微,天子庭,大人惡之。」一曰:「有改王。翼,又楚分野。北斗主殺罰,三台為三公。」三年正月,星孛於西方。三月,星孛於胃。胃,徐州分。四月,星孛女御。女御為後宮。五月,又孛于東方。七月,星孛紫宮。上曰:「天下易主。」四年四月,蚩尤旗見東井。後二年,傾三方伐吳,是其應也。五年三月,星孛于柳。四月,又孛于女御。七月,孛于紫宮。占曰:「外臣陵主。柳,又三河分野。大角、太微、紫宮、女御並為王者。」明年吳亡,是其應也。孛主兵喪。征吳之役,三河、
 徐、兗之兵悉出,交戰於吳楚之地,吳丞相都督以下梟戮十數,偏裨行陣之徒馘斬萬計,皆其徵也。



 太康二年八月,有星孛于張。占曰:「為兵喪。」十一月,星孛于斬轅。占曰:「後宮當之。」四年三月戊申,星孛於西南。是年,齊王攸、任城王陵、瑯邪王伷、新都王該薨。八年九月,星孛于南斗,長數十丈餘日滅。占曰:「斗主爵祿,國有大憂。」一曰:「孛于斗,王者疾病,天下易政,大亂兵起。」



 太熙元年四月,客星在紫宮。占曰:「為兵喪。」太康未,武帝
 耽宴遊,多疾病。是月己酉,帝崩。永平元年,賈后誅楊駿及其黨與,皆夷三族,楊太后亦見弒。又誅汝南王亮、太保衛瓘、楚王瑋,王室兵喪之應也。



 惠帝元康五年四月,有星孛于奎,至軒轅、太微,經三台、太陵。占曰:「奎為魯,又為庫兵,軒轅為後宮,太微天子庭,三台為三司,太陵有積尸死喪之事。」其後武庫火,西羌反。後五年,司空張華遇禍,賈后廢死,魯公賈謐誅。又明年,趙王倫篡位。於是三王興兵討倫,兵士戰死十餘萬人。



 永康元年三月,妖星見南方。占曰:「妖星出,天下大兵將
 起。」是月賈后殺太子,趙王倫尋廢殺后,斬司空張華,又廢帝自立。於是三王並起,迭總天權。其十二月,彗星出牽牛之西,指天市。占曰:「牛者七政始,彗出之,改元易號之象也。天市一名天府,一名天子旗,帝坐在其中。」明年,趙王倫篡位,改元,尋為大兵所滅。二年四月,彗星見齊分。占曰:「齊有兵喪。」是時,齊王冏起兵討趙王倫。倫滅,冏擁兵不朝,專權淫奢。明年,誅死。



 太安元年四月,彗星晝見。二年三月,彗星見東方,指三台。占曰:「兵喪之象。三台為三公。」三年正月,東海王越執太尉、長沙王乂,張方又殺
 之。



 永興元年五月,客星守畢。占曰:「天子絕嗣。」一曰:「大臣有誅。」時諸王擁兵,其後惠帝失統,終無繼嗣。二年八月,有星孛于昴畢。占曰:「為兵喪。昴畢又趙魏分野。」十月丁丑,有星孛于北斗。占曰:「璇璣更授,天子出走。」又曰:「彊國發兵,諸侯爭權。」是後,諸王交兵,皆有應。明年,惠帝崩。



 成帝咸和四年七月,有星孛于西北,犯斗,二十三日滅。占曰:「為兵亂。」十二月,郭默殺江州刺史劉胤,荊州刺史陶侃討默,斬之。時石勒又始僭號。



 咸康二年正月辛巳,彗星夕見西方,在奎。占曰:「為兵喪。奎,又為邊兵。」三年正月,石季龍僭天王位。四年,石季龍伐慕容皝,不剋。既退,皝追擊之,又破麻秋。時皝稱蕃,邊兵之應也。六年二月庚辰,有星孛于太微。七年三月,杜皇后崩。



 康帝建元元年十一月六日,彗星見亢,長七尺,白色。占曰:「亢為朝廷,主兵喪。」二年,康帝崩。



 穆帝永和五年十一月乙卯,彗星見於亢。芒西向,色白,長一丈。六年正月丁丑,彗星又見于亢。占曰:「為兵喪、疾疫。」其五年八月,褚裒北征,兵敗。十一月,冉閔殺石遵,又
 盡殺胡十餘萬人,於是中土大亂。十二月,褚裒薨。是年,大疫。



 升平二年五月丁亥,彗星出天船,在胃。占曰:「為兵喪,除舊布新。出天船,外夷侵。」一曰:「為大水。」四年五月,天下大水。五年,穆帝崩。



 哀帝興寧元年八月,有星孛於角亢,入天市。案占曰:「為兵喪。」三年正月,皇后王氏崩。二月,帝崩。三月,慕容恪攻沒洛陽,沈勁等戰死。



 海西太和四年二月,客星見紫宮西垣,至七月乃滅。占曰:「客星守紫宮,臣弒主。」六年,桓溫廢帝為海西公。



 孝武寧康二年正月丁巳,有星孛于女虛,經氐、亢、角、軫、翼、張。至三月丙戌,彗星,見於氐。九月丁丑,有星孛于天市。占曰:「為兵喪。」太元元年七月,苻堅破涼州,虜張天錫。



 太元十一年三月,客星在南斗,至六月乃沒。占曰:「有兵,有赦。」是後司、雍、兗、冀常有兵役。十二年正月大赦,八月又大赦。十五年七月壬申,有星孛于北河戍,經太微、三台、文昌,入北斗,色白,長十餘丈。八月戊戌,入紫宮乃減。占曰:「北河戍一名胡門,胡有兵喪。掃太微,入紫微,王者當之。三台為三公,文昌為將相,將相三公有災。入北斗,諸侯
 戮。」一曰:「掃北斗,彊國發兵,諸侯爭權,大人憂。」二十一年,帝崩。隆安元年,王恭、殷仲堪、桓玄等並發兵,表以誅王國寶為名。朝廷順而殺之,并斬其從弟緒,司馬道子由是失勢,禍亂成矣。十八年二月,客星在尾中,至九月乃滅。占曰:「燕有兵喪。」二十年,慕容垂息寶伐魏,為所破,死者數萬人。二十一年,垂死,國遂衰亡。二十年九月,有蓬星如粉絮,東南行,歷女虛,至哭星。占曰:「蓬星見,不出三年,必有亂臣戮死於市。」是時,王國寶交構朝廷。二十一年九月,帝崩。隆安元年,王恭等興兵,
 而朝廷殺王國寶、王緒。



 安帝隆安四年二月己丑,有星孛於奎,長三丈,上至閣道、紫宮西蕃,入北斗魁,至三台,三月,遂經于太微帝坐端門。占曰:「彗星掃天子庭閣道,易主之象。」經三台入北斗。占同上條。十二月戊寅,有星孛于貫索、天市、天津。占曰:「貴臣獄死,內外有兵喪。天津為賊斷,王道天下不通。」案占:「災在吳越。」五年二月,有孫恩兵亂,攻侵郡國。於是內外戒嚴,營陣屯守,柵斷淮口。九月,桓玄表至,逆旨陵上。其後玄遂篡位,亂京都,大饑,人相食,百姓流亡,皆其應
 也。



 元興元年十月,有客星色白如粉絮,在太微西,至十二月入太微。占曰:「兵入天子庭。」二年十二月,桓玄篡位,放遷帝、后於尋陽,以永安何皇后為零陵君。三年二月,劉裕盡誅桓氏。



 義熙十一年五月甲申,彗星二出天市,掃帝坐,在房心北。房心,宋之分野。案占:「得彗柄者興,除舊布新,宋興之象。」十四年五月庚子,有星孛於北斗魁中。七月癸亥,彗星出太微西,柄起上相星下,芒漸長至十餘丈,進掃北斗、紫微、中台。占曰:「彗出太微,社稷亡,天下易王;入北斗、紫
 微,帝宮空。」十四年,劉裕還彭城,受宋公。十二月,帝崩。



 恭帝元年正月戊戌,有星孛于太微西蕃。占曰:「革命之徵。」其年,宋有天下。



 ○星流隕



 蜀後主建興十三年,諸葛亮帥大眾伐魏,屯于渭南。有長星赤而芒角,自東北西南流,投亮營,三投再還,往大還小。占曰:「兩軍相當,有大流星來走軍上及墜軍中者,皆破敗之徵也。」九月,亮卒于軍,焚營而退,群帥交怨,多相誅殘。



 魏明帝景初二年,宣帝圍公孫文懿於襄平。八月丙寅
 夜,有大流星長數十丈,白色有芒鬣,從首山東北流,墜襄平城東南。占曰:「圍城而有流星來走城上及墜城中者破。」又曰:「星墜,當其下有戰場。」又曰:「凡星所墜,國易姓。」九月,文懿突圍走,至星墜所被斬,屠城,坑其眾。



 元帝景元四年六月,有大流星二並如斗,見西方,分流南北,光照地,隆隆有聲。案占:「流星為貴使,星大者使大。」是年,鐘、鄧剋蜀,二星蓋二帥之象。二帥相背,又分流南北之應。鐘會既叛,三軍憤怒,隆隆有聲,兵將怒之徵也。



 武帝泰始四年七月,星隕如雨,皆西流。占曰:「星隕為百姓叛。西流,吳人歸晉之象好。」二年,吳夏口督孫秀率部
 曲二千餘人來降。



 太康九年八月壬子,星隕如雨。《劉向傳》云:「下去其上之象。」後三年,帝崩而惠帝立,天下自此亂矣。



 惠帝元康四年九月甲午,枉矢東北行,竟天。六年六月丙午夜,有枉矢自斗魁東南行。案占曰:「以亂伐亂。北斗主執殺,出斗魁,居中執殺者,不直之象也。」是後,趙王殺張、裴,廢賈后,以理太子之冤,因自篡盜,以至屠滅,以亂伐亂之應也。一曰,氐帥齊萬年反之應也。



 太安二年十一月辛巳,有星晝隕中天北下,光變白,有聲如雷。案占:「名曰營首。營首所在,下有大兵,流血。」明年,
 劉元海、石勒攻略並州,多所殘滅。王浚起燕代,引鮮卑攻掠鄴中,百姓塗地。有聲如雷,怒之象也。



 永興元年七月乙丑,星隕有聲。二年十月,星又隕有聲。占同上。是後,遂亡中夏。



 光熙元年五月,枉矢西南流。是時,司馬越西破河間兵,奉迎大駕,尋收繆胤、何綏等,肆無君之心,天下惡之。及死而石勒焚其屍柩,是其應也。



 懷帝永嘉元年九月辛卯,有大星如日,自西南流于東北,小者如斗,相隨,天盡赤,聲如雷。占曰:「流星為貴使,星大者使大。」是年五月,汲桑殺東燕王騰,遂據河北。十一
 月,始遣和郁為征北將軍,鎮鄴西。田甄等大破汲桑,斬於樂陵。於是以甄為汲郡太守,弟蘭鉅鹿太守。小星相隨者,小將別帥之象也。司馬越忿魏郡以東平原以南皆黨於桑,以賞甄等,於是侵掠赤地。有聲如雷,忿怒之象也。四年十月庚子,大星西北墜,有聲。尋而帝蒙塵于平陽。



 元帝太興三年四月壬辰,枉矢出虛、危,沒翼、軫。占曰:「枉矢所觸,天下之所伐。翼、軫,荊州之分野。」太寧二年,王敦殺譙王承及甘卓,而敦又梟夷,枉矢觸翼之應也。



 永昌元年七月甲午,有流星大如甕,長百餘丈,青赤色,
 從西方來,尾分為百餘岐,或散。時王敦之亂,百姓流亡之應也。



 成帝咸康三年元月辛末,流星大如二斗魁,色青赤,光耀地,出奎中,沒婁北。案占:「為饑,五穀不藏。」是月,大旱,饑。六年二月庚午朔,有流星大如斗,光耀地,出天市,西行人太微。占曰:「大人當之。」八年六月,成帝崩。



 穆帝永和八年六月辛巳,日未入,有流星大如三斗魁,從辰巳上,東南行。晷度推之,在箕、斗之間,蓋燕分也。案占:「為營首。營首之下,流血滂沱。」是時,慕容俊僭稱大燕,攻伐無已。
 十年四月癸未,流星大如斗,色赤黃,出織女,沒造父,有聲如雷。占曰:「燕齊有兵,百姓流亡。」其年十二月,慕容俊遂據臨漳,盡有幽、並、青、冀之地。緣河諸將奔散,河津隔絕。慕容恪攻齊。



 升平二年十一月,枉矢自東南流于西北,其長半天。四年十月庚戌,天狗見西南。占曰:「有大兵,流血。」



 海西太和四年十月壬申,有大流星西下,有聲如雷。明年,遣使免袁真為庶人。桓溫在壽春,真病死,息瑾代立,求救於苻堅。溫破苻堅軍。六年,壽春城陷。



 孝武太元六年十月乙卯,有奔星東南經翼、軫,聲如雷。
 占曰:「楚地有兵,軍破,百姓流亡。」十二月,苻堅荊州刺史梁成、襄陽太守閻震率眾伐竟陵,桓石虔擊大破之,生擒震,斬首七千,獲生口萬人。聲如雷,將帥怒之象也。十三年閏月戊辰,天狗東北下,有聲。占曰:「有大戰,流血。」自是後,慕容垂、翟遼、姚萇、苻登、慕容永並阻兵爭彊。十四年正月,彭城妖賊又稱偽號於皇丘,劉牢之破滅之。三月,張道破合鄉、太山,向欽之擊走之。



 安帝隆安五年三月甲寅,流星赤色,眾多西行,經牽牛、虛、危、天津、閣道,貫太微、紫宮。占曰:「星庶人類,眾多西行,眾將西流之象。經天子庭,主弱臣彊,諸侯兵不制。」其年
 五月,孫恩侵吳郡,殺內史。六月,至京口。於是內外戒嚴,營陣屯守,劉裕追破之。元興元年七月,大饑,人相食。浙江以東流亡十六七,吳郡、吳興戶口減半,又流奔而西者萬計。十月,桓玄遣將擊劉軌,破走之。軌奔青州。



 ○雲氣



 惠帝永興元年十二月壬寅夜,有赤氣亙天,砰隱有聲。二年十月丁丑,赤氣見北方,東西竟天。占曰:「並為大兵。砰隱有聲,怒之象也。」是後,四海雲擾,九服交兵。



 光熙元年十二月甲申,有白氣若虹,中天北下至地,夜見五日乃滅。占曰:「大兵起。」明年,王彌起青徐,汲桑亂河
 北,毒流天下。



 懷帝永嘉三年十一月乙亥,有白氣如帶,出南北方各二,起地至天,貫參伐中。占曰:「天下大兵起。」四年三月,司馬越收繆胤等。又,三方雲擾,攻戰不休。五年三月,司馬越死於寧平城,石勒攻破其眾,死者十餘萬人。六月,京都焚滅,帝如虜庭。



 愍帝建興元年十月己巳夜,有赤氣曜於西北。荊州刺史陶侃討杜弢之黨於石城,戰敗。



\end{pinyinscope}