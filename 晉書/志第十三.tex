\article{志第十三}

\begin{pinyinscope}

 樂下



 永嘉之亂,海內分崩,伶官樂器,皆沒於劉、石。江左初立宗廟,尚書下太常祭祀所用樂名。太常賀循答云:「魏氏增損漢樂,以為一代之禮,未審大晉樂名所以為異。遭離喪亂,舊典不存。然此諸樂皆和之以鐘律,文之以五聲,詠之於歌辭,陳之於舞列。宮懸在庭,琴瑟在堂,八音迭奏,雅樂並作,登歌下管,各有常詠,周人之舊也。自漢
 氏以來,依仿此禮,自造新詩而已。舊京荒廢,今既散亡,音韻曲折,又無識者,則於今難以意言。」於時以無雅樂器及伶人,省太樂並鼓吹令。是後頗得登歌,食舉之樂,猶有未備。太寧末,明帝又訪阮孚等增益之。咸和中,成帝乃復置太樂官,鳩集遺逸,而尚未有金石也。庾亮為荊州,與謝尚脩復雅樂,未具而亮薨。庾翼、桓溫專事軍旅,樂器在庫,遂至朽壞焉。及慕容人雋平冉閔,兵戈之際,而鄴下樂人亦頗有來者。永和十一年,謝尚鎮壽陽,於是採拾樂人,以備太樂,并制石磬,雅樂始頗具。面王猛平鄴,慕容氏所得樂聲又入關右。太元中,破苻堅,又獲
 其樂工楊蜀等,閑習舊樂,於是四廂金石始備焉。乃使曹毗、王珣等增造宗廟歌詩,然郊祀遂不設樂。今列其詞於後云。



 歌宣帝曹毗



 於赫高祖,德協靈符。應運撥亂,釐整天衢。勛格宇宙,化動八區。肅以典刑,陶以玄珠。神石吐瑞,靈芝自敷。肇基天命,道均唐虞。



 歌景帝曹毗



 景皇承運,纂隆洪緒。皇羅重抗,天暉再舉。蠢矣二寇,擾我揚楚。乃整元戎,以膏齊斧。
 亹亹神算,赫赫王旅。鯨鯢既平,功冠帝宇。



 歌文帝曹毗



 太祖齊聖,王猷誕融。仁教四塞,天基累崇。皇室多難,嚴清紫宮。威厲秋霜,惠過春風。平蜀夷楚,以文以戎。奄有參墟,聲流無窮。



 歌武帝曹毗



 於穆武皇,允龔欽明。應期登禪,龍飛紫庭。百揆時序,聽斷以情。殊域既賓,偽吳亦平。晨流甘露,宵映朗星。野有擊壤,路垂頌聲。



 歌元帝曹毗



 運屯百六,天羅解貫。元皇勃興,網籠江漢。仰齊七政,俯平禍亂。化若風行,澤猶雨散。淪光更曜,金輝復煥。德冠千載,蔚有餘粲。



 歌明帝曹毗



 明明肅祖,闡弘帝祚。英風夙發,清暉載路。奸逆縱忒,罔式皇度。躬振朱旗,遂豁天步。宏猷允塞,高羅雲布。品物咸寧,洪基永固。



 歌成帝曹毗



 於休顯宗,道澤玄播。式宣德音,暢物以和。邁德蹈仁,匪禮不過。敷以純風,濯以清波。
 連理映阜,鳴鳳棲柯。同規放勛,義蓋山河。



 歌康帝曹毗



 康皇穆穆,仰嗣洪德。為而不宰,雅音四塞。閑邪以誠,鎮物以默。威靜區宇,道宣邦國。



 歌穆帝曹毗



 孝宗夙哲,休音久臧。如彼晨離,耀景扶桑。垂訓華幄,流潤八荒。幽贊玄妙,爰該典章。西平僭蜀,北靜舊疆。高猷遠暢,朝有遺芳。



 歌哀帝曹毗



 於穆哀皇,聖心虛遠。雅好玄古,大庭是踐。
 道尚無為,治存易簡。化若風行,時猶草偃。雖曰登遐,徽音彌闡。愔愔《雲》《韶》,盡美盡善。



 歌簡文帝王珣



 皇矣簡文,於昭于天。靈明若神,周淡如川。沖應其來,實與其遷。亹亹心化,日用不言。易而有親,簡而可傳。觀流彌遠,求本踰玄。



 歌孝武帝王珣



 天監有晉,欽哉烈宗。同規文考,玄默允恭。威而不猛,約而能通。神鉦一震,九域來同。道積淮海,雅頌自東。氣陶醇露,化協時雍。



 四時祠祀曹毗



 肅肅清廟,巍巍聖功。萬國來賓,禮儀有容。鐘鼓振,金石熙。宣兆祚,武開基。神斯樂兮!理管絃,有來斯和。說功德,吐清歌。神斯樂兮!洋洋玄化,潤被九壤。民無不悅,道無不往。禮有儀,樂有式。詠九功,永無極。神斯樂兮!



 漢時有《短簫鐃歌》之樂,其曲有《朱鷺》、《思悲翁》、《艾如張》、《上之回》、《雍離》、《戰城南》、《巫山高》、《上陵》、《將進酒》、《君馬黃》、《芳樹》、《有所思》、《雉子班》、《
 聖人出》、《上邪》、《臨高臺》、《遠如期》、《石留》、《務成》、《玄雲》、《黃爵行》《釣竿》等曲,列於鼓吹,多序戰陣之事。



 及魏受命,改其十二曲,使繆襲為詞,述以功德代漢。改《朱鷺》為《楚之平》,言魏也。改《思悲翁》為《戰滎陽》,言曹公也。改《艾如張》為《獲呂布》,言曹公東圍臨淮,擒呂布也。改《上之回》為《克官渡》,言曹公與袁紹戰,破之於官渡也。改《雍離》為《舊邦》,言曹公勝袁紹於官渡,還譙收藏死亡士卒也。改《戰城南》為《定武功》,言曹公初破鄴,武功之定始乎此也。改《巫山高》為《屠柳城》,言曹公越北塞,歷白檀,破三郡烏桓於柳城也。改《上陵》為《平
 南荊》,言曹公平荊州也。改《將進酒》為《平關中》,言曹公征馬超,定關中也。改《有所思》為《應帝期》,言文帝以聖德受命,應運期也。改《芳樹》為《邕熙》,言魏氏臨其國,君臣邕穆,庶績咸熙也。改《上邪》為《太和》,言明帝繼體承統,太和改元,德澤流布也。其餘並同舊名。



 是時吳亦使韋昭製十二曲名,以述功德受命。改《朱鷺》為《炎精缺》,言漢室衰,孫堅奮迅猛志,念在匡救,王迹始乎此也。改《思悲翁》為《漢之季》,言堅悼漢之微,痛董卓之亂,興兵奮擊,功蓋海內也。改《艾如張》為《攄武師》,言權卒父之業而征伐也。改《上之回》為《烏林》,言魏武既破荊州,順流東下,欲來爭鋒,權
 命將周瑜逆擊之於烏林而破走也。改《雍離》為《秋風》,言權悅以使人,人忘其死也。改《戰城南》為《克皖城》,言魏武志圖並兼,而權親征,破之於皖也。改《巫山高》為《關背德》,言蜀將關羽背棄吳德,權引師浮江而擒之也。改《上陵曲》為通荊州,言權與蜀交好齊盟,中有關羽自失之愆,終復初好也。改《將進酒》為《章洪德》,言權章其大德,而遠方來附也。改《有所思》為《順歷數》,言權順籙圖之符,而建大號也。改《芳樹》為《承天命》,言其時主聖德踐位,道化至盛也。改《上邪曲》為《玄化》,言其時主脩文武,則天而行,仁澤流洽,天下喜樂也。其餘亦用舊名不改。



 及武帝受禪,
 乃令傅玄製為二十二篇,亦述以功德代魏。改《朱鷺》為《靈之祥》,言宣帝之佐魏,猶虞舜之事堯,既有石瑞之徵,又能用武以誅孟達之逆命也。改《思悲翁》為《宣受命》,言宣帝禦諸葛亮,養威重,運神兵,亮震怖而死也。改《艾如張》為《征遼東》,言宣帝陵大海之表,討滅公孫氏而梟其首也。改《上之回》為《宣輔政》,言宣帝聖道深遠,撥亂反正,網羅文武之才,以定二儀之序也。改《雍離》為《時運多難》,言宣帝致討吳方,有征無戰也。改《戰城南》為《景龍飛》,言景帝克明威教,賞順夷逆,隆無疆,崇洪基也。改《巫山高》為《平玉衡》,言景帝一萬國之殊風,齊四海之乖心,禮賢
 養士,而纂洪業也。改《上陵》為《文皇統百揆》,言文帝始統百揆,用人有序,以敷太平之化也。改《將進酒》為《因時運》,言因時運變,聖謀潛施,解長蛇之交,離群桀之黨,以武濟文,以邁其德也。改《有所思》為《惟庸蜀》,言文帝既平萬乘之蜀,封建萬國,復五等之爵也。改《芳樹》為《天序》,言聖皇應歷受禪,弘濟大化,用人各盡其才也。改《上邪》為《大晉承運期》,言聖皇應籙受圖,化象神明也。改《君馬黃》為《金靈運》,言聖皇踐阼,致敬宗廟,而孝道行於天下也。改《雉子班》為《於穆我皇》,言聖皇受禪,德合神明也。改《聖人出》為《仲春振旅》,言大晉申文武之教,畋獵以時也。改《臨
 高臺》為《夏苗田》,言大晉畋狩順時,為苗除害也。改《遠如期》為《仲秋獮田》,言大晉雖有文德,不廢武事,順時以殺伐也。改《石留》為《順天道》,言仲冬大閱,用武脩文,大晉之德配天也。改《務成》為《唐堯》,言聖皇陟帝位,德化光四表也。《玄雲》依舊名,言聖皇用人,各盡其材也。改《黃爵行》為《伯益》,言赤烏銜書,有周以興,今聖皇受命,神雀來也。《釣竿》依舊名,言聖皇德配堯舜,又有呂望之佐,濟大功,致太平也。其辭並列之於後云。



 靈之祥



 靈之祥,石瑞章。旌金德,出西方。天降命,
 授宣皇。應期運,時龍驤。繼大舜,佐陶唐。贊武文,建帝綱。孟氏叛,據南疆。追有扈,亂五常。吳寇叛,蜀虜彊。交誓盟,連遐荒。宣赫怒,奮鷹揚。震乾威,曜電光。陵九天,陷石城。梟逆命,拯有生。萬國安,四海寧。



 宣受命



 宣受命,應天機,風雲時動神龍飛。禦葛亮,鎮雍梁。邊境安,夷夏康。務節事,勤定傾。攬英雄,保持盈。深穆穆,赫明明。沖而泰,天之經。養威重,運神兵。亮乃震斃,天下安寧。



 征遼東



 征遼東,敵失據,威靈邁日域。公孫既授首,群逆破膽,咸震怖。朔北響應,海表景附。武功赫赫,德雲布。



 宣輔政



 宣皇輔政,聖烈深。撥亂反正,順天心。網羅文武才,慎厥所生。所生賢,遺教施。安上治民,化風移。肇創帝基,洪業垂。於鑠明明,時赫戲。功濟萬世,定二儀。定二儀,雲行雨施,海外風馳。



 時運多難



 時運多難,道教痡。天地變化,有盈虛。蠢爾吳蠻,武視江湖。我皇赫斯,致天誅。有征無戰,弭其圖。天威橫被,廓東隅。



 景龍飛



 景龍飛,御天威。聰鑒玄察,動與神明協機。從之者顯,逆之者滅夷。文教敷,武功巍。普被四海,萬邦望風,莫不來綏。聖德潛斷,先天弗違。弗違祥,享世永長。猛以致寬,道化光。赫明明,祚隆無疆。帝績惟期,
 有命既集,崇此洪基。



 平玉衡



 平玉衡,糾姦回。萬國殊風,四海乖。禮賢養士,羈御英雄,思心齊。纂戎洪業,崇皇階。品物咸亨,聖敬日躋。聰鑒盡下情,明明綜天機。



 文皇統百揆



 文皇統百揆,繼天理萬方。武將鎮四隅,英佐盈朝堂,謀言協秋蘭,清風發其芳。洪澤所漸潤,礫石為珪璋。大道侔五帝,盛德踰三王。咸光大,上參天與地,
 至化無內外。無內外,六合並康乂。並康乂,遘茲嘉會。在昔羲與農,大晉德斯邁。鎮征及諸州,為籓衛。功濟四海,洪烈流萬世。



 因時運



 因時運,聖策施。長蛇交解,群桀離。勢窮奔吳,獸騎厲。惟武進,審大計。時邁其德,清一世。



 惟庸蜀



 惟庸蜀,僭號天一隅。劉備逆帝命,禪亮承其餘。擁眾數十萬,窺隙乘我虛。
 驛騎進羽檄,天下不遑居。姜維屢寇邊,隴上為荒蕪。文皇愍斯民,歷世受罪辜。外謨籓屏臣,內謨眾士夫。爪牙應指受,腹心獻良圖。良圖協成文,大興百萬軍。雷鼓震地起,猛勢陵浮雲。逋虜畏天誅,面縛造壘門。萬里同風教,逆命稱妾臣。光建五等,紀綱天人。



 天序



 天序,應歷受禪,承靈祜。御群龍,勒螭武。弘濟大化,英雋作輔。明明統萬機,
 赫赫鎮四方,咎繇稷契之疇,協蘭芳。禮王臣,覆兆民。化之如天與地,誰敢愛其身?



 大晉承運期



 大晉承運期,德隆聖皇。時清晏,白日垂光。應籙圖,陟帝位,繼天正玉衡。化行象神明,至哉道隆虞與唐。元首敷洪化,百僚股肱並忠良。時太康,隆隆赫赫,福祚盈無疆。



 金靈運



 金靈運,天符發。聖徵見,參日月。
 惟我皇,體神聖。受魏禪,應天命。皇之興,靈有徵。登大麓,御萬乘。皇之輔,若闞武。爪牙奮,莫之禦。皇之佐,贊清化。百事理,萬邦賀。神祗應,嘉瑞章。恭享禮,薦先皇。樂時奏,磬管鏘。鼓殷殷,鐘鍠鍠。奠樽俎,實玉觴。神歆饗,咸悅康。宴孫子,祐無疆。大孝蒸蒸,德教被萬方。



 於穆我皇



 於穆我皇,盛德聖且明。受禪君世,光濟群生。
 普天率土,莫不來庭。顒顒六合內,望風仰泰清。萬國雍雍,興頌聲。大化洽,地平而天成。七政齊,玉衡惟平。峨峨佐命,濟濟群英。夙夜乾乾,萬機是經。雖治興,匪荒寧。謙道樂,沖不盈。天地合德,日月同榮。赫赫煌煌,曜幽冥。三光克從,於顯天,垂景星。龍鳳臻,甘露宵零。肅神祗,祗上靈。萬物欣戴,自天效其成。



 仲春振旅



 仲春振旅,大致人,武教於時日新。
 師執提,工執鼓。坐作從,節有序。盛矣允文允武!搜田表祃,申法誓。遂圍禁,獻社祭。允以時,明國制。文武並用,禮之經。列車如戰,大教明,古今誰能去兵?大晉繼天,濟群生。



 夏苗田



 夏苗田,運將徂。軍國異容,文武殊。乃命群吏,撰車徒,辯其號名,贊契書。王軍啟八門,行同上帝居。時路建大麾,雲旗翳紫虛。百官象其事,疾則疾,徐則徐。
 回衡旋軫,罷陣弊車。獻禽享祀,蒸蒸配有虞。惟大晉,德參兩儀,化雲敷。



 仲秋獮田



 仲秋獮田,金德常綱。涼風清且厲,凝露結為霜。白藏司辰,金隼時鷹揚。鷹揚猶尚父,順天以殺伐,春秋時序。雷霆震威曜,進退由鉦鼓。致禽祀祊,羽毛之用充軍府。赫赫大晉德,芬烈陵三五。敷化以文,雖安不廢武。光宅四海,永享天之祜。



 順天道



 順天道,握神契,三時示,講武事。冬大閱,鳴鐲振鼓鐸,旌旗象虹霓。文制其中,武不窮武。動軍誓眾,禮成而義舉。三驅以崇仁,進止不失其序。兵卒練,將如闞武。惟闞武,氣陵青雲。解圍三面,殺不殄群。偃旌麾,班六軍。獻享蒸,修典文。嘉大晉,德配天。祿報功,爵俟賢。饗燕樂,受茲百祿,壽萬年。



 唐堯



 唐堯諮務成,謙謙德其興。積漸終光大,
 履霜致堅冰。神明道自成,河海猶可凝。舜禹統百揆,元凱以次升。禪讓應大歷,睿聖世相承。我皇陟帝位,平衡正準繩。德化飛四表,祥氣見其徵。興王坐俟旦,亡主恬自矜。致遠由近始,覆簣成山陵。披圖案先籍,有其證靈液。



 玄雲



 玄雲起丘山,祥氣萬里會。龍飛何蜿蜿,鳳翔何翽翽。昔在唐虞朝,時見青雲際。今親遊萬國,流興溢天外。鶴鳴在後園,
 清音隨風邁。成湯隆顯命,伊摯來如飛。周文獵渭濱,遂載呂望歸。符合如影響,先天天不違。輟耕綜地綱,解褐衿天維。元功配二王,芬馨世所稀。我皇敘群才,洪烈何巍巍。桓桓征四表,濟濟理萬機。神化感無方,髦才盈帝畿。丕顯惟昧旦,日新孔所諮。茂哉明聖德,日月同光輝。



 伯益



 伯益佐舜禹,職掌山與川。德侔十六相,思心入無間。智理周萬物,下知眾鳥言。
 黃雀應清化,翔習何翩翩。和鳴棲庭樹,徘徊雲日間。夏桀為無道,密網施山河。酷祝振纖網,當柰黃雀何。殷湯崇天德,去其三面羅。逍遙群飛來,鳴聲乃復和。朱雀作南宿,鳳皇統羽群。赤烏銜書至,天命瑞周文。神雀今來遊,為我受命君。嘉祥致天和,膏澤隆青雲。蘭風發芳氣,蓋世同其芬。



 釣竿



 釣竿何冉冉,甘餌芳且鮮。臨川運思心,
 微綸沈九泉。太公寶此術,乃在《靈祕》篇。機變隨物移,精妙貫未然。游魚驚著釣,潛龍飛戾天。戾天安所至?撫翼翔太清。太清一何異,兩儀出渾成。玉衡正三辰,造化賦群形。退願輔聖君,與神合其靈。我君弘遠略,天人不足并。天人初並時,昧昧何芒芒。日月有徵兆,文象興二皇。蚩尤亂生靈,黃帝用兵征萬方。逮夏禹而德衰,三代不及虞與唐。我皇盛德配堯舜,受禪即阼享天祥。率土蒙祐,靡不肅,庶事康。庶事康,
 穆穆明明。荷百祿,保無極,永太平。



 鼙舞,未詳所起,然漢代已施於燕享矣。傅毅、張衡所賦,皆其事也。舊曲有五篇,一、《關東有賢女》,二、《章和二年中》,三、《樂久長》,四、《四方皇》,五、《殿前生桂樹》,其辭並亡。曹植《鼙舞詩序》云:「故漢靈帝西園鼓吹有李堅者,能鼙舞,遭世荒亂,堅播越關西,隨將軍段煨。先帝聞其舊伎,下書召堅。堅年踰七十,中間廢而不為,又古曲甚多謬誤,異代之文,未必相襲,故依前曲作新歌五篇。」及泰始中,又製其辭焉。其舞故常二八,桓玄將僭位,尚書殿中郎袁明子啟增滿八佾。泰始中歌辭今列之後云。



 鼙舞歌詩五篇



 洪業篇當魏曲《明明魏皇帝》,古曲《關東有賢女》。



 宣文創洪業,盛德在泰始。聖皇應靈符,受命君四海。萬國何所樂?上有明天子。唐堯禪帝位,虞舜惟恭己。恭己正南面,道化與時移。大赦盪萌漸,文教被黃支。象天則地,體無為。聰明配日月,神聖參兩儀。雖有三凶類,靜言無所施。象天則地,體無為。稷契並佐命,伊呂升王臣。蘭芷登朝肆,下無失宿人。聲發響自應,表立景來附。哮闞順羈制,
 潛龍升天路。備物立成器,變通極其數。百事以時敘,萬機有常度。訓之以克讓,納之以忠恕。群下仰清風,海外同懽慕。象天則地,化雲布。昔日貴彫飾,今尚儉與素。昔日多纖介,今去情與故。象天則地,化雲布。濟濟大朝士,夙夜綜萬機。萬機無廢理,明明降訓諮。臣譬列星景,君配朝日輝。事業並通濟,功烈何巍巍。五帝繼三皇,三皇世所歸。聖德應期運,天天地不能違。仰之彌已高,猶天不可階。將復御龍氏,
 鳳皇在庭棲。



 天命篇當魏曲《太和有聖帝》,古曲《章和二年中》。



 聖祖受天命,應期輔魏皇。入則綜萬機,出則徵四方。朝廷無遺理,方表寧且康。道隆舜臣堯,積德踰太王。孟度阻窮險,造亂天一隅。神兵出不意,奉命致天誅。赦善罰有罪,元惡宗為虛。威風震勁蜀,武烈懾強吳。諸葛不知命,肆逆亂天常。擁徒十餘萬,數來寇邊疆。我皇邁神武,執鉞鎮雍涼。亮乃畏天威,未戰先仆僵。盈虛自然運,時變故多艱。東征陵海表,萬里克朝鮮。
 受遺齊七政,曹爽又滔天。群凶受誅殛,百祿咸來臻。黃華應福始,王凌為禍先。



 景皇篇當魏曲《魏歷長》,古曲《樂久長》。



 景皇帝,聰明命世生,盛德參天地。帝王道大,創基既已難,繼世亦未易。外則夏侯玄,內則張與李,三凶構逆,亂帝紀。順天行誅,窮其姦宄。邊將禦其漸,潛謀不得起。罪人咸伏辜,威風振萬里。平衡綜萬機,萬機無不理。召陵桓不君,內外何紛紛。眾小便成群,蒙昧恣心,治亂不分。睿聖獨斷,濟武常以文。順天惟廢立,掃霓披
 浮雲。雲霓既已闢,清和未幾間,羽檄首尾至,變起東南籓。儉欽為長蛇,外則憑吳蠻。萬國紛騷擾,戚戚天下懼不安。神武御六軍,我皇執鉞征。儉欽起壽春,前鋒據項城。出其不意,並縱奇兵。奇兵誠難御,廟勝實難支。兩軍不期遇,敵退計無施。豹騎惟武進,大戰沙陽陂。欽乃亡魂走,奔虜若雲披。天因赦有罪,東土放鯨鯢。



 大晉篇當魏曲《天生蒸民》,古曲《四方皇》。



 赫赫大晉,於穆文王。蕩蕩巍巍,道邁陶唐。世稱三皇五帝,及今重其光。九德克明,文既顯,
 武又彰。思弘六合,兼濟萬方。內舉元凱,朝政以綱。外簡武臣,時惟鷹揚。靡順不懷,逆命斯亡。仁配春日,威踰秋霜。濟濟多士,同茲蘭芳。唐虞至治,四凶滔天。致討儉欽,罔不肅虔。化感海內,海外來賓。獻其聲樂,並稱妾臣。西蜀猾夏,僭號方域。命將致討,委國稽服。吳人放命,馮海阻江。飛書告喻,響應來同。先王建萬國,九服為籓衛。亡秦壞諸侯,序祚不二世。歷代不能復,忽逾五百歲。我皇邁聖德,應期創典制。分土五等,籓國正封界。
 莘莘文武佐,千秋遘嘉會。洪澤溢區內,仁風翔海外。



 明君篇當魏曲《為君既不易》,古曲《殿前生桂樹》。



 明君御四海,聽鑒盡物情。顧望有譴罰,謁忠身必榮。蘭芷出荒野,萬里升紫庭。茨草穢堂階,掃截不得生。能否莫相蒙,百官正其名。恭己慎有為,有為無不成。闇君不自信,群下執異端。正直羅浸潤,姦臣奪其權。雖欲盡忠誠,結舌不敢言。結舌亦何憚,盡忠為身患。清流豈不潔,飛塵濁其源。岐路令人迷,未遠勝不還。忠臣立
 君朝,正色不顧身。邪正不並存,譬若胡與秦。胡秦有合時,邪正各異津。忠臣遇明君,乾乾惟日新。群目統在綱,眾星共北辰。設令遭暗主,斥退為凡人。雖薄供時用,白茅猶為珍。冰霜晝夜結,蘭桂摧為薪。邪臣多端變,用心何委曲。便辟順情指,動隨君所欲。偷安樂目前,不問清與濁。積偽罔時主,養交以持祿。言行恒相違,難饜甚谿谷。昧死射乾沒,覺露則滅族。



 拂舞,出自江左。舊云吳舞,檢其歌,非吳辭也。亦陳於殿庭。楊泓序云:「自到江南見《白符舞》,或言《白鳧鳩舞》,云有
 此來數十年矣。察其辭旨,乃是吳人患孫皓虐政,思屬晉也。」今列之於後云。



 拂舞歌詩五篇



 白鳩篇



 翩翩白鳩,再飛再鳴。懷我君德,來集君庭。白雀呈瑞,素羽明鮮。翔庭舞翼,以應仁乾。皎皎鳴鳩,或丹或黃。樂我君惠,振羽來翔。東壁餘光,魚在江湖。惠而不費,敬我微軀。策我良駟,習我驅馳。與君周旋,樂道忘飢。我心虛靜,我志霑濡。彈琴鼓瑟,聊以自娛。
 陵雲登臺,浮遊太清。攀龍附鳳,自望身輕。



 濟濟篇



 暢暢飛舞氣流芳,追念三五大綺黃。去失有,時可行,去來時同此未央。時冉冉,近桑榆,但當飲酒為歡娛。衰老逝,有何期,多憂耿耿內懷思,深池曠,魚獨希,願得黃浦眾所依。恩感人,世無比,悲歌且舞無極已。



 獨祿篇



 獨獨祿祿,水深泥濁。泥濁尚可,水深殺我。
 雍雍雙雁,遊戲田畔。我欲射雁,念子孤散。翩翩浮萍,得風搖輕。我心何合,與之同並。空床低幃,誰知無人。夜衣錦繡,誰別偽真。刀鳴鞘中,倚床無施。父冤不報,欲活何為。猛獸班班,遊戲山間。獸欲嚙人,不避豪賢。



 碣石篇



 東臨碣石,以觀滄海。水何淡淡,山島竦峙。樹木叢生,百草豐茂。秋風蕭瑟,洪波涌起。日月之行,若出其中。星漢燦爛,若出其裏。幸甚至哉,歌以詠志。《觀滄海》



 孟冬十月,北風徘徊。天氣肅清,繁霜霏霏。鵾雞晨鳴,鴈過南飛。鷙鳥潛藏,熊羆窟棲。耨鑮停置,農收積場。逆旅整設,以通賈商。幸甚至哉,歌以詠志。《冬十月》



 鄉土不同,河朔隆塞。流澌浮漂,舟船行難。錐不之地,豐籟深奧。水竭不流,冰堅可蹈。士隱者貧,勇俠輕非。心常嘆怨,戚戚多悲。幸甚至哉,歌以詠志。《土不同》



 神龜雖壽,猶有竟時。騰蛇乘霧,終為土灰。驥老伏櫪,志在千里。烈士暮年,壯心不已。
 盈縮之期,不但在天。養怡之福,可得永年。幸甚至哉,歌以詠志。《龜雖壽》



 淮南王篇



 淮南王,自言尊,百尺高樓與天連。後園鑿井銀作床,金瓶素綆汲寒漿。汲寒漿,飲少年,少年窈窕何能賢。揚聲悲歌音絕天。我欲渡河河無梁,願作雙黃鵠,還故鄉。還故鄉,入故里,徘徊故鄉,若身不已。繁舞奇歌無不泰,徘徊桑梓遊天外。



 鼓角橫吹曲。鼓,案《周禮》「以GW鼓鼓軍事」。角,說者云,蚩尤氏帥魑魅與黃帝戰於涿鹿,帝乃始命吹角為龍鳴以禦之。其後魏武北征烏丸,越沙漠而軍士思歸,於是減為中鳴,而尤更悲矣。



 胡角者,本以應胡笳之聲,後漸用之橫吹,有雙角,即胡樂也。張博望入西域,傳其法於西京,惟得《摩訶兜勒》一曲。李延年因胡曲更造新聲二十八解,乘輿以為武樂。後漢以給邊將,和帝時,萬人將軍得用之。魏晉以來,二十八解不復具存,用者有《黃鵠》、《隴頭》、《出關》、《入關》、《出塞》、《入塞》、《折楊柳》、《黃覃子》、《赤之楊》、《望行人》十曲。



 案魏晉之世,有孫氏
 善弘舊曲,宋識善擊節唱和,陳左善清歌,列和善吹笛,郝索善彈箏,朱生善琵琶,尤發新聲。故傅玄著書曰:「人若欽所聞而忽所見,不亦惑乎?設此六人生於上世,越今古而無儷,何但夔牙同契哉!」案此說,則自茲以後,皆孫朱等之遺則也。



 相和,漢舊歌也,絲竹更相和,執節者歌。本一部,魏明帝分為二,更遞夜宿。本十七曲,朱生、宋職、列和等復合之為十三曲。



 但歌,四曲,出自漢世。無絃節,作伎最先唱,一人唱,三人和。魏武帝尤好之。時有宋容華者,清徹好聲,善唱此曲,當
 時之特妙。自晉以來不復傳,遂絕。



 凡樂章古辭,今之存者,並漢世街陌謠謳,《江南可採蓮》、《烏生十五子》、《白頭吟》之屬也。吳歌雜曲並出江南,東晉以來,稍有增廣。



 《子夜歌》者,女子名子夜,造此聲。孝武太元中,瑯邪王軻之家有鬼歌《子夜》,則子夜是此時以前人也。



 《鳳將雛歌》者,舊曲也。應璩《百一詩》云「言是《鳳將雛》」,然則其來久矣。《前溪歌》者,車騎將軍沈充所制。



 《阿子》及《懽聞歌》者,穆帝升平初,歌畢輒呼「阿子,汝聞不?」
 語在《五行志》。後人衍其聲,以為此二曲。



 《團扇歌》者,中書令王氏與嫂婢有情,愛好甚篤,嫂捶撻婢過苦,婢素善歌,而氏好捉白團扇,故制此歌。



 《懊憹歌》者,隆安初俗聞訛謠之曲,語在《五行志》。



 《長史變》者,司徒左長史王廞臨敗所制。



 凡此諸曲,始皆徒歌,即而被之管絃。又有因絲竹金石,造歌以被之,魏世三調歌辭之類是也。



 《杯柈舞》,案太康中天下為《晉世寧舞》,務手以接杯柈反覆之。此則漢世惟有



 柈舞,而晉加之以杯,反覆之也。



 《公莫舞》,今之《巾舞》也。相傳云項莊劍舞,項伯以袖隔之,
 使不得害漢高祖,且語項莊云「公莫」!古人相呼曰公,言公莫害漢王也。今之用巾蓋像項伯衣袖之遺式。然案《琴操》有《公莫渡河曲》,然則其聲所從來已久,俗云項伯,非也。



 《白糸寧舞》,案舞辭有巾袍之言。糸寧本吳地所出,宜是吳舞也。晉《俳歌》又云:「皎皎白緒,節節為雙。」吳音呼緒為糸寧,疑白糸寧即白緒也。



 《鐸舞歌》一篇,《幡舞歌》一篇,《鼓舞伎》六曲,並陳於元會。



 後漢正旦,天子臨德陽殿受朝賀,舍利從西方來,戲於殿前,激水化成比目魚,跳躍嗽水,作霧翳日。畢,又化成龍,
 長八九丈,出水遊戲,炫耀日光。以兩大絲繩繫兩柱頭,相去數丈,兩倡女對舞,行於繩上,相逢切肩而不傾。魏晉訖江左,猶有《夏育扛鼎》、《巨象行乳》、《神龜抃舞》、《背負靈岳》、《桂樹白雪》、《畫地成川》之樂。



 成帝咸康七年,尚書蔡謨奏:「八年正會儀注,惟作鼓吹鐘鼓,其餘伎樂盡不作。」侍中張澄、給事黃門侍郎陳逵駮,以為「王者觀時設教,至於吉凶殊斷,不易之道也。今四方觀禮,陵有儐弔之位,庭奏宮懸之樂,二禮兼用,哀樂不分,體國經制,莫大於此」。詔曰:「今既以天下體大,禮從權宜,三正之饗,宜盡用吉禮也。至娛耳目之樂,所不
 忍聞,故闕之耳。事之大者,不過上壽酒,稱萬歲,已許其大,不足復闕鐘鼓鼓吹也。」



 澄、逵又啟:「今大禮雖降,事吉於朝。然儐弔顯於園陵,則未滅有哀;禮服定於典文,義無盡吉。是以咸寧之會,有撤樂之典,實先朝稽古憲章,垂式萬世者也。」詔曰:「若元日大饗,萬國朝宗,庭廢鐘鼓之奏,遂闕起居之節,朝無磬制之音,賓無蹈履之度,其於事義,不亦闕乎!惟可量輕重,以制事中。」



 散騎侍郎顧臻表曰:「臣聞聖王制樂,贊揚政道,養以仁義,防其淫佚,上享宗廟,下訓黎元,體五行之正音,協八風以陶物。宮聲正方而好義,角聲堅齊而率禮,絃歌鐘鼓金石之作
 備矣。故通神至化,有率舞之感,移風易俗,致和樂之極。末世之伎,設禮外之觀,逆行連倒,頭足入筥之屬,皮膚外剝,肝心內摧,敦彼行葦,猶謂勿踐,矧伊生靈,而不側愴。加四海朝覲,言觀帝庭,耳聆《雅》《頌》之聲,目睹威儀之序,足以蹋天,頭以履地,反天地之至順,傷彝倫之大方。今夷狄對岸,外禦為急,兵食七升,忘身赴難,過泰之戲,日廩五斗。方掃神州,經略中甸,若此之事,不可示遠。宜下太常,纂備雅樂,簫《韶》九成,惟新於盛運,功德頌聲,永著于來葉,此乃所以『燕及皇天,克昌厥後』者也。諸伎而傷人者,皆宜除之。流簡儉之德,邁康哉之詠,清風既行,
 下應如草,此之謂也。愚管之誠,惟垂採察!」於是除《高絙》、《紫鹿》、《跂行》、《鱉食》及《齊王卷衣》、《笮兒》等樂,又減其廩。其後復《高絙》、《紫鹿》焉。



\end{pinyinscope}