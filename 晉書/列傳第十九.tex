\article{列傳第十九}

\begin{pinyinscope}
阮籍
 \gezhu{
  兄子咸咸子瞻瞻弟孚從子脩族弟放放弟裕}
 嵇康向秀劉伶謝鯤胡毋輔之
 \gezhu{
  子謙之}
 畢卓王尼羊曼光逸



 阮籍,字嗣宗,陳留尉氏人也。父瑀,魏丞相掾,知名於世。籍容貌瑰傑,志氣宏放,傲然獨得,任性不羈,而喜怒不形於色。或閉戶視書,累月不出;或登臨山水,經日忘歸。博覽群籍,尤好《莊》《老》。嗜酒能嘯,善彈琴。當其得意,忽忘形骸。時人多謂之癡,惟族兄文業每嘆服之,以為勝己,由是咸共稱異。



 籍嘗隨叔父至東郡,兗州刺史王昶請
 與相見,終日不開一言,自以不能測。太尉蔣濟聞其有雋才而辟之,籍詣都亭奏記曰:「伏惟明公以含一之德,據上台之位,英豪翹首,俊賢抗足。開府之日,人人自以為掾屬;辟書始下,而下走為首。昔子夏在於西河之上,而文侯擁篲;鄒子處於黍谷之陰,而昭王陪乘。夫布衣韋帶之士,孤居特立,王公大人所以禮下之者,為道存也。今籍無鄒、卜之道,而有其陋,猥見採擇,無以稱當。方將耕於東皋之陽,輸黍稷之餘稅。負薪疲病,足力不彊,補吏之召,非所克堪。乞回謬恩,以光清舉。」初,濟恐籍不至,得記欣然。遣卒迎之,而籍已去,濟大怒。於是鄉親共
 喻之,乃就吏。後謝病歸。復為尚書郎,少時,又以病免。及曹爽輔政,召為參軍。籍因以疾辭,屏於田里。歲餘而爽誅,時人服其遠識。宣帝為太傅,命籍為從事中郎。及帝崩,復為景帝大司馬從事中郎。高貴鄉公即位,封關內侯,徙散騎常侍。



 籍本有濟世志,屬魏、晉之際,天下多故,名士少有全者,籍由是不與世事,遂酣飲為常。文帝初欲為武帝求婚於籍,籍醉六十日,不得言而止。鐘會數以時事問之,欲因其可否而致之罪,皆以酣醉獲免。及文帝輔政,籍嘗從容言於帝曰:「籍平生曾游東平,樂其風土。」帝大悅,即拜東平相。籍乘驢到郡,壞府舍屏鄣,使
 內外相望,法令清簡,旬日而還。帝引為大將軍從事中郎。有司言有子殺母者,籍曰:「嘻!殺父乃可,至殺母乎!」坐者怪其失言。帝曰:「殺父,天下之極惡,而以為可乎?」籍曰:「禽獸知母而不知父,殺父,禽獸之類也。殺母,禽獸之不若。」眾乃悅服。



 籍聞步兵廚營人善釀,有貯酒三百斛,乃求為步兵校尉。遺落世事,雖去佐職,恒游府內,朝宴必與焉。會帝讓九錫,公卿將勸進,使籍為其辭。籍沈醉忘作,臨詣府,使取之,見籍方據案醉眠。使者以告,籍便書案,使寫之,無所改竄。辭甚清壯,為時所重。



 籍雖不拘禮教,然發言玄遠,口不臧否人物。性至孝,母終,正與人圍
 棋,對者求止,籍留與決賭。既而飲酒二斗,舉聲一號,吐血數升。及將葬,食一蒸肫,飲二斗酒,然後臨訣,直言窮矣,舉聲一號,因又吐血數升,毀瘠骨立,殆致滅性。裴楷往弔之,籍散髮箕踞,醉而直視,楷弔唁畢便去。或問楷:「凡弔者,主哭,客乃為禮。籍既不哭,君何為哭?」楷曰:「阮籍既方外之士,故不崇禮典。我俗中之士,故以軌儀自居。」時人歎為兩得。籍又能為青白眼,見禮俗之士,以白眼對之。及嵇喜來弔,籍作白眼,喜不懌而退。喜弟康聞之,乃齎酒挾琴造焉,籍大悅,乃見青眼。由是禮法之士疾之若仇,而帝每保護之。



 籍嫂嘗歸寧,籍相見與別。或譏
 之,籍曰:「禮豈為我設邪!」鄰家少婦有美色,當壚沽酒。籍嘗詣飲,醉,便臥其側。籍既不自嫌,其夫察之,亦不疑也。兵家女有才色,未嫁而死。籍不識其父兄,徑往哭之,盡哀而還。其外坦蕩而內淳至,皆此類也。時率意獨駕,不由徑路,車迹所窮,輒慟哭而反。嘗登廣武,觀楚、漢戰處,歎曰:「時無英雄,使豎子成名!」登武牢山,望京邑而歎,於是賦《豪傑詩》。景元四年冬卒,時年五十四。



 籍能屬文,初不留思。作《詠懷詩》八十餘篇,為世所重。著《達莊論》,敘無為之貴。文多不錄。



 籍嘗於蘇門山遇孫登,與商略終古及棲神導氣之術,登皆不應,籍因長嘯而退。至半嶺,聞
 有聲若鸞鳳之音,響乎巖谷,乃登之嘯也。遂歸著《大人先生傳》,其略曰:「世人所謂君子,惟法是修,惟禮是克。手執圭璧,足履繩墨。行欲為目前檢,言欲為無窮則。少稱鄉黨,長聞鄰國。上欲圖三公,下不失九州牧。獨不見群虱之處褌中,逃乎深縫,匿乎壞絮,自以為吉宅也。行不敢離縫際,動不敢出褌襠,自以為得繩墨也。然炎丘火流,焦邑滅都,群虱處於褌中而不能出也。君子之處域內,何異夫虱之處褌中乎!」此亦籍之胸懷本趣也。



 子渾,字長成,有父風。少慕通達,不飾小節。籍謂曰:「仲容已豫吾此流,汝不得復爾!」太康中,為太子庶子。



 咸字仲容。父熙,武都太守。咸任達不拘,與叔父籍為竹林之游,當世禮法者譏其所為。咸與籍居道南,諸阮居道北,北阮富而南阮貧。七月七日,北阮盛曬衣服,皆錦綺粲目,咸以竿挂大布犢鼻於庭。人或怪之,答曰:「未能免俗,聊復爾耳!」



 歷仕散騎侍郎。山濤舉咸典選,曰:「阮咸貞素寡欲,深識清濁,萬物不能移。若在官人之職,必絕於時。」武帝以咸耽酒浮虛,遂不用。太原郭奕高爽有識量,知名於時,少所推先,見咸心醉,不覺歎焉。而居母喪,縱情越禮。素幸姑之婢,姑當歸於夫家,初云留婢,既而自從去。時方有客,咸聞之,遽借客馬追婢,既及,與婢累
 騎而還,論者甚非之。



 咸妙解音律,善彈琵琶。雖處世不交人事,惟共親知弦歌酣宴而已。與從子修特相善,每以得意為歡。諸阮皆飲酒,咸至,宗人間共集,不復用杯觴斟酌,以大盆盛酒,圓坐相向,大酌更飲。時有群豕來飲其酒,咸直接去其上,便共飲之。群從昆弟莫不以放達為行,籍弗之許。荀勖每與咸論音律,自以為遠不及也,疾之,出補始平太守。以壽終。二子:瞻、孚。



 瞻字千里。性清虛寡欲,自得於懷。讀書不甚研求,而默識其要,遇理而辯,辭不足而旨有餘。善彈琴,人聞其能,多往求聽,不問貴賤長幼,皆為彈之。神氣沖和,而不知
 向人所在。內兄潘岳每令鼓琴,終日達夜,無忤色。由是識者歎其恬澹,不可榮辱矣。舉止灼然。見司徒王戎,戎問曰:「聖人貴名教,老莊明自然,其旨同異?」瞻曰:「將無同。」戎咨嗟良久,即命辟之。時人謂之「三語掾」。太尉王衍亦雅重之。瞻嘗群行,冒熱渴甚,逆旅有井,眾人競趨之,瞻獨逡巡在後,須飲者畢乃進,其夷退無競如此。



 東海王越鎮許昌,以瞻為記室參軍,與王承、謝鯤、鄧攸俱在越府。越與瞻等書曰:「禮,年八歲出就外傅,明始可以加師訓之則;十年曰幼學,明可漸先王之教也。然學之所入淺,體之所安深。是以閑習禮容,不如式瞻儀度;諷誦遺
 言,不若親承音旨。小兒毗既無令淑之質,不聞道德之風,望諸君時以閑豫,周旋誨接。」



 永嘉中,為太子舍人。瞻素執無鬼論,物莫能難,每自謂此理足可以辯正幽明。忽有一客通名詣瞻,寒溫畢,聊談名理。客甚有才辯,瞻與之言,良久及鬼神之事,反覆甚苦。客遂屈,乃作色曰:「鬼神,古今聖賢所共傳,君何得獨言無!即僕便是鬼。」於是變為異形,須臾消滅。瞻默然,意色大惡。後歲餘,病卒於倉垣,時年三十。



 孚字遙集。其母,即胡婢也。孚之初生,其姑取王延壽《魯靈光殿賦》曰「胡人遙集於上楹」而以字焉。初辟太傅府,
 遷騎兵屬。避亂渡江,元帝以為安東參軍。蓬髮飲酒,不以王務嬰心。時帝既用申、韓以救世,而孚之徒未能棄也。雖然,不以事任處之。轉丞相從事中郎。終日酣縱,恆為有司所按,帝每優容之。



 瑯邪王裒為車騎將軍,鎮廣陵,高選綱佐,以孚為長史。帝謂曰:「卿既統軍府,郊壘多事,宜節飲也。」孚答曰:「陛下不以臣不才,委之以戎旅之重。臣FC勉從事,不敢有言者,竊以今王蒞鎮,威風赫然,皇澤遐被,賊寇斂迹,氛昆既澄,日月自朗,臣亦何可爵火不息?正應端拱嘯詠,以樂當年耳。」遷黃門侍郎、散騎常侍。嘗以金貂換酒,復為所司彈劾,帝宥之。轉太子中
 庶子、左衛率,領屯騎校尉。



 明帝即位,遷侍中。從平王敦,賜爵南安縣侯。轉吏部尚書,領東海王師,稱疾不拜。詔就家用之,尚書令郗鑒以為非禮。帝曰:「就用之誠不快,不爾便廢才。」及帝疾大漸,溫嶠入受顧命,過孚,要與同行。升車,乃告之曰:「主上遂大漸,江左危弱,實資群賢,共康世務。卿時望所歸,今欲屈卿同受顧託。」孚不答,固求下車,嶠不許。垂至臺門,告嶠內迫,求暫下,便徒步還家。



 初,祖約性好財,孚性好屐,同是累而未判其得失。有詣約,見正料財物,客至,屏當不盡,餘兩小簏,以著背後,傾身障之,意未能平。或有詣阮,正見自蠟屐,因自嘆曰:「未
 知一生當著幾量屐!」神色甚閑暢。於是勝負始分。



 咸和初,拜丹陰尹。時太后臨朝,政出舅族。孚謂所親曰:「今江東雖累世,而年數實淺。主幼時艱,運終百六,而庾亮年少,德信未孚,以吾觀之,將兆亂矣。」會廣州刺史劉顗卒,遂苦求出。王導等以孚疏放,非京尹才,乃除都督交、廣、寧三州軍事、鎮南將軍、領平越中郎將、廣州刺史、假節。未至鎮,卒,年四十九。尋而蘇峻作逆,識者以為知幾。無子,從孫廣嗣。



 修字宣子。好《易》《老》,善清言。嘗有論鬼神有無者,皆以人死者有鬼,修獨以為無,曰:「今見鬼者云著生時衣服,若
 人死有鬼,衣服有鬼邪?」論者服焉。後遂伐社樹,或止之,修曰:「若社而為樹,伐樹則社移;樹而為社,伐樹則社亡矣。」



 性簡任,不修人事。絕不喜見俗人,遇便舍去。意有所思,率爾褰裳,不避晨夕,至或無言,但欣然相對。常步行,以百錢掛杖頭,至酒店,便獨酣暢。雖當世富貴而不肯顧,家無儋石之儲,宴如也。與兄弟同志,常自得於林阜之間。



 王衍當時談宗,自以論《易》略盡,然有所未了,研之終莫悟,每云「不知比沒當見能通之者不」。衍族子敦謂衍曰:「阮宣子可與言。」衍曰:「吾亦聞之,但未知其亹癖之處定何如耳!」及與修談,言寡而旨暢,衍乃歎服焉。



 梁國
 張偉志趣不常,自隱於屠釣,修愛其才美,而知其不真。偉後為黃門郎、陳留內史,果以世事受累。



 修居貧,年四十餘未有室,王敦等斂錢為婚,皆名士也,時慕之者求入錢而不得。



 修所著述甚寡,嘗作《大鵬贊》曰:「蒼蒼大鵬,誕自北溟。假精靈鱗,神化以生。如雲之翼,如山之形。海運水擊,扶搖上征。翕然層舉,背負太清。志存天地,不屑唐庭。鸴鳩仰笑,尺鷃所輕。超世高逝,莫知其情。」



 王敦時為鴻臚卿,謂修曰:「卿常無食,鴻臚丞差有祿,能作不?」修曰:「亦復可爾耳!」遂為之。轉太傅行參軍、太子洗馬。避亂南行,至西陽期思縣,為賊所害,時年四十二。



 放字思度。祖略,齊郡太守。父顗,淮南內史。放少與孚並知名。中興,除太學博士、太子中舍人、庶子。時雖戎車屢駕,而放侍太子,常說《老》《莊》,不及軍國。明帝甚友愛之。轉黃門侍郎,遷吏部郎,在銓管之任,甚有稱績。



 時成帝幼沖,庾氏執政,放求為交州,乃除監交州軍事、揚威將軍、交州刺史。行達寧浦,逢陶侃將高寶平梁碩自交州還,放設饌請寶,伏兵殺之。寶眾擊放,敗走,保簡陽城,得免。到州少時,暴發渴,見寶為祟,遂卒,朝廷甚悼惜之,年四十四。追贈廷尉。



 放素知名,而性清約,不營產業,為吏部郎,不免飢寒。王導、庾亮以其名士,常供給衣食。子晞之,
 南頓太守。



 裕字思曠。宏達不及放,而以德業知名。弱冠辟太宰掾。大將軍王敦命為主簿,甚被知遇。裕以敦有不臣之心,乃終日酣觴,以酒廢職。敦謂裕非當世實才,徒有虛譽而已,出為溧陽令,復以公事免官。由是得違敦難,論者以此貴之。



 咸和初,除尚書郎。時事故之後,公私弛廢,裕遂去職還家,居會稽剡縣。司徒王導引為從事中郎,固辭不就。朝廷將欲征之,裕知不得已,乃求為王舒撫軍長史。舒薨,除吏部郎,不就。即家拜臨海太守,少時去職。司空郗鑒請為長史,詔徵祕書監,皆以疾辭。復除東陽
 太守。尋徵侍中,不就。還剡山,有肥遁之志。有以問王羲之,羲之曰:「此公近不驚寵辱,雖古之沈冥,何以過此!」人云,裕骨氣不及逸少,簡秀不如真長,韶潤不如仲祖,思致不如殷浩,而兼有諸人之美。成帝崩,裕赴山陵,事畢便還。諸人相與追之,裕亦審時流必當逐己,而疾去,至方山不相及。劉惔歎曰:「我入東,正當泊安石渚下耳,不敢復近思曠傍。」



 裕雖不博學,論難甚精。嘗問謝萬云:「未見《四本論》,君試為言之。」萬敘說既畢,裕以傅嘏為長,於是構辭數百言,精義入微,聞者皆嗟味之。裕嘗以人不須廣學,正應以禮讓為先故終日靜默,無所修綜,而物
 自宗焉。在剡曾有好車,借無不給。有人葬母,意欲借而不敢言。後裕聞之,乃嘆曰:「吾有車而使人不敢借,何以車為!」遂命焚之。



 在東山久之,復徵散騎常侍,領國子祭酒。俄而復以為金紫光祿大夫,領瑯邪王師。經年敦逼,並無所就。御史中丞周閔奏裕及謝安違詔累載,並應有罪,禁錮終身,詔書貰之。或問裕曰:「子屢辭徵聘,而宰二郡,何邪?」裕曰:「雖屢辭王命,非敢為高也。吾少無宦情,兼拙於人間,既不能躬耕自活,必有所資,故曲躬二郡。豈以騁能,私計故耳。」年六十二卒。三子:傭、寧、普。



 傭,早卒。寧,鄱陽太守。普,驃騎諮議參軍。傭子歆之,中領軍。寧子
 腆,祕書監。腆弟萬齡及歆之子彌之,元熙中並列顯位。



 嵇康,字叔夜,譙國銍人也。其先姓奚,會稽上虞人,以避怨,徙焉。銍有嵇山,家于其側,因而命氏。兄喜,有當世才,歷太僕、宗正。康早孤,有奇才,遠邁不群。身長七尺八寸,美詞氣,有風儀,而土木形骸,不自藻飾,人以為龍章鳳姿,天質自然。恬靜寡慾,含垢匿瑕,寬簡有大量。學不師受,博覽無不該通,長好《老》《莊》。與魏宗室婚,拜中散大夫。常修養性服食之事,彈琴詠詩,自足於懷。以為神仙稟之自然,非積學所得,至於導養得理,則安期、彭祖之倫
 可及,乃著《養生論》。又以為君子無私,其論曰:「夫稱君子者,心不措乎是非,而行不違乎道者也。何以言之?夫氣靜神虛者,心不存於矜尚;體亮心達者,情不繫於所欲。矜尚不存乎心,故能越名教而任自然;情不繫於所欲,故能審貴賤而通物情。物情順通,故大道無違;越名任心,故是非無措也。是故言君子則以無措為主,以通物為美;言小人則以匿情為非,以違道為闕。何者?匿情矜吝,小人之至惡;虛心無措,君子之篤行也。是以大道言『及吾無身,吾又何患』。無以生為貴者,是賢於貴生也。由斯而言,夫至人之用心,固不存有措矣。故曰『君子行
 道,忘其為身』,斯言是矣。君子之行賢也,不察於有度而後行也;任心無邪,不議於善而後正也;顯情無措,不論於是而後為也。是故傲然忘賢,而賢與度會;忽然任心,而心與善遇;儻然無措,而事與是俱也。」其略如此。蓋其胸懷所寄,以高契難期,每思郢質。所與神交者惟陳留阮籍、河內山濤,豫其流者河內向秀、沛國劉伶、籍兄子咸、瑯邪王戎,遂為竹林之游,世所謂「竹林七賢」也。戎自言與康居山陽二十年,未嘗見其喜慍之色。



 康嘗採藥游山澤,會其得意,忽焉忘反。時有樵蘇者遇之,咸謂為神。至汲郡山中見孫登,康遂從之游。登沈默自守,無所言
 說。康臨去,登曰:「君性烈而才雋,其能免乎!」康又遇王烈,共入山,烈嘗得石髓如飴,即自服半,餘半與康,皆凝而為石。又於石室中見一卷素書,遽呼康往取,輒不復見。烈乃嘆曰:「叔夜志趣非常而輒不遇,命也!」其神心所感,每遇幽逸如此。



 山濤將去選官,舉康自代。康乃與濤書告絕,曰:



 聞足下欲以吾自代,雖事不行,知足下故不知之也。恐足下羞庖人之獨割,引尸祝以自助,故為足下陳其可否。



 老子、莊周,吾之師也,親居賤職;柳下惠、東方朔,達人也,安乎卑位。吾豈敢短之哉!又仲尼兼愛,不羞執鞭;子文無欲卿相,而三為令尹,是乃君子思濟物之意
 也。所謂達能兼善而不渝,窮則自得而無悶。以此觀之,故知堯、舜之居世,許由之巖棲,子房之佐漢,接輿之行歌,其揆一也。仰瞻數君,可謂能遂其志者也。故君子百行,殊途同致,循性而動,各附所安。故有「處朝廷而不出,入山林而不反」之論。且延陵高子臧之風,長卿慕相如之節,意氣所托,亦不可奪也。



 吾每讀《尚子平、臺孝威傳》,慨然慕之,想其為人。加少孤露,母兄驕恣,不涉經學,又讀《老》《莊》,重增其放,故使榮進之心日頹,任逸之情轉篤。阮嗣宗口不論人過,吾每師之,而未能及。至性過人,與物無傷,惟飲酒過差耳,至為禮法之士所繩,疾之如仇
 仇,幸賴大將軍保持之耳。吾以不如嗣宗之資,而有慢弛之闕;又不識物情,闇於機宜;無萬石之慎,而有好盡之累;久與事接,疵釁日興,雖欲無患,其可得乎!



 又聞道士遺言,餌術黃精,令人久壽,意甚信之。游山澤,觀魚鳥,心甚樂之。一行作吏,此事便廢,安能舍其所樂,而從其所懼哉!



 夫人之相知,貴識其天性,因而濟之。禹不逼伯成子高,全其長也;仲尼不假蓋於子夏,護其短也。近諸葛孔明不迫元直以入蜀,華子魚不彊幼安以卿相,此可謂能相終始,真相知者也。自卜已審,若道盡途殫則已耳,足下無事冤之令轉於溝壑也。



 吾新失母兄之歡,
 意常悽切。女年十三,男年八歲,未及成人,況復多疾,顧此悢悢,如何可言。今但欲守陋巷,教養子孫,時時與親舊敘離闊,陳說平生,濁酒一盃,彈琴一曲,志意畢矣,豈可見黃門而稱貞哉!若趣欲共登王途,期於相致,時為歡益,一旦迫之,必發狂疾。自非重仇,不至此也。既以解足下,並以為別。



 此書既行,知其不可羈屈也。性絕巧而好鍛。宅中有一柳樹甚茂,乃激水圜之,每夏月,居其下以鍛。東平呂安服康高致,每一相思,輒千里命駕,康友而善之。後安為兄所枉訴,以事繫獄,辭相證引,遂復收康。康性慎言行,一旦縲紲,乃作《幽憤詩》,曰:



 嗟餘薄祜,少
 遭不造,哀煢靡識,越在襁褓。母兄鞠育,有慈無威,恃愛肆姐,不訓不師。爰及冠帶,憑寵自放,抗心希古,任其所尚。託好《莊》《老》,賤物貴身,志在守樸,養素全真。



 曰予不敏,好善闇人,子玉之敗,屢增惟塵。大人含弘,藏垢懷恥。人之多僻,政不由己。惟此褊心,顯明臧否;感悟思愆,怛若創磐。欲寡其過,謗議沸騰,性不傷物,頻致怨憎。昔慚柳惠,今愧孫登,內負宿心,外恧良朋。仰慕嚴、鄭,樂道閑居,與世無營,神氣晏如。



 咨予不淑,嬰累多虞。匪降自天,實由頑疏,理弊患結,卒致囹圄。對答鄙訊,縶此幽阻,實恥訟冤,時不我與。雖曰義直,神辱志沮,澡身滄浪,曷云能
 補。雍雍鳴鴈,厲翼北游,順時而動,得意忘憂。嗟我憤歎,曾莫能疇。事與願違,遘茲淹留,窮達有命,亦又何求?



 古人有言,善莫近名。奉時恭默,咎悔不生。萬石周慎,安親保榮。世務紛紜,只攪餘情,安樂必誡,乃終利貞。煌煌靈芝,一年三秀;予獨何為,有志不就。懲難思復,心焉內疚,庶勖將來,無馨無臭。採薇山阿,散髮巖岫,永嘯長吟,頤神養壽。



 初,康居貧,嘗與向秀共鍛於大樹之下,以自贍給。潁川鐘會,貴公子也,精練有才辯,故往造焉。康不為之禮,而鍛不輟。良久會去,康謂曰:「何所聞而來?何所見而去?」會曰:「聞所聞而來,見所見而去。」會以此憾之。及是,
 言於文帝曰:「嵇康,臥龍也,不可起。公無憂天下,顧以康為慮耳。」因譖「康欲助毌丘儉,賴山濤不聽。昔齊戮華士,魯誅少正卯,誠以害時亂教,故聖賢去之。康、安等言論放蕩,非毀典謨,帝王者所不宜容。宜因釁除之,以淳風俗」。帝既暱聽信會,遂並害之。



 康將刑東市,太學生三千人請以為師,弗許。康顧視日影,索琴彈之,曰:「昔袁孝尼嘗從吾學《廣陵散》,吾每靳固之,《廣陵散》於今絕矣!」時年四十。海內之士,莫不痛之。帝尋悟而恨焉。初,康嘗游於洛西,暮宿華陽亭,引琴而彈。夜分,忽有客詣之,稱是古人,與康共談音律,辭致清辯,因索琴彈之,而為《廣陵散》,
 聲調絕倫,遂以授康,仍誓不傳人,亦不言其姓字。



 康善談理,又能屬文,其高情遠趣,率然玄遠。撰上古以來高士為之傳贊,欲友其人於千載也。又作《太師箴》,亦足以明帝王之道焉。復作《聲無哀樂論》,甚有條理。子紹,別有傳。



 向秀,字子期,河內懷人也。清悟有遠識,少為山濤所知,雅好老莊之學。莊周著內外數十篇,歷世才士雖有觀者,莫適論其旨統也,秀乃為之隱解,發明奇趣,振起玄風,讀之者超然心悟,莫不自足一時也。惠帝之世,郭象
 又述而廣之,儒墨之迹見鄙,道家之言遂盛焉。始,秀欲注,嵇康曰:「此書詎復須注,正是妨人作樂耳。」及成,示康曰:「殊復勝不?」又與康論養生,辭難往復,蓋欲發康高致也。



 康善鍛,秀為之佐,相對欣然,傍若無人。又共呂安灌園於山陽。康既被誅,秀應本郡計入洛。文帝問曰:「聞有箕山之志,何以在此?」秀曰:「以為巢許狷介之士,未達堯心,豈足多慕。」帝甚悅。秀乃自此役,作《思舊賦》云:



 餘與嵇康、呂安居止接近,其人並有不羈之才,嵇意遠而疏,呂心曠而放,其後並以事見法。嵇博綜伎藝,於絲竹特妙,臨當就命,顧視日影,索琴而彈之。逝將西邁,經其舊廬。
 于時日薄虞泉,寒冰淒然。鄰人有吹笛者,發聲寥亮。追想曩昔游宴之好,感音而歎,故作賦曰:



 將命適於遠京兮,遂旋反以北徂。濟黃河以汎舟兮,經山陽之舊居。瞻曠野之蕭條兮,息余駕乎城隅。踐二子之遺迹兮,歷窮巷之空廬。歎《黍離》之愍周兮,悲《麥秀》於殷墟。惟追昔以懷今兮,心徘徊以躊躇。棟宇在而弗毀兮,形神逝其焉如。昔李斯之受罪兮,歎黃犬而長吟。悼嵇生之永辭兮,顧日影而彈琴。託運遇於領會兮,寄餘命於寸陰。聽鳴笛之慷慨兮,妙聲絕而復尋。佇駕言其將邁兮,故援翰以寫心。



 後為散騎侍郎,轉黃門侍郎、散騎常侍,在朝不任
 職,容迹而已。卒於位。二子:純、悌。



 劉伶,字伯倫,沛國人也。身長六尺,容貌甚陋。放情肆志,常以細宇宙齊萬物為心。澹默少言,不妄交游,與阮籍、嵇康相遇,欣然神解,攜手入林。初不以家產有無介意。常乘鹿車,攜一壺酒,使人荷鍤而隨之,謂曰:「死便埋我。」其遺形骸如此。嘗渴甚,求酒於其妻。妻捐酒毀器,涕泣諫曰:「君酒太過,非攝生之道,必宜斷之。」伶曰:「善!吾不能自禁,惟當祝鬼神自誓耳。便可具酒肉。」妻從之。伶跪祝曰:「天生劉伶,以酒為名。一飲一斛,五斗解酲。婦兒之言,
 慎不可聽。」仍引酒御肉,隗然復醉。嘗醉與俗人相忤,其人攘袂奮拳而往。伶徐曰:「雞肋不足以安尊拳。」其人笑而止。



 伶雖陶兀昏放,而機應不差。未嘗厝意文翰,惟著《酒德頌》一篇。其辭曰:



 有大人先生,以天地為一朝,萬期為須臾,日月為扃牖,八荒為庭衢。行無轍迹,居無室廬,幕天席地,縱意所如。止則操卮執觚,動則挈榼提壺,惟酒是務,焉知其餘。有貴介公子、搢紳處士,聞吾風聲,議其所以,乃奮袂攘襟,怒目切齒,陳說禮法,是非蜂起。先生於是方捧甕承槽,銜盃漱醪,奮髯箕踞,枕曲藉糟,無思無慮,其樂陶陶。兀然而醉,怳爾而醒。靜聽不聞雷霆
 之聲,熟視不睹泰山之形。不覺寒暑之切肌,利欲之感情。俯觀萬物,擾擾焉若江海之載浮萍。二豪侍側焉,如蜾蠃之與螟蛉。



 嘗為建威參軍。泰始初對策,盛言無為之化。時輩皆以高第得調,伶獨以無用罷。竟以壽終。



 謝鯤,字幼輿,陳國陽夏人也。祖纘,典農中郎將。父衡,以儒素顯,仕至國子祭酒。鯤少知名,通簡有高識,不修威儀,好《老》《易》,能歌,善鼓琴,王衍、嵇紹並奇之。



 永興中,長沙王乂入輔政,時有疾鯤者,言其將出奔。乂欲鞭之,鯤解衣就罰,曾無忤容。既舍之,又無喜色。太傅東海王越聞
 其名,辟為掾,任達不拘,尋坐家僮取官稿除名。于時名士王玄、阮修之徒,並以鯤初登宰府,便至黜辱,為之歎恨。鯤聞之,方清歌鼓琴,不以屑意,莫不服其遠暢,而恬於榮辱。鄰家高氏女有美色,鯤嘗挑之,女投梭,折其兩齒。時人為之語曰:「任達不已,幼輿折齒。」鯤聞之,敖然長嘯曰:「猶不廢我嘯歌。」越尋更辟之,轉參軍事。鯤以時方多故,乃謝病去職,避地於豫章。嘗行經空亭中夜宿,此亭舊每殺人。將曉,有黃衣人呼鯤字令開戶,鯤憺然無懼色,便於窗中度手牽之,胛斷,視之,鹿也,尋血獲焉。爾後此亭無復妖怪。



 左將軍王敦引為長史,以討杜弢功
 封咸亭侯。母憂去職,服闋,遷敦大將軍長史。時王澄在敦坐,見鯤談話無勌,惟嘆謝長史可與言,都不眄敦,其為人所慕如此。鯤不徇功名,無砥礪行,居身於可否之間,雖自處若穢,而動不累高。敦有不臣之迹,顯於朝野。鯤知不可以道匡弼,乃優游寄遇,不屑政事,從容諷議,卒歲而已。每與畢卓、王尼、阮放、羊曼、桓彞、阮孚等縱酒,敦以其名高,雅相賓禮。



 嘗使至都,明帝在東宮見之,甚相親重。問曰:「論者以君方庾亮,自謂何如?」答曰:「端委廟堂,使百僚準則,鯤不如亮。一丘一壑,自謂過之。」溫嶠嘗謂鯤子尚曰:「尊大君豈惟識量淹遠,至於神鑒沈深,雖
 諸葛瑾之喻孫權不過也。」



 及敦將為逆,謂鯤曰:「劉隗奸邪,將危社稷。吾欲除君側之惡,匡主濟時,何如?」對曰:「隗誠始禍,然城狐社鼠也。」敦怒曰:「君庸才,豈達大理。」出鯤為豫章太守,又留不遣,藉其才望,逼與俱下。敦至石頭,歎曰:「吾不復得為盛德事矣。」鯤曰:「何為其然?但使自今以往,日忘日去耳。」初,敦謂鯤曰:「吾當以周伯仁為尚書令,戴若思為僕射。」及至都,復曰:「近來人情何如?」鯤對曰:「明公之舉,雖欲大存社稷,然悠悠之言,實未達高義。周顗、戴若思,南北人士之望,明公舉而用之,群情帖然矣。」是日,敦遣兵收周、戴,而鯤弗知,敦怒曰:「君粗疏邪!二子
 不相當,吾已收之矣。」鯤與顗素相親重,聞之愕然,若喪諸己。參軍王驕以敦誅顗,諫之甚切,敦大怒,命斬嶠,時人士畏懼,莫敢言者。鯤曰:「明公舉大事,不戮一人。嶠以獻替忤旨,便以釁鼓,不亦過乎!」敦乃止。



 敦既誅害忠賢,而稱疾不朝,將還武昌。鯤喻敦曰:「公大存社稷,建不世之勛,然天下之心實有未達。若能朝天子,使君臣釋然,萬物之心於是乃服。杖眾望以順群情,盡沖退以奉主上,如斯則勛侔一匡,名垂千載矣。」敦曰:「君能保無變乎?」對曰:「鯤近日入覲,主上側席,遲得見公,宮省穆然,必無虞矣。公若入朝,鯤請侍從。」敦勃然曰:「正復殺君等數百
 人,亦復何損於時!」竟不朝而去。是時朝望被害,皆為其憂。而鯤推理安常,時進正言。敦既不能用,內亦不悅。軍還,使之郡,涖政清肅,百姓愛之。尋卒官,時年四十三。敦死後,追贈太常,謚曰康。子尚嗣,別有傳。



 胡毋輔之,字彥國,泰山奉高人也。高祖班,漢執金吾。父原,練習兵馬,山濤稱其才堪邊任,舉為太尉長史,終河南令。輔之少擅高名,有知人之鑒。性嗜酒,任縱不拘小節。與王澄、王敦、庾敳俱為太尉王衍所暱,號曰四友。澄嘗與人書曰:「彥國吐佳言如鋸木屑,霏霏不絕,誠為後
 進領袖也。」



 辟別駕、太尉掾,並不就。以家貧,求試守繁昌令,始節酒自厲,甚有能名。遷尚書郎。豫討齊王冏,賜爵陰平男。累轉司徒左長史。復求外出,為建武將軍、樂安太守。與郡人光逸晝夜酣飲,不視郡事。成都王穎為太弟,召為中庶子,遂與謝鯤、王澄、阮修、王尼、畢卓俱為放達。



 嘗過河南門下飲,河南騶王子博箕坐其傍,輔之叱使取火。子博曰:「我卒也,惟不乏吾事則已,安復為人使!」輔之因就與語,歎曰:「吾不及也!」薦之河南尹樂廣,廣召見,甚悅之,擢為功曹。其甄拔人物若此。



 東海王越聞輔之名,引為從事中郎,復補振威將軍、陳留太守。王彌經
 其郡,輔之不能討,坐免官。尋除寧遠將軍、揚州刺史,不之職,越復以為右司馬、本州大中正。越薨,避亂渡江,元帝以為安東將軍諮議祭酒,遷揚武將軍、湘州刺史、假節。到州未幾卒,時年四十九。子謙之。



 謙之字子光。才學不及父,而傲縱過之。至酣醉,常呼其父字,輔之亦不以介意,談者以為狂。輔之正酣飲,謙之規而厲聲曰:「彥國年老,不得為爾!將令我尻背東壁。」輔之歡笑,呼入與共飲。其所為如此。年未三十卒。



 畢卓字茂世,新蔡鮦陽人也。父諶,中書郎。卓少希放達,為
 胡毋輔之所知。太興末,為吏部郎,常飲酒廢職。比舍郎釀熟,卓因醉夜至其甕間盜飲之,為掌酒者所縛,明旦視之,乃畢吏部也,遽釋其縛。卓遂引主人宴於甕側,致醉而去。卓嘗謂人曰:「得酒滿數百斛船,四時甘味置兩頭,右手持酒杯,左手持蟹螯,拍浮酒船中,便足了一生矣。」及過江,為溫嶠平南長史,卒官。



 王尼,字孝孫,城陽人也,或云河內人。本兵家子,寓居洛陽,卓犖不羈。初為護軍府軍士,胡毋輔之與瑯邪王澄、北地傅暢、中山劉輿、潁川荀邃、河東裴遐迭屬河南功
 曹甄述及洛陽令曹攄請解之。攄等以制旨所及,不敢。輔之等齎羊酒詣護軍門,門吏疏名呈護軍,護軍歎曰:「諸名士持羊酒來,將有以也。」尼時以給府養馬,輔之等入,遂坐馬廄下,與尼炙羊飲酒,醉飽而去,竟不見護軍。護軍大驚,即與尼長假,因免為兵。東嬴公騰辟為車騎府舍人,不就。時尚書何綏奢侈過度,尼謂人曰:「綏居亂世,矜豪乃爾,將死不久。」人曰:「伯蔚聞言,必相危害。」尼曰:「伯蔚比聞我語,已死矣。」未幾,綏果為東海王越所殺。初入洛,尼詣越不拜。越問其故,尼曰:「公無宰相之能,是以不拜。」因數之,言甚切。又云:「公負尼物。」越大驚曰:「寧有是
 也?」尼曰:「昔楚人亡布,謂令尹盜之。今尼屋舍資財,悉為公軍人所略,尼今飢凍,是亦明公之負也。」越大笑,即賜絹五十匹。諸貴人聞,競往餉之。洛陽陷,避亂江夏。時王登為荊州刺史,遇之甚厚。尼早喪婦,止有一子。無居宅,惟畜露車,有牛一頭,每行,輒使子御之,暮則共宿車上。常歎曰:「滄海橫流,處處不安也。」俄而澄卒,荊土饑荒,尼不得食,乃殺牛壞車,煮肉啖之。既盡,父子俱餓死。



 羊曼,字祖延,太傅祜兄孫也。父暨,陽平太守。曼少知名,本州禮命,太傅辟,皆不就。避難渡江,元帝以為鎮東參
 軍,轉丞相主簿,委以機密。歷黃門侍郎、尚書吏部郎、晉陵太守,以公事免。曼任達頹縱,好飲酒。溫嶠、庾亮、阮放、桓彞同志友善,並為中興名士。時州里稱陳留阮放為宏伯,高平郗鑒為方伯,泰山胡毋輔之為達伯,濟陰卞壺為裁伯,陳留蔡謨為朗伯,阮孚為誕伯,高平劉綏為委伯,而曼為濌伯,凡八人,號兗州八伯,蓋擬古之八雋也。



 王敦既與朝廷乖貳,羈錄朝士,曼為右長史。曼知敦不臣,終日酣醉,諷議而已。敦以其士望,厚加禮遇,不委以事,故得不涉其難。敦敗,代阮孚為丹陽尹。時朝士過江初拜官,相飾供饌。曼拜丹陽,客來早者得佳設,日宴
 則漸罄,不復及精,隨客早晚而不問貴賤。有羊固拜臨海太守,竟日皆美,雖晚至者猶獲盛饌。論者以固之豐腆,乃不如曼之真率。



 蘇峻作亂,加前將軍,率文武守雲龍門。王師不振,或勸曼避峻。曼曰:「朝廷破敗,吾安所求生?」勒眾不動,為峻所害,年五十五。峻平,追贈太常。子賁嗣,少知名,尚明帝女南郡悼公主,除祕書郎,早卒。弟聃。



 聃字彭祖。少不經學,時論皆鄙其凡庸。先是,兗州有八伯之號,其後更有四伯。大鴻臚陳留江泉以能食為穀伯,豫章太守史疇以大肥為笨伯,散騎郎高平張嶷以狡妄為猾伯,而聃以狼戾為瑣伯,蓋擬古之四凶。



 聃初
 辟元帝丞相府,累遷廬陵太守。剛克粗暴,恃國戚,縱恣尤甚,睚眥之嫌輒加刑殺。疑郡人簡良等為賊,殺二百餘人,誅及嬰孩,所髡鎖復百餘。庾亮執之,歸于京都。有司奏聃罪當死,以景獻皇后是其祖姑,應八議。成帝詔曰:「此事古今所無,何八議之有!猶未忍肆之市朝,其賜命獄所。」兄子賁尚公主,自表求解婚。詔曰:「罪不相及,古今之令典也。聃雖極法,於賁何有!其特不聽離婚。」琅邪太妃山氏,聃之甥也,入殿叩頭請命。王導又啟:「聃罪不容恕,宜極重法。山太妃憂戚成疾,陛下罔極之恩,宜蒙生全之宥。」於是詔下曰:「太妃惟此一舅,發言摧咽,乃至
 吐血,情慮深重。朕往丁荼毒,受太妃撫育之恩,同於慈親。若不堪難忍之痛,以致頓弊,朕亦何顏以寄。今便原聃生命,以慰太妃渭陽之思。」於是除名。頃之,遇疾,恒見簡良等為祟,旬日而死。



 光逸,字孟祖,樂安人也。初為博昌小吏,縣令使逸送客,冒寒舉體凍濕,還遇令不在,逸解衣炙之,入令被中臥。令還,大怒,將加嚴罰。逸曰:「家貧衣單,沾濕無可代。若不暫溫,勢必凍死,奈何惜一被而殺一人乎!君子仁愛,必不爾也,故寢而不疑。」令奇而釋之。後為門亭長,迎新令
 至京師。胡毋輔之與荀邃共詣令家,望見逸,謂邃曰:「彼似奇才。」便呼上車,與談良久,果俊器。令怪客不入,吏白與光逸語。令大怒,除逸名,斥遣之。



 後舉孝廉,為州從事,棄官投輔之。輔之時為太傅越從事中郎,薦逸於越,越以門寒而不召。越後因閑宴,責輔之無所舉薦。輔之曰:「前舉光逸,公以非世家不召,非不舉也。」越即辟焉。書到郡縣,皆以為誤,審知是逸,乃備禮遣之。尋以世難,避亂渡江,復依輔之。初至,屬輔之與謝鯤、阮放、畢卓、羊曼、桓彞、阮孚散髮裸袒,閉室酣飲已累日。逸將排戶入,守者不聽,逸便於戶外脫衣露頭於狗竇中窺之而大叫。輔
 之驚曰:「他人決不能爾,必我孟祖也。」遽呼入,遂與飲,不捨晝夜。時人謂之八達。元帝以逸補軍諮祭酒。中興建,為給事中,卒官。



 史臣曰:夫學非常道,則物靡不通;理有忘言,則在情斯遣。其進也,撫俗同塵,不居名利;其退也,餐和履順,以保天真。若乃一其本原,體無為之用,分其華葉,開寓言之道,是以伯陽垂範,鳴謙置式,欲崇諸己,先下於人,猶大樂無聲,而蹌鸞斯應者也。莊生放達其旨,而馳辯無窮;棄彼榮華,則俯輕爵位,懷其道術,則顧蔑王公;舐痔兼車,鳴鳶吞腐。以茲自口,於焉玩物,殊異虛舟,有同攘臂。
 嵇、阮竹林之會,劉、畢芳樽之友,馳騁莊門,排登李室。若夫儀天布憲,百官從軌,經禮之外,棄而不存。是以帝堯縱許由於埃盍之表,光武舍子陵於潺湲之瀨,松蘿低舉,用以優賢,巖水澄華,茲焉賜隱;臣行厥志,主有嘉名。至於嵇康遺巨源之書,阮氏創先生之傳,軍諮散髮,吏部盜樽,豈以世疾名流,茲焉自垢?臨鍛灶而不迴,登廣武而長歎,則嵇琴絕響,阮氣徒存。通其旁徑,必凋風俗;召以效官,居然尸素。軌躅之外,或有可觀者焉。咸能符契情靈,各敦終始,愴神交於晚笛,或相思而動駕。史臣是以拾其遺事,附于篇云。



 贊曰:老篇爰植,孔教提衡。各存其趣,道貴無名。相彼非禮,遵乎達生。秋水揚波,春雲斂映。旨酒厥德,憑虛其性。不玩斯風,誰虧王政?



\end{pinyinscope}