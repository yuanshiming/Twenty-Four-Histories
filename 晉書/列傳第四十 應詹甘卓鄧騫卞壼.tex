\article{列傳第四十 應詹甘卓鄧騫卞壼}

\begin{pinyinscope}
應詹甘卓鄧騫卞壼
 \gezhu{
  從父兄敦劉超鐘雅}



 應詹,字思遠,汝南南頓人,魏侍中璩之孫也。詹幼孤,為祖母所養。年十餘歲,祖母又終,居喪毀頓,杖而後起,遂以孝聞。家富於財,年又稚弱,乃請族人共居,委以資產,情若至親,世以此異焉。弱冠知名,性質素弘雅,物雖犯而弗之校,以學藝文章稱。司徒何劭見之曰:「君子哉若人!」



 初辟公府,為太子舍人。趙王倫以為征東長史。倫誅,
 坐免。成都王穎辟為掾。時驃騎從事中郎諸葛玫委長沙王乂奔鄴,盛稱乂之非。玫浮躁有才辯,臨漳人士無不詣之。詹與玫有舊,歎曰:「諸葛成林,何與樂毅之相詭乎!」卒不見之。玫聞甚愧。鎮南大將軍劉弘,詹之祖舅也,請為長史,謂之曰:「君器識弘深,後當代老子於荊南矣。」仍委以軍政。弘著績漢南,詹之力也。遷南平太守。



 王澄為荊州,假詹督南平、天門、武陵三郡軍事。及洛陽傾覆,詹攘袂流涕,勸澄赴援。澄使詹為檄,詹下筆便成,辭義壯烈,見者慷慨,然竟不能從也。天門、武陵谿蠻並反,詹討降之。時政令不一,諸蠻怨望,並謀背叛。詹召蠻酋,破
 銅券與盟,由是懷詹,數郡無虞。其後天下大亂,詹境獨全。百姓歌之曰:「亂離既普,殆為灰朽。僥倖之運,賴茲應后。歲寒不凋,孤境獨守。拯我塗炭,惠隆丘阜。潤同江海,恩猶父母。」鎮南將軍山簡復假詹督五郡軍事。會蜀賊杜疇作亂,來攻詹郡,力戰摧之。尋與陶侃破杜弢於長沙,賊中金寶溢目,詹一無所取,唯收圖書,莫不歎之。元帝假詹建武將軍,王敦又上詹監巴東五郡軍事,賜爵潁陽鄉侯。陳人王沖擁眾荊州,素服詹名,迎為刺史。詹以沖等無賴,棄還南平,沖亦不怨。其得人情如此。遷益州刺史,領巴東監軍。詹之出郡也,士庶攀車號泣,若戀
 所生。



 俄拜後軍將軍。詹上疏陳便宜,曰:「先王設官,使君有常尊,臣有定卑,上無茍且之志,下無覬覦之心。下至亡奏,罷侯置守,本替末陵,綱紀廢絕。漢興,雖未能興復舊典,猶雜建侯守,故能享年享世,殆參古迹。今大荒之後,制度改創,宜因斯會,釐正憲則,先舉盛德元功以為封首,則聖世之化比隆唐虞矣。」又曰:「性相近,習相遠,訓導之風,宜慎所好。魏正始之間,蔚為文林。元康以來,賤經尚道,以玄虛宏放為夷達,以儒術清儉為鄙俗。永嘉之弊,未必不由此也。今雖有儒官,教養未備,非所以長育人才,納之軌物也。宜修辟雍,崇明教義,先令國子受
 訓,然後皇儲親臨釋奠,則普天尚德,率土知方矣。」元帝雅重其才,深納之。



 頃之,出補吳國內史,以公事免。鎮北將軍劉隗出鎮,以詹為軍司。加散騎常侍,累遷光祿勳。詹以王敦專制自樹,故優游諷詠,無所標明。及敦作逆,明帝問詹計將安出。詹厲然慷慨曰:「陛下宜奮赫斯之威,臣等當得負戈前驅,庶憑宗廟之靈,有征無戰。如其不然,王室必危。」帝以詹為都督前鋒軍事、護軍將軍、假節,都督朱雀橋南。賊從竹格渡江,詹與建威將軍趙胤等擊敗之,斬賊率杜發,梟首數千級。賊平,封觀陽縣侯,食邑一千六百戶,賜絹五千匹。上疏讓曰:「臣聞開國承
 家,光啟土宇,唯令德元功乃宜封錫。臣雖忝當一隊,策無微略,勞不汗馬。猥以疏賤,倫亞親密,暫廁被練,列勤司勛。乞迴謬恩,聽其所守。」不許。



 遷使持節、都督江州諸軍事、平南將軍、江州刺史。詹將行,上疏曰:



 夫欲用天下之智力者,莫若使天下信之也。商鞅移木,豈禮也哉?有由而然。自經荒弊,綱紀頹陵,清直之風既澆,糟秕之俗猶在,誠宜濯以滄浪之流,漉以吞舟之網,則幽顯明別,於變時雍矣。弘濟茲務,在乎官人。今南北雜錯,屬託者無保負之累,而輕舉所知,此博采所以未精,職理所以多闕。今凡有所用,宜隨其能否而與舉主同乎褒貶,則
 人有慎舉之恭,官無廢職之吝。昔冀缺有功,胥臣蒙先茅之賞;子玉敗軍,子文受蒍賈之責。古既有之,今亦宜然。漢朝使刺史行部,乘傳奏事,猶恐不足以辨彰幽明,弘宣政道,故復有繡衣直指。今之艱弊,過於往昔,宜分遣黃、散若中書郎等循行天下,觀採得失,舉善彈違,斷截茍且,則入不敢為非矣。漢宣帝時,二千石有居職修明者,則入為公卿;其不稱職免官者,皆還為平人。懲勸必行,故歷世長久。中間以來,遷不足競,免不足懼。或有進而失意,退而得分。蒞官雖美,當以素論降替;在職實劣,直以舊望登敘。校游談為多少,不以實事為先後。以
 此責成,臣未見其兆也。今宜峻左降舊制,可二千石免官,三年乃得敘用,長史六年,戶口折半,道里倍之。此法必明,便天下知官難得而易失,必人慎其職,朝無惰官矣。都督可課佃二十頃,州十頃,郡五頃,縣三頃。皆取文武吏醫卜,不得撓亂百姓。三臺九府,中外諸軍,有可減損,皆令附農。市息末伎,道無游人,不過一熟,豐穰可必。然後重居職之俸,使祿足以代耕。頃大事之後,遐邇皆想宏略,而寂然未副,宜早振綱領,肅起群望。



 時王敦新平,人情未安,詹撫而懷之,莫不得其歡心,百姓賴之。



 疾篤,與陶侃書曰:「每憶密計,自沔入湘,頡頏繾綣,齊好斷
 金。子南我東,忽然一紀,其間事故,何所不有。足下建功嶠南,旋鎮舊楚。吾承乏幸會,來忝此州,圖與足下進共竭節本朝,報恩幼主,退以申尋平生,纏綿舊好。豈悟時不我與,長即幽冥,永言莫從,能不慨悵!今神州未夷,四方多難,足下年德並隆,功名俱盛,宜務建洪範,雖休勿休,至公至平,至謙至順,即自天祐之,吉無不利。人之將死,其言也善,足下察吾此誠。」以咸和六年卒,時年五十三。冊贈鎮南大將軍、儀同三司,謚曰烈,祠以太牢。子玄嗣,位至散騎侍郎。玄弟誕,有器幹,歷六郡太守、龍驤將軍,追贈冀州刺史。



 初,京兆韋泓喪亂之際,親屬遇飢疫
 並盡,客遊洛陽,素聞詹名,遂依託之。詹與分甘共苦,情若弟兄。遂隨從積年,為營伉儷,置居宅,並薦之於元帝曰:「自遭喪亂,人士易操,至乃任運固窮,耿介守節者鮮矣。伏見議郎韋泓,年三十八,字元量,執心清沖,才識備濟,躬耕隴畝,不煩人役,靜默居常,不豫政事。昔年流移,來在詹境,經寇喪資,一身特立,短褐不掩形,菜蔬不充朝,而抗志彌厲,不遊非類。顏回稱不改其樂,泓有其分。明公輔亮皇室,恢維宇宙,四門開闢,英彥鳧藻,收春華於京輦,採秋實於巖藪。而泓抱璞荊山,未剖和璧。若蒙銓召,付以列曹,必能協隆鼎味,緝熙庶績者也。」帝即辟
 之。自後位至少府卿。既受詹生成之惠,詹卒,遂製朋友之服,哭止宿草,追趙氏祀程嬰、杵臼之義,祭詹終身。



 甘卓,字季思,丹陽人,秦丞相茂之後也。曾祖寧,為吳將。祖述,仕吳為尚書。父昌,太子太傅。吳平,卓退居自守。郡命主簿、功曹,察孝謙,州舉秀才,為吳王常侍。討石冰,以功賜爵都亭侯。東海王越引為參軍,出補離狐令。卓見天下大亂,棄官東歸,前至歷陽,與陳敏相遇。敏甚悅,共圖縱橫之計,遂為其子景娶卓女,共相結託。會周唱義,密使錢廣攻敏弟昶,敏遣卓討廣,頓朱雀橋南。會廣
 殺昶,告丹陽太守顧榮共邀說卓。卓素敬服榮,且以昶死懷懼,良久乃從之。遂詐疾迎女,斷橋,收船南岸,共滅敏,傳首於京都。



 元帝初渡江,授卓前鋒都督、揚威將軍、歷陽內史。其後討周馥,征杜弢,屢經苦戰,多所擒獲。以前後功,進爵南鄉侯,拜豫章太守。尋遷湘州刺史,將軍如故。復進爵於湖侯。



 中興初,以邊寇未靜,學校陵遲,特聽不試孝廉,而秀才猶依舊策試。卓上疏以為:「答問損益,當須博通古令,明達政體,必求諸墳索,乃堪其舉。臣所忝州往遭寇亂,學校久替,人士流播,不得比之餘州。策試之由,當藉學功,謂宜同孝廉例,申與期限。」疏奏,
 朝議不許。卓於是精加隱括,備禮舉桂陽谷儉為秀才。儉辭不獲命,州厚禮遣之。諸州秀才聞當考試,皆憚不行,惟儉一人到臺,遂不復策試。儉恥其州少士,乃表求試,以高第除中郎。儉少有志行,寒苦自立,博涉經史。于時南土凋荒,經籍道息,儉不能遠求師友,唯在家研精。雖所得實深,未有名譽,又恥衒耀取達,遂歸,終身不仕,卒於家。



 卓尋遷安南將軍、梁州刺史、假節、督沔北諸軍,鎮襄陽。卓外柔內剛,為政簡惠,善於綏撫,估稅悉除,市無二價。州境所有魚池,先恒責稅,卓不收其利,皆給貧民,西土稱為惠政。



 王敦稱兵,遣使告卓。卓乃偽許,而心不
 同之。及敦升舟,而卓不赴,使參軍孫雙詣武昌諫止敦。敦聞雙言,大驚曰:「甘侯前與吾語云何,而更有異!正當慮吾危朝廷邪?吾今下唯除姦凶耳。卿還言之,事濟當以甘侯作公。」雙還報卓,卓不能決。或說卓且偽許敦,待敦至都而討之。卓曰:「昔陳敏之亂,吾亦先從後圖,而論者謂懼逼面謀之。雖吾情本不爾,而事實有似,心恒愧之。今若復爾,誰能明我!」時湘州刺史譙王承遣主簿鄧騫說卓曰:「劉大連雖乘權寵,非有害於天下也。大將軍以其私憾稱兵象魏,雖託討亂之名,實失天下之望,此忠臣義士匡救之時也。昔魯連匹夫,猶懷蹈海之志,況
 受任方伯,位同體國者乎!今若因天人之心,唱桓文之舉,杖大順以掃逆節,擁義兵以勤王室,斯千載之運,不可失也。」卓笑曰:「桓文之事,豈吾所能。至於盡力國難,乃其心也。當共詳思之。」參軍李梁說卓曰:「昔隗囂亂隴右,竇融保河西以歸光武,今日之事,有似於此。將軍有重名於天下,但當推亡固存,坐而待之。使大將軍勝,方當崇將軍以方面之重;如其不勝,朝廷必以將軍代之。何憂不富貴,而釋此廟勝,決存亡於一戰邪!」騫謂梁曰:「光武創業,中國未平,故隗囂斷隴右,竇融兼河西,各據一方,鼎足之勢,故得文服天子,從容顧望。及海內已定,君
 臣正位,終於隴右傾覆,河西入朝。何則?向之文服,義所不容也。今將軍之於本朝,非竇融之喻也。襄陽之於大府,非河西之固也。且人臣之義,安忍國難而不陳力,何以北面於天子邪!使大將軍平劉隗,還武昌,增石城之守,絕荊湘之粟,將軍安歸乎?勢在人手,而曰我處廟勝,未之聞也。」卓尚持疑未決,騫又謂卓曰:「今既不義舉,又不承大將軍檄,此必至之禍,愚智所見也。且議者之所難,以彼彊我弱,是不量虛實者也。今大將軍兵不過萬餘,其留者不能五千,而將軍見眾既倍之矣。將軍威名天下所聞也,此府精銳,戰勝之兵也。擁彊眾,藉威名,杖
 節而行,豈王含所能御哉!溯流之眾,勢不自救,將軍之舉武昌,若摧枯拉朽,何所顧慮乎!武昌既定,據其軍實,鎮撫二州,施惠士卒,使還者如歸,此呂蒙所以剋敵也。如是,大將軍可不戰而自潰。今釋必勝之策,安坐以待危亡,不可言知計矣。願將軍熟慮之。」



 時敦以卓不至,慮在後為變,遣參軍樂道融苦要卓俱下。道融本欲背敦,因說卓襲之,語在融傳。卓既素不欲從敦,得道融說,遂決曰:「吾本意也。」乃與巴東監軍柳純、南平太守夏侯承、宜都太守譚該等十餘人,俱露檄遠近,陳敦肆逆,率所統致討。遣參軍司馬讚、孫雙奉表詣臺,參軍羅英至廣
 州,與陶侃剋期,參軍鄧騫、虞沖至長沙,令譙王承堅守。征西將軍戴若思在江西,先得卓書,表上之,臺內皆稱萬歲。武昌驚,傳卓軍至,人皆奔散。詔書遷卓為鎮南大將軍、侍中、都督荊梁二州諸軍事、荊州牧,梁州刺史如故,陶侃得卓信,即遣參軍高寶率兵下。



 卓雖懷義正,而性不果毅,且年老多疑,計慮猶豫,軍次豬口,累旬不前。敦大懼,遣卓兄子行參軍仰求和,謝卓曰:「君此自是臣節,不相責也。吾家計急,不得不爾。想便旋軍襄陽,當更結好。」時王師敗績,敦求臺騶虞幡駐卓。卓聞周顗、戴若思遇害,流涕謂仰曰:「吾之所憂,正謂今日。每得朝廷
 人書,常以胡寇為先,不悟忽有蕭牆之禍。且使聖上元吉,太子無恙,吾臨敦上流,亦未敢便危社稷。吾適徑據武昌,敦勢逼,必劫天子以絕四海之望。不如還襄陽,更思後圖。」即命旋軍。都尉秦康說卓曰:「今分兵取敦不難,但斷彭澤,上下不得相赴,自然離散,可一戰擒也。將軍既有忠節,中道而廢,更為敗軍將,恐將軍之下亦各便求西還,不可得守也。」卓不能從。樂道融亦日夜勸卓速下。卓性先寬和,忽便彊塞,徑還襄陽,意氣騷擾,舉動失常,自照鏡不見其頭,視庭樹而頭在樹上,心甚惡之。其家金櫃鳴,聲似槌鏡,清而悲。巫云:「金櫃將離,是以悲鳴。」
 主簿何無忌及家人皆勸令自警。卓轉更很愎,聞諫輒怒。方散兵使大佃,而不為備。功曹榮建固諫,不納。襄陽太守周慮等密承敦意,知卓無備,詐言湖中多魚,勸卓遣左右皆捕魚,乃襲害卓于寢,傳首於敦。四子散騎郎蕃等皆被害。太寧中,追贈驃騎將軍,謚曰敬。



 鄧騫,子長真,長沙人。少有志氣,為鄉鄰所重。常推誠行己,能以正直全於多難之時。刺史譙王承命為主簿,便說甘卓。卓留為參軍,欲與同行,以母老辭卓而反。承為魏乂所敗,以虞悝兄弟為承黨,乂盡誅之,而求騫甚急。
 鄉人皆為之懼,騫笑曰:「欲用我耳。彼新得州,多殺忠良,是其求賢之時,豈以行人為罪!」乃往詣乂。乂喜曰:「君所謂古之解揚也。」以為別駕。騫有節操忠信,兼識量弘遠,善與人交,久而益敬。太尉庾亮稱之,以為長者。歷武陵、始興太守,遷大司農,卒於官。



 卞壼,字望之,濟陰冤句人也。祖統,琅邪內史。父粹,以清辯鑒察稱。兄弟六人並登宰府,世稱「卞氏六龍,玄仁無雙」。玄仁,粹字也。弟裒,嘗忤其郡將,郡將怒訐其門內之私,粹遂以不訓見譏議,陵遲積年。惠帝初,為尚書郎。楊
 駿執政,人多附會,而粹正直不阿。及駿誅,超拜右丞,封成陽子,稍遷至右軍將軍。張華之誅,粹以華婿免官。齊王冏輔政,為侍中、中書令,進爵為公。及長沙王乂專權,粹立朝正色,乂忌而害之。初,粹如廁,見物若兩眼,俄而難作。



 壼弱冠有名譽,司兗二州、齊王冏辟,皆不就。遇家禍,還鄉里。永嘉中,除著作郎,襲父爵。征東將軍周馥請為從事中郎,不就。遭本州傾覆,東依妻兄徐州刺史裴盾。盾以壼行廣陵相。元帝鎮建鄴,召為從事中郎,委以選舉,甚見親杖。出為明帝東中郎長史。遭繼母憂,既葬,起復舊職,累辭不就。元帝遣中使敦逼,壼箋自陳曰:



 壼
 天性狷狹,不能和俗,退以情事,欲畢志家門。亡父往為中書令,時壼蒙大例,望門見辟,信其所執,得不祗就。門戶遇禍,迸竄易名,得存視息,私志有素。加嬰極難,流寄蘭陵,為茍晞所召,恐見逼迫,依下邳裴盾,又見假授,思暫之郡,規得托身。尋蒙見召,為從事中郎,豈曰貪榮,直欲自致,規暫恭命,行當乞退。屬華軼之難,不敢自陳。軼既梟懸,壼亦嬰病,具自歸聞,未蒙恕遣。世子北征,選寵顯望,復以無施,忝充元佐。榮則榮矣,實非素懷。顧以命重人輕,不敢辭憚。聞西臺召壼為尚書郎,實欲因此以避賢路,未及陳誠,奄丁窮罰。



 壼年九歲,為先母弟表所
 見孤背。十二,蒙亡母張所見覆育。壼以陋賤,不能榮親,家產屢空,養道多闕,存無歡娛,終不備禮,拊心永恨,五內抽割。於公無效如彼,私情艱苦如此,實無情顏昧冒榮進。若廢壼一人,江北便有傾危之慮,壼居事之日功績以隆者,誠不得私其身。今東中郎岐嶷自然,神明日茂,軍司馬、諸參佐並以明德宣力王事,壼之去留,會無損益。賀循、謝端、顧景、丁琛、傅晞等皆荷恩命,高枕家門。壼委質二府,漸冉五載,考效則不能已彰,論心則頻累恭順,奈何哀孤之日不見愍恕哉!



 帝以其辭苦,不奪其志。



 服闋,為世子師。壼前後居師佐之任,盡匡輔之節,一
 府貴而憚焉。中興建,補太子中庶子,轉散騎常侍,侍講東宮。遷太子詹事,以公事免。尋復職,轉御史中丞。忠於事上,權貴屏迹。



 時淮南小中正王式繼母,前夫終,更適式父。式父終,喪服訖,議還前夫家。前夫家亦有繼子,奉養至終,遂合葬於前夫。式自云:「父臨終,母求去,父許諾。」於是制出母齊衰期。壼奏曰:「就如式父臨終許諾,必也正名,依禮為無所據。若夫有命,須顯七出之責,當存時棄之,無緣以絕義之妻留家制服。若式父臨困謬亂,使去留自由者,此必為相要以非禮,則存亡無所得從,式宜正之以禮。魏顆父命不從其亂,陳乾昔欲以二婢子
 殉,其子以非禮不從,《春秋》、《禮記》善之。並以妾勝,猶正以禮,況其母乎!式母於夫,生事奉終,非為既絕之妻。夫亡制服,不為無義之婦。自云守節,非為更嫁。離絕之斷,在夫沒之後。夫之既沒,是其從子之日,而式以為出母,此母以子出也。致使存無所容居,沒無所託也。寄命於他人之門,埋尸於無名之冢。若式父亡後,母尋沒於式家,必不以為出母明矣。許諾之命一耳,以為母于同居之時,至沒前子之門而不以為母,此為制離絕於二居,裁出否於意斷。離絕之斷,非式而誰!假使二門之子皆此母之生,母戀前子,求去求絕,非禮於後家,還反又非禮
 於前門,去不可去,還不可還,則為無寄之人也。式必內盡匡諫,外極防閑,不絕明矣。何至守不移於至親,略情禮於假繼乎!繼母如母,聖人之教。式為國士,閏門之內犯禮違義,開闢未有,於父則無追亡之善,於母則無孝敬之道,存則去留自由,亡則合葬路人,可謂生事不以禮,死葬不以禮者也。虧損世教,不可以居人倫詮正之任。案侍中、司徒、臨潁公組敷宣五教,實在任人,而含容違禮,曾不貶黜,揚州大中正、侍中、平望亭侯曄,淮南大中正、散騎侍郎弘,顯執邦論,朝野取信,曾不能率禮正違,崇孝敬之教,並為不勝其任。請以見事免組、曄、弘官,
 大鴻臚削爵土,廷尉結罪。」疏奏,詔特原組等,式付鄉邑清議,廢棄終身。壼遷吏部尚書。王含之難,加中軍將軍。含滅,以功封建興縣公,尋遷領軍將軍。



 明帝不豫,領尚書令,與王導等俱受顧命輔幼主。復拜右將軍,加給事中、尚書令。帝崩,成帝即位,群臣進璽,司徒王導以疾不至。壼正色於朝曰:「王公豈社稷之臣邪!大行大殯,嗣皇未立,寧是人臣辭疾之時!」導聞之,乃輿疾而至。皇太后臨朝,壼與庾亮對直省中,共參機要。時召南陽樂謨為郡中正,潁川庾怡為廷尉評。謨、怡各稱父命不就。壼奏曰:「人無非父而生,職無非事而立。有父必有命,居職必
 有悔。有家各私其子,此為王者無人,職不軌物,官不立政。如此則先聖之言廢,五教之訓塞,君臣之道散,上下之化替矣。樂廣以平夷稱,庾氏以忠篤顯,受寵聖世,身非己有,況及後嗣而可專哉!所居之職若順夫群心,則戰戍者之父母皆當以命子,不以處也。若順謨父之意,則人皆不為郡中正,人倫廢矣。順怡父之意,人皆不為獄官,則刑辟息矣。凡如是者,其可聽歟?若不可聽,何以許謨、怡之得稱父命乎!此為謨以名父子可虧法,怡是親戚可以自專。以此二途服人示世,臣所未悟也。宜一切班下,不得以私廢公。絕其表疏,以為永制。」朝議以
 為然。謨、怡不得已,各居所職。是時王導稱疾不朝,而私送車騎將軍郗鑒,壼奏以導虧法從私,無大臣之節。御史中丞鐘雅阿撓王典,不加準繩,並請免官。雖事寢不行,舉朝震肅。壼斷裁切直,不畏彊禦,皆此類也。



 壼乾實當官,以褒貶為己任,勤於吏事,欲軌正督世,不肯茍同時好。然性不弘裕,才不副意,故為諸名士所少,而無卓爾優譽。明帝深器之,於諸大臣而最任職。阮孚每謂之曰;「卿恒無閑泰,常如含瓦石,不亦勞乎?」壼曰:「諸君以道德恢弘,風流相尚,執鄙吝者,非壼而誰!」時貴遊子弟多慕王澄、謝鯤為達,壼厲色於朝曰:「悖禮傷教,罪莫斯甚!
 中朝傾覆,實由於此。」欲奏推之。王導、庾亮不從,乃止,然而聞者莫不折節。時王導以勛德輔政,成帝每幸其宅,嘗拜導婦曹氏。侍中孔坦密表不宜拜。導聞之曰:「王茂弘駑痾耳,若卞望之之巖巖,刁玄亮之察察,戴若思之峰岠,當敢爾邪!」壼廉潔儉素,居甚貧約。息當婚,詔特賜錢五十萬,固辭不受。後患面創,累乞解職。



 拜光祿大夫,加散騎常侍。時庾亮將征蘇峻,言於朝曰:「峻狼子野心,終必為亂。今日征之,縱不順命,為禍猶淺。若復經年,為惡滋蔓,不可復制。此是朝錯勸漢景帝早削七國事也。」當時議者無以易之。壼固爭,謂亮曰:「峻擁彊兵,多藏無
 賴,且逼近京邑,路不終朝,一旦有變,易為蹉跌。宜深思遠慮,恐未可倉卒。」亮不納。壼知必敗,與平南將軍溫嶠書曰:「元規召峻意定,懷此於邑。溫生足下,柰此事何!吾今所慮,是國之大事,且峻已出狂意,而召之更速,必縱其群惡以向朝廷。朝廷威力誠桓桓,交須接鋒履刃,尚不知便可即擒不?王公亦同此情。吾與之爭甚懇切,不能如之何。本出足下為外籓任,而今恨出足下在外。若卿在內俱諫,必當相從。今內外戒嚴,四方有備,峻凶狂必無所至耳,恐不能使無傷,如何?」壼司馬任台勸壼宜畜良馬,以備不虞。壼笑曰:「以順逆論之,理無不濟。若萬
 一不然,豈須馬哉!」峻果稱兵。壼復為尚書令、右將軍、領右衛將軍,餘官如故。



 峻至東陵口,詔以壼都督大桁東諸軍事、假節,復加領軍將軍、給事中,壼率郭默、趙胤等與峻大戰於西陵,為峻所破。壼與鐘雅皆退還,死傷者以千數。壼、雅並還節,詣闕謝罪。峻進攻青溪,壼與諸軍距擊,不能禁。賊放火燒宮寺,六軍敗績。壼時發背創,猶未合,力疾而戰,率厲散眾及左右吏數百人,攻賊麾下,苦戰,遂死之,時年四十八。二子、盱見父沒,相隨赴賊,同時見害。



 峻平,朝議贈壼左光祿大夫,加散騎常侍。尚書郎弘訥議以為「死事之臣古今所重,卞令忠貞之節,
 當書於竹帛。今之追贈,實未副眾望,謂宜加鼎司之號,以旌忠烈之勳」。司徒王導見議,進贈驃騎將軍,加侍中。訥重議曰:「夫事親莫大於孝,事君莫尚於忠。唯孝也,故能盡敬竭誠;唯忠也,故能見危授命。此在三之大節,臣子之極行也。案壼委質三朝,盡規翼亮,遭世險難,存亡以之。受顧託之重,居端右之任,擁衛至尊,則有保傅之恩;正色在朝,則有匪躬之節。賊峻造逆,戮力致討,身當矢KQ,再對賊鋒,父子并命,可謂破家為國,守死勤事。昔許男疾終,猶蒙二等之贈,況壼伏節國難者乎!夫賞疑從重,況在不疑!謂可上準許穆,下同嵇紹,則允合典謨,
 克厭眾望。」於是改贈壼侍中、驃騎將軍、開府儀同三司,謚曰忠貞,祠以太牢。贈世子散騎侍郎,弟盱奉車都尉。珍母裴氏撫二子尸哭曰:「父為忠臣,汝為孝子,夫何恨乎!」徵士翟湯聞之歎曰:「父死於君,子死於父,忠孝之道,萃于一門。」子誕嗣。



 咸康六年,成帝追思壼,下詔曰:「壼立朝忠恪,喪身兇寇,所封懸遠,租秩薄少,妻息不瞻,以為慨然!可給實口廩。」其後盜發壼墓,尸僵,鬢髮蒼白,面如生,兩手悉拳,爪甲穿達手背。安帝詔給錢十萬,以修塋兆。



 壺第三子瞻,位至廣州刺史。瞻弟眈,尚書郎。



 敦字仲仁。父俊,清真有檢識,以名理著稱。其鄉人傲郤
 詵恃才陵傲俊兄弟,俊等亦以門盛輕詵,相視如仇。詵以楊駿故吏被繫,俊時為尚書郎,案其獄,詵懼不免,俊平心斷決正之,詵卒以免,而猶不悛。後為左丞,復奏陷卞氏。俊歷位汝南相、廷尉卿。



 敦弱冠仕州郡,辟司空府,稍遷太子舍人、尚書郎,朝士多稱之。東海王越聞,召以為主簿。王彌逼洛,敦及胡毋輔之勸越擊王彌,而王衍、潘滔共執不聽,敦庭爭苦至,眾咸壯之。出補汝南內史。元帝之為鎮東,請為軍諮祭酒,不就。征南將軍山簡以為司馬。尋而王如、杜曾相繼為亂,簡乃使敦監沔北七郡軍事、振威將軍、領江夏相,戍夏口。敦攻討沔中皆平。既
 而杜弢寇湘中,加敦征討大都督。伐弢有功,賜爵安陵亭侯。鎮東大將軍王敦請為軍司。



 中興建,拜太子左衛率。時石勒侵逼淮泗,帝備求良將可以式遏邊境者,公卿舉敦,除征虜將軍、徐州刺史,鎮泗口。及勒寇彭城,敦自度力不能支,與征北將軍王邃退保盱眙,賊勢遂張,淮北諸郡多為所陷,竟以畏懦貶秩三等,為鷹揚將軍。徵拜大司農。王敦表為征虜將軍、都督石頭軍事。明帝之討王敦也。以為鎮南將軍、假節。事平,更拜尚書,以功封益陽侯。徙光祿勛,出為都督安南將軍、湘州刺史、假節。尋進征南將軍,固辭不拜。



 蘇峻反,溫嶠、庾亮移檄征
 鎮同赴京師。敦擁兵不下,又不給軍糧,唯遣督護荀璲領數百人隨大軍而已。時朝野莫不怪歎,獨陶侃亦切齒忿之。峻平,侃奏敦阻軍顧望,不赴國難,無大臣之節,請檻車收付廷尉。丞相王導以喪亂之後宜加寬宥轉安南將軍、廣州刺史。病不之職。徵為光祿大夫,領少府。敦既不討蘇峻,常懷愧恥,名論自此虧矣。尋以憂卒,追贈本官,加散騎常侍,謚曰敬。子滔嗣。



 劉超,字世瑜,琅邪臨沂人,漢城陽景王章之後也。章七世孫封臨沂縣慈鄉侯,子孫因家焉。父和,為琅邪國上
 軍將軍。超少有志尚,為縣小吏,稍遷琅邪國記室掾。以忠謹清慎為元帝所拔,恒親侍左右,遂從渡江,轉安東府舍人,專掌文檄。相府建,又為舍人。于時天下擾亂,伐叛討貳,超自以職在近密,而書跡與帝手筆相類,乃絕不與人交書。時出休沐,閉門不通賓客,由是漸得親密。以左右勤勞,賜爵原鄉亭侯,食邑七百戶,轉行參軍。



 中興建,為中書舍人,拜騎都尉、奉朝請。時臺閣初建,庶績未康,超職典文翰,而畏慎靜密,彌見親待。加以處身清苦,衣不重帛,家無儋石之儲。每帝所賜,皆固辭曰:「凡陋小臣,橫竊賞賜,無德而祿,殃咎足懼。」帝嘉之,不奪其志。
 尋出補句容令,推誠於物,為百姓所懷。常年賦稅,主者常自四出詰評百姓家貲。至超,但作大函,村別付之,使各自書家產,投函中訖,送還縣。百姓依實投上,課輸所入,有踰常年。入為中書通事郎。以父憂去官。既葬,屬王敦稱兵,詔超復職,又領安東上將軍。尋六軍敗散,唯超案兵直衛,帝感之,遣歸終喪禮。及錢鳳構禍,超招合義士,從明帝徵鳳。事平,以功封零陵伯。超家貧,妻子不贍,帝手詔褒之,賜以魚米,超辭不受。超後須純色牛,市不可得,啟買官外廄牛,詔便以賜之。出為義興太守。未幾,徵拜中書侍郎。拜受往還,朝廷莫有知者。會帝崩,穆后
 臨朝,遷射聲校尉。時軍校無兵,義興人多義隨超,因統其眾以宿衛,號為「君子營」。咸和初,遭母憂去官,衰服不離身,朝夕號泣,朔望輒步至墓所,哀感路人。



 及蘇峻謀逆,超代趙胤為左衛將軍。時京邑大亂,朝士多遣家人入東避難。義興故吏欲迎超家,而超不聽,盡以妻孥入處宮內。及王師敗績,王導以超為右衛將軍,親侍成帝。屬太后崩,軍衛禮章損闕,超躬率將士奉營山陵。峻遷車駕石頭,時天大雨,道路沈陷,超與侍中鐘雅步侍左右,賊給馬不肯騎,而悲哀慷慨。峻聞之,甚不平,然未敢加害,而以其所親信許方等補司馬督、殿中監,外託宿
 衛,內實防禦超等。時饑饉米貴,峻等問遺,一無所受,繾綣朝夕,臣節愈恭。帝時年八歲,雖幽厄之中,超猶啟授《孝經》、《論語》。溫嶠等至,峻猜忌朝士,而超為帝所親遇,疑之尤甚。後王導出奔,超與懷德令匡術、建康令管旆等密謀,將欲奉帝而出。未及期,事泄,峻使任讓將兵入收超及鐘雅。帝抱持悲泣曰:「還我侍中、右衛!」任讓不奉詔,因害之。及峻平,任讓與陶侃有舊,侃欲特不誅之,乃請於帝。帝曰:「讓是殺我侍中、右衛者,不可宥。」由是遂誅讓。及超將改葬,帝痛念之不已,詔遷高顯近地葬之,使出入得瞻望其墓。追贈衛尉,謚曰忠。超天性謙慎,歷事三
 帝,恒在機密,並蒙親遇,而不敢因寵驕諂,故士人皆安而敬之。



 子訥嗣,謹飭有石慶之風,歷中書侍郎、下邳內史。訥子享,亦清慎,為散騎郎。



 鐘雅,字彥胄,潁川長社人也。父曄,公府掾,早終。雅少孤,好學有才志,舉四行,除汝陽令,入為佐著作郎。母憂去官,服闋復職。東海王越請為參軍,遷尚書郎。



 避亂東渡,元帝以為丞相記室參軍,遷臨淮內史、振威將軍。頃之,徵拜散騎侍郎,轉尚書右丞。時有事於太廟,雅奏曰:「陛下繼承世數,於京兆府君為玄孫,而今祝文稱曾孫,恐
 此因循之失,宜見改正。又禮,祖之昆弟,從祖父也。景皇帝自以功德為世宗,不以伯祖而登廟,亦宜除伯祖之文。」詔曰:「禮,事宗廟,自曾孫已下皆稱曾孫,此非因循之失也。義取於重孫,可歷世共其名,無所改也。稱伯祖不安,如所奏。」轉北軍中候。大將軍王敦請為從事中郎,補宣城內史。錢鳳作逆,加廣武將軍,率眾屯青弋。時廣德縣人周為鳳起兵攻雅,雅退據涇縣,收合士庶,討,斬之。鳳平,徵拜尚書左丞。



 時帝崩,遷御史中丞。時國喪未期,而尚書梅陶私奏女妓,雅劾奏曰:「臣聞放勛之殂,八音遏密,雖在凡庶,猶能三載。自茲以來,歷代所同。肅
 祖明皇帝崩背萬國,當期來月。聖主縞素,泣血臨朝,百僚慘愴,動無歡容。陶無大臣忠慕之節,家庭侈靡,聲妓紛葩,絲竹之音,流聞衢路,宜加放黜,以整王憲。請下司徒,論正清議。」穆后臨朝,特原不問。雅直法繩違,百僚皆憚之。



 北中郎將劉遐卒,遐部曲作亂,詔郭默討之,以雅監征討軍事、假節。事平,拜驍騎將軍。蘇峻之難,詔雅為前鋒監軍、假節,領精勇千人以距峻。雅以兵少,不敢擊,退還。拜侍中。尋王師敗績,雅與劉超並侍衛天子。或謂雅曰:「見可而進,知難而退,古之道也。君性亮直,必不容於寇仇,何不隨時之宜而坐待其斃。」雅曰:「國亂不能匡,
 君危不能濟,各遜遁以求免,吾懼董狐執簡而至矣。」庾亮臨去,顧謂雅曰:「後事深以相委。」雅曰:「棟折榱崩,誰之責也。」亮曰:「今日之事,不容復言,卿當期剋復之效耳。」雅曰:「想足下不愧荀林父耳。」及峻逼遷車駕幸石頭,雅、超流涕步從。明年,並為賊所害。賊平,追贈光祿勳。其後以家貧,詔賜布帛百匹。子誕,位至中軍參軍,早卒。



 史臣曰:應詹行業聿修,文史足用,入居列位,則嘉謀屢陳;出撫籓條,則惠政斯洽。甘卓伐暴寧亂,庸績克宣,作鎮扞城,威略具舉。及兇渠犯順,志在勤王。既而人撓其謀,天奪其鑒,疑留不斷,自取誅夷。卞壼束帶立朝,以匡
 正為己任;褰裳衛主,蹈忠義以成名。遂使臣死於君,子死於父,惟忠與孝,萃其一門。古稱社稷之臣,忠貞之謂矣。劉超勤肅奉上,鐘雅正直當官。屬臣猾滔天,幼君危逼,乃崎嶇寇難,契闊艱虞,匪石為心,寒松比操,貞軌皆沒,亮跡雙升。雖高赫在難彌恭,荀息繼之以死,方之二子,曾何足雲!



 贊曰:卓臨南服,詹蒞西州。政刑克舉,威惠兼修。應嗟運促,甘斃疑留。望之徇義,處死為易。惟子惟臣,名節斯寄。鐘劉入仕,忠貞攸履。竭其股肱,繼之以死。



\end{pinyinscope}