\article{載記第十四 苻堅下}

\begin{pinyinscope}
苻堅
 \gezhu{
  下}
 \gezhu{
  王猛苻融苻朗}



 太元七年,堅饗群臣於前殿,樂奏賦詩。秦州別駕天水姜平子詩有「丁」字,直而不曲。堅問其故,平子曰:「臣丁至剛,不可以屈,且曲下者之不正之物,未足獻也。」堅笑曰:「名不虛行。」因擢為上第。



 堅兄法子東海公陽與王猛子散騎侍郎皮謀反,事洩,堅問反狀,陽曰:「《禮》云,父母之仇,不同天地。臣父哀公,死不以罪,齊襄復九世之仇,而況臣
 也!」皮曰:「臣父丞相有佐命之勛,而臣不免貧餒,所以圖富也。」堅流涕謂陽曰:「哀公之薨,事不在朕,卿寧不知之!」讓皮曰:「丞相臨終,託卿以十具牛為田,不聞為卿求位。知子莫若父,何斯言之徵也!」皆赦不誅,徙陽于高昌,皮于朔方之北。苻融以位忝宗正,不能肅遏奸萌,上疏請待罪私籓。堅不許。將以融為司徒,融固辭。堅銳意荊、揚,將謀入寇,乃改授融征南大將軍、開府儀同三司。



 新平郡獻玉器。初,堅即偽位,新平王彫陳說圖讖,堅大悅,以彫為太史令。嘗言於堅曰:「謹案讖云:『古月之末亂中州,洪水大起健西流,惟有雄子定八州。』此即三祖、陛下之
 聖諱也。又曰:『當有KI付臣又土,滅東燕,破白虜,氐在中,華在表。』案圖讖之文,陛下當滅燕,平六州。願徙汧、隴諸氐于京師,三秦大戶置於邊地,以應圖讖之言。」堅訪之王猛,猛以彫為左道惑眾,勸堅誅之。彫臨刑上疏曰:「臣以趙建武四年,從京兆劉湛學,明於圖記,謂臣曰:『新平地古顓頊之墟,里名曰雞閭。記云,此里應出帝王寶器,其名曰延壽寶鼎。顓頊有云,河上先生為吾隱之于咸陽西北,吾之孫有KI付臣又土應之。』湛又云:『吾嘗齋于室中,夜有流星大如半月,落於此地,斯蓋是乎!』願陛下誌之,平七州之後,出於壬午之年。」至是而新平人得
 之以獻,器銘篆書文題之法,一為天王,二為王后,三為三公,四為諸侯,五為伯子男,六為卿大夫,七為元士。自此已下,考載文記,列帝王名臣,自天子王后,內外次序,上應天文,象紫宮布列,依玉牒版辭,不違帝王之數。從上元人皇起,至中元,窮於下元,天地一變,盡三元而止。堅以彫言有征,追贈光祿大夫。



 幽州蝗,廣袤千里,堅遣其散騎常侍劉蘭持節為使者,發青、冀、幽、并百姓討之。



 以苻朗為使持節、都督青徐兗三州諸軍事、鎮東將軍、青州刺史,以諫議大夫裴元略為陵江將軍、西夷校尉、巴西梓潼二郡太守,密授規模,令與王撫備舟師于蜀,
 將以入寇。



 車師前部王彌窴、鄯善王休密馱朝於堅,堅賜以朝服,引見西堂。窴等觀其宮宇壯麗,儀衛嚴肅,甚懼,因請年年貢獻。堅以西域路遙,不許,令三年一貢,九年一朝,以為永制。窴等請曰:「大宛諸國雖通貢獻,然誠節未純,請乞依漢置都護故事。若王師出關,請為鄉導。」堅於是以驍騎呂光為持節、都督西討諸軍事,與陵江將軍姜飛、輕騎將軍彭晃等配兵七萬,以討定西域。苻融以虛秏中國,投兵萬里之外,得其人不可役,得其地不可耕,固諫以為不可。堅曰:「二漢力不能制匈奴,猶出師西域。今匈奴既平,易若摧朽,雖勞師遠役,可傳檄而
 定,化被昆山,垂芳千載,不亦美哉!」朝臣又屢諫,皆不納。



 晉將軍朱綽焚踐沔北屯田,掠六百餘戶而還。堅引群臣會議,曰:「吾統承大業垂二十載,芟夷逋穢,四方略定,惟東南一隅未賓王化。吾每思天下不一,未嘗不臨食輟餔,今欲起天下兵以討之。略計兵杖精卒,可有九十七萬,吾將躬先啟行,薄伐南裔,於諸卿意何如?」祕書監朱彤曰:「陛下應天順時,恭行天罰,嘯吒則五嶽摧覆,呼吸則江海絕流,若一舉百萬,必有征無戰。晉主自當銜璧輿櫬,啟顙軍門,若迷而弗悟,必逃死江海,猛將追之,即可賜命南巢。中州之人,還之桑梓。然後迴駕岱宗,告
 成封禪,起白雲於中壇,受萬歲於中嶽,爾則終古一時,書契未有。」堅大悅曰:「吾之志也。」左僕射權翼進曰:「臣以為晉未可伐。夫以紂之無道,天下離心,八百諸侯不謀而至,武王猶曰彼有人焉,迴師止旆。三仁誅放,然後奮戈牧野。今晉道雖微,未聞喪德,君臣和睦,上下同心。謝安、桓沖,江表偉才,可謂晉有人焉。臣聞師克在和,今晉和矣,未可圖也。」堅默然久之,曰:「諸君各言其志。」太子左衛率石越對曰:「吳人恃險偏隅,不賓王命,陛下親御六師,問罪衡、越,誠合人神四海之望。但今歲鎮星守斗牛,福德有吳。懸象無差,弗可犯也。且晉中宗,籓王耳,夷夏
 之情,咸共推之,遺愛猶在於人。昌明,其孫也,國有長江之險,朝無昏貳之釁。臣愚以為利用修德,未宜動師。孔子曰:『遠人不服,脩文德以來之。』願保境養兵,伺其虛隙。」堅曰:「吾聞武王伐紂,逆歲犯星。天道幽遠,未可知也。昔夫差威陵上國,而為句踐所滅。仲謀澤洽全吳,孫皓因三代之業,龍驤一呼,君臣面縛,雖有長江,其能固乎!以吾之眾旅,投鞭於江,足斷其流。」越曰:「臣聞紂為無道,天下患之。夫差淫虐,孫皓昏暴,眾叛親離,所以敗也。今晉雖無德,未有斯罪,深願厲兵積粟以待天時。」群臣各有異同,庭議者久之。堅曰:「所謂築室于道,沮計萬端,吾當
 內斷於心矣。」群臣出後,獨留苻融議之。堅曰:「自古大事,定策者一兩人而已,群議紛紜,徒亂人意,吾當與汝決之」融曰:「歲鎮在斗牛,吳、越之福,不可以伐一也。晉主休明,朝臣用命,不可以伐二也。我數戰,兵疲將倦,有憚敵之意,不可以伐三也。諸言不可者,策之上也,願陛下納之。」堅作色曰:「汝復如此,天下之事,吾當誰與言之!今有眾百萬,資仗如山,吾雖未稱令主,亦不為闇劣。以累捷之威,擊垂亡之寇,何不克之有乎!吾終不以賊遺子孫,為宗廟社稷之憂也。」融泣曰:「吳之不可伐昭然,虛勞大舉,必無功而反。臣之所憂,非此而已。陛下寵育鮮卑、羌、
 羯,布諸畿甸,舊人族類,斥徙遐方。今傾國而去,如有風塵之變者,其如宗廟何!監國以弱卒數萬留守京師,鮮卑、羌、羯攢聚如林,此皆國之賊也,我之仇也。臣恐非但徒返而已,亦未必萬全。臣智識愚淺,誠不足采;王景略一時奇士,陛下每擬之孔明,其臨終之言不可忘也。」堅不納。游于東苑,命沙門道安同輦。權翼諫曰:「臣聞天子之法駕,侍中陪乘,清道而行,進止有度。三代末主,或虧大倫,適一時之情,書惡來世。故班姬辭輦,垂美無窮。道安毀形賤士,不宜參穢神輿。」堅作色曰:「安公道冥至境,德為時尊。朕舉天下之重,未足以易之。非公與輦之榮,此
 乃朕之顯也。」命翼扶安升輦,顧謂安曰:「朕將與公南遊吳、越,整六師而巡狩,謁虞陵於疑嶺,瞻禹穴于會稽,泛長江,臨滄海,不亦樂乎!」安曰:「陛下應天御世,居中土而制四維,逍遙順時,以適聖躬,動則鳴鑾清道,止則神棲無為,端拱而化,與堯、舜比隆,何為勞身于馳騎,口倦于經略,櫛風沐雨。蒙塵野次乎?且東南區區,地下氣癘,虞舜游而不返,大禹適而弗歸,何足以上勞神駕,下困蒼生。《詩》云:『惠此中國,以綏四方。』茍文德足以懷遠,可不煩寸兵而坐賓百越。」堅曰:「非為地不廣、人不足也,但思混一六合,以濟蒼生。天生蒸庶,樹之君者,所以除煩去亂,
 安得憚勞!朕既大運所鐘,將簡天心以行天罰。高辛有熊泉之役,唐堯有丹水之師,此皆著之前典,昭之後王。誠如公言,帝王無省方之文乎?且朕此行也,以義舉耳,使流度衣冠之胄,還其墟墳,復其桑梓,止為濟難銓才,不欲窮兵極武。」安曰:「若鑾駕必欲親動,猶不願遠涉江、淮,可暫幸洛陽,明授勝略,馳紙檄于丹陽,開其改迷之路。如其不庭,伐之可也。」堅不納。先是,群臣以堅信重道安,謂安曰:「主上欲有事於東南,公何不為蒼生致一言也!」故安因此而諫。苻融及尚書原紹、石越等上書面諫,前後數十,堅終不從。堅少子中山公詵有寵於堅,又諫
 曰:「臣聞季梁在隨,楚人憚之;宮奇在虞,晉不窺兵。國有人焉故也。及謀之不用,而亡不淹歲。前車之覆軌,後車之明鑒。陽平公,國之謀主,而陛下違之;晉有謝安、桓沖,而陛下伐之。是行也,臣竊惑焉。」堅曰:「國有元龜。可以決大謀;朝有公卿,可以定進否。孺子言焉,將為戮也。」



 所司奏劉蘭討蝗幽州,經秋冬不滅,請徵下廷尉詔獄。堅曰:「災降自天,殆非人力所能除也。此自朕之政違所致,蘭何罪焉!」



 明年,呂光發長安,堅送于建章宮,謂光曰:「西戎荒俗,非禮義之邦。羈縻之道,服而赦之,示以中國之威,導以王化之法,勿極武窮兵,過深殘掠。」加鄯善王休密
 馱使持節、散騎常侍、都督西域諸軍事、寧西將軍,車師前部王彌窴使持節、平西將軍、西域都護,率其國兵為光鄉導。



 是年,益州西南夷、海南諸國皆遣使貢其方物。



 堅南游灞上,從容謂群臣曰:「軒轅,大聖也,其仁若天,其智若神,猶隨不順者從而征之,居無常所,以兵為衛,故能日月所照,風雨所至,莫不率從。今天下垂平,惟東南未殄。朕忝荷大業,巨責攸歸,豈敢優游卒歲,不建大同之業!每思桓溫之寇也,江東不可不滅。今有勁卒百萬,文武如林,鼓行而摧遺晉,若商風之隕秋籜。朝廷內外,皆言不可,吾實未解所由。晉武若信朝士之言而不征
 吳者,天下何由一軌!吾計決矣,不復與諸卿議也。」太子宏進曰:「吳今得歲,不可伐也。且晉主無罪,人為之用;謝安、桓沖兄弟皆一方之俊才,君臣戮力,阻險長江,未可圖也。但可厲兵積粟,以待暴主,一舉而滅之。今若動而無功,則威名損於外,資財竭於內。是故聖王之行師也,內斷必誠,然後用之。彼若憑長江以固守,徙江北百姓於江南,增城清野,杜門不戰,我已疲矣,彼未引弓。土下氣癘,不可久留,陛下將若之何?」堅曰:「往年車騎滅燕,亦犯歲而捷之。天道幽遠,非汝所知也。昔始皇之滅六國,其王豈皆暴乎?且吾內斷於心久矣,舉必克之,何為無
 功!吾方命蠻夷以攻其內,精甲勁兵以攻其外,內外如此,安有不克!」道安曰:「太子之言是也,願陛下納之。」堅弗從。冠軍慕容垂言於堅曰:「陛下德侔軒、唐,功高湯、武,威澤被于八表,遠夷重譯而歸。司馬昌明因餘燼之資,敢距王命,是而不誅,法將安措!孫氏跨僭江東,終併於晉,其勢然也。臣聞小不敵大,弱不御彊,況大秦之應符,陛下之聖武,彊兵百萬,韓、白盈朝,而令其偷魂假號,以賊虜遺子孫哉!《詩》云:『築室于道謀,是用不潰于成。』陛下內斷神謀足矣,不煩廣訪朝臣以亂聖慮。昔晉武之平吳也,言可者張、杜數賢而已,若採群臣之言,豈能建不世
 之功!諺云憑天俟時,時已至矣,其可已乎!」堅大悅,曰:「與吾定天下者,其惟卿耳。」賜帛五百匹。



 彗星掃東井。自堅之建元十七年四月,長安有水影,遠觀若水,視地則見人,至是則止。堅惡之。上林竹死,洛陽地陷。



 晉車騎將軍桓沖率眾十萬伐堅,遂攻襄陽。遣前將軍劉波、冠軍桓石虔、振威桓石民攻沔北諸城;輔國楊亮伐蜀,攻拔伍城,進攻涪城,龍驤胡彬攻下蔡;鷹揚郭銓攻武當;沖別將攻萬歲城,拔之。堅大怒,遣其子征南睿及冠軍慕容垂、左衛毛當率步騎五萬救襄陽,揚武張崇救武當,後將軍張蠔、步兵校尉姚萇救涪城。睿次新野,垂次鄧城。王
 師敗張崇于武當,掠二千餘戶而歸。睿遣垂及驍騎石越為前鋒,次于沔水。垂、越夜命三軍人持十炬火,繫炬于樹枝,光照十數里中。沖懼,退還上明。張蠔出斜谷,楊亮亦引兵退歸。



 堅下書悉發諸州公私馬,人十丁遣一兵。門在灼然者,為崇文義從。良家子年二十已下,武藝驍勇,富室材雄者,皆拜羽林郎。下書期克捷之日,以帝為尚書左僕射,謝安為吏部尚書,桓沖為侍中,並立第以待之。良家子至者三萬餘騎。其秦州主簿金城趙盛之為建威將軍、少年都統。遣征南苻融、驃騎張蠔、撫軍苻方、衛軍梁成、平南慕容、冠軍慕容垂率步騎二
 十五萬為前鋒。堅發長安,戎卒六十餘萬,騎二十七萬,前後千里,旗鼓相望。堅至項城,涼州之兵始達咸陽,蜀漢之軍順流而下,幽、冀之眾至於彭城,東西萬里,水陸齊進。運漕萬艘,自河入石門,達于汝、潁。



 融等攻陷壽春,執晉平虜將軍徐元喜、安豐太守王先。垂攻陷鄖城,害晉將軍王太丘。梁成與其揚州刺史王顯、弋陽太守王詠等率眾五萬,屯于洛澗,柵淮以遏東軍。成頻敗王師。晉遣都督謝石、徐州刺史謝玄、豫州刺史桓伊、輔國謝琰等水陸七萬,相繼距融,去洛澗二十五里,憚成不進。龍驤將軍胡彬先保硤石,為融所逼,糧盡,詐揚沙以示
 融軍,潛遣使告石等曰:「今賊盛糧盡,恐不見大軍。」融軍人獲而送之。融乃馳使白堅曰:「賊少易俘,但懼其越逸,宜速進眾軍,掎禽賊帥。」堅大悅,恐石等遁也,捨大軍于項城,以輕騎八千兼道赴之,令軍人曰:「敢言吾至壽春者拔舌。」故石等弗知。晉龍驤將軍劉牢之率勁卒五千,夜襲梁成壘,克之,斬成及王顯、王詠等十將,士卒死者萬五千。謝石等以既敗梁成,水陸繼進。堅與苻融登城而望王師,見部陣齊整,將士精銳,又北望八公山上草木,皆類人形,顧謂融曰:「此亦勍敵也,何謂少乎!」憮然有懼色。初,朝廷聞堅入寇,會稽王道子以威儀鼓吹求
 助於鐘山之神,奉以相國之號。及堅之見草木狀人,若有力焉。



 堅遣其尚書朱序說石等以眾盛,欲脅而降之。序詭謂石曰:「若秦百萬之眾皆至,則莫可敵也。及其眾軍未集,宜在速戰。若挫其前鋒,可以得志。」石聞堅在壽春也,懼,謀不戰以疲之。謝琰勸從序言,遣使請戰,許之。時張蠔敗謝石于肥南,謝玄、謝琰勒卒數萬,陣以待之。蠔乃退,列陣逼肥水。王師不得渡,遣使謂融曰:「君懸軍深入,置陣逼水,此持久之計,豈欲戰者乎?若小退師,令將士周旋,僕與君公緩轡而觀之,不亦美乎!」融於是麾軍卻陣,欲因其濟水,覆而取之。軍遂奔退,制之不可止。
 融馳騎略陣,馬倒被殺,軍遂大敗。王師乘勝追擊,至于青岡,死者相枕。堅為流矢所中,單騎遁還於淮北,飢甚,人有進壺飧豚髀者,堅食之,大悅,曰:「昔公孫豆粥何以加也!」使賜帛十匹,綿十斤。辭曰:「臣聞白龍厭天池之樂而見困豫且,陛下目所睹也,耳所聞也。今蒙塵之難,豈自天乎!且妄施不為惠,妄受不為忠。陛下,臣之父母也,安有子養而求報哉!」弗顧而退。堅大慚,顧謂其夫人張氏曰:「朕若用朝臣之言,豈見今日之事邪!當何面目復臨天下乎?」潸然流涕而去。聞風聲鶴唳,皆謂晉師之至。其僕射張天錫、尚書朱序及徐元喜等皆歸順。初,諺言「
 堅不出項」,群臣勸堅停項,為六軍聲鎮,堅不從,故敗。



 諸軍悉潰,惟慕容垂一軍獨全,堅以千餘騎赴之。垂子寶勸垂殺堅,垂不從,乃以兵屬堅。初,慕容屯鄖城,姜成等守漳口,晉隨郡太守夏侯澄攻姜成,斬之,棄其眾奔還。堅收離集散,比至洛陽,眾十餘萬,百官威儀軍容粗備。未及關而垂有貳志,說堅請巡撫燕、岱,并求拜墓,堅許之。權翼固諫以為不可,堅不從。尋懼垂為變,悔之,遣驍騎石越率卒三千戍鄴,驃騎張蠔率羽林五千戍并州,留兵四千配鎮軍毛當戍洛陽。堅至自淮南,次于長安東之行宮,哭苻融而後入,告罪于其太廟,赦殊死
 已下,文武增位一級,厲兵課農,存恤孤老,諸士卒不返者皆復其家終世。贈融大司馬,謚曰哀公。



 衛軍從事中郎丁零、翟斌反于河南,長樂公苻丕遣慕容垂及苻飛龍討之。垂南結丁零,殺飛龍,盡坑其眾。豫州牧、平原公苻暉遣毛當擊翟斌,為斌所敗,當死之。垂子農亡奔列人,招集群盜,眾至萬數千。丕遣石越擊之,為農所敗,越死之。垂引丁零、烏丸之眾二十餘萬,為飛梯地道以攻鄴城。



 慕容弟燕故濟北王泓先為北地長史,聞垂攻鄴,亡命奔關東,收諸馬牧鮮卑,眾至數千,還屯華陰。慕容乃潛使諸弟及宗人起兵於外。堅遣將軍強永率
 騎擊之,為泓所敗,泓眾遂盛,自稱使持節、大都督陜西諸軍事、大將軍、雍州牧、濟北王,推叔父垂為丞相、都督陜東諸軍事、領大司馬、冀州牧、吳王。



 堅謂權翼曰:「吾不從卿言,鮮卑至是。關東之地,吾不復與之爭,將若泓何?」翼曰:「寇不可長。慕容垂正可據山東為亂,不暇近逼。今及宗族種類盡在京師,鮮卑之眾布於畿甸,實社稷之元憂,宜遣重將討之。」堅乃以廣平公苻熙為使持節、都督雍州雜戎諸軍事、鎮東大將軍、雍州刺史,鎮蒲阪。征苻睿為都督中外諸軍事、衛大將軍、司隸校尉、錄尚書事,配兵五萬以左將軍竇衝為長史,龍驤姚萇為司
 馬,討泓於華澤。平陽太守慕容沖起兵河東,有眾二萬,進攻蒲阪,堅命竇衝討之。苻睿勇果輕敵,不恤士眾。泓聞其至也,懼,率眾將奔關東,睿馳兵要之。姚萇諫曰:「鮮卑有思歸之心,宜驅令出關,不可遏也。」睿弗從,戰于華澤,睿敗績,被殺。堅大怒。萇懼誅,遂叛。竇衝擊慕容沖於河東,大破之,沖率騎八千奔于泓軍。泓眾至十餘萬,遣使謂堅曰:「秦為無道,滅我社稷。今天誘其衷,使秦師傾敗,將欲興復大燕。吳王已定關東,可速資備大駕,奉送家兄皇帝並宗室功臣之家。泓當率關中燕人,翼衛皇帝,還返鄴都,與秦以武牢為界,分王天下,永為鄰好,不復
 為秦之患也。鉅鹿公輕戇銳進,為亂兵所害,非泓之意。」堅大怒,召慕容責之曰:「卿父子干紀僭亂,乖逆人神,朕應天行神,盡兵勢而得卿。卿非改迷歸善,而合宗蒙宥,兄弟布列上將、納言,雖曰破滅,其實若歸。奈何因王師小敗,便猖悖若此!垂為長蛇於關東,泓、沖稱兵內侮。泓書如此,卿欲去者,朕當相資。卿之宗族,可謂人面獸心,殆不可以國士期也。」叩頭流血,泣涕陳謝。堅久之曰:「《書》云,父子兄弟無相及也。卿之忠誠,實簡朕心,此自三豎之罪,非卿之過。」復其位而待之如初。命以書招喻垂及泓、沖,使息兵還長安,恕其反叛之咎。而密遣
 使者謂泓曰:「今秦數已終,長安怪異特甚,當不復能久立。吾既籠中之人,必無還理。昔不能保守宗廟,致令傾喪若斯,吾罪人也,不足復顧吾之存亡。社稷不輕,勉建大業,以興復為務。可以吳王為相國,中山王為太宰、領大司馬,汝可為大將軍、領司徒,承制封拜。聽吾死問,汝使即尊位。」泓於是進向長安,改年曰燕興。是時鬼夜哭,三旬而止。



 堅率步騎二萬討姚萇于北地,次于趙氏塢,使護軍楊璧游騎三千,斷其奔路,右軍徐成、左軍竇衝、鎮軍毛盛等屢戰敗之,仍斷其運水之路。馮翊游欽因淮南之敗,聚眾數千,保據頻陽,遣軍運水及粟,以饋姚
 萇,楊璧盡獲之。萇軍渴甚,遣其弟鎮北尹買率勁卒二萬決堰。竇沖率眾敗其軍于鸛雀渠,斬尹買及首級萬三千。萇眾危懼,人有渴死者。俄而降雨於萇營,營中水三尺,周營百步之外,寸餘而已,於是萇軍大振。堅方食,去案怒曰:「天其無心,何故降澤賊營!」萇又東引慕容泓為援。



 泓謀臣高蓋、宿勤崇等以泓德望後沖,且持法苛峻,乃殺泓,立沖為皇太弟,承制行事,自相署置。



 姚萇留其弟征虜緒守楊渠川大營,率眾七萬來攻堅。堅遣楊璧等擊之,為萇所敗,獲楊璧、毛盛、徐成及前軍齊午等數十人,皆禮而遣之。



 苻暉率洛陽、陜城之眾七萬歸于
 長安。益州刺史王廣遣將軍王蠔率蜀漢之眾來赴難。堅聞慕容沖去長安二百餘里,引師而歸,使撫軍苻方戍驪山,拜苻暉使持節、散騎常侍、都督中外諸軍事、車騎大將軍、司隸校尉、錄尚書,配兵五萬距沖,河間公苻琳為中軍大將軍,為暉後繼。沖乃令婦人乘牛馬為眾,揭竿為旗,揚土為塵,督厲其眾,晨攻暉營于鄭西。暉出距戰,沖揚塵鼓噪,暉師敗績。堅又以尚書姜宇為前將軍,與苻琳率眾三萬,擊沖于灞上,為沖所敗,宇死之,琳中流矢,沖遂據阿房城。初,堅之滅燕,沖姊為清河公主,年十四,有殊色,堅納之,寵冠後庭。沖年十二,亦有龍陽之
 姿,堅又幸之。姊弟專寵,宮人莫進。長安歌之曰:「一雌復一雄,雙飛入紫宮。」咸懼為亂。王猛切諫,堅乃出沖。長安又謠曰:「鳳皇鳳皇止阿房。」堅以鳳皇非梧桐不棲,非竹實不食,乃植桐竹數十萬株于阿房城以待之。沖小字鳳皇,至是,終為堅賊,入止阿房城焉。



 晉西中郎將桓石虔進據魯陽,遣河南太守高茂北戍洛陽。晉冠軍謝玄次于下邳,徐州刺史趙遷棄彭城奔還。玄前鋒張願追遷及於碭山,轉戰而免。玄進據彭城。



 時呂光討平西域三十六國,所獲珍寶以萬萬計。堅下書以光為使持節、散騎常侍、都督玉門以西諸軍事、安西將軍、西域校尉,
 進封順鄉侯,增邑一千戶。



 劉牢之伐兗州,堅刺史張崇棄鄄城奔于慕容垂。牢之遣將軍劉襲追崇,戰于河南,斬其東平太守楊光而退。牢之遂據鄄城。



 慕容沖進逼長安,堅登城觀之,歎曰:「此虜何從出也?其彊若斯!」大言責沖曰:「爾輩群奴正可牧牛羊,何為送死!」沖曰:「奴則奴矣,既厭奴苦,復欲取爾見代。」堅遣使送錦袍一領遺沖,稱詔曰:「古人兵交,使在其間。卿遠來草創,得無勞乎?今送一袍,以明本懷。朕於卿恩分如何,而於一朝忽為此變!」沖命詹事答之,亦稱「皇太弟有令:孤今心在天下,豈顧一袍小惠。茍能知命,便可君臣束手,早送皇帝,自當
 寬貸苻氏,以酬曩好,終不使既往之施獨美於前」。堅大怒曰:「吾不用王景略、陽平公之言,使白虜敢至於此。」



 苻丕在鄴糧竭,馬無草,削松木而食之。會丁零叛慕容垂,垂引師去鄴,始具西問,知苻睿等喪敗,長安危逼,乃遣其陽平太守邵興率騎一千,將北引重合侯苻謨、高邑侯苻亮、阜城侯苻定於常山,固安侯苻鑒、中山太守王兗于中山,以為己援。垂遣將軍張崇要興,獲之于襄國南。又遣其參軍封孚西引張蠔、并州刺史王騰于晉陽,蠔、騰以眾寡不赴。丕進退路窮,乃謀於群僚。司馬楊膺唱歸順之計,丕猶未從。會晉遣濟北太守丁匡據碻磝,濟陽
 太守郭滿據滑臺,將軍顏肱、劉襲次于河北,丕遣將軍桑據距之,為王師所敗。襲等進攻黎陽,克之。丕懼,乃遣從弟就與參軍焦逵請救于謝玄。丕書稱假途求糧,還赴國難,須軍援既接,以鄴與之,若西路不通,長安陷沒,請率所領保守鄴城。乃羈縻一方,文降而已。逵與參軍姜讓密謂楊膺曰:「今禍難如此,京師阻隔,吉凶莫審,密邇寇仇,三軍罄絕,傾危之甚,朝不及夕。觀公豪氣不除,非救世之主,既不能竭盡誠款,速致糧援,方設兩端,必無成也。今日之殆,疾於轉機,不容虛設,徒成反覆。宜正書為表,以結殷勤。若王師之至,必當致身。如其不從,可逼
 縛與之。茍不義服,一人力耳。古人行權,寧濟為功,況君侯累葉載德,顯祖初著名於晉朝,今復建崇勛,使功業相繼,千載一時,不可失也。」膺素輕丕,自以力能逼之,乃改書而遣逵等,并遣濟南毛蜀、毛鮮等分房為任於晉。



 堅遣鴻臚郝稚徵處士王嘉于到獸山。既至,堅每日召嘉與道安於外殿,動靜咨問之。慕容入見東堂,稽首謝曰:「弟沖不識義方,孤背國恩,臣罪應萬死。陛下垂天地之容,臣蒙更生之惠。臣二子昨婚,明當三日,愚欲暫屈鑾駕,幸臣私第。」堅許之。出,嘉曰:「椎蘆作蘧蒢,不成文章,會天大雨,不得殺羊。」堅與群臣莫之能解。是夜大
 雨,晨不果出。初,之遣諸弟起兵於外也,堅防守甚嚴,謀應之而無因。時鮮卑在城者猶有千餘人,乃密結鮮卑之眾,謀伏兵請堅,因而殺之。令其豪帥悉羅騰、屈突鐵侯等潛告之曰:「官今使侯外鎮,聽舊人悉隨,可於某日會集某處。」鮮卑信之。北部人突賢與其妹別,妹為左將軍竇衝小妻,聞以告衝,請留其兄。衝馳入白堅,堅大驚,召騰問之,騰具首服。堅乃誅父子及其宗族,城內鮮卑無少長及婦女皆殺之。



 慕容垂復圍鄴城。焦逵既至,朝廷果欲征丕任子,然後出師。逵固陳丕款誠無貳,并宣楊膺之意,乃遣劉牢之等率眾二萬,水陸運漕
 救鄴。



 時長安大饑,人相食,諸將歸而吐肉以飴妻子。



 慕容沖僭稱尊號于阿房,改年更始。堅與沖戰,各有勝負。嘗為沖軍所圍,殿中上將軍鄧邁、左中郎將鄧綏、尚書郎鄧瓊相謂曰:「吾門世荷榮寵,先君建殊功於國家,不可不立忠效節,以成先君之志。且不死君難者,非丈夫也。」於是與毛長樂等蒙獸皮,奮矛而擊沖軍。沖軍潰,堅獲免,嘉其忠勇,並拜五校,加三品將軍,賜爵關內侯。沖又遣其尚書令高蓋率眾夜襲長安,攻陷南門,入于南城。左將軍竇衝、前禁將軍李辯等擊敗之,斬首千八百級,分其尸而食之。堅尋敗沖于城西,追奔至於阿城。諸
 將請乘勝入城,堅懼為沖所獲,乃擊金以止軍。



 是時劉牢之至枋頭。征東參軍徐義、宦人孟豐告苻丕,楊膺、姜讓等謀反,丕收膺、讓戮之。牢之以丕自相屠戮,盤桓不進。



 苻暉屢為沖所敗,堅讓之曰:「汝,吾之子也,擁大眾,屢為白虜小兒所摧,何用生為!」暉憤恚自殺。關中堡壁三千餘所,推平遠將軍馮翊、趙敖為統主,相率結盟,遣兵糧助堅。左將軍茍池、右將軍俱石子率騎五千,與沖爭麥,戰于驪山,為沖所敗,池死之,石子奔鄴。堅大怒,復遣領軍楊定率左右精騎二千五百擊沖,大敗之,俘掠鮮卑萬餘而還。堅怒,悉坑之。定果勇善戰,沖深憚之,遂穿
 馬埳以自固。



 劉牢之至鄴,慕容垂北如新城。鄴中饑甚,丕率鄴城之眾就晉穀于枋頭。牢之入屯鄴城。慕容垂軍人飢甚,多奔中山,幽、冀人相食。初,關東謠曰:「幽州,生當滅。若不滅,百姓絕。」,垂之本名。與丕相持經年,百姓死幾絕。



 先是,姚萇攻新平,新平太守茍輔將降之,郡人遼西太守馮傑、蓮勺令馮翊等諫曰:「天下喪亂,忠臣乃見。昔田單守一城而存齊,今秦之所有,猶連州累鎮,郡國百城。臣子之於君父,盡心焉,盡力焉,死而後已,豈宜貳哉!」輔大悅,於是憑城固守。萇為土山地道,輔亦為之。或戰山峰,萇眾死者萬有餘人。輔乃詐降,萇將入,覺
 之,引眾而退。輔馳出擊之,斬獲萬計。至是,糧竭矢盡,外救不至,萇遣吏謂輔曰:「吾方以義取天下,豈仇忠臣乎?卿但率見眾男女還長婁,吾須此城置鎮。」輔以為然,率男女萬五千口出城,萇圍而坑之,男女無遺。初,石季龍末,清河崔悅為新平相,為郡人所殺。悅子液後仕堅,為尚書郎,自表父仇不同天地,請還冀州。堅愍之,禁錮新平人,缺其城角以恥之。新平酋望深以為慚,故相率距萇,以立忠義。



 時有群烏數萬,翔鳴于長安城上,其聲甚悲,占者以為斗羽不終年,有甲兵入城之象。沖率眾登城,堅身貫甲胄,督戰距之,飛矢滿身,血流被體。時雖兵
 寇危逼,馮翊諸堡壁猶有負糧冒難而至者,多為賊所殺。堅謂之曰:「聞來者率不善達,誠是忠臣赴難之義。當今寇難殷繁,非一人之力所能濟也。庶明靈有照,禍極災返,善保誠順,為國自愛,蓄糧厲甲,端聽師期,不可徒喪無成,相隨獸口。」三輔人為沖所略者,咸遣使告堅,請放火以為內應。堅曰:「哀諸卿忠誠之意也,何復已已。但時運圮喪,恐無益於國,空使諸卿坐自夷滅,吾所不忍也。且吾精兵若獸,利器如霜,而衄於烏合疲鈍之賊,豈非天也!宜善思之。」眾固請曰:「臣等不愛性命,投身為國,若上天有靈,單誠或冀一濟,沒無遺恨矣。」堅遣騎七百
 應之。而沖營放火者為風焰所燒,其能免者十有一二。堅深痛之,身為設祭而招之曰:「有忠有靈,來就此庭。歸汝先父,勿為妖形。」歔欷流涕,悲不自勝。眾咸相謂曰:「至尊慈恩如此,吾等有死無移。」沖毒暴關中,人皆流散,道路斷絕,千里無煙。堅以甘松護軍仇騰為馮翊太守,加輔國將軍,與破虜將軍蜀人蘭犢慰勉馮翊諸縣之眾。眾咸曰:「與陛下同死共生,誓無有貳。」



 每夜有周城大呼曰:「楊定健兒應屬我,宮殿臺觀應坐我,父子同出不共汝。」且尋而不見人跡。城中有書曰《古符傳賈錄》,載「帝出五將久長得」。先是,又謠曰:「堅入五將山長得。」堅大信
 之,告其太子宏曰:「脫如此言,天或導予。今留汝兼總戎政,勿與賊爭利,朕當出隴收兵運糧以給汝。天其或者正訓予也。」於是遣衛將軍楊定擊沖于城西,為沖所擒。堅彌懼,付宏以後事,將中山公詵、張夫人率騎數百出如五將,宣告州郡,期以孟冬救長安。宏尋將母妻宗室男女數千騎出奔,百僚逃散。慕容沖入據長安,從兵大掠,死者不可勝計。



 初,秦之未亂也,關中土然,無火而煙氣大起,方數十里中,月餘不滅。堅每臨聽訟觀,令百姓有怨者舉煙于城北,觀而錄之。長安為之語曰:「欲得必存當舉煙。」又為謠曰:「長鞘馬鞭擊左股,太歲南行當復
 虜。」秦人呼鮮卑為白虜。慕容垂之起於關東,歲在癸末。堅之分氐戶於諸鎮也,趙整因侍,援琴而歌曰:「阿得脂,阿得脂,博勞舊父是仇綏,尾長翼短不能飛,遠徙種人留鮮卑,一旦緩急語阿誰!」堅笑而不納。至是,整言驗矣。



 堅至五將山,姚萇遣將軍吳忠圍之。堅眾奔散,獨侍御十數人而已。神色自若,坐而待之,召宰人進食。俄而忠至,執堅以歸新平,幽之於別室。萇求傳國璽於堅曰:「萇次膺符歷,可以為惠。」堅瞋目叱之曰:「小羌乃敢干逼天子,豈以傳國璽授汝羌也,圖緯符命,何所依據?五胡次序,無汝羌名。違天不祥,其能久乎!璽已送晉,不可得也。」
 萇又遣尹緯說堅,求為堯、舜禪代之事。堅責緯曰:「禪代者,聖賢之事。姚萇叛賊,奈何擬之古人!」堅既不許萇以禪代,罵而求死,萇乃縊堅于新平佛寺中,時年四十八。中山公詵及張夫人並自殺。是歲太元十年也。



 宏之奔也,歸其南秦州刺史楊璧于下辯,璧距之,乃奔武翥氐豪強熙,假道歸順,朝廷處宏于江州。宏歷位輔國將軍。桓玄篡位,以宏為梁州刺史。義熙初,以謀叛被誅。



 初,堅彊盛之時,國有童謠云:「河水清復清,苻詔死新城。」堅聞而惡之,每征伐,戒軍候云:「地有名新者避之。」時又童謠云:「阿堅連牽三十年,若後欲敗當在江、淮間。」堅在位二
 十七年,因壽春之敗,其國大亂,後二年,竟死於新平佛寺,咸應謠言矣。丕僭號,偽追謚堅曰世祖宣昭皇帝。



 王猛,字景略,北海劇人也,家于魏郡。少貧賤,以鬻畚為業。嘗貨畚于洛陽,乃有一人貴買其畚,而云無直,自言:「家去此無遠,可隨我取直。」猛利其貴而從之,行不覺遠,忽至深山,見一父老,鬚髮皓然,踞胡床而坐,左右十許人,有一人引猛進拜之。父老曰:「王公何緣拜也!」乃十倍償畚直,遣人送之。猛既出,顧視,乃嵩高山也。



 猛瑰姿俊偉。博學好兵書,謹重嚴毅,氣度雄遠,細事不干其慮,自
 不參其神契,略不與交通,是以浮華之士咸輕而笑之。猛悠然自得,不以屑懷。少游於鄴都,時人罕能識也。惟徐統見而奇之,召為功曹。遁而不應,遂隱于華陰山。懷佐世之志,希龍顏之主,斂翼待時,候風雲而後動。桓溫入關,猛被褐而詣之,一面談當世之事,捫虱而言,旁若無人。溫察而異之,問曰:「吾奉天子之命,率銳師十萬,杖義討逆,為百姓除殘賊,而三秦豪傑未有至者何也?」猛曰:「公不遠數千里,深入寇境,長安咫尺而不渡灞水,百姓未見公心故也,所以不至。」溫默然無以酬之。溫之將還,賜猛車馬,拜高官督護,請與俱南。猛還山咨師,師曰:「
 卿與桓溫豈並世哉!在此自可富貴,何為遠乎!」猛乃止。



 苻堅將有大志,聞猛名,遣呂婆樓招之,一見便若平生。語及廢興大事,異符同契,若玄德之遇孔明也。及堅僭位,以猛為中書侍郎。時始平多枋頭西歸之人,豪右縱橫,劫盜充斥,乃轉猛為始平令。猛下車,明法峻刑,澄察善惡,禁勒彊豪。鞭殺一吏,百姓上書訟之,有司劾奏,檻車徵下廷尉詔獄。堅親問之,曰:「為政之體,德化為先,蒞任未幾而殺戮無數,何其酷也!」猛曰:「臣聞宰寧國以禮,治亂邦以法。陛下不以臣不才,任臣以劇邑,謹為明君翦除兇猾。始殺一姦,餘尚萬數,若以臣不能窮殘盡暴,
 肅清軌法者,敢不甘心鼎鑊,以謝孤負。酷政之刑,臣實未敢受之。」堅謂群臣曰:「王景略固是夷吾、子產之儔也。」於是赦之。



 遷尚書左丞、咸陽內史、京兆尹。未幾,除吏部尚書、太子詹事,又遷尚書左僕射、輔國將軍、司隸校尉,加騎都尉,居中宿衛。時猛年三十六,歲中五遷,權傾內外,宗戚舊臣皆害其寵。尚書仇騰、丞相長史席寶數譖毀之,堅大怒,黜騰為甘松護軍,寶白衣領長史。爾後上下咸服,莫有敢言。頃之,遷尚書令、太子太傅,加散騎常侍。猛頻表累讓,堅竟不許。又轉司徒、錄尚書事,餘如故。猛辭以無功,不拜。



 後率諸軍討慕容,軍禁嚴明,師無
 私犯。猛之未至鄴也,劫盜公行,及猛之至,遠近帖然,燕人安之。軍還,以功進封清河郡侯,賜以美妾五人,上女妓十二人,中妓三十八人,馬百匹,車十乘。猛上疏固辭不受。



 時既留鎮冀州,堅遣猛於六州之內聽以便宜從事,簡召英俊,以補關東守宰,授訖,言臺除正。居數月,上疏曰:「臣前所以朝聞夕拜,不顧艱虞者,正以方難未夷,軍機權速,庶竭命戎行,甘驅馳之役,敷宣皇威,展筋骨之效,故FC俛從事,叨據負乘,可謂恭命於濟時,俟太平於今日。今聖德格于皇天,威靈被于八表,弘化已熙,六合清泰,竊敢披貢丹誠,請避賢路。設官分職,各有司存,
 豈應孤任愚臣,以速傾敗!東夏之事,非臣區區所能康理,願徙授親賢,濟臣顛墜。若以臣有鷹犬微勤,未忍捐棄者,乞待罪一州,效盡力命。徐方始賓,淮、汝防重,六州處分,府選便宜,輒以悉停。督任弗可虛曠,深願時降神規。」堅不許,遣其侍中梁讜詣鄴喻旨,猛乃視事如前。



 俄入為丞相、中書監、尚書令、太子太傅、司隸校尉,持節、常侍、將軍、侯如故。稍加都督中外諸軍事。猛表讓久之。堅曰:「卿昔螭蟠布衣,朕龍潛弱冠,屬世事紛紜,厲士之際,顛覆厥德。朕奇卿於暫見,擬卿為臥龍,卿亦異朕於一言,回《考槃》之雅志,豈不精契神交,千載之會!雖傅巖入
 夢,姜公悟兆,今古一時,亦不殊也。自卿輔政,幾將二紀,內釐百揆,外蕩群凶,天下向定,彞倫始敘。朕且欲從容於上,望卿勞心於下,弘濟之務,非卿而誰!」遂不許。其後數年,復授司徒。猛復上疏曰:「臣聞乾象盈虛,惟后則之;位稱以才,官非則曠。鄭武翼周,仍世載詠;王叔昧寵,政替身亡,斯則成敗之殷監,為臣之炯戒。竊惟鼎宰崇重,參路太階,宜妙盡時賢,對揚休命。魏祖以文和為公,貽笑孫后;千秋一言致相,匈奴吲之。臣何庸狷,而應斯舉!不但取嗤鄰遠,實令為虜輕秦。昔東野窮馭,顏子知其將弊。陛下不復料度臣之才力,私懼敗亡是及。且上
 虧憲典,臣何顏處之!雖陛下私臣,其如天下何!願回日月之鑒,矜臣後悔,使上無過授之謗,臣蒙覆燾之恩。」堅竟不從。猛乃受命。軍國內外萬機之務,事無巨細,莫不歸之。



 猛宰政公平,流放尸素,拔幽滯,顯賢才,外修兵革,內綜儒學,勸課農桑,教以廉恥,無罪而不刑,無才而不任,庶績咸熙,百揆時敘。於是兵彊國富,垂及升平,猛之力也。堅嘗從容謂猛曰:「卿夙夜匪懈,憂勤萬機,若文王得太公,吾將優游以卒歲。」猛曰:「不圖陛下知臣之過,臣何足以擬古人!」堅曰:「以吾觀之,太公豈能過也。」常敕其太子宏、長樂公丕等曰:「汝事王公,如事我也。」其見重如
 此。



 廣平麻思流寄關右,因母亡歸葬,請還冀州。猛謂思曰:「便可速裝,是暮已符卿發遣。」及始出關,郡縣已被符管攝。其令行禁整,事無留滯,皆此類也。性剛明清肅,於善惡尤分。微時一餐之惠,睚柴之忿,靡不報焉,時論頗以此少之。



 其年寢疾,堅親祈南北郊、宗廟、社稷,分遣侍臣禱河嶽諸祀,靡不周備。猛疾未瘳,乃大赦其境內殊死已下。猛疾甚,因上疏謝恩,并言時政,多所弘益。堅覽之流涕,悲慟左右。及疾篤,堅親臨省病,問以後事。猛曰:「晉雖僻陋吳、越,乃正朔相承。親仁善鄰,國之寶也。臣沒之後,願不以晉為圖。鮮卑、羌虜,我之仇也,終為人患,宜
 漸除之,以便社稷。」言終而死,時年五十一。堅哭之慟。比斂,三臨,謂太子宏曰:「天不欲使吾平一六合邪?何奪吾景略之速也!」贈侍中,丞相餘如故。給東園溫明秘器,帛三千匹,穀萬石。謁者僕射監護喪事,葬禮一依漢大將軍故事。謚曰武侯。朝野巷哭三日。



 苻融,字博休,堅之季弟也。少而岐嶷夙成,魁偉美姿度。健之世封安樂王,融上疏固辭,健深奇之,曰:「且成吾兒箕山之操。」乃止。苻生愛其器貌,常侍左右,未弱冠便有台輔之望。長而令譽彌高,為朝野所屬。堅僭號,拜侍中,
 尋除中軍將軍。融聰辯明慧,下筆成章,至於談玄論道,雖道安無以出之。耳聞則誦,過目不忘,時人擬之王粲。嘗著《浮圖賦》,壯麗清贍,世咸珍之。未有升高不賦,臨喪不誄,朱彤、趙整等推其妙速。旅力雄勇,騎射擊刺,百夫之敵也。銓綜內外,刑政脩理,進才理滯,王景略之流也。尤善斷獄,姦無所容,故為堅所委任。



 後為司隸校尉。京兆人董豐游學三年而返,過宿妻家,是夜妻為賊所殺。妻兄疑豐殺之,送豐有司。豐不堪楚掠,誣引殺妻。融察而疑之,問曰:「汝行往還,頗有怪異及卜筮以不?」豐曰:「初將發,夜夢乘馬南渡水,返而北渡,復自北而南,馬停水
 中,鞭策不去。俯而視之,見兩日在于水下,馬左白而濕,右黑而燥。寤而心悸,竊以為不祥。還之夜,復夢如初,問之筮者,筮者云:『憂獄訟,遠三枕,避三沐。』既至,妻為具沐,夜授豐枕。豐記筮者之言,皆不從之。妻乃自沐,枕枕而寢。」融曰:「吾知之矣。《周易》《坎》為水,馬為《離》,夢乘馬南渡,旋北而南者,從《坎》之《離》。三爻同變,變而成《離》。《離》為中女,《坎》為中男。兩日,二夫之象。《坎》為執法吏。吏詰其夫,婦人被流血而死。《坎》二陰一陽,《離》二陽一陰,相承易位。《離》下《坎》上,《既濟》,文王遇之囚牖里,有禮而生,無禮而死。馬左而濕,濕,水也,左水右馬,馮字也。兩日,昌字也。其馮昌殺之
 乎!」於是推檢,獲昌而詰之,昌具首服,曰:「本與其妻謀殺董豐,期以新沐枕枕為驗,是以誤中婦人。」在冀州,有老母遇劫於路,母揚聲唱盜,行人為母逐之。既擒劫者,劫者返誣行人為盜。時日垂暮,母及路人莫知孰是,乃俱送之。融見而笑曰:「此易知耳,可二人並走,先出鳳陽門者非盜。」既而還入,融正色謂後出者曰:「汝真是盜,何以誣人!」其發奸摘伏,皆此類也。所在盜賊止息,路不拾遺。堅及朝臣雅皆歎服,州郡疑獄莫不折之於融。融觀色察形,無不盡其情狀。雖鎮關東,朝之大事靡不馳驛與融議之。



 性至孝,初屆冀州,遣使參問其母動止,或日有
 再三。堅以為煩,月聽一使。後上疏請還侍養,堅遣使慰喻不許。久之,徵拜侍中、中書監、都督中外諸軍事、車騎大將軍、司隸校尉、太子太傅、領宗正、錄尚書事。俄轉司徒,融苦讓不受。融為將善謀略,好施愛士,專方征伐,必有殊功。



 堅既有意荊、揚,時慕容垂、姚萇等常說堅以平吳封禪之事,堅謂江東可平,寢不暇旦。融每諫曰:「知足不辱,知止不殆,窮兵極武,未有不亡。且國家,戎族也,正朔會不歸人。江東雖不絕如綖,然天之所相,終不可滅。」堅曰:「帝王歷數豈有常哉,惟德之所授耳!汝所以不如吾者,正病此不達變通大運。劉禪可非漢之遺祚,然終
 為中國之所并。吾將任汝以天下之事,奈何事事折吾,沮壞大謀!汝尚如此,況於眾乎!」堅之將入寇也,融又切切諫曰:「陛下聽信鮮卑、羌虜諂諛之言,採納良家少年利口之說,臣恐非但無成,亦大事去矣。垂、萇皆我之仇敵,思聞風塵之變,冀因之以逞其凶德。少年等皆富足子弟,希關軍旅,茍說佞諂之言,以會陛下之意,不足採也。」堅弗納。及淮南之敗,垂、萇之叛,堅悼恨彌深。



 苻朗,字元達,堅之從兄子也。性宏達,神氣爽邁,幼懷遠操,不屑時榮。堅嘗目之曰:「吾家千里駒也。」徵拜鎮東將
 軍、青州刺史,封樂安男,不得已起而就官。及為方伯,有若素士,耽玩經籍,手不釋卷,每談虛語玄,不覺日之將夕;登涉山水,不知老之將至。在任甚有稱績。



 後晉遣淮陰太守高素伐青州,朗遣使詣謝玄於彭城求降,玄表朗許之,詔加員外散騎侍郎。既至揚州,風流邁於一時,超然自得,志陵萬物,所與悟言,不過一二人而已。驃騎長史王忱,江東之俊秀,聞而詣之,朗稱疾不見。沙門釋法汰問朗曰:「見王吏部兄弟未?」朗曰:「吏部為誰?非人面而狗心、狗面而人心兄弟者乎?」王忱醜而才慧,國寶美貌而才劣於弟,故朗云然。汰悵然自失。其忤物侮人,皆此類
 也。



 謝安常設宴請之,朝士盈坐,並機褥壺席。朗每事欲誇之,唾則令小兒跪而張口,既唾而含出,頃復如之,坐者為不及之遠也。又善識味,鹹酢及肉皆別所由。會稽王司馬道子為朗設盛饌,極江左精肴。食訖,問曰:「關中之食孰若此?」答曰:「皆好,惟鹽味小生耳。」既問宰夫,皆如其言。或人殺雞以食之,既進,朗曰:「此雞棲恒半露。」檢之,皆驗。又食鵝肉,知黑白之處。人不信,記而試之,無豪釐之差。時人咸以為知味。



 後數年,王國寶譖而殺之。王忱將為荊州刺史,待殺朗而後發。臨刑,志色自若,為詩曰:「四大起何因?聚散無窮已。既過一生中,又入一死理。
 冥心乘和暢,未覺有終始。如何箕山夫,奄焉處東市!曠此百年期,遠同嵇叔子。命也歸自天,委化任冥紀。」著《苻子》數十篇行於世,亦《老》《莊》之流也。



\end{pinyinscope}