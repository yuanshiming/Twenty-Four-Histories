\article{列傳第二十三}

\begin{pinyinscope}
愍懷太子
 \gezhu{
  子臧尚}



 愍懷太子遹,字熙祖,惠帝長子,母曰謝才人。幼而聰慧,武帝愛之,恒在左右。嘗與諸皇子共戲殿上,惠帝來朝,執諸皇子手,次至太子,帝曰:「是汝兒也。」惠帝乃止。宮中嘗夜失火,武帝登樓望之。太子時年五歲,牽帝裾入闇中。帝問其故,太子曰:「暮夜倉卒,宜備非常,不宜令照見人君也。」由是奇之。嘗從帝觀豕牢,言於帝曰:「豕甚肥,何
 不殺以享士,而使久費五穀?」帝嘉其意,即使烹之。因撫其背,謂廷尉傅祗曰:「此兒當興我家。」嘗對群臣稱太子似宣帝,於是令譽流於天下。



 時望氣者言廣陵有天子氣,故封為廣陵王,邑五萬戶。以劉寔為師,孟珩為友,楊準、馮蓀為文學。惠帝即位,立為皇太子。盛選德望以為師傅,以何劭為太師,王戎為太傅,楊濟為太保,裴楷為少師,張華為少傅,和嶠為少保。元康元年,出就東宮,又詔曰:「遹尚幼蒙,今出東宮,惟當賴師傅群賢之訓。其游處左右,宜得正人使共周旋,能相長益者。」於是使太保衛瓘息庭、司空泰息略、太子太傅楊濟息毖、太子少師
 裴楷息憲、太子少傅張華息禕、尚書令華暠息恒與太子游處,以相輔導焉。



 及長,不好學,惟與左右嬉戲,不能尊敬保傅。賈后素忌太子有令譽,因此密敕黃門閹宦媚諛於太子曰:「殿下誠可及壯時極意所欲,何為恒自拘束?」每見喜怒之際,輒歎曰:「殿下不知用威刑,天下豈得畏服!」太子所幸蔣美人生男,又言宜隆其賞賜,多為皇孫造玩弄之器,太子從之。於是慢弛益彰,或廢朝侍,恒在後園游戲。愛埤車小馬,令左右馳騎,斷其鞅勒,使墮地為樂。或有犯忤者,手自捶擊之。性拘小忌,不許繕壁修墻,正瓦動屋。而於宮中為市,使人屠酤,手揣斤兩,
 輕重不差。其母本屠家女也,故太子好之。又令西園賣葵菜、藍子、雞、面之屬,而收其利。東宮舊制,月請錢五十萬,備於眾用,太子恒探取二月,以供嬖寵。洗馬江統陳五事以諫之,太子不納,語在《統傳》中。舍人杜錫以太子非賈后所生,而后性凶暴,深以為憂,每盡忠規勸太子修德進善,遠於讒謗。太子怒,使人以針著錫常所坐氈中而剌之。



 太子性剛,知賈謐恃后之貴,不能假借之。謐至東宮,或捨之而於後庭游戲。詹事裴權諫曰:「賈謐甚有寵於中宮,而有不順之色,若一旦交構,大事去矣。宜深自謙屈,以防其變,廣延賢士,用自輔翼。」太子不能從。
 初,賈后母郭槐欲以韓壽女為太子妃,太子亦欲婚韓氏以自固。而壽妻賈午及后皆不聽,而為太子聘王衍小女惠風。太子聞衍長女美,而賈后為謐聘之,心不能平,頗以為言。謐嘗與太子圍棋,爭道,成都王穎見而訶謐,謐意愈不平,因此譖太子於后曰:「太子廣買田業,多畜私財以結小人者,為賈氏故也。密聞其言云:『皇后萬歲後,吾當魚肉之。』非但如是也,若宮車晏駕,彼居大位,依楊氏故事,誅臣等而廢后於金墉,如反手耳。不如早為之所,更立慈順者以自防衛。」后納其言,又宣揚太子之短,布諸遠近。于時朝野咸知賈后有害太子意。中護
 軍趙俊請太子廢后,太子不聽。



 九年六月,有桑生於宮西廂,日長尺餘,數日而枯。十二月,賈后將廢太子,詐稱上不和,呼太子入朝。既至,后不見,置于別室,遣婢陳舞賜以酒棗,逼飲醉之。使黃門侍郎潘岳作書草,若禱神之文,有如太子素意,因醉而書之,令小婢承福以紙筆及書草使太子書之。文曰:「陛下宜自了;不自了,吾當入了之。中宮又宜速自了;不了,吾當手了之。并謝妃共要剋期而兩發,勿疑猶豫,致後患。茹毛飲血於三辰之下,皇天許當掃除患害,立道文為王,蔣為內主。願成,當三牲祠北君,大赦天下。要疏如律令。」太子醉迷不覺,遂依
 而寫之,其字半不成。既而補成之,后以呈帝。帝幸式乾殿,召公卿入,使黃門令董猛以太子書及青紙詔曰:「遹書如此,今賜死。」遍示諸公王,莫有言者,惟張華、裴頠證明太子。賈后使董猛矯以長廣公主辭白帝曰:「事宜速決,而群臣各有不同,若有不從詔,宜以軍法從事。」議至日西不決。后懼事變,乃表免太子為庶人,詔許之。於是使尚書和郁持節,解結為副,及大將軍梁王肜、鎮東將軍淮南王允、前將軍東武公澹、趙王倫、太保何劭詣東宮,廢太子為庶人。是日太子游玄圃,聞有使者至,改服出崇賢門,再拜受詔,步出承華門,乘粗犢車。澹以兵仗
 送太子妃王氏、三皇孫於金墉城,考竟謝淑妃及太子保林蔣俊。明年正月,賈后又使黃門自首,欲與太子為逆。詔以黃門首辭班示公卿。又遣澹以千兵防送太子,更幽于許昌宮之別坊,令治書御史劉振持節守之。先是,有童謠曰:「東宮馬子莫聾空,前至臘月纏汝閤。」又曰:「南風起兮吹白沙,遙望魯國鬱嵯峨,千歲髑髏生齒牙。」南風,后名;沙門,太子小字也。



 初,太子之廢也,妃父王衍表請離婚。太子至許,遺妃書曰:「鄙雖頑愚,心念為善,欲盡忠孝之節,無有惡逆之心。雖非中宮所生,奉事有如親母。自為太子以來,敕見禁檢,不得見母。自宜城君亡,
 不見存恤,恒在空室中坐。去年十二月,道文疾病困篤,父子之情,實相憐愍。於時表國家乞加徽號,不見聽許。疾病既篤,為之求請恩福,無有惡心。自道文病,中宮三遣左右來視,云:『天教呼汝。』到二十八日暮,有短函來,題言東宮發,疏云:『言天教欲見汝。』即便作表求入。二十九日早入見國家,須臾遣至中宮。中宮左右陳舞見語:『中宮旦來吐不快。』使住空屋中坐。須臾中宮遣陳舞見語:『聞汝表陛下為道文乞王,不得王是成國耳。』中宮遙呼陳舞:『昨天教與太子酒棗。』便持三升酒、大盤棗來見與,使飲酒啖棗盡。鄙素不飲酒,即便遣舞啟說不堪三升
 之意。中宮遙呼曰:『汝常陛下前持酒可喜,何以不飲?天與汝酒,當使道文差也。』便答中宮:『陛下會同一日見賜,故不敢辭,通日不飲三升酒也。且實未食,恐不堪。又未見殿下,飲此或至顛倒。』陳舞復傳語云:『不孝那!天與汝酒飲,不肯飲,中有惡物邪?』遂可飲二升,餘有一升,求持還東宮飲盡。逼迫不得已,更飲一升。飲已,體中荒迷,不復自覺。須臾有一小婢持封箱來,云:『詔使寫此文書。』鄙便驚起,視之,有一白紙,一青紙。催促云:『陛下停待。』又小婢承福持筆研墨黃紙來,使寫。急疾不容復視,實不覺紙上語輕重。父母至親,實不相疑,事理如此,實為見誣,
 想眾人見明也。」



 太子既廢非其罪,眾情憤怨。右衛督司馬雅,宗室之疏屬也,與常從督許超並有寵於太子,二人深傷之,說趙王倫謀臣孫秀曰:「國無適嗣,社稷將危,大臣之禍必起。而公奉事中宮,與賈后親密,太子之廢,皆云豫知,一旦事起,禍必及矣。何不先謀之!」秀言於趙王倫,倫深納焉。計既定,而秀說倫曰:「太子為人剛猛,若得志之日,必肆其情性矣。明公素事賈后,街談巷議,皆以公為賈氏之黨。今雖欲建大功於太子,太子雖將含忍宿忿,必不能加賞於公,當謂公逼百姓之望,翻覆以免罪耳。若有瑕釁,猶不免誅。不若遷延卻期,賈后必害
 太子,然後廢賈后,為太子報仇,猶足以為功,乃可以得志。」倫然之。秀因使反間,言殿中人欲廢賈后,迎太子。賈后聞之憂怖,乃使太醫令程據合巴豆杏子丸。三月,矯詔使黃門孫慮齋至許昌以害太子。初,太子恐見鴆,恒自煮食於前。慮以告劉振,振乃徙太子於小坊中,絕不與食,宮中猶於墻壁上過食與太子。慮乃逼太子以藥,太子不肯服,因如廁,慮以藥杵椎殺之,太子大呼,聲聞于外。時年二十三。將以庶人禮葬之,賈后表曰:「遹不幸喪亡,傷其迷悖,又早短折,悲痛之懷,不能自己。妾私心冀其刻肌刻骨,更思孝道,規為稽顙,正其名號。此志不
 遂,重以酸恨。遹雖罪在莫大,猶王者子孫,便以匹庶送終,情實憐愍,特乞天恩,賜以王禮。妾誠闇淺不識禮義,不勝至情,冒昧陳聞。」詔以廣陵王禮葬之。



 及賈庶人死,乃誅劉振、孫慮、程據等,冊復太子曰:「皇帝使使持節、兼司空、衛尉伊策故皇太子之靈曰:嗚呼!維爾少資岐嶷之質,荷先帝殊異之寵,大啟土宇,奄有淮陵。朕奉遵遺旨,越建爾儲副,以光顯我祖宗。祗爾德行,以從保傅,事親孝敬,禮無違者。而朕昧于凶構,致爾于非命之禍,俾申生、孝己復見於今。賴宰相賢明,人神憤怨,用啟朕心,討厥有罪,咸伏其辜。何補於荼毒冤魂酷痛哉?是用忉
 怛悼恨,震動於五內。今追復皇太子喪禮,反葬京畿,祠以太牢。魂而有靈,尚獲爾心。」帝為太子服長子斬衰,群臣齊衰,使尚書和郁率東宮官屬具吉凶之制,迎太子喪於許昌。



 喪之發也,大風雷電,幃蓋飛裂。又為哀策曰:「皇帝臨軒,使洗馬劉務告于皇太子之殯曰:咨爾遹!幼稟英挺,芬馨誕茂。既表髫齔,高明逸秀。昔爾聖祖,嘉爾淑美。顯詔仍崇,名振同軌。是用建爾儲副,永統皇基。如何凶戾潛構,禍害如茲!哀感和氣,痛貫四時。嗚呼哀哉!爾之降廢,實我不明。牝亂沈[C102],釁結禍成。爾之逝矣,誰百其形?昔之申生,含枉莫訟。今爾之負,抱冤于東。悠悠
 有識,孰不哀慟!壺關干主,千秋悟己。異世同規,古今一理。皇孫啟建,隆祚爾子。雖悴前終,庶榮後始。窀穸既營,將寧爾神。華髦電逝,戎車雷震。芒芒羽蓋,翼翼縉紳。同悲等痛,孰不酸辛!庶光來葉,永世不泯。」謚曰愍懷。六月己卯,葬于顯平陵。帝感閻纘之言,立思子臺,故臣江統、陸機並作誄頌焉。太子三子:[A170]、臧、尚,並與父同幽金墉。



 [A170]字道文,永康元年正月,薨。四月,追封南陽王。



 臧字敬文。永康元年四月,封臨淮王。己巳,詔曰:「咎徵數發,姦回作變,遹既逼廢,非命而沒。今立臧為皇太孫。還妃王氏以母之,稱太孫太妃。太子官屬即轉為太孫官
 屬。趙王倫行太孫太傅。」五月,倫與太孫俱之東宮,太孫自西掖門出,車服侍從皆愍懷之舊也。到銅駝街,宮人哭,侍從者皆哽咽,路人抆淚焉。桑復生于西廂,太孫廢,乃枯。永寧元年正月,趙王倫篡位,廢為濮陽王,與帝俱遷金墉,尋被害。太安初,追謚曰哀。



 尚字敬仁。永康元年四月,封為襄陽王。永寧元年八月,立為皇太孫。太安元年三月癸卯,薨,帝服齊衰期,謚曰沖太孫。



 史臣曰:愍懷挺岐嶷之姿,表夙成之質。武皇鐘愛,既深詒厥之謀;天下歸心,頗有後來之望。及于繼明宸極,守
 器春坊,四教不勤,三朝或闕,豹姿未變,鳳德已衰,信惑奸邪,疏斥正士,好屠酤之賤役,耽苑囿之佚游,可謂靡不有初,鮮克有終者也。既而中宮兇忍,久懷危害之心,外戚諂諛,競進讒邪之說;坎牲之謀已構,斃犬之譖遂行;一人乏探隱之聰,百闢無爭臣之節。遂使冤逾楚建,酷甚戾園。雖復禮備哀榮,情深憫慟,亦何補於荼毒者哉!



 贊曰:愍懷聰穎,諒惟天挺。皇祖鐘心,庶僚引領。震宮肇建,儲德不恢。掇蜂構隙,歸胙生災。既罹兇忍,徒望歸來。



\end{pinyinscope}