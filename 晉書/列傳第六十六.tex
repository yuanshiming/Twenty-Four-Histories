\article{列傳第六十六}

\begin{pinyinscope}

 列女



 羊耽妻辛氏杜有道妻嚴氏王渾妻鐘氏鄭袤妻曹氏愍懷
 太子妃王氏鄭休妻石氏陶侃母湛氏賈渾妻宗氏梁緯妻辛氏許延妻杜氏虞潭母孫氏周顗母李氏張茂妻陸氏尹虞二女荀崧小女灌王凝之妻謝氏劉臻妻陳氏皮京妻龍氏孟昶妻周氏何無忌母劉氏劉聰妻劉氏王廣女陜婦人靳康女韋逞母宋氏張天錫妾閻氏薛氏苻堅妾張氏竇滔妻蘇氏苻登妻毛氏慕容垂妻段氏段豐妻慕容氏呂纂妻楊氏李玄盛後尹氏


夫三才分位,室家之道克隆;二族交嘆,貞烈之風斯著。振高情而獨秀,魯冊於是飛華;挺峻節而孤標,周篇於焉騰茂。徽烈兼劭,柔順無愆,隔代相望,諒非一緒。然則虞興媯汭,夏盛塗山,有娀、有
 \gezhu{
  新女}
 廣隆殷之業,大任、大姒衍昌姬之化,馬鄧恭儉,漢朝推德,宣昭懿淑,魏代揚芬,斯皆禮極中闈,義殊月室者矣。至若恭姜誓節,孟母求
 仁,華率傅而經齊,樊授規而霸楚,譏文伯於奉劍,讓子發於分菽,少君之從約禮,孟光之符隱志,既昭婦則,且擅母儀。子政緝之於前,元凱編之於後,具宣閨範,有裨陰訓。故上從泰始,下迄恭安,一操可稱,一藝可紀,咸皆撰錄,為之傳云。或位極后妃,或事因夫子,各隨本傳,今所不錄。在諸偽國,暫阻王猷,天下之善,足以懲勸,亦同搜次,附于篇末。



 羊耽妻辛氏,字憲英,隴西人,魏侍中毗之女也。聰朗有才鑒。初,魏文帝得立為太子,抱毗項謂之曰:「辛君知我
 喜不?」毗以告憲英,憲英歎曰:「太子,代君主宗廟社稷者也。代君不可以不戚,主國不可以不懼,宜戚而喜,何以能久!魏其不昌乎?」



 弟敞為大將軍曹爽參軍,宣帝將誅爽,因其從魏帝出而閉城門,爽司馬魯芝率府兵斬關赴爽,呼敞同去。敞懼,問憲英曰:「天子在外,太傅閉城門,人云將不利國家,於事可得爾乎?」憲英曰:「事有不可知,然以吾度之,太傅殆不得不爾。明皇帝臨崩,把太傅臂,屬以後事,此言猶在朝士之耳。且曹爽與太傅俱受寄託之任,而獨專權勢,於王室不忠,於人道不直,此舉不過以誅爽耳。」敞曰:「然則敞無出乎?」憲英曰:「安可以不出!
 職守,人之大義也。凡人在難,猶或恤之;為人執鞭而棄其事,不祥也。且為人任,為人死,親暱之職也,汝從眾而已。」敞遂出。宣帝果誅爽。事定後,敞歎曰:「吾不謀於姊,幾不獲於義!」



 其後鐘會為鎮西將軍,憲英謂耽從子祜曰:「鐘士季何故西出?」祐曰:「將為滅蜀也。」憲英曰:「會在事縱恣,非持久處下之道,吾畏其有他志也。」及會將行,請其子琇為參軍,憲英憂曰:「他日吾為國憂,今日難至吾家矣。」琇固請於文帝,帝不聽。憲英謂琇曰:「行矣,戒之!古之君子入則致孝於親,出則致節於國;在職思其所司,在義思其所立,不遺父母憂患而已。軍旅之間可以濟者,
 其惟仁恕乎!」會至蜀果反,琇竟以全歸。祜嘗送錦被,憲英嫌其華,反而覆之,其明鑒儉約如此。泰始五年卒,年七十九。



 杜有道妻嚴氏,字憲,京兆人也。貞淑有識量。年十三,適于杜氏,十八而嫠居。子植、女韡並孤藐,憲雖少,誓不改節,撫育二子,教以禮度,植遂顯名於時,韡亦有淑德,傳玄求為繼室,憲便許之。時玄與何晏、鄧揚不穆,晏等每欲害之,時人莫肯共婚。及憲許玄,內外以為憂懼。或曰:「何、鄧執權,必為玄害,亦由排山壓卵,以湯沃雪耳,奈何
 與之為親?」憲曰:「爾知其一,不知其他。晏等驕移,必當自敗,司馬太傅獸睡耳,吾恐卵破雪銷,行自有在。」遂與玄為婚。晏等尋亦為宣帝所誅。植後為南安太守。



 植從兄預為秦州刺史,被誣,徵還,憲與預書戒之曰:「諺云忍辱至三公。卿今可謂辱矣,能忍之,公是卿坐。」預後果為儀同三司。玄前妻子咸年六歲,嘗隨其繼母省憲,謂咸曰:「汝千里駒也,必當遠至。」以其妹之女妻之。咸後亦有名於海內。其知人之鑒如此。年六十六卒。



 王渾妻鐘氏,字琰,潁川人,魏太傅繇曾孫也。父徽,黃門
 郎。琰數歲能屬文,及長,聰慧弘雅,博覽記籍。美容止,善嘯詠,禮儀法度為中表所則。既適渾,生濟。渾嘗共琰坐,濟趨庭而過,渾欣然曰:「生子如此,足慰人心。」琰笑曰:「若使新婦得配參軍,生子故不翅如此。」參軍,謂渾中弟淪也。琰女亦有才淑,為求賢夫。時有兵家子甚俊,濟欲妻之,白琰,琰曰:「要令我見之。」濟令此兵與群小雜處,琰自幃中察之,既而謂濟曰:「緋衣者非汝所拔乎?」濟曰:「是。」琰曰:「此人才足拔萃,然地寒壽促,不足展其器用,不可與婚。」遂止。其人數年果亡。琰明鑒遠識,皆此類也。



 渾弟湛妻郝氏亦有德行,琰雖貴門,與郝雅相親重,郝不以賤
 下琰,琰不以貴陵郝,時人稱鐘夫人之禮,郝夫人之法云。



 鄭袤妻曹氏。魯國薛人也。袤先娶孫氏,早亡,娉之為繼室。事舅姑甚孝,躬紡織之勤,以充奉養,至於叔妹群娣之間,盡其禮節,咸得歡心。及袤為司空,其子默等又顯朝列,時人稱其榮貴。曹氏深懼盛滿,每默等升進,輒憂之形於聲色。然食無重味,服浣濯之衣,袤等所獲祿秩,曹氏必班散親姻,務令周給,家無餘貲。



 初,孫氏瘞于黎陽,及袤薨,議者以久喪難舉,欲不合葬。曹氏曰:「孫氏元
 妃,理當從葬,不可使孤魂無所依邪。」於是備吉凶導從之儀以迎之,具衣衾几筵,親執雁行之禮,聞者莫不歎息,以為趙姬之下叔隗,不足稱也。太康元年卒,年八十三。



 愍懷太子妃王氏,太尉衍女也,字惠風。貞婉有志節。太子既廢居于金墉,衍請絕婚,惠風號哭而歸,行路為之流涕。及劉曜陷洛陽,以惠風賜其將喬屬,屬將妻之。惠風拔劍距屬曰:「吾太尉公女,皇太子妃,義不為逆胡所辱。」屬遂害之。



 鄭休妻石氏,不知何許人也。少有德操,年十餘歲,鄉邑稱之。既歸鄭氏,為九族所重。休前妻女既幼,又休父布臨終,有庶子沈生,命棄之,石氏曰:「奈何使舅之胤不存乎!」遂養沈及前妻女。力不兼舉,九年之中,三不舉子。



 陶侃母湛氏,豫章新淦人也。初,侃父丹娉為妾,生侃,而陶氏貧賤,湛氏每紡績資給之,使交結勝己。侃少為尋陽縣吏,嘗監魚梁,以一坩鮓遺母。湛氏封鮓及書,責侃曰:「爾為吏,以官物遺我,非惟不能益吾,乃以增吾憂矣。」
 鄱陽孝廉范逵寓宿於侃,時大雪,湛氏乃徹所臥親薦,自銼給其馬,又密截髮賣與鄰人,供肴饌。逵聞之,歎息曰:「非此母不生此子!」侃竟以功名顯。



 賈渾妻宗氏,不知何許人也。渾為介休令,被劉元海將喬晞攻破,死之。宗氏有姿色,晞欲納之。宗氏罵曰:「屠各奴!豈有害人之夫而欲加無禮,於爾安乎?何不促殺我!」因仰天大哭。晞遂害之,時年二十餘。



 梁緯妻辛氏,隴西狄道人也。緯為散騎常侍,西都陷沒,
 為劉曜所害。辛氏有殊色,曜將妻之。辛氏據地大哭,仰謂曜曰:「妾聞男以義烈,女不再醮。妾夫已死,理無獨全。且婦人再辱,明公亦安用哉!乞即就死。下事舅姑。逐號哭不止。曜曰:「貞婦也,任之。」自縊而死曜以禮葬之。



 許延妻杜氏,不知何許人也。延為益州別駕,為李驤所害。驤欲納杜氏為妻,杜氏號哭守夫尸,罵驤曰:「汝輩逆賊無道,死有先後,寧當久活!我杜家女,豈為賊妻也!」驤怒,遂害之。



 虞潭母孫氏,
 吳郡富春人,孫權族孫女也。初適潭父忠,恭順貞和,甚有婦德。及忠亡,遺孤藐爾,孫氏雖少,誓不改節,躬自撫養,劬勞備至。性聰敏,識鑒過人。潭始自幼童,便訓以忠義,故得聲望允洽,為朝廷所稱。永嘉末,潭為南康太守,值杜弢構逆,率眾討之。孫氏勉潭以必死之義,俱傾其資產以饋戰士,潭遂克捷。及蘇峻作亂,潭時守吳興,又假節徵峻。孫氏戒之曰:「吾聞忠臣出孝子之門,汝當捨生取義,勿以吾老為累也。」仍盡發其家僮,令隨潭助戰,貿其所服環珮以為軍資。于時會稽內史王舒遣子允之為督護,孫氏又謂潭曰:「王府君遣兒征,
 汝何為獨不?」潭即以子楚為督護,與舒允之合勢。其憂國之誠如此。拜武昌侯太夫人,加金章紫綬。潭立養堂於家,王導以下皆就拜謁。咸和末卒,所九十五。成帝遣使弔祭,謚曰定夫人。



 周顗母李氏,字絡秀,汝南人也。少時在室,顗父浚為安東將軍,時嘗出獵,遇雨,過止絡秀之家。會其家父兄不在,絡秀聞浚至,與一婢於內宰豬羊,具數十人之饌,甚精辦而不聞人聲。浚怪使覘之,獨見一女子甚美,浚因求為妾。其父兄不許,絡秀曰:「門戶殄瘁,何惜一女!若連姻
 貴族,將來庶有大益矣。」父兄許之。遂生顗及嵩、謨。而顗等既長,絡秀謂之曰:「我屈節為汝家作妾,門戶計耳。汝不與我家為親親者,吾亦何惜餘年!」顗等從命,由此李氏遂得為方雅之族。



 中興時,顗等並列顯位。嘗冬至置酒,絡秀舉觴賜三子曰:「吾本渡江,託足無所,不謂爾等並貴,列吾目前,吾復何憂!」高起曰:「恐不如尊旨。伯仁志大而才短,名重而識闇,好乘人之弊,此非自全之道。嵩性抗直,亦不容於世。唯阿奴碌碌,當在阿母目下耳。」阿奴,謨小字也。後果如其言。



 張茂妻陸氏,吳郡人也。茂為吳郡太守,被沈充所害,陸氏傾家產,率茂部曲為先登以討充。充敗,陸詣闕上書,為茂謝不剋之責。詔曰:「茂夫妻忠誠,舉門義烈,宜追贈茂太僕。」



 尹虞二女,長沙人也。虞前任始興太守,起兵討杜弢,戰敗,二女為弢所獲,並有國色,弢將妻之。女曰:「我父二千石,終不能為賊婦,有死而已!」弢並害之。



 荀崧小女灌,幼有奇節。崧為襄城太守,為杜曾所圍,力
 弱食盡,欲求救於故吏平南將軍石覽,計無從出。灌時年十三,乃率勇士數千人,踰城突圍夜出。賊追甚急,灌督厲將士,且戰且前,得入魯陽山獲免。自詣覽乞師,又為崧書與南中郎將周訪請援,仍結為兄弟,訪即遣子撫率三千人會石覽俱救崧。賊聞兵至,散走,灌之力也。



 五凝之妻謝氏,字道韞,安西將軍奕之女也。聰識有才辯。叔父安嘗問:「《毛詩》何句最佳?」道韞稱:「吉甫作頌,穆如清風。仲山甫永懷,以慰其心。」安謂有雅人深致。又嘗內集,俄而雪驟下,安曰:「何所似也?」安兄子朗曰:「散鹽空中
 差可擬。」道韞曰:「未若柳絮因風起。」安大悅。



 初適凝之,還,甚不樂。安曰:「王郎,逸少子,不惡,汝何恨也?」答曰:「一門叔父則有阿大、中郎,群從兄弟復有封、胡、羯、末,不意天壤之中乃有王郎!」封謂謝韶,胡謂謝朗,羯謂謝玄,末謂謝川,皆其小字也。又嘗譏玄學植不進,曰:「為塵務經心,為天分有限邪?」凝之弟獻之嘗與賓客談議,詞理將屈,道韞遣婢白獻之曰:「欲為小郎解圍。」乃施青綾步鄣自蔽,申獻之前議,客不能屈。



 及遭孫恩之難,舉厝自若,既聞夫及諸子已為賊所害,方命婢肩輿抽刃出門。亂兵稍至,手殺數人,乃被虜。其外孫劉濤時年數歲,賊又欲害
 之,道韞曰:「事在王門,何關他族!必其如此,寧先見殺。」恩雖毒虐,為之改容,乃不害濤。自爾嫠居會稽,家中莫不嚴肅。太守劉柳聞其名,請與談議。道韞素知柳名,亦不自阻,乃簪髻素褥坐於帳中,柳束脩整帶造于別榻。道韞風韻高邁,敘致清雅,先及家事,慷慨流漣,徐酬問旨,詞理無滯。柳退而歎曰:「實頃所未見,瞻察言氣,使人心形俱服。」道韞亦云:「親從凋亡,始遇此士,聽其所問,殊開人胸府。」



 初,同郡張玄妹亦有才質,適於顧氏,玄每稱之,以敵道韞。有濟尼者,游於二家,或問之,濟尼答曰:「王夫人神情散朗,故有林下風氣。顧家婦清心玉映,自是閨
 房之秀。」道韞所著詩賦誄頌並傳於世。



 劉臻妻陳氏者,亦聰辯能屬文。嘗正旦獻《椒花頌》,其詞曰:「旋穹周回,三朝肇建。青陽散輝,澄景載煥。標美靈葩,爰採爰獻。聖容映之,永壽於萬。」又撰元日及冬至進見之儀,行於世。



 皮京妻龍氏,字憐,西道縣人也。年十三適京,未逾年而京卒,京二弟亦相次而隕,既無胤嗣,又無期功之親。憐貨其嫁時資裝,躬自紡織,數年間三喪俱舉,葬斂既畢,每
 時享祭無闕。州里聞其賢,屢有娉者,憐誓不改醮,守節窮居五十餘載而卒。



 孟昶妻周氏,昶弟顗妻又其從妹也。二家並豐財產。初,桓玄雅重昶而劉邁毀之,昶知,深自惋失。及劉裕將建義,與昶定謀,昶欲盡散財物以供軍糧,其妻非常婦人,可語以大事,乃謂之曰:「劉邁毀我於桓公,便是一生淪陷,決當作賊。卿幸可早爾離絕,脫得富貴,相迎不晚也。」周氏曰:「君父母在堂,欲建非常之謀,豈婦人所諫!事之不成,當於奚官中奉養大家,義無歸志也。」昶愴然久
 之而起。周氏追昶坐,云:「觀君舉厝,非謀及婦人者,不過欲得財物耳。」時其所生女在抱,推而示之曰:「此而可賣,亦當不惜,況資財乎!」遂傾資產以給之,而託以他用。及事之將舉,周氏謂顗妻云:「一昨夢殊不好,門內宜浣濯沐浴以除之,且不宜赤色,我當悉取作七日藏厭。」顗妻信之,所有絳色者悉斂以付焉。乃置帳中,潛自剔綿,以絳與昶,遂得數十人被服赫然,悉周氏所出,而家人不之知也。



 何無忌母劉氏,征虜將軍建之女也。少有志節。弟牢之
 為桓玄所害,劉氏每銜之,常思報復。及無忌與劉裕定謀,而劉氏察其舉厝有異,喜而不言。會無忌夜於屏風裹制檄文,劉氏潛以器覆燭,徐登橙於屏風上窺之,既知,泣而撫之曰:「我不如東海呂母明矣!既孤其誠,常恐壽促,汝能如此,吾仇恥雪矣。」因問其同謀,知事在裕,彌喜,乃說桓玄必敗、義師必成之理以勸勉之。後果如其言。



 劉聰妻劉氏,名娥,字麗華,偽太保殷女也。幼而聰慧,晝營女工,夜誦書籍,傅母恒止之,娥敦習彌厲。每與諸兄論經義,理趣超遠,諸兄深以歎伏。性孝友,善風儀進止。
 聰既僭位,召為右貴嬪,甚寵之。俄拜為后,將起䳨儀殿以居之,其廷尉陳元達切諫,聰大怒,將斬之。娥時在後堂,私敕左右停刑,手疏啟曰:「伏聞將為妾營殿,今昭德足居,䳨儀非急。四海未一,禍難猶繁,動須人力資財,尤宜慎之。廷尉之言,國家大政。夫忠臣之諫,豈為身哉?帝王距之,亦非顧身也。妾仰謂陛下上尋明君納諫之昌,下忿闇主距諫之禍,宜賞廷尉以美爵,酬廷尉以列土,如何不惟不納,而反欲誅之?陛下此怒由妾而起,廷尉之禍由妾而招,人怨國疲,咎歸於妾,距諫害忠,亦妾之由。自古敗國喪家,未始不由婦人者也。妾每覽古事,忿
 之忘食,何意今日妾自為之!後人之觀妾,亦猶妾之視前人也,復何面目仰侍巾櫛,請歸死此堂,以塞陛下誤惑之過。」聰覽之色變,謂其群下曰:「朕比得風疾,喜怒過常。元達,忠臣也,朕甚愧之。」以娥表示元達曰:「外輔如公,內輔如此后,朕無憂矣。」及娥死,偽謚武宣皇后。



 其姊英,字麗芳,亦聰敏涉學,而文詞機辯,曉達政事,過於娥。初與娥同召拜左貴嬪,尋卒,偽追謚武德皇后。



 王廣女者,不知何許人也。容質甚美,慷慨有丈夫之節。廣仕劉聰,為西揚州刺史。蠻帥梅芳攻陷揚州,而廣被
 殺。王時年十五,芳納之。俄於闇室擊芳,不中,芳驚起曰:「何故反邪?」王罵曰:「蠻畜!我欲誅反賊,何謂反乎?吾聞父仇不同天,母仇不同地,汝反逆無狀,害人父母,而復以無禮陵人,吾所以不死者,欲誅汝耳!今死自吾分,不待汝殺,但恨不得梟汝首於通逵,以塞大恥。」辭氣猛厲,言終乃自殺,芳止之不可。



 陜婦人,不知姓字,年十九。劉曜時嫠居陜縣,事叔姑甚謹,其家欲嫁之,此婦毀面自誓。後叔姑病死,其叔姑有女在夫家,先從此婦乞假不得,因而誣殺其母,有司不
 能察而誅之。時有群鳥悲鳴尸上,其聲甚哀,盛夏暴尸十日,不腐,亦不為蟲獸所敗,其境乃經歲不雨。曜遣呼延謨為太守,既知其冤,乃斬此女,設少牢以祭其墓,謚曰孝烈貞婦,其日大雨。



 靳康女者,不知何許人也。美姿容,有志操。劉曜之誅靳氏,將納靳女為妾,靳曰:「陛下既滅其父母兄弟,復何用妾為!妾聞逆人之誅也,尚污宮伐樹,而況其子女乎!」因號泣請死,曜哀之,免康一子。



 韋逞母宋氏,不知何郡人也,家世以儒學稱。宋氏幼喪母,其父躬自養之。及長,授以《周官》音義,謂之曰:「吾家世學《周官》,傳業相繼,此又周以所制,經紀典誥,百官品物,備於此矣。吾今無男可傳,汝可受之,勿令經世。」屬天下喪亂,宋氏諷誦不輟。其後為石季龍徙之於山東,宋氏與夫在徙中,推鹿車,背負父所授書,到冀州,依膠東富人程安壽,壽養護之。逞時年小,宋氏晝則樵採,夜則教逞,然紡績無廢。壽每歎曰:「學家多士大夫,得無是乎!」逞遂學成名立,仕苻堅為太常。堅嘗幸其太學,問博士經典,乃憫禮樂遣闕。時博士盧壼對曰:「廢學既久,書傳零落,
 此年綴撰,正經粗集,唯周官禮注未有其師。窺見太常韋逞母宋氏世學家女,傳其父業,得周官音義,今年八十,視聽無闕,自非此母無可以傳授後生。」於是就宋氏家立講堂,置生員百二十人,隔絳紗幔而受業,號宋氏為宣文君,賜侍婢十人。周官學復行於世,時稱韋氏宋母焉。



 張天錫妾閻氏、薛氏,並不知何許人也,咸有寵於天錫。天錫寢疾,謂之曰:「汝二人將何以報我?吾死後,豈可為人妻乎!」皆曰:「尊若不諱,妾請效死,供灑掃地下,誓無他
 志。」及其疾篤,二姬皆自刎。天錫疾瘳,追悼之,以夫人禮葬焉。



 苻堅妾張氏,不知何許人,明辯有才識。堅將入寇江左,群臣切諫不從。張氏進曰:「妾聞天地之生萬物,聖王之馭天下,莫不順其性而暢之,故黃帝服牛乘馬,因其性也,禹鑿龍門,決洪河,因水之勢也;后稷之播殖百穀,因地之氣也;湯武之滅夏商,因人之欲也。是以有因成,無因敗。今朝臣上下皆言不可,陛下復何所因也?書曰:『天聰明自我民聰明。』天猶若此,況于人主乎!妾聞人君有
 伐國之志者,必上觀乾象,下採眾祥。天道崇遠,非妾所知。以人事言之,未見其可。諺言:「雞夜鳴者不利行師,犬群嗦者宮室必空,兵動馬驚,軍敗不歸。」秋冬已來,每夜群犬大嗥,眾雞夜鳴,伏聞廄馬驚逸,武庫兵器有聲,吉凶之理,誠非微妾所論,願陛下詳而思之。」堅曰:「軍旅之事非婦人所豫也。」遂興兵。張氏請從。堅是大敗於壽春,張氏乃自殺。



 竇滔妻蘇氏,始平人也,名蕙,字若蘭,善屬文。滔苻堅時為秦州刺史,被徙流沙,蘇氏思之,織錦為回文旋圖詩
 以贈滔。宛轉循環以讀之,詞甚悽惋,凡八百四十字,文多不錄。



 苻登妻毛氏,不知何許人,壯勇善騎射。登為姚萇所襲,營壘既陷,毛氏猶彎弓跨馬,率壯士數百人,與萇交戰,殺傷甚眾。眾寡不敵,為萇所執。萇欲納之,毛氏罵曰:「吾天子后,豈為賊羌所辱,何不速殺我!」因仰天大哭曰:「姚萇無道,前害天子,今辱皇后,皇天后土,寧不鑒照!」萇怒,殺之。



 慕容垂妻段氏,字元妃,偽右光祿大夫儀之女也。少而婉慧,有志操,常謂妹季妃曰:「我終不作凡人妻。」委妃亦曰:「妹亦不為庸夫婦。」鄰人聞而笑之。垂之稱燕王,納元妃為繼室,遂有殊寵。偽范陽王德亦娉季妃焉。姊妹俱為垂、德之妻,卒如其志。垂既僭位,拜為皇后。



 垂立其子寶為太子也,元妃謂垂曰:「太子姿質雍容,柔而不斷,承平則為仁明之主,處難則非濟世之雄,陛下託之以大業,妾未見克昌之美。遼西、高陽二王,陛下兒之賢者,宜擇一以樹之。趙王麟奸詐負氣,常有輕太子之心,陛下一旦不諱,必有難作。此陛下之家事,宜深圖之。」垂不納。
 寶及麟聞之,深以為恨。其後元妃又言之,垂曰:「汝欲使我為晉獻公乎?」元妃泣而退,告季妃曰:「太子不令,群下所知,而主上比吾為驪戎之女,何其苦哉!主上百年之後,太子必亡社稷。范陽王有非常器度,若燕祚未終,其在王乎!」



 垂死,寶嗣偽位,遣麟逼元妃曰:「后常謂主上不能嗣守大統,今竟何如?宜早自裁,以全段氏。」元妃怒曰:「汝兄弟尚逼殺母,安能保守社稷!吾豈惜死,念國滅不久耳。」遂自殺。寶議以元妃謀廢嫡統,無母后之道,不宜成喪,群下咸以為然。偽中書令眭邃大言於朝曰:「子無廢母之義,漢之安思閻后親廢順帝,猶配饗安皇,先后
 言虛實尚未可知,宜依閻后故事。」寶從之。其後麟果作亂,寶亦被殺,德後僭稱尊號,終如元妃之言。



 段豐妻慕容氏,德之女也。有才慧,善書史,能鼓琴,德既僭位,署為平原公主。年十四,適於豐。豐為人所譖,被殺,慕容氏寡歸,將改適偽壽光公餘熾。慕容氏謂侍婢曰:「我聞忠臣不事二君,貞女不更二夫。段氏既遭無辜,己不能同死,豈復有心於重行哉!今主上不顧禮義嫁我,若不從,則違嚴君之命矣。」於是剋日交禮。慕容氏姿容婉麗,服飾光華,熾睹之甚喜。經再宿,慕容氏偽辭以疾,
 熾亦不之逼。三日還第,沐浴置酒,言笑自若,至夕,密書其裙帶云:「死後當埋我於段氏墓側,若魂魄有知,當歸彼矣。」遂於浴室自縊而死。及葬,男女觀者數萬人,莫不歎息曰:「貞哉公主!」路經餘熾宅前,熾聞挽歌之聲,慟絕良久。



 呂纂妻楊氏,弘農人也。美艷有義烈。纂被呂超所殺,楊氏與侍婢十數人殯纂於城西。將出宮,超慮齎珍物出外,使人搜之。楊氏厲聲責超曰:「爾兄弟不能和睦,手刃相屠,我旦夕死人,何用金寶!」超慚而退。又問楊氏玉璽
 所在,楊氏怒曰:「盡毀之矣。」超將妻之,謂其父桓曰:「后若自殺,禍及卿宗。」桓以告楊氏,楊氏曰:「大人本賣女與氏以圖富貴,一之已甚,其可再乎!」乃自殺。



 時呂紹妻張氏亦有操行,年十四,紹死,便請為尼。呂隆見而悅之,欲穢其行,張氏曰:「欽樂至道,誓不受辱。」遂昇樓自投於地,二脛俱折,口誦佛經,俄然而死。



 涼武昭王李玄盛后尹氏,天水冀人也。幼好學,清辯有志節。初適扶風馬元正,元正卒,為玄盛繼室。以再醮之故,三年不言。撫前妻子踰於己生。玄盛之創業也,謨謀
 經略多所毗贊,故西州諺曰:「李、尹王敦煌。」



 及玄盛薨,子士業嗣位,尊為太后。士業將攻沮渠蒙遜,尹氏謂士業曰:「汝新造之國,地狹人稀,靖以守之猶懼其失,云何輕舉,窺冀非望!蒙遜驍武,善用兵,汝非其敵。吾觀其數年已來有并兼之志,且天時人事似欲歸之。今國雖小,足以為政。知足不辱,道家明誡也。且先王臨薨,遺令殷勤,志令汝曹深慎兵戰,俟時而動。言猶在耳,柰何忘之!不如勉修德政,蓄力以觀之。彼若淫暴,人將歸汝;汝茍德之不建,事之無日矣。汝此行也,非唯師敗,國亦將亡。」士業不從,果為蒙遜所滅。



 尹氏至姑臧,蒙遜引見勞之,對
 曰:「李氏為胡所滅,知復何言!」或諫之曰:「母子命懸人手,柰何倨傲!且國敗子孫屠滅,何獨無悲?」尹氏曰:「興滅死生,理之大分,何為同凡人之事,起兒女之悲!吾一婦人,不能死亡,豈憚斧鉞之禍,求為臣妾乎!若殺我者,吾之願矣。」蒙遜嘉之,不誅,為子茂虔娉其女為妻。及魏氏以武威公主妻茂虔,尹氏及女遷居酒泉。既而女卒,撫之不哭,曰:「汝死晚矣!」沮渠無諱時鎮酒泉,每謂尹氏曰:「后諸孫在伊吾,后能去不?」尹氏未測其言,答曰:「子孫流漂,託身醜虜,老年餘命,當死於此,不能作氈裘鬼也。」俄而潛奔伊吾,無諱遣騎追及之。尹氏謂使者曰:」沮渠酒泉
 許我歸北,何故來追?汝可斬吾首歸,終不回矣。」使者不敢逼而還。年七十五,卒于伊吾。



 史臣曰:夫繁霜降節,彰勁心於後凋;橫流在辰,表貞期於上德,匪伊尹子,抑亦婦人焉。自晉政陵夷,罕樹風檢,虧閑爽操,相趨成俗,薦之以劉石,汩之以苻姚。三月歌胡,唯見爭新之飾;一朝辭漢,曾微戀舊之情。馳騖風埃,脫落名教,頹縱忘反,於茲為極。至若惠風之數喬屬,道韞之對孫恩,荀女釋急於重圍,張妻報怨於強寇,僭登之后,蹈死不迴,偽纂之妃,捐生匪吝,宗辛抗情而致夭,王靳守節而就終,斯皆冥踐義途,匪因教至。聳清漢之
 喬葉,有裕徽音;振幽谷之貞蕤,無慚雅引,比夫懸梁靡顧,齒劍如歸,異日齊風,可以激揚千載矣。



 贊曰:從容陰禮,婉娩柔則。載循六行,爰昭四德。操潔風霜,譽流邦國。彤管貽訓,清芬靡忒。



\end{pinyinscope}