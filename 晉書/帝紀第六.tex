\article{帝紀第六}

\begin{pinyinscope}

 元
 帝明帝



 元皇帝諱睿,字景文,宣帝曾孫,瑯邪恭王覲之子也。咸寧二年生於洛陽,有神光之異,一室盡明,所藉槁如始刈。及長,白豪生於日角之左,隆準龍顏,目有精曜,顧眄煒如也。年十五,嗣位瑯邪王。幼有令聞。及惠皇之際,王室多故,帝每恭儉退讓,以免於禍。沈敏有度量,不顯灼然之跡,故時人未之識焉。惟侍中嵇紹異之,謂人曰:「瑯
 邪王毛骨非常,殆非人臣之相也。」元康二年,拜員外散騎常侍。累遷左將軍,從討成都王穎。蕩陰之敗也,叔父東安王繇為穎所害。帝懼禍及,將出奔。其夜月正明,而禁衛嚴警,帝無由得去,甚窘迫。有頃,雲霧晦冥,雷雨暴至,徼者皆馳,因得潛出。穎先令諸關無得出貴人,帝既至河陽,為津吏所止。從者宋典後來,以策鞭帝馬而笑曰:「舍長!官禁貴人,汝亦被拘邪!」吏乃聽過。至洛陽,迎太妃俱歸國。東海王越之收兵下邳也,假帝輔國將軍。尋加平東將軍、監徐州諸軍事,鎮下邳。俄遷安東將軍、都督揚州諸軍事。越西迎大駕,留帝居守。永嘉初,用王導
 計,始鎮建鄴,以顧榮為軍司馬,賀循為參佐,王敦、王導、周顗、刁協並為腹心股肱,賓禮名賢,存問風俗,江東歸心焉。屬太妃薨于國,自表奔喪,葬畢,還鎮,增封宣城郡二萬戶,加鎮東大將軍、開府儀同三司。受越命,討征東將軍周馥,走之。及懷帝蒙塵於平陽,司空荀籓等移檄天下,推帝為盟主。江州刺史華軼不從,使豫章內史周廣、前江州刺史衛展討禽之。愍帝即位,加左丞相。歲餘,進位丞相、大都督中外諸軍事。遣諸將分定江東,斬叛者孫弼於宣城,平杜弢於湘州,承制赦荊揚。及西都不守,帝出師露次,躬擐甲胄,移檄四方,徵天下之兵,剋日
 進討。于時有玉冊見於臨安,白玉麒麟神璽出於江寧,其文曰「長壽萬年」,日有重暈,皆以為中興之象焉。



 建武元年春二月辛巳,平東將軍宋哲至,宣愍帝詔曰:「遭運迍否,皇綱不振。朕以寡德,奉承洪緒,不能祈天永命,紹隆中興,至使兇胡敢帥犬羊,逼迫京輦。朕今幽塞窮城,憂慮萬端,恐一旦崩潰。卿指詔丞相,具宣朕意,使攝萬機,時據舊都,脩復陵廟,以雪大恥。」三月,帝素服出次,舉哀三日。西陽王羕及群僚參佐、州徵牧守等上尊號,帝不許。羕等以死固請,至於再三。帝慨然流涕曰:「孤,罪人也,惟有蹈節死義,以雪天下之恥,庶贖鈇鉞之誅。
 吾本瑯邪王,諸賢見逼不已!」乃呼私奴命駕,將反國。群臣乃不敢逼,請依魏晉故事為晉王,許之。辛卯,即王位,大赦,改元。其殺祖父母、父母,及劉聰、石勒,不從此令。諸參軍拜奉車都尉,掾屬駙馬都尉。辟掾屬百餘人,時人謂之「百六掾」。乃備百官,立宗廟社稷於建康。時四方競上符瑞,帝曰:「孤負四海之責,未能思愆,何徵祥之有?」丙辰,立世子紹為晉王太子。以撫軍大將軍、西陽王羕為太保,征南大將軍、漢安侯王敦為大將軍,右將軍王導都督中外諸軍事、驃騎將軍,左長史刁協為尚書左僕射。封王子宣城公裒瑯邪王。六月丙寅,司空、并州刺
 史、廣武侯劉琨,幽州刺史、左賢王、渤海公段匹磾,領護烏丸校尉、鎮北將軍劉翰,單于、廣寧公段辰,遼西公段眷,冀州刺史、祝阿子劭續,青州刺史、廣饒侯曹嶷,兗州刺史、定襄侯劉演,東夷校尉崔毖,鮮卑大都督慕容廆等一百八十人上書勸進,曰:



 臣聞天生蒸民,樹之以君,所以對越天地,司牧黎元。聖帝明王監其若此,知天地不可以乏饗,故屈其身以奉之;知蒸黎不可以無主,故不得已而臨之。社稷時難,則戚籓定其傾;郊廟或替,則宗哲纂其祀。是以弘振遐風,式固萬世,三五以降,靡不由之。伏惟高祖宣皇帝肇基景命,世祖武皇帝遂造區
 夏,三葉重光,四聖繼軌,惠澤侔於有虞,卜世過於周氏。自元康以來,艱難繁興,永嘉之際,氛厲彌昏,宸極失御,登遐醜裔,國家之危,有若綴旒。賴先后之德、宗廟之靈,皇帝嗣建,舊物克甄。誕授欽明,服膺聰哲,玉質幼彰,金聲夙振。塚宰攝其綱,百辟輔其政,四海想中興之美,群生懷來蘇之望。不圖天不悔禍,大災薦臻,國未忘難,寇害尋興。逆胡劉曜,縱逸西都,敢肆犬羊,陵虐天邑。臣奉表使還,乃承西朝以去年十一月不守,主上幽劫,復沈虜庭,神器流離,再辱荒逆。臣每覽史籍,觀之前載,厄運之極,古今未有。茍在食土之毛,含血之類,莫不叩心絕
 氣,行號巷哭。況臣等荷寵三世,位廁鼎司,聞問震惶,精爽飛越,且驚且惋,五情無主,舉哀朔垂,上下泣血。



 臣聞昏明迭用,否泰相濟,天命無改,歷數有歸。或多難以固邦國,或殷憂以啟聖明。是以齊有無知之禍,而小白為五伯之長;晉有麗姬之難,而重耳以主諸侯之盟。社稷靡安,必將有以扶其危;黔首幾絕,必將有以繼其緒。伏惟陛下,玄德通於神明,聖姿合於兩儀,應命世之期,紹千載之運。符瑞之表,天人有徵;中興之兆,圖讖垂典。自京畿隕喪,九服崩離,天下囂然,無所歸懷,雖有夏之遘夷羿,宗姬之離犬戎,蔑以過之。陛下撫征江左,奄有舊
 吳,柔服以德,伐叛以刑,抗明威以攝不類,杖大順以號宇內。純化既敷,則率土宅心;義風既暢,則遐方企踵。百揆時敘於上,四門穆穆於下。昔少康之隆,夏訓以為美談;宣王中興,周詩以為休詠。況茂勳格于皇天,清暉光於四海,蒼生顒然,莫不欣戴,聲教所加,願為臣妾者哉!且宣皇之胤,惟有陛下,意兆攸歸,曾無與二。天祚大晉,必將有主,主晉祀者,非陛下而誰!是以邇無異言,遠無異望,謳歌者無不吟諷徽猷,獄訟者無不思於聖德。天地之際既交,華夷之情允洽。一角之獸,連理之木,以為休徵者,蓋有百數。冠帶之倫,要荒之眾,不謀同辭者,動
 以萬計。是以臣等敢考天地之心,因函夏之趣,昧死上尊號。願陛下存舜禹至公之情,狹由巢抗矯之節;以社稷為務,不以小行為先;以黔首為憂,不以克讓為事;上尉宗廟乃顧之懷,下釋普天傾首之勤。則所謂生繁華於枯荑,育豐肌於朽骨,神人獲安,無不幸甚。



 臣聞尊位不可久虛,萬機不可久曠。虛之一日,則尊位以殆;曠之浹辰,則萬機以亂。方今踵百王之季,當陽九之會,狡寇窺窬,伺國瑕隙,黎元波蕩,無所繫心,安可廢而不恤哉?陛下雖欲逡巡,其若宗廟何?其若百姓何?昔者惠公虜秦,晉國震駭,呂去阜之謀,欲立子圉,外以絕敵人之志,內
 以固闔境之情。故曰「喪君有君,群臣輯睦,好我者勸,惡我者懼。」前事之不忘,後代之元龜也。陛下明並日月,無幽不燭,深謀遠猷,出自胸懷。不勝犬馬憂國之情,遲睹人神開泰之路,是以陳其乃誠,布之執事。臣等忝於方任,久在遐外,不得陪列闕庭,與睹盛禮,踴躍之懷,南望罔極。



 帝優令答之。語在琨傳。



 石勒將石季龍圍譙城,平西將軍祖逖擊走之。己巳,帝傅檄天下曰:「逆賊石勒,肆虐河朔,逋誅歷載,遊魂縱逸。復遣凶黨石季龍犬羊之眾,越河南渡,縱其鴆毒。平西將軍祖逖帥眾討擊,應時潰散。今遣車騎將軍,瑯邪王裒等九軍,銳卒三萬,水陸
 四道,逕造賊場,受逖節度。有能梟季龍首者。賞絹三千匹,金五十斤,封縣侯,食邑二千戶。又賊黨能梟送季龍首,封賞亦同之。」七月,散騎侍郎朱嵩、尚書郎顧球卒,帝痛之,將為舉哀。有司奏,舊尚書郎不在舉哀之例。帝曰:「衰亂之弊,特相痛悼。」於是遂舉哀,哭之甚慟。丁未,梁王悝薨。以太尉荀組為司徒。弛山澤之禁。八月甲午,封梁王世子翹為梁王。荊州刺史第五猗為賊帥杜曾所推,遂與曾同反。九月戊寅,王敦使武昌太守趙誘、襄陽太守朱軌、陵江將軍黃峻討猗,為其將杜曾所敗,誘等皆死之。石勒害京兆太守華住。梁州刺史周訪討杜曾,大
 破之。十月丁未,瑯邪王裒薨。十一月甲子,封汝南王子弼為新蔡王。丁卯,以司空劉琨為太尉。置史官,立太學。是歲,揚州大旱。



 太興元年春正月戊申朔,臨朝,懸而不樂。三月癸丑,愍帝崩問至,帝斬縗居廬。丙辰,百僚上尊號。令曰:「孤以不德,當厄運之極,臣節未立,匡救未舉,夙夜所以忘寢食也。今宗廟廢絕,億兆無係,群官庶尹,咸勉之以大政,亦何敢辭,輒敬從所執。」是日,即皇帝位。詔曰:「昔我高祖宣皇帝,誕應期運,廓開王基。景、文皇帝,奕世重光,緝熙諸夏。爰暨世祖,應天順時,受茲明命。功格天地,仁濟宇宙。
 昊天不融,降此鞠凶,懷帝短世,越去王都。天禍薦臻,大行皇帝崩殂,社稷無奉。肆群後三司六事之人,疇咨庶尹,至于華戎,致葺大命于朕躬。予一人畏天之威,用弗敢違。遂登壇南獄,受終文祖,焚柴頒瑞,告類上帝。惟朕寡德,纘我洪緒,若涉大川,罔知攸濟。惟爾股肱爪牙之佐,文武熊羆之臣,用能弼寧晉室,輔餘一人。思與萬國,共同休慶。」於是大赦,改元,文武增位二等。庚午,立王太子紹為皇太子。壬申,詔曰:「昔之為政者,動人以行不以言,應天以實不以文,故我清靜而人自正。其次聽言觀行,明試以功。其有政績可述,刑獄得中,人無怨訟,久而
 日新,及當官軟弱,茹柔吐剛,行身穢濁,修飾時譽者,各以名聞。令在事之人,仰鑒前烈,同心戮力,深思所以寬眾息役,惠益百姓,無廢朕命。遠近禮贄,一切斷之。」夏四月丁丑朔,日有食之。加大將軍王敦江州牧,進驃騎將軍王導開府儀同三司。戊寅,初禁招魂葬。乙酉,西平地震。五月癸丑,使持節、侍中、都督、太尉、并州刺史、廣武侯劉琨為段匹磾所害。六月,旱,帝親雩。改丹陽內史為丹陽尹。甲申,以尚書左僕射刁協為尚書令,平南將軍、曲陵公荀崧為尚書左僕射。庚寅,以滎陽太守李矩為都督司州諸軍事、司州刺史。戊戌,封皇子晞為武陵王。初
 置諫鼓謗木。秋七月戊申,詔曰:「王室多故,姦凶肆暴,皇綱馳墜,顛覆大猷。朕以不德,統承洪緒,夙夜憂危,思改其弊。二千石令長當祗奉舊憲,正身明法,抑齊豪強,存恤孤獨,隱實戶口,勸課農桑。州牧刺史當互相檢察,不得顧私虧公。長吏有志在奉公而不見進用者,有貪惏穢濁而以財勢自安者,若有不舉,當受故縱蔽善之罪,有而不知,當受暗塞之責。各明慎奉行。」劉聰死,其子粲嗣偽位。八月,冀、徐、青三州蝗。靳準弒劉粲,自號漢王。冬十月癸未,加廣州刺史陶侃平南將軍。劉曜僭即皇帝位于赤壁。十一月乙卯,日夜出,高三丈,中有赤青珥。新
 蔡王弼薨。加大將軍王敦荊州牧。庚申,詔曰:「朕以寡德,纂承洪緒,上不能調和陰陽,下不能濟育群生,災異屢興,咎徵仍見。壬子、乙卯,雷震暴雨,蓋天災譴戒,所以彰朕之不德也。群公卿士,其各上封事,具陳得失,無有所諱,將親覽焉。」新作聽訟觀。故歸命侯孫皓子璠謀反,伏誅。十二月,劉聰故將王騰、馬忠等誅靳準,送傳國璽於劉曜。武昌地震。丁丑,封顯義亭侯煥為瑯邪王。己卯,瑯邪王煥薨。癸巳,詔曰:「漢高經大梁,美無忌之賢;齊師入魯,脩柳下惠之墓。其吳之高德名賢或未旌錄者,具條列以聞。」江東三郡饑,遣使振給之。彭城內史周撫殺沛
 國內史周默以反。



 二年春正月丁卯,崇陽陵毀,帝素服哭三日;使冠軍將軍梁堪、守太常馬龜等脩復山陵。迎梓宮于平陽,不剋而還。二月,太山太守徐龕斬周撫,傳首京師。夏四月,龍驤將軍陳川以浚儀叛。降于石勒。太山太守徐龕以郡叛,自號兗州刺史,寇濟岱。秦州刺史陳安叛,降于劉曜。五月癸丑,太陽陵毀,帝素服哭三日。徐楊及江西諸郡蝗。吳郡大饑。平北將軍祖逖及石勒將石季龍戰于浚儀,王師敗績。壬戌,詔曰:「天下凋弊,加以災荒,百姓困窮,國用並匱,吳郡饑人死者百數。天生蒸黎而樹之以君,
 選建明哲以左右之,當深思以救其弊。昔吳起為楚悼王明法審令,捐不急之官,除廢公族疏遠,以附益將士,而國富兵強。況今日之弊,百姓凋困邪!且當去非急之務,非軍事所須者皆省之。」甲子,梁州刺史訪及杜曾戰于武當,斬之,禽第五猗。六月丙子,加周訪安南將軍。罷御府及諸郡丞,置博士員五人。己亥,加太常賀循開府儀同三司。秋七月乙丑,太常賀循卒。八月,肅慎獻楛矢石砮。徐龕寇東莞,遣太子左衛率羊鑒行征虜將軍,統徐州刺史蔡豹討之。冬十月,平北將軍祖逖使督護陳超襲石勒將桃豹,超敗,沒於陣。十一月戊寅,石勒僭
 即王位,國號趙。十二月乙亥,大赦,詔百官各上封事,并省眾役。鮮卑慕容廆襲遼東,東夷校尉、平州刺史崔毖奔高句驪。是歲,南陽王保稱晉王於祁山。三吳大饑。



 三年春正月丁酉朔,晉王保為劉曜所逼,遷于桑城。二月辛未,石勒將石季龍寇厭次,平北將軍、冀州刺史邵續擊之,續敗,沒於陣。三月,慕容廆奉送玉璽三紐。閏月,以尚書周顗為尚書僕射。夏四月壬辰,枉矢流于翼軫。五月丙寅,孝懷帝太子詮遇害於平陽,帝三日哭。庚寅,地震。是月,晉王保為其將張春所害。劉曜使陳安攻春,滅之,安因叛曜。石勒將徐龕帥眾來降。六月,大水。丁酉,
 盜殺西中郎將、護羌校尉、涼州刺史、西平公張寔,寔弟茂嗣,領平西將軍、涼州刺史。秋七月丁亥,詔曰:「先公武王、先考恭王臨君瑯邪四十餘年,惠澤加於百姓,遺愛結於人情。朕應天符,創基江表,兆庶宅心,襁負子來。瑯邪國人在此者近有千戶,今立為懷德縣,統丹陽郡。昔漢高祖以沛為湯沐邑,光武亦復南頓,優復之科一依漢氏故事。」祖逖部將衛策大破石勒別軍於汴水。加逖為鎮西將軍。八月戊午,尊敬王后虞氏為敬皇后。辛酉,遷神主于太廟。辛未,梁州刺史、安南將軍周訪卒。皇太子釋尊于太學。以湘州刺史甘卓為安南將軍、梁州刺
 史。九月,徐龕又叛,降于石勒。冬十月丙辰,徐州刺史蔡豹以畏懦伏誅。王敦殺武陵內史向碩。



 四年春二月,徐龕又帥眾來降。鮮卑末波奉送皇帝信璽。庚戌,告于太廟,乃受之。癸亥,日鬥。三月,置周易、儀禮、公羊博士。癸酉,以平東將軍曹嶷為安東將軍。夏四月辛亥,帝親覽庶獄。石勒攻厭次,陷之。撫軍將軍、幽州刺史段匹磾沒于勒。五月,旱。庚申,詔曰:「昔漢二祖及魏武皆免良人,武帝時,涼州覆敗,諸為奴婢亦皆復籍,此累代成規也。其免中州良人遭難為揚州諸郡僮客者,以備征役。」秋七月,大水。甲戌,以尚書戴若思為征西將軍、
 都督司兗豫并冀雍六州諸軍事、司州刺史,鎮合肥;丹陽尹劉隗為鎮北將軍、都督青徐幽平四州諸軍事、青州刺史,鎮淮陰。壬千,以驃騎將軍王導為司空。八月,常山崩。九月壬寅,鎮西將軍、豫州刺史祖逖卒。冬十月壬午,以逖弟侍中約為平西將軍、豫州刺史。十二月,以慕容廆為持節、都督幽平二州東夷諸軍事、平州牧,封遼東郡公。



 永昌元年正月乙卯,大赦,改元。戊辰,大將軍王敦舉兵於武昌,以誅劉隗為名,龍驤將軍沈充帥眾應之。三月,徵征西將軍戴若思、鎮北將軍劉隗還衛京都。以司
 空王導為前鋒大都督,以戴若思為驃騎將軍,丹陽諸郡皆加軍號。加僕射周顗尚書左僕射,領軍王邃尚書右僕射。以太子右衛率周筵行冠軍將軍,統兵三千討沈充。甲午,封皇子昱為瑯邪王。劉隗軍于金城,右將軍周札守石頭,帝親被甲徇六師於郊外。遣平南將軍陶侃領江州,安南將軍甘卓領荊州,各帥所統以躡敦後。四月,敦前鋒攻石頭,周札開城門應之,奮威將軍侯禮死之。敦據石頭,戴若思、劉隗帥眾攻之,王導、周顗、郭逸、虞潭等三道出戰,六軍敗績。尚書令刁協奔於江乘,為賊所害。鎮北將軍劉隗奔于石勒。帝遣使謂敦曰:「公若不
 忘本朝,於此息兵,則天下尚可共安也。如其不然,騰當歸於瑯邪,以避賢路。」辛未,大赦。敦乃自為丞相、都督中外諸軍、錄尚書事,封武昌郡公,邑萬戶。丙子,驃騎將軍、秣陵侯戴若思,尚書左僕射、護軍將軍、武城侯周顗為敦所害。敦將沈充陷吳國,魏乂陷湘州,吳國內史張茂,湘州刺史、譙王承並遇害。五月壬申,敦以太保、西陽王羕為太宰,加司空王導尚書令。乙亥,鎮南大將軍甘卓為襄陽太守周慮所害。蜀賊張龍寇巴東,建平太守柳純擊走之。石勒遣騎寇河南。六月,旱。秋七月,王敦自加兗州刺史郗鑒為安北將軍。石勒將石季龍攻陷太山,
 執守將徐龕。兗州刺史郗鑒自鄒山退守合肥。八月,敦以其見含為衛將軍,自領寧、益二州都督。瑯邪太守孫默叛,降于石勒。冬十月,大疫,死者十二三。己丑,都督荊梁二州諸軍事、平南將軍、荊州刺史、武陵侯王暠卒。辛卯,以下邳內史王邃為征北將軍、都督青徐幽平四州諸軍事,鎮淮陰。新昌太守梁碩起兵反。京師大務,黑氣蔽天,日月無光。石勒攻陷襄城、城父,遂圍譙,破祖約別軍,約退據壽春。十一月,以司徒荀組為太尉。己酉,太尉荀組薨。罷司徒,并丞相。閏月己丑,帝崩於內殿,時年四十七,葬建平陵,廟號中宗。



 帝性簡儉沖素,容納直言,虛
 己待物。初鎮江東,頗以酒廢事,王導深以為言,帝命酌,引觴覆之,於此遂絕。有司嘗奏太極殿廣室施絳帳,帝曰:「漢文集上書皁囊為帷。」遂令冬施青布,夏施青綀帷帳。將拜貴人,有司請市雀釵,帝以煩費不許。所幸鄭夫人衣無文彩。從母弟王暠為母立屋過制,流涕止之。然晉室遘紛,皇輿播越,天命未改,人謀葉贊。元戎屢動,不出江畿,經略區區,僅全吳楚。終於下陵上辱,憂憤告謝。恭儉之德雖充,雄武之量不足。始秦時望氣者云「五百年後金陵有天子氣」,故始皇東遊以厭之,改其地曰秣陵,塹北山以絕其勢。及孫權之稱號。自謂當之。孫盛以
 為始皇逮于孫氏四百三十七載,考其歷數,猶為未及;元帝之渡江也,乃五百二十六年,真人之應在於此矣。咸寧初,風吹太社樹折,社中有青氣,占者以為東莞有帝者之祥。由是徙封東莞王於瑯邪,即武王也。及吳之亡,王濬實先至建鄴,而皓之降款,遠歸璽於瑯邪。天意人事,又符中興之兆。太安之際,童謠云:「五馬浮渡江,一馬化為龍。」及永嘉中,歲、鎮、熒惑、太白聚斗、牛之間,識者以為吳越之地當興王者。是歲,王室淪覆,帝與西陽、汝南、南頓、彭城五王獲濟,而帝竟登大位焉。初,玄石圖有「牛繼馬後」,故宣帝深忌牛氏,遂為二榼,共一口,以貯酒
 焉,帝先飲佳者,而以毒酒鴆其將牛金。而恭王妃夏侯氏竟通小吏牛氏而生元帝,亦有符云。



 史臣曰:晉氏不虞,自中流外,五胡扛鼎,七廟隳尊,滔天方駕,則民懷其舊德者矣。昔光武以數郡加名,元皇以一州臨極,豈武宣余化猶暢於瑯邪,文景垂仁傳芳於南頓,所謂後乎天時,先諸人事者也。馳章獻號,高蓋成陰,星斗呈祥,金陵表慶。陶士行擁三州之旅,郢外以安;王茂弘為分陜之計,江東可立。或高旌未拂,而遐心斯偃,迴首朝陽,仰希乾棟,帝猶六讓不居,七辭而不免也。布帳綀帷,詳刑簡化,抑揚前軌,光啟中興。古首私家不
 蓄甲兵,大臣不為威福,王之常制,以訓股肱。中宗失馭強臣,自亡齊斧,兩京胡羯,風埃相望。雖復六月之駕無聞,而鴻鴈之歌方遠,享國無幾,哀哉!



 明皇帝諱紹,字道畿,元皇帝長子也。幼而聰哲,為元帝所寵異。年數歲,嘗坐置膝前,屬長安使來,因問帝曰:「汝謂日與長安孰遠?」對曰:「長安近。不聞人從日邊來,居然可知也。」元帝異之。明日,宴群僚,又問之。對曰:「日近。」元帝失色,曰:「何乃異間者之言乎?」對曰:「舉目則見日,不見長安。」由是益奇之。



 建興初,拜東中郎將,鎮廣陵。元帝為晉
 王,立為晉王太子。及帝即尊號,立為皇太子。性至孝,有文武才略,欽賢愛客,雅好文辭。當時名臣,自王導、庚亮、溫嶠、桓彞、阮放等,咸見親待。嘗論聖人真假之意,導等不能屈。又習武藝,善撫將士。于時東朝濟濟,遠近屬心焉。及王敦之亂,六軍敗績,帝欲帥將士決戰,升車將出,中庶子溫嶠固諫,抽劍斬鞅,乃止。敦素以帝神武明略,朝野之所欽信,欲誣以不孝而廢焉。大會百官而問溫嶠曰:「皇太子以何德稱?」聲色俱厲,必欲使有言。嶠對曰:「鉤深致遠,蓋非淺局所量。以禮觀之,可稱為孝矣。」眾皆以為信然,敦謀遂止。



 永昌元年閏月己丑,元帝崩。庚寅,太子即皇帝位,大赦,尊所生荀氏為建安郡君。



 太寧元年春正月癸巳,黃霧四塞,京師火。李雄使其將李驤、任回寇臺登,將軍司馬玖死之。越巂太守李釗、漢嘉太守王載以郡叛,降于驤。二月,葬元帝於建平陵,帝徒跣至于陵所。以特進華恒為驃騎將軍、都督石頭水陸軍事。乙丑,黃霧四塞。丙寅,隕霜。壬申,又隕霜,殺穀。三月戊寅朔,改元,臨軒,停饗宴之禮,懸而不樂。丙戌,隕霜,殺草。饒安、東光、安陵三縣災,燒七千餘家,死者萬五千人。石勒攻陷下邳,徐州刺史卞敦退保盱眙。王敦獻皇
 帝信璽一紐。敦將謀篡逆,諷朝廷徵己,帝乃手詔徵之。夏四月,敦下屯于湖,轉司空王導為司徒,自領揚州牧。巴東監軍柳純為敦所害。以尚書陳為都督幽平二州諸軍事、幽州刺史。五月,京師大水。李驤等寇寧州,刺史王遜遣將姚岳距戰於堂狼,大破之。梁碩攻陷交州,刺史王諒死之。六月壬子,立皇后庾氏。平南將軍陶侃遣參軍高寶攻梁碩,斬之,傳首京師。進侃位征南大將軍、開府儀同三司。秋七月丙子朔,震太極殿柱。是月,劉曜攻陳安於隴城,滅之。八月,以安北將軍郗鑒為尚書令。石勒將石季龍攻陷青州,刺史曹嶷遇害。冬十一月,
 王敦以其兄征南大將軍含為征東大將軍、都督揚州江西諸軍事。以軍事饑乏,調刺史以下米各有差。



 二年春正月丁丑,帝臨朝,停饗宴之禮,懸而不樂。庚辰,赦五歲刑以下。術人李脫造妖書惑眾,斬于建康市。石勒將石季龍寇兗州,刺史劉遐自彭城退保泗口。三月,劉曜將康平寇魏興,及南陽。夏五月,王敦矯詔拜其子應為武衛將軍,兄含為驃騎大將軍。帝所親信常從督公乘雄、冉曾並為敦所害。六月,敦將舉兵內向,帝密知之,乃乘巴滇駿馬微行,至于湖,陰察敦營壘而出。有軍士疑帝非常人。又敦正書寢,夢日環其城,驚起曰:「此必
 黃鬚鮮卑奴來也。」帝母荀氏,燕代人,帝狀類外氏,鬚黃,敦故謂帝云。於是使五騎物色追帝。帝亦馳去,馬有遺糞,輒以水灌之。見逆旅賣食嫗,以七寶鞭與之,曰:「後有騎來,可以此示也。」俄而追者至,問嫗。嫗曰:「去巳遠矣。」因以鞭示之。五騎傳玩,稽留遂久,又見馬糞冷,以為信遠而止不追。帝僅而獲免。丁卯,加司徒王導大都督、假節,領揚州刺史,以丹陽尹溫嶠為中壘將軍,與右將軍卞敦守石頭,以光祿勛應詹為護軍將軍、假節、督朱雀橋南諸軍事,以尚書令郗鑒行衛將軍、都督從駕諸軍事,以中書監庾亮領左衛將軍,以尚書卞壼行中軍將軍。
 征平北將軍、徐州刺史王邃,平西將軍、豫州刺史祖約,北中郎將、兗州刺史劉遐,奮武將軍、臨淮太守蘇峻,奮威將軍、廣陵太守陶瞻等還衛京師。帝次于中堂。秋七月壬申朔,敦遣其兄含及錢鳳、周撫、鄧岳等水陸五萬,至于南岸。溫嶠移屯水北,燒朱雀桁,以挫其鋒。帝躬率六軍,出次南皇堂。至癸酉夜,募壯士,遣將軍段秀、中軍司馬曹渾、左衛參軍陳嵩、鐘寅等甲卒千人渡水,掩其未備。平旦,戰于越城,大破之,斬其前鋒將何康。王敦憤惋而死。前宗正虞潭起義師於會稽。沈充帥萬餘人來會含等,庚辰,築壘于陵口。丁亥,劉遐、蘇峻等帥精卒萬
 人以至,帝夜見,勞之,賜將士各有差。義興人周蹇殺敦所署太守劉芳,平西將軍祖約逐敦所署淮南太守任台于壽春。乙未,賊眾濟水,護軍將軍應詹帥建威將軍趙胤等距戰,不利。賊至宣陽門,北中郎將劉遐、蘇峻等自南塘橫擊,大破之。劉遐又破沈充于青溪。丙申,賊燒營宵遁。丁西,帝還宮,大赦,惟敦黨不原。於是分遣諸將追其黨與,悉平之。封司徒王導為始興郡公,邑三千戶,賜絹九千匹;丹陽尹溫嶠建寧縣公,尚書卞壼建興縣公,中書監庾亮永昌縣公,北中郎將劉遐泉陵縣公,奮武將軍蘇峻邵陵縣公,邑各一千八百戶,絹各五千四
 百匹;尚書令郗鑒高平縣侯,護軍將軍應詹觀陽縣侯,邑各千六百戶,絹各四千八百匹;建威將軍趙胤湘南縣侯,右將軍卞敦益陽縣侯,邑各千六百戶,絹各三千二百匹。其餘封賞各有差。冬十月,以司徒王導為太保、領司徒,太宰、西陽王羕領太尉,應詹為平南將軍、都督江州諸軍事、江州刺史,劉遐為監淮北諸軍事、徐州刺史,庾亮為護軍將軍。詔王敦群從一無所問。是時,石勒將石生屯洛陽,豫州刺史祖約退保壽陽。十二月壬子,帝謁建平陵,從大祥之禮。梁水太守爨亮、盜竊州太守李逷以興古叛,降于李雄。沈充故將顧颺反於武康,攻燒
 城邑,州縣討斬之。



 三年春二月戊辰,復三族刑,惟不及婦人。三月,幽州刺史段末波卒,以弟牙嗣。戊辰,立皇子衍為皇太子,大赦,增文武位二等,大酺三日,賜鰥寡孤獨帛,人二匹。癸巳,徵處士臨海任旭、會稽虞喜並為博士。夏四月,詔曰:「大事初定,其命惟新。其令太宰、司徒巳下,詣都坐參議政道,諸所因革,務盡事中。」又詔曰:「餐直言,引亮正,想群賢達吾此懷矣。予違汝弼,堯舜之相君臣也。吾雖虛闇,庶不距逆耳之談。稷契之任,君居之矣。望共勖之。」己亥,雨雹。石勒將石良寇兗州,刺史檀贇力戰,死之。將軍李矩
 等並眾潰而歸,石勒盡陷司、兗、豫三州之地。五月,以征南大將軍陶侃為征西大將軍、都督荊湘雍梁四州諸軍事、荊州刺史,王舒為安南將軍、都督廣州諸軍事、廣州刺史。六月,石勒將石季龍攻劉曜將劉岳於新安,陷之。以廣州刺史王舒為都督湘州諸軍事、湘州刺史,湘州刺史劉顗為平越中郎將、都督廣州諸軍事、廣州刺史。大旱,自正月不雨,至于是月。秋七月辛未,以尚書令郗鑒為車騎將軍、都督青兗二州諸軍事、假節,鎮廣陵,領軍將軍卞壼為尚書令。詔曰:「三恪二王,世代之所重;興滅繼絕,政道之所先。又宗室哲王有功勛於大晉受
 命之際者,佐命功臣,碩德名賢,三祖所與共維大業,咸開國胙土、誓同山河者,而並廢絕,禋祀不傳,甚用懷傷。主者其祥議諸應立後者以聞。」又詔曰:「郊祀天地,帝王之重事。自中興以來,惟南郊,未曾北郊,四時五郊之禮都不復設,五嶽、四瀆、名山、大川載在祀典應望秩者,悉廢而未舉。主者其依舊詳處。」八月,詔曰:「昔周武克殷,封比干之墓;漢高過趙,錄樂毅之後,追顯既往,以勸將來也。吳時將相名賢之胄,有能纂修家訓,又忠孝仁義,靜己守真,不聞于時者,州郡中正亟以名聞,勿有所遺。」閏月,以尚書左僕射荀崧為光祿大夫、錄尚書事,尚書鄧
 攸為尚書左僕射。壬午,帝不豫,召太宰、西陽王羕,司徒王導,尚書令卞壼,車騎將軍郗鑒,護軍將軍庾亮,領軍將軍陸曄,丹陽尹溫嶠並受遺詔,輔太子。丁亥,詔曰:「自古有死,賢聖所同,壽夭窮達,歸於一概,亦何足特痛哉!朕枕疾已久,常慮忽然。仰惟祖宗洪基,不能克終堂構,大恥未雪,百姓塗炭,所以有慨耳。不幸之日,斂以時服,一遵先度,務從簡約,勞眾崇飾,皆勿為也。衍以幼弱,猥當大重,當賴忠賢,訓而成之。昔周公匡輔成王,霍氏擁育孝昭,義行前典,功冠二代,豈非宗臣之道乎?凡此公卿,時之望也。敬聽顧命,任托付之重,同心斷金,以謀王
 室。諸方嶽征鎮,刺史將守,皆朕扞城,推轂於外,雖事有內外,其致一也。故不有行者,誰扞牧圉?譬若脣齒,表裏相資。宜戮力一心,若合符契,思美焉之美,以緝事為期。百辟卿士,其總己以聽於塚宰,保祐沖幼,弘濟艱難,永令祖宗之靈,寧於九天之上,則朕沒于地下,無恨黃泉。」戊子,帝崩于東堂,年二十七,葬武平陵,廟號肅祖。



 帝聰明有機斷,尤精物理。于時兵凶歲饑,死疫過半,虛弊既甚,事極艱虞。屬王敦挾震主之威,將移神器。帝騎驅遵養,以弱制強,潛謀獨斷,廓清大昆。改授荊、湘等四州,以分上流之勢,撥亂反正,強本弱枝。雖享國日淺,而規模
 弘遠矣。



 史臣曰:維揚作宇,憑帶洪流,楚江恒戰,方城對敵,不得不推誠將相,以總戎麾。樓船萬計,兵倍王室,處其利而無心者,周公其人也。威權外假,嫌隙內興,彼有順流之師,此無強籓之援。商逢九亂,堯止八音,明皇負圖,屬在茲日。運龍韜於掌握,起天旆於江靡,燎其餘燼,有若秋原。去縗絰而踐戎場,斬鯨鯢而拜園闕。鎮削威權,州分江漢,覆車不踐。貽厥孫謀。其後七十餘年,終罹敬道之害。或曰:「興亡在運,非止上流。」豈創制不殊,而弘之者異
 也。



 贊曰:傾天起害,猛獸呈災。瑯邪之子,仁義歸來,龔行趙璧,命箠荊臺。云瞻北晦,江望南開。晉陽禦敵,河西全壤。胡寇雖艱,靈心弗爽。三方馳騖,百蠻從響。寶命還昌,金輝載明。明後岐嶷,軍書接要。莽首晨懸,董臍昏燎。厥德不回,餘風可劭。



\end{pinyinscope}