\article{志第十七}

\begin{pinyinscope}

 五行上



 夫
 帝王者,配德天地,葉契陰陽,發號施令,動關幽顯,休咎之徵,隨感而作,故《書》曰:「惠迪吉,從逆凶,惟影響。」昔伏羲氏繼天而王,受《河圖》,則而畫之,八卦是也。禹治洪水,賜《洛書》,法而陳之,《洪範》是也。聖人行其道,寶其真,自天祐之,吉無不利。三五已降,各有司存。爰及殷之箕子,在父師之位,典斯大範。周既克殷,以箕子歸,武王虛己而
 問焉。箕子對以禹所得《雒書》,授之以垂訓。然則《河圖》、《雒書》相為經緯,八卦、九章更為表裏。殷道絕,文王演《周易》;周道弊,孔子述《春秋》。奉乾坤之陰陽,郊洪範之休咎,天人之道粲然著矣。



 漢興,承秦滅學之後,文帝時,虙生創紀《大傳》,其言五行庶徵備矣。後景武之際,董仲舒治《公羊春秋》,始推陰陽,為儒者之宗。宣元之間,劉向治《穀梁春秋》,數其禍福,傳以《洪範》,與仲舒多所不同。至向子歆治《左氏傳》,其言《春秋》及五行,又甚乖異。班固據《大傳》,采仲舒、劉向、劉歆著《五行志》,而傳載眭孟、夏侯勝、京房、谷永、李尋之徒所陳行事,訖于王莽,博通祥變,以傳《春秋》。



 綜
 而為言,凡有三術。其一曰,君治以道,臣輔克忠,萬物咸遂其性,則和氣應,休徵效,國以安。二曰,君違其道,小人在位,眾庶失常,則乖氣應,咎徵效,國以亡。三曰,人君大臣見災異,退而自省,責躬修德,共禦補過,則消禍而福至。此其大略也。輒舉斯例,錯綜時變,婉而成章,有足觀者。及司馬彪纂光武之後以究漢事,災眚之說不越前規。今採黃初以降言祥異者,著于此篇。



 《經》曰:「五行:一曰水,二曰火,三曰木,四曰金,五曰土。水曰潤下,火曰炎上,木曰曲直,金曰從革,土爰稼穡。」



 《傳》曰:「田獵不宿,飲食不享,出入不節,奪農時及有姦謀,
 則木不曲直。」



 說曰:木,東方也。於《易》,地上之木為《觀》。於王事,威儀容貌亦可觀者也。故行步有佩玉之度,登車有和鸞之節,三驅之制,飲食有享獻之禮;出入有名,使人以時,務在勸農桑,謀在安百姓,如此,則木得其性矣。若乃田獵馳騁,不反宮室;飲食沈湎,不顧法度;妄興徭役,以奪農時;作為姦詐,以傷人財,則木失其性矣。蓋工匠之為輪矢者多傷敗,及木為變怪,是為不曲直。



 魏文帝黃初六年正月,雨,木冰。案劉歆說,上陽施不下通,下陰施不上達,故雨,而木為之冰,氛氣寒,木不曲直
 也。劉向曰,冰者陰之盛,木者少陽,貴臣卿大夫象也。此人將有害,則陰氣脅木,木先寒,故得雨而冰也。是年六月,利成郡兵蔡方等殺太守徐質,據郡反。太守,古之諸侯,貴臣有害之應也。一說以木冰為木介,介者甲兵之象。是歲,既討蔡方,又八月天子自將以舟師征吳,戍卒十餘萬,連旌數百里,臨江觀兵,又屬常雨也。



 元帝太興三年二月辛未,雨,木冰。後二年,周顗等遇害,是陽施不下通也。



 穆帝永和八年正月乙巳,雨,木冰。是年殷浩北伐,明年軍敗,十年廢黜。又曰,荀羨、殷浩北伐,桓溫入關之象也。



 孝武帝太元十四年十二月乙巳,雨,木冰。明年二月王恭為北籓,八月庾楷為西籓,九月王國寶為中書令,尋加領軍將軍,十七年殷仲堪為荊州,雖邪正異規,而終同夷滅,是其應也。



 吳孫亮建興二年,諸葛恪征淮南,後所坐聽事棟中折。恪妄興征役,奪農時,作邪謀,傷國財力,故木失其性致毀折也。及旋師而誅滅,於《周易》又為「棟撓之凶」也。



 武帝太康五年五月,宣帝廟地陷,梁折。八年正月,太廟殿又陷,改作廟,築基及泉。其年九月,遂更營新廟,遠致名材,雜以銅柱,陳勰為匠,作者六萬人。至十年四月乃
 成,十一月庚寅梁又折。天戒若曰,地陷者分離之象,梁折者木不曲直也。明年帝崩,而王室遂亂。



 惠帝太安二年,成都王穎使陸機率眾向京都,擊長沙王乂,及軍始引而牙竿折,俄而戰敗,機被誅,穎遂奔潰,卒賜死。此姦謀之罰,木不曲直也。



 元帝太興四年,王敦在武昌,鈴下儀仗生華如蓮華,五六日而萎落。此木失其性。干寶以為狂華生枯木,又在鈴閣之間,言威儀之富,榮華之盛,皆如狂華之發,不可久也。其後王敦終以逆命加戮其尸。一說亦華孽也,於《周易》為「枯楊生華」。



 桓玄始篡,龍旂竿折。時玄田獵無度,飲食奢恣,土木妨農,又多姦謀,故木失其性。天戒若曰,旂所以掛三辰,章著明也,旂竿之折,高明去矣。玄果敗。



 《傳》:「棄法津,逐功臣,殺太子,以妾為妻,則火不炎上。」



 說曰:火,南方,揚光輝為明者也。其於王者,南面嚮明而治。《書》云:「知人則哲,能官人。」故堯舜舉群賢而命之朝,遠四佞而放諸野。孔子曰:「浸潤之譖,膚受之愬,不行焉,可謂明矣。」賢佞分別,官人有序,帥由舊章,敬重功勛,殊別嫡庶,如此則火得其性矣。若乃信道不篤,或耀虛偽,讒夫昌,邪勝正,則火失其性矣。自上而降,及濫炎妄起,焚
 宗廟,燒宮館,雖興師眾,不能救也,是為火不炎上。



 魏明帝太和五年五月,清商殿災。初,帝為平原王,納河南虞氏為妃。及即位,不以為后,更立典虞車工卒毛嘉女為后。后本仄微,非所宜升,以妾為妻之罰也。



 青龍元年六月,洛陽宮鞠室災。二年四月,崇華殿災,延於南閣,繕復之。至三年七月,此殿又災。帝問高堂隆:「此何咎也?於禮寧有祈禳之義乎?」對曰:「夫災變之發,皆所以明教誡也,惟率禮修德可以勝之。《易傳》曰:『上不儉,下不節,孽火燒其室。』又曰:『君高其臺,天火為災。』此人君茍飾宮室,不知百姓空竭,故天應之以旱,火從高殿起也。
 案《舊占》曰:『災火之發,皆以臺榭宮室為誡。』今宜罷散作役,務從節約,清掃所災之處,不敢於此有所營造,萐莆嘉禾必生此地,以報陛下虔恭之德。」帝不從。遂復崇華殿,改曰九龍。以郡國前後言龍見者九,故以為名。多棄法度,疲眾逞欲,以妾為妻之應也。



 吳孫亮建興元年十二月,武昌端門災,改作,端門又災。內殿門者,號令所出;殿者,聽政之所。是時諸葛恪執政,而矜慢放肆,孫峻總禁旅,而險害終著。武昌,孫氏尊號所始。天戒若曰,宜除其貴要之首者,恪果喪眾殄人,峻授政於綝,綝廢亮也。或曰,孫權毀撤武昌以增太初宮,
 諸葛恪有遷都意,更起門殿,事非時宜,故見災也。京房《易傳》曰:「君不思道,厥妖火燒宮。」



 太平元年二月朔,建鄴火,人之火也。是秋,孫綝始執政,矯以亮詔殺呂據、滕胤。明年,又輒殺朱異。棄法律逐功臣之罰也。



 孫休永安五年二月,城西門北樓災。六年十月,石頭小城火,燒西南百八十丈。是時嬖人張布專擅國勢,多行無禮,而韋昭、盛沖終斥不用,兼遣察戰等為內史,驚擾州郡,致使交止反亂,是其咎也。



 孫皓建衡二年三月,大火,燒萬餘家,死者七百人。案《春
 秋》齊大災,劉向以為桓公好內,聽女口,妻妾數更之罰也。時皓制令詭暴,蕩棄法度,勞臣名士,誅斥甚眾,後宮萬餘,女謁數行,其中隆寵佩皇后璽綬者又多矣,故有大火。



 武帝太康八年三月乙丑,震災西閣楚王所止坊及臨商觀窗。十年四月癸丑,崇賢殿災。十一月庚辰,含章鞠室、修成堂前廡、景坊東屋、暉章殿南閣火。時有上書曰:「漢王氏五侯,兄弟迭任,今楊氏三公,並在大位,故天變屢見,竊為陛下憂之。」由是楊珧求退。是時帝納馮紞之間,廢張華
 之功,聽楊駿之讒,離衛瓘之寵,此逐功臣之罰也。明年,宮車宴駕。其後楚王承竊發之旨,戮害二公,身亦不免。震災其坊,又天意乎。



 惠帝元康五年閏月庚寅,武庫火。張華疑有亂,先命固守,然後救火。是以累代異寶,王莽頭,孔子屐,漢高祖斷白蛇劍及二百八萬器械,一時蕩盡。是後愍懷太子見殺之罰也。天戒若曰,夫設險擊柝,所以固其國,儲積戒器,所以戒不虞。今冢嗣將傾,社稷將泯,禁兵無所復施,皇旅又將誰衛。帝后不悟,終喪四海,是其應也。張華、閻纂皆曰,武庫火而氐羌反,太子見廢,則四海可知。」



 八年十一月,高原陵火。是時賈后凶恣,賈謐擅朝,惡積罪稔,宜見誅絕。天戒若曰,臣妾之不可者,雖親貴莫比,猶宜忍而誅之,如吾燔高原陵也。帝既眊弱,而張華又不納裴頠、劉卞之謀,故后遂與謐殺太子也。干寶以為「高原陵火,太子廢之應。漢武帝世,高園便殿火,董仲舒對與此占同」。



 永康元年,帝納皇后羊氏,后將入宮,衣中忽有火,眾咸怪之。永興元年,成都王遂廢后,處之金墉城。是後還立,立而復廢者四。又詔賜死,荀籓表全之。雖來還在位,然憂逼折辱,終古未聞。此孽火之應也。



 永興二年七月甲午,尚書諸曹火起,延崇禮闥及閣道。夫百揆王化之本,王者棄法津之應也。後清河王覃入嗣,不終於位,又殺太子之罰也。



 孝懷帝永嘉四年十一月,襄陽火,燒死者三千餘人。是時王如自號大將軍、司雍二州牧,眾四五萬,攻略郡縣。此下陵上,陽失其節之應也。



 元帝太興中,王敦鎮武昌,武昌災,火起,興眾救之,救於此而發於彼,東西南北數十處俱應,數日不絕。舊說所謂「濫炎妄起,雖興師眾,不能救之」之謂也。干寶以為「此臣而君行,亢陽失節,是為王敦陵上,有無君之心,故災
 也。」



 永昌二年正月癸巳,京師大火。三月,饒安、東光、安陵三縣火,燒七千餘家,死者萬五千人。



 明帝太寧元年正月,京都火。是時王敦威侮朝廷,多行無禮,內外臣下咸懷怨毒,極陰生陽也。



 成帝咸和二年五月,京師火。



 康帝建元元年七月庚申,吳郡災。



 穆帝永和五年六月,震災石季龍太武殿及兩廟端門。震災月餘乃滅,金石皆盡。其後季龍死,大亂,遂滅亡。



 海西公太和中,郗愔為會稽太守。六月大旱災,火燒數
 千家。延及山陰倉米數百萬斛,炎煙蔽天,不可撲滅。此亦桓溫強盛,將廢海西,極陰生陽之應也。



 孝武帝寧康元年三月,京師風火大起。是時桓溫入朝,志在陵上,少主踐位,人懷憂恐,此與太寧火事同。



 太元十年正月,國子學生因風放火,焚房百餘間。是後考課不厲,賞黜無章。蓋有育才之名,而無收賢之實,此不哲之罰先兆也。



 十三年十二月乙未,延賢堂災。是月丙申,螽斯則百堂及客館、驃騎府庫皆災。于時朝多弊政,衰陵日兆,不哲之罰,皆有象類。主相不悟,終至亂亡。會稽王道子寵幸
 尼及姏母,各樹用其親戚,乃至出入宮掖,禮見人主。天戒若曰,登延賢堂及客館者多非其人,故災之也。又,孝武帝更不立皇后,寵幸微賤張夫人,夫人驕妒,皇子不繁,乖「螽斯則百」之道,故災其殿焉。道子復賞賜不節,故府庫被災,斯亦其罰也。



 安帝隆安二年三月,龍舟二乘災,是水沴火也。其後桓玄篡位,帝乃播越。天戒若曰,王者流遷,不復御龍舟,故災之耳。



 元興元年八月庚子,尚書下舍曹火。時桓玄遙錄尚書,故天火,示不復居也。



 三年,盧循攻略廣州,刺史吳隱之閉城固守。其十月壬戌夜,火起。時百姓避寇盈滿城內,隱之懼有應賊者,但務嚴兵,不先救火。由是府舍焚蕩,燒死者萬餘人,因遂散潰,悉為賊擒。



 義熙四年七月丁酉,尚書殿中吏部曹火。九年,京都大火,燒數千家。十一年,京都所在大行火災,吳界尤甚。火防甚峻,猶自不絕。王弘時為吳郡,晝在聽事,見天上有一赤物下,狀如信幡,遙集路南人家屋上,火即大發。弘知天為之災,故不罪火主。此帝室衰微之應也。



 《
 傳》曰:「修宮室,飾臺榭,內淫亂,犯親戚,侮兄弟,則稼穡不成。」



 說曰:土,中央,生萬物者也。其於王者,為內事,宮室、夫婦、親屬,亦相生者也。古者天子諸侯,宮廟大小高卑有制,后夫人媵妾多少有度,九族親疏長幼有序。孔子曰:「禮,與其奢也,寧儉。」故禹卑宮室,文王刑於寡妻,此聖人之所以昭教化也。如此,則土得其性矣。若乃奢淫驕慢,則土失其性。亡水旱之災而草木百穀不熟,是為稼穡不成。



 吳孫皓時,常歲無水旱,苗稼豐美而實不成,百姓以飢,
 闔境皆然,連歲不已。吳人以為傷露,非也。案劉向《春秋說》曰「水旱當書,不書水旱而曰大無麥禾者,土氣不養,稼穡不成」,此其義也。皓初遷都武昌,尋還建鄴,又起新館,綴飾珠玉,壯麗過甚,破壞諸營,增廣苑囿,犯暑妨農,官私疲怠。《月令》,季夏不可以興土功,



 皓皆冒之。此修宮室飾臺榭之罰也。



 元帝太興二年,吳郡、吳興、東陽無麥禾,大饑。



 成帝咸和五年,無麥禾,天下大饑。



 穆帝永和十年,三麥不登。十二年,大無麥。



 孝武太元六年,無麥禾,天下大饑。



 安帝元興元年,無麥禾,天下大饑。



 《傳》曰:「好戰攻,輕百姓,飾城郭,侵邊境,則金不從革。」



 說曰:金,西方,萬物既成,殺氣之始也。故立秋而鷹隼擊,秋分而微霜降。其於王事,出軍行師,把旄杖鉞,誓士眾,抗威武,所以征叛逆,止暴亂也。《詩》云:「有虔執鉞,如火烈烈。」又曰:「載戢干戈,載橐弓矢。」動靜應宜,說以犯難,人忘其死,金得其性矣。若乃貪慾恣睢,務立威勝,不重人命,則金失其性。蓋工冶鑄金鐵,冰滯涸堅,不成者眾,乃為變怪,是為金不從革。



 魏時張掖石瑞,雖是晉之符命,而
 於魏為妖。好攻戰,輕百姓,飾城郭,侵邊境,魏氏三祖皆有其事。石圖發於非常之文,此不從革之異也。晉定大業,多斃曹氏,石瑞文「大討曹」之應也。案劉歆以《春秋》石言于晉,為金石同類也,是為金不從革,失其性也,劉向以為石白色為主,屬白祥。



 魏明帝青龍中,盛修宮室,西取長安金狄,承露槃折,聲聞數十里,金狄泣,於是因留霸城。此金失其性而為異也。



 吳時,歷陽縣有巖穿,似印,咸云「石印封發,天下太平」。孫皓天璽元年,印發。又,陽羨山有石穴,長十餘丈。皓初修
 武昌宮,有遷都之意。是時武昌為離宮。班固云「離宮與城郭同占」,飾城郭之謂也。其寶鼎三年後,皓出東關,遣丁奉至合肥,建衡三年皓又大舉出華里,侵邊境之謂也。故令金失其性,卒面縛而吳亡。



 惠帝元康三年閏二月,殿前六鐘皆出涕,五刻止。前年賈后殺楊太后於金墉城,而賈后為惡不止,故鐘出涕,猶傷之也。



 永興元年,成都伐長沙,每夜戈戟鋒有火光如懸燭。此輕人命,好攻戰,金失其性而為光變也。天戒若曰,兵猶火也,不戢將自焚。成都不悟,終以敗亡。



 懷帝永嘉元年,項縣有魏豫州刺史賈逵石碑,生金可採,此金不從革而為變也。五月,汲桑作亂,群寇飆起。



 清河王覃為世子時,所佩金鈴忽生起如粟者,康王母疑不祥,毀棄之。及後為惠帝太子,不終于位,卒為司馬越所殺。



 愍帝建興五年,石言于平陽。是時帝蒙塵亦在平陽,故有非言之物而言,妖之大者。俄而帝為逆胡所弒。



 元帝永昌元年,甘卓將襲王敦,既而中止。及還,家多變怪,照鏡不見其頭。此金失其性而為妖也。尋為敦所襲,遂夷滅。



 石季龍時,鄴城鳳陽門上金鳳皇二頭飛入漳河。



 海西太和中,會稽山陰縣起倉,鑿地得兩大船,滿中錢,錢皆輪文大形。時日向暮,鑿者馳以告官,官夜遣防守甚嚴。至明旦,失錢所在,惟有船存。視其狀,悉有錢處。



 安帝義熙初,東陽太守殷仲文照鏡不見其頭,尋亦誅翦,占與甘卓同也。



 《傳》曰:「簡宗廟,不禱祠,廢祭祀,逆天時,則水不潤下。」



 說曰:水,北方,終藏萬物者也。其於人道,命終而形藏,精神放越。聖人為之宗廟,以收魂氣,春秋祭祀,以終孝道。王者即位,必郊祀天地,禱祈神祇,望秩山川,懷柔百神,
 亡不宗事。慎其齋戒,致其嚴敬,是故鬼神歆饗,多獲福助。此聖王所以順事陰氣,和神人也。及至發號施令,亦奉天時。十二月咸得其氣,則陰陽調而終始成。如此,則水得其性矣。若乃不敬鬼神,政令逆時,水失其性。霧水暴出,百川逆溢,壞鄉邑,溺人民,及淫雨傷稼穡,是為水不潤下。



 京房《易傳》曰:「顓事者加,誅罰絕理,厥災水。其水也,雨,殺人,以隕霜,大風天黃。饑而不損,茲謂泰,厥大水,水殺人。避遏有德,茲謂狂,厥水,水流殺人也。已水則地生蟲。歸獄不解,茲謂追非,厥水寒,殺人。追誅不解,茲謂不理,厥
 水五穀不收。大敗不解,茲謂皆陰,厥水流入國邑,隕霜殺穀。」董仲舒曰:「交兵結仇,伏尸流血,百姓愁怨,陰氣盛,故大水也。」



 魏文帝黃初四年六月,大雨霖,伊洛溢,至津陽城門,漂數千家,殺人。初,帝即位,自鄴遷洛,營造宮室,而不起宗廟。太祖神主猶在鄴,嘗於建始殿饗祭如家人禮,終黃初不復還鄴。又郊社神祇,未有定位。此簡宗廟廢祭祀之罰也。



 吳孫權赤烏八年夏,茶陵縣鴻水溢出,漂二百餘家。十三年秋,丹陽、故鄣等縣又鴻水溢出。案權稱帝三十
 年,竟不於建鄴創七廟。惟父堅一廟遠在長沙,而郊祀禮闕。嘉禾初,群臣奏宜郊祀,又不許。末年雖一南郊,而北郊遂無聞焉。吳楚之望亦不見秩,反祀羅陽妖神,以求福助。天戒若曰,權簡宗廟,不禱祠,廢祭祀,故示此罰,欲其感悟也。



 太元元年,吳又有大風涌水之異。是冬,權南郊,宜是鑒咎徵乎!還而寢疾,明年四月薨。一曰,權時信納譖訴,雖陸遜勳重,子和儲貳,猶不得其終,與漢安帝聽讒免楊震、廢太子同事也。且赤烏中無年不用兵,百姓愁怨。八年秋,將軍馬茂等又圖逆。



 魏明帝景初元年九月,淫雨,冀、兗、徐、豫四州水出,沒溺殺人,漂失財產。帝自初即位,便淫奢極慾,多占幼女,或奪士妻,崇飾宮室,妨害農戰,觸情恣慾,至是彌甚,號令逆時,饑不損役。此水不潤下之應也。吳孫亮五鳳元年夏,大水。亮即位四年,乃立權廟。又終吳世不上祖宗之號,不修嚴父之禮,昭穆之數有闕。亮及休、皓又並廢二郊,不秩群神。此簡宗廟不祭祀之罰也。又,是時孫峻專政,陰勝陽之應乎!



 孫休永安四年五月,大雨,水泉涌溢。昔歲作浦里塘,功費無數,而田不可成,士卒死叛,或自賊殺,百姓愁怨,陰
 氣盛也。休又專任張布,退盛沖等,吳人賊之應也。五年八月壬午,大雨震電,水泉湧溢。



 武帝泰始四年九月,青、徐、兗、豫四州大水。七年六月,大雨霖,河、洛、伊、沁皆溢,殺二百餘人。自帝即尊位,不加三后祖宗之號。泰始二年又除明堂南郊五帝座,同稱昊天上帝,一位而已。又省先后配地之祀。此簡宗廟廢祭祀之罰也。



 咸寧元年九月,徐州大水。二年七月癸亥,河南、魏郡暴水,殺百餘人。閏月,荊州郡國五大水,流四千餘家。去年採擇良家子女,露面入殿,
 帝親簡閱,務在姿色,不訪德行,有蔽匿者以不敬論,搢紳愁怨,天下非之,陰盛之應也。



 三年六月,益、梁二州郡國八暴水,殺三百餘人。七月,荊州大水。九月,始平郡大水。十月,青、徐、兗、豫、荊、益、梁七州又大水。是時賈充等用事專恣,而正人疏外者多,陰氣盛也。



 四年七月,司、冀、兗、豫、荊、揚郡國二十大水,傷秋稼,壞屋室,有死者。



 太康二年六月,泰山、江夏大水,泰山流三百家,殺六十餘人,江夏亦殺人。時平吳後,王浚為元功而詆劾妄加,荀、賈為無謀而並蒙重賞,收吳姬五千,納之後宮,此其
 應也。



 四年七月,兗州大水。十二月,河南及荊、揚六州大水。五年九月,郡國四大水,又隕霜。是月,南安等五郡大水。六年四月,郡國十大水,壞廬舍。七年九月,郡國八大水。八月六月,郡國八大水。



 惠帝元康二年,有水災。五年五月,潁川、淮南大水。六月,城陽、東莞大水,殺人,荊、揚、徐、兗、豫五州又水。是時帝即位已五載,猶未郊祀,其蒸嘗亦多不親行事。此簡宗廟廢祭祀之罰。



 六年五月,刑、揚二州大水。是時賈后亂朝,寵樹賈、郭,女主專政,陰氣盛之應也。



 八年五月,金墉城井溢。《漢志》,成帝時有此妖,後王莽僭逆。今有此妖,趙王倫篡位,倫廢帝於此城,井溢所在,其天意也。九月,荊、揚、徐、冀、豫五州大水。是時賈后暴戾滋甚,韓謐驕猜彌扇,卒害太子,旋以禍滅。九年四月,宮中井水沸溢。



 永寧元年七月,南陽、東海大水。是時齊王冏專政,陰盛之應也。



 太安元年七月,兗、豫、徐、冀四州水。時將相力政,無尊主
 心,陰盛故也。



 孝懷帝永嘉四年四月,江東大水。時王導等潛懷翼戴之計,陰氣盛也。



 元帝太興三年六月,大水。是時王敦內懷不臣,傲很陵上,此陰氣盛也。四年七月,又大水。



 永昌二年五月,荊州及丹陽、宣城、吳興、壽春大水。



 明帝太寧元年五月,丹陽、宣城、吳興、壽春大水。是時王敦威權震主,陰氣盛故也。



 成帝咸和元年五月,大水。是時嗣主幼沖,母后稱制,庾
 亮以元舅決事禁中,陰勝陽故也。



 二年五月戊子,京都大水。是冬,以蘇峻稱兵,都邑塗地。



 四年七月,丹陽、宣城、吳興、會稽大水。是冬,郭默作亂,荊豫共討之,半歲乃定,兵役之應也。



 七年五月,大水。是時帝未親機務,政在大臣,陰勝陽也。



 咸康元年八月,長沙、武陵大水。



 穆帝永和四年五月,大水。五年五月,大水。六年五月,又大水。時幼主沖弱,母后臨朝,又將相大臣各執權政,與咸和初同事也。



 七年七月甲辰夜,濤水入石頭,死者數百人。是時殷浩以私忿廢蔡謨,遐邇非之。又幼主在上而殷桓交惡,選徒聚甲,各崇私權,陰勝陽之應也。一說,濤水入石頭,以為兵占。是後殷浩、桓溫、謝尚、荀羨連年征伐,百姓愁怨也。



 升平二年五月,大水。五年四月,又大水。是時桓溫權制朝廷,專征伐,陰勝陽也。



 海西太和六年六月,京師大水,平地數尺,浸及太廟。朱雀大航纜斷,三艘流入大江。丹陽、晉陵、吳郡、吳興、臨海
 五郡又大水,稻稼蕩沒,黎庶饑饉。初,四年桓溫北伐敗績,十喪其九,五年又征淮南,踰歲乃剋,百姓愁怨之應也。



 簡文帝咸安元年十二月壬午,濤水入石頭。明年,妖賊盧竦率其屬數百人入殿,略取武庫三庫甲仗,游擊將軍毛安之討滅之,兵興、陰盛之應也。



 孝武帝太元三年六月,大水。是時帝幼弱,政在將相。五年五月,大水。六年六月,揚、荊、江三州大水。八年三月,始興、南康、廬陵大水,平地五丈。
 十年五月,大水。自八年破苻堅後,有事中州,役無寧歲,愁怨之應也。



 十三年十二月,濤水入石頭,毀大航,殺人。明年,慕容氏寇擾司兗,鎮戍西北,疲於奔命,愁怨之應也。



 十五年七月,沔中諸郡及兗州大水。是時緣河紛爭,征戍勤瘁之應也。



 十七年六月甲寅,濤水入石頭,毀大航,漂船舫,有死者。京口西浦亦濤入殺人。永嘉郡潮水湧起,近海四縣人多死。後四年帝崩,而王恭再攻京師,京師亦發眾以禦之,兵彼頻興,百姓愁怨之應也。



 十八年六月己亥,始興、南康、廬陵大水,深五丈。十九年七月,荊徐大水,傷秋稼。二十年六月,荊徐又大水。二十一年五月癸卯,大水。是時政事多弊,兆庶非之。



 安帝隆安三年五月,荊州大水,平地三丈。去年殷仲堪舉兵向京師,是年春又殺郗恢,陰盛作威之應也。仲堪尋亦敗亡。



 五年五月,大水。是時會稽王世子元顯作威陵上,又桓玄擅西夏,孫恩亂東國,陰勝陽之應也。



 元興二年十二月,桓玄篡位。其明年二月庚寅夜,濤水
 入石頭。商旅方舟萬計,漂敗流斷,骸胔相望。江左雖頻有濤變,未有若斯之甚。三月,義軍剋京都,玄敗走,遂夷滅之。



 三年二月己丑朔夜,濤水入石頭,漂沒殺人,大航流敗。



 義熙元年十二月己未,濤水入石頭。二年十二月己未夜,濤水入石頭。明年,駱球父環潛結桓胤、殷仲文等謀作亂,劉稚亦謀反,凡所誅滅數十家。



 三年五月丙午,大水。四年十二月戊寅,濤水入石頭。明年,王旅北討。



 六年五月丁巳,大水。乙丑,盧循至蔡洲。



 八年六月,大水。九年五月辛巳,大水。十年五月丁丑,大水。戊寅,西明門地穿,涌水出,毀門扇及限,亦水沴土也。七月乙丑,淮北風災,大水殺人。十一年七月丙戌,大水,淹漬太廟,百官赴救。明年,王旅北討關河。


《經》曰:「敬用五事:一曰貌,二曰言,三曰視,四曰聽,五曰思。貌曰恭,言曰從,視曰明,聽曰聰,思曰睿。恭作肅,從作乂,明作哲,聰作謀,睿作聖。休徵:曰肅,時雨若;乂,時晹若;哲,時燠若;謀,時寒若;聖,時風若。咎徵:曰狂,恒雨若;僭,恒晹
 若;豫,恒燠若;急,恒寒若;
 \gezhu{
  雨瞀}
 ,恒風若。」



 《傳》曰:「貌之不恭,是謂不肅,厥咎狂,厥罰恒雨,厥極惡。時則有服妖,時則有龜孽,時則有雞禍,時則有下體生上之痾,時則有青眚青祥。惟金沴木。」



 說曰:凡草木之類謂之妖。妖猶夭胎,言尚微也。蟲豸之類謂之孽。孽則芽孽矣。及六畜,謂之禍,言其著也。及人,謂之痾。痾,病貌也,言浸深也。甚則有異物生,謂之眚;自外來,謂之祥。祥,猶禎也。氣相傷,謂之沴。沴猶臨蒞,不和意也。每一事云「時則」以絕之,言非必俱至,或有或亡,或在前或在後。孝武時,夏侯始昌通《五經》,善推《五行傳》,以
 傳族子夏侯勝,下及許商,皆以教所賢弟子。其傳與劉向同,惟劉歆傳獨異。



 貌之不恭,是謂不肅。肅,敬也。內曰恭,外曰敬。人君行己,體貌不恭,怠慢驕蹇,則不能敬萬事,失則狂易,故其咎狂也。上慢下暴,則陰氣勝,故其罰常雨也。水傷百穀,衣食不足,則姦宄並作,故其極惡也。



 一曰,人多被刑,或形貌醜惡,亦是也。風俗狂慢,變節易度,則為剽輕奇怪之服,故有服妖。水類動,故有龜孽。於《易》,《巽》為雞。雞有冠、距,文武之貌。而不為威,貌氣毀,故有雞禍。一曰,水歲多雞死及為怪,亦是也。上失威儀,則有
 彊臣害君上者,故有下體生於上之痾。木色青,故有青眚青祥。凡貌傷者病木氣,木氣病則金沴之,衝氣相通也。於《易》,《震》在東方,為春為木;《兌》在西方,為秋為金;《離》在南方,為夏為火;《坎》在北方,為冬為水。春與秋日夜分,寒暑平,是以金木之氣易以相變,故貌傷則致秋陰常雨,言傷則致春陽常旱也。至於冬夏,日夜相反,寒暑殊絕,水火之氣不得相并,故視傷常燠、聽傷常寒者,其氣然也。逆之,其極曰惡;順之,其福曰攸好德。劉歆《貌傳》曰有鱗蟲之孽,羊禍,鼻痾。說以為於天文東方辰為龍星,故為鱗蟲。於《易》,《兌》為羊,木為金所病,故致羊禍,與常雨同應。此
 說非是。春與秋氣陰陽相敵,木病金盛,故能相并,惟此一事耳。禍與妖痾祥眚同類,不得獨異。



 魏尚書鄧颺揚行步馳縱,筋不束體,坐起傾倚,若無手足,此貌之不恭也。管輅謂之鬼躁。鬼躁者,凶終之徵,後卒誅也。



 惠帝元康中,貴游子弟相與為散髮惈身之飲,對弄婢妾,逆之者傷好,非之者負譏,希世之士恥不與焉。蓋貌之不恭,胡狄侵中國之萌也。其後遂有五胡之亂,此又失在狂也。



 元康中,賈謐親貴,數入二宮,與儲君遊戲,無降下心。又
 嘗因弈棋爭道,成都王穎厲色曰:「皇太子國之儲貳,賈謐何敢無禮!」謐猶不悛,故及於禍,貌不恭之罰也。



 齊王冏既誅趙王倫,因留輔政,坐拜百官,符敕臺府,淫JT專驕,不一朝覲,此狂恣不肅之咎也。天下莫不高其功而慮其亡也,冏終弗改,遂致夷滅。



 司馬道子於府園內列肆,使姬人酤鬻,身自貿易。干寶以為貴者失位,降在皁隸之象也。俄而道子見廢,以庶人終,此貌不恭之應也。



 安帝義熙七年,將拜授劉毅世子,毅以王命之重,當設饗宴,親請吏佐臨視。至拜日,國僚不重白,默拜於廄中。
 王人將反命,毅方知之,大以為恨,免郎中令劉敬叔官。天戒若曰,此惰略嘉禮不肅之妖也。其後毅遂被殺焉。



 庶徵恒雨,劉歆以為《春秋》大雨,劉向以為大水。



 魏明帝太和元年秋,數大雨,多暴卒,雷電非常,至殺鳥雀。案楊阜上疏,此恒雨之罰也。時天子居喪不哀,出入弋獵無度,奢侈繁興,奪農時,故水失其性而恒雨為罰。



 太和四年八月,大雨霖三十餘日,伊、洛、河、漢皆溢,歲以凶饑。



 吳孫亮太平二年二月甲寅,大雨,震電。乙卯,雪,大寒。案劉歆說,此時當雨而不當大,大雨,恒雨之罰也。於始震
 電之,明日而雪,大寒,又常寒之罰也。劉向以為既已雷電,則雪不當復降,皆失時之異也。天戒若曰,為君失時,賊臣將起。先震電而後雪者,陰見間隙,起而勝陽,逆弒之禍將成也。亮不悟,尋見廢。此與《春秋》魯隱同。



 武帝泰始六年六月,大雨霖。甲辰,河、洛、伊、沁水同時並溢,流四千九百餘家,殺二百餘人,沒秋稼千三百六十餘頃。



 太康五年七月,任城、梁國暴雨,害豆麥。九月,南安郡霖雨暴雪,樹木摧折,害秋稼。是秋,魏郡西平郡九縣、淮南、平原霖雨暴水,霜傷秋稼。



 惠帝永寧元年十月、義陽、南陽、東海霖雨,淹害秋麥。



 元帝太興三年,春雨至于夏。是時王敦執權,不恭之罰也。



 永昌元年,春雨四十餘日,晝夜雷電震五十餘日。是時王敦興兵,王師敗績之應也。



 成帝咸和四年,春雨五十餘日,恒雷電。是時雖斬蘇峻,其餘黨猶據守石頭,至其滅後,淫雨乃霽。



 咸康元年八月乙丑,荊州之長沙攸、醴陵、武陵之龍陽,三縣雨水,浮漂屋室,殺人,損秋稼。是時帝幼,權在於下。



 服妖



 魏武帝以天下凶荒,資財乏匱,始擬古皮弁,裁縑帛為白帢,以易舊服。傅玄曰;「白乃軍容,非國容也。」干寶以為「縞素,凶喪之象也」。名之為帢,毀辱之言也,蓋革代之後,劫殺之妖也。



 魏明帝著繡帽,披縹紈半袖,常以見直臣楊阜,諫曰:「此禮何法服邪!」帝默然。近服妖也。夫縹,非禮之色。褻服尚不以紅紫,況接臣下乎?人主親御非法之章,所謂自作孽不可禳也。帝既不享永年,身沒而祿去王室,後嗣不終,遂亡天下。



 景初元年,發銅鑄為巨人二,號曰翁仲,置之司馬門外。
 案古長人見,為國亡。長狄見臨洮,為秦亡之禍。始皇不悟,反以為嘉祥,鑄銅人以象之。魏法亡國之器,而於義竟無取焉。蓋服妖也。



 尚書何晏好服婦人之服,傅玄曰:「此妖服也。夫衣裳之制,所以定上下殊內外也。《大雅》云『玄袞赤舄,鉤膺鏤錫』,歌其文也。《小雅》云『有嚴有翼,共武之服』,詠其武也。若內外不殊,王制失敘,服妖既作,身隨之亡。妹嬉冠男子之冠,桀亡天下;何晏服婦人之服,亦亡其家,其咎均也。」



 吳婦人修容者,急束其髮而劘角過于耳,蓋其俗自操束太急,而廉隅失中之謂也。故吳之風俗,相驅以急,言
 論彈射,以刻薄相尚。居三年之喪者,往往有致毀以死。諸葛患之,著《正交論》,雖不可以經訓整亂,蓋亦救時之作也。



 孫休後,衣服之制上長下短,又積領五六而裳居一二。干寶曰:「上饒奢,下儉逼,上有餘下不足之妖也。」至孫皓,果奢暴恣情於上,而百姓彫困於下,卒以亡國,是其應也。



 武帝泰始初,衣服上儉下豐,著衣者皆厭衣要,此君衰弱,臣放縱,下掩上之象也。至元康末,婦人出兩襠,加乎交領之上,此內出外也。為車乘者茍貴輕細,又數變易其
 形,皆以白篾為純,蓋古喪車之遺象也。夫乘者,君子之器。蓋君子立心無恒,事不崇實也。干寶以為晉之禍徵也。及惠帝踐阼,權制在於寵臣,下掩上之應也。至永嘉末,六宮才人流冗沒於戎狄,內出外之應也。及天下撓亂,宰輔方伯多負其任,又數改易不崇實之應也。



 泰始之後,中國相尚用胡床貊槃,及為羌煮貊炙,貴人富室,必畜其器,吉享嘉會,皆以為先。太康中,又以氈為絈頭及絡帶褲口。百姓相戲曰,中國必為胡所破。夫氈毳產於胡,而天下以為絈頭、帶身、褲口,胡既三制之矣,能無敗乎!至元康中,氐羌互反,永嘉後,劉、石遂篡中都,
 自後四夷迭據華土,是服妖之應也。



 初作屐者,婦人頭圓,男子頭方。圓者順之義,所以別男女也。至太康初,婦人屐乃頭方,與男無別。此賈后專妒之徵也。



 太康中,天下為《晉世寧》之舞,手接杯盤而反覆之,歌曰「晉世寧,舞杯盤」。識者曰:「夫樂生人心,所以觀事也。今接杯盤於手上而反覆之,至危之事也。杯盤者,酒食之器,而名曰《晉世寧》,言晉世之士茍偷於酒食之間,而知不及遠,晉世之寧猶杯盤之在手也。」



 惠帝元康中,婦人之飾有五兵佩,又以金銀玳瑁之屬,
 為斧鉞戈戟,以當笄。干寶以為「男女之別,國之大節,故服物異等,贄幣不同。今婦人而以兵器為飾,此婦人妖之甚者。於是遂有賈后之事」。終亡天下。是時婦人結髮者既成,以繒急束其環,名曰擷子紒。始自中宮,天下化之。其後賈后廢害太子之應也。



 元康中,天下始相傚為烏杖以柱掖,其後稍施其鐓,住則植之。夫木,東方之行,金之臣也。杖者扶體之器,烏其頭者,尤便用也。必旁柱掖者,旁救之象也。施其金,柱則植之,言木因於金,能孤立也。及懷愍之世,王室多故,而此中都喪敗,元帝以籓臣樹德東方,維持天下,柱掖之
 應也。至社稷無主,海內歸之,遂承天命,建都江外,獨立之應也。



 元康、太安之間,江淮之域有敗屩自聚于道,多者至四五十量,人或散投坑谷,明日視之復如故。或云,見狸銜聚之。干寶以為『夫屩者,人之賤服,處于勞辱,黔庶之象也。敗者,疲弊之象;道者,四方往來,所以交通王命也。今敗屩聚於道者,象黔庶罷病,將相聚為亂,以絕王命也」。太安中,發壬午兵,百姓怨叛。江夏張昌唱亂,荊楚從之如流。於是兵革歲起,服妖也。



 初,魏造白帢,橫縫其前以別後,名之曰顏帢,傳行之。至
 永嘉之間,稍去其縫,名無顏帢,而婦人束髮,其緩彌甚,紒之堅不能自立,髮被于額,目出而已。無顏者,愧之言也。覆額者,慚之貌也。其緩彌甚者,言天下亡禮與義,放縱情性,及其終極,至于大恥也。永嘉之後,二帝不反,天下媿焉。



 孝懷帝永嘉中,士大夫競服生箋單衣。識者指之曰:「此則古者繐衰,諸侯所以服天子也。今無故服之,殆有應乎!」其後遂有胡賊之亂,帝遇害焉。



 元帝太興中,兵士以絳囊縛紒。識者曰:「紒者在首,為乾,君道也。囊者坤,臣道也。今以朱囊縛紒,臣道上侵君之
 象也。」於是王敦陵上焉。



 舊為羽扇柄者,刻木象其骨形,列羽用十,取全數也。自中興初,王敦南征,始改為長柄,下出可捉,而減其羽用八。識者尤之曰:「夫羽扇,翼之名也。創為長柄者,將執其柄以制羽翼也。改十為八者,將未備奪已備也。此殆敦之擅權以制朝廷之柄,又將以無德之材欲竊非據也。」是時,為衣者又上短,帶纔至於掖,著帽者又以帶縛項。下逼上,上無地也。為褲者直幅為口,無殺,下大之象。尋而王敦謀逆,再攻京師。



 海西嗣位,忘設豹尾。天戒若曰,夫豹尾,儀服之主,大人所以豹變也。而海西豹變之日,非所宜忘而忘之。非主
 社稷之人,故忘其豹尾,示不終也。尋而被廢焉。



 孝武太元中,人不復著帩頭。天戒若曰,頭者元首,帩者助元首為儀飾者也。今忽廢之,若人君獨立無輔佐,以至危亡也。至安帝,桓玄乃篡位焉。



 舊為屐者,齒皆達楄上,名曰露卯。太元中忽不徹,名日陰卯。識者以為卯,謀也,必有陰謀之事。至烈宗末,驃騎參軍袁悅之始攬構內外,隆安中遂謀詐相傾,以致大亂。



 太元中,公主婦女必緩鬢傾髻,以為盛飾。用髲既多,不可恒戴,乃先於木及籠上裝之,名曰假髻,或名假頭。至
 於貧家,不能自辦,自號無頭,就人借頭。遂布天下,亦服妖也。無幾時,孝武晏駕而天下騷動,刑戮無數,多喪其元。至於大殮,皆刻木及蠟或縛菰草為頭,是假頭之應云。



 桓玄篡立,殿上施絳帳,鏤黃金為顏,四角金龍銜五色羽葆流蘇。群下相謂曰:「頗類轜車。」尋而玄敗,此服之妖也。



 晉末皆冠小而衣裳博大,風流相放,輿臺成俗。識者曰:「上小而下大,此禪代之象也。」尋而宋受終焉。



 雞禍



 魏明帝景初二年,廷尉府中雌雞化為雄,不鳴不將。干寶曰:「是歲宣帝平遼東,百姓始有與能之義,此其象也。然晉三后並以人臣終,不鳴不將,又天意也。」



 惠帝元康六年,陳國有雞生雄雞無翅,既大,墜坑而死。王隱以為:「雄者,胤嗣子之象。坑者,母象。今雞生無翅,墜坑而死,此子無羽翼,為母所陷害乎?」於後賈后誣殺愍懷,此其應也。



 太安中,周家雌雞逃承溜中,六七日而下,奮翼鳴將,獨毛羽不變。其後有陳敏之事。敏雖控制江表,終無紀綱文章,殆其象也。卒為所滅。雞禍見家,又天意也。
 京房《易傳》曰:「牝雞雄鳴,主不榮。」



 元帝太興中,王敦鎮武昌,有雌雞化為雄。天戒若曰,雌化為雄,臣陵其上。其後王敦再攻京師。



 孝武太元十三年四月,廣陵高平閻嵩家雌雞生無右翅,彭城人劉象之家雞有三足。京房《易傳》曰:「君用婦人言,則雞生妖。」是時,主相並用尼媼之言,寵賜過厚,故妖象見焉。



 安帝隆安元年八月,瑯邪王道子家青雌雞化為赤雄雞,不鳴不將。桓玄將篡,不能成業之象。



 四年,荊州有雞生角,角尋墮落。是時桓玄始擅西夏,狂
 慢不肅,故有雞禍。天戒若曰,角,兵象,尋墮落者,暫起不終之妖也。後皆應也。



 元興二年,衡陽有雌雞化為雄,八十日而冠萎。天戒若曰,衡陽,桓玄楚國之邦略也。及桓玄篡位,果八十日而敗,此其應也。



 青祥



 武帝咸寧元年八月丁酉,大風折大社樹,有青氣出焉,此青祥也。占曰:「東莞當有帝者。」明年,元帝生。是時,帝大父武王封東莞,由是徙封琅邪。孫盛以為中興之表。晉室之亂,武帝子孫無孑遺,社樹折之應,又常風之罰。



 惠帝元康中,洛陽南山有虻作聲,曰「韓尸尸」。識者曰:「韓氏將尸也,言尸尸者,盡死意也。」其後韓謐誅而韓族殲焉,此青祥也。



 金沴木



 魏文帝黃初七年正月,幸許昌。許昌城南門無故自崩,帝心惡之,遂不入,還洛陽。此金沴木,木動之也。五月,宮車晏駕。京房《易傳》曰:「上下咸悖,厥妖也城門壞。」



 元帝太興二年六月,吳郡米廡無故自壞。天戒若曰,夫米廡,貨糴之屋,無故自壞,此五穀踴貴,所以無糴賣也。是歲遂大饑,死者千數焉。



 明帝太寧元年,周莚自歸王敦,既立其宅宇,所起五間六梁,一時躍出墜地,餘桁猶亙柱頭。此金沴木也。明年五月,錢鳳謀亂,遂族滅莚,而湖熟尋亦為墟矣。



 安帝元興元年正月丙子,會稽王世子元顯將討桓玄,建牙竿於揚州南門,其東者難立,良久乃正。近沴妖也。而元顯尋為玄所擒。



 三年五月,樂賢堂壞。時帝嚚眊,無樂賢之心,故此堂是沴。



 義熙九年五月,國子聖堂壞。天戒若曰,聖堂,禮樂之本,無故自壞,業祚將墜之象。未及十年而禪位焉。



\end{pinyinscope}