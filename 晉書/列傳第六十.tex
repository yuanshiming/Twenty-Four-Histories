\article{列傳第六十}

\begin{pinyinscope}

 良吏



 魯芝胡威杜軫竇允王宏曹攄潘京範晷丁紹喬智明鄧攸吳隱之



 漢宣帝有言:「百姓所以安其田里而無歎息愁恨之心者,政平訟理也。與我共此者,其唯良二千石乎!」此則長吏之官,實為撫導之本。是以東里相鄭,西門宰鄴,潁川黃霸,蜀郡文翁,或吏不敢欺,或人懷其惠,或教移齊魯,或政務寬和,斯並惇史播其徽音,良能以為準的。



 有晉肇茲王業,光啟霸圖,授方任能,經文緯武。泰始受禪,改
 物君臨,纂三葉之鴻基,膺百王之大寶,勞心庶績,垂意黎元,申敕守宰之司,婁發憂矜之詔,辭旨懇切,誨諭殷勤,欲使直道正身,抑末敦本。當此時也,可謂農安其業,吏盡其能者歟!而帝寬厚足以君人,明威未能厲俗,政刑以之私謁,賄賂於此公行,結綬者以放濁為通,彈冠者以茍得為貴,流遁忘反,浸以為常。劉毅抗賣官之言,當時以為矯枉,察其風俗,豈虛也哉!爰及惠懷,中州鼎沸,逮於江左,晉政多門,元帝比少康之隆,處仲為梗,海西微昌邑之罪,元子亂常,既權偪是憂,故羈縻成俗。蒞職者為身擇利,銓綜者為人擇官,下僚多英俊之才,勢
 位必高門之胄,遂使良能之績僅有存焉。雖復茂弘以明允贊經綸,安石以時宗鎮雅俗,然外虞孔熾,內難方殷,而匡救彌縫,方免傾覆,弘風革弊,彼則未遑。今采其政績可稱者,以為《良吏傳》。



 魯芝,字世英,扶風郿人也。世有名德,為西州豪族。父為郭氾所害,芝襁褓流離,年十七,乃移居雍,耽思墳籍。郡舉上計吏,州辟別駕。魏車騎將軍郭淮為雍州刺史,深敬重之。舉孝廉,除郎中。會蜀相諸葛亮侵隴右,淮復請芝為別駕。事平,薦於公府,辟大司馬曹真掾,轉臨淄侯
 文學。鄭袤薦於司空王朗,朗即加禮命。後拜騎都尉、參軍事、行安南太守,遷尚書郎。曹真出督關右,又參大司馬軍事。真薨,宣帝代焉,乃引芝參驃騎軍事,轉天水太守。郡鄰于蜀,數被侵掠,戶口減削,寇盜充斥,芝傾心鎮衛,更造城市,數年間舊境悉復。遷廣平太守。天水夷夏慕德,老幼赴闕獻書,乞留芝。魏明帝許焉,仍策書嘉歎,勉以黃霸之美,加討寇將軍。



 曹爽輔政,引為司馬。芝屢有讜言嘉謀,爽弗能納。及宣帝起兵誅爽,芝率餘眾犯門斬關,馳出赴爽,勸爽曰:「公居伊周之位,一旦以罪見黜,雖欲牽黃犬,復可得乎!若挾天子保許昌,杖大威以
 羽檄徵四方兵,孰敢不從!捨此而去,欲就東市,豈不痛哉!」爽懦惑不能用,遂委身受戮。芝坐爽下獄,當死,而口不訟直,志不茍免。宣帝嘉之,赦而不誅。俄而起為使持節、領護匈奴中郎將、振威將軍、並州刺史。以綏緝有方,遷大鴻臚。



 高貴鄉公即位,賜爵關內侯,邑二百戶。毌丘儉平,隨例增邑二百戶,拜揚武將軍、邢州刺史。諸葛誕以壽春叛,文帝奉魏帝出征,徵兵四方,芝率荊州文武以為先驅。誕平,進爵武進亭侯,又增邑九百戶。遷大尚書,掌刑理。常道鄉公即位,進爵斄城鄉侯,又增邑八百戶,遷監青州諸軍事、振武將軍、青州刺史,轉平東將軍。
 五等建,封陰平伯。



 武帝踐阼,轉鎮東將軍,進爵為侯。帝以芝清忠履正,素無居宅,使軍兵為作屋五十間。芝以年及懸車,告老遜位,章表十餘上,於是徵為光祿大夫,位特進,給吏卒,門施行馬。羊祜為車騎將軍,乃以位讓芝,曰:「光祿大夫魯芝潔身寡欲,和而不同,服事華髮,以禮終始,未蒙此選,臣更越之,何以塞天下之望!」上不從。其為人所重如是。泰始九年卒,年八十四。帝為舉哀,賵贈有加,謚曰貞,賜塋田百畝。



 胡威,字伯武,一名貔。淮南壽春人也。父質,以忠清著稱,
 少與鄉人蔣濟、朱績俱知名於江淮間,仕魏至征東將軍、荊州刺史。威早厲志尚。質之為荊州也,威自京都定省,家貧,無車馬僮僕,自驅驢單行。每至客舍,躬放驢,取樵炊爨,食畢,復隨侶進道。既至,見父,停廄中十餘日。告歸,父賜絹一匹為裝。威曰:「大人清高,不審於何得此絹?」質曰:「是吾俸祿之餘,以為汝糧耳。」威受之,辭歸。質帳下都督先威未發,請假還家,陰資裝於百餘里,要威為伴,每事佐助。行數百里,威疑而誘問之,既知,乃取所賜絹與都督,謝而遣之。後因他信以白質,質杖都督一百,除吏名。其父子清慎如此。於是名譽著聞。拜侍御史,歷南
 鄉侯、安豐太守,遷徐州刺史。勤於政術,風化大行。



 後入朝,武帝語及平生,因歎其父清,謂威曰:「卿孰與父清?」對曰:「臣不如也。」帝曰:「卿父以何勝耶?」對曰:「臣父清恐人知,臣清恐人不知,是臣不及遠也。」帝以威言直而婉,謙而順。累遷監豫州諸軍事、右將軍、豫州刺史,入為尚書,加奉車都尉。



 威嘗諫時政之寬,帝曰:「尚書郎以下,吾無所假借。」威曰:「臣之所陳,豈在丞郎令史,正謂如臣等輩,始可以肅化明法耳。」拜前將軍、監青州諸軍事、青州刺史,以功封平春侯。太康元年,卒于位,追贈使持節、都督青州諸軍事、鎮東將軍,餘如故,謚曰烈。子奕嗣。



 奕字次
 孫,仕至平東將軍。威弟羆,字季象,亦有乾用,仕至益州刺史、安東將軍。



 杜軫,字超宗,蜀郡成都人也。父雄,綿竹令。軫師事譙周,博涉經書。州辟不就,為郡功曹史。時鄧艾至成都,軫白太守曰:「今大軍來征,必除舊布新,明府宜避之,此全福之道也。」太守乃出。艾果遣其參軍牽弘自之郡,弘問軫前守所在,軫正色對曰:「前守達去就之機,輒自出官舍以俟君子。」弘器之,命復為功曹,軫固辭。察孝廉,除建寧令,導以德政,風化大行,夷夏悅服。秩滿將歸,群蠻追送,
 賂遺甚多,軫一無所受,去如初至。又除池陽令,為雍州十一郡最。百姓生為立祠,得罪者無怨言。累遷尚書郎。軫博聞廣涉,奏議駁論多見施用。時涪人李驤亦為尚書郎,與軫齊名,每有論議,朝廷莫能踰之,號蜀有二郎。軫後拜犍為太守,甚有聲譽。當遷,會病卒,年五十一。子毗。



 毗字長基。州舉秀才,成都王穎辟大將軍掾,遷尚書郎,參太傅軍事。及洛陽覆沒,毗南渡江,王敦表為益州刺史,將與宜都太守柳純共固白帝。杜弢遣軍要毗,遂遇害。



 毗弟秀,字彥穎,為羅尚主簿。州沒,為氏賊李驤所得,欲用為司馬。秀不受,見害。毗次子歆,舉秀才。



 軫弟烈,
 明政事,察孝廉,歷平康、安陽令,所居有異績,遷衡陽太守。聞軫亡,因自表兄子幼弱,求去官,詔轉犍為太守,蜀土榮之。後遷湘東太守,為成都王穎郎中令,病卒。



 烈弟良,舉秀才,除新都令、涪陵太守,不就,補州大中正,卒。



 竇允,字雅,始平人也。出自寒門,清尚自修。少仕縣,稍遷郡主簿。察孝廉,除浩亹長。勤於為政,勸課田蠶,平均調役,百姓賴之。遷謁者。泰始中,詔曰:「當官者能潔身修己,然後在公之節乃全。身善有章,雖賤必賞,此興化立教之務也。謁者竇允前為浩亹長,以修勤清白見稱河右。
 是輩當擢用,使立行者有所勸。主者詳復參訪,有以旌表之。」拜臨水令。克己厲俗,改修政事,士庶悅服,咸歌詠之。遷鉅鹿太守,甚有政績。卒於官。



 王宏,字正宗,高平人,魏侍中粲之從孫也。魏時辟公府,累遷尚書郎,歷給事中。泰始初,為汲郡太守,撫百姓如家,耕桑樹藝,屋宇阡陌,莫不躬自教示,曲盡事宜,在郡有殊績。司隸校尉石鑒上其政術,武帝下詔稱之曰:「朕惟人食之急,而懼天時水旱之運,夙夜警戒,念在於農。雖詔書屢下,敕厲殷勤,猶恐百姓廢惰以損生植之功。
 而刺史、二千石、百里長吏未能盡勤,至使地有遺利而人有餘力,每思聞監司糾舉能不,將行其賞罰,以明沮勸。今司隸校尉石鑒上汲郡太守王宏勤恤百姓,導化有方,督勸開荒五千餘頃,而熟田常課頃畝不減。比年普饑,人食不足,而宏郡界獨無匱乏,可謂能矣。其賜宏穀千斛,布告天下,咸使聞知。」



 俄遷衛尉、河南尹、大司農,無復能名,更為苛碎。坐桎梏罪人,以泥墨塗面,置深坑中,餓不與食,又擅縱五歲刑以下二十一人,為有司所劾。帝以宏累有政績,聽以贖罪論。太康中,代劉毅為司隸校尉,於是檢察士庶,使車服異制,庶人不得衣紫絳
 及綺繡錦繢。帝常遣左右微行,觀察風俗,宏緣此復遣吏科檢婦人衵服,至褰發於路。論者以為暮年謬妄,由是獲譏於世,復坐免官。後起為尚書。太康五年卒,追贈太常。



 曹攄,字顏遠,譙國譙人也。祖肇,魏衛將軍。攄少有孝行,好學善屬文,太尉王衍見而器之,調補臨淄令。縣有寡婦,養姑甚謹。姑以其年少,勸令改適,婦守節不移。姑愍之,密自殺。親黨告婦殺姑,官為考鞫,寡婦不勝苦楚,乃自誣。獄當決,適值攄到。攄知其有冤,更加辯究,具得情
 實,時稱其明。獄有死囚,歲夕,攄行獄,愍之,曰:「卿等不幸致此非所,如何?新歲人情所重,豈不欲暫見家邪?」眾囚皆涕泣曰:「若得暫歸,死無恨也。」攄悉開獄出之,剋日令還。掾吏固爭,咸謂不可。攄曰:「此雖小人,義不見負,自為諸君任之。」至日,相率而還,並無違者,一縣歎服,號曰聖君。入為尚書郎,轉洛陽令,仁惠明斷,百姓懷之。時天大雨雪,宮門夜失行馬,群官檢察,莫知所在。攄使收門士,眾官咸謂不然。攄曰:「宮掖禁嚴,非外人所敢盜,必是門士以燎寒耳。」詰之,果服。以病去官。復為洛陽令。



 及齊王冏輔政,攄與左思俱為記室督。冏嘗從容問攄曰:「天子
 為賊臣所逼,莫有能奮。吾率四海義兵興復王室,今入輔朝廷,匡振時艱,或有勸吾還國,於卿意如何?」攄曰:「蕩平國賊,匡復帝祚,古今人臣之功未有如大王之盛也。然道罔隆而不殺,物無盛而不衰,非唯人事,抑亦天理。竊預下問,敢不盡情。願大王居高慮危,在盈思沖,精選百官,存公屏欲,舉賢進善,務得其才,然後脂車秣馬,高揖歸籓,則上下同慶,攄等幸甚。」冏不納。尋轉中書侍郎。長沙王乂以為驃騎司馬。乂敗,免官。因丁母憂。惠帝末,起為襄城太守。



 永嘉二年,高密王簡鎮襄陽,以攄為征南司馬。其年
 流人王逌等聚眾屯冠軍,寇掠城邑。簡遣參軍崔曠討之,令攄督護曠。曠,奸凶人也,譎攄前戰,期為後繼,既而不至。攄獨與逌戰於酈縣,軍敗死之。故吏及百姓並奔喪會葬,號哭即路,如赴父母焉。



 潘京,字世長,武陵漢壽人也。弱冠,郡辟主簿,太守趙廞甚器之,嘗問曰:「貴郡何以名武陵?」京曰:「鄙郡本名義陵,在辰陽縣界,與夷相接,數為所攻,光武時移東出,遂得全完,共議易號。《傳》曰止戈為武,《詩》稱高平曰陵,於是名焉。」為州所辟,因謁見問策,探得「不孝」字,刺史戲京曰:「辟
 士為不孝邪?」京舉版答曰:「今為忠臣,不得復為孝子。」其機辯皆此類。後太廟立,州郡皆遣使賀,京白太守曰:「夫太廟立,移神主,應問訊,不應賀。」遂遣京作文,使詣京師,以為永式。京仍舉秀才,到洛。尚書令樂廣,京州人也,共談累日,深歎其才,謂京曰:「君天才過人,恨不學耳。若學,必為一代談宗。」京感其言,遂勤學不倦。時武陵太守戴昌亦善談論,與京共談,京假借之,昌以為不如己,笑而遣之,令過其子若思,京方極其言論。昌竊聽之,乃歎服曰:「才不可假。」遂父子俱屈焉。歷巴丘、邵陵、泉陵三令。京明於政術,路不拾遺。遷桂林太守,不就,歸家,年五十卒。



 范晷,字彥長,南陽順陽人也。少遊學清河,遂徙家僑居。郡命為五官掾,歷河內郡丞。太守裴楷雅知之,薦為侍御史。調補上谷太守,遭喪,不之官。後為司徒左長史,轉馮翊太守,甚有政能,善於綏撫,百姓愛悅之。徵拜少府,出為涼州刺史,轉雍州。於時西土荒毀,氏羌蹈藉,田桑失收,百姓困弊,晷傾心化導,勸以農桑,所部甚賴之。元康中,加左將軍,卒於官。二子:廣、稚。



 廣字仲將。舉孝廉,除靈壽令,不之官。姊適孫氏,早亡,有孫名邁,廣負以南奔,雖盜賊艱急,終不棄之。元帝承制,以為堂邑令。丞劉榮
 坐事當死,郡劾以付縣。榮即縣人,家有老母,至節,廣輒聽暫還,榮亦如期而反。縣堂為野火所及,榮脫械救火,事畢,還自著械。後大旱,米貴,廣散私穀振飢人,至數千斛,遠近流寓歸投之,戶口十倍。卒於官。



 稚少知名,辟大將軍掾,早卒。子汪,別有傳。



 丁紹,字叔倫,譙國人也。少開朗公正,早歷清官,為廣平太守,政平訟理,道化大行。于時河北騷擾,靡有完邑,而廣平一郡四境乂安,是以皆悅其法而從其令。及臨漳被圍,南陽王模窘急,紹率郡兵赴之,模賴以獲全。模感
 紹恩,生為立碑。遷徐州刺史,士庶戀慕,攀附如歸。未之官,復轉荊州刺史。從車千乘,南渡河至許。時南陽王模為都督,留紹,啟轉為冀州刺史。到鎮,率州兵討破汲桑有功,加寧北將軍、假節、監冀州諸軍事。時境內羯賊為患,紹捕而誅之,號為嚴肅,河北人畏而愛之。紹自以為才足為物雄,當官蒞政,每事剋舉,視天下之事若運於掌握,遂慨然有董正四海之志矣。是時王浚盛於幽州,茍晞盛於青州,然紹視二人蔑如也。永嘉三年,暴疾而卒,臨終歎曰:「此乃天亡冀州,豈吾命哉!」懷帝策贈車騎將軍。



 喬智明,字元達,鮮卑前部人也。少喪二親,哀毀過禮,長而以德行著稱。成都王穎辟為輔國將軍。穎之敗趙王倫也,表智明為殄寇將軍、隆慮、共二縣令。二縣愛之,號為「神君」。部人張兌為父報仇,母老單身,有妻無子,智明愍之,停其獄。歲餘,令兌將妻入獄,兼陰縱之。人有勸兌逃者,兌曰:「有君如此,吾何忍累之!縱吾得免,作何面目視息世間!」於獄產一男。會赦,得免。其仁感如是。惠帝之伐鄴也,穎以智明為折衝將軍、參丞相前鋒軍事。智明勸穎奉迎乘輿,穎大怒曰:「卿名曉事,投身事孤。主上為
 群小所逼,將加非罪於孤,卿奈何欲使孤束手就刑邪!共事之義,正若此乎?」智明乃止。尋屬永嘉之亂,仕於劉曜。



 鄧攸,字伯道,平陽襄陵人也。祖殷,亮直彊正。鐘會伐蜀,奇其才,自黽池令召為主簿。賈充伐吳,請殷為長史。後授皇太子《詩》,為淮南太守。夢行水邊,見一女子,猛獸自後斷其盤囊。占者以為水邊有女,汝字也,斷盤囊者,新獸頭代故獸頭也,不作汝陰,當汝南也。果遷汝陰太守。後為中庶子。



 攸七歲喪父,尋喪母及祖母,居喪九年,以
 孝致稱。清和平簡,貞正寡欲。少孤,與弟同居。初,祖父殷有賜官,敕攸受之。後太守勸攸去王官,欲舉為孝廉,攸曰:「先人所賜,不可改也。」嘗詣鎮軍賈混,混以人訟事示攸,使決之。攸不視,曰:「孔子稱聽訟吾猶人也,必也使無訟乎!」混奇之,以女妻焉。舉灼然二品,為吳王文學,歷太子洗馬、東海王越參軍。越欽其為人,轉為世子文學、吏部郎。越弟騰為東中郎將,請攸為長史。出為河東太守。



 永嘉末,沒于石勒。然勒宿忌諸官長二千石,聞攸在營,馳召,將殺之。攸至門,門幹乃攸為郎時幹,識攸,攸求紙筆作辭。幹候勒和悅,致之。勒重其辭,乃勿殺。勒長史
 張賓先與攸比舍,重攸名操,因稱攸於勒。勒召至幕下,與語,悅之,以為參軍,給車馬。勒每東西,置攸車營中。勒夜禁火,犯之者死。攸與胡鄰轂,胡夜失火燒車。吏按問,胡乃誣攸。攸度不可與爭,遂對以弟婦散發溫酒為辭。勒赦之。既而胡人深感,自縛詣勒以明攸,而陰遺攸馬驢,諸胡莫不歎息宗敬之。石勒過泗水,攸乃斫壞車,以牛馬負妻子而逃。又遇賊,掠其牛馬,步走,擔其兒及其弟子綏。度不能兩全,乃謂其妻曰:「吾弟早亡,唯有一息,理不可絕,止應自棄我兒耳。幸而得存,我後當有子。」妻泣而從之,乃棄之。其子朝棄而暮及。明日,攸繫之於樹
 而去。



 至新鄭,投李矩。三年,將去,而矩不聽。荀組以為陳郡、汝南太守,愍帝徵為尚書左丞、長水校尉,皆不果就。後密捨矩去,投荀組於許昌,矩深恨焉,久之,乃送家屬還攸。攸與刁協、周顗素厚,遂至江東。元帝以攸為太子中庶子。時吳郡闕守,人多欲之,帝以授攸。攸載米之郡,俸祿無所受,唯飲吳水而已。時郡中大饑,攸表振貸,未報,乃輒開倉救之。臺遣散騎常侍桓彞、虞斐慰勞饑人,觀聽善不,乃劾攸以擅出穀。俄而有詔原之。攸在郡刑政清明,百姓歡悅,為中興良守。後稱疾去職。郡常有送迎錢數百萬,攸去郡,不受一錢。百姓數千人留牽攸船,
 不得進,攸乃小停,夜中發去。吳人歌之曰:「紞如打五鼓,雞鳴天欲曙。鄧侯挽不留,謝令推不去。」百姓詣臺乞留一歲,不聽。拜侍中。歲餘,轉吏部尚書。蔬食弊衣,周急振乏。性謙和,善與人交,賓無貴賤,待之若一,而頗敬媚權貴。



 永昌中,代周顗為護軍將軍。太寧二年,王敦反,明帝密謀起兵,乃遷攸為會稽太守。初,王敦伐都之後,中外兵數每月言之於敦。攸已出在家,不復知護軍事,有惡攸者,誣攸尚白敦兵數。帝聞而未之信,轉攸為太常。時帝南郊,攸病不能從。車駕過攸問疾,攸力病出拜。有司奏攸不堪行郊而拜道左,坐免。攸每有進退,無喜慍之
 色。久之,遷尚書右僕射。咸和元年卒,贈光祿大夫,加金章紫綬,祠以少年。



 攸棄子之後,妻子不復孕。過江,納妾,甚寵之,訊其家屬,說是北人遭亂,憶父母姓名,乃攸之甥。攸素有德行,聞之感恨,遂不復畜妾,卒以無嗣。時人義而哀之,為之語曰:「天道無知,使鄧伯道無兒。」弟子綏服攸喪三年。



 吳隱之,字處默,濮陽鄄城人,魏侍中質六世孫也。隱之美姿容,善談論,博涉文史,以儒雅標名。弱冠而介立,有清操,雖日晏歠菽,不饗非其粟,儋石無儲,不取非其道。
 年十餘,丁父憂,每號泣,行人為之流涕。事母孝謹,及其執喪,哀毀過禮。家貧,無人鳴鼓,每至哭臨之時,恒有雙鶴警叫,及祥練之夕,復有群雁俱集,時人咸以為孝感所至。嘗食咸菹,以其味旨,掇而棄之。



 與太常韓康伯鄰居,康伯母,殷浩之姊,賢明婦人也,每聞隱之哭聲,輟餐投箸,為之悲泣。既而謂康伯曰:「汝若居銓衡,當舉如此輩人。」及康伯為吏部尚書,隱之遂階清級,解褐輔國功曹,轉參征虜軍事。兄坦之為袁真功曹,真敗,將及禍,隱之詣桓溫,乞代兄命,溫矜而釋之。遂為溫所知賞,拜奉朝請、尚書郎,累遷晉陵太守。在郡清儉,妻自負薪。入為
 中書侍郎、國子博士、太子右衛率,轉散騎常侍,領著作郎。孝武帝欲用為黃門郎,以隱之貌類簡文帝,乃止。尋守廷尉、祕書監、御史中丞,領著作如故,遷左衛將軍。雖居清顯,祿賜皆班親族,冬月無被,嘗浣衣,乃披絮,勤苦同於貧庶。



 廣州包帶山海,珍異所出,一篋之寶,可資數世,然多瘴疫,人情憚焉。唯貧窶不能自立者,求補長史,故前後刺史皆多黷貨。朝廷欲革嶺南之弊,隆安中,以隱之為龍驤將軍、廣州刺史、假節,領平越中郎將。未至州二十里,地名石門,有水曰貪泉,飲者懷無厭之欲。隱之既至,語其親人曰:「不見可欲,使心不亂。越嶺喪清,吾
 知之矣。」乃至泉所,酌而飲之,因賦詩曰:「古人云此水,一歃懷千金。試使夷齊飲,終當不易心。」及在州,清操踰厲,常食不過菜及乾魚而已,帷帳器服皆付外庫,時人頗謂其矯,然亦終始不易。帳下人進魚,每剔去骨存肉,隱之覺其用意,罰而黜焉。元興初,詔曰:「夫孝行篤於閨門,清節厲乎風霜,實立人之所難,而君子之美致也。龍驤將軍、廣州刺史吳隱之孝友過人,祿均九族,菲己潔素,儉愈魚飧。夫處可欲之地,而能不改其操,饗惟錯之富,而家人不易其服,革奢務嗇,南域改觀,朕有嘉焉。可進號前將軍,賜錢五十萬、穀千斛。」



 及盧循寇南海,隱之率
 厲將士,固守彌時,長子曠之戰沒。循攻擊百有餘日,踰城放火,焚燒三千餘家,死者萬餘人,城遂陷。隱之攜家累出,欲奔還都,為循所得。循表朝廷,以隱之黨附桓玄,宜加裁戮,詔不許。劉裕與循書,令遣隱之還,久方得反。歸舟之日,裝無餘資。及至,數畝小宅,籬垣仄陋,內外茅屋六間,不容妻子。劉裕賜車牛,更為起宅,固辭。尋拜度支尚書、太常,以竹篷為屏風,坐無氈席。後遷中領軍,清儉不革,每月初得祿,裁留身糧,其餘悉分振親族,家人績紡以供朝夕。時有困絕,或並日而食,身恒布衣不完,妻子不霑寸祿。



 義熙八年,請老致事,優詔許之,授光祿
 大夫,加金章紫綬,賜錢十萬、米三百斛。九年,卒,追贈左光祿大夫,加散騎常侍。隱之清操不渝,屢被褒飾,致事及於身沒,常蒙優錫顯贈,廉士以為榮。



 初,隱之為奉朝請,謝石請為衛將軍主簿。隱之將嫁女,石知其貧素,遣女必當率薄,乃令移廚帳助其經營。使者至,方見婢牽犬賣之,此外蕭然無辦。後至自番禺,其妻劉氏齎沈香一斤,隱之見之,遂投於湖亭之水。



 子延之復厲清操,為鄱陽太守。延之弟及子為郡縣者,常以廉慎為門法,雖才學不逮隱之,而孝悌潔敬猶為不替。



 史臣曰:魯芝等建旟剖竹,布政宣條,存樹威恩,沒留遺
 愛,咸見知明主,流譽當年。若伯武之潔己克勤,顏遠之申冤緩獄,鄧攸贏糧以述職,吳隱酌水以厲精,晉代良能,此焉為最。而攸棄子存侄,以義斷恩,若力所不能,自可割情忍痛,何至預加徽纆,絕其奔走者乎!斯豈慈父仁人之所用心也?卒以絕嗣,宜哉!勿謂天道無知,此乃有知矣。世英盡節曹氏,犯門斬關,宣帝收雷霆之威,獎忠貞之烈,豈非既已在我,欲其罵人者歟!



 贊曰:猗歟良宰,嗣美前賢。威同御黠,靜若烹鮮。唯嘗吳水,但挹貪泉。人風既偃,俗化斯遷。



\end{pinyinscope}