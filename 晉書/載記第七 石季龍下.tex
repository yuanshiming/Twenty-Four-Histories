\article{載記第七 石季龍下}

\begin{pinyinscope}

 石季龍
 下



 永和三年,季龍親耕藉田于其桑梓苑,其妻杜氏祠先蠶于近郊,遂如襄國謁勒墓。



 以中書監石寧為征西將軍,率并、司州兵二餘人為麻秋等後繼。張重華將宋秦等率戶二萬來降。河湟間氐羌十餘萬落與張璩相首尾,麻秋憚之,不進。重華金城太守張沖又以郡降石寧。麻秋尋次曲柳,劉寧、王擢進攻晉興武街。重華將楊
 康等與寧戰于沙阜,寧敗績,乃引還金城。王擢剋武街,執重華護軍曹權、胡宣,徙七千餘戶于雍州。季龍又以孫伏都為征西將軍,與麻秋率步騎三萬長驅濟河,且城長最。重華大懼,遣將謝艾逆擊,敗之,秋退歸金城。



 勒及季龍並貪而無禮,既王有十州之地,金帛珠玉及外國珍奇異貨不可勝紀,而猶以為不足,曩代帝王及先賢陵墓靡不發掘,而取其寶貨焉。邯鄲城西石子岡上有趙簡子墓,至是季龍令發之,初得炭深丈餘,次得木板厚一尺,積板厚八尺,乃及泉,其水清冷非常,作絞車以牛皮囊汲之,月餘而水不盡,不可發而止。又使掘秦
 始皇冢,取銅柱鑄以為器。



 時沙門吳進言于季龍曰:「胡運將衰,晉當復興,宜若役晉人以厭其氣。」季龍於是使尚書張群發近郡男女十六萬,車十萬乘,運土築華林苑及長牆於鄴北,廣長數十里。趙攬、申鐘、石璞等上疏陳天文錯亂,蒼生凋弊,及因引見,又面諫,辭旨甚切。季龍大怒曰:「牆朝戌夕沒,吾無恨矣。」乃促張群以燭夜作。起三觀、四門,三門通漳水,皆為鐵扉。暴風大雨,死者數萬人。揚州送黃鵠雛五,頸長一丈,聲聞十餘里,泛之于玄武池。郡國前後送蒼麟十六,白鹿七,季龍命司虞張曷柱調之,以駕芝蓋,列于充庭之乘。鑿北城,引水於華
 林園。城崩,壓死者百餘人。



 命石宣祈于山川,因而游獵,乘大輅,羽葆、華蓋,建天子旌旗,十有六軍,戎卒十八萬,出自金明門。季龍從其後宮升陵霄觀望之,笑曰:「我家父子如是,自非天崩地陷,當復何愁,但抱子弄孫日為樂耳!」宣既馳逐無厭,所在陳列行宮,四面各以百里為度,驅圍禽獸,皆幕集其所。文武跪立,圍守重行,烽炬星羅,光燭如晝,命勁騎百餘馳射其中。宣與嬖姬顯德美人乘輦觀之,嬉娛忘反,獸殫乃止。其有禽獸奔逸,當之者坐,有爵者奪馬步驅一日,無爵者鞭之一百。峻制嚴刑,文武戰慄,士卒飢凍而死者萬有餘人。宣弓馬衣食
 皆號為御,有亂其間者,以冒禁罪罪之。所過三州十五郡,資儲靡有孑遺。季龍復命石韜亦如之,出自并州,游于秦、晉。宣素惡韜寵,是行也,嫉之彌甚。宦者趙生得幸於宣而無寵于韜,微勸宣除之,於是相圖之計起矣。



 麻秋又襲張重華將張瑁於河、陜,敗之,斬首三千餘級。枹罕護軍李逵率眾七千降于季龍。自河已南,氐、羌皆降。



 石韜起堂于太尉府,號曰宣光殿,梁長九丈。宣視而大怒,斬匠,截梁而去。韜怒,增之十丈。宣聞之,恚甚,謂所幸楊柸、牟成曰:「韜凶豎勃逆,敢違我如是!汝能殺之者,吾入西宮,當盡以韜之國邑分封汝等。韜既死,主上必親
 臨喪,因行大事,蔑不濟矣。」柸等許諾。時東南有黃黑雲,大如數畝,稍分為三,狀若匹布,東西經天,色黑而青,酉時貫日,日沒後分為七道,每相去數十丈,間有白雲如魚鱗,子時乃滅。韜素解天文,見而惡之,顧謂左右曰:「此變不小,當有刺客起于京師,不知誰定當之?」是夜,韜宴其僚屬於東明觀,樂奏,酒酣,愀然長歎曰:「人居世無常,別易會難。各付一杯,開意為吾飲,令必醉。知後會復何期而不飲乎!」因泫然流涕,左右莫不歔欷,因宿于佛精舍。宣使楊柸、牟皮、牟成、趙生等緣獼猴梯而入,殺韜,置其刀箭而去。旦,宣奏之。季龍哀驚氣絕,久之方蘇。將出
 臨之,其司空李農諫曰:「害秦公者恐在蕭牆之內,慮生非常,不可以出。」季龍乃止。嚴兵發哀于太武殿。宣乘素車,從千人,臨韜喪,不哭,直言呵呵,使舉衾看尸,大笑而去。收大將軍記室參軍鄭靖、尹武等,將委之以罪。



 季龍疑宣之害韜也,謀召之,懼其不入,乃偽言其母哀過危惙。宣不虞己之見疑也,入朝中宮,因而止之。建興人史科告稱:「韜死夜,宿東宮長上楊丕家,柸夜與五人從外來,相與語曰:『大事已定,但願大家老壽,吾等何患不富貴』。語訖便入。科寢闇中,柸不見也。科尋出逃匿。俄而柸與二人出求科不得,柸曰:『宿客聞人向語,當殺之斷口
 舌。今而得去,作大事矣。』科踰牆獲免。」季龍馳使收之,獲楊柸、牟皮、趙生等。柸、皮尋皆亡去,執趙生而詰之,生具首服。季龍悲怒彌甚,幽宣於席庫,以鐵環穿其頷而鎖之,作數斗木槽,和羹飯,以豬狗法食之。取害韜刀箭舐其血,哀號震動宮殿。積柴鄴北,樹標於其上,標末置鹿盧,穿之以繩,倚梯柴積,送宣於標所,使韜所親宦者郝稚、劉霸拔其髮,抽其舌,牽之登梯,上於柴積。郝稚雙繩貫其頷,鹿盧絞上,劉霸斷其手足,斫眼潰腹,如韜之傷。四面縱火,煙炎際天。季龍從昭儀已下數千登中臺以觀之。火滅,取灰分置諸門交道中。殺其妻子九人。宣小
 子年數歲,季龍甚愛之,抱之而泣。兒曰:「非兒罪。」季龍欲赦之,其大臣不聽,遂於抱中取而戮之,兒猶挽季龍衣而大叫,時人莫不為之流涕,季龍因此發病。又誅其四率已下三百人,宦者五十人,皆車裂節解,棄之漳水。洿其東宮,養豬牛。東宮衛士十餘萬人皆謫戍涼州。先是,散騎常侍趙攬言於季龍曰:「中宮將有變,宜防之。」及宣之殺韜也,季龍疑其知而不告,亦誅之。廢宣母杜氏為庶人。貴嬪柳氏,尚書耆之女也,以才色特幸,坐其二兄有寵于宣,亦殺之。季龍追其姿色,復納耆少女于華林園。



 季龍議立太子,其太尉張舉進曰:「燕公斌、彭城公遵
 並有武藝文德。陛下神齒已衰,四海未一,請擇二公而樹之。」初,戎昭張豺之破上邽也,獲劉曜幼女,年十二,有殊色,季龍得而嬖之,生子世,封齊公。至是,豺以季龍年長多疾,規立世為嗣,劉當為太后,己得輔政,說季龍曰:「陛下再立儲宮,皆出自倡賤,是以禍亂相尋。今宜擇母貴子孝者立之。」季龍曰:「卿且勿言,吾知太子處矣。」又議于東堂,季龍曰:「吾欲以純灰三斛洗吾腹,腹穢惡,故生凶子,兒年二十餘便欲殺公。今世方十歲,比其二十,吾已老矣。」於是與張舉、李農定議,敕公卿上書請立世。大司農曹莫不署名,季龍使張豺問其故。莫頓首曰:「天下
 業重,不宜立少,是以不敢署也。」季龍曰:「莫,忠臣也,然未達朕意。張舉、李農知吾心矣,其令諭之。」遂立世為皇太子,劉氏為皇后。季龍召太常條攸、光祿勛杜嘏謂之曰:「煩卿傅太子,實希改轍,吾之相託,卿宜明之。」署攸太傅,嘏為少傅。



 季龍時疾瘳,以永和五年僭即皇帝位于南郊,大赦境內,建元曰太寧。百官增位一等,諸子進爵郡王。以尚書張良為右僕射。



 故東宮謫卒高力等萬餘人當戍涼州,行達雍城,既不在赦例,又敕雍州刺史張茂送之。茂皆奪其馬,令步推鹿車,致糧戍所。高力督定陽梁犢等害眾心之怨,謀起兵東還,陰令胡人頡獨鹿微
 告戍者,戍者皆踴抃大呼。梁犢乃自稱晉征東大將軍,率眾攻陷下辯,逼張茂為大都督、大司馬,載以軺車。安西劉寧自安定擊之,大敗而還。秦、雍間城戍無不摧陷,斬二千石長史,長驅而東。高力等皆多力善射,一當十餘人,雖無兵甲,所在掠百姓大斧,施一丈柯,攻戰若神,所向崩潰,戍卒皆隨之,比至長安,眾已十萬。其樂平王石苞時鎮長安,盡銳距之,一戰而敗。犢遂東出潼關,進如洛川。季龍以李農為大都督,行大將軍事,統衛軍張賀度、征西張良、征虜石閔等,率步騎十萬討之。戰于新安,農師不利。又戰于洛陽,農師又敗,乃退壁成皋。犢東
 掠滎陽、陳留諸郡,季龍大懼,以燕王石斌為大都督中外諸軍事,率精騎一萬,統姚弋仲、苻洪等擊犢于滎陽東,大敗之,斬犢首而還,討其餘黨,盡滅之。



 俄而晉將軍王龕拔其沛郡。始平人馬勖起兵於洛氏葛谷,自稱將軍。石苞攻滅之,誅三千餘家。



 時熒惑犯積尸,又犯昴、月,及熒惑北犯河鼓。未幾,季龍疾甚,以石遵為大將軍,鎮關右,石斌為丞相、錄尚書事,張豺為鎮衛大將軍、領軍將軍、吏部尚書,並受遺輔政。劉氏懼斌之輔政也害世,與張豺謀誅之。斌時在襄國,乃遣使詐斌曰:「主上患已漸損,王須獵者,可小停也。」斌性好酒耽獵,遂游畋縱飲。
 劉氏矯命稱斌無忠孝之心,免斌官,以王歸第,使張豺弟雄率龍騰五百人守之。石遵自幽州至鄴,敕朝堂受拜,配禁兵三萬遣之,遵慟泣而去。是日季龍疾小瘳,問曰:「遵至未?」左右答言久已去矣。季龍曰:「恨不見之。」季龍臨於西閣,龍騰將軍、中郎二百餘人列拜于前。季龍曰:「何所求也?」皆言聖躬不和,宜令燕王入宿衛,典兵馬,或言乞為皇太子。季龍不知斌之廢也,責曰:「燕王不在內邪?呼來!」左右言王酒病,不能入。季龍曰:「促持輦迎之,當付其璽綬。」亦竟無行者。尋昏眩而入。張豺使弟雄等矯季龍命殺斌,劉氏又矯命以豺為太保、都督中外諸軍、
 錄尚書事,加千兵百騎,一依霍光輔漢故事。侍中徐統歎曰:「禍將作矣,吾無為豫之。」乃仰藥而死。俄而季龍亦死。季龍始以咸康元年僭立,至此太和六年,凡在位十五歲。



 於是世即偽位,尊劉氏為皇太后,臨朝,進張豺為丞相。豺請石遵、石鑒為左右丞相,以尉其心,劉氏從之。豺與張舉謀誅李農,而舉與農素善,以豺謀告之。農懼,率騎百餘奔廣宗,率乞活數萬家保於上白。劉氏使張舉等統宿衛精卒圍之。豺以張離為鎮軍大將軍、監中外諸軍事、司隸校尉,為己之副。鄴中群盜大起,迭相劫掠。



 石遵聞季龍之死,屯于河內。姚弋仲、苻洪、石閔、劉寧
 及武衛王鸞、寧西王午、石榮、王鐵、立義將軍段勤等既平秦、洛,班師而歸,遇遵于李城,說遵曰:「殿下長而且賢,先帝亦有意于殿下矣。但以末年惛惑,為張豺所誤。今上白相持未下,京師宿衛空虛,若聲張豺之罪,鼓行而討之,孰不倒戈開門而迎殿下者邪!」遵從之。洛州刺史劉國等亦率洛陽之眾至於李城。遵檄至鄴,張豺大懼,馳召上白之軍。遵次于蕩陰,戎卒九萬,石閔為前鋒。豺將出距之,耆舊羯士皆曰:「天子兒來奔喪,吾當出迎之,不能為張豺城戍也。」踰城而出,豺斬之不能止。張離率龍騰二千斬關迎遵。劉氏懼,引張豺入,對之悲哭曰:「先
 帝梓宮未殯,而禍難繁興。今皇嗣沖幼,託之於將軍,將軍何以匡濟邪?加遵重官,可以弭不?」豺惶怖失守,無復籌計,但言唯唯。劉氏令以遵為丞相、領大司馬、大都督中外諸軍、錄尚書事,加黃鉞、九錫,增封十郡,委以阿衡之任。遵至安陽亭,張豺懼而出迎,遵命執之。於是貫甲曜兵,入自鳳陽門,升于太武前殿,擗踴盡哀,退如東閣。斬張豺于平樂市,夷其三族。假劉氏令曰:「嗣子幼沖,先帝私恩所授,皇業至重,非所克堪。其以遵嗣位。」遵偽讓至于再三,群臣敦勸,乃受之,僭即尊位于太武前殿,大赦殊死已下,罷上白圍。封世為譙王,邑萬戶待以不臣
 之禮,廢劉氏為太妃,尋皆殺之。世凡立三十三日。



 於是李農歸請罪,遵復其位,待之如初。尊其母鄭氏為皇太后,其妻張氏為皇后,以石斌子衍為皇太子,石鑒為侍中,石沖為太保,石苞為大司馬,石琨為大將軍,石閔為中外諸軍事、輔國大將軍、錄尚書事,輔政。暴風拔樹,震雷,雨雹大如盂升。太武、暉華殿災,諸門觀閣蕩然,其乘輿服御燒者太半,光焰照天,金石皆盡,火月餘乃滅。雨血周遍鄴城。



 石沖時鎮于薊,聞遵殺世而自立,乃謂其僚佐曰:「世受先帝之命,遵輒廢殺,罪逆莫大,其敕內外戎嚴,孤將親討之。」於是留寧北沭堅戍幽州,帥眾五萬,
 自薊討遵,傳檄燕、趙,所在雲集,比及常山,眾十餘萬。次於苑鄉,遇遵赦書,謂左右曰:「吾弟一也,死者不可復追,何為復相殘乎!吾將歸矣。」其將陳暹進曰:「彭城篡弒自尊,為罪大矣。王雖北旆,臣將南轅,平京師,擒彭城,然後奉迎大駕。」沖從之。遵馳遣王擢以書喻沖,沖弗聽。遵假石閔黃鉞、金鉦,與李農等率精卒十萬討之。戰于平棘,沖師大敗,獲沖于元氏,賜死,坑其士卒三萬餘人。



 始葬季龍,號其墓為顯原陵,偽謚武皇帝,廟號太祖。



 遵揚州刺史王浹以淮南歸順。晉西中郎將陳逵進據壽春。征北將軍褚裒率師伐遵,次于下邳,遵以李農為南討
 大都督,率騎二萬來距。裒不能進,退屯廣陵。陳逵聞之,懼,遂焚壽春積聚,毀城而還。



 石苞時鎮長安,謀帥關中之眾攻鄴,左長史石光、司馬曹曜等固諫。苞怒,誅光等百餘人。苞性貪而無謀,雍州豪石知其無成,並遣使告晉梁州刺史司馬勳。勳於是率眾赴之,壁于懸鉤,去長安二百餘里,使治中劉煥攻京兆太守劉秀離,斬之。三輔豪右多殺其令長,擁三十餘壁,有眾五萬以應勳。苞輟攻鄴之謀,使麻秋、姚國等率騎距勳。遵遣車騎王朗率精騎二萬,以外討勳為名,因劫苞,送之于鄴。勳又為朗所距,釋懸鉤,拔宛城,殺遵南陽太守袁景而還。



 初,遵
 之發李城也,謂石閔曰:「努力!事成,以爾為儲貳。」既而立衍,閔甚失望,自以勳高一時,規專朝政,遵忌而不能任。閔既為都督,總內外兵權,乃懷撫殿中將士及故東宮高力萬餘人,皆奏為殿中員外將軍,爵關外侯,賜以宮女,樹己之恩。遵弗之猜也,而更題名善惡以挫抑之,眾咸怨矣。而又納中書令孟準、左衛將軍王鸞之計,頗疑憚於閔,稍奪兵權。閔益有恨色,準等咸勸誅之。遵召石鑒等入,議于其太后鄭氏之前,皆請誅之。鄭氏曰:「李城迴師,無棘奴豈有今日!小驕縱之,不可便殺也。」鑒出,遣宦者楊環馳以告閔,閔遂劫李農及右衛王基,密謀廢
 遵。使將軍蘇亥、周成率甲士三十執遵于如意觀。遵時方與婦人彈棋,問成等曰:「反者誰也?」成曰:「義陽王鑒當立。」遵曰:「我尚如是,汝等立鑒,復能幾時!」乃殺之于琨華殿,誅鄭氏及其太子衍、上光祿張斐、中書令孟準、左衛王鸞等。遵凡在位一百八十三日。



 鑒乃僭位,大赦殊死已下。以石閔為大將軍,封武德王,李農為大司馬,並錄尚書事;郎闓為司空,秦州刺史劉群為尚書左僕射,侍中盧諶為中書監。



 鑒使石苞及中書令李松、殿中將軍張才等夜誅閔、農於琨華殿,不克,禁中擾亂。鑒恐閔為變,偽若不知者,夜斬松、才於西中華門,並誅石苞。



 時石
 祗在襄國,與姚弋仲、苻洪等通和,連兵檄誅閔、農。鑒遣石琨為大都督,與張舉及侍中呼延盛率步騎七萬分討祗等。中領軍石成、侍中石啟、前河東太守石暉謀誅閔、農,閔、農殺之。



 龍驤孫伏都、劉銖等結羯士三千伏于胡天,亦欲誅閔等。時鑒在中臺,伏都率三十餘人將升臺挾鑒以攻之。臨見伏都毀閣道,鑒問其故。伏都曰:「李農等反,巳在東掖門,臣嚴率衛士,謹先啟知。」鑒曰:「卿是功臣,好為官陳力。朕從臺觀卿,勿慮無報也。」於是伏都及銖率眾攻閔、農,不剋,屯於鳳陽門。閔、農率眾數千毀金明門而入。鑒懼閔之誅己也,馳招閔、農,開門內之,謂
 曰:「孫伏都反,卿宜速討之。」閔、農攻斬伏都等,自鳳陽至琨華,橫尸相枕,流血成渠。宣令內外六夷敢稱兵杖者斬之。胡人或斬關,或踰城而出者,不可勝數。使尚書王簡、少府王鬱帥眾數千,守鑒于御龍觀,懸食給之。令城內曰:「與官同心者住,不同心者各任所之。」敕城門不復相禁。於是趙人百里內悉入城,胡羯去者填門。閔知胡之不為己用也,班令內外趙人,斬一胡首送鳳陽門者,文官進位三等,武職悉拜牙門。一日之中,斬首數萬。閔躬率趙人誅諸胡羯,無貴賤男女少長皆斬之,死者二十餘萬,尸諸城外,悉為野犬豺狼所食。屯據四方者,所
 在承閔書誅之,于時高鼻多鬚至有濫死者半。



 太宰趙鹿、太尉張舉、中軍張春、光祿石岳、撫軍石寧、武衛張季及諸公侯、卿、校、龍騰等萬餘人出奔襄國。石琨奔據冀州,撫軍張沈屯滏口,張賀度據石瀆,建義段勤據黎陽,寧南楊群屯桑壁,劉國據陽城,段龕據陳留,姚弋仲據混橋,苻洪據枋頭,眾各數萬。王朗、麻秋自長安奔于洛陽。秋承閔書,誅朗部胡千餘。朗奔于襄國。麻秋率眾奔于苻洪。



 石琨及張舉、王朗率眾七萬伐鄴,石閔率騎千餘,距之城北。閔執兩刃矛,馳騎擊之,皆應鋒摧潰,斬級三千。琨等大敗,遂歸于冀州。



 閔與李農率騎三萬討張
 賀度于石瀆,鑒密遣宦者齎書召張沈等,使承虛襲鄴。宦者以告閔、農,閔、農馳還,廢鑒殺之,誅季龍孫三十八人,盡殪石氏。鑒在位一百三日。



 季龍小男混,永和八年將妻妾數人奔京師,敕收付廷尉,俄而斬之於建康市。季龍十三子,五人為冉閔所殺,八人自相殘害,混至此又死。初,讖言滅石者陵,尋而石閔徙蘭陵公,季龍惡之,改蘭陵為武興郡,至是終為閔所滅。始勒以成帝咸和三年僭立,二主四子,凡二十三年,以穆帝永和五年滅。



 閔字永曾,小字棘奴,季龍之養孫也。父瞻,字弘武,本姓
 冉,名良,魏郡內黃人也。其先漢黎陽騎都督,累世牙門。勒破陳午,獲瞻,時年十二,命季龍子之。驍猛多力,攻戰無前。歷位左積射將軍、西華侯。閔幼而果銳,季龍撫之如孫。及長,身長八尺,善謀策,勇力絕人。拜建節將軍,徙封修成侯,歷位北中郎將、游擊將軍。季龍之敗於昌黎,閔軍獨全,由此功名大顯。及敗梁犢之後,威聲彌振,胡夏宿將莫不憚之。



 永和六年,殺石鑒,其司徒申鐘、司空郎闓等四十八人上尊號于閔,閔固讓李農,農以死固請,於是僭即皇帝位于南郊,大赦,改元曰永興,國號大魏,復姓冉氏。追尊其祖隆元皇帝,考瞻烈祖高皇帝,尊
 母王氏為皇太后,立妻董氏為皇后,子智為皇太子。以李農為太宰、領太尉、錄尚書事,封齊王,農諸子皆封為縣公。封其子胤、明、裕皆為王。文武進位三等,封爵有差。遣使者持節赦諸屯結,皆不從。



 石祗聞鑒死,僭稱尊號于襄國,諸六夷據州郡擁兵者皆應之。閔遣使臨江告晉曰:「胡逆亂中原,今已誅之。若能共討者,可遣軍來也。」朝廷不答。閔誅李農及其三子,并尚書令王謨、侍中王衍、中常侍嚴震、趙昇等。晉盧江太守袁真攻其合肥,執南蠻校尉桑坦,遷其百姓而還。



 石祗遣其相國石琨率眾十萬伐鄴,進據邯鄲。祗鎮南劉國自繁陽會琨。閔大
 敗琨於邯鄲,死者萬餘。劉國還屯繁陽。苻健自枋頭入關。張賀度、段勤與劉國、靳豚會于昌城,將攻鄴。閔遣尚書左僕射劉群為行臺都督,使其將王泰、崔通、周成等帥步騎十二萬次于黃城,閔躬統精卒八萬繼之,戰于蒼亭。賀度等大敗,死者二萬八千,追斬勒豚于陰安鄉,盡俘其眾,振旅而歸。戎卒三十餘萬,旌旗鐘鼓綿亙百餘里,雖石氏之盛無以過之。閔至自蒼亭,行飲至之禮,清定九流,準才授任,儒學後門多蒙顯進,于時翕然,方之為魏晉之初。



 閔率步騎十萬攻石祗于襄國,署其子太原王胤為大單于、驃騎大將軍。,以降胡一千配為麾
 下。光祿大夫韋謏啟諫甚切,閔覽之大怒,誅謏及其子孫。閔攻襄國百餘日,為土山地道,築室反耕。祗大懼,去皇帝之號,稱趙王,遣使詣慕容俊、姚弋仲以乞師。會石琨自冀州援祗,弋仲復遣其子襄率騎三萬八千至自滆頭,俊遣將軍悅綰率甲卒三萬自龍城,三方勁卒合十餘萬。閔遣車騎胡睦距襄下場長蘆,將軍孫威候琨于黃丘,皆為敵所敗,士卒略盡,睦、威單騎而還。琨等軍且至,閔將出擊之,衛將軍王泰諫曰:「窮寇固迷,希望外援。今彊救雲集,欲吾出戰,腹背擊我。宜固壘勿出,觀勢而動,以挫其謀。今陛下親戎,如失萬全,大事去矣。請慎無
 出,臣請率諸將為陛下滅之。」閔將從之,道士法饒進曰:「太白經昴,當殺胡王,一戰百克,不可失也。」閔攘袂大言曰:「吾戰決矣,敢諫者斬!」於是盡眾出戰。姚襄、悅綰、石琨等三面攻之,祗衝其後,閔師大敗。閔潛於襄國行宮,與十餘騎奔鄴。降胡慄特康等執冉胤及左僕射劉琦等送于祗,盡殺之。司空石璞、尚書令徐機、車騎胡睦、侍中李琳、中書監盧諶、少府王鬱、尚書劉欽、劉休等諸將士死者十餘萬人,於是人物殲矣。賊盜蜂起,司、冀大饑,人相食。自季龍末年而閔盡散倉庫以樹私恩。與羌胡相攻,無月不戰。青、雍、幽、荊州徙戶及諸氐、羌、胡、蠻數百
 餘萬,各還本土,道路交錯,互相殺掠,且饑疫死亡,其能達者十有二三。諸夏紛亂,無復農者。閔悔之,誅法饒父子,支解之,贈韋謏大司徒。



 石祗使劉顯帥眾七萬攻鄴。時閔潛還,莫有知者,內外兇兇,皆謂閔已沒矣。射聲校尉張艾勸閔親郊,以安眾心,閔從之,訛言乃止。劉顯次於明光宮,去鄴二十三里,閔懼,召衛將軍王泰議之。泰恚其謀之不從,辭以瘡甚。閔親臨問之,固稱疾篤。閔怒,還宮,顧謂左右曰:「巴奴,乃公豈假汝為命邪!要將先滅群胡,卻斬王泰。」於是盡眾而戰,大敗顯軍,追奔及于陽平,斬首三萬餘級。顯懼,密使請降,求殺祗為效,閔振旅
 而歸。會有告王泰招集秦人,將奔關中,閔怒,誅泰,夷其三族。劉顯果殺祗及其太宰趙鹿等十餘人,傳首于鄴,送質請命。驃騎石寧奔于柏人。閔命焚祗首於通衢。



 閔徐州刺史劉啟以鄄城歸順。劉顯復率眾伐鄴,閔擊敗之。還,稱號于襄國。閔徐州刺史周成、兗州刺史魏統、豫州牧冉遇、荊州刺史樂弘皆以城歸順。平南高崇、征虜呂護執洛州刺史鄭系,以三河歸順。慕容彪攻陷中山,殺閔寧北白同、幽州刺史劉準,降于慕容俊。時有雲黃赤色,起東北,長百餘丈,一白鳥從雲間西南去,占者惡之。



 劉顯率眾伐常山,太守蘇亥告難於閔。閔留其大
 將軍蔣幹等輔其太子智守鄴,親率騎八千救之。顯所署大司馬、清河王寧以棗強降於閔,收其餘眾,擊顯,敗之,追奔及于襄國。顯大將曹伏駒開門為應,遂入襄國,誅顯及其公卿已下百餘人,焚襄國宮室,遷其百姓于鄴。顯領軍范路率眾千餘,斬關奔于枋頭。



 時慕容俊已克幽、薊,略地至于冀州。閔帥騎距之,與慕容恪相遇於魏昌城。閔大將軍董閏、車騎張溫言於閔曰:「鮮卑乘勝氣勁,不可當也,請避之以溢其氣,然後濟師以擊之,可以捷也。」閔怒曰:「吾成師以出,將平幽州,斬慕容雋。今遇恪而避之,人將侮我矣。」乃與恪遇,十戰皆敗之。恪乃以
 鐵鎖連馬,簡善射鮮卑勇而無剛者五千,方陣而前。閔所乘赤馬曰朱龍,日行千里,左杖雙刃矛,右執鉤戟,順風擊之,斬鮮卑三百餘級。俄而燕騎大至,圍之數周。閔眾寡不敵,躍馬潰圍東走,行二十餘里,馬無故而死,為恪所擒,及董閏、張溫等送之于薊。俊立閔而問之曰:「汝奴僕下才,何自妄稱天子?」閔曰:「天下大亂,爾曹夷狄,人面獸心,尚欲篡逆。我一時英雄,何為不可作帝王邪!」俊怒,鞭之三百,送于龍城,告廆、皝廟。



 遣慕容評率眾圍鄴。劉寧及弟崇帥胡騎三千奔于晉陽,蘇亥棄常山奔于新興。鄴中饑,人相食,季龍時宮人被食略盡。冉智尚幼,
 蔣幹遣侍中繆嵩、詹事劉猗奉表歸順,且乞師于晉。濮陽太守戴施自倉垣次于棘津,止猗,不聽進,責其傳國璽。猗使嵩還鄴復命,幹沈吟未決,施乃率壯士百餘人入鄴,助守三臺,譎之曰:「且出璽付我。今凶寇在外,道路不通,未敢送也。須得璽,當馳白天子耳。天子聞璽已在吾處,信卿至誠,必遣軍糧厚相救餉。」乾以為然,乃出璽付之。施宣言使督護何融迎糧,陰令懷璽送于京師。長水校尉馬願、龍驤田香開門降評。施、融、蔣幹懸縋而下,奔于倉垣。評送閔妻董氏、太子智、太尉申鐘、司空條攸、中書監聶熊,司隸校尉籍羆、中書令李垣及諸王公卿士于
 薊。尚書令王簡、左僕射張乾、右僕射郎肅自殺。



 俊送閔既至龍城,斬於遏陘山。山左右七里草木悉枯,蝗蟲大起,五月不雨,至於十二月。俊遣使者祀之,謚曰武悼天王,其日大雪。是歲永和八年也。



 史臣曰:夫拯溺救焚,帝王之師也;窮凶騁暴,戎狄之舉也。蠢茲雜種,自古為虞,限以塞垣,猶懼侵軼,況乃入居中壤,窺我王政,乘弛紊之機,睹危亡之隙,而莫不嘯群鳴鏑,汨亂天常者乎!



 石勒出自羌渠,見奇醜類。聞鞞上黨,季子鑒其非凡;倚嘯洛城,夷甫識其為亂。及惠皇失統,宇內崩離,遂乃招聚蟻徒,乘間煽禍,虔劉我都邑,翦
 害我黎元。朝市淪胥,若沈航於鯨浪;王公顛仆,譬游魂於龍漠。豈天厭晉德而假茲妖孽者歟!觀其對敵臨危,運籌賈勇,奇謨間發,猛氣橫飛。遠嗤魏武,則風情慷慨;近答劉琨,則音詞倜儻。焚元超於苦縣,陳其亂政之愆;戮彭祖於襄國,數以無君之罪。於是跨躡燕、趙,并吞韓、魏,杖奇材而竊徽號,擁舊都而抗王室,褫毯裘,襲冠帶,釋介胄,開庠序,鄰敵懼威而獻款,絕域承風而納貢,則古之為國,曷以加諸!雖曰兇殘,亦一時傑也。而託授非所,貽厥無謀,身隕嗣滅,業歸攜養,斯乃知人之暗焉。



 季龍心昧德義,幼而輕險,假豹姿於羊質,騁梟心於狼性,
 始懷怨懟,終行篡奪。於是窮驕極侈,勞役繁興,畚鍤相尋,干戈不息,刑政嚴酷,動見誅夷,惵惵遺黎,求哀無地,戎狄殘獷,斯為甚乎!既而父子猜嫌,兄弟仇隙,自相屠膾,取笑天下。墳土未燥,禍亂薦臻,釁起於張豺,族傾於冉閔,積惡致滅,有天道哉!夫從逆則兇,事符影響;為咎必應,理若循環。世龍之殪晉人,既窮其酷;永曾之誅羯士,亦殲其類。無德不報,斯之謂乎!



 贊曰:中朝不競,蠻狄爭衡。塵飛五岳,霧晻三精。狡焉石氏,怙亂窮兵。流災肆慝,剽邑屠城。始自群盜,終假鴻名。勿謂兇醜,亦曰時英。季龍篡奪,淫虐播聲。身喪國泯,其
 由禍盈。



\end{pinyinscope}