\article{列傳第六 衛瓘 張華 劉卞}

\begin{pinyinscope}

 衛瓘子恆孫璪玠張華子禕韙劉卞



 衛瓘,字伯玉,河河東安邑人也。高祖暠,漢明帝時,以儒學自代郡徵,至河東安邑卒,因賜所亡地而葬之,子孫遂家焉。父覬,魏尚書。瓘年十歲喪父,至孝過人。性貞靜有名理,以明識清允稱。襲父爵閿鄉侯。弱冠為魏尚書郎。時魏法嚴苛,母陳氏憂之,瓘自請得徙為通事郎,轉中書郎。時權臣專政,瓘優游其間,無所親疏,甚為傅嘏所
 重,謂之甯武子。在位十年,以任職稱,累遷散騎常侍。陳留王即位,拜侍中,持節慰勞河北。以定議功,增邑戶。數歲轉廷尉卿。瓘明法理,每至聽訟,小大以情。



 鄧艾、鐘會之伐蜀也,瓘以本官持節監艾、會軍事,行鎮西軍司,給兵千人。蜀既平,艾輒承制封拜。會陰懷異志,因艾專擅,密與瓘俱奏其狀。詔使檻車徵之,會遣瓘先收艾。會以瓘兵少,欲令艾殺瓘,因加艾罪。瓘知欲危己,然不可得而距,乃夜至成都,檄艾所統諸將,稱詔收艾,其餘一無所問。若來赴官軍,爵賞如先;敢有不出,誅及三族。比至雞鳴,悉來赴瓘,唯艾帳內在焉。平旦開門,瓘乘使者車,
 徑入至成都殿前。艾臥未起,父子俱被執。艾諸將圖欲劫艾,整仗趣瓘營。瓘輕出迎之,偽作表草,將申明艾事,諸將信之而止。俄而會至,乃悉請諸將胡烈等,因執之,囚益州解舍,遂發兵反。於是士卒思歸,內外騷動,人情憂懼。會留瓘謀議,乃書版云「欲殺胡烈等」,舉以示瓘,瓘不許,因相疑貳。瓘如廁,見胡烈故給使,使宣語三軍,言會反。會逼瓘定議,經宿不眠,各橫刀膝上。在外諸軍已潛欲攻會。瓘既不出,未敢先發。會使瓘慰勞諸軍。瓘心欲去,且堅其意,曰:「卿三軍主,宜自行。」會曰:「卿監司,且先行,吾當後出。」瓘便下殿。會悔遣之,使呼瓘。瓘辭眩疾動,
 詐仆地。比出閣,數十信追之。瓘至外解,服鹽湯,大吐。瓘素羸,便似困篤。會遣所親人及醫視之,皆言不起,會由是無所憚。及暮,門閉,瓘作檄宣告諸軍。諸軍並已唱義,陵旦共攻會。會率左右距戰,諸將擊敗之,唯帳下數百人隨會繞殿而走,盡殺之。瓘於是部分諸將,群情肅然。鄧艾本營將士復追破檻車出艾,還向成都。瓘自以與會共陷艾,懼為變,又欲專誅會之功,乃遣護軍田續至綿竹,夜襲艾於三造亭,斬艾及其子忠。初,艾之入江由也,以續不進,將斬之,既而赦焉。及瓘遣續,謂之曰:「可以報江由之辱矣。」



 事平,朝議封瓘。瓘以剋蜀之功,群帥之
 力,二將跋扈,自取滅亡,雖運智謀,而無搴旗之效,固讓不受。除使持節、都督關中諸軍事、鎮西將軍,尋遷都督徐州諸軍事、鎮東將軍,增封菑陽侯,以餘爵封弟實開陽亭侯。泰始初,轉征東將軍,進爵為公,都督青州諸軍事、青州刺史,加征東大將軍、青州牧。所在皆有政績。除征北大將軍、都督幽州諸軍事、幽州刺史、護烏桓校尉。至鎮,表立平州,後兼督之。于時幽并東有務桓,西有力微,並為邊害。瓘離間二虜,遂致嫌隙,於是務桓降而力微以憂死。朝廷嘉其功,賜一子亭侯。瓘乞以封弟,未受命而卒,子密受封為亭侯。瓘六男無爵,悉讓二弟,遠近
 稱之。累求入朝,既至,武帝善遇之,俄使旋鎮。咸寧初,徵拜尚書令,加侍中。性嚴整,以法御下,視尚書若參佐,尚書郎若掾屬。瓘學問深博,明習文藝,與尚書郎敦煌索靖俱善草書,時人號為「一臺二妙」。漢末張芝亦善草書,論者謂瓘得伯英筋,靖得伯英肉。太康初,遷司空,侍中、令如故。為政清簡,甚得朝野聲譽。武帝敕瓘第四子宣尚繁昌公主。瓘自以諸生之胄,婚對微素,抗表固辭,不許。又領太子少傅,加千兵百騎鼓吹之府。以日蝕,瓘與太尉汝南王亮、司徒魏舒俱遜位,帝不聽。



 瓘以魏立九品,是權時之制,非經通之道,宜復古鄉舉里選。與太尉
 亮等上疏曰:「昔聖王崇賢,舉善而教,用使朝廷德讓,野無邪行。誠以閭伍之政,足以相檢,詢事考言,必得其善,人知名不可虛求,故還修其身。是以崇賢而俗益穆,黜惡而行彌篤。斯則鄉舉里選者,先王之令典也。自茲以降,此法陵遲。魏氏承顛覆之運,起喪亂之後,人士流移,考詳無地,故立九品之制,粗且為一時選用之本耳。其始造也,鄉邑清議,不拘爵位,褒貶所加,足為勸勵,猶有鄉論餘風。中間漸染,遂計資定品,使天下觀望,唯以居位為貴,人棄德而忽道業,爭多少於錐刀之末,傷損風俗,其弊不細。今九域同規,大化方始,臣等以為宜皆蕩除
 末法,一擬古制,以土斷,定自公卿以下,皆以所居為正,無復懸客遠屬異土者。如此,則同鄉鄰伍,皆為邑里,郡縣之宰,即以居長,盡除中正九品之制,使舉善進才,各由鄉論。然則下敬其上,人安其教,俗與政俱清,化與法並濟。人知善否之教,不在交遊,即華競自息,各求於己矣。今除九品,則宜準古制,使朝臣共相舉任,於出才之路既博,且可以厲進賢之公心,核在位之明闇,誠令典也。」武帝善之,而卒不能改。



 惠帝之為太子也,朝臣咸謂純質,不能親政事。瓘每欲陳啟廢之,而未敢發。後會宴陵雲臺,瓘託醉,因跪帝床前曰:「臣欲有所啟。」帝曰:「公所
 言何耶?」瓘欲言而止者三,因以手撫床曰:「此座可惜!」帝意乃悟,因謬曰:「公真大醉耶?」瓘於此不復有言。賈后由是怨瓘。



 宣尚公主,數有酒色之過。楊駿素與瓘不平,駿復欲自專權重,宣若離婚,瓘必遜位,於是遂與黃門等毀之,諷帝奪宣公主。瓘慚懼,告老遜立。乃下詔曰:「司空瓘年未致仕,而遜讓歷年,欲及神志未衰,以果本情,至真之風,實感吾心。今聽其所執,進位太保,以公就第。給親兵百人,置長史、司馬、從事中郎掾屬;及大車、官騎、麾蓋、鼓吹諸威儀,一如舊典。給廚田十頃、園五十畝、錢百萬、絹五百匹;床帳簟褥,主者務令優備,以稱吾崇賢之
 意焉。」有司又奏收宣付廷尉,免瓘位,詔不許。帝後知黃門虛構,欲還復主,而宣疾亡。



 惠帝即位,復瓘千兵。及楊駿誅,以瓘錄尚書事,加綠綟綬,劍履上殿,入朝不趨,給騎司馬,與汝南王亮共輔朝政。亮奏遣諸王還籓,與朝臣廷議,無敢應者,唯瓘贊其事,楚王瑋由是憾焉。賈后素怨瓘,且忌其方直,不得騁己淫虐;又聞瓘與瑋有隙,遂謗瓘與亮欲為伊霍之事,啟帝作手詔,使瑋免瓘等官。黃門齎詔授瑋,瑋性輕險,欲聘私怨,夜使清河王遐收瓘。左右疑遐矯詔,咸諫曰:「禮律刑名,台輔大臣,未有此比,且請距之。須自表得報,就戮未晚也。」瓘不從,遂與
 子恆、嶽、裔及孫等九人同被害,時年七十二。恆二子璪、玠,時在醫家得免。



 初,杜預聞瓘殺鄧艾,言於眾曰:「伯玉其不免乎!身為名士,位居總帥,既無德音,又不御下以正,是小人而乘君子之器,當何以堪其責乎?」瓘聞之,不俟駕而謝。終如預言。初,瓘家人炊飯,墮地盡化為螺,歲餘而及禍。太保主簿劉繇等冒難收瓘而葬之。



 初,瓘為司空,時帳下督榮晦有罪,瓘斥遣之。及難作,隨兵討瓘,故子孫皆及于禍。



 楚王瑋之伏誅也,瓘女與國臣書曰:「先公名謚未顯,無異凡人,每怪一國蔑然無言。《春秋》之失,其咎安在?悲憤感慨,故以示意。」於是繇等執黃幡,撾登
 聞鼓,上言曰:「初,矯詔者至,公承詔當免,即便奉送章綬,雖有兵仗,不施一刃,重敕出第,單車從命。如矯詔之文唯免公官,右軍以下即承詐偽,違其本文,輒戮宰輔,不復表上,橫收公子孫輒皆行刑,賊害大臣父子九人。伏見詔書『為楚王所誑誤,非本同謀者皆弛遣』。如書之旨,謂里舍人被驅逼齎白杖者耳。律,受教殺人,不得免死。況乎手害功臣,賊殺忠良,雖云非謀,理所不赦。今元惡雖誅,殺賊猶存。臣懼有司未詳事實,或有縱漏,不加精盡,使公父子仇賊不滅,冤魂永恨,訴於穹蒼,酷痛之臣,悲於明世。臣等身被創痍,殯斂始訖。謹條瓘前在司空
 時,帳下給使榮晦無情被黜,知瓘家人數、小孫名字。晦後轉給右軍,其夜晦在門外揚聲大呼,宣詔免公還第。及門開,晦前到中門,復讀所齎偽詔,手取公章綬貂蟬,催公出第。晦按次錄瓘家口及其子孫,皆兵仗將送,著東亭道北圍守,一時之間,便皆斬斫。害公子孫,實由於晦。及將人劫盜府庫,皆晦所為。考晦一人,眾姦皆出。乞驗盡情偽,加以族誅。」詔從之。



 朝廷以瓘舉門無辜受禍,乃追瓘伐蜀勳,封蘭陵郡公、增邑三千戶,謚曰成,贈假黃鉞。



 恆字巨山,少辟司空齊王府,轉太子舍人、尚書郎、祕書丞、太子庶子、黃門郎。



 恆善草隸書,為《四體書勢》曰:



 昔在黃帝,創制造物。有沮誦、倉頡者,始作書契,以代結繩,蓋睹鳥跡以興思也。因而遂滋,則謂之字,有六義焉。一曰指事,上、下是也。二曰象形,日、月是也。三曰形聲,江、河是也。四曰會意,武、信是也。五曰轉注,老、考是也。六曰假借,令、長是也。夫指事者,在上為上,在下為下。象形者,日滿月虧,效其形也。形聲者,以類為形,配以聲也。會意者,止戈為武,人言為信也。轉注者,以老壽考也。假借者,數言同字,其聲雖異,文意一也。自黃帝至三代,其文不改。及秦用篆書,焚燒先典,而古文絕矣。漢武時,魯恭王壞孔子宅,得《尚書》、《春秋》、《論語》、《孝經》。時人以不復知有古
 文,謂之科斗書。漢世祕藏,希得見之。魏初傳古文者,出於邯鄲淳。恆祖敬侯寫淳《尚書》,後以示淳,而淳不別。至正始中,立三字石經,轉失淳法,因科斗之名,遂效其形。太康元年,汲縣人盜發魏襄王塚,得策書十餘萬言。案敬侯所書,猶有仿佛。古書亦有數種,其一卷論楚事者最為工妙。恆竊悅之,故竭愚思,以贊其美,愧不足廁前賢之作,冀以存古人之象焉。古無別名,謂之字勢云。



 「黃帝之史,沮誦、倉頡,眺彼鳥跡,始作書契。紀綱萬事,垂法立制,帝典用宣,質文著世。爰暨暴秦,滔天作戾,大道既泯,古文亦滅。魏文好古,世傳丘墳,歷代莫發,真偽靡分。
 大晉開元,弘道敷訓,天垂其象,地耀其文。其文乃耀,粲矣其章,因聲會意,類物有方:日處君而盈其度,月執臣而虧其旁;雲委蛇而上布,星離離以舒光;禾卉苯䔿以垂穎,山嶽峨嵯而連岡;蟲跂跂其若動,鳥似飛而未揚。觀其錯筆綴墨,用心精專。勢和體均,發止無間。或守正循檢,矩折規旋。或方員靡則,因事制權。其曲如弓,其直如弦。矯然特出,若龍騰于川。森爾下頹,若雨墜于天。或引筆奮力,若鴻鴈高飛,邈邈翩翩。或縱肆阿那,若流蘇懸羽,靡靡綿綿。是故遠而望之,若翔風厲水,清波漪漣。就而察之,有若自然。信黃唐之遺跡,為六藝之範先。籀
 篆蓋其子孫,隸草乃其曾玄。睹物象以致思,非言辭之可宣。」



 昔周宣王時,史籀始著《大篆》十五篇,或與古同,或與古異,世謂之籀書者也。及平王東遷,諸侯力政,家殊國異,而文字乖形。秦始皇帝初兼天下。丞相李斯乃奏益之,罷不合秦文者,斯作《倉頡篇》,中車府令趙高作《爰歷篇》,太史令胡毋敬作《博學篇》,皆取史籀大篆,或頗省改,所謂小篆者。或曰,下土人程邈為衙獄吏,得罪始皇,幽繫雲陽十年,從獄中作大篆,少者增益,多者損減,方者使員,員者使方,奏之始皇。始皇善之,出以為御史,使定書。或曰,邈所定乃隸字也。自秦壞古文,有八體,一曰
 大篆,二曰小篆,三曰刻符,四曰蟲書,五曰摹印,六曰署書,七曰殳書,八曰隸書。王莽時,使司空甄豐校文字部,改定古文,復有六書。一曰古文,孔氏壁中書也。二曰奇字,即古文而異者也。三曰篆書,秦篆書也。四曰佐書,即隸書也。五曰繆篆,所以摹印也。六曰鳥書,所以書幡信也。及許慎撰《說文》,用篆書為正,以為體例,最可得而論也。秦時李斯號為二篆,諸山及銅人銘皆斯書也。漢建初中,扶風曹喜少異於斯,而亦稱善。邯鄲淳師焉,略究其妙,韋誕師淳而不及也。太和中,誕為武都太守,以能書,留補侍中,魏氏寶器銘題皆誕書也。漢末又有蔡邕,
 採斯喜之法,為古今雜形,然精密閑理不如淳也。



 邕作《篆勢》曰:「鳥遺跡,皇頡循。聖作則,制斯文。體有六,篆為真。形要妙,巧入神,或龜文金咸列,櫛比龍鱗;紓體放尾,長短復身;頹若黍稷之垂穎,蘊若蟲蛇之焚縕;揚波振撆,鷹歭鳥震;延頸脅翼,勢似陵雲。或輕筆內投,微本濃末,若絕若連;似水露綠絲,凝垂下端;從者如懸,衡者如編;杳杪邪趣,不方不員;若行若飛,跂歉胗胗。遠而望之,象鴻鵠群游,駱驛遷延;迫而視之,端際不可得見。指捴不可勝原。研桑不能數其詰屈,離婁不能睹其郤間,般倕揖讓而辭巧,籀誦拱手而韜翰。處篇籍之首目,粲斌斌其
 可觀。摛華艷於紈素,為學藝之範先。喜文德之弘懿,慍作者之莫刊。思字體之俯仰,舉大略而論旃。」



 秦既用篆,奏事繁多,篆字難成,即令隸人佐書,曰隸字。漢因行之,獨符、印璽、幡信、題署用篆。隸書者,篆之捷也。上谷王次仲始作楷法。至靈帝好書,時多能者,而師宜官為最,大則一字徑丈,小則方寸千言,甚矜其能。或時不持錢詣酒家飲,因書其壁,顧觀者以酬酒,討錢足而滅之。每書輒削而焚其柎。。梁鵠乃益為版而飲之酒,候其醉而竊其柎。鵠卒以書至選部尚書。宜官後為袁術將,今鉅鹿宋子有《耿球碑》,是術所立,其書甚工,云是宜官也。梁鵠
 奔劉表,魏武帝破荊州,募求鵠。鵠之為選部也,魏武欲為洛陽令,而以為北部尉,故懼而自縛詣門,署軍假司馬;在秘書以勤書自效,是以今者多有鵠手跡。魏武帝懸著帳中,及以釘壁玩之,以為勝宜官。今宮殿題署多是鵠篆。鵠宜為大字,邯鄲淳宜為小字。鵠謂淳得次仲法,然鵠之用筆盡其勢矣。鵠弟子毛弘教於祕書,今八分皆弘法也。漢末有左子邑,小與淳鵠不同,然亦有名。



 魏初有鐘胡二家為行書法,俱學之於劉德升,而鐘氏小異,然亦各有巧,今大行於世云。作《隸勢》曰:「鳥跡之變,乃惟佐隸。蠲彼繁文,崇此簡易。厥用既弘,體象有度。煥
 若星陳,鬱若雲布。其大徑尋,細不容髮。隨事從宜,靡有常制。或穹隆恢廓,或櫛比鍼列,或砥平繩直,或蜿蜒膠戾,或長邪角趣,或規旋矩折。修短相副,異體同勢。奮筆輕舉,離而不絕。纖波濃點,錯落其間,若鍾虡設張,庭燎盡煙,嶄巖截嵯,高下屬連。似崇臺重宇,增雲冠山。遠而望之,若飛龍在天;近而察之,心亂目眩。奇姿譎詭,不可勝原。研桑所不能計,宰賜所不能言。何草篆之足算,而斯文之未宣。豈體大之難睹,將祕奧之不傳?聊俯仰而詳觀,舉大較而論旃。」



 漢興而有草書,不知作者姓名。至章帝時,齊相杜度號善作篇。後有崔瑗、崔寔,亦皆稱工,
 杜氏殺字甚安,而書體微瘦。崔氏甚得筆勢,而結字小疏。弘農張伯英者,因而轉精甚巧。凡家之衣帛,必書而後練之。臨池學書,池水盡黑。下筆必為楷則,號匆匆不暇草書,寸紙不見遺,至今世尤寶其書,韋仲將謂之草聖。伯英弟文舒者,次伯英。又有姜孟穎、梁孔達,田彥和及韋仲將之徒,皆伯英弟子,有名於世,然殊不及文舒也。羅叔景、趙元嗣者,與伯英並時,見稱於西州,而矜巧自與,眾頗惑之。故英自稱「上比崔杜不足,下方羅趙有餘。」河間張超亦有名,然雖與崔氏同州,不如伯英之得其法也。



 崔瑗作《草書勢》曰:「書契之興,始自頡皇。寫彼鳥
 跡,以定文章,爰暨末葉,典籍彌繁。時之多僻,政之多權。官事荒蕪,剿其墨翰。惟作佐隸,舊字是刪。草書之法,蓋又簡略。應時諭指,用於卒迫。兼功並用,愛日省力。純儉之變,豈必古式。觀其法象,俯仰有儀。方不中矩,員不副規;抑左揚右,望之若崎。竦企鳥歭,志大飛移。狡獸暴駭,將奔未馳。或點,狀似連珠,絕而不離;畜怒怫鬱,放逸生奇。或凌邃惴心慄,若據槁臨危;旁點邪附,似蜩螗挶枝。絕筆收勢,餘綖糾結,若杜伯揵毒緣戲,螣蛇赴穴,頭沒尾垂。是故遠而望之,崔焉若沮岑崩崖;就而察之,一畫不可移。機微要妙,臨時從宜。略舉大較,仿佛若斯。」



 及瓘為楚王瑋所構,恆聞變,以何劭,嫂之父也,從牆孔中詣之,以問消息。劭知而不告。恆還經廚下,收人正食,因而遇害。後贈長水校尉,謚蘭陵貞世子。二子:璪、玠。



 璪字仲寶,襲瓘爵。後東海王越以蘭陵益其國,改封江夏郡公,邑八千五百戶。懷帝即位,為散騎侍郎。永嘉五年,沒於劉聰。元帝以瓘玄孫崇嗣。



 玠字叔寶,年五歲,風神秀異。祖父瓘曰:「此兒有異於眾,顧吾年老,不見其成長耳!」總角乘羊車入市,見者皆以為玉人,觀之者傾都。驃騎將軍王濟,玠之舅也,俊爽有風姿,每見玠,輒歎曰:「珠玉在側,覺我形穢。」又嘗語人曰:「與玠同遊,冏若明珠之
 在側,朗然照人。」及長,好言玄理。其後多病體羸,母恆禁其語。遇有勝日,親友時請一言,無不咨嗟,以為入微。琅邪王澄有高名,少所推服,每聞玠言,輒歎息絕倒。故時人為之語曰:「衛玠談道,平子絕倒。」澄及王玄、王濟並有盛名,皆出玠下,世云「王家三子,不如衛家一兒。」玠妻父樂廣,有海內重名,議者以為「婦公冰清,女婿玉潤。」



 辟命屢至,皆不就。久之,為太傅西閣祭酒,拜太子洗馬。璪為散騎侍郎,內侍懷帝。玠以天下大亂,欲移家南行。母曰:「我不能舍仲寶去也。」玠啟諭深至,為門戶大計,母涕泣從之。臨別,玠謂兄曰:「在三之義,人之所重。今可謂致身
 之日,兄其勉之。」乃扶輿母轉至江夏。



 玠妻先亡。征南將軍山簡見之,甚相欽重。簡曰:「昔戴叔鸞嫁女,唯賢是與,不問貴賤,況衛氏權貴門戶令望之人乎!」於是以女妻焉。遂進豫章,是時大將軍王敦鎮豫章,長史謝鯤先雅重玠,相見欣然,言論彌日。敦謂鯤曰:「昔王輔嗣吐金聲於中朝,此子復玉振於江表,微言之緒,絕而復續。不意永嘉之末,復聞正始之音,何平叔若在,當復絕倒。」玠嘗以人有不及,可以情恕;非意相干,可以理遣,故終身不見喜慍之容。



 以王敦豪爽不群,而好居物上,恐非國之忠臣,求向建鄴。京師人士聞其姿容,觀者如堵。玠勞疾
 遂甚,永嘉六年卒,時年二十七,時人謂玠被看殺。葬於南昌。謝鯤哭之慟,人問曰:「子有何恤而致斯哀?」答曰:「棟梁折矣,不覺哀耳。」咸和中,改塋於江寧。丞相王導教曰:「衛洗馬明當改葬。此君風流名士,海內所瞻,可修薄祭,以敦舊好。」後劉惔、謝尚共論中朝人士,或問:「杜乂可方衛洗馬不?」尚曰:「安得相比,其間可容數人。」惔又云:「杜乂膚清,叔寶神清。」其為有識者所重若此。于時中興名士,唯王承及玠為當時第一云。



 恆族弟展字道舒,歷尚書郎、南陽太守。永嘉中,為江州刺史,累遷晉王大理。詔有考子證父,或鞭父母問子所在,展以為恐傷正教,並奏
 除之。中興建,為廷尉,上疏宜復肉刑,語在《刑法志》。卒,贈光祿大夫。



 張華,字茂先,范陽方城人也。父平,魏漁陽郡守。華少孤貧,自牧羊,同郡盧欽見而器之。鄉人劉放亦奇其才,以女妻焉。華學業優博,辭藻溫麗,朗贍多通,圖緯方伎之書莫不詳覽。少自修謹,造次必以禮度。勇於赴義,篤於周急。器識弘曠,時人罕能測之。初未知名,著《鷦鷯賦》以自寄。其詞曰:



 何造化之多端,播群形於萬類。惟鷦鷯之微禽,亦攝生而受氣,育翩翾之陋體,無玄黃以自貴;毛
 無施於器用,肉不登乎俎味。鷹鸇過猶戢翼,尚何懼於罿罻!翳薈蒙籠,是焉游集。飛不飄揚,翔不翕集。其居易容,其求易給;巢林不過一枝,每食不過數粒。棲無所滯。游無所盤;匪陋荊棘,匪榮茝蘭。動翼而逸,投足而安。委命順理,與物無患。伊茲禽之無知,而處身之似智。不懷寶以賈害,不飾表以招累。靜守性而不矜,動因循而簡易。任自然以為資,無誘慕於世偽。雕鶡介其觜距,鵠鷺軼於雲際,鵾雞竄於幽險,孔翠生乎遐裔,彼晨鳧與歸鴈,又矯翼而增逝,咸美羽而豐肌,故無罪而皆斃;徒銜蘆以避繳,終為戮於此世。蒼鷹鷙而受紲,鸚鵡慧而入
 籠,屈猛志以服養,塊幽縶於九重;變音聲以順旨,思摧翮而為庸。戀鍾岱之林野,慕隴坻之高松。雖蒙幸於於日,未若疇昔之從容。海鳥爰居,避風而至;條支巨爵,踰嶺自致;提挈萬里,飄颻逼畏。夫惟體大妨物,而形瑰足偉也。陰陽陶烝,萬品一區。巨細舛錯,種繁類殊。鷦冥巢於蚊睫,大鵬彌乎天隅,將以上方不足而下比有餘。普天壤而遐觀,吾又安知大小之所如。



 陳留阮籍見之,歎曰:「王佐之才也!」由是聲名始著。郡守鮮于嗣薦華為太常博士。盧欽言之於文帝,轉河南尹丞,未拜,除佐著作郎。頃之,遷長史,兼中書郎。朝議表奏,多見施用,遂即真。
 晉受禪,拜黃門侍郎,封關內侯。



 華彊記默識,四海之內,若指諸掌。武帝嘗問漢宮室制度及建章千門萬戶,華應對如流,聽者忘倦,畫地成圖,左右屬目。帝甚異之,時人比之子產。數歲,拜中書令,後加散騎常侍。遭母憂,哀毀過禮,中詔勉勵,逼令攝事。



 初,帝潛與羊祜謀伐吳,而群臣多以為不可,唯華贊成其計。其後,祜疾篤,帝遣華詣祜,問以伐吳之計,語在《祜傳》。及將大舉,以華為度支尚書,乃量計運漕,決定廟算。眾軍既進,而未有剋獲,賈充等奏誅華以謝天下。帝曰:「此是吾意,華但與吾同耳。」時大臣皆以為未可輕進,華獨堅執,以為必剋。及吳滅,
 詔曰:「尚書、關內侯張華,前與故太傅羊祜共創大計,遂典掌軍事,部分諸方,算定權略,運籌決勝,有謀謨之勳。其進封為廣武縣侯,增邑萬戶,封子一人為亭侯,千五百戶,賜絹萬匹。」



 華名重一世,眾所推服,晉史及儀禮憲章並屬於華,多所損益。當時詔誥皆所草定,聲譽益盛,有台輔之望焉。而荀勖自以大族,恃帝恩深,憎疾之,每伺間隙,欲出華外鎮。會帝問華:「誰可託寄後事者?」對曰:「明德至親,莫如齊王攸。」既非上意所在,微為忤旨,間言遂行。乃出華為持節、都督幽州諸軍事、領護烏桓校尉、安北將軍。撫納新舊,戎夏懷之。東夷馬韓、新彌諸國依
 山帶海,去州四千餘里,歷世未附者二十餘國,並遣使朝獻。於是遠夷賓服,四境無虞,頻歲豐稔,士馬彊盛。



 朝議欲徵華入相,又欲進號儀同。初,華毀徵士馮恢於帝,紞即恢之弟也,深有寵於帝。紞嘗侍帝,從容論魏晉事,因曰;「臣竊謂鍾會之釁,頗由太祖。」帝變色曰:「卿何言邪!」紞免冠謝曰;「臣愚冗瞽言,罪應萬死。然臣微意,猶有可申。」帝曰:「何以言之」紞曰:「臣以為善御者必識六轡盈縮之勢,善政者必審官方控帶之宜,故仲由以兼人被抑,冉求以退弱被進,漢高八王以寵過夷滅,光武諸將由抑損克終。非上有仁暴之殊,下有愚智之異,蓋抑揚與
 奪使之然耳。鍾會才見有限,而太祖誇獎太過,嘉其謀猷,盛其名器,居以重勢,委以大兵,故使會自謂算無遺策,功在不賞,輈張跋扈,遂構凶逆耳。向令太祖錄其小能,節以大禮,抑之以權勢,納之以軌則,則亂心無由而生,亂事無由而成矣。」帝曰:「然。」紞稽首曰:「陛下既已然微臣之言,宜思堅冰之漸,無使如會之徒復致覆喪。」帝曰:「當今豈有如會者乎?」紞曰:「東方朔有言『談何容易』,《易》曰:『臣不密則失身』。」帝乃屏左右曰:「卿極言之。」紞曰:「陛下謀謨之臣,著大功於天下,海內莫不聞知,據方鎮總戎馬之任者,皆在陛下聖慮矣。」帝默然。頃之,徵華為太常。以
 太廟屋棟折,免官。遂終帝之世,以列侯朝見。



 惠帝即位,以華為太子少傅,與王戎、裴楷、和嶠俱以德望為楊駿所忌,皆不與朝政。及駿誅後,將廢皇太后,會群臣於朝堂,議者皆承望風旨,以為《春秋》絕文姜,今太后自絕於宗廟,亦宜廢黜。」惟華議以為「夫婦之道,父不能得之於子,子不能得之於父,皇太后非得罪於先帝者也。今黨其所親,為不母於聖世,宜依漢廢趙太后為孝成后故事,貶太后之號,還稱武皇后,居異宮,以全貴終之恩」。不從,遂廢太后為庶人。



 楚王瑋受密詔殺太宰汝南王亮、太保衛瓘等,內外兵擾,朝廷大恐,計無所出。華白帝以「
 瑋矯詔擅害二公,將士倉卒,謂是國家意,故從之耳。今可遣騶虞幡使外軍解嚴,理必風靡。」上從之,瑋兵果敗。及瑋誅,華以首謀有功,拜右光祿大夫、開府儀同三司、侍中、中書監,金章紫綬。固辭開府。



 賈謐與后共謀,以華庶族,儒雅有籌略,進無逼上之嫌,退為眾望所依,欲倚以朝綱,訪以政事。疑而未決,以問裴頠,頠素重華,深贊其事。華遂盡忠匡輔,彌縫補闕,雖當闇主虐后之朝,而海內晏然,華之功也。華懼后族之盛,作《女史箴》以為諷。賈后雖凶妒,而知敬重華。久之,論前後忠勳,進封壯武郡公。華十餘讓,中詔敦譬,乃受。數年,代下邳王晃為司
 空,領著作。



 及賈后謀廢太子,左衛率劉卞甚為太子所信遇,每會宴,卞必預焉。屢見賈謐驕傲,太子恨之,形于言色,謐亦不能平。卞以賈后謀問華,華曰:「不聞。」卞曰:「卞以寒悴,自須昌小吏受公成拔,以至今日。士感知己,是以盡言,而公更有疑於卞邪!」華曰:「假令有此,君欲如何?」卞曰:「東宮俊乂如林,四率精兵萬人。公居阿衡之任,若得公命,皇太子因朝入錄尚書事,廢賈后於金墉城,兩黃門力耳。」華曰:「今天子當陽,太子,人子也,吾又不受阿衡之命,忽相與行此,是無其君父,而以不孝示天下也。雖能有成,猶不免罪,況權戚滿朝,威柄不一,而可以安
 乎!」及帝會群臣於式乾殿,出太子手書,遍示群臣,莫敢有言者。惟華諫曰;「此國之大禍。自漢武以來,每廢黜正嫡,恆至喪亂。且國家有天下日淺,願陛下詳之。」尚書左僕射裴頠以為宜先檢校傳書者,又請比校太子手書,不然,恐有詐妄。賈后乃內出太子素啟事十餘紙,眾人比視,亦無敢言非者,議至日西不決,后知華等意堅,因表乞免為庶人,帝乃可其奏。



 初,趙王倫為鎮西將軍,撓亂關中,氐羌反叛,乃以梁王肜代之。或說華曰:「趙王貪昧,信用孫秀,所在為亂,而秀變詐,姦人之雄。今可遣梁王斬秀,刈趙之半,以謝關右,不亦可乎!」華從之,肜許諾。
 秀友人辛冉從西來,言於肜曰:「氐羌自反,非秀之為。」故得免死。倫既還,諂事賈后,因求錄尚書事,後又求尚書令。華與裴頠皆固執不可,由是致怨,倫、秀疾華如仇。武庫火,華懼因此變作,列兵固守,然後救之,故累代之寶及漢高斬蛇劍、王莽頭、孔子屐等盡焚焉。時華見劍穿屋而飛,莫知所向。



 初,華所封壯武郡有桑化為柏,識者以為不詳。又華第舍及監省數有妖怪。少子韙以中台星坼,勸華遜位。華不從,曰;「天道玄遠,惟修德以應之耳。不如靜以待之,以俟天命。」及倫、秀將廢賈后,秀使司馬雅夜告華曰:「今社稷將危,趙王欲與公共匡朝廷,為霸者
 之事。」華知秀等必成篡奪,乃距之。雅怒曰:「刃將加頸,而吐言如此!」不顧而出。華方晝臥,忽夢見屋壞,覺而惡之。是夜難作,詐稱詔召華,遂與裴頠俱被收。華將死,謂張林曰:「卿欲害忠臣耶?」林稱詔詰曰:「卿為宰相,任天下事,太子之廢,不能死節,何也」華曰:「式乾之議,臣諫事具存,非不諫也。」林曰:「諫若不從,何不去位?」華不能答。須臾,使者至曰:「詔斬公。」華曰:「臣先帝老臣,中心如丹。臣不愛死,懼王室之難,禍不可測也。」遂害之於前殿馬道南,夷三族,朝野莫不悲痛之。時年六十九。



 華性好人物,誘進不倦,至于窮賤候門之士有一介之善者,便咨嗟稱詠,
 為之延譽。雅愛書籍,身死之日,家無餘財,惟有文史溢於機篋。嘗徙居,載書三十乘。秘書監摯虞撰定官書,皆資華之本以取正焉。天下奇秘,世所希有者,悉在華所。由是博物洽聞,世無與比。



 惠帝中,人有得鳥毛三丈,以示華。華見,慘然曰:「此謂海鳧毛也,出則天下亂矣。」陸機嘗餉華鮓,于時賓客滿座,華發器,便曰:「此龍肉也。」眾未之信,華曰:「試以苦酒濯之,必有異。」既而五色光起。機還問鮓主,果云:「園中茅積下得一白魚,質狀殊常,以作鮓,過美,故以相獻。」武庫封閉甚密,其中忽有雉雊。華曰:「此必蛇化為雉也。」開視,雉側果有蛇蛻焉。吳郡臨平岸崩,
 出一石鼓,槌之無聲。帝以問華,華曰:「可取蜀中桐材,刻為魚形,扣之則鳴矣。」於是如其言,果聲聞數里。



 初,吳之未滅也,斗牛之間常有紫氣,道術者皆以吳方彊盛,未可圖也,惟華以為不然。及吳平之後,紫氣愈明。華聞豫章人雷煥妙達緯象,乃要煥宿,屏人曰:「可共尋天文,知將來吉凶。」因登樓仰觀,煥曰:「僕察之久矣,惟斗牛之間頗有異氣。」華曰:「是何祥也?」煥曰:「寶劍之精,上徹於天耳。」華曰:「君言得之。吾少時有相者言,吾年出六十,位登三事,當得寶劍佩之。斯言豈效與!」因問曰:「在何郡?」煥曰:「在豫章豐城。」華曰:「欲屈君為宰,密共尋之,可乎?」煥許之。華大
 喜,即補煥為豐城令。煥到縣,掘獄屋基,入地四丈餘,得一石函,光氣非常,中有雙劍,並刻題,一曰龍泉,一曰太阿。其夕,斗牛間氣不復見焉。煥以南昌西山北巖下土以拭劍,光芒艷發。大盆盛水,置劍其上,視之者精芒炫目。遣使送一劍並土與華,留一自佩。或謂煥曰:「得兩送一,張公豈可欺乎?」煥曰:「本朝將亂,張公當受其禍。此劍當繫徐君墓樹耳。靈異之物,終當化去,不永為人服也。」華得劍,寶愛之,常置坐側。華以南昌土不如華陰赤土,報煥書曰:「詳觀劍文,乃干將也,莫邪何復不至?雖然,天生神物,終當合耳。」因以華陰土一斤致煥。煥更以拭劍,
 倍益精明。華誅,失劍所在。煥卒,子華為州從事,持劍行經延平津,劍忽於腰間躍出墮水,使人沒水取之,不見劍,但見兩龍各長數丈,蟠縈有文章,沒者懼而反。須臾光彩照水,波浪驚沸,於是失劍。華歎曰:「先君化去之言,張公終合之論,此其驗乎!」華之博物多此類,不可詳載焉。



 後倫、秀伏誅,齊王冏輔政,摯虞致箋於冏曰:「間於張華沒後入中書省,得華先帝時答詔本草。先帝問華可以輔政持重付以後事者,華答:「明德至親,莫如先王,宜留以為社稷之鎮。」其忠良之謀,款誠之言,信於幽冥,沒而後彰,與茍且隨時者不可同世而論也。議者有責華
 以愍懷太子之事不抗節廷爭。當此之時,諫者必得違命之死。先聖之教,死而無益者,不以責人。故晏嬰,齊之正卿,不死崔杼之難;季札,吳之宗臣,不爭逆順之理。理盡而無所施者,固聖教之所不責也。」冏於是奏曰:「臣聞興微繼絕,聖王之高政;貶惡嘉善,《春秋》之美義。是以武王封比干之墓,表商容之閭,誠幽明之故有以相通也。孫秀逆亂,滅佐命之國,誅骨鯁之臣,以斲喪王室;肆其虐戾,功臣之後,多見泯滅。張華、裴頠各以見憚取誅於時,解系、解結同以羔羊並被其害,歐陽建等無罪而死,百姓憐之。今陛下更日月之光,布維新之命,然此等諸
 族未蒙恩理。昔欒郤降在皁隸,而《春秋》傳其違;幽王絕功臣之後,棄賢者子孫,而詩人以為刺。臣備忝在職,思納愚誠。若合聖意,可令群官通議。」議者各有所執,而多稱其冤。壯武國臣竺道又詣長沙王,求復華爵位,依違者久之。



 太安二年,詔曰:「夫愛惡相攻,佞邪醜正,自古而有。故司空、壯武公華竭其忠貞,思翼朝政,謀謨之勳,每事賴之。前以華弼濟之功,宜同封建,而華固讓至于八九,深陳大制不可得爾,終有顛敗危辱之慮,辭義懇誠,足勸遠近。華之至心,誓於神明。華以伐吳之勳,受爵於先帝。後封既非國體,又不宜以小功踰前大賞,華之見
 害,俱以姦逆圖亂,濫被枉賊。其復華侍中、中書監、司空、公、廣武侯及所沒財物與印綬符策,遣使弔祭之。」



 初,陸機兄弟志氣高爽,自以吳之名家,初入洛,不推中國人士,見華一面如舊,欽華德範,如師資之禮焉。華誅後,作誄,又為《詠德賦》以悼之。



 華著《博物志》十篇,及文章並行於世。二子:禕、韙。



 禕字彥仲,好學,謙敬有父風,歷位散騎常侍。韙儒博,曉天文,散騎侍郎。同時遇害。禕子輿,字公安,襲華爵。避難過江,辟丞相掾、太子舍人。



 劉卞,字叔龍,東平須昌人也。本兵家子,質直少言。少為
 縣小吏,功曹夜醉如廁,使卞執燭,不從,功曹銜之,以他事補亭子。有祖秀才者,於亭中與刺史箋,久不成,卞教之數言,卓犖有大致。秀才謂縣令曰:「卞,公府掾之精者,卿云何以為亭子?」令即召為門下史,百事疏簡,不能周密。令問卞:「能學不?」答曰:「願之。」即使就學。無幾,卞兄為太子長兵,即死,兵例須代,功曹請以卞代兄役。令曰:「祖秀才有言。」遂不聽。卞後從令至洛,得入太學,試《經》為臺四品吏。訪問令寫黃紙一鹿車,卞曰:「劉卞非為人寫黃紙者也。」訪問知怒,言於中正,退為尚書令吏。或謂卞曰:「君才簡略,堪大不堪小,不如作守舍人。」卞從其言。



 後為吏部
 令史,遷齊王攸司空主簿,轉太常丞、司徒左西曹掾、尚書郎,所歷皆稱職。累遷散騎侍郎,除并州刺史,入為左衛率,知賈后廢太子之謀,甚憂之。以計干張華而不見用,益以不平。賈后親黨微服聽察外間,頗聞卞言,乃遷卞為輕車將軍、雍州刺史,卞知言泄,恐為賈后所誅,乃飲藥卒。初,卞之並州,昔同時為須昌小吏者十餘人祖餞之,其一人輕卞,卞遣扶出之,人以此少之。



 史臣曰:夫忠為令德,學乃國華,譬眾星之有禮義,人倫之有冠冕也。衛瓘撫武帝之床,張華距趙倫之命,進諫則伯玉居多,臨危則茂先為美。遵乎險轍,理有可言:昏
 亂方凝,則事睽其趣;松筠無改,則死勝於生,固以赴蹈為期,而不辭乎傾覆者也。俱陷淫網,同嗟承劍,邦家殄瘁,不亦傷哉!



 贊曰:賢人委質,道映陵寒。尸祿觀敗,吾生未安。衛以賈滅,張由趙殘。忠於亂世,自古為難。



\end{pinyinscope}