\article{載記第二十七}

\begin{pinyinscope}

 慕容德



 慕容德,字玄明,皝之少子也,母公孫氏夢日入臍中,晝寢而生德。年未弱冠,身長八尺二寸,姿貌雄偉,額有日角偃月重文。博觀群書,性清慎,多才藝。慕容俊之僭立也,封為梁公,歷幽州刺史、左衛將軍。及嗣位,改范陽王,稍遷魏尹,加散騎常侍。俄而苻堅將苻雙據陜以叛,堅將苻柳起兵枹罕,將應之。德勸乘釁討堅,辭旨
 慷慨,識者言其有遠略,竟不能用。德兄垂甚壯之,因共論軍國大謀,言必切至。垂謂之曰:「汝器識長進,非復吳下阿蒙也。」枋頭之役,德以征南將軍與垂擊敗晉師。及垂奔苻堅,德坐免職。後遇敗,徙於長安,苻堅以為張掖太守,數歲免歸。



 及堅以兵臨江,拜德為奮威將軍。堅之敗也,堅與張夫人相失,慕容將護致之,德正色謂曰:「昔楚莊滅陳,納巫臣之諫而棄夏姬。此不祥之人,惑亂人主,戎事不邇女器,秦之敗師當由於此。宜掩目而過,奈何將衛之也!」不從,德馳馬而去之。還次滎陽,言於曰:「昔句踐棲於會稽,終獲吳國。聖人相時而
 動,百舉百全。天將悔禍,故使秦師喪敗,宜乘其弊以復社稷。」不納。乃從垂如鄴。



 及垂稱燕王,以德為車騎大將軍,復封范陽王,居中鎮衛,參斷政事。久之,遷司徙。于時慕容永據長子,有眾十萬,垂議討之。群臣咸以為疑,德進曰:「昔三祖積德,遺詠在耳,故陛下龍飛,不謀而會,雖由聖武,亦緣舊愛,燕、趙之士樂為燕臣也。今永既建偽號,扇動華戎,致令群豎從橫,逐鹿不息,宜先除之,以一眾聽。昔光武馳蘇茂之難,不顧百官之疲,夫豈不仁?機急故也。兵法有不得已而用之,陛下容得已乎!」垂笑謂其黨曰:「司徒議與吾同。二人同心,其利斷金,吾計決
 矣。」遂從之。垂臨終,敕其子寶以鄴城委德。寶既嗣位,以德為使持節、都督冀、兗、青、徐、荊、豫六州諸軍事、特進、車騎大將軍、冀州牧,領南蠻校尉,鎮鄴,罷留臺,以都督專總南夏。



 魏將拓拔章攻鄴,德遣南安王慕容青等夜擊,敗之。魏師退次新城,青等請擊之。別駕韓言卓進曰:「古人先決勝廟堂,然後攻戰。今魏不可擊者四,燕不宜動者三。魏懸軍遠入,利在野戰,一不可擊也。深入近畿,頓兵死地,二不可擊也。前鋒既敗,後陣方固,三不可擊也。彼眾我寡,四不可擊也。官軍自戰其地,一不宜動。動而不勝,眾心難固,二不宜動。城郭未修,敵來無備,三不宜動。
 此皆兵家所忌,不如深溝高壘,以逸待勞。彼千里饋糧,野無所掠,久則三軍靡資,攻則眾旅多斃,師老釁生,詳而圖之,可以捷矣。」德曰:「韓別駕之言,良、平之策也。」於是召青還師。魏又遣遼西公賀賴盧率騎與章圍鄴,德遣其參軍劉藻請救於姚興,且參母兄之問,而興師不至,眾大懼。德於是親饗戰士,厚加撫接,人感其恩,皆樂為致死。會章、盧內相乖爭,各引軍潛遁。章司馬丁建率眾來降,言章師老,可以敗之。德遣將追破章軍,人心始固。



 時魏師入中山,慕容寶出奔于薊,慕容詳又僭號。會劉藻自姚興而至,興太史令高魯遣其甥王景暉隨藻送
 玉璽一紐,并圖識祕文,曰:「有德者昌,無德者亡。德受天命,柔而復剛。」又有謠曰:「大風蓬勃揚塵埃,八井三刀卒起來,四海鼎沸中山頹,惟有德人據三臺。」於是德之群臣議以慕容詳僭號中山,魏師盛于冀州,未審寶之存亡,因勸德即尊號。德不從。會慕容達自龍城奔鄴,稱寶猶存,群議乃止。尋而寶以德為丞相,領冀州牧,承制南夏。



 德兄子麟自義臺奔鄴,因說德曰:「中山既沒,魏必乘勝攻鄴,雖糧儲素積,而城大難固,且人情沮動,不可以戰。及魏軍未至,擁眾南渡,就魯陽王和,據滑臺而聚兵積穀,伺隙而動,計之上也。魏雖拔中山,勢不久留,不過
 驅掠而返。人不樂徙,理自生變,然後振威以援之,魏則內外受敵,使戀舊之士有所依憑,廣開恩信,招集遺黎,可一舉而取之。」先是,慕容和亦勸德南徙,於是許之。隆安二年,乃率戶四萬、車二萬七千乘,自鄴將徙于滑臺。遇風,船沒,魏軍垂至,眾懼,議欲退保黎陽。其夕流澌凍合,是夜濟師,旦,魏師至而冰泮,若有神焉。遂改黎陽津為天橋津。及至滑臺,景星見于尾箕。漳水得白玉,狀若璽。於是德依燕元故事,稱元年,大赦境內殊死已下,置百官。以慕容麟為司空、領尚書令,慕容法為中軍將軍,慕輿拔為尚書左僕射,丁通為尚書右僕射,自餘封授
 各有差。初,河間有麟見,慕容麟以為已瑞。及此,潛謀為亂,事覺,賜死。其夏,魏將賀賴盧率眾附之。



 至是,慕容寶自龍城南奔至黎陽,遣其中黃門令趙思召慕容鍾來迎。鍾本首議勸德稱尊號,聞而惡之,執思付獄,馳使白狀。德謂其下曰:「卿等前以社稷大計,勸吾攝政。吾亦以嗣帝奔亡,人神曠主,故權順群議,以繫眾望。今天方悔禍,嗣帝得還,吾將具駕奉迎,謝罪行闕,然後角巾私第,卿等以為何如?」其黃門侍郎張華進曰:「夫爭奪之世,非雄才不振;從橫之時,豈懦夫能濟!陛下若蹈匹婦之仁,捨天授之業,威權一去,則身首不保,何退讓之有乎!」德
 曰:「吾以古人逆取順守,其道未足,所以中路徘徊,悵然未決耳。」慕輿護請馳問寶虛實,德流涕而遣之。乃率壯士數百,隨思而北,因謀殺寶。初,寶遣思之後,知德攝位,懼而北奔。護至無所見,執思而還。德以思閑習典故,將任之。思曰:「昔關羽見重曹公,猶不忘先主之恩。思雖刑餘賤隸,荷國寵靈,犬馬有心,而況人乎!乞還就上,以明微節。」德固留之,思怒曰:「周室衰微,晉、鄭夾輔;漢有七國之難,實賴梁王。殿下親則叔父,位則上台,不能率先群后以匡王室,而幸根本之傾為趙倫之事。思雖無申胥哭秦之效,猶慕君賓不生莽世。」德怒,斬之。



 晉南陽太守
 閭丘羨、寧朔將軍鄧啟方率眾二萬來伐,師次管城。德遣其中軍慕容法、撫軍慕容和等距之,王師敗績。德怒法不窮追晉師,斬其撫軍司馬靳瑰。



 初,苻登既為姚興所滅,登弟廣率部落降於德,拜冠軍將軍,處之乞活堡。會熒惑守東井,或言秦當復興者,廣乃自稱秦王,敗德將慕容鐘。時德始都滑臺,介於晉、魏之間,地無十城,眾不過數萬。及鐘喪師,反側之徒多歸於廣。德乃留慕容和守滑臺,親率眾討廣,斬之。



 初,寶之至黎陽也,和長史李辯勸和納之,和不從。辯懼謀泄,乃引晉軍至管城,冀德親率師,於後作亂。會德不出,愈不自安。及德此行也,
 辯又勸和反,和不從。辯怒,殺和,以滑臺降于魏。時將士家悉在城內,德將攻之,韓範言於德曰:「魏師已入城,據國成資,客主之勢,翻然復異,人情既危,不可以戰。宜先據一方,為關中之基,然後畜力而圖之,計之上也。」德乃止。德右衛將軍慕容雲斬李辯,率將士家累二萬餘人而出,三軍慶悅。德謀於眾曰:「苻廣雖平,而撫軍失據,進有彊敵,退無所托,計將安出?」張華進曰:「彭城阻帶山川,楚之舊都,地險人殷,可攻而據之,以為基本。」慕容鐘、慕輿護、封逞、韓言卓等固勸攻滑臺,潘聰曰:「滑臺四通八達,非帝王之居。且北通大魏,西接彊秦,此二國者,未可以
 高枕而待之。彭城土曠人稀,地平無險,晉之歸鎮,必距王師。又密邇江、淮,水路通浚,秋夏霖潦,千里為湖。且水戰國之所短,吳之所長,今雖剋之,非久安之計也。青、齊沃壤,號曰東秦,土方二千,戶餘十萬,四塞之固,負海之饒,可謂用武之國。三齊英傑,蓄志以待,孰不思得明主以立尺寸之功!廣固者,曹嶷之所營,山川阻峻,足為帝王之都。宜遣辯士馳說于前,大兵繼進于後,避閭渾昔負國恩,必翻然向化。如其守迷不順,大軍臨之,自然瓦解。既據之後,閉關養銳,伺隙而動,此亦二漢之有關中、河內也。」德猶豫未決。沙門郎公素知占候,德因訪其所
 適。郎曰:「敬覽三策,潘尚書之議可謂興邦之術矣。今歲初,長星起於奎婁,遂掃虛危,而虛危,齊之分野,除舊布新之象。宜先定舊魯,巡撫瑯邪,待秋風戒節,然後北圍臨齊,天之道也。」德大悅,引師而南,兗州北鄙諸縣悉降,置守宰以撫之。存問高年,軍無私掠,百姓安之,牛酒屬路。



 德遣使喻齊郡太守避閭渾,渾不從,遣慕容鐘率步騎二萬擊之。德進據瑯邪,徐、兗之土附者十餘萬,自瑯邪而北,迎者四萬餘人。德進寇莒城,守將任安委城而遁,以潘聰鎮莒城。鐘傳檄青州諸郡曰:「隆替有時,義列昔經;困難啟聖,事彰中籙。是以宣王龍飛於危周,光武
 鳳起於絕漢,斯蓋歷數大期,帝王之興廢也。自我永康多難,長鯨逸網,華夏四分,黎元五裂。逆賊辟閭渾父蔚,昔同段龕阻亂淄川,太宰東征,剿絕凶命。渾於覆巢之下,蒙全卵之施,曾微犬馬識養之心,復襲凶父樂禍之志,盜據東秦,遠附吳、越,割剝黎元,委輸南海。皇上應期,大命再集,矜彼營丘,暫阻王略,故以七州之眾二十餘萬,巡省貸宗,問罪齊、魯。昔韓信以裨將伐齊,有征無戰;耿弇以偏軍討步,剋不移朔。況以萬乘之師,掃一隅之寇,傾山碎卵,方之非易。孤以不才,忝荷先驅,都督元戎一十二萬,皆烏丸突騎,三河猛士,奮劍與夕火爭光,揮
 戈與秋月競色。以此攻城,何城不克;以此眾戰,何敵不平!昔竇融以河西歸漢,榮被於後裔;彭寵盜逆漁陽,身死於奴僕。近則曹嶷跋扈,見擒於後趙;段龕干紀,取滅於前朝。此非古今之吉凶,已然之成敗乎?渾若先迷後悟,榮寵有加。如其敢抗王師,敗滅必無遣燼。稷下之雄,岱北之士,有能斬送渾者,賞同佐命。脫履機不發,必玉石俱摧。」渾聞德軍將至,從八千餘家入廣固。諸郡皆承檄降於德。渾懼,將妻子奔于魏。德遣射聲校尉劉綱追斬於莒城。渾參軍張瑛常與渾作檄,辭多不遜。及此,德擒而讓之。瑛神色自若,徐對曰:「渾之有臣,猶韓信之有
 蒯通。通遇漢祖而蒙恕,臣遭陛下而嬰戮,比之古人,竊為不幸。防風之誅,臣實甘之,但恐堯、舜之化未弘於四海耳。」德初善其言,後竟殺之。德遂入廣固。



 四年,僭即皇帝位于南郊,大赦,改元為建平,設行廟於宮南,遣使奉策告成焉。進慕容鐘為司徒,慕輿拔為司空,封孚為左僕射,慕輿護為右僕射。遣其度支尚書封愷、中書侍郎封逞觀省風俗,所在大饗將士。以其妻段氏為皇后。建立學官,簡公卿已下子弟及二品士門二百人為太學生。



 後因宴其群臣,酒酣,笑而言曰:「朕雖寡薄,恭己南面而朝諸侯,在上不驕,夕惕於位,可方自古何等主也?」其
 青州刺史鞠仲曰:「陛下中興之聖后,少康、光武之儔也。」德顧命左右賜仲帛千匹。仲以賜多為讓,德曰:「卿知調朕,朕不知調卿乎!卿飾對非實,故亦以虛言相賞,賞不謬加,何足謝也!」韓範進曰:「臣聞天子無戲言,忠臣無妄對。今日之論,上下相欺,可謂君臣俱失。」德大悅,賜範絹五十匹。自是昌言競進,朝多直士矣。



 德母兄先在長安,遣平原人杜弘如長安問存否,弘曰:「臣至長安,若不奉太后動止,便即西如張掖,以死為效。臣父雄年踰六十,未沾榮貴,乞本縣之祿,以申烏鳥之情。」張華進曰:「杜弘未行而求祿,要利情深,不可使也。」德曰:「吾方散所輕之
 財,招所重之死,況為親尊而可吝乎!且弘為君迎親,為父求祿,雖外如要利,內實忠孝。」乃以雄為平原令。弘至張掖,為盜所殺,德聞而悲之,厚撫其妻子。



 明年,德如齊城,登營丘,望晏嬰塚,顧謂左右曰:「禮,大夫不逼城葬。平仲古之賢人,達禮者也,而生居近市,死葬近城,豈有意乎?」青州秀才晏謨對曰:「孔子稱臣先人平仲賢,則賢矣。豈不知高其梁,豐其禮?蓋政在家門,故儉以矯世。存居湫隘,卒豈擇地而葬乎!所以不遠門者,猶冀悟平生意也。」遂以謨從至漢城陽景王廟,宴庶老于申池,北登社首山,東望鼎足,因目牛山而歎曰:「古無不死!」愴然有終
 焉之志。遂問謨以齊之山川丘陵,賢哲舊事。謨歷對詳辯,畫地成圖。德深嘉之,拜尚書郎。立冶於商山,置鹽官於烏常澤,以廣軍國之用。



 德故吏趙融自長安來,始具母兄凶問,德號慟吐血,因而寢疾。其司隸校尉慕容達因此謀反,遣牙門皇璆率眾攻端門,殿中師侯赤眉開門應之。中黃門遜進扶德踰城,隱於進舍。段宏等聞宮中有變,勒兵屯四門。德入宮,誅赤眉等,達懼而奔魏。慕容法及魏師戰于濟北之摽榆俗,魏師敗績。



 其尚書韓言卓上疏曰:「二寇逋誅,國恥未雪,關西為豺鋃之藪,楊越為鴟鴞之林,三京社稷,鞠為丘墟,四祖園陵,蕪而不守,
 豈非義夫憤歎之日,烈士忘身之秋。而皇室多難,威略未振,是使長蛇弗翦,封豕假息。人懷憤慨,常謂一日之安不可以永久,終朝之逸無卒歲之憂。陛下中興大業,務在遵養,矜遷萌之失土,假長復而不役,愍黎庶之息肩,貴因循而不擾。斯可以保寧于營丘,難以經措于秦、越。今群凶僭逆,實繁有徒,據我三方,伺國瑕釁。深宜審量虛實,大校成敗,養兵厲甲,廣農積糧,進為雪恥討寇之資,退為山河萬全之固。而百姓因秦、晉之弊,迭相陰憲,或百室合戶,或千丁共籍,依託城社,不懼燻燒,公避課役,擅為姦宄,損風毀憲,法所不容,但檢今未宣,弗可
 加戮。今宜隱實黎萌,正其編貫,庶上增皇朝理物之明,下益軍國兵資之用。若蒙採納,冀裨山海,雖遇商鞅之刑,悅綰之害,所不辭也。」德納之,遣其車騎將軍慕容鎮率騎三千,緣邊嚴防,備百姓逃竄。以言卓為使持節、散騎常侍、行臺尚書,巡郡縣隱實,得蔭戶五萬八千。言卓公廉正直,所在野次,人不擾焉。



 德大集諸生,親臨策試。既而饗宴,乘高遠矚,顧謂其尚書魯邃曰:「齊、魯固多君子,當昔全盛之時,接、慎、巴生、淳于、鄒、田之徒,蔭修簷,臨清沼,馳朱輪,佩長劍,恣非馬之雄辭,奮談天之逸辯,指麾則紅紫成章,俯仰則丘陵生韻,至於今日,荒草頹墳,氣消
 煙滅,永言千載,能不依然!」邃答曰:「武王封比干之墓,漢祖祭信陵之墳,皆留心賢哲,每懷往事。陛下慈深二主,澤被九泉,若使彼而有知,寧不銜荷矣。」



 先是,妖賊王始聚眾於太山,自稱太平皇帝,號其父為太上皇,兄為征東將軍,弟征西將軍。慕容鎮討擒之,斬於都市。臨刑,或問其父及兄弟所在,始答曰:「太上皇帝蒙塵於外,征東、征西亂兵所害。惟朕一身,獨無聊賴。」其妻怒之曰:「止坐此口,以至於此,奈何復爾!」始曰:「皇后!自古豈有不破之家,不亡之國邪!」行刑者以刀環築之,仰視曰:「崩即崩矣,終不改帝號。」德聞而哂之。



 時桓玄將行篡逆,誅不附己
 者。冀州刺史劉軌、襄城太守司馬休之、征虜將軍劉敬宣、廣陵相高雅之、江都長張誕並內不自安,皆奔於德。於是德中書侍郎韓範上疏曰:「夫帝王之道,必崇經略。有其時無其人,則弘濟之功闕;有其人無其時,則英武之志不申。至於能成王業者,惟人時合也。自晉國內難,七載于茲。桓玄逆篡,虐踰董卓,神怒人怨,其殃積矣。可乘之機,莫過此也。以陛下之神武,經而緯之,驅樂奮之卒,接厭亂之機,譬猶聲發響應,形動影隨,未足比其易也。且江、淮南北戶口未幾,公私戎馬不過數百,守備之事蓋亦微矣。若以步騎一萬,建雷霆之舉,卷甲長驅,指
 臨江、會,必望旗草偃,壺漿屬路。跨地數千,眾踰十萬,可以西並彊秦,北抗大魏。夫欲拓境開疆,保寧社稷,無過今也。如使後機失會,豪桀復起,梟除桓玄,布惟新之化,遐邇既寧,物無異望,非但建鄴難屠,江北亦不可冀。機過患生,憂必至矣。天與不取,悔將及焉。惟陛下覽之。」德曰:「自頃數纏百六,宏綱暫弛,遂令姦逆亂華,舊京墟穢,每尋否運,憤慨兼懷。昔少康以一旅之眾,復夏配天,況朕據三齊之地,藉五州之眾,教之以軍旅,訓之以禮讓,上下知義,人思自奮,繕甲待釁,為日久矣。但欲先定中原,掃除逋孽,然後宣布淳風,經理九服,飲馬長江,懸旌
 隴阪。此志未遂,且韜戈耳。今者之事,王公其詳議之。」咸以桓玄新得志,未可圖,乃止。於是講武於城西,步兵三十七萬,車一萬七千乘,鐵騎五萬三千,周亙山澤,旌旗彌漫,鉦鼓之聲,振動天地。德登高望之,顧謂劉軌、高雅之曰:「昔郤克仇齊,子胥怨楚,終能暢其剛烈,名流千載。卿等既知投身有道,當使無慚昔人也。」雅之等頓首答曰:「幸蒙陛下天覆之恩,大造之澤,存亡繼絕,實在聖時,雖則萬隕,何以上報!」俄聞桓玄敗,德以慕容鎮為前鋒,慕容鐘為大都督,配以步卒二萬,騎五千,剋期將發,而德寢疾,於是罷兵。



 初,德迎其兄子超於長安,及是而至。
 德夜夢其父曰:「汝既無子,何不早立超為太子,不爾,惡人生心。」寐而告其妻曰:「先帝神明所敕,觀此夢意,吾將死矣。」乃下書以超為皇太子,大赦境內,子為父後者人爵二級。其月死,即義熙元年也,時年七十。乃夜為十餘棺,分出四門,潛葬山谷,竟不知其尸之所在。在位五年。偽謚獻武皇帝。



\end{pinyinscope}