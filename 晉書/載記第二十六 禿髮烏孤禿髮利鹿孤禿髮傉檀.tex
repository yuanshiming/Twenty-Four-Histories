\article{載記第二十六 禿髮烏孤禿髮利鹿孤禿髮傉檀}

\begin{pinyinscope}

 禿髮烏孤禿髮利鹿孤禿髮傉檀



 禿髮烏孤,河西鮮卑人也。其先與後魏同出。八世祖匹孤率其部自塞北遷于河西,其地東至麥田、牽屯,西至濕羅,南至澆河,北接大漠。匹孤卒,子壽闐立。初,壽闐之在孕,母胡掖氏因寢而產於被中,鮮卑謂被為「禿髮」,因而氏焉。壽闐卒,孫樹機能立,壯果多謀略。泰始中,殺秦州刺史胡烈於萬斛堆,敗涼州刺史蘇愉于金山,盡有
 涼州之地,武帝為之旰食。後為馬隆所敗,部下殺之以降。從弟務丸立。死,孫推斤立。死,子思復鞬立,部眾稍盛。烏孤即思復鞬之子也。及嗣位,務農桑,修鄰好。呂光遣使署為假節、冠軍大將軍、河西鮮卑大都統、廣武縣侯。烏孤謂諸將曰:「呂氏遠來假授,當可受不?」眾咸曰:「吾士眾不少,何故屬人!」烏孤將從之,其將石真若留曰:「今本根未固,理宜隨時。光德刑修明,境內無虞,若致死於我者,大小不敵,後雖悔之,無所及也。不如受而遵養之,又待其釁耳。」烏孤乃受之。



 烏孤討乙弗、折掘二部,大破之,遣其將石亦干築廉川堡以都之。烏孤登廉川大山,泣
 而不言。石亦干進曰:「臣聞主憂臣辱,主辱臣死,大王所為不樂者,將非呂光乎!光年已衰老,師徒屢敗。今我以士馬之盛,保據大川,乃可以一擊百,光何足懼也。」烏孤曰:「光之衰老,亦吾所知。但我祖宗以德懷遠,殊俗憚威,盧陵、契汗萬里委順。及吾承業,諸部背叛,邇既乖違,遠何以附,所以泣耳。」其將苻渾曰:「大王何不振旅誓眾,以討其罪。」烏孤從之,大破諸部。呂光封烏孤廣武郡公。又討意云鮮卑,大破之。



 光又遣使署烏孤征南大將軍、益州牧、左賢王。烏孤謂使者曰:「呂王昔以專征之威,遂有此州,不能以德柔遠,惠安黎庶。諸子貪淫,三甥肆暴,郡
 縣土崩,下無生賴。吾安可違天下之心,受不義之爵!帝王之起,豈有常哉!無道則滅,有德則昌,吾將順天人之望,為天下主。」留其鼓吹羽儀,謝其使而遣之。



 隆安元年,自稱大都督、大將軍、大單于、西平王,赦其境內,年號太初。曜兵廣武,攻剋金城。光遣將軍竇茍來伐,戰于街亭,大敗之。降光樂都、湟河、澆河三郡,嶺南羌胡數萬落皆附之。光將楊軌、王乞基率戶數千來奔。烏孤更稱武威王。後三歲,徙于樂都,署弟利鹿孤為驃騎大將軍、西平公,鎮安夷,傉檀為車騎大將軍、廣武公,鎮西平。以楊軌為賓客。金石生、時連珍,四夷之豪雋;陰訓、郭倖,西州之
 德望;楊統、楊貞、衛殷、麴丞明、郭黃、郭奮、史暠、鹿嵩,文武之秀傑;梁昶、韓疋、張昶、郭韶,中州之才令;金樹、薛翹、趙振、王忠、趙晁、蘇霸,秦雍之世門,皆內居顯位,外宰郡縣。官方授才,咸得其所。



 烏孤從容謂其群下曰:「隴右區區數郡地耳!因其兵亂,分裂遂至十餘。乾歸擅命河南,段業阻兵張掖,虐氐假息,偷據姑臧。吾藉父兄遣烈。思郭清西夏。兼弱攻昧,三者何先?」楊統進曰:「乾歸本我所部,終必歸服。段業儒生,才非經世,權臣擅命,制不由已,千里伐人,糧運懸絕,且與我鄰好,許以分災共患,乘其危弊,非義舉也。呂光衰老,嗣紹沖闇,二子纂、弘,雖頗有文
 武,而內相猜忌。若天威臨之,必應鋒瓦解。宜遣車騎鎮浩亹,鎮北據廉川,乘虛迭出,多方以誤之,救右則擊其左,救左則擊其右,使纂疲於奔命,人不得安其農業。兼弱攻昧,於是乎在,不出二年,可以坐定姑臧。姑臧既拔,二寇不待兵戈,自然服矣。」烏孤然之,遂陰有吞並之志。



 段業為呂纂所侵,遣利鹿孤救之。纂懼,燒氐池、張掖穀麥而還。以利鹿孤為涼州牧,鎮西平,追傉檀入錄府國事。



 是歲,烏孤因酒墜馬傷脅,笑曰:「幾使呂光父子大喜。」俄而患甚,顧謂群下曰:「方難未靜,宜立長君。」言終而死,在王位三年,偽謚武王,廟號烈祖。弟利鹿孤立。



 利鹿孤以隆安三年即偽位,赦其境內殊死已下,又徙居于西平。使記室監麴梁明聘于段業。業曰:「貴主先王創業啟運,功高先世,宜為國之太祖,有子何以不立?」梁明曰:「有子羌奴,先王之命也。」業曰:「昔成王弱齡,周召作宰;漢昭八歲,金、霍夾輔。雖嗣子沖幼,而二叔休明,左提右挈,不亦可乎?」明曰:「宋宣能以國讓,《春秋》美之;孫伯符委事仲謀,終開有吳之業。且兄終弟及,殷湯之制也,亦聖人之格言,萬代之通式,何必胤已為是,紹兄為非。」業曰:「美哉!使乎之義也。」



 利鹿孤聞呂光死,遣其將金樹、蘇翹率騎五千屯于昌松漠口。



 既逾年,赦其境內,改元曰
 建和。二千石長吏清高有惠化者,皆封亭侯、關內侯。



 呂纂來伐,使傉檀距之。纂士卒精銳,進度三堆,三軍擾懼。傉檀下馬據胡床而坐,士眾心乃始安。與纂戰,敗之,斬二千餘級。纂西擊段業,傉檀率騎一萬,乘虛襲姑臧。纂弟緯守南北城以自固。傉檀置酒於朱明門上,鳴鐘鼓以饗將士,耀兵于青陽門,虜八千餘戶而歸。



 乞伏乾歸為姚興所敗,率騎數百來奔,處之晉興,待以上賓之禮。乾歸遣子謙等質于西平。鎮北將軍俱延言於利鹿孤曰:「乾歸本我之屬國,妄自尊立,理窮歸命,非有款誠;若奔東秦,必引師西侵,非我利也。宜徙於乙弗之間,防其
 越逸之路。」利鹿孤曰:「吾方弘信義以收天下之心,乾歸投誠而徙之,四海將謂我不可以誠信託也。」俄而乾歸果奔于姚興。利鹿孤謂延曰:「不用卿言,乾歸果叛,卿為吾行也。」延追乾歸至河,不及而還。



 利鹿孤立二年,龍見於長寧,麒麟游于綏羌,於是群臣勸進,以隆安五年僭稱河西王。其將鍮勿崙進曰:「昔我先君肇自幽、朔,被髮左衽,無冠冕之義,遷徙不常,無城邑之制,用能中分天下,威振殊境。今建大號,誠順天心。然寧居樂士,非貽厥之規;倉府粟帛,生敵人之志。且首兵始號,事必無成,陳勝、項籍,前鑒不遠。宜置晉人於諸城,勸課農桑,以供軍
 國之用,我則習戰法以誅未賓。若東西有變,長算以縻之;如其敵強於我,徙而以避其鋒,不亦善乎!」利鹿孤然其言。



 於是率師伐呂隆,大敗之,獲其右僕射楊桓。傉檀謂之曰:「安寢危邦,不思擇木,老為囚虜,豈曰智也!」桓曰:「受呂氏厚恩,位忝端貳,雖洪水滔天,猶欲濟彼俱溺,實恥為叛臣以見明主。」傉檀曰:「卿忠臣也!」以為左司馬。



 利鹿孤謂其群下曰:「吾無經濟之才,忝承業統,自負乘在位,三載于茲。雖夙夜惟寅,思弘道化,而刑政未能允中,風俗尚多凋弊;戎車屢駕,無闢境之功;務進賢彥,而下猶蓄滯。豈所任非才,將吾不明所致也?二三君子其極
 言無諱,吾將覽焉。」祠部郎中史暠對曰:「古之王者,行師以全軍為上,破國次之,拯溺救焚,東征西怨。今不以綏寧為先,惟以徙戶為務,安土重遷,故有離叛,所以斬將剋城,土不加廣。今取士拔才,必先弓馬,文章學藝為無用之條,非所以來遠人,垂不朽也。孔子曰:『不學禮,無以立。』宜建學校,開庠序,選耆德碩儒以訓胄子。」利鹿孤善之,於是以田玄沖、趙誕為博士祭酒,以教胄子。



 時利鹿孤雖僭位,尚臣姚興。楊桓兄經佐命姚萇,早死,興聞桓有德望,征之。利鹿孤餞桓于城東,謂之曰:「本期與卿共成大業,事乖本圖,分歧之感,實情深古人。但鯤非溟海,
 無以運其軀;鳳非脩梧,無以晞其翼。卿有佐時之器,夜光之寶,當振纓雲閣,耀價連城,區區河右,未足以逞卿才力。善勖日新,以成大美。」桓泣曰:「臣往事呂氏,情節不建。陛下宥臣於俘虜之中,顯同賢舊,每希攀龍附風,立尺寸之功,龍門既開,而臣違離,公衡之戀,豈曰忘之!」利鹿孤為之流涕。



 遣傉檀又攻呂隆昌松太守孟禕于顯美,剋之。傉檀執禕而數之曰:「見機而作,賞之所先;守迷不變,刑之所及。吾方耀威玉門,掃平秦、隴,卿固守窮城,稽淹王憲,國有常刑,於分甘乎?」禕曰:「明公開翦河右,聲播宇內,文德以綏遠人,威武以懲不恪,況禕蔑爾,敢距
 天命!釁鼓之刑,禕之分也。但忠於彼者,亦忠於此。荷呂氏厚恩,受籓屏之任,明公至而歸命,恐獲罪於執事,惟公圖之。」傉檀大悅,釋其縛,待之客禮。徙顯美、麗靬二千餘戶而歸。嘉禕忠烈,拜左司馬。禕請曰:「呂氏將亡,聖朝之并河右,昭然已定。但為人守而不全,復忝顯任,竊所未安。明公之恩,聽禕就戮於姑臧,死且不朽。」傉檀義而許之。



 呂隆為沮渠蒙遜所伐,遣使乞師,利鹿孤引群下議之。尚書左丞婆衍崙曰:「今姑臧饑荒殘弊,穀石萬錢,野無青草,資食無取。蒙遜千里行師,糧運不屬,使二寇相殘,以乘其釁。若蒙遜拔姑臧,亦不能守,適可為吾取
 之,不宜救也。」傉檀曰:「崙知其一,未知其二。姑臧今雖虛弊,地居形勝,可西一都之會,不可使蒙遜據之,宜在速救。」利鹿孤曰:「車騎之言,吾之心也。」遂遣傉檀率騎一萬救之。至昌松而蒙遜已退,傉檀徙涼澤、段冢五百餘家而歸。



 利鹿孤寢疾,令曰:「內外多虞,國機務廣,其令車騎嗣業,以成先王之志。」在位三年而死,葬于西平之東南,偽謚曰康王。弟傉檀嗣。



 傉檀少機警,有才略。其父奇之,謂諸子曰:「傉檀明識幹藝,非汝等輩也。」是以諸兄不以授子,欲傳之於傉檀。及利鹿孤即位,垂拱而已,軍國大事皆以委之。以元興元
 年僭號涼王,遷于樂都,改元曰弘昌。



 初,乞伏乾歸之在晉興也,以世子熾磐為質。後熾磐逃歸,為追騎所執,利鹿孤命殺之。傉檀曰:「臣子逃歸君父,振古通義,故魏武善關羽之奔,秦昭恕頃襄之逝。熾磐雖逃叛,孝心可嘉,宜垂全宥,以弘海岳之量。」乃赦之。至是,熾磐又奔允街,傉檀歸其妻子。



 姚興遣使拜傉檀車騎將軍、廣武公。傉檀大城樂都。姚興遣將齊難率眾迎呂隆于姑臧,傉檀攝昌松、魏安二戍以避之。



 興涼州刺史王尚遣主薄宗敞來聘。敞父燮,呂光時自湟河太守入為尚書郎,見傉檀于廣武,執其手曰:「君神爽宏拔,逸氣陵雲,命世之傑
 也,必當剋清世難。恨吾年老不及見耳,以敞兄弟託君。」至是,傉檀謂敞曰:「孤以常才,謬為尊先君所見稱,每自恐有累大人水鏡之明。及忝家業,竊有懷君子。《詩》云:『中心藏之,何日忘之。』不圖今日得見卿也。」敞曰:「大王仁侔魏祖,存念先人,雖朱暉眄張堪之孤,叔向撫汝齊之子,無以加也。」酒酣,語及平生。傉檀曰:「卿魯子敬之儔,恨不與卿共成大業耳。」



 傉檀以姚興之盛,又密圖姑臧,乃去其年號,罷尚書丞郎官,遣參軍關尚聘于興。興謂尚曰:「車騎投誠獻款,為國籓屏,擅興兵眾,輒造大城,為臣之道固若是乎?」尚曰:「王侯設險以自固,先王之制也,所以
 安人衛眾,預備不虞。車騎僻在遐籓,密邇勍寇,南則逆羌未賓,西則蒙遜跋扈,蓋為國家重門之防,不圖陛下忽以為嫌。」興笑曰:「卿言是也。」



 傉檀遣其將文支討南羌、西虜,大破之。上表姚興,求涼州,不許,加傉檀散騎常侍,增邑二千戶。傉檀於是率師伐沮渠蒙遜,次于氐池。蒙遜嬰城固守,芟其禾苗,至於赤泉而還。獻興馬三千匹,羊三萬頭。興乃署傉檀為使持節、都督河右諸軍事、車騎大將軍、領護匈奴中郎將、涼州刺史,常侍、公如故,鎮姑臧。傉檀率步騎三萬次于五澗,興涼州刺史王尚遣辛晁、孟禕、彭敏出迎。尚出自清陽門,鎮南文支入自涼
 風門。宗敞以別駕送尚還長安,傉檀曰:「吾得涼州三千餘家,情之所寄,唯卿一人,奈何捨我去乎?」敞曰:「今送舊君,所以忠於殿下。」傉檀曰:「吾今新牧貴州,懷遠安邇之略,為之若何?」敞曰:「涼土雖弊,形勝之地,道由人弘,實在殿下。段懿、孟禕、武威之宿望;辛晁、彭敏,秦、隴之冠冕;斐敏、馬輔,中州之令族;張昶,涼國之舊胤;張穆、邊憲、文齊、楊班、梁崧、趙昌,武同飛、羽。以大王之神略,撫之以威信,農戰並脩,文教兼設,可以從橫於天下,河右豈足定乎!」傉檀大悅,賜敞馬二十匹。於是大饗文武於謙光殿,班賜金馬各有差。



 遣西曹從事史暠聘于姚興。興謂暠曰:「
 車騎坐定涼州,衣錦本國,其德我乎?」暠曰:「車騎積德河西,少播英問,王威未接,投誠萬里,陛下官方任才,量功授職,彝倫之常,何德之有!」興曰:「朕不以州授車騎者,車騎何從得之。」暠曰:「使河西雲擾、呂氏顛狽者,實由車騎兄弟傾其根本。陛下雖鴻羅遐被,涼州猶在天網之外。故征西以周、召之重,力屈姑臧;齊難以王旅之盛,勢挫張掖。王尚孤城獨守,外逼群狄,陛下不連兵十年,殫竭中國,涼州未易取也。今以虛名假人,內收大利,乃知妙算自天,聖與道合,雖云遷授,蓋亦時宜。」興悅其言,拜騎都尉。



 傉檀宴群僚于宣德堂,仰視而歎曰:「古人言作者不居,
 居者不作,信矣。」孟禕進曰:「張文王築城苑,繕宗廟,為貽厥之資,萬世之業,秦師濟河,漼然瓦解。梁熙據全州之地,擁十萬之眾,軍敗於酒泉,身死於彭濟。呂氏以排山之勢,王有西夏,率土崩離,銜璧秦、雍。寬饒有言:『富貴無常,忽輒易人。』此堂之建,年垂百載,十有二主,唯信順可以久安,仁義可以永固,願大王勉之。」傉檀曰:「非君無以聞讜言也。」傉檀雖受制於姚興,然車服禮章一如王者。以宗敞為太府主簿、錄記室事。



 傉檀偽游澆河,襲徙西平、湟河諸羌三萬餘戶于武興、番禾、武威、昌松四郡。徵集戎夏之兵五萬餘人,大閱于方亭,遂伐沮渠蒙遜,入
 西陜。蒙遜率眾來距,戰于均石,為蒙遜所敗。傉檀率騎二萬,運穀四萬石以給西郡。蒙遜攻西郡,陷之。其後傉檀又與赫連勃勃戰于陽武,為勃勃所敗,將佐死者十餘人,傉檀與數騎奔南山,幾為追騎所得。傉檀懼東西寇至,徙三百里內百姓入于姑臧,國中駭怨。屠各成七兒因百姓之擾也,率其屬三百人,叛傉檀於北城。推梁貴為盟主,貴閉門不應。一夜眾至數千。殿中都尉張猛大言於眾曰:「主上陽武之敗,蓋恃眾故也。責躬悔過,明君之義,諸君何故從此小人作不義之事!殿內武旅正爾相尋,目前之危,悔將無及。」眾聞之,咸散。七兒奔晏然,
 殿中騎將白路等追斬之。軍諮祭酒梁裒、輔國司馬邊憲等七人謀反,傉檀悉誅之。



 姚興以傉檀外有陽武之敗,內有邊、梁之亂,遣其尚書郎韋宗來觀釁。傉檀與宗論六國從橫之規,三家戰爭之略,遠言天命廢興,近陳人事成敗,機變無窮,辭致清辯。宗出而歎曰:「命世大才、經綸名教者,不必華宗夏士;撥煩理亂、澄氣濟世者,亦未必《八索》、《九丘》。五經之外,冠冕之表,復自有人。車騎神機秀發,信一代之偉人,由余、日磾豈足為多也!」宗還長安,言於興曰:「涼州雖殘弊之後,風化未頹,傉檀權詐多方,憑山河之固,未可圖也。」興曰:「勃勃以烏合之眾尚能
 破之,吾以天下之兵,何足剋也!」宗曰:「形移勢變,終始殊途,陵人者易敗,自守者難攻。陽武之役,傉檀以輕勃勃致敗。今以大軍臨之,必自固求全,臣竊料群臣無傉檀匹也。雖以天威臨之,未見其利。」興不從,乃遣其將姚弼及斂成等率步騎三萬來伐,又使其將姚顯為弼等後繼,遺傉檀書云「遣尚書左僕射齊難討勃勃,懼其西逸,故令弼等於河西邀之。」傉檀以為然,遂不設備。弼眾至漠口,昌松太守蘇霸嬰城固守,弼喻霸令降,霸曰:「汝違負盟誓,伐委順之籓,天地有靈,將不祐汝!吾寧為涼鬼,何降之有!」城陷,斬霸。弼至姑臧,屯于西苑。州人王鐘、宋
 鐘、王娥等密為內應,候人執其使送之。傉檀欲誅其元首,前軍伊力延侯曰:「今強敵在外,內有姦豎,兵交勢踧,禍難不輕,宜悉坑之以安內外。」傉檀從之,殺五千餘人,以婦女為軍賞。命諸郡縣悉驅牛羊於野,斂成縱兵虜掠。傉檀遣其鎮北俱延、鎮軍敬歸等十將率騎分擊,大敗之,斬首七千餘級。姚弼固壘不出,傉檀攻之未剋,乃斷水上流,欲以持久斃之。會雨甚,堰壞,弼軍乃振。姚顯聞弼敗,兼道赴之,軍勢甚盛。遣射將孟欽等五人挑戰于涼風門,弦未及發,材官將軍宋益等馳擊斬之。顯乃委罪斂成。遣使謝傉檀,引師而歸。



 傉檀於是僭即涼王
 位,赦其境內,改年為嘉平,置百官。立夫人折掘氏為五后,世子武臺為太子、錄尚書事,左長史趙晁、右長史郭倖為尚書左右僕射,鎮北俱延為太尉,鎮軍敬歸為司隸校尉,自餘封署各有差。



 遣其左將軍枯木、駙馬都尉胡康伐沮渠蒙遜,掠臨松人千餘戶而還。蒙遜大怒,率騎五千至于顯美方亭,破車蓋鮮卑而還。俱延又伐蒙遜,大敗而歸。傉檀將親率眾伐蒙遜,趙晁及太史令景保諫曰:「今太白未出,歲星在西,宜以自守,難以伐人。比年天文錯亂,風霧不時,唯修德責躬可以寧吉。」傉檀曰:「蒙遜往年無狀,入我封畿,掠我邊疆,殘我禾稼。吾蓄力
 待時,將報東門之恥,今大軍已集,卿欲沮眾邪?」保曰:「陛下不以臣不肖,使臣主察乾象,若見事不言,非為臣之體。天文顯然,動必無利。」傉檀曰:「吾以輕騎五萬伐之,蒙遜若以騎兵距我,則眾寡不敵;兼步而來,則舒疾不同;救右則擊其左,赴前則攻其後,終不與之交兵接戰,卿何懼乎?」保曰:「天文不虛,必將有變。」傉檀怒,鎖保而行,曰:「有功當殺汝以徇,無功封汝百戶侯,」既而蒙遜率眾來距,戰於窮泉,傉檀大敗,單馬奔還。景保為蒙遜所擒,讓之曰:「卿明於天文,為彼國所任,違天犯順,智安在乎?」保曰:「臣匪為無智,但言而不從。」蒙遜曰:「昔漢祖困于平
 城,以婁敬為功;袁紹敗於官渡,而田豐為戮。卿策同二子,貴主未可量也。卿必有婁敬之賞者,吾今放卿,但恐有田豐之禍耳。」保曰:「寡君雖才非漢祖,猶不同本初,正可不得封侯,豈慮禍也。」蒙遜乃免之。至姑臧,傉檀謝之曰:「卿,孤之蓍龜也,而不能從之,孤之深罪。」封保安亭侯。



 蒙遜進圖姑臧,百姓懲東苑之戮,悉皆驚散。壘掘、麥田、車蓋諸部盡降于蒙遜。傉檀遣使請和,蒙遜許之,乃遣司隸校尉敬歸及子他為質,歸至胡坑,逃還,他為追兵所執。蒙遜徙其眾八千餘戶而歸。右衛折掘奇鎮據石驢山以叛。傉檀懼為蒙遜所滅,又慮奇鎮剋嶺南,乃遷於
 樂都,留大司農成公緒守姑臧。傉檀始出城,焦諶、王侯等閉門作難,收合三千餘家,保據南城。諶推焦朗為大都督、龍驤大將軍,諶為涼州刺史,降于蒙遜。鎮軍敬歸討奇鎮於石驢山,戰敗,死之。



 蒙遜因剋姑臧之威來伐,傉檀遣其安北段茍、左將軍雲連乘虛出番禾以襲其後,徙三千餘家於西平。蒙遜圍樂都,三旬不剋,遣使謂傉檀曰:「若以寵子為質,我當還師。」傉檀曰:「去否任卿兵勢。卿違盟無信,何質以供!」蒙遜怒,築室返耕,為持久之計。群臣固請,乃以子安周為質。蒙遜引歸。



 吐谷渾樹洛干率眾來伐,傉檀遣其太子武臺距之,為洛干所敗。



 傉檀又將伐蒙遜,邯川護軍孟愷諫曰:「蒙遜初并姑臧,凶勢甚盛,宜固守伺隙,不可妄動。」不從。五道俱進,至番禾、苕藋,掠五千餘戶。其將屈右進曰:「陛下轉戰千里,前元完陣,徙戶資財,盈溢衢路,宜倍道旋師,早度峻險。蒙遜善於用兵,士眾習戰,若輕軍卒至,出吾慮表,大敵外逼,徙戶內攻,危之道也。」衛尉伊力延曰:「我軍勢方盛,將士勇氣自倍,彼徒我騎,勢不相及,若倍道旋師,必捐棄資財,示人以弱,非計也。」屈右出而告其諸弟曰:「吾言不用,天命也。此吾兄弟死地。」俄而昏霧風雨,蒙遜軍大至,傉檀敗績而還。蒙遜進圍樂都,傉檀嬰城固守,以子染
 干為質,蒙遜乃歸,久之,遣安西紇勃耀兵西境。蒙遜侵西平,徙戶掠牛馬而還。



 邯川護軍孟愷表鎮南、湟河太守文支荒酒愎諫,不血阜政事。傉檀謂伊力延曰:「今州土傾覆,所杖者文支而已,將若之何?」延曰:「宜召而訓之,使改往脩來。」傉檀乃召文支,既到,讓之曰:「二兄英姿早世,吾以不才嗣統,不能負荷大業,顛狽如是,胡顏視世,雖存若隕。庶憑子鮮存衛,藉文種復吳,卿之謂也。聞卿唯酒是耽,荒廢庶事。吾年已老,卿復若斯,祖宗之業將誰寄也。」文支頓首陳謝。



 邯川人衛章等謀殺孟愷,南啟乞伏熾磐。郭越止之曰:「孟尹寬以惠下,何罪而殺之!吾寧違
 眾而死,不負君以生。」乃密告之愷,誘章等飲酒,殺四十餘人。愷懼熾磐軍之至,馳告文支,文支遣將軍匹珍赴之。熾磐軍到城,聞珍將至,引歸。



 蒙遜又攻樂都,二旬不剋而還。鎮南文支以湟河降蒙遜,徙五千餘戶於姑臧。蒙遜又來伐,傉檀以太尉俱延為質,蒙遜乃引還。



 傉檀議欲西征乙弗,孟愷諫曰:「連年不收,上下飢弊,南逼熾磐,北迫蒙遜,百姓騷動,下不安業。今遠征雖剋,後患必深,不如結盟熾磐,通糴濟難,慰喻雜部,以廣軍資,畜力繕兵,相時而動。《易》曰:『其亡其亡,繫于苞桑。』惟陛下圖之。」傉檀曰:「孤將略地,卿無沮眾。」謂其太子武臺曰:「今不
 種多年,內外俱窘,事宜西行,以拯此弊。蒙遜近去,不能卒來,旦夕所慮,唯在熾盤。彼名微眾寡,易以討禦,吾不過一月,自足周旋。汝謹守樂都,無使失墮。」傉檀乃率騎七千襲乙弗,大破之,獲牛馬羊四十餘萬。



 熾磐乘虛來襲,撫軍從事中郎尉肅言於武臺曰:「今外城廣大,難以固守,宜聚國人於內城,肅等率諸晉人距戰於外,如或不捷,猶有萬全。」武臺曰:「小賊蕞爾,旦夕當走,卿何慮之過也。」武臺懼晉人有二心也,乃召豪望有勇謀者閉之於內。孟愷泣曰:「熾磐不道,人神同憤,愷等進則荷恩重遷,退顧妻子之累,豈有二乎!今事已急矣,人思自效,有
 何猜邪?」武臺曰:「吾豈不知子忠,實懼餘人脫生慮表,以君等安之耳。」一旬而城潰。



 安西樊尼自西平奔告傉檀,傉檀謂眾曰:「今樂都為熾磐所陷,男夫盡殺,婦女賞軍,雖欲歸還,無所赴也。卿等能與吾藉乙弗之資,取契汗以贖妻子者,是所望也。不爾,歸熾磐便為奴僕矣,豈忍見妻子在他人懷抱中!」遂引師而西,眾多逃返,遣鎮北段茍追之,茍亦不還。於是將士皆散,惟中軍紇勃、後軍洛肱、安西樊尼、散騎侍郎陰利鹿在焉。傉檀曰:「蒙遜、熾磐昔皆委質於吾,今而歸之,不亦鄙哉!四海之廣,匹夫無所容其身,何其痛也!蒙遜與吾名齊年比,熾磐姻好少
 年,俱其所忌,勢皆不濟。與其聚而同死,不如分而或全。樊尼長兄之子,宗部所寄,吾眾在北者戶垂二萬,蒙遜方招懷遐邇,存亡繼絕,汝其西也。紇勃、洛肱亦與尼俱。吾年老矣,所適不容,寧見妻子而死!」遂歸熾磐,唯陰利鹿隨之。傉檀謂利鹿曰:「去危就安,人之常也。吾親屬皆散,卿何獨留?」利鹿曰:「臣老母在家,方寸實亂。但忠孝之義,勢不俱全。雖不能西哭沮渠,申包胥之誠;東感秦援,展毛遂之操,負羈靮而侍陛下者,臣之分也。惟願開弘遠猷,審進止之算。」傉檀歎曰:「知人固未易,人亦未易知。大臣親戚皆棄我去,終紿不虧者,唯卿一人。歲寒不凋,
 見之於卿。」傉檀至西平,熾磐遣使郊迎,待以上賓之禮。



 初,樂都之潰也,諸城皆降于熾磐,傉檀將尉賢政固守浩亹不下。熾磐呼之曰:「樂都已潰,卿妻子皆在吾間,孤城獨守,何所為也!」賢政曰:「受涼王厚恩,為國家籓屏,雖知樂都已陷,妻子為擒,先歸獲賞,後順受誅,然不知主上存亡,未敢歸命。妻子小事,豈足動懷!昔羅憲待命,晉文亮之;文聘後來,魏武不責。邀一時之榮,忘委付之重,竊用恥焉,大王亦安用之哉!」熾磐乃遣武臺手書喻政,政曰:「汝為國儲,不能盡節,面縛於人,棄父負君,虧萬世之業,賢政義士,豈如汝乎!」既而聞傉檀至左南,乃降。



 熾
 磐以傉檀為驃騎大將軍,封左南公。歲餘,為熾磐所鴆。左右勸傉檀解藥,傉檀曰:「吾病豈宜療邪!」遂死,時年五十一,在位十三年,偽謚景王。武臺後亦為熾磐所殺。傉檀少子保周、臘于破羌、俱延子覆龍、鹿孤孫副周、烏孤孫承缽皆奔沮渠蒙遜。久之,歸魏,魏以保周為張掖王,覆龍酒泉公,破羌西平公,副周永平公,承缽昌松公。



 烏孤以安帝隆安元年僭立,至傉檀三世,凡十九年,以安帝義熙十年滅。



 史臣曰:禿髮累葉酋豪,擅強邊服,控弦玉塞,躍馬金山,候滿月而窺兵,乘折膠而縱鏑,禮容弗被,聲教斯阻。烏
 孤納苻渾之策,治兵以討不賓;鹿孤從史暠之言,建學而延胄子。遂能開疆河右,抗衡強國。道由人弘,抑此之謂!



 傉檀承累捷之銳,藉二昆之資,摧呂氏算無遣策,取姑臧兵不血刃,武略雄圖,比蹤前烈。既而叨竊重位,盈滿易期,窮兵以逞其心,縱慝自貽其弊,地奪於蒙遜,勢衄於赫連,覆國喪身,猶為幸也。昔宋殤好戰,致災於華督;楚靈黷武,取殺於乾溪。異代同亡,其於傉檀見之矣。



 贊曰:禿發弟兄,擅雄群虜。開疆河外,清氛西土。傉檀傑出,騰駕時英。窮兵黷武,喪國頹聲。



\end{pinyinscope}