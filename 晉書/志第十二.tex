\article{志第十二}

\begin{pinyinscope}

 樂上



 夫性靈之表,不知所以發於詠歌;感動之端,不知所以關於手足。生於心者謂之道,成於形者謂之用。譬諸天地,其猶影響,百獸率舞,而況於人乎!美其和平而哀其喪亂,以茲援律,乃播其聲焉。



 農瑟羲琴,倕鐘和磬,達靈成性,象物昭功,由此言之,其來自遠。殷氏不綱,遺風餘孽,淫奏既興,雅章奔散,《英》《莖》之制,蓋已微矣。孔子曰:「人
 能弘道,非道弘人。」周始二《南》,《風》兼六代。昔黃帝作《雲門》,堯作《咸池》,舜作《大韶》,禹作《大夏》,殷作《大濩》,周作《大武》,所謂因前王之禮,設俯仰之容,和順積中,英華發外。《書》稱命夔典樂,教胄子,則《周官》所謂奏大呂,歌黃鐘。天貺來下,人祗動色,抑揚周監,以弘雅音。及褒艷興災,平王逢亂,禮廢親疏,樂沈河海。是以延陵季子聞歌《小雅》曰:「其周德之衰乎!猶有先王之遺風焉。」而列壤稱孤,各興吟詠。魏文侯聆古樂而恐臥,晉平公聽新聲而忘食,先王之道,漸以陵夷。八方殊風,九州異則。秦氏并吞,遂專刑憲,至於絃歌《詩》《頌》,干戚旄羽,投諸煙火,掃地無遺。



 漢祖
 提劍寰中,削平天下,文匪躬於德化,武有心於制作。太后擯儒家之道,大臣排賈氏之言,搢紳先生所以長歎,而子政、仲舒猶不能已也。炎漢中興,明皇帝即位,表圭景而陳《清廟》,樹槐陰而疏璧流;祀光武於明堂,以配上帝;召桓榮於太學,袒而割牲;濟濟焉,皇皇焉,有足觀者。自斯厥後,禮樂彌殷。永平三年,官之司樂,改名大予,式揚典禮,旁求圖讖,道鄰《雅》《頌》,事邇中和。其有五方之樂者,則所謂「大樂九變,天神可得而禮」也。其有宗廟之樂者,則所謂「肅雍和鳴,先祖是聽」者也。其有社稷之樂者,則所謂「琴瑟擊鼓,以迓田祖」者也。其有辟雍之樂者,則
 所謂「移風易俗,莫善於樂」者也。其有黃門之樂者,則所謂「宴樂群臣,蹲蹲舞我」者也。其有短簫之樂者,則所謂「王師大捷,令軍中凱歌」者也。



 魏武挾天子而令諸侯,思一戎而匡九服,時逢吞滅,憲章咸盪。及削平劉表,始獲杜夔,揚鼙總干,式遵前記。三祖紛綸,咸工篇什,聲歌雖有損益,愛玩在乎雕章。是以王粲等各造新詩,抽其藻思,吟詠神靈,贊揚來饗。



 武皇帝採漢魏之遺範,覽景文之垂則,鼎鼐唯新,前音不改。泰始九年,光祿大夫荀勖始作古尺,以調聲韻,仍以張華等所制高文,陳諸下管。永嘉之亂,伶官既減,曲臺宣榭,咸變污萊。雖復《象舞》歌
 工,自胡歸晉,至於孤竹之管,雲和之瑟,空桑之琴,泗濱之磬,其能備者,百不一焉。夫人受天地之靈,蘊菁華之氣,剛柔遞用,哀樂分情。經春陽而自喜,遇秋彫而不悅。遊乎金石之端,出乎管絃之外,因物遷逝,乘流不反。是以楚王升輕軒於彭蠡,漢順聽鳴鳥於樊衢。聖人功成作樂,化平裁曲,乃揚節奏,以暢中和,飾其歡欣,止於哀思者也。



 凡樂之道,五聲、八音、六律、十二管,為之綱紀云。



 五聲:宮為君,宮之為言中也。中和之道,無往而不理焉。商為臣,商之為言強也,謂金性之堅強也。角為民,角之為言觸也,謂象諸陽氣觸物而生也。徵為事,徵之為言
 止也,言物盛則止也。羽為物,羽之為言舒也,言陽氣將復,萬物孳育而舒生也。古人有言曰:「禮樂不可斯須去身。」化上遷善,有如不及。是以聞其宮聲,使人溫良而寬大;聞其商聲,使人方廉而好義;聞其角聲,使人惻隱而仁愛;聞其徵聲,使人樂養而好施;聞其羽聲,使人恭儉而好禮。



 八音,八方之風也。乾之音石,其風不周。坎之音革,其風廣莫。艮之音匏,其風融。震之音竹,其風明庶。巽之音木,其風清明。離之音絲,其風景。坤之音土,其風涼。兌之音金,其風閶闔。



 陽六為律,謂黃鐘、太蔟、姑洗、蕤賓、夷則、無射;陰六為呂,謂大呂、應鐘、南呂、林鐘、仲呂、夾鐘:凡有十二,以配十二辰焉。律之為言法也,言陽氣施生各有法也;呂之為言助也,所以助成陽功也。



 正月之辰謂之寅,寅者津也,謂生物之津塗也。二月之辰名為卯,卯者茂也,言陽氣生而孳茂也。三月之辰名為辰,辰者震也,謂時物盡震動而長也。四月之辰謂為巳,巳者起也,物至此時畢盡而起也。五月之辰謂為午,午者長也,大也,言物皆長大也。六月之辰謂之未,未者味也,言時萬物向成,有滋味也。七月之辰謂為申,申者身也,言時萬物身體皆成就也。
 八月之辰謂為酉,酉者緧也,謂時物皆綇縮也。九月之辰謂為戌,戌者滅也,謂時物皆衰滅也。十月之辰謂為亥,亥者劾也,言時陰氣劾殺萬物也。十一月之辰謂為子,子者孳也,謂陽氣至此更孳生也。十二月之辰謂為丑,丑者紐也,言終始之際,以紐結為名也。



 十一月之管謂之黃鐘,黃者,陰陽之中色也。天有六氣,地有五才,而天地數畢焉。或曰,冬至德氣為土,土色黃,故曰黃鐘。正月之管謂為太蔟,蔟者蔟也,謂萬物隨於陽氣太蔟而生也。三月之管名為姑洗,姑洗者:姑,枯也;洗,濯也,謂物生新潔,洗除其枯,改柯易葉也。五月之管名為蕤賓,葳
 蕤,垂下貌也;賓,敬也,謂時陽氣下降,陰氣始起,相賓敬也。七月之管名為夷則,夷,平也;則,法也,謂萬物將成,平均皆有法則也。九月之管名為無射,射者出也,言時陽氣上升,萬物收藏無復出也。十二月之管名為大呂,呂者助也,謂陽氣方之,陰氣助也。十月之管名為應鐘,應者和也,謂歲功皆成,應和陽功,收而聚之也。八月之管名為南呂,南者任也,謂時物皆秀,有懷任之象也。六月之管名為林鐘,林者茂也,謂時物茂盛於野也。四月之管名為仲呂者,呂,助也,謂陽氣盛長,陰助成功也。二月之管名為夾鐘者,夾,佐也,謂時物尚未盡出,陰德佐
 陽而出物也。



 漢自東京大亂,絕無金石之樂,樂章亡缺,不可復知。及魏武平荊州,獲漢雅樂郎河南杜夔,能識舊法,以為軍謀祭酒,使創定雅樂。時又有散騎侍郎鄧靜、尹商善訓雅樂,歌師尹胡能歌宗廟郊祀之曲,舞師馮肅、服養曉知先代諸舞,夔悉總領之。遠詳經籍,近採故事,考會古樂,始設軒懸鐘磬。而黃初中柴玉、左延年之徒,復以新聲被寵,改其聲韻。



 及武帝受命之初,百度草創。泰始二年,詔郊祀明堂禮樂權用魏儀,遵周室肇稱殷禮之義,但改樂章而已,使傅玄為之詞云。



 祀天地五郊夕牲歌



 天命有晉,穆穆明明。我其夙夜,祗事上靈。常於時假,迄用其成。於薦玄牡,進夕其牲。崇德作樂,神祇是聽。



 祀天地五郊迎送神歌



 宣文蒸哉,日靖四方。永言保之,夙夜匪康。光天之命,上帝是皇。嘉樂殷薦,靈祚景祥。神祗降假,享福無疆。



 饗天地五郊歌



 天祚有晉,其命惟新。受終于魏,奄有黎民。燕及皇天,懷和百神。丕顯遺烈,之德之純。享其
 玄牡,式用肇禋。神祗來格,福祿是臻。



 時邁其猶,昊天子之。祐享有晉,肇庶戴之。畏天之威,敬授人時。丕顯丕承,於猶繹思。皇極斯建,庶績咸熙。庶幾夙夜,惟晉之祺。



 宣文惟后,克配彼天。撫寧四海,保有康年。於乎緝熙,肆用靖民。爰立典制,爰脩禮紀。作民之極,莫匪資始。克昌厥後,永言保之。



 天地郊明堂夕牲歌



 皇矣有晉,時邁其德。受終于天,光濟萬國。萬國既光,神定厥祥。虔于郊祀,祗事上皇。祗事
 上皇,百福是臻。巍巍祖考,克配彼天。嘉牲匪歆,德馨惟饗。受天之祐,神化四方。



 天地郊明堂降神歌



 於赫大晉,應天景祥。二帝邁德,宣此重光。我皇受命,奄有萬方。郊祀配享,禮樂孔章。神祗嘉享,祖考是皇。克昌厥後,保祚無疆。



 天郊饗神歌



 整泰壇,禮皇神。精氣感,百靈賓。蘊朱火,繚芳薪。紫煙遊,冠青雲。神之體,靡象形。曠無方,幽以清。神之來,光景昭。聽無聞,視無兆。
 神之至,舉歆歆。靈爽協,動餘心。神之坐,同歡娛。澤雲翔,化風舒。嘉樂奏,文中聲。八音諧,神是聽。咸契齊,並芬芳。烹牷牲,享玉觴。神悅饗,歆禋祀。祐大晉,降繁祉。作京邑,廣四海。保天年,窮地紀。



 地郊饗神歌



 整泰折,竢皇祗。眾神感,群靈儀。陰祀設,吉禮施。夜將極,時未移。祗之體,無形象。潛泰幽,洞忽荒。祗之出,薆若有。靈無遠,天下母。祗之來,遺光景。昭若存,終冥冥。祗之至,
 舉欣欣。舞象德,歌成文。祗既坐,同歡豫。澤雨施,化雲布。樂八變,聲教敷。物咸亨,祗是娛。齊既潔,侍者肅。玉觴進,咸穆穆。饗嘉豢,歆德馨。祚有晉,暨群生。溢九壤,格天庭。保萬壽,延億齡。



 明堂饗神歌



 經始明堂,享祀匪懈。於皇烈考,光配上帝。赫赫上帝,既高既崇。聖考是配,明德顯融。率土敬職,萬方來祭。常于時假,保祚永世。



 祠廟夕牲歌



 我夕我牲,猗歟敬止。嘉豢孔時,供茲享祀。神鑒厥誠,博碩斯歆。祖考降饗,以虞孝孫之心。



 祠廟迎送神歌



 嗚呼悠哉,日監在茲。以時享祀,神明降之。神明斯降,既祐饗之。祚我無疆,受天之祜。赫赫太上,巍巍聖祖。明明烈考,丕承繼序。



 祠征西將軍登歌



 經始宗廟,神明戾止。申錫無疆,祗承享祀。假哉皇祖,綏予孫子。燕及後昆,錫茲繁祉。



 祠豫章府君登歌



 嘉樂肆筵,薦祀在堂。皇皇宗廟,乃祖乃皇。濟濟辟公,相予蒸嘗。享祀不忒,降福穰穰。



 祠潁川府君登歌



 於邈先后,實司于天。顯矣皇祖,帝祉肇臻。本枝克昌,資始開元。惠我無疆,享祚永年。



 祠京兆府君登歌



 於惟曾皇,顯顯令德。商明清亮,匪競柔克,保乂命祐,基命惟則。篤生聖祖,光濟四國。



 祠宣皇帝登歌



 於鑠皇祖,聖德欽明。勤施四方,夙夜敬止。載
 敷文教,載揚武烈。匡定社稷,龔行天罰。經始大業,造創帝基。畏天之命,于時保之。



 祠景皇帝登歌



 執競景皇,克明克哲。旁作穆穆,惟祗惟畏。纂宣之緒,耆定厥功。登此雋乂,糾彼群凶。業業在位,帝既勤止。惟天之命,於穆之已。



 祠文皇帝登歌



 於皇時晉,允文文皇,聰明睿智,聖敬神武。萬機莫綜,皇斯清之。蛇豕放命,皇斯平之。柔遠能邇,簡授英賢。創業垂統,勳格皇天。



 祠廟饗神歌二篇



 曰晉是常,享祀時序。宗廟致敬,禮樂具舉。惟其來祭,普天率土。犧樽既奠,清酤既載。亦有和羹,薦羞斯備。蒸蒸永慕,感時興思。登歌奏舞,神樂其和。祖考來格,祐我邦家。溥天之下,罔不休嘉。



 肅肅在位,濟濟臣工。四海來格,神儀有容。鐘鼓振,管絃理,舞開元,歌永始,神胥樂兮!肅肅在位,臣工濟濟。小大咸敬,上下有禮。理管絃,振鼓鐘,舞象德,歌詠功,神胥樂兮!肅肅
 在位,有來雍雍。穆穆天子,相維辟公。禮有儀,樂有則,舞象功,歌詠德,神胥樂兮!



 杜夔傳舊雅樂四曲,一曰《鹿鳴》,二曰《騶虞》,三曰《伐檀》,四曰《文王》,皆古聲辭。及太和中,左延年改夔《騶虞》、《伐檀》、《文王》三曲,更自作聲節,其名雖存,而聲實異。唯因夔《鹿鳴》,全不改易。每正旦大會,太尉奉璧,群后行禮,東廂雅樂常作者是也。後又改三篇之行禮詩。第一日《於赫篇》,詠武帝,聲節與古《鹿鳴》同。第二曰《巍巍篇》,詠文帝,用延年所改《騶虞》聲。第三日《洋洋篇》,詠明帝,用延年所改《文王》聲。第四曰復用《鹿鳴》。《鹿鳴》之聲重用,而除古《伐檀》。及晉
 初,食舉亦用《鹿鳴》。至泰始五年,尚書奏,使太僕傅玄、中書監荀勖、黃門侍郎張華各造正旦行禮及王公上壽酒、食舉樂歌詩。荀勖云:『魏氏行禮、食舉,再取周詩《鹿鳴》以為樂章。又《鹿鳴》以宴嘉賓,無取於朝,考之舊聞,未知所應。」勖乃除《鹿鳴》舊歌更作行禮詩四篇,先陳三朝朝宗之義。又為正旦大會、王公上壽歌詩并食舉樂歌詩,合十三篇。又以魏氏歌詩或二言,或三言,或四言,或五言,與古詩不類,以問司律中郎將陳頎。頎曰:「被之金石,未必皆當。」故勖造晉歌,皆為四言,唯王公上壽酒一篇為三言五言焉。張華以為「魏上壽、食舉詩及漢氏所施
 用,其文句長短不齊,未皆合古。蓋以依詠弦節,本有因循,而識樂知音,足以制聲度曲,法用率非凡近之所能改。二代三京,襲而不變,雖詩章辭異,興廢隨時,至其韻逗留曲折,皆繫於舊,有由然也。是以一皆因就,不敢有所改易。」此則華、勖所明異旨也。時詔又使中書侍郎成公綏亦作焉。今並採列之云。



 四廂樂歌



 正旦大會行禮歌成公綏



 穆穆天子,光臨萬國。多士盈朝,莫匪俊德。流化罔極,王猷允塞。嘉會置酒,嘉賓充庭。羽旄
 曜宸極,鐘鼓振泰清。百辟朝三朝,彧彧明儀形。濟濟鏘鏘,金聲玉振。



 禮樂具,宴嘉賓。眉壽祚聖皇,景福惟日新。群后戾止,有來雍雍。獻酬納贄,崇此禮容。豐羞萬俎,旨酒千鐘。嘉樂盡宴樂,福祿咸攸同。



 樂哉!天下安寧。道化行,風俗清。簫《韶》作,詠九成。年豐穰,世泰平。至治哉,樂無窮。元首聰明,股肱忠。澍豐澤,揚清風。



 嘉瑞出,靈應彰。麒麟見,鳳皇翔。醴泉湧,流中唐。嘉禾生,穗盈箱。降繁祉,祚聖皇。承天
 位,統萬國。受命應期,授聖德,四世重光。宣開洪業,景克昌,文欽明,德彌彰。肇啟晉邦,流祚無疆。



 泰始建元,鳳皇龍興。龍興伊何,享祚萬乘。奄有八荒,化育黎蒸。圖書既煥,金石有徵。德光大,道熙隆。被四表,格皇穹。奕奕萬嗣,明明顯融,高朗令終。保茲永祚,與天比崇。



 聖皇君四海,順人應天期。三葉合重光,泰始開洪基。明曜參日月,功化侔四時。宇宙清且泰,黎庶咸雍熙,善哉雍熙!



 惟天降命,翼仁祐聖。於穆三皇,載德彌盛。總齊璇璣,光統七政。百揆時序,化若神聖。四海同風,興至仁。濟民育物,擬陶均。擬陶均,垂惠潤。皇皇群賢,峨峨英雋。德化宣,芬芳播來胤。播來胤,垂後昆。清廟何穆穆,皇極闢四門。皇極闢四門,萬機無不綜。亹亹翼翼,樂不及荒,飢不遑食。大禮既行,樂無極。



 登崑崙,上層城。乘飛龍,升泰清。冠日月,佩五星。揚虹霓,建篲旌。披慶雲,蔭繁榮。覽八極,遊天庭。順天地,和陰陽。序四時,曜三光。
 張帝綱,正皇綱。播仁風,流惠康。邁洪化,振靈威。懷萬方,納九夷。朝閶闔,宴紫微。建五旗,羅鐘虡。列四懸,奏《韶》《武》。鏗金石,揚旌羽。縱八佾,《巴渝舞》。詠雅頌,和律呂。于胥樂,樂聖主。



 化蕩蕩,清風泄。總英雄,御俊傑。開宇宙,掃四裔。光緝熙,美聖哲。超百代,揚休烈。流景祚,顯萬世。



 皇皇顯祖,翼世佐時。寧濟六合,受命應期。神武鷹揚,大化咸熙。廊開皇衢,用成帝基。



 光光景皇,無競惟烈。匡時拯俗,休功蓋世。宇宙既康,九域有截。天命降監,啟祚明哲。



 穆穆烈考,克明克雋。實天生德,誕應靈運。肇建帝業,開國有晉。載德奕世,垂慶洪胤。



 明明聖帝,龍飛在天。與靈合契,通德幽玄。仰化青雲,俯育重川。受靈之祐,於萬斯年。



 正旦大會王公上壽酒歌荀勖



 踐元辰,延顯融。獻羽觴,祈令終。我皇壽而隆,我皇茂而嵩。本枝奮百世,休祚鐘聖躬。



 食舉樂東西廂歌荀勖



 煌煌七曜,重明交暢。我有嘉賓,是應是貺。邦政既圖,接以大饗。人之好我,式遵德讓。



 賓之初筵,藹藹濟濟。既朝乃宴,以洽百禮。頒以位敘,或庭或陛。登儐台叟,亦有兄弟。胥子陪寮,憲茲度楷。觀頤養正,降福孔偕。



 昔我三后,大業是維。今我聖皇,焜炔前暉。奕世重規,明照九畿。思輯用光,時罔有違。陟禹之跡,莫不來威。天被顯祿,福履是綏。



 赫矣太祖,克廣明德。廊開宇宙,正世立則。變化不經,民無瑕慝。創業垂統,兆我晉國。



 烈文伯考,時維帝景。夷險平亂,威而不猛。御衡不迷,皇塗煥景。七德咸宣,其寧惟永。



 猗歟盛歟!先皇聖文。則天作孚,大哉為君。慎徽五典,帝載是勤。文武發揮,茂建嘉勛。脩己濟治,民用寧殷。懷遠燭幽,玄教氤氳。善世不伐,服事三分。德博化隆,道昌無垠。



 隆化洋洋,帝命溥將。登我晉道,越惟聖王。龍飛革運,臨燾八荒。睿哲欽明,配蹤虞唐。封建厥福,駿發其祥。三朝習吉,終然允臧。其臧維何,總彼萬方。元侯列辟,四嶽籓王。時見世享,
 率茲有常。旅揖在庭,嘉客在堂。宋衛既臻,陳留山陽。有賓有使,觀國之光。貢賢納計,獻璧奉璋。保祐命之,申錫無疆。



 振鷺于飛,鴻漸其翼。京邑穆穆,四方是式。無競維人,王綱允敕。君子來朝,言觀其極。



 暠鄖大君,民之攸暨。信理天工,惠康不匱。將遠不仁,訓以醇粹。幽明有倫,俊乂在位。九族既睦,庶邦順比。開元布憲,四海鱗萃。協時正統,殊塗同致。厚德載物,靈心隆貴。敷奏讜言,納以無諱。樹之典象,誨之義類。上教如風,
 下應如卉。一人有廢,群萌以遂。我后宴喜,令問不墜。



 既宴既喜,翕是萬邦。禮儀卒度,物有其容。晰晰庭燎,喤々鼓鐘。笙磬詠德,萬舞象功。八音克諧,俗易化從。其和如樂,庶品時邕。



 時邕斌斌,六合同塵。往我祖宣,威靜殊鄰。首定荊楚,遂平燕秦。亹亹文皇,邁德流仁。爰造草昧,應乾順民。靈瑞告符,休徵響震。天地弗違,以和神人。既禽庸蜀,吳會是賓。肅慎率職,楛矢來陳。韓濊進樂,宮徵清鈞。西旅獻獒,
 扶南效珍。蠻裔重譯,玄齒文身。我皇撫之,景命惟新。



 愔愔嘉會,有聞無聲。清酤既奠,籩豆既升。禮充樂備,簫《韶》九成。愷樂飲酒,酣而不盈。率土歡豫,邦國以寧。王猷允塞,萬載無傾。



 冬至初歲小會歌張華



 日月不留,四氣回周。節慶代序,萬國同休。庶尹群后,奉壽升朝。我有壽禮,式宴百僚。繁肴綺錯,旨酒泉渟。笙鏞和奏,磬管流聲。上隆其愛,下盡其心。宣其壅滯,訓之德音。乃宣乃訓,
 配享交泰。永載仁風,長撫無外。



 宴會歌張華



 亹亹我皇,配天垂光。留精日昃,經覽無方。聽朝有暇,延命眾臣。冠蓋雲集,樽俎星陳。肴蒸多品,八珍代變。羽爵無算,究樂極宴。歌者流聲,舞者投袂。動容有節,絲竹並設。宜揚四體,繁手趣摯。懽足發和,酣不忘禮。好樂無荒,翼翼濟濟。



 命將出征歌張華



 重華隆帝道,戎蠻或不賓。徐夷興有周,鬼方亦
 違殷。今在盛明世,寇虐動四垠。豺狼染牙爪,群生號穹旻。元帥統方夏,出車撫涼秦。眾貞必以律,臧否實在人。威信加殊類,疏逖思自親。單醪豈有味,挾纊感至仁。武功尚止戈,七德美安民。遠跡由斯舉,永世無風塵。



 勞還師歌張華



 玁犬允背天德,構亂擾邦畿。戎車震朔野,群帥贊皇威。將士齊心旅,感義忘其私。積勢如郭弩,赴節如發機。囂聲動山谷,金光曜素暉。揮戈陵勁敵,武步蹈橫屍。鯨鯢皆授首,北土永清夷。
 昔往冒隆暑,今來白雪霏。征夫信勤瘁,自古詠《採薇》。收榮於舍爵,燕喜在凱歸。



 中宮所歌張華



 先王統大業,玄化漸八維。儀刑孚萬邦,內訓隆壼闈。皇英垂帝典,《大雅》詠三妃。執德宣隆教,正位理厥機。含章體柔順,帥禮蹈謙祗。《螽斯》弘慈惠,《樛木》逮幽微。徽音穆清風,高義邈不追。遺榮參日月,百世仰餘暉。



 宗親會歌張華



 族燕明禮順,啜食序親親。骨肉散不殊,昆弟豈
 他人。本枝篤同慶,《棠棣》著先民。於皇聖明后,天覆弘且仁。降禮崇親戚,旁施協族姻。式宴盡酣娛,飲御備羞珍。和樂既宣洽,上下同懽欣。德教加四海,敦睦被無垠。



 泰始九年,光祿大夫荀勖以杜夔所制律呂,校太樂、總章、鼓吹八音,與律呂乖錯,乃制古尺,作新律呂,以調聲韻。事具《律歷志》。律成,遂班下太常,使太樂、總章、鼓吹、清商施用。勖遂典知樂事,啟朝士解音律者共掌之。使郭夏、宋識等造《正德》、《大豫》二舞,其樂章亦張華之所作云。



 正德舞歌張華



 日皇上天,玄鑒惟光。神器周回,五德代章。祚命于晉,世有哲王。弘濟區夏,陶甄萬方。大明垂曜,旁燭無疆。蚩蚩庶類,風德永康。皇道惟清,禮樂斯經。金石在懸,萬舞在庭。象容表慶,協律被聲。軼《武》超《濩》,取節《六英》。同進退讓,化漸無形。大和宣洽,通於幽冥。



 大豫舞歌張華



 惟天之命,符運有歸。赫赫大晉,三后重暉。繼明紹世,光撫九圍。我皇紹期,遂在璇璣。群生屬命,奄有庶邦。慎徽五典,玄教遐通。萬方同
 軌,率土咸雍。爰制《大豫》,宣德舞功。醇化既穆,王道協隆。仁及草木,惠加昆蟲。億兆夷人,悅仰皇風。丕顯大業,永世彌崇。



 荀勖又作新律笛十二枚,以調律呂,正雅樂,正會殿庭作之,自謂宮商克諧,然論者猶謂勖暗解。時阮咸妙達八音,論者謂之神解。咸常心譏勖新律聲高,以為高近哀思,不合中和。每公會樂作,勖意咸謂之不調,以為異己,乃出咸為始平相。後有田父耕於野,得周時玉尺,勖以校己所治鐘鼓金石絲竹,皆短校一米,於此伏咸之妙,復徵咸歸。勖既以新律造二舞,次更脩正鐘聲。會勖
 薨,未竟其業。元康三年,詔其子籓脩定金石,以施郊廟。尋值喪亂,莫有記之者。



 漢高祖自蜀漢將定三秦,閬中范因率賨人以從帝,為前鋒。及定秦中,封因為閬中侯,復賨人七姓。其俗喜舞,高祖樂其猛銳,數觀其舞,後使樂人習之。閬中有渝水,因其所居,故名曰《巴渝舞》。舞曲有《矛渝本歌曲》、《安弩渝本歌曲》、《安臺本歌曲》、《行辭本歌曲》,總四篇。其辭既古,莫能曉其句度。魏初,乃使軍謀祭酒王粲改創其詞。粲問巴渝帥李管、種玉歌曲意,試使歌,聽之,以考校歌曲,而為之改為《矛渝新福歌曲》、《弩渝新福歌曲》、《安臺新福歌
 曲》、《行辭新福歌曲》,《行辭》以述魏德。黃初三年,又改《巴渝舞》曰《昭武舞》。至景初元年,尚書奏,考覽三代禮樂遺曲,據功象德,奏作《武始》、《咸熙》、《章斌》三舞,皆執羽籥。及晉又改《昭武舞》曰《宣武舞》,《羽籥舞》曰《宣文舞》。咸寧元年,詔定祖宗之號,而廟樂乃停《宣武》、《宣文》二舞,而同用荀勖所使郭夏、宋識等所造《正德》、《大豫》二舞云。



\end{pinyinscope}