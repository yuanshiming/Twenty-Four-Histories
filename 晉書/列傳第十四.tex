\article{列傳第十四}

\begin{pinyinscope}
鄭袤
 \gezhu{
  子默默子球}
 李胤盧欽
 \gezhu{
  弟珽珽子志志子諶}
 華表
 \gezhu{
  子廙廙子恆廙弟嶠}
 石鑒溫羨



 鄭袤,字林叔,滎陽開封人也。高祖眾,漢大司農。父泰,揚州刺史,有高名。袤少孤,早有識鑒。荀攸見之曰:「鄭公業為不亡矣。」隨叔父渾避難江東。時華歆為豫章太守,渾往依之,歆素與泰善,撫養袤如己子。年十七,乃還鄉里。性清正。時濟陰魏諷為相國掾,名重當世,袤同郡任覽與結交。袤以諷奸雄,終必為禍,勸覽遠之。及諷敗,論者
 稱焉。



 魏武帝初封諸子為侯,精選賓友,袤與徐幹俱為臨淄侯文學,轉司隸功曹從事。司空王朗辟為掾,袤舉高陽許允、扶風魯芝、東萊王基,朗皆命之,後咸至大位,有重名。袤遷尚書郎。出為黎陽令,吏民悅服。太守班下屬城,特見甄異,為諸縣之最。遷尚書右丞。轉濟陰太守,下車旌表孝悌,敬禮賢能,興立庠序,開誘後進。調補大將軍從事中郎,拜散騎常侍。會廣平太守缺,宣帝謂袤曰:「賢叔大匠垂稱於陽平、魏郡,百姓蒙惠化。且盧子家、王子雍繼踵此郡,使世不乏賢,故復相屈。」袤在廣平,以德化為先,善作條教,郡中愛之。徵拜侍中,百姓戀慕,涕
 泣路隅。遷少府。高貴鄉公即位,袤與河南尹王肅備法駕奉迎於元城,封廣昌亭侯。徙光祿勳,領宗正。



 毌丘儉作亂,景帝自出征之,百官祖送於城東,袤疾病不任會。帝謂中領軍王肅曰:「唯不見鄭光祿為恨。」肅以語袤,袤自輿追帝,及於近道。帝笑曰:「故知侯生必來也。」遂與袤共載,曰:「計將何先?」袤曰:「昔與儉俱為臺郎,特所知悉。其人好謀而不達事情,自昔建勳幽州,志望無限。文欽勇而無算。今大軍出其不意,江、淮之卒銳而不能固,深溝高壘以挫其氣,此亞夫之長也。」帝稱善。轉太常。高貴鄉公議立明堂辟雍,精選博士,袤舉劉毅、劉寔、程咸、庾峻,
 後並至公輔大位。及常道鄉公立,與議定策,進封安城鄉侯,邑千戶。景元初,疾病失明,屢乞骸骨,不許。拜光祿大夫。五等初建,封密陵伯。



 武帝踐阼,進爵為侯。雖寢疾十餘年,而時賢並相推薦。泰始中,詔曰:「光祿密陵侯袤,履行純正,守道沖粹,退有清和之風,進有素絲之節,宜登三階之曜,補袞職之闕。今以袤為司空。」天子臨軒,遣五官中郎將國坦就第拜授。袤前後辭讓,遣息稱上送印綬,至于十數。謂坦曰:「魏以徐景山為司空,吾時為侍中,受詔譬旨。徐公語吾曰:『三公當上應天心,茍非其人,實傷和氣,不敢以垂死之年,累辱朝廷也。』終於不
 就。遵大雅君子之跡,可不務乎!」固辭,久之見許,以侯就第,拜儀同三司,置舍人官騎,賜床帳簟褥、錢五十萬。



 九年薨,時年八十五。帝於東堂發哀,賜祕器、朝服一具、衣一襲、錢三十萬、絹布各百匹,以供喪事。謚曰元。有子六人,長子默嗣,次質、舒、詡、稱、予,位並列卿。



 默字思元。起家祕書郎,考核舊文,刪省浮穢。中書令虞松謂曰:「而今而後,朱紫別矣。」轉尚書考功郎,專典伐蜀事,封關內侯,遷司徒左長史。武帝受禪,與太原郭奕俱為中庶子。朝廷以太子官屬宜稱陪臣。默上言:「皇太子體皇極之尊,無私於天下。宮臣皆受命天朝,不得同之
 籓國。」事遂施行。出為東郡太守,值歲荒人飢,默輒開倉振給,乃舍都亭,自表待罪。朝廷嘉默憂國,詔書褒歎,比之汲黯。班告天下,若郡縣有此比者,皆聽出給。入為散騎常侍。



 初,帝以貴公子當品,鄉里莫敢與為輩,求之州內,於是十二郡中正僉共舉默。文帝與袤書曰:「小兒得廁賢子之流,愧有竅賢之累。」及武帝出祀南郊,詔使默驂乘,因謂默曰:「卿知何以得驂乘乎?昔州里舉卿相輩,常愧有累清談。」遂問政事,對曰:「勸穡務農,為國之基。選人得才,濟世之道。居官久職,政事之宜。明慎黜陟,勸戒之由。崇尚儒素,化導之本。如此而已矣。」帝善之。



 後以父
 喪去官,尋起為廷尉。是時鬲令袁毅坐交通貨賂,大興刑獄。在朝多見引逮,唯默兄弟以潔慎不染其流。遷太常。時僕射山濤欲舉一親親為博士,謂默曰:「卿似尹翁歸,令吾不敢復言。」默為人敦重,柔而能整,皆此類也。



 及齊王攸當之國,下禮官議崇錫典制。博士祭酒曹志等並立異議,默容過其事,坐免。尋拜大鴻臚。遭母喪,舊制,既葬還職,默自陳懇至,久而見許。遂改法定令,聽大臣終喪,自默始也。服闋,為大司農,轉光祿勳。



 太康元年卒,時年六十八,謚曰成。尚書令衛瓘奏:「默才行名望,宜居論道,五升九卿,位未稱德,宜贈三司。」而后父楊駿先欲
 以女妻默子豫,默曰:「吾每讀《雋不疑傳》,常想其人。畏遠權貴,奕世所守。」遂辭之。駿深為恨。至此,駿議不同,遂不施行。默寬沖博愛,謙虛溫謹,不以才地矜物,事上以禮,遇下以和,雖僮豎廝養不加聲色,而猶有嫌怨,故士君子以為居世之難。子球。



 球字子瑜。少辟宰府,入侍二宮。成都王為大將軍,起義討趙王倫,球自頓丘太守為右長史,以功封平壽公。累遷侍中、尚書、散騎常侍、中護軍、尚書右僕射,領吏部。永嘉二年卒,追贈金紫光祿大夫,謚曰元。球弟豫,永嘉末為尚書。



 李胤,字宣伯,遼東襄平人也。祖敏,漢河內太守,去官還鄉里,遼東太守公孫度欲彊用之,敏乘輕舟浮滄海,莫知所終。胤父信追求積年,浮海出塞,竟無所見,欲行喪制服,則疑父尚存,情若居喪而不聘娶。後有鄰居故人與其父同年者亡,因行喪制服。燕國徐邈與之同州里,以不孝莫大於無後,勸使娶妻。既生胤,遂絕房室,恒如居喪禮,不堪其憂,數年而卒。胤既幼孤,母又改行,有識之後,降食哀戚,亦以喪禮自居。又以祖不知存亡,設木主以事之。由是以孝聞。容貌質素,頹然若不足者,而知
 度沈邃,言必有則。



 初仕郡上計掾,州辟部從事、治中,舉孝廉,參鎮北軍事。遷樂平侯相,政尚清簡。入為尚書郎,遷中護軍司馬、吏部郎,銓綜廉平。賜爵關中侯,出補安豐太守。文帝引為大將軍從事中郎,遷御史中丞,恭恪直繩,百官憚之。伐蜀之役,為西中郎將、督關中諸軍事。後為河南尹,封廣陸伯。泰始初,拜尚書,進爵為侯。胤奏以為:「古者三公坐而論道,內參六官之事,外與六卿之教,或處三槐,兼聽獄訟,稽疑之典,謀及卿士。陛下聖德欽明,垂心萬機,猥發明詔,儀刑古式,雖唐、虞疇諮,周文翼翼,無以加也。自今以往,國有大政,可親延群公,詢納
 讜言。其軍國所疑,延詣省中,使侍中、尚書諮論所宜。若有疾病,不任覲會,臨時遣侍臣訊訪。」詔從之。遷吏部尚書僕射,尋轉太子少傅。詔以胤忠允高亮,有匪躬之節,使領司隸校尉。胤屢自表讓,忝傅儲宮,不宜兼監司之官。武帝以二職並須忠賢,故每不許。



 咸寧初,皇太子出居東宮,帝以司錄事任峻重,而少傅有旦夕輔導之務,胤素羸,不宜久勞之,轉拜侍中,加特進。俄遷尚書令,侍中、特進如故。胤雖歷職內外,而家至貧儉,兒病無以市藥。帝聞之,賜錢十萬。其後帝以司徒舊丞相之職,詔以胤為司徒。在位五年,簡亮持重,稱為任職。以吳會初平,
 大臣多有勛勞,宜有登進,乃上疏遜位。帝不聽,遣侍中宣旨,優詔敦諭,絕其章表。胤不得已,起視事。



 太康三年薨,詔遣御史持節監喪致祠,謚曰成。皇太子命舍人王贊誄之,文義甚美。帝後思胤清節,詔曰:「故司徒李胤,太常彭灌,並履忠清儉,身沒,家無餘積,賜胤家錢二百萬、穀千斛,灌家半之。」三子,固、真長、脩。固字萬基,散騎郎,先胤卒,固子志嗣爵。志字彥道,歷位散騎侍郎、建威將軍、陽平太守。真長位至太僕卿。脩黃門侍郎、太弟中庶子。



 盧欽,字子若,范陽涿人也。祖植,漢侍中。父毓,魏司空。世
 以儒業顯。欽清淡有遠識,篤志經史,舉孝廉,不行,魏大將軍曹爽辟為掾。爽弟嘗有所屬請,欽白爽子弟不宜干犯法度,爽深納之,而罰其弟。除尚書郎。爽誅,免官。後為侍御史,襲父爵大利亭侯,累遷瑯邪太守。宣帝為太傅,辟從事中郎,出為陽平太守,遷淮北都督、伏波將軍,甚有稱績。徵拜散騎常侍、大司農,遷吏部尚書,進封大梁侯。武帝受禪,以為都督沔北諸軍事、平南將軍、假節,給追鋒軺臥車各一乘、第二駙馬二乘、騎具刀器、御府人馬鎧等,及錢三十萬。欽在鎮寬猛得中,疆埸無虞。入為尚書僕射,加侍中、奉車都尉,領吏部。以清貧,特賜絹
 百匹。欽舉必以材,稱為廉平。



 咸寧四年卒,詔曰:「欽履道清正,執德貞素。文武之稱,著於方夏。入躋機衡,惟允庶事。肆勤內外,有匪躬之節。不幸薨沒,朕甚悼之。其贈衛將軍、開府儀同三司,賜祕器、朝服一具、衣一襲、布五十匹、錢三十萬。」謚曰元。又以欽忠清高潔,不營產業,身沒之後,家無所庇,特賜錢五十萬,為立第舍。復下詔曰:「故司空王基、衛將軍盧欽、領典軍將軍楊囂,並素清貧,身沒之後,居無私積。頃者饑饉,聞其家大匱,其各賜穀三百斛。」欽歷宰州郡,不尚功名,唯以平理為務。祿俸散之親故,不營貲產。動循禮典,妻亡,制廬杖,終喪居外。所著
 詩賦論難數十篇,名曰《小道》。子浮嗣。



 浮字子雲,起家太子舍人。病疽截手,遂廢。然朝廷器重之,以為國子博士、祭酒、祕書監,皆不就。



 欽弟珽字子笏,衛尉卿。珽子志。



 志字子道,初闢公府掾、尚書郎,出為鄴令。成都王穎之鎮鄴也,愛其才量,委以心膂,遂為謀主。齊王冏起義,遣使告穎。穎召志計事,志曰:「趙王無道,肆行篡逆,四海人神,莫不憤怒。今殿下總率三軍,應期電發,子來之眾,不召自至。掃夷凶逆,必有征無戰。然兵事至重,聖人所慎。宜旌賢任才,以收時望。」穎深然之,改選上佐,高辟掾屬,
 以志為諮議參軍,仍補左長史,專掌文翰。穎前鋒都督趙驤為倫所敗,士眾震駭,議者多欲還保朝歌。志曰:「今我軍失利,敵新得勝,必有輕易陵轢之情,若頓兵不進,三軍畏衄,懼不可用。且戰何能無勝負,宜更選精兵,星行倍道,出賊不意,此用兵之奇也。」穎從之。及倫敗,志勸穎曰:「齊王眾號百萬,與張泓等相持不能決,大王逕得濟河,此之大勛,莫之與比,而齊王今當與大王共輔朝政。志聞兩雄不俱處,功名不並立,今宜因太妃微疾,求還定省,推崇齊王,徐結四海之心,此計之上也。」穎納之,遂以母疾還籓,委重於冏。由是穎獲四海之譽,天下歸
 心。朝廷封志為武強侯,加散騎常侍。



 及河間王顒納李含之說,欲內除二王,樹穎儲副,遣報穎,穎將應之,志正諫,不從。及冏滅,穎遙執期權,遂懷觖望之心。以長沙王乂在內,不得恣其所欲,密欲去乂。時荊州有張昌之亂,穎表求親征,朝廷許之。會昌等平,乃迴兵以討乂。志諫曰:「公前有復皇祚之大勳,及事平,歸功於齊,辭九錫之賞,不當朝政之權,振陽翟飢人,葬黃橋白骨,皆盛德之事,四海之人莫不荷賴矣。逆寇縱肆,猾擾荊、楚,今公掃清群難,南土以寧,振旅而旋,頓軍關外,文服入朝,此霸王者之事也。」穎不納。



 及乂死,穎表志為中書監,留鄴,參
 署相府事。乘與敗於蕩陰,穎遣志督兵迎帝。及王浚攻鄴,志勸穎奉天子還洛陽。時甲士尚萬五千人,志夜部分,至曉,眾皆成列,而程太妃戀鄴不欲去,穎未能決。俄而眾潰,唯志與子謐、兄子綝、殿中武賁千人而已,志復勸穎早發。時有道士姓黃,號曰聖人,太妃信之。及使呼人,道士求兩杯酒,飲訖,拋杯而去,於是志計始決。而人馬復散,志於營陣間尋索,得數乘鹿車,司馬督韓玄收集黃門,得百餘人。志入,帝問志曰:「何故散敗至此?」志曰:「賊去鄴尚八十里,而人士一朝駭散,太弟今欲奉陛下還洛陽。」帝曰:「甚佳。」於是御犢車便發。屯騎校尉郝昌先
 領兵八千守洛陽,帝召之,至汲郡而昌至,兵仗甚盛。志喜於復振,啟天子宜下赦書,與百姓同其休慶。既達洛陽,志啟以滿奮為司隸校尉。奔散者多還,百官粗備,帝悅,賜志絹二百匹、綿百斤、衣一襲、鶴綾袍一領。



 初,河間王顒聞王浚起兵,遣右將軍張方救鄴。方聞成都軍敗,頓兵洛陽,不敢進,縱兵虜掠,密欲遷都長安,將焚宗廟宮室,以絕人心。志說方曰:「昔董卓無道,焚燒洛陽,怨毒之聲,百年猶存,何為襲之!」乃止。方遂逼天子幸其壘。帝垂泣就輿,唯志侍側,曰:「陛下今日之事,當一從右將軍。臣駑怯,無所云補,唯知盡微誠,不離左右而已。」停方壘三
 日便西,志復從至長安。穎被黜,志亦免官。



 及東海王越奉迎大駕,顒啟帝復穎還鄴,以志為魏郡太守,加左將軍,隨穎北鎮。行達洛陽,而平昌公模遣前鋒督護馮嵩距穎。穎還長安,未至而聞顒斬張方,求和於越。穎住華陰,志進長安,詣闕陳謝,即還就穎於武關。奔南陽,復為劉陶所驅,回詣河北。及穎薨,官屬奔散,唯志親自殯送,時人嘉之。越命志為軍諮祭酒,遷衛尉,永嘉末,轉尚書。洛陽沒,志將妻子北投并州刺史劉琨。至陽邑,為劉粲所虜,與次子謐、詵等俱遇害於平陽。長子諶。



 諶字子諒,清敏有理思,好《老》《莊》,善屬文。選尚武帝女滎
 陽公主,拜駙馬都尉,未成禮而公主卒。後州舉秀才,辟太尉掾。洛陽沒,隨志北依劉琨,與志俱為劉粲所虜。粲據晉陽,留諶為參軍。琨收散卒,引猗盧騎還攻粲。粲敗走,諶得赴琨,先父母兄弟在平陽者,悉為劉聰所害。琨為司空,以諶為主薄,轉從事中郎。琨妻即諶之從母,既加親愛,又重其才地。



 建興末,隨琨投段匹磾。匹磾自領幽州,取諶為別駕。匹磾既害琨,尋亦敗喪。時南路阻絕,段末波在遼西,諶往投之。元帝之初,末波通使於江左,諶因其使抗表理琨,文旨甚切,於是即加弔祭。累徵諶為散騎中書侍郎,而為末波所留,遂不得南渡。末波死,
 弟遼代立,諶流離世故且二十載。石季龍破遼西,復為季龍所得,以為中書侍郎、國子祭酒、侍中、中書監。屬冉閔誅石氏,諶隨閔軍,於襄國遇害,時年六十七,是歲永和六年也。



 諶名家子,早有聲譽,才高行潔,為一時所推。值中原喪亂,與清河崔悅、穎川荀綽、河東裴憲、北地傅暢並淪陷非所,雖俱顯於石氏,恒以為辱。諶每謂諸子曰:「吾身沒之後,但稱晉司空從事中郎爾。」撰《祭法》,注《莊子》,及文集,皆行於世。



 悅字道儒,魏司空林曾孫,劉琨妻子之姪也。與諶俱為琨司空從事中郎,後為末波佐史。沒石氏,亦居大官。其綽、憲、暢並別有傳。



 華表,字偉容,平原高唐人也,父歆,清德高行,為魏太尉。表年二十,拜散騎黃門郎,累遷侍中。正元初,石苞來朝,盛稱高貴鄉公,以為魏武更生。時聞者流汗沾背,表懼禍作,頻稱疾歸下舍,故免於大難。後遷尚書。五等建,封觀陽伯。坐供給喪事不整,免。泰始中,拜太子少傅,轉光祿勳。遷太常卿。數歲,以老病乞骸骨。詔曰:「表清貞履素,有老成之美,久乾王事,靜恭匪懈。而以疾固辭,章表懇至。今聽如所上,以為太中大夫,賜錢二十萬,床帳褥席祿賜與卿同,門施行馬。」表以苦節垂名,司徒李胤、司隸
 王宏等並歎美表清澹退靜,以為不可得貴賤而親疏也。咸寧元年八月卒,時年七十二,謚曰康,詔賜朝服。有六子:暠、岑、嶠、鑒、澹、簡。



 暠字長駿,弘敏有才義。妻父盧毓典選,難舉姻親,故暠年三十五不得調,晚為中書通事郎。泰始初,遷冗從僕射。少為武帝所禮,歷黃門侍郎、散騎常侍、前軍將軍、侍中、南中郎將、都督河北諸軍事。父疾篤輒還,仍遭喪舊例,葬訖復任,暠固辭,迕旨。



 初,表有賜客在鬲,使暠因縣令袁毅錄名,三客各代以奴。及毅以貨賕致罪,獄辭迷謬,不復顯以奴代客,直言送三奴與暠,而毅亦盧氏婿
 也。又中書監荀勖先為中子求暠女,暠不許,為恨,因密啟帝,以袁毅貨賕者多,不可盡罪,宜責最所親者一人,因指暠當之。又綠暠有違忤之咎,遂於喪服中免暠官,削爵土。大鴻臚何遵奏暠免為庶人,不應襲封,請以表世孫混嗣表。有司奏曰:「暠所坐除名削爵,一時之制。暠為世子,著在名簿,不聽襲嗣,此為刑罰再加。諸侯犯法,八議平處者,褒功重爵也。嫡統非犯終身棄罪,廢之為重,依律應聽襲封。」詔曰:「諸侯薨,子踰年即位,此古制也。應即位而廢之,爵命皆去矣,何為罪罰再加?且吾之責暠,以肅貪穢,本不論常法也。諸賢不能將明此意,乃更
 詭易禮律,不顧憲度,君命廢之,而群下復之,此為上下正相反也。」於是有司奏免議者官,詔皆以贖論。混以世孫當受封,逃避,斷髮陽狂,病喑不能語,故得不拜,世咸稱之。



 暠棲遲家巷垂十載,教誨子孫,講誦經典。集經書要事,名曰《善文》,行於世。與陳勰共造豬闌於宅側,帝嘗出視之,問其故,左右以實對,帝心憐之。帝後又登陵雲臺,望見廙苜蓿園,阡陌甚整,依然感舊。太康初大赦,乃得襲封。久之,拜城門校尉,遷左衛將軍。數年,以為中書監。惠帝即位,加侍中、光祿大夫、尚書令,進爵為公。暠應楊駿召,不時還,有司奏免官。尋遷太子少傅,加散騎常
 侍,動遵禮典,得傅導之義。後年衰病篤,詔遣太醫療病,進位光祿大夫、開府儀同三司。時河南尹韓壽因託賈後求以女配暠孫陶,暠距而不許,后深以為恨,故遂不登台司。年七十五卒,謚曰元。三子:混、薈、恒。



 混字敬倫,嗣父爵,清貞簡正,歷位侍中、尚書,卒官。子陶嗣,補鞏令,沒於石勒。



 薈字敬叔,為河南尹。與荀籓、荀組俱避賊,至臨穎,父子並遇害。



 恒字敬則,博學以清素為稱。尚武帝女滎陽長公主,拜駙馬都尉。元康初,東宮建,恒以選為太子賓友,賜爵關
 內侯,食邑百戶。辟司徒王渾倉曹掾,屬除散騎侍郎,累遷散騎常侍、北軍中候,俄拜領軍,加散騎常侍。



 愍帝即位,以恒為尚書,進爵苑陵縣公。頃之,劉聰逼長安,詔出恒為鎮軍將軍,領潁川太守,以為外援。恒興合義軍,得二千人,未及西赴,而關中陷沒。時群賊方盛,所在州郡相繼奔敗,恒亦欲棄郡東渡,而從兄軼為元帝所誅,以此為疑。先書與驃騎將軍王導,導言於帝。帝曰:「兄弟罪不相及,況群從乎!」即召恒,補光祿勛。恒到,未及拜,更以為衛將軍,加散騎常侍、本州大中正。



 尋拜太常,議立郊祀。尚書刁協、國子祭酒杜彞議,須還洛乃脩郊祀。恒議,
 漢獻帝居許,即便郊柴,宜於此修立。司徒荀組、驃騎將軍王導同恒議,遂定郊祀。尋以疾求解,詔曰:「太常職主宗廟,烝嘗敬重,而華恒所疾,不堪親奉職事。夫子稱『吾不與祭,如不祭』,況宗伯之任職所司邪!今轉恒為廷尉。」頃之,加特進。



 太寧初,遷驃騎將軍,加散騎常侍,督石頭水陸諸軍事。王敦表轉恒為護軍,疾病不拜。授金紫光祿大夫,又領太子太保。成帝即位,加散騎常侍,領國子祭酒。咸和初,以愍帝時賜爵進封一皆削除,恒更以討王敦功封苑陵縣侯,復領太常。蘇峻之亂,恒侍帝左右,從至石頭,備履艱危,困悴踰年。



 初,恒為州大中正,鄉人
 任讓輕薄無行,為恒所黜。及讓在峻軍中,任勢多所殺害,見恒輒恭敬,不肆其虐。鐘雅、劉超之死,亦將及恒,讓盡心救衛,故得免。



 及帝加元服,又將納后。寇難之後,典籍靡遺,婚冠之禮,無所依據。恒推尋舊典,撰定禮儀,并郊廟辟雍朝廷軌則,事並施用。遷左光祿大夫、開府,常侍如故,固讓未拜。會卒,時年六十九,冊贈侍中、左光祿大夫、開府,謚曰敬。



 恒清恪儉素,雖居顯列,常布衣蔬食,年老彌篤。死之日,家無餘財,唯有書數百卷,時人以此貴之。子俊嗣,為尚書郎。俊子仰之,大長秋。



 嶠字叔駿,才學深博,少有令聞。文帝為大將軍,辟為掾
 屬,補尚書郎,轉車騎從事中郎。泰始初,賜爵關內侯。遷太子中庶子。出為安平太守。辭親老不行,更拜散騎常侍,典中書著作,領國子博士,遷侍中。



 太康末,武帝頗親宴樂,又多疾病。屬小瘳,嶠與侍臣表賀,因微諫曰:「伏惟聖體漸就平和,上下同慶,不覺抃舞。臣等愚戇,竊有微懷,以為收功於所忽,事乃無悔;慮福於垂成,祚乃日新。唯願陛下深垂聖明,遠思所忽之悔,以成日新之福。沖靜和氣,嗇養精神,頤身於清簡之宇,留心於虛曠之域。無厭世俗常戒,以忽群下之言,則豐慶日延,天下幸甚!」帝手詔報曰:「輒自消息,無所為慮。」元康初,封宣昌亭侯。
 誅楊駿,改封樂鄉侯,遷尚書。



 後以嶠博聞多識,屬書典實,有良史之志,轉秘書監,加散騎常侍,班同中書。寺為內臺,中書、散騎、著作及治禮音律,天文數術,南省文章,門下撰集,皆典統之。初,嶠以《漢紀》煩穢,慨然有改作之意。會為臺郎,典官制事,由是得遍觀祕籍,遂就其緒,起于光武,終於孝獻,一百九十五年,為帝紀十二卷、皇后紀二卷、十典十卷、傳七十卷及三譜、序傳、目錄,凡九十七卷。嶠以皇后配天作合,前史作外戚傳以繼末編,非其義也,故易為皇后紀,以次帝紀。又改志為典,以有《堯典》故也。而改名《漢後書》奏之。詔朝臣會議。時中書監荀
 勖、令和嶠、太常張華、侍中王濟咸以嶠文質事核,有遷固之規,實錄之風,藏之祕府。後太尉汝南王亮、司空衛瓘為東宮傅,列上通講,事遂施行。嶠所著論議難駁詩賦之屬數十萬言,其所奏官制、太子宜還宮及安邊、雩祭、明堂辟雍、浚導河渠,巡禹之舊跡置都水官,修蠶宮之禮置長秋,事多施行。元康三年卒,追贈少府,謚曰簡。



 嶠性嗜酒,率常沈醉。所撰書十典未成而終,秘書監何劭奏嶠中子徹為佐著作郎,使踵成之,未竟而卒。後監繆徵又奏嶠少子暢為佐著作郎,克成十典,并草魏、晉紀傳,與著作郎張載等俱在史官。永嘉喪亂,經籍遺沒,
 嶠書存者五十餘卷。



 嶠有三子:頤、徹、暢。頤嗣,官至長樂內史。暢有才思,所著文章數萬言。遭寇亂,避難荊州,為賊所害,時年四十。



 石鑒,字林伯,樂陵厭次人也。出自寒素,雅志公亮。仕魏,歷尚書郎、侍御史、尚書左丞、御史中丞,多所糾正,朝廷憚之,出為并州刺史、假節、護匈奴中郎將。武帝受禪,封堂陽子。入為司隸校尉,轉尚書。時秦、涼為虜所敗,遣鑒都督隴右諸軍事,坐論功虛偽免官。後為鎮南將軍、豫州刺史,坐討吳賊虛張首級。詔曰:「昔雲中守魏尚以斬
 首不實受刑,武牙將軍田順以詐增虜獲自殺,誣罔敗法,古今所疾。鑒備大臣,吾所取信。往者西事,公欺朝廷,以敗為得,竟不推究。中間黜免未久,尋復授用,冀能補過,而乃與下同詐。所謂大臣,義得爾乎!有司奏是也,顧未忍耳。今遣歸田里,終身不得復用,勿削爵土也。」久之,拜光祿勳,復為司隸校尉,稍加特進,遷右光祿大夫、開府,領司徒。前代三公冊拜,皆設小會,所以崇宰輔之制也。自魏末已後,廢不復行。至鑒,有詔令會,遂以為常。太康末,拜司空,領太子太傅。



 武帝崩,鑒與中護軍張劭監統山陵。時大司馬、汝南王亮為太傅楊駿所疑,不敢臨
 喪,出營城外。時有告亮欲舉兵討駿,駿大懼,白太后令帝為手詔,詔鑒及張劭使率陵兵討亮。劭,駿甥也,便率所領催鑒速發,鑒以為不然,保持之,遣人密覘視亮,已別道還許昌,於是駿止,論者稱之。山陵訖,封昌安縣侯。元康初,為太尉。年八十餘,克壯慷慨,自遇若少年,時人美之。尋薨,謚曰元。子陋,字處賤,襲封,歷屯騎校尉。



 溫羨,字長卿,太原祁人,漢護羌校尉序之後也。祖恢,魏揚州刺史。父恭,濟南太守。兄弟六人並知名於世,號曰「六龍」。羨少以朗寤見稱,齊王攸辟為掾,遷尚書郎。惠帝
 即位,拜豫州刺史,入為散騎常侍,累遷尚書。及齊王冏輔政,以羨攸之故吏,意特親之,轉吏部尚書。



 先是,張華被誅,冏建議欲復其官爵。論者或以為非,羨駁之曰:「自天子已下,爭臣各有差,不得歸罪於一人也。故晏子曰:『為已死亡,非其親暱,誰能任之?」里克之殺二庶,陳乞之立陽生,漢朝之誅諸呂,皆積年之後乃得立事。未有事主見存,而得行其志於數月之內者也。式乾之會,張華獨諫。上宰不和,不能承風贊善,望其指麾從命,不亦難乎!況今皇后譖害其子,內難不預,禮非所在。且后體齊於帝,尊同皇極,罪在枉子,事不為逆,義非所討。今以華
 不能廢枉子之后,與趙盾不討殺君之賊同,而貶責之,於義不經通也。」華竟得追復爵位。



 其後以從駕討成都王穎有勳,封大陵縣公,邑千八百戶。出為冀州刺史,加後將軍,范陽王虓敗於許昌也,自牧冀州,羨乃避之。惠帝之幸長安,以羨為中書令,不就。及帝還洛陽,徵為中書監,加散騎常侍。未拜,會帝崩。懷帝即位,遷左光祿大夫、開府,領司徒。論者僉謂為速。在位未幾,病卒,贈司徒,謚曰元。有三子:祗、允、裕。



 祗字敬齊,太傅西曹掾。允字敬咸,太子舍人。裕字敬嗣,尚武安長公主,官至左光祿大夫。



 史臣曰:晉氏中朝,承累世之資,建兼並之業,衣冠斯盛,英彥如林。此數公者,或以雅望處臺槐,或以高名居保傅,自非一時之秀,亦曷能至於斯。惜其參緘於論道之辰,獨善於兼濟之日,良圖鯁議,無足多談。然退已進賢,林叔弘推讓之美;自家刑國,宣伯協恭孝之規。子若之儒素為基,偉容之苦節流譽,慶垂來葉,不亦宜哉!石鑒以公亮升,溫羨以明寤顯,屬於危亂,不隕其名。歲寒見松柏之後凋,斯人之謂矣。



 贊曰:讓矣密陵,孝哉廣陸。欽既博雅,表亦貞肅。鑒績克宣,溫聲載穆。同鏘玉振,爭芬蘭鬱。



\end{pinyinscope}