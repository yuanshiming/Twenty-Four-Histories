\article{載記第十三 苻堅上}

\begin{pinyinscope}
苻堅
 \gezhu{
  上}



 苻堅,字永固,一名文玉,雄之子也。祖洪,從石季龍徙鄴,家于永貴里。其母茍氏嘗游漳水,祈子於西門豹祠,其夜夢與神交,因而有孕,十二月而生堅焉。有神光自天燭其庭。背有赤文,隱起成字,曰「草付臣又土王咸陽。」臂垂過膝,目有紫光。洪奇而愛之,名曰堅頭。年七歲,聰敏好施,舉止不踰規矩。每侍洪側,輒量洪舉措,取與不失
 機候。洪每曰:「此兒姿貌瑰偉,質性過人,非常相也。」高平徐統有知人之鑒,遇堅於路,異之,執其手曰:「苻郎,此官之御街,小兒敢戲于此,不畏司隸縛邪?」堅曰:「司隸縛罪人,不縛小兒戲也。」統謂左右曰:「此兒有霸王之相。」左右怪之,統曰:「非爾所及也。」後又遇之,統下車屏人,密謂之曰:「苻郎骨相不恒,後當大貴,但僕不見,如何!」堅曰:「誠如公言,不敢忘德。」八歲,請師就家學。洪曰:「汝戎狄異類,世知飲酒,今乃求學邪!」欣而許之。



 健之入關也,夢天神遣使者朱衣赤冠,命拜堅為龍驤將軍,健翌日為壇於曲沃以授之。健泣謂堅曰:「汝祖昔受此號,今汝復為神明
 所命,可不勉之!」堅揮劍捶馬,志氣感厲,士卒莫不憚服焉。性至孝,博學多才藝,有經濟大志,要結英豪,以圖緯世之宜。王猛、呂婆樓、強汪、梁平老等並有王佐之才,為其羽翼。太原薛贊、略陽權翼見而驚曰:「非常人也!」



 及苻生嗣偽位,贊、翼說堅曰:「今主上昏虐,天下離心。有德者昌,無德受殃,天之道也。神器業重,不可令他人取之,願君王行湯、武之事,以順天人之心。」堅深然之,納為謀主。生既殘虐無度,梁平老等亟以為言,堅遂弒生,以偽位讓其兄法。法自以庶孽,不敢當。堅及母茍氏並慮眾心未服,難居大位,群僚固請,乃從之。以升平元年僭稱大
 秦天王,誅生幸臣董龍、趙韶等二十餘人,赦其境內,改元曰永興。追謚父雄為文桓皇帝,尊母茍氏為皇太后,妻茍氏為皇后,子宏為皇太子。兄法為使持節、侍中、都督中外諸軍事、丞相、錄尚書,從祖侯為太尉,從兄柳為車騎大將軍、尚書令,封弟融為陽平公,雙河南公,子丕長樂公,暉平原公,熙廣平公,睿鉅鹿公。李威為衛將軍、尚書左僕射;梁平老為右僕射;強汪為領軍將軍;仇騰為尚書,領選;席寶為丞相長史、行太子詹事;呂婆樓為司隸校尉;王猛、薛贊為中書侍郎;權翼為給事黃門侍郎,與猛、讚並掌機密。追復魚遵、雷弱兒、毛貴、王墮、梁
 楞、梁安、段純、辛牢等本官,以禮改葬之,其子孫皆隨才擢授。初,堅母以法長而賢,又得眾心,懼終為變,至此,遣殺之。堅性仁友,與法決於東堂,慟哭嘔血,贈以本官,謚曰哀,封其子陽為東海公,敷為清河公。於是脩廢職,繼絕世,禮神祗,課農桑,立學校,鰥寡孤獨高年不自存者,賜穀帛有差,其殊才異行、孝友忠義、德業可稱者,令在所以聞。



 其將張平以并州叛,堅率眾討之,以其建節將軍鄧羌為前鋒,率騎五千據汾上。堅至銅壁,平盡眾拒戰,為羌所敗,獲其養子蠔,送之,平懼,乃降于堅。堅赦其罪,署為右將軍,蠔武賁中郎將,加廣武將軍,徙其所部
 三千餘戶于長安。



 堅自臨晉登龍門,顧謂其群臣曰:「美載山河之固!婁敬有言,『關中四塞之國』,真不虛也。」權翼、薛贊對曰:「臣聞夏、殷之都非不險也,周、秦之眾非不多也,終於身竄南巢,首懸白旗,軀殘於犬戎,國分於項籍昔何也?德之不脩故耳。吳起有言:『在德不在險。』深願陛下追蹤唐、虞,懷遠以德,山河之固不足恃也。」堅大悅,乃還長安。賜為父後者爵一級,鰥寡高年穀帛有差,丐所過田租之半。是秋,大旱,堅減膳撤懸,金玉綺繡皆散之戎士,後宮悉去羅紈,衣不曳地。開山澤之利,公私共之,偃甲息兵,與境內休息。



 王猛親寵愈密,朝政莫不由之。
 特進樊世,氐豪也,有大勳于苻氏,負氣倨傲,眾辱猛曰:「吾輩與先帝共興事業,而不預時權;君無汗馬之勞,何敢專管大任?是為我耕稼而君食之乎!」猛曰:「方當使君為宰夫,安直耕稼而已。」世大怒曰:「要當懸汝頭於長安城門,不爾者,終不處於世也。」猛言之於堅,堅怒曰:「必須殺此老氐,然後百僚可整。」俄而世入言事,堅謂猛曰:「吾欲以楊璧尚主,璧何如人也?」世勃然曰:「楊璧,臣之婿也,婚已久定,陛下安得令之尚主乎!」猛讓世曰:「陛下帝有海內,而君敢競婚,是為二天子,安有上下!」世怒起,將擊猛,左右止之。世遂醜言大罵,堅由此發怒,命斬之於西
 廄。諸氐紛紜,競陳猛短,堅恚甚,慢罵,或有鞭撻於殿庭者。權翼進曰:「陛下宏達大度,善馭英豪,神武卓犖,錄功舍過,有漢祖之風。然慢易之言,所宜除之。」堅笑曰:「朕之過也。」自是公卿以下無不憚猛焉。



 堅起明堂,繕南北郊,郊祀其祖洪以配天,宗祀其伯健於明堂以配上帝。親耕藉田,其妻茍氏親蠶于近郊。



 堅南游霸陵,顧謂群臣曰:「漢祖起自布衣,廓平四海,佐命功臣孰為首乎?」權翼進曰:「《漢書》以蕭、曹為功臣之冠。」堅曰:「漢祖與項羽爭天下,困於京索之間,身被七十餘創,通中六七,父母妻子為楚所囚。平城之下,七日不火食,賴陳平之謀,太上、妻
 子克全,免匈奴之禍。二相何得獨高也!雖有人狗之喻,豈黃中之言乎!」於是酣飲極歡,命群臣賦詩。大赦,復改元曰甘露。以王猛為侍中、中書令、京兆尹。



 其特進強德,健妻之弟也,昏酒豪橫,為百姓之患。猛捕而殺之,陳尸於市。其中丞鄧羌,性鯁直不撓,與猛協規齊志,數旬之間,貴戚強豪誅死者二十有餘人。於是百僚震肅,豪右屏氣,路不拾遺,風化大行。堅歎曰:「吾今始知天下之有法也,天子之為尊也!」於是遣使巡察四方及戎夷種落,州郡有高年孤寡,不能自存,長史刑罰失中、為百姓所苦,清脩疾惡、勸課農桑、有便於俗,篤學至孝、義烈力田
 者,皆令具條以聞。



 時匈奴左賢王衛辰遣使降於堅,遂請田內地,堅許之。雲中護軍賈雍遣其司馬徐斌率騎襲之,因縱兵掠奪。堅怒曰:「朕方脩魏絳和戎之術,不可以小利忘大信。昔荊吳之戰,事興蠶婦;澆瓜之惠,梁、宋息兵。夫怨不在大,事不在小,擾邊動眾,非國之利也。所獲資產,其悉以歸之。」免雍官,以白衣領護軍,遣使脩和,示之信義。辰於是入居塞內,貢獻相尋。烏丸獨孤、鮮卑沒奕於率眾數萬又降於堅。堅初欲處之塞內,苻融以「匈奴為患,其興自古。比虜馬不敢南首者,畏威故也。今處之於內地,見其弱矣,方當窺兵郡縣,為北邊之害。不
 如徙之塞外,以存荒服之義。」堅從之。



 堅僭位五年,鳳皇集于東闕,大赦其境內,百僚進位一級。初,堅之將為赦也,與王猛、苻融密議於露堂,悉屏左右。堅親為赦文,猛、融供進紙墨。有一大蒼蠅入自牖間,鳴聲甚大,集於筆端,驅而復來。俄而張安街巷市里人相告曰:「官今大赦。」有司以聞。堅驚謂融、猛曰:「禁中無耳屬之理,事何從泄也?」於是敕外窮推之,咸言有一小人衣黑衣,大呼於市曰:「官今大赦。」須臾不見。堅歎曰:「其向蒼蠅乎?聲狀非常,吾固惡之。諺曰:『欲人勿知,莫若勿為。』聲無細而弗聞,事未形而必彰者,其此之謂也。」堅廣脩學官,召郡國學生通
 一經以上充之,公卿已下子孫並遣受業。其有學為通儒、才堪幹事、清脩廉直、孝悌力田者,皆旌表之。於是人思勸勵,號稱多士,盜賊止息,請託路絕,田疇脩闢,帑藏充盈,典章法物靡不悉備。堅親臨太學,考學生經義優劣,品而第之。問難五經,博士多不能對。堅謂博士王實曰:「朕一月三臨太學,黜陟幽明,躬親獎勵,罔敢倦違,庶幾周、孔微言不由朕而墜,漢之二武其可追乎!」實對曰:「自劉石擾覆華畿,二都鞠為茂草,儒生罕有或存,墳籍滅而莫紀,經淪學廢,奄若秦皇。陛下神武撥亂,道隆虞、夏,開庠序之美,弘儒教之風,化盛隆周,垂馨千祀,漢
 之二武焉足論哉!」堅自是每月一臨太學,諸生競勸焉。



 屠各張罔聚眾數千,自稱大單于,寇掠郡縣。堅以其尚書鄧羌為建節將軍,率眾七千討平之。



 時商人趙掇、丁妃、鄒瓫等皆家累千金,車服之盛,擬則王侯,堅之諸公競引之為國二卿。黃門侍郎程憲言於堅曰:「趙掇等皆商販醜豎,市郭小人,車馬衣服僭同王者,官齊君子,為籓國列卿,傷風敗俗,有塵聖化,宜肅明典法,使清濁顯分。」堅於是推檢引掇等為國卿者,降其爵。乃下制:「非命士已上,不得乘車馬於都城百里之內。金銀錦繡,工商、皂隸、婦女不得服之,犯者棄市。」



 興寧三年,堅又改元為
 建元。慕容瑋遣其太宰慕容恪攻拔洛陽,略地至於崤、澠。堅懼其入關,親屯陜城以備之。



 匈奴右賢王曹轂、左賢王衛辰舉兵叛,率眾二萬攻其杏城已南郡縣,屯于馬蘭山。索虜烏延等亦叛堅而通于辰、轂。堅率中外精銳以討之,以其前將軍楊安、鎮軍毛盛等為前鋒都督。轂遣弟活距戰于同官川,安大敗之,斬活並四千餘級,轂懼而降。堅徙其酋豪六千餘戶於長安。進擊烏延,斬之。鄧羌討衛辰,擒之於木根山。堅自驄馬城如朔方,巡撫夷狄,以衛辰為夏陽公以統其眾。轂尋死,分其部落,貳城已西二萬餘落封其長子璽為駱川侯,貳城已東二
 萬餘落封其小子寅為力川侯,故號東、西曹。



 秦、雍二州地震裂,水泉湧出,金象生毛,長安大風震電,壞屋殺人,堅懼而愈修德政焉。



 使王猛、楊安等率眾二萬寇荊州北鄙諸郡,掠漢陽萬餘戶而還。羌斂岐叛堅,自稱益州刺史,率部落四千餘家西依張天錫叛將李儼。堅遣王猛與隴西太守姜衡、南安太守邵羌討斂岐于略陽。張天錫率步騎三萬擊李儼,攻其大夏、武始二郡,克之。天錫將掌據又敗儼諸軍于葵谷,儼懼,遣兄子純謝罪於堅,仍請救。尋而猛攻破略陽,斂岐奔白馬。堅遣楊安與建威王撫率眾會猛以救儼。猛遣邵羌追斂岐,使王撫
 守侯和,姜衡守白石。猛與楊安救枹罕,及天錫將楊遹戰于枹罕東,猛不利。邵羌擒斂岐於白馬,送之長安。天錫遂引師而歸。儼猶憑城未出,猛乃服白乘輿,從數十人,請與相見。儼開門延之,未及設備,而將士續入,遂虜儼而還。堅以其將軍彭越為平西將軍、涼州刺史,鎮枹罕。以儼為光祿勳、歸安侯。



 是歲,苻雙據上邽、苻柳據蒲阪叛于堅,苻庾據陜城、苻武據安定並應之,將共伐長安。堅遣使諭之,各齧梨以為信,皆不受堅命,阻兵自守。堅遣後禁將軍楊成世、左將軍毛嵩等討雙、武,王猛、鄧羌攻蒲阪,楊安、張蠔攻陜城。成世、毛嵩為雙、武所敗,堅
 又遣其武衛王鑒、寧朔呂光等率中外精銳以討之,左衛苻雅、左禁竇衝率羽林騎七千繼發。雙、武乘勝至於榆眉,鑒等擊敗之,斬獲萬五千人。武棄安定,隨雙奔上邽,鑒等攻之。苻柳出挑戰,猛閉壘不應。柳以猛為憚己,留其世子良守蒲阪,率眾二萬,將攻長安。長安去蒲阪百餘里,鄧羌率勁騎七千夜襲敗之,柳引軍還,猛又盡眾邀擊,悉俘其卒,柳與數百騎入於蒲阪。鑒等攻上邽,克之,斬雙、武。猛又尋破蒲阪,斬柳及其妻子,傳首長安。猛屯蒲阪,遣鄧羌與王鑒等攻陷陜城,克之,送庾于長安,殺之。



 太和四年,晉大司馬桓溫伐慕容瑋,次於枋頭。
 瑋眾屢敗,遣使乞師于堅,請割武牢以西之地。堅亦欲與瑋連橫,乃遣其將茍池等率步騎二萬救瑋。王師尋敗,引歸,池乃還。



 是時慕容垂避害奔于堅,王猛言於堅曰:「慕容垂,燕之戚屬,世雄東夏,寬仁惠下,恩結士庶,燕、趙之間咸有奉戴之意。觀其才略,權智無方,兼其諸子明毅有乾藝,人之傑也。蛟龍猛獸,非可馴之物,不如除之。」堅曰:「吾方以義致英豪,建不世之功。且其初至,吾告之至誠,今而害之,人將謂我何!」



 王師既旋,慕容瑋悔割武牢之地,遣使謂堅曰:「頃者割地,行人失辭。有國有家,分災救患,理之常也。」堅大怒,遣王猛輿建威梁成、鄧羌
 率步騎三萬,署慕容垂為冠軍將軍,以為鄉導,攻瑋洛州刺史慕容築於洛陽。瑋遣其將慕容臧率精卒十萬,將解築圍。猛使梁成等以精銳萬人卷甲赴之,大破臧於滎陽。築懼而請降,猛陳師以受之,留鄧羌鎮金墉,猛振旅而歸。



 太和五年,又遣猛率楊安、張蠔、鄧羌等十將率步騎六萬伐瑋。堅親送猛于霸東,謂曰:「今授卿精兵,委以重任,便可從壺關、上黨出潞川,此捷濟之機,所謂捷雷不及掩耳。吾當躬自率眾以繼卿後,於鄴相見。已敕運漕相繼,但憂賊,不煩後慮也。」猛曰:「臣庸劣孤生,操無豪介,蒙陛下恩榮,內侍帷幄,出總戎旅,藉宗廟之靈,
 稟陛下神算,殘胡不足平也。願不煩鑾軫,冒犯霜露。臣雖不武,望克不淹時。但願速敕有司,部置鮮卑之所。」堅大悅。於是進師。楊安攻晉陽。猛攻壺關,執瑋上黨太守慕容越,所經郡縣皆降于猛,猛留屯騎校尉茍萇戍壺關。會楊安攻晉陽,為地道,遣張蠔率壯士數百人入其城中,大呼斬關,猛、安遂入晉陽,執瑋並州刺史慕容莊。瑋遣其太傅慕容評率眾四十餘萬以救二城,評憚猛不敢進,屯於潞川。猛留將軍毛當戍晉陽,進師與評相持。遣游擊郭慶以銳卒五千,夜從間道出評營後,傍山起火,燒其輜重,火見鄴中。瑋懼,遣使讓評,催之速戰。猛
 知評賣水鬻薪,有可乘之會,評又求戰,乃陣於渭原而誓眾曰:「王景略受國厚恩,任兼內外,今與諸君深入賊地,宜各勉進,不可退也。願戮力行間,以報恩顧,受爵明君之朝,慶觴父母之室,不亦美乎!」眾皆勇奮,破釜棄糧,大呼競進。猛望評師之眾也,惡之,謂鄧羌曰:「今日之事,非將軍莫可以捷。成敗之機,在斯一舉。將軍其勉之!」羌曰:「若以司隸見與者,公無以為憂。」猛曰:「此非吾之所及也。必以安定太守、萬戶侯相處。」羌不悅而退。俄而兵交,猛召之,羌寢而弗應。猛馳就許之,羌於是大飲帳中,與張蠔、徐成等跨馬運矛,馳入評軍,出入數四,旁若無人,
 搴旗斬將,殺傷甚眾。及日中,評眾大敗,俘斬五萬有餘,乘勝追擊,又降斬十萬,於是進師圍鄴。堅聞之,留李威輔其太子宏守長安,以苻融鎮洛陽,躬率精銳十萬向鄴。七日而至於安陽,過舊閭,引諸耆老語及祖父之事,泫然流涕,乃停信宿。猛潛至安陽迎堅,堅謂之曰:「昔亞夫不出軍迎漢文,將軍何以臨敵而棄眾也?」猛曰:「臣每覽亞夫之事,嘗謂前卻人主,以此而為名將,竊未多之。臣奉陛下神算,擊垂亡之虜,若摧枯拉朽,何足慮也!監國沖幼,鑾駕遠臨,脫有不虞,其如宗廟何!」堅遂攻鄴,陷之。慕容瑋出奔高陽,堅將郭慶執而送之。堅入鄴宮,閱
 其名籍,幾郡百五十七,縣一千五百七十九,戶二百四十五萬八千九百六十九,口九百九十八萬七千九百三十五。諸州郡牧守及六夷渠帥盡降於堅。郭慶窮追餘燼,慕容評奔于高句麗,慶追至遼海,句麗縛評送之。堅散瑋宮人珍寶以賜將士,論功封賞各有差。以王猛為使持節、都督關東六州諸軍事、車騎大將軍、開府儀同三司、冀州牧、鎮鄴;以郭慶為持節、都督幽州諸軍事、揚武將軍、幽州刺史,鎮薊。



 堅自鄴如枋頭,宴諸父老,改枋頭為永昌縣,復之終世。堅至自永昌,行飲至之禮,歌勞止之詩,以饗其群臣。赦慕容瑋及其王公已下,皆徙
 於長安,封授有差。堅於是行禮于辟雍,祀先師孔子,其太子及公侯卿大夫士之元子,皆束脩釋奠焉。徙關東豪傑及諸雜夷十萬戶于關中,處烏丸雜類于馮翊、北地,丁零翟斌于新安,徙陳留、東阿萬戶以實青州。諸因亂流移,避仇遠徙,欲還舊業者,悉聽之。



 晉叛臣袁瑾固守壽春,為大司馬桓溫所圍,遣使請救于堅。堅遣王鑒、張蠔率步騎二萬救之,鑒據洛澗,蠔屯八公山。桓溫遣諸將夜襲鑒、蠔,敗之,鑒、蠔屯慎城。



 初,仇池氐楊世以地降于堅,堅署為平南將軍、秦州刺史、仇池公。既而歸順於晉。世死,子纂代立,遂受天子爵命而絕於堅。世弟統
 驍武得眾,起兵武都,與纂分爭。堅遣其將苻雅、楊安與益州刺史王統率步騎七萬,先取仇池,進圖寧、益。雅等次于鷲陜,纂率眾五萬距雅。晉梁州刺史楊亮遣督護郭寶率騎千餘救之,戰於陜中,為雅等所敗,纂收眾奔還。雅進攻仇池,楊統帥武都之眾降於雅。纂將楊他遣子碩密降于雅,請為內應。纂懼,面縛出降。雅釋其縛,送之長安。以楊統為平遠將軍、南秦州刺史,加楊安都督,鎮仇池。



 先是,王猛獲張天錫將敦煌陰據及甲士五千,堅既東平六州,西擒楊纂,欲以德懷遠,且跨威河右,至是悉送所獲還涼州。天錫懼而遣使謝罪稱籓,堅大悅,即署天錫
 為使持節、散騎常侍、都督河右諸軍事、驃騎大將軍、開府儀同三司、涼州刺史、西域都護、西平公。



 吐谷渾碎奚以楊纂既降,懼而遣使送馬五千匹、金銀五百斤。堅拜奚安遠將軍、漒川侯。



 堅嘗如鄴,狩于西山,旬餘,樂而忘返。伶人王洛叩馬諫曰:「臣聞千金之子坐不垂堂,萬乘之主行不履危。故文帝馳車,袁公止轡;孝武好田,相如獻規。陛下為百姓父母,蒼生所繫,何可盤于游田,以玷聖德。若禍起須臾,變在不測者,其如宗廟何!其如太后何!」堅曰:「善。昔文公悟愆於虞人,朕聞罪於王洛,吾過也。」自是遂不復獵。



 堅聞桓溫廢海西公也,謂群臣曰:「溫前
 敗灞上,後敗枋頭,十五年間,再傾國師。六十歲公舉動如此,不能思愆免退,以謝百姓,方廢君以自悅,將如四海何!諺云『怒其室而作色於父』者,其桓溫之謂乎!」



 堅以境內旱,課百姓區種。懼歲不登,省節穀帛之費,太官、後官減常度二等,百僚之秩以次降之。復魏、晉士籍,使役有常,聞諸非正道,典學一皆禁之。堅臨太學,考學生經義,上第擢敘者八十三人。自永嘉之亂,庠序無聞,及堅之僭,頗留心儒學,王猛整齊風俗,政理稱舉,學校漸興。關、隴清晏,百姓豐樂,自長安至于諸州,皆夾路樹槐柳,二十里一亭,四十里一驛,旅行者取給於途,工商貿販
 於道。百姓歌之曰:「長安大街,夾樹楊槐。下走朱輪,上有鸞栖。英彥雲集,誨我萌黎。」



 是歲,有大風從西南來,俄而晦冥,恒星皆見,又有赤星見于西南。太史令魏延言於堅曰:「於占西南國亡,明年必當平蜀漢。」堅大悅,命秦梁密嚴戎備。乃以王猛為丞相,以苻融為鎮東大將軍。代猛為冀州牧。融將發,堅祖於霸東,奏樂賦詩。堅母茍氏以融少子,甚愛之,比發,三至灞上,其夕又竊如融所,內外莫知。是夜,堅寢于前殿,魏延上言:「天市南門屏內后妃星失明,左右閽寺不見,后妃移動之象。」堅推問知之,驚曰:「天道與人何其不遠!」遂重星官。王猛至長安,加都
 督中外諸軍事,猛辭讓再三,堅不許。



 其後天鼓鳴,有彗星出于尾箕,長十餘丈,名蚩尤旗,經太微,掃東井,自夏及秋冬不滅。太史令張孟言於堅曰:「彗起尾箕,而掃東井,此燕滅秦之象。」因勸堅誅慕容瑋及其子弟。堅不納,更以瑋為尚書,垂為京兆尹,沖為平陽太守。苻融聞之,上疏於堅曰:「臣聞東胡在燕,歷數彌久,逮于石亂,遂據華夏,跨有六州,南面稱帝。陛下爰命六師,大舉征討,勞卒頻年,勤而後獲,非慕義懷德歸化。而今父子兄弟列官滿朝,執權履職,勢傾勞舊,陛下親而幸之。臣愚以為猛獸不可養,狼子野心。往年星異,災起於燕,願少留意,
 以思天戒。臣據可言之地,不容默已。《詩》曰:『兄弟急難』,『朋友好合』。昔劉向以肺腑之親,尚能極言,況於臣乎!」堅報之曰:「汝為德未充而懷是非,立善未稱而名過其實。《詩》云:『德輶如毛,人鮮克舉。』君子處高,戒懼傾敗,可不務乎!今四海事曠,兆庶未寧,黎元應撫,夷狄應和,方將混六合以一家,同有形於赤子,汝其息之,勿懷耿介。夫天道助順,修德則禳災。茍求諸己,何懼外患焉。」



 晉梁州刺史楊亮遣子廣襲仇池,與堅將楊安戰,廣敗績,晉沮水諸戍皆委城奔潰,亮懼而退守磬險,安遂進寇漢川。堅遣王統、朱彤率卒二萬為前鋒寇蜀,前禁將軍毛當、鷹揚
 將軍徐成率步騎三萬入自劍閣。楊亮率巴獠萬餘拒之,戰于青谷,王師不利,亮奔固西城。彤乘勝陷漢中,徐成又攻二劍,克之,楊安進據梓潼。晉奮威將軍、西蠻校尉周虓降于彤。揚武將軍、益州刺史周仲孫勒兵距彤等于綿竹,聞堅將毛當將至成都,仲孫率騎五千奔于南中。安、當進兵,遂陷益州。於是西南夷邛、莋、夜郎等皆歸之。堅以安為右大將軍、益州牧,鎮成都;毛當為鎮西將軍、梁州刺史,鎮漢中;姚萇為寧州刺史、領西蠻校尉;王統為南秦州刺史,鎮仇池。



 蜀人張育、楊光等起兵,與巴獠相應,以叛於堅。晉益州刺史竺瑤、威遠將軍桓石
 虔率眾三萬據墊江。育乃自號蜀王,遣使歸順,與巴獠酋帥張重、尹萬等五萬餘人進圍成都。尋而育與萬爭權,舉兵相持,堅遣鄧羌與楊安等擊敗之,育、光退屯綿竹。安又敗張重、尹萬于成都南,重死之,及首級二萬三千。鄧羌復擊張育、楊光于綿竹,皆害之。桓石虔敗姚萇于墊江,萇退據五城,石虔與竺瑤移屯巴東。



 時有人於堅明光殿大呼謂堅曰:「甲申乙酉,魚羊食人,悲哉無復遺。」堅命執之,俄而不見。秘書監朱彤等因請誅鮮卑,堅不從。遣使巡行四方,觀風俗,問政道,明黜陟,恤孤獨不能自存者。以安車蒲輪徵隱士樂陵王歡為國子祭酒。
 及王猛卒,堅置聽訟觀於未央之南。禁《老》、《莊》、圖讖之學。中外四禁、二衛、四軍長上將士,皆令修學。課後宮,置典學,立內司,以授于掖庭,選閹人及女隸有聰識者署博士以授經。



 遣其武衛茍萇、左將軍毛盛、中書令梁熙、步兵校尉姚萇等率騎十三萬伐張天賜于姑臧。遣尚書朗閻負、梁殊銜命軍前,下書徵天錫。堅嚴飾鹵簿,親餞萇等於城西,賞行將各有差。又遣其秦州刺史茍池、河州刺史李辯、涼州刺史王統,率三州之眾以繼之。閻負等到涼州,天錫自以晉之列籓,志在保境,命斬之,遣將軍馬建出距萇等。俄而梁熙、王統等自清石津攻其將
 梁粲於河會城,陷之。茍萇濟自石城津,與梁熙等會攻纏縮城,又陷之。馬建懼,自楊非退還清塞。天錫又遣將軍掌據率眾三萬,與馬建陣于洪池。茍萇遣姚萇以甲卒三千挑戰,諸將勸據擊之,以挫其鋒,據不從。天錫乃率中軍三萬次金昌。萇、熙聞天錫來逼,急攻據、建,建降於萇,遂攻據,害之,及其軍司席仂。萇進軍入清塞,乘高列陣。天錫又遣司兵趙充哲為前鋒,率勁勇五萬,與萇等戰于赤岸,哲大敗。天錫懼而奔還,至箋請降。萇至姑臧,天錫乘素車白馬,面縛輿櫬,降于軍門。萇釋縛焚櫬,送之于長安,諸郡縣悉降。堅以梁熙為持節、西中郎將、
 涼州刺史,領護西羌校尉,鎮姑臧。徙豪右七千餘戶于關中,五品稅百姓金銀一萬三千斤以賞軍士,餘皆安堵如故。堅封天錫重光縣之東寧鄉二百戶,號歸義侯。初,萇等將征天錫,堅為其立第於長安,至是而居之。



 堅既平涼州,又遣其安北將軍、幽州刺史苻洛為北討大都督,率幽州兵十萬討代王涉翼犍。又遣後將軍俱難與鄧羌等率步騎二十萬東出和龍,西出上郡,與洛會于涉翼犍庭。翼犍戰敗,遁于弱水。苻洛逐之,勢窘迫,退還陰山。其子翼圭縛父請降,洛等振旅而還,封賞有差。堅以翼犍荒俗,未參仁義,令入太學習禮。以翼圭執父
 不孝,遷之於蜀。散其部落於漢鄣邊故地,立尉、監行事,官僚領押,課之治業營生,三五取丁,優復三年無稅租。其渠帥歲終令朝獻,出入行來為之制限。堅嘗之太學,召涉翼犍問曰:「中國以學養性,而人壽考,漠北啖牛羊而人不壽,何也?」翼犍不能答。又問:「卿種人有堪將者,可召為國家用。」對曰:「漠北人能捕六畜,善馳走,逐水草而已,何堪為將!」又問:「好學否?」對曰:「若不好學,陛下用教臣何為?」堅善其答。



 堅以關中水旱不時,議依鄭白故事,發其王侯已下及豪望富室僮隸三萬人,開涇水上源,鑿山起堤,通渠引瀆,以溉岡鹵之田。及春而成,百姓賴其
 利。以涼州新附,復租賦一年。為父後者賜爵一級,孝悌力田爵二級,孤寡高年穀帛有差,女子百戶牛酒,大酺三日。



 遣其尚書令苻丕率司馬慕容瑋、茍萇等步騎七萬寇襄陽。使楊安將樊鄧之眾為前鋒,屯騎校尉石越率精騎一萬出魯陽關,募容垂與姚萇出自南鄉,茍池等與彊駑王顯將勁卒四萬從武當繼進,大會漢陽。師次沔北,晉南中郎將朱序以丕軍無舟楫,不以為虞,石越遂游馬以渡。序大懼,固守中城。越攻陷外郛,獲船百餘艘以濟軍。丕率諸將進攻中城,遣茍池、石越、毛當以眾五萬屯于江陵。晉車騎將軍桓沖擁眾七萬為序聲
 援,憚池等不進,保據上明。兗州刺史彭超遣使上言於堅曰:「晉沛郡太守戴逯以卒數千戍彭城,臣請率精銳五萬攻之,願更遣重將討淮南諸城。」堅於是又遣其後將軍俱難率右將軍毛當、後禁毛盛、陵江邵保等步騎七萬寇淮陰、盱眙。揚武彭超寇鼓城。梁州刺史韋鐘寇魏興,攻太守吉挹于西城。晉將軍毛武生率眾五萬距之,與俱難等相持于淮南。



 先是,梁熙遣使西域,稱揚堅之威德,并以繒彩賜諸國王,於是朝獻者十有餘國。大宛獻天馬千里駒,皆汗血、朱鬣、五色、鳳膺、麟身,及諸珍異五百餘種。堅曰:「吾思漢文之返千里馬,咨嗟美詠。今
 所獻馬,其悉反之,庶克念前王,仿佛古人矣。」乃命群臣作《止馬詩》而遣之,示無欲也。其下以為盛德之事,遠同漢文,於是獻詩者四百餘人。



 是時苻丕久圍襄陽,御史中丞李柔劾丕以師老無功,請徵下廷尉。堅曰:「丕等費廣無成,實宜貶戮。但師已淹時,不可虛然中返,其特原之,令以功成贖罪。」因遣其黃門郎韋華持節切讓丕等,仍賜以劍,曰:「來春不捷者,汝可自裁,不足復持面見吾也。」初,丕之寇襄陽也,將急攻之,茍萇諫曰:「今以十倍之眾,積粟如山,但掠徙荊、楚之人內於許、洛,絕其糧運,使外援不接,糧盡無人,不攻自潰,何為促攻以傷將士之命?」丕
 從之。及堅讓至,眾咸疑懼,莫知所為。征南主簿河東王施進曰:「以大將軍英秀,諸將勇銳,以攻小城,何異洪爐燎羽毛。所以緩攻,欲以計制之。若決一旦之機,可指日而定。今破襄陽,上明自遁,復何所疑!願請一旬之期,以展三軍之勢。如其不捷,施請為戮首。」丕於是促圍攻之。堅將親率眾助丕等,使苻融將關東甲卒會于壽春,梁熙統河西之眾以繼中軍。融、熙並上言,以為未可興師,乃止。



 太元四年,晉兗州刺史謝玄率眾數萬次于泗汭,將救彭城。苻丕陷襄陽,執南中郎將朱序,送于長安,堅署為度支尚書。以其中壘梁成為南中郎將、都督荊、揚
 州諸軍事、荊州刺史,領護南蠻校尉,配兵一萬鎮襄陽,以征南府器杖給之。彭超圍彭城也,置輜重于留城。至是,晉將謝玄遣將軍何謙之、高衡率眾萬餘,聲趣留城,超引軍赴之。戴逯率彭城之眾奔於謝玄,超留其治中徐褒守彭城而復寇盱眙。俱難既陷淮陰,留邵保戍之,與超會師而南。晉將毛武生救魏興,遣前鋒督護趙福、將軍袁虞等將水軍一萬,溯江而上。堅南巴校尉姜宇遣將張紹、仇生等水陸五千距之,戰于南縣,王師敗績。尋而韋鐘攻陷魏興,執太守吉挹。毛當與王顯自襄陽而東,會攻淮南。彭超陷盱眙,獲晉建威將軍、高密內史
 毛璪之,遂攻晉幽州刺史田洛于三阿,去廣陵百里,京都大震,臨江列戍。孝武帝遣征虜將軍謝石率水軍次于塗中,右衛將軍毛安之、游擊將軍河間王曇之次于堂邑,謝玄自廣陵救三阿。毛當、毛盛馳襲安之,王師敗績。玄率眾三萬次于白馬塘,俱難遣其將都顏率騎逆玄,戰于塘西,玄大敗之,斬顏。玄進兵至三阿,與難、超戰,超等又敗,退保盱眙。玄進次石梁,與田洛攻盱眙,難、超出戰,復敗,退屯淮陰。玄遣將軍何謙之、督護諸葛侃率舟師乘潮而上,焚淮橋,又與難等合戰,謙之斬其將邵保,難、超退師淮北。難歸罪彭超,斬其司馬柳渾。堅聞之,
 大怒,檻車徵超下獄,超自殺,難免為庶人。



 堅以毛當為平南將軍、徐州刺史,鎮彭城;毛盛為平東將軍、兗州刺史,鎮胡陸;王顯為平吳校尉、揚州刺史,戍下邳:賞堂邑之功也。又以苻洛為散騎常侍、持節、都督益、寧、西南夷諸軍事、征南大將軍、益州牧,領護西夷校尉,鎮成都,命從伊闕自襄陽溯漢而上。洛,健之兄子也。雄勇多力,而猛氣絕人、堅深忌之,故常為邊牧。洛有征伐之功而未賞,及是遷也,恚怒,謀於眾曰:「孤於帝室,至親也,主上不能以將相任孤,常擯孤於外,既投之西裔,復不聽過京師,此必有伏計,令梁成沈孤於漢水矣。為宜束手就命,
 為追晉陽之事以匡社稷邪?諸君意如何?」其治中平顏妄陳祥瑞,勸洛舉兵。洛因攘袂大言曰:「孤計決矣,沮謀者斬!」於是自稱大將軍、大都督、秦王,署置官司,以平顏輔國將軍、幽州刺史,為其謀主。分遣使者徵兵於鮮卑、烏丸、高句麗、百濟及薛羅、休忍等諸國,並不從。洛懼而欲止,平顏曰:「且宜聲言受詔,盡幽、并之兵出自中山、常山,陽平公必郊迎於路,因而執之,進據冀州,總關東之眾以圖秦、雍,可使百姓不覺易主而大業定矣。」洛從之,乃率眾七萬發和龍,將圖長安。於是關中騷動,盜賊並起。堅遣使數之曰:「天下未一家,兄弟匪他,何為而反?
 可還和龍,當以幽州永為世封。」洛謂使者曰:「汝還白東海王,幽州褊阨,不足容萬乘,須還王咸陽,以承高祖之業。若能候駕潼關者,位為上公,爵歸本國。」堅大怒,遣其左將軍竇衝及呂光率步騎四萬討之,右將軍都貴馳傳詣鄴,率冀州兵三萬為前鋒,以苻融為大都督,授之節度。使石越率騎一萬,自東萊出石徑,襲和龍,海行四百餘里。苻重亦盡薊城之眾會洛,次于中山,有眾十萬。衝等與洛戰于中山,大敗之,執洛及其將蘭殊,送于長安。呂光追斬苻重於幽州,石越克和龍,斬平顏及其黨與百餘人。堅赦蘭殊,署為將軍,徙洛于涼州,徵苻融為
 車騎大將軍、領宗正、錄尚書事。



 洛既平,堅以關東地廣人殷,思所以鎮靜之,引其群臣于東堂議曰:「凡我族類,支胤彌繁,今欲分三原、九嵕、武都、汧、雍十五萬戶於諸方要鎮,不忘舊德,為磐石之宗,於諸君之意如何?」皆曰:「此有周所以祚隆八百,社稷之利也。」於是分四帥子弟三千戶,以配苻丕鎮鄴,如世封諸侯,為新券主。堅送丕於灞上,流涕而別。諸戎子弟離其父兄者,皆悲號哀慟,酸感行人,識者以為喪亂流離之象。於是分幽州置平州,以石越為平州刺史,領護鮮卑中郎將,鎮龍城;大鴻臚韓胤領護赤沙中郎將,移烏丸府于代郡之平城;中
 書令梁讜為安遠將軍、幽州刺史,鎮薊城;毛興為鎮西將軍、河州刺史,鎮枹罕;王騰為鷹揚將軍、並州刺史,領護匈奴中郎將,鎮晉陽;二州各配支戶三千;苻暉為鎮東大將軍、豫州牧,鎮洛陽;苻睿為安東將軍、雍州刺史,鎮蒲阪。



 先是,高陸人穿井得龜,大三尺,背有八卦文,堅命太卜池養之,食以粟,及此而死,藏其骨於太廟。其夜廟丞高虜夢龜謂之曰:「我本出將歸江南,遭時不遇,隕命秦庭。」又有人夢中謂虜曰:「龜三千六百歲而終,終必妖興,亡國之徵也。」



 堅自平諸國之後,國內殷實,遂示人以侈,懸珠簾于正殿,以朝群臣,宮宇車乘,器物服御,悉以
 珠璣、瑯玕、奇寶、珍怪飾之。尚書郎裴元略諫曰:「臣聞堯、舜茅茨,周卑宮室,故致和平,慶隆八百。始皇窮極奢麗,嗣不及孫。願陛下則采椽之不琢,鄙瓊室而不居,敷純風於天下,流休範於無窮,賤金玉,珍穀帛,勤恤人隱,勸課農桑,捐無用之器,棄難得之貨,敦至道以厲薄俗,修文德以懷遠人。然後一軌九州,同風天下,刑措既登,告成東嶽,蹤軒皇以齊美,哂二漢之徙封,臣之願也。」堅大悅,命去珠簾,以元略為諫議大夫。



 鄯善王、車師前部王來朝,大宛獻汗血馬,肅慎貢楛矢,天竺獻火浣布,康居、于闐及海東諸國,凡六十有二王,皆遣使貢其方物。



 初,
 堅母少寡,將軍李威有闢陽之寵,史官載之。至是,堅收起居注及著作所錄而觀之,見其事,慚怒,乃焚其書,大檢史官,將加其罪。著作郎趙泉、車敬等已死,乃止。



 荊州刺史都貴遣其司馬閻振、中兵參軍吳仲等率眾二萬寇竟陵,留輜重於管城,水陸輕進。桓沖遣南平太守桓石虔、竟陵太守郭銓等水陸二萬距之,相持月餘,戰於滶水。振等大敗,退保管城。石虔乘勝攻破之,斬振及仲,俘斬萬七千。



\end{pinyinscope}