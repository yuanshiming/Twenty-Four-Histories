\article{列傳第十六}

\begin{pinyinscope}

 劉
 頌李重



 劉頌,字子雅,廣陵人,漢廣陵厲王胥之後也。世為名族。同郡有雷、蔣、穀、魯四姓,皆出其下,時人為之語曰「雷、蔣、穀、魯,劉最為祖。」父觀,平陽太守。頌少能辨物理,為時人所稱。察孝廉,舉秀才,皆不就。文帝辟為相府掾,奉使于蜀。時蜀新平,人飢土荒,頌表求振貸,不待報而行,由是除名。武帝踐阼,拜尚書三公郎,典科律,申冤訟。累遷中
 書侍郎。咸寧中,詔頌與散騎郎白褒巡撫荊、揚,以奉使稱旨,轉黃門郎。遷議郎,守廷尉。時尚書令史扈寅非罪下獄,詔使考竟,頌執據無罪,寅遂得免,時人以頌比張釋之。在職六年,號為詳平。會滅吳,諸將爭功,遣頌校其事,以王渾為上功,王濬為中功。帝以頌持法失理,左遷京兆太守,不行,轉任河內。臨發,上便宜,多所納用。郡界多公主水碓,遏塞流水,轉為浸害,頌表罷之,百姓獲其便利。尋以母憂去職。服闋,除淮南相。在官嚴整,甚有政績。舊修芍陂,年用數萬人,豪彊兼並,孤貧失業,頌使大小戮力,計功受分,百姓歌其平惠。



 頌在郡,上疏曰:



 臣昔
 忝河內,臨辭受詔:「卿所言悉要事,宜大小數以聞。恒苦多事,或不能悉有報,勿以為疑。」臣受詔之日,喜懼交集,益思自竭,用忘其鄙,願以螢燭,增暉重光。到郡草具所陳如左,未及書上,會臣嬰丁天罰,寢頓累年,今謹封上前事。臣雖才不經國,言淺多違,猶願陛下垂省,使臣微誠得經聖鑒,不總棄於常案。如有足採,冀補萬一。



 伏見詔書,開啟土宇,以支百世,封建戚屬,咸出之籓,夫豈不懷,公理然也。樹國全制,始成於今,超秦、漢、魏氏之局節,紹五帝三代之絕跡。功被無外,光流後裔,巍巍盛美,三五之君殆有慚德。何則?彼因自然而就之,異乎絕跡之
 後更創之。雖然,封幼稚皇子於吳、蜀,臣之愚慮,謂未盡善。夫吳、越剽輕,庸、蜀險絕,此故變釁之所出,易生風塵之地。且自吳平以來,東南六州將士更守江表,此時之至患也。又內兵外守,吳人有不自信之心,宜得壯主以鎮撫之,使內外各安其舊。又孫氏為國,文武眾職,數擬天朝,一旦堙替,同於編戶。不識所蒙更生之恩,而災困逼身,自謂失地,用懷不靖。今得長王以臨其國,隨才授任,文武並敘,士卒百役不出其鄉,求富貴者取之於國內。內兵得散,新邦乂安,兩獲其所,於事為宜。宜取同姓諸王年二十以上人才高者,分王吳、蜀。以其去近就遠,
 割裂土宇,令倍於舊。以徙封故地,用王幼稚,須皇子長乃遣君之,於是無晚也。急所須地,交得長主,此事宜也。臣所陳封建,今大義已舉,然餘眾事,儻有足採,以參成制,故皆並列本事。



 臣聞:不憚危悔之患,而願獻所見者,盡忠之臣也;垂聽逆耳,甘納苦言者,濟世之君也。臣以期運,幸遇無諱之朝。雖嘗抗疏陳辭,氾論政體,猶未悉所見,指言得失,徒荷恩寵,不異凡流。臣竊自愧,不盡忠規,無以上報,謹列所見如左。臣誠未自許所言必當,然要以不隱所懷為上報之節。若萬一足採,則微臣更生之年;如皆瞽妄,則國之福也。願陛下缺半日之間,垂省
 臣言。



 伏惟陛下雖應天順人,龍飛踐阼,為創基之主,然所遇之時,實是叔世。何則?漢末陵遲,閹豎用事,小人專朝,君子在野,政荒眾散,遂以亂亡。魏武帝以經略之才,撥煩理亂,兼肅文教,積數十年,至于延康之初,然後吏清下順,法始大行。逮至文、明二帝,奢淫驕縱,傾殆之主也。然內盛臺榭聲色之娛,外當三方英豪嚴敵,事成克舉,少有愆違,其故何也?實賴前緒,以濟勳業。然法物政刑,固已漸頹矣。自嘉平之初,晉祚始基,逮于咸熙之末,其間累年。雖鈇鉞屢斷,翦除凶醜,然其存者咸蒙遭時之恩,不軌於法。泰始之初,陛下踐阼,其所服乘皆先代
 功臣之胤,非其子孫,則其曾玄。古人有言,膏粱之性難正,故曰時遇叔世。當此之秋,天地之位始定,四海洗心整綱之會也。然陛下猶以用才因宜,法寬有由,積之在素,異於漢、魏之先。三祖崛起,易朝之為,未可一旦直繩御下,誠時宜也。然至所以為政,矯世眾務,自宜漸出公塗,法正威斷,日遷就肅。譬由行舟,雖不橫截迅流,然俄向所趣,漸靡而往,終得其濟。積微稍著,以至于今,可以言政。而自泰始以來,將三十年,政功美績,未稱聖旨,凡諸事業,不茂既往。以陛下明聖,猶未及叔世之弊,以成始初之隆,傳之後世,不無慮乎!意者,臣言豈不少概聖
 心夫!



 顧惟萬載之事,理在二端。天下大器,一安難傾,一傾難正。故慮經後世者,必精目下之政,政安遺業,使數世賴之。若乃兼建諸侯而樹籓屏,深根固蒂,則祚延無窮,可以比跡三代。如或當身之政,遺風餘烈不及後嗣,雖樹親戚,而成國之制不建,使夫後世獨任智力以安大業。若未盡其理,雖經異時,憂責猶追在陛下,將如之何!願陛下善當今之政,樹不拔之勢,則天下無遺憂矣。



 夫聖明不世及,後嗣不必賢,此天理之常也。故善為天下者,任勢而不任人。任勢者,諸侯是也;任人者,郡縣是也。郡縣之察,小政理而大勢危;諸侯為邦,近多違而遠
 慮固。聖王推終始之弊,,權輕重之理,包彼小違以據大安,然後足以籓固內外,維鎮九服。夫武王聖主也,成王賢嗣也,然武王不恃成王之賢而廣封建者,慮經無窮也。且善言今者,必有驗之於古。唐、虞以前,書文殘缺,其事難詳。至於三代,則並建明德,及興王之顯親,列爵五等,開國承家,以籓屏帝室,延祚久長,近者五六百歲,遠者僅將千載。逮至秦氏,罷侯置守,子弟不分尺土,孤立無輔,二世而亡。漢承周、秦之後,雜而用之,前後二代各二百餘年。揆其封建不用,雖彊弱不適,制度舛錯,不盡事中,然跡其衰亡,恒在同姓失職,諸侯微時,不在彊盛。
 昔呂氏作亂,幸賴齊、代之援,以寧社稷。七國叛逆,梁王捍之,卒弭其難。自是之後,威權削奪,諸侯止食租奉,甚者至乘牛車。是以王莽得擅本朝,遂其姦謀,傾蕩天下,毒流生靈。光武紹起,雖封樹子弟,而不建成國之制,祚亦不延。魏氏承之,圈閉親戚,幽囚子弟,是以神器速傾,天命移在陛下。長短之應,禍福之徵,可見於此。又魏氏雖正位居體,南面稱帝,然三方未賓,正朔有所不加,實有戰國相持之勢。大晉之興,宣帝定燕,太祖平蜀,陛下滅吳,可謂功格天地,土廣三王,舟車所至,人迹所及,皆為臣妾,四海大同,始於今日。宜承大勳之籍,及陛下聖
 明之時,開啟土宇,使同姓必王,建久安於萬載,垂長世於無窮。



 臣又聞國有任臣則安,有重臣則亂。而王制,人君立子以嫡不以長,立嫡以長不以賢,此事情之不可易者也。而賢明至少,不肖至眾,此固天理之常也。物類相求,感應而至,又自然也。是以暗君在位,則重臣盈朝;明后臨政,則任臣列職。夫任臣之與重臣,俱執國統而立斷者也。然成敗相反,邪正相背,其故何也?重臣假所資以樹私,任臣因所籍以盡公。盡公者,政之本也;樹私者,亂之源也。推斯言之,則泰日少,亂日多,政教漸頹,欲國之無危,不可得也。又非徒唯然而已。借令愚劣之嗣,
 蒙先哲之遺緒,得中賢之佐,而樹國本根不深,無幹輔之固,則所謂任臣者化而為重臣矣。何則?國有可傾之勢,則執權者見疑,眾疑難以自信,而甘受死亡者非人情故也。若乃建基既厚,籓屏彊禦,雖置幼君赤子而天下不懼,曩之所謂重臣者,今悉反忠而為任臣矣。何則?理無危勢,懷不自猜,忠誠得著,不惕於邪故也。聖王知賢哲之不世及,故立相持之勢以御其臣。是以五等既列,臣無忠慢,同於竭節,以徇其上。群后既建,繼體賢鄙,亦均一契,等於無慮。且樹國茍固,則所任之臣,得賢益理,次委中智,亦足以安。何則?勢固易持故也。



 然則建邦
 茍盡其理,則無向不可。是以周室自成、康以下,逮至宣王,宣王之後,到于赧王,其間歷載,朝無名臣,而宗廟不隕者,諸侯維持之也。故曰,為社稷計,莫若建國。夫邪正逆順者,人心之所繫服也。今之建置,宜審量事勢,使諸侯率義而動,同忿俱奮,令其力足以維帶京邑。若包藏禍心,惕於邪而起,孤立無黨,所蒙之籍不足獨以有為。然齊此甚難,陛下宜與達古今善識事勢之士深共籌之。建侯之理,使君樂其國,臣榮其朝,各流福祚,傳之無窮。上下一心,愛國如家,視百姓如子,然後能保荷天祿,兼翼王室。今諸王裂土,皆兼於古之諸侯,而君賤其爵,
 臣恥其位,莫有安志,其故何也?法同郡縣,無成國之制故也。今之建置,宜使率由舊章,一如古典。然人心繫常,不累十年,好惡未改,情願未移。臣之愚慮,以為宜早創大制,遲迴眾望,猶在十年之外,然後能令君臣各安其位,榮其所蒙,上下相持,用成籓輔。如今之為,適足以虧天府之藏,徒棄穀帛之資,無補鎮國衛上之勢也。



 古者封建既定,各有其國,後雖王之子孫,無復尺土,此今事之必不行者也。若推親疏,轉有所廢,以有所樹,則是郡縣之職,非建國之制。今宜豫開此地,令十世之內,使親者得轉處近。十世之遠,近郊地盡,然後親疏相維,不得
 復如十世之內。然猶樹親有所,遲天下都滿,已彌數百千年矣。今方始封而親疏倒施,甚非所宜。宜更大量天下土田方里之數,都更裂土分人,以王同姓,使親疏遠近不錯其宜,然後可以永安。古者封國,大者不過土方百里,然後人數殷眾,境內必盈其力,足以備充制度。今雖一國周環近將千里,然力實寡,不足以奉國典。所遇不同,故當因時制宜,以盡事適今。宜令諸王國容少而軍容多,然於古典所應有者悉立其制,然非急所須,漸而備之,不得頓設也。須車甲器械既具,群臣乃服彩章;倉廩已實,乃營宮室;百姓已足,乃備官司;境內充實,乃
 作禮樂。唯宗廟社稷,則先建之。至於境內之政,官人用才,自非內史、國相命於天子,其餘眾職及死生之斷、穀帛資實、慶賞刑威、非封爵者,悉得專之。今臣所舉二端,蓋事之大較,其所不載,應在二端之屬者,以此為率。今諸國本一郡之政耳,若備舊典,則官司以數,事所不須,而以虛制損實力。至於慶賞刑斷,所以衛下之權,不重則無以威眾人而衛上。故臣之愚慮,欲令諸侯權具,國容少而軍容多,然亦終於必備今事為宜。



 周之建侯,長享其國,與王者並,遠者僅將千載,近者猶數百年;漢之諸王,傳祚暨至曾玄。人性不甚相遠,古今一揆,而短長
 甚違,其故何邪?立意本殊而制不同故也。周之封建,使國重於君,公侯之身輕於社稷,故無道之君不免誅放。敦興滅繼絕之義,故國祚不泯。不免誅放,則群后思懼;胤嗣必繼,是無亡國也。諸侯思懼,然後軌道,下無亡國,天子乘之,理勢自安,此周室所以長在也。漢之樹置君國,輕重不殊,故諸王失度,陷於罪戮,國隨以亡。不崇興滅繼絕之序,故下無固國。下無固國,天子居上,勢孤無輔,故姦臣擅朝,易傾大業。今宜反漢之弊,修周舊跡。國君雖或失道,陷於誅絕,又無子應除,茍有始封支胤,不問遠近,必紹其祚。若無遺類,則虛建之,須皇子生,以繼
 其統,然後建國無滅。又班固稱「諸侯失國亦猶網密」,今又宜都寬其檢。且建侯之理,本經盛衰,大制都定,班之群后,著誓丹青,書之玉版,藏之金匱,置諸宗廟,副在有司。寡弱小國猶不可危,豈況萬乘之主!承難傾之邦而加其上,則自然永久居重固之安,可謂根深華嶽而四維之也。臣之愚,願陛下置天下於自安之地,寄大業於固成之勢,則可以無遺憂矣。



 今閻閭少名士,官司無高能,其故何也?清議不肅,人不立德,行在取容,故無名士。下不專局,又無考課,吏不竭節,故無高能。無高能,則有疾世事;少名士,則後進無準,故臣思立吏課而肅清議。
 夫欲富貴而惡貧賤,人理然也。聖王大諳物情,知不可去,故直同公私之利,而詭其求道,使夫欲富者必先由貧,欲貴者必先安賤。安賤則不矜,不矜然後廉恥厲;守貧者必節欲,節欲然後操全。以此處務,乃得盡公。盡公者,富貴之徒也。為無私者終得其私,故公私之利同也。今欲富者不由貧自得富,欲貴者不安賤自得貴,公私之塗既乖,而人情不能無私,私利不可以公得,則恒背公而橫務。是以風節日頹,公理漸替,人士富貴,非軌道之所得。以此為政,小大難期。然教頹來既久,難反一朝。又世放都靡,營欲比肩,群士渾然,庸行相似,不可頓肅,
 甚殊黜陟也。且教不求盡善,善在抑尤,同侈之中,猶有甚泰。使夫昧適情之樂者,捐其顯榮之貴,俄在不鮮之地;約己潔素者,蒙儉德之報,列于清官之上。二業分流,令各有蒙。然俗放都奢,不可頓肅,故臣私慮,願先從事於漸也。



 天下至大,萬事至眾,人君至少,同於天日,故非垂聽所得周覽。是以聖王之化,執要而已,委務於下而不以事自嬰也。分職既定,無所與焉,非憚日昃之勤,而牽於逸豫之虞,誠以政體宜然,事勢致之也。何則?夫造創謀始,逆闇是非,以別能否,甚難察也。既以施行,因其成敗,以分功罪,甚易識也。易識在考終,難察在造始,故
 人君恒居其易則安,人臣不處其難則亂。今陛下每精事始而略於考終,故群吏慮事懷成敗之懼輕,飾文采以避目下之譴重,此政功所以未善也。今人主能恒居易執要以御其下,然後人臣功罪形於成敗之徵,無逃其誅賞。故罪不可蔽,功不可誣。功不可誣,則能者勸;罪不可蔽,則違慢日肅,此為國之大略也。臣竊惟陛下聖心,意在盡善,懼政有違,故精事始,以求無失。又以眾官勝任者少,故不委務,寧居日昃也。臣之愚慮,竊以為今欲盡善,故宜考終。何則?精始難校故也。又群官多不勝任,亦宜委務,使能者得以成功,不能者得以著敗。敗著
 可得而廢,功成可得遂任,然後賢能常居位以善事,闇劣不得以尸祿害政。如此不已,則勝任者漸多,經年少久,即群司遍得其人矣。此校才考實,政之至務也。今人主不委事仰成,而與諸下共造事始,則功罪難分。下不專事,居官不久,故能否不別。何以驗之?今世士人決不悉良能也,又決不悉疲軟也。然今欲舉一忠賢,不知所賞;求一負敗,不知所罰。及其免退,自以犯法耳,非不能也。登進者自以累資及人間之譽耳,非功實也。若謂不然,則當今之政未稱聖旨,此其徵也。陛下御今法為政將三十年,而功未日新,其咎安在?古人有言:「琴瑟不調,
 甚者必改而更張。」凡臣所言,誠政體之常,然古今異宜,所遇不同。陛下縱未得盡仰成之理,都委務於下,至如今事應奏御者,蠲除不急,使要事得精可三分之二。



 古者六卿分職,冢宰為師。秦、漢已來,九列執事,丞相都總。今尚書制斷,諸卿奉成,於古制為重,事所不須,然今未能省並。可出眾事付外寺,使得專之,尚書為其都統,若丞相之為。惟立法創制,死生之斷,除名流徙,退免大事,及連度支之事,臺乃奏處。其餘外官皆專斷之,歲終臺閤課功校簿而已。此為九卿造創事始,斷而行之,尚書書主,賞罰繩之,其勢必愈考成司非而已。於今親掌者
 動受成於上,上之所失,不得復以罪下,歲終事功不建,不知所責也。夫監司以法舉罪,獄官案劾盡實,法吏據辭守文,大較雖同,然至於施用,監司與夫法獄體宜小異。獄官唯實,法吏唯文,監司則欲舉大而略小。何則?夫細過微闕,謬妄之失,此人情之所必有,而悉糾以法,則朝野無全人,此所謂欲理而反亂者也。



 故善為政者綱舉而網疏,綱舉則所羅者廣,網疏則小必漏,所羅者廣則為政不苛,此為政之要也。而自近世以來,為監司者,類大綱不振而微過必舉。微過不足以害政,舉之則微而益亂;大綱不振,則豪彊橫肆,豪彊橫肆,則百姓失職
 矣,此錯所急而倒所務之由也。今宜令有司反所常之政,使天下可善化。及此非難也,人主不善碎密之案,必責犯彊舉尤之奏,當以盡公,則害政之姦自然禽矣。夫大姦犯政而亂兆庶之罪者,類出富彊,而豪富者其力足憚,其貨足欲,是以官長顧勢而頓筆。下吏縱姦,懼所司之不舉,則謹密網以羅微罪。使奏劾相接,狀似盡公,而撓法不亮固已在其中矣。非徒無益於政體,清議乃由此而益傷。古人有言曰:「君子之過,如日之蝕焉。」又曰:「過而能改」又曰「不貳過」。凡此數者,皆是賢人君子不能無過之言也。茍不至於害政,則皆天網之所漏;所犯在
 甚泰,然後王誅所必加,此舉罪淺深之大例者也。



 故君子得全美以善事,不善者必夷戮以警眾,此為政誅赦之準式也。何則?所謂賢人君子,茍不能無過,小疵不可以廢其身,而輒繩以法,則愧於明時。何則?雖有所犯,輕重甚殊,於士君子之心受責不同而名不異者,故不軌之徒得引名自方,以惑眾聽,因名可亂,假力取直,故清議益傷也。凡舉過彈違,將以肅風論而整世教,今舉小過,清議益頹。是以聖人深識人情而達政體,故其稱曰:「不以一眚掩大德。」又曰:「赦小過,舉賢才。」又曰:「無求備於一人。」故冕而前旒,充纊塞耳,意在善惡之報必取其尤,
 然後簡而不漏,大罪必誅,法禁易全也。何則?害法在犯尤,而謹搜微過,何異放兕豹於公路,而禁鼠盜於隅隙。古人有言,「鈇鉞不用而刀鋸日弊,不可以為政」,此言大事緩而小事急也。時政所失,少有此類,陛下宜反而求之,乃得所務也。



 夫權制不可以經常,政乖不可以守安,此言攻守之術異也。百姓雖愚,望不虛生,必因時而發。有因而發,則望不可奪;事變異前,則時不可違。明聖達政,應赴之速,不及下車,故能動合事機,大得人情。昔魏武帝分離天下,使人役居戶,各在一方;既事勢所須,且意有曲為,權假一時,以赴所務,非正典也。然逡巡至今,
 積年未改,百姓雖身丁其困,而私怨不生,誠以三方未悉蕩並,知時未可以求安息故也。是以甘役如歸,視險若夷。至於平吳之日,天下懷靜,而東南二方,六州郡兵,將士武吏,戍守江表,或給京城運漕,父南子北,室家分離,咸更不寧。又不習水土,運役勤瘁,並有死亡之患,勢不可久。此宜大見處分,以副人望。魏氏錯役,亦應改舊。此二者各盡其理,然黔首感恩懷德,謳吟樂生必十倍於今也。自董卓作亂以至今,近出百年,四海勤瘁,丁難極矣。六合渾並,始於今日,兆庶思寧,非虛望也。然古今異宜,所遇不同,誠亦未可以希遵在昔,放息馬牛。然
 使受百役者不出其國,兵備待事其鄉,實在可為。縱復不得悉然為之,茍盡其理,可靜三分之二,吏役可不出千里之內。但如斯而已,天下所蒙已不訾矣。



 政務多端,世事之未盡理者,難遍以疏舉,振領總綱,要在三條。凡政欲靜,靜在息役,息役在無為。倉廩欲實,實在利農,利農在平糴。為政欲著信,著信在簡賢,簡賢在官久。官久非難也,連其班級,自非才宜,不得傍轉以終其課,則事善矣。平糴已有成制,其未備者可就周足,則穀積矣。無為匪他,卻功作之勤,抑似益而損之利。如斯而已,則天下靜矣。此三者既舉,雖未足以厚化,然可以為安有餘
 矣。夫王者之利,在生天地自然之財,農是也。所立為指於此,事誠有功益。茍或妨農,皆務所息,此悉似益而損之謂也。然今天下自有事所必須,不得止已,或用功甚少而所濟至重。目下為之,雖少有廢,而計終已大益。農官有十百之利,及有妨害,在始似如未急,終作大患,宜逆加功,以塞其漸。如河、汴將合,沈萊茍善,則役不可息。諸如此類,亦不得已已。然事患緩急,權計輕重,自非近如此類,準以為率,乃可興為,其餘皆務在靜息。然能善算輕重,權審其宜,知可興可廢,甚難了也,自非上智遠才,不乾此任。夫創業之美,勳在垂統,使夫後世蒙賴以
 安。其為安也,雖昏猶明,雖愚若智。濟世功者,實在善化之為,要在靜國。至夫修飾宮署,凡諸作役務為恆傷過泰,不患不舉,此將來所不須於陛下而自能者也。至於仰蒙前緒,所憑日月者,實在遺風繫人心,餘烈匡幼弱,而今勤所不須,以傷所憑。鈞此二者,何務孰急,陛下少垂恩迴慮,詳擇所安,則大理盡矣。



 世之私議,竊比陛下於孝文。臣以為聖德隆殺,將在乎後,不在當今。何則?陛下龍飛鳳翔,應期踐阼,有創業之勛矣。掃滅彊吳,奄征南海,又有之矣。以天子之貴,而躬行布衣之所難,孝儉之德,冠于百王,又有之矣。履宜無細,動成軌度,又有之
 矣。若善當身之政,建籓屏之固,使晉代久長,後世仰瞻遺跡,校功考事,實與湯、武比隆,何孝文足云!臣之此言,非臣下褒上虛美常辭,其事實然。若所以資為安之理,或未盡善,則恐良史書勛,不得遠盡弘美,甚可惜也。然不可使夫知政之士得參聖慮,經年少久,終必有成。願陛下少察臣言。



 又論肉刑,見《刑法志》。詔答曰:「得表陳封國之制,宜如古典,任刑齊法,宜復肉刑,及六州將士之役,居職之宜,諸所陳聞,具知卿之乃心為國也。動靜數以聞。」



 元康初,從準南王允入朝。會誅楊駿,頌屯衛殿中,其夜,詔以頌為三公尚書。又上疏論律令事,為時論所
 美。久之,轉吏部尚書,建九班之制,欲令百官居職希遷,考課能否,明其賞罰。賈郭專朝,仕者欲速,竟不施行。



 及趙王倫之害張華也,頌哭之甚慟。聞華子得逃,喜曰:「茂先,卿尚有種也!」倫黨張林聞之,大怒,憚頌持正而不能害也。孫秀等推崇倫功,宜加九錫,百僚莫敢異議。頌獨曰:「昔漢之錫魏,魏之錫晉,皆一時之用,非可通行。今宗廟乂安,雖嬖后被退,勢臣受誅,周勃誅諸呂而尊孝文,霍光廢昌邑而奉孝宣,並無九錫之命。違舊典而習權變,非先王之制。九錫之議,請無所施。」張林積忿不已,以頌為張華之黨,將害之。孫秀曰:「誅張、裴已傷時望,不可
 復誅頌。」林乃止。於是以頌為光祿大夫,門施行馬。尋病卒,使使者弔祭,賜錢二十萬、朝服一具,謚曰貞。中書侍郎劉沈議,頌當時少輩,應贈開府。孫秀素恨之,不聽。頌無子,養弟和子雍早卒,更以雍弟詡子焉為嫡孫,襲封。永康元年,詔以頌誅賈謐督攝眾事有功,追封梁鄒縣侯,食邑千五百戶。



 頌弟彪字仲雅,參安東軍事。伐吳,獲張悌,累官積弩將軍。及武庫火,彪建計斷屋,得出諸寶器。歷荊州刺史。次弟仲字世混,歷黃門郎、滎陽太守,未之官,卒。



 初,頌嫁女臨淮陳矯,矯本劉氏子,與頌近親,出養於姑,改姓陳氏。中正劉友譏之,頌曰:「舜後姚虞、陳田
 本同根系,而世皆為婚,禮律不禁。今與此同義,為婚可也。」友方欲列上,為陳騫所止,故得不劾。頌問明法掾陳默、蔡畿曰:「鄉里誰最屈?」二人俱云:「劉友屈。」頌作色呵之,畿曰:「友以私議冒犯明府為非,然鄉里公論稱屈。」友辟公府掾、尚書郎、黃沙御史。



 李重字茂曾,江夏鐘武人也。父景,秦州刺史、都亭定侯。重少好學,有文辭;早孤,與群弟居,以友愛著稱。弱冠為本國中正,遜讓不行。後為始平王文學,上疏陳九品曰:「先王議制,以時因革,因革之理,唯變所適。九品始於喪
 亂,軍中之政,誠非經國不刊之法也。且其檢防轉碎,徵刑失實,故朝野之論,僉謂驅動風俗,為弊已甚。而至於議改,又以為疑。臣以革法創制,當先盡開塞利害之理,舉而錯之,使體例大通而無否滯亦未易故也。古者諸侯之治,分土有常,國有定主,人無異望,卿大夫世祿,仕無出位之思,臣無越境之交,上下體固,人德歸厚。秦反斯道,罷侯置守,風俗淺薄,自此來矣。漢革其弊,斟酌周、秦,並建侯守,亦使分土有定,而牧司必各舉賢,貢士任之鄉議,事合聖典,比蹤三代。方今聖德之隆,光被四表,兆庶顒顒,欣睹太平。然承魏氏凋弊之跡,人物播越,仕
 無常朝,人無定處,郎吏蓄於軍府,豪右聚於都邑,事體駁錯,與古不同。謂九品既除,宜先開移徙,聽相並就。且明貢舉之法,不濫於境外,則冠帶之倫將不分而自均,即土斷之實行矣。又建樹官司,功在簡久。階級少,則人心定;久其事,則政化成而能否著,此三代所以直道而行也。以為選例九等,當今之要,所宜施用也。聖王知天下之難,常從事於其易,故寄隱括於閭伍,則邑屋皆為有司。若任非所由,事非所核,則雖竭聖智,猶不足以贍其事。由此而觀,誠令二者既行,即人思反本,修之於鄉,華競自息,而禮讓日隆矣。」



 遷太子舍人,轉尚書郎。時太
 中大夫恬和表陳便宜,稱漢孔光、魏徐幹等議,使王公已下制奴婢限數,及禁百姓賣田宅。中書啟可,屬主者為條制。重奏曰:「先王之制,士農工商有分,不遷其業,所以利用厚生,各肆其力也。《周官》以土均之法,經其土地井田之制,而辨其五物九等貢賦之序,然後公私制定,率土均齊。自秦立阡陌,建郡縣,而斯制已沒。降及漢、魏,因循舊跡,王法所峻者,唯服物車器有貴賤之差,令不僭擬以亂尊卑耳。至於奴婢私產,則實皆未嘗曲為之立限也。八年《己巳詔書》申明律令,諸士卒百工以上,所服乘皆不得違制。若一縣一歲之中,有違犯者三家,洛
 陽縣十家已上,官長免。如詔書之旨,法制已嚴。今如和所陳而稱光、乾之議,此皆衰世踰侈,當時之患。然盛漢之初不議其制,光等作而不行,非漏而不及,能而不用也。蓋以諸侯之軌既滅,而井田之制未復,則王者之法不得制人之私也。人之田宅既無定限,則奴婢不宜偏制其數,懼徒為之法,實碎而難檢。方今聖明垂制,每尚簡易,法禁已具,和表無施。」



 又司隸校尉石鑒奏,鬱林太守介登役使所監,求召還;尚書荀愷以為遠郡非人情所樂,奏登貶秩居官。重駁曰:「臣聞立法無制,所以齊眾檢邪,非必曲尋事情,而理無所遺也。故所滯者寡,而所
 濟者眾。今如登郡比者多,若聽其貶秩居官,動為準例,懼庸才負遠,必有黷貨之累,非所以肅清王化,輯寧殊域也。臣愚以為宜聽鑒所上,先召登還,且使體例有常,不為遠近異制。」詔從之。



 太熙初,遷廷尉平。駁廷尉奏邯鄲醉等,文多不載。再遷中書郎,每大事及疑議,輒參以經典處決,多皆施行。遷尚書吏部郎,務抑華競,不通私謁,特留心隱逸,由是群才畢舉。拔用北海西郭湯、琅邪劉珩、燕國霍原、馮翊吉謀等為祕書郎及諸王文學,故海內莫不歸心。時燕國中正劉沈舉霍原為寒素,司徒府不從,沈又抗詣中書奏原,而中書復下司徒參論。司
 徒左長史荀組以為:「寒素者,當謂門寒身素,無世祚之資。原為列侯,顯佩金紫,先為人間流通之事,晚乃務學,少長異業,年踰始立,草野之譽未洽,德禮無聞,不應寒素之目。」重奏曰:「案如《癸酉詔書》,廉讓宜崇,浮競宜黜。其有履謙寒素靖恭求己者,應有以先之。如詔書之旨,以二品繫資,或失廉退之士,故開寒素以明尚德之舉。司徒總御人倫,實掌邦教,當務峻準評,以一風流。然古之厲行高尚之士,或棲身巖穴,或隱跡丘園,或克己復禮,或耄期稱道,出處默語,唯義所在。未可以少長異操,疑其所守之美,而遠同終始之責,非所謂擬人必於其倫
 之義也。誠當考之於邦黨之倫,審之於任舉之主。沈為中正,親執銓衡。陳原隱居求志,篤古好學,學不為利,行不要名,絕迹窮山,韞韣道藝,外無希世之容,內全遁逸之節,行成名立,搢紳慕之,委質受業者千里而應,有孫、孟之風,嚴、鄭之操。始舉原,先諮侍中、領中書監華,前州大中正、後將軍嬰,河南尹軼。去三年,諸州還朝,幽州刺史許猛特以原名聞,擬之西河,求加徵聘。如沈所列,州黨之議既舉,又刺史班詔表薦,如此而猶謂草野之譽未洽,德禮無聞,舍所征檢之實,而無明理正辭,以奪沈所執。且應二品,非所求備。但原定志窮山,修述儒道,義
 在可嘉。若遂抑替,將負幽邦之望,傷敦德之教。如詔書所求之旨,應為二品。」詔從之。



 重與李毅同為吏部郎,時王戎為尚書,重以清尚見稱,毅淹通有智識,雖二人操異,然俱處要職,戎以識會待之,各得其所。毅字茂彥,舊史闕其行事。于時內官重,外官輕,兼階級繁多,重議之,見《百官志》。又上疏曰:「凡山林避寵之士,雖違世背時,出處殊軌,而先王許之者,嘉其服膺高義也。昔先帝患風流之弊,而思反純朴,乃諮詢朝眾,搜求隱逸。咸寧二年,始以太子中庶子征安定皇甫謐,四年又以博士征南安朱沖,太康元年,復以太子庶子征沖,雖皆以病疾不
 至,而朝野悅服。陛下遠邁先帝禮賢之旨,臣訪沖州邑,言其雖年近耋耋,而志氣克壯,耽道窮藪,老而彌新,操尚貞純,所居成化,誠山棲耆德,足以表世篤俗者也。臣以為宜垂聖恩,及其未沒,顯加優命。」時朝廷政亂,竟不能從。出為行討虜護軍、平陽太守,崇德化,修學校,表篤行,拔賢能,清簡無欲,正身率下,在職三年,彈黜四縣。弟嶷亡,表去官。



 永康初,趙王倫用為相國左司馬,以憂逼成疾而卒,時年四十八。家貧,宅宇狹小,無殯斂之地,詔於典客署營喪。追贈散騎常侍,謚曰成。子式,有美名,官至侍中,咸和初卒。



 史臣曰:子雅束發登朝,竭誠奉國,廣陳封建,深中機宜,詳辨刑名,該核政體。雖文慚華婉,而理歸切要。游目西京,望賈誼而非遠;眷言東國,顧郎顗而有餘。逮元康之間,賊臣專命,舉朝戰慄,茍避菹醢;頌以此時,忠鯁不撓,哭張公之非罪,拒趙王之妄錫,雖古遺直,何以尚茲。至於緣其私議,不平劉友,異夫憎而知善,舉不避仇者歟!李重言因革之理,駁田產之制,詞愜事當,蓋亹癖可觀。及銳志銓衡,留心隱逸,浚沖期之識會,豈虛也哉!



 贊曰:劉頌剛直,義形於詞。自下摩上,彼實有之。李重清雅,志乃無私。推賢拔滯,嘉言在茲。懋哉兩哲,邦家之基。



\end{pinyinscope}