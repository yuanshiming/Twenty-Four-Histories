\article{載記第十二 苻洪苻健苻生}

\begin{pinyinscope}

 苻
 洪苻健苻生苻雄王墮



 苻洪,字廣世,略陽臨渭氐人也。其先蓋有扈之苗裔,世為西戎酋長。始其家池中蒲生,長五丈,五節如竹形,時咸謂之蒲家,因以為氏焉。父懷歸,部落小帥。先是,隴右大雨,百姓苦之,謠曰:「雨若不止,洪水必起。」故因名曰洪。好施,多權略,驍武善騎射。屬永嘉之亂,乃散千金,召英傑之士訪安危變通之術。宗人蒲光、蒲突遂推洪為盟
 主。劉曜僭號長安,光等逼洪歸曜,拜率義侯。曜敗,洪西保隴山。石季龍將攻上邽,洪又請降。季龍大悅,拜冠軍將軍,委以西方之事。季龍滅石生,洪說季龍宜徙關中豪傑及羌戎內實京師。季龍從之,以洪為龍驤將軍、流人都督,處于枋頭。累有戰功,封西平郡公,其部下賜爵關內侯者二千餘人,以洪為關內領侯將。冉閔言於季龍曰:「苻洪雄果,其諸子並非常才,宜密除之。」季龍待之愈厚。及石遵即位,閔又以為言,遵乃去洪都督,餘如前。洪怨之,乃遣使降晉。後石鑒殺遵,所在兵起,洪有眾十餘萬。



 永和六年,帝以洪為征北大將軍、都督河北諸軍
 事、冀州刺史、廣川郡公。時有說洪稱尊號者,洪亦以讖文有「草付應王」,又其孫堅背有「草付」字,遂改姓苻氏,自稱大將軍、大單于、三秦王。洪謂博士胡文曰:「孤率眾十萬,居形勝之地,冉閔、慕容俊可指辰而殄,姚襄父子克之在吾數中,孤取天下,有易於漢祖。」初,季龍以麻秋鎮枹罕,冉閔之亂,秋歸鄴,洪使子雄擊而獲之,以秋為軍師將軍。秋說洪西都長安,洪深然之。既而秋因宴鴆洪,將并其眾,世子健收而斬之。洪將死,謂健曰:「所以未入關者,言中州可指時而定。今見困豎子,中原非汝兄弟所能辦。關中形勝,吾亡後便可鼓行而西。」言終而死,年
 六十六。健僭位,偽謚惠武帝。



 苻健,字建業,洪第三子也。初,母姜氏夢大羆而孕之,及長,勇果便弓馬,好施,善事人,甚為石季龍父子所親愛。季龍雖外禮苻氏,心實忌之,乃陰殺其諸兄,而不害健也。及洪死,健嗣位,去秦王之號,稱晉爵,遣使告喪于京師,且聽王命。



 時京兆杜洪竊據長安,自稱晉征北將軍、雍州刺史,戎夏多歸之。健密圖關中,懼洪知之,乃偽受石祗官,繕宮室於枋頭,課所部種麥,示無西意,有知而不種者,健殺之以徇。既而自稱晉征西大將軍、都督關
 中諸軍事、雍州刺史,盡眾西行,起浮橋於盟津以濟。遣其弟雄率步騎五千入潼關,兄子菁自軹關入河東。健執菁手曰:「事若不捷,汝死河北,我死河南,不及黃泉,無相見也。」既濟,焚橋,自統大眾繼雄而進。杜洪遣其將張先要健於潼關,健逆擊破之。健雖戰勝,猶修箋于洪,并送名馬珍寶,請至長安上尊號。洪曰:「幣重言甘,誘我也。」乃盡召關中之眾來距。健筮之,遇《泰》之《臨》,健曰:「小往大來,吉亨。昔往東而小,今還西而大,吉孰大焉!」是時眾星夾河西流,占者以為百姓還西之象。健遂進軍,次赤水,遣雄略地渭北,又敗張先於陰槃,擒之,諸城盡陷,菁所
 至無不降者,三輔略定。健引兵至長安,洪奔司竹。健入而都之,遣使獻捷京師,并修好於桓溫。



 健軍師將軍賈玄碩等表健為侍中、大都督關中諸軍事、大單于、秦王,健怒曰:「我官位輕重,非若等所知。」既而潛使諷玄碩等使上尊號。永和七年,僭稱天王、大單于,赦境內死罪,建元皇始,繕宗廟社稷,置百官於長安。立妻強氏為天王皇后,子萇為天王皇太子,弟雄為丞相、都督中外諸軍事、車騎大將軍、領雍州刺史,自餘封授各有差。



 初,杜洪之奔也,招晉梁州刺史司馬勳。至是,勳率步騎三萬入秦川,健敗之於五丈原。



 八年,健僭即皇帝位于太極前
 殿,諸公進為王,以大單于授其子萇。



 杜洪屯宜秋,為其將張琚所殺,琚自立為秦王,置百官。健率步騎二萬攻琚,斬其首。健至自宜秋,遣雄、菁率眾掠關東,并援石季龍豫州刺史張遇於許昌,與晉鎮西將軍謝尚戰于潁水之上,王師敗績。雄乘勝逐北,至于壘門,殺傷太半,遂虜遇及其眾歸于長安,拜遇司空、豫州刺史,鎮許昌。雄攻王擢於隴上,擢奔涼州,雄屯隴東。張重華拜擢征東大將軍,使與其將張弘、宋修連兵伐雄。雄與菁率眾擊敗之,獲弘、修送長安。



 初,張遇自許昌來降,健納遇後母韓氏為昭儀,每於眾中謂遇曰:「卿,吾子也。」遇慚恨,引關
 中諸將欲以雍州歸順,乃與健中黃門劉晃謀夜襲健,事覺,遇害。於是孔特起池陽,劉珍、夏侯顯起鄠,喬景起雍,胡陽赤起司竹,呼延毒起霸城,眾數萬人,並遣使詣征西桓溫、中軍殷浩請救。



 雄遣菁掠上洛郡,於豐陽縣立荊州,以引南金奇貨、弓竿漆蠟,通關市,來遠商,於是國用充足,而異賄盈積矣。



 十年,溫率眾四萬趨長安,遣別將入淅川,攻上洛,執健荊州刺史郭敬,而遣司馬勛掠西鄙。健遣其子萇率雄、菁等眾五萬,距溫于堯柳城、愁思堆。溫轉戰而前,次于灞上,萇等退營城南。健以羸兵六千固守長安小城,遣精銳三萬為游軍以距溫。三
 輔郡縣多降于溫。健別使雄領騎七千,與桓沖戰於白鹿原,王師敗績,又破司馬勳于子午谷。初,健聞溫之來也,收麥清野以待之,故溫眾大飢。至是,徙關中三千餘戶而歸。及至潼關,又為萇等所敗,司馬勳奔還漢中。



 其年,西虜乞沒軍邪遣子入侍,健於是置來賓館於平朔門以懷遠人。起靈臺於杜門。與百姓約法三章,薄賦卑宮,垂心政事,優禮耆老,修尚儒學,而關右稱來蘇焉。



 新平有長人見,語百姓張靖曰:「苻氏應天受命,今當太平,外面者歸中而安泰。」問姓名,弗答,俄而不見。新平令以聞,健以為妖,下靖獄。會大雨霖,河、渭溢,蒲津監冠登得
 一屐於河,長七盡三寸,人跡稱之,指長尺餘,文深一寸。健歎曰:「覆載之中何所不有,張靖所見定不虛也。」赦之。蝗蟲大起,自華澤至隴山,食百草無遺。牛馬相敢毛,猛獸及狼食人,行路斷絕。健自蠲百姓租稅,減膳撤懸,素服避正殿。



 初,桓溫之入關也,其太子萇與溫戰,為流矢所中死。至是,立其子生為太子。健寢疾,菁勒兵入東宮,將殺苻生自立。時生侍健疾,菁以健為死,回攻東掖門。健聞變,升端門陳兵,眾皆舍杖逃散,執菁殺之。數日,健死,時年三十九,在位四年。偽謚明皇帝,廟號世宗,後改曰高祖。



 生字長生,健第三子也。幼而無賴,祖洪甚惡之。生無一目,為兒童時,洪戲之,問侍者曰:「吾聞瞎兒一淚,信乎?」侍者曰:「然。」生怒,引佩刀自刺出血,曰:「此亦一淚也。」洪大驚,鞭之。生曰:「性耐刀槊,不堪鞭捶。」洪曰:「汝為爾不已,吾將以汝為奴。」生曰:「可不如石勒也。」洪懼,跣而掩其口,謂健曰:「此兒狂勃,宜早除之,不然,長大必破人家。」健將殺之,雄止之曰:「兒長成自當修改,何至便可如此!」健乃止。及長,力舉千鈞,雄勇好殺,手格猛獸,走及奔馬,擊刺騎射,冠絕一時。桓溫之來伐也,生單馬入陣,搴旗斬將者前
 後十數。



 萇既死,健以讖言三羊五眼應符,故立為太子。健卒,僭即皇帝位,大赦境內,改年壽光,時永和十二年也。尊其母強氏為皇太后,立妻梁氏為皇后。以呂婆樓為侍中、左大將軍,苻安領太尉,苻柳為征東大將軍、并州牧,鎮蒲阪,苻謏為鎮東大將軍、豫州牧,鎮陜城,自餘封授有差。



 初,生將強懷與桓溫戰沒,其子延未及封而健死。會生出游,懷妻樊氏於道上書,論懷忠烈,請封其子。生怒,射而殺之。偽中書監胡文、中書令王魚言於生曰:「比頻有客星孛于大角,熒惑入於東井。大角為帝坐,東井秦之分野,於占,不出三年,國有大喪,大臣戮死。願
 陛下遠追周文,修德以禳之,惠和群臣,以成康哉之美。」生曰:「皇后與朕對臨天下,亦足發塞大喪之變。毛太傅、梁車騎、梁僕射受遺輔政,可謂大臣也。」於是殺其妻梁氏及太傅毛貴,車騎、尚書令梁楞,左僕射梁安。未凡,又誅侍中、丞相雷弱兒及其九子、二十七孫。諸羌悉叛。弱兒,南安羌酋也,剛鯁好直言,見生嬖臣趙韶、董榮亂政,每大言於朝,故榮等譖而誅之。



 生雖在諒闇,游飲自若,荒耽淫虐,殺戮無道,常彎弓露刃以見朝臣,錘鉗鋸鑿備置左右。又納董榮之言,誅其司空王墮以應日蝕之災。饗群臣於太極前殿,飲酣樂奏,生親歌以和之。命其
 尚書辛牢典勸,既而怒曰:「何不彊酒?猶有坐者!」引弓射牢而殺之。於是百僚大懼,無不引滿昏醉,污服失冠,蓬頭僵仆,生以為樂。



 生聞張祚見殺,玄靚幼沖,命其征東苻柳參軍閻負、梁殊使涼州,以書喻之。負、殊至姑臧,玄靚年幼,不見殊等。其涼州牧張瓘謂負、殊曰:「孤之本朝,世執忠節,遠宗大晉,臣無境外之交,君等何為而至?」負、殊曰:「晉王以鄰籓義好,有自來矣。雖擁阻山河,然風通道會,不欲使羊、陸二公獨美於前。主上以欽明紹統,八表宅心,光被四海,格於天地。晉王思與張王齊曜大明,交玉帛之好,兼與君公同金蘭之契,是以不遠而來,
 有何怪乎!」瓘曰:「羊、陸一時之事,亦非純臣之義也。本朝六世重光,固忠不貳,若與苻征東交玉帛之好者,便是上違先公純誠雅志,下乘河右遵奉之情。」負、殊曰:「昔微去殷,項伯歸漢,雖背君違親,前史美其先覺。亡晉之餘,遠逃江會,天命去之,子故尊先王翻然改圖,北面二趙,蓋神算無方,鑒機而作。君公若欲稱制河西,眾旅非秦之敵,如欲宗歸遺晉,深乖先君雅旨,孰若遠蹤竇融附漢之規,近述先王歸趙之事,垂祚無窮,永享遐祉乎?」瓘曰:「中州無信,好食誓言。往與石氏通好,旋見寇襲。中國之風,誡在昔日,不足復論通和之事也。」負、殊曰:「三王異
 政,五帝殊風,趙多姦詐,秦以義信,豈可同年而語哉!張先、楊初皆擅兵一方,不供王貢,先帝命將擒之,宥其難恕之罪,加以爵封之榮。今上道合二儀,慈弘山海,信符陰陽,御物無際,不可以二趙相況也。」瓘曰:「秦若兵彊化盛,自可先取江南,天下自然盡為秦有,何辱征東之命!」負、殊曰:「先帝以大聖神武,開構鴻基,彊燕納款,八州順軌。主上欽明,道必隆世,慨徽號擁於河西,正朔未加吳會,以吳必須兵,涼可以義,故遣行人先申大好。如君公不能蹈機而發者,正可緩江南數年之命,迴師西旆,恐涼州弗可保也。」瓘曰:「我跨據三州,帶甲十萬,西包崑域,
 東阻大河,伐人有餘,而況自固!秦何能為患!」負、殊曰:「貴州險塞,孰若崤、函?五郡之眾,何如秦、雍?張琚、杜洪因趙之成資,據天阻之固,策三秦之銳,藉陸海之饒,勁士風集,驍騎如雲,自謂天下可平,關中可固,先帝神矛一指,望旗冰解,人詠來蘇,不覺易主。燕雖武視關東,猶以地勢之義,逆順之理,北面稱籓,貢不踰月。致肅慎楛矢,通九夷之珍;單于屈膝,名王內附。控弦之士百有餘萬,鼓行而濟西河者,君公何以抗之?盍追遵先王臣趙故事,世享大美,為秦之西籓。」瓘曰:「然秦之德義加於天下,江南何以不賓?」負、殊曰:「文身之俗,負阻江山,道洿先叛,化
 盛後賓,自古而然,豈但今也!故《詩》曰:『蠢爾蠻荊,大邦為仇。』言其不可以德義懷也。」瓘曰:「秦據漢舊都,地兼將相,文武輔臣,領袖一時者誰也?」負、殊曰:「皇室懿籓,忠若公旦者,則大司馬、武都王安,征東大將軍、晉王柳;文武兼才,神器秀拔,入可允釐百工,出能折衝萬里者,衛大將軍、廣平王黃眉,後將軍、清河王法,龍驤將軍、東海王堅之兄弟;其耆年碩德,德侔尚父者,則太師、錄尚書事、廣寧公魚遵;其清素剛嚴,骨鯁貞亮,則左光祿大夫強平,金紫光祿程肱、牛夷;博聞強識,探賾索幽,則中書監胡文,中書令王魚,黃門侍郎李柔;雄毅厚重,權智無方,則
 左衛將軍李威,右衛將軍苻雅;才識明達,令行禁止,則特進、領御史中丞梁平老,特進、光祿大夫強汪,侍中、尚書呂婆樓;文史富贍,鬱為文宗,則尚書右僕射董榮,秘書監王颺,著作郎梁讜;驍勇多權略,攻必取,戰必勝,關、張之流,萬人之敵者,則前將軍、新興王飛,建節將軍鄧羌,立忠將軍彭越,安遠將軍范俱難,建武將軍徐盛;常伯納言,卿校牧守,則人皆文武,莫非才賢;其餘懷經世之才,蘊佐時之略,守南山之操,遂而不奪者,王猛、朱肜之倫,相望於巖谷。濟濟多士,焉可罄言!姚襄、張平一時之傑,各擁眾數萬,狼顧偏方,皆委忠獻款,請為臣妾。小
 不事大,《春秋》所誅,惟君公圖之。」瓘笑曰:「此事決之主上,非身所了。」負、殊曰:「涼王雖天縱英睿,然尚幼沖,君公居伊、霍之任,安危所繫,見機之義,實在君公。」瓘新輔政,河西所在兵起,懼秦師之至,乃言於玄靚,遣使稱籓,生因其所稱而授之。



 慕容俊遣將慕輿長卿等率眾七千入自軹關,攻幽州刺史張哲于裴氏堡。晉將軍劉度等率眾四千,攻青州刺史袁朗于盧氏。生遣其前將軍苻飛距晉,建節鄧羌距燕。飛未至而度退。羌及長卿戰于堡南,大敗之,獲長卿及甲首二千七百餘級。



 姚襄率眾萬餘,攻其平陽太守苻產于匈奴堡,苻柳救之,為襄所敗,
 引還蒲阪。襄遂攻堡,剋之,殺苻產,盡坑其眾,遣使從生假道,將還隴西。生將許之,苻堅諫曰:「姚襄,人傑也,今還隴西,必為深害,不如誘以厚利,伺隙而擊之。」生乃止。遣使拜襄官爵,襄不受,斬其使者,焚所送章策,寇掠河東。生怒,命其大將軍張平討之。襄乃卑辭厚幣與平結為兄弟,平更與襄通和。



 生發三輔人營渭橋,金紫光祿大夫程肱以妨農害時,上疏極諫。生怒,殺之。



 長安大風,發屋拔樹,行人顛頓,宮中奔擾,或稱賊至,宮門晝閉,五日乃止。生推告賊者,殺之,刳而出其心。左光祿大夫強平諫曰:「元正盛旦,日有蝕之,正陽神朔,昏風大起,兼水旱
 不時,獸災未息,此皆由陛下不勉彊於政事,乖和氣所致也。願陛下務養元元,平章百姓,棄纖介之嫌,含山嶽之過,致敬宗社,愛禮公卿,去秋霜之威,垂三春之澤,則姦回寢止,妖昆自消,乾靈祗祐皇家,永保無窮之美矣。」生怒,以為妖言,鑿其頂而殺之。



 平之囚也,偽衛將軍苻黃眉、前將軍苻飛、建節鄧羌侍宴禁中,叩頭固諫,以太后為言。平即生母強氏之弟也。生既弗許,強氏憂恨而死。



 生下書曰:「朕受皇天之命,承祖宗之業,君臨萬邦,子育百姓,嗣統已來,有何不善,而謗讟之音扇滿天下。殺不過千,而謂刑虐。行者比肩,未足為稀。方當峻刑極罰,
 復如朕何!」時猛獸及狼大暴,晝則斷道,夜則發屋,惟害人而不食六畜。自生立一年,獸殺七百餘人,百姓苦之,皆聚而邑居。為害滋甚,遂廢農桑,內外兇懼。群臣奏請禳災,生曰:「野獸飢則食人,飽當自止,終不能累年為患也。天豈不子愛群生,而年年降罰,正以百姓犯罪不已,將助朕專殺而施刑教故耳。但勿犯罪,何為怨天而尤人哉!」



 生如阿房,遇兄與妹俱行者,逼令為非禮,不從,生怒殺之。又宴群臣于咸陽故城,有後至者,皆斬之。嘗使太醫令程延合安胎藥,問人參好惡并藥分多少,延曰:「雖小小不具,自可堪用。」生以為譏其目,鑿延目出,然後
 斬之。



 有司奏:「太白犯東井。東井,秦之分也,太白罰星,必有暴兵起于京師。」生曰:「星入井者,必將渴耳,何所怪乎!」



 姚襄遣姚蘭、王欽盧待招動鄜城、定陽、北地、芹川諸羌胡,皆應之,有眾二萬七千,進據黃落。生遣苻黃眉、苻堅、鄧羌率步騎萬五千討之。襄深溝高壘,固守不戰。鄧羌說黃眉曰:「傷弓之鳥,落於虛發。襄頻為桓溫、張平所敗,銳氣喪矣。今謀固壘不戰,是窮寇也。襄性剛很,易以剛動,若長驅鼓行,直壓其壘,襄必忿而出師,可一戰擒也。」黃眉從之,遣羌率騎三千軍於壘門。襄怒,盡銳出戰。羌偽不勝,引騎而退,襄追之于三原,羌迴騎距襄。俄而黃
 眉與堅至,大戰,斬之,盡俘其眾,黃眉等振旅而歸。黃眉雖有大功,生不加旌賞,每於眾中辱之。黃眉怒,謀殺生自立,事發,伏誅,其王公親戚多有死者。



 初,生夢大魚食蒲,又長安謠曰:「東海大魚化為龍,男便為王女為公。問在何所洛門東。」東海,苻堅封也,時為龍驤將軍,第在洛門之東。生不知是堅,以謠夢之故,誅其侍中、太師、錄尚書事魚遵及其七子、十孫。時又謠曰:「百里望空城,鬱鬱何青青。瞎兒不知法,仰不見天星。」於是悉壞諸空城以禳之。金紫光祿大夫牛夷懼不免禍,請出鎮上洛。生曰:「卿忠肅篤敬,宜左右朕躬,豈有外鎮之理。」改授中軍。夷
 懼,歸而自殺。



 初,生少凶暴嗜酒,健臨死,恐其不能保全家業,誡之曰:「酋師、大臣若不從汝命,可漸除之。」及即偽位,殘虐滋甚,耽湎於酒,無復晝夜。群臣朔望朝謁,罕有見者,或至暮方出,臨朝輒怒,惟行殺戮。動連月昏醉,文奏因之遂寢。納姦佞之言,賞罰失中。左右或言陛下聖明宰世,天下惟歌太平。生曰:「媚於我也。」引而斬之。或言陛下刑罰微過。曰:「汝謗我也。」亦斬之。所幸妻妾小有忤旨,便殺之,流其尸于渭水。又遣宮人與男子裸交於殿前。生剝牛羊驢馬,活爓雞豚鵝,三五十為群,放之殿中。或剝死囚面皮,令其歌舞,引群臣觀之,以為嬉樂。宗室、
 勛舊、親戚、忠良殺害略盡,王公在位者悉以疾告歸,人情危駭,道路以目。既自有目疾,其所諱者不足、不具、少、無、缺、傷、殘、毀、偏、隻之言皆不得道,左右忤旨而死者不可勝紀,至於截脛、刳胎、拉脅、鋸頸者動有千數。



 太史令康權言於生曰:「昨夜三月並出,勃星入於太微,遂入於東井。兼自去月上旬沈陰不雨,迄至于今,將有下人謀上之禍,深願陛下脩德以消之。」生怒,以為妖言,撲而殺之。



 生夜對侍婢曰:「阿法兄弟亦不可信,明當除之。」是夜清河王苻法夢神告之曰:「旦將禍集汝門,惟先覺者可以免之。」寤而心悸。會侍婢來告,乃與特進梁平老、強汪
 等率壯士數百人潛入雲龍門,苻堅與呂婆樓率麾下三百餘人鼓噪繼進,宿衛將士皆捨杖歸堅。生猶昏寐未寤。堅眾既至,引生置於別室,廢之為越王,俄而殺之。生臨死猶飲酒數斗,昏醉無所知矣。時年二十三,在位二年,偽謚厲王。



 苻雄,字元才,洪之季子也。少善兵書,而多謀略,好施下士,便弓馬,有政術。健僭位,為佐命元勳,權侔人主,而謙恭奉法。健常曰:「元才,吾姬旦也。」及卒,健哭之歐血,曰:「天不欲吾定四海邪?何奪元才之速也!」子堅,別有載記。



 王墮,字安生,京兆霸城人也。博學有雄才,明天文圖緯。苻洪征梁犢,以墮為司馬,謂洪曰:「讖言苻氏應王,公其人也。」洪深然之。及為宰相,著匪躬之稱。健常嘆曰:「天下群官皆如王令君者,陰陽曷不和乎!」甚敬重之。性剛峻疾惡,雅好直言。疾董榮、強國如仇讎,每於朝見之際,略不與言。人謂之曰:「董尚書貴幸一時,公宜降意。」墮曰:「董龍是何雞狗,而令國士與之言乎!」榮聞而慚恨,遂勸生誅之。及刑,榮謂墮曰:「君今復敢數董龍作雞狗?」墮瞋目而叱之。龍,榮之小字也。



\end{pinyinscope}