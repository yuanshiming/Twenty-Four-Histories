\article{列傳第三十九 劉隗刁協戴若思周顗}

\begin{pinyinscope}
劉隗
 \gezhu{
  孫波}
 刁協
 \gezhu{
  子彞彞子逵}
 戴若思
 \gezhu{
  弟邈周顗}



 劉隗,字大連,彭城人,楚元王交之後也。父砥,東光令。隗少有文翰,起家祕書郎,稍遷冠軍將軍、彭城內史。避亂渡江,元帝以為從事中郎。隗雅習文史,善求人主意,帝深器遇之。遷丞相司直,委以刑憲。時建康尉收護軍士,而為府將篡取之,隗奏免護軍將軍戴若思官。世子文學王籍之居叔母喪而婚,隗奏之,帝下令曰:「《詩》稱殺禮
 多婚,以會男女之無夫家,正今日之謂也,可一解禁止。自今以後,宜為其防。」東閣祭酒顏含在叔父喪嫁女,隗又奏之。廬江太守梁龕明日當除婦服,今日請客奏伎,丞相長史周顗等三十餘人同會,隗奏曰:「夫嫡妻長子皆杖居廬,故周景王有三年之喪,既除而宴,《春秋》猶譏,況龕匹夫,暮宴朝祥,慢服之愆,宜肅喪紀之禮。請免龕官,削侯爵。顗等知龕有喪,吉會非禮,宜各奪俸一月,以肅其違。」從之。丞相行參軍宋挺,本揚州刺史劉陶門人,陶亡後,挺娶陶愛妾以為小妻。建興中,挺又割盜官布六百餘匹,正刑棄市,遇赦免。既而奮武將軍阮抗請為
 長史。隗劾奏曰:「挺蔑其死主而專其室,悖在三之義,傷人倫之序,當投之四裔以禦魑魅。請除挺名,禁錮終身。而奮武將軍、太山太守阮抗請為長史。抗緯文經武,剖符東籓,當庸勛忠良,暱近仁賢,而褒求贓污,舉頑用嚚。請免抗官,下獄理罪。」奏可,而挺病死。隗又奏:「符旨:挺已喪亡,不復追貶。愚蠢意闇,未達斯義。昔鄭人JX子家之棺,漢明追討史遷,經傳褒貶,皆追書先世數百年間,非徒區區欲釐當時,亦將作法垂於來世,當朝亡夕沒便無善惡也。請曹如前追除挺名為民,錄妾還本,顯證惡人,班下遠近。」從之。南中郎將王含以族彊顯貴,驕傲自
 恣,一請參佐及守長二十許人,多取非其才。隗劾奏文致甚苦,事雖被寢,王氏深忌疾之。而隗之彈奏不畏彊禦,皆此類也。



 建興中,丞相府斬督運令史淳于伯而血逆流,隗又奏曰:「古之為獄必察五聽,三槐九棘以求民情。雖明庶政,不敢折獄。死者不得復生,刑者不可復續,是以明王哀矜用刑。曹參去齊,以市獄為寄。自頃蒸荒,殺戮無度,罪同斷異,刑罰失宜。謹按行督運令史淳于伯刑血著柱,遂逆上終極柱末二丈三尺,旋復下流四尺五寸。百姓喧華,士女縱觀,咸曰其冤。伯息忠訴辭稱枉,云伯督運訖去二月,事畢代還,無有稽乏。受賕使役,
 罪不及死。軍是戍軍,非為征軍,以乏軍興論,於理為枉。四年之中,供給運漕,凡諸徵發租調百役,皆有稽停,而不以軍興論,至於伯也,何獨明之?捶楚之下,無求不得,囚人畏痛,飾辭應之。理曹,國之典刑,而使忠等稱冤明時。謹按從事中郎周筵、法曹參軍劉胤、屬李匡幸荷殊寵,並登列曹,當思敦奉政道,詳法慎殺,使兆庶無枉,人不稱訴。而令伯枉同周青,冤魂哭於幽都,訴靈恨於黃泉,嗟歎甚於杞梁,血妖過於崩城,故有隕霜之人,夜哭之鬼。伯有晝見,彭生為豕,刑殺失中,妖眚並見,以古況今,其揆一也。皆由筵等不勝其任,請皆免官。」於是右將
 軍王導等上疏引咎,請解職。帝曰:「政刑失中,皆吾暗塞所由。尋示愧懼,思聞忠告,以補其闕。而引過求退,豈所望也!」由是導等一無所問。



 晉國既建,拜御史中丞。周嵩嫁女,門生斷道解廬,斫傷二人,建康左尉赴變,又被斫。隗劾嵩兄顗曰:「顗幸荷殊寵,列位上僚,當崇明憲典,協和上下,刑于左右,以御于家邦。而乃縱肆小人,群為兇害,公於廣都之中白日刃尉,遠近洶嚇,百姓喧華,虧損風望,漸不可長。既無大臣檢御之節,不可對揚休命。宜加貶黜,以肅其違。」顗坐免官。



 太興初,長兼侍中,賜爵都鄉侯,尋代薛兼為丹陽尹,與尚書令刁協並為元帝所
 寵,欲排抑豪彊。諸刻碎之政,皆云隗、協所建。隗雖在外,萬機秘密皆豫聞之。拜鎮北將軍、都督青徐幽平四州軍事、假節,加散騎常侍,率萬人鎮泗口。



 初,隗以王敦威權太盛,終不可制,勸帝出腹心以鎮方隅,故以譙王承為湘州,續用隗及戴若思為都督。敦甚惡之,與隗書曰:「頃承聖上顧眄足下,今大賊未滅,中原鼎沸,欲與足下周生之徒戮力王室,共靜海內。若其泰也,則帝祚於是乎隆;若其否也,則天下永無望矣。」隗答曰:「魚相忘於江湖,人相忘於道術。竭股肱之力,效之以忠貞,吾之志也。」敦得書甚怒。及敦作亂,以討隗為名,詔征隗還京師,百
 官迎之於道,隗岸幘大言,意氣自若。及入見,與刁協奏請誅王氏。不從,有懼色,率眾屯金城。及敦剋石頭,隗攻之不拔,入宮告辭,帝雪涕與之別。隗至淮陰,為劉遐所襲,攜妻子及親信二百餘人奔于石勒,勒以為從事中郎、太子太傅。卒年六十一。子綏,初舉秀才,除駙馬都尉、奉朝請。隨隗奔勒,卒。孫波嗣。



 波字道則。初為石季龍冠軍將軍王洽參軍,及季龍死,洽與波俱降。穆帝以波為襄城太守,累遷桓沖中軍諮議參軍。大司馬桓溫西征袁真,朝廷空虛,以波為建威將軍、淮南內史,領五千人鎮石頭。壽陽平,除尚書左丞,
 不拜,轉冠軍將軍、南郡相。時苻堅弟融圍雍州刺史朱序於襄陽,波率眾八千救之,以敵彊不敢進,序竟陷沒。波以畏懦免官。後復以波為冠軍將軍,累遷散騎常侍。



 苻堅敗,朝廷欲鎮靖北方,出波督淮北諸軍、冀州刺史,以疾未行。上疏曰:



 臣聞天地以弘濟為仁,君道以惠下為德,是以禹湯有身勤之績,唐虞有在予之誥,用能惠被蒼生,勛流後葉。宣帝開拓洪圖,始基成命;爰及文武,歷數在躬,而猶虛心側席,卑己崇物。然後知積累之功重,勤王之業艱,先君之德弘,貽厥之賜厚。惠皇不懷,委政內任,遂使神器幽淪,三光翳曜;園陵懷九泉之感,宮
 廟集胡馬之跡;所謂肉食失之於朝,黎庶暴骸於外也。賴元皇帝神武應期,祚隆淮海,振乾綱於已墜,紐絕維而更張。陛下承宣帝開始之宏基,受元帝克終之成烈,保大定功,戢兵靜亂。故使負鱗橫海之鯨,僭位滔天之寇,望雲旗而宵潰,睹太陽而霧散,巍巍蕩蕩,人無名焉。而頃年已來,天文違錯,妖怪屢生。會稽先帝本封,而地動經年。昔周之文武有魚烏之瑞,君臣猶懷震悚,況今災變眾集,曾莫之疑。公旦有勿休之誡,賈誼有積薪之喻。臣鑒先征,竊惟今事,是以敢肆狂瞽,直言無諱。



 往者先帝以玄風御世,責成群后,坐運天綱,隨化委順,故忘
 日計之功,收歲成之用。今禮樂征伐自天子出,相王賢俊,協和百揆,六合承風,天下響振,而鈞臺之詠弗聞,景毫之命未布。將群臣之不稱,陛不用之不盡乎?



 凡聖王之化,莫不敦崇忠信,存正棄邪。傷化毀俗者,雖親雖貴,必疏而遠之;清公貞修者,雖微雖賤,必親而近之。今則不然。此風既替,利競滋甚,朋黨比周,毀譽交興,鑽求茍進,人希分外。見賢而居其上,受祿每過其量,希旨承意者以為奉公,共相贊白者以為忠節。舉世見之,誰敢正言。陛下不明必行之法以絕穿鑒之源者,恐脫因疲倦以誤視聽。且苻堅滅亡,於今五年,舊京殘毀,山陵無衛,
 百姓塗炭,未蒙拯接。伏願遠觀漢魏衰滅之由,近覽西朝傾覆之際,超然易慮,為於未有,則靈根永固,社稷無虞。臣豈誣一朝之人皆無忠節,但任非其才,求之不至耳。



 今政煩役殷,所在凋弊,倉廩空虛,國用傾竭,下民侵削,流亡相屬。略計戶口,但咸安已來,十分去三。百姓懷浮游之歎,《下泉》興周京之思。昔漢宣有云:「與我共治天下者,其惟良二千石乎!」是以臨下有方者就加璽贈,法苛政亂者恤刑不赦,事簡於上,人悅於下。今則不然。告時乞職者以家弊為辭,振窮恤滯者以公爵為施。古者為百姓立君,使之司牧;今者以百姓恤君,使之蠶食,至
 乃貪污者謂之清勤,慎法者謂之怯劣。何反古道一至於此!



 陛下雖躬自節儉,哀矜於上,而群僚肆欲,縱心於下,六司垂翼,三事拱默,故有識者睹人事以歎息,觀妖眚而大懼。昔宋景退熒惑之災,殷宗消鼎雉之異。伏願陛下仰觀大禹過門之志,俯察商辛沈湎之失,遠思《國風》恭公之刺,深惟定姜小臣之喻。暫迴聖恩,大詢群后,延納眾賢,訪以得失;令百僚率職,人言損益。察其所由,觀其所以,審識群才,助鼎和味。克念作聖,以答天休。則四海宅心,天下幸甚。



 臣亡祖先臣隗,昔荷殊寵,匪躬之操,猶存舊史,有志無時,懷恨黃泉。及臣凡劣,復蒙罔極
 之眷,恩隆累世,實非糜身傾宗所能上報。前作此表,未及得通。暴嬰篤疾,恐命在奄忽,貪及視息,望達愚情。氣力懾然,不能自宣。



 疏奏而卒。追贈前將軍。子淡嗣。元熙初,為廬江太守。



 隗伯父訥,字令言,有人倫鑒識。初入洛,見諸名士而歎曰:「王夷甫太鮮明,樂彥輔我所敬,張茂先我所不解,周弘武巧於用短,杜方叔拙於用長。」終於司隸校尉。



 子疇,字王喬,少有美譽,善談名理。曾避亂塢壁,賈胡百數欲害之,疇無懼色,援笳而吹之,為《出塞》、《入塞》之聲,以動其游客之思。於是群胡皆垂泣而去之。永嘉中,位至司徒左長史,尋為閻鼎所殺。司空蔡謨每歎
 曰:「若使劉王喬得南渡,司徒公之美選也。」又王導初拜司徒,謂人曰:「劉王喬若過江,我不獨拜公也。」其為名流之所推服如此。



 疇兄子劭,有才幹,辟瑯邪王丞相掾。咸康世,歷御史中丞、侍中、尚書、豫章太守,秩中二千石。



 邵族子黃老,太元中,為尚書郎,有義學,注《慎子》、《老子》,並傳於世。



 刁協,字玄亮,渤海饒安人也。祖恭,魏齊郡太守。父攸,武帝時御史中丞。協少好經籍,博聞彊記,釋褐濮陽王文學,累轉太常博士、本郡大中正。成都王穎請為平北司
 馬,後歷趙王倫相國參軍,長沙王乂驃騎司馬。及東嬴公騰鎮臨漳,以協為長史,轉潁川太守。永嘉初,為河南尹,未拜,避難渡江。元帝以為鎮東軍諮祭酒,轉長史。愍帝即位,徵為御史中丞,例不行。元帝為丞相,以協為左長史。中興建,拜尚書左僕射。於時朝廷草創,憲章未立,朝臣無習舊儀者。協久在中朝,諳練舊事,凡所制度,皆稟於協焉,深為當時所稱許。太興初,遷尚書令,在職數年,加金紫光祿大夫,令如故。



 協性剛悍,與物多忤,每崇上抑下,故為王氏所疾。又使酒放肆,侵毀公卿,見者莫不側目。然悉力盡心,志在匡救,帝甚信任之。以奴為兵,
 取將吏客使轉運,皆協所建也,眾庶怨望之。及王敦構逆,上疏罪協。帝使協出督六軍。既而王師敗績,協與劉隗俱侍帝於太極東除,帝執協、隗手,流涕嗚咽,勸令避禍。協曰:「臣當守死,不敢有貳。」帝曰:「今事逼矣,安可不行!」乃令給協、隗人馬,使自為計。協年老,不堪騎乘,素無恩紀,募從者,皆委之行。至江乘,為人所殺,送首於敦,敦德刁氏,收葬之。帝痛協不免,密捕送協首者而誅之。



 敦平後,周顗、戴若思等皆被顯贈,惟協以出奔不在其例。咸康中,協子彞上疏訟之。在位者多以明帝之世褒貶已定,非所得更議,且協不能抗節隕身,乃出奔遇害,不可
 復其官爵也。丹陽尹殷融議曰:「王敦惡逆,罪不容誅,則協之善亦不容賞。若以忠非良圖,謀事失算,以此為責者,蓋在於譏議之間耳。即凶殘之誅以為國刑,將何以沮勸乎!當敦專逼之時,慶賞威刑專自己出,是以元帝慮深崇本,以協為比,事由國計,蓋不為私。昔孔寧、儀行父從君於昏,楚復其位者,君之黨故也。況協之比君,在於義順。且中興四佐,位為朝首。于時事窮計屈,奉命違寇,非為逃刑。謂宜顯贈,以明忠義。」時庾冰輔政,疑不能決。左光祿大夫蔡謨與冰書曰:



 夫爵人者,宜顯其功;罰人者,宜彰其罪,此古今之所慎也。凡小之人猶尚如此,
 刁令中興上佐,有死難之名,天下不聞其罪,而見其貶,致令刁氏稱冤,此乃為王敦復仇也。內沮忠臣之節,論者惑之。若實有大罪,宜顯其事,令天下知之,明聖朝不貶死難之臣。《春秋》之義,以功補過。過輕功重者,得以加封;功輕過重者,不免誅絕;功足贖罪者無黜。雖先有邪佞之罪,而臨難之日黨於其君者,不絕之也。孔寧、儀行父親與靈公淫亂於朝,君殺國滅,由此二臣,而楚尚納之。傳稱有禮不絕其位者,君之黨也。若刁令有罪,重於孔儀,絕之可也。若無此罪,宜見追論。



 或謂明帝之世已見寢廢,今不宜復改,吾又以為不然。夫大道宰世,殊塗
 一致。萬機之事,或異或同,同不相善,異不相譏。故堯抑元凱而舜舉之,堯不為失,舜不為非,何必前世所廢便不宜改乎?漢蕭何之後坐法失侯,文帝不封而景帝封之,後復失侯,武昭二帝不封而宣帝封之。近去元年,車駕釋奠,拜孔子之坐,此亦元明二帝所不行也。又刁令但是明帝所不贈耳,非誅之也。王平子、第五猗皆元帝所誅,而今日所贈,豈以改前為嫌乎!凡處事者,當上合古義,下準今例,然後談者不惑,受罪者無怨耳。案周僕射、戴征西本非王敦唱檄所仇也,事定後乃見害耳;周筵、郭璞等並亦非為主禦難也,自平居見殺耳,皆見褒
 贈,刁令事義豈輕於此乎?自頃員外散騎尚得追贈,況刁令位亞三司。若先自壽終,不失員外散騎之例也。就不蒙贈,不失以本官殯葬也。此為一人之身,壽終則蒙贈,死難則見絕,豈所以明事君之道,厲為臣之節乎!宜顯評其事,以解天下疑惑之論。



 又聞談者亦多謂宜贈。凡事不允當,而得眾助者,若以善柔得眾,而刁令粗剛多怨;若以貴也,刁氏今賤;若以富也,刁氏今貧。人士何故反助寒門而此言之?足下宜察此意。



 冰然之。事奏,成帝詔曰:「協情在忠主,而失為臣之道,故令王敦得託名公義,而實肆私忌,遂令社稷受屈,元皇銜恥,致禍之原,
 豈不有由!若極明國典,則曩刑非重。今正當以協之勤有可書,敦之逆命不可長,故議其事耳。今可復協本位,加之冊祭,以明有忠於君者纖介必顯,雖於貶裁未盡,然或足有勸矣。」於是追贈本官,祭以太牢。



 彝字大倫。少遭家難。王敦誅後,彝斬仇人黨,以首祭父墓,詣廷尉請罪,朝廷特宥之,由是知名,歷尚書吏部郎、吳國內史,累遷北中郎將、徐兗二州刺史、假節,鎮廣陵,卒於官。



 子逵,字伯道,逵弟暢,字仲遠;次子弘,字叔仁,並歷顯職。隆安中,達為廣州刺史,領平越中郎將、假節;暢為始興相;弘為冀州刺史。兄弟子侄並不拘名行,以貨
 殖為務,有田萬頃,奴婢數千人,餘資稱是。



 桓玄篡位,以逵為西中郎將、豫州刺史,鎮歷陽;暢右衛將軍;弘撫軍桓脩司馬。劉裕起義,斬桓脩,時暢、弘謀起兵襲裕,裕遣劉毅討之,暢伏誅;弘亡,不知所在。逵在歷陽執劉裕參軍諸葛長民,檻車送於桓玄,至當利而玄敗,送人共破檻出長民,遂趣歷陽。逵棄城而走,為下人所執,斬於石頭。子姪無少長皆死,惟小弟騁被宥,為給事中,尋謀反伏誅,刁氏遂滅。刁氏素殷富,奴客縱橫,固吝山澤,為京口之蠹。裕散其資蓄,今百姓稱力而取之,彌日不盡。時天下饑弊,編戶賴之以濟焉。



 戴若思,廣陵人也,名犯高祖廟諱。祖烈,吳左將軍。父昌,會稽太守。若思有風儀,性閒爽,少好遊俠,不拘操行。遇陸機赴洛,船裝甚盛,遂與其徒掠之。若思登岸,據胡床,指麾同旅,皆得其宜。機察見之,知非常人,在舫屋上遙謂之曰:「卿才器如此,乃復作劫邪!」若思感悟,因流涕,投劍就之。機與言,深加賞異,遂與定交焉。



 若思後舉孝廉,入洛,機薦之於趙王倫曰:「蓋聞繁弱登御,然後高墉之功顯;孤竹在肆,然後降神之曲成。是以高世之主必假遠邇之器,蘊櫝之才思託太音之和。伏見處士廣陵戴
 若思,年三十,清沖履道,德量允塞;思理足以研幽,才鑒足以辯物;安窮樂志,無風塵之慕,砥節立行,有井渫之潔;誠東南之遺寶,宰朝之奇璞也。若得託迹康衢,則能結軌驥騄;曜質廊廟,必能垂光璵璠矣。惟明公垂神採察,不使忠允之言以人而廢。」倫乃辟之,除沁水令,不就,遂往武陵省父。時同郡人潘京素有理鑒,名知人,其父遣若思就京與語,既而稱若思有公輔之才。累轉東海王越軍諮祭酒,出補豫章太守,加振威將軍,領義軍都督。以討賊有功,賜爵秣陵侯,遷治書侍御史、驃騎司馬,拜散騎侍郎。



 元帝召為鎮東右司馬。將征杜弢,加若思
 前將軍,未發而弢滅。帝為晉王,以為尚書。中興建,為中護軍,轉護軍將軍、尚書僕射,皆辭不拜。出為征西將軍、都督兗豫幽冀雍並六州諸軍事、假節,加散騎常侍。發投刺王官千人為軍吏,調揚州百姓家奴萬人為兵配之,以散騎常侍王遐為軍司,鎮壽陽,與劉隗同出。帝親幸其營,勞勉將士,臨發祖餞,置酒賦詩。



 若思至合肥,而王敦舉兵,詔追若思還鎮京都,進驃騎將軍,與右衛將軍郭逸夾道築壘於大桁之北。尋而石頭失守,若思與諸軍攻石頭,王師敗績。若思率麾下百餘人赴宮受詔,與公卿百官於石頭見敦。敦問若思曰:「前日之戰有餘
 力乎?」若思不謝而答曰:「豈敢有餘,但力不足耳。」又曰:「吾此舉動,天下以為如何?」若思曰:「見形者謂之逆,體誠者謂之忠。」敦笑曰:「卿可謂能言。」敦參軍呂猗昔為臺郎,有刀筆才,性尤姦諂,若思為尚書,惡其為人,猗亦深憾焉。至是,乃說敦曰:「周顗、戴若思皆有高名,足以惑眾,近者之言曾無愧色。公若不除,恐有再舉之患,為將來之憂耳。」敦以為然,又素忌之,俄而遣鄧嶽、繆坦收若思而害之。若思素有重望,四海之士莫不痛惜焉。賊平,冊贈右光祿大夫、儀同三司,謚曰簡。



 邈字望之。少好學,尤精《史》《漢》,才不逮若思,儒博過之。弱
 冠舉秀才,尋遷太子洗馬,出補西陽內史。永嘉中,元帝版行邵陵內史、丞相軍諮祭酒,出為征南軍司。于時凡百草創,學校未立,邈上疏曰:



 臣聞天道之所大,莫大於陰陽;帝王之至務,莫重於禮學。是以古之建國,有明堂辟雍之制,鄉有庠序1111校之儀,皆所以抽導幽滯,啟廣才思。蓋以六四有困蒙之吝,君子大養正之功也。昔仲尼列國之大夫耳,興禮修學於洙泗之間,四方髦俊斐然向風,身達者七十餘人。自茲以來,千載絕塵。豈天下小於魯衛,賢哲乏於曩時?勵與不勵故也。



 自頃國遭無妄之禍,社稷有綴旒之危,寇羯飲馬於長江,兇狡鴟張
 於萬里,遂使神州蕭條,鞠為茂草,四海之內,人跡不交。霸主有旰食之憂,黎元懷荼毒之苦,戎首交拜於中原,何遽籩豆之事哉!然三年不為禮,禮必壞;三年不為樂,樂必崩,況曠戴累紀如此之久邪!今末進後生目不睹揖讓升降之儀,耳不聞鐘鼓管弦之音,文章散滅,圖讖無遺,此蓋聖達之所深悼,有識之所嗟歎也。夫平世尚文,遭亂尚武,文武遞用,長久之道,譬之天地昏明之迭,自古以來未有不由之者也。



 今或以天下未一,非興禮學之時,此言似之而不其然。夫儒道深奧,不可倉卒而成。古之俊乂必三年而通一經,比天下平泰然後修之,
 則功成事定,誰與制禮作樂者哉?又貴游之子未必有斬將搴旗之才,亦未有從軍征戍之役,不及盛年講肄道義,使明珠加磨瑩之功,荊璞發採琢之榮,不亦良可惜乎!



 臣愚以世喪道久,人情玩於所習;純風日去,華競日彰,猶火之消膏而莫之覺也。今天地告始,萬物權輿,聖朝以神武之德,值革命之運,蕩近世之流弊,繼千載之絕軌,篤道崇儒,創立大業。明主唱之於上,宰輔督之於下。夫上之所好,下必有過之者焉,是故雙劍之節崇,而飛白之俗成;挾琴之容飾,而赴曲之和作;君子之德風,小人之德草,實在感之而已。臣以闇淺,不能遠識格
 言;奉誦明令,慷慨下風,謂宜以三時之隙漸就修建。



 疏奏,納焉,於是始脩禮學。



 代劉隗為丹陽尹。王敦作逆,加左將軍。及敦得志,而若思遇害,邈坐免官。敦誅後,拜尚書僕射。卒官,贈衛將軍,謚曰穆。子謐嗣,歷義興太守、大司農。



 周顗,字伯仁,安東將軍浚之子也。少有重名,神彩秀徹,雖時輩親狎,莫能媟也。司徒掾同郡賁嵩有清操,見顗,歎曰:「汝潁固多奇士!自頃雅道陵遲,今復見周伯仁,將振起舊風,清我邦族矣。」廣陵戴若思東南之美,舉秀才,
 入洛,素聞顗名,往候之,終坐而出,不敢顯其才辯。顗從弟穆亦有美譽,欲陵折顗,顗陶然弗與之校,於是人士益宗附之。州郡辟命皆不就。弱冠,襲父爵武城侯,拜祕書郎,累遷尚書吏部郎。東海王越子毗為鎮軍將軍,以顗為長史。



 元帝初鎮江左,請為軍諮祭酒,出為寧遠將軍、荊州刺史、領護南蠻校尉、假節。始到州,而建平流人傅密等叛迎蜀賊杜弢,顗狼狽失據。陶侃遣將吳寄以兵救之,故顗得免,因奔王敦於豫章。敦留之。軍司戴邈曰:「顗雖退敗,未有蒞眾之咎,德望素重,宜還復之。」敦不從。帝召為揚威將軍、兗州刺史。顗還建康,帝留顗不遣,
 復以為軍諮祭酒,尋轉右長史。中興建,補吏部尚書。頃之,以醉酒為有司所糾,白衣領職。復坐門生斫傷人,免官。



 太興初,更拜太子少傅,尚書如故。顗上疏讓曰:「臣退自循省,學不通一經,智不效一官,止足良難,未能守分,遂忝顯任,名位過量。不悟天鑒忘臣頑弊,乃欲使臣內管銓衡,外忝傅訓,質輕蟬翼,事重千鈞,此之不可,不待識而明矣。若臣受負乘之責,必貽聖朝惟塵之恥,俯仰愧懼,不知所圖。」詔曰:「紹幼沖便居儲副之貴,當賴軌匠以祛蒙蔽。望之儼然,斯不言之益,何學之習邪,所謂與田蘇遊忘其鄙心者。便當副往意,不宜沖讓。」轉尚書左
 僕射,領吏部如故。



 庾亮嘗謂顗曰:「諸人咸以君方樂廣。」顗曰:「何乃刻畫無鹽,唐突西施也。」帝宴群公于西堂,酒酣,從容曰:「今日名臣共集,何如堯舜時邪?」顗因醉厲聲曰:「今雖同人主,何得復比聖世!」帝大怒而起,手詔付廷尉,將加戮,累日方赦之。及出,諸公就省,顗曰:「近日之罪,固知不至于死。」尋代戴若思為護軍將軍。尚書紀瞻置酒請顗及王導等,顗荒醉失儀,復為有司所奏。詔曰:「顗參副朝右,職掌銓衡,當敬慎德音,式是百辟。屢以酒過,為有司所繩。吾亮其極嘆之情,然亦是濡首之誡也。顗必能克己復禮者,今不加黜責。」



 初,顗以雅望獲海內盛
 名,後頗以酒失。為僕射,略無醒日,時人號為「三日僕射」。庾亮曰:「周侯末年,所謂鳳德之衰也。」顗在中朝時,能飲酒一石,及過江,雖日醉,每稱無對。偶有舊對從北來,顗遇之欣然,乃出酒二石共飲,各大醉。及顗醒,使視客,已腐脅而死。



 顗性寬裕而友愛過人,弟嵩嘗因酒真目謂顗曰:「君才不及弟,何乃橫得重名!」以所燃蠟燭投之。顗神色無忤,徐曰:「阿奴火攻,固出下策耳。」王導甚重之,嘗枕顗膝而指其腹曰:「此中何所有也?」答曰:「此中空洞無物,然足容卿輩數百人。」導亦不以為忤。又於導坐傲然嘯詠,導云:「卿欲希嵇、阮邪?」顗曰:「何敢近捨明公,遠希
 嵇、阮。」



 及王敦構逆,溫嶠謂顗曰:「大將軍此舉似有所在,當無濫邪?」顗曰:「君少年未更事。人主自非堯舜,何能無失,人臣豈可得舉兵以協主!共相推戴,未能數年,一旦如此,豈云非亂乎!處仲剛愎彊忍,狼抗無上,其意寧有限邪!」既而王師敗績,顗奉詔詣敦,敦曰:「伯仁,卿負我!」顗曰:「公戎車犯順,下官親率六軍,不能其事,使王旅奔敗,以此負公。」敦憚其辭正,不知所答。帝召顗於廣室,謂之曰:「近日大事,二宮無恙,諸人平安,大將軍故副所望邪?」顗曰:「二宮自如明詔,於臣等故未可知。」護軍長史郝嘏等勸顗避敦,顗曰:「吾備位大臣,朝廷喪敗,寧可復草間
 求活,外投胡越邪!」俄而與戴若思俱被收,路經太廟,顗大言曰:「天地先帝之靈;賊臣王敦傾覆社稷,枉殺忠臣,陵虐天下,神祇有靈,當速殺敦,無令縱毒,以傾王室。」語未終,收人以戟傷其口,血流至踵,顏色不變,容止自若,觀者皆為流涕。遂於石頭南門外石上害之,時年五十四。



 顗之死也,敦坐有一參軍樗蒱,馬於博頭被殺,因謂敦曰:「周家奕世令望,而位不至公,及伯仁將登而墜,有似下官此馬。」敦曰:「伯仁總角於東宮相遇,一面披襟,便許之三事,何圖不幸自貽王法。」敦素憚顗,每見顗輒面熱,雖復冬月,扇面手不得休。敦使繆坦籍顗家,收得素
 簏數枚,盛故絮而已,酒五甕,米數石,在位者服其清約。敦卒後,追贈左光祿大夫、儀同三司,謚曰康,祀以少牢。



 初,敦之舉兵也,劉隗勸帝盡除諸王,司空導率群從詣闕請罪,值顗將入,導呼顗謂曰:「伯仁,以百口累卿!」顗直入不顧。既見帝,言導忠誠,申救甚至,帝納其言。顗喜飲酒,致醉而出。導猶在門,又呼顗。顗不與言,顧左右曰:「今年殺諸賊奴,取金印如斗大繫肘。」既出,又上表明導,言甚切至。導不知救己,而甚銜之。敦既得志,問導曰:「周顗、戴若思南北之望,當登三司,無所疑也。」導不答。又曰:「若不三司,便應令僕邪?」又不答。敦曰:「若不爾,正當誅爾。」導
 又無言。導後料檢中書故事,見顗表救己,殷勤款至。導執表流涕,悲不自勝,告其諸子曰:「吾雖不殺伯仁,伯仁由我而死。幽冥之中,負此良友!」顗三子:閔、恬、頤。



 閔字子騫,方直有父風。歷衡陽、建安、臨川太守,侍中,中領軍,吏部尚書,尚書左僕射,加中軍將軍,轉護軍,領秘書監。卒,追贈金紫光祿大夫,謚曰烈。無子,以弟頤長子琳為嗣。琳仕至東陽太守。恬、頤並歷卿守。琳少子文,驃騎諮議參軍



 史臣曰:夫太剛則折,至察無徒,以之為政,則害于而國;用之行己,則凶於乃家。誠以器乖容眾,非先王之道也。
 大連司憲,陰候主情,當約法之秋,獻斫棺之議。玄亮剛愎,與物多違,雖有崇上之心,專行刻下之化,同薄相濟,並運天機。是使賢宰見疏,致物情於解體;權臣發怒,借其名以誓師。既而謀人之國,國危而茍免;見暱於主,主辱而圖生。自取流亡,非不幸也。若思閑爽,照理研幽。伯仁凝正,處腴能約。咸以高才雅道,參豫疇咨。及京室淪胥,抗言無撓,甘赴鼎而全操,蓋事君而盡節者歟!顗招時論,尤其酒德,《禮經》曰「瑕不掩瑜」,未足韜其美也。



 贊曰:劉刁亮直,志奉興王。奸回醜正,終致奔亡。周戴英爽,忠謨允塞。道屬屯蒙,禍罹兇慝。



\end{pinyinscope}