\article{帝紀第九}

\begin{pinyinscope}

 簡文帝孝武帝



 簡文皇帝諱昱,字道萬,元帝之少子也。幼而岐嶷,為元帝所愛。郭璞見而謂人曰:「興晉祚者,必此人也。」及長,清虛寡欲,尤善玄言。永昌元年,元帝詔曰:「先公武王、先考恭王君臨瑯邪。繼世相承,國嗣未立,蒸嘗靡主,朕常悼心。子昱仁明有智度,可以虔奉宗廟,以慰罔極之恩。其封昱為瑯邪王,食會稽、宣城如舊。」咸和元年,所生鄭夫
 人薨。帝時年七歲,號慕泣血,固請服重。成帝哀而許之,故徙封會稽王,拜散騎常侍。九年,遷右將軍,加侍中。咸康六年,進撫軍將軍,領秘書監。建元元年夏五月癸丑,康帝詔曰:「太常職奉天地,兼掌宗廟,其為任也,可謂重矣。是以古今選建,未嘗不妙簡時望,兼之儒雅。會稽王叔履尚清虛,志道無倦,優游上列,諷議朝肆。其領太常本官如故。」永和元年,崇德太后臨朝,進位撫軍大將軍、錄尚書六條事。二年,驃騎何充卒,崇德太后詔帝專總萬機。八年,進位司徒,固讓不拜。穆帝始冠,帝稽首歸政,不許。廢帝即位,以瑯邪王絕嗣,復徙封瑯邪,而
 封王子昌明為會稽王。帝固讓,故雖封瑯邪而不去會稽之號。太和元年,進位丞相、錄尚書事,入朝不趨,贊拜不名,劍履上殿,給羽葆鼓吹班劍六十人,又固讓。及廢帝廢,皇太后詔曰:「丞相、錄尚書、會稽王體自中宗,明德劭令,英秀玄虛,神棲事外。以具瞻允塞,故阿衡三世。道化宣流,人望攸歸,為日已久。宜從天人之心,以統皇極。主者明依舊典,以時施行。」於是大司馬桓溫率百官進太極前殿,具乘輿法駕,奉迎帝於會稽邸,於朝堂變服,著平巾幘單衣,東向拜受璽綬。



 咸安元年冬十一月己酉,即皇帝位。桓溫出次中堂,令
 兵屯衛。乙卯,溫奏廢太宰、武陵王晞及子總。詔魏郡太守毛安之帥所領宿衛殿內,改元為咸安。庚戌,使兼太尉周頤告于太廟。辛亥,桓溫遣弟秘逼新蔡王晃詣西堂,自列與太宰、武陵王晞等謀反。帝對之流涕,溫皆收付廷尉。癸丑,殺東海二子及其母。初,帝以沖虛簡貴,歷宰三世,溫素所敬憚。及初即位,溫乃撰辭欲自陳述,帝引見,對之悲泣,溫懼不能言。至是,有司承其旨,奏誅武陵王晞,帝不許。溫固執至于再三,帝手詔報曰:「若晉祚靈長,公便宜奉行前詔。如其大運去矣,請避賢路。」溫覽之,流汗變色,不復敢言。乙卯,廢晞及其三子,徙于新
 安。丙辰,放新葵王晁于衡陽。戊午,詔曰:「王室多故,穆哀早世,皇胤夙遷,神器無主。東海王以母弟近屬,入纂大統,嗣位經年,昏闇亂常,人倫虧喪,大禍將及,則我祖宗之靈靡知所託。皇太后深懼皇基,時定大計。大司馬因順天人,協同神略,親帥群后,恭承明命。雲霧既除,皇極載清,乃顧朕躬,仰承弘緒。雖伊尹之寧殷朝,博陸之安漢室,無以尚也。朕以寡德,猥居元首,實懼眇然,不克負荷,戰戰兢兢,罔知攸濟。思與兆庶更始,其大赦天下,大酺五日,增文武位二等,孝順忠貞鰥寡孤獨米人五斛。」己未,賜溫軍三萬人,人布一匹,米一斛。庚申,加大司馬
 桓溫為丞相,不受。辛酉,溫旋自白石,因鎮姑孰。以冠軍將軍毛武生都督荊州之沔中、揚州之義城諸軍事。十二月戊子,詔以京都有經年之儲,權停一年之運。庚寅,廢東海王奕為海西公,食邑四千戶。辛卯,初薦酃淥酒於太廟。



 二年春正月辛丑,百濟、林邑王各遣使貢方物。二月,苻堅伐慕容桓於遼東,滅之。三月丁酉,詔曰:「朕居阿衡三世,不能濟彼時雍,乃至海西失德,殆傾皇祚。賴祖宗靈祗之德,皇太后淑體應期,籓輔忠賢,百官戮力,用能蕩氣務於昊蒼,耀晨輝於宇宙。遂以眇身,託于王公之上,
 思賴群賢,以弼其闕。夫敦本息末,抑絕華競,使清濁異流,能否殊貫,官無秕政,士無謗讟,不有懲勸,則德禮焉施?且彊寇未殄,勞役未息,自非軍國戎祀之耍,其華飾煩費之用皆省之。夫肥遁窮谷之賢,滑泥揚波之士,雖抗志玄霄,潛默幽岫,貪屈高尚之道,以隆協贊之美,孰與自足山水,棲遲丘壑,徇匹夫之潔,而忘兼濟之大邪?古人不借賢於曩代,朕所以虛想於今日。內外百官,各勤所司,使善無不達,惡無不聞,令詩人元素餐之刺,而吾獲虛心之求焉。」癸丑,詔曰:「吾承祖宗洪基,而昧于政道,懼不能允釐天工,克隆先業,夕惕惟憂,若涉泉冰。賴
 宰輔忠德,道濟伊望,群后竭誠,協契斷金,內外盡匡翼之規,文武致匪躬之節,冀因斯道,終克弘濟。每念干戈未戢,公私疲悴,籓鎮有疆理之務,征戍懷東山之勤,或白首戎陣,忠勞未敘,或行役彌久,擔石靡儲,何嘗不昧旦晨興,夜分忘寢。雖未能撫而巡之,且欲達其此心。可遣大使詣大司馬,并問方伯,逮于邊戍,宣詔大饗,求其所安。又籌量賜給,悉令周普。」乙卯,詔曰:「往事故之後,百度未充,群僚常俸,並皆寡約,蓋隨時之義也。然退食在朝,而祿不代耕,非經通之制。今資儲漸豐,可籌量增俸。」騶虞見豫章。夏四月,徙海西公於吳縣西柴里。追貶庾
 后曰夫人。六月,遣使拜百濟王餘句為鎮東將軍,領樂浪太守。戊子,前護軍將軍庚希舉兵反,自海陵入京口,晉陵太守卞眈奔于曲阿。秋七月壬辰,桓溫遣東海內史周少孫討希,擒之,斬于建康市。己未,立會稽王昌明為皇太子,皇子道子為瑯邪王,領會稽內史。是日,帝崩于東堂,時年五十三。葬高平陵,廟號太宗。遺詔以桓溫輔政,依諸葛亮、王導故事。



 帝少有風儀,善容止,留心典籍,不以居處為意,凝塵滿席,湛如也。嘗與桓溫及武陵王晞同載遊版橋,溫遽令鳴鼓吹角,車馳卒奔,欲觀其所為。晞大恐,求下車,而帝安然無懼色,溫由此憚服。溫
 既仗文武之任,屢建大功,加以廢立,威振內外。帝雖處尊位,拱默守道而已,常懼廢黜。先是,熒惑入太微,尋而海西廢。及帝登阼,熒惑又入太微,帝甚惡焉。時中書郎郗超在直,帝乃引入,謂曰:「命之修短,本所不計,故當無復近日事邪!」超曰:「大司馬臣溫方內固社稷,外恢經略,非常之事,臣以百口保之。」及超請急省其父,帝謂之曰:「致意尊公,家國之事,遂至於此!由吾不能以道匡衛,愧歎之深,言何能喻。」因詠庾闡詩云「志士痛朝危,忠臣哀主辱」,遂泣下霑襟。帝雖神識恬暢,而無濟世大略,故謝安稱為惠帝之流,清談差勝耳。沙門支道林嘗言「會嶴
 有遠體而無遠神」。謝靈運迹其行事,亦以為赧獻之輩云。



 孝武皇帝諱曜,字昌明,簡文帝第三子也。興寧三年七月甲申,初封會稽王。咸安二年秋七月己未,立為皇太子。是日,簡文帝崩,太子即皇帝位。詔曰:「朕以不造,奄丁閔凶,號天扣地。靡知所訴。藐然幼沖,眇若綴旒,深惟社稷之重,大懼不克負荷。仰憑祖宗之靈,積德之祀。先帝淳風玄化,遺詠在民。宰輔英賢,勛隆德盛。顧命之託,實賴匡訓。群后率職,百僚勤政。冀孤弱之躬有寄,皇極之
 基不墜。先恩遺惠,播于四海,思弘餘潤,以康黎庶。其大赦天下,與民更始。」九月甲寅,追尊皇妣會稽王妃曰順皇后。冬十月丁卯,葬簡文皇帝于高平陵。十一月甲午,妖賊盧悚晨人殿庭,游擊將軍毛安之等討擒之。是歲,三吳大旱,人多餓死,詔所在振給。苻堅陷仇池,執秦州刺史楊世。



 寧康元年春正月己丑朔,改元。二月,大司馬桓溫來朝。三月癸丑,詔除丹陽竹格等四桁稅。夏五月,旱。秋七月己亥,使持節、侍中、都督中外諸軍事、丞相、錄尚書、大司馬、揚州牧、平北將軍、徐兗二州刺史、南郡公桓溫薨。庚
 戌,進右將軍桓豁為征西將軍。以江州刺史桓沖為中軍將軍、都督揚豫江三州諸軍事、揚州刺史,鎮姑孰。八月壬子,崇德太后臨朝攝政。九月,苻堅將楊安寇成都。丙申,以尚書僕射王彪之為尚書令,吏部尚書謝安為尚書僕射,吳國內史刁彞為北中郎將、徐兗二州刺史,鎮廣陵。復置光祿勛、大司農、少府官。冬十月,西平公張天錫貢方物。十一月,苻堅將楊安陷梓潼及梁、益二州,刺史周仲孫帥騎五千南遁。



 二年春正月癸未朔,大赦。追封謚故會稽世子郁為臨川獻王。己酉,北中郎將、徐兗二州刺史刁彞卒。二月癸
 丑,以丹陽尹王坦之為北中郎將、徐兗二州刺史。丁巳,有星孛于女虛。三月丙戌,彗星見于氐。夏四月壬戌,皇太后詔曰:「頃玄象忒愆,上天表異,仰觀斯變,震懼于懷。夫因變致休,自古之道,朕敢不剋意復心,以思厥中?又三吳奧壤,股肱望郡,而水旱併臻,百姓失業,夙夜惟憂,不能忘懷,宜時拯恤,救其彫困。三吳義興、晉陵及會稽遭水之縣尤甚者,全除一年租布,其次聽除半年,受振貸者即以賜之。」五月,蜀人張育自號蜀王,帥眾圍成都,遣使稱籓。秋七月,涼州地震,山崩。苻堅將鄧羌攻張育,滅之。八月,以長秋將建,權停婚姻。九月丁丑,有星孛于
 天市。冬十一月己酉,天門蜑賊攻郡,太守王匪死之,征西將軍桓豁遣師討平之。長城人錢步射、錢弘等作亂,吳興太守朱序討平之。癸酉,鎮遠將軍桓石虔破苻堅將姚萇於墊江。



 三年春正月辛亥,大赦。夏五月丙午,北中郎將、徐兗二州刺史、藍田侯王坦之卒。甲寅,以中軍將軍、揚州刺史桓沖為鎮北將軍、徐州刺史,鎮丹徒,尚書僕射謝安領揚州刺史。秋八月癸巳,立皇后王氏,大赦,加文武位一等。九月,帝講《孝經》。冬十月癸酉朔,日有蝕之。十二月癸未,神獸門災。甲申,皇太后詔曰:「頃日蝕告變,水旱不適,
 雖克己思救,未盡其方。其賜百姓窮者米,人五斛。」癸巳,帝釋奠于中堂,祠孔子,以顏回配。



 太元元年春正月壬寅朔,帝加元服,見于太廟。皇太后歸政。甲辰,大赦,改元。丙午,帝始臨朝。以征西將軍桓豁為征西大將軍,領軍將軍郗愔為鎮軍大將軍,中軍將軍桓沖為車騎將軍,加尚書僕射謝安中書監、錄尚書事。甲子,謁建平等四陵。夏五月癸丑,地震。甲寅,詔曰:「頃者上天垂監,譴告屢彰,朕有懼焉,震惕于心。思所以議獄緩死,赦過宥罪,庶因大變,與之更始。」於是大赦,增文武位各一等。六月,封河間王欽子範之為章武王。秋七
 月,苻堅將茍萇陷涼州,虜刺史張天錫,盡有其地。乙巳,除度田收租之制,公王以下口稅米三斛,蠲在役之身。冬十月,移準北流人於淮南。十一月己巳朔,日有蝕之。詔太官撤膳。十二月,苻堅使其將苻洛攻代,執代王涉翼犍。



 二年春正月,繼絕世,紹功臣。三月,以兗州刺史朱序為南中郎將、梁州刺史、監沔中諸軍,鎮襄陽。閏月壬午,地震。甲申,暴風,折木發屋。夏四月己酉,雨雹。五月丁丑,地震。六月己巳,暴風,揚沙石。林邑貢方物。秋七月乙卯,老人星見。八月壬辰,四騎將軍桓沖來朝。丁未,以尚書僕
 射謝安為司徒。丙辰,使持節、都督荊梁寧益交廣六州諸軍事、荊州刺史、征西大將軍桓豁卒。冬十月辛丑,以車騎將軍桓沖都督荊江梁益寧交廣七州諸軍事、領護南蠻校尉、荊州刺史,尚書王蘊為徐州刺史、督江南晉陵諸軍,征西司馬謝玄為兗州刺史、廣陵相、監江北諸軍。壬寅。散騎常侍、左光祿大夫、尚書令王彪之卒。十二月庚寅,以尚書王劭為尚書僕射。



 三年春二月乙巳,作新宮,帝移居會稽王邸。三月乙丑,雷雨,暴風,發屋折木。夏五月庚午,陳留王曹恢薨。六月,大水。秋七月辛巳,帝入新宮。乙酉,老人星見南方。



 四年春正月辛酉,大赦,郡縣遭水旱者減租稅。丙子,謁建平等七陵。二月戊午,苻堅使其子丕攻陷襄陽,執南中郎將朱序。又陷順陽。三月,大疫。壬戌,詔曰:「狡寇縱逸,籓守傾沒,疆埸之虞,事兼平日。其內外眾官,各悉心戮力,以康庶事。又年穀不登,百姓多匱。其詔御所供,事從儉約,九親供給,眾官廩俸,權可減半。凡諸役費,自非軍國事要,皆宜停省,以周時務。」癸未,使右將軍毛武生帥師伐蜀。夏四月,苻堅將韋鐘陷魏興,太守吉挹死之。五月,苻堅將句難、彭超陷盱眙、高密內史毛璪之為賊所執。六月,大旱。戊子,征虜將軍謝玄及超、難戰于君川,大
 破之。秋八月丁亥,以左將軍王蘊為尚書僕射。乙未,暴風,揚沙石。九月,盜殺安太守傅湛。冬十二月己酉朔,日有蝕之。



 五年春正月乙巳,謁崇平陵。夏四月,大旱。癸酉,大赦五歲刑以下。五月,大水。以司徒謝安為衛將軍、儀同三司。六月甲寅,震含章殿四柱,并殺內侍二人。甲子,以比歲荒儉,大赦,自太元三年以前逋租宿債皆蠲除之,其鰥寡窮獨孤老不能自存者,人賜米五斛。丁卯,以驃騎將軍、瑯邪王道子為司徒。秋九月癸未,皇后王氏崩。冬十月,九真太守李遜據交州反。十一月乙酉,葬定皇后于
 隆平陵。



 六年春正月,帝初奉佛法,立精舍於殿內,引諸沙門以居之。丁酉,以尚書謝石為尚書僕射。初置督運御史官。夏六月庚子朔,日有蝕之。揚、荊、江三州大水。己巳,改制度,減煩費,損吏士員七百人。秋七月丙子,赦五歲刑已下。甲午,交址太守杜瑗斬李遜,交州平。大饑。冬十一月己亥,以鎮軍大將軍郗愔為司空。會稽人檀元之反,自號安東將軍,鎮軍參軍謝藹之討平之。十二月甲辰,苻堅遣其襄陽太守閻震寇竟陵,襄陽太守桓石虔討擒之。



 七年春三月,林邑范熊遣使獻方物。秋八月癸卯,大赦。九月,東夷五國遣使來貢方物。苻堅將都貴焚燒沔北田穀,略襄陽百姓而去。冬十月丙子,雷。



 八年春二月癸未,黃霧四塞。三月,始興、南康、廬陵大水,平地五丈。丁巳,大赦。夏五月,輔國將軍楊亮伐蜀,拔五城,擒苻堅將魏光。秋七月,鷹揚將軍郭洽及苻堅將張崇戰于武當,大敗之。八月,苻堅帥眾渡淮,遣征討都督謝石、冠軍將軍謝玄、輔國將軍謝琰、西中郎將桓伊等距之。九月,詔司徒、瑯邪王道子錄尚書六條事。冬十月,苻堅弟融陷壽春。乙亥,諸將及苻堅戰于肥水,大破之,
 俘斬數萬計,獲堅輿輦及雲母車。十一月庚申,詔衛將軍謝安勞旋師於金城。壬子,立陳留王世子靈誕為陳留王。十二月庚午,以寇難初平,大赦。以中軍將軍謝石為尚書令。開酒禁。始增百姓稅米,口五石。前句町王翟遼背苻堅,舉兵於河南,慕容垂自鄴與遼合,遂攻堅子暉於洛陽。仇池公楊世奔還隴右,遣使稱籓。



 九年春正月庚子,封武陵王孫寶為臨川王。戊午,立新寧王晞子遵為新寧王。辛亥,謁建平等四陵。龍驤將軍劉牢之克譙城。車騎將軍桓沖部將郭寶伐新城、魏興、上庸三郡,降之。二月辛巳,使持節、都督荊江梁寧益交
 廣七州諸軍事、車騎將軍、荊州刺史桓沖卒。慕容垂自洛陽與翟遼攻苻堅子丕於鄴。三月,以衛將軍謝安為太保。苻堅北地長史慕容泓、平陽太守慕容沖並起兵背堅。夏四月己卯,增置太學生百人。封張天錫為西平公。使竟陵太守趙統伐襄陽,克之。苻堅將姚萇背堅,起兵於北地,自立為王,國號秦。六月癸丑朔,崇德皇太后褚氏崩。慕容泓為其叔父沖所殺,沖自稱皇太弟。秋七月戊戌,遣兼司空、高密王純之修謁洛陽五陵。己酉,葬康獻皇后于崇平陵。百濟遣使來貢方物。苻堅及慕容沖戰于鄭西,堅師敗績。八月戊寅,司空郗愔薨。九月辛
 卯,前鋒都督謝玄攻苻堅將兗州刺史張崇於鄄城,克之。甲午,加太保謝安大都督揚、江、荊、司、豫、徐、兗、青、冀、幽、并、梁、益、雍、涼十五州諸軍事。冬十月辛亥朔,日有蝕之。丁巳,河間王曇之薨。乙丑,以玄象乖度,大赦。庚午,立前新蔡王晃弟崇為新蔡王。苻堅青州刺史苻朗帥眾來降。十二月,苻堅將呂光稱制于河右,自號酒泉公。慕容沖僭即皇帝位于阿房。



 十年春正月甲午,謁諸陵。二月,立國學。蜀郡太守任權斬苻堅益州刺史李平,益州平。三月,滎陽人鄭燮以郡來降。苻堅國亂,使使奉表請迎。龍驤將軍劉牢之及慕
 容垂戰于黎陽,王師敗績。夏四月丙辰,劉牢之與沛郡太守周次及垂戰于五橋澤,王師又敗績。壬戌,太保謝安帥眾救苻堅。五月,大水。苻堅留太子宏守長安,奔於五將山。六月,宏來降,慕容沖入長安。秋七月,苻丕自枋頭西走,龍驤將軍檀玄追之,為丕所敗。旱,饑。丁巳,老人星見。八月甲午,大赦。丁酉,使持節、侍中、中書監、大都督十五州諸軍事、衛將軍、太保謝安薨。庚子,以瑯邪王道子為都督中外諸軍事。是月,姚萇殺苻堅而僭即皇帝位。九月,呂光據姑臧,自稱涼州刺史。苻丕僭即皇帝位于晉陽。冬十月丁亥,論淮肥之功,追封謝安廬陵郡公,
 封謝石南康公,謝玄康樂公,謝琰望蔡公,桓伊永修公,自餘封拜各有差。是歲,乞伏國仁自稱大單于、秦河二州牧。



 十一年春正月辛未,慕容垂僭即皇帝位于中山。壬午,翟遼襲黎陽,執太守滕恬之。乙酉。謁諸陵。慕容沖將許木末殺慕容沖於長安。三月,大赦。太山太守張願以郡叛,降於翟遼。夏四月,以百濟王世子餘暉為使持節、都督、鎮東將軍、百濟王。代王拓拔珪始改稱魏。癸巳,以尚書僕射陸納為尚書左僕射,譙王恬為尚書右僕射。六月己卯,地震。庚寅,以前輔國將軍楊亮為西戎校尉、雍
 州刺史,鎮衛山陵。秋八月庚午,封孔靖之為奉聖亭侯,奉宣尼祀。丁亥,安平王邃之薨。翟遼寇譙,龍驤將軍朱序擊走之。冬十月,慕容垂破苻丕於河東,丕走東垣,揚威將軍馮該擊斬之,傳首京都。甲申,海西公奕薨。十一月,苻丕將苻登僭即皇帝位於隴東。



 十二年春正月乙巳,以豫州刺史朱序為青、兗二州刺史,鎮淮陰。丁未,大赦。壬子,暴風,發屋折木。戊午,慕容垂寇河東,濟北太守溫詳奔彭城。翟遼遣子釗寇陳、潁,朱序擊走之。夏四月戊辰,尊夫人李氏為皇太妃。己丑,雨雹。高平人翟暢執太守徐含遠,以郡降于翟遼。六月癸
 卯,束帛聘處士戴逵、襲玄之。秋八月辛巳,立皇子德宗為皇太子,大赦,增文武位二等,大酺五日,賜百官布帛各有差。九月戊午,復新寧王遵為武陵王,立梁王逢子和為梁王。冬十一月,松滋太守王遐之討翟遼于洛口,敗之。



 十三年夏四月戊午,以青兗二州刺史朱序為持節、都督雍梁沔中九郡諸軍事、雍州刺史,譙王恬之為鎮北將軍、青兗二州刺史。夏六月,旱。乞伏國仁死,弟乾歸嗣偽位,僭號河南王。秋七月,翟遼將翟發寇洛陽,河南太守郭給距破之。冬十二月戊子,濤水入石頭,毀大桁,殺
 人。乙未,大風,晝晦,延賢堂災。丙申,螽斯則百堂、客館、驃騎庫皆災。己亥,加尚書令謝石衛將軍、開府儀同三司。庚子,尚書令、衛將軍、開府儀同三司謝石薨。



 十四年春正月癸亥,詔淮南所獲俘虜付諸作部者一皆散遣,男女自相配匹,賜百日廩,其沿線為軍賞者悉贖出之,以襄陽、淮南饒沃地各立一縣以居之。彭城妖賊劉黎僭稱皇帝於皇丘,龍驤將軍劉宰之討平之。二月,扶南獻方物。呂光僭號三河王。夏四月甲辰,彭城王弘之薨。翟遼寇滎陽,執太守張卓。六月壬寅,使持節、都督荊益寧三州諸軍事、荊州刺史桓石虔卒。秋七月甲寅,
 宣陽門四柱災。八月,姚萇襲破苻登,獲其偽后毛氏。丁亥,汝南王羲薨。九月庚午,以尚書左僕射陸納為尚書令。冬十二月乙巳,雨,木冰。



 十五年春正月乙亥,鎮北將軍、譙王恬之薨。龍驤將軍劉牢之及翟遼、張願戰于太山,王師敗績。征虜將軍朱序破慕容永於太行。二月辛己,以中書令王恭為都督青兗幽并冀五州諸軍事、前將軍、青兗二州刺史。三月己酉朔,地震。戊辰,大赦。秋七月丁巳,有星孛于北河。八月,永嘉人李耽舉兵反,太守劉懷之討平之。己丑,京師地震。有星孛于北斗,犯紫微。沔中諸郡及兗州大水。龍
 驤將軍朱序攻翟遼于滑臺,大敗之,張願來降。九月丁未,以吳郡太守王珣為尚書僕射。冬十二月己未,地震。



 十六年春正月庚申,改築太廟。夏六月,慕容永寇河南,太守楊佺期擊破之。己未,章武王範之薨。秋九月癸未,以尚書右僕射王珣為尚書左僕射,以太子詹事謝琰為尚書右僕射。新廟成。冬十一月,姚萇敗苻登于安定。



 十七年春正月己巳朔,大赦,除逋租宿債。夏四月,齊國內史蔣喆殺樂安太守辟閭濬,據青州反,北平原太守辟閭渾討平之。五月丁卯朔,日有蝕之。六月癸卯,京師地震。甲寅,濤水入石頭,毀大桁。永嘉郡潮水湧起,近海
 四縣人多死者。乙卯,大風,折木。戊午,梁王薨。慕容垂襲翟釗于黎陽,敗之,釗奔于慕容永。秋七月丁丑,太白晝見。八月,新作東宮。冬十月丁酉,太白晝見。辛亥,都督荊益寧三州諸軍事、荊州刺史王忱卒。十一月癸酉,以黃門郎殷仲堪為都督荊益梁三州諸軍事、荊州刺史。庚寅,徙封瑯邪王道子為會稽王,封皇子德文為瑯邪王。十二月己未,地震。是歲,自秋不雨,至于冬。



 十八年春正月癸亥朔,地震。二月乙未,地又震。三月,翟釗寇河南。夏六月己亥,始興、南康、廬陵大水,深五丈。秋七月,旱。閏月,妖賊司馬徽聚黨於馬頭山,劉牢之遣部
 將討平之。九月丙戌,龍驤將軍楊佺期擊氐帥楊佛嵩于潼谷,敗之。冬十月,姚萇死,子興嗣偽位。



 十九年夏六月壬子,追尊會稽王太妃鄭氏為簡文宣太后。秋七月,荊、徐二州大水,傷秋稼,遣使振恤之。八月己巳,尊皇太妃李氏為皇太后,宮曰崇訓。慕容垂擊慕容永於長子,斬之。冬十月,慕容垂遣其子惡奴寇廩丘,東平太守韋簡及垂將尹國戰于平陸,簡死之。是歲,苻登為姚興所殺,登太子崇奔于湟中,僭稱皇帝。



 二十年春二月,作宣太后廟。甲寅,散騎常侍、光祿大夫、開府儀同三司、尚書令陸納卒。三月庚辰朔,日有蝕之。
 夏六月,荊、徐二州大水。十一月,魏王拓拔珪擊慕容垂子寶于黍谷,敗之。



 二十一年春正月,造清暑殿。三月,慕容垂攻平城,拔之。夏四月,新作永安宮。丁亥,雨雹。慕容垂死,子寶嗣偽位。五月甲子,以望蔡公謝琰為尚書左僕射。大水。六月,呂光僭即天王位。秋九月庚申,帝崩于清暑殿,時年三十五。葬隆平陵。



 帝幼稱聰悟。簡文之崩也,時年十歲,至晡不臨,左右進諫,答曰:「哀至則哭,何常之有?」謝安嘗嘆以為精理不減先帝。既威權己出,雅有人主之量。既而溺於酒色,殆為長夜之飲。末年長星見,帝心甚惡之,於華
 林園舉酒祝之曰:「長星,勸汝一杯酒,自古何有萬歲天子邪!」太白連年晝見,地震水旱為變者相屬。醒日既少,而傍無正人,竟不能改焉。時張貴人有寵,年幾三十,帝戲之曰:「汝以年當廢矣。」貴人潛怒,向夕,帝醉,遂暴崩。時道子昏惑,元顯專權,竟不推其罪人。初,簡文帝見讖云:「晉祚盡昌明。」及帝之在孕也,李太后夢神人謂之曰:「汝生男,以『昌明』為字。」及產,東方始明,因以為名焉。簡文帝後悟,乃流涕。及為清暑殿,有識者以為「清暑」反為「楚」聲,哀楚之徵也。俄而帝崩,晉祚自此傾矣。



 史臣曰:前史稱「不有廢也,吾何以興」,若乃天挺惟神,光
 膺嗣位,邁油雲而驤首,濟沈川而能躍。少康一旅之眾,所以闡帝圖;成湯七十之基,所以興王業;靜河海於既泄,補穹圓於已紊;事異於斯,則弗由也。簡皇以虛白之姿,在屯如之會,政由桓氏,祭則寡人。太宗晏駕,寧康纂業,天誘其衷,姦臣自隕。于時西踰劍岫而跨靈山,北振長河而臨清洛;荊吳戰旅,嘯叱成雲;名賢間出,舊德斯在:謝安可以鎮雅俗,彪之足以正紀綱,桓沖之夙夜王家,謝玄之善斷軍事。於時上天乃眷,彊氐自泯。五盡童子,振袂臨江,思所以挂旆天山,封泥函谷;而條綱弗垂,威思罕樹,道子荒乎朝政,國寶匯以小人,拜授之榮,初
 非天旨,鬻刑之貨,自走權門,毒賦年滋,愁民歲廣。是以聞人、許榮馳書詣闕,烈宗知其抗直,而惡聞逆耳,肆一醉於崇朝,飛千觴於長夜。雖復「昌明」表夢,安聽神言?而金行頹弛,抑亦人事,語曰「大國之政未陵夷,小邦之亂已傾覆」也。屬苻堅百六之秋,棄肥水之眾,帝號為「武」,不亦優哉!



 贊曰:君若綴旒,道非交泰。簡皇凝寂,不貽伊害。孝武登朝,奸雄自消。燕之擊路,鄭叔分鑣。倡臨帝席,酒勸天妖。金風不競,人事先凋。



\end{pinyinscope}