\article{列傳第四十六}

\begin{pinyinscope}
王舒
 \gezhu{
  子允之}
 王廙
 \gezhu{
  弟彬彬子彪之彬從兄棱}
 虞潭
 \gezhu{
  孫嘯父兄子斐}
 顧
 眾張闓



 王舒,字處明,丞相導之從弟也。父會,侍御史。舒少為從兄敦所知,以天下多故,不營當時名,恒處私門,潛心學植。年四十餘,州禮命,太傅辟,皆不就。及敦為青州,舒往依焉。時敦被徵為祕書監,以寇難路險,輕騎歸洛陽,委棄公主。時輜重金寶甚多,親賓無不競取,惟舒一無所眄,益為敦所賞。



 及元帝鎮建康,因與諸父兄弟俱渡江
 委質焉。參鎮東軍事,出補溧陽令。明帝之為東中郎將,妙選上佐,以舒為司馬。轉後將軍、宣城公褚裒諮議參軍,遷軍司,固辭不受。裒鎮廣陵,復以舒為車騎司馬。頻領望府,咸稱明練。裒薨,遂代裒鎮,除北中郎將、監青徐二州軍事。頃之,徵國子博士,加散騎常侍,未拜,轉少府。太寧初,徙廷尉。敦表舒為鷹揚將軍、荊州刺史、領護南蠻校尉、監荊州沔南諸軍事。及敦敗,王含父子俱奔舒,舒遣軍逆之,並沈於江。進都督荊州、平西將軍、假節。尋以陶侃代舒,遷舒為安南將軍、廣州刺史。舒疾病,不樂越嶺,朝議亦以其有功,不應遠出,乃徙為湘州刺史,將
 軍、都督、持節如故。徵代鄧攸為尚書僕射。



 時將征蘇峻,司徒王導欲出舒為外援,乃授撫軍將軍、會稽內史,秩中二千石。舒上疏辭以父名,朝議以字同音異,於禮無嫌。舒復陳音雖異而字同,求換他郡。於是改「會」字為「鄶」。舒不得已而行。在郡二年而蘇峻作逆,乃假舒節都督,行揚州刺史事。時吳國內史庾冰棄郡奔舒,舒移告屬縣,以吳王師虞斐為軍司,御史中丞謝藻行龍驤將軍、監前鋒征討軍事,率眾一萬,與庾冰俱渡浙江。前義興太守顧眾、護軍參軍顧颺等,皆起義軍以應舒。舒假眾揚威將軍、督護吳中軍事,颺監晉陵軍事,於御亭築壘。
 峻聞舒等兵起,乃赦庾亮諸弟,以悅東軍。舒率眾次郡之西江,為冰、藻後繼。冰、颺等遣前鋒進據無錫,遇賊將張健等數千人,交戰,大敗,奔還御亭,復自相驚擾,冰、颺等並退於錢唐,藻守嘉興。賊遂入吳,燒府舍,掠諸縣,所在塗地。舒以輕進奔敗,斬二軍主者,免冰、颺督護,以白衣行事。更以顧眾督護吳晉陵軍,屯兵章埭。吳興太守虞潭率所領討健,屯烏苞亭,並不敢進。時暴雨大水,賊管商乘船旁出,襲潭及眾。潭等奔敗。潭還保吳興,眾退守錢唐。舒更遣將軍陳孺率精銳千人增戍海浦,所在築壘。或勸舒宜還都,使謝藻守西陵,扶海立柵。舒不聽,
 留藻守錢唐,使眾、颺守紫壁。於是賊轉攻吳興,潭諸軍復退。賊復掠東遷、餘杭、武康諸縣。舒遣子允之行揚烈將軍,與將軍徐遜、陳孺及揚烈司馬朱燾,以精銳三千,輕邀賊於武康,出其不意,遂破之,斬首數百級,賊悉委舟步走。允之收其器械,進兵助潭。時賊韓晃既破宣城,轉人故鄣、長城。允之遣朱燾、何準等於之,戰擊於湖。潭以彊弩射之,晃等退走,斬首千餘級,納降二千人。潭由是得保郡。是時臨海、新安諸山縣並反應賊,舒分兵悉討平之。會陶侃等至京都,舒、潭等並以屢戰失利,移書盟府,自貶去節。侃遣使敦喻,不聽。及侃立行臺,上舒監
 浙江東五郡軍事,允之督護吳郡、義興、晉陵三郡征討軍事。既而晃等南走,允之追躡於長塘湖,復大破之。賊平,以功封彭澤縣侯,尋卒官,贈車騎大將軍、儀同三司,謚曰穆。



 長子晏之,蘇峻時為護軍參軍,被害。晏之子崐之嗣。卒,子陋之嗣。宋受禪,國除。晏之弟允之最知名



 允之字深猷。總角,從伯敦謂為似己,恒以自隨,出則同輿,入則共寢。敦嘗夜飲,允之辭醉先臥。敦與錢鳳謀為逆,允之已醒,悉聞其言,慮敦或疑己,便於臥處大吐,衣面並污。鳳既出,敦果照視,見允之臥吐中,以為大醉,不復疑之。時父舒始拜廷尉,允之求還定省,敦許之。至都,
 以敦、鳳謀議事白舒,舒即與導俱啟明帝。



 舒為荊州,允之隨在西府。及敦平,帝欲令允之仕,舒請曰:「臣子尚少,不樂早官。」帝許隨舒之會稽。及蘇峻反,允之討賊有功,封番禺縣侯,邑千六百戶,除建武將軍、錢唐令,領司鹽都尉。舒卒,去職。既葬,除義興太守,以憂哀不拜,從伯導與其書曰:「太保、安豐侯以孝聞天下,不得辭司隸;和長輿海內名士,不免作中書令。吾群從死亡略盡,子弟零落,遇汝如親,如其不爾,吾復何言!」允之固不肯就。咸和末,除宣城內史、監揚州江西四郡事、建武將軍,鎮于湖。咸康中,進號西中郎將、假節。尋遷南中郎將、江州刺史。
 蒞政甚有威惠。時王恬服闋,除豫章郡。允之聞之驚愕,以為恬丞相子,應被優遇,不可出為遠郡,乃求自解州,欲與庾冰言之。冰聞甚愧,即以恬為吳郡,而以允之為衛將軍、會稽內史。未到,卒,年四十。謚曰忠。



 子晞之嗣。卒,子肇之嗣。



 王廙,字世將,丞相導從弟,而元帝姨弟也。父正,尚書郎。廙少能屬文,多所通涉,工書畫,善音樂、射御、博弈、雜伎。辟太傅掾,轉參軍。豫迎大駕,封武陵縣侯,拜尚書郎,出為濮陽太守。元帝作鎮江左,廙棄郡過江。帝見之大悅,
 以為司馬。頻守廬江、鄱陽二郡。豫討周馥、杜韜,以功累增封邑,除冠軍將軍,鎮石頭,領丞相軍諮祭酒。王敦啟為寧遠將軍、荊州刺史。



 及帝即位,廙奏《中興賦》,上疏曰:



 臣託備肺腑,幼蒙洪潤,愛自齠齔,至於弱冠,陛下之所撫育,恩侔於兄弟,義同於交友,思欲攀龍鱗附鳳翼者,有年矣,是以昔忝濮陽,棄官遠跡,扶持老母,攜將細弱,越長江歸陛下者,誠以道之所存,願託餘廕故也。天誘其願,遇陛下中興,當大明之盛,而守局遐外,不得奉瞻大禮,聞問之日,悲喜交集。昔司馬相如不得睹封禪之事,慷慨發憤,況臣情則骨肉,服膺聖化哉!



 又臣昔嘗侍
 於先后,說陛下誕育之日,光明映室,白毫生於額之左,相者謂當王有四海。又臣以壬申歲見用為鄱陽內史,七月,四星聚于牽牛。又臣郡有枯樟更生。及臣後還京都,陛下見臣白兔,命臣作賦。時瑯邪郡又獻甘露,陛下命臣嘗之。又驃騎將軍導向臣說晉陵有金鐸之瑞,郭璞云必致中興。璞之爻筮,雖京房、管輅不過也。明天之歷數在陛下矣。



 臣少好文學,志在史籍,而飄放遐外,嘗與桀寇為對。臣犬馬之年四十三矣,未能上報天施,而愆負屢彰。恐先朝露,填溝壑,令微情不得上達,謹竭其頑,獻《中興賦》一篇。雖未雖以宣揚盛美,亦是詩人嗟
 嘆詠歌之義也。



 文多不載。



 初,王敦左遷陶侃,使廙代為荊州。將吏馬俊、鄭攀等上書請留侃,敦不許。廙為俊等所襲,奔於江安。賊杜曾與俊、攀北迎第五猗以距廙。廙督諸軍討曾,又為曾所敗。敦命湘州刺史甘卓、豫章太守周廣等助廙擊曾,曾眾潰,廙得到州。廙性俊率,嘗從南下,旦自尋陽,迅風飛帆,暮至都,倚舫樓長嘯,神氣甚逸。王導謂庾亮曰:「世將為傷時識事。」亮曰:「正足舒其逸氣耳。」廙在州大誅戮侃時將佐,及徵士皇甫方回,於是大失荊土之望,人情乖阻。帝乃徵廙為輔國將軍,加散騎常侍。以母喪去職。服闋,拜征虜將軍,進左衛將軍。



 及
 王敦構禍,帝遣廙喻敦,既不能諫其悖逆,乃為敦所留,受任助亂。敦得志,以廙為平南將領護南蠻校尉、荊州刺史。尋病卒。帝猶以親故,深痛愍之。喪還京都,皇太子親臨拜柩,如家人之禮。贈侍中、驃騎將軍,謚曰康。明帝與大將軍溫嶠書曰:「痛謝鯤未絕於口,世將復至於此。並盛年雋才,不遂其志,痛切于心。廙明古多通,鯤遠有識致。其言雖未足令人改聽,然味之不倦,近未易有也。坐相視盡,如何!」



 子頤之嗣,仕至東海內史。頤之弟胡之,字修齡,弱冠有聲譽,歷郡守、侍中、丹陽尹。素有風眩疾,發動甚數,而神明不損。石季龍死,朝廷欲綏輯河洛,
 以胡之為西中郎將、司州刺史、假節,以疾固辭,未行而卒。子茂之亦有美譽,官至晉陵太守。子敬弘,義熙末為尚書。



 彬字世儒。少稱雅正,弱冠,不就州郡之命。光祿大夫傅祗辟為掾。後與兄廙俱渡江,為揚州刺史劉機建武長史。元帝引為鎮東賊曹參軍,轉典兵參軍。豫討華軼功,封都亭侯,愍帝召為尚書郎,以道險不就。遷建安太守,徙義興內史,未之職,轉軍諮祭酒。



 中興建,稍遷侍中。從兄敦舉兵石頭,帝使彬勞之。會周顗遇害,彬素與顗善,先往哭顗,甚慟。既而見敦,敦怪其有慘容,而問其所
 以。彬曰:「向哭伯仁,情未能已。」敦怒曰:「伯仁自致刑戮,且凡人遇汝,復何為者哉!」彬曰:「伯仁長者,君之親友,在朝雖無謇諤,亦非阿黨,而赦後加以極刑,所以傷惋也。」因勃然數敦曰:「兄抗旌犯順,殺戮忠良,謀圖不軌,禍及門戶。」音辭慷慨,聲淚俱下。敦大怒,厲聲曰:「爾狂悖乃可至此,為吾不能殺汝邪!」時王導在坐,為之懼,勸彬起謝。彬曰:「有腳疾已來,見天子尚欲不拜,何跪之有!此復何所謝!」敦曰:「腳痛孰若頸痛?」彬意氣自若,殊無懼容。後敦議舉兵向京師,彬諫甚苦。敦變色目左右,將收彬,彬正色曰:「君昔歲害兄,今又殺弟邪?」先是,彬從兄豫章太守棱
 為敦所害,敦以彬親故容忍之。俄而以彬為豫章太守。彬為人樸素方直,乏風味之好,雖居顯貴,常布衣蔬食。遷前將軍、江州刺史。



 及敦死,王含欲投王舒,王應勸含投彬。含曰:「大將軍平素與江州云何,汝欲歸之?」應曰:「此乃所以宜往也。江州當人彊盛時,能立同異,此非常人所及。睹衰厄,必興愍惻。荊州守文,豈能意外行事!」含不從,遂共投舒,舒果沈含父子于江。彬聞應來,密具船以待之。既不至,深以為恨。



 敦平,有司奏彬及兄子安成太守籍之,並是敦親,皆除名。詔曰:「司徒導以大義滅親,其後昆雖或有違,猶將百世宥之,況彬等公之近親。」乃原
 之。徵拜光祿勛,轉度支尚書。蘇峻平後,改築新宮,彬為大匠。以營創勛勞,賜爵關內侯,遷尚書右僕射。卒官,年五十九。贈特進、衛將軍,加散騎常侍,謚曰肅。長子彭之嗣,位至黃門郎。次彪之,最知名。



 彪之字叔武。年二十,鬚鬢皓白,時人謂之王白鬚。初除佐著作郎、東海王文學。從伯導謂曰:「選官欲以汝為尚書郎,汝幸可作諸王佐邪!」彪之曰:「位之多少既不足計,自當任之於時,至於超遷,是所不願。」遂為郎。鎮軍將軍、武陵王晞以為司馬,累遷尚書左丞、司徒左長史、御史中丞、侍中、廷尉。



 時永嘉太守謝毅。赦後殺郡人周矯,矯
 從兄球詣州訴冤。揚州刺史殷浩遣從事疏收毅,付廷尉。彪之以球為獄主,身無王爵,非廷尉所料,不肯受,與州相反復。穆帝發詔令受之。彪之又上疏執據,時人比之張釋之。時當南郊,簡文帝為撫軍,執政,訪彪之應有赦不。答曰:「中興以來,郊祀往往有赦,愚意嘗謂非宜。何者?黎庶不達其意,將謂效祀必赦,至此時,凶愚之輩復生心於僥倖矣。」遂從之。



 轉吏部尚書。簡文有命用秣陵令曲安遠補句容令,殿中侍御史奚郎補湘東郡。彪之執不從,曰:「秣陵令三品縣耳,殿下昔用安遠,談者紛然。句容近幾,三品佳邑,敢可處卜術之人無才用者邪!湘東
 雖復遠小,所用未有朗比,談者謂頗兼卜術得進。殿下若超用寒悴,當充人才可拔。朗等凡器,實未足充此選。」



 太尉桓溫欲北伐,屢詔不許。溫輒下武昌,人情震懼。或勸殷浩引身告退,彪之言於簡文曰:「此非保社稷為殿下計,皆自為計耳。若殷浩去職,人情崩駭,天子獨坐。既爾,當有任其責者,非殿下而誰!」又謂浩曰:「彼抗表問罪,卿為其首。事任如此,猜釁已構,欲作匹夫,豈有全地邪?且當靜以待之。令相王與手書,示以款誠,陳以成敗,當必旋旆。若不順命,即遣中詔。如復不奉,乃當以正義相裁。,無故匆匆,先自猖蹶。」浩曰:「決大事正自難,頃日來欲
 使人悶,聞卿此謀,意始得了。」溫亦奉帝旨,果不進。



 時眾官漸多,而遷徙每速,彪之上議曰:



 為政之道,以得賢為急,非謂雍容廊廟,標的而已,固將蒞任贊時,職思其憂也。得賢之道,在於蒞任;蒞任之道,在於能久;久於其道,天下化成。是以三載考績,三考黜陟,不收一切之功,不採速成之譽。故勛格辰極,道融四海,風流遐邈,聲冠百代。凡庸之族眾,賢能之才寡,才寡於世而官多於朝,焉得不賢鄙共貫,清濁同官!官眾則闕多,闕多則遷速,前後去來,更相代補,非為故然,理固然耳。所以職事未修,朝風未澄者也。職事之修,在於省官;朝風之澄,在於並
 職。官省則選清而得久,職並則吏簡而俗靜;選清則勝人久於其事,事久則中才猶足有成。



 今內外百官,較而計之,固應有並省者矣。六卿之任,太常望雅而職重,然其所司,義高務約。宗正所統蓋鮮,可以並太常。宿衛之重,二衛任之,其次驍騎、左軍各有所領,無兵軍校皆應罷廢。四軍皆罷,則左軍之名不宜獨立,宜改游擊以對驍騎。內官自侍中以下,舊員皆四,中興之初,二人而已。二人對直,或有不周,愚謂三人,於事則無闕也。凡餘諸官,無綜事實者,可令大官隨才位所帖而領之,若未能頓廢,自可因缺而省之。委之以職分,責之以有成,能否
 因考績而著,清濁隨黜陟而彰。雖緝熙之隆、康哉之歌未可,使庶官之選差清,蒞職之日差久,無奉祿之虛費,簡吏寺之煩役矣。



 永和末,多疾疫。舊制,朝臣家有時疾,染易三人以上者,身雖無病,百日不得入宮。至是,百官多列家疾,不入。彪之又言:「疾疫之年,家無不染。若以之不復人宮,則直侍頓闕,王者宮省空矣。」朝廷從之。



 既而長安人雷弱兒、梁安等詐云殺苻健、苻眉,請兵應接。時殷浩鎮壽陽,便進據洛,營復山陵。屬彪之疾歸,上簡文帝箋,陳弱兒等容有詐偽,浩未應輕進。尋而弱兒果詐,姚襄反叛,浩大敗,退守譙城。簡文笑謂彪之
 曰:「果如君言。自頃以來,君謀無遺策,張、陳何以過之!」



 轉領軍將軍,遷尚書僕射,以疾病,不拜。徙太常,領崇德衛尉。時或謂簡文曰:「武陵第中大修器杖,將謀非常也。」簡文以彪之。彪之曰:「武陵王志意盡於馳騁田獵耳。願深靜之,以懷異同者。」或復以此為言,簡文甚悅。



 復轉尚書僕射。時豫州刺史謝奕卒,簡文遽使彪之舉可以代奕者。對曰:「當今時賢,備簡高監。」簡文曰:「人有舉桓雲者,君謂如何?」彪之曰:「雲不必非才,然溫居上流,割天下之半。其弟復處西籓,兵權盡出一門,亦非深根固蒂之宜也。人才非可豫量,但當令不與殿下作異者耳。」簡文
 頷曰:「君言是也。」



 後以彪之為鎮軍將軍、會稽內史,加散騎常侍。居郡八年,豪右斂跡,亡戶歸者三萬餘口。桓溫下鎮姑孰,威勢震主,四方修敬,皆遣上佐綱紀。彪之獨曰:「大司馬誠為富貴,朝廷既有宰相,動靜之宜自當諮稟。修敬若遣綱紀,致貢天子復何以過之!」竟不遣。溫以山陰縣折布米不時畢,郡不彈糾,上免彪之。彪之去郡,郡見罪謫未上州臺者,皆原散之。溫復以為罪,乃檻收下吏。會赦,免,左降謫為尚書。



 頃之,復僕為射。是時溫將廢海西公,百僚震慄,溫亦色動,莫知所為。彪之既知溫不臣迹已著,理不可奪。乃謂溫曰:「公阿衡皇家,便當倚傍
 先代耳。」命取《霍光傳》。禮度儀制,定於須臾,曾無懼容。溫歎曰:「作元凱不當如是邪!」時廢立之儀既絕於曠代,朝臣莫有識其故典者。彪之神彩毅然,朝服當階,文武儀準莫不取定,朝廷以此服之。溫又廢武陵王遵,以事示彪之。彪之曰:「武陵親尊,未有顯罪,不可以猜嫌之間,便相廢徙。公建立聖明,遐邇歸心,當崇獎王室,伊周同美。此大事,宜更深詳。」溫曰:「此已成事,卿勿復言。」



 及簡文崩,群臣疑惑,未敢立嗣。或云,宜當須大司馬處分。彪之正色曰:「君崩,太子代立,大司馬何容得異!若先面諮,必反為所責矣。」於是朝議乃定。及孝武帝即位,太皇太后令
 以帝沖幼,加在諒闇,令溫依周公居攝故事。事已施行,彪之曰:「此異常大事,大司馬必當固讓,使萬機停滯,稽廢山陵,未敢奉令。謹具封還內,請停。」事遂不行。



 溫遇疾,諷朝廷求九錫,袁宏為文,以示彪之。彪之視訖,歎其文辭之美,謂宏曰:「卿固大才,安可以此示人!」時謝安見其文,又頻使宏改之,宏遂逡巡其事。既屢引日,乃謀於彪之。彪之曰:「聞彼病日增,亦當不復支久,自可更小遲迴。」宏從之,溫亦尋薨。



 時桓沖及安夾輔朝政,安以新喪元輔,主上未能親覽萬機,太皇太后宜臨朝,彪之曰:「先代前朝,主在襁抱,母子一體,故可臨朝。太后亦不能決政
 事,終是顧問僕與君諸人耳。今上年出十歲,垂婚冠,反令從嫂臨朝,示人君幼弱,豈是翼戴贊揚立德之謂乎!二君必行此事,豈僕所制,所惜者大體耳。」時安不欲委任桓沖,故使太后臨朝決政,獻替專在乎自己。彪之不達安旨,故以為言。安竟不從。



 尋遷尚書令,與安共掌朝政。安每曰:「朝之大事,眾不能決者,諮王公無不得判。」以年老,上疏乞骸骨,詔不許。轉拜護軍將軍,加散騎常侍。安欲更營宮室,彪之曰:「中興初,即位東府,殊為儉陋,元明二帝亦不改制。蘇峻之亂,成帝止蘭臺都坐,殆不蔽寒暑,是以更營修築。方之漢魏,誠為儉狹,復不至陋,殆合
 豐約之中,今自可隨宜增益修補而已。彊寇未殄,正是休兵養士之時,何可大興功力,勞擾百姓邪!」安曰:「宮室不壯,後世謂人無能。」彪之曰:」任天下事,當保國寧家,朝政惟允,豈以修屋宇為能邪!」安無以奪之。」故終彪之之世,不改營焉。



 加光祿大夫、儀同三司,未拜。疾篤,帝遣黃門侍郎問所苦,賜錢三十萬以營醫藥。太元二年卒,年七十三。即以光祿為贈,謚曰簡。二子:越之,撫軍參軍;臨之,東陽太守。



 棱字文子,彬季父國子祭酒琛之子也。少歷清官。渡江,為元帝丞相從事中郎。從兄導以棱有政事,宜守大郡,
 乃出為豫章太守,加廣武將軍。棱知從兄敦驕傲自負,有罔上心,日夕諫諍,以為宜自抑損,推崇盟主,且群從一門,並相與服事,應務相崇高,以隆勛業。每言苦切。敦不能容,潛使人害之。



 弟侃,亦知名,少歷顯職,位至吳國內史。



 虞潭,字思奧,會稽餘姚人,吳騎都尉翻之孫也。父忠,仕至宜都太守。吳之亡也,堅壁不降,遂死之。潭清貞有檢操,州辟從事、主簿,舉秀才,大司馬、齊王冏請為祭酒,除祁鄉令,徙醴陵令。值張昌作亂,郡縣多從之,潭獨起兵
 斬昌別率鄧穆等。襄陽太守華恢上潭領建平太守,以疾固辭。遂周旋征討,以軍功賜爵都亭侯。陳敏反,潭東下討敏弟贊於江州。廣州刺史王矩上潭領廬陵太守。綏撫荒餘,咸得其所。又與諸軍共平陳恢,仍轉南康太守,進爵東鄉侯。尋被元帝檄,使討江州刺史華軼。潭至廬陵,會軼已平,而湘川賊杜弢猶盛。江州刺史衛展上潭并領安成太守。時甘卓屯宜陽,為杜弢所逼。潭進軍救卓,卓上潭領長沙太守,固辭不就。王敦版潭為湘東太守,復以疾辭。弢平後,元帝召補丞相軍諮祭酒,轉瑯邪國中尉。



 帝為晉王,除屯騎校尉,徙右衛將軍,遷宗正卿,
 以疾告歸。會王含、沈充等攻逼京都,潭遂於本縣招合宗人,及郡中大姓,共起義軍,眾以萬數,自假明威將軍。乃進赴國難,至上虞。明帝手詔潭為冠軍將軍,領會稽內史。潭即受命,義眾雲集。時有野鷹飛集屋梁,眾咸懼。潭曰:「起大義,而剛鷙之鳥來集,破賊必矣。」遣長史孔坦領前鋒過浙江,追躡充。潭次於西陵,為坦後繼。會充已擒,罷兵,徵拜尚書,尋補右衛將軍,加散騎常侍。



 成帝即位,出為吳興太守,秩中二千石,加輔國將軍。以討充功,進爵零縣侯。蘇峻反,加潭督三吳、晉陵、宣城、義興五郡軍事。會王師敗績,大駕逼遷,潭勢弱,不能獨振,乃固
 守以俟四方之舉。會陶侃等下,潭與郗鑒、王舒協同義舉。侃等假潭節、監揚州浙江西軍事。潭率眾與諸軍并勢,東西猗角。遣督護沈伊距管商於吳縣,為商所敗,潭自貶還節。



 尋而峻平,潭以母老,輒去官還餘姚。詔轉鎮軍將軍、吳國內史。復徙會稽內史,未發,還復吳郡。以前後功,進爵武昌縣侯,邑一千六百戶。是時軍荒之後,百姓饑饉,死亡塗地,潭乃表出倉米振救之。又修滬瀆壘,以防海抄,百轉賴之。



 咸康中,進衛將軍。潭貌雖和弱,而內堅明,有膽決,雖屢統軍旅,而鮮有傾敗。以毋憂去職。服闕,以侍中、衛將軍征。既至,更拜光祿大人、開府儀
 同三司,給親兵三百人,侍中如故。年七十九,卒於位。追贈左光祿大夫,開府、侍中如故,謚曰孝烈。子仡嗣,官至右將軍司馬。仡卒,子嘯父嗣。


嘯父少歷顯位,後至侍中,為孝武帝所親愛,嘗侍飲宴,帝從容問曰:「卿在門下,初不聞有所獻替邪?」嘯父家近海,謂帝有所求,對曰:「天時尚溫,
 \gezhu{
  制魚}
 魚蝦鮓未可致,尋當有所上獻。」帝大笑。因飲大醉,出,拜不能起,帝顧曰:「扶虞侍中。」嘯父曰:「臣位未及扶,醉不及亂,非分之賜,所不敢當。」帝甚悅。隆安初,為吳國內史。徵補尚書,未發,而王廞舉兵,版嘯父行吳興太守。嘯父即入吳興應廞。廞敗,有
 司奏嘯父與廞同謀,罪應斬。詔以祖潭舊勳,聽以疾贖為庶人。四年,復拜尚書。桓玄用事,以為太尉左司馬。尋遷護軍將軍,出為會稽內史。義熙初,去職,卒於家。



 斐字思行,潭之兄子也。雖機乾不及於潭,然而素行過之。與譙國桓彞俱為吏部郎,情好甚篤。彞遣溫拜斐,斐使子谷拜彞。歷吳興太守、金紫光祿大夫。王導嘗謂斐曰:「孔愉有公才而無公望,丁潭有公望而無公才,兼之者,其在卿乎!」官未達而喪,時人惜之。子谷,位至吳國內史。



 顧眾,字長始,吳郡吳人,驃騎將軍榮之族弟也。父秘,交州刺史,有文武才幹。眾出後伯父,早終,事伯母以孝聞。光祿朱誕器之。州辟主簿,舉秀才,除餘杭、秣陵令,並不行。元帝為鎮東將軍。命為參軍。以討華軼功,封東鄉侯,辟丞相掾。秘卒,州人立眾兄壽為刺史,為州人所害,眾往交州迎喪,值杜弢之亂,崎嶇六年乃還。祕曾蒞吳興,吳興義故以眾經離寇難,共遺錢二百萬,一無所受。



 及帝踐阼,徵拜駙馬都尉、奉朝請,轉尚書郎。大將軍王敦請為從事中郎,上補南康太守。會詔除鄱陽太守,加廣武將軍。眾徑之鄱陽,不過敦,敦甚怪焉。及敦構逆,
 令眾出軍,眾遲回不發。敦大怒,以軍期召眾還,詰之,聲色甚厲。眾不為動容,敦意漸釋。時敦又怒宣城內史陸喈,眾又辨明之。敦長史陸玩在坐,代眾危懼,出謂眾曰:「卿真所謂剛亦不吐,柔亦不茹,雖仲山甫何以加之!」敦事捷,欲以眾為吳興內史。眾固辭,舉吏部郎桓彝,彞亦讓眾,事並不行。敦鎮姑孰,復以眾為從事中郎。敦平,除太子中庶子,為義興太守,加揚威將軍。



 蘇峻反,王師敗績,眾還吳,潛圖義舉。時吳國內史庾冰奔于會稽,峻以蔡謨代之。前陵江將軍張悊為峻收兵於吳,眾遣人喻悊,悊從之。眾乃遣郎中徐機告謨曰:「眾已潛合家兵,待
 時而奮,又與張悊剋期效節。」謨乃檄眾為本國督護,揚威將軍仍舊,眾從弟護軍將軍颺為威遠將軍、前鋒督護。吳中人士同時響應。



 峻遣將弘徽領甲卒五百,鼓行而前。眾與颺、悊要擊徽,戰於高莋,大破之,收其軍實。謨以冰當還任,故便去郡。眾遣颺率諸軍屯無錫。冰至,鎮御亭,恐賊從海虞道入,眾自往備之。而賊率張健、馬流攻無錫,颺等大敗,庚冰亦失守,健等遂據吳城。眾自海虞由婁縣東倉與賊別率交戰,破之,義軍又集進屯烏苞。會稽內史王舒、吳興內史虞潭並檄眾為五郡大督護,統諸義軍討健。潭遣將姚休為眾前鋒,與賊戰沒。眾
 還守紫壁。



 時賊黨方銳,義軍沮退,人咸勸眾過浙江。眾曰:「不然。今保固紫壁,可得全錢唐以南五縣。若越他境,便為寓軍,控引無所,非長計也。」臨平人范明亦謂眾曰:「此地險要,可以制寇,不可委也。」眾乃版明為參軍。明率宗黨五百人,合諸軍,凡四千人,復進討健。健退于曲阿,留錢弘為吳令。軍次路丘,即斬弘首。眾進住吳城,遣督護朱祈等九軍,與蘭陵太守李閎共守庱亭。健遣馬流、陶陽等往攻之。閎與祈等逆擊,大破之,斬首二千餘級。



 峻平,論功,眾以承檄備義,推功于謨,謨以眾唱謀,非己之力,俱表相讓,論者美之。封鄱陽縣伯,除平南軍司,不
 就。更拜丹陽尹、本國大中正,入為侍中,轉尚書。咸康末,遷領軍將軍、揚州大中正,固讓不拜。以母憂去職。



 穆帝即位,何充執政,復徵眾為領軍,不起。服闕,乃就。是時充與武陵王不平,眾會通其間,遂得和釋。充崇信佛教,眾議其糜費,每以為言。嘗與充同載,經佛寺,充要眾入門。眾不下車。充以眾州里宿望,每優遇之。以年老,上疏乞骸骨,詔書不許。遷尚書僕射。永和二年卒,時年七十三。追贈特進、光祿大夫,謚曰靖。長子昌嗣,為建康令。第三子會,中軍諮議參軍。時稱美士。



 張闓,
 字敬緒,丹陽人,吳輔吳將軍昭之曾孫也。少孤,有志操。太常薛兼進之於元帝,言闓才幹貞固,當今之良器。即引為安東參軍,甚加禮遇。轉丞相從事中郎,以母憂去職。既葬,帝強起之,闓固辭疾篤。優命敦逼,遂起視事。及帝為晉王,拜給事黃門侍郎,領本郡大中正。以佐翼勛,賜爵丹陽縣侯,遷侍中。



 帝踐阼,出補晉陵內史,在郡甚有威惠。帝下詔曰:「夫二千石之任,當勉勵其德,綏齊所蒞,使寬而不縱,嚴而不苛,其於勤功督察,便國利人,抑彊扶弱,使無雜濫,真太守之任也。若聲過其實,古人所不取。功乎異端,為政之甚害,蓋所貴者本也。」闓遵
 而行之。時所部四縣並以旱失田,闓乃立曲阿新豐塘,溉田八百餘頃,每歲豐稔。葛洪為其頌。計用二十一萬一千四百二十功,以擅興造免官。後公卿並為之言曰:「張闓興陂溉田,可謂益國,而反被黜,使臣下難復為善。」帝感悟,乃下詔曰:「丹陽侯闓昔以勞役部人免官,雖從吏議,猶未掩其忠節之志也。倉廩國之大本,宜得其才,今以闓為大司農。」闓陳黜免始爾,不宜便居九列。疏奏,不許,然後就職。帝晏駕,以闓為大匠卿,營建平陵,事畢,遷尚書。蘇峻之役,闓與王導俱入宮侍衛。峻使闓持節權督東軍。王導潛與闓謀,密宣太后詔於三吳,令速起
 義軍。陶侃等至,假闓節,行征虜將軍,與振威將軍陶回共督丹陽義軍。闓到晉陵,使內史劉耽盡以一部穀,並遣吳郡度支運四部穀,以給車騎將軍郗鑒。又與吳郡內史蔡謨、前吳興內史虞潭、會稽內史王舒等招集義兵,以討峻。峻平,以尚書加散騎常侍,賜爵宜陽伯。遷廷尉,以疾解職,拜金紫光祿大夫。尋卒,時年六十四。子混嗣。闓箋表文議傳於世。


史臣曰:季孫行父稱見有禮於其君者,如孝子之養父母;無禮於其君者,如鷹鸇之逐鳥雀。是以石碏戮厚,叔向誅鮒,前史以為美譚。王敦之惡,不足矜其類。然而朱
 家容布,為大俠之首;酈寄載呂,興賣友之譏。亦所以激揚風俗,弘長名教。王彬艤船而厚其所薄,王舒沈江而薄其所厚,較之優劣,斷乎可知。思行、彪之厲風規於多僻之日,虞潭、顧眾徇貞心於危蹙之辰。龍管為出納之端,
 \gezhu{
  制魚}
 魚非獻替之術,嘯父之對,何其鄙歟!



 贊曰:處明夙令,聲頹暮年。允之騂角,無棄山川。廙稱多藝,綢繆哲後。二三其德,亦孔之醜。世儒憤發,慟顗陵敦。彪之不撓,寧浩旋溫。顧實南金,虞惟東箭。銑質無改,筠心不變,公望公才,斐為其選。



\end{pinyinscope}