\article{列傳第二十四}

\begin{pinyinscope}
陸機
 \gezhu{
  孫拯弟雲雲弟耽從父兄喜}



 陸機,字士衡,吳郡人也。祖遜,吳丞相。父抗,吳大司馬。機身長七尺,其聲如鐘。少有異才,文章冠世,伏膺儒術,非禮不動。抗卒,領父兵為牙門將。年二十而吳滅,退居舊里,閉門勤學,積有十年。以孫氏在吳,而祖父世為將相,有大勳於江表,深慨孫皓舉而棄之,乃論權所以得,皓所以亡,又欲述其祖父功業,遂作《辯亡論》二篇。其上篇
 曰:



 昔漢氏失御,姦臣竊命,禍基京畿,毒遍宇內,皇綱弛頓,王室遂卑。於是群雄蜂駭,義兵四合。吳武烈皇帝慷慨下國,電發荊南,權略紛紜,忠勇伯世,威棱則夷羿震盪,兵交則醜虜授馘,遂掃清宗祊,蒸禋皇祖。於時雲興之將帶州,猋起之師跨邑,哮闞之群風驅,熊羆之族霧合。雖兵以義動,同盟戮力,然皆苞藏禍心,阻兵怙亂,或師無謀律,喪威稔寇。忠規武節,未有如此其著者也。



 武烈既沒,長沙桓王逸才命世,弱冠秀發,招攬遺老,與之述業。神兵東驅,奮寡犯眾,攻無堅城之將,戰無交鋒之虜。誅叛柔服,而江外底定;飭法脩師,則威德翕赫。賓禮
 名賢,而張公為之雄;交御豪俊,而周瑜為之傑。彼二君子皆弘敏而多奇,雅達而聰哲,故同方者以類附,等契者以氣集,江東蓋多士矣。將北伐諸華,誅鉏干紀,旋皇輿於夷庚,反帝坐於紫闥,挾天子以令諸侯,清天步而歸舊物。戎車既次,群凶側目,大業未就,中世而殞。



 用集我大皇帝,以奇蹤襲逸軌,睿心因令圖,從政咨於故實,播憲稽乎遺風;而加之以篤敬,申之以節儉,疇諮俊茂,好謀善斷,束帛旅於丘園,旌命交乎塗巷。故豪彥尋聲而響臻,志士晞光而景騖,異人輻輳,猛士如林。於是張公為師傅;周瑜、陸公、魯肅、呂蒙之儔,入為腹心,出為股
 肱;甘寧、凌統、程普、賀齊、朱桓、朱然之徒奮其威,韓當、潘璋、黃蓋、蔣欽、周泰之屬宣其力;風雅則諸葛瑾、張承、步騭以名聲光國,政事則顧雍、潘浚、呂範、呂岱以器任幹職,奇偉則虞翻、陸績、張惇以風義舉政,奉使則趙咨、沈珩以敏達延譽,術數則吳範、趙達以禨祥協德;董襲、陳武殺身以衛主,駱統、劉基彊諫以補過。謀無遺計,舉不失策。故遂割據山川,跨制荊、吳,而與天下爭衡矣。魏氏嘗藉戰勝之威,率百萬之師,浮鄧塞之舟,下漢陰之眾,羽楫萬計,龍躍順流,銳師千旅,武步原隰,謨臣盈室,武將連衡,喟然有吞江滸之志,壹宇宙之氣。而周瑜驅我
 偏師,黜之赤壁,喪旗亂轍,僅而獲免,收迹遠遁。漢王亦憑帝王之號,帥巴、漢之人,乘危騁變,結壘千里,志報關羽之敗,圖收湘西之地。而我陸公亦挫之西陵,覆師敗績,困而後濟,絕命永安。續以濡須之寇,臨川摧銳;蓬蘢之戰,孑輪不反。由是二邦之將,喪氣挫鋒,勢財匱,而吳莞然坐乘其弊,故魏人請好,漢氏乞盟,遂躋天號,鼎峙而立。西界庸、益之郊,北裂淮、漢之涘,東苞百越之地,南括群蠻之表。於是講八代之禮,搜三王之樂,告類上帝,拱揖群后。武臣毅卒,循江而守;長棘勁鎩,望猋而奮。庶尹盡規於上,黎元展業於下,化協殊裔,風衍遐圻。乃
 俾一介行人,撫巡外域,巨象逸駿,擾於外閑,明珠瑋寶,耀於內府,珍瑰重迹而至,奇玩應響而赴;輶軒騁於南荒,衝輣息於朔野;黎庶免干戈之患,戎馬無晨服之虞,而帝業固矣。



 大皇既沒,幼主蒞朝,姦回肆虐。景皇聿興,虔修遺憲,政無大闕,守文之良主也。降及歸命之初,典刑未滅,故老猶存。大司馬陸公以文武熙朝,左丞相陸凱以謇諤盡規,而施績、范慎以威重顯,丁奉、鐘離斐以武毅稱,孟宗、丁固之徒為公卿,樓玄、賀邵之屬掌機事,元首雖病,股肱猶良。爰逮末葉,群公既喪,然後黔首有瓦解之患,皇家有土崩之釁,歷命應化而微,王師躡運
 而發,卒散於陳,眾奔于邑,城池無籓籬之固,山川無溝阜之勢,非有工輸雲梯之械,智伯灌激之害,楚子築室之圍,燕人濟西之隊,軍未浹辰而社稷夷矣。雖忠臣孤憤,烈士死節,將奚救哉!



 夫曹、劉之將非一世所選,向時之師無曩日之眾,戰守之道抑有前符,險阻之利俄然未改,而成敗貿理,古今詭趣,何哉?彼此之化殊,授任之才異也。



 其下篇曰:



 昔三方之王也,魏人據中夏,漢氏有岷、益,吳制荊、揚而掩有交、廣。曹氏雖功濟諸華,虐亦深矣,其人怨。劉翁因險以飾智,功已薄矣,其俗陋。夫吳,桓王基之以武,太祖成之以德,聰明睿達,懿度弘遠矣。其
 求賢如弗及,血阜人如稚子,接士盡盛德之容,親仁罄丹府之愛。拔呂蒙於戎行,試潘濬於係虜。推誠信士,不恤人之我欺;量能授器,不患權之我偪。執鞭鞠躬,以重陸公之威;悉委武衛,以濟周瑜之師。卑宮菲食,豐功臣之賞;披懷虛己,納謨士之算。故魯肅一面而自託,士燮蒙險而效命。高張公之德,而省游田之娛;賢諸葛之言,而割情欲之歡;感陸公之規,而除刑法之煩;奇劉基之議,而作三爵之誓;屏氣跼蹐,以伺子明之疾;分滋損甘,以育凌統之孤;登壇慷愾,歸魯子之功;削投怨言,信子瑜之節。是以忠臣競盡其謨,志士咸得肆力,洪規遠略,固
 不厭夫區區者也。故百官茍合,庶務未遑。初都建鄴,群臣請備禮秩,天子辭而弗許,曰:「天下其謂朕何!」宮室輿服,蓋慊如也。爰及中葉,天人之分既定,故百度之缺粗修,雖醲化懿綱,未齒乎上代,抑其體國經邦之具,亦足以為政矣。地方幾萬里,帶甲將百萬,其野沃,其兵練,其器利,其財豐;東負滄海,西阻險塞,長江制其區宇,峻山帶其封域,國家之利未見有弘於茲者也。借使守之以道,御之以術,敦率遺典,勤人謹政,修定策,守常險,則可以長世永年,未有危亡之患也。



 或曰:「吳、蜀脣齒之國也,夫蜀滅吳亡,理則然矣。」夫蜀,蓋籓援之與國,而非吳人
 之存亡也。其郊境之接,重山積險,陸無長轂之徑;川阨流迅,水有驚波之艱。雖有銳師百萬,啟行不過千夫;軸轤千里,前驅不過百艦。故劉氏之伐,陸公喻之長蛇,其勢然也。昔蜀之初亡,朝臣異謀,或欲積石以險其流,或欲機械以禦其變。天子總群議以諮之大司馬陸公,公以四瀆天地之所以節宣其氣,固無可遏之理,而機械則彼我所共,彼若棄長技以就所屈,即荊、楚而爭舟楫之用,是天贊我也,將謹守峽口以待擒耳。逮步闡之亂,憑寶城以延彊寇,資重幣以誘群蠻。于時大邦之眾,雲翔電發,懸旍江介,築壘遵渚,衿帶要害,以止吳人之西,
 巴、漢舟師,沿江東下。陸公偏師三萬,北據東坑,深溝高壘,按甲養威。反虜宛迹待戮,而不敢北窺生路,彊寇敗績宵遁,喪師太半。分命銳師五千,西禦水軍,東西同捷,獻俘萬計。信哉賢人之謀,豈欺我哉!自是烽燧罕驚,封域寡虞。陸公沒而潛謀兆,吳釁深而六師駭。夫太康之役,眾未盛乎曩日之師;廣州之亂,禍有愈乎向時之難,而邦家顛覆,宗廟為墟。嗚呼!「人之云亡,邦國殄瘁」,不其然歟!



 《易》曰「湯、武革命順乎天」,或曰「亂不極則治不形」,言帝王之因天時也。古人有言曰「天時不如地利」,《易》曰「王侯設險以守其國」,言為國之恃險也。又曰「地利不如人
 和」,「在德不在險」,言守險之在人也。吳之興也,參而由焉,孫卿所謂合其參者也。及其亡也,恃險而已,又孫卿所謂舍其參者也。夫四州之萌非無眾也,大江以南非乏俊也,山川之險易守也,勁利之器易用也,先政之策易修也,功不興而禍遘何哉?所以用之者失也。故先王達經國之長規,審存亡之至數,謙己以安百姓,敦惠以致人和,寬沖以誘俊乂之謀,慈和以結士庶之愛。是以其安也,則黎元與之同慶,及其危也,則兆庶與之同患。安與眾同慶,則其危不可得也;危與下同患,則其難不足血阜也。夫然,故能保其社稷而固其土宇,《麥秀》無悲殷之
 思,《黍離》無愍周之感也。



 至太康末,與弟雲俱入洛,造太常張華。華素重其名,如舊相識,曰:「伐吳之役,利獲二俊。」又嘗詣侍中王濟,濟指羊酪謂機曰:「卿吳中何以敵此?」答云:「千里蓴羹,未下鹽豉。」時人稱為名對。張華薦之諸公。後太傅楊駿辟為祭酒。會駿誅,累遷太子洗馬、著作郎。范陽盧志於眾中問機曰:「陸遜、陸抗於君近遠?」機曰:「如君於盧毓、盧廷。」志默然。既起,雲謂機曰:「殊邦遐遠,容不相悉,何至於此!」機曰:「我父祖名播四海,寧不知邪!」議者以此定二陸之優劣。



 吳王晏出鎮淮南,以機為郎中令,遷尚書中兵郎,轉殿中郎。趙王倫輔政,引為相國參
 軍。豫誅賈謐功,賜爵關中侯。倫將篡位,以為中書郎。倫之誅也,齊王冏以機職在中書,九錫文及禪詔疑機與焉,遂收機等九人付廷尉。賴成都王穎、吳王晏並救理之,得減死徙邊,遇赦而止。



 初機有駿犬,名曰黃耳,甚愛之。既而羈寓京師,久無家問,笑語犬曰:「我家絕無書信,汝能齎書取消息不?」犬搖尾作聲。機乃為書以竹筒盛之而繫其頸,犬尋路南走,遂至其家,得報還洛。其後因以為常。時中國多難,顧榮、戴若思等咸勸機還吳,機負其才望,而志匡世難,故不從。



 冏既矜功自伐,受爵不讓,機惡之,作《豪士賦》以刺焉。其序曰:



 夫立德之基有常,而
 建功之路不一。何則?修心以為量者存乎我,因物以成務者係乎彼。存乎我者,隆殺止乎其域;係乎彼者,豐約惟所遭遇。落葉俟微飆以隕,而風之力蓋寡;孟嘗遭雍門以泣,而琴之感以末。何哉?欲隕之葉無所假烈風,將墜之泣不足煩哀響也。是故茍時啟於天,理盡於人,庸夫可以濟聖賢之功,斗筲可以定烈士之業。故曰「才不半古,功已倍之」,蓋得之於時世也。歷觀今古,徼一時之功而居伊、周之位者有矣。



 夫我之自我,智士猶嬰其累;物之相物,昆蟲皆有此情。夫以自我之量而挾非常之勛,神器暉其顧眄,萬物隨其俯仰,心玩居常之安,耳飽從
 諛之說,豈識乎功在身外,任出才表者哉!且好榮惡辱,有生之所大期,忌盈害上,鬼神猶且不免,人主操其常柄,天下服其大節,故曰天可仇乎。而時有玄服荷戟,立乎廟門之下,援旗誓眾,奮於阡陌之上,況乎世主制命,自下裁物者乎!廣樹恩不足以敵怨,勤興利不足以補害,故曰代大匠斲者必傷其手。且夫政由寧氏,忠臣所以慷慨;祭則寡人,人主所不久堪。是以君奭怏怏,不悅公旦之舉;高平師師,側目博陸之勢。而成王不遣嫌吝於懷,宣帝若負芒刺於背,非其然者歟?



 嗟乎!光於四表,德莫富焉。王曰叔父,親莫暱焉。登帝天位,功莫厚焉。守節
 沒齒,忠莫至焉。而傾側顛沛,僅而自全,則伊生抱明允以嬰戮,文子懷忠敬而齒劍,固其所也。因斯以言,夫以篤聖穆親,如彼之懿,大德至忠,如此之盛,尚不能取信於人主之懷,止謗於眾多之口,過此以往,惡睹其可!安危之理,斷可識矣。又況乎饕大名以冒道家之忌,運短才而易聖哲所難者哉!身危由於勢過,而不知去勢以求安;禍積起於寵盛,而不知辭寵以招福。見百姓之謀己,則申宮警守,以崇不畜之威;懼萬方之不服,則嚴刑峻制,以賈傷心之怨。然後威窮乎震主,而怨行乎上下,眾心日陊,危機將發,而方偃仰瞪眄,謂足以夸世,笑古
 人之未工,忘己事之已拙,知曩勳之可矜,闇成敗之有會。是以事窮運盡,必有顛仆;風起塵合,而禍至常酷也。聖人忌功名之過己,惡寵祿之踰量,蓋為此也。



 夫惡欲之大端,賢愚所共有,而遊子殉高位於生前,志士思垂名於身後,受生之分,惟此而已。夫蓋世之業,名莫盛焉;率意無違,欲莫順焉。借使伊人頗覽天道,知盡不可益,盈難久持,超然自引,高揖而退,則巍巍之盛,仰邈前賢,洋洋之風,俯觀來籍,而大欲不止於身,至樂無愆乎舊,節彌效而德彌廣,身逾逸而名逾劭。此之不為,而彼之必昧,然後河海之迹堙為窮流,一匱之釁積成山嶽,名
 編凶頑之條,身厭荼毒之痛,豈不謬哉!故聊為賦焉,庶使百世少有悟云。



 冏不之悟,而竟以敗。



 機又以聖王經國,義在封建,因採其遠指,著《五等論》曰:



 夫體國經野,先王所慎,創制垂基,思隆後葉。然而經略不同,長世異術。五等之制,始於黃、唐,郡縣之治,創於秦、漢,得失成敗,備在典謨,是以其詳可得而言。



 夫王者知帝業至重,天下至廣。廣不可以偏制,重不可以獨任。任重必於借力,制廣終乎因人。故設官分職,所以輕其任也;並建伍長,所以弘其制也。於是乎立其封疆之典,裁其親疏之宜,使萬國相維,以成盤石之固;宗庶雜居,而定維城之業。又
 有以見綏世之長御,識人情之大方,知其為人不如厚己,利物不如圖身;安上在於悅下,為己存乎利人。故《易》曰「悅以使人,人忘其勞」,孫卿曰「不利而利之,不如利而後利之利也」。是以分天下以厚樂,則己得與之同憂;饗天下以豐利,而己得與之共害。利博而恩篤,樂遠則憂深,故諸侯享食土之實,萬國受傳世之祚。夫然,則南面之君各務其政,九服之內知有定主,上之子愛於是乎生,下之禮信於是乎結,世平足以敦風,道衰足以禦暴。故彊毅之國不能擅一時之勢,雄俊之人無所寄霸王之志。然後國安由萬邦之思化,主尊賴群后之圖身,譬
 猶眾目營方,則天網自昶;四體辭難,而心膂獲乂。蓋三代所以直道,四王所以垂業也。



 夫盛衰隆弊,理所固有,教之廢興,繫乎其人,原法期於必諒,明道有時而闇。故世及之制弊於彊禦,厚下之典漏於末折,侵弱之釁遘自三委,陵夷之禍終乎七雄。昔成湯親照夏后之鑒,公旦目涉商人之戒,文質相濟,損益有物。然五等之禮,不革於時,封畛之制,有隆爾者,豈玩二王之禍而闇經世之算乎?固知百世非可懸御,善制不能無弊,而侵弱之辱愈於殄祀,土崩之困痛於陵夷也。是以經始獲其多福,慮終取其少禍,非謂侯伯無可亂之符,郡縣非興化
 之具。故國憂賴其釋位,主弱憑於翼戴。及承微積弊,王室遂卑,猶保名位,祚垂後嗣,皇統幽而不輟,神器否而必存者,豈非事勢使之然歟!



 降及亡秦,棄道任術,懲周之失,自矜其得。尋斧始於所庇,制國昧於弱下,國慶獨饗其利,主憂莫與共害。雖速亡趨亂,不必一道,顛沛之釁,實由孤立。是蓋思五等之小怨,亡萬國之大德,知陵夷之可患,闇土崩之為痛也。周之不競,有自來矣。國乏令主,十有餘世。然片言勤王,諸侯必應,一朝振矜,遠國先叛,故彊晉收其請隧之圖,暴楚頓其觀鼎之志,豈劉、項之能窺關,勝、廣之敢號澤哉!借使秦人因循其制,雖
 則無道,有與共亡,覆滅之禍,豈在曩日!



 漢矯秦枉,大啟王侯,境土踰溢,不遵舊典,故賈生憂其危,晁錯痛其亂。是以諸侯岨其國家之富,憑其士庶之力,勢足者反疾,土狹者逆遲,六臣犯其弱綱,七子衝其漏網,皇祖夷於黔徒,西京病於東帝。是蓋過正之災,而非建侯之累也。然呂氏之難,朝士外顧;宋昌策漢,必稱諸侯。逮至中葉,忌其失節,割削宗子,有名無實,天下曠然,復襲亡秦之軌矣。是以五侯作威,不忌萬國;新都襲漢,易於拾遺也。光武中興,纂隆皇統,而由遵覆車之遺轍,養喪家之宿疾,僅及數世,姦宄棄斥。卒有彊臣專朝,則天下風靡,一
 夫從衡,而城池自夷,豈不危哉!



 在周之衰,難興王室,放命者七臣,乾位者三子,嗣王委其九鼎,凶族據其天邑,鉦鼙震於閫宇,鋒鏑流於絳闕,然禍止畿甸,害不覃及,天下晏然,以安待危。是以宣王興於共和,襄惠振於晉、鄭。豈若二漢階闥暫擾,而四海已沸,嬖臣朝入,九服夕亂哉!



 遠惟王莽篡逆之事,近覽董卓擅權之際,億兆悼心,愚智同痛。然周以之存,漢以之亡,夫何故哉?豈世乏曩時之臣,士無匡合之志歟?蓋遠績屈於時異,雄心挫於卑勢耳。故烈士扼腕,終委寇仇之手;中人變節,以助虐國之桀。雖復時有鳩合同志以謀王室,然上非奧主,
 下皆市人,師旅無先定之班,君臣無相保之志,是以義兵雲合,無救劫殺之禍,眾望未改,而已見大漢之滅矣。



 或以「諸侯世位,不必常全,昏主暴君,有時比迹,故五等所以多亂。今之牧守,皆官方庸能,雖或失之,其得固多,故郡縣易以為政」。夫德之休明,黜陟日用,長率連屬,咸述其職,而淫昏之郡無所容過,何則其不治哉!故先代有以興矣。茍或衰陵,百度自悖,鬻官之吏以貨準財,則貪殘之萌皆群后也,安在其不亂哉!故後王有以之廢矣。且要而言之,五等之君,為己思政;郡縣之長,為吏圖物。何以征之?蓋企及進取,仕子之常志;修己安人,良士
 所希及。夫進取之情銳,而安人之譽遲,是故侵百姓以利己者,在位所不憚;損實事以養名者,官長所夙慕也。君無卒歲之圖,臣挾一時之志。五等則不然。知國為己土,眾皆我民;民安,己受其利;國傷,家嬰其病。故前人欲以垂後,後嗣思其堂構,為上無茍且之心,群下知膠固之義。使其並賢居政,則功有厚薄;兩愚處亂,則過有深淺。然則八代之制,幾可以一理貫;秦、漢之典,殆可以一言蔽也。



 時成都王穎推功不居,勞謙下士。機既感全濟之恩,又見朝廷屢有變難,謂穎必能康隆晉室,遂委身焉。穎以機參大將軍軍事,表為平原內史。太安初,穎與
 河間王顒起兵討長沙王乂,假機後將軍、河北大都督,督北中郎將王粹、冠軍牽秀等諸軍二十餘萬人。機以三世為將,道家所忌,又羈旅入宦,屯居群士之右,而王粹、牽秀等皆有怨心,固辭都督。穎不許。機鄉人孫惠亦勸機讓都督於粹,機曰:「將謂吾為首鼠避賊,適所以速禍也。」遂行。穎謂機曰:「若功成事定,當爵為郡公,位以台司,將軍勉之矣!」機曰:「昔齊桓任夷吾以建九合之功,燕惠疑樂毅以失垂成之業,今日之事,在公不在機也。」穎左長史盧志心害機寵,言於穎曰:「陸機自比管、樂,擬君闇主,自古命將遣師,未有臣陵其君而可以濟事者也。」
 穎默然。機始臨戎,而牙旗折,意甚惡之。列軍自朝歌至于河橋,鼓聲聞數百里,漢、魏以來,出師之盛,未嘗有也。長沙王乂奉天子與機戰於鹿苑,機軍大敗,赴七里澗而死者如積焉,水為之不流,將軍賈棱皆死之。



 初,宦人孟玖弟超並為穎所嬖寵。超領萬人為小都督,未戰,縱兵大掠。機錄其主者。超將鐵騎百餘人,直入機麾下奪之,顧謂機曰:「貉奴能作督不!」機司馬孫拯勸機殺之,機不能用。超宣言於眾曰:「陸機將反。」又還書與玖言機持兩端,軍不速決。及戰,超不受機節度,輕兵獨進而沒。玖疑機殺之,遂譖機於穎,言其有異志。將軍王闡、郝昌、公
 師籓等皆玖所用,與牽秀等共證之。穎大怒,使秀密收機。其夕,機夢黑幰繞車,手決不開,天明而秀兵至。機釋戎服,著白帢,與秀相見,神色自若,謂秀曰:「自吳朝傾覆,吾兄弟宗族蒙國重恩,入侍帷幄,出剖符竹。成都命吾以重任,辭不獲已。今日受誅,豈非命也!」因與穎箋,詞甚悽惻。既而歎曰:「華亭鶴唳,豈可復聞乎!」遂遇害於軍中,時年四十三。二子蔚、夏亦同被害。機既死非其罪,士卒痛之,莫不流涕。是日昏霧晝合,大風折木,平地尺雪,議者以為陸氏之冤。



 機天才秀逸,辭藻宏麗,張華嘗謂之曰:「人之為文,常恨才少,而子更患其多。」弟雲嘗與書曰:「
 君苗見兄文,輒欲燒其筆硯。」後葛洪著書,稱「機文猶玄圃之積玉,無非夜光焉,五河之吐流,泉源如一焉。其弘麗妍贍,英銳漂逸,亦一代之絕乎!」其為人所推服如此。然好游權門,與賈謐親善,以進趣獲譏。所著文章凡三百餘篇,並行於世。



 孫拯者,字顯世,吳都富春人也。能屬文,仕吳為黃門郎。孫皓世,侍臣多得罪,惟拯與顧榮以智全。吳平後,為涿令,有稱績。機既為孟玖等所誣收拯考掠,兩踝骨見,終不變辭。門生費慈、宰意二人詣獄明拯,拯譬遣之曰:「吾義不可誣枉知故,卿何宜復爾?」二人曰:「僕亦安得負君!」
 拯遂死獄中,而慈、意亦死。



 雲字士龍,六歲能屬文,性清正,有才理。少與兄機齊名,雖文章不及機,而持論過之,號曰「二陸」。幼時吳尚書廣陵閔鴻見而奇之,曰:「此兒若非龍駒,當是鳳雛。」後舉雲賢良,時年十六。吳平,入洛。機初詣張華,華問雲何在。機曰:「雲有笑疾,未敢自見。」俄而雲至。華為人多姿制,又好帛繩纏鬚。雲見而大笑,不能自已。先是,嘗著縗絰上船,於水中顧見其影,因大笑落水,人救獲免。雲與荀隱素未相識,嘗會華坐,華曰:「今日相遇,可勿為常談。」雲因抗手曰:「雲間陸士龍。」隱曰:「日下荀鳴鶴。」鳴鶴,隱字也。雲又
 曰:「既開青雲睹白雉,何不張爾弓,挾爾矢?」隱曰:「本謂是雲龍騤騤,乃是山鹿野麋。獸微弩強,是以發遲。」華撫手大笑。刺史周浚召為從事,謂人曰:「陸士龍當今之顏子也。」



 俄以公府掾為太子舍人,出補浚儀令。縣居都會之要,名為難理。雲到官肅然,下不能欺,市無二價。人有見殺者,主名不立,雲錄其妻,而無所問。十許日遣出,密令人隨後,謂曰:「其去不出十里,當有男子候之與語,便縛來。」既而果然。問之具服,云:「與此妻通,共殺其夫,聞妻得出,欲與語,憚近縣,故遠相要候。」於是一縣稱其神明。郡守害其能,屢譴責之,雲乃去官。百姓追思之,圖畫形象,
 配食縣社。



 尋拜吳王晏郎中令。晏於西園大營第室,雲上書曰:「臣竊見世祖武皇帝臨朝拱默,訓世以儉,即位二十有六載,宮室臺榭無所新營,屢發明詔,厚戒豐奢。國家纂承,務在遵奉,而世俗陵遲,家競盈溢,漸漬波蕩,遂已成風。雖嚴詔屢宣,而侈俗滋廣。每觀詔書,眾庶歎息。清河王昔起墓宅時,手詔追述先帝節儉之教,懇切之旨,形于四海。清河王毀壞成宅以奉詔命,海內聽望,咸用欣然。臣愚以先帝遺教日以陵替,今與國家協崇大化、追闡前蹤者,實在殿下。先敦素朴而後可以訓正四方;凡在崇麗,一宜節之以制,然後上厭帝心,下允時
 望。臣以凡才,特蒙拔擢,亦思竭忠效節以報所受之施,是以不慮犯迕,敢陳所懷。如愚臣言有可采,乞垂三省。」



 時晏信任部將,使覆察諸官錢帛,雲又陳曰:「伏見令書,以部曲將李咸、馮南、司馬吳定、給使徐泰等覆校諸官市買錢帛簿。臣愚以聖德龍興,光有大國,選眾官材,庶工肄業。中尉該、大農誕皆清廉淑慎,恪居所司,其下眾官,悉州閭一介,疏闇之咎,雖可日聞,至於處義用情,庶無大戾。今咸、南軍旅小人,定、泰士卒廝賤,非有清慎素著,忠公足稱。大臣所關,猶謂未詳,咸等督察,然後得信,既非開國勿用之義,又傷殿下推誠曠蕩之量。雖使咸
 等能盡節益國,而功利百倍,至於光輔國美,猶未若開懷信士之無失。況所益不過姑息之利,而使小人用事,大道陵替,此臣所以慷慨也。臣備位大臣,職在獻可,茍有管見,敢不盡規。愚以宜發明令,罷此等覆察,眾事一付治書,則大信臨下,人思盡節矣。」



 雲愛才好士,多所貢達。移書太常府薦同郡張贍曰:「蓋聞在昔聖王,承天御世,殷薦明德,思和人神,莫不崇典謨以教思,興禮學以陶遠。是以帝堯昭煥而道協人天,西伯質文而周隆二代。大晉建皇,崇配天地,區夏既混,禮樂將庸。君侯應歷運之會,贊天人之期,博延俊茂,熙隆載典。伏見衛將
 軍舍人同郡張贍,茂德清粹,器思深通。初慕聖門,棲心重仞,啟塗及階,遂升樞奧。抽靈匱於祕宮,披金滕於玄夏,思樂百氏,博採其珍;辭邁翰林,言敷其藻。探微集逸,思心洞神;論道屬書,篇章光覿。含奇宰府,婆娑公門。棲靜隱寶,淪虛藏器;褧裳襲錦,緇衣被玉。曾泉改路,懸車將邁,考槃下位,歲聿屢遷。搢紳之士,具懷愾恨。方今太清闢宇,四門啟籥,玄綱括地,天網廣羅;慶雲興以招龍,和風起而儀鳳,誠巖穴耀穎之秋,河津託乘之日也。而贍沈淪下位,群望悼心。若得端委太學,錯綜先典;垂纓玉階,論道紫宮,誠帝室之瑰寶,清廟之偉器。廣樂九奏,
 必登昊天之庭;《韶》《夏》六變,必饗上帝之祀矣。」



 入為尚書郎、侍御史、太子中舍人、中書侍郎。成都王穎表為清河內史。穎將討齊王冏,以雲為前鋒都督。會冏誅,轉大將軍右司馬。穎晚節政衰,雲屢以正言忤旨。孟玖欲用其父為邯鄲令,左長史盧志等並阿意從之,而雲固執不許,曰:「此縣皆公府掾資,豈有黃門父居之邪!」玖深忿怨。張昌為亂,穎上雲為使持節、大都督、前鋒將軍以討昌。會伐長沙王,乃止。



 機之敗也,并收雲。穎官屬江統、蔡克、棗嵩等上疏曰:「統等聞人主聖明,臣下盡規,茍有所懷,不敢不獻。昨聞教以陸機後失軍期,師徒敗績,以法加
 刑,莫不謂當。誠足以肅齊三軍,威示遠近,所謂一人受戮,天下知誡者也。且聞重教,以機圖為反逆,應加族誅,未知本末者,莫不疑惑。夫爵人於朝,與眾共之;刑人於市,與眾棄之。惟刑之恤,古人所慎。今明公興舉義兵,以除國難,四海同心,雲合響應,罪人之命,懸於漏刻,泰平之期,不旦則夕矣。機兄弟並蒙拔擢,俱受重任,不當背罔極之恩,而向垂亡之寇;去泰山之安,而赴累卵之危也。直以機計慮淺近,不能董攝群帥,致果殺敵,進退之間,事有疑似,故令聖鑒未察其實耳。刑誅事大,言機有反逆之徵,宜令王粹、牽秀檢校其事。令事驗顯然,暴之
 萬姓,然後加雲等之誅,未足為晚。今此舉措,實為太重,得則足令天下情服,失則必使四方心離,不可不令審諦,不可不令詳慎。統等區區,非為陸雲請一身之命,實慮此舉有得失之機,敢竭愚戇,以備誹謗。」穎不納。統等重請,穎遲迴者三日。盧志又曰:「昔趙王殺中護軍趙浚,赦其子驤,驤詣明公而擊趙,即前事也。」蔡克入至穎前,叩頭流血,曰:「雲為孟玖所怨,遠近莫不聞。今果見殺,罪無彰驗,將令群心疑惑,竊為明公惜之。」僚屬隨克入者數十人,流涕固請,穎惻然有宥雲色。孟玖扶穎入,催令殺雲。時年四十二。有二女,無男。門生故吏迎喪葬清河,
 修墓立碑,四時祠祭。所著文章三百四十九篇,又撰《新書》十篇,並行於世。



 初,雲嘗行,逗宿故人家,夜暗迷路,莫知所從。忽望草中有火光,於是趣之。至一家,便寄宿,見一年少,美風姿,共談老子,辭致深遠。向曉辭去,行十許里,至故人家,云此數十里中無人居,雲意始悟。卻尋昨宿處,乃王弼冢。雲本無玄學,自此談老殊進。



 雲弟耽為平東祭酒,亦有清譽,與雲同遇害。大將軍參軍孫惠與淮南內史朱誕書曰:「不意三陸相攜闇朝,一旦湮滅,道業淪喪,痛酷之深,荼毒難言。國喪俊望,悲豈一人!」其為州里所痛悼如此。後東海王越討穎,移檄天下,亦以機、
 雲兄弟枉害罪狀穎云。



 喜字恭仲。父瑁,吳吏部尚書。喜仕吳,累遷吏部尚書。少有聲名,好學有才思。嘗為自敘,其略曰:「劉向省《新語》而作《新序》,桓譚詠《新序》而作《新論》。餘不自量,感子雲之《法言》而作《言道》,睹賈子之美才而作《訪論》,觀子政《洪範》而作《古今歷》,鑒蔣子通《萬機》而作《審機》,讀《幽通》、《思玄》、《四愁》而作《娛賓》、《九思》,真所謂忍愧者也。」其書近百篇。



 吳平,又作《西州清論》傳於世,借稱諸葛孔明以行其書也。有《較論格品篇》曰:「或問予,薛瑩最是國士之第一者乎?答曰:『以理推之,在乎四五之間,問者愕然請問。答曰:『夫孫皓
 無道,肆其暴虐,若龍蛇其身,沈默其體,潛而勿用,趣不可測,此第一人也。避尊居卑,祿代耕養,玄靜守約,沖退澹然,此第二人也。侃然體國思治,心不辭貴,以方見憚,執政不懼,此第三人也。斟酌時宜,在亂猶顯,意不忘忠,時獻微益,此第四人也。溫恭修慎,不為諂首,無所云補,從容保寵,此第五人也。過此已往,不足復數。故第二已上,多淪沒而遠悔吝,第三已下,有聲位而近咎累。是以深識君子,晦其明而履柔順也。』問者曰:『始聞高論,終年啟寤矣。』」



 太康中,下詔曰:「偽尚書陸喜等十五人,南士歸稱,並以貞潔不容皓朝,或忠而獲罪,或退身修志,放在
 草野。主者可皆隨本位就下拜除,敕所在以禮發遣,須到隨才授用。」乃以喜為散騎常侍,尋卒。子育,為尚書郎、弋陽太守。



 制曰:古人云:「雖楚有才,晉實用之。」觀夫陸機、陸雲,實荊、衡之杞梓,挺珪璋於秀實,馳英華於早年,風鑒澄爽,神情俊邁。文藻宏麗,獨步當時;言論慷慨,冠乎終古。高詞迥映,如朗月之懸光;疊意迴舒,若重巖之積秀。千條析理,則電坼霜開;一緒連文,則珠流璧合。其詞深而雅,其義博而顯,故足遠超枚、馬,高躡王、劉,百代文宗,一人而已。然其祖考重光,羽楫吳運,文武奕葉,將相連華。而機
 以廊廟蘊才,瑚璉標器,宜其承俊乂之慶,奉佐時之業,申能展用,保譽流功。屬吳祚傾基,金陵畢氣,君移國滅,家喪臣遷。矯翮南辭,翻棲火樹;飛鱗北逝,卒委湯池。遂使穴碎雙龍,巢傾兩鳳。激浪之心未騁,遽骨修鱗;陵雲之意將騰,先灰勁翮。望其翔躍,焉可得哉!夫賢之立身,以功名為本;士之居世,以富貴為先。然則榮利人之所貪,禍辱人之所惡,故居安保名,則君子處焉;冒危履貴,則哲士去焉。是知蘭植中塗,必無經時之翠;桂生幽壑,終保彌年之丹。非蘭怨而桂親,豈塗害而壑利?而生滅有殊者,隱顯之勢異也。故曰,衒美非所,罕有常安;韜奇
 擇居,故能全性。觀機、雲之行己也,智不逮言矣。睹其文章之誡,何知易而行難?自以智足安時,才堪佐命,庶保名位,無忝前基。不知世屬未通,運鐘方否,進不能闢昏匡亂,退不能屏跡全身,而奮力危邦,竭心庸主,忠抱實而不諒,謗緣虛而見疑,生在己而難長,死因人而易促。上蔡之犬,不誡於前,華亭之鶴,方悔於後。卒令覆宗絕祀,良可悲夫!然則三世為將,釁鐘來葉;誅降不祥,殃及後昆。是知西陵結其兇端,河橋收其禍末,其天意也,豈人事乎!



\end{pinyinscope}