\article{帝紀第八}

\begin{pinyinscope}

 穆
 帝哀帝海西公



 穆皇帝諱聃,字彭子,康帝子也。建元二年九月丙申,立為皇太子。戊戌,康帝崩。己亥,太子即皇帝位,時年二歲。大赦,尊皇后為皇太后。壬寅,皇太后臨朝攝政。冬十月乙丑,葬康皇帝於崇平陵。十一月庚辰,車騎將軍庾冰卒。



 永和元年春正月甲戌朔,皇太后設白紗帷於太極殿,
 抱帝臨軒。改元。甲申,進鎮軍將軍、武陵王晞為鎮軍大將軍、開府儀同三司,以鎮軍將軍顧眾為尚書右僕射。夏四月壬戌,詔會稽王昱錄尚書六條事。五月戊寅,大雩。尚書令、金紫光祿大夫、建安伯諸葛恢卒。六月癸亥,地震。秋七月庚午,持節、都尉江荊司梁雍益寧七州諸軍事、江州刺史、征西將軍、都亭侯庾翼卒。翼部將于瓚、戴羲等殺冠軍將軍曹據,舉兵反,安西司馬朱燾討平之。八月,豫州刺史路永叛奔于石季龍。庚辰,以輔國將軍、徐州刺史桓溫為安西將軍、持節、都督荊司雍益梁寧六州諸軍事,領護南蠻校尉、荊州刺史。石季龍將路
 永屯于壽春。九月丙申,皇太后詔曰:「今百姓勞弊,其共思詳所以振恤之宜。及歲常調非軍國要急者,並宜停之。」冬十二月,李勢將爨頠來奔。涼州牧張駿伐焉耆,降之。



 二年春正月丙寅,大赦。己卯,使持節、侍中、都督揚州諸軍事、揚州刺史、驃騎將軍、錄尚書事、都鄉侯何充卒。二月癸丑,以左光祿大夫葵謨領司徒,錄尚書六條事、撫軍大將軍、會稽王昱及謨並輔政。三月丙子,以前司徒左長史殷浩為建武將軍、揚州刺史。夏四月己酉朔,日有蝕之。五月丙戌,涼州牧張駿卒,子重華嗣。六月。石季
 龍將王擢襲武街,執張重華護軍胡宣。又使麻秋、孫伏都伐金城,太守張沖降之。重華將謝艾擊秋,敗之。秋七月,以兗州刺史褚裒為征北大將軍,開府儀同三司。冬十月,地震。十一月辛未,安西將軍桓溫帥征虜將軍周撫,輔國將軍、譙王無忌,建武將軍袁喬伐蜀,拜表輒行。十二月,枉矢自東南流于西北,其長竟天。



 三年春正月乙卯,桓溫攻成都,剋之。丁亥,李勢降,益州平。林邑范文攻陷日南,害太守夏侯覽,以尸祭天。夏四月,地震。蜀人鄧定、隗文舉兵反,桓溫又擊破之,使益州刺史周撫鎮彭模。丁巳,鄧定、隗文復入據成都,征虜將
 軍楊謙棄涪城,退保德陽。五月戊申,進慕容皝為安北將軍。石季龍又使其將石寧、麻秋等伐涼州,次于曲柳。張重華使將軍牛旋禦之,退守枹罕。六月辛酉,大赦。秋七月,范文復陷日南,害督護劉雄。隗文立范賁為帝。八月戊午,張重華將謝艾進擊麻秋,大敗之。九月,地震。冬十月乙丑,假涼州刺史張重華大都督隴右關中諸軍事、護羌校尉、大將軍,武都氐王楊初為征南將軍、雍州刺史、平羌校尉、仇池公,並假節。十二月,振威護軍蕭敬文害征虜將軍楊謙,攻涪城,陷之。遂取巴西,通于漢中。



 四年夏四月,范文寇九德,多所殺害。五月,大水。秋八月,
 進安西將軍桓溫為征西大將軍、開府儀同三司,封臨賀郡公;西中郎將謝尚為安西將軍。九月丙申,慕容皝死,子雋嗣偽位。冬十月己未,地震。石季龍使其將苻健寇竟陵。十二月,豫章人黃韜自號孝神皇帝,聚眾數千,寇臨川,太守庾條討平之。



 五年春正月辛巳朔,大赦。庚寅,地震。石季龍僭即皇帝位於鄴。二月,征北大將軍褚裒使部將王龕北伐,獲石季龍將支重。夏四月,益州刺史周撫、龍驤將軍朱燾擊范賁,獲之,益州平。封周撫為建城公。假慕容雋大將軍、幽平二州牧、大單于、燕王。征西大將軍桓溫遣督軍滕
 畯討范文,為文所敗。石季龍死,子世嗣偽位。五月,石遵廢世而自立。六月,桓溫屯安陸,遣諸將討河北。石遵揚州刺史王浹以壽陽來降。秋七月,褚裒進次彭城,遣部將王龕、李邁及石遵將李農戰于代陂,王師敗績,王龕為農所執,李邁死之。八月,褚裒退屯廣陵,西中郎將陳逵焚壽春而遁。梁州刺史司馬勳功石遵長城戍,仇池公楊初襲西城,皆破之。冬十月,石遵將石遇攻宛,陷之,執南陽太守郭啟。司馬勛進次懸鉤,石季龍故將麻秋距之,勛退還梁州。十一月丙辰,石鑒弒石遵而自立。十二月己酉,使持節、都督徐兗二州諸軍事、徐州刺史、征
 北大將軍、開府儀同三司、都鄉侯褚裒卒。以建武將軍、吳國內史荀羨為使持節、監徐兗二州諸軍事、北中郎將、徐州刺史。



 六年春正月,帝臨朝,以褚裒喪故,懸而不樂。閏月,冉閔弒石鑒,僭稱天王,國號魏。鑒弟祗僭帝號于襄國。丁丑,彗星見于亢。己丑,加中軍將軍殷浩督揚豫徐兗青五州諸軍事、假節。氐帥苻洪遣使來降,以為氐王,封廣川郡公。假洪子健節,監河北諸軍事、右將軍,封襄國縣公。三月,石季龍故將麻秋鴆殺苻洪於枋頭。夏五月,大水。廬江太守袁真攻合肥,剋之。六月,石祗遣其弟琨攻冉
 閔將王泰於邯鄲,琨師敗績。秋八月,輔國將軍、譙王無忌薨。苻健帥眾入關。冬十一月,冉閔圍襄國。十二月,免司徒蔡謨為庶人。是歲,大疫。



 七年春正月丁酉,日有蝕之。辛丑,鮮卑段龕以青州來降。苻健僭稱王,國號秦。二月戊寅,以段龕為鎮北將軍,封齊公。石祗大敗冉閔於襄國。夏四月,梁州刺史司馬勛出步騎三萬,自漢中入秦川,與苻健戰于五丈原,王師敗績。加尚書令顧和開府儀同三司。劉顯殺石祗。五月,祗兗州刺史劉啟自鄄城來奔。秋七月,尚書令、左光祿大夫、開府儀同三司顧和卒。甲辰,濤水入石頭,溺死
 者數百人。八月,冉閔豫州牧張遇以許昌來降,拜鎮西將軍。九月,峻陽、太陽二陵崩。甲辰,帝素服臨於太極殿三日,遣兼太常趙拔修復山陵。冬十月,雷雨,震電。十一月,石祗將姚弋仲、冉閔將魏脫各遣使來降,以弋仲為車騎將軍、大單于,封高陵郡公;弋仲子襄為平北將軍、都督并州諸軍事、并州刺史、平鄉縣公;脫為安北將軍、監冀州諸軍事、冀州刺史。十二月辛未,征西大將軍桓溫帥眾北伐,次于武昌而止。時石季龍故將周成屯廩丘,高昌屯野王,樂立屯許昌,李歷屯衛國,皆相次來降。



 八年春正月辛卯,日有蝕之。劉顯僭帝號於襄國,冉閔
 擊破,殺之。苻健僭帝號於長安。二月,峻平、崇陽二陵崩。戊辰,帝臨三日,遣殿中都尉王惠如洛陽,以衛五陵。鎮西將軍張遇反於許昌,使其黨上官恩據洛陽。樂弘攻督護戴施於倉垣。三月,使北中郎荀羨鎮淮陰。苻健別帥侵順陽,太守薛珍擊破之。夏四月,冉閔為慕容雋所滅。雋僭帝號於中山,稱燕。安西將軍謝尚帥姚襄與張遇戰於許昌之誡橋,王師敗績。苻健使其弟雄襲遇,虜之。秋七月,大雩。石季龍故將王擢遣使請降,拜征西將軍、秦州刺史。丁酉,以鎮軍大將軍、武陵王晞為太宰,撫軍大將軍、會稽王昱為司徒,征西大將軍桓溫為太
 尉。八月,平西將軍周撫討蕭敬文於涪城,斬之。冉閔子智以鄴降,督護戴施獲其傳國璽,送之,文曰「受天之命,皇帝壽昌」,百僚畢賀。九月,冉智為其將馬願所執,降於慕容恪。中軍將軍殷浩帥眾北伐,次泗口,遣河南太守戴施據石門,滎陽太守劉遂戍倉垣。冬十月,秦州刺史王擢為苻健所逼,奔于涼州。



 九年春正月乙卯朔,大赦。張重華使王擢與苻健將苻雄戰,擢師敗績。丙寅,皇太后與帝同拜建平陵。三月,旱。交州刺史阮敷討林邑范佛於日南,破其五十餘壘。夏四月,以安西將軍謝尚為尚書僕射。五月,大疫。張重華
 復使王擢襲秦州,取之。仇池公楊初為苻雄所敗。秋七月丁酉,地震,有聲如雷。八月,遣兼太尉、河間王欽修復五陵。冬十月,中軍將軍殷浩進次山桑,使平北將軍姚襄為前鋒,襄叛,反擊浩,浩棄輜重,退保譙城。丁未,涼州牧張重華卒,子耀靈嗣。是月,張祚弒耀靈而自稱涼州牧。十一月,殷浩使部將劉啟、王彬之討姚襄,復為襄所敗,襄遂進據芍陂。十二月,加尚書僕射謝尚為都督豫、揚、江西諸軍事,領豫州刺史,鎮歷陽。



 十年春正月己酉朔,帝臨朝,以五陵未復,懸而不樂。涼州牧張祚僭帝位。冉閔降將周成舉兵反,自宛陵襲洛
 陽。辛酉,河南太守戴施奔鮪渚。丁卯,地震,有聲如雷。二月己丑,太尉、征西將軍桓溫帥師伐關中。廢揚州刺史殷浩為庶人,以前會稽內史王述為揚州刺史。夏四月己亥,溫及苻健子萇戰于藍田,大敗之。五月,江西乞活郭敞等執陳留內史劉仕而叛,京師震駭,以吏部尚書周閔為中軍將軍,屯于中堂,豫州刺史謝尚自歷陽還衛京師。六月,苻健將苻雄悉眾及桓溫戰於白鹿原,王師敗績。秋九月辛酉,桓溫糧盡,引還。



 十一年春正月甲辰,侍中、汝南王統薨。平羌校尉、仇池公楊初為其部將梁式所害,初子國嗣位,因拜鎮北將
 軍、秦州刺史。齊公段龕襲慕容雋將榮國於郎山,敗之。夏四月壬申,隕霜。乙酉,地震。姚襄帥眾寇外黃,冠軍將軍高季大破之。五月丁未,地又震。六月,苻健死,其子生嗣偽位。秋七月,宋混、張瓘弒張祚,而立耀靈弟玄靚為大將軍、涼州牧,遣使來降。以吏部尚書周閔為尚書左僕射,領軍將軍王彪之為尚書右僕射。冬十月,進豫州刺史謝尚督并冀幽三州諸軍事、鎮西將軍,鎮馬頭。十二月,慕容恪帥眾寇廣固。壬戌,上黨人馮鴦自稱太守,背苻生遣使來降。



 十二年春正月丁卯,帝臨朝,以皇太后母喪,懸而不樂。
 鎮北將軍段龕及慕容恪戰於廣固,大敗之,恪退據安平。二月辛丑,帝講《孝經》。三月,姚襄入于許昌,以太尉桓溫為征討大都督以討之。秋八月己亥,桓溫及姚襄戰於伊水,大敗之,襄走平陽,徙其餘眾三千餘家於江漢之間,執周成而歸。使揚武將軍毛穆之,督護陳午,輔國將軍、河南太守戴施鎮洛陽。冬十月癸巳朔,日有蝕之。慕容恪攻段龕於廣固,使北中郎將荀羨帥師次于瑯邪以救之。十一月,遣兼司空、散騎常侍車灌,龍驤將軍袁真等持節如洛陽,修五陵。十二月庚戌,以有事於五陵,告於太廟,帝及群臣皆服緦,於太極殿臨三日。是歲,
 仇池公楊國為其從父俊所殺,俊自立。



 升平元年春正月壬戌朔,帝加元服,告天太廟,始親萬機。大赦,改元,增文武位一等。皇太后居崇德宮。丁丑,隕石於槐里一。是月,鎮北將軍、齊公段龕為慕容恪所陷,遇害。扶南竺旃檀獻馴象,詔曰:「昔先帝以殊方異獸或為人患,禁之。今及其未至,可令還本土。」三月,帝講《孝經》。壬申,親釋奠于中堂。夏五月庚午,鎮西將軍謝尚卒。苻生將苻眉、苻堅擊姚襄,戰于三原,斬之。六月,苻堅殺苻生而自立。以軍司謝奕為使持節、都督、安西將軍、豫州刺史。秋七月,苻堅將張平以并州降,遂以為並州刺
 史。八月丁未,立皇后何氏,大赦,賜孝悌鰥寡米,人五斛,逋租宿債皆勿收,大酺三日。冬十月,皇后見于太廟。十一月,雷。十二月,以太常王彪之為尚書左僕射。



 二年春正月,司徒、會稽王昱稽首歸政,帝不許。三月,慕容雋陷冀州諸郡,詔安西將軍謝奕、北中郎將荀羨北伐。三月,佽飛督王饒獻鴆鳥,帝怒,鞭之二百,使殿中御史焚其鳥於四達之衢。夏五月,大水。有星孛于天船。六月,并州刺史張平為苻堅所逼,帥眾三千奔于平陽,堅追敗之。慕容恪進據上黨,冠軍將軍馮鴦以眾叛歸慕容雋,雋盡陷河北之地。秋八月,安西將軍謝奕卒。壬申,
 以吳興太守謝萬為西中郎將、持節、監司豫冀并四州諸軍事、豫州刺史。以散騎常侍郗曇為北中郎將、持節、都督徐兗青冀幽五州諸軍事、徐兗二州刺史,鎮下邳。冬十月乙丑,陳留王曹勱薨。十一月庚子,雷。辛酉,地震。十二月,北中郎將荀羨及慕容雋戰于山荏,王師敗績。



 三年春三月甲辰,詔以比年出軍,糧運不繼,王公已下十三戶借一人一年助運。秋七月,平北將軍高昌為慕容雋所逼,自白馬奔於滎陽。冬十月慕容雋寇東阿,遣西中郎將謝萬次下蔡,北中郎將郗曇次高平以擊之,王師敗績。十一月戊子,進揚州刺史王述為衛將軍。十
 二月,又以中軍將軍、琅邪王丕為驃騎將軍,東海王奕為車騎將軍。封武陵王晞子逢為梁王。交州刺史溫放之帥兵討林邑參黎、耽潦,並降之。



 四年春正月,仇池公楊俊卒,子世嗣。丙戌,慕容雋死,子嗣偽位。二月,鳳皇將九雛見于豐城。秋七月,以軍役繁興,省用撤膳。八月辛丑朔,日有蝕之,既。冬十月,天狗流于西南。十一月,封太尉桓溫為南郡公,溫弟沖為豐城縣公,子濟為臨賀郡公。鳳皇復見豐城,眾鳥隨之。



 五年春正月戊戌,大赦,賜鰥寡孤獨不能自存者,人米五斛。北中郎將、都督徐兗青冀幽五州諸軍事、徐兗二
 州刺史郗曇卒。二月,以鎮軍將軍范汪為都督徐兗青冀幽五州諸軍事、安北將軍、徐兗二州刺史。平南將軍、廣州刺史、陽夏侯滕含卒。夏四月,大水。太尉桓溫鎮宛,使其弟豁將兵取許昌。鳳皇見于沔北。



 五月丁巳,帝崩于顯陽殿,時年十九。葬永平陵,廟號孝宗。



 哀皇帝諱丕,字千齡,成帝長子也。咸康八年,封為瑯邪王。永和元年拜散騎常侍,十二年加中軍將軍,升平三年除驃騎將軍。五年五月丁巳,穆帝崩。皇太后令曰:「帝奄不救疾,胤嗣未建。瑯邪王丕,中興正統,明德懋親。昔
 在咸康,屬當儲貳。以年在幼沖,未堪國難,故顯宗高讓。今義望情地,莫與為比,其以王奉大統。」於是百官備法駕,迎于琅邪第。庚申,即皇帝位,大赦。壬戌,詔曰:「朕獲承明命,入纂大統。顧惟先王宗廟,蒸嘗無主,太妃喪庭,廓然靡寄,悲痛感摧,五內抽割。宗國之尊,情禮兼隆,胤嗣之重,義無與二。東海王奕,戚屬親近,宜奉本統,其以奕為琅邪王。」秋七月戊午,葬穆皇帝于永平陵。慕容恪攻陷野王,守將呂護退保滎陽。八月己卯夜,天裂,廣數丈,有聲如雷。九月戊申,立皇后王氏。穆帝皇后何氏稱永安宮。呂護叛奔于莫容。冬十月,安北將軍范汪有罪
 廢為庶人。十一月丙辰,詔曰:「顯宗成皇帝顧命,以時事多艱,弘高世之風,樹德博重,以隆社稷。而國故不巳,康穆早世,胤祚不融。朕以寡德,復承先緒,感惟永慕,悲育兼摧。夫昭穆之義,固宜本之天屬。繼體承基,古今常道。宜上嗣顯宗,以脩本統。」十二月,加涼州刺史張玄靚為大都督隴右諸軍事、護羌校尉、西平公。



 隆和元年春正月壬子,大赦,改元。甲寅,減田稅,畝收二升。是月,慕容將呂護、傅末波攻陷小壘,以逼洛陽。二月辛未,以輔國將軍、吳國內史庾希為北中郎將、徐兗二州刺史,鎮下邳;前鋒監軍、龍驤將軍袁真為西中郎
 將、監護豫司並冀四州諸軍事、豫州刺史,鎮汝南,並假節。丙子,尊所生周氏為皇太妃。三月甲寅朔,日有蝕之。夏四月,旱。詔出輕繫,振困乏。丁丑,梁州地震,浩釁山崩。呂護復寇洛陽。乙酉,輔國將軍、河南太守戴施奔于宛。五月丁巳,遣北中郎將庾希、竟陵太守鄧遐以舟師救洛陽。秋七月,呂護等退守小平津。進琅邪王奕為侍中、驃騎大將軍、開府。鄧遐進屯新城,庾希部將何謙及慕容將劉則戰于檀丘,破之。八月,西中郎將袁真進次汝南,運米五萬斛以饋洛陽。冬十月,賜貧乏者米,人五斛。章武王珍薨。十二月戊午朔,日有蝕之。詔曰:「戎旅路
 次,未得輕簡賦役。玄象失度,亢旱為患,豈政事未洽,將有板築、渭濱之士邪!其搜揚隱滯,蠲除苛碎,詳議法令,咸從損耍。」庾希自下邳退鎮山陽,袁真自汝南退鎮壽陽。



 興寧元年春二月己亥,大赦,改元。三月壬寅,皇太妃薨於瑯邪第。癸卯,帝奔喪,詔司徒、會稽王昱總內外眾務。夏四月,慕容寇滎陽,太守劉遠奔魯陽。甲戌,揚州地震,湖瀆溢。五月,加征西大將軍桓溫侍中、大司馬、都督中外諸軍事、錄尚書事、假黃鉞。復以西中郎將袁真都督司、冀、并三州諸軍事,北中郎將瘦希都督青州諸軍
 事。癸卯,慕容陷密城,滎陽太守劉遠奔于江陵。秋七月,張天錫弒涼州刺史、西平公張玄靚,自稱大將軍、護羌校尉、涼州牧、西平公。丁酉,葬章皇太妃。八月,有星孛于角亢,入天市。九月壬戌,大司馬桓溫帥眾北伐。癸亥,以皇子生,大赦。冬十月甲申,立陳留王世子恢為王。十一月,姚襄故將張駿殺江州督護趙毗,焚武昌,略府藏以叛,江州刺史桓沖討斬之。是歲,慕容將慕容塵攻陳留太守袁披于長平。汝南太守硃斌承虛襲許昌,剋之。



 二年春二月庚寅,江陵地震。慕容將慕容評襲許昌,
 潁川太守李福死之。評遂侵汝南,太守朱斌遁于壽陽。又進圍陳郡,太守朱輔嬰城固守。桓溫遣江夏相劉岵擊退之。改左軍將軍為遊擊將軍,罷右軍、前軍、後軍將軍五校三將官。癸卯,帝親耕藉田。三月庚戌朔,大閱戶人,嚴法禁,稱為庚戌制。辛未,帝不豫。帝雅好黃老,斷穀,餌長生藥,服食過多,遂中毒,不識萬機,崇德太后復臨朝攝政。夏四月甲申,慕容遣其將李洪侵許昌,王師敗績于懸瓠,朱斌奔於淮南,朱輔退保彭城。桓溫遣西中郎將袁真、江夏相劉岵等鑿陽儀道以通運,溫帥舟師次於合肥,慕容塵復屯許昌。五月,遷陳人于陸以避
 之。戊辰,以揚州刺史王述為尚書令、衛將軍。以桓溫為揚州牧、錄尚書事。壬申,遣使喻溫入相,溫不從。秋七月丁卯,復徵溫入朝。八月,溫至赭圻,遂城而居之。苻堅別帥侵河南,慕容寇洛陽。九月,冠軍將軍陳祐留長史沈勁守洛陽,帥眾奔新城。



 三年春正月庚申,皇后王氏崩。二月乙未,以右將軍桓豁監荊州揚州之義城雍州之京兆諸軍事、領南蠻校尉、荊州刺史;桓沖監江州荊州之江夏隨郡豫州之汝南西陽新蔡潁川六郡諸軍事、南中郎將、江州刺史,領南蠻校尉,並假節。



 丙申,帝崩於西堂,時年二十五。葬安
 平陵。



 廢帝諱奕,字延齡,哀帝之母弟也。咸康八年封為東海王。永和八年拜散騎常侍,尋加鎮軍將軍;升平四年拜車騎將軍。五年,改封琅邪王。隆和初,轉侍中、驃騎大將軍、開府儀同三司。興寧三年二月丙申,哀帝崩,無嗣。丁酉,皇太后詔曰:「帝遂不救厥疾,艱禍仍臻,遺緒泯然,哀慟切心。琅邪王奕,明德茂親,屬當儲嗣,宜奉祖宗,纂承大統。便速正大禮,以寧人神。」於是百官奉迎于琅邪第。是日,即皇帝位,大赦。三月壬申,葬哀皇帝於安平陵。癸
 酉,散騎常侍、河間王欽薨。丙子,慕容將慕容恪陷洛陽,寧朔將軍竺瑤奔于襄陽,冠軍長史、揚武將軍沈勁死之。夏六月戊子,使持節、都督益寧二州諸軍事、鎮西將軍、益州刺史、建城公周撫卒。秋七月,匈奴左賢王衛辰、右賢王曹穀帥眾二萬侵苻堅杏城。己酉,改封會稽王昱為琅邪王。壬子,立皇后庾氏。封琅邪王昱子昌明為會稽王。冬十月,梁州刺史司馬勳反,自稱成都王。十一月,帥眾人劍閣,攻涪,西夷校尉毌丘棄城而遁。乙卯,圍益州刺史周楚於成都,桓溫遣江夏相朱序救之。十二月戊戌,以會稽內史王彪之為尚書僕射。



 太和元年春二月己丑,以涼州刺史張天錫為大將軍、都督隴右關中諸軍事、西平郡公。丙申,以宣城內史桓秘為持節、監梁益二州征討諸軍事。三月辛亥,新蔡王邈薨。荊州刺史桓豁遣督護桓羆攻南鄭,魏興人畢欽舉兵以應羆。夏四月,旱。五月戊寅,皇后庾氏崩。朱序攻司馬勛於成都,眾潰,執勳,斬之。秋七月癸酉,葬孝皇后於敬平陵。九月甲午,曲赦梁、益二州。冬十月辛丑,苻堅將王猛、楊安攻南鄉,荊州刺史桓豁救之,師次新野而猛、安退。以會稽王昱為丞相。十二月,南陽人趙弘、趙憶等據宛城反,太守桓澹走保新野。慕容將慕容厲陷
 魯郡、高平。



 二年春正月,北中郎將庾希有罪,走入于海。夏四月,慕容將慕容塵寇竟陵,太守羅崇擊破之。苻堅將王猛寇涼州,張天錫距之,猛師敗績。五月,右將軍桓豁擊趙憶,走之,進獲慕容將趙槃,送于京師。秋九月,以會稽內史郗愔為都督徐兗青幽四州諸軍事、平北將軍、徐州刺史。冬十月乙巳,彭城王玄薨。



 三年春三月丁巳朔,日有蝕之。癸亥,大赦。夏四月癸巳,雨雹,大風折木。秋八月壬寅,尚書令、衛將軍、藍田侯王述卒。



 四年夏四月庚戌,大司馬桓溫帥眾伐慕容。秋七月辛卯,將慕容垂帥眾距溫,溫擊敗之。九月戊寅,桓溫裨將鄧遐、朱序遇將傅末波於林渚,又大破之。戊子,溫至枋頭。丙申,以糧運不繼,焚舟而歸。辛丑,慕容垂追敗溫後軍於襄邑。冬十月,大星西流,有聲如雷。己巳,溫收散卒,屯于山陽。豫州刺史袁真以壽陽叛。十一月辛丑,桓溫自山陽及會稽王昱會于塗中,將謀後舉。十二月,遂城廣陵而居之。



 五年春正月己亥,袁真子雙之、愛之害梁國內史朱憲、汝南內史朱斌。二月癸酉,袁真死,陳郡太守朱輔立真
 子瑾嗣事,求救於慕容。夏四月辛未,桓溫部將竺瑤破瑾于武丘。秋七月癸酉朔,日有蝕之。八月癸丑,桓溫擊袁瑾於壽陽,敗之。九月,苻堅將猛伐慕容,陷其上黨。廣漢妖賊李弘與益州妖賊李金根聚眾反,弘自稱聖王,眾萬餘人,梓潼太守周虓討平之。冬十月,王猛大破慕容將慕容評於潞川。十一月,猛剋鄴,獲慕容,盡有其地。



 六年春正月,苻堅遣將王鑒來援袁瑾,將軍桓伊逆擊,大破之。丁亥,桓溫剋壽陽,斬袁瑾。三月壬辰,監益寧二州諸軍事、冠軍將軍、益州刺史、建城公周楚卒。夏四月
 戊午,大赦,賜窮獨米,人五斛。苻堅將苻雅伐仇池,仇池公楊纂降之。六月,京都及丹陽、晉陵、吳郡、吳興、臨海並大水。秋八月,以前寧州刺史周仕孫為假節、監益梁二州諸軍事、益州刺史。冬十月壬子,高密王俊薨。十一月癸卯,桓溫自廣陵屯于白石。丁未,詣闕,因圖廢立,誣帝在籓夙有痿疾,嬖人相龍、計好、朱靈寶等參侍內寢,而二美人田氏、孟氏生三男,長欲封樹,時人惑之,溫因諷太后以伊霍之舉。己酉,集百官於朝堂,宣崇德太后令曰:「王室艱難,穆、哀短祚,國嗣不育儲宮靡立。瑯邪王奕親則母弟,故以入纂大位。不圖德之不建,乃至於斯。昏
 濁潰亂,動違禮度。有此三孽,莫知誰子。人倫道喪,醜聲遐布。既不可以奉守社稷,敬承宗廟,且昏孽並大,便欲建樹儲籓。誣罔祖宗,頌移皇基,是而可忍,孰不可懷!今廢奕為東海王,以王還第,供衛之儀,皆如漢朝昌邑故事。但未亡人不幸,罹此百憂,感念存沒,心焉如割。社稷大計,義不獲已。臨紙悲塞,如何可言。」於是百官入太極前殿,即日桓溫使散騎侍郎劉享收帝璽綬。帝著白帢單衣,步下西堂,乘犢車出神獸門。群臣拜辭,莫不覷欷。侍御史、殿中監將兵百人衛送東海第。



 初,桓溫有不臣之心,欲先立功河朔,以收時望。及枋頭之敗,威名頓挫,
 逐潛謀廢立,以長威權。然憚帝守道,恐招時議。以宮闡重悶,床笫易誣,乃言帝為閹,遂行廢辱。初,帝平生每以為慮,嘗召術人扈謙筮之,卦成,答曰:「晉室有盤石之固,陛下有出宮之象。」竟如其言。



 咸安二年正月,降封帝為海西縣公。四月,徙居吳縣,敕吳國內史刁彞防衛,又遣御史顧允監察之。十一月,妖賊盧悚遣弟子殿中監許龍晨到其門,稱太后密詔,奉迎興復。帝初欲從之,納保母諫而止。龍曰:「大事將捷,焉用兒女子言乎?」帝曰:「我得罪於此,幸蒙寬宥,豈敢妄動哉!且太后有詔,便應官屬來,何獨使汝也?汝必為亂。」因叱左右縛之,龍懼而走。帝
 知天命不可再,深慮橫禍,乃杜塞聰明,無思無慮,終日酣暢,耽于內寵,有子不育,庶保天年。時人憐之,為作歌焉。朝廷以帝安于屈辱,不復為虞。太元十一年十月甲申,薨于吳,時年四十五。



 史臣曰:孝宗因繦抱之姿,用母氏之化,中外無事,十有餘年。以武安之才,啟之疆埸;以文王之風,被乎江漢,則孔子所謂吾無間然矣。哀皇寬惠,可以為君,而鴻祀禳天,用塵其德。東海違許龍之駕,屈放命之臣,所謂柔弱勝剛彊,得盡於天年者也。



 贊曰:委裘稱化,大孝為宗,遵彼聖善,成茲允恭。西旌玉
 壘,北旆金墉。遷殷舊僰,莫不來從。哀后寬仁,惟靈既集。海西多故,時災見及。彼異阿衡,我非昌邑。



\end{pinyinscope}