\article{列傳第四十八}

\begin{pinyinscope}

 孔愉子汪安國弟祗從子坦嚴從弟群群子沉丁潭張茂陶回



 孔愉,字敬康,會稽山陰人也。其先世居梁國。曾祖潛,太子少傅,漢末避地會稽,因家焉。祖竺,吳豫章太守。父恬,湘東太守。從兄侃,大司農。俱有名江左。愉年十三而孤,養祖母以孝聞,與同郡張茂字偉康、丁潭字世康齊名,時人號曰「會稽三康」。吳平,愉遷於洛。惠帝末,歸鄉里,行至江淮間,遇石冰、封雲為亂,雲逼愉為參軍,不從將殺
 之,賴雲司馬張統營救獲免。東還會稽,人新安山中,改姓孫氏,以稼穡讀書為務,信著鄉里。後忽捨去,皆謂為神人,而為之立祠。永嘉中,元帝始以安東將軍鎮揚土,命愉為參軍。邦族尋求,莫知所在。建興初,始出應召。為丞相掾,仍除駙馬都尉、參丞相軍事,時年已五十矣。以討華軼功,封餘不亭侯。愉嘗行經餘不亭,見籠龜於路者,愉買而放之溪中,龜中流左顧者數四。及是,鑄侯印,而印龜左顧,三鑄如初。印工以告,愉乃悟,遂佩焉。



 帝為晉王,使長兼中書郎。於時刁協、劉隗用事,王導頗見疏遠。愉陳導忠賢,有佐命之勛,謂事無大小皆宜諮訪。由
 是不合旨,出為司徒左長史,累遷吳興太守。沈充反,愉棄官還京師,拜御史中丞,遷侍中、太常。及蘇峻反,愉朝服守宗廟。初,愉為司徒長史,以平南將軍溫嶠母亡遭亂不葬,乃不過其品。至是,峻平,而嶠有重功,愉往石頭詣嶠,嶠執愉手而流涕曰:「天下喪亂,忠孝道廢。能持古人之節,歲寒不凋者,唯君一人耳。」時人咸稱嶠居公而重愉之守正。尋徙大尚書,遷安南將軍、江州刺史,不行。轉尚書右僕射,領東海王師。尋遷左僕射。



 咸和八年,詔曰:「尚書令玩、左僕射愉並恪居官次,祿不代耕。端右任重,先朝所崇,其給玩親信三十人,愉二十人,稟賜。」愉上
 疏固讓,優詔不許。重表曰:「臣以朽暗,忝廁朝右,而以惰劣,無益毗佐。方今彊寇未殄,疆場日駭,政煩役重,百姓困苦,奸吏擅威,暴人肆虐。大弊之後,倉庫空虛,功勞之士,賞報不足,困悴之餘,未見拯恤,呼嗟之怨,人鬼感動。宜并官省職,貶食節用,勤撫其人,以濟其艱。臣等不能贊揚大化,糾明刑政,而偷安高位,橫受寵給,無德而祿,殃必及之,不敢橫受殊施,以重罪戾。」從之。王導聞而非之,於都坐謂愉曰:「君言姦吏擅威,暴人肆虐,為患是誰?」愉欲大論朝廷得失,陸玩抑之乃止。後導將以趙胤為護軍,愉謂導曰:「中興以來,處此官者,周伯仁、應思遠耳。
 今誠乏才,豈宜以趙胤居之邪!」導不從。其守正如此。由是為導所銜。



 後省左右僕射,以愉為尚書僕射。愉年在懸車,累乞骸骨,不許,轉護軍將軍,加散騎常侍。復徙領軍將軍,加金紫光祿大夫,領國子祭酒。頃之,出為鎮軍將軍、會稽內史,加散騎常侍。句章縣有漢時舊陂,毀廢數百年。愉自巡行,修復故堰,溉田二百餘頃,皆成良業。在郡三年,乃營山陰湖南侯山下數畝地為宅,草屋數間,便棄官居之。送資數百萬,悉無所取。病篤,遺令斂以時服,鄉邑義賵,一不得受。年七十五,咸康八年卒。贈車騎將軍、開府儀同三司,謚曰貞。



 三子:訚、汪、安國。訚嗣爵,
 位至建安太守。訚子靜,字季恭,再為會稽內史,累遷尚書左僕射,加後將軍。



 汪字德澤,好學有志行,孝武帝時位至侍中。時茹千秋以佞媚見幸於會稽王道子,汪屢言之於帝,帝不納。遷尚書太常卿,以不合意,求出。為假節、都督交廣二州諸軍事、征虜將軍、平越中郎將、廣州刺史,甚有政績,為嶺表所稱。太元十七年卒。



 安國字安國,年小諸兄三十餘歲。群從諸兄並乏才名,以富彊自立,唯安國與汪少厲孤貧之操。汪既以直亮稱,安國亦以儒素顯。孝武帝時甚蒙禮遇,仕歷侍中、太
 常。及帝崩,安國形素贏瘦,服衰絰,涕泗竟日,見者以為真孝,再為會稽內史、領軍將軍。安帝隆安中下詔曰:「領軍將軍孔安國貞慎清正,出內播譽,可以本官領東海王師,必能導達津梁,依仁游藝。」後歷尚書左右僕射。義熙四年卒,贈左光祿大夫。



 祗字承祖。太守周札命為功曹史。札為沈充所害,故人賓吏莫敢近者。祗冒刃號哭,親行殯禮,送喪還義興,時人義之。



 坦字君平。祖沖,丹陽太守。父侃,大司農。坦少方直,有雅望,通《左氏傳》,解屬文。完帝為晉王,以坦為世子文學。東
 宮建,補太子舍人,遷尚書郎。時臺郎初到,普加策試,帝手策問曰:「吳興徐馥為賊,殺郡將,郡今應舉孝廉不?」坦對曰:「四罪不相及,殛鯀而興禹。徐馥為逆,何妨一郡之賢!」又問:「姦臣賊子弒君,污宮瀦宅,莫大之惡也。鄉舊廢四科之選,今何所依?」坦曰:「季平子逐魯昭公,豈可以廢仲尼也!」竟不能屈



 先是,以兵亂之後,務存慰悅,遠方秀孝到,不策試,普皆除署。至是,帝申明舊制,皆令試《經》,有不中科,刺史、太守免官。太興三年,秀孝多不敢行,其有到者,並託疾。帝欲除署孝廉,而秀才如前制。坦奏議曰:



 臣聞經邦建國,教學為先,移風崇化,莫尚斯矣。古者且耕
 且學,三年而通一經,以平康之世,猶假漸漬,積以日月。自喪亂以來,十有餘年,於戈載揚,俎豆禮戢,家廢講誦,國闕庠序,率爾責試,竊以為疑。然宣下以來,涉歷三載,累遇慶會,遂未一試。揚州諸郡,接近京都,懼累及君父,多不敢行。其遠州邊郡,掩誣朝廷,冀於不試,冒昧來赴,既到審試,遂不敢會。臣愚以不會與不行,其為闕也同。若當偏加除署,是為肅法奉憲者失分,僥倖投射者得官,頹風傷教,懼於是始。



 夫王言如絲,其出如綸,臨事改制,示短天下,人聽有惑,臣竊惜之。愚以王命無貳,憲制宜信。去年察舉,一皆策試。如不能試,可不拘到,遣歸不
 署。又秀才雖以事策,亦汜問經義,茍所未學,實難闇通,不足復曲碎垂例,違舊造異。謂宜因其不會,徐更革制。可申明前下,崇修學校,普延五年,以展講習,鈞法齊訓,示人軌則。夫信之與法,為政之綱,施之家室,猶弗可貳,況經國之典而可玩黷乎!



 帝納焉。聽孝廉申至七年,秀才如故。



 時典客令萬默領諸胡,胡人相誣,朝廷疑默有所偏助,將加大辟。坦獨不署,由是被譴,遂棄官歸會稽。久之,除領軍司馬,未赴召。會王敦反,與右衛將軍虞潭俱在會稽起義,而討沈充。事平,始就職。揚州刺史王導請為別駕。



 咸和初,遷尚書左丞,深為臺中之所敬憚。尋
 屬蘇峻反,坦與司徒司馬陶回白王導曰:「及峻未至,宜急斷阜陵之界,守江西當利諸口,彼少我眾,一戰決矣。若峻未至,可往逼其城。今不先往,峻必先至。先人有奪人之功,時不可失。」導然之。庾亮以為峻脫徑來,是襲朝廷虛也,故計不行。峻遂破姑熟,取鹽米,亮方悔之。坦謂人曰:「觀峻之勢,必破臺城。自非戰士,不須戎服。」既而臺城陷,戎服者多死,白衣者無他,時人稱其先見。及峻挾天子幸石頭,坦奔陶侃,侃引為長史。時侃等夜築白石壘,至曉而成。聞峻軍嚴聲,咸懼來攻。坦曰:「不然。若峻攻壘,必須東北風急,令我水軍不得往救。今天清靜,賊必
 不動,決遣軍出江乘,掠京口以東矣。」果如所籌。時郗鑒鎮京口,侃等各以兵會。既至,坦議以為本不應須召郗公,遂使東門無限。今宜遣還,雖晚,猶勝不也。侃等猶疑,坦固爭甚切,始令鑒還據京口,遣郭默屯大業,又令驍將李閎、曹統、周光與默并力,賊遂勢分,卒如坦計。



 及峻平,以坦為吳郡太守。自陳吳多賢豪,而坦年少,未宜臨之。王導、庾亮並欲用坦為丹陽尹。時亂離之後,百姓凋弊,坦固辭之。導等猶未之許。坦慨然曰:「昔肅祖臨崩,諸君親據御床,共奉遺詔。孔坦疏賤,不在顧命之限。既有艱難,則以微臣為先。今由俎上肉,任人膾截耳!」乃拂衣
 而去。導等亦止。於是遷吳興內史,封晉陵男,加建威將軍。以歲饑,運家米以振窮乏,百姓賴之。時使坦募江淮流人為軍,有殿中兵,因亂東還,來應坦募,坦不知而納之。或諷朝廷,以坦藏臺叛兵,遂坐免。尋拜侍中。



 三康元年,石聰寇歷陽,王導為大司馬,討之,請坦為司馬。會石勒新死,季龍專恣,石聰及譙郡太守彭彪等各遣使請降。坦與聰書曰:



 華狄道乖,南北回邈,瞻河企宋,每懷飢渴。數會陽九,天禍晉國,姦凶猾夏,乘釁肆虐。我德雖衰,天命未改。乾符啟再集之慶,中興應靈期之會,百六之艱既過,惟新之美日隆。而神州振蕩,遺氓波散,誓命戎
 狄之手,局蹐豺狼之穴,朝廷每臨寐永歎,痛心疾首。天罰既集,罪人斯隕,王旅未加,自相魚肉。豈非人怨神怒,天降其災!蘭艾同焚,賢愚所歎,哀矜勿喜,我后之仁,大赦曠廓,唯季龍是討。彭譙使至,粗具動靜,知將軍忿疾醜類,翻然同舉。承問欣豫,慶若在己。何知幾之先覺,砎石之易悟哉!引領來儀,怪無聲息。



 將軍出自名族,誕育洪胄。遭世多故,國傾家覆,生離親屬,假養異類。雖逼偽寵,將亦何賴!聞之者猶或有悼,況身嬰之,能不憤慨哉!非我族類,其心必異,誠反族歸正之秋,圖義建功之日也。若將軍喻納往言,宣之同盟,率關右之眾,輔河南之
 卒,申威趙魏,為國前驅,雖竇融之保西河,黥布之去項羽,比諸古今,未足為喻。聖上寬明,宰輔弘納,雖射鉤之隙,賞之故行,雍齒之恨,侯之列國。況二三子無曩人之嫌,而遇天啟之會,當如影響,有何遲疑!



 今六軍誡嚴,水陸齊舉,熊羆踴躍,齕噬爭先,鋒鏑一交,玉石同碎,雖復後悔,何嗟及矣!僕以不才,世荷國寵,雖實不敏,誠為行李之主,區區之情,還信所具。夫機事不先,鮮不後悔,自求多福,唯將軍圖之。



 朝廷遂不果北伐,人皆懷恨。



 坦在職數年,遷侍中。時成帝每幸丞相王導府,拜導妻曹氏,有同家人,坦每切諫。時帝刻日納后,而尚書左僕射王
 彬卒,議者以為欲卻期。坦曰:「婚禮之重,重於救日蝕。救日蝕,有后之喪,太子墮井,則止。納后盛禮,豈可以臣喪而廢!」從之。及帝既加元服,猶委政王導,坦每發憤,以國事為己憂,嘗從容言於帝曰:「陛下春秋以長,聖敬日躋,宜博納朝臣,諮諏善道。」由是忤導,出為廷尉,怏怏不悅,以疾去職。加散騎常侍,遷尚書,未拜。



 疾篤,庾冰省之,乃流涕。坦慨然曰:「大丈夫將終不問安國寧家之術,乃作兒女子相問邪!」冰深謝焉。臨終,與庾亮書曰:「不謂疾苦,遂至頓弊,自省綿綿,奄忽無日。脩短命也,將何所悲!但以身往名沒,朝恩不報,所懷未敘,即命多恨耳!足下以
 伯舅之尊,居方伯之重,抗威顧眄,名震天下,榱椽之佐,常願下風。使九服式序,四海一統,封京觀於中原,反紫極於華壤,是宿昔之味詠,慷慨之本誠矣。今中道而斃,豈不惜哉!若死而有靈,潛聽風烈。」俄卒,時年五十一。追贈光祿勳,謚曰簡。亮報書曰:「廷尉孔君,神游體離,嗚呼哀哉!得八月十五日書,知疾患轉篤,遂不起濟,悲恨傷楚,不能自勝。足下方在中年,素少疾患,雖天命有在,亦禍出不圖。且足下才經於世,世常須才,況於今日,倍相痛惜。吾以寡乏,忝當大任,國恥未雪,夙夜憂憤。常欲足下同在外籓,戮力時事。此情未果,來書奄至。申尋往
 復,不覺涕隕。深明足下慷慨之懷,深痛足下不遂之志。邈然永隔,夫復何言!謹遣報答,並致薄祭,望足下降神饗之。」子混嗣。



 嚴字彭祖。祖父奕,全椒令,明察過人。時有遺其酒者,始提入門,奕遙呵之曰:「人餉吾兩罌酒,其一何故非也?」檢視之,一罌果是水。或問奕何以知之,笑曰:「酒重水輕,提酒者手有輕重之異故耳。」在官有惠化,及卒,市人若喪慈親焉。父倫,黃門郎。嚴少仕州郡,歷司徒掾、尚書殿中郎。殷浩臨揚州,請為別駕。遷尚書左丞。時朝廷崇樹浩,以抗擬桓溫,溫深以不平。浩又引接荒人,謀立功於閫
 外。嚴言於浩曰:「當今時事艱難,可謂百六之運,使君屈己應務,屬當其會。聖懷所以日昃匪懈,臨朝斤斤,每欲深根固本,靜邊寧國耳,亦豈至私哉!而處任者所志不同,所見各異,人口云云,無所不至。頃來天時人情,良可寒心。古人為政,防人之口甚於防川。間日侍座,亦已粗申所懷,不審竟當何以鎮之?《老子》云『夫唯不爭,則萬物不難與之爭』,此言不可不察也。愚意故謂朝廷宜更明授任之方,韓彭可專征伐,蕭曹守管籥,內外之任,各有攸司。深思廉藺屈申之道,平勃相和之義,令婉然通順,人無間言,然後乃可保大定功,平濟天下也。又觀頃日
 降附之徒,皆人面獸心,貪而無親,難以義感。而聚著都邑,雜處人間,使君常疲聖體以接之,虛府庫以拯之,足以疑惑視聽耳。」浩深納之。



 及哀帝踐阼,議所承統,時多異議。嚴與丹陽尹庾和議曰:「順本居正,親親不可奪,宜繼成皇帝。」諸儒咸以嚴議為長,竟從之。



 隆和元年,詔曰:「天文失度,太史雖有禳祈之事,猶釁眚屢彰。今欲依鴻祀之制,於太極殿前庭親執虔肅。」嚴諫曰:「鴻祀雖出《尚書大傳》,先儒所不究,歷代莫之興,承天接神,豈可以疑殆行事乎!天道無親,唯德是輔,陛下祗順恭敬,留心兆庶,可以消災復異。皆已蹈而行之,德合神明,丘禱久矣,
 豈須屈萬乘之尊,修雜祀之事!君舉必書,可不慎歟!」帝嘉之而止。以為揚州大中正,嚴不就。有司奏免,詔特以侯領尚書



 時東海王奕求海鹽、錢塘以水牛牽埭稅取錢直,帝初從之,嚴諫乃止。初,帝或施私恩,以錢帛賜左右。嚴又啟諸所別賜及給廚食,皆應減省。帝曰:「左右多困乏,故有所賜,今通斷之。又廚膳宜有減撤,思詳具聞。」嚴多所匡益。



 太和中,拜吳興太守,加秩中二千石。善於宰牧,甚得人和。餘杭婦人經年荒,賣其子以活夫之兄子。武康有兄弟二人,妻各有孕,弟遠行未反,遇荒歲,不能兩全,棄其子而活弟子。嚴並褒薦之。又甄賞才能之
 士,論者美焉。五年,以疾去職,卒于家。



 三子:道民,宣城內史;靜民,散騎侍郎;福民,太子洗馬,皆為孫恩所害。



 群字敬林,嚴叔父也。有智局,志尚不羈。蘇峻入石頭,時匡術有寵於峻,賓從甚盛。群與從兄愉同行於橫塘,遇之,愉止與語,而群初不視術。術怒,欲刃之。愉下車抱術曰:「吾弟發狂,卿為我宥之。」乃獲免。後峻平,王導保存術,嘗因眾坐,令術勸群酒,以釋橫塘之憾。群答曰:「群非孔子,厄同匡人。雖陽和布氣,鷹化為鳩,至於識者,猶憎其目。」導有愧色。仕歷中丞。性嗜酒,導嘗戒之曰:「卿恒飲,不見酒家覆瓿布,日月久糜爛邪?」答曰:「公不見肉糟淹更
 堪久邪?」嘗與親友書云:「今年田得七百石秫米,不足了曲糵事。」其耽湎如此。卒於官。嗣子沉。



 沉字德度,有美名。何充薦沉於王導曰:「文思通敏,宜登宰門。」辟丞相司徒掾、瑯邪王文學,並不就。從兄坦以裘遺之,辭不受。坦曰:「晏平仲儉,祀其先人,豚肩不掩豆,猶狐裘數十年,卿復何辭!」於是受而服之。是時沉與魏顗、虞球、虞存、謝奉並為四族之俊。



 沉子廞,位至吳興太守、廷尉。廞子琳之,以草書擅名,又為吳興太守,侍中。



 丁潭,字世康,會稽山陰人也。祖固,吳司徒。父彌,梁州刺
 史。潭初為郡功曹,察孝廉,除郎中,稍遷丞相西閣祭酒。時元帝稱制,使各陳時事損益,潭上書曰:



 為國者恃人須才,蓋二千石長吏是也。安可不明簡其才,使必允當。既然得其人,使久於其職,在官者無茍且,居下者有恆心,此為政之較也。今之長吏,遷轉既數,有送迎之費。古人三載考績,三考黜陟,中才處局,故難以速成矣。



 夫兵所以防禦未然,鎮壓姦凶,周雖三聖,功成由武。今戎戰之世,益宜留心,簡選精銳,以備不虞。無事則優其身,有難則責其力。竊聞今之兵士,或私有役使,而營陳不充。夫為國者,由為家也。計財力之所任,審趨舍之舉動,不營
 難成之功,損棄分外之役。今兵人未彊,當審其宜,經塗遠舉,未獻大捷,更使力單財盡而威望挫弱也。



 及帝踐阼,拜駙馬都尉、奉朝請、尚書祠部郎。時瑯邪王裒始受封,帝欲引朝賢為其國上卿,將用潭,以問中書令賀循。循曰:「郎中令職望清重,實宜審授。潭清淳貞粹,雅有隱正,聖明所簡,才實宜之。」遂為瑯邪王郎中令。會裒薨,潭上疏求行終喪禮,曰:「在三之義,禮有達制,近代已來,或隨時降殺,宜一匡革,以敦于後,輒案令文,王侯之喪,官僚服斬,既葬而除。今國無繼統,喪庭無主,臣實陋賤,不足當重,謬荷首任,禮宜終喪。」詔下博議。國子祭酒杜夷
 議:「古者諒闇,三年不言。下及周世,稅衰效命。春秋之時,天子諸侯既葬而除。此所謂三代損益,禮有不同。故三年之喪,由此而廢。然則漢文之詔,合於隨時,凡有國者,皆宜同也,非唯施於帝皇而已。案禮,殤與無後,降於成人。有後,既葬而除。今不得以無後之故而獨不除也。愚以丁郎中應除衰麻,自宜主祭,以終三年。」太常賀循議:「禮,天子諸侯俱以至尊臨人,上下之義,群臣之禮,自古以來,其例一也。故禮盛則並全其重,禮殺則從其降。春秋之事,天子諸侯不行三年。至於臣為君服,亦宜以君為節,未有君除而臣服,君服而臣除者。今法令,諸侯卿
 相官屬為君斬衰,既葬而除。以令文言之,明諸侯不以三年之喪與天子同可知也。君若遂服,則臣子輕重無應除者也。若當皆除,無一人獨重之文。禮有攝主而無攝重,故大功之親主人喪者,必為之再祭練祥,以大功之服,主人三年喪者也。茍謂諸侯與天子同制,國有嗣王,自不全服,而人主居喪,素服主祭,三年不攝吉事,以尊令制。若當遠迹三代,令復舊典,不依法令者,則侯之服貴賤一例,亦不得唯一人論。」於是詔使除服,心喪三年。



 太興三年,遷王導驃騎司馬,轉中書郎,出為廣武將軍、東陽太守,以清潔見稱。徵為太子左衛率,不拜。成帝
 踐阼,以為散騎常侍、侍中。蘇峻作亂,帝蒙塵於石頭,唯潭及侍中鐘雅、劉超等隨從不離帝側。峻誅,以功賜爵永安伯,遷大尚書,徙廷尉,累遷左光祿大夫、領國子祭酒、本國大中正,加散騎常侍。



 康帝即位,屢表乞骸骨。詔以光祿大夫還第,門施行馬,祿秩一如舊制,給傳詔二人,賜錢二十萬,床帳褥席。年八十,卒。贈侍中,大夫如故,謚曰簡。王導嘗謂孔敬康有公才而無公望,丁世康有公望而無公才。子話,位至散騎侍郎。



 張茂,字偉康,少單貧,有志行,為鄉里所敬信。初起義兵,
 討賊陳斌,一郡用全。元帝辟為掾屬。官有老牛數十,將賣之,茂曰:「殺牛有禁,買者不得輒屠,齒力疲老,又不任耕駕,是以無用之物收百姓利也。」帝乃止。遷太子右衛率,出補吳興內史。沈充之反也,茂與三子並遇害。茂弟盎,為周札將軍,充討札,盎又死之。贈茂太僕。茂少時夢得大象,以問占夢萬推。推曰:「君當為大郡,而不善也。」問其故,推曰:「象者大獸,獸者守也,故知當得大郡。然象以齒焚,為人所害。」果如其言。



 陶回,丹陽人也。祖基,吳交州刺史。父抗,太子中庶子。回
 辟司空府中軍、主簿,並不就。大將軍王敦命為參軍,轉州別駕。敦死,司徒王導引為從事中郎,遷司馬。蘇峻之役,回與孔坦言於導,請早出兵守江口,語在坦傳。峻將至,回復謂亮曰:「峻知石頭有重戍,不敢直下,必向小丹陽南道步來,宜伏兵要之,可一戰而擒。」亮不從。峻果由小丹陽經秣陵,迷失道,逢郡人,執以為鄉導。時峻夜行,甚無部分。亮聞之,深悔不從回等之言。尋王師敗績,回還本縣,收合義軍,得千餘人,並為步軍,與陶侃、溫嶠等并力攻峻,又別破韓晁,以功封康樂伯。



 時大賊新平,綱維弛廢,司徒王導以回有器幹,擢補北軍中候,俄轉中
 護軍。久之,遷征虜將軍、吳興太守。時人飢穀貴,三吳尤甚。詔欲聽相鬻賣,以拯一時之急。回上疏曰:「當今天下不普荒儉,唯獨東土穀價偏貴,便相鬻賣,聲必遠流,北賊聞之,將窺疆場。如愚臣意,不如開倉廩以振之。」乃不待報,輒便開倉,及割府郡軍資數萬斛米以救乏絕,由是一境獲全。既而下詔,并敕會稽、吳郡依回振恤,二郡賴之。在郡四年,徵拜領軍將軍,加散騎常侍,征虜將軍如故。



 回性雅正,不憚彊禦。丹陽尹桓景佞事王導,甚為導所暱。回常慷慨謂景非正人,不宜親狎。會熒惑守南斗經旬,導語回曰:「南斗,揚州分,而熒惑守之,吾當遜位
 以厭此謫。」回答曰:「公以明德作相,輔弼聖主,當親忠貞,遠邪佞,而與桓景造膝,熒惑何由退舍!」導深愧之咸和二年,以疾辭職,帝不許。徙護軍將軍,常侍、領軍如故,未拜,卒,年五十一。謚曰威。



 四子:汪、陋、隱、無忌。汪嗣爵,位至輔國將軍、宣城內史,陋冠軍將軍,隱少府,無忌光祿勳,兄弟咸有於用。



 史臣曰:孔愉父子暨丁潭等,咸以筱簜之材,邀締構之運,策名霸府,騁足高衢,歷試清階,遂登顯要,外宣政績,內盡謀猷,罄心力以佐時,竭股肱以衛主,並能保全名節,善始令終。而愉高謝百萬之貲,辭榮數畝之宅,弘止
 足之分,有廉讓之風者矣。陶回陳邪佞之宜遠,明鬻賣之非宜,並補闕弼違,良可稱也。



 贊曰:愉既公才,潭唯公望。領軍儒雅,平越忠亮。君平料敵,彭祖弘益。茂以象焚,群由匡厄。陶回規過,言同金石。



\end{pinyinscope}