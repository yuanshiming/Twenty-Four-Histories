\article{列傳第六十九 桓玄卞範之段仲文}

\begin{pinyinscope}

 桓玄卞範之段仲文



 桓玄,字敬道,一名靈寶,大司馬溫之孽子也。其母馬氏嘗與同輩夜坐,於月下見流星墜銅盆水中,忽如二寸火珠,冏然明凈,競以瓢接取,馬氏得而吞之,若有感,遂有娠。及生玄,有光照室,占者奇之,故小名靈寶。妳媼每抱詣溫,輒易人而後至,云其重兼常兒,溫甚愛異之。臨終,命以為嗣,襲爵南郡公。年七歲,溫服終,府州文武辭
 其叔父沖,沖撫玄頭曰:「此汝家之故吏也。」玄因涕淚覆面,眾並異之。及長,形貌瑰奇,風神疏朗,博綜藝術,善屬文。常負其才地,以雄豪自處,眾咸憚之,朝廷亦疑而未用。年二十三,始拜太子洗馬,時議謂溫有不臣之跡,故折玄兄弟而為素官。



 太元末,出補義興太守,鬱鬱不得志。嘗登高望震澤,歎曰:「父為九州伯,兒為五湖長!」棄官歸國。自以元勳之門而負謗於世,乃上疏曰:



 臣聞周公大聖而四國流言,樂毅王佐而被謗騎劫,《巷伯》有豺獸之慨,蘇公興飄風之刺,惡直醜正,何代無之!先臣蒙國殊遇,姻婭皇極,常欲以身報德,投袂乘機,西平巴蜀,北
 清伊洛,使竊號之寇繫頸北闕,園陵修復,大恥載雪,飲馬灞滻懸旌趙魏,勤王之師,功非一捷。太和之末,皇基有潛移之懼,遂乃奉順天人,翼登聖朝,明離既朗,四凶兼澄。向使此功不建,此事不成,宗廟之事豈可孰念!昔太甲雖迷,商祚無憂;昌邑雖昏,弊無三孽。因茲而言,晉室之機危於殷漢,先臣之功高於伊霍矣。而負重既往,蒙謗清時,聖世明王黜陟之道,不聞廢忽顯明之功,探射冥冥之心,啟嫌謗之塗,開邪枉之路者也。先臣勤王艱難之勞,匡復剋平之勳,朝廷若其遺之,臣亦不復計也。至於先帝龍飛九五,陛下之所以繼明南面,請問談
 者,誰之由邪?誰之德邪?豈惟晉室永安,祖宗血食,於陛下一門,實奇功也。



 自頃權門日盛,醜政實繁,咸稱述時旨,互相扇附,以臣之兄弟皆晉之罪人,臣等復何理可以茍存聖世?何顏可以尸饗封祿?若陛下忘先臣大造之功,信貝錦萋菲之說,臣等自當奉還三封,受戮市朝,然後下從先臣,歸先帝於玄宮耳。若陛下述遵先旨,追錄舊勳,竊望少垂愷悌覆蓋之恩。



 疏寢不報。



 玄在荊楚積年,優游無事,荊州刺史殷仲堪甚敬憚之。及中書令王國寶用事,謀削弱方鎮,內外騷動,知王恭有憂國之言,玄潛有意於功業,乃說仲堪曰:「國寶與君諸人素已
 為對,唯患相弊之不速耳。今既執權要,與王緒相為表裏,其所迴易,罔不如志。孝伯居元舅之地,正情為朝野所重,必未便動之,唯當以君為事首。君為先帝所拔,超居方任,人情未以為允,咸謂君雖有思致,非方伯人。若發詔徵君為中書令,用殷顗為荊州,君何以處之?」仲堪曰:「憂之久矣,君謂計將安出?」玄曰:「國寶姦兇,天下所知,孝伯疾惡之情每至而當,今日之會,以理推之,必當過人。君若密遣一人,信說王恭,宜興晉陽之師,以內匡朝廷,己當悉荊楚之眾順流而下,推王為盟主,僕等亦皆投袂,當此無不響應。此事既行,桓文之舉也。」仲堪持疑
 未決。俄而王恭信至,招仲堪及玄匡正朝廷。國寶既死,於是兵罷。玄乃求為廣州,會稽王道子亦憚之,不欲使在荊楚,故順其意。



 隆安初,詔以玄督交廣二州、建威將軍、平越中郎將、廣州刺史、假節,玄受命不行。其年,王恭又與庾楷起兵討江州刺史王愉及譙王尚之兄弟。玄、仲堪謂恭事必剋捷,一時響應。仲堪給玄五千人,與楊佺期俱為前鋒。軍至湓口,王愉奔于臨川,玄遣偏將軍追獲之。玄、佺期至石頭,仲堪至蕪湖。恭將劉牢之背恭歸順。恭既死,庾楷戰敗,奔于玄軍。既而詔以玄為江州,仲堪等仲皆被換易,乃各回舟西還,屯于尋陽,共相結約,
 推玄為盟主。玄始得志,乃連名上疏申理王恭,求誅尚之、牢之等。朝廷深憚之,乃免桓脩、復仲堪以相和解。



 初,玄在荊州豪縱,士庶憚之,甚於州牧。仲堪親黨勸殺之,仲堪不聽。及還尋陽,資其聲地,故推為盟主,玄逾自矜重。佺期為人驕悍,常自謂承藉華胄,江表莫比,而玄每以寒士裁之,佺期甚憾,即欲於壇所襲玄,仲堪惡佺期兄弟虓勇,恐剋玄之後復為己害,苦禁之。於是各奉詔還鎮。玄亦知佺期有異謀,潛有吞并之計,於是屯于夏口。



 隆安中,詔加玄都督荊州四郡,以兄偉為輔國將軍、南蠻校尉。仲堪慮玄跋扈,遂與佺期結婚為援。初,玄既
 與仲堪、佺期有隙,恒臣掩襲,求廣其所統。朝廷亦欲成其釁隙,故分佺斯所督四郡與玄,佺期甚忿懼。會姚興侵洛陽,佺期乃建牙,聲云援洛,密欲與仲堪共襲玄。仲堪雖外結佺期而疑其心,距而不許,猶慮弗能禁,復遣從弟遹屯于北境以遏佺期。佺期既不能獨舉,且不測仲堪本意,遂息甲。南蠻校尉楊廣,佺期之兄也,欲距桓偉,仲堪不聽,乃出廣為宜都、建平二郡太守,加征虜將軍。佺期弟孜敬先為江夏相,玄以兵襲而召之。既至,以為諮議參軍。玄於是興軍西征,亦聲云救洛,與仲堪書,說佺期受國恩而棄山陵,宜共罪之。今親率戎旅,逕造
 金墉,使仲堪收楊廣,如其不爾,無以相信。仲堪本計欲兩全之,既得玄書,知不能禁,乃曰:「君自沔而行,不得一人入江也。」玄乃止。



 後荊州大水,仲堪振恤饑者,倉廩空竭。玄乘其虛而伐之,先遣軍襲巴陵。梁州刺史郭銓當之所鎮,路經夏口,玄聲云朝廷遣銓為己前鋒,乃授以江夏之眾,使督諸軍並進,密報兄偉令為內應。偉遑遽不知所為,乃自齎疏示仲堪。仲堪執偉為質,令與玄書,辭甚苦至。玄曰:「仲堪為人不得專決,常懷成敗之計,為兒子作慮,我兄必無憂矣。」



 玄既至巴陵,仲堪遣眾距之,為玄所敗。玄進至楊口,又敗仲堪弟子道護,乘勝至
 零口,去江陵二十里,仲堪遣軍數道距之。佺期自襄陽來赴,與兄廣共擊玄,玄懼其銳,乃退軍馬頭。佺期等方復追玄苦戰,佺期敗,走還襄陽,仲堪出奔酂城,玄遣將軍馮該躡佺期,獲之。廣為人所縛,送玄,並殺之。仲堪聞佺期死,乃將數百人奔姚興,至冠軍城,為該所得,玄令害之。



 於是遂平荊雍,乃表求領江、荊二州。詔以玄都督荊司雍秦梁益寧七州、後將軍、荊州刺史、假節,以桓脩為江州刺史。玄上疏固爭江州,於是進督八州及楊豫八郡,復領江州刺史。玄又輒以偉為冠軍將軍、雍州刺史。時寇賊未平,朝廷難違其意,許之。玄於是樹用腹心,
 兵馬日盛,屢上疏求討孫恩,詔輒不許。其後恩逼京都,玄建牙聚眾,外託勤王,實欲觀釁而進,復上疏請討之。會恩已走,玄又奉詔解嚴。以偉為江州,鎮夏口;司馬刁暢為輔國將軍,督八郡,鎮襄陽;遣桓振、皇甫敷、馮該等戍湓口。移沮漳蠻二千戶于江南,立武寧郡;更招集流人,立綏安郡。又置諸郡丞。詔徵廣州刺史刁逵、豫章太守郭昶之,玄皆留不遣。自謂三分有二,知勢運所歸,屢上禎祥以為己瑞。



 初。庾楷既奔於玄,玄之求討孫恩也,以為右將軍。玄既解嚴,楷亦去職。楷以玄方與朝廷構怨,恐事不剋,禍及於己,乃密結於後將軍元顯,許為內
 應。元興初,元顯稱詔伐玄,玄從兄石生時為太傅長史,密書報玄。玄本謂揚土饑饉,孫恩未滅,必未遑討己,可得蓄力養眾,觀釁而動。既聞元顯將伐之,甚懼,欲保江陵。長史卞範之說玄曰:「公英略威名振于天下,元顯口尚乳臭,劉牢之大失物情,若兵臨近畿,示以威賞,則土崩之勢可翹足而待,何有延敵入境自取蹙弱者乎!」玄大悅,乃留其兄偉守江陵,抗表率眾,下至尋陽,移檄京邑,罪狀元顯。檄至。元顯大懼,下船而不克發。玄既失人情,而興師犯順,慮眾不為用,恒有迴旆之計。既過尋陽,不見王師,意甚悅,其將吏亦振。庾楷謀泄,收縶之。至姑
 孰,使其將馮該、苻宏、皇甫敷、索元等先攻譙王尚之。尚之敗。劉牢之遣子敬宣詣玄降。



 玄至新亭,元顯自潰。玄入京師,矯詔曰:「義旗雲集,罪在元顯。太傅已別有教,其解嚴息甲,以副義心。」又矯詔加己總百揆,侍中、都督中外諸軍事、丞相、錄尚書事、揚州牧,領徐州刺史,又加假黃鉞、羽葆鼓吹、班劍二十人,置左右長史、司馬、從事中郎四人,甲杖二百人上殿。玄表列太傅道子及元顯之惡,徙道子于安成郡,害元顯于市。於是玄入居太傅府,害太傅中郎毛泰、泰弟游擊將軍邃,太傅參軍荀遜、前豫州刺史庾楷父子、吏部郎袁遵、譙王尚之等,流尚之
 弟丹陽尹恢之、廣晉伯允之、驃騎長史王誕、太傅主簿毛遁等於交廣諸郡,尋追害恢之、允之于道。以兄偉為安西將軍、荊州刺史,領南蠻校尉,從兄謙為左僕射、加中軍將軍、領選,脩為右將軍、徐兗二州刺史,石生為前將軍、江州刺史,長史為卞範之為建武將軍、丹陽尹、王謐為中書令、領軍將軍。大赦,改元為大亨。玄讓丞相,自署太尉、領平西將軍、豫州刺史。又加袞冕之服,綠糸戾綬,增班劍為六十人,劍履上殿,入朝不趨,贊奏不名。



 玄將出居姑孰,訪之於眾,王謐對曰:「《公羊》有言,周公何以不之魯?欲天下一乎周也。願靜根本,以公旦為心。」玄善其對
 而不能從。遂大築城府,臺館山池莫不壯麗,乃出鎮焉。既至姑孰,固辭錄尚書事,詔許之,而大政皆諮焉,小事則決於桓謙、卞範之。



 自禍難屢構,干戈不戢,百姓厭之。思歸一統。及玄初至也,黜凡佞,擢俊賢,君子之道粗備,京師欣然。後乃陵侮朝廷,幽擯宰輔,豪奢縱欲,眾務繁興,於是朝野失望,人不安業。時會稽饑荒,玄令賑貸之。百姓散在江湖採穭,內史王愉悉召之還。請米,米既不多,吏不時給,頓仆道路死者十八九焉。玄又害吳興太守高素、輔國將軍竺謙之、謙之從兄高平相朗之、輔國將軍劉襲、襲弟彭城內史季武、冠軍將軍孫無終等,皆
 牢之之黨,北府舊將也。襲兄冀州刺史軌及寧朔將軍高雅之、牢之子敬宣並奔慕容德。玄諷朝廷以己平元顯功,封豫章公,食安成郡地方二百二十五里,邑七千五百戶;平仲堪、佺期功,封桂陽郡公,地方七十五里,邑二千五百戶;本封南郡如故。玄以豫章改封息昇,桂陽郡公賜兄子浚,降為西道縣公。又發詔為桓溫諱,有姓名同者一皆改之,贈其母馬氏豫章公太夫人。元興二年,玄詐表請平姚興,又諷朝廷作詔,不許。玄本無資力,而好為大言,既不克行,乃云奉詔故止。初欲飾裝,無他處分,先使作輕舸,載服玩及書畫等物。或諫之,玄曰:「
 書畫服玩既宜恒在左右,且兵凶戰危,脫有不意,當使輕而易運。」眾咸笑之。



 是歲,玄兄偉卒,贈開府、驃騎將軍,以桓脩代之。從事中郎曹靖之說玄以桓脩兄弟職居內外,恐權傾天下,玄納之,乃以南郡相桓石康為西中郎將、荊州刺史。偉服始以公除,玄便作樂。初奏,玄撫節慟哭,既而收淚盡懽,玄所親仗唯偉,偉既死,玄乃孤危。而不臣之迹已著,自知怨滿天下,欲速定篡逆,殷仲文、卞範之等又共催促之,於是先改授群司,解瑯邪王司徒,遷太宰,加殊禮,以桓謙為侍中、衛將軍、開府、錄尚書事,王謐散騎常侍、中書監,領司徒,桓胤中書令,加桓脩
 散騎常侍、撫軍大將軍。置學官,教授二品子弟數百人。又矯詔加其相國,總百揆,封南郡、南平、宜都、天門、零陵、營陽、桂陽、衡陽、義陽、建平十郡為楚王,揚州牧,領平西將軍、豫州刺史如故,加九錫備物,楚國置丞相已下,一遵舊典。又諷天子御前殿而策授焉。玄屢偽讓,詔遣百僚敦勸,又云:「當親降鑾輿乃受命。」矯詔贈父溫為楚王,南康公主為楚王后。以平西長史劉瑾為尚書,刁逵為中領軍,王嘏為太常,殷叔文為左衛,皇甫敷為右衛,凡眾官合六十餘人,為楚官屬。玄解平西、豫州,以平西文武配相國府。



 新野人庾仄聞玄受九錫,乃起義兵,襲馮
 該於襄陽,走之。仄有眾七千,於城南設壇,祭祖宗七廟。南蠻參軍庾彬、安西參軍楊道護、江安令鄧襄子謀為內應。仄本仲堪黨,桓偉既死,石康未至,故乘間而發,江陵震動。桓濟之子亮起兵于羅縣,自號平南將軍、湘州刺史,以討仄為名。南蠻校尉羊僧壽與石康共攻襄陽,仄眾散,奔姚興,彬等皆遇害。長沙相陶延壽以亮乘亂起兵,遣收之。玄徙亮于衡陽,誅其同謀桓奧等。



 玄偽上表求歸籓,又自作詔留之,遣使宣旨,玄又上表固請,又諷天子作手詔固留焉。玄好逞偽辭,塵穢簡牘,皆此類也。謂代謝之際宜有禎祥,乃密令所在上臨平湖開除
 清朗,使眾官集賀。矯詔曰:「靈瑞之事非所敢聞也。斯誠相國至德,故事為之應。太平之化,於是乎始,六合同悅,情何可言!」又詐云江州甘露降王成基家竹上。玄以歷代咸有肥遁之士,而己世獨無,乃徵皇甫謐六世孫希之為著作,并給其資用,皆令讓而不受,號曰高士,時人名為「充隱」。議復肉刑,斷錢貨,迴復改異,造革紛紜,志無一定,條制森然,動害政理。性貪鄙,好奇異,尤愛寶物,珠玉不離于手。人士有法書好畫及佳園宅者,悉欲歸己,猶難逼奪之,皆蒱博而取。遣臣佐四出,掘果移竹,不遠數千里,百姓佳果美竹無復遺餘。信悅諂譽,逆忤讜言,
 或奪其所憎與其所愛。



 十一月,玄矯制加其冕十有二旒,建天子旌旗,出警入蹕,乘金根車,駕六馬,備五時副車,置旄頭雲罕,樂儛八佾,設鐘虡宮縣,妃為王后,世子為太子,其女及孫爵命之號皆如舊制。玄乃多斥朝臣為太宰僚佐,又矯詔使王謐兼太保,領司徒,奉皇帝璽禪位於己。又諷帝以禪位告廟,出居永安宮,移晉神主于瑯邪廟。



 初,玄恐帝不肯為手詔,又慮璽不可得,逼臨川王寶請帝自為手詔,因奪取璽。比臨軒,璽已久出,玄甚喜。百官到姑孰勸玄僭偽位,玄偽讓,朝臣固請,玄乃於城南七里立郊,登壇篡位,以玄牡告天,百僚陪列,而
 儀注不備,忘稱萬歲,又不易帝諱。榜為文告天皇后帝云:「晉帝欽若景運,敬順明命,以命于玄。夫天工人代,帝王所以興,匪君莫治,惟德司其元,故承天理物,必由一統。並聖不可以二君,非賢不可以無主,故世換五帝,鼎遷三代。爰暨漢魏,咸歸勳烈。晉自中葉,仍世多故,海西之亂,皇祚殆移,九代廓寧之功,升明黜陟之勳,微禹之德,左衽將及。太元之末,君子道消,積釁基亂。鐘于隆安,禍延士庶,理絕人倫。玄雖身在草澤,見棄時班,義情理感,胡能無慨!投袂剋清之勞,阿衡撥亂之績,皆仰憑先德遺愛之利,玄何功焉!屬當理運之會,猥集樂推之數,
 以寡昧之身踵下武之重,膺革泰之始,託王公之上,誠仰藉洪基,德漸有由。夕惕祗懷,罔知攸厝。君位不可以久虛,人神不可以乏饗,是用敢不奉以欽恭大禮,敬簡良辰,升壇受禪,告類上帝,以永綏眾望,式孚萬邦,惟明靈是饗。」乃下書曰:「夫三才相資,天人所以成功,理由一統,貞夫所以司契,帝王之興,其源深矣。自三五已降,世代參差,雖所由或殊,其歸一也。朕皇考宣武王聖德高邈,誕啟洪基,景命攸歸,理貫自昔。中間屯險,弗克負荷,仰瞻宏業,殆若綴旒。藉否終之運,遇時來之會,用獲除姦救溺,拯拔人倫。晉氏以多難薦臻,歷數唯既,典章唐
 虞之準,述遵漢魏之則,用集天祿於朕躬。惟德不敏,辭不獲命,稽若令典,遂升壇燎于南郊,受終于文祖。思覃斯慶,願與億兆聿茲更始。」於是大赦,改元永始,賜天下爵二級,孝悌力田人三級,鰥寡孤獨不能自存者穀人五斛。其賞賜之制,徒設空文,無其實也。初出偽詔,改年為建始,右丞王悠之曰:「建始,趙王倫偽號也。」又改為永始,復是王莽始執權之歲,其兆號不祥,冥符僭逆如此。又下書曰:「夫三恪作賓,有自來矣。爰暨漢魏,咸建疆宇。晉氏欽若歷數,禪位于朕躬,宜則是古訓,授茲茅土。以南康之平固縣奉晉帝為平固王,車旗正朔一如舊典。」
 遷帝居尋陽,即陳留王處鄴宮故事。降永安皇后為零陵君,瑯邪王為石陽縣公,武陵王遵為彭澤縣侯。追尊其父溫宣武皇帝,廟稱太廟,南康公主為宣皇后。封子昇為豫章郡王,叔父雲孫放之為寧都縣王,豁孫稚玉為臨沅縣王,豁次子石康為右將軍、武陵郡王,祕子蔚為醴陵縣王,贈沖太傳、宣城郡王,加殊禮,依晉安平王故事,以孫胤襲爵,為吏部尚書,沖次子謙為揚州刺史、新安郡王,謙弟脩為撫軍大將軍、安成郡王,兄歆臨賀縣王,禕富陽縣王,贈偉侍中、大將軍、義興郡王,以子濬襲爵,為輔國將軍,浚弟邈西昌縣王。封王謐為武昌公,
 班劍二十人,卞範之為臨汝公,殷仲文為東興公,馮該為魚復侯。又降始安郡公為縣公,長沙為臨湘縣公,盧陵為巴丘縣公,各千戶。其康樂、武昌、南昌、望蔡、建興、永脩、觀陽皆降封百戶,公侯之號如故。又普進諸征鎮軍號各有差。以相國左長史王綏為中書令。崇桓謙母庾氏為宣城太妃,加殊禮,給以輦乘。號溫墓曰永崇陵,置守衛四十人。



 玄入建康宮,逆風迅激,旍旗儀飾皆傾偃。及小會于西堂,設妓樂,殿上施絳綾帳,縷黃金為顏,四角作金龍,頭銜五色羽葆旒蘇,群臣竊相謂曰:「此頗似輀車,亦王莽仙蓋之流也。龍角,所謂亢龍有悔者也。」又
 造金根車,駕六馬。是月,玄臨聽訟觀閱囚徒,罪無輕重,多被原放。有乾輿乞者,時或恤之。其好行小惠如此。自以水德,壬辰,臘於祖。改尚書都官郎為賊曹,又增置五校、三將及彊弩、積射武衛官。元興三年,玄之永始二年也,尚書答「春蒐」字誤為「春菟」,凡所關署皆被降黜。玄大綱不理,而糾摘纖微,皆此類也。以其妻劉氏為皇后,將脩殿宇,乃移入東宮。又開東掖、平昌、廣莫及宮殿諸門,皆為三道。更造大輦,容三千人坐,以二百人舁之。性好畋遊,以體大不堪乘馬,又作徘徊輿,施轉關,令迴動無滯。既不追尊祖曾,疑其禮義,問於群臣。散騎常侍徐廣
 據晉典宜追立七廟,又敬其父則子悅,位彌高者情理得申,道愈廣者納敬必普也。玄曰:「《禮》云三昭、三穆,與太祖為七,然則太祖必居廟之主也,昭穆皆自下之稱,則非逆數可知也。禮,太祖東向,左昭右穆。如晉室之廟,則宣帝在昭穆之列,不得在太祖之位。昭穆既錯,太祖無寄,失之遠矣。」玄曾祖以上名位不顯,故不欲序列,且以王莽九廟見譏於前史,遂以一廟矯之,郊廟齋二日而已。祕書監卞承之曰:「祭不及祖,知楚德之不長也。」又毀晉小廟以廣臺榭。其庶母蒸嘗,靡有定所,忌日見賓客遊宴,唯至亡時一哭而已。期服之內,不廢音樂。玄出遊
 水門,飄風飛其儀蓋。夜,濤水入石頭,大桁流壞,殺人甚多。大風吹朱雀門樓,上層墜地。



 玄自篡盜之後,驕奢荒侈,遊獵無度,以夜繼晝。兄偉葬日,旦哭晚遊,或一日之中屢出馳騁。性又急暴,呼召嚴速,直官咸系馬省前,禁內嘩雜,無復朝廷之體。於是百姓疲苦,朝野勞瘁,怨怒思亂者十室八九焉。於是劉裕、劉毅、何無忌等共謀興復。裕等斬桓脩於京口,斬桓弘于廣陵,河內太守辛扈興、弘農太守王元德、振威將軍童厚之、竟陵太守劉邁謀為內應。至期,裕遣周安穆報之,而邁惶遽,遂以告玄。玄震駭,即殺扈興等,安穆馳去得免。封邁重安侯,一宿
 又殺之。



 裕率義軍至竹里,玄移還上宮,百僚步從,召侍官皆入止省中。赦揚、豫、徐、兗、青、冀六州,加桓謙征討都督、假節,以殷仲文代桓脩,遣頓丘太守吳甫之、右衛將軍皇甫敷北距義軍。裕等於江乘與戰,臨陣斬甫之,進至羅落橋,與敷戰,復梟其首。玄聞之大懼,乃召諸道術人推算數為厭勝之法,乃問眾曰:「朕其敗乎?」曹靖之對曰:「神怒人怨,臣實懼焉。」玄曰:「人或可怨,神何為怒?」對曰:「移晉宗廟,飄泊失所,大楚之祭,不及於祖,此其所以怒也。」玄曰:「卿何不諫?」對曰:「輦上諸君子皆以為堯舜之世,臣何敢言!」玄愈忿懼,使桓謙、何澹之屯東陵,卞範之屯
 覆舟山西,眾合二萬,以距義軍。裕至蔣山,使羸弱貫油帔登山,分張旗幟,數道並前。玄偵侯還云:「裕軍四塞,不知多少。」玄益憂惶,遣武衛將軍庾頤之配以精卒,副援諸軍。于時東北風急,義軍放火,煙塵張天,鼓噪之音震駭京邑。劉裕執鉞麾而進,謙等諸軍一時奔潰。玄率親信數千人聲言赴戰,遂將其子升、兄子濬出南掖門,西至石頭,使殷仲文具船,相與南奔。



 初,玄在姑孰,將相星屢有變;篡位之夕,月及太白,又入羽林,玄甚惡之。及敗走,腹心勸其戰,玄不暇答,直以策指天。而經日不得食,左右進以粗飯,咽不能下。昇時年數歲,抱玄胸而撫之,
 玄悲不自勝。



 劉裕以武陵王遵攝萬機,立行臺,總百官。遣劉毅、劉道規躡玄,誅玄諸兄子及石康兄權、振兄洪等。



 玄至尋陽,江州刺史郭昶之給其器用兵力。殷仲文自後至,望見玄舟,旌旗輿服備帝者之儀,歎息曰:「敗中復振,故可也。」玄於是逼乘輿西上。桓歆聚黨向歷陽,宣城內史諸葛長民擊破之。玄於道作起居注,敘其距義軍之事,自謂經略指授,算無遺策,諸將違節度,以致虧喪,非戰之罪。於是不遑與群下謀議,唯耽思誦述,宣示遠近。玄至江陵,石康納之,張幔屋於城南,署置百官,以卞範之為尚書僕射,其餘職多用輕資。於是大修舟師,
 曾未三旬,眾且二萬,樓船器械甚盛。謂其群黨曰:「卿等並清塗翼從朕躬,都下竊位者方應謝罪軍門,其觀卿等入石頭,無異雲霄中人也。」



 玄以奔敗之後,懼法令不肅,遂輕怒妄殺,人多離怨。殷仲文諫曰:「陛下少播英譽,遠近所服,遂掃平荊雍,一匡京室,聲被八荒矣。既據有極位,而遇此圮運,非為威不足也。百姓喁喁,想望皇澤,宜弘仁風,以收物情。」玄怒曰:「漢高、魏武幾遇敗,但諸將失利耳!以天文惡,故還都舊楚,而群小愚惑,妄生是非,方當糾之以猛,未宜施之以恩也。」玄左右稱玄為「桓詔」,桓胤諫曰:「詔者,施於辭令,不以為稱謂也。漢魏之主皆
 無此言,唯聞北虜以苻堅為『苻詔』耳。願陛下稽古帝則,令萬世可法。」玄曰:「此事已行,今宣敕罷之,更為不祥。必其宜革,可待事平也。」荊州郡守以玄播越,或遣使通表,有匪寧之辭,玄悉不受,仍乃更令所在表賀遷都。



 玄遣游擊將軍何澹之、武衛將軍庾稚祖、江夏太守桓道恭就郭銓以數千人守湓口。又遣輔國將軍桓振往義陽聚眾,至弋陽,為龍驤將軍胡譁所破,振單騎走還。何無忌、劉道規等破郭銓、何澹之、郭昶之於桑落洲,進師尋陽。玄率舟艦二百發江陵,使苻宏、羊僧壽為前鋒。以鄱陽太守徐放為散騎常侍,欲遣說解義軍,謂放曰:「諸人
 不識天命,致此妄作,遂懼禍屯結,不能自反。卿三州所信,可明示朕心,若退軍散甲,當與之更始,各授位任,令不失分。江水在此,朕不食言。」放對曰:「劉裕為唱端之主,劉毅兄為陛下所誅,並不可說也。輒當申聖旨於何無忌。」玄曰:「卿使若有功,當以吳興相敘。」放遂受使,入無忌軍。



 魏詠之破桓歆于歷陽,諸葛長民又敗歆于芍陂,歆單馬渡淮。毅率道規及下邳太守孟懷玉與玄戰於崢嶸洲。於時義軍數千,玄兵甚盛,而玄懼有敗衄,常漾輕舸於舫側,故其眾莫有鬥心。義軍乘風縱火,盡銳爭先,玄眾大潰,燒輜重夜遁,郭銓歸降。玄故將劉統、馮稚等
 聚黨四百人,襲破尋陽城,毅遣建威將軍劉懷肅討平之。玄留永安皇后及皇后於巴陵。殷仲文時在玄艦,求出別船收集散軍,因叛玄,奉二后奔于夏口。玄入江陵城,馮該勸使更下戰,玄不從,欲出漢川,投梁州刺史桓希,而人情乖阻,制令不行。玄乘馬出城,至門,左右於闇中斫之,不中,前後相殺交橫,玄僅得至船。於是荊州別駕王康產奉帝入南郡府舍,太守王騰之率文武營衛。



 時益州刺史毛璩使其從孫祐之、參軍費恬送弟璠喪葬江陵,有眾二百,璩弟子脩之為玄屯騎校尉,誘玄以入蜀,玄從之。達枚回洲,恬與祐之迎擊玄,矢下如雨。玄
 嬖人丁仙期、萬蓋等以身敝玄,並中數十箭而死。玄被箭,其子昇輒拔去之。益州督護馮遷抽刀而前,玄拔頭上玉導與之,仍曰:「是何人邪?敢殺天子!」遷曰:「欲殺天子之賊耳。」遂斬之,時年三十六。又斬石康及濬等五級,庾頤之戰死。昇云:「我是豫章王,諸君勿見殺。」送至江陵市斬之。



 初,玄在宮中,恆覺不安,若為鬼神所擾,語其所親云:「恐己當死,故與時競。」元興中,衡陽有雌雞化為雄,八十日而冠萎。及玄建國於楚,衡陽屬焉,自篡盜至敗,時凡八旬矣。其時有童謠云:「長干巷,巷長干,今年殺郎君,後年斬諸桓。」其兇兆符會如此。郎君,謂元顯也。



 是月,
 王騰之奉帝入居太府。桓謙亦聚眾沮中,為玄舉哀,立喪庭,偽謚為武悼皇帝。毅等傳送玄首,梟于大桁,百姓觀者莫不欣幸。



 何無忌等攻桓謙于馬頭,桓蔚于龍洲,皆破之。義軍乘勝競進,振、該等距戰於靈溪,道規等敗績,死沒者千餘人。義軍退次尋陽,更繕舟甲。毛璩自領梁州,遣將攻漢中,殺桓希。江夏相張暢之、高平太守劉懷肅攻何澹之於西塞磯,破之。振遣桓蔚代王曠守襄陽。道規進討武昌,破偽太守王旻。魏詠之、劉籓破桓石綏於白茅。義軍發尋陽。桓亮自號江州刺史,侵豫章,江州刺史劉敬宣討走之。義軍進次夏口。偽鎮東將軍
 馮該等守夏口,揚武將軍孟山圖據魯城,輔國將軍桓山客守偃月壘。劉毅攻魯城,道規攻偃月壘,無忌與檀祗列艦中流,以防越逸。義軍騰赴,叫聲動山谷,自辰及午,二城俱潰,馮該散走,生擒山客。毅等平巴陵。毛璩遣涪陵太守文處茂東下,振遣桓放之為益州,屯夷陵,處茂距戰,放之敗走,還江陵。



 義熙元年正月,南陽太守魯宗之起義兵襲襄陽,破偽雍州刺史桓蔚。無忌諸軍次江陵之馬頭,振擁帝出營江津。魯宗之率眾於柞溪,破偽武賁中郎溫楷,進至紀南。振自擊宗之,宗之失利。時蜀軍據靈溪,毅率無忌、道規等破馮該軍,推鋒而前,即
 平江陵。振見火起,知城已陷,乃與謙等北走。是日,安帝反正。大赦天下,唯逆黨就戮,詔特免桓胤一人。桓亮自豫章,自號鎮南將軍、湘州刺史。苻宏寇安成、廬陵,劉敬宣遣將討之,宏走入湘中。二月,桓謙、何澹之、溫楷等奔于姚興。桓振與宏出自溳城,襲破江陵,劉懷肅自雲杜伐振等,破之。廣武將軍唐興斬振及偽輔國將軍桓珍,毅於臨鄣斬偽零陵太守劉叔祖。桓亮、苻宏復出冠湘中,害郡守長吏,檀祗討宏於湘東,斬之,廣武將軍郭彌斬亮於益陽,其餘擁眾假號皆討平之。詔徙桓胤及諸黨與於新安諸郡。



 三年,東陽太守殷仲文與永嘉太守
 駱球謀反,欲建桓胤為嗣,曹靖之、桓石松、卞承之、劉延祖等潛相交結,劉裕以次收斬之,并誅其家屬。後桓謙走入蜀,蜀賊譙縱以謙為荊州刺史,使率兵而下,荊楚之眾多應之。謙至枝江,荊州刺史劉道規斬之,梁州刺史傅歆又斬桓石綏,桓氏遂滅。



 卞範之字敬祖,濟陰宛句人也,識悟聰敏,見美於當世。太元中,自丹陽丞為始安太守。桓玄少與之遊,及玄為江州,引為長史,委以心膂之任,潛謀密計,莫不決之。後玄將為篡亂,以範之為丹陽尹。範之與殷仲文陰撰策
 命,進範之為征虜將軍、散騎常侍。玄僭位,以範之為侍中,班劍二十人,進號後將軍,封臨汝縣公。其禪詔,即範之文也。



 玄既奢侈無度,範之亦盛營館第。自以佐命元勛,深懷矜伐,以富貴驕人,子弟慠慢,眾咸畏嫉之。義軍起,範之屯兵於覆舟山西,為劉毅所敗,隨玄西走,玄又以範之為尚書僕射。玄為劉毅等所敗,左右分散,唯範之在側。玄平,斬於江陵。



 殷仲文,南蠻校尉覬之弟也。少有才藻,美容貌。從兄仲堪薦之於會稽王道子,即引為驃騎參軍,甚相賞待。俄
 轉諮議參軍,後為元顯征虜長史。會桓玄與朝廷有隙,玄之姊,仲文之妻,疑而間之,左遷新安太守。仲文於玄雖為姻親,而素不交密,及聞玄平京師,便棄郡投焉。玄甚悅之,以為諮議參軍。時王謐見禮而不親,卞範之被親而少禮,而寵遇隆重,兼於王、卞矣。玄將為亂,使總領詔命,以為侍中,領左衛將軍。玄九錫,仲文之辭也。



 初,玄篡位入宮,其床忽陷,群下失色,仲文曰:「將由聖德深厚,地不能載。」玄大悅。」以佐命親貴,厚自封崇,輿馬器服,窮極綺麗,後房伎妾數十,絲竹不絕音。性貪吝,多納貨賄,家累千金,常若不足。玄為劉裕所敗,隨玄西走,其珍寶
 玩好悉藏地中,皆變為土。至巴陵,因奉二后投義軍,而為鎮軍長史,轉尚書。



 帝初反正,抗表自解曰:「臣聞洪波振壑,川無恬鱗;驚飆拂野,林無靜柯。何者?勢弱則受制於巨力,質微則無以自保。於理雖可得而言,於臣實非所敢譬。昔桓玄之代,誠復驅逼者眾。至如微臣,罪實深矣,進不能見危授命,亡身殉國;退不能辭粟首陽,拂衣高謝。遂乃宴安昏寵,叨昧偽封,錫文篡事,曾無獨固。名義以之俱淪,情節自茲兼撓,宜其極法,以判忠邪。會鎮軍將軍劉裕匡復社稷,大弘善貸,佇一戮於微命,申三驅於大信,既惠之以首領,又申之以縶維。于時皇輿否
 隔,天人未泰,用忘進退,是以僶俯從事,自同令人。今宸極反正,唯新告始,憲章既明,品物思舊,臣亦胡顏之厚,可以顯居榮次!乞解所職,待罪私門。違離闕庭,乃心慕戀。」詔不許。



 仲文因月朔與眾至大司馬府,府中有老槐樹,顧之良久而歎曰:「此樹婆娑,無復生意!」仲文素有名望,自謂必當朝政,又謝混之徒疇昔所輕者,並皆比肩,常怏怏不得志。忽遷為東陽太守,意彌不平。劉毅愛才好士,深相禮接,臨當之郡,游宴彌日。行至富陽,慨然歎曰:「看此山川形勢,當復出一伯符。」何無忌甚慕之。東陽,無忌所統,仲文許當便道修謁,無忌故益飲遲之,令府
 中命文人殷闡、孔寧子之徒撰義構文,以俟其至。仲文失志恍惚,遂不過府。無忌疑其薄己,大怒,思中傷之。時屬慕容超南侵,無忌言於劉裕曰:「桓胤、殷仲文並乃腹心之疾,北虜不足為憂。」義熙三年,又以仲文與駱球等謀反,及其弟南蠻校尉叔文伏誅。仲文時照鏡不見其面,數日而遇禍。



 仲文善屬文,為世所重,謝靈運嘗云:「若殷仲文讀書半袁豹,則文才不減班固。」言其文多而見書少也。



 史臣曰:桓玄纂凶,父之餘基。挾姦回之本性,含怒於失職;苞藏其豕心,抗表以稱冤。登高以發憤,觀釁而動,竊
 圖非望。始則假寵於仲堪,俄而戮殷以逞欲,遂得據全楚之地,驅勁勇之兵,因晉政之陵遲,乘會稽之酗JT,縱其狙詐之計,扇其陵暴之心,敢率犬羊,稱兵內侮。天長喪亂,兇力實繁,踰年之間,奄傾晉祚,自謂法堯禪舜,改物君臨,鼎業方隆,卜年惟永。俄而義旗電發,忠勇雷奔,半辰而都邑廓清,踰月而兇渠即戮,更延墜歷,復振頹綱。是知神器不可以闇干,天祿不可以妄處者也。夫帝王者,功高宇內,道濟含靈,龍宮鳳歷表其祥,彤雲玄石呈其瑞,然後光臨大寶,克享鴻名,允彳奚后之心,副樂推之望。若桓玄之麼麼,豈足數哉!適所以干紀亂常,傾宗
 絕嗣,肇金行之禍難,成宋氏之驅除者乎!



 贊曰:靈寶隱賊,世載兇德。信順未孚,姦回是則。肆逆遷鼎,憑威縱慝。違天虐人,覆宗殄國。



\end{pinyinscope}