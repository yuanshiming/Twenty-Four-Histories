\article{列傳第六十七}

\begin{pinyinscope}

 四夷



 東夷夫餘國馬韓辰韓肅慎氏倭人裨離等十國



 西戎吐谷渾焉耆國龜茲國大宛國康居國大秦國



 南蠻林邑扶南



 北狄匈奴



 夫恢恢乾德,萬類之所資始;蕩蕩坤儀,九區之所均載。考羲軒於往統,肇承天而理物;訊炎昊於前辟,爰制地而疏疆。襲冠帶以辨諸華,限要荒以殊遐裔,區分中外,其來尚矣。九夷八狄,被青野而亙玄方;七戎六蠻,綿西宇而橫南極。繁種落,異君長,遇有道則時遵聲教,鐘無妄則爭肆虔劉,趨扇風塵,蓋其常性也。詳求遐議,歷選
 深謨,莫不待以羈縻,防其猾夏。



 武帝受終衰魏,廓境全吳,威略既申,招攜斯廣,迷亂華之議,矜來遠之名,撫舊懷新,歲時無怠,凡四夷入貢者,有二十三國。既而惠皇失德,中宗遷播,凶徒分據,天邑傾淪,朝化所覃,江外而已,賝貢之禮,於茲殆絕,殊風異俗,所未能詳。故採其可知者,為之傳云。北狄竊號中壤,備于載記;在其諸部種類,今略書之。



 東夷,夫餘國、馬韓、辰韓、肅慎氏、倭人、裨離等十國。



 夫餘國,在玄菟北千餘里,南接鮮卑,北有弱水,地方二
 千里,戶八萬,有城邑宮室,地宜五穀。其人強勇,會同揖讓之儀有似中國。其出使,乃衣錦罽,以金銀飾腰。其法,殺人者死,沒入其家;盜者一責十二;男女淫,婦人妒,皆殺之。若有軍事,殺牛祭天,以其蹄占吉凶,蹄解者為凶,合者為吉。死者以生人殉葬,有槨無棺。其居喪,男女皆衣純白,婦人著布面衣,去玉佩。出善馬及貂豽、美珠,珠大如酸棗。其國殷富,自先世以來,未嘗被破。其王印文稱「穢王之印」。國中有古穢城,本穢貃之城也。



 武帝時,頻來朝貢,至太康六年,為慕容廆所襲破,其王依慮自殺,子弟走保沃沮。帝為下詔曰:「夫餘王世守忠孝,為惡虜
 所滅,其愍念之。若其遺類足以復國者,當為之方計,使得存立。」有司奏護東夷校尉鮮於嬰不救夫餘,失於機略。詔免嬰,以何龕代之。明年,夫餘後王依羅遣詣龕,求率見人還復舊國。仍請援。龕上列,遣督郵賈沈以兵送之。廆又要之於路,沈與戰,大敗之,廆眾退,羅得復國。爾後每為廆掠其種人,賣於中國。帝愍之,又發詔以官物贖還,下司、冀二州,禁市夫餘之口。



 韓種有三:一曰馬韓,二曰辰韓,三曰弁韓。辰韓在帶方南,東西以海為限。



 馬韓居山海之間,無城郭,凡有小國
 五十六所,大者萬戶,小者數千家,各有渠帥。俗少綱紀,無跪拜之禮。居處作土室,形如冢,其戶向上,舉家共在其中,無長幼男女之別。不知乘牛馬,畜者但以送葬。俗不重金銀錦罽,而貴瓔珠,用以綴衣或飾髮垂耳。其男子科頭露紒,衣布袍,履草蹻,性勇悍。國中有所調役,及起築城隍,年少勇健者皆鑿其背皮,貫以大繩,以杖搖繩,終日歡呼力作,不以為痛。善用弓楯矛櫓,雖有鬥爭攻戰,而貴相屈服。俗信鬼神,常以五月耕種畢,群聚歌舞以祭神;至十月農事畢,亦如之。國邑各立一人主祭天神,謂為天君。又置別邑,名曰蘇塗,立大木,懸鈴鼓。其
 蘇塗之義,有似西域浮屠也,而所行善惡有異。



 武帝太康元年、二年,其主頻遣使入貢方物,七年、八年、十年,又頻至。太熙元年,詣東夷校尉何龕上獻。咸寧三年復來,明年又請內附。



 辰韓在馬韓之東,自言秦之亡人避役入韓,韓割東界以居之,立城柵,言語有類秦人,由是或謂之為秦韓。初有六國,後稍分為十二,又有弁辰,亦十二國,合四五萬戶,各有渠帥,皆屬於辰韓。辰韓常用馬韓人作主,雖世世相承,而不得自立,明其流移之人,故為馬韓所制也。
 地宜五穀,俗饒蠶桑,善作縑布,服牛乘馬。其風俗可類馬韓,兵器亦與之同。初生子,便以石押其頭使扁。喜舞,善彈瑟,瑟形似築。



 武帝太康元年,其王遣使獻方物。二年復來朝貢,七年又來。



 肅慎氏一名挹婁,在不咸山北,去夫餘可六十日行。東濱大海,西接寇漫汗國,北極弱水。其土界廣袤數千里,居深山窮谷,其路險阻,車馬不通。夏則巢居,冬則穴處。父子世為君長。無文墨,以言語為約。有馬不乘,但以為財產而已。無牛羊,多畜豬,食其肉,衣其皮,績毛以為布。
 有樹名雒常,若中國有聖帝代立,則其木生皮可衣。無井灶,作瓦鬲,受四五升以食。坐則箕踞,以足挾肉而啖之,得凍肉,坐其上令暖。土無鹽鐵,燒木作灰,灌取汁而食之。俗皆編髮,以布作衣詹,徑尺餘,以蔽前後。將嫁娶,男以毛羽插女頭,女和則持歸,然後致禮娉之。婦貞而女淫,貴壯而賤老,死者其日即葬之於野,交木作小槨,殺豬積其上,以為死者之糧。性凶悍,以無憂哀相尚。父母死,男子不哭泣,哭者謂之不壯。相盜竊,無多少皆殺之,故雖野處而不相犯。有石砮,皮骨之甲,檀弓三尺五寸,楛矢長尺有咫。其國東北有山出石,其利入鐵,將取之,
 必先祈神。



 周武王時,獻其楛矢、石砮。逮于周公輔成王,復遣使入賀,爾後千餘年,雖秦漢之盛,莫之致也。及文帝作相,魏景元末,來貢楛矢、石砮、弓甲、貂皮之屬。魏帝詔歸于相府,賜其王傉雞錦罽、綿帛。至武帝元康初,復來貢獻。元帝中興,又詣江左貢其石砮。至成帝時,通貢於石季龍,四年方達。季龍問之,答曰:「每候牛馬向西南眠者三年矣,是知有大國所在,故來一云。



 倭人在帶方東南大海中,依山島為國,地多山林,無良田,食海物。舊有百餘小國相接,至魏時,有三十國通好。
 戶有七萬。男子無大小,悉黥面文身。自謂太伯之後,又言上古使詣中國,皆自稱大夫。昔夏少康之子封於會稽,繼髮文身以避蛟龍之害,今倭人好沈沒取魚,亦文身以厭水禽。計其道里,當會稽東冶之東。其男子衣以橫幅,但結束相連,略無縫綴。婦人衣如單被,穿其中央以貫頭,而皆被髮徒跣。其地溫暖,俗種禾稻糸寧麻而蠶桑織績。土無牛馬,有刀楯弓箭,以鐵為鏃。有屋宇,父母兄弟臥息異處。食飲用俎豆。嫁娶不持錢帛,以衣迎之。死有棺無槨,封土為冢。初喪,哭泣,不食肉。已葬,舉家入水澡浴自潔,以除不祥。其舉大事,輒灼骨以占吉凶。不
 知正歲四節,但計秋收之時以為年紀。人多壽百年,或八九十。國多婦女,不淫不妒。無爭訟,犯輕罪者沒其妻孥,重者族滅其家。舊以男子為主。漢末,倭人亂,攻伐不定,乃立女子為王,名曰卑彌呼。



 宣帝之平公孫氏也,其女王遣使至帶方朝見,其後貢聘不絕。及文帝作相,又數至。泰始初,遣使重譯入貢。



 裨離國在肅慎西北,馬行可二百日,領戶二萬。養雲國去裨離馬行又五十日,領戶二萬。寇莫汗國去養雲國又百日行,領戶五萬餘。一群國去莫汗又百五十日,計
 去肅慎五萬餘里。其風俗土壤並未詳。



 泰始三年,各遣小部獻其方物。至太熙初,復有牟奴國帥逸芝惟離、模盧國帥沙支臣芝、于離末利國帥加牟臣芝、蒲都國帥因末、繩全國帥馬路、沙樓國帥釤加,各遣正副使詣東夷校尉何龕歸化。



 西戎,吐谷渾、焉耆國、龜茲國、大宛國、康居國、大秦國、吐谷渾、吐延、葉延、辟奚、視連、視羆、樹洛干。



 吐谷渾,慕容廆之庶長兄也,其父涉歸分部落一千七百家以隸之。及涉歸卒,廆嗣位,而二部馬鬥,廆怒曰:「先公分建有別,奈何不相遠離,而令馬鬥!」吐谷渾曰:「馬為
 畜耳,鬥其常性,何怒於人!乖別甚易,當去汝於萬里之外矣。」於是遂行。廆悔之,遣其長史史那蔞馮及父時耆舊追還之。吐谷渾曰:「先公稱卜筮之言,當有二子克昌,祚流後裔。我卑庶也、理無並大,今因馬而別,殆天所啟乎!諸君試驅馬令東,馬若還東,我當相隨去矣。」樓馮遣從者二千騎,擁馬東出數百步,輒悲鳴西走。如是者十餘輩,樓馮跪而言曰:「此非人事也。」遂止。鮮卑謂兄為阿干,廆追思之,作《阿干之歌》,歲暮窮思,常歌之。



 吐谷渾謂其部落曰:「我兄弟俱當享國,廆及曾玄纔百餘年耳。我玄孫已後,庶其昌乎!」於是乃西附陰山。屬永嘉之亂,始
 度隴而西,其後子孫據有西零已西甘松之界,極乎白蘭數千里。然有城郭而不居,隨逐水草,廬帳為屋,以肉酪為糧。其官置長史、司馬、將軍,頗識文字。其男子通服長裙,帽或戴冪䍦。婦人以金花為首飾,辮髮縈後,綴以珠貝。其婚姻,富家厚出娉財,竊女而去。父卒,妻其群母;兄亡,妻其諸嫂。喪服制,葬訖而除。國無常稅,調用不給,輒斂富室商人,取足而止。殺人及盜馬者罪至死,他犯則徵物以贖。地宜大麥,而多蔓菁,頗有菽粟。出蜀馬、犛牛。西北雜種謂之為阿柴虜,或號為野虜焉。吐谷渾年七十二卒,有子六十人,長曰吐延,嗣。



 吐延身長七尺八寸,雄姿魁傑,羌虜憚之,號曰項羽。性俶儻不群,嘗慷慨謂其下曰:「大丈夫生不在中國,當高光之世,與韓、彭、吳、鄧並驅中原,定天下雌雄,使名垂竹帛,而潛竄窮山,隔在殊俗,不聞禮教於上京,不得策名於天府,生與麋鹿同群,死作氈裘之鬼,雖偷觀日月,獨不愧於心乎!」性酷忍,而負其智,不能恤下,為羌酋姜聰所刺。劍猶在其身,謂其將紇拔泥曰:「豎子刺吾,吾之過也,上負先公,下愧士女。所以控制諸羌者,以吾故也。吾死之後,善相葉延,速保白蘭。」言終而卒。在位十三年,有子十二人,長子葉延嗣。



 葉延年十歲,其父為羌酋姜聰所害,每旦縛草為姜聰之象,哭而射之,中之則號泣,不中則瞋目大呼。其母謂曰:「姜聰,諸將已屠鱠之矣,汝何為如此?」葉延泣曰:「誠知射草人不益於先仇,以申罔極之志耳。」性至孝,母病,五日不食,葉延亦不食。長而沈毅,好問天地造化、帝王年曆。司馬薄洛鄰曰:「臣等不學,實未審三皇何父之子,五帝誰母所生。」延曰:「自羲皇以來,符命玄象昭言著見,而卿等面牆,何其鄙哉!語曰『夏蟲不知冬冰』,良不虛也。」又曰:「《禮》云公孫之子得以王父字為氏,吾祖始自昌黎光宅於此,今以吐谷渾為氏,尊祖之義也。」在位二十三年
 卒,年三十三。有子四人,長子辟奚嗣。



 辟奚性仁厚慈惠。初聞苻堅之盛,遣使獻馬五十匹,金銀五百斤。堅大悅,拜為安遠將軍。時辟奚三弟皆專恣,長史鐘惡地恐為國害,謂司馬乞宿雲曰:「昔鄭莊公、秦昭王以一弟之寵,宗祀幾傾,況今三孽並驕,必為社稷之患。吾與公忝當元輔,若獲保首領以沒于地,先君有問,其將何辭!吾今誅之矣。」宿雲請白辟奚,惡地曰:「吾王無斷,不可以告。」於是因群下入覲,遂執三弟而誅之。辟奚自投于床,惡地等奔而扶之,曰:「臣昨夢先王告臣云:『三弟將為逆亂,汝速除之。』臣謹奉先王之命矣。」辟奚素
 友愛,因恍惚成疾,謂世子視連曰:「吾禍滅同生,何以見之於地下!國事大小,汝宜攝之,吾餘年殘命,寄食而已。」遂以憂卒。在位二十五年,時年四十二。有子六人,視連嗣。



 視連既立,通娉於乞伏乾歸,拜為白蘭王。視連幼廉慎有志性,以父憂卒,不知政事,不飲酒遊田七年矣。鐘惡地進曰:「夫人君者,以德御世,以威齊眾,養以五味,娛以聲色。此四者,聖帝明王之所先也,而公皆略之。昔昭公儉嗇而喪,偃王仁義而亡,然則仁義所以存身,亦所以亡己。經國者,德禮也;濟世者,刑法也。二者或差,則綱維
 失緒。明公奕葉重光,恩結西夏,雖仁孝發於天然,猶宜憲章周孔,不可獨追徐偃之仁,使刑德委而不建。」視連泣曰:「先王追友于之痛,悲憤升遐,孤雖纂業,尸存而已。聲色遊娛,豈所安也!綱維刑禮,付之將來。」臨終,謂其子視羆曰:「我高祖吐谷渾公常言子孫必有興者,永為中國之西籓,慶流百世。吾已不及,汝亦不見,當在汝之子孫輩耳。」在位十五年而卒。有二子,長曰視羆,少曰烏紇堤。



 視羆性英果,有雄略,嘗從容謂博士金城騫苞曰:「《易》云:『動靜有常,剛柔斷矣。』先王以仁宰世,不任威刑,所以剛
 柔靡斷,取輕鄰敵。當仁不讓,豈宜拱默者乎!今將秣馬厲兵,爭衡中國,先生以為何如?」苞曰:「大王之言,高世之略,秦隴英豪所願聞也。」於是虛襟撫納,眾赴如歸。乞伏乾歸遣使拜為使持節、都督龍涸已西諸軍事、沙州牧、白蘭王。視羆不受,謂使者曰:「自晉道不綱,姦雄競逐,劉、石虐亂,秦、燕跋扈,河南王處形勝之地,宜當糾合義兵,以懲不順,奈何私相假署,擬僭群兇!寡人承五祖之休烈,控弦之士二萬,方欲掃氛秦隴,清彼沙涼,然後飲馬涇渭,戮問鼎之豎,以一丸尼封東關,閉燕趙之路,迎天子于西京,以盡遐籓之節,終不能如季孟、子陽妄自尊
 大。為吾白河南王,何不立勳帝室,策名王府,建當年之功,流芳來葉邪!」乾歸大怒,然憚其彊,初猶結好,後竟遣眾擊之。視羆大敗,退保白蘭。在位十一年,年三十三卒。子樹洛干年少,傳位於烏紇堤。



 烏紇堤一名大孩,性軟弱,耽酒淫色,不恤國事。乞伏乾歸之入長安也,烏紇堤屢抄其境。乾歸怒,率騎討之。烏紇堤大敗,亡失萬餘口,保于南涼,遂卒於胡國。在位八年,時年三十五。視羆之子樹洛干立。



 樹洛乾九歲而孤,其母念氏聰惠有姿色,烏紇堤妻之,有寵,遂專國事。洛干十歲便自稱世子,年十六嗣立,率
 所部數千家奔歸莫何川,自稱大都督、車騎大將軍、大單于、吐谷渾王。化行所部,眾庶樂業,號為戊寅可汗,沙漒雜種莫不歸附。乃宣言曰:「孤先祖避地於此,暨孤七世,思與群賢共康休緒。今士馬桓桓,控弦數萬,孤將振威梁益,稱霸西戎,觀兵三秦,遠朝天子,諸君以為何如?」眾咸曰:「此盛德之事也,願大王自勉!」乞伏乾歸甚忌之,率騎二萬,攻之於赤水。樹洛干大敗,遂降乾歸,乾歸拜為平狄將軍、赤水都護,又以其弟吐護真為捕虜將軍、層城都尉。其後屢為乞伏熾磐所破,又保白蘭,慚憤發病而卒。在位九年,時年二十四。熾磐聞其死,喜曰:「此虜
 矯矯,所謂有豕白蹄也。」有子四人,世子拾虔嗣。其後世嗣不絕。



 焉耆國西去洛陽八千二百里,其地南至尉犁,北與烏孫接,方四百里。四面有大山,道險隘,百人守之,千人不過。其俗丈夫翦髮,婦人衣襦,著大褲。婚姻同華夏。好貨利,任姦詭。王有侍衛數十人,皆倨慢無尊卑之禮。



 武帝太康中,其王龍安遣子入侍。安夫人獪胡之女,妊身十二月,剖脅生子,曰會,立之為世子。會少而勇傑,安病篤,謂會曰:「我嘗為龜茲王白山所辱,不忘於心。汝能雪之,
 乃吾子也。」及會立,襲滅白山,遂據其國,遣子熙歸本國為王。會有膽氣籌略,遂霸西胡,葱嶺以東莫不服。然恃勇輕率,嘗出宿於外,為龜茲國人羅雲所殺。



 其後張駿遣沙州刺史楊宣率眾疆理西域,宣以部將張植為前鋒,所向風靡。軍次其國,熙距戰於賁崙城,為植所敗。植時屯鐵門,未至十餘里,熙又率眾先要之於遮留谷。植將至,或曰:「漢祖畏於柏人,岑彭死於彭亡,今谷名遮留,殆將有伏?」植單騎嘗之,果有伏發。植馳擊敗之,進據尉犁,熙率群下四萬人肉袒降于宣。呂光討西域,復降于光。及光僭位,熙又遣子入侍。



 龜茲國西去洛陽八千二百八十里,俗有城郭,其城三重,中有佛塔廟千所。人以田種畜牧為業,男女皆翦髮垂項。王宮壯麗,煥若神居。



 武帝太康中,其王遣子入侍。惠懷末,以中國亂,遣使貢方物於張重華。苻堅時,堅遣其將呂光率眾七萬伐之,其王白純距境不降,光進軍討平之。



 大宛國去洛陽萬三千三百五十里,南至大月氏,北接康居,大小七十餘城。土宜稻麥,有蒲陶酒,多善馬,馬汗
 血。其人皆深目多鬚。其俗娶婦先以金同心指鈽為娉,又以三婢試之。不男者絕婚。姦淫有子,皆卑其母。與人馬乘不調墜死者,馬主出斂具。善市賈,爭分銖之利,得中國金銀,輒為器物,不用為幣也。



 太康六年,武帝遣使楊顥拜其王藍庾為大宛王。藍庾卒,其子摩之立,遣使貢汗血馬,



 康居國在大宛西北可二千里,與粟弋、伊列鄰接。其王居蘇薤城。風俗及人貌、衣服略同大宛。地和暖,饒桐柳蒲陶,多牛羊,出好馬。泰始中,其王那鼻遣使上封事,并
 獻善馬。



 大秦國一名犁鞬,在西海之西,其地東西南北各數千里。有城邑,其城周回百餘里。屋宇皆以珊瑚為棁栭,琉璃為牆壁,水精為柱礎。其王有五宮,其宮相去各十里,每旦於一宮聽事,終而復始。若國有災異,輒更立賢人,放其舊王,被放者亦不敢怨。有官曹簿領,而文字習胡,亦有白蓋小車、旌旗之屬,及郵驛制置,一如中州。其人長大,貌類中國人而胡服。其土多出金玉寶物、明珠、大貝,有夜光璧、駭雞犀及火浣布,又能刺金縷繡及積錦
 縷罽。以金銀為錢,銀錢十當金錢之一。安息、天竺人與之交市於海中,其利百倍。鄰國使到者,輒廩以金錢。途經大海,海水鹹苦不可食,商客往來皆齎三歲糧,是以至者稀少。



 漢時都護班超遣掾甘英使其國,入海,船人曰:「海中有思慕之物,往者莫不悲懷。若漢使不戀父母妻子者,可入。」英不能渡。武帝太康中,其王遣使貢獻。



 南蠻,林邑、扶南。



 林邑國本漢時象林縣,則馬援鑄柱之處也,去南海三千里。後漢末,縣功曹姓區,有子曰連,殺令自立為王,子
 孫相承。其後王無嗣,外孫范熊代立。熊死,子逸立。其俗皆開北戶以向日,至於居止,或東西無定。人性凶悍,果於戰鬥,便山習水,不閑平地。四時暄暖,無霜無雪,人皆惈露徒跣,以黑色為美。貴女賤男,同姓為婚,婦先娉婿。女嫁之時,著迦盤衣,橫幅合縫如井欄,首戴寶花。居喪翦鬢謂之孝,燔尸中野謂之葬。其王服天冠,被纓絡,每聽政,子弟侍臣皆不得近之。



 自孫權以來,不朝中國。至武帝太康中,始來貢獻。咸康二年,范逸死,奴文纂位。文,日南西卷縣夷帥范椎奴也。嘗牧牛澗中,獲二鯉魚,化成鐵,用以為刀。刀成,乃對大石嶂而咒之曰:「鯉魚變
 化,冶成雙刀,石嶂破者,是有神靈。」進斫之,石即瓦解。文知其神,乃懷之。隨商賈往來,見上國制度,至林邑,遂教逸作宮室、城邑及器械。逸甚愛信之,使為將。文乃譖逸諸子,或徙或奔。及逸死,無嗣,文遂自立為王。以逸妻妾悉置之高樓,從己者納之,不從者絕其食。於是乃攻大岐界、小岐界、式僕、徐狼、屈都、乾魯、扶單等諸國,并之,有眾四五萬人。遣使通表入貢於帝,其書皆胡字。至永和三年,文率其眾攻陷日南,害太守夏侯覽,殺五六千人,餘奔九真,以覽尸祭天,鏟平西卷縣城,遂據日南。告交州刺史朱蕃,求以日南北鄙橫山為界。



 初,徼外諸國嘗
 齎寶物自海路來貿貨,而交州刺史、日南太守多貪利侵侮,十折二三。至刺史姜壯時,使韓戢領日南太守,戢估較太半,又伐船調枹,聲云征伐,由是諸國恚憤。且林邑少田,貪日南之地,戢死絕,繼以謝擢,侵刻如初。及覽至郡,又耽荒于酒,政教愈亂,故被破滅。



 既而文還林邑。是歲,朱蕃使督護劉雄戍于日南,文復攻陷之。四年,文又襲九真,害士庶十八九。明年,征西督護滕畯率交廣之兵伐文於盧容,為文所敗,退次九真。其年,文死,子佛嗣。



 升平末,廣州刺史勝含率眾伐之,佛懼,請降,含與盟而還。至孝武帝寧康中,遣使貢獻。至義熙中,每歲又
 來寇日南、九真、九德等諸郡,殺傷甚眾,交州遂致虛弱,而林邑亦用疲弊。



 佛死,子胡達立,上疏貢金盤椀及金鉦等物。



 扶南西去林邑三千餘里,在海大灣中,其境廣袤三千里,有城邑宮室。人皆醜黑拳髮,惈身跣行。性質直,不為寇盜,以耕種為務,一歲種,三歲獲。又好雕文刻鏤,食器多以銀為之,貢賦以金銀珠香。亦有書記府庫,文字有類於胡。喪葬婚姻略同林邑。



 其王本是女子,字葉柳。時有外國人混潰者,先事神,夢神賜之弓,又教載舶入海。
 混潰旦詣神祠,得弓,遂隨賈人泛海至扶南外邑。葉柳率眾禦之,混潰舉弓,葉柳懼,遂降之。於是混潰納以為妻,而據其國。後胤衰微,子孫不紹,其將范尋復世王扶南矣。



 武帝泰始初,遣使貢獻。太康中,又頻來。穆帝升平初,復有竺旃檀稱王,遣使貢馴象。帝以殊方異獸,恐為人患,詔還之。



 北狄,匈奴。



 匈奴之類,總謂之北狄。匈奴地南接燕趙,北暨沙漠,東連九夷,西距六戎。世世自相君臣,不稟中國正朔。夏曰:
 薰鬻,殷曰鬼方,周曰獫狁,漢曰匈奴。其強弱盛衰、風俗好尚區域所在,皆列于前史。



 前漢末,匈奴大亂,五單于爭立,而呼韓邪單于失其國,攜率部落,入臣於漢。漢嘉其意,割并州並界以安之。於是匈奴五千餘落入居朔方諸郡,與漢人雜處。呼韓邪感漢恩,來朝,漢因留之,賜其邸舍,猶因本號,聽稱單于,歲給綿絹錢穀,有如列侯。子孫傳襲,歷代不絕。其部落隨所居郡縣,使宰牧之,與編戶大同,而不輸貢賦。多歷年所,戶口漸滋,彌漫北朔,轉難禁制。後漢末,天下騷動,群臣競言胡人猥多,懼必為寇,宜先為其防。建安中,魏武帝始分其眾為五部,部
 立其中貴者為帥,選漢人為司馬以監督之。魏末,復改帥為都尉。其左部都尉所統可萬餘落,居于太原故茲氏縣;右部都尉可六千餘落,居祁縣;南部都尉可三千餘落,居蒲子縣;北部都尉可四千餘落,居新興縣;中部都尉可六千餘落,居大陵縣。



 武帝踐阼後,塞外匈奴大水,塞泥、黑難等二萬餘落歸化,帝復納之,使居河西故宜陽城下。後復與晉人雜居,由是平陽、西河、太原、新興、上黨、樂平諸郡靡不有焉。泰始七年,單于猛叛,屯孔邪城。武帝遣婁侯何楨持節討之。楨素有志略,以猛眾凶悍,非少兵所制,乃潛誘猛左部督李恪殺猛,於是匈奴
 震服,積年不敢復反。其後稍因忿恨,殺害長史,漸為邊患。侍御史西河郭欽上疏曰:「戎狄彊獷,歷古為患。魏初人寡,西北諸郡皆為戎居。今雖服從,若百年之後有風塵之警,胡騎自平陽、上黨不三日而至孟津,北地、西河、太原、馮翊、安定、上郡盡為狄庭矣。宜及平吳之威,謀臣猛將之略,出北地、西河、安定,復上郡,實馮翊,於平陽已北諸縣募取死罪,徙三河、三魏見士四萬家以充之。裔不亂華,漸徙平陽、弘農、魏郡、京兆、上黨雜胡,峻四夷出入之防,明先王荒服之制,萬世之長策也。」帝不納。至太康五年,復有匈奴胡太阿厚率其部落二萬九千三百
 人歸化。七年,又有匈奴胡都大博及萎莎胡等各率種類大小凡十萬餘口,詣雍州刺史扶風王駿降附。明年,匈奴都督大豆得一育鞠等復率種落大小萬一千五百口,牛二萬二千頭,羊十萬五千口,車廬什物不可勝紀,來降,并貢其方物,帝並撫納之。



 北狄以部落為類,其入居塞者有屠各種、鮮支種、寇頭種、烏譚種、赤勒種、捍蛭種、黑狼種、赤沙種、鬱鞞種、萎莎種、禿童種、勃蔑種、羌渠種、賀賴種、鐘跂種、大樓種、雍屈種、真樹種、力羯種,凡十九種,皆有部落,不相雜錯。屠各最豪貴,故得為單于,統領諸種。其國號有左賢王、右賢王、左奕蠡王、右奕蠡
 王、左於陸王、右於陸王、左漸尚王、右漸尚王、左朔方王、右朔方王、左獨鹿王、右獨鹿王、左顯祿王、右顯祿王、左安樂王、右安樂王、凡十六等,皆用單于親子弟也。其左賢王最貴,唯太子得居之。其四姓,有呼延氏、卜氏、蘭氏、喬氏。而呼延氏最貴,則有左日逐、右日逐,世為輔相;卜氏則有左沮渠、右沮渠;蘭氏則有左當戶、右當戶;喬氏則有左都侯、右都侯。又有車陽、沮渠、餘地諸雜號,猶中國百官也。其國人有綦毋氏、勒氏、皆勇健,好反叛。武帝時,有騎督綦毋伣邪伐吳有功,遷赤沙都尉。



 惠帝元康中,匈奴郝散攻上黨,殺長吏,入守上郡。明年,散弟度元
 又率馮翊、北地羌胡攻破二郡。自此已後,北狄漸盛,中原亂矣。



 史臣曰:夫宵形稟氣,是稱萬物之靈,繫土隨方,乃有群分之異。蹈仁義者為中寓,肆凶獷者為外夷,譬諸草木,區以別矣。夷狄之徒,名教所絕,窺邊侯隙,自古為患,稽諸前史,憑陵匪一。軒皇北逐,唐帝南征,殷后東戡,周王西狩,皆所以禦其侵亂也。嬴劉之際,匈奴最彊;元成之間,呼韓委質,漢嘉其節,處之中壤。歷年斯永,種類逾繁,舛號殊名,不可勝載。爰及泰始,匪革前迷,廣闢塞垣,更招種落,納萎莎之後附,開育鞠之新降,接帳連韝,充郊
 掩甸。既而沸脣成俗,鳴鏑為群,振鴞響而挻災,恣狼心而逞暴。何楨縱策,弗沮於姦萌;郭欽馳疏,無救於妖漸。未環星紀,坐傾都邑,黎元塗地,凶族滔天。迹其所由,抑武皇之失也。吐谷渾分緒偽燕,遠辭正嫡,率東胡之餘眾,掩西羌之舊宇,綱疏政暇,地廣兵全,廓萬里之基,貽一匡之訓,弗忘忠義,良可嘉焉。吐延夙標宏偉,見方於項籍,始遵朝化,遽夭於姜聰,高節不群,亦殊籓之秀也。葉延至孝,寄新哀於射草;辟奚深友,邁古烈於分荊;視連蒸蒸,光奉先之義;視羆矯矯,蘊經時之略;洛干童幼,早擅英規,未騁雄心,先摧凶手,奉順者必敗,豈天亡晉
 乎!且渾廆連枝,生自邊極,各謀孫而翼子,咸革裔而希華。廆胤奸兇,假鳳圖而竊號,渾嗣忠謹,距龍涸而歸誠。懷奸者數世而亡,資忠者累葉彌劭,積善餘慶,斯言信矣。



 贊曰:逖矣前王,區別群方。叛由德弛,朝因化昌。武后升圖,智昧遷胡。遽淪家國,多謝明謨。谷渾英奮,思矯穨運;克昌其緒,實資忠訓。



\end{pinyinscope}