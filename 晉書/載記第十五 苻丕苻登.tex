\article{載記第十五 苻丕苻登}

\begin{pinyinscope}

 苻
 丕苻登



 苻丕,字永叔,堅之長庶子也。少而聰彗好學,博綜經史。堅與言將略,嘉之,命鄧羌教以兵法。文武才幹亞於苻融,為將善收士卒情,出鎮于鄴,東夏安之。堅敗歸長安,丕為慕容垂所逼,自鄴奔枋頭。堅之死也,丕復入鄴城,將收兵趙、魏,西赴長安。會幽州刺史王永、平州刺史苻沖頻為垂將平規等所敗,乃遣昌黎太守宋敞焚燒和
 龍、薊城宮室,率眾三萬進屯壺關,遣使招丕。丕乃去鄴,率男女六萬餘口進如潞川。驃騎張蠔、並州刺史王騰迎之,入據晉陽,始知堅死問,舉哀于晉陽,三軍縞素。王永留苻沖守壺關,率騎一萬會丕,勸稱尊號,丕從之,乃以太元十年僭即皇帝位于晉陽南。立堅行廟,太赦境內,改元曰太安。置百官,以張蠔為侍中、司空,封上黨郡公;王永為使持節、侍中、都督中外諸軍事、車騎大將軍、尚書令,進封清河公;王騰為散騎常侍、中軍大將軍、司隸校尉、陽平郡公;苻沖為左光祿大夫、尚書左僕射、西平王;俱石子為衛將軍、濮陽公;楊輔為尚書右僕射、濟
 陽公;王亮為護軍將軍、彭城公;強益耳、梁暢為侍中,徐義為吏部尚書,並封縣公。自餘封授各有差。



 是時安西呂光自西域還師,至于宜禾,堅涼州刺史梁熙謀閉境距之。高昌太守楊翰言於熙曰:「「呂光新定西國,兵強氣銳,其鋒不可當也。度其事意,必有異圖。且今關中擾亂,京師存亡未知,自河已西迄于流沙,地方萬里,帶甲十萬,鼎峙之勢實在今日。若光出流沙,其勢難測。高梧谷口,水險之要,宜先守之而奪其水。彼既窮渴,自然投戈。如其以遠不守,伊吾之關亦可距也。若度此二要,雖有子房之策,難為計矣。地有所必爭,真此機也。」熙弗從。美
 水令犍為張統說熙曰:「主上傾國南討,覆敗而還。慕容垂擅兵河北,泓、沖寇逼京師,丁零雜虜,跋扈關、洛,州郡姦豪,所在風扇,王綱弛絕,人懷利己。今呂光回師,將軍何以抗也?」熙曰:「誠深憂之,未知計之所出。」統曰:「光雄果勇毅,明略絕人,今以蕩西域之威,擁歸師之銳,鋒若猛火之盛於原,弗可敵也。將軍世受殊恩,忠誠夙著,立勛王室,宜在於今。行唐公洛,上之從弟,勇冠一時。為將軍計者,莫若奉為盟主,以攝眾望,推忠義以總率群豪,則光無異心也。資其精銳,東兼毛興,連王統、楊璧,集四州之眾,掃凶逆於諸夏,寧帝室於關中,此桓文之舉也。」熙
 又不從。殺洛于西海,以子胤為鷹揚將軍,率眾五萬距光于酒泉。敦煌太守姚靜、晉昌太守李純以郡降光。胤及光戰于安彌,為光所敗。武威太守彭濟執熙迎光,光殺之。建威、西郡太守索泮,奮威、督洪池已南諸軍事、酒泉太守宋皓等,並為光所殺。



 堅尚書令、魏昌公苻纂自關中來奔,拜太尉,進封東海王。以中山太守王兗為平東將軍、平州刺史、阜城侯,苻定為征東將軍、冀州牧、高城侯,苻紹為鎮東將軍、督冀州諸軍事、重合侯,苻謨為征西將軍、幽州牧、高邑侯,苻亮為鎮北大將軍、督幽、並二州諸軍事,並進爵郡公。定、紹據信都,謨、亮先據常山,慕
 容垂之圍鄴城也,並降于垂,聞丕稱尊號,遣使謝罪。王兗固守博陵,與垂相持。左將軍竇衝、秦州刺史王統、河州刺史毛興、益州刺史王廣、南秦州刺史楊璧、衛將軍楊定,並據隴右,遣使招丕,請討姚萇。丕大悅,以定為驃騎大將軍、雍州牧,衝為征西大將軍、梁州牧,統鎮西大將軍,興車騎大將軍,璧征南大將軍,並開府儀同三司,加散騎常侍,廣安西將軍,皆進位州牧。



 於是王永宣檄州郡曰:「大行皇帝棄背萬國,四海無主。征東大將軍、長樂公,先帝元子,聖武自天,受命荊南,威振衡海,分陜東都,道被夷夏,仁澤光于宇宙,德聽侔于《下武》。永與司空
 蠔等謹順天人之望,以季秋吉辰奉公紹承大統,銜哀即事,栖谷總戎,枕戈待旦,志雪大恥。慕容垂為封豕于關東,泓、沖繼凶于京邑,致乘輿播越,宗社淪傾。羌賊姚萇,我之牧士,乘釁滔天,親行大逆,有生之巨賊也。永累葉受恩,世荷將相,不與驪山之戎、滎澤之狄共戴皇天,同履厚土。諸牧伯公侯或宛沛宗臣,或四七勛舊,豈忍舍破國之醜豎,縱殺君之逆賊乎!主上飛龍九五,實協天心,靈祥休瑞,史不輟書,投戈效義之士三十餘萬,少康、光武之功可旬朔而成。今以衛將軍俱石子為前軍師,司空張蠔為中軍都督。武將猛士,風烈雷震,志殄元
 兇,義無他顧。永謹奉乘輿,恭行天罰。君臣終始之義,在三忘軀之誠,戮力同之,以建晉、鄭之美。」



 先是,慕容驎攻王兗于博陵,至是糧竭矢盡,郡功曹張猗踰城聚眾應驎。袞臨城數之曰:「卿,秦之人也。吾,卿之君也。起眾應賊,號稱義兵,何名實相違之甚!卿兄往合鄉宗,親逐城主,天地不容,為世大戮。身滅未幾,卿復續之。卿見為吾吏,親尋干戈,競為戎首,為爾君者,不亦難乎!今人何取卿一切之功,寧能忘卿不忠不孝之事!古人有云,求忠臣必出孝子之門,卿母在城,不能顧之,何忠義之可望!惡不絕世,卿之謂也。不圖中州禮義之邦邦,而卿門風若斯。
 卿去老母如脫屣,吾復何論哉!」既而城陷,兗及固安侯苻鑒,並為驎所殺。



 丕復以王永為司徒、錄尚書事,徐義為尚書令,加右光祿大夫。



 初,王廣還自成都也,奔其兄秦州刺史統。及長安不守,廣攻河州牧毛興于枹罕。興遣建節將軍、臨清柏衛平率其宗人千七百夜擊廣軍,大敗之。王統復遣兵助廣,興於是嬰城固守。既而襲王廣,敗之,廣亡奔秦州,為隴西鮮卑匹蘭所執,送詣姚萇。興既敗王廣,謀伐王統,平上邽。袍罕諸氐皆窘於兵革而疲不堪命,乃殺興,推衛平為使持節、安西將軍、河州刺史,遣使請命。



 刁雲殺慕容忠,乃推慕容永為使持節、
 大都督中外諸軍事、大將軍、大單于、雍、秦、梁、涼四州牧、錄尚書事、河東王、稱籓于垂。征東苻定、鎮東苻紹、征北苻謨、鎮北苻亮皆降于慕容垂。



 丕又進王永為左丞相,苻纂為大司馬,張蠔為太尉,王騰為驃騎大將軍、儀同三司,徐義為司空,苻沖為車騎大將軍、尚書令、儀同三司,俱石子為衛大將軍、尚書左僕射,領官皆如故。永又檄州郡曰:「昔夏有窮夷之難,少康起焉;王莽毒殺平帝,世祖重光漢道;百六之運,何代無之!天降喪亂,羌胡猾夏,先帝晏駕賊庭,京師鞠為戎穴,神州蕭條,生靈塗炭。天未亡秦,社稷有奉。主上聖德恢弘,道侔光武,所在宅
 心,天人歸屬,必當隆中興之功,復配天之美。姚萇殘虐,慕容垂凶暴,所過滅戶夷煙,毀發丘墓,毒遍存亡,痛纏幽顯,雖黃巾之害於九州,赤眉之暴于四海,方之未為甚也。今素秋將及,行師令辰,公侯牧守,壘主鄉豪,或戮力國家,乃心王室,各率所統,以孟冬上旬會大駕于臨晉。」於是天水姜延、馮翊寇明、河東王昭、新平張晏、京兆杜敏、扶風馬郎、建忠高平牧官都尉王敏等咸承檄起兵,各有眾數萬,遣使應丕。皆就拜將軍、郡守,封列侯。冠軍鄧景擁眾五千據彭池,與竇衝為首尾,擊萇平涼太守金熙。安定北部都尉鮮卑沒奕于率鄯善王胡員吒、
 護羌中郎將梁茍奴等,與萇左將軍姚方成、鎮遠強京戰于孫丘谷,大敗之。



 枹罕諸氐以衛平年老,不可以成事業,議廢之,而憚其宗彊,連日不決。氐有啖青者,謂諸將曰:「大事宜定,東討姚萇,不可沈吟猶豫。一旦事發,反為人害。諸軍但請衛公會儲眾將,青為諸軍決之。」眾以為然。於是大饗諸將,青抽劍而前曰:「今天下大亂,豺狼塞路,吾曹今日可謂休戚是同,非賢明之主莫可濟艱難也。衛公朽耄,不足以成大事,宜反初服,以避賢路,狄道長苻登雖王室疏屬,而志略雄明,請共立之,以赴大駕。諸君若有不同者,便下異議。」乃奮劍攘袂,將斬貳己
 者,眾皆從之,莫敢仰視。於是推登為帥,遣使於丕請命。丕以登為征西大將軍、開府儀同三司、南安王、持節及州郡督因其所稱而授之。又以徐義為右丞相。



 丕留王騰守晉陽,楊輔戍壺關,率眾四萬進據平陽。王統以秦州降姚萇。慕容永以丕至平陽,恐不自固,乃遣使求假道還東,丕弗許。遣王永及苻纂攻之,以俱石子為前鋒都督,與慕容永戰于襄陵。王永大敗,永及石子皆死之。



 初,苻纂之奔丕也,部下壯士三千餘人,丕猜而忌之。及永之敗,懼為纂所殺,率騎數千南奔東垣。晉揚威將軍馮該自陜要擊,敗之,斬丕首,執其太子寧、長樂王壽,送
 於京師,朝廷赦而不誅,歸之於苻宏。徐義為慕容永所獲,械埋其足,將殺之。義誦《觀世音經》,至夜中,土開械脫,於重禁之中若有人導之者,遂奔楊佺期,佺期以為洛陽令。苻纂及弟師奴率丕餘眾數萬,奔據杏城。苻登稱尊號,偽謚丕為哀平皇帝。丕之臣佐皆沒慕容永,永乃進據上黨之長子,僭稱大號,改元曰中興。丕在位二年而敗。



 登字文高,堅之族孫也。父敞,健之世為太尉司馬、隴東太守、建節將軍,後為苻生所殺。堅即偽位,追贈右將軍、涼州刺史,以登兄同成嗣。毛興之鎮上邽,以為長史。登
 少而雄勇,有壯氣,粗險不脩細行,故堅弗之奇也。長而折節謹厚,頗覽書傳。拜殿上將軍,稍遷羽林監、揚武將軍、長安令,坐事黜為狄道長。及關中亂,去縣歸毛興。同成言於興,請以登為司馬,常在營部。登度量不群,好為奇略,同成常謂之曰:「汝聞不在其位,不謀其政,無數干時,將為博識者不許。吾非疾汝,恐或不喜人妄豫耳,自是可止。汝後得政,自可專意。」時人聞同成言,多以為疾登而抑蔽之。登乃屏迹不妄交游。興有事則召之,戲謂之曰:「小司馬可坐評事。」登出言輒析理中,興內服焉,然敬憚而不能委任。姚萇作亂,遣其弟碩德率眾伐毛興,
 相持久之。興將死,告同成曰:「與卿累年共擊逆羌,事終不克,何恨之深!可以後事付卿小弟司馬,殄碩德者,必此人也。卿可換攝司馬事。」



 登既代衛平,遂專統征伐。是時歲旱眾飢,道殣相望,登每戰殺賊,名為熟食,謂軍人曰:「汝等朝戰,幕便飽肉,何憂於飢!」士眾從之,啖死人肉,輒飽健能鬥。姚萇聞之,急召碩德曰:「汝不來,必為苻登所食盡。」碩德於是下隴奔萇。



 及丕敗,丕尚書寇遺奉丕子渤海王懿、濟北王昶自杏城奔登。登乃具丕死問,於是為丕發喪行服,三軍縞素。登請立懿為主,眾咸曰:「渤海王雖先帝之子,然年在幼沖,未堪多難。國亂而立長
 君,《春秋》之義也。三虜跨僭,寇旅殷彊,豺狼梟鏡,舉目而是,自厄運之極,莫甚於斯。大王挺劍西州,鳳翔秦、隴,偏師暫接,姚萇奔潰,一戰之功,可謂光格天地。宜龍驤武奮,拯拔舊京,以社稷宗廟為先,不可顧曹臧、吳札一介微節,以失圖運之機,不建中興之業也。」登於是以太元十一年僭即皇帝位,大赦境內,改元曰太初。



 立堅神主於軍中,載以輜軿,羽葆青蓋,車建黃旗,武賁之士三百人以衛之,將戰必告,凡欲所為,啟主而後行。繕甲纂兵,將引師而東,乃告堅神主曰:「維曾孫皇帝臣登,以太皇帝之靈恭踐寶位。昔五將之難,賊羌肆害于聖躬,實
 登之罪也。今合義旅,眾餘五萬,精甲勁兵,足以立功,年穀豐穰,足以資贍。即日星言電邁,直造賊庭,奮不顧命,隕越為期,庶上報皇帝酷冤,下雪臣子大恥。惟帝之靈,降監厥誠。」因覷欷流涕。將士莫不悲慟,皆刻鉾鎧為「死休」字,示以戰死為志。每戰以長槊鉤刃為方圓大陣,知有厚薄,從中分配,故人自為戰,所向無前。



 初,長安之將敗也,堅中壘將軍徐嵩、屯騎校尉胡空各聚眾五千,據險築堡以自固,而受姚萇官爵。及萇之害堅,嵩等以王禮葬堅于二堡之間。至是,各率眾降登。拜嵩鎮軍將軍、雍州刺史,空輔國將軍、京兆尹。登復改葬堅以天子之
 禮。又僭立其妻毛氏為皇后,弟懿為皇太弟。遣使拜苻纂為使持節、侍中、都督中外諸軍事、太師,領大司馬,進封魯王,纂弟師奴為撫軍大將軍、並州牧、朔方公。纂怒謂使者曰:「渤海王世祖之孫,先帝之子,南安王何由不立而自尊乎?」纂長史王旅諫曰:「南安已立,理無中改。賊虜未平,不可宗室之中自為仇敵,原大王遠蹤光武推聖公之義,梟二虜之後,徐更圖之。」纂乃受命。於是貳縣虜帥彭沛穀、屠各董成、張龍世、新平羌雷惡地等盡應之,有眾十餘萬。纂遣師奴攻上郡羌酋金大黑、金洛生,大黑等逆戰,大敗之,斬首五千八百。



 登以竇衝為車騎
 大將軍、南秦州牧,楊定為大將軍、益州牧,楊璧為司空、梁州牧。



 苻纂敗姚碩德于涇陽,姚萇自陰密距纂,纂退屯敷陸。竇衝攻萇汧、雍二城,剋之,斬其將軍姚元平、張略等。又與萇戰于汧東,為萇所敗。登次于瓦亭。萇攻彭沛穀堡,陷之,沛穀奔杏城,萇遷陰密。登征虜、馮翊太守蘭犢率眾二萬自頻陽入于和寧,與苻纂首尾,將圖長安。師奴勸其兄纂稱尊號,纂不從,乃殺纂,自立為秦公。蘭犢絕之,皆為姚萇所敗。



 登進所胡空堡,戎夏歸之者十有餘萬。姚萇遣其將軍姚方成攻陷徐嵩堡,嵩被殺,悉坑戎士。登率眾下隴入朝那,姚萇據武都相持,累戰
 互有勝負。登軍中大飢,收葚以供兵士。立其子崇為皇太子,弁為南安王,尚為北海王。姚萇退還安定。登就食新平,留其大軍于胡空堡,率騎萬餘圍萇營,四面大哭,哀聲動人。萇惡之,乃命三軍哭以應登,登乃引退。



 萇以登頻戰輒勝,謂堅有神驗,亦於軍中立堅神主,請曰:「往年新平之禍,非萇之罪。臣兄襄從陜北渡,假路求西,狐死首丘,欲暫見鄉里。陛下與苻眉要路距擊,不遂而沒。襄敕臣行殺,非臣之罪。苻登陛下末族,尚欲復讎,臣為兄報恥,於情理何負!昔陛下假臣龍驤之號,謂臣曰:『朕以龍驤建業,卿其勉之!』明詔昭然,言猶在耳。陛下雖過
 世為神,豈假手于苻登而圖臣,忘前征時言邪!今為陛下立神象,可歸休於此,勿計臣過,聽臣至誠。」登進師攻萇,既而升樓謂萇曰:「自古及今,安有殺君而反立神象請福,望有益乎!」大呼曰:「殺君賊姚萇出來,吾與汝決之,何為枉害無辜!」萇憚而不應。萇自立堅神象,戰未有利,軍中每夜驚恐,乃嚴鼓斬象首以送登。



 登將軍竇洛、竇于等謀反發覺,出奔于萇。登進討彭池不剋,攻彌姐營及繁川諸堡,皆剋之。萇連戰屢敗,乃遣其中軍姚崇襲大界,登引師要之,大敗崇于安丘,俘斬二萬五千,進攻萇將吳忠、唐匡於平涼,剋之,以尚書苻碩原為前禁將
 軍、滅羌校尉,戍平涼。登進據茍頭原以逼安定。萇率騎三萬夜襲大界營,陷之,殺登妻毛氏及其子弁、尚,擒名將數十人,驅掠男女五萬餘口而去。



 登收合餘兵,退據胡空堡,遣使齎書加竇衝大司馬、驃騎將軍、前鋒大都督、都督隴東諸軍事,楊定左丞相、上大將軍、都督中外諸軍事,楊璧大將軍、都督隴右諸軍事。遣衝率見眾為先驅,自繁川趣長安。登率眾從新平逕據新豐之千戶固。使定率隴上諸軍為其後繼,璧留守仇池。又命其并州刺史楊政、冀州刺史楊楷率所統大會長安。萇遣其將軍王破虜略地秦州,楊定及破虜戰於清水之格奴
 阪,大敗之。登攻張龍世於鴦泉堡,姚萇救之,登引退。萇密遣其將任瓫、宗度詐為內應,遣使招登,許開門納之。登以為然。雷惡地馳謂登曰:「姚萇多計略,善御人,必為姦變,願深宜詳思。」登乃止。萇聞惡地之詣登也,謂諸將曰:「此羌多姦智,今其詣登,事必無成。」登聞萇懸門以待之,大驚,謂左右曰:「雷征東其殆聖乎!微此公,朕幾為豎子所誤。」萇攻陷新羅堡。萇撫風太守齊益男奔登。登將軍路柴、強武等並以眾降於萇。登攻萇將張業生於隴東,萇救之,不剋而退。登將軍魏褐飛攻姚當成于杏城,為萇所殺。



 馮翊郭質起兵廣鄉以應登,宣檄三輔曰:「義
 感君子,利動小人。吾等生逢先帝堯、舜之化,累世受恩,非常伯納言之子,即卿校牧守之胤,而可坐視豺狼忍害君父!裸尸薦棘,痛結幽泉,山陵無松隧之兆,靈主無清廟之頌,賊臣莫大之甚,自古所未聞。雖茹荼之苦,銜蓼之辛,何以諭之!姚萇窮凶肆害,毒被人神,於圖讖歷數萬無一分,而敢妄竊重名,厚顏瞬息,日月固所不照,二儀實亦不育。皇天雖欲絕之,亦將假手于忠節。凡百君子,皆夙漸神化,有懷義方,含恥而存,孰若蹈道而沒乎!」眾咸然之。唯鄭縣人茍曜不從,聚眾數千應姚萇。登以質為平東將軍、馮翊太守。質遣部將伐曜,大敗而歸。
 質乃東引楊楷,以為聲援,又與曜戰于鄭東,為曜所敗,遂歸于萇,萇以為將軍。質眾皆潰散。



 登自雍攻萇將金溫于范氏堡,剋之,遂渡渭水,攻萇京兆太守韋范于段氏堡,不剋,進據曲牢。茍曜有眾一萬,據逆方堡,密應登,登去曲牢繁川,次于馬頭願。萇率騎來距,大戰敗之,斬其尚書吳忠,進攻新平。萇率眾救之,登引退,復攻安定,為萇所敗,據路承堡。



 是時萇疾病,見苻堅為崇。登聞之,秣馬萬兵,告堅神主曰:「曾孫登自受任執戈,幾將一紀,未嘗不上天錫祐,皇鑒垂矜,所在必剋,賊旅冰摧。今太皇帝之靈降災疢于逆羌,以形類推之,醜虜必將不振。
 登當因其隕斃,順行天誅,拯復梓宮,謝罪清廟。」於是大赦境內,百僚進位二等。與萇將姚崇爭麥于清水,累為崇所敗。進逼安定,去城九十餘里。萇疾小瘳,率眾距登,登去營逆萇,萇遣其將姚熙隆別攻登營,登懼,退還。萇夜引軍過登營三十餘里以躡登後。旦而候人告曰:「賊諸營已空,不知所向。」登驚曰:「此為何人,去令我不知,來令我不覺,謂其將死,忽然復來,朕與此羌同世,何其厄哉!」遂罷師還雍。



 以竇衝為右丞相。尋而衝叛,自稱秦王,建年號。登攻之于野人堡,衝請救於姚萇,萇遣其太子興攻胡空堡以救之。登引兵還赴胡空堡,衝遂與萇連
 和。



 至是萇死,登聞之喜曰:「姚興小兒,吾將折杖以笞之。」於是大赦,盡眾而東,攻屠各姚奴、帛蒲二堡,剋之,自甘泉向關中。興追登不及數十里,登從六陌趣廢橋,興將尹緯據橋以待之。登爭水不得,眾渴死者十二三。與緯大戰,為緯所敗,其夜眾潰,登單馬奔雍。



 初,登之東也,留其弟司徒廣守雍,太子崇守胡空堡。廣、崇聞登敗,出奔,眾散。登至,無所歸,遂奔平涼,收集遺眾入馬毛山。興率眾攻之,登遣子汝陰王宗質于隴西鮮卑乞伏乾歸,結婚請援,乾歸遣騎二萬救登。登引軍出迎,與興戰于山南,為興所敗,登被殺。在位九年,時年五十二。崇奔于湟
 中,僭稱尊號,改元延初。偽謚登曰高皇帝,廟號太宗。崇為乾歸所逐,崇、定皆死。



 始,健以穆帝永和七年僭立,至登五世,凡四十有四歲,以孝武帝太元十九年滅。



 索泮,字德林,敦煌人也。世為冠族。泮少時游俠,及長,變節好學,有佐世才器。張天錫輔政,以泮為冠軍、記室參軍。天錫即位,拜司兵,歷位禁中錄事。執法御掾,州府肅然,郡縣改迹。遷羽林左監,有勤幹之稱。出為中壘將軍、西郡武威太守、典戎校尉。政務寬和,戎夏懷其惠,天錫甚敬之。苻堅見而歎曰:「涼州信多君子!」既而以泮河西德望,拜別駕。呂光既剋姑臧,泮固郡不降,光攻而獲之。
 光曰:「孤既平西域,將赴難京師,梁熙無狀,絕孤歸路,此朝延之罪人,卿何意阻郡固迷,自同元惡!」泮厲色責光曰:「將軍受詔討叛胡,可受詔亂涼州邪?寡君何罪,而將軍害之?泮但苦力寡,不能固守以報君父之讎,豈如逆氐彭濟望風反叛!主滅臣死,禮之常也。」乃就刑于市,神色不變。



 弟菱,有俊才,仕張天錫為執法中郎、冗從右監。苻堅世至伏波將軍、典農都尉,與泮俱被害。



 徐嵩,字元高,盛之子也。少以清白著稱。苻堅時舉賢良,為郎中,稍遷長安令,貴戚子弟犯法者,嵩一皆考竟,請託路絕。堅甚奇之,謂其叔父成曰:「人為長吏,故當應耳。
 此年少落落,有端貳之才。」遷守始平郡,甚有威惠。及壘陷,姚方成執而數之,嵩厲色謂方成曰:「汝姚萇罪應萬死,主上止黃眉之斬而宥之,叨據內外,位為列將,無犬馬識養之誠,首為大逆。汝曹羌輩豈可以人理期也!何不速殺我,早見先帝,取姚萇於地下。」方成怒,三斬嵩,漆其首為便器。登哭之哀慟,贈車騎大將軍、儀同三司,謚曰忠武。



 史臣曰:自兩京殄覆,九土分崩,赤縣成蛇豕之墟,紫宸遷蛙黽之穴,干戈日用,戰爭方興,猶逐鹿之並驅,若瞻烏之靡定。苻洪擅蠻陬之桀黠,乘羯虜之危亡,乃附款
 江東而志圖關右,禍生蠆毒,未逞狼心。健既承家,克隆凶緒,率思歸之眾,投山西之隙,據億丈之巖險,總三秦之果銳,敢窺大寶,遂竊鴻名,校數姦雄,有可言矣。長生慘虐,稟自率由。睹辰象之災,謂法星之夜飲;忍生靈之命,疑猛獸之朝飢。但肆毒於刑殘,曾無心於戒懼。招亂速禍,不亦宜乎!



 永固雅量瑰姿,變夷眾夏,葉魚龍之謠詠,挺草付之休徵,剋翦姦回,纂承偽歷,遵明王之德教,闡先聖之儒風,撫育黎元,憂勤庶政。王猛以宏材緯軍國,苻融以懿戚贊經綸,權薛以諒直進規謨,鄧、張以忠勇恢威略,俊賢效足,杞梓呈才,文武兼施,德刑具舉。乃
 平燕定蜀,擒代吞涼,跨三分之二,居九州之七,遐荒慕義,幽險宅心,因止馬而獻歌,託栖鸞以成頌,因以功侔曩烈,豈直化洽當年!雖五胡之盛,莫之比也。



 既而足己夸世,愎諫違謀,輕敵怒鄰,窮兵黷武。懟三正之未葉,恥五運之猶乖,傾率土之師,起滔天之寇,負其犬羊之力,肆其吞噬之能。自謂戰必勝,攻必取,便欲鳴鸞禹穴,駐蹕疑山,疏爵以侯楚材,築館以須歸命。曾弗知人道助順,神理害盈,雖矜涿野之彊,終致昆陽之敗。遂使凶渠候隙,狡寇伺間,步搖啟其禍先,燒當乘其亂極,宗社遷於他族,身首罄於賊臣,貽戒將來,取笑天下,豈不哀哉!
 豈不謬哉!



 苻丕承亂僭竊,尋及傾敗,斯可謂天之所廢,人不能支。苻登集離散之兵,厲死休之志,雖眾寡不敵,難以立功,而義烈慷慨,有足稱矣。



 贊曰:洪惟壯勇,威棱氐種。健藉世資,遂雄關、隴。長生昏虐,敗不旋踵。永固禎祥,肇自龍驤。垂旒負扆,竊帝圖王。患生縱敵,難起矜強。丕、登僭假,淪胥以亡。



\end{pinyinscope}