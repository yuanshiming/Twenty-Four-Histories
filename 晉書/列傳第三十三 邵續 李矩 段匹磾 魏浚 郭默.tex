\article{列傳第三十三 邵續 李矩 段匹磾 魏浚 郭默}

\begin{pinyinscope}
邵續李矩段匹磾魏浚
 \gezhu{
  族子該郭默}



 邵續,字嗣祖,魏郡安陽人也。父乘,散騎侍郎。續朴素有志烈,博覽經史,善談理義,妙解天文。初為成都王穎參軍,穎將討長沙王乂,續諫曰:「續聞兄弟如左右手,今明公當天下之敵,而欲去一手乎?續竊惑之。」穎不納。後為茍晞參軍,除沁水令。



 時天下漸亂,續去縣還家,糾合亡命,得數百人。王浚假續綏集將軍、樂陵太守,屯厭次,以
 續子乂為督護。續綏懷流散,多歸附之。石勒既破浚,遣乂還招續,續以孤危無援,權附於勒,勒亦以乂為督護。既而段匹磾在薊,遺書要續俱歸元帝,續從之。其下諫曰:「今棄勒歸匹磾,任子危矣。」續垂泣曰:「我出身為國,豈得顧子而為叛臣哉!」遂絕於勒,勒乃害乂。續懼勒攻,先求救於匹磾,匹磾遣弟文鴦救續。文鴦未至,勒已率八千騎圍續。勒素畏鮮卑,又聞文鴦至,乃棄攻具東走。續與文鴦追勒至安陵,不及,虜勒所署官,並驅三千餘家,又遣騎入抄勒北邊,掠常山,亦二千家而還。



 匹磾既殺劉琨,夷晉多怨叛,遂率其徒依續。勒南和令趙領等率
 廣川、渤海千餘家背勒歸續。而帝以續為平原樂安太守、右將軍、冀州刺史,進平北將軍、假節,封祝阿子。續遣兄子武邑內史存與文鴦率匹磾眾就食平原,為石季龍所破。續先與曹嶷互相侵掠,嶷因存等敗,乃破續屯田,又抄其戶口,續首尾相救,疲於奔命。太興初,續遣存及文鴦屯濟南黃巾固,因以逼嶷,嶷懼,求和。俄而匹磾率眾攻段末杯,石勒知續孤危,遣季龍乘虛圍續。季龍騎至城下,掠其居人,續率眾出救,季龍伏騎斷其後,遂為季龍所得,使續降其城。續呼其兄子竺等曰:「吾志雪國難,以報所受,不幸至此。汝等努力自勉,便奉匹磾為
 主,勿有二心。」



 時帝既聞續沒,下詔曰:「邵續忠烈在公,義誠慷慨,綏集荒餘,憂國亡身。功勳未遂,不幸陷沒,朕用悼恨于懷。所統任重,宜時有代。其部曲文武,已共推其息緝為營主。續之忠誠,著于公私,今立其子,足以安眾,一以續本位即授緝,使總率所統,效節國難,雪其家仇。」



 季龍遣使送續於勒,勒使使徐光讓之曰:「國家應符撥亂,八表宅心,遺晉怖威,遠竄揚越。而續蟻封海阿,跋扈王命,以夷狄不足為君邪?何無上之甚也!國有常刑,於分甘乎?」續對曰:「晉末饑亂,奔控無所,保合鄉宗,庶全老幼。屬大王龍飛之始,委命納質,精誠無感,不蒙慈恕。言
 歸遺晉,仍荷寵授,誓盡忠節,實無二心。且受彼厚榮,而復二三其趣者,恐亦不容於明朝矣。周文生于東夷,大禹出於西羌,帝王之興,蓋惟天命所屬,德之所招,當何常邪!伏惟大王聖武自天,道隆虞夏,凡在含生,孰不延首神化,恥隔皇風,而況囚乎!使囚去真即偽,不得早叩天門者,大王負囚,囚不負大王也。釁鼓之刑,囚之恒分,但恨天實為之,謂之何哉!」勒曰:「其言慨至,孤愧之多矣。夫忠於其君者,乃吾所求也。」命張寶延之于館,厚撫之,尋以為從事中郎。今自後諸剋敵擒俊,皆送之,不得輒害,冀獲如續之流。



 初,季龍之攻續也,朝廷有王敦之逼,
 不遑救恤。續既為勒所執,身灌園鬻菜,以供衣食。勒屢遣察之,歎曰;「此真高人矣。不如是,安足貴乎!」嘉其清苦,數賜穀帛。每臨朝嗟歎,以勵群官。



 續被獲之後,存及竺、緝等與匹磾嬰城距寇,而帝又假存揚武將軍、武邑太守。勒屢遣季龍攻之,戰守疲苦,不能自立。久之,匹磾及其弟文鴦與竺、緝等悉見獲,惟存得潰圍南奔,在道為賊所殺。續竟亦遇害。



 李矩,字世迴,平陽人也。童齔時,與群兒聚戲,便為其率,計畫指授,有成人之量。及長,為吏,送故縣令於長安,征
 西將軍梁王肜以為牙門。伐氐齊萬年有殊功,封東明亭侯。還為本郡督護。太守宋胄欲以所親吳畿代之,矩謝病去。畿恐矩復還,陰使人刺矩,會有人救之,故得免。屬劉元海攻平陽,百姓奔走,矩素為鄉人所愛,乃推為塢主,東屯滎陽,後移新鄭。



 矩勇毅多權略,志在立功,東海王越以為汝陰太守。永嘉初,使矩與汝南太守袁孚率眾修洛陽千金堨,以利運漕。及洛陽不守,太尉荀籓奔陽城,衛將軍華薈奔成皋。時大饑,賊帥侯都等每略人而食之,籓、薈部曲多為所啖。矩討都等滅之,乃營護籓、薈,各為立屋宇,輸穀以給之。及籓承制,建行臺,假矩
 滎陽太守。矩招懷離散,遠近多附之。



 石勒親率大眾襲矩,矩遣老弱入山,令所在散牛馬,因設伏以待之。賊爭取牛馬。伏發,齊呼,聲動山谷,遂大破之,斬獲甚眾,勒乃退。籓表元帝,加矩冠軍將軍,軺車幢蓋,進封陽武縣侯,領河東、平陽太守。時饑饉相仍,又多疫癘,矩垂心撫恤,百姓賴焉。會長安群盜東下,所在多虜掠,矩遣部將擊破之,盡得賊所略婦女千餘人。諸將以非矩所部。欲遂留之。矩曰:「俱是國家臣妾,焉有此彼此!」乃一時遣之。



 時劉琨所假河內太守郭默為劉元海所逼,乞歸於矩,矩將使其甥郭誦迎致之,而不敢進。會劉琨遣參軍張肇,率
 鮮卑范勝等五百餘騎往長安,屬默被圍,道路不通,將還依邵續,行至矩營,矩謂肇曰:「默是劉公所授,公家之事,知無不為。」屠各舊畏鮮卑,遂邀肇為聲援,肇許之。賊望見鮮卑,不戰而走。誦潛遣輕舟濟河,使勇士夜襲懷城,掩賊留營,又大破之。默遂率其屬歸於矩。後劉聰遣從弟暢步騎三萬討矩,屯于韓王故壘,相去七里,遣使招矩。時暢卒至,矩未暇為備,遣使奉牛酒詐降于暢,潛匿精勇,見其老弱。暢不以為虞,大饗渠帥,人皆醉飽。矩謀夜襲之,兵士以賊眾,皆有懼色。矩令郭誦禱鄭子產祠曰:「君昔相鄭,惡鳥不鳴。凶胡臭羯,何得過庭!」使巫揚
 言:「東里有教,當遣神兵相助。」將士聞之,皆踴躍爭進。乃使誦及督選楊璋等選勇敢千人,夜掩暢營,獲鎧馬甚多,斬首數千級,暢僅以身免。



 先是,郭默聞矩被攻,遣弟芝率眾援之。既而聞破暢,芝復馳來赴矩。矩乃與芝馬五百匹,分軍為三道,夜追賊,復大獲而旋。



 先是,聰使其將趙固鎮洛陽,長史周振與固不協,密陳固罪。矩之破暢也,帳中得聰書,敕暢平矩訖,過洛陽,收固斬之,便以振代固。矩送以示固,固即斬振父子,遂率騎一千來降,矩還令守洛。後數月,聰遣其太子粲率劉雅生等步騎十萬屯孟津北岸,分遣雅生攻趙固於洛。固奔陽城山,
 遣弟告急,矩遣郭誦屯洛口以救之。誦使將張皮簡精卒千人夜渡河。粲候者告有兵至,粲恃其眾,不以為虞。既而誦等奄至,十道俱攻,粲眾驚擾,一時奔潰,殺傷太半,因據其營,獲其器械軍資不可勝數。及旦,粲見皮等人少,更與雅生悉餘眾攻之,苦戰二十餘日不能下。矩進救之,使壯士三千泛舟迎皮。賊臨河列陣,作長鉤以鉤船,連戰數日不得渡。矩夜遣部將格增潛濟入皮壘,與皮選精騎千餘,而殺所獲牛馬,焚燒器械,夜穴圍而出,奔武牢。聰追之,不及而退。聰因憤恚,發病而死。帝嘉其功,除矩都督河南三郡軍事、安西將軍、滎陽太守,封
 脩武縣侯。



 及劉粲嗣位,昏虐日甚,其將靳準乃起兵殺粲,并其宗族,發聰冢,斬其尸,遣使歸矩,稱「劉元海屠各小醜,因大晉事故之際,作亂幽並,矯稱天命,至令二帝幽沒虜庭。輒率眾扶侍梓宮,因請上聞」。矩馳表于帝,帝遣太常韓胤等奉迎梓宮,未至而準已為石勒、劉曜所沒。矩以眾少不足立功,每慷慨憤歎。及帝踐阼,以為都督司州諸軍事、司州刺史,改封平陽縣侯,將軍如故。時弘農太守尹安、振威將軍宋始等四軍並屯洛陽,各相疑阻,莫有固志。矩、默各遣千騎至洛以鎮之。安等乃同謀告石勒,勒遣石生率騎五千至洛陽,矩、默軍皆退還。
 俄而四將復背勒,遣使乞迎,默又遣步卒五百人入洛。石生以四將相謀,不能自安,乃虜宋始一軍,渡河而南。百姓相率歸矩,於是洛中遂空。矩乃表郭誦為揚武將軍、陽翟令,阻水築壘,且耕且守,為滅賊之計。屬趙固死,石生遣騎襲誦,誦多計略,賊至,輒設伏破之,虜掠無所得。生怒,又自率四千餘騎暴掠諸縣,因攻誦壘,接戰須臾,退軍堮阪。誦率勁勇五百追及生於磐脂故亭,又大破之。矩以誦功多,表加赤幢曲蓋,封吉陽亭侯。



 郭默欲侵祖約,矩禁之不可,遂為約所破。石勒遣其養子匆襲默,默懼後患未已,將降於劉曜,遣參軍鄭雄詣矩謀之,
 矩距而不許。後勒遣其將石良率精兵五千襲矩,矩逆擊不利。郭誦弟元復為賊所執,賊遣元以書說矩曰:「去年東平曹嶷,西賓猗盧,矩如牛角,何不歸命?」矩以示誦,誦曰:「昔王陵母在賊,猶不改意,弟當何論!」勒復遺誦麈尾馬鞭,以示殷勤,誦不答。勒將石生屯洛陽,大掠河南,矩、默大飢,默因復說矩降曜。矩既為石良所破遂,從默計,遣使於曜。曜遣從弟岳軍於河陰,欲與矩謀攻石生。勒遣將圍岳,岳閉門不敢出。默後為石匆所敗,自密南奔建康。矩聞之大怒,遣其將郭誦等齎書與默,又敕誦曰:「汝識脣亡之談不?迎接郭默,皆由於卿,臨難逃走,其
 必留之。」誦追及襄城,默自知負矩,棄妻子而遁。誦擁其餘眾而歸,矩待其妻子如初。劉岳以外援不至,降於石季龍。



 矩所統將士有陰欲歸勒者,矩知之而不能討,乃率眾南走,將歸朝廷,眾皆道亡,惟郭誦及參軍郭方,功曹張景,主簿茍遠,將軍騫韜、江霸、梁志、司馬尚、季弘、李瑰、段秀等百餘人棄家送矩。至於魯陽縣,矩墜馬卒,葬襄陽之峴山。



 段匹磾,東部鮮卑人也。種類勁健,世為大人。父務勿塵,遣軍助東海王越征討有功,王浚表為親晉王,封遼西
 公,嫁女與務勿塵,以結鄰援。懷帝即位,以務勿塵為大單于,匹磾為左賢王,率眾助國征討,假撫軍大將軍。務勿塵死,弟涉復辰以務勿塵子疾陸眷襲號。



 劉曜逼洛陽,王浚遣督護王昌等率疾陸眷及弟文鴦、從弟末杯攻石勒於襄國。勒敗還壘,末杯追入壘門,為勒所獲。勒質末杯,遣使求和於疾陸眷,疾陸眷將許之,文鴦諫曰:「受命討勒,寧以末杯一人,故縱成擒之冠?既失浚意,且有後憂,必不可許。」疾陸眷不聽,以鎧馬二百五十匹、金銀各一簏贈末杯。勒歸之,又厚以金寶採絹報疾陸眷。疾陸眷令文鴦與石季龍同盟,約為兄弟,遂引騎還。昌
 等不能獨守,亦還。



 建武初,匹磾推劉琨為大都督,結盟討勒,並檄涉復辰、疾陸眷、末杯等三面俱集襄國,琨、匹磾進屯固安,以候眾軍。勒懼,遣間使厚賂末杯。然末杯既思報其舊恩,且因匹磾在外,欲襲奪其國,乃間匹磾於涉復辰、疾陸眷曰:「以父兄而從子弟邪?雖一旦有功,匹磾獨收之矣。」涉復辰等以為然,引軍而還。匹磾亦止。會疾陸眷病死,匹磾自薊奔喪,至于右北平。末杯宣言匹磾將篡,出軍擊敗之。末杯遂害涉復辰及其子弟黨與二百餘人,自立為單于。



 及王浚敗,匹磾領幽州刺史,劉琨自并州依之,復與匹磾結盟,俱討石勒。匹磾復為
 末杯所敗,士眾離散,懼琨圖己,遂害之,於是晉人離散矣。匹磾不能自固,北依邵續,末杯又攻敗之。匹磾被瘡,謂續曰:「吾夷狄慕義,以至破家,君若不忘舊要,與吾進討,君之惠也。」續曰:「賴公威德,續得效節。今公有難,豈敢不俱!遂並力追末杯,斬獲略盡。又令文鴦北討末杯弟於薊城,及還,去城八十里,聞續已沒,眾懼而散,復為石季龍所遮,文鴦以其親兵數百人力戰破之,始得入城。季龍復抄城下,文鴦登城臨見,欲出擊之,匹磾不許。文鴦曰:「我以勇聞,故百姓杖我。見人被略而不救,非丈夫也。令眾失望,誰復為我致死乎!」遂將壯士數十騎出戰,
 殺胡甚多。遇馬乏,伏不能起。季龍呼曰:「大兄與我俱是戎狄,久望共同。天不違願,今日相見,何故復戰?請釋杖。」文鴦罵曰:「汝為寇虐,久應合死,吾兄不用吾計,故令汝得至此,吾寧死,不為汝擒。」遂下馬苦戰,槊折,執刀力戰不已。季龍軍四面解馬羅披自鄣,前捉文鴦。文鴦戰自辰至申,力極而後被執。城內大懼。



 匹磾欲單騎歸朝,續弟樂安內史洎協兵,不許,洎復欲執臺使王英送於季龍,匹磾正色責之曰:「卿不能遵兄之志,逼吾不得歸朝,亦以甚矣,復欲執天子使者,我雖胡素,所未聞也。」因謂英曰:「匹磾世受重恩,不忘忠孝。今日事逼,欲歸罪朝廷,
 而見逼迫,忠款不遂。若得假息,未死之日,心不忘本。」遂渡黃河南。匹磾著朝服,持節,賓從出見季龍曰:「我受國恩,志在滅汝。不幸吾國自亂,以至於此。既不能死,又不能為汝敬也。」勒及季龍素與匹磾結為兄弟,季龍起而拜之。匹磾到襄國,又不為勒禮,常著朝服,持晉節。經年,國中謀推匹磾為主,事露,被害。文鴦亦遇鴆而死,惟末波存焉。及死,弟牙立。牙死,其後從祖就陸眷之孫遼立。



 自務勿塵已後,值晉喪亂,自稱位號,據有遼西之地,而臣御晉人。其地西盡幽州,東界遼水。然所統胡晉可三萬餘家,控弦可四五萬騎,而與石季龍遞相侵掠,連兵不
 息,竟為季龍所破,徙其遺黎數萬家於司雍之地。其子蘭復聚兵,與季龍為患久之。及石氏之亡,末波之子勤鳩集胡羯得萬餘人,保枉人山,自稱趙王,附于慕容俊。俄為冉閔所敗,徙于繹幕,僭即尊號。俊遣慕容恪擊之,勤懼而降。



 魏浚,東郡東阿人也,寓居關中。初為雍州小吏,河間王顒敗亂之際,以為武威將軍。後為度支校尉,有乾用。永嘉末,與流人數百家東保河陰之硤石。時京邑荒儉,浚劫掠得穀麥,獻之懷帝,帝以為揚威將軍、平陽太守,度
 支如故。以亂不之官。及洛陽陷,屯于洛北石梁塢,撫養遺眾,漸修軍器。其附賊者,皆先解喻,說大晉運數靈長,行已建立,歸之者甚眾。其有恃遠不從命者,遣將討之,服從而已,不加侵暴。於是遠近感悅,襁負至者漸眾。劉琨承制,假浚河南尹。時太尉荀籓建行臺在密縣,浚詣籓諮謀軍事,籓甚悅,要李矩同會。矩將夜赴之,矩官屬以浚不可信,不宜夜往。矩曰:「忠臣同心,將何疑乎!」及會,客主盡嘆,浚因與矩相結而去。劉曜忌浚得眾,率眾軍圍之。劉演、郭默遣軍來救,曜分兵逆於河北,乃伏兵深隱處,以邀演、默軍,大破之,盡虜演等騎。浚夜遁走,為曜
 所得,遂死之。追贈平西將軍。族子該領其眾。



 該一名亥,本僑居京兆陰磐。河間王顒之伐趙王倫,以該為將兵都尉。及劉曜攻洛陽,隨浚赴難,先領兵守金墉城,故得無他。曜引去,餘眾依之。



 時杜預子尹為弘農太守,屯宜陽界一泉塢,數為諸賊所抄掠。尹要該共距之,該遣其將馬瞻將三百人赴尹。瞻知其無備,夜襲尹殺之,迎該據塢。塢人震懼,並服從之。乃與李炬、郭默相結以距賊。荀籓即以該為武威將軍,統城西雍涼人,使討劉曜。元帝承制,加冠軍將軍、河東太守。督護河東、河南、平陽三郡。



 曜嘗攻李矩,該破之。及矩將迎郭默,該遣
 軍助之,又與河南尹任愔相連結。後漸饑弊,曜寇日至,欲率眾南徙,眾不從,該遂單騎走至南陽。帝又以為前鋒都督、平北將軍、雍州刺史。馬瞻率該餘眾降曜。曜徵發既苦,瞻又驕虐,部曲遣使呼該,該密往赴之,其眾殺瞻而納該。該遷於新野,率眾助周訪討平杜曾,詔以該為順陽太守。



 王敦之反也,梁州刺史甘卓不從,欲觀該去就,試以敦旨動之。該曰:「我本去賊,惟忠於國。今王公舉兵向天子,非吾所宜與也。」遂距而不應。及蘇峻反,率眾救臺,軍次石頭,受陶侃節度。峻未平,該病篤還屯,卒於道,葬於武陵。從子雄統其眾。



 郭默,河內懷人。少微賤,以壯勇事太守裴整,為督將。永嘉之亂,默率遺眾自為塢主,以漁舟抄東歸行旅,積年遂致巨富,流人依附者漸眾。撫循將士,甚得其歡心。默婦兄同郡陸嘉取官米數石餉妹,默以為違制,將殺嘉,嘉懼,奔石勒。默乃自射殺婦,以明無私。遣使謁劉琨,琨加默河內太守。劉元海遣從子曜討默,曜列三屯圍之,欲使餓死。默送妻子為質,並請糴焉,糴畢,設守。曜怒,沈默妻子于河而攻之。默遣弟芝求救於劉琨,琨知默狡猾,留之而緩其救。默更遣人告急。會芝出城浴馬,使強
 與俱歸。默乃遣芝質於石勒,勒以默多詐,封默書與劉曜。默使人伺得勒書,便突圍投李矩。後與矩並力距劉、石,事見矩傳。



 太興初,除潁川太守。默與石匆戰敗,矩轉蹙弱,默深憂懼,解印授其參軍殷嶠,謂之曰:「李使君遇吾甚厚,今遂棄去,無顏謝之,三日可白吾去也。」乃奔陽翟。矩聞之,大怒,遣其將郭誦追默,至襄城,及之。默棄家人,單馬馳去。默至京都,明帝授征虜將軍。劉遐卒,以默為北中郎將、監淮北軍事、假節。遐故部曲李龍等謀反,詔默與右衛將軍趙胤討平之。



 朝廷將征蘇峻,懼其為亂,召默拜後將軍,領屯騎校尉。初戰有功,及六軍敗績,南
 奔。郗鑒議於曲阿北大業里作壘,以分賊勢,使默守之。峻遣韓晁等攻默甚急,壘中頗乏水,默懼,分人馬出外,乃潛從南門盪出,留人堅守。會峻死,圍解,徵為右軍將軍。



 默樂為邊將,不願宿衛,及赴召,謂平南將軍劉胤曰:「我能禦胡而不見用。右軍主禁兵,若疆場有虞,被使出征,方始配給,將卒無素,恩信不著,以此臨敵,少有不敗矣。時當為官擇才,若人臣自擇官,安得不亂乎」胤曰:「所論事雖然,非小人所及也。」當發,求資於胤。時胤被詔免官,不即歸罪,方自申理,而驕侈更甚,遠近怪之。



 初,默之被徵距蘇峻也,下次尋陽,見胤,胤參佐張滿等輕默,惈
 露視之,默常切齒。至是,胤臘日餉默酒一器,肫一頭,默對信投之水中,忿憤益甚。又僑人蓋肫先略取祖煥所殺孔煒女為妻,煒家求之,張滿等使還其家,肫不與,因與胤、滿有隙。至是,肫謂默曰:「劉江州不受免,密有異圖,與長史司馬張滿、荀楷等日夜計謀,反逆已形,惟忌郭侯一人,云當先除郭侯而後起事。禍將至矣,宜深備之。」默既懷恨,便率其徒候旦門開襲胤。胤將吏欲距默,默句之曰:「我被詔有所討,動者誅及三族。」遂入至內寢。胤尚與妾臥,默牽下斬之。出取胤僚佐張滿、荀楷等,誣以大逆。傳胤首於京師,詐作詔書,宜視內外。掠胤女及諸
 妾,并金寶還船。初云下都,俄而還,停胤故府,招桓宣、王愆期。愆期懼逼,勸默為平南、江州,默從之。愆期因逃廬山,桓宣固守不應。



 司徒王導懼不可制,乃大赦天下,梟胤首於大航,以默為西中郎將、豫州刺史。武昌太守鄧嶽馳白太尉陶侃,侃聞之,投袂起曰:「此必詐也。」即日率眾討默,上疏陳默罪惡。導聞之,乃收胤首,詔庾亮助侃討默。默欲南據豫章,而侃已至城下築土山以臨之。諸軍大集,圍之數重。侃惜默驍勇,欲活之,遣郭誦見默,默許降,而默將張丑、宋侯等恐為侃所殺,故致進退,不時得出。攻之轉急,宋侯遂縛默求降,即斬于軍門,同黨死
 者四十人,傳首京師。



 史臣曰:邵、李、魏、郭等諸將,契闊喪亂之辰,驅馳戎馬之際,威懷足以容眾,勇略足以制人,乃保據危城,折衝千里,招集義勇,抗禦仇讎,雖艱阻備嘗,皆乃心王室。而矩能以少擊眾,戰勝獲多,遂使玄明憤恚,世龍挫衄。惜其寡弱,功虧一簣。方之數子,其最優乎!默既拔迹危亡,參陪朝伍,忿因眥睚,禍及誅夷,非夫狂悖,豈宜至此!段匹磾本自遐方,而係心朝廷,始則盡忠國難,終乃抗節虜廷,自蘇子卿以來,一人而已。越石之見誅段氏,實以威名;匹磾之取戮世龍,亦由眾望:禍福之應,何其速哉!《詩》
 云:「無言不酬,無德不報」,此之謂也。



 贊曰:邵李諸將,實惟忠壯。蒙犯艱危,驅馳亭鄣。力小任重,功虧身喪。匹磾勁烈,隕身全節。默實兇殘,自貽罪戾。



\end{pinyinscope}