\article{載記第九 慕容皝}

\begin{pinyinscope}

 慕容皝



 慕容皝,字元真,廆第三子也。龍顏版齒,身長七尺八寸。雄毅多權略,尚經學,善天文。廆為遼東公,立為世子。建武初,拜為冠軍將軍、左賢王,封望平侯,率眾征討,累有功。太寧末,拜平北將軍,進封朝鮮公。廆卒,嗣位,以平北將軍行平州刺史,督攝部內。尋而宇文乞得龜為其別部逸豆歸所逐,奔死于外,皝率騎討之,逸豆歸懼而請
 和,遂築榆陰、安晉二城而還。



 初,皝庶兄建威翰驍武有雄才,素為皝所忌,母弟征虜仁、廣武昭並有寵于廆,皝亦不平之。及廆卒,並懼不自容。至此,翰出奔段遼,仁勸昭舉兵廢皝。皝殺昭,遣使按檢仁之虛實,遇仁於險瀆。仁知事發,殺皝使,東歸平郭。皝遣其弟建武幼、司馬佟壽等討之。仁盡眾距戰,幼等大敗,皆沒於仁。襄平令王冰、將軍孫機以遼東叛于皝,東夷校尉封抽、護軍乙逸、遼東相韓矯、玄菟太守高詡等棄城奔還。仁於是盡有遼左之地,自稱車騎將軍、平州刺史、遼東公。宇文歸、段遼及鮮卑諸部並為之援。



 咸和九年,皝遣其司馬封弈
 攻鮮卑木堤于白狼,揚威淑虞攻烏丸悉羅侯於平岡,皆斬之。材官劉佩攻乙連,不剋。段遼遂寇徒河,皝將張萌逆擊,敗之。遼弟蘭與翰寇柳城,都尉石琮擊敗之。旬餘,蘭、翰復圍柳城,皝遣寧遠慕容汗及封弈等救之。皝戒汗曰:「賊眾氣銳,難與爭鋒,宜顧萬全,慎勿輕進,必須兵集陣整,然後擊之。」汗性驍銳,遣千餘騎為前鋒而進,封弈止之,汗不從,為蘭所敗,死者大半。蘭復攻柳城,為飛梯、地道,圍守二旬,石琮躬勒將士出擊,敗之,斬首千五百級,蘭乃遁歸。



 是歲,成帝遣謁者徐孟、閭丘幸等持節拜皝鎮軍大將軍、平州刺史、大單于、遼東公,持節、都
 督、承制封拜,一如廆故事。



 皝自征遼東,剋襄平。仁所署居就令劉程以城降,新昌人張衡執縣宰以降。於是斬仁所置守宰,分徙遼東大姓於棘城,置和陽、武次、西樂三縣而歸。



 咸康初,遣封弈襲宇文別部涉奕於,大獲而還。涉奕于率騎追戰於渾水,又敗之。皝將乘海討仁,群下咸諫,以海道危陰,宜從陸路。皝曰:「舊海水無凌,自仁反已來,凍合者三矣。昔漢光武因滹沱之冰以濟大業,天其或者欲吾乘此而無之乎!吾計決矣,有沮謀者斬!」乃率三軍從昌黎踐凌而進。仁不虞皝之至也,軍去平郭七里,候騎乃告,仁狼狽出戰,為皝所擒,殺仁而還。



 立
 藉田於朝陽門東,置官司以主之。



 段遼遣將李詠夜襲武興,遇雨,引還,都尉張萌追擊,擒詠。段蘭擁眾數萬屯於曲水亭,將攻柳城,宇文歸入寇安晉,為蘭聲援。皝以步騎五萬擊之,師次柳城,蘭、歸皆遁。遣封弈率輕騎追擊,敗之,收其軍實,館穀二旬而還。謂諸將曰:「二虜恥無功而歸,必復重至,宜於柳城左右設伏以待之。」遣封弈率騎潛于馬兒山諸道。俄而遼騎果至,弈夾擊,大敗之,斬其將榮保。遣兼長史劉斌、郎中令陽景送徐孟等歸于京師。使其世子俊伐段遼諸城,封弈攻宇文別部,皆大捷而歸。



 立納諫之木,以開讜言之路。



 後徙昌黎郡,
 築好城於乙連東,使將軍蘭勃戍之,以逼乙連。又城曲水,以為勃援。乙連飢甚,段遼輸之粟,蘭勃要擊獲之。遼遣將屈雲攻興國,與皝將慕容遵大戰於五官水上,雲敗,斬之,盡俘其眾。



 封弈等以皝任重位輕,宜稱燕王,皝於是以咸康三年僭即王位,赦其境內。以封弈為國相,韓壽為司馬,裴開、陽騖、王寓、李洪、杜群、宋該、劉瞻、石琮、皇甫真、陽協、宋晃、平熙、張泓等並為列卿將帥。起文昌殿,乘金根車,駕六馬,出入稱警蹕。以其妻段氏為王后,世子俊為太子,皆如魏武、晉文輔政故事。



 皝以段遼屢為邊患,遣將軍宋回稱籓于石季龍,請師討遼。季龍於
 是總眾而至。皝率諸軍攻遼令支以北諸城,遼遣其將段蘭來距,大戰,敗之,斬級數千,掠五千餘戶而歸。季龍至徐無,遼奔密雲山。季龍進入令支,怒皝之不會師也,進軍擊之,至于棘城,戎卒數十萬,四面進攻,郡縣諸部叛應季龍者三十六城。相持旬餘,左右勸皝降。皝曰:「孤方取天下,何乃降人乎!」遣子恪等率騎二千,晨出擊之。季龍諸軍驚擾,棄甲而遁。恪乘勝追之,斬獲三萬餘級,築戍凡城而還。段遼遣使詐降於季龍,請兵應接。季龍遣其將麻秋率眾迎遼,恪伏精騎七千於密雲山,大敗之,獲其司馬陽裕、將軍鮮于亮,擁段遼及其部眾以歸。



 帝又遣使進皝為征北大將軍、幽州牧,領平州刺史,加散騎常侍,增邑萬戶,持節、都督、單于、公如故。



 皝前軍帥慕容評敗季龍將石成等於遼西,斬其將呼延晃、張支,掠千餘戶以歸。段遼謀叛,皝誅之。



 季龍又使石成入攻凡城,不剋,進陷廣城。皝雖稱燕王,未有朝命,乃遣其長史劉祥獻捷京師,兼言權假之意,並請大舉討平中原。又聞庾亮薨,弟冰、翼繼為將相,乃表曰:



 臣究觀前代昏明之主,若能親賢並建,則功致升平;若親黨后族,必有傾辱之禍。是以周之申伯號稱賢舅,以其身籓于外,不握朝權。降及秦昭,足為令主,委信二舅,幾至亂國。逮于
 漢武,推重田蚡,萬機之要,無不決之。及蚡死後,切齒追恨。成帝闇弱,不能自立,內惑艷妻,外恣五舅,卒令王莽坐取帝位。每覽斯事,孰不痛惋!設使舅氏賢若穰侯、王鳳,則但聞有二臣,不聞有二主。若其不才,則有竇憲、梁冀之禍。凡此成敗,亦既然矣。茍能易軌,可無覆墜。



 陛下命世天挺,當隆晉道,而遭國多難,殷憂備嬰,追述往事,至今楚灼。迹其所由,實因故司空亮居元舅之尊,勢業之重,執政裁下,輕侮邊將,故令蘇峻、祖約不勝其忿,遂致敗國。至今太后發憤,一旦升遐。若社稷不靈,人神無助,豺狼之心當可極邪!前事不忘,後事之表,而中書監、
 左將軍冰等內執樞機,外擁上將,昆弟並列,人臣莫疇。陛下深敦渭陽,冰等自宜引領。臣常謂世主若欲崇顯舅氏,何不封以籓國,豐其祿賜,限其勢利,使上無偏優,下無私論。如此,榮辱何從而生!噂沓何辭而起!往者惟亮一人,宿有名望,尚致世變,況今居之者素無聞焉!且人情易惑,難以戶告,縱今陛下無私於彼,天下之人誰謂不私乎!



 臣與冰等名位殊班,出處懸邈,又國之戚暱,理應降悅,以適事會。臣獨矯抗此言者,上為陛下,退為冰計,疾茍容之臣,坐鑒得失。顛而不扶,焉用彼相!昔徐福陳霍氏之戒,宣帝不從,至令忠臣更為逆族,良由察
 之不審,防之無漸。臣今所陳,可謂防漸矣。但恐陛下不明臣之忠,不用臣之計,事過之日,更處焦爛之後耳。昔王章、劉向每上封事,未嘗不指斥王氏,故令二子或死或刑。谷永、張禹依違不對,故容身茍免,取譏於世。臣被髮殊俗,位為上將,夙夜惟憂,罔知所報,惟當外殄寇仇,內盡忠規,陳力輸誠,以答國恩。臣若不言,誰當言者!



 又與冰書曰:



 君以椒房之親,舅氏之暱,總據樞機,出內王命,兼擁列將州司之位,昆弟網羅,顯布畿甸。自秦、漢以來,隆赫之極,豈有若此者乎!以吾觀之,若功就事舉,必享申伯之名;如或不立,將不免梁竇之迹矣。



 每睹史傳,
 未嘗不寵恣母族,使執權亂朝,先有殊世之榮,尋有負乘之累,所謂愛之適足以為害。吾常忿歷代之主,不盡防萌終寵之術,何不業以一土之封,令籓國相承,如周之齊、陳?如此則永保南面之尊,復何黜辱之憂乎!竇武、何進好善虛己。賢士歸心,雖為閹豎所危,天下嗟痛,猶有能履以不驕,圖國亡身故也。



 方今四海有倒懸之急,中夏逋僭逆之寇,家有漉血之怨,人有復仇之憾,寧得安枕逍遙,雅談卒歲邪!吾雖寡德,過蒙先帝列將之授,以數郡之人,尚欲並吞彊虜,是以自頃迄今,交鋒接刃,一時務農,三時用武,而猶師徒不頓,倉有餘粟,敵人日
 畏,我境日廣,況乃王者之威,堂堂之勢,豈可同年而語哉!



 冰見表及書甚懼,以其絕遠,非所能制,遂與何充等奏聽皝稱燕王。



 其年皝伐高句麗,王釗乞盟而還。明年,釗遣其世子朝於皝。



 初,段遼之敗也,建威翰奔于宇文歸,自以威名夙振,終不保全,乃陽狂恣酒,被髮歌呼。歸信而不禁,故得周游自任,至於山川形便,攻戰要路,莫不練之。皝遣商人王車陰使察翰,翰見車無言,撫膺而已。車還以白,皝曰:「翰欲來也。」乃遣車遺翰弓矢,翰乃竊歸駿馬,攜其二子而還。



 皝將圖石氏,從容謂諸將曰:「石季龍自以安樂諸城守防嚴重,城之南北必不設備,今
 若詭路出其不意,冀之北土盡可破也。」於是率騎二萬出蠮螉塞,長驅至于薊城,進渡武遂津,入于高陽,所過焚燒積聚,掠徙幽、冀三萬餘戶。



 使陽裕、唐柱等築龍城,構宮廟,改柳城為龍城縣。於是成帝使兼大鴻臚郭希持節拜皝侍中、大都督河北諸軍事、大將軍、燕王,其餘官皆如故。封諸功臣百餘人。



 咸康七年,皝遷都龍城。率勁卒四萬,入自南陜,以伐宇文、高句麗,又使翰及子垂為前鋒,遣長史王寓等勒眾萬五千,從北置而進。高句麗王釗謂皝軍之從北路也,乃遣其弟武統精銳五萬距北置,躬率弱卒以防南陜。翰與釗戰于木底,大敗之,
 乘勝遂入丸都,釗單馬而遁。皝掘釗父利墓,載其尸並其母妻珍寶,掠男女五萬餘口,焚其宮室,毀丸都而歸。明年,釗遣使稱臣於皝,貢其方物,乃歸其父尸。



 宇文歸遣其國相莫淺渾伐皝,諸將請戰,皝不許。渾以皝為憚之,荒酒縱獵,不復設備。皝曰:「渾奢忌已甚,今則可一戰矣。」遣翰率騎擊之,渾大敗,僅以身免,盡俘其眾。



 皝躬巡郡縣,勸課農桑,起龍城宮闕。



 尋又率騎二萬親伐宇文歸,以翰及垂為前鋒。歸使其騎將涉奕于盡眾距翰,皝馳遣謂翰曰:「奕于雄悍,宜小避之,待虜勢驕,然後取也。」翰曰:「歸之精銳,盡在於此,今若剋之,則歸可不勞兵而滅。奕
 于徒有虛名,其實易與耳,不宜縱敵挫吾兵氣。」於是前戰,斬奕于,盡俘其眾,歸遠遁漠北。皝開地千餘里,徙其部人五萬餘落於昌黎,改涉奕于城為威德城。行飲至之禮,論功行賞各有差。



 以牧牛給貧家,田于苑中,公收其八,二分入私。有牛而無地者,亦田苑中,公收其七,三分入私。皝記室參軍封裕諫曰:



 臣聞聖王之宰國也,薄賦而藏于百姓,分之以三等之田,十一而稅之;寒者衣之,飢者食之,使家給人足。雖水旱而不為災者,何也?高選農官,務盡勸課,人治周田百畝,亦不假牛力;力田者受旌顯之賞,惰農者有不齒之罰。又量事置官,量官
 置人,使官必稱須,人不虛位,度歲入多少,裁而祿之。供百僚之外,藏之太倉,三年之耕,餘一年之粟。以斯而積,公用於何不足?水旱其如百姓何!雖務農之令屢發,二千石令長莫有志勤在公、銳盡地利者。故漢祖知其如此,以墾田不實,徵殺二千石以十數,是以明、章之際,號次升平。



 自永嘉喪亂,百姓流亡,中原蕭條,千里無煙,飢寒流隕,相繼溝壑。先王以神武聖略,保全一方,威以殄姦,德以懷遠,故九州之人,塞表殊類,襁負萬里,若赤子之歸慈父,流人之多舊土十倍有餘,人殷地狹,故無田者十有四焉。殿下以英聖之資,克廣先業,南摧彊趙,東
 滅句麗,開境三千,戶增十萬,繼武闡廣之功,有高西伯。宜省罷諸苑,以業流人。人至而無資產者,賜之以牧牛。人既殿下之人,牛豈失乎!善藏者藏於百姓,若斯而已矣。邇者深副樂土之望,中國之人皆將壺餐奉迎,石季龍誰與居乎!且魏、晉雖道消之世,猶削百姓不至於七八,持官牛田者官得六分,百姓得四分,私牛而官田者與官中分,百姓安之,人皆悅樂。臣猶曰非明王之道,而況增乎!且水旱之厄,堯、湯所不免,王者宜濬治溝澮,循鄭白、西門、史起溉灌之法,旱則決溝為雨,水則入于溝瀆,上無《雲漢》之憂,下無昏墊之患。



 句麗、百濟及宇文、段
 部之人,皆兵勢所徙,非如中國慕義而至,咸有思歸之心。今戶垂十萬,狹湊都城,恐方將為國家深害,宜分其兄弟宗屬,徙于西境諸城,撫之以恩,檢之以法,使不得散在居人,知國之虛實。



 今中原未平,資畜宜廣,官司猥多,游食不少,一夫不耕,歲受其飢。必取於耕者而食之,一人食一人之力,游食數萬,損亦如之,安可以家給人足,治致升平!殿下降覽古今之事多矣,政之巨患莫甚於斯。其有經略出世,才稱時求者,自可隨須置之列位。非此已往,其耕而食,蠶而衣,亦天之道也。



 殿下聖性寬明,思言若渴,故人盡芻蕘,有犯無隱。前者參軍王憲、大
 夫劉明並竭忠獻款,以貢至言,雖頗有逆鱗,意在無責。主者奏以妖言犯上,至之於法,殿下慈弘苞納,恕其大辟,猶削黜禁錮,不齒於朝。其言是也,殿下固宜納之;如其非也,宜亮其狂狷。罪諫臣而求直言,亦猶北行詣越,豈有得邪!右長史宋該等阿媚茍容,輕劾諫士,己無骨鯁,嫉人有之,掩蔽耳目,不忠之甚。



 四業者國之所資,教學者有國盛事。習戰務農,尤其本也。百工商賈,猶其末耳。宜量軍國所須,置其員數,已外歸之於農,教之戰法,學者三年無成,亦宜還之於農,不可徒充大員,以塞聰俊之路。



 臣之所言當也,願時速施行;非也,登加罪戮,使
 天下知朝廷從善如流,罰惡不淹。王憲、劉明,忠臣也,願宥忤鱗之愆,收其藥石之效。



 皝乃令曰:「覽封記室之諫,孤實懼焉。君以黎元為國,黎元以穀為命。然則農者,國之本也,而二千石令長不遵孟春之令,惰農弗勸,宜以尤不脩闢者措之刑法,肅厲屬城。主者明詳推檢,具狀以聞。苑囿悉可罷之,以給百姓無田業者。貧者全無資產,不能自存,各賜牧牛一頭。若私有餘力,樂取官牛墾官田者,其依魏、晉舊法。溝洫溉灌,有益官私,主者量造,務盡水陸之勢。中州未平,兵難不息,勳誠既多,官僚不可以減也。待剋平凶醜,徐更議之。百工商賈數,四佐與
 列將速定大員,餘者還農。學生不任訓教者,亦除員錄。夫人臣關言於人主,至難也,妖妄不經之事皆應蕩然不問,擇其善者而從之。王憲、劉明雖其罪應禁黜,亦猶孤之無大量也。可悉復本官,仍居諫司。封生蹇蹇,深得王臣之體。《詩》不云乎:『無言不酬。』其賜錢五萬,明宣內外,有欲陳孤過者,不拘貴賤,勿有所諱。」



 時有黑龍、白龍各一,見於龍山,皝親率群僚觀之,去龍二百餘步,祭以太宰。二龍交首嬉翔,解角而去。皝大悅,還宮,赦其境內,號新宮曰和龍,立龍翔佛寺于山上。



 賜其大臣子弟為官學生者號高門生,立東庠于舊宮,以行鄉射之禮,每月
 臨觀,考試優劣。皝雅好文籍,勤於講授,學徒甚盛,至千餘人。親造《太上章》以代《急就》,又著《典誡》十五篇,以教胄子。



 慕容恪攻高句麗南蘇,克之,置戍而還。三年,遣其世子俊與恪率騎萬七千東襲夫餘,剋之,虜其王及部眾五萬餘口以還。



 皝親臨東庠考試學生,其經通秀異者,擢充近侍。以久旱,丐百姓田租。罷成周、冀陽、營丘等郡。以勃海人為興集縣,河間人為寧集縣,廣平、魏郡人為興平縣,東萊、北海人為育黎縣,吳人為吳縣,悉隸燕國。



 皝嘗畋于西鄙,將濟河,見一父老,服硃衣,乘白馬,舉手麾皝曰:「此非獵所,王其還也。」秘之不言,遂濟河,連日大
 獲。後見白兔,馳射之,馬倒被傷,乃說所見。輦而還宮,引俊屬以後事。以永和四年死,在位十五年,時年五十二。俊僭號,追謚文明皇帝。



 慕容翰,字元邕,廆之庶長子也。性雄豪,多權略,猿臂工射,膂力過人。廆甚奇之,委以折衝之任。行師征伐,所在有功,威聲大振,為遠近所憚。作鎮遼東,高句麗不敢為寇。善撫接,愛儒學,自士大夫至於卒伍,莫不樂而從之。



 及奔段遼,深為遼所敬愛。柳城之敗,段蘭欲乘勝深入,翰慮成本國之害,詭說於蘭,蘭遂不進。後石季龍征遼,皝親將三軍略令支以北,遼議欲追之,翰知皝躬自總
 戎,戰必克勝,乃謂遼曰:「今石氏向至,方對大故,不宜復以小小為事。燕王自來,士馬精銳。兵者凶器,戰有危慮,若其失利,何以南禦乎!」蘭怒曰:「吾前聽卿誑說,致成今患,不復入卿計中矣。」乃率眾追皝,蘭果大敗。翰雖處仇國,因事立忠,皆此類也。



 及遼奔走,翰又北投宇文歸。既而逃,歸乃遣勁騎百餘追之。翰遙謂追者曰:「吾既思戀而歸,理無反面。吾之弓矢,汝曹足知,無為相逼,自取死也。吾處汝國久,恨不殺汝。汝可百步豎刀,吾射中者,汝便宜反;不中者,可來前也。」歸騎解刀豎之,翰一發便中刀鐶,追騎乃散。



 既至,皝甚加恩禮。建元二年,從皝討宇
 文歸,臨陣為流矢所中,臥病積時。後疾漸愈,於其家中騎馬自試,或有人告翰私習騎,疑為非常。皝素忌之,遂賜死焉。翰臨死謂使者曰:「翰懷疑外奔,罪不容誅,不能以骸骨委賊庭,故歸罪有司。天慈曲愍,不肆之市朝,今日之死,翰之生也。但逆胡跨據神州,中原未靖,翰常剋心自誓,志吞醜虜,上成先王遺旨,下謝山海之責。不圖此心不遂,沒有餘恨,命也奈何!」仰藥而死。



 陽裕,字士倫,右北平無終人也。少孤,兄弟皆早亡,單煢獨立,雖宗族無能識者,惟叔父耽幼而奇之,曰:「此兒非
 惟吾門之標秀,乃佐時之良器也。」刺史和演辟為主簿。王浚領州,轉治中從事,忌而不能任。



 石勒既剋薊城,問棗嵩曰:「幽州人士,誰最可者?」嵩曰:「燕國劉翰,德素長者。北平陽裕,幹事之才。」勒曰:「若如君言,王公何以不任?」嵩曰:「王公由不能任,所以為明公擒也。」勒方任之,裕乃微服潛遁。



 時鮮卑單於段眷為晉驃騎大將軍、遼西公,雅好人物,虛心延裕。裕謂友人成泮曰:「仲尼喜佛肸之召,以匏瓜自喻,伊尹亦稱何事非君,何使非民,聖賢尚如此,況吾曹乎!眷今召我,豈徒然哉!」泮曰:「今華夏分崩,九州幅裂,軌迹所及,易水而已。欲偃蹇考槃,以待大通者,俟
 河之清也。人壽幾何?古人以為白駒之歎。少游有云,郡掾足以蔭後,況國相乎!卿追蹤伊、孔,抑亦知機其神也。」裕乃應之。拜郎中令、中軍將軍,處上卿位。歷事段氏五主,甚見尊重。



 段遼與皝相攻,裕諫曰:「臣聞親仁善鄰,國之寶也。慕容與國世為婚姻,且皝令德之主,不宜連兵構怨,凋殘百姓。臣恐禍害之興,將由於此。願兩追前失,通款如初,使國家有太山之安,蒼生蒙息肩之惠。」遼不從。出為燕郡太守。石季龍剋令支,裕以郡降,拜北平太守,徵為尚書左丞。



 段遼之請迎於季龍也,裕以左丞領征東麻秋司馬。秋敗,裕為軍人所執,將詣皝。皝素聞
 裕名,即命釋其囚,拜郎中令,遷大將軍左司馬。東破高句麗,北滅宇文歸,皆豫其謀,皝甚器重之。及遷都和龍,裕雅有巧思,皝所制城池宮闔,皆裕之規模。裕雖仕皝日近,寵秩在舊人之右,性謙恭清儉,剛簡慈篤,雖歷居朝端,若布衣之士。士大夫流亡羈絕者,莫不經營收葬,存恤孤遺,士無賢不肖皆傾身待之,是以所在推仰。



 初,範陽盧諶每稱之曰:「吾及晉之清平,歷觀朝士多矣,忠清簡毅,篤信義烈,如陽士倫者,實亦未幾。」及死,皝甚悼之,時年六十二。



\end{pinyinscope}