\article{帝紀第三 世祖武帝}

\begin{pinyinscope}

 武帝



 武皇帝諱炎,字安世,文帝長子也。寬惠仁厚,沈深有度量。魏嘉平中,封北平亭侯,歷給事中、奉車都尉、中壘將軍,加散騎常侍,累遷中護軍、假節。迎常道鄉公於東武陽,遷中撫軍,進封新昌鄉侯。及晉國建,立為世子,拜撫軍大將軍,開府、副貳相國。初,文帝以景帝既宣帝之嫡,早世無後,以帝弟攸為嗣,特加愛異,自謂攝居相位,百
 年之後,大業宜歸攸。每曰:「此景王之天下也,吾何與焉。」將議立世子,屬意於攸。何曾等固爭曰:「中撫軍聰明神武,有超世之才。髮委地,手過膝,此非人臣之相也。」由是遂定。



 咸熙二年五月,立為晉王太子。八月辛卯,文帝崩,太子嗣相國、晉王位。下令寬刑宥罪,撫眾息役,國內行服三日。是月,長人見於襄武,長三丈,告縣人王始曰:「今當太平。」九月戊午,以魏司徒何曾為丞相,鎮南將軍王沈為御史大夫,中護軍賈充為衛將軍,議郎裴秀為尚書令、光祿大夫,皆開府。十一月,初置四護軍,以統城外諸軍。
 乙未,令諸郡中正以六條舉淹滯:一曰忠恪匪躬,二曰孝敬盡禮,三曰友于兄弟,四曰潔身勞謙,五曰信義可復,六曰學以為己。是時晉德既洽,四海宅心。於是天子知歷數有在,乃使太保鄭沖奉策曰:「咨爾晉王:我皇祖有虞氏誕膺靈運,受終于陶唐,亦以命于有夏。惟三后陟配于天,而咸用光敷聖德。自茲厥後,天又輯大命于漢。火德既衰,乃眷命我高祖。方軌虞夏四代之明顯,我不敢知。惟王乃祖乃父,服膺明哲,輔亮我皇家,勳德光於四海。格爾上下神祗,罔不克順,地平天成,萬邦以乂。應受上帝之命,協皇極之中。肆予一人,祗承天序,以敬
 授爾位,歷數實在爾躬。允執其中,天祿永終。於戲!王其欽順天命。率循訓典,底綏四國,用保天休,無替我二皇之弘烈。」帝初以禮讓,魏朝公卿何曾、王沈等固請,乃從之。



 泰始元年冬十二月丙寅,設壇于南郊,百僚在位及匈奴南單于四夷會者數萬人,柴燎告類于上帝曰:「皇帝臣炎敢用玄牡明告于皇皇后帝:魏帝稽協皇運,紹天明命以命炎。昔者唐堯,熙隆大道,禪位虞舜,舜又以禪禹,邁德垂訓,多歷年載。暨漢德既衰,太祖武皇帝撥亂濟時,扶翼劉氏,又用受命于漢。粵在魏室,仍世多故,幾
 於顛墜,實賴有晉匡拯之德,用獲保厥肆祀,弘濟于艱難,此則晉之有大造於魏也。誕惟四方,罔不祗順,郭清梁岷,包懷揚越,八紘同軌,祥瑞屢臻,天人協應,無思不服。肆予憲章三后,用集大命于茲。炎維德不嗣,辭不獲命。於是群公卿士,百辟庶僚,黎獻陪隸,暨於百蠻君長,僉曰:『皇天鑒下,求人之瘼,既有成命,固非克讓所得距違。天序不可以無統,人神不可以曠主。』炎虔奉皇運。寅畏天威,敬簡元辰,升壇受禪,告類上帝,永答眾望。」禮畢,即洛陽宮幸太極前殿,詔曰:「昔朕皇祖宣王,聖哲欽明,誕應期運,熙帝之載,肇啟洪基。伯考景王,履道宣猷,緝
 熙諸夏。至于皇考文王,睿哲光遠,允協靈祗,應天順時,受茲明命。仁濟于宇宙,功格于上下。肆魏氏弘鑒於古訓,儀刑于唐虞,疇咨群后,爰輯大命于朕身。予一人畏天之命,用不敢違。惟朕寡德,負荷洪烈,託於王公之上,以君臨四海,惴惴惟懼,罔知所濟。惟爾股肱爪牙之佐,文武不貳之臣,乃祖乃父,實左右我先王,光隆我大業。思與萬國,共享休祚。」於是大赦,改元。賜天下爵,人五級;鰥寡孤獨不能自存者穀,人五斛。復天下租賦及關市之稅一年,逋債宿負皆勿收。除舊嫌,解禁錮,亡官失爵者悉復之。丁卯,遣太僕劉原告于太廟。封魏帝為陳留
 王,邑萬戶,居于鄴宮;魏氏諸王皆為縣侯。迫尊宣王為宣皇帝,景王為景皇帝,文王為文皇帝,宣王妃張氏為宣穆皇后。尊太妃王氏曰皇太后,宮曰崇化。封皇叔祖父孚為安平王,皇叔父乾為平原王,亮為扶風王,伷為東莞王,駿為汝陰王,肜為梁王,倫為瑯邪王,皇弟攸為齊王,鑒為樂安王,幾為燕王,皇從伯父望為義陽王,皇從叔父輔為渤海王,晃為下邳王,瑰為太原王,珪為高陽王,衡為常山王,子文為沛王,泰為隴西王,權為彭城王,綏為范陽王,遂為濟南王,遜為譙王,睦為中山王,凌為北海王,斌為陳王,皇從父兄洪為河間王,皇從父弟
 楙為東平王。以驃騎將軍石苞為大司馬,封樂陵公,車騎將軍陳騫為高平公,衛將軍賈充為車騎將軍、魯公,尚書令裴秀為巨鹿公,侍中荀勖為濟北公,太保鄭沖為太傅、壽光公,太尉王祥為太保、睢陵公,丞相何曾為太尉、郎陵公,御史大夫王沈為驃騎將軍、博陵公,司空荀顗為臨淮公,鎮北大將軍衛瓘為菑陽公。其餘增封進爵各有差,文武普增位二等。改景初歷為太始歷,臘以酉,社以丑。戊辰,下詔大弘儉約,出御府珠玉玩好之物,頒賜王公以下各在差。置中軍將軍,以統宿衛七軍。己巳,詔陳留王載天子旌旗,備五時副車,行魏正朔,郊
 祀天地,禮樂制度皆如魏舊,上書不稱臣。賜山陽公劉康、安樂公劉禪子弟一人為附馬都尉。乙亥,以安平王孚為太宰、假黃鉞、大都督中外諸軍事。詔曰:「昔王凌謀廢齊王,而王竟不足以守位。鄧艾雖矜功失節,然束手受罪。今大赦其家,還使立後。興滅繼絕,約法省刑。除魏氏宗室禁錮。諸將吏遭三年喪者,遣寧終喪。百姓復其徭役。罷部曲將長吏以下質任。省郡國御調,禁樂府靡麗百戲之伎及雕文游畋之具。開直言之路,置諫官以掌之。」是月,鳳皇六、青龍三、白龍二、麒麟各一見于郡國。



 二年春正月丙戌,遣兼侍中侯史光等持節四方,循省
 風俗,除禳祝之不在祀典者。丁亥,有司請建七廟,帝重其役,不許。庚寅,罷雞鳴鼓。辛丑,尊景皇帝夫人羊氏曰景皇后,宮曰弘訓。丙午,立皇后楊氏。二月,除漢宗室禁錮。己未,常山王衡薨。詔曰:「五等之封,皆錄舊勛。本為縣侯者傳封次子為亭侯,鄉侯為關內侯,亭侯為關中侯,皆食本戶十分之一。」丁丑,郊祀宣皇帝以配天,宗祀文皇帝於明堂以配上帝。庚午,詔曰:「古者百官,官箴王闕。然保氏特以諫諍為職,今之侍中、常侍實處此位。擇其能正色弼違匡救不逮者,以兼此選。」三月戊戌,吳人來弔祭,有司奏為答詔。帝曰:「昔漢文、光武懷撫尉佗、公
 孫述,皆未正君臣之儀,所以羈糜未賓也。皓遣使之始,未知國慶,但以書答之。」夏五月戊辰,詔曰:「陳留王操尚謙沖,每事輒表,非所以優崇之也。主者喻意,非大事皆使王官表上之。」壬子,驃騎將軍博陵公王沈卒。六月壬申,濟南王遂薨。秋七月辛巳,營太廟,致荊山之木,采華山之石』鑄銅柱十二,塗以黃金,鏤以百物,綴以明珠。戊戌,譙王遜薨。丙午晦,日有蝕之。八月丙辰,省右將軍官。初,帝雖從漢魏之制,既葬除服。而深衣素冠,降席撤膳,哀敬如喪者。戊辰,有司奏改服進膳,不許,遂禮終而後復吉。及太后之喪,亦如之。九月乙未,散騎常侍皇甫陶、
 傅玄領諫官,上書諫諍,有司奏請寢之。詔曰:「凡關言人主,人臣所至難,而苦不能聽納,自古忠臣直士之所慷慨也。每陳事出付主者,多從深刻,乃云恩貸當由主上,是何言乎?其詳評議。」戊戌,有司奏:「大晉繼三皇之蹤,蹈舜禹之跡,應天順時,受禪有魏,宜一用前代正朔服色,皆如虞遵唐故事。」奏可。冬十月丙午朔,日有蝕之。丁未,詔曰:「昔舜葬蒼梧,農不易畝;禹葬成紀,市不改肆。上惟祖考清簡之旨,所徙陵十里內居人,動為煩擾,一切停之。」十一月己卯,倭人來獻方物。並圜丘、方丘於南、北郊,二至之祀合於二郊。罷山陽公國督軍,除其禁制。己丑,
 追尊景帝夫人夏侯氏為景懷皇后。辛卯,遷祖禰神主於太廟。十二月,罷農官為郡縣。是歲,鳳皇六、青龍十、黃龍九、麒麟各一見于郡國。



 三年春正月癸丑,白龍二見於弘農澠池。丁卯,立皇子衷為皇太子。詔曰:「朕以不德,託于四海之上,兢兢祗畏,懼無以康濟寓內,思與天下式明王度,正本清源,於置胤樹嫡,非所先務。又近世每建太子,寬宥施惠之事,間不獲已,順從王公卿士之議耳。方今世運垂平,將陳之以德義,示之以好惡,使百姓蠲多幸之慮,篤終始之行,曲惠小仁,故無取焉。咸使知聞。」三月戊寅,初令二千石
 得終三年喪。丁未,晝昏。罷武衛將軍官。以李憙為太子太傅。太山石崩。夏四月戊午,張掖太守焦勝上言,氐池縣大柳谷口有玄石一所,白畫成文,實大晉之休祥,圖之以獻。詔以制幣告於太廟,藏之天府。秋八月,罷都護將軍,以其五署還光祿勛。九月甲申,詔曰:「古者以德詔爵,以庸制祿,雖下士猶食上農,外足以奉公忘私,內足以養親施惠。今在位者祿不代耕,非所以崇化之本也。其議增吏俸。」賜王公以下帛各有差。以太尉何曾為太保,義陽王望為太尉,司空荀顗為司徒。冬十月,聽士卒遭父母喪者,非在疆場,皆得奔赴。十二月,徙宗聖侯孔
 震為奉聖亭侯。山陽公劉康來朝。禁星氣讖緯之學。



 四年春正月辛未,以尚書令裴秀為司空。丙戌,律令成,封爵賜帛各有差。有星孛於軫。丁亥,帝耕于藉田。戊子,詔曰:「古設象刑而眾不犯,今雖參夷而姦不絕,何德刑相去之遠哉!先帝深愍黎元,哀矜庶獄,乃命群后,考正典刑。朕守遺業,永惟保乂皇基,思與萬國以無為為政。方今陽春養物,東作始興,朕親率王公卿士耕藉田千畝。又律令既就,班之天下,將以簡法務本,惠育海內。宜寬有罪,使得自新,其大赦天下。長吏、郡丞、長史各賜馬一匹。」二月庚子,增置山陽公國相、郎中令、陵令、雜工宰
 人、鼓吹車馬各有差。罷中軍將軍,置北軍中候官。甲寅,以東海劉儉有至行,拜為郎。以中軍將軍羊祜為尚書左僕射,東莞王伷為尚書右僕射。三月戊子,皇太后王氏崩。夏四月戊戌,太保、睢陵公王祥薨。己亥,祔葬文明皇后王氏於崇陽陵。罷振威、揚威護軍官,置左右積弩將軍。六月甲申朔,詔曰:「郡國守相,三載一巡行屬縣,必以春,此古者所以述職宣風展義也。見長吏,觀風俗,協禮律,考度量,存問耆老,親見百年。錄囚徒,理冤枉,詳察政刑得失,知百姓所患苦。無有遠近,便若朕親臨之。敦喻五教,勸務農功,勉勵學者,思勤正典,無為百家庸末,
 致遠必泥。士庶有好學篤道,孝弟忠信,清白異行者,舉而進之;有不孝敬於父母,不長悌於族黨,悖禮棄常,不率法令者,糾而罪之。田疇闢,生業修,禮教設,禁令行,則長吏之能也。人窮匱,農事荒,姦盜起刑,獄煩,下陵上替,禮義不興,斯長吏之否也。若長吏在官公廉,慮不及私,正色直節,不飾名譽者,及身行貪穢,謅黷求容,公節不立,而私門日富者,並謹察之。揚清激濁,舉善彈違,此朕所以垂拱總綱,責成於良二千石也。於戲戒哉!」秋七月,太山石崩,眾星西流。戊午,遣使者侯史光循行天下。己卯,謁崇陽陵。九月,青、徐、兗、豫四州大水,伊洛溢,合於河,
 開倉以振之。詔曰:「雖詔有所欲,及奏得可而於事不便者,皆不可隱情。」冬十月,吳將施績入江夏,萬郁寇襄陽。遣太尉義陽王望屯龍陂。荊州刺史胡烈擊敗郁。吳將顧容寇鬱林,太守毛炅大破之,斬其交州刺史劉俊、將軍修則。十一月,吳將丁奉等出芍陂,安東將軍汝陰王駿與義陽王望擊走之。己未,詔王公卿尹及郡國守相,舉賢良方正直言之士。十二月,班五條詔書於郡國:一曰正身,二曰勤百姓,三曰撫孤寡,四曰敦本息末,五曰去人事。庚寅,帝臨聽訟觀,錄廷尉洛陽獄囚,親平決焉。扶南、林邑各遣使來獻。



 五年春正月癸巳,申戒郡國計吏守相令長,務盡地利,禁游食商販。丙申,帝臨聽訟觀錄囚徒,多所原遣。青龍二見於滎陽。二月,以雍州隴右五郡及涼州之金城、梁州之陰平置秦州。辛巳,白龍二見於趙國。青、徐、兗三州水,遣使振恤之。壬寅,以尚書左僕射羊祜都督荊州諸軍事,征東大將軍衛瓘都督青州諸軍事,東莞王伷為鎮東大將軍都督徐州諸軍事。丁亥,詔曰:「古者歲書群吏之能否,三年而誅賞之。諸令史前後,但簡遣疏劣,而無有勸進,非黜陟之謂也。其條勤能有稱尤異者,歲以為常。吾將議其功勞。」己未,詔蜀相諸葛亮孫京隨才署
 吏。夏四月,地震。五月辛卯朔,鳳皇見于趙國。曲赦交趾、九真、日南五歲刑。六月,鄴奚官督郭暠上疏陳五事以諫,言甚切直,擢為屯留令,西平人曲路伐登聞彭,言多襖謗,有司奏棄市。帝曰:「朕之過也。」捨而不問。罷鎮軍將軍,復置左右將軍官。秋七月,延群公,詢讜言。九月,有星孛於紫宮。冬十月丙子,以汲郡太守王宏有政績,賜穀千斛。十一月,追封謚皇弟兆為城陽哀王,以皇子景度嗣。十二月,詔州郡舉勇猛秀異之才。



 六年春正月丁亥朔,帝臨軒,不設樂。吳將丁奉入渦口,揚州刺史牽弘擊走之。三月,赦五歲刑已下。夏四月,白
 龍二見於東莞。五月,立壽安亭侯承為南宮王。六月戊午,秦州刺史胡烈擊叛虜於萬斛堆,力戰,死之。詔遣尚書石鑒行安西將軍、都督秦州諸軍事,與奮威護軍田章討之。秋七月丁酉,復隴右五郡遇寇害者租賦,不能自存者稟貸之。乙巳,城陽王景度薨。詔曰:「自泰始以來,大事皆撰錄秘書,寫副。後有其事,輒宜綴集以為常。」丁未,以汝陰王駿為鎮西大將軍、都督雍涼二州諸軍事。九月,大宛獻汗血馬,焉耆來貢方物。冬十一月,幸辟雍,行鄉飲酒之禮,賜太常博士、學生帛牛酒各有差。立皇子柬為汝南王。十二月,吳夏口督、前將軍孫秀帥眾來奔,拜
 驃騎將軍、開府儀同三司,封會稽公。戊辰,復置鎮軍官。



 七年春正月丙子,皇太子冠,賜王公以下帛各有差。匈奴帥劉猛叛出塞。二月,孫皓帥眾趨壽陽,遣大司馬望屯淮北以距之。三月,丙戌,司空、巨鹿公裴秀薨。癸巳,以中護軍王業為尚書左僕射,高陽王珪為尚書右僕射。孫秀部將何崇帥眾五千人來降。夏四月,九真太守董元為吳將虞氾所攻,軍敗,死之。北地胡寇金城,涼州刺史牽弘討之。群虜內叛,圍弘於青山,弘軍敗,死之。五月,立皇子憲為城陽王。雍、涼、秦三州饑,赦其境內殊死以下。閏月,大雩,太官減膳。詔交趾三郡、南中諸郡無出今
 年戶調。六月,詔公卿以下舉將帥各一人。辛丑,大司馬義陽王望薨。大雨霖,伊、洛、河溢,流居人四千餘家,殺三百餘人,有詔振貸給棺。秋七月癸酉,以車騎將軍賈充為都督秦、涼二州諸軍事。吳將陶璜等圍交趾,太守楊稷與鬱林太守毛炅及日南等三郡降於吳。八月丙戌,以征東大將軍衛瓘為征北大將軍、都督幽州諸軍事。丙申,城陽王憲薨。分益州之南中四郡置寧州,曲赦四郡殊死已下。冬十月丁丑,日有蝕之。十一月丁巳,衛公姬署薨。十二月,大雪。罷中領軍,并北軍中候。以光祿大夫鄭袤為司空。



 八年春正月,監軍何楨討匈奴劉猛,累破之,左部帥李恪殺猛而降。癸亥,帝耕于藉田。二月乙亥,禁彫文綺組非法之物。壬辰,太宰、安平王孚薨。詔內外群官舉任邊郡者各三人。帝與右將軍皇甫陶論事,陶與帝爭言,散騎常侍鄭徽表請罪之。帝曰:「讜言謇諤,所望於左右也。人主常以阿媚為患,豈以爭臣為損哉!徽越職妄奏,豈朕之意。」遂免徽官。夏四月,置後將軍,以備四軍。六月,益州牙門張弘誣其刺史皇甫晏反,殺之,傳首京師。弘坐伏誅,夷三族。壬辰,大赦。丙申,詔復隴右四郡遇寇害者田租。秋七月,以車騎將軍賈充為司空。九月,吳西陵督
 步闡來降,拜衛將軍、開府儀同三司,封宜都公。吳將陸抗攻闡,遣車騎將軍羊祜帥眾出江陵,荊州刺史楊肇迎闡於西陵,巴東監軍徐胤擊建平以救闡。冬十月辛未朔,日有蝕之。十二月,肇攻抗,不克而還。闡城陷,為抗所禽。



 九年春正月辛酉,司空、密陵侯鄭袤薨。二月癸巳,司徒、樂陵公石苞薨。立安平亭侯隆為安平王。三月,立皇子祗為東海王。夏四月戊辰朔,日有蝕之。五月,旱。以太保何曾領司徒。六月乙未,東海王祗薨。秋七月丁酉朔,日有蝕之。吳將魯淑圍弋陽,征虜將軍王渾擊敗之。罷五
 官左右中郎將、弘訓太僕、衛尉、大長秋等官。鮮卑寇廣寧,殺略五千人。詔聘公卿以下子女以備六宮,採擇未畢,權禁斷婚姻。冬十月辛巳,制女年十七父母不嫁者,使長吏配之。十一月丁酉,臨宣武觀大閱諸軍,甲辰乃罷。



 十年春正月辛亥,帝耕于藉田。閏月癸酉,太傅、壽光公鄭沖薨。己卯,高陽王珪薨。庚辰,太原王瑰薨。丁亥,詔曰:「嫡庶之別,所以辨上下,明貴賤。而近世以來,多皆內寵,登妃后之職,亂尊卑之序。自今以後,皆不得登用妾媵以為嫡正。」二月,分幽州五郡置平州。三月癸亥,日有蝕
 之。夏四月己未,太尉、臨淮公荀顗薨。六月癸巳,臨聽訟觀錄囚徒,多所原遣。是夏,大蝗。秋七月丙寅,皇后楊氏崩。壬午,吳平虜將軍孟泰、偏將軍王嗣等帥眾降。八月,涼州虜寇金城諸郡,鎮西將軍、汝陰王駿討之,斬其帥乞文泥等。戊申,葬元皇后于峻陽陵。九月癸亥,以大將軍陳騫為大尉。攻拔吳枳里城,獲吳立信校尉莊祜。吳將孫遵、李承帥眾寇江夏,太守嵇喜擊破之。立河橋于富平津。冬十一月,立城東七里澗石橋。庚午,帝臨宣武觀,大閱諸軍。十二月,有星孛于軫。置藉田令。立太原王子緝為高陽王。吳威北將軍嚴聰、揚威將軍嚴整、偏將
 軍朱買來降。是歲,鑿陜南山,決河,東注洛,以通運漕。



 咸寧元年春正月戊午朔,大赦,改元。二月,以將士應已娶者多,家有五女者給復。辛酉,以故鄴令夏謖有清稱,賜穀百斛。以奉祿薄,賜公卿以下帛有差。叛虜樹機能送質請降。夏五月,下邳、廣陵大風,拔木,壞廬舍。六月,鮮卑力微遣子來獻。吳人寇江夏。西域戊己校尉馬循討叛鮮卑,破之,斬其渠帥。戊申,置太子詹事官。秋七月甲申晦,日有蝕之。郡國螟。八月壬寅,沛王子文薨。以故太傅鄭沖、太尉荀顗、司徒石苞、司空裴秀、驃騎將軍王沈、安平獻王孚等及太保何曾、司空賈充、太尉陳騫、中書
 監荀勖、平南將軍羊祜、齊王攸等皆列於銘饗。九月甲子,青州螟,徐州大水。冬十月乙酉,常山王殷薨。癸巳,彭城王權薨。十一月癸亥,大閱於宜武觀,至于己巳。十二月丁亥,追尊宣帝廟曰高祖,景帝曰世宗,文帝曰太祖。是月大疫,洛陽死者大半。封裴頠為鉅鹿公。



 二年春正月,以疾疫廢朝。賜諸散吏至於士卒絲各有差。二月丙戌,河間王洪薨。甲午,赦五歲刑以下。東夷八國歸化。并州虜犯塞,監並州諸軍事胡奮擊破之。初,燉煌太守尹璩卒,州以燉煌令梁澄領太守事。議郎令狐豐廢澄,自領郡事。豐死,弟宏代之。至是,涼州刺史楊欣
 斬宏,傳首洛陽。先是,帝不豫,及瘳,群臣上壽。詔曰:「每念頃遇疫氣死亡,為之愴然。豈以一身之休息,忘百姓之艱邪?諸上禮者皆絕之。」夏五月,鎮西大將軍、汝陰王駿討北胡,斬其渠帥吐敦。立國子學。庚午,大雩。六月癸丑,薦荔支于太廟。甲戌,有星孛于氐。自春旱,至于是月始雨。吳京下督孫楷帥眾來降,以為車騎將軍,封丹陽侯。白龍二見于新興井中。秋七月,有星孛于大角。吳臨平湖自漢末壅塞,至是自開。父老相傳云:「此湖塞,天下亂;此湖開,天下平。」癸丑,安平王隆薨。東夷十七國內附。河南、魏郡暴水,殺百餘人,詔給棺。鮮卑阿羅多等寇邊,西
 域戊己校尉馬循討之,斬首四千餘級,獲生九千餘人,於是來降。八月庚辰,河東、平陽地震。己亥,以太保何曾為太傅,太尉陳騫為大司馬,司空賈充為太尉,鎮軍大將軍齊王攸為司空。有星孛于太微,九月又孛于翼。丁未,起太倉於城東,常平倉於東西市。閏月,荊州五郡水,流四千餘家。冬十月,以汝陰王駿為征西大將軍,平南將軍羊祜為征南大將軍。丁卯,立皇后楊氏,大赦,賜王公以下及於鰥寡各有差。十一月,白龍二見於梁國。十二月,徵處士安定皇甫謐為太子中庶子,封后父鎮軍將軍楊駿為臨晉侯。是月,以平州刺史傅詢、前廣平太
 守孟桓清白有聞,詢賜帛二百匹,桓百匹。



 三年春正月丙子朔,日有蝕之。立皇子裕為始平王,安平穆王隆弟敦為安平王。詔曰:「宗室戚屬,國之枝葉,欲令奉率德義,為天下式。然處富貴而能慎行者寡,召穆公糾合兄弟而賦《棠棣》之詩,此姬氏所以本枝百世也。今以衛將軍、扶風王亮為宗師,所當施行,皆咨之於宗師也。」庚寅,始平王裕薨。有星孛於西方。使征北大將軍衛言雚討鮮卑力微。三月,平虜護軍文淑討叛虜樹機能等,破之。有星孛于胃。乙未,帝將射雉,慮損麥苗而止。夏五月戊子,吳將邵凱、夏祥帥眾七千餘人來降。六月,益、
 梁八郡水,殺三百餘人,沒邸閣別倉。秋七月,以都督豫州諸軍事王渾為都督揚州諸軍事。中山王睦以罪廢為丹水侯。八月癸亥,徙扶風王亮為汝南王,東莞王伷為琅邪王,汝陰王駿為扶風王,瑯邪王倫為趙王,渤海王輔為太原王,太原王顒為河間王,北海王陵為任城王,陳王斌為西河王,汝南王柬為南陽王,濟南王耽為中山王,河間王威為章武王。立皇子瑋為始平王,允為濮陽王,該為新都王,遐為清河王,鉅平侯羊祜為南城侯。以汝南王亮為鎮南大將軍。大風拔樹,暴寒且冰,郡國五隕霜,傷穀。九月戊子,以左將軍胡奮為都督江北
 諸軍事。兗、豫、徐、青、荊、益、梁七州大水,傷秋稼,詔振給之。立齊王子蕤為遼東王,贊為廣漢王。冬十一月丙戌,帝臨宣武觀大閱,至于壬辰。十二月,吳將孫慎入江夏、汝南,略千餘家而去。是歲,西北雜虜及鮮卑、匈奴、五溪蠻夷、東夷三國前後十餘輩,各帥種人部落內附。



 四年春正月庚午朔,日有蝕之。三月甲申,尚書左僕射盧欽卒。辛酉,以尚書右僕射山濤為尚書左僕射。東夷六國來獻。夏四月,蚩尤旗見於東井。六月丁未,陰平、廣武地震,甲子又震。涼州刺史楊欣與虜若羅拔能等戰于武威,敗績,死之。弘訓皇后羊氏崩。秋七月己丑,祔葬
 景獻皇后羊氏于峻平陵。庚寅,高陽王緝薨。癸巳,范陽王綏薨。荊、揚郡國二十皆大水。九月,以大傅何曾為太宰。辛巳,以尚書令李胤為司徒。冬十月,以征北大將軍衛瓘為尚書令。揚州刺史應綽伐吳皖城,斬首五千級,焚穀米百八十萬斛。十一月辛巳,太醫司馬程據獻雉頭裘,帝以奇技異服典禮所禁,焚之於殿前。甲申,敕內外敢有犯者罪之。吳昭武將軍劉翻、厲武將軍祖始來降。辛卯,以尚書杜預都督荊州諸軍事。征南大將軍羊祜卒。十二月乙未,西河王斌薨。丁未,太宰郎陵公何曾薨。是歲,東夷九國內附。



 五年春正月,虜帥樹機能攻陷涼州。乙丑,使討虜護軍武威太守馬隆擊之。二月甲午,白麟見于平原。三月,匈奴都督拔弈虛帥部落歸化。乙亥,以百姓饑饉,減御膳之半。有星孛于柳。夏四月,又孛於女御。大赦,降除部曲督以下質任。丁亥,郡國八雨雹,傷秋稼,壞百姓廬舍。秋七月,有星孛于紫宮。九月甲午,麟見于河南。冬十月戊寅,匈奴餘渠都督獨雍等帥部落歸化。汲郡人不準掘魏襄王塚,得竹簡小篆古書十餘萬言,藏于秘府。十一月,大舉伐吳,遣鎮軍將軍、瑯邪王伷出塗中,安東將軍王渾出江西,建威將軍王戎出武昌,平南將軍胡奮出
 夏口,鎮南大將軍杜預出江陵,龍驤將軍王浚、廣武將軍唐彬率巴蜀之卒浮江而下,東西凡二十餘萬。以太尉賈充為大都督,行冠軍將軍楊濟為副,總統眾軍。十二月,馬隆擊叛虜樹機能,大破,斬之,涼州平。肅慎來獻楛矢石砮。



 太康元年春正月己丑朔,五色氣冠日。癸丑,王渾克吳尋陽賴鄉諸城,獲吳武威將軍周興。二月戊午,王濬、唐棼等剋丹陽城。庚申,又克西陵,殺西陵都督、鎮軍將軍留憲,征南將軍成璩,西陵監鄭廣。壬戌,浚又克夷道樂鄉城,殺夷道監陸晏、水軍都督陸景。甲戌,杜預克江陵,
 斬吳江陵督王延;平南將軍胡奮剋江安。於是諸軍並進,樂鄉、荊門諸戍相次來降。乙亥,以浚為都督益、梁二州諸軍事,復下詔曰:「濬、彬東下,掃除巴丘,與胡奮、王戎共平夏口、武昌,順流長鶩,直造秣陵,與奮、戎審量其宜。杜預當鎮靜零、桂,懷輯衡陽。大兵既過,荊州南境固當傳檄而定,預當分萬人給浚,七千給彬。夏口既平,奮宜以七千人給浚。武昌既了,戎當以六千人增彬。太尉充移屯項,總督諸方。」浚進破夏口、武昌,遂泛舟東下,所至皆平。王渾、周浚與吳丞相張悌戰於版橋,大敗之,斬悌及其將孫震、沈瑩,傳首洛陽。孫皓窮蹙請降,送璽綬於
 瑯邪王伷。三月壬申,王浚以舟師至于建鄴之石頭,孫皓大懼,面縛輿櫬,降于軍門。浚杖節解縛焚櫬,送于京都。收其圖籍,得州四,郡四十三,縣三百一十三,戶五十二萬三千,吏三萬三千,兵二十三萬,男女口二百三十萬。其牧守下皆因吳所置,除其苛政,示之簡易,吳人大悅。乙酉大赦,改元,大酺五日,恤孤老困窮。夏四月,河東、高平雨雹,傷秋稼。遣兼侍中張側、黃門侍郎朱震分使揚越,尉其初附。白麟見于頓丘。三河、魏郡、弘農雨雹,傷宿麥。五月辛亥,封孫皓為歸命侯,拜其太子為中郎,諸子為郎中。吳之舊望,隨才擢敘。孫氏大將戰亡之家
 徙於壽陽,將吏渡江復十年,百姓及百工復二十年。丙寅,帝臨軒大會,引皓升殿,群臣咸稱萬歲。丁卯,薦酃淥酒于太廟。郡國六雹,傷秋稼。庚午,詔諸士卒年六十以上罷歸于家。庚辰,以王浚為輔國大將軍、襄陽侯,杜預當陽侯,王戎安豐侯,唐彬上庸侯,賈充、琅邪王伷以下增封。於是論功行封,賜公卿以下帛各有差。六月丁丑,初置翊軍校尉官。封丹水侯睦為高陽王。甲申,東夷十國歸化。秋七月,虜軻成泥寇西平、浩亹,殺督將以下三百餘人。東夷二十國朝獻。庚寅,以尚書魏舒為尚書右僕射。八月,車師前部遣子入侍。己未,封皇弟延祚為樂
 平王。白龍三見于永昌。九月,群臣以天下一統,屢請封禪,帝謙讓弗許。冬十月丁巳,除五女復。十二月戊辰,廣漢王贊薨。



 二年春二月,淮南、丹陽地震。三月丙申,安平王敦薨。賜王公以下吳生口各有差。詔選孫皓妓妾五千人入宮。東夷五國朝獻。夏六月,東夷五國內附。郡國十六雨雹,大風拔樹,壞百姓廬舍。江夏、泰山水,流居人三百餘家。秋七月,上黨又暴風雨雹,傷秋稼。八月,有星孛于張。冬十月,鮮卑慕容廆寇昌黎。十一月壬寅,大司馬陳騫薨。有星孛于軒轅。鮮卑寇遼西,平州刺史鮮于嬰討破之。



 三年春正月丁丑,罷秦州,并雍州。甲午,以尚書張華都督幽州諸軍事。三月,安北將軍嚴詢敗鮮卑慕容廆於昌黎,殺傷數萬人。夏四月庚午,太尉、魯公賈充薨。閏月丙子,司徒、廣陸侯李胤薨。癸丑,白龍二見于濟南。秋八月,罷平州、寧州刺史三年一入奏事。九月,東夷二十九國歸化,獻其方物。吳故將莞恭、帛奉舉兵反,攻害建鄴令,遂圍揚州,徐州刺史嵇喜討平之。冬十二月甲申,以司空齊王攸為大司馬、督青州諸軍事,鎮東大將軍、瑯邪王伷為撫軍大將軍,汝南王亮為太尉,光祿大夫山濤為司徒,尚書令衛瓘為司空。丙申,詔四方水旱甚者
 無出田租。



 四年春正月甲申,以尚書右僕射魏舒為尚書左僕射,下邳王晃為尚書右僕射。戊午,司徒山濤薨。二月己丑,立長樂亭侯寔為北海王。三月辛丑朔,日有蝕之。癸丑,大司馬齊王攸薨。夏四月,任城王陵薨。五月己亥,大將軍、瑯邪王伷薨。徙遼東王蕤為東萊王。六月,增九卿禮秩。牂柯獠二千餘落內屬。秋七月壬子,以尚書右僕射、下邳王晃為都督青州諸軍事。丙寅,袞州大水,復田租。八月,鄯善國遣子人侍,假其歸義侯。以隴西王泰為尚書右僕射。冬十一月戊午,新都王該薨。以尚書左僕
 射魏舒為司徒。十二月庚午,大閱于宣武觀。是歲,河南及荊州、揚州大水。



 五年春正月己亥,青龍二見於武庫井中。二月丙寅,立南宮王子玷為長樂王。壬辰,地震。夏四月,任城、魯國池水赤如血。五月丙午,宣帝廟梁折。六月,初置黃沙獄。秋七月戊申,皇子恢薨。任城、梁國、中山雨雹,傷秋稼。減天下戶課三分之一。九月,南安大風折木。郡國五大水,隕霜,傷秋稼。冬十一月甲辰,太原王輔薨。十二月庚午,大赦。林邑、大秦國各遣使來獻。閏月,鎮南大將軍、當陽侯杜預卒。



 六年春正月庚申朔,以比歲不登,免租貸宿負。戊辰,以征南大將軍王渾為尚書左僕射,尚書褚契都督揚州諸軍事,揚濟都督荊州諸軍事。三月,郡國六隕霜,傷桑麥。夏四月,扶南等十國來獻,參離四千餘落內附。郡國四旱,十大水,壞百姓廬舍。秋七月,巴西地震。八月丙戌朔,日有蝕之。減百姓綿絹三分之一。白龍見于京兆。以鎮軍大將軍王浚為撫軍大將軍。九月丙子,山陽公劉康薨。冬十月,南安山崩,水出。南陽郡獲兩足獸。龜茲、焉耆國遣子人侍。十二月甲申,大閱于宣武觀,旬日而罷。庚寅,撫軍大將軍、襄陽侯王濬卒。



 七年春正月甲寅朔,日有蝕之。乙卯,詔曰:「比年災異屢發,日蝕三朝,地震山崩。邦之不臧,實在朕躬。公卿大臣各上封事,極言其故,勿有所諱。」夏五月,郡國十三旱。鮮卑慕容廆寇遼東。秋七月,硃提山崩,犍為地震。八月,東夷十一國內附。京兆地震。九月戊寅,驃騎將軍、扶風王駿薨。郡國八大水。冬十一月壬子,以隴西王泰都督關中諸軍事。十二月,遣侍御史巡遭水諸郡。出後宮才人、妓女以下三百七十人歸于家。始制大臣聽終喪三年。己亥,河陰雨赤雪二頃。是歲,扶南等二十一國、馬韓等十一國遣使來獻。



 八年春正月戊申朔,日有蝕之。太廟殿陷。三月乙丑,臨商觀震。夏四月,齊國、天水隕霜,傷麥。六月,魯國大風,拔樹木,壞百姓廬舍。郡國八大水。秋七月,前殿地陷,深數丈,中有破船。八月,東夷二國內附。九月,改營太廟。冬十月,南康平固縣吏李豐反,聚眾攻郡縣,自號將軍。十一月,海安令蕭輔聚眾反。十二月,吳興人蔣迪聚黨反,圍陽羨縣,州郡捕討,皆伏誅。南夷扶南、西域康居國各遣使來獻。是歲,郡國五地震。



 九年春正月壬申朔,日有蝕之。詔曰:「興化之本,由政平訟理也。二千石長吏不能勤恤人隱,而輕挾私故,興長
 刑獄,又多貪濁,煩撓百姓。其敕刺史二千石糾其穢濁,舉其公清,有司議其黜陟。令內外群官舉清能,拔寒素。」江東四郡地震。二月,尚書右僕射、陽夏侯胡奮卒,以尚書朱整為尚書右僕射。三月丁丑,皇后親桑于西郊,賜帛各有差。壬辰,初並二社為一。夏四月,江南郡國八地震;隴西隕霜,傷宿麥。五月,義陽王奇有罪,黜為三縱亭侯。詔內外群官舉守令之才。六月庚子朔,日有蝕之。徙章武王威為義陽王。郡國三十二大旱,傷麥。秋八月壬子,星隕如雨。詔郡國五歲刑以下決遣,無留庶獄。九月,東夷七國詣校尉內附。郡國二十四螟。冬十二月癸卯,
 立河間平王洪子英為章武王。戊申,青龍、黃龍各一見於魯國。



 十年夏四月,以京兆太守劉霄、陽平太守梁柳有政績,各賜穀千斛。郡國八隕霜。太廟成。乙巳,遷神主于新廟,帝迎于道左,遂袷祭。大赦,文武增位一等,作廟者二等。丁未,尚書右僕射、廣興侯朱整卒。癸丑,崇聖殿災。五月,鮮卑慕容廆來降,東夷十一國內附。六月庚子,山陽公劉瑾薨。復置二社。冬十月壬子,徙南宮王承為武邑王。十一月丙辰,守尚書令、左光祿大夫荀勖卒。帝疾瘳,賜王公以下帛有差。含章殿鞠室火。甲申,以汝南王亮為
 大司馬、大都督、假黃鉞。改封南陽王柬為秦王,始平王瑋為楚王,濮陽王允為淮南王,並假節之國,各統方州軍事。立皇子乂為長沙王,潁為成都王,晏為吳王,熾為豫章王,演為代王,皇孫遹為廣陵王。立濮陽王子迪為漢王,始平王子儀為毗陵王,汝南王次子羕為西陽公。徙扶風王暢為順陽王,暢弟歆為新野公,瑯邪王覲弟澹為東武公,繇為東安公,漼為廣陵公,卷為東莞公。改諸王國相為內史。十二月庚寅,太廟梁折。是歲,東夷絕遠三十餘國、西南夷二十餘國來獻。壬戌,虜奚軻男女十萬口來降。



 太熙元年春正月辛酉朔,改元。乙巳,以尚書左僕射王渾為司徒,司空衛瓘為太保。二月辛丑,東夷七國朝貢。瑯邪王覲薨。三月甲子,以右光祿大夫石鑒為司空。夏四月辛丑,以侍中車騎將軍楊駿為太尉、都督中外諸軍、錄尚書事。己酉,帝崩於含章殿,時年五十五,葬峻陽陵,廟號世祖。



 帝宇量弘厚,造次必於仁恕;容納讜正,未嘗失色於人;明達善謀,能斷大事,故得撫寧萬國,綏靜四方。承魏氏奢侈革弊之後,百姓思古之遺風,乃厲以恭儉,敦以寡慾。有司嘗奏御牛青絲紖斷,詔以青麻代之。臨朝寬裕,法度有恒。高陽許允既為文帝所殺,允子
 奇為太常丞。帝將有事於太廟,朝議以奇受害之門,不欲接近左右,請出為長史。帝乃追述允夙望,稱奇之才,擢為祠部郎,時論稱其夷曠。平吳之後,天下乂安,遂怠於政術,耽於遊宴,寵愛后黨,親貴當權,舊臣不得專任,彞章紊廢,請謁行矣。爰至未年,知惠帝弗克負荷,然恃皇孫聰睿,故無廢立之心。復慮非賈后所生,終致危敗,遂與腹心共圖後事。說者紛然,久而不定,竟用王佑之謀,遣太子母弟秦王柬都督關中,楚王瑋、淮南王允並鎮守要害,以彊帝室。又恐楊氏之偪,復以佑為北軍中候,以典禁兵。既而寢疾彌留,至于大漸,佐命元勳,皆已先沒,
 群臣惶惑,計無所從。會帝小差,有詔以汝南王亮輔政,又欲令朝士之有名望年少者數人佐之,楊駿祕而不宣。帝復尋至迷亂,楊后輒為詔以駿輔政,促亮進發。帝尋小間,問汝南王來未,意欲見之,有所付託。左右答言未至,帝遂困篤。中朝之亂,實始于斯矣。



 制曰:武皇承基,誕膺天命,握圖御宇,敷化導民,以佚代勞。以治易亂。絕縑絕之貢,去雕琢之飾,制奢俗以變儉約,止澆風而反淳朴。雅好直言,留心采擢,劉毅、裴楷以質直見容,嵇紹、許奇雖仇讎不棄。仁以御物,寬而得眾,宏略大度,有帝王之量焉。於是民和俗靜,家給人足,聿
 修武用,思啟封疆。決神算於深衷,斷雄圖於議表。馬隆西伐,王濬南征,師不延時,獯虜削迹,兵無血刃,揚越為墟。通上代之不通,服前王之未服。禎祥顯應,風教肅清,天人之功成矣,霸王之業大矣。雖登封之禮,讓而不為,驕泰之心,因斯而起。見土地之廣,謂萬棄而無虞;睹天下之安,謂千年而永治。不知處廣以思狹,則廣可長廣;居治而忘危,則治無常治。加之建立非所,委寄失才,志欲就於升平,行先迎於禍亂。是猶將適越者指沙漠以遵途,欲登山者涉舟航而覓路,所趣逾遠,所尚轉難,南北倍殊,高下相反,求其至也,不亦難乎!況以新集易動
 之基,而久安難拔之慮,故賈充凶豎,懷姦志以擁權;楊駿豺狼,苞禍心以專輔。及乎宮車晚出,諒闇未周,籓翰變親以成疏,連兵競滅其本;棟梁回忠而起偽,擁眾各舉其威。曾未數年,網紀大亂,海內版蕩,宗廟播遷。帝道王猷,反居文身之俗;神州赤縣,翻成被髮之鄉。棄所大以資人,掩其小而自託,為天下笑,其故何哉?良由失慎於前,所以貽患於後。且知子者賢父,知臣者明君;子不肖則家亡,臣不忠則國亂;國亂不可以安也,家亡不可以全也。是以君子防其始,聖人閑其端。而世祖惑荀勖之奸謀,迷王渾之偽策,心屢移於眾口,事不定於己
 圖。元海當除而不除,卒令擾亂區夏;惠帝可廢而不廢,終使傾覆洪基。夫全一人者德之輕,拯天下者功之重,棄一子者忍之小,安社稷者孝之大;況乎資三世而成業,延二孽以喪之,所謂取輕德而舍重功,畏小忍而忘大孝。聖賢之道,豈若斯乎!雖則善始於初,而乖令終於末,所以殷勤史策,不能無慷慨焉。



\end{pinyinscope}