\article{載記第五 石勒下}

\begin{pinyinscope}
石勒
 \gezhu{
  下}



 太興二年,勒偽稱趙王,赦殊死已下,均百姓田租之半,賜孝悌力田死義之孤帛各有差,孤老鰥寡穀人三石,大酺七日。依春秋列國、漢初侯王每世稱元,改稱趙王元年。始建社稷,立宗廟,營東西宮。署從事中郎裴憲、參軍傅暢、杜嘏並領經學祭酒,參軍續咸、庾景為律學祭酒,任播、崔濬為史學祭酒。中壘支雄、游擊王陽並領門
 臣祭酒,專明胡人辭訟,以張離、張良、劉群、劉謨等為門生主書,司典胡人出內,重其禁法,不得侮易衣冠華族。號胡為國人。遣使循行州郡,勸課農桑。加張賓大執法,專總朝政,位冠僚首。署石季龍為單于元輔、都督禁衛諸軍事,署前將軍李寒領司兵勳,教國子擊刺戰射之法。命記室佐明楷、程機撰《上黨國記》,中大夫傅彪、賈蒲、江軌撰《大將軍起居注》,參軍石泰、石同、石謙、孔隆撰《大單于志》。自是朝會常以天子禮樂饗其群下,威儀冠冕從容可觀矣。群臣議請論功,勒曰:「自孤起軍,十六年于茲矣。文武將士從孤征伐者,莫不蒙犯矢石,備嘗艱阻,
 其在葛陂之役,厥功尤著,宜為賞之先也。若身見存,爵封輕重隨功位為差,死事之孤,賞加一等,庶足以尉答存亡,申孤之心也。」又下書禁國人不聽報嫂及在喪婚娶,其燒葬令如本俗。



 孔萇攻邵續別營十一,皆下之。續尋為石季龍所獲,送于襄國。劉曜將尹安、宋始據洛陽,降于勒。



 晉徐州刺史蔡豹敗徐龕于檀丘,龕遣使詣勒,陳討豹之計。勒遣將王步都為龕前鋒,使張敬率騎繼之。敬達東平,龕疑敬之襲已也,斬步都等三百餘人,復降于晉。勒大怒,命張敬據其襟要以守之。



 大雨霖,中山、常山尤甚,滹沲汛溢,衝陷山谷,巨松僵拔,浮于滹沲,東
 至渤海,原隰間皆如山積。



 孔萇攻陷文鴦十餘營,萇不設備,鴦夜擊之,大敗而歸。



 勒始制軒懸之樂,八佾之舞,為金根大輅,黃屋左纛,天子車旗,禮樂備矣。



 使石季龍率步騎四萬討徐龕,龕遣長史劉霄詣勒乞降,送妻子為質,納之。時蔡豹屯于譙城,季龍攻豹,豹夜遁,季龍引軍城封丘而旋。



 徙朝臣掾屬已上士族者三百戶於襄國崇仁里,置公族大夫以領之。勒宮殿及諸門始就,制法令甚嚴,諱胡尤峻。有醉胡乘馬突入止車門,勒大怒,謂宮門小執法馮翥曰:「夫人君為令,尚望威行天下,況宮闕之間乎!向馳馬入門為是何人,而不彈白邪?」翥
 惶懼忘諱,對曰:「向有醉胡乘馬馳入,甚呵禦之,而不可與語。」勒笑曰:「胡人正自難與言。」恕而不罪。



 使石季龍擊託候部掘咄哪於岍北,大破之,俘獲牛馬二十餘萬。



 勒清定五品,以張賓領選。復續定九品。署張班為左執法郎,孟卓為右執法郎,典定士族,副選舉之任。今群僚及州郡歲各舉秀才、至孝、廉清、賢良、直言、武勇之士各一人。置署都部從事各一部一州,秩二千石,職準丞相司直。



 勒下令曰:「去年水出巨材,所在山積,將皇天欲孤繕修宮宇也!其擬洛陽之太極起建德殿。」遣從事中郎任汪帥使工匠五千採木以供之。黎陽人陳武妻一
 產三男一女,武攜其妻子詣襄國上書自陳。勒下書以為二儀諧暢,和氣所致,賜其乳婢一口,穀一百石,雜彩四十匹。



 石季龍攻段匹磾于厭次。孔萇討匹磾部內諸城,陷之。匹磾勢窮,乃率其臣下輿櫬出降。季龍送之襄國,勒署匹磾為冠軍將軍,以其其弟文鴦、亞將衛麟為左右中郎將,皆金章紫綬。散諸流人三萬餘戶,復其本業,置守宰以撫之,於是冀、并、幽州、遼西巴西諸屯結皆陷於勒。



 時晉征北將軍祖逖據譙,將平中原。逖善於撫納,自河以南多背勒歸順。勒憚之,不敢為寇,乃下書曰:「祖逖屢為邊患。逖,北州士望也,儻有首丘之思。其下幽州,修祖
 氏墳墓,為置守塚二家。冀逖如趙他感恩,輟其寇暴。」逖聞之甚悅,遣參軍王愉使于勒,贈以方物,修結和好。勒厚賓其使,遣左常侍董樹報聘,以馬百匹、金五十斤答之。自是兗豫乂安,人得休息矣。



 從事中郎劉奧坐營建德殿井木斜縮,斬于殿中。勒悔之,贈太常。



 建德校尉王和掘得員石,銘曰:「律權石,重四鈞,同律度量衡,有新氏造。」議者未詳,或以為瑞。參軍續咸曰:「王莽時物也。」其時兵亂之後,典度堙滅,遂命下禮官為準程定式。又得一鼎,容四升,中有大錢三十文,曰:「百當千,千當萬。」鼎銘十三字,篆書不可曉,藏之於永豐倉。因此令公私行錢,而
 人情不樂,乃出公絹市錢,限中絹匹一千二百,下絹八百。然百姓私買中絹四千,下絹二千,巧利者賤買私錢,貴賣於官,坐死者十數人,而錢終不行。勒徙洛陽銅馬、翁仲二於襄國,列之永豐門。



 祖逖牙門童建害新蔡內史周密,遣使降于勒。勒斬之,送首于祖逖,曰:「天下之惡一也。叛臣逃吏,吾之深仇,將軍之惡,猶吾惡也。」逖遣使報謝。自是兗豫間壘壁叛者,逖皆不納,二州之人率多兩屬矣。



 勒令武鄉耆舊赴襄國。既至,勒親與鄉老齒坐歡飲,語及平生。初,勒與李陽鄰居,歲常爭麻池,迭相驅擊。至是,謂父老曰:「李陽,壯士也,何以不來?漚麻是布衣
 之恨,孤方崇信於天下,寧讎匹夫乎!」乃使召陽。既至,勒與酣謔,引陽臂笑曰:「孤往日厭卿老拳,卿亦飽孤毒手。」因賜甲第一區,拜參軍都尉。令曰:「武鄉,吾之豐沛,萬歲之後,魂靈當歸之,其復之三世。」勒以百姓始復業,資儲未豐,於是重制禁釀,郊祀宗廟皆以醴酒,行之數年,無復釀者。



 尋署石季龍為車騎將軍,率騎三萬討鮮卑鬱粥于離石,俘獲及牛馬十餘萬,鬱粥奔烏丸,悉降其眾城。



 先是,勒世子興死,至是,立子弘為世子,領中領軍。



 遣季龍統中外精卒四萬討徐龕,龕堅守不戰,於是築室返耕,列長圍以守之。晉鎮北將軍劉隗降于勒,拜鎮南
 將軍,封列侯。石季龍攻陷徐龕,送之襄國,勒囊盛於百尺樓自上Ξ殺之,令步都等妻子刳而食之,坑龕降卒三千。晉兗州刺史劉遐懼,自鄒山退屯于下邳。瑯邪內史孫默以瑯邪叛降于勒。徐兗間壘壁多送任請降,皆就拜守宰。



 清河張披為程遐長史,遐甚委暱之,張賓舉為別駕,引參政事。遐疾披去己,又惡賓之權盛。勒世子弘,即遐之甥也,自以有援,欲收威重於朝,乃使弘之母譖之曰:「張披與張賓為游俠,門客日百餘乘,物望皆歸之,非社稷之利也,宜除披以便國家。」勒然之。至是,披取急召不時至,因此遂殺之。賓知遐之間己,遂弗敢請。無
 幾,以遐為右長史,總執朝政,自是朝臣莫不震懼,赴於程氏矣。



 時祖逖卒,勒始侵寇邊戍。勒征虜石他敗王師于酂西,執將軍衛榮而歸。征北將軍祖約懼,退如壽春。勒境內大疫,死者十二三,乃罷徽文殿作。遣其將王陽屯于豫州,有窺窬之志,於是兵難日尋,梁鄭之間騷然矣。



 又遣季龍統中外步騎四萬討曹嶷。先是,嶷議欲徙海中,保根餘山,會疾疫甚,計未及就。季龍進兵圍廣固,東萊太守劉巴、長廣太守呂披皆以郡降。以石他為征東將軍,擊羌胡于河西。左軍石挺濟師于廣固,曹嶷降,送于襄國。勒害之,坑其眾三萬。季龍將盡殺嶷眾,其青
 州刺史劉徵曰:「今留征,使牧人也;無人焉牧,征將歸矣。」季龍乃留男女七百口配征,鎮廣固。青州諸郡縣壘壁盡陷。



 勒司州刺史石生攻晉揚武將軍郭誦于陽翟,不克,進寇襄城,俘獲千餘而還。



 勒以參軍樊垣清貧,擢授章武內史。既而入辭,勒見坦衣冠弊壞,大驚曰:「樊參軍何貧之甚也!」坦性誠朴,率然而對曰:「頃遭羯賊無道,資財蕩盡。」勒笑曰:「羯賊乃爾暴掠邪!今當相償耳。」坦大懼,叩頭泣謝。勒曰:「孤律自防俗士,不關卿輩老書生也。」賜車馬衣服裝錢三百萬,以勵貪俗。



 勒將兵都尉石瞻寇下邳,敗晉將軍劉長,遂寇蘭陵,又敗彭城內史劉續。東
 莞太守竺珍、東海太守蕭誕以郡叛降于勒。



 勒親臨大小學,考諸學生經義,尤高者賞帛有差。勒雅好文學,雖在軍旅,常令儒生讀史書而聽之,每以其意論古帝王善惡,朝賢儒士聽者莫不歸美焉。嘗使人讀《漢書》,聞酈食其勸立六國後,大驚曰:「此法當失,何得遂成天下!」至留侯諫,乃曰:「賴有此耳。」其天資英達如此。



 勒征徐、揚州兵,會石瞻于下邳,劉遐懼,又自下邳奔于泗汭。



 石生攻劉曜河內太守尹平于新安,斬之,剋壘壁十餘,降掠五千餘戶而歸。自是劉、石禍結,兵戈日交,河東、弘農間百姓無聊矣。



 以右常侍霍皓為勸課大夫,與典農使者朱
 表、典勸都尉陸充等循行州郡,核定戶籍,勸課農桑。農桑最修者賜五大夫。



 使石生自延壽關出寇許潁,俘獲萬餘,降者二萬,生遂攻陷康城。晉將軍郭誦追生,生大敗,死者千餘。生收散卒,屯于康城。勒汲郡內史石聰聞生敗,馳救之,進攻郭默,俘獲男女二千餘人。石聰攻敗晉將李矩、郭默等。



 勒將狩於近郊,主簿程瑯諫曰:「劉、馬刺客,離布如林,變起倉卒,帝王亦一夫之敵耳。孫策之禍可不慮乎!且枯木朽株盡能為害,馳騁之弊,今古戒之。」勒勃然曰:「吾幹力自可,足能裁量。但知卿文書事,不須白此輩也。」是日逐獸,馬觸木而死,勒亦幾殆,乃曰:「
 不用忠臣言,吾之過也。」乃賜瑯朝服錦絹,爵關內侯。於是朝臣謁見,忠言競進矣。



 晉都尉魯潛叛,以許昌降於勒。石瞻攻陷晉兗州刺史檀斌於鄒山,斌死之。勒西夷中郎將王勝襲殺并州刺史崔琨、上黨內史王昚,以並州叛於勒。先是,石季龍攻劉曜將劉嶽於石梁,至是,石梁潰,執嶽送襄國。季龍又攻王勝于并州,殺之。李矩以劉嶽之敗也,懼,自滎陽遁歸。矩長史崔宣率矩眾二千降于勒。於是盡有司之地,徐豫濱淮諸郡縣皆降之。



 勒命徙洛陽晷影于襄國,列之單于庭。銘佐命功臣三十九人于石函,置于建德前殿。立桑梓苑于襄國。



 勒嘗
 夜微行,檢察營衛,齎繒帛金銀以賂門者求出。永昌門門候王假欲收捕之,從者至,乃止。旦召假以為振忠都尉,爵關內侯。勒如苑鄉,召記室參軍徐光,光醉不至。以光物情所湊,常不平之,因此發怒,退為牙門。勒自苑鄉如鄴,徐光侍直,慍然攘袂振紛,仰視不顧。勒因而惡之,讓光曰:「何負卿而敢怏怏邪!」於是幽光并其妻子于獄。



 勒既將營鄴宮,又欲以其世子弘為鎮,密與程遐謀之。石季龍自以勳效之重,仗鄴為基,雅無去意。及修構三臺,遷其家室,季龍深恨遐,遣左右數十人夜入遐宅,姦其妻女,掠衣物而去。勒以弘鎮鄴,配禁兵萬人,車騎所
 統五十四營悉配之,以驍騎領門臣祭酒王陽專統六夷以輔之。



 石聰攻壽春,不剋,遂寇逡遒、阜陵,殺掠五千餘人,京師大震。



 濟岷太守劉闓、將軍張闔等叛,害下邳內史夏侯嘉,以下邳降于石生。



 石瞻攻河南太守王羨于邾,陷之。



 龍驤將軍王國叛,以南郡降于勒。晉彭城內史劉續復據蘭陵、石城,石瞻攻陷之。



 勒令州郡,有墳發掘不掩覆者推劾之,骸骨暴露者縣為備棺衾之具。以牙門將王波為記室參軍,典定九流,始立秀、孝試經之制。



 茌平令師懽獲黑兔,獻之於勒,程遐等以為勒「龍飛革命之祥,於晉以水承金,兔陰精之獸,玄為水色,此示殿
 下宜速副天人之望也。」於是大赦,以咸和三年改年曰太和。



 石堪攻晉豫州刺史祖約于壽春,屯師淮上。晉龍驤將軍王國以南郡叛降于堪。南陽都尉董幼叛,率襄陽之眾又降于堪。祖約諸將佐皆陰遣使附于勒。石聰與堪濟淮,陷壽春,祖約奔歷陽,壽春百姓陷於聰者二萬餘戶。



 劉曜敗季龍于高候,遂圍洛陽。勒滎陽太守尹矩、野王太守張進等皆降之,襄國大震。勒將親救洛陽,左右長史、司馬郭敖、程遐等固諫曰:「劉曜乘勝雄盛,難與爭鋒,金墉糧豐,攻之未可卒拔。曜懸軍千里,勢不支久。不可親動,動無萬全,大業去矣。」勒大怒,按劍叱遐等
 出。於是赦徐光,召而謂之曰:「劉曜乘高候之勢,圍守洛陽,庸人之情皆謂其鋒不可當也。然曜帶甲十萬,攻一城而百日不剋,師老卒殆,以我初銳擊之,可一戰而擒。若洛陽不守,曜必送死冀州,自河已北,席卷南向,吾事去矣。程遐等不欲吾親行,卿以為何如?」光對曰:「劉曜乘高候之勢而不能進臨襄國,更守金墉,此其無能為也。懸軍三時,亡攻戰之利,若鸞旗親駕,必望旌奔敗。定天下之計,在今一舉。今此機會,所謂天授,授而弗應,禍之攸集。」勒笑曰:「光之言是也。」佛圖澄亦謂勒曰:「大軍若出,必擒劉曜。」勒尤悅,使內外戒嚴,有諫者斬。命石堪、石聰
 及豫州刺史桃豹等各統見眾會滎陽,使石季龍進據石門,以左衛石邃都督中軍事,勒統步騎四萬赴金墉,濟自大堨。先是,流澌風猛,軍至,冰泮清和,濟畢,流澌大至,勒以為神靈之助也,命曰靈昌津。勒顧謂徐光曰:「曜盛兵成皋關,上計也;阻洛水,其次也;坐守洛陽者成擒也。」諸軍集于成皋,步卒六萬,騎二萬七千。勒見曜無守軍,大悅,舉手指天,又自指額曰:「天也!」乃卷甲銜枚而詭道兼路,出於鞏、訾之間。知曜陳其軍十餘萬于城西,彌悅,謂左右曰:「可以賀我矣!」勒統步騎四萬人自宣陽門,升故太極前殿。季龍步卒三萬,自城北而西,攻其中軍,
 石堪、石聰等各以精騎八千,城西而北,擊其前鋒,大戰於西陽門。勒躬貫甲胄,出自閶闔,夾擊之。曜軍大潰,石堪執曜,送之以徇于軍,斬首五萬餘級,枕尸於金谷。勒下令曰:「所欲擒者一人耳,今已獲之,其敕將士抑鋒止銳,縱其歸命之路。」乃旋師。使征東石邃等帥騎衛曜而北。



 及是,祖約舉兵敗,降於勒,勒使王波讓之曰:「卿逆極勢窮,方來歸命,吾朝豈逋逃之藪邪?而卿敢有靦面目也。」示之以前後檄書,乃赦之。



 劉曜子熙等去長安,奔于上邽,遣季龍討之。



 勒巡行州諸郡,引見高年、孝悌、力田、文學之士,班賜穀帛有差。令遠近牧守宣告屬城,諸
 所欲言,靡有隱諱,使知區區之朝虛渴讜言也。



 季龍剋上邽,遣主簿趙封送傳國玉璽、金璽、太子玉璽各一于勒。季龍進攻集木且羌于河西,剋之,俘獲數萬,秦、隴悉平。涼州牧張駿大懼,遣使稱籓,貢方物于勒,徙氐羌十五萬落于司、冀州。



 勒群臣議以勒功業既隆,祥符並萃,宜時革徽號以答乾坤之望,於是石季龍等奉皇帝璽綬,上尊號于勒,勒弗許。群臣固請,勒乃以咸和五年僭號趙天王,行皇帝事。尊其祖邪曰宣王,父周曰元王。立其妻劉氏為王后,世子弘為太子。署其子宏持節、散騎常侍、都督中外諸軍事、驃騎大將軍、大單于,封秦王;
 左衛將軍斌太原王;小子恢為輔國將軍、南陽王;中山公季龍為太尉、守尚書令、中山王;石生河東王;石堪彭城王;以季龍子邃為冀州刺史,封齊王,加散騎常侍、武衛將軍;宣左將軍;挺侍中、梁王。署左長史郭敖為尚書左僕射,右長史程遐為右僕射、領吏部尚書,左司馬夔安、右司馬郭殷、從事中郎李鳳、前郎中令裴憲為尚書,署參軍事徐光為中書令、領秘書監。論功封爵,開國郡公文武二十一人,侯二十四人,縣公二十六人,侯二十二人,其餘文武各有差。侍中任播等參議,以趙承金為水德,旗幟尚玄,牲牡尚白,子社丑臘,勒從之。勒下書曰:「
 自今有疑難大事,八坐及委丞郎齎詣東堂,詮詳平決。其有軍國要務須啟,有令僕尚書隨局入陳,勿避寒暑昏夜也。」



 勒以祖約不忠於本朝,誅之,及其諸子至親屬百餘人。



 群臣固請勒宜即尊號,勒乃僭即皇帝位,大赦境內,改元曰建平,自襄國都臨漳。追尊其高祖曰順皇,曾祖曰威皇,祖曰宣皇,父曰世宗元皇帝,妣曰元昭皇太后,文武封進各有差。立其妻劉氏為皇后,又定昭儀、夫人位視上公,貴嬪、貴人視列侯,員各一人;三英、九華視伯,淑媛、淑儀視子,容華、美人視男,務簡賢淑,不限員數。



 勒荊州監軍郭敬、南蠻校尉董幼寇襄陽。勒驛敕敬
 退屯樊城,戒之使偃藏旗幟,寂若無人,彼若使人觀察,則告之曰:「自愛堅守,後七八日大騎將至,相策不復得走矣。」敬使人浴馬于津,周而復始,晝夜不絕。偵諜還告南中郎將周撫,撫以為勒軍大至,懼而奔武昌。敬入襄陽,軍無私掠,百姓安之。晉平北將軍魏該弟遐等率該部眾自石城降於敬。敬毀襄陽,遷其百姓于沔北,城樊城以戍之。



 秦州休屠王羌叛于勒,刺史臨深遣司馬管光帥州軍討之,為羌所敗,隴右大擾,氐羌悉叛。勒遣石生進據隴城。王羌兄子擢與羌有仇,生乃賂擢,與掎擊之。羌敗,奔涼州。徙秦州夷豪五千餘戶於雍州。



 勒下書
 曰:「自今諸有處法,悉依科令。吾所忿戮、怒發中旨者,若德位已高,不宜訓罰,或服勤死事之孤,邂逅罹譴,門下皆各列奏之,吾當思擇而行也。」堂陽人陳豬妻一產三男,賜其衣帛廩食,乳婢一口,復三歲勿事。時高句麗、肅慎致其楛矢,宇文屋孤並獻名馬于勒。涼州牧張駿遣長史馬詵奉圖送高昌、于闐、鄯善、大宛使,獻其方物。晉荊州牧陶侃遣兼長史王敷聘于勒,致江南之珍寶奇獸。秦州送白獸、白鹿,荊州送白雉、白兔,濟陰木連理,甘露降苑鄉。勒以休瑞並臻,遐方慕義,赦三歲刑已下,均百姓去年逋調;特赦涼州殊死,涼州計吏皆拜郎中,賜
 絹十匹,綿十斤。勒南郊,有白氣自壇屬天,勒大悅,還宮,赦四歲刑。遣使封張駿武威郡公,食涼州諸郡。勒親耕藉田,還宮,赦五歲刑,賜其公卿已下金帛有差。勒以日蝕,避正殿三日,令群公卿士各上封事。禁州郡諸祠堂非正典者皆除之,其能興雲致雨,有益於百姓者,郡縣更為立祠堂,殖嘉樹,準嶽瀆已下為差等。



 勒將營鄴宮,廷尉續咸上書切諫。勒大怒,曰:「不斬此老臣。朕宮不得成也!」敕御史收之。中書令徐光進曰:「陛下天資聰睿,超邁唐虞,而更不欲聞忠臣之言,豈夏癸、商辛之君邪?其言可用用之,不可用故當容之,奈何一旦以直言而斬
 列卿乎!」勒嘆曰:「為人君不得自專如是!豈不識此言之忠乎?向戲之爾。人家有百匹資,尚欲市別宅,況有天下之富,萬乘之尊乎!終當繕之耳。且敕停作,成吾直臣之氣也。」因賜咸絹百匹,稻百斛。又下書令公卿百僚歲薦賢良、方正、直言、秀異、至孝、廉清各一人,答策上第者拜議郎,中第中郎,下第郎中。其舉人得遞相薦引,廣招賢之路。起明堂、辟雍、靈臺于襄國城西。時大雨霖,中山西北暴水,流漂巨木百餘萬根,集于堂陽。勒大悅,謂公卿曰:「諸卿知不?此非為災也,天意欲吾營鄴都耳。」於是令少府任汪、都水使者張漸等監營鄴宮,勒親授規模。



 蜀
 梓潼、建平、漢固三郡蠻巴降于勒。



 勒以成周土中,漢晉舊京,復欲有移都之意,乃命洛陽為南都,置行臺治書侍御史于洛陽。



 勒因饗高句麗、宇文屋孤使,酒酣,謂徐光曰:「朕方自古開基何等主也?」對曰:「陛下神武籌略邁於高皇,雄藝卓犖超絕魏祖,自三王已來無可比也,其軒轅之亞乎!」勒笑曰:「人豈不自知,卿言亦以太過。朕若逢高皇,當北面而事之,與韓彭競鞭而爭先耳。脫遇光武,當並驅于中原,未知鹿死誰手。大丈夫行事當礌礌落落,如日月皎然,終不能如曹孟德、司馬仲達父子,欺他孤兒寡婦,狐媚以取天下也。朕當在二劉之間耳,軒
 轅豈所擬乎!」其群臣皆頓首稱萬歲。



 晉將軍趙胤攻剋馬頭,石堪遣將軍韓雍救之,至則無及,遂寇南沙、海虞,俘獲五千餘人。初,郭敬之退據樊城也,王師復戍襄陽。至是,敬又攻陷之,留戍而歸。



 暴風大雨,震電建德殿端門、襄國市西門,殺五人。雹起西河介山,大如雞子,平地三尺,洿下丈餘,行人禽獸死者萬數,歷太原、樂平、武鄉、趙郡、廣平、鉅鹿千餘里,樹木摧折,禾稼蕩然。勒正服於東堂,以問徐光曰:「歷代已來有斯災幾也?」光對曰:「周、漢、魏、晉皆有之,雖天地之常事,然明主未始不為變,所以敬天之怒也。去年禁寒食,介推,帝鄉之神也,歷代所尊,
 或者以為未宜替也。一人吁嗟,王道尚為之虧,況群神怨憾而不怒動上帝乎!縱不能令天下同爾,介山左右,晉文之所封也,宜任百姓奉之。」勒下書曰:「寒食既并州之舊風,朕生其俗,不能異也。前者外議以子推諸侯之臣,王者不應為忌,故從其議,倘或由之而致斯災乎!子推雖朕鄉之神,非法食者亦不得亂也,尚書其促檢舊典定議以聞。」有司奏以子推歷代攸尊,請普復寒食,更為植嘉樹,立祠堂,給戶奉祀。勒黃門郎韋謏駁曰:「案《春秋》,藏冰失道,陰氣發泄為雹。自子推已前,雹者復何所致?此自陰陽乖錯所為耳。且子推賢者,曷為暴害如此!
 求之冥趣,必不然矣。今雖為冰室,懼所藏之冰不在固陰沍寒之地,多皆山川之側,氣泄為雹也。以子推忠賢,令綿、介之間奉之為允,於天下則不通矣。」勒從之。於是遷冰室於重陰凝寒之所,并州復寒食如初。



 勒令其太子省可尚書奏事,使中常侍嚴震參綜可否,征伐刑斷大事乃呈之。自是震威權之盛過于主相矣。季龍之門可設雀羅,季龍愈怏怏不悅。



 郭敬南掠江西,晉南中郎將桓宣承其虛攻樊城,取城中之眾而去。敬旋師救樊,追戰于涅水。敬前軍大敗,宣亦死傷太半,盡取所掠而止。宣遂南取襄陽,留軍戍之。



 勒如鄴,臨石季龍第,謂之
 曰:「功力不可並興,待宮殿成後,當為王起第,勿以卑小悒悒也。」季龍免冠拜謝,勒曰:「與王共有天下,何所謝也!」有流星大如象,尾足蛇形,自北極西南流五十餘丈,光明燭地,墜於河,聲聞九百餘里。黑龍見鄴井中,勒觀龍有喜色。朝其群臣于鄴。



 命郡國立學官,每郡置博士祭酒二人,弟子百五十人,三考修成,顯升台府。於是擢拜太學生五人為佐著作郎,錄述時事。時大旱,勒親臨廷尉錄囚徒,五歲刑已下皆輕決遣之,重者賜酒食,聽沐浴,一須秋論。還未及宮,澍雨大降。



 勒如其灃水宮,因疾甚而還。召石季龍與其太子弘、中常侍嚴震等待疾禁
 中。季龍矯命絕弘、震及內外群臣親戚,勒疾之增損莫有知者。詐召石宏、石堪還襄國。勒疾小瘳,見宏,驚曰:「秦王何故來邪?使王籓鎮,正備今日。有呼者邪?自來也?有呼者誅之!」季龍大懼曰:「秦王思慕暫還耳,今謹遣之。」數日復問之,季龍曰:「奉詔即遣,今已半路矣。」更諭宏在外,遂不遣之。



 廣阿蝗。季龍密遣其子邃率騎三千游于蝗所。熒惑人昴。星隕于鄴東北六十里,初赤黑黃雲如幕,長數十匹,交錯,聲如雷震,墜地氣熱如火,塵起連天。時有耕者往視之,土猶燃沸,見有一石方尺餘,青色而輕,擊之間聲如磬。



 勒疾甚,遺令:「三日而葬,內外百僚既葬除
 服,無禁婚娶、祭祀、飲酒、食肉,征鎮牧守不得輒離所司以奔喪,斂以時服,載以常車,無藏金寶,無內器玩。大雅沖幼,恐非能構荷朕志。中山已下其各司所典,無違朕命。大雅與斌宜善相維持,司馬氏汝等之殷鑒,其務於敦穆也。中山王深可三思周霍,勿為將來口實。」以咸和七年死,時年六十,在位十五年。夜瘞山谷,莫知其所,備文物虛葬,號高平陵。偽謚明皇帝,廟號高祖。



 弘字大雅,勒之第二子也。幼有孝行,以恭謙自守,受經於杜嘏,誦律於續咸。勒曰:「今世非承平,不可專以文業教也。」於是使劉徵、任播授以兵書,王陽教之擊刺。立為
 世子,領中領軍,尋暑衛將軍,使領開府辟召,後鎮鄴。



 勒僭位,立為太子。虛襟愛士,好為文詠,其所親暱,莫非儒素。勒謂徐光曰:「大雅愔愔,殊不似將家子。」光曰:「漢祖以馬上取天下,孝文以玄默守之,聖人之後,必世勝殘,天之道也。」勒大悅。光因曰:「皇太子仁孝溫恭,中山王雄暴多詐,陛下一旦不諱,臣恐社稷必危,宜漸奪中山威權,使太子早參朝政。」勒納之。程遐又言於勒曰:「中山王勇武權智,群臣莫有及者。觀其志也,自陛下之外,視之蔑如。兼荷專征歲久,威振外內,性又不仁,殘忍無賴。其諸子並長,皆預兵權。陛下在,自當無他,恐其怏怏不可輔
 少主也。宜早除之,以便大計。」勒曰:「今天下未平,兵難未已,大雅沖幼,宜任強輔。中山佐命功臣,親同魯衛,方委以伊霍之任,何至如卿言也。卿當恐輔幼主之日,不得獨擅帝舅之權故耳。吾亦當參卿於顧命,勿為過懼也。」遐泣曰:「臣所言者至公,陛下以私賜距,豈明主開襟納說,忠臣必盡之義乎!中山雖為皇太后所養,非陛下天屬,不可以親義期也。杖陛下神規,微建鷹犬之效,陛下酬其父子以恩榮,亦以足矣。魏任司馬懿父子,終於鼎祚淪移,以此而觀,中山豈將來有益者乎!臣因緣多幸,託瓜葛於東宮,臣而不竭言於陛下,而誰言之!陛下若
 不除中山,臣已見社稷不復血食矣。」勒不聽。遐退告徐光曰:「主上向言如此,太子必危,將若之何?」光曰:「中山常切齒於吾二人,恐非但國危,亦為家禍,當為安國寧家之計,不可坐而受禍也。」光復承間言於勒曰:「陛下廓平八州,帝有海內,而神色不悅者何也?」勒曰:「吳、蜀未平,書軌不一,司馬家猶不絕於丹陽,恐後之人將以吾為不應符錄,每一思之,不覺見於神色。」光曰:「臣以陛下為憂腹心之患,而何暇更憂四支手!何則?魏承漢運,為正朔帝王,劉備雖紹興巴、蜀,亦不可謂漢不滅也。吳雖跨江東,豈有虧魏美?陛下既苞括二都,為中國帝王,彼司馬
 家兒復何異玄德,李氏亦猶孫權。符籙不在陛下,竟欲安歸?此四支之輕患耳。中山王藉陛下指授神略,天下皆言其英武亞於陛下,兼其殘暴多姦,見利忘義,無伊、霍之忠。父子爵位之重,勢傾王室。觀其耿耿,常有不滿之心。近於東宮曲宴,有輕皇太子之色。陛下隱忍容之,臣恐陛下萬年之後,宗廟必生荊刺,此心腹之重疾也,惟陛下圖之。」勒默然,而竟不從。



 及勒死,季龍執弘使臨軒,命收程遐、徐光下廷尉,召其子邃率兵入宿衛,文武靡不奔散。弘大懼,讓位於季龍。季龍曰:「君薨而世子立,臣安敢亂之!」弘泣而固讓,季龍怒曰:「若其不堪,天下自
 當有大議,何足預論!」遂以咸和七年逼立之,改年曰延熙,文武百僚進位一等。誅程遐、徐光。弘策拜季龍為丞相、魏王、大單于,加九錫,以魏郡等十三郡為邑,總攝百揆。季龍偽固讓,久而受命,赦其境內殊死已下,立季龍妻鄭氏為魏王后,子邃為魏太子,加使持節、侍中、大都督中外諸軍事、大將軍、錄尚書事;宣為使持節、車騎大將軍、冀州刺史,封河間王;韜為前鋒將軍、司隸校尉,封樂安王;遵齊王,鑒代王,苞樂平王;徙太原王斌為章武王。勒文武舊臣皆補左右丞相閑任,季龍府僚舊暱悉署臺省禁要。命太子宮曰崇訓宮,勒妻劉氏已下皆居
 之。簡其美淑及勒車馬、珍寶、服御之上者,皆入于己署。鎮軍夔安領左僕射,尚書郭殷為右僕射。



 劉氏謂石堪曰:「皇祚之滅不復久矣,王將何以圖之?」堪曰:「先帝舊臣皆已斥外,眾旅不復由人,宮殿之內無所措籌,臣請出奔袞州,據廩丘,挾南陽王為盟主,宣太后詔於諸牧守征鎮,令各率義兵同討桀逆,蔑不濟也。」劉氏曰:「事急矣,便可速發,恐事淹變生。」堪許諾,微服輕騎襲袞州,失期,不剋,遂南奔譙城。季龍遣其將郭太等追擊之,獲堪於城父,送襄國,炙而殺之。征石恢還于襄國。劉氏謀泄,季龍殺之。尊弘母程氏為皇太后。



 時石生鎮關中,石朗
 鎮洛陽,皆起兵于二鎮。季龍留子邃守襄國,統步騎七萬攻郎于金墉。金墉潰,獲朗,刖而斬之。進師攻長安,以石挺為前鋒大都督。生遣將軍郭權率鮮卑涉璝部眾二萬為前鋒距之,生統大軍繼發,次于蒲阪。前鋒及挺大戰潼關,敗績,挺及丞相左長史劉隗皆戰死,季龍退奔澠池,枕尸三百餘里。鮮卑密通于季龍,背生而擊之。生時停蒲阪,不知挺之死也,懼,單馬奔長安。郭權乃復收眾三千,與越騎校尉石廣相持于渭汭。生遂去長安,潛于雞頭山。將軍蔣英固守長安。季龍聞生之奔也,進師入關,進攻長安,旬餘拔之,斬蔣英等。分遣諸將屯于
 汧。徙雍、秦州華戎十餘萬戶於關東。生部下斬生于雞頭山。季龍還襄國,大赦,諷弘命己建魏臺,一如魏輔漢故事。



 郭權以生敗,據上邽以歸順,詔以權為鎮西將軍、秦州刺史,於是京兆、新平、扶風、馮翊、北地皆應之。弘鎮西石廣與權戰,敗績。季龍遣郭敖及其子斌等率步騎四萬討之,次于華陰。上邽豪族害權以降。徙秦州三萬餘戶于青、并二州諸郡。南氐、楊難敵等送任通和。長安陳良夫奔于黑羌,招誘北羌四角王薄句大等擾北地、馮翊,與石斌相持。石韜等率騎掎句大之後,與斌夾擊,敗之,句大奔于馬蘭山。郭敖等懸軍追北,為羌所敗,死
 者十七八。斌等收軍還于三城。季龍聞而大怒,遣使殺郭敖。石宏有怒言,季龍幽之。



 弘齎璽綬親詣季龍,諭禪位意。季龍曰:「天下人自當有議,何為自論此也!」弘還宮,對其母流涕曰:「先帝真無復遺矣!」俄而季龍遣丞相郭殷持節入,廢弘為海陽王。弘安步就車,容色自若,謂群臣曰:「不堪纂承大統,顧慚群后,此亦天命去矣,又何言!」百官莫不流涕,宮人慟哭。咸康元年,幽弘及程氏並宏、恢于崇訓宮,尋殺之,在位二年,時年二十二。



 張賓,字孟孫,趙郡中丘人也。父瑤,中山太守。賓少好學,
 博涉經史,不為章句,闊達有大節,常謂昆弟曰:「吾自言智算鑒識不後子房,但不遇高祖耳。」為中丘王帳下都督,非其好也,病免。及永嘉大亂,石勒為劉元海輔漢將軍,與諸將下山東,賓謂所親曰:「吾歷觀諸將多矣,獨胡將軍可與共成大事。」乃提劍軍門,大呼請見,勒亦未之奇也。後漸進規謨,乃異之,引為謀主。機不虛發,算無遺策,成勒之基業,皆賓之勳也。及為右長史、大執法,封濮陽侯,任遇優顯,寵冠當時,而謙虛敬慎,開襟下士,士無賢愚,造之者莫不得盡其情焉。肅清百僚,屏絕私暱,入則格言,出則歸美。勒甚重之,每朝,常為之正容貌,簡辭
 令,呼曰「右侯」而不名之,勒朝莫與為比也。



 及卒,勒親臨哭之,哀慟左右,贈散騎常侍、右光祿大夫、儀同三司,謚曰景。將葬,送於正陽門,望之流涕,顧左右曰:「天欲不成吾事邪,何奪吾右侯之早也!」程遐代為右長史,勒每與遐議,有所不合,輒嘆曰:「右侯舍我去,令我與此輩共事,豈非酷乎!」因流涕彌日。



\end{pinyinscope}