\article{帝紀第五}

\begin{pinyinscope}

 孝懷帝孝愍帝



 孝懷皇帝諱熾,字豐度,武帝第二十五子也。太熙元年,封豫章郡王。屬孝惠之時,宗室構禍,帝沖素自守,門絕賓游,不交世事,專玩史籍,有譽於時。初拜散騎常侍,及趙王倫篡,見收。倫敗,為射聲校尉。累遷車騎大將軍、都督青州諸軍事。未之鎮。永興元年,改授鎮北大將軍、都督鄴城守諸軍事。十二月丁亥,立為皇太弟。帝以清河王
 覃本太子也,懼不敢當。典書令廬陵脩肅曰:「二相經營王室,志寧社稷,儲貳之重,宜歸時望,親賢之舉,非大王而誰?清河幼弱,未允眾心,是以既升東宮,復贊籓國。今乘輿播越,二宮久曠,常恐氐羌飲馬於涇川,蟻眾控弦於霸水。宜及吉辰,時登儲副,上翼大駕,早寧東京,下允黔首喁喁之望。」帝曰:「卿,吾之宋昌也。」乃從之。



 光熙元年十一月庚午,孝惠帝崩。羊皇后以於太弟為嫂,不得為太后,催清河王覃入,已至尚書閣,侍中華混等急召太弟。癸酉,即皇帝位,大赦,尊皇后羊氏為惠皇后,居弘訓宮,追尊所生太妃王氏為皇太后,立妃梁氏
 為皇后。十二月壬午朔,日有蝕之。己亥,封彭城王植子融為樂城縣王。南陽王模殺河間王顒於雍谷。辛丑,以中書監溫羨為司徒,尚書左僕射王衍為司空。己酉,葬孝惠皇帝于太陽陵。李雄別帥李離寇梁州。



 永嘉元年春正月癸丑朔,大赦,改元,除三族刑。以太傅、東海王越輔政,殺御史中丞諸葛玫。二月辛巳,東萊人王彌起兵反,寇青、徐二州,長廣太守宋羆、東牟太守龐伉並遇害。三月己未朔,平東將軍周馥斬送陳敏首。丁卯,改葬武悼楊皇后。庚午,立豫章王詮為皇太子。辛未,大赦。庚辰,東海王越出鎮許昌。以征東將軍、高密王簡
 為征南大將軍、都督荊州諸軍事,鎮襄陽;改封安北將軍、東燕王騰為新蔡王、都督司冀二州諸軍事,鎮鄴;以征南將軍、南陽王模為征西大將軍、都督秦雍梁益四州諸軍事,鎮長安。並州諸郡為劉元海所陷,刺史劉琨獨保晉陽。夏五月,馬牧帥汲桑聚眾反,敗魏郡太守馮嵩,遂陷鄴城,害新蔡王騰。燒鄴宮,火旬日不滅。又殺前幽州刺史石鮮於樂陵,入掠平原,山陽公劉秋遇害。洛陽步廣里地陷,有二鵝出,色蒼者沖天,白者不能飛。建寧郡夷攻陷寧州,死者三千餘人。秋七月己酉朔,東海王越進屯官渡,以討汲桑。己未,以平東將軍、瑯邪王睿
 為安東將軍、都督揚州江南諸軍事、假節,鎮建鄴。八月己卯朔,撫軍將軍茍晞敗汲桑於鄴。甲辰,曲赦幽、并、司、冀、兗、豫等六州。分荊州、江州八郡為湘州。九月戊申,茍晞又破汲桑,陷其九壘。辛亥,有大星如日,小者如斗,自西方流於東北,天盡赤,俄有聲如雷。始脩千金堨於許昌以通運。冬十一月戊申朔,日有蝕之。甲寅,以尚書右僕射和郁為征北將軍,鎮鄴。十二月戊寅,并州人田蘭、薄盛等斬汲桑於樂陵。甲午,以前太傅劉寔為太尉。庚子,以光祿大夫、延陵公高光為尚書令。東海王越矯詔囚清河王覃於金墉城。癸卯,越自為丞相。以撫軍將軍
 茍晞為征東大將軍。



 二年春正月丙子朔,日有蝕之,日有蝕之。丁未,大赦。二月辛卯,清河王覃為東海王越所害。庚子,石勒寇常山,安北將軍王浚討破之。三月,東海王越鎮鄄城。劉元海侵汲郡,略有頓丘、河內之地。王彌寇青、徐、兗、豫四州。夏四月丁亥,入許昌,諸郡守將皆奔走。五月甲子,彌遂寇洛陽,司徒王衍帥眾禦之,彌退走。秋七月甲辰,劉元海寇平陽,太守宋抽奔京師,河東太守路述力戰,死之。八月丁亥,東海王越自鄄城遷屯于濮陽。九月,石勒寇趙郡,征北將軍和郁自鄴奔于衛國。冬十月甲戌,劉元海僭帝號于平陽,
 仍稱漢。十一月乙巳,尚書令高光卒;丁卯,以太子少傅荀籓為尚書令。己酉,石勒寇鄴,魏郡太守王粹戰敗,死之。十二月辛未朔,大赦。立長沙王乂子碩為長沙王,鮮為臨淮王。



 三年春正月甲午,彭城王釋薨。三月戊申,征南大將軍、高密王簡薨。以尚書左僕射山簡為征南將軍、都尉荊湘交廣等四州諸軍事,司隸校尉劉暾為尚書左僕射。丁巳,東海王越歸京師。乙丑,勒兵人宮,於帝側收近臣中書令繆播、帝舅王延等十餘人,並害之。丙寅,曲赦河南郡。丁卯,太尉劉寔請老,以司徒王衍為太尉。東海王
 越領司徒。劉元海冠黎陽,遣車騎將軍王堪擊之,王師敗績于延津,死者三萬餘人。大旱,江、漢、河、洛皆竭,可涉。夏四月,左積弩將軍朱誕叛奔于劉元海。石勒攻陷冀州郡縣百餘壁。秋七月戊辰,當陽地裂三所,各廣三丈,長三百餘步。辛未,平陽人劉芒蕩自稱漢後,誑誘羌戎,僭帝號於馬蘭山。支胡五斗叟、郝索聚眾數千為亂,屯新豐,與芒蕩合黨。劉元海遣子聰及王彌寇上黨,圍壺關。并州刺史劉琨使兵救之,為聰所敗。淮南內史王曠、將軍施融、曹超及聰戰,又敗,超、融死之。上黨太守龐淳以郡降賊。九月丙寅,劉聰圍浚儀,遣平北將軍曹武討
 之。丁丑,王師敗績。東海王越人保京城。聰至西明門,越禦之,戰于宣陽門外,大破之。石勒寇常山,安北將軍王浚使鮮卑騎救之,大破勒於飛龍山。征西大將軍、南陽王模使其將淳于定破劉芒蕩、五斗叟,並斬之。使車騎將軍王堪、平北將軍曹武討劉聰,王師敗績,堪奔還京師。李雄別帥羅羨以梓潼歸順。劉聰攻洛陽西明門,不克。宜都夷道山崩,荊、湘二州地震。冬十一月,石勒陷長樂,安北將軍王斌遇害。因屠黎陽。乞活帥李惲、薄盛等帥眾救京師,聰退走。惲等又破王彌於新汲。十二月乙亥,夜有白氣如帶,自地升天,南北各二丈。



 四年春正月乙丑朔,大赦。二月,石勒襲鄄城,兗州刺史袁孚戰敗,為其部下所害。勒又襲白馬,車騎將軍王堪死之。李雄將文碩殺雄大將軍李國,以巴西歸順。戊午,吳興人錢璯反,自稱平西將軍。三月,丞相倉曹屬周帥鄉人討璯,斬之。夏四月,大水。將軍祁弘破劉元海將劉靈曜于廣宗。李雄陷梓潼。兗州地震。五月,石勒寇汲郡,執太守胡寵,遂南濟河,滎陽太守裴純奔建鄴。大風折木。地震。幽、并、司、冀、秦、雍等六州大蝗,食草木,牛馬毛皆盡。六月,劉元海死,其子和嗣偽位,和弟聰殺和而自立。秋七月,劉聰從弟曜及其將石勒圍懷,詔征虜將軍
 宋抽救之,為曜所敗,抽死之。九月,河內人樂仰執太守裴整叛,降于石勒。徐州監軍王隆自下邳棄軍奔于周馥。雍州人王如舉兵反于宛,殺害令長,自號大將軍、司雍二州牧,大掠漢沔,新平人龐寔、馮翊人嚴嶷、京兆人侯脫等各起兵應之。征南將軍山簡、荊州刺史王澄、南中郎將杜蕤並遣兵援京師,及如戰于宛,諸軍皆大敗;王澄獨以眾進至沶口,眾潰而歸。冬十月辛卯,晝昏,至於庚子。大星西南墜,有聲。壬寅,石勒圍倉垣,陳留內史王讚擊敗之,勒走河北。壬子,以驃騎將軍王浚為司空,平北將軍劉琨為平北大將軍。京師饑。東海王越羽檄
 徵天下兵,帝謂使者曰:「為我語諸征鎮,若今日,尚可救,後則無逮矣。」時莫有至者。石勒陷襄城,太守崔曠遇害,遂至宛。王浚遣鮮卑文鴦帥騎救之,勒退。浚又遣別將王申始討勒于汶石津,大破之。十一月甲戌,東海王越帥眾出許昌,以行臺自隨。宮省無復守衛,荒饉日甚,殿內死人交橫,府寺營署並掘塹自守,盜賊公行,桴鼓之音不絕。越軍次項,自領豫州牧,以太尉王衍為軍司。丁丑,流氐隗伯等襲宜都,太守嵇晞奔建鄴。王申始攻劉曜、王彌於瓶壘,破之。鎮東將軍周馥表迎大駕遷都壽陽,越使裴碩討馥,為馥所敗,走保東城,請救於瑯邪
 王睿。襄陽大疫,死者三千餘人。加涼州刺史張軌安西將軍。十二月,征東大將軍茍晞攻王彌別帥曹嶷,破之。乙酉,平陽人李洪帥流人入定陵作亂。



 五年春正月,帝密詔茍晞討東海王越。壬申,晞為曹嶷所破。乙未,越遣從事中郎將楊瑁、徐州刺史裴盾共擊晞。癸酉,石勒入江夏,太守楊氏奔于武昌。乙亥,李雄攻陷涪城,梓潼太守譙登遇害。湘州流人杜弢據長沙反。戊寅,安東將軍、瑯邪王睿使將軍甘卓攻鎮東將軍周馥於壽春,馥眾潰。庚辰,太保、平原王乾薨。二月,石勒寇汝南,汝南王祐奔建鄴。三月戊午,詔下東海王越罪狀,
 告方鎮討之。以證東大將軍茍晞為大將軍。丙子,東海王越薨。四月戊子,石勒追東海王越喪,及於東郡,將軍錢端戰死,軍潰,太尉王衍、吏部尚書劉望、廷尉諸葛銓、尚書鄭豫、武陵王澹等皆遇害,王公已下死者十餘萬人。東海世子毗及宗室四十八王尋又沒于石勒。賊王桑、冷道陷徐州,刺史裴盾遇害,桑遂濟淮,至于歷陽。五月,益州流人汝班、梁州流人蹇撫作亂於湘州,虜刺史茍眺,南破零、桂諸郡,東掠武昌,安城太守郭察、劭陵太守鄭融、衡陽內史滕育並遇害。進司空王浚為大司馬,征西大將軍、南陽王模為太尉,太子太傅傅祗為司徒,
 尚書令荀籓為司空,安東將軍、瑯邪王睿為鎮東大將軍。東海王越之出也,使河南尹潘滔居守。大將軍茍晞表遷都倉垣,帝將從之,諸大臣畏滔,不敢奉詔,且宮中及黃門戀資財,不欲出。至是饑甚,人相食,百官流亡者十八九。帝召群臣會議,將行而警衛不備。帝撫手歎曰:「如何會無車輿!」乃使司徒傅祗出詣河陰,脩舟楫,為水行之備。朝士數人導從。帝步出西掖門。至銅馳街,為盜所掠,不得進而還。六月癸未,劉曜、王彌、石勒同寇洛川,王師頻為賊所敗,死者甚眾。庚寅,司空荀籓、光祿大夫荀組奔轘轅,太子左率溫幾夜開廣莫門奔小平
 津。丁酉、劉曜、王彌入京師。帝開華林園門,出河陰藕池,欲幸長安,為曜等所追及。曜等遂焚燒宮廟,逼辱妃后,吳王晏、竟陵王楙、尚書左僕射和郁、右僕射曹馥、尚書閭丘沖、袁粲、王緄、河南尹劉默等皆遇害,百官士庶死者三萬餘人。帝蒙塵於平陽,劉聰以帝為會稽公。荀籓移檄州鎮,以瑯邪王為盟主。豫章王端東奔茍晞,晞立為皇太子,自領尚書令,具置官屬,保梁國之蒙縣。百姓饑儉,米斛萬餘價。秋七月,大司馬王浚承制假立太子,置百官,署征鎮。石勒寇穀陽,沛王滋戰敗遇害。八月,劉聰使子粲攻陷長安,太尉、征西將軍、南陽王模遇害,長
 安遺人四千餘家奔漢中。九月癸亥,石勒襲陽夏,至於蒙縣,大將軍茍晞、豫章王端並沒于賊。冬十月,勒寇豫州,諸軍至江而還。十一月,猗盧寇太原,平北將軍劉琨不能制,徙五縣百姓於新興,以其地居之。



 六年春正月,帝在平陽。劉聰寇太原。故鎮南府牙門將胡亢聚眾寇荊土,自號楚公。二月壬子,日有蝕之。癸丑,鎮東大將軍、瑯邪王睿上尚書,檄四方以討石勒。大司馬王浚移檄天下,稱被中詔承制,以荀籓為太尉。汝陽王熙為石勒所害。夏四月丙寅,征南將軍山簡卒。秋七月,歲星、熒惑、太白聚於牛斗。石勒寇冀州。劉粲寇晉陽,
 平北將軍劉琨遣部將郝詵帥眾禦粲,詵敗績,死之,太原太守高喬以晉陽降粲。八月庚戌,劉琨奔於常山。辛亥,陰平都尉董沖逐太守王鑒,以郡叛降于李雄。乙亥,劉琨乞師于猗盧,表盧為代公。九月己卯,猗盧使子利孫赴琨,不得進。辛巳,前雍州刺史賈疋討劉粲於三輔,走之,關中小定,乃與衛將軍梁芬、京兆太守梁綜共奉秦王鄴為皇太子於長安。冬十月,猗盧自將六萬騎次於盆城。十一月甲午,劉粲遁走,劉琨收其遺眾,保于陽曲。是歲大疫。



 七年春正月,劉聰大會,使帝著青衣行酒。侍中庾氏號
 哭,聰惡之。丁未,帝遇弒,崩于平陽,時年三十。



 帝初誕,有嘉禾生于豫章之南昌。先是望氣者云「豫章有天子氣」,其後竟以豫章王為皇太弟。在東宮,恂恂謙損,接引朝士,講論書籍。及即位,始遵舊制,臨太極殿,使尚書郎讀時令,又於東堂聽政。至於宴會,輒與群官論眾務,考經籍。黃門侍郎傅宣歎曰:「今日復見武帝之世矣!」秘書監荀崧又常謂人曰:「懷帝天姿清劭,少著英猷,若遭承平,足為守文佳主。而繼惠帝擾亂之後,東海專政,無幽厲之釁,而有流亡之禍。」



 孝愍皇帝諱鄴,字彥旗,武帝孫,吳孝王晏之子也。出繼後伯父秦獻王柬,襲封秦王。永嘉二年,拜散騎常侍、撫軍將軍。及洛陽傾覆,避難於滎陽密縣,與舅荀籓、荀組相遇,自密南趨許潁。豫州刺史閻鼎與前撫軍長史王毗、司徒長史劉疇、中書郎李昕及籓、組等同謀奉帝歸于長安,而疇等中塗復叛,鼎追殺之,籓、組僅而獲免。鼎遂挾帝乘牛車,自宛趣武關,頻遇山賊,士卒亡散,次于藍田。鼎告雍州刺史賈疋,疋遽遣州兵迎衛,達于長安,又使輔國將軍梁綜助守之。時有玉龜出霸水,神馬鳴城南焉。六年九月辛巳,奉秦王為皇太子,登壇告類,建
 宗廟社稷,大赦。加疋征西大將軍,以秦州刺史、南陽王保為大司馬。賈疋討賊張連,遇害,眾推始平太守麴允領雍州刺史,為盟主,承制選置。



 建興元年夏四月丙午,奉懷帝崩問,舉哀成禮。壬申,即皇帝位,大赦,改元。以衛將軍梁芬為司徒,雍州刺史麴允為使持節、領軍將軍、錄尚書事,京兆大守索綝為尚書右僕射。石勒攻龍驤將軍李惲於上白,惲敗,死之。五月壬辰,以鎮東大將軍、瑯邪王睿為侍中、左丞相、大都督陜東諸軍事,大司馬、南陽王保為右丞相、大都督陜西諸軍事。又詔二王曰:「夫陽九百六之災,雖在盛世,猶或遘之。朕以幼沖,纂承
 洪緒,庶憑祖宗之靈,群公義士之力,蕩滅兇寇,拯拔幽宮,瞻望未達,肝心分裂。昔周邵分陜,姬氏以隆;平王東遷,晉鄭為輔。今左右丞相茂德齊聖,國之暱屬,當恃二公,掃除鯨鯢,奉迎梓宮,克復中興。令幽、並兩州勒卒三十萬,直造平陽。右丞相宜帥秦、涼、梁、雍武旅三十萬,徑詣長安。左丞相帥所領精兵二十萬,徑造洛陽。分遣前鋒,為幽并後駐。赴同大限,克成元勳。」又詔瑯邪王曰:「朕以沖昧,纂承洪緒,未能梟夷凶逆,奉迎梓宮,枕戈煩冤,肝心抽裂。前得魏浚表,知公帥先三軍,已據壽春,傳檄諸侯,協齊威勢,想今漸進,已達洛陽。涼州刺史張軌,乃
 心王室,連旗萬里,已到水幵隴;梁州刺史張光,亦遣巴漢之卒,屯在駱谷:秦川驍勇,其會如林。間遣使適還,具知平陽定問,云幽並隆盛,餘胡衰破,然猶恃險,當須大舉。未知公今所到,是以息兵秣馬,未便進軍。今為已至何許,當須來旨,便乘輿自出,會除中原也。公宜思弘謀猷,勖濟遠略,使山陵旋反,四海有賴。故遣殿中都尉劉蜀、蘇馬等具宣朕意。公茂德暱屬,宣隆東夏,恢融六合,非公而誰!但洛都陵廟,不可空曠,公宜鎮撫,以綏山東。右丞相當入輔弼,追蹤周邵,以隆中興也。」六月,石勒害兗州刺史田徽。是時,山東郡邑相繼陷于勒。秋八月癸亥,
 劉蜀等達于揚州。改建鄴為建康,改鄴為臨漳。杜弢寇武昌,焚燒城邑。弢別將王真襲沔陽,荊州刺史周顗奔于健康。九月,司空荀籓薨于滎陽。劉聰寇河南,河南尹張髦死之。冬十月,荊州刺史陶侃討杜弢黨杜曾於石城,為曾所敗。己巳,大雨雹。庚午,大雪。十一月,流人楊武攻陷梁州。十二月,河東地震,雨肉。



 二年春正月己巳朔,黑霧著人如墨,連夜,五日乃止。辛未,辰時日隕于地。又有三日相承,出于西方而東行。丁丑,大赦。楊武大略漢中,遂奔李雄。二月壬寅,以司空王浚為大司馬,衛將軍荀組為司空,涼州刺史張軌為太
 尉,封西平郡公,並州刺史劉琨為大將軍。三月癸酉,石勒陷幽州,殺侍中、大司馬、幽州牧、博陵公王浚,焚燒城邑,害萬餘人。杜弢別帥王真襲荊州刺史陶侃於林鄣,侃奔灄中。夏四月甲辰,地震。五月壬辰,太尉、領護羌校尉、涼州刺史、西平公張軌薨。六月,劉曜、趙冉寇新豐諸縣,安東將軍索綝討破之。秋七月,曜、冉等又逼京都,領軍將軍麴允討破之,冉中流矢而死。九月,北中郎將劉演剋頓丘,斬石勒所署太守邵攀。丙戌,麟見襄平。單于代公猗盧遣使獻馬。蒲子馬生人。



 三年春正月,盜殺晉昌太守趙佩。吳興人徐馥害太守
 袁琇。以侍中宋哲為平東將軍。屯華陰。二月丙子,進左丞相、瑯邪王睿為大都督、督中外諸軍事,右丞相、南陽王保為相國,司空荀組為太尉,大將軍劉琨為司空。進封代公猗盧為代王。荊州刺史陶侃破王真於巴陵。杜弢別將杜弘、張彥與臨川內史謝摛戰于海昏,摛敗績,死之。三月,豫章內史周訪擊杜弘,走之,斬張彥於陳。夏四月,大赦。五月,劉聰寇並州。六月,盜發漢霸、杜二陵及薄太后陵,太后面如生,得金玉彩帛不可勝記。時以朝廷草創,服章多闕,敕收其餘,以實內府。丁卯,地震。辛巳,大赦。敕雍州掩骼埋胔,脩復陵墓,有犯者誅及三族。秋
 七月,石勒陷濮陽,害太守韓弘。劉聰寇上黨,劉琨遣將救之。八月癸亥,戰於襄垣,王師敗績。荊州刺史陶侃攻杜弢,弢敗走,道死,湘州平。九月,劉曜寇北地,命領軍將軍麴允討之。冬十月,允進攻青白城。以豫州牧、征東將軍索綝為尚書僕射、都督宮城諸軍事。劉聰陷馮翊,太守梁肅奔萬年。十二月,涼州刺史張寔送皇帝行璽一紐。盜殺安定太守趙班。



 四年春三月,代王猗盧薨,其眾歸于劉琨。夏四月丁丑,劉曜寇上郡,太守籍韋率其眾奔於南鄭。涼州刺史張寔遣步騎五千來赴京都。石勒陷廩丘,北中郎將劉演
 出奔。五月,平夷太守雷照害南廣太守孟桓,帥二郡三千餘家叛降于李雄。六月丁巳朔,日有蝕之。大蝗。秋七月,劉曜攻北地,麴允帥步騎三萬救之。王師不戰而潰,北地太守麴昌奔于京師。曜進至涇陽,渭北諸城悉潰,建威將軍魯克、散騎常侍梁緯、少府皇甫陽等皆死之。八月,劉曜逼京師,內外斷絕,鎮西將軍焦嵩、平東將軍宋哲、始平太守竺恢等同赴國難,麴允與公卿守長安小城以自固,散騎常侍華輯監京兆、馮翊、弘農、上洛四郡兵東屯霸上,鎮軍將軍胡崧帥城西諸郡兵屯遮馬橋,並不敢進。冬十月,京師饑甚,米斗金二兩,人相食,死
 者太半。太倉有曲數餅,麴允屑為粥以供帝,至是復盡。帝泣謂允曰:「今窘厄如此,外無救援,死於社稷,是朕事也。然念將士暴離斯酷,今欲因城未陷為羞死之事,庶令黎元免屠爛之苦。行矣遣書,朕意決矣。」十一月乙末,使侍中宋敞送箋於曜,帝乘羊車,肉袒銜壁,輿櫬出降。群臣號泣攀車,執帝之手,帝亦悲不自勝。御史中丞吉朗自殺。曜焚櫬受壁,使宋敞奉帝還宮。初,有童謠曰:「天子何在豆田中。」時王浚在幽州,以豆有藿,殺隱士霍原以應之。及帝如曜營,營實在城東豆田壁。辛丑,帝蒙塵於平陽,麴允及群官並從。劉聰假帝光祿大夫、懷安
 侯。壬寅,聰臨殿,帝稽首于前,麴允伏地慟哭,因自殺。尚書梁允、侍中梁浚、散騎常侍嚴敦、左丞臧振、黃門侍郎任播、張偉、杜曼及諸郡守並為曜所害,華輯奔南山。石勒圍樂平,司空劉琨遣兵援之,為勒所敗,樂平太守韓據出奔。司空長史李弘以并州叛降于勒。十二月甲申朔,日有蝕之。己未,劉琨奔薊,依段匹磾。



 五年春正月,帝在平陽。庚子,虹霓彌天,三日並照。平東將軍宋哲奔江左。李雄使其將李恭、羅寅寇巴東。二月,劉聰使其將劉暢攻滎陽,太守李矩擊破之。三月,瑯邪王睿承制改元,稱晉王於建康。夏五月丙子,日有蝕之。
 秋七月,大暑,司、冀、青、雍等四州螽蝗。石勒亦競取百姓禾,時人謂之「胡蝗」。八月,劉聰使趙固襲衛將軍華薈於定潁,遂害之。冬十月丙子,日有蝕之。劉聰出獵,令帝行車騎將軍,戎服執戟為導,百姓聚而觀之,故老或歔欷流涕,聰聞而惡之。聰後因大會,使帝行酒洗爵,反而更衣,又使帝執蓋,晉臣在坐者多失聲而泣,尚書郎辛賓抱帝慟哭,為聰所害。十二月戊戌,帝遇弒,崩于平陽,時年十八。



 帝之繼皇統也,屬永嘉之亂,天下崩離,長安城中戶不盈百,牆宇頹毀,蒿棘成林。朝廷無車馬章服,唯桑版署號而已。眾唯一旅,公私有車四乘,器械多闕,運
 饋不繼。巨猾滔天,帝京危急,諸侯無釋位之志,征鎮闕勤王之舉,故君臣窘迫,以至殺辱云。



 史臣曰:昔炎暉杪暮,英雄多假於宗室。金德韜華,顛沛共推於懷愍。樊陽寂寥,兵車靡會,豈力不足而情有餘乎?喋喋遺萌,茍存其主,譬彼詩人,愛其棠樹。夫有非常之事,而無非常之功,詳觀發迹,用非天啟,是以輿棺齒劍,可得而言焉。于時五嶽三塗,並皆淪寇,龍州、牛首,故以立君。股肱非挑戰之秋,劉石有滔天之勢,療饑中斷,嬰戈外絕,兩京淪狄,再駕徂戎。周王隕首於驪峰,衛公亡肝於淇上,思為一郡,其可得乎!干寶有言曰:



 昔高祖
 宣皇帝以雄才碩量,應時而仕,值魏太祖創基之初,籌畫軍國,嘉謀屢中,遂服輿軫,驅馳三世。性深阻有若城府,而能寬綽以容納;行任數以御物,而知人善采拔。故賢愚咸懷,大小畢力。爾乃取鄧艾於農隙,引州泰於行役,委以文武,各善其事。故能西禽孟達,東舉公孫,內夷曹爽,外襲王凌。神略獨斷,征伐四克,維御群后,大權在己。於是百姓與能,大象始構。世宗承基,太祖繼業,玄豐亂內,欽誕寇外,潛謀雖密,而在機必兆;淮浦再擾,而許洛不震:咸黜異圖,用融前烈。然後推轂鐘鄧,長驅庸蜀,三關電埽,而劉禪入臣,天符人事,於是信矣。始當非常之
 禮,終受備物之錫。至于世祖,遂享皇極。仁以厚下,儉以足用,和而不馳,寬而能斷,故民詠維新,四海悅勸矣。聿修祖宗之志,思輯戰國之苦。腹心不同,公卿異議,而獨納羊祜之策,杖王杜之決,役不二時,江湘來同。掩唐虞之舊域,班正朔於八荒,天下書同文,車同軌,牛馬被野,餘糧委畝,故于時有「天下無窮人」之諺。雖太平未洽,亦足以明吏奉其法,民樂其生矣。武皇既崩,山陵未乾,而楊駿被誅,母后廢黜。尋以二公、楚王之變,宗子無維城之助,師尹無具瞻之貴,至乃易天子以太上之號,而有免官之謠。民不見德,惟亂是聞,朝為伊周,夕成桀蹠,善
 惡陷於成敗,毀譽脅於世利,內外混淆,庶官失才,名實反錯,天綱解紐。國政迭移於亂人,禁兵外散於四方,方岳無鈞石之鎮,關門無結草之固。李辰、石冰傾之於荊楊,元海、王彌撓之於青冀,戎羯稱制,二帝失尊,何哉?樹立失權,託付非才,四維不張,而茍且之政多也。



 夫作法於治,其弊猶亂;作法於亂,誰能救之!彼元海者,離石之將兵都尉;王彌者,青州之散吏也。蓋皆弓馬之士,驅走之人,非有吳先主、諸葛孔明之能也;新起之寇,烏合之眾,非吳蜀之敵也;脫耒為兵,裂裳為旗,非戰國之器也;自下逆上,非鄰國之勢也。然而擾天下如驅群羊,舉二
 都如拾遺芥,將相王侯連頸以受戮,后嬪妃主虜辱於戎卒,豈不哀哉!天下,大器也;群生,重畜也。愛惡相攻,利害相奪,其勢常也。若積水於防,燎火于原,未嘗暫靜也。器大者,不可以小道治;勢重者,不可以爭競擾。古先哲王知其然也,是以扞其大患,禦其大災。百姓皆知上德之生己,而不謂浚己以生也,是以感而應之,悅而歸之,如晨風之鬱北林,龍魚之趣藪澤也。然後設禮文以理之,斷刑罰以威之,謹好惡以示之,審禍福以喻之,求明察以官之,尊慈愛以固之。故眾知向方,皆樂其生而哀其死,悅其教而安其俗;君子勤禮,小人盡力,廉恥篤於
 家閭,邪闢消於胸懷。故其民有見危以授命,而不求生以害義,又況可奮臂大呼,聚之以干紀作亂乎!基廣則難傾,根深則難拔,理節則不亂,膠結則不遷,是以昔之有天下者之所以長久也。夫豈無僻主,賴道德典刑以維持之也。



 昔周之興也,后稷生於姜嫄,而天命昭顯,文武之功起於后稷。至於公劉,遭夏人之亂,去邰之豳,身服厥勞。至於太王,為戎翟所逼,而不忍百姓之命,杖策而去之。故從之如歸市,一年成邑,二年成都,三年五倍其初。至于王季,能貊其德音;至于文王,而維新其命。由此觀之,周家世積忠厚,仁及草木,內隆九族,外尊事黃
 耇,以成其福祿者也。而其妃后躬行四教,尊敬師傅,服瀚濯之衣,修煩辱之事,化天下以成婦道。是以漢濱之女,守潔白之志,中林之士,有純一之德,始於憂勤,終於逸樂。以三聖之知,伐獨夫之紂,猶正其名教,曰逆取順守。及周公遭變,陳后稷先公風化之所由,致王業之艱難者,則皆農夫女工衣食之事也。故自后稷之始基靖民,十五王而文始平之,十六王而武始居之,十八王而康克安之。故其積基樹本,經緯禮俗,節理人情,恤隱民事,如此之纏綿也。今晉之興也,功烈於百王,事捷於三代。宣景遭多難之時,誅庶孽以便事,不及修公劉、太王
 之仁也。受遺輔政,屢遇廢置,故齊王不明,不獲思庸於亳;高貴沖人,不得復子明辟也。二祖逼禪代之期,不暇待參分八百之會也。是其創基立本,異於先代者也。加以朝寡純德之人,鄉乏不貳之老,風俗淫僻,恥尚失所,學者以老莊為宗而黜《六經》,談者以虛蕩為辨而賤名檢,行身者以放濁為通而狹節信,進仕者以茍得為貴而鄙居正,當官者以望空為高而笑勤恪。是以劉頌屢言治道,傅咸每糾邪正,皆謂之俗吏;其倚杖虛曠,依阿無心者皆名重海內。若夫文王日旰不暇食,仲山甫夙夜匪懈者,蓋共嗤黜以為灰塵矣。由是毀譽亂於善惡
 之實,情慝奔於貨欲之塗。選者為人擇官,官者為身擇利,而執鈞當軸之士,身兼官以十數。大極其尊,小錄其耍,而世族貴戚之子弟,陵邁超越,不拘資次。悠悠風塵,皆奔競之士,列官千百,無讓賢之舉。子真著崇讓而莫之省,子雅制九班而不得用。其婦女,莊櫛織紝皆取成於婢僕,未嘗知女工絲枲之業,中饋酒食之事也。先時而婚,任情而動,故皆不恥淫泆之過,不拘妒忌之惡,父兄不之罪也,天下莫之非也,又況責之聞四教於古,修貞順於今,以輔佐君子者哉!禮法刑政於此大壞,如水斯積而決其提防,如火斯畜而離其薪燎也。國之將亡,
 未必先顛,其此之謂乎!故觀阮籍之行,而覺禮教崩馳之所由也。察庾純、賈充之爭,而見師尹之多僻;考平吳之功,而知將帥之不讓;思郭欽之謀,而寤戎狄之有釁;覽傅玄、劉毅之言,而得百官之邪;核傅咸之奏、《錢神》之論,而睹寵賂之彰。民風國勢如此,雖以中庸之主治之,辛有必見之於祭祀,季札必得之於聲樂,范燮必為之請死,賈誼必為之痛哭,又況我惠帝以放蕩之德臨之哉!懷帝承亂得位,羈於強臣,愍帝奔播之後,徒廁其虛名,天下之政既去,非命世之雄才,不能取之矣!淳耀之烈未渝,故大命重集于中宗皇帝。



 贊曰:懷佩玉璽,愍居黃屋。鰲墜三山,鯨吞九服,獯入金商,穹居未央。圜顱盡僕,方趾咸殭。大夫反首,徙我平陽。主憂臣哭,於何不臧!



\end{pinyinscope}