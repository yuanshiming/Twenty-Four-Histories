\article{列傳第十三 山濤子簡簡子遐王戎從弟衍衍弟澄郭舒樂廣}

\begin{pinyinscope}
山濤
 \gezhu{
  子簡簡子遐}
 王戎
 \gezhu{
  從弟衍衍弟澄}
 郭舒樂廣



 山濤,字巨源,河內懷人也。父曜,宛句令。濤早孤,居貧,少有器量,介然不群。性好《莊》《老》,每隱身自晦。與嵇康、呂安善,後遇阮籍,便為竹林之交,著忘言之契。康後坐事,臨誅,謂子紹曰:「巨源在,汝不孤矣。」



 濤年四十,始為郡主簿、功曹、上計掾。舉孝廉,州辟部河南從事。與石鑒共宿,濤夜起蹴鑒曰:「今為何等時而眠邪!知太傅臥何意?」鑒曰:「
 宰相三不朝,與尺一令歸第,卿何慮也!」濤曰:「咄!石生無事馬蹄間邪!」投傳而去。未二年,果有曹爽之事,遂隱身不交世務。



 與宣穆后有中表親,是以見景帝。帝曰:「呂望欲仕邪?」命司隸舉秀才,除郎中。轉驃騎將軍王昶從事中郎。久之,拜趙國相,遷尚書吏部郎。文帝與濤書曰:「足下在事清明,雅操邁時。念多所乏,今致錢二十萬、穀二百斛。」魏帝嘗賜景帝春服,帝以賜濤。又以母老,並賜藜杖一枚。



 晚與尚書和逌交,又與鐘會、裴秀並申款暱。以二人居勢爭權,濤平心處中,各得其所,而俱無恨焉。遷大將軍從事中郎。鐘會作亂於蜀,而文帝將西征。時魏
 氏諸王公並在鄴,帝謂濤曰:「西偏吾自了之,後事深以委卿。」以本官行軍司馬,給親兵五百人,鎮鄴。



 咸熙初,封新沓子。轉相國左長史,典統別營。時帝以濤鄉閭宿望,命太子拜之。帝以齊王攸繼景帝後,素又重攸,嘗問裴秀曰:「大將軍開建未遂,吾但承奉後事耳。故立攸,將歸功於兄,何如?」秀以為不可,又以問濤。濤對曰:「廢長立少,違禮不祥。國之安危,恒必由之。」太子位於是乃定。太子親拜謝濤。及武帝受禪,以濤守大鴻臚,護送陳留王詣鄴。泰始初,加奉車都尉,進爵新沓伯。



 及羊祜執政,時人欲危裴秀,濤正色保持之。由是失權臣意,出為冀州刺
 史,加寧遠將軍。冀州俗薄,無相推轂。濤甄拔隱屈,搜訪賢才,旌命三十餘人,皆顯名當時。人懷慕尚,風俗頗革。轉北中郎將,督鄴城守事。入為侍中,遷尚書。以母老辭職,詔曰:「君雖乃心在於色養,然職有上下,旦夕不廢醫藥,且當割情,以隆在公。」濤心求退,表疏數十上,久乃見聽。除議郎,帝以濤清儉無以供養,特給日契,加賜床帳茵褥。禮秩崇重,時莫為比。



 後除太常卿,以疾不就。會遭母喪,歸鄉里。濤年踰耳順,居喪過禮,負土成墳,手植松柏。詔曰:「吾所共致化者,官人之職是也。方今風欲陵遲,人心進動,宜崇明好惡,鎮以退讓。山太常雖尚居諒闇,
 情在難奪,方今務殷,何得遂其志邪!其以濤為吏部尚書。」濤辭以喪病,章表懇切。會元皇后崩,遂扶興還洛。逼迫詔命,自力就職。前後選舉,周遍內外,而並得其才。



 咸寧初,轉太子少傅,加散騎常侍;除尚書僕射,加侍中,領吏部。固辭以老疾,上表陳情。章表數十上,久不攝職,為左丞白褒所奏。帝曰:「濤以病自聞,但不聽之耳。使濤坐執銓衡則可,何必上下邪!不得有所問。」濤不自安,表謝曰:「古之王道,正直而已。陛下不可以一老臣為加曲私,臣亦何必屢陳日月。乞如所表,以章典刑。」帝再手詔曰:「白褒奏君甚妄,所以不即推,直不喜凶赫耳。君之明度,
 豈當介意邪!便當攝職,令斷章表也。」濤志必欲退,因發從弟婦喪,輒還外舍。詔曰:「山僕射近日暫出,遂以微苦未還,豈吾側席之意。其遣丞掾奉詔諭旨,若體力故未平康者,便以輿車輿還寺舍。」濤辭不獲已,乃起視事。



 濤再居選職十有餘年,每一官缺,輒啟擬數人,詔旨有所向,然後顯奏,隨帝意所欲為先。故帝之所用,或非舉首,眾情不察,以濤輕重任意。或譖之於帝,故帝手詔戒濤曰:「夫用人惟才,不遺疏遠單賤,天下便化矣。」而濤行之自若,一年之後眾情乃寢。濤所奏甄拔人物,各為題目,時稱《山公啟事》。



 濤中立於朝,晚值后黨專權,不欲任楊
 氏,多有諷諫,帝雖悟而不能改。後以年衰疾篤,上疏告退曰:「臣年垂八十,救命旦夕,若有毫末之益,豈遺力於聖時,迫以老耄,不復任事。今四海休息,天下思化,從而靜之,百姓自正。但當崇風尚教以敦之耳,陛下亦復何事。臣耳目聾瞑,不能自勵。君臣父子,其間無文,是以直陳愚情,乞聽所請。」乃免冠徒跣,上還印綬。詔曰:「天下事廣,加吳土初平,凡百草創,當共盡意化之。君不深識往心而以小疾求退,豈所望於君邪!朕猶側席,未得垂拱,君亦何得高尚其事乎!當崇至公,勿復為虛飾之煩。」濤苦表請退,詔又不許。尚書令衛瓘奏:「濤以微苦,久不視
 職。手詔頻煩,猶未順旨。參議以為無專節之尚,違在公之義。若實沈篤,亦不宜居位。可免濤官。」中詔瓘曰:「濤以德素為朝之望,而常深退讓,至於懇切。故比有詔,欲必奪其志,以匡輔不逮。主者既不思明詔旨,而反深加詆案。虧崇賢之風,以重吾不德,何以示遠近邪!」濤不得已,又起視事。



 太康初,遷右僕射,加光祿大夫,侍中、掌選如故。濤以老疾固辭,手詔曰:「君以道德為世模表,況自先帝識君遠意。吾將倚君以穆風俗,何乃欲舍遠朝政,獨高其志耶!吾之至懷故不足以喻乎,何來言至懇切也。且當以時自力,深副至望。君不降志,朕不安席。」濤又上
 表固讓,不許。



 吳平之後,帝詔天下罷軍役,示海內大安,州郡悉去兵,大郡置武吏百人,小郡五十人。帝嘗講武於宣武場,濤時有疾,詔乘步輦從。因與盧欽論用兵之本,以為不宜去州郡武備,其論甚精。于時咸以濤不學孫、吳,而闇與之合。帝稱之曰:「天下名言也。」而不能用。及永寧之後,屢有變難,寇賊猋起,郡國皆以無備不能制,天下遂以大亂,如濤言焉。



 後拜司徒,濤復固讓。」詔曰:「郡年耆德茂,朝之碩老,是以授君台輔之位。而遠崇克讓,至於反覆,良用於邑。君當終始朝政,翼輔朕躬。」濤又表曰:「臣事天朝三十餘年,卒無毫釐以崇大化。陛下私臣
 無已,猥授三司。臣聞德薄位高,力少任重,上有折足之凶,下有廟門之咎,願陛下垂累世之恩,乞臣骸骨。詔曰:「君翼贊朝政,保乂皇家,匡佐之勳,朕所倚賴。司徒之職,實掌幫教,故用敬授,以答群望。豈宜沖讓以自抑損邪!」已敕斷章表,使者乃臥加章綬。濤曰:「垂沒之人,豈可污官府乎!」輿疾歸家。以太康四年薨,時年七十九,詔賜東園秘器、朝服一具、衣一襲、錢五十萬、布百匹,以供喪事,策贈司徒,蜜印紫綬,侍中貂蟬,新沓伯蜜印青朱綬,祭以太牢,謚曰康。將葬,賜錢四十萬、布百匹。左長史范晷等上言:「濤舊第屋十間,子孫不相容。」帝為之立室。



 初,濤
 布衣家貧,謂妻韓氏曰:「忍饑寒,我後當作三公,但不知卿堪公夫人不耳!」及居榮貴,貞慎儉約,雖爵同千乘,而無嬪媵。祿賜俸秩,散之親故。



 初,陳郡袁毅嘗為鬲令,貪濁而賂遺公卿,以求虛譽,亦遺濤絲百斤,濤不欲異於時,受而藏於閣上。後毅事露,檻車送廷尉,凡所以賂,皆見推檢。濤乃取絲付吏,積年塵埃,印封如初。



 濤飲酒至八斗方醉,帝欲試之,乃以酒八鬥飲濤,而密益其酒,濤極本量而止。有五子:該、淳、允、謨、簡。



 該字伯倫,嗣父爵,仕至并州刺史、太子左率,贈長水校尉。該子瑋字彥祖,翊軍校尉。次子世回,吏部郎、散騎常
 侍。淳字子玄,不仕,允字叔真,奉車都尉,並少尪病,形甚短小,而聰敏過人。武帝聞而欲見之,濤不敢辭,以問於允。允自以尪陋,不肯行。濤以為勝己,乃表曰:「臣二子尪病,宜絕人事,不敢受詔。」謨字季長,明惠有才智,官至司空掾。



 簡字季倫。性溫雅,有父風,年二十餘,濤不之知也。簡歎曰:「吾年幾三十,而不為家公所知!」後與譙國嵇紹、水市郡劉謨、弘農楊準齊名。初為太子舍人,累遷太子庶子、黃門郎,出為青州刺史。徵拜侍中,頃之,轉尚書。歷鎮軍將軍、荊州刺史,領南蠻校尉,不行,復拜尚書。光熙初,轉吏
 部尚書。永嘉初,出為雍州刺史、鎮西將軍。徵為尚書左僕射,領吏部。



 簡欲令朝臣各舉所知,以廣得才之路。上疏曰:「臣以為自古興替,實在官人;茍得其才,則無物不理。《書》言:『知人則哲,惟帝難之。』唐、虞之盛,元愷登庸;周室之隆,濟濟多士。秦、漢已來,風雅漸喪。至於後漢,女君臨朝,尊官大位,出於阿保,斯亂之始也。是以郭泰、許劭之倫,明清議於草野;陳蕃、李固之徒,守忠節於朝廷。然後君臣名節,古今遺典,可得而言。自初平之元,訖於建安之末,三十年中,萬姓流散,死亡略盡,斯亂之極也。世祖武皇帝應天順人,受禪于魏,泰始之初,躬親萬機,佐命之
 臣,咸皆率職。時黃門侍郎王恂、庾純始於太極東堂聽政,評尚書奏事,多論刑獄,不論選舉。臣以為不先所難,而辨其所易。陛下初臨萬國,人思盡誠,每於聽政之日,命公卿大臣先議選舉,各言所見後進俊才、鄉邑尤異、才堪任用者,皆以名奏,主者隨缺先敘。是爵人於朝,與眾共之之義也。」朝廷從之。



 永嘉三年,出為征南將軍、都督荊、湘、交、廣四州諸軍事、假節,鎮襄陽。于時四方寇亂,天下分崩,王威不振,朝野危懼。簡優游卒歲,唯酒是耽。諸習氏,荊土豪族,有佳園池,簡每出嬉遊,多之池上,置酒輒醉,名之曰高陽池。時有童兒歌曰:「山公出何許,往
 至高陽池。日夕倒載歸,酩酊無所知。時時能騎馬,倒著白接䍦。舉鞭問葛疆:何如並州兒?」疆家在並州,簡愛將也。



 尋加督寧、益軍事。時劉聰入寇,京師危逼。簡遣督護王萬率師赴難,次于涅陽,為宛城賊王如所破,遂嬰城自守。及洛陽陷沒,簡又為賊嚴嶷所逼,乃遷于夏口。招納流亡,江、漢歸附。時華軼以江州作難,或勸簡討之。簡曰:「與彥夏舊友,為之惆悵。簡豈利人之機,以為功伐乎!」其篤厚如此。時樂府伶人避難,多奔沔漢,宴會之日,僚佐或勸奏之。簡曰:「社稷傾覆,不能匡救,有晉之罪人也,何作樂之有!」因流涕慷慨,坐者咸愧焉。年六十卒,追贈
 征南大將軍、儀同三司。子遐。



 遐字彥林,為餘姚令。時江左初基,法禁寬弛,豪族多挾藏戶口,以為私附。遐繩以峻法,到縣八旬,出口萬餘。縣人虞喜以藏戶當棄市,遐欲繩喜。諸豪彊莫不切齒於遐,言於執事,以喜有高節,不宜屈辱。又以遐輒造縣舍,遂陷其罪。遐與會稽內史何充箋:「乞留百日,窮翦捕逃,退而就罪,無恨也。」充申理,不能得。竟坐免官。後為東陽太守,為政嚴猛。康帝詔曰:「東陽頃來竟囚,每多入重。豈郡多罪人,將捶楚所求,莫能自固邪!」遐處之自若,郡境肅然。卒於官。



 史臣曰:若夫居官以潔其務,欲以啟天下之方,事親以終其身,將以勸天下之俗,非山公之具美,其孰能與於此者哉!自東京喪亂,吏曹湮滅,西園有三公之錢,蒲陶有一州之任,貪饕方駕,寺署斯滿。時移三代,世歷九王,拜謝私庭,此焉成俗。若乃餘風稍殄,理或可言。委以銓綜,則群情自抑;通乎魚水,則專用生疑。將矯前失,歸諸後正,惠絕臣名,恩馳天口,世稱《山公啟事》者,豈斯之謂歟!若盧子家之前代,何足算也。



 王戎,字濬沖,瑯邪臨沂人也。祖雄,幽州刺史。父渾,涼州
 刺史、貞陵亭侯。戎幼而穎悟,神彩秀徹。視日不眩,裴楷見而目之曰:「戎眼燦燦,如巖下電。」年六七歲,於宣武場觀戲,猛獸在檻中虓吼震地,眾皆奔走,戎獨立不動,神色自若。魏明帝於閣上見而奇之。又嘗與群兒嬉於道側,見李樹多實,等輩兢趣之,戎獨不往。或問其故,其曰:「樹在道邊而多子,必苦李也。」取之信然。



 阮籍與渾為友。戎年十五,隨渾在郎舍。戎少籍二十歲,而籍與之交。籍每適渾,俄頃輒去,過視戎,良久然後出。謂渾曰:「濬沖清賞,非卿倫也。共卿言,不如共阿戎談。」及渾卒於涼州,故吏賻贈數百萬,戎辭而不受,由是顯名。為人短小,任率
 不修威儀,善發談端,賞其要會。朝賢嘗上巳示契洛,或問王濟曰:「昨游有何言談?」濟曰:「張華善說《史》《漢》;裴頠論前言往行,袞袞可聽;王戎談子房、季札之間,超然玄著。」其為識鑒者所賞如此。



 戎嘗與阮籍飲,時兗州刺史劉昶字公榮在坐,籍以酒少,酌不及昶,昶無恨色。戎異之,他日問籍曰:「彼何如人也?」答曰:「勝公榮,不可不與飲;若減公榮,則不敢不共飲;惟公榮可不與飲。」戎每與籍為竹林之游,戎嘗後至。籍曰:「俗物已復來敗人意。」戎笑曰:「卿輩意亦復易敗耳!



 鐘會伐蜀,過與戎別,問計將安出。戎曰:「道家有言,『為而不恃』,非成功難,保之難也。」及會敗,議
 者以為知言。



 襲父爵,辟相國掾,歷吏部黃門郎、散騎常侍、河東太守、荊州刺史,坐遣吏修園宅,應免官,詔以贖論。遷豫州刺史,加建威將軍,受詔伐吳。戎遣參軍羅尚、劉喬領前鋒,進攻武昌,吳將楊雍、孫述、江夏太守劉朗各率眾詣戎降。戎督大軍臨江,吳牙門將孟泰以蘄春、邾二縣降。吳平,進爵安豐侯,增邑六千戶,賜絹六千匹。



 戎渡江,綏慰新附,宣揚威惠。吳光祿勛石偉方直,不容皓朝,稱疾歸家。戎嘉其清節,表薦之。詔拜偉為議郎,以二千石祿終其身。荊土悅服。徵為侍中。南郡太守劉肇賂戎筒中細布五十端,為司隸所糾,以知而未納,故
 得不坐,然議者尤之。帝謂朝臣曰:「戎之為行,豈懷私茍得,正當不欲為異耳!」帝雖以是言釋之,然為清慎者所鄙,由是損名。



 戎在職雖無殊能,而庶績脩理。後遷光祿勛、吏部尚書,以母憂去職。性至孝,不拘禮制,飲酒食肉,或觀弈棋,而容貌毀悴,杖然後起。裴頠往弔之,謂人曰:「若使一慟能傷人,濬沖不免滅性之譏也。」時和嶠亦居父喪,以禮法自持,量米而食,哀毀不踰於戎。帝謂劉毅曰:「和嶠毀頓過禮,使人憂之。」毅曰:「嶠雖寢苫食粥,乃生孝耳。至於王戎,所謂死孝,陛下當先憂之。」戎先有吐疾,居喪增甚。帝遣醫療之,并賜藥物,又斷賓客。



 楊駿執政,
 拜太子太傅。駿誅之後,東安公繇專斷刑賞,威震外內。戎誡繇曰:「大事之後,宜深遠之。」繇不從,果得罪。轉中書令,加光祿大夫,給恩信五十人。遷尚書左僕射,領吏部。



 戎始為甲午制,凡選舉皆先治百姓,然後授用。司隸傅咸奏戎,曰:「《書》稱『三載考績,三考黜陟幽明』。今內外群官,居職未期而戎奏還,既未定其優劣,且送故迎新,相望道路,巧詐由生,傷農害政。戎不仰依堯舜典謨,而驅動浮華,虧敗風俗,非徒無益,乃有大損。宜免戎官,以敦風俗。」戎與賈、郭通親,竟得不坐。尋轉司徒。以王政將圮,茍媚取容,屬愍懷太子之廢,竟無一言匡諫。



 裴頠,戎之婿
 也,頠誅,戎坐免官。齊王冏起義,孫秀祿戎於城內,趙王倫子欲取戎為軍司。博士王繇曰:「濬沖譎詐多端,安肯為少年用?」乃止。惠帝反宮,以戎為尚書令。既而河間王顒遣使就說成都王穎,將誅齊王冏。檄書至,冏謂戎曰:「孫秀作逆,天子幽逼。孤糾合義兵,掃除元惡,臣子之節,信著神明。二王聽讒,造構大難,當賴忠謀,以和不協。卿其善為我籌之。」戎曰:「公首舉義眾,匡定大業,開闢以來,未始有也。然論功報嘗,不及有勞,朝野失望,人懷貳志。今二王帶甲百萬,其鋒不可當,若以王就第,不失故爵。委權崇讓,此求安之計也。」冏謀臣葛旟怒曰:「漢魏以來,
 王公就第,寧有得保妻子乎!議者可斬。」於是百官震悚,戎偽藥發墮廁,得不及禍。



 戎以晉室方亂,慕蘧伯玉之為人,與時舒卷,無蹇諤之節。自經典選,未嘗進寒素,退虛名,但與時浮沈,戶調門選而已。尋拜司徒,雖位總鼎司,而委事僚採。間乘小馬,從便門而出游,見者不知其三公也。故吏多至大官,道路相遇輒避之。性好興利,廣收八方園田水碓,周遍天下。積實聚錢,不知紀極,每自執牙籌,晝夜算計,恒若不足。而又儉嗇,不自奉養,天下人謂之膏肓之疾。女適裴頠,貸錢數萬,久而未還。女後歸寧,戎色不悅,女遽還直,然後乃懽。從子將婚,戎遣其
 一單衣,婚訖而更責取。家有好李,常出貨之,恐人得種,恆鑽其核。以此獲譏於世。



 其後從帝北伐,王師敗績於蕩陰,戎復詣鄴,隨帝還洛陽。車駕之西遷也,戎出奔於郟。在危難之間,親接鋒刃,談笑自若,未嘗有懼容。時召親賓,歡娛永日。永興二年,薨于郟縣,時年七十二,謚曰元。



 戎有人倫鑒識,嘗目山濤如璞玉渾金,人皆欽其寶,莫知名其器;王衍神姿高徹,如瑤林瓊樹,自然是風塵表物。謂裴頠拙於用長,荀勖工於用短,陳道寧糸畟糸畟如束長竿。族弟敦有高名,戎惡之。敦每候戎,輒託疾不見。敦後果為逆亂。其鑒嘗先見如此。嘗經黃公酒壚下過,
 顧謂後車客曰:「吾昔與嵇叔夜、阮嗣宗酣暢於此,竹林之游亦預其末。自嵇、阮云亡,吾便為時之所羈紲。今日視之雖近,邈若山河!」初,孫秀為瑯邪郡吏,求品於鄉議。戎從弟衍將不許,戎勸品之。及秀得志,朝士有宿怨者皆被誅,而戎、衍獲濟焉。



 子萬,有美名。少而大肥,戎令食穅,而肥愈甚。年十九卒。有庶子興,戎所不齒。以從弟陽平太守愔子為嗣。



 衍字夷甫,神情明秀,風姿詳雅。總角嘗造山濤,濤嗟歎良久,既去,目而送之曰:「何物老嫗,生寧馨兒!然誤天下蒼生者,未必非此人也。」父乂,為平北將軍,常有公事,使
 行人列上,不時報。衍年十四,時在京師,造僕射羊祜,申陳事狀,辭甚清辯。祜名德貴重,而衍幼年無屈下之色,眾咸異之。楊駿欲以女妻焉,衍恥之,遂陽狂自免。武帝聞其名,問戎曰:「夷甫當世誰比?」戎曰:「未見其比,當從古人中求之。」



 泰始八年,詔舉奇才可以安邊者,衍初好論從橫之術,故尚書盧欽舉為遼東太守。不就,於是口不論世事,唯雅詠玄虛而已。嘗因宴集,為族人所怒,舉累擲其面。衍初無言,引王導共載而去。然心不能平,在車中攬鏡自照,謂導曰:「爾看吾目光乃在牛背上矣。」父卒於北平,送故甚厚,為親識之所借貸,因以捨之。數年之
 間,家資罄盡,出就洛城西田園而居焉。後為太子舍人,還尚書郎。出補元城令,終日清談,而縣務亦理。入為中庶子、黃門侍郎。



 魏正始中,何晏、王弼等祖述《老》《莊》,立論以為:「天地萬物皆以無為本。無也者,開物成務,無往不存者也。陰陽恃以化生,萬物恃以成形,賢者恃以成德,不肖恃以免身。故無之為用,無爵而貴矣。」衍甚重之。惟裴頠以為非,著論以譏之,而衍處之自若。衍既有盛才美貌,明悟若神,常自比子貢。兼聲名藉甚,傾動當世。妙善玄言,唯談《老》《莊》為事。每捉玉柄麈尾,與手同色。義理有所不安,隨即改更,世號「口中雌黃。」朝野翕然,謂之「
 一世龍門」矣。累居顯職,後進之士,莫不景慕放效。選舉登朝,皆以為稱首。矜高浮誕,遂成風俗焉。衍嘗喪幼子,山簡弔之。衍悲不自勝,簡曰:「孩抱中物,何至於此!」衍曰:「聖人忘情,最下不及於情。然則情之所鐘,正在我輩。」簡服其言,更為之慟。



 衍妻郭氏,賈后之親,藉中宮之勢,剛愎貪戾,聚斂無厭,好干預人事,衍患之而不能禁。時有鄉人幽州刺史李陽,京師大俠也,郭氏素憚也。衍謂郭曰:「非但我言卿不可,李陽亦謂不可。」郭氏為之小損。衍疾郭之貪鄙,故口未嘗言錢。郭欲試之,令婢以錢繞床,使不得行。衍晨起見錢,謂婢曰:「舉阿堵物卻!」其措意如
 此。



 後歷北軍中候、中領軍、尚書令。女為愍懷太子妃,太子為賈后所誣,衍懼禍,自表離婚。賈后既廢,有司奏衍,曰:「衍與司徒梁王肜書,寫呈皇太子手與妃及衍書,陳見誣見狀。肜等伏讀,辭旨懇惻。衍備位大臣,應以議責也。太子被誣得罪,衍不能守死善道,即求離婚。得太子手書,隱蔽不出。志在茍免,無忠蹇之操。宜加顯責,以厲臣節。可禁錮終身。」從之。



 衍素輕趙王倫之為人。及倫篡位,衍陽狂斫婢以自免。及倫誅,拜河南尹,轉尚書,又為中書令。時齊王乂有匡復之功,而專權自恣,公卿皆為之拜,衍獨長揖焉。以病去官。成都王穎以衍為中軍師,
 累遷尚書僕射,領吏部,後拜尚書令、司空、司徒。衍雖居宰輔之重,不以經國為念,而思自全之計。說東海王越曰:「中國已亂,當賴方伯,宜得文武兼資以任之。」乃以弟澄為荊州,族弟敦為青州。因謂澄、敦曰:「荊州有江、漢之固,青州有負海之險,卿二人在外,而吾留此,足以為三窟矣。」識者鄙之。



 及石勒、王彌寇京師,以衍都督征討諸軍事、持節、假黃鉞以距之。衍使前將軍曹武、左衛將軍王景等擊賊,退之,獲其輜重。遷太尉,尚書令如故。封武陵侯,辭封不受。時洛陽危逼,多欲遷都以避其難,而衍獨賣車牛以安眾心。



 越之討茍晞也,衍以太尉為太傅
 軍司。及越薨,眾共推為元帥。衍以賊寇鋒起,懼不敢當。辭曰:「吾少無宦情,隨牒推移,遂至於此。今日之事,安可以非才處之。」俄而舉軍為石勒所破,勒呼王公,與之相見,問衍以晉故。衍為陳禍敗之由,云計不在己。勒甚悅之,與語移日。衍自說少不豫事,欲求自免,因勸勒稱尊號。勒怒曰:「君名蓋四海,身居重任,少壯登朝,至於白首,何得言不豫世事邪!破壞天下,正是君罪。」使左右扶出。謂其黨孔萇曰:「吾行天下多矣,未嘗見如此人,當可活不?」萇曰:「彼晉之三公,必不為我盡力,又何足貴乎!」勒曰:「要不可加以鋒刃也。」使人夜排牆填殺之。衍將死,顧而
 言曰:「嗚呼!吾曹雖不如古人,向若不祖尚浮虛,戮力以匡天下,猶可不至今日。」時年五十六。



 衍俊秀有令望,希心玄遠,未嘗語利。王敦過江,常稱之曰:「夷甫處眾中,如珠玉在瓦石間。」顧愷之作畫贊,亦稱衍巖巖清峙,壁立千仞。其為人所尚如此。



 子玄,字眉子,少慕簡曠,亦有俊才,與衛玠齊名。荀籓用為陳留太守,屯尉氏。玄素名家,有豪氣,荒弊之時,人情不附,將赴祖逖,為盜所害焉。



 澄字平子。生而警悟,雖未能言,見人舉動,便識其意。衍妻郭性貪鄙,欲令婢路上擔糞。澄年十四,諫郭以為不可。郭大怒,謂澄曰:「昔夫人臨終,以小郎屬新婦,不以新
 婦屬小郎。」因捉其衣裾,將杖之。澄爭得脫,踰窗而走。



 衍有重名於世,時人許以人倫之鑒。尤重澄及王敦、庾敳,嘗為天下人士目曰:「阿平第一,子嵩第二,處仲第三。」澄嘗謂衍曰:「兄形似道,而神鋒太俊。」衍曰:「誠不如卿落落穆穆然也。」澄由是顯名。有經澄所題目者,衍不復有言,輒云「已經平子矣」。



 少歷顯位,累遷成都王穎從事中郎。穎嬖豎孟玖譖殺陸機兄弟,天下切齒。澄發玖私姦,勸穎殺玖,穎乃誅之,士庶莫不稱善。及穎敗,東海王越請為司空長史。以迎大駕勳,封南鄉侯。遷建威將軍、雍州刺史,不之職。時王敦、謝鯤、庾敳、阮脩皆為衍所親善,號
 為四友,而亦與澄狎,又有光逸、胡毋輔之等亦豫焉。酣宴縱誕,窮懽極娛。



 惠帝末,衍白越以澄為荊州刺史、持節、都督,領南蠻校尉,敦為青州。衍因問以方略,敦曰:「當臨事制變,不可豫論。」澄辭義鋒出,算略無方,一坐嗟服。澄將之鎮,送者傾朝。澄見樹上鵲巢,便脫衣上樹,探而弄之,神氣蕭然,傍若無人。劉琨謂澄曰:「卿形雖散朗,而內實動俠,以此處世,難得其死。」澄默然不答。



 澄既至鎮,日夜縱酒,不親庶事,雖寇戎急務,亦不以在懷。擢順陽人郭舒於寒悴之中,以為別駕,委以州府。時京師危逼,澄率眾軍,將赴國難,而飄風折其節柱。會王如寇襄
 陽,澄前鋒至宜城,遣使詣山簡,為如黨嚴嶷所獲。嶷偽使人從襄陽來而問之曰:「襄陽拔未?」答云:「昨旦破城,已獲山簡。」乃陰緩澄使,令得亡去。澄聞襄陽陷,以為信然,散眾而還。既而恥之,託糧運不贍,委罪長史蔣俊而斬之,竟不能進。巴蜀流人散在荊、湘者,與土人忿爭,遂殺縣令,屯聚樂鄉。澄使成都內史王機討之。賊請降,澄偽許之,既而襲之於寵洲,以其妻子為賞,沈八千餘人於江中。於是益、梁流人四五萬家一時俱反,推杜弢為主,,南破零桂,東掠武昌,敗王機于巴陵。澄亦無憂懼之意,但與機日夜縱酒,投壺博戲,數十局俱起。殺富人李才,
 取其家資以賜郭舒。南平太守應詹驟諫,不納。於是上下離心,內外怨叛。澄望實雖損,猶傲然自得。後出軍擊杜弢,次於作塘。山簡參軍王沖叛于豫州,自稱荊州刺史。澄懼,使杜蕤守江陵。澄遷于孱陵,尋奔沓中。郭舒諫曰:「使君臨州,雖無異政,未失眾心。今西收華容向義之兵,足以擒此小醜,奈何自棄。」澄不能從。



 初,澄命武陵諸郡同討杜弢,天門太守扈瑰次于益陽。武陵內史武察為其郡夷所害,瑰以孤軍引還。澄怒,以杜曾代瑰。夷袁遂,瑰故吏也,託為瑰報仇,遂舉兵逐曾,自稱平晉將軍。澄使司馬毌丘邈討之,為遂所敗。會元帝徵澄為軍諮
 祭酒,於是赴召。



 時王敦為江州,鎮豫章,澄過詣敦。澄夙有盛名,出於敦右,士庶莫不傾慕之。兼勇力絕人,素為敦所憚,澄猶以舊意侮敦。敦益忿怒,請澄入宿,陰欲殺之。而澄左右有二十絕人,持鐵馬鞭為衛,澄手嘗捉玉枕以自防,故敦未之得發。後敦賜澄左右酒,皆醉,借玉枕觀之。因下床而謂澄曰:「何與杜弢通信?」澄曰:「事自可驗。」敦欲入內,澄手引敦衣,至于絕帶。乃登於梁,因罵敦曰:「行事如此,殃將及焉。」敦令力士路戎搤殺之,時年四十四,載尸還其家。劉琨聞澄之死,歎曰:「澄自取之。」及敦平,澄故吏佐著作郎桓稚上表理澄,請加贈謚。詔復澄本
 官,謚曰憲。長子詹,早卒。次子徽,右軍司馬。



 郭舒,字稚行。幼請其母從師,歲餘便歸,粗識大義。鄉人少府范晷、宗人武陵太守郭景,咸稱舒當為後來之秀,終成國器。始為領軍校尉,坐擅放司馬彪,繫廷尉,世多義之。刺史夏侯含辟為西曹,轉主簿。含坐事,舒自繫理含,事得釋。刺史宗岱命為治中,喪母去職。劉弘牧荊州,引為治中。弘卒,舒率將士推弘子璠為主,討逆賊郭勱。滅之,保全一州。



 王澄聞其名,引為別駕。澄終日酣飲,不以眾務在意,舒常切諫之。及天下大亂,又勸澄修德養
 威,保完州境。澄以為亂自京都起,非復一州所能匡禦,雖不能從,然重其忠亮。荊土士人宗庾廞嘗因酒忤澄,澄怒,叱左右棒廞。舒厲色謂左右曰:「使君過醉,汝輩何敢妄動!」澄恚曰:「別駕狂邪,誑言我醉!」因遣掐其鼻,灸其眉頭,舒跪而受之。澄意少釋,而廞遂得免。



 澄之奔敗也,以舒領南郡。澄又欲將舒東下,舒曰:「舒為萬里紀綱,不能匡正,令使君奔亡,不忍渡江。」乃留屯沌口,採穭湖澤以自給。鄉人盜食舒牛,事覺,來謝。舒曰:「卿饑,所以食牛耳,餘肉可共啖之。」世以此服其弘量。



 舒少與杜曾厚,曾嘗召之,不往,曾銜之。至是,澄又轉舒為順陽太守,曾密遣
 兵襲舒,遁逃得免。



 王敦召為參軍,轉從事中郎。襄陽都督周訪卒,敦遣舒監襄陽軍。甘卓至,乃還。朝廷徵舒為右丞,敦留不遣。敦謀為逆,舒諫不從,使守武昌。荊州別駕宗澹忌舒才能,數譖之於王暠。暠疑舒與甘卓同謀,密以白敦,敦不受。高官督護繆坦嘗請武昌城西地為營,太守樂凱言於敦曰:「百姓久買此地,種菜自贍,不宜奪之。」敦大怒曰:「王處仲不來江湖,當有武昌地不,而人云是我地邪!」凱懼,不敢言。舒曰:「公聽舒一言。」敦曰:「平子以卿病狂,故掐鼻灸眉頭,舊疢復發邪!」舒曰:「古之狂也直,周昌、汲黯、朱雲不狂也。昔堯立誹謗之木,舜置敢諫
 之鼓,然後事無枉縱。公為勝堯、舜邪?乃逆折舒,使不得言。何與古人相遠!」敦曰:「卿欲何言?」舒曰:「繆坦可謂小人,疑誤視聽,奪人私地,以強陵弱。晏子稱:君曰其可,臣獻其否,以成其可。是以舒等不敢不言。」敦即使還地,眾咸壯之。敦重舒公亮,給賜轉豐,數詣其家。表為梁州刺史。病卒。



 樂廣,字彥輔,南陽淯陽人也。父方,參魏征西將軍夏侯玄軍事。廣時年八歲,玄常見廣在路,因呼與語,還謂方曰:「向見廣神姿郎徹,當為名士。卿家雖貧,可令專學,必
 能興卿門戶也。」方早卒。廣孤貧,僑居山陽,寒素為業,人無知者。性沖約,有遠識,寡嗜慾,與物無競。尤善談論,每以約言析理,以厭人之心,其所不知,默如也。裴楷嘗引廣共談,自夕申旦,雅相欽挹,歎曰:「我所不如也。」王戎為荊州刺史,聞廣為夏侯玄所嘗,乃舉為秀才。楷又薦廣於賈充,遂辟太尉掾,轉太子舍人。尚書令衛瓘,朝之耆舊,逮與魏正始中諸名士談論,見廣而奇之,曰:「自昔諸賢既沒,常恐微言將絕,而今乃復聞斯言於君矣。」命諸子造焉,曰:「此人之水鏡,見之瑩然,若披雲霧而睹青天也。」王衍自言:「與人語甚簡至,及見廣,便覺己之煩。」其為
 識者所歎美如此。



 出補元城令,遷中書侍郎,轉太子中庶子,累遷侍中、河南尹。廣善清言而不長於筆,將讓尹,請潘岳為表。岳曰:「當得君意。」廣乃作二百句語,述己之志。岳因取次比,便成名筆。時人咸云:「若廣不假岳之筆,岳不取廣之旨,無以成斯美也。」



 嘗有親客,久闊不復來,廣問其故,答曰:「前在坐,蒙賜酒,方欲飲,見盃中有蛇,意甚惡之,既飲而疾。」于時河南聽事壁上有角,漆畫作蛇,廣意盃中蛇即角影也。復置酒於前處,謂客曰:「酒中復有所見不?」答曰:「所見如初。」廣乃告其所以,客豁然意解,沈痾頓愈。衛玠總角時,嘗問廣夢,廣云是想。玠曰:「神形
 所不接而夢,豈是想邪!」廣曰:「因也。」玠思之經月不得,遂以成疾。廣聞故,命駕為剖析之,玠病即愈。廣歎曰:「此賢胸中當必無膏肓之疾!」



 廣所在為政,無當時功譽,然每去職,遺愛為人所思。凡所論人,必先稱其所長,則所短不言而自見矣。人有過,先盡弘恕,然後善惡自彰矣。廣與王衍俱宅心事外,名重於時。故天下言風流者,謂王、樂為稱首焉。



 少與弘農楊準相善。準之二子曰喬曰髦,皆知名於世。準使先詣裴頠,頠性弘方,愛喬有高韻。謂準曰:「喬當及卿,髦少減也。」又使詣廣,廣性清淳,愛髦有神檢。謂準曰:「喬自及卿,然髦亦清出。」準笑曰:「我二兒之
 優劣,乃裴、樂之優劣也。」論者以為喬雖有高韻,而神檢不足,樂為得之矣。



 是時王澄、胡毋輔之等,皆亦任放為達,或至裸體者。廣聞而笑曰:「名教內自有樂地,何必乃爾!」其居才愛物,動有理中,皆此類也。值世道多虞,朝章紊亂,清己中立,任誠保素而已。時人莫有見其際焉。



 先是河南官舍多妖怪,前尹多不敢處正寢,廣居之不疑。嘗外戶自閉,左右皆驚,廣獨自若。顧見牆有孔,使人掘牆,得狸而殺之,其怪亦絕。



 愍懷太子之廢也,詔故臣不得辭送,眾官不勝憤歎,皆冒禁拜辭。司隸校尉滿奮敕河南中部收縛拜者送獄,廣即便解遣。眾人代廣危懼。
 孫琰說賈謐曰:「前以太子罪惡,有斯廢黜,其臣不懼嚴詔,冒罪而送。今若繫之,是彰太子之善,不如釋去。」謐然其言,廣故得不坐。



 遷吏部尚書左僕射,後東安王繇當為僕射,轉廣為右僕射,領吏部,代王戎為尚書令,始戎薦廣,而終踐其位,時人美之。



 成都王穎,廣之婿也,及與長沙王乂遘難,而廣既處朝望,群小讒謗之。乂以問廣,廣神色不變,徐答曰:「廣豈以五男易一女。」乂猶以為疑,廣竟以憂卒。荀籓聞廣之不免也,為之流涕。三子:凱、肇、謨。



 凱字弘緒,大司馬齊王掾,參驃騎軍事。肇字弘茂,太傅東海王掾。洛陽陷,兄弟相攜南渡江。謨字弘範,征虜
 將軍、吳郡內史。



 史臣曰:漢相清靜,見機於曠務;周史清虛,不嫌於尸祿。豈台揆之任,有異於常班者歟!濬沖善發談端,夷甫仰希方外,登槐庭之顯列,顧漆圓而高視。彼既憑虛,朝章已亂。戎則取容於世,旁委貨財;衍則自保其身,寧論宗稷?及三方搆亂,六戎藉手,犬羊之侶,鋒鏑如雲。夷甫區區焉,佞彼兇渠,以求容貸,頹牆之隕,猶有禮也。平子肆情傲物,對鏡難堪,終失厥生,自貽伊敗。且夫衣服表容,珪璋範德,聲移宮羽,採照山華,布武有章,立言成訓。澄之箕踞,不已甚矣。若乃解衵登枝,裸形捫鵲,以此為達,
 謂之高致,輕薄是效,風流詎及。道睽將聖,事乖跰指,操情獨往,自夭其生者焉。昔晏嬰哭莊公之尸,樂令解愍懷之客,豈聞伯夷之風歟,懦夫能立志者也。



 贊曰:晉家求士,乃構仙臺,陵云切漢,山叟知材。濬沖居鼎,談優務劣。夷甫兩顧,退求三穴。神亂當年,忠乖曩列。平子陵侮,多於用拙。樂令披雲,高天澄徹。



\end{pinyinscope}