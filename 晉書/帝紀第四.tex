\article{帝紀第四}

\begin{pinyinscope}

 惠帝



 孝惠皇帝諱衷,字正度,武帝第二子也。泰始三年,立為皇太子,時年九歲。太熙元年四月己酉,武帝崩。是日,皇太子即皇帝位,大赦,改元為永熙。尊皇后楊氏曰皇太后,立妃賈氏為皇后。夏五月辛未,葬武皇帝於峻陽陵。丙子。增天下位一等,預喪事者二等,復租調一年,二千石已上皆封關中侯。以太尉楊駿為太傅,輔政。秋八月
 壬午,立廣陵王遹為皇太子,以中書監何劭為太子太師,吏部尚書王戎為太子太傅,衛將軍楊濟為太子太保。遣南中郎將石崇、射聲校尉胡奕、長水校尉趙俊、揚烈將軍趙歡將屯兵四出。冬十月辛酉,以司空石鑒為太尉,前鎮西將軍、隴西王泰為司空。



 永平元年春正月乙酉朔,臨朝,不設樂。詔曰:「朕夙遭不造,淹恤在疚。賴祖宗遺靈,宰輔忠賢,得以眇身託于群后之上,昧於大道,不明于訓,戰戰兢兢,夕惕若厲。乃者哀迷之際,三事股肱,惟社稷之重,率遵翼室之典,猶欲長奉先皇之制,是以有永熙之號。然日月踰邁,已涉新
 年,開元易紀,禮之舊章。其改永熙二年為永平元年。」又詔子弟及郡官並不得謁陵。丙午,皇太子冠,丁未,見於太廟。二月甲寅,賜王公已下帛各有差。癸酉,鎮南將軍楚王瑋、鎮東將軍淮南王允來朝。戊寅,復置秘書監官。三月辛卯,誅太傅楊駿,駿弟衛將軍珧,太子太保濟,中護軍張劭,散騎常侍段廣、楊邈。左將軍劉預,河同尹李斌,中書令符俊,東夷校尉文淑,尚書武茂,皆夷三族。壬辰,大赦,改元。賈后矯詔廢皇太后為庶人,徙於金墉城,告于天地宗廟。誅太后母龐氏。壬寅,征大司馬、汝南王亮為太宰,與太保衛瓘輔政。以秦王柬為大將軍,東平
 王楙為撫軍大將軍,鎮南將軍、楚王瑋為衛將軍,領北軍中候,下邳王晃為尚書令,東安公繇為尚書左僕射,進封東安王。督將侯者千八十一人。庚戌,免東安王繇及東平王楙,繇徙帶方。夏四月癸亥,以征東將軍、梁王肜為征西大將軍、都督關西諸軍事,太子少傅阮垣為平東將軍、監青徐二州諸軍事。己巳,以太子太傅王戎為尚書右僕射。五月甲戌,毗陵王軌薨。壬午,除天下戶調綿絹,賜孝悌、高年、鰥寡、力田者帛,人三匹。六月,賈后矯詔使楚王瑋殺太宰、汝南王亮,太保、菑陽公衛瓘。乙丑,以瑋擅害亮、瓘,殺之。曲赦洛陽。以廣陵王師劉寔為
 太子太保,司空、隴西王泰錄尚書事。秋七月,分揚州、荊州十郡為江州。八月庚申,以趙王倫為征東將軍、都督徐兗二州諸軍事;河間王顒為北中郎將,鎮鄴;太子太師何劭為都督豫州諸軍事,鎮許昌。徙長沙王乂為常山王。己巳,進西陽公羕爵為王。辛未,立隴西世子越為東海王。九月甲午,大將軍、秦王柬薨。辛丑,徵征西大將軍、梁王肜為衛將軍、錄尚書事,以趙王倫為征西大將軍、都督雍梁二州諸軍事。冬十二月辛酉,京師地震。是歲,東夷十七國、南夷二十四部並詣校尉內附。



 二年春二月己酉,賈后弒皇太后於金墉城。秋八月壬
 子,大赦。九月乙酉,中山王耽薨。冬十一月,大疫。是歲,沛國雨雹,傷麥。



 三年夏四月,滎陽雨雹。六月,弘農郡雨雹,深三尺。冬十月,太原王泓薨。



 四年春正月丁酉朔,侍中、太尉、安昌公石鑒薨。夏五月,蜀郡山移,淮南壽春洪水出,山崩地陷,壞城府及百姓廬舍。匈奴郝散反,攻上黨,殺長吏。六月,壽春地大震,死者二十餘家。上庸郡山崩,殺二十餘人。秋八月,郝散帥眾降,馮翊都尉殺之。上谷居庸、上庸並地陷裂,水泉涌出,人有死者。大饑。九月丙辰,赦諸州之遭地災者。甲午,
 枉矢東北竟天。是歲,京師及郡國八地震。



 五年夏四月,彗星見于西方,孛于奎,至軒轅。六月,金城地震。東海雨雹,深五寸。秋七月,下邳暴風,壞廬舍。九月,鴈門、新興、太原、上黨大風,傷禾稼。冬十月,武庫火,焚累代之寶。十二月丙戌,新作武庫,大調兵器。丹楊雨雹。有石生于京師宜年里。是歲,荊、揚、兗、豫、青、徐等六州大水,詔遣御史巡行振貸。



 六年春正月,大赦。司空、下邳王晃薨。以中書監張華為司空,大尉、隴西王泰為尚書令,衛將軍、梁王肜為太子太保。丁丑,地震。三月,東海隕霜,傷桑麥。彭城呂縣有流
 血,東西百餘步。夏四月,大風。五月,荊、揚二州大水。匈奴郝散弟度元帥馮翊、北地馬蘭羌、廬水胡反,攻北地,太守張損死之。馮翊太守歐陽建與度元戰,建敗績。徵征西大將軍、趙王倫為車騎將軍,以太子太保、梁王肜為征西大將軍、都督雍梁二州諸軍事,鎮關中。秋八月,雍州刺史解系又為度元所破。秦雍氐、羌悉叛,推氐帥齊萬年僭號稱帝,圍涇陽。冬十月乙未,曲赦雍、涼二州。十一月丙子,遣安西將軍夏侯俊、建威將軍周處等討萬年,梁王肜屯好畤。關中饑,大疫。



 七年春正月癸丑,周處及齊萬年戰於六陌,王師敗績,
 處死之。夏五月,魯國雨雹。秋七月,雍、梁州疫。大旱,隕霜,殺秋稼。關中饑,米斛萬錢。詔骨肉相賣者不禁。丁丑,司徒、京陵公王渾薨。九月,以尚書右僕射王戎為司徒,太子太師何劭為尚書左僕射。



 八年春正月丙辰,地震。詔發倉稟,振雍州饑人。三月壬戌,大赦。夏五月,郊禖石破為二。秋九月,荊、豫、揚、徐、冀等五州大水。雍州有年。



 九年春正月,左積弩將軍孟觀伐氐,戰于中亭,大破之,獲齊萬年。徵征西大將軍、梁王肜錄尚書事。以北中郎將、河間王顒為鎮西將軍,鎮關中;成都王潁為鎮北大
 將軍,鎮鄴。夏四月,鄴人張承基等妖言署置,聚黨數千。郡縣逮捕,皆伏誅。六月戊戌,太尉、隴西王泰薨。秋八月,以尚書裴頠為尚書僕射。冬十一月甲子朔,日有蝕之。京師大風,發屋折木。十二月壬戌,廢皇太子遹為庶人,及其三子幽于金墉城,殺太子母謝氏。



 永康元年春正月癸亥朔,大赦,改元。己卯,日有蝕之。丙子,皇孫霖卒。二月丁酉,大風,飛沙拔木。三月,尉氏雨血,妖星見於南方。癸未,賈后矯詔害庶人遹於許昌。夏四月辛卯,日有蝕之。癸巳,梁王肜、趙王倫矯詔廢賈后為庶人,司空張華、尚書僕射裴頠皆遇害,侍中賈謐及黨
 與數十人皆伏誅。甲午,倫矯詔大赦,自為相國、都督中外諸軍,如宣文輔魏故事,追復故皇太子位。丁酉,以梁王肜為太宰,左光祿大夫何劭為司徒,右光祿大夫劉寔為司空,淮南王允為驃騎將軍。己亥,趙王倫矯詔害賈庶人于金墉城。五月己巳,立皇孫臧為皇太孫,尚為襄陽王。六月壬寅,葬懷愍太子于顯平陵。撫軍將軍、清河王遐薨。癸卯,震崇陽陵標。秋八月,淮南王允舉兵討趙王倫,不剋,允及其二子秦王郁、漢王迪皆遇害。曲赦洛陽。平東將軍、彭城王植薨。改封吳王晏為賓徒縣王。以齊王冏為平東將軍,鎮許昌;光祿大夫陳準為太尉、錄
 尚書事。九月,改司徒為丞相,以梁王肜為之。冬十月,黃務四塞。十一月戊午,大風飛沙石,六日乃止。甲子,立皇后羊氏,大赦,大酺三日。十二月,彗星見于東方。益州刺史趙廞與洛陽流人李庠害成都內史耿勝、犍為太守李密、汶山太守霍固、西夷校尉陳總,據成都反。



 永寧元年春正月乙丑,趙王倫篡帝位。丙寅,遷帝於金墉城,號曰太上皇,改金墉曰永昌宮。廢皇太孫臧為濮陽王。五星經天,縱橫無常。癸酉,倫害濮陽王臧。洛陽流人李特殺趙廞,傳首京師。三月,平東將軍、齊王冏起兵以討倫,傳檄州郡,屯于陽翟。征北大將軍、成都王穎,征
 西大將軍、河間王顒,常山王乂,豫州刺史李毅,兗州刺史王彥,南中朗將、新野公歆,皆舉兵應之,眾數十萬。倫遣其將閭和出伊闕,張泓、孫輔出堮阪以距冏,孫會、士猗、許超出黃橋以距潁。及穎將趙驤、石超戰於溴水,會等大敗,棄軍走。閏月丙戌朔,日有蝕之。夏四月,歲星晝見。冏將何勖等擊張泓於陽翟,大破之,斬孫輔等。辛酉,左衛將軍王輿與尚書、淮陵王漼勒兵入宮,禽倫黨孫秀、孫會、許超、士猗、駱休等,皆斬之。逐倫歸第,即日乘輿反正。群臣頓首謝罪,帝曰:「非諸卿之過也。癸亥,詔曰:「朕以不德,纂承皇統,遠不能光濟大業,靖綏四方;近不能
 開明刑威,式遏奸宄,至使逆臣孫秀敢肆凶虐,窺間王室,遂奉趙王倫饕據天位。鎮東大將軍、齊王冏,征北大將軍、成都王穎,征西大將軍、河間王顒,並以明德茂親,忠規允著,首建大策,匡救國難。尚書漼共立大謀,左衛將軍王輿與群公卿士,協同謀略,親勒本營,斬秀及其二子。前趙王倫為秀所誤,與其子等已詣金墉迎朕幽宮,旋軫閶闔。豈在予一人獨饗其慶,宗廟社稷實有賴焉。」於是大赦,改元,孤寡賜穀五斛,大酺五日。誅趙王倫、義陽王威、九門侯質等及倫之黨與。五月,立襄陽王尚為皇太孫。六月戌辰,大赦,增吏位二等。復封賓徒王晏
 為吳王。庚午,東萊王蕤、左衛將軍王輿謀廢齊王冏,事泄,蕤廢為庶人,輿伏誅,夷三族。甲戌,以齊王冏為大司馬、都督中外諸軍事,成都王潁為大將軍、錄尚書事,河間王顒為太尉。罷丞相,復置司徒官。已卯,以梁王肜為太宰,領司徒。封齊王冏功臣葛CM牟平公,路季小黃公,衛毅平陰公,劉真安鄉公,韓泰封丘公。秋七月甲午,立吳王晏子國為漢王,復封常山王乂為長沙王。八月,大赦。戊辰,原徙邊者。益州刺史羅尚討羌,破之,己巳,徙南平王祥為宜都王。下邳王韡薨。以東平王楙為平東將軍、都督徐州諸軍事。九月,追東安王繇復其爵。丁丑,封
 楚王瑋子範為襄陽王。冬十月,流人李特反於蜀。十二月,司空何劭薨。封齊王冏子冰為樂安王,英為濟陽王,超為淮南王。是歲,郡國十二旱,六蝗。



 太安元年春正月庚子,安東將軍、譙王隨薨。三月癸卯,赦司、冀、兗、豫四州。皇太孫尚薨。夏四月,彗星晝見。五月;乙酉,侍中、太宰、領司徒、梁王肜薨。以右光祿大夫劉寔為太傅。太尉、河間王顒遣將衙博擊李特於蜀,為特所敗。特遂陷梓潼、巴西,害廣漢太守張微,自號大將軍。癸卯,以清河王遐子覃為皇太子,賜孤寡帛,大酺五日。以齊王冏為太師,東海王越為司空。秋七月,兗、豫、徐、冀等四
 州大水。冬十月,地震。十二月丁卯,河間王顒表齊王冏窺伺神器,有無君之心,與成都王潁、新野王歆、范陽王虓同會洛陽,請廢冏還第。長沙王乂奉乘輿屯南止車門,攻冏,殺之,幽其諸子于金墉城,廢冏弟北海王寔。大赦,改元。以長沙王乂為太尉、都督中外諸軍事。封東萊王蕤子照為齊王。



 二年春正月甲子朔,赦五歲刑。三月,李特攻陷益州。荊州刺史宋岱擊特,斬之,傳首京師。夏四月,特子雄復據益州。五月,義陽蠻張昌舉兵反,以山都人丘沈為主,改姓劉氏,偽號漢,建元神鳳,攻破郡縣,南陽太守劉彬,平
 南將軍羊尹,鎮南大將軍、新野王歆並遇害。六月,遣荊州刺史劉弘等討張昌于方城,王師敗績。秋七月,中書令卞粹、侍中馮蓀、河南尹李含等貳於長沙王乂,乂疑而害之。張昌陷江南諸郡,武陵太守賈隆、零陵太守孔紘、豫章太守閻濟、武昌太守劉根皆遇害。昌別帥石冰寇揚州,刺史陳徽與戰,大敗,諸郡盡沒。臨淮人封雲舉兵應之,自阜陵寇徐州。八月,河間王顒、成都王穎舉兵討長沙王乂,帝以乂為大都督,帥軍禦之。庚申,劉弘及張昌戰於清水,斬之。顒遣其將張方,潁遣其將陸機、牽秀、石超等來逼京師。乙丑,帝幸十三里橋,遣將軍皇甫
 商距方於宜陽。己巳,帝旋軍于宣武。庚午,舍于石樓。天中裂,無雲而雷。九月丁丑,帝次於河橋。壬午,皇甫商為張方所敗。甲申,帝軍於芒山。丁亥,幸偃師。辛卯,舍於豆田。癸巳,尚書右僕射、興晉侯羊玄之卒。帝旋于城東。丙申,進軍緱氏,擊牽秀,走之。大赦。張方入京城,燒清明、開陽二門,死者萬計。石超逼乘輿於緱氏。冬十月壬寅,帝旋於宮。石超焚緱氏,服御無遺。丁未,破牽秀、范陽王虓於東陽門外。戊申,破陸機於建春門,石超走,斬其大將賈崇等十六人,懸首銅駝街。張方退屯十三里橋。十一月辛巳,星晝隕,聲如雷。師王攻方壘,不利。方決千金堨,
 水碓皆涸。乃發王公奴婢手舂給兵稟,一品已下不從征者、男子十三以上皆從役。又發奴助兵,號為四部司馬。公私窮踧,米石萬錢。詔命所至,一城而已。壬寅夜,赤氣竟天,隱隱有聲。丙辰,地震。癸亥,東海王越執長沙王乂,幽於金墉城,尋為張方所害。甲子,大赦。丙寅,揚州秀才周、前南平內史王矩、前吳興內史顧秘起義軍以討石冰。冰退,自臨淮趣壽陽。征東將軍劉準遣廣陵度支陳敏擊冰。李雄自郫城攻益州刺史羅尚,尚委城而遁,雄盡有成都之地。封鮮卑段勿塵為遼西公。



 永興元年春正月丙午,尚書令樂廣卒。成都王穎自鄴
 諷于帝,乃大赦,改元為永安。帝逼於河間王顒,密詔雍州刺史劉沈、秦州刺史皇甫重以討之。沈舉舉兵攻長安,為顒所敗。張方大掠洛中,還長安。於是軍中大餒,人相食。以成都王穎為丞相。穎遣從事中郎盛夔等以兵五萬屯十二城門,殿中宿所忌者,潁皆殺之,以三部兵代宿衛。二月乙酉,廢皇后羊氏,幽於金墉城,黜皇太子覃復為清河王。三月,陳敏攻石冰,斬之,揚、徐二州平。河間王顒表請立成都王穎為太弟。戊申,詔曰:「朕以不德,纂承鴻緒,于茲十有五載。禍亂滔天,奸逆仍起,至乃幽廢重宮,宗廟圮絕。成都王穎溫仁惠和,剋平暴亂。其以穎
 為皇太弟、都督中外諸軍事,承相如故。」大赦,賜鰥寡高年帛三匹,大酺五日。丙辰,盜竊太廟服器。以太尉顒為太宰,太傅劉實為太尉。六月,新作三城門。秋七月丙申朔,右衛將軍陳以詔召百僚入殿中,因勒兵討成都王穎。戊戌,大赦,復皇后羊氏及皇太子覃。己亥,司徒王戎、東海王越、高密王簡、平昌公模、吳王晏、豫章王熾、襄陽王範、右僕射荀籓等奉帝北征,至安陽,眾十餘萬,穎遣其將石超距戰。己未,六軍敗績于蕩陰,矢及乘輿,百官分散,侍中嵇紹死之。帝傷頰,中三矢,亡六璽。帝遂幸超軍,餒甚,超進水,左右奉秋桃。超遣弟熙奉帝之鄴,穎
 帥群官迎謁道左。帝下輿涕泣,其夕幸於穎軍。穎府有九錫之儀,陳留王送貂蟬文衣鶡尾,明日,乃備法駕幸于鄴,唯豫章王熾、司徒王戎、僕射荀籓從。庚申,大赦,改元為建武。八月戊辰,穎殺東安王繇。張方復入洛陽,廢皇后羊氏及皇太子覃。匈奴左賢王劉元海反於離石,自號大單于。安北將軍王浚遣烏丸騎攻成都王穎于鄴,大敗之。穎輿帝單車走洛陽,服御分散,倉卒上下無齎,侍中黃門被囊中齎私錢三千,詔貸用。所在買飯以供,宮人止食于道中客舍。宮人有持升餘粇米飯及燥蒜鹽豉以進帝,帝啖之,御中黃門布被。次獲嘉,市粗米
 飯,盛以瓦盆,帝啖兩盂。有老父獻蒸雞,帝受之。至溫,將謁陵,帝喪履,納從者之履,下拜流涕,左右皆歔欷。及濟河,張方帥騎三千,以陽燧青蓋車奉迎。方拜謁,帝躬止之。辛巳,大赦,賞從者各有差。冬十一月乙未,方請帝謁廟,因劫帝幸長安。方以所乘車入殿中,帝馳避後園竹中。方逼帝升車,左右中黃門鼓吹十二人步從,唯中書監盧志侍側。方以帝幸其壘,帝令方具車載宮人寶物,軍人因妻略後宮,分爭府藏。魏晉已來之積,掃地無遺矣。行次新安,寒甚,帝墮馬傷足,尚書高光進面衣,帝嘉之。河間王顒帥官屬步騎三萬,迎于霸上。顒前拜謁,帝
 下車止之。以征西府為宮。唯僕射荀籓、司隸劉暾、太常鄭球、河南尹周馥與其遺官在洛陽,為留臺,承制行事,號為東西臺焉。丙午,留臺大赦,改元復為永安。辛丑,復皇后羊氏。李雄僭號成都王,劉元海僭號漢王。十二月丁亥,詔曰:「天禍晉邦,冢嗣莫繼。成都王穎自在儲貳,政績虧損,四海失望,不可承重,其以王還第。豫章王熾先帝愛子,令聞日新,四海注意,今以為皇太弟,以隆我晉邦。以司空越為太傅,與太宰顒夾輔朕躬。司徒王戎參錄朝政,光祿大夫王衍為尚書左僕射。安南將軍虓、安北將軍濬、平北將軍騰各守本鎮。高密王簡為鎮南將
 軍,領司隸校尉,權鎮洛陽;東中郎將模為寧北將軍、都督冀州,鎮于鄴;鎮南大將軍劉弘領荊州,以鎮南土。周馥、繆胤各還本部,百官皆復職。齊王冏前應還第,長沙王乂輕陷重刑,封其子紹為樂平縣王,以奉其嗣。自頃戎車屢征,勞費人力,供御之物皆減三分之二,戶調田租三分減一。蠲除苛政,愛人務本。清通之後,當還東京。」大赦,改元。以河間王顒都督中外諸軍事。



 二年春正月甲午朔,帝在長安。夏四月,詔封樂平王紹為齊王。丙子,張方廢皇后羊氏。六月甲子,侍中、司徒、安豐侯王戎薨。隴西太守韓稚攻秦州刺史張輔,殺之。李
 雄僭即帝位,國號蜀。秋七月甲午,尚書諸曹火,燒崇禮闥。東海王越嚴兵徐方,將西迎大駕。成都王穎部將公師籓等聚眾攻陷郡縣,害陽平太守李志、汲郡太守張延等,轉攻鄴,平昌公模遣將軍趙驤擊破之。八月辛丑,大赦。驃騎將軍、范陽王虓逐冀州刺史李義。揚州刺史曹武殺丹陽太守朱建。李雄遣其將李驤寇漢安。車騎大將軍劉弘逐平南將軍、彭城王釋於宛。九月庚寅朔,公師籓又害平原太守王景、清河太守馮熊。庚子,豫州刺史劉喬攻范陽王虓於許昌,敗之。壬子,以成都王穎為鎮軍大將軍、都督河北諸軍事,鎮鄴。河間王顒遣將
 軍呂郎屯洛陽。冬十月丙子,詔曰:「得豫州刺史劉喬檄,稱潁川太守劉輿迫脅驃騎將軍虓,距逆詔令,造構凶逆,擅劫郡縣,合聚兵眾,擅用茍晞為兗州,斷截王命。鎮南大將軍、荊州刺史劉弘,平南將軍、彭城王釋等,其各勒所統,徑會許昌,與喬並力。今遣右將軍張方為大都督,統精卒十萬,建武將軍呂郎、廣武將軍騫貙、建威將軍刁默等為軍前鋒,共會許昌,除輿兄弟。」丁丑,使前車騎將軍石超、北中郎將王闡討輿等。赤氣見于北方,東西竟天。有星孛於北斗。平昌公模遣將軍宋胄等屯河橋。十一月,立節將軍周權詐被檄,自稱平西將軍,復皇
 后羊氏。洛陽令何喬攻權,殺之,復廢皇后。十二月,呂朗等東屯滎陽,成都王穎進據洛陽,張方、劉弘等並桉兵不能禦。范陽王虓濟自官渡,拔滎陽,斬石超,襲許昌,破劉喬於蕭,喬奔南陽。右將軍陳敏舉兵反,自號楚公,矯稱被中詔,從沔漢奉迎天子;逐揚州刺史劉機、丹楊太守王曠;遣弟恢南略江州,刺史應邈奔弋陽。



 光熙元年春正月戊子朔,日有蝕之。帝在長安。河間王顒聞劉喬破,大懼,遂殺張方,請和於東海王越,越不聽。宋胄等破穎將樓裒,進逼洛陽,穎奔長安。甲子,越遣其將祁弘、宋胄、司馬纂等迎帝。三月,東萊惤令劉柏根反,
 自稱惤公,襲臨淄,高密王簡奔聊城。王浚遣將討柏根,斬之。夏四月己巳,東海王越屯于溫。顒遣弘農太守彭隨、北地太守刁默距祁弘等於湖。五月,枉矢西南流。范陽國地燃,可以爨。壬辰,祁弘等與刁默戰,默大敗,顒、潁走南山,奔于宛。弘等所部鮮卑大掠長安,殺二萬餘人。是日,日光四散,赤如血。甲午又如之。己亥,弘等奉帝還洛陽,帝乘牛車,行宮藉草,公卿跋涉。戊申,驃騎、范陽王虓殺司隸校尉刑喬。己酉,盜取太廟金匱及策文各四。六月丙辰朔,至自長安,升舊殿,哀感流涕。謁于太廟。復皇后羊氏。辛未,大赦,改元。秋七月乙酉朔,日有蝕之。太
 廟吏賈苞盜太廟靈衣及劍,伏誅。八月,以太傅、東海王越錄尚書,驃騎將軍、范陽王虓為司空。九月,頓丘太守馮嵩執成都王穎,送之于鄴。進東嬴公騰爵為車燕王,平昌公模為南陽王。冬十月,司空、范陽王虓薨。虓長史劉輿害成都王穎。十一月庚午,帝崩于顯陽殿,時年四十八,葬太陽陵。



 帝之為太子也,朝廷咸知不堪政事,武帝亦疑焉。嘗悉召東宮官屬,使以尚書事令太子決之,帝不能對。賈妃遣左右代對,多引古義。給事張泓曰:「太子不學,陛下所知,今宜以事斷,不可引書。」妃從之。泓乃具草,令帝書之。武帝覽而大悅,太子遂安。及居大位,政
 出群下,綱紀大壞,貨賂公行,勢位之家,以貴陵物,忠賢路絕,讒邪得志,更相薦舉,天下謂之互市焉。高平王沈作《釋時論》,南陽魯褒作《錢神論》,廬江杜嵩作《任子春秋》,皆疾時之作也。帝文嘗在華林園,聞蝦蟆聲,謂左右曰:「此鳴者為官乎,私乎?」或對曰:「在官地為官,在私地為私。」及天下荒亂,百姓餓死,帝曰:「何不食肉糜?」其蒙蔽皆此類也。後因食餅中毒而崩,或云司馬越之鴆。



 史臣曰:不才之子,則天稱大,權非帝出,政邇宵人。褒姒共叔帶並興,襄后與犬戎俱運。昔者,丹朱不肖,赧王逃責,相彼凶德,事關休咎,方乎土梗,以墜其情。溽暑之氣
 將闌,淫蛙之音罕記,乃彰蚩笑,用符顛隕。豈通才俊彥猶形於前代,增淫助虐獨擅於當今者歟?物號忠良,於茲拔本,人稱襖孽,自此疏源。長樂不祥,承華非命,生靈版蕩,社稷丘墟。古者敗國亡身,分鑣共軫,不有亂常,則多庸暗。豈明神喪其精魄,武皇不知其子也!



 贊曰:惠皇居尊,臨朝聽言。厥體斯昧,其情則昏。高臺望子,長夜奚冤。金墉毀冕,湯陰釋胄,及爾皆亡,滔天來遘。



\end{pinyinscope}