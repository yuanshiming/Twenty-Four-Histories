\article{列傳第五十二}

\begin{pinyinscope}

 陳壽王長文虞溥司馬彪王隱虞預孫盛干寶鄧粲謝沉習鑿齒徐廣



 陳壽,字承祚,巴西安漢人也。少好學,師事同郡譙周,仕蜀為觀閣令史。宦人黃皓專弄威權,大臣皆曲意附之,壽獨不為之屈,由是屢被譴黜。遭父喪,有疾,使婢丸藥,客往見之,鄉黨以為貶議。及蜀平,坐是沈滯者累年。司空張華愛其才,以壽雖不遠嫌,原情不至貶廢,舉為孝廉,除佐著作郎,出補陽平令。撰《蜀相諸葛亮集》,奏之。除
 著作郎,領本郡中正。撰魏吳蜀《三國志》,凡六十五篇。時人稱其善敘事,有良史之才。夏侯湛時著《魏書》,見壽所作,便壞己書而罷。張華深善之,謂壽曰:「當以《晉書》相付耳。」其為時所重如此。或云丁儀、丁暠有盛名於魏,壽謂其子曰:「可覓千斛米見與,當為尊公作佳傳。」丁不與之,竟不為立傳。壽父為馬謖參軍,謖為諸葛亮所誅,壽父亦坐被髡,諸葛瞻又輕壽。壽為亮立傳,謂亮將略非長,無應敵之才,言瞻惟工書,名過其實。議者以此少之



 張華將舉壽為中書郎,荀勖忌華而疾壽,遂諷吏部遷壽為長廣太守。辭母老不就。杜預將之鎮,復薦之於帝,宜
 補黃散。由是授御史治書。以母憂去職。母遺言令葬洛陽,壽遵其志。又坐不以母歸葬,竟被貶議。初,譙周嘗謂壽曰:「卿必以才學成名,當被損折,亦非不幸也。宜深慎之。」壽至此,再致廢辱,皆如周言。後數歲,起為太子中庶子,未拜。



 元康七年,病卒,時年六十五。梁州大中正、尚書郎范頵等上表曰:「昔漢武帝詔曰:『司馬相如病甚,可遣悉取其書。」使者得其遺書,言封禪事,天子異焉。臣等案:故治書侍御史陳壽作《三國志》,辭多勸誡,明乎得失,有益風化,雖文艷不若相如,而質直過之,願垂採錄。」於是詔下河南尹、洛陽令,就家寫其書。壽又撰《古國志》五十
 篇、《益都耆舊傳》十篇,餘文章傳於世。



 王長文,字德睿,廣漢郪人也。少以才學知名,而蕩不羈,州府辟命皆不就。州辟別駕,乃微服竊出,舉州莫知所之。後於成都市中蹲踞齧胡餅。刺史知其不屈,禮遣之。閉門自守,不交人事。著書四卷,擬《易》,名曰《通玄經》,有《文言》、《卦象》,可用卜筮,時人比之揚雄《太玄》。同郡馬秀曰:「揚雄作《太玄》,惟桓譚以為必傳後世。晚遭陸績,玄道遂明。長文《通玄經》未遭陸績、君山耳。」



 太康中,蜀土荒饉,開倉振貸。長文居貧,貸多,後無以償。郡縣切責,送長文到
 州。刺史徐幹捨之,不謝而去。後成都王穎引為光源令。或問:「前不降志,今何為屈?」長文曰:「祿以養親,非為身也。」梁王肜為丞相,引為從事中郎。在洛出行,輒著白旃小鄣以載車,當時異焉。後終於洛。



 虞溥,字允源,高平昌邑人也。父祕,為偏將軍。鎮隴西。溥從父之官,專心墳籍。時疆場閱武,人爭視之,溥未嘗寓目。郡察孝廉,除郎中,補尚書都令史。尚書令衛瓘、尚書褚並器重之。溥謂瓘曰:「往者金馬啟符,大晉應天,宜復先王五等之制,以綏久長。不可承暴秦之法,遂漢魏
 之失也。」瓘曰:「歷代歎此,而終未能改。」



 稍遷公車司馬令,除鄱陽內史。大修庠序,廣詔學徒,移告屬縣曰:「學所以定情理性而積眾善者也。情定於內而行成於外,積善於心而名顯於教,故中人之性隨教而移,積善則習與性成。唐虞之時,皆比屋而可封,及其廢也,而云可誅,豈非化以成俗,教移人心者哉!自漢氏失御,天下分崩,江表寇隔,久替王教,庠序之訓,廢而莫修。今四海一統,萬里同軌,熙熙兆庶,咸休息乎太和之中,宜崇尚道素,廣開學業,以贊協時雍,光揚盛化。」乃具為條制。於是至者七百餘人。溥乃作誥以獎訓之,曰:



 文學諸生皆冠帶之
 流,年盛志美,始涉學庭,講修典訓,此大成之業,立德之基也。夫聖人之道淡而寡味,故始學者不好也。及至期月,所觀彌博,所習彌多,日聞所不聞,日見所不見,然後心開意朗,敬業樂群,忽然不覺大化之陶己,至道之入神也。故學之染人,甚於丹青。丹青吾見其久而渝矣,未見久學而渝者也。



 夫工人之染,先修其質,後事其色,質修色積,而染工畢矣。學亦有質,孝悌忠信是也。君子內正其心,外修其行,行有餘力,則以學文,文質彬彬,然後為德。夫學者不患才不及,而患志不立,故曰希驥之馬,亦驥之乘,希顏之徒,亦顏之倫也。又曰契而舍之,朽木
 不知;契而不舍,金石可虧。斯非其效乎!



 今諸生口誦聖人之典,體閒庠序之訓,比及三年,可以小成。而令名宣流,雅譽日新,朋友欽而樂之,朝士敬而歎之。於是州府交命擇官而仕,不亦美乎!若乃含章舒藻,揮翰流離,稱述世務,探賾究奇,使楊斑韜筆,仲舒結舌,亦惟才所居,固無常人也。然積一勺以成江河,累微塵以崇峻極,匪至匪勤,理無由濟也。諸生若絕人間之務,心專親學,累一以貫之,積漸以進之,則亦或遲或速,或先或後耳,何滯而不通,何遠而不至邪!



 時祭酒求更起屋行禮,溥曰:「君子行禮,無常處也,故孔子射於矍相之圃,而行禮於
 大樹之下。況今學庭庠序,高堂顯敞乎!」



 溥為政嚴而不猛,風化大行,有白烏集于郡庭。注《春秋》經、傳,撰《江表傳》及文章詩賦數十篇。卒於洛,時年六十二。子勃,過江上《江表傳》於元帝,詔藏于祕書。



 司馬彪,字紹統,高陽王睦之長子也。出後宣帝弟敏。少篤學不倦,然好色薄行,為睦所責,故不得為嗣,雖名出繼,實廢之也。彪由此不交人事,而專精學習,故得博覽群籍,終其綴集之務。初拜騎都尉。泰始中,為祕書郎,轉丞。注《莊子》,作《九州春秋》。以為「先王立史官以書時事,載
 善惡以為沮勸,撮教世之要也。是以《春秋》不修,則仲尼理之;《關雎》既亂,則師摯修之。前哲豈好煩哉?蓋不得已故也。漢氏中興,訖于建安,忠臣義土亦以昭著,而時無良史,記述煩雜,譙周雖已刪除,然猶未盡,安順以下,亡缺者多。」彪乃討論眾書,綴其所聞,起於世祖,終于孝獻,編年二百,錄世十二,通綜上下,旁貫庶事,為紀、志、傳凡八十篇,號曰《續漢書》。



 泰始初,武帝親祠南郊,彪上疏定議,語在《效祀志》。後拜散騎侍郎。惠帝末年卒,時所六十餘。



 初,譙周以司馬遷《史記》書周秦以上,或採俗語百家之言,不專據正經,周於是作《古史考》二十五篇,皆憑舊
 典,以糾遷之謬誤。彪復以周為未盡善也,條《古史考》中凡百二十二事為不當,多據《汲塚紀年》之義,亦行於世。



 王隱,字處叔,陳郡陳人也。世寒素。父銓,歷陽令,少好學,有著述之志,每私錄晉事及功臣行狀,未就而卒。隱以儒素自守,不交勢援,博學多聞,受父遺業,西都舊事多所諳究。



 建興中,過江,丞相軍諮祭酒涿郡祖納雅相知重。納好博弈,每諫止之。納曰:「聊用忘憂耳。」隱曰:「蓋古人遭時,則以功達其道;不遇,則以言達其才,故否泰不窮也。當今晉未有書,天下大亂,舊事蕩滅,非凡才所能立。
 君少長五都,游宦四方,華夷成敗皆在耳目,何不述而裁之!應仲遠作《風俗通》,崔子真作《政論》,蔡伯喈作《勸學篇》,史游作《急就章》,猶行於世,便為沒而不朽。當其同時,人豈少哉?而了無聞,皆由無所述作也。故君子疾沒世而無聞,《易》稱自強不息,況國史明乎得失之跡,何必博弈而後忘憂哉」納喟然嘆曰:「非不悅子道,力不足也。」乃上疏薦隱。元帝以草創務殷,未遑史官,遂寢不報。



 太興初,典章稍備,乃召隱及郭璞俱為著作郎,令撰晉史。豫平王敦功,賜爵平陵鄉侯。時著作郎虞預私撰《晉書》,而生長東南,不知中朝事,數訪於隱,並借隱所著書竊
 寫之,所聞漸廣。是後更疾隱,形于言色。預既豪族,交結權貴,共為朋黨,以斥隱,竟以謗免,黜歸于家。貧無資用,書遂不就,乃依征西將軍庾亮於武昌。亮供其紙筆,書乃得成,詣闕上之。隱雖好著述,而文辭鄙拙,蕪舛不倫。其書次第可觀者,皆其父所撰;文體混漫義不可解者,隱之作也。年七十餘,卒於家。



 隱兄瑚,字處仲。少重武節,成都王穎舉兵向洛,以為冠軍參軍,積功,累遷游擊將軍,與司隸滿奮、河南尹周馥等俱屯大司馬門,以衛宮掖。時上官已縱暴,瑚與奮等共謀除之,反為所害。



 虞預,字叔寧,徵士喜之弟也,本名茂,犯明穆皇后母諱,故改焉。預十二而孤,少好學,有文章。餘姚風俗,各有朋黨,宗人共薦預為縣功曹,欲使沙汰穢濁。預書與其從叔父曰:「近或聞諸君以預入寺,便應委質,則當親事,不得徒已。然預下愚,過有所懷。邪黨互瞻,異同蜂至,一旦差跌,眾鼓交鳴。毫釐之失,差以千里,此古人之炯戒,而預所大恐也。」卒如預言,未半年,遂見斥退。



 太守庾琛命為主簿,預上記陳時政所失,曰:「軍寇以來,賦役繁數,兼值年荒,百姓失業,是輕徭薄斂,寬刑省役之時也。自頃
 長吏輕多去來,送故迎新,交錯道路。受迎者惟恐船馬之不多,見送者惟恨吏卒之常少。窮奢竭費謂之忠義,省煩從簡呼為薄俗,轉相放效,流而不反,雖有常防,莫肯遵修。加以王途未夷,所在停滯,送者經年,永失播植。一夫不耕,十夫無食,況轉百數,所妨不訾。愚謂宜勒屬縣,若令、尉先去官者,人船吏侍皆具條列,到當依法減省,使公私允當。又今統務多端,動加重制,每有特急,輒立督郵。計今直兼三十餘人,人船吏侍皆當出官,益不堪命,宜復減損,嚴為之防。」琛善之,即皆施行。太守紀瞻到,預復為主簿,轉功曹史。察孝廉,不行。安東從事中郎
 諸葛恢、參軍庾亮等薦預,召為丞相行參軍兼記室。遭母憂,服竟,除佐著作郎。



 太興二年,大旱,詔求讜言直諫之士,預上書諫曰:



 大晉受命,于今五十餘載。自元康以來,王德始闕,戎翟及於中國,宗廟焚為灰燼,千里無煙爨之氣,華夏無冠帶之人,自天地開闢,書籍所載,大亂之極,未有若茲者也。



 陛下以聖德先覺,超然遠鑒,作鎮東南,聲教遐被,上天眷顧,人神贊謀,雖云中興,其實受命,少康、宣王誠未足喻。然《南風》之歌可著,而陵遲之俗未改者,何也?臣愚謂為國之要在於得才,得才之術在於抽引。茍其可用,仇賤必舉。高宗、文王思佐發夢,拔巖
 徒以為相,載釣老而師之。下至列國,亦有斯事,故燕重郭隗而三士競至,魏式乾木而秦兵退舍。今天下雖弊,人士雖寡,十室雖寡,十室之邑,必有忠信,世不乏驥,求則可致。而束帛未賁於丘園,蒲輪頓轂而不駕,所以大化不洽而用雍熙有闕者也。



 預以寇賊未平,當須良將,又上疏曰:



 臣聞承平之世,其教先文,撥亂之運,非武不剋;故牧野之戰,呂望杖鉞;淮夷作難,召伯專征;玁狁為暴,衛霍長驅。故陰陽不和,擢士為相;三軍不勝,拔卒為將。漢帝既定天下,猶思猛士以守四方;孝文志存鉅鹿,馮唐進說,魏尚復守。《詩》稱「赳赳武夫,公侯干城」,折衝之佐,豈可忽哉!
 況今中州荒弊,百無一存,牧守官長非戎貊之族類,即寇竊之幸脫。陛下登阼,威暢四遠,故令此等反善向化。然狼子獸心,輕薄易動,羯虜未殄,益使難安。周撫、陳川相係背叛;徐龕驕黠,無所拘忌,放兵侵掠,罪已彰灼。



 昔葛伯違道,湯獻之牛;吳濞失禮,錫以几杖,惡成罪著,方復加戮。龕之小醜,可不足滅。然豫備不虞,古之善教,矧乃有虞,可不為防!為防之術,宜得良將。將不素簡,難以應敵。壽春無鎮,祖逖孤立,前有勁虜,後無係援,雖有智力,非可持久。願陛下諮之群公,博舉於眾。若當局之才,必允其任,則宜獎厲,使不顧命。旁料冗猥。或有可者,厚
 加寵待,足令忘身。昔英布見慢,恚欲自裁,出觀供置,然後致力。禮遇之恩,可不隆哉!



 誠知山河之量非塵露可益,神鑒之慮非愚淺所測;然匹夫嫠婦猶有憂國之言,況臣得廁朝堂之末,蒙冠帶之榮者乎!



 轉瑯邪國常侍,遷祕書丞、著作郎。



 咸和初,夏旱,詔眾官各陳致雨之意。預議曰:



 臣聞天道貴信,地道貴誠。誠信者,蓋二儀所以生植萬物,人君所以保乂黎蒸。是以殺伐擬於震電,推恩象於雲雨。刑罰在於必信,慶賞貴於平均。臣聞間者以來,刑獄轉繁,多力者則廣牽連逮,以稽年月;無援者則嚴其檟楚,期於入重。是以百姓嗷然,感傷和氣。臣愚
 以為輕刑耐罪,宜速決遣,殊死重囚,重加以請。寬徭息役,務遵節儉,砥礪朝臣,使各知禁。



 蓋老牛不犧,禮有常制,而自頃眾官拜授祖贈,轉相夸尚,屠殺牛犢,動有十數,醉酒流湎,無復限度,傷財敗俗,所虧不少。



 昔殷宗修德以消桑穀之異,宋景善言以退熒惑之變,楚國無災,莊王是懼。盛德之君,未嘗無眚,應以信順,天祐乃隆。臣學見淺暗,言不足採。



 從平王含,賜爵西鄉侯。蘇峻作亂,預先假歸家,太守王舒請為諮議參軍。峻平,進爵平康縣侯,遷散騎侍郎,著作如故。除散騎常侍,仍領著作。以年老歸,卒于家。



 預雅好經史,憎疾玄虛,其論阮籍裸袒,
 比之伊川被髮,所以胡虜遍於中國,以為過衰周之時。著《晉書》四十餘卷、《會稽典錄》二十篇、《諸虞傳》十二篇,皆行於世。所著詩賦碑誄論難數十篇。



 孫盛,字安國,太原中都人。祖楚,馮翊太守。父恂,潁川太守。恂在郡遇賊,被害。盛年十歲,避難渡江。及長,博學,善言名理。于時殷浩擅名一時,與抗論者,惟盛而已。盛嘗詣浩談論,對食,奮擲麈尾,毛悉落飯中,食冷而復暖者數四,至暮忘餐,理竟不定。盛又著醫卜及《易象妙於見形論》,浩等竟無以難之,由是遂知名。



 起家佐著作郎,以
 家貧親老,求為小邑,出補瀏陽令。太守陶侃請為參軍。庾亮代侃,引為征西主簿,轉參軍。時丞相王導執政,亮以元舅居外,南蠻校尉陶稱讒構其間,導、亮頗懷疑貳。盛密諫亮曰:「王公神情朗達,常有世外之懷,豈肯為凡人事邪!此必佞邪之徒欲間內外耳。」亮納之。庾翼代亮,以盛為安西諮議參軍,尋遷廷尉正。會桓溫代翼,留盛為參軍,與俱伐蜀,軍次彭模,溫自以輕兵入蜀,盛領贏老輜重在後,賊數千忽至,眾皆遑遽。盛部分諸將,并力距之,應時敗走。蜀平,賜爵安懷縣侯,累遷溫從事中郎。從入關平洛,以功進封吳昌縣侯,出補長沙太守。以家
 貧,頗營資貨,部從事至郡察知之,服其高名而不劾之。盛與溫箋,而辭旨放蕩,稱州遣從事觀採風聲,進無威鳳來儀之美,退無鷹鸇搏擊之用,徘徊湘川,將為怪鳥。溫得盛箋,復遣從事重案之,臟私鋃籍,檻車收盛到州,捨而不罪。累遷祕書監,加給事中。年七十二卒。



 盛篤學不倦,自少至老,手不釋卷。著《魏氏春秋》、《晉陽秋》,并造詩賦論難復數十篇。《晉陽秋》詞直而理正,咸稱良史焉。既而桓溫見之,怒謂盛子曰:「枋頭誠為失利,何至乃如尊君所說!若此史遂行,自是關君門戶事。」其子遽拜謝,謂請刪改之。時盛年老還家,性方嚴有軌憲,雖子孫白,
 而庭訓愈峻。至此,諸子乃共號泣稽顙,請為百口切計。盛大怒。諸子遂爾改之。盛寫兩定本,寄於慕容俊。太元中,孝武帝博求異聞,始於遼東得之,以相考校,多有不同,書遂兩存。子潛、放。



 潛字齊由,為豫章太守。殷仲堪之討王國寶也,潛時在郡,仲堪逼以為諮議參軍,固辭不就,以憂卒。



 放字齊莊,幼稱令慧。年七八歲,在荊州,與父俱從庾亮獵,亮謂曰:「君亦來邪?」應聲答曰:「無小無大,從公于邁。」亮又問:「欲齊何莊邪?」放曰:「欲齊莊周。」亮曰:「不慕仲尼邪?」答曰:「仲尼生而知之,非希企所及。」亮大奇之,曰:「王輔嗣弗
 過也。」庾翼子爰客嘗候盛,見放而問曰:「安國何在?」放答曰:「庾稚恭家。」爰客大笑曰:「諸孫太盛,有兒如此也!」放又曰:「未若諸庾翼翼。」既而語人曰:「我故得重呼奴父也。」終於長沙相。



 干寶,字令升,新蔡人也。祖統,吳奮武將軍、都亭侯。父瑩,丹陽丞。寶少勤學,博覽書記,以才器召為著作郎。平杜弢有功,賜爵關內侯。



 中興草創,未置史官,中書監王導上疏曰:「夫帝王之跡,莫不必書,著為令典,垂之無窮。宣皇帝廓定四海,武皇帝受禪於魏,至德大勳,等蹤上聖,
 而紀傳不存於王府,德音未被乎管弦。陛下聖明,當中興之盛,宜建立國史,撰集帝紀,上敷祖宗之烈,下紀佐命之勳,務以實錄,為後代之準,厭率土之望,悅人神之心,斯誠雍熙之至美,王者之弘基也。宜備史官,敕佐著作郎干寶等漸就撰集。」元帝納焉。寶於是始領國史。以家貧,求補山陰令,遷始安太守。王導請為司徒右長史,遷散騎常侍,著《晉紀》,自宣帝迄于愍帝五十三年,凡二十卷,奏之。其書簡略,直而能婉,咸稱良史。



 性好陰陽術數,留思京房、夏侯勝等傳。寶父先有所寵侍婢,母甚妒忌,及父亡,母乃生推婢於墓中。寶兄弟年小,不之審也。
 後十餘年,母喪,開墓,而婢伏棺如生,載還,經日乃蘇。言其父常取飲食與之,恩情如生,在家中吉凶輒語之,考校悉驗,地中亦不覺為惡。既而嫁之,生子。又寶兄嘗病氣絕,積日不冷,後遂悟,云見天地間鬼神事,如夢覺,不自知死。寶以此遂撰集古今神祇靈異人物變化。名為《搜神記》,凡三十卷。以示劉惔,惔曰:「卿可謂鬼之董狐。」寶既博採異同,遂混虛實,因作序以陳其志曰:



 雖考先志於載籍,收遺逸於當時,蓋非一耳一目之所親聞睹也,亦安敢謂無失實者哉!衛朔失國,二傳互其所聞;呂望事周,子長存其兩說,若此比類,往往有焉。從此觀之,聞
 見之難一,由來尚矣。夫書赴告之定辭,據國史之方策,猶尚若茲,況仰述千載之前,記殊俗之表,綴片言於殘闕,訪行事於故老,將使事不二迹,言無異途,然後為信者,固亦前史之所病。然而國家不廢注記之官,學士不絕誦覽之業,豈不以其所失者小,所存者大乎!今之所集,設有承於前載者,則非余之罪也。若使采訪近世之事,茍有虛錯,願與先賢前儒分其譏謗。及其著述,亦足以明神道之不誣也。



 群言百家不可勝覽,耳目所受不可勝載,今粗取足以演八略之旨,成其微說而已。幸將來好事之士錄其根體,有以游心寓目而無尤焉。



 寶又
 為《春秋左氏義外傳》,注《周易》、《周官》凡數十篇,及雜文集皆行於世。



 鄧粲,長沙人。少以高潔著名,與南陽劉驎之、南郡劉尚公同志友善,並不應州郡辟命。荊州刺史桓沖卑辭厚禮請粲為別駕,粲嘉其好賢,乃起應召。驎之、尚公謂之曰:「卿道廣學深,眾所推懷,忽然改節,誠失所望。」粲笑答曰:「足下可謂有志於隱而未知隱。夫隱之為道,朝亦可隱,市亦可隱。隱初在我,不在於物。」尚公等無以難之,然粲亦於此名譽減半矣,後患足疾,不能朝拜,求去職,不
 聽,令臥視事。後以病篤,乞骸骨,許之。粲以父騫有忠信言而世無知者,著《元明紀》十篇,注《老子》,並行於世。



 謝沈,字行思,會稽山陰人也。曾祖斐,吳豫章太守。父秀,吳翼正都尉。沈少孤,事母至孝,博學多識,明練經史。郡命為主簿、功曹,察孝廉,太尉郗鑒辟,並不就。會稽內史何充引為參軍,以母老去職。平西將軍庾亮命為功曹,征北將軍蔡謨版為參軍,皆不就。閑居養母,不交人事,耕耘之暇,研精墳籍。康帝即位,朝議疑七廟迭毀,乃以太學博士徵,以質疑滯。以母憂去職。服闋,除尚書度支
 郎。何充、庾冰並稱沉有史才,遷著作郎,撰《晉書》三十餘卷。會卒,時年五十二。沉先著《後漢書》百卷及《毛詩》、《漢書外傳》,所著述及詩賦文論皆行於世。其才學在虞預之右云。



 習鑿齒,字彥威,襄陽人也。宗族富盛,世為鄉豪。鑿齒少有志氣,博學洽聞,以文筆著稱。荊州刺史桓溫辟為從事,江夏相袁喬深器之,數稱其才於溫,轉西曹主簿,親遇隆密。



 時溫有大志,追蜀人知天文者至,夜執手問國家祚運修短。答曰:「世祀方永。」疑其難言,乃飾辭云:「如
 君言,豈獨吾福,乃蒼生之幸。然今日之語自可令盡,必有小小厄運,亦宜說之。」星人曰:「太微、紫微、文昌三宮氣候如此,決無憂虞。至五十年外不論耳。」溫不悅,乃止。異日,送絹一匹、錢五千文以與之。星人乃馳詣鑿齒曰:「家在益州,被命遠下,今受旨自裁,無由致其骸骨。緣君仁厚,乞為標碣棺木耳。」鑿齒問其故,星人曰:「賜絹一匹,令僕自裁,惠錢五千,以買棺耳。」鑿齒曰:「君幾誤死!君嘗聞前知星宿有不覆之義乎?此以絹戲君,以錢供道中資,是聽君去耳。」星人大喜,明便詣溫別。溫問去意,以鑿齒言答。溫笑曰:「鑿齒憂君誤死,君定是誤活。然徒三十年
 看儒書,不如一詣習主簿。」



 累遷別駕。溫出征伐,鑿齒或從或守,所在任職,每處機要,蒞事有績,善尺牘論議,溫甚器遇之。時清談文章之士韓伯、伏滔等並相友善,後使至京師。簡文亦雅重焉。既還,溫問:「相王何似?」答曰:「生平所未見。」以此大忤溫旨,左遷戶曹參軍。時有桑門釋道安,俊辯有高才,自北至荊州,與鑿齒初相見。道安曰:「彌天釋道安。」鑿齒曰:「四海習鑿齒。」時人以為佳對。



 初,鑿齒與其二舅羅崇、羅友俱為州從事。及遷別駕,以坐越舅右,屢經陳請。溫後激怒既盛,乃超拔其二舅,相繼為襄陽都督,出鑿齒為滎陽太守。溫弟祕亦有才氣,素與
 鑿齒相親善。鑿齒既罷郡歸,與祕書曰:



 吾以去五三日來達襄陽,觸目悲感,略無懽情,痛惻之事,故非書言之所能具也。每定省家舅,從北門入,西望隆中,想臥龍之吟;東眺白沙,思鳳雛之聲;北臨樊墟,存鄧老之高;南眷城邑,懷羊公之風;縱目檀溪,念崔徐之友;肆睇魚梁,追二德之遠,未嘗不徘徊移日,惆悵極多,撫乘躊躇,慨爾而泣。曰若乃魏武之所置酒,孫堅之所隕斃,裴杜之故居,繁王之舊宅,遺事猶存,星列滿目。瑣瑣常流,碌碌凡士,焉足以感其方寸哉!



 夫芬芳起於椒蘭,清響生乎琳瑯。命世而作佐者,必垂可大之餘風;高尚而邁德者,
 必有明勝之遺事。若向八君子者,千載猶使義想其為人,況相去不遠乎!彼一時也,此一時也,焉知今日之才不如疇辰,百年之後,吾與足下不並為景升乎!



 其風期俊邁如此。



 是時溫覬覦非望,鑿齒在郡,著《漢晉春秋》以裁正之。起漢光武,終於晉愍帝。於三國之時,蜀以宗室為正,魏武雖受漢禪晉,尚為篡逆,至文帝平蜀,乃為漢亡而晉始興焉。引世祖諱炎興而為禪受,明天心不可以勢力強也。凡五十四卷。後以腳疾,遂廢于里巷。



 及襄陽陷於苻堅,堅素聞其名,與道安俱輿而致焉。既見,與語,大悅之,賜遺甚厚。又以其蹇疾,與諸鎮書:「昔晉氏
 平吳,利在二陸;今破漢南,獲士裁一人有半耳。」俄以疾歸襄陽。尋而襄鄧反正,朝廷欲征鑿齒,使典國史,會卒,不果。臨終上疏曰:



 臣每謂皇晉宜越魏繼漢,不應以魏後為三恪。而身微官卑,無由上達,懷抱愚情,三十餘年。今沈淪重疾,性命難保,遂嘗懷此,當與之朽爛,區區之情,切所悼惜,謹力疾著論一篇,寫上如左。願陛下考尋古義,求經常之表,超然遠覽,不以臣微賤廢其所言。論曰:



 或問:「魏武帝功蓋中夏,文帝受禪於漢,而吾子謂漢終有晉,豈實理乎?且魏之見廢,晉道亦病,晉之臣子寧可以同此言哉!」



 答曰:「此乃所以尊晉也,但絕節赴曲,非
 常耳所悲,見殊心異,雖奇莫察,請為子言焉。



 「昔漢氏失御,九州殘隔,三國乘間,鼎歭數世,干戈日尋,流血百載,雖各有偏平,而其實亂也,宣皇帝勢逼當年,力制魏氏,蠖屈從時,遂羈戎役,晦明掩耀,龍潛下位,俯首重足,鞠躬屏息,道有不容之難,躬蹈履霜之險,可謂危矣!魏武既亡,大難獲免,始南擒孟達,東蕩海隅,西抑勁蜀,旋撫諸夏,摧吳人入侵之鋒,掃曹爽見忌之黨,植靈根以跨中嶽,樹群才以翼子弟,命世之志既恢,非常之業亦固。景文繼之,靈武冠世,剋伐貳違,以定厥庸,席卷梁益,奄征西極,功格皇天,勳侔古烈,豐規顯祚,故以灼如也。至
 於武皇,遂并彊吳,混一宇宙,乂清四海,同軌二漢。除三國之大害,靜漢末之交爭,開九域之蒙晦,定千載之盛功者,皆司馬氏也。而推魏繼漢,以晉承魏,比義唐虞,自託純臣,豈不惜哉!



 「今若以魏有代王之德,則其道不足;有靜亂之功,則孫劉鼎立。道不足則不可謂制當年,當年不制於魏,則魏未曾為天下之主;王道不足於曹,則曹未始為一日之王矣。昔共工伯有九州,秦政奄平區夏,鞭撻華戎,專總六合,猶不見序於帝王,淪沒於戰國,何況暫制數州之人,威行境內而已,便可推為一代者乎!



 「若以晉嘗事魏,懼傷皇德,拘惜禪名,謂不可割,則惑
 之甚者也。何者?隗囂據隴,公孫帝蜀,蜀隴之人雖服其役,取之大義,於彼何有!且吳楚僭號,周室未亡,子文、延陵不見貶絕。宜皇帝官魏,逼於性命,舉非擇木,何虧德美,禪代之義,不同堯舜,校實定名,必彰於後,人各有心,事胡可掩!定空虛之魏以屈於己,孰若杖義而以貶魏哉!夫命世之人正情遇物,假之際會,必兼義勇。宣皇祖考立功于漢,世篤爾勞,思報亦深。魏武超越,志在傾主,德不素積,義險冰薄,宣帝與之,情將何重!雖形屈當年,意申百世,降心全己,憤慨於下,非道服北面,有純臣之節,畢命曹氏,忘濟世之功者也。



 「夫成業者係於所為,不
 係所藉;立功者言其所濟,不言所起。是故漢高稟命於懷王,劉氏乘斃於亡秦,超二偽以遠嗣,不論近而計功,考五德於帝典,不疑道於力政,季無承楚之號,漢有繼周之業,取之既美,而己德亦重故也。凡天下事有可借喻於古以曉於今,定之往昔而足為來證者。當陽秋之時,吳楚二國皆僭號之王也,若使楚莊推鄢郢以尊有德,闔閭舉三江以奉命世,命世之君、有德之主或藉之以應天,或撫之而光宅,彼必自係於周室,不推吳楚以為代明矣。況積勳累功,靜亂寧眾,數之所錄,眾之所與,不資於燕噲之授,不賴於因藉之力,長轡廟堂,吳蜀兩
 斃,運奇二紀而平定天下,服魏武之所不能臣,蕩累葉之所不能除者哉!



 「自漢末鼎沸五六十年,吳魏犯順而強,蜀人杖正而弱,三家不能相一,萬姓曠而無主。夫有定天下之大功,為天下之所推,孰如見推於闇人,受尊於微弱?配天而為帝,方駕於三代,豈比俯首於曹氏,側足於不正?即情而恆實,取之而無慚,何與詭事而託偽,開亂於將來者乎?是故故舊之恩可封魏後,三恪之數不宜見列。以晉承漢,功實顯然,正名當事,情體亦厭,又何為虛尊不正之魏而虧我道於大通哉!



 「昔周人詠祖宗之德,追述翦商之功;仲尼明大孝之道,高稱配天之
 義。然后稷勤於所職,聿來未以翦商,異於司馬氏仕乎曹族,三祖之寓於魏世矣。且夫魏自君之道不正,則三祖臣魏之義未盡。義未盡,故假塗以運高略;道不正,故君臣之節有殊。然則弘道不以輔魏而無逆取之嫌,高拱不勞汗馬而有靜亂之功者,蓋勛足以王四海,義可以登天位,雖我德慚於有周,而彼道異於殷商故也。



 「今子不疑共工之不得列於帝王,不嫌漢之係周而不係秦,何至於一魏猶疑滯而不化哉!夫欲尊其君而不知推之於堯舜之道,欲重其國而反厝之於不勝之地,豈君子之高義!若猶未悟,請於是止矣。」



 子辟強,才學有父
 風,位至驃騎從事中郎。



 徐廣,字野民,東莞姑幕人,侍中邈之弟也。世好學,至廣尤為精純,百家數術無不研覽。謝玄為兗州,辟從事。譙王恬為鎮北,補參軍。孝武世,除祕書郎,典校祕書省。增置省職,轉員外散騎侍郎,仍領校書。尚書令王珣深相欽重,舉為祠部郎,會稽世子元顯時錄尚書,欲使百僚致敬,內外順之,使廣為議,廣常以為愧焉。元顯引為中軍參軍,遷領軍長史。桓玄輔政,以為大將軍文學祭酒,義熙初,奉詔撰車服儀注,除鎮軍諮議,領記室,封樂成
 侯,轉員外散騎常侍,領著作。尚書奏:「左史述言,右官書事,《乘》《志》顯於晉鄭,《春秋》著乎魯史。自聖代有造《中興記》者,道風帝典,煥乎史策。而太和以降,世歷三朝,玄風聖迹,倏為疇古。臣等參詳,宜敕著作郎徐廣撰成國史。」於是敕廣撰集焉。遷驍騎將軍,領徐州大中正,轉正員常侍、大司家、仍領著作如故。十二年,勒成《晉紀》,凡四十六卷,表上之。因乞解史任,不許。遷秘書監。



 初,桓玄篡位,帝出宮,廣陪列,悲動左右。及劉裕受禪,恭帝遜位,廣獨哀感,涕泗交流。謝晦見之,謂曰:「徐公將無小過也。」廣收淚而言曰:「君為宋朝佐命,吾乃晉室遺老,憂喜之事固不
 同時。」乃更歔欷。因辭衰老,乞歸桑梓。性好讀書,老猶不倦。年七十四,卒於家。廣《答禮問》行於世。



 史臣曰:古之王者咸建史臣,昭法立訓,莫近於此。若夫原始要終,紀情括性,其言微而顯,其義皎而明,然後可以茵藹緹油,作程遐世者也。丘明即沒,班馬迭興,奮鴻筆於西京,騁直詞於東觀。自斯已降,分明競爽,可以繼明先典者,陳壽得之乎!江漢英靈,信有之矣。允源將率之子,篤志典墳;紹統戚籓之胤,研機載籍。咸能綜緝文,垂諸不朽,豈必克傳門業,方擅箕裘者哉!處叔區區,勵精著述,混淆蕪舛,良不足觀。叔寧寡聞,穿窬王氏,雖
 勒成一家,未足多尚。令升、安國有良史之才,而所著之書惜非正典。悠悠晉室,斯文將墜。鄧粲、謝沉祖述前史,葺宇重軒之下,施床連榻之上,奇詞異義,罕見稱焉。習氏、徐公俱云筆削,彰善癉惡,以為懲勸。夫蹈忠履正,貞士之心;背義圖榮,君子不敢。而彥威跡淪寇壤,逡巡於偽國;野民運遭革命,流漣於舊朝。行不違言,廣得之矣。



 贊曰:陳壽含章,巖巖孤峙。彪溥勵節,摛辭綜理。王恧雅才,虞慚惇史。幹孫撫翰,前良可擬。鄧謝懷鉛,異聞無紀。習亦研思,徐非絢美,咸被簡冊,共傳遙祀。



\end{pinyinscope}