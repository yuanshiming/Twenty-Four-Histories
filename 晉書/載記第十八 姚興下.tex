\article{載記第十八 姚興下}

\begin{pinyinscope}
姚興
 \gezhu{
  下}



 晉義熙二年,平北將軍、梁州督護苻宣入漢中,興梁州別駕呂營、漢中徐逸、席難起兵應宣,求救於楊盛。盛遣軍臨濜口,南梁州刺史王敏退守武興。楊盛復通于晉。



 興以太子泓錄尚書事。



 慕容超司徒、北地王鐘,右僕射、濟陽王嶷,高都公始,皆來奔。



 華山郡地涌沸,廣袤百餘步,燒生物皆熟,歷五月乃止。



 赫連勃勃殺高平公沒奕
 於,收其眾以叛。



 先是,魏主拓跋珪送馬千匹,求婚于興,興許之。以魏別立后,遂絕婚,故有柴壁之戰。至是,復與魏通和,魏放狄伯支、姚伯禽、唐小方、姚良國、康宦還長安,皆復其爵位。



 時禿髮傉檀、沮渠蒙遜迭相攻擊,傉檀遂東招河州刺史西羌彭奚念,奚念阻河以叛。



 蜀譙縱遣使稱籓,請桓謙,欲令順流東伐劉裕。興以問謙,謙請行,遂許之。



 使中軍姚弼、後軍斂成、鎮遠乞伏乾歸等率步騎三萬伐傉檀,左僕射齊難等率騎二萬討勃勃。吏部尚書尹昭諫曰:「傉檀恃遠,輕敢違逆,宜詔蒙遜及李玄盛,使自相攻擊。待其斃也,然後取之,此卞莊之舉也。」
 興不從。勃勃退保河曲。弼濟自金城,弼部將姜紀言於弼曰:「今王師聲討勃勃,傉檀猶豫,未為嚴防,請給輕騎五千,掩其城門,則山澤之人皆為吾有,孤城獨立,坐可剋也。」弼不從,進拔昌松,長驅至姑臧。傉檀嬰城固守,出其兵擊弼,弼敗,退據西苑。興又遣衛大將軍姚顯率騎二萬,為諸軍節度。至高平,聞弼敗績,兼道赴之,撫慰河外,率眾而還。傉檀遣使人徐宿詣興謝罪。



 齊難為勃勃所擒。興遣平北姚沖、征虜狄伯支、輔國斂曼嵬、鎮東楊佛嵩率騎四萬討勃勃。沖次于嶺北,欲回師襲長安,伯支不從,乃止,懼其謀泄,遂鴆殺伯支。



 時王師伐譙縱,大
 敗之,縱遣使乞師於興。興遣平西姚賞、南梁州刺史王敏率眾二萬救之,王師引還。縱遣使拜師,仍貢其方物。興遣其兼司徒韋華持節策拜縱為大都督、相國、蜀王,加九錫,備物典策一如魏、晉故事,承制封拜悉如王者之儀。



 興自平涼如朝那,聞沖謀逆,以其弟中最少,雄武絕人,猶欲隱忍容之。斂成泣謂興曰:「沖凶險不仁,每侍左右,臣常寢不安席,願早為之所。」興曰:「沖何能為也!但輕害名將,吾欲明其罪於四海。」乃下書賜沖死,葬以庶人之禮。



 晉河間王子國璠、章武王子叔道來奔,興謂之曰:「劉裕匡復晉室,卿等何故來也」國璠等曰:「裕與不逞
 之徒削弱王室,宗門能自修立者莫不害之。是避之來,實非誠款,所以避死耳。」興嘉之,以國璠為建義將軍、揚州刺史,叔道為平南將軍、兗州刺史,賜以甲第。



 興如貳城,將討赫連勃勃,遣安遠姚詳及斂曼嵬、鎮軍彭白狼分督租運。諸軍未集而勃勃騎大至,興欲留步軍,輕如嵬營。眾咸惶懼,群臣固以為不可,興弗納。尚書郎韋宗希旨勸興行。蘭臺侍御史姜楞越次而進曰:「韋宗傾險不忠,沮敗國計,宜先腰斬以謝天下。脫車駕動軫,六軍駭懼,人無守志,取危之道也,宜遣單使以征詳等。」興默然。右僕射韋華等諫曰:「若車騎輕動,必不戰自潰,嵬營
 亦未必可至,惟陛下圖之。」興乃遣左將軍姚文宗率禁兵距戰,中壘齊莫統氐兵以繼之。文宗與莫皆勇果兼人,以死力戰,勃勃乃退。留禁兵五千配姚詳守貳城,興還長安。



 譙縱遣其侍中譙良、太常楊軌朝於興,請大舉以寇江東。遣其荊州刺史桓謙、梁州刺史譙道福率眾二萬東寇江陵。興乃遣前將軍茍林率騎會之。謙屯枝江,林屯江津。謙,江左貴族,部曲遍於荊、楚,晉之將士皆有叛心。荊州刺史劉道規大懼,嬰城固守。雍州刺史魯宗之率襄陽之眾救之,道規乃留宗之守江陵,率軍逆戰。謙等舟師大盛,兼列步騎以待之。大戰枝江,謙敗績,
 乘輕舸奔就茍林,晉人獲而斬之。茍林懼而引歸。



 興以國用不足,增關津之稅,鹽竹山木皆有賦焉。群臣咸諫,以為天殖品物以養群生,王者子育萬邦,不宜節約以奪其利。興曰:「能踰關梁通利於山水者,皆豪富之家。吾損有餘以裨不足,有何不可!」乃遂行之。



 興從朝門游於文武苑,及昏而還,將自平朔門入。前驅既至,城門校尉王滿聰被甲持杖,閉門距之,曰:「今已昏闇,姦良不辨,有死而已,門不可開。」興乃迴從朝門而入。旦而召滿聰,進位二等。



 乞伏乾歸以眾叛,攻陷金城,執太守任蘭。蘭厲色責乾歸以背恩違義,乾歸怒而囚之,蘭遂不食而死。



 赫連勃勃遣其將胡金纂將萬餘騎攻平涼。興如貳城,因救平涼,纂眾大潰,生擒纂。勃勃遣兄子提攻陷定陽,執北中郎將姚廣都。興將曹熾、曹雲、王肆佛等各將數千戶避勃勃內徙,興處佛于湟山澤,熾、雲于陳倉。勃勃寇隴右,攻白崖堡,破之,遂趣清水。略陽太守姚壽都委守奔秦州,勃勃又收其眾而歸。興自安定追之,至壽渠川,不及而還。



 初,天水人姜紀,呂氏之叛臣,阿謅姦詐,好問人之親戚。興子弼有寵於興,紀遂傾心附之。弼時為雍州刺史,鎮安定,與密謀還朝,令傾心事常山公顯,樹黨左右。至是,興以弼為尚書令、侍中、大將軍。既居將相,
 虛襟引納,收結朝士,勢傾東宮,遂有奪嫡之謀矣。



 興以勃勃、乾歸作亂西北,傉檀、蒙遜擅兵河右,疇咨將帥之臣,欲鎮撫二方。隴東太守郭播言於興曰:「嶺北二州鎮戶皆數萬,若得文武之才以綏撫之,足以靖塞姦略。」興曰;「吾每思得廉頗、李牧鎮撫四方,使便宜行事。然任非其人,恒致負敗。卿試舉之。」播曰:「清潔善撫邊,則平陸子王元始;雄武多奇略,則建威王煥;賞罰必行,臨敵不顧,則奮武彭蠔。」興曰:「蠔令行禁止則有之,非綏邊之才也。始、煥年少,吾未知其為人。」播曰:「廣平公弼才兼文武,宜鎮督一方,願陛下遠鑒前車,近悟後轍。」興不從,以其太
 常索棱為太尉,領隴西內史,綏誘乾歸。政績既美,乾歸感而歸之。太史令任猗言於興曰:「白氣出於北方,東西竟天五百里,當有破軍流血。」乞伏乾歸遣使送所掠守宰,謝罪請降。興以勃勃之難,權宜許之,假乾歸及其子熾磐官爵。



 姚詳時鎮杏城,為赫連勃勃所逼,糧盡委守,南奔大蘇。勃勃耍之,眾散,為勃勃所執。時遣衛大將軍顯迎詳,詳敗,遂屯杏城,因令顯都督安定嶺北二鎮事。



 潁川太守姚平都自許昌來朝,言於興曰:「劉裕敢懷奸計,屯聚芍陂,有擾邊之志,宜遣燒之,以散其眾謀。」興曰:「裕之輕弱,安敢窺吾疆埸!茍有奸心,其在子孫乎!」召其
 尚書楊佛嵩謂之曰:「吳兒不自知,乃有非分之意。待至孟冬,當遣卿率精騎三萬焚其積聚。」嵩曰:「陛下若任臣以此役者,當從肥口濟淮,直趣壽春,舉大眾以屯城,縱輕騎以掠野,使淮南蕭條,兵粟俱了,足令吳兒俯仰回惶,神爽飛越。」興大悅。



 時西胡梁國兒於平涼作壽冢,每將妻妾入冢飲宴,酒酣,升靈床而歌。時人或譏之,國兒不以為意。前後征伐,屢有大功,興以為鎮北將軍,封平輿男,年八十餘乃死。



 時客星入東井,所在地震,前後一百五十六。興公卿抗表請罪,興曰:「災譴之來,咎在元首;近代或歸罪三公,甚無謂也。公等其悉冠履復位。」



 仇池
 公楊盛叛,侵擾祁山。遣建威趙琨率騎五千為前鋒,立節楊伯壽統步卒繼之,前將軍姚恢、左將軍姚文宗入自鷲陜,鎮西、秦州刺史姚嵩入羊頭陜,右衛胡翼度從陰密出自汧城,討盛。興將輕騎五千,自雍赴之,與諸將軍會于隴口。天水太守王松忿言於嵩曰:「先皇神略無方,威武冠世,冠軍徐洛生猛毅兼人,佐命英輔,再入仇池,無功而還。非楊盛智勇能全,直是地勢然也。今以趙琨之眾,使君之威,準之先朝,實未見成功。使君具悉形便,何不表聞?」嵩不從。盛率眾與琨相持,伯壽畏懦弗進,琨眾寡不敵,為盛所敗,興斬伯壽而還。嵩乃具陳松忿之
 言,興善之。



 乾歸為其下人所殺,子熾磐新立,群下咸勸興取之。興曰:「乾歸先已返善,吾方當懷撫,因喪伐之,非朕本志也。」



 以楊佛嵩都督嶺北討虜諸軍事、安遠將軍、雍州刺史,率領北見兵以討赫連勃勃。嵩發數日,興謂群臣曰:「佛嵩驍勇果銳,每臨敵對寇,不可制抑,吾常節之,配兵不過五千。今眾旅既多,遇賊必敗。今去已遠,追之無及,吾深憂之。」其下咸以為不然。佛嵩果為勃勃所執,絕亢而死。



 興立昭儀齊氏為皇后。又下書以其故丞相姚緒、太宰姚碩德、太傅姚旻、大司馬姚崇、司徒尹緯等二十四人配饗於萇廟。興以大臣屢喪,令所司更詳
 臨赴之制。所司白興,依故事東堂發哀。興不從,每大臣死,皆親臨之。



 姚文宗有寵於姚泓,姚弼深疾之,誣文宗有怨言,以侍御史廉桃生為證。興怒,賜文宗死。是後群臣累足,莫敢言弼之短。



 時貳縣羌叛興,興遣後將軍斂成、鎮軍彭白狼、北中郎將姚洛都討之。斂成為羌所敗,甚懼,詣趙興太守姚穆歸罪。穆欲送殺之,成怒,奔赫連勃勃。



 興遣姚紹與姚弼率禁衛諸軍鎮撫嶺北。遼東侯彌姐亭地率其部人南居陰密,劫掠百姓。弼收亭地送之,殺其眾七百餘人,徙二千餘戶於鄭城。



 弼寵愛方隆,所欲施行,無不信納。乃以嬖人尹沖為給事黃門侍郎,
 唐盛為治書侍御史,左右機耍,皆其黨人,漸欲廣樹爪牙,彌縫其闕。右僕射梁喜、侍中任謙、京兆尹尹昭承間言於興曰:「父子之際,人罕得而言。然君臣亦猶父子,臣等理不容默。並后匹嫡,未始不傾國亂家。廣平公弼奸凶無狀,潛有陵奪之志,陛下寵之不道,假其威權,傾險無賴之徙,莫不鱗湊其側。市巷諷議,皆言陛下欲有廢立之志。誠如此者,臣等有死而已,不敢奉詔。」興曰:「安有此乎!」昭等曰:「若無廢立之事,陛下愛弼,適所以禍之,願去其左右,減其威權。非但弼有太山之安,宗廟社稷亦有磐石之固矣。」興默然。



 興寢疾,妖賊李弘反于貳原,貳
 原氐仇常起兵應弘。興輿疾討之,斬常,執弘而還,徙常部人五百餘戶于許昌。



 興疾篤,其太子泓屯兵于東華門,侍疾於諮議堂。姚弼潛謀為亂,招集數千人,被甲伏於其第。撫軍姚紹及侍中任謙、右僕射梁喜、冠軍姚贊、京兆尹尹昭、輔國斂曼嵬並典禁兵,宿衛於內。姚裕遣使告姚懿于蒲阪,并密信諸籓,論弼逆狀。懿流涕以告將士曰:「上今寢疾,臣子所宜冠履不整。而廣平公弼擁兵私第,不以忠於儲宮,正是孤徇義亡身之日。諸君皆忠烈之士,亦當同孤徇斯舉也。」將士無不奮怒攘袂曰:「惟殿下所為,死生不敢貳。」於是盡赦囚徙,散布帛數萬
 匹以賜其將士,建牙誓眾,將赴長安。鎮東、豫州牧姚洸起兵洛陽,平西姚諶起兵於雍,將以赴泓之難。興疾瘳,朝其群臣,征虜劉羌泣謂興曰:「陛下寢疾數旬,奈何忽有斯事!」興曰:「朕過庭無訓,使諸子不穆,愧于四海。卿等各陳所懷,以安社稷。」尹昭曰:「廣平公弼恃寵不虔,阻兵懷貳,自宜置之刑書,以明典憲。陛下若含忍未便加法者,且可削奪威權,使散居籓國,以紓窺窬之禍,全天性之恩。」興謂梁喜曰:「卿以為何如?」喜曰:「臣之愚見,如昭所陳。」興以弼才兼文武,未忍致法,免其尚書令,以將軍、公就第。懿等聞興疾瘳,各罷兵還鎮。懿、恢及弟諶等皆抗
 表罪弼,請致之刑法,興弗許。



 時魏遣使聘于興,且請婚。會平陽太守姚成都來朝,興謂之曰:「卿久處東籓,與魏鄰接,應悉彼事形。今來求婚,吾已許之,終能分災共患,遠相接援以不?」成都曰:「魏自柴壁剋捷已來,戎甲未曾損失,士馬桓桓,師旅充盛。今修和親,兼婚姻之好,豈但分災共患而已,實亦永安之福也。」興大悅,遣其吏部郎嚴康報聘,并致方物。



 時姚懿、姚洸、姚宣、姚諶來朝,使姚裕言於興曰:「懿等今悉在外,欲有所陳。」興曰:「汝等正欲道弼事耳,吾已知之。」裕曰:「弼茍有可論,陛下所宜垂聽。若懿等言違大義,便當肆之刑辟,奈何距之!」於是引見
 諮議堂。宣流涕曰:「先帝以大聖起基,陛下以神武定業,方隆七百之祚,為萬世之美,安可使弼謀傾社稷。宜委之有司,肅明刑憲。臣等敢以死請。」興曰:「吾自處之,非汝等所憂。」先是,大司農竇溫、司徙左長史王弼皆有密表,勸興廢立。興雖不從,亦不以為責。撫軍東曹屬姜虯上疏曰:「廣平公弼懷姦積年,謀禍有歲,傾諂群豎為之畫足,釁成逆著,取嗤戎裔。文王之化,刑于寡妻;聖朝之亂,起自愛子。今雖欲含忍其瑕,掩蔽其罪,而逆黨猶繁,扇惑不已,弼之亂心其可革耶!宜斥散凶徒,以絕禍始。」興以虯表示梁喜曰:「天下之人莫不以吾兒為口實,將何
 以處之?」喜曰:「信如虯言,陛下宜早裁決。」興默然。



 太子詹事王周亦虛襟引士,樹黨東宮,弼惡之,每規陷害周。周抗志確然,不為之屈。興嘉其守正,以周為中書監。



 興如三原,顧謂群臣曰:「古人有言,關東出相,關西出將,三秦饒俊異,汝潁多奇士。吾應天明命,跨據中原,自流沙已東,淮、漢已北,未嘗不傾己招求,冀匡不逮。然明不照下,弗感懸魚。至於智效一官,行著一善,吾歷級而進之,不使有後門之歎。卿等宜明揚仄陋,助吾舉之。」梁喜對曰:「奉旨求賢,弗曾休倦,未見儒亮大才王佐之器,可謂世之乏賢。」興曰:「自古霸王之起也,莫不將則韓、吳,相兼蕭、
 鄧,終不採將於往賢,求相於後哲。卿自識拔不明,求之不至,奈何厚誣四海乎!」群臣咸悅。



 晉荊州刺史司馬休之據江陵,雍州刺史魯宗之據襄陽,與劉裕相攻,遣使求援。興遣姚成王、司馬國璠率騎八千赴之。



 弼恨姚宣之毀己,遂譖宣於興。會宣司馬權丕至長安,興責丕以無匡輔之益,將戮之。丕性傾巧,因誣宣罪狀。興大怒,遂收宣于杏城,下獄,而使弼將三萬人鎮秦州。尹昭言於興曰:「廣平公與皇太子不平,握彊兵于外,陛下一旦不諱,恐社稷必危。小不忍以致大亂者,陛下之謂也。」興弗納。赫連勃勃攻杏城,興又遣弼救之,至冠泉而杏城陷。
 興如北地,弼次于三樹,遣弼及斂曼嵬向新平,興還長安。



 姚成王至于南陽,司馬休之等為劉裕所敗,引歸。休之、宗之等遂與譙王文思,新蔡王道賜,寧朔將軍、梁州刺史馬敬,輔國將軍、竟陵太守魯軌,寧朔將軍、南陽太守魯範奔于興。



 勃勃遣其將赫連建率眾寇貳縣,數千騎入平涼。姚恢與建戰于五井,平涼太守姚興都為建所獲,遂入新平。姚弼討之,戰於龍尾堡,大破之,擒建,送于長安。初,勃勃攻彭雙方于石堡,方力戰距守,積年不能剋。至是,聞建敗,引歸。



 休之等至長安,興謂之曰:「劉裕崇奉晉帝,豈便有闕乎?」休之曰:「臣前下都,瑯邪王德文
 泣謂臣曰:『劉裕供御主上,克薄奇深。』以事勢推之,社稷之憂方未可測。」興將以休之為荊州刺史,任以東南之事。休之固辭,請與魯宗之等擾動襄陽、淮、漢。乃以休之為鎮南將軍、揚州刺史,宗之等並有拜授。休之將行,侍御史唐盛言於興曰:「符命所記,司馬氏應復河、洛。休之既得濯鱗南翔,恐非復池中之物,可以崇禮,不宜放之。」興曰:「司馬氏脫如所記,留之適足為患。」遂遣之。



 揚武、安鄉侯康宦驅略白鹿原氏胡數百家奔上洛,太守宋林距之。商洛人黃金等起義兵以掎宦,宦乃率眾歸罪。興赦之,復其爵位。



 時白虹貫日,有術人言於興曰:「將有不
 祥之事,終當自消。」時興藥動,姚弼稱疾不朝,集兵於第。興聞之怒甚,收其黨殿中侍御史唐盛、孫玄等殺之。泓言於興曰:「臣誠不肖,不能訓諧於弟,致弼構造是非,仰慚天日,陛下若以臣為社稷之憂,除臣而國寧,亦家之福也。若垂天性之恩,不忍加臣刑戮者,乞聽臣守籓。」興慘然改容,召姚贊、梁喜、尹昭、斂曼嵬於諮議堂,密謀收弼。時姚紹屯兵雍城,馳遣告之,數日不決。弼黨兇懼。興慮其為變,乃收弼,囚之中曹,窮責黨與,將殺之。泓流涕固請之,乃止。興謂梁喜曰:「泓天心平和,性少猜忌,必能容養群賢,保全吾子。」於是皆赦弼黨。



 靈臺令張泉又言
 於興曰:「熒惑入東井,旬紀而返,未餘月,復來守心。王者惡之,宜修仁虛己,以答天譴。」興納之。



 正旦,興朝群臣于太極前殿,沙門賀僧慟泣不能自勝,眾咸怪焉。賀僧者,莫知其所從來也,言事皆有效驗,興甚神禮之,常與隱士數人預於宴會。



 興如華陰,以泓監國,入居西宮。因疾篤,還長安。泓欲出迎,其宮臣曰:「今主上疾篤,奸臣在側,廣平公每希顗非常,變故難測。今殿下若出,進則不得見主上,退則有弼等之禍,安所歸乎!自宜深抑情禮,以寧宗社。」泓從之,乃拜迎於黃龍門樽下。弼黨見興升輿,咸懷危懼。尹沖等先謀欲因泓出迎害之,尚書姚沙彌
 曰:「若太子有備,不來迎侍,當奉乘輿直趣公第。宿衛者聞上在此,自當來奔,誰與太子守乎!吾等以廣平公之故,陷身逆節。今以乘輿南幸,自當是杖義之理,匪但救廣平之禍,足可以申雪前愆。」沖等不從,欲隨興入殿中作亂,復未知興之存亡,疑而不發。興命泓錄尚書事,使姚紹、胡翼度典兵禁中,防制內外,遣斂曼嵬收弼第中甲杖,內之武庫。



 興疾轉篤,興妹偽南安長公主問疾,不應。興少子耕兒出告其兄愔曰:「上已崩矣,宜速決計。」於是愔與其屬率甲士攻端門,殿中上將軍斂曼嵬勒兵距戰,右衛胡翼度率禁兵閉四門。愔等遣壯士登門,緣
 屋而入,及于馬道。泓時侍疾于諮議堂,遣斂曼嵬率殿中兵登武庫距戰,太子右衛率姚和都率東宮兵入屯馬道南。愔等既不得進,遂燒端門。興力疾臨前殿,賜弼死。禁兵見興,喜躍,貫甲赴賊,賊眾駭擾。和都勒東宮兵自後擊之,愔等奔潰,逃于驪山,愔黨呂隆奔雍,尹沖等奔於京師。興引紹及贊、梁喜、尹昭、斂曼嵬入內寢,受遺輔政。義熙十二年,興死,時年五十一,在位二十二年。偽謚文桓皇帝,廟號高祖,墓曰偶陵。



 尹緯,字景亮,天水人也。少有大志,不營產業。身長八尺,
 腰帶十圍,魁梧有爽氣。每覽書傳至宰相立勳之際,常輟書而歎。苻堅以尹赤之降姚襄,諸尹皆禁錮不仕。緯晚乃為吏部令史,風志豪邁,郎皆憚之。堅末年,祅星見於東井,緯知堅將滅,喜甚,向天再拜,既而流涕長歎。友人略陽桓識怪而問之,緯曰:「天時如此,正是霸王龍飛之秋,吾徒杖策之日。然知己難遭,恐不得展吾才志,是以欣懼交懷。」



 及姚萇奔馬牧,緯與尹詳、龐演等扇動群豪,推萇為盟主,遂為佐命元功。萇既敗苻堅,遣緯說堅,求禪代之事。堅問緯曰:「卿於朕何官?」緯曰:「尚書令史。」堅歎曰:「宰相之才也,王景略之儔。而朕不知卿,亡也不亦
 宜乎!」



 緯性剛簡清亮,慕張子布之為人。馮翊段鏗性傾巧,萇愛其博識,引為侍中。緯固諫以為不可,萇不從。緯屢眾中辱鏗,鏗心不平之。萇聞而謂緯曰:「卿性不好學,何為憎學者?」緯曰:「臣不憎學,憎鏗不正耳。」萇因曰:「卿好不自知,每比蕭何,真何如也?」緯曰:「漢祖與蕭何俱起布衣,是以相貴。陛下起貴中,是以賤臣。」萇曰:「卿實不及,胡為不也?」緯曰:「陛下何如漢祖?」萇曰:「朕實不如漢祖,卿遠蕭何,故不如甚也。」緯曰:「漢祖所以勝陛下者,以能遠段鏗之徒故耳。」萇默然,乃出鏗為北地太守。



 萇死,緯與姚興滅苻登,成興之業,皆緯之力也。歷輔國將軍、司隸校
 尉、尚書左右僕射、清河侯。



 緯友人隴西牛壽率漢中流人歸興,謂緯曰:「足下平生自謂:『時明也,才足以立功立事;道消也,則追二疏、朱雲,發其狂直,不能如胡廣之徒洿隆隨俗。』今遇其時矣,正是垂名竹素之日,可不勉歟!」緯曰:「吾之所庶幾如是,但未能委宰衡於夷吾,識韓信於羈旅,以斯為愧耳。立功立事,竊謂未負昔言。」興聞而謂緯曰:「君之與壽言也,何其誕哉!立功立事,自謂何如古人?」緯曰:「臣實未愧古人。何則?遇時來之運,則輔翼太祖,建八百之基。及陛下龍飛之始,翦滅苻登,盪清秦、雍,生極端右,死饗廟庭,古之君子,正當爾耳。」興大悅。及死,
 興甚悼之,贈司徙,謚曰忠成侯。



\end{pinyinscope}