\article{志第十五}

\begin{pinyinscope}

 輿服



 史臣曰:昔者乘雲效駕,卷領垂衣,則黃帝皁衣纁裳,放勛彤車白馬,葉三微之序,舍寅丑之建,玄戈玉刃,作會相暉。若乃參旗分景,帝車含曜,又所以營衛南宮,增華北極。《月令》季夏之月,「命婦官染彩」,赬丹班次,各有品章矣。高旗有日月之象,式視有威儀之選,衣兼鞙珮,衡載鳴和,是以閑邪屏棄,不可入也。若乃正名百物,補緝四
 維,疏懷山之水,靜傾天之害,功尤彰者飾彌煥,德愈盛者服彌尊,莫不質良,用成其美。《書》曰:「明試以功,車服以庸。」「《禮記》曰:「鸞車,有虞氏之路也。鉤車,夏后氏之路也。大路,殷路也。乘路,周路也。」而韍火山龍,以通其意。前史以為,聖人見鳥獸容貌,草木英華,始創衣冠,而玄黃殊采;見秋蓬孤轉,杓觿旁建,乃作輿輪,而方圓異則。遇物成象,觸類興端。周因於殷,其來已舊。成王之會,壇垂陰羽,五方之盛,有八十物者焉。宗馬鳥旌,奚往不格,殷公、曹叔,此焉低首。《周禮》,巾車氏建大赤以朝,大白以戎。雅制弘多,式遵遺範,賓入異憲,師行殊則,是以有嚴有翼,用
 光其武,鉤膺鞗革,乃暢其文。六服之冕,五時之路,王之常制,各有等差。逮禮業彫訛,人情馳爽,諸侯征伐,憲度淪亡,一紫亂於齊飾,長纓混於鄒玩。孔子曰:「君子其學也博,其服也鄉。」若乃豪傑不經,庶人干典,彯鷸冠於鄭伯之門,躡珠履於春申之第。及秦皇并國,攬其餘軌,豐貂東至,獬豸南來,又有玄旗皁旒之制,旄頭罕車之飾,寫九王之廷於咸陽北阪,車輿之綵,各樹其文,所謂秦人大備,而陳戰國之後車者也。及凝脂布網,經書咸燼,削滅三代,以金根為帝軫,除棄六冕,以袀玄為祭服。高祖入關,既因秦制。世宗挺英雄之略,總文景之資,揚霓
 拂翳,皮軒記鼓,橫汾河而祠后土,登甘泉而祭昊天,奉常獻儀,謂之大駕,車千乘而騎萬匹。至於成帝,以幸姬趙飛燕置屬車間豹尾中,又楊雄所謂彏天狼之威弧,張曜日之靈旄,駢羅列布,霧集雲合者也。於後王氏擅朝,武車常軔,赤眉之亂,文物無遺。建武十三年,吳漢平蜀,始送葆車輿輦,充庭之飾,漸以周備。明帝採《周官》、《禮記》,更服袞章,天子冠通天而佩玉璽。魏明以黼黻之美,有疑於僭,於是隨章儐略,而捐者半焉。高堂隆奏曰:「改正朔、殊徽號者,帝王所以神明其政,變民耳目也。」帝從其議,改青龍五年為景初元年,服色尚黃,從地正也。世
 祖武皇帝接天人之貺,開典午之基,受終之禮,皆如唐虞故事。晉氏金行,而服色尚赤,豈有司失其傳歟!



 玉、金、象、革、木等路,是為五路,並天子之法車,皆朱班漆輪,畫為𣝛文。三十幅,法月之數;重轂,貳轄,以赤油,廣八寸,長三尺,注地,繫兩軸頭,謂之飛軨。金薄繆龍繞之為輿倚較,較重,為文獸伏軾,龍首銜軛,左右吉陽筩,鸞雀立衡,𣝛文畫轅及轓。青蓋,黃為裏,謂之黃屋。金華施尞末,尞二十八以象宿。兩箱之後,皆玳瑁為鵾翅,加以金銀雕飾,故世人亦謂之金鵾車。斜注旂旗於車之左,又加棨戟於車之右,皆橐而施之。棨戟韜以黻繡,上為亞
 字,繫大蛙蟆幡。軛長丈餘。於戟之杪,以犛牛尾,大如斗,置左騑馬軛上,是為左纛。轅皆曲向上,取《禮緯》「山車垂句」之義,言不揉而能自曲。



 玉、金、象三路,各以其物飾車,因以為名。革者漆革,木者漆木。其制,玉路最尊,建太常,十有二旒,九仞委地,畫日月升龍,以祀天。金路建大旂,九旒,以會萬國之賓,亦以賜上公及王子母弟。象路建大赤,通赤無畫,所以視朝,亦以賜諸侯。革路建大白,以即戎兵事,亦以賜四鎮諸侯。木路建大麾,以田獵,其麾色黑,亦以賜籓國。玉路駕六黑馬,餘四路皆駕四馬,馬並以黃金為文髦,插以翟
 尾。象鑣而鏤錫,錫在馬面,所謂當顱者也。金[B080]而方釳,金[B080]謂以金[B080]為文。釳以鐵為之,其大三寸,中央兩頭高,如山形,貫中以翟尾而結著之也。繁纓赤罽易茸,金就十有二。繁纓,馬飾纓,在馬膺前,如索裙。五路皆有錫鸞之飾,和鈴之響,鉤膺玉瓖,鉤膺,即繁纓也。瓖,馬帶玦名也。龍輈華轙,輈,車轅也,頭為龍象。轙,謂車衡上環受鸞者也。朱幩。幩,飾也,人君以朱纏鑣扇汗,以為飾也。法駕行則五路各有所主,不懼出;臨軒大會則陳乘輿車輦旌鼓於其殿庭。



 車,坐乘者謂之安車,倚乘者謂之立車,亦謂之高車。案《周禮》,惟王后有安車也,王亦無之。自漢以來制乘輿,乃有之。有青立車、青安車、赤立車、赤安車、黃立車、黃安車、白立車、白安車、黑立車、黑安車,合十乘,名為五時車,俗
 謂之五帝車。天子所御則駕六,其餘並駕四。建旂十二,各如車色。立車則正豎其旂,安車則邪注。駕馬,馬亦各隨五時之色,白馬則朱其尾,左右騑驂,金[B080]鏤錫,黃屋左纛,如金根之制,行則從後。五牛旗,平吳後所造,以五牛建旗,車設五牛,青赤在左,黃在中,白黑在右。豎旗於牛背,行則使人輿之。牛之為義,蓋取其負重致遠而安穩也。旗常纏不舒,所謂德車結旌也。天子親戎則舒,謂武車綏旌也。



 金根車,駕四馬,不建旗幟,其上如畫輪車,下猶金根之飾。



 耕根車,駕四馬,建赤旗,十有二旒,天子親耕所乘者也。一名芝車,一名三蓋車。置耒耜於軾上。魏景初元年,
 改正朔,易服色,色尚黃,牲用白,戎事乘黑首白馬,建大赤之旂,朝會則建大白,行殷之時也。泰始二年,有司奏:「宜如有虞遵唐故事,皆用前代正朔服色,其金根、耕根車,並以建赤旗。」帝從之。



 輦,案自漢以來為人君之乘,魏晉御小出即乘之。



 戎車,駕四馬,天子親戎所乘者也。載金鼓、羽旂、幢翳,置弩於軾上,其建矛麾悉斜注。



 獵車,駕四馬,天子校獵所乘也。重輞漫輪,繆龍繞之。一名闒戟車,一名蹋豬車。魏文帝改名蹋獸車。《記》云「國君不乘奇車」,奇車亦獵車也。古天子獵則乘木輅,後人代以獵車也。



 遊車,九乘,駕四,先驅之乘是也。



 雲罕車,駕四。



 皮軒車,駕四,以獸皮為軒。



 鸞旗車,駕四,先輅所載也。鸞旗者,謂析羽旄而編之,列系幢傍也。



 建華車,駕四,凡二乘,行則分居左右。



 輕車,駕二,古之戰車也。前後二十乘,分居左右。輿輪洞朱,不巾不蓋,建矛戟麾幢,置弩箙於軾上。大駕法駕出,射聲校尉、司馬、吏士、戰士載,以次屬車。



 司南車,一名指南車,駕四馬,其下制如樓,三級;四角金
 龍銜羽葆;刻木為仙人,衣羽衣,立車上,車雖回運而手常南指。大駕出行,為先啟之乘。



 記里鼓車,駕四,形制如司南,其中有木人執棰向鼓,行一里則打一棰。



 羊車,一名輦車,其上如軺,伏兔箱,漆畫輪軛。武帝時,護軍羊琇輒乘羊車,司隸劉毅糾劾其罪。



 畫輪車,駕牛,以彩漆畫輪轂,故名曰畫輪車。上起四夾杖,左右開四望,綠油幢,朱絲絡,青交路,其上形制事事如輦,其下猶如犢車耳。古之貴者不乘牛車,漢武帝推恩之末,諸侯寡弱,貧者至乘牛車,其後稍見貴之。自靈
 獻以來,天子至士遂以為常乘,至尊出朝堂舉哀乘之。



 屬車,一曰副車,一曰貳車,一曰左車。漢因秦制,大駕屬車八十一乘,行則中央左右分為行。



 法駕屬車三十六乘。最後車懸豹尾,豹尾以前比之省中。屬車皆皂蓋朱里雲。



 御衣車、御書車、御軺車、御藥車,皆駕牛。



 陽遂四望繐窗皁輪小形車,駕牛。



 象車,漢鹵簿最在前。武帝太康中平吳後,南越獻馴象,詔作大車駕之,以載黃門鼓吹數十人,使越人騎之。元正大會,駕象入庭。



 中朝大駕鹵簿



 先象車,鼓吹一部,十三人,中道。



 次靜室令,駕一,中道。式道候二人,駕一,分左右也。



 次洛陽尉二人,騎,分左右。



 次洛陽亭長九人,赤車,駕一,分三道,各吹正二人引。



 次洛陽令,皁車,駕一,中道。次河南中部掾,中道。河橋掾在左,功曹史在右,並駕一。



 次河南尹,駕駟,戟吏六人。



 次河南主簿,駕一,中道。



 次河南主記,駕一,中道。



 次司隸部河南從事,中道。都部從事居左,別駕從事居右,並駕一。



 次司隸校尉,駕三,戟吏八人。



 次司隸主簿,駕一,中道。



 次司隸主記,駕一,中道。



 次廷尉明法掾,中道。五官掾居左,功曹史居右,並駕一。



 次廷尉卿,駕駟,戟吏六人。



 次廷尉主簿、主記,並駕一,在左。太僕引從如廷尉,在中。宗正引從如廷尉,在右。



 次太常,駕駟,中道,戟吏六人。太常外部掾居左,五官掾、
 功曹吏居右,並駕一。



 次光祿引從,中道。太常主簿、主記居左,衛尉引從居右,並駕一。



 次太尉外督令史,駕一,中道。



 次西東賊倉戶等曹屬,並駕一,引從。



 次太尉,駕駟,中道。太尉主簿、舍人各一人,祭酒二人,並駕一,在左。



 次司徒引從,駕駟,中道。



 次司空引從,駕駟,中道。三公騎令史戟各八人,鼓吹各一部,七人。



 次中護軍,中道,駕駟。鹵簿左右各二行,戟楯在外,弓矢在內,鼓吹一部,七人。



 次步兵校尉在左,長水校尉在右,並駕一。各鹵簿左右二行,戟楯在外,刀楯在內,鼓吹各一部,七人。



 次射聲校尉在左,翊軍校尉在右,並駕一。各鹵簿左右各二行,戟楯在外,刀楯在內,鼓吹各一部,七人。



 次驍騎將軍在左,游擊將軍在右,並駕一。皆鹵簿左右引各二行,戟楯在外,刀楯在內,鼓吹各一部,七人。騎隊,五在左,五在右,隊各五十匹,命中督二人分領左右。各有戟吏二人,麾幢獨揭,鼓在隊前。



 次左將軍在左,前將軍在右,並駕一。皆鹵簿左右各二行,戟楯盾在外,刀楯在內,鼓吹各一部,七人。



 次黃門麾騎,中道。



 次黃門前部鼓吹,左右各一部,十三人,駕駟。八校尉佐仗,左右各四行,外大戟楯,次九尺楯,次弓矢,次弩,並熊渠、佽飛督領之。



 次司南車,駕駟,中道。護駕御史,騎,夾左右。



 次謁者僕射,駕駟,中道。



 次御史中丞,駕一,中道。



 次武賁中郎將,騎,中道。



 次九游車,中道,武剛車夾左右,並駕駟。



 次雲罕車,駕駟,中道。



 次闒戟車,駕駟,中道,長戟邪偃向後。



 次皮軒車,駕駟,中道。



 次鸞旗車,中道,建華車分左右,並駕駟。



 次護駕尚書郎三人,都官郎中道,駕部在左,中兵在右,並騎。又有護駕尚書一人,騎,督攝前後無常。



 次相風,中道。



 次司馬督,在前,中道。左右各司馬史三人引仗,左右各
 六行,外大戟楯二行。



 次九尺楯,次刀楯。



 次弓矢,次弩。



 次五時車,左右有遮列騎。



 次典兵中郎,中道,督攝前卻無常。左殿中御史,右殿中監,並騎。



 次高蓋,中道,左罼,右罕。



 次御史,中道,左右節郎各四人。



 次華蓋,中道。



 次殿中司馬,中道。殿中都尉在左,殿中校尉在右,左右
 各四行。細楯一行在弩內,又殿中司馬一行,殿中都尉一行,殿中校尉一行。



 次手罡鼓,中道。



 次金根車,駕六馬,中道。太僕卿御,大將軍參乘。左右又各增三行,為九行。司馬史九人,引大戟楯二行,九尺楯一行,刀楯一行,由基一行,細弩一行,跡禽一行,椎斧一行,力人刀楯一行。連細楯,殿中司馬,殿中都尉,殿中校尉,為左右各十二行。
 金根車建青旂十二。左將軍騎在左,右將軍騎在右,殿中將軍持鑿腦斧夾車,車後衣書主職步從,六行,合左右三十二行。



 次曲華蓋,中道。侍中、散騎常侍、黃門侍郎並騎,分左右。



 次黃鉞車,駕一,在左,御麾騎在右。



 次相風,中道。



 次中書監騎左,秘書監騎右。



 次殿中御史騎左,殿中監騎右。



 次五牛旗,赤青在左,黃在中,白黑在右。



 次大輦,中道。太官令丞在左,太醫令丞在右。



 次金根車,駕駟,不建旗。



 次青立車,次青安車,次赤立車,次赤安車,次黃立車,次黃安車,次白立車,次白安車,次黑立車,次黑安車,合十乘,並駕駟。建旗十二,如車色。立車正豎旗,安東邪拖之。



 次蹋豬車,駕駟,中道,無旗。



 次耕根車,駕駟,中道,赤旗十二,熊渠督左,佽飛督右。



 次御軺車,次御四望車,
 次御衣車,次御書車,次御藥車,並駕牛,中道。



 次尚書令在左,尚書僕射在右,又尚書郎六人,分次左右,並駕。又治書侍御史二人,分左右,又侍御史二人,分次左右,又蘭臺令史分次左右,並騎。



 次豹尾車,駕一。自豹尾車後而鹵簿盡矣。但以神弩二十張夾道,至後部鼓吹。其五張神弩置一將,左右各二將。



 次輕車二十乘,左右分駕。



 次流蘇馬六十匹。



 次金鉞車,駕三,中道。左右護駕尚書郎并令史,並騎,各一人。



 次金鉦車,駕三,中道。左右護駕侍御史并令史等,並騎,各一人。



 次黃門後部鼓吹,左右各十三人。



 次戟鼓車,駕牛,二乘,分左右。次左大鴻臚外部掾,右五官掾、功曹史,並駕。



 次大鴻臚,駕駟,鉞吏六人。



 次大司農引從,中道,左大鴻臚主簿、主記,右少府引從。



 次三卿,並騎,吏四人,鈴下二人,執馬鞭辟車六人,執方扇羽林十人,朱衣。



 次領軍將軍,中道。鹵簿左右各二行,九尺楯在外,弓矢在內,鼓吹如護軍。



 次後軍將軍在左,右將軍在右,各鹵簿鼓吹如左軍、前軍。



 次越騎校尉在左,屯騎校尉在右,各鹵簿鼓吹如步兵、射聲。



 次領護驍騎、游軍校尉,皆騎,吏四人,乘馬夾道,都督兵曹各一人,乘馬在中。騎將軍四人,騎校、鞉角、金鼓、鈴下、
 信幡、軍校並駕一。功曹吏、主簿並騎從。散扇幢麾各一騎,鼓吹一部,七騎。



 次領護軍,加大車斧,五官掾騎從。



 次騎十隊,隊各五十匹。將一人,持幢一人,鞉一人,並騎在前,督戰伯長各一人,並騎在後,羽林騎督、幽州突騎督分領之。郎簿十隊,隊各五十人。絳袍將一人,騎、鞉各一人,在前,督戰伯長各一人,步,在後。騎皆持槊。



 次大戟一隊,九尺楯一隊,刀楯一隊,弓一隊,弩一隊,隊各五十人。黑褲褶將一人,騎校、鞉角各一人,步,在前,督戰伯長各一人,步,在後。金顏督將并領之。



 皇太子安車,駕三,左右騑。朱班輪,倚獸較,伏鹿軾。九旒,畫降龍。青蓋,金華蚤二十八枚。黑𣝛文畫轓,文輈,黃金塗五採。亦謂之鸞路。非法駕則乘畫輪車,上開四望,綠油幢,朱絲繩絡,兩箱裏飾以金錦,黃金塗五采。其副車三乘,形制如所乘,但不畫輪耳。



 王青蓋車,皇孫綠蓋車,並駕三,左右騑。



 雲母車,以雲母飾犢車。臣下不得乘,以賜王公耳。



 皁輪車,駕四牛,形制猶如犢車,但皁漆輪轂,上加青油幢,朱絲繩絡。諸王三公有勛德者特加之。位至公或四望、三望、夾望車。



 油幢車,駕牛,形制如皁輪,
 但不漆轂耳。王公大臣有勛德者特給之。



 通幰車,駕牛,猶如今犢車制,但舉其幰通覆車上也。諸王三公並乘之。



 諸公給朝車駕四、安車黑耳駕三各一乘,皁輪犢車各一乘。自祭酒掾屬以下及令史,皆皁零,辟朝服。其武官公又別給大車。



 特進及車騎將軍驃騎將軍以下諸大將軍不開府非持節都督者,給安車黑耳駕二,軺車施耳後戶一乘。



 三公、九卿、中二千石、二千石、河南尹、謁者僕射、郊廟明堂法出,皆大車立乘,駕駟。前後導從大車駕二,右騑。他
 出乘安車。其去位致仕告老,賜安車駟馬。



 郡縣公侯,安車駕二,右騑。皆朱班輪,倚鹿較,伏熊軾,黑輜,皁繒蓋。



 公旗旂八旒,侯七旒,卿五旒,皆畫降龍。



 中二千石、二千石,皆皁蓋,朱兩轓,銅五采,駕二。中二千石以上,右騑。千石、六百石,朱左轓。車轓長六尺,下屈廣八寸,上業廣尺二寸,九丈,十二初,後謙一寸,若月初生,示不敢自滿也。



 王公之世子攝命理國者,安車,駕三,旗旂七旒,其封侯之世子五旒。



 太康四年,制:「依漢故事,給九卿朝車駕四及安車各一乘。」八年,詔:「諸尚書軍校加侍中常侍者,皆給傳事乘軺車,給劍,得入殿省中,與侍臣升降相隨。」



 大使車,立乘,駕四,赤帷裳,騶騎導從。舊公卿二千石郊廟上陵從駕,乘大使車,他出乘安車也。



 小使車,不立乘,駕四,輕車之流也。蘭輿皆朱,赤轂,赤屏泥,白蓋,赤帷裳,從騶騎四十人。又別有小使車,赤轂皁蓋,追捕考案有所執取者之所乘也。凡諸使車皆朱班輪,赤衡軛。



 追鋒車,去小平蓋,加通幰,如軺車,駕二。追鋒之名,蓋取
 其迅速也,施於戎陣之間,是為傳乘。



 軺車,古之時軍車也。一馬曰軺車,二馬曰軺傳。漢世貴輜軿而賤軺車,魏晉重軺車而賤輜軿。三品將軍以上、尚書令軺車黑耳有後戶,僕射但有後戶無耳,並皂輪。尚書及四品將軍則無後戶,漆轂輪。其中書監令如僕射,侍中、黃門、散騎,初拜及謁陵廟,亦得乘之。



 皇太后、皇后法駕,乘重翟羽蓋金根車,駕青輅,青帷裳,雲𣝛畫轅,黃金塗五采,蓋爪施金華,駕三,左右騑。其廟見小駕,則乘紫罽軿車,雲𣝛畫輈,黃金塗五采,駕三。非法駕則皇太后乘輦,皇后乘畫輪車。皇后先蠶,乘油畫
 雲母安車,駕六騩馬;騩,淺黑色。油畫兩轅安車,駕五騩馬,為副。又,金薄石山軿、紫絳罽軿車,皆駕三騩馬,為副。女旄頭十二人,持棨戟二人,共載安車,儷駕。女尚輦十二人,乘輜車,儷駕。女長御八人,乘安車,儷駕。三夫人油軿車,駕兩馬,左騑。其貴人駕節畫輈。三夫人助蠶,乘青交路,安車,駕三,皆以紫絳罽軿車。九嬪世婦乘軿車,駕三。



 長公主赤罽軿車,駕兩馬。公主、王太妃、王妃,皆油軿車,駕兩馬,右騑。公主油畫安車,駕三,青交路,以紫絳罽軿車駕三為副,王太妃、三夫人亦如之。公主助蠶,乘油畫安車,駕三。公主有先置者,乘青交路安車,駕三。



 諸王妃、公太夫人、夫人、縣鄉君、諸郡公侯特進夫人助蠶,乘皁交路安車,駕三。



 諸侯監國世子之世婦、侍中常侍尚書中書監令卿校世婦、命婦助蠶,乘皁交路安車,儷駕。



 郡縣公侯、中二千石、二千石夫人會朝及蠶,各乘其夫之安車,皆右騑,皁交路,皁帷裳。自非公會則不得乘軺車,止乘漆布輜軿,銅五采而已。



 王妃、特進夫人、封郡君,安車,駕三,皂交路。封縣鄉君油軿車,駕兩馬,右騑。



 自過江之後,舊章多缺。元帝踐極,始造大路、戎路各一,
 皆即古金根之制也,無復充庭之儀。至於郊祀大事,則權飾餘車以周用。六師親征則用戎路,去其蓋而乘之,屬車但五乘而已。加綠油幢,朱絲路,飾青交路,黃金塗五采,其輪轂猶素,兩箱無金錦之飾。其一車又是軺車,舊儀,天子所乘駕六,是時無復六馬之乘,五路皆駕四而已,同用黑,是為玄牡。無復五時車,有事則權以馬車代之,建旗其上。其後但以五色木牛象五時車,豎旗於牛背,行則使人輿之。牛之義,蓋取其負重致遠安而穩也。旗常纏而不舒旆,所謂德車結旌者也。惟天子親戎,五旗舒旆,所謂武車綏旌者也。指南車,過江亡失,及義
 熙五年,劉裕屠廣固,始復獲焉,乃使工人張綱補緝周用。十三年,裕定關中,又獲司南、記里諸車,制度始備。其輦,過江亦亡制度,太元中謝安率意造焉,及破苻堅於淮上,獲京都舊輦,形制無差,大小如一,時人服其精記。義熙五年,劉裕執慕容超,獲金鉦輦、豹尾,舊式猶存。



 元帝太興三年,皇太子釋奠。制曰:「今草創,未有高車,可乘安車也。」太元中,東宮建,乘路有青赤旂,致疑。徐邈議,太子既不備五路,赤旂宜省。漢制,太子鸞路皆以安車為名。自晉過江,禮儀疏舛,王公以下,車服卑雜,惟有東宮禮秩崇異,上次辰極,下納侯王。而安帝為皇太子乘石
 山安車,制如金路,義不經見,事無所出。



 中宮初建及祀先蠶,皆用法駕,太僕妻御,大將軍妻參乘,侍中妻陪乘,丹陽尹建康令及公卿之妻奉引,各乘其夫車服,多以宮人權領其職。



 《周禮》,弁師掌六冕,司服掌六服。自后王之制爰及庶人,各有等差。及秦變古制,郊祭之服皆以袀玄,舊法掃地盡矣。漢承秦弊,西京二百餘年猶未能有所制立。及中興後,明帝乃始採《周官》、《禮記》、《尚書》及諸儒記說,還備袞冕之服。天子車乘冠服從歐陽氏說,公卿以下從大小夏侯氏說,始制天子、三公、九卿、特進之服,侍祠天地明
 堂,皆冠旒冕,兼五冕之制,一服而已。天子備十二章,三公諸侯用山龍九章,九卿以下用華蟲七章,皆具五采。魏明帝以公卿袞衣黼黻之飾,疑於至尊,多所減損,始制天子服刺繡文,公卿服織成文。及晉受命,遵而無改。天子郊祀天地明堂宗廟,元會臨軒,黑介幘,通天冠,平冕。冕,皁表,朱綠裏,廣七寸,長二尺二寸,加於通天冠上,前圓後方,垂白玉珠,十有二旒,以朱組為纓,無緌。佩白玉,垂珠黃大旒,綬黃赤縹紺四采。衣皁上,絳下,前三幅,後四幅,衣畫而裳繡,為日、月、星辰、山、龍、華蟲、藻、火、粉米、黼、黻之象,凡十二章。素帶廣四寸,朱裏,以朱綠裨飾其側。
 中衣以絳緣其領袖。赤皮為韍,絳褲襪,赤舄。未加元服者,空頂介幘。其釋奠先聖,則皁紗袍,絳緣中衣,絳褲襪,黑舄,其臨軒,亦袞冕也。其朝服,通天冠高九寸,金博山顏,黑介幘,絳紗袍,皁緣中衣。其拜陵,黑介幘,單衣。其雜服,有青赤黃白緗黑色,介幘,五色紗袍,五梁進賢冠,遠遊冠,平上幘武冠。其素服,白單衣。後漢以來,天子之冕,前後旒用真白玉珠。魏明帝好婦人之飾,改以珊瑚珠。晉初仍舊不改。及過江,服章多闕,而冕飾以翡翠珊瑚雜珠。侍中顧和奏:「舊禮,冕十二旒,用白玉珠。今美玉難得,不能備,可用白璇珠。」從之。



 通天冠,本秦制。高九寸,正豎,頂少斜卻,乃直下,鐵為卷梁,前有展筒,冠前加金博山述,乘輿所常服也。



 平冕,王公、卿助祭於郊廟服之。王公八旒,卿七旒。以組為纓,色如其綬。王公衣山龍以下九章,卿衣華蟲以下七章。



 遠遊冠,傅玄云秦冠也。似通天而前無山述,有展筒橫于冠前。皇太子及王者後、帝之兄弟、帝之子封郡王者服之。諸王加官者自服其官之冠服,惟太子及王者後常冠焉。太子則以翠羽為緌,綴以白珠,其餘但青絲而
 已。



 緇布冠,蔡邕云即委貌冠也。太古冠布,齊則緇之。緇布冠,始冠之冠也。其制有四形,一似武冠,又一似進賢,其一上方其下如幘顏,其一刺上而方下。行鄉射禮則公卿委貌冠,以皁絹為之。形如覆杯,與皮弁同制,長七寸,高四寸。衣黑而裳素,其中衣以皁緣領袖。其執事之人皮弁,以鹿皮為之。



 進賢冠,古緇布遺象也,斯蓋文儒者之服。前高七寸,後高三寸,長八寸,有五梁、三梁、二梁、一梁。人主元服,始加緇布,則冠五梁進賢。三公及封郡公、縣公、郡侯、縣侯、鄉亭侯,則冠三梁。卿、大夫、八座,尚書,關中內侯、二千石及
 千石以上,則冠兩梁。中書郎、秘書丞郎、著作郎、尚書丞郎、太子洗馬舍人、六百石以下至于令史、門郎、小史、並冠一梁。漢建初中,太官令冠兩梁,親省御膳為重也。博士兩梁,崇儒也。宗室劉氏亦得兩梁冠,示加服也。



 武冠,一名武弁,一名大冠,一名繁冠,一名建冠,一名籠冠,即古之惠文冠。或曰趙惠文王所造,因以為名。亦云,惠者蟪也,其冠文輕細如蟬翼,故名惠文。或云,齊人見千歲涸澤之神,名曰慶忌,冠大冠,乘小車,好疾馳,因象其冠而服焉。漢幸臣閎孺為侍中,皆服大冠。天子元服亦先加大冠,左右侍臣及諸將軍武官通服之。侍中、常
 侍則加金璫,附蟬為飾,插以貂毛,黃金為竿,侍中插左,常侍插右。胡廣曰:「昔趙武靈王為胡服,以金貂飾首。秦滅趙,以其君冠賜侍臣。」應劭《漢官》云:「說者以為金取剛強,百煉不耗。蟬居高飲清,口在掖下。貂內勁悍而外柔縟。」又以蟬取清高飲露而不食,貂則紫蔚柔潤而毛采不彰灼,金則貴其寶瑩,於義亦有所取。或以為北土多寒,胡人常以貂皮溫額,後世效此,遂以附冠。漢貂用赤黑色,王莽用黃貂,各附服色所尚也。



 高山冠,一名側注,高九寸,鐵為卷梁,制似通天。頂直豎,不斜卻,無山述展筒。高山者,《詩》云「高山仰止」,取其矜莊
 賓遠者也。中外官、謁者、謁者僕射所服。胡廣曰:「高山,齊王冠也。傅曰『桓公好高冠大帶』。秦滅齊,以其君冠賜謁者近臣。」應劭曰:「高山,今法冠也,秦行人使官亦服之。」而《漢官儀》云「乘輿冠高山之冠,飛翮之纓」,然則天子亦有時服焉。《傅子》曰:「魏明帝以其制似通天、遠遊,故改令卑下。」



 法冠,一名柱後,或謂之獬豸冠。高五寸,以縰為展筒。鐵為柱卷,取其不曲撓也。侍御史、廷尉正監平,凡執法官皆服之。或謂獬豸神羊,能觸邪佞。《異物志》云:「北荒之中,有獸名獬豸,一角,性別曲直。見人鬥,觸不直者。聞人爭,
 咋不正者。楚王嘗獲此獸,因象其形以制衣冠。」胡廣曰:「《春秋左氏傳》晉侯觀于軍府,見鐘儀,曰『南冠而縶者誰也』?南冠即楚冠。秦滅楚,以其冠服賜執法臣也。」



 長冠,一名齊冠。高七寸,廣三寸,漆纚為之,制如版,以竹為裏。漢高祖微時,以竹皮為此冠,其世因謂劉氏冠。後除竹用漆纚。司馬彪曰:「長冠蓋楚制。人間或謂之鵲尾冠,非也。救日蝕則服長冠,而祠宗廟諸祀冠之。此高祖所造,後世以為祭服,尊敬之至也。」



 建華冠,以鐵為柱卷,貫大銅珠九枚,古用雜木珠,原憲所冠華冠是也。又《春秋左氏傳》鄭子臧好聚鷸冠,謂建
 華是也。祀天地、五郊、明堂,舞人服之。漢《育命舞》樂人所服。



 方山冠,其制似進賢。鄭展曰:「方山冠,以五采縠為之。」漢《大予》、《八佾》、《五行》樂人所服,冠衣各如其行方之色而舞焉。



 巧士冠,前高七寸,要後相通,直豎。此冠不常用,漢氏惟郊天,黃門從官四人冠之;在鹵簿中,夾乘輿車前,以備宦者四星。或云,掃除從官所服。



 卻非冠,高五寸,制似長冠。宮殿門吏僕射冠之。負赤幡,青翅燕尾,諸僕射幡皆如之。



 卻
 敵冠,前高四寸,通長四寸,後高三寸,制似進賢。凡當殿門衛士服之。



 樊噲冠,廣九寸,高七寸,前後出各四寸,制似平冕。昔楚漢會於鴻門,項籍圖危高祖,樊噲常持鐵楯,聞急,乃裂裳苞楯,戴以為冠,排入羽營,因數羽罪,漢王乘間得出。後人壯其意,乃制冠象焉。凡殿門司馬衛士服之。



 術氏冠,前圓,吳制,差池四重。趙武靈王好服之。或曰,楚莊王復仇冠是也。



 鶡冠,加雙鶡尾,豎插兩邊。鶡,鳥名也,形類鷂而微黑,性果勇,其鬥到死乃止。上黨貢之,趙武靈王以表顯壯士。
 至秦漢,猶施之武人。



 皮弁,以鹿皮淺毛黃白色者為之。《禮》「王皮弁,會五採玉綦,象邸玉笄」,謂之合皮為弁。其縫中名曰會,以采玉朱為綦。綦,結也。天子五采,諸侯三採。邸,冠下抵也,象骨為之,音帝也。天子則縫十二,公侯伯七,子男五,孤四,卿大夫三。



 韋弁,制似皮弁,頂上尖,韎草染之,色如淺絳。



 爵弁,一名廣冕。高八寸,長尺二寸,如爵形,前小後大。增其上似爵頭色。有收持笄,所謂夏收殷哻者也。祠天地、五郊、明堂,《雲翹舞》樂人服之。



 幘者,古賤人不冠者之服也。漢元帝額有壯髮,始引幘
 服之。王莽頂禿,又加其屋也。《漢注》曰,冠進賢者宜長耳,今介幘也。冠惠文者宜短耳,今平上幘也。始時各隨所宜,遂因冠為別。介幘服文吏,平上幘服武官也。童子幘無屋者,示不成人也。又有納言幘,幘後收又一重,方三寸。又有赤幘,騎吏、武吏、乘輿鼓吹所服。救日蝕,文武官皆免冠著幘,對朝服,示武威也。



 漢儀,立秋日獵,服緗幘。及江左,哀帝從博士曹弘之等議,立秋御讀令,改用素白。案漢末王公名士多委王服,以幅巾為雅,是以袁紹、崔鈞之徒,雖為將帥,皆著縑巾。魏武以天下凶荒,資財乏匱,擬古皮弁,裁縑帛以為
 ,合乎簡易隨時之義,以色別其貴賤,本施軍飾,非為國容也。徐爰曰:「俗說本未有岐,荀文若巾之行,觸樹枝成岐,謂之為善,因而弗改。」今通以為慶弔服。



 巾,以葛為之,形如而橫著之,古尊卑共服也。故漢末妖賊以黃為巾,世謂黃巾賊。



 帽名猶冠也,義取於蒙覆其首,其本纚也。古者冠無幘,冠下有纚,以繒為之。後世施幘於冠,因或裁纓為帽。自乘輿宴居,下至庶人無爵者皆服之。成帝咸和九年,制聽尚書八座丞郎、門下三省侍官乘車,白低幃,出入掖門。又,二宮直官著烏紗。然則往往士人宴居皆著
 矣。而江左時野人已著帽,人士亦往往而然,但其頂圓耳,後乃高其屋云。



 漢制,自天子至于百官,無不佩劍,其後惟朝帶劍。晉世始代之以木,貴者猶用玉首,賤者亦用蚌、金銀、玳瑁為雕飾。



 乘輿六璽,秦制也。曰「皇帝行璽」、「皇帝之璽」、「皇帝信璽」、「天子行璽」、「天子之璽」、「天子信璽」,漢遵秦不改。又有秦始皇藍田玉璽,螭獸紐,在六璽之外,文曰「受天之命,皇帝壽昌」。漢高祖佩之,後世名曰傳國璽,與斬白蛇劍俱為乘輿所寶。斬白蛇劍至惠帝時武庫火燒之,遂亡。及懷帝
 沒胡,傳國璽沒於劉聰,後又沒於石勒。及石季龍死,胡亂,穆帝世乃還江南。



 革帶,古之鞶帶也,謂之鞶革,文武眾官牧守丞令下及騶寺皆服之。其有囊綬,則以綴於革帶,其戎服則以皮絡帶代之。八坐尚書荷紫,以生紫為袷囊,綴之服外,加於左肩。昔周公負成王,制此服衣,至今以為朝服。或云漢世用盛奏事,負之以行,未詳也。



 車前五百者,卿行旅從,五百人為一旅。漢氏一統,故去其人,留其名也。



 褲褶之制,未詳所起,近世凡車駕親戎、中外戒嚴服之。
 服無定色,冠黑帽,綴紫摽,摽以繒為之,長四寸,廣一寸,腰有絡帶以代鞶。中官紫摽,外官絳摽。又有纂嚴戎服而不綴摽,行留文武悉同。其畋獵巡幸,則惟從官戎服帶鞶革,文官不下纓,武官脫冠。



 漢制,一歲五郊,天子與執事者所服各如方色,百官不執事者服常服絳衣以從。魏祕書監秦靜曰:「漢氏承秦,改六冕之制,但玄冠絳衣而已。」魏已來名為五時朝服,又有四時朝服,又有朝服。自皇太子以下隨官受給。百官雖服五時朝服,據今止給四時朝服,闕秋服。三年一易。



 諸假印綬而官不給鞶囊者,得自具作,其但假印不假綬者,不得佩綬鞶,古制也。漢世著鞶囊者,側在腰間,或謂之傍囊,或謂之綬囊,然則以紫囊盛綬也。或盛或散,各有其時。



 笏,古者貴賤皆執笏,其有事則搢之於腰帶,所謂搢紳之士者,搢笏而垂紳帶也。紳垂長三尺。笏者,有事則書之,故常簪筆,今之白筆是其遺象。三臺五省二品文官簪之,王、公、侯、伯、子、男、卿尹及武官不簪,加內侍位者乃簪之。手版即古笏矣。尚書令、僕射、尚書手版頭復有白筆,以紫皮裹之,名曰笏。



 皇太子金璽龜鈕,朱黃綬,四采:赤、黃、縹、紺。給五時朝服、遠遊冠,介幘、翠緌。佩瑜玉,垂組。朱衣絳紗襮,皁緣白紗,其中衣白曲領。帶劍,火珠素首。革帶,玉鉤燮獸頭鞶囊。其大小會、祠宗廟、朔望、五日還朝皆朝服,常還上宮則朱服,預上宮正會則於殿下脫劍舄。又有三梁進賢冠。其侍祀則平冕九旒,袞衣九章,白紗絳緣中單,絳繒韠,采畫織成袞帶,金辟邪首,紫綠二色帶,采畫廣領、曲領各一,赤舄絳襪。若講,則著介幘單衣。釋奠,則遠遊冠,玄朝服,絳緣中單,絳褲襪,玄舄。若未加元服,則中舍人執冕從,介幘單衣玄服。



 諸王金璽龜鈕,纁朱綬,四采:朱、黃、縹、紺。五時朝服,遠遊冠介幘,亦有三梁進賢冠。朱衣絳紗襮皁緣,中衣表素。革帶,黑舄,佩山玄玉,垂組,大帶。若加餘官,則服其加官之服也。



 皇后謁廟,其服皂上皁下,親蠶則青上縹下,皆深衣制,隱領,袖緣以絳。首飾則假髻,步搖,俗謂之珠松是也,簪珥。步搖以黃金為山題,貫白珠為支相繆。八爵九華,熊、獸、赤羆、天鹿、辟邪、南山豐大特六獸,諸爵獸皆以翡翠為毛羽,金題白珠榼,繞以翡翠為華。元康六年,詔曰:「魏以來皇后蠶服皆以文繡,非古義也。今宜純服青,以為
 永制。」



 貴人、夫人、貴嬪,是為三夫人,皆金章紫綬,章文曰貴人、夫人、貴嬪之章。佩于闐玉。



 淑妃、淑媛、淑儀、脩華、脩容、脩儀、婕妤、容華、充華,是為九嬪,銀印青綬,佩采瓄玉。



 貴人、貴嬪、夫人助蠶,服純縹為上與下,皆深衣制。太平髻,七金奠蔽髻,黑玳瑁,又加簪珥。九嬪及公主、夫人五金奠,世婦三金奠。助蠶之義,自古而然矣。



 皇太子妃金璽龜鈕,纁朱綬,佩瑜玉。



 諸王太妃、妃、諸長公主、公主、封君金印紫綬,佩山玄玉。



 長公主、公主見會,太平髻,七金奠蔽髻。其長公主得有步搖,皆有簪珥,衣服同制。自公主、封君以上皆帶綬,以彩組為緄帶,各如其綬色,金闢邪首為帶玦。



 郡公侯縣公侯太夫人,夫人銀印青綬,佩水蒼玉,其特加乃金紫。



 公特進侯卿校世婦、中二千石二千石夫人紺繒幗,黃金龍首銜白珠,魚須擿長一尺為簪珥。入廟佐祭者皁絹上下。助蠶者縹絹上下,皆深衣制緣。



 自二千石夫人以上至皇后,皆以蠶衣為朝服。



\end{pinyinscope}