\article{蔡邕列傳下}

\begin{pinyinscope}
蔡邕字伯喈,陳留圉人也。六世祖勳,好黃老,平帝時為郿令。王莽初,授以厭戎連率。勳對印綬仰天歎曰:「吾策名漢室,死歸其正。昔曾子不受季孫之賜,況可事二姓哉?」遂攜將家屬,逃入深山,與鮑宣、卓茂等同不仕新室。父棱,亦有清白行,謚曰貞定公。

邕性篤孝,母常滯病三年,邕自非寒暑節變,未嘗解襟帶,不寢寐者七旬。母卒,廬于冢側,動靜以禮。有菟馴擾其室傍,又木生連理,遠近奇之,多往觀焉。與叔父從弟同居,三世不分財,鄉黨高其義。少博學,師事太傅胡廣。好辭章、數術、天文,妙操音律。

桓帝時,中常侍徐璜、左悺等五侯擅恣,聞邕善鼓琴,遂白天子,敕陳留太守督促發遣。邕不得已,行到偃師,稱疾而歸。閑居翫古,不交當世。感東方客難及楊雄、班固、崔駰之徒設疑以自通,及斟酌群言,韙其是而矯其非,作釋誨以戒厲云爾。

有務世公子誨於華顛胡老曰:「蓋聞

聖人之大寶曰位,故以仁守位,以財聚人。然則有位斯貴,有財斯富,行義達道,士之司也。故伊摯有負鼎之衒,仲尼設執鞭之言,甯子有清商之歌,百里有豢牛之事。夫如是,則聖哲之通趣,古人之明志也。夫子生清穆之世,稟醇和之靈,覃思典籍,韞櫝六經,安貧樂賤,與世無營,沈精重淵,抗志高冥,包括無外,綜析無形,其已久矣。曾不能拔萃出群,揚芳飛文,登天庭,序彝倫,埽六合之穢慝,清宇宙之埃塵,連光芒於白日,屬炎氣於景雲。時逝歲暮,默而無聞。小子惑焉,是以有云。方今聖上寬明,輔弼賢知,崇英逸偉,不墜於地,德弘者建宰相而裂土,才羨者荷榮祿而蒙賜。盍亦回塗要至,俛仰取容,輯當世之利,定不拔之功,榮家宗於此時,遺不滅之令蹤?夫獨未之思邪,何為守彼而不通此?」

胡老傲然而笑曰:「若公子,所謂睹曖昧之利,而忘昭晢之害;專必成之功,而忽蹉跌之敗者已。」公子謖爾斂袂而興曰:「胡為其然也?」胡老曰:「居,吾將釋汝。昔自太極,君臣始基,有羲皇之洪寧,唐虞之至時。三代之隆,亦有緝熙,五伯扶微,勤而撫之。于斯已降,天網縱,人紘弛,王塗壞,太極陀,君臣土崩,上下瓦解。於是智者騁詐,辯者馳說,武夫奮略,戰士講銳。電駭風馳,霧散雲披,變詐乖詭,以合時宜。或畫一策而綰萬金,或談崇朝而錫瑞珪。連衡者六印磊落,合從者駢組流離。隆貴翕習,積富無崖,據巧蹈機,以忘其危。夫華離蔕而萎,條去幹而枯,女冶容而淫,士背道而辜。人毀其滿,神疾其邪,利端始萌,害漸亦牙。速速方轂,夭夭是加,欲豐其屋,乃蔀其家。是故天地否閉,聖哲潛形,石門守晨,沮、溺耦耕,顏歜抱璞,蘧瑗保生,齊人歸樂,孔子斯征,雍渠驂乘,逝而遺輕。夫豈傲主而背國乎?道不可以傾也。

「且我聞之,日南至則黃鍾應,融風動而魚上冰,蕤賓統則微陰萌,蒹葭蒼而白露凝。寒暑相推,陰陽代興,運極則化,理亂相承。今大漢紹陶唐之洪烈,盪四海之殘災,隆隱天之高,拆渉地之基。皇道惟融,帝猷顯桧,汦汦庶類,含甘吮滋。檢六合之群品,濟之乎雍熙,群僚恭己於職司,聖主垂拱乎兩楹。君臣穆穆,守之以平,濟濟多士,端委縉綎,鴻漸盈階,振鷺充庭。譬猶鍾山之玉,泗濱之石,累珪璧不為之盈,探浮磬不為之索。曩者,洪源辟而四隩集,武功定而干戈戢,獫狁攘而吉甫宴,城濮捷而晉凱入。故當其有事也,則蓑笠並載,擐甲揚鋒,不給於務;當其無事也,則舒紳緩佩,鳴玉以步,綽有餘裕。

「夫世臣、門子,铣御之族,天隆其祜,主豐其祿。抱膺從容,爵位自從,攝須理髯,餘官委貴。其取進也,順傾轉圓,不足以喻其便;逡巡放屣,不足以況其易。夫有逸群之才,人人有優贍之智。童子不問疑於老成,瞳矇不稽謀於先生。心恬澹於守高,意無為於持盈。粲乎煌煌,莫非華榮。明哲泊焉,不失所寧。狂淫振蕩,乃亂其情。貪夫殉財,夸者死權。瞻仰此事,體躁心煩。闇謙盈之效,迷損益之數。騁駑駘於脩路,慕騏驥而增驅,卑俯乎外戚之門,乞助乎近貴之譽。榮顯未副,從而顛踣,下獲熏胥之辜,高受滅家之誅。前車已覆,襲軌而騖,曾不鑒禍,以知畏懼。予惟悼哉,害其若是!天高地厚,跼而蹐之。怨豈在明,患生不思。戰戰兢兢,必慎厥尤。

「且用之則行,聖訓也;舍之則藏,至順也。夫九河盈溢,非一卖所防;帶甲百萬,非一勇所抗。今子責匹夫以清宇宙,庸可以水旱而累堯、湯乎?懼煙炎之毀熸,何光芒之敢揚哉!且夫地將震而樞星直,井無景則日陰食,元首寬則望舒朓,侯王肅則月側匿。是以君子推微達著,尋端見緒,履霜知冰,踐露知暑。時行則行,時止則止,消息盈沖,取諸天紀。利用遭泰,可與處否,樂天知命,持神任己。群車方奔乎險路,安能與之齊軌?思危難而自豫,故在賤而不恥。方將騁馳乎典籍之崇塗,休息乎仁義之淵藪,槃旋乎周、孔之庭宇,揖儒、墨而與為友。舒之足以光四表,收之則莫能知其所有。若乃丁千載之運,應神靈之符,闓閶闔,乘天衢,擁華蓋而奉皇樞,納玄策於聖德,宣太平於中區。計合謀從,己之圖也;勳績不立,予之辜也。龜鳳山翳,霧露不除,踊躍草萊,祗見其愚。不我知者,將謂之迂。脩業思真,棄此焉如?靜以俟命,不斁不渝。『百歲之後,歸乎其居。』幸其獲稱,天所誘也。罕漫而已,非己咎也。昔伯翳綜聲於鳥語,葛盧辯音於鳴牛,董父受氏於豢龍,奚仲供德於衡輈,倕氏興政於巧工,造父登御於驊騮,非子享土於善圉,狼瞫取右於禽囚,弓父畢精於筋角,佽非明勇於赴流,壽王創基於格五,東方要幸於談優,上官效力於執蓋,弘羊據相於運籌。僕不能參跡於若人,故抱璞而優遊。」

於是公子仰首降階,忸怩而避。胡老乃揚衡含笑,援琴而歌。歌曰:「練余心兮浸太清,滌穢濁兮存正靈。和液暢兮神氣寧,情志泊兮心亭亭,嗜欲息兮無由生。踔宇宙而遺俗兮,眇翩翩而獨征。」

建寧三年,辟司徒橋玄府,玄甚敬待之。出補河平長。召拜郎中,校書東觀。遷議郎。邕以經籍去聖久遠,文字多謬,俗儒穿鑿,疑誤後學,熹平四年,乃與五官中郎將堂谿典、光祿大夫楊賜、諫議大夫馬日磾、議郎張馴、韓說、太史令單颺等,奏求正定六經文字。靈帝許之,邕乃自書冊於碑,使工鐫刻立於太學門外。於是後儒晚學,咸取正焉。及碑始立,其觀視及摹寫者,車乘日千餘兩,填塞街陌。

初,朝議以州郡相黨,人情比周,乃制婚姻之家及兩州人士不得對相監臨。至是復有三互法,禁忌轉密,選用艱難。幽冀二州,久缺不補。邕上疏曰:「伏見幽、冀舊壤,鎧馬所出,比年兵飢,漸至空耗。今者百姓虛縣,萬里蕭條,闕職經時,吏人延屬,而三府選舉,踰月不定。臣經怪其事,而論者云『避三互』。十一州有禁,當取二州而已。又二州之士,或復限以歲月,狐疑遲淹,以失事會。愚以為三互之禁,禁之薄者,今但申以威靈,明其憲令,在任之人豈不戒懼,而當坐設三互,自生留閡邪?昔韓安國起自徒中,朱買臣出於幽賤,並以才宜,還守本邦。又張敞亡命,擢授劇州。豈復顧循三互,繼以末制乎?三公明知二州之要,所宜速定,當越禁取能,以救時敝;而不顧爭臣之義,苟避輕微之科,選用稽滯,以失其人。臣願陛下上則先帝,蠲除近禁,其諸州刺史器用可換者,無拘日月三互,以差厥中。」書奏不省。

初,帝好學,自造皇羲篇五十章,因引諸生能為文賦者。本頗以經學相招,後諸為尺牘及工書鳥篆者,皆加引召,遂至數十人。侍中祭酒樂松、賈護,多引無行趣埶之徒,並待制鴻都門下,憙陳方俗閭里小事,帝甚悅之,待以不次之位。又市賈小民,為宣陵孝子者,復數十人,悉除為郎中、太子舍人。時頻有雷霆疾風,傷樹拔木,地震、隕雹、蝗蟲之害。又鮮卑犯境,役賦及民。六年七月,制書引咎,誥群臣各陳政要所當施行。邕上封事曰:

臣伏讀聖旨,雖周成遇風,訊諸執事,宣王遭旱,密勿祗畏,無以或加。臣聞天降災異,緣象而至。辟歷數發,殆刑誅繁多之所生也。風者天之號令,所以教人也。夫昭事上帝,則自懷多福;宗廟致敬,則鬼神以著。國之大事,實先祀典,天子聖躬所當恭事。臣自在宰府,及備朱衣,迎氣五郊,而車駕稀出,四時至敬,屢委有司,雖有解除,猶為疏廢。故皇天不悅,顯此諸異。鴻範傳曰:「政悖德隱,厥風發屋折木。」坤為地道,易稱安貞。陰氣憤盛,則當靜反動,法為下叛。夫權不在上,則雹傷物;政有苛暴,則虎狼食人;貪利傷民,則蝗蟲損稼。去六月二十八日,太白與月相迫,兵事惡之。鮮卑犯塞,所從來遠,今之出師,未見其利。上違天文,下逆人事。誠當博覽眾議,從其安者。臣不勝憤滿,謹條宜所施行七事表左:

一事:明堂月令,天子以四立及季夏之節,迎五帝於郊,所以導致神氣,祈福豐年。清廟祭祀,追往孝敬,養老辟雍,示人禮化,皆帝者之大業,祖宗所祗奉也。而有司數以蕃國疏喪,宮內產生,及吏卒小汙,屢生忌故。竊見南郊齋戒,未嘗有廢,至於它祀,輒興異議。豈南郊卑而它祀尊哉?孝元皇帝策書曰:「禮之至敬,莫重於祭,所以竭心親奉,以致肅祗者也。」又元和故事,復申先典。前後制書,推心懇惻。而近者以來,更任太史。忘禮敬之大,任禁忌之書,拘信小故,以虧大典。禮,妻妾產者,齋則不入側室之門,無廢祭之文也。所謂宮中有卒,三月不祭者,謂士庶人數堵之室,共處其中耳,豈謂皇居之曠,臣妾之眾哉?自今齋制宜如故典,庶荅風霆災妖之異。

二事:臣聞國之將興,至言數聞,內知己政,外見民情。是故先帝雖有聖明之姿,而猶廣求得失。又因災異,援引幽隱,重賢良、方正、敦朴、有道之選,危言極諫,不絕於朝。陛下親政以來,頻年災異,而未聞特舉博選之旨。誠當思省述脩舊事,使抱忠之臣展其狂直,以解易傳「政悖德隱」之言。

三事:夫求賢之道,未必一塗,或以德顯,或以言揚。頃者,立朝之士,曾不以忠信見賞,恆被謗訕之誅,遂使群下結口,莫圖正辭。郎中張文,前獨盡狂言,聖聽納受,以責三司。臣子曠然,眾庶解悅。臣愚以為宜擢文右職,以勸忠謇,宣聲海內,博開政路。

四事:夫司隸校尉、諸州刺史,所以督察姦枉,分別白黑者也。伏見幽州刺史楊憙、益州刺史龐芝、涼州刺史劉虔,各有奉公疾姦之心,憙等所糾,其效尤多。餘皆枉橈,不能稱職。或有抱罪懷瑕,與下同疾,綱網弛縱,莫相舉察,公府臺閣亦復默然。五年制書,議遣八使,又令三公謠言奏事。是時奉公者欣然得志,邪枉者憂悸失色。未詳斯議,所因寢息。昔劉向奏曰:「夫執狐疑之計者,開群枉之門;養不斷之慮者,來讒邪之口。」今始聞善政,旋復變易,足令海內測度朝政。宜追定八使,糾舉非法,更選忠清,平章賞罰。三公歲盡,差其殿最,使吏知奉公之福,營私之禍,則眾災之原庶可塞矣。

五事:臣聞古者取士,必使諸侯歲貢。孝武之世,郡舉孝廉,又有賢良、文學之選,於是名臣輩出,文武並興。漢之得人,數路而已。夫書畫辭賦,才之小者,匡國理政,未有其能。陛下即位之初,先涉經術,聽政餘日,觀省篇章,聊以游意,當代博弈,非以教化取士之本。而諸生競利,作者鼎沸。其高者頗引經訓風喻之言;下則連偶俗語,有類俳優;或竊成文,虛冒名氏。臣每受詔於盛化門,差次錄第,其未及者,亦復隨輩皆見拜擢。既加之恩,難復收改,但守奉祿,於義已弘,不可復使理人及仕州郡。昔孝宣會諸儒於石渠,章帝集學士於白虎,通經釋義,其事優大,文武之道,所宜從之。若乃小能小善,雖有可觀,孔子以為「致遠則泥」,君子故當志其大者。

六事:墨綬長吏,職典理人,皆當以惠利為績,日月為勞。褒責之科,所宜分明。而今在任無復能省,及其還者,多召拜議郎、郎中。若器用優美,不宜處之冗散。如有釁故,自當極其刑誅。豈有伏罪懼考,反求遷轉,更相放效,臧否無章?先帝舊典,未嘗有此。可皆斷絕,以覈真偽。

七事:伏見前一切以宣陵孝子者為太子

舍人。臣聞孝文皇帝制喪服三十六日,雖繼體之君,父子至親,公卿列臣,受恩之重,皆屈情從制,不敢踰越。今虛偽小人,本非骨肉,既無幸私之恩,又無祿仕之實,惻隱思慕,情何緣生?而群聚山陵,假名稱孝,行不隱心,義無所依,至有姦軌之人,通容其中。恒思皇后祖載之時,東郡有盜人妻者亡在孝中,本縣追捕,乃伏其辜。虛偽雜穢,難得勝言。又前至得拜,後輩被遺;或經年陵次,以暫歸見漏;或以人自代,亦蒙寵榮。爭訟怨恨,凶凶道路。太子官屬,宜搜選令德,豈有但取丘墓凶醜之人?其為不祥,莫與大焉。宜遣歸田里,以明詐偽。

書奏,帝乃親迎氣北郊,及行辟雍之禮。又詔宣陵孝子為舍人者,悉改為丞尉焉。光和元年,遂置鴻都門學,畫孔子及七十二弟子像。其諸生皆敕州郡三公舉用辟召,或出為刺史、太守,入為尚書、侍中,乃有封侯賜爵者,士君子皆恥與為列焉。

時妖異數見,人相驚擾。其年七月,詔召邕與光祿大夫楊賜、諫議大夫馬日磾、議郎張華、太史令單颺詣金商門,引入崇德殿,使中常侍曹節、王甫就問災異及消改變故所宜施行。邕悉心以對,事在五行、天文志。又特詔問曰:「比災變互生,未知厥咎,朝廷焦心,載懷恐懼。每訪群公卿士,庶聞忠言,而各存括囊,莫肯盡心。以邕經學深奧,故密特稽問,宜披露失得,指陳政要,勿有依違,自生疑諱。具對經術,以皁囊封上。」邕對曰:「臣伏惟陛下聖德允明,深悼災咎,褒臣末學,特垂訪及,非臣螻蟻所能堪副。斯誠輸寫肝膽出命之秋,豈可以顧患避害,使陛下不聞至戒哉!臣伏思諸異,皆亡國之怪也。天於大漢,殷勤不已,故屢出祅變,以當譴責,欲令人君感悟,改危即安。今災眚之發,不於它所,遠則門垣,近在寺署,其為監戒,可謂至切。蜺墯雞化,皆婦人干政之所致也。前者乳母趙嬈,貴重天下,生則貲藏侔於天府,死則丘墓踰於園陵,兩子受封,兄弟典郡;續以永樂門史霍玉,依阻城社,又為姦邪。今者道路紛紛,復云有程大人者,察其風聲,將為國患。宜高為隄防,明設禁令,深惟趙、霍,以為至戒。今聖意勤勤,思明邪正。而聞太尉張顥,為玉所進;光祿勳姓璋,有名貪濁;又長水校尉趙玹、屯騎校尉蓋升,並叨時幸,榮富優足。宜念小人在位之咎,退思引身避賢之福。伏見廷尉郭禧,純厚老成;光祿大夫橋玄,聰達方直;故太尉劉寵,忠實守正:並宜為謀主,數見訪問。夫宰相大臣,君之四體,委任責成,優劣已分,不宜聽納小吏,雕琢大臣也。又尚方工技之作,鴻都篇賦之文,可且消息,以示惟憂。《詩》云:『畏天之怒,不敢戲豫。』天戒誠不可戲也。宰府孝廉,士之高選。近者以辟召不慎,切責三公,而今並以小文超取選舉,開請託之門,違明王之典,眾心不厭,莫之敢言。臣願陛下忍而絕之,思惟萬機,以荅天望。聖朝既自約厲,左右近臣亦宜從化。人自抑損,以塞咎戒,則天道虧滿,鬼神福謙矣。臣以愚贛,感激忘身,敢觸忌諱,手書具對。夫君臣不密,上有漏言之戒,下有失身之禍。願寢臣表,無使盡忠之吏,受怨姦仇。」章奏,帝覽而歎息,因起更衣,曹節於後竊視之,悉宣語左右,事遂漏露。其為邕所裁黜者,皆側目思報。

初,邕與司徒劉郃素不相平,叔父衛尉質又與將作大匠楊球有隙。球即中常侍程璜女夫也,璜遂使人飛章言邕、質數以私事請託於郃,郃不聽,邕含隱切,志欲相中。於是詔下尚書,召邕詰狀。邕上書自陳曰:「臣被召,問以大鴻臚劉郃前為濟陰太守,臣屬吏張宛長休百日,郃為司隸,又託河內郡吏李奇為州書佐,及營護故河南尹羊陟、侍御史胡母班,郃不為用致怨之狀。臣征營怖悸,肝膽塗地,不知死命所在。竊自尋案,實屬宛、奇,不及陟、班。凡休假小吏,非結恨之本。與陟姻家,豈敢申助私黨?如臣父子欲相傷陷,當明言臺閣,具陳恨狀所緣。內無寸事,而謗書外發,宜以臣對與郃參驗。臣得以學問特蒙褒異,執事秘館,操管御前,姓名貌狀,微簡聖心。今年七月,召詣金商門,問以災異,齎詔申旨,誘臣使言。臣實愚贛,唯識忠盡,出命忘軀,不顧後害,遂譏刺公卿,內及寵臣。實欲以上對聖問,救消災異,規為陛下建康寧之計。陛下不念忠臣直言,宜加掩蔽,誹謗卒至,便用疑怪。盡心之吏,豈得容哉?詔書每下,百官各上封事,欲以改政思譴,除凶致吉,而言者不蒙延納之福,旋被陷破之禍。今皆杜口結舌,以臣為戒,誰敢為陛下盡忠孝乎?臣季父質,連見拔擢,位在上列。臣被蒙恩渥,數見訪逮。言事者因此欲陷臣父子,破臣門戶,非復發糾姦伏,補益國家者也。臣年四十有六,孤特一身,得託名忠臣,死有餘榮,恐陛下於此不復聞至言矣。臣之愚冗,職當咎患,但前者所對,質不及聞,而衰老白首,橫見引逮,隨臣摧沒,并入阬埳,誠冤誠痛。臣一入牢獄,當為楚毒所迫,趣以飲章,辭情何緣復聞?死期垂至,冒昧自陳。願身當辜戮,饨質不并坐,則身死之日,更生之年也。惟陛下加餐,為萬姓自愛。」於是下邕、質於洛陽獄,劾以仇怨奉公,議害大臣,大不敬,棄市。事奏,中常侍呂強愍邕無罪,請之,帝亦更思其章,有詔減死一等,與家屬髡鉗徙朔方,不得以赦令除。楊球使客追路刺邕,客感其義,皆莫為用。球又賂其部主使加毒害,所賂者反以其情戒邕,故每得免焉。居五原安陽縣。

邕前在東觀,與盧植、韓說等撰補後漢記,會遭事流離,不及得成,因上書自陳,奏其所著十意,分別首目,連置章左。帝嘉其才高,會明年大赦,及宥邕還本郡。邕自徙及歸,凡九月焉。將就還路,五原太守王智餞之。酒酣,智起舞屬邕,邕不為報。智者,中常侍王甫弟也,素貴驕,慚於賓客,詬邕曰:「徒敢輕我!」邕拂衣而去。智銜之,密告邕怨於囚放,謗訕朝廷。內寵惡之。邕慮卒不免,乃亡命江海,遠跡吳會。往來依太山羊氏,積十二年,在吳。

吳人有燒桐以爨者,邕聞火烈之聲,知其良木,因請而裁為琴,果有美音,而其尾猶焦,故時人名曰「焦尾琴」焉。初,邕在陳留也,其鄰人有以酒食召邕者,比往而酒以酣焉。客有彈琴於屏,邕至門試潛聽之,曰:「压!以樂召我而有殺心,何也?」遂反。將命者告主人曰:「蔡君向來,至門而去。」邕素為邦鄉所宗,主人遽自追而問其故,邕具以告,莫不憮然。彈琴者曰:「我向鼓弦,見螳蜋方向鳴蟬,蟬將去而未飛,螳蜋為之一前一卻。吾心聳然,惟恐螳蜋之失之也,此豈為殺心而形於聲者乎?」邕莞然而笑曰:「此足以當之矣。」

中平六年,靈帝崩,董卓為司空,聞邕名高,辟之。稱疾不就。卓大怒,詈曰:「我力能族人,蔡邕遂偃蹇者,不旋踵矣。」又切敕州郡舉邕詣府,邕不得已,到,署祭酒,甚見敬重。舉高第,補侍御史,又轉持書御史,遷尚書。三日之閒,周歷三臺。遷巴郡太守,復留為侍中。

初平元年,拜左中郎將,從獻帝遷都長安,封高陽鄉侯。

董卓賓客部曲議欲尊卓比太公,稱尚父。卓謀之於邕,邕曰:「太公輔周,受命翦商,故特為其號。今明公威德,誠為巍巍,然比之尚父,愚意以為未可。宜須關東平定,車駕還反舊京,然後議之。」卓從其言。

初平二年六月,地震,卓以問邕。邕對曰:「地動者,陰盛侵陽,臣下踰制之所致也。前春郊天,公奉引車駕,乘金華青蓋,爪畫兩轓,遠近以為非宜。」卓於是改乘皁蓋車。

卓重董邕才學,厚相遇待,每集讌,輒令邕鼓琴贊事,邕亦每存匡益。然卓多自佷用,邕恨其言少從,謂從弟谷曰:「董公性剛而遂非,終難濟也。吾欲東奔兗州,若道遠難達,且遯逃山東以待之,何如?」谷曰:「君狀異恆人,每行觀者盈集。以此自匿,不亦難乎?」邕乃止。

及卓被誅,邕在司徒王允坐,殊不意言之而歎,有動於色。允勃然叱之曰:「董卓國之大賊,幾傾漢室。君為王臣,所宜同忿,而懷其私遇,以忘大節!今天誅有罪,而反相傷痛,豈不共為逆哉?」即收付廷尉治罪。邕陳辭謝,乞黥首刖足,繼成漢史。士大夫多矜救之,不能得。太尉馬日磾馳往謂允曰:「伯喈曠世逸才,多識漢事,當續成後史,為一代大典。且忠孝素著,而所坐無名,誅之無乃失人望乎?」允曰:「昔武帝不殺司馬遷,使作謗書,流於後世。方今國祚中衰,神器不固,不可令佞臣執筆在幼主左右。既無益聖德,復使吾黨蒙其訕議。」日磾退而告人曰:「王公其不長世乎?善人,國之紀也;制作,國之典也。滅紀廢典,其能久乎!」邕遂死獄中。允悔,欲止而不及。時年六十一。搢紳諸儒莫不流涕。北海鄭玄聞而歎曰:「漢世之事,誰與正之!」兗州、陳留聞皆畫像而頌焉。

其撰集漢事,未見錄以繼後史。適作靈紀及十意,又補諸列傳四十二篇,因李傕之亂,湮沒多不存。所著詩、賦、碑、誄、銘、讚、連珠、箴、弔、論議、獨斷、勸學、釋誨、敘樂、女訓、篆埶、祝文、章表、書記,凡百四篇,傳於世。

論曰:意氣之感,士所不能忘也。流極之運,有生所共深悲也。當伯喈抱鉗扭,徙幽裔,仰日月而不見照燭,臨風塵而不得經過,其意豈及語平日倖全人哉!及解刑衣,竄歐越,潛舟江壑,不知其遠,捷步深林,尚苦不密,但願北首舊丘,歸骸先壟,又可得乎?董卓一旦入朝,辟書先下,分明枉結,信宿三遷。匡導既申,狂僭屢革,資同人之先號,得北叟之後福。屬其慶者,夫豈無懷?君子斷刑,尚或為之不舉,況國憲倉卒,慮不先圖,矜情變容,而罰同邪黨?執政乃追怨子長謗書流後,放此為戮,未或聞之典刑。

贊曰:季長戚氏,才通情侈。苑囿典文,流悅音伎。邕實慕靜,心精辭綺。斥言金商,南徂北徙。籍梁懷董,名澆身毀。


\end{pinyinscope}