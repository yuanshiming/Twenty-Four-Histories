\article{劉趙淳于江劉周趙列傳}

\begin{pinyinscope}
孔子曰:「夫孝莫大於嚴父,嚴父莫大於配天,則周公其人也。」子路曰:「傷哉貧也!生無以養,死無以葬。」子曰:「啜菽飲水,孝也。」夫鍾鼓非樂云之本,而器不可去;三牲非致孝之主,而養不可廢。存器而忘本,樂之遁也;調器以和聲,樂之成也。崇養以傷行,孝之累也;脩己以致祿,養之大也。故言能大養,則周公之祀,致四海之祭;言以義養,則仲由之菽,甘於東鄰之牲。夫患水菽之薄,干祿以求養者,是以恥祿親也。存誠以盡行,孝積而祿厚者,此能以義養也。

中興,廬江毛義少節,家貧,以孝行稱。南陽人張奉慕其名,往候之。坐定而府檄適至,以義守令,義奉檄而入,喜動顏色。奉者,志尚士也,心賤之,自恨來,固辭而去。及義母死,去官行服。數辟公府,為縣令,進退必以禮。後舉賢良,公車徵,遂不至。張奉歎曰:「賢者固不可測。往日之喜,乃為親屈也。斯蓋所謂『家貧親老,不擇官而仕』者也。」建初中,章帝下詔褒寵義,賜穀千斛,常以八月長吏問起居,加賜羊酒。壽終于家。

安帝時,汝南薛包孟嘗,好學篤行,喪母,以至孝聞。及父娶後妻而憎包,分出之,包日夜號泣,不能去,至被歐杖。不得已,廬於舍外,旦入而洒掃,父怒,又逐之。乃廬於里門,昏晨不廢。積歲餘,父母慚而還之。後行六年服,喪過乎哀。既而弟子求分財異居,包不能止,乃中分其財。奴婢引其老者,曰:「與我共事久,若不能使也。」田廬取其荒頓者,曰:「吾少時所理,意所戀也。」器物取朽敗者,曰:「我素所服食,身口所安也。」弟子數破其產,輒復賑給。建光中,公車特徵,至,拜侍中。包性恬虛,稱疾不起,以死自乞。有詔賜告歸,加禮如毛義。年八十餘,以壽終。

若二子者,推至誠以為行,行信於心而感於人,以成名受祿致禮,斯可謂能以孝養也。若夫江革、劉般數公者之義行,猶斯志也。撰其行事著于篇。

劉平字公子,楚郡彭城人也。本名曠,顯宗後改為平。王莽時為郡吏,守菑丘長,政教大行。其後每屬縣有劇賊,輒令平守之,所至皆理,由是一郡稱其能。

更始時,天下亂,平弟仲為賊所殺。其後賊復忽然而至,平扶侍其母,奔走逃難。仲遺腹女始一歲,平抱仲女而棄其子。母欲還取之,平不聽,曰:「力不能兩活,仲不可以絕類。」遂去不顧,與母俱匿野澤中。平朝出求食,逢餓賊,將亨,平叩頭曰:「今旦為老母求菜,老母待曠為命,願得先歸,食母異,還就死。」因涕泣。賊見其至誠,哀而遣之。平還,既食母訖,因白曰:「屬與賊期,義不可欺。」遂還詣賊。眾皆大驚,相謂曰:「常聞烈士,乃今見之。子去矣,吾不忍食子。」於是得全。

建武初,平狄將軍龐萌反於彭城,攻敗郡守孫萌。平時復為郡吏,冒白刃伏萌身上,被七創,困頓不知所為,號泣請曰:「願以身代府君。」賊乃斂兵止,曰:「此義士也,勿殺。」遂解去。萌傷甚氣絕,有頃蘇,渴求飲。平傾其創血以飲之。後數日萌竟死,平乃裹創,扶送萌喪,至其本縣。

後舉孝廉,拜濟陰郡丞,太守劉育甚重之,任以郡職,上書薦平。會平遭父喪去官。服闋,拜全椒長,政有恩惠,百姓懷感,人或增貲就賦,或減年從役。刺史、太守行部,獄無繫囚,人自以得所,不知所問,唯班詔書而去。後以病免。

顯宗初,尚書僕射鍾離意上書薦平及琅邪王望、東萊王扶曰:「臣竊見琅邪王望、楚國劉曠、東萊王扶,皆年七十,執性恬淡,所居之處,邑里化之,脩身行義,應在朝次。臣誠不足知人,竊慕推士進賢之義。」書奏,有詔徵平等,特賜辦裝錢。至皆拜議郎,並數引見。平再遷侍中,永平三年,拜宗正,數薦達名士承宮、郇恁等。在位八年,以老病上疏乞骸骨,卒於家。

王望字慈卿,客授會稽,自議郎遷青州刺史,甚有威名。是時州郡災旱,百姓窮荒,望行部,道見飢者,裸行草食,五百餘人,愍然哀之,因以便宜出所在布粟,給其廩糧,為作褐衣。事畢上言,帝以望不先表請,章示百官,詳議其罪。時公卿皆以為望之專命,法有常條。鍾離意獨曰:「昔華元、子反,楚、宋之良臣,不稟君命,擅平二國,春秋之義,以為美談。今望懷義忘罪,當仁不讓,若繩之以法,忽其本情,將乖聖朝愛育之旨。」帝嘉意議,赦而不罪。

王扶字子元,掖人也。少脩節行,客居琅邪不其縣,所止聚落化其德。國相張宗謁請,不應,欲強致之,遂杖策歸鄉里。連請,固病不起。太傅鄧禹辟,不至。後拜議郎,會見,恂恂似不能言。然性沈正,不可干以非義,當世高之。永平中,臨邑侯劉復著漢德頌,盛稱扶為名臣云。

趙孝字長平,沛國蘄人也。父普,王莽時為田禾將軍,任孝為郎。每告歸,常白衣步擔。嘗從長安還,欲止郵亭。亭長先時聞孝當過,以有長者客,掃洒待之。孝既至,不自名,長不肯內,因問曰:「聞田禾將軍子當從長安來,何時至乎?」孝曰:「尋到矣。」於是遂去。及天下亂,人相食。孝弟禮為餓賊所得,孝聞之,即自縛詣賊,曰:「禮久餓羸瘦,不如孝肥飽。」賊大驚,並放之,謂曰:「可且歸,更持米糒來。」孝求不能得,復往報賊,願就亨。眾異之,遂不害。鄉黨服其義。州郡辟召,進退必以禮。舉孝廉,不應。

永平中,辟太尉府,顯宗素聞其行,詔拜諫議大夫,遷侍中,又遷長樂衛尉。復徵弟禮為御史中丞。禮亦恭謙行己,類於孝。帝嘉其兄弟篤行,欲寵異之,詔禮十日一就衛尉府,太官送供具,令共相對盡歡。數年,禮卒,帝令孝從官屬送喪歸葬。後歲餘,復以衛尉賜告歸,卒于家。孝無子,拜禮兩子為郎。

時汝南有王琳巨尉者,年十餘歲喪父母。因遭大亂,百姓奔逃,唯琳兄弟獨守塚廬,號泣不絕。弟季,出遇赤眉,將為所哺,琳自縛,請先季死,賊矜而放遣,由是顯名鄉邑。後辟司徒府,薦士而退。

琅邪魏譚少閒者,時亦為飢寇所獲,等輩數十人皆束縛,以次當亨。賊見譚似謹厚,獨令主戏,暮輒執縛。賊有夷長公,特哀念譚,密解其縛,語曰:「汝曹皆應就食,急從此去。」對曰:「譚為諸君戏,恆得遺餘,餘人皆茹草萊,不如食我。」長公義之,相曉赦遣,並得俱免。譚永平中為主家令。

又齊國兒萌子明、梁郡車成子威二人,兄弟並見執於赤眉,將食之,萌、成叩頭,乞以身代,賊亦哀而兩釋焉。

淳于恭字孟孫,北海淳于人也。善說老子,清靜不慕榮名。家有山田果樹,人或侵盜,輒助為收採。又見偷刈禾者,恭念其愧,因伏草中,盜去乃起,里落化之。

王莽末,歲飢兵起,恭兄崇將為盜所亨,恭請代,得俱免。後崇卒,恭養孤幼,教誨學問,有不如法,輒反用杖自箠,以感悟之,兒慚而改過。初遭賊寇,百姓莫事農桑。恭常獨力田耕,鄉人止之曰:「時方淆亂,死生未分,何空自苦為?」恭曰:「縱我不得,它人何傷。」墾耨不輟。後州郡連召,不應,遂幽居養志,潛於山澤。舉動周旋,必由禮度。建武中,郡舉孝廉,司空辟,皆不應,客隱琅邪黔陬山,遂數十年。

建初元年,肅宗下詔美恭素行,告郡賜帛二十匹,遣詣公車,除為議郎。引見極日,訪以政事,遷侍中騎都尉,禮待甚優。其所薦名賢,無不徵用。進對陳政,皆本道德,帝與之言,未嘗不稱善。五年,病篤,使者數存問,卒於官。詔書褒歎,賜穀千斛,刻石表閭。除子孝為太子舍人。

江革字次翁,齊國臨淄人也。少失父,獨與母居。遭天下亂,盜賊並起,革負母逃難,備經阻險,常採拾以為養。數遇賊,或劫欲將去,革輒涕泣求哀,言有老母,辭氣愿款,有足感動人者。賊以是不忍犯之,或乃指避兵之方,遂得俱全於難。革轉客下邳,窮貧裸跣,行傭以供母,便身之物,莫不必給。

建武末年,與母歸鄉里。每至歲時,縣當案比,革以母老,不欲搖動,自在轅中輓車,不用牛馬,由是鄉里稱之曰「江巨孝」。太守嘗備禮召,革以母老不應。及母終,至性殆滅,嘗寢伏冢廬,服竟,不忍除。郡守遣丞掾釋服,因請以為吏。

永平初,舉孝廉為郎,補楚太僕。月餘,自劾去。楚王英馳遣官屬追之,遂不肯還。復使中傅贈送,辭不受。後數應三公命,輒去。

建初初,太尉牟融舉賢良方正,再遷司空長史。肅宗甚崇禮之,遷五官中郎將。每朝會,帝常使虎賁扶侍,及進拜,恆目禮焉。時有疾不會,輒太官送醪膳,恩寵有殊。於是京師貴戚衛尉馬廖、侍中竇憲慕其行,各奉書致禮,革無所報受。帝聞而益善之。後上書乞骸骨,轉拜諫議大夫,賜告歸,因謝病稱篤。

元和中,天子思革至行,制詔齊相曰:「諫議大夫江革,前以病歸,今起居何如?夫孝,百行之冠,眾善之始也。國家每惟志士,未嘗不及革。縣以見穀千斛賜『巨孝』,常以八月長吏存問,致羊酒,以終厥身。如有不幸,祠以中牢。」由是「巨孝」之稱,行於天下。及卒,詔復賜穀千斛。

劉般字伯興,宣帝之玄孫也。宣帝封子囂於楚,是為孝王。孝王生思王衍,衍生王紆,紆生般。自囂至般,積累仁義,世有名節,而紆尤慈篤。早失母,同產弟原鄉侯平尚幼,紆親自鞠養,常與共臥起飲食。及成人,未嘗離左右。平病卒,紆哭泣歐血,數月亦歿。初,紆襲王封,因值王莽篡位,廢為庶人,因家於彭城。

般數歲而孤,獨與母居。王莽敗,天下亂,太夫人聞更始即位,乃將般俱奔長安。會更始敗,復與般轉側兵革中,西行上隴,遂流至武威。般雖尚少,而篤志脩行,講誦不怠。其母及諸舅,以為身寄絕域,死生未必,不宜苦精若此,數以曉般,般猶不改其業。

建武八年,隗囂敗,河西始通,般即將家屬東至洛陽,脩經學於師門。明年,光武下詔,封般為菑丘侯,奉孝王祀,使就國。後以國屬楚王,徙封杼秋侯。

十九年,行幸沛,詔問郡中諸侯行能。太守薦言般束脩至行,為諸侯師。帝聞而嘉之,乃賜般綬,錢百萬,繒二百匹。二十年,復與車駕會沛,因從還洛陽,賜穀什物,留為侍祠侯。

永平元年,以國屬沛,徙封居巢侯,復隨諸侯就國。數年,楊州刺史觀恂薦般在國口無擇言,行無怨惡,宜蒙旌顯。顯宗嘉之。十年,徵般行執金吾事,從至南陽,還為朝侯。明年,兼屯騎校尉。時五校官顯職閑,而府寺寬敞,輿服光麗,伎巧畢給,故多以宗室肺腑居之。每行幸郡國,般常將長水胡騎從。

帝曾欲置常平倉,公卿議者多以為便。般對以「常平倉外有利民之名,而內實侵刻百姓,豪右因緣為姦,小民不能得其平,置之不便」。帝乃止。是時下令禁民二業,又以郡國牛疫,通使區種增耕,而吏下檢結,多失其實,百姓患之。般上言:「郡國以官禁二業,至有田者不得漁捕。今濱江湖郡率少蠶桑,民資漁採以助口實,且以冬春閑月,不妨農事。夫漁獵之利,為田除害,有助穀食,無關二業也。又郡國以牛疫、水旱,墾田多減,故詔敕區種,增進頃畝,以為民也。而吏舉度田,欲令多前,至於不種之處,亦通為租。可申敕刺史、二千石,務令實覈,其有增加,皆使與奪田同罪。」帝悉從之。

肅宗即位,以為長樂少府。建初二年,遷宗正。般妻卒,厚加賵贈,及賜冢塋地於顯節陵下。般在位數言政事。其收恤九族,行義尤著,時人稱之。年六十,建初三年卒。子憲嗣。憲卒,子重嗣。憲兄愷。

愷字伯豫,以當襲般爵,讓與弟憲,遁逃避封。久之,章和中,有司奏請絕愷國,肅宗美其義,特優假之,愷猶不出。積十餘歲,至永元十年,有司復奏之,侍中賈逵因上書曰:「孔子稱『能以禮讓為國,於從政乎何有』。竊見居巢侯劉般嗣子愷,素行孝友,謙遜絜清,讓封弟憲,潛身遠跡。有司不原樂善之心,而繩以循常之法,懼非長克讓之風,成含弘之化。前世扶陽侯韋玄成,近有陵陽侯丁鴻、鄳侯鄧彪,並以高行絜身辭爵,未聞貶削,而皆登三事。今愷景仰前脩,有伯夷之節,宜蒙矜宥,全其先功,以增聖朝尚德之美。」和帝納之,下詔曰:「故居巢侯劉般嗣子愷,當襲般爵,而稱父遺意,致國弟憲,遁亡七年,所守彌篤。蓋王法崇善,成人之美。其聽憲嗣爵。遭事之宜,後不得以為比。」乃徵愷,拜為郎,稍遷侍中。

愷之入朝,在位者莫不仰其風行。遷步兵校尉。十三年,遷宗正,免。復拜侍中,遷長水校尉。永初元年,代周章為太常。愷性篤古,貴處士,每有徵舉,必先巖穴。論議引正,辭氣高雅。永初六年,代張敏為司空。元初二年,代夏勤為司徒。

舊制,公卿、二千石、刺史不得行三年喪,由是內外眾職並廢喪禮。元初中,鄧太后詔長吏以下不為親行服者,不得典城選舉。時有上言牧守宜同此制,詔下公卿,議者以為不便。愷獨議曰:「詔書所以為制服之科者,蓋崇化厲俗,以弘孝道也。今刺史一州之表,二千石千里之師,職在辯章百姓,宣美風俗,尤宜尊重典禮,以身先之。而議者不尋其端,至於牧守則云不宜,是猶濁其源而望流清,曲其形而欲景直,不可得也。」太后從之。

時征西校尉任尚以姦利被徵抵罪。尚曾副大將軍鄧騭,騭黨護之,而太尉馬英、司空李郃承望騭旨,不復先請,即獨解尚臧錮,愷不肯與議。後尚書案其事,二府並受譴咎,朝廷以此稱之。

視事五歲,永寧元年,稱病上書致仕,有詔優許焉,加賜錢三十萬,以千石祿歸養,河南尹常以歲八月致羊酒。時安帝始親政事,朝廷多稱愷之德,帝乃遣問起居,厚加賞賜。會馬英策罷,尚書陳忠上疏薦愷曰:「臣聞三公上則台階,下象山岳,股肱元首,鼎足居職,協和陰陽,調訓五品,考功量才,以序庶僚,遭烈風不迷,遇迅雨不惑,位莫重焉。而今上司缺職,未議其人。臣竊差次諸卿,考合眾議,咸稱太常朱倀、少府荀遷。臣父寵,前忝司空,倀、遷並為掾屬,具知其能。倀能說經書而用心褊狹,遷嚴毅剛直而薄於藝文。伏見前司徒劉愷,沈重淵懿,道德博備,克讓爵土,致祚弱弟,躬浮雲之志,兼浩然之氣,頻歷二司,舉動得禮。以疾致仕,側身里巷,處約思純,進退有度,百僚景式,海內歸懷。往者孔光、師丹,近世鄧彪、張酺,皆去宰相,復序上司。誠宜簡練卓異,以猒眾望。」書奏,詔引愷拜太尉。安帝初,清河相叔孫光坐臧抵罪,遂增錮二世,釁及其子。是時居延都尉范邠復犯臧罪,詔下三公、廷尉議。司徒楊震、司空陳褒、廷尉張皓議依光比。愷獨以為「春秋之義,『善善及子孫,惡惡止其身』,所以進人於善也。尚書曰:『上刑挾輕,下刑挾重。』如今使臧吏禁錮子孫,以輕從重,懼及善人,非先王詳刑之意也」。有詔:「太尉議是。」

視事三年,以疾乞骸骨,久乃許之,下河南尹禮秩如前。歲餘,卒于家。詔使者護喪事,賜東園祕器,錢五十萬,布千匹。

少子茂,字叔盛,亦好禮讓,歷位出納,桓帝時為司空。會司隸校尉李膺等抵罪,而南陽太守成档、太原太守劉暧下獄當死,茂與太尉陳蕃、司徒劉矩共上書訟之。帝不悅,有司承旨劾奏三公,茂遂坐免。建寧中,復為太中大夫,卒於官。

周磐字堅伯,汝南安成人,徵士燮之宗也。祖父業,建武初為天水太守。磐少游京師,學古文尚書、洪範五行、左氏傳,好禮有行,非典謨不言,諸儒宗之。居貧養母,儉薄不充。嘗誦詩至汝墳之卒章,慨然而歎,乃解韋帶,就孝廉之舉。和帝初,拜謁者,除任城長,遷陽夏、重合令,頻歷三城,皆有惠政。後思母,棄官還鄉里。及母歿,哀至幾於毀滅,服終,遂廬于冢側。教授門徒常千人。

公府三辟,皆以有道特徵,磐語友人曰:「昔方回、支父嗇神養和,不以榮利滑其生術。吾親以沒矣,從物何為?」遂不應。建光元年,年七十三,歲朝會集諸生,講論終日,因令其二子曰:「吾日者夢見先師東里先生,與我講於陰堂之奧。」既而長歎:「豈吾齒之盡乎!若命終之日,桐棺足以周身,外槨足以周棺,斂形懸封,濯衣幅巾。編二尺四寸簡,寫堯典一篇,并刀筆各一,以置棺前,云不忘聖道。」其月望日,無病忽終,學者以為知命焉。

磐同郡蔡順,字君仲,亦以至孝稱。順少孤,養母。嘗出求薪,有客卒至,母望順不還,乃噬其指,順即心動,棄薪馳歸,跪問其故。母曰:「有急客來,吾噬指以悟汝耳。」母年九十,以壽終。未及得葬,里中災,火將逼其舍,順抱伏棺柩,號哭叫天,火遂越燒它室,順獨得免。太守韓崇召為東閤祭酒。母平生畏雷,自亡後,每有雷震,順輒圜冢泣,曰:「順在此。」崇聞之,每雷輒為差車馬到墓所。後太守鮑眾舉孝廉,順不能遠離墳墓,遂不就。年八十,終于家。

趙咨字文楚,東郡燕人也。父暢,為博士。咨少孤,有孝行,州郡召舉孝廉,並不就。

延熹元年,大司農陳奇舉咨至孝有道,仍遷博士。靈帝初,太傅陳蕃、大將軍竇武為宦者所誅,咨乃謝病去。太尉楊賜特辟,使飾巾出入,請與講議。舉高第,累遷敦煌太守。以病免還,躬率子孫耕農為養。

盜嘗夜往劫之,咨恐母驚懼,乃先至門迎盜,因請為設食,謝曰:「老母八十,疾病須養,居貧,朝夕無儲,乞少置衣糧。」妻子物餘,一無所請。盜皆慚歎,跪而辭曰:「所犯無狀,干暴賢者。」言畢奔出,咨追以物與之,不及。由此益知名。徵拜議郎,辭疾不到,詔書切讓,州郡以禮發遣,前後再三,不得已應召。

復拜東海相。之官,道經滎陽,令敦煌曹暠,咨之故孝廉也,迎路謁候,咨不為留。暠送至亭次,望塵不及,謂主簿曰:「趙君名重,今過界不見,必為天下笑!」即棄印綬,追至東海。謁咨畢,辭歸家。其為時人所貴若此。

咨在官清簡,計日受奉,豪黨畏其儉節。視事三年,以疾自乞,徵拜議郎。抗疾京師,將終,告其故吏朱祇、蕭建等,使薄斂素棺,籍以黃壤,欲令速朽,早歸后土,不聽子孫改之。乃遺書敕子胤曰:「夫含氣之倫,有生必終,蓋天地之常期,自然之至數。是以通人達士,鑒茲性命,以存亡為晦明,死生為朝夕,故其生也不為娛,亡也不知戚。夫亡者,元氣去體,貞魂游散,反素復始,歸於無端。既已消仆,還合糞土。土為棄物,豈有性情,而欲制其厚薄,調其燥溼邪?但以生者之情,不忍見形之毀,乃有掩骼埋窆之制。《易》曰:『古之葬者,衣以薪,藏之中野,後世聖人易之以棺槨。』棺槨之造,自黃帝始。爰自陶唐,逮于虞、夏,猶尚簡樸,或瓦或木,及至殷人而有加焉。周室因之,制兼二代。復重以牆翣之飾,表以旌銘之儀,招復含斂之禮,殯葬宅兆之期,棺槨周重之制,衣衾稱襲之數,其事煩而害實,品物碎而難備。然而秩爵異級,貴賤殊等。自成、康以下,其典稍乖。至於戰國,漸至穨陵,法度衰毀,上下僭雜。終使晉侯請隧,秦伯殉葬,陳大夫設參門之木,宋司馬造石槨之奢。爰暨暴秦,違道廢德,滅三代之制,興淫邪之法,國貲糜於三泉,人力單於酈墓,玩好窮於糞土,伎巧費於窀穸。自生民以來,厚終之敝,未有若此者。雖有仲尼重明周禮,墨子勉以古道,猶不能禦也。是以華夏之士,爭相陵尚,違禮之本,事禮之末,務禮之華,棄禮之實,單家竭財,以相營赴。廢事生而營終亡,替所養而為厚葬,豈云聖人制禮之意乎?記曰:『喪雖有禮,哀為主矣。』又曰:『喪與其易也寧戚。』今則不然,并棺合槨,以為孝愷,豐貲重襚,以昭惻隱,吾所不取也。昔舜葬蒼梧,二妃不從。豈有匹配之會,守常之所乎?聖主明王,其猶若斯,況於品庶,禮所不及。古人時同即會,時乖則別,動靜應禮,臨事合宜。王孫裸葬,墨夷露骸,皆達於性理,貴於速變。梁伯鸞父沒,卷席而葬,身亡不反其尸。彼數子豈薄至親之恩,亡忠孝之道邪?況我鄙闇,不德不敏,薄意內昭,志有所慕,上同古人,下不為咎。果必行之,勿生疑異。恐爾等目猒所見,耳諱所議,必欲改殯,以乖吾志,故遠采古聖,近揆行事,以悟爾心。但欲制坎,令容棺槨,棺歸即葬,平地無墳。勿卜時日,葬無設奠,勿留墓側,無起封樹。於戲小子,其勉之哉,吾蔑復有言矣!」朱祇、蕭建送喪到家,子胤不忍父體與土并合,欲更改殯,祇、建譬以顧命,於是奉行,時稱咨明達。

贊曰:公子、長平,臨寇讓生。淳于仁悌,「巨孝」以名。居巢好讀,遂承家祿。伯豫逡巡,方跡孤竹。文楚薄終,喪朽惟速。周能感親,嗇神養福。


\end{pinyinscope}