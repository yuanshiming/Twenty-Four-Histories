\article{杜欒劉李劉謝列傳}

\begin{pinyinscope}
杜根字伯堅,潁川定陵人也。父安,字伯夷,少有志節,年十三入太學,號奇童。京師貴戚慕其名,或遺之書,安不發,悉壁藏之。及後捕案貴戚賓客,安開壁出書,印封如故,竟不離其患,時人貴之。位至巴郡太守,政甚有聲。

根性方實,好絞直。永初元年,舉孝廉,為郎中。時和熹鄧后臨朝,權在外戚。根以安帝年長,宜親政事,乃與同時郎上書直諫。太后大怒,收執根等,令盛以縑囊,於殿上撲殺之。執法者以根知名,私語行事人使不加力,既而載出城外,根得蘇。太后使人檢視,根遂詐死,三日,目中生蛆,因得逃竄,為宜城山中酒家保。積十五年,酒家知其賢,厚敬待之。

及鄧氏誅,左右皆言根等之忠。帝謂根已死,乃下詔布告天下,錄其子孫。根方歸鄉里,徵詣公車,拜侍御史。初,平原郡吏成翊世亦諫太后歸政,坐抵罪,與根俱徵,擢為尚書郎,並見納用。或問根曰:「往者遇禍,天下同義,知故不少,何至自苦如此?」根曰:「周旋民閒,非絕跡之處,邂逅發露,禍及知親,故不為也。」順帝時,稍遷濟陰太守。去官還家,年七十八卒。

翊世字季明,少好學,深明道術。延光,中常侍樊豐、帝乳母王聖共譖皇太子,廢為濟陰王。翊世連上書訟之,又言樊豐、王聖誣罔之狀。帝既不從,而豐等陷以重罪,下獄當死,有詔免官歸本郡。及濟陰王立,是為順帝,司空張皓辟之。皓以翊世前訟太子之廢,薦為議郎。翊世自以其功不顯,恥於受位,自劾歸。三公比辟,不應。尚書僕射虞詡雅重之,欲引與共參朝政,乃上書薦之,徵拜議郎。後尚書令左雄、僕射郭虔復舉為尚書。在朝正色,百僚敬之。

欒巴字叔元,魏郡內黃人也。順帝世,以宦者給事掖庭,補黃門令,非其好也。性質直,學覽經典,雖在中官,不與諸常侍交接。後陽氣通暢,白上乞退,擢拜郎中,四遷桂楊太守。以郡處南垂,不閑典訓,為吏人定婚姻喪紀之禮,興立校學,以獎進之。雖幹吏卑末,皆課令習讀,程試殿最,隨能升授。政事明察。視事七年,以病乞骸骨。

荊州刺史李固薦巴治跡,徵拜議郎,守光祿大夫,與杜喬、周舉等八人徇行州郡。

巴使徐州還,再遷豫章太守。郡土多山川鬼怪,小人常破貲產以祈禱。巴素有道術,能役鬼神,乃悉毀壞房祀,翦理姦巫,於是妖異自消。百姓始頗為懼,終皆安之。遷沛相。所在有績,徵拜尚書。會帝崩,營起憲陵。陵左右或有小人墳冢,主者欲有所侵毀,巴連上書苦諫。時梁太后臨朝,詔詰巴曰:「大行皇帝晏駕有日,卜擇陵園,務從省約,塋域所極,裁二十頃,而巴虛言主者壞人冢墓。事既非實,寢不報下,巴猶固遂其愚,復上誹謗。苟肆狂瞽,益不可長。」巴坐下獄,抵罪,禁錮還家。

二十餘年,靈帝即位,大將軍竇武、太傅陳蕃輔政,徵拜議郎。蕃、武被誅,巴以其黨,復謫為永昌太守。以功自劾,辭病不行,上書極諫,理陳、竇之冤。帝怒,下詔切責,收付廷尉。巴自殺。子賀,官至雲中太守。

劉陶字子奇,一名偉,潁川潁陰人,濟北貞王勃之後。陶為人居簡,不脩小節。所與交友,必也同志。好尚或殊,富貴不求合;情趣苟同,貧賤不易意。同宗劉愷,以雅德知名,獨深器陶。

時大將軍梁冀專朝,而桓帝無子,連歲荒飢,災異數見。陶時游太學,乃上疏陳事曰:

臣聞人非天地無以為生,天地非人無以為靈,是故帝非人不立,人非帝不寧。夫天之與帝,帝之與人,猶頭之與足,相須而行也。伏惟陛下年隆德茂,中天稱號,襲常存之慶,循不易之制,目不視鳴條之事,耳不聞檀車之聲,天災不有痛於肌膚,震食不即損於聖體,故蔑三光之謬,輕上天之怒。伏念高祖之起,始自布衣,拾暴秦之敝,追亡周之鹿,合散扶傷,克成帝業。功既顯矣,勤亦至矣。流福遺祚,至於陛下。陛下既不能增明烈考之軌,而忽高祖之勤,妄假利器,委授國柄,使群醜刑隸,芟刈小民,彫敝諸夏,虐流遠近,故天降眾異,以戒陛下。陛下不悟,而競令虎豹窟於麑場,豺狼乳於春囿。斯豈唐咨禹、稷,益典朕虞,議物賦土蒸民之意哉?又令牧守長吏,上下交競;封豕長蛇,蠶食天下;貨殖者為窮冤之魂,貧餒者作飢寒之鬼;高門獲東觀之辜,豐室羅妖叛之罪;死者悲於窀穸,生者戚於朝野;是愚臣所為咨嗟長懷歎息者也。且秦之將亡,正諫者誅,諛進者賞,嘉言結於忠舌,國命出於讒口,擅閻樂於咸陽,授趙高以車府。權去己而不知,威離身而不顧。古今一揆,成敗同埶。願陛下遠覽強秦之傾,近察哀、平之變,得失昭然,禍福可見。

臣又聞危非仁不扶,亂非智不救,故武丁得傅說,以消鼎雉之災,周宣用申、甫,以濟夷、厲之荒。竊見故冀州刺史南陽朱穆,前烏桓校尉臣同郡李膺,皆履正清平,貞高絕俗。穆前在冀州,奉憲操平,摧破姦黨,掃清萬里。膺歷典牧守,正身率下,及掌戎馬,威揚朔北。斯實中興之良佐,國家之柱臣也。宜還本朝,挾輔王室,上齊七燿,下鎮萬國。臣敢吐不時之義於諱言之朝,猶冰霜見日,必至消滅。臣始悲天下之可悲,今天下亦悲臣之愚惑也。

書奏不省。

時有上書言人以貨輕錢薄,故致貧困,宜改鑄大錢。事下四府群僚及太學能言之士。陶上議曰:

聖王承天制物,與人行止,建功則眾悅其事,興戎而師樂其旅。是故靈臺有子來之人,武旅有鳧藻之士,皆舉合時宜,動順人道也。臣伏讀鑄錢之詔,平輕重之議,訪覃幽微,不遺窮賤,是以藿食之人,謬延逮及。

蓋以為當今之憂,不在於貨,在乎民飢。夫生養之道,先食後民。是以先王觀象育物,敬授民時,使男不逋畝,女不下機。故君臣之道行,王路之教通。由是言之,食者乃有國之所寶,生民之至貴也。竊見比年已來,良苗盡於蝗螟之口,杼柚空於公私之求,所急朝夕之餐,所患靡盬之事,豈謂錢貨之厚薄,銖兩之輕重哉?就使當今沙礫化為南金,瓦石變為和玉,使百姓渴無所飲,飢無所食,雖皇羲之純德,唐虞之文明,猶不能以保蕭牆之內也。蓋民可百年無貨,不可一朝有飢,故食為至急也。議者不達農殖之本,多言鑄冶之便,或欲因緣行詐,以賈國利。國利將盡,取者爭競,造鑄之端於是乎生。蓋萬人鑄之,一人奪之,猶不能給;況今一人鑄之,則萬人奪之乎?雖以陰陽為炭,萬物為銅,役不食之民,使不飢之士,猶不能足無猒之求也。夫欲民殷財阜,要在止役禁奪,則百姓不勞而足。陛下聖德,愍海內之憂戚,傷天下之艱難,欲鑄錢齊貨以救其敝,此猶養魚沸鼎之中,棲鳥烈火之上。水木本魚鳥之所生也,用之不時,必至燋爛。願陛下寬鍥薄之禁,後冶鑄之議,聽民庶之謠吟,問路叟之所憂,瞰三光之文耀,視山河之分流。天下之心,國家大事,粲然皆見,無有遺惑者矣。

臣嘗誦詩,至於鴻鴈于野之勞,哀勤百堵之事,每喟爾長懷,中篇而歎。近聽征夫飢勞之聲,甚於斯歌。是以追悟匹婦吟魯之憂,始於此乎?見白駒之意,屏營傍偟,不能監寐。伏念當今地廣而不得耕,民眾而無所食。群小競進,秉國之位,鷹揚天下,鳥鈔求飽,吞肌及骨,並噬無猒。誠恐卒有役夫窮匠,起於板築之閒,投斤攘臂,登高遠呼,使愁怨之民,嚮應雲合,八方分崩,中夏魚潰。雖方尺之錢,何能有救!其危猶舉函牛之鼎,絓纖枯之末,詩人所以眷然顧之,潸焉出涕者也。

臣東野狂闇,不達大義,緣廣及之時,對過所問,知必以身脂鼎鑊,為天下笑。

帝竟不鑄錢。

後陶舉孝廉,除順陽長。縣多姦猾,陶到官,宣募吏民有氣力勇猛,能以死易生者,不拘亡命姦臧,於是剽輕劍客之徒過晏等十餘人,皆來應募。陶責其先過,要以後效,使各結所厚少年,得數百人,皆嚴兵待命。於是覆案姦軌,所發若神。以病免,吏民思而歌之曰:「邑然不樂,思我劉君。何時復來,安此下民。」

陶明尚書、春秋,為之訓詁。推三家尚書及古文,是正文字七百餘事,名曰中文尚書。

頃之,拜侍御史。靈帝宿聞其名,數引納之。時鉅鹿張角偽託大道,妖惑小民,陶與奉車都尉樂松、議郎袁貢連名上疏言之,曰:「

聖王以天下耳目為視聽,故能無不聞見。今張角支黨不可勝計。前司徒楊賜奏下詔書,切敕州郡,護送流民,會賜去位,不復捕錄。唯會赦令,而謀不解散。四方私言,云角等竊入京師,覘視朝政,鳥聲獸心,私共鳴呼。州郡忌諱,不欲聞之,但更相告語,莫肯公文。宜下明詔,重募角等,賞以國土。有敢回避,與之同罪。」帝殊不悟,方詔陶次第春秋條例。明年,張角反亂,海內鼎沸,帝思陶言,封中陵鄉侯,三遷尚書令。以所舉將為尚書,難與齊列,乞從冗散,拜侍中。以數切諫,為權臣所憚,徙為京兆尹。到職,當出脩宮錢直千萬,陶既清貧,而恥以錢買職,稱疾不聽政。帝宿重陶才,原其罪,徵拜諫議大夫。

是時天下日危,寇賊方熾,陶憂致崩亂,復上疏曰:「臣聞事之急者不能安言,心之痛者不能緩聲。竊見天下前遇張角之亂,後遭邊章之寇,每聞羽書告急之聲,心灼內熱,四體驚竦。今西羌逆類,私署將帥,皆多段熲時吏,曉習戰陳,識知山川,變詐萬端。臣常懼其輕出河東、馮翊,鈔西軍之後,東之函谷,據阨高望。今果已攻河東,恐遂轉更豕突上京。如是則南道斷絕,車騎之軍孤立,關東破膽,四方動搖,威之不來,叫之不應,雖有田單、陳平之策,計無所用。臣前驛馬上便宜,急絕諸郡賦調,冀尚可安。事付主者,留連至今,莫肯求問。今三郡之民皆以奔亡,南出武關,北徙壺谷,冰解風散,唯恐在後。今其存者尚十三四,軍吏士民悲愁相守,民有百走退死之心,而無一前鬥生之計。西寇浸前,去營咫尺,胡騎分布,已至諸陵。將軍張溫,天性精勇,而主者旦夕迫促,軍無後殿,假令失利,其敗不救。臣自知言數見厭,而言不自裁者,以為國安則臣蒙其慶,國危則臣亦先亡也。謹復陳當今要急八事,乞須臾之閒,深垂納省。」其八事,大較言天下大亂,皆由宦官。宦官事急,共讒陶曰:「前張角事發,詔書示以威恩,自此以來,各各改悔。今者四方安靜,而陶疾害聖政,專言妖孽。州郡不上,陶何緣知?疑陶與賊通情。」於是收陶,下黃門北寺獄,掠按日急。陶自知必死,對使者曰:「朝廷前封臣云何?今反受邪譖。恨不與伊、呂同疇,而以三仁為輩。」遂閉氣而死,天下莫不痛之。

陶著書數十萬言,又作七曜論、匡老子、反韓非、復孟軻,及上書言當世便事、條教、賦、奏、書、記、辯疑,凡百餘篇。

時司徒東海陳耽,亦以非罪與陶俱死。耽以忠正稱,歷位三司。光和五年,詔公卿以謠言舉刺史、二千石為民蠹害者。時太尉許戫、司空張濟承望內官,受取貨賂,其宦者子弟賓客,雖貪汙穢濁,皆不敢問,而虛糾邊遠小郡清脩有惠化者二十六人。吏人詣闕陳訴,耽與議郎曹操上言:「公卿所舉,率黨其私,所謂放鴟梟而囚鸞鳳。」其言忠切,帝以讓戫、濟,由是諸坐謠言徵者悉拜議郎。宦官怨之,遂誣陷耽死獄中。

李雲字行祖,甘陵人也。性好學,善陰陽。初舉孝廉,再遷白馬令。

桓帝延熹二年,誅大將軍梁冀,而中常侍單超等五人皆以誅冀功並封列侯,專權選舉。又立掖庭民女亳氏為皇后,數月閒,后家封者四人,賞賜巨萬。是時地數震裂,眾災頻降。雲素剛,憂國將危,心不能忍,乃露布上書,移副三府,曰:「臣聞皇后天下母,德配坤靈,得其人則五氏來備,不得其人則地動搖宮。比年災異,可謂多矣,皇天之戒,可謂至矣。高祖受命,至今三百六十四歲,君期一周,當有黃精代見,姓陳、項、虞、田、許氏,不可令此人居太尉、太傅典兵之官。舉厝至重,不可不慎。班功行賞,宜應其實。梁冀雖持權專擅,虐流天下,今以罪行誅,猶召家臣搤殺之耳。而猥封謀臣萬戶以上,高祖聞之,得無見非?西北列將,得無解體?孔子曰:『帝者,諦也。』今官位錯亂,小人諂進,財貨公行,政化日損,尺一拜用不經御省。是帝欲不諦乎?」帝得奏震怒,下有司逮雲,詔尚書都護劍戟送黃門北寺獄,使中常侍管霸與御史廷尉雜考之。時弘農五官掾杜眾傷雲以忠諫獲罪,上書願與雲同日死。帝愈怒,遂并下廷尉。大鴻臚陳蕃上疏救雲曰:「李雲所言,雖不識禁忌,干上逆旨,其意歸於忠國而已。昔高祖忍周昌不諱之諫,成帝赦朱雲腰領之誅。今日殺雲,臣恐剖心之譏復議於世矣。故敢觸龍鱗,冒昧以請。」太常楊秉、洛陽市長沐茂、郎中上官資並上疏請雲。帝恚甚,有司奏以為大不敬。詔切責蕃、秉,免歸田里;茂、資貶秩二等。時帝在濯龍池,管霸奏雲等事。霸跪言曰:「李雲野澤愚儒,杜眾郡中小吏,出於狂戇,不足加罪。」帝謂霸曰:「帝欲不諦,是何等語,而常侍欲原之邪?」顧使小黃門可其奏,雲、眾皆死獄中。後冀州刺史賈琮使行部,過祠雲墓,刻石表之。

論曰:禮有五諫,諷為上。若夫託物見情,因文載旨,使言之者無罪,聞之者足以自戒,貴在於意達言從,理歸乎正。曷其絞訐摩上,以衒沽成名哉?李雲草茅之生,不識失身之義,遂乃露布帝者,班檄三公,至於誅死而不顧,斯豈古之狂也!夫未信而諫,則以為謗己,故說者識其難焉。

劉瑜字季節,廣陵人也。高祖父廣陵靖王。父辯,清河太守。瑜少好經學,尤善圖讖、天文、歷筭之術。州郡禮請不就。

延熹八年,太尉楊秉舉賢良方正,及到京師,上書陳事曰:

臣瑜自念東國鄙陋,得以豐沛枝胤,被蒙復除,不給卒伍。故太尉楊秉知臣竊闚典籍,猥見顯舉,誠冀臣愚直,有補萬一。而秉忠謨不遂,命先朝露。臣在下土,聽聞歌謠,驕臣虐政之事,遠近呼嗟之音,竊為辛楚,泣血漣如。幸得引錄,備荅聖問,泄寫至情,不敢庸回。誠願陛下且以須臾之慮,覽今往之事,人何為咨嗟,天曷為動變。

蓋諸侯之位,上法四七,垂文炳燿,關之盛衰者也。今中官邪孽,比肩裂土,皆競立胤嗣,繼體傳爵,或乞子疏屬,或買兒市道,殆乖開國承家之義。

古者天子一娶九女,娣姪有序,河圖授嗣,正在九房。今女嬖令色,充積閨帷,皆當盛其玩飾冗食空宮,勞散精神,生長六疾。此國之費也,生之傷也。且天地之性,陰陽正紀,隔絕其道,則水旱為并。《詩》云:「五日為期,六日不詹。」怨曠作歌,仲尼所錄。況從幼至長,幽藏歿身。又常侍、黃門,亦廣妻娶。怨毒之氣,結成妖眚。行路之言,官發略人女,取而復置,轉相驚懼。孰不悉然,無緣空生此謗。鄒衍匹夫,杞氏匹婦,尚有城崩霜隕之異;況乃群輩咨怨,能無感乎!

昔秦作阿房,國多刑人。今第舍增多,窮極奇巧,掘山攻石,不避時令。促以嚴刑,威以法正。民無罪而覆入之,民有田而覆奪之。州郡官府,各自考事,姦情賕賂,皆為吏餌。民悉鬱結,起入賊黨,官輒興兵,誅討其罪。貧困之民,或有賣其首級以要酬賞,父兄相代殘身,妻孥相見分裂。窮之如彼,伐之如此,豈不痛哉!

又陛下以北辰之尊,神器之寶,而微行近習之家,私幸宦者之舍,賓客市買,熏灼道路,因此暴縱,無所不容。今三公在位,皆博達道蓺,而各正諸己,莫或匡益者,非不智也,畏死罰也。惟陛下設置七臣,以廣諫道,及開東序金縢史官之書,從堯舜禹湯文武致興之道,遠佞邪之人,放鄭衛之聲,則政致和平,德感祥風矣。臣悾悾推情,言不足採,懼以觸忤,征營慴悸。

於是特詔召瑜問災咎之徵,指事案經讖以對。執政者欲令瑜依違其辭,而更策以它事。瑜復悉心以對,八千餘言,有切於前,帝竟不能用。拜為議郎。

及帝崩,大將軍竇武欲大誅宦官,乃引瑜為侍中,又以侍中尹勳為尚書令,共同謀畫。及武敗,瑜、勳並被誅。事在武傳。

勳字伯元,河南人。從祖睦為太尉,睦孫頌為司徒。勳為人剛毅直方。少時每讀書,得忠臣義士之事,未嘗不投書而仰歎。自以行不合於當時,不應州郡公府禮命。桓帝時,以有道徵,四遷尚書令。延熹中,誅大將軍梁冀,帝召勳部分眾職,甚有方略,封宜陽鄉侯。僕射霍諝,尚書張敬、歐陽參、李偉、虞放、周永,並封亭侯。勳後再遷至九卿,以病免,拜為侍中。八年,中常侍具瑗、左悺等有罪免,奪封邑,因黜勳等爵。

瑜誅後,宦官悉焚其上書,以為訛言。

子琬,傳瑜學,明占候,能著災異。舉方正,不行。

謝弼字輔宣,東郡武陽人也。中直方正,為鄉邑所宗師。建寧二年,詔舉有道之士,弼與東海陳敦、玄菟公孫度俱對策,皆除郎中。

時青蛇見前殿,大風拔木,詔公卿以下陳得失。弼上封事曰:

臣聞和氣應於有德,妖異生乎失政。上天告譴,則王者思其愆;政道或虧,則姦臣當其罰。夫蛇者,陰氣所生;鱗者,甲兵之符也。鴻範傳曰:「厥極弱,時則有蛇龍之孽。」又熒惑守亢,裴回不去,法有近臣謀亂,發於左右。不知陛下所與從容帷幄之內,親信者為誰。宜急斥黜,以消天戒。臣又聞「惟虺惟蛇,女子之祥」。伏惟皇太后定策宮闥,援立聖明,書云:「父子兄弟,罪不相及。」竇氏之誅,豈宜咎延太后?幽隔空宮,愁感天心,如有霧露之疾,陛下當何面目以見天下?昔周襄王不能敬事其母,戎狄遂至交侵。孝和皇帝不絕竇后之恩,前世以為美談。禮為人後者為之子,今以桓帝為父,豈得不以太后為母哉?援神契曰:「天子行孝,四夷和平。」方今邊境日蹙,兵革蜂起,自非孝道,何以濟之!願陛下仰慕有虞蒸蒸之化,俯思凱風慰母之念。

臣又聞爵賞之設,必酬庸勳;開國承家,小人勿用。今功臣久外,未蒙爵秩,阿母寵私,乃享大封,大風雨雹,亦由於茲。又故太傅陳蕃,輔相陛下,勤身王室,夙夜匪懈,而見陷群邪,一旦誅滅。其為酷濫,駭動天下,而門生故吏,並離徙錮。蕃身已往,人百何贖!宜還其家屬,解除禁網。夫台宰重器,國命所繼。今之四公,唯司空劉寵斷斷守善,餘皆素餐致寇之人,必有折足覆餗之凶。可因災異,並加罷黜。徵故司空王暢,長樂少府李膺,並居政事,庶災變可消,國祚惟永。臣山藪頑闇,未達國典。策曰「無有所隱」,敢不盡愚,用忘諱忌。伏惟陛下裁其誅罰。

左右惡其言,出為廣陵府丞。去官歸家。

中常侍曹節從子紹為東郡太守,忿疾於弼,遂以它罪收考掠按,死獄中,時人悼傷焉。初平二年,司隸校尉趙謙訟弼忠節,求報其怨,乃收紹斬之。

贊曰:鄧不明辟。梁不損陵。慊慊欒、杜,諷辭以興。黃寇方熾,子奇有識。武謀允臧,瑜亦協志。弼忤宦情,雲犯時忌。成仁喪己,同方殊事。


\end{pinyinscope}