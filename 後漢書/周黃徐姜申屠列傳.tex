\article{周黃徐姜申屠列傳}

\begin{pinyinscope}
《易》曰:「君子之道,或出或處,或默或語。」孔子稱「蘧伯玉邦有道則仕,邦無道則可卷而懷也」。然用舍之端,君子之所以存其誠也。故其行也,則濡足蒙垢,出身以效時;及其止也,則窮棲茹菽,臧寶以迷國。

太原閔仲叔者,世稱節士,雖周黨之潔清,自以弗及也。黨見其含菽飲水,遺以生蒜,受而不食。建武中,應司徒侯霸之辟,既至,霸不及政事,徒勞苦而已。仲叔恨曰:「始蒙嘉命,且喜且懼;今見明公,喜懼皆去。以仲叔為不足問邪,不當辟也。辟而不問,是失人也。」遂辭出,投劾而去。復以博士徵,不至。客居安邑。老病家貧,不能得肉,日買豬肝一片,屠者或不肯與,安邑令聞,敕吏常給焉。仲叔怪而問之,知,乃歎曰:「閔仲叔豈以口腹累安邑邪?」遂去,客沛。以壽終。

仲叔同郡荀恁,字君大,少亦脩清節。資財千萬,父越卒,悉散與九族。隱居山澤,以求厥志。王莽末,匈奴寇其本縣廣武,聞恁名節,相約不入荀氏閭。光武徵,以病不至。永平初,東平王蒼為驃騎將軍,開東閤延賢俊,辟而應焉。及後朝會,顯宗戲之曰:「先帝徵君不至,驃騎辟君而來,何也?」對曰:「先帝秉德以惠下,故臣可得不來。驃騎執法以檢下,故臣不敢不至。」後月餘,罷歸,卒於家。

桓帝時,安陽人魏桓,字仲英,亦數被徵。其鄉人勸之行。桓曰:「夫干祿求進,所以行其志也。今後宮千數,其可損乎?廄馬萬匹,其可減乎?左右悉權豪,其可去乎?」皆對曰:「不可。」桓乃慨然歎曰:「使桓生行死歸,於諸子何有哉!」遂隱身不出。

若二三子,可謂識去就之概,候時而處。夫然,豈其枯槁苟而已哉?蓋詭時審己,以成其道焉。余故列其風流,區而載之。

周燮字彥祖,汝南安城人,法曹掾燕之後也。燮生而欽頤折頞,醜狀駭人。其母欲棄之,其父不聽,曰:「吾聞賢聖多有異貌。興我宗者,乃此兒也。」於是養之。

始在髫鬌,而知廉讓;十歲就學,能通詩、論;及長,專精禮、易。不讀非聖之書,不脩賀問之好。有先人草廬結于罔畔,下有陂田,常肆勤以自給。非身所耕漁,則不食也。鄉黨宗族希得見者。

舉孝廉、賢良方正,特徵,皆以疾辭,延光二年,安帝以玄纁羔幣聘燮,及南陽馮良,二郡各遣丞掾致禮。宗族更勸之曰:「

夫修德立行,所以為國。自先世以來,勳寵相承,君獨何為守東岡之陂乎?」燮曰:「吾既不能隱處巢穴,追綺季之跡,而猶顯然不遠父母之國,斯固以滑泥揚波,同其流矣。夫修道者,度其時而動。動而不時,焉得亨乎!」因自載到潁川陽城,遣生送敬,遂辭疾而歸。良亦載病到近縣,送禮而還。詔書告二郡,歲以羊酒養病。

良字君郎。出於孤微,少作縣吏。年三十,為尉從佐。奉檄迎督郵,即路慨然,恥在冢役,因壞車殺馬,毀裂衣冠,乃遁至犍為,從杜撫學。妻子求索,蹤跡斷絕。後乃見草中有敗車死馬,衣裳腐朽,謂為虎狼盜賊所害,發喪制服。積十許年,乃還鄉里。志行高整,非禮不動,遇妻子如君臣,鄉黨以為儀表。燮、良年皆七十餘終。

黃憲字叔度,汝南慎陽人也。世貧賤,父為牛醫。

潁川荀淑至慎陽,遇憲於逆旅,時年十四,淑竦然異之,揖與語,移日不能去。謂憲曰:「子,吾之師表也。」既而前至袁閎所,未及勞問,逆曰:「子國有顏子,寧識之乎?」閎曰:「見吾叔度邪?」是時,同郡戴良才高倨傲,而見憲未嘗不正容,及歸,罔然若有失也。其母問曰:「汝復從牛醫兒來邪?」對曰:「良不見叔度,不自以為不及;既睹其人,則瞻之在前,忽焉在後,固難得而測矣。」同郡陳蕃、周舉常相謂曰:「時月之閒不見黃生,則鄙吝之萌復存乎心。」及蕃為三公,臨朝歎曰:「叔度若在,吾不敢先佩印綬矣。」太守王龔在郡,禮進賢達,多所降致,卒不能屈憲。郭林宗少游汝南,先過袁閎,不宿而退;進往從憲,累日方還。或以問林宗。林宗曰:「奉高之器,譬諸汎濫,雖清而易挹。叔度汪汪若千頃陂,澄之不清,淆之不濁,不可量也。」

憲初舉孝廉,又辟公府,友人勸其仕,憲亦不拒之,暫到京師而還,竟無所就。年四十八終,天下號曰「徵君」。

論曰:黃憲言論風旨,無所傳聞,然士君子見之者,靡不服深遠,去玼吝。將以道周性全,無德而稱乎?余曾祖穆侯以為憲隤然其處順,淵乎其似道,淺深莫臻其分,清濁未議其方。若及門於孔氏,其殆庶乎!故嘗著論云。

徐稚字孺子,豫章南昌人也。家貧,常自耕稼,非其力不食。恭儉義讓,所居服其德。屢辟公府,不起。

時陳蕃為太守,以禮請署功曹,稚不免之,既謁而退。蕃在郡不接賓客,唯稚來特設一榻,去則縣之。後舉有道,家拜太原太守,皆不就。

延熹二年,尚書令陳蕃、僕射胡廣等上疏薦稚等曰:「臣聞善人天地之紀,政之所由也。《詩》云:『思皇多士,生此王國。』天挺俊乂,為陛下出,當輔弼明時,左右大業者也。伏見處士豫章徐稚、彭城姜肱、汝南袁閎、京兆韋著、潁川李曇,德行純備,著于人聽。若使擢登三事,協亮天工,必能翼宣盛美,增光日明矣。」桓帝乃以安車玄纁,備禮徵之,並不至。帝因問蕃曰:「徐稚、袁閎、韋著誰為先後?」蕃對曰:「閎出生公族,聞道漸訓。著長於三輔禮義之俗,所謂不扶自直,不鏤自雕。至於稚者,爰自江南卑薄之域,而角立傑出,宜當為先。」

稚嘗為太尉黃瓊所辟,不就。及瓊卒歸葬,稚乃負糧徒步到江夏赴之,設雞酒薄祭,哭畢而去,不告姓名。時會者四方名士郭林宗等數十人,聞之,疑其稚也,乃選能言語生茅容輕騎追之。及於塗,容為設飯,共言稼穡之事。臨訣去,謂容曰:「為我謝郭林宗,大樹將顛,非一繩所維,何為栖栖不遑寧處?」及林宗有母憂,稚往弔之,置生芻一束於廬前而去。眾怪,不知其故。林宗曰:「此必南州高士徐孺子也。詩不云乎,『生芻一束,其人如玉。』吾無德以堪之。」

靈帝初,欲蒲輪聘稚,會卒,時年七十二。

子胤字季登,篤行孝悌,亦隱居不仕。太守華歆禮請相見,固病不詣。漢末寇賊從橫,皆敬胤禮行,轉相約敕,不犯其閭。建安中卒。

李曇字雲,少孤,繼母嚴酷,曇事之愈謹,為鄉里所稱法。養親行道,終身不仕。

姜肱字伯淮,彭城廣戚人也。家世名族。肱與二弟仲海、季江,俱以孝行著聞。其友愛天至,常共臥起。及各娶妻,兄弟相戀,不能別寑,以係嗣當立,乃遞往就室。

肱博通五經,兼明星緯,士之遠來就學者三千餘人。諸公爭加辟命,皆不就。二弟名聲相次,亦不應徵聘,時人慕之。

肱嘗與季江謁郡,夜於道遇盜,欲殺之。肱兄弟更相爭死,賊遂兩釋焉,但掠奪衣資而已。既至郡中,見肱無衣服,怪問其故,肱託以它辭,終不言盜。盜聞而感悔,後乃就精廬,求見徵君。肱與相見,皆叩頭謝罪,而還所略物。肱不受,勞以酒食而遣之。

後與徐稚俱徵,不至。桓帝乃下彭城使畫工圖其形狀。肱臥於幽闇,以被韜面,言患眩疾,不欲出風。工竟不得見之。

中常侍曹節等專執朝事,新誅太傅陳蕃、大將軍竇武,欲借寵賢德,以釋眾望,乃白徵肱為太守。肱得詔,乃私告其友曰:「吾以虛獲實,遂藉聲價。明明在上,猶當固其本志,況今政在閹豎,夫何為哉!」乃隱身遯命,遠浮海濱。再以玄纁聘,不就。即拜太中大夫,詔書至門,肱使家人對云「久病就醫」。遂羸服閒行,竄伏青州界中,賣卜給食。召命得斷,家亦不知其處,歷年乃還。年七十七,熹平二年終于家。弟子陳留劉操追慕肱德,共刊石頌之。

申屠蟠字子龍,陳留外黃人也。九歲喪父,哀毀過禮。服除,不進酒肉十餘年。每忌日,輒三日不食。

同郡緱氏女玉為父報讎,殺夫氏之黨,吏執玉以告外黃令梁配,配欲論殺玉。蟠時年十五,為諸生,進諫曰:「玉之節義,足以感無恥之孫,激忍辱之子。不遭明時,尚當表旌廬墓,況在清聽,而不加哀矜!」配善其言,乃為讞得減死論。鄉人稱美之。

家貧,傭為漆工。郭林宗見而奇之。同郡蔡邕深重蟠,及被州辟,乃辭讓之曰:「申屠蟠稟氣玄妙,性敏心通,喪親盡禮,幾於毀滅。至行美義,人所鮮能。安貧樂潛,味道守真,不為燥濕輕重,不為窮達易節。方之於邕,以齒則長,以德則賢。」

後郡召為主簿,不行。遂隱居精學,博貫五經,兼明圖緯。始與濟陰王子居同在太學,子居臨歿,以身託蟠,蟠乃躬推輦車,送喪歸鄉里。遇司隸從事於河鞏之閒,從事義之,為封傳護送,蟠不肯受,投傳於地而去。事畢還學。

太尉黃瓊辟,不就。及瓊卒,歸葬江夏,四方名豪會帳下者六七千人,互相談論,莫有及蟠者。唯南郡一生與相酬對,既別,執蟠手曰:「君非聘則徵,如是相見於上京矣。」蟠勃然作色曰:「始吾以子為可與言也,何意乃相拘教樂貴之徒邪?」因振手而去,不復與言。再舉有道,不就。

先是京師游士汝南范滂等非訐朝政,自公卿以下皆折節下之。太學生爭慕其風,以為文學將興,處士復用。蟠獨歎曰:「昔戰國之世,處士橫議,列國之王,至為擁篲先驅,卒有阬儒燒書之禍,今之謂矣。」乃絕跡於梁碭之閒,因樹為屋,自同傭人。居二年,滂等果罹黨錮,或死或刑者數百人,蟠確然免於疑論。後蟠友人陳郡馮雍坐事繫獄,豫州牧黃琬欲殺之。或勸蟠救雍,蟠不肯行,曰:「黃子琰為吾故邪,未必合罪。如不用吾言,雖往何益!」琬聞之,遂免雍罪。

大將軍何進連徵不詣,進必欲致之,使蟠同郡黃忠書勸曰:「前莫府初開,至如先生,特加殊禮,優而不名,申以手筆,設几杖之坐。經過二載,而先生抗志彌高,所尚益固。竊論先生高節有餘,於時則未也。今潁川荀爽載病在道,北海鄭玄北面受署。彼豈樂羈牽哉,知時不可逸豫也。昔人之隱,遭時則放聲滅跡,巢棲茹薇。其不遇也,則裸身大笑,被髮狂歌。今先生處平壤,游人閒,吟典籍,襲衣裳,事異昔人,而欲遠蹈其跡,不亦難乎!孔氏可師,何必首陽。」蟠不荅。

中平五年,復與爽、玄及潁川韓融、陳紀等十四人並博士徵,不至。明年,董卓廢立,蟠及爽、融、紀等復俱公車徵,唯蟠不到。眾人咸勸之,蟠笑而不應。居無幾,爽等為卓所脅迫,西都長安,京師擾亂。及大駕西遷,公卿多遇兵飢,室家流散,融等僅以身脫。唯蟠處亂末,終全高志。年七十四,終于家。

贊曰:琛寶可懷,貞期難對。道苟違運,理用同廢。與其遐棲,豈若蒙穢?悽悽碩人,陵阿窮退。韜伏明姿,甘是堙曖。


\end{pinyinscope}