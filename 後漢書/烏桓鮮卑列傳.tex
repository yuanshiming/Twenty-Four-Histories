\article{烏桓鮮卑列傳}

\begin{pinyinscope}
烏桓者,本東胡也。漢初,匈奴冒頓滅其國,餘類保烏桓山,因以為號焉。俗善騎射,弋獵禽獸為事。隨水草放牧,居無常處。以穹廬為舍,東開向日。食肉飲酪,以毛毳為衣。貴少而賤老,其性悍塞。怒則殺父兄,而終不害其母,以母有族類,父兄無相仇報故也。有勇健能理決鬥訟者,推為大人,無世業相繼。邑落各有小帥,數百千落自為一部。大人有所召呼,則刻木為信,雖無文字,而部眾不敢違犯。氏姓無常,以大人健者名字為姓。大人以下,各自畜牧營產,不相傜役。其嫁娶則先略女通情,或半歲百日,然後送牛馬羊畜,以為娉幣。婿隨妻還家,妻家無尊卑,旦旦拜之,而不拜其父母。為妻家僕役,一二年閒,妻家乃厚遣送女,居處財物一皆為辦。其俗妻後母,報寡泽,死則歸其故夫。計謀從用婦人,唯鬥戰之事乃自決之。父子男女相對踞蹲。以髡頭為輕便。婦人至嫁時乃養髮,分為髻,著句決,飾以金碧,猶中國有簂步搖。婦人能刺韋作文繡,織氀毼。男子能作弓矢鞍勒,鍛金鐵為兵器。其土地宜穄及東牆。東牆似蓬草,實如穄子,至十月而熟。見鳥獸孕乳,以別四節。

俗貴兵死,斂屍以棺,有哭泣之哀,至葬則歌舞相送。肥養一犬,以彩繩纓牽,并取死者所乘馬衣物,皆燒而送之,言以屬累犬,使護死者神靈歸赤山。赤山在遼東西北數千里,如中國人死者魂神歸岱山也。敬鬼神,祠天地日月星辰山川及先大人有健名者。祠用牛羊,畢皆燒之。其約法:違大人言者,罪至死;若相賊殺者,令部落自相報,不止,詣大人告之,聽出馬牛羊以贖死;其自殺父兄則無罪;若亡畔為大人所捕者,邑落不得受之,皆徙逐於雍狂之地,沙漠之中。其土多蝮蛇,在丁令西南,烏孫東北焉。

烏桓自為冒頓所破,眾遂孤弱,常臣伏匈奴,歲輸牛馬羊皮,過時不具,輒沒其妻子。及武帝遣驃騎將軍霍去病擊破匈奴左地,因徙烏桓於上谷、漁陽、右北平、遼西、遼東五郡塞外,為漢偵察匈奴動靜。其大人歲一朝見,於是始置護烏桓校尉,秩二千石,擁節監領之,使不得與匈奴交通。

昭帝時,烏桓漸強,乃發匈奴單于冢墓,以報冒頓之怨。匈奴大怒,乃東擊破烏桓。大將軍霍光聞之,因遣度遼將軍范明友將二萬騎出遼東邀匈奴,而虜已引去。明友乘烏桓新敗,遂進擊之,斬首六千餘級,獲其三王首而還。由是烏桓復寇幽州,明友輒破之。宣帝時,乃稍保塞降附。

及王莽篡位,欲擊匈奴,興十二部軍,使東域將嚴尤領烏桓、丁令兵屯代郡,皆質其妻子於郡縣。烏桓不便水土,懼久屯不休,數求謁去。莽不肯遣,遂自亡畔,還為抄盜,而諸郡盡殺其質,由是結怨於莽。匈奴因誘其豪帥以為吏,餘者皆羈縻屬之。

光武初,烏桓與匈奴連兵為寇,代郡以東尤被其害。居止近塞,朝發穹廬,暮至城郭,五郡民庶,家受其辜,至於郡縣損壞,百姓流亡。其在上谷塞外白山者,最為強富。

建武二十一年,遣伏波將軍馬援將三千騎出五阮關掩擊之。烏桓逆知,悉相率逃走,追斬百級而還。烏桓復尾擊援後,援遂晨夜奔歸,比入塞,馬死者千餘匹。

二十二年,匈奴國亂,烏桓乘弱擊破之,匈奴轉北徙數千里,漠南地空,帝乃以幣帛賂烏桓。二十五年,遼西烏桓大人郝旦等九百二十二人率眾向化,詣闕朝貢,獻奴婢牛馬及弓虎豹貂皮。

是時四夷朝賀,絡驛而至,天子乃命大會勞饗,賜以珍寶。烏桓或願留宿衛,於是封其渠帥為侯王君長者八十一人,皆居塞內,布於緣邊諸郡,令招來種人,給其衣食,遂為漢偵候,助擊匈奴、鮮卑。時司徒掾班彪上言:「烏桓天性輕黠,好為寇賊,若久放縱而無總領者,必復侵掠居人,但委主降掾史,恐非所能制。臣愚以為宜復置烏桓校尉,誠有益於附集,省國家之邊慮。」帝從之。於是始復置校尉於上谷甯城,開營府,并領鮮卑,賞賜質子,歲時互市焉。

及明、章、和三世,皆保塞無事。安帝永初三年夏,漁陽烏桓與右北平胡千餘寇代郡、上谷。秋,鴈門烏桓率眾王無何允,與鮮卑大人丘倫等,及南匈奴骨都侯,合七千騎寇五原,與太守戰於九原高渠谷,漢兵大敗,殺郡長吏。乃遣車騎將軍何熙、度遼將軍梁慬等擊,大破之。無何乞降,鮮卑走還塞外。是後烏桓稍復親附,拜其大人戎朱廆為親漢都尉。

順帝陽嘉四年冬,烏桓寇雲中,遮截道上商賈車牛千餘兩,度遼將軍耿曄率二千餘人追擊,不利,又戰於沙南,斬首五百級。烏桓遂圍曄於蘭池城,於是發積射士二千人,度遼營千人,配上郡屯,以討烏桓,烏桓乃退。永和五年,烏桓大人阿堅、羌渠等與南匈奴左部句龍吾斯反畔,中郎將張耽擊破斬之,餘眾悉降。桓帝永壽中,朔方烏桓與休著屠各並畔,中郎將張奐搫平之。延熹九年夏,烏桓復與鮮卑及南匈奴鮮卑寇緣邊九郡,俱反,張奐討之,皆出塞去。

靈帝初,烏桓大人上谷有難樓者,眾九千餘落,遼西有丘力居者,眾五千餘落,皆自稱王;又遼東蘇僕延,眾千餘落,自稱峭王;右北平烏延,眾八百餘落,自稱汗魯王:並勇建而多計策。中平四年,前中山太守張純畔,入丘力居眾中,自號彌天安定王,遂為諸郡烏桓元帥,寇掠青、徐、幽、冀四州。五年,以劉虞為幽州牧,虞購募斬純首,北州乃定。

獻帝初平中,丘力居死,子樓班年少,從子蹋頓有武略,代立,總攝三郡,眾皆從其號令。建安初,冀州牧袁紹與前將軍公孫瓚相持不決,蹋頓遣使詣紹求和親,遂遣兵助擊瓚,破之。紹矯制賜蹋頓、難樓、蘇僕延、烏延等,皆以單于印綬。後難樓、蘇僕延率其部眾奉樓班為單于,蹋頓為王,然蹋頓猶秉計策。廣陽人閻柔,少沒烏桓、鮮卑中,為其種人所歸信,柔乃因鮮卑眾,殺烏桓校尉邢舉而代之。袁紹因寵慰柔,以安北邊。及紹子尚敗,奔蹋頓。時幽、冀吏人奔烏桓者十萬餘戶,尚欲憑其兵力,復圖中國。會曹操平河北,閻柔率鮮卑、烏桓歸附,操即以柔為校尉。建安十二年,曹操自征烏桓,大破蹋頓於柳城,斬之,首虜二十餘萬人。袁尚與樓班、烏延等皆走遼東,遼東太守公孫康並斬送之。其餘眾萬餘落,悉徙居中國云。

鮮卑者,亦東胡之支也,別依鮮卑山,故因號焉。其言語習俗與烏桓同。唯婚姻先髡頭,以季春月大會於饒樂水上,飲讌畢,然後配合。又禽獸異於中國者,野馬、原羊、角端牛,以角為弓,俗謂之角端弓者。又有貂、豽、鼲子,皮毛柔蝡,故天下以為名裘。

漢初,亦為冒頓所破,遠竄遼東塞外,與烏桓相接,未常通中國焉。光武初,匈奴強盛,率鮮卑與烏桓寇抄北邊,殺略吏人,無有寧歲。建武二十一年,鮮卑與匈奴入遼東,遼東太守祭肜擊破之,斬獲殆盡,事已具肜傳,由是震怖。及南單于附漢,北虜孤弱,二十五年,鮮卑始通驛使。

其後都護偏何等詣祭肜求自效功,因令擊北匈奴左伊育訾部,斬首二千餘級。其後偏何連歲出兵擊北虜,還輒持首級詣遼東受賞賜。三十年,鮮卑大人於仇賁、滿頭等率種人詣闕朝賀,慕義內屬。帝封於仇賁為王,滿頭為侯。時漁陽赤山烏桓歆志賁等數寇上谷。永平元年,祭肜復賂偏何擊歆志賁,破斬之,於是鮮卑大人皆來歸附,並詣遼東受賞賜,青徐二州給錢歲二億七千萬為常。明章二世,保塞無事。

和帝永元中,大將軍竇憲遣右校尉耿夔擊破匈奴,北單于逃走,鮮卑因此轉徙據其地。匈奴餘種留者尚有十餘萬落,皆自號鮮卑,鮮卑由此漸盛。九年,遼東鮮卑攻肥如縣,太守祭參坐沮敗,下獄死。十三年,遼東鮮卑寇右北平,因入漁陽,漁陽太守擊破之。延平元年,鮮卑復寇漁陽,太守張顯率數百人出塞追之。兵馬掾嚴授諫曰:「前道險阻,賊埶難量,宜且結營,先令輕騎偵視之。」顯意甚銳,怒欲斬之。因復進兵,遇虜伏發,士卒悉走,唯授力戰,身被十創,手殺數人而死。顯中流矢,主簿衛福、功曹徐咸皆自投赴顯,俱歿於陣。鄧太后策書褒歎,賜顯錢六十萬,以家二人為郎;授、福、咸各錢十萬,除一子為郎。

安帝永初中,鮮卑大人燕荔陽詣闕朝賀,鄧太后賜燕荔陽王印綬,赤車參駕,令止烏桓校尉所居甯城下,通胡市,因築南北兩部質館。鮮卑邑落百二十部,各遣入質。是後或降或畔,與匈奴、烏桓更相攻擊。

元初二年秋,遼東鮮卑圍無慮縣,州郡合兵固保清野,鮮卑無所得。復攻扶黎營,殺長吏。四年,遼西鮮卑連休等遂燒塞門,寇百姓。烏桓大人於秩居等與連休有宿怨,共郡兵奔擊,大破之,斬首千三百級,悉獲其生口牛馬財物。五年秋,代郡鮮卑萬餘騎遂穿塞入寇,分攻城邑,燒官寺,殺長吏而去。乃發緣邊甲卒、黎陽營兵,屯上谷以備之。冬,鮮卑入上谷,攻居庸關,復發緣邊諸郡、黎陽營兵、積射士步騎二萬人,屯列衝要。六年秋,鮮卑入馬城塞,殺長吏,度遼將軍鄧遵發積射士三千人,及中郎將馬續率南單于,與遼西、右北平兵馬會,出塞追擊鮮卑,大破之,獲生口及牛羊財物甚眾。又發積射士三千人,馬三千匹,詣度遼營屯守。

永寧元年,遼西鮮卑大人烏倫、其至鞬率眾詣鄧遵降,奉貢獻。詔封烏倫為率眾王,其至鞬為率眾侯,賜綵繒各有差。

建光元年秋,其至鞬復畔,寇居庸,雲中太守成嚴擊之,兵敗,功曹楊穆以身捍嚴,與俱戰歿。鮮卑於是圍烏桓校尉徐常於馬城。度遼將軍耿夔與幽州刺史龐參發廣陽、漁陽、涿郡甲卒,分為兩道救之;常夜得潛出,與夔等并力並進,攻賊圍,解之。鮮卑既累殺郡守,膽意轉盛,控弦數萬騎。延光元年冬,復寇鴈門、定襄,遂攻太原,掠殺百姓。二年冬,其至鞬自將萬餘騎入東領候,分為數道,攻南匈奴於曼柏,薁鞬日逐王戰死,殺千餘人。三年秋,復寇高柳,擊破南匈奴,殺漸將王。

順帝永建元年秋,鮮卑其至鞬寇代郡,太守李超戰死。明年春,中郎將張國遣從事將南單于兵步騎萬餘人出塞,擊破之,獲其資重二千餘種。時遼東鮮卑六千餘騎亦寇遼東玄菟,烏桓校尉耿曄發緣邊諸郡兵及烏桓率眾王出塞擊之,斬首數百級,大獲其生口牛馬什物,鮮卑乃率種眾三萬人詣遼東乞降。三年,四年,鮮卑頻寇漁陽、朔方。六年秋,耿曄遣司馬將胡兵數千人,出塞擊破之。冬,漁陽太守又遣烏桓兵擊之,斬首八百級,獲牛馬生口。烏桓豪人扶漱官勇健,每與鮮卑戰,輒陷敵,詔賜號「率眾君」。

陽嘉元年冬,耿曄遣烏桓親漢都尉戎朱廆率眾王侯咄歸等,出塞抄擊鮮卑,大斬獲而還,賜咄歸等已下為率眾王、侯、長,賜綵繒各有差。鮮卑後寇遼東屬國,於是耿曄乃移屯遼東無慮城拒之。二年春,匈奴中郎將趙稠遣從事將南匈奴骨都侯夫沈等,出塞擊鮮卑,破之,斬獲甚眾,詔賜夫沈金印紫綬及縑綵各有差。秋,鮮卑穿塞入馬城,代郡太守擊之,不能克。後其至鞬死,鮮卑抄盜差稀。

桓帝時,鮮卑檀石槐者,其父投鹿侯、初從匈奴軍三年,其妻在家生子。投鹿侯歸,怪欲殺之。妻言嘗晝行聞雷震,仰天視而雹入其口,因吞之,遂妊身,十月而產,此子必有奇異,且宜長視。投鹿侯不聽,遂棄之。妻私語家令收養焉,名檀石槐。年十四五,勇健有智略。異部大人抄取其外家牛羊,檀石槐單騎追擊之,所向無前,悉還得所亡者,由是部落畏服。乃施法禁,平曲直,無敢犯者,遂推以為大人。檀石槐乃立庭於彈汗山歠仇水上,去高柳北三百餘里,兵馬甚盛,東西部大人皆歸焉。因南抄緣邊,北拒丁零,東卻夫餘,西擊烏孫,盡據匈奴故地,東西萬四千餘里,南北七千餘里,網羅山川水澤鹽池。

永壽二年秋,檀石槐遂將三四千騎寇雲中。延熹元年,鮮卑寇北邊。冬,使匈奴中郎將張奐率南單于出塞擊之,斬首二百級。二年,復入鴈門,殺數百人,大抄掠而去。六年夏,千餘騎寇遼東屬國。九年夏,遂分騎數萬人入緣邊九郡,並殺掠吏人,於是復遣張奐擊之,鮮卑乃出塞去。朝廷積患之,而不能制,遂遣使持印綬封檀石槐為王,欲與和親。檀石槐不肯受,而寇抄滋甚。乃自分其地為三部,從右北平以東至遼東,接夫餘、濊貊二十餘邑為東部,從右北平以西至上谷十餘邑為中部,從上谷以西至敦煌、烏孫二十餘邑為西部,各置大人主領之,皆屬檀石槐。

靈帝立,幽、并、涼三州緣邊諸郡無歲不被鮮卑寇抄,殺略不可勝數。熹平三年冬,鮮卑入北地,太守夏育率休著屠各追擊破之。遷育為護烏桓校尉。五年,鮮卑寇幽州。六年夏,鮮卑寇三邊。秋,夏育上言:「鮮卑寇邊,自春以來,三十餘發,請徵幽州諸郡兵出塞擊之,一冬二春,必能禽滅。」朝廷未許。先是護羌校尉田晏坐事論刑被原,欲立功自效,乃請中常侍王甫求得為將,甫因此議遣兵與育并力討賊。帝乃拜晏為破鮮卑中郎將。大臣多有不同,乃召百官議朝堂。議郎蔡邕議曰:

書戒猾夏,易伐鬼方,周有獫狁、蠻荊之師,漢有闐顏、瀚海之事。征討殊類,所由尚矣。然而時有同異,埶有可否,故謀有得失,事有成敗,不可齊也。

武帝情存遠略,志闢四方,南誅百越,北討強胡,西伐大宛,東并朝鮮。因文、景之蓄,藉天下之饒,數十年閒,官民俱匱。乃興鹽鐵酒榷之利,設告緡重稅之令,民不堪命,起為盜賊,關東紛擾,道路不通。繡衣直指之使,奮鈇鉞而並出。既而覺悟,乃息兵罷役,丞相為富人侯。故主父偃曰:「夫務戰勝,窮武事,未有不悔者也。」夫以世宗神武,將相良猛,財賦充實,所拓廣遠,猶有悔焉。況今人財並乏,事劣昔時乎!

自匈奴遁逃,鮮卑強盛,據其故地,稱兵十萬,才力勁健,意智益生。加以關塞不嚴,禁網多漏,精金良鐵,皆為賊有;漢人逋逃,為之謀主,兵利馬疾,過於匈奴。昔段熲良將,習兵善戰,有事西羌,猶十餘年。今育、晏才策,未必過熲,鮮卑種眾,不弱于曩時。而虛計二載,自許有成,若禍結兵連,豈得中休?當復徵發眾人,轉運無已,是為耗竭諸夏,并力蠻夷。夫邊垂之患,手足之蚧搔;中國之困,胸背之瘭疽。方今郡縣盜賊尚不能禁,況此醜虜而可伏乎!

昔高祖忍平城之恥,呂后棄慢書之詬,方之於今,何者為甚?

天設山河,秦築長城,漢起塞垣,所以別內外,異殊俗也。苟無瞩國內侮之患則可矣,豈與蟲螘校寇計爭往來哉!雖或破之,豈可殄盡,而方今本朝為之旰食乎?

夫專勝者未必克,挾疑者未必敗,眾所謂危,聖人不任,朝議有嫌,明主不行也。昔淮南王安諫伐越曰:「天子之兵,有征無戰。言其莫敢校也。如使越人蒙死以逆執事廝輿之卒,有一不備而歸者,雖得越王之首,而猶為大漢羞之。」而欲以齊民易醜虜,皇威辱外夷,就如其言,猶已危矣,況乎得失不可量邪!昔珠崖郡反,孝元皇帝納賈捐之言,而下詔曰:「珠崖背畔,今議者或曰可討,或曰棄之。朕日夜惟思,羞威不行,則欲誅之;通于時變,復憂萬民。夫萬民之飢與遠蠻之不討,何者為大?宗廟之祭,凶年猶有不備,況避不嫌之辱哉!今關東大困,無以相贍,又當動兵,非但勞民而已。其罷珠崖郡。」此元帝所以發德音也。夫卹民救急,雖成郡列縣,尚猶棄之,況障塞之外,未嘗為民居者乎!守邊之術,李牧善其略,保塞之論,嚴尤申其要,遺業猶在,文章具存,循二子之策,守先帝之規,臣曰可矣。

帝不從。遂遣夏育出高柳,田晏出雲中,匈奴中郎將臧旻率南單于出鴈門,各將萬騎,三道出塞二千餘里。檀石槐命三部大人各帥眾逆戰,育等大敗,喪其節傳輜重,各將數十騎奔還,死者十七八。三將檻車徵下獄,贖為庶人。冬,鮮卑寇遼西。光和元年冬,又寇酒泉,緣邊莫不被毒。種眾日多,田畜射獵不足給食,檀石槐乃自徇行,見烏侯秦水廣從數百里,水停不流,其中有魚,不能得之。聞倭人善網捕,於是東擊倭人國,得千餘家,徙置秦水上,令捕魚以助糧食。

光和中,檀石槐死,時年四十五,子和連代立。和連才力不及父,亦數為寇抄,性貪淫,斷法不平,眾畔者半。後出攻北地,廉人善弩射者射中和連,即死。其子騫曼年小,兄子魁頭立。後騫曼長大,與魁頭爭國,眾遂離散。魁頭死,弟步度根立。自檀石槐後,諸大人遂世相傳襲。

論曰:四夷之暴,其埶互彊矣。匈奴熾於隆漢,西羌猛於中興。而靈獻之閒,二虜迭盛,石槐驍猛,盡有單于之地,蹋頓凶桀,公據遼西之土。其陵跨中國,結患生人者,靡世而寧焉。然制御上略,歷世無聞;周、漢之策,僅得中下。將天之冥數,以至於是乎?

贊曰:二虜首施,鯁我北垂。道暢則馴,時薄先離。


\end{pinyinscope}