\article{竇何列傳}

\begin{pinyinscope}
竇武字游平,扶風平陵人,安豐戴侯融之玄孫也。父奉,定襄太守。武少以經行著稱,常教授於大澤中,不交時事,名顯關西。

延熹八年,長女選入掖庭,桓帝以為貴人,拜武郎中。其冬,貴人立為皇后,武遷越騎校尉,封槐里侯,五千戶。明年冬,拜城門校尉。在位多辟名士,清身疾惡,禮賂不通,妻子衣食裁充足而已。是時羌蠻寇難,歲儉民飢,武得兩宮賞賜,悉散與太學諸生,及載肴糧於路,饨施貧民。兄子紹,為虎賁中郎將,性疏簡奢侈。武每數切厲相戒,猶不覺悟,乃上書求退紹位,又自責不能訓導,當先受罪。由是紹更遵節,大小莫敢違犯。

時國政多失,內官專寵,李膺、杜密等為黨事考逮。永康元年,上疏諫曰:「臣聞明主不諱譏刺之言,以探幽暗之實;忠臣不卹諫爭之患,以暢萬端之事。是以君臣並熙,名奮百世。臣幸得遭盛明之世,逢文武之化,豈敢懷祿逃罪,不竭其誠!陛下初從藩國,爰登聖祚,天下逸豫,謂當中興。自即位以來,未聞善政。梁、孫、寇、鄧雖或誅滅,而常侍黃門續為禍虐,欺罔陛下,競行譎詐,自造制度,妄爵非人,朝政日衰,姦臣日彊。伏尋西京放恣王氏,佞臣執政,終喪天下。今不慮前事之失,復循覆車之軌,臣恐二世之難,必將復及,趙高之變,不朝則夕。近者姦臣牢脩,造設黨議,遂收前司隸校尉李膺、太僕杜密、御史中丞陳翔、太尉掾范滂等逮考,連及數百人,曠年拘錄,事無效驗。臣惟膺等建忠抗節,志經王室,此誠陛下稷、党、伊、呂之佐,而虛為姦臣賊子之所誣枉,天下寒心,海內失望。惟陛下留神澄省,時見理出,以厭人鬼喁喁之心。臣聞古之明君,必須賢佐,以成政道。今臺閣近臣,尚書令陳蕃,僕射胡廣,尚書朱宇、荀緄、劉祐、魏朗、劉矩、尹勳等,皆國之貞士,朝之良佐。尚書郎張陵、媯皓、苑康、楊喬、邊韶、戴恢等,文質彬彬,明達國典。內外之職,群才並列。而陛下委任近習,專樹饕餮,外典州郡,內幹心膂。宜以次貶黜,案罪糾罰,抑奪宦官欺國之封,案其無狀誣罔之罪,信任忠良,平決臧否,使邪正毀譽,各得其所,寶愛天官,唯善是授。如此,咎徵可消,天應可待。閒者有嘉禾、芝草、黃龍之見。夫瑞生必於嘉士,福至實由善人,在德為瑞,無德為災。陛下所行,不合天意,不宜稱慶。」書奏,因以病上還城門校尉、槐里侯印綬。帝不許,有詔原李膺、杜密等,自黃門北寺、若盧、都內諸獄,繫囚罪輕者皆出之。

其冬帝崩,無嗣。武召侍御史河閒劉儵,參問其國中王子侯之賢者,儵稱解瀆亭侯宏。武入白太后,遂徵立之,是為靈帝。拜武為大將軍,常居禁中。帝既立,論定策功,更封武為聞喜侯;子機渭陽侯,拜侍中;兄子紹鄠侯,遷步兵校尉;紹弟靖西鄉侯,為侍中,監羽林左騎。

武既輔朝政,常有誅翦宦官之意,太傅陳蕃亦素有謀。時共會朝堂,蕃私謂武曰:「中常侍曹節、王甫等,自先帝時操弄國權,濁亂海內,百姓匈匈,歸咎於此。今不誅節等,後必難圖。」武深然之。蕃大喜,以手推席而起。武於是引同志尹勳為尚書令,劉瑜為侍中,馮述為屯騎校尉;又徵天下名士廢黜者前司隸李膺、宗正劉猛、太僕杜密、廬江太守朱宇等,列於朝廷;請前越巂太守荀翌為從事中郎,辟潁川陳寔為屬:共定計策。於是天下雄俊,知其風旨,莫不延頸企踵,思奮其智力。

會五月日食,蕃復說武曰:「昔蕭望之困一石顯,近者李、杜諸公禍及妻子,況今石顯數十輩乎!蕃以八十之年,欲為將軍除害,今可且因日食,斥罷宦官,以塞天變。又趙夫人及女尚書,旦夕亂太后,急宜退絕。惟將軍慮焉。」武乃白太后曰:「故事,黃門、常侍但當給事省內,典門戶,主近署財物耳。今乃使與政事而任權重,子弟布列,專為貪暴。天下匈匈,正以此故。宜悉誅廢,以清朝廷。」太后曰:「漢來故事世有,但當誅其有罪,豈可盡廢邪?」時中常侍管霸頗有才略,專制省內。武先白誅霸及中常侍蘇康等,竟死。武復數白誅曹節等,太后冘豫未忍,故事久不發。

至八月,太白出西方。劉瑜素善天官,惡之,上書皇太后曰:「

太白犯房左驂,上將星入太微,其占宮門當閉,將相不利,姦人在主傍。願急防之。」又與武、蕃書,以星辰錯繆,不利大臣,宜速斷大計。武、蕃得書將發,於是以朱宇為司隸校尉,劉祐為河南尹,虞祁為洛陽令。武乃奏免黃門令魏彪,以所親小黃門山冰代之。使冰奏素狡猾尤無狀者長樂尚書鄭骆,送北寺獄。蕃謂武曰:「此曹子便當收殺,何復考為!」武不從,令冰與尹勳、侍御史祝档雜考骆,辭連及曹節、王甫。勳、冰即奏收節等,使劉瑜內奏。

時武出宿歸府,典中書者先以告長樂五官史朱瑀。瑀盜發武奏,罵曰:「中官放縱者,自可誅耳。我曹何罪,而當盡見族滅?」因大呼曰:「陳蕃、竇武奏白太后廢帝,為大逆!」乃夜召素所親壯健者長樂從官史共普、張亮等十七人,喢血共盟誅武等。曹節聞之,驚起,白帝曰:「外閒切切,請出御德陽前殿。」令帝拔劍踊躍,使乳母趙嬈等擁衛左右,取棨信,閉諸禁門。召尚書官屬,脅以白刃,使作詔板。拜王甫為黃門令,持節至北寺獄收尹勳、山冰。冰疑,不受詔,甫格殺之。遂害勳,出鄭骆。還共劫太后,奪璽書。令中謁者守南宮,閉門,絕複道。使鄭骆等持節,及侍御史、謁者捕收武等。武不受詔,馳入步兵營,與紹共射殺使者。召會北軍五校士數千人屯都亭下,令軍士曰:「黃門常侍反,盡力者封侯重賞。」詔以少府周靖行車騎將軍,加節,與護匈奴中郎將張奐率五營士討武。夜漏盡,王甫將虎賁、羽林、廄騶、都候、劍戟士,合千餘人,出屯朱雀掖門,與奐等合。明旦悉軍闕下,與武對陳。甫兵漸盛,使其士大呼武軍曰:「竇武反,汝皆禁兵,當宿衛宮省,何故隨反者乎?先降有賞!」營府素畏服中官,於是武軍稍稍歸甫。自旦至食時,兵降略盡。武、紹走,諸軍追圍之,皆自殺,梟首洛陽都亭。收捕宗親、賓客、姻屬,悉誅之,及劉瑜、馮述,皆夷其族。徙武家屬日南,遷太后於雲臺。

當是時,凶豎得志,士大夫皆喪其氣矣。武府掾桂陽胡騰,少師事武,獨殯斂行喪,坐以禁錮。

武孫輔,時年二歲,逃竄得全。事覺,節等捕之急。胡騰及令史南陽張敞共逃輔於零陵界,詐云已死,騰以為己子,而使聘娶焉。後舉桂陽孝廉。至建安中,荊州牧劉表聞而辟焉,以為從事,使還竇姓,以事列上。會表卒,曹操定荊州,輔與宗人徙居於鄴,辟丞相府。從征馬超,為流矢所中死。

初,武母產武而并產一蛇,送之林中。後母卒,及葬未窆,有大蛇自榛草而出,徑至喪所,以頭擊柩,涕血皆流,俯仰蛣屈,若哀泣之容,有頃而去。時人知為竇氏之祥。

騰字子升。初,桓帝巡狩南陽,以騰為護駕從事。公卿貴戚車騎萬計,徵求費役,不可勝極。騰上言:「天子無外,乘輿所幸,即為京師。臣請以荊州刺史比司隸校尉,臣自同都官從事。」帝從之。自是肅然,莫敢妄有干欲,騰以此顯名。黨錮解,官至尚書。

張敞者,太尉溫之弟也。

何進字遂高,南陽宛人也。異母女弟選入掖庭為貴人,有寵於靈帝,拜進郎中,再遷虎賁中郎將,出為潁川太守。光和二年,貴人立為皇后,徵進入,拜侍中、將作大匠、河南尹。

中平元年,黃巾賊張角等起,以進為大將軍,率左右羽林五營士屯都亭,修理器械,以鎮京師。張角別黨馬元義謀起洛陽,進發其姦,以功封慎侯。

四年,滎陽賊數千人群起,攻燒郡縣,殺中牟縣令,詔使進弟河南尹苗出擊之。苗攻破群賊,平定而還。詔遣使者迎於成皋,拜苗為車騎將軍,封濟陽侯。

五年,天下滋亂,望氣者以為京師當有大兵,兩宮流血。大將軍司馬許涼、假司馬伍宕說進曰:「太公六韜有天子將兵事,可以威厭四方。」進以為然,入言之於帝。於是乃詔進大發四方兵,講武於平樂觀下。起大壇,上建十二重五采華蓋,高十丈,壇東北為小壇,復建九重華蓋,高九丈,列步兵,騎士數萬人,結營為陳。天子親出臨軍,駐大華蓋下,進駐小華蓋下。禮畢,帝躬擐甲介馬,稱「無上將軍」,行陳三匝而還。詔使進悉領兵屯於觀下。是時置西園八校尉,以小黃門蹇碩為上軍校尉,虎賁中郎將袁紹為中軍校尉,屯騎都尉鮑鴻為下軍校尉,議郎曹操為典軍校尉,趙融為助軍校尉,淳于瓊為佐軍校尉,又有左右校尉。帝以蹇碩壯健而有武略,特親任之,以為元帥,督司隸校尉以下,雖大將軍亦領屬焉。

碩雖擅兵於中,而猶畏忌於進,乃與諸常侍共說帝遣進西擊邊章、韓遂。帝從之,賜兵車百乘,虎賁斧鉞。進陰知其謀,乃上遣袁紹東擊徐兗二州兵,須紹還,即戎事,以稽行期。

初,何皇后生皇子辯,王貴人生皇子協。群臣請立太子,帝以辯輕佻無威儀,不可為人主,然皇后有寵,且進又居重權,故久不決。

六年,帝疾篤,屬協於蹇碩。碩既受遺詔,且素輕忌於進兄弟,及帝崩,碩時在內,欲先誅進而立協。及進從外入,碩司馬潘隱與進早舊,迎而目之。進驚,馳從儳道歸營,引兵入屯百郡邸,因稱疾不入。碩謀不行,皇子辯乃即位,何太后臨朝,進與太傅袁隗輔政,錄尚書事。

進素知中官天下所疾,兼忿蹇碩圖己,及秉朝政,陰規誅之。袁紹亦素有謀,因進親客張津勸之曰:「黃門常侍權重日久,又與長樂太后專通姦利,將軍宜更清選賢良,整齊天下,為國家除患。」進然其言。又以袁氏累世寵貴,海內所歸,而紹素善養士,能得豪傑用,其從弟虎賁中郎將術亦尚氣俠,故並厚待之。因復博徵智謀之士龐紀、何顒、荀攸等,與同腹心。

蹇碩疑不自安,與中常侍趙忠等書曰:「大將軍兄弟秉國專朝,今與天下黨人謀誅先帝左右,埽滅我曹。但以碩典禁兵,故且沈吟。今宜共閉上閤,急捕誅之。」中常侍郭勝,進同郡人也。太后及進之貴幸,勝有力焉。故勝親信何氏,遂共趙忠等議,不從碩計,而以其書示進。進乃使黃門令收碩,誅之,因領其屯兵。

袁紹復說進曰:「前竇武欲誅內寵而反為所害者

,以其言語漏泄,而五營百官服畏中人故也。今將軍既有元舅之重,而兄弟並領勁兵,部曲將吏皆英俊名士,樂盡力命,事在掌握,此天贊之時也。將軍宜一為天下除患,名垂後世。雖周之申伯,何足道哉!今大行在前殿,將軍宜受詔領禁兵,不宜輕出入宮省。」進甚然之,乃稱疾不入陪喪,又不送山陵。遂與紹定籌策,而以其計白太后。太后不聽,曰:「中官統領禁省,自古及今,漢家故事,不可廢也。且先帝新棄天下,我柰何楚楚與士人對共事乎?」進難違太后意,且欲誅其放縱者。紹以為中官親近至尊,出入號令,今不悉廢,後必為患。而太后母舞陽君及苗數受諸宦官賂遺,知進欲誅之。數白太后,為其障蔽。又言:「大將軍專殺左右,擅權以弱社稷。」太后疑以為然。中官在省闥者或數十年,封侯貴寵,膠固內外。進新當重任,素敬憚之,雖外收大名而內不能斷,故事久不決。

紹等又為畫策,多召四方猛將及諸豪傑,使並引兵向京城,以脅太后。進然之。主簿陳琳入諫曰:「易稱『即鹿無虞』,諺有『掩目捕雀』。夫微物尚不可欺以得志,況國之大事,其可以詐立乎?今將軍總皇威,握兵要,龍驤虎步,高下在心,此猶鼓洪爐燎毛髮耳。夫違經合道,天人所順,而反委釋利器,更徵外助。大兵聚會,彊者為雄,所謂倒持干戈,授人以柄,功必不成,秖為亂階。」進不聽。遂西召前將軍董卓屯關中上林苑,又使府掾太山王匡東發其郡強弩,并召東郡太守橋瑁屯城皋,使武猛都尉丁原燒孟津,火照城中,皆以誅宦官為言。太后猶不從。

苗謂進曰:「始共從南陽來,俱以貧賤,依省內以致貴富。國家之事,亦何容易!覆水不可收。宜深思之,且與省內和也。」進意更狐疑。紹懼進變計,乃脅之曰:「交搆已成,形埶已露,事留變生,將軍復欲何待,而不早決之乎?」進於是以紹為司隸校尉,假節,專命擊斷;從事中郎王允為河南尹。紹使洛陽方略武吏司察宦者,而促董卓等使馳驛上,欲進兵平樂觀。太后乃恐,悉罷中常侍小黃門,使還里舍,唯留進素所私人,以守省中。諸常侍小黃門皆詣進謝罪,唯所措置。進謂曰:「天下匈匈,正患諸君耳。今董卓垂至,諸君何不早各就國?」袁紹勸進便於此決之,至于再三。進不許。紹又為書告諸州郡,詐宣進意,使捕案中官親屬。

進謀積日,頗泄,中官懼而思變。張讓子婦,太后之妹也。讓向子婦叩頭曰:「老臣得罪,當與新婦俱歸私門。惟受恩累世,今當遠離宮殿,情懷戀戀,願復一入直,得暫奉望太后、陛下顏色,然後退就溝壑,死不恨矣。」子婦言於舞陽君,入白太后,乃詔諸常侍皆復入直。

八月,進入長樂白太后,請盡誅諸常侍以下,選三署郎入守宦官廬。諸宦官相謂曰:「大將軍稱疾不臨喪,不送葬,今欻入省,此意何為?竇氏事竟復起邪?」又張讓等使人潛聽,具聞其語,乃率常侍段珪、畢嵐等數十人,持兵竊自側闥入,伏省中。及進出,因詐以太后詔召進。入坐省闥,讓等詰進曰:「天下憒憒,亦非獨我曹罪也。先帝嘗與太后不快,幾至成敗,我曹涕泣救解,各出家財千萬為禮,和悅上意,但欲託卿門戶耳。今乃欲滅我曹種族,不亦太甚乎?卿言省內穢濁,公卿以下忠清者為誰?」於是尚方監渠穆拔劍斬進於嘉德殿前。讓、珪等為詔,以故太尉樊陵為司隸校尉,少府許相為河南尹。尚書得詔板,疑之,曰:「請大將軍出共議。」中黃門以進頭擲與尚書,曰:「何進謀反,已伏誅矣。」

進部曲將吳匡、張璋,素所親幸,在外聞進被害,欲將兵入宮,宮閤閉。袁術與匡共斫攻之,中黃門持兵守閤。會日暮,術因燒南宮九龍門及東西宮,欲以脅出讓等。讓等入白太后,言大將軍兵反,燒宮,攻尚書闥,因將太后、天子及陳留王,又劫省內官屬,從複道走北宮。尚書盧植執戈於閣道窗下,仰數段珪。段珪等懼,乃釋太后。太后投閣得免。

袁紹與叔父隗矯詔召樊陵、許相,斬之。苗、紹乃引兵屯朱雀闕下,捕得趙忠等,斬之。吳匡等素怨苗不與進同心,而又疑其與宦官同謀,乃令軍中曰:「殺大將軍者即車騎也,士吏能為報讎乎?」進素有仁恩,士卒皆流涕曰:「願致死!」匡遂引兵與董卓弟奉車都尉旻攻殺苗,棄其屍於苑中。紹遂閉北宮門,勒兵捕宦者,無少長皆殺之。或有無須而誤死者,至自發露然後得免。者二千餘人。紹因進兵排宮,或上端門屋,以攻省內。

張讓、段珪等困迫,遂將帝與陳留王數十人步出穀門,奔小平津。公卿並出平樂觀,無得從者,唯尚書盧植夜馳河上,王允遣河南中部掾閔貢隨植後。貢至,手劍斬數人,餘皆投河而死。明日,公卿百官乃奉迎天子還宮,以貢為郎中,封都亭侯。

董卓遂廢帝,又迫殺太后,殺舞陽君,何氏遂亡,而漢室亦自此敗亂。

論曰:竇武、何進藉元舅之資,據輔政之權,內倚太后臨朝之威,外迎群英乘風之埶,卒而事敗閹豎,身死功穨,為世所悲,豈智不足而權有餘乎?傳曰:「天之廢商久矣,君將興之。」斯宋襄公所以敗於泓也。

贊曰:武生蛇祥,進自屠羊。惟女惟弟,來儀紫房。上惛下嬖,人靈動怨。將糾邪慝,以合人願。道之屈矣,代離凶困。


\end{pinyinscope}