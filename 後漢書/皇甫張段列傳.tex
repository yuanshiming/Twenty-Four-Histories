\article{皇甫張段列傳}

\begin{pinyinscope}
皇甫規字威明,安定朝那人也。祖父棱,度遼將軍。父旗,扶風都尉。

永和六年,西羌大寇三輔,圍安定,征西將軍馬賢將諸郡兵擊之,不能克。規雖在布衣,見賢不卹軍事,審其必敗,乃上書言狀。尋而賢果為羌所沒。郡將知規有兵略,乃命為功曹,使率甲士八百,與羌交戰,斬首數級,賊遂退卻。舉規上計掾。其後羌眾大合,攻燒隴西,朝廷患之。規乃上疏求乞自效,曰:「臣比年以來,數陳便宜。羌戎未動,策其將反,馬賢始出,頗知必敗。誤中之言,在可考校。臣每惟賢等擁眾四年,未有成功,懸師之費且百億計,出於平人,回入姦吏。故江湖之人,群為盜賊,青、徐荒飢,襁負流散。夫羌戎潰叛,不由承平,皆由邊將失於綏御。乘常守安,則君侵暴,苟競小利,則致大害,微勝則虛張首級,軍敗則隱匿不言。軍士勞怨,困於猾吏,進不得快戰以徼功,退不得溫飽以全命,餓死溝渠,暴骨中原。徒見王師之出,不聞振旅之聲。酋豪泣血,驚懼生變。是以安不能久,敗則經年。臣所以搏手叩心而增歎者也。願假臣兩營二郡,屯列坐食之兵五千,出其不意,與護羌校尉趙沖共相首尾。土地山谷,臣所曉習;兵埶巧便,臣已更之。可不煩方寸之印,尺帛之賜,高可以滌患,下可以納降。若謂臣年少官輕,不足用者,凡諸敗將,非官爵之不高,年齒之不邁。臣不勝至誠,沒死自陳。」時帝不能用。

沖質之閒,梁太后臨朝,規舉賢良方正。對策曰:

伏惟孝順皇帝,初勤王政,紀綱四方,幾以獲安。後遭姦偽,威分近習,畜貨聚馬,戲謔是聞;又因緣嬖倖,受賂賣爵,輕使賓客,交錯其閒,天下擾擾,從亂如歸,故每有征戰,鮮不挫傷,官民並竭,上下窮虛。臣在關西,竊聽風聲,未聞國家有所先後,而威福之來,咸歸權倖。陛下體兼乾坤,聰哲純茂。攝政之初,拔用忠貞,其餘維綱,多所改正。遠近翕然,望見太平。而地震之後,霧氣白濁,日月不光,旱魃為虐,大賊從橫,流血丹野,庶品不安,譴誠累至,殆以姦臣權重之所致也。其常侍尤無狀者,亟便黜遣,披埽凶黨,收入財賄,以塞痛怨,以荅天誡。

今大將軍梁冀、河南尹不疑,處周、邵之任,為社稷之鎮,加與王室世為姻族,今日立號雖尊可也,實宜增脩謙節,輔以儒術,省去遊娛不急之務,割減廬第無益之飾。夫君者舟也,人者水也。群臣乘舟者也,將軍兄弟操楫者也。若能平志畢力,以度元元,所謂福也。如其怠弛,將淪波濤。可不慎乎!夫德不稱祿,猶鑿墉之趾,以益其高。豈量力審功安固之道哉?凡諸宿猾、酒徒、戲客,皆耳納邪聲,口出諂言,甘心逸遊,唱造不義。亦宜貶斥,以懲不軌。令冀等深思得賢之福,失人之累。又在位素餐,尚書怠職,有司依違,莫肯糾察,故使陛下專受諂諛之言,不聞戶牖之外。臣誠知阿諛有福,深言近禍,豈敢隱心以避誅責乎!臣生長邊遠,希涉紫庭,怖慴失守,言不盡心。

梁冀忿其刺己,以規為下第,拜郎中。託疾免歸,州郡承冀旨,幾陷死者再三。遂以詩、易教授,門徒三百餘人,積十四年。後梁冀被誅,旬月之閒,禮命五至,皆不就。

時太山賊叔孫無忌侵亂郡縣,中郎將宗資討之未服。公車特徵規,拜太山太守。規到官,廣設方略,寇賊悉平。延熹四年秋,叛羌零吾等與先零別種寇鈔關中,護羌校尉段熲坐徵。後先零諸種陸梁,覆沒營塢。規素悉羌事,志自奮效,乃上疏曰:「自臣受任,志竭愚鈍,實賴兗州刺史牽顥之清猛,中郎將宗資之信義,得承節度,幸無咎譽。今猾賊就滅,太山略平,復聞群羌並皆反逆。臣生長邠岐,年五十有九,昔為郡吏,再更叛羌,豫籌其事,有誤中之言。臣素有固疾,恐犬馬齒窮,不報大恩,願乞冗官,備單車一介之使,勞來三輔,宣國威澤,以所習地形兵埶,佐助諸軍。臣窮居孤危之中,坐觀郡將,已數十年矣。自鳥鼠至于東岱,其病一也。力求猛敵,不如清平;勤明吳、孫,未若奉法。前變未遠,臣誠戚之。是以越職,盡其區區。」

至冬,羌遂大合,朝廷為憂。三公舉規為中郎將,持節監關西兵,討零吾等,破之,斬首八百級。先零諸種羌慕規威信,相勸降者十餘萬。明年,規因發其騎共討隴右,而道路隔絕,軍中大疫,死者十三四。規親入菴廬,巡視將士,三軍感悅。東羌遂遣使乞降,涼州復通。

先是安定太守孫雋受取狼籍,屬國都尉李翕、督軍御史張稟多殺降羌,涼州刺史郭閎、漢陽太守趙熹並老弱不堪任職,而皆倚恃權貴,不遵法度。規到州界,悉條奏其罪,或免或誅。羌人聞之,翕然反善。沈氐大豪滇昌、飢恬等十餘萬口,復詣規降。

規出身數年,持節為將,擁眾立功,還督鄉里,既無它私惠,而多所舉奏,又惡絕宦官,不與交通,於是中外並怨,遂共誣規貨賂群羌,令其文降。天子璽書誚讓相屬。規懼不免,上疏自訟曰:「四年之秋,戎醜蠢戾,爰自西州,侵及涇陽,舊都懼駭,朝廷西顧。明詔不以臣愚駑,急使軍就道。幸蒙威靈,遂振國命,羌戎諸種,大小稽首,輒移書營郡,以訪誅納,所省之費,一億以上。以為忠臣之義,不敢告勞,故恥以片言自及微效。然比方先事,庶免罪悔。前踐州界,先奏郡守孫雋,次及屬國都尉李翕、督軍御史張稟;旋師南征,又上涼州刺史郭閎、漢陽太守趙熹,陳其過惡,執據大辟。凡此五臣,支黨半國,其餘墨綬,下至小吏,所連及者,復有百餘。吏託報將之怨,子思復父之恥,載贄馳車,懷糧步走,交搆豪門,競流謗讟,云臣私報諸羌,謝其錢貨。若臣以私財,則家無擔石;如物出於官,則文簿易考。就臣愚惑,信如言者,前世尚遺匈奴以宮姬,鎮烏孫以公主。今臣但費千萬,以懷叛羌。則良臣之才略,兵家之所貴,將有何罪,負義違理乎?自永初以來,將出不少,覆軍有五,動資巨億。有旋車完封,寫之權門,而名成功立,厚加爵封。今臣還督本土,糾舉諸郡,絕交離親,戮辱舊故,眾謗陰害,固其宜也。臣雖汙穢,廉絜無聞,今見覆沒,恥痛實深。傳稱『鹿死不擇音』,謹冒昧略上。」

其年冬,徵還拜議郎。論功當封。而中常侍徐璜、左悺欲從求貨,數遣賓客就問功狀,規終不荅。璜等忿怒,陷以前事,下之於吏。官屬欲賦斂請謝,規誓而不聽,遂以餘寇不絕,坐繫廷尉,論輸左校。諸公及太學生張鳳等三百餘人詣闕訟之。會赦,歸家。

徵拜度遼將軍,至營數月,上書薦中郎將張奐以自代。曰:「臣聞人無常俗,而政有治亂;兵無強弱,而將有能否。伏見中郎將張奐,才略兼優,宜正元帥,以從眾望。若猶謂愚臣宜充軍事者,願乞冗官,以為奐副。」朝庭從之,以奐代為度遼將軍,規為使匈奴中郎將。及奐遷大司農,規復代為度遼將軍。

規為人多意筭,自以連在大位,欲退身避第,數上病,不見聽。會友人上郡太守王旻喪還,規縞素越界,到下亭迎之。因令客密告并州刺史胡芳,言規擅遠軍營,公違禁憲,當急舉奏。芳曰:「威明欲避第仕塗,故激發我耳。吾當為朝廷愛才,何能申此子計邪!」遂無所問。及黨事大起,天下名賢多見染逮,規雖為名將,素譽不高。自以西州豪桀,恥不得豫,乃先自上言:「臣前薦故大司農張奐,是附黨也。又臣昔論輸左校時,太學生張鳳等上書訟臣,是為黨人所附也。臣宜坐之。」朝廷知而不問,時人以為規賢。

在事數歲,北邊威服。永康元年,徵為尚書。其夏日食,詔公卿舉賢良方正,下問得失。規對曰:「天之於王者,如君之於臣,父之於子也。誡以災妖,使從福祥。陛下八年之中,三斷大獄,一除內嬖,再誅外臣。而災異猶見,人情未安者,殆賢愚進退,威刑所加,有非其理也。前太尉陳蕃、劉矩,忠謀高世,廢在里巷;劉祐、馮緄、趙典、尹勳,正直多怨,流放家門;李膺、王暢、孔翊,絜身守禮,終無宰相之階。至於鉤黨之釁,事起無端,虐賢傷善,哀及無辜。今興改善政,易於覆手,而群臣杜口,鑒畏前害,互相瞻顧,莫肯正言。伏願陛下暫留聖明,容受謇直,則前責可弭,後福必降。」對奏,不省。

遷規弘農太守,封壽成亭侯,邑二百戶,讓封不受。再轉為護羌校尉。熹平三年,以疾召還,未至,卒于穀城,年七十一。所著賦、銘、碑、讚、禱文、弔、章表、教令、書、檄、牋記,凡二十七篇。

論曰:孔子稱「其言之不怍,則其為之也難」。察皇甫規之言,其心不怍哉!夫其審己則干祿,見賢則委位,故干祿不為貪,而委位不求讓;稱己不疑伐,而讓人無懼情。故能功成於戎狄,身全於邦家也。

張奐字然明,敦煌酒泉人也。父惇,為漢陽太守。奐少遊三輔,師事太尉朱寵,學歐陽尚書。初,牟氏章句浮辭繁多,有四十五萬餘言,奐減為九萬言。後辟大將軍梁冀府,乃上書桓帝,奏其章句,詔下東觀。以疾去官,復舉賢良,對策第一,擢拜議郎。

永壽元年,遷安定屬國都尉。初到職,而南匈奴左薁鞬臺耆、且渠伯德等七千餘人寇美稷,東羌復舉種應之,而奐壁唯有二百許人,聞即勒兵而出。軍吏以為力不敵,叩頭爭止之。奐不聽,遂進屯長城,收集兵士,遣將王衛招誘東羌,因據龜茲,使南匈奴不得交通東羌。諸豪遂相率與奐和親,共擊薁鞬等,連戰破之。伯德惶恐,將其眾降,郡界以寧。

羌豪帥感奐恩德,上馬二十匹,先零酋長又遺金鐻八枚。奐並受之,而召主簿於諸羌前,以酒酹地曰:「使馬如羊,不以入廄;使金如粟,不以入懷。」悉以金馬還之。羌性貪而貴吏清,前有八都尉率好財貨,為所患苦,及奐正身絜己,威化大行。

遷使匈奴中郎將。時休屠各及朔方烏桓並同反叛,燒度遼將軍門,引屯赤阬,煙火相望。兵眾大恐,各欲亡去。奐安坐帷中,與弟子講誦自若,軍士稍安。乃潛誘烏桓陰與和通,遂使斬屠各渠帥,襲破其眾。諸胡悉降。

延熹元年,鮮卑寇邊,奐率南單于擊之,斬首數百級。

明年,梁冀被誅,奐以故吏免官禁錮。奐與皇甫規友善,奐既被錮,凡諸交舊莫敢為言,唯規薦舉前後七上。在家四歲,復拜武威太守。平均徭賦,率厲散敗,常為諸郡最,河西由是而全。其俗多妖忌,凡二月、五月產子及與父母同月生者,悉殺之。奐示以義方,嚴加賞罰,風俗遂改,百姓生為立祠。舉尤異,遷度遼將軍。數載閒,幽、并清靜。

九年春,徵拜大司農。鮮卑聞奐去,其夏,遂招結南匈奴、烏桓數道入塞,或五六千騎,或三四千騎,寇掠緣邊九郡,殺略百姓。秋,鮮卑復率八九千騎入塞,誘引東羌與共盟詛。於是上郡沈氐、安定先零諸種共寇武威、張掖,緣邊大被其毒。朝廷以為憂,復拜奐為護匈奴中郎將,以九卿秩督幽、并、涼三州及度遼、烏桓二營,兼察刺史、二千石能否,賞賜甚厚。匈奴、烏桓聞奐至,因相率還降,凡二十萬口。奐但誅其首惡,餘皆慰納之。唯鮮卑出塞去。

永康元年春,東羌、先零五六千騎寇關中,圍祋祤,掠雲陽。夏,復攻沒兩營,殺千餘人。冬,羌岸尾、摩蟞等脅同種復鈔三輔。奐遣司馬尹端、董卓並擊,大破之,斬其酋豪,首虜萬餘人,三州清定。論功當封,奐不事宦官,故賞遂不行,唯賜錢二十萬,除家一人為郎。並辭不受,而願徙屬弘農華陰。舊制邊人不得內移,唯奐因功特聽,故始為弘農人焉。

建寧元年,振旅而還。時竇太后臨朝,大將軍竇武與太傅陳蕃謀誅宦官,事泄,中常侍曹節等於中作亂,以奐新徵,不知本謀,矯制使奐與少府周靖率五營士圍武。武自殺,蕃因見害。奐遷少府,又拜大司農,以功封侯。奐深病為節所賣,上書固讓,封還印綬,卒不肯當。

明年夏,青蛇見於御坐軒前,又大風雨雹,霹靂拔樹,詔使百僚各言災應。奐上疏曰:「臣聞風為號令,動物通氣。木生於火,相須乃明。蛇能屈申,配龍騰蟄。順至為休徵,逆來為殃咎。陰氣專用,則凝精為雹。故大將軍竇武、太傅陳蕃,或志寧社稷,或方直不回,前以讒勝,並伏誅戮,海內默默,人懷震憤。昔周公葬不如禮,天乃動威。今武、蕃忠貞,未被明宥,妖眚之來,皆為此也。宜急為改葬,徙還家屬。其從坐禁錮,一切蠲除。又皇太后雖居南居,而恩禮不接,朝臣莫言,遠近失望。宜思大義顧復之報。」天子深納奐言,以問諸黃門常侍,左右皆惡之,帝不得自從。

轉奐太常,與尚書劉猛、刁韙、衛良同薦王暢、李膺可參三公之選,而曹節等彌疾其言,遂下詔切責之。奐等皆自囚廷尉,數日乃得出,並以三月俸贖罪。司隸校尉王寓,出於宦官,欲借寵公卿,以求薦舉,百僚畏憚,莫不許諾,唯奐獨拒之。寓怒,因此遂陷以黨罪,禁錮歸田里。

奐前為度遼將軍,與段熲爭擊羌,不相平。及熲為司隸校尉,欲逐奐歸敦煌,將害之。奐憂懼,奏記謝熲曰:「小人不明,得過州將,千里委命,以情相歸。足下仁篤,照其辛苦,使人未反,復獲郵書。恩詔分明,前以寫白,而州期切促,郡縣惶懼,屏營延企,側待歸命。父母朽骨,孤魂相託,若蒙矜憐,壹惠咳唾,則澤流黃泉,施及冥寞,非奐生死所能報塞。夫無毛髮之勞,而欲求人丘山之用,此淳于髡所以拍髀仰天而笑者也。誠知言必見譏,然猶未能無望。何者?朽骨無益於人,而文王葬之;死馬無所復用,而燕昭寶之。黨同文、昭之德,豈不大哉!凡人之情,冤則呼天,窮則叩心。今呼天不聞,叩心無益,誠自傷痛。俱生聖世,獨為匪人。孤微之人,無所告訴。如不哀憐,便為魚肉。企心東望,無所復言。」熲雖剛猛,省書哀之,卒不忍也。時禁錮者多不能守靜,或死或徙。奐閉門不出,養徒千人,著尚書記難三十餘萬言。

奐少立志節,嘗與士友言曰:「大丈夫處世,當為國家立功邊境。」及為將帥,果有勳名。董卓慕之,使其兄遺縑百匹。奐惡卓為人,絕而不受。光和四年卒,年七十八。遺命曰:「吾前後仕進,十要銀艾,不能和光同塵,為讒邪所忌。通塞命也,始終常也。但地厎冥冥,長無曉期,而復纏以纊綿,牢以釘密,為不喜耳。幸有前窀,朝殞夕下,措屍靈床,幅巾而已。奢非晉文,儉非王孫,推情從意,庶無咎吝。」諸子從之。武威多為立祠,世世不絕。所著銘、頌、書、教、誡述、志、對策、章表二十四篇。

長子芝,字伯英,最知名。芝及弟昶,字文舒,並善草書,至今稱傳之。

初,奐為武威太守,其妻懷孕,夢帶奐印綬登樓而歌。訊之占者,曰:「必將生男,復臨茲邦,命終此數。」既而生子猛,以建安中為武威太守,殺刺史邯鄲商,州兵圍之急,猛恥見擒,乃登樓自燒而死,卒如占云。

論曰:自鄛鄉之封,中官世盛,暴恣數十年閒,四海之內,莫不切齒憤盈,願投兵於其族。陳蕃、竇武奮義草謀,徵會天下,名士有識所共聞也,而張奐見欺豎子,揚戈以斷忠烈。雖恨毒在心,辭爵謝咎。《詩》云:「啜其泣矣,何嗟及矣!」

段熲字紀明,武威姑臧人也。其先出鄭共叔段,西域都護會宗之從曾孫也。熲少便習弓馬,尚遊俠,輕財賄,長乃折節好古學。初舉孝廉,為憲陵園丞、陽陵令,所在能政。

遷遼東屬國都尉。時鮮卑犯塞,熲即率所領馳赴之。既而恐賊驚去,乃使驛騎詐齎璽書詔熲,熲於道偽退,潛於還路設伏。虜以為信然,乃入追熲。熲因大縱兵,悉斬獲之。坐詐璽書伏重刑,以有功論司寇。刑竟,徵拜議郎。

時太山、琅邪賊東郭竇、公孫舉等聚眾三萬人,破壞郡縣,遣兵討之,連年不克。永壽二年,桓帝詔公卿選將有文武者,司徒尹訟薦熲,乃拜為中郎將。擊竇、舉等,大破斬之,獲首萬餘級,餘黨降散。封熲為列侯,賜錢五十萬,除一子為郎中。

延熹二年,遷護羌校尉。會燒當、燒何、當煎、勒姐等八種羌寇隴西、金城塞,熲將兵及湟中義從羌萬二千騎出湟谷,擊破之。追討南度河,使軍吏田晏、夏育募先登,懸索相引,復戰於羅亭,大破之,斬其酋豪以下二千級,獲生口萬餘人,虜皆奔走。

明年春,餘羌復與燒何大豪寇張掖,攻沒鉅鹿塢,殺屬國吏民,又招同種千餘落,并兵晨奔熲軍。熲下馬大戰,至日中,刀折矢盡,虜亦引退。熲追之,且鬥且行,晝夜相攻,割肉食雪,四十餘日,遂至河首積石山,出塞二千餘里,斬燒何大帥,首虜五千餘人。又分兵擊石城羌,斬首溺死者千六百人。燒當種九十餘口詣熲降。又雜種羌屯聚白石,熲復進擊,首虜三千餘人。冬,勒姐、零吾種圍允街,殺略吏民,熲排營救之,斬獲數百人。

四年冬,上郡沈氐、隴西牢姐、烏吾諸種羌共寇并涼二州,熲將湟中義從討之。涼州刺史郭閎貪共其功,稽固熲軍,使不得進。義從役久,戀鄉舊,皆悉反叛。郭閎歸罪於熲,熲坐徵下獄,輸作左校。羌遂陸梁,覆沒營塢,轉相招結,唐突諸郡,於是吏人守闕訟熲以千數。朝廷知熲為郭閎所誣,詔問其狀。熲但謝罪,不敢言枉,京師稱為長者。起於徒中,復拜議郎,遷并州刺史。

時滇那等諸種羌五六千人寇武威、張掖、酒泉,燒人廬舍。六年,寇埶轉盛,涼州幾亡。冬,復以熲為護羌校尉,乘驛之職。明年春,羌封僇、良多、滇那等酋豪三百五十五人率三千落詣熲降。當煎、勒姐種猶自屯結。冬,熲將萬餘人擊破之,斬其酋豪,首虜四千餘人。

八年春,熲復擊勒姐種,斬首四百餘級,降者二千餘人。夏,進軍擊當煎種於湟中,熲兵敗,被圍三日,用隱士樊志張策,潛師夜出,鳴鼓還戰,大破之,首虜數千人。熲遂窮追,展轉山谷閒,自春及秋,無日不戰,虜遂飢困敗散,北略武威閒。

熲凡破西羌,斬首二萬三千級,獲生口數萬人,馬牛羊八百萬頭,降者萬餘落。封熲都鄉侯,邑五百戶。

永康元年,當煎諸種復反,合四千餘人,欲攻武威,熲復追擊於鸞鳥,大破之,殺其渠帥,斬首三千餘級,西羌於此弭定。

而東羌先零等,自覆沒征西將軍馬賢後,朝廷不能討,遂數寇擾三輔。其後度遼將軍皇甫規、中郎將張奐招之連年,既降又叛。桓帝詔問熲曰:「先零東羌造惡反逆,而皇甫規、張奐各擁強眾,不時輯定。欲熲移兵東討,未識其宜,可參思術略。」熲因上言曰:「臣伏見先零東羌雖數叛逆,而降於皇甫規者,已二萬許落,善惡既分,餘寇無幾。今張奐躊躇久不進者,當慮外離內合,兵往必驚。且自冬踐春,屯結不散,人畜疲羸,自亡之埶,徒更招降,坐制強敵耳。臣以為狼子野心,難以恩納,埶窮雖服,兵去復動。唯當長矛挾脅,白刃加頸耳。計東種所餘三萬餘落,居近塞內,路無險折,非有燕、齊、秦、趙從橫之埶,而久亂并、涼,累侵三輔,西河、上郡,已各內徙,安定、北地,復至單危,自雲中、五原,西至漢陽二千餘里,匈奴、種羌,並擅其地,是為帻疽伏疾,留滯脅下,如不加誅,轉就滋大。今若以騎五千,步萬人,車三千兩,三冬二夏,足以破定,無慮用費為錢五十四億。如此,則可令群羌破盡,匈奴長服,內徙郡縣,得反本土。伏計永初中,諸羌反叛,十有四年,用二百四十億;永和之末,復經七年,用八十餘億。費耗若此,猶不誅盡,餘孽復起,于茲作害。今不暫疲人,則永寧無期。臣庶竭駑劣,伏待節度。」帝許之,悉聽如所上。

建寧元年春,熲將兵萬餘人,齎十五日糧,從彭陽直指高平,與先零諸種戰於逢義山。虜兵盛,熲眾恐。熲乃令軍中張鏃利刃,長矛三重,挾以強弩,列輕騎為左右翼。淚怒兵將曰:「今去家數千里,進則事成,走必盡死,努力共功名!」因大呼,眾皆應騰赴,熲馳騎於傍,突而擊之,虜眾大潰,斬首八千餘級,獲牛馬羊二十八萬頭。

時竇太后臨朝,下詔曰:「先零東羌歷載為患,熲前陳狀,欲必埽滅。涉履霜雪,兼行晨夜,身當矢石,感厲吏士。曾未浹日,凶醜奔破,連尸積俘,掠獲無筭。洗雪百年之逋負,以慰忠將之亡魂。功用顯著,朕甚嘉之。須東羌盡定,當并錄功勤。今且賜熲錢二十萬,以家一人為郎中。」敕中藏府調金錢綵物,增助軍費。拜熲羌將軍。

夏,熲復追羌出橋門,至走馬水上。尋聞虜在奢延澤,乃將輕兵兼行,一日一夜二百餘里,晨及賊,擊破之。餘虜走向落川,復相屯結。熲乃分遣騎司馬田晏將五千人出其東,假司馬夏育將二千人繞其西。羌分六七千人攻圍晏等,晏等與戰,羌潰走。熲急進,與晏等共追之於令鮮水上。熲士卒飢渴,乃勒眾推方奪其水,虜復散走。熲遂與相連綴,且鬥且引,及於靈武谷。熲乃被甲先登,士卒無敢後者。羌遂大敗,棄兵而走。追之三日三夜,士皆重繭。既到涇陽,餘寇四千落,悉散入漢陽山谷閒。

時張奐上言:「東羌雖破,餘種難盡,熲性輕果,慮負敗難常。宜且以恩降,可無後悔。」詔書下熲。熲復上言:「臣本知東羌雖眾,而軟弱易制,所以比陳愚慮,思為永寧之筭。而中郎將張奐,說虜強難破,宜用招降。聖朝明監,信納瞽言,故臣謀得行,奐計不用。事埶相反,遂懷猜恨。信叛羌之訴,飾潤辭意,云臣兵累見折衄,又言羌一氣所生,不可誅盡,山谷廣大,不可空靜,血流汙野,傷和致災。臣伏念周秦之際,戎狄為害,中興以來,羌寇最盛,誅之不盡,雖降復叛。今先零雜種,累以反覆,攻沒縣邑,剽略人物,發冢露尸,禍及生死,上天震怒,假手行誅。昔邢為無道,衛國伐之,師興而雨。臣動兵涉夏,連獲甘澍,歲時豐稔,人無疵疫。上占天心,不為災傷;下察人事,眾和師克。自橋門以西,落川以東,故宮縣邑,更相通屬,非為深險絕域之地,車騎安行,無應折衄。案奐為漢吏,身當武職,駐軍二年,不能平寇,虛欲修文戢戈,招降獷敵,誕辭空說,僭而無徵。何以言之?昔先零作寇,趙充國徙令居內,煎當亂邊,馬援遷之三輔,始服終叛,至今為鯁。故遠識之士,以為深憂。今傍郡戶口單少,數為羌所創毒,而欲令降徒與之雜居,是猶種枳棘於良田,養虺蛇於室內也。故臣奉大漢之威,建長久之策,欲絕其本根,不使能殖。本規三歲之費,用五十四億,今適期年,所耗未半,而餘寇殘燼,將向殄滅。臣每奉詔書,軍不內御,願卒斯言,一以任臣,臣時量宜,不失權便。」

二年,詔遣謁者馮禪說降漢陽散羌。熲以春農,百姓布野,羌雖暫降,而縣官無廩,必當復為盜賊,不如乘虛放兵,埶必殄滅。夏,熲自進營,去羌所屯凡亭山四五十里,遣田晏、夏育將五千人據其山上。羌悉眾攻之,厲聲問曰:「田晏、夏育在此不?湟中義從羌悉在何面?今日欲決死生。」軍中恐,晏等勸激兵士,殊死大戰,遂破之。羌眾潰,東奔,復聚射虎谷,分兵守諸谷上下門。熲規一舉滅之,不欲復令散走,乃遣千人於西縣結木為柵,廣二十步,長四十里,遮之。分遣晏、育等將七千人,銜枚夜上西山,結營穿塹,去虜一里許。又遣司馬張愷等將三千人上東山。虜乃覺之,遂攻晏等,分遮汲水道。熲自率步騎進擊水上,羌卻走,因與愷等挾東西山,縱兵擊破之,羌復敗散。熲追至谷上下門窮山深谷之中,處處破之,斬其渠帥以下萬九千級,獲牛馬驢騾氈裘廬帳什物,不可勝數。馮禪等所招降四千人,分置安定、漢陽、隴西三郡,於是東羌悉平。

凡百八十戰,斬三萬八千六百餘級,獲牛馬羊騾驢駱駝四十二萬七千五百餘頭,費用四十四億,軍士死者四百餘人。更封新豐縣侯,邑萬戶。熲行軍仁愛,士卒疾病者,親自瞻省,手為裹創。在邊十餘年,未嘗一日蓐寢。與將士同苦,故皆樂為死戰。

三年春,徵還京師,將秦胡步騎五萬餘人,及汗血千里馬,生口萬餘人。詔遣大鴻臚持節慰勞於鎬。軍至,拜侍中。轉執金吾河南尹。有盜發馮貴人冢,坐左轉諫議大夫,再遷司隸校尉。

熲曲意宦官,故得保其富貴,遂黨中常侍王甫,枉誅中常侍鄭颯、董騰等,增封四千戶,并前萬四千戶。

明年,伐李咸為太尉,其冬病罷,復為司隸校尉。數歲,轉潁川太守,徵拜太中大夫。

光和二年,復代橋玄為太尉。在位月餘,會日食自劾,有司舉奏,詔收印綬,詣廷尉。時司隸校尉陽球奏誅王甫,并及熲,就獄中詰責之,遂飲鴆死,家屬徙邊。後中常侍呂強上疏,追訟熲功,靈帝詔熲妻子還本郡。

初,熲與皇甫威明、張然明,並知名顯達,京師稱為「涼州三明」云。

贊曰:山西多猛,「三明」儷蹤。戎驂糾結,塵斥河、潼。規、奐審策,亟遏囂凶。文會志比,更相為容。段追兩狄,束馬縣鋒。紛紜騰突,谷靜山空。


\end{pinyinscope}