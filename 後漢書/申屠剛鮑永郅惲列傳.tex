\article{申屠剛鮑永郅惲列傳}

\begin{pinyinscope}
申屠剛字巨卿,扶風茂陵人也。七世祖嘉,文帝時為丞相。剛質性方直,常慕史怅、汲黯之為人。仕郡功曹。

平帝時,王莽專政,朝多猜忌,遂隔絕帝外家馮衛二族,不得交宦,剛常疾之。及舉賢良方正,因對策曰:

臣聞王事失則神祇怨怒,姦邪亂正,故陰陽謬錯。此天所以譴告王者,欲令失道之君,曠然覺悟,懷邪之臣,懼然自刻者也。今朝廷不考功校德,而虛納毀譽,數下詔書,張設重法,抑斷誹謗,禁割論議,罪之重者,乃至腰斬。傷忠臣之情,挫直士之銳,殆乖建進善之旌,縣敢諫之鼓,闢四門之路,明四目之義也。

臣聞成王幼少,周公攝政,聽言下賢,均權布寵,無舊無新,唯仁是親,動順天地,舉措不失。然近則召公不悅,遠則四國流言。夫子母之性,天道至親。今聖主幼少,始免繈褓,即位以來,至親分離,外戚杜隔,恩不得通。且漢家之制,雖任英賢,猶援姻戚。親疏相錯,杜塞閒隙,誠所以安宗廟,重社稷也。今馮、衛無罪,久廢不錄,或處窮僻,不若民庶,誠非慈愛忠孝承上之意。夫為人後者,自有正義,至尊至卑,其埶不嫌,是以人無賢愚,莫不為怨,姦臣賊子,以之為便,不諱之變,誠難其慮。今之保傅,非古之周公。周公至聖,猶尚有累,何況事失其衷,不合天心者哉?昔周公先遣伯禽守封於魯,以義割恩,寵不加後,故配天郊祀,三十餘世。霍光秉政,輔翼少主,修善進士,名為忠直,而尊其宗黨,摧抑外戚,結貴據權,至堅至固,終沒之後,受禍滅門。方今師傅皆以伊、周之位,據賢保之任,以此思化,則功何不至?不思其危,則禍何不到?損益之際,孔父攸歎,持滿之戒,老氏所慎。蓋功冠天下者不安,威震人主者不全。今承衰亂之後,繼重敝之世,公家屈竭,賦斂重數,苛吏奪其時,貪夫侵其財,百姓困乏,疾疫夭命。盜賊群輩,且以萬數,軍行眾止,竊號自立,攻犯京師,燔燒縣邑,至乃訛言積弩入宮,宿衛驚懼。自漢興以來,誠未有也。國家微弱,姦謀不禁,六極之效,危於累卵。王者承天順地,典爵主刑,不敢以天官私其宗,不敢以天罰輕其親。陛下宜遂聖明之德,昭然覺悟,遠述帝王之跡,近遵孝文之業,差五品之屬,納至親之序,亟遣使者徵中山太后,置之別宮,令時朝見。又召馮衛二族,裁與冗職,使得執戟,親奉宿衛,以防未然之符,以抑患禍之端。上安社稷,下全保傅,內和親戚,外絕邪謀。

後莽篡位,剛遂避地河西,轉入巴蜀,往來二十許年。及隗囂據隴右,欲背漢而附公孫述。剛說之曰:「愚聞人所歸者天所與,人所畔者天所去也。伏念本朝躬聖德,舉義兵,龔行天罰,所當必摧,誠天之所福,非人力也。將軍本無尺土,孤立一隅,宜推誠奉順,與朝并力,上應天心,下酬人望,為國立功,可以永年。嫌疑之事,聖人所絕。以將軍之威重,遠在千里,動作舉措,可不慎與?今璽書數到,委國歸信,欲與將軍共同吉凶。布衣相與,尚有沒身不負然諾之信,況於萬乘者哉!今何畏何利,久疑如是?卒有非常之變,上負忠孝,下愧當世。夫未至豫言,固常為虛,及其已至,又無所及,是以忠言至諫,希得為用。誠願反覆愚老之言。」囂不納,遂畔從述。

建武七年,詔書徵剛。剛將歸,與囂書曰:「愚聞專己者孤,拒諫者塞,孤塞之政,亡國之風也。雖有明聖之姿,猶屈己從眾,故慮無遺策,舉無過事。夫聖人不以獨見為明,而以萬物為心。順人者昌,逆人者亡,此古今之所共也。將軍以布衣為鄉里所推,廊廟之計,既不豫定,動軍發眾,又不深料。今東方政教日睦,百姓平安,而西州發兵,人人懷憂,騷動惶懼,莫敢正言,群眾疑惑,人懷顧望。非徒無精銳之心,其患無所不至。夫物窮則變生,事急則計易,其埶然也。夫離道德,逆人情,而能有國有家者,古今未有也。將軍素以忠孝顯聞,是以士大夫不遠千里,慕樂德義。今苟欲決意徼幸,此何如哉?夫天所祐者順,人所助者信。如未蒙祐助,令小人受塗地之禍,毀壞終身之德,敗亂君臣之節,污傷父子之恩,眾賢破膽,可不慎哉!」囂不納。剛到,拜侍御史,遷尚書令。

光武嘗欲出游,剛以隴蜀未平,不宜宴安逸豫。諫不見聽,遂以頭軔乘輿輪,帝遂為止。

時內外群官,多帝自選舉,加以法理嚴察,職事過苦,尚書近臣,至乃捶撲牽曳於前,群臣莫敢正言。剛每輒極諫,又數言皇太子宜時就東宮,簡任賢保,以成其德,帝並不納。以數切諫失旨,數年,出為平陰令。復徵拜太中大夫,以病去官,卒於家。

鮑永字君長,上黨屯留人也。父宣,哀帝時任司隸校尉,為王莽所殺。永少有志操,習歐陽尚書。事後母至孝,妻嘗於母前叱狗,而永即去之。

初為郡功曹。莽以宣不附己,欲滅其子孫。都尉路平承望風旨,規欲害永。太守苟諫擁護,召以為吏,常置府中。永因數為諫陳興復漢室,翦滅篡逆之策。諫每戒永曰:「君長幾事不密,禍倚人門。」永感其言。及諫卒,自送喪歸扶風。路平遂收永弟升。太守趙興到,聞乃歎曰:「我受漢茅土,不能立節,而鮑宣死之,豈可害其子也!」敕縣出升,復署永功曹。時有矯稱侍中止傳舍者,興欲謁之。永疑其詐,諫不聽而出,興遂駕往,永乃拔佩刀截馬當匈,乃止。後數日,莽詔書果下捕矯稱者,永由是知名。舉秀才,不應。

更始二年徵,再遷尚書僕射,行大將軍事,持節將兵,安集河東、并州、朔部,得自置偏裨,輒行軍法。永至河東,因擊青犢,大破之,更始封為中陽侯。永雖為將率,而車服敝素,為道路所識。

時赤眉害更始,三輔道絕。光武即位,遣諫議大夫儲大伯,持節徵永詣行在所。永疑不從,乃收繫大伯,遣使馳至長安。既知更始已亡,乃發喪,出大伯等,封上將軍列侯印綬,悉罷兵,但幅巾與諸將及同心客百餘人詣河內。帝見永,問曰:「卿眾所在?」永離席叩頭曰:「臣事更始,不能令全,誠慚以其眾幸富貴,故悉罷之。」帝曰:「卿言大!」而意不悅。時攻懷未拔,帝謂永曰:「我攻懷三日而兵不下,關東畏服卿,可且將故人自往城下譬之。」即拜永諫議大夫。至懷,乃說更始河內太守,於是開城而降。帝大喜,賜永洛陽商里宅,固辭不受。

時董憲裨將屯兵於魯,侵害百姓,乃拜永為魯郡太守。永到,擊討,大破之,降者數千人。唯別帥彭豐、虞休、皮常等各千餘人,稱「將軍」,不肯下。頃之,孔子闕里無故荊棘自除,從講堂至于里門。永異之,謂府丞及魯令曰:「方今危急而闕里自開,斯豈夫子欲令太守行禮,助吾誅無道邪?」乃會人眾,修鄉射之禮,請豐等共會觀視,欲因此禽之。豐等亦欲圖永,乃持牛酒勞饗,而潛挾兵器。永覺之,手格殺豐等,禽破黨與。帝嘉其略,封為關內侯,遷楊州牧。時南土尚多寇暴,永以吏人痍傷之後,乃緩其銜轡,示誅彊橫而鎮撫其餘,百姓安之。會遭母憂,去官,悉以財產與孤弟子。

建武十一年,徵為司隸校尉。帝叔父趙王良尊戚貴重,永以事劾良大不敬,由是朝廷肅然,莫不戒慎。乃辟扶風鮑恢為都官從事,恢亦抗直不避彊禦。帝常曰:「貴戚且宜斂手,以避二鮑。」其見憚如此。

永行縣到霸陵,路經更始墓,引車入陌,從事諫止之。永曰:「親北面事人,寧有過墓不拜!雖以獲罪,司隸所不避也。」遂下拜,哭盡哀而去。西至扶風,椎牛上苟諫冢。帝聞之,意不平,問公卿曰:「奉使如此何如?」太中大夫張湛對曰:「仁者行之宗,忠者義之主也。仁不遺舊,忠不忘君,行之高者也。」帝意乃釋。

後大司徒韓歆坐事,永固請之不得,以此忤帝意,出為東海相。坐度田事不實,被徵,諸郡守多下獄。永至城皋,詔書逆拜為兗州牧,便道之官。視事三年,病卒。子昱。

論曰:鮑永守義於故主,斯可以事新主矣。恥以其眾受寵,斯可以受大寵矣。若乃言之者雖誠,而聞之未譬,豈苟進之悅,易以情納,持正之忤,難以理求乎?誠能釋利以循道,居方以從義,君子之概也。

昱字文泉。少傳父學,客授於東平。建武初,太行山中有劇賊,太守戴涉聞昱鮑永子,有智略,乃就謁,請署守高都長。昱應之,遂討擊群賊,誅其渠帥,道路開通,由是知名。後為沘陽長,政化仁愛,境內清淨。

荊州刺史表上之,再遷,中元元年,拜司隸校尉。詔昱詣尚書,使封胡降檄。光武遣小黃門問昱有所怪不?對曰:「臣聞故事通官文書不著姓,又當司徒露布,怪使司隸下書而著姓也。」帝報曰:「吾故欲令天下知忠臣之子復為司隸也。」昱在職,奉法守正,有父風,永平五年,坐救火遲,免。

後拜汝南太守。郡多陂池,歲歲決壞,年費常三千餘萬。昱乃上作方梁石洫,水常饒足,溉田倍多,人以殷富。

十七年,代王敏為司徒,賜錢帛什器帷帳,除子得為郎。建初元年,大旱,穀貴。肅宗召昱問曰:「旱既大甚,將何以消復災眚?」對曰:「臣聞聖人理國,三年有成。今陛下始踐天位,刑政未著,如有失得,何能致異?但臣前在汝南,典理楚事,繫者千餘人,恐未能盡當其罪。先帝詔言,大獄一起,冤者過半。又諸徙者骨肉離分,孤魂不祀。一人呼嗟,王政為虧。宜一切還諸徙家屬,蠲除禁錮,興滅繼絕,死生獲所。如此,和氣可致。」帝納其言。

四年,代牟融為太尉。六年,薨,年七十餘。

子德,修志節,有名稱,累官為南陽太守。時歲多荒災,唯南陽豐穰,吏人愛悅,號為神父。時郡學久廢,德乃修起橫舍,備俎豆黻冕,行禮奏樂。又尊饗國老,宴會諸儒。百姓觀者,莫不勸服。在職九年,徵拜大司農,卒于官。

子昂,字叔雅,有孝義節行。初,德被病數年,昂俯伏左右,衣不緩帶;及處喪,毀瘠三年,抱負乃行;服闋,遂潛于墓次,不關時務。舉孝廉,辟公府,連徵不至,卒於家。

郅惲字君章,汝南西平人也。年十二失母,居喪過禮。及長,理韓詩、嚴氏春秋,明天文歷數。

王莽時,寇賊群發,惲乃仰占玄象,歎謂友人曰:「方今鎮、歲、熒惑並在漢分翼、軫之域,去而復來,漢必再受命,福歸有德。如有順天發策者,必成大功。」時左隊大夫逯並素好士,惲說之曰:「當今上天垂象,智者以昌,愚者以亡。昔伊尹自鬻輔商,立功全人。惲竊不遜,敢希伊尹之蹤,應天人之變。明府儻不疑逆,俾成天德。」並奇之,使署為吏。惲不謁,曰:「昔文王拔呂尚於渭濱,高宗禮傅說於巖築,桓公取管仲於射鉤,故能立弘烈,就元勳。未聞師相仲父,而可為吏位也。非闚天者不可與圖遠。君不授驥以重任,驥亦俛首裹足而去耳。」遂不受署。

西至長安,乃上書王莽曰:「臣聞天地重其人,惜其物,故運機衡,垂日月,含元包一,甄陶品類,顯表紀世,圖錄豫設。漢歷久長,孔為赤制,不使愚惑,殘人亂時。智者順以成德,愚者逆以取害,神器有命,不可虛獲。上天垂戒,欲悟陛下,令就臣位,轉禍為福。劉氏享天永命,陛下順節盛衰,取之以天,還之以天,可謂知命矣。若不早圖,是不免於竊位也。且堯舜不以天顯自與,故禪天下,陛下何貪非天顯以自累也?天為陛下嚴父,臣為陛下孝子。父教不可廢,子諫不可拒,惟陛下留神。」莽大怒,即收繫詔獄,劾以大逆。猶以惲據經讖,難即害之,使黃門近臣脅惲,令自告狂病恍忽,不覺所言。惲乃瞋目詈曰:「所陳皆天文聖意,非狂人所能造。」遂繫須冬,會赦得出,乃與同郡鄭敬南遁蒼梧。

建武三年,又至廬江,因遇積弩將軍傅俊東徇揚州。俊素聞惲名,乃禮請之,上為將兵長史,授以軍政。惲乃誓眾曰:「無掩人不備,窮人於厄,不得斷人支體,祼人形骸,放淫婦女。」俊軍士猶發冢陳尸,掠奪百姓。惲諫俊曰:「昔文王不忍露白骨,武王不以天下易一人之命,故能獲天地之應,剋商如林之旅。將軍如何不師法文王,而犯逆天地之禁,多傷人害物,虐及枯尸,取罪神明?今不謝天改政,無以全命。願將軍親率士卒,收傷葬死,哭所殘暴,以明非將軍本意也。」從之,百姓悅服,所向皆下。

七年,俊還京師,而上論之。惲恥以軍功取位,遂辭歸鄉里。縣令卑身崇禮,請以為門下掾。惲友人董子張者,父先為鄉人所害。及子張病,將終,惲往候之。子張垂歿,視惲,歔欷不能言。惲曰:「吾知子不悲天命,而痛讎不復也。子在,吾憂而不手;子亡,吾手而不憂也。」子張但目擊而已。惲即起,將客遮仇人,取其頭以示子張。子張見而氣絕。惲因而詣縣,以狀自首。令應之遲,惲曰:「為友報讎,吏之私也。奉法不阿,君之義也。虧君以生,非臣節也。」趨出就獄。令跣而追惲,不及,遂自至獄,令拔刃自向以要惲曰:「子不從我出,敢以死明心。」惲得此乃出,因病去。

久之,太守歐陽歙請為功曹。汝南舊俗,十月饗會,百里內縣皆齎牛酒到府讌飲。時臨饗禮訖,歙教曰:「西部督郵繇延,天資忠貞,稟性公方,摧破姦凶,不嚴而理。今與眾儒共論延功,顯之于朝。太守敬嘉厥休,牛酒養德。」主簿讀書教,戶曹引延受賜。惲於下坐愀然前曰:「司正舉觥,以君之罪,告謝于天。案延資性貪邪,外方內員,朋黨搆姦,罔上害人,所在荒亂,怨慝並作。明府以惡為善,股肱以直從曲,此既無君,又復無臣,惲敢再拜奉觥。」歙色慚動,不知所言。門下掾鄭敬進曰:「君明臣直,功曹言切,明府德也,可無受觥哉?」歙意少解,曰:「實歙罪也,敬奉觥。」惲乃免冠謝曰:「昔虞舜輔堯,四罪咸服,讒言弗庸,孔任不行,故能作股肱,帝用有歌。惲不忠,孔任是昭,豺虎從政,既陷誹謗,又露所言,罪莫重焉。請收惲、延,以明好惡。」歙曰:「是重吾過也。」遂不讌而罷。惲歸府,稱病,延亦自退。

鄭敬素與惲厚,見其言忤歙,乃相招去,曰:「子廷爭繇延,君猶不納。延今雖去,其埶必還。直心無諱,誠三代之道。然道不同者不相為謀,吾不能忍見子有不容君之危,盍去之乎!」惲曰:「孟軻以彊其君之所不能為忠,量其君之所不能為賊。惲業已彊之矣。障君於朝,既有其直,而不死職,罪也。延退而惲又去,不可。」敬乃獨隱於弋陽山中。居數月,歙果復召延,惲於是乃去,從敬止,漁釣自娛,留數十日。惲志在從政,既乃喟然而歎,謂敬曰:「天生俊士,以為人也。烏獸不可與同群,子從我為伊呂乎?將為巢許,而父老堯舜乎?」敬曰:「吾足矣。初從生步重華於南野,謂來歸為松子,今幸得全軀樹類,還奉墳墓,盡學問道,雖不從政,施之有政,是亦為政也。吾年耄矣,安得從子?子勉正性命,勿勞神以害生。」惲於是告別而去。敬字次都,清志高世,光武連徵不到。

惲遂客居江夏教授,郡舉孝廉,為上東城門候。帝嘗出獵,車駕夜還,惲拒關不開。帝令從者見面於門閒。惲曰:「火明遼遠。」遂不受詔。帝乃迴從東中門入。明日,惲上書諫曰:「昔文王不敢槃于游田,以萬人惟憂。而陛下遠獵山林,夜以繼晝,其如社岜宗廟何?暴虎馮河,未至之戒,誠小臣所竊憂也。」書奏,賜布百匹,貶東中門候為參封尉。

後令惲授皇太子韓詩,侍講殿中。及郭皇后廢,惲乃言於帝曰:「臣聞夫婦之好,父不能得之於子,況臣能得之於君乎?是臣所不敢言。雖然,願陛下念其可否之計,無令天下有議社稷而已。」帝曰:「惲善恕己量主,知我必不有所左右而輕天下也。」后既廢,而太子意不自安,惲乃說太子曰:「久處疑位,上違孝道,下近危殆。昔高宗明君,吉甫賢臣,及有纖介,放逐孝子。春秋之義,母以子貴。太子宜因左右及諸皇子引愆退身,奉養母氏,以明聖教,不背所生。」太子從之,帝竟聽許。

惲再遷長沙太守。先是長沙有孝子古初,遭父喪未葬,鄰人失火,初匍匐柩上,以身扞火,火為之滅。惲甄異之,以為首舉。後坐事左轉芒長,又免歸,避地教授,著書八篇。以病卒。子壽。

壽字伯考,善文章,以廉能稱,舉孝廉,稍遷冀州刺史。時冀部屬郡多封諸王,賓客放縱,類不檢節,壽案察之,無所容貸。乃使部從事專住王國,又徙督郵舍王宮外,動靜失得,即時騎驛言上奏王罪及劾傅相,於是藩國畏懼,並為遵節。視事三年,冀土肅清。三遷尚書令。朝廷每有疑議,常獨進見。肅宗奇其智策,擢為京兆尹。郡多彊豪,姦暴不禁。三輔素聞壽在冀州,皆懷震竦,各相檢敕,莫敢干犯。壽雖威嚴,而推誠下吏,皆願效死,莫有欺者。以公事免。

復徵為尚書僕射。是時大將軍竇憲以外戚之寵,威傾天下。憲嘗使門生齎書詣壽,有所請託,壽即送詔獄。前後上書陳憲驕恣,引王莽以誡國家。是時憲征匈奴,海內供其役費,而憲及其弟篤、景並起第宅,驕奢非法,百姓苦之。壽以府臧空虛,軍旅未休,遂因朝會譏刺憲等,厲音正色,辭旨甚切。憲怒,陷壽以買公田誹謗,下吏當誅。侍御史何敞上疏理之曰:「臣聞聖王闢四門,開四聰,延直言之路,下不諱之詔,立敢諫之旗,聽歌謠於路,爭臣七人,以自鑒照,考知政理,違失人心,輒改更之,故天人並應,傳福無窮。臣伏見尚書僕射郅壽坐於臺上,與諸尚書論擊匈奴,言議過差,及上書請買公田,遂繫獄考劾大不敬。臣愚以為壽機密近臣,匡救為職。若懷默不言,其罪當誅。今壽違眾正議,以安宗廟,豈其私邪?又臺閣平事,分爭可否,雖唐虞之隆,三代之盛,猶謂諤諤以昌,不以誹謗為罪。請買公田,人情細過,可裁隱忍。壽若被誅,臣恐天下以為國家橫罪忠直,賊傷和氣,忤逆陰陽。臣所以敢犯嚴威,不避夷滅,觸死瞽言,非為壽也。忠臣盡節,以死為歸。臣雖不知壽,度其甘心安之。誠不欲聖朝行誹謗之誅,以傷晏晏之化,杜塞忠直,垂譏無窮。臣敞謬豫機密,言所不宜,罪名明白,當填牢獄,先壽僵仆,萬死有餘。」書奏,壽得減死,論徙合浦。未行,自殺,家屬得歸鄉里。

贊曰:鮑永沈吟,晚乃歸正。志達義全,先號後慶。申屠對策,郅惲上書。有道雖直,無道不愚。


\end{pinyinscope}