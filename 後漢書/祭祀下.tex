\article{祭祀下}

\begin{pinyinscope}
宗廟社稷靈星先農迎春

光武帝建武二年正月,立高廟于雒陽。四時祫祀,高帝為太祖,文帝為太宗,武帝為世宗,如舊。餘帝四時春以正月,夏以四月,秋以七月,冬以十月及臘,一歲五祀。三年正月,立親廟雒陽,祀父南頓君以上至舂陵節侯。時寇賊未夷,方務征伐,祀儀未設。至十九年,盜賊討除,戎事差息,於是五官中郎將張純與太僕朱浮奏議:「禮,為人子事大宗,降其私親。禮之設施,不授之與自得之異意。當除今親廟四。孝宣皇帝以孫後祖,為父立廟於奉明,曰皇考廟,獨群臣侍祠。願下有司議先帝四廟當代親廟者及皇考廟事。」下公卿、博士、議郎。大司徒涉等議:「宜奉所代,立平帝、哀帝、成帝、元帝廟,代今親廟。兄弟以下,使有司祠。宜為南頓君立皇考廟,祭上至舂陵節侯,群臣奉祠。」時議有異,不著。上可涉等議,詔曰:「以宗廟處所未定,且祫祭高廟。其成、哀、平且祠祭長安故高廟。其南陽舂陵歲時各且因故園廟祭祀。園廟去太守治所遠者,在所令長行太守事侍祠。惟孝宣帝有功德,其上尊號曰中宗。」於是雒陽高廟四時加祭孝宣、孝元,凡五帝。其西廟成、哀、平三帝主,四時祭於故高廟。東廟京兆尹侍祠,冠衣車服如太常祠陵廟之禮。南頓君以上至節侯,皆就園廟。南頓君稱皇考廟,鉅鹿都尉稱皇祖考廟,鬱林太守稱皇曾祖考廟,節侯稱皇高祖考廟,在所郡縣侍祠。

二十六年,有詔問張純,禘祫之禮不施行幾年。純奏:「禮,三年一祫,五年一禘。毀廟之主,陳於太祖;未毀廟之主,皆升合食太祖;五年再殷祭。舊制,三年一祫,毀廟主合食高廟,存廟主未嘗合。元始五年,始行禘禮。父為昭,南嚮;子為穆,北嚮。父子不並坐,而孫從王父。禘之為言諦。諦諟昭穆,尊卑之義。以夏四月陽氣在上,陰氣在下,故正尊卑之義。祫以冬十月,五穀成熟,故骨肉合飲食。祖宗廟未定,且合祭。今宜以時定。」語在純傳。上難復立廟,遂以合祭高廟為常。後以三年冬祫五年夏禘之時,但就陳祭毀廟主而已,謂之殷。太祖東面,惠、文、武、元帝為昭,景、宣帝為穆。惠、景、昭三帝非殷祭時不祭。光武皇帝崩,明帝即位,以光武帝撥亂中興,更為起廟,尊號曰世祖廟。以元帝於光武為穆,故雖非宗,不毀也。後遂為常。

明帝臨終遺詔,遵儉無起寢廟,藏主於世祖廟更衣。孝章即位,不敢違,以更衣有小別,上尊號曰顯宗廟,閒祠於更衣,四時合祭於世祖廟。語在章紀。章帝臨崩,遺詔無起寢廟,廟如先帝故事。和帝即位不敢違,上尊號曰肅宗。後帝承尊,皆藏主于世祖廟,積多無別,是後顯宗但為陵寢之號。永元中,和帝追尊其母梁貴人曰恭懷皇后,陵。以竇后配食章帝,恭懷后別就陵寢祭之。和帝崩,上尊號曰穆宗。殤帝生三百餘日而崩,鄧太后攝政,以尚嬰孫,故不列于廟,就陵寢祭之而已。安帝以清河孝王子即位,建光元年,追尊其祖母宋貴人曰敬隱后,陵曰敬北陵。亦就陵寢祭,太常領如西陵。追尊父清河孝王曰孝德皇,母曰孝德后,清河嗣王奉祭而已。安帝以讒害大臣,廢太子,及崩,無上宗之奏。後以自建武以來無毀者,故遂常祭,因以其陵號稱恭宗。順帝即位,追尊其母曰恭愍后,陵曰恭北陵。就陵寢祭,如敬北陵。順帝崩,上尊號曰敬宗。沖質帝皆小崩,梁太后攝政,以殤帝故事,就陵寢祭。凡祠廟訖,三公分祭之。桓帝以河閒孝王孫蠡吾侯即位,亦追尊祖考,王國奉祀。語在章和八王傳。帝崩,上尊號曰威宗,無嗣。靈帝以河閒孝王曾孫解犢侯即位,亦追尊祖考。語在章和八王傳。靈帝時,京都四時所祭高廟五主,世祖廟七主,少帝三陵,追尊后三陵,凡牲用十八太牢,皆有副倅。故高廟三主親毀之後,亦但殷祭之歲奉祠。靈帝崩,獻帝即位。初平中,相國董卓、左中郎將蔡邕等以和帝以下,功德無殊,而有過差,不應為宗,及餘非宗者追尊三后,皆奏毀之。四時所祭,高廟一祖二宗,及近帝四,凡七帝。

古不墓祭,漢諸陵皆有園寢,承秦所為也。說者以為古宗廟前制廟,後制寢,以象人之居前有朝,後有寢也。月令有「先薦寢廟」,詩稱「寢廟弈弈」,言相通也。廟以藏主,以四時祭。寢有衣冠几杖象生之具,以薦新物。秦始出寢,起於墓側,漢因而弗改,故陵上稱寢殿,起居衣服象生人之具,古寢之意也。建武以來,關西諸陵以轉久遠,但四時特牲祠;帝每幸長安謁諸陵,乃太牢祠。自雒陽諸陵至靈帝,皆以晦望二十四氣伏臘及四時祠。廟日上飯,太官送用物,園令、食監典省,其親陵所宮人隨鼓漏理被枕,具盥水,陳嚴具。

建武二年,立太社稷于雒陽,在宗廟之右,方壇,無屋,有牆門而已。二月八月及臘,一歲三祠,皆太牢具,使有司祠。孝經援神契曰:「社者,土地之主也。稷者,五穀之長也。」禮記及國語皆謂共工氏之子曰句龍,為后土官,能平九土,故祀以為社。烈山氏之子曰柱,能植百穀疏,自夏以上祀以為稷,至殷以柱久遠,而堯時棄為后稷,亦植百穀,故廢柱,祀棄為稷。大司農鄭玄說,古者官有大功,則配食其神。故句龍配食於社,棄配食於稷。郡縣置社稷,太守、令、長侍祠,牲用羊豖。唯州所治有社無稷,以其使官。古者師行平有載社主,不載稷也。國家亦有五祀之祭,有司掌之,其禮簡於社稷云。

漢興八年,有言周興而邑立后稷之祀,於是高帝令天下立靈星祠。言祠后稷而謂之靈星者,以后稷又配食星也。舊說,星謂天田星也。一曰,龍左角為天田官,主穀。祀用壬辰位祠之。壬為水,辰為龍,就其類也。牲用太牢,縣邑令長侍祠。舞者用童男十六人。舞者象教田,初為芟除,次耕種、芸耨、驅爵及穫刈、舂簸之形,象其功也。

縣邑常以乙未日祠先農於乙地,以丙戌日祠風伯於戌地,以己丑日祠雨師於丑地,用羊豕。

立春之日,皆青幡幘,迎春于東郭外。令一童男冒青巾,衣青衣,先在東郭外野中。迎春至者,自野中出,則迎者拜之而還,弗祭。三時不迎。

論曰:臧文仲祀爰居,而孔子以為不知。漢書郊祀志著自秦以來迄于王莽,典祀或有未修,而爰居之類眾焉。世祖中興,蠲除非常,修復舊祀,方之前事邈殊矣。嘗聞儒言,三皇無文,結繩以治,自五帝始有書契。至於三王,俗化彫文,詐偽漸興,始有印璽以檢姦萌,然猶未有金玉銀銅之器也。自上皇以來,封泰山者,至周七十二代。封者,謂封土為壇,柴祭告天,代興成功也。禮記所謂「因名山升中于天」者也。易姓則改封者,著一代之始,明不相襲也。繼世之王巡狩,則修封以祭而已。自秦始皇、孝武帝封泰山,本由好僊信方士之言,造為石檢印封之事也。所聞如此。雖誠天道難可度知,然其大較猶有本要。天道質誠,約而不費者也。故牲有牘,器用陶匏,殆將無事於檢封之閒,而樂難攻之石也。且唯封為改代,故曰岱宗。夏康、周宣,由廢復興,不聞改封。世祖欲因孝武故封,實繼祖宗之道也。而梁松固爭,以為必改。乃當夫既封之後,未有福,而松卒被誅死。雖罪由身,蓋亦誣神之咎也。且帝王所以能大顯于後者,實在其德加於民,不聞其在封矣。言天地者莫大於易,易無六宗在中之象。若信為天地四方所宗,是至大也。而比太社,又為失所,難以為誠矣!

贊曰:天地禋郊,宗廟享祀,咸秩無文,山川具止。淫乃國紊,典惟皇紀。肇自盛敬,孰崖厥始?


\end{pinyinscope}