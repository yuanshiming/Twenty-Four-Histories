\article{百官五}

\begin{pinyinscope}
烏桓校尉護羌校尉王國宋

衛國

列侯關內侯四夷國百官奉

外十二州,每州刺史一人,六百石。本注曰:秦有監御史,監諸郡,漢興省之,但遣丞相史分刺諸州,無常官。孝武帝初置刺史十三人,秩六百石。成帝更為牧,秩二千石。建武十八年,復為刺史,十二人各主一州,其一州屬司隸校尉。諸州常以八月巡行所部郡國,錄囚徒,考殿最。初歲盡詣京都奏事,中興但因計吏。

皆有從事史、假佐。本注曰:員職略與司隸同,無都官從事,其功曹從事為治中從事。

豫州部郡國六,冀州部九,兗州部八,徐州部五,青州部六,荊州部七,揚州部六,益州部十二,涼州部十二,并州部九,幽州部十一,交州部七,凡九十八。其二十七王國相,其七十一郡太守。其屬國都尉。屬國,分郡離遠縣置之,如郡差小,置本郡名。世祖并省郡縣四百餘所,後世稍復增之。

凡州所監都為京都,置尹一人,二千石,丞一人。每郡置太守一人,二千石,丞一人。郡當邊戍者,丞為長史。王國之相亦如之。每屬國置都尉一人,比二千石,丞一人。本注曰:凡郡國皆掌治民,進賢勸功,決訟檢姦。常以春行所主縣,勸民農桑,振救乏絕。秋冬遣無害吏案訊諸囚,平其罪法,論課殿最。歲盡遣吏上計。并舉孝廉,郡口二十萬舉一人。典兵禁,備盜賊,景帝更名都尉。武帝又置三輔都尉各一人,譏出入。邊郡置農都尉,主屯田殖穀。又置屬國都尉,主蠻夷降者。中興建武六年,省諸郡都尉,并職太守,無都試之役。省關都尉,唯邊郡往往置都尉及屬國都尉,稍有分縣,治民比郡。安帝以羌犯法,三輔有陵園之守,乃復置右扶風都尉,京兆虎牙都尉。皆置諸曹掾史。本注曰:諸曹略如公府曹,無東西曹。有功曹史,主選署功勞。有五官掾,署功曹及諸曹事。其監屬縣,有五部督郵,曹掾一人。正門有亭長一人。主記室史,主錄記書,催期會。無令史。閤下及諸曹各有書佐,幹主文書。

屬官,每縣、邑、道,大者置令一人,千石;其次置長,四百石;小者置長,三百石;侯國之相,秩次亦如之。本注曰:皆掌治民,顯善勸義,禁姦罰惡,理訟平賊,恤民時務,秋冬集課,上計於所屬郡國。

凡縣主蠻夷曰道。公主所食湯沐曰國。縣萬戶以上為令,不滿為長。侯國為相。皆秦制也。丞各一人。尉大縣二人,小縣一人。本注曰:丞署文書,典知倉獄。尉主盜賊。凡有賊發,主名不立,則推索行尋,案察姦宄,以起端緒。各署諸曹掾史。本注曰:諸曹略如郡員,五官為廷掾,監鄉五部,春夏為勸農掾,秋冬為制度掾。

鄉置有秩、三老、游徼。本注曰:有秩,郡所署,秩百石,掌一鄉人;其鄉小者,縣置嗇夫一人。皆主知民善惡,為役先後,知民貧富,為賦多少,平其差品。三老掌教化。凡有孝子順孫,貞女義婦,讓財救患,及學士為民法式者,皆扁表其門,以興善行。游徼掌徼循,禁司姦盜。又有鄉佐,屬鄉,主民收賦稅。

亭有亭長,以禁盜賊。本注曰:亭長,主求捕盜賊,承望都尉。

里有里魁,民有什伍,善惡以告。本注曰:里魁掌一里百家。什主十家,伍主五家,以相檢察。民有善事惡事,以告監官。

邊縣有障塞尉。本注曰:掌禁備羌夷犯塞。其郡有鹽官、鐵官、工官、都水官者,隨事廣狹置令、長及丞,秩次皆如縣、道,無分士,給均本吏。本注曰:凡郡縣出鹽多者置鹽官,主鹽稅。出鐵多者置鐵官,主鼓鑄。有工多者置工官,主工稅物。有水池及魚利多者置水官,主平水收漁稅。在所諸縣均差吏更給之,置吏隨事,不具縣員。

使匈奴中郎將一人,比二千石。本注曰:主護南單于。置從事二人,有事隨事增之,掾隨事為員。護羌、烏桓校尉所置亦然。

護烏桓校尉一人,比二千石。本注曰:主烏桓胡。

護羌校尉一人,比二千石。本注曰:主西羌。

皇子封王,其郡為國,每置傅一人,相一人,皆二千石。本注曰:傅主導王以善,禮如師,不臣也。相如太守。有長史,如郡丞。

漢初立諸王,因項羽所立諸王之制,地既廣大,且至千里。又其官職傅為太傅,相為丞相,又有御史大夫及諸卿,皆秩二千石,石官皆如朝廷。國家唯為置丞相,其御史大夫以下皆自置之。至景帝時,吳、楚七國恃其國大,遂以作亂,幾危漢室。及其誅滅,景帝懲之,遂令諸王不得治民,令內史主治民,改丞相曰相,省御史大夫、廷尉、少府、宗正、博士官。武帝改漢內史、中尉、郎中令之名,而王國如故,員職皆朝廷為署,不得自置。至漢成帝省內史治民,更令相治民,太傅但曰傅。

中尉一人,比二千石。本注曰:職如郡都尉,主盜賊。郎中令一人,僕一人,皆千石。本注曰:郎中令掌王大夫、郎中宿衛,官如光祿勳。自省少府,職皆并焉。僕主車及馭,如太僕。本注曰太僕,比二千石,武帝改,但曰僕,又皆減其秩。治書,比六百石。本注曰:治書本尚書更名。大夫,比六百石。本注曰:無員。掌奉王使至京都,奉璧賀正月,及使諸國。本皆持節,後去節。謁者,比四百石。本注曰:掌冠長冠。本員十六人,後減。禮樂長。本注曰:主樂人。衛士長。本注曰:主衛士。醫工長。本注曰:主醫藥。永巷長。本注曰:宦者,主宮中婢使。祠祀長。本注曰:主祠祀。皆比四百石。郎中,二百石。本注曰:無員。

衛公、宋公。本注曰:建武二年,封周後姬常為周承休公;五年,封殷後孔安為殷紹嘉公。十三年,改常為衛公,安為宋公,以為漢賓,在三公上。

列侯,所食縣為侯國。本注曰:承秦爵二十等,為徹侯,金印紫綬,以賞有功。功大者食縣,小者食鄉、亭,得臣其所食吏民。後避武帝諱,為列侯。武帝元朔二年,令諸王得推恩分眾子土,國家為封,亦為列侯。舊列侯奉朝請在長安者,位次三公。中興以來,唯以功德賜位特進者,次車騎將軍;賜位朝侯,次五校尉;賜位侍祠侯,次大夫。其餘以胏附及公主子孫奉墳墓於京都者,亦隨時見會,位在博士、議郎下。

諸王封者受茅土,歸以立社稷,禮也。列土、特進、朝侯賀正月執璧云。

每國置相一人,其秩各如本縣。本注曰:主治民,如令、長,不臣也。但納租于侯,以戶數為限。其家臣,置家丞、庶子各一人。本注曰:主侍侯,使理家事。列侯舊有行人、洗馬、門大夫,凡五官。中興以來,食邑千戶已上置家丞、庶子各一人,不滿千戶不置家丞,又悉省行人、洗馬、門大夫。

關內侯,承秦賜爵十九等,為關內侯,無土,寄食在所縣,民租多少,各有戶數為限。

四夷國王,率眾王,歸義侯,邑君,邑長,皆有丞,比郡、縣。

百官受奉例:大將軍、三公奉,月三百五十斛。中二千石奉,月百八十斛。二千石奉,月百二十斛。比二千石奉,月百斛。千石奉,月八十斛。六百石奉,月七十斛。比六百石奉,月五十斛。四百石奉,月四十五斛。比四百石奉,月四十斛。三百石奉,月四十斛。比三百石奉,月三十七斛。二百石奉,月三十斛。比二百石奉,月二十七斛。一百石奉,月十六斛。斗食奉,月十一斛。佐史奉,月八斛。凡諸受奉,皆半錢半穀。

贊曰:帝道淵默,冢帥修德。寡以御眾,分職乃克。不置不監,無驕無忒。程是師徒,寧民康國。


\end{pinyinscope}