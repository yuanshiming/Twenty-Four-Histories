\article{班彪列傳下}

\begin{pinyinscope}
主人喟然而歎曰:「痛乎風俗之移人也!子實

秦人,矜夸館室,保界河山,信識昭襄而知始皇矣,惡睹大漢之云為乎?夫大漢之開原也,奮布衣以登皇極,繇數期而創萬世,蓋六籍所不能談,前聖靡得而言焉。當此之時,功有橫而當天,討有逆而順人,故婁敬度埶而獻其說,蕭公權宜以拓其制。時豈泰而安之哉?計不得以已也。吾子曾不是睹,顧燿後嗣之末造,不亦闇乎?今將語子以建武之理,永平之事,監乎泰清,以變子之或志。

往者王莽作逆,漢祚中缺,天人致誅,六合相滅。于時之亂,生民幾亡,鬼神泯絕,壑無完柩,郛罔遺室,原野猒人之肉,川谷流人之血,秦、項之災猶不克半,書契已來未之或紀也。故下民號而上愬,上帝懷而降鑒,致命于聖皇。於是聖皇乃握乾符,闡坤珍,披皇圖,稽帝文,赫爾發憤,應若興雲,霆發昆陽,憑怒雷震。遂超大河,跨北嶽,立號高邑,建都河洛。紹百王之荒屯,因造化之盪滌,體元立制,繼天而作。系唐統,接漢緒,茂育群生,恢復疆宇,動兼乎在昔,事勤乎三五。豈特方軌並跡,紛綸后辟,理近古之所務,蹈一聖之險易云爾哉?且夫建武之元,天地革命,四海之內,更造夫婦,肇有父子,君臣初建,人倫寔始,斯乃虙羲氏之所以基皇德也。分州土,立市朝,作舟車,造器械,斯軒轅氏之所以開帝功也。龔行天罰,應天順民,斯乃湯武之所以昭王業也。遷都改邑,有殷宗中興之則焉;即土之中,有周成隆平之制焉。不階尺土一人之柄,同符乎高祖。克己復禮,以奉終始,允恭乎孝文。憲章稽古,封岱勒成,儀炳乎世宗。案六經而校德,妙古昔而論功,仁聖之事既該,帝王之道備矣。

至于永平之際,重熙而累洽,盛三雍之上儀,修袞龍之法服,敷洪藻,信景鑠,揚世廟,正予樂。人神之和允洽,君臣之序既肅。乃動大路,遵皇衢,省方巡狩,窮覽萬國之有無,考聲教之所被,散皇明以燭幽。然後增周舊,修洛邑,翩翩巍巍,顯顯翼翼,光漢京于諸夏,總八方而為之極。是以皇城之內,宮室光明,闕庭神麗,奢不可踰,儉不能侈。外則因原野以作苑,順流泉而為沼,發蘋藻以潛魚,豐圃草以毓獸,制同乎梁騶,義合乎靈囿。若乃順時節而蒐狩,簡車徒以講武,則必臨之以王制,考之以風雅。歷騶虞,覽四驖,嘉車攻,采吉日,禮官正儀,乘輿乃出。於是發鯨魚,鏗華鍾,登玉輅,乘時龍,鳳蓋颯灑,和鸞玲瓏,天官景從,祲威盛容。山靈護野,屬御方神,雨師汎灑,風伯清塵,千乘雷起,萬騎紛紜,元戎竟野,戈鋋彗雲,羽旄掃霓,旌旗拂天。焱焱炎炎,揚光飛文,吐爓生風,吹野燎山,日月為之奪明,丘陵為之搖震。遂集乎中囿,陳師案屯,駢部曲,列校隊,勒三軍,誓將帥。然後舉烽伐鼓,以命三驅,輕車霆發,驍騎電騖,游基發射,范氏施御,弦不失禽,轡不詭遇,飛者未及翔,走者未及去。指顧倏忽,獲車已實,樂不極般,殺不盡物,馬踠餘足,士怒未泄,先驅復路,屬車案節。於是薦三犧,效五牲,禮神祇,懷百靈,御明堂,臨辟雍,揚緝熙,宣皇風,登靈臺,考休徵。俯仰乎乾坤,參象乎聖躬,目中夏而布德,瞰四裔而抗棱。西盪河源,東澹海漘,北動幽崖,南趯朱垠。殊方別區,界絕而不鄰,自孝武所不能征,孝宣所不能臣,莫不陸讋水慄,奔走而來賓。遂綏哀牢,開永昌,春王三朝,會同漢京。是日也,天子受四海之圖籍,膺萬國之貢珍,內撫諸夏,外接百蠻。乃盛禮樂供帳,置乎雲龍之庭,陳百僚而贊群后,究皇儀而展帝容。於是庭實千品,旨酒萬鍾,列金罍,班玉觴,嘉珍御,大牢饗。爾乃食舉雍徹,太師秦樂,陳金石,布絲竹,鐘鼓鏗鎗,管絃曄煜。抗五聲,極六律,歌九功,舞八佾,韶武備,太古畢。四夷閒奏,德廣所及,仱峤兜離,罔不具集。萬樂備,百禮暨,皇歡浹,群臣醉,降煙熅,調元氣,然後撞鍾告罷,百僚遂退。

於是聖上親萬方之歡娛,久沐浴乎膏澤,懼其侈心之將萌,而怠於東作也,乃申舊章,下明詔,命有司,班憲度,昭節儉,示大素。去後宮之麗飾,損乘輿之服御,除工商之淫業,興農桑之上務。遂令海內棄末而反本,背偽而歸真,女脩織紝,男務耕耘,器用陶匏,服尚素玄,恥纖靡而不服,賤奇麗而不珍,捐金於山,沈珠於淵。於是百姓滌瑕盪穢而鏡至清,形神寂漠,耳目不營,嗜欲之原滅,廉正之心生,莫不優游而自得,玉潤而金聲。是以四海之內,學校如林,庠序盈門,獻酬交錯,俎豆莘莘,下舞上歌,蹈德詠仁。登降飫宴之禮既畢,因相與嗟歎玄德,讜言弘說,咸含和而吐氣,頌曰「盛哉乎斯世」!

今論者但知誦虞夏之書,詠殷周之詩,講羲文之易,論孔氏之春秋,罕能精古今之清濁,究漢德之所由。唯子頗識舊典,又徒馳騁乎末流。溫故知新已難,而知德者鮮矣!且夫辟界西戎,險阻四塞,脩其防禦,孰與處乎土中,平夷洞達,萬方輻湊?秦領九嵕,涇渭之川,曷若四瀆五岳,帶河泝洛,圖書之淵?建章甘泉,館御列仙,孰與靈臺明堂,統和天人?太液昆明,鳥獸之囿,曷若辟雍海流,道德之富?游俠踰侈,犯義侵禮,孰與同履法度,翼翼濟濟也?子徒習秦阿房之造天,而不知京洛之有制也;識函谷之可關,而不知王者之無外也。」

主人之辭未終,西都賓矍然失容,逡巡降階,惵然意下,捧手欲辭。主人曰:「復位,今將喻子五篇之詩。」賓既卒業,乃稱曰:「美哉乎此詩!義正乎楊雄,事實乎相如,非唯主人之好學,蓋乃遭遇乎斯時也。小子狂簡,不知所裁,既聞正道,請終身誦之。」其詩曰:

明堂詩:於昭明堂,明堂孔陽;聖皇宗祀,穆穆煌煌。上帝宴饗,五位時序;誰其配之,世祖光武。普天率土,各以其職;猗與緝熙,允懷多福。

辟雍詩:迺流辟雍,辟雍湯湯;聖皇蒞止,造舟為梁。皤皤國老,迺父迺兄;抑抑威儀,孝友光明。於赫太上,示我漢行;鴻化惟神,永觀厥成。

靈臺詩:迺經靈臺,靈臺既崇;帝勤時登,爰考休徵。三光宣精,五行布序;習習祥風,祁祁甘雨。百穀溱溱,庶卉蕃蕪;屢惟豐年,於皇樂胥。

寶鼎詩:嶽脩貢兮川效珍,吐金景兮歊浮雲。寶鼎見兮色紛縕,煥其炳兮被龍文。登祖廟兮享聖神,昭靈德兮彌億年。

白雉詩:啟靈篇兮披瑞圖,獲白雉兮效素烏。發皓羽兮奮翹英,容絜朗兮於淳精。章皇德兮侔周成,永延長兮膺天慶。

及肅宗雅好文章,固愈得幸,數入讀書禁中,或連日繼夜。每行巡狩,輒獻上賦頌,朝廷有大議,使難問公卿,辯論於前,賞賜恩寵甚渥。固自以二世才術,位不過郎,感東方朔、楊雄自論,以不遭蘇、張、范、蔡之時,作賓戲以自通焉。後遷玄武司馬。天子會諸儒講論五經,作白虎通德論,令固撰集其事。

時北單于遣使貢獻,求欲和親,詔問群僚。議者或以為「匈奴變詐之國,無內向之心,徒以畏漢威靈,逼憚南虜,故希望報命,以安其離叛。今若遣使,恐失南虜親附之歡,而成北狄猜詐之計,不可」。固議曰:「竊自惟思,漢興已來,曠世歷年,兵纏夷狄,尤事匈奴。綏御之方,其塗不一,或脩文以和之,或用武以征之,或卑下以就之,或臣服而致之。雖屈申無常,所因時異,然未有拒絕棄放,不與交接者也。故自建武之世,復脩舊典,數出重使,前後相繼,至於其末,始乃暫絕。永平八年,復議通之。而廷爭連日,異同紛回,多執其難,少言其易。先帝聖德遠覽,瞻前顧後,遂復出使,事同前世。以此而推,未有一世闕而不修者也。今烏桓就闕,稽首譯官,康居、月氏,自遠而至,匈奴離析,名王來降,三方歸服,不以兵威,此誠國家通於神明自然之徵也。臣愚以為宜依故事,復遣使者,上可繼五鳳、甘露至遠人之會,下不失建武、永平羈縻之義。虜使再來,然後一往,既明中國主在忠信,且知聖朝禮義有常,豈同逆詐示猜,孤其善意乎?絕之未知其利,通之不聞其害。設後北虜稍彊,能為風塵,方復求為交通,將何所及?不若因今施惠,為策近長。」

固又作典引篇,述敘漢德。以為相如封禪,靡而不典,楊雄美新,典而不實,蓋自謂得其致焉。其辭曰:

太極之原,兩儀始分,煙撰熅熅,有沈而奧,有浮而清。沈浮交錯,庶類混成。肇命人主,五德初始,同于草昧,玄混之中。踰繩越契,寂寥而亡詔者,系不得而綴也。厥有氏號,紹天闡繹者,莫不開元於大昊皇初之首,上哉夐乎,其書猶可得而脩也。亞斯之世,通變神化,函光而未曜。

若夫上稽乾則,降承龍翼,而炳諸典謨,以冠德卓蹤者,莫崇乎陶唐。陶唐舍胤而禪有虞,虞亦命夏后,稷契熙載,越成湯武。股肱既周,天乃歸功元首,將授漢劉。俾其承三季之荒末,值亢龍之災孽,懸象暗而恆文乖,彝倫斁而舊章缺。故先命玄聖,使綴學立制,宏亮洪業,表相祖宗,贊揚迪哲,備哉燦爛,真神明之式也。雖前皋、夔、衡、旦密勿之輔,比茲褊矣。是以高、光二聖,辰居其域,時至氣動,乃龍見淵躍。拊翼而未舉,則威靈紛紜,海內雲蒸,雷動電熛,胡縊莽分,不蒞其誅。然後欽若上下,恭揖群后,正位度宗,有于德不台淵穆之讓,靡號師矢敦奮撝之容。蓋以膺當天之正統,受克讓之歸運,蓄炎上之烈精,蘊孔佐之弘陳云爾。

洋洋乎若德,帝者之上儀,誥誓所不及已。鋪觀二代洪纖之度,其賾可探也。並開跡於一匱,同受侯甸之所服,奕世勤民,以伯方統牧。乘其命賜彤弧黃戚之威,用討韋、顧、黎、崇之不格。至乎三五華夏,京遷鎬亳,遂自北面,虎離其師,革滅天邑。是故義士偉而不敦,武稱未盡,護有慚德,不其然與?然猶於穆猗那,翕純皦繹,以崇嚴祖考,殷薦宗祀配帝,發祥流慶,對越天地者,舄奕乎千載。豈不克自神明哉!誕略有常,審言行於篇籍,光藻朗而不渝耳。

矧夫赫赫聖漢,巍巍唐基,泝測其源,乃先孕虞育夏,甄殷陶周,然後宣二祖之重光,襲四宗之緝熙。神靈日燭,光被六幽,仁風翔乎海表,威靈行於鬼區,慝亡迥而不泯,微胡瑣而不頤。故夫顯定三才昭登之績,匪堯不興,鋪聞遺策在下之訓,匪漢不弘。厥道至乎經緯乾坤,出入三光,外運混元,內浸豪芒,性類循理,品物咸亨,其已久矣。

盛哉!皇家帝世,德臣列辟,功君百王,榮鏡宇宙,尊無與抗。乃始虔鞏勞讓,兢兢業業,貶成抑定,不敢論制作。至令遷正黜色賓監之事煥揚宇內,而禮官儒林屯朋篤論之士而不傳祖宗之仿佛,雖云優慎,無乃葸歟!

於是三事嶽牧之僚,僉爾而進曰:陛下仰監唐典,中述祖則,俯蹈宗軌。躬奉天經,惇睦辯章之化洽。巡靖黎蒸,懷保鰥寡之惠浹。燔瘞縣沈,肅祗群神之禮備。是以鳳皇來儀集羽族於觀魏,肉角馴毛宗於外囿,擾緇文皓質於郊,升黃暉采鱗於沼,甘露宵零於豐草,三足軒翥於茂樹。若乃嘉穀靈草,奇獸神禽,應圖合諜,窮祥極瑞者,朝夕坰牧,日月邦畿,卓犖乎方州,羨溢乎要荒。昔姬有素雉、朱烏、玄秬、黃咛之事耳,君臣動色,左右相趨,濟濟翼翼,峨峨如也。蓋用昭明寅畏,承聿懷之福。亦以寵靈文武,貽燕後昆,覆以懿鑠,豈其為身而有顓辭也?若然受之,宜亦勤恁旅力,以充厥道,啟恭館之金縢,御東序之祕寶,以流其占。

夫圖書亮章,天哲也;孔猷先命,聖孚也;體行德本,正性也;逢吉丁辰,景命也。順命以創制,定性以和神,荅三靈之繁祉,展放唐之明文,茲事體大而允,寤寐次于聖心。瞻前顧後,豈蔑清廟憚敕天乎?伊考自邃古,乃降戾爰茲,作者七十有四人,有不俾而假素,罔光度而遺章,今其如台而獨闕也!

是時聖上固已垂精游神,包舉蓺文,屢訪群儒,諭咨故老,與之乎斟酌道德之淵源,肴覈仁義之林藪,以望元符之臻焉。既成群后之讜辭,又悉經五繇之碩慮矣。將絣萬嗣,煬洪暉,奮景炎,扇遺風,播芳烈,久而愈新,用而不竭,汪汪乎丕天之大律,其疇能异之哉?唐哉皇哉,皇哉唐哉!

固後以母喪去官。永元初,大將軍竇憲出征匈奴,以固為中護軍,與參議。北單于聞漢軍出,遣使款居延塞,欲脩呼韓邪故事,朝見天子,請大使。憲上遣固行中郎將事,將數百騎與虜使俱出居延塞迎之。會南匈奴掩破北庭,固至私渠海,聞虜中亂,引還。及竇憲敗,固先坐免官。

固不教學諸子,諸子多不遵法度,吏人苦之。初,洛陽令种兢嘗行,固奴干其車騎,吏椎呼之,奴醉罵,兢大怒,畏憲不敢發,心銜之。及竇氏賓客皆逮考,兢因此捕繫固,遂死獄中。時年六十一。詔以譴責兢,抵主者吏罪。

固所著典引、賓戲、應譏、詩、賦、銘、誄、頌

、書、文、記、論、議、六言,在者凡四十一篇。

論曰:司馬遷、班固父子,其言史官載籍之作,大義粲然著矣。議者咸稱二子有良史之才。遷文直而事覈,固文贍而事詳。若固之序事,不激詭,不抑抗,贍而不穢,詳而有體,使讀之者亹亹而不猒,信哉其能成名也。彪、固譏遷,以為是非頗謬於聖人。然其論議常排死節,否正直,而不敘殺身成仁之為美,則輕仁義,賤守節愈矣。固傷遷博物洽聞,不能以智免極刑;然亦身陷大戮,智及之而不能守之。嗚呼,古人所以致論於目睫也!

贊曰:二班懷文,裁成帝墳。比良遷、董,兼麗卿、雲。彪識皇命,固迷世紛。


\end{pinyinscope}