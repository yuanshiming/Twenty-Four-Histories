\article{列女傳}

\begin{pinyinscope}
詩書之言女德尚矣。若夫賢妃助國君之政,哲婦隆家人之道,高士弘清淳之風,貞女亮明白之節,則其徽美未殊也,而世典咸漏焉。故自中興以後,綜其成事,述為列女篇。如馬、鄧、梁后別見前紀,梁嫕、李姬各附家傳,若斯之類,並不兼書。餘但鹑次才行尤高秀者,不必專在一操而已。

勃海鮑宣妻者,桓氏之女也,字少君。宣嘗就少君父學,父奇其清苦,故以女妻之,裝送資賄甚盛。宣不悅,謂妻曰:「少君生富驕,習美飾,而吾實貧賤,不敢當禮。」妻曰:「大人以先生脩德守約,故使賤妾侍執巾櫛。既奉承君子,唯命是從。」宣笑曰:「能如是,是吾志也。」妻乃悉歸侍御服飾,更著短布裳,與宣共挽鹿車歸鄉里。拜姑禮畢,提甕出汲。脩行婦道,鄉邦稱之。

宣、哀帝時官至司隸校尉。子永,中興初為魯郡太守。永子昱從容問少君曰:「太夫人寧復識挽鹿車時不?」對曰:「先姑有言:『存不忘亡,安不忘危。』吾焉敢忘乎!」永、昱已見前傳。

太原王霸妻者,不知何氏之女也。霸少立高節,光武時,連徵不仕。霸已見逸人傳。妻亦美志行。初,霸與同郡令狐子伯為友,後子伯為楚相,而其子為郡功曹。子伯乃令子奉書於霸,車馬服從,雍容如也。霸子時方耕於野,聞賓至,投耒而歸,見令狐子,沮怍不能仰視。霸目之,有愧容,客去而久臥不起。妻怪問其故,始不肯告,妻請罪,而後言曰:「吾與子伯素不相若,向見其子容服甚光,舉措有適,而我兒曹蓬髮歷齒,未知禮則,見客而有慚色。父子恩深,不覺自失耳。」妻曰:「君少修清節,不顧榮祿。今子伯之貴孰與君之高?柰何忘宿志而慚兒女子乎!」霸屈起而笑曰:「有是哉!」遂共終身隱遯。

廣漢姜詩妻者,同郡龐盛之女也。詩事母至孝,妻奉順尤篤。母好飲江水,水去舍六七里,妻常泝流而汲。後值風,不時得還,母渴,詩責而遣之。妻乃寄止鄰舍,晝夜紡績,市珍羞,使鄰母以意自遺其姑。如是者久之,姑怪問鄰母,鄰母具對。姑感慚呼還,恩養愈謹。其子後因遠汲溺死,妻恐姑哀傷,不敢言,而託以行學不在。姑嗜魚鱠,又不能獨食,夫婦常力作供鱠,呼鄰母共之。舍側忽有涌泉,味如江水,每旦輒出雙鯉魚,常以供二母之膳。赤眉散賊經詩里,弛兵而過,曰:「驚大孝必觸鬼神。」時歲荒,賊乃遺詩米肉,受而埋之,比落蒙其安全。

永平三年,察孝廉,顯宗詔曰:「大孝入朝,凡諸舉者一聽平之。」由是皆拜郎中。詩尋除江陽令,卒于官。所居治,鄉人為立祀。

沛郡周郁妻者,同郡趙孝之女也,字阿。少習儀訓,閑於婦道,而郁驕淫輕躁,多行無禮。郁父偉謂阿曰:「新婦賢者女,當以道匡夫。郁之不改,新婦過也。」阿拜而受命,退謂左右曰:「我無樊衛二姬之行,故君以責我。我言而不用,君必謂我不奉教令,則罪在我矣。若言而見用,是為子違父而從婦,則罪在彼矣。生如此,亦何聊哉!」乃自殺。莫不傷之。

扶風曹世叔妻者,同郡班彪之女也,名昭,字惠班,一名姬。博學高才。世叔早卒,有節行法度。兄固著漢書,其八表及天文志未及竟而卒,和帝詔昭就東觀臧書閣踵而成之。帝數召入宮,令皇后諸貴人師事焉,號曰大家。每有貢獻異物,輒詔大家作賦頌。及鄧太后臨朝,與聞政事。以出入之勤,特封子成關內侯,官至齊相。時漢書始出,多未能通者,同郡馬融伏於閣下,從昭受讀,後又詔融兄續繼昭成之。

永初中,太后兄大將軍鄧騭以母憂,上書乞身,太后不欲許,以問昭。昭因上疏曰:「伏惟皇太后陛下,躬盛德之美,隆唐虞之政,闢四門而開四聰,采狂夫之瞽言,納芻蕘之謀慮。妾昭得以愚朽,身當盛明,敢不披露肝膽,以效萬一。妾聞謙讓之風,德莫大焉,故典墳述美,神祇降福。昔夷齊去國,天下服其廉高;太伯違邠,孔子稱為三讓。所以光昭令德,揚名于後者也。論語曰:『能以禮讓為國,於從政乎何有。』由是言之,推讓之誠,其致遠矣。今四舅深執忠孝,引身自退,而以方垂未靜,拒而不許;如後有毫毛加於今日,誠恐推讓之名不可再得。緣見逮及,故敢昧死竭其愚情。自知言不足采,以示蟲螘之赤心。」太后從而許之。於是騭等各還里第焉。

作女誡七篇,有助內訓。其辭曰:

鄙人愚暗,受性不敏,蒙先君之餘寵,賴母師之典訓。年十有四,執箕帚於曹氏,于今四十餘載矣。戰戰兢兢,常懼黜辱,以增父母之羞,以益中外之累。夙夜劬心,勤不告勞,而今而後,乃知免耳。吾性疏頑,教道無素,恆恐子穀負辱清朝。聖恩橫加,猥賜金紫,實非鄙人庶幾所望也。男能自謀矣,吾不復以為憂也。但傷諸女方當適人,而不漸訓誨,不聞婦禮,懼失容它門,取恥宗族。吾今疾在沈滯,性命無常,念汝曹如此,每用惆悵。閒作女誡七章,願諸女各寫一通,庶有補益,裨助汝身。去矣,其勗勉之!

卑弱第一:古者生女三日,臥之床下,弄之瓦塼,而齋告焉。臥之床下,明其卑弱,主下人也。弄之瓦塼,明其習勞,主執勤也。齋告先君,明當主繼祭祀也。三者蓋女人之常道,禮法之典教矣。謙讓恭敬,先人後己,有善莫名,有惡莫辭,忍辱含垢,常若畏懼,是謂卑弱下人也。晚寢早作,勿憚夙夜,執務私事,不辭劇易,所作必成,手跡整理,是謂執勤也。正色端操,以事夫主,清靜自守,無好戲笑,絜齊酒食,以供祖宗,是謂繼祭祀也。三者苟備,而患名稱之不聞,黜辱之在身,未之見也。三者苟失之,何名稱之可聞,黜辱之可遠哉!

夫婦第二:夫婦之道,參配陰陽,通達神明,信天地之弘義,人倫之大節也。是以禮貴男女之際,詩著關雎之義。由斯言之,不可不重也。夫不賢,則無以御婦;婦不賢,則無以事夫。夫不御婦,則威儀廢缺;婦不事夫,則義理墮闕。方斯二事,其用一也。察今之君子,徒知妻婦之不可不御,威儀之不可不整,故訓其男,檢以書傳,殊不知夫主之不可不事,禮義之不可不存也。但教男而不教女,不亦蔽於彼此之數乎!禮,八歲始教之書,十五而至於學矣。獨不可依此以為則哉!

敬慎第三:陰陽殊性,男女異行。陽以剛為德,陰以柔為用,男以彊為貴,女以弱為美。故鄙諺有云:「生男如狼,猶恐其尪;生女如鼠,猶恐其虎。」然則修身莫若敬,避彊莫若順。故曰敬順之道,婦人之大禮也。夫敬非它,持久之謂也。夫順非它,寬裕之謂也。持久者,知止足也。寬裕者,尚恭下也。夫婦之好,終身不離。房室周旋,遂生媟黷。媟黷既生,語言過矣。語言既過,縱恣必作。縱恣既作,則侮夫之心生矣。此由於不知止足者也。夫事有曲直,言有是非。直者不能不爭,曲者不能不訟。訟爭既施,則有忿怒之事矣。此由於不尚恭下者也。侮夫不節,譴呵從之;忿怒不止,楚撻從之。夫為夫婦者,義以和親,恩以好合,楚撻既行,何義之存?譴呵既宣,何恩之有?恩義俱廢,夫婦離矣。

婦行第四:女有四行,一曰婦德,二曰婦言,三曰婦容,四曰婦功。夫云婦德,不必才明絕異也;婦言,不必辯口利辭也;婦容,不必顏色美麗也;婦功,不必工巧過人也。清閑貞靜,守節整齊,行己有恥,動靜有法,是謂婦德。擇辭而說,不道惡語,時然後言,不厭於人,是謂婦言。盥浣塵穢,服飾鮮絜,沐浴以時,身不垢辱,是謂婦容。專心紡績,不好戲笑,絜齊酒食,以奉賓客,是謂婦功。此四者,女人之大德,而不可乏之者也。然為之甚易,唯在存心耳。古人有言:「仁遠乎哉?我欲仁,而仁斯至矣。」此之謂也。

專心第五:禮,夫有再娶之義,婦無二適之文,故曰夫者天也。天固不可逃,夫固不可離也。行違神祇,天則罰之;禮義有愆,夫則薄之。故女憲曰:「得意一人,是謂永畢;失意一人,是謂永訖。」由斯言之,夫不可不求其心。然所求者,亦非謂佞媚苟親也,固莫若專心正色。禮義居絜,耳無塗聽,目無邪視,出無冶容,入無廢飾,無聚會群輩,無看視門戶,此則謂專心正色矣。若夫動靜輕脫,視聽陜輸,入則亂髮壞形,出則窈窕作態,說所不當道,觀所不當視,此謂不能專心正色矣。

曲從第六:夫得意一人,是謂永畢;失意一人,是謂永訖。欲人定志專心之言也。舅姑之心,豈當可失哉?物有以恩自離者,亦有以義自破者也。夫雖云愛,舅姑云非,此所謂以義自破者也。然則舅姑之心柰何?固莫尚於曲從矣。姑云不爾而是,固宜從令;姑云爾而非,猶宜順命。勿得違戾是非,爭分曲直。此則所謂曲從矣。故女憲曰:「婦如影響,焉不可賞。」

和叔妹第七:婦人之得意於夫主,由舅姑之愛己也;舅姑之愛己,由叔妹之譽己也。由此言之,我臧否譽毀,一由叔妹,叔妹之心,復不可失也。皆莫知叔妹之不可失,而不能和之以求親,其蔽也哉!自非聖人,鮮能無過。故顏子貴於能改,仲尼嘉其不貳,而況婦人者也!雖以賢女之行,聰哲之性,其能備乎!是故室人和則謗掩,外內離則惡揚。此必然之埶也。《易》曰:「二人同心,其利斷金。同心之言,其臭如蘭。」此之謂也。夫嫂妹者,體敵而尊,恩疏而義親。若淑媛謙順之人,則能依義以篤好,崇恩以結援,使徽美顯章,而瑕過隱塞,舅姑矜善,而夫主嘉美,聲譽曜于邑鄰,休光延於父母。若夫憃愚之人,於嫂則託名以自高,於妹則因寵以驕盈。驕盈既施,何和之有!恩義既乖,何譽之臻!是以美隱而過宣,姑忿而夫慍,毀訾布於中外,恥辱集于厥身,進增父母之羞,退益君子之累。斯乃榮辱之本,而顯否之基也。可不慎哉!然則求叔妹之心,固莫尚於謙順矣。謙則德之柄,順則婦之行。凡斯二者,足以和矣。《詩》云:「在彼無惡,在此無射。」其斯之謂也。

馬融善之,令妻女習焉。

昭女妹曹豐生,亦有才惠,為書以難之,辭有可觀。

昭年七十餘卒,皇太后素服舉哀,使者監護喪事。所著賦、頌、銘、誄、問、注、哀辭、書、論、上疏、遺令,凡十六篇。子婦丁氏為撰集之,又作大家讚焉。

河南樂羊子之妻者,不知何氏之女也。羊子嘗行路,得遺金一餅,還以與妻。妻曰:「妾聞志士不飲盜泉之水,廉者不受嗟來之食,況拾遺求利,以污其行乎!」羊子大慚,乃捐金於野,而遠尋師學。一年來歸,妻跪問其故。羊子曰:「久行懷思,無它異也。」妻乃引刀趨機而言曰:「此織生自蠶繭,成於機杼,一絲而累,以至於寸,累寸不已,遂成丈匹。今若斷斯織也,則捐失成功,稽廢時月。夫子積學,當日知其所亡,以就懿德。若中道而歸,何異斷斯織乎?」羊子感其言,復還終業,遂七年不反。妻常躬勤養姑,又遠饋羊子。

嘗有它舍雞謬入園中,姑盜殺而食之,妻對雞不餐而泣。姑怪問其故。妻曰:「自傷居貧,使食有它肉。」姑竟棄之。

後盜欲有犯妻者,乃先劫其姑。妻聞,操刀而出。盜人曰:「釋汝刀從我者可全,不從我者,則殺汝姑。」妻仰天而歎,舉刀刎頸而死。盜亦不殺其姑。太守聞之,即捕殺賊盜,而賜妻縑帛,以禮葬之,號曰「貞義」。

漢中程文矩妻者,同郡李法之姊也,字穆姜。有二男,而前妻四子。文矩為安眾令,喪於官。四子以母非所生,憎毀日積,而穆姜慈愛溫仁,撫字益隆,衣食資供皆兼倍所生。或謂母曰:「四子不孝甚矣,何不別居以遠之?」對曰:「吾方以義相導,使其自遷善也。」及前妻長子興遇疾困篤,母惻隱自然,親調藥膳,恩情篤密。興疾久乃瘳,於是呼三弟謂曰:「繼母慈仁,出自天受。吾兄弟不識恩養,禽獸其心。雖母道益隆,我曹過惡亦已深矣!」遂將三弟詣南鄭獄,陳母之德,狀己之過,乞就刑辟。縣言之於郡,郡守表異其母,蠲除家徭,遣散四子,許以脩革,自後訓導愈明,並為良士。

穆姜年八十餘卒。臨終敕諸子曰:「吾弟伯度,智達士也。所論薄葬,其義至矣。又臨亡遺令,賢聖法也。令汝曹遵承,勿與俗同,增吾之累。」諸子奉行焉。

孝女曹娥者,會稽上虞人也。父盱,能絃歌,為巫祝。漢安二年五月五日,於縣江泝濤迎婆娑神,溺死,不得屍骸。娥年十四,乃沿江號哭,晝夜不絕聲,旬有七日,遂投江而死。至元嘉元年,縣長度尚改葬娥於江南道傍,為立碑焉。

吳許升妻者,呂氏之女也,字榮。升少為博徒,不理操行,榮嘗躬勤家業,以奉養其姑。數勸升修學,每有不善,輒流涕進規。榮父積忿疾升,乃呼榮欲改嫁之。榮歎曰:「命之所遭,義無離貳!」終不肯歸。升感激自厲,乃尋師遠學,遂以成名。尋被本州辟命,行至壽春,道為盜所害。刺史尹耀捕盜得之。榮迎喪於路,聞而詣州,請甘心讎人。耀聽之。榮乃手斷其頭,以祭升靈。後郡遭寇賊,賊欲犯之,榮踰垣走,賊拔刀追之。賊曰:「從我則生,不從我則死。」榮曰:「義不以身受辱寇虜也!」遂殺之。是日疾風暴雨,雷電晦冥,賊惶懼叩頭謝罪,乃殯葬之。

汝南袁隗妻者,扶風馬融之女也。字倫。隗已見前傳。倫少有才辯。融家世豐豪,裝遣甚盛。及初成禮,隗問之曰:「婦奉箕嶹而已,何乃過珍麗乎?」對曰:「慈親垂愛,不敢逆命。君若欲慕鮑宣、梁鴻之高者,妾亦請從少君、孟光之事矣。」隗又曰:「弟先兄舉,世以為笑。今處姊未適,先行可乎?」對曰:「妾姊高行殊毙,未遭良匹,不似鄙薄,苟然而已。」又問曰:「南郡君學窮道奧,文為辭宗,而所在之職,輒以貨財為損,何邪?」對曰:「孔子大聖,不免武叔之毀;子路至賢,猶有伯寮之愬。家君獲此,固其宜耳。」隗默然不能屈,帳外聽者為慚。隗既寵貴當時,倫亦有名於世。年六十餘卒。

倫妹芝,亦有才義。少喪親長而追感,乃作申情賦云。

酒泉龐淯母者,趙氏之女也,字娥。父為同縣人所殺,而娥兄弟三人,時俱病物故,讎乃喜而自賀,以為莫己報也。娥陰懷感憤,乃潛備刀兵,常帷車以候讎家。十餘年不能得。後遇於都亭,刺殺之。因詣縣自首。曰:「父仇已報,請就刑戮。」福祿長尹嘉義之,解印綬欲與俱亡。娥不肯去。曰:「怨塞身死,妾之明分;結罪理獄,君之常理。何敢苟生,以枉公法!」後遇赦得免。州郡表其閭。太常張奐嘉歎,以束帛禮之。

沛劉長卿妻者,同郡桓鸞之女也。鸞已見前傳。生一男五歲而長卿卒,妻防遠嫌疑,不肯歸寧。兒年十五,晚又夭歿。妻慮不免,乃豫刑其耳以自誓。宗婦相與愍之,共謂曰:「若家殊無它意;假令有之,猶可因姑姊妹以表其誠,何貴義輕身之甚哉!」對曰:「昔我先君五更,學為儒宗,尊為帝師。五更已來,歷代不替,男以忠孝顯,女以貞順稱。《詩》云:『無忝爾祖,聿脩厥德。』是以豫自刑翦,以明我情。」沛相王吉上奏高行,顯其門閭,號曰「行義桓釐」,縣邑有祀必膰焉。

安定皇甫規妻者,不知何氏女也。規初喪室家,後更娶之。妻善屬文,能草書,時為規荅書記,眾人怪其工。及規卒時,妻年猶盛,而容色美。後董卓為相國,承其名,娉以軿輜百乘,馬二十匹,奴婢錢帛充路。妻乃輕服詣卓門,跪自陳請,辭甚酸愴。卓使傅奴侍者悉拔刀圍之,而謂曰:「孤之威教,欲令四海風靡,何有不行於一婦人乎!」妻知不免,乃立罵卓曰:「君羌胡之種,毒害天下猶未足邪!妾之先人,清德奕世。皇甫氏文武上才,為漢忠臣。君親非其趣使走吏乎?敢欲行非禮於爾君夫人邪!」卓乃引車庭中,以其頭縣軶,鞭撲交下。妻謂持杖者曰:「何不重乎?速盡為惠。」遂死車下。後人圖畫,號曰「禮宗」云。

南陽陰瑜妻者,潁川荀爽之女也,名采,字女荀。聰敏有才蓺。年十七,適陰氏。十九產一女,而瑜卒。采時尚豐少,常慮為家所逼,自防禦甚固。後同郡郭奕喪妻,爽以采許之,因詐稱病篤,召采。既不得已而歸,懷刃自誓。爽令傅婢執奪其刃,扶抱載之,猶憂致憤激,敕衛甚嚴。女既到郭氏,乃偽為歡悅之色,謂左右曰:「我本立志與陰氏同穴,而不免逼迫,遂至於此,素情不遂,柰何?」乃命使建四燈,盛裝飾,請奕入相見,共談,言辭不輟。亦敬憚之,遂不敢逼,至曙而出。采因敕令左右辨浴。既入室而掩戶,權令侍人避之,以粉書扉上曰:「尸還陰。」「陰」字未及成,懼有來者,遂以衣帶自縊。左右翫之不為意,比視,已絕,時人傷焉。

犍為盛道妻者,同郡趙氏之女也,字媛姜。建安五年,益部亂,道聚眾起兵,事敗,夫妻執繫,當死。媛姜夜中告道曰:「法有常刑,必無生望,君可速潛逃,建立門戶,妾自留獄,代君塞咎。」道依違未從。媛姜便解道桎梏,為齎糧貨。子翔時年五歲,使道攜持而走。媛姜代道持夜,應對不失。度道已遠,乃以實告吏,應時見殺。道父子會赦得歸。道感其義,終身不娶焉。

孝女叔先雄者,犍為人也。父泥和,永建初為縣功曹。縣長遣泥和拜檄謁巴郡太守,乘船墯湍水物故,尸喪不歸。雄感念怨痛,號泣晝夜,心不圖存,常有自沈之計。所生男女二人,並數歲,雄乃各作囊,盛珠環以繫兒,數為訣別之辭。家人每防閑之,經百許日後稍懈,雄因乘小船,於父墯處慟哭,遂自投水死。弟賢,其夕夢雄告之:「卻後六日,當共父同出。」至期伺之,果與父相持,浮於江上。郡縣表言,為雄立碑,圖象其形焉。

陳留董祀妻者,同郡蔡邕之女也,名琰,字文姬。博學有才辯,又妙於音律。適河東衛仲道。夫亡無子,歸寧于家。興平中,天下喪亂,文姬為胡騎所獲,沒於南匈奴左賢王,在胡中十二年,生二子。曹操素與邕善,痛其無嗣,乃遣使者以金璧贖之,而重嫁於祀。

祀為屯田都尉,犯法當死,文姬詣曹操請之。時公卿名士及遠方使驛坐者滿堂,操謂賓客曰:「蔡伯喈女在外,今為諸君見之。」及文姬進,蓬首徒行,叩頭請罪,音辭清辯,旨甚酸哀,眾皆為改容。操曰:「誠實相矜,然文狀已去,柰何?」文姬曰:「明公廄馬萬匹,虎士成林,何惜疾足一騎,而不濟垂死之命乎!」操感其言,乃追原祀罪。時且寒,賜以頭巾履襪。操因問曰:「聞夫人家先多墳籍,猶能憶識之不?」文姬曰:「昔亡父賜書四千許卷,流離塗炭,罔有存者。今所誦憶,裁四百餘篇耳。」操曰:「今當使十吏就夫人寫之。」文姬曰:「妾聞男女之別,禮不親授。乞給紙筆,真草唯命。」於是繕書送之,文無遺誤。

後感傷亂離,追懷悲憤,作詩二章。其辭曰:

漢季失權柄,董卓亂天常。志欲圖篡弒,先害諸賢良。逼迫遷舊邦,擁主以自彊。海內興義師,欲共討不祥。卓眾來東下,金甲耀日光。平土人脆弱,來兵皆胡羌。獵野圍城邑,所向悉破亡。斬涞無孑遺,尸骸相牚拒。馬邊縣男頭,馬後載婦女。長驅西入關,迥路險且阻。還顧邈冥冥,肝脾為爛腐。所略有萬計,不得令屯聚。或有骨肉俱,欲言不敢語。失意機微閒,輒言斃降虜。要當以亭刃,我曹不活汝。豈復惜性命,不堪其詈罵。或便加棰杖,毒痛參并下。旦則號泣行,夜則悲吟坐。欲死不能得,欲生無一可。彼蒼者何辜,乃遭此厄禍!邊荒與華異,人俗少義理。處所多霜雪,胡風春夏起。翩翩吹我衣,肅肅入我耳。感時念父母,哀歎無窮已。有客從外來,聞之常歡喜。迎問其消息,輒復非鄉里。邂逅徼時願,骨肉來迎己。己得自解免,當復棄兒子。天屬綴人心,念別無會期。存亡永乖隔,不忍與之辭。兒前抱我頸,問母欲何之。「人言母當去,豈復有還時。阿母常仁惻,今何更不慈?我尚未成人,柰何不顧思!」見此崩五內,恍惚生狂癡。號泣手撫摩,當發復回疑。兼有同時輩,相送告離別。慕我獨得歸,哀叫聲摧裂。馬為立踟躕,車為不轉轍。觀者皆歔欷,行路亦嗚咽。去去割情戀,遄征日遐邁。悠悠三千里,何時復交會?念我出腹子,匈臆為摧敗。既至家人盡,又復無中外。城郭為山林,庭宇生荊艾。白骨不知誰,從橫莫覆蓋。出門無人聲,豺狼號且吠。煢煢對孤景,怛吒糜肝肺。登高遠眺望,魂神忽飛逝。奄若壽命盡,旁人相寬大。為復彊視息,雖生何聊賴!託命於新人,竭心自勗厲。流離成鄙賤,常恐復捐廢。人生幾何時,懷憂終年歲!

其二章曰:

嗟薄祐兮遭世患,宗族殄兮門戶單。身執略兮入西關,歷險阻兮之羌蠻。山谷眇兮路曼曼,眷東顧兮但悲歎。冥當寢兮不能安,飢當食兮不能餐,常流涕兮眥不乾,薄志節兮念死難,雖苟活兮無形顏。惟彼方兮遠陽精,陰氣凝兮雪夏零。沙漠壅兮塵冥冥,有草木兮春不榮。人似禽兮食臭腥,言兜離兮狀窈停。歲聿暮兮時邁征,夜悠長兮禁門赖。不能寐兮起屏營,登胡殿兮臨廣庭。玄雲合兮翳月星,北風厲兮肅泠泠。胡笳動兮邊馬鳴,孤雁歸兮聲嚶嚶。樂人興兮彈琴箏,音相和兮悲且清。心吐思兮匈憤盈,欲舒氣兮恐彼驚,含哀咽兮涕沾頸。家既迎兮當歸寧,臨長路兮捐所生。兒呼母兮號失聲,我掩耳兮不忍聽。追持我兮走煢煢,頓復起兮毀顏形。還顧之兮破人情,心怛絕兮死復生。

贊曰:端操有蹤,幽閑有容。區明風烈,昭我管彤。


\end{pinyinscope}