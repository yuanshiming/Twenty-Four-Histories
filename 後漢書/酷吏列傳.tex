\article{酷吏列傳}

\begin{pinyinscope}
漢承戰國餘烈,多豪猾之民。其并兼者則陵橫邦邑,桀健者則雄張閭里。且宰守曠遠,戶口殷大。故臨民之職,專事威斷,族滅姦軌,先行後聞。肆情剛烈,成其不橈之威。違眾用己,表其難測之智。至於重文橫入,為窮怒之所遷及者,亦何可勝言。故乃積骸滿阱,漂血十里。致溫舒有虎冠之吏,延年受屠伯之名,豈虛也哉!若其揣挫彊埶,摧勒公卿,碎裂頭腦而不顧,亦為壯也。

自中興以後,科網稍密,吏人之嚴害者,方於前世省矣。而閹人親婭,侵虐天下。至使陽球磔王甫之屍,張儉剖曹節之墓。若此之類,雖厭快眾憤,亦云酷矣!儉知名,故附黨人篇。

董宣字少平,陳留圉人也。初為司徒侯霸所辟,舉高第,累遷北海相。到官,以大姓公孫丹為五官掾。丹新造居宅,而卜工以為當有死者,丹乃令其子殺道行人,置屍舍內,以塞其咎。宣知,即收丹父子殺之。丹宗族親黨三十餘人,操兵詣府,稱冤叫號。宣以丹前附王莽,慮交通海賊,乃悉收繫劇獄,使門下書佐水丘岑盡殺之。青州以其多濫,奏宣考岑,宣坐徵詣廷尉。在獄,晨夜諷誦,無憂色。及當出刑,官屬具饌送之,宣乃厲色曰:「董宣生平未曾食人之食,況死乎!」升車而去。時同刑九人,次應及宣,光武馳使騶騎特原宣刑,且令還獄。遣使者詰宣多殺無辜,宣具以狀對,言水丘岑受臣旨意,罪不由之,願殺臣活岑。使者以聞,有詔左轉宣懷令,令青州勿案岑罪。岑官至司隸校尉。

後江夏有劇賊夏喜等寇亂郡境,以宣為江夏太守。到界,移書曰:「朝廷以太守能禽姦賊,故辱斯任。今勒兵界首,檄到,幸思自安之宜。」喜等聞,懼,即時降散。外戚陰氏為郡都尉,宣輕慢之,坐免。

後特徵為洛陽令。時湖陽公主蒼頭白日殺人,因匿主家,吏不能得。及主出行,而以奴驂乘,宣於夏門亭候之,乃駐車叩馬,以刀畫地,大言數主之失,叱奴下車,因格殺之。主即還宮訴帝,帝大怒,召宣,欲箠殺之。宣叩頭曰:「願乞一言而死。」帝曰:「欲何言?」宣曰:「陛下聖德中興,而縱奴殺良人,將何以理天下乎?臣不須箠,請得自殺。」即以頭擊楹,流血被面。帝令小黃門持之,使宣叩頭謝主,宣不從,彊使頓之,宣兩手據地,終不肯俯。主曰:「文叔為白衣時,臧亡匿死,吏不敢至門。今為天子,威不能行一令乎?」帝笑曰:「天子不與白衣同。」因敕彊項令出。賜錢三十萬,宣悉以班諸吏。由是搏擊豪彊,莫不震慄。京師號為「臥虎」。歌之曰:「枹鼓不鳴董少平。」

在縣五年。年七十四,卒於官。詔遣使者臨視,唯見布被覆屍,妻子對哭,有大麥數斛、敝車一乘。帝傷之,曰:「董宣廉絜,死乃知之!」以宣嘗為二千石,賜艾綬,葬以大夫禮。拜子並為郎中,後官至齊相。

樊曄字仲華,南陽新野人也。與光武少游舊。建武初,徵為侍御史,遷河東都尉,引見雲臺。初,光武微時,嘗以事拘於新野,曄為市吏,餽餌一笥,帝德之不忘,仍賜曄御食,及乘輿服物。因戲之曰:「一笥餌得都尉,何如?」曄頓首辭謝。及至郡,誅討大姓馬適匡等。盜賊清,吏人畏之。數年,遷楊州牧,教民耕田種樹理家之術。視事十餘年,坐法左轉軹長。

隗囂滅後,隴右不安,乃拜曄為天水太守。政嚴猛,好申韓法,善惡立斷。人有犯其禁者,率不生出獄,吏人及羌胡畏之。道不拾遺。行旅至夜,聚衣裝道傍,曰「以付樊公」。涼州為之歌曰:「游子常苦貧,力子天所富。寧見乳虎穴,不入冀府寺。大笑期必死,忿怒或見置。嗟我樊府君,安可再遭值!」視事十四年,卒官。

永平中,顯宗追思曄在天水時政能,以為後人莫之及,詔賜家錢百萬。子融,有俊才,好黃老,不肯為吏。

李章字第公,河內懷人也。五世二千石。章習嚴氏春秋,經明教授,歷州郡吏。光武為大司馬,平定河北,召章置東曹屬,數從征伐。

光武即位,拜陽平令。時趙、魏豪右往往屯聚,清河大姓趙綱遂於縣界起塢壁,繕甲兵,為在所害。章到,乃設饗會,而延謁綱。綱帶文劍,被羽衣,從士百餘人來到。章與對讌飲,有頃,手劍斬綱,伏兵亦悉殺其從者,因馳詣塢壁,掩擊破之,吏人遂安。

遷千乘太守,坐誅斬盜賊過濫,徵下獄免。歲中拜侍御史,出為琅邪太守。時北海安丘大姓夏長思等反,遂囚太守處興,而據營陵城。章聞,即發兵千人,馳往擊之。掾吏止章曰:「二千石行不得出界,兵不得擅發。」章按劍怒曰:「逆虜無狀,囚劫郡守,此何可忍!若坐討賊而死,吾不恨也。」遂引兵安丘城下,募勇敢燒城門,與長思戰,斬之,獲三百餘級,得牛馬五百餘頭而還。興歸郡,以狀上帝,悉以所得班勞吏士。後坐度人田不實徵,以章有功,但司寇論。月餘免刑歸。復徵,會病卒。

周锱字文通,下邳徐人也。為人刻削少恩,好韓非之術。少為廷尉史。

永平中,補南行唐長。到官,曉吏人曰:「朝廷不以長不肖,使牧黎民,而性讎猾吏,志除豪賊,且勿相試!」遂殺縣中尤無狀者數十人,吏人大震。遷博平令。收考姦臧,無出獄者。以威名遷齊相,亦頗嚴酷,專任刑法,而善為辭案條教,為州內所則。後坐殺無辜,復左轉博平令。

建初中,為勃海太守。每赦令到郡,輒隱閉不出,先遣使屬縣盡決刑罪,乃出詔書。坐徵詣廷尉,免歸。

锱廉絜無資,常築墼以自給。肅宗聞而憐之,復以為郎,再遷召陵侯相。廷掾憚锱嚴明,欲損其威,乃晨取死人斷手足,立寺門。锱聞,便往至死人邊,若與死人共語狀。陰察視口眼有稻芒,乃密問守門人曰:「悉誰載卧入城者?」門者對:「唯有廷掾耳。」又問鈴下:「外頗有疑令與死人語者不?」對曰:「廷掾疑君。」乃收廷掾考問,具服「不殺人,取道邊死人」。後人莫敢欺者。

徵拜洛陽令,下車,先問大姓主名,吏數閭里豪彊以對。锱厲聲怒曰:「本問貴戚若馬、竇等輩,豈能知此賣菜傭乎?」於是部吏望風旨,爭以激切為事。貴戚跼蹐,京師肅清。皇后弟黃門郎竇篤從宮中歸,夜至止姦亭,亭長霍延遮止篤,篤蒼頭與爭,延遂拔劍擬篤,而肆詈恣口。篤以表聞。詔召司隸校尉、河南尹詣尚書譴問,遣劍戟士收锱送廷尉詔獄。數日貰出。帝知锱奉法疾姦,不事貴戚,然苛慘失中,數為有司所奏,八年,遂免官。

後為御史中丞。和帝即位,太傅鄧彪奏锱在任過酷,不宜典司京輦。免歸田里。後竇氏貴盛,篤兄弟秉權,睚眥宿怨,無不僵仆。锱自謂無全,乃柴門自守,以待其禍。然篤等以锱公正,而怨隙有素,遂不敢害。

永元五年,復徵為御史中丞。諸竇雖誅,而夏陽侯瑰猶尚在朝。锱疾之,乃上疏曰:「臣聞臧文仲之事君也,見有禮於君者,事之如孝子之養父母;見無禮於君者,誅之如鷹鸇之逐鳥雀。案夏陽侯瑰,本出輕薄,志在邪僻,學無經術,而妄搆講舍,外招儒徒,實會姦桀。輕忽天威,侮慢王室,又造作巡狩封禪之書,惑眾不道,當伏誅戮,而主者營私,不為國計。夫涓流雖寡,浸成江河;爝火雖微,卒能燎野。履霜有漸,可不懲革?宜尋呂產專竊之亂,永惟王莽篡逆之禍,上安社稷之計,下解萬夫之惑。」會瑰歸國,锱遷司隸校尉。

六年夏旱,車駕自幸洛陽錄囚徒,二人被掠生蟲,坐左轉騎都尉。七年,遷將作大匠。九年,卒於官。

黃昌字聖真,會稽餘姚人也。本出孤微。居近學官,數見諸生修庠序之禮,因好之,遂就經學。又曉習文法,仕郡為決曹。刺史行部,見昌,甚奇之,辟從事。

後拜宛令,政尚嚴猛,好發姦伏。人有盜其車蓋者,昌初無所言,後乃密遣親客至門下賊曹家掩取得之,悉收其家,一時殺戮。大姓戰懼,皆稱神明。

朝廷舉能,遷蜀郡太守。先太守李根年老多悖政,百姓侵冤。及昌到,吏人訟者七百餘人,悉為斷理,莫不得所。密捕盜帥一人,脅使條諸縣彊暴之人姓名居處,乃分遣掩討,無有遺脫。宿惡大姦,皆奔走它境。

初,昌為州書佐,其婦歸寧於家,遇賊被獲,遂流轉入蜀為人妻。其子犯事,乃詣昌自訟。昌疑母不類蜀人,因問所由。對曰:「妾本會稽餘姚戴次公女,州書佐黃昌妻也。妾嘗歸家,為賊所略,遂至於此。」昌驚,呼前謂曰:「何以識黃昌邪?」對曰:「昌左足心有黑子,常自言當為二千石。」昌乃出足示之。因相持悲泣,還為夫婦。

視事四年,徵,再遷陳相。縣人彭氏舊豪縱,造起大舍,高樓臨道。昌每出行縣,彭氏婦人輒升樓而觀。昌不喜,遂敕收付獄,案殺之。

又遷為河內太守,又再遷潁川太守。永和五年,徵拜將作大匠。漢安元年,進補大司農,左轉太中大夫,卒於官。

陽球字方正,漁陽泉州人也。家世大姓冠蓋。球能擊劍,習弓馬。性嚴厲,好申韓之學。郡吏有辱其母者,球結少年數十人,殺吏,滅其家,由是知名。初舉孝廉,補尚書侍郎,閑達故事,其章奏處議,常為臺閣所崇信。出為高唐令,以嚴苛過理,郡守收舉,會赦見原。

辟司徒劉寵府,舉高第。九江山賊起,連月不解。三府上球有理姦才,拜九江太守。球到,設方略,凶賊殄破,收郡中姦吏盡殺之。

遷平原相。出教曰:「相前蒞高唐,志埽姦鄙,遂為貴郡所見枉舉。昔桓公釋管仲射鉤之讎,高祖赦季布逃亡之罪。雖以不德,敢忘前義。況君臣分定,而可懷宿昔哉!今一蠲往愆,期諸來效。若受教之後而不改姦狀者,不得復有所容矣。」郡中咸畏服焉。時天下大旱,司空張顥條奏長吏苛酷貪污者,皆罷免之。球坐嚴苦,徵詣廷尉,當免官。靈帝以球九江時有功,拜議郎。

遷將作大匠,坐事論。頃之,拜尚書令。奏罷鴻都文學,曰:「伏承有詔敕中尚方為鴻都文學樂松、江覽等三十二人圖象立贊,以勸學者。臣聞傳曰:『君舉必書。書而不法,後嗣何觀!』案松、覽等皆出於微蔑,斗筲小人,依憑世戚,附託權豪,俛眉承睫,徼進明時。或獻賦一篇,或鳥篆盈簡,而位升郎中,形圖丹青。亦有筆不點牘,辭不辯心,假手請字,妖偽百品,莫不被蒙殊恩,蟬蛻滓濁。是以有識掩口,天下嗟歎。臣聞圖象之設,以昭勸戒,欲令人君動鑒得失。未聞豎子小人,詐作文頌,而可妄竊天官,垂象圖素者也。今太學、東觀足以宣明聖化。願罷鴻都之選,以消天下之謗。」書奏不省。

時中常侍王甫、曹節等姦虐弄權,扇動外內,球嘗拊髀發憤曰:「若陽球作司隸,此曹子安得容乎?」光和二年,遷為司隸校尉。王甫休沐里舍,球詣闕謝恩,奏收甫及中常侍淳于登、袁赦、封锟、中黃門劉毅、小黃門龐訓、朱禹、齊盛等,及子弟為守令者,姦猾縱恣,罪合滅族。太尉段熲諂附佞倖,宜並誅戮。於是悉收甫、熲等送洛陽獄,及甫子永樂少府萌、沛相吉。球自臨考甫等,五毒備極。萌謂球曰:「父子既當伏誅,少以楚毒假借老父。」球曰:「若罪惡無狀,死不滅責,乃欲求假借邪?」萌乃罵曰:「爾前奉事吾父子如奴,奴敢反汝主乎!今日困吾,行自及也!」球使以土窒萌口,箠朴交至,父子悉死杖下。熲亦自殺。乃僵磔甫屍於夏城門,大署牓曰「賊臣王甫」。盡沒入財產,妻子皆徙比景。

球既誅甫,復欲以次表曹節等,乃敕中都官從事曰:「且先去大猾,當次案豪右。」權門聞之,莫不屏氣。諸奢飾之物,皆各緘縢,不敢陳設。京師畏震。

時順帝虞貴人葬,百官會喪還,曹節見磔甫屍道次,慨然抆淚曰:「我曹自可相食,何宜使犬舐其汁乎?」語諸常侍,今且俱入,勿過里舍也。節直入省,白帝曰:「陽球故酷暴吏,前三府奏當免官,以九江微功,復見擢用。愆過之人,好為妄作,不宜使在司隸,以騁毒虐。」帝乃徙球為衛尉。時球出謁陵,節敕尚書令召拜,不得稽留尺一。球被召急,因求見帝,叩頭曰:「臣無清高之行,橫蒙鷹犬之任。前雖糾誅王甫、段熲,蓋簡落狐狸,未足宣示天下。願假臣一月,必令豺狼鴟梟,各服其辜。」叩頭流血。殿上呵叱曰:「衛尉扞詔邪!」至於再三,乃受拜。

其冬,司徒劉郃與球議收案張讓、曹節,節等知之,共誣白郃等。語已見陳球傳。遂收球送洛陽獄,誅死,妻子徙邊。

王吉者,陳留浚儀人,中常侍甫之養子也。甫在宦者傳。吉少好誦讀書傳,喜名聲,而性殘忍。以父秉權寵,年二十餘,為沛相。曉達政事,能斷察疑獄,發起姦伏,多出眾議。課使郡內各舉姦吏豪人諸常有微過酒肉為臧者,雖數十年猶加貶棄,注其名籍。專選剽悍吏,擊斷非法。若有生子不養,即斬其父母,合土棘埋之。凡殺人皆磔屍車上,隨其罪目,宣示屬縣。夏月腐爛,則以繩連其骨,周遍一郡乃止,見者駭懼。視事五年,凡殺萬餘人。其餘慘毒刺刻,不可勝數。郡中惴恐,莫敢自保。及陽球奏甫,乃就收執,死於洛陽獄。

論曰:古者敦庬,善惡易分。至於畫衣冠,異服色,而莫之犯。叔世偷薄,上下相蒙,德義不足以相洽,化導不能以懲違,遂乃嚴刑痛殺,隨而繩之,致刻深之吏,以暴理姦,倚疾邪之公直,濟忍苛之虐情。漢世所謂酷能者,蓋有聞也。皆以敢捍精敏,巧附文理,風行霜烈,威譽諠赫。與夫斷斷守道之吏,何工否之殊乎!故嚴君蚩黃霸之術,密人笑卓茂之政,猛既窮矣,而猶或未勝。然朱邑不以笞辱加物,袁安未嘗鞫人臧罪,而猾惡自禁,人不欺犯。何者?以為威辟既用,而苟免之行興;仁信道孚,故感被之情著。苟免者威隙則姦起,感被者人亡而思存。由一邦以言天下,則刑訟繁措,可得而求乎!

贊曰:大道既往,刑禮為薄。斯人散矣,機詐萌作。去殺由仁,濟寬非虐。末暴雖勝,崇本或略。


\end{pinyinscope}