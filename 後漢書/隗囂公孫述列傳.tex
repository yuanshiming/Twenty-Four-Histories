\article{隗囂公孫述列傳}

\begin{pinyinscope}
隗囂字季孟,天水成紀人也。少仕州郡。王莽國師劉歆引囂為士。歆死,囂歸鄉里。季父崔,素豪俠,能得眾。聞更始立而莽兵連敗,於是乃與兄義及上邽人楊廣、冀人周宗謀起兵應漢。囂止之曰:「夫兵,凶事也。宗族何辜!」崔不聽,遂聚眾數千人,攻平襄,殺莽鎮戎大尹。崔、廣等以為舉事宜立主以一眾心,咸謂囂素有名,好經書,遂共推為上將軍。囂辭讓不得已,曰:「諸父眾賢不量小子。必能用囂言者,乃敢從命。」眾皆曰「諾」。

囂既立,遣使聘請平陵人方望,以為軍師。望至,說囂曰:「足下欲承天順民,輔漢而起,今立者乃在南陽,王莽尚據長安,雖欲以漢為名,其實無所受命,將何以見信於眾乎?宜急立高廟,稱臣奉祠,所謂『神道設教』,求助人神者也。且禮有損益,質文無常。削地開兆,茅茨土階,以致其肅敬。雖未備物,神明其舍諸。」囂從其言,遂立廟邑東,祀高祖、太宗、世宗。囂等皆稱臣執事,史奉璧而告。祝畢,有司穿坎于庭,牽馬操刀,奉盤錯鍉,遂割牲而盟。曰:「凡我同盟三十一將,十有六姓,允承天道,興輔劉宗。如懷姦慮,明神殛之。高祖、文皇、武皇,俾墜厥命,厥宗受兵,族類滅亡。」有司奉血鍉進,護軍舉手揖諸將軍曰:「鍉不濡血,歃不入口,是欺神明也,厥罰如盟。」既而薶血加書,一如古禮。

事畢,移檄告郡國曰:

「漢復元年七月己酉朔。己巳,上將軍隗囂、白虎將軍隗崔、左將軍隗義、右將軍楊廣、明威將軍王遵、雲旗將軍周宗等,告州牧、部監、郡卒正、連率、大尹、尹、尉隊大夫、屬正、屬令:故新都侯王莽,慢侮天地,悖道逆理。鴆殺孝平皇帝,篡奪其位。矯託天命,偽作符書,欺惑眾庶,震怒上帝。反戾飾文,以為祥瑞。戲弄神祇,歌頌禍殃。楚、越之竹,不足以書其惡。天下昭然,所共聞見。今略舉大端,以喻吏民。

蓋天為父,地為母,禍福之應,各以事降。莽明知之。而冥昧觸冒,不顧大忌,詭亂天術,援引史傳。昔秦始皇毀壞謚法,以一二數欲至萬世,而莽下三萬六千歲之歷,言身當盡此度。循亡秦之軌,推無窮之數。是其逆天之大罪也。

分裂郡國,斷截地絡。田為王田,賣買不得。規錮山澤,奪民本業。造起九廟,窮極土作。發冢河東,攻劫丘壟。此其逆地之大罪也。

尊任殘賊,信用姦佞,誅戮忠正,覆按口語,赤車奔馳,法冠晨夜,冤繫無辜,妄族眾庶。行炮格之刑,除順時之法,灌以醇醯,裂以五毒。政令日變,官名月易,貨幣歲改,吏民昏亂,不知所從,商旅窮窘,號泣市道。設為六管,增重賦斂,刻剝百姓,厚自奉養,苞苴流行,財入公輔,上下貪賄,莫相檢考。民坐挾銅炭,沒入鍾官,徒隸殷積,數十萬入,工匠飢死,長安皆臭。既亂諸夏,狂心益悖,北攻強胡,南擾勁越,西侵羌戎,東摘濊貊。使四境之外,並入為害,緣邊之郡,江海之瀕,滌地無類。故攻戰之所敗,苛法之所陷,飢饉之所夭,疾疫之所及,以萬萬計。其死者則露屍不掩,生者則奔亡流散,幼孤婦女,流離係虜。此其逆人之大罪也。

是故上帝哀矜,降罰于莽,妻子顛殞,還自誅刈。大臣反據,亡形已成。大司馬董忠,國師劉歆,衛將軍王涉,皆結謀內潰;司命孔仁,納言嚴尤,秩宗陳茂,舉眾外降。今山東之兵二百餘萬,已平齊、楚,下蜀、漢,定宛、洛,據敖倉,守函谷,威命四布,宣風中岳。興滅繼絕,封定萬國,遵高祖之舊制,修孝文之遺德。有不從命,武軍平之。馳使四夷,復其爵號。然後還師振旅,櫜弓臥鼓。申命百姓,各安其所,庶無負子之責。」

囂乃勒兵十萬,擊殺雍州牧陳慶。將攻安定。安定大尹王向,莽從弟平阿侯譚之子也,威風獨能行其邦內,屬縣皆無叛者。囂乃移書於向,喻以天命,反覆誨示,終不從。於是進兵虜之,以徇百姓,然後行戮,安定悉降。而長安中亦起兵誅王莽。囂遂分遣諸將徇隴西、武都、金城、武威、張掖、酒泉、敦煌,皆下之。

更始二年,遣使徵囂及崔、義等。囂將行,方望以為更始未可知,固止之,囂不聽。望以書辭謝而去,曰:「足下將建伊、呂之業,弘不世之功,而大事草創,英雄未集。以望異域之人,疵瑕未露,欲先崇郭隗,想望樂毅,故欽承大旨,順風不讓。將軍以至德尊賢,廣其謀慮,動有功,發中權,基業已定,大勳方緝。今俊乂並會,羽翮並肩,望無耆户之德,而猥託賓客之上,誠自愧也。雖懷介然之節,欲絜去就之分,誠終不背其本,貳其志也。何則?范蠡收責句踐。偏舟於五湖;舅犯謝罪文公,亦逡巡於河上。夫以二子之賢,勒銘兩國,猶削跡歸愆,請命乞身,望之無勞,蓋其宜也。望聞烏氏有龍池之山,微徑南通,與漢相屬,其傍時有奇人,聊及閑暇,廣求其真。願將軍勉之。」囂等遂至長安,更始以為右將軍,崔、義皆即舊號。其冬,崔、義謀欲叛歸,囂懼并禍,即以事告之,崔、義誅死。更始感囂忠,以為御史大夫。

明年夏,赤眉入關,三輔擾亂。流聞光武即位河北,囂即說更始歸政於光武叔父國三老良,更始不聽。諸將欲劫更始東歸,囂亦與通謀。事發覺,更始使使者召囂,囂稱疾不入,因會客王遵、周宗等勒兵自守。更始使執金吾鄧曄將兵圍囂,囂閉門拒守;至昏時,遂潰圍,與數十騎夜斬平城門關,亡歸天水。復招聚其眾,據故地,自稱西州上將軍。

及更始敗,三輔耆老士大夫皆奔歸囂。

囂素謙恭愛士,傾身引接為布衣交。以前王莽平河大尹長安谷恭為掌野大夫,平陵范逡為師友,趙秉、蘇衡、鄭興為祭酒,申屠剛、杜林為持書,楊廣、王遵、周宗及平襄人行巡、阿陽人王捷、長陵人王元為大將軍,杜陵、金丹之屬為賓客。由此名震西州,聞於山東。

建武二年,大司徒鄧禹西擊赤眉,屯雲陽。禹裨將馮愔引兵叛禹,西向天水,囂逆擊,破之於高平,盡獲輜重。於是禹承制遣使持節命囂為西州大將軍,得專制涼州、朔方事。及赤眉去長安,欲西上隴,囂遣將軍楊廣迎擊,破之,又追敗之於烏氏、涇陽閒。

囂既有功於漢,又受鄧禹爵,署其腹心,議者多勸通使京師。三年,囂乃上書詣闕。光武素聞其風聲,報以殊禮,言稱字,用敵國之儀,所以慰藉之良厚。時陳倉人呂鮪擁眾數萬,與公孫述通,寇三輔。囂復遣兵佐征西大將軍馮異擊之,走鮪,遣使上狀。帝報以手書曰:「慕樂德義,思相結納。昔文王三分,猶服事殷。但駑馬鈆刀,不可強扶。數蒙伯樂一顧之價,而蒼蠅之飛,不過數步,即託驥尾,得以絕群。隔於盜賊,聲問不數。將軍操執款款,扶傾救危,南距公孫之兵,北禦羌胡之亂,是以馮異西征,得以數千百人躑躅三輔。微將軍之助,則咸陽已為他人禽矣。今關東寇賊,往往屯聚,志務廣遠,多所不暇,未能觀兵成都,與子陽角力。如令子陽到漢中、三輔,願因將軍兵馬,鼓旗相當。儻肯如言,蒙天之福,即智士計功割地之秋也。管仲曰:『生我者父母,成我者鮑子。』自今以後,手書相聞,勿用傍人解構之言。」自是恩禮愈篤。

其後公孫述數出兵漢中,遣使以大司空扶安王印綬授囂。囂自以與述敵國,恥為所臣,乃斬其使,出兵擊之,連破述軍,以故蜀兵不復北出。

時關中將帥數上書,言蜀可擊之狀,帝以示囂,因使討蜀,以效其信。囂乃遣長史上書,盛言三輔單弱,劉文伯在邊,未宜謀蜀。帝知囂欲持兩端,不願天下統一,於是稍黜其禮,正君臣之儀。

初,囂與來歙、馬援相善,故帝數使歙、援奉使往來,勸令入朝,許以重爵。囂不欲東,連遣使深持謙辭,言無功德,須四方平定,退伏閭里。五年,復遣來歙說囂遣子入侍,囂聞劉永、彭寵皆已破滅,乃遣長子恂隨歙詣闕。以為胡騎校尉,封鐫羌侯。而囂將王元、王捷常以為天下成敗未可知,不願專心內事。元遂說囂曰:「昔更始西都,四方響應,天下喁喁,謂之太平。一旦敗壞,大王幾無所厝。今南有子陽,北有文伯,江湖海岱,王公十數,而欲牽儒生之說,棄千乘之基,羇旅危國,以求萬全,此循覆車之軌,計之不可者也。今天水完富,士馬最強,北收西河、上郡,東收三輔之地,案秦舊跡,表裏河山。元請以一丸泥為大王東封函谷關,此萬世一時也。若計不及此,且畜養士馬,據隘自守,曠日持久,以待四方之變,圖王不成,其弊猶足以霸。要之,魚不可脫於淵,神龍失埶,即還與蚯蚓同。」囂心然元計,雖遣子入質,猶負其險阨,欲專方面,於是游士長者,稍稍去之。

六年,關東悉平。帝積苦兵閒,以囂子內侍,公孫述遠據邊陲,乃謂諸將曰:「且當置此兩子於度外耳。」因數騰書隴、蜀,告示禍福。囂賓客、掾史多文學生,每所上事,當世士大夫皆諷誦之,故帝有所辭荅,尤加意焉。囂復遣使周游詣闕,先到馮異營,游為仇家所殺。帝遣衛尉銚期持珍寶繒帛賜囂,期至鄭被盜,亡失財物。帝常稱囂長者,務欲招之,聞而歎曰:「吾與隗囂事欲不諧,使來見殺,得賜道亡。」

會公孫述遣兵寇南郡,乃詔囂當從天水伐蜀,因此欲以潰其心腹。囂復上言:「白水險阻,棧閣絕敗。」又多設支閡。帝知其終不為用,叵欲討之。遂西幸長安,遣建威大將軍耿弇等七將軍從隴道伐蜀,先使來歙奉璽書喻旨。囂疑懼,即勒兵,使王元據隴坻,伐木塞道,謀欲殺歙。歙得亡歸。

諸將與囂戰,大敗,各引退。囂因使王元、巡侵三輔,征西大將軍馮異、征虜將軍祭遵等擊破之。囂乃上疏謝曰:「吏人聞大兵卒至,驚恐自救,臣囂不能禁止。兵有大利,不敢廢臣子之節,親自追還。昔虞舜事父,大杖則走,小杖則受。臣雖不敏,敢忘斯義。今臣之事,在於本朝,賜死則死,加刑則刑。如遂蒙恩,更得洗心,死骨不朽。」有司以囂言慢,請誅其子恂,帝不忍,復使來歙至汧,賜囂書曰:「昔柴將軍與韓信書云:『陛下寬仁,諸侯雖有亡叛而後歸,輒復位號,不誅也。』以囂文吏,曉義理,故復賜書。深言則似不遜,略言則事不決。今若束手,復遣恂弟歸闕庭者,則爵祿獲全,有浩大之福矣。吾年垂四十,在兵中十歲,厭浮語虛辭。即不欲,勿報。」囂知帝審其詐,遂遣使稱臣於公孫述。

明年,述以囂為朔寧王,遣兵往來,為之援埶。秋,囂將步騎三萬侵安定,至陰槃,馮異率諸將拒之。囂又令別將下隴,攻祭遵於汧,兵並無利,乃引還。

帝因令來歙以書招王遵,遵乃與家屬東詣京師,拜為太中大夫,封向義侯。遵字子春,霸陵人也。父為上郡太守。遵少豪俠,有才辯,雖與囂舉兵,而常有歸漢意。曾於天水私於來歙曰:「吾所以戮力不避矢石者,豈要爵位哉!徒以人思舊主,先君蒙漢厚恩,思效萬分耳。」又數勸囂遣子入侍,前後辭諫切甚,囂不從,故去焉。

八年春,來歙從山道襲得略陽城。囂出不意,懼更有大兵,乃使王元拒隴坻,行巡守番須口,王孟塞雞頭道,牛邯軍瓦亭,囂自悉其大眾圍來歙。公孫述亦遣其將李育、田弇助囂攻略陽,連月不下。帝乃率諸將西征之,數道上隴,使王遵持節監大司馬吳漢留屯於長安。

遵知囂必敗滅,而與牛邯舊故,知其有歸義意,以書喻之曰:「遵與隗王歃盟為漢,自經歷虎口,踐履死地,已十數矣。于時周洛以西無所統壹,故為王策,欲東收關中,北取上郡,進以奉天人之用,退以懲外夷之亂。數年之閒,冀聖漢復存,當挈河隴奉舊都以歸本朝。生民以來,臣人之埶,未有便於此時者也。而王之將吏,群居穴處之徒,人人扺掌,欲為不善之計。遵與孺卿日夜所爭,害幾及身者,豈一事哉!前計抑絕,後策不從,所以吟嘯扼腕,垂涕登車。幸蒙封拜,得延論議,每及西州之事,未嘗敢忘孺卿之言。今車駕大眾,已在道路,吳、耿驍將,雲集四境,而孺卿以奔離之卒,拒要扼,當軍衝,視其形埶何如哉?夫智者睹危思變,賢者泥而不滓,是以功名終申,策畫復得。故夷吾束縛而相齊,黥布杖劍以歸漢,去愚就義,功名並著。今孺卿當成敗之際,遇嚴兵之鋒,可為怖慄。宜斷之心胸,參之有識。」邯得書,沈吟十餘日,乃謝士眾,歸命洛陽,拜為太中大夫。於是囂大將十三人,屬縣十六,眾十餘萬,皆降。

王元入蜀求救,囂將妻子奔西城,從楊廣,而田弇、李育保上邽。詔告囂曰:「若束手自詣,父子相見,保無佗也。高皇帝云:『橫來,大者王,小者侯。』若遂欲為黥布者,亦自任也。」囂終不降。於是誅其子恂,使吳漢與征南大將軍岑彭圍西城,耿弇與虎牙大將軍蓋延圍上邽。車駕東歸。月餘,楊廣死,囂窮困。其大將王捷別在戎丘,登城呼漢軍曰:「隗王城守者,皆必死無二心!願諸軍亟罷,請自殺以明之。」遂自刎頸死。數月,王元、行巡、周宗將蜀救兵五千餘人,乘高卒至,鼓譟大呼曰:「百萬之眾方至!」漢軍大驚,未及成陳,元等決圍,殊死戰,遂得入城,迎囂歸冀。會吳漢等食盡退去,於是安定、北地、天水、隴西復反為囂。

九年春,囂病且餓,出城餐糗糒,恚憤而死。王元、周宗立囂少子純為王。明年,來歙、耿弇、蓋延等攻破落門,周宗、行巡、苟宇、趙恢等將純降。宗、恢及諸隗分徙京師以東,純與巡、宇徙弘農。唯王元留為蜀將。及輔威將軍臧宮破延岑,元舉眾詣宮降。

元字惠孟,初拜上蔡令,遷東平相,坐墾田不實,下獄死。

牛邯字孺卿,狄道人。有勇力才氣,雄於邊垂。及降,大司空司直杜林、太中大夫馬援並薦之,以為護羌校尉,與來歙平隴右。

十八年,純與賓客數十騎亡入胡,至武威,捕得,誅之。

論曰:隗囂援旗糾族,假制明神,跡夫創圖首事,有以識其風矣。終於孤立一隅,介于大國,隴坻雖隘,非有百二之埶,區區兩郡,以禦堂堂之鋒,至使窮廟策,竭征徭,身歿眾解,然後定之。則知其道有足懷者,所以棲有四方之桀,士至投死絕亢而不悔者矣。夫功全則譽顯,業謝則釁生,回成喪而為其議者,或未聞焉。若囂命會符運,敵非天力,雖坐論西伯,豈多嗤乎?

公孫述字子陽,扶風茂陵人也。哀帝時,以父任為郎。後父仁為河南都尉,而述補清水長。仁以述年少,遣門下掾隨之官。月餘,掾辭歸,白仁曰:「述非待教者也。」後太守以其能,使兼攝五縣,政事修理,姦盜不發,郡中謂有鬼神。王莽天鳳中,為導江卒正,居臨邛,復有能名。

及更始立,豪傑各起其縣以應漢,南陽人宗成自稱「虎牙將軍」,入略漢中;又商人王岑亦起兵於雒縣,自稱「定漢將軍」,殺王莽庸部牧以應成,眾合數萬人。述聞之,遣使迎成等。成等至成都,虜掠暴橫。述意惡之,召縣中豪桀謂曰:「天下同苦新室,思劉氏久矣,故聞漢將軍到,馳迎道路。今百姓無辜而婦子係獲,室屋燒燔,此寇賊,非義兵也。吾欲保郡自守,以待真主。諸卿欲并力者即留,不欲者便去。」豪桀皆叩頭曰:「願效死。」述於是使人詐稱漢使者自東方來,假述輔漢將軍、蜀郡太守兼益州牧印綬。乃選精兵千餘人,西擊成等。比至成都,眾數千人,遂攻成,大破之。成將垣副殺成,以其眾降。二年秋,更始遣柱功侯李寶、益州刺史張忠,將兵萬餘人徇蜀、漢。述恃其地險眾附,有自立志,乃使其弟恢於綿竹擊寶、忠,大破走之。由是威震益部。

功曹李熊說述曰:「方今四海波蕩,匹夫橫議。將軍割據千里,地什湯武,若奮威德以投天隙,霸王之業成矣。宜改名號,以鎮百姓。」述曰:「吾亦慮之,公言起我意。」於是自立為蜀王,都成都。

蜀地肥饒,兵力精強,遠方士庶多往歸之,邛、笮君長皆來貢獻。李熊復說述曰:「今山東飢饉,人庶相食;兵所屠滅,城邑丘墟。蜀地沃野千里,土壤膏腴,果實所生,無穀而飽。女工之業,覆衣天下。名材竹幹,器械之饒,不可勝用。又有魚鹽銅銀之利,浮水轉漕之便。北據漢中,杜褒、斜之險;東守巴郡,拒扞關之口;地方數千里,戰士不下百萬。見利則出兵而略地,無利則堅守而力農。東下漢水以窺秦地,南順江流以震荊、楊。所謂用天因地,成功之資。今君王之聲,聞於天下,而名號未定,志士狐疑,宜即大位,使遠人有所依歸。」述曰:「帝王有命,吾何足以當之?」熊曰:「天命無常,百姓與能。能者當之,王何疑焉!」述夢有人語之曰:「八厶子系,十二為期。」覺,謂其妻曰:「雖貴而祚短,若何?」妻對曰:「朝聞道,夕死尚可,況十二乎!」會有龍出其府殿中,夜有光耀,述以為符瑞,因刻其掌,文曰「公孫帝」。建武元年四月,遂自立為天子,號成家。色尚白。建元曰龍興元年。以李熊為大司徒,以其弟光為大司馬,恢為大司空。改益州為司隸校尉,蜀郡為成都尹。

越巂任貴亦殺王莽大尹而據郡降。述遂使將軍侯丹開白水關,北守南鄭;將軍任滿從閬中下江州,東據扞關。於是盡有益州之地。

自更始敗後,光武方事山東,未遑西伐。關中豪桀呂鮪等往往擁眾以萬數,莫知所屬,多往歸述,皆拜為將軍。遂大作營壘,陳車騎,肄習戰射,會聚兵甲數十萬人,積糧漢中,築宮南鄭。又造十層赤樓帛蘭船。多刻天下牧守印章,備置公卿百官。使將軍李育、程烏將數萬眾出陳倉,與呂鮪徇三輔。三年,征西將軍馮異擊鮪、育於陳倉,大敗之,鮪、育奔漢中。五年,延岑、田戎為漢兵所敗,皆亡入蜀。

岑字叔牙,南陽人。始起據漢中,又擁兵關西,關西所在破散,走至南陽,略有數縣。戎,汝南人。初起兵夷陵,轉寇郡縣,眾數萬人。岑、戎並與秦豐合,豐俱以女妻之。及豐敗,故二人皆降於述。述以岑為大司馬,封汝寧王,戎翼江王。六年,述遣戎與將軍任滿出江關,下臨沮、夷陵閒,招其故眾,因欲取荊州諸郡,竟不能剋。

是時,述廢銅錢,置鐵官錢,百姓貨幣不行。蜀中童謠言曰:「黃牛白腹,五銖當復。」好事者竊言王莽稱「黃」,述自號「白」,五銖錢,漢貨也,言天下當并還劉氏。述亦好為符命鬼神瑞應之事,妄引讖記。以為孔子作春秋,為赤制而斷十二公,明漢至平帝十二代,歷數盡也,一姓不得再受命。又引錄運法曰:「廢昌帝,立公孫。」括地象曰:「帝軒轅受命,公孫氏握。」援神契曰:「西太守,乙卯金。」謂西方太守而乙絕卯金也。五德之運,黃承赤而白繼黃,金據西方為白德,而代王氏,得其正序。又自言手文有奇,及得龍興之瑞。數移書中國,冀以感動眾心。帝患之,乃與述書曰:「圖讖言『公孫』,即宣帝也。代漢者當塗高,君豈高之身邪?乃復以掌文為瑞,王莽何足效乎!君非吾賊臣亂子,倉卒時人皆欲為君事耳,何足數也。君日月已逝,妻子弱小,當早為定計,可以無憂。天下神器,不可力爭,宜留三思。」署曰「公孫皇帝」。述不荅。

明年,隗囂稱臣於述。述騎都尉平陵人荊邯見東方將平,兵且西向,說述曰:「兵者,帝王之大器,古今所不能廢也。昔秦失其守,豪桀並起,漢祖無前人之跡,立錐之地,起於行陣之中,躬自奮擊,兵破身困者數矣。然軍敗復合,創愈復戰。何則?前死而成功,踰於卻就於滅亡也。隗囂遭遇運會,割有雍州,兵強士附,威加山東。遇更始政亂,復失天下,眾庶引領,四方瓦解。囂不及此時推危乘勝,以爭天命,而退欲為西伯之事,尊師章句,賓友處士,偃武息戈,卑辭事漢,喟然自以文王復出也。令漢帝釋關隴之憂,專精東伐,四分天下而有其三;使西州豪傑咸居心於山東,發閒使,招攜貳,則五分而有其四;若舉兵天水,必至沮潰,天水既定,則九分而有其八。陛下以梁州之地,內奉萬乘,外給三軍,百姓愁困,不堪上命,將有王氏自潰之變。臣之愚計,以為宜及天下之望未絕,豪傑尚可招誘,急以此時發國內精兵,令田戎據江陵,臨江南之會,倚巫山之固,築壘堅守,傳檄吳、楚,長沙以南必隨風而靡。令延岑出漢中,定三輔,天水、隴西拱手自服。如此,海內震搖,冀有大利。」述以問群臣。博士吳柱曰:「昔武王伐殷,先觀兵孟津,八百諸侯不期同辭,然猶還師以待天命。未聞無左右之助,而欲出師千里之外,以廣封疆者也。」邯曰:「今東帝無尺土之柄,驅烏合之眾,跨馬陷敵,所向輒平。不亟乘時與之分功,而坐談武王之說,是效隗囂欲為西伯也。」述然邯言,欲悉發北軍屯士及山東客兵,使延岑、田戎分出兩道,與漢中諸將合兵并埶。蜀人及其弟光以為不宜空國千里之外,決成敗於一舉,固爭之,述乃止。延岑、田戎亦數請兵立功,終疑不聽。

述性苛細,察於小事。敢誅殺而不見大體,好改易郡縣官名。然少為郎,習漢家制度,出入法駕,鑾旗旄騎,陳置陛戟,然後輦出房闥。又立其兩子為王,食犍為、廣漢各數縣。群臣多諫,以為成敗未可知,戎士暴露,而遽王皇子,示無大志,傷戰士心。述不聽。唯公孫氏得任事,由此大臣皆怨。

八年,帝使諸將攻隗囂,述遣李育將萬餘人救囂。囂敗,并沒其軍,蜀地聞之恐動。述懼,欲安眾心。成都郭外有秦時舊倉,述改名白帝倉,自王莽以來常空。述即詐使人言白帝倉出穀如山陵,百姓空市里往觀之。述乃大會群臣,問曰:「白帝倉竟出穀乎?」皆對言「無」。述曰:「訛言不可信,道隗王破者復如此矣。」俄而囂將王元降,述以為將軍。明年,使元與領軍環安拒河池,又遣田戎及大司徒任滿、南郡太守程汎將兵下江關,破虜將軍馮駿等,拔巫及夷陵、夷道,因據荊門。

十一年,征南大將軍岑彭攻之,滿等大敗,述將王政斬滿首降于彭。田戎走保江州。城邑皆開門降,彭遂長驅至武陽。帝乃與述書,陳言禍福,以明丹青之信。述省書歎息,以示所親太常常少、光祿勳張隆。隆、少皆勸降。述曰:「廢興命也。豈有降天子哉!」左右莫敢復言。

中郎將來歙急攻王元、環安,安使刺客殺歙;述復令刺殺岑彭。十二年,述弟恢及子婿史興並為大司馬吳漢、輔威將軍臧宮所破,戰死。自是將帥恐懼,日夜離叛,述雖誅滅其家,猶不能禁。帝必欲降之,乃下詔喻述曰:「往年詔書比下,開示恩信,勿以來歙、岑彭受害自疑。今以時自詣,則家族完全;若迷惑不喻,委肉虎口,痛哉柰何!將帥疲倦,吏士思歸,不樂久相屯守,詔書手記,不可數得,朕不食言。」述終無降意。

九月,吳漢又破斬其大司徒謝豐、執金吾袁吉,漢兵遂守成都。述謂延岑曰:「事當柰何?」岑曰:「男兒當死中求生,可坐窮乎!財物易聚耳,不宜有愛。」述乃悉散金帛,募敢死士五千餘人,以配岑於市橋,偽建旗幟,鳴鼓挑戰,而潛遣奇兵出吳漢軍後,襲擊破漢。漢墯水,緣馬尾得出。

十一月,臧宮軍至咸門。述視占書,云「虜死城下」,大喜,謂漢等當之。乃自將數萬人攻漢,使延岑拒宮。大戰,岑三合三勝。自旦及日中,軍士不得食,並疲,漢因令壯士突之,述兵大亂,被刺洞胸,墯馬。左右輿入城。述以兵屬延岑,其夜死。明旦,岑降吳漢。乃夷述妻子,盡滅公孫氏,并族延岑。遂放兵大掠,焚述宮室。帝聞之怒,以譴漢。又讓漢副將劉尚曰:「城降三日,吏人從服,孩兒老母,口以萬數,一旦於兵縱火,聞之可為酸鼻!尚宗室子孫,嘗更吏職,何忍行此?仰視天,俯視地,觀放麑啜羹,二者孰仁?良失斬將弔人之義也!」

初,常少、張隆勸述降,不從,並以憂死。帝下詔追贈少為太常,隆為光祿勳,以禮改葬之。其忠節志義之士,並蒙旌顯。程烏、李育以有才幹,皆擢用之。於是西土咸悅,莫不歸心焉。

論曰:昔趙佗自王番禺,公孫亦竊帝蜀漢,推其無他功能,而至於後亡者,將以地邊處遠,非王化之所先乎?述雖為漢吏,無所馮資,徒以文俗自憙,遂能集其志計。道未足而意有餘,不能因隙立功,以會時變,方乃坐飾邊幅,以高深自安,昔吳起所以慚魏侯也。及其謝臣屬,審廢興之命,與夫泥首銜玉者異日談也。

贊曰:公孫習吏,隗王得士,漢命已還,二隅方跱。天數有違,江山難恃。


\end{pinyinscope}