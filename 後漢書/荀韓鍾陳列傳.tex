\article{荀韓鍾陳列傳}

\begin{pinyinscope}
荀淑字季和,潁川潁陰人也,荀卿十一世孫也。少有高行,博學而不好章句,多為俗儒所非,而州里稱其知人。

安帝時,徵拜郎中,後再遷當塗長。去職還鄉里。當世名賢李固、李膺等皆師宗之。及梁太后臨朝,有日食地震之變,詔公卿舉賢良方正,光祿勳杜喬、少府房植舉淑對策,譏刺貴倖,為大將軍梁冀所忌,出補朗陵侯相。蒞事明理,稱為神君。頃之,棄官歸,閑居養志。產業每增,輒以贍宗族知友。年六十七,建和三年卒。李膺時為尚書,自表師喪。二縣皆為立祠。有子八人:儉,緄,靖,燾,汪,爽,肅,專,並有名稱,時人謂「八龍。」

初,荀氏舊里名西豪,潁陰令勃海苑康以為昔高陽氏有才子八人,今荀氏亦有八子,故改其里曰高陽里。

靖有至行,不仕,年五十而終,號曰玄行先生。

淑兄子昱字伯條,曇字元智。昱為沛相,曇為廣陵太守。兄弟皆正身疾惡,志除閹宦。其支黨賓客有在二郡者,纖罪必誅。昱後共大將軍竇武謀誅中官,與李膺俱死。曇亦禁錮終身。

爽字慈明,一名諝。幼而好學,年十二,能通春秋、論語。太尉杜喬見而稱之,曰:「可為人師。」爽遂耽思經書,慶弔不行,徵命不應。潁川為之語曰:「荀氏八龍,慈明無雙。」

延熹九年,太常趙典舉爽至孝,拜郎中。對策陳便宜曰:

臣聞之於師曰:「漢為火德,火生於木,木盛於火,故其德為孝,其象在周易之離。」夫在地為火,在天為日。在天者用其精,在地者用其形。夏則火王,其精在天,溫暖之氣,養生百木,是其孝也。冬時則廢,其形在地,酷烈之氣,焚燒山林,是其不孝也。故漢制使天下誦孝經,選吏舉孝廉。夫喪親自盡,孝之終也。今之公卿及二千石,三年之喪,不得即去,殆非所以增崇孝道而克稱火德者也。往者孝文勞謙,行過乎儉,故有遺詔以日易月。此當時之宜,不可貫之萬世。古今之制雖有損益,而諒闇之禮未嘗改移,以示天下莫遺其親。今公卿群寮皆政教所瞻,而父母之喪不得奔赴。夫仁義之行,自上而始;敦厚之俗,以應乎下。傳曰:「喪祭之禮闕,則人臣之恩薄,背死忘生者眾矣。」曾子曰:「人未有自致者,必也親喪乎!」春秋傳曰:「上之所為,民之歸也。」夫上所不為而民或為之,故加刑罰;若上之所為,民亦為之,又何誅焉?昔丞相翟方進,以自備宰相,而不敢踰制。至遭母憂,三十六日而除。夫失禮之源,自上而始。古者大喪三年不呼其門,所以崇國厚俗篤化之道也。事失宜正。過勿憚改。天下通喪,可如舊禮。

臣聞有夫婦然後有父子,有父子然後有君臣,有君臣然後有上下,有上下然後有禮義。禮義備,則人知所厝矣。夫婦人倫之始,王化之端,故文王作易,上經首乾、坤,下經首咸、恆。孔子曰:「天尊地卑,乾坤定矣。」夫婦之道,所謂順也。《堯典》曰:「釐降二女於媯汭,嬪于虞。」降者下也,嬪者婦也。言雖帝堯之女,下嫁於虞,猶屈體降下,勤修婦道。《易》曰:「帝乙歸妹,以祉元吉。」婦人謂嫁曰歸,言湯以娶禮歸其妹於諸侯也。春秋之義,王姬嫁齊,使魯主之,不以天子之尊加於諸侯也。今漢承秦法,設尚主之儀,以妻制夫,以卑臨尊,違乾坤之道,失陽唱之義。孔子曰:「昔聖人之作易也,仰則觀象於天,俯則察法於地,睹鳥獸之文,與地之宜。近取諸身,遠取諸物,以通神明之德,以類萬物之情。」今觀法於天,則北極至尊,四星妃后。察法於地,則崑山象夫,卑澤象妻。睹鳥獸之文,鳥則雄者鳴鴝,雌能順服;獸則牡為唱導,牝乃相從。近取諸身,則乾為人首,坤為人腹。遠取諸物,則木實屬天,根荄屬地。陽尊陰卑,蓋乃天性。且詩初篇實首關雎;禮始冠、婚,先正夫婦。天地六經,其旨一揆。宜改尚主之制,以稱乾坤之性。遵法堯、湯,式是周、孔。合之天地而不謬,質之鬼神而不疑。人事如此,則嘉瑞降天,吉符出地,五韙咸備,各以其敘矣。

昔者聖人建天地之中而謂之禮,禮者,所以興福祥之本,而止禍亂之源也。人能枉欲從禮者,則福歸之;順情廢禮者,則禍歸之。推禍福之所應,知興廢之所由來也。眾禮之中,婚禮為首。故天子娶十二,天之數也;諸侯以下各有等差,事之降也。陽性純而能施,陰體順而能化,以禮濟樂,節宣其氣。故能豐子孫之祥,致老壽之福。及三代之季,淫而無節。瑤臺、傾宮,陳妾數百。陽竭於上,陰隔於下。故周公之戒曰:「不知稼穡之艱難,不聞小人之勞,惟耽樂之從,時亦罔或克壽。」是其明戒。後世之人,好福不務其本,惡禍不易其軌。傳曰:「涞趾適履,孰云其愚?何與斯人,追欲喪軀?」誠可痛也。臣竊聞後宮采女五六千人,從官侍使復在其外。冬夏衣服,朝夕稟糧,耗費縑帛,空竭府藏,徵調增倍,十而稅一,空賦不辜之民,以供無用之女,百姓窮困於外,陰陽隔塞于內。故感動和氣,災異屢臻。臣愚以為諸非禮聘未曾幸御者,一皆遣出,使成妃合。一曰通怨曠,和陰陽。二曰省財用,實府藏。三曰脩禮制,綏眉壽。四曰配陽施,祈螽斯。五曰寬役賦,安黎民。此誠國家之弘利,天人之大福也。

夫寒熱晦明,所以為歲;尊卑奢儉,所以為禮:故以晦明寒暑之氣,尊卑侈約之禮為其節也。《易》曰:「天地節而四時成。」春秋傳曰:「唯器與名不可以假人。」《孝經》曰:「安上治民,莫善於禮。」禮者,尊卑之差,上下之制也。昔季氏八佾舞於庭,非有傷害困於人物,而孔子猶曰「是可忍也,孰不可忍」。洪範曰:「惟辟作威,惟辟作福,惟辟玉食。」凡此三者,君所獨行而臣不得同也。今臣僭君服,下食上珍,所謂害于而家,凶於而國者也。宜略依古禮尊卑之差,及董仲舒制度之別,嚴篤有司,必行其命。此則禁亂善俗足用之要。奏聞,即棄官去。

後遭黨錮,隱於海上,又南遁漢濱,積十餘年,以著述為事,遂稱為碩儒。黨禁解,五府並辟,司空袁逢舉有道,不應。及逢卒,爽制服三年,當世往往化以為俗。時人多不行妻服,雖在親憂猶有弔問喪疾者,又私謚其君父及諸名士,爽皆引據大義,正之經典,雖不悉變,亦頗有改。

後公車徵為大將軍何進從事中郎。進恐其不至,迎薦為侍中,及進敗而詔命中絕。獻帝即立,董卓輔政,復徵之。爽欲遁命,吏持之急,不得去,因復就拜平原相。行至宛陵,復追為光祿勳。視事三日,進拜司空。爽自被徵命及登台司,九十五日。因從遷都長安。

爽見董卓忍暴滋甚,必危社稷,其所辟舉皆取才略之士,將共圖之,亦與司徒王允及卓長史何顒等為內謀。會病薨,年六十三。

著禮、易傳、詩傳、尚書正經、春秋條例,又集漢事成敗可為鑒戒者,謂之漢語。又作公羊問及辯讖,并它所論敘,題為新書。凡百餘篇,今多所亡缺。

兄子悅、彧並知名。彧自有傳。

論曰:荀爽、鄭玄、申屠蟠俱以儒行為處士,累徵並謝病不詣。及董卓當朝,復備禮召之。蟠、玄竟不屈以全其高。爽已黃髮矣,獨至焉,未十旬而取卿相。意者疑其乖趣舍,余竊商其情,以為出處君子之大致也,平運則弘道以求志,陵夷則濡跡以匡時。荀公之急急自勵,其濡跡乎?不然,何為違貞吉而履虎尾焉?觀其遜言遷都之議,以救楊、黃之禍。及後潛圖董氏,幾振國命,所謂「大直若屈」,道固逶迤也。

悅字仲豫,儉之子也。儉早卒。悅年十二,能說春秋。家貧無書,每之人閒,所見篇牘,一覽多能誦記。性沈靜,美姿容,尤好著述。靈帝時閹官用權,士多退身窮處,悅乃託疾隱居,時人莫之識,雖從弟彧特稱敬焉。初辟鎮東將軍曹操府,遷黃門侍郎。獻帝頗好文學,悅與彧及少府孔融侍講禁中,旦夕談論。累遷祕書監、侍中。

時政移曹氏,天子恭己而已。悅志在獻替,而謀無所用,乃作申鑒五篇。其所論辯,通見政體,既成而奏之。其大略曰:

夫道之本,仁義而已矣。五典以經之,群籍以緯之,詠之歌之,弦之舞之,前監既明,後復申之。故古之聖王,其於仁義也,申重而已。

致政之術,先屏四患,乃崇五政。

一曰偽,二曰私,三曰放,四曰奢。偽亂俗,私壞法,放越軌,奢敗制。四者不除,則政末由行矣。夫俗亂則道荒,雖天地不得保其性矣;法壞則世傾,雖人主不得守其度矣;軌越則禮亡,雖聖人不得全其道矣;制敗則欲肆,雖四表不得充其求矣。是謂四患。

興農桑以養其性,審好惡以正其俗,宣文教以章其化,立武備以秉以其威,明賞罰以統其法。是謂五政。

人不畏死,不可懼以罪。人不樂生,不可勸以善。雖使契布五教,皋陶作士,政不行焉。故在上者先豐人財以定其志,帝耕籍田,后桑蠶宮,國無遊人,野無荒業,財不賈用,力不妄加,以周人事。是謂養生。

君子之所以動天地,應神明,正萬物而成王化者,必乎真定而已。故在上者審定好醜焉。善惡要乎功罪,毀譽效於準驗。聽言責事,舉名察實,無惑詐偽,以蕩眾心。故事無不覈,物無不切,善無不顯,惡無不章,俗無姦怪,民無淫風。百姓上下睹利害之存乎己也,故肅恭其心,慎修其行,內不回惑,外無異望,則民志平矣。是謂正俗。

君子以情用,小人以刑用。榮辱者,賞罰之精華也。故禮教榮辱,以加君子,化其情也;桎梏鞭撲,以加小人,化其刑也。君子不犯辱,況於刑乎!小人不忌刑,況於辱乎!若教化之廢,推中人而墜於小人之域;教化之行,引中人而納於君子之塗。是謂章化。小人之情,緩則驕,驕則恣,恣則怨,怨則叛,危則謀亂,安則思欲,非威強無以懲之。故在上者,必有武備,以戒不虞,以遏寇虐。安居則寄之內政,有事則用之軍旅。是謂秉威。

賞罰,政之柄也。明賞必罰,審信慎令,賞以勸善,罰以懲惡。人主不妄賞,非徒愛其財也,賞妄行則善不勸矣。不妄罰,非矜其人也,罰妄行則惡不懲矣。賞不勸謂之止善,罰不懲謂之縱惡。在上者能不止下為善,不縱下為惡,則國法立矣。是謂統法。

四患既蠲,五政又立,行之以誠,守之以固,簡而不怠,疏而不失,無為為之,使自施之,無事事之,使自交之。不肅而成,不嚴而化,垂拱揖讓,而海內平矣。是謂為政之方。

又言:

尚主之制非古。釐降二女,陶唐之典。歸妹元吉,帝乙之訓。王姬歸齊,宗周之禮。以陰乘陽違天,以婦陵夫違人。違天不祥,違人不義。又古者天子諸侯有事,必告于廟。朝有二史,左史記言,右史書事。事為春秋,言為尚書。君舉必記,善惡成敗,無不存焉。下及士庶,苟有茂異,咸在載籍。或欲顯而不得,或欲隱而名章。得失一朝,而榮辱千載。善人勸焉,淫人懼焉。宜於今者備置史官,掌其典文,紀其行事。每於歲盡,舉之尚書。以助賞罰,以弘法教。

帝覽而善之。

帝好典籍,常以班固漢書文繁難省,乃令悅依左氏傳體以為漢紀三十篇,詔尚書給筆札。辭約事詳,論辨多美。其序之曰:「昔在上聖,惟建皇極,經緯天地,觀象立法,乃作書契,以通宇宙,揚于王庭,厥用大焉。先王光演大業,肆于時夏。亦惟厥後,永世作典。夫立典有五志焉:一曰達道義,二曰章法式,三曰通古今,四曰著功勳,五曰表賢能。於是天人之際,事物之宜,粲然顯著,罔不備矣。世濟其軌,不隕其業。損益盈虛,與時消息。臧否不同,其揆一也。漢四百有六載,撥亂反正,統武興文,永惟祖宗之洪業,思光啟乎萬嗣。聖上穆然,惟文之恤,瞻前顧後,是紹是繼,闡崇大猷,命立國典。於是綴敘舊書,以述漢紀。中興以前,明主賢臣得失之軌,亦足以觀矣。」

又著崇德、正論及諸論數十篇。年六十二,建安十四年卒。

韓韶字仲黃,潁川舞陽人也。少仕郡,辟司徒府。時太山賊公孫舉偽號歷年,守令不能破散,多為坐法。尚書選三府掾能理劇者,乃以韶為贏長。賊聞其賢,相戒不入贏境。餘縣多被寇盜,廢耕桑,其流入縣界求索衣糧者甚眾。韶愍其飢困,乃開倉賑之,所稟贍萬餘戶。主者爭謂不可。韶曰:「長活溝壑之人,而以此伏罪,含笑入地矣。」太守素知韶名德,竟無所坐。以病卒官。同郡李膺、陳寔、杜密、荀淑等為立碑頌焉。

子融,字元長。少能辯理而不為章句學。聲名甚盛,五府並辟。獻帝初,至太僕。年七十卒。

鍾皓字季明,潁川長社人也。為郡著姓,世善刑律。皓少以篤行稱,公府連辟,為二兄未仕,避隱密山,以詩律教授門徒千餘人。同郡陳寔,年不及皓,皓引與為友。皓為郡功曹,會辟司徒府,臨辭,太守問:「誰可代卿者?」皓曰:「明府欲必得其人,西門亭長陳寔可。」寔聞之,曰:「鍾君似不察人,不知何獨識我?」皓頃之自劾去。前後九辟公府,徵為廷尉正、博士、林慮長,皆不就。時皓及荀淑並為士大夫所歸慕。李膺常歎曰:「荀君清識難尚,鍾君至德可師。」

皓兄子瑾母,膺之姑也。瑾好學慕古,有退讓風,與膺同年,俱有聲名。膺祖太尉脩,常言:「瑾似我家性,邦有道,不廢;邦無道,免於刑戮。」復以膺妹妻之。瑾辟州府,未嘗屈志。膺謂之曰:「孟子以為『人無是非之心,非人也』。弟何期不與孟軻同邪?」瑾常以膺言白皓。皓曰:「昔國武子好昭人過,以致怨本。卒保身全家,爾道為貴。」其體訓所安,多此類也。

年六十九,終於家。諸儒頌之曰:「林慮懿德,非禮不處。悅此詩書,弦琴樂古。五就州招,九應台輔。逡巡王命,卒歲容與。」

皓孫繇,建安中為司隸校尉。

陳寔字仲弓,潁川許人也。出於單微。自為兒童,雖在戲弄,為等類所歸。少作縣吏,常給事廝役,後為都亭刺佐。而有志好學,坐立誦讀。縣令鄧邵試與語,奇之,聽受業太學。後令復召為吏,乃避隱陽城山中。時有殺人者,同縣楊吏以疑寔,縣遂逮繫,考掠無實,而後得出。及為督郵,乃密託許令,禮召楊吏。遠近聞者,咸歎服之。

家貧,復為郡西門亭長,尋轉功曹。時中常侍侯覽託太守高倫用吏,倫教署為文學掾。寔知非其人,懷檄請見。言曰:「此人不宜用,而侯常侍不可違。寔乞從外署,不足以塵明德。」倫從之。於是鄉論怪其非舉,寔終無所言。倫後被徵為尚書,郡中士大夫送至輪氏傳舍。倫謂眾人言曰:「吾前為侯常侍用吏,陳君密持教還,而於外白署。比聞議者以此少之,此咎由故人畏憚強禦,陳君可謂善則稱君,過則稱己者也。」寔固自引愆,聞者方歎息,由是天下服其德。

司空黃瓊辟選理劇,補聞喜長,旬月,以期喪去官。復再遷除太丘長。修德清靜,百姓以安。鄰縣人戶歸附者,寔輒訓導譬解,發遣各令還本司官行部。吏慮有訟者,白欲禁之。寔曰:「訟以求直,禁之理將何申?其勿有所拘。」司官聞而歎息曰:「陳君所言若是,豈有怨於人乎?」亦竟無訟者。以沛相賦斂違法,及解印綬去,吏人追思之。

及後逮捕黨人,事亦連寔。餘人多逃避求免,寔曰:「吾不就獄,眾無所恃。」乃請囚焉。遇赦得出。靈帝初,大將軍竇武辟以為掾屬。時中常侍張讓權傾天下。讓父死,歸葬潁川,雖一郡畢至,而名士無往者,讓甚恥之,寔乃獨弔焉。乃後復誅黨人,讓感寔,故多所全宥。

寔在鄉閭,平心率物。其有爭訟,輒求判正,曉譬曲直,退無怨者。至乃歎曰:「寧為刑罰所加,不為陳君所短。」時歲荒民儉,有盜夜入其室,止於梁上。寔陰見,乃起自整拂,呼命子孫,正色訓之曰:「夫人不可不自勉。不善之人未必本惡,習以性成,遂至於此。梁上君子者是矣!」盜大驚,自投於地,稽顙歸罪。寔徐譬之曰:「視君狀貌,不似惡人,宜深剋己反善。然此當由貧困。」令遺絹二匹。自是一縣無復盜竊。

太尉楊賜、司徒陳耽,每拜公卿,群僚畢賀,賜等常歎寔大位未登,愧於先之。及黨禁始解,大將軍何進、司徒袁隗遣人敦寔,欲特表以不次之位。寔乃謝使者曰:「寔久絕人事,飾巾待終而已。」時三公每缺,議者歸之,累見徵命,遂不起,閉門懸車,棲遲養老。中平四年,年八十四,卒于家。何進遣使弔祭,海內赴者三萬餘人,制衰麻者以百數。共刊石立碑,謚為文範先生。

有六子,紀、諶最賢。

紀字元方,亦以至德稱。兄弟孝養,閨門廱和,後進之士皆推慕其風。及遭黨錮,發憤著書數萬言,號曰陳子。黨禁解,四府並命,無所屈就。遭父憂,每哀至,輒歐血絕氣,雖衰服已除,而積毀消瘠,殆將滅性。豫州刺史嘉其至行,表上尚書,圖象百城,以厲風俗。董卓入洛陽,乃使就家拜五官中郎將,不得已,到京師,遷侍中。出為平原相,往謁卓,時欲徙都長安。乃謂紀曰:「三輔平敞,四面險固,土地肥美,號為陸海。今關東兵起,恐洛陽不可久居。長安猶有宮室,今欲西遷何如?」紀曰:「天下有道,守在四夷。宜脩德政,以懷不附。遷移至尊,誠計之末者。愚以公宜事委公卿,專精外任。其有違命,則威之以武。今關東兵起,民不堪命。若謙遠朝政,率師討伐,則塗炭之民,庶幾可全。若欲徙萬乘以自安,將有累卵之危,崢嶸之險也。」卓意甚忤,而敬紀名行,無所復言。時議欲以為司徒,紀見禍亂方作,不復辨嚴,即時之郡。璽書追拜太僕,又徵為尚書令。建安初,袁紹為太尉,讓於紀;紀不受,拜大鴻臚。年七十一,卒於官。

子群,為魏司空。天下以為公慚卿,卿慚長。

弟諶,字季方。與紀齊德同行,父子並著高名,時號三君。每宰府辟召,常同時旌命,羔鴈成群,當世者靡不榮之。諶早終。

論曰:漢自中世以下,閹豎擅恣,故俗遂以遁身矯絜放言為高。士有不談此者,則芸夫牧豎已叫呼之矣。故時政彌惛,而其風愈往。唯陳先生進退之節,必可度也。據於德故物不犯,安於仁故不離群,行成乎身而道訓天下,故凶邪不能以權奪,王公不能以貴驕,所以聲教廢於上,而風俗清乎下也。

贊曰:二李師淑,陳君友皓。韓韶就吏,贏寇懷道。太丘奧廣,模我彝倫。曾是淵軌,薄夫以淳。慶基既啟,有蔚潁濱,二方承則,八慈繼塵。


\end{pinyinscope}