\article{獨行列傳}

\begin{pinyinscope}
孔子曰:「與其不得中庸,必也狂狷乎!」又云:「狂者進取,狷者有所不為也。」此蓋失於周全之道,而取諸偏至之端者也。然則有所不為,亦將有所必為者矣;既云進取,亦將有所不取者矣。如此,性尚分流,為否異適矣。

中世偏行一介之夫,能成名立方者,蓋亦眾也。或志剛金石,而剋扞於強禦。或意嚴冬霜,而甘心於小諒。亦有結朋協好,幽明共心;蹈義陵險,死生等節。雖事非通圓,良其風軌有足懷者。而情跡殊雜,難為條品;片辭特趣,不足區別。措之則事或有遺,載之則貫序無統。以其名體雖殊,而操行俱絕,故總為獨行篇焉。庶備諸闕文,紀志漏脫云爾。

譙玄字君黃,巴郡閬中人也。少好學,能說易、春秋。仕於州郡。成帝永始二年,有日食之災,乃詔舉敦樸遜讓有行義者各一人。州舉玄,詣公車,對策高第,拜議郎。

帝始作期門,數為微行。立趙飛燕為皇后,后專寵懷忌,皇太子多橫夭。玄上書諫曰:「臣聞王者承天,繼宗統極,保業延祚,莫急胤嗣,故易有幹蠱之義,詩詠眾多之福。今陛下聖嗣未立,天下屬望,而不惟社稷之計,專念微行之事,愛幸用於所惑,曲意留於非正。竊聞後宮皇子產而不育。臣聞之怛然,痛心傷剝,竊懷憂國,不忘須臾。夫警衛不脩,則患生非常。忽有醉酒狂夫,分爭道路,既無尊嚴之儀,豈識上下之別。此為胡狄起於轂下,而賊亂發於左右也。願陛下念天之至重,愛金玉之身,均九女之施,存無窮之福,天下幸甚。」

時數有災異,玄輒陳其變。既不省納,故久稽郎官。後遷太常丞,以弟服去職。

平帝元始元年,日食,又詔公卿舉敦朴直言。大鴻臚左咸舉玄詣公車對策,復拜議郎,遷中散大夫。四年,選明達政事能班化風俗者八人。時並舉玄,為繡衣使者,持節,與太僕任惲等分行天下,觀覽風俗,所至專行誅賞。事未及終,而王莽居攝,玄於是縱使者車,變易姓名,閒竄歸家,因以隱遁。

後公孫述僭號於蜀,連聘不詣。述乃遣使者備禮徵之;若玄不肯起,使陽以毒藥。太守乃自齎璽書至玄廬,曰:「君高節已著,朝廷垂意,誠不宜復辭,自招凶禍。」玄仰天歎曰:「唐堯大聖,許由恥仕;周武至德,伯夷守餓。彼獨何人,我亦何人。保志全高,死亦奚恨!」遂受毒藥。玄子瑛泣血叩頭於太守曰:「方今國家東有嚴敵,兵師四出,國用軍資或不常充足,願奉家錢千萬,以贖父死。」太守為請,述聽許之。玄遂隱藏田野,終述之世。

時兵戈累年,莫能脩尚學業,玄獨訓諸子勤習經書。建武十一年卒。明年,天下平定,玄弟慶以狀詣闕自陳。光武美之,策詔本郡祠以中牢,敕所在還玄家錢。

時亦有犍為費貽,不肯仕述,乃漆身為厲,陽狂以避之,退藏山藪十餘年。述破後,仕至合浦太守。

瑛善說易,以授顯宗,為北宮衛士令。

李業字巨游,廣漢梓潼人也。少有志操,介特。習魯詩,師博士許晃。元始中,舉明經,除為郎。

會王莽居攝,業以病去官,杜門不應州郡之命。太守劉咸強召之,業乃載病詣門。咸怒,出教曰:「賢者不避害,譬猶殼弩射市,薄命者先死。聞業名稱,故欲與之為治,而反託疾乎?」令詣獄養病,欲殺之。客有說咸曰:「趙殺鳴犢,孔子臨河而逝。未聞求賢而脅以牢獄者也。」咸乃出之,因舉方正。王莽以業為酒士,病不之官,遂隱藏出谷,絕匿名跡,終莽之世。

及公孫述僭號,素聞業賢,徵之,欲以為博士,業固疾不起。數年,述羞不致之,乃使大鴻臚尹融持毒酒奉詔命以劫業:若起,則受公侯之位:不起,賜之以藥。融譬旨曰:「方今天下分崩,孰知是非,而以區區之身,試於不測之淵乎!朝廷貪慕名德,曠官缺位,于今七年,四時珍御,不以忘君。宜上奉知己,下為子孫,身名俱全,不亦優乎!今數年不起,猜疑寇心,凶禍立加,非計之得者也。」業乃歎曰:「危國不入,亂國不居。親於其身為不善者,義所不從。君子見危授命,何乃誘以高位重餌哉?」融見業辭志不屈,復曰:「宜呼室家計之。」業曰:「丈夫斷之於心久矣,何妻子之為?」遂飲毒而死。述聞業死,大驚,又恥有殺賢之名,乃遣使弔祠,賻贈百匹。業子翬逃辭不受。

蜀平,光武下詔表其閭,益部紀載其高節,圖畫形象。

初,平帝時,蜀郡王皓為美陽令,王嘉為郎。王莽篡位,並棄官西歸。及公孫述稱帝,遣使徵皓、嘉,恐不至,遂先繫其妻子。使者謂嘉曰:「速裝,妻子可全。」對曰:「犬馬猶識主,況於人乎!」王皓先自刎,以首付使者。述怒,遂誅皓家屬。王嘉聞而歎曰:「後之哉!」乃對使者伏劍而死。

是時犍為任永君業同郡馮信,並好學博古。公孫述連徵命,待以高位,皆託青盲以避世難。永妻淫於前,匿情無言;見子入井,忍而不救。信侍婢亦對信姦通。及聞述誅,皆盥洗更視曰:「世適平,目即清。」淫者自殺。光武聞而徵之,並會病卒。

劉茂字子衛,太原晉陽人也。少孤,獨侍母居。家貧,以筋力致養,孝行著於鄉里。及長,能習禮經,教授常數百人。哀帝時,察孝廉,再遷五原屬國候,遭母憂去官。服竟後為沮陽令。會王莽篡位,茂棄官,避世弘農山中教授。

建武二年,歸,為郡門下掾。時赤眉二十餘萬眾攻郡縣,殺長吏及府掾史。茂負太守孫福踰牆藏空穴中,得免。其暮,俱奔盂縣。晝則逃隱,夜求糧食。積百餘日,賊去,乃得歸府。明年,詔書求天下義士。福言茂曰:「臣前為赤眉所攻,吏民壞亂,奔走趣山,臣為賊所圍,命如絲髮,賴茂負臣踰城,出保盂縣。茂與弟觸冒兵刃,緣山負食,臣及妻子得度死命,節義尤高。宜蒙表擢,以厲義士。」詔書即徵茂拜議郎,遷宗正丞。後拜侍中,卒官。

元初中,鮮卑數百餘騎寇漁陽,太守張顯率吏士追出塞,遙望虜營煙火,急趣之。兵馬掾嚴授慮有伏兵,苦諫止,不聽。顯蹙令進,授不獲已,前戰,伏兵發,授身被十創,歿於陣。顯拔刃追散兵,不能制,虜射中顯,主簿衛福、功曹徐咸遽起之,顯遂墮馬,福以身擁蔽,虜并殺之。朝廷愍授等節,詔書褒歎,厚加賞賜,各除子一人為郎中。

永初二年,劇賊畢豪等入平原界,縣令劉雄將吏士乘船追之。至厭次河,與賊合戰。雄敗,執雄,以矛刺之。時小吏所輔前叩頭求哀,願以身代雄。豪等縱雄而刺輔,貫心洞背即死。東郡太守捕得豪等,具以狀上。詔書追傷之,賜錢二十萬,除父奉為郎中。

溫序字次房,太原祁人也。仕州從事。建武二年,騎都尉弓里戍將兵平定北州,到太原,歷訪英俊大人,問以策謀。戍見序奇之,上疏薦焉。於是徵為侍御史。遷武陵都尉,病免官。

六年,拜謁者,遷護羌校尉。序行部至襄武,為隗囂別將苟宇所拘劫。宇謂序曰:「子若與我并威同力,天下可圖也。」序曰:「受國重任,分當效死,義不貪生苟背恩德。」宇等復曉譬之。序素有氣力,大怒,叱宇等曰:「虜何敢迫脅漢將!」因以節檛殺數人。賊眾爭欲殺之。宇止之曰:「此義士死節,可賜以劍。」序受劍,銜鬚於口,顧左右曰:「既為賊所迫殺,無令鬚汙土。」遂伏劍而死。

序主簿韓遵、從事王忠持屍歸斂。光武聞而憐之,命忠送喪到洛陽,賜城傍為冢地,賻穀千斛、縑五百匹,除三子為郎中。長子壽,服竟為鄒平侯相。夢序告之曰:「久客思鄉里。」壽即棄官,上書乞骸骨歸葬。帝許之,乃反舊塋焉。

彭脩字子陽,會稽毗陵人也。年十五時,父為郡吏,得休,與脩俱歸,道為盜所劫,脩困迫,乃拔佩刀前持盜帥曰:「

父辱子死,卿不顧死邪?」盜相謂曰:「此童子義士也,不宜逼之。」遂辭謝而去。鄉黨稱其名。

後仕郡為功曹。時西部都尉宰晁行太守事,以微過收吳縣獄吏,將殺之,主簿鐘離意爭諫甚切,晁怒,使收縛意,欲案之,掾吏莫敢諫。脩排閤直入,拜於庭,曰:「明府發雷霆於主簿,請聞其過。」晁曰:「受教三日,初不奉行,廢命不忠,豈非過邪?」脩因拜曰:「昔任座面折文侯,朱雲攀毀欄檻,自非賢君,焉得忠臣?今慶明府為賢君,主簿為忠臣。」晁遂原意罰,貰獄吏罪。

後州辟從事。時賊張子林等數百人作亂,郡言州,請脩守吳令。脩與太守俱出討賊,賊望見車馬,競交射之,飛矢雨集。脩障扞太守,而為流矢所中死,太守得全。賊素聞其恩信,即殺弩中脩者,餘悉降散。言曰:「自為彭君故降,不為太守服也。」

索盧放字君陽,東郡人也。以尚書教授千餘人。初署郡門下掾。更始時,使者督行郡國,太守有事,當就斬刑。放前言曰:「今天下所以苦毒王氏,歸心皇漢者,實以聖政寬仁故也。而傳車所過,未聞恩澤。太守受誅,誠不敢言,但恐天下惶懼,各生疑變。夫使功者不如使過,願以身代太守之命。」遂前就斬。使者義而赦之,由是顯名。

建武六年,徵為洛陽令,政有能名。以病乞身,徙諫議大夫,數納忠言,後以疾去。

建武末,復徵不起,光武使人輿之,見於南宮雲臺,賜穀二千斛,遣歸,除子為太子中庶子。卒於家。

周嘉字惠文,汝南安城人也。高祖父燕,宣帝時為郡決曹掾。太守欲枉殺人,燕諫不聽,遂殺囚而黜燕。囚家守闕稱冤。詔遣覆考,燕見太守曰:「願謹定文書,皆著燕名,府君但言時病而已。」出謂掾史曰:「諸君被問,悉當以罪推燕。如有一言及於府君,燕手劍相刃。」使乃收燕繫獄。屢被掠楚,辭無屈橈。當下蠶室,乃歎曰:「我平王之後,正公玄孫,豈可以刀鋸之餘下見先君?」遂不食而死。燕有五子,皆至刺史、太守。

嘉仕郡為主簿。王莽末,群賊入汝陽城,嘉從太守何敞討賊,敞為流矢所中,郡兵奔北,賊圍繞數十重,白刃交集,嘉乃擁敞,以身扞之。因呵賊曰:「卿曹皆人隸也。為賊既逆,豈有還害其君者邪?嘉請以死贖君命。」因仰天號泣。群賊於是兩兩相視,曰:「此義士也!」給其車馬,遣送之。

後太守寇恂舉為孝廉,拜尚書侍郎。光武引見,問以遭難之事。嘉對曰:「太守被傷,命懸寇手,臣實駑怯,不能死難。」帝曰:「此長者也。」詔嘉尚公主,嘉稱病篤,不肯當。

稍遷零陵太守,視事七年,卒,零陵頌其遺愛,吏民為立祠焉。

嘉從弟暢,字伯持,性仁慈,為河南尹。永初二年,夏旱,久禱無應,暢因收葬洛城傍客死骸骨凡萬餘人,應時澍雨,歲乃豐稔。位至光祿勳。

范式字巨卿,山陽金鄉人也,一名氾。少遊太學,為諸生,與汝南張劭為友。劭字元伯。二人並告歸鄉里。式謂元伯曰:「後二年當還,將過拜尊親,見孺子焉。」乃共剋期日。後期方至,元伯具以白母,請設饌以候之。母曰:「二年之別,千里結言,爾何相信之審邪?」對曰:「巨卿信士,必不乖違。」母曰:「若然,當為爾醞酒。」至其日,巨卿果到,升堂拜飲,盡歡而別。

式仕為郡功曹。後元伯寢疾篤,同郡郅君章、殷子徵晨夜省視之。元伯臨盡,歎曰:「恨不見吾死友!」子徵曰:「吾與君章盡心於子,是非死友,復欲誰求?」元伯曰:「若二子者,吾生友耳。山陽范巨卿,所謂死友也。」尋而卒。式忽夢見元伯玄冕垂纓屣履而呼曰:「巨卿,吾以某日死,當以爾時葬,永歸黃泉。子未我忘,豈能相及?」式怳然覺寤,悲歎泣下,具告太守,請往奔喪。太守雖心不信而重違其情,許之。式便服朋友之服,投其葬日,馳往赴之。式未及到,而喪已發引,既至壙,將窆,而柩不肯進。其母撫之曰:「元伯,豈有望邪?」遂停柩移時,乃見有素車白馬,號哭而來。其母望之曰:「是必范巨卿也。」巨卿既至,叩喪言曰:「行矣元伯!死生路異,永從此辭。」會葬者千人,咸為揮涕。式因執紼而引,柩於是乃前。式遂留止冢次,為脩墳樹,然後乃去。

後到京師,受業太學。時諸生長沙陳平子亦同在學,與式未相見,而平子被病將亡,謂其妻曰:「吾聞山陽范巨卿,烈士也,可以託死。吾歿後,但以屍埋巨卿戶前。」乃裂素為書,以遺巨卿。既終,妻從其言。時式出行適還,省書見瘞,愴然感之,向墳揖哭,以為死友。乃營護平子妻兒,身自送喪於臨湘。未至四五里,乃委素書於柩上,哭別而去。其兄弟聞之,尋求不復見。長沙上計掾史到京師,上書表式行狀,三府並辟,不應。

舉州茂才,四遷荊州刺史。友人南陽孔嵩,家貧親老,乃變名姓,傭為新野縣阿里街卒。式行部到新野,而縣選嵩為導騎迎式。式見而識之,呼嵩,把臂謂曰:「子非孔仲山邪?」對之歎息,語及平生。曰:「昔與子俱曳長裾,遊集帝學,吾蒙國恩,致位牧伯,而子懷道隱身,處於卒伍,不亦惜乎!」嵩曰:「侯嬴長守於賤業,晨門肆志於抱關。子欲居九夷,不患其陋。貧者士之宜,豈為鄙哉!」式敕縣代嵩,嵩以為先傭未竟,不肯去。

嵩在阿里,正身厲行,街中子弟皆服其訓化。遂辟公府。之京師,道宿下亭,盜共竊其馬,尋問知其嵩也,乃相責讓曰:「孔仲山善士,豈宜侵盜乎!」於是送馬謝之。嵩官至南海太守。

式後遷廬江太守,有威名,卒於官。

李善字次孫,南陽淯陽人,本同縣李元蒼頭也。建武中疫疾,元家相繼死沒,唯孤兒續始生數旬,而貲財千萬,諸奴婢私共計議,欲謀殺續,分其財產。善深傷李氏而力不能制,乃潛負續逃去,隱山陽瑕丘界中,親自哺養,乳為生湩,推燥居溼,備嘗艱勤。續雖在孩抱,奉之不異長君,有事輒長跪請白,然後行之。閭里感其行,皆相率脩義。續年十歲,善與歸本縣,脩理舊業。告奴婢於長吏,悉收殺之。時鍾離意為瑕丘令,上書薦善行狀。光武詔拜善及續並為太子舍人。

善,顯宗時辟公府,以能理劇,再遷日南太守。從京師之官,道經淯陽,過李元冢。未至一里,乃脫朝服,持鉏去草。及拜墓,哭泣甚悲,身自炊爨,埶鼎俎以脩祭祀。垂泣曰:「君夫人,善在此。」盡哀,數日乃去。到官,以愛惠為政,懷來異俗。遷九江太守,未至,道病卒。

續至河閒相。

王忳字少林,廣漢新都人也。忳嘗詣京師,於空舍中見一書生疾困,龟而視之。書生謂忳曰:「我當到洛陽,而被病,命在須臾,腰下有金十斤,願以相贈,死後乞藏骸骨。」未及問姓名而絕。忳即鬻金一斤,營其殯葬,餘金悉置棺下,人無知者。後歸數年,縣署忳大度亭長。初到之日,有馬馳入亭中而止。其日,大風飄一繡被,復墯忳前,即言之於縣,縣以歸忳。忳後乘馬到雒縣,馬遂奔走,牽忳入它舍。主人見之喜曰:「今禽盜矣。」問忳所由得馬,忳具說其狀,并及繡被。主人悵然良久,乃曰:「被隨旋風與馬俱亡,卿何陰德而致此二物?」忳自念有葬書生事,因說之,并道書生形貌及埋金處。主人大驚號曰:「是我子也。姓金名彥。前往京師,不知所在,何意卿乃葬之。大恩久不報,天以此章卿德耳。」忳悉以被馬還之,彥父不取,又厚遺忳,忳辭讓而去。時彥父為州從事,因告新都令,假忳休,自與俱迎彥喪,餘金俱存。忳由是顯名。

仕郡功曹,州治中從事。舉茂才,除郿令。到官,至斄亭。亭長曰:「亭有鬼,數殺過客,不可宿也。」忳曰:「仁勝凶邪,德除不祥,何鬼之避!」即入亭止宿。夜中聞有女子稱冤之聲。忳沟曰:「有何枉狀,可前求理乎?」女子曰:「無衣,不敢進。」忳便投衣與之。女子乃前訴曰:「妾夫為涪令,之官過宿此亭,亭長無狀,賊殺妾家十餘口,埋在樓下,悉取財貨。」忳問亭長姓名。女子曰:「即今門下游徼者也。」忳曰:「汝何故數殺過客?」對曰:「妾不得白日自訴,每夜陳冤,客輒眠不見應,不勝感恚,故殺之。」忳曰:「當為汝理此冤,勿復殺良善也。」因解衣於地,忽然不見。明旦召游徼詰問,具服罪,即收繫,及同謀十餘人悉伏辜,遣吏送其喪歸鄉里,於是亭遂清安。

張武者,吳郡由拳人也。父業,郡門下掾,送太守妻子還鄉里,至河內亭,盜夜劫之,業與賊戰死,遂亡屍。武時年幼,不及識父。後之太學受業,每節,常持父遺劍,至亡處祭醊,而還。太守第五倫嘉其行,舉孝廉。遭母喪過毀,傷父魂靈不返,因哀慟絕命。

陵續字智初,會稽吳人也。世為族姓。祖父閎,字子春,建武中為尚書令。美姿貌,喜著越布單衣,光武見而好之,自是常敕會稽郡獻越布。

續幼孤,仕郡戶曹史。時歲荒民飢,太守尹興使續於都亭賦民饘粥。續悉簡閱其民,訊以名氏。事畢,興問所食幾何?續因口說六百餘人,皆分別姓字,無有差謬。興異之,刺史行部,見續,辟為別駕從事。以病去,還為郡門下掾。

是時楚王英謀反,陰疏天下善士,及楚事覺,顯宗得其錄,有尹興名,乃徵興詣廷尉獄。續與主簿梁宏、功曹史駟勳及掾史五百餘人詣洛陽詔獄就考,諸吏不堪痛楚,死者大半,唯續、宏、勳掠考五毒,肌肉消爛,終無異辭。續母遠至京師,覘候消息,獄事特急,無緣與續相聞,母但作饋食,付門卒以進之。續雖見考苦毒,而辭色慷慨,未嘗易容,唯對食悲泣,不能自勝。使者怪而問其故。續曰:「母來不得相見,故泣耳。」使者大怒,以為門卒通傳意氣,召將案之。續曰:「因食餉羹,識母所自調和,故知來耳,非人告也。」使者問:「何以知母所作乎?」續曰:「母嘗截肉未嘗不方,斷蔥以寸為度,是以知之。」使者問諸謁舍,續母果來,於是陰嘉之,上書說續行狀。帝即赦興等事,還鄉里,禁錮終身。續以老病卒。

長子稠,廣陵太守,有理名。中子逢,樂安太守。少子褒,力行好學,不慕榮名,連徵不就。褒子康,已見前傳。

戴封字平仲,濟北剛人也。年十五,詣太學,師事鄮令東海申君。申君卒,送喪到東海,道當經其家。父母以封當還,豫為娶妻。封暫過拜親,不宿而去。還京師卒業。時同學石敬平溫病卒,封養視殯斂,以所齎糧巿小棺,送喪到家。家更斂,見敬平行時書物皆在棺中,乃大異之。封後遇賊,財物悉被略奪,唯餘縑七匹,賊不知處,封乃追以與之,曰:「知諸君乏,故送相遺。」賊驚曰:「此賢人也。」盡還其器物。

後舉孝廉,光祿主事,遭伯父喪去官。詔書求賢良方正直言之士,有至行能消災伏異者,公卿郡守各舉一人。郡及大司農俱舉封。公車徵,陛見,對策第一,擢拜議郎。遷西華令。時汝、潁有蝗災,獨不入西華界。時督郵行縣,蝗忽大至,督郵其日即去,蝗亦頓除,一境奇之。其年大旱,封禱請無獲,乃積薪坐其上以自焚。火起而大雨暴至,於是遠近歎服。

遷中山相。時諸縣囚四百餘人,辭狀已定,當行刑。封哀之,皆遣歸家,與剋期日,皆無違者。詔書策美焉。

永元十二年,徵拜太常,卒官。

李充字大遜,陳留人也。家貧,兄弟六人同食遞衣。妻竊謂充曰:「今貧居如此,難以久安,妾有私財,願思分異。」充偽酬之曰:「如欲別居,當醞酒具會,請呼鄉里內外,共議其事。」婦從充置酒讌客。充於坐中前跪白母曰:「此婦無狀,而教充離閒母兄,罪合遣斥。」便呵叱其婦,逐令出門,婦銜涕而去。坐中驚肅,因遂罷散。充後遭母喪,行服墓次,人有盜其墓樹者,充手自殺之。服闋,立精舍講授。

太守魯平請署功曹,不就。平怒,乃援充以捐溝中,因謫署縣都亭長。不得已,起親職役。後和帝公車徵,不行。延平中,詔公卿、中二千石各舉隱士大儒,務取高行,以勸後進,特徵充為博士。時魯平亦為博士,每與集會,常歎服焉。

充遷侍中。大將軍鄧騭貴戚傾時,無所下借,以充高節,每卑敬之。嘗置酒請充,賓客滿堂,酒酣,騭跪曰:「幸託椒房,位列上將,幕府初開,欲辟天下奇偉,以匡不逮,惟諸君博求其器。」充乃為陳海內隱居懷道之士,頗有不合。騭欲絕其說,以肉啖之。充抵肉於地,曰:「說士猶甘於肉!」遂出,徑去。騭甚望之。同坐汝南張孟舉往讓充曰:「一日聞足下與鄧將軍說士未究,激刺面折,不由中和,出言之責,非所以光祚子孫者也。」充曰:「大丈夫居世,貴行其意,何能遠為子孫計哉!」由是見非於貴戚。

遷左中郎將,年八十八,為國三老。安帝常特進見,賜以几杖。卒於家。

繆肜字豫公,汝南召陵人也。少孤,兄弟四人,皆同財業。及各娶妻,諸婦遂求分異,又數有鬥爭之言。肜深懷憤歎,乃掩戶自撾曰:「繆肜,汝脩身謹行,學聖人之法,將以齊整風俗,柰何不能正其家乎!」弟及諸婦聞之,悉叩頭謝罪,遂更為敦睦之行。

仕縣為主簿。時縣令被章見考,吏皆畏懼自誣,而肜獨證據其事,掠考苦毒,至乃體生蟲蛆,因復傳換五獄,踰涉四年,令卒以自免。

太守隴西梁湛召為決曹史。安帝初,湛病卒官,肜送喪還隴西。始葬,會西羌反叛,湛妻子悉避亂它郡,肜獨留不去,為起墳冢,乃潛穿井旁以為窟室,晝則隱竄,夜則負土,及賊平而墳已立。其妻子意肜已死,還見大驚。關西咸稱傳之,共給車馬衣資,肜不受而歸鄉里。

辟公府,舉尤異,遷中牟令。縣近京師,多權豪,肜到,誅諸姦吏及託名貴戚賓客者百有餘人,威名遂行。卒於官。

陳重字景公,豫章宜春人也。少與同郡雷義為友,俱學魯詩、顏氏春秋。太守張雲舉重孝廉,重以讓義,前後十餘通記,雲不聽。義明年舉孝廉,重與俱在郎署。

有同署郎負息錢數十萬,責主日至,詭求無已,重乃密以錢代還。郎後覺知而厚辭謝之。重曰:「非我之為,將有同姓名者。」終不言惠。又同舍郎有告歸寧者,誤持鄰舍郎恊以去。主疑重所取,重不自申說,而巿恊以償之。後寧喪者歸,以恊還主,其事乃顯。

重後與義俱拜尚書郎,義代同時人受罪,以此黜退,重見義去,亦以病免。

後舉茂才,除細陽令。政有異化,舉尤異,當遷為會稽太守,遭姊憂去官。後為司徒所辟,拜侍御史,卒。

雷義字仲公,豫章鄱陽人也。初為郡功曹,皆擢舉善人,不伐其功。義嘗濟人死罪,罪者後以金二斤謝之,義不受,金主伺義不在,默投金於承塵上。後葺理屋宇,乃得之,金主已死,無所復還,義乃以付縣曹。

後舉孝廉,拜尚書侍郎,有同時郎坐事當居刑作,義默自表取其罪,以此論司寇。同臺郎覺之,委位自上,乞贖義罪。順帝詔皆除刑。

義歸,舉茂才,讓於陳重,刺史不聽,義遂陽狂被髮走,不應命。鄉里為之語曰:「膠漆自謂堅,不如雷與陳。」三府同時俱辟二人。義遂為守灌謁者。使持節督郡國行風俗,太守令長坐者凡七十人。旋拜侍御史,除南頓令,卒官。

子授,官至蒼梧太守。

范冉字史雲,陳留外黃人也。少為縣小吏,年十八,奉檄迎督郵,冉恥之,乃遁去。到南陽,受業於樊英。又遊三輔,就馬融通經,歷年乃還。

冉好違時絕俗,為激詭之行。常慕梁伯鸞、閔仲叔之為人。與漢中李固、河內王奐親善,而鄙賈偉節、郭林宗焉。奐後為考城令,境接外黃,屢遣書請冉,冉不至。及奐遷漢陽太守,將行,冉乃與弟協步齎麥酒,於道側設壇以待之。冉見奐車徒駱驛,遂不自聞,惟與弟共辯論於路。奐識其聲,即下車與相揖對。奐曰:「行路倉卒,非陳闊之所,可共到前亭宿息,以敘分隔。」冉曰:「子前在考城,思欲相從,以賤質自絕豪友耳。今子遠適千里,會面無期,故輕行相候,以展訣別。如其相追,將有慕貴之譏矣。」便起告違,拂衣而去。奐瞻望弗及,冉長逝不顧。

桓帝時,以冉為萊蕪長,遭母憂,不到官。後辟太尉府,以狷急不能從俗,常佩韋於朝。議者欲以為侍御史,因遁身逃命於梁沛之閒,徒行敝服,賣卜於巿。

遭黨人禁錮,遂推鹿車,載妻子,捃拾自資,或寓息客廬,或依宿樹蔭。如此十餘年,乃結草室而居焉。所止單陋,有時糧粒盡,窮居自若,言貌無改,閭里歌之曰:「甑中生塵范史雲,釜中生魚范萊蕪。」

及黨禁解,為三府所辟,乃應司空命。是時西羌反叛,黃巾作難,制諸府掾屬不得妄有去就。冉首自劾退,詔書特原不理罪。又辟太尉府,以疾不行。

中平二年,年七十四,卒於家。臨命遺令刔其子曰:「吾生於昏闇之世,值乎淫侈之俗,生不得匡世濟時,死何忍自同於世!氣絕便斂,斂以時服,衣足蔽形,棺足周身,斂畢便穿,穿畢便埋。其明堂之奠,干飯寒水,飲食之物,勿有所下。墳封高下,令足自隱。知我心者李子堅、王子炳也。今皆不在,制之在爾,勿令鄉人宗親有所加也。」於是三府各遣令史奔弔。大將軍何進移書陳留太守,累行論謚,僉曰宜為貞節先生。會葬者二千餘人,刺史郡守各為立碑表墓焉。

戴就字景成,會稽上虞人也。仕郡倉曹掾,楊州刺史歐陽參奏太守成公浮臧罪,遣部從事薛安案倉庫簿領,收就於錢唐縣獄。幽囚考掠,五毒參至。就慷慨直辭,色不變容。又燒鋘斧,使就挾於肘腋。就語獄卒:「可熟燒斧,勿令冷。」每上彭考,因止飯食不肯下,肉焦毀墯地者,掇而食之。主者窮竭酷慘,無復餘方,乃臥就覆船下,以馬通薰之。一夜二日,皆謂已死,發船視之,就方張眼大罵曰:「何不益火,而使滅絕!」又復燒地,以大鍼刺指爪中,使以把土,爪悉墯落。主者以狀白安,安呼見就,謂曰:「太守罪穢狼藉,受命考實,君何故以骨肉拒扞邪?」就據地荅言:「太守剖符大臣,當以死報國。卿雖銜命,固宜申斷冤毒,柰何誣枉忠良,強相掠理,令臣謗其君,子證其父!薛安庸騃,忸行無義,就考死之日,當白之於天,與群鬼殺汝於亭中。如蒙生全,當手刃相裂!」安深奇其壯節,即解械,更與美談,表其言辭,解釋郡事。徵浮還京師,免歸鄉里。

太守劉寵舉就孝廉,光祿主事,病卒。

趙苞字威豪,甘陵東武城人。從兄忠,為中常侍,苞深恥其門族有宦官名埶,不與忠交通。

初仕州郡,舉孝廉,再遷廣陵令。視事三年,政教清明,郡表其狀,遷遼西太守。抗厲威嚴,名振邊俗。以到官明年,遣使迎母及妻子,垂當到郡,道經柳城,值鮮卑萬餘人入塞寇鈔,苞母及妻子遂為所劫質,載以擊郡。苞率步騎二萬,與賊對陣。賊出母以示苞,苞悲號謂母曰:「為子無狀,欲以微祿奉養朝夕,不圖為母作禍。昔為母子,今為王臣,義不得顧私恩,毀忠節,唯當萬死,無以塞罪。」母遙謂曰:「威豪,人各有命,何得相顧,以虧忠義!昔王陵母對漢使伏劍,以固其志,爾其勉之。」苞即時進戰,賊悉摧破,其母妻皆為所害。苞殯斂母畢,自上歸葬。靈帝遣策弔慰,封鄃侯。

苞葬訖,謂鄉人曰:「食祿而避難,非忠也;殺母以全義,非孝也。如是,有何面目立於天下!」遂歐血而死。

向栩字甫興,河內朝歌人,向長之後也。少為書生,性卓詭不倫。恆讀老子,狀如學道。又似狂生,好被髮,著絳綃頭。常於灶北坐板床上,如是積久,板乃有膝踝足指之處。不好語言而喜長嘯。賓客從就,輒伏而不視。有弟子,名為「顏淵」、「子貢」、「季路」、「冉有」之輩。或騎驢入市,乞饨於人。或悉要諸乞兒俱歸止宿,為設酒食。時人莫能測之。郡禮請辟,舉孝廉、賢良方正、有道,公府辟,皆不到。又與彭城姜肱、京兆韋著並徵,栩不應。

後特徵,到,拜趙相。及之官,時人謂其必當脫素從儉,而栩更乘鮮車,御良馬,世疑其始偽。及到官,略不視文書,舍中生蒿萊。

徵拜侍中,每朝廷大事,侃然正色,百官憚之。會張角作亂,栩上便宜,頗譏刺左右,不欲國家興兵,但遣將於河上北向讀孝經,賊自當消滅。中常侍張讓讒栩不欲令國家命將出師,疑與角同心,欲為內應。收送黃門北寺獄,殺之。

諒輔字漢儒,廣漢新都人也。仕郡為五官掾。時夏大旱,太守自出祈禱山川,連日而無所降。輔乃自暴庭中,慷慨沟曰:「輔為股肱,不能進諫納忠,薦賢退惡,和調陰陽,承順天意,至令天地否隔,萬物焦枯,百姓喁喁,無所訴告,咎盡在輔。今郡太守改服責己,為民祈福,精誠懇到,未有感徹。輔今敢自祈請,若至中不雨,乞以身塞無狀。」於是積薪柴聚茭茅以自環,搆火其傍,將自焚焉。未及日中時,而天雲晦合,須臾澍雨,一郡沾潤。世以此稱其至誠。

劉翊字子相,潁川潁陰人也。家世豐產,常能周施而不有其惠。曾行於汝南界中,有陳國張季禮遠赴師喪,遇寒冰車毀,頓滯道路。翊見而謂曰:「君慎終赴義,行宜速達。」即下車與之,不告姓名,自策馬而去。季禮意其子相也,後故到潁陰,還所假乘。翊閉門辭行,不與相見。

常守志臥疾,不屈聘命。河南种拂臨郡,引為功曹,翊以拂名公之子,乃為起焉。拂以其擇時而仕,甚敬任之。陽翟黃綱恃程夫人權力,求占山澤以自營植。拂召翊問曰:「程氏貴盛,在帝左右,不聽則恐見怨,與之則奪民利,為之柰何?」翊曰:「名山大澤不以封,蓋為民也。明府聽之,則被佞倖之名矣。若以此獲禍,貴子申甫,則自以不孤也。」拂從翊言,遂不與之。乃舉翊為孝廉,不就。

後黃巾賊起,郡縣飢荒,翊救給乏絕,資其食者數百中。鄉族貧者,死亡則為具殯葬,嫠獨則助營妻娶。

獻帝遷都西京,翊舉上計掾。是時寇賊興起,道路隔絕,使驛稀有達者。翊夜行晝伏,乃到長安。詔書嘉其忠勤,特拜議郎,遷陳留太守。翊散所握珍玩,唯餘車馬,自載東歸。出關數百里,見士大夫病亡道次,翊以馬易棺,脫衣斂之。又逢知故困餒於路,不忍委去,因殺所駕牛,以救其乏。眾人止之,翊曰:「視沒不救,非志士也。」遂俱餓死。

王烈字彥方,太原人也。少師事陳寔,以義行稱。鄉里有盜牛者,主得之,盜請罪曰:「刑戮是甘,乞不使王彥方知也。」烈聞而使人謝之,遺布一端。或問其故,烈曰:「盜懼吾聞其過,是有恥惡之心。既懷恥惡,必能改善,故以此激之。」後有老父遺劍於路,行道一人見而守之,至暮,老父還,尋得劍,怪而問其姓名,以事告烈。烈使推求,乃先盜牛者也。諸有爭訟曲直,將質之於烈,或至塗而反,或望廬而還。其以德感人若此。

察孝廉,三府並辟,皆不就。遭黃巾、董卓之亂,乃避地遼東,夷人尊奉之。太守公孫度接以昆弟之禮,訪酬政事。欲以為長史,烈乃為商賈自穢,得免。曹操聞烈高名,遣徵不至。建安二十四年,終於遼東,年七十八。

贊曰:乘方不忒,臨義罔惑。惟此剛絜,果行育德。


\end{pinyinscope}