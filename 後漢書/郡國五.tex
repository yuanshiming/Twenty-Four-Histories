\article{郡國五}

\begin{pinyinscope}
右幽州

南海蒼梧鬱林合浦交趾

九真日南

右交州

漢中郡

九城,戶五萬七千三百四十四,口二十六萬七千四百二。

南鄭

成固

媯

墟在西北。西城褒中沔陽有鐵。安陽錫有錫,春秋時曰錫穴。上庸本庸國。房陵

巴郡

十四城,戶三十一萬六百九十一,口百八萬六千四十九。

江州

宕渠

有

鐵。朐忍閬中

魚復

扞水有扦關。臨江枳

涪陵

出丹墊江

安漢

平都

充

國

永元二年分閬中置。宣漢

漢昌

永元中置。

廣漢郡

十一城,戶十三萬九千八百六十五,口五十萬九千四百三十八。

雒

州刺史治。新都綿竹什邡涪梓潼

白水

葭萌

郪

廣漢

有沈水

。德陽

蜀郡

十一城,戶三十萬四百五十二,口百三十五萬四百七十六。

成都

郫

江原

繁

廣

都

臨邛

有鐵。湔氐道岷山在西徼外。汶江道八陵廣柔

綿虒道

犍為郡

九城,戶十三萬七千七百一十三,口四十一萬一千三百七十八。

武陽

有彭亡聚。資中牛鞞南安

有魚泣津。僰道

江陽

荷

節

南廣

漢安

牂牁郡

十六城,戶三萬一千五百二十三,口二十六萬七千二百五十三。

故且蘭

平

夷

鄨

毋

斂

談指

出丹。夜郎出雄黃、雌黃。同並

談稿

漏江

毋

單

宛溫

鐔

封

漏臥

句

町

進乘

西

隨

越巂郡

十四城,戶十三萬一百二十,口六十二萬三千四百一十八。

邛都

南山出銅。遂久靈關道臺登出鐵。青蛉有禺同山,俗謂有金馬碧雞。卑水

三縫

會

無

出鐵。定莋

闡

蘇谀

大莋

莋秦

姑復

益州郡

十

七城,戶二萬九千三十六,口十一萬八百二。

滇池

出鐵。有池澤。北有黑水祠。勝休俞元裝山出銅。律高石室山出錫。抛町山出銀、鉛。賁古采山出銅、錫。羊山出銀、鉛。母掇建伶穀昌牧靡味

昆澤

同瀨

同

勞

雙柏

出銀。連然

梇棟

秦臧

永昌郡

八城,戶二十三萬一千八百九十七,口百八十九萬七千三百四十四。

不韋

出鐵。巂唐比蘇楪榆邪龍雲南哀牢永平中置,故牢王國。博南永平中置。南界出金。

廣漢屬國

戶三萬七千一百一十,口二十萬五千六百五十二。

陰平道

甸氐道

剛氐道

蜀郡屬國

戶十一萬

一千五百六十八,口四十七萬五千六百二十九。

漢嘉

故青衣,陽嘉二年改。有蒙山。嚴道有邛僰九折阪者,邛刻置。徙

旄牛

犍為屬國

戶七千九

百三十八,口三萬七千一百八十七。

朱提

山出銀、銅。漢陽

右益州刺史部,郡、國十二,縣、道百一十八。

隴西郡

十一城,戶五千六百二十八,口二萬九千六百三十七。

狄道

安故

氐

道

養水出此。首陽有鳥鼠同穴山,渭水出。大夏

襄武

有五雞聚。臨洮

有西頃山。枹罕故屬金城。白石故屬金城。鄣

河關

故屬金城。積石山在西南,河水出。

漢陽郡

十

三城,戶二萬七千四百二十三,口十三萬一百三十八。

冀

有朱圄山。有緹群山。有雒門聚。望恒

阿陽

略陽

有街泉亭

。勇士成紀

隴

州刺史治。有大阪名隴坻。豲坻聚有秦亭。豲道蘭干平襄顯親上邽故屬隴西。西故屬隴西。有嶓冢山,西漢水。

武都郡

七城,戶二萬一百二,口八萬一千七百二十八。

下辨

武都

道

上祿

故

道

河池

沮

沔水出東狼谷。羌道

金城郡

十城,戶三千八百五十八,口萬八千九百四十七。

允吾

浩亹

令居

枝

陽

金城

榆中

臨羌

有昆崙山。破羌

安夷

允街

安定郡

八城,戶六千九十四,口二萬九千六十。

臨涇

高平

有第一城。朝那

烏枝

有瓦亭,出薄落谷。三水陰盤

彭陽

鶉觚

故

屬北地。

北地郡

六城,戶三千一百二十二,口萬八千六百三十七。

富平

泥陽

有

五柞亭。弋居有鐵。廉

參讀

故屬安定。靈州

武威郡

十四城,戶萬四十二,口三萬四千二百二十六。

姑臧

張掖

武威

休屠

揟次

鸞鳥

樸

峦

媼圍

宣威

倉松

鸇陰

故

屬安定。租厲故屬安定。顯美故屬張掖。左騎千人官。

張掖郡

八

城,戶六千五百五十二,口二萬六千四十。

觻得

昭武

刪丹

弱水出。氐池

屋蘭

日勒

驪

靬

番和

酒泉郡

九城,戶萬二千七百六。

福祿

表氏

樂涫

玉門

會

水

沙頭

安彌

故

曰緩彌。乾齊延壽

敦煌郡

六城,戶七百四十八,口二萬九千一百七十。

敦煌

古瓜州,出美瓜。冥安效穀拼泉

廣至

龍勒

有玉門

關。

張掖屬國

戶四千

六百五十六,口萬六千九百五十二。

候官

左騎

千人

司馬官

千人官。

張掖居延屬國

戶一千五百六十,口四千七百三十三。

居延

有居延澤,古流沙。

右涼州刺史部,郡國十二,縣、道、候官九十八。

上黨郡

十三城,戶二萬六千二百二十二,口十二萬七千四百三。

長子

屯留

絳水出。銅鞮

沾

涅

有閼

與聚。襄垣

壺關

有黎亭,故黎國。泫氏有長平亭。高都潞本國。猗氏陽阿侯國。穀遠

太原郡

十

六城,戶三萬九百二,口二十萬一百二十四。

晉陽

本唐國。有龍山,晉水所出。刺史治。界休有界山,有綿上聚。有千畝聚。榆次有鑿壺。中都

于離

茲氏

狼孟鄔

盂

平陶

京陵

春

秋時九京。陽曲大陵有鐵。祁慮虒陽邑有箕城。

上郡

十

城,戶五千一百六十九,口二萬八千五百九十九。

膚施

白土

漆垣

奢延

雕

陰

楨林

定陽

高奴

龜茲屬國

候官

西河郡

十三城,戶五千六百九十八,口二萬八百三十八。

離石

平定

美稷

樂街

中

陽

皋狼

平周

平陸

益蘭圜陰

藺

圜陽

廣

衍

五原郡

十城,戶四千六百六十七,口二萬二千九百五十七。

九原

五原

臨沃

父國

河

除武都

宜梁

曼柏

成宜

西安陽

北有陰

山。

雲中郡

十

一城,戶五千三百五十一,口二萬六千四百三十。

雲中

咸陽

箕

陵

沙陵

沙南

北輿

武泉

原

陽

定襄

故屬定襄。成樂故屬定襄。武進故屬定襄。

定襄郡

五

城,戶三千一百五十三,口萬三千五百七十一。

善無

故屬鴈門。桐過武成駱中陵故屬鴈門。

鴈門郡

十四城,戶三萬一千八百六十二,口二十四萬九千。

陰館

繁畤

樓煩

武州

汪陶

劇陽

崞

平城

埒

馬邑

鹵城

故屬

代郡。廣武故屬太原。有夏屋山。原平故屬太原。彊陰

朔方郡

六

城,戶千九百八十七,口七千八百四十三。

臨戎

三封

朔

方

沃野

廣牧

大城

故屬西河。

右并州刺史部,郡九,縣、邑、侯國九十八。

涿郡

七城,戶十萬二千二百一十八,口六十三萬三千七百五十四。

涿

迺

侯國。故安易水出,雹水出。范陽侯國。良鄉

北

新城

有汾水門。方城故屬廣陽。有臨鄉。有督亭。

廣陽郡

世祖省并上谷,永平

八年復。五城,戶四萬四千五百五十,口二十八萬六百。

薊

本燕國。刺史治。廣陽昌平故屬上谷。軍都故屬上谷。安次故屬勃海。

代郡

十一城,戶二萬一百二十三,口十二萬六千一百八十八。

高柳

桑乾

道

人

當城

馬城

班氏

狋氏

北

平邑

永元八年復。東安陽平舒代

上谷郡

八城,戶萬三百五十二,口五萬一千二百四。

沮陽

潘永元十一年復。甯

廣甯

居庸

雊

瞀

涿鹿

下落

漁陽郡

九城,戶六萬八千四百五十六,口四十三萬五千七百四十。

漁陽

有鐵。狐奴

潞

雍奴

泉

州

有鐵。平谷安樂傂奚獷平

右北平郡

四城,戶九千一百七十,口五萬三千四百七十五。

土垠

徐無

俊靡

無終

遼西郡

五城,戶萬四千一百五十,口八萬一千七百一十四。

陽樂

海陽

令

支

有孤竹城。肥如臨渝

遼東郡

十一城,戶六萬四千一百五十八,口八萬一千七百一十四。

襄平

新昌

無

慮

望平

候城

安巿

平郭

有鐵。西安平汶

番汗

沓氏

玄菟郡

六城,戶一千五百九十四,口四萬三千一百六十三。

高句驪

遼山,遼水出。西蓋鳥上殷台

高顯

故屬遼東。候城

故屬遼東。遼陽故屬遼東。

樂浪郡

十八城,戶六萬一千四百九十二,口二十五萬七千五十。

朝鮮

烫邯

浿

水

含資

占蟬

遂城

增地

帶

方

駟望海冥

列口

長岑

屯有

昭

明

鏤方

提奚

渾彌

樂都

遼東屬國

雒陽東

北三千二百六十里。

昌遼

故天遼,屬遼西。賓徒故屬遼西。徒河故屬遼西。無慮有醫無慮山。險瀆房

右幽州刺史部,郡、國十一,縣、邑、侯國九十。南海郡

七城,戶七萬一千四百七十七,口二十五萬二百八十二。

番禺

博羅

中宿

龍川

四

會

揭陽

增城

有

勞領山。

蒼梧郡

十一城,戶十一萬一千三百九十五,口四十六萬六千九百七十五。

廣信

謝沐

高要

封陽

臨

賀

端谿

馮乘

富川

荔浦

猛

陵

鄣平

鬱林郡

十一城。

布山

安廣

阿

林

廣鬱

中溜

桂林

潭中

臨

塵

定周增食

領方

合浦郡

五城,戶二萬三千一百二十一,口八萬六千六百一十七。

合浦

徐聞

高涼

臨元

朱崖

交趾郡

十二城。

龍編

羸啮

定安

〈定〉苟

漏

麊泠

曲陽

北帶稽徐

西于

朱觏

封谿

建武十九年

置。望海建武十九年置。

九真郡

五城,戶四萬六千五百一十三,口二十萬九千八百九十四。

胥浦

居風

咸懽

無功

無

編

日南郡

五城,戶萬八千二百六十三,口十萬六百七十六。西卷朱吾

盧容

象林

比景

右交州刺史部,郡七,縣五十六。

漢書地理志承秦三十六郡,縣邑數百,後稍分析,至于孝平,凡郡、國百三,縣、邑、道、侯國千五百八十七。世祖中興,惟官多役煩,乃命并合,省郡、國十,縣、邑、道、侯國四百餘所。至明帝置郡一,章帝置郡、國二,和帝置三,安帝又命屬國別領比郡者六,又所省縣漸復分置,至于孝順,凡郡、國百五,縣、邑、道、侯國千一百八十,民戶九百六十九萬八千六百三十,口四千九百一十五萬二百二十。

贊曰:眾安后載,政洽區分;侯罷守列,民無常君。稱號遷隔,封割糾紛;略存減益,多證前聞。


\end{pinyinscope}