\article{馬援列傳}

\begin{pinyinscope}
馬援字文淵,扶風茂陵人也。其先趙奢為趙將,號曰馬服君,子孫因為氏。武帝時,以吏二千石自邯鄲徙焉。曾祖父通,以功封重合侯,坐兄何羅反,被誅,故援再世不顯。援三兄況、余、員,並有才能,王莽時皆為二千石。

援年十二而孤,少有大志,諸兄奇之。嘗受齊詩,意不能守章句,乃辭況,欲就邊郡田牧。況曰:「汝大才,當晚成。良工不示人以朴,且從所好。」會況卒,援行服期年,不離墓所;敬事寡桦,不冠不入廬。後為郡督郵,送囚至司命府,囚有重罪,援哀而縱之,遂亡命北地。遇赦,因留牧畜,賓客多歸附者,遂役屬數百家。轉游隴漢閒,常謂賓客曰:「丈夫為志,窮當益堅,老當益壯。」因處田牧,至有牛馬羊數千頭,穀數萬斛。既而歎曰:「凡殖貨財產,貴其能施賑也,否則守錢虜耳。」乃盡散以班昆弟故舊,身衣羊裘皮恊。

王莽末,四方兵起,莽從弟衛將軍林廣招雄俊,乃辟援及同縣原涉為掾,薦之於莽。莽以涉為鎮戎大尹,援為新成大尹。及莽敗,援兄員時為增山連率,與援俱去郡,復避地涼州。世祖即位,員先詣洛陽,帝遣員復郡,卒於官。援因留西州,隗囂甚敬重之,以援為綏德將軍,與決籌策。

是時公孫述稱帝於蜀,囂使援往觀之。援素與述同里閈,相善,以為既至當握手歡如平生,而述盛陳陛衛,以延援入,交拜禮畢,使出就館,更為援制都布單衣、交讓冠,會百官於宗廟中,立舊交之位。述鸞旗旄騎,警蹕就車,磬折而入,禮饗官屬甚盛,欲授援以封侯大將軍位。賓客皆樂留,援曉之曰:「天下雄雌未定,公孫不吐哺走迎國士,與圖成敗,反修飾邊幅,如偶人形。此子何足久稽天下士乎?」因辭歸,謂囂曰:「子陽井底蛙耳,而妄自尊大,不如專意東方。」

建武四年冬,囂使援奉書洛陽。援至,引見於宣德殿。世祖迎笑謂援曰:「卿遨遊二帝閒,今見卿,使人大慚。」援頓首辭謝,因曰:「當今之世,非獨君擇臣也,臣亦擇君矣。臣與公孫述同縣,少相善。臣前至蜀,述陛戟而後進臣。臣今遠來,陛下何知非刺客姦人,而簡易若是?」帝復笑曰:「卿非刺客,顧說客耳。」援曰:「天下反覆,盜名字者不可勝數。今見陛下,恢廓大度,同符高祖,乃知帝王自有真也。」帝甚壯之。援從南幸黎丘,轉至東海。及還,以為待詔,使太中大夫來歙持節送援西歸隴右。

隗囂與援共臥起,問以東方流言及京師得失。援說囂曰:「前到朝廷,上引見數十,每接讌語,自夕至旦,才明勇略,非人敵也。且開心見誠,無所隱伏,闊達多大節,略與高帝同。經學博覽,政事文辯,前世無比。」囂曰:「卿謂何如高帝?」援曰:「不如也。高帝無可無不可;今上好吏事,動如節度,又不喜飲酒。」囂意不懌,曰:「如卿言,反復勝邪?」然雅信援,故遂遣長子恂入質。援因將家屬隨恂歸洛陽。居數月而無它職任。援以三輔地曠土沃,而所將賓客猥多,乃上書求屯田上林苑中,帝許之。

會隗囂用王元計,意更狐疑,援數以書記責譬於囂。囂怨援背己,得書增怒,其後遂發兵拒漢。援乃上疏曰:「臣援自念歸身聖朝,奉事陛下,本無公輔一言之薦,左右為容之助。臣不自陳,陛下何因聞之。夫居前不能令人輊,居後不能令人軒,與人怨不能為人患,臣所恥也。故敢觸冒罪忌,昧死陳誠。臣與隗囂,本實交友。初,囂遣臣東,謂臣曰:『本欲為漢,願足下往觀之。於汝意可,即專心矣。』及臣還反,報以赤心,實欲導之於善,非敢譎以非義。而囂自挾姦心,盜憎主人,怨毒之情遂歸於臣。臣欲不言,則無以上聞。願聽詣行在所,極陳滅囂之術,得空匈腹,申愚策,退就隴畝,死無所恨。」帝乃召援計事,援具言謀畫。因使援將突騎五千,往來游說囂將高峻、任禹之屬,下及羌豪,為陳禍福,以離囂友黨。

援又為書與囂將楊廣,使曉勸於囂,曰:「春卿無恙。前別冀南,寂無音驛。援閒還長安,因留上林。竊見四海已定,兆民同情,而季孟閉拒背畔,為天下表的。常懼海內切齒,思相屠裂,故遺書戀戀,以致惻隱之計。乃聞季孟歸罪於援,而納王游翁諂邪之說,自謂函谷以西,舉足可定,以今而觀,竟何如邪?援閒至河內,過存伯春,見其奴吉從西方還,說伯春小弟仲舒望見吉,欲問伯春無它否,竟不能言,曉夕號泣,婉轉塵中。又說其家悲愁之狀,不可言也。夫怨讎可刺不可毀,援聞之,不自知泣下也。援素知季孟孝愛,曾、閔不過。夫孝於其親,豈不慈於其子?可有子抱三木,而跳梁妄作,自同分羹之事乎?季孟平生自言所以擁兵眾者,欲以保全父母之國而完墳墓也,又言苟厚士大夫而已。而今所欲全者將破亡之,所欲完者將毀傷之,所欲厚者將反薄之。季孟嘗折愧子陽而不受其爵,今更共陸陸,欲往附之,將難為顏乎?若復責以重質,當安從得子主給是哉!往時子陽獨欲以王相待,而春卿拒之;今者歸老,更欲低頭與小兒曹共槽櫪而食,併肩側身於怨家之朝乎?男兒溺死何傷而拘游哉!今國家待春卿意深,宜使牛孺卿與諸耆老大人共說季孟,若計畫不從,真可引領去矣。前披輿地圖,見天下郡國百有六所,柰何欲以區區二邦以當諸夏百有四乎?春卿事季孟,外有君臣之義,內有朋友之道。言君臣邪,固當諫爭;語朋友邪,應有切磋。豈有知其無成,而但萎腇咋舌,叉手從族乎?及今成計,殊尚善也;過是,欲少味矣。且來君叔天下信士,朝廷重之,其意依依,常獨為西州言。援商朝廷,尤欲立信於此,必不負約。援不得久留,願急賜報。」廣竟不荅。

八年,帝自西征囂,至漆,諸將多以王師之重,不宜遠入險阻,計冘豫未決。會召援,夜至,帝大喜,引入,具以群議質之。援因說隗囂將帥有土崩之埶,兵進有必破之狀。又於帝前聚米為山谷,指畫形埶,開示眾軍所從道徑往來,分析曲折,昭然可曉。帝曰:「虜在吾目中矣。」明旦,遂進軍至第一,囂眾大潰。

九年,拜援為太中大夫,副來歙監諸將平涼州。自王莽末,西羌寇邊,遂入居塞內,金城屬縣多為虜有。來歙奏言隴西侵殘,非馬援莫能定。十一年夏,璽書拜援隴西太守。援迺發步騎三千人,擊破先零羌於臨洮,斬首數百級,獲馬牛羊萬餘頭。守塞諸羌八千餘人詣援降。諸種有數萬,屯聚寇鈔,拒浩亹隘。援與揚武將軍馬成擊之。羌因將其妻子輜重移阻於允吾谷,援乃潛行閒道,掩赴其營。羌大驚壞,復遠徙唐翼谷中,援復追討之。羌引精兵聚北山上,援陳軍向山,而分遣數百騎繞襲其後,乘夜放火,擊鼓叫譟,虜遂大潰,凡斬首千餘級。援以兵少,不得窮追,收其穀糧畜產而還。援中矢貫脛,帝以璽書勞之,賜牛羊數千頭,援盡班諸賓客。

是時,朝臣以金城破羌之西,塗遠多寇,議欲棄之。援上言,破羌以西城多完牢,易可依固;其田土肥壤,灌溉流通。如令羌在湟中,則為害不休,不可棄也。帝然之,於是詔武威太守,令悉還金城客民。歸者三千餘口,使各反舊邑。援奏為置長吏,繕城郭,起塢候,開導水田,勸以耕牧,郡中樂業。又遣羌豪楊封譬說塞外羌,皆來和親。又武都氐人背公孫述來降者,援皆上復其侯王君長,賜印綬,帝悉從之。乃罷馬成軍。

十三年,武都參狼羌與塞外諸種為寇,殺長吏。援將四千餘人擊之,至氐道縣,羌在山上,援軍據便地,奪其水草,不與戰,羌遂窮困,豪帥數十萬戶亡出塞,諸種萬餘人悉降,於是隴右清靜。

援務開寬信,恩以待下,任吏以職,但總大體而已。賓客故人,日滿其門。諸曹時白外事,援輒曰:「此丞、掾之任,何足相煩。頗哀老子,使得遨游。若大姓侵小民,黠羌欲旅距,此乃太守事耳。」傍縣嘗有報仇者,吏民驚言羌反,百姓奔入城郭。狄道長詣門,請閉城發兵。援時與賓客飲,大笑曰:「燒虜何敢復犯我。曉狄道長歸守寺舍,良怖急者,可床下伏。」後稍定,郡中服之。視事六年,徵入為虎賁中郎將。

初,援在隴西上書,言宜如舊鑄五銖錢。事下三府,三府奏以為未可許,事遂寑。及援還,從公府求得前奏,難十餘條,乃隨牒解釋,更具表言。帝從之,天下賴其便。援自還京師,數被進見。為人明須髮,眉目如畫。閑於進對,尤善述前世行事。每言及三輔長者,下至閭里少年,皆可觀聽。自皇太子、諸王侍聞者,莫不屬耳忘倦。又善兵策,帝常言「伏波論兵,與我意合」,每有所謀,未嘗不用。

初,卷人維汜,訞言稱神,有弟子數百人,坐伏誅。後其弟子李廣等宣言汜神化不死,以誑惑百姓。十七年,遂共聚會徒黨,攻沒晥城,殺晥侯劉閔,自稱「南岳大師」。遣謁者張宗將兵數千人討之,復為廣所敗。於是使援發諸郡兵,合萬餘人,擊破廣等,斬之。

又交阯女子徵側及女弟徵貳反,攻沒其郡,九真、日南、合浦蠻夷皆應之,寇略嶺外六十餘城,側自立為王。於是璽書拜援伏波將軍,以扶樂侯劉隆為副,督樓船將軍段志等南擊交阯。軍至合浦而志病卒,詔援并將其兵。遂緣海而進,隨山刊道千餘里。十八年春,軍至浪泊上,與賊戰,破之,斬首數千級,降者萬餘人。援追徵側等至禁谿,數敗之,賊遂散走。明年正月,斬徵側、徵貳,傳首洛陽。封援為新息侯,食邑三千戶。援乃擊牛釃酒,勞饗軍士。從容謂官屬曰:「吾從弟少游常哀吾慷慨多大志,曰:『士生一世,但取衣食裁足,乘下澤車,御款段馬,為郡掾史,守墳墓,鄉里稱善人,斯可矣。致求盈餘,但自苦耳。』當吾在浪泊、西里閒,虜未滅之時,下潦上霧,毒氣重蒸,仰視飛鳶跕跕墯水中,臥念少游平生時語,何可得也!今賴士大夫之力,被蒙大恩,猥先諸君紆佩金紫,且喜且慚。」吏士皆伏稱萬歲。

援將樓船大小二千餘艘,戰士二萬餘人,進擊九真賊徵側餘黨都羊等,自無功至居風,斬獲五千餘人,嶠南悉平。援奏言西于縣戶有三萬二千,遠界去庭千餘里,請分為封溪、望海二縣,許之。援所過輒為郡縣治城郭,穿渠灌溉,以利其民。條奏越律與漢律駮者十餘事,與越人申明舊制以約束之,自後駱越奉行馬將軍故事。

二十年秋,振旅還京師,軍吏經瘴疫死者十四五。賜援兵車一乘,朝見位次九卿。

援好騎,善別名馬,於交阯得駱越銅鼓,乃鑄為馬式,還上之。因表曰:「夫行天莫如龍,行地莫如馬。馬者甲兵之本,國之大用。安寧則以別尊卑之序,有變則以濟遠近之難。昔有騏驥,一日千里,伯樂見之,昭然不惑。近世有西河子輿,亦明相法。子輿傳西河儀長孺,長孺傳茂陵丁君都,君都傳成紀楊子阿,臣援嘗師事子阿,受相馬骨法。考之於事,輒有驗效。臣愚以為傳聞不如親見,視景不如察形。今欲形之於生馬,則骨法難備具,又不可傳之於後。孝武皇帝時,善相馬者東門京鑄作銅馬法獻之,有詔立馬於魯班門外,則更名魯班門曰金馬門。臣謹依儀氏昙,中帛氏口齒,謝氏脣舰,丁氏身中,備此數家骨相以為法。」馬高三尺五寸,圍四尺五寸。有詔置於宣德殿下,以為名馬式焉。

初,援軍還,將至,故人多迎勞之,平陵人孟冀,名有計謀,於坐賀援。援謂之曰:「吾望子有善言,反同眾人邪?昔伏波將軍路博德開置七郡,裁封數百戶;今我微勞,猥饗大縣,功薄賞厚,何以能長久乎?先生奚用相濟?」冀曰:「愚不及。」援曰:「方今匈奴、烏桓尚擾北邊,欲自請擊之。男兒要當死於邊野,以馬革裹屍還葬耳,何能臥床上在兒女子手中邪?」冀曰:「諒為烈士,當如此矣。」

還月餘,會匈奴、烏桓寇扶風,援以三輔侵擾,園陵危逼,因請行,許之。自九月至京師,十二月復出屯襄國。詔百官祖道。援謂黃門郎梁松、竇固曰:「凡人為貴,當使可賤,如卿等欲不可復賤,居高堅自持,勉思鄙言。」松後果以貴滿致災,固亦幾不免。

明年秋,援乃將三千騎出高柳,行鴈門、代郡、上谷障塞。烏桓候者見漢軍至,虜遂散去,援無所得而還。

援嘗有疾,梁松來候之,獨拜床下,援不荅。松去後,諸子問曰:「梁伯孫帝婿,貴重朝廷,公卿已下莫不憚之,大人柰何獨不為禮?」援曰:「我乃松父友也。雖貴,何得失其序乎?」松由是恨之。

二十四年,武威將軍劉尚擊武陵五溪蠻夷,深入,軍沒,援因復請行。時年六十二,帝愍其老,未許之。援自請曰;「臣尚能被甲上馬。」帝令試之。援據鞍顧眄,以示可用。帝笑曰:「矍鑠哉是翁也!」遂遣援率中郎將馬武、耿舒、劉匡、孫永等,將十二郡募士及弛刑四萬餘人征五溪。援夜與送者訣,謂友人謁者杜愔曰:「吾受厚恩,年迫餘日索,常恐不得死國事。今獲所願,甘心瞑目,但畏長者家兒或在左右,或與從事,殊難得調;介介獨惡是耳。」明年春,軍至臨鄉,遇賊攻縣,援迎擊,破之,斬獲二千餘人,皆散走入竹林中。

初,軍次下雋,有兩道可入,從壺頭則路近而水嶮,從充則塗夷而運遠,帝初以為疑。及軍至,耿舒欲從充道,援以為棄日費糧,不如進壺頭,搤其喉咽,充賊自破。以事上之,帝從援策。三月,進營壺頭。賊乘高守隘,水疾,船不得上。會暑甚。士卒多疫死,援亦中病,遂困,乃穿岸為室,以避炎氣。賊每升險鼓譟,援輒曳足以觀之,左右哀其壯意,莫不為之流涕。耿舒與兄好畤侯弇書曰:「前舒上書當先擊充,糧雖難運而兵馬得用,軍人數萬爭欲先奮。今壺頭竟不得進,大眾怫鬱行死,誠可痛惜。前到臨鄉,賊無故自致,若夜擊之,即可殄滅。伏波類西域賈胡,到一處輒止,以是失利。今果疾疫,皆如舒言。」弇得書,奏之。帝乃使虎賁中郎將梁松乘驛責問援,因代監軍。會援病卒,松宿懷不平,遂因事陷之。帝大怒,追收援新息侯印綬。

初,兄子嚴、敦並喜譏議,而通輕俠客。援前在交阯,還書誡之曰:「吾欲汝曹聞人過失,如聞父母之名,耳可得聞,口不可得言也。好論議人長短,妄是非正法,此吾所大惡也,寧死不願聞子孫有此行也。汝曹知吾惡之甚矣,所以復言者,施衿結褵,申父母之戒,欲使汝曹不忘之耳。龍伯高敦厚周慎,口無擇言,謙約節儉,廉公有威,吾愛之重之,願汝曹效之。杜季良豪俠好義,憂人之憂,樂人之樂,清濁無所失,父喪致客,數郡畢至,吾愛之重之,不願汝曹效也。效伯高不得,猶為謹敕之士,所謂刻鵠不成尚類鶩者也。效季良不得,陷為天下輕薄子,所謂畫虎不成反類狗者也。訖今季良尚未可知,郡將下車輒切齒,州郡以為言,吾常為寒心,是以不願子孫效也。」季良名保,京兆人,時為越騎司馬。保仇人上書,訟保「為行浮薄,亂群惑眾,伏波將軍萬里還書以誡兄子,而梁松、竇固以之交結,將扇其輕偽,敗亂諸夏」。書奏,帝召責松、固,以訟書及援誡書示之,松、固叩頭流血,而得不罪。詔免保官。伯高名述,亦京兆人,為山都長,由此擢拜零陵太守。

初,援在交阯,常餌薏苡實,用能輕身省慾,以勝瘴氣。南方薏苡實大,援欲以為種,軍還,載之一車。時人以為南土珍怪,權貴皆望之。援時方有寵,故莫以聞。及卒後,有上書譖之者,以為前所載還,皆明珠文犀。馬武與於陵侯侯昱等皆以章言其狀,帝益怒。援妻孥惶懼,不敢以喪還舊塋,裁買城西數畝地槁葬而已。賓客故人莫敢弔會。嚴與援妻子草索相連,詣闕請罪。帝乃出松書以示之,方知所坐,上書訴冤,前後六上,辭甚哀切,然後得葬。

又前雲陽令同郡朱勃詣闕上書曰:

臣聞王德聖政,不忘人之功,採其一美,不求備於眾。故高祖赦蒯通而以王禮葬田橫,大臣曠然,咸不自疑。夫大將在外,讒言在內,微過輒記,大功不計,誠為國之所慎也。故章邯畏口而奔楚,燕將據聊而不下。豈其甘心末規哉,悼巧言之傷類也。

竊見故伏波將軍新息侯馬援,拔自西州,欽慕聖義,閒關險難,觸冒萬死,孤立群貴之閒,傍無一言之佐,馳深淵,入虎口,豈顧計哉!寧自知當要七郡之使,徼封侯之福邪?八年,車駕西討隗囂,國計狐疑,眾營未集,援建宜進之策,卒破西州。及吳漢下隴,冀路斷隔,唯獨狄道為國堅守,士民飢困,寄命漏刻。援奉詔西使,鎮慰邊眾,乃招集豪傑,曉誘羌戎,謀如涌泉,埶如轉規,遂救倒縣之急,存幾亡之城,兵全師進,因糧敵人,隴、冀略平,而獨守空郡,兵動有功,師進輒克。銖鋤先零,緣入山谷,猛怒力戰,飛矢貫脛。又出征交阯,土多瘴氣,援與妻子生訣,無悔吝之心,遂斬滅徵側,克平一州。閒復南討,立陷臨鄉,師已有業,未竟而死,吏士雖疫,援不獨存。夫戰或以久而立功,或以速而致敗,深入未必為得,不進未必為非。人情豈樂久屯絕地,不生歸哉!惟援得事朝廷二十二年,北出塞漠,南度江海,觸冒害氣,僵死軍事,名滅爵絕,國土不傳。海內不知其過,眾庶未聞其毀,卒遇三夫之言,橫被誣罔之讒,家屬杜門,葬不歸墓,怨隙並興,宗親怖慄。死者不能自列,生者莫為之訟,臣竊傷之。

夫明主醲於用賞,約於用刑。高祖嘗與陳平金四萬斤以閒楚軍,不問出入所為,豈復疑以錢穀閒哉?夫操孔父之忠而不能自免於讒,此鄒陽之所悲也。《詩》云:「取彼讒人,投畀豺虎,谥虎不食,投畀有北。有北不受,投畀有昊。」此言欲令上天而平其惡。惟陛下留思豎儒之言,無使功臣懷恨黃泉。臣聞春秋之義,罪以功除;聖王之祀,臣有五義。若援,所謂以死勤事者也。願下公卿平援功罪,宜絕宜續,以厭海內之望。

臣年已六十,常伏田里,竊感欒布哭彭越之義,冒陳悲憤,戰慄闕庭。

書奏,報,歸田里。書奏,報,歸田里。

勃字叔陽,年十二能誦詩、書。常候援兄況。勃衣方領,能矩步,辭言嫺雅,援裁知書,見之自失。況知其意,乃自酌酒慰援曰:「朱勃小器速成,智盡此耳,卒當從汝稟學,勿畏也。」朱勃未二十,右扶風請試守渭城宰,及援為將軍,封侯,而勃位不過縣令。援後雖貴,常待以舊恩而卑侮之,勃愈身自親,及援遇讒,唯勃能終焉。肅宗即位,追賜勃子穀二千斛。

初,援兄子婿王磐子石,王莽從兄平阿侯仁之子也。莽敗,磐擁富貲居故國,為人尚氣節而愛士好施,有名江淮閒。後游京師,與衛尉陰興、大司空朱浮、齊王章共相友善。援謂姊子曹訓曰:「王氏,廢姓也。子石當屏居自守,而反游京師長者,用氣自行,多所陵折,其敗必也。」後歲餘,磐果與司隸校尉蘇鄴、丁鴻事相連,坐死洛陽獄。而磐子肅復出入北宮及王侯邸第。援謂司馬呂种曰:「建武之元,名為天下重開。自今以往,海內日當安耳。但憂國家諸子並壯,而舊防未立,若多通賓客,則大獄起矣。卿曹戒慎之!」及郭后薨,有上書者,以為肅等受誅之家,客因事生亂,慮致貫高、任章之變。帝怒,乃下郡縣收捕諸王賓客,更相牽引,死者以千數。呂种亦豫其禍,臨命嘆曰:「馬將軍誠神人也!」

永平初,援女立為皇后。顯宗圖畫建武中名臣、列將於雲臺,以椒房故,獨不及援。東平王蒼觀圖,言於帝曰:「何故不畫伏波將軍像?」帝笑而不言。至十七年,援夫人卒,乃更脩封樹,起祠堂。

建初三年,肅宗使五官中郎將持節追策,謚援曰忠成侯。

四子:廖,防,光,客卿。

客卿幼而歧嶷,年六歲,能應接諸公,專對賓客。嘗有死罪亡命者來過,客卿逃匿不令人知。外若訥而內沈敏。援甚奇之,以為將相器,故以客卿字焉。援卒後,客卿亦夭沒。

論曰:馬援騰聲三輔,遨游二帝,及定節立謀,以干時主,將懷負鼎之願,蓋為千載之遇焉。然其戒人之禍,智矣,而不能自免於讒隙。豈功名之際,理固然乎?夫利不在身,以之謀事則智;慮不私己,以之斷義必厲。誠能回觀物之智而為反身之察,若施之於人則能恕,自鑒其情亦明矣。

廖字敬平,少以父任為郎。明德皇后既立,拜廖為羽林左監、虎賁中郎將。顯宗崩,受遺詔典掌門禁,遂代趙憙為衛尉,肅宗甚尊重之。

時皇太后躬履節儉,事從簡約,廖慮美業難終,上疏長樂宮以勸成德政,曰:「臣案前世詔令,以百姓不足,起於世尚奢靡,故元帝罷服官,成帝御浣衣,哀帝去樂府。然而侈費不息,至於衰亂者,百姓從行不從言也。夫改政移風,必有其本。傳曰:『吳王好劍客,百姓多創瘢;楚王好細腰,宮中多餓死。』長安語曰:『城中好高髻,四方高一尺;城中好廣眉,四方且半額;城中好大袖,四方全匹帛。』斯言如戲,有切事實。前下制度未幾。後稍不行。雖或吏不奉法,良由慢起京師。今陛下躬服厚繒,斥去華飾,素簡所安,發自聖性。此誠上合天心,下順民望,浩大之福,莫尚於此。陛下既已得之自然,猶宜加以勉勗,法太宗之隆德,戒成、哀之不終。《易》曰:『不恆其德,或承之羞。』誠令斯事一竟,則四海誦德,聲薰天地,神明可通,金石可勒,而況於行仁心乎,況於行令乎!願置章坐側,以當瞽人夜誦之音。」太后深納之。朝廷大議,輒以詢訪。

廖性質誠畏慎,不愛權埶聲名,盡心納忠,不屑毀譽。有司連據舊典,奏封廖等,累讓不得已,建初四年,遂受封為順陽侯,以特進就第。每有賞賜,輒辭讓不敢當,京師以是稱之。

子豫,為步兵校尉。太后崩後,馬氏失埶,廖性寬緩,不能教勒子孫,豫遂投書怨誹。又防、光奢侈,好樹黨與。八年,有司奏免豫,遣廖、防、光就封。豫隨廖歸國,考擊物故。後詔還廖京師。永元四年,卒。和帝以廖先帝之舅,厚加賵賻,使者弔祭,王主會喪,謚曰安侯。

子遵嗣,徙封程鄉侯。遵卒,無子,國除。元初三年,鄧太后詔封廖孫度為潁陽侯。

防字江平,永平十二年,與弟光俱為黃門侍郎。肅宗即位,拜防中郎將,稍遷城門校尉。

建初二年,金城、隴西保塞羌皆反,拜防行車騎將軍事,以長水校尉耿恭副,將北軍五校兵及諸郡積射士三萬人擊之。軍到冀,而羌豪布橋等圍南部都尉於臨洮。防欲救之,臨洮道險,車騎不得方駕,防乃別使兩司馬將數百騎,分為前後軍,去臨洮十餘里為大營,多樹幡幟,揚言大兵旦當進。羌候見之,馳還言漢兵盛不可當。明旦遂鼓譟而前,羌虜驚走,因追擊破之,斬首虜四千餘人,遂解臨洮圍。防開以恩信,燒當種皆降,唯布橋等二萬餘人在臨洮西南望曲谷。十二月,羌又敗耿恭司馬及隴西長史於和羅谷,死者數百人。明年春,防遣司馬夏駿將五千人從大道向其前,潛遣司馬馬彭將五千人從閒道衝其心腹,又令將兵長史李調等將四千人繞其西,三道俱擊,復破之,斬獲千餘人,得牛羊十餘萬頭。羌退走,夏駿追之,反為所敗。防乃引兵與戰於索西,又破之。布橋迫急,將種人萬餘降。詔徵防還,拜車騎將軍,城門校尉如故。

防貴寵最盛,與九卿絕席。光自越騎校尉遷執金吾。四年,封防潁陽侯,光為許侯,兄弟二人各六千戶。防以顯宗寑疾,入參醫藥,又平定西羌,增邑千三百五十戶。屢上表讓位,俱以特進就第。皇太后崩,明年,拜防光祿勳,光為衛尉。防數言政事,多見採用。是冬始施行十二月迎氣樂,防所上也。子鉅,為常從小侯。六年正月,以鉅當冠,特拜為黃門侍郎。肅宗親御章臺下殿,陳鼎俎,自臨冠之。明年,防復以病乞骸骨,詔賜故中山王田廬,以特進就第。

防兄弟貴盛,奴婢各千人已上,資產巨億,皆買京師膏腴美田,又大起第觀,連閣臨道,彌亙街路,多聚聲樂,曲度比諸郊廟。賓客奔湊,四方畢至,京兆杜篤之徒數百人,常為食客,居門下。刺史、守、令多出其家。歲時賑給鄉閭,故人莫不周洽。防又多牧馬畜,賦斂羌胡。帝不喜之,數加譴敕,所以禁遏甚備,由是權埶稍損,賓客亦衰。八年,因兄子豫怨謗事,有司奏防、光兄弟奢侈踰僭,濁亂聖化,悉免就國。臨上路,詔曰:「舅氏一門,俱就國封,四時陵廟無助祭先后者,朕甚傷之。其令許侯思愆田廬,有司勿復請,以慰朕渭陽之情。」

光為人小心周密,喪母過哀,帝以是特親愛之,乃復位特進。子康,黃門侍郎。永元二年,光為太僕,康為侍中。及竇憲誅,光坐與厚善,復免就封。後憲奴誣光與憲逆,自殺,家屬歸本郡。本郡復殺康,而防及廖子遵皆坐徙封丹陽。防為翟鄉侯,租歲限三百萬,不得臣吏民。防後以江南下溼,上書乞歸本郡,和帝聽之。十三年,卒。

子鉅嗣,後為長水校尉。永初七年,鄧太后詔諸馬子孫還京師,隨四時見會如故事,復紹封光子朗為合鄉侯。

嚴字威卿。父余,王莽時為楊州牧。嚴少孤,而好擊劍,習騎射。後乃白援,從平原楊太伯講學,專心墳典,能通春秋左氏,因覽百家群言,遂交結英賢,京師大人咸器異之。仕郡督郵,援常與計議,委以家事。弟敦,字孺卿,亦知名。援卒後,嚴乃與敦俱歸安陵、居鉅下,三輔稱其義行,號曰「鉅下二卿」。

明德皇后既立,嚴乃閉門自守,猶復慮致譏嫌,遂更徙北地,斷絕賓客。永平十五年,皇后敕使移居洛陽。顯宗召見,嚴進對閑雅,意甚異之,有詔留仁壽闥,與校書郎杜撫、班固等雜定建武注記。常與宗室近親臨邑侯劉復等論議政事,甚見寵幸。後拜將軍長史,將北軍五校士、羽林禁兵三千人,屯西河美稷,衛護南單于,聽置司馬、從事。牧守謁敬,同之將軍。敕嚴過武庫,祭蚩尤,帝親御阿閣,觀其士眾,時人榮之。

肅宗即位,徵拜侍御史中丞,除子鱄為郎,令勸學省中。其冬,有日食之災,嚴上封事曰:「臣聞日者眾陽之長,食者陰侵之徵。書曰:『無曠庶官,天工人其代之。』言王者代天官人也。故考績黜陟,以明褒貶。無功不黜,則陰盛陵陽。臣伏見方今刺史太守專州典郡,不務奉事盡心為國,而司察偏阿,取與自己,同則舉為尤異,異則中以刑法,不即垂頭塞耳,採求財賂。今益州刺史朱酺、楊州刺史倪說、涼州刺史尹業等,每行考事,輒有物故,又選舉不實,曾無貶坐,是使臣下得作威福也。故事,州郡所舉上奏,司直察能否以懲虛實。今宜加防檢,式遵前制。舊丞相、御史親治職事,唯丙吉以年老優游,不案吏罪,於是宰府習為常俗,更共罔養,以崇虛名,或未曉其職,便復遷徙,誠非建官賦祿之意。宜敕正百司,各責以事,州郡所舉,必得其人。若不如言,裁以法令。傳曰:『上德以寬服民,其次莫如猛。故火烈則人望而畏之,水懦則人狎而翫之。為政者寬以濟猛,猛以濟寬。』如此,綏御有體,災眚消矣。」書奏,帝納其言而免酺等官。

建初元年,遷五官中郎將,除三子為郎。嚴數薦達賢能,申解冤結,多見納用。復以五官中郎將行長樂衛尉事。二年,拜陳留太守。嚴當之職,乃言於帝曰:「昔顯親侯竇固誤先帝出兵西域,置伊吾盧屯,煩費無益。又竇勳受誅,其家不宜親近京師。」是時勳女為皇后,竇氏方寵,時有側聽嚴言者,以告竇憲兄弟,由是失權貴心。嚴下車,明賞罰,發姦慝,郡界清靜。時京師訛言賊從東方來,百姓奔走,轉相驚動,諸郡遑急,各以狀聞。嚴察其虛妄,獨不為備。詔書敕問,使驛係道,嚴固執無賊,後卒如言。典郡四年,坐與宗正劉軼、少府丁鴻等更相屬託,徵拜太中大夫;十餘日,遷將作大匠。七年,復坐事免。後既為竇氏所忌,遂不復在位。及帝崩,竇太后臨朝,嚴乃退居自守,訓教子孫。永元十年,卒於家,時年八十二。

弟敦,官至虎賁中郎將。嚴七子,唯續、融知名。續字季則,七歲能通論語,十三明尚書,十六治詩,博觀群籍,善九章筭術。順帝時,為護羌校尉,遷度遼將軍,所在有威恩稱。融自有傳。

棱字伯威,援之族孫也。少孤,依從兄毅共居業,恩猶同產,毅卒無子,棱心喪三年。

建初中,仕郡功曹,舉孝廉。及馬氏廢,肅宗以棱行義,徵拜謁者。章和元年,遷廣陵太守。時穀貴民飢,奏罷鹽官,以利百姓,賑貧羸,薄賦稅,興復陂湖,溉田二萬餘頃,吏民刻石頌之。永元二年,轉漢陽太守,有威嚴稱。大將軍竇憲西屯武威,棱多奉軍費,侵賦百姓,憲誅,坐抵罪。後數年,江湖多劇賊,以棱為丹陽太守。棱發兵掩擊,皆禽滅之。轉會稽太守,治亦有聲。轉河內太守。永初中,坐事抵罪,卒于家。

贊曰:伏波好功,爰自冀、隴。南靜駱越,西屠燒種。徂年已流,壯情方勇。明德既升,家祚以興。廖乏三趣,防遂驕陵。


\end{pinyinscope}