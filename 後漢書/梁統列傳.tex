\article{梁統列傳}

\begin{pinyinscope}
梁統字仲寧,安定烏氏人,晉大夫梁益耳,即其先也。統高祖父子都,自河東遷居北地,子都子橋,以貲千萬徙茂陵,至哀、平之末,歸安定。

統性剛毅而好法律。初仕州郡。更始二年,召補中郎將,使安集涼州,拜酒泉太守。會更始敗,赤眉入長安,統與竇融及諸郡守起兵保境,謀共立帥。初以位次,咸共推統,統固辭曰:「昔陳嬰不受王者,以有老母也。今統內有尊親,又德薄能寡,誠不足以當之。」遂共推融為河西大將軍,更以統為武威太守。為政嚴猛,威行鄰郡。

建武五年,統等各遣使隨竇融長史劉鈞詣闕奉貢,願得詣行在所,詔加統宣德將軍。八年夏,光武自征隗囂,統與竇融等將兵會車駕。及囂敗,封統為成義侯,同產兄巡、從弟騰並為關內侯,拜騰酒泉典農都尉,悉遣還河西。十二年,統與融等俱詣京師,以列侯奉朝請,更封高山侯,拜太中大夫,除四子為郎。

統在朝廷,數陳便宜。以為法令既輕,下姦不勝,宜重刑罰,以遵舊典,乃上疏曰:

臣竊見元哀二帝輕殊死之刑以一百二十三事

,手殺人者減死一等,自是以後,著為常準,故人輕犯法,吏易殺人。

臣聞立君之道,仁義為主,仁者愛人,義者政理,愛人以除殘為務,政理以去亂為心。刑罰在衷,無取於輕,是以五帝有流、殛、放、殺之誅,三王有大辟、刻肌之法。故孔子稱「仁者必有勇」,又曰「理財正辭,禁民為非曰義」。高帝受命誅暴,平蕩天下,約令定律,誠得其宜。文帝寬惠柔克,遭世康平,唯除省肉刑、相坐之法,它皆率由,無革舊章。武帝值中國隆盛,財力有餘,征伐遠方,軍役數興,豪桀犯禁,姦吏弄法,故重首匿之科,著知從之律,以破朋黨,以懲隱匿。宣帝聰明正直,總御海內,臣下奉憲,無所失墜,因循先典,天下稱理。至哀、平繼體,而即位日淺,聽斷尚寡,丞相王嘉輕為穿鑿,虧除先帝舊約成律,數年之閒,百有餘事,或不便於理,或不厭民心。謹表其尤害於體者傅奏於左。

伏惟陛下包元履德,權時撥亂,功踰文武,德侔高皇,誠不宜因循季末衰微之軌。回神明察,考量得失,宣詔有司,詳擇其善,定不易之典,施無窮之法,天下幸甚。

事下三公、廷尉,議者以為隆刑峻法,非明王急務,施行日久,豈一朝所釐。統今所定,不宜開可。

統復上言曰:「有司以臣今所言,不可施行。尋臣之所奏,非曰嚴刑。竊謂高帝以後,至乎孝宣,其所施行,多合經傳,宜比方今事,驗之往古,聿遵前典,事無難改,不勝至願。願得召見,若對尚書近臣,口陳其要。」帝令尚書問狀,統對曰:

聞聖帝明王,制立刑罰,故雖堯舜之盛,猶誅四凶。經曰:「天討有罪,五刑五庸哉。」又曰:「爰制百姓于刑之衷。」孔子曰:「刑罰不衷,則人無所厝手足。」衷之為言,不輕不重之謂也。春秋之誅,不避親戚,所以防患救亂,全安眾庶,豈無仁愛之恩,貴絕殘賊之路也?

自高祖之興,至于孝宣,君明臣忠,謨謀深博,猶因循舊章,不輕改革,海內稱理,斷獄益少。至初元、建平,所減刑罰百有餘條,而盜賊浸多,歲以萬數。閒者三輔從橫,群輩並起,至燔燒茂陵,火見未央。其後隴西、北地、西河之賊,越州度郡,萬里交結,攻取庫兵,劫略吏人,詔書討捕,連年不獲。是時以天下無難,百姓安平,而狂狡之埶,猶至於此,皆刑罰不衷,愚人易犯之所致也。

由此觀之,則刑輕之作,反生大患;惠加姦軌,而害及良善也。故臣統願陛下采擇賢臣孔光、師丹等議。

議上,遂寑不報。

後出為九江太守,定封陵鄉侯。統在郡亦有治跡,吏人畏愛之。卒於官。子松嗣。

松字伯孫,少為郎,尚光武女舞陰長公主,再遷虎賁中郎將。松博通經書,明習故事,與諸儒脩明堂、辟廱、郊祀、封禪禮儀,常與論議,寵幸莫比。光武崩,受遺詔輔政。永平元年,遷太僕。

松數為私書請託郡縣,二年,發覺免官,遂懷怨望。四年冬,乃縣飛書誹謗,下獄死,國除。

子扈,後以恭懷皇后從兄,永元中,擢為黃門侍郎,歷位卿、校尉。溫恭謙讓,亦敦詩書。永初中,為長樂少府。松弟竦。

竦字叔敬,少習孟氏易,弱冠能教授。後坐兄松事,與弟恭俱徙九真。既徂南土,歷江、湖,濟沅、湘,感悼子胥、屈原以非辜沈身,乃作悼騷賦,繫玄石而沈之。

顯宗後詔聽還本郡。竦閉門自養,以經籍為娛,著書數篇,名曰七序。班固見而稱曰:「孔子著春秋而亂臣賊子懼,梁竦作七序而竊位素餐者慚。」性好施,不事產業。長嫂舞陰公主贍給諸梁,親疏有序,特重敬竦,雖衣食器物,必有加異。竦悉分與親族,自無所服。

竦生長京師,不樂本土,自負其才,鬱鬱不得意。嘗登高遠望,歎息言曰:「大丈夫居世,生當封侯,死當廟食。如其不然,閑居可以養志,詩書足以自娛,州郡之職,徒勞人耳。」後辟命交至,並無所就。有三男三女,肅宗納其二女,皆為貴人。小貴人生和帝,竇皇后養以為子,而竦家私相慶。後諸竇聞之,恐梁氏得志,終為己害,建初八年,遂譖殺二貴人,而陷竦等以惡逆。詔使漢陽太守鄭據傳考竦罪,死獄中,家屬復徙九真。辭語連及舞陰公主,坐徙新城,使者護守。宮省事密,莫有知和帝梁氏生者。

永元九年,竇太后崩,松子扈遣從兄襢奏記三府,以為漢家舊典,崇貴母氏,而梁貴人親育聖躬,不蒙尊號,求得申議。太尉張酺引襢訊問事理,會後召見,因白襢奏記之狀。帝感慟良久,曰:「於君意若何?」酺對曰:「春秋之義,母以子貴。漢興以來,母氏莫不隆顯,臣愚以為宜上尊號,追慰聖靈,存錄諸舅,以明親親。」帝悲泣曰:「非君孰為朕思之!」會貴人姊南陽樊調妻嫕上書自訟曰:「妾同產女弟貴人,前充後宮,蒙先帝厚恩,得見寵幸。皇天授命,誕生聖明。而為竇憲兄弟所見譖訴,使妾父竦冤死牢獄,骸骨不掩。老母孤弟,遠徙萬里。獨妾遺脫,逸伏草野,常恐沒命,無由自達。今遭值陛下神聖之運,親統萬機,群物得所。憲兄弟姦惡,既伏辜誅,海內曠然,各獲其宜。妾得蘇息,拭目更視,乃敢昧死自陳所天。妾聞太宗即位,薄氏蒙榮;宣帝繼統,史族復興。妾門雖有薄、史之親,獨無外戚餘恩,誠自悼傷。妾父既冤,不可復生,母氏年殊七十,及弟棠等,遠在絕域,不知死生。願乞收竦朽骨,使母弟得歸本郡,則施過天地,存歿幸賴。」帝覽章感悟,乃下中常侍、掖庭令驗問之,嫕辭證明審,遂得引見,具陳其狀。乃留嫕止宮中,連月乃出,賞賜衣被錢帛第宅奴婢,旬月之閒,累資千萬。嫕素有行操,帝益愛之,加號梁夫人;擢樊調為羽林左監。調,光祿大夫宏兄曾孫也。

於是追尊恭懷皇后。其冬,制詔三公、大鴻臚曰:「夫孝莫大於尊尊親親,其義一也。《詩》云:『父兮生我,母兮鞠我,撫我畜我,長我育我,顧我復我,出入腹我。欲報之德,昊天罔極。』朕不敢興事,覽于前世。太宗、中宗,寔有舊典,追命外祖,以篤親親。其追封謚皇太后父竦為褒親愍侯,比靈文、順成、侯。魂而有靈,嘉斯寵榮,好爵顯服,以慰母心。」遣中謁者與嫕及扈,備禮西迎竦喪,詣京師改殯,賜東園畫棺、玉匣、衣衾,建塋於恭懷皇后陵傍。帝親臨送葬,百官畢會。

徵還竦妻子,封子棠為樂平侯,棠弟雍乘氏侯,雍弟翟單父侯,邑各五千戶,位皆特進,賞賜第宅奴婢車馬兵弩什物以巨萬計,寵遇光於當世。諸梁內外以親疏並補郎、謁者。

棠官至大鴻臚,雍少府。棠卒,子安國嗣,延光中為侍中,有罪免官,諸梁為郎吏者皆坐免。

商字伯夏,雍之子也。少以外戚拜郎中,遷黃門侍郎。永建元年,襲父封乘氏侯。三年,順帝選商女及妹入掖庭,遷侍中、屯騎校尉。陽嘉元年,女立為皇后,妹為貴人,加商位特進,更增國土,賜安車駟馬,其歲拜執金吾。二年,封子冀為襄邑侯,商讓不受。三年,以商為大將軍,固稱疾不起。四年,使太常桓焉奉策就第即拜,商乃詣闕受命。明年,夫人陰氏薨,追號開封君,贈印綬。

商自以戚屬居大位,每存謙柔,虛己進賢,辟漢陽巨覽、上黨陳龜為掾屬,李固、周舉為從事中郎,於是京師翕然,稱為良輔,帝委重焉。每有飢饉,輒載租穀於城門,賑與貧餧,不宣己惠。檢御門族,未曾以權盛干法。而性慎弱無威斷,頗溺於內豎。以小黃門曹節等用事於中,遂遣子冀、不疑與為交友,然宦者忌商寵任,反欲陷之。永和四年,中常侍張逵、蘧政,內者令石光,尚方令傅福,冗從僕射杜永連謀,共譖商及中常侍曹騰、孟賁,云欲徵諸王子,圖議廢立,請收商等案罪。帝曰:「大將軍父子我所親,騰、賁我所愛,必無是,但汝曹共妒之耳。」逵等知言不用,懼迫,遂出矯詔收縛騰、賁於省中。帝聞震怒,敕宦者李歙急呼騰、賁釋之,收逵等,悉伏誅。辭所連染及在位大臣,商懼多侵枉,乃上疏曰:「春秋之義,功在元帥,罪止首惡,故賞不僭溢,刑不淫濫,五帝、三王所以同致康乂也。竊聞考中常侍張逵等,辭語多所牽及。大獄一起,無辜者眾,死囚久繫,纖微成大,非所以順迎和氣,平政成化也。宜早訖竟,以止逮捕之煩。」帝乃納之,罪止坐者。

六年秋,商病篤,敕子冀等曰:「吾以不德,享受多福。生無以輔益朝廷,死必耗費帑臧,衣衾飯唅玉匣珠貝之屬,何益朽骨。百僚勞擾,紛華道路,秖增塵垢,雖云禮制,亦有權時。方今邊境不寧,盜賊未息,豈宜重為國損!氣絕之後,載至冢舍,即時殯斂。斂以時服,皆以故衣,無更裁制。殯已開冢,冢開即葬。祭食如存,無用三牲。孝子善述父志,不宜違我言也。」及薨,帝親臨喪,諸子欲從其誨,朝廷不聽,賜以東園朱壽之器、銀鏤、黃腸、玉匣、什物二十八種,錢二百萬,布三千匹。皇后錢五百萬,布萬匹。及葬,贈輕車介士,賜謚忠侯。中宮親送,帝幸宣陽亭,瞻望車騎。

子冀嗣。

冀字伯卓。為人鳶肩豺目,洞精矘眄,口吟舌言,裁能書計。少為貴戚,逸游自恣。性嗜酒,能挽滿、彈棋、格五、六博、蹴鞠、意錢之戲,又好臂鷹走狗,騁馬鬥雞。初為黃門侍郎,轉侍中,虎賁中郎將,越騎、步兵校尉,執金吾。

永和元年,拜河南尹。冀居職暴恣,多非法,父商所親客洛陽令呂放,頗與商言及冀之短,商以讓冀,冀即遣人於道刺殺放。而恐商知之,乃推疑於放之怨仇,請以放弟禹為洛陽令,使捕之,盡滅其宗親、賓客百餘人。

商薨未及葬,順帝乃拜冀為大將軍,弟侍中不疑為河南尹。

及帝崩,沖帝始在繈褓,太后臨朝,詔冀與太傅趙峻、太尉李固參錄尚書事。冀雖辭不肯當,而侈暴滋甚。

沖帝又崩,冀立質帝。帝少而聰慧,知冀驕橫,嘗朝群臣,目冀曰:「此跋扈將軍也。」冀聞,深惡之,遂令左右進鴆加煮餅,帝即日崩。

復立桓帝,而枉害李固及前太尉杜喬,海內嗟懼,語在李固傳。建和元年,益封冀萬三千戶,增大將軍府舉高第茂才,官屬倍於三公。又封不疑為潁陽侯,不疑弟蒙西平侯,冀子胤襄邑侯,各萬戶。和平元年,重增封冀萬戶,并前所襲合三萬戶。

弘農人宰宣素性佞邪,欲取媚於冀,乃上言大將軍有周公之功,今既封諸子,則其妻宜為邑君。詔遂封冀妻孫壽為襄城君,兼食陽翟租,歲入五千萬,加賜赤紱,比長公主。壽色美而善為妖態,作愁眉,啼诽,墯馬髻,折腰步,齲齒笑,以為媚惑。冀亦改易輿服之制,作平上軿車,埤幘,狹冠,折上巾,擁身扇,狐尾單衣。壽性鉗忌,能制御冀,冀甚寵憚之。

初,父商獻美人友通期於順帝,通期有微過,帝以歸商,商不敢留而出嫁之,冀即遣客盜還通期。會商薨,冀行服,於城西私與之居。壽伺冀出,多從倉頭,篡取通期歸,截髮刮面,笞掠之,欲上書告其事。冀大恐,頓首請於壽母,壽亦不得已而止。冀猶復與私通,生子伯玉,匿不敢出。壽尋知之,使子胤誅滅友氏。冀慮壽害伯玉,常置複壁中。冀愛監奴秦宮,官至太倉令,得出入壽所。壽見宮,輒屏御者,託以言事,因與私焉。宮內外兼寵,威權大震,刺史、二千石皆謁辭之。

冀用壽言,多斥奪諸梁在位者,外以謙讓,而實崇孫氏宗親。冒名而為侍中、卿、校尉、郡守、長吏者十餘人,皆貪叨凶淫,各遣私客籍屬縣富人,被以它罪,閉獄掠拷,使出錢自贖,貲物少者至於死徙。扶風人士孫奮居富而性吝,冀因以馬乘遺之,從貸錢五千萬,奮以三千萬與之,冀大怒,乃告郡縣,認奮母為其守臧婢,云盜白珠十斛、紫金千斤以叛,遂收考奮兄弟,死於獄中,悉沒貲財億七千餘萬。

其四方調發,歲時貢獻,皆先輸上第於冀,乘輿乃其次焉。吏人齎貨求官請罪者,道路相望。冀又遣客出塞,交通外國,廣求異物。因行道路,發取妓女御者,而使人復乘埶橫暴,妻略婦女,敺擊吏卒,所在怨毒。

冀乃大起第舍,而壽亦對街為宅,殫極土木,互相誇競。堂寢皆有陰陽奧室,連房洞戶。柱壁雕鏤,加以銅漆;吓牖皆有綺疏青瑣,圖以雲氣仙靈。臺閣周通,更相臨望;飛梁石蹬,陵跨水道。金玉珠璣,異方珍怪,充積臧室。遠致汗血名馬。又廣開園囿,採土築山,十里九阪,以像二崤,深林絕澗,有若自然,奇禽馴獸,飛走其閒。冀壽共乘輦車,張羽蓋,飾以金銀,游觀第內,多從倡伎,鳴鍾吹管,酣謳竟路。或連繼日夜,以騁娛恣。客到門不得通,皆請謝門者,門者累千金。又多拓林苑,禁同王家,西至弘農,東界滎陽,南極魯陽,北達河、淇,包含山藪,遠帶丘荒,周旋封域,殆將千里。又起菟苑於河南城西,經亙數十里,發屬縣卒徒,繕修樓觀,數年乃成。移檄所在,調發生菟,刻其毛以為識,人有犯者,罪至刑死。嘗有西域賈胡,不知禁忌,誤殺一兔,轉相告言,坐死者十餘人。冀二弟嘗私遣人出獵上黨,冀聞而捕其賓客,一時殺三十餘人,無生還者。冀又起別第於城西,以納姦亡。或取良人,悉為奴婢,至數千人,名曰「自賣人」。

元嘉元年,帝以冀有援立之功,欲崇殊典,乃大會公卿,共議其禮。於是有司奏冀入朝不趨,劍履上殿,謁讚不名,禮儀比蕭何;悉以定陶、陽成餘戶增封為四縣,比鄧禹;賞賜金錢、奴婢、綵帛、車馬、衣服、甲第,比霍光:以殊元勳。每朝會,與三公絕席。十日一入,平尚書事。宣布天下,為萬世法。冀猶以所奏禮薄,意不悅。專擅威柄,凶恣日積,機事大小,莫不諮決之。宮衛近侍,並所親樹,禁省起居,纖微必知。百官遷召,皆先到冀門牋檄謝恩,然後敢詣尚書。下邳人吳樹為宛令,之官辭冀,冀賓客布在縣界,以情託樹。樹對曰:「小人姦蠹,比屋可誅。明將軍以椒房之重,處上將之位,宜崇賢善,以補朝闕。宛為大都,士之淵藪,自侍坐以來,未聞稱一長者,而多託非人,誠非敢聞!」冀嘿然不悅。樹到縣,遂誅殺冀客為人害者數十人,由是深怨之。樹後為荊州刺史,臨去辭冀,冀為設酒,因鴆之,樹出,死車上。又遼東太守侯猛,初拜不謁,冀託以它事,乃腰斬之。

時郎中汝南袁著,年十九,見冀凶縱,不勝其憤,乃詣闕上書曰:「臣聞仲尼歎鳳鳥不至,河不出圖,自傷卑賤,不能致也。今陛下居得致之位,又有能致之資,而和氣未應,賢愚失序者,埶分權臣,上下壅隔之故也。夫四時之運,功成則退,高爵厚寵,鮮不致災。今大將軍位極功成,可為至戒,宜遵懸車之禮,高枕頤神。傳曰:『木實繁者,披枝害心。』若不抑損權盛,將無以全其身矣。左右聞臣言,將側目切齒,臣特以童蒙見拔,故敢忘忌諱。昔舜、禹相戒無若丹朱,周公戒成王無如殷王紂,願除誹謗之罪,以開天下之口。」書得奏御,冀聞而密遣掩捕著。著乃變易姓名,後託病偽死,結蒲為人,市棺殯送。冀廉問知其詐,陰求得,笞殺之,隱蔽其事。學生桂陽劉常,當世名儒,素善於著,冀召補令史以辱之。時太原郝絜、胡武,皆危言高論,與著友善。先是絜等連名奏記三府,薦海內高士,而不詣冀,冀追怒之,又疑為著黨,敕中都官移檄捕前奏記者並殺之,遂誅武家,死者六十餘人。絜初逃亡,知不得免,因輿櫬奏書冀門。書入,仰藥而死,家乃得全。及冀誅,有詔以禮祀著等。冀諸忍忌,皆此類也。

不疑好經書,善待士,冀陰疾之,因中常侍白帝,轉為光祿勳。又諷眾人共薦其子胤為河南尹。胤一名胡狗,時年十六,容貌甚陋,不勝冠帶,道路見者,莫不蚩笑焉。不疑自恥兄弟有隙,遂讓位歸第,與弟蒙閉門自守。冀不欲令與賓客交通,陰使人變服至門,記往來者,南郡太守馬融、江夏太守田明,初除,過謁不疑,冀諷州郡以它事陷之,皆髡笞徙朔方。融自刺不殊,明遂死於路。

永興二年,封不疑子馬為潁陰侯,胤子桃為城父侯。冀一門前後七封侯,三皇后,六貴人,二大將軍,夫人、女食邑稱君者七人,尚公主者三人,其餘卿、將、尹、校五十七人。在位二十餘年,窮極滿盛,威行內外,百僚側目,莫敢違命,天子恭己而不得有所親豫。

帝既不平之。延熹元年,太史令陳授因小黃門徐璜,陳災異日食之變,咎在大將軍,冀聞之,諷洛陽收考授,死於獄。帝由此發怒。

初,掖庭人鄧香妻宣生女猛,香卒,宣更適梁紀。梁紀者,冀妻壽之舅也。壽引進猛入掖庭,見幸,為貴人,冀因欲認猛為其女以自固,乃易猛姓為梁。時猛姊婿邴尊為議郎,冀恐尊沮敗宣意,乃結刺客於偃城,刺殺尊,而又欲殺宣。宣家在延熹里,與中常侍袁赦相比。冀使刺客登赦屋,欲入宣家。赦覺之,鳴鼓會眾以告宣。宣馳入以白帝,帝大怒,遂與中常侍單超、具瑗、唐衡、左悺、徐璜等五人成謀誅冀。語在宦者傳。

冀心疑超等,乃使中黃門張惲入省宿,以防其變。具瑗敕吏收惲,以輒從外入,欲圖不軌。帝因是御前殿,召諸尚書入,發其事,使尚書令尹勳持節勒丞郎以下皆操兵守省閣,斂諸符節送省中。使黃門令具瑗將左右廄騶、虎賁、羽林、都候劍戟士,合千餘人,與司隸校尉張彪共圍冀第。使光祿勳袁盱持節收冀大將軍印綬,徙封比景都鄉侯。冀及妻壽即日皆自殺。悉收子河南尹胤、叔父屯騎校尉讓,及親從衛尉淑、越騎校尉忠、長水校尉戟等,諸梁及孫氏中外宗親送詔獄,無長少皆棄市。不疑、蒙先卒。其它所連及公卿列校刺史二千石死者數十人,故吏賓客免黜者三百餘人,朝廷為空,唯尹勳、袁盱及廷尉邯鄲義在焉。是時事卒從中發,使者交馳,公卿失其度,官府市里鼎沸,數日乃定,百姓莫不稱慶。

收冀財貨,縣官斥賣,合三十餘萬萬,以充王府,用減天下稅租之半。散其苑囿,以業窮民。錄誅冀功者,封尚書令尹勳以下數十人。

論曰:順帝之世,梁商稱為賢輔,豈以其地居亢滿,而能以愿謹自終者乎?夫宰相運動樞極,感會天人,中於道則易以興政,乖於務則難乎御物。商協回天之埶,屬彫弱之期,而匡朝卹患,未聞上術,憔悴之音,載謠人口。雖輿粟盈門,何救阻飢之厄;永言終制,未解尸官之尤。況乃傾側孽臣,傳寵凶嗣,以至破家傷國,而豈徒然哉!

贊曰:河西佐漢,統亦定筭。褒親幽憤,升高累歎。商恨善柔,冀遂貪亂。


\end{pinyinscope}