\article{班梁列傳}

\begin{pinyinscope}
班超字仲升,扶風平陵人,徐令彪之少子也。為人有大志,不修細節。然內孝謹,居家常執勤苦,不恥勞辱。有口辯,而涉獵書傳。永平五年,兄固被召詣校書郎,超與母隨至洛陽。家貧,常為官傭書以供養。久勞苦,嘗輟業投筆歎曰:「大丈夫無它志略,猶當效傅介子、張騫立功異域,以取封侯,安能久事筆研閒乎?」左右皆笑之。超曰:「小子安知壯士志哉!」其後行詣相者,曰:「祭酒,布衣諸生耳,而當封侯萬里之外。」超問其狀。相者指曰:「生燕頷虎頸,飛而食肉,此萬里侯相也。」久之,顯宗問固「卿弟安在」,固對「為官寫書,受直以養老母」。帝乃除超為蘭臺令史,後坐事免官。

十六年,奉車都尉竇固出擊匈奴,以超為假司馬,將兵別擊伊吾,戰於蒲類海,多斬首虜而還。固以為能,遣與從事郭恂俱使西域。

超到鄯善,鄯善王廣奉超禮敬甚備,後忽更疏懈。超謂其官屬曰:「寧覺廣禮意薄乎?此必有北虜使來,狐疑未知所從故也。明者睹未萌,況已著邪。」乃召侍胡詐之曰:「匈奴使來數日,今安在乎?」侍胡惶恐,具服其狀。超乃閉侍胡,悉會其吏士三十六人,與共飲,酒酣,因激怒之曰:「卿曹與我俱在絕域,欲立大功,以求富貴。今虜使到裁數日,而王廣禮敬即廢;如令鄯善收吾屬送匈奴,骸骨長為豺狼食矣。為之柰何?」官屬皆曰:「今在危亡之地,死生從司馬。」超曰:「不入虎穴,不得虎子。當今之計,獨有因夜以火攻虜,使彼不知我多少,必大震怖,可殄盡也。滅此虜,則鄯善破膽,功成事立矣。」眾曰:「當與從事議之。」超怒曰:「吉凶決於今日。從事文俗吏,聞此必恐而謀泄,死無所名,非壯士也!」眾曰:「善」。初夜,遂將吏士往奔虜營。會天大風,超令十人持鼓藏虜舍後,約曰:「見火然,皆當鳴鼓大呼。」餘人悉持兵弩夾門而伏。超乃順風縱火,前後鼓噪。虜眾驚亂,超手格殺三人,吏兵斬其使及從士三十餘級,餘眾百許人悉燒死。明日乃還告郭恂,恂大驚,既而色動。超知其意,舉手曰:「掾雖不行,班超何心獨擅之乎?」恂乃悅。超於是召鄯善王廣,以虜使首示之,一國震怖。超曉告撫慰,遂納子為質。還奏於竇固,固大喜,具上超功效,并求更選使使西域。帝壯超節,詔固曰:「吏如班超,何故不遣而更選乎?今以超為軍司馬,令遂前功。」超復受使,固欲益其兵,超曰:「願將本所從三十餘人足矣。如有不虞,多益為累。」

是時于窴王廣德新攻破莎車,遂雄張南道,而匈奴遣使監護其國。超既西,先至于窴。廣德禮意甚疏。且其俗信巫。巫言:「神怒何故欲向漢?漢使有騧馬,急求取以祠我。」廣德乃遣使就超請馬。超密知其狀,報許之,而令巫自來取馬。有頃,巫至,超即斬其首以送廣德,因辭讓之。廣德素聞超在鄯善誅滅虜使,大惶恐,即攻殺匈奴使者而降超。超重賜其王以下,因鎮撫焉。

時龜茲王建為匈奴所立,倚恃虜威,據有北道,攻破疏勒,殺其王,而立龜茲人兜題為疏勒王。明年春,超從閒道至疏勒。去兜題所居槃橐城九十里,逆遣吏田慮先往降之。敕慮曰:「兜題本非疏勒種,國人必不用命。若不即降,便可執之。」慮既到,兜題見慮輕弱,殊無降意。慮因其無備,遂前劫縛兜題。左右出其不意,皆驚懼奔走。慮馳報超,超即赴之,悉召疏勒將吏,說以龜茲無道之狀,因立其故王兄子忠為王,國人大悅。忠及官屬皆請殺兜題,超不聽,欲示以威信,釋而遣之。疏勒由是與龜茲結怨。

十八年,帝崩。焉耆以中國大喪,遂攻沒都護陳睦。超孤立無援,而龜茲、姑墨數發兵攻疏勒。超守盤橐城,與忠為首尾,士吏單少,拒守歲餘。肅宗初即位,以陳睦新沒,恐超單危不能自立,下詔徵超。超發還,疏勒舉國憂恐。其都尉黎弇曰:「漢使棄我,我必復為龜茲所滅耳。誠不忍見漢使去。」因以刀自剄。超還至于窴,王侯以下皆號泣曰:「依漢使如父母,誠不可去。」互抱超馬腳,不得行。超恐于窴終不聽其東,又欲遂本志,乃更還疏勒。疏勒兩城自超去後,復降龜茲,而與尉頭連兵。超捕斬反者,擊破尉頭,殺六百餘人,疏勒復安。

建初三年,超率疏勒、康居、于窴、拘彌兵一萬人攻姑墨石城,破之,斬首七百級。超欲因此叵平諸國,乃上疏請兵。曰:「臣竊見先帝欲開西域,故北擊匈奴,西使外國,鄯善、于窴即時向化。今拘彌、莎車、疏勒、月氏、烏孫、康居復願歸附,欲共并力破滅龜茲,平通漢道。若得龜茲,則西域未服者百分之一耳。臣伏自惟念,卒伍小吏,實願從谷吉效命絕域,庶幾張騫棄身曠野。昔魏絳列國大夫,尚能和輯諸戎,況臣奉大漢之威,而無鈆刀一割之用乎?前世議者皆曰取三十六國,號為斷匈奴右臂。今西域諸國,自日之所入,莫不向化,大小欣欣,貢奉不絕,唯焉耆、龜茲獨未服從。臣前與官屬三十六人奉使絕域,備遭艱厄。自孤守疏勒,於今五載,胡夷情數,臣頗識之。問其城郭小大,皆言『倚漢與依天等』。以是效之,則蔥領可通,蔥領通則龜茲可伐。今宜拜龜茲侍子白霸為其國王,以步騎數百送之,與諸國連兵,歲月之閒,龜茲可禽。以夷狄攻夷狄,計之善者也。臣見莎車、疏勒田地肥廣,草牧饒衍,不比敦煌,鄯善閒也,兵可不費中國而糧食自足。且姑墨、溫宿二王,特為龜茲所置,既非其種,更相厭苦,其埶必有降反。若二國來降,則龜茲自破。願下臣章,參考行事。誠有萬分,死復何恨。臣超區區,特蒙神靈,竊冀未便僵仆,目見西域平定,陛下舉萬年之觴,薦勳祖廟,布大喜於天下。」書奏,帝知其功可成,議欲給兵。平陵人徐幹素與超同志,上疏願奮身佐超。五年,遂以幹為假司馬,將弛刑及義從千人就超。

先是莎車以為漢兵不出,遂降於龜茲,而疏勒都尉番辰亦復反叛。會徐幹適至,超遂與幹擊番辰,大破之,斬首千餘級,多獲生口。超既破番辰,欲進攻龜茲。以烏孫兵彊,宜因其力,乃上言:「烏孫大國,控弦十萬,故武帝妻以公主,至孝宣皇帝,卒得其用。今可遣使招慰,與共合力。」帝納之。八年,拜超為將兵長史,假鼓吹幢麾。以徐幹為軍司馬,別遣衛候李邑護送烏孫使者,賜大小昆彌以下錦帛。

李邑始到于窴,而值龜茲攻疏勒,恐懼不敢前,因上書陳西域之功不可成,又盛毀超擁愛妻,抱愛子,安樂外國,無內顧心。超聞之,歎曰:「身非曾參而有三至之讒,恐見疑於當時矣。」遂去其妻。帝知超忠,乃切責邑曰:「縱超擁愛妻,抱愛子,思歸之士千餘人,何能盡與超同心乎?」令邑詣超受節度。詔超:「若邑任在外者,便留與從事。」超即遣邑將烏孫侍子還京師。徐幹謂超曰:「邑前親毀君,欲敗西域,今何不緣詔書留之,更遣它吏送侍子乎?」超曰:「是何言之陋也!以邑毀超,故今遣之。內省不疚,何卹人言!快意留之,非忠臣也。」

明年,復遣假司馬和恭等四人將兵八百詣超,超因發疏勒、于窴兵擊莎車。莎車陰通使疏勒王忠,啖以重利,忠遂反從之,西保烏即城。超乃更立其府丞成大為疏勒王,悉發其不反者以攻忠。積半歲,而康居遣精兵救之,超不能下。是時月氏新與康居婚,相親,超乃使使多齎錦帛遺月氏王,令曉示康居王,康居王乃罷兵,執忠以歸其國,烏即城遂降於超。

後三年,忠說康居王借兵,還據損中,密與龜茲謀,遣使詐降於超。超內知其姦而外偽許之。忠大喜,即從輕騎詣超。超密勒兵待之,為供張設樂。酒行,乃叱吏縛忠斬之。因擊破其眾,殺七百餘人,南道於是遂通。

明年,超發于窴諸國兵二萬五千人,復擊莎車。而龜茲王遣左將軍發溫宿、姑墨、尉頭合五萬人救之。超召將校及于窴王議曰:「今兵少不敵,其計莫若各散去。于窴從是而東,長史亦於此西歸,可須夜鼓聲而發。」陰緩所得生口。龜茲王聞之大喜,自以萬騎於西界遮超,溫宿王將八千騎於東界徼于窴。超知二虜已出,密召諸部勒兵,雞鳴馳赴莎車營,胡大驚亂奔走,追斬五千餘級,大獲其馬畜財物。莎車遂降,龜茲等因各退散,自是威震西域。

初,月氏嘗助漢擊車師有功,是歲貢奉珍寶、符拔、師子,因求漢公主。超拒還其使,由是怨恨。永元二年,月氏遣其副王謝將兵七萬攻超。超眾少,皆大恐。超譬軍士曰:「月氏兵雖多,然數千里踰蔥領來,非有運輸,何足憂邪?但當收穀堅守,彼飢窮自降,不過數十日決矣。」謝遂前攻超,不下,又鈔掠無所得。超度其糧將盡,必從龜茲求救,乃遣兵數百於東界要之。謝果遣騎齎金銀珠玉以賂龜茲。超伏兵遮擊,盡殺之,持其使首以示謝。謝大驚,即遣使請罪,願得生歸。超縱遣之。月氏由是大震,歲奉貢獻。

明年,龜茲、姑墨、溫宿皆降,乃以超為都護,徐幹為長史。拜白霸為龜茲王,遣司馬姚光送之。超與光共脅龜茲廢其王尤利多而立白霸,使光將尤利多還詣京師。超居龜茲它乾城,徐幹屯疏勒。西域唯焉耆、危須、尉犁以前沒都護,懷二心,其餘悉定。

六年秋,超遂發龜茲、鄯善等八國兵合七萬人,及吏士賈客千四百人討焉耆。兵到尉犁界,而遣曉說焉耆、尉犁、危須曰:「都護來者,欲鎮撫三國。即欲改過向善,宜遣大人來迎,當賞賜王侯已下,事畢即還。今賜王綵五百匹。」焉耆王廣遣其左將北鞬支奉牛酒迎超。超結鞬支曰:「汝雖匈奴侍子,而今秉國之權。都護自來,王不以時迎,皆汝罪也。」或謂超可便殺之。超曰:「非汝所及。此人權重於王,今未入其國而殺之,遂令自疑,設備守險,豈得到其城下哉!」於是賜而遣之。廣乃與大人迎超於尉犁,奉獻珍物。

焉耆國有葦橋之險,廣乃絕橋,不欲令漢軍入國。超更從它道厲度。七月晦,到焉耆,去城二十里,正營大澤中。廣出不意,大恐,乃欲悉驅其人共入山保。焉耆左候元孟先嘗質京師,密遣使以事告超,超即斬之,示不信用。乃期大會諸國王,因揚聲當重加賞賜,於是焉耆王廣、尉犁王汎及北鞬支等三十人相率詣超。其國相腹久等十七人懼誅,皆亡入海,而危須王亦不至。坐定,超怒詰廣曰:「危須王何故不到?腹久等所緣逃亡?」遂叱吏士收廣、汎等於陳睦故城斬之,傳首京師。因縱兵鈔掠,斬首五千餘級,獲生口萬五千人,馬畜牛羊三十餘萬頭,更立元孟為焉耆王。超留焉耆半歲,慰撫之。於是西域五十餘國悉皆納質內屬焉。

明年,下詔曰:「往者匈奴獨擅西域,寇盜河西,永平之末,城門晝閉。先帝深愍邊萌嬰羅寇害,乃命將帥擊右地,破白山,臨蒲類,取車師,城郭諸國震慴響應,遂開西域,置都護。而焉耆王舜、舜子忠獨謀悖逆,持其險隘,覆沒都護,并及吏士。先帝重元元之命,憚兵役之興,故使軍司馬班超安集于窴以西。超遂踰蔥領,迄縣度,出入二十二年,莫不賓從。改立其王,而綏其人。不動中國,不煩戎士,得遠夷之和,同異俗之心,而致天誅,蠲宿恥,以報將士之讎。司馬法曰:『賞不踰月,欲人速睹為善之利也。』其封超為定遠侯,邑千戶。」

超自以久在絕域,年老思土。十二年,上疏曰:「臣聞太公封齊,五世葬周,狐死首丘,代馬依風。夫周齊同在中土千里之閒,況於遠處絕域,小臣能無依風首丘之思哉?蠻夷之俗,畏壯侮老。臣超犬馬齒殲,常恐年衰,奄忽僵仆,孤魂棄捐。昔蘇武留匈奴中尚十九年,今臣幸得奉節帶金銀護西域,如自以壽終屯部,誠無所恨,然恐後世或名臣為沒西域。臣不敢望到酒泉郡,但願生入玉門關。臣老病衰困,冒死瞽言,謹遣子勇隨獻物入塞。及臣生在,令勇目見中土。」而超妹同郡曹壽妻昭亦上書請超曰:

妾同產兄西域都護定遠侯超,幸得以微功特蒙重賞,爵列通侯,位二千石。天恩殊絕,誠非小臣所當被蒙。超之始出,志捐軀命,冀立微功,以自陳效。會陳睦之變,道路隔絕,超以一身轉側絕域,曉譬諸國,因其兵眾,每有攻戰,輒為先登,身被金夷,不避死亡。賴蒙陛下神靈,且得延命沙漠,至今積三十年。骨肉生離,不復相識。所與相隨時人士眾,皆已物故。超年最長,今且七十。衰老被病,頭髮無黑,兩手不仁,耳目不聰明,扶杖乃能行。雖欲竭盡其力,以報塞天恩,迫於歲暮,犬馬齒索。蠻夷之性,悖逆侮老,而超旦暮入地,久不見代,恐開姦宄之源,生逆亂之心。而卿大夫咸懷一切,莫肯遠慮。如有卒暴,超之氣力不能從心,便為上損國家累世之功,下棄忠臣竭力之用,誠可痛也。故超萬里歸誠,自陳苦急,延頸踰望,三年於今,未蒙省錄。

妾竊聞古者十五受兵,六十還之,亦有休息不任職也。緣陛下以至孝理天下,得萬國之歡心,不遺小國之臣,況超得備侯伯之位,故敢觸死為超求哀,饨超餘年。一得生還,復見闕庭,使國永無勞遠之慮,西域無倉卒之憂,超得長蒙文王葬骨之恩,子方哀老之惠。《詩》云:「民亦勞止,汔可小康,惠此中國,以綏四方。」超有書與妾生訣,恐不復相見。妾誠傷超以壯年竭忠孝於沙漠,疲老則便捐死於曠野,誠可哀憐。如不蒙救護,超後有一旦之變,冀幸超家得蒙趙母、衛姬先請之貸。妾愚戇不知大義,觸犯忌諱。書奏,帝感其言,乃徵超還。

超在西域三十一歲。十四年八月至洛陽,拜為射聲校尉。超素有匈脅疾,既至,病遂加。帝遣中黃門問疾,賜醫藥。其年九月卒,年七十一。朝廷愍惜焉,使者弔祭,贈賵甚厚。子雄嗣。

初,超被徵,以戊己校尉任尚為都護。與超交代。尚謂超曰:「君侯在外國三十餘年,而小人猥承君後,任重慮淺,宜有以誨之。」超曰:「年老失智,任君數當大位,豈班超所能及哉!必不得已,願進愚言。塞外吏士,本非孝子順孫,皆以罪過徙補邊屯。而蠻夷懷鳥獸之心,難養易敗。今君性嚴急,水清無大魚,察政不得下和。宜蕩佚簡易,寬小過,總大綱而已。」超去後,尚私謂所親曰:「我以班君當有奇策,今所言平平耳。」尚至數年,而西域反亂,以罪被徵,如超所戒。

有三子。長子雄,累遷屯騎校尉。會叛羌寇三輔,詔雄將五營兵屯長安,就拜京兆尹。雄卒,子始嗣,尚清河孝王女陰城公主。主順帝之姑,貴驕淫亂,與嬖人居帷中,而召始入,使伏床下。始積怒,永建五年,遂拔刃殺主。帝大怒,腰斬始,同產皆棄巿。超少子勇。

勇字宜僚,少有父風。永初元年,西域反叛,以勇為軍司馬。與兄雄俱出敦煌,迎都護及西域甲卒而還。因罷都護。後西域絕無漢吏十餘年。

元初六年,敦煌太守曹宗遣長史索班將千餘人屯伊吾,車師前王及鄯善王皆來降班。後數月,北單于與車師後部遂共攻沒班,進擊走前王,略有北道。鄯善王急,求救於曹宗,宗因此請出兵五千人擊匈奴,報索班之恥,因復取西域。鄧太后召勇詣朝堂會議。先是公卿多以為宜閉玉門關,遂棄西域。勇上議曰:「昔孝武皇帝患匈奴彊盛,兼總百蠻,以逼障塞。於是開通西域,離其黨與,論者以為奪匈奴府藏,斷其右臂。遭王莽篡盜,徵求無猒,胡夷忿毒,遂以背叛。光武中興,未遑外事,故匈奴負彊,驅率諸國。及至永平,再攻敦煌,河西諸郡,城門晝閉。孝明皇帝深惟廟策,乃命虎臣,出征西域,故匈奴遠遁,邊境得安。及至永元,莫不內屬。會閒者羌亂,西域復絕,北虜遂遣責諸國,備其逋租,高其價直,嚴以期會。鄯善、車師皆懷憤怨,思樂事漢,其路無從。前所以時有叛者,皆由牧養失宜,還為其害故也。今曹宗徒恥於前負,欲報雪匈奴,而不尋出兵故事,未度當時之宜也。夫要功荒外,萬無一成,若兵連禍結,悔無及已。況今府藏未充,師無後繼,是示弱於遠夷,暴短於海內,臣愚以為不可許也。舊敦煌郡有營兵三百人,今宜復之,復置護西域副校尉,居於敦煌,如永元故事。又宜遣西域長史將五百人屯樓蘭,西當焉耆、龜茲徑路,南彊鄯善、于窴心膽,北扞匈奴,東近敦煌。如此誠便。」

尚書問勇曰:「今立副校尉,何以為便?又置長史屯樓蘭,利害云何?」勇對曰:「昔永平之末,始通西域,初遣中郎將居敦煌,後置副校於車師,既為胡虜節度,又禁漢人不得有所侵擾。故外夷歸心,匈奴畏威。今鄯善王尤還,漢人外孫,若匈奴得志,則尤還必死。此等雖同鳥獸,亦知避害。若出屯樓蘭,足以招附其心,愚以為便。」長樂衛尉鐔顯、廷尉綦母參、司隸校尉崔據難曰:「朝廷前所以棄西域者,以其無益於中國而費難供也。今車師已屬匈奴,鄯善不可保信,一旦反覆,班將能保北虜不為邊害乎?」勇對曰:「今中國置州牧者,以禁郡縣姦猾盜賊也。若州牧能保盜賊不起者,臣亦願以要斬保匈奴之不為邊害也。今通西域則虜埶必弱,虜埶必弱則為患微矣。孰與歸其府藏,續其斷臂哉!今置校尉以扞撫西域,設長史以招懷諸國,若棄而不立,則西域望絕。望絕之後,屈就北虜,緣邊之郡將受困害,恐河西城門必復有晝閉之儆矣。今不廓開朝廷之德,而拘屯戍之費,若北虜遂熾,豈安邊久長之策哉!」太尉屬毛軫難曰:「今若置校尉,則西域駱驛遣使,求索無猒,與之則費難供,不與則失其心。一旦為匈奴所迫,當復求救,則為役大矣。」勇對曰:「今設以西域歸匈奴,而使其恩德大漢,不為鈔盜則可矣。如其不然,則因西域租入之饒,兵馬之眾,以擾動緣邊,是為富仇讎之財,增暴夷之埶也。置校尉者,宣威布德,以繫諸國內向之心,以疑匈奴覬覦之情,而無財費耗國之慮也。且西域之人無它求索,其來入者,不過稟食而已。今若拒絕,埶歸北屬,夷虜并力以寇并、涼,則中國之費不止千億。置之誠便。」於是從勇議,復敦煌郡營兵三百人,置西域副校尉居敦煌。雖復羈縻西域,然亦未能出屯。其後匈奴果數與車師共入寇鈔,河西大被其害。

延光二年夏,復以勇為西域長史,將兵五百人出屯柳中。明年正月,勇至樓蘭,以鄯善歸附,特加三綬。而龜茲王白英猶自疑未下,勇開以恩信,白英乃率姑墨、溫宿自縛詣勇降。勇因發其兵步騎萬餘人到車師前王庭,擊走匈奴伊蠡王於伊和谷,收得前部五千餘人,於是前部始復開通。還,屯田柳中。

四年秋,勇發敦煌、張掖、酒泉六千騎及鄯善、疏勒、車師前部兵擊後部王軍就,大破之。首虜八千餘人,馬畜五萬餘頭。捕得軍就及匈奴持節使者,將至索班沒處斬之,以報其恥,傳首京師。永建元年,更立後部故王子加特奴為王。勇又使別校誅斬東且彌王,亦更立其種人為王,於是車師六國悉平。

其冬,勇發諸國兵擊匈奴呼衍王,呼衍王亡走,其眾二萬餘人皆降。捕得單于從兄,勇使加特奴手斬之,以結車師匈奴之隙。北單于自將萬餘騎入後部,至金且谷,勇使假司馬曹俊馳救之。單于引去,俊追斬其貴人骨都侯,於是呼衍王遂徙居枯梧河上。是後車師無復虜跡,城郭皆安。唯焉耆王元孟未降。

二年,勇上請攻元孟,於是遣敦煌太守張朗將河西四郡兵三千人配勇。因發諸國兵四萬餘人,分騎為兩道擊之。勇從南道,朗從北道,約期俱至焉耆。而朗先有罪,欲徼功自贖,遂先期至爵離關,遣司馬將兵前戰,首虜二千餘人。元孟懼誅,逆遣使乞降,張朗徑入焉耆受降而還。元孟竟不肯面縛,唯遣子詣闕貢獻。朗遂得免誅。勇以後期,徵下獄,免。後卒于家。

梁慬字伯威,北地弋居人也。父諷,歷州宰。永元元年,車騎將軍竇憲出征匈奴,除諷為軍司馬,令先齎金帛使北單于,宣國威德,其歸附者萬餘人。後坐失憲意,髡輸武威,武威太守承旨殺之。竇氏既滅,和帝知其為憲所誣,徵慬,除為郎中。

慬有勇氣,常慷慨好功名。初為車騎將軍鄧鴻司馬,再遷,延平元年拜西域副校尉。慬行至河西,會西域諸國反叛,攻都護任尚於疏勒。尚上書求救,詔慬將河西四郡羌胡五千騎馳赴之,慬未至而尚已得解。會徵尚還,以騎都尉段禧為都護,西域長史趙博為騎都尉。禧、博守它乾城。它乾城小,慬以為不可固,乃譎說龜茲王白霸,欲入共保其城,白霸許之。吏人固諫,白霸不聽。慬既入,遣將急迎禧、博,合軍八九千人。龜茲吏人並叛其王,而與溫宿、姑墨數萬兵反,共圍城。慬等出戰,大破之。連兵數月,胡眾敗走,乘勝追擊,凡斬首萬餘級,獲生口數千人,駱駝畜產數萬頭,龜茲乃定。而道路尚隔,檄書不通。歲餘,朝廷憂之。公卿議者以為西域阻遠,數有背叛,吏士屯田,其費無已。永初元年,遂罷都護,遣騎都尉王弘發關中兵迎慬、禧、博及伊吾盧、柳中屯田吏士。

二年春,還至敦煌。會眾羌反叛,朝廷大發兵西擊之,逆詔慬留為諸軍援。慬至張掖日勒。羌諸種萬餘人攻亭候,殺略吏人。慬進兵擊,大破之,乘勝追至昭武,虜遂散走,其能脫者十二三。及至姑臧,羌大豪三百餘人詣慬降,並尉譬遣還故地,河西四郡復安。

慬受詔當屯金城,聞羌轉寇三輔,迫近園陵,即引兵赴擊之,轉戰武功美陽關。慬臨陣被創,不顧,連破走之,盡還得所掠生口,獲馬畜財物甚眾,羌遂奔散。朝廷嘉之,數璽書勞勉,委以西方事,令為諸軍節度。

三年冬,南單于與烏桓大人俱反。以大司農何熙行車騎將軍事,中郎將龐雄為副,將羽林五校營士,及發緣邊十郡兵二萬餘人,又遼東太守耿夔率將鮮卑種眾共擊之,詔慬行度遼將軍事。龐雄與耿夔共擊匈奴奧鞬日逐王,破之。單于乃自將圍中郎將耿种於美稷,連戰數月,攻之轉急,种移檄求救。明年正月,慬將八千餘人馳往赴之,至屬國故城,與匈奴左將軍、烏桓大人戰,破斬其渠帥,殺三千餘人,虜其妻子,獲財物甚眾。單于復自將七八千騎迎攻,圍慬。慬被甲奔擊,所向皆破,虜遂引還虎澤。三月,何熙軍到五原曼柏,暴疾,不能進,遣龐雄與慬及耿种步騎萬六千人攻虎澤。連營稍前,單于惶怖,遣左奧鞬日逐王詣慬乞降,慬乃大陳兵受之。單于脫帽徒跣,面縛稽顙,納質。會熙卒于師,即拜慬度遼將軍。龐雄還為大鴻臚。雄,巴郡人,有勇略,稱為名將。

明年,安定、北地、上郡皆被羌寇,穀貴人流,不能自立。詔慬發邊兵迎三郡太守,使將吏人徙扶風界。慬即遣南單于兄子優孤塗奴將兵迎之。既還,慬以塗奴接其家屬有勞,輒授以羌侯印綬,坐專擅,徵下獄,抵罪。明年,校書郎馬融上書訟慬與護羌校尉龐參,有詔原刑。語在龐參傳。

會叛羌寇三輔,關中盜賊起,拜慬謁者,將兵擊之。至湖縣,病卒。

何熙字孟孫,陳國人。少有大志。永元中,為謁者。身長八尺五寸,善為威容,贊拜殿中,音動左右。和帝偉之,擢為御史中丞,歷司隸校尉、大司農。及在軍臨歿,遺言薄葬。三子:臨,瑾,阜。臨、瑾並有政能。阜俊才早沒。臨子衡,為尚書,以正直稱,坐訟李膺等下獄,免官,廢于家。

論曰:時政平則文德用,而武略之士無所奮其力能,故漢世有發憤張膽,爭膏身於夷狄以要功名,多矣。祭肜、耿秉啟匈奴之權,班超、梁慬奮西域之略,卒能成功立名,享受爵位,薦功祖廟,勒勳于後,亦一時之志士也。

贊曰:定遠慷慨,專功西遐。坦步蔥、雪,咫尺龍沙。慬亦抗憤,勇乃負荷。


\end{pinyinscope}