\article{任李萬邳劉耿列傳}

\begin{pinyinscope}
任光字伯卿,南陽宛人也。少忠厚,為鄉里所愛。初為鄉嗇夫,郡縣吏。漢兵至宛,軍人見光冠服鮮明,令解衣,將殺而奪之。會光祿勳劉賜適至,視光容貌長者,乃救全之。光因率黨與從賜,為安集掾,拜偏將軍,與世祖破王尋、王邑。

更始至洛陽,以光為信都太守。及王郎起,郡國皆降之,光獨不肯,遂與都尉李忠、令萬脩、功曹阮況、五官掾郭唐等同心固守。廷掾持王郎檄詣府白光,光斬之於巿,以徇百姓,發精兵四千人城守。更始二年春,世祖自薊還,狼狽不知所向,傳聞信都獨為漢拒邯鄲,即馳赴之。光等孤城獨守,恐不能全,聞世祖至,大喜,吏民皆稱萬歲,即時開門,與李忠、萬脩率官屬迎謁。世祖入傳舍,謂光曰:「伯卿。今埶力虛弱,欲俱入城頭子路、力子都兵中,何如邪?」光曰:「不可。」世祖曰:「卿兵少,何如?」光曰:「可募發奔命,出攻傍縣,若不降者,恣聽掠之。人貪財物,則兵可招而致也。」世祖從之。拜光為左大將軍,封武成侯,留南陽宗廣領信都太守事,使光將兵從。光乃多作檄文曰:「大司馬劉公將城頭子路、力子都兵百萬眾從東方來,擊諸反虜。」遣騎馳至鉅鹿界中。吏民得檄,傳相告語。世祖遂與光等投暮入堂陽界,使騎各持炬火,彌滿澤中,光炎燭天地,舉城莫不震驚惶怖,其夜即降。旬日之閒,兵眾大盛,因攻城邑,遂屠邯鄲,迺遣光歸郡。

城頭子路者,東平人,姓爰,名曾,字子路,與肥城劉詡起兵盧城頭,故號其兵為「城頭子路」。曾自稱「都從事」,詡稱「

校三老」,寇掠河、濟閒,眾至二十餘萬。更始立,曾遣使降,拜曾東萊郡太守,詡濟南太守,皆行大將軍事。是歲,曾為其將所殺,眾推詡為主,更始封詡助國侯,令罷兵歸本郡。

力子都者,東海人也。起兵鄉里,鈔擊徐、兗界,眾有六七萬。更始立,遣使降,拜子都徐州牧。為其部曲所殺,餘黨復相聚,與諸賊會於檀鄉,因號為檀鄉。檀鄉渠帥董次仲始起茌平,遂渡河入魏郡清河,與五校合,眾十餘萬。建武元年,世祖入洛陽,遣大司馬吳漢等擊檀鄉,明年春,大破降之。

是歲,更封光阿陵侯,食邑萬戶。五年,徵詣京師,奉朝請。其冬卒。子隗嗣。

後阮況為南陽太守,郭唐至河南尹,皆有能名。

隗字仲和,少好黃老,清靜寡欲,所得奉秩,常以賑卹宗族,收養孤寡。顯宗聞之,擢奉朝請,遷羽林左監、虎賁中郎將,又遷長水校尉。肅宗即位,雅相敬愛,數稱其行,以為將作大匠。將作大匠自建武以來常謁者兼之,至隗迺置真焉。建初五年,遷太僕,八年,代竇固為光祿勳,所歷皆有稱。章和元年,拜司空。

隗義行內修,不求名譽,而以沈正見重於世。和帝即位,大將軍竇憲秉權,專作威福,內外朝臣莫不震懾。時憲擊匈奴,國用勞費,隗奏議徵憲還,前後十上。獨與司徒袁安同心畢力,持重處正,鯁言直議,無所回隱,語在袁安傳。

永元四年薨,子屯嗣。帝追思隗忠,擢屯為步兵校尉,徙封西陽侯。

屯卒,子勝嗣。勝卒,子世嗣,徙封北鄉侯。

李忠字仲都,東萊黃人也。父為高密都尉。忠元始中以父任為郎,署中數十人,而忠獨以好禮修整稱。王莽時為新博屬長,郡中咸敬信之。

更始立,使使者行郡國,即拜忠都尉官。忠遂與任光同奉世祖,以為右大將軍,封武固侯。時世祖自解所佩綬以帶忠,因從攻下屬縣。至苦陘,世祖會諸將,問所得財物,唯忠獨無所掠。世祖曰:「我欲特賜李忠,諸卿得無望乎?」即以所乘大驪馬及繡被衣物賜之。

進圍鉅鹿,未下,王郎遣將攻信都,信都大姓馬寵等開城內之,收太守宗廣及忠母妻,而令親屬招呼忠。時寵弟從忠為校尉,忠即時召見,責數以背恩反城,因格殺之。諸將皆驚曰:「家屬在人手中,殺其弟,何猛也!」忠曰:「若縱賊不誅,則二心也。」世祖聞而美之,謂忠曰:「今吾兵已成矣,將軍可歸救老母妻子,宜自募吏民能得家屬者,賜錢千萬,來從我取。」忠曰:「蒙明公大恩,思得效命,誠不敢內顧宗親。」世祖迺使任光將兵救信都,光兵於道散降王郎,無功而還。會更始遣將攻破信都,忠家屬得全。世祖因使忠還,行太守事,收郡中大姓附邯鄲者,誅殺數百人。及任光歸郡,忠迺還復為都尉。建武二年,更封中水侯,食邑三千戶。其年,徵拜五官中郎將,從平龐萌、董憲等。

六年,遷丹陽太守。是時海內新定,南方海濱江淮,多擁兵據土。忠到郡,招懷降附,其不服者悉誅之,旬月皆平。忠以丹陽越俗不好學,嫁娶禮儀,衰於中國,乃為起學校,習禮容,春秋鄉飲,選用明經,郡中向慕之。墾田增多,三歲閒流民占著者五萬餘口。十四年,三公奏課為天下第一,遷豫章太守。病去官,徵詣京師。十九年,卒。

子威嗣,威卒,子純嗣,永平九年,坐母殺純叔父,國除。永初七年,鄧太后復封純琴亭侯。純卒,子廣嗣。

萬脩字君游,扶風茂陵人也。更始時,為信都令,與太守任光、都尉李忠共城守,迎世祖,拜為偏將軍,封造義侯。及破邯鄲,拜右將軍,從平河北。建武二年,更封槐里侯。與揚化將軍堅鐔俱擊南陽,未剋而病,卒于軍。

子普嗣,徙封泫氏侯。普卒,子親嗣,徙封扶柳侯。親卒,無子,國除。永初七年,鄧太后紹封脩曾孫豐為曲平亭侯。豐卒,子熾嗣。永建元年,熾卒,無子,國除。延熹二年,桓帝紹封脩玄孫恭為門德亭侯。

邳彤字偉君,信都人也。父吉,為遼西太守。彤初為王莽和成卒正。世祖徇河北,至下曲陽,彤舉城降,復以為太守,留止數日。世祖北至薊,會王郎兵起,使其將徇地,所到縣莫不奉迎,唯和成、信都堅守不下。彤聞世祖從薊還,失軍,欲至信都,乃先使五官掾張萬、督郵尹綏,選精騎二千餘匹,掾路迎世祖軍。彤尋與世祖會信都。世祖雖得二郡之助,而兵眾未合,議者多言可因信都兵自送,西還長安。彤廷對曰:「議者之言皆非也。吏民歌吟思漢久矣,故更始舉尊號而天下響應,三輔清宮除道以迎之。一夫荷戟大呼,則千里之將無不捐城遁逃,虜伏請降。自上古以來,亦未有感物動民其如此者也。又卜者王郎,假名因埶,驅集烏合之眾,遂震燕、趙之地;況明公奮二郡之兵,揚響應之威,以攻則何城不克,以戰則何軍不服!今釋此而歸,豈徒空失河北,必更驚動三輔,墮損威重,非計之得者也。若明公無復征伐之意,則雖信都之兵猶難會也。何者?明公既西,即邯鄲城民不肯捐父母,背城主,而千里送公,其離散亡逃可必也。」世祖善其言而止。即日拜彤為後大將軍,和成太守如故,使將兵居前。比至堂陽,堂陽已反屬王郎,彤使張萬、尹綏先曉譬吏民,世祖夜至,即開門出迎。引兵擊破白奢賊於中山。自此常從戰攻。

信都復反為王郎,郎所置信都王捕繫彤父弟及妻子,使為手書呼彤曰:「降者封爵,不降族滅。」彤涕泣報曰:「事君者不得顧家。彤親屬所以至今得安於信都者,劉公之恩也。公方爭國事,彤不得復念私也。」會更始所遣將攻拔信都,郎兵敗走,彤家屬得免。

及拔邯鄲,封武義侯。建武元年,更封靈壽侯,行大司空事。帝入洛陽,拜彤太常,月餘日轉少府,是年免。復為左曹侍中,常從征伐。六年,就國。

彤卒,子湯嗣,九年,徙封樂陵侯。十九年,湯卒,子某嗣;無子,國除。元初元年,鄧太后紹封彤孫音為平亭侯。音卒,子柴嗣。

初,張萬、尹綏與彤俱迎世祖,皆拜偏將軍,亦從征伐。萬封重平侯,綏封平臺侯。

論曰:凡言成事者,以功著易顯;謀幾初者,以理隱難昭。斯固原情比跡,所宜推察者也。若迺議者欲因二郡之眾,建入關之策,委成業,臨不測,而世主未悟,謀夫景同,邳彤之廷對,其為幾乎!語曰「一言可以興邦」,斯近之矣。

劉植字伯先,鉅鹿昌城人也。王郎起,植與弟喜、從兄歆率宗族賓客,聚兵數千人據昌城。聞世祖從薊還,迺開門迎世祖,以植為驍騎將軍,喜、歆偏將軍,皆為列侯。時真定王劉揚起兵以附王郎,眾十餘萬,世祖遣植說揚,揚迺降。世祖因留真定,納郭后,后即揚之甥也,故以此結之。迺與揚及諸將置酒郭氏漆里舍,揚擊筑為歡,因得進兵拔邯鄲,從平河北。

建武二年,更封植為昌城侯。討密縣賊,戰歿。子向嗣。帝使喜代將植營,復為驍騎將軍,封觀津侯。喜卒,復以歆為驍騎將軍,封浮陽侯。喜、歆從征伐,皆傳國于後。向徙封東武陽侯,卒,子述嗣,永平十五年,坐與楚王英謀反,國除。

耿純字伯山,鉅鹿宋子人也。父艾,為王莽濟平尹。純學於長安,因除為納言士。

王莽敗,更始立,使舞陰王李軼降諸郡國,純父艾降,還為濟南太守。時李軼兄弟用事,專制方面,賓客游說者甚眾。純連求謁不得通,久之迺得見,因說軼曰:「大王以龍虎之姿,遭風雲之時,奮迅拔起,期月之閒兄弟稱王,而德信不聞於士民,功勞未施於百姓,寵祿暴興,此智者之所忌也。兢兢自危,猶懼不終,而況沛然自足,可以成功者乎?」軼奇之,且以其鉅鹿大姓,迺承制拜為騎都尉,授以節,令安集趙、魏。

會世祖度河至邯鄲,純即謁見,世祖深接之。純退,見官屬將兵法度不與它將同,遂求自結納,獻馬及縑帛數百匹。世祖北至中山,留純邯鄲。會王郎反,世祖自薊東南馳,純與從昆弟訢、宿、植共率宗族賓客二千餘人,老病者皆載木自隨,奉迎於育。拜純為前將軍,封耿鄉侯,訢、宿、植皆偏將軍,使與純居前,降宋子,從攻下曲陽及中山。

是時郡國多降邯鄲者,純恐宗家懷異心,迺使訢、宿歸燒其廬舍。世祖問純故,對曰:「竊見明公單車臨河北,非有府臧之蓄,重賞甘餌,可以聚人者也,徒以恩德懷之,是故士眾樂附。今邯鄲自立,北州疑惑,純雖舉族歸命,老弱在行,猶恐宗人賓客半有不同心者,故燔燒屋室,絕其反顧之望。」世祖歎息。及至鄗,世祖止傳舍,鄗大姓蘇公反城開門內王郎將李惲。純先覺知,將兵逆與惲戰,大破斬之。從平邯鄲,又破銅馬。

時赤眉、青犢、上江、大彤、鐵脛、五幡十餘萬

眾並在射犬,世祖引兵將擊之。純軍在前,去眾營數里,賊忽夜攻純,雨射營中,士多死傷。純勒部曲,堅守不動。選敢死二千人,俱持彊弩,各傅三矢,使銜枚閒行,繞出賊後,齊聲呼譟,彊弩並發,賊眾驚走,追擊,遂破之。馳騎白世祖。世祖明旦與諸將俱至營,勞純曰:「昨夜困乎?」純曰:「賴明公威德,幸而獲全。」世祖曰:「大兵不可夜動,故不相救耳。軍營進退無常,卿宗族不可悉居軍中。」迺以純族人耿伋為蒲吾長,悉令將親屬居焉。

世祖即位,封純高陽侯。擊劉永於濟陰,下定陶。初,純從攻王郎,墯馬折肩,時疾發,迺還詣懷宮。帝問「卿兄弟誰可使者」,純舉從弟植,於是使植將純營,純猶以前將軍從。

時真定王劉揚復造作讖記云:「赤九之後,癭揚為主。」揚病癭,欲以惑眾,與綿曼賊交通。建武二年春,遣騎都尉陳副、游擊將軍鄧隆徵揚,揚閉城門,不內副等。乃復遣純持節,行赦令於幽、冀,所過並使勞慰王侯。密敕純曰:「劉揚若見,因而收之。」純從吏士百餘騎與副、隆會元氏,俱至真定,止傳舍。揚稱病不謁,以純真定宗室之出,遣使與純書,欲相見。純報曰:「奉使見王侯牧守,不得先詣,如欲面會,宜出傳舍。」時揚弟林邑侯讓及從兄細各擁兵萬餘人,揚自恃眾強而純意安靜,即從官屬詣之,兄弟並將輕兵在門外。揚入見純,純接以禮敬,因延請其兄弟,皆入,迺閉閤悉誅之,因勒兵而出。真定震怖,無敢動者。帝憐揚、讓謀未發,並封其子,復故國。

純還京師,因自請曰:「臣本吏家子孫,幸遭大漢復興,聖帝受命,備位列將,爵為通侯。天下略定,臣無所用志,願試治一郡,盡力自效。」帝笑曰:「卿既治武,復欲修文邪?」迺拜純為東郡太守。時東郡未平,純視事數月,盜賊清寧。四年,詔純將兵擊更始東平太守范荊,荊降。進擊太山濟南及平原賊,皆平之。居東郡四歲,時發干長有罪,純案奏,圍守之,奏未下,長自殺。純坐免,以列侯奉朝請。從擊董憲,道過東郡,百姓老小數千隨車駕涕泣。云「願復得耿君」。帝謂公卿曰:「純年少被甲冑為軍吏耳。治郡迺能見思若是乎?」

六年,定封為東光侯。純辭就國,帝曰:「文帝謂周勃『丞相吾所重,君為我率諸侯就國』,今亦然也。」純受詔而去。至鄴,賜穀萬斛。到國,弔死問病,民愛敬之。八年,東郡、濟陰盜賊群起,遣大司空李通、橫野大將軍王常擊之。帝以純威信著於衛地,遣使拜太中大夫,使與大兵會東郡。東郡聞純入界,盜賊九千餘人皆詣純降,大兵不戰而還。璽書復以為東郡太守,吏民悅服。十三年,卒官,謚曰成侯。子阜嗣。

植後為輔威將軍,封武邑侯。宿至代郡太守,封遂鄉侯。訢為赤眉將軍,封著武侯,從鄧禹西征,戰死雲陽。凡宗族封列侯者四人,關內侯者三人,為二千石者九人。

阜徙封莒鄉侯,永平十四年,坐同族耿歙與楚人顏忠辭語相連,國除。建初二年,肅宗追思純功,紹封阜子盱為高亭侯。盱卒,無嗣,帝復封盱弟騰。卒,子忠嗣。忠卒,孫緒嗣。

贊曰:任、邳識幾,嚴城解扉。委佗還旅,二守焉依。純、植義發,奉兵佐威。


\end{pinyinscope}