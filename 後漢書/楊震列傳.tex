\article{楊震列傳}

\begin{pinyinscope}
楊震字伯起,弘農華陰人也。八世祖喜,高祖時有功,封赤泉侯。高祖敞,昭帝時為丞相,封安平侯。父寶,習歐陽尚書。哀、平之世,隱居教授。居攝二年,與兩龔、蔣詡俱徵,遂遁逃,不知所處。光武高其節。建武中,公車特徵,老病不到,卒於家。

震少好學,受歐陽尚書於太常桓郁,明經博覽,無不窮究。諸儒為之語曰:「關西孔子楊伯起。」常客居於湖,不荅州郡禮命數十年,眾人謂之晚暮,而震志愈篤。後有冠雀銜三鱣魚,飛集講堂前,都講取魚進曰:「蛇鱣者,卿大夫服之象也。數三者,法三台也。先生自此升矣。」年五十,乃始仕州郡。

大將軍鄧騭聞其賢而辟之,舉茂才,四遷荊州刺史、東萊太守。當之郡,道經昌邑,故所舉荊州茂才王密為昌邑令,謁見,至夜懷金十斤以遺震。震曰:「故人知君,君不知故人,何也?」密曰:「暮夜無知者。」震曰:「天知,神知,我知,子知。何謂無知!」密愧而出。後轉涿郡太守。性公廉,不受私謁。子孫常蔬食步行,故舊長者或欲令為開產業,震不肯,曰:「使後世稱為清白吏子孫,以此遺之,不亦厚乎!」

元初四年,徵入為太僕,遷太常。先是博士選舉多不以實,震舉薦明經名士陳留楊倫等,顯傳學業,諸儒稱之。

永寧元年,代劉愷為司徒。明年,鄧太后崩,內寵始橫。安帝乳母王聖,因保養之勤,緣恩放恣;聖子女伯榮出入宮掖,傳通姦賂。震上疏曰:「臣聞政以得賢為本,理以去穢為務。是以唐虞俊乂在官,四凶流放,天上咸服,以致雍熙。方今九德未事,嬖倖充庭。阿母王聖出自賤微,得遭千載,奉養聖躬,雖有推燥居溼之勤,前後賞惠,過報勞苦,而無厭之心,不知紀極,外交屬託,擾亂天下,損辱清朝,塵點日月。書誡牝雞牡鳴,詩刺哲婦喪國。昔鄭嚴公從母氏之欲,恣驕弟之情,幾至危國,然後加討,春秋貶之,以為失教。夫女子小人,近之喜,遠之怨,實為難養。《易》曰:『無攸遂,在中饋。』言婦人不得與於政事也。宜速出阿母,令居外舍,斷絕伯榮,莫使往來,令恩德兩隆,上下俱美。惟陛下絕婉孌之私,割不忍之心,留神萬機,誡慎拜爵,減省獻御,損節徵發。令野無鶴鳴之歎,朝無小明之悔,大東不興於今,勞止不怨於下。擬蹤往古,比德哲王,豈不休哉!」奏御,帝以示阿母等,內倖皆懷忿恚。而伯榮驕淫尤甚,與故朝陽侯劉護從兄瑰交通,瑰遂以為妻,得襲護爵,位至侍中。震深疾之,復詣闕上疏曰:「臣聞高祖與群臣約,非功臣不得封,故經制父死子繼,兄亡弟及,以防篡也。伏見詔書封故朝陽侯劉護再從兄瑰襲護爵為侯。護同產弟威,今猶見在。臣聞天子專封封有功,諸侯專爵爵有德。今瑰無佗功行,但以配阿母女,一時之閒,既位侍中,又至封侯,不稽舊制,不合經義,行人諠譁,百姓不安。陛下宜覽鏡既往,順帝之則。」書奏不省。

延光二年,代劉愷為太尉。帝舅大鴻臚耿寶薦中常侍李閏兄於震,震不從。寶乃自往候震曰:「李常侍國家所重,欲令公辟其兄,寶唯傳上意耳。」震曰:「如朝廷欲令三府辟召,故宜有尚書敕。」遂拒不許,寶大恨而去。皇后兄執金吾閻顯亦薦所親厚於震,震又不從。司空劉授聞之,即辟此二人,旬日中皆見拔擢。由是震益見怨。

時詔遣使者大為阿母脩第,中常侍樊豐及侍中周廣、謝惲等更相扇動,傾搖朝廷。震復上疏曰:「臣聞古者九年耕必有三年之儲,故堯遭洪水,人無菜色。臣伏念方今災害發起,彌彌滋甚,百姓空虛,不能自贍。重以螟蝗,羌虜鈔掠,三邊震擾,戰鬥之役至今未息,兵甲軍糧不能復給。大司農帑藏匱乏,殆非社稷安寧之時。伏見詔書為阿母興起津城門內第舍,合兩為一,連里竟街,雕修繕飾,窮極巧伎。今盛夏土王,而攻山採石,其大匠左校別部將作合數十處,轉相迫促,為費巨億。周廣、謝惲兄弟,與國無肺腑枝葉之屬,依倚近倖姦佞之人,與樊豐、王永等分威共權,屬託州郡,傾動大臣。宰司辟召,承望旨意,招來海內貪汙之人,受其貨賂,至有臧錮棄世之徒復得顯用。白黑溷淆,清濁同源,天下讙譁,咸曰財貨上流,為朝結譏。臣聞師言:『上之所取,財盡則怨,力盡則叛。』怨叛之人,不可復使,故曰:『百姓不足,君誰與足?』惟陛下度之。」豐、惲等見震連切諫不從,無所顧忌,遂詐作詔書,調發司農錢穀、大匠見徒材木,各起家舍、園池、廬觀,役費無數。

震因地震,復上疏曰:「臣蒙恩備台輔,不能奉宣政化,調和陰陽,去年十一月四日,京師地動。臣聞師言:『地者陰精,當安靜承陽。』而今動搖者,陰道盛也。其日戊辰,三者皆土,位在中宮,此中臣近官盛於持權用事之象也。臣伏惟陛下以邊境未寧,躬自菲薄,宮殿垣屋傾倚,枝柱而已,無所興造,欲令遠近咸知政化之清流,商邑之翼翼也。而親近倖臣,未崇斷金,驕溢踰法,多請徒士,盛修第舍,賣弄威福。道路讙譁,眾所聞見。地動之變,近在城郭,殆為此發。又冬無宿雪,春節未雨,百僚燋心,而繕修不止,誠致旱之徵也。書曰:『僭恆陽若,臣無作威作福玉食。』唯陛下奮乾剛之德,棄驕奢之臣,以掩訞言之口,奉承皇天之戒,無令威福久移於下。」

震前後所上,轉有切至,帝既不平之,而樊豐等皆側目憤怨,俱以其名儒,未敢加害。尋有河閒男子趙騰詣闕上書,指陳得失。帝發怒,遂收考詔獄,結以罔上不道。震復上疏救之曰:「臣聞堯舜之世,諫鼓謗木,立之於朝;殷周哲王,小人怨詈,則還自敬德。所以達聰明,開不諱,博採負薪,盡極下情也。今趙騰所坐激訐謗語為罪,與手刃犯法有差。乞為虧除,全騰之命,以誘芻蕘輿人之言。」帝不省,騰竟伏尸都市。

會三年春,東巡岱宗,樊豐等因乘輿在外,競修第宅,震部掾高舒召大匠令史考校之,得豐等所詐下詔書,具奏,須行還上之。豐等聞,惶怖,會太史言星變逆行,遂共譖震云:「自趙騰死後,深用怨懟;且鄧氏故吏,有恚恨之心。」及車駕行還,便時太學,夜遣使者策收震太尉印綬,於是柴門絕賓客。豐等復惡之,乃請大將軍耿寶奏震大臣不服罪,懷恚望,有詔遣歸本郡。震行至城西几陽亭,乃慷慨謂其諸子門人曰:「死者士之常分。吾蒙恩居上司,疾姦臣狡猾而不能誅,惡嬖女傾亂而不能禁,何面目復見日月!身死之日,以雜木為棺,布單被裁足蓋形,勿歸冢次,勿設祭祠。」因飲酖而卒,時年七十餘。弘農太守移良承樊豐等旨,遣吏於陝縣留停震喪,露棺道側,謫震諸子代郵行書,道路皆為隕涕。

歲餘,順帝即位,樊豐、周廣等誅死,震門生虞放、陳翼詣闕追訟震事。朝廷咸稱其忠,乃下詔除二子為郎,贈錢百萬,以禮改葬於華陰潼亭,遠近畢至。先葬十餘日,有大鳥高丈餘,集震喪前,俯仰悲鳴,淚下霑地,葬畢,乃飛去。郡以狀上。時連有災異,帝感震之枉,乃下詔策曰:「故太尉震,正直是與,俾匡時政,而青蠅點素,同茲在藩。上天降威,災眚屢作,爾卜爾筮,惟震之故。朕之不德,用彰厥咎,山崩棟折,我其危哉!今使太守丞以中牢具祠,魂而有靈,儻其歆享。」於是時人立石鳥象於其墓所。

震之被譖也,高舒亦得罪,以減死論。及震事顯,舒拜侍御史,至荊州刺史。

震五子。長子牧,富波相。

牧孫奇,靈帝時為侍中,帝嘗從容問奇曰:「朕何如桓帝?」對曰:「陛下之於桓帝,亦猶虞舜比德唐堯。」帝不悅曰:「卿強項,真楊震子孫,死後必復致大鳥矣。」出為汝南太守。帝崩後,復入為侍中衛尉,從獻帝西遷,有功勤。及李傕脅帝歸其營,奇與黃門侍郎鍾繇誘傕部曲將宋曄、楊昂令反傕,傕由此孤弱,帝乃得東。後徙都許,追封奇子亮為陽成亭侯。

震少子奉,奉子敷,篤志博聞,議者以為能世其家。敷早卒,子眾,亦傳先業,以謁者僕射從獻帝入關,累遷御史中丞。及帝東還,夜走度河,眾率諸官屬步從至太陽,拜侍中。建安二年,追前功封蓩亭侯。

震中子秉。

秉字叔節,少傳父業,兼明京氏易,博通書傳,常隱居教授。年四十餘,乃應司空辟,拜待御史,頻出為豫、荊、徐、兗四州刺史,遷任城相。自為刺史、二千石,計日受奉,餘祿不入私門。故吏齎錢百萬遺之,閉門不受。以廉潔稱。

桓帝即位,以明尚書徵入勸講,拜太中大夫、左中郎將,遷侍中、尚書。帝時微行,私過幸河南尹梁胤府舍。是日大風拔樹,晝昏,秉因上疏諫曰:「臣聞瑞由德至,災應事生。傳曰:『禍福無門,唯人所召。』天不言語,以災異譴告,是以孔子迅雷風烈必有變動。《詩》云:『敬天之威,不敢驅馳。』王者至尊,出入有常,警蹕而行,靜室而止,自非郊廟之事,則鑾旗不駕。故詩稱『自郊徂宮』,《易》曰『王假有廟,致孝享也』。諸侯如臣之家,春秋尚列其誡,況以先王法服而私出槃游!降亂尊卑,等威無序,侍衛守空宮,紱璽委女妾,設有非常之變,任章之謀,上負先帝,下悔靡及。臣奕世受恩,得備納言,又以薄學,充在講勤,特蒙哀識,見照日月,恩重命輕,義使士死,敢憚摧折,略陳其愚。」帝不納。秉以病乞退,出為右扶風。太尉黃瓊惜其去朝廷,上秉勸講帷幄,不宜外遷,留拜光祿大夫。是時大將軍梁冀用權,秉稱病。六年,冀誅後,乃拜太僕,遷太常。

延熹三年,白馬令李雲以諫受罪,秉爭之不能得,坐免官,歸田里。其年冬,復徵拜河南尹。先是中常侍單超弟匡為濟陰太守,以臧罪為刺史第五種所劾,窘急,乃賂客任方刺兗州從事衛羽。事已見種傳。及捕得方,囚繫洛陽,匡慮秉當窮竟其事,密令方等得突獄亡走。尚書召秉詰責,秉對曰:「春秋不誅黎比而魯多盜,方等無狀,釁由單匡。刺執法之吏,害奉公之臣,復令逃竄,寬縱罪身,元惡大憝,終為國害。乞檻車徵匡考覈其事,則姦慝蹤緒,必可立得。」而秉竟坐輸作左校,以久旱赦出。

會日食,太山太守皇甫規等訟秉忠正,不宜久抑不用。有詔公車徵秉及處士韋著,二人各稱疾不至。有司並劾秉、著大不敬,請下所屬正其罪。尚書令周景與尚書邊韶議奏:「秉儒學侍講,常在謙虛;著隱居行義,以退讓為節。俱徵不至,誠違側席之望,然逶迤退食,足抑苟進之風。夫明王之世,必有不召之臣,聖朝弘養,宜用優游之禮。可告在所屬,喻以朝庭恩意。如遂不至,詳議其罰。」於是重徵,乃到,拜太常。

五年冬,代劉矩為太尉。是時宦官方熾,任人及子弟為官,布滿天下,競為貪淫,朝野嗟怨。秉與司空周景上言:「內外吏職,多非其人,自頃所徵,皆特拜不試,致盜竊縱恣,怨訟紛錯。舊典,中臣子弟不得居位秉埶,而今枝葉賓客布列職署,或年少庸人,典據守宰,上下忿患,四方愁毒。可遵用舊章,退貪殘,塞災謗。請下司隸校尉、中二千石、二千石、城門五營校尉、北軍中候,各實覈所部,應當斥罷,自以狀言,三府廉察有遺漏,續上。」帝從之。於是秉條奏牧守以下匈奴中郎將燕瑗、青州刺史羊亮、遼東太守孫諠等五十餘人,或死或免,天下莫不肅然。

時郡國計吏多留拜為郎,秉上言三署見郎七百餘人,帑臧空虛,浮食者眾,而不良守相,欲因國為池,澆濯釁穢。宜絕橫拜,以塞覬覦之端。自此終桓帝世,計吏無復留拜者。

七年,南巡園陵,特詔秉從。南陽太守張彪與帝微時有舊恩,以車駕當至,因傍發調,多以入私。秉聞之,下書責讓荊州刺史,以狀副言公府。及行至南陽,左右並通姦利,詔書多所除拜。秉復上疏諫曰:「臣聞先王建國,順天制官。太微積星,名為郎位,入奉宿衛,出牧百姓。皋陶誡虞,在於官人。頃者道路拜除,恩加豎隸,爵以貨成,化由此敗,所以俗夫巷議,白駒遠逝,穆穆清朝,遠近莫觀。宜割不忍之恩,以斷求欲之路。」於是詔除乃止。

時中常侍侯覽弟參為益州刺史,累有臧罪,暴虐一州。明年,秉劾奏參,檻車徵詣廷尉。參惶恐,道自殺。秉因奏覽及中常侍具瑗曰:「臣案國舊典,宦豎之官,本在給使省闥,司昏守夜,而今猥受過寵,執政操權。其阿諛取容者,則因公褒舉,以報私惠;有忤逆於心者,必求事中傷,肆其凶忿。居法王公,富擬國家,飲食極肴⑾,僕妾盈紈素,雖季氏專魯,穰侯擅秦,何以尚茲!案中常侍侯覽弟參,貪殘元惡,自取禍滅,覽顧知釁重,必有自疑之意,臣愚以為不宜復見親近。昔懿公刑邴歜之父,奪閻職之妻,而使二人參乘,卒有竹中之難,春秋書之,以為至戒。蓋鄭詹來而國亂,四佞放而眾服。以此觀之,容可近乎?覽宜急屏斥,投畀有虎。若斯之人,非恩所宥,請免官送歸本郡。」書奏,尚書召對秉掾屬曰:「公府外職,而奏劾近官,經典漢制有故事乎?」秉使對曰:「春秋趙鞅以晉陽之甲,逐君側之惡。傳曰:『除君之惡,唯力是視。』鄧通懈慢,申屠嘉召通詰責,文帝從而請之。漢世故事,三公之職無所不統。」尚書不能詰。帝不得已,竟免覽官,而削瑗國。每朝廷有得失,輒盡忠規諫,多見納用。

秉性不飲酒,又早喪夫人,遂不復娶,所在以淳白稱。嘗從容言曰:「我有三不惑:酒,色,財也。」八年薨,時年七十四,賜塋陪陵。子賜。

賜字伯獻。少傳家學,篤志博聞。常退居隱約,教授門徒,不荅州郡禮命。後辟大將軍梁冀府,非其好也。出除陳倉令,因病不行。公車徵不至,連辭三公之命。後以司空高第,再遷侍中、越騎校尉。

建寧初,靈帝當受學,詔太傅、三公選通尚書桓君章句宿有重名者,三公舉賜,乃侍講于華光殿中。遷少府、光祿勳。

熹平元年,青蛇見御坐,帝以問賜,賜上封事曰:「臣聞和氣致祥,乖氣致災,休徵則五福應,咎徵則六極至。夫善不妄來,災不空發。王者心有所惟,意有所想,雖未形顏色,而五星以之推移,陰陽為其變度。以此而觀,天之與人,豈不符哉?尚書曰:『天齊乎人,假我一日。』是其明徵也。夫皇極不建,則有蛇龍之孽。《詩》云:『惟虺惟蛇,女子之祥。』故春秋兩蛇鬥於鄭門,昭公殆以女敗;康王一朝晏起,關睢見幾而作。夫女謁行則讒夫昌,讒夫昌則苞苴通,故殷湯以之自戒,終濟亢旱之災。惟陛下思乾剛之道,別內外之宜,崇帝乙之制,受元吉之祉,抑皇甫之權,割豔妻之愛,則蛇變可消,禎祥立應。殷戊、宋景,其事甚明。」

二年,代唐珍為司空,以災異免。復拜光祿大夫,秩中二千石。五年,代袁隗為司徒。是時朝廷爵授,多不以次,而帝好微行,遊幸外苑。賜復上疏曰:「臣聞天生蒸民,不能自理,故立君長使司牧之,是以唐虞兢兢業業,周文日昃不暇,明慎庶官,俊乂在職,三載考績,以觀厥成。而今所序用無佗德,有形埶者,旬日累遷,守真之徒,歷載不轉,勞逸無別,善惡同流,北山之詩,所為訓作。又聞數微行出幸苑囿,觀鷹犬之埶,極槃遊之荒,政事日墮,大化陵遲。陛下不顧二祖之勤止,追慕五宗之美蹤,而欲以望太平,是由曲表而欲直景,卻行而求及前人也。宜絕慢傲之戲,念官人之重,割用板之恩,慎貫魚之次,無令醜女有四殆之歎,遐邇有憤怨之聲。臣受恩偏特,忝任師傅,不敢自同凡臣,括囊避咎。謹自手書密上。」

後坐辟黨人免。復拜光祿大夫。光和元年,有虹蜺晝降於嘉德殿前,帝惡之,引賜及議郎蔡邕等入金商門崇德署,使中常待曹節、王甫問以祥異禍福所在。賜仰天而歎,謂節等曰:「吾每讀張禹傳,未嘗不憤恚歎息,既不能竭忠盡情,極言其要,而反留意少子,乞還女婿。朱游欲得尚方斬馬劍以理之,固其宜也。吾以微薄之學,充先師之末,累世見寵,無以報國。猥當大問,死而後已。」乃書對曰:「臣聞之經傳,或得神以昌,或得神以亡。國家休明,則鑒其德;邪辟昏亂,則視其禍。今殿前之氣,應為虹蜺,皆妖邪所生,不正之象,詩人所謂蝃蝀者也。於中孚經曰:『蜺之比,無德以色親。』方今內多嬖倖,外任小臣,上下並怨,諠譁盈路,是以災異屢見,前後丁寧。今復投蜺,可謂孰矣。案春秋讖曰:『天投蜺,天下怨,海內亂。』加四百之期,亦復垂及。昔虹貫牛山,管仲諫桓公無近妃宮。《易》曰:『天垂象,見吉凶,聖人則之。』今妾媵嬖人閹尹之徒,共專國朝,欺罔日月。又鴻都門下,招會群小,造作賦說,以蟲篆小技見寵於時,如驩兜、共工更相薦說,旬月之閒,並各拔擢,樂松處常伯,任芝居納言。纳儉、梁鵠俱以便辟之性,佞辯之心,各受豐爵不次之寵,而令搢紳之徒委伏髂畝,口誦堯舜之言,身蹈絕俗之行,棄捐溝壑,不見逮及。冠履倒易,陵谷代處,從小人之邪意,順無知之私欲,不念板、蕩之作,虺蜴之誡。殆哉之危,莫過於今。幸賴皇天垂象譴告。周書曰:『天子見怪則修德,諸侯見怪則修政,卿大夫見怪則修職,士庶人見怪則修身。』惟陛下慎經典之誡,圖變復之道,斥遠佞巧之臣,速徵鶴鳴之士,內親張仲,外任山甫,斷絕尺一,抑止槃游,留思庶政,無敢怠遑。冀上天還威,眾變可弭。老臣過受師傅之任,數蒙寵異之恩,豈敢愛惜垂沒之年,而不盡其慺慺之心哉!」書奏,甚忤曹節等。蔡邕坐直對抵罪,徙朔方。賜以師傅之恩,故得免咎。

其冬,行辟雍禮,引賜為三老。復拜少府、光祿勳,代劉郃為司徒。帝欲造畢圭靈琨苑,賜復上疏諫曰:「竊聞使者並出,規度城南人田,欲以為苑。昔先王造囿,裁足以脩三驅之禮,薪萊芻牧,皆悉往焉。先帝之制,左開鴻池,右作上林,不奢不約,以合禮中。今猥規郊城之地,以為苑囿,壞沃衍,廢田園,驅居人,畜禽獸,殆非所謂『若保赤子』之義。今城外之苑已有五六,可以逞情意,順四節也,宜惟夏禹卑宮,太宗露臺之意,以尉下民之勞。」書奏,帝欲止,以問侍中任芝、中常侍樂松。松等曰:「昔文王之囿百里,人以為小;齊宣五里,人以為大。今與百姓共之,無害於政也。」帝悅,遂令築苑。

四年,賜以病罷。居無何,拜太常,詔賜御府衣一襲,自所服冠幘綬,玉壺革帶,金錯鉤佩。

五年冬,復拜太尉。中平元年,黃巾賊起,賜被召會議詣省閤,切諫忤旨,因以寇賊免。

先是黃巾帥張角等執左道,稱大賢,以誑燿百姓,天下繈負歸之。賜時在司徒,召掾劉陶告曰:「張角等遭赦不悔,而稍益滋蔓,今若下州郡捕討,恐更騷擾,速成其患。且欲切敕刺史、二千石,簡別流人,各護歸本郡,以孤弱其黨,然後誅其渠帥,可不勞而定,何如?」陶對曰:「此孫子所謂不戰而屈人之兵,廟勝之術也。」賜遂上書言之。會去位,事留中。後帝徙南宮,閱錄故事,得賜所上張角奏及前侍講注籍,乃感悟,下詔封賜臨晉侯,邑千五百戶。初,賜與太尉劉寬、司空張濟並入侍講,自以不宜獨受封賞,上書願分戶邑於寬、濟。帝嘉歎,復封寬及濟子,拜賜尚書令。數日出為廷尉,賜自以代非法家,言曰:「三后成功,惟殷于民,皋陶不與焉,蓋吝之也。」遂固辭,以特進就第。

二年九月,復代張溫為司空。其月薨。天子素服,三日不臨朝,贈東園梓器襚服,賜錢三百萬,布五百匹。策曰:「故司空臨晉侯賜,華嶽所挺,九德純備,三葉宰相,輔國以忠。朕昔初載,授道帷幄,遂階成勳,以陟大猷。師範之功,昭于內外,庶官之務,勞亦勤止。七在卿校,殊位特進,五登袞職,弭難乂寧。雖受茅土,未荅厥勳,哲人其萎,將誰諮度!朕甚懼焉。禮設殊等,物有服章。今使左中郎將郭儀持節追位特進,贈司空驃騎將軍印綬。」及葬,又使侍御史持節送喪,蘭臺令史十人發羽林騎輕車介士,前後部鼓吹,又敕驃騎將軍官屬司空法駕,送至舊塋。公卿已下會葬。謚文烈侯。及小祥,又會焉。子彪嗣。

彪字文先,少傳家學。初舉孝廉,州舉茂才,辟公府,皆不應。熹平中,以博習舊聞,公車徵拜議郎,遷侍中、京兆尹。光和中,黃門令王甫使門生於郡界辜榷官財物七千餘萬,彪發其姦,言之司隸。司隸校尉陽球因此奏誅甫,天下莫不愜心。徵還為侍中、五官中郎將,遷潁川、南陽太守,復拜侍中,三遷永樂少府、太僕、衛尉。

中平六年,代董卓為司空,其冬,代黃琬為司徒。明年,關東兵起,董卓懼,欲遷都以違其難。乃大會公卿議曰:「高祖都關中十有一世,光武宮洛陽,於今亦十世矣。案石包讖,宜徙都長安,以應天人之意。」百官無敢言者。彪曰:「移都改制,天下大事,故盤庚五遷,殷民胥怨。關中遭王莽變亂,宮室焚蕩,民庶塗炭,百不一在。光武受命,更都洛邑。今天下無虞,百姓樂安,明公建立聖主,光隆漢祚,無故捐宗廟,棄園陵,恐百姓驚動,必有糜沸之亂。石包室讖,妖邪之書,豈可信用?」卓曰:「關中肥饒,故秦得并吞六國。且隴右材木自出,致之甚易。又杜陵南山下有武帝故瓦陶灶數千所,并功營之,可使一朝而辨。百姓何足與議!若有前卻,我以大兵驅之,可令詣滄海。」彪曰:「天下動之至易,安之甚難,惟明公慮焉。」卓作色曰:「公欲沮國計邪?」太尉黃琬曰:「此國之大事,楊公之言得無可思?」卓不荅。司空荀爽見卓意壯,恐害彪等,因從容言曰:「相國豈樂此邪?山東兵起,非一日可禁,故當遷以圖之,此秦、漢之埶也。」卓意小解。爽私謂彪曰:「諸君堅爭不止,禍必有歸,故吾不為也。」議罷,卓使司隸校尉宣播以災異奏免琬、彪等,詣闕謝,即拜光祿大夫。十餘日,遷大鴻臚。從入關,轉少府、太常,以病免。復為京兆尹、光祿勳,再遷光祿大夫。三年秋,代淳于嘉為司空,以地震免。復拜太常。興平元年,代朱雋為太尉,錄尚書事。及李傕、郭汜之亂,彪盡節衛主,崎嶇危難之閒,幾不免於害。語在董卓傳。及車駕還洛陽,復守尚書令。

建安元年,從東都許。時天子新遷,大會公卿,兗州刺史曹操上殿,見彪色不悅,恐於此圖之,未得讌設,託疾如廁,因出還營。彪以疾罷。時袁術僭亂,操託彪與術婚姻,誣以欲圖廢置,奏收下獄,劾以大逆。將作大匠孔融聞之,不及朝服,往見操曰:「楊公四世清德,海內所瞻。周書父子兄弟罪不相及,況以袁氏歸罪楊公。易稱『積善餘慶』,徒欺人耳。」操曰:「此國家之意。」融曰:「假使成王殺邵公,周公可得言不知邪?今天下纓緌搢紳所以瞻仰明公者,以公聰明仁智,輔相漢朝,舉直厝枉,致之雍熙也。今橫殺無辜,則海內觀聽,誰不解體!孔融魯國男子,明日便當拂衣而去,不復朝矣。」操不得已,遂理出彪。

四年,復拜太常,十年免。十一年,諸以恩澤為侯者皆奪封。彪見漢祚將終,遂稱腳攣不復行,積十年。後子脩為曹操所殺,操見彪問曰:「公何瘦之甚?」對曰:「愧無日磾先見之明,猶懷老牛舐犢之愛。」操為之改容。

脩字德祖,好學,有俊才,為丞相曹操主簿,用事曹氏。及操自平漢中,欲因討劉備而不得進,欲守之又難為功,護軍不知進止何依。操於是出教,唯曰「雞肋」而已。外曹莫能曉,脩獨曰:「夫雞肋,食之則無所得,棄之則如可惜,公歸計決矣。」乃令外白稍嚴,操於此迴師。脩之幾決,多有此類。脩又嘗出行,籌操有問外事,乃逆為荅記,敕守舍兒:「若有令出,依次通之。」既而果然。如是者三,操怪其速,使廉之,知狀,於此忌脩。且以袁術之甥,慮為後患,遂因事殺之。

脩所著賦、頌、碑、讚、詩、哀辭、表、記、書

凡十五篇。

及魏文帝受禪,欲以彪為太尉,先遣使示旨。彪辭曰:「彪備漢三公,遭世傾亂,不能有所補益。耄年被病,豈可贊惟新之朝?」遂固辭。乃授光祿大夫,賜几杖衣袍,因朝會引見,令彪著布單衣、鹿皮冠,杖而入,待以賓客之禮。年八十四,黃初六年卒于家。自震至彪,四世太尉,德業相繼,與袁氏俱為東京名族云。

論曰:孔子稱「危而不持,顛而不扶,則將焉用彼相矣」。誠以負荷之寄,不可以虛冒,崇高之位,憂重責深也。延、光之閒,震為上相,抗直方以臨權枉,先公道而後身名,可謂懷王臣之節,識所任之體矣。遂累葉載德,繼踵宰相。信哉,「積善之家,必有餘慶」。先世韋、平,方之蔑矣。

贊曰:楊氏載德,仍世柱國。震畏四知,秉去三惑。賜亦無諱,彪誠匪忒。脩雖才子,渝我淳則。


\end{pinyinscope}