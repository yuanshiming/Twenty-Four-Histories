\article{禮儀中}

\begin{pinyinscope}
案戶祠星立冬冬至臘

大儺土牛遣衛士朝會

立夏之日,夜漏未盡五刻,京都百官皆衣赤,至季夏衣黃,郊。其禮:祠特,祭灶。

自立春至立夏盡立秋,郡國上雨澤。若少,府郡縣各掃除社稷;其旱也,公卿官長以次行雩禮求雨。閉諸陽,衣皁,興土龍,立土人舞僮二佾,七日一變如故事。反拘朱索社,伐朱鼓。禱賽以少牢如禮。

拜皇太子之儀:百官會,位定,謁者引皇太子當御坐殿下,北面;司空當太子西北,東面立。讀策書畢,中常侍持皇太子璽綬東向授太子。太子再拜,三稽首。謁者贊皇太子臣某,甲謁者稱制曰「可」。三公升階上殿,賀壽萬歲。因大赦天下。供賜禮畢,罷。

拜諸侯王公之儀:石官會,位定,謁者引光祿勳前。謁者引當拜前,當坐伏殿下。光祿勳前,一拜,舉手曰;「制詔其以某為某。」讀策書畢,謁者稱臣某再拜。尚書郎以璽印綬付侍御史。侍御史前,東面立,授璽印綬。王公再拜頓首三下。贊謁者曰:「某王臣某新封,某公某初,謝。」中謁者報謹謝。贊者立曰:「謝皇帝為公興。」皆冠謝,起就位。供賜禮畢,罷。

仲夏之月,萬物方盛。日夏至,陰氣萌作,恐物不楙。其禮:以朱索連葷菜,彌牟蠱鍾。以桃印長六寸,方三寸,五色書文如法,以施門戶。代以所尚為飾。夏后氏金行,作葦茭,言氣交也。殷人水德,以螺首,慎其閉塞,使如螺也。周人木德,以桃為更,言氣相更也。漢兼用之,故以五月五日,朱索五色印為門戶飾,以難止惡氣。日夏至,禁舉大火,止炭鼓鑄,消石冶皆絕止。至立秋,如故事。是日浚井改水,日冬至,鑽燧改火云。

先立秋十八日,郊黃帝。是日夜漏未盡五刻,京都百官皆衣黃。至立秋,迎氣於黃郊,樂奏黃鍾之宮,歌帝臨,冕而執干戚,舞雲翹、育命,所以養時訓也。

立秋之日,夜漏未盡五刻,京都百官皆衣白,施皁領緣中衣,迎氣白郊。禮畢,皆衣絳,至立冬。

立秋之日,自郊禮畢,始揚威武,斬牲於郊東門,以薦陵廟。其儀:乘輿御戎路,白馬朱鬣,躬執弩射牲。牲以鹿麛。太宰令、謁者各一人,載獲車,馳駟送陵廟。還宮,遣使者齎束帛以賜武官。武官肄兵,習戰陣之儀、斬牲之禮,名曰貙劉。兵、官皆肄孫、吳兵法六十四陣,名曰乘之。立春,遣使者齎束帛以賜文官。貙劉之禮:祠先虞,執事告先虞已,烹鮮時,有司,乃逡巡射牲。獲車畢,有司告事畢。

仲秋之月,縣道皆案戶比民。年始七十者,授之以王杖,餔之糜粥。八十九十,禮有加賜。王杖長〈九〉尺,端以鳩鳥為飾。鳩者,不噎之鳥也。欲老人不噎。是月也,祀老人星于國都南郊老人廟。

季秋之月,祠星于城南壇心星廟。

立冬之日,夜漏未盡五刻,京都百官皆衣皁,迎氣於黑郊。禮畢,皆衣絳,至冬至絕事。

冬至前後,君子安身靜體,百官絕事,不聽政,擇吉辰而後省事。絕事之日,夜漏未盡五刻,京都百官皆衣絳,至立春。諸五時變服,執事者先後其時皆一日。

日冬至、夏至,陰陽晷景長短之極,微氣之所生也。故使八能之士八人,或吹黃鍾之律閒竽;或撞黃鍾之鍾;或度晷景,權水輕重,水一升,冬重十三兩;或擊黃鍾之磬;或鼓黃鍾之瑟,軫閒九尺,二十五弦,宮處于中,左右為商、徵、角、羽;或擊黃鍾之鼓。先之三日,太史謁之。至日,夏時四孟,冬則四仲,其氣至焉。

先氣至五刻,太史令與八能之士郎坐于端門左塾。太子具樂器,夏赤冬黑,列前殿之前西上,鍾為端。守宮設席于器南,北面東上,正德席,鼓南西面,令晷儀東北。三刻,中黃門持兵,引太史令、八能之士入自端門,就位。二刻,侍中、尚書、御史、謁者皆陛。一刻,乘輿親御臨軒,安體靜居以聽之。太史令前,當軒溜北面跪。舉手曰:「八能之士以備,請行事。」制曰「可」。太史令稽首曰「諾」。起立少退,顧令正德曰:「可行事。」正德曰「諾」。皆旋復位。正德立,命八能士曰:「以次行事,閒音以竽。」八能曰「諾」。五音各三十為闋。正德曰:「合五音律。」先唱,五音並作,二十五闋,皆音以竽。訖,正德曰:「八能士各言事。」八能士各書板言事。文曰:「臣某言,今月若干日甲乙日冬至,黃鍾之音調,君道得,孝道褒。」商臣,角民,徵事,羽物,各一板。否則召太史令各板書,封以皁囊,送西陛,跪授尚書,施當軒,北面稽首,拜上封事。尚書授侍中常侍迎受,報聞。以小黃門幡麾節度。太史令前曰禮畢。制曰「可」。太史令前稽首曰「諾」。太史命八能士詣太官受賜。陛者以次罷。日夏至禮亦如之。

季冬之月,星迴歲終,陰陽以交,勞農大享臘。

先臘一日,大儺,謂之逐疫。其儀:選中黃門子弟年十歲以上,十二以下,百二十人為侲子。皆赤幘皁製,執大浅。方相氏黃金四目,蒙熊皮,玄衣朱裳,執戈揚盾。十二獸有衣毛角。中黃門行之,冗從僕射將之,以逐惡鬼于禁中。夜漏上水,朝臣會,侍中、尚書、御史、謁者、虎賁、羽林郎將執事,皆赤幘陛衛。乘輿御前殿。黃門令奏曰:「侲子備,請逐疫。」於是中黃門倡,振子和,曰:「甲作食杂,胇胃食虎,雄伯食魅,騰簡食不祥,攬諸食咎,伯奇食夢,強梁、祖明共食磔死寄生,委隨食觀,錯斷食巨,窮奇、騰根共食蠱。凡使十二神追惡凶,赫女軀,拉女幹,節解女肉,抽女肺腸。女不急去,後者為糧!」因作方相與十二獸觯。嚾呼,周遍前後省三過,持炬火,送疫出端門;門外騶騎傳炬出宮,司馬闕門門外五營騎士傳火棄雒水中。百官官府各以木面獸能為儺人師訖,設桃梗、鬱櫑、葦茭畢,執事陛者罷。葦戟、桃杖以賜公、卿、將軍、特侯、諸侯云。

是月也,立土牛六頭於國都郡縣城外丑地,以送大寒。

饗遣故衛士儀:百官會,位定,謁者持節引故衛士入自端門。衛司馬執幡鉦護行。行定,侍御史持節慰勞,以詔恩問所疾苦,受其章奏所欲言。畢饗,賜作樂,觀以角抵。樂闋罷遣,勸以農桑。

每月朔歲首,為大朝受賀。其儀:夜漏未盡七刻,鍾鳴,受賀。及贄,公、侯璧,中二千石、二千石羔,千石、六百石鴈,四百石以下雉。百官賀正月。二千石以上上殿稱萬歲。舉觴御坐前。司空奉羹,大司農奉飯,奏食舉之樂。百官受錫宴饗,大作樂。其每朔,唯十月旦從故事者,高祖定秦之月,元年歲首也。


\end{pinyinscope}