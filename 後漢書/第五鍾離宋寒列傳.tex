\article{第五鍾離宋寒列傳}

\begin{pinyinscope}
第五倫字伯魚,京兆長陵人也。其先齊諸田,諸田徙園陵者多,故以次第為氏。

倫少介然有義行。王莽末,盜賊起,宗族閭里爭往附之。倫乃依險固築營壁,有賊,輒奮厲其眾,引彊持滿以拒之,銅馬、赤眉之屬前後數十輩,皆不能下。倫始以營長詣郡尹鮮于褒,褒見而異之,署為吏。後褒坐事左轉高唐令,臨去,握倫臂訣曰:「恨相知晚。」

倫後為鄉嗇夫,平傜賦,理怨結,得人歡心。自以為久宦不達,遂將家屬客河東,變名姓,自稱王伯齊,載鹽往來太原、上黨,所過輒為糞除而去,陌上號為道士,親友故人莫知其處。

數年,鮮于褒薦之於京兆尹閻興,興即召倫為主簿。時長安鑄錢多姦巧,乃署倫為督鑄錢掾,領長安巿。倫平銓衡,正斗斛,巿無阿枉,百姓悅服。每讀詔書,常歎息曰:「此聖主也,一見決矣。」等輩笑之曰:「爾說將尚不下,安能動萬乘乎?」倫曰:「未遇知己,道不同故耳。」

建武二十七年,舉孝廉,補淮陽國醫工長,隨王之國。光武召見,甚異之。二十九年,從王朝京師,隨官屬得會見,帝問以政事,倫因此酬對政道,帝大悅。明日,復特召入,與語至夕。帝戲謂倫曰:「聞卿為吏篣婦公,不過從兄飯,寧有之邪?」倫對曰:「臣三娶妻皆無父。少遭飢亂,實不敢妄過人食。」帝大笑。倫出,有詔以為扶夷長,未到官,追拜會稽太守。雖為二千石,躬自斬芻養馬,妻執炊爨。受俸裁留一月糧,餘皆賤貿與民之貧羸者。會稽俗多淫祀,好卜筮。民常以牛祭神,百姓財產以之困匱,其自食牛肉而不以薦祠者,發病且死先為牛鳴,前後郡將莫敢禁。倫到官,移書屬縣,曉告百姓。其巫祝有依託鬼神詐怖愚民,皆案論之。有妄屠牛者,吏輒行罰。民初頗恐懼,或祝詛妄言,倫案之愈急,後遂斷絕,百姓以安。永平五年,坐法徵,老小攀車叩馬,啼呼相隨,日裁行數里,不得前。倫乃偽止亭舍,陰乘船去。眾知,復追之。及詣廷尉,吏民上書守闕者千餘人。是時顯宗方案梁松事,亦多為松訟者。帝患之,詔公車諸為梁氏及會稽太守上書者勿復受。會帝幸廷尉錄囚徒,得免歸田里。身自耕種,不交通人物。

數歲,拜為宕渠令,顯拔鄉佐玄賀,賀後為九江、沛二郡守,以清絜稱,所在化行,終於大司農。

倫在職四年,遷蜀郡太守。蜀地肥饒,人吏富實,掾史家貲多至千萬,皆鮮車怒馬,以財貨自達。倫悉簡其豐贍者遣還之,更選孤貧志行之人以處曹任,於是爭賕抑絕,文職修理。所舉吏多至九卿、二千石,時以為知人。

視事七歲,肅宗初立,擢自遠郡,代牟融為司空。帝以明德太后故,尊崇舅氏馬廖,兄弟並居職任。廖等傾身交結,冠蓋之士爭赴趣之。倫以后族過盛,欲令朝廷抑損其權,上疏曰:「臣聞忠不隱諱,直不避害。不勝愚狷,昧死自表。書曰:『臣無作威作福,其害于而家,凶于而國。』傳曰:『大夫無境外之交,束脩之饋。』近代光烈皇后,雖友愛天至,而卒使陰就歸國,徙廢陰興賓客;其後梁、竇之家,互有非法,明帝即位,竟多誅之。自是洛中無復權戚,書記請託一皆斷絕。又譬諸外戚曰:『苦身待士,不如為國,戴盆望天,事不兩施。』臣常刻著五臧,書諸紳帶。而今之議者,復以馬氏為言。竊聞衛尉廖以布三千匹,城門校尉防以錢三百萬,私贍三輔衣冠,知與不知,莫不畢給。又聞臘日亦遺其在洛中者錢各五千,越騎校尉光,臘用羊三百頭,米四百斛,肉五千斤。臣愚以為不應經義,惶恐不敢不以聞。陛下情欲厚之,亦宜所以安之。臣今言此,誠欲上忠陛下,下全后家,裁蒙省察。」及馬防為車騎將軍,當出征西羌,倫又上疏曰:「臣愚以為貴戚可封侯以富之,不當職事以任之。何者?繩以法則傷恩,私以親則違憲。伏聞馬防今當西征,臣以太后恩仁,陛下至孝,恐卒有纖介,難為意愛。聞防請杜篤為從事中郎,多賜財帛。篤為鄉里所廢,客居美陽,女弟為馬氏妻,恃此交通,在所縣令苦其不法,收繫論之。今來防所,議者咸致疑怪,況乃以為從事,將恐議及朝廷。今宜為選賢能以輔助之,不可復令防自請人,有損事望。苟有所懷,敢不自聞。」並不見省用。

倫雖峭直,然常疾俗吏苛刻。及為三公,值帝長者,屢有善政,乃上疏褒稱盛美,因以勸成風德,曰:「陛下即位,躬天然之德,體晏晏之姿,以寬弘臨下,出入四年,前歲誅刺史、二千石貪殘者六人。斯皆明聖所鑒,非群下所及。然詔書每下寬和而政急不解,務存節儉而奢侈不止者,咎在俗敝,群下不稱故也。光武承王莽之餘,頗以嚴猛為政,後代因之,遂成風化。郡國所舉,類多辨職俗吏,殊未有寬博之選以應上求者也。陳留令劉豫,冠軍令駟協,並以刻薄之姿,臨人宰邑,專念掠殺,務為嚴苦,吏民愁怨,莫不疾之,而今之議者反以為能,違天心,失經義,誠不可不慎也。非徒應坐豫、協,亦當宜譴舉者。務進仁賢以任時政,不過數人,則風俗自化矣。臣嘗讀書記,知秦以酷急亡國,又目見王莽亦以苛法自滅,故勤勸懇懇,實在於此。又聞諸王主貴戚,驕奢踰制,京師尚然,何以示遠?故曰:『其身不正,雖令不行。』以身教者從,以言教者訟。夫陰陽和歲乃豐,君臣同心化乃成也。其刺史、太守以下,拜除京師及道出洛陽者,宜皆召見,可因博問四方,兼以觀察其人。諸上書言事有不合者,可但報歸田里,不宜過加喜怒,以明在寬。臣愚不足採。」及諸馬得罪歸國,而竇氏始貴,倫復上疏曰:「臣得以空虛之質,當輔弼之任。素性駑怯,位尊爵重,拘迫大義,思自策厲,雖遭百死,不敢擇地,又況親遇危言之世哉!今承百王之敝,人尚文巧,咸趨邪路,莫能守正。伏見虎賁中郎將竇憲,椒房之親,典司禁兵,出入省闥,年盛志美,卑謙樂善,此誠其好士交結之方。然諸出入貴戚者,類多瑕舗禁錮之人,尤少守約安貧之節,士大夫無志之徒更相販賣,雲集其門。眾喣飄山,聚蚊成雷,蓋驕佚所從生也。三輔論議者,至云以貴戚廢錮,當復以貴戚浣濯之,猶解酲當以酒也。詖險趣埶之徒,誠不可親近。臣愚願陛下中宮嚴敕憲等閉門自守,無妄交通士大夫,防其未萌,慮於無形,令憲永保福祿,君臣交歡,無纖介之隙。此臣之至所願也。」

倫奉公盡節,言事無所依違。諸子或時諫止,輒叱遣之,吏人奏記及便宜者,亦并封上,其無私若此。性質愨,少文采,在位以貞白稱,時人方之前朝貢禹。然少蘊藉,不修威儀,亦以此見輕。或問倫曰:「公有私乎?」對曰:「昔人有與吾千里馬者,吾雖不受,每三公有所選舉,心不能忘,而亦終不用也。吾兄子常病,一夜十往,退而安寢;吾子有疾,雖不省視而竟夕不眠。若是者,豈可謂無私乎?」連以老病上疏乞身。元和三年,賜策罷,以二千石奉終其身,加賜錢五十萬,公宅一區。後數年卒,時年八十餘,詔賜秘器、衣衾、錢布。

少子頡嗣,歷桂陽、廬江、南陽太守,所在見稱。順帝之為太子廢也,頡為太中大夫,與太僕來歷等共守闕固爭。帝即位,擢為將作大匠,卒官。倫曾孫種。

論曰:第五倫峭覈為方,非夫愷悌之士,省其奏議,惇惇歸諸寬厚,將懲苛切之敝使其然乎?昔人以弦韋為佩,蓋猶此矣。然而君子侈不僭上,儉不偪下,豈尊臨千里而與牧圉等庸乎?詎非矯激,則未可以中和言也。

種字興先,少厲志義,為吏,冠名州郡。永壽中,以司徒掾清詔使冀州,廉察災害,舉奏刺史、二千石以下,所刑免甚眾,棄官奔走者數十人。還,以奉使稱職,拜高密侯相。是時徐兗二州盜賊群輩,高密在二州之郊,種乃大儲糧蓄,勤厲吏士,賊聞皆憚之,桴鼓不鳴,流民歸者,歲中至數千家。以能換為衛相。

遷兗州刺史。中常侍單超兄子匡為濟陰太守,負埶貪放,種欲收舉,未知所使。會聞從事衛羽素抗厲,乃召羽具告之。謂曰:「聞公不畏彊禦,今欲相委以重事,若何?」對曰:「願庶幾於一割。」羽出,遂馳至定陶,閉門收匡賓客親吏四十餘人,六七日中,糾發其臧五六千萬。種即奏匡,并以劾超。匡窘迫,遣刺客刺羽,羽覺其姦,乃收繫客,具得情狀。州內震慄,朝廷嗟歎之。

是時太山賊叔孫無忌等暴橫一境,州郡不能討。羽說種曰:「中國安寧,忘戰日久,而太山險阻,寇猾不制。今雖有精兵,難以赴敵,羽請往譬降之。」種敬諾。羽乃往,備說禍福,無忌即帥其黨與三千餘人降。單超積懷忿恨,遂以事陷種,竟坐徙朔方。超外孫董援為朔方太守,蓄怒以待之。初,種為衛相,以門下掾孫斌賢,善遇之。及當徙斥,斌具聞超謀,乃謂其友人同縣閭子直及高密甄子然曰:「蓋盜憎其主,從來舊矣。第五使君當投裔土,而單超外屬為彼郡守。夫危者易仆,可為寒心。吾今方追使君,庶免其難。若奉使君以還,將以付子。」二人曰:「子其行矣,是吾心也。」於是斌將俠客晨夜追種,及之於太原,遮險格殺送吏,因下馬與種,斌自步從。一日一夜行四百餘里,遂得脫歸。

種匿於閭、甄氏數年,徐州從事臧旻上書訟之曰:「臣聞士有忍死之辱,必有就事之計,故季布屈節於朱家,管仲錯行於召忽。此二臣以可死而不死者,非愛身於須臾,貪命於苟活,隱其智力,顧其權略,庶幸逢時有所為耳。卒遭高帝之成業,齊桓之興伯,遺其亡逃之行,赦其射鉤之讎,拔於囚虜之中,信其佐國之謀,勳效傳於百世,君臣載於篇籍。假令二主紀過於纖介,則此二臣同死於犬馬,沈名於溝壑,當何由得申其補過之功,建其奇奧之術乎?伏見故兗州刺史第五種,傑然自建,在鄉曲無苞苴之嫌,步朝堂無擇言之闕,天性疾惡,公方不曲,故論者說清高以種為上,序直士以種為首。春秋之義,選人所長,棄其所短,錄其小善,除其大過。種所坐以盜賊公負,筋力未就,罪至徵徙,非有大惡。昔虞舜事親,大杖則走。故種逃亡,苟全性命,冀有朱家之路,以顯季布之會。願陛下無遺須臾之恩,令種有持忠入地之恨。」會赦出,卒於家。

鍾離意字子阿,會稽山陰人也。少為郡督郵。時部縣亭長有受人酒禮者,府下記案考之。意封還記,入言於太守曰:「春秋先內後外,《詩》云『刑於寡妻,以御于家邦』,明政化之本,由近及遠。今宜先清府內,且闊略遠縣細微之愆。」太守甚賢之,遂任以縣事。建武十四年,會稽大疫,死者萬數,意獨身自隱親,經給醫藥,所部多蒙全濟。

舉孝廉,再遷,辟大司徒侯霸府。詔部送徒詣河內,時冬寒,徒病不能行。路過弘農,意輒移屬縣使作徒衣,縣不得已與之,而上書言狀,意亦具以聞。光武得奏,以見霸,曰:「君所使掾何乃仁於用心?誠良吏也!」意遂於道解徒桎梏,恣所欲過,與剋期俱至,無或違者。還,以病免。

後除瑕丘令。吏有檀建者,盜竊縣內,意屏人問狀,建叩頭服罪,不忍加刑,遣令長休。建父聞之,為建設酒,謂曰:「吾聞無道之君以刃殘人,有道之君以義行誅。子罪,命也。」遂令建進藥而死。二十五年,遷堂邑令。人防廣為父報讎,繫獄,其母病死,廣哭泣不食。意憐傷之,乃聽廣歸家,使得殯斂。丞掾皆爭,意曰:「罪自我歸,義不累下。」遂遣之。廣斂母訖,果還入獄。意密以狀聞,廣竟得以減死論。

顯宗即位,徵為尚書。時交阯太守張恢,坐臧千金,徵還伏法,以資物簿入大司農,詔班賜群臣。意得珠璣,悉以委地而不拜賜。帝怪而問其故。對曰:「臣聞孔子忍渴於盜泉之水,曾參回車於勝母之閭,惡其名也。此臧穢之寶,誠不敢拜。」帝嗟歎曰:「清乎尚書之言!」乃更以庫錢三十萬賜意。轉為尚書僕射。車駕數幸廣成苑,意以為從禽廢政,常當車陳諫般樂遊田之事,天子即時還宮。永平三年夏旱,而大起北宮,意詣闕免冠上疏曰:「伏見陛下以天時小旱,憂念元元,降避正殿,躬自克責,而比日密雲,遂無大潤,豈政有未得應天心者邪?昔成湯遭旱,以六事自責曰:『政不節邪?使人疾邪?宮室榮邪?女謁盛邪?苞苴行邪?讒夫昌邪?』竊見北宮大作,人失農時,此所謂宮室榮也。自古非苦宮室小狹,但患人不安寧。宜且罷止,以應天心。臣意以匹夫之才,無有行能,久食重祿,擢備近臣,比受厚賜,喜懼相并,不勝愚戇征營,罪當萬死。」帝策詔報曰:「湯引六事,咎在一人。其冠履,勿謝。比上天降旱,密雲數會,朕戚然慚懼,思獲嘉應,故分布禱請,闚候風雲,北祈明堂,南設雩場。今又敕大匠止作諸宮,減省不急,庶消災譴。」詔因謝公卿百僚,遂應時澍雨焉。

時詔賜降胡子縑,尚書案事,誤以十為百。帝見司農上簿,大怒,召郎將笞之。意因入叩頭曰:「過誤之失,常人所容。若以懈慢為愆,則臣位大,罪重,郎位小,罪輕,咎皆在臣,臣當先坐。」乃解衣就格。帝意解,使復冠而貰郎。

帝性褊察,好以耳目隱發為明,故公卿大臣數被詆毀,近臣尚書以下至見提拽。常以事怒郎藥崧,以杖撞之。崧走入床下,帝怒甚,疾言曰:「郎出!郎出!」崧曰:「天子穆穆,諸侯煌煌。未聞人君自起撞郎。」帝赦之。朝廷莫不悚慄,爭為嚴切,以避誅責;唯意獨敢諫爭,數封還詔書,臣下過失輒救解之。會連有變異,意復上疏曰:「伏惟陛下躬行孝道,修明經術,郊祀天地,畏敬鬼神,憂恤黎元,勞心不怠。而天氣未和,日月不明,水泉湧溢,寒暑違節者,咎在群臣不能宣化理職,而以苛刻為俗。吏殺良人,繼踵不絕。百官無相親之心,吏人無雍雍之志。至於骨肉相殘,毒害彌深,感逆和氣,以致天災。百姓可以德勝,難以力服。先王要道,民用和睦,故能致天下和平,災害不生,禍亂不作。鹿鳴之詩必言宴樂者,以人神之心洽,然後天氣和也。願陛下垂聖德,揆萬機,詔有司,慎人命,緩刑罰,順時氣,以調陰陽,垂之無極。」帝雖不能用,然知其至誠。亦以此故不得久留,出為魯相。後德陽殿成,百官大會。帝思意言,謂公卿曰:「鍾離尚書若在,此殿不立。」

意視事五年,以愛利為化,人多殷富。以久病卒官。遺言上書陳升平之世,難以急化,宜少寬假。帝感傷其意,下詔嗟歎,賜錢二十萬。

藥崧者,河內人,天性朴忠。家貧為郎,常獨直臺上,無被,枕励,食糟糠。帝每夜入臺,輒見崧,問其故,甚嘉之,自此詔太官賜尚書以下朝夕餐,給帷被皁袍,及侍史二人。崧官至南陽太守。

宋均字叔庠,南陽安眾人也。父伯,建武初為五官中郎將。均以父任為郎,時年十五,好經書,每休沐日,輒受業博士,通詩禮,善論難。至二十餘,調補辰陽長。其俗少學者而信巫鬼,均為立學校,禁絕淫祀,人皆安之。以祖母喪去官,客授潁川。

後為謁者。會武陵蠻反,圍武威將軍劉尚,詔使均乘傳發江夏奔命三千人往救之。既至而尚已沒。會伏波將軍馬援至,詔因令均監軍,與諸將俱進,賊拒阨不得前。及馬援卒於師,軍士多溫溼疾病,死者太半。均慮軍遂不反,乃與諸將議曰:「今道遠士病,不可以戰,欲權承制降之何如?」諸將皆伏地莫敢應。均曰:「夫忠臣出竟,有可以安國家,專之可也。」乃矯制調伏波司馬呂种守沅陵長,命种奉詔書入虜營,告以恩信,因勒兵隨其後。蠻夷震怖,即共斬其大帥而降,於是入賊營,散其眾,遣歸本郡,為置長吏而還。均未至,先自劾矯制之罪。光武嘉其功,迎賜以金帛,令過家上冢。其後每有四方異議,數訪問焉。

遷上蔡令。時府下記,禁人喪葬不得侈長。均曰:「夫送終踰制,失之輕者。今有不義之民,尚未循化,而遽罰過禮,非政之先。」竟不肯施行。

遷九江太守。郡多虎暴,數為民患,常募設檻阱而猶多傷害。均到,下記屬縣曰:「夫虎豹在山,黿鼉在水,各有所託。且江淮之有猛獸,猶北土之有雞豚也。今為民害,咎在殘吏,而勞勤張捕,非憂恤之本也。其務退姦貪,思進忠善,可一去檻阱,除削課制。」其後傳言虎相與東游度江。中元元年,山陽、楚、沛多蝗,其飛至九江界者,輒東西散去,由是名稱遠近。浚遒縣有唐、后二山,民共祠之,眾巫遂取百姓男女以為公嫗,歲歲改易,既而不敢嫁娶,前後守令莫敢禁。均乃下書曰:「自今以後,為山娶者皆娶巫家,勿擾良民。」於是遂絕。

永平元年,遷東海相,在郡五年,坐法免官,客授潁川。而東海吏民思均恩化,為之作歌,詣闕乞還者數千人。顯宗以其能,七年,徵拜尚書令。每有駮議,多合上旨。均嘗刪翦疑事,帝以為有姦,大怒,收郎縛格之。諸尚書惶恐,皆叩頭謝罪。均顧厲色曰:「蓋忠臣執義,無有二心。若畏威失正,均雖死,不易志。」小黃門在傍,入具以聞。帝善其不撓,即令貰郎,遷均司隸校尉。數月,出為河內太守,政化大行。

均常寢病,百姓耆老為禱請,旦夕問起居,其為民愛若此。以疾上書乞免,詔除子條為太子舍人。均自扶輿詣闕謝恩,帝使中黃門慰問,因留養疾。司徒缺,帝以均才任宰相,召入視其疾,令兩騶扶之。均拜謝曰:「天罰有罪,所苦浸篤,不復奉望帷幄!」因流涕而辭。帝甚傷之,召條扶侍均出,賜錢三十萬。

均性寬和,不喜文法,常以為吏能弘厚,雖貪汙放縱,猶無所害;至於苛察之人,身或廉法,而巧黠刻削,毒加百姓,災害流亡所由而作。及在尚書,恆欲叩頭爭之,以時方嚴切,故遂不敢陳。帝後聞其言而追悲之。建初元年,卒於家。族子意。

意字伯志。父京,以大夏侯尚書教授,至遼東太守。意少傳父業,顯宗時舉孝廉,以召對合旨,擢拜阿陽侯相。建初中,徵為尚書。

肅宗性寬仁,而親親之恩篤,故叔父濟南、中山二王每數入朝,特加恩寵,及諸昆弟並留京師,不遣就國。意以為人臣有節,不宜踰禮過恩,乃上疏諫曰:「陛下至孝烝烝,恩愛隆深,以濟南王康、中山王焉先帝昆弟,特蒙禮寵,聖情戀戀,不忍遠離,比年朝見,久留京師,崇以叔父之尊,同之家人之禮,車入殿門,即席不拜,分甘損膳,賞賜優渥。昔周公懷聖人之德,有致太平之功,然後王曰叔父,加以錫幣。今康、焉幸以支庶享食大國,陛下即位,蠲除前過,還所削黜,衍食它縣,男女少長,並受爵邑,恩寵踰制,禮敬過度。春秋之義,諸父昆弟無所不臣,所以尊尊卑卑,彊幹弱枝者也。陛下德業隆盛,當為萬世典法,不宜以私恩損上下之序,失君臣之正。又西平王羡等六王,皆妻子成家,官屬備其,當早就蕃國,為子孫基阯。而室第相望,久磐京邑,婚姻之盛,過於本朝,僕馬之眾,充塞城郭,驕奢僭擬,寵祿隆過。今諸國之封,並皆膏腴,風氣平調,道路夷近,朝聘有期,行來不難。宜割情不忍,以義斷恩,發遣康、焉各歸蕃國,令羡等速就便時,以塞眾望。」帝納之。

章和二年,鮮卑擊破北匈奴,而南單于乘此請兵北伐,因欲還歸舊庭。時竇太后臨朝,議欲從之。意上疏曰:「夫戎狄之隔遠中國,幽處北極,界以沙漠,簡賤禮義,無有上下,彊者為雄,弱即屈服。自漢興以來,征伐數矣,其所剋獲,曾不補害。光武皇帝躬服金革之難,深昭天地之明,故因其來降,羈縻畜養,邊人得生,勞役休息,於茲四十餘年矣。今鮮卑奉順,斬獲萬數,中國坐享大功,而百姓不知其勞,漢興功烈,於斯為盛。所以然者,夷虜相攻,無損漢兵者也。臣察鮮卑侵伐匈奴,正是利其抄掠,及歸功聖朝,實由貪得重賞。今若聽南虜還都北庭,則不得不禁制鮮卑。鮮卑外失暴掠之願,內無功勞之賞,豺狼貪婪,必為邊患。今北虜西遁,請求和親,宜因其歸附,以為外扞,巍巍之業,無以過此。若引兵費賦,以順南虜,則坐失上略,去安即危矣。誠不可許。」會南單于竟不北徙。

遷司隸校尉。永元初,大將軍竇憲兄弟貴盛,步兵校尉鄧疊、河南尹王調、故蜀郡太守廉范等群黨,出入憲門,負埶放縱。意隨違舉奏,無所回避,由是與竇氏有隙。二年,病卒。

孫俱,靈帝時為司空。

寒朗字伯奇,魯國薛人也。生三日,遭天下亂,棄之荊棘;數日兵解,母往視,猶尚氣息,遂收養之。及長,好經學,博通書傳,以尚書教授。舉孝廉。

永平中,以謁者守侍御史。與三府掾屬共考案楚獄顏忠、王平等,辭連及隧鄉侯耿建、朗陵侯臧信、護澤侯鄧鯉、曲成侯劉建。建等辭未嘗與忠、平相見。是時顯宗怒甚,吏皆惶恐,諸所連及,率一切陷入,無敢以情恕者。朗心傷其冤,試以建等物色獨問忠、平,而二人錯龙不能對。朗知其詐,乃上言建等無姦,專為忠、平所誣,疑天下無辜類多如此。帝乃召朗入,問曰:「建等即如是,忠、平何故引之?」朗對曰:「忠、平自知所犯不道,故多有虛引,冀以自明。」帝曰:「即如是,四侯無事,何不早奏,獄竟而久繫至今邪?」朗對曰:「臣雖考之無事,然恐海內別有發其姦者,故未敢時上。」帝怒罵曰:「吏持兩端,促提下。」左右方引去,朗曰:「願一言而死。小臣不敢欺,欲助國耳。」帝問曰:「誰與共為章?」對曰:「臣自知當必族滅,不敢多污染人,誠冀陛下一覺悟而已。臣見考囚在事者,咸共言妖惡大故,臣子所宜同疾,今出之不如入之,可無後責。是以考一連十,考十連百。又公卿朝會,陛下問以得失,皆長跪言,舊制大罪禍及九族,陛下大恩,裁止於身,天下幸甚。及其歸舍,口雖不言,而仰屋竊歎,莫不知其多冤,無敢啎陛下者。臣今所陳,誠死無悔。」帝意解,詔遣朗出。後二日,車駕自幸洛陽獄錄囚徒,理出千餘人。後平、忠死獄中,朗乃自繫。會赦,免官。復舉孝廉。

建初中,肅宗大會群臣,朗前謝恩,詔以朗納忠先帝,拜為易長。歲餘,遷濟陽令,以母喪去官,百姓追思之。章和元年,上行東巡狩,過濟陽,三老吏人上書陳朗前政治狀。帝至梁,召見朗,詔三府為辟首,由是辟司徒府。永元中,再遷清河太守。坐法免。

永初三年,太尉張禹薦朗為博士,徵詣公車,會卒,時年八十四。

論曰:左丘明有言:「仁人之言,其利博哉!」晏子一言,齊侯省刑。若鍾離意之就格請過,寒朗之廷爭冤獄,篤矣乎,仁者之情也!夫正直本於忠誠則不詭,本於諫爭則絞切。彼二子之所本得乎天,故言信而志行也。

贊曰:伯魚、子阿,矯急去苛。臨官以絜,匡帝以奢。宋均達政,禁此妖禜。禽蟲畏德,子民請病。意明尊尊,割恩蕃屏。惵惵楚黎,寒君為命。


\end{pinyinscope}