\article{文苑列傳下}

\begin{pinyinscope}
張升字彥真,陳留尉氏人,富平侯放之孫也。升少好學,多關覽,而任情不羈。其意相合者,則傾身交結,不問窮賤;如乖其志好者,雖王公大人,終不屈從。常歎曰:「死生有命,富貴在天。其有知我,雖胡越可親;苟不相識,從物何益?」

仕郡為綱紀,以能出守外黃令。吏有受賕者,即論殺之。或譏升守領一時,何足趨明威戮乎?對曰:「昔仲尼暫相,誅齊之侏儒,手足異門而出,故能威震強國,反其侵地。君子仕不為己,職思其憂,豈以久近而異其度哉?」遇黨錮去官,後竟見誅,年四十九。

著賦、誄、頌、碑、書,凡六十篇。

趙壹字元叔,漢陽西縣人也。體貌魁梧,身長九尺,美須豪眉,望之甚偉。而恃才倨傲,為鄉黨所擯,乃作解擯。後屢抵罪,幾至死,友人救得免。壹乃貽書謝恩曰:

昔原大夫贖桑下絕氣,傳稱其仁;秦越人還虢太子結脈,世著其神。設曩之二人不遭仁遇神,則結絕之氣竭矣。然而糒脯出乎車軨,鍼石運乎手爪。今所賴者,非直車軨之糒脯,手爪之鍼石也。乃收之於斗極,還之於司命,使乾皮復含血,枯骨復被肉,允所謂遭仁遇神,真所宜傳而著之。余畏禁,不敢班班顯言,竊為窮鳥賦一篇。其辭曰:

有一窮鳥,戢翼原野。罼網加上,機阱在下,前見蒼隼,後見驅者,繳彈張右,羿子彀左,飛丸激矢,交集于我。思飛不得,欲鳴不可,舉頭畏觸,搖足恐墯。內獨怖急,乍冰乍火。幸賴大賢,我矜我憐,昔濟我南,今振我西。鳥也雖頑,猶識密恩,內以書心,外用告天。天乎祚賢,歸賢永年,且公且侯,子子孫孫。

又作刺世疾邪賦,以舒其怨憤。曰:

伊五帝之不同禮,三王亦又不同樂,數極自然變化,非是故相反駮。德政不能救世溷亂,賞罰豈足懲時清濁?春秋時禍敗之始,戰國愈復增其荼毒。秦、漢無以相踰越,乃更加其怨酷。寧計生民之命,唯利己而自足。

于茲迄今,情偽萬方。佞諂日熾,剛克消亡。舐痔結駟,正色徒行。嫗悯名埶,撫拍豪強。偃蹇反俗,立致咎殃。捷懾逐物,日富月昌。渾然同惑,孰溫孰涼。邪夫顯進,直士幽藏。

原斯瘼之攸興,寔執政之匪賢。女謁掩其視聽兮,近習秉其威權。所好則鑽皮出其毛羽,所惡則洗垢求其瘢痕。雖欲竭誠而盡忠,路絕嶮而靡緣。九重既不可啟,又群吠之狺狺。安危亡於旦夕,肆嗜慾於目前。奚異涉海之失杝,積薪而待燃。榮納由於閃揄,孰知辨其蚩妍。故法禁屈撓於埶族,恩澤不逮於單門。寧飢寒於堯舜之荒歲兮,不飽暖於當今之豐年。乘理雖死而非亡,違義雖生而匪存。

有秦客者,乃為詩曰:河清不可俟,人命不可延。順風激靡草,富貴者稱賢。文籍雖滿腹,不如一囊錢。伊優北堂上,抗髒倚門邊。

魯生聞此辭,繫而作歌曰:埶家多所宜,欬唾自成珠。被褐懷金玉,蘭蕙化為芻。賢者雖獨悟,所困在群愚。且各守爾分,勿復空馳驅。哀哉復哀哉,此是命矣夫!

光和元年,舉郡上計到京師。是時司徒袁逢受計,計吏數百人皆拜伏庭中,莫敢仰視,壹獨長揖而已。逢望而異之,令左右往讓之,曰:「下郡計史而揖三公,何也?」對曰:「昔酈食其長揖漢王,今揖三公,何遽怪哉?」逢則斂衽下堂,執其手,延置上坐,因問西方事,大悅,顧謂坐中曰:「此人漢陽趙元叔也。朝臣莫有過之者,吾請為諸君分坐。」坐者皆屬觀。既出,往造河南尹羊陟,不得見。壹以公卿中非陟無足以託名者,乃日往到門,陟自強許通,尚臥未起,壹逕入上堂,遂前臨之,曰:「竊伏西州,承高風舊矣,乃今方遇而忽然,柰何命也!」因舉聲哭,門下驚,皆奔入滿側。陟知其非常人,乃起,延與語,大奇之。謂曰:「子出矣。」陟明旦大從車騎奉謁造壹。時諸計吏多盛飾車馬帷幕,而壹獨柴車草屏,露宿其傍,延陟前坐於車下,左右莫不歎愕。陟遂與言談,至熏夕,極歡而去,執其手曰:「良璞不剖,必有泣血以相明者矣!」陟乃與袁逢共稱薦之。名動京師,士大夫想望其風采。

及西還,道經弘農,過候太守皇甫規,門者不即通,壹遂遁去。門吏懼,以白之。規聞壹名大驚,乃追書謝曰:「蹉跌不面,企德懷風,虛心委質,為日久矣。側聞仁者愍其區區,冀承清誨,以釋遙悚。今旦外白有一尉兩計吏,不道屈尊門下,更啟乃知已去。如印綬可投,夜豈待旦。惟君明叡,平其夙心。寧當慢傲,加於所天。事在悖惑,不足具責。儻可原察,追脩前好,則何福如之!謹遣主簿奉書。下筆氣結,汗流竟趾。」壹報曰:「君學成師範,縉紳歸慕,仰高希驥,歷年滋多。旋轅兼道,渴於言侍,沐浴晨興,昧旦守門,實望仁兄,昭其懸遲。以貴下賤,握髮垂接,高可敷翫墳典,起發聖意,下則抗論當世,消弭時災。豈悟君子,自生怠倦,失恂恂善誘之德,同亡國驕惰之志!蓋見機而作,不俟終日,是以夙退自引,畏使君勞。昔人或歷說而不遇,或思士而無從,皆歸之於天,不尤於物。今壹自譴而已,豈敢有猜!仁君忽一匹夫,於德何損?而遠辱手筆,追路相尋,誠足愧也。壹之區區,曷云量己,其嗟可去,謝也可食,誠則頑薄,實識其趣。但關節疢動,膝灸塊潰,請俟它日,乃奉其情。輒誦來貺,永以自慰。」遂去不顧。

州郡爭致禮命,十辟公府,並不就,終於家。初袁逢使善相者相壹,云「仕不過郡吏」,竟如其言。

著賦、頌、箴、誄、書、論及雜文十六篇。

劉梁字曼山,一名岑,東平寧陽人也。梁宗室子孫,而少孤貧,賣書於市以自資。

常疾世多利交,以邪曲相黨,乃著破群論。時之覽者,以為「仲尼作春秋,亂臣知懼,今此論之作,俗士豈不愧心」。其文不存。

又著辯和同之論。其辭曰:

夫事有違而得道,有順而失義,有愛而為害,有惡而為美。其故何乎?蓋明智之所得,闇偽之所失也。是以君子之於事也,無適無莫,必考之以義焉。

得由和興,失由同起,故以可濟否謂之和,好惡不殊謂之同。春秋傳曰:「和如羹焉,酸苦以劑其味,君子食之以平其心。同如水焉,若以水濟水,誰能食之?琴瑟之專一,誰能聽之?」是以君子之行,周而不比,和而不同,以救過為正,以匡惡為忠。經曰:「將順其美,匡救其惡,則上下和睦能相親也。」

昔楚恭王有疾,召其大夫曰:「不穀不德,少主社稷。失先君之緒,覆楚國之師,不穀之罪也。若以宗廟之靈,得保首領以歿,請為靈若厲。」大夫許諸。及其卒也,子囊曰:「不然。夫事君者,從其善,不從其過。赫赫楚國,而君臨之,撫正南海,訓及諸夏,其寵大矣。有是寵也,而知其過,可不謂恭乎!」大夫從之。此違而得道者也。及靈王驕淫,暴虐無度,芋尹申亥從王之欲,以殯於乾溪,殉之二女。此順而失義者也。鄢陵之役,晉楚對戰,陽穀獻酒,子反以斃。此愛而害之者也。臧武仲曰:「孟孫之惡我,藥石也;季孫之愛我,美疢也。疢毒滋厚,石猶生我。」此惡而為美者也。孔子曰:「智之難也!有臧武仲之智,而不容於魯國。抑有由也,作不順而施不恕也。」蓋善其知義,譏其違道也。

夫知而違之,偽也;不知而失之,闇也。闇與偽焉,其患一也。患之所在,非徒在智之不及,又在及而違之者矣。故曰「智及之仁不能守之,雖得之,必失之」也。夏書曰:「念茲在茲,庶事恕施。」忠智之謂矣。

故君子之行,動則思義,不為利回,不為義疚,進退周旋,唯道是務。苟失其道,則兄弟不阿;苟得其義,雖仇讎不廢。故解狐蒙祁奚之薦,二叔被周公之害,勃鞮以逆文為成,傅瑕以順厲為敗,管蘇以憎忤取進,申侯以愛從見退,考之以義也。故曰:「不在逆順,以義為斷;不在憎愛,以道為貴。」《禮記》曰:「愛而知其惡,憎而知其善。」考義之謂也。

桓帝時,舉孝廉,除北新城長。告縣人曰:「昔文翁在蜀,道著巴漢,庚桑瑣隸,風移碨磥。吾雖小宰,猶有社稷,苟赴期會,理文墨,豈本志乎!」乃更大作講舍,延聚生徒數百人,朝夕自往勸誡,身執經卷,試策殿最,儒化大行。此邑至後猶稱其教焉。

特召入拜尚書郎,累遷。後為野王令,未行。光和中,病卒。

孫楨,亦以文才知名。

邊讓字文禮,陳留浚儀人也。少辯博,能屬文。作章華賦,雖多淫麗之辭,而終之以正,亦如相如之諷也。其辭曰:

楚靈王既遊雲夢之澤,息於荊臺之上。前方淮之水,左洞庭之波,右顧彭蠡之隩,南眺巫山之阿。延目廣望,騁觀終日。顧謂左史倚相曰:「盛哉斯樂,可以遺老而忘死也!」於是遂作章華之臺,築乾谿之室,窮木土之技,單珍府之實,舉國營之,數年乃成。設長夜之淫宴,作北里之新聲。於是伍舉知夫陳、蔡之將生謀也。乃作斯賦以諷之:

冑高陽之苗胤兮,承聖祖之洪澤。建列藩於南楚兮,等威靈於二伯。超有商之大彭兮,越隆周之兩虢。達皇佐之高勳兮,馳仁聲之顯赫。惠風春施,神武電斷,華夏肅清,五服攸亂。旦垂精於萬機兮,夕回輦於門館。設長夜之歡飲兮,展中情之嬿婉。竭四海之妙珍兮,盡生人之秘玩。

爾乃攜窈窕,從好仇,徑肉林,登糟丘,蘭肴山竦,椒酒淵流。激玄醴於清池兮,靡微風而行舟。登瑤臺以回望兮,冀彌日而消憂。於是招宓妃,命湘娥,齊倡列,鄭女羅。揚激楚之清宮兮,展新聲而長歌。繁手超於北里,妙舞麗於陽阿。金石類聚,絲竹群分。被輕褂,曳華文,羅衣飄颻,組綺繽紛。縱輕軀以迅赴,若孤鵠之失群;振華袂以逶迤,若遊龍之登雲。於是歡嬿既洽,長夜向半,琴瑟易調,繁手改彈,清聲發而響激,微音逝而流散。振弱支而紆繞兮,若綠繁之垂幹,忽飄颻以輕逝兮,似鸞飛於天漢。舞無常態,鼓無定節,尋聲響應,修短靡跌。長袖奮而生風,清氣激而繞結。爾乃妍媚遞進,巧弄相加,俯仰異容,忽兮神化。體迅輕鴻,榮曜春華,進如浮雲,退如激波。雖復柳惠,能不咨嗟!於是天河既回,淫樂未終,清籥發徵,激楚揚風。於是音氣發於絲竹兮,飛響軼於雲中。比目應節而雙躍兮,孤雌感聲而鳴雄。美繁手之輕妙兮,嘉新聲之彌隆。於是眾變已盡,群樂既考。歸乎生風之廣夏兮,脩黃軒之要道。攜西子之弱腕兮,援毛嬪之素肘。形便娟以嬋媛兮,若流風之靡草。美儀操之姣麗兮,忽遺生而忘老。

爾乃清夜晨,妙技單,收尊俎,徹鼓盤。惘焉若酲,撫劍而歎。慮理國之須才,悟稼穡之艱難。美呂尚之佐周,善管仲之輔桓。將超世而作理,焉沈湎於此歡!於是罷女樂,墮瑤臺。思夏禹之卑宮,慕有虞之土階。舉英奇於仄陋,拔髦秀於蓬萊。君明哲以知人,官隨任而處能。百揆時敘,庶績咸熙。諸侯慕義,不召同期。繼高陽之絕軌,崇成、莊之洪基。雖齊桓之一匡,豈足方於大持?爾乃育之以仁,臨之以明。致虔報於鬼神,盡肅恭乎上京。馳淳化於黎元,永歷世而太平。

大將軍何進聞讓才名,欲辟命之,恐不至,詭以軍事徵召,既到,署令史,進以禮見之。讓善占謝,能辭對,時賓客滿堂,莫不羨其風。府掾孔融、王朗並修刺候焉。

議郎蔡邕深敬之,以為讓宜處高任,乃薦於何進曰:「伏惟幕府初開,博選清英,華髮舊德,並為元龜。雖振鷺之集西雍,濟濟之在周庭,無以或加。竊見令史陳留邊讓,天授逸才,聰明賢智。髫蛮夙孤,不盡家訓。及就學廬,便受大典,初涉諸經,見本知義,授者不能對其問,章句不能逮其意。心通性達,口辯辭長。非禮不動,非法不言。若處狐疑之論,定嫌審之分,經典交至,撿括參合,眾夫寂焉,莫之能奪也。使讓生在唐、虞,則元、凱之次,運值仲尼,則顏、冉之亞,豈徒俗之凡偶近器而已者哉!階級名位,亦宜超然,若復隨輩而進,非所以章瑰偉之高價,昭知人之絕明也。傳曰:『函牛之鼎以亨雞,多汁則淡而不可食,少汁則熬而不可熟。』此言大器之於小用,固有所不宜也。邕竊悁邑,怪此寶鼎未受犧牛大羹之和,久在煎熬臠割之閒,願明將軍回謀垂慮,裁加少納,貢之機密,展之力用。若以年齒為嫌,則顏回不得貫德行之首,子奇終無理阿之功。苟堪其事,古今一也。」

讓後以高才擢進,屢遷,出為九江太守,不以為能也。

初平中,王室大亂,讓去官還家。恃才氣,不屈曹操,多輕侮之言。建安中,其鄉人有搆讓於操,操告郡就殺之。文多遺失。

酈炎字文勝,范陽人,酈食其之後也。炎有文才,解音律,言論給捷,多服其能理。靈帝時,州郡辟命,皆不就。有志氣,作詩二篇曰:

大道夷且長,窘路狹且促。脩翼無與栖,遠趾不步局。舒吾陵霄羽,奮此千里足。超邁絕塵驅,倏忽誰能逐。賢愚豈常類,稟性在清濁。富貴有人籍,貧賤無天錄。通塞苟由己,志士不相卜。陳平敖里社,韓信釣河曲。終居天下宰,食此萬鍾祿。德音流千載,功名重山岳。

靈芝生河洲,動搖因洪波。蘭榮一何晚,嚴霜瘁其柯。哀哉二芳草,不植太山阿。文質道所貴,遭時用有嘉。絳、灌臨衡宰,謂誼崇浮華。賢才抑不用,遠投荊南沙。抱玉乘龍驥,不逢樂與和。安得孔仲尼,為世陳四科!

炎後風病慌忽。性至孝,遭母憂,病甚發動。妻始產而驚死,妻家訟之,收繫獄。炎病不能理對,熹平六年,遂死獄中,時年二十八。尚書盧植為之誄讚,以昭其懿德。

侯瑾字子瑜,敦煌人也。少孤貧,依宗人居。性篤學,恆傭作為資,暮還輒钢柴以讀書。常以禮自牧,獨處一房,如對嚴賓焉。州郡累召,公車有道徵,並稱疾不到。作矯世論以譏切當時。而徙入山中,覃思著述。以莫知於世,故作應賓難以自寄。又案漢記撰中興以後行事,為皇德傳三十篇,行於世。餘所作雜文數十篇,多亡失。西河人敬其才而不敢名之,皆稱為侯君云。

高彪字義方,吳郡無錫人也。家本單寒,至彪為諸生,遊太學。有雅才而訥於言。嘗從馬融欲訪大義,融疾不獲見,乃覆刺遺融書曰:「承服風問,從來有年,故不待介者而謁大君子之門,冀一見龍光,以敘腹心之願。不圖遭疾,幽閉莫啟。昔周公旦父文兄武,九命作伯,以尹華夏,猶揮沐吐餐,垂接白屋,故周道以隆,天下歸德。公今養痾傲士,故其宜也。」融省書慚,追謝還之,彪逝而不顧。

後郡舉孝廉,試經第一,除郎中,校書東觀,數奏賦、頌、奇文,因事諷諫,靈帝異之。

時京兆第五永為督軍御史,使督幽州,百官大會,祖餞於長樂觀。議郎蔡邕等皆賦詩,彪乃獨作箴曰:「文武將墜,乃俾俊臣。整我皇綱,董此不虔。古之君子,即戎忘身。明其果毅,尚其桓桓。呂尚七十,氣冠三軍,詩人作歌,如鷹如鸇。天有太一,五將三門;地有九變,丘陵山川;人有計策,六奇五閒:總茲三事,謀則咨詢。無曰己能,務在求賢,淮陰之勇,廣野是尊。周公大聖,石碏純臣,以威克愛,以義滅親。勿謂時險,不正其身。勿謂無人,莫識己真。忘富遺貴,福祿乃存。枉道依合,復無所觀。先公高節,越可永遵。佩藏斯戒,以厲終身。」邕等甚美其文,以為莫尚也。

後遷內黃令,帝敕同僚臨送,祖於上東門,詔東觀畫彪像以勸學者。彪到官,有德政,上書薦縣人申徒蟠等。病卒於官,文章多亡。

子岱,亦知名。

張超字子並,河閒鄚人也,留侯良之後也。有文才。靈帝時,從車騎將軍朱雋征黃巾,為別部司馬。著賦、頌、碑文、薦、檄、牋、書、謁文、嘲,凡十九篇。超又善於草書,妙絕時人,世共傳之。

禰衡字正平,平原般人也。少有才辯,而尚氣剛傲,好矯時慢物。興平中,避難荊州。建安初,來遊許下。始達潁川,乃陰懷一刺,既而無所之適,至於刺字漫滅。是時許都新建,賢士大夫四方來集。或問衡曰:「盍從陳長文、司馬伯達乎?」對曰:「吾焉能從屠沽兒耶!」又問:「荀文若、趙稚長云何?」衡曰:「文若可借面弔喪,稚長可使監廚請客。」唯善魯國孔融及弘農楊脩。常稱曰:「大兒孔文舉,小兒楊德祖。餘子碌碌,莫足數也。」融亦深愛其才。

衡始弱冠,而融年四十,遂與為交友。上疏薦之曰:「臣聞洪水橫流,帝思俾乂,旁求四方,以招賢俊。昔孝武繼統,將弘祖業,疇咨熙載,群士響臻。陛下叡聖,纂承基緒,遭遇厄運,勞謙日昃。惟岳降神,異人並出。竊見處士平原禰衡,年二十四,字正平,淑質貞亮,英才卓礫。初涉蓺文,升堂睹奧,目所一見,輒誦於口,耳所瞥聞,不忘於心。性與道合,思若有神。弘羊潛計,安世默識,以衡準之,誠不足怪。忠果正直,志懷霜雪,見善若驚,疾惡若讎。任座抗行,史魚厲節,殆無以過也。鷙鳥累伯,不如一鶚。使衡立朝,必有可觀。飛辯騁辭,溢氣坌涌,解疑釋結,臨敵有餘。昔賈誼求試屬國,詭係單于;終軍欲以長纓,牽致勁越。弱冠慷慨,前世美之。近日路粹、嚴象,亦用異才擢拜臺郎,衡宜與為比。如得龍躍天衢,振翼雲漢,楊聲紫微,垂光虹蜺,足以昭近署之多士,增四門之穆穆。鈞天廣樂,必有奇麗之觀;帝室皇居,必蓄非常之寶。若衡等輩,不可多得。激楚、楊阿,至妙之容,臺牧者之所貪;飛兔、騕褭,絕足奔放,良、樂之所急。臣等區區,敢不以聞。」

融既愛衡才,數稱述於曹操。操欲見之,而衡素相輕疾,自稱狂病,不肯往,而數有恣言。操懷忿,而以其才名,不欲殺之。聞衡善擊鼓,乃召為鼓史,因大會賓客,閱試音節,諸史過者,皆令脫其故衣,更著岑牟單絞之服。次至衡,衡方為漁陽參撾,蹀钱而前,容態有異,聲節悲壯,聽者莫不慷慨。衡進至操前而止,吏訶之曰:「鼓史何不改裝,而輕敢進乎?」衡曰:「諾。」於是先解衵衣,次釋餘服,裸身而立,徐取岑牟、單絞而著之,畢,復參撾而去,顏色不怍。操笑曰:「本欲辱衡,衡反辱孤。」

孔融退而數之曰:「正平大雅,固當爾邪?」因宣操區區之意。衡許往。融復見操,說衡狂疾,今求得自謝。操喜,敕門者有客便通,待之極晏。衡乃著布單衣、疏巾,手持三尺梲杖,坐大營門,以杖捶地大罵。吏白:外有狂生,坐於營門,言語悖逆,請收案罪。操怒,謂融曰:「禰衡豎子,孤殺之猶雀鼠耳。顧此人素有虛名,遠近將謂孤不能容之,今送與劉表,視當何如。」於是遣人騎送之。臨發,眾人為之祖道,先供設於城南,乃更相戒曰:「禰衡勃虐無禮,今因其後到,咸當以不起折之也。」及衡至,眾人莫肯興,衡坐而大號。眾問其故,衡曰:「坐者為冢,臥者為屍,屍冢之閒,能不悲乎!」

劉表及荊州士大夫先服其才名,甚賓禮之,文章言議,非衡不定。表嘗與諸文人共草章奏,並極其才思。時衡出,還見之,開省未周,因毀以抵地。表憮然為駭。衡乃從求筆札,須臾立成,辭義可觀。表大悅,益重之。

後復侮慢於表,表恥不能容,以江夏太守黃祖性急,故送衡與之,祖亦善待焉。衡為作書記,輕重疏密,各得體宜。祖持其手曰:「處士,此正得祖意,如祖腹中之所欲言也。」

祖長子射為章陵太守,尤善於衡。嘗與衡俱遊,共讀蔡邕所作碑文,射愛其辭,還恨不繕寫。衡曰:「吾雖一覽,猶能識之,唯其中石缺二字為不明耳。」因書出之,射馳使寫碑還校,如衡所書,莫不歎伏。射時大會賓客,人有獻鸚鵡者,射舉卮於衡曰:「願先生賦之,以娛嘉賓。」衡覽筆而作,文無加點,辭采甚麗。

後黃祖在蒙衝船上,大會賓客,而衡言不遜順,祖慚,乃訶之,衡更熟視曰:「死公!云等道?」祖大怒,令五百將出,欲加箠,衡方大罵,祖恚,遂令殺之。祖主簿素疾衡,即時殺焉。射徒跣來救,不及。祖亦悔之,乃厚加棺斂。衡時年二十六,其文章多亡云。

贊曰:情志既動,篇辭為貴。抽心呈貌,非彫非蔚。殊狀共體,同聲異氣。言觀麗則,永監淫費。


\end{pinyinscope}