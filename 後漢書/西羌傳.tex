\article{西羌傳}

\begin{pinyinscope}
西羌之本,出自三苗,姜姓之別也。其國近南岳。及舜流四凶,徙之三危,河關之西南羌地是也。濱於賜支,至乎河首,綿地千里。賜支者,禹貢所謂析支者也。南接蜀、漢徼外蠻夷,西北鄯善、車師諸國。所居無常,依隨水草。地少五穀,以產牧為業。其俗氏族無定,或以父名母姓為種號。十二世後,相與婚姻,父沒則妻後母,兄亡則納递泽,故國無鰥寡,種類繁熾。不立君臣,無相長一,強則分種為酋豪,弱則為人附落,更相抄暴,以力為雄。殺人償死,無它禁令。其兵長在山谷,短於平地,不能持久,而果於觸突,以戰死為吉利,病終為不祥。堪耐寒苦,同之禽獸。雖婦人產子,亦不避風雪。性堅剛勇猛,得西方金行之氣焉。

王政脩則賓服,德教失則寇亂。昔夏后氏太康失國,四夷背叛。及后相即位,乃征畎夷,七年然後來賓。至于后泄,始加爵命,由是服從。后桀之亂,畎夷入居邠岐之閒,成湯既興,伐而攘之。及殷室中衰,諸夷皆叛。至于武丁,征西戎、鬼方,三年乃克。故其詩曰:「自彼氐羌,莫敢不來王。」

及武乙暴虐,犬戎寇邊,周古公踰梁山而避于岐下。及子季歷,遂伐西落鬼戎。太丁之時,季歷復伐燕京之戎,戎人大敗周師。後二年,周人克余無之戎,於是太丁命季歷為牧師。自是之後,更伐始呼、翳徒之戎,皆克之。及文王為西伯,西有昆夷之患,北有獫狁之難,遂攘戎狄而戍之,莫不賓服。及率西戎,征殷之叛國以事紂。

及武王伐商,羌、髳率師會于牧野。至穆王時,戎狄不貢,王乃西征犬戎,獲其五王,又得四白鹿,四白狼,王遂遷戎于太原。夷王衰弱,荒服不朝,乃命总公率六師伐太原之戎,至于俞泉,獲馬千匹。厲王無道,戎狄寇掠,乃入犬丘,殺秦仲之族,王命伐戎,不克。及宣王立四年,使秦仲伐戎,為戎所殺,王乃召秦仲子莊公,與兵七千人,伐戎破之,由是少卻。後二十七年,王遣兵伐太原戎,不克。後五年,王伐條戎、奔戎,王師敗績。後二年,晉人敗北戎于汾隰,戎人滅姜侯之邑。明年,王征申戎,破之。後十年,幽王命伯士伐六濟之戎,軍敗,伯士死焉。其年,戎圍犬丘,虜秦襄公之兄伯父。時幽王昏虐,四夷交侵,遂廢申后而立褒姒。申侯怒,與戎寇周,殺幽王於酈山,周乃東遷洛邑,秦襄公攻戎救周。後二年,邢侯大破北戎。

及平王之末,周遂陵遲,戎逼諸夏,自隴山以東,及乎伊、洛,往往有戎。於是渭首有狄、铠、邽、冀之戎,涇北有義渠之戎,洛川有大荔之戎,渭南有驪戎,伊、洛閒有楊拒、泉皋之戎,潁首以西有蠻氏之戎。當春秋時,閒在中國,與諸夏盟會。魯莊公伐秦取邽、冀之戎。後十餘歲,晉滅驪戎。是時,伊、洛戎強,東侵曹、魯,後十九年,遂入王城,於是秦、晉伐戎以救周。後二年,又寇京師,齊桓公徵諸侯戍周。後九年,陸渾戎自瓜州遷于伊川,允姓戎遷于渭汭,東及轘轅。在河南山北者號曰陰戎,陰戎之種遂以滋廣。晉文公欲脩霸業,乃賂戎狄通道,以匡王室。秦穆公得戎人由余,遂霸西戎,開地千里。及晉悼公,又使魏絳和諸戎,復脩霸業。是時楚、晉強盛,威服諸戎,陸渾、伊、洛、陰戎事晉,而蠻氏從楚。後陸渾叛晉,晉令荀吳滅之。後四十四年,楚執蠻氏而盡囚其人。是時義渠、大荔最強,築城數十,皆自稱王。

至周貞王八年,秦厲公滅大荔,取其地。趙亦滅代戎,即北戎也。韓、魏復共稍并伊、洛、陰戎,滅之。其遺脫者皆逃走,西踰汧、隴。自是中國無戎寇,唯餘義渠種焉。至貞王二十五年,秦伐義渠,虜其王。後十四年,義渠侵秦至渭陰。後百許年,義渠敗秦師于洛。後四年,義渠國亂,秦惠王遣庶長操將兵定之,義渠遂臣於秦。後八年,秦伐義渠,取郁郅。後二年,義渠敗秦師于李伯。明年,秦伐義渠,取徒涇二十五城。及昭王立,義渠王朝秦,遂與昭王母宣太后通,生二子。至王赧四十三年,宣太后誘殺義渠王於甘泉宮,因起兵滅之,始置隴西、北地、上郡焉。

戎本無君長,夏后氏末及商周之際,或從侯伯征伐有功,天子爵之,以為藩服。春秋時,陸渾、蠻氏戎稱子,戰國世,大荔、義渠稱王,及其衰亡,餘種皆反舊為酋豪云。

羌無弋爰劍者,秦厲公時為秦所拘執,以為奴隸。不知爰劍何戎之別也。後得亡歸,而秦人追之急,藏於巖穴中得免。羌人云爰劍初藏穴中,秦人焚之,有景象如虎,為其蔽火,得以不死。既出,又與劓女遇於野,遂成夫婦。女恥其狀,被髮覆面,羌人因以為俗,遂俱亡入三河閒。諸羌見爰劍被焚不死,怪其神,共畏事之,推以為豪。河湟閒少五穀,多禽獸,以射獵為事,爰劍教之田畜,遂見敬信,廬落種人依之者日益眾。羌人謂奴為無弋,以爰劍嘗為奴隸,故因名之。其後世世為豪。

至爰劍曾孫忍時,秦獻公初立,欲復穆公之跡,兵臨渭首,滅狄铠戎。忍季父卬畏秦之威,將其種人附落而南,出賜支河曲西數千里,與眾羌絕遠,不復交通。其後子孫分別,各自為種,任隨所之。或為氂牛種,越巂羌是也;或為白馬種,廣漢羌是也;或為參狼種,武都羌是也。忍及弟舞獨留湟中,並多娶妻婦。忍生九子為九種,舞生十七子為十七種,羌之興盛,從此起矣。

及忍子研立,時秦孝公雄強,威服羌戎。孝公使太子駟率戎狄九十二國朝周顯王。研至豪健,故羌中號其後為研種。及秦始皇時,務并六國,以諸侯為事,兵不西行,故種人得以繁息。秦既兼天下,使蒙恬將兵略地,西逐諸戎,北卻眾狄,築長城以界之,眾羌不復南度。

至于漢興,匈奴冒頓兵強,破東胡,走月氏,威震百蠻,臣服諸羌。景帝時,研種留何率種人求守隴西塞,於是徙留何等於狄道、安故,至臨洮、氐道、羌道縣。及武帝征伐四夷,開地廣境,北卻匈奴,西逐諸羌,乃度河、湟,築令居塞;初開河西,列置四郡,通道玉門,隔絕羌胡,使南北不得交關。於是障塞亭燧出長城外數千里。時先零羌與封養牢姐種解仇結盟,與匈奴通,合兵十餘萬,共攻令居、安故,遂圍枹罕。漢遣將軍李息、郎中令徐自為將兵十萬人擊平之。始置護羌校尉,持節統領焉。羌乃去湟中,依西海、鹽池左右。漢遂因山為塞,河西地空,稍徙人以實之。

至宣帝時,遣光祿大夫義渠安國覘行諸羌,其先零種豪言:「願得度湟水,逐人所不田處以為畜牧。」安國以事奏聞,後將軍趙充國以為不可聽。後因緣前言,遂度湟水,郡縣不能禁。至元康三年,先零乃與諸羌大共盟誓,將欲寇邊。帝聞,復使安國將兵觀之。安國至,召先零豪四十餘人斬之,因放兵擊其種,斬首千餘級。於是諸羌怨怒,遂寇金城。乃遣趙充國與諸將將兵六萬人擊破平之。至研十三世孫燒當立。元帝時,彡姐等七種羌寇隴西,遣右將軍馮奉世擊破降之。從爰劍種五世至研,研最豪健,自後以研為種號。十三世至燒當,復豪健,其子孫更以燒當為種號。自彡姐羌降之後數十年,四夷賓服,邊塞無事。至王莽輔政,欲燿威德,以懷遠為名,乃令譯諷旨諸羌,使共獻西海之地,初開以為郡,築五縣,邊海亭燧相望焉。

滇良者,燒當之玄孫也。時王莽末,四夷內侵,及莽敗,眾羌遂還據為寇。更始、赤眉之際,羌遂放縱,寇金城、隴西。隗囂雖擁兵而不能討之,乃就慰納,因發其眾與漢相拒。建武九年,隗囂死,司徒掾班彪上言:「今涼州部皆有降羌,羌胡被髮左衽,而與漢人雜處,習俗既異,言語不通,數為小吏黠人所見侵奪,窮恚無聊,故致反叛。夫蠻夷寇亂,皆為此也。舊制益州部置蠻夷騎都尉,幽州部置領烏桓校尉,涼州部置護羌校尉,皆持節領護,理其怨結,歲時循行,問所疾苦。又數遣使驛通動靜,使塞外羌夷為吏耳目,州郡因此可得儆備。今宜復如舊,以明威防。」光武從之,即以牛邯為護羌校尉,持節如舊。及邯卒而職省。十年,先零豪與諸種相結,復寇金城、隴西,遣中郎將來歙等擊之,大破。事已具歙傳。十一年夏,先零種復寇臨洮,隴西太守馬援破降之。後悉歸服,徙置天水、隴西、扶風三郡。明年,武都參狼羌反,援又破降之。事已具援傳。

自燒當至滇良,世居河北大允谷,種小人貧。而先零、卑湳並皆強富,數侵犯之。滇良父子積見陵易,憤怒,而素有恩信於種中,於是集會附落及諸雜種,乃從大榆入,掩擊先零、卑湳,大破之,殺三千人,掠取財畜,奪居其地大榆中,由是始強。

滇良子滇吾立。中元元年,武都參狼羌反,殺略吏人,太守與戰不勝,隴西太守劉盱遣從事辛都、監軍掾李苞,將五千人赴武都,與羌戰,斬其酋豪,首虜千餘人。時武都兵亦更破之,斬首千餘級,餘悉降。時滇吾附落轉盛,常雄諸羌,每欲侵邊者,滇吾轉教以方略,為其渠帥。二年秋,燒當羌滇吾與弟滇岸率步騎五千寇隴西塞,劉盱遣兵於枹罕擊之,不能克,又戰於允街,為羌所敗,殺五百餘人。於是守塞諸羌皆復相率為寇。遣謁者張鴻領諸郡兵擊之,戰於允吾、唐谷,軍敗,鴻及隴西長史田颯皆沒。又天水兵為牢姐種所敗於白石,死者千餘人。

時燒何豪有婦人比銅鉗者,年百餘歲,多智筭,為種人所信向,皆從取計策。時為盧水胡所擊,比銅鉗乃將其眾來依郡縣。種人頗有犯法者,臨羌長收繫比銅鉗,而誅殺其種六七百人。顯宗憐之,乃下詔曰:「昔桓公伐戎而無仁惠,故春秋貶曰『齊人』。今國家無德,恩不及遠,羸弱何辜,而當并命!夫長平之暴,非帝者之功,咎由太守長吏妄加殘戮。比銅鉗尚生者,所在致醫藥養視,令招其種人,若欲歸故地者,厚遣送之。其小種若束手自詣,欲效功者,皆除其罪。若有逆謀為吏所捕,而獄狀未斷,悉以賜有功者。」

永平元年,復遣中郎將竇固、捕虜將軍馬武等擊滇吾於西邯,大破之。事已具武等傳。滇吾遠引去,餘悉散降,徙七千口置三輔。以謁者竇林領護羌校尉,居狄道。林為諸羌所信,而滇岸遂詣林降。林為下吏所欺,謬奏上滇岸以為大豪,承制封為歸義侯,加號漢大都尉。明年,滇吾復降,林復奏其第一豪,與俱詣闕獻見。帝怪一種兩豪,疑其非實,以事詰林。林辭窘,乃偽對曰:「滇岸即滇吾,隴西語不正耳。」帝窮驗知之,怒而免林官。會涼州刺史又奏林臧罪,遂下獄死。謁者郭襄代領校尉事,到隴西,聞涼州羌盛,還詣闕,抵罪,於是復省校尉官。滇吾子東吾立,以父降漢,乃入居塞內,謹愿自守。而諸弟迷吾等數為寇盜。

肅宗建初元年,安夷縣吏略妻卑湳種羌婦,吏為其夫所殺,安夷長宗延追之出塞,種人恐見誅,遂共殺延,而與勒姐及吾良二種相結為寇。隴西太守孫純遣從事李睦及金城兵會和羅谷,與卑湳等戰,斬首虜數百人。復拜故度遼將軍吳棠領護羌校尉,居安夷。二年夏,迷吾遂與諸眾聚兵,欲叛出塞。金城太守郝崇追之,戰於荔谷,崇兵大敗,崇輕騎得脫,死者二千餘人。於是諸種及屬國盧水胡悉與相應,吳棠不能制,坐徵免。武威太守傅育代為校尉,移居臨羌。迷吾又與封養種豪布橋等五萬餘人共寇隴西、漢陽,於是遣行車騎將軍馬防,長水校尉耿恭副,討破之。於是臨洮、索西、迷吾等悉降。防乃築索西城,徙隴西南部都尉戍之,悉復諸亭候。至元和三年,迷吾復與弟號吾諸雜種反叛。秋,號吾先輕入寇隴西界,郡督烽掾李章追之,生得號吾,將詣郡。號吾曰:「獨殺我,無損於羌。誠得生歸,必悉罷兵,不復犯塞。」隴西太守張紆權宜放遣,羌即為解散,各歸故地,迷吾退居河北歸義城。傅育不欲失信伐之,乃募人鬥諸羌胡,羌胡不肯,遂復叛出塞,更依迷吾。

章和元年,育上請發隴西、張掖、酒泉各五千人,諸郡太守將之,育自領漢陽、金城五千人,合二萬兵,與諸郡剋期擊之,令隴西兵據河南,張掖、酒泉兵遮其西。並未及會,育軍獨進。迷吾聞之,徙廬落去。育選精騎三千窮追之,夜至建威南三兜谷,去虜數里,須旦擊之,不設備。迷吾乃伏兵三百人,夜突育營,營中驚壞散走,育下馬手戰,殺十餘人而死,死者八百八十人。及諸郡兵到,羌遂引去。育,北地人也。顯宗初,為臨羌長,與捕虜將軍馬武等擊羌滇吾,功冠諸軍;及在武威,威聲聞於匈奴。食祿數十年,秩奉盡贍給知友,妻子不免操井臼。肅宗下詔追褒美之。封其子毅為明進侯,七百戶。以隴西太守張紆代為校尉,將萬人屯臨羌。

迷吾既殺傅育,狃觇邊利。章和元年,復與諸種步騎七千人入金城塞。張紆遣從事司馬防將千餘騎及金城兵會戰於木乘谷,迷吾兵敗走,因譯使欲降,紆納之。遂將種人詣臨羌縣,紆設兵大會,施毒酒中,羌飲醉,紆因自擊,伏兵起,誅殺酋豪八百餘人。斬迷吾等五人頭,以祭育冢。復放兵擊在山谷閒者,斬首四百餘人,得生口二千餘人。迷吾子迷唐及其種人向塞號哭,與燒何、當煎、當闐等相結,以子女及金銀娉納諸種,解仇交質,將五千人寇隴西塞,太守寇盱與戰於白石,迷唐不利,引還大、小榆谷,北招屬國諸胡,會集附落,種眾熾盛,張紆不能討。永元元年,紆坐徵,以張掖太守鄧訓代為校尉,稍以賞賂離閒之,由是諸種少解。

東吾子東號立。是時號吾將其種人降。校尉鄧訓遣兵擊迷唐,迷唐去大、小榆谷,徙居頗巖谷。和帝永元四年,訓病卒,蜀郡太守聶尚代為校尉。尚見前人累征不克,欲以文德服之,乃遣驛使招呼迷唐,使還居大、小榆谷。迷唐既還,遣祖母卑缺詣尚,尚自送至塞下,為設祖道,令譯田汜等五人護送至廬落。迷唐因而反叛,遂與諸種共生屠裂汜等,以血盟詛,復寇金城塞。五年,尚坐徵免,居延都尉貫友代為校尉。友以迷唐難用德懷,終於叛亂,乃遣驛使搆離諸種,誘以財貨,由是解散。友乃遣兵出塞,攻迷唐於大、小榆谷,獲首虜八百餘人,收麥數萬斛,遂夾逢留大河築城塢,作大航,造河橋,欲度兵擊迷唐。迷唐乃率部落遠依賜支河曲。至八年,友病卒,漢陽太守史充代為校尉。充至,遂發湟中羌胡出塞擊迷唐,而羌迎敗充兵,殺數百人。明年,充坐徵,代郡太守吳祉代為校尉。其秋,迷唐率八千人寇隴西,殺數百人,乘勝深入,脅塞內諸種羌共為寇盜,眾羌復悉與相應,合步騎三萬人,擊破隴西兵,殺大夏長。遣行征西將軍劉尚、越騎校尉趙代副,將北軍五營、黎陽、雍營、三輔積射及邊兵羌胡三萬人討之。尚屯狄道,代屯枹罕。尚遣司馬寇盱監諸郡兵,四面並會。迷唐懼,棄老弱奔入臨洮南。尚等追至高山。迷唐窮迫,率其精強大戰。盱斬虜千餘人,得牛馬羊萬餘頭。迷唐引去。漢兵死傷亦多,不能復追,乃還入塞。明年,尚、代並坐畏懦徵下獄,免。謁者王信領尚營屯枹罕,謁者耿譚領代營屯白石。譚乃設購賞,諸種頗來內附。迷唐恐,乃請降。信、譚遂受降罷兵,遣迷唐詣闕。其餘種人不滿二千,飢窘不立,入居金城。和帝令迷唐將其種人還大、小榆谷。迷唐以為漢作河橋,兵來無常,故地不可復居,辭以種人飢餓,不肯遠出。吳祉等乃多賜迷唐金帛,令糴穀巿畜,促使出塞,種人更懷猜驚。十二年,遂復背叛,乃脅將湟中諸胡,寇鈔而去。王信、耿譚、吳祉皆坐徵,以酒泉太守周鮪代為校尉。明年,迷唐復還賜支河曲。

初,累姐種附漢,迷唐怨之,遂擊殺其酋豪,由是與諸種為讎,黨援益疏。其秋,迷唐復將兵向塞,周鮪與金城太守侯霸,及諸郡兵、屬國湟中月氏諸胡、隴西牢姐羌,合三萬人,出塞至允川,與迷唐戰。周鮪還營自守,唯侯霸兵陷陳,斬首四百餘級。羌眾折傷,種人瓦解,降者六千餘口,分徙漢陽、安定、隴西。迷唐遂弱,其種眾不滿千人,遠踰賜支河首,依發羌居。明年,周鮪坐畏懦徵,侯霸代為校尉。安定降羌燒何種脅諸羌數百人反叛,郡兵擊滅之,悉沒入弱口為奴婢。

時西海及大、小榆谷左右無復羌寇。隃麋相曹鳳上言:「西戎為害,前世所患,臣不能紀古,且以近事言之。自建武以來,其犯法者,常從燒當種起。所以然者,以其居大、小榆谷,土地肥美,又近塞內,諸種易以為非,難以攻伐。南得鍾存以廣其眾,北阻大河因以為固,又有西海魚鹽之利,緣山濱水,以廣田蓄,故能彊大,常雄諸種,恃其權勇,招誘羌胡。今者衰困,黨援壞沮,親屬離叛,餘勝兵者不過數百,亡逃棲竄,遠依發羌。臣愚以為宜及此時,建復西海郡縣,規固二榆,廣設屯田,隔塞羌胡交關之路,遏絕狂狡窺欲之源。又殖穀富邊,省委輸之役,國家可以無西方之憂。」於是拜鳳為金城西部都尉,將徙士屯龍耆。後金城長史上官鴻上開置歸義、建威屯田二十七部,侯霸復上置東西邯屯田五部,增留、逢二部,帝皆從之。列屯夾河,合三十四部。其功垂立。至永初中,諸羌叛,乃罷。迷唐失眾,病死。有一子來降,戶不滿數十。

東號子麻奴立。初隨父降,居安定。時諸降羌布在郡縣,皆為吏人豪右所徭役,積以愁怨。安帝永初元年夏,遣騎都尉王弘發金城、隴西、漢陽羌數百千騎征西域,弘迫促發遣,群羌懼遠屯不還,行到酒泉,多有散叛。諸郡各發兵儌遮,或覆其廬落。於是勒姐、當煎大豪東岸等愈驚,遂同時奔潰。麻奴兄弟因此遂與種人俱西出塞。

先零別種滇零與鍾羌諸種大為寇掠,斷隴道。時羌歸附既久,無復器甲,或持竹竿木枝以代戈矛,或負板案以為楯,或執銅鏡以象兵,郡縣畏懦不能制。冬,遣車騎將軍鄧騭,征西校尉任尚副,將五營及三河、三輔、汝南、南陽、潁川、太原、上黨兵合五萬人,屯漢陽。明年春,諸郡兵未及至,鍾羌數千人先擊敗騭軍於冀西,殺千餘人。校尉侯霸坐眾羌反叛徵免,以西域都護段禧代為校尉。其冬,騭使任尚及從事中郎司馬鈞率諸郡兵與滇零等數萬人戰於平襄,尚軍大敗,死者八千餘人。於是滇零等自稱「天子」於北地,招集武都、參狼、上郡、西河諸雜種,眾遂大盛,東犯趙、魏,南入益州,殺漢中太守董炳,遂寇鈔三輔,斷隴道。湟中諸縣粟石萬錢,百姓死亡不可勝數。朝廷不能制,而轉運難劇,遂詔騭還師,留任尚屯漢陽,為諸軍節度。朝廷以鄧太后故,迎拜騭為大將軍,封任尚樂亭侯,食邑三百戶。

三年春,復遣騎都尉任仁督諸郡屯兵救三輔。仁戰每不利,眾羌乘勝,漢兵數挫。當煎、勒姐種攻沒破羌縣,鍾羌又沒臨洮縣,生得隴西南部都尉。明年春,滇零遣人寇褒中,燔燒郵亭,大掠百姓。於是漢中太守鄭勤移屯褒中。軍營久出無功,有廢農桑,乃詔任尚將吏兵還屯長安,罷遣南陽、潁川、汝南吏士,置京兆虎牙都尉於長安,扶風都尉於雍,如西京三輔都尉故事。時羌復攻褒中,鄭勤欲擊之。主簿段崇諫,以為虜乘勝,鋒不可當,宜堅守待之。勤不從,出戰,大敗,死者三千餘人,段崇及門下史王宗、原展以身扞刃,與勤俱死。於是徙金城郡居襄武。任仁戰累敗,而兵士放縱,檻車徵詣廷尉詔獄死。段禧病卒,復以前校尉侯霸代之,遂移居張掖。五年春,任尚坐無功徵免。羌遂入寇河東,至河內,百姓相驚,多奔南度河。使北軍中候朱寵將五營士屯孟津,詔魏郡、趙國、常山、中山繕作塢候六百一十六所。

羌既轉盛,而二千石、令、長多內郡人,並無守戰意,皆爭上徙郡縣以避寇難。朝廷從之,遂移隴西徙襄武,安定徙美陽,北地徙池陽,上郡徙衙。百姓戀土,不樂去舊,遂乃刈其禾稼,發徹室屋,夷營壁,破積聚。時連旱蝗飢荒,而驅蹙劫略,流離分散,隨道死亡,或棄捐老弱,或為人僕妾,喪其太半。復以任尚為侍御史,擊眾羌於上黨羊頭山,破之,誘殺降者二百餘人,乃罷孟津屯。其秋,漢陽人杜琦及弟季貢、同郡王信等與羌通謀,聚眾入上邽城,琦自稱安漢將軍。於是詔購募得琦首者,封列侯,賜錢百萬,羌胡斬琦者賜金百斤,銀二百斤。漢陽太守趙博遣刺客杜習刺殺琦,封習討姦侯,賜錢百萬。而杜季貢、王信等將其眾據樗泉營。侍御史唐喜領諸郡兵討破之,斬王信等六百餘級,沒入妻子五百餘人,收金錢綵帛一億已上。杜貢亡從滇零。六年,任尚復坐徵免。滇零死,子零昌代立,年尚幼少,同種狼莫為其計策,以杜貢為將軍,別居丁奚城。七年夏,騎都尉馬賢與侯霸掩擊零昌別部牢羌於安定,首虜千人,得驢騾駱駝馬牛羊二萬餘頭,以畀得者。

元初元年春,遣兵屯河內,通谷衝要三十三所,皆作塢壁,設鳴鼓。零昌遣兵寇雍城,又號多與當煎、勒姐大豪共脅諸種,分兵鈔掠武都、漢中。巴郡板楯蠻將兵救之,漢中五官掾程信率壯士與蠻共擊破之。號多退走,還斷隴道,與零昌通謀。侯霸、馬賢將湟中吏人及降羌胡於枹罕擊之,斬首二百餘級。涼州刺史皮楊擊羌於狄道,大敗,死者八百餘人,楊坐徵免。侯霸病卒,漢陽太守龐參代為校尉。參以恩信招誘之。二年春,號多等率眾七千餘人詣參降,遣詣闕,賜號多侯印綬遣之。參紿還居令居,通河西道。而零昌種眾復分寇益州,遺中郎將尹就將南陽兵,因發益部諸郡屯兵擊零昌黨呂叔都等。至秋,蜀人陳省、羅橫應募,刺殺叔都,皆封侯賜錢。又使屯騎校尉班雄屯三輔,遣左馮翊司馬鈞行征西將軍,督右扶風仲光、安定太守杜恢、北地太守盛包、京兆虎牙都尉耿溥、右扶風都尉皇甫旗等,合八千餘人,又龐參將羌胡兵七千餘人,與鈞分道並北擊零昌。參兵至勇士東,為杜季貢所敗,於是引退。鈞等獨進,攻拔丁奚城,大克獲。杜秀貢率眾偽逃。鈞令光、恢、包等收羌禾稼、光等違鈞節度,散兵深入,羌乃設伏要擊之。鈞在城中,怒而不救,光並沒,死者三千餘人。鈞乃遣還,坐徵自殺。龐參以失期軍敗抵罪,以馬賢代領校尉事。後遣任尚為中郎將,將羽林、緹騎、五營子弟三千五百人,代班雄屯三輔。尚臨行,懷令虞詡說尚曰:「使君頻奉國命討逐寇賊,三州屯兵二十餘萬人,棄農桑,疲苦徭役,而未有功效,勞費日滋。若此出不克,誠為使君危之。」尚曰:「憂惶久矣,不知所如。」詡曰:「兵法弱不攻強,走不逐飛,自然之埶也。今虜皆馬騎,日行數百,來如風雨,去如絕弦,以步追之,埶不相及,所以曠而無功也。為使君計者,莫如罷諸郡兵,各令出錢數千,二十人共市一馬,如此,可捨甲冑,馳輕兵,以萬騎之眾,逐數千之虜,追尾掩涞,其道自窮。便人利事,大功立矣。」尚大喜,即上言用其計。乃遣輕騎鈔擊杜季貢於丁奚城,斬首四百餘級,獲牛馬羊數千頭。

明年夏,度遼將軍鄧遵率南單于及左鹿蠡王須沈萬騎,擊零昌於靈州,斬首八百餘級,封須沈為破虜侯,金印紫綬,賜金帛各有差。任尚遣兵擊破先零羌於丁奚城。秋,築馮翊北界候塢五百所。任尚又遣假司馬募陷陳士,擊零昌於北地,殺其妻子,得牛馬羊二萬頭,燒其廬落,斬首七百餘級,得僭號文書及所沒諸將印綬。

四年春,尚遣當闐種羌榆鬼等五人刺殺杜季貢,封榆鬼為破羌侯。其夏,尹就以不能定益州,坐徵抵罪,以益州刺史張喬領尹就軍屯。招誘叛羌,稍稍降散。秋,任尚復募效功種號封刺殺零昌,封號封為羌王。冬,任尚將諸郡兵與馬賢並進北地擊狼莫,賢先至安定青石岸,狼莫逆擊敗之。會尚兵到高平,因合埶俱進,狼莫等引退,乃轉營迫之,至北地,相持六十餘日,戰於富平河上,大破之,斬首五千級,還得所略人男女千餘人,牛馬驢羊駱馳十餘萬頭,狼莫逃走,於是西河虔人種羌萬一千口詣鄧遵降。

五年,鄧遵募上郡全無種羌雕何等刺殺狼莫,賜雕何為羌侯,封遵武陽侯,三千戶。遵以太后從弟故,爵封優大。任尚與遵爭功,又詐增首級,受賕枉法,臧千萬已上,檻車徵棄市,沒入田廬奴婢財物。自零昌、狼莫死後,諸羌瓦解,三輔、益州無復寇儆。

自羌叛十餘年閒,兵連師老,不暫寧息。軍旅之費,轉運委輸,用二百四十餘億,府帑空竭。延及內郡,邊民死者不可勝數,并涼二州遂至虛耗。

六年春,勒姐種與隴西種羌號良等通謀欲反,馬賢逆擊之於安故,斬號良及種人數百級,皆降散。

永寧元年春,上郡沈氐種羌五千餘人復寇張掖。其夏,馬賢將萬人擊之。初戰失利,死者數百人,明日復戰,破之,斬首千八百級,獲生口千餘人,馬牛羊以萬數,餘虜悉降。時當煎種大豪飢〈五〉等,以賢兵在張掖,乃乘虛寇金城,賢還軍追之出塞,斬首數千級而還。燒當、燒何種聞賢軍還,率三千餘人復寇張掖,殺長吏。初,飢五同種大豪盧璴、忍良等千餘戶別留允街,而首施兩端。建光元年春,馬賢率兵召盧璴斬之,因放兵擊其種人,首虜二千餘人,掠馬牛羊十萬頭,忍良等皆亡出塞。璽書封賢安亭侯,食邑千戶。忍良等以麻奴兄弟本燒當世嫡,而賢撫恤不至,常有怨心。秋,遂相結共脅將諸種步騎三千人寇湟中,攻金城諸縣。賢將先零種赴擊之,戰於牧苑,兵敗,死者四百餘人。麻奴等又敗武威、張掖郡兵於令居,因脅將先零、沈氐諸種四千餘戶,緣山西走,寇武威。賢追到鸞鳥,招引之,諸種降者數千,麻奴南還湟中。延光元年春,賢追到湟中,麻奴出塞度河,賢復追擊戰破之,種眾散遁,詣涼州刺史宗漢降。麻奴等孤弱飢困,其年冬,將種眾三千餘戶詣漢陽太守耿种降。安帝假金印紫綬,賜金銀綵繒各有差。是歲,虔人種羌與上郡胡反,攻穀羅城,度遼將軍耿夔將諸郡兵及烏桓騎赴擊破之。三年秋,隴西郡始還狄道焉。麻奴弟犀苦立。

順帝永建元年,隴西鍾羌反,校尉馬賢將七千餘人擊之,戰於臨洮,斬首千餘級,皆率種人降。進封賢都鄉侯。自是涼州無事。

至四年,尚書僕射虞詡上疏曰:「臣聞子孫以奉祖為孝,君上以安民為明,此高宗、周宣所以上配湯、武也。禹貢雍州之域,厥田惟上。且沃野千里,穀稼殷積,又有龜茲鹽池以為民利。水草豐美,土宜產牧,牛馬銜尾,群羊塞道。北阻山河,乘阨據險。因渠以溉,水舂河漕。用功省少,而軍糧饒足。故孝武皇帝及光武築朔方,開西河,置上郡,皆為此也。而遭元元無妄之災,眾羌內潰,郡縣兵荒二十餘年。夫棄沃壤之饒,損自然之財,不可謂利;離河山之阻,守無險之處,難以為固。今三郡未復,園陵單外,而公卿選懦,容頭過身,張解設難,但計所費,不圖其安。宜開聖德,考行所長。」書奏,帝乃復三郡。使謁者郭璜督促徙者,各歸舊縣,繕城郭,置候驛。既而激河浚渠為屯田,省內郡費歲一億計。遂令安定、北地、上郡及隴西、金城常儲穀粟,令周數年。

馬賢以犀苦兄弟數背叛,因繫質於令居。其冬,賢坐徵免,右扶風韓皓代為校尉。明年,犀苦詣皓自言求歸故地,皓復不遣。因轉湟中屯田,置兩河閒,以逼群羌。皓復坐徵,張掖太守馬續代為校尉。兩河閒羌以屯田近之,恐必見圖,乃解仇詛盟,各自儆備。續欲先示恩信,乃上移屯田還湟中,羌意乃安。至陽嘉元年,以湟中地廣,更增置屯田五部,并為十部。二年夏,復置隴西南部都尉如舊制。

三年,鍾羌良封等復寇隴西、漢陽,詔拜前校尉馬賢為謁者,鎮撫諸種。馬續遣兵擊良封,斬首數百級。四年,馬賢亦發隴西吏士及羌胡兵擊殺良封,斬首千八百級,獲馬牛羊五萬餘頭,良封親屬並詣實降。賢復進擊鍾羌且昌,且昌等率諸種十餘萬詣涼州刺史降。永和元年,馬續遷度遼將軍,復以馬賢代為校尉。初,武都塞上白馬羌攻破屯官,反叛連年。二年春,廣漢屬國都尉擊破之,斬首六百餘級,馬賢又擊斬其渠帥飢指累祖等三百級,於是隴右復平。明年冬,燒當種那離等三千餘騎寇金城塞,馬賢將兵赴擊,斬首四百餘級,獲馬千四百匹。那離等復西招羌胡,殺傷吏民。

四年,馬賢將湟中義從兵及羌胡萬餘騎掩擊那離等,斬之,獲首虜千二百餘級,得馬騾羊十萬餘頭。徵賢為弘農太守,以來機為并州刺史,劉秉為涼州刺史,並當之職。大將軍梁商謂機等曰:「戎狄荒服,蠻夷要服,言其荒忽無常。而統領之道,亦無常法,臨事制宜,略依其俗。今三君素性疾惡,欲分明白黑。孔子曰:『人而不仁,疾之已甚,亂也。』況戎狄乎!其務安羌胡,防其大故,忍其小過。」機等天性虐刻,遂不能從。到州之日,多所擾發。

五年夏,且凍、傅難種羌等遂反叛,攻金城,與西塞及湟中雜種羌胡大寇三輔,殺害長吏。機、秉並坐徵。於是發京師近郡及諸州兵討之,拜馬賢為征西將軍,以騎都尉耿叔副,將左右羽林、五校士及諸州郡兵十萬人屯漢陽。又於扶風、漢陽、隴道作塢壁三百所,置屯兵,以保聚百姓。且凍分遣種人寇武都,燒隴關,掠苑馬。六年春,馬賢將五六千騎擊之,到射姑山,賢軍敗,賢及二子皆戰歿。順帝愍之,賜布三千匹,穀千斛,封賢孫光為舞陽亭侯,租入歲百萬。遣侍御史督錄征西營兵,存恤死傷。

於是東西羌遂大合。鞏唐種三千餘騎寇隴西,又燒園陵,掠關中,殺傷長吏,郃陽令任頵追擊,戰死。遣中郎將龐浚募勇士千五百人頓美陽,為涼州援。武威太守趙沖追擊鞏唐羌,斬首四百餘級,得馬牛羊驢萬八千餘頭,羌二千餘人降。詔沖督河西四郡兵為節度。罕種羌千餘寇北地,北地太守賈福與趙沖擊之,不利。秋,諸種八九千騎寇武威,涼部震恐。於是復徙安定居扶風,北地居馮翊,遣行車騎將軍執金吾張喬將左右羽林、五校士及河內、南陽、汝南兵萬五千屯三輔。漢安元年,以趙沖為護羌校尉。沖招懷叛羌,罕種乃率邑落五千餘戶詣沖降。於是罷張喬軍屯。唯燒何種三千餘落據參讀北界。三年夏,趙沖與漢陽太守張貢掩擊之,斬首千五百級,得牛羊驢十八萬頭。冬,沖擊諸種,斬首四千餘級。詔沖一子為郎。沖復追擊於阿陽,斬首八百級。於是諸種前後三萬餘戶詣涼州刺史降。

建康元年春,護羌從事馬玄遂為諸羌所誘,將羌眾亡出塞,領護羌校尉衛瑤追擊玄等,斬首八百餘級,得牛馬羊二十餘萬頭。趙沖復追叛羌到建威鸇陰河。軍度竟,所將降胡六百餘人叛走,沖將數百人追之,遇羌伏兵,與戰歿。沖雖身死,而前後多所斬獲,羌由是衰耗。永嘉元年,封沖子愷義陽亭侯。以漢陽太守張貢代為校尉。左馮翊梁並稍以恩信招誘之,於是離湳、狐奴等五萬餘戶詣並降,隴右復平。並,大將軍冀之宗人。封為鄠侯,邑二千戶。

自永和羌叛,至乎是歲,十餘年閒,費用八十餘億。諸將多斷盜牢稟,私自潤入,皆以珍寶貨賂左右,上下放縱,不恤軍事,士卒不得其死者,白骨相望於野。

桓帝建和二年,白馬羌寇廣漢屬國,殺長吏。是時西羌及湟中胡復畔為寇,益州刺史率板楯蠻討破之,斬首招降二十萬人。

永壽元年,校尉張貢卒,以前南陽太守第五訪代為校尉,甚有威惠,西垂無事。延熹二年,訪卒,以中郎將段熲代為校尉。時燒當八種寇隴右,熲擊大破之。四年,零吾復與先零及上郡沈氐、牢姐諸種并力寇并、涼及三輔。會段熲坐事徵,以濟南相胡閎代為校尉。閎無威略,羌遂陸梁,覆沒營塢,寇患轉盛,中郎將皇甫規擊破之。五年,沈氐諸種復寇張掖、酒泉,皇甫規招之,皆降。事已具規傳。鳥吾種復寇漢陽,隴西、金城諸郡兵共擊破之,各還降附。至冬,滇那等五六千人復攻武威、張掖、酒泉,燒民廬舍。六年,隴西太守孫羌擊破之,斬首溺死三千餘人。胡閎疾,復以段熲為校尉。

永康元年,東羌岸尾等脅同種連寇三輔,中郎將張奐追破斬之,事已具奐傳。當煎羌寇武威,破羌將軍段熲復破滅之,餘悉降散。事已具熲傳。靈帝建寧三年,燒當羌奉使貢獻。中平元年,北地降羌先零種因黃巾大亂,乃與漢中羌、義從胡北宮伯玉等反,寇隴右。事已具董卓傳。興平元年,馮翊降羌反,寇諸縣,郭汜、樊稠擊破之,斬首數千級。

自爰劍後,子孫支分凡百五十種。其九種在賜支河首以西,及在蜀、漢徼北,前史不載口數。唯參狼在武都,勝兵數千人。其五十二種衰少,不能自立,分散為附落,或絕滅無後,或引而遠去。其八十九種,唯鍾最強,勝兵十餘萬。其餘大者萬餘人,小者數千人,更相鈔盜,盛衰無常,無慮順帝時勝兵合可二十萬人。發羌、唐旄等絕遠,未嘗往來。氂牛、白馬羌在蜀、漢,其種別名號,皆不可紀知也。建武十三年,廣漢塞外白馬羌豪樓登等率種人五千餘戶內屬,光武封樓登為歸義君長。至和帝永元六年,蜀郡徼外大牂夷種羌豪造頭等率種人五十餘萬口內屬,拜造頭為邑君長,賜印綬。至安帝永初元年,蜀郡徼外羌龍橋等六種萬七千二百八十口內屬。明年,蜀郡徼外羌薄申等八種三萬六千九百口復舉土內屬。冬,廣漢塞外參狼種羌二千四百口復來內屬。桓帝建和二年,白馬羌千餘人寇廣漢屬國,殺長吏,益州刺史率板楯蠻討破之。

湟中月氏胡,其先大月氏之別也,舊在張掖、酒泉地。月氏王為匈奴冒頓所殺,餘種分散,西踰冈領。其羸弱者南入山阻,依諸羌居止,遂與共婚姻。及驃騎將軍霍去病破匈奴,取西河地,開湟中,於是月氏來降,與漢人錯居。雖依附縣官,而首施兩端。其從漢兵戰鬥,隨埶強弱。被服飲食言語略與羌同,亦以父名母姓為種。其大種有七,勝兵合九千餘人,分在湟中及令居。又數百戶在張掖,號曰義從胡。中平元年,與北宮伯玉等反,殺護羌校尉泠徵、金城太守陳懿,遂寇亂隴右焉。

論曰:羌戎之患,自三代尚矣。漢世方之匈奴,頗為衰寡,而中興以後,邊難漸大。朝規失綏御之和,戎帥騫然諾之信。其內屬者,或倥傯於豪右之手,或屈折於奴僕之勤。塞候時清,則憤怒而思禍;桴革暫動,則屬鞬以鳥驚。故永初之閒,群種蜂起。遂解仇嫌。結盟詛,招引山豪,轉相嘯聚,揭木為兵,負柴為械。穀馬揚埃,陸梁於三輔;建號稱制,恣睢於北地。東犯趙、魏之郊,南入漢、蜀之鄙,塞湟中,斷隴道,燒陵園,剽城市,傷敗踵係,羽書日聞。并、涼之士,特衝殘斃,壯悍則委身於兵場,女婦則徽纆而為虜,發冢露胔,死生塗炭。自西戎作逆,未有陵斥上國若斯其熾也。和熹以女君親政,威不外接。朝議憚兵力之損,情存苟安。或以邊州難援,宜見捐棄;或懼疽食浸淫,莫知所限。謀夫回遑,猛士疑慮,遂徙西河四郡之人,雜寓關右之縣。發屋伐樹,塞其戀土之心;燔破貲積,以防顧還之思。於是諸將鄧騭、任尚、馬賢、皇甫規、張奐之徒,爭設雄規,更奉征討之命,徵兵會眾,以圖其隙。馳騁東西,奔救首尾,搖動數州之境,日耗千金之資。至於假人增賦,借奉侯王,引金錢縑綵之珍,徵糧粟鹽鐵之積。所以賂遺購賞,轉輸勞來之費,前後數十巨萬。或梟剋酋健,摧破附落,降俘載路,牛羊滿山。軍書未奏其利害,而離叛之狀已言矣。故得不酬失,功不半勞。暴露師徒,連年而無所勝。官人屈竭,烈士憤喪,段熲受事,專掌軍任,資山西之猛性,練戎俗之態情,窮武思盡飆銳以事之。被羽前登,身當百死之陳,蒙沒冰雪,經履千折之道,始殄西種,卒定東寇。若乃陷擊之所殲傷,追走之所崩籍,頭顱斷落於萬丈之山,支革判解於重崖之上,不可校計。其能穿竄草石,自脫於鋒鏃者,百不一二。而張奐盛稱「戎狄一氣所生,不宜誅盡,流血汙野,傷和致妖」。是何言之迂乎!羌雖外患,實深內疾,若攻之不根,是養疾槙於心腹也。惜哉寇敵略定矣,而漢祚亦衰焉。嗚呼!昔先王疆理九土,判別畿荒,知夷貊殊性,難以道御,故斥遠諸華,薄其貢職,唯與辭要而已。若二漢御戎之方,失其本矣。何則?先零侵境,趙充國遷之內地;當煎作寇,馬文淵徙之三輔。貪其暫安之埶,信其馴服之情,計日用之權宜,忘經世之遠略,豈夫識微者之為乎?故微子垂泣於象箸,辛有浩歎於伊川也。

贊曰:金行氣剛,播生西羌。氐豪分種,遂用殷彊。虔劉隴北。假僭涇陽。朝勞內謀,兵憊外攘。


\end{pinyinscope}