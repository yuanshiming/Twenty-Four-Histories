\article{伏侯宋蔡馮趙牟韋列傳}

\begin{pinyinscope}
伏湛字惠公,琅邪東武人也。九世祖勝,字子賤,所謂濟南伏生者也。湛高祖父孺,武帝時,客授東武,因家焉。父理,為當世名儒,以詩授成帝,為高密太傅,別自名學。

湛性孝友,少傳父業,教授數百人。成帝時,以父任為博士弟子。五遷,至王莽時為繡衣執法,使督大姦,遷後隊屬正。

更始立,以為平原太守。時倉卒兵起,天下驚擾,而湛獨晏然,教授不廢。謂妻子曰:「夫一穀不登,國君徹膳;今民皆飢,奈何獨飽?」乃共食麤糲,悉分奉祿以賑鄉里,來客者百餘家。時門下督素有氣力,謀欲為湛起兵,湛惡其惑眾,即收斬之,徇首城郭,以示百姓,於是吏人信向,郡內以安。平原一境,湛所全也。

光武即位,知湛名儒舊臣,欲令幹任內職,徵拜尚書,使典定舊制。時大司徒鄧禹西征關中,帝以湛才任宰相,拜為司直,行大司徒事。車駕每出征伐,常留鎮守,總攝群司。建武三年,遂代鄧禹為大司徒,封陽都侯。

時彭寵反於漁陽,帝欲自征之,湛上疏諫曰:「臣聞文王受命而征伐五國,必先詢之同姓,然後謀於群臣,加占蓍龜,以定行事,故謀則成,卜則吉,戰則勝。其《詩》曰:『帝謂文王,詢爾仇方,同爾弟兄,以爾鉤援,與爾臨衝,以伐崇庸。』崇國城守,先退後伐,所以重人命,俟時而動,故參分天下而有其二。陛下承大亂之極,受命而帝,興明祖宗,出入四年,而滅檀鄉,制五校,降銅馬,破赤眉,誅鄧奉之屬,不為無功。今京師空匱,資用不足,未能服近而先事邊外;且漁陽之地,逼接北狄,黠虜困迫,必求其助。又今所過縣邑,尤為困乏。種麥之家,多在城郭,聞官兵將至,當已收之矣。大軍遠涉二千餘里,士馬罷勞,轉糧艱阻。今兗、豫、青、冀,中國之都,而寇賊從橫,未及從化。漁陽以東,本備邊塞,地接外虜,貢稅微薄。安平之時,尚資內郡,況今荒耗,豈足先圖?而陛下捨近務遠,棄易求難,四方疑怪,百姓恐懼,誠臣之所惑也。復願遠覽文王重兵博謀,近思征伐前後之宜,顧問有司,使極愚誠,采其所長,擇之聖慮,以中土為憂念。」帝覽其奏,竟不親征。

時賊徐異卿等萬餘人據富平,連攻之不下,唯云「願降司徒伏公」。帝知湛為青、徐所信向,遣到平原,異卿等即日歸降,護送洛陽。

湛雖在倉卒,造次必於文德,以為禮樂政化之首,顛沛猶不可違。是歲奏行鄉飲酒禮,遂施行也。

其冬,車駕征張步,留湛居守。時蒸祭高廟,而河南尹、司隸校尉於廟中爭論,湛不舉奏,坐策免。六年,徙封不其侯,邑三千六百戶,遣就國。後南陽太守杜詩上疏薦湛曰:「臣聞唐、虞以股肱康,文王以多士寧,是故詩稱『濟濟』,書曰『良哉』。臣詩竊見故大司徒陽都侯伏湛,自行束脩,訖無毀玷,篤信好學,守死善道,經為人師,行為儀表。前在河內朝歌及居平原,吏人畏愛,則而象之。遭時反覆,不離兵凶,秉節持重,有不可奪之志。陛下深知其能,顯以宰相之重,眾賢百姓,仰望德義。微過斥退,久不復用,有識所惜,儒士痛心,臣竊傷之。湛容貌堂堂,國之光暉;智略謀慮,朝之淵藪。髫髮厲志,白首不衰。實足以先後王室,名足以光示遠人。古者選擢諸侯以為公卿,是故四方回首,仰望京師。柱石之臣,宜居輔弼,出入禁門,補缺拾遺。臣詩愚戇,不足以知宰相之才,竊懷區區,敢不自竭。臣前為侍御史,上封事,言湛公廉愛下,好惡分明,累世儒學,素持名信,經明行修,通達國政,尤宜近侍,納言左右,舊制九州五尚書,令一郡二人,可以湛代。頗為執事所非。但臣詩蒙恩深渥,所言誠有益於國,雖死無恨,故復越職觸冒以聞。」

十三年夏,徵,敕尚書擇拜吏日,未及就位,因讌見中暑,病卒。賜祕器,帝親弔祠,遣使者送喪脩冢。

二子:隆,翕。

翕嗣爵,卒,子光嗣。光卒,子晨嗣。晨謙敬博愛,好學尤篤,以女孫為順帝貴人,奉朝請,位特進。卒,子無忌嗣,亦傳家學,博物多識,順帝時,為侍中屯騎校尉。永和元年,詔無忌與議郎黃景校定中書五經、諸子百家、蓺術。元嘉中,桓帝復詔無忌與黃景、崔寔等共撰漢記。又自采集古今,刪著事要,號曰伏侯注。無忌卒,子質嗣,官至大司農。質卒,子完嗣,尚桓帝女陽安長公主。女為孝獻皇后。曹操殺后,誅伏氏,國除。

初,自伏生已後,世傳經學,清靜無競,故東州號為「伏不鬥」云。

隆字伯文,少以節操立名,仕郡督郵。建武二年,詣懷宮,光武甚親接之。

時張步兄弟各擁彊兵,據有齊地,拜隆為太中大夫,持節使青徐二州,招降郡國。隆移檄告曰:「乃者,猾臣王莽,殺帝盜位。宗室興兵,除亂誅莽,故群下推立聖公,以主宗廟。而任用賊臣,殺戮賢良,三王作亂,盜賊從橫,忤逆天心,卒為赤眉所害。皇天祐漢,聖哲應期,陛下神武奮發,以少制眾。故尋、邑以百萬之軍,潰散於昆陽,王郎以全趙之師,土崩於邯鄲,大肜、高胡望旗消靡,鐵脛、五校莫不摧破。梁王劉永,幸以宗室屬籍,爵為侯王,不知厭足,自求禍棄,遂封爵牧守,造為詐逆。今虎牙大將軍屯營十萬,已拔睢陽,劉永奔迸,家已族矣。此諸君所聞也。不先自圖,後悔何及?」青、徐群盜得此惶怖,獲索賊右師郎等六校即時皆降。張步遣使隨隆,詣闕上書,獻鰒魚。

其冬,拜隆光祿大夫,復使於步,并與新除青州牧守及都尉俱東,詔隆輒拜令長以下。隆招懷綏緝,多來降附。帝嘉其功,比之酈生。即拜步為東萊太守,而劉永亦復遣使立步為齊王。步貪受王爵,冘豫未決。隆曉譬曰:「高祖與天下約,非劉氏不王,今可得為十萬戶侯耳。」步欲留隆與共守二州,隆不聽,求得反命,步遂執隆而受永封。隆遣閒使上書曰:「臣隆奉使無狀,受執凶逆,雖在困厄,授命不顧。又吏人知步反畔,心不附之,願以時進兵,無以臣隆為念。臣隆得生到闕廷,受誅有司,此其大願;若令沒身寇手,以父母昆弟長累陛下。陛下與皇后、太子永享萬國,與天無極。」帝得隆奏,召父湛流涕以示之曰:「隆可謂有蘇武之節。恨不且許而遽求還也!」其後步遂殺之,時人莫不憐哀焉。

五年,張步平,車駕幸北海,詔隆中弟咸收隆喪,賜給棺斂,太中大夫護送喪事,詔告琅邪作冢,以子瑗為郎中。

侯霸字君房,河南密人也。族父淵,以宦者有才辯,任職元帝時,佐石顯等領中書,號曰大常侍。成帝時,任霸為太子舍人。霸矜嚴有威容,家累千金,不事產業。篤志好學,師事九江太守房元,治穀梁春秋,為元都講。王莽初,五威司命陳崇舉霸德行,遷隨宰。縣界曠遠,濱帶江湖,而亡命者多為寇盜。霸到,即案誅豪猾,分捕山賊,縣中清靜。再遷為執法刺姦,糾案埶位者,無所疑憚。後為淮平大尹,政理有能名。及王莽之敗,霸保固自守,卒全一郡。

更始元年,遣使徵霸,百姓老弱相攜號哭,遮使者車,或當道而臥。皆曰:「願乞侯君復留期年。」民至乃戒乳婦勿得舉子,侯君當去,必不能全。使者慮霸就徵,臨淮必亂,不敢授璽書,具以狀聞。會更始敗,道路不通。

建武四年,光武徵霸與車駕會壽春,拜尚書令。時無故典,朝廷又少舊臣,霸明習故事,收錄遺文,條奏前世善政法度有益於時者,皆施行之。每春下寬大之詔,奉四時之令,皆霸所建也。明年,代伏湛為大司徒,封關內侯。在位明察守正,奉公不回。

十三年,霸薨,帝深傷惜之,親自臨弔。下詔曰:「惟霸積善清絜。視事九年。漢家舊制,丞相拜日,封為列侯。朕以軍師暴露,功臣未封,緣忠臣之義,不欲相踰,未及爵命,奄然而終。嗚呼哀哉!」於是追封謚霸則鄉哀侯,食邑二千六百戶。子昱嗣。臨淮吏人共為立祠,四時祭焉。以沛郡太守韓歆代霸為大司徒。

歆字翁君,南陽人,以從攻伐有功,封扶陽侯。好直言,無隱諱,帝每不能容。嘗因朝會,聞帝讀隗囂、公孫述相與書,歆曰:「亡國之君皆有才,桀紂亦有才。」帝大怒,以為激發。歆又證歲將飢凶,指天畫地,言甚剛切,坐免歸田里。帝猶不釋,復遣使宣詔責之。司隸校尉鮑永固請不能得,歆及子嬰竟自殺。歆素有重名,死非其罪,眾多不厭,帝乃追賜錢穀,以成禮葬之。

後千乘歐陽歙、清河戴涉相代為大司徒,坐事下獄死,自是大臣難居相任。其後河南蔡茂,京兆玉況,魏郡馮勤,皆得薨位。況字文伯,性聰敏,為陳留太守,以德行化人,遷司徒,四年薨。

昱後徙封於陵侯,永平中兼太僕。昱卒,子建嗣。建卒,子昌嗣。

宋弘字仲子,京兆長安人也。父尚,成帝時至少府;哀帝立,以不附董賢,違忤抵罪。弘少而溫順,哀平閒作侍中,王莽時為共工。赤眉入長安,遣使徵弘,逼迫不得已,行至渭橋,自投於水,家人救得出,因佯死獲免。

光武即位,徵拜太中大夫。建武二年,代王梁為大司空,封栒邑侯。所得租奉分贍九族,家無資產,以清行致稱。徙封宣平侯。

帝嘗問弘通博之士,弘乃薦沛國桓譚才學洽聞,幾能及楊雄、劉向父子。於是召譚拜議郎、給事中。帝每讌,輒令鼓琴,好其繁聲。弘聞之不悅,悔於薦舉,伺譚內出,正朝服坐府上,遣吏召之。譚至,不與席而讓之曰:「吾所以薦子者,欲令輔國家以道德也,而今數進鄭聲以亂雅頌,非忠正者也。能自改邪?將令相舉以法乎?」譚頓首辭謝,良久乃遣之。後大會群臣,帝使譚鼓琴,譚見弘,失其常度。帝怪而問之。弘乃離席免冠謝曰:「臣所以薦桓譚者,望能以忠正導主,而令朝廷耽悅鄭聲,臣之罪也。」帝改容謝,使反服,其後遂不復令譚給事中。弘推進賢士馮翊、桓梁三十餘人,或相及為公卿者。

弘當讌見,御坐新屏風,圖畫列女,帝數顧視之。弘正容言曰:「未見好德如好色者。」帝即為徹之。笑謂弘曰:「聞義則服,可乎?」對曰:「陛下進德,臣不勝其喜。」

時帝姊湖陽公主新寡,帝與共論朝臣,微觀其意。主曰:「宋公威容德器,群臣莫及。」帝曰:「方且圖之。」後弘被引見,帝令主坐屏風後,因謂弘曰:「諺言貴易交,富易妻,人情乎?」弘曰:「臣聞貧賤之知不可忘,糟糠之妻不下堂。」帝顧謂主曰:「事不諧矣。」

弘在位五年,坐考上黨太守無所據,免歸第。數年卒,無子,國除。

弘弟嵩,以剛彊孝烈著名,官至河南尹。嵩子由,章和閒為太尉,坐阿黨竇憲,策免歸本郡,自殺。由二子:漢,登。登在儒林傳。

漢字仲和,以經行著名,舉茂才,四遷西河太守。永建元年,為東平相、度遼將軍,立名節,以威恩著稱。遷太僕,上病自乞,拜太中大夫,卒。策曰:「太中大夫宋漢,清修雪白,正直無邪。前在方外,仍統軍實,懷柔異類,莫匪嘉績,戎車載戢,邊人用寧。予錄乃勳,引登九列。因病退讓,守約彌堅,將授三事,未剋而終。朝廷癏悼,怛其愴然。詩不云乎:『肇敏戎功,用錫爾祉。』其令將相大夫會葬,加賜錢十萬,及其在殯,以全素絲羔羊之絜焉。」

子則,字元矩,為鄢陵令,亦有名跡。拔同郡韋著、扶風法真,稱為知人。則子年十歲,與蒼頭共弩射,蒼頭弦斷矢激,誤中之,即死。奴叩頭就誅,則察而恕之。潁川荀爽深以為美,時人亦服焉。

論曰:中興以後,居台相總權衡多矣,其能以任職取名者,豈非先遠業後小數哉?故惠公造次,急於鄉射之禮;君房入朝,先奏寬大之令。夫器博者無近用,道長者其功遠,蓋志士仁人所為根心者也。君子以之得,固貴矣;以之失,亦得矣。宋弘止繁聲,戒淫色,其有關雎之風乎!

蔡茂字子禮,河內懷人也。哀平閒以儒學顯,徵試博士,對策陳災異,以高等擢拜議郎,遷侍中。遇王莽居攝,以病自免,不仕莽朝。

會天下擾亂,茂素與竇融善,因避難歸之。融欲以為張掖太守,固辭不就;每所餉給,計口取足而已。後與融俱徵,復拜議郎,再遷廣漢太守,有政績稱。時陰氏賓客在郡界多犯吏禁,茂輒糾案,無所回避。會洛陽令董宣舉糾湖陽公主,帝始怒收宣,既而赦之。茂喜宣剛正,欲令朝廷禁制貴戚,乃上書曰:「臣聞興化致教,必由進善;康國寧人,莫大理惡。陛下聖德係興,再隆大命,即位以來,四海晏然。誠宜夙興夜寐,雖休勿休。然頃者貴戚椒房之家,數因恩埶,干犯吏禁,殺人不死,傷人不論。臣恐繩墨棄而不用,斧斤廢而不舉。近湖陽公主奴殺人西市,而與主共輿,出入宮省,逋罪積日,冤魂不報。洛陽令董宣,直道不顧,干主討姦。陛下不先澄審,召欲加箠。當宣受怒之初,京師側耳;及其蒙宥,天下拭目。今者,外戚憍逸,賓客放濫,宜敕有司案理姦罪,使執平之吏永申其用,以厭遠近不緝之情。」光武納之。

建武二十年,代戴涉為司徒,在職清儉匪懈。二十三年薨于位,時年七十二。賜東園梓棺,賻贈甚厚。

茂初在廣漢,夢坐大殿,極上有三穗禾,茂跳取之,得其中穗,輒復失之。以問主簿郭賀,賀離席慶曰:「大殿者,宮府之形象也。極而有禾,人臣之上祿也。取中穗,是中台之位也。於字禾失為秩,雖曰失之,乃所以得祿秩也。袞職有闕,君其補之。」旬月而茂徵焉,乃辟賀為掾。

賀字喬卿,雒陽人。祖父堅伯,父游君,並修清節,不仕王莽。賀能明法,累官,建武中為尚書令,在職六年,曉習故事,多所匡益。拜荊州刺史,引見賞賜,恩寵隆異。及到官,有殊政。百姓便之,歌曰:「厥德仁明郭喬卿,忠正朝廷上下平。」顯宗巡狩到南陽,特見嗟歎,賜以三公之服,黼黻冕旒。敕行部去襜帷,使百姓見其容服,以章有德。每所經過,吏人指以相示,莫不榮之。永平四年,徵拜河南尹,以清靜稱。在官三年卒,詔書癏惜,賜車一乘,錢四十萬。

馮勤字偉伯,魏郡繁陽人也。曾祖父揚,宣帝為弘農太守。有八子,皆為二千石,趙魏閒榮之,號曰「萬石君」焉。兄弟形皆偉壯,唯勤祖父偃,長不滿七尺,常自恥短陋,恐子孫之似也,乃為子伉娶長妻。伉生勤,長八尺三寸。八歲善計。

初為太守銚期功曹,有高能稱。期常從光武征伐,政事一以委勤。勤同縣馮巡等舉兵應光武,謀未成而為豪右焦廉等所反,勤乃率將老母兄弟及宗親歸期,期悉以為腹心,薦於光武。初未被用,後乃除為郎中,給事尚書。以圖議軍糧,在事精勤,遂見親識。每引進,帝輒顧謂左右曰:「佳乎吏也!」由是使典諸侯封事。勤差量功次輕重,國土遠近,地埶豐薄,不相踰越,莫不厭服焉。自是封爵之制,非勤不定。帝益以為能,尚書眾事,皆令總錄之。

司徒侯霸薦前梁令閻楊。楊素有譏議,帝常嫌之,既見霸奏,疑其有姦,大怒,賜霸璽書曰:「崇山、幽都何可偶,黃鉞一下無處所。欲以身試法邪?將殺身以成仁邪?」使勤奉策至司徒府。勤還,陳霸本意,申釋事理,帝意稍解,拜勤尚書僕射。職事十五年,以勤勞賜爵關內侯。遷尚書令,拜大司農,三歲遷司徒。

先是三公多見罪退,帝賢勤,欲令以善自終,乃因讌見從容戒之曰:「朱浮上不忠於君,下陵轢同列,竟以中傷至今,死生吉凶未可知,豈不惜哉!人臣放逐受誅,雖復追加賞賜賻祭,不足以償不訾之身。忠臣孝子,覽照前世,以為鏡誡。能盡忠於國,事君無二,則爵賞光乎當世,功名列於不朽,可不勉哉!」勤愈恭約盡忠,號稱任職。

勤母年八十,每會見,詔敕勿拜,令御者扶上殿,顧謂諸王主曰:「使勤貴寵者,此母也。」其見親重如此。

中元元年,薨,帝悼惜之,使者弔祠,賜東園祕器,賵贈有加。

勤七子。長子宗嗣,至張掖屬國都尉。中子順,尚平陽長公主,終於大鴻臚。建初八年,以順中子奮襲主爵為平陽侯,薨,無子。永元七年,詔書復封奮兄羽林右監勁為平陽侯,奉公主之祀。奮弟由,黃門侍郎,尚平安公主。勁薨,子卯嗣。卯延光中為侍中,薨,子留嗣。

趙憙字伯陽,南陽宛人也。少有節操。從兄為人所殺,無子,憙年十五,常思報之。乃挾兵結客,後遂往復仇。而仇家皆疾病,無相距者。憙以因疾報殺,非仁者心,且釋之而去。顧謂仇曰:「爾曹若健,遠相避也。」仇皆臥自搏。後病愈,悉自縛詣憙,憙不與相見,後竟殺之。

更始即位,舞陰大姓李氏擁城不下,更始遣柱天將軍李寶降之,不肯,云「聞宛之趙氏有孤孫憙,信義著名,願得降之」。更始乃徵憙。憙年未二十,既引見,更始笑曰:「繭栗犢,豈能負重致遠乎?」即除為郎中,行偏將軍事,使詣舞陰,而李氏遂降。憙因進入潁川,擊諸不下者,歷汝南界,還宛。更始大悅,謂憙曰:「卿名家駒,努力勉之。」會王莽遣王尋、王邑將兵出關,更始乃拜憙為五威偏將軍,使助諸將拒尋、邑於昆陽。光武破尋、邑,憙被創,有戰勞,還拜中郎將,封勇功侯。

更始敗,憙為赤眉兵所圍,迫急,乃踰屋亡走,與所友善韓仲伯等數十人,攜小弱,越山阻,徑出武關。仲伯以婦色美,慮有彊暴者,而己受其害,欲棄之於道。憙責怒不聽,因以泥塗仲伯婦面,載以鹿車,身自推之。每道逢賊,或欲逼略,憙輒言其病狀,以此得免。既入丹水,遇更始親屬,皆祼跣塗炭,飢困不能前。憙見之悲感,所裝縑帛資糧,悉以與之,將護歸鄉里。

時鄧奉反於南陽,憙素與奉善,數遺書切責之,而讒者因言憙與奉合謀,帝以為疑。及奉敗,帝得憙書,乃驚曰:「趙憙真長者也。」即徵憙,引見,賜卝馬,待詔公車。時江南未賓,道路不通,以憙守簡陽侯相。憙不肯受兵,單車馳之簡陽。吏民不欲內憙,憙乃告譬,呼城中大人,示以國家威信,其帥即開門面縛自歸,由是諸營壁悉降。荊州牧奏憙才任理劇,詔以為平林侯相。攻擊群賊,安集已降者,縣邑平定。

後拜懷令。大姓李子春先為琅邪相,豪猾并兼,為人所患,憙下車,聞其二孫殺人事未發覺,即窮詰其姦,收考子春,二孫自殺。京師為請者數十,終不聽。時趙王良疾病將終,車駕親臨王,問所欲言。王曰:「素與李子春厚,今犯罪,懷令趙憙欲殺之,願乞其命。」帝曰:「吏奉法,律不可枉也,更道它所欲。」王無復言。既薨,帝追感趙王,乃貰出子春。

其年,遷憙平原太守。時平原多盜賊,憙與諸郡討捕,斬其渠帥,餘黨當坐者數千人。憙上言「惡惡止其身,可一切徙京師近郡」。帝從之,乃悉移置潁川、陳留。於是擢舉義行,誅鋤姦惡。後青州大蝗,侵入平原界輒死,歲屢有年,百姓歌之。

二十六年,帝延集內戚讌會,歡甚,諸夫人各各前言「趙憙篤義多恩,往遭赤眉出長安,皆為憙所濟活」。帝甚嘉之。後徵憙入為太僕,引見謂曰:「卿非但為英雄所保也,婦人亦懷卿之恩。」厚加賞賜。

二十七年,拜太尉,賜爵關內侯。時南單于稱臣,烏桓、鮮卑並來入朝,帝令憙典邊事,思為久長規。憙上復緣邊諸郡,幽并二州由是而定。

三十年,憙上言宜封禪,正三雍之禮。中元元年,從封泰山。及帝崩,憙受遺詔,典喪禮。是時藩王皆在京師,自王莽篡亂,舊典不存,皇太子與東海王等雜止同席,憲章無序。憙乃正色,橫劍殿階,扶下諸王,以明尊卑。時藩國官屬出入宮省,與百僚無別,憙乃表奏謁者將護,分止它縣,諸王並令就邸,唯朝晡入臨。整禮儀,嚴門衛,內外肅然。

永平元年,封節鄉侯。三年春,坐考中山相薛脩事不實免。其冬,代竇融為衛尉。八年,代虞延行太尉事,居府如真。後遭母憂,上疏乞身行喪禮,顯宗不許,遣使者為釋服,賞賜恩寵甚渥。憙內典宿衛,外幹宰職,正身立朝,未嘗懈惰。及帝崩,復典喪事,再奉大行,禮事脩舉。肅宗即位,進為太傅,錄尚書事。擢諸子為郎吏者七人。長子代,給事黃門。

建初五年,憙疾病,帝親幸視。及薨,車駕往臨弔。時年八十四。謚曰正侯。

子代嗣,官至越騎校尉。永元中,副行征西將軍劉尚征羌,坐事下獄,疾病物故。和帝憐之,賜祕器錢布。贈越騎校尉、節鄉侯印綬。子直嗣,官至步兵校尉。直卒,子淑嗣,無子,國除。

牟融字子優,北海安丘人也。少博學,以大夏侯尚書教授,門徒數百人,名稱州里。以司徒茂才為豐令,視事三年,縣無獄訟,為州郡最。

司徒范遷薦融忠正公方,經行純備,宜在本朝,并上其理狀。永平五年,入代鮑昱為司隸校尉,多所舉正,百僚敬憚之。八年,代包咸為大鴻臚。十一年,代鮭陽鴻為大司農。

是時顯宗方勤萬機,公卿數朝會,每輒延謀政事,判折獄訟。融經明才高,善論議,朝廷皆服其能;帝數嗟歎,以為才堪宰相。明年,代伏恭為司空,舉動方重,甚得大臣節。肅宗即位,以融先朝名臣,代趙憙為太尉,與憙參錄尚書事。

建初四年薨,車駕親臨其喪。時融長子麟歸鄉里,帝以其餘子幼弱,敕太尉掾史教其威儀進止,贈賵恩寵篤密焉。又賜冢塋地於顯節陵下,除麟為郎。

韋彪字孟達,扶風平陵人也。高祖賢,宣帝時為丞相。祖賞,哀帝時為大司馬。

彪孝行純至,父母卒,哀毀三年,不出廬寑。服竟,羸瘠骨立異形,醫療數年乃起。好學洽聞,雅稱儒宗。建武末,舉孝廉,除郎中,以病免,復歸教授。安貧樂道,恬於進趣,三輔諸儒莫不慕仰之。

顯宗聞彪名,永平六年,召拜謁者,賜以車馬衣服,三遷魏郡太守。肅宗即位,以病免。徵為左中郎將、長樂衛尉,數陳政術,每歸寬厚。比上疏乞骸骨,拜為奉車都尉,秩中二千石,賞賜恩寵,侔於親戚。

建初七年,車駕西巡狩,以彪行太常從,數召入,問以三輔舊事,禮儀風俗。彪因建言:「今西巡舊都,宜追錄高祖、中宗功臣,褒顯先勳,紀其子孫。」帝納之。行至長安,乃制詔京兆尹、右扶風求蕭何、霍光後。時光無苗裔,唯封何末孫熊為酇侯。建初二年已封曹參後曹湛為平陽侯,故不復及焉。乃厚賜彪錢珍羞食物,使歸平陵上冢。還,拜大鴻臚。

是時陳事者,多言郡國貢舉率非功次,故守職益懈而吏事寖疏,咎在州郡。有詔下公卿朝臣議。彪上議曰:「伏惟明詔,憂勞百姓,垂恩選舉,務得其人。夫國以簡賢為務,賢以孝行為首。孔子曰:『事親孝故忠可移於君,是以求忠臣必於孝子之門。』夫人才行少能相兼,是以孟公綽優於趙、魏老,不可以為滕、薛大夫。忠孝之人,持心近厚;鍛鍊之吏,持心近薄。三代之所以直道而行者,在其所以磨之故也。士宜以才行為先,不可純以閥閱。然其要歸,在於選二千石。二千石賢,則貢舉皆得其人矣。」帝深納之。

彪以世承二帝吏化之後,多以苛刻為能,又置官選職,不必以才,因盛夏多寒,上疏諫曰:「臣聞政化之本,必順陰陽。伏見立夏以來,當暑而寒,殆以刑罰刻急,郡國不奉時令之所致也。農人急於務而苛吏奪其時,賦發充常調而貪吏割其財,此其巨患也。夫欲急人所務,當先除其所患。天下樞要,在於尚書,尚書之選,豈可不重?而閒者多從郎官超升此位,雖曉習文法,長於應對,然察察小慧,類無大能。宜簡嘗歷州宰素有名者,唯進退舒遲,時有不逮,然端心向公,奉職周密。宜鑒嗇夫捷急之對,深思絳侯木訥之功也。往時楚獄大起,故置令史以助郎職,而類多小人,好為姦利。今者務簡,可皆停省。又諫議之職,應用公直之士,通才謇正,有補益於朝者。今或從徵試輩為大夫。又御史外遷,動據州郡。並宜清選其任,責以言績。其二千石視事雖久,而為吏民所便安者,宜增秩重賞,勿妄遷徙。惟留聖心。」書奏,帝納之。

元和二年春,東巡狩,以彪行司徒事從行。還,以病乞身,帝遣小黃門、太醫問病,賜以食物。彪遂稱困篤。章和二年夏,使謁者策詔曰:「彪以將相之裔,勤身飭行,出自州里,在位歷載。中被篤疾,連上求退。君年在耆艾,不可復以加增,恐職事煩碎,重有損焉。其上大鴻臚印綬。其遣太子舍人詣中臧府,受賜錢二十萬。」永元元年,卒,詔尚書:「故大鴻臚韋彪,在位無愆,方欲錄用,奄忽而卒。其賜錢二十萬,布百匹,穀三千斛。」

彪清儉好施,祿賜分與宗族,家無餘財。著書十二篇,號曰韋卿子。

族子義。義字季節。高祖父玄成,元帝時為丞相。初,彪獨徙扶風,故義猶為京兆杜陵人焉。

兄順,字叔文,平輿令。有高名。次兄豹,字季明。數辟公府,輒以事去。司徒劉愷復辟之,謂曰:「卿以輕好去就,爵位不躋。今歲垂盡,當選御史,意在相薦,子其宿留乎?」豹曰:「犬馬齒衰,旅力已劣,仰慕崇恩,故未能自割。且眩瞀滯疾,不堪久待,選薦之私,非所敢當。」遂跣而起。愷追之,徑去不顧。安帝西巡,徵拜議郎。

義少與二兄齊名,初仕州郡。太傅桓焉辟舉理劇,為廣都長,甘陵、陳二縣令,政甚有績,官曹無事,牢獄空虛。數上書順帝,陳宜依古典,考功黜陟,徵集名儒,大定其制。又譏切左右,貶刺竇氏。言既無感,而久抑不遷,以兄順喪去官。比辟公府,不就。廣都為生立廟。及卒,三縣吏民為義舉哀,若喪考妣。

豹子著,字休明。少以經行知名,不應州郡之命。大將軍梁冀辟,不就。延熹二年,桓帝公車備禮徵,至霸陵,稱病歸,乃入雲陽山,采藥不反。有司舉奏加罪,帝特原之。復詔京兆尹重以禮敦勸,著遂不就徵。靈帝即位,中常侍曹節以陳蕃、竇氏既誅,海內多怨,欲借寵時賢以為名,白帝就家拜著東海相。詔書逼切,不得已,解巾之郡。政任威刑,為受罰者所奏,坐論輸左校。又後妻憍恣亂政,以之失名,竟歸,為姦人所害,隱者恥之。

贊曰:湛、霸奮庸,維寧兩邦。淮人孺慕,徐寇要降。弘實體遠,仁不忘本。憙政多跡,彪明理損。牟公簡帝,身終上袞。


\end{pinyinscope}