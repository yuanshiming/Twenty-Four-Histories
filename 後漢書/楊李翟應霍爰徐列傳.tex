\article{楊李翟應霍爰徐列傳}

\begin{pinyinscope}
楊終字子山,蜀郡成都人也。年十三,為郡小吏,太守奇其才,遣詣京師受業,習春秋。顯宗時,徵詣蘭臺,拜校書郎。

建初元年,大旱穀貴,終以為廣陵、楚、淮陽、濟南之獄,徙者萬數,又遠屯絕域,吏民怨曠,乃上疏曰:「臣聞『善善及子孫,惡惡止其身』,百王常典,不易之道也。秦政酷烈,違啎天心,一人有罪,延及三族。高祖平亂,約法三章。太宗至仁,除去收孥。萬姓廓然,蒙被更生,澤及昆蟲,功垂萬世。陛下聖祖,德被四表。今以比年久旱,災疫未息,躬自菲薄,廣訪失得,三代之隆,無以加焉。臣竊桉春秋水旱之變,皆應暴急,惠不下流。自永平以來,仍連大獄,有司窮考,轉相牽引,掠考冤濫,家屬徙邊。加以北征匈奴,西開三十六國,頻年服役,轉輸煩費。又遠屯伊吾、樓蘭、車師、戊己,民懷土思,怨結邊域。傳曰:『安土重居,謂之眾庶。』昔殷民近遷洛邑,且猶怨望,何況去中土之肥饒,寄不毛之荒極乎?且南方暑濕,障毒互生。愁困之民,足以感動天地,移變陰陽矣。陛下留念省察,以濟元元。」書奏,肅宗下其章。司空第五倫亦同終議。太尉牟融、司徒鮑昱、校書郎班固等難倫,以施行既久,孝子無改父之道,先帝所建,不宜回異。終復上書曰:「秦築長城,功役繁興,胡亥不革,卒亡四海。故孝元棄珠崖之郡,光武絕西域之國,不以介鱗易我衣裳。魯文公毀泉臺,春秋譏之曰『先祖為之而己毀之,不如勿居而已』,以其無妨害於民也。襄公作三軍,昭公舍之,君子大其復古,以為不舍則有害於民也。今伊吾之役,樓蘭之屯,久而未還,非天意也。」帝從之,聽還徙者,悉罷邊屯。

終又言:「宣帝博徵群儒,論定五經於石渠閣。方今天下少事,學者得成其業,而章句之徒,破壞大體。宜如石渠故事,永為後世則。」於是詔諸儒於白虎觀論考同異焉。會終坐事繫獄,博士趙博、校書郎班固、賈逵等,以終深曉春秋,學多異聞,表請之,終又上書自訟,即日貰出,乃得與於白虎觀焉。後受詔刪太史公書為十餘萬言。

時太后兄衛尉馬廖,謹篤自守,不訓諸子。終與廖交善,以書戒之曰:「終聞堯舜之民,可比屋而封;桀紂之民,可比屋而誅。何者?堯舜為之隄防,桀紂示之驕奢故也。《詩》曰:『皎皎練絲,在所染之。』上智下愚,謂之不移;中庸之流,要在教化。春秋殺太子母弟,直稱君甚惡之者,坐失教也。禮制,人君之子年八歲,為置少傅,教之書計,以開其明;十五置太傅,教之經典,以道其志。漢興,諸侯王不力教誨,多觸禁忌,故有亡國之禍,而乏嘉善之稱。今君位地尊重,海內所望,豈可不臨深履薄,以為至戒!黃門郎年幼,血氣方盛,既無長君退讓之風,而要結輕狡無行之客,縱而莫誨,視成任性,鑒念前往,可為寒心。君侯誠宜以臨深履薄為戒。」廖不納。子豫後坐縣書誹謗,廖以就國。

終兄鳳為郡吏,太守廉范為州所考,遣鳳候終,終為范游說,坐徙北地。帝東巡狩,鳳皇黃龍並集,終贊頌嘉瑞,上述祖宗鴻業,凡十五章,奏上,詔貰還故郡。著春秋外傳十二篇,改定章句十五萬言。永元十二年,徵拜郎中,以病卒。

李法字伯度,漢中南鄭人也。博通群書,性剛而有節。和帝永元九年,應賢良方正對策,除博士,遷侍中、光祿大夫。歲餘,上疏以為朝政苛碎,違永平、建初故事;宦官權重,椒房寵盛;又譏史官記事不實,後世有識,尋功計德,必不明信。坐失旨,下有司,免為庶人。還鄉里,杜門自守。故人儒生時有候之者,言談之次,問其不合上意之由,法未嘗應對。友人固問之,法曰:「鄙夫可與事君乎哉?苟患失之,無所不至。孟子有言:『夫仁者如射,正己而後發。發而不中,不怨勝己者,反諸身而已矣。』」在家八年,徵拜議郎、諫議大夫,正言極辭,無改於舊。出為汝南太守,政有聲跡。後歸鄉里,卒於家。

翟酺字子超,廣漢雒人也。四世傳詩。酺好老子,尤善圖緯、天文、歷筭。以報舅讎,當徙日南,亡於長安,為卜相工,後牧羊涼州。遇赦還。仕郡,徵拜議郎,遷侍中。

時尚書有缺,詔將大夫六百石以上試對政事、天文、道術,以高第者補之。酺自恃能高,而忌故太史令孫懿,恐其先用,乃往候懿。既坐,言無所及,唯涕泣流連。懿怪而問之,酺曰:「圖書有漢賊孫登,將以才智為中官所害。觀君表相,似當應之。酺受恩接,悽愴君之禍耳!」懿憂懼,移病不試。由是酺對第一,拜尚書。

時安帝始親政事,追感祖母宋貴人,悉封其家。又元舅耿寶及皇后兄弟閻顯等並用威權。酺上疏諫曰:

臣聞微子佯狂而去殷,叔孫通背秦而歸漢,彼非自疏其君,時不可也。臣荷殊絕之恩,蒙值不諱之政,豈敢雷同受寵,而以戴天履地。伏惟陛下應天履祚,歷值中興,當建太平之功,而未聞致化之道。蓋遠者難明,請以近事徵之。昔竇、鄧之寵,傾動四方,兼官重紱,盈金積貨,至使議弄神器,改更社稷。豈不以埶尊威廣,以致斯患乎?及其破壞,頭顙墯地,願為孤豚,豈可得哉!夫致貴無漸失必暴,受爵非道殃必疾。今外戚寵幸,功均造化,漢元以來,未有等比。陛下誠仁恩周洽,以親九族。然祿去公室,政移私門,覆車重尋,寧無摧折。而朝臣在位,莫肯正議,翕翕訾訾,更相佐附。臣恐威權外假,歸之良難,虎翼一奮,卒不可制。故孔子曰「吐珠於澤,誰能不含」;老子稱「國之利器,不可以示人。」此最安危戒,社稷之深計也。

夫儉德之恭,政存約節。故文帝愛百金於露臺,飾帷帳於皁囊。或有譏其儉者,上曰:「朕為天下守財耳,豈得妄用之哉!」至倉穀腐而不可食,錢貫朽而不可校。今自初政已來,日月未久,費用賞賜已不可筭。斂天下之財,積無功之家,帑藏單盡,民物彫傷,卒有不虞,復當重賦百姓,怨叛既生,危亂可待也。

昔成王之政,周公在前,邵公在後,畢公在左,史佚在右,四子挾而維之。目見正容,耳聞正言,一日即位,天下曠然,言其法度素定也。今陛下有成王之尊而無數子之佐,雖欲崇雍熙,致太平,其可得乎?

自去年已來,災譴頻數,地坼天崩,高岸為谷。脩身恐懼,則轉禍為福;輕慢天戒,則其害彌深。願陛下親自勞恤,研精致思,勉求忠貞之臣,誅遠佞諂之黨,損玉堂之盛,尊天爵之重,割情欲之歡,罷宴私之好。帝王圖籍,陳列左右,心存亡國所以失之,鑒觀興王所以得之,庶災害可息,豐年可招矣。

書奏不省,而外戚寵臣咸畏惡之。

延光三年,出為酒泉太守。叛羌千餘騎徙敦煌來鈔郡界,酺赴擊,斬首九百級,羌眾幾盡,威名大震。遷京兆尹。順帝即位,拜光祿大夫,遷將作大匠。損省經用,歲息四五千萬。屢因災異,多所匡正。由是權貴共誣酺及尚書令高堂芝等交通屬託,坐減死歸家。復被章云酺前與河南張楷等謀反,逮詣廷尉。及杜真等上書訟之,事得明釋。卒於家。

著援神、鉤命解詁十二篇。

初,酺之為大匠,上言:「孝文皇帝始置一經博士,武帝大合天下之書,而孝宣論六經於石渠,學者滋盛,弟子萬數。光武初興,愍其荒廢,起太學博士舍、內外講堂,諸生橫巷,為海內所集。明帝時辟雍始成,欲毀太學,太尉趙憙以為太學、辟雍皆宜兼存,故並傳至今。而頃者穨廢,至為園採芻牧之處。宜更修繕,誘進後學。」帝從之。酺免後,遂起太學,更開拓房室,學者為酺立碑銘於學云。

應奉字世叔,汝南南頓人也。曾祖父順,字華仲。和帝時為河南尹、將作大匠,公廉約己,明達政事。生十子,皆有才學。中子疊,江夏太守。疊生郴,武陵太守。郴生奉。

奉少聰明,自為童兒及長,凡所經履,莫不暗記。讀書五行並下。為郡決曹史,行部四十二縣,錄囚徒數百千人。及還,太守備問之,奉口說罪繫姓名,坐狀輕重,無所遺脫,時人奇之。著漢書後序,多所述載。大將軍梁冀舉茂才。

先是,武陵蠻詹山等四千餘人反叛,執縣令,屯結連年。詔下公卿議,四府舉奉才堪將帥。永興元年,拜武陵太守。到官慰納,山等皆悉降散。於是興學校,舉仄陋,政稱變俗。坐公事免。

延熹中,武陵蠻復寇亂荊州,車騎將軍馮緄以奉有威恩,為蠻夷所服,上請與俱征。拜從事中郎。奉勤設方略,賊破軍罷,緄推功於奉,薦為司隸校尉。糾舉姦違,不避豪戚,以嚴厲為名。

及鄧皇后敗,而田貴人見幸,桓帝有建立之議。奉以田氏微賤,不宜超登后位,上書諫曰:「臣聞周納狄女,襄王出居于鄭;漢立飛燕,成帝胤嗣泯絕。母后之重,興廢所因。宜思關睢之所求,遠五禁之所忌。」帝納其言,竟立竇皇后。

及黨事起,奉乃慨然以疾自退。追愍屈原,因以自傷,著感騷三十篇,數萬言。諸公多薦舉,會病卒。子劭。

劭字仲遠。少篤學,博覽多聞。靈帝時舉孝廉,辟車騎將軍何苗掾。

中平二年,漢陽賊邊章、韓遂與羌胡為寇,東侵三輔,時遣車騎將軍皇甫嵩西討之。嵩請發烏桓三千人。北軍中候鄒靖上言:「烏桓眾弱,宜開募鮮卑。」事下四府,大將軍掾韓卓議,以為「烏桓兵寡,而與鮮卑世為仇敵,若烏桓被發,則鮮卑必襲其家。烏桓聞之,當復棄軍還救。非唯無益於實,乃更沮三軍之情。鄒靖居近邊塞,究其態詐。若令靖募鮮卑輕騎五千,必有破敵之效」。劭駮之曰:「鮮卑隔在漠北,犬羊為群,無君長之帥,廬落之居,而天性貪暴,不拘信義,故數犯障塞,且無寧歲。唯至互巿,乃來靡服。苟欲中國珍貨,非為畏威懷德。計獲事足,旋踵為害。是以朝家外而不內,蓋為此也。往者匈奴反叛,度遼將軍馬續、烏桓校尉王元發鮮卑五千餘騎,又武威太守趙沖亦率鮮卑征討叛羌。斬獲醜虜,既不足言,而鮮卑越溢,多為不法。裁以軍令,則忿戾作亂;制御小緩,則陸掠殘害。劫居人,鈔商旅,噉人牛羊,略人兵馬。得賞既多,不肯去,復欲以物買鐵。邊將不聽,便取縑帛聚欲燒之。邊將恐怖,畏其反叛,辭謝撫順,無敢拒違。今狡寇未殄,而羌為巨害,如或致悔,其可追乎!臣愚以為可募隴西羌胡守善不叛者,簡其精勇,多其牢賞。太守李參沈靜有謀,必能獎厲得其死力。當思漸消之略,不可倉卒望也。」韓卓復與劭相難反覆。於是詔百官大會朝堂,皆從劭議。

三年,舉高第,再遷,六年,拜太山太守。初平二年,黃巾三十萬眾入郡界。劭糾率文武連與賊戰,前後斬首數千級,獲生口老弱萬餘人,輜重二千兩,賊皆退卻,郡內以安。興平元年,前太尉曹嵩及子德從琅邪入太山,劭遣兵迎之,未到,而徐州牧陶謙素怨嵩子操數擊之,乃使輕騎追嵩、德,並殺之於郡界。劭畏操誅,棄郡奔冀州牧袁紹。

初,安帝時河閒人尹次、潁川人史玉皆坐殺人當死,次兄初及玉母軍並詣官曹求代其命,因縊而物故。尚書陳忠以罪疑從輕,議活次、玉。劭後追駮之,據正典刑,有可存者。其議曰:

《尚書》稱「天秩有禮,五服五章哉。天討有罪,五刑五用哉」。而孫卿亦云「凡制刑之本,將以禁暴惡,且懲其末也。凡爵列、官秩、賞慶、刑威,皆以類相從,使當其實也」。若德不副位,能不稱官,賞不酬功,刑不應罪,不祥莫大焉。殺人者死,傷人者刑,此百王之定制,有法之成科。高祖入關,雖尚約法,然殺人者死,亦無寬降。夫時化則刑重,時亂則刑輕。書曰「刑罰時輕時重」,此之謂也。

今次、玉公以清時釋其私憾,阻兵安忍,僵屍道路。朝恩在寬,幸至冬獄,而初、軍愚狷,妄自投斃。昔召忽親死子糾之難,而孔子曰「經於溝瀆,人莫之知」。朝氏之父非錯刻峻,遂能自隕其命,班固亦云「不如趙母指括以全其宗」。傳曰「僕妾感慨而致死者,非能義勇,顧無慮耳」。夫刑罰威獄,以類天之震燿殺戮也;溫慈和惠,以放天之生殖長育也。是故春一草枯則為災,秋一木華亦為異。今殺無罪之初、軍,而活當死之次、玉,其為枯華,不亦然乎?陳忠不詳制刑之本,而信一時之仁,遂廣引八議求生之端。夫親故賢能功貴勤賓,豈有次、玉當罪之科哉?若乃小大以情,原心定罪,此為求生,非謂代死可以生也。敗法亂政,悔其可追。劭凡為駮議三十篇,皆此類也。

又刪定律令為漢儀,建安元年乃奏之。曰:「夫國之大事,莫尚載籍。載籍也者,決嫌疑,明是非,賞刑之宜,允獲厥中,俾後之人永為監焉。故膠東相董仲舒老病致仕,朝廷每有政議,數遣廷尉張湯親至陋巷,問其得失。於是作春秋決獄二百三十二事,動以經對,言之詳矣。逆臣董卓,蕩覆王室,典憲焚燎,靡有孑遺,開辟以來,莫或茲酷。今大駕東邁,巡省許都,拔出險難,其命惟新。臣累世受恩,榮祚豐衍,竊不自揆,貪少云補,輒撰具律本章句、尚書舊事、廷尉板令、決事比例、司徒都目、五曹詔書及春秋斷獄凡二百五十篇。蠲去復重,為之節文。又集駮議三十篇,以類相從,凡八十二事。其見漢書二十五,漢記四,皆刪敘潤色,以全本體。其二十六,博採古今瑰瑋之士,文章煥炳,德義可觀。其二十七,臣所創造。豈繄自謂必合道衷,心焉憤邑,聊以藉手。昔鄭人以乾鼠為璞,鬻之於周;宋愚夫亦寶燕石,緹俦十重。夫睹之者掩口盧胡而笑,斯文之族,無乃類旃。左氏實云雖有姬姜絲麻,不棄憔悴菅蒯,蓋所以代匱也。是用敢露頑才,廁于明哲之末。雖未足綱紀國體,宣洽時雍,庶幾觀察,增闡聖聽。惟因萬機之餘暇,游意省覽焉。」獻帝善之。

二年,詔拜劭為袁紹軍謀校尉。時始遷都於許,舊章堙沒,書記罕存。劭慨然歎息,乃綴集所聞,著漢官禮儀故事,凡朝廷制度,百官典式,多劭所立。

初,父奉為司隸時,並下諸官府郡國,各上前人像贊,劭乃連綴其名,錄為狀人紀。又論當時行事,著中漢輯序。撰風俗通,以辯物類名號,釋時俗嫌疑。文雖不典,後世服其洽聞。凡所著述百三十六篇。又集解漢書,皆傳于時。後卒於鄴。

弟子瑒、璩,並以文才稱。

中興初,有應嫗者,生四子而寡。見神光照社,試探之,乃得黃金。自是諸子宦學,並有才名,至瑒七世通顯。

霍諝字叔智,魏郡鄴人也。少為諸生,明經。有人誣諝舅宋光於大將軍梁商者,以為妄刊章文,坐繫洛陽詔獄,掠考困極。諝時年十五,奏記於商曰:

將軍天覆厚恩,愍舅光冤結,前者溫教許為平議,雖未下吏斷決其事,已蒙神明顧省之聽。皇天后土,寔聞德音。竊獨踴躍,私自慶幸。諝聞春秋之義,原情定過,赦事誅意,故許止雖弒君而不罪,趙盾以縱賊而見書。此仲尼所以垂王法,漢世所宜遵前脩也。傳曰:「人心不同,譬若其面。」斯蓋謂大小窳隆醜美之形,至於鼻目眾竅毛髮之狀,未有不然者也。情之異者,剛柔舒急倨敬之閒。至於趨利避害,畏死樂生,亦復均也。諝與光骨肉,義有相隱,言其冤濫,未必可諒,且以人情平論其理。

光衣冠子孫,徑路平易,位極州郡,日望徵辟,亦無瑕穢纖介之累,無故刊定詔書,欲以何名?就有所疑,當求其便安,豈有觸冒死禍,以解細微?譬猶療飢於附子,止渴於酖毒,未入腸胃,已絕咽喉,豈可為哉!昔東海孝婦見枉不辜,幽靈感革,天應枯旱。光之所坐,情既可原,守闕連年,而終不見理。呼嗟紫宮之門,泣血兩觀之下,傷和致災,為害滋甚。凡事更赦令,不應復案。夫以罪刑明白,尚蒙天恩,豈有冤謗無徵,反不得理?是為刑宥正罪,戮加誣侵也。不偏不黨,其若是乎?明將軍德盛位尊,人臣無二,言行動天地,舉厝移陰陽,誠能留神,沛然曉察,必有于公高門之福,和氣立應,天下幸甚。

商高諝才志,即為奏原光罪,由是顯名。

仕郡,舉孝廉,稍遷金城太守。性明達篤厚,能以恩信化誘殊俗,甚為羌胡所敬服。遭母憂,自上歸行喪。服闋,公車徵,再遷北海相,入為尚書僕射。是時大將軍梁冀貴戚秉權,自公卿以下莫敢違啎。諝與尚書令尹勳數奏其事,又因陛見陳聞罪失。及冀誅後,桓帝嘉其忠節,封鄴都亭侯。前後固讓,不許。出為河南尹,遷司隸校尉,轉少府、廷尉,卒官。

子雋,安定太守。

爰延字季平,陳留外黃人也。清苦好學,能通經教授。性質愨,少言辭。縣令隴西牛述好士知人,乃禮請延為廷掾,范丹為功曹,濮陽潛為主簿,常共言談而已。後令史昭以為鄉嗇夫,仁化大行,人但聞嗇夫,不知郡縣。在事二年,州府禮請,不就。桓帝時徵博士,太尉楊秉等舉賢良方正,再遷為侍中。

帝游上林苑,從容問延曰:「朕何如主也?」對曰:「陛下為漢中主。」帝曰:「何以言之?」對曰:「尚書令陳蕃任事則化,中常侍黃門豫政則亂,是以知陛下可與為善,可與為非。」帝曰:「昔朱雲廷折欄檻,今侍中面稱朕違,敬聞闕矣。」拜五官中郎將,轉長水校尉,遷魏郡太守,徵拜大鴻臚。

帝以延儒生,常特宴見。時太史令上言客星經帝坐,帝密以問延。延因上封事曰:「臣聞天子尊無為上,故天以為子,位臨臣庶,威重四海。動靜以禮,則星辰順序;意有邪僻,則晷度錯違。陛下以河南尹鄧萬有龍潛之舊,封為通侯,恩重公卿,惠豐宗室。加頃引見,與之對博,上下媟黷,有虧尊嚴。臣聞之,帝左右者,所以咨政德也。故周公戒成王曰『其朋其朋』,言慎所與也。昔宋閔公與彊臣共博,列婦人於側,積此無禮,以致大災。武帝與倖臣李延年、韓嫣同臥起,尊爵重賜,情欲無猒,遂生驕淫之心,行不義之事,卒延年被戮,嫣伏其辜。夫愛之則不覺其過,惡之則不知其善,所以事多放濫,物情生怨。故王者賞人必酬其功,爵人必甄其德。善人同處,則日聞嘉訓;惡人從游,則日生邪情。孔子曰:『益者三友,損者三友。』邪臣惑君,亂妾危主,以非所言則悅於耳,以非所行則翫於目,故令人君不能遠之。仲尼曰:『唯女子與小人為難養,近之則不遜,遠之則怨。』蓋聖人之明戒也!昔光武皇帝與嚴光俱寢,上天之異,其夕即見。夫以光武之聖德,嚴光之高賢,君臣合道,尚降此變,豈況陛下今所親幸,以賤為貴,以卑為尊哉?惟陛下遠讒諛之人,納謇謇之士,除左右之權,寤宦官之敝。使積善日熙,佞惡消殄,則乾災可除。」帝省其奏。因以病自上,乞骸骨還家。靈帝復特徵,不行,病卒。

子驥,白馬令,亦稱善士。

徐璆字孟玉,廣陵海西人也。父淑,度遼將軍,有名於邊。璆少博學,辟公府,舉高第。稍遷荊州刺史。時董太后姊子張忠為南陽太守,因埶放濫,臧罪數億。璆臨當之部,太后遣中常侍以忠屬璆。璆對曰:「臣身為國,不敢聞命。」太后怒,遽徵忠為司隸校尉,以相威臨。璆到州,舉奏忠臧餘一億,使冠軍縣上簿詣大司農,以彰暴其事。又奏五郡太守及屬縣有臧汙者,悉徵案罪,威風大行。中平元年,與中郎將朱雋擊黃巾賊於宛,破之。張忠怨璆,與諸閹官構造無端,璆遂以罪徵。有破賊功,得免官歸家。後再徵,遷汝南太守,轉東海相,所在化行。

獻帝遷許,以廷尉徵,當詣京師,道為袁術所劫,授璆以上公之位。璆乃歎曰:「龔勝、鮑宣,獨何人哉?守之必死!」術不敢逼。術死軍破,璆得其盜國璽,及還許,上之,并送前所假汝南、東海二郡印綬。司徒趙溫謂璆曰:「君遭大難,猶存此邪?」璆曰:「昔蘇武困於匈奴,不隊七尺之節,況此方寸印乎?」

後拜太常,使持節拜曹操為丞相。操以相讓璆,璆不敢當。卒於官。

論曰:孫懿以高明見忌,而受欺於陰計;翟酺資譎數取通,而終之以謇諫。豈性智自有周偏,先後之要殊度乎?應氏七世才聞,而奉、劭采章為盛。及撰著篇籍,甄紀異知,雖云小道,亦有可觀者焉。延、璆應對辯正,而不可犯陵上之尤,斯固辭之不可以已也。

贊曰:楊終、李法,華陽有聞。二應克聰,亦表汝濆。翟酺詐懿,霍諝請舅。延能訐帝,璆亦啎后。


\end{pinyinscope}