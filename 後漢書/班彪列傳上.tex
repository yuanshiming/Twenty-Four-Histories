\article{班彪列傳上}

\begin{pinyinscope}
班彪字叔皮,扶風安陵人也。祖況,成帝時為越騎校尉。父稚,哀帝時為廣平太守。

彪性沈重好古。年二十餘,更始敗,三輔大亂。時隗囂擁眾天水,彪乃避難從之。囂問彪曰:「往者周亡,戰國並爭,天下分裂,數世然後定。意者從橫之事復起於今乎?將承運迭興,在於一人也?願生試論之。」對曰:「周之廢興,與漢殊異。昔周爵五等,諸侯從政,本根既微,枝葉彊大,故其末流有從橫之事,埶數然也。漢承秦制,改立郡縣,主有專己之威,臣無百年之柄。至於成帝,假借外家,哀、平短祚,國嗣三絕,故王氏擅朝,因竊號位。危自上起,傷不及下,是以即真之後,天下莫不引領而歎。十餘年閒,中外搔擾,遠近俱發,假號雲合,咸稱劉氏,不謀同辭。方今雄桀帶州域者,皆無七國世業之資,而百姓謳吟,思仰漢德,已可知矣。」囂曰:「生言周、漢之埶可也;至於但見愚人習識劉氏姓號之故,而謂漢家復興,疏矣。昔秦失其鹿,劉季逐而羈之,時人復知漢乎?」

彪既疾囂言,又傷時方艱,乃著王命論,以為漢德承堯,有靈命之符,王者興祚,非詐力所致,欲以感之,而囂終不寤,遂避河西。河西大將軍竇融以為從事,深敬待之,接以師友之道。彪乃為融畫策事漢,總西河以拒隗囂。

及融徵還京師,光武問曰:「所上章奏,誰與參之?」融對曰:「皆從事班彪所為。」帝雅聞彪才,因召入見,舉司隸茂才,拜徐令,以病免。後數應三公之命,輒去。

彪既才高而好述作,遂專心史籍之閒。武帝時,司馬遷著史記,自太初以後,闕而不錄,後好事者頗或綴集時事,然多鄙俗,不足以踵繼其書。彪乃繼採前史遺事,傍貫異聞,作後傳數十篇,因斟酌前史而譏正得失。其略論曰:

唐虞三代,詩書所及,世有史官,以司典籍,暨於諸侯,國自有史,故孟子曰「楚之檮杌,晉之乘,魯之春秋,其事一也」。定哀之閒,魯君子左丘明論集其文,作左氏傳三十篇,又撰異同,號曰國語,二十一篇,由是乘、檮杌之事遂闇,而左氏、國語獨章。又有記錄黃帝以來至春秋時帝王公侯卿大夫,號曰世本,一十五篇。春秋之後,七國並爭,秦并諸侯,則有戰國策三十三篇。漢興定天下,太中大夫陸賈記錄時功,作楚漢春秋九篇。孝武之世,太史令司馬遷採左氏、國語,刪世本、戰國策,據楚、漢列國時事,上自黃帝,下訖獲麟,作本紀、世家、列傳、書、表凡百三十篇,而十篇缺焉。遷之所記,從漢元至武以絕,則其功也。至於採經摭傳,分散百家之事,甚多疏略,不如其本,務欲以多聞廣載為功,論議淺而不篤。其論術學,則崇黃老而薄五經;序貨殖,則輕仁義而羞貧窮;道游俠,則賤守節而貴俗功:此其大敝傷道,所以遇極刑之咎也。然善述序事理,辯而不華,質而不野,文質相稱,蓋良史之才也。誠令遷依五經之法言,同聖人之是非,意亦庶幾矣。

夫百家之書,猶可法也。若左氏、國語、世本、戰國策、楚漢春秋、太史公書,今之所以知古,後之所由觀前,聖人之耳目也。司馬遷序帝王則曰本紀,公侯傳國則曰世家,卿士特起則曰列傳。又進項羽、陳涉而黜淮南、衡山,細意委曲,條列不經。若遷之著作,採獲古今,貫穿經傳,至廣博也。一人之精,文重思煩,故其書刊落不盡,尚有盈辭,多不齊一。若序司馬相如,舉郡縣,著其字,至蕭、曹、陳平之屬,及董仲舒並時之人,不記其字,或縣而不郡者,蓋不暇也。今此後篇,慎覈其事,整齊其文,不為世家,唯紀、傳而已。傳曰:「殺史見極,平易正直,春秋之義也。」

彪復辟司徒玉況府。時東宮初建,諸王國並開,而官屬未備,師保多闕。彪上言曰:

孔子稱「性相近,習相遠也」。賈誼以為「習與善人居,不能無為善,猶生長於齊,不能無齊言也。習與惡人居,不能無惡,猶生長於楚,不能無楚言也」。是以聖人審所與居,而戒慎所習。昔成王之為孺子,出則周公、邵公、太公史佚,入則大顛、閎夭、南宮括、散宜生,左右前後,禮無違者,故成王一日即位,天下曠然太平。是以春秋「愛子教以義方,不納於邪。驕奢淫佚,所自邪也」。《詩》云:「詒厥孫謀,以宴翼子。」言武王之謀遺子孫也。

漢興,太宗使晁錯導太子以法術,賈誼教梁王以詩書。及至中宗,亦令劉向、王褒、蕭望之、周堪之徒,以文章儒學保訓東宮以下,莫不崇簡其人,就成德器。今皇太子諸王,雖結髮學問,脩習禮樂,而傅相未值賢才,官屬多闕舊典。宜博選名儒有威重明通政事者,以為太子太傅,東宮及諸王國,備置官屬。又舊制,太子食湯沐十縣,設周衛交戟,五日一朝,因坐東箱,省視膳食,其非朝日,使僕、中允旦旦請問而已,明不媟黷,廣其敬也。

書奏,帝納之。

後察司徒廉為望都長,吏民愛之。建武三十年,年五十二,卒官。所著賦、論、書、記、奏事合九篇。

二子:固,超。超別有傳。

論曰:班彪以通儒上才,傾側危亂之閒,行不踰方,言不失正,仕不急進,貞不違人,敷文華以緯國典,守賤薄而無悶容。彼將以世運未弘,非所謂賤焉恥乎?何其守道恬淡之篤也!

固字孟堅。年九歲,能屬文誦詩賦,及長,遂博貫載籍,九流百家之言,無不窮究。所學無常師,不為章句,舉大義而已。性寬和容眾,不以才能高人,諸儒以此慕之。

永平初,東平王蒼以至戚為驃騎將軍輔政,開東閤,延英雄。時固始弱冠,奏記說蒼曰:

將軍以周、邵之德,立乎本朝,承休明之策,建威靈之號,昔在周公,今也將軍,詩書所載,未有三此者也。傳曰:「必有非常之人,然後有非常之事;有非常之事,然後有非常之功。」固幸得生於清明之世,豫在視聽之末,私以螻螘,竊觀國政,誠美將軍擁千載之任,躡先聖之蹤,體弘懿之姿,據高明之埶,博貫庶事,服膺六蓺,白黑簡心,求善無猒,採擇狂夫之言,不逆負薪之議。竊見幕府新開,廣延群俊,四方之士,顛倒衣裳。將軍宜詳唐、殷之舉,察伊、皋之薦,令遠近無偏,幽隱必達,期於總覽賢才,收集明智,為國得人,以寧本朝。則將軍養志和神,優游廟堂,光名宣於當世,遺烈著於無窮。

竊見故司空掾桓梁,宿儒盛名,冠德州里,七十從心,行不踰矩,蓋清廟之光暉,當世之俊彥也。京兆祭酒晉馮,結髮修身,白首無違,好古樂道,玄默自守,古人之美行,時俗所莫及。扶風掾李育,經明行著,教授百人,客居杜陵,茅室土階。京兆、扶風二郡更請,徒以家貧,數辭病去。溫故知新,論議通明,廉清修絜,行能純備,雖前世名儒,國家所器,韋、平、孔、翟,無以加焉。宜令考續,以參萬事。京兆督郵郭基,孝行著於州里,經學稱於師門,政務之績,有絕異之效。如得及明時,秉事下僚,進有羽翮奮翔之用,退有杞梁一介之死。涼州從事王雍,躬卞嚴之節,文之以術蓺,涼州冠蓋,未有宜先雍者也。古者周公一舉則三方怨,曰「奚為而後己」。宜及府開,以慰遠方。弘農功曹史殷肅,達學洽聞,才能絕倫,誦詩三百,奉使專對。此六子者,皆有殊行絕才,德隆當世,如蒙徵納,以輔高明,此山梁之秋,夫子所為歎也。昔卞和獻寶,以離斷趾,靈均納忠,終於沈身,而和氏之璧,千載垂光,屈子之篇,萬世歸善。願將軍隆照微之明,信日昃之聽,少屈威神,咨嗟下問,令塵埃之中,永無荊山、汨羅之恨。

蒼納之。

父彪卒,歸鄉里。固以彪所續前史未詳,乃潛精研思,欲就其業。既而有人上書顯宗,告固私改作國史者,有詔下郡,收固繫京兆獄,盡取其家書。先是扶風人蘇朗偽言圖讖事,下獄死。固弟超恐固為郡所覈考,不能自明,乃馳詣闕上書,得召見,具言固所著述意,而郡亦上其書。顯宗甚奇之,召詣校書部,除蘭臺令史,與前睢陽令陳宗、長陵令尹敏、司隸從事孟異共成世祖本紀。遷為郎,典校祕書。固又撰功臣、平林、新市、公孫述事,作列傳、載記二十八篇,奏之。帝乃復使終成前所著書。

固以為漢紹堯運,以建帝業,至於六世,史臣乃追述功德,私作本紀,編於百王之末,廁於秦、項之列,太初以後,闕而不錄,故探撰前記,綴集所聞,以為漢書。起元高祖,終于孝平王莽之誅,十有二世,二百三十年,綜其行事,傍貫五經,上下洽通,為春秋考紀、表、志、傳凡百篇。固自永平中始受詔,潛精積思二十餘年,至建初中乃成。當世甚重其書,學者莫不諷誦焉。

自為郎後,遂見親近。時京師脩起宮室,濬繕城隍,而關中耆老猶望朝廷西顧。固感前世相如、壽王、東方之徒,造搆文辭,終以諷勸,乃上兩都賦,盛稱洛邑制度之美,以折西賓淫侈之論。其辭曰:

有西都賓問於東都主人曰:「蓋聞皇

漢之初經營也,嘗有意乎都河洛矣。輟而弗康,寔用西遷,作我上都。主人聞其故而睹其制乎?」主人曰:「未也。願賓攄懷舊之蓄念,發思古之幽情,博我以皇道,弘我以漢京。」賓曰:「唯唯。」

漢之西都,在于雍州,寔曰長安。左據函谷、二崤之阻,表以泰華、終南之山。右界褒斜、隴首之險,帶以洪河、涇、渭之川。華實之毛,則九州之上腴焉;防禦之阻,則天下之奧區焉。是故橫被六合,三成帝畿,周以龍興,秦以虎視。及至大漢受命而都之也,仰寤東井之精,俯協河圖之靈,奉春建策,留侯演成,天人合應,以發皇明,乃眷西顧,寔惟作京。於是睎秦領,睋北阜,挾酆霸,據龍首。圖皇基於億載,度宏規而大起,肇自高而終平,世增飾以崇麗,歷十二之延祚,故窮奢而極侈。建金城其萬雉,呀周池而成淵,披三條之廣路,立十二之通門。內則街衢洞達,閭閻且千,九市開場,貨別隧分,人不得顧,車不得旋,闐城溢郭,傍流百廛,紅塵四合,煙雲相連。於是既庶且富,娛樂無疆,都人士女,殊異乎五方,游士擬於公侯,列肆侈於姬、姜。鄉曲豪俊游俠之雄,節慕原、嘗,名亞春、陵,連交合眾,騁騖乎其中。

若乃觀其四郊,浮遊近縣,則南望杜、霸,北眺五陵,名都對郭,邑居相承,英俊之域,黻冕所興,冠蓋如雲,七相五公。與乎州郡之豪桀,五都之貨殖,三選七遷,充奉陵邑,蓋以彊幹弱枝,隆上都而觀萬國。封畿之內,厥土千里,逴犖諸夏,兼其所有。其陽則崇山隱天,幽林穹谷,陸海珍藏,藍田美玉,商、洛緣其隈,鄠、杜濱其足,源泉灌注,陂池交屬,竹林果園,芳草甘木,郊野之富,號曰近蜀。其陰則冠以九嵕,陪以甘泉,乃有靈宮起乎其中。秦、漢之所極觀,淵、雲之所頌歎,於是乎存焉。下有鄭、白之沃,衣食之源,隄封五萬,疆埸綺分,溝塍刻鏤,原隰龍鱗,決渠降雨,荷臿成雲,五穀垂穎,桑麻敷棻。東郊則有通溝大漕,潰渭洞河,泛舟山東,控引淮、湖,與海通波。西郊則有上囿禁苑,林麓藪澤,陂池連乎蜀、漢,繚以周牆,四百餘里,離宮別館,三十六所,神池靈沼,往往而在。其中乃有九真之麟,大宛之馬,黃支之犀,條枝之鳥,踰崑崙,越巨海,殊方異類,至三萬里。

其宮室也,體象乎天地,經緯乎陰陽,據坤靈之正位,放泰、紫之圓方。樹中天之華闕,豐冠山之朱堂,因瑰材而究奇,抗應龍之虹梁,列棼橑以布翼,荷棟桴而高驤。雕玉瑱以居楹,裁金璧以飾璫,發五色之渥采,光爓朗以景彰。於是左屦右平,重軒三階,閨房周通,門闥洞開,列鍾虡於中庭,立金人於端闈,仍增崖而衡閾,臨峻路而啟扉。徇以離殿別寢,承以崇臺閒館,煥若列星,紫宮是環。清涼宣溫,神仙長年,金華玉堂,白虎麒麟,區宇若茲,不可殫論。增槃業峨,登降炤爛,殊形詭制,每各異觀,乘茵步輦,唯所息宴。後宮則有掖庭椒房,后妃之室,合歡增成,安處常寧,茞若椒風,披香發越,蘭林蕙草,鴛鸞飛翔之列。昭陽特盛,隆乎孝成,屋不呈材,牆不露形,裛以藻繡,絡以綸連,隨侯明月,錯落其閒,金釭銜璧,是為列錢,翡翠火齊,流燿含英,懸黎垂棘,夜光在焉。於是玄墀釦切,玉階彤庭,礝磩采緻,琳禄青熒,珊瑚碧樹,周阿而生。紅羅颯纚,綺組繽紛,精曜華燭,俯仰如神。後宮之號,十有四位,窈窕繁華,更盛迭貴,處乎斯列者,蓋以百數。左右廷中,朝堂百僚之位,蕭曹魏邴,謀謨乎其上。佐命則垂統,輔翼則成化,流大漢之愷悌,蕩亡秦之毒螫。故令斯人揚樂和之聲,作畫一之歌,功德著於祖宗,膏澤洽于黎庶。又有天祿石渠,典籍之府,命夫諄誨故老,名儒師傅,講論乎六蓺,稽合乎同異。又有承明金馬,著作之庭,大雅宏達,於茲為群,元元本本,周見洽聞,啟發篇章,校理祕文。周以鉤陳之位,衛以嚴更之署,總禮官之甲科,群百郡之廉孝。虎賁贅衣,閹尹閽寺,陛戟百重,各有攸司。周廬千列,徼道綺錯。輦路經營,脩涂飛閣。自未央而連桂宮,北彌明光而渉長樂,陵墱道而超西墉,混建章而外屬,設璧門之鳳闕,上柧棱而棲金雀。內則別風之嶕嶢,眇麗巧而竦擢,張千門而立萬戶,順陰陽以開闔。爾乃正殿崔巍,層構厥高,臨乎未央,經駘盪而出馺娑,洞枍詣與天梁,上反宇以蓋戴,激日景而納光。神明鬱其特起,遂偃蹇而上躋,軼雲雨於太半,虹霓回帶於棼楣,雖輕迅與僄狡,猶愕眙而不敢階。攀井幹而未半,目眴轉而意迷,舍櫺檻而卻倚,若顛墜而復稽,魂怳怳以失度,巡回涂而下低。既懲懼於登望,降周流以彷徨,步甬道以縈紆,又杳窱而不見陽。排飛闥而上出,若游目於天表,似無依之洋洋。前唐中而後太液,攬滄海之湯湯,揚波濤於碣石,激神嶽之嶈嶈,濫瀛洲與方壺,蓬萊起乎中央。於是靈草冬榮,神木叢生,巖峻崔崒,金石崢嶸。抗仙掌與承露,擢雙立之金莖,軼埃壒之混濁,鮮顥氣之清英。騁文成之丕誕,馳五利之所刑,庶松喬之群類,時游從乎斯庭,實列仙之攸館,匪吾人之所寧。

爾乃盛娛游之壯觀,奮大武乎上囿,因茲以威戎夸狄,燿威而講事。命荊州使起鳥,詔梁野而驅獸,毛群內闐,飛羽上覆,接翼側足,集禁林而屯聚。水衡虞人,理其營表,種別群分,部曲有署。罘罔連紘,籠山絡野,列卒周匝,星羅雲布。於是乘鑾輿備法駕,帥群臣,披飛廉,入苑門。遂繞酆鎬,歷上蘭,六師發冑,百獸駭殫,震震爚爚,雷奔電激,草木塗地,山淵反覆,蹂蹸其十二三,乃拗怒而少息。爾乃期門佽飛,列刃鑽鍭,要趹追蹤,鳥驚觸絲,獸駭值鋒,機不鸪掎,弦不再控,矢無單殺,中必疊雙,颮颮紛紛,矰繳相纏,風毛雨血,灑野蔽天。平原赤,勇士厲,猿狖失木,豺狼攝竄。爾乃移師趨險,並蹈潛穢,窮虎奔突,狂兕觸蹶。許少施巧,秦成力折,掎僄狡,扼猛噬,脫角挫脰,徒搏獨殺。挾師豹,拖熊螭,頓犀犛,曳豪羆,超迥壑,越峻崖,蹶巉巖,鉅石隤,松柏仆,叢林摧,草木無餘,禽獸殄夷。於是天子乃登屬玉之館,歷長楊之榭,覽山川之體埶,觀三軍之殺獲,原野蕭條,目極四裔,禽相鎮厭,獸相枕藉。然後收禽會眾,論功賜胙,陳輕騎以行炰,騰酒車而斟酌,割鮮野食,舉燧命爵。饗賜畢,勞逸齊,大輅鳴鸞,容與裴回,集乎豫章之宇,臨乎昆明之池。左牽牛而右織女,似雲漢之無崖,茂樹蔭蔚,芳草被堤,蘭茞發色,曄曄猗猗,若摛錦布繡,燭燿乎其陂。玄鶴白鷺,黃鵠鵁鸛,鶬鴰鴇鶂,鳧鷖鴻鴈,朝發河海,夕宿江漢,沈浮往來,雲集霧散。於是後宮乘輚路,登龍舟,張鳳蓋,建華旗,袪黼帷,鏡清流,靡微風,澹淡浮。櫂女謳,鼓吹震,聲激越,謍厲天,鳥群翔,魚闚淵。招白閒,下雙鵠,揄文竿,出比目。撫鴻幢,御矰繳,方舟並騖,俛仰極樂。遂風舉雲搖,浮遊普覽,前乘秦領,後越九嵕,東薄河華,西涉岐雍,宮館所歷,百有餘區,行所朝夕,儲不改供。禮上下而接山川,究休祐之所用,採遊童之歡謠,第從臣之嘉頌。于斯之時,都都相望,邑邑相屬,國藉十世之基,家承百年之業,士食舊德之名氏,農服先疇之畎畝,商修族世之所鬻,工用高曾之規矩,粲乎隱隱,各得其所。

若臣者,徒觀跡乎舊墟,聞之乎故老,什分而未得其一端,故不能遍舉也。


\end{pinyinscope}