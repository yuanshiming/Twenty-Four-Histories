\article{禮儀下}

\begin{pinyinscope}
大喪諸侯王列侯始封貴人公主薨

不豫,太醫令丞將醫人,就進所宜藥。嘗藥監、近臣中常侍、小黃門皆先嘗藥,過量十二。公卿朝臣問起居無閒。太尉告請南郊,司徒、司空告請宗廟,告五獄、四瀆、群祀,並禱求福。疾病,公卿復如禮。

三公奏尚書顧命,太子即日即天子位于柩前,請太子即皇帝位,皇后為皇太后。奏可。群臣皆出,吉服入會如儀。太尉升自阼階,當柩御坐北面稽首,讀策畢,以傳國玉璽綬東面跪授皇太子,即皇帝位。中黃門掌兵以玉具、隨侯珠、斬蛇寶劍授太尉,告令群臣,群臣皆伏稱萬歲。或大赦天下。遣使者詔開城門、宮門,罷屯衛兵。群臣百官罷,入成喪服如禮。兵官戎。三公,太常如禮。

故事:百官五日一會臨,故吏二千石、刺史、在京都郡國上計掾史皆五日一會。天下吏民發喪臨三日。先葬二日,皆旦晡臨。既葬,釋服,無禁嫁娶、祠祀。佐史以下,布衣冠幘,絰帶無過三寸,臨庭中。武吏布幘大冠。大司農出見錢穀,給六丈布直。以葬,大紅十五日,小紅十四日,纖七日,釋服。部刺史、二千石、列侯在國者及關內侯、宗室長吏及因郵奉奏,諸侯王遣大夫一人奉奏,弔臣請驛馬露布,奏可。

以木為重,高九尺,廣容八歷,裹以葦席。巾門、喪帳皆以簟。車皆去輔轓,疏布惡輪。走卒皆布恳幘。太僕四輪輈為賓車,大練為屋幙。中黃門、虎賁各二十人執紼。司空擇土造穿。太史卜日。謁者二人,中謁者僕射、中謁者副將作,油緹帳以覆坑。方石治黃腸題湊便房如禮。

大駕,太僕御。方相氏黃金四目,蒙熊皮,玄衣朱裳,執戈揚楯,立乘四馬先驅。旂之制,長三仞,十有二游,曳地,畫日、月、升龍,書旐曰「天子之柩」。謁者二人立乘六馬為次。大駕甘泉鹵簿,金根容車,蘭臺法駕。喪服大行載飾如金根車。皇帝從送如禮。太常上啟奠。夜漏二十刻,太尉冠長冠,衣齋衣,乘高車,詣殿止車門外。使者到,南向立,太尉進伏拜受詔。太尉詣南郊。未盡九刻,大鴻臚設九賓隨立,群臣入位,太尉行禮。執事皆冠長冠,衣齋衣。太祝令跪讀謚策,太尉再拜稽首。治禮告事畢。太尉奉謚策,還詣殿端門。太常上祖奠,中黃門尚衣奉衣登容根車。東園武士載大行,司徒卻行道立車前。治禮引太尉入就位,大行車西少南,東面奉策,太史令奉哀策立後。太常跪曰「進」,皇帝進。太尉讀謚策,藏金匱。皇帝次科藏于廟。太史奉哀策葦篋詣陵。太尉旋復公位,再拜立哭。太常跪曰「哭」,大鴻臚傳哭,十五舉音,止哭。太常行遣奠皆如禮。請哭止哭如儀。

晝漏上水,請發。司徒、河南尹先引車轉,太常跪曰「請拜送」。載車著白系參繆紼,長三十丈,大七寸為輓,六行,行五十人。公卿以下子弟凡三百人,皆素幘委貌冠,衣素裳。校尉三人,皆赤幘不冠,絳科單衣,持幢幡。候司馬丞為行首,皆銜枚。羽林孤兒、巴俞擢歌者六十人,為六列。鐸司馬八人,執鐸先。大鴻臚設九賓,隨立陵南羨門道東,北面;諸侯、王公、特進道西,北面東上;中二千石、二千石、列侯宜九賓東,北面西上。皇帝白布幕素裏,夾羨道東,西向如禮。容車幄坐羨道西,南向,車當坐,南向,中黃門尚衣奉衣就幄坐。車少前,太祝進醴獻如禮。司徒跪曰「大駕請舍」,太史令自車南,北面讀哀策,掌故在後,已哀哭。太常跪曰「哭」,大鴻臚傳哭如儀。司徒跪曰「請就下位」,東園武士奉下車。司徒跪曰「請就下房」,都導東園武士奉車入房。司徒、太史令奉謚、哀策。

東園武士執事下明器。筲八盛,容三升,黍一,稷一,麥一,粱一,稻一,麻一,菽一,小豆一。甕三,容三升,醯一,醢一,屑一。黍飴。載以木桁,覆以疏布。甒二,容三升,醴一,酒一。載以木桁,覆以功布。瓦鐙一。彤矢四,軒輖中,亦短衛。彤矢四,骨,短衛。彤弓一。墒八,牟八,豆八,籩八,形方酒壺八。槃匜一具。杖、几各一。蓋一。鍾十六,無虡。鎛四,無虡。磬十六,無虡。壎一,簫四,笙一,箎一,柷一,敔一,瑟六,琴一,竽一,筑一,坎侯一。干、戈各一,笮一,甲一,冑一。輓車九乘,芻靈三十六匹。瓦灶二,瓦釜二,瓦甑一。瓦鼎十二,容五升。匏勺一,容一升。瓦案九。瓦大杯十六,容三升。瓦小杯二十,容二升。瓦飯槃十。瓦酒樽二,容五斗。匏勺二,容一升。

祭服衣送皆畢,東園匠曰「可哭」,在房中者皆哭。太常、大鴻臚請哭止如儀。司徒曰「百官事畢,臣請罷」,從入房者皆再拜,出,就位。太常導皇帝就贈位。司徒跪曰「請進贈」,侍中奉持鴻洞。贈王珪長尺四寸,薦以紫巾,廣袤各三寸,緹裏,赤纁周緣;贈幣,玄三纁二,各長尺二寸,廣充幅。皇帝進跪,臨羨道房戶,西向,手下贈,投鴻洞中,三。東園匠奉封入藏房中。太常跪曰「皇帝敬再拜,請哭」,大鴻臚傳哭如儀。太常跪曰「贈事畢」,皇帝促就位。容根車游載容衣。司徒至便殿,并蚂騎皆從容車玉帳下。司徒跪曰「請就幄」,導登。尚衣奉衣,以次奉器衣物,藏於便殿。太祝進醴獻。凡下,用漏十刻。禮畢,司空將校復土。

皇帝、皇后以下皆去麤服,服大紅,還宮反廬,立主如禮。桑木主尺二寸,不書謚。虞禮畢,祔於廟,如禮。

先大駕日游冠衣于諸宮諸殿,群臣皆吉服從會如儀。皇帝近臣喪服如禮。醳大紅,服小紅,十一升都布練冠。醳小紅,服纖。醳纖,服留黃,冠常冠。近臣及二千石以下皆服留黃冠。百官衣皁。每變服,從哭詣陵會如儀。祭以特牲,不進毛血首。司徒、光祿勳備三爵如禮。

太皇太后、皇太后崩,司空以特牲告謚于祖廟如儀。長樂太僕、少府、大長秋典喪事,三公奉制度,他皆如禮儀。

合葬:羨道開通,皇帝謁便房,太常導至羨道,去杖,中常侍受,至柩前,謁,伏哭止如儀。辭,太常導出,中常侍授杖,升車歸宮。已下,反虞立主如禮。諸郊廟祭服皆下便房。五時朝服各一襲在陵寢,其餘及宴服皆封以篋笥,藏宮殿後閤室。

諸侯王、列侯、始封貴人、公主薨,皆令贈印璽、玉柙銀縷;大貴人、長公主銅縷。諸侯王、貴人、公主、公、將軍、特進皆賜器,官中二十四物。使者治喪,穿作,柏槨,百官會送,如故事。諸侯王、公主、貴人皆樟棺,洞朱,雲氣畫。公、特進樟棺黑漆。中二千石以下坎侯漆。朝臣中二千石、將軍,使者弔祭,郡國二千石、六百石以至黃綬,皆賜常車驛牛贈祭。宜自佐史以上達,大斂皆以朝服。君臨弔若遣使者,主人免絰去杖望馬首如禮。免絰去杖,不敢以戚凶服當尊者。自王、主、貴人以下至佐史,送車騎導從吏卒,各如其官府。載飾以蓋,龍首魚尾,華布牆,纁上周,交絡前後,雲氣畫帷裳。中二千石以上有輜,左龍右虎,朱鳥玄武;公侯以上加倚鹿伏熊。千石以下,緇布蓋牆,魚龍首尾而已。二百石黃綬以下至于處士,皆以簟席為牆蓋。其正妃、夫人、妻皆如之。諸侯王,傅、相、中尉、內史典喪事,大鴻臚奏謚,天子使者贈璧帛,載日命謚如禮。下陵,群臣醳麤服如儀,主人如禮。

贊曰:大禮雖簡,鴻儀則容。天尊地卑,君莊臣恭。質文通變,哀敬交從。元序斯立,家邦迺隆。


\end{pinyinscope}