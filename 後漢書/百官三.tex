\article{百官三}

\begin{pinyinscope}
宗正大司農少府

宗正,卿一人,中二千石。本注曰:掌序錄王國嫡庶之次,及諸宗室親屬遠近,郡國歲因計上宗室名籍。若有犯法當髡以上,先上諸宗正,宗正以聞,乃報決。丞一人,比千石。

諸公主,每主家令一人,六百石。丞一人,三百石。本注曰:其餘屬吏增減無常。

右屬宗正。本注曰:中興省都司空令、丞。

大司農,卿一人,中二千石。本注曰:掌諸錢穀金帛諸貨幣。郡國四時上月旦見錢穀簿,其逋未畢,各具別之。邊郡諸官請調度者,皆為報給,損多益寡,取相給足。丞一人,比千石。部丞一人,六百石。本注曰:部丞主帑藏。

太倉令一人,六百石。本注曰:主受郡國傳漕穀。丞一人。

平準令一人,六百石。本注曰:掌知物賈,主練染,作采色。丞一人。

導官令一人,六百石。本注曰:主舂御米,及作乾糒。導,擇也。丞一人。

右屬大司農。本注曰:郡國鹽官、鐵官本屬司農,中興皆屬郡縣。又有廩犧令,六百石,掌祭祀犧牲鴈鶩之屬。及雒陽巿長、滎陽敖倉官,中興皆屬河南尹。餘均輸等皆省。

少府,卿一人,中二千石。本注曰:掌中服御諸物,衣服寶貨珍膳之屬。丞一人,比千石。

太醫令一人,六百石。本注曰:掌諸醫。藥丞、方丞各一人。本注曰:藥丞主藥。方丞主藥方。

太官令一人,六百石。本注曰:掌御飲食。左丞、甘丞、湯官丞、果丞各一人。本注曰:左丞主飲食。甘丞主膳具。湯官丞主酒。果丞主果。

守宮令一人,六百石。本注曰:主御紙筆墨,及尚書財用諸物及封泥。丞一人。

上林苑令一人,六百石。本注曰:主苑中禽獸。頗有民居,皆主之。捕得其獸送太官。丞、尉各一人。

侍中,比二千石。本注曰:無員。掌侍左右,贊導眾事,顧問應對。法駕出,則多識者一人參乘,餘皆騎在乘輿車後。本有僕射一人,中興轉為祭酒,或置或否。

中常侍,千石。本注曰:宦者,無員。後增秩比二千石。掌侍左右,從入內宮,贊導內眾事,顧問應對給事。

黃門侍郎,六百石。本注曰:無員。掌侍從左右,給事中,關通中外。及諸王朝見於殿上,引王就坐。

小黃門,六百石。:宦者,無員。掌侍左右,受尚書事。上在內宮,關通中外,及中宮已下眾事。諸公主及王太妃等有疾苦,則使問之。

黃門令一人,六百石。本注曰:宦者。主省中諸宦者。丞、從丞各一人。本注曰:宦者。從丞主出入從。

黃門署長、畫室署長、玉堂署長各一人。丙署長七人。皆四百石,黃綬。本注曰:宦者。各主中宮別處。

中黃門冗從僕射一人,六百石。本注曰:宦者。主中黃門冗從。居則宿衛,直守門戶;出則騎從,夾乘輿車。

中黃門,比百石。本注曰:宦者,無員。後增比三百石。掌給事禁中。

掖庭令一人,六百石。本注曰:宦者。掌後宮貴人采女事。左右丞、暴室丞各一人。本注曰:宦者。暴室丞主中婦人疾病者,就此室治;其皇后、貴人有罪,亦就此室。

永巷令一人,六百石。本注曰:宦者。典官婢侍使。丞一人。本注曰:宦者。

御府令一人,六百石。本注曰:宦者。典官婢作中衣服及補浣之屬。丞、織室丞各一人。本注曰:宦者。

祠祀令一人,六百石。本注曰:典中諸小祠祀。丞一人。本注曰:宦者。

鉤盾令一人,六百石。本注曰:宦者。典諸近池苑囿遊觀之處。丞、永安丞各一人,三百石。本注曰:宦者。永安,北宮東北別小宮名,有園觀。苑中丞、果丞、鴻池丞、南園丞各一人,二百石。本注曰:苑中丞主苑中離宮。果丞主果園。鴻池,池名,在雒陽東二十里。南園在雒水南。濯龍監、直里監各一人,四百石。本注曰:濯龍亦園名,近北宮。直里亦園名也,在雒陽城西南角。

中藏府令一人,六百石。本注曰:掌中幣帛金銀諸貨物。丞一人。

內者令一人,六百石。本注曰:掌中布張諸衣物。左右丞各一人。

尚方令一人,六百石。本注曰:掌上手工作御刀劍諸好器物。丞一人。

尚書令一人,千石。本注曰:承秦所置,武帝用宦者,更為中書謁者令,成帝用士人,復故。掌凡選署及奏下尚書曹文書眾事。

尚書僕射一人,六百石。本注曰:署尚書事,令不在則奏下眾事。

尚書六人,六百石。本注曰:成帝初置尚書四人,分為四曹:常侍曹尚書主公卿事;二千石曹尚書主郡國二千石事;民曹尚書主凡吏上書事;客曹尚書主外國夷狄事。世祖承遵,後分二千石曹,又分客曹為南主客曹、北主客曹,凡六曹。左右丞各一人,四百石。本注曰:掌錄文書期會。左丞主吏民章報及騶伯史。右丞假署印綬,及紙筆墨諸財用庫藏。侍郎三十六人,四百石。本注曰:一曹有六人,主作文書起草。令史十八人,二百石。本注曰:曹有三,主書。後增劇曹三人,合二十一人。

符節令一人,六百石。本注曰:為符節臺率,主符節事。凡遣使掌授節。尚符璽郎中四人。本注曰:舊二人在中,主璽及虎符、竹符之半者。符節令史,二百石。本注曰:掌書。

御史中丞一人,千石。本注曰:御史大夫之丞也。舊別監御史在殿中,密舉非法。及御史大夫轉為司空,因別留中,為御史臺率,後又屬少府。治書侍御史二人,六百石。本注曰:掌選明法律者為之。凡天下諸讞疑事,掌以法律當其是非。侍御史十五人,六百石。本注曰:掌察舉非法,受公卿群吏奏事,有違失舉劾之。凡郊廟之祠及大朝會、大封拜,則一人監威儀,有違失則劾奏。

蘭臺令史,六百石。本注曰:掌奏及印工文書。

右屬少府。本注曰:職屬少府者,自太醫、上林凡四官。自侍中至御史,皆以文屬焉。承秦,凡山澤陂池之稅,名曰禁錢,屬少府。世祖改屬司農,考工轉屬太僕,都水屬郡國。孝武帝初置水衡都尉,秩比二千石,別主上林苑有離宮燕休之處,世祖省之,并其職於少府。每立秋貙劉之日,輒暫置水衡都尉,事訖乃罷之。少府本六丞,省五。又省湯官、織室令,置丞。又省上林十池監,胞人長丞,宦者、昆臺、佽飛三令,二十一丞。又省水衡屬官令、長、丞、尉二十餘人。章和以下,中官稍廣,加嘗藥、太官、御者、鉤盾、尚方、考工、別作監,皆六百石,宦者為之,轉為兼副,或省,故錄本官。


\end{pinyinscope}