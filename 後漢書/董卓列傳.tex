\article{董卓列傳}

\begin{pinyinscope}
董卓字仲穎,隴西臨洮人也。性麤猛有謀。少嘗遊羌中,盡與豪帥相結。後歸耕於野,諸豪帥有來從之者,卓為殺耕牛,與共宴樂,豪帥感其意,歸相斂得雜畜千餘頭以遺之,由是以健俠知名。為州兵馬掾,常徼守塞下。卓膂力過人,雙帶兩鞬,左右馳射,為羌胡所畏。

桓帝末,以六郡良家子為羽林郎,從中郎將張奐為軍司馬,共擊漢陽叛羌,破之,拜郎中,賜縑九千匹。卓曰:「為者則己,有者則士。」乃悉分與吏兵,無所留。稍遷西域戊己校尉,坐事免。後為并州刺史,河東太守。

中平元年,拜東中郎將,持節,代盧植擊張角於下曲陽,軍敗抵罪。其冬,北地先零羌及枹罕河關群盜反叛,遂共立湟中義從胡北宮伯玉、李文侯為將軍,殺護羌校尉泠徵。伯玉等乃劫致金城人邊章、韓遂,使專任軍政,共殺金城太守陳懿,攻燒州郡。明年春,將數萬騎入寇三輔,侵逼園陵,托誅宦官為名。詔以卓為中郎將,副左車騎將軍皇甫嵩征之。嵩以無功免歸,而邊章、韓遂等大盛。朝廷復以司空張溫為車騎將軍,假節,執金吾袁滂為副。拜卓破虜將軍,與盪寇將軍周慎並統於溫。并諸郡兵步騎合十餘萬,屯美陽,以衛園陵。章、遂亦進兵美陽。溫、卓與戰,輒不利。十一月,夜有流星如火,光長十餘丈,照章、遂營中,驢馬盡鳴。賊以為不祥,欲歸金城。卓聞之喜,明日,乃與右扶風鮑鴻等并兵俱攻,大破之,斬首數千級。章、遂敗走榆中,溫乃遣周慎將三萬人追討之。溫參軍事孫堅說慎曰:「賊城中無穀,當外轉糧食。堅願得萬人斷其運道,將軍以大兵繼後,賊必困乏而不敢戰。若走入羌中,并力討之,則涼州可定也。」慎不從,引軍圍榆中城。而章、遂分屯葵園狹,反斷慎運道。慎懼,乃棄車重而退。溫時亦使卓將兵三萬討先零羌,卓於望垣北為羌胡所圍,糧食乏絕,進退逼急。乃於所度水中偽立衝,以為捕魚,而潛從衝下過軍。比賊追之,決水已深,不得度。時眾軍敗退,唯卓全師而還,屯於扶風,封斄鄉侯,邑千戶。

三年春,遣使者持節就長安拜張溫為太尉。三公在外,始之於溫。其冬,徵溫還京師,韓遂乃殺邊章及伯玉、文侯,擁兵十餘萬,進圍隴西。太守李相如反,與遂連和,共殺涼州刺史耿鄙。而鄙司馬扶風馬騰,亦擁兵反叛,又漢陽王國,自號「合眾將軍」,皆與韓遂合。共推王國為主,悉令領其眾,寇掠三輔。五年,圍陳倉。乃拜卓前將軍,與左將軍皇甫嵩擊破之。韓遂等復共廢王國,而劫故信都令漢陽閻忠,使督統諸部。忠恥為眾所脅,感恚病死。遂等稍爭權利,更相殺害,其諸部曲並各分乖。

六年,徵卓為少府,不肯就,上書言:「所將湟中義從及秦胡兵皆詣臣曰:『牢直不畢,稟賜斷絕,妻子飢凍。』牽挽臣車,使不得行。羌胡敝腸狗態,臣不能禁止,輒將順安慰。增異復上。」朝廷不能制,頗以為慮。及靈帝寑疾,璽書拜卓為并州牧,令以兵屬皇甫嵩。卓復上書言曰:「臣既無老謀,又無壯事,天恩誤加,掌戎十年。士卒大小相狎彌久,戀臣畜養之恩,為臣奮一旦之命。乞將之北州,效力邊垂。」於是駐兵河東,以觀時變。

及帝崩,大將軍何進、司隸校尉袁紹謀誅閹宦,而太后不許,乃私呼卓將兵入朝,以脅太后。卓得召,即時就道。並上書曰:「中常侍張讓等竊倖承寵,濁亂海內。臣聞揚湯止沸,莫若去薪;潰帻雖痛,勝於內食。昔趙鞅興晉陽之甲,以逐君側之惡人。今臣輒鳴鍾鼓如洛陽,請收讓等,以清姦穢。」卓未至而何進敗,虎賁中郎將袁術乃燒南宮,欲討宦官,而中常侍段珪等劫少帝及陳留王夜走小平津。卓遠見火起,引兵急進,未明到城西,聞少帝在北芒,因往奉迎。帝見卓將兵卒至,恐怖涕泣。卓與言,不能辭對;與陳留王語,遂及禍亂之事。卓以王為賢,且為董太后所養,卓自以與太后同族,有廢立意。

初,卓之入也,步騎不過三千,自嫌兵少,恐不為遠近所服,率四五日輒夜潛出軍近營,明旦乃大陳旌鼓而還,以為西兵復至,洛中無知者。尋而何進及弟苗先所領部曲皆歸於卓,卓又使呂布殺執金吾丁原而并其眾,卓兵士大盛。乃諷朝廷策免司空劉弘而自代之。因集議廢立。百僚大會,卓乃奮首而言曰:「大者天地,其次君臣,所以為政。皇帝闇弱,不可以奉宗廟,為天下主。今欲依伊尹、霍光故事,更立陳留王,何如?」公卿以下莫敢對。卓又抗言曰:「昔霍光定策,延年案劍。有敢沮大議,皆以軍法從之。」坐者震動。尚書盧植獨曰:「昔太甲既立不明,昌邑罪過千餘,故有廢立之事。今上富於春秋,行無失德,非前事之比也。」卓大怒,罷坐。明日復集群僚於崇德前殿,遂脅太后,策廢少帝。曰:「皇帝在喪,無人子之心,威儀不類人君,今廢為弘農王。」乃立陳留王,是為獻帝。又議太后蹙迫永樂太后,至令憂死,逆婦姑之禮,無孝順之節,遷於永安宮,遂以弒崩。

卓遷太尉,領前將軍事,加節傳斧鉞虎賁,更封郿侯。卓乃與司徒黃琬、司空楊彪,俱帶鈇鑕詣闕上書,追理陳蕃、竇武及諸黨人,以從人望。於是悉復蕃等爵位,擢用子孫。

尋進卓為相國,入朝不趨,劍履上殿。封母為池陽君,置丞令。

是時洛中貴戚室第相望,金帛財產,家家殷積。卓縱放兵士,突其廬舍,淫略婦女,剽虜資物,謂之「搜牢」。人情崩恐,不保朝夕。及何后葬,開文陵,卓悉取藏中珍物。又姦亂公主,妻略宮人,虐刑濫罰,睚眥必死,群僚內外莫能自固。卓嘗遣軍至陽城,時人會於社下,悉令就斬之,駕其車重,載其婦女,以頭繫車轅,歌呼而還。又壞五銖錢,更鑄小錢,悉取洛陽及長安銅人、鍾虡、飛廉、銅馬之屬,以充鑄焉。故貨賤物貴,穀石數萬。又錢無輪郭文章,不便人用。時人以為秦始皇見長人於臨洮,乃鑄銅人。卓,臨洮人也,而今毀之。雖成毀不同,凶暴相類焉。

卓素聞天下同疾閹官誅殺忠良,及其在事,雖行無道,而猶忍性矯情,擢用群士。乃任吏部尚書漢陽周珌、侍中汝南伍瓊、尚書鄭公業、長史何顒等。以處士荀爽為司空。其染黨錮者陳紀、韓融之徒,皆為列卿。幽滯之士,多所顯拔。以尚書韓馥為冀州刺史,侍中劉岱為兗州刺史,陳留孔骸為豫州刺史,潁川張咨為南陽太守。卓所親愛,並不處顯職,但將校而已。初平元年,馥等到官,與袁紹之徒十餘人,各興義兵,同盟討卓,而伍瓊、周珌陰為內主。

初,靈帝末,黃巾餘黨郭太等復起西河白波谷,轉寇太原,遂破河東,百姓流轉三輔,號為「白波賊」,眾十餘萬。卓遣中郎將牛輔擊之,不能卻。及聞東方兵起,懼,乃鴆殺弘農王,欲徙都長安。會公卿議,太尉黃琬、司徒楊彪廷爭不能得,而伍瓊、周珌又固諫之。卓因大怒曰:「卓初入朝,二子勸用善士,故相從,而諸君到官,舉兵相圖。此二君賣卓,卓何用相負!」遂斬瓊、珌。而彪、琬恐懼,詣卓謝曰:「小人戀舊,非欲沮國事也,請以不及為罪。」卓既殺瓊、珌,旋亦悔之,故表彪、琬為光祿大夫。於是遷天子西都。

初,長安遭赤眉之亂,宮室營寺焚滅無餘,是時唯有高廟、京兆府舍,遂便時幸焉。後移未央宮。於是盡徙洛陽人數百萬口於長安,步騎驅蹙,更相蹈藉,飢餓寇掠,積尸盈路。卓自屯留畢圭苑中,悉燒宮廟官府居家,二百里內無復孑遺。又使呂布發諸帝陵,及公卿已下冢墓,收其珍寶。

時長沙太守孫堅亦率豫州諸郡兵討卓。卓先遣將徐榮、李蒙四出虜掠。榮遇堅於梁,與戰,破堅,生禽潁川太守李旻,亨之。卓所得義兵士卒,皆以布纏裹,倒立於地,熱膏灌殺之。

時河內太守王匡屯兵河陽津,將以圖卓。卓遣疑兵挑戰,而潛使銳卒從小平津過津北,破之,死者略盡。明年,孫堅收合散卒,進屯梁縣之陽人。卓遣將胡軫、呂布攻之,布與軫不相能,軍中自驚恐,士卒散亂。堅追擊之,軫、布敗走。卓遣將李傕詣堅求和,堅拒絕不受,進軍大谷,距洛九十里。卓自出與堅戰於諸陵墓閒,卓敗走,卻屯黽池,聚兵於陝。堅進洛陽宣陽城門,更擊呂布,布復破走。堅乃埽除宗廟,平塞諸陵,分兵出函谷關,至新安、黽池閒,以涞卓後。卓謂長史劉艾曰:「關東諸將數敗矣,無能為也。唯孫堅小戇,諸將軍宜慎之。」乃使東中郎將董越屯黽池,中郎將段煨屯華陰,中郎將牛輔屯安邑,其餘中郎將、校尉布在諸縣,以禦山東。

卓諷朝廷使光祿勳宣璠持節拜卓為太師,位在諸侯王上。乃引還長安。百官迎路拜揖,卓遂僭擬車服,乘金華青蓋,爪畫兩轓,時人號「竿摩車」,言其服飾近天子也。以弟旻為左將軍,封鄠侯,兄子璜為侍中、中軍校尉,皆典兵事。於是宗族內外,並居列位。其子孫雖在髫齔,男皆封侯,女為邑君。

數與百官置酒宴會,淫樂縱恣。乃結壘於長安城東以自居。又築塢於郿,高厚七丈,號曰「萬歲塢」。積穀為三十年儲。自云:「事成,雄據天下;不成,守此足以畢老。」嘗至郿行塢,公卿已下祖道於橫門外。卓施帳幔飲設,誘降北地反者數百人,於坐中殺之。先斷其舌,次斬手足,次鑿其眼目,以鑊煮之。未及得死,偃轉柸案閒。會者戰慄,亡失匕箸,而卓飲食自若。諸將有言語蹉跌,便戮於前。又稍誅關中舊族,陷以叛逆。

時太史望氣,言當有大臣戮死者。卓乃使人誣衛尉張溫與袁術交通,遂笞溫於市,殺之,以塞天變。前溫出屯美陽,令卓與邊章等戰無功,溫召又不時應命,既到而辭對不遜。時孫堅為溫參軍,勸溫陳兵斬之。溫曰:「卓有威名,方倚以西行。」堅曰:「明公親帥王師,威振天下,何恃於卓而賴之乎?堅聞古之名將,杖鉞臨眾,未有不斷斬以示威武者也。故穰苴斬莊賈,魏絳戮楊干。今若縱之,自虧威重,後悔何及!」溫不能從,而卓猶懷忌恨,故及於難。

溫字伯慎,少有名譽,累登公卿,亦陰與司徒王允共謀誅卓,事未及發而見害。越騎校尉汝南伍孚忿卓凶毒,志手刃之,乃朝服懷佩刀以見卓。孚語畢辭去,卓起送至閤,以手撫其背,孚因出刀刺之,不中。卓自奮得免,急呼左右執殺之,而大詬曰:「虜欲反耶!」孚大言曰:「恨不得磔裂姦賊於都市,以謝天地!」言未畢而斃。

時王允與呂布及僕射士孫瑞謀誅卓。有人書「呂」字於布上,負而行於市,歌曰:「布乎!」有告卓者,卓不悟。三年四月,帝疾新愈,大會未央殿。卓朝服升車,既而馬驚墯泥,還入更衣。其少妻止之,卓不從,遂行。乃陳兵夾道,自壘及宮,左步右騎,屯衛周匝,令呂布等扞衛前後。王允乃與士孫瑞密表其事,使瑞自書詔以授布,令騎都尉李肅與布同心勇士十餘人,偽著衛士服於北掖門內以待卓。卓將至,馬驚不行,怪懼欲還。呂布勸令進,遂入門。肅以戟刺之,卓衷甲不入,傷臂墯車,顧大呼曰:「呂布何在?」布曰:「有詔討賊臣。」卓大罵曰:「庸狗敢如是邪!」布應聲持矛刺卓,趣兵斬之。主簿田儀及卓倉頭前赴其尸,布又殺之。馳齎赦書,以令宮陛內外。士卒皆稱萬歲,百姓歌舞於道。長安中士女賣其珠玉衣裝市酒肉相慶者,填滿街肆。使皇甫嵩攻卓弟旻於郿塢,殺其母妻男女,盡滅其族。乃尸卓於市。天時始熱,卓素充肥,脂流於地。守尸吏然火置卓臍中,光明達曙,如是積日。諸袁門生又聚董氏之尸,焚灰揚之於路。塢中珍藏有金二三萬斤,銀八九萬斤,錦綺繢縠紈素奇玩,積如丘山。

初,卓以牛輔子婿,素所親信,使以兵屯陝。輔分遣其校尉李傕、郭汜、張濟將步騎數萬,擊破河南尹朱雋於中牟。因掠陳留、潁川諸縣,殺略男女,所過無復遺類。呂布乃使李肅以詔命至陝討輔等,輔等逆與肅戰,肅敗走弘農,布誅殺之。其後牛輔營中無故大驚,輔懼,乃齎金寶踰城走。左右利其貨,斬輔,送首長安。

傕、汜等以王允、呂布殺董卓,故忿怒并州人,并州人其在軍者男女數百人,皆誅殺也。牛輔既敗,眾無所依,欲各散去。傕等恐,乃先遣使詣長安,求乞赦免。王允以為一歲不可再赦,不許之。傕等益懷憂懼,不知所為。武威人賈詡時在傕軍,說之曰:「聞長安中議欲盡誅涼州人,諸君若棄軍單行,則一亭長能束君矣。不如相率而西,以攻長安,為董公報仇。事濟,奉國家以正天下;若其不合,走未後也。」傕等然之,各相謂曰:「京師不赦我,我當以死決之。若攻長安剋,則得天下矣;不剋,則鈔三輔婦女財物,西歸鄉里,尚可延命。」眾以為然,於是共結盟,率軍數千,晨夜西行。王允聞之,乃遣卓故將胡軫、徐榮擊之於新豐。榮戰死,軫以眾降。傕隨道收兵,比至長安,已十餘萬,與卓故部曲樊稠、李蒙等合,圍長安。城峻不可攻,守之八日,呂布軍有叟兵內反,引傕眾得入。城潰,放兵虜掠,死者萬餘人。殺衛尉种拂等。呂布戰敗出奔。王允奉天子保宣平城門樓上。於是大赦天下。李傕、郭汜、樊稠等皆為將軍。遂圍門樓,共表請司徒王允出,問「太師何罪」?允窮蹙乃下,後數日見殺。傕等葬董卓於郿,并收董氏所焚尸之灰,合斂一棺而葬之。葬日,大風雨,霆震卓墓,流水入藏,漂其棺木。

傕又遷車騎將軍,開府,領司隸校尉,假節。汜後將軍,稠右將軍,張濟為鎮東將軍,並封列侯。傕、汜、稠共秉朝政。濟出屯弘農。以賈詡為左馮翊,欲侯之。詡曰:「此救命之計,何功之有!」固辭乃止。更以為尚書典選。

明年夏,大雨晝夜二十餘日,漂沒人庶,又風如冬時。帝使御史裴茂訊詔獄,原繫者二百餘人。其中有為傕所枉繫者,傕恐茂赦之,乃表奏茂擅出囚徒,疑有姦故,請收之。詔曰:「災異屢降,陰雨為害,使者銜命宣布恩澤,原解輕微,庶合天心。欲釋冤結而復罪之乎!一切勿問。」

初,卓之入關,要韓遂、馬騰共謀山東。遂、騰見天下方亂,亦欲倚卓起兵。興平元年,馬騰從隴右來朝,進屯霸橋。時騰私有求於傕,不獲而怒,遂與侍中馬宇、右中郎將劉範、前涼州刺史种劭、中郎將杜稟合兵攻傕,連日不決。韓遂聞之,乃率眾來欲和騰、傕,既而復與騰合。傕使兄子利共郭汜、樊稠與騰等戰於長平觀下。遂、騰敗,斬首萬餘級,种劭、劉範等皆死。遂、騰走還涼州,稠等又追之。韓遂使人語稠曰:「天下反覆未可知,相與州里,今雖小違,要當大同,欲共一言。」乃駢馬交臂相加,笑語良久。軍還,利告傕曰:「樊、韓駢馬笑語,不知其辭,而意愛甚密。」於是傕、稠始相猜疑。猶加稠及郭汜開府,與三公合為六府,皆參選舉。

時長安中盜賊不禁,白日虜掠,傕、汜、稠乃參分城內,各備其界,猶不能制,而其子弟縱橫,侵暴百姓。是時穀一斛五十萬,豆麥二十萬,人相食啖,白骨委樍,臭穢滿路。帝使侍御史侯汶出太倉米豆為飢人作糜,經日而死者無降。帝疑賦卹有虛,乃親於御前自加臨檢。既知不實,使侍中劉艾出讓有司。於是尚書令以下皆詣省閣謝,奏收侯汶考實。詔曰:「未忍致汶于理,可杖五十。」自是後多得全濟。

明年春,傕因會刺殺樊稠於坐,由是諸將各相疑異,傕、汜遂復理兵相攻。安西將軍楊定者,故卓部曲將也。懼傕忍害,乃與汜合謀迎天子幸其營。傕知其計,即使兄子暹將數千人圍宮。以車三乘迎天子、皇后。太尉楊彪謂暹曰:「古今帝王,無在人臣家者。諸君舉事,當上順天心,柰何如是!」暹曰:「將軍計決矣。」帝於是遂幸傕營,彪等皆徒從。亂兵入殿,掠宮人什物,傕又徙御府金帛乘輿器服,而放火燒宮殿官府居人悉盡。帝使楊彪與司空張喜等十餘人和傕、汜,汜不從,遂質留公卿。彪謂汜曰:「將軍達人閒事,柰何君臣分爭,一人劫天子,一人質公卿,此可行邪?」汜怒,欲手刃彪。彪曰:「卿尚不奉國家,吾豈求生邪!」左右多諫,汜乃止。遂引兵攻傕,矢及帝前,又貫傕耳。傕將楊奉本白波賊帥,乃將兵救傕,於是汜眾乃退。

是日,傕復移帝幸其北塢,唯皇后、宋貴人俱。傕使校尉監門,隔絕內外。尋復欲徙帝於池陽黃白城,君臣惶懼。司徒趙溫深解譬之,乃止。詔遣謁者僕射皇甫酈和傕、汜。酈先譬汜,汜即從命。又詣傕,傕不聽。曰:「郭多,盜馬虜耳,何敢欲與我同邪!必誅之。君觀我方略士眾,足辦郭多不?多又劫質公卿。所為如是,而君苟欲左右之邪!」汜一名多。酈曰:「今汜質公卿,而將軍脅主,誰輕重乎?」傕怒,呵遣酈,因令虎賁王昌追殺之。昌偽不及,酈得以免。傕乃自為大司馬。與郭汜相攻連月,死者以萬數。

張濟自陝來和解二人,仍欲遷帝權幸弘農。帝亦思舊京,因遣使敦請傕求東歸,十反乃許。車駕即日發邁。李傕出屯曹陽。以張濟為驃騎將軍,復還屯陝。遷郭汜車騎將軍,楊定後將軍,楊奉興義將軍。又以故牛輔部曲董承為安集將軍。汜等並侍送乘輿。汜遂復欲脅帝幸郿,定、奉、承不聽。汜恐變生,乃棄軍還就李傕。車駕進至華陰。寧輯將軍段煨乃具服御及公卿以下資儲,請帝幸其營。初,楊定與煨有隙,遂誣煨欲反,乃攻其營,十餘日不下。而煨猶奉給御膳,稟贍百官,終無二意。

李傕、郭汜既悔令天子東,乃來救段煨,因欲劫帝而西,楊定為汜所遮,亡奔荊州。而張濟與楊奉、董承不相平,乃反合傕、汜,共追乘輿,大戰於弘農東澗。承、奉軍敗,百官士卒死者不可勝數,皆棄其婦女輜重,御物符策典籍,略無所遺。射聲校尉沮雋被創墜馬。李傕謂左右曰:「尚可活不?」雋罵之曰:「汝等凶逆,逼迫天子,亂臣賊子,未有如汝者!」傕使殺之。天子遂露次曹陽。承、奉乃譎傕等與連和,而密遣閒使至河東,招故白波帥李樂、韓暹、胡才及南匈奴右賢王去卑,並率其眾數千騎來,與承、奉共擊傕等,大破之,斬首數千級,乘輿乃得進。董承、李樂擁衛左右,胡才、楊奉、韓暹、去卑為後距。傕等復來戰,奉等大敗,死者甚於東澗。自東澗兵相連綴四十里中,方得至陝,乃結營自守。時殘破之餘,虎賁羽林不滿百人,皆有離心。承、奉等夜乃潛議過河,使李樂先度具舟舡,舉火為應。帝步出營,臨河欲濟,岸高十餘丈,乃以絹縋而下。餘人或匍匐岸側,或從上自投,死亡傷殘,不復相知。爭赴舡者,不可禁制,董承以戈擊披之,斷手指於舟中者可掬。同濟唯皇后、宋貴人、楊彪、董承及后父執金吾伏完等數十人。其宮女皆為傕兵所掠奪,凍溺死者甚眾。既到大陽,止於人家,然後幸李樂營。百官飢餓,河內太守張楊使數千人負米貢餉。帝乃御牛車,因都安邑。河東太守王邑奉獻綿帛,悉賦公卿以下。封邑為列侯,拜胡才征東將軍,張楊為安國將軍,皆假節、開府。其壘壁群豎,競求拜職,刻印不給,至乃以錐畫之。或齎酒肉就天子燕飲。又遣太僕韓融至弘農,與傕、汜等連和。傕乃放遣公卿百官,頗歸宮人婦女,及乘輿器服。

初,帝入關,三輔戶口尚數十萬,自傕汜相攻,天子東歸後,長安城空四十餘日,強者四散,羸者相食,二三年閒,關中無復人跡。建安元年春,諸將爭權,韓暹遂攻董承,承奔張楊,楊乃使承先繕修洛宮。七月,帝還至洛陽,幸楊安殿。張楊以為己功,故因以「楊」名殿。乃謂諸將曰:「天子當與天下共之,朝廷自有公卿大臣,楊當出扞外難,何事京師?」遂還野王。楊奉亦出屯梁。乃以張楊為大司馬,楊奉為車騎將軍,韓暹為大將軍,領司隸校尉,皆假節鉞。暹與董承並留宿衛。

暹矜功恣睢,干亂政事,董承患之,潛召兗州牧曹操。操乃詣闕貢獻,稟公卿以下,因奏韓暹、張楊之罪。暹懼誅,單騎奔楊奉。帝以暹、楊有翼車駕之功,詔一切勿問。於是封衛將軍董承、輔國將軍伏完等十餘人為列侯,贈沮雋為弘農太守。曹操以洛陽殘荒,遂移帝幸許。楊奉、韓暹欲要遮車駕,不及,曹操擊之,奉、暹奔袁術,遂縱暴楊、徐閒。明年,左將軍劉備誘奉斬之。暹懼,走還并州,道為人所殺。胡才、李樂留河東,才為怨家所害,樂自病死。張濟飢餓,出至南陽,攻穰,戰死。郭汜為其將伍習所殺。

三年,使謁者僕射裴茂詔關中諸將段煨等討李傕,夷三族。以段煨為安南將軍,封閺鄉侯。

四年,張楊為其將楊醜所殺。以董承為車騎將軍,開府。

自都許之後,權歸曹氏,天子總己,百官備員而已。帝忌操專偪,乃密詔董承,使結天下義士共誅之。承遂與劉備同謀,未發,會備出征,承更與偏將軍王服、長水校尉种輯、議郎吳碩結謀。事泄,承、服、輯、碩皆為操所誅。

韓遂與馬騰自還涼州,更相戰爭,乃下隴據關中。操方事河北,慮其乘閒為亂,七年,乃拜騰征南將軍,遂征西將軍,並開府。後徵段煨為大鴻臚,病卒。復徵馬騰為衛尉,封槐里侯。騰乃應召,而留子超領其部曲。十六年,超與韓遂舉關中背曹操,操擊破之,遂、超敗走,騰坐夷三族。超攻殺涼州刺史韋康,復據隴右。十九年,天水人楊阜破超,超奔漢中,降劉備。韓遂走金城羌中,為其帳下所殺。初,隴西人宗建在枹罕,自稱「河首平漢王」,署置百官三十許年。曹操因遣夏侯淵擊建,斬之,涼州悉平。

論曰:董卓初以虓闞為情,因遭崩剝之埶,故得蹈藉彝倫,毀裂畿服。夫以刳肝斮趾之性,則群生不足以厭其快,然猶折意縉紳,遲疑陵奪,尚有盜竊之道焉。及殘寇乘之,倒山傾海,崑岡之火,自茲而焚,版蕩之篇,於焉而極。嗚呼,人之生也難矣!天地之不仁甚矣!

贊曰:百六有會,過、剝成災。董卓滔天,干逆三才。方夏崩沸,皇京煙埃。無禮雖及,餘祲遂廣。矢延王輅,兵纏魏象。區服傾回,人神波蕩。


\end{pinyinscope}