\article{天文中}

\begin{pinyinscope}
安四十六順二十三質三

孝明永平元年四月丁酉,流星大如斗,起天市樓,西南行,光照地。流星為外兵,西南行為西南夷。是時益州發兵擊姑復蠻夷大牟替滅陵,斬首傳詣雒陽。

三年六月丁卯,彗星出天船北,長二尺所,稍北行至亢南,百三十五日去。天船為水,彗出之為大水。是歲伊、雒水溢,到津城門,壞伊橋;郡七縣三十二皆大水。

四年八月辛酉,客星出梗河,西北指貫索,七十日去。梗河為胡兵。至五年十一月,北匈奴七千騎入五原塞,十二月又入雲中,至原陽。貫索,貴人之牢。其十二月,陵鄉侯梁松坐怨望懸飛書誹謗朝廷下獄死,妻子家屬徙九真。

七年正月戊子,流星大如杯,從織女西行,光照地。織女,天之真女,流星出之,女主憂。其月癸卯,光烈皇后崩。

八年六月壬午,長星出柳、張三十七度,犯軒轅,刺天船,陵太微,氣至上陛,凡見五十六日去。柳,周地。是歲多雨水,郡十四傷稼。

九年正月戊申,客星出牽牛,長八尺,歷建星至房南,滅見至五十日。牽牛主吳、越,房、心為宋。後廣陵王荊與沈涼,楚王英與顏忠各謀逆,事覺,皆自殺。廣陵屬吳,彭城古宋地。

十三年閏月丁亥,火犯輿鬼,為大喪,質星為大臣誅戮。其十二月,楚王英與顏忠等造作妖謀反,事覺,英自殺,忠等皆伏誅。

十四年正月戊子,客星出昴,六十日,在軒轅右角稍滅。昴主邊兵。後一年,漢遣奉車都尉顯親侯竇固、駙馬都尉耿秉、騎都尉耿忠、開陽城門候秦彭、太僕祭肜,將兵擊匈奴。一曰,軒轅右角為貴相,昴為獄事,客星守之為大獄。是時考楚事未訖,司徒虞延與楚王英黨與黃初、公孫弘等交通,皆自殺,或下獄伏誅。

十五年十一月乙丑,太白入月中,為大將戮,人主亡,不出三年。後三年,孝明帝崩。

十六年正月丁丑,歲星犯房右驂,北第一星不見,辛巳乃見。房右驂為貴臣,歲星犯之為見誅。是後司徒邢穆,坐與阜陵王延交通知逆謀自殺。四月癸未,太白犯畢。畢為邊兵。後北匈奴寇,入雲中,至咸陽。使者高弘發三郡兵追討,無所得。太僕祭肜坐不進下獄。

十八年六月己未,彗星出張,長三尺,轉在郎將,南入太微,皆屬張。張,周地,為東都。太微,天子廷。彗星犯之為兵喪。其八月壬子,孝明帝崩。

孝章建初元年,正月丁巳,太白在昴西一尺。八月庚寅,彗星出天市,長二尺所,稍行入牽牛三度,積四十日稍滅。太白在昴為邊兵,彗星出天市為外軍,牽牛為吳、越。是時蠻夷陳縱等及哀牢王類反,攻蕉唐城。永昌太守王尋走奔楪榆,安夷長宋延為羌所殺。以武威太守傅育領護羌校尉,馬防行車騎將軍,征西羌。又阜陵王延與子男魴謀反,大逆無道,得不誅。廢為侯。

二月九日甲寅,流星過紫宮中,長數丈,散為三,滅。十二月戊寅,彗星出婁三度,長八九尺,稍入紫宮中,百六日稍滅。流星過,入紫宮,皆大人忌。後四年六月癸丑,明德皇后崩。

元和元年四月丁巳,客星晨出東方,在胃八度,長三尺,歷閣道入紫宮,留四十日滅。閣道、柴宮,天子之宮也。客星犯入留久為大喪。後四年,孝章帝崩。

孝和永元元年正月辛卯,有流星起參,長四丈,有光,色黃白。二月,流星起天棓,東北行三丈所滅,色青白。壬申,夜有流星起太微東蕃,長三丈。三月丙辰,流星起天津。壬戌,有流星起天將軍,東北行。參為邊兵,天棓為兵,太微天廷,天津為水,天將軍為兵,流星起之皆為兵。其六月,漢遣車騎將軍竇憲、執金吾耿秉,與度遼將軍鄧鴻出朔方,並進兵臨私渠北鞮海,斬虜首萬餘級,獲生口牛馬羊百萬頭。日逐王等八十一部降,凡三十餘萬人。追單于至西海。是歲七月,又雨水漂人民,是其應。

二年正月乙卯,金、木俱在奎,丙寅,水又在奎。奎主武庫兵,三星會又為兵喪。辛未,水、金、木在婁,亦為兵,又為匿謀。二月丁酉,有流星大如桃,起紫宮東蕃,西北行五丈稍滅。四月丙辰,有流星大如瓜,起文昌東北,西南行至少微西滅。有頃音如雷聲,已而金在軒轅大星東北二尺所。八月丁未,有流星如雞子,起太微西,東南行四丈所消。十月癸未,有流星大如桃,起天津,西行六丈所消。十一月辛酉,有流星大如拳,起紫宮,西行到胃消。

三年九月丁卯,有流星大如雞子,起紫宮,西南至北斗柄閒消。紫宮天子宮,文昌、少微為貴臣,天津為水,北斗主殺。流星起,歷紫宮、文昌、少微、天津,文昌為天子使,出有兵誅也。竇憲為大將軍,憲弟篤、景等皆卿、校尉,憲女弟婿郭舉為侍中、射聲校尉,與衛尉鄧疊母元俱出入宮中,謀為不軌。至四年六月丙寅發覺,和帝幸北宮,詔執金吾、五校勒兵屯南、北宮,閉城門,捕舉。舉父長樂少府璜及疊,疊弟步兵校尉磊,母元,皆下獄誅。憲弟篤、景等皆自殺。金犯軒轅,女主失勢。竇氏被誅,太后失勢。

五年四月癸巳,太白、熒惑、辰星俱在東井。七月壬午,歲星犯軒轅大星。九月,金在南斗魁中。火犯房北第一星。東井,秦地,為法。三星合,內外有兵,又為法令及水。金入斗口中,為大將將死。火犯房北第一星,為將相。其六年正月,司徒丁鴻薨。七月水,大漂殺人民,傷五穀。許侯馬光有罪自殺。九月,行車騎將軍事鄧鴻、越騎校尉馮柱發左右羽林、北軍五校士及八郡跡射、烏桓、鮮卑,合四萬騎,與度遼將軍朱徵、護烏桓校尉任尚、中郎將杜崇征叛胡。十二月,車騎將軍鴻坐追虜失利,下獄死;度遼將軍徵、中郎將崇皆抵罪。

七年正月丁未,有流星起天津,入紫宮中滅。色青黃,有光。二月癸酉,金、火俱在參。戊寅,金、火俱在東井。八月甲寅,水、土、金俱在軫。十一月甲戌,金、火俱在心。十二月己卯,有流星起文昌,入紫宮消。丙辰,火、金、水俱在斗。流星入紫宮,金、火在心,皆為大喪。三星合軫為白衣之會,金、火俱在參、東井,皆為外兵,有死將。三星俱在斗,有戮將,若有死相。八年四月樂成王黨,七月樂成王宗皆薨。將兵長史吳棽坐事徵下獄誅。十月,北海王威自殺。十二月,陳王羨薨。其九年閏月,皇太后竇氏崩。遼東鮮卑,太守祭參不追虜,徵下獄誅。九月,司徒劉方坐事免官,自殺。隴西羌反,遣執金吾劉尚行征西將軍事,越騎校尉節鄉侯趙世發北軍五校、黎陽、雍營及邊胡兵三萬騎,征西羌。

十一年五月丙午,流星大如瓜,起氐,西南行,稍有光,白色。占曰:「流星白,為有使客,大為大使,小亦小使。疾期疾,遲亦遲。大如瓜為近小,行稍有光為遲也。又正王日,邊方有受王命者也。」明年二月,蜀郡旄牛徼外夷白狼樓薄種王唐繒等率種人口十七萬歸義內屬,賜金印紫綬錢帛。

十二年十一月癸酉,夜有蒼白氣,長三丈,起天園,東北指軍市,見積十日。占曰:「兵起,十日期歲。」明年十一月,遼東鮮卑二千餘騎寇右北平。

十三年十一月乙丑,軒轅第四星閒有小客星,色青黃。軒轅為後宮,星出之,為失勢。其十四年六月辛卯,陰皇后廢。

十六年四月丁未,紫宮中生白氣如粉絮。戊午,客星出紫宮西行至昂,五月壬申滅。七月庚午,水在輿鬼中。十月辛亥,流星起鉤陳,北行三丈,有光,色黃。白氣生紫宮中為喪。客星從紫宮西行至昴為趙。輿鬼為死喪。鉤陳為皇后,流星出之為中使。後一年,元興元年十〈二〉月二日,和帝崩,殤帝即位一年又崩,無嗣,鄧太后遣使者迎清河孝王子即位,是為孝安皇帝,是其應也。清河,趙地也。

元興元年二月庚辰,有流星起角、亢五丈所。四月辛亥,有流星起斗,東北行到須女。七月己巳,有流星起天市五丈所,光色赤。閏月辛亥,水、金俱在氐。流星起斗,東北行至須女。須女,燕地。天市為外軍。水、金會為兵誅。其年,遼東貊人反,鈔六縣,發上谷、漁陽、右北平、遼西烏桓討之。

孝殤帝延平元年正月丁酉,金、火在婁。金、火合為爍,為大人憂。是歲八月辛亥,孝殤帝崩。

孝安永初元年五月戊寅,熒惑逆行守心前星。八月戊申,客星在東井、弧星西南。心為天子明堂,熒惑逆行守之,為反臣。客星在東井,為大水。是時,安帝未臨朝,鄧太后攝政,鄧騭為車騎將軍,弟弘、悝、閶皆以校尉封侯,秉國勢。司空周章意不平,與王尊、叔元茂等謀,欲閉宮門,捕將軍兄弟,誅常侍鄭眾、蔡倫,劫刺尚書,廢皇太后,封皇帝為遠國王。事覺,章自殺。東井、弧皆秦地。是時羌反,斷隴道,漢遣騭將左右羽林、北軍五校及諸郡兵征之。是歲郡國四十一縣三百一十五雨水。四瀆溢,傷秋稼,壞城郭,殺人民,是其應也。

二年正月戊子,太白晝見。

三年正月庚戌,月犯心後星。己亥,太白入斗中。十二月,彗星起天菀南,東北指,長六七尺,色蒼白。太白晝見,為強臣。是時鄧氏方盛,月犯心後星,不利子。心為宋。五月丁酉,沛王牙薨。太白入斗中,為貴相凶。天菀為外軍,彗星出其南為外兵。是後使羌、氐討賊李貴,又使烏桓擊鮮卑,又使中郎將任尚、護羌校尉馬賢擊羌,皆降。

四年六月甲子,客星大如李,蒼白,芒氣長二尺,西南指上階星。癸酉,太白入輿鬼。指上階,為三公。後太尉張敏免官。太白入輿鬼,為將凶。後中郎將任尚坐贓千萬,檻車徵,棄巿。

五年六月辛丑,太白晝見,經天。元初元年三月癸酉,熒惑入輿鬼。二年九月辛酉,熒惑入輿鬼中。三年三月,熒惑入輿鬼中。五月丙寅,太白入畢口。七月甲寅,歲星入輿鬼。閏月己未,太白犯太微左執法。十一月甲午,客星見西方,己亥在虛、危,南至胃、昴。四年正月丙戌,歲星留輿鬼中。乙未,太白晝見丙上。四月壬戌,太白入輿鬼中。己巳,辰星入輿鬼中。五月己卯,辰星犯歲星。六月丙申,熒惑入輿鬼中,戊戌,犯輿鬼大星。九月辛巳,太白入南斗口中。五年三月丙申,鎮星犯東井鉞星。五月庚午,辰星犯輿鬼質星。丙戌,太白犯鉞星。六年四月癸丑,太白入輿鬼。六月丙戌,熒惑在輿鬼中。丁卯,鎮星在輿鬼中。辛巳,太白犯左執法。自永初五年到永寧,十年之中,太白一晝見經天,再入輿鬼,一守畢,再犯左執法,入南斗,犯鉞星。熒惑五入輿鬼。鎮星一犯東井鉞星,一入輿鬼。歲星、辰星再入輿鬼。凡五星入輿鬼中,皆為死喪。熒惑、太白甚犯鉞、質星為誅戮。斗為貴將。執法為近臣。客星在虛、危為喪,為哭泣。昴、畢為邊兵,又為獄事。至建光元年三月癸巳,鄧太后崩;五月庚辰,太后兄車騎將軍騭等七侯皆免官,自殺,是其應也。

延光二年八月己亥,熒惑出太微端門。三年二月辛未,太白犯昴。五月癸丑,太白入畢。九月壬寅,鎮星犯左執法。四年,太白入輿鬼中。六月壬辰,太白出太微。九月甲子,太白入斗口中。十一月,客星見天巿。熒惑出太微,為亂臣。太白犯昴、畢,為近兵,一曰大人當之。鎮星犯左執法,有誅臣。太白入輿鬼中,為大喪。太白出太微,為中宮有兵;入斗口,為貴將相有誅者。客星見天巿中,為貴喪。是時大將軍耿寶、中常侍江京、樊豐、小黃門劉安與阿母王聖、聖子女永等并構譖太子保,并惡太子乳母男、廚監邴吉。三年九月丁酉,廢太子為濟陰王,以北鄉侯懿代。殺男、吉,徙其父母妻子日南。四年三月丁卯,安帝巡狩,從南陽還,道寢疾,至葉崩,閻后與兄衛尉顯、中常侍江京等共隱匿,不令群臣知上崩,遣司徒劉喜等分詣郊廟,告天請命,載入北宮。庚午夕發喪,尊閻氏為太后。北鄉侯懿病薨,京等又不欲立保,白太后,更徵諸王子擇所立。中黃門孫程、王國、王康等十九人,共合謀誅顯、京等,立保為天子,是為孝順皇帝。皆姦人強臣狂亂王室,其於死亡誅戮,兵起宮中,是其應。

孝順永建二年二月癸未,太白晝見三十九日。閏月乙酉,太白晝見東南維四十一日。八月乙巳,熒惑入輿鬼。太白晝見,為強臣。熒惑為凶。輿鬼為死喪。質星為誅戮。是時中常侍高梵、張防、將作大匠翟酺、尚書令高堂芝、僕射張敦、尚書尹就、郎姜述、楊鳳等,及兗州刺史鮑就、使匈奴中郎張國、金城太守張篤、敦煌太守張朗,相與交通,漏泄,就、述棄巿,梵、防、酺、芝、敦、鳳、就、國皆抵罪。又定遠侯班始尚陰城公主堅得,鬥爭殺堅得,坐要斬馬巿,同產皆棄巿。

六年四月,熒惑入太微中,犯左、右執法西北方六寸所。十月乙卯,太白晝見。十二月壬申,客星芒氣長二尺餘,西南指,色蒼白,在牽牛六度。客星芒氣白為兵。牽牛為吳、越。後一年,會稽海賊曾於等千餘人燒句章,殺長吏,又殺鄞、鄮長,取官兵,拘殺吏民,攻東部都尉;揚州六郡逆賊章何等稱將軍,犯四十九縣,大攻略吏民。

陽嘉元年閏月戊子,客星氣白,廣二尺,長五丈,起天菀西南。主馬牛,為外軍,色白為兵。是時,敦煌太守徐白使疏勒王盤等兵二萬人入于窴界,虜掠斬首三百餘級。烏桓校尉耿嘩使烏桓親漢都尉戎末瘣等出塞,鈔鮮卑,斬首,獲生口財物;鮮卑怨恨,鈔遼東、代郡,殺傷吏民。是後,西戎、北狄為寇害,以馬牛起兵,馬牛亦死傷於兵中,至十餘年乃息。

永和二年五月戊申,太白晝見。八月庚子,熒惑犯南斗。斗為吳。明年五月,吳郡太守行丞事羊珍與越兵弟葉、吏民吳銅等二百餘人起兵反,殺吏民,燒官亭民舍,攻太守府。太守王衡距守,吏兵格殺珍等。又〈九〉江賊蔡伯流等數百人攻廣陵、九江,燒城郭,殺都長。

三年二月辛巳,太白晝見,戊子,在熒惑西南,光芒相犯。辛丑,有流星大如斗,從西北東行,長八九尺,色赤黃,有聲隆隆如雷。三月壬子,太白晝見。六月丙午,太白晝見。八月乙卯,太白晝見。閏月甲寅,辰星入輿鬼。己酉,熒惑入太微。乙卯,太白晝見。太白者,將軍之官,又為西州。晝見,陰盛,與君爭明。熒惑與太白相犯,為兵喪。流星為使,聲隆隆,怒之象也。辰星入輿鬼,為大臣有死者。熒惑入太微,亂臣在廷中。是時,大將軍梁商父子秉勢,故太白常晝見也。其四年正月,祀南郊,夕牲,中常侍張逵、蘧政、陽定、內者令石光、尚方令傅福等與中常侍曹騰、孟賁爭權,白帝言騰、賁與商謀反,矯詔命收騰、賁,賁自解說,順帝寤,解騰、賁縛。逵等自知事不從,各奔走,或自刺,解貂蟬投草中逃亡,皆得免。其六年,征西將軍馬賢擊西羌於北地謝姑山下,父子為羌所沒殺,是其應也。

四年七月壬午,熒惑入南斗犯第三星。五年四月戊午,太白晝見。八月己酉,熒惑入太微。斗為貴相,為揚州,熒惑犯入之為兵喪。其六年,大將軍商薨。九江、丹陽賊周生、馬勉等起兵攻沒郡縣。梁氏又專權於天廷中。

六年二月丁巳,彗星見東方,長六七尺,色青白,西南指營室及墳墓星。丁丑,彗星在奎一度,長六尺,癸未昏見,西北歷昴、畢,甲申,在東井,遂歷輿鬼、柳、七星、張,光炎及三台,至軒轅中滅。營室者,天子常宮。墳墓主死。彗星起而在營室、墳墓,不出五年,天下有大喪。後四年,孝順帝崩。昴為邊兵,又為趙。羌周馬父子後遂為寇。又劉文凇清河相射暠,欲立王蒜為天子,暠不聽,殺暠,王閉門距文,官兵捕誅文,蒜以惡人所凇,廢為尉氏侯,又徙為犍陽都鄉侯,薨,國絕。歷東井、輿鬼為秦,皆羌所攻鈔。炎及三台,為三公。是時,太尉杜喬及故太尉李固為梁翼所陷入,坐文書死。及至注、張為周,滅於軒轅中為後宮。其後懿獻后以憂死,梁氏被誅,是其應也。

漢安二年,正月己亥,太白晝見。五月丁亥,辰星犯輿鬼。六月乙丑,熒惑光芒犯鎮星。七月甲申,太白晝見。辰星犯輿鬼為大喪。熒惑犯鎮星為大人忌。明年八月,孝順帝崩,孝沖明年正月又崩。

孝質本初元年,三月癸丑,熒惑入輿鬼,四月辛巳,太白入輿鬼,皆為大喪。五月庚戌,太白犯熒惑,為逆謀。閏月一日,孝質帝為梁冀所鴆,崩。


\end{pinyinscope}