\article{李王鄧來列傳}

\begin{pinyinscope}
李通字次元,南陽宛人也。世以貨殖著姓。父守,身長九尺,容貌絕異,為人嚴毅,居家如官廷。初事劉歆,好星歷讖記,為王莽宗卿師。通亦為五威將軍從事,出補巫丞,有能名。莽末,百姓愁怨,通素聞守說讖云「劉氏復興,李氏為輔」,私常懷之。且居家富逸,為閭里雄,以此不樂為吏,乃自免歸。

及下江、新巿兵起,南陽騷動,通從弟軼,亦素好事,乃共計議曰:「今四方擾亂,新室且亡,漢當更興。南陽宗室,獨劉伯升兄弟汎愛容眾,可與謀大事。」通笑曰:「吾意也。」會光武避事在宛,通聞之,即遣軼往迎光武。光武初以通士君子相慕也,故往荅之。及相見,共語移日,握手極歡。通因具言讖文事,光武初殊不意,未敢當之。時守在長安,光武乃微觀通曰:「即如此,當如宗卿師何?」通曰:「已自有度矣。」因復備言其計。光武既深知通意,乃遂相約結,定謀議,期以材官都試騎士日,欲劫前隊大夫及屬正,因以號令大眾。乃使光武與軼歸舂陵,舉兵以相應。遣從兄子季之長安,以事報守。

季於道病死,守密知之,欲亡歸。素與邑人黃顯相善,時顯為中郎將,聞之,謂守曰:「今關門禁嚴,君狀貌非凡,將以此安之?不知詣闕自歸。事既未然,脫可免禍。」守從其計,即上書歸死,章未及報,留闕下。會事發覺,通得亡走,莽聞之,乃繫守於獄。而黃顯為請曰:「守聞子無狀,不敢逃亡,守義自信,歸命宮闕。臣顯願質守俱東,曉說其子。如遂悖逆,令守北向刎首,以謝大恩。」莽然其言。會前隊復上通起兵之狀,莽怒,欲殺守,顯爭之,遂并被誅,及守家在長安者盡殺之。南陽亦誅通兄弟、門宗六十四人,皆焚屍宛巿。

時漢兵亦已大合。通與光武、李軼相遇棘陽,遂共破前隊,殺甄阜、梁丘賜。

更始立,以通為柱國大將軍、輔漢侯。從至長安,更拜為大將軍,封西平王;軼為舞陰王;通從弟松為丞相。更始使通持節還鎮荊州,通因娶光武女弟伯姬,是為寧平公主。光武即位,徵通為衛尉。建武二年,封固始侯,拜大司農。帝每征討四方,常令通居守京師,鎮撫百姓,修宮室,起學官。五年春,代王梁為前將軍。六年夏,領破姦將軍侯進、捕虜將軍王霸等十營擊漢中賊。公孫述遣兵赴救,通等與戰於西域,破之,還屯田順陽。

時天下略定,通思避榮寵,以病上書乞身。詔下公卿群臣議。大司徒侯霸等曰:「王莽篡漢,傾亂天下,通懷伊、呂、蕭、曹之謀,建造大策,扶助神靈,輔成聖德。破家為國,忘身奉主,有扶危存亡之義。功德最高,海內所聞。通以天下平定,謙讓辭位。夫安不忘危,宜令通居職療疾。欲就諸侯,不可聽。」於是詔通勉致醫藥,以時視事。其夏,引拜為大司空。

通布衣唱義,助成大業,重以寧平公主故,特見親重。然性謙恭,常欲避權埶。素有消疾,自為宰相,謝病不視事,連年乞骸骨,帝每優寵之。令以公位歸第養疾,通復固辭。積二歲,乃聽上大司空印綬,以特進奉朝請。有司奏請封諸皇子,帝感通首創大謀,即日封通少子雄為召陵侯。每幸南陽,常遣使者以太牢祠通父冢。十八年卒,謚曰恭侯。帝及皇后親臨弔,送葬。

子音嗣。音卒,子定嗣。定卒,子黃嗣。黃卒,子壽嗣。

李軼後為朱鮪所殺。更始之敗,李松戰死,唯通能以功名終。永平中,顯宗幸宛,詔諸李隨安眾宗室會見,並受賞賜,恩寵篤焉。

論曰:子曰「富與貴是人之所欲,不以其道得之,不處也」。李通豈知夫所欲而未識以道者乎!夫天道性命,聖人難言之,況乃億測微隱,猖狂無妄之福,汙滅親宗,以觖一切之功哉!昔蒙穀負書,不徇楚難;即墨用齊,義雪燕恥。彼之趣舍所立,其殆與通異乎?

王常字顏卿,潁川舞陽人也。王莽末,為弟報仇,亡命江夏。久之,與王鳳、王匡等起兵雲杜綠林中,聚眾數萬人,以常為偏裨,攻傍縣。後與成丹、張卬別入南郡藍口,號下江兵。王莽遣嚴尤、陳茂擊破之。常與丹、卬收散卒入蔞谿,劫略鍾、龍閒,眾復振。引軍與荊州牧戰於上唐,大破之,遂北至宜秋。

是時,漢兵與新巿、平林眾俱敗於小長安,各欲解去。伯升聞下江軍在宜秋,即與光武及李通俱造常壁,曰:「願見下江一賢將,議大事。」成丹、張卬共推遣常。伯升見常,說以合從之利。常大悟,曰:「王莽篡弒,殘虐天下,百姓思漢,故豪傑並起。今劉氏復興,即真主也。誠思出身為用,輔成大功。」伯升曰:「如事成,豈敢獨饗之哉!」遂與常深相結而去,常還,具為丹、卬言之。丹、卬負其眾,皆曰:「大丈夫既起,當各自為主,何故受人制乎?」常心獨歸漢,乃稍曉說其說將帥曰:「往者成、哀衰微無嗣,故王莽得承閒篡位。既有天下,而政令苛酷,積失百姓之心。民之謳吟思漢,非一日也,故使吾屬因此得起。夫民所怨者,天所去也;民所思者,天所與也。舉大事必當下順民心,上合天意,功乃可成。若負強恃勇,觸情恣欲,雖得天下,必復失之。以秦、項之埶,尚至夷覆,況今布衣相聚草澤?以此行之,滅亡之道也。今南陽諸劉舉宗起兵,觀其來議事者,皆有深計大慮,王公之才,與之并合,必成大功,此所以祐吾屬也。」下江諸將雖屈強少識,然素敬常,乃皆謝曰:「無王將軍,吾屬幾陷於不義。願敬受教。」即引兵與漢軍及新巿、平林合。於是諸部齊心同力,銳氣益壯,遂俱進,破殺甄阜、梁丘賜。

及諸將議立宗室,唯常與南陽士大夫同意欲立伯升,而朱鮪、張卬等不聽。及更始立,以常為廷尉、大將軍,封知命侯。別徇汝南、沛郡,還入昆陽,與光武共擊破王尋、王邑。更始西都長安,以常行南陽太守事,令專命誅賞,封為鄧王,食八縣,賜姓劉氏。常性恭儉,遵法度,南方稱之。

更始敗,建武二年夏,常將妻子詣洛陽,肉袒自歸。光武見常甚歡,勞之曰:「王廷尉良苦。每念往時,共更艱厄,何日忘之。莫往莫來,豈違平生之言乎?」常頓首謝曰:「臣蒙大命,得以鞭策託身陛下。始遇宜秋,後會昆陽,幸賴靈武,輒成斷金。更始不量愚臣,任以南州。赤眉之難,喪心失望,以為天下復失綱紀。聞陛下即位河北,心開目明,今得見闕庭,死無遺恨。」帝笑曰:「吾與廷尉戲耳。吾見廷尉,不憂南方矣。」乃召公卿將軍以下大會,具為群臣言:「常以匹夫興義兵,明于知天命,故更始封為知命侯。與吾相遇兵中,尤相厚善。」特加賞賜,拜為左曹,封山桑侯。

後帝於大會中指常謂群臣曰:「此家率下江諸將

輔翼漢室,心如金石,真忠臣也。」是日遷常為漢忠將軍,遣南擊鄧奉、董訢,令諸將皆屬焉。又詔常北擊河閒、漁陽,平諸屯聚。五年秋,攻拔湖陵,又與帝會任城,因從破蘇茂、龐萌。進攻下邳,常部當城門戰,一日數合,賊反走入城,常追迫之,城上射矢雨下,帝從百餘騎自城南高處望,常戰力甚,馳遣中黃門詔使引還,賊遂降。又別率騎都尉王霸共平沛郡賊。六年春,徵還洛陽,令夫人迎常於舞陽,歸家上冢。西屯長安,拒隗囂。七年,使使者持璽書即拜常為橫野大將軍,位次與諸將絕席。常別擊破隗囂將高峻於朝那。囂遣將過烏氏,常要擊破之。轉降保塞羌諸營壁,皆平之。九年,擊內黃賊,破降之。後北屯故安,拒盧芳。十二年,薨于屯所,謚曰節侯。

子廣嗣。三十年,徙封石城侯。永平十四年,坐與楚事相連,國除。

鄧晨字偉卿,南陽新野人也。世吏二千石。父宏,豫章都尉。晨初娶光武姊元。王莽末,光武嘗與兄伯升及晨俱之宛,與穰人蔡少公等讌語。少公頗學圖讖,言劉秀當為天子。或曰:「是國師公劉秀乎?」光武戲曰:「何用知非僕邪?」坐者皆大笑,晨心獨喜。及光武與家屬避吏新野,舍晨廬,甚相親愛。晨因謂光武曰:「王莽悖暴,盛夏斬人,此天亡之時也。往時會宛,獨當應邪?」光武笑不荅。

及漢兵起,晨將賓客會棘陽。漢兵敗小長安,諸將多亡家屬,光武單馬遁走,遇女弟伯姬,與共騎而奔。前行復見元,趣令上馬。元以手撝曰:「行矣,不能相救,無為兩沒也。」會追兵至,元及三女皆遇害。漢兵退保棘陽,而新野宰乃汙晨宅,焚其冢墓。宗族皆恚怒,曰:「家自富足,何故隨婦家人入湯鑊中?」晨終無恨色。

更始立,以晨為偏將軍。與光武略地潁川,俱夜出昆陽城,擊破王尋、王邑。又別徇陽翟以東,至京、密,皆下之。更始北都洛陽,以晨為常山太守。會王郎反,光武自薊走信都,晨亦閒行會於鉅鹿下,自請從擊邯鄲。光武曰:「偉卿以一身從我,不如以一郡為我北道主人。」乃遣晨歸郡。光武追銅馬、高胡群賊於冀州,晨發積射士千人,又遣委輸給軍不絕。光武即位,封晨房子侯。帝又感悼姊沒於亂兵,追封謚元為新野節義長公主,立廟于縣西。封晨長子汎為吳房侯,以奉公主之祀。

建武三年,徵晨還京師,數讌見,說故舊平生為歡。晨從容謂帝曰:「僕竟辯之。」帝大笑。從幸章陵,拜光祿大夫,使持節監執金吾賈復等擊平邵陵、新息賊。四年,從幸壽春,留鎮九江。

晨好樂郡職,由是復拜為中山太守,吏民稱之,常為冀州高第。十三年,更封南讀侯。入奉朝請,復為汝南太守。十八年,行幸章陵,徵晨行廷尉事。從至新野,置酒酣讌,賞賜數百十萬,復遣歸郡。晨興鴻郤陂數千頃田,汝土以殷,魚稻之饒,流衍它郡。明年,定封西華侯,復徵奉朝請。二十五年卒,詔遣中謁者備公主官屬禮儀,招迎新野主魂,與晨合葬於北芒。乘輿與中宮親臨喪送葬。謚曰惠侯。

小子棠嗣,後徙封武當。棠卒,子固嗣。固卒,子國嗣。國卒,子福嗣,永建元年卒,無子,國除。

來歙字君叔,南陽新野人也。六世祖漢,有才力,武帝世,以光祿大夫副樓船將軍楊僕,擊破南越、朝鮮。父仲,哀帝時為諫大夫,娶光武祖姑,生歙。光武甚親敬之,數共往來長安。

漢兵起,王莽以歙劉氏外屬,乃收繫之,賓客共篡奪,得免。更始即位,以歙為吏,從入關。數言事不用,以病去。歙女弟為漢中王劉嘉妻,嘉遣人迎歙。因南之漢中。更始敗,歙勸嘉歸光武,遂與嘉俱東詣洛陽。

帝見歙,大歡,即解衣以衣之,拜為太中大夫。是時方以隴、蜀為憂,獨謂歙曰:「今西州未附,子陽稱帝,道里阻遠,諸將方務關東,思西州方略,未知所任,其謀若何?」歙因自請曰:「臣嘗與隗囂相遇長安。其人始起,以漢為名。今陛下聖德隆興,臣願得奉威命,開以丹青之信,囂必束手自歸,則述自亡之埶,不足圖也。」帝然之。建武三年,歙始使隗囂。五年,復持節送馬援,因奉璽書於囂。既還,復往說囂,囂遂遣子恂隨歙入質,拜歙為中郎將。時山東略定,帝謀西收囂兵,與俱伐蜀,復使歙喻旨。囂將王元說囂,多設疑故,久冘豫不決。歙素剛毅,遂發憤質責囂曰:「國家以君知臧否,曉廢興,故以手書暢意。足下推忠誠,遣伯春委質,是臣主之交信也。今反欲用佞惑之言,為族滅之計,叛主負子,違背忠信乎?吉凶之決,在於今日。」欲前刺囂,囂起入,部勒兵,將殺歙,歙徐杖節就車而去。囂愈怒,王元勸囂殺歙,使牛邯將兵圍守之。囂將王遵諫曰:「愚聞為國者慎器與名,為家者畏怨重禍。俱慎名器,則下服其命;輕用怨禍,則家受其殃。今將軍遣子質漢,內懷它志,名器逆矣;外人有議欲謀漢使,輕怨禍矣。古者列國兵交,使在其閒,所以重兵貴和而不任戰也,何況承王命籍重質而犯之哉?君叔雖單車遠使,而陛下之外兄也。害之無損於漢,而隨以族滅。昔宋執楚使,遂有析骸易子之禍。小國猶不可辱,況於萬乘之主,重以伯春之命哉!」歙為人有信義,言行不違,及往來游說,皆可案覆,西州士大夫皆信重之,多為其言,故得免而東歸。

八年春,歙與征虜將軍祭遵襲略陽,遵道病還,分遣精兵隨歙,合二千餘人,伐山開道,從番須、回中徑至略陽,斬囂守將金梁,因保其城。囂大驚曰:「何其神也!」乃悉兵數萬人圍略陽,斬山築堤,激水灌城。歙與將士固死堅守,矢盡,乃發屋斷木以為兵。囂盡銳攻之,自春至秋,其士卒疲弊。帝乃大發關東兵,自將上隴,囂眾潰走,圍解。於是置酒高會,勞賜歙,班坐絕席,在諸將之右,賜歙妻縑千匹。詔使留屯長安,悉監護諸將。

歙因上書曰:「公孫述以隴西、天水為藩蔽,故得延命假息。今二郡平蕩,則述智計窮矣。宜益選兵馬,儲積資糧。昔趙之將帥多賈人,高帝懸之以重賞。今西州新破,兵人疲饉,若招以財穀,則其眾可集。臣知國家所給非一,用度不足,然有不得已也。」帝然之。於是大轉糧運,詔歙率征西大將軍馮異、建威大將軍耿弇、虎牙大將軍蓋延、揚武將軍馬成、武威將軍劉尚入天水,擊破公孫述將田弇、趙匡。明年,攻拔落門,隗囂支黨周宗、趙恢及天水屬縣皆降。

初王莽世,羌虜多背叛,而隗囂招懷其酋豪,遂得為用。及囂亡後,五谿、先零諸種數為寇掠,皆營塹自守,州郡不能討。歙乃大修攻具,率蓋延、劉尚及太中大夫馬援等進擊羌於金城,大破之,斬首虜數千人,獲牛羊萬餘頭,穀數十萬斛。又擊破襄武賊傅栗卿等。隴西雖平,而人飢,流者相望。歙乃傾倉廩,轉運諸縣,以賑贍之,於是隴右遂安,而涼州流通焉。

十一年,歙與蓋延、馬成進攻公孫述將王元、環安於河池、下辯,陷之,乘勝遂進。蜀人大懼,使刺客刺歙,未殊,馳召蓋延。延見歙,因伏悲哀,不能仰視。歙叱延曰:「虎牙何敢然!今使者中刺客,無以報國,故呼巨卿,欲相屬以軍事,而反效兒女子涕泣乎!刃雖在身,不能勒兵斬公邪!」延收淚強起,受所誡。歙自書表曰:「臣夜人定後,為何人所賊傷,中臣要害。臣不敢自惜,誠恨奉職不稱,以為朝廷羞。夫理國以得賢為本,太中大夫段襄,骨鯁可任,願陛下裁察。又臣兄弟不肖,終恐被罪,陛下哀憐,數賜教督。」投筆抽刃而絕。

帝聞大驚,省書覽涕,乃賜策曰:「中郎將來歙,攻戰連年,平定羌、隴,憂國忘家,忠孝彰著。遭命遇害,嗚呼哀哉!」使太中大夫贈歙中郎將、征羌侯印綬,謚曰節侯,謁者護喪事。喪還洛陽,乘輿縞素臨弔送葬。以歙有平羌、隴之功,故改汝南之當鄉縣為征羌國焉。

子褒嗣。十三年,帝嘉歙忠節,復封歙弟由為宜西侯。褒子稜,尚顯宗女武安公主。稜早歿,褒卒,以稜子歷為嗣。

論曰:世稱來君叔天下信士。夫專使乎二國之閒,豈厭詐謀哉?而能獨以信稱者,良其誠心在乎使兩義俱安,而己不私其功也。

歷字伯珍,少襲爵,以公主子,永元中,為侍中,監羽林右騎。永初三年,遷射聲校尉。永寧元年,代馮石為執金吾。延光元年,尊歷母為長公主。二年,遷歷太僕。

明年,中常侍樊豐與大將軍耿寶、侍中周廣、謝惲等共讒陷太尉楊震,震遂自殺。歷謂侍御史虞詡曰:「耿寶託元舅之親,榮寵過厚,不念報國恩,而傾側姦臣,誣奏楊公,傷害忠良,其天禍亦將至矣。」遂絕周廣、謝惲,不與交通。時皇太子驚病不安,避幸安帝乳母野王君王聖舍。太子乳母王男、廚監邴吉等以為聖舍新繕修,犯土禁,不可久御。聖及其女永與大長秋江京及中常侍樊豐、王男、邴吉等互相是非,聖、永遂誣譖男、吉,皆幽囚死,家屬徙比景。太子思男等,數為歎息。京、豐懼有後害,妄造虛無,構讒太子及東宮官屬。帝怒,召公卿以下會議廢立。耿寶等承旨,皆以為太子當廢。歷與太常桓焉、廷尉張皓議曰:「經說,年未滿十五,過惡不在其身。且男、吉之謀,皇太子容有不知,宜選忠良保傅,輔以禮義。廢置事重,此誠聖恩所宜宿留。」帝不從,是日遂廢太子為濟陰王。時監太子家小黃門籍建、中傅高梵等皆以無罪徙朔方。歷乃要結光祿勳祋諷,宗正劉瑋,將作大匠薛皓,侍中閭丘弘、陳光、趙代、施延,太中大夫朱倀、第五頡,中散大夫曹成,諫議大夫李尤,符節令張敬,持書侍御史龔調,羽林右監孔顯,城門司馬徐崇,衛尉守丞樂闈,長樂、未央廄令鄭安世等十餘人,俱詣鴻都門證太子無過。龔調據法律明之,以為男、吉犯罪,皇太子不當坐。帝與左右患之,乃使中常侍奉詔脅群臣曰:「父子一體,天性自然。以義割恩,為天下也。歷、諷等不識大典,而與群小共為讙譁,外見忠直而內希後福,飾邪違義,豈事君之禮?朝廷廣開言事之路,故且一切假貸;若懷迷不反,當顯明刑書。」諫者莫不失色。薛皓先頓首曰:「固宜如明詔。」歷怫然,廷詰皓曰:「屬通諫何言,而今復背之?大臣乘朝車,處國事,固得輾轉若此乎!」乃各稍自引起,歷獨守闕,連日不肯去。帝大怒,乃免歷兄弟官,削國租,黜公主不得會見。歷遂杜門不與親戚通,時人為之震慄。

及帝崩,閻太后起歷為將作大匠。順帝即位,朝廷咸稱社稷臣,於是遷為衛尉。祋諷、劉瑋、閭丘弘等先卒,皆拜其子為郎;朱倀、施延、陳光、趙代等並為公卿,任職;徵王男、邴吉家屬還京師,厚加賞賜;籍建、高梵等悉蒙顯擢。永建元年,拜歷車騎將軍,弟祉為步兵校尉,超為黃門侍郎。三年,母長公主薨,歷稱病歸第;服闋,復為大鴻臚。陽嘉二年,卒官。

子定嗣。定尚安帝妹平氏長公主,順帝時,為虎賁中郎將。定卒,子虎嗣,桓帝時,為屯騎校尉。弟豔,字季德,少好學下士,開館養徒,少歷顯位,靈帝時,再遷司空。

贊曰:李、鄧豪贍,舍家從讖。少公雖孚,宗卿未驗。王常知命,功惟帝念。款款君叔,斯言無玷。方獻三捷,永墜一劍。


\end{pinyinscope}