\article{郎顗襄楷列傳下}

\begin{pinyinscope}
郎顗字雅光,北海安丘人也。父宗,字仲綏,學京氏易,善風角、星筭、六日七分,能望氣占候吉凶,常賣卜自奉。安帝徵之,對策為諸儒表,後拜吳令。時卒有暴風,宗占知京師當有大火,記識時日,遣人參候,果如其言。諸公聞而表上,以博士徵之。宗恥以占驗見知,聞徵書到,夜縣印綬於縣廷而遁去,遂終身不仕。

顗少傳父業,兼明經典,隱居海畔,延致學徒常數百人。晝研精義,夜占象度,勤心銳思,朝夕無倦。州郡辟召,舉有道、方正,不就。

順帝時,災異屢見,陽嘉二年正月,公車徵,顗乃詣闕拜章曰:

臣聞天垂妖象,地見災符,所以譴告人主,責躬脩德,使正機平衡,流化興政也。易內傳曰:「凡災異所生,各以其政。變之則除,消之亦除。」伏惟陛下躬日昃之聽,溫三省之勤,思過念咎,務消祇悔。

方今時俗奢佚,淺恩薄義。夫救奢必於儉約,拯薄無若敦厚,安上理人,莫善於禮。修禮遵約,蓋惟上興,革文變薄,事不在下。故周南之德,關雎政本。本立道生,風行草從,澄其源者流清,溷其本者末濁。天地之道,其猶鼓籥,以虛為德,自近及遠者也。伏見往年以來,園陵數災,炎光熾猛,驚動神靈。易天人應曰:「君子不思遵利,茲謂無澤,厥災孽火燒其宮。」又曰:「君高臺府,犯陰侵陽,厥災火。」又曰:「上不儉,下不節,炎火並作燒君室。」自頃繕理西苑,修復太學,宮殿官府,多所搆飾。昔盤庚遷殷,去奢即儉,夏后卑室,盡力致美。又魯人為長府,閔子騫曰:「仍舊貫,何必改作。」臣愚以為諸所繕修,事可省減,稟卹貧人,賑贍孤寡,此天之意也,人之慶也,仁之本也,儉之要也。焉有應天養人,為仁為儉,而不降福者哉?

土者地祇,陰性澄靜,宜以施化之時,敬而勿擾,竊見正月以來,陰闇連日。易內傳曰:「久陰不雨,亂氣也,蒙之比也。蒙者,君臣上下相冒亂也。」又曰:「欲德不用,厥異常陰。」夫賢者化之本,雲者雨之具也。得賢而不用,猶久陰而不雨也。又頃前數日,寒過其節,冰既解釋,還復凝合。夫寒往則暑來,暑往則寒來,此言日月相推,寒暑相避,以成物也。今立春之後,火卦用事,當溫而寒,違反時節,由功賞不至,而刑罰必加也。宜須立秋,順氣行罰。

臣伏案飛候,參察眾政,以為立夏之後,當有震裂涌水之害。又比熒惑失度,盈縮往來,涉歷輿鬼,環繞軒轅。火精南方,夏之政也。政有失禮,不從夏令,則熒惑失行。正月三日至乎九日,三公卦。三公上應台階,下同元首。政失其道,則寒陰反節。「節彼南山」,詠自周詩;「股肱良哉」,著於虞典。而今之在位,競託高虛,納累鐘之奉,忘天下之憂,棲遲偃仰,寑疾自逸,被策文,得賜錢,即復起矣。何疾之易而愈之速?以此消伏災眚,興致升平,其可得乎?今選舉牧守,委任三府。長吏不良,既咎州郡,州郡有失,豈得不歸責舉者?而陛下崇之彌優,自下慢事愈甚,所謂大網疏,小網數。三公非臣之仇,臣非狂夫之作,所以發憤忘食,懇懇不已者,誠念朝廷欲致興平,非不能面譽也。

臣生長草野,不曉禁忌,披露肝膽,書不擇言。伏鑕鼎鑊,死不敢恨。謹詣闕奉章,伏待重誅。

書奏,帝復使對尚書。顗對曰:

臣聞明王聖主好聞其過,忠臣孝子言無隱情。臣備生人倫視聽之類,而稟性愚愨,不識忌諱,故出死忘命,懇懇重言。誠欲陛下修乾坤之德,開日月之明,披圖籍,案經典,覽帝王之務,識先後之政。如有闕遺,退而自改。本文武之業,擬堯舜之道,攘災延慶,號令天下。此誠臣顗區區之願,夙夜夢寤,盡心所計。謹條序前章,暢其旨趣,條便宜七事,具如狀對:

一事:陵園至重,聖神攸馮,而災火炎赫,迫近寑殿,魂而有靈,猶將驚動。尋宮殿官府,近始永平,歲時未積,便更修造。又西苑之設,禽畜是處,離房別觀,本不常居,而皆務精土木,營建無已,消功單賄,巨億為計。易內傳曰:「人君奢侈,多飾宮室,其時旱,其災火。」是故魯僖遭旱,修政自敕,下鐘鼓之縣,休繕治之官,雖則不寧,而時雨自降。由此言之,天之應人,敏於景響。今月十七日戊午,徵日也,日加申,風從寅來,丑時而止。丑、寅、申皆徵也,不有火災,必當為旱。願陛下校計繕修之費,永念百姓之勞,罷將作之官,減彫文之飾,損庖廚之饌,退宴私之樂。易中孚傳曰:「陽感天,不旋日。」如是,則景雲降集,眚沴息矣。

二事:去年已來,兌卦用事,類多不效。易傳曰:「有貌無實,佞人也;有實無貌,道人也。」寒溫為實,清濁為貌。今三公皆令色足恭,外厲內荏,以虛事上,無佐國之實,故清濁效而寒溫不效也,是以陰寒侵犯消息。占曰:「日乘則有妖風,日蒙則有地裂。」如是三年,則致日食,陰侵其陽,漸積所致。立春前後溫氣應節者,詔令寬也。其後復寒者,無寬之實也。夫十室之邑,必有忠信,率土之人,豈無貞賢,未聞朝廷有所賞拔,非所以求善贊務,弘濟元元。宜採納良臣,以助聖化。

三事:臣聞天道不遠,三五復反。今年少陽之歲,法當乘起,恐後年已往,將遂驚動,涉歷天門,災成戊己。今春當旱,夏必有水,臣以六日七分候之可知。夫災眚之來,緣類而應。行有玷缺,則氣逆于天,精感變出,以戒人君。王者之義,時有不登,則損滋徹膳。數年以來,穀收稍減,家貧戶饉,歲不如昔。百姓不足,君誰與足?水旱之災,雖尚未至,然君子遠覽,防微慮萌。老子曰:「人之飢也,以其上食稅之多也。」故孝文皇帝綈袍革舄,木器無文,約身薄賦,時致升平。今陛下聖德中興,宜遵前典,惟節惟約,天下幸甚。《易》曰:「天道無親,常與善人。」是故高宗以享福,宋景以延年。

四事:臣竊見皇子未立,儲宮無主,仰觀天文,太子不明。熒惑以去年春分後十六日在婁五度,推步三統,熒惑今當在翼九度,今反在柳三度,則不及五十餘度。去年八月二十四日戊辰,熒惑歷輿鬼東入軒轅,出后星北,東去四度,北旋復還。軒轅者,後宮也。熒惑者,至陽之精也,天之使也,而出入軒轅,繞還往來。《易》曰:「天垂象,見吉凶。」其意昭然可見矣。禮,天子一娶九女,嫡媵畢具。今宮人侍御,動以千計,或生而幽隔,人道不通,鬱積之氣,上感皇天,故遣熒惑入軒轅,理人倫,垂象見異,以悟主上。昔武王下車,出傾宮之女,表商容之閭,以理人倫,以表賢德,故天授以聖子,成王是也。今陛下多積宮人,以違天意,故皇胤多夭,嗣體莫寄。《詩》云:「敬天之怒,不敢戲豫。」方今之福,莫若廣嗣,廣嗣之術,可不深思?宜簡出宮女,恣其姻嫁,則天自降福,子孫千億。惟陛下丁寧再三,留神於此。左右貴倖,亦宜惟臣之言,以悟陛下。蓋善言古者合於今,善言天者合於人。願訪問百僚,有違臣言者,臣當受苟言之罪。

五事:臣竊見去年閏十月十七日己丑夜

,有白氣從西方天苑趨左足,入玉井,數日乃滅。春秋曰:「有星孛于大辰。大辰者何?大火也。大火為大辰,伐又為大辰,北極亦為大辰。」所以孛一宿而連三宿者,言北辰王者之宮也。凡中宮無節,政教亂逆,威武衰微,則此三星以應之也。罰者白虎,其宿主兵,其國趙、魏,變見西方,亦應三輔。凡金氣為變,發在秋節。臣恐立秋以後,趙、魏、關西將有羌寇畔戾之患。宜豫宣告諸郡,使敬授人時,輕徭役,薄賦斂,勿妄繕起,堅倉獄,備守衛,回選賢能,以鎮撫之。金精之變,責歸上司。宜以五月丙午,遣太尉服干戚,建井旟,書玉板之策,引白氣之異,於西郊責躬求愆,謝咎皇天,消滅妖氣。蓋以火勝金,轉禍為福也。

六事:臣竊見今月十四日乙卯巳時,白虹貫日。凡日傍氣色白而純者名為虹。貫日中者,侵太陽也;見於春者,政變常也。方今中官外司,各各考事,其所考者,或非急務。又恭陵火災,主名未立,多所收捕,備經考毒。尋火為天戒,以悟人君,可順而不可違,可敬而不可慢。陛下宜恭己內省,以備後災。凡諸考案,并須立秋。又易傳曰:「公能其事,序賢進士,後必有喜。」反之,則白虹貫日。以甲乙見者,則譴在中台。自司徒居位,陰陽多謬,久無虛己進賢之策,天下興議,異人同咨。且立春以來,金氣再見,金能勝木,必有兵氣,宜黜司徒以應天意。陛下不早攘之,將負臣言,遺患百姓。

七事:臣伏惟漢興以來三百三十九歲。於詩三基,高祖起亥仲二年,今在戌仲十年。詩氾歷樞曰:「卯酉為革政,午亥為革命,神在天門,出入候聽。」言神在戌亥,司候帝王興衰得失,厥善則昌,厥惡則亡。於易雄雌祕歷,今值困乏。凡九二困者,眾小人欲共困害君子也。經曰:「困而不失其所,其唯君子乎!」唯獨賢聖之君,遭困遇險,能致命遂志,不去其道。陛下乃者潛龍養德,幽隱屈厄,即位之元,紫宮驚動,歷運之會,時氣已應。然猶恐妖祥未盡,君子思患而豫防之。臣以為戌仲已竟,來年入季,文帝改法,除肉刑之罪,至今適三百載。宜因斯際,大蠲法令,官名稱號,輿服器械,事有所更,變大為小,去奢就儉,機衡之政,除煩為簡。改元更始,招求幽隱,舉方正,徵有道,博採異謀,開不諱之路。

臣陳引際會,恐犯忌諱,書不盡言,未敢究暢。

臺詰顗曰:「對云『白虹貫日,政變常也』。朝廷率由舊章,何所變易而言變常?又言『當大蠲法令,革易官號』。或云變常以致災,或改舊以除異,何也?又陽嘉初建,復欲改元,據何經典?其以實對。」顗對曰:

方春東作,布德之元,陽氣開發,養導萬物。王者因天視聽,奉順時氣,宜務崇溫柔,遵其行令。而今立春之後,考事不息,秋冬之政,行乎春夏,故白虹春見,掩蔽日曜。凡邪氣乘陽,則虹蜺在日,斯皆臣下執事刻急所致,殆非朝廷優寬之本。此其變常之咎也。又今選舉皆歸三司,非有周召之才,而當則哲之重,每有選用,輒參之掾屬,公府門巷,賓客填集,送去迎來,財貨無已。其當遷者,競相薦謁,各遣子弟,充塞道路,開長姦門,興致浮偽,非所謂率由舊章也。尚書職在機衡,宮禁嚴密,私曲之意,羌不得通,偏黨之恩,或無所用。選舉之任,不如還在機密。臣誠愚戇,不知折中,斯固遠近之論,當今之宜。又孔子曰:「漢三百載,計歷改憲。」三百四歲為一德,五德千五百二十歲,五行更用。王者隨天,譬猶自春徂夏,改青服絳者也。自文帝省刑,適三百年,而輕微之禁,漸已殷積。王者之法,譬猶江河,當使易避而難犯也。故《易》曰:「易則易知,簡則易從,易簡而天下之理得矣。」今去奢即儉,以先天下,改易名號,隨事稱謂。《易》曰:「君子之道,或出或處,同歸殊塗,一致百慮。」是知變常而善,可以除災,變常而惡,必致於異。今年仲竟,來年入季,仲終季始,歷運變改,故可改元,所以順天道也。

臣顗愚蔽,不足以荅聖問。

顗又上書薦黃瓊、李固,并陳消災之術曰:

臣前對七事,要政急務,宜於今者,所當施用。誠知愚淺,不合聖聽,人賤言廢,當受誅罰,征營惶怖,靡知厝身。

臣聞刳舟剡楫,將欲濟江海也;聘賢選佐,將以安天下也。昔唐堯在上,群龍為用,文武創德,周召作輔,是以能建天地之功,增日月之耀者也。《詩》云:「赫赫王命,仲山甫將之。邦國若否,仲山甫明之。」宣王是賴,以致雍熙。陛下踐祚以來,勤心庶政,而三九之位,未見其人,是以災害屢臻,四國未寧。臣考之國典,驗之聞見,莫不以得賢為功,失士為敗。且賢者出處,翔而後集,爵以德進,則其情不苟,然後使君子恥貧賤而樂富貴矣。若有德不報,有言不酬,來無所樂,進無所趨,則皆懷歸藪澤,修其故志矣。夫求賢者,上以承天,下以為人。不用之,則逆天統,違人望。逆天統則災眚降,違人望則化不行。災眚降則下呼嗟,化不行則君道虧。四始之缺,五際之厄,其咎由此。豈可不剛健篤實,矜矜慄慄,以守天功盛德大業乎?

臣伏見光祿大夫江夏黃瓊,耽道樂術,清亮自然,被褐懷寶,含味經籍,又果於從政,明達變復。朝廷前加優寵,賓于上位。瓊入朝日淺,謀謨未就,因以喪病,致命遂志。老子曰:「大音希聲,大器晚成。」善人為國,三年乃立。天下莫不嘉朝廷有此良人,而復怪其不時還任。陛下宜加隆崇之恩,極養賢之禮,徵反京師,以慰天下。又處士漢中李固,年四十,通游夏之蓺,履顏閔之仁。絜白之節,情同皦日,忠貞之操,好是正直,卓冠古人,當世莫及。元精所生,王之佐臣,天之生固,必為聖漢,宜蒙特徵,以示四方。夫有出倫之才,不應限以官次。昔顏子十八,天下歸仁;子奇稚齒,化阿有聲。若還瓊徵固,任以時政,伊尹、傅說,不足為比,則可垂景光,致休祥矣。臣顗明不知人,伏聽眾言,百姓所歸,臧否共歎。願汎問百僚,覈其名行,有一不合,則臣為欺國。惟留聖神,不以人廢言。

謹復條便宜四事,附奏於左:

一事:孔子作春秋,書「正月」者,敬歲之始也。王者則天之象,因時之序,宜開發德號,爵賢命士,流寬大之澤,垂仁厚之德,順助元氣,含養庶類。如此,則天文昭爛,星辰顯列,五緯循軌,四時和睦。不則太陽不光,天地溷濁,時氣錯逆,霾霧蔽日。自立春以來,累經旬朔,未見仁德有所施布,但聞罪罰考掠之聲。夫天之應人,疾於景響,而自從入歲,常有蒙氣,月不舒光,日不宣曜。日者太陽,以象人君。政變於下,日應於天。清濁之占,隨政抑揚。天之見異,事無虛作。豈獨陛下倦於萬機,帷幄之政有所闕歟?何天戒之數見也!臣願陛下發揚乾剛,援引賢能,勤求機衡之寄,以獲斷金之利。臣之所陳,輒以太陽為先者,明其不可久闇,急當改正。其異雖微,其事甚重。臣言雖約,其旨甚廣。惟陛下乃眷臣章,深留明思。

二事:孔子曰:「雷之始發大壯始,君弱臣彊從解起。」今月九日至十四日,大壯用事,消息之卦也。於此六日之中,雷當發聲,發聲則歲氣和,王道興也。《易》曰:「雷出地奮,豫,先王以作樂崇德,殷薦之上帝。」雷者,所以開發萌牙,辟陰除害。萬物須雷而解,資雨而潤。故經曰:「雷以動之,雨以潤之。」王者崇寬大,順春令,則雷應節,不則發動於冬,當震反潛。故易傳曰:「當雷不雷,太陽弱也。」今蒙氣不除,日月變色,則其效也。天網恢恢,疏而不失,隨時進退,應政得失。大人者,與天地合其德,與日月合其明,琁璣動作,與天相應。雷者號令,其德生養。號令殆廢,當生而殺,則雷反作,其時無歲。陛下若欲除災昭祉,順天致和,宜察臣下尤酷害者,亟加斥黜,以安黎元,則太皓悅和,雷聲乃發。

三事:去年十月二十日癸亥,太白與歲星合於房、心。太白在北,歲星在南,相離數寸,光芒交接。房、心者,天帝明堂布政之宮。孝經鉤命決曰:「歲星守心年穀豐。」尚書洪範記曰:「月行中道,移節應期,德厚受福,重華留之。」重華者,謂歲星在心也。今太白從之,交合明堂,金木相賊,而反同合,此以陰陵陽,臣下專權之異也。房、心東方,其國主宋。石氏經曰:「歲星出左有年,出右無年。」今金木俱東,歲星在南,是為出右,恐年穀不成,宋人飢也。陛下宜審詳明堂布政之務,然後妖異可消,五緯順序矣。

四事:易傳曰:「陽無德則旱,陰僭陽亦旱。」陽無德者,人君恩澤不施於人也。陰僭陽者,祿去公室,臣下專權也。自冬涉春,訖無嘉澤,數有西風,反逆時節。朝廷勞心,廣為禱祈,薦祭山川,暴龍移市。臣聞皇天感物,不為偽動,災變應人,要在責己。若令雨可請降,水可攘止,則歲無隔并,太平可待。然而災害不息者,患不在此也。立春以來,未見朝廷賞錄有功,表顯有德,存問孤寡,賑恤貧弱,而但見洛陽都官奔車東西,收繫纖介,牢獄充盈。臣聞恭陵火處,比有光曜,明此天災,非人之咎。丁丑大風,掩蔽天地。風者號令,天之威怒,皆所以感悟人君忠厚之戒。又連月無雨,將害宿麥。若一穀不登,則飢者十三四矣。陛下誠宜廣被恩澤,貸贍元元。昔堯遭九年之水,人有十載之蓄者,簡稅防災,為其方也。願陛下早宣德澤,以應天功。若臣言不用,朝政不改者,立夏之後乃有澍雨,於今之際未可望也。若政變於朝而天不雨,則臣為誣上,愚不知量,分當鼎鑊。

書奏,特詔拜郎中,辭病不就,即去歸家。至四月京師地震,遂陷。其夏大旱。秋,鮮卑入馬邑城,破代郡兵。明年,西羌寇隴右。皆略如顗言。後復公車徵,不行。

同縣孫禮者,積惡凶暴,好游俠,與其同里人常慕顗名德,欲與親善。顗不顧,以此結怨,遂為禮所殺。

襄楷字公矩,平原隰陰人也。好學博古,善天文陰陽之術。

桓帝時,宦官專朝,政刑暴濫,又比失皇子,災異尤數。延熹九年,楷自家詣闕上疏曰:

臣聞皇天不言,以文象設教。堯舜雖聖,必歷象日月星辰,察五緯所在,故能享百年之壽,為萬世之法。臣竊見去歲五月,熒惑入太微,犯帝坐,出端門,不軌常道。其閏月庚辰,太白入房,犯心小星,震動中耀。中燿,天王也;傍小星者,天王子也。夫太微天廷,五帝之坐,而金火罰星揚光其中,於占,天子凶;又俱入房、心,法無繼嗣。今年歲星久守太微,逆行西至掖門,還切執法。歲為木精,好生惡殺,而淹留不去者,咎在仁德不修,誅罰太酷。前七年十二月,熒惑與歲星俱入軒轅,逆行四十餘日,而鄧皇后誅。其冬大寒,殺鳥獸,害魚鱉,城傍竹柏之葉有傷枯者。臣聞於師曰:「柏傷竹枯,不出三年,天子當之。」今洛陽城中人夜無故叫呼,云有火光,人聲正諠,於占亦與竹柏枯同。自春夏以來,連有霜雹及大雨雷,而臣作威作福,刑罰急刻之所感也。

太原太守劉暧、南陽太守成档,志除姦邪,其所誅翦,皆合人望,而陛下受閹豎之譖,乃遠加考逮。三公上書乞哀暧等,不見採察,而嚴被譴讓。憂國之臣,將遂杜口矣。

臣聞殺無罪,誅賢者,禍及三世。自陛下即位以來,頻行誅伐,梁、寇、孫、鄧,並見族滅,其從坐者,又非其數。李雲上書,明主所不當諱,杜眾乞死,諒以感悟聖朝,曾無赦宥,而并被殘戮,天下之人,咸知其冤。漢興以來,未有拒諫誅賢,用刑太深如今者也。

永平舊典,諸當重論皆須冬獄,先請後刑,所以重人命也。頃數十歲以來,州郡翫習,又欲避請讞之煩,輒託疾病,多死牢獄。長吏殺生自己,死者多非其罪,魂神冤結,無所歸訴,淫厲疾疫,自此而起。昔文王一妻,誕致十子,今宮女數千,未聞慶育。宜修德省刑,以廣螽斯之祚。

又七年六月十三日,河內野王山上有龍死,長可數十丈。扶風有星隕為石,聲聞三郡。夫龍形狀不一,小大無常,故周易況之大人,帝王以為符瑞。或聞河內龍死,諱以為蛇。夫龍能變化,蛇亦有神,皆不當死。昔秦之將衰,華山神操璧以授鄭客,曰「今年祖龍死」,始皇逃之,死於沙丘。王莽天鳳二年,訛言黃山宮有死龍之異,後漢誅莽,光武復興。虛言猶然,況於實邪?夫星辰麗天,猶萬國之附王者也。下將畔上,故星亦畔天。石者安類,墜者失埶。春秋五石隕宋,其後襄公為楚所執。秦之亡也,石隕東郡。今隕扶風,與先帝園陵相近,不有大喪,必有畔逆。

案春秋以來及古帝王,未有河清及學門自壞者也。臣以為河者,諸侯位也。清者屬陽,濁者屬陰。河當濁而反清者,陰欲為陽,諸侯欲為帝也。太學,天子教化之宮,其門無故自壞者,言文德將喪,教化廢也。京房易傳曰:「河水清,天下平。」今天垂異,地吐妖,人厲疫,三者並時而有河清,猶春秋麟不當見而見,孔子書之以為異也。

臣前上琅邪宮崇受干吉神書,不合明聽。臣聞布穀鳴於孟夏,蟋蟀吟於始秋,物有微而志信,人有賤而言忠。臣雖至賤,誠願賜清閒,極盡所言。

書奏不省。

十餘日,復上書曰:

臣伏見太白北入數日,復出東方,其占當有大兵,中國弱,四夷彊。臣又推步,熒惑今當出而潛,必有陰謀。皆由獄多冤結,忠臣被戮。德星所以久守執法,亦為此也。陛下宜承天意,理察冤獄,為劉暧、成档虧除罪辟,追錄李雲、杜眾等子孫。

夫天子事天不孝,則日食星鬥。比年日食於正朔,三光不明,五緯錯戾。前者宮崇所獻神書,專以奉天地順五行為本,亦有興國廣嗣之術。其文易曉,參同經典,而順帝不行,故國胤不興,孝沖、孝質頻世短祚。

臣又聞之,得主所好,自非正道,神為生虐。故周衰,諸侯以力征相尚,於是夏育、申休、宋萬、彭生、任鄙之徒生於其時。殷紂好色,妲己是出。葉公好龍,真龍游廷。今黃門常侍,天刑之人,陛下愛待,兼倍常寵,係嗣未兆,豈不為此?天官宦者星不在紫宮而在天市,明當給使主市里也。今乃反處常伯之位,實非天意。

又聞宮中立黃老、浮屠之祠。此道清虛,貴尚無為,好生惡殺,省慾去奢。今陛下嗜欲不去,殺罰過理,既乖其道,豈獲其祚哉!或言老子入夷狄為浮屠。浮屠不三宿桑下,不欲久生恩愛,精之至也。天神遺以好女,浮屠曰:「此但革囊盛血。」遂不眄之。其守一如此,乃能成道。今陛下婬女豔婦,極天下之麗,甘肥飲美,單天下之味,柰何欲如黃老乎?

書上,即召詔尚書問狀。楷曰:「臣聞古者本無宦臣,武帝末,春秋高,數游後宮,始置之耳。後稍見任,至於順帝,遂益繁熾。今陛下爵之,十倍於前,至今無繼嗣者,豈獨好之而使之然乎?」尚書上其對,詔下有司處正,尚書承旨奏曰:「其宦者之官,非近世所置。漢初張澤為大謁者,佐絳侯誅諸呂;孝文使趙談參乘,而子孫昌盛。楷不正辭理,指陳要務,而析言破律,違背經蓺,假借星宿,偽託神靈,造合私意,誣上罔事。請下司隸,正楷罪法,收送洛陽獄。」帝以楷言雖激切,然皆天文恆象之數,故不誅,猶司寇論刑。

初,順帝時,琅邪宮崇詣闕,上其師干吉於曲陽泉水上所得神書百七十卷,皆縹白素朱介青首朱目,號太平清領書。其言以陰陽五行為家,而多巫覡雜語。有司奏崇所上妖妄不經,乃收臧之。後張角頗有其書焉。

及靈帝即位,以楷書為然。太傅陳蕃舉方正,不就。鄉里宗之,每太守至,輒致禮請。中平中,與荀爽、鄭玄俱以博士徵,不至,卒于家。

論曰:古人有云:「善言天者,必有驗於人。」而張衡亦云:「天文歷數,陰陽占候,今所宜急也。」郎顗、襄楷能仰瞻俯察,參諸人事,禍福吉凶既應,引之教義亦明。此蓋道術所以有補於時,後人所當取鑒者也。然而其敝好巫,故君子不以專心焉。

贊曰:仲桓術深,蒲車屢尋。蘇竟飛書,清我舊陰。襄、郎災戒,寔由政淫。


\end{pinyinscope}