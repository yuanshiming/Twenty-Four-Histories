\article{祭祀中}

\begin{pinyinscope}
迎氣增祀六宗老子

是年初營北郊,明堂、辟雍、靈臺未用事。遷呂太后于園。上薄太后尊號曰高皇后,當配地郊高廟。語在光武紀。

北郊在雒陽城北四里,為方壇四陛。三十三年正月辛未,郊。別祀地祇,位南面西上,高皇后配,西面北上,皆在壇上,地理群神從食,皆在壇下,如元始中故事。中嶽在未,四嶽各在其方孟辰之地,中營內。海在東;四瀆河西,濟北,淮東,江南;他山川各如其方,皆在外營內。四陛醊及中外營門封神如南郊。地祇、高后用犢各一頭,五嶽共牛一頭,海、四瀆共牛一頭,群神共二頭。奏樂亦如南郊。既送神,瘞俎實于壇北。

明帝即位,永平二年正月辛未,初祀五帝於明堂,光武帝配。五帝坐位堂上,各處其方。黃帝在未,皆如南郊之位。光武帝位在青帝之南少退,西面。牲各一犢,奏樂如南郊。卒事,遂升靈臺,以望雲物。

迎時氣,五郊之兆。自永平中,以禮讖及月令有五郊迎氣服色,因采元始中故事,兆五郊于雒陽四方。中兆在未,壇皆三尺,階無等。

立春之日,迎春于東郊,祭青帝句芒。車旗服飾皆青。歌青陽,八佾舞雲翹之舞。及因賜文官太傅、司徒以下縑各有差。

立夏之日,迎夏于南郊,祭赤帝祝融。車旗服飾皆赤。歌朱明,八佾舞雲翹之舞。

先立秋十八日,迎黃靈于中兆,祭黃帝后土。車旗服飾皆黃。歌朱明,八佾舞雲翹、育命之舞。

立秋之日,迎秋于西郊,祭白帝蓐收。車旗服飾皆白。歌西皓,八佾舞育命之舞。使謁者以一特牲先祭先虞于壇,有事,天子入囿射牲,以祭宗廟,名曰貙劉。語在禮儀志。

立冬之日,迎冬于北郊,祭黑帝玄冥。車旗服飾皆黑。歌玄冥,八佾舞育命之舞。

章帝即位,元和二年正月,詔曰:「山川百神,應祀者未盡。其議增修群祀宜享祀者。」

二月,上東巡狩,將至泰山,道使使者奉一太牢祠帝堯於濟陰成陽靈臺。上至泰山,修光武山南壇兆。辛未,柴祭天地群神如故事。壬申,宗祀五帝於孝武所作汶上明堂,光武帝配,如雒陽明堂祀。癸酉,更告祀高祖、太宗、世宗、中宗、世祖、顯宗於明堂,各一太牢。卒事,遂覲東后。饗賜王侯群臣。因行郡國,幸魯,祠東海恭王,及孔子、七十二弟子。四月,還京都。庚申,告至,祠高廟、世祖,各一特牛。又為靈臺十二門作詩,各以其月祀而奏之。和帝無所增改。

安帝即位,元初六年,以尚書歐陽家說,謂六宗者,在天地四方之中,為上下四方之宗。以元始中故事,謂六宗易六子之氣日、月、雷公、風伯、山、澤者為非是。三月庚辰,初更立六宗,祀於雒陽西北戌亥之地,禮比太社也。

延光三年,上東巡狩,至泰山,柴祭,及祠汶上明堂,如元和三年故事。順帝即位,修奉常祀。

桓帝即位十八年,好神僊事。延熹八年,初使中常侍之陳國苦縣祠老子。九年,親祠老子於濯龍。文罽為壇,飾淳金釦器,設華蓋之坐,用郊天樂也。


\end{pinyinscope}