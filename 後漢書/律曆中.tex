\article{律曆中}

\begin{pinyinscope}
賈逵論曆永元論曆延光論曆

漢安論曆熹平論曆

論月食

自太初元年始用三統曆,施行百有餘年,曆稍後天,朔先曆,朔或在晦,月見。考其行,日有退無進,月有進無退。建武八年中,太僕朱浮、太中大夫許淑等數上書,言曆不正,宜當改更。時分度覺差尚微,上以天下初定,未遑考正。至永平五年,官曆署七月十六日食。待詔楊岑見時月食多先曆,即縮用筭上為日,上言「月當十五日食,官曆不中」。詔書令岑普,與官課。起七月,盡十一月,弦望凡五,官曆皆失,岑皆中。庚寅,詔令岑署弦望月食官,復令待詔張盛、景防、鮑鄴等以四分法與岑課。歲餘,盛等所中多岑六事。十二年十一月丙子,詔書令盛、防代岑署弦望月食加時。四分之術,始頗施行。是時盛、防等未能分明曆元,綜校分度,故但用其弦望而已。

先是,九年,太史待詔董萌上言曆不正,事下三公、太常知曆者雜議,訖十年四月,無能分明據者。至元和二年,太初失天益遠,日、月宿度相覺浸多,而候者皆知冬至之日日在斗二十一度,未至牽牛五度,而以為牽牛中星,從天四分日之三,晦朔弦望差天一日,宿差五度。章帝知其謬錯,以問史官,雖知不合,而不能易,故召治曆編訢、李梵等綜校其狀。二月甲寅,遂下詔曰:「朕聞古先聖王,先天而天不違,後天而奉天時。河圖曰:『赤九會昌,十世以光,十一以興。』又曰:『九名之世,帝行德,封刻政。』朕以不德,奉承大業,夙夜祗畏,不敢荒寧。予末小子,託在於數終,曷以續興,崇弘祖宗,拯濟元元?尚書琁璣鈐曰:『述堯世,放唐文。』帝命驗曰:『堯考德,顧期立象。』且三、五步驟,優劣殊軌,況乎頑陋,無以克堪,雖欲從之,末由也已。每見圖書,中心恧焉。閒者以來,政治不得,陰陽不和,災異不息,癘疫之氣,流傷於牛,農本不播。夫庶徵休咎,五事之應,咸在朕躬,信有闕矣,將何以補之?書曰:『惟先假王正厥事。』又曰:『歲二月,東巡狩,至岱宗,柴,望秩于山川。遂覲東后,協時月正日。』祖堯岱宗,同律度量,考在璣衡,以正曆象,庶乎有益。春秋保乾圖曰:『三百年斗曆改憲。』史官用太初鄧平術,有餘分一,在三百年之域,行度轉差,浸以謬錯。琁璣不正,文象不稽。冬至之日日在斗二十二度,而曆以為牽牛中星。先立春一日,則四分數之立春日也。以折獄斷大刑,於氣已迕;用望平和隨時之義,蓋亦遠矣。今改行四分,以遵於堯,以順孔聖奉天之文。冀百君子越有民,同心敬授,獲咸喜,以明予祖之遺功。」於是四分施行。而訢、梵猶以為元首十一月當先大,欲以合耦弦望,命有常日,而十九歲不得七閏,晦朔失實。行之未期,章帝復發聖思,考之經讖,使左中郎將賈逵問治曆者衛承、李崇、太尉屬梁鮪、司徒嚴勗、太子舍人徐震、鉅鹿公乘蘇統及訢、梵等十人。以為月當先小,據春秋經書朔不書晦者,朔必有明晦,不朔必在其月也。即先大,則一月再朔,後月無朔,是明不可必。梵等以為當先大,無文正驗,取欲諧耦十六日,月朓昏,晦當滅而已。又晦與合同時,不得異日。又上知訢、梵穴見,敕毋拘曆已班,天元始起之月常小。定,後年曆數遂正。永元中,復令史官以九道法候弦望,驗無有差跌。逵論集狀,後之議者,用得折衷,故詳錄焉。

逵論曰:「太初曆冬至日在牽牛初者,牽牛中星也。古黃帝、夏、殷、周、魯冬至日在建星,建星即今斗星也。太初曆斗二十六度三百八十五分,牽牛八度。案行事史官注,冬、夏至日常不及太初曆五度,冬至日在斗一十一度四分度之一。石氏星經曰:『黃道規牽牛初直斗二十度,去極二十五度。』於赤道,斗二十一度也。四分法與行事候注天度相應。尚書考靈曜『斗二十二度,無餘分,冬至在牽牛所起』。又編訢等據今日所在牽牛中星五度,於斗二十一度四分一,與考靈曜相近,即以明事。元和二年八月,詔書曰『石不可離』,令兩候,上得筭多者。太史令玄等候元和二年至永元元年,五歲中課日行及冬夏至斗一十一度四分一,合古曆建星考靈曜日所起,其星閒距度皆如石氏故事。他術以為冬至日在牽牛初者,自此遂黜也。」

逵論曰:「以太初曆考漢元盡太初元年日朔

二十三事,其十七得朔,四得晦,二得二日;新曆七得朔,十四得晦,二得三日。以太初曆考太初元年盡更始二年二十四事,十得晦;以新曆十六得朔,七得二日,一得晦。以太初曆考建武元年盡永元元年二十三事,五得朔,十八得晦;以新曆十七得朔,三得晦,三得二日。又以新曆上考春秋中有日朔者二十四事,失不中者二十三事。天道參差不齊,必有餘,餘又有長短,不可以等齊。治曆者方以七十六歲斷之,則餘分稍長,稍得一日。故易金火相革之卦象曰:『君子以治曆明時。』又曰:『湯、武革命,順乎天應乎人。』言聖人必曆象日月星辰,明數不可貫數千萬歲,其閒必改更,先距求度數,取合日月星辰所在而已。故求度數,取合日月星辰,有異世之術。太初曆不能下通於今,新曆不能上得漢元。一家曆法必在三百年之閒。故讖文曰『三百年斗曆改憲』。漢興,當用太初而不改,下至太初元年百二歲乃改。故其前有先晦一日合朔,下至成、哀,以二日為朔,故合朔多在晦,此其明效也。」

逵論曰:「臣前上傅安等用黃道度日月弦望多近

。史官一以赤道度之,不與日月同,於今曆弦望至差一日以上,輒奏以為變,至以為日卻縮退行。於黃道,自得行度,不為變。願請太史官日月宿簿及星度課,與待詔星象考校。奏可。臣謹案:前對言冬至日去極一百一十五度,夏至日去極六十七度,春秋分日去極九十一度。洪範『日月之行,則有冬夏』。五紀論『日月循黃道,南至牽牛,北至東井,率日日行一度,月行十三度十九分度七』也。今史官一以赤道為度,不與日月行同,其斗、牽牛、、輿鬼,赤道得十五,而黃道得十三度半;行東壁、奎、婁、軫、角、亢,赤道十〈七〉度,黃道八度;或月行多而日月相去反少,謂之日卻。案黃道值牽牛,出赤道南二十五度,其直東井、輿鬼,出赤道北〈二十〉五度。赤道者為中天,去極俱九十度,非日月道,而以遙準度日月,失其實行故也。以今太史官候注考元和二年九月已來月行牽牛、東井四十九事,無行十一度者;行婁、角三十七事,無行十五六度者,如安言。問典星待詔姚崇、井畢等十二人,皆曰『星圖有規法,日月實從黃道,官無其器,不知施行』。案甘露二年大司農中丞耿壽昌奏,以圖儀度日月行,考驗天運狀,日月行至牽牛、東井,日過〈一〉度,月行十五度,至婁、角,日行一度,月行十三度,赤道使然,此前世所共知也。如言黃道有驗,合天,日無前卻,弦望不差一日,比用赤道密近,宜施用。上中多臣校。」案逵論,永元四年也。至十五年七月甲辰,詔書造太史黃道銅儀,以角為十三度,亢十,氐十六,房五,心五,尾十八,箕十,斗二十四四分度之一,牽牛七,須女十一,虛十,危十六,營室十八,東壁十,奎十七,婁十二,胃十五,昴十二,畢十六,觜三,參八,東井三十,輿鬼四,柳十四,星七,張十七,翼十九,軫十八,凡三百六十五度四分度之一。冬至日在斗十九度四分度之一。史官以郭日月行,參弦望,雖密近而不為注日。儀,黃道與度轉運,難以候,是以少循其事。

逵論曰:「又今史官推合朔、弦、望、月食加時

,率多不中,在於不知月行遲疾意。永平中,詔書令故太史待詔張隆以四分法署弦、望、月食加時。隆言能用易九、六、七、八支知月行多少。今案隆所署多失。臣使隆逆推前手所署,不應,或異日,不中天乃益遠,至十餘度。梵、統以史官候注考校,月行當有遲疾,不必在牽牛、東井、婁、角之閒,又非所謂朓、側匿,乃由月所行道有遠近出入所生,率一月移故所疾處三度,九歲九道一復,凡九章,百七十一歲,復十一月合朔旦冬至,合春秋、三統九道終數,可以知合朔、弦、望、月食加時。據官注天度為分率,以其術法上考建武以來月食凡三十八事,差密近,有益,宣課試上。」

案史官舊有九道術,廢而不修。熹平中,故治曆郎梁國宗整上九道術,詔書下太史,以參舊術,相應。部太子舍人馮恂課校,恂亦復作九道術,增損其分,與整術並校,差為近。太史令颺上以恂術參弦、望。然而加時猶復先後天,遠則十餘度。

永元十四年,待詔太史霍融上言:「官漏刻率九日增減一刻,不與天相應,或時差至二刻半,不如夏曆密。」詔書下太常,令史官與融以儀校天,課度遠近。太史令舒、承、梵等對:「案官所施漏法令甲第六常符漏品,孝宣皇帝三年十二月乙酉下,建武十年二月壬午詔書施行。漏刻以日長短為數,率日南北二度四分而增減一刻。一氣俱十五日,日去極各有多少。今官漏率九日移一刻,不隨日進退。夏曆漏隨日南北為長短,密近於官漏,分明可施行。」其年十一月甲寅,詔曰:「告司徒、司空:漏所以節時分,定昏明。昏明長短,起於日去極遠近,日道周,不可以計率分,當據儀度,下參晷景。今官漏以計率分昏明,九日增減一刻,違失其實,至為疏數以耦法。太史待詔霍融上言,不與天相應。太常史官運儀下水,官漏失天者至三刻。以晷景為刻,少所違失,密近有驗。今下晷景漏刻四十八箭,立成斧官府當用者,計吏到,班予四十八箭。」文多,故魁取二十四氣日所在,并黃道去極、晷景、漏刻、昏明中星刻于下。

昔太初曆之興也,發謀於元封,啟定於天鳳,積百三十年,是非乃審。及用四分,亦於建武,施於元和,訖於永元,七十餘年,然后儀式備立,司候有準。天事幽微,若此其難也。中興以來,圖讖漏泄,而考靈曜、命曆序皆有甲寅元。其所起在四分庚申元後百一十四歲,朔差卻二日。學士修之於草澤,信向以為得正。及太初曆以後大為疾,而修之者云「百四十四歲而太歲超一表,百七十一歲當棄朔餘六十三,中餘千一百九十七,乃可常行」。自太初元年至永平十一年,百七十一,當去分而不去,故令益有疏闊。此二家常挾其術,庶幾施行,每有訟者,百寮會議,群儒騁思,論之有方,益於多聞識之,故詳錄焉。

安帝延光二年,中謁者亶誦言當用甲寅元,河南梁豐言當復用太初。尚書郎張衡、周興皆能曆,數難誦、豐,或不對,或言失誤。衡、興參案儀注者,考往校今,以為九道法最密。詔書下公卿詳議。太尉愷等上侍中施延等議:「太初過天,日一度,弦望失正,月以晦見西方,食不與天相應;元和改從四分,四分雖密於太初,復不正,皆不可用。甲寅元與天相應,合圖讖,可施行。」博士黃廣、大行令任僉議,如九道。河南尹祉、太子舍人李泓等四十人議:「即用甲寅元,當除元命苞天地開闢獲麟中百一十四歲,推閏月六直其日,或朔、晦、弦、望,二十四氣宿度不相應者非一。用九道為朔,月有比三大二小,皆疏遠。元和變曆,以應保乾圖『三百歲斗曆改憲』之文。四分曆本起圖讖,最得其正,不宜易。」愷等八十四人議,宜從太初。尚書令忠上奏:「諸從太初者,皆無他效驗,徒以世宗攘夷廓境,享國久長為辭。或云孝章改四分,災異卒甚,未有善應。臣伏惟聖王興起,各異正朔,以通三統。漢祖受命,因秦之紀,十月為年首,閏常在歲後。不稽先代,違於帝典。太宗遵修,三階以平,黃龍以至,刑犴以錯,五是以備。哀平之際,同承太初,而妖孽累仍,痾禍非一。議者不以成數相參,考真求實,而汎采妄說,歸福太初,致咎四分。太初曆眾賢所立,是非已定,永平不審,復革其弦望。四分有謬,不可施行。元和鳳鳥不當應曆而翔集。遠嘉前造,則喪其休;近譏後改,則隱其福。漏見曲論,未可為是。臣輒復重難衡、興,以為五紀論推步行度,當時比諸術為近,然猶未稽於古。及向子歆欲以合春秋,橫斷年數,損夏益周,考之表紀,差謬數百。兩曆相課,六千一百五十六歲,而太初多一日。冬至日直斗,而云在牽牛。嫔闊不可復用,昭然如此。史官所共見,非獨衡、興。前以為九道密近,今議者以為有闕,及甲寅元復多違失,皆未可取正。昔仲尼順假馬之名,以崇君之義。況天之曆數,不可任疑從虛,以非易是。」上納其言,遂改曆事。

順帝漢安二年,尚書侍郎邊韶上言:「世微於數虧,道盛於得常。數虧則物衰,得常則國昌。孝武皇帝攄發聖思,因元封七年十一月甲子朔旦冬至,乃詔太史令司馬遷、治曆鄧平等更建太初,改元易朔,行夏之正,乾鑿度八十〈一〉分之四十三為日法。設清臺之候,驗六異,課效觕密,太初為最。其後劉歆研機極深,驗之春秋,參以易道,以河圖帝覽嬉、雒書甄曜度推廣九道,百七十一歲進退六十三分,百四十四歲一超次,與天相應,少有闕謬。從太初至永平十一年,百七十〈一〉歲,進退餘分六十三,治曆者不知處之。推得十二度弦望不效,挾廢術者得竄其說。至永和二年,小終之數寖過,餘分稍增,月不用晦朔而先見。孝章皇帝以保乾圖『三百年斗曆改憲』,就用四分。以太白復樞甲子為癸亥,引天從筭,耦之目前。更以庚申為元,既無明文;託之於獲麟之歲,又不與感精符單閼之歲同。史官相代,因成習疑,少能鉤深致遠;案弦望足以知之。」詔書下三公、百官雜議。太史令虞恭、治曆宗訢等議:「建曆之本,必先立元,元正然後定日法,法定然後度周天以定分至。三者有程,則曆可成也。四分曆仲紀之元,起於孝文皇帝後元三年,歲在庚辰。上四十五歲,歲在乙未,則漢興元年也。又上二百七十五歲,歲在庚申,則孔子獲麟。二百七十六萬歲,尋之上行,復得庚申。歲歲相承,從下尋上,其執不誤。此四分曆元明文圖讖所著也。太初元年歲在丁丑,上極其元,當在庚戌,而曰丙子,言百四十四歲超一辰,凡九百九十三超,歲有空行八十二周有奇,乃得丙子。案歲所超,於天元十一月甲子朔旦冬至,日月俱超。日行一度,積三百六十五度四分度一而周天一匝,名曰歲。歲從一辰,日不得空周天,則歲無由超辰。案百七十〈一〉歲二蔀一章,小餘六十三,自然之數也。夫數出於杪曶,以成毫氂,亮氂積累,以成分寸。兩儀既定,日月始離。初行生分,積分成度。日行一度,一歲而周,故為術者,各生度法,或以九百四十,或以八十一。法有細觕,以生兩科,其歸一也。日法者,日之所行分也。日垂令明,行有常節,日法所該,通遠無已,損益毫氂,差以千里。自此言之,數無緣得有虧棄之意也。今欲飾平之失,斷法垂分,恐傷大道。以步日月行度,終數不同,四章更不得朔餘一。雖言九道去課進退,恐不足以補其闕。且課曆之法,晦朔變弦,以月食天驗,昭著莫大焉。今以去六十三分之法為曆,驗章和元年以來日變二十事,月食二十八事,與四分曆更失,定課相除,四分尚得多,而又便近。孝章皇帝曆度審正,圖儀晷漏,與天相應,不可復尚。文曜鉤曰:『高辛受命,重黎說文。唐堯即位,羲和立禪。夏后制德,昆吾列神。成周改號,萇弘分官。』運斗樞曰:『常占有經,世史所明。』洪範五紀論曰:『民閒亦有黃帝諸曆,不如史官記之明也。』自古及今,聖帝明王,莫不取言於羲和、常占之官,定精微於晷儀,正眾疑,祕藏中書,改行四分之原。及光武皇帝數下詔書,草創其端,孝明皇帝課校其實,孝章皇帝宣行其法。君更三聖,年歷數十,信而徵之,舉而行之。其元則上統開闢,其數則復古四分。宜如甲寅詔書故事。」奏可。

靈帝熹平四年,五官郎中馮光、沛相上計掾陳晃言:「曆元不正,故妖民叛寇益州,盜賊相續為。曆用甲寅為元而用庚申,圖緯無以庚為元者。近秦所用代周之元。太史治治曆中郭香、劉固意造妄說,乞與本庚申元經緯有明,受虛欺重誅。」乙卯,詔書下三府,與儒林明道者詳議,務得道真。以群臣會司徒府議。

議郎蔡邕議,以為:

曆數精微,去聖久遠,得失更迭,術術無常是。以承秦,曆用顓頊,元用乙卯。百有二歲,孝武皇帝始改正朔,曆用太初,元用丁丑,行之百八十九歲。孝章皇帝改從四分,元用庚申。今光、晃各以庚申為非,甲寅為是。案曆法,黃帝、顓頊、夏、殷、周、魯,凡六家,各自有元。光、晃所據,則殷曆元也。他元雖不明於圖讖,各家術,皆當有效於其當時。黃帝始用太初丁丑之元,有六家紛錯,爭訟是非。太史令張壽王挾甲寅元以非漢曆,雜候清臺,課在下第,卒以疏闊,連見劾奏,太初效驗,無所漏失。是則雖非圖讖之元,而有效於前者也。及用四分以來,考之行度,密於太初,是又新元效於今者也。延光元年,中謁者亶誦亦非四分庚申,上言當用命曆序甲寅元。公卿百寮參議正處,竟不施行。且三光之行,遲速進退,不必若一。術家以筭追而求之,取合於當時而已。故有古今之術。今之不能上通於古,亦猶古術之不能下通於今也。元命苞、乾鑿度皆以為開闢至獲麟二百七十六萬歲;及命曆序積獲麟至漢,起庚子蔀之二十三歲,竟己酉、戊子及丁卯蔀六十九歲,合為二百七十五歲。漢元年歲在乙未,上至獲麟則歲在庚申。推此以上,上極開闢,則不在庚申。讖雖無文,其數見存。而光、晃以為開闢至獲麟二百七十五萬九千八百八十六歲,獲麟至漢百六十二歲,轉差少一百一十四歲。云當滿足,則上違乾鑿度、元命苞,中使獲麟不得在哀公十四年,下不及命曆序獲麟漢相去四蔀年數,與奏記譜注不相應。

當今曆正月癸亥朔,光、晃以為乙丑朔。乙丑之與癸亥,無題勒款識可與眾共別者,須以弦望晦朔光魄虧滿可得而見者,考其符驗。而光、晃曆以考靈曜,二十八宿度數及冬至日所在,與今史官甘、石舊文錯異,不可考校;以今渾天圖儀檢天文,亦不合於考靈曜。光、晃誠能自依其術,更造望儀,以追天度,遠有驗於圖書,近有效於三光,可以易奪甘、石,窮服諸術者,實宜用之。難問光、晃,但言圖讖,所言不服。元和二年二月甲寅制書曰:『朕聞古先聖王,先天而天不違,後天而奉天時。史官用太初鄧平術,冬至之日,日在斗二十二度,而曆以為牽牛中星,先立春一日,則四分數之立春也,而以折獄斷大刑,於氣已迕,用望平和,蓋亦遠矣。今改行四分,以遵於堯,以順孔聖奉天之文。』是始用四分曆庚申元之詔也。深引河雒圖讖以為符驗,非史官私意獨所興構。而光、晃以為、固意造妄說,違反經文,謬之甚者。昔堯命羲和曆象日月星辰,舜協時月正日,湯、武革命,治曆明時,可謂正矣,且猶遇水遭旱,戒以『蠻夷猾夏,寇賊姦宄』。而光、晃以為陰陽不和,姦臣盜賊,皆元之咎,誠非其理。元和二年乃用庚申,至今九十二歲,而光、晃言秦所用代周之元,不知從秦來,漢三易元,不常庚申。光、晃區區信用所學,亦妄虛無造欺語之愆。至於改朔易元,往者壽王之術已課不效,亶誦之議不用,元和詔書文備義著,非群臣議者所能變易。

太尉耽、司徒隗、司空訓以邕議劾光、晃不敬,正鬼薪法。詔書勿治罪。

太初曆推月食多失。四分因太初法,以河平癸巳為元,施行五年。永元元年,天以七月後閏食,術以八月。其十二年正月十二日,蒙公乘宗紺上書言:「今月十六日月當食,而曆以二月。」至期如紺言。太史令巡上紺有益官用,除待詔。甲辰,詔書以紺法署。施行五十六歲。至本初元年,天以十二月食,曆以後年正月,於是始差。到熹平三年,二十九年之中,先曆食者十六事。常山長史劉洪上作七曜術。甲辰詔屬太史部郎中劉固、舍人馮恂等課效,復作八元術,固等作月食術,並已相參。固術與七曜術同。月食所失,皆以歲在己未當食四月,恂術以三月,官曆以五月。太史上課,到時施行中者。丁巳,詔書報可。

其四年,紺孫誠上書言:「受紺法術,當復改,今年十二月當食,而官曆以後年正月。」到期如言,拜誠為舍人。丙申,詔書聽行誠法。

光和二年歲在己未,三月、五月皆陰,太史令修、部舍人張恂等推計行度,以為三月近,四月遠。誠以四月。奏廢誠術,施用恂術。其三年,誠兄整前後上書言:「去年三月不食,當以四月。史官廢誠正術,用恂不正術。」整所上五屬太史,太史主者終不自言三月近,四月遠。食當以見為正,無遠近。詔書下太常:「其詳案注記,平議術之要,效驗虛實。」太常就耽上選侍中韓說、博士蔡較、穀城門候劉洪、右郎中陳調於太常府,覆校注記,平議難問。恂、誠各對。恂術以五千六百四十日有九百六十一食為法,而除成分,空加縣法,推建武以來,俱得三百二十七食,其十五食錯。案其官素注,天見食九十八,與兩術相應,其錯辟二千一百。誠術以百三十五月二十三食為法,乘除成月,從建康以上減四十一,建康以來減三十五,以其俱不食。恂術改易舊法,誠術中復減損,論其長短,無以相踰。各引書緯自證,文無義要,取追天而已。夫日月之術,日循黃道,月從九道。以赤道儀,日冬至去極俱一百一十五度。其入宿也,赤道在斗中十一,而黃道在斗十九。兩儀相參,日月之行,曲直有差,以生進退。故月行井、牛,十四度以上;其在角、婁,十二度以上。皆不應率不行。以是言之,則術不差不改,不驗不用。天道精微,度數難定,術法多端,曆紀非一,未驗無以知其是,未差無以知其失。失然後改之,是然後用之,此謂允執其中。今誠術未有差錯之謬,恂術未有獨中之異,以無驗改未失,是以檢將來為是者也。誠術百三十五月月二十三食,其文在書籍,學者所修,施行日久,官守其業,經緯日月,厚而未愆,信於天文,述而不作。恂久在候部,詳心善意,能揆儀度,定立術數,推前校往,亦與見食相應。然協曆正紀,欽若昊天,宜率舊章,如甲辰、丙申詔書,以見食為比。今宜施用誠術,棄放恂術,史官課之,後有效驗,乃行其法,以審術數,以順改易。耽以說等議奏聞,詔書可。恂、整、誠各復上書,恂言不當施誠術,整言不當復棄恂術。為洪議所侵,事下永安臺覆實,皆不如恂、誠等言。劾奏謾欺。詔書報,恂、誠各以二月奉贖罪,整適作左校二月。遂用洪等,施行誠術。

光和二年,萬年公乘王漢上月食注。自章和元年到今年凡九十三歲,合百九十六食;與官曆河平元年月錯,以己巳為元。事下太史令修,上言「漢所作注不與見食相應者二事,以同為異者二十九事」。尚書召穀城門候劉洪。敕曰:「前郎中馮光、司徒掾陳晃各訟曆,故議郎蔡邕共補續其志。今洪其詣修,與漢相參,推元謂分,考校月食。審己巳元密近,有師法,洪便從漢受;不能,對。」洪上言:「推元漢己巳元,則考靈曜旃蒙之歲乙卯元也,與光、晃甲寅元相經緯。於以追天作曆,校三光之步,今為疏闊。孔子緯一事見二端者,明曆興廢,隨天為節。甲寅曆於孔子時效;己巳顓頊秦所施用,漢興草創,因而不易,至元封中,迂闊不審,更用太初,應期三百改憲之節。甲寅、己巳讖雖有文,略其年數,是以學人各傳所聞,至於課校,罔得厥正。夫甲寅元天正正月甲子朔旦冬至,七曜之起,始於牛初。乙卯之元人正己巳朔旦立春,三光聚天廟五度。課兩元端,閏餘差自五十〈二〉分二之三,朔三百四,中節之餘二十九。以效信難聚,漢不解說,但言先人有書而已。以漢成注參官施行,術不同二十九事,不中見食二事。案漢習書,見己巳元,謂朝不聞,不知聖人獨有興廢之義,史官有附天密術。甲寅、己巳,前已施行,效後格而已不用。河平疏闊,史官已廢之,而漢以去事分爭,殆非其意。雖有師法,與無同。課又不近密。其說蔀數,術家所共知,無所采取。」遣漢歸鄉里。


\end{pinyinscope}