\article{郭杜孔張廉王蘇羊賈陸列傳}

\begin{pinyinscope}
郭伋字細侯,扶風茂陵人也。高祖父解,武帝時以任俠聞。父梵,為蜀郡太守。伋少有志行,哀平閒辟大司空府,三遷為漁陽都尉。王莽時為上谷大尹,遷并州牧。

更始新立,三輔連被兵寇,百姓震駭,強宗右姓各擁眾保營,莫肯先附。更始素聞伋名,徵拜左馮翊,使鎮撫百姓。世祖即位,拜雍州牧,再轉為尚書令,數納忠諫爭。

建武四年,出為中山太守。明年,彭寵滅,轉為漁陽太守。漁陽既離王莽之亂,重以彭寵之敗,民多猾惡,寇賊充斥。伋到,示以信賞,糾戮渠帥,盜賊銷散。時匈奴數抄郡界,邊境苦之。伋整勒士馬,設攻守之略,匈奴畏憚遠跡,不敢復入塞,民得安業。在職五歲,戶口增倍。後潁川盜賊群起,九年,徵拜潁川太守。召見辭謁,帝勞之曰:「賢能太守,去帝城不遠,河潤九里,冀京師并蒙福也。君雖精於追捕,而山道險阨,自鬥當一士耳,深宜慎之。」伋到郡,招懷山賊陽夏趙宏、襄城召吳等數百人,皆束手詣伋降,悉遣歸附農。因自劾專命,帝美其策,不以咎之。後宏、吳等黨與聞伋威信,遠自江南,或從幽、冀,不期俱降,駱驛不絕。

十一年,省朔方刺史屬并州。帝以盧芳據北土,乃調伋為并州牧。過京師謝恩,帝即引見,并召皇太子諸王宴語終日,賞賜車馬衣服什物。伋因言選補眾職,當簡天下賢俊,不宜專用南陽人。帝納之。伋前在并州,素結恩德,及後入界,所到縣邑,老幼相攜,逢迎道路。所過問民疾苦,聘求耆德雄俊,設几杖之禮,朝夕與參政事。

始至行部,到西河美稷,有童兒數百,各騎竹馬,道次迎拜。伋問「兒曹何自遠來」。對曰:「聞使君到,喜,故來奉迎。」伋辭謝之。及事訖,諸兒復送至郭外,問「使君何日當還」。伋謂別駕從事,計日當告之。行部既還,先期一日,伋為違信於諸兒,遂止于野亭,須期乃入。

是時朝廷多舉伋可為大司空,帝以并部尚有盧芳之儆,且匈奴未安,欲使久於其事,故不召。伋知盧芳夙賊,難卒以力制,常嚴烽候,明購賞,以結寇心。芳將隋昱遂謀脅芳降伋,芳乃亡入匈奴。

伋以老病上書乞骸骨。二十二年,徵為太中大夫,賜宅一區,及帷帳錢穀,以充其家,伋輒散與宗親九族,無所遺餘。明年卒,時年八十六。帝親臨弔,賜冢塋地。

杜詩字公君,河內汲人也。少有才能,仕郡功曹,有公平稱。更始時,辟大司馬府。建武元年,歲中三遷為侍御史,安集洛陽。時將軍蕭廣放縱兵士,暴橫民閒,百姓惶擾,詩敕曉不改,遂格殺廣,還以狀聞。世祖召見,賜以棨戟,復使之河東,誅降逆賊楊異等。詩到大陽,聞賊規欲北度,乃與長史急焚其船,部勒郡兵,將突騎趁擊,斬異等,賊遂翦滅。拜成皋令,視事三歲,舉政尤異。再遷為沛郡都尉,轉汝南都尉,所在稱治。

七年,遷南陽太守。性節儉而政治清平,以誅暴立威,善於計略,省愛民役。造作水排,鑄為農器,用力少,見功多,百姓便之。又修治陂池,廣拓土田,郡內比室殷足。時人方於召信臣,故南陽為之語曰:「前有召父,後有杜母。」

詩自以無勞,不安久居大郡,求欲降避功臣,乃上疏曰:

陛下亮成天工,克濟大業,偃兵脩文,群帥反旅,海內合和,萬世蒙福,天下幸甚。唯匈奴未譬聖德,威侮二垂,陵虐中國,邊民虛耗,不能自守,臣恐武猛之將雖勤,亦未得解甲櫜弓也。夫勤而不息亦怨,勞而不休亦怨,怨恨之師,難復責功。臣伏睹將帥之情,功臣之望,冀一休足於內郡,然後即戎出命,不敢有恨。臣愚以為「師克在和不在眾」,陛下雖垂念北邊,亦當頗泄用之。昔湯武善御眾,故無忿鷙之師。陛下起兵十有三年,將帥和睦,士卒鳧竡。今若使公卿郡守出於軍壘,則將帥自厲;士卒之復,比於宿衛,則戎士自百。何者?天下已安,各重性命,大臣以下,咸懷樂土,不讎其功而厲其用,無以勸也。陛下誠宜虛缺數郡,以俟振旅之臣,重復厚賞,加於久役之士。如此,緣邊屯戍之師,競而忘死,乘城拒塞之吏,不辭其勞,則烽火精明,守戰堅固。聖王之政,必因人心。今猥用愚薄,塞功臣之望,誠非其宜。

臣詩伏自惟忖,本以史吏一介之才,遭陛下創制大業,賢俊在外,空乏之閒,超受大恩,收養不稱,奉職無效,久竊祿位,令功臣懷慍,誠惶誠恐。八年,上書乞避功德,陛下殊恩,未許放退。臣詩蒙恩尤深,義不敢苟冒虛請,誠不勝至願,願退大郡,受小職。及臣齒壯,力能經營劇事,如使臣詩必有補益,復受大位,雖析珪授爵,所不辭也。惟陛下哀矜!

帝惜其能,遂不許之。

詩雅好推賢,數進知名士清河劉統及魯陽長董崇等。

初,禁網尚簡,但以璽書發兵,未有虎符之信,詩上疏曰:「臣聞兵者國之凶器,聖人所慎。舊制發兵,皆以虎符,其餘徵調,竹使而已。符第合會,取為大信,所以明著國命,斂持威重也。閒者發兵,但用璽書,或以詔令,如有姦人詐偽,無由知覺。愚以為軍旅尚興,賊虜未殄,徵兵郡國,宜有重慎,可立虎符,以絕姦端。昔魏之公子,威傾鄰國,猶假兵符,以解趙圍,若無如姬之仇,則其功不顯。事有煩而不可省,費而不得已,蓋謂此也。」書奏,從之。

詩身雖在外,盡心朝廷,讜言善策,隨事獻納。視事七年,政化大行。十四年,坐遣客為弟報仇,被徵,會病卒。司隸校尉鮑永上書言詩貧困無田宅,喪無所歸。詔使治喪郡邸,賻絹千匹。

孔奮字君魚,扶風茂陵人也。曾祖霸,元帝時為侍中。奮少從劉歆受春秋左氏傳,歆稱之,謂門人曰:「吾已從君魚受道矣。」

遭王莽亂,奮與老母幼弟避兵河西。建武五年,河西大將軍竇融請奮署議曹掾,守姑臧長。八年,賜爵關內侯。時天下擾亂,唯河西獨安,而姑臧稱為富邑,通貨羌胡,市日四合,每居縣者,不盈數月輒致豐積。奮在職四年,財產無所增。事母孝謹,雖為儉約,奉養極求珍膳。躬率妻子,同甘菜茹。時天下未定,士多不修節操,而奮力行清絜,為眾人所笑,或以為身處脂膏,不能以自潤,徒益苦辛耳。奮既立節,治貴仁平,太守梁統深相敬待,不以官屬禮之,常迎於大門,引入見母。

隴蜀既平,河西守令咸被徵召,財貨連轂,彌竟川澤。唯奮無資,單車就路。姑臧吏民及羌胡更相謂曰:「孔君清廉仁賢,舉縣蒙恩,如何今去,不共報德!」遂相賦斂牛馬器物千萬以上,追送數百里。奮謝之而已,一無所受。既至京師,除武都郡丞。

時隴西餘賊隗茂等夜攻府舍,殘殺郡守,賊畏奮追急,乃執其妻子,欲以為質。奮年已五十,唯有一子,終不顧望,遂窮力討之。吏民感義,莫不倍用命焉。郡多氐人,便習山谷,其大豪齊鍾留者,為群氐所信向。奮乃率厲鍾留等令要遮鈔擊,共為表裏。賊窘懼逼急,乃推奮妻子以置軍前,冀當退卻,而擊之愈厲,遂禽滅茂等,奮妻子亦為所殺。世祖下詔褒美,拜為武都太守。

奮自為府丞,已見敬重,及拜太守,舉郡莫不改操。為政明斷,甄善疾非,見有美德,愛之如親,其無行者,忿之若讎,郡中稱為清平。

弟奇,游學洛陽。奮以奇經明當仕,上病去官,守約鄉閭,卒于家。奇博通經典,作春秋左氏刪。奮晚有子嘉,官至城門校尉,作左氏說云。

張堪字君游,南陽宛人也,為郡族姓。堪早孤,讓先父餘財數百萬與兄子。年十六,受業長安,志美行厲,諸儒號曰「聖童」。

世祖微時,見堪志操,常嘉焉。及即位,中郎將來歙薦堪,召拜郎中,三遷為謁者。使送委輸縑帛,并領騎七千匹,詣大司馬吳漢伐公孫述,在道追拜蜀郡太守。時漢軍餘七日糧,陰具船欲遁去。堪聞之,馳往見漢,說述必敗,不宜退師之策。漢從之,乃示弱挑敵,述果自出,戰死城下。成都既拔,堪先入據其城,撿閱庫藏,收其珍寶,悉條列上言,秋毫無私。慰撫吏民,蜀人大悅。

在郡二年,徵拜騎都尉,後領票騎將軍杜茂營,擊破匈奴於高柳,拜漁陽太守。捕擊姦猾,賞罰必信,吏民皆樂為用。匈奴嘗以萬騎入漁陽,堪率數千騎奔擊,大破之,郡界以靜。乃於狐奴開稻田八千餘頃,勸民耕種,以致殷富。百姓歌曰:「桑無附枝,麥穗兩岐。張君為政,樂不可支。」視事八年,匈奴不敢犯塞。

帝嘗召見諸郡計吏,問其風土及前後守令能否。蜀郡計掾樊顯進曰:「漁陽太守張堪昔在蜀,其仁以惠下,威能討姦。前公孫述破時,珍寶山積,捲握之物,足富十世,而堪去職之日,乘折轅車,布被囊而已。」帝聞,良久歎息,拜顯為魚復長。方徵堪,會病卒,帝深悼惜之,下詔褒揚,賜帛百匹。

廉范字叔度,京兆杜陵人,趙將廉頗之後也。漢興,以廉氏豪宗,自苦陘徙焉。世為邊郡守,或葬隴西襄武,故因仕焉。曾祖父褒,成哀閒為右將軍,祖父丹,王莽時為大司馬庸部牧,皆有名前世。范父遭喪亂,客死於蜀漢,范遂流寓西州。西州平,歸鄉里。年十五,辭母西迎父喪。蜀郡太守張穆,丹之故吏,乃重資送范,范無所受,與客步負喪歸葭萌。載船觸石破沒,范抱持棺柩,遂俱沈溺。眾傷其義,鉤求得之,療救僅免於死。穆聞,復馳遣使持前資物追范,范又固辭。歸葬服竟,詣京師受業,事博士薛漢。京兆、隴西二郡更請召,皆不應。永平初,隴西太守鄧融備禮謁范為功曹,會融為州所舉案,范知事譴難解,欲以權相濟,乃託病求去,融不達其意,大恨之。范於是東至洛陽,變名姓,求代廷尉獄卒。居無幾,融果徵下獄,范遂得衛侍左右,盡心勸勞。融怪其貌類范而殊不意,乃謂曰:「卿何似我故功曹邪?」范訶之曰:「君困厄瞀亂邪!」語遂絕。融繫出困病,范隨而養視,及死,竟不言,身自將車送喪致南陽,葬畢乃去。

後辟公府,會薛漢坐楚王事誅,故人門生莫敢視,范獨往收斂之,吏以聞,顯宗大怒,召范入,詰責曰:「薛漢與楚王同謀,交亂天下,范公府掾,不與朝廷同心,而反收斂罪人,何也?」范叩頭曰:「臣無狀愚戇,以為漢等皆已伏誅,不勝師資之情,罪當萬坐。」帝怒稍解,問范曰:「卿廉頗後邪?與右將軍褒、大司馬丹有親屬乎?」范對曰:「褒,臣之曾祖;丹,臣之祖也。」帝曰:「怪卿志膽敢爾!」因貰之。由是顯名。

舉茂才,數月,再遷為雲中太守。會匈奴大入塞,烽火日通。故事,虜人過五千人,移書傍郡。吏欲傳檄求救,范不聽,自率士卒拒之。虜眾盛而范兵不敵。會日暮,令軍士各交縛兩炬,三頭奋火,營中星列。虜遙望火多,謂漢兵救至,大驚。待旦將退,范乃令軍中蓐食,晨往赴之,斬首數百級,虜自相轔藉,死者千餘人,由此不敢復向雲中。

後頻歷武威、武都二郡太守,隨俗化導,各得治宜。建初中,遷蜀郡太守,其俗尚文辯,好相持短長,范每厲以淳厚,不受偷薄之說。成都民物豐盛,邑宇逼側,舊制禁民夜作,以防火災,而更相隱蔽,燒者日屬。范乃毀削先令,但嚴使儲水而已。百姓為便,乃歌之曰:「廉叔度,來何暮?不禁火,民安作。平生無襦今五恊。」在蜀數年,坐法免歸鄉里。范世在邊,廣田地,積財粟,悉以賑宗族朋友。

肅宗崩,范奔赴敬陵。時廬江郡掾嚴麟奉章弔國,俱會於路。麟乘小車,塗深馬死,不能自進,范見而愍然,命從騎下馬與之,不告而去。麟事畢,不知馬所歸,乃緣蹤訪之。或謂麟曰:「故蜀郡太守廉叔度,好周人窮急,今奔國喪,獨當是耳。」麟亦素聞范名,以為然,即牽馬造門,謝而歸之。世伏其好義,然依倚大將軍竇憲,以此為譏。卒於家。

初,范與洛慶鴻為刎頸交,時人稱曰:「前有管鮑,後有慶廉。」鴻慷慨有義節,位至琅邪、會稽二郡太守,所在有異跡。

論曰:張堪、廉范皆以氣俠立名,觀其振危急,赴險阨,有足壯者。堪之臨財,范之忘施,亦足以信意而感物矣。若夫高祖之召欒布,明帝之引廉范,加怒以發其志,就戮更延其寵,聞義能徙,誠君道所尚,然情理之樞,亦有開塞之感焉。

王堂字敬伯,廣漢郪人也。初舉光祿茂才,遷穀城令,治有名跡。永初中,西羌寇巴郡,為民患,詔書遣中郎將尹就攻討,連年不剋。三府舉堂治劇,拜巴郡太守。堂馳兵赴賊,斬虜千餘級,巴、庸清靜,吏民生為立祠。刺史張喬表其治能,遷右扶風。

安帝西巡,阿母王聖、中常侍江京等並請屬於堂,堂不為用。掾吏固諫之,堂曰:「吾蒙國恩,豈可為權寵阿意,以死守之!」即日遣家屬歸,閉閤上病。果有誣奏堂者,會帝崩,京等悉誅,堂以守正見稱。永建二年,徵入為將作大匠。四年,坐公事左轉議郎。復拜魯相,政存簡一,至數年無辭訟。遷汝南太守,搜才禮士,不苟自專,乃教掾吏曰:「古人勞於求賢,逸於任使,故能化清於上,事緝於下。其憲章朝右,簡覈才職,委功曹陳蕃。匡政理務,拾遺補闕,任主簿應嗣。庶循名責實,察言觀效焉。」自是委誠求當,不復妄有辭教,郡內稱治。時大將軍梁商及尚書令袁湯,以求屬不行,並恨之。後廬江賊迸入弋陽界,堂勒兵追討,即便奔散,而商、湯猶因此風州奏堂在任無警,免歸家。

年八十六卒。遺令薄斂,瓦棺以葬。子稚,清行不仕。曾孫商,益州牧劉焉以為蜀郡太守,有治聲。

蘇章字孺文,扶風平陵人也。八世祖建,武帝時為右將軍。祖父純,字桓公,有高名,性強切而持毀譽,士友咸憚之,至乃相謂曰:「見蘇桓公,患其教責人,不見,又思之。」三輔號為「大人」。永平中,為奉車都尉竇固軍,出擊北匈奴、車師有功,封中陵鄉侯,官至南陽太守。

章少博學,能屬文。安帝時,舉賢良方正,對策高第,為議郎。數陳得失,其言甚直。出為武原令,時歲飢,輒開倉廩,活三千餘戶。順帝時,遷冀州刺史。故人為清河太守,章行部案其姦臧。乃請太守,為設酒肴,陳平生之好甚歡。太守喜曰:「人皆有一天,我獨有二天。」章曰:「今夕蘇孺文與故人飲者,私恩也;明日冀州刺史案事者,公法也。」遂舉正其罪。州境知章無私,望風畏肅。換為并州刺吏,以摧折權豪,忤旨,坐免。隱身鄉里,不交當世。後徵為河南尹,不就。時天下日敝,民多悲苦,論者舉章有幹國才,朝廷不能復用,卒于家。兄曾孫不韋。

不韋字公先。父謙,初為郡督郵。時魏郡李暠為美陽令,與中常侍具瑗交通,貪暴為民患,前後監司畏其埶援,莫敢糾問。及謙至,部案得其臧,論輸左校。謙累遷至金城太守,去郡歸鄉里。漢法,免罷守令,自非詔徵,不得妄到京師。而謙後私至洛陽,時暠為司隸校尉,收謙詰掠,死獄中,暠又因刑其屍,以報昔怨。

不韋時年十八,徵詣公車,會謙見殺,不韋載喪歸鄉里,瘞而不葬,仰天嘆曰:「伍子胥獨何人也!」乃藏母於武都山中,遂變名姓,盡以家財募劍客,邀暠於諸陵閒,不剋。會暠遷大司農,時右校芻廥在寺北垣下,不韋與親從兄弟潛入廥中,夜則鑿地,晝則逃伏。如此經月,遂得傍達暠之寢室,出其床下。值暠在廁,因殺其妾并及小兒,留書而去。暠大驚懼,乃布棘於室,以板籍地,一夕九徙,雖家人莫知其處。每出,輒劍戟隨身,壯士自衛。不韋知暠有備,乃日夜飛馳,徑到魏郡,掘其父阜冢,斷取阜頭,以祭父墳,又標之於市曰「李君遷父頭」。暠匿不敢言,而自上退位,歸鄉里,私掩塞冢槨。捕求不韋,歷歲不能得,憤恚感傷,發病歐血死。

不韋後遇赦還家,乃始改葬,行喪。士大夫多譏其發掘冢墓,歸罪枯骨,不合古義,唯任城何休方之伍員。太原郭林宗聞而論之曰:「子胥雖云逃命,而見用強吳,憑闔廬之威,因輕悍之眾,雪怨舊郢,曾不終朝,而但鞭墓戮屍,以舒其憤,竟無手刃後主之報。豈如蘇子單特孑立,靡因靡資,強讎豪援,據位九卿,城闕天阻,宮府幽絕,埃塵所不能過,霧露所不能沾。不韋毀身燋慮,出於百死,冒觸嚴禁,陷族禍門,雖不獲逞,為報己深。況復分骸斷首,以毒生者,使暠懷忿結,不得其命,猶假手神靈以斃之也。力唯匹夫,功隆千乘,比之於員,不以優乎?」議者於是貴之。

後太傅陳蕃辟,不應,為郡五官掾。初,弘農張奐睦於蘇氏,而武威段熲與暠素善,後奐熲有隙。及熲為司隸,以禮辟不韋,不韋懼之,稱病不詣。熲既積憤於奐,因發怒,乃追咎不韋前報暠事,以為暠表治謙事,被報見誅,君命天也,而不韋仇之。又令長安男子告不韋多將賓客奪舅財物,遂使從事張賢等就家殺之。乃先以鴆與賢父曰:「若賢不得不韋,便可飲此。」賢到扶風,郡守使不韋奉謁迎賢,即時收執,并其一門六十餘人盡誅滅之,諸蘇以是衰破。及段熲為陽球所誅,天下以為蘇氏之報焉。

羊續字興祖,太山平陽人也。其先七世二千石卿校。祖父侵,安帝時司隸校尉。父儒,桓帝時為太常。

續以忠臣子孫拜郎中,去官後,辟大將軍竇武府。及武敗,坐黨事,禁錮十餘年,幽居守靜。及黨禁解,復辟太尉府,四遷為廬江太守。後揚州黃巾賊攻舒,焚燒城郭,續發縣中男子二十以上,皆持兵勒陳,其小弱者,悉使負水灌火,會集數萬人,并埶力戰,大破之,郡界平。後安風賊戴風等作亂,續復擊破之,斬首三千餘級,生獲渠帥,其餘黨輩原為平民,賦與佃器,使就農業。

中平三年,江夏兵趙慈反叛,殺南陽太守秦頡,攻沒六縣,拜續為南陽太守。當入郡界,乃羸服閒行,侍童子一人,觀歷縣邑,採問風謠,然後乃進。其令長貪絜,吏民良猾,悉逆知其狀,郡內驚竦,莫不震懾。乃發兵與荊州刺史王敏共擊慈,斬之,獲首五千餘級。屬縣餘賊並詣續降,續為上言,宥其枝附。賊既清平,乃班宣政令,候民病利,百姓歡服。時權豪之家多尚奢麗,續深疾之,常敝衣薄食,車馬羸敗。府丞嘗獻其生魚,續受而懸於庭;丞後又進之,續乃出前所懸者以杜其意。續妻後與子祕俱往郡舍,續閉門不內,妻自將祕行,其資藏唯有布衾、敝袛裯,鹽、麥數斛而已,顧敕祕曰:「吾自奉若此,何以資爾母乎?」使與母俱歸。

六年,靈帝欲以續為太尉。時拜三公者,皆輸東園禮錢千萬,令中使督之,名為「左騶」。其所之往,輒迎致禮敬,厚加贈賂。續乃坐使人於單席,舉縕袍以示之,曰:「臣之所資,唯斯而已。」左驂白之,帝不悅,以此不登公位。而徵為太常,未及行,會病卒,時年四十八。遺言薄斂,不受賵遺。舊典,二千石卒官賻百萬,府丞焦儉遵續先意,一無所受。詔書褒美,敕太山太守以府賻錢賜續家云。

賈琮字孟堅,東郡聊城人也。舉孝廉,再遷為京兆令,有政理跡。

舊交阯土多珍產,明璣、翠羽、犀、象、玳瑁、異香、美木之屬,莫不自出。前後刺史率多無清行,上承權貴,下積私賂,財計盈給,輒復求見遷代,故吏民怨叛。中平元年,交阯屯兵反,執刺史及合浦太守,自稱「柱天將軍」。靈帝特敕三府精選能吏,有司舉琮為交阯刺史。琮到部,訊其反狀,咸言賦斂過重,百姓莫不空單,京師遙遠,告冤無所,民不聊生自活,故聚為盜賊。琮即移書告示,各使安其資業,招撫荒散,蠲復傜役,誅斬渠帥為大害者,簡選良吏試守諸縣,歲閒蕩定,百姓以安。巷路為之歌曰:「賈父來晚,使我先反;今見清平,吏不敢飯。」在事三年,為十三州最,徵拜議郎。

時黃巾新破,兵凶之後,郡縣重斂,因緣生姦。詔書沙汰刺史、二千石,更選清能吏,乃以琮為冀州刺史。舊典,傳車驂駕,垂赤帷裳,迎於州界。及琮之部,升車言曰:「刺史當遠視廣聽,糾察美惡,何有反垂帷裳以自掩塞乎?」乃命御者褰之。百城聞風,自然竦震。其諸臧過者,望風解印綬去,唯癭陶長濟陰董昭、觀津長梁國黃就當官待琮,於是州界翕然。

靈帝崩,大將軍何進表琮為度遼將軍,卒於官。

陸康字季寧,吳郡吳人也。祖父續,在獨行傳。父褒,有志操,連徵不至。

康少仕郡,以義烈稱,刺史臧旻舉為茂才,除高成令。縣在邊垂,舊制,令戶一人具弓弩以備不虞,不得行來。長吏新到,輒發民繕修城郭。康至,皆罷遣,百姓大悅。以恩信為治,寇盜亦息,州郡表上其狀。光和元年,遷武陵太守,轉守桂陽、樂安二郡,所在稱之。

時靈帝欲鑄銅人,而國用不足,乃詔調民田,畝斂十錢。而比水旱傷稼,百姓貧苦。康上疏諫曰:「臣聞先王治世,貴在愛民。省傜輕賦,以寧天下,除煩就約,以崇簡易,故萬姓從化,靈物應德。末世衰主,窮奢極侈,造作無端,興制非一,勞割自下,以從苟欲,故黎民吁嗟,陰陽感動。陛下聖德承天,當隆盛化,而卒被詔書,畝斂田錢,鑄作銅人,伏讀惆悵,悼心失圖。夫十一而稅,周謂之徹。徹者通也,言其法度可通萬世而行也。故魯宣稅畝,而蝝災自生;哀公增賦,而孔子非之。豈有聚奪民物,以營無用之銅人;捐捨聖戒,自蹈亡王之法哉!傳曰:『君舉必書,書而不法,後世何述焉?』陛下宜留神省察,改敝從善,以塞兆民怨恨之望。」書奏,內倖因此譖康援引亡國,以譬聖明,大不敬,檻車徵詣廷尉。侍御史劉岱典考其事,岱為表陳解釋,免歸田里。復徵拜議郎。

會廬江賊黃穰等與江夏蠻連結十餘萬人,攻沒四縣,拜康廬江太守。康申明賞罰,擊破穰等,餘黨悉降。帝嘉其功,拜康孫尚為郎中。獻帝即位,天下大亂,康蒙險遣孝廉計吏奉貢朝廷,詔書策勞,加忠義將軍,秩中二千石。時袁術屯兵壽春,部曲飢餓,遣使求委輸兵甲。康以其叛逆,閉門不通,內修戰備,將以禦之。術大怒,遣其將孫策攻康,圍城數重。康固守,吏士有先受休假者,皆遁伏還赴,暮夜緣城而入。受敵二年,城陷。月餘,發病卒,年七十。宗族百餘人,遭離飢厄,死者將半。朝廷愍其守節,拜子雋為郎中。

少子績,仕吳為鬱林太守,博學善政,見稱當時。幼年曾謁袁術,懷橘墯地者也,有名稱。

贊曰:伋牧朔藩,信立童昏。詩守南楚,民作謠言。奮馳單乘,堪駕毀轅。范得其朋,堂任良肱。二蘇勁烈,羊、賈廉能。季寧拒策,城隕衝輣。


\end{pinyinscope}