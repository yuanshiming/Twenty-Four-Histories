\article{郡國四}

\begin{pinyinscope}
濟南國

十城,戶七萬八千五百四十四,口四十五萬三千三百八。

東平陵

有鐵。有譚城。有天山。著於陵

臺

菅

有賴亭。土鼓梁鄒鄒平東朝陽歷城有鐵。有巨里聚。

平原郡

九城,戶十五萬五千五百八十八,口百萬二千六百五十八。

平原

高唐

濕水出。般鬲

侯國。夏時有鬲君,滅浞立少康。祝阿春秋時曰祝柯。有野井亭。樂陵濕陰安德侯國。厭次本富平,明帝更名。

樂安國

九

城,戶七萬四千四百,口四十二萬四千七十五。

臨濟

本狄,安帝更名。千乘高菀樂安博昌有薄姑城。有貝中聚。有時水。蓼城侯國。利故屬齊。益

侯國,故屬北海。壽光故屬北海。有灌亭。

北海國

十八城,戶十五萬八千六百四十一,口八十五萬三千六百四。

劇

有紀亭,古紀國。營陵平壽有斟城。有寒亭,古寒國,浞封此。都昌安丘有渠丘亭。淳于永元九年復。有密鄉。平昌侯國,故屬琅邪。有蔞鄉。朱虛侯國,故屬琅邪,永初元年屬。東安平故屬菑川。六國時曰安平。有酅亭。高密侯國。昌安侯國,安帝復。夷安侯國,安帝復。膠東侯國。即墨侯國。有棠鄉。壯武安帝復。下密安帝復。拒觀陽

東萊郡

十三城,戶十萬四千二百九十七,口四十八萬四千三百九十三。

黃

牟平

惤

侯國。曲成侯國。掖侯國。有過鄉。當利侯國。東牟侯國。昌陽盧鄉

長廣

故屬琅邪。黔陬侯國,故屬琅邪。有介亭。葛盧有尤涉亭。不期侯國,故屬琅邪。

齊國

六城,戶六萬四千四百一十五,口四十九萬一千七百六十五。

臨菑

本齊,刺史治。西安有棘里亭。有蘧丘里,古渠丘。昌國臨朐有三亭,古郱邑。廣般陽故屬濟南。

右青州刺史部,郡、國六,縣六十五。南陽郡

三十七城,戶五十二萬八千五百五十一,口二百四十三萬九千六百一十八。

宛

。本申伯國。有南就聚。有瓜里津。有夕陽聚。有東武亭。冠軍邑。葉有長山,曰方城。有卷城。新野

有東鄉,故新都。有黃郵聚。章陵故舂陵,世祖更名。有上唐鄉。西鄂雉

魯陽

有魯山。有牛蘭累亭。犨堵陽博望舞陰邑。比陽復陽侯國。有杏聚。平氏桐柏大復山,淮水出。有宜秋聚。棘陽

有藍鄉。有黃淳聚。湖陽邑。隨西有斷蛇丘。育陽邑。有小長安。有東陽聚。涅陽陰酇鄧

有鄾聚。山都侯國。酈侯國。穰

朝陽

蔡

陽

侯國。安眾侯國。筑陽侯國。有涉都鄉。武當有和成聚。順陽侯國,故博山。有須聚。成都襄鄉南鄉

丹水

故屬弘農。有章密鄉。有三戶亭。析故屬弘農,故楚白羽邑。有武關,在縣西。有豐鄉城。

南郡

十七城,戶十六萬二千五百七十,口七十四萬七千六百四。

江陵

有津鄉。巫西有白帝城。秭歸本歸國。中盧侯國。編有藍口聚。當陽華容侯國。雲夢澤在南。襄陽有阿頭山。邔

侯國。有犁丘城。宜城侯國。鄀侯國,永平元年復。臨沮侯國。有荊山。枝江侯國。本羅國。有丹陽聚。夷道夷陵有荊門,虎牙山。州陵很山故屬武陵。

江夏郡

十四城,戶五萬八千四百三十四,口二十六萬五千四百六十四。

西陵

西陽

軑

侯國。鄳竟陵侯國。有鄖鄉。立章山,本內方。雲杜沙羡邾

下雉

蘄春

侯國

。鄂平春侯國。南新市侯國。安陸

零陵郡

十三城,戶二十一萬二千二百八十四,口百萬一千五百七十八。

泉陵零陵

陽朔山,湘水出。營道南有九疑山。營浦泠道洮陽都梁有路山。夫夷侯國故屬長沙。始安侯國。重安侯國,故鍾武,永建三年更名。湘鄉昭陽侯國。烝陽侯國,故屬長沙。

桂陽郡

十一城,戶十三萬五千二十九,口五十萬一千四百三。

郴

有客嶺山。便

耒陽

有鐵。陰山

南平

臨武

桂

陽含洭

湞陽

有苲領山。曲江漢寧永和元年置。

武陵郡

十二城,戶四萬六千六百七十二,口二十五萬九百一十三。

臨沅

漢壽

故

索,陽嘉三年更名,刺史治。孱陵

零陽

充

沅陵

先有壺頭山。辰陽酉陽遷陵

鐔成

沅南

建武二十六

年置。作唐

長沙郡

十三城,戶二十五萬五千八百五十四,口百五萬九千三百七十二。

臨湘

攸

荼

陵

安城

酃

湘南

侯國。衡山在東南。連道昭陵益陽下雋羅

醴陵

容陵

右荊州刺史部,郡七,縣、邑、侯國百一十七。

九江郡

十四城,戶八萬九千四百三十六,口四十三萬二千四百二十六。

陰陵

壽春

浚遒

成德

西曲陽

合肥

侯國。歷陽侯國,刺史治。當塗有馬丘聚,徐鳳反於此。全椒鍾離侯國。阜陵下蔡故屬沛。平阿故屬沛。有塗山。義成故屬沛。

丹陽郡

十六城,戶十三萬六千五百一十八,口六十三萬五百四十五。

宛陵

溧陽

丹

陽

故鄣

於潛

涇

歙

黝

陵陽

蕪湖

中江在西。秣陵南有牛渚。湖熟侯國。句容江乘春穀石城

廬江郡

十四城,戶十萬一千三百九十二,口四十二萬四千六百八十三。

舒

有桐鄉。雩婁侯國。尋陽南有九江,東合為大江。潛臨湖侯國。龍舒侯國。襄安晥有鐵。居巢

侯國。六安國。蓼

侯國。安豐有大別山。陽泉侯國。安風侯國。

會稽郡

十四城

,戶十二萬三千九十,口四十八萬一千一百九十六。

山陰

會稽山在南,上有禹冢。有浙江。鄮烏傷諸暨餘暨太末上虞

剡

餘姚

句

章

鄞

章安

故

治,閩越地,光武更名。永寧永和三年以章安縣東甌鄉為縣。東部

侯國。

吳郡

十三城,戶十六萬四千一百六十四,口七十萬七百八十二。

吳

本國。震澤在西,後名具區澤。海鹽烏程餘杭毗陵季札所居。北江在北。丹徒

曲阿

由拳

安

富春

陽

羨

邑。無錫侯國。婁

豫章郡

二十一城,戶四十萬六千四百九十六,口百六十六萬八千九百六。

南昌

建城

新淦

宜春

廬

陵

贛

有豫章水。雩都南野有臺領山。南城鄱陽有鄱水。黃金采。歷陵有傅昜山。餘汗鄡陽

彭澤

彭蠡澤在西。柴桑

艾

海昏

侯國。平都侯國,故安平。石陽臨汝永元八年置。建昌永元十六年分海昏置。

右揚州刺史部,郡六,縣、邑、侯國九十二。


\end{pinyinscope}