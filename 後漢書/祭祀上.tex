\article{祭祀上}

\begin{pinyinscope}
光武即位告天郊封禪

祭祀之道,自生民以來則有之矣。豺獺知祭祀,而況人乎!故人知之至於念想,猶豺獺之自然也,顧古質略而後文飾耳。自古以來王公所為群祀,至於王莽,漢書郊祀志既著矣,故今但列自中興以來所修用者,以為祭祀志。

建武元年,光武即位于鄗,為壇營於鄗之陽。祭告天地,采用元始中郊祭故事。六宗群神皆從,未以祖配。天地共犢,餘牲尚約。其文曰:「皇天上帝,后土神祇,睠顧降命,屬秀黎元,為民父母,秀不敢當。群下百僚,不謀同辭。咸曰王莽篡弒竊位,秀發憤興義兵,破王邑百萬眾於昆陽,誅王郎、銅馬、赤眉、青犢賊,平定天下,海內蒙恩,上當天心,下為元元所歸。讖記曰:『劉秀發兵捕不道,卯金修德為天子。』秀猶固辭,至于再,至于三。群下曰:『皇天大命,不可稽留。』敢不敬承。」

二年正月,初制郊兆於雒陽城南七里,依鄗。采元始中故事。為圓壇八陛,中又為重壇,天地位其上,皆南鄉,西上。其外壇上為五帝位。青帝位在甲寅之地,赤帝位在丙巳之地,黃帝位在丁未之地,白帝位在庚申之地,黑帝位在壬亥之地。其外為壝,重營皆紫,以像紫宮;有四通道以為門。日月在中營內南道,日在東,月在西,北斗在北道之西,皆別位,不在群神列中。八陛,陛五十八醊,合四百六十四醊。五帝陛郭,帝七十二醊,合三百六十醊。中營四門,門五十四神,合二百一十六神。外營四門,門百八神,合四百三十二神。皆背營內鄉。中營四門,門封神四,外營四門,門封神四,合三十二神。凡千五百一十四神。營即壝也。封,封土築也。背中營神,五星也,及中宮宿五官神及五嶽之屬也。背外營神,二十八宿外宮星,雷公、先農、風伯、雨師、四海、四瀆、名山、大川之屬也。

至七年五月,詔三公曰:「漢當郊堯。其與卿大夫、博士議。」時侍御史杜林上疏,以為「漢起不因緣堯,與殷周異宜,而舊制以高帝配。方軍師在外,且可如元年郊祀故事」。上從之。語在林傳。

隴、蜀平後,乃增廣郊祀,高帝配食,位在中壇上,西面北上。天、地、高帝、黃帝各用犢一頭,青帝、赤帝共用犢一頭,白帝、黑帝共用犢一頭,凡用犢六頭。日、月、北斗共用牛一頭,四營群神共用牛四頭,凡用牛五頭。凡樂奏青陽、朱明、西皓、玄冥,及雲翹、育命舞。中營四門,門用席十八枚,外營四門,門用席三十六枚,凡用席二百一十六枚,皆莞簟,率一席三神。日、月、北斗無陛郭醊。既送神,芜俎實於壇南巳地。

建武三十年二月,群臣上言,即位三十年,宜封禪泰山。詔書曰:「即位三十年,百姓怨氣滿腹,吾誰欺,欺天乎?曾謂泰山不如林放,何事汙七十二代之編錄!桓公欲封,管仲非之。若郡縣遠遣吏上壽,盛稱虛美,必髡,兼令屯田。」從此群臣不敢復言。三月,上幸魯,過泰山,告太守以上過故,承詔祭山及梁父。時虎賁中郎將梁松等議:「記曰『齊將有事泰山,先有事配林』,蓋諸侯之禮也。河獄視公侯,王者祭焉。宜無即事之漸,不祭配林。」

三十二年正月,上齋,夜讀河圖會昌符,曰「赤劉之九,會命岱宗。不慎克用,何益於承。誠善用之,姦偽不萌」。感此文,乃詔松等復案索河雒讖文言九世封禪事者。松等列奏,乃許焉。

初,孝武帝欲求神仙,以扶方者言黃帝由封禪而後僊,於是欲封禪。封禪不常,時人莫知。元封元年,上以方士言作封禪器,以示群儒,多言不合古,於是罷諸儒不用。三月,上東上泰山,乃上石立之泰山顛。遂東巡海上,求僊人,無所見而還。四月,封泰山。恐所施用非是,乃祕其事。語在漢書郊祀志。

上許梁松等奏,乃求元封時封禪故事,議封禪所施用。有司奏當用方石再累置壇中,皆方五尺,厚一尺,用玉牒書藏方石。牒厚五寸,長尺三寸,廣五寸,有玉檢。又用石檢十枚,列於石傍,東西各三,南北各二,皆長三尺,廣一尺,厚七寸。檢中刻三處,深四寸,方五寸,有蓋。檢用金鏤五周,以水銀和金以為泥。王璽一方寸二分,一枚方五寸。方石四角又有距石,皆再累。枚長一丈,厚一尺,廣二尺,皆在圓壇上。其下用距石十八枚,皆高三尺,厚一尺,廣二尺,如小碑,環壇立之,去壇三步。距石下皆有石跗,入地四尺。又用石碑,高九尺,廣三尺五寸,厚尺二寸,立壇丙地,去壇三丈以上,以刻書。上以用石功難,又欲及二月封,故詔松欲因故封石空檢,更加封而已。松上疏爭之,以為「登封之禮,告功皇天,垂後無窮,以為萬民也。承天之敬,尤宜章明。奉圖書之瑞,尤宜顯著。今因舊封,竄寄玉牒故石下,恐非重命之義。受命中興,宜當特異,以明天意」。遂使泰山郡及魯趣石工,宜取完青石,無必五色。時以印工不能刻玉牒,欲用丹漆書之;會求得能刻玉者,遂書。書祕刻方石中,合容玉牒。

二月,上至奉高,遣侍御史與蘭臺令史,將工先上山刻石。文曰:「維建武三十有二年二月,皇帝東巡狩,至于岱宗,柴,望秩於山川,班于群神,遂覲東后。從臣太尉熹、行司徒事特進高密侯禹等。漢賓二王之後在位。孔子之後褒成侯,序在東后,蕃王十二,咸來助祭。河圖赤伏符曰:『劉秀發兵捕不道,四夷雲集龍鬥野,四七之際火為主。』河圖會昌符曰:『赤帝九世,巡省得中,治平則封,誠合帝道孔矩,則天文靈出,地祇瑞興。帝劉之九,會命岱宗,誠善用之,姦偽不萌。赤漢德興,九世會昌,巡岱皆當。天地扶九,崇經之常。漢大興之,道在九世之主。封于泰山,刻石著紀,禪于梁父,退省考五。』河圖合古篇曰:『帝劉之秀,九名之世,帝行德,封刻政。』河圖提劉予曰:『九世之帝,方明聖,持衡拒,九州平,天下予。』雒書甄曜度曰:『赤三德,昌九世,會修符,合帝際,勉刻封。』孝經鉤命決曰:『予誰行,赤劉用帝,三建孝,九會修,專茲竭行封岱青。』河雒命后,經讖所傳。昔在帝堯,聰明密微,讓與舜庶,後裔握機。王莽以舅后之家,三司鼎足冢宰之權勢,依託周公、霍光輔幼歸政之義,遂以篡叛,僭號自立。宗廟墮壞,社稷喪亡,不得血食,十有八年。楊、徐、青三州首亂,兵革橫行,延及荊州,豪傑并兼,百里屯聚,往往僭號。北夷作寇,千里無煙,無雞鳴狗吠之聲。皇天睠顧皇帝,以匹庶受命中興,年二十八載興兵,起是以中次誅討,十有餘年,罪人則斯得。黎庶得居爾田,安爾宅。書同文,車同軌,人同倫。舟輿所通,人跡所至,靡不貢職。建明堂,立辟雍,起靈臺,設庠序。同律、度、量、衡。修五禮,五玉,三帛,二牲,一死,贄。吏各修職,復于舊典。在位三十有二年,年六十二。乾乾日昃,不敢荒寧,涉危歷險,親巡黎元,恭肅神祇,惠恤耆老,理庶遵古,聰允明恕。皇帝唯慎河圖、雒書正文,是月辛卯,柴,登封泰山。甲午,禪于梁陰。以承靈瑞,以為兆民,永茲一宇,垂于後昆。百寮從臣,郡守師尹,咸蒙祉福,永永無極。秦相李斯燔詩書,樂崩禮壞。建武元年已前,文書散亡,舊典不具,不能明經文,以章句細微相況八十一卷,明者為驗,又其十卷,皆不昭晢。子貢欲去告朔之餼羊,子曰:『賜也,爾愛其羊,我愛其禮。』後有聖人,正失誤,刻石記。」

二十二日辛卯晨,燎祭天於泰山下南方,群神皆從,用樂如南郊。諸王、王者後二公、孔子後褒成君,皆助祭位事也。事畢,將升封。或曰:「泰山雖已從食於柴祭,今親升告功,宜有禮祭。」於是使謁者以一特牲於常祠泰山處,告祠泰山,如親耕、貙劉、先祠、先農、先虞故事。至食時,御輦升山,日中後到山上更衣,早晡時即位于壇,北面。群臣以次陳後,西上,畢位升壇。尚書令奉玉牒檢,皇帝以寸二分璽親封之,訖,太常命人發壇上石,尚書令藏玉牒已,復石覆訖,尚書令以五寸印封石檢。事畢,皇帝再拜,群臣稱萬歲。命人立所刻石碑,乃復道下。

二十五日甲午,禪,祭地于梁陰,以高后配,山川群神從,如元始中北郊故事。

四月己卯,大赦天下,以建武三十二年為建武中元元年,復博、奉高、嬴勿出元年租、芻稿。以吉日刻玉牒書函藏金匱,璽印封之。乙酉,使太尉行事,以特告至高廟。太尉奉匱以告高廟,藏于廟室西壁石室高主室之下。


\end{pinyinscope}