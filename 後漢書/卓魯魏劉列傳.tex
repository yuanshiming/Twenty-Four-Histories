\article{卓魯魏劉列傳}

\begin{pinyinscope}
卓茂字子康,南陽宛人也。父祖皆至郡守。茂,元帝時學於長安,事博士江生,習詩、禮及歷筭,究極師法,稱為通儒。性寬仁恭愛。鄉黨故舊,雖行能與茂不同,而皆愛慕欣欣焉。

初辟丞相府史,事孔光,光稱為長者。時嘗出行,有人認其馬。茂問曰:「子亡馬幾何時?」對曰:「月餘日矣。」茂有馬數年,心知其謬,嘿解與之,挽車而去,顧曰:「若非公馬,幸至丞相府歸我。」他日,馬主別得亡者,乃詣府送馬,叩頭謝之。茂性不好爭如此。

後以儒術舉為侍郎,給事黃門,遷密令。勞心諄諄,視人如子,舉善而教,口無惡言,吏人親愛而不忍欺之。人嘗有言部亭長受其米肉遺者,茂辟左右問之曰:「亭長為從汝求乎?為汝有事囑之而受乎?將平居自以恩意遺之乎?」人曰:「往遺之耳。」茂曰:「遺之而受,何故言邪?」人曰:「竊聞賢明之君,使人不畏吏,吏不取人。今我畏吏,是以遺之,吏既卒受,故來言耳。」茂曰:「汝為敝人矣。凡人所以貴於禽獸者,以有仁愛,知相敬事也。今鄰里長老尚致饋遺,此乃人道所以相親,況吏與民乎?吏顧不當乘威力強請求耳。凡人之生,群居雜處,故有經紀禮義以相交接。汝獨不欲修之,寧能高飛遠走,不在人閒邪?亭長素善吏,歲時遺之,禮也。」人曰:「苟如此,律何故禁之?」茂笑曰:「律設大法,禮順人情。今我以禮教汝,汝必無怨惡;以律治汝,何所措其手足乎?一門之內,小者可論,大者可殺也。且歸念之!」於是人納其訓,吏懷其恩。初,茂到縣,有所廢置,吏人笑之,鄰城聞者皆蚩其不能。河南郡為置守令,茂不為嫌,理事自若。數年,教化大行,道不拾遺。平帝時,天下大蝗,河南二十餘縣皆被其災,獨不入密縣界。督郵言之,太守不信,自出案行,見乃服焉。

是時王莽秉政,置大司農六部丞,勸課農桑,遷茂為京部丞,密人老少皆涕泣隨送。及莽居攝,以病免歸郡,常為門下掾祭酒,不肯作職吏。

更始立,以茂為侍中祭酒,從至長安,知更始政亂,以年老乞骸骨歸。

時光武初即位,先訪求茂,茂詣河陽謁見。乃下詔曰:「前密令卓茂,束身自修,執節淳固,誠能為人所不能為。夫名冠天下,當受天下重賞,故武王誅紂,封比干之墓,表商容之閭。今以茂為太傅,封褒德侯,食邑二千戶,賜几杖車馬,衣一襲,絮五百斤」。復以茂長子戎為太中大夫,次子崇為中郎,給事黃門。建武四年,薨,賜棺槨冢地,車駕素服親臨送葬。

子崇嗣,徙封汎鄉侯,官至大司農。崇卒,子棽嗣。棽卒,子訢嗣。訢卒,子隆嗣。永元十五年,隆卒,無子,國除。

初,茂與同縣孔休、陳留蔡勳、安眾劉宣、楚國龔勝、上黨鮑宣六人同志,不仕王莽時,並名重當時。休字子泉,哀帝初,守新都令。後王莽秉權,休去官歸家。及莽篡位,遣使齎玄纁、束帛,請為國師,遂歐血託病,杜門自絕。光武即位,求休、勳子孫,賜穀以旌顯之。劉宣字子高,安眾侯崇之從弟,知王莽當篡,乃變名姓,抱經書隱避林藪。建武初乃出,光武以宣襲封安眾侯。擢龔勝子賜為上谷太守。勝、鮑宣事在前書。勳事在玄孫邕傳。

論曰:建武之初,雄豪方擾,虓呼者連響,嬰城者相望,斯固倥傯不暇給之日。卓茂斷斷小宰,無它庸能,時已七十餘矣,而首加聘命,優辭重禮,其與周、燕之君表閭立館何異哉?於是蘊憤歸道之賓,越關阻,捐宗族,以排金門者眾矣。夫厚性寬中近於仁,犯而不校鄰於恕,率斯道也,怨悔曷其至乎!

魯恭字仲康,扶風平陵人也。其先出於魯傾公,為楚所滅,遷於下邑,因氏焉。世吏二千石,哀平閒,自魯而徙。祖父匡,王莽時,為羲和,有權數,號曰「智囊」。父某,建武初,為武陵太守,卒官。時恭年十二,弟丕七歲,晝夜號踴不絕聲,郡中賻贈無所受,乃歸服喪,禮過成人,鄉里奇之。十五,與母及丕俱居太學,習魯詩,閉戶講誦,絕人閒事,兄弟俱為諸儒所稱,學士爭歸之。

太尉趙憙慕其志,每歲時遣子問以酒糧,皆辭不受。恭憐丕小,欲先就其名,託疾不仕。郡數以禮請,謝不肯應,母強遣之,恭不得已而西,因留新豐教授。建初初,丕舉方正,恭始為郡吏。太傅趙憙聞而辟之。肅宗集諸儒於白虎觀,恭特以經明得召,與其議。

憙復舉恭直言,待詔公車,拜中牟令。恭專以德化為理,不任刑罰。訟人許伯等爭田,累守令不能決,恭為平理曲直,皆退而自責,輟耕相讓。亭長從人借牛而不肯還之,牛主訟於恭。恭召亭長,敕令歸牛者再三,猶不從。恭歎曰:「是教化不行也。」欲解印綬去。掾史泣涕共留之,亭長乃慚悔,還牛,詣獄受罪,恭貰不問。於是吏人信服。建初七年,郡國螟傷稼,犬牙緣界,不入中牟。河南尹袁安聞之,疑其不實,使仁恕掾肥親往廉之。恭隨行阡陌,俱坐桑下,有雉過,止其傍。傍有童兒,親曰:「兒何不捕之?」兒言「雉方將雛」。親瞿然而起,與恭訣曰:「所以來者,欲察君之政跡耳。今蟲不犯境,此一異也;化及鳥獸,此二異也;豎子有仁心,此三異也。久留,徒擾賢者耳。」還府,具以狀白安,是歲,嘉禾生恭便坐廷中,安因上書言狀,帝異之。會詔百官舉賢良方正,恭薦中牟名士王方,帝即徵方詣公車,禮之與公卿所舉同,方致位侍中。恭在事三年,州舉尤異,會遭母喪去官,吏人思之。

後拜侍御史。和帝初立,議遣車騎將軍竇憲與征西將軍耿秉擊匈奴,恭上疏諫曰:

陛下親勞聖思,日昃不食,憂在軍役,誠欲以安定北垂,為人除患,定萬世之計也。臣伏獨思之,未見其便。社稷之計,萬人之命,在於一舉。數年以來,秋稼不熟,人食不足,倉庫空虛,國無畜積。會新遭大憂,人懷恐懼。陛下躬大聖之德,履至孝之行,盡諒陰三年,聽於冢宰。百姓闕然,三時不聞警蹕之音,莫不懷思皇皇,若有求而不得。今乃以盛春之月,興發軍役,擾動天下,以事戎夷,誠非所以垂恩中國,改元正時,由內及外也。

萬民者,天之所生。天愛其所生,猶父母愛其子。一物有不得其所者,則天氣為之舛錯,況於人乎?故愛人者必有天報。昔太王重人命而去邠,故獲上天之祐。夫戎狄者,四方之異氣也。蹲夷踞肆,與鳥獸無別。若雜居中國,則錯亂天氣,汙辱善人,是以聖王之制,羈縻不絕而已。

今邊境無事,宜當脩仁行義,尚於無為,令家給人足,安業樂產。夫人道乂於下,則陰陽和於上,祥風時雨,覆被遠方,夷狄重譯而至矣。《易》曰:『有孚盈缶,終來有它吉。』言甘雨滿我之缶,誠來有我而吉已。夫以德勝人者昌,以力勝人者亡。今匈奴為鮮卑所殺,遠臧於史侯河西,去塞數千里,而欲乘其虛耗,利其微弱,是非義之所出也。前太僕祭肜遠出塞外,卒不見一胡而兵已困矣。白山之難,不絕如綖,都護陷沒,士卒死者如積,迄今被其辜毒。孤寡哀思之心未弭,仁者念之,以為累息,柰何復欲襲其跡,不顧患難乎?今始徵發,而大司農調度不足,使者在道,分部督趣,上下相迫,民閒之急亦已甚矣。三輔、并、涼少雨,麥根枯焦,牛死日甚,此其不合天心之效也。群僚百姓,咸曰不可,陛下獨柰何以一人之計,棄萬人之命,不卹其言乎?上觀天心,下察人志,足以知事之得失。臣恐中國不為中國,豈徒匈奴而已哉!惟陛下留聖恩,休罷士卒,以順天心。

書奏,不從。每政事有益於人,恭輒言其便,無所隱諱。

其後拜為魯詩博士,由是家法學者日盛。遷侍中,數召讌見,問以得失,賞賜恩禮寵異焉。遷樂安相。是時東州多盜賊,群輩攻劫,諸郡患之。恭到,重購賞,開恩信,其渠帥張漢等率支黨降,恭上以漢補博昌尉,其餘遂自相捕擊,盡破平之,州郡以安。

永元九年,徵拜議郎。八月,飲酎,齋會章臺,詔使小黃門特引恭前。其夜拜侍中,敕使陪乘,勞問甚渥。冬,遷光祿勳,選舉清平,京師貴戚莫能枉其正。十二年,代呂蓋為司徒。十五年,從巡狩南陽,除子撫為郎中,賜駙馬從駕。時弟丕亦為侍中。兄弟父子並列朝廷。後坐事策免。殤帝即位,以恭為長樂衛尉。永初元年,復代梁鮪為司徒。

初,和帝末,下令麥秋得案驗薄刑,而州郡好以苛察為政,因此遂盛夏斷獄。恭上疏諫曰:

臣伏見詔書,敬若天時,憂念萬民,為崇和氣,罪非殊死,且勿案驗。進柔良,退貪殘,奉時令。所以助仁德,順昊天,致和氣,利黎民者也。

舊制至立秋乃行薄刑,自永元十五年以來,改用孟夏,而刺吏、太守不深惟憂民息事之原,進良退殘之化,因以盛夏徵召農人,拘對考驗,連滯無已,司隸典司京師,四方是則,而近於春月分行諸部,託言勞來貧人,而無隱惻之實,煩擾郡縣,廉考非急,逮捕一人,罪延十數,上逆時氣,下傷農業。案易五月姤用事。經曰:「后以施令誥四方。」言君以夏至之日,施命令止四方行者,所以助微陰也。行者尚止之,況於逮召考掠,奪其時哉!

比年水旱傷稼,人飢流冗。今始夏,百穀權輿,陽氣胎養之時。自三月以來,陰寒不暖,物當化變而不被和氣。月令:「孟夏斷薄刑,出輕繫。行秋令則苦雨數來,五穀不熟。」又曰:「仲夏挺重囚,益其食。行秋令則草木零落,人傷於疫。」夫斷薄刑者,謂其輕罪已正,不欲令久繫,故時斷之也。臣愚以為今孟夏之制,可從此令,其決獄案考,皆以立秋為斷,以順時節,育成萬物,則天地以和,刑罰以清矣。

初,肅宗時,斷獄皆以冬至之前,自後論者互多駮異。鄧太后詔公卿以下會議,恭議奏曰:

夫陰陽之氣,相扶而行,發動用事,各有時節。若不當其時,則物隨而傷。王者雖質文不同,而茲道無變,四時之政,行之若一。月令,周世所造,而所據皆夏之時也,其變者唯正朔、服色、犧牲、徽號、器械而已。故曰:「殷因於夏禮,周因於殷禮,所損益可知也。」《易》曰:「潛龍勿用。」言十一月、十二月陽氣潛臧,未得用事。雖煦噓萬物,養其根荄,而猶盛陰在上,地涷水冰,陽氣否隔,閉而成冬。故曰:「履霜堅冰,陰始凝也。馴致其道,至堅冰也。」言五月微陰始起,至十一月堅冰至也。

夫王者之作,因時為法。孝章皇帝深惟古人之道,助三正之微,定律著令,冀承天心,順物性命,以致時雍。然從變改以來,年歲不熟,穀價常貴,人不寧安。小吏不與國同心者,率入十一月得死罪賊,不問曲直,便即格殺。雖有疑罪,不復讞正。一夫吁嗟,王道為虧,況於眾乎?易十一月「君子以議獄緩死」可令疑罪使詳其法,大辟之科,盡冬月乃斷。其立春在十二月中者,勿以報囚如故事。

後卒施行。

恭再在公位,選辟高第,至列卿郡守者數十人。而其耆舊大姓,或不蒙薦舉,至有怨望者。恭聞之,曰:「學之不講,是吾憂也。諸生不有鄉舉者乎?」終無所言。恭性謙退,奏議依經,潛有補益,然終不自顯,故不以剛直為稱。三年,以老病策罷。六年,年八十一,卒於家。

以兩子為郎。長子謙,為隴西太守,有名績。謙子旭,官至太僕,從獻帝西入關,與司徒王允同謀共誅董卓。及李傕入長安。旭與允俱遇害。

丕字叔陵,性沈深好學,孳孳不倦,遂杜絕交游,不荅候問之禮。士友常以此短之,而丕欣然自得。遂兼通五經,以魯詩、尚書教授,為當世名儒。後歸郡,為督郵功曹,所事之將,無不師友待之。

建初元年,肅宗詔舉賢良方正,大司農劉寬舉丕。時對策者百有餘人,唯丕在高第,除為議郎,遷新野令。視事期年,州課第一,擢拜青州刺史。務在表賢明,慎刑罰。七年,坐事下獄司寇論。

元和元年徵,再遷,拜趙相。門生就學者常百餘人,關東號之曰「五經復興魯叔陵」。趙王商嘗欲避疾,便時移住學官,丕止不聽。王乃上疏自言,詔書下丕。丕奏曰:「臣聞禮,諸侯薨於路寢,大夫卒於嫡室,死生有命,未有逃避之典也。學官傳五帝之道,修先王禮樂教化之處,王欲廢塞以廣游讌,事不可聽。」詔從丕言,王以此憚之。其後帝巡狩之趙,特被引見,難問經傳,厚加賞賜。在職六年,嘉瑞屢降,吏人重之。

永元二年,遷東郡太守。丕在二郡,為人修通溉灌,百姓殷富。數薦達幽隱名士。明年,拜陳留太守。視事三期,後坐稟貧人不實,徵司寇論。

十一年復徵,再遷中散大夫。時侍中賈逵薦丕道蓺深明,宜見任用。和帝因朝會,召見諸儒,丕與侍中賈逵、尚書令黃香等相難數事,帝善丕說,罷朝,特賜冠幘履陉衣一襲。丕因上疏曰:「臣以愚頑,顯備大位,犬馬氣衰,猥得進見,論難於前,無所甄明,衣服之賜,誠為優過。臣聞說經者,傳先師之言,非從己出,不得相讓;相讓則道不明,若規矩權衡之不可枉也。難者必明其據,說者務立其義,浮華無用之言不陳於前,故精思不勞而道術愈章。法異者,各令自說師法,博觀其義。覽詩人之旨意,察雅頌之終始,明舜、禹、皋陶之相戒,顯周公、箕子之所陳,觀乎人文,化成天下。陛下既廣納謇謇以開四聰,無令芻蕘以言得罪;既顯巖穴以求仁賢,無使幽遠獨有遺失。」

十三年,遷為侍中,免。

永初二年,詔公卿舉儒術篤學者,大將軍鄧騭舉丕,再遷,復為侍中、左中郎將,再為三老。五年,年七十五,卒於官。

魏霸字喬卿,濟陰句陽人也。世有禮義。霸少喪親,兄弟同居,州里慕其雍和。

建初中,舉孝廉,八遷。和帝時為鉅鹿太守。以簡朴寬恕為政。掾史有過,霸先誨其失,不改者乃罷之。吏或相毀訴,霸輒稱它吏之長,終不及人短,言者懷慚,譖訟遂息。

永元十六年,徵拜將作大匠。明年,和帝崩,典作順陵。時盛冬地凍,中使督促,數罰縣吏以厲霸。霸撫循而已,初不切責,而反勞之曰:「令諸卿被辱,大匠過也。」吏皆懷恩,力作倍功。

延平元年,代尹勤為太常。明年,以病致仕,為光祿大夫。永初五年,拜長樂衛尉,以病乞身,復為光祿大夫,卒於官。

劉寬字文饒,弘農華陰人也。父崎,順帝時為司徒。寬嘗行,有人失牛者,乃就寬車中認之。寬無所言,下駕步歸。有頃,認者得牛而送還,叩頭謝曰:「慚負長者,隨所刑罪。」寬曰:「物有相類,事容脫誤,幸勞見歸,何為謝之?」州里服其不校。

桓帝時,大將軍辟,五遷司徒長史。時京師地震,特見詢問,再遷,出為東海相。延熹八年,徵拜尚書令,遷南陽太守。典歷三郡,溫仁多恕,雖在倉卒,未嘗疾言遽色。常以為「齊之以刑,民免而無恥」。吏人有過,但用蒲鞭罰之,示辱而已,終不加苦。事有功善,推之自下。災異或見,引躬克責。每行縣止息亭傳,輒引學官祭酒及處士諸生執經對講。見父老慰以農里之言,少年勉以孝悌之訓。人感德興行,日有所化。

靈帝初,徵拜太中大夫,侍講華光殿。遷侍中,賜衣一襲。轉屯騎校尉,遷宗正,轉光祿勳。熹平五年,代許訓為太尉。靈帝頗好學蓺,每引見寬,常令講經。寬嘗於坐被酒睡伏。帝問:「太尉醉邪?」寬仰對曰:「臣不敢醉,但任重責大,憂心如醉。」帝重其言。

寬簡略嗜酒,不好盥浴,京師以為諺。嘗坐客,遣蒼頭市酒,迂久,大醉而還。客不堪之,罵曰:「畜產。」寬須臾遣人視奴,疑必自殺。顧左右曰:「此人也,罵言畜產,辱孰甚焉!故吾懼其死也。」夫人欲試寬令恚,伺當朝會,裝嚴已訖,使侍婢奉肉羹,侴汙朝衣。婢遽收之,寬神色不異,乃徐言曰:「羹爛汝手?」其性度如此。海內稱為長者。

後以日食策免。拜衛尉。光和二年,復代段熲為太尉。在職三年,以日變免。又拜永樂少府,遷光祿勳。以先策黃巾逆謀,以事上聞,封逯鄉侯六百戶。中平二年卒,時年六十六。贈車騎將軍印綬,位特進,謚曰昭烈侯。子松嗣,官至宗正。

贊曰:卓、魯款款,情愨德滿。仁感昆蟲,愛及胎卵。寬、霸臨政,亦稱優緩。


\end{pinyinscope}