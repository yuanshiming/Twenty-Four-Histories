\article{李杜列傳}

\begin{pinyinscope}
李固字子堅,漢中南鄭人,司徒郃之子也。郃在數術傳。固貌狀有奇表,鼎角匿犀,足履龜文。少好學,常步行尋師,不遠千里。遂究覽墳籍,結交英賢。四方有志之士,多慕其風而來學。京師咸歎曰:「是復為李公矣。」司隸、益州並命郡舉孝廉,辟司空掾,皆不就。

陽嘉二年,有地動、山崩、火災之異,公卿舉固對策,詔又特問當世之敝,為政所宜。

臣聞王者父天母地,寶有山川。王道得則陰陽和穆,政化乖則崩震為災。斯皆關之天心,效於成事者也。夫化以職成,官由能理。古之進者,有德有命;今之進者,唯財與力。伏聞詔書務求寬博,疾惡嚴暴。而今長吏多殺伐致聲名者,必加遷賞;其存寬和無黨援者,輒見斥逐。是以淳厚之風不宣,彫薄之俗未革。雖繁刑重禁,何能有益?前孝安皇帝變亂舊典,封爵阿母,因造妖孽,使樊豐之徒乘權放恣,侵奪主威,改亂嫡嗣,至令聖躬狼狽,親遇其艱。既拔自困殆,龍興即位,天下喁喁,屬望風政。積敝之後,易致中興,誠當沛然思惟善道;而論者猶云,方今之事,復同於前。臣伏從山草,痛心傷臆。實以漢興以來,三百餘年,賢聖相繼,十有八主。豈無阿乳之恩?豈忘貴爵之寵?然上畏天威,俯案經典,知義不可,故不封也。今宋阿母雖有大功勤謹之德,但加賞賜,足以酬其勞苦;至於裂土開國,實乖舊典。聞阿母體性謙虛,必有遜讓,陛下宜許其辭國之高,使成萬安之福。

夫妃后之家所以少完全者,豈天性當然?但以爵位尊顯,專總權柄,天道惡盈,不知自損,故至顛仆。先帝寵遇閻氏,位號太疾,故其受禍,曾不旋時。老子曰:「其進銳,其退速也。」今梁氏戚為椒房,禮所不臣,尊以高爵,尚可然也。而子弟群從,榮顯兼加,永平、建初故事,殆不如此。宜令步兵校尉冀及諸侍中還居黃門之官,使權去外戚,政歸國家,豈不休乎!

又詔書所以禁侍中尚書中臣子弟不得為吏察

孝廉者,以其秉威權,容請託故也。而中常侍日月之側,聲埶振天下,子弟祿仕,曾無限極。雖外託謙默,不干州郡,而諂偽之徒,望風進舉。今可為設常禁,同之中臣。

昔館陶公主為子求郎,明帝不許,賜錢千萬。所以輕厚賜,重薄位者,為官人失才,害及百姓也。竊聞長水司馬武宣、開陽城門候羊迪等,無它功德,初拜便真。此雖小失,而漸壞舊章。先聖法度,所宜堅守,政教一跌,百年不復。《詩》云:「上帝板板,下民卒癉。」刺周王變祖法度,故使下民將盡病也。

今陛下之有尚書,猶天之有北斗也。斗為天喉舌,尚書亦為陛下喉舌。斗斟酌元氣,運平四時。尚書出納王命,賦政四海,權尊埶重,責之所歸。若不平心,災眚必至。誠宜審擇其人,以毗聖政。今與陛下共理天下者,外則公卿尚書,內則常侍黃門,譬猶一門之內,一家之事,安則共其福慶,危則通其禍敗。刺史、二千石,外統職事,內受法則。夫表曲者景必邪,源清者流必絜,猶叩樹本,百枝皆動也。周頌曰:「薄言振之,莫不震疊。」此言動之於內,而應於外者也。猶此言之,本朝號令,豈可蹉跌?閒隙一開,則邪人動心;利競暫啟,則仁義道塞。刑罰不能復禁,化導以之寑壞。此天下之紀綱,當今之急務。陛下宜開石室,陳圖書,招會群儒,引問失得,指擿變象,以求天意。其言有中理,即時施行,顯拔其人,以表能者。則聖聽日有所聞,忠臣盡其所知。又宜罷退宦官,去其權重,裁置常侍二人,方直有德者,省事左右;小黃門五人,才智閑雅者,給事殿中。如此,則論者厭塞,升平可致也。臣所以敢陳愚瞽,冒昧自聞者,儻或皇天欲令微臣覺悟陛下。陛下宜熟察臣言,憐赦臣死。

順帝覽其對,多所納用,即時出阿母還弟舍,諸常侍悉叩頭謝罪,朝廷肅然。以固為議郎。而阿母宦者疾固言直,因詐飛章以陷其罪,事從中下。大司農黃尚等請之於大將軍梁商,又僕射黃瓊救明固事,久乃得拜議郎。

出為廣漢雒令,至白水關,解印綬,還漢中,杜門不交人事。歲中,梁商請為從事中郎。商以后父輔政,而柔和自守,不能有所整裁,災異數見,下權日重。固欲令商先正風化,退辭高滿,乃奏記曰:「春秋褒儀父以開義路,貶無駭以閉利門。夫義路閉則利門開,利門開則義路閉也。前孝安皇帝內任伯榮、樊豐之屬,外委周廣、謝惲之徒,開門受賂,署用非次,天下紛然,怨聲滿道。朝廷初立,頗存清靜,未能數年,稍復墮損。左右黨進者,日有遷拜,守死善道者,滯涸窮路,而未有改敝立德之方。又即位以來,十有餘年,聖嗣未立,群下繼望。可令中宮博簡嬪媵,兼採微賤宜子之人,進御至尊,順助天意。若有皇子,母自乳養,無委保妾醫巫,以致飛燕之禍。明將軍望尊位顯,當以天下為憂,崇尚謙省,垂則萬方。而新營祠堂,費功億計,非以昭明令德,崇示清儉。自數年以來,災怪屢見,比無雨潤,而沈陰鬱泱。宮省之內,容有陰謀。孔子曰:『智者見變思刑,愚者睹怪諱名。』天道無親,可為祗畏。加近者月食既於端門之側。月者,大臣之體也。夫窮高則危,大滿則溢,月盈則缺,日中則移。凡此四者,自然之數也。天地之心,福謙忌盛,是以賢達功遂身退,全名養壽,無有怵迫之憂。誠令王綱一整,道行忠立,明公踵伯成之高,全不朽之譽,豈與此外戚凡輩耽榮好位者同日而論哉!固狂夫下愚,不達大體,竊感古人一飯之報,況受顧遇而容不盡乎!」商不能用。

永和中,荊州盜賊起,彌年不定,乃以固為荊州刺史。固到,遣吏勞問境內,赦寇盜前釁,與之更始。於是賊帥夏密等斂其魁黨六百餘人,自縛歸首。固皆原之,遣還,使自相招集,開示威法。半歲閒,餘類悉降,州內清平。

上奏南陽太守高賜等臧穢。賜等懼罪,遂共重賂大將軍梁冀,冀為千里移檄,而固持之愈急。冀遂令徙固為太山太守。時太山盜賊屯聚歷年,郡兵常千人,追討不能制。固到,悉罷遣歸農,但選留任戰者百餘人,以恩信招誘之。未滿歲,賊皆弭散。

遷將作大匠。上疏陳事曰:「臣聞氣之清者為神,人之清者為賢。養身者以練神為寶,安國者以積賢為道。昔秦欲謀楚,王孫圉設壇西門,陳列名臣,秦使戄然,遂為寑兵。魏文侯師卜子夏,友田子方,軾段干木,故群俊競至,名過齊桓,秦人不敢闚兵於西河,斯蓋積賢人之符也。陛下撥亂龍飛,初登大位,聘南陽樊英、江夏黃瓊、廣漢楊厚、會稽賀純,策書嗟歎,待以大夫之位。是以巖穴幽人,智術之士,彈冠振衣,樂欲為用,四海欣然,歸服聖德。厚等在職,雖無奇卓,然夕惕孳孳,志在憂國。臣前在荊州,聞厚、純等以病免歸,誠以悵然,為時惜之。一日朝會,見諸侍中並皆年少,無一宿儒大人可顧問者,誠可歎息。宜徵還厚等,以副群望。瓊久處議郎,已且十年,眾人皆怪始隆崇,今更滯也。光祿大夫周舉,才謨高正,宜在常伯,訪以言議。侍中杜喬,學深行直,當世良臣,久託疾病,可敕令起。」又薦陳留楊倫、河南尹存、東平王惲、陳國何臨、清河房植等。是日有詔徵用倫、厚等,而遷瓊、舉,以固為大司農。

先是周舉等八使案察天下,多所劾奏,其中並是宦者親屬,輒為請乞,詔遂令勿考。又舊任三府選令史,光祿試尚書郎,時皆特拜,不復選試。固乃與廷尉吳雄上疏,以為八使所糾,宜急誅罰,選舉署置,可歸有司。帝感其言,乃更下免八使所舉刺史、二千石,自是稀復特拜,切責三公,明加考察,朝廷稱善。乃復與光祿勳劉宣上言:「自頃選舉牧守,多非其人,至行無道,侵害百姓。又宜止槃遊,專心庶政。」帝納其言,於是下詔諸州劾奏守令以下,政有乖枉,遇人無惠者,免所居官;其姦穢重罪,收付詔獄。

及沖帝即位,以固為太尉,與梁冀參錄尚書事。明年帝崩,梁太后以楊、徐盜賊盛強,恐驚擾致亂,使中常侍詔固等,欲須所徵諸王侯到乃發喪。固對曰:「帝雖幼少,猶天下之父。今日崩亡,人神感動,豈有臣子反共掩匿乎?昔秦皇亡於沙丘,胡亥、趙高隱而不發,卒害扶蘇,以至亡國。近北鄉侯薨,閻后兄弟及江京等亦共掩祕,遂有孫程手刃之事。此天下大忌,不可之甚者也。」太后從之,即暮發喪。

固以清河王蒜年長有德,欲立之,謂梁冀曰:「今當立帝,宜擇長年高明有德,任親政事者,願將軍審詳大計,察周、霍之立文、宣,戒鄧、閻之利幼弱。」冀不從,乃立樂安王子纘,年八歲,是為質帝。時沖帝將北卜山陵,固乃議曰:「今處處寇賊,軍興用費加倍,新創憲陵,賦發非一。帝尚幼小,可起陵於憲陵塋內,依康陵制度,其於役費三分減一。」乃從固議。時太后以比遭不造,委任宰輔,固所匡正,每輒從用,其黃門宦者一皆斥遣,天下咸望遂平,而梁冀猜專,每相忌疾。

初,順帝時諸所除官,多不以次,及固在事,奏免百餘人。此等既怨,又希望冀旨,遂共作飛章虛誣固罪曰:「臣聞君不稽古,無以承天;臣不述舊,無以奉君。昔堯殂之後,舜仰慕三年,坐則見堯於牆,食則睹堯於羹。斯所謂聿追來孝,不失臣子之節者。太尉李固,因公假私,依正行邪,離閒近戚,自隆支黨。至於表舉薦達,例皆門徒;及所辟召,靡非先舊。或富室財賂,或子婿婚屬,其列在官牒者凡四十九人。又廣選賈豎,以補令史;募求好馬,臨吓呈試。出入踰侈,輜軿曜日。大行在殯,路人掩涕,固獨胡粉飾貌,搔頭弄姿,槃旋偃仰,從容冶步,曾無慘怛傷悴之心。山陵未成,違矯舊政,善則稱己,過則歸君,斥逐近臣,不得侍送,作威作福,莫固之甚。臣聞台輔之位,實和陰陽,琁機不平,寇賊姦軌,則責在太尉。固受任之後,東南跋扈,兩州數郡,千里蕭條,兆人傷損,大化陵遲,而詆疵先主,苟肆狂狷。存無廷爭之忠,沒有誹謗之說。夫子罪莫大於累父,臣惡莫深於毀君。固之過釁,事合誅辟。」事奏,冀以白太后,使下其事。太后不聽,得免。

冀忌帝聰慧,恐為後患,遂令左右進鴆。帝苦煩甚,使促召固。固入,前問:「陛下得患所由?」帝尚能言,曰:「食煮餅,今腹中悶,得水尚可活。」時冀亦在側,曰:「恐吐,不可飲水。」語未絕而崩。固伏尸號哭,推舉侍醫。冀慮其事泄,大惡之。

因議立嗣,固引司徒胡廣、司空趙戒,先與冀書曰:「天下不幸,仍遭大憂。皇太后聖德當朝,攝統萬機,明將軍體履忠孝,憂存社稷,而頻年之閒,國祚三絕。今當立帝,天下重器,誠知太后垂心,將軍勞慮,詳擇其人,務存聖明。然愚情眷眷,竊獨有懷。遠尋先世廢立舊儀,近見國家踐祚前事,未嘗不詢訪公卿,廣求群議,令上應天心,下合眾望。且永初以來,政事多謬,地震宮廟,彗星竟天,誠是將軍用情之日。傳曰:『以天下與人易,為天下得人難。』昔昌邑之立,昏亂日滋,霍光憂愧發憤,悔之折骨。自非博陸忠勇,延年奮發,大漢之祀,幾將傾矣。至憂至重,可不熟慮!悠悠萬事,唯此為大。國之興衰,在此一舉。」冀得書,乃召三公、中二千石、列侯大議所立。固、廣、戒及大鴻臚杜喬皆以為清河王蒜明德著聞,又屬最尊親,宜立為嗣。先是蠡吾侯志當取冀妹,時在京師,冀欲立之。眾論既異,憤憤不得意,而未有以相奪。中常侍曹騰等聞而夜往說冀曰:「將軍累世有椒房之親,秉攝萬機,賓客縱橫,多有過差。清河王嚴明,若果立,則將軍受禍不久矣。不如立蠡吾侯,富貴可長保也。」冀然其言。明日重會公卿,冀意氣凶凶,而言辭激切。自胡廣、趙戒以下,莫不懾憚之。皆曰:「惟大將軍令。」而固獨與杜喬堅守本議。冀厲聲曰:「罷會。」固意既不從,猶望眾心可立,復以書勸冀。冀愈激怒,乃說太后先策免固,竟立蠡吾侯,是為桓帝。

後歲餘,甘陵劉文、魏郡劉鮪各謀立蒜為天子,梁冀因此誣固與文、鮪共為妖言,下獄。門生勃海王調貫械上書,證固之枉,河內趙承等數十人亦要鈇鑕詣闕通訴,太后明之,乃赦焉。及出獄,京師市里皆稱萬歲。冀聞之大驚,畏固名德終為己害,乃更據奏前事,遂誅之,時年五十四。

臨命,與胡廣、趙戒書曰:「固受國厚恩,是以竭其股肱,不顧死亡,志欲扶持王室,比隆文、宣。何圖一朝梁氏迷謬,公等曲從,以吉為凶,成事為敗乎?漢家衰微,從此始矣。公等受主厚祿,顛而不扶,傾覆大事,後之良史,豈有所私?固身已矣,於義得矣,夫復何言!」廣、戒得書悲慚,皆長歎流涕。

州郡收固二子基、茲於郾城,皆死獄中。小子燮得脫亡命。冀乃封廣、戒而露固尸於四衢,令有敢臨者加其罪。固弟子汝南郭亮,年始成童,遊學洛陽,乃左提章鉞,右秉鈇鑕,詣闕上書,乞收固屍。不許,因往臨哭,陳辭於前,遂守喪不去。夏門亭長呵之曰:「李、杜二公為大臣,不能安上納忠,而興造無端。卿曹何等腐生,公犯詔書,干試有司乎?」亮曰:「亮含陰陽以生,戴乾履坤。義之所動,豈知性命,何為以死相懼?」亭長歎曰:「居非命之世,天高不敢不跼,地厚不敢不蹐。耳目適宜視聽,口不可以妄言也。」太后聞而不誅。南陽人董班亦往哭固,而殉尸不肯去。太后憐之,乃聽得襚斂歸葬。二人由此顯名,三公並辟。班遂隱身,莫知所歸。

固所著章、表、奏、議、教令、對策、記、銘凡

十一篇。弟子趙承等悲歎不已,乃共論固言跡,以為德行一篇。

燮字德公。初,固既策罷,知不免禍,乃遣三子歸鄉里。時燮年十三,姊文姬為同郡趙伯英妻,賢而有智,見二兄歸,具知事本,默然獨悲曰:「李氏滅矣!自太公已來,積德累仁,何以遇此?」密與二兄謀豫藏匿燮,託言還京師,人咸信之。有頃難作,下郡收固三子。二兄受害,文姬乃告父門生王成曰:「君執義先公,有古人之節。今委君以六尺之孤,李氏存滅,其在君矣。」成感其義,乃將燮乘江東下,入徐州界內,令變名姓為酒家傭,而成賣卜於巿。各為異人,陰相往來。

燮從受學,酒家異之,意非恆人,以女妻燮。燮專精經學。十餘年閒,梁冀既誅而災眚屢見。明年,史官上言宜有赦令,又當存錄大臣冤死者子孫,於是大赦天下,并求固後嗣。燮乃以本末告酒家,酒家具車重厚遣之,皆不受,遂還鄉里,追服。姊弟相見,悲感傍人。既而戒燮曰:「先公正直,為漢忠臣,而遇朝廷傾亂,梁冀肆虐,令吾宗祀血食將絕。今弟幸而得濟,豈非天邪!宜杜絕眾人,勿妄往來,慎無一言加於梁氏。加梁氏則連主上,禍重至矣。唯引咎而已。」燮謹從其誨。後王成卒,燮以禮葬之,感傷舊恩,每四節為設上賓之位而祠焉。

州郡禮命,四府並辟,皆無所就,後徵拜議郎。及其在位,廉方自守,所交皆舍短取長,好成人之美。時潁川荀爽、賈彪,雖俱知名而不相能,燮並交二子,情無適莫,世稱其平正。

靈帝時拜安平相。先是安平王續為張角賊所略,國家贖王得還,朝廷議復其國。燮上奏曰:「續在國無政,為妖賊所虜,守藩不稱,損辱聖朝,不宜復國。」時議者不同,而續竟歸藩。燮以謗毀宗室,輸作左校。未滿歲,王果坐不道被誅,乃拜燮為議郎。京師語曰:「

父不肯立帝,子不肯立王。」

擢遷河南尹。時既以貨賂為官,詔書復橫發錢三億,以實西園。燮上書陳諫,辭義深切,帝乃止。先是潁川甄邵諂附梁冀,為鄴令。有同歲生得罪於冀,亡奔邵,邵偽納而陰以告冀,冀即捕殺之。邵當遷為郡守,會母亡,邵且埋屍於馬屋,先受封,然後發喪。邵還至洛陽,燮行塗遇之,使卒投車於溝中,笞捶亂下,大署帛於其背曰「諂貴賣友,貪官埋母」。乃具表其狀。邵遂廢錮終身。燮在職二年卒,時人感其世忠正,咸傷惜焉。

杜喬字叔榮,河內林慮人也。少為諸生,舉孝廉,辟司徒楊震府。稍遷為南郡太守,轉東海相,入拜侍中。

漢安元年,以喬守光祿大夫,使徇察兗州。表奏太山太守李固政為天下第一;陳留太守梁讓、濟陰太守汜宮、濟北相崔瑗等臧罪千萬以上。讓即大將軍梁冀季父,宮、瑗皆冀所善。還,拜太子太傅,遷大司農。

時梁冀子弟五人及中常侍等以無功並封,喬上書諫曰:「陛下越從藩臣,龍飛即位,天人屬心,萬邦攸賴。不急忠賢之禮,而先左右之封,傷善害德,興長佞諛。臣聞古之明君,褒罰必以功過;末世闇主,誅賞各緣其私。今梁氏一門,宦者微孽,並帶無功之紱,裂勞臣之土,其為乖濫,胡可勝言!夫有功不賞,為善失其望;姦回不詰,為惡肆其凶。故陳資斧而人靡畏,班爵位而物無勸。苟遂斯道,豈伊傷政,為亂而已,喪身亡國,可不慎哉!」書奏不省。

益州刺史种暠舉劾永昌太守劉君世以金蛇遺梁冀

,事發覺,以蛇輸司農。冀從喬借觀之,喬不肯與,冀始為恨。累遷大源臚。時冀小女死,令公卿會喪,喬獨不往,冀又銜之。

遷光祿勳。建和元年,代胡廣為太尉。桓帝將納梁冀妹,冀欲令以厚禮迎之,喬據執舊典,不聽。又冀屬喬舉汜宮為尚書,喬以宮臧罪明著,遂不肯用,因此日忤於冀。先是李固見廢,內外喪氣,群臣側足而立,唯喬正色無所回橈。由是海內歎息,朝野瞻望焉。在位數月,以地震免。宦者唐衡、左悺等因共譖於帝曰:「陛下前當即位,喬與李固抗議言上不堪奉漢宗祀。」帝亦怨之。及清河王蒜事起,梁冀遂諷有司劾喬及李固與劉鮪等交通,請逮案罪。而梁太后素知喬忠,但策免而已。冀愈怒,使人脅喬曰:「早從宜,妻子可得全。」喬不肯。明日冀遣騎至其門,不聞哭者,遂白執繫之,死獄中。妻子歸故郡。與李固俱暴尸於城北,家屬故人莫敢視者。

喬故掾陳留楊匡聞之,號泣星行到洛陽,乃著故赤幘,託為夏門亭吏,守衛尸喪,驅護蠅蟲,積十二日,都官從事執之以聞。梁太后義而不罪。匡於是帶鈇鑕詣闕上書,并乞李、杜二公骸骨。太后許之。成禮殯殮,送喬喪還家,葬送行服,隱匿不仕。匡初好學,常在外黃大澤教授門徒。補蘄長,政有異績,遷平原令。時國相徐曾,中常侍璜之兄也,匡恥與接事,託疾牧豕云。

論曰:夫稱仁人者,其道弘矣!立言踐行,豈徒徇名安己而已哉,將以定去就之概,正天下之風,使生以理全,死與義合也。夫專為義則傷生,專為生則騫義,專為物則害智,專為己則損仁。若義重於生,舍生可也;生重於義,全生可也。上以殘闇失君道,下以篤固盡臣節。臣節盡而死之,則為殺身以成仁,去之不為求生以害仁也。順桓之閒,國統三絕,太后稱制,賊臣虎視。李固據位持重,以爭大義,確乎而不可奪。豈不知守節之觸禍,恥夫覆折之傷任也。觀其發正辭,及所遺梁冀書,雖機失謀乖,猶戀戀而不能已。至矣哉,社稷之心乎!其顧視胡廣、趙戒,猶糞土也。

贊曰:李、杜司職,朋心合力。致主文、宣,抗情伊、稷。道亡時晦,終離罔極。燮同趙孤,世載弦直。


\end{pinyinscope}