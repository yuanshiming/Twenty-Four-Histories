\article{五行三}

\begin{pinyinscope}
冬雷山鳴魚孽蝗

五行傳曰:「簡宗廟,不禱祠,廢祭祀,逆天時,則水不潤下。」謂水失其性而為災也。又曰:「聽之不聰,是謂不謀。厥咎急,厥罰恆寒,厥極貧。時則有鼓妖,時則有魚孽,時則有豕禍,時則有耳痾,時則有黑眚、黑祥,惟火沴水。」魚孽,劉歆傳以為介蟲之孽,謂蝗屬也。

和帝永元元年七月,郡國九大水,傷稼。京房易傳曰:「顓事有知,誅罰絕理,厥災水。其水也,而殺人,隕霜,大風,天黃。飢而不損,茲謂泰,厥水水殺人。辟遏有德,茲謂狂,厥水水流殺人,已水則地生蟲。歸獄不解,茲謂追非,厥水寒殺人。追誅不解,茲謂不理,厥水五穀不收。大敗不解,茲謂皆陰,厥水流入國邑,隕霜殺穀。」是時和帝幼,竇太后攝政,其兄竇憲幹事,及憲諸弟皆貴顯,並作威虣虐,嘗所怨恨,輒任客殺之。其後竇氏誅滅。

十二年六月,潁川大水,傷稼。是時和帝幸鄧貴人,陰有欲廢陰后之意,陰后亦懷恚怨。一曰,先是恭懷皇后葬禮有闕,竇太后崩後,乃改殯梁后,葬西陵,徵舅三人皆為列侯,位特進,賞賜累千金。

殤帝延平元年五月,郡國三十七大水,傷稼。董仲舒曰:「水者,陰氣盛也。」是時帝在襁抱,鄧太后專政。

安帝永初元年冬十月辛酉,河南新城山水虣出,突壞民田,壞處泉水出,深三丈。是時司空周章等以鄧太后不立皇太子勝而立清河王子,故謀欲廢置。十一月,事覺,章等被誅。是年郡國四十一水出,漂沒民人。讖曰:「水者,純陰之精也。陰氣盛洋溢者,小人專制擅權,妒疾賢者,依公結私,侵乘君子,小人席勝,失懷得志,故涌水為災。」

二年,大水。

三年,大水。

四年,大水。

五年,大水。

六年,河東池水變色,皆赤如血。是時鄧太后猶專政。

延光三年,大水,流殺民人,傷苗稼。是時安帝信江京、樊豐及阿母王聖等讒言,免太尉楊震,廢皇太子。

質帝本初元年五月,海水溢樂安、北海,溺殺人物。是時帝幼,梁太后專政。

桓帝建和二年七月,京師大水。去年冬,梁冀枉殺故太尉李固、杜喬。

三年八月,京都大水。是時梁太后猶專政。

永興元年秋,河水溢,漂害人物。

二年六月,彭城泗水增長,逆流。

永壽元年六月,雒水溢至津陽城門,漂流人物。是時梁皇后兄冀秉政,疾害忠直,威權震主。後遂誅滅。

延熹八年四月,濟北水清。九年四月,濟陰、東郡、濟北、平原河水清。襄楷上言:「河者諸侯之象,清者陽明之徵,豈獨諸侯有規京都計邪?」其明年,宮車晏駕,徵解犢亭侯為漢嗣,即尊位,是為孝靈皇帝。

永康元年八月,六州大水,勃海海溢,沒殺人。是時桓帝奢侈淫祀,其十一月崩,無嗣。

靈帝建寧四年二月,河水清。五月,山水大出,漂壞廬舍五百餘家。

熹平二年六月,東萊、北海海水溢出,漂沒人物。

三年秋,雒水出。

四年夏,郡國三水,傷害秋稼。

光和六年秋,金城河溢,水出二十餘里。

中平五年,郡國六水大出。

獻帝建安二年九月,漢水流,害民人。是時天下大亂。

十八年六月,大水。

二十四年八月,漢水溢流,害民人。

庶徵之恆寒。

靈帝光和六年冬,大寒,北海、東萊、琅邪井中冰厚尺餘。

獻帝初平四年六月,寒風如冬時。

和帝永元五年六月,郡國三雨雹,大如雞子。是時和帝用酷吏周紆為司隸校尉,刑誅深刻。

安帝永初元年,雨雹。二年,雨雹,大如雞子。三年,雨雹,大如鴈子,傷稼。劉向以為雹,陰脅陽也。是時鄧太后以陰專陽政。

元初四年六月戊辰,郡國三雨雹,大如杅杯及雞子,殺六畜。

延光元年四月,郡國二十一雨雹,大如雞子,傷稼。是時安帝信讒,無辜死者多。

三年,雨雹,大如雞子。

桓帝延熹四年五月己卯,京都雨雹,大如雞子。是時桓帝誅殺過差,又寵小人。

七年五月己丑,京都雨雹。是時皇后鄧氏僭侈,驕恣專幸。明年廢,以憂死,其家皆誅。

靈帝建寧二年四月,雨雹。

四年五月,河東雨雹。

光和四年六月,雨雹,大如雞子。是時常侍、黃門用權。

中平二年四月庚戌,雨雹,傷稼。

獻帝初平四年六月,右扶風雹如斗。

和帝元興元年冬十一月壬午,郡國四冬雷。是時皇子數不遂,皆隱之民閒。是歲,宮車晏駕,殤帝生百餘日,立以為君;帝兄有疾,封為平原王,卒,皆夭無嗣。

殤帝延平元年九月乙亥,陳留雷,有石隕地四。

安帝永初六年十月丙戌,郡六冬雷。

七年十月戊子,郡國三冬雷。

元初元年十月癸巳,郡國三冬雷。

三年十月辛亥,汝南、樂浪冬雷。

四年十月辛酉,郡國五冬雷。

六年十月丙子,郡國五冬雷。

永寧元年十月,郡國七冬雷。

建光元年十月,郡國七冬雷。

延光四年,郡國十九冬雷。是時太后攝政,上無所與。太后既崩,阿母王聖及皇后兄閻顯兄弟更秉威權,上遂不親萬機,從容寬仁任臣下。

桓帝建和三年六月乙卯,雷震憲陵寢屋。先是梁太后聽兄冀枉殺李固、杜喬。

靈帝熹平六年冬十月,東萊冬雷。

中平四年十二月晦,雨水,大雷電,雹。

獻帝初平三年五月丙申,無雲而雷。

四年五月癸酉,無雲而雷。

建安七八年中,長沙醴陵縣有大山常大鳴如牛呴聲,積數年。後豫章賊攻沒醴陵縣,殺略吏民。

靈帝熹平二年,東萊海出大魚二枚,長八九丈,高二丈餘。明年,中山王暢、任城王博並薨。

和帝永元四年,蝗。

八年五月,河內、陳留蝗。九月,京都蝗。九年,蝗從夏至秋。先是西羌數反,遣將軍將北軍五校征之。

安帝永初四年夏,蝗。是時西羌寇亂,軍眾征距,連十餘年。

五年夏,九州蝗。

六年三月,去蝗處復蝗子生。

七年夏,蝗。

元初元年夏,郡國五蝗。

二年夏,郡國二十蝗。

延光元年六月,郡國蝗。

順帝永建五年,郡國十二蝗。是時鮮卑寇朔方,用眾征之。

永和元年秋七月,偃師蝗。去年冬,烏桓寇沙南,用眾征之。

桓帝永興元年七月,郡國三十二蝗。是時梁冀秉政無謀憲,苟貪權作虐。

二年六月,京都蝗。

永壽三年六月,京都蝗。

延熹元年五月,京都蝗。

靈帝熹平六年夏,七州蝗。先是鮮卑前後三十餘犯塞,是歲護烏桓校尉夏育、破鮮卑中郎將田晏、使匈奴中郎將臧旻將南單于以下,三道並出討鮮卑。大司農經用不足,殷斂郡國,以給軍糧。三將無功,還者少半。

光和元年詔策問曰:「連年蝗蟲至冬踊,其咎焉在?」蔡邕對曰:「臣聞易傳曰:『大作不時,天降災,厥咎蝗蟲來。』河圖祕徵篇曰:『帝貪則政暴而吏酷,酷則誅深必殺,主蝗蟲。』蝗蟲,貪苛之所致也。」是時百官遷徙,皆私上禮西園以為府。

獻帝興平元年夏,大蝗。是時天下大亂。

建安二年五月,蝗。


\end{pinyinscope}