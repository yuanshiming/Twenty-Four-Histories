\article{方術列傳下}

\begin{pinyinscope}
唐檀字子產,豫章南昌人也。少遊太學,習京氏易、韓詩、顏氏春秋,尤好災異星占。後還鄉里,教授常百餘人。

元初七年,郡界有芝草生,太守劉祗欲上言之,以問檀。檀對曰:「方今外戚豪盛,陽道微弱,斯豈嘉瑞乎?」祗乃止。永寧元年,南昌有婦人生四子,祗復問檀變異之應。檀以為京師當有兵氣,其禍發於蕭牆。至延光四年,中黃門孫程揚兵殿省,誅皇后兄車騎將軍閻顯等,立濟陰王為天子,果如所占。

永建五年,舉孝廉,除郎中。是時白虹貫日,檀因上便宜三事,陳其咎徵。書奏,棄官去。著書二十八篇,名為唐子。卒於家。

公沙穆字文乂,北海膠東人也。家貧賤。自為兒童不好戲弄,長習韓詩、公羊春秋,尤銳思河洛推步之術。居建成山中,依林阻為室,獨宿無侶。時暴風震雷,有聲於外呼穆者三,穆不與語。有頃,呼者自牖而入,音狀甚怪,穆誦經自若,終亦無它妖異,時人奇之。後遂隱居東萊山,學者自遠而至。

有富人王仲,致產千金。謂穆曰:「方今之世,以貨自通,吾奉百萬與子為資,何如?」對曰:「來意厚矣。夫富貴在天,得之有命,以貨求位,吾不忍也。」

後舉孝廉,以高第為主事,遷繒相。時繒侯劉敞,東海恭王之後也,所為多不法,廢嫡立庶,傲很放恣。穆到官,謁曰:「臣始除之日,京師咸謂臣曰『繒有惡侯』,以弔小相。明侯何因得此醜聲之甚也?幸承先人之支體,傳茅土之重,不戰戰兢兢,而違越法度,故朝廷使臣為輔。願改往修來,自求多福。」乃上沒敞所侵官民田地,廢其庶子,還立嫡嗣。其蒼頭兒客犯法,皆收考之。因苦辭諫敞。敞涕泣為謝,多從其所規。

遷弘農令。縣界有螟蟲食稼,百姓惶懼。穆乃設壇謝曰:「百姓有過,罪穆之由,請以身禱。」於是暴雨,既霽而螟蟲自銷,百姓稱曰神明。永壽元年,霖雨大水,三輔以東莫不湮沒。穆明曉占候,乃豫告令百姓徙居高地,故弘農人獨得免害。

遷遼東屬國都尉,善得吏人歡心。年六十六卒官。六子皆知名。

許曼者,汝南平輿人也。祖父峻,字季山,善卜占之術,多有顯驗,時人方之前世京房。自云少嘗篤病,三年不愈,乃謁太山請命,行遇道士張巨君,授以方術。所著易林,至今行於世。

曼少傳峻學。桓帝時,隴西太守馮緄始拜郡,開綬笥,有兩赤蛇分南北走。緄令曼筮之。卦成,曼曰:「三歲之後,君當為邊將,官有東名,當東北行三千里。復五年,更為大將軍,南征。」延熹元年,緄出為遼東太守,討鮮卑,至五年,復拜車騎將軍,擊武陵蠻賊,皆如占。其餘多此類云。

趙彥者,琅邪人也。少有術學。延熹三年,琅邪賊勞丙與太山賊叔孫無忌殺都尉,攻沒琅邪屬縣,殘害吏民。朝廷以南陽宗資為討寇中郎將,杖鉞將兵,督州郡合討無忌。彥為陳孤虛之法,以賊屯在莒,莒有五陽之地,宜發五陽郡兵,從孤擊虛以討之。資具以狀上,詔書遣五陽兵到。彥推遁甲,教以時進兵,一戰破賊,燔燒屯塢,徐兗二州一時平夷。

樊志張者,漢中南鄭人也。博學多通,隱身不仕。嘗遊隴西,時破羌將軍段熲出征西羌,請見志張。其夕,熲軍為羌所圍數重,因留軍中,三日不得去。夜謂熲曰:「東南角無復羌,宜乘虛引出,住百里,還師攻之,可以全勝。」熲從之,果以破賊。於是以狀表聞。又說其人既有梓慎、焦、董之識,宜翼聖朝,咨詢奇異。於是有詔特徵,會病終。

單颺字武宣,山陽湖陸人也。以孤特清苦自立,善明天官、筭術。舉孝廉,稍遷太史令,侍中。出為漢中太守,公事免。後拜尚書,卒於官。

初,熹平末,黃龍見譙,光祿大夫橋玄問颺:「此何祥也?」颺曰:「其國當有王者興。不及五十年,龍當復見,此其應也。」魏郡人殷登密記之。至建安二十五年春,黃龍復見譙,其冬,魏受禪。

韓說字叔儒,會稽山陰人也。博通五經,尤善圖緯之學。舉孝廉。與議郎蔡邕友善。數陳災眚,及奏賦、頌、連珠。稍遷侍中。光和元年十月,說言於靈帝,云其晦日必食,乞百官嚴裝。帝從之,果如所言。中平二年二月,又上封事,剋期宮中有災。至日南宮大火。遷說江夏太守,公事免。年七十,卒於家。

董扶字茂安,廣漢綿竹人也。少遊太學,與鄉人任安齊名,俱事同郡楊厚,學圖讖。還家講授,弟子自遠而至。前後宰府十辟,公車三徵,再舉賢良方正、博士、有道,皆稱疾不就。

靈帝時,大將軍何進薦扶,徵拜侍中,甚見器重。扶私謂太常劉焉曰:「京師將亂,益州分野有天子氣。」焉信之,遂求出為益州牧,扶亦為蜀郡屬國都尉,相與入蜀。去後一歲,帝崩,天下大亂,乃去官還家。年八十二卒。

後劉備稱天子於蜀,皆如扶言。蜀丞相諸葛亮問廣漢秦密,董扶及任安所長。密曰「董扶褒秋毫之善,貶纖介之惡。任安記人之善,忘人之過」云。

郭玉者,廣漢雒人也。初,有老父不知何出,常漁釣於涪水,因號涪翁。乞食人閒,見有疾者,時下針石,輒應時而效,乃著針經、診脈法傳於世。弟子程高尋求積年,翁乃授之。高亦隱跡不仕。玉少師事高,學方診六微之技,陰陽隱側之術。和帝時,為太醫丞,多有效應。帝奇之,仍試令嬖臣美手腕者與女子雜處帷中,使玉各診一手,問所疾苦。玉曰:「左陽右陰,脈有男女,狀若異人。臣疑其故。」帝歎息稱善。

玉仁愛不矜,雖貧賤廝養,必盡其心力,而醫療貴人,時或不愈。帝乃令貴人羸服變處,一針即差。召玉詰問其狀。對曰:「醫之為言意也。腠理至微,隨氣用巧,針石之閒,毫芒即乖。神存於心手之際,可得解而不可得言也。夫貴者處尊高以臨臣,臣懷怖懾以承之。其為療也,有四難焉:自用意而不任臣,一難也;將身不謹,二難也;骨節不彊,不能使藥,三難也;好逸惡勞,四難也。針有分寸,時有破漏,重以恐懼之心,加以裁慎之志,臣意且猶不盡,何有於病哉!此其所為不愈也。」帝善其對。年老卒官。

華佗字元化,沛國譙人也,一名归。遊學徐土,兼通數經。曉養性之術,年且百歲而猶有壯容,時人以為仙。沛相陳珪舉孝廉,太尉黃琬辟,皆不就。

精於方藥,處齊不過數種,心識分銖,不假稱量。針灸不過數處。若疾發結於內,針藥所不能及者,乃令先以酒服麻沸散,既醉無所覺,因刳破腹背,抽割積聚。若在腸胃,則斷截湔洗,除去疾穢,既而縫合,傅以神膏,四五日創愈,一月之閒皆平復。

佗嘗行道,見有病咽塞者,因語之曰:「向來道隅有賣餅人,萍齏甚酸,可取三升飲之,病自當去。」即如佗言,立吐一蛇,乃懸於車而候佗。時佗小兒戲於門中,逆見,自相謂曰:「客車邊有物,必是逢我翁也。」及客進,顧視壁北,懸蛇以十數,乃知其奇。

又有一郡守篤病久,佗以為盛怒則差。乃多受其貨而不加功。無何棄去,又留書罵之。太守果大怒,令人追殺佗,不及,因瞋恚,吐黑血數升而愈。

又有疾者,詣佗求療,佗曰:「君病根深,應當剖破腹。然君壽亦不過十年,病不能相殺也。」病者不堪其苦,必欲除之,佗遂下療,應時愈,十年竟死。

廣陵太守陳登忽患匈中煩懣,面赤,不食。佗脈之,曰:「府君胃中有蟲,欲成內疽,腥物所為也。」即作湯二升,再服,須臾,吐出三升許蟲,頭赤而動,半身猶是生魚膾,所苦便愈。佗曰:「此病後三期當發,遇良醫可救。」登至期疾動,時佗不在,遂死。

曹操聞而召佗,常在左右。操積苦頭風眩,佗針,隨手而差。

有李將軍者,妻病,呼佗視脈。佗曰:「傷身而胎不去。」將軍言閒實傷身,胎已去矣。佗曰:「案脈,胎未去也。」將軍以為不然。妻稍差百餘日復動,更呼佗。佗曰:「脈理如前,是兩胎,先生者去,血多,故後兒不得出也。胎既已死,血脈不復歸,必燥著母脊。」乃為下針,并令進湯。婦因欲產而不通。佗曰:「死胎枯燥,埶不自生。」使人探之,果得死始,人形可識,但其色已黑。佗之絕技,皆此類也。

為人性惡難得意,且恥以醫見業,又去家思歸,乃就操求還取方,因託妻疾,數期不反。操累書呼之,又敕郡縣發遣,佗恃能厭事,猶不肯至。操大怒,使人廉之,知妻詐疾,乃收付獄訊,考驗首服。荀彧請曰:「佗方術實工,人命所懸,宜加全宥。」操不從,竟殺之。佗臨死,出一卷書與獄吏,曰:「此可以活人。」吏畏法不敢受,佗不強與,索火燒之。

初,軍吏李成苦欬,晝夜不寐。佗以為腸帻,與散兩錢服之,即吐二升膿血,於此漸愈。乃戒之曰:「後十八歲,疾當發動,若不得此藥,不可差也。」復分散與之。後五六歲,有里人如成先病,請藥甚急,成愍而與之,乃故往譙更從佗求,適值見收,意不忍言。後十八年,成病發,無藥而死。

廣陵吳普、彭城樊阿皆從佗學。普依準佗療,多所全濟。

佗語普曰:「人體欲得勞動,但不當使極耳。動搖則穀氣得銷,血脈流通,病不得生,譬猶戶樞,終不朽也。是以古之仙者為導引之事,熊經鴟顧,引挽腰體,動諸關節,以求難老。吾有一術,名五禽之戲:一曰虎,二曰鹿,三曰熊,四曰猿,五曰鳥。亦以除疾,兼利蹄足,以當導引。體有不快,起作一禽之戲,怡而汗出,因以著粉,身體輕便而欲食。」普施行之,年九十餘,耳目聰明,齒牙完堅。

阿善針術。凡醫咸言背及匈藏之閒不可妄針,針之不可過四分,而阿針背入一二寸,巨闕匈藏乃五六寸,而病皆瘳。阿從佗求方可服食益於人者,佗授以漆葉青办散:漆葉屑一斗,青办十四兩,以是為率。言久服,去三蟲,利五藏,輕體,使人頭不白。阿從其言,壽百餘歲。漆葉處所而有。青办生於豐、沛、彭城及朝歌閒。

漢世異術之士甚眾,雖云不經,而亦有不可誣,故簡其美者列于傳末:

泠壽光、唐虞、魯女生三人者,皆與華佗同時。壽光年可百五六十歲,行容成公御婦人法,常屈頸鷮息,須髮盡白,而色理如三四十時,死於江陵。唐虞道赤眉、張步家居里落,若與相及,死於鄉里不其縣。魯女生數說顯宗時事,甚明了,議者疑其時人也。董卓亂後,莫知所在。

徐登者,閩中人也。本女子,化為丈夫。善為巫術。又趙炳,字公阿,東陽人,能為越方。時遭兵亂,疾疫大起,二人遇於烏傷溪水之上,遂結言約,共以其術療病。各相謂曰:「今既同志,且可各試所能。」登乃禁溪水。水為不流,炳復次禁枯樹,樹即生荑,二人相視而笑,共行其道焉。

登年長,炳師事之。貴尚清儉,禮神唯以東流水為酌,削桑皮為脯。但行禁架,所療皆除。

後登物故,炳東入章安,百姓未之知也。炳乃故升茅屋,梧鼎而爨,主人見之驚懅,炳笑不應,既而爨孰,屋無損異。又嘗臨水求度,船人不和之,炳乃張蓋坐其中,長嘯呼風,亂流而濟。於是百姓神服,從者如歸。章安令惡其惑眾,收殺之。人為立祠室於永康,至今蚊蚋不能入也。

費長房者,汝南人也。曾為市掾。市中有老翁賣藥,懸一壺於肆頭,及市罷,輒跳入壺中。市人莫之見,唯長房於樓上睹之,異焉,因往再拜奉酒脯。翁知長房之意其神也,謂之曰:「子明日可更來。」長房旦日復詣翁,翁乃與俱入壺中。唯見玉堂嚴麗,旨酒甘肴盈衍其中,共飲畢而出。翁約不聽與人言之。後乃就樓上候長房曰:「我神仙之人,以過見責,今事畢當去,子寧能相隨乎?樓下有少酒,與卿為別。」長房使人取之,不能勝,又令十人扛之,猶不舉。翁聞,笑而下樓,以一指提之而上。視器如一升許,而二人飲之終日不盡。

長房遂欲求道,而顧家人為憂。翁乃斷一青竹,度與長房身齊,使懸之舍後。家人見之,即長房形也,以為縊死,大小驚號,遂殯葬之。長房立其傍,而莫之見也。於是遂隨從入深山,踐荊棘於群虎之中。留使獨處,長房不恐。又臥於空室,以朽索懸萬斤石於心上,眾蛇競來齧索且斷,長房亦不移。翁還,撫之曰:「子可教也。」復使食糞,糞中有三蟲,臭穢特甚,長房意惡之。翁曰:「子幾得道,恨於此不成,如何!」

長房辭歸,翁與一竹杖,曰:「騎此任所之,則自至矣。既至,可以杖投葛陂中也。」又為作一符,曰:「以此主地上鬼神。」長房乘杖,須臾來歸,自謂去家適經旬日,而已十餘年矣。即以杖投陂,顧視則龍也。家人謂其久死,不信之。長房曰:「往日所葬,但竹杖耳。」乃發冢剖棺,杖猶存焉。遂能醫療眾病,鞭笞百鬼,及驅使社公。或在它坐,獨自恚怒,人問其故,曰:「吾責鬼魅之犯法者耳。」

汝南歲歲常有魅,偽作太守章服,詣府門椎鼓者,郡中患之。時魅適來,而逢長房謁府君,惶懼不得退,便前解衣冠,叩頭乞活。長房呵之云:「便於中庭正汝故形!」即成老鱉,大如車輪,頸長一丈。長房復令就太守服罪,付其一札,以敕葛陂君。魅叩頭流涕,持札植於陂邊,以頸繞之而死。

後東海君來見葛陂君,因淫其夫人,於是長房劾繫之三年,而東海大旱。長房至海上,見其人請雨,乃謂之曰:「東海君有罪,吾前繫於葛陂,今方出之使作雨也。」於是雨立注。

長房曾與人共行,見一書生黃巾被裘,無鞍騎馬,下而叩頭。長房曰:「還它馬,赦汝死罪。」人問其故,長房曰:「此狸也,盜社公馬耳。」又嘗坐客,而使至宛市鮓,須臾還,乃飯。或一日之閒,人見其在千里之外者數處焉。

後失其符,為眾鬼所殺。

薊子訓者,不知所由來也。建安中,客在濟陰宛句。有神異之道。嘗抱鄰家嬰兒,故失手墯地而死,其父母驚號怨痛,不可忍聞,而子訓唯謝以過誤,終無它說,遂埋藏之。後月餘,子訓乃抱兒歸焉。父母大恐,曰:「死生異路,雖思我兒,乞不用復見也。」兒識父母,軒渠笑悅,欲往就之,母不覺攬取,乃實兒也。雖大喜慶,心猶有疑,乃竊發視死兒,但見衣被,方乃信焉。於是子訓流名京師,士大夫皆承風向慕之。

後乃駕驢車,與諸生俱詣許下。道過滎陽,止主人舍,而所駕之驢忽然卒僵,蛆蟲流出,主遽白之。子訓曰:「乃爾乎?」方安坐飯,食畢,徐出以杖扣之,驢應聲奮起,行步如初,即復進道。其追逐觀者常有千數。既到京師,公卿以下候之者,坐上恆數百人,皆為設酒脯,終日不匱。

後因遁去,遂不知所止。初去之日,唯見白雲騰起,從旦至暮,如是數十處。時有百歲翁,自說童兒時見子訓賣藥於會稽市,顏色不異於今。後人復於長安東霸城見之,與一老公共摩挲銅人,相謂曰:「適見鑄此,已近五百歲矣。」顧視見人而去,猶駕昔所乘驢車也。見者呼之曰:「薊先生小住。」並行應之,視若遲徐,而走馬不及,於是而絕。

劉根者,潁川人也。隱居嵩山中。諸好事者自遠而至,就根學道,太守史祈以根為妖妄,乃收執詣郡,數之曰:「汝有何術,而誣惑百姓?若果有神,可顯一驗事。不爾,立死矣。」根曰:「實無它異,頗能令人見鬼耳。」祈曰:「促召之,使太守目睹,爾乃為明。」根於是左顧而嘯,有頃,祈之亡父祖近親數十人,皆反縛在前,向根叩頭曰:「小兒無狀,分當萬坐。」顧而叱祈曰:「汝為子孫,不能有益先人,而反累辱亡靈!可叩頭為吾陳謝。」祈驚懼悲哀,頓首流血,請自甘罪坐。根嘿而不應,忽然俱去,不知在所。

左慈字元放,廬江人也。少有神道。嘗在司空曹操坐,操從容顧眾賓曰:「今日高會,珍羞略備,所少吳松江鱸魚耳。」放於下坐應曰:「此可得也。」因求銅盤貯水,以竹竿餌釣於盤中,須臾引一鱸魚出。操大拊掌笑,會者皆驚。操曰:「一魚不周坐席,可更得乎?」放乃更餌鉤沈之,須臾復引出,皆長三尺餘,生鮮可愛。操使目前鱠之,周浹會者。操又謂曰:「既已得魚,恨無蜀中生薑耳。」放曰:「亦可得也。」操恐其近即所取,因曰:「吾前遣人到蜀買錦,可過敕使者,增市二端。」語頃,即得薑還,并獲操使報命。後操使蜀反,驗問增錦之狀及時日早晚,若符契焉。

後操出近郊,士大夫從者百許人,慈乃為齎酒一升,脯一斤,手自斟酌,百官莫不醉飽。操怪之,使尋其故,行視諸鑪,悉亡其酒脯矣。操懷不喜,因坐上收欲殺之,慈乃卻入壁中,霍然不知所在。或見於市者,又捕之,而市人皆變形與慈同,莫知誰是。後人逢慈於陽城山頭,因復逐之,遂入走羊群。操知不可得,乃令就羊中告之曰:「不復相殺,本試君術耳。」忽有一老羝屈前兩膝,人立而言曰:「遽如許。」即競往赴之,而群羊數百皆變為羝,並屈前膝人立,云「遽如許」,遂莫知所取焉。

計子勳者,不知何郡縣人。皆謂數百歲,行來於人閒。一旦忽言日中當死,主人與之葛衣,子勳服而正寢,至日中果死。

上成公者,宓縣人也。其初行久而不還,後歸,語其家云:「我已得仙。」因辭家而去。家人見其舉步稍高,良久乃沒云。陳寔、韓韶同見其事。

解奴辜、張貂者,亦不知是何郡國人也。皆能隱淪,出入不由門戶。奴辜能變易物形,以誑幻人。

又河南有麴聖卿,善為丹書符劾,厭殺鬼神而使命之。

又有編盲意,亦與鬼物交通。

初,章帝時有壽光侯者,能劾百鬼眾魅,令自縛見形。其鄉人有婦為魅所病,侯為劾之,得大蛇數丈,死於門外。又有神樹,人止者輒死,鳥過者必墜,侯復劾之,樹盛夏枯落,見大蛇長七八丈,懸死其閒。帝聞而徵之。乃試問之:「吾殿下夜半後,常有數人絳衣被髮,持火相隨,豈能劾之乎?」侯曰:「此小怪,易銷耳。」帝偽使三人為之,侯劾三人,登時仆地無氣。帝大驚曰:「非魅也,朕相試耳。」解之而蘇。

甘始、東郭延年、封君達三人者,皆方士也。率能行容成御婦人術,或飲小便,或自倒懸,愛嗇精氣,不極視大言。甘始、元放、延年皆為操所錄,問其術而行之。君達號「青牛師」。凡此數人,皆百餘歲及二百歲也。

王真、郝孟節者,皆上黨人也。王真年且百歲,視之面有光澤,似未五十者。自云:「周流登五岳名山,悉能行胎息胎食之方,嗽舌下泉咽之,不絕房室。」孟節能含棗核,不食可至五年十年。又能結氣不息,身不動搖,狀若死人,可至百日半年。亦有室家。為人質謹不妄言,似士君子。曹操使領諸方士焉。

北海王和平,性好道術,自以當仙。濟南孫邕少事之,從至京師。會和平病歿,邕因葬之東陶。有書百餘卷,藥數囊,悉以送之。後弟子夏榮言其尸解,邕乃恨不取其寶書仙藥焉。

贊曰:幽貺罕徵,明數難校。不探精遠,曷感靈效?如或遷訛,實乖玄奧。


\end{pinyinscope}