\article{皇甫嵩朱雋列傳}

\begin{pinyinscope}
皇甫嵩字義真,安定朝那人,度遼將軍規之兄子也。父節,鴈門太守。嵩少有文武志介,好詩書,習弓馬。初舉孝廉、茂才。太尉陳蕃、大將軍竇武連辟,並不到。靈帝公車徵為議郎,遷北地太守。

初,鉅鹿張角自稱「大賢良師」,奉事黃老道,畜養弟子,跪拜首過,符水咒說以療病,病者頗愈,百姓信向之。角因遣弟子八人使於四方,以善道教化天下,轉相誑惑。十餘年閒,眾徒數十萬,連結郡國,自青、徐、幽、冀、荊、楊、兗、豫八州之人,莫不畢應。遂置三十六方。方猶將軍號也。大方萬餘人,小方六七千,各立渠帥。訛言「蒼天已死,黃天當立,歲在甲子,天下大吉」。以白土書京城寺門及州郡官府,皆作「甲子」字。中平元年,大方馬元義等先收荊、楊數萬人,期會發於鄴。元義數往來京師,以中常侍封諝、徐奉等為內應,約以三月五日內外俱起。未及作亂,而張角弟子濟南唐周上書告之,於是車裂元義於洛陽。靈帝以周章下三公、司隸,使鉤盾令周斌將三府掾屬,案驗宮省直衛及百姓有事角道者,誅殺千餘人,推考冀州,逐捕角等。角等知事已露,晨夜馳敕諸方,一時俱起。皆著黃巾為摽幟,時人謂之「黃巾」,亦名為「蛾賊」。殺人以祠天。角稱「天公將軍」,角弟寶稱「地公將軍」,寶弟梁稱「人公將軍」,所在燔燒官府,劫略聚邑,州郡失據,長吏多逃亡。旬日之閒,天下嚮應,京師震動。

詔敕州郡修理攻守,簡練器械,自函谷、大谷、廣城、伊闕、轘轅、旋門、孟津、小平津諸關,並置都尉。召群臣會議。嵩以為宜解黨禁,益出中藏錢、西園廄馬,以班軍士。帝從之。於是發天下精兵,博選將帥,以嵩為左中郎將,持節,與右中郎將朱雋,共發五校、三河騎士及募精勇,合四萬餘人,嵩、雋各統一軍,共討潁川黃巾。

雋前與賊波才戰,戰敗,嵩因進保長社。波才引大眾圍城,嵩兵少,軍人皆恐,乃召軍吏謂曰:「兵有奇變,不在眾寡。今賊依草結營,易為風火。若因夜縱燒,必大驚亂。吾出兵擊之,四面俱合,田單之功可成也。」其夕遂大風,嵩乃約敕軍士皆束苣乘城,使銳士閒出圍外,縱火大呼,城上舉燎應之,嵩因鼓而奔其陳,賊驚亂奔走。會帝遣騎都尉曹操將兵適至,嵩、操與朱雋合兵更戰,大破之,斬首數萬級。封嵩都鄉侯。嵩、雋乘勝進討汝南、陳國黃巾,追波才於陽翟,擊彭脫於西華,並破之。餘賊降散,三郡悉平。

又進擊東郡黃巾卜己於倉亭,生禽卜己,斬首七千餘級。時北中郎將盧植及東中郎將董卓討張角,並無功而還,乃詔嵩進兵討之。嵩與角弟梁戰於廣宗。梁眾精勇,嵩不能剋。明日,乃閉營休士,以觀其變。知賊意稍懈,乃潛夜勒兵,雞鳴馳赴其陳,戰至晡時,大破之,斬梁,獲首三萬級,赴河死者五萬許人,焚燒車重三萬餘兩,悉虜其婦子,繫獲甚眾。角先已病死,乃剖棺戮屍,傳首京師。

嵩復與鉅鹿太守馮翊郭典攻角弟寶於下曲陽,又斬之。首獲十餘萬人,築京觀於城南。即拜嵩為左車騎將車,領冀州牧,封槐里侯,食槐里、美陽兩縣,合八千戶。

以黃巾既平,故改年為中平。嵩奏請冀州一年田租,以贍飢民,帝從之。百姓歌曰:「天下大亂兮市為墟,母不保子兮妻失夫,賴得皇甫兮復安居。」嵩溫卹士卒,甚得眾情,每軍行頓止,須營幔修立,然後就舍帳。軍士皆食,爾乃嘗飯。吏有因事受賂者,嵩更以錢物賜之,吏懷慚,或至自殺。

嵩既破黃巾,威震天下,而朝政日亂,海內虛困。故信都令漢陽閻忠干說嵩曰:「難得而易失者,時也;時至不旋踵者,幾也。故聖人順時以動,智者因幾以發。今將軍遭難得之運,蹈易駭之機,而踐運不撫,臨機不發,將何以保大名乎?」嵩曰:「何謂也?」忠曰:「天道無親,百姓與能。今將軍受鉞於暮春,收功於末冬。兵動若神,謀不再計,摧強易於折枯,消堅甚於湯雪,旬月之閒,神兵電埽,封尸刻石,南向以報,威德震本朝,風聲馳海外,雖湯武之舉,未有高將軍者也。今身建不賞之功,體兼高人之德,而北面庸主,何以求安乎?」嵩曰:「夙夜在公,心不忘忠,何故不安?」忠曰:「不然。昔韓信不忍一餐之遇,而棄三分之業,利劍已揣其喉,方發悔毒之歎者,機失而謀乖也。今主上埶弱於劉、項,將軍權重於淮陰,指撝足以振風雲,叱吒可以興雷電。赫然奮發,因危扺穨,崇恩以綏先附,振武以臨後服,徵冀方之士,動七州之眾,羽檄先馳於前,大軍響振於後,蹈流漳河,飲馬孟津,誅閹官之罪,除群凶之積,雖僮兒可使奮拳以致力,女子可使褰裳以用命,況厲熊羆之卒,因迅風之埶哉!功業已就,天下已順,然後請呼上帝,示以天命,混齊六合,南面稱制,移寶器於將興,推亡漢於已墜,實神機之至會,風發之良時也。夫既朽不雕,衰世難佐。若欲輔難佐之朝,雕朽敗之木,是猶逆阪走丸,迎風縱棹,豈云易哉?且今豎宦群居,同惡如市,上命不行,權歸近習,昏主之下,難以久居,不賞之功,讒人側目,如不早圖,後悔無及。」嵩懼曰:「非常之謀,不施於有常之埶。創圖大功,豈庸才所致。黃巾細孽,敵非秦、項,新結易散,難以濟業。且人未忘主,天不祐逆。若虛造不冀之功,以速朝夕之禍,孰與委忠本朝,守其臣節。雖云多讒,不過放廢,猶有令名,死且不朽。反常之論,所不敢聞。」忠知計不用,因亡去。

會邊章、韓遂作亂隴右,明年春,詔嵩迴鎮長安,以衛園陵。章等遂復入寇三輔,使嵩因討之。

初,嵩討張角,路由鄴,見中常侍趙忠舍宅踰制,乃奏沒入之。又中常侍張讓私求錢五千萬,嵩不與,二人由此為憾,奏嵩連戰無功,所費者多。其秋徵還,收左車騎將軍印綬,削戶六千,更封都鄉侯,二千戶。

五年,梁州賊王國圍陳倉,復拜嵩為左將軍,督前將軍董卓,各率二萬人拒之。卓欲速進赴陳倉,嵩不聽。卓曰:「智者不後時,勇者不留決。速救則城全,不救則城滅,全滅之埶,在於此也。」嵩曰:「不然。百戰百勝,不如不戰而屈人之兵。是以先為不可勝,以待敵之可勝。不可勝在我,可勝在彼。彼守不足,我攻有餘。有餘者動於九天之上,不足者陷於九地之下。今陳倉雖小,城守固備,非九地之陷也。王國雖強,而攻我之所不救,非九天之埶也。夫埶非九天,攻者受害;陷非九地,守者不拔。國今已陷受害之地,而陳倉保不拔之城,我可不煩兵動眾,而取全勝之功,將何救焉!」遂不聽。王國圍陳倉,自冬迄春,八十餘日,城堅守固,竟不能拔。賊眾疲敝,果自解去。嵩進兵擊之。卓曰:「不可。兵法,窮寇勿迫,歸眾勿追。今我追國,是迫歸眾,追窮寇也。困獸猶鬥,蜂蠆有毒,況大眾乎!」嵩曰:「不然。前吾不擊,避其銳也。今而擊之,待其衰也。所擊疲師,非歸眾也。國眾且走,莫有鬥志。以整擊亂,非窮寇也。」遂獨進擊之,使卓為後拒。連戰大破之,斬首萬餘級,國走而死。卓大慚恨,由是忌嵩。

明年,卓拜為并州牧,詔使以兵委嵩,卓不從。嵩從子酈時在軍中,說嵩曰:「本朝失政,天下倒懸,能安危定傾者,唯大人與董卓耳。今怨隙已結,埶不俱存。卓被詔委兵,而上書自請,此逆命也。又以京師昏亂,躊躇不進,此懷姦也。且其凶戾無親,將士不附。大人今為元帥,杖國威以討之,上顯忠義,下除凶害,此桓文之事也。」嵩曰:「專命雖罪,專誅亦有責也。不如顯奏其事,使朝廷裁之。」於是上書以聞。帝讓卓,卓又增怨於嵩。及後秉政,初平元年,乃徵嵩為城門校尉,因欲殺之。嵩將行,長史梁衍說曰:「漢室微弱,閹豎亂朝,董卓雖誅之,而不能盡忠於國,遂復寇掠京邑,廢立從意。今徵將軍,大則危禍,小則困辱。今卓在洛陽,天子來西,以將軍之眾,精兵三萬,迎接至尊,奉令討逆,發命海內,徵兵群帥,袁氏逼其東,將軍迫其西,此成禽也。」嵩不從,遂就徵。有司承旨,奏嵩下吏,將遂誅之。

嵩子堅壽與卓素善,自長安亡走洛陽,歸投於卓。卓方置酒歡會,堅壽直前質讓,責以大義,叩頭流涕。坐者感動,皆離席請之。卓乃起,牽與共坐。使免嵩囚,復拜嵩議郎,遷御史中丞。及卓還長安,公卿百官迎謁道次。卓風令御史中丞已下皆拜以屈嵩,既而扺手言曰:「義真犕未乎?」嵩笑而謝之,卓乃解釋。

及卓被誅,以嵩為征西將軍,又遷車騎將軍。其年秋,拜太尉,冬,以流星策免。復拜光祿大夫,遷太常。尋李傕作亂,嵩亦病卒,贈驃騎將軍印綬,拜家一人為郎。

嵩為人愛慎盡勤,前後上表陳諫有補益者五百餘事,皆手書毀草,不宣於外。又折節下士,門無留客。時人皆稱而附之。

堅壽亦顯名,後為侍中,辭不拜,病卒。

朱雋字公偉,會稽上虞人也。少孤,母嘗販繒為業。雋以孝養致名,為縣門下書佐,好義輕財,鄉閭敬之。時同郡周規辟公府,當行,假郡庫錢百萬,以為冠幘費,而後倉卒督責,規家貧無以備,雋乃竊母繒帛,為規解對。母既失產業,深恚責之。雋曰:「小損當大益,初貧後富,必然理也。」

本縣長山陽度尚見而奇之,薦於太守韋毅,稍歷郡職。後太守尹端以雋為主簿。熹平二年,端坐討賊許昭失利,為州所奏,罪應棄市。雋乃羸服閒行,輕齎數百金到京師,賂主章吏,遂得刊定州奏,故端得輸作左校。端喜於降免而不知其由,雋亦終無所言。

後太守徐珪舉雋孝廉,再遷除蘭陵令,政有異能,為東海相所表。會交阯部群賊並起,牧守軟弱不能禁。又交阯賊梁龍等萬餘人,與南海太守孔芝反叛,攻破郡縣。光和元年,即拜雋交阯刺史,令過本郡簡募家兵及所調,合五千人,分從兩道而入。既到州界,按甲不前,先遣使詣郡,觀賊虛實,宣揚威德,以震動其心;既而與七郡兵俱進逼之,遂斬梁龍,降者數萬人,旬月盡定。以功封都亭侯,千五百戶,賜黃金五十斤,徵為諫議大夫。

及黃巾起,公卿多薦雋有才略,拜為右中郎將,持節,與左中郎將皇甫嵩討潁川、汝南、陳國諸賊,悉破平之。嵩乃上言其狀,而以功歸雋,於是進封西鄉侯,遷鎮賊中郎將。

時南陽黃巾張曼成起兵,稱「神上使」,眾數萬,殺郡守褚貢,屯宛下百餘日。後太守秦頡擊殺曼成,賊更以趙弘為帥,眾浸盛,遂十餘萬,據宛城。雋與荊州刺史徐璆及秦頡合兵萬八千人圍弘,自六月至八月不拔。有司奏欲徵雋。司空張溫上疏曰:「昔秦用白起,燕任樂毅,皆曠年歷載,乃能克敵。雋討潁川,以有功效,引師南指,方略已設,臨軍易將,兵家所忌,宜假日月,責其成功。」靈帝乃止。雋因急擊弘,斬之。賊餘帥韓忠復據宛拒雋。雋兵少不敵,乃張圍結壘,起土山以臨城內,因鳴鼓攻其西南,賊悉眾赴之。雋自將精卒五千,掩其東北,乘城而入。忠乃退保小城,惶懼乞降。司馬張超及徐璆、秦頡皆欲聽之。雋曰:「兵有形同而埶異者。昔秦項之際,民無定主,故賞附以勸來耳。今海內一統,唯黃巾造寇,納降無以勸善,討之足以懲惡。今若受之,更開逆意,賊利則進戰,鈍則乞降,縱敵長寇,非良計也。」因急攻,連戰不剋。雋登土山望之,顧謂張超曰:「吾知之矣。賊今外圍周固,內營逼急,乞降不受,欲出不得,所以死戰也。萬人一心,猶不可當,況十萬乎!其害甚矣。不如徹圍,并兵入城。忠見圍解,埶必自出,出則意散,易破之道也。」既而解圍,忠果出戰,雋因擊,大破之。乘勝逐北數十里,斬首萬餘級。忠等遂降。而秦頡積忿忠,遂殺之。餘眾懼不自安,復以孫夏為帥,還屯宛中。雋急攻之。夏走,追至西鄂精山,又破之。復斬萬餘級,賊遂解散。明年春,遣使者持節拜雋右車騎將軍,振旅還京師,以為光祿大夫,增邑五千,更封錢塘侯,加位特進。以母喪去官,起家,復為將作大匠,轉少府、太僕。

自黃巾賊後,復有黑山、黃龍、白波、左校、郭大賢、于氐根、青牛角、張白騎、劉石、左髭丈八、平漢、大計、司隸、掾哉、雷公、浮雲、飛燕、白雀、楊鳳、于毒、五鹿、李大目、白繞、畦固、苦唒之徒,並起山谷閒,不可勝數。其大聲者稱雷公,騎白馬者為張白騎,輕便者言飛燕,多髭者號于氐根,大眼者為大目,如此稱號,各有所因。大者二三萬,小者六七千。

賊帥常山人張燕,輕勇趫捷,故軍中號曰飛燕。善得士卒心,乃與中山、常山、趙郡、上黨、河內諸山谷寇賊更相交通,眾至伯萬,號曰黑山賊。河北諸郡縣並被其害,朝廷不能討。燕乃遣使至京師,奏書乞降,遂拜燕平難中郎將,使領河北諸山谷事,歲得舉孝廉、計吏。

燕後漸寇河內,逼近京師,於是出雋為河內太守,將家兵擊卻之。其後諸賊多為袁紹所定,事在紹傳。復拜雋為光祿大夫,轉屯騎,尋拜城門校尉、河南尹。

時董卓擅政,以雋宿將,外甚親納而心實忌之。及關東兵盛,卓懼,數請公卿會議,徙都長安,雋輒止之。卓雖惡雋異己,然貪其名重,乃表遷太僕,以為己副。使者拜,雋辭不肯受。因曰:「國家西遷,必孤天下之望,以成山東之釁,臣不見其可也。」使者詰曰:「召君受拜而君拒之,不問徙事而君陳之,其故何也?」雋曰:「副相國,非臣所堪也;遷都計,非事所急也。辭所不堪,言所非急,臣之宜也。」使者曰:「遷都之事,不聞其計,就有未露,何所承受?」雋曰:「相國董卓具為臣說,所以知耳。」使人不能屈,由是止不為副。

卓後入關,留雋守洛陽,而雋與山東諸將通謀為內應。既而懼為卓所襲,乃棄官奔荊州。卓以弘農楊懿為河南尹,守洛陽。雋聞,復進兵還洛,懿走。雋以河南殘破無所資,乃東屯中牟,移書州郡,請師討卓。徐州刺史陶謙遣精兵三千,餘州郡稍有所給,謙乃上雋行車騎將軍。董卓聞之,使其將李傕、郭汜等數萬人屯河南拒雋。雋逆擊,為傕、汜所破。雋自知不敵,留關下不敢復前。

及董卓被誅,傕、汜作亂,雋時猶在中牟。陶謙以雋名臣,數有戰功,可委以大事,乃與諸豪桀共推雋為太師,因移檄牧伯,同討李傕等,奉迎天子。乃奏記於雋曰:「徐州刺史陶謙、前楊州刺史周乾、琅邪相陰德、東海相劉馗、彭城相汲廉、北海相孔融、沛相袁忠、太山太守應劭、汝南太守徐璆、前九江太守服虔、博士鄭玄等,敢言之行車騎將軍河南尹莫府:國家既遭董卓,重以李傕、郭汜之禍,幼主劫執,忠良殘敝,長安隔絕,不知吉凶。是以臨官尹人,搢紳有識,莫不憂懼,以為自非明哲雄霸之士,曷能剋濟禍亂!自起兵已來,于茲三年,州郡轉相顧望,未有奮擊之功,而互爭私變,更相疑惑。謙等並共諮諏,議消國難。僉曰:『將軍君侯,既文且武,應運而出,凡百君子,靡不顒顒。』故相率厲,簡選精悍,堪能深入,直指咸陽,多持資糧,足支半歲,謹同心腹,委之元帥。」會李傕用太尉周忠、尚書賈詡策,徵雋入朝。軍吏皆憚入關,欲應陶謙等。雋曰:「以君召臣,義不俟駕,況天子詔乎!且傕、汜小豎,樊稠庸兒,無他遠略,又埶力相敵,變難必作。吾乘其閒,大事可濟。」遂辭謙議而就傕徵,復為太僕,謙等遂罷。

初平四年,代周忠為太尉,錄尚書事。明年秋,以日食免,復行驃騎將軍事,持節鎮關東。未發,會李傕殺樊稠,而郭汜又自疑,與傕相攻,長安中亂,故雋止不出,留拜大司農。獻帝詔雋與太尉楊彪等十餘人譬郭汜,令與李傕和。汜不肯,遂留質雋等。雋素剛,即日發病卒。

子皓,亦有才行,官至豫章太守。

論曰:皇甫嵩、朱雋並以上將之略,受脤倉卒之時。及其功成師剋,威聲滿天下。值弱主蒙塵,獷賊放命,斯誠葉公投袂之幾,翟義鞠旅之日,故梁衍獻規,山東連盟,而舍格天之大業,蹈匹夫之小諒,卒狼狽虎口,為智士笑。豈天之長斯亂也?何智勇之不終甚乎!前史晉平原華嶠,稱其父光祿大夫表,每言其祖魏太尉歆稱「時人說皇甫嵩之不伐,汝豫之戰,歸功朱雋,張角之捷,本之於盧植,收名斂策,而己不有焉。蓋功名者,世之所甚重也。誠能不爭天下之所甚重,則怨禍不深矣」。如皇甫公之赴履危亂,而能終以歸全者,其致不亦貴乎!故顏子願不伐善為先,斯亦行身之要與!

贊曰:黃妖衝發,嵩乃奮鉞。孰是振旅,不居不伐。雋捷陳、潁,亦弭于越。言肅王命,並遘屯蹶。


\end{pinyinscope}