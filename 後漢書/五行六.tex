\article{五行六}

\begin{pinyinscope}
日中黑虹貫日月蝕非其月

光武帝建武二年正月甲子朔,日有蝕之。在危八度。日蝕說曰:「日者,太陽之精,人君之象。君道有虧,為陰所乘,故蝕。蝕者,陽不克也。」其候雜說,漢書五行志著之必矣。儒說諸侯專權,則其應多在日所宿之國。諸象附從,則多為王者事。人君改修其德,則咎害除。是時世祖初興,天下賊亂未除。虛、危,齊也。賊張步擁兵據齊,上遣伏隆諭步,許降,旋復叛稱王,至五年中乃破。

三年五月乙卯晦,日有蝕之,在柳十四度。柳,河南也。時世祖在雒陽,赤眉降賊樊崇謀作亂,其七月發覺,皆伏誅。

六年九月丙寅晦,日有蝕之。史官不見,郡以聞。在尾八度。

七年三月癸亥晦,日有蝕之,在畢五度。畢為邊兵。秋,隗囂反,侵安定。冬,盧芳所置朔方、雲中太守各舉郡降。

十六年三月辛丑晦,日有蝕之,在昴七度。昴為獄事。時諸郡太守坐度田不實,世祖怒,殺十餘人,然後深悔之。

十七年二月乙未晦,日有蝕之,在胃九度。胃為廩倉。時諸郡新坐租之後,天下憂怖,以穀為言,故示象。或曰:胃,供養之官也。其十月,廢郭皇后,詔曰「不可以奉供養」。

二十二年五月乙未晦,日有蝕之,在柳七度,京都宿也。柳為上倉,祭祀穀也。近輿鬼,輿鬼為宗廟。十九年中,有司奏請立近帝四廟以祭之,有詔「廟處所未定,且就高廟祫祭之」。至此三年,遂不立廟。有簡墮心,奉祖宗之道有闕,故示象也。

二十五年三月戊申晦,日有蝕之,在畢十五度。畢為邊兵。其冬十月,以武谿蠻夷為寇害,伏波將軍馬援將兵擊之。

二十九年二月丁巳朔,日有蝕之,在東壁五度。東壁為文章,一名娵訾之口。先是皇子諸王各招來文章談說之士,去年中,有人上奏:「諸王所招待者,或真偽雜,受刑罰者子孫,宜可分別。」於是上怒,詔捕諸王客,皆被以苛法,死者甚多。世祖不早為明設刑禁,一時治之過差,故天示象。世祖於是改悔,遣使悉理侵枉也。

三十一年五月癸酉晦,日有蝕之,在柳五度,京都宿也。自二十一年示象至此十年,後二年,宮車晏駕。

中元元年十一月甲子晦,日有蝕之,在斗二十度。斗為廟,主爵祿。儒說十一月甲子,時王日也,又為星紀,主爵祿,其占重。

明帝永平三年八月壬申晦,日有蝕之,在氐二度。氐為宿宮。是時明帝作北宮。

八年十月壬寅晦,日有蝕之,既,在斗十一度。斗,吳也。廣陵於天文屬吳。後二年,廣陵王荊坐謀反自殺。

十三年十月甲辰晦,日有蝕之,在尾十七度。

十六年五月戊午晦,日有蝕之,在柳十五度。儒說五月戊午,猶十一月甲子也,又宿在京都,其占重。後二歲,宮車晏駕。

十八年十一月甲辰晦,日有蝕之,在斗二十一度。是時明帝既崩,馬太后制爵祿,故陽不勝。

章帝建初五年二月庚辰朔,日有蝕之,在東壁八度。例在前建武二十九年。是時群臣爭經,多相非毀者。

六年六月辛未晦,日有蝕之,在翼六度。翼主遠客。冬,東平王蒼等來朝,明年正月,蒼薨。

元和元年八月乙未晦,日有蝕之。史官不見,佗官以聞。日在氐四度。

和帝永元二年二月壬午,日有蝕之。史官不見,涿郡以聞。日在奎八度。

四年六月戊戌朔,日有蝕之,在七星二度,主衣裳。又曰行近軒轅,在左角,為太后族。是月十九日,上免太后兄弟竇憲等官,遣就國,選嚴能相,於國蹙迫自殺。

七年四月辛亥朔,日有蝕之,在觜觿,為葆旅,主收斂。儒說葆旅宮中之象,收斂貪妒之象。是歲鄧貴人始入。明年三月,陰皇后立,鄧貴人有寵,陰后妒忌之,後遂坐廢。一曰是將入參,參、伐為斬刈。明年七月,越騎校尉馮柱捕斬匈奴溫禺犢王烏居戰。

十二年秋七月辛亥朔,日有蝕之,在翼八度,荊州宿也。明年冬,南郡蠻夷反為寇。

十五年四月甲子晦,日有蝕之,在東井二十二度。東井,主酒食之宿也。婦人之職,無非無儀,酒食是議。去年冬,鄧皇后立,有丈夫之性,與知外事,故天示象。是年水,雨傷稼。

安帝永初元年三月二日癸酉,日有蝕之,在胃二度。胃主廩倉。是時鄧太后專政,去年大水傷稼,倉廩為虛。

五年正月庚辰朔,日有蝕之,在虛八度。正月,王者統事之正日也。虛,空名也。是時鄧太后攝政,安帝不得行事,俱不得其正,若王者位虛,故於正月陽不克,示象也。於是陰預乘陽,故夷狄並為寇害,西邊諸郡皆至虛空。

七年四月丙申晦,日有蝕之,在東井一度。

元初元年十月戊子朔,日有蝕之,在尾十度。尾為後宮,繼嗣之宮也。是時上甚幸閻貴人,將立,故示不善,將為繼嗣禍也。明年四月,遂立為后。後遂與江京、耿寶等共讒太子廢之。

二年九月壬午晦,日有蝕之,在心四度。心為王者,明久失位也。

三年三月二日辛亥,日有蝕之,在婁五度。史官不見,遼東以聞。

四年二月乙亥朔,日有蝕之,在奎九度。史官不見,七郡以聞。奎主武庫兵。其十月八日壬戌,武庫火,燒兵器也。

五年八月丙申朔,日有蝕之,在翼十八度。史官不見,張掖以聞。

六年十二月戊午朔,日有蝕之,幾盡,地如昏狀。在須女十一度,女主惡之。後二歲三月,鄧太后崩。

永寧元年七月乙酉朔,日有蝕之,在張十五度。史官不見,酒泉以聞。

延光三年九月庚寅晦,日有食之,在氐十五度。氐為宿宮。宮,中宮也。時上聽中常侍江京、樊豐及阿母王聖等讒言,廢皇太子。

四年三月戊午朔,日有蝕之,在胃十二度。隴西、酒泉、朔方各以狀上,史官不覺。

順帝永建二年七月甲戌朔,日有蝕之,在翼九度。

陽嘉四年閏月丁亥朔,日有蝕之,在角五度。史官不見,零陵以聞。

永和三年十二月戊戌朔,日有蝕之,在須女十一度。史官不見,會稽以聞。明年,中常侍張逵等謀譖皇后父梁商欲作亂,推考,逵等伏誅也。

五年五月己丑晦,日有蝕之,在東井三十三度。東并,三輔宿。又近輿鬼,輿鬼為宗廟。其秋,西羌為寇,至三輔陵園。

六年九月辛亥晦,日有蝕之,在尾十一度。尾主後宮,繼嗣之宮也。以為繼嗣不興之象。

桓帝建和元年正月辛亥朔,日有蝕之,在營室三度。史官不見,郡國以聞。是時梁太后攝政。

三年四月丁卯晦,日有蝕之,在東井二十三度。例在永元十五年。東井主法,梁太后又聽兄冀枉殺公卿,犯天法也。明年,太后崩。

元嘉二年七月二日庚辰,日有蝕之,在翼四度。史官不見,廣陵以聞。翼主倡樂。時上好樂過。

永興二年九月丁卯朔,日有蝕之,在角五度。角,鄭宿也。十一月,泰山盜賊群起,劫殺長吏。泰山於天文屬鄭。

永壽三年閏月庚辰晦,日有蝕之,在七星二度。史官不見,郡國以聞。例在永元四年。後二歲,梁皇后崩,冀兄弟被誅。

延熹元年五月甲戌晦,日有蝕之,在柳七度,京都宿也。

八年正月丙申晦,日有蝕之,在營室十三度。營室之中,女主象也。其二月癸亥,鄧皇后坐酗,上送暴室,令自殺,家屬被誅。呂太后崩時亦然。

九年正月辛卯朔,日有蝕之,在營室三度。史官不見,郡國以聞。谷永以為三朝尊者惡之。其明年,宮車晏駕。

永康元年五月壬子晦,日有蝕之,在輿鬼一度。儒說壬子淳水日,而陽不克,將有水害。其八月,六州大水,勃海盜賊。

靈帝建寧元年五月丁未朔,日有蝕之。冬十月甲辰晦,日有蝕之。

二年十月戊戌晦,日有蝕之。右扶風以聞。

三年三月丙寅晦,日有蝕之。梁相以聞。

四年三月辛酉朔,日有蝕之。

熹平二年十二月癸酉晦,日有蝕之,在虛二度。是時中常侍曹節、王甫等專權。

六年十月癸丑朔,日有蝕之,趙相以聞。

光和元年二月辛亥朔,日有蝕之。十月丙子晦,日有蝕之,在箕四度。箕為後宮口舌。是月,上聽讒廢宋皇后。

二年四月甲戌朔,日有蝕之。

四年九月庚寅朔,日有蝕之,在角六度。

中平三年五月壬辰晦,日有蝕之。

六年四月丙午朔,日有蝕之。其月浹辰,宮車晏駕。

獻帝初平四年正月甲寅朔,日有蝕之,在營室四度。是時李傕、郭汜專政。

興平元年六月乙巳晦,日有蝕之。

建安五年九月庚午朔,日有蝕之。

六年十月癸未朔,日有蝕之。

十三年十月癸未朔,日有蝕之,在尾十二度。

十五年二月乙巳朔,日有蝕之。

十七年六月庚寅晦,日有蝕之。

二十一年五月己亥朔,日有蝕之。

二十四年二月壬子晦,日有蝕之。

凡漢中興十二世,百九十六年,日蝕七十二:朔三十二,晦三十七,月二日三。

光武建武七年四月丙寅,日有暈抱,白虹貫暈,在畢八度。畢為邊兵。秋,隗囂反,侵安定。

靈帝時,日數出東方,正赤如血,無光,高二丈餘乃有景。且入西方,去地二丈,亦如之。其占曰,事天不謹,則日月赤。是時月出入去地二三丈,皆赤如血者數矣。

光和四年二月己巳,黃氣抱日,黃白珥在其表。

中平四年三月丙申,黑氣大如瓜,在日中。

五年正月,日色赤黃,中有黑氣如飛鵲,數月乃銷。

六年二月乙未,白虹貫日。

獻帝初平元年二月壬辰,白虹貫日。

桓帝永壽三年十二月壬戌,月蝕非其月。

延熹八年正月辛巳,月蝕非其月。

贊曰:皇極惟建,五事剋端。罰咎入沴,逆亂浸干。火下水騰,木弱金酸。妖豈或妄,氣炎以觀。


\end{pinyinscope}