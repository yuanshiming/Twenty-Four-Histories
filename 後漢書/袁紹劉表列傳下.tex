\article{袁紹劉表列傳下}

\begin{pinyinscope}
譚自稱車騎將軍,出軍黎陽。尚少與其兵,而使逢紀隨之。譚求益兵,審配等又議不與。譚怒,殺逢紀。

曹操度河攻譚,譚告急於尚,尚乃留審配守鄴,自將助譚,與操相拒於黎陽。自九月至明年二月,大戰城下,譚、尚敗退。操將圍之,乃夜遁還鄴。操進軍,尚逆擊破操,操軍還許,譚謂尚曰:「我鎧甲不精,故前為曹操所敗。今操軍退,人懷歸志,及其未濟,出兵掩之,可令大潰,此策不可失也。」尚疑而不許,既不益兵,又不易甲。譚大怒,郭圖、辛評因此謂譚曰:「使先公出將軍為兄後者,皆是審配之所構也。」譚然之。遂引兵攻尚,戰於外門。譚敗,乃引兵還南皮。

別駕王脩率吏人自青州往救譚,譚還欲更攻尚,問脩曰:「計將安出?」脩曰:「兄弟者,左右手也。譬人將鬥而斷其右手,曰『我必勝若』,如是者可乎?夫棄兄弟而不親,天下其誰親之?屬有讒人交鬥其閒,以求一朝之利,願塞耳勿聽也。若斬佞臣數人,復相親睦,以御四方,可橫行於天下。」譚不從。尚復自將攻譚,譚戰大敗,嬰城固守。尚圍之急,譚奔平原,而遣潁川辛毗詣曹操請救。

劉表以書諫譚曰:

天降災害,禍難殷流,初交殊族,卒成同盟,使王室震蕩,彝倫攸斁。是以智達之士,莫不痛心入骨,傷時人不能相忍也。然孤與太公,志同願等,雖楚魏絕邈,山河迥遠,戮力乃心,共獎王室,使非族不干吾盟,異類不絕吾好,此孤與太公無貳之所致也。功績未卒,太公殂隕,賢胤承統,以繼洪業。宣奕世之德,履丕顯之祚,摧嚴敵於鄴都,揚休烈於朔土,顧定疆宇,虎視河外,凡我同盟,莫不景附。何悟青蠅飛於竿旌,無忌游於二壘,使股肱分成二體,匈膂絕為異身。初聞此問,尚謂不然,定聞信來,乃知閼伯、實沈之忿已成,棄親即讎之計已決,旃旆交於中原,暴尸累於城下。聞之哽咽,若存若亡。昔三王、五伯,下及戰國,君臣相弒,父子相殺,兄弟相殘,親戚相滅,蓋時有之。然或欲以成王業,或欲以定霸功,皆所謂逆取順守,而徼富強於一世也。未有棄親即異,兀其根本,而能全於長世者也。

昔齊襄公報九世之讎,士饨卒荀偃之事,是故春秋美其義,君子稱其信。夫伯游之恨於齊,未若太公之忿於曹也;宣子之臣承業,未若仁君之繼統也。且君子違難不適讎國,交絕不出惡聲,況忘先人之讎,棄親戚之好,而為萬世之戒,遺同盟之恥哉!蠻夷戎狄將有誚讓之言,況我族類,而不痛心邪!

夫欲立竹帛於當時,全宗祀於一世,豈宜同生分謗,爭校得失乎?若冀州有不弟之傲,無慚順之節,仁君當降志辱身,以濟事為務。事定之後,使天下平其曲直,不亦為高義邪?今仁君見憎於夫人,未若鄭莊之於姜氏;昆弟之嫌,未若重華之於象敖。然莊公卒崇大隧之樂,象敖終受有鼻之封。願捐棄百痾,追攝舊義,復為母子昆弟如初。今整勒士馬,瞻望鵠立。

又與尚書諫之,並不從。

曹操遂還救譚,十月至黎陽。尚聞操度河,乃釋平原還鄴。尚將呂曠、高翔畔歸曹氏,譚復陰刻將軍印,以假曠、翔。操知譚詐,乃以子整娉譚女以安之,而引軍還。

九年三月,尚使審配守鄴,復攻譚於平原。配獻書於譚曰:「配聞良藥苦口而利於病,忠言逆耳而便於行。願將軍緩心抑怒,終省愚辭。蓋春秋之義,國君死社稷,忠臣死君命。苟圖危宗廟,剝亂國家,親疏一也。是以周公垂涕以斃管、蔡之獄,季友歔欷而行叔牙之誅。何則?義重人輕,事不獲已故也。昔先公廢黜將軍以續賢兄,立我將軍以為嫡嗣,上告祖靈,下書譜牒,海內遠近,誰不備聞!何意凶臣郭圖,妄畫蛇足,曲辭諂媚,交亂懿親。至令將軍忘孝友之仁,襲閼、沈之跡,放兵鈔突,屠城殺吏,冤魂痛於幽冥,創痍被於草棘。又乃圖獲鄴城,許賞賜秦胡,其財物婦女,豫有分數。又云:『孤雖有老母,趣使身體完具而已。』聞此言者,莫不悼心揮涕,使太夫人憂哀憤隔,我州君臣監寐悲歎。誠拱默以聽執事之圖,則懼違春秋死命之節,詒太夫人不測之患,損先公不世之業。我將軍辭不獲命,以及館陶之役。伏惟將軍至孝蒸蒸,發於岐嶷,友于之性,生於自然,章之以聰明,行之以敏達,覽古今之舉措,睹興敗之徵符,輕榮財於糞土,貴名高於丘岳。何意奄然迷沈,墮賢哲之操,積怨肆忿,取破家之禍!翹企延頸,待望讎敵,委慈親於虎狼之牙,以逞一朝之志,豈不痛哉!若乃天啟尊心,革圖易慮,則我將軍匍匐悲號於將軍股掌之上,配等亦當鲍躬布體以聽斧鑕之刑。如又不悛,禍將及之。願熟詳吉凶,以賜環玦。」譚不納。

曹操因此進攻鄴,審配將馮札為內應,開突門內操兵三百餘人。配覺之,從城上以大石擊門,門閉,入者皆死。操乃鑿塹圍城,周回四十里,初令淺,示若可越。配望見,笑而不出爭利。操一夜濬之,廣深二丈,引漳水以灌之。自五月至八月,城中餓死者過半。尚聞鄴急,將軍萬餘人還救城,操逆擊破之。尚走依曲漳為營,操復圍之,未合,尚懼,遣陰夔、陳琳求降,不聽。尚還走藍口,操復進,急圍之。尚將馬延等臨陣降,眾大潰,尚奔中山。盡收其輜重,得尚印綬節鉞及衣物,以示城中,城中崩沮。審配令士卒曰:「堅守死戰,操軍疲矣。幽州方至,何憂無主!」操出行圍,配伏弩射之,幾中。以其兄子榮為東門校尉,榮夜開門內操兵,配拒戰城中,生獲配。操謂配曰:「吾近行圍,弩何多也?」配曰:「猶恨其少。」操曰:「卿忠於袁氏,亦自不得不爾。」意欲活之。配意氣壯烈,終無撓辭,見者莫不歎息,遂斬之。全尚母妻子,還其財寶。高幹以并州降,復為刺史。

曹操之圍鄴也,譚復背之,因略取甘陵、安平、勃海、河閒,攻尚於中山。尚敗,走故安從熙,而譚悉收其眾,還屯龍湊。

十二月,曹操討譚,軍其門。譚夜遁奔南皮,臨清河而屯。明年正月,急攻之。譚欲出戰,軍未合而破。譚被髮驅馳,追者意非恆人,趨奔之。譚墯馬,顧曰:「咄,兒過我,我能富貴汝。」言未絕口,頭已斷地。於是斬郭圖等,戮其妻子。

熙、尚為其將焦觸、張南所攻,奔遼西烏桓。觸自號幽州刺史,驅率諸郡太守令長背袁向曹,陳兵數萬。殺白馬盟,令曰:「違者斬!」眾莫敢仰視,各以次歃。至別駕代郡韓珩,曰:「吾受袁公父子厚恩,今其破亡,智不能救,勇不能死,於義闕矣。若乃北面曹氏,所不能為也!」一坐為珩失色。觸曰:「夫舉大事,當立大義。事之濟否,不待一人,可卒珩志,以厲事君。」曹操聞珩節,甚高之,屢辟不至,卒於家。

高幹復叛,執上黨太守,舉兵守壺口關。十一年,曹操自征幹,幹乃留其將守城,自詣匈奴求救,不得,獨與數騎亡,欲南奔荊州。上洛都尉捕斬之。

十二年,曹操征遼西,擊烏桓。尚、熙與烏桓逆操軍,戰敗走,乃與親兵數千人奔公孫康於遼東。尚有勇力,先與熙謀曰:「今到遼東,康必見我,我獨為兄手擊之,且據其郡,猶可以自廣也。」康亦心規取尚以為功,乃先置精勇於廄中,然後請尚、熙。熙疑不欲進,尚彊之,遂與俱入。未及坐,康叱伏兵禽之,坐於凍地。尚謂康曰:「未死之閒,寒不可忍,可相與席。」康曰:「卿頭顱方行萬里,何席之為!」遂斬首送之。

康,遼東人。父度。初避吏為玄兔小吏,稍仕。中平元年,還為本郡守。在職敢殺伐,郡中名豪與己夙無恩者,遂誅滅百餘家。因東擊高句驪,西攻烏桓,威行海畔。時王室方亂,度恃其地遠,陰獨懷幸。會襄平社生大石丈餘,下有三小石為足,度以為己瑞。初平元年,乃分遼東為遼西、中遼郡,並置太守,越海收東萊諸縣,為營州刺史,自立為遼東侯、平州牧,追封父延為建義侯。立漢二祖廟。承制設壇墠於襄平城南,郊祀天地,藉田理兵,乘鸞輅九旒旄頭羽騎。建安九年,司空曹操表為奮威將軍,封永寧鄉侯。度死,康嗣,故遂據遼土焉。

劉表字景升,山陽高平人,魯恭王之後也。身長八尺餘,姿貌溫偉。與同郡張儉等俱被訕議,號為「八顧」。詔書捕案黨人,表亡走得免。黨禁解,辟大將軍何進掾。

初平元年,長沙太守孫堅殺荊州刺史王叡,詔書以表為荊州刺史。時江南宗賊大盛,又袁術阻兵屯魯陽,表不能得至,乃單馬入宜城,請南郡人蒯越、襄陽人蔡瑁與共謀畫。表謂越曰:「宗賊雖盛而眾不附,若袁術因之,禍必至矣。吾欲徵兵,恐不能集,其策焉出?」對曰:「理平者先仁義,理亂者先權謀。兵不在多,貴乎得人。袁術驕而無謀,宗賊率多貪暴。越有所素養者,使人示之以利,必持眾來。使君誅其無道,施其才用,威德既行,襁負而至矣。兵集眾附,南据江陵,北守襄陽,荊州八郡可傳檄而定。公路雖至,無能為也。」表曰:「善。」乃使越遣人誘宗賊帥,至者十五人,皆斬之而襲取其眾。唯江夏賊張虎、陳坐擁兵據襄陽城,表使越與龐季往譬之,乃降。江南悉平。諸守令聞表威名,多解印綬去。表遂理兵襄陽,以觀時變。

袁術與其從兄紹有隙,而紹與表相結,故術共孫堅合從襲表。表敗,堅遂圍襄陽。會表將黃祖救至,堅為流箭所中死,餘眾退走。及李傕等入長安,冬,表遣使奉貢。傕以表為鎮南將軍、荊州牧,封成武侯,假節,以為己援。

建安元年,驃騎將軍張濟自關中走南陽,因攻穰城,中飛矢而死。荊州官屬皆賀。表曰:「濟以窮來,主人無禮,至於交鋒,此非牧意,牧受弔不受賀也。」使人納其眾,眾聞之喜,遂皆服從。三年,長沙太守張羨率零陵、桂陽三郡畔表,表遣兵攻圍,破羨,平之。於是開土遂廣,南接五領,北據漢川,地方數千里,帶甲十餘萬。初,荊州人情好擾,加四方駭震,寇賊相扇,處處麋沸。表招誘有方,威懷兼洽,其姦猾宿賊更為效用,萬里肅清,大小咸悅而服之。關西、兗、豫學士歸者蓋有千數,表安慰賑贍,皆得資全。遂起立學校,博求儒術,綦母闓、宋忠等撰立五經章句,謂之後定。愛民養士,從容自保。

及曹操與袁紹相持於官度,紹遣人求助,表許之,不至,亦不援曹操,且欲觀天下之變。從事中郎南陽韓嵩、別駕劉先說表曰:「今豪桀並爭,兩雄相持,天子之重在於將軍。若欲有為,起乘其敝可也;如其不然,固將擇所宜從。豈可擁甲十萬,坐觀成敗,求援而不能助,見賢而不肯歸!此兩怨必集於將軍,恐不得中立矣。曹操善用兵,且賢俊多歸之,其埶必舉袁紹,然後移兵以向江漢,恐將軍不能禦也。今之勝計,莫若舉荊州以附曹操,操必重德將軍,長享福祚,垂之後嗣,此萬全之策也。」蒯越亦勸之。表狐疑不斷,乃遣嵩詣操,觀望虛實。謂嵩曰:「今天下未知所定,而曹操擁天子都許,君為我觀其釁。」嵩對曰:「嵩觀曹公之明,必得志於天下。將軍若欲歸之,使嵩可也;如其猶豫,嵩至京師,天子假嵩一職,不獲辭命,則成天子之臣,將軍之故吏耳。在君為君,不復為將軍死也。惟加重思。」表以為憚使,強之。至許,果拜嵩侍中、零陵太守。及還,盛稱朝廷曹操之德,勸遣子入侍。表大怒,以為懷貳,陳兵詬嵩,將斬之。嵩不為動容,徐陳臨行之言。表妻蔡氏知嵩賢,諫止之。表猶怒,乃考殺從行者。知無它意,但囚嵩而已。

六年,劉備自袁紹奔荊州,表厚相待結而不能用也。十三年,曹操自將征表,未至。八月,表疽發背卒。在荊州幾二十年,家無餘積。

二子:琦,琮。表初以琦貌類於己,甚愛之,後為琮娶其後妻蔡氏之姪,蔡氏遂愛琮而惡琦,毀譽之言日聞於表。表寵耽後妻,每信受焉。又妻弟蔡瑁及外甥張允並得幸於表,又睦於琮。而琦不自寧,嘗與琅邪人諸葛亮謀自安之術。亮初不對。後乃共升高樓,因令去梯,謂亮曰:「今日上不至天,下不至地,言出子口而入吾耳,可以言未?」亮曰:「君不見申生在內而危,重耳居外而安乎?」琦意感悟,陰規出計。會表將江夏太守黃祖為孫權所殺,琦遂求代其任。

及表病甚,琦歸省疾,素慈孝,允等恐其見表而父子相感,更有託後之意,乃謂琦曰:「將軍命君撫臨江夏,其任至重。今釋眾擅來,必見譴怒。傷親之歡,重增其疾,非孝敬之道也。」遂遏于戶外,使不得見。琦流涕而去,人眾聞而傷焉。遂以琮為嗣。琮以侯印授琦。琦怒,投之地,將因奔喪作難。會曹操軍至新野,琦走江南。蒯越、韓嵩及東曹掾傅巽等說琮歸降。琮曰:「今與諸君據全楚之地,守先君之業,以觀天下,何為不可?」巽曰:「逆順有大體,強弱有定埶。以人臣而拒人主,逆道也;以新造之楚而禦中國,必危也;以劉備而敵曹公,不當也。三者皆短,欲以抗王師之鋒,必亡之道也。將軍自料何與劉備?」琮曰:「不若也。」巽曰:「誠以劉備不足禦曹公,則雖全楚不能以自存也。誠以劉備足禦曹公,則備不為將軍下也。願將軍勿疑。」

及操軍到襄陽,琮舉州請降,劉備奔夏口。操以琮為青州刺史,封列侯。蒯越等侯者十五人。乃釋嵩之囚,以其名重,甚加禮待,使條品州人優劣,皆擢而用之。以嵩為大鴻臚,以交友禮待之。蒯越光祿勳,劉光尚書令。初,表之結袁紹也,侍中從事鄧義諫不聽。義以疾退,終表世不仕,操以為侍中。其餘多至大官。

操後敗於赤壁,劉備表琦為荊州刺史。明年卒。

論曰:袁紹初以豪俠得眾,遂懷雄霸之圖,天下勝兵舉旗者,莫不假以為名。及臨場決敵,則悍夫爭命;深籌高議,則智士傾心。盛哉乎,其所資也!韓非曰:「佷剛而不和,愎過而好勝,嫡子輕而庶子重,斯之謂亡徵。」劉表道不相越,而欲臥收天運,擬蹤三分,其猶木禺之於人也。

贊曰:紹姿弘雅,表亦長者。稱雄河外,擅強南夏。魚儷漢舳,雲屯冀馬。闚圖訊鼎,禋天類社。既云天工,亦資人亮。矜彊少成,坐談奚望。回皇冢嬖,身穨業喪。


\end{pinyinscope}