\article{禮儀上}

\begin{pinyinscope}
耕高禖養老先蠶祓禊

夫威儀,所以與君臣,序六親也。若君亡君之威,臣亡臣之儀,上替下陵,此謂大亂。大亂作,則群生受其殃,可不慎哉!故記施行威儀,以為禮儀志。

禮威儀,每月朔旦,太史上其月曆,有司、侍郎、尚書見讀其令,奉行其政。朔前後各二日,皆牽羊酒至社下以祭日。日有變,割羊以祠社,用救日日變。執事者冠長冠,衣皁單衣,絳領袖綠中衣,絳蔥远,以行禮,如故事。

立春之日,夜漏未盡五刻,京師百官皆衣青衣,郡國縣道官下至斗食令史皆服青幘,立青幡,施土牛耕人于門外,以示兆民,至立夏。唯武官不。立春之日,下寬大書曰:「制詔三公:方春東作,敬始慎微,動作從之。罪非殊死,且勿案驗,皆須麥秋。退貪殘,進柔良,下當用者,如故事。」

正月上丁,祠南郊。禮畢,次北郊,明堂,高廟,世祖廟,謂之五供。五供畢,以次上陵。

西都舊有上陵。東都之儀,百官、四姓親家婦女、公主、諸王大夫、外國朝者侍子、郡國計吏會陵。晝漏上水,大鴻臚設九賓,隨立寢殿前。鍾鳴,謁者治禮引客,群臣就位如儀。乘輿自東廂下,太常導出,西向拜,止旋升阼階,拜神坐。退坐東廂,西向。侍中、尚書、陛者皆神坐後。公卿群臣謁神坐,太官上食,太常樂奏食舉,文始、五行之舞。禮樂闋,君臣受賜食畢,郡國上計吏以次前,當神軒占其郡穀價,民所疾苦,欲神知其動靜。孝子事親盡禮,敬愛之心也。周遍如禮。最後親陵,遣計吏,賜之帶佩。八月飲酎,上陵,禮亦如之。

凡齋,天地七日,宗廟、山川五日,小祠三日。齋日內有汙染,解齋,副倅行禮。先齋一日,有汙穢災變,齋祀如儀。大喪,唯天郊越紼而齋,地以下皆百日後乃齋,如故事。

正月甲子若丙子為吉日,可加元服,儀從冠禮。乘輿初緇布進賢,次爵弁,次武弁,次通天。以據,皆於高祖廟如禮謁。王公以下,初加進賢而已。

正月,天郊,夕牲。晝漏未盡十八刻初納,夜漏未盡八刻初納,進熟獻,太祝送,旋,皆就燎位,宰祝舉火燔柴,火然,天子再拜,興,有司告事畢也。明堂、五郊、宗廟、太社稷、六宗夕牲,皆以晝漏十四刻初納,夜漏未盡七刻初納,進熟獻,送神,還,有司告事畢。六宗燔燎,火大然,有司告事畢。

正月始耕。晝漏上水初納,執事告祠先農,已享。耕時,有司請行事,就耕位,天子、三公、九卿、諸侯、百官以次耕。力田種各耰訖,有司告事畢。是月令曰:「郡國守相皆勸民始耕,如儀。諸行出入皆鳴鍾,皆作樂。其有災眚,有他故,若請雨、止雨,皆不鳴鍾,不作樂。」

仲春之月,立高禖祠于城南,祀以特牲。

明帝永平二年三月,上始帥群臣躬養三老、五更于辟雍。行大射大禮。郡、縣、道行鄉飲酒于學校,皆祀聖師周公、孔子,牲以犬。於是七郊禮樂三雍之義備矣。

養三老、五更之儀,先吉日,司徒上太傅若講師故三公人名,用其德行年耆高者一人為老,次一人為更也。皆服都紵大袍單衣,皁緣領袖中衣,冠進賢,扶玉杖。五更亦如之,不杖。皆齋于太學講堂。其日,乘輿先到辟雍禮殿,御坐東廂,遣使者安車迎三老、五更。天子迎于門屏,交禮,道自阼階,三老升自賓階。至階,天子揖如禮。三老升,東面,三公設几,九卿正履,天子親袒割牲,執醬而饋,執爵而酳,祝鯁在前,祝饐在後。五更南面,公進供禮,亦如之。明日皆詣闕謝恩,以見禮遇大尊顯故也。

是月,皇后帥公卿諸侯夫人蠶。祠先蠶,禮以少牢。

是月上巳,官民皆絜於東流水上,曰洗濯祓除去宿垢疢為大絜。絜者,言陽氣布暢,萬物訖出,始絜之矣。


\end{pinyinscope}