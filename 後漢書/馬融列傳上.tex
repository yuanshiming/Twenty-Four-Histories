\article{馬融列傳上}

\begin{pinyinscope}
馬融字季長,扶風茂陵人也,將作大匠嚴之子。為人美辭貌,有俊才。初,京兆摯恂以儒術教授,隱于南山,不應徵聘,名重關西,融從其遊學,博通經籍。恂奇融才,以女妻之。

永初二年,大將軍鄧騭聞融名,召為舍人,非其好也,遂不應命,客於涼州武都、漢陽界中。會羌虜飆起,邊方擾亂,米穀踴貴,自關以西,道殣相望。融既飢困,乃悔而歎息,謂其友人曰:「古人有言:『左手據天下之圖,右手刎其喉,愚夫不為。』所以然者,生貴於天下也。今以曲俗咫尺之羞,滅無貲之軀,殆非老莊所謂也。」故往應騭召。

四年,拜為校書郎中,詣東觀典校秘書。是時鄧太后臨朝,騭兄弟輔政。而俗儒世士,以為文德可興,武功宜廢,遂寢蒐狩之禮,息戰陳之法,故猾賊從橫,乘此無備。融乃感激,以為文武之道,聖賢不墜,五才之用,無或可廢。元初二年,上廣成頌以諷諫。其辭曰:

臣聞孔子曰:「奢則不遜,儉則固。」奢儉之中,以禮為界。是以蟋蟀、山樞之人,並刺國君,諷以太康馳驅之節。夫樂而不荒,憂而不困,先王所以平和府藏,頤養精神,致之無疆。故戛擊鳴球,載於虞謨;吉日車攻,序於周詩。聖主賢君,以增盛美,豈徒為奢淫而已哉!伏見元年已來,遭值厄運,陛下戒懼災異,躬自菲薄,荒棄禁苑,廢弛樂懸,勤憂潛思,十有餘年,以過禮數。重以皇太后體唐堯親九族篤睦之德,陛下履有虞烝烝之孝,外舍諸家,每有憂疾,聖恩普勞,遣使交錯,稀有曠絕。時時寧息,又無以自娛樂,殆非所以逢迎太和,裨助萬福也。臣愚以為雖尚頗有蝗蟲,今年五月以來,雨露時澍,祥應將至。方涉冬節,農事閒隙,宜幸廣成,覽原隰,觀宿麥,收藏,因講武校獵,使寮庶百姓,復睹羽旄之美,聞鍾鼓之音,歡嬉喜樂,鼓舞疆畔,以迎和氣,招致休慶。小臣螻蟻,不勝區區。職在書籍,謹依舊文,重述蒐狩之義,作頌一篇,并封上。淺陋鄙薄,不足觀省。

臣聞昔命師於鞬櫜,偃伯於靈臺,或人嘉而稱焉。彼固未識夫雷霆之為天常,金革之作昏明也。自黃炎之前,傳道罔記;三五以來,越可略聞。且區區之酆郊,猶廓七十里之囿,盛春秋之苗。詩詠囿草,樂奏騶虞。是以大漢之初基也,宅茲天邑,總風雨之會,交陰陽之和。揆厥靈囿,營于南郊。徒觀其坰場區宇,恢胎曠蕩,摆夐勿罔,寥豁鬱泱,騁望千里,天與地莽。於是周阹環瀆,右矕三塗,左概嵩嶽,面據衡陰,箕背王屋,浸以波、溠,夤以滎、洛。金山、石林,殷起乎其中,峨峨磑磑,鏘鏘锐锐,隆穹槃回,嵎峗錯崔。神泉側出,丹水涅池,怪石浮磬,燿焜于其陂。其土毛則搉牧薦草,芳茹甘荼,茈萁、芸蒩,昌本、深蒱,芝荋、菫、荁,蘘荷、芋渠,桂荏、鳧葵,格、韭、菹、于。其植物則玄林包竹,藩陵蔽京,珍林嘉樹,建木叢生,椿、梧、栝、柏,柜、柳、楓、楊,豐彤對蔚,崟哣槮爽。翕習春風,含津吐榮,鋪于布铐,蓶扈蘳熒,惡可殫形。

至于陽月,陰慝害作,百草畢落,林衡戒田,焚萊柞木。然後舉天網,頓八紘,揫斂九藪之動物,繯橐四野之飛征。鳩之乎茲囿之中,山敦雲移,群鳴膠膠,鄙騃譟讙,子野聽聳,離朱目眩,隸首策亂,陳子籌昏。於時營圍恢廓,充斥川谷,罦罝羅羉,彌綸阬澤,皋牢陵山。校隊案部,前後有屯,甲乙相伍,戊己為堅。

乘輿乃以吉月之陽朔,登于疏鏤之金路,六驌垩之玄龍,建雄虹之旌夏,揭鳴鳶之脩橦。曳長庚之飛髾,載日月之太常,棲招搖與玄弋,注枉矢於天狼。羽毛紛其髟鼬,揚金鲻而扡玉瓖。屯田車於平原,播同徒於高岡,旃旝摻其如林,錯五色以摛光。清氛埃,埽野場,誓六師,搜俊良。司徒勒卒,司馬平行,車攻馬同,教達戒通。伐咎鼓,撞華鍾,獵徒縱,赴榛叢。徽嫿霍奕,別騖分奔,騷擾聿皇,往來交舛,紛紛回回,南北東西。風行雲轉,匈磕隱訇,黃塵勃滃,闇若霧昏。日月為之籠光,列宿為之翳昧,僄狡課才,勁勇程氣。狗馬角逐,鷹鸇競鷙,驍騎旁佐,輕車橫厲,相與陸梁,聿皇于中原。絹猑蹄,鏦特肩,脰完羝,撝介鮮,散毛族,梏羽群。然後飛鋋電激,流矢雨墜,各指所質,不期俱殪,竄伏扔輪,發作梧禇。祋殳狂擊,頭陷顱碎,獸不得猭,禽不得瞥。或夷由未殊,顛狽頓躓,蝡蝡蟫蟫,充衢塞隧,葩華锑布,不可勝計。

若夫鷙獸适蟲,倨牙黔口,大匈哨後,縕巡歐紆,負隅依阻,莫敢嬰禦。乃使鄭叔、晉婦之徒,睽孤刲刺,裸裎袒裼。冒甯柘,槎棘枳,窮浚谷,底幽嶰,暴斥虎,搏狂兕,獄辊熊,抾封狶。或輕訬趬悍,廋疏嶁領,犯歷嵩巒,陵喬松,履脩樠,踔攳枝,杪標端,尾蒼蜼,掎玄猿,木產盡,寓屬單。罕罔合部,罾弋同曲,類行並驅,星布麗屬,曹伍相保,各有分局。矰碆飛流,纖羅絡縸,遊雉群驚,晨鳧輩作,翬然雲起,霅爾雹落。

爾乃摆觀高蹈,改乘回轅,泝恢方,撫馮夷,策句芒,超荒忽,出重陽,厲雲漢,橫天潢。導鬼區,徑神場,詔靈保,召方相,驅厲疫,走蜮祥。捎罔兩,拂游光,枷天狗,惞墳羊。然後緩節舒容,裴回安步,降集波烃,川衡澤虞,矢魚陳罟。茲飛、宿沙,田開、古蠱,翬終葵,揚關斧,刊重冰,撥蟄戶,測潛鱗,踵介旅。逆獵湍瀨,渀薄汾橈,淪滅潭淵,左挈夔龍,右提蛟鼉,春獻王鮪,夏薦鱉黿。於是流覽遍照,殫變極態,上下究竟,山谷蕭條,原野嵺愀,上無飛鳥,下無走獸,虞人植旍,獵者效具,車弊田罷,旋入禁囿。棲遲乎昭明之觀,休息乎高光之榭,以臨乎宏池。鎮以瑤臺,純以金堤,樹以蒱柳,被以綠莎,瀇瀁沆漭,錯紾槃委,天地虹洞,固無端涯,大明生東,月朔西陂。乃命壺涿,驅水蠱,逐罔、螭,滅短狐,簎鯨、鯢。然後方餘皇,連舼舟,張雲帆,施蜺幬,靡颸風,陵迅流,發櫂歌,縱水謳,淫魚出,蓍蔡浮,湘靈下,漢女游。水禽鴻鵠,鴛鴦、鷗、鷖,鶬鴰、鸕、鷁,鷺、鴈、鷿辍,乃安斯寢,戢翮其涯。魴、鱮、媂、凄,鰋、鯉、鱨、魦,樂我純德,騰踊相隨,雖靈沼之白鳥,孟津之躍魚,方斯蔑矣。然猶詠歌於伶蕭,載陳於方策,豈不哀哉!

於是宗廟既享,庖廚既充,車徒既簡,器械既攻。然後擺牲班禽,淤賜犒功,群師疊伍,伯校千重,山罍常滿,房俎無空。酒正案隊,膳夫巡行,清醪車湊,燔炙騎將,鼓駭舉爵,鍾鳴既觴。若乃陽阿衰斐之晉制,闡杀華羽之南音,所以洞蕩匈臆,發明耳目,疏越蘊慉,駭恫底伏,鍠鍠鎗鎗,奏于農郊大路之衢,與百姓樂之。是以明德曜乎中夏,威靈暢乎四荒,東鄰浮巨海而入享,西旅越蔥領而來王,南徼因九譯而致貢,朔狄屬象胥而來同。蓋安不忘危,治不忘亂,道在乎茲,斯固帝王之所以曜神武而折遐衝者也。

方今大漢收功於道德之林,致獲於仁義之淵,忽蒐狩之禮,闕槃虞之佃。闇昧不睹日月之光,聾昏不聞雷霆之震,于今十二年,為日久矣。亦方將刊禁臺之秘藏,發天府之官常,由質要之故業,率典刑之舊章。采清原,嘉岐陽,登俊桀,命賢良,舉淹滯,拔幽荒。察淫侈之華譽,顧介特之實功,聘畎畝之群雅,宗重淵之潛龍。乃儲精山藪,歷思河澤,目辆鼎俎,耳聽康衢,營傅說於胥靡,求伊尹於庖廚,索膠鬲於魚鹽,聽甯戚於大車。俾之昌言而宏議,軼越三家,馳騁五帝,悉覽休祥,總括群瑞。遂棲鳳皇於高梧,宿麒麟於西園,納僬僥之珍羽,受王母之白環。永逍搖乎宇內,與二儀乎無疆,貳造化於后土,參神施於昊乾,超特達而無儔,煥巍巍而無原。豐千億之子孫,歷萬載而永延。禮樂既闋,北轅反旆,至自新城,背伊闕,反洛京。

頌奏,忤鄧氏,滯於東觀,十年不得調。因兄子喪自劾歸。太后聞之怒,謂融羞薄詔除,欲仕州郡,遂令禁錮之。

太后崩,安帝親政,召還郎署,復在講部。出為河閒王廄長史。時車駕東巡岱宗,融上東巡頌,帝奇其文,召拜郎中。及北鄉侯即位,融移病去,為郡功曹。

陽嘉二年,詔舉敦樸,城門校尉岑起舉融,徵詣公車,對策,拜議郎。大將軍梁商表為從事中郎,轉武都太守。時西羌反叛,征西將軍馬賢與護羌校尉胡疇征之,而稽久不進。融知其將敗,上疏乞自效,曰:「今雜種諸羌轉相鈔盜,宜及其未并,亟遣深入,破其支黨,而馬賢等處處留滯。羌胡百里望塵,千里聽聲,今逃匿避回,漏出其後,則必侵寇三輔,為民大害。臣願請賢所不可用關東兵五千,裁假部隊之號,盡力率厲,埋根行首,以先吏士,三旬之中,必克破之。臣少習學蓺,不更武職,猥陳此言,必受誣罔之辜。昔毛遂廝養,為眾所蚩,終以一言,克定從要。臣懼賢等專守一城,言攻於西而羌出於東,且其將士必有高克潰叛之變。」朝廷不能用。又陳:「星孛參、畢,參西方之宿,畢為邊兵,至於分野,并州是也。西戎北狄,殆將起乎!宜備二方。」尋而隴西羌反,烏桓寇上郡,皆卒如融言。

三遷,桓帝時為南郡太守。先是融有事忤大將軍梁冀旨,冀諷有司奏融在郡貪濁,免官,髡徙朔方。自刺不殊,得赦還,復拜議郎,重在東觀著述,以病去官。

融才高博洽,為世通儒,教養諸生,常有千數。涿郡盧植,北海鄭玄,皆其徒也。善鼓琴,好吹笛,達生任性,不拘儒者之節。居宇器服,多存侈飾。常坐高堂,施絳紗帳,前授生徒,後列女樂,弟子以次相傳,鮮有入其室者。嘗欲訓左氏春秋,及見賈逵、鄭眾注,乃曰:「賈君精而不博,鄭君博而不精。既精既博,吾何加焉!」但著三傳異同說。注孝經、論語、詩、易、三禮、尚書、列女傳、老子、淮南子、離騷,所著賦、頌、碑、誄、書、記、表、奏、七言、琴歌、對策、遺令,凡二十一篇。

初,融懲於鄧氏,不敢復違忤埶家,遂為梁冀草奏李固,又作大將軍西第頌,以此頗為正直所羞。年八十八,延熹九年卒于家。遺令薄葬。族孫日磾,獻帝時位至太傅。

論曰:馬融辭命鄧氏,逡巡隴漢之閒,將有意於居貞乎?既而羞曲士之節,惜不貲之軀,終以奢樂恣性,黨附成譏,固知識能匡欲者鮮矣。夫事苦,則矜全之情薄;生厚,故安存之慮深。登高不懼者,胥靡之人也;坐不垂堂者,千金之子也。原其大略,歸於所安而已矣。物我異觀,亦更相笑也。


\end{pinyinscope}