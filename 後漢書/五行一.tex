\article{五行一}

\begin{pinyinscope}
屋自壞訛言旱謠狼食人

五行傳說及其占應,漢書五行志錄之詳矣。故泰山太守應劭、給事中董巴、散騎常侍譙周並撰建武以來災異。今合而論之,以續前志云。

五行傳曰:「田獵不宿,飲食不享,出入不節,奪民農時,及有姦謀,則木不曲直。」謂木失其性而為災也。又曰:「貌之不恭,是謂不肅。厥咎狂,厥罰恆雨,厥極惡。時則有服妖,時則有龜孽,時則有雞禍,時則有下體生上之痾,時則有青眚、青祥,惟金沴木。」說云:氣之相傷謂之沴。

建武元年,赤眉賊率樊崇、逢安等共立劉盆子為天子。然崇等視之如小兒,百事自由,初不恤錄也。後正旦至,君臣欲共饗,既坐,酒食未下,群臣更起,亂不可整。時大司農楊音案劍怒曰:「小兒戲尚不如此!」其後遂破壞,崇、安等皆誅死。唯音為關內侯,以壽終。

光武崩,山陽王荊哭不哀,作飛書與東海王,勸使作亂。明帝以荊同母弟,太后在,故隱之。後徙王廣陵,荊遂坐復謀反自殺也。

章帝時,竇皇后兄憲以皇后甚幸於上,故人人莫不畏憲。憲於是強請奪沁水長公主田,公主畏憲,與之,憲乃賤顧之。後上幸公主田,覺之,問憲,憲又上言借之。上以后故,但譴敕之,不治其罪。後章帝崩,竇太后攝政,憲秉機密,忠直之臣與憲忤者,憲多害之,其後憲兄弟遂皆被誅。

桓帝時,梁冀秉政,兄弟貴盛自恣,好驅馳過度,至於歸家,猶馳驅入門,百姓號之曰「梁氏滅門驅馳」。後遂誅滅。

和帝永元十年,十三年,十四年,十五年,皆淫雨傷稼。

安帝元年四年秋,郡國十淫雨傷稼。

永寧元年,郡國三十三淫雨傷稼。

建光元年,京都及郡國二十九淫雨傷稼。是時羌反久未平,百姓屯戍,不解愁苦。

延光元年,郡國二十七淫雨傷稼。

二年,郡國五連雨傷稼。

順帝永建四年,司隸、荊、豫、兗、冀部淫雨傷稼。

六年,冀州淫雨傷稼。

桓帝延熹二年夏,霖雨五十餘日。是時,大將軍梁冀秉政,謀害上所幸鄧貴人母宣,冀又擅殺議郎邴尊。上欲誅冀,懼其持權日久,威勢強盛,恐有逆命,害及吏民,密與近臣中常侍單超等圖其方略。其年八月,冀卒伏罪誅滅。

靈帝建寧元年夏,霖雨六十餘日。是時大將軍竇武謀變廢中官。其年九月,長樂五官史朱瑀等共與中常侍曹節起兵,先誅武,交兵闕下,敗走,追斬武兄弟,死者數百人。

熹平元年夏,霖雨七十餘日。是時中侍曹節等,共誣曰勃海王悝謀反,其十月誅悝。

中平六年夏,霖雨八十餘日。是時靈帝新棄群臣,大行尚在梓宮,大將軍何進與佐軍校尉袁紹等共謀欲誅廢中官。下文陵畢,中常侍張讓等共殺進,兵戰京都,死者數千。

更始諸將軍過雒陽者數十輩,皆幘而衣婦人衣繡擁萧。時智者見之,以為服之不中,身之災也,乃奔入邊郡避之。是服妖也。其後更始遂為赤眉所殺。

桓帝元嘉中,京都婦女作愁眉、啼荫、墮馬髻、折要步、齲齒笑。所謂愁眉者,細而曲折。啼荫者,薄拭目下,若啼處。墮馬髻者,作一邊。折要步者,足不在體下。齲齒笑者,若齒痛,樂不欣欣。始自大將軍梁冀家所為,京都歙然,諸夏皆放效。此近服妖也。梁冀二世上將,婚媾王室,大作威福,將危社稷。天誡若曰:兵馬將往收捕,婦女憂愁,踧眉啼泣,吏卒掣頓,折其要脊,令髻傾邪,雖強語笑,無復氣味也。到延熹二年,舉宗誅夷。

延熹中,梁冀誅後,京都幘顏短耳長,短上長下。時中常侍單超、左悺、徐璜、具瑗、唐衡在帝左右,縱其姦慝。海內慍曰:一將軍死,五將軍出。家有數侯,子弟列布州郡,賓客雜襲騰翥,上短下長,與梁冀同占。到其八年,桓帝因日蝕之變,乃拜故司徒韓寅為司隸校尉,以次誅鉏,京都正清。

延熹中,京都長者皆著木屐;婦女始嫁,至作漆畫五采為系。此服妖也。到九年,黨事始發,傳黃門北寺,臨時惶惑,不能信天任命,多有逃走不就考者,九族拘繫,及所過歷,長少婦女皆被桎梏,應木屐之象也。

靈帝建寧中,京都長者皆以葦方笥為荫具,下士盡然。時有識者竊言:葦方笥,郡國讞篋也;今珍用之,此天下人皆當有罪讞於理官也。到光和三年癸丑赦令詔書,吏民依黨禁錮者赦除之,有不見文,他以類比疑者讞。於是諸有黨郡皆讞廷尉,人名悉入方笥中。

靈帝好胡服、胡帳、胡床、胡坐、胡飯、胡空侯

、胡笛、胡舞,京都貴戚皆競為之。此服妖也。其後董卓多擁胡兵,填塞街衢,虜掠宮掖,發掘園陵。

靈帝於宮中西園駕四白驢,躬自操轡,驅馳周旋,以為大樂。於是公卿貴戚轉相放效,至乘輜軿以為騎從,互相侵奪,賈與馬齊。案《易》曰:「時乘六龍以御天。」行天者莫若龍,行地者莫如馬。《詩》云:「四牡騤騤,載是常服。」「檀車煌煌,四牡彭彭。」夫驢乃服重致遠,上下山谷,野人之所用耳,何有帝王君子而驂服之乎!遲鈍之畜,而今貴之。天意若曰:國且大亂,賢愚倒植,凡執政者皆如驢也。其後董卓陵虐王室,多援邊人以充本朝,胡夷異種,跨蹈中國。

熹平中,省內冠狗帶綬,以為笑樂。有一狗突出,走入司徒府門,或見之者,莫不驚怪。京房易傳曰:「君不正,臣欲篡,厥妖狗冠出。」後靈帝寵用便嬖子弟,永樂賓客、鴻都群小,傳相汲引,公卿牧守,比肩是也。又遣御史於西鄉賣官,關內侯顧五百萬者,賜與金紫;詣闕上書占令長,隨縣好醜,豐約有賈。強者貪如豺虎,弱者略不類物,實狗而冠者也。司徒古之丞相,壹統國政。天戒若曰:宰相多非其人,尸祿素餐,莫能據正持重,阿意曲從;今在位者皆如狗也,故狗走入其門。

靈帝數遊戲於西園中,令後宮采女為客舍主人,身為商賈服。行至舍,采女下酒食,因共飲食以為戲樂。此服妖也。其後天下大亂。

獻帝建安中,男子之衣,好為長躬而下甚短,女子好為長裙而上甚短。時益州從事莫嗣以為服妖,是陽無下而陰無上也,天下未欲平也。後還,遂大亂。

靈帝光和元年,南宮侍中寺雌雞欲化雄,一身毛皆似雄,但頭冠尚未變。詔以問議郎蔡邕。邕對曰:「貌之不恭,則有雞禍。宣帝黃龍元年,未央宮雌雞化為雄,不鳴無距。是歲元帝初即位,立王皇后。至初元元年,丞相史家雌雞化為雄,冠距鳴將。是歲后父禁為平陽侯,女立為皇后。至哀帝晏駕,后攝政,王莽以后兄子為大司馬,由是為亂。臣竊推之,頭,元首,人君之象;今雞一身已變,未至於頭,而上知之,是將有其事而不遂成之象也。若應之不精,政無所改,頭冠或成,為患茲大。」是後張角作亂稱黃巾,遂破壞。四方疲於賦役,多叛者。上不改政,遂至天下大亂。

桓帝永興二年四月丙午,光祿勳吏舍壁下夜有青氣,視之,得玉鉤、玦各一。鉤長七寸二分,周五寸四分,身中皆雕鏤。此青祥也。玉,金類也。七寸二分,商數也。五寸四分,徵數也。商為臣,徵為事,蓋為人臣引決事者不肅,將有禍也。是時梁冀秉政專恣,後四歲,梁氏誅滅也。

延熹五年,太學門無故自壞。襄楷以為太學前疑所居,其門自壞,文德將喪,教化廢也。是後天下遂至喪亂。

永康元年十月壬戍,南宮平城門內屋自壞。金沴木,木動也。其十二月,宮車晏駕。

靈帝光和元年,南宮平城門內屋、武庫屋及外東垣屋前後頓壞。蔡邕對曰:「平城門,正陽之門,與宮連,郊祀法駕所由從出,門之最尊者也。武庫,禁兵所藏。東垣,庫之外障。易傳曰:『小人在位,上下咸悖,厥妖城門內崩。』潛潭巴曰:『宮瓦自墮,諸侯強陵主。』此皆小人顯位亂法之咎也。」其後黃巾賊先起東方,庫兵大動。皇后同父兄何進為大將軍,同母弟苗為車騎將軍,兄弟並貴盛,皆統兵在京都。其後進欲誅廢中官,為中常侍張讓、段珪等所殺,兵戰宮中闕下,更相誅滅,天下兵大起。

三年二月,公府駐駕廡自壞,南北三十餘閒。

中平二年二月癸亥,廣陽城門外上屋自壞也。

獻帝初平二年三月,長安宣平城門外屋無故自壞。至三年夏,司徒王允使中郎將呂布殺太師董卓,夷三族。

興平元年十月,長安市門無故自壞。至二年春,李傕、郭汜鬥長安中,傕迫劫天子,移置傕塢,盡燒宮殿、城門、官府、民舍,放兵寇鈔公卿以下。冬,天子東還雒陽,傕、汜追上到曹陽,虜掠乘輿輜重,殺光祿勳鄧淵、廷尉宣璠、少府田邠等數十人。

五行傳曰:「好攻戰,輕百姓,飾城郭,侵邊境,則金不從革。」謂金失其性而為災也。又曰:「言之不從,是謂不乂。厥咎僭,厥罰恆陽,厥極憂。時則有詩妖,時則有介蟲之孽,時則有犬禍,時則有口舌之痾,時則有白眚、白祥,惟木沴金。」介蟲,劉歆傳以為毛蟲。乂,治也。

安帝永初元年十一月,民訛言相驚,司隸、并、冀州民人流移。時鄧太后專政。婦人以順為道,故禮「夫死從子」之命。今專王事,此不從而僭也。

世祖建武五年夏,旱。京房傳曰:「欲德不用,茲謂張,厥災荒,其旱陰雲不雨,變而赤因四陰。眾出過時,茲謂廣,其旱不生。上下皆蔽,茲謂隔,其旱天赤三月,時有雹殺飛禽。上緣求妃,茲謂僭,其旱三月大溫亡雲。君高臺府,茲謂犯,陰侵陽,其旱萬物根死,有火災。庶位踰節,茲謂僭,其旱澤物枯,為火所傷。」是時天下僭逆者未盡誅,軍多過時。

章帝章和二年夏,旱。時章帝崩後,竇太后兄弟用事奢僭。

和帝永元六年秋,京都旱。時雒陽有冤囚,和帝幸雒陽寺,錄囚徒,理冤囚,牧令下獄抵罪。行未還宮,澍雨降。

安帝永初六年夏,旱。

七年夏,旱。

元初元年夏,旱。

二年夏,旱。

六年夏,旱。

順帝永建三年夏,旱。

五年夏,旱。

陽嘉二年夏,旱。時李固對策,以為奢僭所致也。

沖帝永嘉元年夏,旱。時沖帝幼崩,太尉李固勸太后及兄梁冀立嗣帝,擇年長有德者,天下賴之,則功名不朽。年幼未可知,如後不善,悔無所及。時太后及冀貪立年幼,欲久自專,遂立質帝,八歲。此不用德。

桓帝元嘉元年夏,旱。是時梁冀秉政,妻子並受封,寵踰節。

延熹元年六月,旱。

靈帝熹平五年夏,旱。

六年夏,旱。

光和五年夏,旱。

六年夏,旱。是時常侍、黃門僭作威福。

獻帝興平元年秋,長安旱。是時李傕、郭汜專權縱肆。

更始時,南陽有童謠曰:「諧不諧,在赤眉。得不得,在河北。」是時更始在長安,世祖為大司馬平定河北。更始大臣並僭專權,故謠妖作也。後更始遂為赤眉所殺,是更始之不諧在赤眉也。世祖自河北興。

世祖建武六年,蜀童謠曰:「黃牛白腹,五銖當復。」是時公孫述僭號於蜀,時人竊言王莽稱黃,述欲繼之,故稱白;五銖,漢家貨,明當復也。述遂誅滅。王莽末,天水童謠曰:「出吳門,望緹群。見一蹇人,言欲上天;令天可上,地上安得民!」時隗囂初起兵於天水,後意稍廣,欲為天子,遂破滅。囂少病蹇。吳門,冀郭門名也。緹群,山名也。

順帝之末,京都童謠曰:「直如弦,死道邊。曲如鉤,反封侯。」案順帝即世,孝質短祚,大將軍梁冀貪樹疏幼,以為己功,專國號令,以贍其私。太尉李固以為清河王雅性聰明,敦詩悅禮,加又屬親,立長則順,置善則固。而冀建白太后,策免固,徵蠡吾侯,遂即至尊。固是日幽斃于獄,暴屍道路,而太尉胡廣封安樂鄉侯、司徒趙戒廚亭侯、司空袁湯安國亭侯云。

桓帝之初,天下童謠曰:「小麥青青大麥枯,誰當穫者婦與姑。丈人何在西擊胡,吏買馬,君具車,請為諸君鼓嚨胡。」案元嘉中涼州諸羌一時俱反,南入蜀、漢,東抄三輔,延及并、冀,大為民害。命將出眾,每戰常負,中國益發甲卒,麥多委棄,但有婦女穫刈之也。吏買馬,君具車者,言調發重及有秩者也。請為諸君鼓嚨胡者,不敢公言,私咽語。

桓帝之初,京都童謠曰:「城上烏,尾畢逋。公為吏,子為徒。一徒死,百乘車。車班班,入河閒。河閒嚯女工數錢,以錢為室金為堂。石上慊慊舂黃粱。梁下有懸鼓,我欲擊之丞卿怒。」案此皆謂為政貪也。城上烏,尾畢逋者,處高利獨食,不與下共,謂人主多聚斂也。公為吏,子為徒者,言蠻夷將畔逆,父既為軍吏,其子又為卒徒往擊之也。一徒死,百乘車者,言前一人往討胡既死矣,後又遣百乘車往。車班班,入河閒者,言上將崩,乘輿班班入河閒迎靈帝也。河閒嚯女工數錢,以錢為室金為堂者,靈帝既立,其母永樂太后好聚金以為堂也。石上慊慊舂黃粱者,言永樂雖積金錢,慊慊常苦不足,使人舂黃粱而食之也。梁下有懸鼓,我欲擊之丞卿怒者,言永樂主教靈帝,使賣官受錢,所祿非其人,天下忠篤之士怨望,欲擊懸鼓以求見,丞卿主鼓者,亦復諂順,怒而止我也。

桓帝之初,京都童謠曰:「游平賣印自有平,不辟豪賢及大姓。」案到延熹之末,鄧皇后以譴自殺,乃以竇貴人代之,其父名武字游平,拜城門校尉。及太后攝政,為大將軍,與太傅陳蕃合心戮力,惟德是建,印綬所加,咸得其人,豪賢大姓,皆絕望矣。

桓帝之末,京都童謠曰:「茅田一頃中有井,四方纖纖不可整。嚼復嚼,今年尚可後年鐃。」案《易》曰:「拔茅茹以其彙,征吉。」茅喻群賢也。井者,法也。于時中常侍管霸、蘇康憎疾海內英哲,與長樂少府劉囂、太常許詠、尚書柳分、尋穆、史佟、司隸唐珍等,代作脣齒。河內牢川詣闕上書:「汝、潁、南陽,上采虛譽,專作威福;甘陵有南北二部,三輔尤甚。」由是傳考黃門北寺,始見廢閣。茅田一頃者,言群賢眾多也。中有井者,言雖阨窮,不失其法度也。四方纖纖不可整者,言姦慝大熾,不可整理。嚼復嚼者,京都飲酒相強之辭也。言食肉者鄙,不恤王政,徒耽宴飲歌呼而已也。今年尚可者,言但禁錮也。後年鐃者,陳、竇被誅,天下大壞。

桓帝之末,京都童謠曰:「白蓋小車何延延。河閒來合諧,河閒來合諧!」案解犢亭屬饒陽河閒縣也。居無幾何而桓帝崩,使者與解犢侯皆白蓋車從河閒來。延延,眾貌也。是時御史劉儵建議立靈帝,以儵為侍中,中常侍侯覽畏其親近,必當閒己,白拜儵泰山太守,因令司隸迫促殺之。朝廷必長,思其功效,乃拔用其弟郃,致位司徒,此為合諧也。

靈帝之末,京都童謠曰:「侯非侯,王非王,千乘萬騎上北芒。」案到中平六年,史侯登躡至尊,獻帝未有爵號,為中常侍段珪等數十人所執,公卿百官皆隨其後,到河上,乃得來還。此為非侯非王上北芒者也。

靈帝中平中,京都歌曰:「承樂世董逃,遊四郭董逃,蒙天恩董逃,帶金紫董逃,行謝恩董逃,整車騎董逃,垂欲發董逃,與中辭董逃,出西門董逃,瞻宮殿董逃,望京城董逃,日夜絕董逃,心摧傷董逃。」案「董」謂董卓也,言雖跋扈,縱其殘暴,終歸逃竄,至於滅族也。

獻帝踐祚之初,京都童謠曰:「千里草,何青青。十日卜,不得生。」案千里草為董,十日卜為卓。凡別字之體,皆從上起,左右離合,無有從下發端者也。今二字如此者,天意若曰:卓自下摩上,以臣陵君也。青青者,暴盛之貌也。不得生者,亦旋破亡。

建安初,荊州童謠曰:「八九年閒始欲衰,至十三年無孑遺。」言自中興以來,荊州無破亂,及劉表為牧,又豐樂,至此逮八九年。當始衰者,謂劉表妻當死,諸將並零落也。十三年無孑遺者,言十三年表又當死,民當移詣冀州也。

順帝陽嘉元年十月中,望都蒲陰狼殺童兒九十七人。時李固對策,引京房易傳曰「君將無道,害將及人,去之深山全身,厥災狼食人」。陛下覺寤,比求隱滯,故狼災息。

靈帝建寧中,群狼數十頭入晉陽南城門齧人。


\end{pinyinscope}