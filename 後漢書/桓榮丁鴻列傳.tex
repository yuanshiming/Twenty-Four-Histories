\article{桓榮丁鴻列傳}

\begin{pinyinscope}
桓榮字春卿,沛郡龍亢人也。少學長安,習歐陽尚書,事博士九江朱普。貧窶無資,常客傭以自給,精力不倦,十五年不闚家園。至王莽篡位乃歸。會朱普卒,榮奔喪九江,負土成墳,因留教授,徒眾數百人。莽敗,天下亂。榮抱其經書與弟子逃匿山谷,雖常飢困而講論不輟,後復客授江淮閒。

建武十九年,年六十餘,始辟大司徒府。時顯宗始立為皇太子,選求明經,乃擢榮弟子豫章何湯為虎賁中郎將,以尚書授太子。世祖從容問湯本師為誰,湯對曰:「事沛國桓榮。」帝即召榮,令說尚書,甚善之。拜為議郎,賜錢十萬,入使授太子。每朝會,輒令榮於公卿前敷奏經書。帝稱善,曰:「得生幾晚!」會歐陽博士缺,帝欲用榮。榮叩頭讓曰:「臣經術淺薄,不如同門生郎中彭閎、揚州從事皋弘。」帝曰:「俞,往,女諧。」因拜榮為博士,引閎、弘為議郎。

車駕幸大學,會諸博士論難於前,榮被服儒衣,溫恭有蘊籍,辯明經義,每以禮讓相猒,不以辭長勝人,儒者莫之及,特加賞賜。又詔諸生雅吹擊磬,盡日乃罷。後榮入會庭中,詔賜奇果,受者皆懷之,榮獨舉手捧之以拜。帝笑指之曰:「此真儒生也。」以是愈見敬厚,常令止宿太子宮。積五年,榮薦門下生九江胡憲侍講,乃聽得出,旦一入而已。榮嘗寢病,太子朝夕遣中傅問病,賜以珍羞、帷帳、奴婢,謂曰:「如有不諱,無憂家室也。」後病愈,復入侍講。

二十八年,大會百官,詔問誰可傅太子者,群臣承望上意,皆言太子舅執金吾原鹿侯陰識可。博士張佚正色曰:「今陛下立太子,為陰氏乎?為天下乎?即為陰氏,則陰侯可;為天下,則固宜用天下之賢才。」帝稱善,曰:「欲置傅者,以輔太子也。今博士不難正朕,況太子乎?」即拜佚為太子太傅,而以榮為少傅,賜以輜車、乘馬。榮大會諸生,陳其車馬、印綬,曰:「今日所蒙,稽古之力也,可不勉哉!」榮以太子經學成畢,上疏謝曰:「臣幸得侍帷幄,執經連年,而智學淺短,無以補益萬分。今皇太子以聰叡之姿,通明經義,觀覽古今,儲君副主莫能專精博學若此者也。斯誠國家福祐,天下幸甚。臣師道已盡,皆在太子,謹使掾臣汜再拜歸道。」太子報書曰:「莊以童蒙,學道九載,而典訓不明,無所曉識。夫五經廣大,聖言幽遠,非天下之至精,豈能與於此!況以不才,敢承誨命。昔之先師謝弟子者有矣,上則通達經旨,分明章句,下則去家慕鄉,求謝師門。今蒙下列,不敢有辭,願君慎疾加餐,重愛玉體。」

三十年,拜為太常。榮初遭倉卒,與族人桓元卿同飢厄,而榮講誦不息。元卿嗤榮曰:「但自苦氣力,何時復施用乎?」榮笑不應。及為太常,元卿歎曰:「我農家子,豈意學之為利乃若是哉!」

顯宗即位,尊以師禮,甚見親重,拜二子為郎。榮年踰八十,自以衰老,數上書乞身,輒加賞賜。乘輿嘗幸太常府,令榮坐東面,設几杖,會百官驃騎將軍東平王蒼以下及榮門生數百人,天子親自執業,每言輒曰「大師在是」。既罷,悉以太官供具賜太常家。其恩禮若此。

永平二年,三雍初成,拜榮為五更。每大射養老禮畢,帝輒引榮及弟子升堂,執經自為下說。乃封榮為關內侯,食邑五千戶。

榮每疾病,帝輒遣使者存問,太官、太醫相望於道。及篤,上疏謝恩,讓還爵土。帝幸其家問起居,入街下車,擁經而前,撫榮垂涕,賜以床茵、帷帳、刀劍、衣被,良久乃去。自是諸侯將軍大夫問疾者,不敢復乘車到門,皆拜床下。榮卒,帝親自變服,臨喪送葬,賜冢塋于首山之陽。除兄子二人補四百石,都講生八人補二百石,其餘門徒多至公卿。子郁嗣。

論曰:張佚訐切陰侯,以取高位,危言犯眾,義動明后,知其直有餘也。若夫一言納賞,志士為之懷恥;受爵不讓,風人所以興歌。而佚廷議戚援,自居全德,意者以廉不足乎?昔樂羊食子,有功見疑;西巴放麑,以罪作傅。蓋推仁審偽,本乎其情。君人者能以此察,則真邪幾於辨矣。

郁字仲恩,少以父任為郎。敦厚篤學,傳父業,以尚書教授,門徒常數百人。榮卒,郁當襲爵,上書讓於兄子汎,顯宗不許,不得已受封,悉以租入與之。帝以郁先師子,有禮讓,甚見親厚,常居中論經書,問以政事,稍遷侍中。帝自制五家要說章句,令郁校定於宣明殿,以侍中監虎賁中郎將。

永平十五年,入授皇太子經,遷越騎校尉,詔敕太子、諸王各奉賀致禮。郁數進忠言,多見納錄。肅宗即位,郁以母憂乞身,詔聽以侍中行服。建初二年,遷屯騎校尉。

和帝即位,富於春秋,侍中竇憲自以外戚之重,欲令少主頗涉經學,上疏皇太后曰:「禮記云:『天下之命,懸於天子;天子之善,成乎所習。習與智長,則切而不勤;化與心成,則中道若性。昔成王幼小,越在襁保,周公在前,史佚在後,太公在左,召公在右。中立聽朝,四聖維之。是以慮無遺計,舉無過事。』孝昭皇帝八歲即位,大臣輔政,亦選名儒韋賢、蔡義、夏侯勝等入授於前,平成聖德。近建初元年,張酺、魏應、召訓亦講禁中。臣伏惟皇帝陛下,躬天然之姿,宜漸教學,而獨對左右小臣,未聞典義。昔五更桓榮,親為帝師,子郁,結髮敦尚,繼傳父業,故再以校尉入授先帝,父子給事禁省,更歷四世,今白首好禮,經行篤備。又宗正劉方,宗室之表,善為詩經,先帝所褒。宜令郁、方並入教授,以崇本朝,光示大化。」由是遷長樂少府,復入侍講。頃之,轉為侍中奉車都尉。永元四年,代丁鴻為太常。明年,病卒。

郁經授二帝,恩寵甚篤,賞賜前後數百千萬,顯於當世。門人楊震、朱寵,皆至三公。

初,榮受朱普學章句四十萬言,浮辭繁長,多過其實。及榮入授顯宗,減為二十三萬言。郁復刪省定成十二萬言。由是有桓君大小太常章句。

子普嗣,傳爵至曾孫。郁中子焉,能世傳其家學。孫鸞、曾孫彬,並知名。

焉字叔元,少以父任為郎。明經篤行,有名稱。永初元年,入授安帝,三遷為侍中步兵校尉。永寧中,順帝立為皇太子,以焉為太子少傅,月餘,遷太傅,以母憂自乞,聽以大夫行喪。踰年,詔使者賜牛酒,奪服,即拜光祿大夫,遷太常。時廢皇太子為濟陰王,焉與太僕來歷、廷尉張皓諫,不能得,事已具來歷傳。

順帝即位,拜太傅,與太尉朱寵並錄尚書事。焉復入授經禁中,因讌見,建言宜引三公、尚書入省事,帝從之。以焉前廷議守正,封陽平侯,固讓不受。視事三年,坐辟召禁錮者為吏免。復拜光祿大夫。陽嘉二年,代來歷為大鴻臚,數日,遷為太常。永和五年,代王龔為太尉。漢安元年,以日食免。明年,卒於家。

弟子傳業者數百人,黃瓊、楊賜最為顯貴。焉孫典。

典字公雅,復傳其家業,以尚書教授潁川,門徒數百人。舉孝廉為郎。居無幾,會國相王吉以罪被誅,故人親戚莫敢至者。典獨棄官收斂歸葬,服喪三年,負土成墳,為立祠堂,盡禮而去。

辟司徒袁隗府,舉高第,拜侍御史。是時宦官秉權,典執政無所回避。常乘驄馬,京師畏憚,為之語曰:「行行且止,避驄馬御史。」及黃巾賊起滎陽,典奉使督軍。賊破,還,以啎宦官賞不行。在御史七年不調,後出為郎。

靈帝崩,大將軍何進秉政,典與同謀議,三遷羽林中郎將。獻帝即位,三公奏典前與何進謀誅閹官,功雖不遂,忠義炳著。詔拜家一人為郎,賜錢二十萬。

從西入關,拜御史中丞,賜爵關內侯。車駕都許,遷光祿勳。建安六年,卒官。

鸞字始春,焉弟子也。少立操行,褞袍糟食,不求盈餘。以世濁,州郡多非其人,恥不肯仕。

年四十餘,時太守向苗有名跡,乃舉鸞孝廉,遷為膠東令。始到官而苗卒,鸞即去職奔喪,終三年然後歸,淮汝之閒高其義。後為巳吾、汲二縣令,甚有名跡。諸公並薦,復徵辟拜議郎。上陳五事:舉賢才,審授用,黜佞倖,省苑囿,息役賦。書奏御,啎內豎,故不省。以病免。中平元年,年七十七,卒于家。子曄。

曄字文林,一名嚴,尤修志介。姑為司空楊賜夫人。初鸞卒,姑歸寧赴哀,將至,止於傳舍,整飾從者而後入,曄心非之。及姑勞問,終無所言,號哭而已。賜遣吏奉祠,因縣發取祠具,曄拒不受。後每至京師,未嘗舍宿楊氏。其貞忮若此。賓客從者,皆祗其志行,一餐不受於人。仕為郡功曹。後舉孝廉、有道、方正、茂才,三公並辟,皆不應。

初平中,天下亂,避地會稽,遂浮海客交阯,越人化其節,至閭里不爭訟。為凶人所誣,遂死于合浦獄。

彬字彥林,焉之兄孫也。

父麟,字元鳳,早有才惠。桓帝初,為議郎,入侍講禁中,以直道啎左右,出為許令,病免。會母終,麟不勝喪,未祥而卒,年四十一。所著碑、誄、讚、說、書凡二十一篇。

彬少與蔡邕齊名。初舉孝廉,拜尚書郎。時中常侍曹節女婿馮方亦為郎,彬厲志操,與左丞劉歆、右丞杜希同好交善,未嘗與方共酒食之會,方深怨之,遂章言彬等為酒黨。事下尚書令劉猛,雅善彬等,不舉正其事,節大怒,劾奏猛,以為阿黨,請收下詔獄,在朝者為之寒心,猛意氣自若,旬日得出,免官禁錮。彬遂以廢。光和元年,卒於家,年四十六。諸儒莫不傷之。

所著七說及書凡三篇,蔡邕等共論序其志,僉以為彬有過人者四:夙智早成,岐嶷也;學優文麗,至通也;仕不苟祿,絕高也;辭隆從窊,絜操也。乃共樹碑而頌焉。

劉猛,琅邪人。桓帝時為宗正,直道不容,自免歸家。靈帝即位,太傅陳蕃、大將軍竇武輔政,復徵用之。

論曰:伏氏自東西京相襲為名儒,以取爵位。中興而桓氏尤盛,自榮至典,世宗其道,父子兄弟代作帝師,受其業者皆至卿相,顯乎當世。子曰:「古之學者為己,今之學者為人。」為人者,憑譽以顯物;為己者,因心以會道。桓榮之累世見宗,豈其為己乎!

丁鴻字孝公,潁川定陵人也。

父綝,字幼春,王莽末守潁陽尉。世祖略地潁陽,潁陽城守不下,綝說其宰,遂與俱降,世祖大喜,厚加賞勞,以綝為偏將軍,因從征伐。綝將兵先度河,移檄郡國,攻營略地,下河南、陳留、潁川二十一縣。

建武元年,拜河南太守。及封功臣,帝令各言所樂,諸將皆占豐邑美縣,唯綝願封本鄉。或謂綝曰:「人皆欲縣,子獨求鄉,何也?」綝曰:「昔孫叔敖敕其子,受封必求墝埆之地,今綝能薄功微,得鄉亭厚矣。」帝從之,封定陵新安鄉侯,食邑五千戶,後徙封陵陽侯。

鴻年十三,從桓榮受歐陽尚書,三年而明章句,善論難,為都講,遂篤志精銳,布衣荷擔,不遠千里。

初,綝從世祖征伐,鴻獨與弟盛居,憐盛幼小而共寒苦。及綝卒,鴻當襲封,上書讓國於盛,不報。既葬,乃挂縗絰於冢廬而逃去,留書與盛曰:「鴻貪經書,不顧恩義,弱而隨師,生不供養,死不飯唅,皇天先祖,並不祐助,身被大病,不任茅土。前上疾狀,願辭爵仲公,章寢不報,迫且當襲封。謹自放棄,逐求良醫。如遂不瘳,永歸溝壑。」鴻初與九江人鮑駿同事桓榮,甚相友善,及鴻亡封,與駿遇於東海,陽狂不識駿。駿乃止而讓之曰:「昔伯夷、吳札亂世權行,故得申其志耳。春秋之義,不以家事廢王事。今子以兄弟私恩而絕父不滅之基,可謂智乎?」鴻感悟,垂涕歎息,乃還就國,開門教授。鮑駿亦上書言鴻經學至行,顯宗甚賢之。

永平十年詔徵,鴻至即召見,說文侯之命篇,賜御衣及綬,稟食公車,與博士同禮。頃之,拜侍中。十三年,兼射聲校尉。建初四年,徙封魯陽鄉侯。

肅宗詔鴻與廣平王羨及諸儒樓望、成封、桓郁、

賈逵等,論定五經同異於北宮白虎觀,使五官中郎將魏應主承制問難,侍中淳于恭奏上,帝親稱制臨決。鴻以才高,論難最明,諸儒稱之,帝數嗟美焉。時人嘆曰:「殿中無雙丁孝公。」數受賞賜,擢徙校書,遂代成封為少府。門下由是益盛,遠方至者數千人。彭城劉愷、北海巴茂、九江朱倀皆至公卿。元和三年,徙封馬亭鄉侯。

和帝即位,遷太常。永元四年,代袁安為司徒。是時竇太后臨政,憲兄弟各擅威權。鴻因日食,上封事曰:

臣聞日者陽精,守實不虧,君之象也;月者陰精,盈毀有常,臣之表也。故日食者,臣乘君,陰陵陽;月滿不虧,下驕盈也。昔周室衰季,皇甫之屬專權於外,黨類強盛,侵奪主埶,則日月薄食,故《詩》曰:「十月之交,朔月辛卯,日有食之,亦孔之醜。」春秋日食三十六,弒君三十二。變不空生,各以類應。夫威柄不以放下,利器不可假人。覽觀往古,近察漢興,傾危之禍,靡不由之。是以三桓專魯,田氏擅齊,六卿分晉;諸呂握權,統嗣幾移;哀、平之末,廟不血食。故雖有周公之親,而無其德,不得行其埶也。

今大將軍雖欲敕身自約,不敢僭差,然而天下遠近皆惶怖承旨,刺史二千石初除謁辭,求通待報,雖奉符璽,受臺敕,不敢便去,久者至數十日。背王室,向私門,此乃上威損,下權盛也。人道悖於下,效驗見於天,雖有隱謀,神照其情,垂象見戒,以告人君。閒者月滿先節,過望不虧,此臣驕溢背君,專功獨行也。陛下未深覺悟,故天重見戒,誠宜畏懼,以防其禍。《詩》云:「敬天之怒,不敢戲豫。」若敕政責躬,杜漸防萌,則凶妖銷滅,害除福湊矣。

夫壞崖破巖之水,源自涓涓;干雲蔽日之木,起於蔥青。禁微則易,救末者難,人莫不忽於微細,以致其大。恩不忍誨,義不忍割,去事之後,未然之明鏡也。臣愚以為左官外附之臣,依託權門,傾覆諂諛,以求容媚者,宜行一切之誅。閒者大將軍再出,威振州郡,莫不賦斂吏人,遣使貢獻。大將軍雖云不受,而物不還主,部署之吏無所畏憚,縱行非法,不伏罪辜,故海內貪猾,競為姦吏,小民吁嗟,怨氣滿腹。臣聞天不可以不剛,不剛則三光不明;王不可以不彊,不彊則宰牧從橫。宜因大變,改政匡失,以塞天意。

書奏十餘日,帝以鴻行太尉兼衛尉,屯南、北宮。於是收竇憲大將軍印綬,憲及諸弟皆自殺。

時大郡口五六十萬舉孝廉二人,小郡口二十萬并有蠻夷者亦舉二人,帝以為不均,下公卿會議。鴻與司空劉方上言:「凡口率之科,宜有階品,蠻夷錯雜,不得為數。自今郡國率二十萬口歲舉孝廉一人,四十萬二人,六十萬三人,八十萬四人,百萬五人,百二十萬六人。不滿二十萬二歲一人,不滿十萬三歲一人。」帝從之。

六年,鴻薨,賜贈有加常禮。子湛嗣。卒,子浮嗣。浮卒,子夏嗣。

論曰:孔子曰「太伯三以天下讓,民無得而稱焉」。孟子曰「聞伯夷之風者,貪夫廉,懦夫有立志」。若乃太伯以天下而違周,伯夷率絜情以去國,並未始有其讓也。故太伯稱至德,伯夷稱賢人。後世聞其讓而慕其風,徇其名而昧其致,所以激詭行生而取與妄矣。至夫鄧彪、劉愷,讓其弟以取義,使弟受非服而己厚其名,於義不亦薄乎!君子立言,非苟顯其理,將以啟天下之方悟者;立行,非獨善其身,將以訓天下之方動者。言行之所開塞,可無慎哉!原丁鴻之心,主於忠愛乎?何其終悟而從義也!異夫數子類乎徇名者焉。

贊曰:五更待問,應若鳴鍾。庭列輜駕,堂修禮容。穆穆帝則,擁經以從。丁鴻翼翼,讓而不飾。高論白虎,深言日食。


\end{pinyinscope}