\article{宗室四王三侯列傳}

\begin{pinyinscope}
齊武王縯字伯升,光武之長兄也。性剛毅,慷慨有大節。自王莽篡漢,常憤憤,懷復社稷之慮,不事家人居業,傾身破產,交結天下雄俊。

莽末,盜賊群起,南方尤甚。伯升召諸豪傑計議曰:「王莽暴虐,百姓分崩。今枯旱連年,兵革並起。此亦天亡之時,復高祖之業,定萬世之秋也。」眾皆然之。於是分遣親客,使鄧晨起新野,光武與李通、李軼起於宛。伯升自發舂陵子弟,合七八千人,部署賓客,自稱柱天都部。使宗室劉嘉往誘新市、平林兵王匡、陳牧等,合軍而進,屠長聚及唐子鄉,殺湖陽尉,進拔棘陽,因欲攻宛。至小長安,與王莽前隊大夫甄阜、屬正梁丘賜戰。時天密霧,漢軍大敗,姊元弟仲皆遇害,宗從死者數十人。伯升復收會兵眾,還保棘陽。

阜、賜乘勝,留輜重於藍鄉,引精兵十萬南渡黃淳水,臨泚水,阻兩川閒為營,絕後橋,示無還心。新巿、平林見漢兵數敗,阜、賜軍大至,各欲解去,伯升甚患之。會下江兵五千餘人至宜秋,乃往為說合從之埶,下江從之。語在王常傳。伯升於是大饗軍士,設盟約。休卒三日,分為六部,潛師夜起,襲取藍鄉。盡獲其輜重。明旦,漢軍自西南攻甄阜,下江兵自東南攻梁丘賜。至食時,賜陳潰,阜軍望見散走,漢兵急追之,卻迫黃淳水,斬首溺死者二萬餘人,遂斬阜、賜。

王莽納言將軍嚴尤、秩宗將軍陳茂聞阜、賜軍敗

,引欲據宛,伯升乃陳兵誓眾,焚積聚,破釜甑,鼓行而前,與尤、茂遇育陽下,戰,大破之,斬首三千餘級。尤、茂棄軍走,伯升遂進圍宛,自號柱天大將軍。王莽素聞其名,大震懼,購伯升邑五萬戶,黃金十萬斤,位上公。使長安中官署及天下鄉亭皆畫伯升像於塾,旦起射之。

自阜、賜死後,百姓日有降者,眾至十餘萬。諸將會議立劉氏以從人望,豪傑咸歸於伯升。而新巿、平林將帥樂放縱,憚伯升威明而貪聖公懦弱,先共定策立之,然後使騎召伯升,示其議。伯升曰:「諸將軍幸欲尊立宗室,其德甚厚,然愚鄙之見,竊有未同。今赤眉起青、徐,眾數十萬,聞南陽立宗室,恐赤眉復有所立,如此,必將內爭。今王莽未滅,而宗室相攻,是疑天下而自損權,非所以破莽也。且首兵唱號,鮮有能遂,陳勝、項籍,即其事也。舂陵去宛三百里耳,未足為功。遽自尊立,為天下準的,使後人得承吾敝,非計之善者也。今且稱王以號令。若赤眉所立者賢,相率而往從之;若無所立,破莽降赤眉,然後舉尊號,亦未晚也。願各詳思之。」諸將多曰「善」。將軍張卬拔劍擊地曰:「疑事無功。今日之議,不得有二。」眾皆從之。

聖公既即位,拜伯升為大司徒,封漢信侯。由是豪傑失望,多不服。平林後部攻新野,不能下。新野宰登城言曰:「得司徒劉公一信,願先下。」及伯升軍至,即開城門降。五月,伯升拔宛。六月,光武破王尋、王邑。自是兄弟威名益甚。

更始君臣不自安,遂共謀誅伯升,乃大會諸將,以成其計。更始取伯升寶劍視之,繡衣御史申屠建隨獻玉玦,更始竟不能發。及罷會,伯升舅樊宏謂伯升曰:「昔鴻門之會,范增舉玦以示項羽。今建此意,得無不善乎?」伯升笑而不應。初,李軼諂事更始貴將,光武深疑之,常以戒伯升曰:「此人不可復信。」又不受。

伯升部將宗人劉稷,數陷陳潰圍,勇冠三軍。時將兵擊魯陽,聞更始立,怒曰:「本起兵圖大事者,伯升兄弟也,今更始何為者邪?」更始君臣聞而心忌之,以稷為抗威將軍,稷不肯拜。更始乃與諸將陳兵數千人,先收稷,將誅之,伯升固爭。李軼、朱鮪因勸更始并執伯升,即日害之。

有二子。建武二年,立長子章為太原王,興為魯王。十一年,徙章為齊王。十五年,追謚伯升為齊武王。

章少孤,光武感伯升功業不就,撫育恩愛甚篤,以其少貴,欲令親吏事,故使試守平陰令,遷梁郡太守。立二十一年薨,謚曰哀王。子煬王石嗣。建武二十七年,石始就國。三十年,封石弟張為下博侯。永平十四年,封石二子為鄉侯。石立二十四年薨,子晃嗣。

下博侯張以善論議,十六年,與奉車都尉竇固等並出擊匈奴,後進者多害其能,數被譖訴。建初中卒,肅宗下詔褒揚之,復封張子它人奉其祀。

晃及弟利侯剛與母太姬宗更相誣告。章和元年,有司奏請免晃、剛爵為庶人,徙丹陽。帝不忍,下詔曰:「朕聞人君正屏,有所不聽。宗尊為小君,宮衛周備,出有輜軿之飾,入有牖戶之固,殆不至如譖者之言。晃、剛愆乎至行,濁乎大倫,甫刑三千,莫大不孝。朕不忍置之于理,其貶晃爵為蕪湖侯,削剛戶三千。於戲!小子不勗大道,控于法理,以墮宗緒。其遣謁者收晃及太姬璽綬。」晃立十七年而降爵。晃卒,子無忌嗣。

帝以伯升首創大業,而後嗣罪廢,心常愍之。時北海亦絕無後。及崩,遺詔令復二國。永元二年,乃復封無忌為齊王,是為惠王。立五十二年薨,子頃王喜嗣。立五年薨,子承嗣。建安十一年,國除。

論曰:大丈夫之鼓動拔起,其志致蓋遠矣。若夫齊武王之破家厚士,豈游俠下客之為哉!其慮將存乎配天之絕業,而痛明堂之不祀也。及其發舉大謀,在倉卒擾攘之中,使信先成於敵人,赦岑彭以顯義,若此足以見其度矣。志高慮遠,禍發所忽。嗚呼!古人以蜂蠆為戒,蓋畏此也。《詩》云:「敬之敬之,命不易哉!」

北海靖王興,建武二年封為魯王,嗣光武兄仲。

初,南頓君娶同郡樊重女,字嫺都。嫺都性婉順,自為童女。不正容服不出於房,宗族敬焉,生三男三女:長男伯升,次仲,次光武;長女黃,次元,次伯姬。皇妣以初起兵時病卒,宗人樊巨公收斂焉。建武二年,封黃為湖陽長公主,伯姬為寧平長公主。元與仲俱歿於小長安,追爵元為新野長公主,十五年,追謚仲為魯哀王。

興其歲試守緱氏令。為人有明略,善聽訟,甚得名稱。遷弘農太守,亦有善政。視事四年,上疏乞骸骨,徵還京師,奉朝請。二十七年,始就國。明年,以魯國益東海,故徙興為北海王。三十年,封興子復為臨邑侯。中元二年,又封興二子為縣侯。顯宗器重興,每有異政,輒乘驛問焉。立三十九年薨,子敬王睦嗣。

睦少好學,博通書傳,光武愛之,數被延納。顯宗之在東宮,尤見幸待,入侍諷誦,出則執轡。中興初,禁網尚闊,而睦性謙恭好士,千里交結,自名儒宿德,莫不造門,由是聲價益廣。永平中,法憲頗峻,睦乃謝絕賓客,放心音樂。然性好讀書,常為愛翫。歲終,遣中大夫奉璧朝賀,召而謂之曰:「朝廷設問寡人,大夫將何辭以對?」使者曰:「大王忠孝慈仁,敬賢樂士。臣雖螻蟻,敢不以實?」睦曰:「吁,子危我哉!此乃孤幼時進趣之行也。大夫其對以孤襲爵以來,志意衰惰,聲色是娛,犬馬是好。」使者受命而行。其能屈申若此。

初,靖王薨,悉推財產與諸弟,雖王車服珍寶非列侯制,皆以為分,然後隨以金帛贖之。睦能屬文,作春秋旨義終始論及賦頌數十篇。又善史書,當世以為楷則。及寢病,帝驛馬令作草書尺牘十首。立十年薨,子哀王基嗣。

永平十八年,封基二弟為縣侯,二弟為鄉侯。建初二年,又封基弟毅為平望侯。基立十四年薨,無子,肅宗憐之,不除其國。

永元二年,和帝封睦庶子斟鄉侯威為北海王,奉睦後。立七年,威以非睦子,又坐誹謗,檻車徵詣廷尉,道自殺。

永初元年,鄧太后復封睦孫壽光侯普為北海王,是為頃王。延光二年,復封睦少子為亭侯。普立〈十〉七年薨,子恭王翼嗣;立十四年薨,子康王嗣,無後,建安十一年,國除。

初,臨邑侯復好學,能文章。永平中,每有講學事,輒令復典掌焉。與班固、賈逵共述漢史,傅毅等皆宗事之。復子騊駼及從兄平望侯毅,並有才學。永寧中,鄧太后召毅及騊駼入東觀,與謁者僕射劉珍著中興以下名臣列士傳。騊駼又自造賦、頌、書、論凡四篇。

趙孝王良字次伯,光武之叔父也。平帝時舉孝廉,為蕭令。光武兄弟少孤,良撫循甚篤。及光武起兵,以事告,良大怒,曰:「汝與伯升志操不同,今家欲危亡,而反共謀如是!」既而不得已,從軍至小長安,漢兵大敗,良妻及二子皆被害。更始立,以良為國三老,從入關。更始敗,良聞光武即位,乃亡奔洛陽。建武二年,封良為廣陽王。五年,徙為趙王,始就國。十三年,降為趙公。頻歲來朝。十七年,薨于京師。凡立十六年。子節王栩嗣。建武三十年,封栩二子為鄉侯。建初二年,復封栩十子為亭侯。

栩立四十年薨,子頃王商嗣。永元三年,封商三弟為亭侯。元年,封商四子為亭侯。

商立二十三年薨,子靖王宏。立十二年薨,子惠王乾嗣。

元初五年,封乾二弟為亭侯。是歲,趙相奏乾居父喪私娉小妻,又白衣出司馬門,坐削中丘縣。時郎中南陽程堅素有志行,拜為乾傅。堅輔以禮義,乾改悔前過,堅列上,復所削縣。本初元年,封乾一子為亭侯。乾立四十八年薨,子懷王豫嗣。豫薨,子獻王赦嗣。赦薨,子珪嗣,建安十八年徙封博陵王。立九年,魏初以為崇德侯。

城陽恭王祉字巨伯,光武族兄舂陵康侯敞之子也。

敞曾祖父節侯買,以長沙定王子封於零道之舂陵鄉,為舂陵侯。買卒,子戴侯熊渠嗣。熊渠卒,子考侯仁嗣。仁以舂陵地埶下溼,山林毒氣,上書求減邑內徙。元帝初元四年,徙封南陽之白水鄉,猶以舂陵為國名,遂與從弟鉅鹿都尉回及宗族往家焉。仁卒,子敞嗣。敞謙儉好義,盡推父時金寶財產與昆弟,荊州刺史上其義行,拜廬江都尉。歲餘,會族兄安眾侯劉崇起兵,王莽畏惡劉氏,徵敞至長安,免歸國。

先是平帝時,敞與崇俱朝京師,助祭明堂。崇見莽將危漢室,私謂敞曰:「安漢公擅國權,群臣莫不回從,社稷傾覆至矣。太后春秋高,天子幼弱,高皇帝所以分封子弟,蓋為此也。」敞心然之。及崇事敗,敞懼,欲結援樹黨,乃為祉娶高陵侯翟宣女為妻。會宣弟義起兵欲攻莽,南陽捕殺宣女,祉坐繫獄。敞因上書謝罪,願率子弟宗族為士卒先。莽新居攝,欲慰安宗室,故不被刑誅。及莽篡立,劉氏為侯者皆降稱子,食孤卿祿,後皆奪爵。及敞卒,祉遂特見廢,又不得官為吏。

祉以故侯嫡子,行淳厚,宗室皆敬之。及光武起兵,祉兄弟相率從軍,前隊大夫甄阜盡收其家屬繫宛獄。及漢兵敗小長安,祉挺身還保棘陽,甄阜盡殺其母弟妻子。更始立,以祉為太常將軍,紹封舂陵侯。從西入關,封為定陶王。別將擊破劉嬰於臨涇。

及更始降於赤眉,祉乃閒行亡奔洛陽。是時宗室唯祉先至,光武見之歡甚。建武二年,封為城陽王,賜乘輿、御物、車馬、衣服。追謚敞為康侯。十一年,祉疾病,上城陽王璽綬,願以列侯奉先人祭祀。帝自臨其疾。祉薨,年四十三,謚曰恭王,竟不之國,葬於洛陽北芒。

十三年,封祉嫡子平為蔡陽侯,以奉祉祀;平弟堅為高鄉侯。

初,建武二年,以皇祖、皇考墓為昌陵,置陵令守視;後改為章陵,因以舂陵為章陵縣。十八年,立考侯、康侯廟,比園陵,置嗇夫。詔零陵郡奉祠節侯、戴侯廟,以四時及臘歲五祠焉。置嗇夫、佐吏各一人。

平後坐與諸王交通,國除。永平五年,顯宗更封平為竟陵侯。平卒,子真嗣。真卒,子禹嗣。禹卒,子嘉嗣。

泗水王歙字經孫,光武族父也。歙子終,與光武少相親愛。漢兵起,始及唐子,終誘殺湖陽尉。更始立,歙從入關,封為元氏王,終為侍中。更始敗,歙、終東奔洛陽。建武二年,立歙為泗水王,終為淄川王。十年,歙薨,封小子燀為堂谿侯,奉歙後。終居喪思慕,哭泣二十餘日,亦薨。封長子柱為邔侯,以奉終祀,又封終子鳳曲陽侯。

歙從父弟茂,年十八,漢兵之起,茂自號劉失職,亦聚眾京、密閒,稱厭新將軍。攻下潁川、汝南,眾十餘萬人。光武既至河內,茂率眾降,封為中山王。十三年,宗室為王者皆降為侯,更封茂為穰侯。

茂弟匡,亦與漢兵俱起。建武二年,封宜春侯。為人謙遜,永平中為宗正。子浮嗣,封朝陽侯。

浮弟尚,永元中為征西將軍。浮傳國至孫護,無子,封絕。延光中,護從兄瑰與安帝乳母王聖女伯榮私通,遂取伯榮為妻,得紹護封為朝陽侯,位侍中。及王聖敗,貶爵為亭侯。

安成孝侯賜字子琴,光武族兄也。祖父利,蒼梧太守。賜少孤。兄顯報怨殺人,吏捕顯殺之。賜與顯子信賣田宅,同拋財產,結客報吏,皆亡命逃伏,遭赦歸。會伯升起兵,乃隨從攻擊諸縣。

更始既立,以賜為光祿勳,封廣漢侯。及伯升被害,代為大司徒,將兵討汝南。未及平,更始又以信為奮威大將軍,代賜擊汝南,賜與更始俱到洛陽。更始欲令親近大將徇河北,未知所使。賜言諸家子獨有文叔可用,大司馬朱鮪等以為不可,更始狐疑,賜深勸之,乃拜光武行大司馬,持節過河。是日以賜為丞相,令先入關,修宗廟宮室。還迎更始都長安,封賜為宛王,拜前大司馬,使持節鎮撫關東。二年春,賜就國於宛,典將六部兵。後赤眉破更始,賜所領六部亦稍散畔,乃去宛保育陽。

聞光武即位,乃西之武關,迎更始妻子將詣洛陽。帝嘉賜忠,建武二年,封為慎侯。十三年,更增戶邑,定封為安成侯,奉朝請。以賜有恩信,故親厚之,數蒙讌私,時幸其第,恩賜特異。賜輒賑與故舊,無有遺積。帝為營冢堂,起祠廟,置吏卒,如舂陵孝侯。二十八年卒,子閔嗣。

三十年,帝復封閔弟嵩為白牛侯。坐楚事,辭語相連,國除。閔卒,子商嗣,徙封為白牛侯。商卒,子昌嗣。

初,信為更始討平汝南,因封為汝陰王。信遂將兵平定江南,據豫章。光武即位,桂陽太守張隆擊破之,信乃詣洛陽降,以為汝陰侯。永平十三年,亦坐楚事國除。

成武孝侯順字平仲,光武族兄也。父慶,舂陵侯敞同產弟。順與光武同里閈,少相厚。

更始即位,以慶為燕王,順為虎牙將軍。會更始降赤眉,慶為亂兵所叔,順乃閒行詣光武,拜為南陽太守。建武二年,封成武侯,邑戶最大,租入倍宗室諸家。八年,使擊破六安賊,因拜為六安太守。數年,帝欲徵之,吏人上書請留。十一年卒,帝使使者迎喪,親自臨弔。子遵嗣,坐與諸王交通,降為端氏侯。遵卒,子弇嗣。弇卒,無嗣,國除。永平十年,顯宗幸章陵,追念舊恩,封順弟子三人為鄉侯。

初,順叔父弘娶於樊氏,皇妣之從妹也。生二子:敏,國。與母隨更始在長安。建武二年,詣洛陽,光武封敏為甘里侯,國為弋陽侯。敏通經有行,永平初,官至越騎校尉。

弘弟梁,以俠氣聞,更始元年,起兵豫章,欲徇江東,自號「就漢大將軍」,暴病卒。

順陽懷侯嘉字孝孫,光武族兄也。父憲,舂陵侯敞同產弟。嘉少孤,性仁厚,南頓君養視如子,後與伯升俱學長安,習尚書、春秋。

及義兵起,嘉隨更始征伐。漢軍之敗小長安也,嘉妻子遇害。更始即位,以為偏將軍。及攻破宛,封興德侯,遷大將軍。擊延岑於冠軍,降之。更始既都長安,以嘉為漢中王、扶威大將軍,持節就國,都於南鄭,眾數十萬。建武二年,延岑復反,攻漢中,圍南鄭,嘉兵敗走。岑遂定漢中,進兵武都,為更始柱功侯李寶所破。岑走天水,公孫述遣將侯丹取南鄭。嘉收散卒,得數萬人,以寶為相,從武都南擊侯丹,不利,還軍河池、下辨。復與延岑連戰,岑引北入散關,至陳倉,嘉追擊破之。更始鄧王廖湛將赤眉十八萬攻嘉,嘉與戰於谷口,大破之。嘉手殺湛,遂到雲陽就穀。

李寶等聞鄧禹西征,擁兵自守,勸嘉且觀成敗。光武聞之,告禹曰:「孝孫素謹善,少且親愛,當是長安輕薄兒誤之耳。」禹即宣帝旨,嘉乃因來歙詣禹於雲陽。三年,到洛陽,從征伐,拜為千乘太守。六年,病,上書乞骸骨,徵詣京師。十三年,封為順陽侯。秋,復封嘉子廧為黃李侯。十五年,嘉卒。子參嗣,有罪,削為南鄉侯。永平中,參為城門校尉。參卒,子循嗣。循卒,子章嗣。

贊曰:齊武沈雄,義戈乘風。倉卒匪圖,亡我天工。城陽早協,趙孝晚同。泗水三侯,或恩或功。


\end{pinyinscope}