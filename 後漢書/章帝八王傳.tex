\article{章帝八王傳}

\begin{pinyinscope}
孝章皇帝八子:宋貴人生清河孝王慶,梁貴人生和帝,申貴人生濟北惠王壽,河閒孝王開,四王不載母氏。

千乘貞王伉,建初四年封。和帝即位,以伉長兄,甚見尊禮。立十五年薨。

子寵嗣,一名伏胡。永元七年,改國名樂安。立二十八年薨,是為夷王。父子薨于京師,皆葬洛陽。

子鴻嗣。安帝崩,始就國。鴻生質帝,質帝立,梁太后下詔,以樂安國土卑溼,租委鮮薄,改鴻封勃海王。立二十六年薨,是為孝王。

無子,太后立桓帝弟蠡吾侯悝為勃海王,奉鴻嗣。延熹八年,悝謀為不道,有司請廢之。帝不忍,乃貶為廮陶王,食一縣。

悝後因中常侍王甫求復國,許謝錢五千萬。帝臨崩,遺詔復為勃海王。悝知非甫功,不肯還謝錢。甫怒,陰求其過。初,迎立靈帝,道路流言悝恨不得立,欲鈔徵書。而中常侍鄭颯、中黃門董騰並任俠通剽輕,數與悝交通。王甫司察,以為有姦,密告司隸校尉段熲。熹平元年,遂收颯送北寺獄。使尚書令廉忠誣奏颯等謀迎立悝,大逆不道。遂詔冀州刺史收悝考實,又遣大鴻臚持節與宗正、廷尉之勃海,迫責悝。悝自殺。妃妾十一人,子女七十人,伎女二十四人,皆死獄中。傅、相以下,以輔導王不忠,悉伏誅。悝立二十五年國除。眾庶莫不憐之。

平春悼王全,以建初四年封。其年薨,葬於京師。無子,國除。

清河孝王慶,母宋貴人。貴人,宋昌八世孫,扶風平陵人也。父楊,以恭孝稱於鄉閭,不應州郡之命。楊姑即明德馬后之外祖母也。馬后聞楊二女皆有才色,迎而訓之。永平末,選入太子宮,甚有寵。肅宗即位,並為貴人。建初三年,大貴人生慶,明年立為皇太子,徵楊為議郎,褒賜甚渥。貴人長於人事,供奉長樂宮,身埶饋饌,太后憐之。太后崩後,竇皇后寵盛,以貴人姊妹並幸,慶為太子,心內惡之。與母比陽主謀陷宋氏。外令兄弟求其纖過,內使御者偵伺得失。後於掖庭門邀遮得貴人書,云「病思生菟,令家求之」,因誣言欲作蠱道祝詛,以菟為厭勝之術,日夜毀譖,貴人母子遂漸見疏。

慶出居承祿觀,數月,竇后諷掖庭令誣奏前事,請加驗實。七年,帝遂廢太子慶而立皇太子肇。肇,梁貴人子也。乃下詔曰:「皇太子有失惑無常之性,爰自孩乳,至今益章,恐襲其母凶惡之風,不可以奉宗廟,為天下主。大義滅親,況降退乎!今廢慶為清河王。皇子肇保育皇后,承訓懷衽,導達善性,將成其器。蓋庶子慈母,尚有終身之恩,豈若嫡后事正義明哉!今以肇為皇太子。」遂出貴人姊妹置丙舍,使小黃門蔡倫考實之,皆承諷旨傅致其事,乃載送暴室。二貴人同時飲藥自殺。帝猶傷之,敕掖庭令葬於樊濯聚。於是免楊歸本郡。郡縣因事復捕繫之,楊友人前懷令山陽張峻、左馮翊沛國劉均等奔走解釋,得以免罪。楊失志憔悴,卒于家。慶時雖幼,而知避嫌畏禍,言不敢及宋氏,帝更憐之,敕皇后令衣服與太子齊等。太子特親愛慶,入則共室,出則同輿。及太子即位,是為和帝,待慶尤渥,諸王莫得為比,常共議私事。

後慶以長,別居丙舍。永元四年,帝移幸北宮章德殿,講於白虎觀,慶得入省宿止。帝將誅竇氏,欲得外戚傳,懼左右不敢使,乃令慶私從千乘王求,夜獨內之;又令慶傳語中常侍鄭眾求索故事。及大將軍竇憲誅,慶出居邸,賜奴婢三百人,輿馬、錢帛、帷帳、珍寶、玩好充仞其第,又賜中傅以下至左右錢帛各有差。

慶多被病,或時不安,帝朝夕問訊,進膳藥,所以垂意甚備。慶小心恭孝,自以廢黜,尤畏事慎法。每朝謁陵廟,常夜分嚴裝,衣冠待明;約敕官屬,不得與諸王車騎競驅。常以貴人葬禮有闕,每竊感恨,至四節伏臘,輒祭於私室。竇氏誅後,始使乳母於城北遙祠。及竇太后崩,慶求上冢致哀,帝許之,詔太官四時給祭具。慶垂涕曰:「生雖不獲供養,終得奉祭祀,私願足矣。」欲求作祠堂,恐有自同恭懷梁后之嫌,遂不敢言。常泣向左右,以為沒齒之恨。後上言外祖母王年老,遭憂病,下土無毉藥,願乞詣洛陽療疾。於是詔宋氏悉歸京師,除慶舅衍、俊、蓋、暹等皆為郎。

十五年,有司以日食陰盛,奏遣諸王侯就國。詔曰:「甲子之異,責由一人。諸王幼稚,早離顧復,弱冠相育,常有蓼莪、凱風之哀。選懦之恩,知非國典,且復須留。」至冬,從祠章陵,詔假諸王羽林騎各四十人。後中傅衛訢私為臧盜千餘萬,詔使案理之,并責慶不舉之狀,慶曰:「訢以師傅之尊,選自聖朝,臣愚唯知言從事聽,不甚有所糾察。」帝嘉其對,悉以訢臧財賜慶。及帝崩,慶號泣前殿,嘔血數升,因以發病。

明年,諸王就國,鄧太后特聽清河王置中尉、內史,賜什物皆取乘輿上御,以宋衍等並為清河中大夫。慶到國,下令:「寡人生於深宮,長於朝廷,仰恃明主,垂拱受成。既以薄祐,早離顧復,屬遭大憂,悲懷感傷。蒙恩大國,職惟藩輔,新去京師,憂心煢煢,夙夜屏營,未知所立。蓋聞智不獨理,必須明賢。今官屬並居爵任,失得是均,庶望上遵策戒,下免悔咎。其糾督非枉,明察典禁,無令孤獲怠慢之罪焉。」

鄧太后以殤帝襁抱,遠慮不虞,留慶長子祐與嫡母耿姬居清河邸。至秋,帝崩,立祐為嗣,是為安帝。太后使中黃門送耿姬歸國。

帝所生母左姬,字小娥,小娥姊字大娥,犍為人也。初,伯父聖坐妖言伏誅,家屬沒官,二娥數歲入掖庭,及長,並有才色。小娥善史書,喜辭賦。和帝賜諸王宮人,因入清河第。慶初聞其美,賞傅母以求之。及後幸愛極盛,姬妾莫比。姊妹皆卒,葬於京師。

慶立凡二十五年,乃歸國。其年病篤,謂宋衍等曰:「清河埤薄,欲乞骸骨於貴人冢傍下棺而已。朝廷大恩,猶當應有祠室,庶母子并食,魂靈有所依庇,死復何恨?」乃上書太后曰:「臣國土下溼,願乞骸骨,下從貴人於樊濯,雖歿且不朽矣。及今口目尚能言視,冒昧干請。命在呼吸,願蒙哀憐。」遂薨,年二十九。遣司空持節與宗正奉弔祭;又使長樂謁者僕射、中謁者二人副護喪事;賜龍旂九旒,虎賁百人,儀比東海恭王。太后使掖庭丞送左姬喪,與王合葬廣丘。

子愍王虎威嗣。永初元年,太后封宋衍為盛鄉侯,分清河為二國,封慶少子常保為廣川王,子女十一人皆為鄉公主,食邑奉。明年,常保薨,無子,國除。

虎威立三年薨,亦無子。鄧太后復立樂安王寵子延平為清河王,是為恭王。

太后崩,有司上言:「清河孝王至德淳懿,載育明聖,承天奉祚,為郊廟主。漢興,高皇帝尊父為太上皇,宣帝號父為皇考,序昭穆,置園邑。太宗之義,舊章不忘。宜上尊號曰孝德皇,皇妣左氏曰孝德后,孝德皇母宋貴人追謚曰敬隱后。」乃告祠高廟,使司徒持節與大鴻臚奉策書璽綬清河,追上尊號;又遣中常侍奉太牢祠典,護禮儀侍中劉珍等及宗室列侯皆往會事。尊陵曰甘陵,廟曰昭廟,置令、丞,設兵車周衛,比章陵。復以廣川益清河國。尊耿姬為甘陵大貴人。又封女弟侍男為涅陽長公主,別得為舞陰長公主,久長為濮陽長公主,直得為平氏長公主。餘七主並早卒,故不及進爵。追贈敬隱后女弟小貴人印綬,追封謚宋楊為當陽穆侯。楊四子皆為列侯,食邑各五千戶。宋氏為卿、校、侍中、大夫、謁者、郎吏十餘人。孝德后異母弟次及達生二人,諸子九人,皆為清河國郎中。耿貴人者,牟平侯舒之孫也。貴人兄寶,襲封牟平侯。帝以寶嫡舅,寵遇甚渥,位至大將軍,事已見耿舒傳。

立三十五年薨,子蒜嗣。沖帝崩,徵蒜詣京師,將議為嗣。會大將軍梁冀與梁太后立質帝,罷歸國。

蒜為人嚴重,動止有度,朝臣太尉李固等莫不歸心焉。初,中常侍曹騰謁蒜,蒜不為禮,宦者由此惡之。及帝崩,公卿皆正議立蒜,而曹騰說梁冀不聽,遂立桓帝。語在李固傳。蒜由此得罪。

建和元年,甘陵人劉文與南郡妖賊劉鮪交通,訛言清河王當統天子,欲共立蒜。事發覺,文等遂劫清河相謝暠,將至王宮司馬門,曰:「當立王為天子,暠為公。」暠不聽,罵之,文因刺殺暠。於是捕文、鮪誅之。有司因劾奏蒜,坐貶爵為尉氏侯,徙桂陽,自殺。立三年,國絕。

梁冀惡清河名,明年,乃改為甘陵。梁太后立安平孝王子經侯理為甘陵王,奉孝德皇祀,是為威王。

理立二十五年薨,子貞王定嗣。

定立四年薨,子獻王忠嗣。黃巾賊起,忠為國人所執,既而釋之。靈帝以親親故,詔復忠國。忠立十三年薨,嗣子為黃巾所害,建安十一年,以無後,國除。

濟北惠王壽,母申貴人,潁川人也,世吏二千石。貴人年十三,入掖庭。壽以永元二年封,分太山郡為國。和帝遵肅宗故事,兄弟皆留京師,恩寵篤密。有司請遣諸王歸藩,不忍許之,及帝崩,乃就國。永初元年,鄧太后封壽舅申轉為新亭侯。壽立三十一年薨。自永初已後,戎狄叛亂,國用不足,始封王薨,減賻錢為千萬,布萬匹;嗣王薨,五百萬,布五千匹。時唯壽最尊親,特賻錢三千萬,布三萬匹。

子節王登嗣。永寧元年,封登弟五人為鄉侯,皆別食太山邑。

登立十五年薨,子哀王多嗣。

多立三年薨,無子。永和四年,立戰鄉侯安國為濟北王,是為釐王。

安國立十年薨,子孝王次嗣。本初元年,封次弟猛為亭侯。次九歲喪父,至孝。建和元年,梁太后下詔曰:「濟北王次以幼年守藩,躬履孝道,父沒哀慟,焦毀過禮,草廬土席,衰杖在身,頭不枇沐,體生瘡腫。諒闇已來二十八月,自諸國有憂,未之聞也,朝廷甚嘉焉。《書》不云乎:『用德章厥善。』《詩》云:『孝子不匱,永錫爾類。』今增次封五千戶,廣其土宇,以慰孝子惻隱之勞。」

次立〈十〉七年薨,子鸞嗣。鸞薨,子政嗣。政薨,無子,建安十一年,國除。

河閒孝王開,以永元二年封,分樂成、勃海、涿郡為國。延平元年就國。開奉遵法度,吏人敬之。永寧元年,鄧太后封開子翼為平原王,奉懷王勝祀;子德為安平王,奉樂成王黨祀。

開立四十二年薨,子惠王政嗣。政傲佷,不奉法憲。順帝以侍御史吳郡沈景有彊能稱,故擢為河閒相。景到國謁王,王不正服,箕踞殿上。侍郎贊拜,景峙不為禮。問王所在,虎賁曰:「是非王邪?」景曰:「王不服,常人何別!今相謁王,豈謁無禮者邪!」王慚而更服,景然後拜。出住宮門外,請王傅責之曰:「前發京師,陛下見受詔,以王不恭,使檢督。諸君空受爵祿,而無訓導之義。」因奏治罪。詔書讓政而詰責傅。景因捕諸姦人上案其罪,殺戮尤惡者數十人,出冤獄百餘人。政遂為改節,悔過自脩。陽嘉元年,封政弟十三人皆為亭侯。

政立十年薨,子貞王建嗣。建立十年薨,子安王利嗣。利立二十八年薨,子陔嗣。陔立四十一年,魏受禪,以為崇德侯。

蠡吾侯翼,元初六年鄧太后徵濟北、河閒王諸子詣京師,奇翼美儀容,故以為平原懷王後焉。留在京師。歲餘,太后崩。安帝乳母王聖與中常侍江京等譖鄧騭兄弟及翼,云與中大夫趙王謀圖不軌,闚覦神器,懷大逆心。貶為都鄉侯,遣歸河閒。翼於是謝賓客,閉門自處。永建五年,父開上書,願分蠡吾縣以封翼,順帝從之。

翼卒,子志嗣,為大將軍梁冀所立,是為桓帝。梁太后詔追尊河閒孝王為孝穆皇,夫人趙氏曰孝穆后,廟曰清廟,陵曰樂成陵;蠡吾先侯曰孝崇皇,廟曰烈廟,陵曰博陵。皆置令、丞,使司徒持節奉策書、璽綬,祠以太牢。建和二年,更封帝兄都鄉侯碩為平原王,留博陵,奉翼後。尊翼夫人馬氏為孝崇博園貴人,以涿郡之良鄉、故安,河閒之蠡吾三縣為湯沐邑。碩嗜酒,多過失,帝令馬貴人領王家事。建安十一年,國除。

解瀆亭侯淑,以河閒孝王子封。淑卒,子長嗣。長卒,子宏嗣,為大將軍竇武所立,是為靈帝。建寧元年,竇太后詔追尊皇祖淑為孝元皇,夫人夏氏曰孝元后,陵曰敦陵,廟曰靖廟;皇考長為孝仁皇,夫人董氏為慎園貴人,陵曰慎陵,廟曰奐廟。皆置令、丞,使司徒持節之河閒奉策書、璽綬,祠以太牢,常以歲時遣中常侍持節之河閒奉祠。

熹平三年,使使拜河閒安王利子康為濟南王,奉孝仁皇祀。

康薨,子贇嗣,建安十二年,為黃巾賊所害。子開嗣,立十三年,魏受禪,以為崇德侯。

城陽懷王淑,以永元二年分濟陰為國。立五年薨,葬於京師。無子,國除,還并濟陰。

廣宗殤王萬歲,以永元五年封,分鉅鹿為國。其年薨,葬於京師。無子,國除,還并鉅鹿。

平原懷王勝,和帝長子也。不載母氏。少有痼疾,延平元年封。立八年薨,葬於京師。無子,鄧太后立樂安夷王寵子得為平原王,奉勝後,是為哀王。

得立六年薨,無子,永寧元年,太后又立河閒王開子都鄉侯翼為平原王嗣。安帝廢之,國除。

論曰:傳稱吳子夷昧,甚德而度,有吳國者,必其子孫。章帝長者,事從敦厚,繼祀漢室,咸其苗裔,古人之言信哉!

贊曰:章祚不已,本枝流祉。質惟伉孫,安亦慶子。河閒多福,桓、靈承祀。濟北無驕,皇恩寵饒。平原抱痼,三王薨朝。振振子孫,或秀或苗。


\end{pinyinscope}