\article{輿服上}

\begin{pinyinscope}
青蓋車綠車皁蓋車夫人安

車大駕法駕小駕

輕車大使車小使車載車

導從車車馬飾

《書》曰:「明試以功,車服以庸。」言昔者聖人興天下之大利,除天下之大害,躬親其事,身履其勤,憂之勞之,不避寒暑,使天下之民物,各得安其性命,無夭昏暴陵之災。是以天下之民,敬而愛之,若親父母;則而養之,若仰日月。夫愛之者欲其長久,不憚力役,相與起作宮室,上棟下宇,以雍覆之,欲其長久也;敬之者欲其尊嚴,不憚勞煩,相與起作輿輪旌旗章表,以尊嚴之。斯愛之至,敬之極也。苟心愛敬,雖報之至,情由未盡。或殺身以為之,盡其情也;弈世以祀之,明其功也。是以流光與天地比長。後世聖人,知恤民之憂思深大者,必饗其樂;勤仁毓物使不夭折者,必受其福。故為之制禮以節之,使夫上仁繼天統物,不伐其功,民物安逸,若道自然,莫知所謝。老子曰:「聖人不仁,以百姓為芻狗。」此之謂也。

夫禮服之興也,所以報功章德,尊仁尚賢。故禮尊貴貴,不得相踰,所以為禮也。非其人不得服其服,所以順禮也。順則上下有序,德薄者退,德盛者縟。故聖人處乎天子之位,服玉藻邃延,日月升龍,山車金根飾,黃屋左纛,所以副其德,章其功也。賢仁佐聖,封國愛民,黼黻文繡,降龍路車,所以顯其仁,光其能也。及其季末,聖人不得其位,賢者隱伏,是以天子微弱,諸侯脅矣。於此相貴以等,相讟以貨,相賂以利,天下之禮亂矣。至周夷王下堂而迎諸侯,此天子失禮,微弱之始也。自是諸侯宮縣樂食,祭以白牡,擊玉谫,朱干設鍚,冕而觯大武。大夫臺門旅樹反坫,繡黼丹朱中衣,鏤簋朱紘,此大夫之僭諸侯禮也。詩刺「彼己之子,不稱其服」,傷其敗化。易譏「負且乘,致寇至」,言小人乘君子器,盜思奪之矣。自是禮制大亂,兵革並作;上下無法,諸侯陪臣,山楶藻梲。降及戰國,奢僭益熾,削滅禮籍,蓋惡有害己之語。競修奇麗之服,飾以輿馬,文罽玉纓,象鑣金鞍,以相夸上。爭錐刀之利,殺人若刈草然,其宗祀亦旋夷滅。榮利在己,雖死不悔。及秦并天下,攬其輿服,上選以供御,其次以錫百官。漢興,文學既缺,時亦草創,承秦之制,後稍改定,參稽六經,近於雅正。孔子曰:「其或繼周者,行夏之正,乘殷之輅,服周之冕,樂則韶舞。」故撰輿服著之于篇,以觀古今損益之義云。

上古聖人,見轉蓬始知為輪。輪行可載,因物知生,復為之輿。輿輪相乘,流運罔極,任重致遠,天下獲其利。後世聖人觀於天,視斗周旋,魁方杓曲,以攜龍、角為帝車,於是迺曲其輈,乘牛駕馬,登險赴難,周覽八極。故易震乘乾,謂之大壯,言器莫能有上之者也。自是以來,世加其飾。至奚仲為夏車正,建其斿旐,尊卑上下,各有等級。周室大備,官有六職,百工與居一焉。一器而群工致巧者,車最多,是故具物以時,六材皆良。輿方法地,蓋圓象天;三十輻以象日月;蓋弓二十八以象列星;龍旂九斿,七仞齊軫,以象大火;鳥旟七斿,五仞齊較,以象鶉火;熊旗六斿,五仞齊肩,以象參、伐;龜旐四斿,四仞齊首,以象營室;弧旌枉矢,以象弧也:此諸侯以下之所建者也。

天子五路,以玉為飾,錫樊纓十有再就,建太常,十有二斿,九仞曳地,日月升龍,象天明也。夷王以下,周室衰弱,諸侯大路。秦并天下,閱三代之禮,或曰殷瑞山車,金根之色。漢承秦制,御為乘輿,所謂孔子乘殷之路者也。

乘輿、金根、安車、立車,輪皆朱班重牙,貳轂兩轄,金薄繆龍,為輿倚較,文虎伏軾,龍首銜軛,左右吉陽筩,鸞雀立衡,肤文畫輈,羽蓋華蚤,建大旂,十有二斿,畫日月升龍,駕六馬,象鑣鏤錫,金鍐方釳,插翟尾,朱兼樊纓,赤罽易茸,金就十有二,左纛以氂牛尾為之,在左騑馬軛上,大如斗,是為德車。五時車,安、立亦皆如之,各如方色,馬亦如之。白馬者,朱其髦尾為朱鬣云。所御駕六,餘皆駕四,後從為副車。

耕車,其飾皆如之。有三蓋。一曰芝車,置胶耒耜之箙,上親耕所乘也。

戎車,其飾皆如之。蕃以矛麾金鼓羽析幢翳,胶冑甲弩之箙。

獵車,其飾皆如之。重輞縵輪,繆龍繞之。一曰闟豬車,親校獵乘之。

太皇太后、皇太后法駕,皆御金根,加交路帳裳。非法駕,則乘紫罽軿車,雲肤文畫輈,黃金塗五末、蓋蚤。左右騑,駕三馬。長公主赤罽軿車。大貴人、貴人、公主、王妃、封君油畫軿車。大貴人加節畫輈。皆右騑而已。

皇太子、皇子皆安車,朱班輪,青蓋,金華蚤,黑肤文,畫轓文輈,金塗五末。皇子為王,錫以乘之,故曰王青蓋車。皇孫綠車以從。皆左右騑,駕三。公、列侯安車,朱班輪,倚鹿較,伏熊軾,皁繒蓋,黑轓,右騑。

中二千石、二千石皆皁蓋,朱兩轓。其千石、六百石,朱左轓。轓長六尺,下屈廣八寸,上業廣尺二寸,九文,十二初,後謙一寸,若月初生,示不敢自滿也。景帝中元五年,始詔六百石以上施車轓,得銅五末,軛有吉陽筩。中二千石以上右騑,三百石以上皁布蓋,千石以上皁繒覆蓋,二百石以下白布蓋,皆有四維杠衣。賈人不得乘馬車。除吏赤畫杠,其餘皆青云。

公、列侯、中二千石、二千石夫人,會朝若蠶,各乘其夫之安車,右騑,加交路帷裳,皆皁。非公會,不得乘朝車,得乘漆布輜軿車,銅五末。

乘輿大駕,公卿奉引,太僕御,大將軍參乘。屬車八十一乘,備千乘萬騎。西都行祠天郊,甘泉備之。官有其注,名曰甘泉鹵簿。東都唯大行乃大駕。大駕,太僕校駕;法駕,黃門令校駕。

乘輿法駕,八卿不在鹵簿中。河南尹、執金吾、雒陽令奉引,奉車郎御,侍中參乘。屬車四十六乘。前驅有九斿雲罕,鳳皇闟戟,皮軒鸞旗,皆大夫載。鸞旗者,編羽旄,列繫幢旁。民或謂之雞翹,非也。後有金鉦黃鉞,黃門鼓車。

古者諸侯貳車九乘。秦滅九國,兼其車服,故大駕屬車八十一乘,法駕半之。屬車皆皁蓋赤裏,木轓,戈矛弩箙,尚書、御史所載。最後一車懸豹尾,豹尾以前比省中。

行祠天郊以法駕,祠地、明堂省什三,祠宗廟尤省,謂之小駕。每出,太僕奉駕上鹵簿,中常侍、小黃門副;尚書主者,郎令史副;侍御史,蘭臺令史副。皆執注,以督整車騎,謂之護駕。春秋上陵,尤省於小駕,直事尚書一人從,其餘令以下,皆先行後罷。

輕車,古之戰車也。洞朱輪輿,不巾不蓋,建矛戟幢麾,胶輒弩服。藏在武庫。大駕、法駕出,射聲校尉、司馬史士載,以次屬車,在鹵簿中。諸車有矛戟,其飾幡斿旗幟皆五采,制度從周禮。吳孫兵法云:「有巾有蓋,謂之武剛車。」武剛車者,為先驅。又為屬車輕車,為後殿焉。

大使車,立乘,駕駟,赤帷。持節者,重導從:賊曹車、斧車、督車、功曹車皆兩;大車,伍伯璅弩十二人;辟車四人;從車四乘。無節,單導從,減半。

小使車,不立乘,有騑,赤屏泥油,重絳帷。導無斧車。

近小使車,蘭輿赤轂,白蓋赤帷。從騶騎四十人。此謂追捕考案,有所敕取者之所乘也。

諸使車皆朱班輪,四輻,赤衡軛。其送葬,白堊已下,洒車而後還。公、卿、中二千石、二千石,郊廟、明堂、祠陵,法出,皆大車,立乘,駕駟。他出,乘安車。

大行載車,其飾如金根車,加施組連璧交絡四角,金龍首銜璧,垂五采,析羽流蘇前後,雲氣畫帷裳,肤文畫曲轓,長懸車等。太僕御,駕六布施馬。布施馬者,淳白駱馬也,以黑藥灼其身為虎文。既下,馬斥賣,車藏城北祕宮,皆不得入城門。當用,太僕考工乃內飾治,禮吉凶不相干也。

公卿以下至縣三百石長導從,置門下五吏、賊曹、督盜賊功曹,皆帶劍,三車導;主簿、主記,兩車為從。縣令以上,加導斧車。公乘安車,則前後并馬立乘。長安、雒陽令及王國都縣加前後兵車,亭長,設右騑,駕兩。璅弩車前伍伯,公八人,中二千石、二千石、六百石皆四人,自四百石以下至二百石皆二人。黃綬,武官伍伯,文官辟車。鈴下、侍閤、門蘭、部署、街里走卒,皆有程品,多少隨所典領。驛馬三十里一置,卒皆赤幘絳韝云。

古者軍出,師旅皆從;秦省其卒,取其師旅之名焉。公以下至二千石,騎吏四人,千石以下至三百石,縣長二人,皆帶劍,持棨戟為前列,揵弓韣九鞬。諸侯王法駕,官屬傅相以下,皆備鹵簿,似京都官騎,張弓帶鞬,遮迾出入稱課促。列侯,家丞、庶子導從。若會耕祠,主縣假給辟車鮮明卒,備其威儀。導從事畢,皆罷所假。

諸車之文:乘輿,倚龍伏虎,肤文畫輈,龍首鸞衡,重牙班輪,升龍飛軨。皇太子、諸侯王,倚虎伏鹿,肤文畫輈轓,吉陽筩,朱班輪,鹿文飛軨,旂旗九斿降龍。公、列侯,倚鹿伏熊,黑轓,朱班輪,鹿文飛軨,九斿降龍。卿,朱兩轓,五斿降龍。二千石以下各從科品。諸轓車以上,軛皆有吉陽筩。

諸馬之文:案乘輿,金鍐方釳,插翟象鑣,龍畫總,沫升龍,赤扇汗,青兩翅,鷰尾。駙馬,左右赤珥流蘇,飛鳥節,赤膺兼。皇太子或亦如之。王、公、列侯,鏤錫现髦,朱鑣朱鹿,朱文,絳扇汗,青翅鷰尾。卿以下有騑者,緹扇汗,青翅尾,當盧现髦,上下皆通。中二千石以上及使者,乃有騑駕云。


\end{pinyinscope}