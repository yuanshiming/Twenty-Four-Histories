\article{南蠻西南夷列傳}

\begin{pinyinscope}
昔高辛氏有犬戎之寇,帝患其侵暴,而征伐不剋。乃訪募天下,有能得犬戎之將吳將軍頭者,購黃金千鎰,邑萬家,又妻以少女。時帝有畜狗,其毛五采,名曰槃瓠。下令之後,槃瓠遂銜人頭造闕下,群臣怪而診之,乃吳將軍首也。帝大喜,而計槃瓠不可妻之以女,又無封爵之道,議欲有報而未知所宜。女聞之,以為帝皇下令,不可違信,因請行。帝不得已,乃以女配槃瓠。槃瓠得女,負而走入南山,止石室中。所處險絕,人跡不至。於是女解去衣裳,為僕鑒之結,著獨力之衣。帝悲思之,遣使尋求,輒遇風雨震晦,使者不得進。經三年,生子一十二人,六男六女。槃瓠死後,因自相夫妻。織績木皮,染以草實,好五色衣服,製裁皆有尾形。其母後歸,以狀白帝,於是使迎致諸子。衣裳班蘭,語言侏離,好入山壑,不樂平曠。帝順其意,賜以名山廣澤。其後滋蔓,號曰蠻夷。外癡內黠,安土重舊。以先父有功,母帝之女,田作賈販,無關梁符傳,租稅之賦。有邑君長,皆賜印綬,冠用獺皮。名渠帥曰精夫,相呼為姎徒。今長沙武陵蠻是也。

其在唐虞,與之要質,故曰要服。夏商之時,漸為邊患。逮于周世,黨眾彌盛。宣王中興,乃命方叔南伐蠻方,詩人所謂「蠻荊來威」者也。又曰:「蠢爾蠻荊,大邦為讎。」明其黨眾繁多,是以抗敵諸夏也。

平王東遷,蠻遂侵暴上國。晉文侯輔政,乃率蔡共侯擊破之。至楚武王時,蠻與羅子共敗楚師,殺其將屈瑕。莊王初立,民飢兵弱,復為所寇。楚師既振,然後乃服,自是遂屬於楚。鄢陵之役,蠻與恭王合兵擊晉。及吳起相悼王,南并蠻越,遂有洞庭、蒼梧。秦昭王使白起伐楚,略取蠻夷,始置黔中郡。漢興,改為武陵。歲令大人輸布一匹,小口二丈,是謂賨布。雖時為寇盜,而不足為郡國患。

光武中興,武陵蠻夷特盛。建武二十三年,精夫相單程等據其險隘,大寇郡縣。遣武威將軍劉尚發南郡、長沙、武陵兵萬餘人,乘船泝沅水入武谿擊之。尚輕敵入險,山深水疾,舟船不得上。蠻氏知尚糧少入遠,又不曉道徑,遂屯聚守險。尚食盡引還,蠻緣路徼戰,尚軍大敗,悉為所沒。二十四年,相單程等下攻臨沅,遣謁者李嵩、中山太守馬成擊之,不能剋。明年春,遣伏波將軍馬援、中郎將劉匡、馬武、孫永等,將兵至臨沅,擊破之。單程等飢困乞降,會援病卒,謁者宗均聽悉受降。為置吏司,群蠻遂平。

肅宗建初元年,武陵澧中蠻陳從等反叛,入零陽蠻界。其冬,零陽蠻五里精夫為郡擊破從,從等皆降。三年冬,漊中蠻覃兒健等復反,攻燒零陽、作唐、孱陵界中。明年春,發荊州七郡及汝南、潁川施刑徒吏士五千餘人,拒守零陽,募充中五里蠻精夫不叛者四千人,擊澧中賊。五年春,覃兒健等請降,不許。郡因進兵與戰於宏下,大破之,斬兒健首,餘皆棄營走還漊中,復遣乞降,乃受之。於是罷武陵屯兵,賞賜各有差。

和帝永元四年冬,漊中、澧中蠻潭戎等反,燔燒郵亭,殺略吏民,郡兵擊破降之。安帝元初二年,澧中蠻以郡縣徭稅失平,懷怨恨,遂結充中諸種二千餘人,攻城殺長吏。州郡募五里蠻六亭兵追擊破之,皆散降。賜五里、六亭渠帥金帛各有差。明年秋,漊中、澧中蠻四千人並為盜賊。又零陵蠻羊孫、陳湯等千餘人,著赤幘,稱將軍,燒官寺,抄掠百姓。州郡募善蠻討平之。

順帝永和元年,武陵太守上書,以蠻夷率服,可比漢人,增其租賦。議者皆以為可。尚書令虞詡獨奏曰:「自古聖王不臣異俗,非德不能及,威不能加,知其獸心貪婪,難率以禮。是故羇縻而綏撫之,附則受而不逆,叛則棄而不追。先帝舊典,貢稅多少,所由來久矣。今猥增之,必有怨叛。計其所得,不償所費,必有後悔。」帝不從。其冬澧中、漊中蠻果爭貢布非舊約,遂殺鄉吏,舉種反叛。明年春,蠻二萬人圍充城,八千人寇夷道。遣武陵太守李進討破之,斬首數百級,餘皆降服。進乃簡選良吏,得其情和。在郡九年,梁太后臨朝,下詔增進秩二千石,賜錢二十萬。桓帝元嘉元年秋,武陵蠻詹山等四千餘人反叛,拘執縣令,屯結深山。至永興元年,太守應奉以恩信招誘,皆悉降散。

永壽三年十一月,長沙蠻反叛,屯益陽。至延熹三年秋,遂抄掠郡界,眾至萬餘人,殺傷長吏。又零陵蠻入長沙。冬,武陵蠻六千餘人寇江陵,荊州刺史劉度、謁者馬睦、南郡太守李肅皆奔走。肅主簿胡爽扣馬首諫曰:「蠻夷見郡無儆備,故敢乘閒而進。明府為國大臣,連城千里,舉旄鳴鼓,應聲十萬,柰何委符守之重,而為逋逃之人乎!」肅拔刃向爽曰:「掾促去!太守今急,何暇此計。」爽抱馬固諫,肅遂殺爽而走。帝聞之,徵肅棄巿,度、睦減死一等,復爽門閭,拜家一人為郎。於是以右校令度尚為荊州刺史,討長沙賊,平之。又遣車騎將軍馮緄討武陵蠻,並皆降散。軍還,賊復寇桂陽,太守廖析奔走。武陵蠻亦更攻其郡,太守陳奉率吏人擊破之,斬首三千餘級,降者二千餘人。至靈帝中平三年,武陵蠻復叛,寇郡界,州郡擊破之。

禮記稱「南方曰蠻,雕題交阯」。其俗男女同川而浴,故曰交阯。其西有噉人國,生首子輒解而食之,謂之宜弟。味旨,則以遺其君,君喜而賞其父。取妻美,則讓其兄。今烏滸人是也。

交阯之南有越裳國。周公居攝六年,制禮作樂,天下和平,越裳以三象重譯而獻白雉,曰:「道路悠遠,山川岨深,音使不通,故重譯而朝。」成王以歸周公。公曰:「德不加焉,則君子不饗其質;政不施焉,則君子不臣其人。吾何以獲此賜也!」其使請曰:「吾受命吾國之黃耇曰:『久矣,天之無烈風雷雨,意者中國有聖人乎?有則盍往朝之。』」周公乃歸之於王,稱先王之神致,以薦于宗廟。周德既衰,於是稍絕。

及楚子稱霸,朝貢百越。秦并天下,威服蠻夷,始開領外,置南海、桂林、象郡。漢興,尉佗自立為南越王,傳國五世。至武帝元鼎五年,遂滅之,分置九郡,交阯刺史領焉。其珠崖、儋耳二郡在海洲上,東西千里,南北五百里。其渠帥貴長耳,皆穿而縋之,垂肩三寸。武帝末,珠崖太守會稽孫幸調廣幅布獻之,蠻不堪役,遂攻郡殺幸。幸子豹合率善人還復破之,自領郡事,討擊餘黨,連年乃平。豹遣使封還印綬,上書言狀,制詔即以豹為珠崖太守。威政大行,獻命歲至。中國貪其珍賂,漸相侵侮,故率數歲一反。元帝初元三年,遂罷之。凡立郡六十五歲。

逮王莽輔政,元始二年,日南之南黃支國來獻犀牛。凡交阯所統,雖置郡縣,而言語各異,重譯乃通。人如禽獸,長幼無別。項髻徒跣,以布貫頭而著之。後頗徙中國罪人,使雜居其閒,乃稍知言語,漸見禮化。

光武中興,錫光為交阯,任延守九真,於是教其耕稼,制為冠履,初設媒娉,始知姻娶,建立學校,導之禮義。

建武十二年,九真徼外蠻里張游,率種人慕化內屬,封為歸漢里君。明年,南越徼外蠻夷獻白雉、白菟。至十六年,交阯女子徵側及其妹徵貳反,攻郡。徵側者,麊泠縣雒將之女也。嫁為朱觏人詩索妻,甚雄勇。交阯太守蘇定以法繩之,側忿,故反。於是九真、日南、合浦蠻里皆應之,凡略六十五城,自立為王。交阯刺史及諸太守僅得自守。光武乃詔長沙、合浦、交阯具車船,修道橋,通障谿,儲糧穀。十八年,遣伏波將軍馬援、樓船將軍段志,發長沙、桂陽、零陵、蒼梧兵萬餘人討之。明年夏四月,援破交阯,斬徵側、徵貳等,餘皆降散。進擊九真賊都陽等,破降之。徙其渠帥三百餘口於零陵。於是領表悉平。

肅宗元和元年,日南徼外蠻夷究不事人邑豪獻生犀、白雉。和帝永元十二年夏四月,日南、象林蠻夷二千餘人寇掠百姓,燔燒官寺,郡縣發兵討擊,斬其渠帥,餘眾乃降。於是置象林將兵長史,以防其患。安帝永初元年,九真徼外夜郎蠻夷舉土內屬,開境千八百四十里。元初二年,蒼梧蠻夷反叛,明年,遂招誘鬱林、合浦蠻漢數千人攻蒼梧郡。鄧太后遣侍御史任逴奉詔赦之,賊皆降散。延光元年,九真徼外蠻貢獻內屬。三年,日南徼外蠻復來內屬。順帝永建六年,日南徼外葉調王便遣使貢獻,帝賜調便金印紫綬。

永和二年,日南、象林徼外蠻夷區憐等數千人攻象林縣,燒城寺,殺長吏。交阯刺史樊演發交阯、九真二郡兵萬餘人救之。兵士憚遠役,遂反,攻其府。二郡雖擊破反者,而賊埶轉盛。會侍御史賈昌使在日南,即與州郡并力討之,不利,遂為所攻。圍歲餘而兵穀不繼,帝以為憂。明年,召公卿百官及四府掾屬,問其方略,皆議遣大將,發荊、楊、兗、豫四萬人赴之。大將軍從事中郎李固駮曰:「若荊、楊無事,發之可也。今二州盜賊槃結不散,武陵、南郡蠻夷未輯,長沙、桂陽數被徵發,如復擾動,必更生患。其不可一也。又兗、豫之人卒被徵發,遠赴萬里,無有還期,詔書迫促,必致叛亡。其不可二也。南州水土溫暑,加有瘴氣,致死亡者十必四五。其不可三也。遠涉萬里,士卒疲勞,比至領南,不復堪鬥。其不可四也。軍行三十里為程,而去日南九千餘里,三百日乃到,計人稟五升,用米六十萬斛,不計將吏驢馬之食,但負甲自致,費便若此。其不可五也。設軍到所在,死亡必眾,既不足禦敵,當復更發,此為刻割心腹以補四支。其不可六也。九真、日南相去千里,發其吏民,猶尚不堪,何況乃苦四州之卒,以赴萬里之艱哉!其不可七也。前中郎將尹就討益州叛羌,益州諺曰:『虜來尚可,尹來殺我。』後就徵還,以兵付刺史張喬。喬因其將吏,旬月之閒,破殄寇虜。此發將無益之效,州郡可任之驗也。宜更選有勇略仁惠任將帥者,以為刺史、太守,悉使共住交阯。今日南兵單無穀,守既不足,戰又不能。可一切徙其吏民北依交阯,事靜之後,又命歸本。還募蠻夷,使自相攻,轉輸金帛,以為其資。有能反閒致頭首者,許以封侯列土之賞。故并州刺史長沙祝良,性多勇決,又南陽張喬,前在益州有破虜之功,皆可任用。昔太宗就加魏尚為雲中守,哀帝即拜龔舍為太山太守。宜即拜良等,便道之官。」四府悉從固議,即拜祝良為九真太守,張喬為交阯刺史。喬至,開示慰誘,並皆降散。良到九真,單車入賊中,設方略,招以威信,降者數萬人,皆為良築起府寺。由是嶺外復平。

建康元年,日南蠻夷千餘人復攻燒縣邑,遂扇動九真,與相連結。交阯刺史九江夏方開恩招誘,賊皆降服。時梁太后臨朝,美方之功,遷為桂陽太守。桓帝永壽三年,居風令貪暴無度,縣人朱達等及蠻夷相聚,攻殺縣令,眾至四五千人,進攻九真,九真太守兒式戰死。詔賜錢六十萬,拜子二人為郎。遣九真都尉魏朗討破之,斬首二千級,渠帥猶屯據日南,眾轉彊盛。延熹三年,詔復拜夏方為交阯刺史。方威惠素著,日南宿賊聞之,二萬餘人相率詣方降。靈帝建寧三年,鬱林太守谷永以恩信招降烏滸人十餘萬內屬,皆受冠帶,開置七縣。熹平二年冬十二月,日南徼外國重譯貢獻。光和元年,交阯、合浦烏滸蠻反叛,招誘九真、日南,合數萬人,攻沒郡縣。四年,刺史朱雋擊破之。六年,日南徼外國復來貢獻。

巴郡南郡蠻,本有五姓:巴氏,樊氏,瞫氏,相氏,鄭氏。皆出於武落鍾離山。其山有赤黑二穴,巴氏之子生於赤穴,四姓之子皆生黑穴。未有君長,俱事鬼神,乃共擲劍於石穴,約能中者,奉以為君。巴氏子務相乃獨中之,眾皆歎。又令各乘土船,約能浮者,當以為君。餘姓悉沈,唯務相獨浮。因共立之,是為廩君。乃乘土船,從夷水至鹽陽。鹽水有神女,謂廩君曰:「此地廣大,魚鹽所出,願留共居。」廩君不許。鹽神暮輒來取宿,旦即化為蟲,與諸蟲群飛,掩蔽日光,天地晦冥。積十餘日,廩君思其便,因射殺之,天乃開明。廩君於是君乎夷城,四姓皆臣之。廩君死,魂魄世為白虎。巴氏以虎飲人血,遂以人祠焉。

及秦惠王并巴中,以巴氏為蠻夷君長,世尚秦女,其民爵比不更,有罪得以爵除。其君長歲出賦二千一十六錢,三歲一出義賦千八百錢。其民戶出幏布八丈二尺,雞羽三十鍭。漢興,南郡太守靳彊請一依秦時故事。

至建武二十三年,南郡潳山蠻雷遷等始反叛,寇掠百姓,遣武威將軍劉尚將萬餘人討破之,徙其種人七千餘口置江夏界中,今沔中蠻是也。和帝永元十三年,巫蠻許聖等以郡收稅不均,懷怨恨,遂屯聚反叛。明年夏,遣使者督荊州諸郡兵萬餘人討之。聖等依憑阻隘,久不破。諸軍乃分道並進,或自巴郡、魚復數路攻之,蠻乃散走,斬其渠帥,乘勝追之,大破聖等。聖等乞降,復悉徙置江夏。靈帝建寧二年,江夏蠻叛,州郡討平之。光和三年,江夏蠻復反,與廬江賊黃穰相連結,十餘萬人,攻沒四縣,寇患累年。廬江太守陸康討破之,餘悉降散。

板楯蠻夷者,秦昭襄王時有一白虎,常從群虎數遊秦、蜀、巴、漢之境,傷害千餘人。昭王乃重募國中有能殺虎者,賞邑萬家,金百鎰。時有巴郡閬中夷人,能作白竹之弩,乃登樓射殺白虎。昭王嘉之,而以其夷人,不欲加封,乃刻石盟要,復夷人頃田不租,十妻不筭,傷人者論,殺人者得以倓錢贖死。盟曰:「秦犯夷,輸黃龍一雙;夷犯秦,輸清酒一鍾。」夷人安之。

至高祖為漢王,發夷人還伐三秦。秦地既定,乃遣還巴中,復其渠帥羅、朴、督、鄂、度、夕、龔七姓,不輸租賦,餘戶乃歲入賨錢,口四十。世號為板楯蠻夷。閬中有渝水,其人多居水左右。天性勁勇,初為漢前鋒,數陷陳。俗喜歌舞,高祖觀之,曰:「此武王伐紂之歌也。」乃命樂人習之,所謂巴渝舞也。遂世世服從。

至于中興,郡守常率以征伐。桓帝之世,板楯數反,太守蜀郡趙溫以恩信降服之。靈帝光和三年,巴郡板楯復叛,寇掠三蜀及漢中諸郡。靈帝遣御史中丞蕭瑗督益州兵討之,連年不能剋。帝欲大發兵,乃問益州計吏,考以征討方略。漢中上計程包對曰:「板楯七姓,射殺白虎立功,先世復為義人。其人勇猛,善於兵戰。昔永初中,羌入漢州,郡縣破壞,得板楯救之,羌死敗殆盡,故號為神兵。羌人畏忌,傳語種輩,勿復南行。至建和二年,羌復大入,實賴板楯連摧破之。前車騎將軍馮緄南征武陵,雖受丹陽精兵之銳,亦倚板楯以成其功。近益州郡亂,太守李顒亦以板楯討而平之。忠功如此,本無惡心。長吏鄉亭更賦至重,僕役箠楚,過於奴虜,亦有嫁妻賣子,或乃至自頸割。雖陳冤州郡,而牧守不為通理。闕庭悠遠,不能自聞。含怨呼天,叩心窮谷。愁苦賦役,困罹酷刑。故邑落相聚,以致叛戾。非有謀主僭號,以圖不軌。今但選明能牧守,自然安集,不煩征伐也。」帝從其言,遣太守曹謙宣詔赦之,即皆降服。至中平五年,巴郡黃巾賊起,板楯蠻夷因此復叛,寇掠城邑,遣西園上軍別部司馬趙瑾討平之。

西南夷者,在蜀郡徼外。有夜郎國,東接交阯,西有滇國,北有邛都國,各立君長。其人皆椎結左衽,邑聚而居,能耕田。其外又有巂、昆明諸落,西極同師,東北至葉榆,地方數千里。無君長,辮髮,隨畜遷徙無常。自巂東北有莋都國,東北有冉駹國,或土著,或隨畜遷徙。自冉駹東北有白馬國,氐種是也。此三國亦有君長。

夜郎者,初有女子浣於遯水,有三節大竹流入足閒,聞其中有號聲,剖竹視之,得一男兒,歸而養之。及長,有才武,自立為夜郎侯,以竹為姓。武帝元鼎六年,平南夷,為牂柯郡,夜郎侯迎降,天子賜其王印綬。後遂殺之。夷獠咸以竹王非血氣所生,甚重之,求為立後。牂柯太守霸以聞,天子乃封其三子為侯。死,配食其父。今夜郎縣有竹王三郎神是也。

初,楚頃襄王時,遣將莊豪從沅水伐夜郎,軍至且蘭,椓船於岸而步戰。既滅夜郎,因留王滇池。以且蘭椓船牂柯處,乃改其名為牂柯。牂柯地多雨潦,俗好巫鬼禁忌,寡畜生,又無蠶桑,故其郡最貧。句町縣有桄桹木,可以為弃,百姓資之。公孫述時,大姓龍、傅、尹、董氏,與郡功曹謝暹保境為漢,乃遣使從番禺江奉貢。光武嘉之,並加褒賞。桓帝時,郡人尹珍自以生於荒裔,不知禮義,乃從汝南許慎、應奉受經書圖緯,學成,還鄉里教授,於是南域始有學焉。珍官至荊州刺史。

滇王者,莊蹻之後也。元封二年,武帝平之,以其地為益州郡,割牂柯、越巂各數縣配之。後數年,復并昆明地,皆以屬之此郡。有池,周回二百餘里,水源深廣,而末更淺狹,有似倒流,故謂之滇池。河土平敞,多出鸚鵡、孔雀,有鹽池田漁之饒,金銀畜產之富。人俗豪头。居官者皆富及累世。

及王莽政亂,益州郡夷棟蠶、若豆等起兵殺郡守,越巂姑復夷人大牟亦皆叛,殺略吏人。莽遣寧始將軍廉丹,發巴蜀吏人及轉兵穀卒徒十餘萬擊之。吏士飢疫,連年不能剋而還。以廣漢文齊為太守,造起陂池,開通溉灌,墾田二千餘頃。率厲兵馬,修障塞,降集群夷,甚得其和。及公孫述據益土,齊固守拒險,述拘其妻子,許以封侯,齊遂不降。聞光武即位,乃閒道遣使自聞。蜀平,徵為鎮遠將軍,封成義侯。於道卒,詔為起祠堂,郡人立廟祀之。

建武十八年,夷渠帥棟蠶與姑復、楪榆、梇棟、連然、滇池、建怜、昆明諸種反叛,殺長吏。益州太守繁勝與戰而敗,退保朱提。十九年,遣武威將軍劉尚等發廣漢、犍為、蜀郡人及朱提夷,合萬三千人擊之。尚軍遂度瀘水,入益州界。群夷聞大兵至,皆棄壘奔走,尚獲其羸弱、穀畜。二十年,進兵與棟蠶等連戰數月,皆破之。明年正月,追至不韋,斬棟蠶帥,凡首虜七千餘人,得生口五千七百人,馬三千疋,牛羊三萬餘頭,諸夷悉平。

肅宗元和中,蜀郡王追為太守,政化尤異,有神馬四匹出滇池河中,甘露降,白烏見,始興起學校,漸遷其俗。靈帝熹平五年,諸夷反叛,執太守雍陟。遣御史中丞朱龜討之,不能剋。朝議以為郡在邊外,蠻夷喜叛,勞師遠役,不如棄之。太尉掾巴郡李顒建策討伐,乃拜顒益州太守,與刺史龐芝發板楯蠻擊破平之,還得雍陟。顒卒後,夷人復叛,以廣漢景毅為太守,討定之。毅初到郡,米斛萬錢,漸以仁恩,少年閒,米至數十云。

哀牢夷者,其先有婦人名沙壹,居于牢山。嘗捕魚水中,觸沈木若有感,因懷妊,十月,產子男十人。後沈木化為龍,出水上。沙壹忽聞龍語曰:「若為我生子,今悉何在?」九子見龍驚走,獨小子不能去,背龍而坐,龍因舐之。其母鳥語,謂背為九,謂坐為隆,因名子曰九隆。及後長大,諸兄以九隆能為父所舐而黠,遂共推以為王。後牢山下有一夫一婦,復生十女子,九隆兄弟皆娶以為妻,後漸相滋長。種人皆刻畫其身,象龍文,衣皆著尾。九隆死,世世相繼。乃分置小王,往往邑居,散在谿谷。絕域荒外,山川阻深,生人以來,未嘗交通中國。

建武二十三年,其王賢栗遣兵乘箄船,南下江、漢,擊附塞夷鹿茤。鹿茤人弱,為所禽獲。於是震雷疾雨,南風飄起,水為逆流,侴涌二百餘里,箄船沈沒,哀牢之眾,溺死數千人。賢栗復遣其六王將萬人以攻鹿茤,鹿茤王與戰,殺其六王。哀牢耆老共埋六王,夜虎復出其尸而食之,餘眾驚怖引去。賢栗惶恐,謂其耆老曰:「我曹入邊塞,自古有之,今攻鹿茤,輒被天誅,中國其有聖帝乎?天祐助之,何其明也!」二十七年,賢栗等遂率種人戶二千七百七十,口萬七千六百五十九,詣越巂太守鄭鴻降,求內屬。光武封賢栗等為君長。自是歲來朝貢。

永平十二年,哀牢王柳貌遣子率種人內屬,其稱邑王者七十七人,戶五萬一千八百九十,口五十五萬三千七百一十一。西南去洛陽七千里,顯宗以其地置哀牢、博南二縣,割益州郡西部都尉所領六縣,合為永昌郡。始通博南山,度蘭倉水,行者苦之。歌曰:「漢德廣,開不賓。度博南,越蘭津。度蘭倉,為它人。」

哀牢人皆穿鼻儋耳,其渠帥自謂王者,耳皆下肩三寸,庶人則至肩而已。土地沃美,宜五穀、蠶桑。知染采文繡,罽毲帛疊,蘭干細布,織成文章如綾錦。有梧桐木華,績以為布,幅廣五尺,絜白不受垢汙。先以覆亡人,然後服之。其竹節相去一丈,名曰濮竹。出銅、鐵、鉛、錫、金、銀、光珠、虎魄、水精、琉璃、軻蟲、蚌珠、孔雀、翡翠、犀、象、猩猩、貊獸。雲南縣有神鹿兩頭,能食毒草。

先是,西部都尉廣漢鄭純為政清絜,化行夷貊,君長感慕,皆獻土珍,頌德美。天子嘉之,即以為永昌太守。純與哀牢夷人約,邑豪歲輸布貫頭衣二領,鹽一斛,以為常賦,夷俗安之。純自為都尉、太守,十年卒官。建初元年,哀牢王類牢與守令忿爭,遂殺守令而反叛,攻越巂唐城。太守王尋奔楪榆。哀牢三千餘人攻博南,燔燒民舍。肅宗募發越巂、益州、永昌夷漢九千人討之。明年春,邪龍縣昆明夷鹵承等應募,率種人與諸郡兵擊類牢於博南,大破斬之。傳首洛陽,賜鹵承帛萬匹,封為破虜傍邑侯。

永元六年,郡徼外敦忍乙王莫延慕義,遣使譯獻犀牛、大象。九年,徼外蠻及撣國王雍由調遣重譯奉國珍寶,和帝賜金印紫綬,小君長皆加印綬、錢帛。

永初元年,徼外僬僥種夷陸類等三千餘口舉種內附,獻象牙、水牛、封牛。永寧元年,撣國王雍由調復遣使者詣闕朝賀,獻樂及幻人,能變化吐火,自支解,易牛馬頭。又善跳丸,數乃至千。自言我海西人。海西即大秦也,撣國西南通大秦。明年元會,安帝作樂於庭,封雍由調為漢大都尉,賜印綬、金銀、綵繒各有差也。

邛都夷者,武帝所開,以為邛都縣。無幾而地陷為汙澤,因名為邛池,南人以為邛河。後復反叛。元鼎六年,漢兵自越巂水伐之,以為越巂郡。其土地平原,有稻田。青蛉縣禺同山有碧雞金馬,光景時時出見。俗多游蕩,而喜謳歌,略與牂柯相類。豪帥放縱,難得制御。

王莽時,郡守枚根調邛人長貴,以為軍候。更始二年,長貴率種人攻殺枚根,自立為邛穀王,領太守事。又降於公孫述。述敗,光武封長貴為邛穀王。建武十四年,長貴遣使上三年計,天子即授越巂太守印綬。十九年,武威將軍劉尚擊益州夷,路由越巂。長貴聞之,疑尚既定南邊,威法必行,己不得自放縱,即聚兵起營臺,招呼諸君長,多釀毒酒,欲先以勞軍,因襲擊尚。尚知其謀,即分兵先據邛都,遂掩長貴誅之,徙其家屬於成都。

永平元年,姑復夷復叛,益州刺史發兵討破之,斬其渠帥,傳首京師。後太守巴郡張翕,政化清平,得夷人和。在郡十七年,卒,夷人愛慕,如喪父母。蘇祈叟二百餘人,齎牛羊送喪,至翕本縣安漢,起墳祭祀。詔書嘉美,為立祠堂。

安帝元初三年,郡徼外夷大羊等八種,戶三萬一千,口十六萬七千六百二十,慕義內屬。時郡縣賦斂煩數,五年,卷夷大牛種封離等反畔,殺遂久令。明年,永昌、益州及蜀郡夷皆叛應之,眾遂十餘萬,破壞二十餘縣,殺長吏,燔燒邑郭,剽略百姓,骸骨委積,千里無人。詔益州刺史張喬選堪能從事討之。喬乃遣從事楊竦將兵至楪榆擊之,賊盛未敢進,先以詔書告示三郡,密徵求武士,重其購賞。乃進軍與封離等戰,大破之,斬首三萬餘級,獲生口千五百人,資財四千餘萬,悉以賞軍士。封離等惶怖,斬其同謀渠帥,詣竦乞降,竦厚加慰納。其餘三十六種皆來降附。竦因奏長吏姦猾侵犯蠻夷者九十人,皆減死。州中論功未及上,會竦病創卒,張喬深痛惜之,乃刻石勒銘,圖畫其像。天子以張翕有遺愛,乃拜其子湍為太守。夷人懽喜,奉迎道路。曰:「郎君儀貌類我府君。」後湍頗失其心,有欲叛者,諸夷耆老相曉語曰:「當為先府君故。」遂以得安。後順桓閒,廣漢馮顥為太守,政化尤多異跡云。

莋都夷者,武帝所開,以為莋都縣。其人皆被髮左衽,言語多好譬類,居處略與汶山夷同。土出長年神藥,仙人山圖所居焉。元鼎六年,以為沈黎郡。至天漢四年,并蜀為西部,置兩都尉,一居旄牛,主徼外夷。一居青衣,主漢人。

永平中,益州刺史梁國朱輔,好立功名,慷慨有大略。在州數歲,宣示漢德,威懷遠夷。自汶山以西,前世所不至,正朔所未加。白狼、槃木、唐菆等百餘國,戶百三十餘萬,口六百萬以上,舉種奉貢,稱為臣僕,輔上疏曰:「臣聞《詩》云:『彼徂者岐,有夷之化。』傳曰:『岐道雖僻,而人不遠。』詩人誦詠,以為符驗。今白狼王唐菆等慕化歸義,作詩三章。路經邛來大山零高阪,峭危峻險,百倍岐道。繈負老幼,若歸慈母。遠夷之語,辭意難正。草木異種,鳥獸殊類。有犍為郡掾田恭與之習狎,頗曉其言,臣輒令訊其風俗,譯其辭語。今遣從事史李陵與恭護送詣闕,并上其樂詩。昔在聖帝,舞四夷之樂;今之所上,庶備其一。」帝嘉之,事下史官,錄其歌焉。

遠夷樂德歌詩曰:

大漢是治,與天合意。吏譯平端,

不從我來。

聞風向化,

所見奇異。多賜贈布,甘美酒食。昌樂肉飛,

屈申悉備

。蠻夷貧薄,

無所報嗣。

願主長壽,子孫昌熾。遠夷慕德歌詩曰:

蠻夷所處,日入之部。慕義向化,

歸日出主。

聖德深恩,

與人富厚。冬多霜雪,夏多和雨。

寒溫時適,

部人多有。涉危歷險,不遠萬里。

去俗歸德,

心歸慈母。遠夷懷德歌曰:

荒服之外,土地墝埆。食肉衣皮,

不見鹽穀。

吏譯傳風,

大漢安樂。攜負歸仁,觸冒險陜。

高山岐峻,

緣崖磻石。木薄發家,百宿到洛。

父子同賜,

懷抱匹帛。傳告種人,長願臣僕。

肅宗初,輔坐事免。是時郡尉府舍皆有雕飾,畫山神海靈奇禽異獸,以眩燿之,夷人益畏憚焉。和帝永元十二年,旄牛徼外白狼、樓薄蠻夷王唐繒等,遂率種人十七萬口,歸義內屬。詔賜金印紫綬,小豪錢帛各有差。

安帝永初元年,蜀郡三襄種夷與徼外汙衍種并兵三千餘人反叛,攻蠶陵城,殺長吏。二年,青衣道夷邑長令田,與徼外三種夷三十一萬口,齎黃金、旄牛毦,舉土內屬。安帝增令田爵號為奉通邑君。延光二年春,旄牛夷叛,攻零關,殺長吏,益州刺史張喬與西部都尉擊破之。於是分置蜀郡屬國都尉,領四縣如太守。桓帝永壽二年,蜀郡夷叛,殺略吏民。延熹二年,蜀郡三襄夷寇蠶陵,殺長吏。四年,犍為屬國夷寇郡界,益州刺史山昱擊破之,斬首千四百級,餘皆解散。靈帝時,以屬郡蜀國為漢嘉郡。

冉駹夷者,武帝所開。元鼎六年,以為汶山郡。至地節三年,夷人以立郡賦重,宣帝乃省并蜀郡為北部都尉。其山有六夷七羌九氐,各有部落。其王侯頗知文書,而法嚴重。貴婦人,黨母族。死則燒其尸。土氣多寒,在盛夏冰猶不釋,故夷人冬則避寒,入蜀為傭,夏則違暑,反其眾邑。皆依山居止,累石為室,高者至十餘丈,為邛籠。又土地剛鹵,不生穀粟麻菽,唯以麥為資,而宜畜牧。有旄牛,無角,一名童牛,肉重千斤,毛可為毦。出名馬。有靈羊,可療毒。又有食藥鹿,鹿麑有胎者,其腸中糞亦療毒疾。又有五角羊、麝香、輕毛毼雞、牲牲。其人能作旄氈、班罽、青頓、毞毲、羊羧之屬。特多雜藥。地有鹹土,煮以為鹽,麡羊牛馬食之皆肥。

其西又有三河、槃于虜,北有黃石、北地、盧水胡,其表乃為徼外。靈帝時,復分蜀郡北部為汶山郡云。

白馬氐者,武帝元鼎六年時,分廣漢西部,合以為武都,土地險阻,有麻田,出名馬、牛、羊、漆、蜜,氐人勇戇抵冒,貪貨死利,居於河池,一名仇池,方百傾,四面斗絕。數為邊寇,郡縣討之,則依固自守,元封三年,氐人反叛,遣兵破之,分徙酒泉郡,昭帝元鳳元年,氐人復叛,遣執金吾馬適建、龍哣侯韓增、大鴻臚田廣明,將三輔、太常徒討破之。

及王莽篡亂,氐人亦叛。建武初,氐一悉附隴蜀,及隗囂滅,其酋豪乃背公孫述降漢,隴西太守馬援上復其王侯君長,賜以印綬。後囂放人隗茂反,殺武都太守。氐人大豪齊鍾留為種類所敬信,威服諸豪,與郡丞孔奮擊茂,破斬之。後亦時為寇盜,郡縣討破之。

論曰:漢氏征伐戎狄,有事邊遠,蓋亦與王業而終始矣,至於傾沒疆垂,喪師敗將者,不出時歲,卒能開四夷之境,款殊俗之附。若乃文約之所沾漸,風聲之所周流,幾將日所出入處也。著自山經、水志者,亦略及焉。雖服叛難常,威澤時曠,及其化行,則緩耳雕腳之倫,獸居鳥語之類,莫不舉種盡落,回面而請吏,陵海越障,累譯以內屬焉。故其錄名中郎、校尉之署,編數都護、部守之曹,動以數百萬計。若乃藏山隱海之靈物,沈沙棲陸之瑋寶,莫不呈表怪麗,雕被宮幄焉。又其賨幏火毳馴禽封獸之賦,軨積於內府;夷歌巴舞殊音異節之技,列倡於外明。豈柔服之道,必足於斯?然亦云致遠者矣。蠻夷雖附阻巖谷,而類有土居,連涉荊、交之區,布護巴、庸之外,不可量極。然其凶勇狡筭,薄於羌狄,故陵暴之害,不能深也。西南之徼,尤為劣焉。故關守永昌,肇自遠離,啟土立人,至今成都焉。

贊曰:百蠻蠢居,仞彼方徼。鏤體卉衣,憑深阻峭。亦有別夷,屯彼蜀表。參差聚落,紆餘岐道。往化既孚,改襟輸寶。俾建永昌,同編億兆。


\end{pinyinscope}