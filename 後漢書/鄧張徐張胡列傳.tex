\article{鄧張徐張胡列傳}

\begin{pinyinscope}
鄧彪字智伯,南陽新野人,太傅禹之宗也。父邯,中興初以功封鄳侯,仕至勃海太守。彪少勵志,修孝行。父卒,讓國於異母弟荊鳳,顯宗高其節,下詔許焉。

後仕州郡,辟公府,五遷桂陽太守。永平十七年,徵入為太僕。數年,喪後母,辭疾乞身,詔以光祿大夫行服。服竟,拜奉車都尉,遷大司農。數月,代鮑昱為太尉。彪在位清白,為百僚式。視事四年,以疾乞骸骨。元和元年,賜策罷,贈錢三十萬,在所以二千石奉終其身。又詔太常四時致宗廟之胙,河南尹遣丞存問,常以八月旦奉羊、酒。

和帝即位,以彪為太傅,錄尚書事,賜爵關中侯。永元初,竇氏專權驕縱,朝廷多有諫爭,而彪在位修身而已,不能有所匡正。又嘗奏免御史中丞周紆,紆前失竇氏旨,故頗以此致譏,然當時宗其禮讓。及竇氏誅,以老病上還樞機職,詔賜養牛酒而許焉。五年春,薨于位,天子親臨弔臨。

張禹字伯達,趙國襄國人也。

祖父況族姊為皇祖考夫人,數往來南頓,見光武。光武為大司馬,過邯鄲,況為郡吏,謁見光武。光武大喜,曰:「乃今得我大舅乎!」因與俱北,到高邑,以為元氏令。遷涿郡太守。後為常山關長。會赤眉攻關城,況戰歿。父歆,初以報仇逃亡,後仕為淮陽相,終於汲令。

禹性篤厚節儉。父卒,汲吏人賻送前後數百萬,悉無所受。又以田宅推與伯父,身自寄止。

永平八年,舉孝廉,稍遷;建初中,拜楊州刺史。當過江行部,中土民皆以江有子胥之神,難於濟涉。禹將度,吏固請不聽。禹厲言曰:「子胥如有靈,知吾志在理察枉訟,豈危我哉?」遂鼓楫而過。歷行郡邑,深幽之處莫不畢到,親錄囚徒,多所明舉。吏民希見使者,民懷喜悅,怨德美惡,莫不自歸焉。

元和二年,轉兗州刺史,亦有清平稱。三年,遷下邳相。徐縣北界有蒲陽坡,傍多良田,而堙廢莫修。禹為開水門,通引灌溉,遂成孰田數百頃。勸率吏民,假與種糧,親自勉勞,遂大收穀實。鄰郡貧者歸之千餘戶,室廬相屬,其下成巿。後歲至墾千餘頃,民用溫給。功曹史戴閏,故太尉掾也,權動郡內。有小譴,禹令自致徐獄,然後正其法。自長史以下,莫不震肅。

永元六年,入為大司農,拜太尉,和帝甚禮之。十五年,南巡祠園廟,禹以太尉兼衛尉留守。聞車駕當進幸江陵,以為不宜冒險遠,驛馬上諫。詔報曰:「祠謁既訖,當南禮大江,會得君奏,臨漢回輿而旋。」及行還,禹特蒙賞賜。

延平元年,遷為太傅,錄尚書事。鄧太后以殤帝初育,欲令重臣居禁內,乃詔禹舍宮中,給帷帳床褥,太官朝夕進食,五日一歸府。每朝見,特贊,與三公絕席。禹上言:「方諒闇密靜之時,不宜依常有事於苑囿。其廣成、上林空地,宜且以假貧民。」太后從之。及安帝即位,數上疾乞身。詔遣小黃門問疾,賜牛一頭,酒十斛,勸令就第。其錢布、刀劍、衣物,前後累至。

永初元年,以定策功封安鄉侯,食邑千二百戶,與太尉徐防、司空尹勤同日俱封。其秋,以寇賊水雨策免防、勤,而禹不自安,上書乞骸骨,更拜太尉。四年,新野君病,皇太后車駕幸其第。禹與司徒夏勤、司空張敏俱上表言:「新野君不安,車駕連日宿止,臣等誠竊惶懼。臣聞王者動設先置,止則交戟,清道而後行,清室而後御,離宮不宿,所以重宿衛也。陛下體烝烝之至孝,親省方藥,恩情發中,久處單外,百官露止,議者所不安。宜且還宮,上為宗廟社稷,下為萬國子民。」比三上,固爭,乃還宮。後連歲災荒,府臧空虛,禹上疏求入三歲租稅,以助郡國稟假。詔許之。五年,以陰陽不和策免。七年,卒于家。使者弔祭。除小子曜為郎中。長子盛嗣。

徐防字謁卿,沛國銍人也。祖父宣,為講學大夫,以易教授王莽。父憲,亦傳宣業。

防少習父祖學,永平中,舉孝廉,除為郎。防體貌矜嚴,占對可觀,顯宗異之,特補尚書郎。職典樞機,周密畏慎,奉事二帝,未嘗有過。和帝時,稍遷司隸校尉,出為魏郡太守。永元十年,遷少府、大司農。防勤曉政事,所在有跡。十四年,拜司空。

防以五經久遠,聖意難明,宜為章句,以悟後學。上疏曰:「臣聞詩書禮樂,定自孔子;發明章句,始於子夏。其後諸家分析,各有異說。漢承亂秦,經典廢絕,本文略存,或無章句。收拾缺遺,建立明經,博徵儒術,開置太學。孔聖既遠,微旨將絕,故立博士十有四家,設甲乙之科,以勉勸學者,所以示人好惡,改敝就善者也。伏見太學試博士弟子,皆以意說,不修家法,私相容隱,開生姦路。每有策試,輒興諍訟,論議紛錯,互相是非。孔子稱『述而不作』,又曰『吾猶及史之闕文』,疾史有所不知而不肯闕也。今不依章句,妄生穿鑿,以遵師為非義,意說為得理,輕侮道術,寖以成俗,誠非詔書實選本意。改薄從忠,三世常道,專精務本,儒學所先。臣以為博士及甲乙策試,宜從其家章句,開五十難以試之。解釋多者為上第,引文明者為高說;若不依先師,義有相伐,皆正以為非。五經各取上第六人,論語不宜射策。雖所失或久,差可矯革。」詔書下公卿,皆從防言。

十六年,拜為司徒。延平元年,遷太尉,與太傅張禹參錄尚書事,數受賞賜,甚見優寵。

安帝即位,以定策封龍鄉侯。食邑千一百戶。其年以災異寇賊策免,就國。凡三公以災異策免,始自防也。

防卒,子衡當嗣,讓封於其弟崇。數歲,不得已,乃出就爵云。

張敏字伯達,河閒鄚人也。建初二年,舉孝廉,四遷,五年,為尚書。

建初中,有人侮辱人父者,而其子殺之,肅宗貰其死刑而降宥之,自後因以為比。是時遂定其議,以為輕侮法。敏駮議曰:「夫輕侮之法,先帝一切之恩,不有成科班之律令也。夫死生之決,宜從上下,猶天之四時,有生有殺。若開相容恕,著為定法者,則是故設姦萌,生長罪隙。孔子曰:『民可使由之,不可使知之。』春秋之義,子不報讎,非子也。而法令不為之減者,以相殺之路不可開故也。今託義者得減,妄殺者有差,使執憲之吏得設巧詐,非所以導『在醜不爭』之義。又輕侮之比,寖以繁滋,至有四五百科,轉相顧望,彌復增甚,難以垂之萬載。臣聞師言:『救文莫如質。』故高帝去煩苛之法,為三章之約。建初詔書,有改於古者,可下三公、廷尉蠲除其敝。」議寢不省。敏復上疏曰:「臣敏蒙恩,特見拔擢,愚心所不曉,迷意所不解,誠不敢苟隨眾議。臣伏見孔子垂經典,皋陶造法律,原其本意,皆欲禁民為非也。未曉輕侮之法將以何禁?必不能使不相輕侮,而更開相殺之路,執憲之吏復容其姦枉。議者或曰:『平法當先論生。』臣愚以為天地之性,唯人為貴,殺人者死,三代通制。今欲趣生,反開殺路,一人不死,天下受敝。記曰:『利一害百,人去城郭。』夫春生秋殺,天道之常。春一物枯即為災,秋一物華即為異。王者承天地,順四時,法聖人,從經律。願陛下留意下民,考尋利害,廣令平議,天下幸甚。」和帝從之。

九年,拜司隸校尉。視事二歲,遷汝南太守。清約不煩,用刑平正,有理能名。坐事免。延平元年,拜議郎,再遷潁川太守。徵拜司空,在位奉法而已。視事三歲,以病乞身,不聽。六年春,行大射禮,陪位頓仆,乃策罷之。因病篤,卒于家。

胡廣字伯始,南郡華容人也。六世祖剛,清高有志節。平帝時,大司徒馬宮辟之。值王莽居攝,剛解其衣冠,縣府門而去,遂亡命交阯,隱於屠肆之閒。後莽敗,乃歸鄉里。父貢,交阯都尉。

廣少孤貧,親執家苦。長大,隨輩入郡為散吏。太守法雄之子真,從家來省其父。真頗知人。會歲終應舉,雄敕真助求其才。雄因大會諸吏,真自於牖閒密占察之,乃指廣以白雄,遂察孝廉。既到京師,試以章奏,安帝以廣為天下第一。旬月拜尚書郎,五遷尚書僕射。

順帝欲立皇后,而貴人有寵者四人,莫知所建,議欲探籌,以神定選。廣與尚書郭虔、史敞上疏諫曰:「竊見詔書以立后事大,謙不自專,欲假之籌策,決疑靈神。篇籍所記,祖宗典故,未嘗有也。恃神任筮,既不必當賢;就值其人,猶非德選。夫岐嶷形於自然,俔天必有異表。宜參良家,簡求有德,德同以年,年鈞以貌,稽之典經,斷之聖慮。政令猶汗,往而不反。詔文一下,形之四方。臣職在拾遺,憂深責重,是以焦心,冒昧陳聞。」帝從之,以梁貴人良家子,定立為皇后。

時尚書令左雄議改察舉之制,限年四十以上,儒者試經學,文吏試章奏。廣復與敞、虔上書駮之,曰:「臣聞君以兼覽博照為德,臣以獻可替否為忠。書載稽疑,謀及卿士;詩美先人,詢于芻蕘。國有大政,必議之於前訓,諮之於故老,是以慮無失策,舉無過事。竊見尚書令左雄議郡舉孝廉,皆限年四十以上,諸生試章句,文吏試牋奏。明詔既許,復令臣等得與相參。竊惟王命之重,載在篇典,當令縣於日月,固於金石,遺則百王,施之萬世。《詩》云:『天難諶斯,不易惟王。』可不慎與!蓋選舉因才,無拘定制。六奇之策,不出經學;鄭、阿之政,非必章奏。甘、奇顯用,年乖彊仕;終、賈揚聲,亦在弱冠。漢承周、秦,兼覽殷、夏,祖德師經,參雜霸軌,聖主賢臣,世以致理,貢舉之制,莫或回革。今以一臣之言,凛戾舊章,便利未明,眾心不猒。矯枉變常,政之所重,而不訪台司,不謀卿士。若事下之後,議者剝異,異之則朝失其便,同之則王言已行。臣愚以為可宣下百官,參其同異,然後覽擇勝否,詳採厥衷。敢以瞽言,冒干天禁,惟陛下納焉。」帝不從。

時陳留郡缺職,尚書史敞等薦廣。曰:「臣聞德以旌賢,爵以建事,『明試以功』,典謨所美,『五服五章』,天秩所作,是以臣竭其忠,君豐其寵,舉不失德,下忘其死。竊見尚書僕射胡廣,體真履規,謙虛溫雅,博物洽聞,探賾窮理,六經典奧,舊章憲式,無所不覽。柔而不犯,文而有禮,忠貞之性,憂公如家。不矜其能,不伐其勞,翼翼周慎,行靡玷漏。密勿夙夜,十有餘年,心不外顧,志不苟進。臣等竊以為廣在尚書,劬勞日久,後母年老,既蒙簡照,宜試職千里,匡寧方國。陳留近郡,今太守任缺。廣才略深茂,堪能撥煩,願以參選,紀綱頹俗,使束脩守善,有所勸仰。」

廣典機事十年,出為濟陰太守,以舉吏不實免。復為汝南太守,入拜大司農。漢安元年,遷司徒。質帝崩,代李固為太尉,錄尚書事。以定策立桓帝,封育陽安樂鄉侯。以病遜位。又拜司空,告老致仕。尋以特進徵拜太常,遷太尉,以日食免。復為太常,拜太尉。

延熹二年,大將軍梁冀誅,廣與司徒韓縯、司空孫朗坐不衛宮,皆減死一等,奪爵土,免為庶人。後拜太中大夫、太常。九年,復拜司徒。

靈帝立,與太傅陳蕃參錄尚書事,復封故國。以病自乞。會蕃被誅,代為太傅,總錄如故。

時年已八十,而心力克壯。繼母在堂,朝夕瞻省,傍無几杖,言不稱老。及母卒,居喪盡哀,率禮無愆。性溫柔謹素,常遜言恭色。達練事體,明解朝章。雖無謇直之風,屢有補闕之益。故京師諺曰:「萬事不理問伯始,天下中庸有胡公。」及共李固定策,大議不全,又與中常侍丁肅婚姻,以此譏毀於時。

自在公台三十餘年,歷事六帝,禮任甚優,每遜位辭病,及免退田里,未嘗滿歲,輒復升進。凡一履司空,再作司徒,三登太尉,又為太傅。其所辟命,皆天下名士。與故吏陳蕃、李咸並為三司。蕃等每朝會,輒稱疾避廣,時人榮之。年八十二,熹平元年薨。使五官中郎將持節奉策贈太傅、安樂鄉侯印綬,給東園梓器,謁者護喪事,賜冢塋于原陵,謚文恭侯,拜家一人為郎中。故吏自公、卿、大夫、博士、議郎以下數百人,皆縗絰殯位,自終及葬。漢興以來,人臣之盛,未嘗有也。

初,楊雄依虞箴作十二州二十五官箴,其九箴亡闕,後涿郡崔駰及子瑗又臨邑侯劉騊駼增補十六篇,廣復繼作四篇,文甚典美。乃悉撰次首目,為之解釋,名曰百官箴,凡四十八篇。其餘所著詩、賦、銘、頌、箴、弔及諸解詁,凡二十二篇。

熹平六年,靈帝思感舊德,乃圖畫廣及太尉黃瓊於省內,詔議郎蔡邕為其頌云。

論曰:爵任之於人重矣,全喪之於生大矣。懷祿以圖存者,仕子之恆情;審能而就列者,出身之常體。夫紆於物則非己,直於志則犯俗,辭其艱則乖義,徇其節則失身。統之,方軌易因,險塗難御。故昔人明慎於所受之分,遲遲於岐路之閒也。如令志行無牽於物,臨生不先其存,後世何貶焉?古人以宴安為戒,豈數公之謂平?

贊曰:鄧、張作傅,無咎無譽。敏正疑律,防議章句。胡公庸庸,飾情恭貌。朝章雖理,據正或橈。


\end{pinyinscope}