\article{張曹鄭列傳}

\begin{pinyinscope}
張純字伯仁,京兆杜陵人也。高祖父安世,宣帝時為大司馬衛將軍,封富平侯。父放,為成帝侍中。純少襲爵土,哀平閒為侍中,王莽時至列卿。遭值篡偽,多亡爵土,純以敦謹守約,保全前封。

建武初,先來詣闕,故得復國。五年,拜太中大夫,使將潁川突騎安集荊、徐、楊部,督委輸,監諸將營。後又將兵屯田南陽,遷五官中郎將。有司奏,列侯非宗室不宜復國。光武曰:「張純宿衛十有餘年,其勿廢,更封武始侯,食富平之半。」

純在朝歷世,明習故事。建武初,舊章多闕,每有疑議,輒以訪純,自郊廟婚冠喪紀禮儀,多所正定。帝甚重之,以純兼虎賁中郎將,數被引見,一日或至數四。純以宗廟未定,昭穆失序,十九年,乃與太僕朱浮共奏言:「陛下興於匹庶,蕩滌天下,誅鉏暴亂,興繼祖宗。竊以經義所紀,人事眾心,雖實同創革,而名為中興,宜奉先帝,恭承祭祀者也。元帝以來,宗廟奉祠高皇帝為受命祖,孝文皇帝為太宗,孝武皇帝為世宗,皆如舊制。又立親廟四世,推南頓君以上盡於舂陵節侯。禮,為人後者則為之子,既事大宗,則降其私親。今禘祫高廟,陳序昭穆,而舂陵四世,君臣並列,以卑廁尊,不合禮意。設不遭王莽,而國嗣無寄,推求宗室,以陛下繼統者,安得復顧私親,違禮制乎?昔高帝以自受命,不由太上,宣帝以孫後祖,不敢私親,故為父立廟,獨群臣侍祠。臣愚謂宜除今親廟,以則二帝舊典,願下有司博採其議。」詔下公卿,大司徒戴涉、大司空竇融議:「宜以宣、元、成、哀、平五帝四世代今親廟,宣、元皇帝尊為祖、父,可親奉祠,成帝以下,有司行事,別為南頓君立皇考廟。其祭上至舂陵節侯,群臣奉祠,以明尊尊之敬,親親之恩。」帝從之。是時宗廟未備,自元帝以上,祭於洛陽高廟,成帝以下,祠於長安高廟,其南頓四世,隨所在而祭焉。

明年,純代朱浮為太僕。二十三年,代杜林為大司空。在位慕曹參之跡,務於無為,選辟掾史,皆知名大儒。明年,上穿陽渠,引洛水為漕,百姓得其利。

二十六年,詔純曰:「禘、祫之祭,不行已久矣。『三年不為禮,禮必壞;三年不為樂,樂必崩』。宜據經典,詳為其制。」純奏曰:「禮,三年一祫,五年一禘。春秋傳曰:『大祫者何?合祭也。』毀廟及未毀廟之主皆登,合食乎太祖,五年而再殷。漢舊制三年一祫,毀廟主合食高廟,存廟主未嘗合祭。元始五年,諸王公列侯廟會,始為禘祭。又前十八年親幸長安,亦行此禮。禮說三年一閏,天氣小備;五年再閏,天氣大備。故三年一祫,五年一禘。禘之為言諦,諦定昭穆尊卑之義也。禘祭以夏四月,夏者陽氣在上,陰氣在下,故正尊卑之義也。祫祭以冬十月,冬者五穀成孰,物備禮成,故合聚飲食也。斯典之廢,於茲八年,謂可如禮施行,以時定議。」帝從之,自是禘、祫遂定。

時南單于及烏桓來降,邊境無事,百姓新去兵革,歲仍有年,家給人足。純以聖王之建辟雍,所以崇尊禮義,既富而教者也。乃案七經讖、明堂圖、河閒古辟雍記、孝武太山明堂制度,及平帝時議,欲具奏之。未及上,會博士桓榮上言宜立辟雍、明堂,章下三公、太常,而純議同榮,帝乃許之。

三十年,純奏上宜封禪,曰:「自古受命而帝,治世之隆,必有封禪,以告成功焉。樂動聲儀曰:『以雅治人,風成於頌。』有周之盛,成康之閒,郊配封禪,皆可見也。《書》曰『歲二月,東巡狩,至于岱宗,』,則封禪之義也。臣伏見陛下受中興之命,平海內之亂,修復祖宗,撫存萬姓,天下曠然,咸蒙更生,恩德雲行,惠澤雨施,黎元安寧,夷狄慕義。《詩》云:『受天之祜,四方來賀。』今攝提之歲,倉龍甲寅,德在東宮。宜及嘉時,遵唐帝之典,繼孝武之業,以二月東巡狩,封于岱宗,明中興,勒功勳,復祖統,報天神,禪梁父,祀地祇,傳祚子孫,萬世之基也。」中元元年,帝乃東巡岱宗,以純視御史大夫從,并上元封舊儀及刻石文。三月,薨,謚曰節侯。

子奮嗣。

奮字稚通。父純,臨終敕家丞曰:「司空無功於時,猥蒙爵土,身死之物,勿議傳國。」奮兄根,少被病,光武詔奮嗣爵,奮稱純遺敕,固不肯受。帝以奮違紹,敕收下獄,奮惶怖,乃襲封。永平四年,隨例歸國。

奮少好學,節儉行義,常分損租奉,贍卹宗親,雖至傾匱,而施與不怠。十〈七〉年,儋耳降附,奮來朝上壽,引見宣平殿,應對合旨,顯宗異其才,以為侍祠侯。建初元年,拜左中郎將,轉五官中郎將,遷長水校尉。七年,為將作大匠,章和元年,免。永元元年,復拜城門校尉。四年,遷長樂衛尉。明年,代桓郁為太常。六年,代劉方為司空。

時歲災旱,祈雨不應,乃上表曰:「比年不登,人用飢匱,今復久旱,秋稼未立,陽氣垂盡,歲月迫促。夫國以民為本,民以穀為命,政之急務,憂之重者也。臣蒙恩尤深,受職過任,夙夜憂懼,章奏不能敘心,願對中常侍疏奏。」即時引見,復口陳時政之宜。明日,和帝召太尉、司徒幸洛陽獄,錄囚徒,收洛陽令陳歆,即大雨三日。

奮在位清白,無它異績。九年,以病罷。在家上疏曰:「聖人所美,政道至要,本在禮樂。五經同歸,而禮樂之用尤急。孔子曰:『安上治民,莫善於禮;移風易俗,莫善於樂。』又曰:『揖讓而化天下者,禮樂之謂也。』先王之道,禮樂可謂盛矣。孔子謂子夏曰:『禮以修外,樂以制內,丘已矣夫!』又曰:『禮樂不興,則刑罰不中;刑罰不中,則民無所厝其手足。』臣以為漢當制作禮樂,是以先帝聖德,數下詔書,愍傷崩缺,而眾儒不達,議多駮異。臣累世台輔,而大典未定,私竊惟憂,不忘寢食。臣犬馬齒盡,誠冀先死見禮樂之定。」十三年,更召拜太常。復上疏曰:「漢當改作禮樂,圖書著明。王者化定制禮,功成作樂。謹條禮樂異議三事,願下有司,以時考定。昔者孝武皇帝、光武皇帝封禪告成,而禮樂不定,事不相副。先帝已詔曹褒,今陛下但奉而成之,猶周公斟酌文武之道,非自為制,誠無所疑。久執謙謙,令大漢之業不以時成,非所以章顯祖宗功德,建太平之基,為後世法。」帝雖善之,猶未施行。其冬,復以病罷。明年,卒於家。

子甫嗣,官至津城門候。甫卒,子吉嗣。永初三年,吉卒,無子,國除。自昭帝封安世,至吉,傳國八世,經歷篡亂,二百年閒未嘗譴黜,封者莫與為比。

曹褒字叔通,魯國薛人也。父充,持慶氏禮,建武中為博士,從巡狩岱宗,定封禪禮,還,受詔議立七郊、三雍、大射、養老禮儀。顯宗即位,充上言:「漢再受命,仍有封禪之事,而禮樂崩闕,不可為後嗣法。五帝不相沿樂,三王不相襲禮,大漢自制禮,以示百世。」帝問:「制禮樂云何?」充對曰:「河圖括地象曰:『有漢世禮樂文雅出。』尚書琁機鈐曰:『有帝漢出,德洽作樂,名予。』」帝善之,下詔曰:「今且改太樂官曰太予樂,歌詩曲操,以俟君子。」拜充侍中。作章句辯難,於是遂有慶氏學。

褒少篤志,有大度,結髮傳充業,博雅疏通,尤好禮事。常感朝廷制度未備,慕叔孫通為漢禮儀,晝夜研精,沈吟專思,寢則懷抱筆札,行則誦習文書,當其念至,忘所之適。

初舉孝廉,再遷圉令,以禮理人,以德化俗。時它郡盜徒五人來入圉界,吏捕得之,陳留太守馬嚴聞而疾惡,風縣殺之。褒敕吏曰:「夫絕人命者,天亦絕之。皋陶不為盜制死刑,管仲遇盜而升諸公。今承旨而殺之,是逆天心,順府意也,其罰重矣。如得全此人命而身坐之,吾所願也。」遂不為殺。嚴奏褒耎弱,免官歸郡,為功曹。

徵拜博士。會肅宗欲制定禮樂,元和二年下詔曰:「河圖稱『赤九會昌,十世以光,十一以興』。尚書琁機鈐曰:『述堯理世,平制禮樂,放唐之文。』予末小子,託于數終,曷以纘興,崇弘祖宗,仁濟元元?帝命驗曰:『順堯考德,題期立象。』且三五步驟,優劣殊軌,況予頑陋,無以克堪,雖欲從之,末由也已。每見圖書,中心恧焉。」褒知帝旨欲有興作,乃上疏曰:「昔者聖人受命而王,莫不制禮作樂,以著功德。功成作樂,化定制禮,所以救世俗,致禎祥,為萬姓獲福於皇天者也。今皇天降祉,嘉瑞並臻,制作之符,甚於言語。宜定文制,著成漢禮,丕顯祖宗盛德之美。」章下太常,太常巢堪以為一世大典,非褒所定,不可許。帝知群僚拘攣,難與圖始,朝廷禮憲,宜時刊立,明年復下詔曰:「朕以不德,膺祖宗弘烈。乃者鸞鳳仍集,麟龍並臻,甘露宵降,嘉穀滋生,赤草之類,紀于史官。朕夙夜祗畏,上無以彰于先功,下無以克稱靈物。漢遭秦餘,禮壞樂崩,且因循故事,未可觀省,有知其說者,各盡所能。」褒省詔,乃歎息謂諸生曰:「昔奚斯頌魯,考甫詠殷。夫人臣依義顯君,竭忠彰主,行之美也。當仁不讓,吾何辭哉!」遂復上疏,具陳禮樂之本,制改之意。拜褒侍中,從駕南巡,既還,以事下三公,未及奏,詔召玄武司馬班固,問改定禮制之宜。固曰:「京師諸儒,多能說禮,宜廣招集,共議得失。」帝曰:「諺言『作舍道邊,三年不成』。會禮之家,名為聚訟,互生疑異,筆不得下。昔堯作大章,一夔足矣。」章和元年正月,乃召褒詣嘉德門,令小黃門持班固所上叔孫通漢儀十二篇,敕褒曰:「此制散略,多不合經,今宜依禮條正,使可施行。於南宮、東觀盡心集作。」褒既受命,乃次序禮事,依準舊典,雜以五經讖記之文,撰次天子至於庶人冠婚吉凶終始制度,以為百五十篇,寫以二尺四寸簡。其年十二月奏上。帝以眾論難一,故但納之,不復令有司平奏。會帝崩,和帝即位,褒乃為作章句,帝遂以新禮二篇冠。擢褒監羽林左騎。永元四年,遷射聲校尉。後太尉張酺、尚書張敏等奏褒擅制漢禮,破亂聖術,宜加刑誅。帝雖寢其奏,而漢禮遂不行。

褒在射聲,營舍有停棺不葬者百餘所,褒親自履行,問其意故。吏對曰:「此等多是建武以來絕無後者,不得埋掩。」褒乃愴然,為買空地,悉葬其無主者,設祭以祀之。遷城門校尉、將作大匠。時有疾疫,褒巡行病徒,為致醫藥,經理饘粥,多蒙濟活。七年,出為河內太守。時春夏大旱,糧穀踊貴。褒到,乃省吏并職,退去姦殘,澍雨數降。其秋大孰,百姓給足,流冗皆還。後坐上災害不實免。有頃徵,再遷,復為侍中。

褒博物識古,為儒者宗。十四年,卒官。作通義十二篇,演經雜論百二十篇,又傳禮記四十九篇,教授諸生千餘人,慶氏學遂行於世。

論曰:漢初天下創定,朝制無文,叔孫通頗採經禮,參酌秦法,雖適物觀時,有救崩敝,然先王之容典蓋多闕矣,是以賈誼、仲舒、王吉、劉向之徒,懷憤歎息所不能已也。資文、宣之遠圖明懿美,而終莫或用,故知自燕而觀,有不盡矣。孝章永言前王,明發興作,專命禮臣,撰定國憲,洋洋乎盛德之事焉。而業絕天筭,議黜異端,斯道竟復墜矣。夫三王不相襲禮,五帝不相箪樂,所以咸、莖異調,中都殊絕。況物運遷回,情數萬化,制則不能隨其流變,品度未足定其滋章,斯固世主所當損益者也。且樂非夔、襄,而新音代起,律謝皋、蘇,而制令亟易,修補舊文,獨何猜焉?禮云禮云,曷其然哉!

鄭玄字康成,北海高密人也。八世祖崇,哀帝時尚書僕射。玄少為鄉嗇夫,得休歸,常詣學官,不樂為吏,父數怒之,不能禁。遂造太學受業,師事京兆第五元先,始通京氏易、公羊春秋、三統歷、九章筭術。又從東郡張恭祖受周官、禮記、左氏春秋、韓詩、古文尚書。以山東無足問者,乃西入關,因涿郡盧植,事扶風馬融。

融門徒四百餘人,升堂進者五十餘生。融素驕貴,玄在門下,三年不得見,乃使高業弟子傳授於玄。玄日夜尋誦,未嘗怠倦。會融集諸生考論圖緯,聞玄善筭,乃召見於樓上,玄因從質諸疑義,問畢辭歸。融喟然謂門人曰:「鄭生今去,吾道東矣。」

玄自游學,十餘年乃歸鄉里。家貧,客耕東萊,學徒相隨已數百千人。及黨事起,乃與同郡孫嵩等四十餘人俱被禁錮,遂隱修經業,杜門不出。時任城何休好公羊學,遂著公羊墨守、左氏膏肓、穀梁廢疾;玄乃發墨守,鍼膏肓,起廢疾。休見而歎曰:「康成入吾室,操吾矛,以伐我乎!」初,中興之後,范升、陳元、李育、賈逵之徒爭論古今學,後馬融荅北地太守劉镬及玄荅何休,義據通深,由是古學遂明。

靈帝末,黨禁解,大將軍何進聞而辟之。州郡以進權戚,不敢違意,遂迫脅玄,不得已而詣之。進為設几杖,禮待甚優。玄不受朝服,而以幅巾見。一宿逃去。時年六十,弟子河內趙商等自遠方至者數千。後將軍袁隗表為侍中,以父喪不行。國相孔融深敬於玄,屣履造門。告高密縣為玄特立一鄉,曰:「昔齊置『士鄉』,越有『君子軍』,皆異賢之意也。鄭君好學,實懷明德。昔太史公、廷尉吳公、謁者僕射鄧公,皆漢之名臣。又南山四皓有園公、夏黃公,潛光隱耀,世嘉其高,皆悉稱公。然則公者仁德之正號,不必三事大夫也。今鄭君鄉宜曰『鄭公鄉』。昔東海于公僅有一節,猶或戒鄉人侈其門閭,矧乃鄭公之德,而無駟牡之路!可廣開門衢,令容高車,號為『通德門』。」

董卓遷都長安,公卿舉玄為趙相,道斷不至。會黃巾寇青部,乃避地徐州,徐州牧陶謙接以師友之禮。建安元年,自徐州還高密,道遇黃巾賊數萬人,見玄皆拜,相約不敢入縣境。玄後嘗疾篤,自慮,以書戒子益恩曰:「吾家舊貧,為父母群弟所容,去廝役之吏,游學周、秦之都,往來幽、并、兗、豫之域,獲覲乎在位通人,處逸大儒,得意者咸從捧手,有所受焉。遂博稽六蓺,粗覽傳記,時睹祕書緯術之奧。年過四十,乃歸供養,假田播殖,以娛朝夕。遇閹尹擅埶,坐黨禁錮,十有四年,而蒙赦令,舉賢良方正有道,辟大將軍三司府。公車再召,比牒併名,早為宰相。惟彼數公,懿德大雅,克堪王臣,故宜式序。吾自忖度,無任於此,但念述先聖之元意,思整百家之不齊,亦庶幾以竭吾才,故聞命罔從。而黃巾為害,萍浮南北,復歸邦鄉。入此歲來,已七十矣。宿素衰落,仍有失誤,案之禮典,便合傳家。今我告爾以老,歸爾以事,將閑居以安性,覃思以終業。自非拜國君之命,問族親之憂,展敬墳墓,觀省野物,胡嘗扶杖出門乎!家事大小,汝一承之。咨爾煢煢一夫,曾無同生相依。其勗求君子之道,研鑽勿替,敬慎威儀,以近有德。顯譽成於僚友,德行立於己志。若致聲稱,亦有榮於所生,可不深念邪!可不深念邪!吾雖無紱冕之緒,頗有讓爵之高。自樂以論贊之功,庶不遺後人之羞。末所憤憤者,徒以亡親墳壟未成,所好群書率皆腐敝,不得於禮堂寫定,傳與其人。日西方暮,其可圖乎!家今差多於昔,勤力務時,無恤飢寒。菲飲食,薄衣服,節夫二者,尚令吾寡恨。若忽忘不識,亦已焉哉!」

時大將軍袁紹總兵冀州,遣使要玄,大會賓客,玄最後至,乃延升上坐。身長八尺,飲酒一斛,秀眉明目,容儀溫偉。紹客多豪俊,並有才說,見玄儒者,未以通人許之,競設異端,百家互起。玄依方辯對,咸出問表,皆得所未聞,莫不嗟服。時汝南應劭亦歸於紹,因自贊曰:「故太山太守應中遠,北面稱弟子何如?」玄笑曰:「仲尼之門考以四科,回、賜之徒不稱官閥。」劭有慚色。紹乃舉玄茂才,表為左中郎將,皆不就。公車徵為大司農,給安車一乘,所過長吏送迎。玄乃以病自乞還家。

五年春,夢孔子告之曰:「起,起,今年歲在辰,來年歲在巳。」既寤,以讖合之,知命當終,有頃寢疾。時袁紹與曹操相拒於官度,令其子譚遣使逼玄隨軍。不得已,載病到元城縣,疾篤不進,其年六月卒,年七十四。遺令薄葬。自郡守以下嘗受業者,縗絰赴會千餘人。

門人相與撰玄荅諸弟子問五經,依論語作鄭志八篇。凡玄所注周易、尚書、毛詩、儀禮、禮記、論語、孝經、尚書大傳、中候、乾象歷,又著天文七政論、魯禮禘祫義、六蓺論、毛詩譜、駮許慎五經異義、荅臨孝存周禮難,凡百餘萬言。

玄質於辭訓,通人頗譏其繁。至於經傳洽孰,稱為純儒,齊魯閒宗之。其門人山陽郗慮至御史大夫,東萊王基、清河崔琰著名於世。又樂安國淵、任嘏,時並童幼,玄稱淵為國器,嘏有道德,其餘亦多所鑒拔,皆如其言。玄唯有一子益恩,孔融在北海,舉為孝廉;及融為黃巾所圍,益恩赴難隕身。有遺腹子,玄以其手文似己,名之曰小同。

論曰:自秦焚六經,聖文埃滅。漢興,諸儒頗修蓺文;及東京,學者亦各名家。而守文之徒,滯固所稟,異端紛紜,互相詭激,遂令經有數家,家有數說,章句多者或乃百餘萬言,學徒勞而少功,後生疑而莫正。鄭玄括囊大典,網羅眾家,刪裁繁誣,刊改漏失,自是學者略知所歸。王父豫章君每考先儒經訓,而長於玄,常以為仲尼之門不能過也。及傳授生徒,並專以鄭氏家法云。

贊曰:富平之緒,承家載世。伯仁先歸,釐我國祭。玄定義乖,褒修禮缺。孔書遂明,漢章中輟。


\end{pinyinscope}