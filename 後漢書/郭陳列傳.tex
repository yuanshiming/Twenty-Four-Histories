\article{郭陳列傳}

\begin{pinyinscope}
郭躬字仲孫,潁川陽翟人也。家世衣冠。父弘,習小杜律。太守寇恂以弘為決曹掾,斷獄至三十年,用法平。諸為弘所決者,退無怨情,郡內比之東海于公。年九十五卒。

躬少傳父業,講授徒眾常數百人。後為郡吏,辟公府。永平中,奉車都尉竇固出擊匈奴,騎都尉秦彭為副。彭在別屯而輒以法斬人,固奏彭專擅,請誅之。顯宗乃引公卿朝臣平其罪科。躬以明法律,召入議。議者皆然固奏,躬獨曰:「於法,彭得斬之。」帝曰:「軍征,校尉一統於督。彭既無斧鉞,可得專殺人乎?」躬對曰:「一統於督者,謂在部曲也。今彭專軍別將,有異於此。兵事呼吸,不容先關督帥。且漢制棨戟即為斧鉞,於法不合罪。」帝從躬議。又有兄弟共殺人者,而罪未有所歸。帝以兄不訓弟,故報兄重而減弟死。中常侍孫章宣詔,誤言兩報重,尚書奏章矯制,罪當腰斬。帝復召躬問之,躬對「章應罰金」。帝曰:「章矯詔殺人,何謂罰金?」躬曰:「法令有故、誤,章傳命之謬,於事為誤,誤者其文則輕。」帝曰:「章與囚同縣,疑其故也。」躬曰:「『周道如砥,其直如矢。』『君子不逆詐。』君王法天,刑不可以委曲生意。」帝曰:「善。」遷躬廷尉正,坐法免。

後三遷,元和三年,拜為廷尉。躬家世掌法,務在寬平,及典理官,決獄斷刑,多依矜恕,乃條諸重文可從輕者四十一事奏之,事皆施行,著于令。章和元年,赦天下繫囚在四月丙子以前減死罪一等,勿笞,詣金城,而文不及亡命未發覺者。躬上封事曰:「聖恩所以減死罪使戍邊者,重人命也。今死罪亡命無慮萬人,又自赦以來,捕得甚眾,而詔令不及,皆當重論。伏惟天恩莫不蕩宥,死罪已下並蒙更生,而亡命捕得獨不沾澤。臣以為赦前犯死罪而繫在赦後者,可皆勿笞詣金城,以全人命,有益於邊。」肅宗善之,即下詔赦焉。躬奏讞法科,多所生全。永元六年,卒官。中子晊,亦明法律,至南陽太守,政有名跡。弟子鎮。

鎮字桓鍾,少修家業。辟太尉府,再遷,延光中為尚書。及中黃門孫程誅中常侍江京等而立濟陰王,鎮率羽林士擊殺衛尉閻景,以成大功,事在宦者傳。再遷尚書令。太傅、三公奏鎮冒犯白刃,手劍賊臣,姦黨殄滅,宗廟以寧,功比劉章,宜顯爵土,以勵忠貞。乃封鎮為定潁侯,食邑二千戶。拜河南尹,轉廷尉,免。永建四年,卒於家。詔賜冢塋地。

長子賀當嗣爵,讓與小弟時而逃去。積數年,詔大鴻臚下州郡追之,賀不得已,乃出受封。累遷,復至廷尉。及賀卒,順帝追思鎮功,下詔賜鎮謚曰昭武侯,賀曰成侯。

賀弟禎,亦以能法律至廷尉。

鎮弟子禧,少明習家業,兼好儒學,有名譽,延熹中亦為廷尉。建寧二年,代劉寵為太尉。禧子鴻,至司隸校尉,封城安鄉侯。

郭氏自弘後,數世皆傳法律,子孫至公者一人,廷尉七人,侯者三人,刺史、二千石、侍中、中郎將者二十餘人,侍御史、正、監、平者甚眾。

順帝時,廷尉河南吳雄季高,以明法律,斷獄平,起自孤宦,致位司徒。雄少時家貧,喪母,營人所不封土者,擇葬其中。喪事趣辨,不問時日,醫巫皆言當族滅,而雄不顧。及子訢孫恭,三世廷尉,為法名家。

初,肅宗時,司隸校尉下邳趙興亦不卹諱忌,每入官舍,輒更繕修館宇,移穿改築,故犯妖禁,而家人爵祿,益用豐熾,官至潁川太守。子峻,太傅,以才器稱。孫安世,魯相。三葉皆為司隸,時稱其盛。

桓帝時,汝南有陳伯敬者,行必矩步,坐必端膝,呵叱狗馬,終不言死,目有所見,不食其肉,行路聞凶,便解駕留止,還觸歸忌,則寄宿鄉亭。年老寢滯,不過舉孝廉。後坐女婿亡吏,太守邵夔怒而殺之。時人罔忌禁者,多談為證焉。

論曰:曾子云:「上失其道,民散久矣。如得其情,則哀矜而勿喜。」夫不喜於得情則恕心用,恕心用則可寄枉直矣。夫賢人君子斷獄,其必主於此乎?郭躬起自佐史,小大之獄必察焉。原其平刑審斷,庶於勿喜者乎?若乃推己以議物,捨狀以貪情,法家之能慶延于世,蓋由此也!

陳寵字昭公,沛國洨人也。曾祖父咸,成哀閒以律令為尚書。平帝時,王莽輔政,多改漢制,咸心非之。及莽因呂寬事誅不附己者何武、鮑宣等,咸乃歎曰:「易稱『君子見幾而作,不俟終日』,吾可以逝矣!」即乞骸骨去職。及莽篡位,召咸以為掌寇大夫,謝病不肯應。時三子參、豐、欽皆在位,乃悉令解官,父子相與歸鄉里,閉門不出入,猶用漢家祖臘。人問其故,咸曰:「我先人豈知王氏臘乎?」其後莽復徵咸,遂稱病篤。於是乃收斂其家律令書文,皆壁藏之。咸性仁恕,常戒子孫曰:「為人議法,當依於輕,雖有百金之利,慎無與人重比。」

建武初,欽子躬為廷尉左監,早卒。

躬生寵,明習家業,少為州郡吏,辟司徒鮑昱府。是時三府掾屬專尚交遊,以不肯視事為高。寵常非之,獨勤心物務,數為昱陳當世便宜。昱高其能,轉為辭曹,掌天下獄訟。其所平決,無不厭服眾心。時司徒辭訟,久者數十年,事類溷錯,易為輕重,不良吏得生因緣。寵為昱撰辭訟比七卷,決事科條,皆以事類相從。昱奏上之,其後公府奉以為法。

三遷,肅宗初,為尚書。是時承永平故事,吏政尚嚴切,尚書決事率近於重。寵以帝新即位,宜改前世苛俗。乃上疏曰:「臣聞先王之政,賞不僭,刑不濫,與其不得已,寧僭不濫。故唐堯著典,『眚災肆赦』;周公作戒,『勿誤庶獄』;伯夷之典,『惟敬五刑,以成三德』。由此言之,聖賢之政,以刑罰為首。往者斷獄嚴明,所以威懲姦慝,姦慝既平,必宜濟之以寬。陛下即位,率由此義,數詔群僚,弘崇晏晏。而有司執事,未悉奉承,典刑用法,猶尚深刻。斷獄者急於篣格酷烈之痛,執憲者煩於詆欺放濫之文,或因公行私,逞縱威福。夫為政猶張琴瑟,大弦急者小弦絕。故子貢非臧孫之猛法,而美鄭喬之仁政。《詩》云:『不剛不柔,布政優優。』方今聖德充塞,假于上下,宜隆先王之道,蕩滌煩苛之法。輕薄箠楚,以濟群生;全廣至德,以奉天心。」帝敬納寵言,每事務於寬厚。其後遂詔有司,絕鉆鑽諸慘酷之科,解妖惡之禁,除文致之請讞五十餘事,定著于令。是後人俗和平,屢有嘉瑞。

漢舊事斷獄報重,常盡三冬之月,是時帝始改用冬初十月而已。元和二年,旱,長水校尉賈宗等上言,以為斷獄不盡三冬,故陰氣微弱,陽氣發泄,招致災旱,事在於此。帝以其言下公卿議,寵奏曰:「夫冬至之節,陽氣始萌,故十一月有蘭、射干、芸、荔之應。時令曰:『諸生蕩,安形體。』天以為正,周以為春。十二月陽氣上通,雉雊雞乳,地以為正,殷以為春。十三月陽氣已至,天地已交,萬物皆出,蟄蟲始振,人以為正,夏以為春。三微成著,以通三統。周以天元,殷以地元,夏以人元。若以此時行刑,則殷、周歲首皆當流血,不合人心,不稽天意。月令曰:『孟冬之月,趣獄刑,無留罪。』明大刑畢在立冬也。又:『孟冬之月,身欲寧,事欲靜。』若以降威怒,不可謂寧;若以行大刑,不可謂靜。議者咸曰:『旱之所由,咎在改律。』臣以為殷、周斷獄不以三微,而化致康平,無有災害。自元和以前,皆用三冬,而水旱之異,往往為患。由此言之,災害自為它應,不以改律。秦為虐政,四時行刑,聖漢初興,改從簡易。蕭何草律,季秋論囚,俱避立春之月,而不計天地之正,二王之春,實頗有違。陛下探幽析微,允執其中,革百載之失,建永年之功,上有迎承之敬,下有奉微之惠,稽春秋之文,當月令之意,聖功美業,不宜中疑。」書奏,帝納之。遂不復改。

寵性周密,常稱人臣之義,苦不畏慎。自在樞機,謝遣門人,拒絕知友,唯在公家而已。朝廷器之。

皇后弟侍中竇憲,薦真定令張林為尚書,帝以問寵,寵對「林雖有才能,而素行貪濁」,憲以此深恨寵。林卒被用,而以臧汙抵罪。及帝崩,憲等秉權,常銜寵,乃白太后,令典喪事,欲因過中之。黃門侍郎鮑德素敬寵,說憲弟夏陽侯瑰曰:「陳寵奉事先帝,深見納任,故久留臺閣,賞賜有殊。今不蒙忠能之賞,而計幾微之故,誠傷輔政容貸之德。」瑰亦好士,深然之。故得出為太山太守。

後轉廣漢太守。西州豪右并兼,吏多姦貪,訴訟日百數。寵到,顯用良吏王渙、鐔顯等,以為腹心,訟者日減,郡中清肅。先是洛縣城南,每陰雨,常有哭聲聞於府中,積數十年。寵聞而疑其故,使吏案行。還言:「世衰亂時,此下多死亡者,而骸骨不得葬,儻在於是?」寵愴然矜歎,即敕縣盡收斂葬之。自是哭聲遂絕。

及竇憲為大將軍征匈奴,公卿以下及郡國無不遣吏子弟奉獻遺者,而寵與中山相汝南張郴、東平相應順守正不阿。後和帝聞之,擢寵為大司農,郴太僕,順左馮翊。

永元六年,寵代郭躬為廷尉。性仁矜。及為理官,數議疑獄,常親自為奏,每附經典,務從寬恕,帝輒從之,濟活著甚眾。其深文刻敝,於此少衰。寵又鉤校律令條法,溢於甫刑者除之。曰:「臣聞禮經三百,威儀三千,故甫刑大辟二百,五刑之屬三千。禮之所去,刑之所取,失禮則入刑,相為表裏者也。今律令死刑六百一十,耐罪千六百九十八,贖罪以下二千六百八十一,溢於甫刑者千九百八十九,其四百一十大辟,千五百耐罪,七十九贖罪。春秋保乾圖曰:『王者三百年一蠲法。』漢興以來,三百二年,憲令稍增,科條無限。又律有三家,其說各異。宜令三公、廷尉平定律令,應經合義者,可使大辟二百,而耐罪、贖罪二千八百,并為三千,悉刪除其餘令,與禮相應,以易萬人視聽,以致刑措之美,傳之無窮。」未及施行,會坐詔獄吏與囚交通抵罪。詔特免刑,拜為尚書。遷大鴻臚。

寵歷二郡三卿,所在有跡,見稱當時。十六年,代徐防為司空。寵雖傳法律,而兼通經書,奏議溫粹,號為任職相。在位三年薨。以太常南陽尹勤代為司空。

勤字叔梁,篤性好學,屏居人外,荊棘生門,時人重其節。後以定策立安帝,封福亭侯,五百戶。永初元年,以雨水傷稼,策免就國。病卒,無子,國除。

寵子忠。

忠字伯始,永始中辟司徒府,三遷廷尉正,以才能有聲稱。司徒劉愷舉忠明習法律,宜備機密,於是擢拜尚書,使居三公曹。忠自以世典刑法,用心務在寬詳。初,父寵在廷尉,上除漢法溢於甫刑者,未施行,及寵免後遂寢。而苛法稍繁,人不堪之。忠略依寵意,奏上二十三條,為決事比,以省請讞之敝。又上除蠶室刑;解臧吏三世禁錮;狂易殺人,得減重論;母子兄弟相代死,聽,赦所代者。事皆施行。

及鄧太后崩,安帝始親朝事。忠以為臨政之初,宜微聘賢才,以宣助風化,數上薦隱逸及直道之士馮良、周燮、杜根、成翊世之徒。於是公車禮聘良、燮等。後連有災異,詔舉有道,公卿百僚各上封事。忠以詔書既開諫爭,慮言事者必多激切,或致不能容,乃上疏豫通廣帝意。曰:「臣聞仁君廣山藪之大,納切直之謀;忠臣盡謇諤之節,不畏逆耳之害。是以高祖舍周昌桀紂之譬,孝文嘉爰盎人豕之譏,武帝納東方朔宣室之正,元帝容薛廣德自刎之切。昔晉平公問於叔向曰:『國家之患孰為大?』對曰:『大臣重祿不極諫,小臣畏罪不敢言,下情不上通,此患之大者。』公曰:『善。』於是下令曰:『吾欲進善,有謁而不通者,罪至死。』今明詔崇高宗之德,推宋景之誠,引咎克躬,諮訪群吏。言事者見杜根、成翊世等新蒙表錄,顯列二臺,必承風響應,爭為切直。若嘉謀異策,宜輒納用。如其管穴,妄有譏刺,雖苦口逆耳,不得事實,且優遊寬容,以示聖朝無諱之美。若有道之士,對問高者,宜垂省覽,特遷一等,以廣直言之路。」書御,有詔拜有道高第士沛國施延為侍中,延後位至太尉。

常侍江京、李閏等皆為列侯,共秉權任。帝又愛信阿母王聖,封為野王君。忠內懷懼懣而未敢陳諫,乃作搢紳先生論以諷,文多故不載。

自帝即位以後,頻遭元二之厄,百姓流亡,盜賊並起,郡縣更相飾匿,莫肯糾發。忠獨以為憂,上疏曰:「臣聞輕者重之端,小者大之源,故隄潰蟻孔,氣洩鍼芒。是以明者慎微,智者識幾。書曰:『小不可不殺。』《詩》云:『無縱詭隨,以謹無良。』蓋所以崇本絕末,鉤深之慮也。臣竊見元年以來,盜賊連發,攻亭劫掠,多所傷殺。夫穿窬不禁,則致彊盜;彊盜不斷,則為攻盜;攻盜成群,必生大姦。故亡逃之科,憲令所急,至於通行飲食,罪致大辟。而頃者以來,莫以為憂。州郡督錄怠慢,長吏防禦不肅,皆欲採獲虛名,諱以盜賊為負。雖有發覺,不務清澄。至有逞威濫怒,無辜僵仆。或有跼蹐比伍,轉相賦斂。或隨吏追赴,周章道路。是以盜發之家,不敢申告,鄰舍比里,共相壓迮,或出私財,以償所亡。其大章著不可掩者,乃肯發露。陵遲之漸,遂且成俗。寇攘誅咎,皆由於此。前年勃海張伯路,可為至戒。覆車之軌,其跡不遠。蓋失之末流,求之本源。宜糾增舊科,以防來事。自今彊盜為上官若它郡縣所糾覺,一發,部吏皆正法,尉貶秩一等,令長三月奉贖罪;二發,尉免官,令長貶秩一等;三發以上,令長免官。便可撰立科條,處為詔文,切敕刺史,嚴加糾罰。冀以猛濟寬,驚懼姦慝。頃季夏大暑,而消息不協,寒氣錯時,水涌為變。天之降異,必有其故。所舉有道之士,可策問國典所務,王事過差,令處煖氣不效之意。庶有讜言,以承天誡。」

元初三年有詔,大臣得行三年喪,服闋還職。忠因此上言:「孝宣皇帝舊令,人從軍屯及給事縣官者,大父母死未滿三月,皆勿徭,令得葬送。請依此制。」太后從之。至建光中,尚書令祝諷、尚書孟布等奏,以為「孝文皇帝定約禮之制,光武皇帝絕告寧之典,貽則萬世,誠不可改。宜復建武故事」。忠上疏曰:「臣聞之孝經,始於愛親,終於哀戚。上自天子,下至庶人,尊卑貴賤,其義一也。夫父母於子,同氣異息,一體而分,三年乃免於懷抱。先聖緣人情而著其節,制服二十五月,是以春秋臣有大喪,君三年不呼其門,閔子雖要絰服事,以赴公難,退而致位,以究私恩,故稱『君使之非也,臣行之禮也』。周室陵遲,禮制不序,蓼莪之人作詩自傷曰:『瓶之罊矣,惟罍之恥。』言己不得終竟子道者,亦上之恥也。高祖受命,蕭何創制,大臣有寧告之科,合於致憂之義。建武之初,新承大亂,凡諸國政,多趣簡易,大臣既不得告寧,而群司營祿念私,鮮循三年之喪,以報顧復之恩者。禮義之方,實為彫損。大漢之興,雖承衰敝,而先王之制,稍以施行。故藉田之耕,起於孝文;孝廉之貢,發於孝武;郊祀之禮,定於元、成;三雍之序,備於顯宗;大臣終喪,成乎陛下。聖功美業,靡以尚茲。孟子有言:『老吾老以及人之老,幼吾幼以及人之幼,天下可運於掌。』臣願陛下登高北望,以甘陵之思,揆度臣子之心,則海內咸得其所。」宦豎不便之,竟寢忠奏而從諷、布議,遂著于令。

忠以久次,轉為僕射。時帝數遣黃門常侍及中使伯榮往來甘陵,而伯榮負寵驕蹇,所經郡國莫不迎為禮謁。又霖雨積時,河水涌溢,百姓騷動。忠上疏曰:「臣聞位非其人,則庶事不敘;庶事不敘,則政有得失;政有得失,則感動陰陽,妖變為應。陛下每引災自厚,不責臣司,臣司狃恩,莫以為負。故天心未得,隔并屢臻,青、冀之域淫雨漏河,徐、岱之濱海水盆溢,兗、豫蝗蝝滋生,荊、楊稻收儉薄,并涼二州羌戎叛戾。加以百姓不足,府帑虛匱,自西徂東,杼柚將空。臣聞洪範五事,一曰貌,貌以恭,恭作肅,貌傷則狂,而致常雨。春秋大水,皆為君上威儀不穆,臨蒞不嚴,臣下輕慢,貴倖擅權,陰氣盛彊,陽不能禁,故為淫雨。陛下以不得親奉孝德皇園廟,比遣中使致敬甘陵,朱軒軿馬,相望道路,可謂孝至矣。然臣竊聞使者所過,威權翕赫,震動郡縣,王侯二千石至為伯榮獨拜車下,儀體上僭,侔於人主。長吏惶怖譴責,或邪諂自媚,發人修道,繕理亭傳,多設儲跱,徵役無度,老弱相隨,動有萬計,賂遺僕從,人數百匹,頓踣呼嗟,莫不叩心。河閒託叔父之屬,清河有陵廟之尊,及剖符大臣,皆猥為伯榮屈節車下。陛下不問,必以陛下欲其然也。伯榮之威重於陛下,陛下之柄在於臣妾。水災之發,必起於此。昔韓嫣託副車之乘,受馳視之使;江都誤為一拜,而嫣受歐刀之誅。臣願明主嚴天元之尊,正乾剛之位,職事巨細,皆任賢能,不宜復令女使干錯萬機。重察左右,得無石顯泄漏之姦;尚書納言,得無趙昌譖崇之詐;公卿大臣,得無朱博阿傅之援;外屬近戚,得無王鳳害商之謀。若國政一由帝命,王事每決於己,則下不得偪上,臣不得干君,常雨大水必當霽止,四方眾異不能為害。」書奏不省。

時三府任輕,機事專委尚書,而災眚變咎,輒切免公台。忠以為非國舊體,上疏諫曰:「臣聞『君使臣以禮,臣事君以忠』。故三公稱曰冢宰,王者待以殊敬,在輿為下,御坐為起,入則參對而議政事,出則監察而董是非。漢典舊事,丞相所請,靡有不聽。今之三公,雖當其名而無其實,選舉誅賞,一由尚書,尚書見任,重於三公,陵遲以來,其漸久矣。臣忠心常獨不安,是故臨事戰懼,不敢穴見有所興造,又不希意同僚,以謬平典,而謗讟日聞,罪足萬死。近以地震策免司空陳褒,今者災異,復欲切讓三公。昔孝成皇帝以妖星守心,移咎丞相,使賁麗納說方進,方進自引,卒不蒙上天之福,徒乖宋景之誠。故知是非之分,較然有歸矣。又尚書決事,多違故典,罪法無例,詆欺為先,文慘言醜,有乖章憲。宜責求其意,割而勿聽。上順國典,下防威福,置方員於規矩,審輕重於衡石,誠國家之典,萬世之法也。」

忠意常在褒崇大臣,待下以禮。其九卿有疾,使者臨問,加賜錢布,皆忠所建奏。頃之,遷尚書令。延光三年,拜司隸校尉。糾正中官外戚賓客,近倖憚之,不欲忠在內。明年,出為江夏太守,復留拜尚書令,會疾卒。

初,太尉張禹、司徒徐防欲與忠父寵共奏追封和熹皇后父護羌校尉鄧訓,寵以先世無奏請故事,爭之連日不能奪,乃從二府議。及訓追加封謚,禹、防復約寵俱遣子奉禮於虎賁中郎將鄧騭,寵不從,騭心不平之,故忠不得志于鄧氏。及騭等敗,眾庶多怨之,而忠數上疏陷成其惡,遂詆劾大司農朱寵。順帝之為太子廢也,諸名臣來歷、祝諷等守闕固爭,時忠為尚書令,與諸尚書復共劾奏之。及帝立,司隸校尉虞詡追奏忠等罪過,當世以此譏焉。

論曰:陳公居理官則議獄緩死,相幼主則正不僭寵,可謂有宰相之器矣。忠能承風,亦庶乎明慎用刑而不留獄。然其聽狂易殺人,開父子兄弟得相代死,斯大謬矣。是則不善人多幸,而善人常代其禍,進退無所措也。

贊曰:陳、郭主刑,人賴其平。寵矜枯胔,躬斷以情。忠用詳密,損益有程。施于孫子,且公且卿。


\end{pinyinscope}