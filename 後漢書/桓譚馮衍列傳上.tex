\article{桓譚馮衍列傳上}

\begin{pinyinscope}
桓譚字君山,沛國相人也。父成帝時為太樂令。譚以父任為郎,因好音律,善鼓琴。博學多通,遍習五經,皆詁訓大義,不為章句。能文章,尤好古學,數從劉歆、楊雄辯析疑異。性嗜倡樂,簡易不修威儀,而憙非毀俗儒,由是多見排扺。

哀平閒,位不過郎。傅皇后父孔鄉侯晏深善於譚。是時高安侯董賢寵幸,女弟為昭儀,皇后日已疏,晏嘿嘿不得意。譚進說曰:「昔武帝欲立衛子夫,陰求陳皇后之過,而陳后終廢,子夫竟立。今董賢至愛而女弟尤幸,殆將有子夫之變,可不憂哉!」晏驚動,曰:「然,為之柰何?」譚曰:「刑罰不能加無罪,邪枉不能勝正人。夫士以才智要君,女以媚道求主。皇后年少,希更艱難,或驅使醫巫,外求方技,此不可不備。又君侯以后父尊重而多通賓客,必借以重埶,貽致譏議。不如謝遣門徒,務執謙愨,此脩己正家避禍之道也。」晏曰「善」。遂罷遣常客,入白皇后,如譚所戒。後賢果風太醫令真欽,使求傅氏罪過,遂逮后弟侍中喜,詔獄無所得,乃解,故傅氏終全於哀帝之時。及董賢為大司馬,聞譚名,欲與之交。譚先奏書於賢,說以輔國保身之術,賢不能用,遂不與通。當王莽居攝篡弒之際,天下之士,莫不競褒稱德美,作符命以求容媚,譚獨自守,默然無言。莽時為掌樂大夫,更始立,召拜太中大夫。

世祖即位,徵待詔,上書言事失旨,不用。後大司空宋弘薦譚,拜議郎給事中,因上疏陳時政所宜,曰:

臣聞國之廢興,在於政事;政事得失,由乎輔佐。輔佐賢明,則俊士充朝,而理合世務;輔佐不明,則論失時宜,而舉多過事。夫有國之君,俱欲興化建善,然而政道未理者,其所謂賢者異也。昔楚莊王問孫叔敖曰:「寡人未得所以為國是也。」叔敖曰:「國之有是,眾所惡也,恐王不能定也。」王曰:「不定獨在君,亦在臣乎?」對曰:「君驕士,曰士非我無從富貴;士驕君,曰君非士無從安存。人君或至失國而不悟,士或至飢寒而不進。君臣不合,則國是無從定矣。」莊王曰:「善。願相國與諸大夫共定國是也。」蓋善政者,視俗而施教,察失而立防,威德更興,文武迭用,然後政調於時,而躁人可定。昔董仲舒言「理國譬若琴瑟,其不調者則解而更張。」夫更張難行,而拂眾者亡,是故賈誼以才逐,而朝錯以智死。世雖有殊能而終莫敢談者,懼於前事也。

且設法禁者,非能盡塞天下之姦,皆合眾人之所欲也,大抵取便國利事多者,則可矣。夫張官置吏,以理萬人,縣賞設罰,以別善惡,惡人誅傷,則善人蒙福矣。今人相殺傷,雖已伏法,而私結怨讎,子孫相報,後忿深前,至於滅戶殄業,而俗稱豪健,故雖有怯弱,猶勉而行之,此為聽人自理而無復法禁者也。今宜申明舊令,若已伏官誅而私相傷殺者,雖一身逃亡,皆徙家屬於邊,其相傷者,加常二等,不得雇山贖罪。如此,則仇怨自解,盜賊息矣。

夫理國之道,舉本業而抑末利,是以先帝禁人二業,錮商賈不得宦為吏,此所以抑并兼長廉恥也。今富商大賈,多放錢貨,中家子弟,為之保役,趨走與臣僕等勤,收稅與封君比入,是以眾人慕效,不耕而食,至乃多通侈靡,以淫耳目。今可令諸商賈自相糾告,若非身力所得,皆以臧畀告者。如此,則專役一己,不敢以貨與人,事寡力弱,必歸功田畝。田畝修,則穀入多而地力盡矣。

又見法令決事,輕重不齊,或一事殊法,同罪異論,姦吏得因緣為市,所欲活則出生議,所欲陷則與死比,是為刑開二門也。今可令通義理明習法律者,校定科比,一其法度,班下郡國,蠲除故條。如此,天下知方,而獄無怨濫矣。

書奏,不省。

是時帝方信讖,多以決定嫌疑。又酬賞少薄,天下不時安定。譚復上疏曰:

臣前獻瞽言,未蒙詔報,不勝憤懣,冒死復陳。愚夫策謀,有益於政道者,以合人心而得事理也。凡人情忽於見事而貴於異聞,觀先王之所記述,咸以仁義正道為本,非有奇怪虛誕之事。蓋天道性命,聖人所難言也。自子貢以下,不得而聞,況後世淺儒,能通之乎!今諸巧慧小才伎數之人,增益圖書,矯稱讖記,以欺惑貪邪,詿誤人主,焉可不抑遠之哉!臣譚伏聞陛下窮折方士黃白之術,甚為明矣;而乃欲聽納讖記,又何誤也!其事雖有時合,譬猶卜數隻偶之類。陛下宜垂明聽,發聖意,屏群小之曲說,述五經之正義,略雷同之俗語,詳通人之雅謀。

又臣聞安平則尊道術之士,有難則貴介冑之臣。今聖朝興復祖統,為人臣主,而四方盜賊未盡歸伏者,此權謀未得也。臣譚伏觀陛下用兵,諸所降下,既無重賞以相恩誘,或至虜掠奪其財物,是以兵長渠率,各生狐疑,黨輩連結,歲月不解。古人有言曰:「天下皆知取之為取,而莫知與之為取。」陛下誠能輕爵重賞,與士共之,則何招而不至,何說而不釋,何向而不開,何征而不剋!如此,則能以狹為廣,以遲為速,亡者復存,失者復得矣。

帝省奏,愈不悅。

其後有詔會議靈臺所處,帝謂譚曰:「吾欲讖決之,何如?」譚默然良久,曰:「臣不讀讖。」帝問其故,譚復極言讖之非經。帝大怒曰:「桓譚非聖無法,將下斬之。」譚叩頭流血,良久乃得解。出為六安郡丞;意忽忽不樂,道病卒,時年七十餘。

初,譚著書言當世行事二十九篇,號曰新論,上書獻之,世祖善焉。琴道一篇未成,肅宗使班固續成之。所著賦、誄、書、奏,凡二十六篇。

元和中,肅宗行東巡狩,至沛,使使者祠譚冢,鄉里以為榮。

馮衍字敬通,京兆杜陵人也。祖野王,元帝時為大鴻臚。衍幼有奇才,年九歲,能誦詩,至二十而博通群書。王莽時,諸公多薦舉之者,衍辭不肯仕。

時天下兵起,莽遣更始將軍廉丹討伐山東。丹辟衍為掾,與俱至定陶。莽追詔丹曰:「倉廩盡矣,府庫空矣,可以怒矣,可以戰矣。將軍受國重任,不捐身於中野,無以報恩塞責。」丹惶恐,夜召衍,以書示之。衍因說丹曰:「衍聞順而成者,道之所大也;逆而功者,權之所貴也。是故期於有成,不問所由;論於大體,不守小節。昔逢丑父伏軾而使其君取飲,稱於諸侯;鄭祭仲立突而出忽,終得復位,美於春秋。蓋以死易生,以存易亡,君子之道也。詭於眾意,寧國存身,賢智之慮也。故《易》曰『窮則變,變則通,通則久,是以自天祐之,吉,無不利』。若夫知其不可而必行之,破軍殘眾,無補於主,身死之日,負義於時,智者不為,勇者不行。且衍聞之,得時無怠。張良以五世相韓,椎秦始皇博浪之中,勇冠乎賁、育,名高乎太山。將軍之先,為漢信臣。新室之興,英俊不附。今海內潰亂,人懷漢德,甚於詩人思召公也,愛其甘棠,而況子孫乎?人所歌舞,天必從之。方今為將軍計,莫若屯據大郡,鎮撫吏士,砥厲其節,百里之內,牛酒日賜,納雄桀之士,詢忠智之謀,要將來之心,待從橫之變,興社稷之利,除萬人之害,則福祿流於無窮,功烈著於不滅。何與軍覆於中原,身膏於草野,功敗名喪,恥及先祖哉?聖人轉禍而為福,智士因敗而為功,願明公深計而無與俗同。」丹不能從。進及睢陽,復說丹曰:「蓋聞明者見於無形,智者慮於未萌,況其昭晢者乎?凡患生於所忽,禍發於細微,敗不可悔,時不可失。公孫鞅曰:『有高人之行,負非於世;有獨見之慮,見贅於人。』故信庸庸之論,破金石之策,襲當世之操,失高明之德。夫決者智之君也,疑者事之役也。時不重至,公勿再計。」丹不聽,遂進及無鹽,與赤眉戰死。衍乃亡命河東。

更始二年,遣尚書僕射鮑永行大將軍事,安集北方。衍因以計說永曰:

衍聞明君不惡切愨之言,以測幽冥之論;忠臣不顧爭引之患,以達萬機之變。是故君臣兩興,功名兼立,銘勒金石,令問不忘。今衍幸逢寬明之日,將值危言之時,豈敢拱默避罪,而不竭其誠哉!

伏念天下離王莽之害久矣。始自東郡之師,繼以西海之役,巴、蜀沒於南夷,緣邊破於北狄,遠征萬里,暴兵累年,禍挐未解,兵連不息,刑法彌深,賦斂愈重。眾彊之黨,橫擊於外,百僚之臣,貪殘於內,元元無聊,飢寒並臻,父子流亡,夫婦離散,廬落丘墟,田疇蕪穢,疾疫大興,災異蜂起。於是江湖之上,海岱之濱,風騰波涌,更相駘藉,四垂之人,肝腦塗地,死亡之數,不啻太半,殃咎之毒,痛入骨髓,匹夫僮婦,咸懷怨怒。皇帝以聖德靈威,龍興鳳舉,率宛、葉之眾,將散亂之兵,喢血昆陽,長驅武關,破百萬之陳,摧九虎之軍,雷震四海,席卷天下,攘除禍亂,誅滅無道,一期之閒,海內大定。繼高祖之休烈,修文武之絕業,社稷復存,炎精更輝,德冠往初,功無與二。天下自以去亡新,就聖漢,當蒙其福而賴其願。樹恩布德,易以周洽,其猶順驚風而飛鴻毛也。然而諸將虜掠,逆倫絕理,殺人父子,妻人婦女,燔其室屋,略其財產,飢者毛食,寒者裸跣,冤結失望,無所歸命。今大將軍以明淑之德,秉大使之權,統三軍之政,存撫并州之人,惠愛之誠,加乎百姓,高世之聲,聞乎群士,故其延頸企踵而望者,非特一人也。且大將軍之事,豈得珪璧其行,束修其心而已哉?將定國家之大業,成天地之元功也。昔周宣中興之主,齊桓霸彊之君耳,猶有申伯、召虎、夷吾、吉甫攘其蝥賊,安其疆宇。況乎萬里之漢,明帝復興,而大將軍為之梁棟,此誠不可以忽也。

且衍聞之,兵久則力屈,人愁則變生。今邯鄲之賊未滅,真定之際復擾,而大將軍所部不過百里,守城不休,戰軍不息,兵革雲翔,百姓震駭,柰何自怠,不為深憂?夫并州之地,東帶名關,北逼彊胡,年穀獨孰,人庶多資,斯四戰之地,攻守之場也。如其不虞,何以待之?故曰「德不素積,人不為用。備不豫具,難以應卒」。今生人之命,縣於將軍,將軍所杖,必須良才,宜改易非任,更選賢能。夫十室之邑,必有忠信。審得其人,以承大將軍之明,雖則山澤之人,無不感德,思樂為用矣。然後簡精銳之卒,發屯守之士,三軍既整,甲兵已具,相其土地之饒,觀其水泉之利,制屯田之術,習戰射之教,則威風遠暢,人安其業矣。若鎮太原,撫上黨,收百姓之歡心,樹名賢之良佐,天下無變,則足以顯聲譽,一朝有事,則可以建大功。惟大將軍開日月之明,發深淵之慮,監六經之論,觀孫吳之策,省群議之是非,詳眾士之白黑,以超周南之跡,垂甘棠之風,令夫功烈施於千載,富貴傳于無窮。伊、望之策,何以加茲!

永既素重衍,為且受使得自置偏裨,乃以衍為立漢將軍,領狼孟長,屯太原,與上黨太守田邑等繕甲養士,扞衛并土。

及世祖即位,遣宗正劉延攻天井關,與田邑連戰十餘合,延不得進。邑迎母弟妻子,為延所獲。後邑聞更始敗,乃遣使詣洛陽獻璧馬,即拜為上黨太守。因遣使者招永、衍,永、衍等疑不肯降,而忿邑背前約,衍乃遣邑書曰:

蓋聞晉文出奔而子犯宣其忠,趙武逢難而程嬰明其賢,二子之義當矣。今三王背畔,赤眉危國,天下螘動,社稷顛隕,是忠臣立功之日,志士馳馬之秋也。伯玉擢選剖符,專宰大郡。夫上黨之地,有四塞之固,東帶三關,西為國蔽,柰何舉之以資彊敵,開天下之匈,假仇讎之刃?豈不哀哉!

衍聞之,委質為臣,無有二心;挈瓶之智,守不假器。是以晏嬰臨盟,擬以曲戟,不易其辭;謝息守郕,脅以晉、魯,不喪其邑。由是言之,內無鉤頸之禍,外無桃萊之利,而被畔人之聲,蒙降城之恥,竊為左右羞之。且邾庶其竊邑畔君,以要大利,曰賤而必書;莒牟夷以土地求食,而名不滅。是以大丈夫動則思禮,行則思義,未有背此而身名能全者也。為伯玉深計,莫若與鮑尚書同情戮力,顯忠貞之節,立超世之功。如以尊親係累之故,能捐位投命,歸之尚書,大義既全,敵人紓怨,上不損剖符之責,下足救老幼之命,申眉高談,無愧天下。若乃貪上黨之權,惜全邦之實,衍恐伯玉必懷周趙之憂,上黨復有前年之禍。昔晏平仲納延陵之誨,終免欒高之難;孫林父違穆子之戒,故陷終身之惡。以為伯玉聞此至言,必若刺心,自非嬰城而堅守,則策馬而不顧也。聖人轉禍而為福,智士因敗以成勝,願自彊於時,無與俗同。

邑報書曰:邑報書曰:

僕雖駑怯,亦欲為人者也,豈苟貪生而畏死哉!曲戟在頸,不易其心,誠僕志也。

閒者,老母諸弟見執於軍,而邑安然不顧者,豈非重其節乎?若使人居天地,壽如金石,要長生而避死地可也。今百齡之期,未有能至,老壯之閒,相去幾何。誠使故朝尚在,忠義可立,雖老親受戮,妻兒橫分,邑之願也。

閒者,上黨黠賊,大眾圍城,義兵兩輩,入據井陘。邑親潰敵圍,拒擊宗正,自試智勇,非不能當。誠知故朝為兵所害,新帝司徒已定三輔,隴西、北地從風響應。其事昭昭,日月經天,河海帶地,不足以比。死生有命,富貴在天。天下存亡,誠云命也。邑雖沒身,能如命何?

夫人道之本,有恩有義,義有所宜,恩有所施。君臣大義,母子至恩。今故主已亡,義無誰為;老母拘執,恩所當留。而厲以貪權,誘以策馬,抑其利心,必其不顧,何其愚乎!

邑年三十,歷位卿士,性少嗜慾,情厭事為。況今位尊身危,財多命殆,鄙人知之,何疑君子?

君長、敬通揭節垂組,自相署立。蓋仲由使門人為臣,孔子譏其欺天。君長據位兩州,加以一郡,而河東畔國,兵不入彘,上黨見圍,不窺大谷,宗正臨境,莫之能援。兵威屈辱,國權日損,三王背畔,赤眉害主,未見兼行倍道之赴,若墨翟累繭救宋,申包胥重胝存楚,衛女馳歸唁兄之志。主亡一歲,莫知定所,虛冀妄言,苟肆鄙塞。未能事生,安能事死?未知為臣,焉知為主?豈厭為臣子,思為君父乎!欲搖太山而蕩北海,事敗身危,要思邑言。

衍不從。或訛言更始隨赤眉在北,永、衍信之,故屯兵界休,方移書上黨,云皇帝在雍,以惑百姓。永遣弟升及子婿張舒誘降涅城,舒家在上黨,邑悉繫之。又書勸永降,永不荅,自是與邑有隙。邑字伯玉,馮翊人也,後為漁陽太守。永、衍審知更始已歿,乃共罷兵,幅巾降於河內。

帝怨衍等不時至,永以立功得贖罪,遂任用之,而衍獨見黜。永謂衍曰:「昔高祖賞季布之罪,誅丁固之功。今遭明主,亦何憂哉!」衍曰:「記有之,人有挑其鄰人之妻者,挑其長者,長者詈之,挑其少者,少者報之,後其夫死而取其長者。或謂之曰:『夫非罵爾者邪?』曰:『在人欲其報我,在我欲其罵人也。』夫天命難知,人道易守,守道之臣,何患死亡?」頃之,帝以衍為曲陽令,誅斬劇賊郭勝等,降五千餘人,論功當封,以讒毀,故賞不行。

建武六年日食,衍上書陳八事:其一曰顯文德,二曰褒武烈,三曰修舊功,四曰招俊傑,五曰明好惡,六曰簡法令,七曰差秩祿,八曰撫邊境。書奏,帝將召見。初,衍為狼孟長,以罪摧陷大姓令狐略,是時略為司空長史,讒之於尚書令王護、尚書周生豐曰:「衍所以求見者,欲毀君也。」護等懼之,即共排閒,衍遂不得入。

後衛尉陰興、新陽侯陰就以外戚貴顯,深敬重衍,衍遂與之交結,由是為諸王所聘請,尋為司隸從事。帝懲西京外戚賓客,故皆以法繩之,大者抵死徙,其餘至貶黜。衍由此得罪,嘗自詣獄,有紹赦不問。西歸故郡,閉門自保,不敢復與親故通。


\end{pinyinscope}