\article{虞傅蓋臧列傳}

\begin{pinyinscope}
虞詡字升卿,陳國武平人也。祖父經,為郡縣獄吏,案法平允,務存寬恕,每冬月上其狀,恆流涕隨之。嘗稱曰:「東海于公高為里門,而其子定國卒至丞相。吾決獄六十年矣,雖不及于公,其庶幾乎!子孫何必不為九卿邪?」故字詡曰升卿。

詡年十二,能通尚書。早孤,孝養祖母。縣舉順孫,國相奇之,欲以為吏。詡辭曰:「祖母九十,非詡不養。」相乃止。後祖母終,服闋,辟太尉李脩府,拜郎中。

永初四年,羌胡反亂,殘破并、涼,大將軍鄧騭以軍役方費,事不相贍,欲棄涼州,并力北邊,乃會公卿集議。騭曰:「譬若衣敗,壞一以相補,猶有所完。若不如此,將兩無所保。」議者咸同。詡聞之,乃說李脩曰:「竊聞公卿定策當棄涼州,求之愚心,未見其便。先帝開拓土字,劬勞後定,而今憚小費,舉而棄之。涼州既棄,即以三輔為塞;三輔為塞,則園陵單外。此不可之甚者也。喭曰:『關西出將,關東出相。』觀其習兵壯勇,實過餘州。今羌胡所以不敢入據三輔,為心腹之害者,以涼州在後故也。其土人所以推鋒執銳,無反顧之心者,為臣屬於漢故也。若棄其境域,徙其人庶,安土重遷,必生異志。如使豪雄相聚,席卷而東,雖賁、育為卒,太公為將,猶恐不足當禦。議者喻以補衣猶有所完,詡恐其疽食侵淫而無限極。棄之非計。」脩曰:「吾意不及此。微子之言,幾敗國事。然則計當安出?」詡曰:「今涼土擾動,人情不安,竊憂卒然有非常之變。誠宜令四府九卿,各辟彼州數人,其牧守令長子弟皆除為冗官,外以勸厲,荅其功勤,內以拘致;防其邪計。」脩善其言,更集四府,皆從詡議。於是辟西州豪桀為掾屬,拜牧守長吏子弟為郎,以安慰之。

鄧騭兄弟以詡異其議,因此不平,欲以吏法中傷詡。後朝歌賊甯季等數千人攻殺長吏,屯聚連年,州郡不能禁,乃以詡為朝歌長。故舊皆弔詡曰:「得朝歌何衰!」詡笑曰:「志不求易,事不避難,臣之職也。不遇槃根錯節,何以別利器乎?」始到,謁河內大守馬棱。棱勉之曰:「君儒者,當謀謨廟堂,反在朝歌邪?」詡曰:「初除之日,士大夫皆見弔勉。以詡譸之,知其無能為也。朝歌者,韓、魏之郊,背太行,臨黃河,去敖倉百里,而青、冀之人流亡萬數。賊不知開倉招眾,劫庫兵,守城皋,斷天下右臂,此不足憂也。今其眾新盛,難與爭鋒。兵不猒權,願寬假轡策,勿令有所拘閡而已。」及到官,設令三科以募求壯士,自掾史以下各舉所知,其攻劫者為上,傷人偷盜者次之,帶喪服而不事家業為下。收得百餘人,詡為饗會,悉貰其罪,使入賊中,誘令劫掠,乃伏兵以待之,遂殺賊數百人。又潛遣貧人能縫者,傭作賊衣,以采綖縫其裾為幟,有出巿里者,吏輒禽之。賊由是駭散,咸稱神明。遷懷令。

後羌寇武都,鄧太后以詡有將帥之略,遷武都太守,引見嘉德殿,厚加賞賜。羌乃率眾數千,遮詡於陳食、崤谷,詡即停軍不進,而宣言上書請兵,須到當發。羌聞之,乃分鈔傍縣,詡因其兵散,日夜進道,兼行百餘里。令吏士各作兩灶,日增倍之,羌不敢逼。或問曰:「孫臏減灶而君增之。兵法日行不過三十里,以戒不虞,而今日且二百里。何也?」詡曰:「虜眾多,吾兵少。徐行則易為所及,速進則彼所不測。虜見吾灶日增,必謂郡兵來迎。眾多行速,必憚追我。孫臏見弱,吾今示彊,埶有不同故也。」

既到郡,兵不滿三千,而羌眾萬餘,攻圍赤亭數十日。詡乃令軍中,使彊弩勿發,而潛發小弩。羌以為矢力弱,不能至,并兵急攻。詡於是使二十彊弩共射一人,發無不中,羌大震,退。詡因出城奮擊,多所傷殺。明日悉陳其兵眾,令從東郭門出,北郭門入,貿易衣服,回轉數周。羌不知其數,更相恐動。詡計賊當退,乃潛遣五百餘人於淺水設伏,候其走路。虜果大奔,因掩擊,大破之,斬獲甚眾,賊由是敗散,南入益州。詡乃占相地埶,築營壁百八十所,招還流亡,假賑貧人,郡遂以安。

先是運道艱險,舟車不通,驢馬負載,僦五致一。詡乃自將吏士,案行川谷,自沮至下辯數十里中,皆燒石翦木,開漕船道,以人僦直雇借傭者,於是水運通利,歲省四千餘萬。詡始到郡,戶裁盈萬。及綏聚荒餘,招還流散,二三年閒,遂增至四萬餘戶。鹽米豐賤,十倍於前。坐法免。

永建元年,代陳禪為司隸校尉。數月閒,奏太傅馮石、太尉劉熹、中常侍程璜、陳秉、孟生、李閏等,百官側目,號為苛刻。三公劾奏詡盛夏多拘繫無辜,為吏人患。詡上書自訟曰:「法禁者俗之堤防,刑罰者人之銜轡。今州曰任郡,郡曰任縣,更相委遠,百姓怨窮,以苟容為賢,盡節為愚。臣所發舉,臧罪非一,二府恐為臣所奏,遂加誣罪。臣將從史魚死,即以尸諫耳。」順帝省其章,乃為免司空陶敦。

時中常侍張防特用權埶,每請託受取,詡輒案之,而屢寑不報。詡不勝其憤,乃自繫廷尉,奏言曰:「昔孝安皇帝任用樊豐,遂交亂嫡統,幾亡社稷。今者張防復弄威柄,國家之禍將重至矣。臣不忍與防同朝,謹自繫以聞,無令臣襲楊震之跡。」書奏,防流涕訴帝,詡坐論輸左校。防必欲害之,二日之中,傳考四獄。獄吏勸詡自引,詡曰:「寧伏歐刀以示遠近。」宦者孫程、張賢等知詡以忠獲罪,乃相率奏乞見。程曰:「陛下始與臣等造事之時,常疾姦臣,知其傾國。今者即位而復自為,何以非先帝乎?司隸校尉虞詡為陛下盡忠,而更被拘繫;常侍張防臧罪明正,反搆忠良。今客星守羽林,其占宮中有姦臣。宜急收防送獄,以塞天變。下詔出詡,還假印綬。」時防立在帝後,程乃叱防曰:「姦臣張防,何不下殿!」防不得已,趨就東箱。程曰:「陛下急收防,無令從阿母求請。」帝問諸尚書,尚書賈朗素與防善,證詡之罪。帝疑焉,謂程曰:「且出,吾方思之。」於是詡子顗與門生百餘人,舉幡候中常侍高梵車,叩頭流血,訴言枉狀。梵乃入言之,防坐徙邊,賈朗等六人或死或黜,即日赦出詡。程復上書陳詡有大功,語甚切激。帝感悟,復徵拜議郎。數日,遷尚書僕射。

是時長吏、二千石聽百姓謫罰者輸贖,號為「義錢」,託為貧人儲,而守令因以聚斂。詡上疏曰:「元年以來,貧百姓章言長吏受取百萬以上者,匈匈不絕,謫罰吏人至數千萬,而三公、刺史少所舉奏。尋永平、章和中,州郡以走卒錢給貸貧人,司空劾案,州及郡縣皆坐免黜。今宜遵前典,蠲除權制。」於是詔書下詡章,切責州郡。謫罰輸贖自此而止。

先是寧陽主簿詣闕,訴其縣令之枉,積六七歲不省。主簿乃上書曰:「臣為陛下子,陛下為臣父。臣章百上,終不見省,臣豈可北詣單于以告怨乎?」帝大怒,持章示尚書,尚書遂劾以大逆。詡駮之曰:「主簿所訟,乃君父之怨;百上不達,是有司之過。愚惷之人,不足多誅。」帝納詡言,笞之而已。詡因謂諸尚書曰:「小人有怨,不遠千里,斷髮刻肌,詣闕告訴,而不為理,豈臣下之義?君與濁長吏何親,而與怨人何仇乎?」聞者皆慚。詡又上言:「臺郎顯職,仕之通階。今或一郡七八,或一州無人。宜令均平,以厭天下之望。」及諸奏議,多見從用。

詡好刺舉,無所回容,數以此忤權戚,遂九見譴考,三遭刑罰,而剛正之性,終老不屈。永和初,遷尚書令,以公事去官。朝廷思其忠,復徵之,會卒。臨終,謂其子恭曰:「吾事君直道,行己無愧,所悔者為朝歌長時殺賊數百人,其中何能不有冤者。自此二十餘年,家門不增一口,斯獲罪於天也。」

恭有俊才,官至上黨太守。

傅燮字南容,北地靈州人也。本字幼起,慕南容三復白珪,乃易字焉。身長八尺,有威容。少師事太尉劉寬。再舉孝廉。聞所舉郡將喪,乃棄官行服。後為護軍司馬,與左中郎皇甫嵩俱討賊張角。

燮素疾中官,既行,因上疏曰:「臣聞天下之禍,不由於外,皆興於內。是故虞舜升朝,先除四凶,然後用十六相。明惡人不去,則善人無由進也。今張角起於趙、魏,黃巾亂於六州。此皆釁發蕭牆,而禍延四海者也。臣受戎任,奉辭伐罪,始到潁川,戰無不剋。黃巾雖盛,不足為廟堂憂也。臣之所懼,在於治水不自其源,末流彌增其廣耳。陛下仁德寬容,多所不忍,故閹豎弄權,忠臣不進。誠使張角梟夷,黃巾變服,臣之所憂,甫益深耳。何者?夫邪正之人不宜共國,亦猶冰炭不可同器。彼知正人之功顯,而危亡之兆見,皆將巧辭飾說,共長虛偽。夫孝子疑於屢至,巿虎成於三夫。若不詳察真偽,忠臣將復有杜郵之戮矣。陛下宜思虞舜四罪之舉,速行讒佞放殛之誅,則善人思進,姦凶自息。臣聞忠臣之事君,猶孝子之事父也。子之事父,焉得不盡其情?使臣身備鈇鉞之戮,陛下少用其言,國之福也。」書奏,宦者趙忠見而忿惡。及破張角,燮功多當封,忠訴譖之,靈帝猶識燮言,得不加罪,竟亦不封,以為安定都尉。以疾免。

後拜議郎。會西羌反,邊章、韓遂作亂隴右,徵發天下,役賦無已。司徒崔烈以為宜棄涼州。詔會公卿百官,烈堅執先議。燮厲言曰:「斬司徒,天下乃安。」尚書郎楊贊奏燮廷辱大臣。帝以問燮。燮對曰:「昔冒頓至逆也,樊噲為上將,願得十萬眾橫行匈奴中,憤激思奮,未失人臣之節,顧計當從與不耳,季布猶曰『噲可斬也』。今涼州天下要衝,國家藩衛。高祖初興,使酈商別定隴右;世宗拓境,列置四郡,議者以為斷匈奴右臂。今牧御失和,使一州叛逆,海內為之騷動,陛下臥不安寢。烈為宰相,不念為國思所以弭之之策,乃欲割棄一方萬里之土,臣竊惑之。若使左衽之虜得居此地,士勁甲堅,因以為亂,此天下之至慮,社稷之深憂也。若烈不知之,是極蔽也;知而故言,是不忠也。」帝從燮議。由是朝廷重其方格,每公卿有缺,為眾議所歸。

頃之,趙忠為車騎將軍,詔忠論討黃巾之功,執金吾甄舉等謂忠曰:「傅南容前在東軍,有功不侯,故天下失望。今將軍親當重任,宜進賢理屈,以副眾心。」忠納其言,遣弟城門校尉延致殷勤。延謂燮曰:「南容少荅我常侍,萬戶侯不足得也。」燮正色拒之曰:「遇與不遇,命也;有功不論,時也。傅燮豈求私賞哉!」忠愈懷恨,然憚其名,不敢害。權貴亦多疾之,是以不得留,出為漢陽太守。

初,郡將范津明知人,舉燮孝廉。及津為漢陽,與燮交代,合符而去,鄉邦榮之。津字文淵,南陽人。燮善卹人,叛羌懷其恩化,並來降附,乃廣開屯田,列置四十餘營。

時刺史耿鄙委任治中程球,球為通姦利,士人怨之。中平四年,鄙率六郡兵討金城賊王國、韓遂等。燮知鄙失眾,必敗,諫曰:「使君統政日淺,人未知教。孔子曰:『不教人戰,是謂棄之。』今率不習之人,越大隴之阻,將十舉十危,而賊聞大軍將至,必萬人一心。邊兵多勇,其鋒難當,而新合之眾,上下未和,萬一內變,雖悔無及。不若息軍養德,明賞必罰。賊得寬挺,必謂我怯,群惡爭埶,其離可必。然後率已教之人,討已離之賊,其功可坐而待也。今不為萬全之福,而就必危之禍,竊為使君不取。」鄙不從。行至狄道,果有反者,先殺程球,次害鄙,賊遂進圍漢陽。城中兵少糧盡,燮猶固守。

時北胡騎數千隨賊攻郡,皆夙懷燮恩,共於城外叩頭,求送燮歸鄉里。子幹年十三,從在官舍。知燮性剛,有高義,恐不能屈志以免,進諫曰:「國家昏亂,遂令大人不容於朝。今天下已叛,而兵不足自守,鄉里羌胡先被恩德,欲令棄郡而歸,願必許之。徐至鄉里,率厲義徒,見有道而輔之,以濟天下。」言未終,燮慨然而歎,呼幹小字曰:「別成,汝知吾必死邪?蓋『聖達節,次守節』。且殷紂之暴,伯夷不食周粟而死,仲尼稱其賢。今朝廷不甚殷紂,吾德亦豈絕伯夷?世亂不能養浩然之志,食祿又欲避其難乎?吾行何之,必死於此。汝有才智,勉之勉之。主簿楊會,吾之程嬰也。」幹哽咽不能復言,左右皆泣下。王國使故酒泉太守黃衍說燮曰:「成敗之事,已可知矣。先起,上有霸王之業,下成伊呂之勳。天下非復漢有,府君寧有意為吾屬師乎?」燮案劍叱衍曰:「若剖符之臣,反為賊說邪!」遂麾左右進兵,臨陣戰歿。謚曰壯節侯。

幹知名,位至扶風太守。

蓋勳字元固,敦煌廣至人也。家世二千石。初舉孝廉,為漢陽長史。時武威太守倚恃權埶,恣行貪橫,從事武都蘇正和案致其罪。涼州刺史梁鵠畏懼貴戚,欲殺正和以免其負,乃訪之於勳。勳素與正和有仇,或勸勳可因此報隙。勳曰:「不可。謀事殺良,非忠也;乘人之危,非仁也。」乃諫鵠曰:「夫紲食鷹鳶欲其鷙,鷙而亨之,將何用哉?」鵠從其言。正和喜於得免,而詣勳求謝。勳不見,曰:「吾為梁使君謀,不為蘇正和也。」怨之如初。

中平元年,北地羌胡與邊章等寇亂隴右,刺史左昌因軍興斷盜數千萬。勳固諫,昌怒,乃使勳別屯阿陽以拒賊鋒,欲因軍事罪之,而勳數有戰功。邊章等遂攻金城,殺郡守陳懿,勳勸昌救之,不從。邊章等進圍昌於冀,昌懼而召勳。勳初與從事辛曾、孔常俱屯阿陽,及昌檄到,曾等疑不肯赴。勳怒曰:「昔莊賈後期,穰苴奮劍。今之從事,豈重於古之監軍哉!」曾等懼而從之。勳即率兵救昌。到,乃誚讓章等,責以背叛之罪。皆曰:「左使君若早從君言,以兵臨我,庶可自改。今罪已重,不得降也。」乃解圍而去。昌坐斷盜徵,以扶風宋梟代之。梟患多寇叛,謂勳曰:「涼州寡於學術,故屢致反暴。今欲多寫孝經,令家家習之,庶或使人知義。」勳諫曰:「昔太公封齊,崔杼殺君;伯禽侯魯,慶父篡位。此二國豈乏學者?今不急靜難之術,遽為非常之事,既足結怨一州,又當取笑朝廷,勳不知其可也。」梟不從,遂奏行之。果被詔書詰責,坐以虛慢徵。時叛羌圍護羌校尉夏育於畜官,勳與州郡合兵救育,至狐槃,為羌所破。勳收餘眾百餘人,為魚麗之陳。羌精騎夾攻之急,士卒多死。勳被三創,堅不動,乃指木表曰:「必尸我於此。」句就種羌滇吾素為勳所厚,乃以兵扞眾曰:「蓋長史賢人,汝曹殺之者為負天。」勳仰罵曰:「死反虜,汝何知?促來殺我!」眾相視而驚。滇吾下馬與勳,勳不肯上,遂為賊所執。羌戎服其義勇,不敢加害,送還漢陽。後刺史楊雍即表勳領漢陽太守。時人飢,相漁食,勳調穀稟之,先出家糧以率眾,存活者千餘人。

後去官,徵拜討虜校尉。靈帝召見,問:「天下何苦而反亂如此?」勳曰:「倖臣子弟擾之。」時宦者上軍校尉蹇碩在坐,帝顧問碩,碩懼,不知所對,而以此恨勳。帝又謂勳曰:「吾已陳師於平樂觀,多出中藏財物以餌士,何如?」勳曰:「臣聞『先王燿德不觀兵。』今寇在遠而設近陳,不足昭果毅,秪黷武耳。」帝曰:「善。恨見君晚,群臣初無是言也。」

勳時與宗正劉虞、佐軍校尉袁紹同典禁兵。勳謂虞、紹曰:「吾仍見上,上甚聰明,但擁蔽於左右耳。若共併力誅嬖倖,然後徵拔英俊,以興漢室,功遂身退,豈不快乎!」虞、紹亦素有謀,因相連結,未及發,而司隸校尉張溫舉勳為京兆尹。帝方欲延接勳,而蹇碩等心憚之,並勸從溫奏,遂拜京兆尹。

時長安令楊黨,父為中常侍,恃埶貪放,勳案得其臧千餘萬。貴戚咸為之請,勳不聽,具以事聞,并連黨父,有詔窮案,威震京師。時小黃門京兆高望為尚藥監,倖於皇太子,太子因蹇碩屬望子進為孝廉,勳不肯用。或曰:「皇太子副主,望其所愛,碩帝之寵臣,而子違之,所謂三怨成府者也。」勳曰:「選賢所以報國也。非賢不舉,死亦何悔!」勳雖在外,每軍國密事,帝常手詔問之。數加賞賜,甚見親信,在朝臣右。

及帝崩,董卓廢少帝,殺何太后,勳與書曰:「昔伊尹、霍光權以立功,猶可寒心,足下小醜,何以終此?賀者在門,弔者在廬,可不慎哉!」卓得書,意甚憚之。徵為議郎。時左將軍皇甫嵩精兵三萬屯扶風,勳密相要結,將以討卓。會嵩亦被徵,勳以眾弱不能獨立,遂並還京師。自公卿以下,莫不卑下於卓,唯勳長揖爭禮,見者皆為失色。卓問司徒王允曰:「欲得快司隸校尉,誰可作者?」允曰:「唯有蓋京兆耳。」卓曰:「此人明智有餘,然不可假以雄職。」乃以為越騎校尉。卓又不欲令久典禁兵,復出為潁川太守。未及至郡,徵還京師。時河南尹朱雋為卓陳軍事。卓折雋曰:「我百戰百勝,決之於心,卿勿妄說,且汙我刀。」勳曰:「昔武丁之明,猶求箴諫,況如卿者,而欲杜人之口乎?」卓曰:「戲之耳。」勳曰:「不聞怒言可以為戲?」卓乃謝雋。勳雖強直不屈,而內厭於卓,不得意,疽發背卒,時年五十一。遺令勿受卓賻贈。卓欲外示寬容,表賜東園祕器賵襚,送之如禮。葬于安陵。

子順,官至永陽太守。

臧洪字子源,廣陵射陽人也。父旻,有幹事才。熹平元年,會稽妖賊許昭起兵句章,自稱「大將軍」,立其父生為越王,攻破城邑,眾以萬數。拜旻揚州刺史。旻率丹揚太守陳夤擊昭,破之。昭遂復更屯結,大為人患。旻等進兵,連戰三年,破平之,獲昭父子,斬首數千級。遷旻為使匈奴中郎將。

洪年十五,以父功拜童子郎,知名太學。洪體貌魁梧,有異姿。舉孝廉,補即丘長。

中平末,棄官還家,太守張超請為功曹。時董卓殺帝,圖危社稷。洪說超曰:「明府歷世受恩,兄弟並據大郡。今王室將危,賊臣虎視,此誠義士效命之秋也。今郡境尚全,吏人殷富,若動桴鼓,可得二萬人。以此誅除國賊,為天下唱義,不亦宜乎!」超然其言,與洪西至陳留,見兄邈計事。邈先謂超曰:「聞弟為郡,委攻臧洪,洪者何如人?」超曰:「臧洪海內奇士,才略智數不比於超矣。」邈即引洪與語,大異之。乃使詣兗州刺史劉岱、豫州刺史孔骸,遂皆相善。邈既先有謀約,會超至,定議,乃與諸牧守大會酸棗。設壇場,將盟,既而更相辭讓,莫敢先登,咸共推洪。洪乃攝衣升壇,操血而盟曰:「漢室不幸,皇綱失統,賊臣董卓,乘釁縱害,禍加至尊,毒流百姓。大懼淪喪社稷,翦覆四海。兗州刺史岱、豫州刺史骸、陳留太守邈、東郡太守瑁、廣陵太守超等,糾合義兵,並赴國難。凡我同盟,齊心一力,以致臣節,隕首喪元,必無二志。有渝此盟,俾墜其命,無克遺育。皇天后土,祖宗明靈,實皆鑒之。」洪辭氣慷慨,聞其言者,無不激揚。自是之後,諸軍各懷遲疑,莫適先進,遂使糧儲單竭,兵眾乖散。

時討虜校尉公孫瓚與大司馬劉虞有隙,超乃遣洪詣虞,共謀其難。行至河閒而值幽冀交兵,行塗阻絕,因寓於袁紹。紹見洪,甚奇之,與結友好,以洪領青州刺史。前刺史焦和好立虛譽,能清談。時黃巾群盜處處飆起,而青部殷實,軍革尚眾。和欲與諸同盟西赴京師,未及得行,而賊已屠城邑。和不理戎警,但坐列巫史,禜禱群神。又恐賊乘凍而過,命多作陷冰丸,以投于河。眾遂潰散,和亦病卒。洪收撫離叛,百姓復安。

在事二年,袁紹憚其能,徙為東郡太守,都東武陽。時曹操圍張超於雍丘,甚危急。超謂軍吏曰:「今日之事,唯有臧洪必來救我。」或曰:「袁曹方穆,而洪為紹所用,恐不能敗好遠來,違福取禍。」超曰:「子源天下義士,終非背本者也,或見制強力,不相及耳。」洪始聞超圍,及徒跣號泣,並勒所領,將赴其難。自以眾弱,從紹請兵,而紹竟不聽之,超城遂陷,張氏族滅。洪由是怨紹,絕不與通。紹興兵圍之,歷年不下,使洪邑人陳琳以書譬洪,示其禍福,責以恩義。洪荅曰:

隔闊相思,發於寤寐。相去步武,而趨舍異規,其為愴恨,胡可勝言!前日不遺,比辱雅況,述敘禍福,公私切至。以子之才,窮該典籍,豈將闇於大道,不達余趣哉?是以損棄翰墨,一無所酬,亦冀遙忖褊心,粗識鄙性。重獲來命,援引紛紜,雖欲無對,而義篤其言。

僕小人也,本乏志用,中因行役,特蒙傾蓋,恩深分厚,遂竊大州,寧樂今日自還接刃乎?每登城臨兵,觀主人之旗鼓,瞻望帳幄,感故友之周旋,撫弦搦矢,不覺涕流之覆面也。何者?自以輔佐主人,無以為悔;主人相接,過絕等倫。受任之初,志同大事,埽清寇逆,共尊王室。豈悟本州被侵,郡將遘厄,請師見拒,辭行被拘,使洪故君,遂至淪滅。區區微節,無所獲申,豈得復全交友之道,重虧忠孝之名乎?所以忍悲揮戈,收淚告絕。若使主人少垂古人忠恕之情,來者側席,去者克己,則僕抗季札之志,不為今日之戰矣。

昔張景明登壇喢血,奉辭奔走,卒使韓牧讓印,主人得地。後但以拜章朝主,賜爵獲傳之故,不蒙觀過之貸,而受夷滅之禍。呂奉先討卓來奔,請兵不獲,告去何罪,復見斫刺。劉子璜奉使踰時,辭不獲命,畏君懷親,以詐求歸,可謂有志忠孝,無損霸道,亦復僵尸麾下,不蒙虧除。慕進者蒙榮,違意者被戮,此乃主人之利,非遊士之願也。是以鑒戒前人,守死窮城,亦以君子之違,不適敵國故也。

足下當見久圍不解,救兵未至,感婚姻之義,推平生之好,以為屈節而苟生,勝守義而傾覆也。昔晏嬰不降志於白刃,南史不曲筆以求存,故身傳圖象,名垂後世。況僕據金城之固,驅士人之力,散三年之畜以為一年之資,匡困補乏,以悅天下,何圖築室反耕哉?但懼秋風揚塵,伯珪馬首南向,張揚、飛燕旅力作難,北鄙將告倒懸之急,股肱奏乞歸之記耳。主人當鑒戒曹輩,反旌退師,何宜久辱盛怒,暴威於吾城之下哉!

足下譏吾恃黑山以為救,獨不念黃巾之合從邪?昔高袓取彭越於鉅野,光武創基兆於綠林,卒能龍飛受命,中興帝業。苟可輔主興化,夫何嫌哉!況僕親奉璽書,與之從事!

行矣孔璋!足下徼利於境外,臧洪投命於君親;吾子託身於盟主,臧洪策名於長安。子謂余身死而名滅,僕亦笑子生死而無聞焉。本同末離,努力努力,夫復何言!

紹見洪書,知無降意,增兵急攻。城中糧盡,外無援救,洪自度不免,呼吏士謂曰:「袁紹無道,所圖不軌,且不救洪郡將,洪於大義,不得不死。念諸君無事,空與此禍,可先城未破,將妻子出。」將吏皆垂泣曰:「明府之於袁氏,本無怨隙,今為郡將之故,自致危困,吏人何忍當捨明府去也?」初尚掘鼠,煮筋角,後無所復食,主簿啟內廚米三斗,請稍為饘粥,洪曰:「何能獨甘此邪?」使為薄糜,遍班士眾。又殺其愛妾,以食兵將。兵將咸流涕,無能仰視。男女七八十人相枕而死,莫有離叛。

城陷,生執洪。紹盛帷幔,大會諸將見洪。謂曰:「臧洪何相負若是!今日服未?」洪據地瞋目曰:「諸袁事漢,四世五公,可謂受恩。今王室衰弱,無扶翼之意,而欲因際會,觖望非冀,多殺忠良,以立姦威。洪親見將軍呼張陳留為兄,則洪府君亦宜為弟,而不能同心戮力,為國除害,坐擁兵眾,觀人屠滅。惜洪力劣,不能推刃為天下報仇,何謂服乎?」紹本愛洪,意欲屈服赦之,見其辭切,知終不為用,乃命殺焉。

洪邑人陳容,少為諸生,親慕於洪,隨為東郡丞。先城未敗,洪使歸紹。時容在坐,見洪當死,起謂紹曰:「將軍舉大事,欲為天下除暴,而專先誅忠義,豈合天意?臧洪發舉為郡將,柰何殺之!」紹慚,使人牽出,謂曰:「汝非臧洪疇,空復爾為?」容顧曰:「夫仁義豈有常所,蹈之則君子,背之則小人。今日寧與臧洪同日死,不與將軍同日生也。」遂復見殺。在紹坐者,無不歎息,竊相謂曰:「如何一日戮二烈士!」

先是洪遣司馬二人出,求救於呂布。比還,城已陷,皆赴敵死。

論曰:雍丘之圍,臧洪之感憤壯矣!想其行跣且號,束甲請舉,誠足憐也。夫豪雄之所趣舍,其與守義之心異乎?若乃締謀連衡,懷詐筭以相尚者,蓋惟利埶所在而已。況偏城既危,曹袁方穆,洪徒指外敵之衡,以紓倒縣之會。忿悁之師,兵家所忌。可謂懷哭秦之節,存荊則未聞也。

贊曰:先零擾疆,鄧、崔棄涼。詡、燮令圖,再全金方。蓋勳抗董,終然允剛。洪懷偏節,力屈志揚。


\end{pinyinscope}