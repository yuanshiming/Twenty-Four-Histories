\article{吳延史盧趙列傳}

\begin{pinyinscope}
吳祐字季英,陳留長垣人也。父恢,為南海太守。祐年十二,隨從到官。恢欲殺青簡以寫經書,祐諫曰:「今大人踰越五領,遠在海濱,其俗誠陋,然舊多珍怪,上為國家所疑,下為權戚所望。此書若成,則載之兼兩。昔馬援以薏苡興謗,王陽以衣囊徼名。嫌疑之閒,誠先賢所慎也。」恢乃止,撫其首曰:「吳氏世不乏季子矣。」及年二十,喪父,居無檐石,而不受贍遺。常牧豕於長垣澤中,行吟經書。遇父故人,謂曰:「卿二千石子而自業賤事,縱子無恥,柰先君何?」祐辭謝而已,守志如初。

後舉孝廉,將行,郡中為祖道,祐越壇共小史雍丘黃真歡語移時,與結友而別。功曹以祐倨,請黜之。太守曰:「吳季英有知人之明,卿且勿言。」真後亦舉孝廉,除新蔡長,世稱其清節。時公沙穆來遊太學,無資糧,乃變服客傭,為祐賃舂。祐與語大驚,遂共定交於杵臼之閒。

祐以光祿四行遷膠東侯相。時濟北戴宏父為縣丞,宏年十六,從在丞舍。祐每行園,常聞諷誦之音,奇而厚之,亦與為友,卒成儒宗,知名東夏,官至酒泉太守。祐政唯仁簡,以身率物。民有爭訴者,輒閉閤自責,然後斷其訟,以道譬之。或身到閭里,重相和解。自是之後,爭隙省息,吏人懷而不欺。嗇夫孫性私賦民錢,市衣以進其父,父得而怒曰:「有君如是,何忍欺之!」促歸伏罪。性慚懼,詣閤持衣自首。祐屏左右問其故,性具談父言。祐曰:「掾以親故,受污穢之名,所謂『觀過斯知人矣』。」使歸謝其父,還以衣遺之。又安丘男子毋丘長與母俱行市,道遇醉客辱其母,長殺之而亡,安丘追蹤於膠東得之。祐呼長謂曰:「子母見辱,人情所恥。然孝子忿必慮難,動不累親。今若背親逞怒,白日殺人,赦若非義,刑若不忍,將如之何?」長以械自繫,曰:「國家制法,囚身犯之。明府雖加哀矜,恩無所施。」祐問長有妻子乎?對曰:「有妻未有子也。」即移安丘逮長妻,妻到,解其桎梏,使同宿獄中,妻遂懷孕。至冬盡行刑,長泣謂母曰:「負母應死,當何以報吳君乎?」乃齧指而吞之,含血言曰:「妻若生子,名之『吳生』,言我臨死吞指為誓,屬兒以報吳君。」因投繯而死。

祐在膠東九年,遷齊相,大將軍梁冀表為長史。及冀誣奏太尉李固,祐聞而請見,與冀爭之,不聽。時扶風馬融在坐,為冀章草,祐因謂融曰:「李公之罪,成於卿手。李公即誅,卿何面目見天下之人乎?」冀怒而起入室,祐亦徑去。冀道出祐為河閒相,因自免歸家,不復仕,躬灌園蔬,以經書教授。年九十八卒。

長子鳳,官至樂浪太守,少子愷,新息令;鳳子馮,鮦陽侯相:皆有名於世。

延篤字叔堅,南陽犨人也。少從潁川唐溪典受左氏傳,旬日能諷之,典深敬焉。又從馬融受業,博通經傳及百家之言,能著文章,有名京師。

舉孝廉,為平陽侯相。到官,表龔遂之墓,立銘祭祠,擢用其後於畎畝之閒。以師喪棄官奔赴,五府並辟不就。

桓帝以博士徵,拜議郎,與朱穆、邊韶共著作東觀。稍遷侍中。帝數問政事,篤詭辭密對,動依典義。遷左馮翊,又徙京兆尹。其政用寬仁,憂恤民黎,擢用長者,與參政事,郡中歡愛,三輔咨嗟焉。先是陳留邊鳳為京兆尹,亦有能名,郡人為之語曰:「前有趙張三王,後有邊延二君。」

時皇子有疾,下郡縣出珍藥,而大將軍梁冀遣客齎書詣京兆,并貨牛黃。篤發書收客,曰:「大將軍椒房外家,而皇子有疾,必應陳進醫方,豈當使客千里求利乎?」遂殺之。冀慚而不得言,有司承旨欲求其事。篤以病免歸,教授家巷。

時人或疑仁孝前後之證,篤乃論之曰:「觀夫仁孝之辯,紛然異端,互引典文,代取事據,可謂篤論矣。夫人二致同源,總率百行,非復銖兩輕重,必定前後之數也。而如欲分其大較,體而名之,則孝在事親,仁施品物。施物則功濟於時,事親則德歸於己。於己則事寡,濟時則功多。推此以言,仁則遠矣。然物有出微而著,事有由隱而章。近取諸身,刵耳有聽受之用,目有察見之明,足有致遠之勞,手有飾衛之功,功雖顯外,本之者心也。遠取諸物,則草木之生,始於萌牙,終於彌蔓,枝葉扶疏,榮華紛縟,末雖繁蔚,致之者根也。夫仁人之有孝,猶四體之有心腹,枝葉之有本根也。聖人知之,故曰:『夫孝,天之經也,地之義也,人之行也。』『君子務本,本立而道生。孝悌也者,其為仁之本與!』然體大難備,物性好偏,故所施不同,事少兩兼者也。如必對其優劣,則仁以枝葉扶疏為大,孝以心體本根為先,可無訟也。或謂先孝後仁,非仲尼序回、參之意。蓋以為仁孝同質而生,純體之者,則互以為稱,虞舜、顏回是也。若偏而體之,則各有其目,公劉、曾參是也。夫曾、閔以孝悌為至德,管仲以九合為仁功,未有論德不先回、參,考功不大夷吾。以此而言,各從其稱者也。」

前越巂太守李文德素善於篤,時在京師,謂公卿曰:「延叔堅有王佐之才,奈何屈千里之足乎?」欲令引進之。篤聞,乃為書止文德曰:「夫道之將廢,所謂命也。流聞乃欲相為求還東觀,來命雖篤,所未敢當。吾嘗昧爽櫛梳,坐於客堂。朝則誦羲、文之易,虞、夏之書,歷公旦之典禮,覽仲尼之春秋。夕則消搖內階,詠詩南軒。百家眾氏,投閒而作。洋洋乎其盈耳也,渙爛兮其溢目也,紛紛欣欣兮其獨樂也。當此之時,不知天之為蓋,地之為輿;不知世之有人,己之有軀也。雖漸離擊筑,傍若無人,高鳳讀書,不知暴雨,方之於吾,未足況也。且吾自束脩已來,為人臣不陷於不忠,為人子不陷於不孝,上交不諂,下交不黷,從此而歿,下見先君遠祖,可不慚赧。如此而不以善止者,恐如教羿射者也。慎勿迷其本,棄其生也。」

後遭黨事禁錮。永康元年,卒于家。鄉里圖其形于屈原之廟。

篤論解經傳,多所駮正,後儒服虔等以為折中。所著詩、論、銘、書、應訊、表、教令,凡二十篇云。

史弼字公謙,陳留考城人也。父敞,順帝時以佞辯至尚書、郡守。弼少篤學,聚徒數百。仕州郡,辟公府,遷北軍中候。

是時桓帝弟渤海王悝素行險辟,僭傲多不法。弼懼其驕悖為亂,乃上封事曰:「臣聞帝王之於親戚,愛雖隆,必示之以威;體雖貴,必禁之以度。如是,和睦之道興,骨肉之恩遂。昔周襄王恣甘昭公,孝景皇帝驕梁孝王,而二弟階寵,終用绗慢,卒周有播蕩之禍,漢有爰盎之變。竊聞勃海王悝,憑至親之屬,恃偏私之愛,失奉上之節,有僭慢之心,外聚剽輕不逞之徒,內荒酒樂,出入無常,所與群居,皆有口無行,或家之棄子,或朝之斥臣,必有羊勝、伍被之變。州司不敢彈糾,傅相不能匡輔。陛下隆於友于,不忍遏絕。恐遂滋蔓,為害彌大。乞露臣奏,宣示百僚,使臣得於清朝明言其失,然後詔公卿平處其法。法決罪定,乃下不忍之詔。臣下固執,然後少有所許。如是,則聖朝無傷親之譏,勃海有享國之慶。不然,懼大獄將興,使者相望於路矣。臣職典禁兵,備禦非常,而妄知藩國,干犯至戚,罪不容誅。不勝憤懣,謹冒死以聞。」帝以至親,不忍下其事。後悝竟坐逆謀,貶為癭陶王。

弼遷尚書,出為平原相。時詔書下舉鉤黨,郡國所奏相連及者多至數百,唯弼獨無所上。詔書前後切卻州郡,髡笞掾史。從事坐傳責曰:「詔書疾惡黨人,旨意懇惻。青州六郡,其五有黨,近國甘陵,亦考南北部,平原何理而得獨無?」弼曰:「先王疆理天下,畫界分境,水土異齊,風俗不同,它郡自有,平原自無,胡可相比?若承望上司,誣陷良善,淫刑濫罰,以逞非理,則平原之人,戶可為黨。相有死而已,所不能也。」從事大怒,即收郡僚職送獄,道舉奏弼。會黨禁中解,弼以俸贖罪得免,濟活者千餘人。

弼為政特挫抑彊豪,其小民有罪,多所容貸。遷河東太守,被一切詔書當舉孝廉。弼知多權貴請託,乃豫敕斷絕書屬。中常侍侯覽果遣諸生齎書請之,并求假鹽稅,積日不得通。生乃說以它事謁弼,而因達覽書。弼大怒曰:「太守忝荷重任,當選士報國,爾何人而偽詐無狀!」命左右引出,楚捶數百,府丞、掾史十餘人皆諫於廷,弼不對。遂付安邑獄,即日考殺之。侯覽大怨,遂詐作飛章下司隸,誣弼誹謗,檻車徵。吏人莫敢近者,唯前孝廉裴瑜送到崤澠之閒,大言於道傍曰:「明府摧折虐臣,選德報國,如其獲罪,足以垂名竹帛,願不憂不懼。」弼曰:「『誰謂荼苦,其甘如薺。』昔人刎頸,九死不恨。」及下廷尉詔獄,平原吏人奔走詣闕訟之。又前孝廉魏劭毀變形服,詐為家僮,瞻護於弼。弼遂受誣,事當棄市。劭與郡人賣郡邸,行賂於侯覽,得減死罪一等,論輸左校。時人或譏曰:「平原行貨以免君,無乃蚩乎!」陶丘洪曰:「昔文王牖里,閎、散懷金。史弼遭患,義夫獻寶。亦何疑焉!」於是議者乃息。刑竟歸田里,稱病閉門不出。數為公卿所薦,議郎何休又訟弼有幹國之器,宜登台相,徵拜議郎。侯覽等惡之。光和中,出為彭城相,會病卒。裴瑜位至尚書。

論曰:夫剛烈表性,鮮能優寬;仁柔用情,多乏貞直。吳季英視人畏傷,發言烝烝,似夫儒者;而懷憤激揚,折讓權枉,又何壯也!仁以矜物,義以退身,君子哉!語曰:「活千人者子孫必封。」史弼頡頏嚴吏,終全平原之黨,而其後不大,斯亦未可論也。

盧植字子幹,涿郡涿人也。身長八尺二寸,音聲如鍾。少與鄭玄俱事馬融,能通古今學,好研精而不守章句。融外戚豪家,多列女倡歌舞於前。植侍講積年,未嘗轉眄,融以是敬之。學終辭歸,闔門教授。性剛毅有大節,常懷濟世志,不好辭賦,能飲酒一石。

時皇后父大將軍竇武援立靈帝,初秉機政,朝議欲加封爵。植雖布衣,以武素有名譽,乃獻書以規之曰:「植聞嫠有不恤緯之事,漆室有倚楹之戚,憂深思遠,君子之情。夫士立爭友,義貴切磋。書陳『謀及庶人』,詩詠『詢于芻蕘』。植誦先王之書久矣,敢愛其瞽言哉!今足下之於漢朝,猶旦、奭之在周室,建立聖主,四海有繫。論者以為吾子之功,於斯為重。天下聚目而視,攢耳而聽,謂準之前事,將有景風之祚。尋春秋之義,王后無嗣,擇立親長,年均以德,德均則決之卜筮。今同宗相後,披圖案牒,以次建之,何勳之有?豈橫叨天功以為己力乎!宜辭大賞,以全身名。又比世祚不競,仍外求嗣,可謂危矣。而四方未寧,盜賊伺隙,恆岳、勃碣,特多姦盜,將有楚人脅比,尹氏立朝之變。宜依古禮,置諸子之官,徵王侯愛子,宗室賢才,外崇訓道之義,內息貪利之心,簡其良能,隨用爵之,彊幹弱枝之道也。」武並不能用。州郡數命,植皆不就。建寧中,徵為博士,乃始起焉。熹平四年,九江蠻反,四府選植才兼文武,拜九江太守,蠻寇賓服。以疾去官。

作尚書章句、三禮解詁。時始立太學石經,以正五經文字,植乃上書曰:「臣少從通儒故南郡太守馬融受古學,頗知今之禮記特多回冗。臣前以周禮諸經,發起秕謬,敢率愚淺,為之解詁,而家乏,無力供繕上。願得將書生二人,共詣東觀,就官財糧,專心研精,合尚書章句,考禮記失得,庶裁定聖典,刊正碑文。古文科斗,近於為實,而厭抑流俗,降在小學。中興以來,通儒達士班固、賈逵、鄭興父子,並敦悅之。今毛詩、左氏、周禮各有傳記,其興春秋共相表裏,宜置博士,為立學官,以助後來,以廣聖意。」

會南夷反叛,以植嘗在九江有恩信,拜為廬江太守。植深達政宜,務存清靜,弘大體而已。

歲餘,復徵拜議郎,與諫議大夫馬日磾、議郎蔡邕、楊彪、韓說等並在東觀,校中書五經記傳,補續漢記。帝以非急務,轉為侍中,遷尚書。光和元年,有日食之異,植上封事諫曰:「臣聞五行傳『日晦而月見謂之朓,王侯其舒』。此謂君政舒緩,故日食晦也。春秋傳曰『天子避位移時』,言其相掩不過移時。而閒者日食自巳過午,既食之後,雲霧晻曖。比年地震,彗孛互見。臣聞漢以火德,化當寬明。近色信讒,忌之甚者,如火畏水故也。案今年之變,皆陽失陰侵,消禦災凶,宜有其道。謹略陳八事:一曰用良,二曰原禁,三曰禦癘,四曰備寇,五曰修禮,六曰遵堯,七曰御下,八曰散利。用良者,宜使州郡覈舉賢良,隨方委用,責求選舉。原禁者,凡諸黨錮,多非其罪,可加赦恕,申宥回枉。禦癘者,宋后家屬,並以無辜委骸橫尸,不得收葬,疫癘之來,皆由於此。宜敕收拾,以安遊魂。備寇者,侯王之家,賦稅減削,愁窮思亂,必致非常,宜使給足,以防未然。脩禮者,應徵有道之人,若鄭玄之徒,陳明洪範,攘服災咎。遵堯者,今郡守刺史一月數遷,宜依黜陟,以章能否,縱不九載,可滿三歲。御下者,謂謁希爵,一宜禁塞,遷舉之事,責成主者。散利者,天子之體,理無私積,宜弘大務,蠲略細微。」帝不省。

中平元年,黃巾賊起,四府舉植,拜北中郎將,持節,以護烏桓中郎將宗員副,將北軍五校士,發天下諸郡兵征之。連戰破賊帥張角,斬獲萬餘人。角等走保廣宗,植築圍鑿塹,造作雲梯,垂當拔之。帝遣小黃門左豐詣軍觀賊形埶,或勸植以賂送豐,植不肯。豐還言於帝曰:「廣宗賊易破耳。盧中郎固壘息軍,以待天誅。」帝怒,遂檻車徵植,減死罪一等。及車騎將車皇甫嵩討平黃巾,盛稱植行師方略,嵩皆資用規謀,濟成其功。以其年復為尚書。

帝崩,大將軍何進謀誅中官,乃召并州牧董卓,以懼太后。植知卓凶悍難制,必生後患,固止之。進不從。及卓至,果陵虐朝廷,乃大會百官於朝堂,議欲廢立。群僚無敢言,植獨抗議不同。卓怒罷會,將誅植,語在卓傳。植素善蔡邕,邕前徙朔方,植獨上書請之。邕時見親於卓,故往請植事。又議郎彭伯諫卓曰:「盧尚書海內大儒,人之望也。今先害,天下震怖。」卓乃止,但免植官而已。

植以老病求歸,懼不免禍,乃詭道從轘轅出。卓果使人追之,到懷,不及。遂隱於上谷,不交人事。冀州牧袁紹請為軍師。初平三年卒。臨困,敕其子儉葬於土穴,不用棺槨,附體單帛而已。所著碑、誄、表、記凡六篇。

建安中,曹操北討柳城,過涿郡,告守令曰:「故北中郎將盧植,名著海內,學為儒宗,士之楷模,國之楨幹也。昔武王入殷,封商容之閭;鄭喪子產,仲尼隕涕。孤到此州,嘉其餘風。春秋之義,賢者之後,宜有殊禮。亟遣丞掾除其墳墓,存其子孫,并致薄醊,以彰厥德。」子毓,知名。

論曰:風霜以別草木之性,危亂而見貞良之節,則盧公之心可知矣。夫螽蠆起懷,雷霆駭耳,雖賁、育、荊、諸之倫,未有不冘豫奪常者也。當植抽白刃嚴閤之下,追帝河津之閒,排戈刃,赴戕折,豈先計哉?君子之於忠義,造次必於是,顛沛必於是也。

趙岐字邠卿,京兆長陵人也。初名嘉,生於御史臺,因字臺卿,後避難,故自改名字,示不忘本土也。岐少明經,有才蓺,娶扶風馬融兄女。融外戚豪家,岐常鄙之,不與融相見。仕州郡,以廉直疾惡見憚。年三十餘,有重疾,臥蓐七年,自慮奄忽,乃為遺令敕兄子曰:「大丈夫生世,遯無箕山之操,仕無伊、呂之勳,天不我與,復何言哉!可立一員石於吾墓前,刻之曰:『漢有逸人,姓趙名嘉。有志無時,命也柰何!』」其後疾瘳。

永興二年,辟司空掾,議二千石得去官為親行服,朝廷從之。其後為大將軍梁冀所辟,為陳損益求賢之策,冀不納。舉理劇,為皮氏長。會河東太守劉祐去郡,而中常侍左悺兄勝代之,岐恥疾宦官,即日西歸。京兆尹延篤復以為功曹。

先是中常侍唐衡兄玹為京兆虎牙都尉,郡人以玹進不由德,皆輕侮之。岐及從兄襲又數為貶議,玹深毒恨。延熹元年,玹為京兆尹,岐懼禍及,乃與從子戩逃避之。玹果收岐家屬宗親,陷以重法,盡殺之。岐遂逃難四方,江、淮、海、岱,靡所不歷。自匿姓名,賣餅北海市中。時安丘孫嵩年二十餘,遊市見岐,察非常人,停車呼與共載。岐懼失色,嵩乃下帷,令騎屏行人。密問岐曰:「視子非賣餅者,又相問而色動,不有重怨,即亡命乎?我北海孫賓石,闔門百口,埶能相濟。」岐素聞嵩名,即以實告之,遂以俱歸。嵩先入白母曰:「出行,乃得死友。」迎入上堂,饗之極歡。藏岐複壁中數年,岐作厄屯歌二十三章。

後諸唐死滅,因赦乃出。三府聞之,同時並辟。九年,乃應司徒胡廣之命。會南匈奴、烏桓、鮮卑反叛,公卿舉岐,擢拜并州刺史。岐欲奏守邊之策,未及上,會坐黨事免,因撰次以為禦寇論。

靈帝初,復遭黨錮十餘歲。中平元年,四方兵起,詔選故刺史、二千石有文武才用者,徵岐拜議郎。車騎將軍張溫西征關中,請補長史,別屯安定。大將軍何進舉為敦煌太守,行至襄武,岐與新除諸郡太守數人俱為賊邊章等所執。賊欲脅以為帥,岐詭辭得免,展轉還長安。

及獻帝西都,復拜議郎,稍遷太僕。及李傕專政,使太傅馬日磾撫慰天下,以岐為副。日磾行至洛陽,表別遣岐宣揚國命,所到郡縣,百姓皆喜曰:「今日乃復見使者車騎。」

是時袁紹、曹操與公孫瓚爭冀州,紹及操聞岐至,皆自將兵數百里奉迎,岐深陳天子恩德,宜罷兵安人之道,又移書公孫瓚,為言利害。紹等各引兵去,皆與岐期會洛陽,奉迎車駕。岐南到陳留,得篤疾,經涉二年,期者遂不至。

興平元年,詔書徵岐,會帝當還洛陽,先遣衛將軍董承修理宮室。岐謂承曰:「今海內分崩,唯有荊州境廣地勝,西通巴蜀,南當交阯,年穀獨登,兵人差全。岐雖迫大命,猶志報國家,欲自乘牛車,南說劉表,可使其身自將兵來衛朝廷,與將軍并心同力,共獎王室。此安上救人之策也。」承即表遣岐使荊州,督租糧。岐至,劉表即遣兵詣洛陽助修宮室,軍資委輸,前後不絕。時孫嵩亦寓於表,表不為禮,岐乃稱嵩素行篤烈,因共上為青州刺史。岐以老病,遂留荊州。

曹操時為司空,舉以自代。光祿勳桓典、少府孔融上書薦之,於是就拜岐為太常。年九十餘,建安六年卒。先自為壽藏,圖季札、子產、晏嬰、叔向四像居賓位,又自畫其像居主位,皆為讚頌。敕其子曰:「我死之日,墓中聚沙為床,布簟白衣,散髮其上,覆以單被,即日便下,下訖便掩。」岐多所述作,蓋孟子章句、三輔決錄傳於時。

贊曰:吳翁溫愛,義干剛烈。延、史字人,風和恩結。梁使顯刑,誣黨潛絕。子幹兼姿,逢掖臨師。邠卿出疆,專出朝威。


\end{pinyinscope}