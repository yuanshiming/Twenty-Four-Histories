\article{五行二}

\begin{pinyinscope}
災火草妖羽蟲孽羊禍

五行傳曰:「棄法律,逐功臣,殺太子,以妾為妻,則火不炎上。」謂火失其性而為災也。又曰:「視之不明,是謂不悊。厥咎舒,厥罰常燠,厥極疾。時則有草妖,時則有蠃蟲之孽,時則有羊禍時則有赤眚、赤祥,惟水沴火。」蠃蟲,劉歆傳以為羽蟲。

建武中,漁陽太守彭寵被徵。書至,明日潞縣火,災起城中,飛出城外,燔千餘家,殺人。京房易傳曰:「上不儉,下不節,盛火數起,燔宮室。」儒說火以明為德而主禮。時寵與幽州牧朱浮有隙,疑浮見浸譖,故意狐疑,其妻勸無應徵,遂反叛攻浮,卒誅滅。

和帝永元八年十二月丁巳,南宮宣室殿火。是時和帝幸北宮,竇太后在南宮。明年,竇太后崩。

十三年八月己亥,北宮盛饌門閤火。是時和帝幸鄧貴人,陰后寵衰怨恨,上有欲廢之意。明年,會得陰后挾偽道事,遂廢遷于桐宮,以憂死,立鄧貴人為皇后。

十五年六月辛酉,漢中城固南城門災。此孝和皇帝將絕世之象也。其後二年,宮車晏駕,殤帝及平原王皆早夭折,和帝世絕。

安帝永初二年四月甲寅,漢陽河陽城中失火,燒殺三千五百七十人。先是和帝崩,有皇子二人,皇子勝長,鄧皇后貪殤帝少,欲自養長立之。延平元年,殤帝崩。勝有厥疾不篤,群臣咸欲立之,太后以前既不立勝,遂更立清河王子,是為安帝。司空周章等心不掩服,謀欲誅鄧氏,廢太后、安帝,而更立勝。元年十一月,事覺,章等被誅。其後涼州叛羌為害大甚,涼州諸郡寄治馮翊、扶風界。及太后崩,鄧氏被誅。

四年三月戊子,杜陵園火。

元初四年二月壬戌,武庫火。是時羌叛,大為寇害,發天下兵以攻禦之,積十餘年未已,天下厭苦兵役。

延光元年八月戊子,陽陵園寢殿火。凡災發于先陵,此太子將廢之象也。若曰:不當廢太子以自翦,如火不當害先陵之寢也。明年,上以讒言廢皇太子為濟陰王。後二年,宮車晏駕。中黃門孫程等十九人起兵殿省,誅賊臣,立濟陰王。

四年秋七月乙丑,漁陽城門樓災。

順帝永建三年七月丁酉,茂陵園寢災。

陽嘉元年,恭陵廡災,及東西莫府火。太尉李固以為奢僭所致。陵之初造,禍及枯骨,規廣治之尤飾。又上欲更造宮室,益臺觀,故火起莫府,燒材木。

永和元年十月丁未,承福殿火。先是爵號阿母宋娥為山陽君;后父梁商本國侯,又多益商封;商長子冀當繼商爵,以商生在,復更封冀為襄邑侯;追號后母為開封君:皆過差非禮。

漢安元年三月甲午,雒陽劉漢等百九十七家為火所燒,後四年,宮車比三晏駕,建和元年君位乃定。

桓帝建和二年五月癸丑,北宮掖庭中德陽殿火,及左掖門。先是梁太后兄冀挾姦枉,以故太尉李固、杜喬正直,恐害其事,令人誣奏固、喬而誅滅之。是後梁太后崩,而梁氏誅滅。

延熹四年正月辛酉,南宮嘉德殿火。戊子,丙署火。二月壬辰,武庫火。五月丁卯,原陵長壽門火。先是亳后因賤人得幸,號貴人,為后。上以后母宣為長安君,封其兄弟,愛寵隆崇,又多封無功者。去年春,白馬令李雲坐直諫死。至此彗除心、尾,火連作。

五年正月壬午,南宮丙署火。四月乙丑,恭北陵東闕火。戊辰,虎賁掖門火。五月,康陵園寢火。甲申,中藏府承祿署火。七月己未,南宮承善闥內火。

六年四月辛亥,康陵東署火。七月甲申,平陵園寢火。

八年二月己酉,南宮嘉德署、黃龍、千秋萬歲殿皆火。四月甲寅,安陵園寢火。閏月,南宮長秋、和歡殿後鉤盾、掖庭朔平署各火。十一月壬子,德陽前殿西閤及黃門北寺火,殺人。

九年三月癸巳,京都夜有火光轉行,民相驚譟。

靈帝熹平四年五月,延陵園災。

光和四年閏月辛酉,北宮東掖庭永巷署災。

五年五月庚申,德陽前殿西北入門內永樂太后宮署火。

中平二年二月己酉,南宮雲臺災。庚戌,樂城門災,延及北闕,道西燒嘉德、和歡殿。案雲臺之災自上起,榱題數百,同時並然,若就縣華鐙,其日燒盡,延及白虎、威興門、尚書、符節、蘭臺。夫雲臺者,乃周家之所造也,圖書、術籍、珍玩、寶怪皆所藏在也。京房易傳曰:「君不思道,厥妖火燒宮。」是時黃巾作慝,變亂天常,七州二十八郡同時俱發,命將出眾,雖頗有所禽,然宛、廣宗、曲陽尚未破壞,役起負海,杼柚空懸,百姓死傷已過半矣。而靈帝曾不克己復禮,虐侈滋甚,尺一雨布,騶騎電激,官非其人,政以賄成,內嬖鴻都,並受封爵。京都為之語曰:「今茲諸侯歲也。」天戒若曰:放賢賞淫,何以舊典為?故焚其臺門祕府也。其後三年,靈帝暴崩,續以董卓之亂,火三日不絕,京都為丘墟矣。

獻帝初平元年八月,霸橋災。其後三年,董卓見殺。

庶徵之恆燠,漢書以冬溫應之。中興以來,亦有冬溫,而記不錄云。

安帝元初三年,有瓜異本共生,一瓜同蔕,時以為嘉瓜。或以為瓜者外延,離本而實,女子外屬之象也。是時閻皇后初立,後閻后與外親耿寶等共譖太子,廢為濟陰王,更外迎濟北王子犢立之,草妖也。

桓帝延熹九年,雒陽城局竹柏葉有傷者。占曰:「天子凶。」

靈帝熹平三年,右校別作中有兩樗樹,皆高四尺所,其一株宿夕暴長,長丈餘,大一圍,作胡人狀,頭目鬢鬚髮備具。京房易傳曰:「王德衰,下人將起,則有木生人狀。」

五年十月壬午,御所居殿後槐樹,皆六七圍,自拔,倒豎根在上。

中平元年夏,東郡,陳留濟陽、長垣,濟陰冤句、離狐縣界,有草生,其莖靡纍腫大如手指,狀似鳩雀龍蛇鳥獸之形,五色各如其狀,毛羽頭目足翅皆具。近草妖也。是歲黃巾賊始起。皇后兄何進,異父兄朱苗,皆為將軍,領兵。後苗封濟陽侯,進、苗遂秉威權,持國柄,漢遂微弱,自此始焉。

中平中,長安城西北六七里空樹中,有人面生鬢。

獻帝興平元年九月,桑復生椹,可食。

安帝延光三年二月戊子,有五色大鳥集濟南臺,十月,又集新豐,時以為鳳皇。或以為鳳皇陽明之應,故非明主,則隱不見。凡五色大鳥似鳳者,多羽蟲之孽。是時安帝信中常侍樊豐、江京、阿母王聖及外屬耿寶等讒言,免太尉楊震,廢太子為濟陰王,不悊之異也。章帝末,號鳳皇百四十九見。時直臣何敞以為羽孽似鳳,翱翔殿屋,不察也。記者以為其後章帝崩,以為驗。案宣帝、明帝時,五色鳥群翔殿屋,賈逵以為胡降徵也。帝多善政,雖有過,不及至衰缺,末年胡降二十萬口,爾其驗也。帝之時,羌胡外叛,讒慝內興,羽孽之時也。樂協圖徵說五鳳皆五色,為瑞者一,為孽者四。

桓帝元嘉元年十一月,五色大鳥見濟陰己氏。時以為鳳皇。此時政治衰缺,梁冀秉政阿枉,上幸亳后,皆羽孽時也。

靈帝光和四年秋,五色大鳥見于新城,眾鳥隨之,時以為鳳皇。時靈帝不恤政事,常侍、黃門專權,羽孽之時也。眾鳥之性,見非常班駮,好聚觀之,至於小爵希見梟者,虣見猶聚。

中平三年八月中,懷陵上有萬餘爵,先極悲鳴,已因亂鬥相殺,皆斷頭,懸著樹枝枳棘。到六年,靈帝崩,大將軍何進以內寵外嬖,積惡日久,欲悉糾黜,以隆更始冗政,而太后持疑,事久不決。進從中出,於省內見殺,因是有司盪滌虔劉,後祿而尊厚者無餘矣。夫陵者,高大之象也。天戒若曰:諸懷爵祿而尊厚者,還自相害至滅亡也。

桓帝建和三年秋七月,北地廉雨肉似羊肋,或大如手。近赤祥也。是時梁太后攝政,兄梁冀專權,枉誅漢良臣故太尉李固、杜喬,天下冤之。其後梁氏誅滅。


\end{pinyinscope}