\article{宣張二王杜郭吳承鄭趙列傳}

\begin{pinyinscope}
宣秉字巨公,馮翊雲陽人也。少修高節,顯名三輔。哀、平際,見王氏據權專政,侵削宗室,有逆亂萌,遂隱遁深山,州郡連召,常稱疾不仕。王莽為宰衡,辟命不應。及莽篡位,又遣使者徵之,秉固稱疾病。更始即位,徵為侍中。建武元年,拜御史中丞。光武特詔御史中丞與司隸校尉、尚書令會同並專席而坐,故京師號曰「三獨坐」。明年,遷司隸校尉。務舉大綱,簡略苛細,百僚敬之。

秉性節約,常服布被,蔬食瓦器。帝嘗幸其府舍,見而歎曰:「楚國二龔,不如雲陽宣巨公。」即賜布帛帳帷什物。四年,拜大司徒司直。所得祿奉,輒以收養親族。其孤弱者,分與田地,自無擔石之儲。六年,卒於官,帝敏惜之,除子彪為郎。

張湛字子孝,扶風平陵人也。矜嚴好禮,動止有則,居處幽室,必自修整,雖遇妻子,若嚴君焉。及在鄉黨,詳言正色,三輔以為儀表。人或謂湛偽詐,湛聞而笑曰:「我誠詐也。人皆詐惡,我獨詐善,不亦可乎?」

成哀閒,為二千石。王莽時,歷太守、都尉。

建武初,為左馮翊。在郡修典禮,設條教,政化大行。後告歸平陵,望寺門而步。主簿進曰:「明府位尊德重,不宜自輕。」湛曰:「禮,下公門,軾輅馬。孔子於鄉黨,恂恂如也。父母之國,所宜盡禮,何謂輕哉?」

五年,拜光祿勳。光武臨朝,或有惰容,湛輒陳諫其失。常乘白馬,帝每見湛,輒言「白馬生且復諫矣」。

七年,以病乞身,拜光祿大夫,代王丹為太子太傅。及郭后廢,因稱疾不朝,拜太中大夫,居中東門候舍,故時人號曰中東門君。帝數存問賞賜。後大司徒戴涉被誅,帝彊起湛以代之。湛至朝堂,遺失溲便,因自陳疾篤,不能復任朝事,遂罷之。後數年,卒於家。

王丹字仲回,京兆下邽人也。哀、平時,仕州郡。王莽時,連徵不至。家累千金,隱居養志,好施周急。每歲農時,輒載酒肴於田閒,候勤者而勞之。其墯编者,恥不致丹,皆兼功自厲。邑聚相率,以致殷富。其輕黠游蕩廢業為患者,輒曉其父兄,使黜責之。沒者則賻給,親自將護。其有遭喪憂者,輒待丹為辦,鄉鄰以為常。行之十餘年,其化大洽,風俗以篤。

丹資性方絜,疾惡彊豪。時河南太守同郡陳遵,關西之大俠也。其友人喪親,遵為護喪事,賻助甚豐。丹乃懷縑一匹,陳之於主人前,曰:「如丹此縑,出自機杼。」遵聞而有慚色。自以知名,欲結交於丹,丹拒而不許。

會前將軍鄧禹西征關中,軍糧乏,丹率宗族上麥一千斛。禹表丹領左馮翊,稱疾不視事,免歸。後徵為太子少傅。

時大司徒侯霸欲與交友,及丹被徵,遣子昱候於道。昱迎拜車下,丹下荅之。昱曰:「家公欲與君結交,何為見拜?」丹曰:「君房有是言,丹未之許也。」

丹子有同門生喪親,家在中山,白丹欲往奔慰。結侶將行,丹怒而撻之,令寄縑以祠焉。或問其故。丹曰:「交道之難,未易言也。世稱管、鮑,次則王、貢。張、陳凶其終,蕭、朱隙其末,故知全之者鮮矣。」時人服其言。

客初有薦士於丹者,因選舉之,而後所舉者陷罪,丹坐以免。客慚懼自絕,而丹終無所言。尋復徵為太子太傅,乃呼客謂曰:「子之自絕,何量丹之薄也?」不為設食以罰之,相待如舊。其後遜位,卒于家。

王良字仲子,東海蘭陵人也。少好學,習小夏侯尚書。王莽時,寑病不仕,教授諸生千餘人。

建武二年,大司馬吳漢辟,大應。三年,徵拜諫議大夫,數有忠言,以禮進止,朝廷敬之。遷沛郡太守。至蘄縣,稱病不之府,官屬皆隨就之,良遂上疾篤,乞骸骨,徵拜太中大夫。

六年,代宣秉為大司徒司直。在位恭儉,妻子不入官舍,布被瓦器。時司徒史鮑恢以事到東海,過候其家,而良妻布裙曳柴,從田中歸。恢告曰:「我司徒史也,故來受書,欲見夫人。」妻曰:「妾是也。苦掾,無書。」恢乃下拜,歎息而還,聞者莫不嘉之。

後以病歸。一歲復徵,至滎陽,疾篤不任進道,乃過其友人。友人不肯見,曰:「不有忠言奇謀而取大位,何其往來屑屑不憚煩也?」遂拒之。良慚,自後連徵,輒稱病。詔以玄纁聘之,遂不應。後光武幸蘭陵,遣使者問良所苦疾,不能言對。詔復其子孫邑中傜役,卒於家。

論曰:夫利仁者或借仁以從利,體義者不期體以合義。季文子妾不衣帛,魯人以為美談。公孫弘身服布被,汲黯譏其多詐。事實未殊而譽毀別議。何也?將體之與利之異乎?宣秉、王良處位優重,而秉甘疏薄,良妻荷薪,可謂行過乎儉。然當世咨其清,人君高其節,豈非臨之以誠哉!語曰:『同言而信,則信在言前;同令而行,則誠在令外。』不其然乎!張湛不屑矜偽之誚,斯不偽矣。王丹難於交執之道,斯知交矣。

杜林字伯山,扶風茂陵人也。父鄴,成哀閒為涼州刺史。林少好學沈深,家既多書,又外氏張竦父子喜文采,林從竦受學,博洽多聞,時稱通儒。

初為郡吏。王莽敗,盜賊起,林與弟成及同郡范逡、孟冀等,將細弱俱客河西。道逢賊數千人,遂掠取財担,褫奪衣服,拔刃向林等將欲殺之。冀仰曰:「願一言而死。將軍知天神乎?赤眉兵眾百萬,所向無前,而殘賊不道,卒至破敗。今將軍以數千之眾,欲規霸王之事,不行仁恩而反遵覆車,不畏天乎?」賊遂釋之,俱免於難。

隗囂素聞林志節,深相敬待,以為持書平。後因疾告去,辭還祿食。囂復欲令彊起,遂稱篤。囂意雖相望,且欲優容之,乃出令曰:「杜伯山天子所不能臣,諸侯所不能友,蓋伯夷、叔齊恥食周粟。今且從師友之位,須道開通,使順所志。」林雖拘於囂,而終不屈節。建武六年,弟成物故,囂乃聽林持喪東歸。既遣而悔,追令刺客楊賢於隴坻遮殺之。賢見林身推鹿車,載致弟喪,乃歎曰:「當今之世,誰能行義?我雖小人,何忍殺義士!」因亡去。

光武聞林已還三輔,乃徵拜侍御史,引見,問以經書故舊及西州事,甚悅之,賜車馬衣被。群寮知林以名德用,甚尊憚之。京師士大夫,咸推其博洽。

河南鄭興、東海衛宏等,皆長於古學。興嘗師事劉歆,林既遇之,欣然言曰:「林得興等固諧矣,使宏得林,且有以益之。」及宏見林,闇然而服。濟南徐巡,始師事宏,後皆更受林學。林前於西州得漆書古文尚書一卷,常寶愛之,雖遭難困,握持不離身。出以示宏等曰:「林流離兵亂,常恐斯經將絕。何意東海衛子、濟南徐生復能傳之,是道竟不墜於地也。古文雖不合時務,然願諸生無悔所學。」宏、巡益重之,於是古文遂行。

明年,大議郊祀制,多以為周郊后稷,漢當祀堯。詔復下公卿議,議者僉同,帝亦然之。林獨以為周室之興,祚由后稷,漢業特起,功不緣堯。祖宗故事,所宜因循。定從林議。

後代王良為大司徒司直。林薦同郡范逡、趙秉、申屠剛及隴西牛邯等,皆被擢用,士多歸之。十一年,司直官罷,以林代郭憲為光祿勳。內奉宿衛,外總三署,周密敬慎,選舉稱平。郎有好學者,輒見誘進,朝夕滿堂。

十四年,群臣上言:「古者肉刑嚴重,則人畏法令;今憲律輕薄,故姦軌不勝。宜增科禁,以防其源。」詔下公卿。林奏曰:「夫人情挫辱,則義節之風損;法防繁多,則苟免之行興。孔子曰:『導之以政,齊之以刑,民免而無恥。導之以德,齊之以禮,有恥且格。』古之明王,深識遠慮,動居其厚,不務多辟,周之五刑,不過三千。大漢初興,詳覽失得,故破矩為圓,斲彫為樸,蠲除苛政,更立疏網,海內歡欣,人懷寬德。及至其後,漸以滋章,吹毛索疵,詆欺無限。果桃菜茹之饋,集以成臧,小事無妨於義,以為大戮,故國無廉士,家無完行。至於法不能禁,令不能止,上下相遁,為敝彌深。臣愚以為宜如舊制,不合翻移。」帝從之。

後皇太子彊求乞自退,封東海王,故重選官屬,以林為王傅。從駕南巡狩。時諸王傅數被引命,或多交游,不得應詔;唯林守慎,有召必至。餘人雖不見譴,而林特受賞賜,又辭不敢受,帝益重之。

明年,代丁恭為少府。二十二年,復為光祿勳。頃之,代朱浮為大司空,博雅多通,稱為任職相。明年薨,帝親自臨喪送葬,除子喬為郎。詔曰:「公侯子孫,必復其始,賢者之後,宜宰城邑。其以喬為丹水長。」

論曰:夫威彊以自禦,力損則身危;飾詐以圖己,詐窮則道屈;而忠信篤敬,蠻貊行焉者,誠以德之感物厚矣。故趙孟懷忠,匹夫成其仁;杜林行義,烈士假其命。《易》曰「人之所助者順」,有不誣矣。

郭丹字少卿,南陽穰人也。父稚,成帝時為廬江太守,有清名。丹七歲而孤,小心孝順,後母哀憐之,為鬻衣担,買產業。後從師長安,買符入函谷關,乃慨然歎曰:「丹不乘使者車,終不出關。」既至京師,常為都講,諸儒咸敬重之。大司馬嚴尤請丹,辭病不就。王莽又徵之,遂與諸生逃於北地。更始二年,三公舉丹賢能,徵為諫議大夫,持節使歸南陽,安集受降。丹自去家十有二年,果乘高車出關,如其志焉。

更始敗,諸將悉歸光武,並獲封爵;丹獨保平氏不下,為更始發喪,衰絰盡哀。建武二年,遂潛逃去,敝衣閒行,涉歷險阻,求謁更始妻子,奉還節傳,因歸鄉里。太守杜詩請為功曹,丹薦鄉人長者自代而去。詩乃歎曰:「昔明王興化,卿士讓位,今功曹推賢,可謂至德。敕以丹事編署黃堂,以為後法。」

十三年,大司馬吳漢辟舉高第,再遷并州牧,有清平稱。轉使匈奴中郎將,遷左馮翊。永平三年,代李訢為司徒。在朝廉直公正,與侯霸、杜林、張湛、郭伋齊名相善。明年,坐考隴西太守鄧融事無所據,策免。五年,卒於家,時年八十七。以河南尹范遷有清行,代為司徒。

遷字子廬,沛國人,初為漁陽太守,以智略安邊,匈奴不敢入界。及在公輔,有宅數畝,田不過一頃,復推與兄子。其妻嘗謂曰:「君有四子而無立錐之地,可餘奉祿,以為後世業。」遷曰:「吾備位大臣而蓄財求利,何以示後世!」在位四年薨,家無擔石焉。

後顯宗因朝會問群臣郭丹家今何如,宗正劉匡對曰:「昔孫叔敖相楚,馬不秣粟,妻不衣帛,子孫竟蒙寑丘之封。丹出典州郡,入為三公,而家無遺產,子孫困匱。」帝乃下南陽訪求其嗣。長子宇,官至常山太守。少子濟,趙相。

吳良字大儀,齊國臨淄人也。初為郡吏,歲旦與掾史入賀,門下掾王望舉觴上壽,諂稱太守功德。良於下坐勃然進曰:「望佞邪之人,欺諂無狀,願勿受其觴。」太守斂容而止。讌罷,轉良為功曹;恥以言受進,終不肯謁。

時驃騎將軍東平王蒼聞而辟之,署為西曹。蒼甚相敬愛,上疏薦良曰:「臣聞為國所重,必在得人;報恩之義,莫大薦士。竊見臣府西曹掾齊國吳良,資質敦固,公方廉恪,躬儉安貧,白首一節;又治尚書,學通師法,經任博士,行中表儀。宜備宿衛,以輔聖政。臣蒼榮寵絕矣,憂責深大,私慕公叔同升之義,懼於臧文竊位之罪,敢秉愚瞽,犯冒嚴禁。」顯宗以示公卿曰:「前以事見良,鬚髮皓然,衣冠甚偉,夫薦賢助國,宰相之職,蕭何舉韓信,設壇而拜,不復考試。今以良為議郎。」

永平中,車駕近出,而信陽侯陰就干突禁衛,車府令徐匡鉤就車,收御者送獄。詔書譴匡,匡乃自繫。良上言曰:「信陽侯就倚恃外戚,干犯乘輿,無人臣禮,為大不敬。匡執法守正,反下于理,臣恐聖化由是而弛。」帝雖赦匡,猶左轉良為即丘長。後遷司徒長史。每處大議,輒據經典,不希旨偶俗,以徼時譽。後坐事免。復拜議郎,卒於官。

承宮字少子,琅邪姑幕人也。少孤,年八歲為人牧豕。鄉里徐子盛者,以春秋經授諸生數百人,宮過息廬下,樂其業,因就聽經,遂請留門下,為諸生拾薪。執苦數年,勤學不倦。經典既明,乃歸家教授。遭天下喪亂,遂將諸生避地漢中,後與妻子之蒙陰山,肆力耕種。禾黍將孰,人有認之者,宮不與計,推之而去,由是顯名。三府更辟,皆不應。

永平中,徵詣公車。車駕臨辟雍,召宮拜博士,遷左中郎將。數納忠言,陳政,論議切愨,朝臣憚其節,名播匈奴。時北單于遣使求得見宮,顯宗敕自整飾,宮對曰:「夷狄眩名,非識實者也。臣狀醜,不可以示遠,宜選有威容者。」帝乃以大鴻臚魏應代之。十七年,拜侍中祭酒。建初元年,卒,肅宗褒歎,賜以冢地。妻上書乞歸葬鄉里,復賜錢三十萬。

鄭均字仲虞,東平任城人也。少好黃老書。兄為縣吏,頗受禮遺,均數諫止,不聽。即脫身為傭,歲餘,得錢帛,歸以與兄。曰:「物盡可復得,為吏坐臧,終身捐棄。」兄感其言。遂為廉絜。均好義篤實,養寡嫂孤兒,恩禮敦至。常稱病家廷,不應州郡辟召。郡將欲必致之,使縣令譎將詣門,既至,卒不能屈。均於是客於濮陽。

建初三年,司徒鮑昱辟之,後舉直言,並不詣。六年,公車特徵,再遷尚書,數納忠言,肅宗敬重之。後以病乞骸骨,拜議郎,告歸,因稱病篤,帝賜以衣冠。

元和元年,詔告廬江太守、東平相曰:「議郎鄭均,束脩安貧,恭儉節整,前在機密,以病致仕,守善貞固,黃髮不怠。又前安邑令毛義,躬履遜讓,比徵辭病,淳絜之風,東州稱仁。書不云乎:『章厥有常,吉哉!』其賜均、義穀各千斛,常以八月長吏存問,賜羊酒,顯茲異行。」明年,帝東巡過任城,乃幸均舍,敕賜尚書祿以終其身,故時人號為「白衣尚書」。永元中,卒於家。

趙典字仲經,蜀郡成都人也。父戒,為太尉,桓帝立,以定策封廚亭侯。典少篤行隱約,博學經書,弟子自遠方至。建和初,四府表薦,徵拜議郎,侍講禁內,再遷為侍中。時帝欲廣開鴻池,典諫曰:「鴻池汎溉,已且百頃,猶復增而深之,非所以崇唐虞之約己,遵孝文之愛人也。」帝納其言而止。

父卒,襲封。出為弘農太守,轉右扶風。公事去官,徵拜城門校尉,轉將作大匠,遷少府,又轉大鴻臚。時恩澤諸侯以無勞受封,群臣不悅而莫敢諫,典獨奏曰:「夫無功而賞,勞者不勸,上忝下辱,亂象干度。且高祖之誓,非功臣不封。宜一切削免爵土,以存舊典。」帝不從。頃之,轉太僕,遷太常。朝廷每有災異疑議,輒諮問之。典據經正對,無所曲折。每得賞賜,輒分與諸生之貧者。後以諫爭違旨,免官就國。

會帝崩,時禁藩國諸侯不得奔弔,典慨然曰:「身從衣褐之中,致位上列。且鳥烏反哺報德,況於士邪!」遂解印綬符策付縣,而馳到京師。州郡及大鴻臚並執處其罪,而公卿百寮嘉典之義,表請以租自贖,詔書許之。再遷長樂少府、衛尉。公卿復表典篤學博聞,宜備國師。會病卒,使者弔祠。竇太后復遣使兼贈印綬,謚曰獻侯。

典兄子謙,謙弟溫,相繼為三公。

謙字彥信,初平元年,代黃琬為太尉。獻帝遷都長安,以謙行車騎將軍,為前置。明年病罷。復為司隸校尉。車師王侍子為董卓所愛,數犯法,謙收殺之。卓大怒,殺都官從事,而素敬憚謙,故不加罪。轉為前將軍,遣擊白波賊,有功,封郫侯。李傕殺司徒王允,復代允為司徒,數月病免,拜尚書令。是年卒,謚曰忠侯。

溫字子柔,初為京兆郡丞,歎曰:「大丈夫當雄飛,安能雌伏!」遂棄官去。遭歲大飢,散家糧以振窮餓,所活萬餘人。獻帝西遷都,為侍中,同輿輦至長安,封江南亭侯,代楊彪為司空,免,頃之,復為司徒,錄尚書事。

時李傕與郭汜相攻,傕遂虜掠禁省,劫帝幸北塢,外內隔絕。傕素疑溫不與己同,乃內溫於塢中,又欲移乘輿於黃白城。溫與傕書曰:「公前託為董公報讎,然實屠陷王城,殺戮大臣,天下不可家見而戶說也。今與郭汜爭睚眥之隙,以成千鈞之讎,人在塗炭,各不聊生。曾不改悟,遂成禍亂。朝廷仍下明詔,欲令和解。上命不行,威澤日損。而復欲移轉乘輿,更幸非所,此誠老夫所不達也。於易,一為過,再為涉,三而弗改,滅其頂,凶。不如早共和解,引軍還屯,上安萬乘,下全人民,豈不幸甚。」傕大怒,欲遣人殺溫。董卓從弟應,溫故掾也,諫之數日,乃獲免。

溫從車駕都許。建安十三年,以辟司空曹操子丕為掾,操怒,奏溫辟忠臣子弟,選舉不實,免官。是歲卒,年七十二。

贊曰:宣、鄭、二王,奉身清方。杜林據古,張湛矜莊。典以義黜,宮由德揚。大儀鵠髮,見表憲王。少卿志仕,終乘高箱。


\end{pinyinscope}