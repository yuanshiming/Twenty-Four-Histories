\article{朱景王杜馬劉傅堅馬列傳}

\begin{pinyinscope}
朱祐字仲先,南陽宛人也。少孤,歸外家復陽劉氏,往來舂陵,世祖與伯升皆親愛之。伯升拜大司徒,以祐為護軍。及世祖為大司馬,討河北,復以祐為護軍,常見親幸,舍止於中。祐侍讌,從容曰:「長安政亂,公有日角之相,此天命也。」世祖曰:「召刺姦收護軍!」祐乃不敢復言。從征河北,常力戰陷陣,以為偏將軍,封安陽侯。世祖即位,拜為建義大將軍。建武二年,更封堵陽侯。冬,與諸將擊鄧奉於淯陽,祐軍敗,為奉所獲。明年,奉破,乃肉袒因祐降。帝復祐位而厚加慰賜。遣擊新野、隨,皆平之。

延岑自敗於穰,遂與秦豐將張成合,祐率征虜將軍祭遵與戰於東陽,大破之,臨陣斬成,延岑敗走歸豐。祐收得印綬九十七。進擊黃郵,降之,賜祐黃金三十斤。四年,率破姦將軍侯進、輔威將軍耿植代征南大將軍岑彭圍秦豐於黎丘,破其將張康於蔡陽,斬之。帝自至黎丘,使御史中丞李由持璽書招豐,豐出惡言,不肯降。車駕引還,敕祐方略,祐盡力攻之。明年夏,城中窮困,豐乃將其母妻子九人肉袒降。祐轞車傳豐送洛陽,斬之。大司馬吳漢劾奏祐廢詔受降,違將帥之任,帝不加罪。祐還,與騎都尉臧宮會擊延岑餘黨陰、酇、筑陽三縣賊,悉平之。

祐為人質直,尚儒學。將兵率眾,多受降,以克定城邑為本,不存首級之功。又禁制士卒不得虜掠百姓,軍人樂放縱,多以此怨之。九年,屯南行唐拒匈奴。十三年,增邑,定封鬲侯,食邑七千三百戶。

十五年,朝京師,上大將軍印綬,因留奉朝請。祐奏古者人臣受封,不加王爵,可改諸王為公。帝即施行,又奏宜令三公並去「大」名,以法經典。後遂從其議。

祐初學長安,帝往候之,祐不時相勞苦,而先升講舍。後車駕幸其第,帝因笑曰;「主人得無捨我講乎?」以有舊恩,數蒙賞賚。二十四年,卒。

子商嗣。商卒,子演嗣,永元十四年,坐從兄伯為外孫陰皇后巫蠱事,免為庶人。永初七年,鄧太后紹封演子沖為鬲侯。

景丹字孫卿,馮翊櫟陽人也。少學長安。王莽時舉四科:丹以言語為固德侯相,有幹事稱,遷朔調連率副貳。

更始立,遣使者徇上谷,丹與連率耿況降,復為上谷長史。王郎起,丹與況共謀拒之。況使丹與子弇及寇恂等將兵南歸世祖,世祖引見丹等,笑曰:「邯鄲將帥數言我發漁陽、上谷兵,吾聊應言然,何意二郡良為吾來!方與士大夫共此功名耳。」拜丹為偏將軍,號奉義侯。從擊王郎將兒宏等於南讀,郎兵迎戰,漢軍退卻,丹等縱突騎擊,大破之,追奔十餘里,死傷者從橫。丹還,世祖謂曰:「吾聞突騎天下精兵,今乃見其戰,樂可言邪?」遂從征河北。

世祖即位,以讖文用平狄將軍孫咸行大司馬,眾咸不悅。詔舉可為大司馬者,群臣所推唯吳漢及丹。帝曰:「景將軍北州大將,是其人也。然吳將軍有建大策之勳,又誅苗幽州、謝尚書,其功大。舊制驃騎將軍官與大司馬相兼也。」乃以吳漢為大司馬,而拜丹為驃騎大將軍。

建武二年,定封丹櫟陽侯。帝謂丹曰:「今關東故王國,雖數縣,不過櫟陽萬戶邑。夫『富貴不歸故鄉,如衣繡夜行』,故以封卿耳。」丹頓首謝。秋,與吳漢、建威大將軍耿弇、建義大將軍朱祐、執金吾賈復、偏將軍馮異、強弩將軍陳俊、左曹王常、騎都尉臧宮等從擊破五校於羛陽,降其眾五萬人。會陝賊蘇況攻破弘農,生獲郡守。丹時病,帝以其舊將,欲令強起領郡事,乃夜召入,謂曰:「賊迫近京師,但得將軍威重,臥以鎮之足矣。」丹不敢辭,乃力疾拜命,將營到郡,十餘日薨。

子尚嗣,徙封余吾侯。尚卒,子苞嗣。苞卒,子臨嗣,無子,國絕。永初七年,鄧太后紹封苞弟遽為監亭侯。

王梁字君嚴,漁陽安陽人也。為郡吏,太守彭寵以梁守狐奴令,與蓋延、吳漢俱將兵南及世祖於廣阿,拜偏將軍。既拔邯鄲,賜爵關內侯。從平河北,拜野王令,與河內太守寇恂南拒洛陽,北守天井關,朱鮪等不敢出兵,世祖以為梁功。及即位,議選大司空,而赤伏符曰「王梁主衛作玄武」,帝以野王衛之所徙,玄武水神之名,司空水土之官也,於是擢拜梁為大司空,封武強侯。

建武二年,與大司馬吳漢等俱擊檀鄉,有詔軍事一屬大司馬,而梁輒發野王兵,帝以其不奉詔敕,令止在所縣,而梁復以便宜進軍。帝以梁前後違命,大怒,遣尚書宗廣持節軍中斬梁。廣不忍,乃檻車送京師。既至,赦之。月餘,以為中郎將,行執金吾事。北守箕關,擊赤眉別校,降之。三年春,轉擊五校,追至信都、趙國,破之,悉平諸屯聚。冬,遣使者持節拜梁前將軍。四年春,擊肥城、文陽,拔之。進與驃騎大將軍杜茂擊佼彊、蘇茂於楚、沛閒,拔大梁、齧桑,而捕虜將軍馬武、偏將軍王霸亦分道並進,歲餘悉平之。五年,從救桃城,破龐萌等,梁戰尤力,拜山陽太守,鎮撫新附,將兵如故。

數月徵入,代歐陽歙為河南尹。梁穿渠引穀水注洛陽城下,東寫鞏川,及渠成而水不流。七年,有司劾奏之,梁慚懼,上書乞骸骨。乃下詔曰:「梁前將兵征伐,眾人稱賢,故擢典京師。建議開渠,為人興利,旅力既愆,迄無成功,百姓怨讟,談者讙譁。雖蒙寬宥,猶執謙退,『君子成人之美』,其以梁為濟南太守。」十三年,增邑,定封封阜成侯。十四年,卒官。

子禹嗣。禹卒,子堅石嗣。堅石追坐父禹及弟平與楚王英謀反,棄市,國除。

杜茂字諸公,南陽冠軍人也。初歸光武於河北,為中堅將軍,常從征伐。世祖即位,拜大將軍,封樂鄉侯。北擊五校於真定,進降廣平。建武二年,更封苦陘侯。與中郎將王梁擊五校賊於魏郡、清河、東郡,悉平諸營保,降其持節大將三十餘人,三郡清靜,道路流通。明年,遣使持節拜茂為驃騎大將軍,擊沛郡,拔芒。時西防復反,迎佼彊。五年春,茂率捕虜將軍馬武進攻西防,數月拔之,彊奔董憲。

東方既平,七年,詔茂引兵北屯田晉陽、廣武,以備胡寇。九年,與鴈門太守郭涼擊盧芳將尹由於繁畤,芳將賈覽率胡騎萬餘救之,茂戰,軍敗,引入樓煩城。時盧芳據高柳,與匈奴連兵,數寇邊民,帝患之。十二年,遣謁者段忠將眾郡弛刑配茂,鎮守北邊,因發邊卒築亭候,修烽火,又發委輸金帛繒絮供給軍士,并賜邊民,冠蓋相望。茂亦建屯田,驢車轉運。先是,鴈門人賈丹、霍匡、解勝等為尹由所略,由以為將帥,與共守平城。丹等聞芳敗,遂共殺由詣郭涼;涼上狀,皆封為列侯,詔送委輸金帛賜茂、涼軍吏及平城降民。自是盧芳城邑稍稍來降,涼誅其豪右郇氏之屬,鎮撫羸弱,旬月閒鴈門且平,芳遂亡入匈奴。帝擢涼子為中郎,宿衛左右。

涼字公文,右北平人也。身長八尺,氣力壯猛,雖武將,然通經書,多智略,尤曉邊事,有名北方。初,幽州牧朱浮辟為兵曹掾,擊彭寵有功,封廣武侯。

十三年,增茂邑,更封脩侯。十五年,坐斷兵馬稟縑,使軍吏殺人,免官,削戶邑,定封參蘧鄉侯。十九年,卒。

子元嗣,永平十四年,坐與東平王等謀反,減死一等,國除。永初七年,鄧太后紹封茂孫奉為安樂亭侯。

馬成字君遷,南陽棘陽人也。少為縣吏。世祖徇潁川,以成為安集掾,調守郟令。及世祖討河北,成即棄官步負,追及於滿陽,以成為期門,從征伐。世祖即位,再遷護軍都尉。

建武四年,拜揚武將軍,督誅虜將軍劉隆、振威將軍宋登、射聲校尉王賞,發會稽、丹陽、九江、六安四郡兵擊李憲,時帝幸壽春,設壇場,祖禮遣之。進圍憲於舒,令諸軍各深溝高壘。憲數挑戰,成堅壁不出,守之歲餘,至六年春,城中食盡,乃攻之,遂屠舒,斬李憲,追擊其黨與,盡平江淮地。

七年夏,封平舒侯。八年,從征破隗囂,以成為天水太守,將軍如故。冬,徵還京師。九年,代來歙守中郎將,率武威將軍劉尚等破河池,遂平武都。明年,大司空李通罷,以成行大司空事,居府如真,數月復拜揚武將軍。

十四年,屯常山、中山以備北邊,并領建義大將軍朱祐營。又代驃騎大將軍杜茂繕治障塞,自西河至渭橋,河上至安邑,太原至井陘,中山至鄴,皆築保壁,起烽燧,十里一候。在事五六年,帝以成勤勞,徵還京師。邊人多上書求請者,復遣成還屯。及南單于保塞,北方無事,拜為中山太守,上將軍印綬,領屯兵如故。二十四年,南擊武谿蠻賊,無功,上太守印綬。

二十七年,定封全椒侯,就國。三十二年卒。

子衛嗣。衛卒,子香嗣,徙封棘陵侯。香卒,子豐嗣。豐卒,子玄嗣。玄卒,子邑嗣。邑卒,子醜嗣,桓帝時以罪失國。延熹二年,帝復封成玄孫昌為益陽亭侯。

劉隆字元伯,南陽安眾侯宗室也。王莽居攝中,隆父禮與安眾侯崇起兵誅莽,事泄,隆以年未七歲,故得免。及壯,學於長安,更始拜為騎都尉。謁歸,迎妻子置洛陽。聞世祖在河內,即追及於射犬,以為騎都尉,與馮異共拒朱鮪、李軼等,軼遂殺隆妻子。建武二年,封亢父侯。四年,拜誅虜將軍,討李憲。憲平,遣隆屯田武當。

十一年,守南郡太守,歲餘,上將軍印綬。十三年,增邑,更封竟陵侯。是時,天下墾田多不以實,又戶口年紀互有增減。十五年,詔下州郡檢覈其事,而刺史太守多不平均,或優饒豪右,侵刻羸弱,百姓嗟怨,遮道號呼。時諸郡各遣使奏事,帝見陳留吏牘上有書,視之,云「潁川、弘農可問,河南、南陽不可問」。帝詰吏由趣,吏不肯服,抵言於長壽街上得之。帝怒。時顯宗為東海公,年十二,在幄後言曰:「吏受郡敕,當欲以墾田相方耳。」帝曰:「即如此,何故言河南、南陽不可問?」對曰:「河南帝城,多近臣,南陽帝鄉,多近親,田宅踰制,不可為準。」帝令虎賁將詰問吏,吏乃實首服,如顯宗對。於是遣謁者考實,具知姦狀。明年,隆坐徵下獄,其疇輩十餘人皆死。帝以隆功臣,特免為庶人。

明年,復封為扶樂鄉侯,以中郎將副伏波將軍馬援擊交阯蠻夷徵側等,隆別於禁谿口破之,獲其帥徵貳,斬首千餘級,降者二萬餘人。還,更封大國,為長平侯。及大司馬吳漢薨,隆為驃騎將軍,行大司馬事。

隆奉法自守,視事八歲,上將軍印綬,罷,賜養牛,上樽酒十斛,以列侯奉朝請。三十年,定封慎侯。中元二年,卒,謚曰靖侯。子安嗣。

傅俊字子衛,潁川襄城人也。世祖徇襄城,俊以縣亭長迎軍,拜為校尉,襄城收其母弟宗族,皆滅之。從破王尋等,以為偏將軍。別擊京、密,破之,遣歸潁川,收葬家屬。

及世祖討河北,俊與賓客十餘人北追,及於邯鄲,上謁,世祖使將潁川兵,常從征伐。世祖即位,以俊為侍中。建武二年,封昆陽侯。三年,拜俊積弩將軍,與征南大將軍岑彭擊破秦豐,因將兵徇江東,揚州悉定。七年,卒,謚曰威侯。

子昌嗣,徙封蕪湖侯。建初中,遭母憂,因上書,以國貧不願之封,乞錢五十萬,為關內侯。肅宗怒,貶為關內侯,竟不賜錢。永初七年,鄧太后復封昌子鐵為高置亭侯。

堅鐔字子伋,潁川襄城人也。為郡縣吏。世祖討河北,或薦鐔者,因得召見。以其吏能,署主簿。又拜偏將軍,從平河北,別擊破大槍於盧奴。世祖即位,拜鐔揚化將軍,封濦強侯。

與諸將攻洛陽,而朱鮪別將守東城者為反閒,私約鐔晨開上東門。鐔與建義大將軍朱祐乘朝而入,與鮪大戰武庫下,殺傷甚眾,至旦食乃罷,朱鮪由是遂降。又別擊內黃,平之。建武二年,與右將軍萬脩徇南陽諸縣,而堵鄉人董訢反宛城,獲南陽太守劉驎。鐔乃引軍赴宛,選敢死士夜自登城,斬關而入,訢遂棄城走還堵鄉。鄧奉復反新野,攻破吳漢。時萬脩病卒,鐔獨孤絕,南拒鄧奉,北當董訢,一年閒道路隔塞,糧饋不至,鐔食蔬菜,與士卒共勞苦。每急,輒先當矢石,身被三創,以此能全其眾。及帝征南陽,擊破訢、奉,以鐔為左曹,常從征伐。六年,定封合肥侯。二十六年,卒。

子鴻嗣。鴻卒,子浮嗣。浮卒,子雅嗣。

馬武字子張,南陽湖陽人也。少時避讎,客居江夏。王莽末,竟陵、西陽三老起兵於郡界,武往從之,後入綠林中,遂與漢軍合。更始立,以武為侍郎,與世祖破王尋等,拜為振威將軍,與尚書令謝躬共攻王郎。

及世祖拔邯鄲,請躬及武等置酒高會,因欲以圖躬,不剋。既罷,獨與武登叢臺,從容謂武曰:「吾得漁陽、上谷突騎,欲令將軍將之,何如?」武曰:「駑怯無方略。」世祖曰:「將軍久將,習兵,豈與我掾史同哉!」武由是歸心。

及謝躬誅死,武馳至射犬降,世祖見之甚悅,引置左右,每勞饗諸將,武輒起斟酌於前,世祖以為歡。復使將其部曲至鄴,武叩頭辭以不願,世祖愈美其意,因從擊群賊。世祖擊尤來、五幡等,敗於慎水,武獨殿,還陷陣,故賊不得迫及。進至安定次、小廣陽,武常為軍鋒,力戰無前,諸將皆引而隨之,故遂破賊,窮追至平谷、浚靡而還。

世祖即位,以武為侍中、騎都尉,封山都侯。建武四年,與虎牙將軍蓋延等討劉永,武別擊濟陰,下成武、楚丘,拜捕虜將軍。明年,龐萌反,攻桃城,武先與戰,破之;會車駕至,萌遂敗走。六年夏,與建威大將軍耿弇西擊隗囂,漢軍不利,引下隴。囂追急,武選精騎還為後拒,身被甲持戟奔擊,殺數千人,囂兵乃退,諸軍得還長安。

十三年,增邑,更封鄃侯。將兵北屯下曲陽,備匈奴。坐殺軍吏,受詔將妻子就國。武徑詣洛陽,上將軍印綬,削戶五百,定封為楊虛侯,因留奉朝請。

帝後與功臣諸侯讌語,從容言曰:「諸卿不遭際會,自度爵祿何所至乎?」高密侯鄧禹先對曰:「臣少嘗學問,可郡文學博士。」帝曰:「何言之謙乎?卿鄧氏子,志行脩整,何為不掾功曹?」餘各以次對,至武,曰:「臣以武勇,可守尉督盜賊。」帝笑曰:「且勿為盜賊,自致亭長,斯可矣。」武為人嗜酒,闊達敢言,時醉在御前面折同列,言其短長,無所避忌。帝故縱之,以為笑樂。帝雖制御功臣,而每能回容,宥其小失。遠方貢珍甘,必先遍賜列侯,而太官無餘。有功,輒增邑賞,不任以吏職,故皆保其福祿,終無誅譴者。

二十五,武以中郎將將兵擊武陵蠻夷,還,上印綬。顯宗初,西羌寇隴右,覆軍殺將,朝廷患之,復拜武捕虜將軍,以中郎將王豐副,與監軍使者竇固、右輔都尉陳訢,將烏桓、黎陽營、三輔募士、涼州諸郡羌胡兵及弛刑,合四萬人擊之。到金城浩亹,與羌戰,斬首六百級。又戰於洛都谷,為羌所敗,死者千餘人。羌乃率眾引出塞,武復追擊到東、西邯,大破之,斬首四千六百級,獲生口千六百人,餘皆降散。武振旅還京師,增邑七百戶,并前千八百戶。永平四年,卒。

子檀嗣,坐兄伯濟與楚王英黨顏忠謀反,國除。永初七年,鄧太后紹封武孫震為漻亭侯。震卒,子側嗣。

論曰:中興二十八將,前世以為上應二十八宿,未之詳也。然咸能感會風雲,奮其智勇,稱為佐命,亦各志能之士也。議者多非光武不以功臣任職,至使英姿茂績,委而勿用。然原夫深圖遠筭,固將有以焉爾。若乃王道既衰,降及霸德,猶能授受惟庸,勳賢皆序,如管、隰之迭升桓世,先、趙之同列文朝,可謂兼通矣。降自秦、漢,世資戰力,至於翼扶王運,皆武人屈起。亦有鬻繒屠狗輕猾之徒,或崇以連城之賞,或任以阿衡之地,故埶疑則隙生,力侔則亂起。蕭、樊且猶縲紲,信、越終見葅戮,不其然乎!自茲以降,迄于孝武,宰輔五世,莫非公侯。遂使縉紳道塞,賢能蔽壅,朝有世及之私,下多抱關之怨。其懷道無聞,委身草莽者,亦何可勝言。故光武鑒前事之違,存矯枉之志,雖寇、鄧之高勳,耿、賈之鴻烈,分土不過大縣數四,所加特進、朝請而已。觀其治平臨政,課職責咎,將所謂「導之以政,齊之以刑」者乎!若格之功臣,其傷已甚。何者?直繩則虧喪恩舊,橈情則違廢禁典,選德則功不必厚,舉勞則人或未賢,參任則群心難塞,並列則其敝未遠。不得不校其勝否,即以事相權。故高秩厚禮,允荅元功,峻文深憲,責成吏職。建武之世,侯者百餘,若夫數公者,則與參國議,分均休咎,其餘並優以寬科,完其封祿,莫不終以功名延慶于後。昔留侯以為高祖悉用蕭、曹故人,而郭伋亦譏南陽多顯,鄭興又戒功臣專任。夫崇恩偏授,易啟私溺之失,至公均被,必廣招賢之路,意者不其然乎!

永平中,顯宗追感前世功臣,乃圖畫二十八將於南宮雲臺,其外又有王常、李通、竇融、卓茂,合三十二人。故依其本弟係之篇末,以志功臣之次云爾。

太傅高密侯鄧禹中山太守全椒侯馬

成

大司馬廣平侯吳漢河南尹阜成侯王梁

左將軍膠東侯賈復琅邪太守祝阿侯陳

俊

建威大將軍好畤侯耿弇驃騎大將軍參蘧侯

杜茂

執金吾雍奴侯寇恂積弩將軍昆陽侯傅

俊

征南大將軍舞陽侯岑彭左曹合肥侯堅鐔

征西大將軍陽夏侯馮異上谷太守淮陽

侯王霸

建義大將軍鬲侯朱祐信都太守阿陵侯任

光

征虜將軍潁陽侯祭遵豫章太守中水侯李

忠

驃騎大將軍櫟陽侯景丹右將軍槐里侯萬脩

虎牙大將軍安平侯蓋延太常靈壽侯邳彤

衛尉安成侯銚期驍騎將軍昌成侯劉

植

東郡太守東光侯耿純橫野大將軍山桑侯

王常

城門校尉朗陵侯臧宮大司空固始侯李通

捕虜將軍楊虛侯馬武大司空安豐侯竇融

驃騎將軍慎侯劉隆太傅宣德侯卓茂

贊曰:帝績思乂,庸功是存。有來群后,捷我戎軒。婉孌龍姿,儷景同侴。


\end{pinyinscope}