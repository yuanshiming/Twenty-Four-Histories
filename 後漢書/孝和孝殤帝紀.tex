\article{孝和孝殤帝紀}

\begin{pinyinscope}
孝和皇帝諱肇,肅宗第四子也。母梁貴人,為竇皇后所譖,憂卒,竇后養帝以為己子。建初七年,立為皇太子。

章和二年二月壬辰,即皇帝位,年十歲。尊皇后曰皇太后,太后臨朝。

三月丁酉,改淮陽為陳國,楚郡為彭城國,西平并汝南郡,六安復為廬江郡。遺詔徙西平王羨為陳王,六安王恭為彭城王。

癸卯,葬孝章皇帝于敬陵。

庚戌,皇太后詔曰:「先帝以明聖,奉承祖宗至德要道,天下清靜,庶事咸寧。今皇帝以幼年,煢煢在疚,朕且佐助聽政。外有大國賢王並為蕃屏,內有公卿大夫統理本朝,恭己受成,夫何憂哉!然守文之際,必有內輔以參聽斷。侍中憲,朕之元兄,行能兼備,忠孝尤篤,先帝所器,親受遺詔,當以舊典輔斯職焉。憲固執謙讓,節不可奪。今供養兩宮,宿衛左右,厥事已重,亦不可復勞以政事。故太尉鄧彪,元功之族,三讓彌高,海內歸仁,為群賢首,先帝褒表,欲以崇化。今彪聰明康彊,可謂老成黃耇矣。其以彪為太傅,賜爵關內侯,錄尚書事,百官總己以聽,朕庶幾得專心內位。於戲!群公其勉率百僚,各修厥職,愛養元元,綏以中和,稱朕意焉。」

辛酉,有司上奏:「孝章皇帝崇弘鴻業,德化普洽,垂意黎民,留念稼穡。文加殊俗,武暢方表,界惟人面,無思不服。巍巍蕩蕩,莫與比隆。周頌曰:『於穆清廟,肅雝顯相。』請上尊廟曰肅宗,共進武德之舞。」制曰:「可。」

癸亥,陳王羨、彭城王恭、樂成王黨、下邳王衍、梁王暢始就國。

夏四月丙子,謁高廟。丁丑,謁世祖廟。

戊寅,詔曰:「昔孝武皇帝致誅胡、越,故權收鹽鐵之利,以奉師旅之費。自中興以來,匈奴未賓,永平末年,復修征伐。先帝即位,務休力役,然猶深思遠慮,安不忘危,探觀舊典,復收鹽鐵,欲以防備不虞,寧安邊境。而吏多不良,動失其便,以違上意。先帝恨之,故遺戒郡國罷鹽鐵之禁,縱民煮鑄,入稅縣官如故事。其申敕刺史、二千石,奉順聖旨,勉弘德化,布告天下,使明知朕意。」

五月,京師旱。詔長樂少府桓郁侍講禁中。

冬十月乙亥,以侍中竇憲為車騎將軍,伐北匈奴。

安息國遣使獻師子、扶拔。

永元元年春三月甲辰,初令郎官詔除者得占丞、尉,以比秩為真。

夏六月,車騎將軍竇憲出雞鹿塞,度遼將軍鄧鴻出棝陽塞,南單于出滿夷谷,與北匈奴戰於稽落山,大破之,追至和渠北鞮海。竇憲遂登燕然山,刻石勒功而還。北單于遣弟右溫禺鞮王奉奏貢獻。

秋七月乙未,會稽山崩。

閏月丙子,詔曰:「匈奴背叛,為害久遠。賴祖宗之靈,師克有捷,醜虜破碎,遂掃厥庭,役不再籍,萬里清蕩,非朕小子眇身所能克堪。有司其案舊典,告類薦功,以章休烈。」

九月庚申,以車騎將軍竇憲為大將軍,以中郎將劉尚為車騎將軍。

冬十月,令郡國弛刑輸作軍營。其徙出塞者,刑雖未竟,皆免歸田里。

庚子,阜陵王延薨。

是歲,郡國九大水。

二年春正月丁丑,大赦天下。

二月壬午,日有食之。

己亥,復置西河、上郡屬國都尉官。

夏五月庚戌,分太山為濟北國,分樂成、涿郡、勃海為河閒國。丙辰,封皇弟壽為濟北王,開為河閒王,淑為城陽王,紹封故淮陽王疠子側為常山王。賜公卿以下至佐史錢布各有差。

己未,遣副校尉閻磐討北匈奴,取伊吾盧地。

丁卯,紹封故齊王晃子無忌為齊王,北海王睦子威為北海王。

車師前後王並遣子入侍。

月氏國遣兵攻西域長史班超,超擊降之。

六月辛卯,中山王焉薨。

秋七月乙卯,大將軍竇憲出屯涼州。九月,北匈奴遣使稱臣。

冬十月,遣行中郎將班固報命南單于。遣左谷蠡王師子出雞鹿塞,擊北匈奴於河雲北,大破之。

三年春正月甲子,皇帝加元服,賜諸侯王、公、將軍、特進、中二千石、列侯、宗室子孫在京師奉朝請者黃金,將、大夫、郎吏、從官帛。賜民爵及粟帛各有差,大酺五日。郡國中都官繫囚死罪贖縑,至司寇及亡命,各有差。庚辰,賜京師民酺,布兩戶共一匹。

二月,大將軍竇憲遣左校尉耿夔出居延塞,圍北單于於金微山,大破之,獲其母閼氏。

夏六月辛卯,尊皇太后母比陽公主為長公主。

辛丑,阜陵王种薨。

冬十月癸未,行幸長安。詔曰:「北狄破滅,名王仍降,西域諸國,納質內附,豈非祖宗迪哲重光之鴻烈歟?寤寐歎息,想望舊京。其賜行所過二千石長吏已下及三老、官屬錢帛,各有差;鰥、寡、孤、獨、篤缮、貧不能自存者粟,人三斛。」

十一月癸卯,祠高廟,遂有事十一陵。詔曰:「高祖功臣,蕭、曹為首,有傳世不絕之義。曹相國後容城侯無嗣。朕望長陵東門,見二臣之壟,循其遠節,每有感焉。忠義獲寵,古今所同。可遣使者以中牢祠,大鴻臚求近親宜為嗣者,須景風紹封,以章厥功。」

十二月,復置西域都護、騎都尉、戊己校尉官。

庚辰,至自長安,減弛刑徒從駕者刑五月。

四年春正月,北匈奴右谷蠡王於除鞬自立為單于,款塞乞降。遣大將軍左校尉耿夔授璽綬。

三月癸丑,司徒袁安薨。閏月丁丑,太常丁鴻為司徒。

夏四月丙辰,大將軍竇憲還至京師。

六月戊戌朔,日有食之。丙辰,郡國十三地震。

竇憲潛圖弒逆。庚申,幸北宮。詔收捕憲黨射聲校尉郭璜,璜子侍中舉,衛尉鄧疊,疊弟步兵校尉磊,皆下獄死。使謁者僕射收憲大將軍印綬,遣憲及弟篤、景就國,到皆自殺。

是夏,旱,蝗。

秋七月己丑,太尉宋由坐黨憲自殺。

八月辛亥,司空任隗薨。

癸丑,大司農尹睦為太尉,錄尚書事。

丁巳,賜公卿以下至佐史錢穀各有差。

冬十月己亥,宗正劉方為司空。

十二月壬辰,詔:「今年郡國秋稼為旱蝗所傷,其什四以上勿收田租、芻稿;有不滿者,以實除之。」

武陵零陵澧中蠻叛。燒當羌寇金城。

五年春正月乙亥,宗祀五帝於明堂,遂登靈臺,望雲物。大赦天下。

戊子,千乘王伉薨。

辛卯,封皇弟萬歲為廣宗王。

二月戊戌,詔有司省減內外廄及涼州諸苑馬。自京師離宮果園上林廣成囿悉以假貧民,恣得采捕,不收其稅。

丁未,詔曰:「去年秋麥入少,恐民食不足。其上尤貧不能自給者戶口人數。往者郡國上貧民,以衣履釜鬵為貲,而豪右得其饒利。詔書實覈,欲有以益之,而長吏不能躬親,反更徵召會聚,令失農作,愁擾百姓。若復有犯者,二千石先坐。」

甲寅,太傅鄧彪薨。

戊午,隴西地震。

三月戊子,詔曰:「選舉良才,為政之本。科別行能,必由鄉曲。而郡國舉吏,不加簡擇,故先帝明敕在所,令試之以職,乃得充選。又德行尤異,不須經職者,別署狀上。而宣布以來,出入九年,二千石曾不承奉,恣心從好,司隸、刺史訖無糾察。今新蒙赦令,且復申敕,後有犯者,顯明其罰。在位不以選舉為憂,督察不以發覺為負,非獨州郡也。是以庶官多非其人。下民被姦邪之傷,由法不行故也。」

庚寅,遣使者分行貧民,舉實流冗,開倉賑稟三十餘郡。

夏四月壬子,封阜陵王种兄魴為阜陵王。

六月丁酉,郡國三雨雹。

秋九月辛酉,廣宗王萬歲薨,無子,國除。

匈奴單于於除鞬叛,遣中郎將任尚討滅之。

壬午,令郡縣勸民蓄蔬食以助五穀。其官有陂池,令得采取,勿收假稅二歲。

冬十月辛未,太尉尹睦薨。十一月乙丑,太僕張酺為太尉。

是歲,武陵郡兵破叛蠻,降之。護羌校尉貫友討燒當羌,羌乃遁去。南單于安國叛,骨都侯喜斬之。

六年春正月,永昌徼外夷遣使譯獻犀牛、大象。

己卯,司徒丁鴻薨。

二月乙未,遣謁者分行稟貸三河、兗、冀、青州貧民。

許侯馬光自殺。

丁未,司空劉方為司徒,太常張奮為司空。

三月庚寅,詔流民所過郡國皆實稟之,其有販賣者勿出租稅,又欲就賤還歸者,復一歲田租、更賦。

丙寅,詔曰:「朕以眇末,承奉鴻烈。陰陽不和,水旱違度,濟河之域,凶饉流亡,而未獲忠言至謀,所以匡救之策。寤寐永歎,用思孔疚。惟官人不得於上,黎民不安于下,有司不念寬和,而競為苛刻,覆案不急,以妨民事,甚非所以上當天心,下濟元元也。思得忠良之士,以輔朕之不逮。其令三公、中二千石、二千石、內郡守相舉賢良方正、能直言極諫之士各一人。昭巖穴,披幽隱,遣詣公車,朕將悉聽焉。」帝乃親臨策問,選補郎吏。

夏四月,蜀郡徼外羌率種人遣使內附。

五月,城陽王淑薨。無子,國除。

六月己酉,初令伏閉盡日。

秋七月,京師旱。詔中都官徒各除半刑,謫其未竟,五月已下皆免遣。丁巳,幸洛陽寺,錄囚徒,舉冤獄。收洛陽令下獄抵罪,司隸校尉、河南尹皆左降。未及還宮而澍雨。

西域都護班超大破焉耆、尉犁,斬其王。自是西域降服,納質者五十餘國。

南單于安國從弟子逢侯率叛胡亡出塞。九月癸丑,以光祿勳鄧鴻行車騎將軍事,與越騎校尉馮柱、行度遼將軍朱徽、使匈奴中郎將杜崇討之。冬十一月,護烏桓校尉任尚率烏桓、鮮卑,大破逢侯,馮柱遣兵追擊。復之。

詔以勃海郡屬冀州。

武陵漊中蠻叛,郡兵討平之。

七年春正月,行車騎將軍鄧鴻、度遼將軍朱徽、中郎將杜崇皆下獄死。

夏四月辛亥朔,日有食之。帝引見公卿問得失,令將、大夫、御史、謁者、博士、議郎、郎官會廷中,各言封事。詔曰:「元首不明,化流無良,政失於民,謫見于天。深惟庶事,五教在寬,是以舊典因孝廉之舉,以求其人。有司詳選郎官寬博有謀才任典城者三十人。」既而悉以所選郎出補長、相。

五月辛卯,改千乘國為樂安國。

六月丙寅,沛王定薨。

秋七月乙巳,易陽地裂。九月癸卯,京師地震。

八年春二月己丑,立貴人陰氏為皇后。賜天下男子爵,人二級,三老、孝悌、力田三級,民無名數及流民欲占者一級;鰥、寡、孤、獨、篤缮、貧不能自存者粟,人五斛。

夏四月癸亥,樂成王黨薨。

甲子,詔賑貸并州四郡貧民。

五月,河內、陳留蝗。

南匈奴右溫禺犢王叛,為寇。秋七月,行度遼將軍龐奮、越騎校尉馮柱追討之,斬右溫禺犢王。

車師後王叛,擊其前王。

八月辛酉,飲酎。詔郡國中都官繫囚減死一等,詣敦煌戍。其犯大逆,募下蠶室;其女子宮。自死罪已下,至司寇及亡命者入贖,各有差。

九月,京師蝗。吏民言事者,多歸責有司。詔曰:「蝗蟲之異,殆不虛生,萬方有罪,在予一人,而言事者專咎自下,非助我者也。朕寤寐恫瘝,思弭憂釁。昔楚嚴無災而懼,成王出郊而反風。將何以匡朕不逮,以塞災變?百僚師尹勉修厥職,刺史、二千石詳刑辟,理冤虐,恤鰥寡,矜孤弱,思惟致災興蝗之咎。」

庚子,復置廣陽郡。

冬十月乙丑,北海王威有罪自殺。

十二月辛亥,陳王羨薨。

丁巳,南宮宣室殿火。

九年春正月,永昌徼外蠻夷及撣國重譯奉貢。

三月庚辰,隴西地震。

癸巳,濟南王康薨。

西域長史王林擊車師後王,斬之。

夏四月丁卯,封樂成王黨子巡為樂成王。

六月,蝗、旱。戊辰,詔:「今年秋稼為蝗蟲所傷,皆勿收租、更、芻稿;若有所損失,以實除之,餘當收租者亦半入。其山林饒利,陂池漁採,以贍元元,勿收假稅。」秋七月,蝗蟲飛過京師。

八月,鮮卑寇肥如,遼東太守祭參下獄死。

閏月辛巳,皇太后竇氏崩。丙申,葬章德皇后。

燒當羌寇隴西,殺長吏,遣行征西將軍劉尚、越騎校尉趙世等討破之。

九月庚申,司徒劉方策免,自殺。

甲子,追尊皇妣梁貴人為皇太后。冬十月乙酉,改葬恭懷梁皇后于西陵。

十一月癸卯,光祿勳河南呂蓋為司徒。十二月丙寅,司空張奮罷。壬申,太僕韓稜為司空。

己丑,復置若盧獄官。

十年春三月壬戌,詔曰:「隄防溝渠,所以順助地理,通利壅塞。今廢慢懈弛,不以為負。刺史、二千石其隨宜疏導。勿因緣妄發,以為煩擾,將顯行其罰。」

夏五月,京師大水。

秋七月己巳,司空韓稜薨。八月丙子,太常太山巢堪為司空。

九月庚戌,復置廩犧官。

冬十月,五州雨水。

十二月,燒當羌豪迷唐等率種人詣闕貢獻。

戊寅、梁王暢薨。

十一年春二月,遣使循行郡國,稟貸被災害不能自存者,令得漁采山林池澤,不收假稅。

丙午,詔郡國中都官徒及篤缮老小女徒各除半刑,其未竟三月者,皆免歸田里。

夏四月丙寅,大赦天下。

己巳,復置右校尉官。

秋七月辛卯,詔曰:「吏民踰僭,厚死傷生,是以舊令節之制度。頃者貴戚近親,百僚師尹,莫肯率從,有司不舉,怠放日甚。又商賈小民,或忘法禁,奇巧靡貨,流積公行。其在位犯者,當先舉正。巿道小民,但且申明憲綱,勿因科令,加虐羸弱。」

十二年春二月,旄牛徼外白狼、貗薄夷率種人內屬。

詔貸被災諸郡民種糧。賜下貧、鰥、寡、孤、獨、不能自存者,及郡國流民,聽入陂池漁采,以助蔬食。

三月丙申,詔曰:「比年不登,百姓虛匱。京師去冬無宿雪,今春無澍雨,黎民流離,困於道路。朕痛心疾首,靡知所濟。『瞻仰昊天,何辜今人?』三公朕之腹心,而未獲承天安民之策。數詔有司,務擇良吏。今猶不改,競為苛暴,侵愁小民,以求虛名,委任下吏,假埶行邪。是以令下而姦生,禁至而詐起。巧法析律,飾文增辭,貨行於言,罪成乎手,朕甚病焉,公卿不思助明好惡,將何以救其咎罰?咎罰既至,復令災及小民。若上下同心,庶或有瘳。其賜天下男子爵,人二級,三老、孝悌、力田三級,民無名數及流民欲占者人一級;鰥、寡、孤、獨、篤缮、貧不能自存者粟,人三斛。」

壬子,賜博士員弟子在太學者布,人三匹。

夏四月,日南象林蠻夷反,郡兵討破之。

閏月,脤貸敦煌、張掖、五原民下貧者穀。

戊辰,賑歸山崩。

六月,舞陽大水,賜被水災尤貧者穀,人三斛。

秋七月辛亥朔,日有食之。

九月戊午,太尉張酺免。丙寅,大司農張禹為太尉。

冬十一月,西域蒙奇、兜勒二國遣使內附,賜其王金印紫綬。

是歲,燒當羌復叛。

十三年春正月丁丑,帝幸東觀,覽書林,閱篇籍,博選術蓺之士以充其官。

二月,任城王尚薨。

丙午,賑貸張掖、居延、朔方、日南貧民及孤、寡、羸弱不能自存者。

秋八月,詔象林民失農桑業者,賑貸種糧,稟賜下貧穀食。

己亥,北宮盛饌門閣火。

護羌校尉周鮪擊燒當羌,破之。

荊州雨水。九月壬子,詔曰:「荊州比歲不節,今茲淫水為害,餘雖頗登,而多不均浹,深惟四民農食之本,慘然懷矜。其令天下半入今年田租、芻稿;有宜以實除者,如故事。貧民假種食,皆勿收責。」

冬十一月,安息國遣使獻師子及條枝大爵。

丙辰,詔曰:「幽、并、涼州戶口率少,邊役眾劇,束脩良吏,進仕路狹。撫接夷狄,以人為本。其令緣邊郡口十萬以上歲舉孝廉一人,不滿十萬二歲舉一人,五萬以下三歲舉一人。」

鮮卑寇右北平,遂入漁陽,漁陽太守擊破之。

戊辰,司徒呂蓋罷。十二月丁丑,光祿勳魯恭為司徒。

辛卯,巫蠻叛,寇南郡。

十四年春二月乙卯,東海王政薨。

繕修故西海郡,徙金城西部都尉以戍之。

三月戊辰,臨辟雍,饗射,大赦天下。

夏四月,遣使者督荊州兵討巫蠻,破降之。

庚辰,賑貸張掖、居延、敦煌、五原、漢陽、會稽流民下貧穀,各有差。

五月丁未,初置象林將兵長史官。

六月辛卯,廢皇后陰氏,后父特進綱自殺。

秋七月甲寅,詔復象林縣更賦、田租、芻稿二歲。

壬子,常山王側薨。

是秋,三州雨水。冬十月甲申,詔:「兗、豫、荊州今年水雨淫過,多傷農功。其令被害什四以上皆半入田租、芻稿;其不滿者,以實除之。」

辛卯,立貴人鄧氏為皇后。

丁酉,司空巢堪罷。十一月癸卯,大司農徐防為司空。

是歲,初復郡國上計補郎官。

十五年春閏月乙未,詔流民欲還歸本而無糧食者,過所實稟之,疾病加致醫藥;其不欲還歸者,勿強。

二月,詔稟貸潁川、汝南、陳留、江夏、梁國、敦煌貧民。

夏四月甲子晦,日有食之。五月戊寅,南陽大風。

六月,詔令百姓鰥寡漁采陂池,勿收假稅二歲。

秋七月丙寅,濟南王錯薨。

復置涿郡故安鐵官。

九月壬午,南巡狩,清河王慶、濟北王壽、河閒王開並從。賜所過二千石長吏以下、三老、官屬及民百年者錢布,各有差。是秋,四州雨水。冬十月戊申,幸章陵,祠舊宅。癸丑,祠園廟,會宗室於舊廬,勞賜作樂。戊午,進幸雲夢,臨漢水而還。十一月甲申,車駕還宮,賜從臣及留者公卿以下錢布,各有差。

十二月庚子,琅邪王宇薨。

有司奏,以為夏至則微陰起,靡草死,可以決小事。

是歲,初令郡國以日北至案薄刑。

十六年春正月己卯,詔貧民有田業而以匱乏不能自農者,貸種糧。

二月己未,詔兗、豫、徐、冀四州比年雨多傷稼,禁沽酒。夏四月,遣三府掾分行四州,貧民無以耕者,為雇犁牛直。

五月壬午,趙王商薨。

秋七月,旱。戊午,詔曰:「今秋稼方穗而旱,雲雨不霑,疑吏行慘刻,不宣恩澤,妄拘無罪,幽閉良善所致。其一切囚徒於法疑者勿決,以奉秋令。方察煩苛之吏,顯明其罰。」

辛酉,司徒魯恭免。庚午,光祿勳張酺為司徒。

辛巳,詔令天下皆半入今年田租、芻稿;其被災害者,以實除之。貧民受貸種糧及田租、芻稿、皆勿收責。

八月己酉,司徒張酺薨。冬十月辛卯,司空徐防為司徒,大鴻臚陳寵為司空。

十一月己丑,行幸緱氏,登百岯山,賜百官從臣布,各有差。

北匈奴遣使稱臣貢獻。

十二月,復置遼東西部都尉官。

元興元年春正月戊午,引三署郎召見禁中,選除七十五人,補謁者、長、相。

高句驪寇郡界。

夏四月庚午,大赦天下,改元元興。宗室以罪絕者,悉復屬籍。

五月癸酉,雍地裂。

秋九月,遼東太守耿夔擊貊人,破之。

冬十二月辛未,帝崩于章德前殿,年二十七。立皇子隆為皇太子。賜天下男子爵,人二級,三老、孝悌、力田人三級,民無名數及流民欲占者人一級;鰥、寡、孤、獨、篤缮、貧不能自存者粟,人三斛。

自竇憲誅後,帝躬親萬機。每有災異,輒延問公卿,極言得失。前後符瑞八十一所,自稱德薄,皆抑而不宣。舊南海獻龍眼、荔支,十里一置,五里一候,奔騰阻險,死者繼路。時臨武長汝南唐羌,縣接南海,乃上書陳狀。帝下詔曰:「遠國珍羞,本以薦奉宗廟。苟有傷害,豈愛民之本。其敕太官勿復受獻。」由是遂省焉。

論曰:自中興以後,逮于永元,雖頗有弛張,而俱存不擾,是以齊民歲增,聞土世廣。偏師出塞,則漠北地空;都護西指,則通譯四萬。豈其道遠三代,術長前世?將服叛去來,自有數也?

孝殤皇帝諱隆,和帝少子也。元興元年十二月辛未夜,即皇帝位,時誕育百餘日。尊皇后曰皇太后,太后臨朝。

北匈奴遣使稱臣,詣敦煌奉獻。

延平元年春正月辛卯,太尉張禹為太傅。司徒徐防為太尉,參錄尚書事,百官總己以聽。封皇兄勝為平原王。癸卯,光祿勳梁鮪為司徒。

三月甲申,葬孝和皇帝于慎陵,尊廟曰穆宗。

丙戌,清河王慶、濟北王壽、河閒王開、常山王章始就國。

夏四月庚申,詔罷祀官不在祀典者。

鮮卑寇漁陽,漁陽太守張顯追擊,戰沒。

丙寅,以虎賁中郎將鄧騭為車騎將軍。

司空陳寵薨。

五月辛卯,皇太后詔曰:「皇帝幼沖,承統鴻業,朕且權佐助聽政,兢兢寅畏,不知所濟。深惟至治之本,道化在前,刑罰在後。將稽中和,廣施慶惠,與吏民更始。其大赦天下。自建武以來諸犯禁錮,詔書雖解,有司持重,多不奉行,其皆復為平民。」

壬辰,河東垣山崩。

六月丁未,太常尹勤為司空。

郡國三十七雨水。己未,詔曰:「自夏以來,陰雨過節,飕氣不效,將有厥咎。寤寐憂惶,未知所由。昔夏后惡衣服,菲飲食,孔子曰『吾無閒然』。今新遭大憂,且歲節未和,徹膳損服,庶有補焉。其減太官、導官、尚方、內署諸服御珍膳靡麗難成之物。」

丁卯,詔司徒、大司農、長樂少府曰:「朕以無德,佐助統政,夙夜經營,懼失厥衷。思惟治道,由近及遠,先內後外。自建武之初以至于今,八十餘年,宮人歲增,房御彌廣。又宗室坐事沒入者,猶託名公族,甚可愍焉。今悉免遣,及掖庭宮人,皆為庶民,以抒幽隔鬱滯之情。諸官府、郡國、王侯家奴婢姓劉及疲缮羸老,皆上其名,務令實悉。」

秋七月庚寅,敕司隸校尉、部刺史曰:「夫天降災戾,應政而至。閒者郡國或有水災,妨害秋稼。朝廷惟咎,憂惶悼懼。而郡國欲獲豐穰虛飾之譽,遂覆蔽災害,多張墾田,不揣流亡,競增戶口,掩匿盜賊,令姦惡無懲,署用非次,選舉乖宜,貪苛慘毒,延及平民。刺史垂頭塞耳,阿私下比,『不畏于天,不愧于人』。假貸之恩,不可數恃,自今以後,將糾其罰。二千石長吏其各實覈所傷害,為除田租、芻稿。」

八月辛亥,帝崩。癸丑,殯于崇德前殿。年二歲。

贊曰:孝和沈烈,率由前則。王赫自中,賜命彊慝。抑沒祥符,登顯時德。殤世何早,平原弗克。


\end{pinyinscope}