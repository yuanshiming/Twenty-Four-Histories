\article{郭符許列傳}

\begin{pinyinscope}
郭太字林宗,太原界休人也。家世貧賤。早孤,母欲使給事縣廷。林宗曰:「大丈夫焉能處斗筲之役乎?」遂辭。就成皋屈伯彥學,三年業畢,博通墳籍。善談論,美音制。乃游於洛陽。始見河南尹李膺,膺大奇之,遂相友善,於是名震京師。後歸鄉里,衣冠諸儒送至河上,車數千兩。林宗唯與李膺同舟而濟,眾賓望之,以為神仙焉。

司徒黃瓊辟,太常趙典舉有道。或勸林宗仕進者,對曰:「吾夜觀乾象,晝察人事,天之所廢,不可支也。」遂並不應。性明知人,好獎訓士類。身長八尺,容貌魁偉,褒衣博帶,周遊郡國。嘗於陳梁閒行遇雨,巾一角墊,時人乃故折巾一角,以為「林宗巾」。其見慕皆如此。或問汝南范滂曰:「郭林宗何如人?」滂曰:「隱不違親,貞不絕俗,天子不得臣,諸侯不得友,吾不知其它。」後遭母憂,有至孝稱。林宗雖善人倫,而不為危言覈論,故宦官擅政而不能傷也。及黨事起,知名之士多被其害,唯林宗及汝南袁閎得免焉。遂閉門教授,弟子以千數。

建寧元年,太傅陳蕃、大將軍竇武為閹人所害,林宗哭之於野,慟。既而歎曰:「『人之云亡,邦國殄瘁』。『瞻烏爰止,不知于誰之屋』耳。」

明年春,卒于家,時年四十二。四方之士千餘人,皆來會葬。同志者乃共刻石立碑,蔡邕為其文,既而謂涿郡盧植曰:「吾為碑銘多矣,皆有慚德,唯郭有道無愧色耳。」

其獎拔士人,皆如所鑒。後之好事,或附益增張,故多華辭不經,又類卜相之書。今錄其章章效於事者,著之篇末。

左原者,陳留人也。為郡學生,犯法見斥。林宗嘗遇諸路,為設酒肴以慰之。謂曰:「昔顏涿聚梁甫之巨盜,段干木晉國之大駔,卒為齊之忠臣,魏之名賢。蘧瑗、顏回尚不能無過,況其餘乎?慎勿恚恨,責躬而已。」原納其言而去。或有譏林宗不絕惡人者。對曰:「人而不仁,疾之以甚,亂也。」原後忽更懷忿,結客欲報諸生。其日林宗在學,原愧負前言,因遂罷去。後事露,眾人咸謝服焉。

茅容字季偉,陳留人也。年四十餘,耕於野,時與等輩避雨樹下,眾皆夷踞相對,容獨危坐愈恭。林宗行見之而奇其異,遂與共言,因請寓宿。旦日,容殺雞為饌,林宗謂為己設,既而以供其母,自以草蔬與客同飯。林宗起拜之曰:「卿賢乎哉!」因勸令學,卒以成德。

孟敏字叔達,鉅鹿楊氏人也。客居太原。荷甑墯地,不顧而去。林宗見而問其意。對曰:「甑以破矣,視之何益?」林宗以此異之,因勸令遊學。十年知名,三公俱辟,並不屈云。

庾乘字世遊,潁川鄢陵人也。少給事縣廷為門士。林宗見而拔之,勸遊學宮,遂為諸生傭。後能講論,自以卑第,每處下坐,諸生博士皆就讎問,由是學中以下坐為貴。後徵辟並不起,號曰「徵君」。

宋果字仲乙,扶風人也。性輕悍,憙與人報讎,為郡縣所疾。林宗乃訓之義方,懼以禍敗。果感悔,叩頭謝負,遂改節自敕。後以烈氣聞,辟公府,侍御史、并州刺史,所在能化。

賈淑字子厚,林宗鄉人也。雖世有冠冕,而性險害,邑里患之。林宗遭母憂,淑來修弔,既而鉅鹿孫威直亦至。威直以林宗賢而受惡人弔,心怪之,不進而去。林宗追而謝之曰:「賈子厚誠實凶德,然洗心向善。仲尼不逆互鄉,故吾許其進也。」淑聞之,改過自厲,終成善士。鄉里有憂患者,淑輒傾身營救,為州閭所稱。

史叔賓者,陳留人也。少有盛名。林宗見而告人曰:「牆高基下,雖得必失。」後果以論議阿枉敗名云。

黃允字子艾,濟陰人也。以侊才知名。林宗見而謂曰:「卿有絕人之才,足成偉器。然恐守道不篤,將失之矣。」後司徒袁隗欲為從女求姻,見允而歎曰:「得婿如是足矣。」允聞而黜遣其妻夏侯氏。婦謂姑曰:「今當見棄,方與黃氏長辭,乞一會親屬,以展離訣之情。」於是大集賓客三百餘人,婦中坐,攘袂數允隱匿穢惡十五事,言畢,登車而去。允以此廢於時。

謝甄字子微,汝南召陵人也。與陳留邊讓並善談論,俱有盛名。每共候林宗,未嘗不連日達夜。林宗謂門人曰:「二子英才有餘,而並不入道,惜乎!」甄後不拘細行,為時所毀。讓以輕侮曹操,操殺之。

王柔字叔優,弟澤,字季道,林宗同郡晉陽縣人也。兄弟總角共候林宗,以訪才行所宜。林宗曰:「叔優當以仕進顯,季道當以經術通,然違方改務,亦不能至也。」後果如所言,柔為護匈奴中郎將,澤為代郡太守。

又識張孝仲芻牧之中,知范特祖郵置之役,召公子、許偉康並出屠酤,司馬子威拔自卒伍,及同郡郭長信、王長文、韓文布、李子政、曹子元、定襄周康子、西河王季然、雲中丘季智、郝禮真等六十人,並以成名。

論曰:莊周有言,人情險於山川,以其動靜可識,而沈阻難徵。故深厚之性,詭於情貌;「則哲」之鑒,惟帝所難。而林宗雅俗無所失,將其明性特有主乎?然而遜言危行,終亨時晦,恂恂善導,使士慕成名,雖墨、孟之徒,不能絕也。

符融字偉明,陳留浚儀人也。少為都官吏,恥之,委去。後遊太學,師事少府李膺。膺風性高簡,每見融,輒絕它賓客,聽其言論。融幅巾奮褎,談辭如雲,膺每捧手歎息。郭林宗始入京師,時入莫識,融一見嗟服,因以介於李膺,由是知名。

時漢中晉文經、梁國黃子艾,並恃其才智,炫曜上京,臥託養疾,無所通接。洛中士大夫好事者,承其聲名,坐門問疾,猶不得見。三公所辟召者,輒以詢訪之,隨所臧否,以為與奪。融察其非真,乃到太學,并見李膺曰:「二子行業無聞,以豪桀自置,遂使公卿問疾,王臣坐門。融恐其小道破義,空譽違實,特宜察焉。」膺然之。二人自是名論漸衰,賓徒稍省,旬日之閒,慚歎逃去。後果為輕薄子,並以罪廢棄。

融益以知名。州郡禮請,舉孝廉,公府連辟,皆不應。太守馮岱有名稱,到官,請融相見。融一往,薦達郡士范冉、韓卓、孔骸等三人,因辭病自絕。會有黨事,亦遭禁錮。

妻亡,貧無殯斂,鄉人欲為具棺服,融不肯受。曰:「古之亡者,棄之中野。唯妻子可以行志,但即土埋藏而已。」

融同郡田盛,字仲嚮,與郭林宗同好,亦名知人,優遊不仕,並以壽終。

許劭字子將,汝南平輿人也。少峻名節,好人倫,多所賞識。若樊子昭、和陽士者,並顯名於世。故天下言拔士者,咸稱許、郭。

初為郡功曹,太守徐璆甚敬之。府中聞子將為吏,莫不改操飾行。同郡袁紹,公族豪俠,去濮陽令歸,車徒甚盛,將入郡界,乃謝遣賓客,曰:「吾輿服豈可使許子將見。」遂以單車歸家。

劭嘗到潁川,多長者之遊,唯不候陳寔。又陳蕃喪妻還葬,鄉人必至,而劭獨不往。或問其故,劭曰:「太后道廣,廣則難周;仲舉性峻,峻則少通。故不造也。」其多所裁量若此。

曹操微時,常卑辭厚禮,求為己目。劭鄙其人而不肯對,操乃伺隙脅劭,劭不得已,曰:「君清平之姦賊,亂世之英雄。」操大悅而去。

劭從祖敬,敬子訓,訓子相,並為三公,相以能諂事宦官,故自致台司封侯,數遣請劭。劭惡其薄行,終不候之。

劭邑人李逵,壯直有高氣,劭初善之,而後為隙,又與從兄靖不睦,時議以此少之。初,劭與靖俱有高名,好共覈論鄉黨人物,每月輒更其品題,故汝南俗有「月旦評」焉。

司空楊彪辟,舉方正、敦樸,徵,皆不就。或勸劭仕,對曰:「方今小人道長,王室將亂,吾欲避地淮海,以全老幼。」乃南到廣陵。徐州刺史陶謙禮之甚厚。劭不自安,告其徒曰:「陶恭祖外慕聲名,內非真正。待吾雖厚,其埶必薄。不如去之。」遂復投揚州刺史劉繇於曲阿。其後陶謙果捕諸寓士。及孫策平吳,劭與繇南奔豫章而卒,時年四十六。

兄虔亦知名,汝南人稱平輿淵有二龍焉。

贊曰:林宗懷寶,識深甄藻。明發周流,永言時道。符融鑒真,子將人倫。守節好恥,並亦逡巡。


\end{pinyinscope}