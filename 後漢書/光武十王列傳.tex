\article{光武十王列傳}

\begin{pinyinscope}
光武皇帝十一子:郭皇后生東海恭王彊、沛獻王輔

、濟南安王康、阜陵質王延、中山簡王焉,許美人生楚王英,光烈皇后生顯宗、東平憲王蒼、廣陵思王荊、臨淮懷公衡、琅邪孝王京。

東海恭王彊。建武二年,立母郭氏為后,彊為皇太子。十七年而郭后廢,彊常慼慼不自安,數因左右及諸王陳其懇誠,願備蕃國。光武不忍,遲回者數歲,乃許焉。十九年,封為東海王,二十八年,就國。帝以彊廢不以過,去就有禮,故優以大封。兼食魯郡,合二十九縣。賜虎賁旄頭,宮殿設鐘虡之縣,擬於乘輿。彊臨之國,數上書讓還東海,又因皇太子固辭。帝不許,深嘉歎之,以彊章宣示公卿。初,魯恭王好宮室,起靈光殿,甚壯麗,是時猶存,故詔彊都魯。中元元年入朝,從封岱山,因留京師。明年春,帝崩。冬,歸國。

永平元年,彊病,顯宗遣中常侍鉤盾令將太醫乘驛視疾,詔沛王輔、濟南王康、淮陽王延詣魯。及薨,臨命上疏謝曰:「臣蒙恩得備蕃輔,特受二國,宮室禮樂,事事殊異,巍巍無量,訖無報稱。而自脩不謹,連年被疾,為朝廷憂念。皇太后、陛下哀憐臣彊,感動發中,數遣使者太醫令丞方伎道術,絡驛不絕。臣伏惟厚恩,不知所言。臣內自省視,氣力羸劣,日夜浸困,終不復望見闕庭,奉承帷幄,孤負重恩,銜恨黃泉。身既夭命孤弱,復為皇太后、陛下憂慮,誠悲誠慚。息政,小人也,猥當襲臣後,必非所以全利之也。誠願還東海郡。天恩愍哀,以臣無男之故,處臣三女小國侯,此臣宿昔常計。今天下新罹大憂,惟陛下加供養皇太后,數進御餐。臣彊困劣,言不能盡意。願並謝諸王,不意永不復相見也。」天子覽書悲慟,從太后出幸津門亭發哀。使大司空持節護喪事,大鴻臚副,宗正、將作大匠視喪事,贈以殊禮,升龍、旄頭、鸞輅、龍旂、虎賁百人。詔楚王英、趙王栩、北海王興、館陶公主、比陽公主及京師親戚四姓夫人、小侯皆會葬。帝追惟彊深執謙儉,不欲厚葬以違其意,於是特詔中常侍杜岑及東海傅相曰:「王恭謙好禮,以德自終,遣送之物,務從約省,衣足斂形,茅車瓦器,物減於制,以彰王卓爾獨行之志。將作大匠留起陵廟。」

彊立十八年,年三十四。子靖王政嗣。政淫欲薄行。後中山簡王薨,政詣中山會葬,私取簡王姬徐妃,又盜迎掖庭出女。豫州刺史、魯相奏請誅政,有詔削薛縣。

立四十四年薨,子頃王肅嗣。永元十六年,封肅弟二十一人皆為列侯。肅性謙儉,循恭王法度。永初中,以西羌未平,上錢二千萬。元初中,復上縑萬匹,以助國費,鄧太后下詔褒納焉。

立二十三年薨,子孝王臻嗣。永建二年,封臻二弟敏、儉為鄉侯。臻及弟蒸鄉侯儉並有篤行,母卒,皆吐血毀眥。至服練紅,兄弟追念初喪父,幼小,哀禮有闕,因復重行喪制。臻性敦厚有恩,常分租秩賑給諸父昆弟。國相籍褒具以狀聞,順帝美之,制詔大將軍、三公、大鴻臚曰:「東海王臻以近蕃之尊,少襲王爵,膺受多福,未知艱難,而能克己率禮,孝敬自然,事親盡愛,送終竭哀,降儀從士,寢苫三年。和睦兄弟,恤養孤弱,至孝純備,仁義兼弘,朕甚嘉焉。夫勸善厲俗,為國所先。曩者東平孝王敞兄弟行孝,喪母如禮,有增戶之封。《詩》云:『永世克孝,念茲皇祖。』今增臻封五千戶,儉五百戶,光啟土宇,以酬厥德。」

立三十一年薨,子懿王祗嗣。初平四年,遣子琬至長安奉章,獻帝封琬汶陽侯,拜為平原相。

祗立四十四年薨,子羡嗣。二十年,魏受禪,以為崇德侯。

沛獻王輔,建武十五年封右馮翊公。十七年,郭后廢為中山太后,故徙輔為中山王,并食常山郡。二十年,復徙封沛王。

時禁網尚疏,諸王皆在京師,競脩名譽,爭禮四方賓客。壽光侯劉鯉,更始子也,得幸於輔。鯉怨劉盆子害其父,因輔結客,報殺盆子兄故式侯恭,輔坐繫詔獄,三日乃得出。自是後,諸王賓客多坐刑罰,各循法度。二十八年,就國。中元二年,封輔子寶為沛侯。永平元年,封寶弟嘉為僮侯。

輔矜嚴有法度,好經書,善說京氏易、孝經、論語傳及圖讖,作五經論,時號之曰沛王通論。在國謹節,終始如一,稱為賢王。顯宗敬重,數加賞賜。

立四十六年薨,子釐王定嗣。元和二年,封定弟十二人為鄉侯。

定立十一年薨,子節王正嗣。元興元年,封正弟二人為縣侯。

正立十四年薨,子孝王廣嗣。有固疾。安帝詔廣祖母周領王家事。周明正有法禮,漢安中薨,順帝下詔曰:「沛王祖母太夫人周,秉心淑慎,導王以仁,使光祿大夫贈以妃印綬。」

廣立三十五年薨,子幽王榮嗣。立二十年薨,子孝王琮嗣。薨,子恭王曜嗣。薨,子契嗣;魏受禪,以為崇德侯。

楚王英,以建武十五年封為楚公,十七年進爵為王,二十八年就國。母許氏無寵,故英國最貧小。三十年,以臨淮之取慮、須昌二縣益楚國。自顯宗為太子時,英常獨歸附太子,太子特親愛之。及即位,數受賞賜。永平元年,特封英舅子許昌為龍舒侯。

英少時好游俠,交通賓客,晚節更喜黃老,學為浮屠齋戒祭祀。八年,詔令天下死罪皆入縑贖。英遣郎中令奉黃縑白紈三十匹詣國相曰:「託在蕃輔,過惡累積,歡喜大恩,奉送縑帛,以贖愆罪。」國相以聞。詔報曰:「楚王誦黃老之微言,尚浮屠之仁祠,絜齋三月,與神為誓,何嫌何疑,當有悔吝?其還贖,以助伊蒲塞桑門之盛饌。」因以班示諸國中傅。英後遂大交通方士,作金龜玉鶴,刻文字以為符瑞。

十三年,男子燕廣告英與漁陽王平、顏忠等造作圖書,有逆謀,事下案驗。有司奏英招聚姦猾,造作圖讖,擅相官秩,置諸侯王公將軍二千石,大逆不道,請誅之。帝以親親不忍,乃廢英,徙丹陽涇縣,賜湯沐邑五百戶。遣大鴻臚持節護送,使伎人奴婢妓士鼓吹悉從,得乘輜軿,持兵弩,行道射獵,極意自娛。男女為侯主者,食邑如故。楚太后勿上璽綬,留住楚宮。

明年,英至丹陽,自殺。立三十三年,國除。詔遣光祿大夫持節弔祠,贈賵如法,加賜列侯印綬,以諸侯禮葬於涇。遣中黃門占護其妻子。悉出楚官屬無辭語者。制詔許太后曰:「國家始聞楚事,幸其不然。既知審實,懷用悼灼,庶欲宥全王身,令保卒天年,而王不念顧太后,竟不自免。此天命也,無可柰何!太后其保養幼弱,勉強飲食。諸許願王富貴,人情也。已詔有司,出其有謀者,令安田宅。」於是封燕廣為折姦侯。楚獄遂至累年,其辭語相連,自京師親戚諸侯州郡豪桀及考案吏,阿附相陷,坐死徙者以千數。

十五年,帝幸彭城,見許太后及英妻子於內殿,悲泣,感動左右。建初二年,肅宗封英子楚侯种,五弟皆為列侯,並不得置相臣吏人。元和三年,許太后薨,復遣光祿大夫持節弔祠,因留護喪事,賻錢五百萬。又遣謁者備王官屬迎英喪,改葬彭城,加王赤綬羽蓋華藻,如嗣王儀,追爵,謚曰楚厲侯。章和元年,帝幸彭城,見英夫人及六子,厚加贈賜。

种後徙封六侯。卒,子度嗣。度卒,子拘嗣,傳國于後。

濟南安王康,建武十五年封濟南公,十七年進爵為王,二十八年就國。三十年,以平原之祝阿、安德、朝陽、平昌、隰陰、重丘六縣益濟南國。中元二年,封康子德為東武城侯。

康在國不循法度,交通賓客。其後,人上書告康招來州郡姦猾漁陽顏忠、劉子產等,又多遺其繒帛,案圖書,謀議不軌。事下考,有司舉奏之,顯宗以親親故,不忍窮竟其事,但削祝阿、隰陰、東朝陽、安德、西平昌五縣。

建初八年,肅宗復還所削地,康遂多殖財貨,大修宮室,奴婢至千四百人,廄馬千二百匹,私田八百頃,奢侈恣欲,游觀無節。永元初,國傅何敞上疏諫康曰:「蓋聞諸侯之義,制節謹度,然後能保其社稷,和其民人。大王以骨肉之親,享食茅土,當施張政令,明其典法,出入進止,宜有期度,輿馬臺隸,應為科品。而今奴婢廄馬皆有千餘,增無用之口,以自蠶食。宮婢閉隔,失其天性,惑亂和氣。又多起內第,觸犯防禁,費以巨萬,而功猶未半。夫文繁者質荒,木勝者人亡,皆非所以奉禮承上,傳福無窮者也。故楚作章華以凶,吳興姑蘇而滅,景公千駟,民無稱焉。今數游諸第,晨夜無節,又非所以遠防未然,臨深履薄之法也。願大王修恭儉,遵古制,省奴婢之口,減乘馬之數,斥私田之富,節游觀之宴,以禮起居,則敞乃敢安心自保。惟大王深慮愚言。」康素敬重敞,雖無所嫌啎,然終不能改。

立五十九年薨,子簡王錯嗣。錯為太子時,愛康鼓吹妓女宋閏,使醫張尊招之不得,錯怒,自以劍刺殺尊。國相舉奏,有詔勿案。永元十一年,封錯弟七人為列侯。

錯立六年薨,子孝王香嗣。永初二年,封香弟四人為列侯。香篤行,好經書。初,叔父篤有罪不得封,西平昌侯昱坐法失侯,香乃上書分爵土封篤子丸、昱子嵩,皆為列侯。

香立二十年薨,無子,國絕。

永建元年,順帝立錯子阜陽侯顯為嗣,是為釐王。立三年薨,子悼王廣嗣。永建五年,封廣弟文為樂城亭侯。

廣立二十五年,永興元年薨,無子,國除。

東平憲王蒼,建武十五年封東平公,十七年進爵為王。

蒼少好經書,雅有智思,為人美須敘,要帶八圍,顯宗甚愛重之。及即位,拜為驃騎將軍,置長史掾史員四十人,位在三公上。

永平元年,封蒼子二人為縣侯。二年,以東郡之壽張、須昌,山陽之南平陽、稿、湖陵五縣益東平國。是時中興三十餘年,四方無虞,蒼以天下化平,宜修禮樂,乃與公卿共議定南北郊冠冕車服制度,及光武廟登歌八佾舞數,語在禮樂、輿服志。帝每巡狩,蒼常留鎮,侍衛皇太后。

四年春,車駕近出,觀覽城第,尋聞當遂校獵河內,蒼即上書諫曰:「臣聞時令,盛春農事,不聚眾興功。傳曰:『田獵不宿,食飲不享,出入不節,則木不曲直。』此失春令者也。臣知車駕今出,事從約省,所過吏人諷誦甘棠之德。雖然,動不以禮,非所以示四方也。惟陛下因行田野,循視稼穡,消搖仿佯,弭節而旋。至秋冬,乃振威靈,整法駕,備周衛,設羽旄。《詩》云:『抑抑威儀,惟德之隅。』臣不勝憤懣,伏自手書,乞詣行在所,極陳至誠。」帝覽奏,即還宮。

蒼在朝數載,多所隆益,而自以至親輔政,聲望日重,意不自安,上疏歸職曰:「臣蒼疲駑,特為陛下慈恩覆護,在家備教導之仁,升朝蒙爵命之首,制書褒美,班之四海,舉負薪之才,升君子之器。凡匹夫一介,尚不忘簞食之惠,況臣居宰相之位,同氣之親哉!宜當暴骸膏野,為百僚先,而愚頑之質,加以固病,誠羞負乘,辱汙輔將之位,將被詩人『三百赤紱』之刺。今方域晏然,要荒無儆,將遵上德無為之時也,文官猶可并省,武職尤不宜建。昔象封有鼻,不任以政,誠由愛深,不忍揚其過惡。前事之不忘,來事之師也。自漢興以來,宗室子弟無得在公卿位者。惟陛下審覽虞帝優養母弟,遵承舊典,終卒厚恩。乞上驃騎將軍印綬,退就蕃國,願蒙哀憐。」帝優詔不聽。其後數陳乞,辭甚懇切。五年,乃許還國,而不聽上將軍印綬。以驃騎長史為東平太傅,掾為中大夫,令史為王家郎。加賜錢五千萬,布十萬匹。

六年冬,帝幸魯,徵蒼從還京師。明年,皇太后崩。既葬,蒼乃歸國,特賜宮人奴婢五百人,布二十五萬匹,及珍寶服御器物。

十一年,蒼與諸王朝京師。月餘,還國。帝臨送歸宮,悽然懷思,乃遣使手詔國中傅曰:「辭別之後,獨坐不樂,因就車歸,伏軾而吟,瞻望永懷,實勞我心,誦及采菽,以增歎息。日者問東平王處家何等最樂,王言為善最樂,其言甚大,副是要腹矣。今送列侯印十九枚,諸王子年五歲已上能趨拜者,皆令帶之。」

十五年春,行幸東平,賜蒼錢千五百萬,布四萬匹。帝以所作光武本紀示蒼,蒼因上光武受命中興頌。帝甚善之,以其文典雅,特令校書郎賈逵為之訓詁。

肅宗即位,尊重恩禮踰於前世,諸王莫與為比。建初元年,地震,蒼上便宜,其事留中。帝報書曰:「丙寅所上便宜三事,朕親自覽讀,反覆數周,心開目明,曠然發矇。閒吏人奏事,亦有此言,但明智淺短,或謂儻是,復慮為非。何者?災異之降,緣政而見。今改元之後,年飢人流,此朕之不德感應所致。又冬春旱甚,所被尤廣,雖內用克責,而不知所定。得王深策,快然意解。詩不云乎:『未見君子,憂心忡忡;既見君子,我心則降。』思惟嘉謀,以次奉行,冀蒙福應。彰報至德,特賜王錢五百萬。」

後帝欲為原陵、顯節陵起縣邑,蒼聞之,遽上疏諫曰:「伏聞當為二陵起立郭邑,臣前頗謂道路之言,疑不審實,近令從官古霸問涅陽主疾,使還,乃知詔書已下。竊見光武皇帝躬履儉約之行,深睹始終之分,勤勤懇懇,以葬制為言,故營建陵地,具稱古典,詔曰『無為山陵,陂池裁令流水而已』。孝明皇帝大孝無違,奉承貫行。至於自所營創,尤為儉省,謙德之美,於斯為盛。臣愚以園邑之興,始自彊秦。古者丘隴且不欲其著明,豈況築郭邑,建都郛哉!上違先帝聖心,下造無益之功,虛費國用,動搖百姓,非所以致和氣,祈豐年也。又以吉凶俗數言之,亦不欲無故繕修丘墓,有所興起。考之古法則不合,稽之時宜則違人,求之吉凶復未見其福。陛下履有虞之至性,追祖禰之深思,然懼左右過議,以累聖心。臣蒼誠傷二帝純德之美,不暢於無窮也。惟蒙哀覽。」帝從而止。自是朝廷每有疑政,輒驛使諮問。蒼悉心以對,皆見納用。

三年,帝饗衛士於南宮,因從皇太后周行掖庭池閣,乃閱陰太后舊時器服,愴然動容,乃命留五時衣各一襲,及常所御衣合五十篋,餘悉分布諸王主及子孫在京師者各有差。特賜蒼及琅邪王京書曰:「中大夫奉使,親聞動靜,嘉之何已!歲月騖過,山陵浸遠,孤心悽愴,如何如何!閒饗衛士於南宮,因閱視舊時衣物,聞於師曰:『其物存,其人亡,不言哀而哀自至。』信矣。惟王孝友之德,亦豈不然!今送光烈皇后假紒帛巾各一,及衣一篋,可時奉瞻,以慰凱風寒泉之思,又欲令後生子孫得見先后衣服之製。今魯國孔氏,尚有仲尼車輿冠履,明德盛者光靈遠也。其光武皇帝器服,中元二年已賦諸國,故不復送。并遺宛馬一匹,血從前髆上小孔中出。常聞武帝歌天馬,霑赤汗,今親見其然也。頃反虜尚屯,將帥在外,憂念遑遑,未有閒寧。願王寶精神,加供養。苦言至戒,望之如渴。」

六年冬,蒼上疏求朝。明年正月,帝許之。特賜裝錢千五百萬,其餘諸王各千萬。帝以蒼冒涉寒露,遣謁者賜貂裘,及太官食物珍果,使大鴻臚竇固持節郊迎。帝乃親自循行邸第,豫設帷床,其錢帛器物無不充備。下詔曰:「伯父歸寧乃國,詩云叔父建爾元子,敬之至也。昔蕭相國加以不名,優忠賢也。況兼親尊者乎!其沛、濟南、東平、中山四王,讚皆勿名。」蒼既至,升殿乃拜,天子親荅之。其後諸王入宮,輒以輦迎,至省閤乃下。蒼以受恩過禮,情不自寧,上疏辭曰:「臣聞貴有常尊,賤有等威,卑高列序,上下以理。陛下至德廣施,慈愛骨肉,既賜奉朝請,咫尺天儀,而親屈至尊,降禮下臣,每賜讌見,輒興席改容,中宮親拜,事過典故。臣惶怖戰慄,誠不自安,每會見,踧踖無所措置。此非所以章示群下,安臣子也。」帝省奏歎息,愈褒貴焉。舊典,諸王女皆封鄉主,乃獨封蒼五女為縣公主。

三月,大鴻臚奏遣諸王歸國,帝特留蒼,賜以祕書、列僊圖、道術祕方。至八月飲酎畢,有司復奏遣蒼,乃許之。手詔賜蒼曰:「骨肉天性,誠不以遠近為親疏,然數見顏色,情重昔時。念王久勞,思得還休,欲署大鴻臚奏,不忍下筆,顧授小黃門,中心戀戀,惻然不能言。」於是車駕祖送,流涕而訣。復賜乘輿服御,珍寶輿馬,錢布以億萬計。

蒼還國,疾病,帝馳遣名醫,小黃門侍疾,使者冠蓋不絕於道。又置驛馬千里,傳問起居。明年正月薨,詔告中傅,封上蒼自建武以來章奏及所作書、記、賦、頌、七言、別字、歌詩,並集覽焉。遣大鴻臚持節,五官中郎將副監喪,及將作使者凡六人,令四姓小侯諸國王主悉會詣東平奔喪,賜錢前後一億,布九萬匹。及葬,策曰:「惟建初八年三月己卯,皇帝曰:咨王丕顯,勤勞王室,親受策命,昭于前世。出作蕃輔,克慎明德,率禮不越,傅聞在下。昊天不弔,不報上仁,俾屏余一人,夙夜煢煢,靡有所終。今詔有司加賜鸞輅乘馬,龍旂九旒,虎賁百人,奉送王行。匪我憲王,其孰離之!魂而有靈,保茲寵榮。嗚呼哀哉!」

立四十五年,子懷王忠嗣。明年,帝乃分東平國封忠弟尚為任城王,餘五人為列侯。

忠立十一年薨,子孝王敞嗣。元和三年,行東巡守,幸東平宮,帝追感念蒼,謂其諸子曰:「思其人,至其鄉;其處在,其人亡。」因泣下沾襟,遂幸蒼陵,為陳虎賁、鸞輅、龍旂,以章顯之,祠以太牢,親拜祠坐,哭泣盡哀,賜御劍于陵前。初,蒼歸國,驃騎時吏丁牧、周栩以蒼敬賢下士,不忍去之,遂為王家大夫,數十年事祖及孫。帝聞,皆引見於前,既愍其淹滯,且欲揚蒼德美,即皆擢拜議郎。牧至齊相,栩上蔡令。永元十年,封蒼孫梁為矜陽亭侯,敞弟六人為列侯。敞喪母至孝,國相陳珍上其行狀。永寧元年,鄧太后增邑五千戶,又封蒼孫二人為亭侯。

敞立四十八年薨,子頃王端嗣。立四十七年薨,子凱嗣;立四十一年,魏受禪,以為崇德侯。

論曰:孔子稱「貧而無諂,富而無驕,未若貧而樂,富而好禮者也。」若東平憲王,可謂好禮者也。若其辭至戚,去母后,豈欲苟立名行而忘親遺義哉!蓋位疑則隙生,累近則喪大,斯蓋明哲之所為歎息。嗚呼!遠隙以全忠,釋累以成孝,夫豈憲王之志哉!東海恭王遜而知廢,「為吳太伯,不亦可乎」!

任城孝王尚,元和元年封,食任城、亢父、樊三縣。

立十八年薨,子貞王安嗣。永元十四年,封母弟福為桃鄉侯。永初四年,封福弟亢為當塗鄉侯。安性輕易貪吝,數微服出入,游觀國中,取官屬車馬刀劍,下至衛士米肉,皆不與直。元初六年,國相行弘奏請廢之。安帝不忍,以一歲租五分之一贖罪。

安立十九年薨,子節王崇嗣。順帝時,羌虜數反,崇輒上錢帛佐邊費。及帝崩,復上錢三百萬助山陵用度,朝廷嘉而不受。立三十一年薨,無子,國絕。

延熹四年,桓帝立河閒孝王子恭為參戶亭侯博為任城王,以奉其祀。博有孝行,喪母服制如禮,增封三千戶。立十三年薨,無子,國絕。

熹平四年,靈帝復立河閒貞王遜新昌侯子佗為任城王,奉孝王後。立四十六年,魏受禪,以為崇德侯。

阜陵質王延,建武十五年封淮陽公,十七年進爵為王,二十八年就國。三十年,以汝南之長平、西華、新陽、扶樂四縣益淮陽國。

延性驕奢而遇下嚴烈。永平中,有上書告延與姬兄謝弇及姊館陶主婿駙馬都尉韓光招姦猾,作圖讖,祠祭祝詛。事下案驗,光、弇被殺,辭所連及,死徙者甚眾。有司奏請誅延。顯宗以延罪薄於楚王英,故特加恩,徙為阜陵王,食二縣。

延既徙封,數懷怨望。建初中,復有告延與子男魴造逆謀者,有司奏請檻車徵詣廷尉詔獄。肅宗下詔曰:「王前犯大逆,罪惡尤深,有同周之管、蔡,漢之淮南。經有正義,律有明刑。先帝不忍親親之恩,枉屈大法,為王受愆,群下莫不惑焉。今王曾莫悔悟,悖心不移,逆謀內潰,自子魴發,誠非本朝之所樂聞。朕惻然傷心,不忍致王于理,今貶爵為阜陵侯,食一縣。獲斯辜者,侯自取焉。於戲誡哉!」赦魴等罪勿驗,使謁者一人監護延國,不得與吏人通。

章和元年,行幸九江,賜延書與車駕會壽春。帝見延及妻子,愍然傷之,乃下詔曰:「昔周之爵封千有八百,而姬姓居半者,所以楨幹王室也。朕南巡,望淮、海,意在阜陵,遂與侯相見。侯志意衰落,形體非故,瞻省懷感,以喜以悲。今復侯為阜陵王,增封四縣,并前為五縣。」以阜陵下溼,徙都壽春,加賜錢千萬,布萬匹,安車一乘,夫人諸子賞賜各有差。明年入朝。

立五十一年薨,子殤王沖嗣。永元二年,下詔盡削除前班下延事。

沖立二年薨,無嗣。和帝復封沖兄魴,是為頃王。永元八年,封魴弟十二人為鄉、亭侯。

魴立三十年薨,子懷王恢嗣。延光三年,封恢兄弟五人為鄉、亭侯。

恢立十年薨,子節王代嗣。陽嘉二年,封代兄便親為勃袭亭侯。

代立十四年薨,無子,國絕。

建和元年,桓帝立勃袭亭侯便親為恢嗣,是為恭王。立十三年薨,子孝王統嗣。立八年薨,子王赦立;建安中薨,無子,國除。

廣陵思王荊,建武十五年封山陽公,十七年進爵為王。

荊性刻急隱害,有才能而喜文法。光武崩,大行在前殿,荊哭不哀,而作飛書,封以方底,令蒼頭詐稱東海王彊舅大鴻臚郭況書與彊曰:「君王無罪,猥被斥廢,而兄弟至有束縛入牢獄者。太后失職,別守北宮,及至年老,遠斥居邊,海內深痛,觀者鼻酸。及太后尸柩在堂,洛陽吏以次捕斬賓客,至有一家三尸伏堂者,痛甚矣!今天下有喪,弓弩張設甚備。閒梁松敕虎賁史曰:『吏以便宜見非,勿有所拘,封侯難再得也。』郎官竊悲之,為王寒心累息。今天下爭欲思刻賊王以求功,寧有量邪!若歸并二國之眾,可聚百萬,君王為之主,鼓行無前,功易於太山破雞子,輕於四馬載鴻毛,此湯、武兵也。今年軒轅星有白氣,星家及喜事者,皆云白氣者喪,軒轅女主之位。又太白前出西方,至午兵當起。又太子星色黑,至辰日輒變赤。夫黑為病,赤為兵,王努力卒事。高祖起亭長,陛下興白水,何況於王陛下長子,故副主哉!上以求天下事必舉,下以雪除沈沒之恥,報死母之讎。精誠所加,金石為開。當為秋霜,無為檻羊。雖欲為檻羊,又可得乎!竊見諸相工言王貴,天子法也。人主崩亡,閭閻之伍尚為盜賊,欲有所望,何況王邪!夫受命之君,天之所立,不可謀也。今新帝人之所置,彊者為右。願君王為高祖、陛下所志,無為扶蘇、將閭叫呼天也。」彊得書惶怖,即執其使,封書上之。

顯宗以荊母弟,祕其事,遣荊出止河南宮。時西羌反,荊不得志,冀天下因羌驚動有變,私迎能為星者與謀議。帝聞之,乃徙封荊廣陵王,遣之國。其後荊復呼相工謂曰:「我貌類先帝。先帝三十得天下,我今亦三十,可起兵未?」相者詣吏告之,荊惶恐,自繫獄。帝復加恩,不考極其事,下詔不得臣屬吏人,唯食租如故,使相、中尉謹宿衛之。荊猶不改。其後使巫祭祀祝詛,有司舉奏,請誅之,荊自殺。立二十九年死。帝憐傷之,賜謚曰思王。

十四年,封荊子元壽為廣陵侯,服王璽綬,食荊故國六縣;又封元壽弟三人為鄉侯。明年,帝東巡狩,徵元壽兄弟會東平宮,班賜御服器物,又取皇子輿馬,悉以與之。建初七年,肅宗詔元壽兄弟與諸王俱朝京師。

元壽卒,子商嗣。商卒,子條嗣,傳國于後。

臨淮懷公衡,建武十五年立,未及進爵為王而薨,無子,國除。

中山簡王焉,建武十五年封左馮翊公,十七年進爵為王。焉以郭太后少子故,獨留京師。三十年,徙封中山王。永平二年冬,諸王來會辟雍,事畢歸蕃,詔焉與俱就國,從以虎賁官騎。焉上疏辭讓,顯宗報曰:「凡諸侯出境,必備左右,故夾谷之會,司馬以從。今五國各官騎百人,稱娖前行,皆北軍胡騎,便兵善射,弓不空發,中必決眥。夫有文事必有武備,所以重蕃職也。王其勿辭。」帝以焉郭太后偏愛,特加恩寵,獨得往來京師。十五年,焉姬韓序有過,焉縊殺之,國相舉奏,坐削安險縣。元和中,肅宗復以安險還中山。

立五十二年,永元二年薨。自中興至和帝時,皇子始封薨者,皆賻錢三千萬,布三萬匹;嗣王薨,賻錢千萬、布萬匹。是時竇太后臨朝,竇憲兄弟擅權,太后及憲等,東海出也,故睦於焉而重於禮,加賻錢一億。詔濟南、東海二王皆會。大為修冢塋,開神道,平夷吏人冢墓以千數,作者萬餘人。發常山、鉅鹿、涿郡柏黃腸雜木,三郡不能備,復調餘州郡工徒及送致者數千人。凡徵發搖動六州十八郡,制度餘國莫及。

子夷王憲嗣。永元四年,封憲弟十一人為列侯。

憲立二十二年薨,子孝王弘嗣。永寧元年,封弘二弟為亭侯。

弘立二十八年薨,子穆王暢嗣。永和六年,封暢弟荊為南鄉侯。

暢立三十四年薨,子節王稚嗣,無子,國除。

琅邪孝王京,建武十五年封琅邪公,十七年進爵為王。

京性恭孝,好經學,顯宗尤愛幸,賞賜恩寵殊異,莫與為比。永平二年,以太山之蓋、南武陽、華,東萊之昌陽、盧鄉、東牟六縣益琅邪。五年,乃就國。光烈皇后崩,帝悉以太后遺金寶財物賜京。京都莒,好修宮室,窮極伎巧,殿館壁帶皆飾以金銀。數上詩賦頌德,帝嘉美,下之史官。京國中有城陽景王祠,吏人奉祠。神數下言,宮中多不便利,京上書願徙宮開陽,以華、蓋、南武陽、厚丘、贛榆五縣易東海之開陽、臨沂,肅宗許之。立三十一年薨,葬東海即丘廣平亭,有詔割亭屬開陽。

子夷王宇嗣。建初七年,封宇弟十三人為列侯。元和元年,封孝王孫二人為列侯。

宇立二十年薨,子恭王壽嗣。永初元年,封壽弟八人為列侯。

立十七年薨,子貞王尊嗣。延光二年,封尊弟四人為鄉侯。

尊立十八年薨,子安王據嗣。永和五年,封據弟三人為鄉侯。

據立四十七年薨,子順王容嗣。初平元年,遣弟邈至長安奉章貢獻,帝以邈為九江太守,封陽都侯。

容立八年薨,國絕。

初,邈至長安,盛稱東郡太守曹操忠誠於帝,操以此德於邈。建安十一年,復立容子熙為王。在位十一年,坐謀欲過江,被誅,國除。

贊曰:光武十子,胙土分王。沛獻尊節,楚英流放。延既怨詛,荊亦觖望。濟南陰謀,琅邪驕宕。中山、臨淮,無聞夭喪。東平好善,辭中委相。謙謙恭王,寔惟三讓。


\end{pinyinscope}