\article{宦者列傳}

\begin{pinyinscope}
《易》曰:「天垂象,聖人則之。」宦者四星,在皇位之側,故周禮置官,亦備其數。閽者守中門之禁,寺人掌女宮之戒。又云「王之正內者五人」。月令:「仲冬,命閹尹審門閭,謹房室。」詩之小雅,亦有巷伯刺讒之篇。然宦人之在王朝者,其來舊矣。將以其體非全氣,情志專良,通關中人,易以役養乎?然而後世因之,才任稍廣。其能者,則勃貂、管蘇有功於楚、晉,景監、繆賢著庸於秦、趙。及其敝也,則豎刁亂齊,伊戾禍宋。

漢興,仍襲秦制,置中常侍官。然亦引用士人,以參其選,皆銀璫左貂,給事殿省。及高后稱制,乃以張卿為大謁者,出入臥內,受宣詔命。文帝時,有趙談、北宮伯子,頗見親倖。至於孝武,亦愛李延年。帝數宴後庭,或潛游離館,故請奏機事,多以宦人主之。至元帝之世,史游為黃門令,勤心納忠,有所補益。其後弘恭、石顯以佞險自進,卒有蕭、周之禍,損穢帝德焉。

中興之初,宦官悉用閹人,不復雜調它士。至永平中,始置員數,中常侍四人,小黃門十人。和帝即祚幼弱,而竇憲兄弟專總權威,內外臣僚,莫由親接,所與居者,唯閹宦而已。故鄭眾得專謀禁中,終除大憝,遂享分土之封,超登宮卿之位。於是中官始盛焉。

自明帝以後,迄乎延平,委用漸大,而其員稍增,中常侍至有十人,小黃門二十人,改以金璫右貂,兼領卿署之職。鄧后以女主臨政,而萬機殷遠,朝臣國議,無由參斷帷幄,稱制下令,不出房闈之閒,不得不委用刑人,寄之國命。手握王爵,口含天憲,非復掖廷永巷之職,閨牖房闥之任也。其後孫程定立順之功,曹騰參建桓之策,續以五侯合謀,梁冀受鉞,跡因公正,恩固主心,故中外服從,上下屏氣。或稱伊、霍之勳,無謝於往載;或謂良、平之畫,復興於當今。雖時有忠公,而竟見排斥。舉動回山海,呼吸變霜露。阿旨曲求,則光寵三族;直情忤意,則參夷五宗。漢之綱紀大亂矣。

若夫高冠長劍,紆朱懷金者,布滿宮闈;苴茅分虎,南面臣人者,蓋以十數。府署第館,棋列於都鄙;子弟支附,過半於州國。南金、和寶、冰紈、霧縠之積,盈仞珍臧;嬙媛、侍兒、歌童、舞女之玩,充備綺室。狗馬飾雕文,土木被緹繡。皆剝割萌黎,競恣奢欲。搆害明賢,專樹黨類。其有更相援引,希附權彊者,皆腐身熏子,以自衒達。同敝相濟,故其徒有繁,敗國蠹政之事,不可單書。所以海內嗟毒,志士窮棲,寇劇緣閒,搖亂區夏。雖忠良懷憤,時或奮發,而言出禍從,旋見孥戮。因復大考鉤黨,轉相誣染。凡稱善士,莫不離被災毒。竇武、何進,位崇戚近,乘九服之囂怨,協群英之埶力,而以疑留不斷,至於殄敗。斯亦運之極乎!雖袁紹龔行,芟夷無餘,然以暴易亂,亦何云及!自曹騰說梁冀,竟立昏弱。魏武因之,遂遷龜鼎。所謂「君以此始,必以此終」,信乎其然矣!

鄭眾字季產,南陽犨人也。為人謹敏有心幾。永平中,初給事太子家。肅宗即位,拜小黃門,遷中常侍。和帝初,加位鉤盾令。

時竇太后秉政,后兄大將軍憲等並竊威權,朝臣上下莫不附之,而眾獨一心王室,不事豪黨,帝親信焉。及憲兄弟圖作不軌,眾遂首謀誅之,以功遷大長秋。策勳班賞,每辭多受少,由是常與議事。中官用權,自眾始焉。

十四年,帝念眾功美,封為鄛鄉侯,食邑千五百戶。永初元年,和熹皇后益封三百戶。

元初元年卒,養子閎嗣。閎卒,子安嗣。後國絕。桓帝延熹二年,紹封眾曾孫石讎為關內侯。

蔡倫字敬仲,桂陽人也。以永平末始給事宮掖,建初中,為小黃門。及和帝即位,轉中常侍,豫參帷幄。

倫有才學,盡心敦慎,數犯嚴顏,匡弼得失。每至休沐,輒閉門絕賓,暴體田野。後加位尚方令。永元九年,監作祕劍及諸器械,莫不精工堅密,為後世法。

自古書契多編以竹簡,其用縑帛者謂之為紙。縑貴而簡重,並不便於人。倫乃造意,用樹膚、麻頭及敝布、魚網以為紙。元興元年奏上之,帝善其能,自是莫不從用焉,故天下咸稱「蔡侯紙」。

元初元年,鄧太后以倫久宿衛,封為龍亭侯,邑三百戶。後為長樂太僕。四年,帝以經傳之文多不正定,乃選通儒謁者劉珍及博士良史詣東觀,各讎校漢家法,令倫監典其事。

倫初受竇后諷旨,誣陷安帝祖母宋貴人。及太后崩,安帝始親萬機,敕使自致廷尉。倫恥受辱,乃沐浴整衣冠,飲藥而死。國除。

孫程字稚卿,涿郡新城人也。安帝時,為中黃門,給事長樂宮。

時鄧太后臨朝,帝不親政事。小黃門李閏與帝乳母王聖常共譖太后兄執金吾悝等,言欲廢帝,立平原王德,帝每忿懼。及太后崩,遂誅鄧氏而廢平原王,封閏雍鄉侯;又小黃門江京以讒諂進,初迎帝於邸,以功封都鄉侯,食邑各三百戶。閏、京並遷中常侍,江京兼大長秋,與中常侍樊豐、黃門令劉安、鉤盾令陳達及王聖、聖女伯榮扇動內外,競為侈虐。又帝舅大將軍耿寶、皇后兄大鴻臚閻顯更相阿黨,遂枉殺太尉楊震,廢皇太子為濟陰王。

明年帝崩,立北鄉侯為天子。顯等遂專朝爭權,乃諷有司奏誅樊豐,廢耿寶、王聖,及黨與皆見死徙。

十月,北鄉侯病篤。程謂濟陰王謁者長興渠曰:「王以嫡統,本無失德,先帝用讒,遂至廢黜。若北鄉疾不起,共斷江京、閻顯,事乃可成。」渠等然之。又中黃門南陽王康,先為太子府史,自太子之廢,常懷歎憤。又長樂太官丞京兆王國,並附同於程。至二十七日,北鄉侯薨。閰顯白太后,徵諸王子簡為帝嗣。未及至。十一月二日,程遂與王康等十八人聚謀於西鍾下,皆涞單衣為誓。四日夜,程等共會崇德殿上,因入章臺門。時江京、劉安及李閏、陳達等俱坐省門下,程與王康共就斬京、安、達,以李閏權埶積為省內所服,欲引為主,因舉刃脅閏曰:「今當立濟陰王,無得搖動。」閏曰:「諾。」於是扶閏起,俱於西鍾下迎濟陰王立之,是為順帝。召尚書令、僕射以下,從輦幸南宮雲臺,程等留守省門,遮扞內外。

閻顯時在禁中,憂迫不知所為,小黃門樊登勸顯發兵,以太后詔召越騎校尉馮詩、虎賁中郎將閻崇,屯朔平門,以禦程等。誘詩入省,太后使授之印,曰:「能得濟陰王者封萬戶侯,得李閏者五千戶侯。」顯以詩所將眾少,使與登迎吏士于左掖門外。詩因格殺登,歸營屯守。顯弟衛尉景遽從省中還外府,收兵至盛德門。程傳召諸尚書使收景。尚書郭鎮時臥病,聞之,即率直宿羽林出南止車門,逢景從吏士,拔白刃,呼曰:「無干兵。」鎮即下車,時節詔之。景曰:「何等詔?」因斫鎮,不中。鎮引劍擊景墯車,左右以戟叉其匈,遂禽之,送廷尉獄,即夜死。旦日,令侍御史收顯等送獄,於是遂定。下詔曰:「夫表功錄善,古今之通義也。故中常侍長樂太僕江京、黃門令劉安、鉤盾令陳達與故車騎將軍閻顯兄弟謀議惡逆,傾亂天下。中黃門孫程、王康、長樂太官丞王國、中黃門黃龍、彭愷、孟叔、李建、王成、張賢、史汎、馬國、王道、李元、楊佗、陳予、趙封、李剛、魏猛、苗光等,懷忠憤發,戮力協謀,遂埽滅元惡,以定王室。詩不云乎:『無言不讎,無德不報。』程為謀首,康、國協同。其封程為浮陽侯,食邑萬戶;康為華容侯,國為酈侯,各九千戶;黃龍為湘南侯,五千戶;彭愷為西平昌侯,孟叔為中廬侯,李建為復陽侯,各四千二百戶;王成為廣宗侯,張賢為祝阿侯,史汎為臨沮侯,馬國為廣平侯,王道為范縣侯,李元為褒信侯,楊佗為山都侯,陳予為下雋侯,趙封為析縣侯,李剛為枝江侯,各四千戶;魏猛為夷陵侯,二千戶;苗光為東阿侯,千戶。」是為十九侯。加賜車馬金銀錢帛各有差。李閏以先不豫謀,故不封。遂擢拜程騎都尉。

永建元年,程與張賢、孟叔、馬國等為司隸校尉虞詡訟罪,懷表上殿,呵叱左右。帝怒,遂免程官,因悉遣十九侯就國,後徙封程為宜城侯。程既到國,怨恨恚懟,封還印綬、符策,亡歸京師,往來山中。詔書追求,復故爵土,賜車馬衣物,遣還國。

三年,帝念程等功勳,悉徵還京師。程與王道、李元皆拜騎都尉,餘悉奉朝請。陽嘉元年,程病甚,即拜奉車都尉,位特進。及卒,使五官郎將追贈車騎將軍印綬,賜謚剛侯。侍御史持節監護喪事,乘輿幸北部尉傳,瞻望車騎。

程臨終,遺言上書,以國傳弟美。帝許之,而分程半,封程養子壽為浮陽侯。後詔書錄微功,封興渠為高望亭侯。四年,詔宦官養子悉聽得為後,襲封爵,定著乎令。

王康、王國、彭愷、王成、趙封、魏猛六人皆早

卒。黃龍、楊佗、孟叔、李建、張賢、史汎、王道、李元、李剛九人與阿母山陽君宋娥更相貨賂,求高官增邑,又誣罔中常侍曹騰、孟賁等。永和二年,發覺,並遣就國,減租四分之一。宋娥奪爵歸田舍。唯馬國、陳予、苗光保全封邑。

初,帝見廢,監太子家小黃門籍建、傅高梵、長秋長趙熹、丞良賀、藥長夏珍皆以無過獲罪,建等坐徙朔方。及帝即位,並擢為中常侍。梵坐臧罪,減死一等。建後封東鄉侯,三百戶。

賀清儉退厚,位至大長秋。陽嘉中,詔九卿舉武猛,賀獨無所薦。帝引問其故,對曰:「臣生自草茅,長於宮掖,既無知人之明,又未嘗交知士類。昔衛鞅因景監以見,有識知其不終。今得臣舉者,匪榮伊辱。」固辭之。及卒,帝思賀忠,封其養子為都鄉侯,三百戶。

曹騰字季興,沛國譙人也。安帝時,除黃門從官。順帝在東宮,鄧太后以騰年少謹厚,使侍皇太子書,特見親愛。及帝即位,騰為小黃門,遷中常侍。桓帝得立,騰與長樂太僕州輔等七人,以定策功,皆封亭侯,騰為費亭侯,遷大長秋,加位特進。

騰用事省闥三十餘年,奉事四帝,未嘗有過。其所進達,皆海內名人,陳留虞放、邊韶、南陽延固、張溫、弘農張奐、潁川堂谿典等。時蜀郡太守因計吏賂遺於騰,益州刺史种暠於斜谷關搜得其書,上奏太守,并以劾騰,請下廷尉案罪。帝曰:「書自外來,非騰之過。」遂寢暠奏。騰不為纖介,常稱暠為能吏,時人嗟美之。

騰卒,養子嵩嗣。种暠後為司徒,告賓客曰:「今身為公,乃曹常侍力焉。」

嵩靈帝時貨賂中官及輸西園錢一億萬,故位至太尉。及子操起兵,不肯相隨,乃與少子疾避亂琅邪,為徐州刺史陶謙所殺。

單超,河南人;徐璜,下邳良城人;具瑗,魏郡元城人;左悺,河南平陰人;唐衡,潁川郾人也。桓帝初,超、璜、瑗為中常侍,悺、衡為小黃門史。

初,梁冀兩妹為順桓二帝皇后,冀代父商為大將軍,再世權戚,威振天下。冀自誅太尉李固、杜喬等,驕橫益甚,皇后乘埶忌恣,多所鴆毒,上下鉗口,莫有言者。帝逼畏久,恆懷不平,恐言泄,不敢謀之。延熹二年,皇后崩,帝因如廁,獨呼衡問:「左右與外舍不相得者皆誰乎?」衡對曰:「單超、左悺前詣河南尹不疑,禮敬小簡,不疑收其兄弟送洛陽獄,二人詣門謝,乃得解。徐璜、具瑗常私忿疾外舍放橫,口不敢道。」於是帝呼超、悺入室,謂曰:「梁將軍兄弟專固國朝,迫脅外內,公卿以下從其風旨。今欲誅之,於常侍意何如?」超等對曰:「誠國姦賊,當誅日久。臣等弱劣,未知聖意何如耳。」帝曰:「審然者,常侍密圖之。」對曰:「圖之不難,但恐陛下復中狐疑。」帝曰:「姦臣脅國,當伏其罪,何疑乎!」於是更召璜、瑗等五人,遂定其議,帝齧超臂出血為盟。於是詔收冀及宗親黨與悉誅之。悺、衡遷中常侍,封超新豐侯,二萬戶,璜武原侯,瑗東武陽侯,各萬五千戶,賜錢各千五百萬;悺上蔡侯,衡汝陽侯,各萬三千戶,賜錢各千三百萬。五人同日封,故世謂之「五侯」。又封小黃門劉普、趙忠等八人為鄉侯。自是權歸宦官,朝廷日亂矣。

超病,帝遣使者就拜車騎將軍。明年薨,賜東園祕器,棺中玉具,贈侯將軍印綬,使者理喪。及葬,發五營騎士,將軍侍御史護喪,將作大匠起冢塋。

其後四侯轉橫,天下為之語曰:「左回天,具獨坐,徐臥虎,唐兩墯。」皆競起第宅,樓觀壯麗,窮極伎巧。金銀罽毦,施於犬馬。多取良人美女以為姬妾,皆珍飾華侈,擬則宮人。其僕從皆乘牛車而從列騎。又養其疏屬,或乞嗣異姓,或買蒼頭為子,並以傳國襲封。兄弟姻戚皆宰州臨郡,辜較百姓,與盜賊無異。

超弟安為河東太守,弟子匡為濟陰太守,璜弟盛為河內太守,悺弟敏為陳留太守,瑗兄恭為沛相,皆為所在蠹害。

璜兄子宣為下邳令,暴虐尤甚。先是求故汝南太守下邳李暠女不能得,及到縣,遂將吏卒至暠家,載其女歸,戲射殺之,埋著寺內。時下邳縣屬東海,汝南黃浮為東海相,有告言宣者,浮乃收宣家屬,無少長悉考之。掾史以下固諫爭。浮曰:「徐宣國賊,今日殺之,明日坐死,足以瞑目矣。」即案宣罪棄市,暴其尸以示百姓,郡中震慄。璜於是訴怨於帝,帝大怒,浮坐髡鉗,輸作右校。五侯宗族賓客虐遍天下,民不堪命,起為寇賊。七年,衡卒,亦贈車騎將軍,如超故事。璜卒,賻贈錢布,賜冢塋地。

明年,司隸校尉韓演因奏悺罪惡,及其兄太僕南鄉侯稱請託州郡,聚斂為姦,賓客放縱,侵犯吏民。悺、稱皆自殺。演又奏瑗兄沛相恭臧罪,徵詣廷尉。瑗詣獄謝,上還東武侯印綬,詔貶為都鄉侯,卒於家。超及璜、衡襲封者,並降為鄉侯,租入歲皆三百萬,子弟分封者,悉奪爵土。劉普等貶為關內侯。

侯覽者,山陽防東人。桓帝初為中常侍,以佞猾進,倚埶貪放,受納貨遺以巨萬計。延熹中,連歲征伐,府帑空虛,乃假百官奉祿,王侯租稅。覽亦上縑五千匹,賜爵關內侯。又託以與議誅梁冀功,進封高鄉侯。

小黃門段珪家在濟陰,與覽並立田業,近濟北界,僕從賓客侵犯百姓,劫掠行旅。濟北相滕延一切收捕,殺數十人,陳尸路衢。覽、珪大怨,以事訴帝,延坐多殺無辜,徵詣廷尉,免。延字伯行,北海人,後為京兆尹,有理名,世稱為長者。

覽等得此愈放縱。覽兄參為益州刺史,民有豐富者,輒誣以大逆,皆誅滅之,沒入財物,前後累億計。太尉楊秉奏參,檻車徵,於道自殺。京兆尹袁逢於旅舍閱參車三百餘兩,皆金銀錦帛珍玩,不可勝數。覽坐免,旋復復官。

建寧二年,喪母還家,大起塋冢。督郵張儉因舉奏覽貪侈奢縱,前後請奪人宅三百八十一所,田百一十八頃。起立第宅十有六區,皆有高樓池苑,堂閣相望,飾以綺畫丹漆之屬,制度重深,僭類宮省。又豫作壽冢,石槨雙闕,高廡百尺,破人居室,發掘墳墓。虜奪良人,妻略婦子,及諸罪釁,請誅之。而覽伺候遮涞,章竟不上。儉遂破覽冢宅,藉沒資財,具言罪狀。又奏覽母生時交通賓客,干亂郡國。復不得御。覽遂誣儉為鉤黨,及故長樂少府李膺、太僕杜密等,皆夷滅之。遂代曹節領長樂太僕。

熹平元年,有司舉奏覽專權驕奢,策收印綬,自殺。阿黨者皆免。

曹節字漢豐,南陽新野人也。其本魏郡人,世吏二千石。順帝初,以西園騎遷小黃門。桓帝時,遷中常侍,奉車都尉。建寧元年,持節將中黃門虎賁羽林千人,北迎靈帝,陪乘入宮。及即位,以定策封長安鄉侯,六百戶。

時竇太后臨朝,后父大將軍武與太傅陳蕃謀誅中官,節與長樂五官史朱瑀、從官史共普、張亮、中黃門王尊、長樂謁者騰是等十七人,共矯詔以長樂食監王甫為黃門令,將兵誅武、蕃等,事已具蕃、武傳。節遷長樂衛尉,封育陽侯,增邑三千戶;甫遷中常侍,黃門令如故;瑀封都鄉侯,千五百戶;普、亮等五人各三百戶;餘十一人皆為關內侯,歲食租二千斛。

先是瑀等陰於明堂中禱皇天曰:「竇氏無道,請皇天輔皇帝誅之,令事必成,天下得寧。」既誅武等,詔令太官給塞具,賜瑀錢五千萬,餘各有差,後更封華容侯。二人,節病困,詔拜為車騎將軍。有頃疾瘳,上印綬,罷,復為中常侍,位特進,秩中二千石,尋轉大長秋。

熹平元年,竇太后崩,有何人書朱雀闕,言「天下大亂,曹節、王甫幽殺太后,常侍侯覽多殺黨人,公卿皆尸祿,無有忠言者。」於是詔司隸校尉劉猛逐捕,十日一會。猛以誹書言直,不肯急捕,月餘,主名不立。猛坐左轉諫議大夫,以御史中丞段熲代猛,乃四出逐捕,及太學游生,繫者千餘人。節等怨猛不已,使熲以它事奏猛,抵罪輸左校。朝臣多以為言,乃免刑,復公車徵之。

節遂與王甫等誣奏桓帝弟勃海王悝謀反,誅之。以功封者十二人。甫封冠軍侯。節亦增邑四千六百戶,并前七千六百戶。父兄子弟皆為公卿列校、牧守令長,布滿天下。

節弟破石為越騎校尉,越騎營五百妻有美色,破石從求之,五百不敢違,妻執意不肯行,遂自殺。其淫暴無道,多此類也。

光和二年,司隸校尉陽球奏誅王甫及子長樂少府萌、沛相吉,皆死獄中。時連有災異,郎中梁人審忠以為朱瑀等罪惡所感,乃上書曰:「臣聞理國得賢則安,失賢則危,故舜有臣五人而天下理,湯舉伊尹不仁者遠。陛下即位之初,未能萬機,皇太后念在撫育,權時攝政,故中常侍蘇康、管霸應時誅殄。太傅陳蕃、大將軍竇武考其黨與,志清朝政。華容侯朱瑀知事覺露,禍及其身,遂興造逆謀,作亂王室,撞蹋省闥,執奪璽綬,迫脅陛下,聚會群臣,離閒骨肉母子之恩,遂誅蕃、武及尹勳等。因共割裂城社,自相封賞。父子兄弟被蒙尊榮,素所親厚布在州郡,或登九列,或據三司。不惟祿重位尊之責,而苟營私門,多蓄財貨,繕修第舍,連里竟巷。盜取御水以作魚釣,車馬服玩擬於天家。群公卿士杜口吞聲,莫敢有言。州牧郡守承順風旨,辟召選舉,釋賢取愚。故蟲蝗為之生,夷寇為之起。天意憤盈,積十餘年。故頻歲日食於上,地震於下,所以譴戒人主,欲令覺悟,誅鉏無狀。昔高宗以雉雊之變,故獲中興之功。近者神祇啟悟陛下,發赫斯之怒,故王甫父子應時馘涞,路人士女莫不稱善,若除父母之讎。誠怪陛下復忍孽臣之類,不悉殄滅。昔秦信趙高,以危其國;吳使刑人,身遘其禍。虞公抱寶牽馬,魯昭見逐乾侯,以不用宮之奇、子家駒以至滅辱。今以不忍之恩,赦夷族之罪,姦謀一成,悔亦何及!臣為郎十五年,皆耳目聞見,瑀之所為,誠皇天所不復赦。願陛下留漏刻之聽,裁省臣表,埽滅醜類,以荅天怒。與瑀考驗,有不如言,願受湯鑊之誅,妻子并徙,以絕妄言之路。」章寢不報。節遂領尚書令。四年,卒,贈車騎將軍。後瑀亦病卒,皆養子傳國。

審忠字公誠,宦官誅後,辟公府。

呂強字漢盛,河南成皋人也。少以宦者為小黃門,再遷中常侍。為人清忠奉公。靈帝時,例封宦者,以強為都鄉侯。強辭讓懇惻,固不敢當,帝乃聽之。因上疏陳事曰:

臣聞諸侯上象四七,下裂王土,高祖重約非功臣不侯,所以重天爵明勸戒也。伏聞中常侍曹節、王甫、張讓等,及侍中許相,並為列侯。節等宦官祐薄,品卑人賤,讒諂媚主,佞邪徼寵,放毒人物,疾妒忠良,有趙高之禍,未被轘裂之誅,掩朝廷之明,成私樹之黨。而陛下不悟,妄授茅土,開國承家,小人是用。又并及家人,重金兼紫,相繼為蕃輔。受國重恩,不念爾祖,述修厥德,而交結邪黨,下比群佞。陛下或其瑣才,特蒙恩澤。又授位乖越,賢才不升,素餐私倖,必加榮擢。陰陽乖剌,稼穡荒蔬,人用不康,罔不由茲。臣誠知封事已行,言之無逮,所以冒死干觸陳愚忠者,實願陛下損改既謬,從此一止。

臣又聞後宮綵女數千餘人,衣食之費,日數百金。比穀雖賤,而戶有飢色。案法當貴而今更賤者,由賦發繁數,以解縣官,寒不敢衣,飢不敢食。民有斯厄,而莫之卹。宮女無用,填積後庭,天下雖復盡力耕桑,猶不能供。昔楚女悲愁,則西宮致災,況終年積聚,豈無憂怨乎!夫天生蒸民,立君以牧之。君道得,則民戴之如父母,仰之猶日月,雖時有征稅,猶望其仁恩之惠。《易》曰:「悅以使民,民忘其勞;悅以犯難,民忘其死。」儲君副主,宜諷誦斯言;南面當國,宜履行其事。

又承詔書,當於河閒故國起解瀆之館。陛下龍飛即位,雖從藩國,然處九天之高,豈宜有顧戀之意。且河閒疏遠,解瀆毙絕,而當勞民單力,未見其便。又今外戚四姓貴倖之家,及中官公族無功德者,造起館舍,凡有萬數,樓閣連接,丹青素堊,雕刻之飾,不可單言。喪葬踰制,奢麗過禮,競相放效,莫肯矯拂。穀梁傳曰:「財盡則怨,力盡則懟。」尸子曰:「君如杅,民如水,杅方則水方,杅圓則水圓。」上之化下,猶風之靡草。今上無去奢之儉,下有縱欲之敝,至使禽獸食民之甘,木土衣民之帛。昔師曠諫晉平公曰:「梁柱衣繡,民無褐衣;池有棄酒,士有渴死;廄馬秣粟,民有飢色。近臣不敢諫,遠臣不得暢。」此之謂也。

又聞前召議郎蔡邕對問於金商門,而令中常侍曹節、王甫等以詔書喻旨。邕不敢懷道迷國,而切言極對,毀刺貴臣,譏呵豎宦。陛下不密其言,至令宣露,群邪項領,膏脣拭舌,競欲咀嚼。造作飛條。陛下回受誹謗,致邕刑罪,室家徙放,老幼流離,豈不負忠臣哉!今群臣皆以邕為戒,上畏不測之難,下懼劍客之害,臣知朝廷不復得聞忠言矣。故太尉段熲,武勇冠世,習於邊事,垂髮服戎,功成皓首,歷事二主,勳烈獨昭。陛下既已式序,位登台司,而為司隸校尉陽球所見誣脅,一身既斃,而妻子遠播。天下惆悵,功臣失望。宜徵邕更授任,反熲家屬,則忠貞路開,眾怨以弭矣。

帝知其忠而不能用。

時帝多蓄私臧,收天下之珍,每郡國貢獻,先輸中署,名為「導行費」。強上疏諫曰:

天下之財,莫不生之陰陽,歸之陛下。歸之陛下,豈有公私?而今中尚方斂諸郡之寶,中御府積天下之繒,西園引司農之臧,中廄聚太僕之馬,而所輸之府,輒有導行之財。調廣民困,費多獻少,姦吏因其利,百姓受其敝。又阿媚之臣,好獻其私,容諂姑息,自此而進。

舊典選舉委任三府,三府有選,參議掾屬,咨其行狀,度其器能,受試任用,責以成功。若無可察,然後付之尚書。尚書舉劾,請下廷尉,覆案虛實,行其誅罰。今但任尚書,或復敕用。如是,三公得免選舉之負,尚書亦復不坐,責賞無歸,豈肯空自苦勞乎!

夫立言無顯過之咎,明鏡無見玼之尤。如惡立言以記過,則不當學也;不欲明鏡之見玼,則不當照也。願陛下詳思臣言,不以記過見玼為責。

書奏不省。

中平元年,黃巾賊起,帝問強所宜施行。強欲先誅左右貪濁者,大赦黨人,料簡刺史、二千石能否。帝納之,乃先赦黨人。於是諸常侍人人求退,又各自徵還宗親子弟在州郡者。中常侍趙忠、夏惲等遂共搆強,云「與黨人共議朝廷,數讀霍光傳。強兄弟所在並皆貪穢」。帝不悅,使中黃門持兵召強。強聞帝召,怒曰:「吾死,亂起矣。丈夫欲盡忠國家,豈能對獄吏乎!」遂自殺。忠、惲復譖曰:「強見召未知所問,而就外草自屏,有姦明審。」遂收捕宗親,沒入財產焉。

時宦者濟陰丁肅、下邳徐衍、南陽郭耽、汝陽李

巡、北海趙祐等五人稱為清忠,皆在里巷,不爭威權。巡以為諸博士試甲乙科,爭弟高下,更相告言,至有行賂定蘭臺漆書經字,以合其私文者,乃白帝,與諸儒共刻五經文於石,於是詔蔡邕等正其文字。自後五經一定,爭者用息。趙祐博學多覽,著作校書,諸儒稱之。

又小黃門甘陵吳伉,善為風角,博達有奉公稱。知不得用,常託病還寺舍,從容養志云。

張讓者,潁川人;趙忠者,安平人也。少皆給事省中,桓帝時為小黃門。忠以與誅梁冀功封都鄉侯。延熹八年,黜為關中侯,食本縣租千斛。

靈帝時,讓、忠並遷中常侍,封列侯,與曹節、王甫等相為表裏。節死後,忠領大長秋。讓有監奴典任家事,交通貨賂,威形諠赫。扶風人孟佗,資產饒贍,與奴朋結,傾竭饋問,無所遺愛。奴咸德之,問佗曰:「君何所欲?力能辦也。」曰:「吾望汝曹為我一拜耳。」時賓客求謁讓者,車恆數百千兩,佗時詣讓,後至,不得進,監奴乃率諸倉頭迎拜於路,遂共轝車入門。賓客咸驚,謂佗善於讓,皆爭以珍玩賂之。佗分以遺讓,讓大喜,遂以佗為涼州刺史。

是時讓、忠及夏惲、郭勝、孫璋、畢嵐、栗嵩、

段珪、高望、張恭、韓悝、宋典十二人,皆為中常侍,封侯貴寵,父兄子弟布列州郡,所在貪殘,為人蠹害。黃巾既作,盜賊糜沸,郎中中山張鈞上書曰:「竊惟張角所以能興兵作亂,萬人所以樂附之者,其源皆由十常侍多放父兄、子弟、婚親、賓客典據州郡,辜榷財利,侵掠百姓,百姓之冤無所告訴,故謀議不軌,聚為盜賊。宜斬十常侍,縣頭南郊,以謝百姓,又遣使者布告天下,可不須師旅,而大寇自消。」天子以鈞章示讓等,皆免冠徒跣頓首,乞自致洛陽詔獄,並出家財以助軍費。有詔皆冠履視事如故。帝怒鈞曰:「此真狂子也。十常侍固當有一人善者不?」鈞復重上,猶如前章,輒寢不報。詔使廷尉、侍御史考為張角道者,御史承讓等旨,遂誣奏鈞學黃巾道,收掠死獄中。而讓等實多與張角交通。後中常侍封諝、徐奏事獨發覺坐誅,帝因怒詰讓等曰:「汝曹常言黨人欲為不軌,皆令禁錮,或有伏誅。今黨人更為國用,汝曹反與張角通,為可斬未?」皆叩頭云:「故中常侍王甫、侯覽所為。」帝乃止。

明年,南宮災。讓、忠等說帝令斂天下田畝稅十錢,以修宮室。發太原、河東、狄道諸郡材木及文石,每州郡部送至京師,黃門常侍輒令譴呵不中者,因強折賤買,十分雇一,因復貨之於宦官,復不為即受,材木遂至腐積,宮室連年不成。刺史、太守復增私調,百姓呼嗟。凡詔所徵求,皆令西園騶密約敕,號曰「中使」,恐動州郡,多受賕賂。刺史、二千石及茂才孝廉遷除,皆責助軍修宮錢,大郡至二三千萬,餘各有差。當之官者,皆先至西園諧價,然後得去。有錢不畢者,或至自殺。其守清者,乞不之官,皆迫遣之。

時鉅鹿太守河內司馬直新除,以有清名,減責三百萬。直被詔,悵然曰:「為民父母,而反割剝百姓,以稱時求,吾不忍也。」辭疾,不聽。行至孟津,上書極陳當世之失,古今禍敗之戒,即吞藥自殺。書奏,帝為暫絕修宮錢。

又造萬金堂於西園,引司農金錢繒帛,仞積其中。又還河閒買田宅,起第觀。帝本侯家,宿貧,每歎桓帝不能作家居,故聚為私臧,復臧寄小黃門常侍錢各數千萬。常云:「張常侍是我公,趙常侍是我母。」宦官得志,無所憚畏,並起第宅,擬則宮室。帝常登永安侯臺,宦官恐其望見居處,乃使中大人尚但諫曰:「天子不當登高,登高則百姓虛散。」自是不敢復升臺榭。

明年,遂使鉤盾令宋典繕修南宮玉堂。又使掖庭令畢嵐鑄銅人四列於倉龍、玄武闕。又鑄四鐘,皆受二千斛,縣於玉堂及雲臺殿前。又鑄天祿蝦蟆,吐水於平門外橋東,轉水入宮。又作翻車渴烏,施於橋西,用灑南北郊路,以省百姓灑道之費。又鑄四出文錢,錢皆四道。識者竊言侈虐已甚,形象兆見,此錢成,必四道而去。及京師大亂,錢果流布四海。復以忠為車騎將軍,百餘日罷。

六年,帝崩。中軍校尉袁紹說大將軍何進,令誅中官以悅天下。謀泄,讓、忠等因進入省,遂共殺進。而紹勒兵斬忠,捕宦官無少長悉斬之。讓等數十人劫質天子走河上。追急,讓等悲哭辭曰:「臣等殄滅,天下亂矣。惟陛下自愛!」皆投河而死。

論曰:自古喪大業絕宗禋者,其所漸有由矣。三世以嬖色取禍,嬴氏以奢虐致災,西京自外戚失祚,東都緣閹尹傾國。成敗之來,先史商之久矣。至於釁起宦夫,其略猶或可言。何者?刑餘之醜,理謝全生,聲榮無暉於門閥,肌膚莫傳於來體,推情未鑒其敝,即事易以取信,加漸染朝事,頗識典物,故少主憑謹舊之庸,女君資出內之命,顧訪無猜憚之心,恩狎有可悅之色。亦有忠厚平端,懷術糾邪;或敏才給對,飾巧亂實;或借譽貞良,先時薦譽。非直苟恣凶德,止於暴橫而已。然真邪並行,情貌相越,故能回惑昏幼,迷瞀視聽,蓋亦有其理焉。詐利既滋,朋徒日廣,直臣抗議,必漏先言之閒,至戚發憤,方啟專奪之隙,斯忠賢所以智屈,社稷故其為墟。《易》曰:「履霜堅冰至。」云所從來久矣。今跡其所以,亦豈一朝一夕哉!

贊曰:任失無小,過用則違。況乃巷職,遠參天機。舞文巧態,作惠作威。凶家害國,夫豈異歸!


\end{pinyinscope}