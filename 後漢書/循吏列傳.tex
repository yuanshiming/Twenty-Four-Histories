\article{循吏列傳}

\begin{pinyinscope}
初,光武長於民閒,頗達情偽,見稼穡艱難,百姓病害,至天下已定,務用安靜,解王莽之繁密,還漢世之輕法。身衣大練,色無重綵,耳不聽鄭衛之音,手不持珠玉之玩,宮房無私愛,左右無偏恩。建武十三年,異國有獻名馬者,日行千里,又進寶劍,賈兼百金,詔以馬駕鼓車,劍賜騎士。損上林池烃之官,廢騁望弋獵之事。其以手跡賜方國者,皆一札十行,細書成文。勤約之風,行于上下。數引公卿郎將,列于禁坐。廣求民瘼,觀納風謠。故能內外匪懈,百姓寬息。自臨宰邦邑者,競能其官。若杜詩守南陽,號為「杜母」,任延、錫光移變邊俗,斯其績用之最章章者也。又第五倫、宋均之徒,亦足有可稱談。然建武、永平之閒,吏事刻深,亟以謠言單辭,轉易守長。故朱浮數上諫書,箴切峻政,鍾離意等亦規諷殷勤,以長者為言,而不能得也。所以中興之美,蓋未盡焉。自章、和以後,其有善績者,往往不絕。如魯恭、吳祐、劉寬及潁川四長,並以仁信篤誠,使人不欺;王堂、陳寵委任賢良,而職事自理:斯皆可以感物而行化也。邊鳳、延篤先後為京兆尹,時人以輩前世趙、張。又王渙、任峻之為洛陽令,明發姦伏,吏端禁止,然導德齊禮,有所未充,亦一時之良能也。今綴集殊聞顯跡,以為循吏篇云。

衛颯字子產,河內脩武人也。家貧好學問,隨師無糧,常傭以自給。王莽時,仕郡歷州宰。

建武二年,辟大司徒鄧禹府。舉能案劇,除侍御史,襄城令。政有名跡,遷桂陽太守。郡與交州接境,頗染其俗,不知禮則。颯下車,修庠序之教,設婚姻之禮。期年閒,邦俗從化。

先是含洭、湞陽、曲江三縣,越之故地,武帝平之,內屬桂陽。民居深山,濱溪谷,習其風土,不出田租。去郡遠者,或且千里。吏事往來,輒發民乘船,名曰「傳役」。每一吏出,傜及數家,百姓苦之。颯乃鑿山通道五百餘里,列亭傳,置郵驛。於是役省勞息,姦吏杜絕。流民稍還,漸成聚邑,使輸租賦,同之平民。又耒陽縣山鐵石,佗郡民庶常依因聚會,私為冶鑄,遂招來亡命,多致姦盜。颯乃上起鐵官,罷斥私鑄,歲所增入五百餘萬。諷理卹民事,居官如家,其所施政,莫不合於物宜。視事十年,郡內清理。

二十五年,徵還。光武欲以為少府,會颯被疾,不能拜起,敕以桂陽太守歸家,須後詔書。居二歲,載病詣闕,自陳困篤,乃收印綬,賜錢十萬,後卒于家。

南陽茨充代颯為桂陽。亦善其政,教民種殖桑柘麻紵之屬,勸令養蠶織屨,民得利益焉。

任延字長孫,南陽宛人也。年十二,為諸生,學於長安,明詩、易、春秋,顯名太學,學中號為「任聖童」。值倉卒,避兵之隴西。時隗囂已據四郡,遣使請延,延不應。

更始元年,以延為大司馬屬,拜會稽都尉,時年十九,迎官驚其壯。及到,靜泊無為,唯先遣饋禮祠延陵季子。時天下新定,道路未通,避亂江南者皆未還中土,會稽頗稱多士。延到,皆聘請高行如董子儀、嚴子陵等,敬待以師友之禮。掾吏貧者,輒分奉祿以賑給之。省諸卒,令耕公田,以周窮急。每時行縣,輒使慰勉孝子,就餐飯之。

吳有龍丘萇者,隱居太末,志不降辱。王莽時,四輔三公連辟,不到。掾史白請召之。延曰:「龍丘先生躬德履義,有原憲、伯夷之節。都尉埽洒其門,猶懼辱焉,召之不可。」遣功曹奉謁,修書記,致醫藥,吏使相望於道。積一歲,萇乃乘輦詣府門,願得先死備錄。延辭讓再三,遂署議曹祭酒。萇尋病卒,延自臨殯,不朝三日。是以郡中賢士大夫爭往宦焉。

建武初,延上書願乞骸骨,歸拜王庭。詔徵為九真太守。光武引見,賜馬雜繒,令妻子留洛陽。九真俗以射獵為業,不知牛耕,民常告糴交阯,每致困乏。延乃令鑄作田器,教之墾闢。田疇歲歲開廣,百姓充給。又駱越之民無嫁娶禮法,各因淫好,無適對匹,不識父子之性,夫婦之道。延乃移書屬縣,各使男年二十至五十,女年十五至四十,皆以年齒相配。其貧無禮娉,令長吏以下各省奉祿以賑助之。同時相娶者二千餘人。是歲風雨順節,穀稼豐衍。其產子者,始知種姓。咸曰:「使我有是子者,任君也。」多名子為「任」。於是徼外蠻夷夜郎等慕義保塞,延遂止罷偵候戍卒。

初,平帝時,漢中錫光為交阯太守,教導民夷,漸以禮義,化聲侔於延。王莽末,閉境拒守。建武初,遣使貢獻,封鹽水侯。領南華風,始於二守焉。

延視事四年,徵詣洛陽,以病稽留,左轉睢陽令,九真吏人生為立祠。拜武威太守,帝親見,戒之曰:「善事上官,無失名譽。」延對曰:「臣聞忠臣不私,私臣不忠。履正奉公,臣子之節。上下雷同,非陛下之福。善事上官,臣不敢奉詔。」帝歎息曰:「卿言是也。」

既之武威,時將兵長史田紺,郡之大姓,其子弟賓客為人暴害。延收紺繫之,父子賓客伏法者五六人。紺少子尚乃聚會輕薄數百人,自號將軍,夜來攻郡。延即發兵破之。自是威行境內,吏民累息。

郡北當匈奴,南接種羌,民畏寇抄,多廢田業。延到,選集武略之士千人,明其賞罰,令將雜種胡騎休屠黃石屯據要害,其有警急,逆擊追討。虜恒多殘傷,遂絕不敢出。

河西舊少雨澤,乃為置水官吏,修理溝渠,皆蒙其利。又造立校官,自掾吏子孫,皆令詣學受業,復其傜役。章句既通,悉顯拔榮進之。郡遂有儒雅之士。

後坐擅誅羌不先上,左轉召陵令。顯宗即位,拜潁川大守。永平二年,徵會辟雍,因以為河內太守。視事九年,病卒。

少子愷,官至太常。

王景字仲通,樂浪俨邯人也。八世祖仲,本琅邪不其人。好道術,明天文。諸呂作亂,齊哀王襄謀發兵,而數問於仲。及濟北王興居反,欲委兵師仲,仲懼禍及,乃浮海東奔樂浪山中,因而家焉。父閎,為郡三老。更始敗,土人王調殺郡守劉憲,自稱大將軍、樂浪太守。建武六年,光武遣太守王遵將兵擊之。至遼東,閎與郡決曹史楊邑等共殺調迎遵,皆封為列侯,閎獨讓爵。帝奇而徵之,道病卒。

景少學易,遂廣闚眾書,又好天文術數之事,沈深多伎蓺。辟司空伏恭府。時有薦景能理水者,顯宗詔與將作謁者王吳共修作浚儀渠。吳用景静流法,水乃不復為害。

初,平帝時,河、汴決壞,未及得修。建武十年,陽武令張汜上言:「河決積久,日月侵毀,濟渠所漂數十許縣。脩理之費,其功不難。宜改脩堤防,以安百姓。」書奏,光武即為發卒。方營河功,而浚儀令樂俊復上言:「昔元光之閒,人庶熾盛,緣隄墾殖,而瓠子河決,尚二十餘年,不即擁塞。今居家稀少,田地饒廣,雖未脩理,其患猶可。且新被兵革,方興役力,勞怨既多,民不堪命。宜須平靜,更議其事。」光武得此遂止。後汴渠東侵,日月彌廣,而水門故處,皆在河中,兗、豫百姓怨歎,以為縣官恒興佗役,不先民急。永平十二年,議修汴渠,乃引見景,問以理水形便。景陳其利害,應對敏給,帝善之。又以嘗修浚儀,功業有成,乃賜景山海經、河渠書、禹貢圖,及錢帛衣物。夏,遂發卒數十萬,遣景與王吳脩渠築隄,自滎陽東至千乘海口千餘里。景乃商度地埶,鑿山阜,破砥績,直涞溝澗,防遏衝要,疏決壅積,十里立一水門,令更相洄注,無復潰漏之患。景雖簡省役費,然猶以百億計。明年夏,渠成。帝親自巡行,詔濱河郡國置河堤員吏,如西京舊制。景由是知名。王吳及諸從事掾史皆增秩一等。景三遷為侍御史。十五年,從駕東巡狩,至無鹽,帝美其功績,拜河堤謁者,賜車馬縑錢。

建初七年,遷徐州刺史。先是杜陵杜篤奏上論都,欲令車駕遷還長安。耆老聞者,皆動懷土之心,莫不眷然佇立西望。景以宮廟已立,恐人情疑惑,會時有神雀諸瑞,乃作金人論,頌洛邑之美,天人之符,文有可採。

明年,遷廬江太守。先是百姓不知牛耕,致地力有餘而食常不足。郡界有楚相孫叔敖所起芍陂稻田。景乃驅率吏民,修起蕪廢,教用犁耕,由是墾闢倍多,境內豐給。遂銘石刻誓,令民知常禁。又訓令蠶織,為作法制,皆著于鄉亭,廬江傳其文辭。卒於官。

初,景以為六經所載,皆有卜筮,作事舉止,質於蓍龜,而眾書錯糅,吉凶相反,乃參紀眾家數術文書,冢宅禁忌,堪輿日相之屬,適於事用者,集為大衍玄基云。

秦彭字伯平,扶風茂陵人也。自漢興之後,世位相承。六世祖襲,為潁川太守,與群從同時為二千石者五人,故三輔號曰「萬石秦氏」。彭同產女弟,顯宗時入掖庭為貴人,有寵。永平七年,以彭貴人兄,隨四姓小侯擢為開陽城門候。十五年,拜騎都尉,副駙馬都尉耿秉北征匈奴。

建初元年,遷山陽太守。以禮訓人,不任刑罰。崇好儒雅,敦明庠序。每春秋饗射,輒修升降揖讓之儀。乃為人設四誡,以定六親長幼之禮。有遵奉教化者,擢為鄉三老,常以八月致酒肉以勸勉之。吏有過咎,罷遣而已,不加恥辱。百姓懷愛,莫有欺犯。興起稻田數千頃,每於農月,親度頃畝,分別肥塉,差為三品,各立文簿,藏之鄉縣。於是姦吏跼蹐,無所容詐。彭乃上言,宜令天下齊同其制。詔書以其所立條式,班令三府,並下州郡。

在職六年,轉潁川太守,仍有鳳皇、麒麟、嘉禾、甘露之瑞,集其郡境。肅宗巡行,再幸潁川,輒賞賜錢穀,恩寵甚異。章和二年卒。

彭弟惇、褒,並為射聲校尉。

王渙字稚子,廣漢郪人也。父順,安定太守。渙少好俠,尚氣力,數通剽輕少年。晚而改節,敦儒學,習尚書,讀律令,略舉大義。為太守陳寵功曹,當職割斷,不避豪右。寵風聲大行,入為大司農。和帝問曰:「在郡何以為理?」寵頓首謝曰:「臣任功曹王渙以簡賢選能,主簿鐔顯拾遺補闕,臣奉宣詔書而已。」帝大悅。渙由此顯名。

州舉茂才,除溫令。縣多姦猾,積為人患。渙以方略討擊,悉誅之。境內清夷,商人露宿於道。其有放牛者,輒云以屬稚子,終無侵犯。在溫三年,遷兗州刺史,繩正部郡,風威大行。後坐考妖言不實論。歲餘,徵拜侍御史。

永元十五年,從駕南巡,還為洛陽令。以平正居身,得寬猛之宜。其冤嫌久訟,歷政所不斷,法理所難平者,莫不曲盡情詐,壓塞群疑。又能以譎數發擿姦伏。京師稱歎,以為渙有神筭。元興元年,病卒。百姓市道莫不咨嗟。男女老壯皆相與賦斂,致奠醊以千數。

渙喪西歸,道經弘農,民庶皆設槃桉於路。吏問其故,咸言平常持米到洛,為卒司所鈔,恆亡其半。自王君在事,不見侵枉,故來報恩。其政化懷物如此。民思其德,為立祠安陽亭西,每食輒弦歌而薦之。

永初二年,鄧太后詔曰:「夫忠良之吏,國家所以為理也。求之甚勤,得之至寡。故孔子曰:『才難不其然乎!』昔大司農朱邑、右扶風尹翁歸,政跡茂異,令名顯聞,孝宣皇帝嘉歎愍惜,而以黃金百斤策賜其子。故洛陽令王渙,秉清脩之節,蹈羔羊之義,盡心奉公,務在惠民,功業未遂,不幸早世,百姓追思,為之立祠。自非忠愛之至,孰能若斯者乎!今以渙子石為郎中,以勸勞勤。」延熹中,桓帝事黃老道,悉毀諸房祀,唯特詔密縣存故太傅卓茂廟,洛陽留王渙祠焉。

鐔顯後亦知名,安帝時豫州刺史。時天下飢荒,競為盜賊,州界收捕且萬餘人。顯愍其困窮,自陷刑辟,輒擅赦之,因自劾奏。有詔勿理。後位至長樂衛尉。

自渙卒後,連詔三公特選洛陽令,皆不稱職。永和中,以劇令勃海任峻補之。峻擢用文武吏,皆盡其能,糾剔姦盜,不得旋踵,一歲斷獄,不過數十。威風猛於渙,而文理不及之。峻字叔高,終於太山太守。

許荊字少張,會稽陽羨人也。祖父武,太守第五倫舉為孝廉。武以二弟晏、普未顯,欲令成名,乃請之曰:「禮有分異之義,家有別居之道。」於是共割財產以為三分,武自取肥田廣宅奴婢強者,二弟所得並悉劣少。鄉人皆稱弟克讓而鄙武貪婪,晏等以此並得選舉。武乃會宗親,泣曰:「吾為兄不肖,盜聲竊位,二弟年長,未豫榮祿,所以求得分財,自取大譏。今理產所增,三倍於前,悉以推二弟,一無所留。」於是郡中翕然,遠近稱之。位至長樂少府。

荊少為郡吏,兄子世嘗報讎殺人,怨者操兵攻之。荊聞,乃出門逆怨者,跪而言曰:「世前無狀相犯,咎皆在荊不能訓導。兄既早沒,一子為嗣,如令死者傷其滅絕,願殺身代之。」怨家扶荊起,曰:「許掾郡中稱賢,吾何敢相侵?」因遂委去。荊名譽益著。太守黃兢舉孝廉。

和帝時,稍遷桂陽太守。郡濱南州,風俗脆薄,不識學義。荊為設喪紀婚姻制度,使知禮禁。嘗行春到耒陽縣,人有蔣均者,兄弟爭財,互相言訟。荊對之歎曰:「吾荷國重任,而教化不行,咎在太守。」乃顧使吏上書陳狀,乞詣廷尉。均兄弟感悔,各求受罪。在事十二年,父老稱歌。以病自上,徵拜諫議大夫,卒於官。桂陽人為立廟樹碑。

荊孫戫,靈帝時為太尉。

孟嘗字伯周,會稽上虞人也。其先三世為郡吏,並伏節死難。嘗少脩操行,仕郡為戶曹史。上虞有寡婦至孝養姑。姑年老壽終,夫女弟先懷嫌忌,乃誣婦厭苦供養,加鴆其母,列訟縣庭。郡不加尋察,遂結竟其罪。嘗先知枉狀,備言之於太守,太守不為理。嘗哀泣外門,因謝病去,婦竟冤死。自是郡中連旱二年,禱請無所獲。後太守殷丹到官,訪問其故,嘗詣府具陳寡婦冤誣之事。因曰:「昔東海孝婦,感天致旱,于公一言,甘澤時降。宜戮訟者,以謝冤魂,庶幽枉獲申,時雨可期。」丹從之,即刑訟女而祭婦墓,天應澍雨,穀稼以登。

嘗後策孝廉,舉茂才,拜徐令。州郡表其能,遷合浦太守。郡不產穀實,而海出珠寶,與交阯比境,常通商販,貿糴糧食。先時宰守並多貪穢,詭人採求,不知紀極,珠遂漸徙於交阯郡界。於是行旅不至,人物無資,貧者餓死於道。嘗到官,革易前敝,求民病利。曾未踰歲,去珠復還,百姓皆反其業,商貨流通,稱為神明。

以病自上,被徵當還,吏民攀車請之。嘗既不得進,乃載鄉民船夜遁去。隱處窮澤,身自耕傭。鄰縣士民慕其德,就居止者百餘家。

桓帝時,尚書同郡楊喬上書薦嘗曰:「臣前後七表言故合浦太守孟嘗,而身輕言微,終不蒙察。區區破心,徒然而已。嘗安仁弘義,耽樂道德,清行出俗,能幹絕群。前更守宰,移風改政,去珠復還,飢民蒙活。且南海多珍,財產易積,掌握之內,價盈兼金,而嘗單身謝病,躬耕壟次,匿景藏采,不揚華藻。實羽翮之美用,非徒腹背之毛也。而沈淪草莽,好爵莫及,廊廟之寶,棄於溝渠。且年歲有訖,桑榆行盡,而忠貞之節,永謝聖時。臣誠傷心,私用流涕。夫物以遠至為珍,士以稀見為貴。槃木朽株,為萬乘用者,左右為之容耳。王者取士,宜拔眾之所貴。臣以斗筲之姿,趨走日月之側。思立微節,不敢苟私鄉曲。竊感禽息,亡身進賢。」嘗竟不見用。年七十,卒于家。

第五訪字仲謀,京兆長陵人。司空倫之族孫也。少孤貧,常傭耕以養兄嫂。有閑暇,則以學文。仕郡為功曹,察孝廉,補新都令。政平化行,三年之閒,鄰縣歸之,戶口十倍。

遷張掖太守。歲飢,粟石數千,訪乃開倉賑給以救其敝。吏懼譴,爭欲上言。訪曰:「若上須報,是棄民也。太守樂以一身救百姓!」遂出穀賦人。順帝璽書嘉之。由是一郡得全。歲餘,官民並豐,界無姦盜。

遷南陽太守,去官。拜護羌校尉,邊境服其威信。卒於官。

劉矩字叔方,沛國蕭人也。叔父光,順帝時為司徒。矩少有高節,以叔父遼未得仕進,遂絕州郡之命。太尉朱寵、太傅桓焉嘉其志義,故叔遼以此為諸公所辟,拜議郎,矩乃舉孝廉。

稍遷雍丘令,以禮讓化之。其無孝義者,皆感悟自革。民有爭訟,矩常引之於前,提耳訓告,以為忿恚可忍,縣官不可入,使歸更尋思。訟者感之,輒各罷去。其有路得遺者,皆推尋其主。在縣四年,以母憂去官。

後太尉胡廣舉矩賢良方正,四遷為尚書令。矩性亮直,不能諧附貴埶,以是失大將軍梁冀意,出為常山相,以疾去官。時冀妻兄孫祉為沛相,矩懼為所害,不敢還鄉里,乃投彭城友人家。歲餘,冀意少悟,乃止。補從事中郎,復為尚書令,遷宗正、太常。

延熹四年,代黃瓊為太尉。瓊復為司空,矩與瓊及司徒种暠同心輔政,號為賢相。時連有災異,司隸校尉以劾三公。尚書朱穆上疏,稱矩等良輔,及言殷湯、高宗不罪臣下之義。帝不省,竟以蠻夷反叛免。後復拜太中大夫。

靈帝初,代周景為太尉。矩再為上公,所辟召皆名儒宿德。不與州郡交通。順辭默諫,多見省用。復以日食免。因乞骸骨,卒於家。

劉寵字祖榮,東萊牟平人,齊悼惠王之後也。悼惠王子孝王將閭,將閭少子封牟平侯,子孫家焉。父丕,博學,號為通儒。

寵少受父業,以明經舉孝廉,除東平陵令,以仁惠為吏民所愛。母疾,棄官去。百姓將送塞道,車不得進,乃輕服遁歸。

後四遷為豫章太守,又三遷拜會稽太守。山民愿朴,乃有白首不入市井者,頗為官吏所擾。寵簡除煩苛,禁察非法,郡中大化。徵為將作大匠。山陰縣有五六老叟,尨眉皓髮,自若邪山谷閒出,人齎百錢以送寵。寵勞之曰:「父老何自苦?」對曰:「山谷鄙生,未嘗識郡朝。它守時吏發求民閒,至夜不絕,或狗吠竟夕,民不得安。自明府下車以來,狗不夜吠,民不見吏。年老遭值聖明,今聞當見棄去,故自扶奉送。」寵曰:「吾政何能及公言邪?勤苦父老!」為人選一大錢受之。

轉為宗正、大鴻臚。延熹四年,代黃瓊為司空,以陰霧愆陽免。頃之,拜將作大匠,復為宗正。建寧元年,代王暢為司空,頻遷司徒、太尉。二年,以日食策免,歸鄉里。

寵前後歷宰二郡,累登卿相,而准約省素,家無貨積。嘗出京師,欲息亭舍,亭吏止之,曰:「整頓洒埽,以待劉公,不可得也。」寵無言而去,時人稱其長者。以老病卒于家。

弟方,官至山陽太守。方有二子:岱字公山,繇字正禮。兄弟齊名稱。

董卓入洛陽,岱從侍中出為兗州刺史。虛己愛物,為士人所附。初平三年,青州黃巾賊入兗州,殺任城相鄭遂,轉入東平。岱擊之,戰死。

興平中,繇為楊州牧、振威將軍。時袁術據淮南,繇乃移居曲阿。值中國喪亂,士友多南奔,繇攜接收養,與同優劇,甚得名稱。袁術遣孫策攻破繇,因奔豫章,病卒。

仇覽字季智,一名香,陳留考城人也。少為書生淳默,鄉里無知者。年四十,縣召補吏,選為蒲亭長。勸人生業,為制科令,至於果菜為限,雞豕有數,農事既畢,乃令子弟群居,還就黌學。其剽輕游恣者,皆役以田桑,嚴設科罰。躬助喪事,賑恤窮寡。期年稱大化。覽初到亭,人有陳元者,獨與母居,而母詣覽告元不孝。覽驚曰:「吾近日過舍,廬落整頓,耕耘以時。此非惡人,當是教化未及至耳。母守寡養孤,苦身投老,柰何肆忿於一朝,欲致子以不義乎?」母聞感悔,涕泣而去。覽乃親到元家,與其母子飲,因為陳人倫孝行,譬以禍福之言。元卒成孝子。鄉邑為之諺曰:「父母何在在我庭,化我鳲梟哺所生。」

時考城令河內王渙,政尚嚴猛,聞覽以德化人,署為主簿。謂覽曰:「主簿聞陳元之過,不罪而化之,得無少鷹鸇之志邪?」覽曰:「以為鷹鸇,不若鸞鳳。」渙謝遣曰:「枳棘非鸞鳳所棲,百里豈大賢之路?今日太學曳長裾,飛名譽,皆主簿後耳。以一月奉為資,勉卒景行。」

覽入太學。時諸生同郡符融有高名,與覽比宇,賓客盈室。覽常自守,不與融言。融觀其容止,心獨奇之,乃謂曰:「與先生同郡壤,鄰房牖。今京師英雄四集,志士交結之秋,雖務經學,守之何固?」覽乃正色曰:「天子脩設太學,豈但使人游談其中!」高揖而去,不復與言。後融以告郭林宗,林宗因與融齎刺就房謁之,遂請留宿。林宗嗟歎,下床為拜。

覽學畢歸鄉里,州郡並請,皆以疾辭。雖在宴居,必以禮自整。妻子有過,輒免冠自責。妻子庭謝,候覽冠,乃敢升堂。家人莫見喜怒聲色之異。後徵方正,遇疾而卒。

三子皆有文史才,少子玄,最知名。

童恢字漢宗,琅邪姑幕人也。父仲玉,遭世凶荒,傾家賑卹,九族鄉里賴全者以百數。仲玉早卒。

恢少仕州郡為吏,司徒楊賜聞其執法廉平,乃辟之。及賜被劾當免,掾屬悉投刺去,恢獨詣闕爭之。及得理,掾屬悉歸府,恢杖策而逝。由是論者歸美。

復辟公府,除不其令。吏人有犯違禁法,輒隨方曉示。若吏稱其職,人行善事者,皆賜以酒肴之禮,以勸勵之。耕織種收,皆有條章。一境清靜,牢獄連年無囚。比縣流人歸化,徙居二萬餘戶。民嘗為虎所害,乃設檻捕之,生獲二虎。恢聞而出,沟虎曰:「天生萬物,唯人為貴。虎狼當食六畜,而殘暴於人。王法殺人者死,傷人則論法。汝若是殺人者,當垂頭服罪;自知非者,當號呼稱冤。」一虎低頭閉目,狀如震懼,即時殺之。其一視恢鳴吼,踴躍自奮,遂令放釋。吏人為之歌頌。青州舉尤異,遷丹陽太守,暴疾而卒。

弟翊字漢文,名高於恢,宰府先辟之。翊陽喑不肯仕,及恢被命,乃就孝廉,除須昌長。化有異政,吏人生為立碑。聞舉將喪,棄官歸。後舉茂才,不就。卒於家。

贊曰:政畏張急,理善亨鮮。推忠以及,眾瘼自蠲。一夫得情,千室鳴弦。懷我風愛,永載遺賢。


\end{pinyinscope}