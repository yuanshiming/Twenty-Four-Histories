\article{袁紹劉表列傳上}

\begin{pinyinscope}
袁紹字本初,汝南汝陽人,司徒湯之孫。父成,五官中郎將,紹壯健好交結,大將軍梁冀以下莫不善之。

紹少為郎,除濮陽長,遭母憂去官。三年禮竟,追感幼孤,又行父服。服闋,徙居洛陽。紹有姿貌威容,愛士養名。既累世台司,賓客所歸,加傾心折節,莫不爭赴其庭,士無貴賤,與之抗禮,輜軿柴轂,填接街陌。內官皆惡之。中常侍趙忠言於省內曰:「袁本初坐作聲價,好養死士,不知此兒終欲何作。」叔父太傅隗聞而呼紹,以忠言責之,紹終不改。

後辟大將軍何進掾,為侍御史、虎賁中郎將。中平五年,初置西園八校尉,以紹為佐軍校尉。

靈帝崩,紹勸何進徵董卓等眾軍,脅太后誅諸宦官,轉紹司隸校尉。語已見何進傳。及卓將兵至,騎都尉太山鮑信說紹曰:「董卓擁制強兵,將有異志,今不早圖,必為所制。及其新至疲勞,襲之可禽也。」紹畏卓,不敢發。頃之,卓議欲廢立,謂紹曰:「天下之主,宜得賢明,每念靈帝,令人憤毒。董侯似可,今當立之。」紹曰:「今上富於春秋,未有不善宣於天下。若公違禮任情,廢嫡立庶,恐眾議未安。」卓案劍叱紹曰:「豎子敢然!天下之事,豈不在我?我欲為之,誰敢不從!」紹詭對曰:「此國之大事,請出與太傅議之。」卓復言「劉氏種不足復遺」。紹勃然曰:「天下健者,豈惟董公!」橫刀長揖徑出。懸節於上東門,而奔冀州。

董卓購募求紹。時侍中周珌、城門校尉伍瓊為卓所信待,瓊等陰為紹說卓曰:「夫廢立大事,非常人所及。袁紹不達大體,恐懼出奔,非有它志。今急購之,埶必為變。袁氏樹恩四世,門生故吏遍於天下,若收豪傑以聚徒眾,英雄因之而起,則山東非公之有也。不如赦之,拜一郡守,紹喜於免罪,必無患矣。」卓以為然,乃遣授紹勃海太守,封邟鄉侯。紹猶稱兼司隸。

初平元年,紹遂以勃海起兵,以從弟後將軍術、冀州牧韓馥、豫州刺史孔骸、兗州刺史劉岱、陳留太守張邈、廣陵太守張超、河內太守王匡、山陽太守袁遺、東郡太守橋瑁、濟北相鮑信等同時俱起,眾各數萬,以討卓為名。紹與王匡屯河內,骸屯潁川,馥屯鄴,餘軍咸屯酸棗,約盟,遙推紹為盟主。紹自號車騎將軍,領司隸校尉。

董卓聞紹起山東,乃誅紹叔父隗,及宗族在京師者,盡滅之。卓乃遣大鴻臚韓融、少府陰循、執金吾胡母班、將作大匠吳循、越騎校尉王瑰譬解紹等諸軍。紹使王匡殺班、瑰、吳循等,袁術亦執殺陰循,惟韓融以名德免。

是時豪傑既多附紹,且感其家禍,人思為報,州郡蜂起,莫不以袁氏為名。韓馥見人情歸紹,忌方得眾,恐將圖己,常遣從事守紹門,不聽發兵。橋瑁乃詐作三公移書,傳驛州郡,說董卓罪惡,天子危逼,企望義兵,以釋國難。馥於是方聽紹舉兵。乃謀於眾曰:「助袁氏乎?助董氏乎?」治中劉惠勃然曰:「興兵為國,安問袁、董?」馥意猶深疑於紹,每貶節軍糧,欲使離散。

明年,馥將麴義反畔,馥與戰失利。紹既恨馥,乃與義相結。紹客逢紀謂紹曰:「夫舉大事,非據一州,無以自立。今冀部強實,而韓馥庸才,可密要公孫瓚將兵南下,馥聞必駭懼。并遣辯士為陳禍福,馥迫於倉卒,必可因據其位。」紹然之,益親紀,即以書與瓚。瓚遂引兵而至,外託董卓,而陰謀襲馥。紹乃使外甥陳留高幹及潁川荀諶等說馥曰:「公孫瓚乘勝來南,而諸郡應之。袁車騎引軍東向,其意未可量也。竊為將軍危之。」馥懼,曰:「然則為之柰何?」諶曰:「君自料寬仁容眾,為天下所附,孰與袁氏?」馥曰:「不如也。」「臨危吐決,智勇邁於人,又孰與袁氏?」馥曰:「不如也。」「世布恩德,天下家受其惠,又孰與袁氏?」馥曰:「不如也。」諶曰:「勃海雖郡,其實州也。今將軍資三不如之埶,久處其上,袁氏一時之傑,必不為將軍下也。且公孫提燕、代之卒,其鋒不可當。夫冀州天下之重資,若兩軍并力,兵交城下,危亡可立而待也。夫袁氏將軍之舊,且為同盟。當今之計,莫若舉冀州以讓袁氏,必厚德將軍,公孫瓚不能復與之爭矣。是將軍有讓賢之名,而身安於太山也。願勿有疑。」馥素性恇怯,因然其計。馥長史耿武、別駕閔純、騎都尉沮授聞而諫曰:「冀州雖鄙,帶甲百萬,穀支十年。袁紹孤客窮軍,仰我鼻息,譬如嬰兒在股掌之上,絕其哺乳,立可餓殺。柰何欲以州與之?」馥曰;「吾袁氏故吏,且才不如本初。度德而讓,古人所貴,諸君獨何病焉?」先是,馥從事趙浮、程渙將強弩萬人屯孟津,聞之,率兵馳還,請以拒紹,馥又不聽。乃避位,出居中常侍趙忠故舍,遣子送印綬以讓紹。

紹遂領冀州牧,承制以馥為奮威將軍,而無所將御。引沮授為別駕,因謂授曰:「今賊臣作亂,朝廷遷移。吾歷世受寵,志竭力命,興復漢室。然齊桓非夷吾不能成霸,句踐非范蠡無以存國。今欲與卿戮力同心,共安社稷,將何以匡濟之乎?」授進曰:「將軍弱冠登朝,播名海內。值廢立之際,忠義奮發,單騎出奔,董卓懷懼,濟河而北,勃海稽服。擁一郡之卒,撮冀州之眾,威陵河朔,名重天下。若舉軍東向,則黃巾可埽;還討黑山,則張燕可滅;回師北首,則公孫必禽;震脅戎狄,則匈奴立定。橫大河之北,合四州之地,收英雄之士,擁百萬之眾,迎大駕於長安,復宗廟於洛邑,號令天下,誅討未服。以此爭鋒,誰能御之!比及數年,其功不難。」紹喜曰:「此吾心也。」即表授為奮武將軍,使監護諸將。

魏郡審配,鉅鹿田豐,並以正直不得志於韓馥。紹乃以豐為別駕,配為治中,甚見器任。馥自懷猜懼,辭紹索去,往依張邈。後紹遣使詣邈,有所計議,因共耳語。馥時在坐,謂見圖謀,無何,如廁自殺。

其冬,公孫瓚大破黃巾,還屯槃河,威震河北,冀州諸城無不望風響應。紹乃自擊之。瓚兵三萬,列為方陳,分突騎萬匹,翼軍左右,其鋒甚銳。紹先令麴義領精兵八百,強弩千張,以為前登。瓚輕其兵少,縱騎騰之,義兵伏楯下,一時同發,瓚軍大敗,斬其所置冀州刺史嚴綱,獲甲首千餘級。麴義追至界橋,瓚斂兵還戰,義復破之,遂到瓚營,拔其牙門,餘眾皆走。紹在後十數里,聞瓚已破,發鞍息馬,唯衛帳下強弩數十張,大戟士百許人。瓚散兵二千餘騎卒至,圍紹數重,射矢雨下。田豐扶紹,使卻入空垣。紹脫兜鍪抵地,曰:「大丈夫當前鬥死,而反逃垣牆閒邪?」促使諸弩競發,多傷瓚騎。眾不知是紹,頗稍引卻。會麴義來迎,騎乃散退。三年,瓚又遣兵至龍湊挑戰,紹復擊破之。瓚遂還幽州,不敢復出。

四年初,天子遣太僕趙岐和解關東,使各罷兵。瓚因此以書譬紹曰:「趙太僕以周、邵之德,銜命來征,宣揚朝恩,示以和睦,曠若開雲見日,何喜如之!昔賈復、寇恂爭相危害,遇世祖解紛,遂同輿並出。釁難既釋,時人美之。自惟邊鄙,得與將軍共同斯好,此誠將軍之羞,而瓚之願也。」紹於是引軍南還。

三月上巳,大會賓徒於薄落津。聞魏郡兵反,與黑山賊干毒等數萬人共覆鄴城,殺郡守。坐中客家在鄴者,皆憂怖失色,或起而啼泣,紹容貌自若,不改常度。賊有陶升者,自號「平漢將軍」,獨反諸賊,將部眾踰西城入,閉府門,具車重,載紹家及諸衣冠在州內者,身自扞衛,送到斥丘。紹還,因屯斥丘,以陶升為建義中郎將。六月,紹乃出軍,入朝歌鹿腸山蒼巖谷口,討干毒。圍攻五日,破之,斬毒及其眾萬餘級。紹遂尋山北行,進擊諸賊左髭丈八等,皆斬之,又擊劉石、青牛角、黃龍、左校、郭大賢、李大目、于氐根等、復斬數萬級,皆屠其屯壁。遂與黑山賊張燕及四營屠各、鴈門烏桓戰於常山。燕精兵數萬,騎數千匹,連戰十餘日,燕兵死傷雖多,紹軍亦疲,遂各退。麴義自恃有功,驕縱不軌,紹召殺之,而并其眾。

興平二年,拜紹右將軍。其冬,車駕為李傕等所追於曹陽,沮授說紹曰:「將軍累葉台輔,世濟忠義。今朝廷播越,宗廟殘毀,觀諸州郡,雖外託義兵,內實相圖,未有憂存社稷卹人之意。且今州城粗定,兵強士附,西迎大駕,即宮鄴都,挾天子而令諸侯,蓄士馬以討不庭,誰能禦之?」紹將從其計。潁川郭圖、淳于瓊曰:「漢室陵遲,為日久矣,今欲興之,不亦難乎?且英雄並起,各據州郡,連徒聚眾,動有萬計,所謂秦失其鹿,先得者王。今迎天子,動輒表聞,從之則權輕,違之則拒命,非計之善者也。」授曰:「今迎朝廷,於義為得,於時為宜。若不早定,必有先之者焉。夫權不失幾,功不猒速,願其圖之。」帝立既非紹意,竟不能從。

紹有三子:譚字顯思,熙字顯雍,尚字顯甫。譚長而惠,尚少而美。紹後妻劉有寵,而偏愛尚,數稱於紹,紹亦奇其姿容,欲使傳嗣。乃以譚繼兄後,出為青州刺史。沮授諫曰:「世稱萬人逐兔,一人獲之,貪者悉止,分定故也。且年均以賢,德均則卜,古之制也。願上惟先代成則之誡,下思逐兔分定之義。若其不改,禍始此矣。」紹曰:「吾欲令諸子各據一州,以視其能。」於是以中子熙為幽州刺史,外甥高幹為并州刺史。

建安元年,曹操迎天子都許,乃下詔書於紹,責以地廣兵多而專自樹黨,不聞勤王之師而但擅相討伐。紹上書曰:

臣聞昔有哀歎而霜隕,悲哭而崩城者。每讀其書,謂為信然,於今況之,乃知妄作。何者?臣出身為國,破家立事,至乃懷忠獲釁,抱信見疑,晝夜長吟,剖肝泣血,曾無崩城隕霜之應,故鄒衍、杞婦何能感徹。

臣以負薪之資,拔於陪隸之中,奉職憲臺,擢授戎校。常侍張讓等滔亂天常,侵奪朝威,賊害忠德,扇動姦黨。故大將軍何進忠國疾亂,義心赫怒,以臣頗有一介之節,可責以鷹犬之功,故授臣以督司,諮臣以方略。臣不敢畏憚強禦,避禍求福,與進合圖,事無違異。忠策未盡而元帥受敗,太后被質,宮室焚燒,陛下聖德幼沖,親遭厄困。時進既被害,師徒喪沮,臣獨將家兵百餘人,抽戈承明,竦劍翼室,虎叱群司,奮擊凶醜,曾不浹辰,罪人斯殄。此誠愚臣效命之一驗也。

會董卓乘虛,所圖不軌。臣父兄親從,並當大位,不憚一室之禍,苟惟寧國之義,故遂解節出奔,創謀河外。時卓方貪結外援,招悅英豪,故即臣勃海,申以軍號,則臣之與卓,未有纖芥之嫌。若使苟欲滑泥揚波,偷榮求利,則進可以享竊祿位,退無門戶之患。然臣愚所守,志無傾奪,故遂引會英雄,興師百萬,飲馬孟津,歃血漳河。會故冀州牧韓馥懷挾逆謀,欲專權埶,絕臣軍糧,不得踵係,至使猾虜肆毒,害及一門,尊卑大小,同日并戮。鳥獸之情,猶知號呼。臣所以蕩然忘哀,貌無隱戚者,誠以忠孝之節,道不兩立,顧私懷己,不能全功。斯亦愚臣破家徇國之二驗也。

又黃巾十萬焚燒青、兗、黑山、張楊蹈藉冀

城。臣乃旋師,奉辭伐畔。金鼓未震,狡敵知亡,故韓馥懷懼,謝咎歸土,張楊、黑山同時乞降。臣時輒承制,竊比竇融,以議郎曹操權領兗州牧。會公孫瓚師旅南馳,陸掠北境,臣即星駕席卷,與瓚交鋒。假天之威,每戰輒克。臣備公族子弟,生長京輦,頗聞俎豆,不習干戈;加自乃祖先臣以來,世作輔弼,咸以文德盡忠,得免罪戾。臣非與瓚角戎馬之埶,爭戰陣之功者也。誠以賊臣不誅,春秋所貶,苟云利國,專之不疑。故冒踐霜雪,不憚劬勤,實庶一捷之福,以立終身之功。社稷未定,臣誠恥之。太僕趙岐銜命來征,宣明陛下含弘之施,蠲除細故,與下更新,奉詔之日,引師南轅。是臣畏怖天威,不敢怠慢之三驗也。

又臣所上將校,率皆清英宿德,令名顯達,登鋒履刃,死者過半,勤恪之功,不見書列。而州郡牧守,競盜聲名,懷持二端,優游顧望,皆列土錫圭,跨州連郡,是以遠近狐疑,議論紛錯者也。臣聞守文之世,德高者位尊;倉卒之時,功多者賞厚。陛下播越非所,洛邑乏祀,海內傷心,志士憤惋。是以忠臣肝腦塗地,肌膚橫分而無悔心者,義之所感故也。今賞加無勞,以攜有德;杜黜忠功,以疑眾望。斯豈腹心之遠圖?將乃讒慝之邪說使之然也?臣爵為通侯,位二千石。殊恩厚德,臣既叨之,豈敢闚覬重禮,以希彤弓玈矢之命哉?誠傷偏裨列校,勤不見紀,盡忠為國,侴成重愆。斯蒙恬所以悲號於邊獄,白起歔欷於杜郵也。太傅日磾位為師保,任配東征,而耗亂王命,寵任非所,凡所舉用,皆眾所捐棄。而容納其策,以為謀主,令臣骨肉兄弟,還為讎敵,交鋒接刃,搆難滋甚。臣雖欲釋甲投戈,事不得已。誠恐陛下日月之明,有所不照,四聰之聽有所不聞,乞下臣章,咨之群賢,使三槐九棘,議臣罪戾。若以臣今行權為釁,則桓、文當有誅絕之刑;若以眾不討賊為賢,則趙盾可無書弒之貶矣。臣雖小人,志守一介。若使得申明本心,不愧先帝,則伏首歐刀,褰衣就鑊,臣之願也。惟陛下垂尸鳩之平,絕邪諂之論,無令愚臣結恨三泉。

於是以紹為太尉,封鄴侯。時曹操自為大將軍,紹恥為之下,偽表辭不受。操大懼,乃讓位於紹。二年,使將作大匠孔融持節拜紹大將軍,錫弓矢節鉞,虎賁百人,兼督冀、青、幽、并四州,然後受之。

紹每得詔書,患有不便於己,乃欲移天子自近,使說操以許下埤溼,洛陽殘破,宜徙都甄城,以就全實。操拒之。田豐說紹曰:「徙都之計,既不克從,宜早圖許,奉迎天子,動託詔令,響號海內,此筭之上者。不爾,終為人所禽,雖悔無益也。」紹不從。四年春,擊公孫瓚,遂定幽土,事在瓚傳。

紹既并四州之地,眾數十萬,而驕心轉盛,貢御稀簡。主簿耿包密白紹曰:「赤德衰盡,袁為黃胤,宜順天意,以從民心。」紹以包白事示軍府僚屬,議者以包妖妄宜誅。紹知眾情未同,不得已乃殺包以弭其跡。於是簡精兵十萬,騎萬匹,欲出攻許,以審配、逢紀統軍事,田豐、荀諶及南陽許攸為謀主,顏良、文醜為將帥。沮授進說曰:「近討公孫,師出歷年,百姓疲敝,倉庫無積,賦役方殷,此國之深憂也。宜先遣使獻捷天子,務農逸人。若不得通,乃表曹操隔我王路,然後進屯黎陽,漸營河南,益作舟船,繕修器械,分遣精騎,抄其邊鄙,令彼不得安,我取其逸。如此可坐定也。」郭圖、審配曰:「兵書之法,十圍五攻,敵則能戰。今以明公之神武,連河朔之強眾,以伐曹操,兵埶譬若覆手。今不時取,後難圖也。」授曰:「蓋救亂誅暴,謂之義兵;恃眾憑強,謂之驕兵。義者無敵,驕者先滅。曹操奉迎天子,建宮許都。今舉師南向,於義則違。且廟勝之策,不在彊弱。曹操法令既行,士卒精練,非公孫瓚坐受圍者也。今棄萬安之術,而興無名之師,竊為公懼之。」圖等曰:「武王伐紂,不為不義;況兵加曹操,而云無名!且公師徒精勇,將士思奮,而不及時早定大業,所謂『天與不取,反受其咎』。此越之所以霸,吳之所以滅也。監軍之計,在於將軍,而非見時知幾之變也。」紹納圖言。圖等因是譖沮授曰:「授監統內外,威震三軍,若其浸盛,何以制之!夫臣與主同者亡,此黃石之所忌也。且御眾於外,不宜知內。」紹乃分授所統為三都督,使授及郭圖、淳于瓊各典一軍,未及行。

五年,左將軍劉備殺徐州刺史車冑,據沛以背曹操。操懼,乃自將征備。田豐說紹曰:「與公爭天下者,曹操也。操今東擊劉備,兵連未可卒解,今舉軍而襲其後,可一往而定。兵以幾動,斯其時也。」紹辭以子疾,未得行。豐舉杖擊地曰:「嗟乎,事去矣!夫遭難遇之幾,而以嬰兒病失其會,惜哉!」紹聞而怒之,從此遂疏焉。

曹操畏紹過河,乃急擊備,遂破之。備奔紹,紹於是進軍攻許。田豐以既失前幾,不宜便行,諫紹曰:「曹操既破劉備,則許下非復空虛。且操善用兵,變化無方,眾雖少,未可輕也。今不如久持之。將軍據山河之固,擁四州之眾,外結英雄,內修農戰,然後簡其精銳,分為奇兵,乘虛迭出,以擾河南,救右則擊其左,救左則擊其右,使敵疲於奔命,人不得安業,我未勞而彼已困,不及三年,可坐剋也。今釋廟勝之策而決成敗於一戰,若不如志,悔無及也。」紹不從。豐強諫忤紹,紹以為沮眾,遂械繫之。乃先宣檄曰:

蓋聞明主圖危以制變,忠臣慮難以立權。曩者強秦弱主,趙高執柄,專制朝命,威福由己,終有望夷之禍,汙辱至今。及臻呂后,祿、產專政,擅斷萬機,決事禁省,下陵上替,海內寒心。於是絳侯、朱虛興威奮怒,誅夷逆暴,尊立太宗,故能道化興隆,光明融顯。此則大臣立權之明表也。

司空曹操祖父騰,故中常侍,與左悺、徐璜並作妖孽,饕餮放橫,傷化虐人。父嵩,乞饨攜養,因臧買位,輿金輦寶,輸貨權門,竊盜鼎司,傾覆重器。操姦閹遺醜,本無令德,僄狡鋒俠,好亂樂禍。幕府董統鷹揚,埽夷凶逆,續遇董卓侵官暴國,於是提劍揮鼓,發命東夏,廣羅英雄,棄瑕錄用,故遂與操參咨策略,謂其鷹犬之才,爪牙可任。至乃愚佻短慮,輕進易退,傷夷折衄,數喪師徒。幕府輒復分兵命銳,修完補輯,表行東郡太守、兗州刺史,被以虎文,授以偏師,嚔就威柄,冀獲秦師一克之報。而遂乘資跋扈,肆行酷烈,割剝元元,殘賢害善。故九江太守邊讓,英才俊逸,以直言正色,論不阿諂,身被梟懸之戮,妻孥受灰滅之咎。自是士林憤痛,人怨天怒,一夫奮臂,舉州同聲,故躬破於徐方,地奪於呂布,彷徨東裔,蹈據無所。幕府惟強幹弱枝之義,且不登畔人之黨,故復援旍擐甲,席卷赴征,金鼓響震,布眾破沮,拯其死亡之患,復其方伯之任。是則幕府無德於兗土,而有大造於操也。

會後鑾駕東反,群虜亂政。時冀州方有北鄙之警,匪遑離局,故使從事中郎徐勳就發遣操,使繕修郊廟,翼衛幼主。而便放志專行,威劫省禁,卑侮王僚,敗法亂紀,坐召三臺,專制朝政,爵賞由心,刑戮在口,所愛光五宗,所怨滅三族,群談者受顯誅,腹議者蒙隱戮,道路以目,百辟鉗口,尚書記期會,公卿充員品而已。

故太尉楊彪,歷典二司,元綱極位。操因睚眥,被以非罪,篣楚并兼,五毒俱至,觸情放慝,不顧憲章。又議郎趙彥,忠諫直言,議有可納,故聖朝含聽,改容加錫。操欲迷奪時明,杜絕言路,擅收立殺,不俟報聞。又梁孝王先帝母弟,墳陵尊顯,松柏桑梓猶宜恭肅。操率將吏士,親臨發掘,破棺裸尸,掠取金寶,至令聖朝流涕,士民傷懷。又署發丘中郎將、摸金校尉,所過毀突,無骸不露。身處三公之官,而行桀虜之態,汙國虐民,毒施人鬼。加其細政苛慘,科防互設,矰繳充蹊,阬阱塞路,舉手挂網羅,動足蹈機埳,是以兗、豫有無聊之人,帝都有呼嗟之怨。

歷觀古今書籍所載,貪殘虐烈無道之臣,於操為甚。莫府方詰外姦,未及整訓,加意含覆,冀可彌縫。而操豺狼野心,潛包禍謀,乃欲橈折棟梁,孤弱漢室,除忠害善,專為梟雄。往歲伐鼓北征,討公孫瓚,強禦桀逆,拒圍一年。操因其未破,陰交書命,欲託助王師,以見掩襲,故引兵造河,方舟北濟。會行人發露,瓚亦梟夷,故使鋒芒挫縮,厥圖不果。屯據敖倉,阻河為固,乃欲運螳蜋之斧,禦隆車之隧。莫府奉漢威靈,折衝宇宙,長戟百萬,胡騎千群,奮中黃、育、獲之士,騁良弓勁弩之埶,并州越太行,青州涉濟、漯,大軍汎黃河以角其前,荊州下宛、葉而掎其後。雷震虎步,並集虜廷,若舉炎火以焚飛蓬,覆滄海而注熛炭,有何不消滅者哉?

當今漢道陵遲,綱弛網絕,操以精兵七百,圍守宮闕,外稱陪衛,內以拘質,懼篡逆之禍,因斯而作。乃忠臣肝腦塗地之秋,烈士立功之會也。可不勗哉!

乃先遣顏良攻曹操別將劉延於白馬,紹自引兵至黎陽。沮授臨行,會其宗族,散資財以與之。曰:「埶存則威無不加,埶亡則不保一身。哀哉!」其弟宗曰:「曹操士馬不敵,君何懼焉?」授曰:「以曹兗州之明略,又挾天子以為資,我雖剋伯珪,眾實疲敝,而主驕將头,軍之破敗,在此舉矣。楊雄有言:『六國蚩蚩,為嬴弱姬。』今之謂乎!」曹操遂救劉延,擊顏良斬之。紹乃度河,壁延津南。沮授臨船歎曰:「上盈其志,下務其功,悠悠黃河,吾其濟乎!」遂以疾退,紹不許而意恨之,復省其所部,并屬郭圖。

紹使劉備、文醜挑戰,曹操又擊破之,斬文醜。再戰而禽二將,紹軍中大震。操還屯官度,紹進保陽武。沮授又說紹曰:「北兵雖眾,而勁果不及南軍;南軍穀少,而資儲不如北。南幸於急戰,北利在緩師。宜徐持久,曠以日月。」紹不從。連營稍前,漸逼官度,遂合戰。操軍不利,復還堅壁。紹為高櫓,起土山,射營中,皆蒙楯而行。操乃發石車擊紹樓,皆破,軍中呼曰「霹靂車」。紹為地道欲襲操,操輒於內為長塹以拒之。又遣奇兵襲紹運車,大破之,盡焚其穀食。

相持百餘日,河南人疲困,多畔應紹。紹遣淳于瓊等將兵萬餘人北迎糧運。沮授說紹可遣蔣奇別為支軍於表,以絕曹操之鈔。紹不從。許攸進曰:「曹操兵少而悉師拒我,許下餘守埶必空弱。若分遣輕軍,星行掩襲,許拔則操為成禽。如其未潰,可令首尾奔命,破之必也。」紹又不能用。會攸家犯法,審配收繫之,攸不得志,遂奔曹操,而說使襲取淳于瓊等,瓊等時宿在烏巢,去紹軍四十里。操自將步騎五千人,夜往攻破瓊等,悉斬之。

初,紹聞操擊瓊,謂長子譚曰:「就操破瓊,吾拔其營,彼固無所歸矣。」乃使高覽、張郃等攻操營,不下。二將聞瓊等敗,遂奔操。於是紹軍驚擾,大潰。紹與譚等幅巾乘馬,與八百騎度河,至黎陽北岸,入其將軍蔣義渠營。至帳下,把其手曰:「孤以首領相付矣。」義渠避帳而處之。使宣令焉。眾聞紹在,稍復集。餘眾偽降,曹操盡阬之,前後所殺八萬人。

沮授為操軍所執,乃大呼曰:「授不降也,為所執耳。」操見授謂曰:「分野殊異,遂用圮絕,不圖今日乃相得也。」授對曰:「冀州失策,自取奔北。授知力俱困,宜其見禽。」操曰:「本初無謀,不相用計。今喪亂過紀,國家未定,方當與君圖之。」授曰:「叔父、母、弟懸命袁氏,若蒙公靈,速死為福。」操歎曰:「孤早相得,天下不足慮也。」遂赦而厚遇焉。授尋謀歸袁氏,乃誅之。

紹外寬雅有局度,憂喜不形於色,而性矜愎自高,短於從善,故至於敗。及軍還,或謂田豐曰:「君必見重。」豐曰:「公貌寬而內忌,不亮吾忠,而吾數以至言迕之。若勝而喜,必能赦我,戰敗而怨,內忌將發。若軍出有利,當蒙全耳,今既敗矣,吾不望生。」紹還,曰:「吾不用田豐言,果為所笑。」遂殺之。

官度之敗,審配二子為曹操所禽。孟岱與配有隙,因蔣奇言於紹曰:「配在位專政,族大兵強,且二子在南,必懷反畔。」郭圖、辛評亦為然。紹遂以岱為監軍,代配守鄴。護軍逢紀與配不睦,紹以問之,紀對曰:「配天性烈直,每所言行,慕古人之節,不以二子在南為不義也,公勿疑之。」紹曰:「君不惡之邪?」紀曰:「先所爭者私情,今所陳者國事。」紹曰「善」。乃不廢配,配、由是更協。

冀州城邑多畔,紹復擊定之。自軍敗後發病,七年夏,薨。未及定嗣,逢紀、審配宿以驕侈為譚所病,辛評、郭圖皆比於譚而與配、紀有隙。眾以譚長,欲立之。配等恐譚立而評等為害,遂矯紹遺命,奉尚為嗣。


\end{pinyinscope}