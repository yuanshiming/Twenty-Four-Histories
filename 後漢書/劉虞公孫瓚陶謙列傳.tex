\article{劉虞公孫瓚陶謙列傳}

\begin{pinyinscope}
劉虞字伯安,東海郯人也。祖父嘉,光祿勳。虞初舉孝廉,稍遷幽州刺史,民夷感其德化,自鮮卑、烏桓、夫餘、穢貊之輩,皆隨時朝貢,無敢擾邊者,百姓歌悅之。公事去官。中平初,黃巾作亂,攻破冀州諸郡,拜虞甘陵相,綏撫荒餘,以蔬儉率下。遷宗正。

後車騎將軍張溫討賊邊章等,發幽州烏桓三千突騎,而牢稟逋懸,皆畔還本國。前中山相張純私謂前太山太守張舉曰:「今烏桓既畔,皆願為亂,涼州賊起,朝廷不能禁。又洛陽人妻生子兩頭,此漢祚衰盡,天下有兩主之徵也。子若與吾共率烏桓之眾以起兵,庶幾可定大業。」舉因然之。四年,純等遂與烏桓大人共連盟,攻薊下,燔燒城郭,虜略百姓,殺護烏桓校尉箕稠、右北平太守劉政、遼東太守陽終等,眾至十餘萬,屯肥如。舉稱「天子」,純稱「彌天將軍安定王」,移書州郡,云舉當代漢,告天子避位,敕公卿奉迎。純又使烏桓峭王等步騎五萬,入青冀二州,攻破清河、平原,殺害吏民。朝廷以虞威信素著,恩積北方,明年,復拜幽州牧。虞到薊,罷省屯兵,務廣恩信。遣使告峭王等以朝恩寬弘,開許善路。又設賞購舉、純。舉、純走出塞,餘皆降散。純為其客王政所殺,送首詣虞。靈帝遣使者就拜太尉,封容丘侯。

及董卓秉政,遣使者授虞大司馬,進封襄賁侯。初平元年,復徵代袁隗為太傅。道路隔塞,王命竟不得達。舊幽部應接荒外,資費甚廣,歲常割青、冀賦調二億有餘,以給足之。時處處斷絕,委輸不至,而虞務存寬政,勸督農植,開上谷胡巿之利,通漁陽鹽鐵之饒,民悅年登,穀石三十。青、徐士庶避黃巾之難歸虞者百餘萬口,皆收視溫恤,為安立生業,流民皆忘其遷徙。虞雖為上公,天性節約,敝衣繩履,食無兼肉,遠近豪俊夙僭奢者,莫不改操而歸心焉。

初,詔令公孫瓚討烏桓,受虞節度。瓚但務會徒眾以自強大,而縱任部曲,頗侵擾百姓,而虞為政仁愛,念利民物,由是與瓚漸不相平。二年,冀州刺史韓馥、勃海太守袁紹及山東諸將議,以朝廷幼沖,逼於董卓,遠隔關塞,不知存否,以虞宗室長者,欲立為主。乃遣故樂浪太守張岐等齎議,上虞尊號。虞見岐等,厲色叱之曰:「今天下崩亂,主上蒙塵。吾被重恩,未能清雪國恥。諸君各據州郡,宜共戮力,盡心王室,而反造逆謀,以相垢誤邪!」固拒之。馥等又請虞領尚書事,承制封拜,復不聽。遂收斬使人。於是選掾右北平田疇、從事鮮于銀蒙險閒行,奉使長安。獻帝既思東歸,見疇等大悅。時虞子和為侍中,因此遣和潛從武關出,告虞將兵來迎。道由南陽,後將軍袁術聞其狀,遂質和,使報虞遣兵俱西。虞乃使數千騎就和奉迎天子,而術竟不遣之。

初,公孫瓚知術詐,固止虞遣兵,虞不從,瓚乃陰勸術執和,使奪其兵,自是與瓚仇怨益深。和尋得逃術還北,復為袁紹所留。瓚既累為紹所敗,而猶攻之不已,虞患其黷武,且慮得志不可復制,固不許行,而稍節其稟假。瓚怒,屢違節度,又復侵犯百姓。虞所賚賞典當胡夷,瓚數抄奪之。積不能禁,乃遣驛使奉章陳其暴掠之罪,瓚亦上虞稟糧不周,二奏交馳,互相非毀,朝廷依違而已。瓚乃築京於薊城以備虞。虞數請瓚,輒稱病不應。虞乃密謀討之,以告東曹掾右北平魏攸。攸曰:「今天下引領,以公為歸,謀臣爪牙,不可無也。瓚文武才力足恃,雖有小惡,固宜容忍。」虞乃止。

頃之攸卒,而積忿不已。四年冬,遂自率諸屯兵眾合十萬人以攻瓚。將行,從事代郡程緒免冑而前曰:「公孫瓚雖有過惡,而罪名未正。明公不先告曉使得改行,而兵起蕭牆,非國之利。加勝敗難保,不如駐兵,以武臨之,瓚必悔禍謝罪,所謂不戰而服人者也。」虞以緒臨事沮議,遂斬之以徇。戒軍士曰:「無傷餘人,殺一伯珪而已。」時州從事公孫紀者,瓚以同姓厚待遇之。紀知虞謀而夜告瓚。瓚時部曲放散在外,倉卒自懼不免,乃掘東城欲走。虞兵不習戰,又愛人廬舍,敕不聽焚燒,急攻圍不下。瓚乃簡募銳士數百人,因風縱火,直衝突之。虞遂大敗,與官屬北奔居庸縣。瓚追攻之,三日城陷,遂執虞并妻子還薊,猶使領州文書。會天子遣使者段訓增虞封邑,督六州事;拜瓚前將軍,封易侯,假節督幽、并、司、冀。瓚乃誣虞前與袁紹等欲稱尊號,脅訓斬虞於薊市。先坐而沟曰:「若虞應為天子者,天當風雨以相救。」時旱埶炎盛,遂斬焉。傳首京師,故吏尾敦於路劫虞首歸葬之。瓚乃上訓為幽州刺史。虞以恩厚得眾,懷被北州,百姓流舊,莫不痛惜焉。

初,虞以儉素為操,冠敝不改,乃就補其穿。及遇害,瓚兵搜其內,而妻妾服羅紈,盛綺飾,時人以此疑之。和後從袁紹報瓚云。

公孫瓚字伯珪,遼西令支人也。家世二千石。瓚以母賤,遂為郡小吏。為人美姿貌,大音聲,言事辯慧。太守奇其才,以女妻之。後從涿郡盧植學於緱氏山中,略見書傳。舉上計吏。太守劉君坐事檻車徵,官法不聽吏下親近,瓚乃改容服,詐稱侍卒,身執徒養,御車到洛陽。太守當徙日南,瓚具豚酒於北芒上,祭辭先人,酹觴祝曰:「昔為人子,今為人臣,當詣日南。日南多瘴氣,恐或不還,便當長辭墳塋。」慷慨悲泣,再拜而去,觀者莫不歎息。既行,於道得赦。

瓚還郡,舉孝廉,除遼東屬國長史。嘗從數十騎出行塞下,卒逢鮮卑數百騎。瓚乃退入空亭,約其從者曰:「今不奔之,則死盡矣。」乃自持兩刃矛,馳出衝賊,殺傷數十人,瓚左右亦亡其半,遂得免。

中平中,以瓚督烏桓突騎,車騎將軍張溫討涼州賊。會烏桓反畔,與賊張純等攻擊薊中,瓚率所領追討純等有功,遷騎都尉。張純復與畔胡丘力居等寇漁陽、河閒、勃海,入平原,多所殺略。瓚追擊戰於屬國石門,虜遂大敗,棄妻子踰塞走,悉得其所略男女。瓚深入無繼,反為丘力居等所圍於遼西管子城,二百餘日,糧盡食馬,馬盡煮弩楯,力戰不敵,乃與士卒辭訣,各分散還。時多雨雪,隊阬死者十五六,虜亦飢困,遠走柳城。詔拜瓚降虜校尉,封都亭侯,復兼領屬國長史。職統戎馬,連接邊寇。每聞有警,瓚輒厲色憤怒,如赴讎敵,望塵奔逐,或繼之以夜戰。虜識瓚聲,憚其勇,莫敢抗犯。

瓚常與善射之士數十人,皆乘白馬,以為左右翼,自號「白馬義從」。烏桓更相告語,避白馬長史。乃畫作瓚形,馳騎射之,中者咸稱萬歲。虜自此之後,遂遠竄塞外。

瓚志埽滅烏桓,而劉虞欲以恩信招降,由是與虞相忤。初平二年,青、徐黃巾三十萬眾入勃海界,欲與黑山合。瓚率步騎二萬人,逆擊於東光南,大破之,斬首三萬餘級。賊棄其車重數萬兩,奔走度河。瓚因其半濟薄之,賊復大破,死者數萬,流血丹水,收得生口七萬餘人,車甲財物不可勝筭,威名大震。拜奮武將軍,封薊侯。

瓚既諫劉虞遣兵就袁術,而懼術知怨之,乃使從弟越將千餘騎詣術自結。術遣越隨其將孫堅,擊袁紹將周昕,越為流矢所中死。瓚因此怒紹,遂出軍屯槃河,將以報紹。乃上疏曰:「臣聞皇羲已來,君臣道著,張禮以導人,設刑以禁暴。今車騎將軍袁紹,託承先軌,爵任崇厚,而性本淫亂,情行浮薄。昔為司隸,值國多難,太后承攝,何氏輔朝。紹不能舉直措枉,而專為邪媚,招來不軌,疑誤社稷,至令丁原焚燒孟津,董卓造為亂始。紹罪一也。卓既無禮,帝主見質。紹不能開設權謀,以濟君父,而棄置節傳,迸竄逃亡。忝辱爵命,背違人主,紹罪二也。紹為勃海,當攻董卓,而默選戎馬,不告父兄,至使太傅一門,纍然同斃。不仁不孝,紹罪三也。紹既興兵,涉歷二載,不恤國難,廣自封植。乃多引資糧,專為不急,割刻無方,考責百姓,其為痛怨,莫不咨嗟。紹罪四也。逼迫韓馥,竊奪其州,矯刻金玉,以為印璽,每有所下,輒皁囊施檢,文稱詔書。昔亡新僭侈,漸以即真。觀紹所擬,將必階亂。紹罪五也。紹令星工伺望祥妖,賂遺財貨,與共飲食,剋會期日,攻鈔郡縣。此豈大臣所當施為?紹罪六也。紹與故虎牙都尉劉勳,首共造兵,勳降服張楊,累有功效,而以小忿枉加酷害。信用讒慝,濟其無道,紹罪七也。故上谷太守高焉,故甘陵相姚貢,紹以貪惏,橫責其錢,錢不備畢,二人并命。紹罪八也。春秋之義,子以母貴。紹母親為傅婢,地實微賤,據職高重,享福豐隆。有苟進之志,無虛退之心,紹罪九也。又長沙太守孫堅,前領豫州刺史,遂能驅走董卓,埽除陵廟,忠勤王室,其功莫大。紹遣小將盜居其位,斷絕堅糧,不得深入,使董卓久不服誅。紹罪十也。昔姬周政弱,王道陵遲,天子遷徙,諸侯背畔,故齊桓立柯會之盟,晉文為踐土之會,伐荊楚以致菁茅,誅曹、衛以章無禮。臣雖闒茸,名非先賢,蒙被朝恩,負荷重任,職在鈇鉞,奉辭伐罪,輒與諸將州郡共討紹等。若大事克捷,罪人斯得,庶續桓文忠誠之效。」遂舉兵攻紹,於是冀州諸城悉畔從瓚。

紹懼,乃以所佩勃海太守印綬授瓚從弟範,遣之郡,欲以相結。而範遂背紹,領勃海兵以助瓚。瓚乃自署其將帥為青、冀、兗三州刺史,又悉置郡縣守令,與紹大戰於界橋。瓚軍敗還薊。紹遣將崔巨業將兵數萬攻圍故安不下,退軍南還。瓚將步騎三萬人追擊於巨馬水,大破其眾,死者七八千。乘勝而南,攻下郡縣,遂至平原,乃遣其青州刺史田揩據有齊地。紹復遣兵數萬與揩連戰二年,糧食並盡,士卒疲困,互掠百姓,野無青草。紹乃遣子譚為青州刺史,揩與戰,敗退還。

是歲,瓚破禽劉虞,盡有幽州之地,猛志益盛。前此有童謠曰:「燕南垂,趙北際,中央不合大如礪,唯有此中可避世。」瓚自以為易地當之,遂徙鎮焉。乃盛修營壘,樓觀數十,臨易河,通遼海。

劉虞從事漁陽鮮于輔等,合率州兵,欲共報瓚。輔以燕國閻柔素有恩信,推為烏桓司馬。柔招誘胡漢數萬人,與瓚所置漁陽太守鄒丹戰于潞北,斬丹等四千餘級。烏桓峭王感虞恩德,率種人及鮮卑七千餘騎,共輔南迎虞子和,與袁紹將麴義合兵十萬,共攻瓚。興平二年,破瓚於鮑丘,斬首二萬餘級。瓚遂保易京,開置屯田,稍得自支。相持歲餘,麴義軍糧盡,士卒飢困,餘眾數千人退走。瓚徼破之,盡得其車重。

是時旱蝗穀貴,民相食。瓚恃其才力,不恤百姓,記過忘善,睚眥必報,州里善士名在其右者,必以法害之。常言「衣冠皆自以職分富貴,不謝人惠」。故所寵愛,類多商販庸兒。所在侵暴,百姓怨之。於是代郡、廣陽、上谷、右北平各殺瓚所置長吏,復與輔、和兵合。瓚慮有非常,乃居於高京,以鐵為門。斥去左右,男人七歲以上不得入易門。專侍姬妾,其文簿書記皆汲而上之。令婦人習為大言聲,使聞數百步,以傳宣教令。疏遠賓客,無所親信,故謀臣猛將,稍有乖散。自此之後,希復攻戰。或問其故。瓚曰:「我昔驅畔胡於塞表,埽黃巾於孟津,當此之時,謂天下指麾可定。至於今日,兵革方始,觀此非我所決,不如休兵力耕,以救凶年。兵法百樓不攻。今吾諸營樓馆千里,積穀三百萬斛,食此足以待天下之變。」

建安三年,袁紹復大攻瓚。瓚遣子續請救於黑山諸帥,而欲自將突騎直出,傍西山以斷紹後。長史關靖諫曰:「今將軍將士,莫不懷瓦解之心,所以猶能相守者,顧戀其老小,而恃將軍為主故耳。堅守曠日,或可使紹自退。若舍之而出,後無鎮重,易京之危,可立待也。」瓚乃止。紹漸相攻逼,瓚眾日蹙,乃卻,築三重營以自固。

四年春,黑山賊帥張燕與續率兵十萬,三道來救瓚。未及至,瓚乃密使行人齎書告續曰:「昔周末喪亂,僵屍蔽地,以意而推,猶為否也。不圖今日親當其鋒。袁氏之攻,狀若鬼神,梯衝舞吾樓上,鼓角鳴於地中,日窮月急,不遑啟處。鳥厄歸人,滀水陵高,汝當碎首於張燕,馳驟以告急。父子天性,不言而動。且厲五千鐵騎於北隰之中,起火為應,吾當自內出,奮揚威武,決命於斯。不然,吾亡之後,天下雖廣,不容汝足矣。」紹候得其書,如期舉火,瓚以為救至,遂便出戰。紹設伏,瓚遂大敗,復還保中小城。自計必無全,乃悉縊其姊妹妻子,然後引火自焚。紹兵趣登臺斬之。

關靖見瓚敗,歎恨曰:「前若不止將軍自行,未必不濟。吾聞君子陷人於危,必同其難,豈可以獨生乎!」乃策馬赴紹軍而死。續為屠各所殺。田揩與袁紹戰死。

鮮于輔將其眾歸曹操,操以輔為度遼將軍,封都亭侯。閻柔將部曲曹操擊烏桓,拜護烏桓校尉,封關內侯。

張燕既為紹所敗,人眾稍散。曹操將定冀州,乃率眾詣鄴降,拜平北將軍,封安國亭侯。

論曰:自帝室王公之冑,皆生長脂腴,不知稼穡,其能厲行飭身,卓然不群者,或未聞焉。劉虞守道慕名,以忠厚自牧。美哉乎,季漢之名宗子也!若虞瓚無閒,同情共力,糾人完聚,蓄保燕、薊之饒,繕兵昭武,以臨群雄之隙,舍諸天運,徵乎人文,則古之休烈,何遠之有!

陶謙字恭祖,丹陽人也。少為諸生,仕州郡,四遷為車騎將軍張溫司馬,西討邊章。會徐州黃巾起,以謙為徐州刺史,擊黃巾,大破走之,境內晏然。

時董卓雖誅,而李傕、郭汜作亂關中。是時四方斷絕,謙每遣使閒行,奉貢西京。詔遷為徐州牧,加安東將軍,封溧陽侯。是時徐方百姓殷盛,穀實甚豐,流民多歸之。而謙信用非所,刑政不理。別駕從事趙昱,知名士也,而以忠直見疏,出為廣陵太守。曹宏等讒慝小人,謙甚親任之,良善多被其害。由斯漸亂。下邳閻宣自稱「天子」,謙始與合從,後遂殺之而并其眾。

初,曹操父嵩避難琅邪,時謙別將守陰平,士卒利嵩財寶,遂襲殺之。初平四年,曹操擊謙,破彭城傅陽。謙退保郯,操攻之不能克,乃還。過拔取慮、雎陵、夏丘,皆屠之。凡殺男女數十萬人,雞犬無餘,泗水為之不流,自是五縣城保,無復行跡。初三輔遭李傕亂,百姓流移依謙者皆殲。

興平元年,曹操復擊謙,略定琅邪、東海諸縣,謙懼不免,欲走歸丹陽。會張邈迎呂布據兗州,操還擊布。是歲,謙病死。

初,同郡人笮融,聚眾數百,往依於謙,謙使督廣陵、下邳、彭城運糧。遂斷三郡委輪,大起浮屠寺。上累金盤,下為重樓,又堂閣周回,可容三千許人,作黃金塗像,衣以錦綵。每浴佛,輒多設飲飯,布席於路,其有就食及觀者且萬餘人。及曹操擊謙,徐方不安,融乃將男女萬口、馬三千匹走廣陵。廣陵太守趙昱待以賓禮。融利廣陵資貨,遂乘酒酣殺昱,放兵大掠,因以過江,南奔豫章,殺郡守朱皓,入據其城。後為楊州刺史劉繇所破,走入山中,為人所殺。

昱字元達,琅邪人。清己疾惡,潛志好學,雖親友希得見之。為人耳不邪聽,目不妄視。太僕种拂舉為方正。

贊曰:襄賁勵德,維城燕北。仁能洽下,忠以衛國。伯珪疏獷,武才趫猛。虞好無終,紹埶難並。徐方殲耗,實謙為梗。


\end{pinyinscope}