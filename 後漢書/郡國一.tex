\article{郡國一}

\begin{pinyinscope}
漢書地理志記天下郡縣本末,及山川奇異,風俗所由,至矣。今但錄中興以來郡縣改異,及春秋、三史會同征伐地名,以為郡國志。凡前志有縣名,今所不載者,皆世祖所并省也。前無今有者,後所置也。凡縣名先書者,郡所治也。

河南尹

二十一城,永和五年戶二十萬八千四百八十六,口百一萬八百二十七。

雒陽

周時號成周。有狄泉,在城中。有唐聚。有上程聚。有士鄉聚。有褚氏聚。有榮錡澗。有前亭。有圉鄉。有大解城。河南周公時所城雒邑也,春秋時謂之王城。東城門名鼎門,北城門名乾祭。又有甘城,有蒯鄉。梁故國,伯翳後。有霍陽山。有注城。熒陽有鴻溝水。有廣武城。有总亭,总叔國。有隴城。有薄亭。有敖亭。有費澤。卷

有長城,經陽武到密。有垣雝城,或曰古衡雍。有扈城亭。原武陽武中牟

有圃田澤。有清口水。有管城。有曲遇聚。有蔡亭。開封菀陵有棐林。有制澤。有瑣侯亭。平陰穀城瀍水出。有函谷關。緱氏有鄔聚。有轘轅關。鞏

有尋谷水。有東訾聚,今名訾城。有坎埳聚。有黃亭。有湟水。有明谿泉。成睪有旃然水。有瓶丘聚。有漫水。有汜水。京密

有大騩山。有梅山。有陘山。新城

有高都城。有廣成聚。有鄤聚,古鄤氏,今名蠻中。匽師

有尸鄉,春秋時曰尸氏。新鄭詩鄭國,祝融墟。平

河內郡

十八城,戶十五萬九千七百七十,口八十萬一千五百五十八。

懷

有隰城。河陽有湛城。軹有原鄉。有湨梁。波有絺城。沁水野王有太行山。有射犬聚。有邘城。溫蘇子所都。濟水出,王莽時大旱,遂枯絕。州平睾有邢丘,故邢國,周公子所封。有李城。山陽邑。有雍城。有蔡城。武德獲嘉侯國。脩武故南陽,秦始皇更名。有南陽城,陽樊、攢茅田。有小脩武聚。有隤城。共本國。淇水出。有汎亭。汲

朝歌

紂所都居

,南有牧野,北有邶國,南有寧鄉。蕩陰有羑里城。林慮故隆慮,殤帝改。有鐵。

河東郡

二十城,戶九萬三千五百四十三,口五十七萬八百三。

安邑

。有鐵,有鹽池。楊有高梁亭。平陽侯國。有鐵。堯都此。臨汾有董亭。汾陰

有介山。蒲阪有雷首山。有沙丘亭。大陽有吳山,上有虞城,有下陽城,有茅津。有顛軨阪。解

有桑泉城。有臼城。有解城。有瑕城。皮氏有耿鄉。有鐵。有冀亭。聞喜邑,本曲沃。有董池陂,古董澤。有稷山亭。有涑水。有洮水。絳邑。有翼城。永安故彘,陽嘉二年更名。有霍大山。河北詩魏國。有韓亭。猗氏垣

有王屋山,兗水出。有壺丘亭。有邵亭。襄陵北屈有壺口山。有采桑津。蒲子濩澤侯國。有祁城山。端氏

弘農郡

九城,戶四萬六千八百一十五,口十九萬九千一百一十三。

弘農

故秦函谷關,燭水出。有枯樅山。有桃丘聚,故桃林。有務鄉。有曹陽亭。陝

本总仲國。有焦城。有陝陌。黽池穀水出。有二崤。新安澗水出。宜陽陸渾西有总略地。盧氏有熊耳山,伊水、清水出。湖故屬京兆。有閺鄉。華陰故屬京兆。有太華山。

京兆尹

十城,戶五萬三千二百九十九,口二十八萬五千五百七十四。

長安

高帝所都。鎬在上林菀中。有細柳聚。有蘭池。有曲郵。有杜郵。霸陵有枳道亭。有長門亭。杜陵

酆在西南。鄭新豐有驪山,東有鴻門亭及戲亭。有嚴城。藍田出美玉。長陵故屬馮翊。商故屬弘農。上雒侯國。有冢領山,雒水出。故屬弘農。有菟和山。有蒼野聚。陽陵故屬馮翊。

左馮翊

十三城

,戶三萬七千九十,口十四萬五千一百九十五。

高陵

池陽

雲陽

祋

祤

永元九年復。頻陽萬年蓮勺重泉

臨晉

本大荔。有河水祠。有芮鄉。有王城。郃陽永平二年復。夏陽有梁山、龍門山。衙粟邑永元九年復。

右扶風

十五城,戶萬七千三百五十二,口九萬三千九十一。

槐里

周曰犬丘,高帝改。安陵平陵

茂陵

鄠

豐

水出。有甘亭。郿有邰亭。武功永平八年復。有太一山,本終南。垂山,本敦物。有斜谷。陳倉汧

有吳嶽山,本名汧,汧水出。有回城,名回中。渝麋侯國。雍

有鐵。栒邑有豳鄉。美陽有岐山,有周城。漆有漆水。有鐵。杜陽永和二年復。

右司隸校尉部,郡七,縣、邑、侯國百六。


\end{pinyinscope}