\article{李陳龐陳橋列傳}

\begin{pinyinscope}
李恂字叔英,安定臨涇人也。少習韓詩,教授諸生常數百人。太守潁川李鴻請署功曹,未及到,而州辟為從事。會鴻卒,恂不應州命,而送鴻喪還鄉里。既葬,留起冢墳,持喪三年。

辟司徒桓虞府。後拜侍御史,持節使幽州,宣布恩澤,慰撫北狄,所過皆圖寫山川、屯田、聚落百餘卷,悉封奏上,肅宗嘉之。拜兗州刺史。以清約率下,常席羊皮,服布被。遷張掖太守,有威重名。時大將軍竇憲將兵屯武威,天下州郡遠近莫不修禮遺,恂奉公不阿,為憲所奏免。

後復徵拜謁者,使持節領西域副校尉。西域殷富,多珍寶,諸國侍子及督使賈胡數遺恂奴婢、宛馬、金銀、香罽之屬,一無所受。北匈奴數斷西域車師、伊吾,隴沙以西使命不得通,恂設購賞,遂斬虜帥,縣首軍門。自是道路夷清,威恩並行。

遷武威太守。後坐事免,步歸鄉里,潛居山澤,結草為廬,獨與諸生織席自給。會西羌反畔,恂到田舍,為所執獲。羌素聞其名,放遣之。恂因詣洛陽謝。時歲荒,司空張敏、司徒魯恭等各遣子饋糧,悉無所受。徙居新安關下,拾橡實以自資。年九十六卒。

陳禪字紀山,巴郡安漢人也。仕郡功曹,舉善黜惡,為邦內所畏。察孝廉,州辟治中從事。時刺史為人所上受納臧賂,禪當傳考,無它所齎,但持喪斂之具而已。及至,笞掠無筭,五毒畢加,禪神意自若,辭對無變,事遂散釋。車騎將軍鄧騭聞其名而辟焉,舉茂才。時漢中蠻夷反畔,以禪為漢中太守。夷賊素聞其聲,即時降服。遷左馮翊,入拜諫議大夫。

永寧元年,西南夷撣國王獻樂及幻人,能吐火,自支解,易牛馬頭。明年元會,作之於庭,安帝與群臣共觀,大奇之。禪獨離席舉手大言曰:「昔齊魯為夾谷之會,齊作侏儒之樂,仲尼誅之。又曰:『放鄭聲,遠佞人。』帝王之庭,不宜設夷狄之技。」尚書陳忠劾奏禪曰:「古者合歡之樂舞於堂,四夷之樂陳於門,故《詩》云『以雅以南,关任朱離』。今撣國越流沙,踰縣度,萬里貢獻,非鄭衛之聲,佞人之比,而禪廷訕朝政,請劾禪下獄。」有詔勿收,左轉為玄菟候城障尉,詔「敢不之官,上妻子從者名」。禪既行,朝廷多訟之。會北匈奴入遼東,追拜禪遼東太守。胡憚其威彊,退還數百里。禪不加兵,但使吏卒往曉慰之,單于隨使還郡。禪於學行禮,為說道義以感化之。單于懷服,遺以胡中珍貨而去。

及鄧騭誅廢,禪以故吏免。復為車騎將軍閻顯長史。順帝即位,遷司隸校尉。明年,卒於官。

子澄,有清名,官至漢中太守。

禪曾孫寶,亦剛壯有禪風,為州別駕從事,顯名州里。

龐參字仲達,河南緱氏人也。初仕郡,未知名,河南尹龐奮見而奇之,舉為孝廉,拜左校令。坐法輸作若盧。

永初元年,涼州先零種羌反畔,遣車騎將軍鄧騭討之。參於徒中使其子俊上書曰:「方今西州流民擾動,而徵發不絕,水潦不休,地力不復。重之以大軍,疲之以遠戍,農功消於轉運,資財竭於徵發。田疇不得墾闢,禾稼不得收入,搏手困窮,無望來秋。百姓力屈,不復堪命。臣愚以為萬里運糧,遠就羌戎,不若總兵養眾,以待其疲。車騎將軍騭宜且振旅,留征西校尉任尚使督涼州士民,轉居三輔。休徭役以助其時,止煩賦以益其財,令男得耕種,女得織紝,然後畜精銳,乘懈沮,出其不意。攻其不備,則邊人之仇報,奔北之恥雪矣。」書奏,會御史中丞樊準上疏薦參曰:「臣聞鷙鳥累百,不如一鶚。昔孝文皇帝悟馮唐之言,而赦魏尚之罪,使為邊守,匈奴不敢南向。夫以一臣之身,折方面之難者,選用得也。臣伏見故左校令河南龐參,勇謀不測,卓爾奇偉,高才武略,有魏尚之風。前坐微法,輸作經時。今羌戎為患,大軍西屯,臣以為如參之人,宜在行伍。惟明詔採前世之舉,觀魏尚之功,免赦參刑,以為軍鋒,必有成效,宣助國威。」鄧太后納其言,即擢參於徒中,召拜謁者,使西督三輔諸軍屯,而徵鄧騭還。

四年,羌寇轉盛,兵費日廣,且連年不登,穀石萬餘。參奏記於鄧騭曰:「比年羌寇特困隴右,供徭賦役為損日滋,官負人責數十億萬。今復募發百姓,調取穀帛,衒賣什物,以應吏求。外傷羌虜,內困徵賦。遂乃千里轉糧,遠給武都西郡。塗路傾阻,難勞百端,疾行則鈔暴為害,遲進則穀食稍損,運糧散於曠野,牛馬死於山澤。縣官不足,輒貸於民。民已窮矣,將從誰求?名救金城,而實困三輔。三輔既困,還復為金城之禍矣。參前數言宜棄西域,乃為西州士大夫所笑。今苟貪不毛之地,營恤不使之民,暴軍伊吾之野,以慮三族之外,果破涼州,禍亂至今。夫拓境不寧,無益於彊;多田不耕,何救飢敝!故善為國者,務懷其內,不求外利;務富其民,不貪廣土。三輔山原曠遠,民庶稀疏,故縣丘城,可居者多。今宜徙邊郡不能自存者,入居諸陵,田戍故縣。孤城絕郡,以權徙之;轉運遠費,聚而近之;徭役煩數,休而息之。此善之善者也。」騭及公卿以國用不足,欲從參議,眾多不同,乃止。

拜參為漢陽太守。郡人任棠者,有奇節,隱居教授。參到,先候之。棠不與言,但以薤一大本,水一盂,置戶屏前,自抱孫兒伏於戶下。主簿白以為倨。參思其微意,良久曰:「棠是欲曉太守也。水者,欲吾清也。拔大本薤者,欲吾擊強宗也。抱兒當戶,欲吾開門恤孤也。」於是歎息而還。參在職,果能抑強助弱,以惠政得民。

元初元年,遷護羌校尉,畔羌懷其恩信。明年,燒當羌種號多等皆降,始復得還都令居,通河西路。時先零羌豪僭號北地,詔參將降羌及湟中義從胡七千人,與行征西將軍司馬鈞期會北地擊之。參於道為羌所敗。既已失期,乃稱病引兵還,坐以詐疾徵下獄,校書郎中馬融上書請之曰:「伏見西戎反畔,寇鈔五州,陛下愍百姓之傷痍,哀黎元之失業,單竭府庫以奉軍師。昔周宣獫狁侵鎬及方,孝文匈奴亦略上郡,而宣王立中興之功,文帝建太宗之號。非惟兩主有明叡之姿,抑亦扞城有虓虎之助,是以南仲赫赫,列在周詩,亞夫赳赳,載於漢策。竊見前護羌校尉龐參,文武昭備,智略弘遠,既有義勇果毅之節,兼以博雅深謀之姿。又度遼將軍梁慬,前統西域,勤苦數年,還留三輔,功效克立,閒在北邊,單于降服。今皆幽囚,陷於法網。昔荀林父敗績於邲,晉侯使復其位;孟明視喪師於崤,秦伯不替其官。故晉景并赤狄之土,秦穆遂霸西戎。宜遠覽二君,使參、慬得在寬宥之科,誠有益於折衝,毗佐於聖化。」書奏,赦參等。

後以參為遼東太守。永建元年,遷度遼將軍。四年,入為大鴻臚。尚書僕射虞詡薦參有宰相器能,順帝時以為太尉,錄尚書事。是時三公之中,參名忠直,數為左右所陷毀,以所舉用忤帝旨,司隸承風案之。時當會茂才孝廉,參以被奏,稱疾不得會。上計掾廣漢段恭因會上疏曰:「伏見道路行人,農夫織婦,皆曰『太尉龐參,竭忠盡節,徒以直道不能曲心,孤立群邪之閒,自處中傷之地』。臣猶冀在陛下之世,當蒙安全,而復以讒佞傷毀忠正,此天地之大禁,人主之至誡。昔白起賜死,諸侯酌酒相賀;季子來歸,魯人喜其紓難。夫國以賢化,君以忠安。今天下咸欣陛下有此忠賢,願卒寵任,以安社稷。」書奏,詔即遣小黃門視參疾,太醫致羊酒。

後參夫人疾前妻子,投於井而殺之。參素與洛陽令祝良不平,良聞之,率吏卒入太尉府案實其事,乃上參罪,遂因災異策免。有司以良不先聞奏,輒折辱宰相,坐繫詔獄。良能得百姓心,洛陽吏人守闕請代其罪者,日有數千萬人,詔乃原刑。

陽嘉四年,復以參為太尉。永和元年,以久病罷,卒於家。

陳龜字叔珍,上黨泫氏人也。家世邊將,便習弓馬,雄於北州。

龜少有志氣。永建中,舉孝廉,五遷五原太守。永和五年,拜使匈奴中郎將。時南匈奴左部反亂,龜以單于不能制下,外順內畔,促令自殺,坐徵下獄免。後再遷,拜京兆尹。時三輔強豪之族,多侵枉小民。龜到,厲威嚴,悉平理其怨屈者,郡內大悅。

會羌胡寇邊,殺長吏,驅略百姓。桓帝以龜世諳邊俗,拜為度遼將軍。龜臨行,上疏曰:「臣龜蒙恩累世,馳騁邊垂,雖展鷹犬之用,頓斃胡虜之庭,魂骸不返,薦享狐狸,猶無以塞厚責,荅萬分也。至臣頑駑,器無鈆刀一割之用,過受國恩,榮秩兼優,生年死日,永懼不報。臣聞三辰不軌,擢士為相;蠻夷不恭,拔卒為將。臣無文武之才,而忝鷹揚之任,上慚聖朝,下懼素餐,雖歿軀體,無所云補。今西州邊鄙,土地塉埆,鞍馬為居,射獵為業,男寡耕稼之利,女乏機杼之饒,守塞候望,懸命鋒鏑,聞急長驅,去不圖反。自頃年以來,匈奴數攻營郡,殘殺長吏,侮略良細。戰夫身膏沙漠,居人首係馬鞍。或舉國掩戶,盡種灰滅,孤兒寡婦,號哭空城,野無青草,室如懸磬。雖含生氣,實同枯朽。往歲并州水雨,災螟互生,稼穡荒耗,租更空闕。老者慮不終年,少壯懼於困厄。陛下以百姓為子,品庶以陛下為父,焉可不日昃勞神,垂撫循之恩哉!唐堯親捨其子以禪虞舜者,是欲民遭聖君,不令遇惡主也。故古公杖策,其民五倍;文王西伯,天下歸之。豈復輿金輦寶,以為民惠乎!近孝文皇帝感一女子之言,除肉刑之法,體德行仁,為漢賢主。陛下繼中興之統,承光武之業,臨朝聽政,而未留聖意。且牧守不良,或出中官,懼逆上旨,取過目前。呼嗟之聲,招致災害,胡虜凶悍,因衰緣隙。而令倉庫單於豺狼之口,功業無銖兩之效,皆由將帥不忠,聚姦所致。前涼州刺史祝良,初除到州,多所糾罰,太守令長,貶黜將半,政未踰時,功效卓然。實應賞異,以勸功能,改任牧守,去斥姦殘。又宜更選匈奴烏桓護羌中郎將校尉,簡練文武,授之法令,除并涼二州今年租更,寬赦罪隸,埽除更始。則善吏知奉公之祐,惡者覺營私之禍,胡馬可不窺長城,塞下無候望之患矣。」帝覺悟,乃更選幽、井刺史,自營郡太守都尉以下,多所革易,下詔「為陳將軍除并、涼一年租賦,以賜吏民」。龜既到職,州郡重足震慄,鮮卑不敢近塞,省息經用,歲以億計。

大將軍梁冀與龜素有隙,譖其沮毀國威,挑取功譽,不為胡虜所畏。坐徵還,遂乞骸骨歸田里。復徵為尚書。冀暴虐日甚,龜上疏言其罪狀,請誅之。帝不省。自知必為冀所害,不食七日而死。西域胡夷,并、涼民庶,咸為舉哀,弔祭其墓。

橋玄字公祖,梁國睢陽人也。七世祖仁,從同郡戴德學,著禮記章句四十九篇,號曰「橋君學」。成帝時為大鴻臚。祖父基,廣陵太守。父肅,東萊太守。

玄少為縣功曹。時豫州刺史周景行部到梁國,玄謁景,因伏地言陳相羊昌罪惡,乞為部陳從事,窮案其姦。景壯玄意,署而遣之。玄到,悉收昌賓客,具考臧罪。昌素為大將軍梁冀所厚,冀為馳檄救之。景承旨召玄,玄還檄不發,案之益急。昌坐檻車徵,玄由是著名。

舉孝廉,補洛陽左尉。時梁不疑為河南尹,玄以公事當詣府受對,恥為所辱,棄官還鄉里。後四遷為齊相,坐事為城旦。刑竟,徵,再遷上谷太守,又為漢陽太守。時上邽令皇甫禎有臧罪,玄收考髡笞,死于冀巿,一境皆震。郡人上邽姜岐,守道隱居,名聞西州。玄召以為吏,稱疾不就。玄怒,敕督郵尹益逼致之,曰:「岐若不至,趣嫁其母。」益固爭不能得,遽曉譬岐。岐堅臥不起。郡內士大夫亦競往諫,玄乃止。時頗以為譏。後謝病免,復公車徵為司徒長史,拜將作大匠。

桓帝末,鮮卑、南匈奴及高句驪嗣子伯固並畔,為寇鈔,四府舉玄為度遼將軍,假黃鉞。玄至鎮,休兵養士,然後督諸將守討擊胡虜及伯固等,皆破散退走。在職三年,邊境安靜。

靈帝初,徵入為河南尹,轉少府、大鴻臚。建寧三年,遷司空,轉司徒。素與南陽太守陳球有隙,及在公位,而薦球為廷尉。玄以國家方弱,自度力無所用,乃稱疾上疏,引眾災以自劾。遂策罷。歲餘,拜尚書令。時太中大夫蓋升與帝有舊恩,前為南陽太守,臧數億以上。玄奏免升禁錮,沒入財賄。帝不從,而遷升侍中。玄託病免,拜光祿大夫。光和元年,遷太尉。數月,復以疾罷,拜太中大夫,就醫里舍。

玄少子十歲,獨游門次,卒有三人持杖劫執之,入舍登樓,就玄求貨,玄不與。有頃,司隸校尉陽球率河南尹、洛陽令圍守玄家。球等恐并殺其子,未欲迫之。玄瞋目呼曰:「姦人無狀,玄豈以一子之命而縱國賊乎!」促令兵進。於是攻之,玄子亦死。玄乃詣闕謝罪,乞下天下:「凡有劫質,皆并殺之,不得贖以財寶,開張姦路。」詔書下其章。初自安帝以後,法禁稍弛,京師劫質,不避豪貴,自是遂絕。

玄以光和六年卒,時年七十五。玄性剛急無大體,然謙儉下士,子弟親宗無在大官者。及卒,家無居業,喪無所殯,當時稱之。

初,曹操微時,人莫知者,嘗往候玄,玄見而異焉,謂曰:「今天下將亂,安生民,者其在君乎!」操常感其知己。及後經過玄墓,輒悽愴致祭。自為其文曰:「故太尉橋公,懿德高軌,汎愛博容。國念明訓,士思令謨。幽靈潛翳,横哉緬矣!操以幼年,逮升堂室,特以頑質,見納君子。增榮益觀,皆由獎助,猶仲尼稱不如顏淵,李生厚歎賈復。士死知己,懷此無忘。又承從容約誓之言:『徂沒之後,路有經由,不以斗酒隻雞過相沃酹,車過三步,腹痛勿怨。』雖臨時戲笑之言,非至親之篤好,胡肯為此辭哉?懷舊惟顧,念之悽愴。奉命東征,屯次鄉里,北望貴土,乃心陵墓。裁致薄奠,公其享之!」

玄子羽,官至任城相。

論曰:任棠、姜岐,世著其清。結甕牖而辭三命,殆漢陽之幽人乎?龐參躬求賢之禮,故民悅其政;橋玄厲邦君之威,而眾失其情。夫豈力不足歟?將有道在焉。如令其道可忘,則彊梁勝矣。語曰:「三軍可奪帥,匹夫不可奪志。」子貢曰:「寧喪千金,不失士心。」昔段干木踰牆而避文侯之命,泄柳閉門不納穆公之請。貴必有所屈,賤亦有所申矣。

贊曰:李叟勤身,甘飢辭饋。禪為君隱,之死靡貳。龜習邊功,參起徒中。橋公識運,先覺時雄。


\end{pinyinscope}