\article{天文上}

\begin{pinyinscope}
王莽三光武十二

《易》曰:「天垂象,聖人則之。庖犧氏之王天下,仰則觀象於天,俯則觀法於地。」觀象於天,謂日月星辰。觀法於地,謂水土州分。形成於下,象見于上。故曰天者北辰星,合元垂燿建帝形,運機授度張百精。三階九列,二十七大夫,八十一元士,斗、衡、太微、攝提之屬百二十官,二十八宿各布列,下應十二子。天地設位,星辰之象備矣。

三皇邁化,協神醇朴,謂五星如連珠,日月若合璧。化由自然,民不犯慝。至於書契之興,五帝是作。軒轅始受河圖鬥苞授,規日月星辰之象,故星官之書自黃帝始。至高陽氏,使南正重司天,北正黎司地。唐、虞之時羲仲、和仲,夏有昆吾,湯則巫咸,周之史佚、萇弘,宋之子韋,楚之唐蔑,魯之梓慎,鄭之裨灶,魏石申夫,齊國甘公,皆掌天文之官。仰占俯視,以佐時政,步變擿微,通洞密至,採禍福之原,睹成敗之勢。秦燔詩書,以愚百姓,六經典籍,殘為灰炭,星官之書,全而不毀。故秦史書始皇之時,彗孛大角,大角以亡,有大星與小星鬥于宮中,是其廢亡之徵。至漢興,景、武之際,司馬談,談子遷,以世黎氏之後,為太史令,遷著史記,作天官書。成帝時,中壘校尉劉向,廣洪範災條作五紀皇極之論,以參往行之事。孝明帝使班固敘漢書,而馬續述天文志。今紹漢書作天文志,起王莽居攝元年,迄孝獻帝建安二十五年,二百一十五載。言其時星辰之變,表象之應,以顯天戒,明王事焉。

王莽地皇三年十一月,有星孛于張,東南行五日不見。孛星者,惡氣所生,為亂兵,其所以孛德。孛德者,亂之象,不明之表。又參然孛焉,兵之類也,故名之曰孛。孛之為言,猶有所傷害,有所妨蔽。或謂之彗星,所以除穢而布新也。張為周地。星孛于張,東南行即翼、軫之分。翼、軫為楚,是周、楚地將有兵亂。後一年正月,光武起兵舂陵,會下江、新巿賊張卬、王常及更始之兵亦至,俱攻破南陽,斬莽前隊大夫甄阜、屬正梁丘賜等,殺其士眾數萬人。更始為天子,都雒陽,西入長安,敗死。光武興於河北,復都雒陽,居周地,除穢布新之象。

四年六月,漢兵起南陽,至昆陽。莽使司徒王尋、司空王邑將諸郡兵,號曰百萬眾,已至者四十二萬人;能通兵法者六十三家,皆為將帥,持其圖書器械。軍出關東,牽從群象虎狼猛獸,放之道路,以示富強,用怖山東。至昆陽山,作營百餘,圍城數重,或為衝車以撞城,為雲車高十丈以瞰城中,弩矢雨集,城中負戶而汲。求降不聽,請出不得。二公之兵自以必克,不恤軍事,不協計慮。莽有覆敗之變見焉。晝有雲氣如壞山,墮軍上,軍人皆厭,所謂營頭之星也。占曰:「營頭之所墮,其下覆軍,流血三千里。」是時光武將兵數千人赴救昆陽,奔擊二公兵,并力猋發,號呼聲動天地,虎豹驚怖敗振。會天大風,飛屋瓦,雨如注水。二公兵亂敗,自相賊,就死者數萬人。競赴滍水,死者委積,滍水為之不流。殺司徒王尋。軍皆散走歸本郡。王邑還長安,莽敗,俱誅死。營頭之變,覆軍流血之應也。

四年秋,太白在太微中,燭地如月光。太白為兵,太微為天廷。太白贏而北入太微,是大兵將入天子廷也。是時莽遣二公之兵至昆陽,已為光武所破。莽又拜九人為將軍,皆以虎為號。九虎將軍至華陰,皆為漢將鄧曄、李松所破。進攻京師,倉將軍韓臣至長門。十月戊申,漢兵自宣平城門入。二日己酉,城中少年朱弟、張魚等數千人起兵攻莽,燒作室,斧敬法闥。商人杜吳殺莽漸臺之上,校尉公賓就斬莽首。大兵蹈藉宮廷之中。仍以更始入長安,赤眉賊立劉盆子為天子,皆以大兵入宮廷,是其應也。

光武建武九年七月乙丑,金犯軒轅大星。十一月乙丑,金又犯軒轅。軒轅者,後宮之官,大星為皇后,金犯之為失勢。是時郭后已失勢見疏,後廢為中山太后,陰貴人立為皇后。

十年三月癸卯,流星如月,從太微出,入北斗魁第六星,色白。旁有小星射者十餘枚,滅則有聲如雷,食頃止。流星為貴使,星大者使大,星小者使小。太微天子廷,北斗魁主殺。星從太微出,抵北斗魁,是天子大使將出,有所伐殺。十二月己亥,大流星如缶,出柳西南行入軫。且滅時,分為十餘,如遺火狀。須臾有聲,隱隱如雷。柳為周,軫為秦、蜀。大流星出柳入軫者,是大使從周入蜀。是時光武帝使大司馬吳漢發南陽卒三萬人,乘船泝江而上,擊蜀白帝公孫述。又命將軍馬武、劉尚、郭霸、岑彭、馮駿平武都、巴郡。十二年十月,漢進兵擊述從弟衛尉永,遂至廣都,殺述女婿史興。威虜將軍馮駿拔江州,斬述將田戎。吳漢又擊述大司馬謝豐,斬首五千餘級。臧宮破涪,殺述弟大司空恢。十一月丁丑,漢護軍將軍高午刺述洞胸,其夜死。明日,漢入屠蜀城,誅述大將公孫晃、延岑等,所殺數萬人,夷滅述妻宗族萬餘人以上。是大將出伐殺之應也。其小星射者,及如遺火分為十餘,皆小將隨從之象。有聲如雷隱隱者,兵將怒之徵也。

十二月年正月己未,小星流百枚以上,或西北,或正北,或東北,二夜止。六月戊戍晨,小流星百枚以上,四面行。小星者,庶民之類。流行者,移徙之象也。或西北,或東北,或四面行,皆小民流移之徵。是時西北討公孫述,北征盧芳。匈奴助芳侵邊,漢遣將軍馬武、騎都尉劉納、閻興軍下曲陽、臨平、呼沱,以備胡。匈奴入河東,中國未安,米穀荒貴,民或流散。後三年,吳漢、馬武又徙鴈門、代郡、上谷、關西縣吏民六萬餘口,置常關、居庸關以東,以避胡寇。是小民流移之應。

十五年正月丁未,彗星見昴,稍西北行入營室,犯離宮,三月乙未,至東壁滅,見四十九日。彗星為兵入除穢,昴為邊兵,彗星出之為有兵至。十一月,定襄都尉陰承反,太守隨誅之。盧芳從匈奴入居高柳,至十六年十月降,上璽綬。一日,昴星為獄事。是時大司徒歐陽歙以事繫獄,踰歲死。營室,天子之常宮;離宮,妃后之所居。彗星入營室,犯離宮,是除宮室也。是時郭皇后已疏,至十七年十月,遂廢為中山太后,立陰貴人為皇后,除宮之象也。

三十年閏月甲午,水在東井二十度,生白氣,東南指,炎長五尺,為彗,東北行,至紫宮西藩止,五月甲子不見,凡見三十一日。水常以夏至放於東井,閏月在四月,尚未當見而見,是贏而進也。東井為水衡,水出之為大水。是歲五月及明年,郡國大水,壞城郭,傷禾稼,殺人民。白氣為喪,有炎作彗,彗所以除穢。紫宮,天子之宮,彗加其藩,除宮之象。後三年,光武帝崩。

三十一年七月戊午,火在輿鬼一度,入鬼中,出尸星南半度,十月己亥,犯軒轅大星。又七日閒有客星,炎二尺所,西南行,至明年二月二十二日,在輿鬼東北六尺所滅,凡見百一十三日。熒惑為凶衰,輿鬼尸星主死亡,熒惑入之為大喪。軒轅為後宮。七星,周地。客星居之為死喪。其後二年,光武崩。

中元二年八月丁巳,火犯太微西南角星,相去二寸。十月戊子,大流星從西南東北行,聲如雷。火犯太微西南角星,為將相。後太尉趙憙、司徒李訢坐事免官。大流星為使。中郎將竇固、揚虛侯馬武、揚鄉侯王賞將兵征西也。


\end{pinyinscope}