\article{五行五}

\begin{pinyinscope}
人化死復生疫投蜺

五行傳曰:「皇之不極,是謂不建。厥咎眊,厥罰恆陰,厥極弱。時則有射妖,時則有龍蛇之孽,時則有馬禍,時則有下人伐上之痾,時則有日月亂行,星辰逆行。」皇,君也。極,中也。眊,不明也。說云:此沴天也。不言沴天者,至尊之辭也。春秋「王師敗績」,以自敗為文。

恆陰,中興以來無錄者。

靈帝光和中,雒陽男子夜龍以弓箭射北闕,吏收考問,辭「居貧負責,無所聊生,因買弓箭以射」。近射妖也。其後車騎將軍何苗,與兄大將軍進部兵還相猜疑,對相攻擊,戰於闕下。苗死兵敗,殺數千人,雒陽宮室內人燒盡。

安帝延光三年,濟南言黃龍見歷城,琅邪言黃龍見諸。是時安帝聽讒,免太尉楊震,震自殺。又帝獨有一子,以為太子,信讒廢之。是皇不中,故有龍孽,是時多用佞媚,故以為瑞應。明年正月,東郡又言黃龍二見濮陽。

桓帝延熹七年六月壬子,河內野王山上有龍死,長可數十丈。襄楷以為夫龍者為帝王瑞,易論大人。天鳳中,黃山宮有死龍,漢兵誅莽而世祖復興,此易代之徵也。至建安二十五年,魏文帝代漢。

永康元年八月,巴郡言黃龍見。時吏傅堅以郡欲上言,內白事以為走卒戲語,不可。太守不聽。嘗見堅語云:「時民以天熱,欲就池浴,見池水濁,因戲相恐『此中有黃龍』,語遂行人閒。聞郡欲以為美,故言。」時史以書帝紀。桓帝時政治衰缺,而在所多言瑞應,皆此類也。又先儒言:瑞興非時,則為妖孽,而民訛言生龍語,皆龍孽也。

熹平元年四月甲午,青蛇見御坐上。是時靈帝委任宦者,王室微弱。

更始二年二月,發雒陽,欲入長安,司直李松奉引,車奔,觸北宮鐵柱門,三馬皆死。馬禍也。時更始失道,將亡。

桓帝延熹五年四月,驚馬與逸象突入宮殿。近馬禍也。是時桓帝政衰缺。

靈帝光和元年,司徒長史馮巡馬生人。京房易傳曰:「上亡天子,諸侯相伐,厥妖馬生人。」後馮巡遷甘陵相,黃巾初起,為所殘殺,而國家亦四面受敵。其後關東州郡各舉義兵,卒相攻伐,天子西移,王政隔塞。其占與京房同。

光和中,雒陽水西橋民馬逸走,遂齧殺人。是時公卿大臣及左右數有被誅者。

安帝永初元年十一月戊子,民轉相驚走,棄什物,去廬舍。

靈帝建寧三年春,河內婦食夫,河南夫食婦。

熹平二年六月,雒陽民訛言虎賁寺東壁中有黃人,形容鬚眉良是,觀者數萬,省內悉出,道路斷絕。到中平元年二月,張角兄弟起兵冀州,自號黃天,三十六方,四面出和,將帥星布,吏士外屬,因其疲餧,牽而勝之。

光和元年五月壬午,何人白衣欲入德陽門,辭「我梁伯夏,教我上殿為天子」。中黃門桓賢等呼門吏僕射,欲收縛何人,吏未到,須臾還走,求索不得,不知姓名。時蔡邕以成帝時男子王褒絳衣入宮,上前殿非常室,曰「天帝令我居此」,後王莽篡位。今此與成帝時相似而有異,被服不同,又未入雲龍門而覺,稱梁伯夏,皆輕於言。以往況今,將有狂狡之人,欲為王氏之謀,其事不成。其後張角稱黃天作亂,竟破壞。

二年,雒陽上西門外女子生兒,兩頭,異肩共胸,俱前向,以為不祥,墮地棄之。自此之後,朝廷霿亂,政在私門,上下無別,二頭之象。後董卓戮太后,被以不孝之名,放廢天子,後復害之。漢元以來,禍莫踰此。

四年,魏郡男子張博送鐵盧詣太官,博上書室殿山居屋後宮禁,落屋讙呼。上收縛考問,辭「忽不自覺知」。

中平元年六月壬申,雒陽男子劉倉居上西門外,妻生男,兩頭共身。

靈帝時,江夏黃氏之母,浴而化為黿,入于深淵,其後時出見。初浴簪一銀釵,及見,猶在其首。

獻帝初平中,長沙有人姓桓氏,死,棺斂月餘,其母聞棺中聲,發之,遂生。占曰:「至陰為陽,下人為上。」其後曹公由庶士起。

建安四年二月,武陵充縣女子李娥,年六十餘,物故,以其家杉木槥斂,瘞於城外數里上,已十四日,有行聞其冢中有聲,便語其家。家往視聞聲,便發出,遂活。

七年,越巂有男化為女子。時周群上言,哀帝時亦有此異,將有易代之事。至二十五年,獻帝封于山陽。

建安中,女子生男,兩頭共身。

安帝元初六年夏四月,會稽大疫。

延光四年冬,京都大疫。

桓帝元嘉元年正月,京都大疫。二月,九江、廬江又疫。

延熹四年正月,大疫。

靈帝建寧四年三月,大疫。

熹平二年正月,大疫。

光和二年春,大疫。

五年二月,大疫。

中平二年正月,大疫。

獻帝建安二十二年,大疫。

靈帝光和元年六月丁丑,有黑氣墮北宮溫明殿東庭中,黑如車蓋,起奮訊,身五色,有頭,體長十餘丈,形貌似龍。上問蔡邕,對曰:「所謂天投蜺者也。不見足尾,不得稱龍。易傳曰:『蜺之比無德,以色親也。』潛潭巴曰:『虹出,后妃陰脅王者。』又曰:『五色迭至,照于宮殿,有兵革之事。』演孔圖曰:『天子外苦兵,威內奪,臣無忠,則天投蜺。』變不空生,占不空言。」先是立皇后何氏,皇后每齋,當謁祖廟,輒有變異不得謁。中平元年,黃巾賊張角等立三十六方,起兵燒郡國,山東七州處處應角。遣兵外討角等,內使皇后二兄為大將統兵。其年,宮車宴駕,皇后攝政,二兄秉權。譴讓帝母永樂后,令自殺。陰呼并州牧董卓欲共誅中官,中官逆殺大將軍進,兵相攻討,京都戰者塞道。皇太后母子遂為太尉卓等所廢黜,皆死。天下之敗,兵先興於宮省,外延海內,二三十歲,其殃禍起自何氏。


\end{pinyinscope}