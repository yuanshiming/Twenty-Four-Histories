\article{鄭孔荀列傳}

\begin{pinyinscope}
鄭太字公業,河南開封人,司農眾之曾孫也。少有才略。靈帝末,知天下將亂,陰交結豪桀。家富於財,有田四百頃,而食常不足,名聞山東。

初舉孝廉,三府辟,公車徵,皆不就。及大將軍何進輔政,徵用名士,以公業為尚書侍郎,遷侍御史。進將誅閹官,欲召并州牧董卓為助。公業謂進曰:「董卓彊忍寡義,志欲無猒。若借之朝政,授以大事,將恣凶慾,必危朝廷。明公以親德之重,據阿衡之權,秉意獨斷,誅除有罪,誠不宜假卓以為資援也。且事留變生,殷鑒不遠。」又為陳時務之所急數事。進不能用,乃棄官去。謂潁川人荀攸曰:「何公未易輔也。」

進尋見害,卓果作亂。公業等與侍中伍瓊、卓長史何顒共說卓,以袁紹為勃海太守,以發山東之謀。及義兵起,卓乃會公卿議,大發卒討之,群僚莫敢忤旨。公業恐其眾多益橫,凶彊難制,獨曰:「夫政在德,不在眾也。」卓不悅,曰,「如卿此言,兵為無用邪?」公業懼,乃詭詞更對曰:「非謂無用,以為山東不足加大兵耳。如有不信,試為明公略陳其要。今山東合謀,州郡連結,人庶相動,非不強盛。然光武以來,中國無警,百姓優逸,忘戰日久。仲尼有言:『不教人戰,是謂棄之。』其眾雖多,不能為害。一也。明公出自西州,少為國將,閑習軍事,數踐戰埸,名振當世,人懷懾服。二也。袁本初公卿子弟,生處京師。張孟卓東平長者,坐不闚堂。孔公緒清談高論,噓枯吹生。並無軍旅之才,埶銳之幹,臨鋒決敵,非公之儔。三也。山東之士,素乏精悍。未有孟賁之勇,慶忌之捷,聊城之守,良、平之謀,可任以偏師,責以成功。四也。就有其人,而尊卑無序,王爵不加,若恃眾怙力,將各基峙,以觀成敗,不肯同心共膽,與齊進退。五也。關西諸郡,頗習兵事,自頃以來,數與羌戰,婦女猶戴戟操矛,挾弓負矢,況其壯勇之士,以當妄戰之人乎!其勝可必。六也。且天下彊勇,百姓所畏者,有并、涼之人,及匈奴、屠各、湟中義從、西羌八種,而明公擁之,以為爪牙,譬驅虎兕以赴犬羊。七也。又明公將帥,皆中表腹心,周旋日久,恩信淳著,忠誠可任,智謀可恃。以膠固之眾,當解合之埶,猶以烈風掃彼枯葉。八也。夫戰有三亡,以亂攻理者亡,以邪攻正者亡,以逆攻順者亡。今明公秉國平正,討滅宦豎,忠義克立。以此三德,持彼三亡,奉辭伐罪,誰敢禦之!九也。東州鄭玄學該古今,北海邴原清高直亮,皆儒生所仰,群士楷式。彼諸將若詢其計畫,足知彊弱。且燕、趙、齊、梁非不盛也,終滅於秦;吳、楚七國非不眾也,卒敗滎陽。況今德政赫赫,股肱惟良,彼豈讚成其謀,造亂長寇哉?其不然。十也。若其所陳少有可採,無事徵兵以驚天下,使患役之民相聚為非,棄德恃眾,自虧威重。」卓乃悅,以公業為將軍,使統諸軍討擊關東。或說卓曰:「鄭公業智略過人,而結謀外寇,今資之士馬,就其黨與,竊為明公懼之。」卓乃收還其兵,留拜議郎。

卓既遷都長安,天下飢亂,士大夫多不得其命。而公業家有餘資,日引賓客高會倡樂,所贍救者甚眾。乃與何顒、荀攸共謀殺卓。事洩,顒等被執,公業脫身自武關走,東歸袁術。術上以為楊州刺史。未至官,道卒,年四十一。

孔融字文舉,魯國人,孔子二十世孫也。七世祖霸,為元帝師,位至侍中。父宙,太山都尉。

融幼有異才。年十歲,隨父詣京師。時河南尹李膺以簡重自居,不妄接士賓客,敕外自非當世名人及與通家,皆不得白。融欲觀其人,故造膺門。語門者曰:「我是李君通家子弟。」門者言之。膺請融,問曰:「高明祖父嘗與僕有恩舊乎?」融曰:「然。先君孔子與君先人李老君同德比義,而相師友,則融與君累世通家。」眾坐莫不歎息。太中大夫陳煒後至,坐中以告煒。煒曰:「夫人小而聰了,大未必奇。」融應聲曰:「觀君所言,將不早惠乎?」膺大笑曰:「高明必為偉器。」

年十三,喪父,哀悴過毀,扶而後起,州里歸其孝。性好學,博涉多該覽。

山陽張儉為中常侍侯覽所怨,覽為刊章下州郡,以名捕儉。儉與融兄褒有舊,亡抵於褒,不遇。時融年十六,儉少之而不告。融見其有窘色,謂曰:「兄雖在外,吾獨不能為君主邪?」因留舍之。後事泄,國相以下,密就掩捕,儉得脫走,遂并收褒、融送獄。二人未知所坐。融曰:「保納舍藏者,融也,當坐之。」褒曰:「彼來求我,非弟之過,請甘其罪。」吏問其母,母曰:「家事任長,妾當其辜。」一門爭死,郡縣疑不能決,乃上讞之。詔書竟坐褒焉。融由是顯名,與平原陶丘洪、陳留邊讓齊聲稱。州郡禮命,皆不就。

辟司徒楊賜府。時隱覈官僚之貪濁者,將加貶黜,融多舉中官親族。尚書畏迫內寵,召掾屬詰責之。融陳對罪惡,言無阿撓。河南尹何進當遷為大將軍,楊賜遣融奉謁賀進,不時通,融即奪謁還府,投劾而去。河南官屬恥之,私遣劍客欲追殺融。客有言於進曰:「孔文舉有重名,將軍若造怨此人,則四方之士引領而去矣。不如因而禮之,可以示廣於天下。」進然之,既拜而辟融,舉高第,為侍御史。與中丞趙舍不同,託病歸家。

後辟司空掾,拜中軍候。在職三日,遷虎賁中郎將。會董卓廢立,融每因對荅,輒有匡正之言。以忤卓旨,轉為議郎。時黃巾寇數州,而北海最為賊衝,卓乃諷三府同舉融為北海相。

融到郡,收合士民,起兵講武,馳檄飛翰,引謀州郡。賊張饒等群輩二十萬眾從冀州還,融逆擊,為饒所敗,乃收散兵保朱虛縣。稍復鳩集吏民為黃巾所誤者男女四萬餘人,更置城邑,立學校,表顯儒術,薦舉賢良鄭玄、彭璆、邴原等。郡人甄子然、臨孝存知名早卒,融恨不及之,乃命配食縣社。其餘雖一介之善,莫不加禮焉。郡人無後及四方游士有死亡者,皆為棺具而斂葬之。時黃巾復來侵暴,融乃出屯都昌,為賊管亥所圍。融逼急,乃遣東萊太史慈求救於平原相劉備。備驚曰:「孔北海乃復知天下有劉備邪?」即遣兵三千救之,賊乃散走。

時袁、曹方盛,而融無所協附。左丞祖者,稱有意謀,勸融有所結納。融知紹、操終圖漢室,不欲與同,故怒而殺之。

融負其高氣,志在靖難,而才疏意廣,迄無成功。在郡六年,劉備表領青州刺史。建安元年,為袁譚所攻,自春至夏,戰士所餘裁數百人,流矢雨集,戈矛內接。融隱几讀書,談笑自若。城夜陷,乃奔東山,妻子為譚所虜。

及獻帝都許,徵融為將作大匠,遷少府。每朝會訪對,融輒引正定議,公卿大夫皆隸名而已。

初,太傅馬日磾奉使山東,及至淮南,數有意於袁術。術輕侮之,遂奪取其節,求去又不聽,因欲逼為軍帥。日磾深自恨,遂嘔血而斃。及喪還,朝廷議欲加禮。融乃獨議曰:「日磾以上公之尊,秉髦節之使,銜命直指,寧輯東夏,而曲媚姦臣,為所牽率,章表署用,輒使首名,附下罔上,姦以事君。昔國佐當晉軍而不撓,宜僚臨白刃而正色。王室大臣,豈得以見脅為辭!又袁術僭逆,非一朝一夕,日磾隨從,周旋歷歲。漢律與罪人交關三日已上,皆應知情。春秋魯叔孫得臣卒,以不發揚襄仲之罪,貶不書日。鄭人討幽公之亂,斲子家之棺。聖上哀矜舊臣,未忍追案,不宜加禮。」朝廷從之。

時論者多欲復肉刑。融乃建議曰:「古者敦庬,善否不別,吏端刑清,政無過失。百姓有罪,皆自取之。末世陵遲,風化壞亂,政撓其俗,法害其人。故曰上失其道,民散久矣。而欲繩之以古刑,投之以殘棄,非所謂與時消息者也。紂斮朝涉之脛,天下謂為無道。夫九牧之地,千八百君,若各刖一人,是下常有千八百紂也。求俗休和,弗可得已。且被刑之人,慮不念生,志在思死,類多趨惡,莫復歸正。夙沙亂齊,伊戾禍宋,趙高、英布,為世大患。不能止人遂為非也,適足絕人還為善耳。雖忠如鬻拳,信如卞和,智如孫臏,冤如巷伯,才如史遷,達如子政,一離刀鋸,沒世不齒。是太甲之思庸,穆公之霸秦,南睢之骨立,衛武之初筵,陳湯之都賴,魏尚之守邊,無所復施也。漢開改惡之路,凡為此也。故明德之君,遠度深惟,棄短就長,不苟革其政者也。」朝廷善之,卒不改焉。

是時荊州牧劉表不供職貢,多行僭偽,遂乃郊祀天地,擬斥乘輿。詔書班下其事。融上疏曰:「竊聞領荊州牧劉表桀逆放恣,所為不軌,至乃郊祭天地,擬儀社稷。雖昏僭惡極,罪不容誅,至於國體,宜且諱之。何者?萬乘至重,天王至尊,身為聖躬,國為神器,陛級縣遠,祿位限絕,猶天之不可階,日月之不可踰也。每有一豎臣,輒云圖之,若形之四方,非所以杜塞邪萌。愚謂雖有重戾,必宜隱忍。賈誼所謂『擲鼠忌器』,蓋謂此也。是以齊兵次楚,唯責包茅;王師敗績,不書晉人。前以露袁術之罪,今復下劉表之事,是使跛牂欲闚高岸,天險可得而登也。案表跋扈,擅誅列侯,遏絕詔命,斷盜貢篚,招呼元惡,以自營衛,專為群逆,主萃淵藪。郜鼎在廟,章孰甚焉!桑落瓦解,其埶可見。臣愚以為宜隱郊祀之事,以崇國防。」

五年,南陽王馮、東海王祗薨,帝傷其早歿,欲為脩四時之祭,以訪於融。融對曰:「聖恩敦睦,感時增思,悼二王之靈,發哀愍之詔,稽度前典,以正禮制。竊觀故事,前梁懷王、臨江愍王、齊哀王、臨淮懷王並薨無後,同產昆弟,即景、武、昭、明四帝是也,未聞前朝修立祭祀。若臨時所施,則不列傳紀。臣愚以為諸在沖齔,聖慈哀悼,禮同成人,加以號謚者,宜稱上恩,祭祀禮畢,而後絕之。至於一歲之限,不合禮意,又違先帝已然之法,所未敢處。」

初,曹操攻屠鄴城,袁氏婦子多見侵略,而操子丕私納袁熙妻甄氏。融乃與操書,稱「武王伐紂,以妲己賜周公」。操不悟,後問出何經典。對曰:「以今度之,想當然耳。」後操討烏桓,又嘲之曰:「大將軍遠征,蕭條海外。昔肅慎不貢楛矢,丁零盜蘇武牛羊,可并案也。」

時年飢兵興,操表制酒禁,融頻書爭之,多侮慢之辭。既見操雄詐漸著,數不能堪,故發辭偏宕,多致乖忤。又嘗奏宜準古王畿之制,千里寰內,不以封建諸侯。操疑其所論建漸廣,益憚之。然以融名重天下,外相容忍,而潛忌正議,慮鯁大業。山陽郗慮承望風旨,以微法奏免融官。因顯明讎怨,操故書激厲融曰:「蓋聞唐虞之朝,有克讓之臣,故麟鳳來而頌聲作也。後世德薄,猶有殺身為君,破家為國。及至其敝,睚眥之怨必讎,一餐之惠必報。故晁錯念國,遘禍於袁盎;屈平悼楚,受譖於椒、蘭;彭寵傾亂,起自朱浮;鄧禹威損,失於宗、馮。由此言之,喜怒怨愛,禍福所因,可不慎與!昔廉、藺小國之臣,猶能相下;寇、賈倉卒武夫,屈節崇好;光武不問伯升之怨;齊侯不疑射鉤之虜。夫立大操者,豈累細故哉!往聞二君有執法之平,以為小介,當收舊好;而怨毒漸積,志相危害,聞之憮然,中夜而起。昔國家東遷,文舉盛歎鴻豫名實相副,綜達經學,出於鄭玄,又明司馬法,鴻豫亦稱文舉奇逸博聞,誠怪今者與始相違。孤與文舉既非舊好,又於鴻豫亦無恩紀,然願人之相美,不樂人之相傷,是以區區思協歡好。又知二君群小所搆,孤為人臣,進不能風化海內,退不能建德和人,然撫養戰士,殺身為國,破浮華交會之徒,計有餘矣。」

融報曰:「猥惠書教,告所不逮。融與鴻豫州里比郡,知之最早。雖嘗陳其功美,欲以厚於見私,信於為國,不求其覆過掩惡,有罪望不坐也。前者黜退,懽欣受之。昔趙宣子朝登韓厥,夕被其戮,喜而求賀。況無彼人之功,而敢枉當官之平哉!忠非三閭,智非晁錯,竊位為過,免罪為幸。乃使餘論遠聞,所以慚懼也。朱、彭、寇、賈,為世壯士,愛惡相攻,能為國憂。至於輕弱薄劣,猶昆蟲之相囓,適足還害其身,誠無所至也。晉侯嘉其臣所爭者大,而師曠以為不如心競。性既遲緩,與人無傷,雖出胯下之負,榆次之辱,不知貶毀之於己,猶蚊虻之一過也。子產謂人心不相似,或矜埶者,欲以取勝為榮,不念宋人待四海之客,大鑪不欲令酒酸也。至於屈穀巨瓠,堅而無竅,當以無用罪之耳。它者奉遵嚴教,不敢失墜。郗為故吏,融所推進。趙衰之拔郤縠,不輕公叔之升臣也。知同其愛,訓誨發中。雖懿伯之忌,猶不得念,況恃舊交,而欲自外於賢吏哉!輒布腹心,脩好如初。苦言至意,終身誦之。」

歲餘,復拜太中大夫。性寬容少忌,好士,喜誘益後進。及退閑職,賓客日盈其門。常歎曰:「坐上客恆滿,尊中酒不空,吾無憂矣。」與蔡邕素善,邕卒後,有虎賁士貌類於邕,融每酒酣,引與同坐,曰:「雖無老成人,且有典刑。」融聞人之善,若出諸己,言有可採,必演而成之,面告其短,而退稱所長,薦達賢士,多所獎進,知而未言,以為己過,故海內英俊皆信服之。

曹操既積嫌忌,而郗慮復搆成其罪,遂令丞相軍謀祭酒路粹枉狀奏融曰:「少府孔融,昔在北海,見王室不靜,而招合徒眾,欲規不軌,云『我大聖之後,而見滅於宋,有天下者,何必卯金刀』。及與孫權使語,謗訕朝廷。又融為九列,不遵朝儀,禿巾微行,唐突宮掖。又前與白衣禰衡跌蕩於言,云『父之於子,當有何親?論其本意,實為情欲發耳。子之於母,亦復奚為?譬如寄物莳中,出則離矣』。既而與衡更相贊揚。衡謂融曰:『仲尼不死。』融荅曰:『顏回復生。』大逆不道,宜極重誅。」書奏,下獄棄市。時年五十六。妻子皆被誅。

初,女年七歲,男年九歲,以其幼弱得全,寄它舍。二子方弈棋,融被收而不動。左右曰:「父執而不起,何也?」荅曰;「安有巢毀而卵不破乎!」主人有遺肉汁,男渴而飲之。女曰:「今日之禍,豈得久活,何賴知肉味乎?」兄號泣而止。或言於曹操,遂盡殺之。及收至,謂兄曰;「若死者有知,得見父母,豈非至願!」乃延頸就刑,顏色不變,莫不傷之。

初,京兆人脂習元升,與融相善,每戒融剛直。及被害,許下莫敢收者,習往撫尸曰:「文舉舍我死,吾何用生為?」操聞大怒,將收習殺之,後得赦出。

魏文帝深好融文辭,每歎曰:「楊、班儔也。」募天下有上融文章者,輒賞以金帛。所著詩、頌、碑文、論議、六言、策文、表、檄、教令、書記凡二十五篇。文帝以習有欒布之節,加中散大夫。

論曰:昔諫大夫鄭昌有言:「山有猛獸者,藜藿為之不採。」是以孔父正色,不容弒虐之謀;平仲立朝,有紓盜齊之望。若夫文舉之高志直情,其足以動義概而忤雄心。故使移鼎之跡,事隔於人存;代終之規,啟機於身後也。夫嚴氣正性,覆折而己。豈有員馅委屈,可以每其生哉!懍懍焉,皜皜焉,其與琨玉秋霜比質可也。

苟彧字文若,潁川潁陰人,朗陵令淑之孫也。父緄,為濟南相。緄畏憚宦官,乃為彧娶中常侍唐衡女。彧以少有才名,故得免於譏議。南陽何顒名知人,見彧而異之,曰:「王佐才也。」

中平六年,舉孝廉,再遷亢父令。董卓之亂,棄官歸鄉里。同邵韓融時將宗親千餘家,避亂密西山中。彧謂父老曰:「潁川,四戰之地也。天下有變,常為兵衝。密雖小固,不足以扞大難,宜亟避之。」鄉人多懷土不能去。會冀州牧同郡韓馥遣騎迎之,彧乃獨將宗族從馥,留者後多為董卓將李傕所殺略焉。

彧比至冀州,而袁紹已奪馥位,紹待彧以上賓之禮。彧明有意數,見漢室崩亂,每懷匡佐之義。時曹操在東郡,彧聞操有雄略,而度紹終不能定大業。初平二年,乃去紹從操。操與語大悅,曰:「吾子房也。」以為奮武司馬,時年二十九。明年,又為操鎮東司馬。

興平元年,操東擊陶謙,使彧守甄城,任以留事。會張邈、陳宮以兗州反操,而潛迎呂布。布既至,諸城悉應之。邈乃使人譎彧曰:「呂將軍來助曹使君擊陶謙,宜亟供軍實。」彧知邈有變,即勒兵設備,故邈計不行。豫州刺史郭貢率兵數萬來到城下,求見彧。彧將往,東郡太守夏侯惇等止之。曰:「何知貢不與呂布同謀,而輕欲見之。今君為一州之鎮,往必危也。」彧曰:「貢與邈等分非素結,今來速者,計必未定,及其猶豫,宜時說之,縱不為用,可使中立。若先懷疑嫌,彼將怒而成謀,不如往也。」貢既見彧無懼意,知城不可攻,遂引而去。彧乃使程昱說范、東阿,使固其守,卒全三城以待操焉。

二年,陶謙死,操欲遂取徐州,還定呂布。彧諫曰:「昔高祖保關中,光武據河內,皆深根固本,以制天下。進可以勝敵,退足以堅守,故雖有困敗,而終濟大業。將軍本以兗州首事,故能平定山東,此實天下之要地,而將軍之關河也。若不先定之,根本將何寄乎?宜急分討陳宮,使虜不得西顧,乘其閒而收熟麥,約食蓄穀,以資一舉,則呂布不足破也。今舍之而東,未見其便。多留兵則力不勝敵,少留兵則後不足固。布乘虛寇暴,震動人心,縱數城或全,其餘非復己有,則將軍尚安歸乎?且前討徐州,威罰實行,其子弟念父兄之恥,必人自為守。就能破之,尚不可保。彼若懼而相結,共為表裏,堅壁清野,以待將軍,將軍攻之不拔,掠之無獲,不出一旬,則十萬之眾未戰而自困矣。夫事固有棄彼取此,以權一時之埶,願將軍慮焉。」操於是大收孰麥,復與布戰。布敗走,因分定諸縣,兗州遂平。

建安元年,獻帝自河東還洛陽,操議欲奉迎車駕,徙都於許。眾多以山東未定,韓暹、楊奉負功恣睢,未可卒制。彧乃勸操曰:「昔晉文公納周襄王,而諸侯景從;漢高祖為義帝縞素,而天下歸心。自天子蒙塵,將軍首唱義兵,徒以山東擾亂,未遑遠赴,雖禦難於外,乃心無不在王室。今鑾駕旋軫,東京榛蕪,義士有存本之思,兆人懷感舊之哀。誠因此時奉主上以從人望,大順也;秉至公以服天下,大略也;扶弘義以致英俊,大德也。四方雖有逆節,其何能為?韓暹、楊奉,安足恤哉!若不時定,使豪桀生心,後雖為慮,亦無及矣。」操從之。

及帝都許,以彧為侍中,守尚書令。操每征伐在外,其軍國之事,皆與彧籌焉。彧又進操計謀之士從子攸,及鍾繇、郭嘉、陳群、杜襲、司馬懿、戲志才等,皆稱其舉。唯嚴象為楊州,韋康為涼州,後並負敗焉。

袁紹既兼河朔之地,有驕氣。而操敗於張繡,紹與操書甚倨。操大怒,欲先攻之,而患力不敵,以謀於彧。彧量紹雖強,終為操所制,乃說先取呂布,然後圖紹,操從之。三年,遂擒呂布,定徐州。

五年,袁紹率大眾以攻許,操與相距。紹甲兵甚盛,議者咸懷惶懼。少府孔融謂彧曰:「袁紹地廣兵彊,田豐、許攸智計之士為其謀,審配、逢紀盡忠之臣任其事,顏良、文醜勇冠三軍,統其兵,殆難克乎?」彧曰:「紹兵雖多而法不整,田豐剛而犯上,許攸貪而不正,審配專而無謀,逢紀果而自用,顏良、文醜匹夫之勇,可一戰而擒也。」後皆如彧之籌,事在袁紹傳。

操保官度,與紹連戰,雖勝而軍糧方盡,與彧議,欲還許以致紹師。彧報曰:「今穀食雖少,未若楚漢在滎陽、成皋閒也。是時劉項莫肯先退者,以為先退則埶屈也。公以十分居一之眾,畫地而守之,搤其喉而不得進,已半年矣。情見埶竭,必將有變,此用奇之時,不可失也。」操從之,乃堅壁持之。遂以奇兵破紹,紹退走。封彧萬歲亭侯,邑一千戶。

六年,操以紹新破,未能為患,但欲留兵衛之,自欲南征劉表,以計問彧。彧對曰:「紹既新敗,眾懼人擾,今不因而定之,而欲遠兵江漢,若紹收離糾散,乘虛以出,則公之事去矣。」操乃止。

九年,操拔鄴,自領冀州牧。有說操宜復置九州者,以為冀部所統既廣,則天下易服。操將從之。彧言曰:「今若依古制,是為冀州所統,悉有河東、馮翊、扶風、西河、幽、并之地也。公前屠鄴城,海內震駭,各懼不得保其土宇,守其兵眾。今若一處被侵,必謂以次見奪,人心易動,若一旦生變,天下未可圖也。願公先定河北,然後脩復舊京,南臨楚郢,責王貢之不入。天下咸知公意,則人人自安。須海內大定,乃議古制,此社稷長久之利也。」操報曰:「微足下之相難,所失多矣!」遂寑九州議。

十二年,操上書表彧曰:「昔袁紹作逆,連兵官度,時眾寡糧單,圖欲還許。尚書令荀彧深建宜住之便,遠恢進討之略,起發臣心,革易愚慮,堅營固守,徼其軍實,遂摧撲大寇,濟危以安。紹既破敗,臣糧亦盡,將舍河北之規,改就荊南之策。彧復備陳得失,用移臣議,故得反旆冀土,克平四州。向使臣退軍官度,紹必鼓行而前,敵人懷利以自百,臣眾怯沮以喪氣,有必敗之形,無一捷之埶。復若南征劉表,委棄兗、豫,飢軍深入,踰越江、沔,利既難要,將失本據。而彧建二策,以亡為存,以禍為福,謀殊功異,臣所不及。是故先帝貴指縱之功,薄搏獲之賞;古人尚帷幄之規,下攻拔之力。原其績效,足享高爵。而海內未喻其狀,所受不侔其功,臣誠惜之。乞重平議,增疇戶邑。」彧深辭讓。操譬之曰:「昔介子推有言:『竊人之財,猶謂之盜。』況君奇謨拔出,興亡所係,可專有之邪?雖慕魯連沖高之跡,將為聖人達節之義乎!」於是增封千戶,并前二千戶。又欲授以正司,彧使荀攸深自陳讓,至于十數,乃止。操將伐劉表,問彧所策。彧曰:「今華夏以平,荊、漢知亡矣,可聲出宛、葉而閒行輕進,以掩其不意。」操從之。會表病死。

十七年,董昭等欲共進操爵國公,九錫備物,密以訪彧。彧曰「曹公本興義兵,以匡振漢朝,雖勳庸崇著,猶秉忠貞之節。君子愛人以德,不宜如此。」事遂寑。操心不能平。會南征孫權,表請彧勞軍于譙,因表留彧曰:「臣聞古之遣將,上設監督之重,下建副二之任,所以尊嚴國命,謀而鮮過者也。臣今當濟江,奉辭伐罪,宜有大使肅將王命。文武並用,自古有之。使持節侍中守尚書令萬歲亭侯彧,國之望臣,德洽華夏,既停軍所次,便宜與臣俱進,宣示國命,威懷醜虜。軍禮尚速,不及先請,臣輒留彧,依以為重。」書奏,帝從之,遂以彧為侍中、光祿大夫,持節,參丞相軍事。至濡須,彧病留壽春,操饋之食,發視,乃空器也,於是飲藥而卒。時年五十。帝哀惜之,祖日為之廢讌樂。謚曰敬侯。明年,操遂稱魏公云。

論曰:自遷帝西京,山東騰沸,天下之命倒縣矣。荀君乃越河、冀,閒關以從曹氏。察其定舉措,立言策,崇明王略,以急國艱,豈云因亂假義,以就違正之謀乎?誠仁為己任,期紓民於倉卒也。及阻董昭之議,以致非命,豈數也夫!世言荀君者,通塞或過矣。常以為中賢以下,道無求備,智筭有所研疏,原始未必要末。斯理之不可全詰者也。夫以衛賜之賢,一說而斃兩國。彼非薄於仁而欲之,蓋有全必有喪也,斯又功之不兼者也。方時運之屯邅,非雄才無以濟其溺,功高埶彊,則皇器自移矣。此又時之不可並也。蓋取其歸正而已,亦殺身以成仁之義也。

贊曰:公業稱豪,駿聲升騰。權詭時偪,揮金僚朋。北海天逸,音情頓挫。越俗易驚,孤音少和。直轡安歸,高謀誰佐?彧之有弼,誠感國疾。功申運改,跡疑心一。


\end{pinyinscope}