\article{南匈奴列傳}

\begin{pinyinscope}
南匈奴虾落尸逐鞮單于比者,呼韓邪單于之孫,烏珠留若鞮單于之子也。自呼韓邪後,諸子以次立,至比季父孝單于輿時,以比為右薁鞬日逐王,部領南邊及烏桓。

建武初,彭寵反畔於漁陽,單于與共連兵,因復權立盧芳,使入居五原。光武初,方平諸夏,未遑外事。至六年,始令歸德侯劉颯使匈奴,匈奴亦遣使來獻,漢復令中郎將韓統報命,賂遺金幣,以通舊好。而單于驕踞,自比冒頓,對使者辭語悖慢,帝待之如初。初,使命常通,而匈奴數與盧芳共侵北邊。九年,遣大司馬吳漢等擊之,經歲無功,而匈奴轉盛,鈔暴日增。十三年,遂寇河東,州郡不能禁。於是漸徙幽、并邊人於常山關、居庸關已東,匈奴左部遂復轉居塞內。朝廷患之,增緣邊兵郡數千人,大築亭候,修烽火。匈奴聞漢購求盧芳,貪得財帛,乃遣芳還降,望得其賞。而芳以自歸為功,不稱匈奴所遣,單于復恥言其計,故賞遂不行。由是大恨,入寇尤深。二十年,遂至上黨、扶風、天水。二十一年冬,復寇上谷、中山,殺略鈔掠甚眾,北邊無復寧歲。

初,單于弟右谷蠡王伊屠知牙師以次當左賢王。左賢王即是單于儲副。單于欲傳其子,遂殺知牙師。知牙師者,王昭君之子也。昭君字嬙,南郡人也。初,元帝時,以良家子選入掖庭。時呼韓邪來朝,帝敕以宮女五人賜之。昭君入宮數歲,不得見御,積悲怨,乃請掖庭令求行。呼韓邪臨辭大會,帝召五女以示之。昭君豐容靚飾,光明漢宮,顧景裴回,竦動左右。帝見大驚,意欲留之,而難於失信,遂與匈奴。生二子。及呼韓邪死,其前閼氏子代立,欲妻之,昭君上書求歸,成帝敕令從胡俗,遂復為後單于閼氏焉。

比見知牙師被誅,出怨言曰:「以兄弟言之,右谷蠡王次當立;以子言之,我前單于長子,我當立。」遂內懷猜懼,庭會稀闊。單于疑之,乃遣兩骨都侯監領比所部兵。二十二年,單于輿死,子左賢王烏達鞮侯立為單于。復死,弟左賢王蒲奴立為單于。比不得立,既懷憤恨。而匈奴中連年旱蝗,赤地數千里,草木盡枯,人畜飢疫,死耗太半。單于畏漢乘其敝,乃遣使詣漁陽求和親。於是遣中郎將李茂報命。而比密遣漢人郭衡奉匈奴地圖,二十三年,詣西河太守求內附。兩骨都侯頗覺其意,會五月龍祠,因白單于,言薁鞬日遂夙來欲為不善,若不誅,且亂國。時比弟漸將王在單于帳下,聞之,馳以報比。比懼,遂斂所主南邊八部眾四五萬人,待兩骨都侯還,欲殺之。骨都侯且到,知其謀,皆輕騎亡去,以告單于。單于遣萬騎擊之,見比眾盛,不敢進而還。

二十四年春,八部大人共議立比為呼韓邪單于,以其大父嘗依漢得安,故欲襲其號。於是款五原塞,願永為蕃蔽,扞禦北虜。帝用五官中郎將耿國議,乃許之。其冬,比自立為呼韓邪單于。

二十五年春,遣弟左賢王莫將兵萬餘人擊北單于弟薁鞬左賢王,生獲之;又破北單于帳下,并得其眾合萬餘人,馬七千匹、牛羊萬頭。北單于震怖,卻地千里。初,帝造戰車,可駕數牛,上作樓櫓,置於塞上,以拒匈奴。時人見者或相謂曰:「讖言漢九世當卻北狄地千里,豈謂此邪?」及是,果拓地焉。北部薁鞬骨都侯與右骨都侯率眾三萬餘人來歸南單于,南單于復遣使詣闕,奉藩稱臣,獻國珍寶,求使者監護,遣侍子,修舊約。

二十六年,遣中郎將段郴、副校尉王郁使南單于,立其庭,去五原西部塞八十里。單于乃延迎使者。使者曰:「單于當伏拜受詔。」單于顧望有頃,乃伏稱臣。拜訖,令譯曉使者曰:「單于新立,誠慚於左右,願使者眾中無相屈折也。」骨都侯等見,皆泣下。郴等反命,詔乃聽南單于入居雲中。遣使上書,獻駱锓二頭,文馬十匹。夏,南單于所獲北虜薁鞬左賢王將其眾及南部五骨都侯合三萬餘人畔歸,去北庭三百餘里,共立薁鞬左賢王為單于。月餘日,更相攻擊,五骨都侯皆死,左賢王遂自殺,諸骨都侯子各擁兵自守。秋,南單于遣子入侍,奉奏詣闕。詔賜單于冠帶、衣裳、黃金璽、盭緺綬,安車羽蓋,華藻駕駟,寶劍弓箭,黑節三,駙馬二,黃金、錦繡、繒布萬匹,絮萬斤,樂器鼓車,棨戟甲兵,飲食什器。又轉河東米糒二萬五千斛,牛羊三萬六千頭,以贍給之。令中郎將置安集掾吏將弛刑五十人,持兵弩隨單于所處,參辭訟,察動靜。單于歲盡輒遣奉奏,送侍子入朝,中郎將從事一人將領詣闕。漢遣謁者送前侍子還單于庭,交會道路。元正朝賀,拜祠陵廟畢,漢乃遣單于使,令謁者將送,賜綵繒千匹,錦四端,金十斤,太官御食醬及橙、橘、龍眼、荔支;賜單于母及諸閼氏、單于子及左右賢王、左右谷蠡王、骨都侯有功善者,繒綵合萬四。歲以為常。

匈奴俗,歲有三龍祠,常以正月、五月、九月戊日祭天神。南單于既內附,兼祠漢帝,因會諸部,議國事,走馬及駱锓為樂。其大臣貴者左賢王,次左谷蠡王,次右賢王,次右谷蠡王,謂之四角;次左右日逐王,次左右溫禺鞮王,次左右漸將王,是為六角:皆單于子弟,次第當為單于者也。異姓大臣左右骨都侯,次左右尸逐骨都侯,其餘日逐、且渠、當戶諸官號,各以權力優劣、部眾多少為高下次第焉。單于姓虛連題。異姓有呼衍氏、須卜氏、丘林氏、蘭氏四姓,為國中名族,常與單于婚姻。呼衍氏為左,蘭氏、須卜氏為右,主斷獄聽訟,當決輕重,口白單于,無文書簿領焉。

冬,前畔五骨都侯子復將其眾三千人歸南部,北單于使騎追擊。悉獲其眾。南單于遣兵拒之,逆戰不利。於是復詔單于徙居西河美稷,因使中郎將段郴及副校尉王郁留西河擁護之,為設官府、從事、掾史。令西河長史歲將騎二千,弛刑五百人,助中郎將衛護單于,冬屯夏罷。自後以為常,及悉復緣邊八郡。

南單于既居西河,亦列置諸部王,助為扞戍。使韓氏骨都侯屯北地,右賢王屯朔方,當于骨都侯屯五原,呼衍骨都侯屯雲中,郎氏骨都侯屯定襄,左南將軍屯鴈門,栗籍骨都侯屯代郡,皆領部眾為郡縣偵羅耳目。北單于惶恐,頗還所略漢人,以示善意。鈔兵每到南部下,還過亭候,輒謝曰:「自擊亡虜薁鞬日逐耳,非敢犯漢人也。」

二十七年,北單于遂遣使詣武威求和親,天子召公卿廷議,不決。皇太子言曰:「南單于新附,北虜懼於見伐,故傾耳而聽,爭欲歸義耳。今未能出兵,而反交通北虜,臣恐南單于將有二心,北虜降者且不復來矣。」帝然之,告武威太守勿受其使。

二十八年,北匈奴復遣使詣闕,貢馬及裘,更乞和親,并請音樂,又求率西域諸國胡客與俱獻見。帝下三府議酬荅之宜。司徒掾班彪奏曰:

臣聞孝宣皇帝敕邊守尉曰:「匈奴大國,多變詐。交接得其情,則卻敵折衝;應對入其數,則反為輕欺。」今北匈奴見南單于來附,懼謀其國,故數乞和親,又遠驅牛馬與漢合巿,重遣名王,多所貢獻,斯皆外示富強,以相欺誕也。臣見其獻益重,知其國益虛,歸親愈數,為懼愈多。然今既未獲助南,則亦不宜絕北,羈縻之義,禮無不荅。謂可頗加賞賜,略與所獻相當,明加曉告以前世呼韓邪、郅支行事。

報荅之辭,令必有適。今立稿草并上,曰:「單于不忘漢恩,追念先祖舊約,欲修和親,以輔身安國,計議甚高,為單于嘉之。往者,匈奴數有乘亂,呼韓邪、郅支自相讎隙,並蒙孝宣皇帝垂恩救護,故各遣侍子稱藩保塞。其後郅支忿戾,自絕皇澤,而呼韓附親,忠孝彌著。及漢滅郅支,遂保國傳嗣,子孫相繼。今南單于攜眾南向,款塞歸命。自以呼韓嫡長,次第當立,而侵奪失職,猜疑相背,數請兵將,歸埽北庭,策謀紛紜,無所不至。惟念斯言不可獨聽,又以北單于比年貢獻,欲修和親,故拒而未許,將以成單于忠孝之義。漢秉威信,總率萬國,日月所照,皆為臣妾。殊俗百蠻,義無親疏,服順者褒賞,畔逆者誅罰,善惡之效,呼韓、郅支是也。今單于欲修和親,款誠已達,何嫌而欲率西域諸國俱來獻見?西域國屬匈奴,與屬漢何異?單于數連兵亂,國內虛耗,貢物裁以通禮,何必獻馬裘?今齎雜繒五百匹,弓鞬韥丸一,矢四發,遣遺單于。又賜獻馬左骨都侯、右谷蠡王雜繒各四百匹,斬馬劍各一。單于前言先帝時所賜呼韓邪竽、瑟、空侯皆敗,願復裁。念單于國尚未安,方厲武節,以戰攻為務,竽瑟之用不如良弓利劍,故未以齎。朕不愛小物於單于,便宜所欲,遣驛以聞。」

帝悉納從之。二十九年,賜南單于羊數萬頭。三十一年,北匈奴復遣使如前,乃璽書報荅,賜以綵繒,不遣使者。

單于比立九年薨,中郎將段郴將兵赴弔,祭以酒米,分兵衛護之。比弟左賢王莫立,帝遣使者齎璽書鎮慰,拜授璽綬,遺冠幘,絳單衣三襲,童子佩刀、緄帶各一,又賜繒綵四千匹,令賞賜諸王、骨都侯已下。其後單于薨,弔祭慰賜,以此為常。

丘浮尤鞮單于莫,中元元年立,一年薨,弟汗立。

伊伐於慮鞮單于汗,中元二年立。永平二年,北匈奴護于丘率眾千餘人來降。南部單于汗立二年薨,單于比之子適立。

虾僮尸逐侯鞮單于適,永平二年立。五年冬,北匈奴六七千騎入于五原塞,遂寇雲中至原陽,南單于擊卻之,西河長史馬襄赴救,虜乃引去。

單于適立四年薨,單于莫子蘇立,是為丘除車林鞮單于。數月復薨,單于適之弟長立。

胡邪尸逐侯鞮單于長,永平六年立。時北匈奴猶盛,數寇邊,朝廷以為憂。會北單于欲合巿,遣使求和親,顯宗冀其交通,不復為寇。乃許之。

八年,遣越騎司馬鄭眾北使報命,而南部須卜骨都侯等知漢與北虜交使,懷嫌怨欲畔,密因北使,令遣兵迎之。鄭眾出塞,疑有異,伺候果得須卜使人,乃上言宜更置大將,以防二虜交通。由是始置度遼營,以中郎將吳棠行度遼將軍事,副校尉來苗、左校尉閻章、右校尉張國將黎陽虎牙營士屯五原曼柏。又遣騎都尉秦彭將兵屯美稷。其年秋,北虜果遣二千騎候望朔方,作馬革船,欲度迎南部畔者,以漢有備,乃引去。復數寇鈔邊郡,焚燒城邑,殺略甚眾,河西城門晝閉。帝患之。

十六年,乃大發緣邊兵,遣諸將四道出塞,北征匈奴。南單于遣左賢王信隨太僕祭肜及吳棠出朔方高闕,攻皋林溫禺犢王於涿邪山。虜聞漢兵來,悉度漠去。肜、棠坐不至涿邪山免,以騎都尉來苗行度遼將軍。其年,北匈奴入雲中,遂至漁陽,太守廉范擊卻之。詔遣使者高弘發三郡兵追之,無所得。

建初元年,來苗遷濟陰太守,以征西大將軍耿秉行度遼將軍。時皋林溫禺犢王復將眾還居涿邪山,南單于聞知,遣輕騎與緣邊郡及烏桓兵出塞擊之,斬首數百級,降者三四千人。其年,南部苦蝗,大飢,肅宗稟給其貧人三萬餘口。七年,耿秉遷執金吾,以張掖太守鄧鴻行度遼將軍。八年,北匈奴三木樓訾大人稽留斯等率三萬八千人、馬二萬匹、牛羊十餘萬,款五原塞降。

元和元年,武威太守孟雲上言北單于復願與吏人合市,詔書聽雲遣驛使迎呼慰納之。北單于乃遣大且渠伊莫訾王等,驅牛馬萬餘頭來與漢賈客交易。諸王大人或前至,所在郡縣為設官邸,賞賜待遇之。南單于聞,乃遣輕騎出上郡,遮略生口,鈔掠牛馬,驅還入塞。

二年正月,北匈奴大人車利、涿兵等亡來入塞,凡七十三輩。時北虜衰耗,黨眾離畔,南部攻其前,丁零寇其後,鮮卑擊其左,西域侵其右,不復自立,乃遠引而去。

單于長立二十三年薨,單于汗之子宣立。

伊屠於閭鞮單于宣,元和二年立。其歲,單于遣兵千餘人獵至涿邪山,卒與北虜溫禺犢王遇,因戰,獲其首級而還。冬,孟雲上言:「北虜以前既和親,而南部復往鈔掠,北單于謂漢欺之,謀欲犯塞,謂宜還南所掠生口,以慰安其意。」肅宗從太僕袁安議,許之。乃下詔曰:「昔獫狁、獯粥之敵中國,其所由來尚矣。往者雖有和親之名,終無絲髮之效。墝埆之人,屢嬰塗炭,父戰於前,子死於後。弱女乘於亭障,孤兒號於道路。老母寡妻設虛祭,飲泣淚,想望歸魂於沙漠之表,豈不哀哉!傳曰:『江海所以能長百川者,以其下之也。』少加屈下,尚何足病?況今與匈奴君臣分定,辭順約明,貢獻累至,豈宜違信自受其曲。其敕度遼及領中郎將龐奮倍雇南部所得生口,以還北虜。其南部斬首獲生,計功受賞如常科。」於是南單于復令薁鞮日逐王師子將輕騎數千出塞掩擊北虜,復斬獲千人。北虜眾以南部為漢所厚,又聞取降者歲數千人。

章和元年,鮮卑入左地擊北匈奴,大破之,斬優留單于,取其匈奴皮而還。北庭大亂,屈蘭、儲卑、胡都須等五十八部,口二十萬,勝兵八千人,詣雲中、五原、朔方、北地降。

單于宣立三年薨,單于長之弟屯屠何立。

休蘭尸逐侯鞮單于屯屠何,章和二年立。時北虜大亂,加以飢蝗,降者前後而至。南單于將并北庭,會肅宗崩,竇太后臨朝。其年七月,單于上言:「臣累世蒙恩,不可勝數。孝章皇帝聖思遠慮,遂欲見成就,故令烏桓、鮮卑討北虜,斬單于首級,破壞其國。今所新降虛渠等詣臣自言:『去歲三月中發虜庭,北單于創刈南兵,又畏丁令、鮮卑,遯逃遠去,依安侯河西。今年正月,骨都侯等復共立單于異母兄右賢王為單于,其人以兄弟爭立,並各離散。』臣與諸王骨都侯及新降渠帥雜議方略,皆曰宜及北虜分爭,出兵討伐,破北成南,并為一國,令漢家長無北念。又今月八日,新降右須日逐鮮堂輕從虜庭遠來詣臣,言北虜諸部多欲內顧,但恥自發遣,故未有至者。若出兵奔擊,必有響應。今年不往,恐復并壹。臣伏念先父歸漢以來,被蒙覆載,嚴塞明候,大兵擁護,積四十年。臣等生長漢地,開口仰食,歲時賞賜,動輒億萬,雖垂拱安枕,慚無報效之義。願發國中及諸部故胡新降精兵,遣左谷蠡王師子、左呼衍日逐王須訾將萬騎出朔方,左賢王安國、右大且渠王交勒蘇將萬騎出居延,期十二月同會虜地。臣將餘兵萬人屯五原、朔方塞,以為拒守。臣素愚淺,又兵眾單少,不足以防內外。願遣執金吾耿秉、度遼將軍鄧鴻及西河、雲中、五原、朔方、上郡太守并力而北,令北地、安定太守各屯要害,冀因聖帝威神,一舉平定。臣國成敗,要在今年。已敕諸部嚴兵馬,訖九月龍祠,悉集河上。唯陛下裁哀省察!」太后以示耿秉。秉上言:「昔武帝單極天下,欲臣虜匈奴,未遇天時,事遂無成。宣帝之世,會呼韓來降,故邊人獲安,中外為一,生人休息六十餘年。及王莽篡位,變更其號,耗擾不止,單于乃畔。光武受命,復懷納之,緣邊壞郡得以還復。烏桓、鮮卑咸脅歸義,威鎮西夷,其效如此。今幸遭天授,北虜分爭,以夷伐夷,國家之利,宜可聽許。」秉因自陳恩,分當出命效用。太后從之。

永元元年,以秉為征西將軍,與車騎將軍竇憲率騎八千,與度遼兵及南單于眾三萬騎,出朔方擊北虜,大破之。北單于奔走,首虜二十餘萬人。事已具竇憲傳。

二年春,鄧鴻遷大鴻臚,以定襄太守皇甫棱行度遼將軍。南單于復上求滅北庭,於是遣左谷蠡王師子等將左右部八千騎出雞鹿塞,中郎將耿譚遣從事將護之。至涿邪山,乃留輜重,分為二部,各引輕兵兩道襲之。左部北過西海至河雲北,右部從匈奴河水西繞天山,南度甘微河,二軍俱會,夜圍北單于。大驚,率精兵千餘人合戰。單于被創,墯馬復上,將輕騎數十遁走,僅而免脫。得其玉璽,獲閼氏及男女五人,斬首八千級,生虜數千口而還。是時南部連剋獲納降,黨眾最盛,領戶三萬四千,口二十三萬七千三百,勝兵五萬一百七十。故從事中郎將置從事二人,耿譚以新降者多,上增從事十二人。

三年,北單于復為右校尉耿夔所破,逃亡不知所在。其弟右谷蠡王於除鞬自立為單于,將右溫禺鞬王、骨都侯已下眾數千人,止蒲類海,遣使款塞。大將軍竇憲上書,立於除鞬為北單于,朝廷從之。四年,遣耿夔即授璽綬,賜玉劍四具,羽蓋一駟,使中郎將任尚持節衛護屯伊吾,如南單于故事。方欲輔歸北庭,會竇憲被誅。五年,於除鞬自畔還北,帝遣將兵長史王輔以千餘騎與任尚共追誘將還斬之,破滅其眾。

單于屯屠何立六年薨,單于宣弟安國立。

單于安國,永元五年立。安國初為左賢王而無稱譽。左谷蠡王師子素勇黠多知,前單于宣及屯屠何皆愛其氣決,故數遣將兵出塞,掩擊北庭,還受賞賜,天子亦加殊異。是以國中盡敬師子,而不附安國。由是疾師子,欲殺之。其諸新降胡初在塞外,數為師子所驅掠,皆多怨之。安國因是委計降者,與同謀議。安國既立為單于,師子以次轉為左賢王,覺單于與新降者有謀,乃別居五原界。單于每龍會議事,師子輒稱病不往。皇甫棱知之,亦擁護不遣,單于懷憤益甚。

六年春,皇甫棱免,以執金吾朱徽行度遼將軍。時單于與中郎將杜崇不相平,迺上書告崇,崇諷西河太守令斷單于章,無由自聞。而崇因與朱徽上言:「南單于安國疏遠故胡,親近新降,欲殺左賢王師子及左臺且渠劉利等。又右部降者謀共迫脅安國,起兵背畔,請西河、上郡、安定為之儆備。」和帝下公卿議,皆以為「蠻夷反覆,雖難測知,然大兵聚會,必未敢動搖。今宜遣有方略使者之單于庭,與杜崇、朱徽及西河太守并力,觀其動靜。如無它變,可令崇等就安國會其左右大臣,責其部眾橫暴為邊害者,共平罪誅。若不從命,令為權時方略,事畢之後,裁行客賜,亦足以威示百蠻」。帝從之。於是徽、崇遂發兵造其庭。安國夜聞漢軍至,大驚,棄帳而去,因舉兵及將新降者欲誅師子。師子先知,乃悉將廬落入曼柏城。安國追到城下,門閉不得入。朱徽遣吏曉譬和之,安國不聽。城既不下,乃引兵屯五原。崇、徽因發諸郡騎追赴之急,眾皆大恐,安國舅骨都侯喜為等慮并被誅,乃格殺安國。

安國立一年,單于適之子師子立。

亭獨尸逐侯鞮單于師子,永元六年立。降胡五六百人夜襲師子,安集掾王恬將衛護士與戰,破之。於是新降胡遂相驚動,十五部二十餘萬人皆反畔,脅立前單于屯屠何子薁逊日逐王逢侯為單于,遂殺略吏人,燔燒郵亭廬帳,將車重向朔方,欲度漠北。於是遣行車騎將軍鄧鴻、越騎校尉馮柱、行度遼將軍朱徽將左右羽林、北軍五校士及郡國積射、緣邊兵,烏桓校尉任尚將烏桓、鮮卑,合四萬人討之。時南單于及中郎將杜崇屯牧師城,逢侯將萬餘騎攻圍之,未下。冬,鄧鴻等至美稷,逢侯乃乘冰度隘,向滿夷谷。南單于遣子將萬騎,及杜崇所領四千騎,與鄧鴻等追擊逢侯於大城塞,斬首三千餘級,得生口及降者萬餘人。馮柱復分兵追擊其別部,斬首四千餘級。任尚率鮮卑大都護蘇拔廆、烏桓大人勿柯八千騎,要擊逢侯於滿夷谷,復大破之。前後凡斬萬七千餘級。逢侯遂率眾出塞,漢兵不能追。七年正月,軍還。

馮柱將虎牙營留屯五原,罷遣鮮卑、烏桓、羌胡兵,封蘇拔廆為率眾王,又賜金帛。鄧鴻還京師,坐逗留失利,下獄死。後帝知朱徽、杜崇失胡和,又禁其上書,以致反畔,皆徵下獄死,以鴈門太守龐奮行度遼將軍。逢侯於塞外分為二部,自領右部屯涿邪山下,左部屯朔方西北,相去數百里。八年冬,左部胡自相疑畔,還入朔方塞,龐奮迎受慰納之。其勝兵四千人,弱小萬餘口悉降,以分處北邊諸郡。南單于以其右溫禺犢王烏居戰始與安國同謀,欲考問之。烏居戰將數千人遂復反畔,出塞外山谷閒,為吏民害。秋,龐奮、馮柱與諸郡兵擊烏居戰,其眾降,於是徙烏居戰眾及諸還降者二萬餘人於安定、北地。馮柱還,遷將作大匠。逢侯部眾飢窮,又為鮮卑所擊,無所歸,竄逃入塞者駱驛不絕。

單于師子立四年薨,單于長之子檀立。

萬氏尸逐鞮單于檀,永元十年立。十二年,龐奮遷河南尹,以朔方太守王彪行度遼將軍。南單于比歲遣兵擊逢侯,多所虜獲,收還生口前後以千數,逢侯轉困迫。十六年,北單于遣使詣闕貢獻,願和親,脩呼韓邪故約。和帝以其舊禮不備,未許之,而厚加賞賜,不荅其使。元興元年,重遣使詣敦煌貢獻,辭以國貧未能備禮,願請大使,當遣子入侍。時鄧太后臨朝,亦不荅其使,但加賜而已。

永初三年夏,漢人韓琮隨南單于入朝,既還,說南單于云:「關東水潦,人民飢餓死盡,可擊也。」單于信其言,遂起兵反畔,攻中郎將耿种於美稷。秋,王彪卒。冬,遣行車騎將軍何熙、副中郎龐雄擊之。四年春,檀遣千餘騎寇常山、中山,以西域校尉梁慬行度遼將軍,與遼東太守耿夔擊破之。事已具慬、夔傳。單于見諸軍並進,大恐怖,顧讓韓琮曰:「汝言漢人死盡,今是何等人也?」乃遣使乞降,許之。單于脫帽徒跣,對龐雄等拜陳,道死罪。於是赦之,遇待如初,乃還所鈔漢民男女及羌所略轉賣入匈奴中者合萬餘人。五年,梁慬免,以雲中太守耿夔行度遼將軍。

元初元年,夔免,以烏桓校尉鄧遵為度遼將軍。遵,皇太后之從弟,故始為真將軍焉。

四年,逢侯為鮮卑所破,部眾分散,皆歸北虜。五年春,逢侯將百餘騎亡還,詣朔方塞降,鄧遵奏徙逢侯於潁川郡。

建光元年,鄧遵免,復以耿夔代為度遼將軍。時鮮卑寇邊,夔與溫禺犢王呼尤徽將新降者連年出塞,討擊鮮卑。還,復各令屯列衝要。而耿夔徵發煩劇,新降者皆悉恨謀畔。

單于檀立二十七年薨,弟拔立。耿夔復免,以太原太守法度代為將軍。

烏稽侯尸逐鞮單于拔,延光三年立。夏,新降一部大人阿族等遂反畔,脅呼尤徽欲與俱去。呼尤徽曰:「我老矣,受漢家恩,寧死不能相隨!」眾欲殺之,有救者,得免。阿族等遂將妻子輜重亡去,中郎將馬翼遣兵與胡騎追擊,破之,斬首及自投河死者殆盡,獲馬牛羊萬餘頭。冬,法度卒。四年,漢陽太守傅眾代為將軍。其冬,傅眾復卒。永建元年,以遼東太守龐參代為將軍。

先是朔方以西障塞多不脩復,鮮卑因此數寇南部,殺漸將王。單于憂恐,上言求復障塞,順帝從之。乃遣黎陽營兵出屯中山北界,增置緣邊諸郡兵,列屯塞下,教習戰射。

單于拔立四年薨,弟休利立。

去特若尸逐就單于休利,永建三年立。四年,龐參遷大鴻臚,以東平相宋漢代為度遼將軍。陽嘉二年,漢遷太僕,以烏桓校尉耿曄代為度遼將軍。永和元年,曄病徵,以護羌校尉馬續代為度遼將軍。

五年夏,南匈奴左部句龍王吾斯、車紐等背畔,率三千餘騎寇西河,因復招誘右賢王,合七八千騎圍美稷,殺朔方、代郡長史。馬續與中郎將梁並、烏桓校尉王元發緣邊兵及烏桓、鮮卑、羌胡合二萬餘人,掩擊破之。吾斯等遂更屯聚,攻沒城邑。天子遣使責讓單于,開以恩義,令相招降。單于本不豫謀,乃脫帽避帳,詣並謝罪。並以病徵,五原太守陳龜代為中郎將。龜以單于本不能制下,逼迫之,單于及其弟左賢王皆自殺。單于休利立十三年。龜又欲徙單于近親於內郡,而降者遂更狐疑。龜坐下獄免。大將軍梁商以羌胡新反,黨眾初合,難以兵服,宜用招降,乃上表曰:「匈奴寇畔,自知罪極,窮鳥困獸,皆知救死,況種類繁熾,不可單盡。今轉運日增,三軍疲苦,虛內給外,非中國之利。竊見度遼將軍馬續素有謀謨,且典邊日久,深曉兵要,每得續書,與臣策合。宜令續深溝高壁,以恩信招降,宣示購賞,明其期約。如此,則醜類可服,國家無事矣。」帝從之,乃詔續招降畔虜。商又移書續等曰:「中國安寧,忘戰日久。良騎野合,交鋒接矢,決勝當時,戎狄之所長,而中國之所短也。強弩乘城,堅營固守,以待其衰,中國之所長也,而戎狄之所短也。宜務先所長,以觀其變,設購開賞,宣示反悔,勿貪小功,以亂大謀。」續及諸郡並各遵行。於是右賢王部抑鞮等萬三千口詣續降。

秋,句龍吾斯等立句龍王車紐為單于。東引烏桓,西收羌戎及諸胡等數萬人,攻破京兆虎牙營,殺上郡都尉及軍司馬,遂寇掠并、涼、幽、冀四州。乃徙西河治離石,上郡治夏陽,朔方治五原。冬,遣中郎將張耽將幽州烏桓諸郡營兵,擊畔虜車紐等,戰於馬邑,斬首三千級,獲生口及兵器牛羊甚眾。車紐等將諸豪帥骨都侯乞降,而吾斯猶率其部曲與烏桓寇鈔。六年春,馬續率鮮卑五千騎到穀城擊之,斬首數百級。張耽性勇銳,而善撫士卒,軍中皆為用命。遂繩索相懸,上通天山,大破烏桓,悉斬其渠帥,還得漢民,獲其畜生財物。夏,馬續復免,以城門校尉吳武代為將軍。

漢安元年秋,吾斯與薁鞮臺耆、且渠伯德等復掠并部。

呼蘭若尸逐就單于兜樓儲先在京師,漢安二年立之。天子臨軒,大鴻臚持節拜授璽綬,引上殿。賜青蓋駕駟、鼓車、安車、駙馬騎、玉具刀劍、什物,給綵布二千匹。賜單于閼氏以下金錦錯雜具,軿車馬二乘。遣行中郎將持節護送單于歸南庭。詔太常、大鴻臚與諸國侍子於廣陽城門外祖會,饗賜作樂,角抵百戲。順帝幸胡桃宮臨觀之。冬,中郎將馬寔募刺殺句龍吾斯,送首洛陽。建康元年,進擊餘黨,斬首千二百級。烏桓七十萬餘口皆詣寔降,車重牛羊不可勝數。

單于兜樓儲立五年薨。

伊陵尸逐就單于居車兒,建和元年立。至永壽元年,匈奴左薁鞮臺耆、且渠伯德等復畔,寇鈔美稷、安定,屬國都尉張奐擊破降之。事已具奐傳。

延熹元年,南單于諸部並畔,遂與烏桓、鮮卑寇緣邊九郡,以張奐為北中郎將討之,單于諸部悉降。奐以單于不能統理國事,乃拘之,上立左谷蠡王。桓帝詔曰:「春秋大居正,居車兒一心向化,何罪而黜!其遣還庭。」

單于居車兒立二十五年薨,子某立。

屠特若尸逐就單于某,熹平元年立。六年,單于與中郎將臧旻出鴈門擊鮮卑檀石槐,大敗而還。是歲,單于薨,子呼徵立。

單于呼徵,光和元年立。二年,中郎將張脩與單于不相能,脩擅斬之,更立右賢王羌渠為單于。脩以不先請而擅誅殺,檻車徵詣廷尉抵罪。

單于羌渠,光和二年立。中平四年,前中山太守張純反畔,遂率鮮卑寇邊郡。靈帝詔發南匈奴兵,配幽州牧劉虞討之。單于遣左賢王將騎詣幽州。國人恐單于發兵無已,五年,右部虾落與休著各胡白馬銅等十餘萬人反,攻殺單于。

單于羌渠立十年,子右賢王於扶羅立。

持至尸逐侯單于於扶羅,中平五年立。國人殺其父者遂畔。共立須卜骨都侯為單于,而於扶羅詣闕自訟。會靈帝崩,天下大亂,單于將數千騎與白波賊合兵寇河內諸郡。時民皆保聚,鈔掠無利,而兵遂挫傷。復欲歸國,國人不受,乃止河東。須卜骨都侯為單于一年而死,南庭遂虛其位,以老王行國事。

單于於扶羅立七年死,弟呼廚泉立。

單于呼廚泉,興平二年立。以兄被逐,不得歸國,數為鮮卑所鈔。建安元年,獻帝自長安東歸,右賢王去卑與白波賊帥韓暹等侍衛天子,拒擊李傕、郭汜。及車駕還洛陽,又徙遷許,然後歸國。二十一年,單于來朝,曹操因留於鄴,而遣去卑歸監其國焉。

論曰:漢初遭冒頓凶黠,種眾強熾。高祖威加四海,而窘平城之圍。太宗政鄰刑措,不雪憤辱之恥。逮孝武亟興邊略,有志匈奴,赫然命將,戎旗星屬,候列郊甸,火通甘泉,而猶鳴鏑揚塵,出入畿內,至於窮竭武力,單用天財,歷紀歲以攘之。寇雖頗折,而漢之疲耗略相當矣。宣帝值虜庭分爭,呼韓邪來臣,乃權納懷柔,因為邊衛,罷關徼之儆,息兵民之勞。龍駕帝服,鳴鍾傳鼓於清渭之上,南面而朝單于,朔、易無復匹馬之蹤,六十餘年矣。後王莽陵篡,擾動戎夷,續以更始之亂,方夏幅裂。自是匈奴得志,狼心復生,乘閒侵佚,害流傍境。及中興之初,更通舊好,報命連屬,金幣載道,而單于驕踞益橫,內暴滋深。世祖以用事諸華,未遑沙塞之外,忍愧思難,徒報謝而已。因徙幽、并之民,增邊屯之卒。及關東稍定,隴、蜀已清,其猛夫扞將,莫不頓足攘手,爭言衛、霍之事。帝方厭兵,閒脩文政,未之許也。其後匈奴爭立,日逐來奔,願脩呼韓之好,以禦北狄之衝,奉藩稱臣,永為外扞。天子總攬群策,和而納焉。乃詔有司開北鄙,擇肥美之地,量水草以處之。馳中郎之使,盡法度以臨之。制衣裳,備文物,加璽紱之綬,正單于之名。於是匈奴分破,始有南北二庭焉。讎釁既深,互伺便隙,控弦抗戈,覘望風塵,雲屯鳥散,更相馳突,至於陷潰創傷者,靡歲或寧,而漢之塞地晏然矣。後亦頗為出師,并兵窮討,命竇憲、耿夔之徒,前後並進,皆用果譎,設奇數,異道同會,究掩其窟穴,躡北追奔三千餘里,遂破龍祠,焚罽幕,阬十角,梏閼氏,銘功封石,倡呼而還。單于震懾屏氣,蒙氈遁走於烏孫之地,而漠北空矣。若因其時埶,及其虛曠,還南虜於陰山,歸河西於內地,上申光武權宜之略,下防戎羯亂華之變,使耿國之筭不謬於當世,袁安之議見從於後王,平易正直,若此其弘也。而竇憲矜三捷之效,忽經世之規,狼戾不端,專行威惠。遂復更立北虜,反其故庭,並恩兩護,以私己福,棄蔑天公,坐樹大鯁。永言前載,何恨憤之深乎!自後經綸失方,畔服不一,其為疢毒,胡可單言!降及後世,翫為常俗,終於吞噬神鄉,丘墟帝宅。嗚呼!千里之差,興自毫端,失得之源,百世不磨矣。

贊曰:匈奴既分,羽書稀聞。野心難悔,終亦紛紜。


\end{pinyinscope}