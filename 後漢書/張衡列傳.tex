\article{張衡列傳}

\begin{pinyinscope}
張衡字平子,南陽西鄂人也。世為著姓。祖父堪,蜀郡太守。衡少善屬文,游於三輔,因入京師,觀太學,遂通五經,貫六蓺。雖才高於世,而無驕尚之情。常從容淡靜,不好交接俗人。永元中,舉孝廉不行,連辟公府不就。時天下承平日久,自王侯以下,莫不踰侈。衡乃擬班固兩都,作二京賦,因以諷諫。精思傅會,十年乃成。文多故不載。大將軍鄧騭奇其才,累召不應。

衡善機巧,尤致思於天文、陰陽、歷筭。常耽好玄經,謂崔瑗曰:「吾觀太玄,方知子雲妙極道數,乃與五經相擬,非徒傳記之屬,使人難論陰陽之事,漢家得天下二百歲之書也。復二百歲,殆將終乎?所以作者之數,必顯一世,常然之符也。漢四百歲,玄其興矣。」安帝雅聞衡善術學,公車特徵拜郎中,再遷為太史令。遂乃研覈陰陽,妙盡琁機之正,作渾天儀,著靈憲、筭罔論,言甚詳明。

順帝初,再轉,復為太史令。衡不慕當世,所居之官,輒積年不徙。自去史職,五載復還,乃設客問,作應閒以見其志云:

有閒余者曰:蓋聞前哲首務,務於下學上達,佐國理民,有云為也。朝有所聞,則夕行之。立功立事,式昭德音。是故伊尹思使君為堯舜,而民處唐虞,彼豈虛言而已哉,必旌厥素爾。咎單、巫咸,寔守王家,申伯、樊仲,實幹周邦,服袞而朝,介圭作瑞。厥跡不朽,垂烈後昆,不亦丕歟!且學非以要利,而富貴萃之。貴以行令,富以施惠,惠施令行,故易稱以「大業」。質以文美,實由華興,器賴彫飾為好,人以輿服為榮。吾子性德體道,篤信安仁,約己博蓺,無堅不鑽,以思世路,斯何遠矣!曩滯日官。今又原之。雖老氏曲全,進道若退,然行亦以需。必也學非所用,術有所仰,故臨川將濟,而舟楫不存焉。徒經思天衢,內昭獨智,固合理民之式也?故嘗見謗于鄙儒。深厲淺揭,隨時為義,曾何貪於支離,而習其孤技邪?參輪可使自轉,木雕猶能獨飛,已垂翅而還故棲,盍亦調其機而銛諸?昔有文王,自求多福。人生在勤,不索何獲。曷若卑體屈己,美言以相剋?鳴于喬木,乃金聲而玉振之。用後勳,雪前吝,婞佷不柔,以意誰靳也。

應之曰:是何觀同而見異也?君子不患位之

不尊,而患德之不崇;不恥祿之不夥,而恥智之不博。是故蓺可學,而行可力也。天爵高懸,得之在命,或不速而自懷,或羨旃而不臻,求之無益,故智者面而不思。阽身以徼幸,固貪夫之所為,未得而豫喪也。枉尺直尋,議者譏之,盈欲虧志,孰云非羞?於心有猜,則簋飧饌餔猶不屑餐,旌瞀以之。意之無疑,則兼金盈百而不嫌辭,孟軻以之。士或解裋褐而襲黼黻,或委臿築而據文軒者,度德拜爵,量績受祿也。輸力致庸,受必有階。

渾元初基,靈軌未紀,吉凶紛錯,人用朣朦。黃帝為斯深慘。有風后者,是焉亮之,察三辰於上,跡禍福乎下,經緯歷數,然後天步有常,則風后之為也。當少昊清陽之末,實或亂德,人神雜擾,不可方物,重黎又相顓頊而申理之,日月即次,則重黎之為也。人各有能,因蓺授任,鳥師別名,四叔三正,官無二業,事不並濟。晝長則宵短,日南則景北。天且不堪兼,況以人該之。夫玄龍,迎夏則陵雲而奮鱗,樂時也;涉冬則淈泥而潛蟠,避害也。公旦道行,故制典禮以尹天下,懼教誨之不從,有人不理。仲尼不遇,故論六經以俟來辟,恥一物之不知,有事之無範。所考不齊,如何可一?

夫戰國交爭,戎車競驅,君若綴旒,人無所麗。燭武縣縋而秦伯退師,魯連係箭而聊城弛柝。從往則合,橫來則離,安危無常,要在說夫。咸以得人為梟,失士為尤。故樊噲披帷,入見高祖;高祖踞洗,以對酈生。當此之會,乃黿鳴而鱉應也。故能同心戮力,勤恤人隱,奄受區夏,遂定帝位,皆謀臣之由也。故一介之策,各有攸建,子長諜之,爛然有第。夫女魃北而應龍翔,洪鼎聲而軍容息;溽暑至而鶉火棲,寒冰沍而黿鼉蟄。今也,皇澤宣洽,海外混同,萬方億醜,并質共劑,若修成之不暇,尚何功之可立!立事有三,言為下列;下列且不可庶矣,奚冀其二哉!

于茲搢紳如雲,儒士成林,及津者風攄,失塗者幽僻,遭遇難要,趨偶為幸。世易俗異,事埶舛殊,不能通其變,而一度以揆之,斯契船而求劍,守株而伺兔也。冒愧逞願,必無仁以繼之,有道者所不履也。越王句踐事此,故厥緒不永。捷徑邪至,我不忍以投步;干進苟容,我不忍以歙肩。雖有犀舟勁楫,猶人涉卬否,有須者也。姑亦奉順敦篤,守以忠信,得之不休,不獲不吝。不見是而不惛,居下位而不憂,允上德之常服焉。方將師天老而友地典,與之乎高睨而大談,孔甲且不足慕,焉稱殷彭及周聃!與世殊技,固孤是求。子憂朱泙曼之無所用,吾恨輪扁之無所教也。子睹木雕獨飛,龟我垂翅故棲,吾感去杀附鴟,悲爾先笑而後號也。

斐豹以斃督燔書,禮至以掖國作銘;弦高以牛餼退敵,墨翟以縈帶全城;貫高以端辭顯義,蘇武以禿節效貞;蒱且以飛矰逞巧,詹何以沈鉤致精;弈秋以棋局取譽,王豹以清謳流聲。僕進不能參名於二立,退又不能群彼數子。愍三墳之既穨,惜八索之不理。庶前訓之可鑽,聊朝隱乎柱史。且韞櫝以待價,踵顏氏以行止。曾不慊夫晉、楚,敢告誠於知己。

陽嘉元年,復造候風地動儀。以精銅鑄成,員徑八尺,合蓋隆起,形似酒尊,飾以篆文山龜鳥獸之形。中有都柱,傍行八道,施關發機。外有八龍,首銜銅丸,下有蟾蜍,張口承之。其牙機巧制,皆隱在尊中,覆蓋周密無際。如有地動,尊則振龍機發吐丸,而蟾蜍銜之。振聲激揚,伺者因此覺知。雖一龍發機,而七首不動,尋其方面,乃知震之所在。驗之以事,合契若神。自書典所記,未之有也。嘗一龍機發而地不覺動,京師學者咸怪其無徵,後數日驛至,果地震隴西,於是皆服其妙。自此以後,乃令史官記地動所從方起。

時政事漸損,權移於下,衡因上疏陳事曰:「伏惟陛下宣哲克明,繼體承天,中遭傾覆,龍德泥蟠。今乘雲高躋,磐桓天位,誠所謂將隆大位,必先倥傯之也。親履艱難者知下情,備經險易者達物偽。故能一貫萬機,靡所疑惑,百揆允當,庶績咸熙。宜獲福祉神祇,受譽黎庶。而陰陽未和,災眚屢見,神明幽遠,冥鑒在茲。福仁禍淫,景響而應,因德降休,乘失致咎,天道雖遠,吉凶可見,近世鄭、蔡、江、樊、周廣、王聖,皆為效矣。故恭儉畏忌,必蒙祉祚,奢淫諂慢,鮮不夷戮,前事不忘,後事之師也。夫情勝其性,流遯忘反,豈唯不肖,中才皆然。苟非大賢,不能見得思義,故積惡成釁,罪不可解也。向使能瞻前顧後,援鏡自戒,則何陷於凶患乎!貴寵之臣,眾所屬仰,其有愆尤,上下知之。褒美譏惡,有心皆同,故怨讟溢乎四海,神明降其禍辟也。頃年雨常不足,思求所失,則洪範所謂『僭恆陽若』者也。懼群臣奢侈,昏踰典式,自下逼上,用速咎徵。又前年京師地震土裂,裂者威分,震者人擾也。君以靜唱,臣以動和,威自上出,不趣於下,禮之政也。竊懼聖思厭倦,制不專己,恩不忍割,與眾共威。威不可分,德不可共。洪範曰:『臣有作威作福玉食,害于而家,凶于而國。』天鑒孔明,雖疏不失,災異示人,前後數矣,而未見所革,以復往悔。自非聖人,不能無過。願陛下思惟所以稽古率舊,勿令刑德八柄,不由天子。若恩從上下,事依禮制,禮制脩則奢僭息,事合宜則無凶咎。然後神望允塞,災消不至矣。」

初,光武善讖,及顯宗、肅宗因祖述焉。自中興之後,儒者爭學圖緯,兼復附以訞言。衡以圖緯虛妄,非聖人之法,乃上疏曰:「臣聞聖人明審律歷以定吉凶,重之以卜筮,雜之以九宮,經天驗道,本盡於此。或觀星辰逆順,寒燠所由,或察龜策之占,巫覡之言,其所因者,非一術也。立言於前,有徵於後,故智者貴焉,謂之讖書。讖書始出,蓋知之者寡。自漢取秦,用兵力戰,功成業遂,可謂大事,當此之時,莫或稱讖。若夏侯勝、眭孟之徒,以道術立名,其所述著,無讖一言。劉向父子領校祕書,閱定九流,亦無讖錄。成、哀之後,乃始聞之。尚書堯使鯀理洪水,九載績用不成,鯀則殛死,禹乃嗣興。而春秋讖云『共工理水』。凡讖皆云黃帝伐蚩尤,而詩讖獨以為『蚩尤敗,然後堯受命』。春秋元命包中有公輸班與墨翟,事見戰國,非春秋時也。又言『別有益州』。益州之置,在於漢世。其名三輔諸陵,世數可知。至於圖中訖于成帝。一卷之書,互異數事,聖人之言,埶無若是,殆必虛偽之徒,以要世取資。往者侍中賈逵摘讖互異三十餘事,諸言讖者皆不能說。至於王莽篡位,漢世大禍,八十篇何為不戒?則知圖讖成於哀平之際也。且河洛、六蓺,篇錄已定,後人皮傅,無所容篡。永元中,清河宋景遂以歷紀推言水災,而偽稱洞視玉版。或者至於棄家業,入山林。後皆無效,而復采前世成事,以為證驗。至於永建復統,則不能知。此皆欺世罔俗,以昧埶位,情偽較然,莫之糾禁。且律歷、卦候、九宮、風角,數有徵效,世莫肯學,而競稱不占之書。譬猶畫工,惡圖犬馬而好作鬼魅,誠以實事難形,而虛偽不窮也。宜收藏圖讖,一禁絕之,則朱紫無所眩,典籍無瑕玷矣。」

後遷侍中,帝引在帷幄,諷議左右。嘗問衡天下所疾惡者。宦官懼其毀己,皆共目之,衡乃詭對而出。閹豎恐終為其患,遂共讒之。

衡常思圖身之事,以為吉凶倚伏,幽微難明,乃作思玄賦,以宣寄情志。其辭曰:

仰先哲之玄訓兮,雖彌高其弗違。匪仁里其焉宅兮,匪義跡其焉追?潛服膺以永靚兮,綿日月而不衰。伊中情之信脩兮,慕古人之貞節。竦余身而順止兮,遵繩墨而不跌。志團團以應懸兮,誠心固其如結。旌性行以制佩兮,佩夜光與瓊枝。驼幽蘭之秋華兮,又綴之以江蘺。美襞積以酷裂兮,允塵邈而難虧。既姱麗而鮮雙兮,非是時之攸珍。奮余榮而莫見兮,播余香而莫聞。幽獨守此仄陋兮,敢怠皇而舍勤。幸二八之遻虞兮,喜傅說之生殷;尚前良之遺風兮,恫後辰而無及。何孤行之煢煢兮,孑不群而介立?感鸞鷖之特棲兮,悲淑人之稀合。

彼無合其何傷兮,患眾偽之冒真。旦獲讟于群弟兮,啟金縢而乃信。覽蒸民之多僻兮,畏立辟以危身。曾煩毒以迷或兮,羌孰可與言己?私湛憂而深懷兮,思繽紛而不理。願竭力以守義兮,雖貧窮而不改。執雕虎而試象兮,阽焦原而跟止。庶斯奉以周旋兮,要既死而後已。俗遷渝而事化兮,泯規矩之圜方。珍蕭艾於重笥兮,謂蕙芷之不香。斥西施而弗御兮,羈要褭以服箱。行陂僻而獲志兮,循法度而離殃。惟天地之無窮兮,何遭遇之無常!不抑操而苟容兮,譬臨河而無航。欲巧笑以干媚兮,非余心之所嘗。襲溫恭之黻衣兮,披禮義之繡裳。辮貞亮以為鞶兮,雜技蓺以為珩。昭綵藻與雕琢兮,璜聲遠而彌長。淹棲遲以恣欲兮,燿靈忽其西藏。恃己知而華予兮,鶗荩鳴而不芳。冀一年之三秀兮,遒白露之為霜。時亹亹而代序兮,疇可與乎比伉?咨妒嫮之難並兮,想依韓以流亡,恐漸冉而無成兮,留則蔽而不章。

心猶與而狐疑兮,即岐阯而攄情。文君為我端蓍兮,利飛遁以保名。歷眾山以周流兮,翼迅風以揚聲。二女感於崇岳兮,或冰折而不營。天蓋高而為澤兮,誰云路之不平!阶自強而不息兮,蹈玉階之嶢崢。懼筮氏之長短兮,鑽東龜以觀禎。遇九皋之介鳥兮,怨素意之不逞。遊塵外而瞥天兮,據冥翳而哀鳴。鵰鶚競於貪婪兮,我脩絜以益榮。子有故於玄鳥兮,歸母氏而後寧。

占既吉而無悔兮,簡元辰而俶裝。旦余沐於清原兮,晞余髮於朝陽。漱飛泉之瀝液兮,咀石菌之流英。翾鳥舉而魚躍兮,將往走乎八荒。過少皞之窮野兮,問三丘乎句芒。何道真之淳粹兮,去穢累而票輕。登蓬萊而容與兮,鼇雖抃而不傾。留瀛洲而採芝兮,聊且以乎長生。憑歸雲而遐逝兮,夕余宿乎扶桑。唿青岑之玉醴兮,餐沆瀣以為糧。發昔夢於木禾兮,穀崑崙之高岡。朝吾行於湯谷兮,從伯禹於稽山。集群神之執玉兮,疾防風之食言。

指長沙以邪徑兮,存重華乎南鄰。哀二妃之未從兮,翩儐處彼湘瀕。流目覜夫衡阿兮,睹有黎之圮墳;痛火正之無懷兮,託山陂以孤魂。愁蔚蔚以慕遠兮,越卬州而愉敖。躋日中于昆吾兮,憩炎天之所陶。揚芒熛而絳天兮,水泫沄而涌濤。溫風翕其增熱兮,惄鬱邑其難聊。顝羈旅而無友兮,余安能乎留茲?

顧金天而歎息兮,吾欲往乎西嬉。前祝融使舉麾兮,纚朱鳥以承旗。躔建木於廣都兮,拓若華而躊躇。超軒轅於西海兮,跨汪氏之龍魚;聞此國之千歲兮,曾焉足以娛余?

思九土之殊風兮,從蓐收而遂徂。欻神化而蟬蛻兮,朋精粹而為徒。蹶白門而東馳兮,云台行乎中野。亂弱水之潺湲兮,逗華陰之湍渚。號馮夷俾清津兮,櫂龍舟以濟予。會帝軒之未歸兮,悵相佯而延佇。呬河林之蓁蓁兮,偉關睢之戒女。黃靈詹而訪命兮,摎天道其焉如。曰近信而遠疑兮,六籍闕而不書。神逵昧其難覆兮,疇克謨而從諸?牛哀病而成虎兮,雖逢昆其必噬。鱉令殪而尸亡兮,取蜀禪而引世。死生錯而不齊兮,雖司命其不晰。竇號行於代路兮,後膺祚而繁廡。王肆侈於漢庭兮,卒銜恤而絕緒。尉尨眉而郎潛兮,逮三葉而遘武。董弱冠而司袞兮,設王隧而弗處。夫吉凶之相仍兮,恆反側而靡所。穆負天以悅牛兮,豎亂叔而幽主。文斷袪而忌伯兮,閹謁賊而寧后。通人闇於好惡兮,豈愛惑之能剖?嬴擿讖而戒胡兮,備諸外而發內。或輦賄而違車兮,孕行產而為對。慎灶顯於言天兮,占水火而妄誶。梁叟患夫黎丘兮,丁厥子而事刃,親所睇而弗識兮,矧幽冥之可信。毋綿攣以涬己兮,思百憂以自疢。彼天監之孔明兮,用棐忱而佑仁。湯蠲體以禱祈兮,蒙厖禠以拯人。景三慮以營國兮,熒惑次於它辰。魏顆亮以從理兮,鬼亢回以敝秦。咎繇邁而種德兮,德樹茂乎英、六。桑末寄夫根生兮,卉既彫而已毓。有無言而不讎兮,又何往而不復?盍遠跡以飛聲兮,孰謂時之可蓄?

仰矯首以遙望兮,魂馁惘而無疇。偪區中之隘陋兮,將北度而宣遊。行積冰之磑磑兮,清泉沍而不流。寒風淒而永至兮,拂穹岫之騷騷。玄武縮於殼中兮,螣蛇蜿而自糾。魚矜鱗而并凌兮,鳥登木而失條。坐太陰之屏室兮,慨含欷而增愁。怨高陽之相寓兮,锢顓頊之宅幽。庸織絡於四裔兮,斯與彼其何瘳?望寒門之絕垠兮,縱余惞乎不周。迅飆潚其媵我兮,騖翩嗁而不禁。趨谽饿之洞穴兮,摽通淵之碄碄。經重陰乎寂寞兮,愍墳羊之潛深。

追慌忽於地底兮,軼無形而上浮。出右密之闇野兮,不識蹊之所由。速燭龍令執炬兮,過鍾山而中休。瞰瑤谿之赤岸兮,弔祖江之見劉。聘王母於銀臺兮,羞玉芝以療飢;戴勝憖其既歡兮,又誚余之行遲。載太華之玉女兮,召洛浦之宓妃。咸姣麗以蠱媚兮,增嫮眼而蛾眉。舒妙婧之纖腰兮,揚雜錯之褂徽。離朱脣而微笑兮,顏的养以遺光。獻環琨與璵縭兮,申厥好以玄黃。雖色豔而賂美兮,志浩盪而不嘉。雙材悲於不納兮,並詠詩而清歌。歌曰:天地煙熅,百卉含蘤。鳴鶴交頸,雎鳩相和。處子懷春,精魂回移。如何淑明,忘我實多。

將荅賦而不暇兮,爰整駕而亟行。瞻崑崙之巍巍兮,臨縈河之洋洋。伏靈龜以負坻兮,亙螭龍之飛梁。登閬風之曾城兮,搆不死而為床。屑瑤繠以為彁兮,巩白水以為漿。抨巫咸以占夢兮,迺貞吉之元符。滋令德於正中兮,合嘉秀以為敷。既垂穎而顧本兮,爾要思乎故居。安和靜而隨時兮,姑純懿之所廬。

戒庶寮以夙會兮,僉恭職而並迓。豐隆軯其震霆兮,列缺曄其照夜。雲師阅以交集兮,涷雨沛其灑塗。轙琱輿而樹葩兮,擾應龍以服輅。百神森其備從兮,屯騎羅而星布。振余袂而就車兮,脩劍揭以低昂。冠咢咢其映蓋兮,佩綝纚以煇煌。僕夫儼其正策兮,八乘攄而超驤。氛旄溶以天旋兮,蜺旌飄而飛揚。撫軨軹而還睨兮,心灼藥其如湯。羨上都之赫戲兮,何迷故而不忘?左青琱以揵芝兮,右素威以司鉦。前長離使拂羽兮,委水衡乎玄冥。屬箕伯以函風兮,徵淟涊而為清。曳雲旗之離離兮,鳴玉鸞之譻譻。涉清霄而升遐兮,浮蔑蒙而上征。紛翼翼以徐戾兮,焱回回其揚靈。叫帝閽使闢扉兮,覿天皇于瓊宮。聆廣樂之九奏兮,展洩洩以肜肜。考理亂於律鈞兮,意建始而思終。惟盤逸之無斁兮,懼樂往而哀來。素撫弦而餘音兮,大容吟曰念哉。既防溢而靜志兮,迨我暇以翱翔。出紫宮之肅肅兮,集大微之閬閬。命王良掌策駟兮,踰高閣之鏘鏘。建罔車之幕幕兮,獵青林之芒芒。彎威弧之撥剌兮,射嶓冢之封狼。觀壁壘於北落兮,伐河鼓之磅硠。乘天潢之汎汎兮,浮雲漢之湯湯。倚招搖、攝提以低回闾流兮,察二紀、五緯之綢繆遹皇。偃蹇夭矯彧以連卷兮,雜沓叢侦颯以方驤。戫汨飂戾沛以罔象兮,爛漫麗靡摆以迭逿。凌驚雷之锂磕兮,弄狂電之淫裔。踰庬澒於宕冥兮,貫倒景而高厲。廓盪盪其無涯兮,乃今窮乎天外。

據開陽而頫盼兮,臨舊鄉之暗藹。悲離居之勞心兮,情悁悁而思歸。魂眷眷而屢顧兮,馬倚輈而俳回。雖遨游以媮樂兮,豈愁慕之可懷。出閶闔兮降天塗,乘飆忽兮馳虛無。雲霏霏兮繞余輪,風眇眇兮震余旟。繽聯翩兮紛暗曖,倏眩眃兮反常閭。

收疇昔之逸豫兮,卷淫放之遐心。脩初服之娑娑兮,長余珮之參參。文章煥以粲爛兮,美紛紜以從風。御六蓺之珍駕兮,遊道德之平林。結典籍而為罟兮,歐儒、墨而為禽。玩陰陽之變化兮,詠雅、頌之徽音。嘉曾氏之歸耕兮,慕歷陵之欽崟。共夙昔而不貳兮,居終始之所服也;夕惕若厲以省伥兮,懼余身之未敕也。苟中情之端直兮,莫吾知而不恧。墨無為以凝志兮,與仁義乎消搖。不出戶而知天下兮,何必歷遠以劬勞?

系曰:天長地久歲不留,俟河之清祗懷憂。願得遠度以自娛,上下無常窮六區。超踰騰躍絕世俗,飄颻神舉逞所欲。天不可階仙夫希,柏舟悄悄吝不飛。松、喬高跱孰能離?結精遠遊使心攜。回志朅來從玄諆,獲我所求夫何思!

永和初,出為河閒相。時國王驕奢,不遵典憲;又多豪右,共為不軌。衡下車,治威嚴,整法度,陰知姦黨名姓,一時收禽,上下肅然,稱為政理。視事三年,上書乞骸骨,徵拜尚書。年六十二,永和四年卒。

著周官訓詁,崔瑗以為不能有異於諸儒也。又欲繼孔子易說彖、象殘缺者,竟不能就。所著詩、賦、銘、七言、靈憲、應閒、七辯、巡誥、懸圖凡三十二篇。

永初中,謁者僕射劉珍、校書郎劉騊駼等著作東觀,撰集漢記,因定漢家禮儀,上言請衡參論其事,會並卒,而衡常歎息,欲終成之。及為侍中,上疏請得專事東觀,收撿遺文,畢力補綴。又條上司馬遷、班固所敘與典籍不合者十餘事。又以為王莽本傳但應載篡事而已,至於編年月,紀災祥,宜為元后本紀。又更始居位,人無異望,光武初為其將,然後即真,宜以更始之號建於光武之初。書數上,竟不聽。及後之著述,多不詳典,時人追恨之。

論曰:崔瑗之稱平子曰「數術窮天地,制作侔造化」。斯致可得而言歟!推其圍範兩儀,天地無所蘊其靈;運情機物,有生不能參其智。故智思引淵微,人之上術。記曰:「德成而上,蓺成而下。」量斯思也,豈夫蓺而已哉?何德之損乎!

贊曰:三才理通,人靈多蔽。近推形筭,遠抽深滯。不有玄慮,孰能昭织?


\end{pinyinscope}