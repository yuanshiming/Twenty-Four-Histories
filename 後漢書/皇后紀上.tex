\article{皇后紀上}

\begin{pinyinscope}
夏、殷以上,后妃之制,其文略矣。周禮王者立后,三夫人,九嬪,二十七世婦,八十一女御,以備內職焉。后正位宮闈,同體天王。夫人坐論婦禮,九嬪掌教四德,世婦主喪、祭、賓客,女御序于王之燕寢。頒官分務,各有典司。女史彤管,記功書過。居有保阿之訓,動有環佩之響。進賢才以輔佐君子,哀窈窕而不淫其色。所以能述宣陰化,修成內則,閨房肅雍,險謁不行也。故康王晚朝,關雎作諷;宣后晏起,姜氏請愆。及周室東遷,禮序凋缺。諸侯僭縱,軌制無章。齊桓有如夫人者六人,晉獻升戎女為元妃,終於五子作亂,冢嗣遘屯。爰逮戰國,風憲逾薄,適情任欲,顛倒衣裳,以至破國亡身,不可勝數。斯固輕禮弛防,先色後德者也。

秦并天下,多自驕大,宮備七國,爵列八品。漢興,因循其號,而婦制莫釐。高祖帷薄不修,孝文衽席無辯。然而選納尚簡,飾翫少華。自武、元之後,世增淫費,至乃掖庭三千,增級十四。妖倖毀政之符,外姻亂邦之跡,前史載之詳矣。

及光武中興,斲彫為朴,六宮稱號,唯皇后、貴人。貴人金印紫綬,奉不過粟數十斛。又置美人、宮人、采女三等,並無爵秩,歲時賞賜充給而已。漢法常因八月筭人,遣中大夫與掖庭丞及相工,於洛陽鄉中閱視良家童女,年十三以上,二十已下,姿色端麗,合法相者,載還後宮,擇視可否,乃用登御。所以明慎聘納,詳求淑哲。明帝聿遵先旨,宮教頗修,登建嬪后,必先令德,內無出閫之言,權無私溺之授,可謂矯其敝矣。向使因設外戚之禁,編著甲令,改正后妃之制,貽厥方來,豈不休哉!雖御己有度,而防閒未篤,故孝章以下,漸用色授,恩隆好合,遂忘淄蠹。

自古雖主幼時艱,王家多釁,必委成冢宰,簡求忠賢,未有專任婦人,斷割重器。唯秦羋太后始攝政事,故穰侯權重於昭王,家富於嬴國。漢仍其謬,知患莫改。東京皇統屢絕,權歸女主,外立者四帝,臨朝者六后,莫不定策帷帟,委事父兄,貪孩童以久其政,抑明賢以專其威。任重道悠,利深禍速。身犯霧露於雲臺之上,家嬰縲絏於圄犴之下。湮滅連踵,傾輈繼路。而赴蹈不息,燋爛為期,終於陵夷大運,淪亡神寶。詩書所歎,略同一揆。故考列行跡,以為皇后本紀。雖成敗事異,而同居正號者,並列于篇。其以私恩追尊,非當時所奉者,則隨它事附出。親屬別事,各依列傳。其餘無所見,則係之此紀,以纘西京外戚云爾。

光武郭皇后諱聖通,真定槁人也。為郡著姓。父昌,讓田宅財產數百萬與異母弟,國人義之。仕郡功曹。娶真定恭王女,號郭主,生后及子況。昌早卒。郭主雖王家女,而好禮節儉,有母儀之德。更始二年春,光武擊王郎,至真定,因納后,有寵。及即位,以為貴人。

建武元年,生皇子彊。帝善況小心謹慎,年始十六,拜黃門侍郎。二年,貴人立為皇后,彊為皇太子,封況綿蠻侯。以后弟貴重,賓客輻湊。況恭謙下士,頗得聲譽。十四年,遷城門校尉。其後,后以寵稍衰,數懷怨懟。十七年,遂廢為中山王太后,進后中子右翊公輔為中山王,以常山郡益中山國。徙封況大國,為陽安侯。后從兄竟,以騎都尉從征伐有功,封為新郪侯,官至東海相。竟弟匡為發干侯,官至太中大夫。后叔父梁,早終,無子。其婿南陽陳茂,以恩澤封南讀侯。

二十年,中山王輔復徙封沛王,后為沛太后。況遷大鴻臚。帝數幸其第,會公卿諸侯親家飲燕,賞賜金錢縑帛,豐盛莫比,京師號況家為金穴。二十六年,后母郭主薨,帝親臨喪送葬,百官大會,遣使者迎昌喪柩,與主合葬,追贈昌陽安侯印綬,謚曰思侯。二十八年,后薨,葬于北芒。

帝憐郭氏,詔況子璜尚淯陽公主,除璜為郎。顯宗即位,況與帝舅陰識、陰就並為特進,數授賞賜,恩寵俱渥。禮待陰、郭,每事必均。永平二年,況卒,贈賜甚厚,帝親自臨喪,謚曰節侯,子璜嗣。

元和三年,肅宗北巡狩,過真定,會諸郭,朝見上壽,引入倡飲甚歡。以太牢具上郭主冢,賜粟萬斛,錢五十萬。永元初,璜為長樂少府,子舉為侍中,兼射聲校尉。及大將軍竇憲被誅,舉以憲女婿謀逆,故父子俱下獄死,家屬徙合浦,宗族為郎吏者,悉免官。新郪侯竟初為騎將,從征伐有功,拜東海相。永平中卒,子嵩嗣;嵩卒,追坐染楚王英事,國廢。建初二年,章帝紹封嵩子勤為伊亭侯,勤無子,國除。發干侯匡,官至太中大夫,建武三十年卒,子勳嗣;勳卒,子駿嗣,永平十三年,亦坐楚王英事,失國。建初三年,復封駿為觀都侯,卒,無子,國除。郭氏侯者凡三人,皆絕國。

論曰:物之興衰,情之起伏,理有固然矣。而崇替去來之甚者,必唯寵惑乎?當其接床笫,承恩色,雖險情贅行,莫不德焉。及至移意愛,析嬿私,雖惠心妍狀,愈獻醜焉。愛升,則天下不足容其高;歡隊,故九服無所逃其命。斯誠志士之所沈溺,君人之所抑揚,未或違之者也。郭后以衰離見貶,恚怨成尤,而猶恩加別館,增寵黨戚。至乎東海逡巡,去就以禮,使後世不見隆薄進退之隙,不亦光於古乎!

光烈陰皇后諱麗華,南陽新野人。初,光武適新野,聞后美,心悅之。後至長安,見執金吾車騎甚盛,因歎曰:「仕宦當作執金吾,娶妻當得陰麗華。」更始元年六月,遂納后於宛當成里,時年十九。及光武為司隸校尉,方西之洛陽,令后歸新野。及鄧奉起兵,后兄識為之將,后隨家屬徙淯陽,止於奉舍。

光武即位,令侍中傅俊迎后,與胡陽、寧平主諸宮人俱到洛陽,以后為貴人。帝以后雅性寬仁,欲崇以尊位,后固辭,以郭氏有子,終不肯當,故遂立郭皇后。建武四年,從征彭寵,生顯宗於元氏。九年,有盜劫殺后母鄧氏及弟訢,帝甚傷之,乃詔大司空曰:「吾微賤之時,娶於陰氏,因將兵征伐,遂各別離。幸得安全,俱脫虎口。以貴人有母儀之美,宜立為后,而固辭弗敢當,列於媵妾。朕嘉其義讓,許封諸弟。未及爵土,而遭患逢禍,母子同命,愍傷于懷。小雅曰:『將恐將懼,惟予與汝。將安將樂,汝轉棄予。』風人之戒,可不慎乎?其追爵謚貴人父陸為宣恩哀侯,弟訢為宣義恭侯,以弟就嗣哀侯後。及尸柩在堂,使太中大夫拜授印綬,如在國列侯禮。魂而有靈,嘉其寵榮!」

十七年,廢皇后郭氏而立貴人。制詔三公曰:「皇后懷執怨懟,數違教令,不能撫循它子,訓長異室。宮闈之內,若見鷹鸇。既無關雎之德,而有呂、霍之風,豈可託以幼孤,恭承明祀。今遣大司徒涉、宗正吉持節,其上皇后璽綬。陰貴人鄉里良家,歸自微賤。『自我不見,于今三年。』宜奉宗廟,為天下母。主者詳案舊典,時上尊號。異常之事,非國休福,不得上壽稱慶。」后在位恭儉,少嗜玩,不喜笑謔。性仁孝,多矜慈。七歲失父,雖已數十年,言及未曾不流涕。帝見,常歎息。

顯宗即位,尊后為皇太后。永平三年冬,帝從太后幸章陵,置酒舊宅,會陰、鄧故人諸家子孫,並受賞賜。七年,崩,在位二十四年,年六十,合葬原陵。

明帝性孝愛。追慕無已。十七年正月,當謁原陵,夜夢先帝、太后如平生歡。既寤,悲不能寐,即案歷,明旦日吉,遂率百官及故客上陵。其日,降甘露於陵樹,帝令百官采取以薦。會畢,帝從席前伏御床,視太后鏡奩中物,感動悲涕,令易脂澤裝具。左右皆泣,莫能仰視焉。

明德馬皇后諱某,伏波將軍援之小女也。少喪父母。兄客卿敏惠早夭,母藺夫人悲傷發疾慌惚。后時年十歲,幹理家事,敕制僮御,內外諮稟,事同成人。初,諸家莫知者,後聞之,咸歎異焉。后嘗久疾,太夫人令筮之,筮者曰:「此女雖有患狀而當大貴,兆不可言也。」後又呼相者使占諸女,見后,大驚曰:「我必為此女稱臣。然貴而少子,若養它子者得力,乃當踰於所生。」

初,援征五溪蠻,卒於師,虎賁中郎將梁松、黃門侍郎竇固等因譖之,由是家益失埶,又數為權貴所侵侮。后從兄嚴不勝憂憤,白太夫人絕竇氏婚,求進女掖庭。乃上書曰:「臣叔父援孤恩不報,而妻子特獲恩全,戴仰陛下,為天為父。人情既得不死,便欲求福。竊聞太子、諸王妃匹未備,援有三女,大者十五,次者十四,小者十三,儀狀髮膚,上中以上。皆孝順小心,婉靜有禮。願下相工,簡其可否。如有萬一,援不朽於黃泉矣。又援姑姊妹並為成帝婕妤。葬於延陵。臣嚴幸得蒙恩更生,冀因緣先姑,當充後宮。」由是選后入太子宮。時年十三。奉承陰后,傍接同列,禮則脩備,上下安之。遂見寵異,常居後堂。

顯宗即位,以后為貴人。時后前母姊女賈氏亦以選入,生肅宗。帝以后無子,命令養之。謂曰:「人未必當自生子,但患愛養不至耳。」后於是盡心撫育,勞悴過於所生。肅宗亦孝性淳篤,恩性天至,母子慈愛,始終無纖介之閒。后常以皇嗣未廣,每懷憂歎,薦達左右,若恐不及。後宮有進見者,每加慰納。若數所寵引,輒增隆遇。永平三年春,有司奏立長秋宮,帝未有所言。皇太后曰:「馬貴人德冠後宮,即其人也。」遂立為皇后。

先是數日,夢有小飛蟲無數赴著身,又入皮膚中而復飛出。既正位宮闈,愈自謙肅。身長七尺二寸,方口,美髮。能誦易,好讀春秋、楚辭,尤善周官、董仲舒書。常衣大練,裙不加緣。朔望諸姬主朝請,望見后袍衣疏麤,反以為綺縠,就視,乃笑。后辭曰:「此繒特宜染色,故用之耳。」六宮莫不歎息。帝嘗幸苑囿離宮,后輒以風邪露霧為戒,辭意款備,多見詳擇。帝幸濯龍中,並召諸才人,下邳王已下皆在側,請呼皇后。帝笑曰:「是家志不好樂,雖來無歡。」是以遊娛之事希嘗從焉。

十五年,帝案地圖,將封皇子,悉半諸國。后見而言曰:「諸子裁食數縣,於制不已儉乎?」帝曰:「我子豈宜與先帝子等乎?歲給二千萬足矣。」時楚獄連年不斷,囚相證引,坐繫者甚眾。后慮其多濫,乘閒言及,惻然。帝感悟之,夜起仿偟,為思所納,卒多有所降宥。時諸將奏事及公卿較議難平者,帝數以試后。后輒分解趣理,各得其情。每於侍執之際,輒言及政事,多所毗補,而未嘗以家私干。欲寵敬日隆,始終無衰。

及帝崩,肅宗即位,尊后曰皇太后。諸貴人當徙居南宮,太后感析別之懷,各賜王赤綬,加安車駟馬,白越三千端,雜帛二千匹,黃金十斤。自撰顯宗起居注,削去兄防參醫藥事。帝請曰:「黃門舅旦夕供養且一年,即無褒異,又不錄勤勞,無乃過乎!」太后曰:「吾不欲令後世聞先帝數親後宮之家,故不著也。」

建初元年,欲封爵諸舅,太后不聽。明年夏,大旱,言事者以為不封外戚之故,有司因此上奏,宜依舊典。太后詔曰:「凡言事者皆欲媚朕以要福耳。昔王氏五侯同日俱封,其時黃霧四塞,不聞澍雨之應。又田蚡、竇嬰,寵貴橫恣,傾覆之禍,為世所傳。故先帝防慎舅氏,不令在樞機之位。諸子之封,裁令半楚、淮陽諸國,常謂『我子不當與先帝子等』。今有司柰何欲以馬氏比陰氏乎!吾為天下母,而身服大練,食不求甘,左右但著帛布,無香薰之飾者,欲身率下也。以為外親見之,當傷心自敕,但笑言太后素好儉。前過濯龍門上,見外家問起居者,車如流水,馬如游龍,倉頭衣綠恳,領袖正白,顧視御者,不及遠矣。故不加譴怒,但絕歲用而已,冀以默愧其心,而猶懈怠,無憂國忘家之慮。知臣莫若君,況親屬乎?吾豈可上負先帝之旨,下虧先人之德,重襲西京敗亡之禍哉!」固不許。

帝省詔悲歎,復重請曰:「漢興,舅氏之封侯,猶皇子之為王也。太后誠存謙虛,柰何令臣獨不加恩三舅乎?且衛尉年尊,兩校尉有大病,如令不諱,使臣長抱刻骨之恨。宜及吉時,不可稽留。」

太后報曰:「吾反覆念之,思令兩善。豈徒欲獲謙讓之名,而使帝受不外施之嫌哉!昔竇太后欲封王皇后之兄,丞相條侯言受高祖約,無軍功,非劉氏不侯。今馬氏無功於國,豈得與陰、郭中興之后等邪?常觀富貴之家,祿位重疊,猶再實之木,其根必傷。且人所以願封侯者,欲上奉祭祀,下求溫飽耳。今祭祀則受四方之珍,衣食則蒙御府餘資,斯豈不足,而必當得一縣乎?吾計之孰矣,勿有疑也。夫至孝之行,安親為上。今數遭變異,穀價數倍,憂惶晝夜,不安坐臥,而欲先營外封,違慈母之拳拳乎!吾素剛急,有匈中氣,不可不順也。若陰陽調和,邊境清靜,然後行子之志。吾但當含飴弄孫,不能復關政矣。」

時新平主家御者失火,延及北閣後殿。太后以為己過,起居不歡。時當謁原陵,自引守備不慎,慚見陵園,遂不行。初,太夫人葬,起墳微高,太后以為言,兄廖等即時減削。其外親有謙素義行者,輒假借溫言,賞以財位。如有纖介,則先見嚴恪之色,然後加譴。其美車服不軌法度者,便絕屬籍,遣歸田里。廣平、鉅鹿、樂成王車騎朴素,無金銀之飾,帝以白太后,太后即賜錢各五百萬。於是內外從化,被服如一,諸家惶恐,倍於永平時。乃置織室,蠶於濯龍中,數往觀視,以為娛樂。常與帝旦夕言道政事,及教授諸小王,論議經書,述敘平生,雍和終日。

四年,天下豐稔,方垂無事,帝遂封三舅廖、防、光為列侯。並辭讓,願就關內侯。太后聞之,曰:「聖人設教,各有其方,知人情性莫能齊也。吾少壯時,但慕竹帛,志不顧命。今雖已老,而復『戒之在得』,故日夜惕厲,思自降損。居不求安,食不念飽。冀乘此道,不負先帝。所以化導兄弟,共同斯志,欲令瞑目之日,無所復恨。何意老志復不從哉?萬年之日長恨矣!」廖等不得已,受封爵而退位歸第焉。

太后其年寢疾,不信巫祝小醫,數敕絕禱祀。至六月,崩。在位二十三年,年四十餘。合葬顯節陵。

賈貴人,南陽人。建武末選入太子宮,中元二年生肅宗,而顯宗以為貴人。帝既為太后所養,專以馬氏為外家,故貴人不登極位,賈氏親族無受寵榮者。及太后崩,乃策書加貴人王赤綬,安車一駟,永巷宮人二百,御府雜帛二萬匹,大司農黃金千斤,錢二千萬。諸史並闕後事,故不知所終。

章德竇皇后諱某,扶風平陵人,大司徒融之曾孫也。祖穆,父勳,坐事死,事在竇融傳。勳尚東海恭王彊女沘陽公主,后其長女也。家既廢壞,數呼相工問息耗,見后者皆言當大尊貴,非臣妾容貌。年六歲能書,親家皆奇之。建初二年,后與女弟俱以選例入見長樂宮,進止有序,風容甚盛。肅宗先聞后有才色,數以訊諸姬傅。及見,雅以為美,馬太后亦異焉,因入掖庭,見於北宮章德殿。后性敏給,傾心承接,稱譽日聞。明年,遂立為皇后,妹為貴人。七年,追爵謚后父勳為安成思侯。后寵幸殊特,專固後宮。

初,宋貴人生皇太子慶,梁貴人生和帝。后既無子,並疾忌之,數閒於帝,漸致疏嫌。因誣宋貴人挾邪媚道,遂自殺,廢慶為清河王,語在慶傳。

梁貴人者,褒親愍侯梁竦之女也。少失母,為伯母舞陰長公主所養。年十六,亦以建初二年與中姊俱選入掖庭為貴人。四年,生和帝。后養為己子。欲專名外家而忌梁氏。八年,乃作飛書以陷竦,竦坐誅,貴人姊妹以憂卒。自是宮房惵息,后愛日隆。

及帝崩,和帝即位,尊后為皇太后。皇太后臨朝,尊母沘陽公主為長公主,益湯沐邑三千戶,兄憲,弟篤、景,並顯貴,擅威權,後遂密謀不軌,永元四年,發覺被誅。

九年,太后崩,未及葬,而梁貴人姊确上書陳貴人枉歿之狀。太尉張酺、司徒劉方、司空張奮上奏,依光武黜呂太后故事,貶太后尊號,不宜合葬先帝。百官亦多上言者。帝手詔曰:「竇氏雖不遵法度,而太后常自減損。朕奉事十年,深惟大義,禮,臣子無貶尊上之文。恩不忍離,義不忍虧。案前世上官太后亦無降黜,其勿復議。」於是合葬敬陵。在位十八年。

帝以貴人酷歿,斂葬禮闕,乃改殯於承光宮,上尊謚曰恭懷皇后,追服喪制,百官縞素,與姊大貴人俱葬西陵,儀比敬園。

和帝陰皇后諱某,光烈皇后兄執金吾識之曾孫也。后少聰慧,善書蓺。永元四年,選入掖庭,以先后近屬,故得為貴人。有殊寵。八年,遂立為皇后。

自和熹鄧后入宮,愛寵稍衰,數有恚恨。后外祖母鄧朱出入宮掖。十四年夏,有言后與朱共挾巫蠱道,事發覺,帝遂使中常侍張慎與尚書陳褒於掖庭獄雜考案之。朱及二子奉、毅與后弟軼、輔、敞辭語相連及,以為祠祭祝詛,大逆無道。奉、毅、輔考死獄中。帝使司徒魯恭持節賜后策,上璽綬,遷于桐宮,以憂死。立七年,葬臨平亭部。父特進綱自殺,軼、敞及朱家屬徙日南比景縣,宗親外內昆弟皆免官還田里。永初四年,鄧太后詔赦陰氏諸徙者悉歸故郡,還其資財五百餘萬。

和熹鄧皇后諱綏,太傅禹之孫也。父訓,護羌校尉;母陰氏,光烈皇后從弟女也。后年五歲,太傅夫人愛之,自為翦髮。夫人年高目冥,誤傷后額,忍痛不言。左右見者怪而問之,后曰:「非不痛也,太夫人哀憐為斷髮,難傷老人意。故忍之耳。」六歲能史書,十二通詩、論語。諸兄每讀經傳,輒下意難問。志在典籍,不問居家之事。母常非之,曰:「汝不習女工以供衣服,乃更務學,寧當舉博士邪?」后重違母言,晝修婦業,暮誦經典,家人號曰「諸生」。父訓異之,事無大小,輒與詳議。

永元四年,當以選入,會訓卒,后晝夜號泣,終三年不食鹽菜,憔悴毀容,親人不識之。后嘗夢捫天,蕩蕩正青,若有鍾乳狀,乃仰嗽飲之。以訊諸占夢,言堯夢攀天而上,湯夢及天而咶之,斯皆聖王之前占,吉不可言。又相者見后驚曰:「此成湯之法也。」家人竊喜而不敢宣。后叔父陔言:「常聞活千人者,子孫有封。兄訓為謁者,使修石臼河,歲活數千人。天道可信,家必蒙福。」初,太傅禹歎曰:「吾將百萬之眾,未嘗妄殺一人,其後世必有興者。」

七年,后復與諸家子俱選入宮。后長七尺二寸,姿顏姝麗,絕異於眾,左右皆驚。八年冬,入掖庭為貴人,時年十六。恭肅小心,動有法度。承事陰后,夙夜戰兢。接撫同列,常克己以下之,雖宮人隸役,皆加恩借。帝深嘉愛焉。及后有疾,特令后母兄弟入視醫藥,不限以日數。后言於帝曰:「宮禁至重,而使外舍久在內省,上令陛下有幸私之譏,下使賤妾獲不知足之謗。上下交損,誠不願也。」帝曰:「人皆以數入為榮,貴人反以為憂,深自抑損,誠難及也。」每有讌會,諸姬貴人競自修整,簪珥光采,褂裳鮮明,而后獨著素,裝服無飾。其衣有與陰后同色者,即時解易。若並時進見,則不敢正坐離立,行則僂身自卑。帝每有所問,常逡巡後對,不敢先陰后言。帝知后勞心曲體,歎曰:「修德之勞,乃如是乎!」後陰后漸疏,每當御見,輒辭以疾。時帝數失皇子,后憂繼嗣不廣,恆垂涕歎息,數選進才人,以博帝意。

陰后見后德稱日盛,不知所為,遂造祝詛,欲以為害。帝嘗寢病危甚,陰后密言:「我得意,不令鄧氏復有遺類!」后聞,乃對左右流涕言曰:「我竭誠盡心以事皇后,竟不為所祐,而當獲罪於天。婦人雖無從死之義,然周公身請武王之命,越姬心誓必死之分,上以報帝之恩,中以解宗族之禍,下不令陰氏有人豕之譏。」即欲飲藥,宮人趙玉者固禁之,因詐言屬有使來,上疾已愈。后信以為然,乃止。明日,帝果瘳。

十四年夏,陰后以巫蠱事廢,后請救不能得,帝便屬意焉。后愈稱疾篤,深自閉絕。會有司奏建長秋宮,帝曰:「皇后之尊,與朕同體,承宗廟,母天下,豈易哉!唯鄧貴人德冠後庭,乃可當之。」至冬,立為皇后。辭讓者三,然後即位。手書表謝,深陳德薄,不足以充小君之選。是時,方國貢獻,競求珍麗之物,自后即位,悉令禁絕,歲時但供紙墨而已。帝每欲官爵鄧氏,后輒哀請謙讓,故兄騭終帝世不過虎賁中郎將。

元興元年,帝崩,長子平原王有疾,而諸皇子夭沒,前後十數,後生者輒隱秘養於人閒。殤帝生始百日,后乃迎立之。尊后為皇太后,太后臨朝。和帝葬後,宮人並歸園,太后賜周、馮貴人策曰:「朕與貴人託配後庭,共歡等列,十有餘年。不獲福祐,先帝早棄天下,孤心煢煢,靡所瞻仰,夙夜永懷,感愴發中。今當以舊典分歸外園,慘結增歎,燕燕之詩,曷能喻焉?其賜貴人王青蓋車,采飾輅,驂馬各一駟,黃金三十斤,雜帛三千匹,白越四千端。」又賜馮貴人王赤綬,以未有頭上步搖、環珮,加賜各一具。

是時新遭大憂,法禁未設。宮中亡大珠一篋,太后念,欲考問,必有不辜。乃親閱宮人,觀察顏色,即時首服。又和帝幸人吉成,御者共枉吉成以巫蠱事,遂下掖庭考訊,辭證明白。太后以先帝左右,待之有恩,平日尚無惡言,今反若此,不合人情,更自呼見實覈,果御者所為。莫不歎服,以為聖明。常以鬼神難徵,淫祀無福,乃詔有司罷諸祠官不合典禮者。又詔赦除建武以來諸犯妖惡,及馬、竇家屬所被禁錮者,皆復之為平人。減大官、導官、尚方、內者服御珍膳靡麗難成之物,自非供陵廟,稻梁米得導擇,朝夕一肉飯而已。舊大官湯官經用歲且二萬萬,太后敕止,曰殺省珍費,自是裁數千萬。及郡國所貢,皆減其過半。悉斥賣上林鷹犬。其蜀、漢釦器九帶佩刀,並不復調。止畫工三十九種。又御府、尚方、織室錦繡、冰紈、綺縠、金銀、珠玉、犀象、玳瑁、彫鏤翫弄之物,皆絕不作。離宮別館儲峙米糒薪炭,悉令省之。又詔諸園貴人,其宮人有宗室同族若羸老不任使者,令園監實覈上名,自御北宮增喜觀閱問之,恣其去留,即日免遣者五六百人。

及殤帝崩,太后定策立安帝,猶臨朝政。以連遭大憂,百姓苦役,殤帝康陵方中祕藏,及諸工作,事事減約,十分居一。

詔告司隸校尉、河南尹、南陽太守曰:「每覽前

代外戚賓客,假借威權,輕薄謥詷,至有濁亂奉公,為人患苦。咎在執法怠懈,不輒行其罰故也。今車騎將軍騭等雖懷敬順之志,而宗門廣大,姻戚不少,賓客姦猾,多干禁憲。其明加檢敕,勿相容護。」自是親屬犯罪,無所假貸。太后愍陰氏之罪廢,赦其徙者歸鄉,敕還資財五百餘萬。永初元年,爵號太夫人為新野君,萬戶供湯沐邑。

二年夏,京師旱,親幸洛陽寺錄冤獄。有囚實不殺人而被考自誣,羸困輿見,畏吏不敢言,將去,舉頭若欲自訴。太后察視覺之。即呼還問狀,具得枉實,即時收洛陽令下獄抵罪。行未還宮,澍雨大降。

三年秋,太后體不安,左右憂惶,禱請祝辭,願得代命。太后聞之,即譴怒,切敕掖庭令以下,但使謝過祈福,不得妄生不祥之言。舊事,歲終當饗遣衛士,大儺逐疫。太后以陰陽不和,軍旅數興,詔饗會勿設戲作樂,減逐疫侲子之半,悉罷象橐駝之屬。豐年復故。太后自入宮掖,從曹大家受經書,兼天文、筭數。晝省王政,夜則誦讀,而患其謬誤,懼乖典章,乃博選諸儒劉珍等及博士、議郎、四府掾史五十餘人,詣東觀讎校傳記。事畢奏御,賜葛布各有差。又詔中官近臣於東觀受讀經傳,以教授宮人,左右習誦,朝夕濟濟。及新野君薨,太后自侍疾病,至乎終盡,憂哀毀損,事加於常。贈以長公主赤綬、東園祕器、玉衣繡衾,又賜布三萬匹,錢三千萬。騭等遂固讓錢布不受。使司空持節護喪事,儀比東海恭王,謚曰敬君。太后諒闇既終,久旱,太后比三日幸洛陽,錄囚徒,理出死罪三十六人,耐罪八十人,其餘減罪死右趾已下至司寇。

七年正月,初入太廟,齋七日,賜公卿百僚各有差。庚戌,謁宗廟,率命婦群妾相禮儀,與皇帝交獻親薦,成禮而還。因下詔曰:「凡供薦新味,多非其節,或鬱養強孰,或穿掘萌牙,味無所至而夭折生長,豈所以順時育物乎!傳曰:『非其時不食。』自今當奉祠陵廟及給御者,皆須時乃上。」凡所省二十三種。

自太后臨朝,水旱十載,四夷外侵,盜賊內起。每聞人飢,或達旦不寐,而躬自減徹,以救災厄,故天下復平,歲還豐穰。

元初五年,平望侯劉毅以太后多德政,欲令早有注記,上書安帝曰:「臣聞易載羲農而皇德著,書述唐虞而帝道崇,故雖聖明,必書功於竹帛,流音於管弦。伏惟皇太后膺大聖之姿,體乾坤之德,齊蹤虞妃,比跡任姒。孝悌慈仁,允恭節約,杜絕奢盈之源,防抑逸欲之兆。正位內朝,流化四海。及元興、延平之際,國無儲副,仰觀乾象,參之人譽,援立陛下為天下主,永安漢室,綏靜四海。又遭水潦,東州飢荒。垂恩元元,冠蓋交路,菲薄衣食,躬率群下,損膳解驂,以贍黎苗。惻隱之恩,猶視赤子。克己引愆,顯揚仄陋。崇晏晏之政,敷在寬之教。興滅國,繼絕世,錄功臣,復宗室。追還徙人,蠲除禁錮。政非惠和,不圖於心;制非舊典,不訪於朝。弘德洋溢,充塞宇宙;洪澤豐沛,漫衍八方。華夏樂化,戎狄混并。丕功著於大漢,碩惠加於生人。巍巍之業,可聞而不可及;蕩蕩之勳,可誦而不可名。古之帝王,左右置史;漢之舊典,世有注記。夫道有夷崇,治有進退。若善政不述,細異輒書,是為堯湯負洪水大旱之責,而無咸熙假天之美;高宗成王有雉雊迅風之變,而無中興康寧之功也。上考詩書,有虞二妃,周室三母,修行佐德,思不踰閾。未有內遭家難,外遇災害,覽總大麓,經營天物,功德巍巍若茲者也。宜令史官著長樂宮注、聖德頌,以敷宣景燿,勒勳金石,縣之日月,攄之罔極,以崇陛下烝烝之孝。」帝從之。

六年,太后詔徵和帝弟濟北、河閒王子男女年五歲以上四十餘人,又鄧氏近親子孫三十餘人,並為開邸第,教學經書,躬自監試。尚幼者,使置師保,朝夕入宮,撫循詔導,恩愛甚渥。乃詔從兄河南尹豹、越騎校尉康等曰:「吾所以引納群子,置之學官者,實以方今承百王之敝,時俗淺薄,巧偽滋生,五經衰缺,不有化導,將遂陵遲,故欲褒崇聖道,以匡失俗。傳不云乎:『飽食終日,無所用心,難矣哉!』今末世貴戚食祿之家,溫衣美飯,乘堅驅良,而面牆術學,不識臧否,斯故禍敗所從來也。永平中,四姓小侯皆令入學,所以矯俗厲薄,反之忠孝。先公既以武功書之竹帛,兼以文德教化子孫,故能束脩,不觸羅網。誠令兒曹上述祖考休烈,下念詔書本意,則足矣。其勉之哉!」

康以太后久臨朝政,心懷畏懼,託病不朝。太后使內人問之。時宮婢出入,多能有所毀譽,其耆宿者皆稱中大人,所使者乃康家先婢,亦自通中大人。康聞,詬之曰:「汝我家出,爾敢爾邪!」婢怒,還說康詐疾而言不遜。太后遂免康官,遣歸國,絕屬籍。

永寧二年二月,寢病漸篤,乃乘輦於前殿,見侍中、尚書,因北至太子新所繕宮。還,大赦天下,賜諸園貴人、王、主、群僚錢布各有差。詔曰:「朕以無德,託母天下,而薄祐不天,早離大憂。延平之際,海內無主,元元厄運,危於累卵。勤勤苦心,不敢以萬乘為樂,上欲不欺天愧先帝,下不違人負宿心,誠在濟度百姓,以安劉氏。自謂感徹天地,當蒙福祚,而喪禍內外,傷痛不絕。頃以廢病沈滯,久不得侍祠,自力上原陵,加欬逆唾血,遂至不解。存亡大分,無可柰何。公卿百官,其勉盡忠恪,以輔朝廷。」三月崩。在位二十年,年四十一。合葬順陵。

論曰:鄧后稱制終身,號令自出,術謝前政之良,身闕明辟之義,至使嗣主側目,斂衽於虛器,直生懷懣,懸書於象魏。借之儀者,殆其惑哉!然而建光之後,王柄有歸,遂乃名賢戮辱,便孽黨進,衰斁之來,茲焉有徵。故知持權引謗,所幸者非己;焦心卹患,自強者唯國。是以班母一說,闔門辭事;愛姪微愆,髡剔謝罪。將杜根逢誅,未值其誠乎!但蹊田之牛,奪之已甚。


\end{pinyinscope}