\article{馮岑賈列傳}

\begin{pinyinscope}
溤異字公孫,潁川父城人也。好讀書,通左氏春秋、孫子兵法。

漢兵起,異以郡掾監五縣,與父城長苗萌共城守,為王莽拒漢。光武略地潁川,攻父城不下,屯兵巾車鄉。異閒出行屬縣,為漢兵所執。時異從兄孝及同郡丁綝、呂晏,並從光武,因共薦異,得召見。異曰:「異一夫之用,不足為彊弱。有老母在城中,願歸據五城,以效功報德。」光武曰「善」。異歸,謂苗萌曰:「今諸將皆壯士屈起,多暴橫,獨有劉將軍所到不虜掠。觀其言語舉止,非庸人也,可以歸身。」苗萌曰:「死生同命,敬從子計。」光武南還宛,更始諸將攻父城者前後十餘輩,異堅守不下;及光武為司隸校尉,道經父城,異等即開門奉牛酒迎。光武署異為主簿,苗萌為從事。異因薦邑子銚期、叔壽、段建、左隆等,光武皆以為掾史,從至洛陽。

更始數欲遣光武徇河北,諸將皆以為不可。是時左丞相曹竟子詡為尚書,父子用事,異勸光武厚結納之。及度河北,詡有力焉。

自伯升之敗,光武不敢顯其悲戚,每獨居,輒不御酒肉,枕席有涕泣處。異獨叩頭寬譬哀情。光武止之曰:「卿勿妄言。」異復因閒進說曰:「天下同苦王氏,思漢久矣。今更始諸將從橫暴虐,所至虜掠,百姓失望,無所依戴。今公專命方面,施行恩德。夫有桀紂之亂,乃見湯武之功;人久飢渴,易為充飽。宜急分遣官屬,徇行郡縣,理冤結,布惠澤。」光武納之。至邯鄲,遣異與銚期乘傳撫循屬縣,錄囚徒,存鰥寡,亡命自詣者除其罪,陰條二千石長吏同心及不附者上之。

及王郎起,光武自薊東南馳,晨夜草舍,至饒陽無蔞亭。時天寒烈,眾皆飢疲,異上豆粥。明旦,光武謂諸將曰:「昨得公孫豆粥,飢寒俱解。」及至南宮,遇大風雨,光武引車入道傍空舍,異抱薪,鄧禹锔火,光武對灶燎衣。異復進麥飯菟肩,因復度虖沱河至信都,使異別收河閒兵。還,拜偏將軍。從破王郎,封應侯。

異為人謙退不伐,行與諸將相逢,輒引車避道。進止皆有表識,軍中號為整齊。每所止舍,諸將並坐論功,異常獨屏樹下,軍中號曰「大樹將軍」。及破邯鄲,乃更部分諸將,各有配隸。軍士皆言願屬大樹將軍,光武以此多之。別擊破鐵脛於北平,又降匈奴于林闟頓王,因從平河北。

時更始遣舞陰王李軼、廩丘王田立、大司馬朱鮪

、白虎公陳僑將兵號三十萬,與河南太守武勃共守洛陽。光武將北徇燕、趙,以魏郡、河內獨不逢兵,而城邑完,倉廩實,乃拜寇恂為河內太守,異為孟津將軍,統二郡軍河上,與恂合埶,以拒朱鮪等。

異乃遺李軼書曰:「愚聞明鏡所以照形,往事所以知今。昔微子去殷而入周,項伯畔楚而歸漢,周勃迎代王而黜少帝,霍光尊孝宣而廢昌邑。彼皆畏天知命,睹存亡之符,見廢興之事,故能成功於一時,垂業於萬世也。笱令長安尚可扶助,延期歲月,疏不閒親,遠不踰近,季文豈能居一隅哉?今長安壞亂,赤眉臨郊,王侯搆難,大臣乖離,綱紀已絕,四方分崩,異姓並起,是故蕭王跋涉霜雪,經營河北。方今英俊雲集,百姓風靡,雖邠岐慕周,不足以喻。季文誠能覺悟成敗,亟定大計,論功古人,轉禍為福,在此時矣。如猛將長驅,嚴兵圍城,雖有悔恨,亦無及已。」初,軼與光武首結謀約,加相親愛,及更始立,反共陷伯升。雖知長安已危,欲降又不自安。乃報異書曰:「軼本與蕭王首謀造漢,結死生之約,同榮枯之計。今軼守洛陽,將軍鎮孟津,俱據機軸,千載一會,思成斷金。唯深達蕭王,願進愚策,以佐國安人。」軼自通書之後,不復與異爭鋒,故異因此得北攻天井關,拔上黨兩城,又南下河南成皋已東十三縣,及諸屯聚,皆平之,降者十餘萬。武勃將萬餘人攻諸畔者,異引軍度河,與勃戰於士鄉下,大破斬勃,獲首五千餘級,軼又閉門不救。異見其信效,具以奏聞。光武故宣露軼書,令朱鮪知之。鮪怒,遂使人刺殺軼。由是城中乖離,多有降者。鮪乃遣討難將軍蘇茂將數萬人攻溫,鮪自將數萬人攻平陰以綴異。異遣校尉護軍將軍將兵,與寇恂合擊茂,破之。異因度河擊鮪,鮪走;異追至洛陽,環城一匝而歸。

移檄上狀,諸將皆入賀,并勸光武即帝位。光武乃召異詣鄗,問四方動靜。異曰:「三王反畔,更始敗亡,天下無主,宗廟之憂,在於大王。宜從眾議,上為社稷,下為百姓。」光武曰:「我昨夜夢乘赤龍上天,覺悟,心中動悸。」異因下席再拜賀曰:「此天命發於精神。心中動悸,大王重慎之性也。」異遂與諸將定議上尊號。

建武二年春,定封異陽夏侯。引擊陽翟賊嚴終、趙根,破之。詔異歸家上冢,使太中大夫齎牛酒,令二百里內太守、都尉已下及宗族會焉。

時赤眉、延岑暴亂三輔,郡縣大姓各擁兵眾,大司徒鄧禹不能定,乃遣異代禹討之。車駕送至河南,賜以乘輿七尺具劍。敕異曰:「三輔遭王莽、更始之亂,重以赤眉、延岑之酷,元元塗炭,無所依訴。今之征伐,非必略地屠城,要在平定安集之耳。諸將非不健鬥,然好虜掠。卿本能御吏士,念自修敕,無為郡縣所苦。」異頓首受命,引而西,所至皆布威信。弘農群盜稱將軍者十餘輩,皆率眾降異。

異與赤眉遇於華陰,相拒六十餘日,戰數十合,降其將劉始、王宣等五千餘人。三年春,遣使者即拜異為征西大將軍。會鄧禹率車騎將軍鄧弘等引歸,與異相遇,禹、弘要異共攻赤眉。異曰:「異與賊相拒且數十日,雖屢獲雄將,餘眾尚多,可稍以恩信傾誘,難卒用兵破也。上今使諸將屯黽池要其東,而異擊其西,一舉取之,此萬成計也。」禹、弘不從。弘遂大戰移日,赤眉陽敗,棄輜重走。車皆載土,以豆覆其上,兵士飢,爭取之。赤眉引還擊弘,弘軍潰亂。異與禹合兵救之,赤眉小卻。異以士卒飢倦,可且休,禹不聽,復戰,大為所敗,死傷者三千餘人。禹得脫歸宜陽。異棄馬步走上回谿阪,與麾下數人歸營。復堅壁,收其散卒,招集諸營保數萬人,與賊約期會戰。使壯士變服與赤眉同,伏於道側。旦日,赤眉使萬人攻異前部,異裁出兵以救之。賊見埶弱,遂悉眾攻異,異乃縱兵大戰。日昃,賊氣衰,伏兵卒起,衣服相亂,赤眉不復識別,眾遂驚潰。追擊,大破於崤底,降男女八萬人。餘眾尚十餘萬,東走宜陽降。璽書勞異曰:「赤眉破平,士吏勞苦,始雖垂翅回谿,終能奮翼黽池,可謂失之東隅,收之桑榆。方論功賞,以荅大勳。」

時赤眉雖降,眾寇猶盛:延岑據藍田,王歆據下邽,芳丹據新豐,蔣震據霸陵,張邯據長安,公孫守據長陵,楊周據谷口,呂鮪據陳倉,角閎據汧,駱蓋延據盩厔,任良據鄠,汝章據槐里,各稱將軍,擁兵多者萬餘,少者數千人,轉相攻擊。異且戰且行,屯軍上林苑中。延岑既破赤眉,自稱武安王,拜置牧守,欲據關中,引張邯、任良共攻異。異擊破之,斬首千餘級,諸營保守附岑者皆來降歸異。岑走攻析,異遣復漢將軍鄧曄、輔漢將軍于匡要擊岑,大破之,降其將蘇臣等八千餘人。岑遂自武關走南陽。時百姓飢餓,人相食,黃金一斤易豆五升。道路斷隔,委輸不至,軍士悉以果實為糧。詔拜南陽趙匡為右扶風,將兵助異,并送縑穀,軍中皆稱萬歲。異兵食漸盛,乃稍誅擊豪傑不從令者,褒賞降附有功勞者,悉遣其渠帥詣京師,散其眾歸本業。威行關中。唯呂鮪、張邯、蔣震遣使降蜀,其餘悉平。

明年,公孫述遣將程焉,將數萬人就呂鮪出屯陳倉。異與趙匡迎擊,大破之,焉退走漢川。異追戰於箕谷,復破之,還擊破呂鮪,營保降者甚眾。其後蜀復數遣將閒出,異輒摧挫之。懷來百姓,申理枉結,出入三歲,上林成都。

異自以久在外,不自安,上書思慕闕廷,願親帷幄,帝不許。後人有章言異專制關中,斬長安令,威權至重,百姓歸心,號為「咸陽王」。帝使以章示異。異惶懼,上書謝曰:「臣本諸生,遭遇受命之會,充備行伍,過蒙恩私,位大將,爵通侯,受任方面,以立微功,皆自國家謀慮,愚臣無所能及。臣伏自思惟:以詔敕戰攻,每輒如意;時以私心斷決,未嘗不有悔。國家獨見之明,久而益遠,乃知『性與天道,不可得而聞也』。當兵革始起,擾攘之時,豪傑競逐,迷惑千數。臣以遭遇,託身聖明,在傾危溷殽之中,尚不敢過差,而況天下平定,上尊下卑,而臣爵位所蒙,巍巍不測乎?誠冀以謹敕,遂自終始。見所示臣章,戰慄怖懼。伏念明主知臣愚性,固敢因緣自陳。」詔報曰:「將軍之於國家,義為君臣,恩猶父子。何嫌何疑,而有懼意?」

六年春,異朝京師。引見,帝謂公卿曰:「是我起兵時主簿也。為吾披荊棘,定關中。」既罷,使中黃門賜以珍寶、衣服、錢帛。詔曰:「倉卒無蔞亭豆粥,虖沱河麥飯,厚意久不報。」異稽首謝曰:「臣聞管仲謂桓公曰:『願君無忘射鉤,臣無忘檻車。』齊國賴之。臣今亦願國家無忘河北之難,小臣不敢忘巾車之恩。」後數引讌見,定議圖蜀,留十餘日,令異妻子隨異還西。

夏,遣諸將上隴,為隗囂所敗,乃詔異軍栒邑。未及至,隗囂乘勝使其將王元、行巡將二萬餘人下隴,因分遣巡取栒邑。異即馳兵,欲先據之。諸將皆曰:「虜兵盛而新乘勝,不可與爭。宜止軍便地,徐思方略。」異曰:「虜兵臨境,忸觇小利,遂欲深入。若得栒邑,三輔動搖,是吾憂也。夫『攻者不足,守者有餘』。今先據城,以逸待勞,非所以爭也。」潛往閉城,偃旗鼓。行巡不知,馳赴之。異乘其不意,卒擊鼓建旗而出。巡軍驚亂奔走,追搫數十里,大破之。祭遵亦破王元於钐。於是北地諸豪長耿定等,悉畔隗囂降。異上書言狀,不敢自伐。諸將或欲分其功,帝患之。乃下璽書曰:「制詔大司馬,虎牙、建威、漢中、捕虜、武威將軍:虜兵猥下,三輔驚恐。栒邑危亡,在於旦夕。北地營保,按兵觀望。今偏城獲全,虜兵挫折,使耿定之屬,復念君臣之義。征西功若丘山,猶自以為不足。孟之反奔而殿,亦何異哉?今遣太中大夫賜征西吏士死傷者醫藥、棺斂,大司馬已下親弔死問疾,以崇謙讓。」於是使異進軍義渠,并領北地太守事。

青山胡率萬餘人降異。異又擊盧芳將賈覽、匈奴薁鞬日逐王,破之。上郡、安定皆降,異復領安定太守事。九年春,祭遵卒,詔異守征虜將軍,并將其營。及隗囂死,其將王元、周宗等復立囂子純,猶總兵據冀,公孫述遣將趙匡等救之,帝復令異行天水太守事。攻匡等且一年,皆斬之。諸將共攻冀,不能拔,欲且還休兵,異固持不動,常為眾軍鋒。

明年夏,與諸將攻落門,未拔,病發,薨于軍,謚曰節侯。

長子彰嗣。明年,帝思異功,復封彰弟訢為析鄉侯。十三年,更封彰東緡侯,食三縣。永平中,徙封平鄉侯。彰卒,子普嗣,有罪,國除。

永初六年,安帝下詔曰:「夫仁不遺親,義不忘勞,興滅繼絕,善善及子孫,古之典也。昔我光武受命中興,恢弘聖緒,橫被四表,昭假上下,光耀萬世,祉祚流衍,垂於罔極。予末小子,夙夜永思,追惟勳烈,披圖案籍,建武元功二十八將,佐命虎臣,讖記有徵。蓋蕭、曹紹封,傳繼於今;況此未遠,而或至乏祀,朕甚愍之。其條二十八將無嗣絕世,若犯罪奪國,其子孫應當統後者,分別署狀上。將及景風,章敘舊德,顯茲遺功焉。」於是紹封普子晨為平鄉侯。明年,二十八將絕國者,皆紹封焉。

岑彭字君然,南陽棘陽人也。王莽時,守本縣長。漢兵起,攻拔棘陽,彭將家屬奔前隊大夫甄阜。阜怒彭不能固守,拘彭母妻,令效功自補。彭將賓客戰鬥甚力。及甄阜死,彭被創,亡歸宛,與前隊貳嚴說共城守。漢兵攻之數月,城中糧盡,人相食,彭乃與說舉城降。

諸將欲誅之,大司徒伯升曰:「彭,郡之大吏,執心堅守,是其節也。今舉大事,當表義士,不如封之,以勸其後。」更始乃封彭為歸德侯,令屬伯升。及伯升遇害,彭復為大司馬朱鮪校尉,從鮪擊王莽楊州牧李聖,殺之,定淮陽城。鮪薦彭為淮陽都尉。更始遣立威王張卬與將軍徭偉鎮淮陽。偉反,擊走卬。彭引兵攻偉,破之。遷潁川太守。

會舂陵劉茂起兵,略下潁川,彭不得之官,乃與麾下數百人從河內太守邑人韓歆。會光武徇河內,歆議欲城守,彭止不聽。既而光武至懷,歆迫急迎降。光武知其謀,大怒,收歆置鼓下,將斬之。召見彭,彭因進說曰:「今赤眉入關,更始危殆,權臣放縱,矯稱詔制,道路阻塞,四方蜂起,群雄競逐,百姓無所歸命。竊聞大王平河北,開王業,此誠皇天祐漢,士人之福也。彭幸蒙司徒公所見全濟,未有報德,旋被禍難,永恨於心。今復遭遇,願出身自效。」光武深接納之。彭因言韓歆南陽大人,可以為用。乃貰歆,以為鄧禹軍師。

更始大將軍呂植將兵屯淇園,彭說降之,於是拜彭為刺姦大將軍,使督察眾營,授以常所持節,從平河北。光武即位,拜彭廷尉,歸德侯如故,行大將軍事。與大司馬吳漢,大司空王梁,建義大將軍朱祐,右將軍萬脩,執金吾賈復,驍騎將軍劉植,揚化將軍堅鐔,積射將軍侯進,偏將軍馮異、祭遵、王霸等,圍洛陽數月。朱鮪等堅守不肯下。帝以彭嘗為鮪校尉,令往說之。鮪在城上,彭在城下,相勞苦歡語如平生。彭因曰:「彭往者得執鞭侍從,蒙薦舉拔擢,常思有以報恩。今赤眉已得長安,更始為三王所反,皇帝受命,平定燕、趙,盡有幽、冀之地,百姓歸心,賢俊雲集,親率大兵,來攻洛陽。天下之事,逝其去矣。公雖嬰城固守,將何待乎?」鮪曰:「大司徒被害時,鮪與其謀,又諫更始無遣蕭王北伐,誠自知罪深。」彭還,具言於帝。帝曰:「夫建大事者,不忌小怨。鮪今若降,官爵可保,況誅罰乎?河水在此,吾不食言。」彭復往告鮪,鮪從城上下索曰:「必信,可乘此上。」彭趣索欲上。鮪見其誠,即許降。後五日,鮪將輕騎詣彭。顧敕諸部將曰:「堅守待我。我若不還,諸君徑將大兵上轘轅,歸郾王。」乃面縛,與彭俱詣河陽。帝即解其縛,召見之,復令彭夜送鮪歸城。明旦,悉其眾出降,拜鮪為平狄將軍,封扶溝侯。鮪,淮陽人,後為少府,傳封累代。

建武二年,使彭擊荊州,下犨、葉等十餘城。是時南方尤亂。南郡人秦豐據黎丘,自稱楚黎王,略十有〈十〉二縣;董訢起堵鄉;許邯起杏;又更始諸將各擁兵據南陽諸城。帝遣吳漢伐之,漢軍所過多侵暴。時破虜將軍鄧奉謁歸新野,怒吳漢掠其鄉里,遂返,擊破漢軍,獲其輜重,屯據淯陽,與諸賊合從。秋,彭破杏,降許邯,遷征南大將軍。復遣朱祐、賈復及建威大將軍耿弇,漢中將軍王常,武威將軍郭守,越騎將軍劉宏,偏將軍劉嘉、耿植等,與彭并力討鄧奉。先擊堵鄉,而奉將萬餘人救董訢。訢、奉皆南陽精兵,彭等攻之,連月不剋。三年夏,帝自將南征,至葉,董訢別將將數千人遮道,車騎不可得前。彭奔擊,大破之。帝至堵陽,鄧奉夜逃歸淯陽,董訢降。彭復與耿弇、賈復及積弩將軍傅俊、騎都尉臧宮等從追鄧奉於小長安。帝率諸將親戰,大破之。奉迫急,乃降。帝憐奉舊功臣,且釁起吳漢,欲全宥之。彭與耿弇諫曰:「鄧奉背恩反逆,暴師經年,致賈復傷痍,朱祐見獲。陛下既至,不知悔善,而親在行陳,兵敗乃降。若不誅奉,無以懲惡。」於是斬之。奉者,西華侯鄧晨之兄子也。

車駕引還,令彭率傅俊、臧宮、劉宏等三萬餘人南趋秦豐,拔黃郵,豐與其大將蔡宏拒彭等於鄧,數月不得進。帝怪以讓彭。彭懼,於是夜勒兵馬,申令軍中,使明旦西擊山都。乃緩所獲虜,令得逃亡,歸以告豐,豐即悉其軍西邀彭。彭乃潛兵度沔水,擊其將張楊於阿頭山,大破之。從川谷閒伐木開道,直襲黎丘,擊破諸屯兵。豐聞大驚,馳歸救之。彭與諸將依東山為營,豐與蔡宏夜攻彭,彭豫為之備,出兵逆擊之,豐敗走,追斬蔡宏。更封彭為舞陰侯。

秦豐相趙京舉宜城降,拜為成漢將軍,與彭共圍豐於黎丘。時田戎擁眾夷陵,聞秦豐被圍,懼大兵方至,欲降。而妻兄辛臣諫戎曰:「今四方豪傑各據郡國,洛陽地如掌耳,不如按甲以觀其變。」戎曰:「以秦王之彊,猶為征南所圍,豈況吾邪?降計決矣。」四年春,戎乃留辛臣守夷陵,自將兵沿江泝沔止黎丘,刻期日當降,而辛臣於後盜戎珍寶,從閒道先降於彭,而以書招戎。戎疑必賣己,遂不敢降,而反與秦豐合。彭出兵攻戎,數月,大破之,其大將伍公詣彭降,戎亡歸夷陵。帝幸黎丘勞軍,封彭吏士有功者百餘人。彭攻秦豐三歲,斬首九萬餘級,豐餘兵裁千人,又城中食且盡。帝以豐轉弱,令朱祐代彭守之,使彭與傅俊南擊田戎,大破之,遂拔夷陵,追至秭歸。戎與數十騎亡入蜀,盡獲其妻子士眾數萬人。

彭以將伐蜀漢,而夾川穀少,水險難漕運,留威虜將軍馮駿軍江州,都尉田鴻軍夷陵,領軍李玄軍夷道,自引兵還屯津鄉,當荊州要會,喻告諸蠻夷,降者奏封其君長。初,彭與交阯牧鄧讓厚善,與讓書陳國家威德,又遣偏將軍屈充移檄江南,班行詔命,於是讓與江夏太守侯登、武陵太守王堂、長沙相韓福、桂陽太守張隆、零陵太守田翕、蒼梧太守杜穆、交阯太守錫光等,相率遣使貢獻,悉封為列侯。或遣子將兵助彭征伐。於是江南之珍始流通焉。

六年冬,徵彭詣京師,數召讌見,厚加賞賜。復南還津鄉,有詔過家上冢,大長秋以朔望問太夫人起居。

八年,彭引兵從車駕破天水,與吳漢圍隗囂於西城。時公孫述將李育將兵救囂,守上邽,帝留蓋延、耿弇圍之,而車駕東歸。敕彭書曰:「兩城若下,便可將兵南擊蜀虜。人苦不知足,既平隴,復望蜀。每一發兵,頭鬚為白。」彭遂壅谷水灌西城,城未沒丈餘,囂將行巡、周宗將蜀救兵到,囂得出還冀。漢軍食盡,燒輜重,引兵下隴,延、弇亦相隨而退。囂出兵尾擊諸營,彭殿為後拒,故諸將能全師東歸。彭還津鄉。

九年,公孫述遣其將任滿、田戎、程汎,將數萬人乘枋箄下江關,擊破馮駿及田鴻、李玄等。遂拔夷道、夷陵,據荊門、虎牙。橫江水起浮橋、鬥樓,立欑柱絕水道,結營山上,以拒漢兵。彭數攻之,不利,於是裝直進樓船、冒突露橈數千艘。

十一年春,彭與吳漢及誅虜將軍劉隆、輔威將軍臧宮、驍騎將軍劉歆,發南陽、武陵、南郡兵,又發桂陽、零陵、長沙委輸棹卒,凡六萬餘人,騎五千匹,皆會荊門。吳漢以三郡棹卒多費糧穀,欲罷之。彭以蜀兵盛,不可遣,上書言狀。帝報彭曰:「大司馬習用步騎,不曉水戰,荊門之事,一由征南公為重而已。」彭乃令軍中募攻浮橋,先登者上賞。於是偏將軍魯奇應募而前。時天風狂急,彭奇船逆流而上,直衝浮橋,而欑柱鉤不得去,奇等乘埶殊死戰,因飛炬焚之,風怒火盛,橋樓崩燒。彭復悉軍順風並進,所向無前。蜀兵大亂,溺死者數千人。斬任滿,生獲程汎,而田戎亡保江州。彭上劉隆為南郡太守,自率臧宮、劉歆長驅入江關,令軍中無得虜掠。所過,百姓皆奉牛酒迎勞。彭見諸耆老,為言大漢哀愍巴蜀久見虜役,故興師遠伐,以討有罪,為人除害。讓不受其牛酒。百姓皆大喜悅,爭開門降。詔彭守益州牧,所下郡,輒行太守事。

彭到江州,以田戎食多,難卒拔,留馮駿守之,自引兵乘利直指墊江,攻破平曲,收其米數十萬石。公孫述使其將延岑、呂鮪、王元及其弟恢悉兵拒廣漢及資中,又遣將侯丹率二萬餘人拒黃石。彭乃多張疑兵,使護軍楊翕與臧宮拒延岑等,自分兵浮江下還江州,泝都江而上,襲擊侯丹,大破之。因晨夜倍道兼行二千餘里,徑拔武陽。使精騎馳廣都,去成都數十里,埶若風雨,所至皆奔散。初,述聞漢兵在平曲,故遣大兵逆之。及彭至武陽,繞出延岑軍後,蜀地震駭。述大驚,以杖擊地曰:「是何神也!」

彭所營地名彭亡,聞而惡之,欲徙,會日暮,蜀刺客詐為亡奴降,夜刺殺彭。

彭首破荊門,長驅武陽,持軍整齊,秋豪無犯。邛穀王任貴聞彭威信,數千里遣使迎降。會彭已薨,帝盡以任貴所獻賜彭妻子,謚曰壯侯。蜀人憐之,為立廟武陽,歲時祠焉。

子遵嗣,徙封細陽侯。十三年,帝思彭功,復封遵弟淮為穀陽侯。遵永平中為屯騎校尉。遵卒,子伉嗣。伉卒,子杞嗣,元初三年,坐事失國。建光元年,安帝復封杞細陽侯,順帝時為光祿勳。

杞卒,子熙嗣,尚安帝妹涅陽長公主。少為侍中、虎賁中郎將,朝廷多稱其能。遷魏郡太守,招聘隱逸,與參政事,無為而化。視事二年,輿人歌之曰:「我有枳棘,岑君伐之。我有蟊賊,岑君遏之。狗吠不驚,足下生氂。含哺鼓腹,焉知凶災?我喜我生,獨丁斯時。美矣岑君,於戲休茲!」

熙卒,子福嗣,為黃門侍郎。

賈復字君文,南陽冠軍人也。少好學,習尚書。事舞陰李生,李生奇之,謂門人曰:「賈君之容貌志氣如此,而勤於學,將相之器也。」王莽末,為縣掾,迎鹽河東,會遇盜賊,等比十餘人皆放散其鹽,復獨完以還縣,縣中稱其信。

時下江、新市兵起,復亦聚眾數百人於羽山,自號將軍。更始立,乃將其眾歸漢中王劉嘉,以為校尉。復見更始政亂,諸將放縱,乃說嘉曰:「臣聞圖堯舜之事而不能至者,湯武是也;圖湯武之事而不能至者,桓文是也;圖桓文事而不能至者,六國是也;定六國之規,欲安守之而不能至者,亡六國是也。今漢室中興,大王以親戚為藩輔,天下未定而安守所保,所保得無不可保乎?」嘉曰:「卿言大,非吾任也。大司馬劉公在河北,必能相施,第持我書往。」復遂辭嘉,受書北度河,及光武於柏人,因鄧禹得召見。光武奇之,禹亦稱有將帥節,於是署復破虜將軍督盜賊。復馬羸,光武解左驂以賜之。官屬以復後來而好陵折等輩,調補鄗尉,光武曰:「賈督有折衝千里之威,方任以職,勿得擅除。」

光武至信都,以復為偏將軍。及拔邯鄲,遷都護將軍。從擊青犢於射犬,大戰至日中,賊陳堅不卻。光武傳召復曰:「吏士皆飢,可且朝飯。」復曰;「先破之,然後食耳。」於是被羽先登,所向皆靡,賊乃敗走。諸將咸服其勇。又北與五校戰於真定,大破之。復傷創甚。光武大驚曰:「我所以不令賈復別將者,為其輕敵也。果然,失吾名將。聞其婦有孕,生女邪,我子娶之,生男邪,我女嫁之,不令其憂妻子也。」復病尋愈,追及光武於薊,相見甚懽,大饗士卒,令復居前,擊鄴賊,破之。

光武即位,拜為執金吾,封冠軍侯。先度河攻朱鮪於洛陽,與白虎公陳僑戰,連破降之。建武二年,益封穰、朝陽二縣。更始郾王尹尊及諸大將在南方未降者尚多,帝召諸將議兵事,未有言,沈吟久之,乃以檄叩地曰:「郾最彊,宛為次,誰當擊之?」復率然對曰:「臣請擊郾。」帝笑曰:「執金吾擊郾,吾復何憂!大司馬當擊宛。」遂遣復與騎都尉陰識、驍騎將軍劉植南度五社津擊郾,連破之。月餘,尹尊降,盡定其地。引東擊更始淮陽太守暴汜,汜降,屬縣悉定。其秋,南擊召陵、新息,平定之。明年春,遷左將軍,別擊赤眉於新城、澠池閒,連破之。與帝會宜陽,降赤眉。

復從征伐,未嘗喪敗,數與諸將潰圍解急,身被十二創。帝以復敢深入,希令遠征,而壯其勇節,常自從之,故復少方面之勳。諸將每論功自伐,復未嘗有言。帝輒曰:「賈君之功,我自知之。」

十三年,定封膠東侯,食郁秩、壯武、下密、即墨、梃胡、觀陽,凡六縣。復知帝欲偃干戈,修文德,不欲功臣擁眾京師,乃與高密侯鄧禹並剽甲兵,敦儒學。帝深然之,遂罷左右將軍。復以列侯就第,加位特進。復為人剛毅方直,多大節。既還私第,闔門養威重。朱祐等薦復宜為宰相,帝方以吏事責三公,故功臣並不用。是時列侯唯高密、固始、膠東三侯與公卿參議國家大事,恩遇甚厚。三十一年卒,謚曰剛侯。

子忠嗣。忠卒,子敏嗣。建初元年,坐誣告母殺人,國除。肅宗更封復小子邯為膠東侯,邯弟宗為即墨侯,各一縣。邯卒,子育嗣。育卒,子長嗣。

宗字武孺,少有操行,多智略。初拜郎中,稍遷,建初中為朔方太守。舊內郡徙人在邊者,率多貧弱,為居人所僕役,不得為吏。宗擢用其任職者,與邊吏參選,轉相監司,以擿發其姦,或以功次補長吏,故各願盡死。匈奴畏之,不敢入塞。徵為長水校尉。宗兼通儒術,每讌見,常使與少府丁鴻等論議於前。章和二年卒,朝廷愍惜焉。

子參嗣。參卒,子建嗣。元初元年,尚和帝女臨潁長公主。主兼食潁陰、許,合三縣,數萬戶。時鄧太后臨朝,光寵最盛,以建為侍中,順帝時為光祿勳。

論曰:中興將帥立功名者眾矣,唯岑彭、馮異建方面之號,自函谷以西,方城以南,兩將之功,實為大焉。若馮、賈之不伐,岑公之義信,乃足以感三軍而懷敵人,故能剋成遠業,終全其慶也。昔高祖忌柏人之名,違之以全福;征南惡彭亡之地,留之以生災。豈幾慮自有明惑,將期數使之然乎?

贊曰:陽夏師克,實在和德。膠東鹽吏。征南宛賊。奇鋒震敵,遠圖謀國。


\end{pinyinscope}