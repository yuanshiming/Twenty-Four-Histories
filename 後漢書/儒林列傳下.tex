\article{儒林列傳下}

\begin{pinyinscope}
前書魯人申公受詩於浮丘伯,為作詁訓,是為魯詩;齊人轅固生亦傳詩,是為齊詩;燕人韓嬰亦傳詩,是為韓詩:三家皆立博士。趙人毛萇傳詩,是為毛詩,未得立。

高詡字季回,平原般人也。曾祖父嘉,以魯詩授元帝,仕至上谷太守。父容,少傳嘉學,哀平閒為光祿大夫。

詡以父任為郎中,世傳魯詩。以信行清操知名。王莽篡位,父子稱盲,逃,不仕莽世。光武即位,大司空宋弘薦詡,徵為郎,除符離長。去官,後徵為博士。建武十一年,拜大司農。在朝以方正稱。十三年,卒官,賜錢及冢田。

包咸字子良,會稽曲阿人也。少為諸生,受業長安,師事博士右師細君,習魯詩、論語。王莽末,去歸鄉里,於東海界為赤眉賊所得,遂見拘執。十餘日,咸晨夜誦經自若,賊異而遣之。因住東海,立精舍講授。光武即位,乃歸鄉里。太守黃讜署戶曹史,欲召咸入授其子。咸曰:「禮有來學,而無往教。」讜遂遣子師之。

舉孝廉,除郎中。建武中,入授皇太子論語,又為其章句。拜諫議大夫、侍中、右中郎將。永平五年,遷大鴻臚。每進見,錫以几杖,入屏不趨,贊事不名。經傳有疑,輒遣小黃門就舍即問。

顯宗以咸有師傅恩,而素清苦,常特賞賜珍玩束帛,奉祿增於諸卿,咸皆散與諸生之貧者。病篤,帝親輦駕臨視。八年,年七十二,卒於官。

子福,拜郎中,亦以論語入授和帝。

魏應字君伯,任城人也。少好學。建武初,詣博士受業,習魯詩。閉門誦習,不交僚黨,京師稱之。後歸為郡吏,舉明經,除濟陰王文學。以疾免官,教授山澤中,徒眾常數百人。永平初,為博士,再遷侍中。十三年,遷大鴻臚。十八年,拜光祿大夫。建初四年,拜五官中郎將,詔入授千乘王伉。

應經明行修,弟子自遠方至,著錄數千人。肅宗甚重之,數進見,論難於前,特受賞賜。時會京師諸儒於白虎觀,講論五經同異,使應專掌難問,侍中淳于恭奏之,帝親臨稱制,如石渠故事。明年,出為上黨太守,徵拜騎都尉,卒於官。

伏恭字叔齊,琅邪東武人,司徒湛之兄子也。湛弟黯,字稚文,以明齊詩,改定章句,作解說九篇,位至光祿勳,無子,以恭為後。

恭性孝,事所繼母甚謹,少傳黯學,以任為郎。建武四年,除劇令。視事十三年,以惠政公廉聞。青州舉為尤異,太常試經第一,拜博士,遷常山太守。敦脩學校,教授不輟,由是北州多為伏氏學。永平二年,代梁松為太僕。四年,帝臨辟雍,於行禮中拜恭為司空,儒者以為榮。

初,父黯章句繁多,恭乃省減浮辭,定為二十萬言。在位九年,以病乞骸骨罷,詔賜千石奉以終其身。十五年,行幸琅邪,引遇如三公儀。建初二年冬,肅宗行饗禮,以恭為三老。年九十,元和元年卒,賜葬顯節陵下。

子壽,官至東郡太守。

任末字叔本,蜀郡繁人也。少習齊詩,遊京師,教授十餘年。友人董奉德於洛陽病亡,末乃躬推鹿車,載奉德喪致其墓所,由是知名。為郡功曹,辭以病免。後奔師喪,於道物故。臨命,敕兄子造曰:「必致我尸於師門,使死而有知,魂靈不慚;如其無知,得土而已。」造從之。

景鸞字漢伯,廣漢梓潼人也。少隨師學經,涉七州之地。能理齊詩、施氏易,兼受河洛圖緯,作易說及詩解,文句兼取河洛,以類相從,名為交集。又撰禮內外記,號曰禮略。又抄風角雜書,列其占驗,作興道一篇。及作月令章句。凡所著述五十餘萬言。數上書陳救災變之術。州郡辟命不就。以壽終。

薛漢字公子,淮陽人也。世習韓詩,父子以章句著名。漢少傳父業,尤善說災異讖緯,教授常數百人。建武初,為博士,受詔校定圖讖。當世言詩者,推漢為長。永平中,為千乘太守,政有異跡。後坐楚事辭相連,下獄死。弟子犍為杜撫、會稽澹臺敬伯、鉅鹿韓伯高最知名。

杜撫字叔和,犍為武陽人也。少有高才。受業於薛漢,定韓詩章句。後歸鄉里教授。沈靜樂道,舉動必以禮。弟子千餘人。後為驃騎將軍東平王蒼所辟,及蒼就國,掾史悉補王官屬,未滿歲,皆自劾歸。時撫為大夫,不忍去,蒼聞,賜車馬財物遣之。辟太尉府。建初中,為公車令,數月卒官。其所作詩題約義通,學者傳之,曰杜君法云。

召馴字伯春,九江壽春人也。曾祖信臣,元帝時為少府。父建武中為卷令,俶儻不拘小節。

馴少習韓詩,博通書傳,以志義聞,鄉里號之曰「德行恂恂召伯春」。累仕州郡,辟司徒府。建初元年,稍遷騎都尉,侍講肅宗。拜左中郎將,入授諸王。帝嘉其義學,恩寵甚崇。出拜陳留太守,賜刀劍錢物。元和二年,入為河南尹。章和二年,代任隗為光祿勳,卒於官,賜冢塋陪園陵。

孫休,位至青州刺史。

楊仁字文義,巴郡閬中人也。建武中,詣師學習韓詩,數年歸,靜居教授。仕郡為功曹,舉孝廉,除郎。太常上仁經中博士,仁自以年未五十,不應舊科,上府讓選。

顯宗特詔補北宮衛士令,引見,問當世政跡。仁對以寬和任賢,抑黜驕戚為先。又上便宜十二事,皆當世急務。帝嘉之,賜以縑錢。

及帝崩,時諸馬貴盛,各爭欲入宮。仁被甲持戟,嚴勒門衛,莫敢輕進者。肅宗既立,諸馬共譖仁刻峻,帝知其忠,愈善之,拜什邡令。寬惠為政,勸課掾史弟子,悉令就學。其有通明經術者,顯之右署,或貢之朝,由是義學大興。墾田千餘頃。行兄喪去官。

後辟司徒桓虞府。掾有宋章者,貪奢不法,仁終不與交言同席,時人畏其節。後為閬中令,卒於官。

趙曄字長君,會稽山陰人也。少嘗為縣吏,奉檄迎督郵,曄恥於廝役,遂棄車馬去。到犍為資中,詣杜撫受韓詩,究竟其術。積二十年,絕問不還,家為發喪制服。曄卒業乃歸。州召補從事,不就。舉有道。卒于家。

曄著吳越春秋、詩細歷神淵。蔡邕至會稽,讀詩細而歎息,以為長於論衡。邕還京師,傳之,學者咸誦習焉。

時山陽張匡,字文通。亦習韓詩,作章句。後舉有道,博士徵,不就。卒於家。

衛宏字敬仲,東海人也。少與河南鄭興俱好古學。

初,九江謝曼卿善毛詩,乃為其訓。宏從曼卿受學,因作毛詩序,善得風雅之旨,于今傳於世。後從大司空杜林更受古文尚書,為作訓旨。時濟南徐巡師事宏,後從林受學,亦以儒顯,由是古學大興。光武以為議郎。

宏作漢舊儀四篇,以載西京雜事;又著賦、頌、誄七首,皆傳於世。

中興後,鄭眾、賈逵傳毛詩,後馬融作毛詩傳,鄭玄作毛詩箋。

前書魯高堂生,漢興傳禮十七篇。後瑕丘蕭奮以授同郡后蒼,蒼授梁人戴德及德兄子聖、沛人慶普。於是德為大戴禮,聖為小戴禮,普為慶氏禮,三家皆立博士。孔安國所獻禮古經五十六篇及周官經六篇,前世傳其書,未有名家。中興已後,亦有大、小戴博士,雖相傳不絕,然未有顯於儒林者。建武中,曹充習慶氏學,傳其子褒,遂撰漢禮,事在褒傳。

董鈞字文伯,犍為資中人也。習慶氏禮。事大鴻臚王臨。元始中,舉明經,遷廩犧令,病去官。建武中,舉孝廉,辟司徒府。

鈞博通古今,數言政事。永平初,為博士。時草創五郊祭祀,及宗廟禮樂,威儀章服,輒令鈞參議,多見從用,當世稱為通儒。累遷五官中郎將,常教授門生百餘人。後坐事左轉騎都尉。年七十餘,卒於家。

中興,鄭眾傳周官經,後馬融作周官傳,授鄭玄,玄作周官注。玄本習小戴禮,後以古經校之,取其義長者,故為鄭氏學。玄又注小戴所傳禮記四十九篇,通為三禮焉。

前書齊胡母子都傳公羊春秋,授東平嬴公,嬴公授東海孟卿,孟卿授魯人眭孟,眭孟授東海嚴彭祖、魯人顏安樂。彭祖為春秋嚴氏學,安樂為春秋顏氏學,又瑕丘江公傳穀梁春秋,三家皆立博士。梁太傅賈誼為春秋左氏傳訓詁,授趙人貫公。

丁恭字子然,山陽東緡人也。習公羊嚴氏春秋。恭學義精明,教授常數百人,州郡請召不應。建武初,為諫議大夫、博士,封關內侯。十一年,遷少府。諸生自遠方至者,著錄數千人,當世稱為大儒。太常樓望、侍中承宮、長水校尉樊儵等皆受業於恭。二十年,拜侍中祭酒、騎都尉,與侍中劉昆俱在光武左右,每事諮訪焉。卒於官。

周澤字稚都,北海安丘人也。少習公羊嚴氏春秋,隱居教授,門徒常數百人。建武末,辟大司馬府,署議曹祭酒。數月,徵試博士。中元元年,遷黽池令。奉公剋己,矜恤孤羸,吏人歸愛之。永平五年,遷右中郎將。十年,拜太常。

澤果敢直言,數有據爭。後北地太守廖信坐貪穢下獄,沒入財產,顯宗以信臧物班諸廉史,唯澤及光祿勳孫堪、大司農常沖特蒙賜焉。是時京師翕然,在位者咸自勉勵。

堪字子稚,河南緱氏人也。明經學,有志操,清白貞正,愛士大夫,然一毫未嘗取於人,以節介氣勇自行。王莽末,兵革並起,宗族老弱在營保閒,堪常力戰陷敵,無所回避,數被創刃,宗族賴之,郡中咸服其義勇。

建武中,仕郡縣。公正廉絜,奉祿不及妻子,皆以供賓客。及為長吏,所在有跡,為吏人所敬仰。喜分明去就。嘗為縣令,謁府,趨步遲緩,門亭長譴堪御吏,堪便解印綬去,不之官。後復仕為左馮翊,坐遇下促急,司隸校尉舉奏免官。數月,徵為侍御史,再遷尚書令。永平十一年,拜光祿勳。

堪清廉,果於從政,數有直言,多見納用。十八年,以病乞身,為侍中騎都尉,卒於官。堪行類於澤,故京師號曰「二稚」。

十二年,以澤行司徒事,如真。澤性簡,忽威儀,頗失宰相之望。數月,復為太常。清絜循行,盡敬宗廟。常臥疾齋宮,其妻哀澤老病,闚問所苦。澤大怒,以妻干犯齋禁,遂收送詔獄謝罪。當世疑其詭激。時人為之語曰:「生世不諧,作太常妻,一歲三百六十日,三百五十九日齋。」十八年,拜侍中騎都尉。後數為三老五更。建初中致仕,卒於家。

鍾興字次文,汝南汝陽人也。少從少府丁恭受嚴氏春秋。恭薦興學行高明,光武召見,問以經義,應對甚明。帝善之,拜郎中,稍遷左中郎將。詔令定春秋章句,去其復重,以授皇太子。又使宗室諸侯從興受章句。封關內侯。興自以無功,不敢受爵。帝曰:「生教訓太子及諸王侯,非大功邪?」興曰:「臣師丁恭。」於是復封恭,而興遂固辭不受爵,卒於官。

甄宇字長文,北海安丘人也。清靜少欲。習嚴氏春秋,教授常數百人。建武中,為州從事,徵拜博士,稍遷太子少傅,卒於官。

傳業子普,普傳子承。承尤篤學,未嘗視家事,講授常數百人。諸儒以承三世傳業,莫不歸服之。建初中,舉孝廉,卒於梁相。子孫傳學不絕。

樓望字次子,陳留雍丘人也。少習嚴氏春秋。操節清白,有稱鄉閭。建武中,趙節王栩聞其高名,遣使齎玉帛請以為師,望不受。後仕郡功曹。永平初,為侍中、越騎校尉,入講省內。十六年,遷大司農。十八年,代周澤為太常。建初五年,坐事左轉太中大夫,後為左中郎將。教授不倦,世稱儒宗,諸生著錄九千餘人。年八十,永元十二年,卒於官,門生會葬者數千人,儒家以為榮。

程曾字秀升,豫章南昌人也。受業長安,習嚴氏春秋,積十餘年,還家講授。會稽顧奉等數百人常居門下。著書百餘篇,皆五經通難,又作孟子章句。建初三年,舉孝廉,遷海西令,卒於官。

張玄字君夏,河內河陽人也。少習顏氏春秋,兼通數家法。建武初,舉明經,補弘農文學,遷陳倉縣丞。清淨無欲,專心經書,方其講問,乃不食終日。及有難者,輒為張數家之說,令擇從所安。諸儒皆伏其多通,著錄千餘人。

玄初為縣丞,嘗以職事對府,不知官曹處,吏白門下責之。時右扶風琅邪徐業,亦大儒也,聞玄諸生,試引見之,與語,大驚曰:「今日相遭,真解矇矣!」遂請上堂,難問極日。

後玄去官,舉孝廉,除為郎。會顏氏博士缺,玄試策第一,拜為博士。居數月,諸生上言玄兼說嚴氏、宣氏,不宜專為顏氏博士。光武且令還署,未及遷而卒。

李育字元春,扶風漆人也。少習公羊春秋。沈思專精,博覽書傳,知名太學,深為同郡班固所重。固奏記薦育於驃騎將軍東平王蒼,由是京師貴戚爭往交之。州郡請召,育到,輒辭病去。

常避地教授,門徒數百。頗涉獵古學。嘗讀左氏傳,雖樂文采,然謂不得聖人深意,以為前世陳元、范升之徒更相非折,而多引圖讖,不據理體,於是作難左氏義四十一事。

建初元年,衛尉馬廖舉育方正,為議郎。後拜博士。四年,詔與諸儒論五經於白虎觀,育以公羊義難賈逵,往返皆有理證,最為通儒。

再遷尚書令。及馬氏廢,育坐為所舉免歸。歲餘復徵,再遷侍中,卒於官。

何休字邵公,任城樊人也。父豹,少府。休為人質朴訥口,而雅有心思,精研六經,世儒無及者。以列卿子詔拜郎中,非其好也,辭疾而去。不仕州郡。進退必以禮。

太傅陳蕃辟之,與參政事。蕃敗,休坐廢錮,乃作春秋公羊解詁,覃思不闚門,十有七年。又注訓孝經、論語、風角七分,皆經緯典謨,不與守文同說。又以春秋駮漢事六百餘條,妙得公羊本意。休善歷筭,與其師博士羊弼,追述李育意以難二傳,作公羊墨守、左氏膏肓、穀梁廢疾。

黨禁解,又辟司徒。群公表休道術深明,宜侍帷幄,倖臣不悅之,乃拜議郎,屢陳忠言。再遷諫議大夫,年五十四,光和五年卒。

服虔字子慎,初名重,又名祇,後改為虔,河南滎陽人也。少以清苦建志,入太學受業。有雅才,善著文論,作春秋左氏傳解,行之至今。又以左傳駮何休之所駮漢事六十條。舉孝廉,稍遷,中平末,拜九江太守。免,遭亂行客,病卒。所著賦、碑、誄、書記、連珠、九憤,凡十餘篇。

潁容字子嚴,陳國長平人也。博學多通,善春秋左氏,師事太尉楊賜。郡舉孝廉,州辟,公車徵,皆不就。初平中,避亂荊州,聚徒千餘人。劉表以為武陵太守,不肯起。著春秋左氏條例五萬餘言,建安中卒。

謝該字文儀,南陽章陵人也。善明春秋左氏,為世名儒,門徒數百千人。建安中,河東人樂詳條左氏疑滯數十事以問,該皆為通解之,名為謝氏釋,行於世。

仕為公車司馬令,以父母老,託疾去官。欲歸鄉里,會荊州道斷,不得去。少府孔融上書薦之曰:「臣聞高祖創業,韓、彭之將征討暴亂,陵賈、叔孫通進說詩書。光武中興,吳、耿佐命,范升、衛宏脩述舊業,故能文武並用,成長久之計。陛下聖德欽明,同符二祖,勞謙厄運,三年乃讙。今尚父鷹揚,方叔翰飛,王師電鷙,群凶破殄,始有櫜弓臥鼓之次,宜得名儒,典綜禮紀。竊見故公車司馬令謝該,體曾、史之淑性,兼商、偃之文學,博通群蓺,周覽古今,物來有應,事至不惑,清白異行,敦悅道訓。求之遠近,少有疇匹。若乃巨骨出吳,隼集陳庭,黃能入寢,亥有二首,非夫洽聞者,莫識其端也。雋不疑定北闕之前,夏侯勝辯常陰之驗,然後朝士益重儒術。今該實卓然比跡前列,閒以父母老疾,棄官欲歸,道路險塞,無由自致。猥使良才抱樸而逃,踰越山河,沈淪荊楚,所謂往而不反者也。後日當更饋樂以釣由余,剋像以求傅說,豈不煩哉?臣愚以為可推錄所在,召該令還。楚人止孫卿之去國,漢朝追匡衡於平原,尊儒貴學,惜失賢也。」書奏,詔即徵還,拜議郎。以壽終。

建武中,鄭興、陳元傳春秋左氏學。時尚書令韓歆上疏,欲為左氏立博士,范升與歆爭之未決,陳元上書訟左氏,遂以魏郡李封為左氏博士。後群儒蔽固者數廷爭之。及封卒,光武重違眾議,而因不復補。

許慎字叔重,汝南召陵人也。性淳篤,少博學經籍,馬融常推敬之,時人為之語曰:「五經無雙許叔重。」為郡功曹,舉孝廉,再遷除洨長。卒于家。

初,慎以五經傳說臧否不同,於是撰為五經異義,又作說文解字十四篇,皆傳於世。

蔡玄字叔陵,汝南南頓人也。學通五經,門徒常千人,其著錄者萬六千人。徵辟並不就。順帝特詔徵拜議郎,講論五經異同,甚合帝意。遷侍中,出為弘農太守,卒官。

論曰:自光武中年以後,干戈稍戢,專事經學,自是其風世篤焉。其服儒衣,稱先王,遊庠序,聚橫塾者,蓋布之於邦域矣。若乃經生所處,不遠萬里之路,精廬暫建,贏糧動有千百,其耆名高義開門受徒者,編牒不下萬人,皆專相傳祖,莫或訛雜。至有分爭王庭,樹朋私里,繁其章條,穿求崖穴,以合一家之說。故楊雄曰:「今之學者,非獨為之華藻,又從而繡其鞶帨。」夫書理無二,義歸有宗,而碩學之徒,莫之或徙,故通人鄙其固焉,又雄所謂「譊譊之學,各習其師」也。且觀成名高第,終能遠至者,蓋亦寡焉,而迂滯若是矣。然所談者仁義,所傳者聖法也。故人識君臣父子之綱,家知違邪歸正之路。

自桓、靈之閒,君道秕僻,朝綱日陵,國隙屢啟,自中智以下,靡不審其崩離;而權彊之臣,息其闚盜之謀,豪俊之夫,屈於鄙生之議者,人誦先王言也,下畏逆順埶也。至如張溫、皇甫嵩之徒,功定天下之半,聲馳四海之表,俯仰顧眄,則天業可移,猶鞠躬昏主之下,狼狽折札之命,散成兵,就繩約,而無悔心。暨乎剝橈自極,人神數盡,然後群英乘其運,世德終其祚。跡衰敝之所由致,而能多歷年所者,斯豈非學之效乎?故先師垂典文,褒勵學者之功,篤矣切矣。不循春秋,至乃比於殺逆,其將有意乎!

贊曰:斯文未陵,亦各有承。塗分流別,專門並興。精疏殊會,通閡相徵。千載不作,淵原誰澂?


\end{pinyinscope}