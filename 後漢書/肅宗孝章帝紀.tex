\article{肅宗孝章帝紀}

\begin{pinyinscope}
肅宗孝章皇帝諱炟,顯宗第五子也。母賈貴人。永平三年,立為皇太子。少寬容,好儒術,顯宗器重之。

十八年八月壬子,即皇帝位,年十九。尊皇后曰皇太后。

壬戌,葬孝明皇帝于顯節陵。

冬十月丁未,大赦天下。賜民爵,人二級,為父後及孝悌、力田人三級,脫無名數及流人欲占者人一級,爵過公乘得移與子若同產子;鰥、寡、孤、獨、篤缮、貧不能自存者粟,人三斛。詔曰:「朕以眇身,託于王侯之上,統理萬機,懼失厥中,兢兢業業,未知所濟。深惟守文之主,必建師傅之官。詩不云乎:『不愆不忘,率由舊章。』行太尉事節鄉侯憙三世在位,為國元老;司空融典職六年,勤勞不怠。其以憙為太傅,融為太尉,並錄尚書事。『三事大夫,莫肯夙夜』,小雅之所傷也。『予違汝弼,汝無面從』,股肱之正義也。群后百僚勉思厥職,各貢忠誠,以輔不逮。申敕四方,稱朕意焉。」

十一月戊戌,蜀郡太守第五倫為司空。

詔征西將軍耿秉屯酒泉。遣酒泉太守段彭救戊己校尉耿恭。

甲辰晦,日有食之。於是避正殿,寑兵,不聽事五日。詔有司各上封事。

十二月癸巳,有司奏言:「孝明皇帝聖德淳茂,劬勞日昃,身御浣衣,食無兼珍。澤臻四表,遠人慕化,僬僥、儋耳,款塞自至。克伐鬼方,開道西域,威靈廣被,無思不服。以烝庶為憂,不以天下為樂。備三雍之教,躬養老之禮。作登歌,正予樂,博貫六蓺,不舍晝夜。聰明淵塞,著在圖讖。至德所感,通於神明。功烈光於四海,仁風行於千載。而深執謙謙,自稱不德,無起寑廟,埽地而祭,除日祀之法,省送終之禮,遂藏主於光烈皇后更衣別室。天下聞之,莫不悽愴。陛下至孝烝烝,奉順聖德。臣愚以為更衣在中門之外,處所殊別,宜尊廟曰顯宗,其四時禘祫,於光武之堂,閒祀悉還更衣,共進武德之舞,如孝文皇帝祫祭高廟故事。」制曰:「可。」

是歲,牛疫。京師及三州大旱,詔勿收兗、豫、徐州田租、芻稿,其以見穀賑給貧人。

建初元年春正月,詔三州郡國:「方春東作,恐人稍受稟,往來煩劇,或妨耕農。其各實覈尤貧者,計所貸并與之。流人欲歸本者,郡縣其實稟,令足還到,聽過止官亭,無雇舍宿。長吏親躬,無使貧弱遺脫,小吏豪右得容姦妄。詔書既下,勿得稽留,刺史明加督察尤無狀者。」

丙寅,詔曰:「比年牛多疾疫,墾田減少,穀價頗貴,人以流亡。方春東作,宜及時務。二千石勉勸農桑,弘致勞來。群公庶尹,各推精誠,專急人事。罪非殊死,須立秋案驗。有司明慎選舉,進柔良,退貪猾,順時令,理冤獄。『五教在寬』,帝典所美;『愷悌君子』,大雅所歎。布告天下,使明知朕意。」

酒泉太守段彭討擊車師,大破之。罷戊己校尉官。

二月,武陵澧中蠻叛。

三月甲寅,山陽、東平地震。己巳,詔曰:「朕以無德,奉承大業,夙夜慄慄,不敢荒寧。而災異仍見,與政相應。朕既不明,涉道日寡;又選舉乖實,俗吏傷人,官職秏亂,刑罰不中,可不憂與!昔仲弓季氏之家臣,子游武城之小宰,孔子猶誨以賢才,問以得人。明政無大小,以得人為本。夫鄉舉里選,必累功勞。今刺史、守相不明真偽,茂才、孝廉歲以百數,既非能顯,而當授之政事,甚無謂也。每尋前世舉人貢士,或起甽畝,不繫閥閱。敷奏以言,則文章可採;明試以功,則政有異跡。文質彬彬,朕甚嘉之。其令太傅、三公、中二千石、二千石、郡國守相舉賢良方正、能直言極諫之士各一人。」

夏五月辛酉,初舉孝廉、郎中寬博有謀,任典城者,以補長、相。

秋七月辛亥,詔以上林池烃田賦與貧人。

八月庚寅,有星孛于天市。

九月,永昌哀牢夷叛。

冬十月,武陵郡兵討叛蠻,破降之。

十一月,阜陵王延謀反,貶為阜陵侯。

二年春三月辛丑,詔曰:「比年陰陽不調,肌饉屢臻。深惟先帝憂人之本,詔書曰『不傷財,不害人』,誠欲元元去末歸本。而今貴戚近親,奢縱無度,嫁聚送終,尤為僭侈。有司廢典,莫肯舉察。春秋之義,以貴理賤。今自三公,並宜明糾非法,宣振威風。朕在弱冠,未知稼穡之艱難,區區管窺,豈能照一隅哉!其科條制度所宜施行,在事者備為之禁,先京師而後諸夏。」

甲辰,罷伊吾盧屯兵。

永昌、越巂、益州三郡民、夷討哀牢,破平之。

夏四月戊子,詔還坐楚、淮陽事徙者四百餘家,令歸本郡。

癸巳,詔齊相省冰紈、方空縠、吹綸絮。

六月,燒當羌叛,金城太守郝崇討之,敗績,羌遂寇漢陽。秋八月,遺行車騎將軍馬防討平之。

十二月戊寅,有星孛于紫宮。

三年春正月己酉,宗祀明堂。禮畢,登靈臺,望雲物。大赦天下。

三月癸巳,立貴人竇氏為皇后。賜爵,人二級,三老、孝悌、力田人三級,民無名數及流民欲占者人一級;鰥、寡、孤、獨、篤缮、貧不能自存者粟,人五斛。

夏四月己巳,罷常山呼沱石臼河漕。

行車騎將軍馬防破燒當羌於臨洮。

閏月,西域假司馬班超擊姑墨,大破之。

冬十二月丁酉,以馬防為車騎將軍。

武陵漊中蠻叛。

是歲,零陵獻芝草。

四年春二月庚寅,太尉牟融薨。

夏四月戊子,立皇子慶為皇太子。賜爵,人二級,三老、孝悌、力田人三級,民無名數及流人欲自占者人一級;鰥、寡、孤、獨、篤缮、貧不能自存者粟,人五斛。

己丑,徙鉅鹿王恭為江陵王,汝南王暢為梁王,常山王疠為淮陽王。辛卯,封皇子伉為千乘王,全為平春王。

五月丙辰,車騎將軍馬防罷。

甲戌,司徒鮑昱為太尉,南陽太守桓虞為司徒。

六月癸丑,皇太后馬氏崩。秋七月壬戌,葬明德皇太后。

冬,牛大疫。

十一月壬戌,詔曰:「蓋三代導人,教學為本。漢承暴秦,褒顯儒術,建立五經,為置博士。其後學者精進,雖曰承師,亦別名家。孝宣皇帝以為去聖久遠,學不厭博,故遂立大、小夏侯尚書,後又立京氏易。至建武中,復置顏氏、嚴氏春秋,大、小戴禮博士。此皆所以扶進微學,尊廣道蓺也。中元元年詔書,五經章句煩多,議欲減省。至永平元年,長水校尉鯈奏言,先帝大業,當以時施行。欲使諸儒共正經義,頗令學者得以自助。孔子曰:『學之不講,是吾憂也。』又曰:『博學而篤志,切問而近思,仁在其中矣。』於戲,其勉之哉!」於是下太常,將、大夫、博士、議郎、郎官及諸生、諸儒會白虎觀,講議五經同異,使五官中郎將魏應承制問,侍中淳于恭奏,帝親稱制臨決,如孝宣甘露石渠故事,作白虎議奏。

是歲,甘露降泉陵、洮陽二縣。

五年春二月庚辰朔,日有食之。詔曰:「朕新離供養,愆咎眾著,上天降異,大變隨之。詩不云乎:『亦孔之醜。』又久旱傷麥,憂心慘切。公卿已下,其舉直言極諫、能指朕過失者各一人,遣詣公車,將親覽問焉。其以巖穴為先,勿取浮華。」

甲申,詔曰:「春秋書『無麥苗』,重之也。去秋雨澤不適,今時復旱,如炎如焚。凶年無時,而為備未至。朕之不德,上累三光,震慄忉忉,痛心疾首。前代聖君,博思咨諏,雖降災咎,輒有開匱反風之應。令予小子,徒慘慘而已。其令二千石理冤獄,錄輕繫;禱五嶽四瀆,及名山能興雲致雨者,冀蒙不崇朝遍雨天下之報。務加肅敬焉。」

三月甲寅,詔曰:「孔子曰:『刑罰不中,則人無所措手足。』今吏多不良,擅行喜怒,或案不以罪,迫脅無辜,致令自殺者,一歲且多於斷獄,甚非為人父母之意也。有司其議糾舉之。」

荊、豫諸郡兵討破武陵漊中叛蠻。

夏五月辛亥,詔曰:「朕思遲直士,側席異聞。其先至者,各以發憤吐懣,略聞子大夫之志矣,皆欲置於左右,顧問省納。建武詔書又曰,堯試臣以職,不直以言語筆札。今外官多曠,並可以補任。」

戊辰,太傅趙憙薨。

冬,始行月令迎氣樂。

是歲,零陵獻芝草。有八黃龍見於泉陵。西域假司馬班超擊疏勒,破之。

六年春二月辛卯,琅邪王京薨。

夏五月辛酉,趙王盱薨。

六月丙辰,太尉鮑昱薨。

辛未晦,日有食之。

秋七月癸巳,以大司農鄧彪為太尉。

七年春正月,沛王輔、濟南王康、東平王蒼、中山王焉、東海王政、琅邪王宇來朝。

夏六月甲寅,廢皇太子慶為清河王,立皇子肇為皇太子。

己未,徙廣平王羨為西平王。

秋八月,飲酎高廟,禘祭光武皇帝、孝明皇帝。甲辰,詔:「書云『祖考來假』,明哲之祀。予末小子,質又菲薄,仰惟先帝烝烝之情,前修禘祭,以盡孝敬。朕得識昭穆之序,寄遠祖之思。今年大禮復舉,加以先帝之坐,悲傷感懷。樂以迎來,哀以送往,雖祭亡如在,而空虛不知所裁,庶或饗之。豈亡克慎肅雍之臣,辟公之相,皆助朕之依依。今賜公錢四十萬,卿半之,及百官執事各有差。」

九月甲戌,幸偃師,東涉卷津,至河內。下詔曰:「車駕行秋稼,觀收穫,因涉郡界。皆精騎輕行,無它輜重。不得輒修道橋,遠離城郭,遣吏逢迎,刺探起居,出入前後,以為煩擾。動務省約,但患不能脫粟瓢飲耳。所過欲令貧弱有利,無違詔書。」遂覽淇園。己酉,進幸鄴,勞饗魏郡守令已下,至于三老、門闌、走卒,賜錢各有差。勞賜常山、趙國吏人,復元氏租賦三歲。辛卯,車駕還宮。詔天下繫囚減死一等,勿笞,詣邊戍;妻子自隨,占著所在;父母同產欲相從者,恣聽之;有不到者,皆以乏軍興論。及犯殊死,一切募下蠶室;其女子宮。繫囚鬼薪、白粲已上,皆減本罪各一等,輸司寇作。亡命贖:死罪入縑二十匹,右趾至髡鉗城旦舂十匹,完城旦至司寇三匹,吏人有罪未發覺,詔書到自告者,半入贖。

冬十月癸丑,西巡狩,幸長安。丙辰,祠高廟,遂有事十一陵。遣使者祠太上皇於萬年,以中牢祠蕭何、霍光。進幸槐里。岐山得銅器,形似酒樽,獻之。又獲白鹿。帝曰:「上無明天子,下無賢方伯。『人之無良,相怨一方。』斯器亦曷為來哉?」又幸長平,御池陽宮,東至高陵,造舟於涇而還。每所到幸,輒會郡縣吏人,勞賜作樂。十一月,詔勞賜河東守、令、掾以下。十二月丁亥,車駕還宮。

是歲,京師及郡國螟。

八年春正月壬辰,東平王蒼薨。三月辛卯,葬東平憲王,賜鑾輅、龍旂。

夏六月,北匈奴大人率眾款塞降。

冬十二月甲午,東巡狩,幸陳留、梁國、淮陽、潁陽。戊申,車駕還宮。

詔曰:「五經剖判,去聖彌遠,章句遺辭,乖疑難正,恐先師微言將遂廢絕,非所以重稽古,求道真也。其令群儒選高才生,受學左氏、穀梁春秋,古文尚書,毛詩,以扶微學,廣異義焉。」

是歲,京師及郡國螟。

元和元年春正月,中山王焉來朝。日南徼外蠻夷獻生犀、白雉。

閏月辛丑,濟陰王長薨。

二月甲戌,詔曰:「王者八政,以食為本,故古者急耕稼之業,致耒耜之勤,節用儲蓄,以備凶災,是以歲雖不登而人無飢色。自牛疫已來,穀食連少,良由吏教未至,刺史、二千石不以為負。其令郡國募人無田欲徙它界就肥饒者,恣聽之。到在所,賜給公田,為雇耕傭,賃種餉,貰與田器,勿收租五歲,除筭三年。其後欲還本鄉者,勿禁。」

夏四月己卯,分東平國,封憲王蒼子尚為任城王。

六月辛酉,沛王輔薨。

秋七月丁未,詔曰:「律云『掠者唯得榜、笞、立』。又令丙,箠長短有數。自往者大獄已來,掠考多酷,鉆鑽之屬,慘苦無極。念其痛毒,怵然動心。書曰『鞭作官刑』,豈云若此?宜及秋冬理獄,明為其禁。」

八月甲子,太尉鄧彪罷,大司農鄭弘為太尉。

癸酉,詔曰:「朕道化不德,吏政失和,元元未諭,抵罪於下。寇賊爭心不息,邊野邑屋不修。永惟庶事,思稽厥衷,與凡百君子,共弘斯道。中心悠悠,將何以寄?其改建初九年為元和元年。郡國中都官繫囚減死一等,勿笞,詣邊縣;妻子自隨,占著在所。其犯殊死,一切募下蠶室;其女子宮。繫囚鬼薪、白粲以上,皆減本罪一等,輸司寇作。亡命者贖,各有差。」

丁酉,南巡狩,詔所經道上,郡縣無得設儲跱。命司空自將徒支柱橋梁。有遣使奉迎,探知起居,二千石當坐。其賜鰥、寡、孤、獨、不能自存者粟,人五斛。

九月乙未,東平王忠薨。

辛丑,幸章陵,祠舊宅園廟,見宗室故人,賞賜各有差。冬十月己未,進幸江陵,詔廬江太守祠南嶽,又詔長沙、零陵太守祠長沙定王、舂陵節侯、鬱林府君。還,幸宛。十一月己丑,車駕還宮,賜從者各有差。

十二月壬子,詔曰:「書云:『父不慈,子不祗,兄不友,弟不恭,不相及也。』往者妖言大獄,所及廣遠,一人犯罪,禁至三屬,莫得垂纓仕宦王朝。如有賢才而沒齒無用,朕甚憐之,非所謂與之更始也。諸以前妖惡禁錮者,一皆蠲除之,以明棄咎之路,但不得在宿衛而已。」

二年春正月乙酉,詔曰:「令云『人有產子者復,勿筭三歲』。今諸懷妊者,賜胎養穀人三斛,復其夫,勿筭一歲,著以為令。」又詔三公曰:「方春生養,萬物莩甲,宜助萌陽,以育時物。其令有司,罪非殊死且勿案驗,及吏人條書相告不得聽受,冀以息事寧人,敬奉天氣。立秋如故。夫俗吏矯飾外貌,似是而非,揆之人事則悅耳,論之陰陽則傷化,朕甚饜之,甚苦之。安靜之吏,悃愊無華,日計不足,月計有餘。如襄城令劉方,吏人同聲謂之不煩,雖未有它異,斯亦殆近之矣。閒敕二千石各尚寬明,而今富姦行賂於下,貪吏枉法於上,使有罪不論而無過被刑,甚大逆也。夫以苛為察,以刻為明,以輕為德,以重為威,四者或興,則下有怨心。吾詔書數下,冠蓋接道,而吏不加理,人或失職,其咎安在?勉思舊令,稱朕意焉。」

二月甲寅,始用四分歷。

詔曰:「今山川鬼神應典禮者,尚未咸秩。其議增修群祀,以祈豐年。」

丙辰,東巡狩。己未,鳳皇集肥城。乙丑,帝耕於定陶。詔曰:「三老,尊年也。孝悌,淑行也。力田,勤勞也。國家甚休之。其賜帛人一匹,勉率農功。」使使者祠唐堯於成陽靈臺。辛未,幸太山,柴告岱宗。有黃鵠三十從西南來,經祠壇上,東北過于宮屋,翱翔升降。進幸奉高。壬申,宗祀五帝于汶上明堂。癸酉,告祠二祖、四宗,大會外內群臣。丙子,詔曰:「朕巡狩岱宗,柴望山川,告祀明堂,以章先勳。其二王之後,先聖之胤,東后蕃衛,伯父伯兄,仲叔季弟,幼子童孫,百僚從臣,宗室眾子,要荒四裔,沙漠之北,蔥領之西,冒耏之類,跋涉懸度,陵踐阻絕,駿奔郊畤,咸來助祭。祖宗功德,延及朕躬。予一人空虛多疚,纂承尊明,盥洗享薦,慚愧祗慄。詩不云乎:『君子如祉,亂庶遄已。』歷數既從,靈燿著明,亦欲與士大夫同心自新。其大赦天下。諸犯罪不當得赦者,皆除之。復博、奉高、嬴,無出今年田租、芻稿。」戊寅,進幸濟南。三月己丑,進幸魯,祠東海恭王陵。庚寅,祠孔子於闕里,及七十二弟子,賜褒成侯及諸孔男女帛。壬辰,進幸東平,祠憲王陵。甲午,遣使者祠定陶太后、恭王陵。乙未,幸東阿,北登太行山,至天井關。夏四月乙巳,客星入紫宮。乙卯,車駕還宮。庚申,假于祖禰,告祠高廟。

五月戊申,詔曰:「乃者鳳皇、黃龍、鸞鳥比集七郡,或一郡再見,及白烏、神雀、甘露屢臻。祖宗舊事,或班恩施。其賜天下吏爵,人三級;高年、鰥、寡、孤、獨帛,人一匹。經曰:「無侮鰥寡,惠此煢獨。』加賜河南女子百戶牛酒,令天下大酺五日。賜公卿已下錢帛各有差;及洛陽人當酺者布,戶一匹,城外三戶共一匹。賜博士員弟子見在太學者布,人三匹。令郡國上明經者,口十萬以上五人,不滿十萬三人。」

改廬江為六安國,江陵復為南郡。徙江陵王恭為六安王。

秋七月庚子,詔曰:「春秋於春每月書『王』者,重三正,慎三微也。律十二月立春,不以報囚。月令冬至之後,有順陽助生之文,而無鞠獄斷刑之政。朕咨訪儒雅,稽之典籍,以為王者生殺,宜順時氣。其定律,無以十一月、十二月報囚。」

九月壬辰,詔:「鳳皇、黃龍所見亭部無出二年租賦。加賜男子爵,人二級;先見者帛二十匹,近者三匹,太守三十匹,令、長十五匹,丞、尉半之。《詩》云:『雖無德與汝,式歌且舞。』它如賜爵故事。」

丙申,徵濟南王康、中山王焉會烝祭。

冬十一月壬辰,日南至,初閉關梁。

三年春正月乙酉,詔曰:「蓋君人者,視民如父母,有憯怛之憂,有忠和之教,匍匐之救。其嬰兒無父母親屬,及有子不能養食者,稟給如律。」

丙申,北巡狩,濟南王康、中山王焉、西平王羨、六安王恭、樂成王黨、淮陽王疠、任城王尚、沛王定皆從。辛丑,帝耕于懷。

二月壬寅,告常山、魏郡、清河、鉅鹿、平原、東平郡太守、相曰:「朕惟巡狩之制,以宣聲教,考同遐樯,解釋怨結也。今『四國無政,不用其良』,駕言出游,欲親知其劇易。前祠園陵,遂望祀華、霍,東祡岱宗,為人祈福。今將禮常山,遂徂北土,歷魏郡,經平原,升踐隄防,詢訪耆老,咸曰『往者汴門未作,深者成淵,淺則泥塗』。追惟先帝勤人之德,底績遠圖,復禹弘業,聖跡滂流,至于海表。不克堂桓,朕甚慚焉。月令,孟春善相丘陵土地所宜。今肥田尚多,未有墾闢。其悉以賦貧民,給與糧種,務盡地力,勿令游手。所過縣邑,聽半入今年田租,以勸農夫之勞。」

乙丑,敕侍御史、司空曰:「方春,所過無得有所伐殺。車可以引避,引避之;騑馬可輟解,輟解之。《詩》云:『敦彼行葦,牛羊勿踐履。』禮,人君伐一草木不時,謂之不孝。俗知順人,莫知順天。其明稱朕意。」

戊辰,進幸中山,遣使者祠北嶽,出長城。癸酉,還幸元氏,祠光武、顯宗於縣舍正堂;明日又祠顯宗于始生堂,皆奏樂。三月丙子,詔高邑令祠光武於即位壇。復元氏七年傜役。己卯,進幸趙。庚辰,祠房山於靈壽。辛卯,車駕還宮。賜從行者各有差。

夏四月丙寅,太尉鄭弘免,大司農宋由為太尉。

五月丙子,司空第五倫罷,太僕袁安為司空。

秋八月乙丑,幸安邑,觀鹽池。九月,至自安邑。

冬十月,北海王基薨。

燒當羌叛,寇隴西。

是歲,西域長史班超擊斬疏勒王。

章和元年春三月,護羌校尉傅育追擊叛羌,戰歿。

夏四月丙子,令郡國中都官繫囚減死一等,詣金城戍。

六月戊辰,司徒桓虞免。癸卯,司空袁安為司徒,光祿勳任隗為司空。

秋七月癸卯,齊王晃有罪,貶為蕪湖侯。壬子,淮陽王疠薨。

鮮卑擊破北單于,斬之。

燒當羌寇金城,護羌校尉劉盱討之,斬其渠帥。

壬戌,詔曰:「朕聞明君之德,啟迪鴻化,緝熙康乂,光照六幽,訖惟人面,靡不率俾,仁風翔于海表,威霆行乎鬼區。然後敬恭明祀,膺五福之慶,獲來儀之貺。朕以不德,受祖宗弘烈。乃者鳳皇仍集,麒麟並臻,甘露宵降,嘉穀滋生,芝草之類,歲月不絕。朕夙夜祗畏上天,無以彰于先功。今改元和四年為章和元年。」

秋,令是月養衰老,授几杖,行糜粥飲食。其賜高年二人共布帛各一匹,以為醴酪。死罪囚犯法在丙子赦前而後捕繫者,皆減死,勿笞,詣金城戍。

八月癸酉,南巡狩。壬午,遣使者祠昭靈后於小黃園。甲申,徵任城王尚會睢陽。戊子,幸梁。己丑,遣使祠沛高原廟,豐枌榆社。乙未,幸沛,祠獻王陵,徵會東海王政。乙未晦,日有食之。九月庚子,幸彭城,東海王政、沛王定、任城王尚皆從。辛亥,幸壽春。壬子,詔郡國中都官繫囚減死罪一等,詣金城戍;犯殊死者,一切募下蠶室;其女子宮;繫囚鬼薪、白粲已上,減罪一等,輸司寇作。亡命者贖:死罪縑二十匹,右趾至髡鉗城旦舂七匹,完城旦至司寇三匹;吏民犯罪未發覺,詔書到自告者,半入贖。復封阜陵侯延為阜陵王。己未,幸汝陰。冬十月丙子,車駕還宮。

北匈奴屋蘭儲等率眾降。

是歲,西域長史班超擊莎車,大破之。月氏國遣使獻扶拔、師子。

二年春正月,濟南王康、阜陵王延、中山王焉來朝。

壬辰、帝崩於章德前殿,年三十三。遺詔無起寑廟,一如先帝法制。

論曰:魏文帝稱「明帝察察,章帝長者」。章帝素知人厭明帝苛切,事從寬厚。感陳寵之義,除慘獄之科。深元元之愛,著胎養之令。奉承明德太后,盡心孝道。割裂名都,以崇建周親。平傜簡賦,而人賴其慶。又體之以忠恕,文之以禮樂。故乃蕃輔克諧,群后德讓。謂之長者,不亦宜乎!在位十三年,郡國所上符瑞,合於圖書者數百千所。烏呼懋哉!

贊曰:肅宗濟濟,天性愷悌。於穆后德,諒惟淵體。左右蓺文,斟酌律禮。思服帝道,弘此長懋。儒館獻歌,戎亭虛候。氣調時豫,憲平人富。


\end{pinyinscope}