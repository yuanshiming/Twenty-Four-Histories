\article{五行四}

\begin{pinyinscope}
地震山崩地陷大風拔樹螟牛疫

五行傳曰:「治宮室,飾臺榭,內淫亂,犯親戚,侮父兄,則稼穡不成。」謂土失其性而為災也。又曰:「思心不容,是謂不聖。厥咎霿,厥罰恆風,厥極凶短折。時則有脂夜之妖,時則有華孽,時則有牛禍,時則有心腹之痾,時則有黃眚、黃祥,惟金、水、木、火沴土。」華孽,劉歆傳為蠃蟲之孽,謂螟屬也。

世祖建武二十二年九月,郡國四十二地震,南陽尤甚,地裂壓殺人。其後武谿蠻夷反,為寇害,至南郡,發荊州諸郡兵,遣武威將軍劉尚擊之,為夷所圍,復發兵赴之,尚遂為所沒。

章帝建初元年三月甲申,山陽、東平地震。

和帝永元四年六月丙辰,郡國十三地震。春秋漢含孳曰:「女主盛,臣制命,則地動坼,畔震起,山崩淪。」是時竇太后攝政,兄竇憲專權,將以是受禍也。後五日,詔收憲印綬,兄弟就國,逼迫皆自殺。

五年二月戊午,隴西地震。儒說民安土者也,將大動,行大震。九月,匈奴單于於除難鞬叛,遣使發邊郡兵討之。

七年九月癸卯,京都地震。儒說奄官無陽施,猶婦人也。是時和帝與中常侍鄭眾謀奪竇氏權,德之,因任用之,及幸常侍蔡倫,二人始並用權。

九年三月庚辰,隴西地震。閏月,塞外羌犯塞,殺略吏民,使征西將軍劉尚擊之。

安帝永初元年,郡國十八地震。李固曰:「地者陰也,法當安靜。今乃越陰之職,專陽之政,故應以震動。」是時鄧太后攝政專事,訖建光中,太后崩,安帝乃得制政,於是陰類並勝,西羌亂夏,連十餘年。

二年,郡國十二地震。

三年十二月辛酉,郡國九地震。

四年三月癸巳,郡國四地震。

五年正月丙戌,郡國十地震。

七年正月壬寅,二月丙午,郡國十八地震。

元初元年,郡國十五地震。

二年十一月庚申,郡國十地震。

三年二月,郡國十地震。十一月癸卯,郡國九地震。

四年,郡國十三地震。

五年,郡國十四地震。

六年二月乙巳,京都、郡國四十二地震,或地坼裂,涌水,壞敗城郭、民室屋,壓人。冬,郡國八地震。

永寧元年,郡國二十三地震。

建光元年九月己丑,郡國三十五地震,或地坼裂,壞城郭室屋,壓殺人。是時安帝不能明察,信宮人及阿母聖等讒云,破壞鄧太后家,於是專聽信聖及宦者,中常侍江京、樊豐等皆得用權。

延光元年七月癸卯,京都、郡國十三地震。九月戊申,郡國二十七地震。

二年,京都、郡國三十二地震。

三年,京都、郡國二十三地震。是時以讒免太尉楊震,廢太子。

四年十〈一〉月丁巳,京都、郡國十六地震。時安帝既崩,閻太后攝政,兄弟閻顯等並用事,遂斥安帝子,更徵諸國王子,未至,中黃門遂誅顯兄弟。

順帝永建三年正月丙子,京都、漢陽地震。漢陽屋壞殺人,地坼涌水出。是時順帝阿母宋娥及中常侍張昉等用權。

陽嘉二年四月己亥,京都地震。是時爵號宋娥為山陽君。

四年十二月甲寅,京都地震。

永和二年四月庚申,京都地震。是時宋娥構姦誣罔,五月事覺,收印綬,歸田里。十一月丁卯,京都地震。是時太尉王龔以中常侍張昉等專弄國權,欲奏誅之,時龔宗親有以楊震行事諫之止云。

三年二月乙亥,京都、金城、隴西地震裂,城郭、室屋多壞,壓殺人。閏月己酉,京都地震。十月,西羌二千餘騎入金城塞,為涼州害。

四年三月乙亥,京都地震。

五年二月戊申,京都地震。

建康元年正月,涼州都郡六,地震。從去年九月以來至四月,凡百八十日震,山谷坼裂,壞敗城寺,傷害人物。三月,護羌校尉趙沖為叛胡所殺。九月丙午,京都地震。是時順帝崩,梁太后攝政,欲為順帝作陵,制度奢廣,多壞吏民冢。尚書欒巴諫事,太后怒,癸卯,詔書收巴下獄,欲殺之。丙午地震,於是太后乃出巴,免為庶人。

桓帝建和元年四月庚寅,京都地震。九月丁卯,京都地震。是時梁太后攝政,兄冀持權。至和平元年,太后崩,然冀猶秉政專事,至延熹二年,乃誅滅。

三年九月己卯,地震,庚寅又震。

元嘉元年十一月辛巳,京都地震。

二年正月丙辰,京都地震。十月乙亥,京都地震。

永興二年二月癸卯,京都地震。

永壽二年十二月,京都地震。

延熹四年,京都、右扶風、涼州地震。

五年五月乙亥,京都地震。是時桓帝與中常侍單超等謀誅除梁冀,聽之,並使用事專權。又鄧皇后本小人,性行無恆,苟有顏色,立以為后,後卒坐執左道廢,以憂死。

八年九月丁未,京都地震。

靈帝建寧四年二月癸卯,地震。是時中常侍曹節、王甫等皆專權。

熹平二年六月,地震。

六年十月辛丑,地震。

光和元年二月辛未,地震。四月丙辰,地震。靈帝時宦者專恣。

二年三月,京兆地震。

三年自秋至明年春,酒泉表氏地八十餘動,涌水出,城中官寺民舍皆頓,縣易處,更築城郭。

獻帝初平二年六月丙戌,地震。

興平元年六月丁丑,地震。

和帝永元元年七月,會稽南山崩。會稽,南方大名山也。京房易傳曰:「山崩,陰乘陽,弱勝強也。」劉向以為山陽,君也;水陰,民也;君道崩壞,百姓失所也。劉歆以為崩猶地也。是時竇太后攝政,兄竇憲專權。

七年七月,趙國易陽地裂。京房易傳曰:「地裂者,臣下分離,不肯相從也。」是時南單于眾乖離,漢軍追討。

十二年夏,閏四月戊辰,南郡秭歸山高四百丈,崩填谿,殺百餘人。明年冬,至蠻夷反,遣使募荊州吏民萬餘人擊之。

元興元年五月癸酉,右扶風雍地裂。是後西羌大寇涼州。

殤帝延平元年五月壬辰,河東恒山崩。是時鄧太后專政。秋八月,殤帝崩。

安帝永初元年六月丁巳,河東楊地陷,東西百四十步,南北百二十步,深三丈五尺。

六年六月壬辰,豫章員谿原山崩,各六十三所。

元初元年三月己卯,日南地坼,長百八十二里。其後三年正月,蒼梧、鬱林、合浦盜賊群起,劫略吏民。

二年六月,河南雒陽新城地裂。

延光二年七月,丹陽山崩四十七所。

三年六月庚午,巴郡閬中山崩。

四年十月丙午,蜀郡越巂山崩,殺四百餘人。丙午,天子會日也。是時閻太后攝政。其十一月,中黃門孫程等殺江京,立順帝,誅閻后兄弟,明年,閻后崩。

順帝陽嘉二年六月丁丑,雒陽宣德亭地坼,長八十五丈,近郊地。時李固對策,以為「陰類專恣,將有分離之象,所以附郊城者,事上帝示象以誡陛下也」。是時宋娥及中常侍各用權分爭,後中常侍張逵、蘧政與大將軍梁商爭權,為商作飛語,欲陷之。

桓帝建和元年四月,郡國六地裂,水涌出,井溢,壞寺屋,殺人。時梁太后攝政,兄冀枉殺李固、杜喬。

三年,郡國五山崩。

和平元年七月,廣漢梓潼山崩。

永興二年六月,東海朐山崩。冬十二月,泰山、琅邪盜賊群起。

永壽三年七月,河東地裂,時梁皇后兄冀秉政,桓帝欲自由,內患之。

延熹元年七月乙巳,左馮翊雲陽地裂。

三年五月戊申,漢中山崩。是時上寵恣中常侍單超等。

四年六月庚子,泰山、博尤來山判解。

八年六月丙辰,緱氏地裂。

永康元年五月丙午,雒陽高平永壽亭、上黨泫氏地各裂。是時朝臣患中常侍王甫等專恣。冬,桓帝崩。明年,竇氏等欲誅常侍、黃門,不果,更為所誅。

靈帝建寧四年五月,河東地裂十二處,裂合長十里百七十步,廣者三十餘步,深不見底。

和帝永元五年五月戊寅,南陽大風,拔樹木。

安帝永初元年,大風拔樹。是時鄧太后攝政,以清河王子年少,號精耳,故立之,是為安帝。不立皇太子勝,以為安帝賢,必當德鄧氏也;後安帝親讒,廢免鄧氏,令郡縣迫切,死者八九人,家至破壞。此為瞉霿也,是後西羌亦大亂涼州十有餘年。

二年六月,京都及郡國四十大風拔樹。

三年五月癸酉,京都大風,拔南郊道梓樹九十六枚。

七年八月丙寅,京都大風拔樹。

元初二年二月癸亥,京都大風拔樹。

六年夏四月,沛國、勃海大風,拔樹三萬餘枚。

延光二年三月丙申,河東、潁川大風拔樹。六月壬午,郡國十一大風拔樹。是時安帝親讒,曲直不分。

三年,京都及郡國三十六大風拔樹。

靈帝建寧二年四月癸巳,京都大風雨雹,拔郊道樹十圍已上百餘枚。其後晨迎氣黃郊,道於雒水西橋,逢暴風雨,道鹵簿車或發蓋,百官霑濡,還不至郊,使有司行禮。迎氣西郊,亦壹如此。

中平五年六月丙寅,大風拔樹。

獻帝初平四年六月,右扶風大風,發屋拔木。

中興以來,脂夜之妖無錄者。

章帝七八年閒,郡縣大螟傷稼,語在魯恭傳,而紀不錄也。是時章帝用竇皇后讒,害宋、梁二貴人,廢皇太子。

靈帝熹平四年六月,弘農、三輔螟蟲為害。是時靈帝用中常侍曹節等讒言,禁錮海內清英之士,謂之黨人。

中平二年七月,三輔螟蟲為害。

明帝永平十八年,牛疫死。是歲遣竇固等征西域,置都護、戊己校尉。固等適還而西域叛,殺都護陳睦、戊己校尉關寵。於是大怒,欲復發興討,會秋明帝崩,是思心不容也。

章帝建初四年冬,京都牛大疫。是時竇皇后以宋貴人子為太子,寵幸,令人求伺貴人過隙,以讒毀之。章帝不知竇太后不善,厥咎霿也。或曰,是年六月馬太后崩,土功非時興故也。


\end{pinyinscope}