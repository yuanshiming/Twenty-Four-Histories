\article{孝靈帝紀}

\begin{pinyinscope}
孝靈皇帝諱宏,肅宗玄孫也。曾祖河閒孝王開,祖淑,父萇。世封解瀆亭侯,帝襲侯爵。母董夫人。桓帝崩,無子,皇太后與父城門校尉竇武定策禁中,使守光祿大夫劉儵持節,將左右羽林至河閒奉迎。

建寧元年春正月壬午,城門校尉竇武為大將軍。己亥,帝到夏門亭,使竇武持節,以王青蓋車迎入殿中。庚子,即皇帝位,年十二。改元建寧。以前太尉陳蕃為太傅,與竇武及司徒胡廣參錄尚書事。

使護羌校尉段熲討先零羌。

二月辛酉,葬孝桓皇帝于宣陵,廟曰威宗。

庚午,謁高廟。辛未,謁世祖廟。大赦天下。賜民爵及帛各有差。

段熲大破先零羌於逢義山。

閏月甲午,追尊皇祖為孝元皇,夫人夏氏為孝元皇后,考為孝仁皇,夫人董氏為慎園貴人。

夏四月戊辰,太尉周景薨。司空宣酆免,長樂衛尉王暢為司空。

五月丁未朔,日有食之。詔公卿以下各上封事,及郡國守相舉有道之士各一人;又故刺史、二千石清高有遺惠,為眾所歸者,皆詣公車。

太中大夫劉矩為太尉。

六月,京師雨水。

秋七月,破羌將軍段熲復破先零羌於涇陽。

八月,司空王暢免,宗正劉寵為司空。

九月丁亥,中常侍曹節矯詔誅太傅陳蕃、大將軍竇武及尚書令尹勳、侍中劉瑜、屯騎校尉馮述,皆夷其族。皇太后遷于南宮。司徒胡廣為太傅,錄尚書事。司空劉寵為司徒,大鴻臚許栩為司空。

冬十月甲辰晦,日有食之。令天下繫囚罪未決入縑贖,各有差。

十一月,太尉劉矩免,太僕沛國聞人襲為太尉。

十二月,鮮卑及濊貊寇幽并二州。

二年春正月丁丑,大赦天下。

三月乙巳,尊慎園董貴人為孝仁皇后。

夏四月癸巳,大風,雨雹。詔公卿以下各上封事。

五月,太尉聞人襲罷,司空許栩免。六月,司徒劉寵為太尉,太常許訓為司徒,太僕長沙劉囂為司空。

秋七月,破羌將軍段熲大破先零羌於射虎塞外谷,東羌悉平。

九月,江夏蠻叛,州郡討平之。

丹陽山越賊圍太守陳夤,夤擊破之。

冬十月丁亥,中常侍侯覽諷有司奏前司空虞放、太僕杜密、長樂少府李膺、司隸校尉朱瑀、潁川太守巴肅、沛相荀翌、河內太守魏朗、山陽太守翟超皆為鉤黨,下獄,死者百餘人,妻子徙邊,諸附從者錮及五屬。制詔州郡大舉鉤黨,於是天下豪桀及儒學行義者,一切結為黨人。

庚子晦,日有食之。

十一月,太尉劉寵免,太僕郭禧為太尉。

鮮卑寇并州。

是歲,長樂太僕曹節為車騎將軍,百餘日罷。

三年春正月,河內人婦食夫,河南人夫食婦。

三月丙寅晦,日有食之。

夏四月,太尉郭禧罷,太中大夫聞人襲為太尉。秋七月,司空劉囂罷。八月,大鴻臚橋玄為司空。

九月,執金吾董寵下獄死。

冬,濟南賊起,攻東平陵。

鬱林烏滸民相率內屬。

四年春正月甲子,帝加元服,大赦天下。賜公卿以下各有差,唯黨人不赦。

二月癸卯,地震,海水溢,河水清。

三月辛酉朔,日有食之。

太尉聞人襲免,太僕李咸為太尉。

詔公卿至六百石各上封事。

大疫,使中謁者巡行致醫藥。

司徒許訓免,司空橋玄為司徒。夏四月,太常來豔為司空。

五月,河東地裂,雨雹,山水暴出。

秋七月,司空來豔免。

癸丑,立貴人宋氏為皇后。

司徒橋玄免。太常宗俱為司空,前司空許栩為司徒。

冬,鮮卑寇并州。

熹平元年春三月壬戌,太傅胡廣薨。

夏五月己巳,大赦天下,改元熹平。

長樂太僕侯覽有罪,自殺。

六月,京師雨水。

癸巳,皇太后竇氏崩。秋七月甲寅,葬桓思皇后。

宦官諷司隸校尉段熲捕繫太學諸生千餘人。冬十月,渤海王悝被誣謀反,丁亥,悝及妻子皆自殺。

十一月,會稽人許生自稱「越王」,寇郡縣,遣楊州刺史臧旻、丹陽太守陳夤討破之。

十二月,司徒許栩罷,大鴻臚袁隗為司徒。

鮮卑寇并州。

是歲,甘陵王恢薨。

二年春正月,大疫,使使者巡行致醫藥。

丁丑,司空宗俱薨。

二月壬午,大赦天下。

以光祿勳楊賜為司空。

三月,太尉李咸免。夏五月,以司隸校尉段熲為太尉。

沛相師遷坐誣罔國王,下獄死。

六月,北海地震。東萊,北海海水溢。

秋七月,司空楊賜免,太常潁川唐珍為司空。

冬十二月,日南徼外國重譯貢獻。

太尉段熲罷。

鮮卑寇幽并二州。

癸酉晦,日有食之。

三年春正月,夫餘國遣使貢獻。

二月己巳,大赦天下。

太常陳耽為太尉。

三月,中山王暢薨,無子,國除。

夏六月,封河閒王利子康為濟南王,奉孝仁皇祀。

秋,洛水溢。

冬十月癸丑,令天下繫囚罪未決,入縑贖。

十一月,楊州刺史臧旻率丹陽太守陳寅,大破許生於會稽,斬之。

任城王博薨。

十二月,鮮卑寇北地,北地太守夏育追擊破之。鮮卑又寇并州。

司空唐珍罷,永樂少府許訓為司空。

四年春三月,詔諸儒正五經文字,刻石立于太學門外。

封河閒王建孫佗為任城王。

夏四月,郡國七大水。

五月丁卯,大赦天下。

延陵園災,遣使者持節告祠延陵。

鮮卑寇幽州。

六月,弘農、三輔螟。

遣守宮令之鹽監,穿渠為民興利。

令郡國遇災者,減田租之半;其傷害十四以上,勿收責。

冬十月丁巳,令天下繫囚罪未決,入縑贖。

拜沖帝母虞美人為憲園貴人,質帝母陳夫人為渤海孝王妃。

改平準為中準,使宦者為令,列於內署。自是諸署悉以閹人為丞、令。

五年夏四月癸亥,大赦天下。

益州郡夷叛,太守李顒討平之。

復崇高山名為嵩高山。

大雩。使侍御史行詔獄亭部,理冤枉,原輕繫,休囚徒。

五月,太尉陳耽罷,司空許訓為太尉。

閏月,永昌太守曹鸞坐訟黨人,棄市。詔黨人門生故吏父兄子弟在位者,皆免官禁錮。

六月壬戌,太常南陽劉逸為司空。

秋七月,太尉許訓罷,光祿勳劉寬為太尉。

冬十月壬午,御殿後槐樹自拔倒豎。

司徒袁隗罷。十一月丙戌,光祿大夫楊賜為司徒。

十二月,甘陵王定薨。

試太學生年六十以上百餘人,除郎中、太子舍人至王家郎、郡國文學吏。

是歲,鮮卑寇幽州。沛國言黃龍見譙。

六年春正月辛丑,大赦天下。

二月,南宮平城門及武庫東垣屋自壞。

夏四月,大旱,七州蝗。

鮮卑寇三邊。

市賈民為宣陵孝子者數十人,皆除太子舍人。

秋七月,司空劉逸免,衛尉陳球為司空。

八月,遣破鮮卑中郎將田晏出雲中,使匈奴中郎將臧旻與南單于出鴈門,護烏桓校尉夏育出高柳,並伐鮮卑,晏等大敗。

冬十月癸丑朔,日有食之。

太尉劉寬免。

帝臨辟雍。

辛丑,京師地震。

辛亥,令天下繫囚罪未決,入縑贖。

十一月,司空陳球免。十二月甲寅,太常河南孟戫為太尉。庚辰,司徒楊賜免。太常陳耽為司空。

鮮卑寇遼西。

永安太僕王旻下獄死。

光和元年春正月,合浦、交阯烏滸蠻叛,招引九真、日南民攻沒郡縣。

太尉孟戫罷。

二月辛亥朔,日有食之。

癸丑,光祿勳陳國袁滂為司徒。

己未,地震。

始置鴻都門學生。

三月辛丑,大赦天下,改元光和。

太常常山張顥為太尉。

夏四月丙辰,地震。

侍中寺雌雞化為雄。

司空陳耽免,太常來豔為司空。

五月壬午,有白衣人入德陽殿門,亡去不獲。六月丁丑,有黑氣墯所御溫德殿庭中。秋七月壬子,青虹見御坐玉堂後殿庭中。八月,有星孛于天巿。

九月,太尉張顥罷,太常陳球為太尉。司空來豔薨。冬十月,屯騎校尉袁逢為司空。

皇后宋氏廢,后父執金吾酆下獄死。

丙子晦,日有食之。

十一月,太尉陳球免。十二月丁巳,光祿大夫橋玄為太尉。

是歲,鮮卑寇酒泉。京師馬生人。初開西邸賣官,自關內侯、虎賁、羽林,入錢各有差。私令左右賣公卿,公千萬,卿五百萬。

二年春,大疫,使常侍、中謁者巡行致醫藥。

三月,司徒袁滂免,大鴻臚劉郃為司徒。乙丑,太尉橋玄罷,太中大夫段熲為太尉。

京兆地震。

司空袁逢罷,太常張濟為司空。

夏四月甲戌朔,日有食之。

辛巳,中常侍王甫及太尉段熲並下獄死。

丁酉,大赦天下,諸黨人禁錮小功以下皆除之。

東平王端薨。

五月,衛尉劉寬為太尉。

秋七月,使匈奴中郎將張脩有罪,下獄死。

冬十月甲申,司徒劉郃、永樂少府陳球、衛尉陽球、步兵校尉劉納謀誅宦者,事泄,皆下獄死。

巴郡板楯蠻叛,遣御史中丞蕭瑗督益州刺史討之,不剋。

十二月,光祿勳楊賜為司徒。

鮮卑寇幽并二州。

是歲,河閒王利薨。洛陽女子生兒,兩頭四臂。

三年春正月癸酉,大赦天下。

二月,公府駐駕廡自壞。

三月,梁王元薨。

夏四月,江夏蠻叛。

六月,詔公卿舉能通尚書、毛詩、左氏、穀梁春秋各一人,悉除議郎。

秋,表是地震,涌水出。

八月,令繫囚罪未決,入縑贖,各有差。

冬閏月,有星孛于狼、弧。

鮮卑寇幽、并二州。

十二月己巳,立貴人何氏為皇后。

是歲,作罼圭、靈昆苑。

四年春正月,初置騄驥廄丞,領受郡國調馬。豪右辜搉,馬一匹至二百萬。

二月,郡國上芝英草。夏四月庚子,大赦天下。

交阯刺史朱雋討交阯、合浦烏滸蠻,破之。

六月庚辰,雨雹。秋七月,河南言鳳皇見新城,群鳥隨之;賜新城令及三老、力田帛,各有差。九月庚寅朔,日有食之。

太尉劉寬免,衛尉許戫為太尉。

閏月辛酉,北宮東掖庭永巷署災。

司徒楊賜罷。冬十月,太常陳耽為司徒。

鮮卑寇幽并二州。

是歲帝作列肆於後宮,使諸釆女販賣,更相盜竊爭鬥。帝著商估服,飲宴為樂。又於西園弄狗,著進賢冠,帶綬。又駕四驢,帝躬自操轡,驅馳周旋,京師轉相放效。

五年春正月辛未,大赦天下。

二月,大疫。

三月,司徒陳耽免。

夏四月,旱。

太常袁隗為司徒。

五月庚申,永樂宮置災。秋七月,有星孛于太微。

巴郡板楯蠻詣太守曹謙降。

癸酉,令繫囚罪未決,入縑贖。

八月,起四百尺觀於阿亭道。

冬十月,太尉許戫罷,太常楊賜為太尉。

校獵上林苑,歷函谷關,遂巡狩于廣成苑。十二月,還,幸太學。

六年春正月,日南徼外國重譯貢獻。

二月,復長陵縣,比豐、沛。三月辛未,大赦天下。

夏,大旱。

秋,金城河水溢。五原山岸崩。

始置圃囿署,以宦者為令。

冬,東海、東萊、琅邪井中冰厚尺餘。

大有年。

中平元年春二月,鉅鹿人張角自稱「黃天」,其部師有三十六萬,皆著黃巾,同日反叛。安平、甘陵人各執其王以應之。

三月戊申,以河南尹何進為大將軍,將兵屯都亭。置八關都尉官。壬子,大赦天下黨人,還諸徙者,唯張角不赦。詔公卿出馬、弩,舉列將子孫及吏民有明戰陣之略者,詣公車。遣北中郎將盧植討張角,左中郎將皇甫嵩、右中郎將朱雋討潁川黃巾。庚子,南陽黃巾張曼成攻殺郡守褚貢。

夏四月,太尉楊賜免,太僕弘農鄧盛為太尉。司空張濟罷,大司農張溫為司空。

朱雋為黃巾波才所敗。

侍中向栩、張鈞坐言宦者,下獄死。

汝南黃巾敗太守趙謙於邵陵。廣陽黃巾殺幽州刺史郭勳及太守劉衛。

五月,皇甫嵩、朱雋復與波才等戰於長社,大破之。

六月,南陽太守秦頡擊張曼成,斬之。

交阯屯兵執刺史及合浦太守來達,自稱「柱天將軍」,遣交阯刺史賈琮討平之。

皇甫嵩、朱雋大破汝南黃巾於西華。詔嵩討東郡,朱雋討南陽。盧植破黃巾,圍張角於廣宗。宦官誣奏植,抵罪。遣中郎將董卓攻張角,不剋。

洛陽女子生兒,兩頭共身。

秋七月,巴郡妖巫張脩反,寇郡縣。

河南尹徐灌下獄死。

八月,皇甫嵩與黃巾戰於倉亭,獲其帥。

乙巳,詔皇甫嵩北討張角。

九月,安平王續有罪誅,國除。

冬十月,皇甫嵩與黃巾賊戰於廣宗,獲張角弟梁。角先死,乃戮其屍。以皇甫嵩為左車騎將軍。十一月,皇甫嵩又破黃巾于下曲陽,斬張角弟寶。

湟中義從胡北宮伯玉與先零羌叛,以金城人邊章、韓遂為軍帥,攻殺護羌校尉伶徵、金城太守陳懿。

癸巳,朱雋拔宛城,斬黃巾別帥孫夏。

詔減太官珍羞,御食一肉;廄馬非郊祭之用,悉出給軍。

十二月己巳,大赦天下,改元中平。

是歲,下邳王意薨,無子,國除。郡國生異草,備龍蛇鳥獸之形。

二年春正月,大疫。

琅邪王據薨。

二月己酉,南宮大災,火半月乃滅。己亥,廣陽門外屋自壞。

稅天下田,畝十錢。

黑山賊張牛角等十餘輩並起,所在寇鈔。

司徒袁隗免。三月,廷尉崔烈為司徒。

北宮伯玉等寇三輔,遣左車騎將軍皇甫嵩討之,不剋。

夏四月庚戌,大風,雨雹。

五月,太尉鄧盛罷,太僕河南張延為太尉。

秋七月,三輔螟。

左車騎將軍皇甫嵩免。八月,以司空張溫為車騎將軍,討北宮伯玉。九月,特進楊賜為司空。冬十月庚寅,司空楊賜薨,光祿大夫許相為司空。

前司徒陳耽、諫議大夫劉陶坐直言,下獄死。

十一月,張溫破北宮伯玉於美陽,因遣盪寇將軍周慎追擊之,圍榆中;又遣中郎將董卓討先零羌。慎、卓並不克。

鮮卑寇幽、并二州。

是歲,造萬金堂於西園。洛陽民生兒,兩頭四臂。

三年春二月,江夏兵趙慈反,殺南陽太守秦頡。

庚戌,大赦天下。

太尉張延罷。車騎將軍張溫為太尉,中常侍趙忠為車騎將軍。

復修玉堂殿,鑄銅人四,黃鍾四,及天祿、蝦蟆,又鑄四出文錢。

五月壬辰晦,日有食之。

六月,荊州刺史王敏討趙慈,斬之。

車騎將軍趙忠罷。

秋八月,懷陵上有雀萬數,悲鳴,因鬥相殺。

冬十月,武陵蠻叛,寇郡界,郡兵討破之。

前太尉張延為宦人所譖,下獄死。

十二月,鮮卑寇幽并二州。

四年春正月己卯,大赦天下。

二月,滎陽賊殺中牟令。

己亥,南宮內殿罘罳自壞。

三月,河南尹何苗討滎陽賊,破之,拜苗為車騎將軍。

夏四月,涼州刺史耿鄙討金城賊韓遂,鄙兵大敗,遂寇漢陽,漢陽太守傅燮戰沒。扶風人馬騰、漢陽人王國並叛,寇三輔。

太尉張溫免,司徒崔烈為太尉。五月,司空許相為司徒,光祿勳沛國丁宮為司空。

六月,洛陽民生男,兩頭共身。

漁陽人張純與同郡張舉舉兵叛,攻殺右北平太守劉政、遼東太守楊終、護烏桓校尉公綦稠等。舉兵自稱天子,寇幽、冀二州。

秋九月丁酉,令天下繫囚罪未決,入縑贖。

冬十月,零陵人觀鵠自稱「平天將軍」,寇桂陽,長沙太守孫堅擊斬之。

十一月,太尉崔烈罷,大司農曹嵩為太尉。

十二月,休屠各胡叛。

是歲,賣關內侯,假金印紫綬,傳世,入錢五百萬。

五年春正月,休屠各胡寇西河,殺郡守邢紀。

丁酉,大赦天下。

二月,有星孛于紫宮。

黃巾餘賊郭太等起於西河白波谷,寇太原、河東。

三月,休屠各胡攻殺并州刺史張懿,遂與南匈奴左部胡合,殺其單于。

夏四月,汝南葛陂黃巾攻沒郡縣。

太尉曹嵩罷。五月,永樂少府樊陵為太尉。

六月丙寅,大風。

太尉樊陵罷。

益州黃巾馬相攻殺刺史郗儉,自稱天子,又寇巴郡,殺郡守趙部,益州從事賈龍擊相,斬之。

郡國七大水。

秋七月,射聲校尉馬日磾為太尉。

八月,初置西園八校尉。

司徒許相罷,司空丁宮為司徒。光祿勳南陽劉弘為司空。衛尉董重為票騎將軍。

九月,南單于叛,與白波賊寇河東。遣中郎將孟益率騎都尉公孫瓚討漁陽賊張純等。

冬十月,壬午御殿後槐樹自拔倒豎青、徐黃巾復起,寇郡縣。

甲子,帝自稱「無上將軍」,燿兵於平樂觀。

十一月,涼州賊王國圍陳倉,右將軍皇甫嵩救之。

遣下軍校尉鮑鴻討葛陂黃巾。

巴郡板楯蠻叛,遣上軍別部司馬趙瑾討平之。

公孫瓚與張純戰於石門,大破之。

是歲,改刺史,新置牧。

六年春二月,左將軍皇甫嵩大破王國於陳倉。

三月,幽州牧劉虞購斬漁陽賊張純。

下軍校尉鮑鴻下獄死。

夏四月丙午朔,日有食之。

太尉馬日磾免,幽州牧劉虞為太尉。

丙辰,帝崩于南宮嘉德殿,年三十四。戊午,皇子辯即皇帝位,年十七。尊皇后曰皇太后,太后臨朝。大赦天下,改元為光喜。封皇弟協為渤海王。後將軍袁隗為太傅,與大將軍何進參錄尚書事。上軍校尉蹇碩下獄死。五月辛巳,票騎將軍董重下獄死。六月辛亥,孝仁皇后董氏崩。

辛酉,葬孝靈皇帝于文陵。

雨水。

秋七月,甘陵王忠薨。

庚寅,孝仁皇后歸葬河閒慎陵。

徙渤海王協為陳留王。司徒丁宮罷。

八月戊辰,中常侍張讓、段珪等殺大將軍何進,於是虎賁中郎將袁術燒東西宮,攻諸宦者。庚午,張讓、段珪等劫少帝及陳留王幸北宮德陽殿。何進部曲將吳匡與車騎將軍何苗戰於朱雀闕下,苗敗斬之。辛未,司隸校尉袁紹勒兵收偽司隸校尉樊陵、河南尹許相及諸閹人,無少長皆斬之。讓、珪等復劫少帝、陳留王走小平津。尚書盧植追讓、珪等,斬數人,其餘投河而死。帝與陳留王協夜步逐熒光行數里,得民家露車,共乘之。

辛未,還宮。大赦天下,改光喜為昭寧。

并州牧董卓殺執金吾丁原。司空劉弘免,董卓自為司空。

九月甲戌,董卓廢帝為弘農王。

自六月雨,至于是月。

論曰:秦本紀說趙高譎二世,指鹿為馬,而趙忠、張讓亦紿靈帝不得登高臨觀,故知亡敝者同其致矣。然則靈帝之為靈也優哉!

贊曰:靈帝負乘,委體宦孽。徵亡備兆,小雅盡缺。麋鹿霜露,遂棲宮衛。


\end{pinyinscope}