\article{竇融列傳}

\begin{pinyinscope}
竇融字周公,扶風平陵人也。七世祖廣國,孝文皇后之弟,封章武侯。融高祖父,宣帝時以吏二千石自常山徙焉。融早孤。王莽居攝中,為強弩將軍司馬,東擊翟義,還攻槐里,以軍功封建武男。女弟為大司空王邑小妻。家長安中,出入貴戚,連結閭里豪傑,以任俠為名;然事母兄,養弱弟,內修行義。王莽末,青、徐賊起,太師王匡請融為助軍,與共東征。

及漢兵起,融復從王邑敗於昆陽下,歸長驅入關,王邑薦融,拜為波水將軍,賜黃金千斤,引兵至新豐。莽敗,融以軍降更始大司馬趙萌,萌以為校尉,甚重之,薦融為鉅鹿太守。

融見更始新立,東方尚擾,不欲出關,而高祖父嘗為張掖太守,從祖父為護羌校尉,從弟亦為武威太守,累世在河西,知其土俗,獨謂兄弟曰:「天下安危未可知,河西殷富,帶河為固,張掖屬國精兵萬騎,一旦緩急,杜絕河津,足以自守,此遺種處也。」兄弟皆然之。融於是日往守萌,辭讓鉅鹿,圖出河西。萌為言更始,乃得為張掖屬國都尉。融大喜,即將家屬而西。既到,撫結雄傑,懷輯羌虜,甚得其歡心,河西翕然歸之。

是時酒泉太守梁統、金城太守厙鈞、張掖

都尉史苞、酒泉都尉竺曾、敦煌都尉辛肜,並州郡英俊,融皆與為厚善。及更始敗,融與梁統等計議曰:「今天下擾亂,未知所歸。河西斗絕在羌胡中,不同心戮力則不能自守;權鈞力齊,復無以相率。當推一人為大將軍,共全五郡,觀時變動。」議既定,而各謙讓,咸以融世任河西為吏,人所敬向,乃推融行河西五郡大將軍事。是時武威太守馬期、張掖太守任仲並孤立無黨,乃共移書告示之,二人即解印綬去。於是以梁統為武威太守,史苞為張掖太守,竺曾為酒泉太守,辛肜為敦煌太守,厙鈞為金城太守。融居屬國,領都尉職如故,置從事監察五郡。河西民俗質樸,而融等政亦寬和,上下相親,晏然富殖。修兵馬,習戰射,明烽燧之警,羌胡犯塞,融輒自將與諸郡相救,皆如符要,每輒破之。其後匈奴懲乂,稀復侵寇,而保塞羌胡皆震服親附,安定、北地、上郡流人避凶飢者,歸之不絕。

融等遙聞光武即位,而心欲東向,以河西隔遠,未能自通。時隗囂先稱建武年號,融等從受正朔,囂皆假其將軍印綬。囂外順人望,內懷異心,使辯士張玄游說河西曰:「更始事業已成,尋復亡滅,此一姓不再興之效。今即有所主,便相係屬,一旦拘制,自令失柄,後有危殆,雖悔無及。今豪傑競逐,雌雄未決,當各據其土宇,與隴、蜀合從,高可為六國,下不失尉佗。」融等於是召豪傑及諸太守計議,其中智者皆曰:「漢承堯運,歷數延長。今皇帝姓號見於圖書,自前世博物道術之士谷子雲、夏賀良等,建明漢有再受命之符,言之久矣,故劉子駿改易名字,冀應其占。及莽末,道士西門君惠言劉秀當為天子,遂謀立子駿。事覺被殺,出謂百姓觀者曰:『劉秀真汝主也。』皆近事暴著,智者所共見也。除言天命,且以人事論之:今稱帝者數人,而洛陽土地最廣,甲兵最彊,號令最明。觀符命而察人事,它姓殆未能當也。」諸郡太守各有賓客,或同或異。融小心精詳,遂決策東向。五年夏,遣長史劉鈞奉書獻馬。

先是,帝聞河西完富,地接隴、蜀,常欲招之以逼囂、述,亦發使遺融書,遇鈞於道,即與俱還。帝見鈞歡甚,禮饗畢,乃遣令還,賜融璽書曰:「制詔行河西五郡大將軍事、屬國都尉:勞鎮守邊五郡,兵馬精彊,倉庫有蓄,民庶殷富,外則折挫羌胡,內則百姓蒙福。威德流聞,虛心相望,道路隔塞,邑邑何已!長史所奉書獻馬悉至,深知厚意。今益州有公孫子陽,天水有隗將軍,方蜀漢相攻,權在將軍,舉足左右,便有輕重。以此言之,欲相厚豈有量哉!諸事具長史所見,將軍所知。王者迭興,千載一會。欲遂立桓、文,輔微國,當勉卒功業;欲三分鼎足,連衡合從,亦宜以時定。天下未并,吾與爾絕域,非相吞之國。今之議者,必有任囂效尉佗制七郡之計。王者有分土,無分民,自適己事而已。今以黃金二百斤賜將軍,便宜輒言。」因授融為涼州牧。

璽書既至,河西咸驚,以為天子明見萬里之外,網羅張立之情。融即復遣鈞上書曰:「臣融竊伏自惟,幸得託先后末屬,蒙恩為外戚,累世二千石。至臣之身,復備列位,假歷將帥,守持一隅。以委質則易為辭,以納忠則易為力。書不足以深達至誠,故遣劉鈞口陳肝膽。自以底裏上露,長無纖介。而璽書盛稱蜀、漢二主,三分鼎足之權,任囂、尉佗之謀,竊自痛傷。臣融雖無識,猶知利害之際,順逆之分。豈可背真舊之主,事姦偽之人;廢忠貞之節,為傾覆之事;棄已成之基,求無冀之利。此三者雖問狂夫,猶知去就,而臣獨何以用心!謹遣同產弟友詣闕,口陳區區。」友至高平,會囂反叛,道絕,馳還,遣司馬席封閒行通書。帝復遣席封賜融、友書,所以尉藉之甚備。

融既深知帝意,乃與隗囂書責讓之曰:「伏惟將軍國富政修,士兵懷附。親遇厄會之際,國家不利之時,守節不回,承事本朝,後遣伯春委身於國,無疑之誠,於斯有效。融等所以欣服高義,願從役於將軍者,良為此也。而忿悁之閒,改節易圖,君臣分爭,上下接兵。委成功,造難就,去從義,為橫謀,百年累之,一朝毀之,豈不惜乎!殆執事者貪功建謀,以至於此,融竊痛之!當今西州地埶局迫,人兵離散,易以輔人,難以自建。計若失路不反,聞道猶迷,不南合子陽,則北入文伯耳。夫負虛交而易強禦,恃遠救而輕近敵,未見其利也。融聞智者不危眾以舉事,仁者不違義以要功。今以小敵大,於眾何如?棄子徼功,於義何如?且初事本朝,稽首北面,忠臣節也。及遣伯春,垂涕相送,慈父恩也。俄而背之,謂吏士何?忍而棄之,謂留子何?自兵起以來,轉相攻擊,城郭皆為丘墟,生人轉於溝壑。今其存者,非鋒刃之餘,則流亡之孤。迄今傷痍之體未愈,哭泣之聲尚聞。幸賴天運少還,而大將軍復重於難,是使積痾不得遂瘳,幼孤將復流離,其為悲痛,尤足愍傷,言之可為酸鼻!庸人且猶不忍,況仁者乎?融聞為忠甚易,得宜實難。憂人大過,以德取怨,知且以言獲罪也。區區所獻,唯將軍省焉。」囂不納。融乃與五郡太守共砥厲兵馬,上疏請師期。

帝深嘉美之,乃賜融以外屬圖及太史公五宗、外戚世家、魏其侯列傳。詔報曰:「每追念外屬,孝景皇帝出自竇氏,定王,景帝之子,朕之所祖。昔魏其一言,繼統以正,長君、少君尊奉師傅,修成淑德,施及子孫,此皇太后神靈,上天祐漢也。從天水來者寫將軍所讓隗囂書,痛入骨髓。畔臣見之,當股慄慚愧,忠臣則酸鼻流涕,義士則曠若發矇,非忠孝愨誠,孰能如此?豈其德薄者所能剋堪!囂自知失河西之助,族禍將及,欲設閒離之說,亂惑真心,轉相解搆,以成其姦。又京師百僚,不曉國家及將軍本意,多能採取虛偽,誇誕妄談,令忠孝失望,傳言乖實。毀譽之來,皆不徒然,不可不思。今關東盜賊已定,大兵今當悉西,將軍其抗厲威武,以應期會。」融被詔,即與諸郡守將兵入金城。

初,更始時,先零羌封何諸種殺金城太守,居其郡,隗囂使使賂遺封何,與共結盟,欲發其眾。融等因軍出,進擊封何,大破之,斬首千餘級,得牛馬羊萬頭,穀數萬斛,因並河揚威武,伺候車駕。時大兵未進,融乃引還。

帝以融信效著明,益嘉之。詔右扶風修理融父墳塋,祠以太牢。數馳輕使,致遺四方珍羞。梁統乃使人刺殺張玄,遂與囂絕,皆解所假將軍印綬。七年夏,酒泉太守竺曾以弟報怨殺人而去郡,融承制拜曾為武鋒將軍,更以辛肜代之。

秋,隗囂發兵寇安定,帝將自西征之,先戒融期。會遇雨,道斷,且囂兵已退,乃止。融至姑臧,被詔罷歸。融恐大兵遂久不出,乃上書曰:「隗囂聞車駕當西,臣融東下,士眾騷動,計且不戰。囂將高峻之屬皆欲逢迎大軍,後聞兵罷,峻等復疑。囂揚言東方有變,西州豪桀遂復附從。囂又引公孫述將,令守突門。臣融孤弱,介在其閒,雖承威靈,宜速救助。國家當其前,臣融促其後,緩急迭用,首尾相資,囂埶排迮,不得進退,此必破也。若兵不早進,久生持疑,則外長寇讎,內示困弱,復令讒邪得有因緣,臣竊憂之。惟陛下哀憐!」帝深美之。

八年夏,車駕西征隗囂,融率五郡太守及羌虜小月氏等步騎數萬,輜重五千餘兩,與大軍會高平第一。融先遣從事問會見儀適,是時軍旅代興,諸將與三公交錯道中,或背使者交私語。帝聞融先問禮儀,甚善之,以宣告百僚。乃置酒高會,引見融等,待以殊禮。拜弟友為奉車都尉,從弟士太中大夫。遂共進軍,囂眾大潰,城邑皆降。帝高融功,下詔以安豐、陽泉、蓼、安安風四縣封融為安豐侯,弟友為顯親侯。遂以次封諸將帥:武鋒將軍竺曾為助義侯,武威太守梁統為成義侯,張掖太守史苞為褒義侯,金城太守厙鈞為輔義侯,酒泉太守辛肜為扶義侯。封爵既畢,乘輿東歸,悉遣融等西還所鎮。

融以兄弟並受爵位,久專方面,懼不自安,數上書求代。詔報曰:「吾與將軍如左右手耳,數執謙退,何不曉人意?勉循士民,無擅離部曲。」

及隴、蜀平,詔融與五郡太守奏事京師,官屬賓客相隨,駕乘千餘兩,馬牛羊被野。融到,詣洛陽城門,上涼州牧、張掖屬國都尉、安豐侯印綬,詔遣使者還侯印綬。引見,就諸侯位,賞賜恩寵,傾動京師。數月,拜為冀州牧,十餘日,又遷大司空。融自以非舊臣,一旦入朝,在功臣之右,每召會進見,容貌辭氣卑恭已甚,帝以此愈親厚之。融小心,久不自安,數辭讓爵位,因侍中金遷口達至誠。又上疏曰:「臣融年五十三。有子年十五,質性頑鈍。臣融朝夕教導以經蓺,不得令觀天文,見讖記。誠欲令恭肅畏事,恂恂循道,不願其有才能,何況乃當傳以連城廣土,享故諸侯王國哉?」因復請閒求見,帝不許。後朝罷,逡巡席後,帝知欲有讓,遂使左右傳出。它日會見,迎詔融曰:「日者知公欲讓職還土,故命公暑熱且自便。今相見,宜論它事,勿得復言。」融不敢重陳請。

二十年,大司徒戴涉坐所舉人盜金下獄,帝以三公參職,不得已乃策免融。明年,加位特進。二十三年,代陰興行衛尉事,特進如故,又兼領將作大匠。弟友為城門校尉,兄弟並典禁兵。融復乞骸骨,輒賜錢帛,太官致珍奇。及友卒,帝愍融年衰,遣中常侍、中謁者即其臥內強進酒食。

融長子穆,尚內黃公主,代友為城門校尉。穆子勳,尚東海恭王彊女沘陽公主,友子固,亦尚光武女涅陽公主。顯宗即位,以融從兄子林為護羌校尉。竇氏一公,兩侯,三公主,四二千石,相與並時。自祖及孫,官府邸第相望京邑,奴婢以千數,於親戚、功臣中莫與為比。

永平二年,林以罪誅,事在西羌傳。帝由是數下詔切責融,戒以竇嬰、田蚡禍敗之事。融惶恐乞骸骨,詔令歸第養病。歲餘,聽上衛尉印綬,賜養牛,上樽酒。融在宿衛十餘年,年老,子孫縱誕,多不法。穆等遂交通輕薄,屬託郡縣,干亂政事。以封在安豐,欲令姻戚悉據故六安國,遂矯稱陰太后詔,令六安侯劉盱去婦,因以女妻之。五年,盱婦家上書言狀,帝大怒,乃盡免穆等官,諸竇為郎吏者皆將家屬歸故郡,獨留融京師。穆等西至函谷關,有詔悉復追還。會融卒,時年七十八,謚曰戴侯,賻送甚厚。

帝以穆不能修尚,而擁富貲,居大第,常令謁者一人監護其家。居數年,謁者奏穆父子自失埶,數出怨望語,帝令將家屬歸本郡,唯勳以沘陽主婿留京師。穆坐賂遺小吏,郡捕繫,與子宣俱死平陵獄,勳亦死洛陽獄。久之,詔還融夫人與小孫一人居洛陽家舍。

十四年,封勳弟嘉為安豐侯,食邑二千戶,奉融後。和帝初,為少府。及勳子大將軍憲被誅,免就國。嘉卒,子萬全嗣。萬全卒,子會宗嗣。萬全弟子武,別有傳。

論曰:竇融始以豪俠為名,拔起風塵之中,以投天隙。遂蟬蛻王侯之尊,終膺卿相之位,此則徼功趣埶之士也。及其爵位崇滿,至乃放遠權寵,恂恂似若不能已者,又何智也!嘗獨詳味此子之風度,雖經國之術無足多談,而進退之禮良可言矣。

固字孟孫,少以尚公主為黃門侍郎。好覽書傳,喜兵法,貴顯用事。中元元年,襲父友封顯親侯。顯宗即位,遷中郎將,監羽林士。後坐從兄穆有罪,廢于家十餘年。時天下乂安,帝欲遵武帝故事,擊匈奴,通西域,以固明習邊事,十五年冬,拜為奉車都尉,以騎都尉耿忠為副,謁者僕射耿秉為駙馬都尉,秦彭為副,皆置從事、司馬,並出屯涼州。明年,固與忠率酒泉、敦煌、張掖甲卒及盧水羌胡萬二千騎出酒泉塞,耿秉、秦彭率武威、隴西、天水募士及羌胡萬騎出居延塞,又太僕祭肜、度遼將軍吳棠將河東北地、西河羌胡及南單于兵萬一千騎出高闕塞,騎都尉來苗、護烏桓校尉文穆將太原、鴈門、代郡、上谷、漁陽、右北平、定襄郡兵及烏桓、鮮卑萬一千騎出平城塞。固、忠至天山,擊呼衍王,斬首千餘級。呼衍王走,追至蒲類海。留吏士屯伊吾盧城。耿秉、秦彭絕漠六百餘里,至三木樓山,來苗、文穆至匈奴河水上,虜皆奔走,無所獲。祭肜、吳棠坐不至涿邪山,免為庶人。時諸將唯固有功,加位特進。明年,復出玉門擊西域,詔耿秉及騎都尉劉張皆去符傳以屬固。固遂破白山,降車師,事已具耿秉傳。固在邊數年,羌胡服其恩信。

肅宗即位,以公主修敕慈愛,累世崇重,加號長公主,增邑三千戶;徵固代魏應為大鴻臚。帝以其曉習邊事,每被訪及。建初三年,追錄前功,增邑一千三百戶。七年,代馬防為光祿勳。明年,復代馬防為衛尉。

固久歷大位,甚見尊貴,賞賜租祿,貲累巨億,而性謙儉,愛人好施,士以此稱之。章和二年卒,謚曰文侯。子彪,至射聲校尉,先固卒,無子,國除。

憲字伯度。父勳被誅,憲少孤。建初二年,女弟立為皇后,拜憲為郎,稍遷侍中、虎賁中郎將;弟篤,為黃門侍郎。兄弟親幸,並侍宮省,賞賜累積,寵貴日盛,自王、主及陰、馬諸家,莫不畏憚。憲恃宮掖聲埶,遂以賤直請奪沁水公主園田,主逼畏,不敢計。後肅宗駕出過園,指以問憲,憲陰喝不得對。後發覺,帝大怒,召憲切責曰:「深思前過,奪主田園時,何用愈趙高指鹿為馬?久念使人驚怖。昔永平中,常令陰黨、陰博,鄧疊三人更相糾察,故諸豪戚莫敢犯法者,而詔書切切,猶以舅氏田宅為言。今貴主尚見枉奪,何況小人哉!國家棄憲如孤雛腐鼠耳。」憲大震懼,皇后為毀服深謝,良久乃得解,使以田還主。雖不繩其罪,然亦不授以重任。

和帝即位,太后臨朝,憲以侍中,內幹機密,出宣誥命。肅宗遺詔以篤為虎賁中郎將,篤弟景、瑰並中常侍,於是兄弟皆在親要之地。憲以前太尉鄧彪有義讓,先帝所敬,而仁厚委隨,故尊崇之,以為太傅,令百官總己以聽。其所施為,輒外令彪奏,內白太后,事無不從,又屯騎校尉桓郁,累世帝師,而性和退自守,故上書薦之,令授經禁中。所以內外協附,莫生疑異。

憲性果急,睚眥之怨莫不報復。初,永平時,謁者韓紆嘗考劾父勳獄,憲遂令客斬紆子,以首祭勳冢。齊殤王子都鄉侯暢來弔國憂,暢素行邪僻,與步兵校尉鄧疊親屬數往來京師,因疊母元自通長樂宮,得幸太后,被詔召詣上東門。憲懼見幸,分宮省之權,遣客刺殺暢於屯衛之中,而歸罪於暢弟利侯剛,乃使侍御史與青州刺史雜考剛等。後事發覺,太后怒,閉憲於內宮。

憲懼誅,自求擊匈奴以贖死。會南單于請兵北伐,乃拜憲車騎將軍,金印紫綬,官屬依司空,以執金吾耿秉為副,發北軍五校、黎陽、雍營、緣邊十二郡騎士,及羌胡兵出塞。明年,憲與秉各將四千騎及南匈奴左谷蠡王師子萬騎出朔方雞鹿塞,南單于屯屠河,將萬餘騎出滿夷谷,度遼將軍鄧鴻及緣邊義從羌胡八千騎,與左賢王安國萬騎出桥稒陽塞,皆會涿邪山。憲分遣副校尉閻盤、司馬耿夔、耿譚將左谷蠡王師子、右呼衍王須訾等,精騎萬餘,與北單于戰於稽落山,大破之,虜眾崩潰,單于遁走,追擊諸部,遂臨私渠比鞮海。斬名王已下萬三千級,獲生口馬牛羊橐駝百餘萬頭。於是溫犢須、日逐、溫吾、夫渠王柳鞮等八十一部率眾降者,前後二十餘萬人。憲、秉遂登燕然山,去塞三千餘里,刻石勒功,紀漢威德,令班固作銘曰:

惟永元元年秋七月,有漢元舅曰車騎將軍竇憲,寅亮聖明,登翼王室,納于大麓,惟清緝熙。乃與執金吾耿秉,述職巡御,理兵於朔方。鷹揚之校,螭虎之士,爰該六師,暨南單于、東烏桓、西戎氐羌侯王君長之群,驍騎三萬。元戎輕武,長轂四分,雲輜蔽路,萬有三千餘乘。勒以八陣,蒞以威神,玄甲耀日,朱旗絳天。遂陵高闕,下雞鹿,經磧鹵,絕大漠,斬溫禺以釁鼓,血尸逐以染鍔。然後四校橫徂,星流彗埽,蕭條萬里,野無遺寇。於是域滅區單,反旆而旋,考傳驗圖,窮覽其山川。遂踰涿邪,跨安侯,乘燕然,躡冒頓之區落,焚老上之龍庭。上以攄高、文之宿憤,光祖宗之玄靈;下以安固後嗣,恢拓境宇,振大漢之天聲。茲所謂一勞而久逸,暫費而永寧者也。乃遂封山刊石,昭銘上德。其辭曰:

鑠王師兮征荒裔,勦凶虐兮涞海外,夐其邈兮亙地界,封神丘兮建隆嵑,熙帝載兮振萬世。

憲乃班師而還。遣軍司馬吳汜、梁諷,奉金帛遺北單于,宣明國威,而兵隨其後。時虜中乖亂,汜、諷所到,輒招降之,前後萬餘人。遂及單于於西海上,宣國威信,致以詔賜,單于稽首拜受。諷因說宜修呼韓邪故事,保國安人之福。單于喜悅,即將其眾與諷俱還,到私渠海,聞漢軍已入塞,乃遣弟右溫禺鞮王奉貢入侍,隨諷詣闕。憲以單于不自身到,奏還其侍弟。南單于於漠北遺憲古鼎,容五斗,其傍銘曰「仲山甫鼎,其萬年子子孫孫永保用」,憲乃上之。詔使中郎將持節即五原拜憲大將軍,封武陽侯,食邑二萬戶。憲固辭封,賜策許焉。

舊大將軍位在三公下,置官屬依太尉。憲威權震朝庭,公卿希旨,奏憲位次太傅下,三公上;長史、司馬秩中二千石,從事中郎二人六百石,自下各有增。振旅還京師。於是大開倉府,勞賜士吏,其所將諸郡二千石子弟從征者,悉除太子舍人。

是時篤為衛尉,景、瑰皆侍中、奉車、駙馬都尉,四家競修第宅,窮極工匠。明年,詔曰:「大將軍憲,前歲出征,克滅北狄,朝加封賞,固讓不受。舅氏舊典,並蒙爵土。其封憲冠軍侯,邑二萬戶;篤郾侯,景汝陽侯,瑰夏陽侯,各六千戶。」憲獨不受封,遂將兵出鎮涼州,以侍中鄧疊行征西將軍事為副。

北單于以漢還侍弟,復遣車諧儲王等款居延塞,欲入朝見,願請大使。憲上遣大將軍中護軍班固行中郎將,與司馬梁諷迎之。會北單于為南匈奴所破,被創遁走,固至私渠海而還。憲以北虜微弱,遂欲滅之。明年,復遣右校尉耿夔、司馬任尚、趙博等將兵擊北虜於金微山,大破之,克獲甚眾。北單于逃走,不知所在。

憲既平匈奴,威名大盛,以耿夔、任尚等為爪牙,鄧疊、郭璜為心腹。班固、傅毅之徒,皆置幕府,以典文章。刺史、守令多出其門。尚書僕射郅壽、樂恢並以忤意,相繼自殺。由是朝臣震懾,望風承旨。而篤進位特進,得舉吏,見禮依三公。景為執金吾,瑰光祿勳,權貴顯赫,傾動京都。雖俱驕縱,而景為尤甚,奴客緹騎依倚形埶,侵陵小人,強奪財貨,篡取罪人,妻略婦女。商賈閉塞,如避寇讎。有司畏懦,莫敢舉奏。太后聞之,使謁者策免景官,以特進就朝位。瑰少好經書,節約自修,出為魏郡,遷潁川太守。竇氏父子兄弟並居列位,充滿朝廷。叔父霸為城門校尉,霸弟褒將作大匠,褒弟嘉少府,其為侍中、將、大夫、郎吏十餘人。

憲既負重勞,陵肆滋甚。四年,封鄧疊為穰侯。疊與其弟步兵校尉磊及母元,又憲女婿射聲校尉郭舉,舉父長樂少府璜,皆相交結。元、舉並出入禁中,舉得幸太后,遂共圖為殺害。帝陰知其謀,乃與近幸中常侍鄭眾定議誅之,以憲在外,慮其懼禍為亂,忍而未發。會憲及鄧疊班師還京師,詔使大鴻臚持節郊迎,賜軍吏各有差。憲等既至,帝乃幸北宮,詔執金吾、五校尉勒兵屯衛南、北宮,閉城門,收捕疊、磊、璜、舉,皆下獄誅,家屬徙合浦。遣謁者僕射收憲大將軍印綬,更封為冠軍侯。憲及篤、景、瑰皆遣就國。帝以太后故,不欲名誅憲,為選嚴能相督察之。憲、篤、景到國,皆迫令自殺,宗族、賓客以憲為官者皆免歸本郡。瑰以素自修,不被逼迫,明年坐稟假貧人,徙封羅侯,不得臣吏人。初,竇后之譖梁氏,憲等豫有謀焉,永元十年,梁棠兄弟徙九真還,路由長沙,逼瑰令自殺。後和熹鄧后臨朝,永初三年,詔諸竇前歸本郡者與安豐侯萬全俱還京師。萬全少子章。

論曰:衛青、霍去病資強漢之眾,連年以事匈奴,國秏太半矣,而猾虜未之勝,後世猶傳其良將,豈非以身名自終邪!竇憲率羌胡邊雜之師,一舉而空朔庭,至乃追奔稽落之表,飲馬比鞮之曲,銘石負鼎,薦告清廟。列其功庸,兼茂於前多矣,而後世莫稱者,章末釁以降其實也。是以下流,君子所甚惡焉。夫二三子得之不過房幄之閒,非復搜揚仄陋,選舉而登也。當青病奴僕之時,竇將軍念咎之日,乃庸力之不暇,思鳴之無晨,何意裂膏腴,享崇號乎?東方朔稱「用之則為虎,不用則為鼠」,信矣。以此言之,士有懷琬琰以就煨塵者,亦何可支哉!

章字伯向。少好學,有文章,與馬融、崔瑗同好,更相推薦。

永初中,三輔遭羌寇,章避難東國,家於外黃。居貧,蓬戶蔬食,躬勤孝養,然講讀不輟,太僕鄧康聞其名,請欲與交,章不肯往,康以此益重焉。是時學者稱東觀為老氏臧室,道家蓬萊山,康遂薦章入東觀為校書郎。

順帝初,章女年十二,能屬文,以才貌選入掖庭,有寵,與梁皇后並為貴人。擢章為羽林郎將,遷屯騎校尉。章謙虛下士,收進時輩,甚得名譽。是時梁、竇並貴,各有賓客,多交搆其閒,章推心待之,故得免於患。

貴人早卒,帝追思之無已,詔史官樹碑頌德,章自為之辭。貴人歿後,帝禮待之無衰。永和五年,遷少府。漢安二年,轉大鴻臚。建康元年,梁后稱制,章自免,卒于家。中子唐,有俊才,官至虎賁中郎將。

贊曰:悃悃安豐,亦稱才雄。提鲭河右,奉圖歸忠。孟孫明邊,伐北開西。憲實空漠,遠兵金山。聽笳龍庭,鏤石燕然。雖則折鼎,王靈以宣。


\end{pinyinscope}