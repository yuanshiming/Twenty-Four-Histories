\article{張法滕馮度楊列傳}

\begin{pinyinscope}
張宗字諸君,南陽魯陽人也。王莽時,為縣陽泉鄉佐。會莽敗,義兵起,宗乃率陽泉民三四百人起兵略地,西至長安,更始以宗為偏將軍。宗見更始政亂,因將家屬客安邑。

及大司徒鄧禹西征,定河東,宗詣禹自歸。禹聞宗素多權謀,乃表為偏將軍。禹軍到栒邑,赤眉大眾且至,禹以栒邑不足守,欲引師進就堅城,而眾人多畏賊追,憚為後拒。禹乃書諸將名於竹簡,署其前後,亂著笥中,令各探之。宗獨不肯探,曰:「死生有命,張宗豈辭難就逸乎!」禹歎息謂曰:「將軍有親弱在營,柰何不顧?」宗曰:「愚聞一卒畢力,百人不當;萬夫致死,可以橫行。宗今擁兵數千,以承大威,何遽其必敗乎!」遂留為後拒。諸營既引兵,宗方勒厲軍士,堅壘壁,以死當之。禹到前縣,議曰:「以張將軍之眾,當百萬之師,猶以小雪投沸湯,雖欲戮力,其埶不全也。」乃遣步騎二千人反還迎宗。宗引兵始發,而赤眉卒至,宗與戰,卻之,乃得歸營,於是諸將服其勇。及還到長安,宗夜將銳士入城襲赤眉,中矛貫胛,又轉攻諸營保,為流矢所激,皆幾至於死。

及鄧禹徵還,光武以宗為京輔都尉,將突騎與征西大將軍馮異共擊關中諸營保,破之,遷河南都尉。建武六年,都尉官省,拜太中大夫。八年,潁川桑中盜賊群起,宗將兵擊定之。後青、冀盜賊屯聚山澤,宗以謁者督諸郡兵討平之。十六年,琅邪、北海盜賊復起,宗督二郡兵討之,乃設方略,明購賞,皆悉破散,於是沛、楚、東海、臨淮群賊懼其威武,相捕斬者數千人,青、徐震慄。後遷琅邪相,其政好嚴猛,敢殺伐。永平二年,卒於官。

法雄字文彊,扶風郿人也,齊襄王法章之後。秦滅齊,子孫不敢稱田姓,故以法為氏。宣帝時,徙三輔,世為二千石。雄初仕郡功曹,辟太傅張禹府,舉雄高第,除平氏長。善政事,好發擿姦伏,盜賊稀發,吏人畏愛之。南陽太守鮑得上其理狀,遷宛陵令。

永初三年,海賊張伯路等三千餘人,冠赤幘,服絳衣,自稱「將軍」,寇濱海九郡,殺二千石令長。初,遣侍御史龐雄督州郡兵擊之,伯路等乞降,尋復屯聚。明年,伯路復與平原劉文河等三百餘人稱「使者」。攻厭次城,殺長吏,轉入高唐,燒官寺,出繫囚,渠帥皆稱「將軍」,共朝謁伯路。伯路冠五梁冠,佩印綬,黨眾浸盛。乃遣御史中丞王宗持節發幽、冀諸郡兵,合數萬人,乃徵雄為青州刺史,與王宗并力討之。連戰破賊,斬首溺死者數百人,餘皆奔走,收器械財物甚眾。會赦詔到,賊猶以軍甲未解,不敢歸降。於是王宗召刺史太守共議,皆以為當遂擊之。雄曰:「不然。兵,凶器;戰,危事。勇不可恃,勝不可必。賊若乘船浮海,深入遠島,攻之未易也。及有赦令,可且罷兵,以慰誘其心,埶必解散,然後圖之,可不戰而定也。」宗善其言,即罷兵。賊聞大喜,乃還所略人。而東萊郡兵獨未解甲,賊復驚恐,遁走遼東,止海島上。五年春,乏食,復抄東萊閒,雄率郡兵擊破之,賊逃還遼東,遼東人李久等共斬平之,於是州界清靜。

雄每行部,錄囚徒,察顏色,多得情偽,長吏不奉法者皆解印綬去。

在州四年,遷南郡太守,斷獄省少,戶口益增。郡濱帶江沔,又有雲夢藪澤,永初中,多虎狼之暴,前太守賞募張捕,反為所害者甚眾。雄乃移書屬縣曰:「凡虎狼之在山林,猶人之居城市。古者至化之世,猛獸不擾,皆由恩信寬澤,仁及飛走。太守雖不德,敢忘斯義。記到,其毀壞檻阱,不得妄捕山林。」是後虎害稍息,人以獲安。在郡數歲,歲常豐稔。元初中卒官。

子真,在逸人傳。

滕撫字叔輔,北海劇人也。初仕州郡,稍遷為涿令,有文武才用。太守以其能,委任郡職,兼領六縣。風政修明,流愛于人,在事七年,道不拾遺。

順帝末,揚、徐盜賊群起,磐牙連歲。建康元年,九江范容、周生等相聚反亂,屯據歷陽,為江淮巨患,遣御史中丞馮緄將兵督揚州刺史尹燿、九江太守鄧顯討之。燿、顯軍敗,為賊所殺。又陰陵人徐鳳、馬勉等復寇郡縣,殺略吏人。鳳衣絳衣,帶黑綬,稱「無上將軍」,勉皮冠黃衣,帶玉印,稱「黃帝」,築營於當塗山中。乃建年號,置百官,遣別帥黃虎攻沒合肥。明年,廣陵賊張嬰等復聚眾數千人反,據廣陵。朝廷博求將帥,三公舉撫有文武才,拜為九江都尉,與中郎將趙序助馮緄合州郡兵數萬人共討之。又廣開賞募,錢、邑各有差。梁太后慮群賊屯結,諸將不能制,又議遣太尉李固。未及行,會撫等進擊,大破之,斬馬勉、范容、周生等千五百級,徐鳳遂將餘眾攻燒東城縣。下邳人謝安應募,率其宗親設伏擊鳳,斬之,封安為平鄉侯,邑三千戶。拜撫中郎將,督揚徐二州事。撫復進擊張嬰,斬獲千餘人。趙序坐畏懦不進,詐增首級,徵還棄市。又歷陽賊華孟自稱「黑帝」,攻九江,殺郡守。撫乘勝進擊,破之,斬孟等三千八百級,虜獲七百餘人,牛馬財物不可勝筭。於是東南悉平,振旅而還。以撫為左馮翊,除一子為郎。撫所得賞賜,盡分於麾下。

性方直,不交權埶,宦官懷忿。及論功當封,太尉胡廣時錄尚書事,承旨奏黜撫,天下怨之。卒於家。

馮緄字鴻卿,巴郡宕渠人也,少學春秋、司馬兵法。父煥,安帝時為幽州刺史,疾忌姦惡,數致其罪。時玄菟太守姚光亦失人和。建光元年,怨者乃詐作璽書譴責煥、光,賜以歐刀。又下遼東都尉龐奮使速行刑,奮即斬光收煥。煥欲自殺,緄疑詔文有異,止煥曰:「大人在州,志欲去惡,實無它故,必是凶人妄詐,規肆姦毒。願以事自上,甘罪無晚。」煥從其言,上書自訟,果詐者所為,徵奮抵罪。會煥病死獄中,帝愍之,賜煥、光錢各十萬,以子為郎中。緄由是知名。

家富好施,賑赴窮急,為州里所歸愛。初舉孝廉,七遷為廣漢屬國都尉,徵拜御史中丞。順帝末,以緄持節督揚州諸郡軍事,與中郎將滕撫擊破群賊,遷隴西太守。後鮮卑寇邊,以緄為遼東太守,曉喻降集,虜皆弭散。徵拜京兆尹,轉司隸校尉,所在立威刑。遷廷尉、太常。

時長沙蠻寇益陽,屯聚積久,至延熹五年,眾轉盛,而零陵蠻賊復反應之,合二萬餘人,攻燒城郭,殺傷長吏。又武陵蠻夷悉反,寇掠江陵閒,荊州刺史劉度、南郡太守李肅並奔走荊南,皆沒。於是拜緄為車騎將軍,將兵十餘萬討之,詔策緄曰:「蠻夷猾夏,久不討攝,各焚都城,蹈籍官人。州郡將吏,死職之臣,相逐奔竄,曾不反顧,可愧言也。將軍素有威猛,是以擢授六師。前代陳湯、馮、傅之徒,以寡擊眾,郅支、夜郎、樓蘭之戎,頭懸都街,衛、霍北征,功列金石,是皆將軍所究覽也。今非將軍,誰與修復前跡?進赴之宜,權時之策,將軍一之,出郊之事,不復內御。已命有司祖于國門。詩不云乎:『進厥虎臣,闞如虓虎,敷敦淮濆,仍執醜虜。』將軍其勉之!」

時天下飢饉,帑藏虛盡,每出征伐,常減公卿奉祿,假王侯租賦,前後所遣將帥,宦官輒陷以折耗軍資,往往抵罪。緄性烈直,不行賄賂,懼為所中,乃上疏曰:「埶得容姦,伯夷可疑;苟曰無猜,盜跖可信。故樂羊陳功,文侯示以謗書。願請中常侍一人監軍財費。」尚書朱穆奏緄以財自嫌,失大臣之節。有詔勿劾。

緄軍至長沙,賊聞,悉詣營道乞降。進擊武陵蠻夷,斬首四千餘級,受降十餘萬人,荊州平定。詔書賜錢一億,固讓不受。振旅還京師,推功於從事中郎應奉,薦以為司隸校尉,而上書乞骸骨,朝廷不許。監軍使者張敞承宦官旨,奏緄將傅婢二人戎服自隨,又輒於江陵刻石紀功,請下吏案理。尚書令黃雋奏議,以為罪無正法,不合致糾。會長沙賊復起,攻桂陽、武陵,緄以軍還盜賊復發,策免。

頃之,拜將作大匠,轉河南尹。上言「舊典,中官子弟不得為牧人職」,帝不納。復為廷尉。時山陽太守單遷以罪繫獄,緄考致其死。遷,故車騎將軍單超之弟,中官相黨,遂共誹章誣緄,坐與司隸校尉李膺、大司農劉祐俱輸左校。應奉上疏理緄等,得免。後拜屯騎校尉,復為廷尉,卒於官。

緄弟允,清白有孝行,能理尚書,善推步之術。拜降虜校尉,終於家。

度尚字博平,山陽湖陸人也。家貧,不修學行,不為鄉里所推舉。積困窮,乃為宦者同郡侯覽視田,得為郡上計吏,拜郎中,除上虞長。為政嚴峻,明於發擿姦非,吏人謂之神明。遷文安令,遇時疾疫,穀貴人飢,尚開倉稟給,營救疾者,百姓蒙其濟。時冀州刺史朱穆行部,見尚甚奇之。

延熹五年,長沙、零陵賊合七八千人,自稱「將軍」,入桂陽、蒼梧、南海、交阯,交阯刺史及蒼梧太守望風逃奔,二郡皆沒。遣御史中丞盛修募兵討之,不能剋。豫章艾縣人六百餘人,應募而不得賞直,怨恚,遂反,焚燒長沙郡縣,寇益陽,殺縣令,眾漸盛。又遣謁者馬睦,督荊州刺史劉度擊之,軍敗,睦、度奔走。桓帝詔公卿舉任代劉度者,尚書朱穆舉尚,自右校令擢為荊州刺史。尚躬率部曲,與同勞逸,廣募雜種諸蠻夷,明設購賞,進擊,大破之,降者數萬人。桂陽宿賊渠帥卜陽、潘鴻等畏尚威烈,徙入山谷。尚窮追數百里,遂入南海,破其三屯,多獲珍寶。而陽、鴻等黨眾猶盛,尚欲擊之,而士卒驕富,莫有鬥志。尚計緩之則不戰,逼之必逃亡,乃宣言卜陽、潘鴻作賊十年,習於攻守,今兵寡少,未易可進,當須諸郡所發悉至,爾乃并力攻之。申令軍中,恣聽射獵。兵士喜悅,大小皆相與從禽。尚乃密使所親客潛焚其營,珍積皆盡。獵者來還,莫不泣涕。尚人人慰勞,深自咎責,因曰:「卜陽等財寶足富數世,諸卿但不并力耳。所亡少少,何足介意!」眾聞咸憤踊,尚敕令秣馬蓐食,明旦,徑赴賊屯。陽、鴻等自以深固,不復設備,吏士乘銳,遂大破平之。

尚出兵三年,群寇悉定。七年,封右鄉侯,遷桂陽太守。明年,徵還京師。時荊州兵朱蓋等,征戍役久,財賞不贍,忿恚,復作亂,與桂陽賊胡蘭等三千餘人復攻桂陽,焚燒郡縣,太守任胤棄城走,賊眾遂至數萬。轉攻零陵,太守陳球固守拒之。於是以尚為中郎將,將幽、冀、黎陽、烏桓步騎二萬六千人救球,又與長沙太守抗徐等發諸郡兵,并埶討擊,大破之,斬蘭等首三千五百級,餘賊走蒼梧。詔賜尚錢百萬,餘人各有差。

時抗徐與尚俱為名將,數有功。徐字伯徐,丹陽人,鄉邦稱其膽智。初試守宣城長,悉移深林遠藪椎髻鳥語之人置於縣下,由是境內無復盜賊。後為中郎將宗資別部司馬,擊太山賊公孫舉等,破平之,斬首三千餘級,封烏程東鄉侯五百戶。遷太山都尉,寇盜望風奔亡。及在長沙,宿賊皆平。卒於官。桓帝下詔追增封徐五百戶,并前千戶。

復以尚為荊州刺史。尚見胡蘭餘黨南走蒼梧,懼為己負,乃偽上言蒼梧賊入荊州界,於是徵交阯刺史張磐下廷尉。辭狀未正,會赦見原。磐不肯出獄,方更牢持械節,獄吏謂磐曰:「天恩曠然而君不出,何乎?」磐因自列曰:「前長沙賊胡蘭作難荊州,餘黨散入交阯。磐身嬰甲冑,涉危履險,討擊凶患,斬殄渠帥,餘盡鳥竄冒遁,還奔荊州。刺史度尚懼磐先言,怖畏罪戾,伏奏見誣。磐備位方伯,為國爪牙,而為尚所枉,受罪牢獄。夫事有虛實,法有是非。磐實不辜,赦無所除。如忍以苟免,永受侵辱之恥,生為惡吏,死為敝鬼。乞傳尚詣廷尉,面對曲直,足明真偽。尚不徵者,磐埋骨牢檻,終不虛出,望塵受枉。」廷尉以其狀上,詔書徵尚到廷尉,辭窮受罪,以先有功得原。磐字子石,丹陽人,以清白稱,終於廬江太守。

尚後為遼東太守,數月,鮮卑率兵攻尚,與戰,破之,戎狄憚畏。年五十,延熹九年,卒於官。

楊琁字機平,會稽烏傷人也。高祖父茂,本河東人,從光武征伐,為威寇將軍,封烏傷新陽鄉侯。建武中就國,傳封三世,有罪國除,因而家焉。父扶,交阯刺史,有理能名。兄喬,為尚書,容儀偉麗,數上言政事,桓帝愛其才貌,詔妻以公主,喬固辭不聽,遂閉口不食,七日而死。

琁初舉孝廉,稍遷,靈帝時為零陵太守。是時蒼梧、桂陽猾賊相聚,攻郡縣,賊眾多而琁力弱,吏人憂恐。琁乃特制馬車數十乘,以排囊盛石灰於車上,繫布索於馬尾,又為兵車,專彀弓弩,剋共會戰。乃令馬車居前,順風鼓灰,賊不得視,因以火燒布,然馬驚,奔突賊陣,因使後車弓弩亂發,鉦鼓鳴震。群盜波駭破散,追逐傷斬無數,梟其渠帥,郡境以清。荊州刺史趙凱,誣奏琁實非身破賊,而妄有其功。琁與相章奏,凱有黨助,遂檻車徵琁。防禁嚴密,無由自訟,乃噬臂出血,書衣為章,具陳破賊形埶,及言凱所誣狀,潛令親屬詣闕通之。詔書原琁,拜議郎,凱反受誣人之罪。

琁三遷為勃海太守,所在有異政,以事免。後尚書令張溫特表薦之,徵拜尚書僕射。以病乞骸骨,卒於家。

論曰:安順以後,風威稍薄,寇攘寖橫,緣隙而生,剽人盜邑者不闋時月,假署皇王者蓋以十數。或託驗神道,或矯妄冕服。然其雄渠魁長,未有聞焉,猶至壘盈四郊,奔命首尾。若夫數將者,並宣力勤慮,以勞定功,而景風之賞未甄,膚受之言互及。以此而推,政道難乎以免。

贊曰:張宗裨禹,敢殿後拒。江、淮、海、岱,虔劉寇阻。其誰清之?雄、尚、緄、撫。琁能用譎,亦云振旅。


\end{pinyinscope}