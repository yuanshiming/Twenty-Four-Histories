\article{張王种陳列傳}

\begin{pinyinscope}
張皓字叔明,犍為武陽人也。六世祖良,高帝時為太子少傅,封留侯。皓少游學京師,初永元中,歸仕州郡,辟大將軍鄧騭府,五遷尚書僕射,職事八年,出為彭城相。

永寧元年,徵拜廷尉。皓雖非法家,而留心刑斷,數與尚書辯正疑獄,多以詳當見從。時安帝廢皇太子為濟陰王,皓與太常桓焉、太僕來歷廷爭之,不能得。事已具來歷傳。退而上疏曰:「昔賊臣江充,造構讒逆,至令戾園興兵,終及禍難。後壺關三老一言,上乃覺悟,雖追前失,悔之何逮!今皇太子春秋方始十歲,未見保傅九德之義,宜簡賢輔,就成聖質。」書奏不省。

及順帝即位,拜皓司空,在事多所薦達,天下稱其推士。時清河趙騰上言災變,譏刺朝政,章下有司,收騰繫考,所引黨輩八十餘人,皆以誹謗當伏重法。皓上疏諫曰:「臣聞堯舜立敢諫之鼓,三王樹誹謗之木,春秋採善書惡,聖主不罪芻蕘。騰等雖干上犯法,所言本欲盡忠正諫。如當誅戮,天下杜口,塞諫爭之源,非所以昭德示後也。」帝乃悟,減騰死罪一等,餘皆司寇。四年,以陰陽不和策免。

陽嘉元年,復為廷尉。其年卒官,時年八十三。遣使者弔祭,賜葬地於河南縣。子綱。

綱字文紀。少明經學。雖為公子,而厲布衣之節。舉孝廉不就,司徒辟高第為御史。時順帝委縱宦官,有識危心。綱常感激,慨然歎曰:「穢惡滿朝,不能奮身出命埽國家之難,雖生吾不願也。」退而上書曰:「《詩》曰:『不愆不忘,率由舊章。』尋大漢初隆,及中興之世,文、明二帝,德化尤盛。觀其理為,易循易見,但恭儉守節,約身尚德而已。中官常侍不過兩人,近倖賞賜裁滿數金,惜費重人,故家給人足。夷狄聞中國優富,任信道德,所以姦謀自消而和氣感應。而頃者以來,不遵舊典,無功小人皆有官爵,富之驕之而復害之,非愛人重器,承天順道者也。伏願陛下少留聖思,割損左右,以奉天心。」書奏不省。

漢安元年,選遣八使徇行風俗,皆耆儒知名,多歷顯位,唯綱年少,官次最微。餘人受命之部,而綱獨埋其車輪於洛陽都亭,曰:「豺狼當路,安問狐狸!」遂奏曰:「大將軍冀,河南尹不疑,蒙外戚之援,荷國厚恩,以芻蕘之資,居阿衡之任,不能敷揚五教,翼讚日月,而專為封豕長蛇,肆其貪叨,甘心好貨,縱恣無底,多樹諂諛,以害忠良。誠天威所不赦,不辟所宜加也。謹條其無君之心十五事,斯皆臣子所切齒者也。」書御,京師震竦。時冀妹為皇后,內寵方盛,諸梁姻族滿朝,帝雖知綱言直,終不忍用。

時廣陵賊張嬰等眾數萬人,殺刺史、二千石,寇亂揚徐閒,積十餘年,朝廷不能討。冀乃諷尚書,以綱為廣陵太守,因欲以事中之。前遣郡守,率多求兵馬,綱獨請單車之職。既到,乃將吏卒十餘人,徑造嬰壘,以慰安之,求得與長老相見,申示國恩。嬰初大驚,既見綱誠信,乃出拜謁。綱延置上坐,問所疾苦。乃譬之曰:「前後二千石多肆貪暴,故致公等懷憤相聚。二千石信有罪矣,然為之者又非義也。今主上仁聖,欲以文德服叛,故遣太守,思以爵祿相榮,不願以刑罰相加,今誠轉禍為福之時也。若聞義不服,天子赫然震怒,荊、楊、兗、豫大兵雲合,豈不危乎?若不料彊弱,非明也;棄善取惡,非智也;去順效逆,非忠也;身絕血嗣,非孝也;背正從邪,非直也;見義不為,非勇也:六者成敗之幾,利害所從,公其深計之。」嬰聞,泣下,曰:「荒裔愚人,不能自通朝廷,不堪侵枉,遂復相聚偷生,若魚遊釜中,喘息須臾閒耳。今聞明府之言,乃嬰等更生之晨也。既陷不義,實恐投兵之日,不免孥戮。」綱約之以天地,誓之以日月,嬰深感悟,乃辭還營。明日,將所部萬餘人與妻子面縛歸降。綱乃單車入嬰壘,大會,置酒為樂,散遣部眾,任從所之;親為卜居宅,相田疇;子弟欲為吏者,皆引召之。人情悅服,南州晏然。朝廷論功當封,梁冀遏絕,乃止。天子嘉美,徵欲擢用綱,而嬰等上書乞留,乃許之。

綱在郡一年,年四十六卒。百姓老幼相攜,詣府赴哀者不可勝數。綱自被疾,吏人咸為祠祀祈福,皆言「千秋萬歲,何時復見此君」。張嬰等五百餘人制服行喪,送到犍為,負土成墳。詔曰:「故廣陵太守張綱,大臣之苗,剖符統務,正身導下,班宣德信,降集劇賊張嬰萬人,息干戈之役,濟蒸庶之困,未升顯爵,不幸早卒。嬰等縗杖,若喪考妣,朕甚愍焉!」拜綱子續為郎中,賜錢百萬。

王龔字伯宗,山陽高平人也。世為豪族。初舉孝廉,稍遷青州刺史,劾奏貪濁二千石數人,安帝嘉之,徵拜尚書。建光元年,擢為司隸校尉,明年遷汝南太守。政崇溫和,好才愛士,引進郡人黃憲、陳蕃等。憲雖不屈,蕃遂就吏。蕃性氣高明,初到,龔不即召見之,乃留記謝病去。龔怒,使除其錄。功曹袁閬請見,言曰:「聞之傳曰『人臣不見察於君,不敢立於朝』。蕃既以賢見引,不宜退以非禮。」龔改容謝曰:「是吾過也。」乃復厚遇待之。由是後進知名之士莫不歸心焉。閬字奉高。數辭公府之命,不修異操,而致名當時。

永建元年,徵龔為太僕,轉太常。四年,遷司空,以地震策免。

永和元年,拜太尉。在位恭慎,自非公事,不通州郡書記。其所辟命,皆海內長者。龔深疾宦官專權,志在匡正,乃上書極言其狀,請加放斥。諸黃門恐懼,各使賓客誣奏龔罪,順帝命亟自實。前掾李固時為大將軍梁商從事中郎,乃奏記於商曰:「今旦聞下太尉王公敕令自實,未審其事深淺何如。王公束脩厲節,敦樂蓺文,不求苟得,不為苟行,但以堅貞之操,違俗失眾,橫為讒佞所構毀,眾人聞知,莫不歎慄。夫三公尊重,承天象極,未有詣理訴冤之義。纖微感概,輒引分決,是以舊典不有大罪,不至重問。王公沈靜內明,不可加以非理。卒有它變,則朝廷獲害賢之名,群臣無救護之節矣。昔絳侯得罪,袁盎解其過,魏尚獲戾,馮唐訴其冤,時君善之,列在書傳。今將軍內倚至尊,外典國柄,言重信著,指撝無違,宜加表救,濟王公之艱難。語曰:『善人在患,飢不及餐。』斯其時也。」商即言之於帝,事乃得釋。

龔在位五年,以老病乞骸骨,卒於家。子暢。

論曰:張皓、王龔,稱為雅士,若其好通汲善,明發升薦,仁人之情也。夫士進則世收其器,賢用即人獻其能。能獻既已厚其功,器收亦理兼天下。其利甚博,而人莫之先,豈同折枝於長者,以不為為難乎?昔柳下惠見抑於臧文,淳于長受稱于方進。然則立德者以幽陋好遺,顯登者以貴塗易引。故晨門有抱關之夫,柱下無朱文之軫也。

暢字叔茂。少以清實為稱,無所交黨。初舉孝廉,辭病不就。大將軍梁商特辟舉茂才,四遷尚書令,出為齊相。徵拜司隸校尉,轉漁陽太守。所在以嚴明為稱。坐事免官。是時政事多歸尚書,桓帝特詔三公,令高選庸能。太尉陳蕃薦暢清方公正,有不可犯之色,由是復為尚書。

尋拜南陽太守。前後二千石逼懼帝鄉貴戚,多不稱職。暢深疾之,下車奮厲威猛,其豪黨有釁穢者,莫不糾發。會赦,事得散。暢追恨之,更為設法,諸受臧二千萬以上不自首實者,盡入財物;若其隱伏,使吏發屋伐樹,堙井夷灶,豪右大震。功曹張敞奏記諫曰:「五教在寬,著之經典。湯去三面,八方歸仁。武王入殷,先去炮格之刑。高祖鑒秦,唯定三章之法。孝文皇帝感一緹縈,蠲除肉刑。卓茂、文翁、召父之徒,皆疾惡嚴刻,務崇溫厚。仁賢之政,流聞後世。夫明哲之君,網漏吞舟之魚,然後三光明於上,人物悅於下。言之若迂,其效甚近。發屋伐樹,將為嚴烈,雖欲懲惡,難以聞遠。以明府上智之才,日月之曜,敷仁惠之政,則海內改觀,實有折枝之易,而無挾山之難。郡為舊都侯甸之國,園廟出於章陵,三后生自新野,士女沾教化,黔首仰風流,自中興以來,功臣將相,繼世而隆。愚以為懇懇用刑,不如行恩;孳孳求姦,未若禮賢。舜舉皋陶,不仁者遠。隨會為政,晉盜奔秦。虞、芮入境,讓心自生。化人在德,不在用刑。」暢深納敞諫,更崇寬政,慎刑簡罰,教化遂行。

郡中豪族多以奢靡相尚,暢常布衣皮褥,車馬羸敗,以矯其敝。同郡劉表時年十七,從暢受學。進諫曰:「夫奢不僭上,儉不逼下,循道行禮,貴處可否之閒。蘧伯玉恥獨為君子。府君不希孔聖之明訓,而慕夷齊之末操,無乃皎然自貴於世乎?」暢曰:「昔公儀休在魯,拔園葵,去織婦;孫叔敖相楚,其子被裘刈薪。夫以約失之鮮矣。聞伯夷之風者,貪夫廉,懦夫有立志。雖以不德,敢慕遺烈。」

後徵為長樂衛尉。建寧元年,遷司空,數月,以水災策免。明年,卒於家。

子謙,為大將軍何進長史。謙子粲,以文才知名。

种暠字景伯,河南洛陽人,仲山甫之後也。父為定陶令,有財三千萬。父卒,暠悉以賑卹宗族及邑里之貧者。其有進趣名利,皆不與交通。始為縣門下史。時河南尹田歆外甥王諶,名知人。歆謂之曰:「今當舉六孝廉,多得貴戚書命,不宜相違,欲自用一名士以報國家,爾助我求之。」明日,諶送客於大陽郭,遙見暠,異之。還白歆曰:「為尹得孝廉矣,近洛陽門下史也。」歆笑曰:「當得山澤隱滯,近洛陽吏邪?」諶曰:「山澤不必有異士,異士不必在山澤。」歆即召暠於庭,辯詰職事。暠辭對有序,歆甚知之,召署主簿,遂舉孝廉,辟太尉府,舉高第。

順帝末,為侍御史。時所遣八使光祿大夫杜喬、周舉等,多所糾奏,而大將軍梁冀及諸宦官互為請救,事皆被寑遏。暠自以職主刺舉,志案姦違,乃復劾諸為八使所舉蜀郡太守劉宣等罪惡章露,宜伏歐刀。又奏請敕四府條舉近臣父兄及知親為刺史、二千石尤殘穢不勝任者,免遣案罪。帝乃從之。擢暠監太子於承光宮。中常侍高梵從中單駕出迎太子,時太傅杜喬等疑不欲從,惶惑不知所為。暠乃手劍當車,曰:「太子國之儲副,人命所係。今常侍來無詔信,何以知非姦邪?今日有死而已。」梵辭屈,不敢對,馳命奏之。詔報,太子乃得去。喬退而歎息,愧暠臨事不惑。帝亦嘉其持重,稱善者良久。

出為益州刺史。暠素慷慨,好立功立事。在職三年,宣恩遠夷,開曉殊俗,岷山雜落皆懷服漢德。其白狼、槃木、唐菆、邛、僰諸國,自前刺史朱輔卒後遂絕;暠至,乃復舉種向化。時永昌太守冶鑄黃金為文蛇,以獻梁冀,暠糾發逮捕,馳傳上言,而二府畏懦,不敢案之,冀由是銜怒於暠。會巴郡人服直聚黨數百人,自稱「天王」,暠與太守應承討捕,不克,吏人多被傷害。冀因此陷之,傳逮暠、承。太尉李固上疏救曰:「臣伏聞討捕所傷,本非暠、承之意,實由縣吏懼法畏罪,迫逐深苦,致此不詳。比盜賊群起,處處未絕。暠、承以首舉大姦,而相隨受罪,臣恐沮傷州縣糾發之意,更共飾匿,莫復盡心。」梁太后省奏,乃赦暠、承罪,免官而已。

後涼州羌動,以暠為涼州刺史,甚得百姓歡心。被徵當遷,吏人詣闕請留之,太后歎曰:「未聞刺史得人心若是。」乃許之。暠復留一年,遷漢陽太守,戎夷男女送至漢陽界,暠與相揖謝,千里不得乘車。及到郡,化行羌胡,禁止侵掠。遷使匈奴中郎將。時遼東烏桓反叛,復轉遼東太守,烏桓望風率服,迎拜於界上。坐事免歸。

後司隸校尉舉暠賢良方正,不應。徵拜議郎,遷南郡太守,入為尚書。會匈奴寇并涼二州,桓帝擢暠為度遼將軍。暠到營所,先宣恩信,誘降諸胡,其有不服,然後加討。羌虜先時有生見獲質於郡縣者,悉遣還之。誠心懷撫,信賞分明,由是羌胡、龜茲、莎車、烏孫等皆來順服。暠乃去烽燧,除候望,邊方晏然無警。

入為大司農。延熹四年,遷司徒。推達名臣橋玄、皇甫規等,為稱職相。在位三年,年六十一薨。并、涼邊人咸為發哀。匈奴聞暠卒,舉國傷惜。單于每入朝賀,望見墳墓,輒哭泣祭祀。二子:岱,拂。

岱字公祖。好學養志。舉孝廉、茂才,辟公府,皆不就。公車特徵,病卒。

初,岱與李固子燮同徵議郎,燮聞岱卒,痛惜甚,乃上書求加禮於岱。曰:「臣聞仁義興則道德昌,道德昌則政化明,政化明而萬姓寧。伏見故處士种岱,淳和達理,耽悅詩書,富貴不能回其慮,萬物不能擾其心。稟命不永,奄然殂殞。若不槃桓難進,等輩皆已公卿矣。昔先賢既沒,有加贈之典,周禮盛德,有銘誄之文,而岱生無印綬之榮,卒無官謚之號。雖未建忠效用,而為聖恩所拔,遐邇具瞻,宜有異賞。」朝廷竟不能從。

拂字穎伯。初為司隸從事,拜宛令。時南陽郡吏好因休沐,游戲市里,為百姓所患。拂出逢之,必下車公謁,以愧其心,自是莫敢出者。政有能名,累遷光祿大夫。初平元年,代荀爽為司空。明年,以地震策免,復為太常。

李傕、郭汜之亂,長安城潰,百官多避兵衝。拂揮劍而出曰:「

為國大臣,不能止戈除暴,致使凶賊兵刃向宮,去欲何之!」遂戰而死。子劭。

劭字申甫。少知名。中平末,為諫議大夫。

大將軍何進將誅宦官,召并州牧董卓,至澠池,而進意更狐疑,遣劭宣詔止之。卓不受,遂前至河南。劭迎勞之,因譬令還軍。卓疑有變,使其軍士以兵脅劭。劭怒,稱詔大呼叱之,軍士皆披,遂前質責卓。卓辭屈,乃還軍夕陽亭。

及進敗,獻帝即位,拜劭為侍中。卓既擅權,而惡劭彊力,遂左轉議郎,出為益涼二州刺史。會父拂戰死,竟不之職。服終,徵為少府、大鴻臚,皆辭不受。曰:「昔我先父以身徇國,吾為臣子,不能除殘復怨,何面目朝覲明主哉!」遂與馬騰、韓遂及左中郎劉範、諫議大夫馬宇共攻李傕、郭汜,以報其仇。與汜戰於長平觀下,軍敗,劭等皆死。勝遂還涼州。

陳球字伯真,下邳淮浦人也。歷世著名。父亹,廣漢太守。球少涉儒學,善律令。陽嘉中,舉孝廉,稍遷繁陽令。時魏郡太守諷縣求納貨賄,球不與之,太守怒而撾督郵,欲令逐球。督郵不肯,曰:「魏郡十五城,獨繁陽有異政,今受命逐之,將致議於天下矣。」太守乃止。

復辟公府,舉高第,拜侍御史。是時,桂陽黠賊李研等群聚寇鈔,陸梁荊部,州郡懦弱,不能禁,太尉楊秉表球為零陵太守。球到,設方略,期月閒,賊虜消散。而州兵朱蓋等反,與桂陽賊胡蘭數萬人轉攻零陵。零陵下溼,編木為城,不可守備,郡中惶恐。掾史白遣家避難,球怒曰:「太守分國虎符,受任一邦,豈顧妻孥而沮國威重乎?復言者斬!」乃悉內吏人老弱,與共城守,弦大木為弓,羽矛為矢,引機發之,遠射千餘步,多所殺傷。賊復激流灌城,球輒於內因地埶反決水淹賊。相拒十餘日,不能下。會中郎將度尚將救兵至,球募士卒,與尚共破斬朱蓋等。賜錢五十萬,拜子一人為郎。遷魏郡太守。

徵拜將作大匠,作桓帝陵園,所省巨萬以上。遷南陽太守,以糾舉豪右,為執家所謗,徵詣廷尉抵罪。會赦,歸家。

復拜廷尉。熹平元年,竇太后崩。太后本遷南宮雲臺,宦者積怨竇氏,遂以衣車載后尸,置城南市舍數日。中常侍曹節、王甫欲用貴人體殯,帝曰:「太后親立朕躬,統承大業。《詩》云:『無德不報,無言不酬。』豈宜以貴人終乎?」於是發喪成禮。及將葬,節等復欲別葬太后,而以馮貴人配祔。詔公卿大會朝堂,令中常侍趙忠監議。太尉李咸時病,乃扶輿而起,擣椒自隨,謂妻子曰:「若皇太后不得配食桓帝,吾不生還矣。」既議,坐者數百人,各瞻望中官,良久莫肯先言。趙忠曰:「議當時定。」怪公卿以下各相顧望。球曰:「皇太后以盛德良家,母臨天下,宜配先帝,是無所疑。」忠笑而言曰:「陳廷尉宜便操筆。」球即下議曰:「皇太后自在椒房,有聰明母儀之德。遭時不造,援立聖明,承繼宗廟,功烈至重。先帝晏駕,因遇大獄,遷居空宮,不幸早世,家雖獲罪,事非太后。今若別葬,誠失天下之望。且馮貴人冢墓被發,骸骨暴露,與賊併尸,魂靈汙染,且無功於國,何宜上配至尊?」忠省球議,作色俛仰,蚩球曰:「陳廷尉建此議甚健!」球曰:「陳、竇既冤,皇太后無故幽閉,臣常痛心,天下憤歎。今日言之,退而受罪,宿昔之願。」公卿以下,皆從球議。李咸始不敢先發,見球辭正,然大言曰:「臣本謂宜爾,誠與臣意合。」會者皆為之愧。曹節、王甫復爭,以為梁后家犯惡逆,別葬懿陵,武帝黜廢衛后,而以李夫人配食。今竇氏罪深,豈得合葬先帝乎?李咸乃詣闕上疏曰:「臣伏惟章德竇后虐害恭懷,安思閻后家犯惡逆,而和帝無異葬之議,順朝無貶降之文。至於衛后,孝武皇帝身所廢棄,不可以為比。今長樂太后尊號在身,親嘗稱制,坤育天下,且援立聖明,光隆皇祚。太后以陛下為子,陛下豈得不以太后為母?子無黜母,臣無貶君,宜合葬宣陵,一如舊制。」帝省奏,謂曹節等曰:「竇氏雖為不道,而太后有德於朕,不宜降黜。」節等無復言,於是議者乃定。咸字元貞,汝南人。累經州郡,以廉幹知名;在朝清忠,權倖憚之。

六年,遷球司空,以地震免。拜光祿大夫,復為廷尉、太常。光和元年,遷太尉,數月,以日食免。復拜光祿大夫。明年,為永樂少府,乃潛與司徒河閒劉郃謀誅宦官。

初,郃兄侍中儵,與大將軍竇武同謀俱死,故郃與球相結。事未及發,球復以書勸郃曰:「公出自宗室,位登台鼎,天下瞻望,社稷鎮衛,豈得雷同容容無違而已?今曹節等放縱為害,而久在左右,又公兄侍中受害節等,永樂太后所親知也。今可表徙衛尉陽球為司隸校尉,以次收節等誅之。政出聖主,天下太平,可翹足而待也。」又尚書劉納以正直忤宦官,出為步兵校尉,亦深勸於郃。郃曰:「凶豎多耳目,恐事未會,先受其禍。」納曰:「公為國棟梁,傾危不持,焉用彼相邪?」郃許諾,亦結謀陽球。

球小妻,程璜之女,璜用事宮中,所謂程大人也。節等頗得聞知,乃重賂於璜,且脅之。璜懼迫,以球謀告節,節因共白帝曰:「郃等常與藩國交通,有惡意。數稱永樂聲埶,受取狼籍。步兵校尉劉納及永樂少府陳球、衛尉陽球交通書疏,謀議不軌。」帝大怒,策免郃,郃與球及劉納、陽球皆下獄死。球時年六十二。

子瑀,吳郡太守;瑀弟琮,汝陰太守;弟子珪,沛相;珪子登,廣陵太守:並知名。

贊曰:安儲遭譖,張卿有請。龔糾便佞,以直為眚。二子過正,埋車堙井。种公自微,臨官以威。陳球專議,桓思同歸。


\end{pinyinscope}