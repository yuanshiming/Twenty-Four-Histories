\article{左周黃列傳}

\begin{pinyinscope}
左雄字伯豪,南郡涅陽人也。安帝時,舉孝廉,稍遷冀州刺史。州部多豪族,好請託,雄常閉門不與交通。奏案貪猾二千石,無所回忌。

永建初,公車徵拜議郎。時順帝新立,大臣懈怠,朝多闕政,雄數言事,其辭深切。尚書僕射虞詡以雄有忠公節,上疏薦之曰:「臣見方今公卿以下,類多拱默,以樹恩為賢,盡節為愚,至相戒曰:『白璧不可為,容容多後福。』伏見議郎左雄,數上封事,至引陛下身遭難厄,以為警戒,實有王臣蹇蹇之節,周公謨成王之風。宜擢在喉舌之官,必有匡弼之益。」由是拜雄尚書,再遷尚書令。上疏陳事曰:

臣聞柔遠和邇,莫大寧人,寧人之務,莫重用賢,用賢之道,必存考黜。是以皋陶對禹,貴在知人。「安人則惠,黎民懷之。」分伯建侯,代位親民,民用和穆,禮讓以興。故《詩》云:「有渰淒淒,興雨祁祁。雨我公田,遂及我私。」及幽、厲昏亂,不自為政,褒豔用權,七子黨進,賢愚錯緒,深谷為陵。故其《詩》云:「四國無政,不用其良。」又曰:「哀今之人,胡為虺蜴?」言人畏吏如虺蜴也。宗周既滅,六國并秦,阬儒泯典,凛革五等,更立郡縣,縣設令長,郡置守尉,什伍相司,封豕其民。大漢受命,雖未復古,然克慎庶官,蠲苛救敝,悅以濟難,撫而循之。至於文、景,天下康乂。誠由玄靖寬柔,克慎官人故也。降及宣帝,興於仄陋,綜覈名實,知時所病,刺史守相,輒親引見,考察言行,信賞必罰。帝乃歎曰:「民所以安而無怨者,政平吏良也。與我共此者,其唯良二千石乎!」以為吏數變易,則下不安業;久於其事,則民服教化。其有政理者,輒以璽書勉勵,增秩賜金,或爵至關內侯,公卿缺則以次用之。是以吏稱其職,人安其業。漢世良吏,於茲為盛,故能降來儀之瑞,建中興之功。

漢初至今,三百餘載,俗浸彫敝,巧偽滋萌,下飾其詐,上肆其殘。典城百里,轉動無常,各懷一切,莫慮長久。謂殺害不辜為威風,聚斂整辨為賢能,以理己安民為劣弱,以奉法循理為不化。髡鉗之戮,生於睚眥;覆尸之禍,成於喜怒。視民如寇讎,稅之如豺虎。監司項背相望,與同疾疢,見非不舉,聞惡不察,觀政於亭傳,責成於期月,言善不稱德,論功不據實,虛誕者獲譽,拘檢者離毀。或因罪而引高,或色斯以求名。州宰不覆,競共辟召,踊躍升騰,超等踰匹。或考奏捕案,而亡不受罪,會赦行賂,復見洗滌。朱紫同色,清濁不分。故使姦猾枉濫,輕忽去就,拜除如流,缺動百數。鄉官部吏,職斯祿薄,車馬衣服,一出於民,廉者取足,貪者充家,特選橫調,紛紛不絕,送迎煩費,損政傷民。和氣未洽,災眚不消,咎皆在此。今之墨綬,猶古之諸侯,拜爵王庭,輿服有庸,而齊於匹豎,叛命避負,非所以崇憲明理,惠育元元也。臣愚以為守相長吏,惠和有顯效者,可就增秩,勿使移徙,非父母喪不得去官。其不從法禁,不式王命,錮之終身,雖會赦令,不得齒列。若被劾奏,亡不就法者,徙家邊郡,以懲其後。鄉部親民之吏,皆用儒生清白任從政者,寬其負筭,增其秩祿,吏職滿歲,宰府州郡乃得辟舉。如此,威福之路塞,虛偽之端絕,送迎之役損,賦斂之源息。循理之吏,得成其化;率土之民,各寧其所。追配文、宣中興之軌,流光垂祚,永世不刊。

帝感其言,申下有司,考其真偽,詳所施行。雄之所言,皆明達政體,而宦豎擅權,終不能用。自是選代交互,令長月易,迎新送舊,勞擾無已,或官寺空曠,無人案事,每選部劇,乃至逃亡。

永建三年,京師、漢陽地皆震裂,水泉涌出。四年,司、冀復有大水。雄推較災異,以為下人有逆上之徵,又上疏言:「宜密為備,以俟不虞。」尋而青、冀、楊州盜賊連發,數年之閒,海內擾亂。其後天下大赦,賊雖頗解,而官猶無備,流叛之餘,數月復起。雄與僕射郭虔共上疏,以為「寇賊連年,死亡太半,一人犯法,舉宗群亡。宜及其尚微,開令改悔。若告黨與者,聽除其罪;能誅斬者,明加其賞」。書奏,並不省。

又上言:「宜崇經術,繕脩太學。」帝從之。陽嘉元年,太學新成,詔試明經者補弟子,增甲乙之科,員各十人。除京師及郡國耆儒年六十以上為郎、舍人、諸王國郎者百三十八人。

雄又上言:「郡國孝廉,古之貢士,出則宰民,宣協風教。若其面牆,則無所施用。孔子曰『四十不惑』,禮稱『強仕』。請自今孝廉年不滿四十,不得察舉,皆先詣公府,諸生試家法,文吏課牋奏,副之端門,練其虛實,以觀異能,以美風俗。有不承科令者,正其罪法。若有茂才異行,自可不拘年齒。」帝從之,於是班下郡國。明年,有廣陵孝廉徐淑,年未及舉,臺郎疑而詰之。對曰:「詔書曰『有如顏回、子奇,不拘年齒』,是故本郡以臣充選。」郎不能屈。雄詰之曰:「昔顏回聞一知十,孝廉聞一知幾邪?」淑無以對,乃譴卻郡。於是濟陰太守胡廣等十餘人皆坐謬舉免黜,唯汝南陳蕃、潁川李膺、下邳陳球等三十餘人得拜郎中。自是牧守畏慄,莫敢輕舉。迄于永嘉,察選清平,多得其人。

雄又奏徵海內名儒為博士,使公卿子弟為諸生。有志操者,加其俸祿。及汝南謝廉,河南趙建,年始十二,各能通經,雄並奏拜童子郎。於是負書來學,雲集京師。

初,帝廢為濟陰王,乳母宋娥與黃門孫程等共議立帝,帝後以娥前有謀,遂封為山陽君,邑五千戶。又封大將軍梁商子冀襄邑侯。雄上封事曰:「夫裂土封侯,王制所重。高皇帝約,非劉氏不王,非有功不侯。孝安皇帝封江京、王聖等,遂致地震之異。永建二年,封陰謀之功,又有日食之變。數術之士,咸歸咎於封爵。今青州飢虛,盜賊未息,民有乏絕,上求稟貸。陛下乾乾勞思,以濟民為務。宜循古法,寧靜無為,以求天意,以消災異。誠不宜追錄小恩,虧失大典。」帝不聽。雄復諫曰:「臣聞人君莫不好忠正而惡讒諛,然而歷世之患,莫不以忠正得罪,讒諛蒙倖者,蓋聽忠難,從諛易也。夫刑罪,人情之所甚惡;貴寵,人情之所甚欲。是以時俗為忠者少,而習諛者多。故令人主數聞其美,稀知其過,迷而不悟,至於危亡。臣伏見詔書顧念阿母舊德宿恩,欲特加顯賞。案尚書故事,無乳母爵邑之制,唯先帝時阿母王聖為野王君。聖造生讒賊廢立之禍,生為天下所咀嚼,死為海內所歡快。桀、紂貴為天子,而庸僕羞與為比者,以其無義也。夷、齊賤為匹夫,而王侯爭與為伍者,以其有德也。今阿母躬蹈約儉,以身率下,群僚蒸庶,莫不向風,而與王聖並同爵號,懼違本操,失其常願。臣愚以為凡人之心,理不相遠,其所不安,古今一也。百姓深懲王聖傾覆之禍,民萌之命,危於累卵,常懼時世復有此類。怵惕之念,未離於心;恐懼之言,未絕乎口。乞如前議,歲以千萬給奉阿母,內足以盡恩愛之歡,外可不為吏民所怪。梁冀之封,事非機急,宜過災厄之運,然後平議可否。」會復有地震、緱氏山崩之異,雄復上疏諫曰:「先帝封野王君,漢陽地震,今封山陽君而京城復震,專政在陰,其災尤大。臣前後瞽言封爵至重,王者可私人以財,不可以官,宜還阿母之封,以塞災異。今冀已高讓,山陽君亦宜崇其本節。」雄言數切至,娥亦畏懼辭讓,而帝戀戀不能已,卒封之。後阿母遂以交遘失爵。

是時大司農劉據以職事被譴,召詣尚書,傳呼促步,又加以捶撲。雄上言:「九卿位亞三事,班在大臣,行有佩玉之節,動有庠序之儀。孝明皇帝始有撲罰,皆非古典。」帝從而改之,其後九卿無復捶撲者。自雄掌納言,多所匡肅,每有章表奏議,臺閣以為故事。遷司隸校尉。

初,雄薦周舉為尚書,舉既稱職,議者咸稱焉。及在司隸,又舉故冀州刺史馮直以為將帥,而直嘗坐臧受罪,舉以此劾奏雄。雄悅曰:「吾嘗事馮直之父而又與直善,今宣光以此奏吾,乃是韓厥之舉也。」由是天下服焉。明年坐法免。後復為尚書。永和三年卒。

周舉字宣光,汝南汝陽人,陳留太守防之子。防在儒林傳。舉姿貌短陋,而博學洽聞,為儒者所宗,故京師為之語曰:「五經從橫周宣光。」

延熹四年,辟司徒李郃府。時宦者孫程等既立順帝,誅滅諸閻,議郎陳禪以為閻太后與帝無母子恩,宜徙別館,絕朝見。群臣議者咸以為宜。舉謂郃曰:「昔鄭武姜謀殺嚴公,嚴公誓之黃泉;秦始皇怨母失行,久而隔絕,後感潁考叔、茅焦之言,循復子道。書傳美之。今諸閻新誅,太后幽在離宮,若悲愁生疾,一日不虞,主上將何以令於天下?如從禪議,後世歸咎明公。宜密表朝廷,令奉太后,率厲群臣,朝覲如舊,以厭天心,以荅人望。」郃即上疏陳之。明年正月,帝乃朝于東宮,太后由此以安。

後長樂少府朱倀代郃為司徒,舉猶為吏。時孫程等坐懷表上殿爭功,帝怒,悉徙封遠縣,敕洛陽令促期發遣。舉說朱倀曰:「朝廷在西鍾下時,非孫程等豈立?雖韓、彭、吳、賈之功,何以加諸!今忘其大德,錄其小過,如道路夭折,帝有殺功臣之譏。及今未去,宜急表之。」倀曰:「今詔怒,二尚書已奏其事,吾獨表此,必致罪譴。」舉曰:「明公年過八十,位為台輔,不於今時竭忠報國,惜身安寵,欲以何求?祿位雖全,必陷佞邪之譏;諫而獲罪,猶有忠貞之名。若舉言不足採,請從此辭。」倀乃表諫,帝果從之。

舉後舉茂才,為平丘令。上書言當世得失,辭甚切正。尚書郭虔、應賀等見之歎息,共上疏稱舉忠直,欲帝置章御坐,以為規誡。

舉稍遷并州刺史。太原一郡,舊俗以介子推焚骸,有龍忌之禁。至其亡月,咸言神靈不樂舉火,由是士民每冬中輒一月寒食,莫敢煙爨,老小不堪,歲多死者。舉既到州,乃作弔書以置子推之廟,言盛冬去火,殘損民命,非賢者之意,以宣示愚民,使還溫食。於是眾惑稍解,風俗頗革。

轉冀州刺史。陽嘉三年,司隸校尉左雄薦舉,徵拜尚書。舉與僕射黃瓊同心輔政,名重朝廷,左右憚之。是歲河南、三輔大旱,五穀災傷,天子親自露坐德陽殿東廂請雨,又下司隸、河南禱祀河神、名山、大澤。詔書以舉才學優深,特下策問曰:「朕以不德,仰承三統,夙興夜寐,思協大中。頃年以來,旱災屢應,稼穡焦枯,民食困乏。五品不訓,王澤未流,群司素餐,據非其位。審所貶黜,變復之徵,厥效何由?分別具對,勿有所諱。」舉對曰:「臣聞易稱『天尊地卑,乾坤以定』。二儀交構,乃生萬物,萬物之中,以人為貴。故聖人養之以君,成之以化,順四節之宜,適陰陽之和,使男女婚娶不過其時。包之以仁恩,導之以德教,示之以災異,訓之以嘉祥。此先聖承乾養物之始也。夫陰陽閉隔,則二氣否塞;二氣否塞,則人物不昌;人物不昌,則風雨不時;風雨不時,則水旱成災。陛下處唐虞之位,未行堯舜之政,近廢文帝、光武之法,而循亡秦奢侈之欲,內積怨女,外有曠夫。今皇嗣不興,東宮未立,傷和逆理,斷絕人倫之所致也。非但陛下行此而已,豎宦之人,亦復虛以形埶,威侮良家,取女閉之,至有白首歿無配偶,逆於天心。昔武王入殷,出傾宮之女;成湯遭災,以六事剋己;魯僖遇旱,而自責祈雨:皆以精誠轉禍為福。自枯旱以來,彌歷年歲,未聞陛下改過之效,徒勞至尊暴露風塵,誠無益也。又下州郡祈神致請。昔齊有大旱,景公欲祀河伯,晏子諫曰:『不可。夫河伯以水為城國,魚鱉為民庶。水盡魚枯,豈不欲雨?自是不能致也。』陛下所行,但務其華,不尋其實,猶緣木希魚,卻行求前。誠宜推信革政,崇道變惑,出後宮不御之女,理天下冤枉之獄,除太官重膳之費。夫五品不訓,責在司徒,有非其位,宜急黜斥。臣自藩外擢典納言,學薄智淺,不足以對。易傳曰:『陽感天,不旋日。』惟陛下留神裁察。」因召見舉及尚書令成翊世、僕射黃瓊,問以得失。舉等並對以為宜慎官人,去斥貪汙,離遠佞邪,循文帝之儉,尊孝明之教,則時雨必應。帝曰:「百官貪汙佞邪者為誰乎?」舉獨對曰:「臣從下州,超備機密,不足以別群臣。然公卿大臣數有直言者,忠貞也;阿諛苟容者,佞邪也。司徒視事六年,未聞有忠言異謀,愚心在此。」其後以事免司徒劉崎,遷舉司隸校尉。

永和元年,災異數見,省內惡之,詔召公、卿、中二千石、尚書詣顯親殿,問曰:「言事者多云,昔周公攝天子事,及薨,成王欲以公禮葬之,天為動變。及更葬以天子之禮,即有反風之應。北鄉侯親為天子而葬以王禮,故數有災異,宜加尊謚,列於昭穆。」群臣議者多謂宜如詔旨,舉獨對曰:「昔周公有請命之應,隆太平之功,故皇天動威,以章聖德。北鄉侯本非正統,姦臣所立,立不踰歲,年號未改,皇天不祐,大命夭昏。春秋王子猛不稱崩,魯子野不書葬。今北鄉侯無它功德,以王禮葬之,於事已崇,不宜稱謚。災眚之來,弗由此也。」於是司徒黃尚、太常桓焉等七十人同舉議,帝從之。尚字伯河,南郡人也,少歷顯位,亦以政事稱。

舉出為蜀郡太守,坐事免。大將軍梁商表為從事中郎,甚敬重焉。六年三月上巳日,商大會賓客,讌于洛水,舉時稱疾不往。商與親暱酣飲極歡,及酒闌倡罷,繼以筹露之歌,坐中聞者,皆為掩涕。太僕張种時亦在焉,會還,以事告舉。舉歎曰:「此所謂哀樂失時,非其所也。殃將及乎!」商至秋果薨。商疾篤,帝親臨幸,問以遺言。對曰:「人之將死,其言也善。臣從事中郎周舉,清高忠正,可重任也。」由是拜舉諫議大夫。

時連有災異,帝思商言,召舉於顯親殿,問以變眚。舉對曰:「陛下初立,遵脩舊典,興化致政,遠近肅然。頃年以來,稍違於前,朝多寵倖,祿不序德。觀天察人,準今方古,誠可危懼。書曰:『僭恆暘若。』夫僭差無度,則言不從而下不正;陽無以制,則上擾下竭。宜密嚴敕州郡,察彊宗大姦,以時禽討。」其後江淮猾賊周生、徐鳳等處處並起,如舉所陳。

時詔遣八使巡行風俗,皆選素有威名者,乃拜舉為侍中,舉侍中杜喬、守光祿大夫周栩、前青州刺史馮羨、尚書欒巴、侍御史張綱、兗州刺史郭遵、太尉長史劉班並守光祿大夫,分行天下。其刺史、二千石有臧罪顯明者,驛馬上之;墨綬以下,便輒收舉。其有清忠惠利,為百姓所安,宜表異者,皆以狀上。於是八使同時俱拜,天下號曰「八俊」。舉於是劾奏貪猾,表薦公清,朝廷稱之。遷河內太守,徵為大鴻臚。

及梁太后臨朝,詔以殤帝幼崩,廟次宜在順帝下。太常馬訪奏宜如詔書,諫議大夫呂勃以為應依昭穆之序,先殤帝,後順帝。詔下公卿。舉議曰:「春秋魯閔公無子,庶兄僖公代立,其子文公遂躋僖於閔上。孔子譏之,書曰:『有事于太廟,躋僖公。』傳曰:『逆祀也。』及定公正其序,經曰『從祀先公』,為萬世法也。今殤帝在先,於秩為父,順帝在後,於親為子,先後之義不可改,昭穆之序不可亂。呂勃議是也。」太后下詔從之。遷光祿勳,會遭母憂去職,後拜光祿大夫。

建和三年卒。朝廷以舉清公亮直,方欲以為宰相,深痛惜之。乃詔告光祿勳、汝南太守曰:「昔在前世,求賢如渴,封墓軾閭,以光賢哲。故公叔見誄,翁歸蒙述,所以昭忠厲俗,作範後昆。故光祿大夫周舉,性侔夷、魚,忠踰隨、管,前授牧守,及還納言,出入京輦,有欽哉之績,在禁闈有密靜之風。予錄乃勳,用登九列。方欲式序百官,亮協三事,不永夙終,用乖遠圖。朝廷愍悼,良為愴然。詩不云乎:『肇敏戎功,用錫爾祉。』其令將大夫以下到喪發日復會弔。加賜錢十萬,以旌委蛇素絲之節焉。」子勰。

勰字巨勝,少尚玄虛,以父任為郎,自免歸家。父故吏河南召夔為郡將,卑身降禮,致敬於勰。勰恥交報之,因杜門自絕。後太守舉孝廉,復以疾去。時梁冀貴盛,被其徵命者,莫敢不應,唯勰前後三辟,竟不能屈。後舉賢良方正,不應。又公車徵,玄纁備禮,固辭廢疾。常隱處竄身,慕老聃清靜,杜絕人事,巷生荊棘,十有餘歲。至延熹二年,乃開門延賓,游談宴樂,及秋而梁冀誅,年終而勰卒,時年五十。蔡邕以為知命。自勰曾祖父揚至勰孫恂,六世一身,皆知名云。

黃瓊字世英,江夏安陸人,魏郡太守香之子也。香在文苑傳。瓊初以父任為太子舍人,辭病不就。遭父憂,服闋,五府俱辟,連年不應。

永建中,公卿多薦瓊者,於是與會稽賀純、廣漢楊厚俱公車徵。瓊至綸氏,稱疾不進。有司劾不敬,詔下縣以禮慰遣,遂不得已。先是徵聘處士多不稱望,李固素慕於瓊,乃以書逆遺之曰:「聞已度伊、洛,近在萬歲亭,豈即事有漸,將順王命乎?蓋君子謂伯夷隘,柳下惠不恭,故傳曰『不夷不惠,可否之閒』。蓋聖賢居身之所珍也。誠遂欲枕山棲谷,擬跡巢、由,斯則可矣;若當輔政濟民,今其時也。自生民以來,善政少而亂俗多,必待堯舜之君,此為志士終無時矣。常聞語曰:『嶢嶢者易缺,皦皦者易汙。』陽春之曲,和者必寡,盛名之下,其實難副。近魯陽樊君被徵初至,朝廷設壇席,猶待神明。雖無大異,而言行所守無缺。而毀謗布流,應時折減者,豈非觀聽望深,聲名太盛乎?自頃徵聘之士,胡元安、薛孟嘗、朱仲昭、顧季鴻等,其功業皆無所採,是故俗論皆言處士純盜虛聲。願先生弘此遠謨,令眾人歎服,一雪此言耳。」瓊至,即拜議郎,稍遷尚書僕射。

初,瓊隨父在臺閣,習見故事。及後居職,達練官曹,爭議朝堂,莫能抗奪。時連有災異,瓊上疏順帝曰:「閒者以來,卦位錯謬,寒燠相干,蒙氣數興,日闇月散。原之天意,殆不虛然。陛下宜開石室,案河洛,外命史官,悉條上永建以前至漢初災異,與永建以後訖于今日,孰為多少。又使近臣儒者參考政事,數見公卿,察問得失。諸無功德者,宜皆斥黜。臣前頗陳災眚,并薦光祿大夫樊英、太中大夫薛包及會稽賀純、廣漢楊厚,未蒙御省。伏見處士巴郡黃錯、漢陽任棠,年皆耆耋,有作者七人之志。宜更見引致,助崇大化。」於是有詔公車徵錯等。

三年,大旱,瓊復上疏曰:「昔魯僖遇旱,以六事自讓,躬節儉,閉女謁,於讒佞者十三人,誅稅民受貨者九人,退舍南郊,天立大雨。今亦宜顧省政事,有所損闕,務存質儉,以易民聽。尚方御府,息除煩費。明敕近臣,使遵法度,如有不移,示以好惡。數見公卿,引納儒士,訪以政化,使陳得失。又囚徒尚積,多致死亡,亦足以感傷和氣,招降災旱。若改敝從善,擇用嘉謀,則災消福至矣。」書奏,引見德陽殿,使中常侍以瓊奏書屬主者施行。

自帝即位以後,不行籍田之禮。瓊以國之大典不宜久廢,上疏奏曰:「自古聖帝哲王,莫不敬恭明祀,增致福祥,故必躬郊廟之禮,親籍田之勤,以先群萌,率勸農功。昔周宣王不籍千畝,总文公以為大譏,卒有姜戎之難,終損中興之名。竊見陛下遵稽古之鴻業,體虔肅以應天,順時奉元,懷柔百神,朝夕觸塵埃於道路,晝暮聆庶政以卹人。雖詩詠成湯之不怠遑,書美文王之不暇食,誠不能加。今廟祀適闋,而祈穀絜齋之事,近在明日。臣恐左右之心,不欲屢動聖躬,以為親耕之禮,可得而廢。臣聞先王制典,籍田有日,司徒咸戒,司空除壇。先時五日,有協風之應,王即齋宮,饗醴載耒,誠重之也。自癸巳以來,仍西北風,甘澤不集,寒涼尚結。迎春東郊,既不躬親,先農之禮,所宜自勉,以逆和氣,以致時風。《易》曰:『君子自強不息。』斯其道也。」書奏,帝從之。

頃之,遷尚書令。瓊以前左雄所上孝廉之選,專用儒學文吏,於取士之義,猶有所遺,乃奏增孝悌及能從政者為四科,事竟施行。又雄前議舉吏先試之於公府,又覆之於端門,後尚書張盛奏除此科。瓊復上言:「覆試之作,將以澄洗清濁,覆實虛濫,不宜改革。」帝乃止。出為魏郡太守,稍遷太常。和平中,以選入侍講禁中。

元嘉元年,遷司空。桓帝欲褒崇大將軍梁冀,使中朝二千石以上會議其禮。特進胡廣、太常羊溥、司隸校尉祝恬、太中大夫邊韶等,咸稱冀之勳德,其制度賚賞,以宜比周公,錫之山川、土田、附庸。瓊獨建議曰:「冀前以親迎之勞,增邑三千,又其子胤亦加封賞。昔周公輔相成王,制禮作樂,化致太平,是以大啟土宇,開地七百。今諸侯以戶邑為制,不以里數為限。蕭何識高祖於泗水,霍光定傾危以興國,皆益戶增封,以顯其功。冀可比鄧禹,合食四縣,賞賜之差,同於霍光,使天下知賞必當功,爵不越德。」朝廷從之。冀意以為恨。會以地動策免。復為太僕。

永興元年,遷司徒,轉太尉。梁冀前後所託辟召,一無所用。雖有善人而為冀所飾舉者,亦不加命。延熹元年,以日食免。復為大司農。明年,梁冀被誅,太尉胡廣、司徒韓縯、司空孫朗皆坐阿附免廢,復拜瓊為太尉。以師傅之恩,而不阿梁氏,乃封為邟鄉侯,邑千戶。瓊辭疾讓封六七上,言旨懇惻,乃許之。梁冀既誅,瓊首居公位,舉奏州郡素行貪汙至死徙者十餘人,海內由是翕然望之。尋而五侯擅權,傾動內外,自度力不能匡,乃稱疾不起。四年,以寇賊免。其年復為司空。秋,以地震免。

七年,疾篤,上疏諫曰:「臣聞天者務剛其氣,君者務彊其政。是以王者處高自持,不可不安;履危任力,不可不據。夫自持不安則顛,任力不據則危。故聖人升高據上,則以德義為首;涉危蹈傾,則以賢者為力。唐堯以德化為冠冕,以稷、契為筋力。高而益崇,動而愈據,此先聖所以長守萬國,保其社稷者也。昔高皇帝應天順民,奮劍而王,埽除秦、項,革命創制,降德流祚。至於哀、平,而帝道不綱,秕政日亂,遂使姦佞擅朝,外戚專恣。所冠不以仁義為冕,所蹈不以賢佐為力,終至顛蹶,滅絕漢祚。天維陵敘,民鬼慘愴,賴皇乾眷命,炎德復輝。光武以聖武天挺,繼統興業,創基冰泮之上,立足枳棘之林。擢賢於眾愚之中,畫功於無形之世。崇禮義於交爭,循道化於亂離。是自歷高而不傾,任力危而不跌,興復洪祚,開建中興,光被八極,垂名無窮。至於中葉,盛業漸衰。陛下初從藩國,爰升帝位,天下拭目,謂見太平。而即位以來,未有勝政。諸梁秉權,豎宦充朝,重封累職,傾動朝廷,卿校牧守之選,皆出其門,羽毛齒革、明珠南金之寶,殷滿其室,富擬王府,埶回天地。言之者必族,附之者必榮。忠臣懼死而杜口,萬夫怖禍而木舌,塞陛下耳目之明,更為聾瞽之主。故太尉李固、杜喬,忠以直言,德以輔政,念國亡身,隕歿為報,而坐陳國議,遂見殘滅。賢愚切痛,海內傷懼。又前白馬令李雲,指言宦官罪穢宜誅,皆因眾人之心,以救積薪之敝。弘農杜眾,知雲所言宜行,懼雲以忠獲罪,故上書陳理之,乞同日而死,所以感悟國家,庶雲獲免。而雲既不辜,眾又并坐,天下尤痛,益以怨結,故朝野之人,以忠為諱。昔趙殺鳴犢,孔子臨河而反。夫覆巢破卵,則鳳皇不翔;刳牲夭胎,則麒麟不臻。誠物類相感,理使其然。尚書周永,昔為沛令,素事梁冀,幸其威埶,坐事當罪,越拜令職。見冀將衰,乃陽毀示忠,遂因姦計,亦取封侯。又黃門協邪,群輩相黨,自冀興盛,腹背相親,朝夕圖謀,共搆姦軌。臨冀當誅,無可設巧,復記其惡,以要爵賞。陛下不加清澂,審別真偽,復與忠臣並時顯封,使朱紫共色,粉墨雜蹂,所謂扺金玉於沙礫,碎珪璧於泥塗。四方聞之,莫不憤歎。昔曾子大孝,慈母投杼;伯奇至賢,終於流放。夫讒諛所舉,無高而不可升;相抑,無深而不可淪。可不察歟?臣至頑駑,世荷國恩,身輕位重,勤不補過,然懼於永歿,負釁益深。敢以垂絕之日,陳不諱之言,庶有萬分,無恨三泉。」其年卒,時年七十九。贈車騎將軍,謚曰忠侯。孫琬。

琬字子琰。少失父。早而辯慧。祖父瓊,初為魏郡太守,建和元年正月日食,京師不見而瓊以狀聞。太后詔問所食多少,瓊思其對而未知所況。琬年七歲,在傍,曰:「何不言日食之餘,如月之初?」瓊大驚,即以其言應詔,而深奇愛之。後瓊為司徒,琬以公孫拜童子郎,辭病不就,知名京師。時司空盛允有疾,瓊遣琬候問,會江夏上蠻賊事副府,允發書視畢,微戲琬曰:「江夏大邦,而蠻多士少。」琬奉手對曰:「蠻夷猾夏,責在司空。」因拂衣辭去。允甚奇之。

稍遷五官中郎將。時陳蕃為光祿勳,深相敬待,數與議事。舊制,光祿舉三署郎,以高功久次才德尤異者為茂才四行。時權富子弟多以人事得舉,而貧約守志者以窮退見遺,京師為之謠曰:「欲得不能,光祿茂才。」於是琬、蕃同心,顯用志士,平原劉醇、河東朱山、蜀郡殷參等並以才行蒙舉。蕃、琬遂為權富郎所見中傷,事下御史丞王暢、侍御史刁韙。韙、暢素重蕃、琬,不舉其事,而左右復陷以朋黨。暢坐左轉議郎而免蕃官,琬、韙俱禁錮。

韙字子榮,彭城人。後陳蕃被徵,而言事者多訟韙,復拜議郎,遷尚書。在朝有鯁直節,出為魯、東海二郡相。性抗厲,有明略,所在稱神。常以法度自整,家人莫見墯容焉。

琬被廢棄幾二十年。至光和末,太尉楊賜上書薦琬有撥亂之才,由是徵拜議郎,擢為青州刺史,遷侍中。中平初,出為右扶風,徵拜將作大匠、少府、太僕。又為豫州牧。時寇賊陸梁,州境彫殘,琬討擊平之,威聲大震。政績為天下表,封關內侯。

及董卓秉政,以琬名臣,徵為司徒,遷太尉,更封陽泉鄉侯。卓議遷都長安,琬與司徒楊彪同諫不從。琬退而駮議之曰:「昔周公營洛邑以寧姬,光武卜東都以隆漢,天之所啟,神之所安。大業既定,豈宜妄有遷動,以虧四海之望?」時人懼卓暴怒,琬必及害,固諫之。琬對曰:「昔白公作亂於楚,屈廬冒刃而前;崔杼弒君於齊,晏嬰不懼其盟。吾雖不德,誠慕古人之節。」琬竟坐免。卓猶敬其名德舊族,不敢害。後與楊彪同拜光祿大夫,及徙西都,轉司隸校尉,與司徒王允同謀誅卓。及卓將李傕、郭汜攻破長安,遂收琬下獄死,時年五十二。

論曰:古者諸侯歲貢士,進賢受上賞,非賢貶爵土。升之司馬,辯論其才,論定然後官之,任官然後祿之。故王者得其人,進仕勸其行,經邦弘務,所由久矣。漢初詔舉賢良、方正,州郡察孝廉、秀才,斯亦貢士之方也。中興以後,復增敦朴、有道、賢能、直言、獨行、高節、質直、清白、敦厚之屬。榮路既廣,觖望難裁,自是竊名偽服,浸以流競。權門貴仕,請謁繁興。自左雄任事,限年試才,雖頗有不密,固亦因識時宜。而黃瓊、胡廣、張衡、崔瑗之徒,泥滯舊方,互相詭駮,循名者屈其短,筭實者挺其效。故雄在尚書,天下不敢妄選,十餘年閒,稱為得人,斯亦效實之徵乎?順帝始以童弱反政,而號令自出,知能任使,故士得用情,天下喁喁仰其風采。遂乃備玄纁玉帛,以聘南陽樊英,天子降寢殿,設壇席,尚書奉引,延問失得。急登賢之舉,虛降己之禮,於是處士鄙生,忘其拘儒,拂巾衽褐,以企旌車之招矣。至乃英能承風,俊乂咸事,若李固、周舉之淵謨弘深,左雄、黃瓊之政事貞固,桓焉、楊厚以儒學進,崔瑗、馬融以文章顯,吳祐、蘇章、种暠、欒巴牧民之良幹,龐參、虞詡將帥之宏規,王龔、張皓虛心以推士,張綱、杜喬直道以糾違,郎顗陰陽詳密,張衡機術特妙:東京之士,於茲盛焉。向使廟堂納其高謀,彊場宣其智力,帷幄容其謇辭,舉厝稟其成式,則武、宣之軌,豈其遠而?《詩》云:「靡不有初,鮮克有終。」可為恨哉!及孝桓之時,碩德繼興,陳蕃、楊秉處稱賢宰,皇甫、張、段出號名將,王暢、李膺彌縫袞闕,朱穆、劉陶獻替匡時,郭有道獎鑒人倫,陳仲弓弘道下邑。其餘宏儒遠智,高心絜行,激揚風流者,不可勝言。而斯道莫振,文武陵隊,在朝者以正議嬰戮,謝事者以黨錮致災。往車雖折,而來軫方遒。所以傾而未顛,決而未潰,豈非仁人君子心力之為乎?嗚呼!

贊曰:雄作納言,古之八元。舉升以彙,越自下蕃。登朝理政,並紓災昏。瓊名夙知,累章國疵。琬亦早秀,位及志差。


\end{pinyinscope}