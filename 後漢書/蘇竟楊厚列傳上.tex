\article{蘇竟楊厚列傳上}

\begin{pinyinscope}
蘇竟字伯況,扶風平陵人也。平帝世,竟以明易為博士講書祭酒。善圖緯,能通百家之言。王莽時,劉歆等共典校書,拜代郡中尉。時匈奴擾亂,北邊多罹其禍,竟終完輯一郡。光武即位,就拜代郡太守,使固塞以拒匈奴。建武五年冬,盧芳略得北邊諸郡,帝使偏將軍隨弟屯代郡。竟病篤,以兵屬弟,詣京師謝罪。拜侍中,數月,以病免。

初,延岑護軍鄧仲況擁兵據南陽陰縣為寇,而劉歆兄子龔為其謀主。竟時在南陽,與龔書曉之曰:

君執事無恙。走昔以摩研編削之才,與國師公從事出入,校定祕書,竊自依依,末由自遠。蓋聞君子愍同類而傷不遇。人無愚智,莫不先避害然後求利,先定志然後求名。昔智果見智伯窮兵必亡,故變名遠逝,陳平知項王為天所棄,故歸心高祖,皆智之至也。聞君前權時屈節,北面延牙,乃後覺悟,棲遲養德。先世數子,又何以加。君處陰中,土多賢士,若以須臾之閒,研考異同,揆之圖書,測之人事,則得失利害,可陳於目,何自負畔亂之困,不移守惡之名乎?與君子之道,何其反也?

世之俗儒末學,醒醉不分,而稽論當世,疑誤視聽。或謂天下迭興,未知誰是,稱兵據土,可圖非冀。或曰聖王未啟,宜觀時變,倚彊附大,顧望自守。二者之論,豈其然乎?夫孔丘祕經,為漢赤制,玄包幽室,文隱事明。且火德承堯,雖昧必亮,承積世之祚,握無窮之符,王氏雖乘閒偷篡,而終嬰大戮,支分體解,宗氏屠滅,非其效歟?皇天所以眷顧踟躕,憂漢子孫者也。論者若不本之於天,參之於聖,猥以師曠雜事輕自眩惑,說士作書,亂夫大道,焉可信哉?

諸儒或曰:今五星失晷,天時謬錯,辰星久而不效,太白出入過度,熒惑進退見態,鎮星繞帶天街,歲星不舍氐、房。以為諸如此占,歸之國家。蓋災不徒設,皆應之分野,各有所主。夫房、心即宋之分,東海是也。尾為燕分,漁陽是也。東海董憲迷惑未降,漁陽彭寵逆亂擁兵,王赫斯怒,命將並征,故熒惑應此,憲、寵受殃。太白、辰星自亡新之末,失行筭度,以至于今,或守東井,或沒羽林,或裴回藩屏,或躑躅帝宮,或經天反明,或潛臧久沈,或衰微闇昧,或煌煌北南,或盈縮成鉤,或偃蹇不禁,皆大運蕩除之祥,聖帝應符之兆也。賊臣亂子,往往錯互,指麾妄說,傳相壞誤。由此論之,天文安得遵度哉!

乃者,五月甲申,天有白虹,自子加午,廣可十丈,長可萬丈,正臨倚彌。倚彌即黎丘,秦豐之都也。是時月入于畢。畢為天網,主網羅無道之君,故武王將伐紂,上祭于畢,求助天也。夫仲夏甲申為八魁。八魁,上帝開塞之將也,主退惡攘逆。流星狀似蚩尤旗,或曰營頭,或曰天槍,出奎而西北行,至延牙營上,散為數百而滅。奎為毒螫,主庫兵。此二變,郡中及延牙士眾所共見也。是故延牙遂之武當,託言發兵,實避其殃。今年比卦部歲,坤主立冬,坎主冬至,水性滅火,南方之兵受歲禍也。德在中宮,刑在木,木勝土,刑制德,今年兵事畢已,中國安寧之效也。五七之家三十五姓,彭、秦、延氏不得豫焉。如何怪惑,依而恃之?葛纍之詩,「求福不回」,其若是乎!

圖讖之占,眾變之驗,皆君所明。善惡之分,去就之決,不可不察。無忽鄙言!

夫周公之善康叔,以不從管蔡之亂也;景帝之悅濟北,以不從吳濞之畔也。自更始以來,孤恩背逆,歸義向善,臧否粲然,可不察歟!良醫不能救無命,彊梁不能與天爭,故天之所壞,人不得支。宜密與太守劉君共謀降議。仲尼棲棲,墨子遑遑,憂人之甚也。屠羊救楚,非要爵祿;茅焦干秦;豈求報利?盡忠博愛之誠,憤滿不能已耳。又與仲況書諫之,文多不載,於是仲況與龔遂降。

龔字孟公,長安人,善論議,扶風馬援、班彪並器重之。竟終不伐其功,潛樂道術,作記誨篇及文章傳於世。年七十,卒于家。

楊厚字仲桓,廣漢新都人也。祖父春卿,善圖讖學,為公孫述將。漢兵平蜀,春卿自殺,臨命戒子統曰:「吾綈胝中有先祖所傳祕記,為漢家用,爾其修之。」統感父遺言,服闋,辭家從犍為周循學習先法,又就同郡鄭伯山受河洛書及天文推步之術。建初中為彭城令,一州大旱,統推陰陽消伏,縣界蒙澤。太守宗湛使統為郡求雨,亦即降澍。自是朝廷災異,多以訪之。統作家法章句及內讖二卷解說,位至光祿大夫,為國三老。年九十卒。

統生厚。厚母初與前妻子博不相安,厚年九歲,思令和親,乃託疾不言不食。母知其旨,懼然改意,恩養加篤。博後至光祿大夫。

厚少學統業,精力思述。初,安帝永初二年,太白入北斗,洛陽大水。時統為侍中,厚隨在京師。朝廷以問統,統對年老耳目不明,子厚曉讀圖書,粗識其意。鄧太后使中常侍承制問之,厚對以為「諸王子多在京師,容有非常,宜亟發遣各還本國」。太后從之,星尋滅不見。又剋水退期日,皆如所言。除為中郎。太后特引見,問以圖讖,厚對不合,免歸。復習業犍為,不應州郡、三公之命,方正、有道、公車特徵皆不就。

永建二年,順帝特徵,詔告郡縣督促發遣。厚不得已,行到長安,以病自上,因陳漢三百五十年之厄,宜蠲法改憲之道,及消伏災異,凡五事。制書褒述,有詔太醫致藥,太官賜羊酒。及至,拜議郎,三遷為侍中,特蒙引見,訪以時政。四年,厚上言「今夏必盛寒,當有疾疫蝗蟲之害」。是歲,果六州大蝗,疫氣流行。後又連上「西北二方有兵氣,宜備邊寇」。車駕臨當西巡,感厚言而止。至陽嘉三年,西羌寇隴右,明年,烏桓圍度遼將軍耿曄。永和元年,復上「京師應有水患,又當火災,三公有免者,蠻夷當反畔。」是夏,洛陽暴水,殺千餘人;至冬,承福殿災,太尉龐參免;荊、交二州蠻夷賊殺長吏,寇城郭。又言「陰臣、近戚、妃黨當受禍」。明年,宋阿母與宦者褒信侯李元等遘姦廢退;後二年,中常侍張逵等復坐誣罔大將軍梁商專恣,悉伏誅。每有災異,厚輒上消救之法,而閹宦專政,言不得信。

時大將軍梁冀威權傾朝,遣弟侍中不疑以車馬、珍玩致遺於厚,欲與相見。厚不荅,固稱病求退。帝許之,賜車馬錢帛歸家。修黃老,教授門生,上名錄者三千餘人。太尉李固數薦言之。太初元年,梁太后詔備古禮以聘厚,遂辭疾不就。建和三年,太后復詔徵之,經四年不至。年八十二,卒於家。策書弔祭。鄉人謚曰文父。門人為立廟,郡文學掾史春秋饗射常祠之。


\end{pinyinscope}