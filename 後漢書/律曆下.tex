\article{律曆下}

\begin{pinyinscope}
曆法

昔者聖人之作曆也,觀琁璣之運,三光之行,道之發斂,景之長短,斗綱之建,青龍所躔,參伍以變,錯綜其數,而制術焉。

天之動也,一晝一夜而運過周,星從天而西,日違天而東。日之所行與運周,在天成度,在曆成日。居以列宿,終于四七,受以甲乙,終于六旬。日月相推,日舒月速,當其同,謂之合朔。舒先速後,近一遠三,謂之弦。相與為衡,分天之中,謂之望。以速及舒,光盡體伏,謂之晦。晦朔合離,斗建移辰,謂之。日月之術,則有冬有夏;冬夏之閒,則有春有秋。是故日行北陸謂之冬,西陸謂之春,南陸謂之夏,東陸謂之秋。日道發南,去極彌遠,其景彌長,遠長乃極,冬乃至焉。日道斂北,去極彌近,其景彌短,近短乃極,夏乃至焉。二至之中,道齊景正,春秋分焉。

日周于天,一寒一暑,四時備成,萬物畢改,攝提遷次,青龍移辰,謂之歲。歲首至也,月首朔也。至朔同日謂之章,同在日首謂之蔀,蔀終六旬謂之紀,歲朔又復謂之元。是故日以實之,月以閏之,時以分之,歲以周之,章以明之,蔀以部之,紀以記之,元以原之。然後雖有變化萬殊,贏朒無方,莫不結系于此而岙正焉。

極建其中,道營于外,琁衡追日,以察斂,光道生焉。孔壺為漏,浮箭為刻,下漏數刻,以考中星,昏明生焉。日有光道,月有九行,九行出入而交生焉。朔會望衡,鄰於所交,虧薄生焉。月有晦朔,星有合見,月有弦望,星有留逆,其歸一也,步術生焉。金、水承陽,先後日下,速則先日,遲而後留,留而後逆,逆與日違,違而後速,速與日競,競又先日,遲速順逆,晨夕生焉。日、月、五緯各有終原,而七元生焉。見伏有日,留行有度,而率數生焉。參差齊之,多少均之,會終生焉。引而伸之,觸而長之,探賾索隱,鉤深致遠,無幽辟潛伏,而不以其精者然。故陰陽有分,寒暑有節,天地貞觀,日月貞明。

若夫祐術開業,淳燿天光,重黎其上也。承聖帝之命若昊天,典曆象三辰,以授民事,立閏定時,以成歲功,羲和其隆也。取象金火,革命創制,治曆明時,應天順民,湯、武其盛也。及王德之衰也,無道之君亂之於上,頑愚之史失之於下。夏后之時,羲和淫湎,廢時亂日,胤乃征之。紂作淫虐,喪其甲子,武王誅之。夫能貞而明之者,其興也勃焉;回而敗之者,其亡也忽焉。巍巍乎若道天地之綱紀,帝王之壯事,是以聖人寶焉,君子勤之。

夫曆有聖人之德六焉:以本氣者尚其體,以綜數者尚其文,以考類者尚其象,以作事者尚其時,以占往者尚其源,以知來者尚其流。大業載之,吉凶生焉,是以君子將有興焉,咨焉而以從事,受命而莫之違也。若夫用天因地,揆時施教,頒諸明堂,以為民極者,莫大乎月令。帝王之大司備矣,天下之能事畢矣。過此而往,群忌苟禁,君子未之或知也。

斗之二十一度,去極至遠也,日在焉而冬至,群物於是乎生。故律首黃鍾,曆始冬至,月先建子,時平夜半。當漢高皇帝受命四十有五歲,陽在上章,陰在執徐,冬十有一月甲子夜半朔旦冬至,日月閏積之數皆自此始,立元正朔,謂之漢曆。又上兩元,而月食五星之元,並發端焉。

曆數之生也,乃立儀、表,以校日景。景長則日遠,天度之端也。日發其端,周而為歲,然其景不復,四周千四百六十一日,而景復初,是則日行之終。以周除日,得三百六十五四分度之一,為歲之日數。日日行一度,亦為天度。察日月俱發度端,日行十九周,月行二百五十四周,復會于端,是則月行之終也。以日周除月周,得一歲周天之數。以日一周減之,餘十二十九分之七,則月行過周及日行之數也,為一歲之月。以除一歲日,為一月之數。月之餘分積滿其法,得一月,月成則其歲。月大四時推移,故置十二中以定月位。有朔而無中者為閏月。中之始日節,與中為二十四氣。以除一歲日,為一氣之日數也。其分積而成日為沒,并歲氣之分,如法為一歲沒。沒分于終中,中終于冬至,冬至之分積如其法得一日,四歲而終。月分成閏,閏七而盡,其歲十九,名之曰章。章首分盡,四之俱終,名之曰蔀。以一歲日乘之,為蔀之日數也。以甲子命之,二十而復其初,是以二十蔀為紀。紀歲青龍未終,三終歲後復青龍為元。

元法,四千五百六十。

紀法,千五百二十。

紀月,萬八千八百。

蔀法,七十六。

蔀月,九百四十。

章法,十九。

章月,二百三十五。

周天,千四百六十一。

日法,四。

蔀日,二萬七千七百五十九。

沒數,二十一。為章閏

通法,四百八十七。

沒法,七,因為章閏。

日餘,百六十八。

中法,四十二。

大周,三十四萬三千三百三十五。

月周千一十六。

月食數之生也,乃記月食之既者。率二十三食而復既,其月食百三十五,率之相除,得五百二十三之二十而一食。以除一歲之月,得歲有再食五百一十三分之五十〈五〉也。分終其法,因以與蔀相約,得四與二十七,互之,會二千五十二,二十而與元會。

元會,四萬一千四十。

蔀會,三千五十三。

歲數,五百一十三。

食數,千八十一。

月數,百二十五。

食法,二十二。

推入蔀術曰:以元法除去上元,其餘以紀法除之,所得數從天紀,筭外則所入紀也。不滿紀法者,入紀年數也。以蔀法除之,所得數從甲子蔀起,筭外,所入紀歲名命之,筭上,即所求年太歲所在。

推月食所入蔀會年,以元會除去上元,其餘以蔀會除之,所得以七十二乘之,滿六十除去之,餘以二十除所得數,從天紀,筭之起外,所以入紀,不滿二十者,數從甲子蔀起,筭外,所入蔀會也。其初不滿蔀會者,入蔀會年數也,各以

不入紀歲名命之,筭上,即所求年蔀。

[TABLE]

推天正術,置入蔀年減一,以章月乘之,滿章法得一,名為積月,不滿為閏餘,十二以上,其歲有閏。

推天正朔日,置入蔀積月,以蔀日乘之,滿蔀月得一,名為積日,不滿為小餘,積日以六十除去之,其餘為大餘,以所入蔀名命之,筭盡之外,則前年天正十一月朔日也。小餘四百四十一以上,其月大。求後月朔,加大餘二十九,小餘四百九十〈九〉,小餘滿蔀月得一,上加大餘,命之如前。

一術,以大周乘年,周天乘減之,餘滿蔀日,則天正朔日也。

推二十四氣術曰:置入蔀年減一,以月餘乘之,滿中法得一,名曰大餘,不滿為小餘,大餘滿六十除去之,其餘以蔀名命之,筭盡之外,則前年冬至之日也。

求次氣,加大餘十五,小餘七,除命之如前,小寒日也。

推閏月所在,以閏餘減章法,餘以十二乘之,滿章閏數得一,滿四以上亦得一筭之數,從前年十一月起,筭盡之外,閏月也。或進退,以中氣定之。

推弦、望日,因其月朔大小餘之數,皆加大餘七,小餘三百五十九四分三,小餘滿蔀月得一,加大餘,大餘命如法,得上弦。又加得望,次下弦,又後月朔。其弦、望小餘二百六十以下,每以百刻乘之,滿蔀月得一刻,不滿其數近節氣夜漏之半者,以筭上為日。

推沒滅術,置入蔀年減一,以沒數乘之,滿日法得一,名為積沒,不盡為沒餘。以通法乘積沒,滿沒法得一,名為大餘,不盡為小餘。大餘滿六十除去之,其餘以蔀名命之,筭盡之外,前年冬至前沒日也。求後沒,加大餘六十九,小餘四,小餘滿沒法,從大餘,命之如前,無分為滅。

一術,以為五乘冬至小餘,以減通法,餘滿沒法得一,則天正後沒也。

推合朔所在度,置入蔀積月以日乘之,滿大周除去之,其餘滿蔀月得一,名為積度,不盡為餘分。積度加斗二十一度,加二百三十五分,以宿次除之,不滿宿,則日月合朔所在星度也。求後合朔,加度二十九,加分四百九十九,分滿蔀月得一度,經斗除二百三十五分。

一術,以閏餘乘周天,以減大周餘,滿蔀月得一,合以斗二十一度四分一,則天正合朔日月所在度。

推日所在度,置入蔀積日之數,以蔀法乘之,滿蔀日除去之,其餘滿蔀法得一,為積度,不盡為餘分。積度加斗二十一度,加十九分,以宿次除去之,則夜半日所在宿度也。

求次日,加一度。求次月,大加三十度,小加二十九度,經斗除十〈九〉分。

一術,以朔小餘減合度分,即日夜半所在。其分三百二十五約之,十九乘之。

推月所在度,置入蔀積日之數,以月周乘之,滿蔀日除去之,其餘滿蔀法得一,為積度,不盡為餘分。積度加斗二十一十〈九〉分,除如上法,則所求之日夜半月所在宿度也。

求次日,加十三度二十八分。求次月,大加三十五度六十一分,月小二十二度三十三分,分滿法得一度,經斗除十九分。其冬下旬月在張、心署之,謂盡漏分後盡漏盡也。

一術,以蔀法除朔小餘,所得以減日半度也。餘以減分,即月夜半所在度也。

推日明所入度分術曰:置其月節氣夜漏之數,以蔀法乘之,二百除之,得一分,即夜半到明所行分也。以增夜半日所在度分,為明所在度分也。

求昏日所入度,以夜半到明日所行分分減蔀法,其餘即夜半到昏所行分也。以加夜半所在度分,為昏日所在度也。

推月明所入度分術曰:置其節氣夜半

之數,以月周乘之,以二百除之,為積分。積分滿蔀法得一,以增夜半度,即明月所在度也。

求昏月所入度:以明積分減月周,其餘滿蔀法得一度,加夜半,則昏月所在度也。

推弦、望日所入星度術曰:置合朔度分之數,加七度三百五十九分四分之三,宿次除之,即得上弦日所入宿度分也。

求望、下弦,加除如前法,小分四從大分,滿蔀月從度。

推弦、望月所入星度術曰:置月合朔度分之數,加度九十八,加分六百五十三半,以宿次除之,即上弦月所入宿度分也。

求望、下弦,加除如前分,滿蔀月從度。

推月食術曰:置入蔀會年數,減一,以食數乘之,滿歲數得一,名曰積食,不滿為食餘。以月數乘積,滿食法得一,名為積月,不滿為月餘分。積月以章月除去之,其餘為入章月數。當先除入章閏,乃以十二除去之,不滿者命以十一月,筭盡之外,則前年十一月前食月也。求入章閏者,置入章月,以章閏乘之,滿章月得一,則入章閏數也。餘分滿二百二十四以上至二百三十一,為食在閏月。閏或進退,以朔日定之。求後食,加五百二十分,滿法得一月數,命之如法,其分盡食筭上。

推月食朔日術曰:置食積月之數,以二十九乘之,為積日。又以四百九十〈九〉乘積月,滿蔀月得一,以并積日,以六十除之,其餘以所會蔀名命之,筭盡之外,則前年天正前食月朔日也。

求食日,加大餘十四,小餘七百一十九半,小餘滿蔀月為大餘,大餘命如前,則食日也。

求後食朔及日,皆加大餘二十七,小餘六百一十五。其月餘分不滿二十者,又加大餘二十九,小餘四百九十九。其食小餘者,當以漏刻課之,夜漏未盡,以筭上為日。

一術,以歲數去上元,餘以為積月,以百一十二乘之,滿月數去之,餘滿食法得一,則天正後食。

推諸加時,以十二乘小餘,先減如法之半,得一時,其餘乃以法除之,所得筭之數從夜半子起,筭盡之外,則所加時也。

推諸上水漏刻:以百乘其小餘,滿其法得一刻;不滿法法什之,滿法得一分。積刻先減所入節氣夜漏之半,其餘為晝上水之數。過晝漏去之,餘為夜上水數。其刻不滿夜漏半者,乃減之,餘為昨夜未晝,其弦望其日。

五星數之生也,各記於日,與周天度相約而為率。以章法乘周率為用法,章月乘日率,如月法,為積月月餘。以月之月乘積,為朔大小餘。乘為入月日餘。以日法乘周率為日度法,以率去日率,餘以乘周天,如日度法,為度之餘也。日率相約取之,得二千九百九十萬一千六百二十一億五十八萬二千三百,而五星終,如蔀之數,與元通。

木,周率,四千三百二十七。日率,四千七百二十五。合積月,十三。月餘,四萬一千六百六。月法,八萬二千二百一十三。大餘,二十三。小餘,八百四十七。虛分,九十三。入月日,十五。日餘,萬四千六百四十七。日度法,萬七千三百八。積度,三十三。度餘,萬三百一十四。

火,周率,八百七十九。日率,千八百七十六。合積月,二十六。月餘,六千六百三十四。月法,萬六千七百一。大餘,四十七。小餘,七百五十四。虛分,一百八十六。入月日,十一。日餘,千八百七十二。日度法,三千五百一十六。積度,四十九。度餘,一百一十四。

土,周率,九千九十六。日率,九千四百一十五。合積月,十二。月餘,十三萬八千六百三十七。月法,十七萬二千八百二十四。大餘,五十四。小餘,三百四十八。虛分,五百九十二。入月日,二十三。日餘,二千一百六十三。日度法,三萬六千三百八十四。積度,十二。度餘,二萬九千四百五十一。

金,周率,五千八百三十。日率,四千六百六十一。合積月,九。月餘,九萬八千四百五。月法,十〈一〉萬七百七十。大餘,二十五。小餘,七百三十一。虛分,二百九。入月日,二十六。日餘,二百八十一。日度法,二萬三千三百二十。積度,二百九十二。度餘,二百八十一。

水,周率,萬一千九百八。日率,千八百八十九。合積月,一。月餘,二十一萬七千六百六十〈三〉。月法,二十二萬六千二百五十二。大餘,二十九。小餘,四百九十九。虛分,四百四十九〈一〉。入月日,二十七。日餘,四萬四千八百五。日度法,四萬七千六百三十一。積度,五十七。度餘四萬四千八百五。

推五星術,置上元以來,盡所求年,以周率乘之,滿日率得一,名為積合;不盡名合餘。餘以周率除之,不得焉退歲;無所得,星合其年,得一合前年,二合前二年。金、水積合奇為晨,偶為夕。其不滿周率者反減之,餘為度分。

推星合月,以合積月乘積合為小積,又以月餘乘積合,滿其月法得一,從小積為月餘。積月滿紀月去之,餘為入紀月。每以章閏乘之,滿章月得一為閏;不盡為閏餘。以閏減入紀月,其餘以十二去之,餘為入歲月數,從天正十一月起,筭外,星合所在之月也。其閏滿二百二十四以上至二百三十一星合閏月。閏或進退,以朔制之。

推朔日,以蔀日乘之入紀月,滿蔀月得一為積日,不盡為小餘。積日滿六十去之,餘為大餘,命以甲子,筭外,星合月朔日。

推入月日,以蔀日乘月餘,以其月法乘朔小餘,從之,以四千四百六十五約之,所得得滿日度法得一,為入月日,不盡為日餘。以朔命入月日,筭外,星合日也。

推合度,以周天乘度分,滿日度法得一為積度,不盡為度餘。以斗二十一四分之一命度,筭外,星合所在度也。

一術,加退歲一,以減上元,滿八十除去之,餘以沒數乘之,滿日法得一,為大餘,不盡為小餘。以甲子命大餘,則星合歲天正冬至日也。以周率小餘,并度餘,餘滿日度法從度,即正後星合日數也,命以冬至。求後合月,加合積月於入歲月,加月餘於月餘,滿其月法得一,從入歲月。入歲月滿十二去之,有閏計焉,餘命如前,筭外,後合月也。餘一加晨得夕,加夕得晨。

求朔日,以大小餘加今所得,其月餘得一月者,又餘二十九,小飾滿蔀月得一,如大餘,大餘命如前。

求入月日,以入月日餘加今所得,餘滿日度法得一,從日。其前合月朔小餘不滿其虛分者,空加一日。日滿月先去二十九,其後合月朔小餘不滿四百九十九,又減一日,其餘命如前。

求合度,以積度度餘加今所得,餘滿日度法得一從度,命如前,經斗除如周率矣。

木,晨伏,十六日七千二百二十分半,行二度萬三千八百一十一分,在日後十三度有奇,而見東方。見順,日行五十八分度之十一,五十八日行十一度,微遲,日行九分,五十八日行九度。留不行,二十五日。旋逆,日行七分度之一,八十四日進十二度。復留,二十五日。復順,五十八日行九度,又五十八日行十一度,在日前十三度有奇,而夕伏西方。除伏逆,一見三百六十六日,行二十八度。伏復十六日七千二百二十分半,行二度萬三千八百一十一分,而與日合。凡一終,三百九十八日有萬四千六百四十一分,行星三十二度與萬三百一十四分,通率日行四千七百二十五分之三百九十八。

火,晨伏,七十一日二千六百九十四分,行五十五度二千二百五十四分半,在日後十六度有奇,而見東方。見順,日行二十三分度之十四,八十四日行一十二度。微遲,日行十二分,九十二日行四十八度。留不行,十一日。旋逆,日行六十二分度之十七,六十二日退十七度。復留,十一日。復順,九十二日,行四十八度,又百八十四日行百一十二度,在日前十六度有奇,而夕伏西方。除伏逆,一見六百三十六日,行〈三〉百三度。伏復,七十一日二千六百九十四分,行五十五度二千二百五十四分半,而與日合。凡一終,七百七十九日有千八百七十二分,行星四百一十四度與九百九十三分。通率日行千八百七十六分之九百九十七。

土,晨伏,十九日千八十一分半,行三度萬四千七百二十五分半,在日後十五度有奇,而見東方。見順,日行四十三分度之三,八十六日行六度。留不行,三十三日。旋逆,日行十七分度之一百二,日退六度。復留,三十三日。復順,八十六日,行六度,在日前十五度有奇,而夕伏西方。除伏逆,〈一〉見三百四十日,行六度。伏復,十九日千八十一分半,行三度萬四千七百二十五分半,與日合。凡一終,三百七十八日有二千一百六十三分,行星十二度與二萬九千四百五十一分。通率日行九千四百一十五分之三百一十九。

金,晨伏,五日,退四度,在日後九度,而見東方。見逆,日行五分度之三,十日,退六度。留不行,八日。順,日行行四十六分度之三十三,四十六日行三十三度。而,日行一度九十〈一〉分度之十五,九十一日行百六度。益疾,日行一度二十二分,九十一日行百一十三度,在日後九度,而晨伏東方。除伏逆,一見二百四十六日,行二百四十六度。伏四十一日二百八十一分,行五十度二百八十一分,而與日合。一合二百九十二日〈二〉百八十一分,行星如之。

金,夕伏,四十一日二百八十一分,行五十度二百八十一分,在日前九度,而見西方。見順,疾,日行一度九十一分度之二十二,九十一日行百一十三度。微遲,日行一度十五分,九十一日行百六度。而進,日行四十六分度之三十三,四十六日行三十三度。留不行,八日。旋逆,日行五分度之三,十日退六度,在日前九度,而夕伏西方。除伏逆,一見二百四十六日,行二百四十六度,伏五日,退四度而後合。凡三合一終,五百八十四日有五百六十二分,行星如之。通率日行一度。

水,晨伏,九日,退七度,在日後十六度,而見東方。見逆,一日退一度。留不得,二日。旋順,日行九分度之八,九日行八度。而疾,日行一度四分度之一,二十日行二十五度,在日後十六度,而晨伏東方。除伏逆,一見,三十二日,行三十二度,伏十六日四萬四千八百五分,行三十二度四萬四千八百五分,而與日合。一合五十七日有四萬四千八百五分,行星如之。

水,夕伏,十六日四萬四千八百五分,行三十二度四萬四千八百五分,在日前十六度,而見西方。見順,疾,日行一度四分度之一,二十日行二十五度。而遲,日行九分度之八,九日行八度。留不行,二日。逆,一日退一度,在日前十六度,而夕伏西方。除伏逆,一見三十二日,行三十〈二〉度,伏九日,退七度而復合。凡再合一終,百一十五日有四萬一千九百七十八分,行星如之。通率日行一度。

步術,以步法伏日度分,如星合日度餘,命之如前,得星見日度也。術分母乘之,分日如度法而一,分不盡如法半以上,亦得一,而日加所行分,滿其母得一度。逆順母不同,以當行之母乘故分,如故母,如一也。留者承前,逆則減之,伏不書度。經斗除如行母,四分具一。其分有損益,前後相放。其以赤道命度,進加退減之。其步以黃道。

[TABLE]

[TABLE]

黃道去極,日景之生,據儀、表也。漏刻之生,以去極遠近差乘節氣之差。如遠近而差一刻,以相增損。昏明之生,以天度乘晝漏,夜漏減三〈之,二〉百而一,為定度。以減天度,餘為明;加定度一為昏。其餘四之,如法為少。〈二為半,三為太,〉不盡,三之,如法為強,餘半法以上以成強。強三為少,少四為度,其強二為少弱也。又以日度餘為少強,而各加焉。

[TABLE]

中星以日所在為正,日行四歲乃終,置所求年二十四氣小餘四之,如法為少、大,餘不盡,三之,如法為強、弱,以減節氣昏明中星,而各定矣。強,正;弱,直也。其強弱相減,同名相去,異名從之。從強進少為弱,從弱退少而強。從上元太歲在庚辰以來,盡熹平三年,歲在甲寅,積九千四百五十五歲也。

論曰:易有太極,是生兩儀。兩儀之分尚矣,乃有皇犧。皇犧之有天下也,未有書計。歷載彌久,暨於黃帝,班示文章,重黎記註,象應著名,始終相驗,準度追元,乃立曆數。天難諶斯,是以五、三迄于來今,各有改作,不通用。故黃帝造曆,元起辛卯,而顓頊用乙卯,虞用戊午,夏用丙寅,殷用甲寅,周用丁巳,魯用庚子。漢興承秦,初用乙卯,至武帝元封,不與天合,乃會術士作太初曆,元以丁丑。王莽之際,劉歆作三統,追太初前世一元,得五星會庚戌之歲,以為上元。太初曆到章帝元和,旋復疏闊,徵能術者課校諸曆,定朔稽元,追漢三十五年庚辰之歲,追朔一日,乃與天合,以為四分曆元。加六百五元一紀,上得庚申。有近於緯,而歲不攝提,以辨曆者得開其說,而其元垬與緯同,同則或不得於天。然曆之興廢,以疏密課,固不主於元。光和元年中,議郎蔡邕、郎中劉洪補續律曆志,邕能著文,清濁鍾律,洪能為筭,述敘三光。今考論其業,義指博通,術數略舉,是以集錄為上下篇,放續前志,以備一家。

贊曰:象因物生,數本杪曶。律均前起,準調後發。該覈衡琁,檢會日月。


\end{pinyinscope}