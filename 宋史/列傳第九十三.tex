\article{列傳第九十三}

\begin{pinyinscope}

 徐禧李稷附高永能沉起劉彞熊本蕭注陶弼林廣



 徐禧,字德占,洪州分寧人。少有志度,博覽周游,以求知古今事變、風俗利疚,不事科舉。熙寧初,王安石行新法,
 禧作《治策》二十四篇以獻。時呂惠卿領修撰經義局,遂以布衣充檢討。神宗見其所上策,曰:「禧言朝廷用經術變士,十已八九,然竊襲人之語,不求心通者相半,此言是也。宜試於有用之地。」即授鎮安軍節度推官、中書戶房習學公事。歲餘召對,顧問久之,曰:「朕多閱人,未見有如卿者。」擢太子中允、館閣校勘、監察御史裏行。



 與中丞鄧綰、知諫院範百祿雜治趙世居獄。李士寧者,挾術出入貴人間,嘗見世居母康,以仁宗御制詩贈之。又許世
 居以寶刀,且曰:「非公不可當此。」世居與其黨皆神之,曰:「士寧,二三百歲人也。」解釋其詩,以為至寶之祥。及鞫世居得之,逮捕士寧,而宰相王安石故與士寧善,百祿劾士寧以妖妄惑世居,致不軌。禧奏:「士寧遺康詩實仁宗制,今獄官以為反,臣不敢同。」百祿言:「士寧有可死之狀,禧故出之以媚大臣。」朝廷以御史雜知、樞密承旨參治,而百祿坐報上不實貶,進禧集賢校理、檢正禮房。



 安石與惠卿交惡,鄧綰言惠卿昔居父喪,嘗貸華亭富人
 錢五百萬買田事,詔禧參鞫。禧陰右惠卿,綰劾之,會綰貶官,獄亦解。禧出為荊湖北路轉運副使。元豐初,召知諫院。惠卿在鄜延,欲更蕃漢兵戰守條約,諸老將不謂然,帝頗採聽,將推其法於他路,遣禧往經畫。禧是惠卿議,渭帥蔡延慶亦以為不然,帝召延慶還,加禧直龍圖閣,使往代,以母憂不行。服除,召試知制誥兼御史中丞。官制行,罷知制誥,專為中丞。鄧綰守長安,禧疏其過,帝知其以惠卿故,雖改綰青州,亦左遷禧給事中。



 種諤西討,
 得銀、夏、宥三州而不能守。延帥沉括欲盡城橫山,瞰平夏,城永樂,詔禧與內侍李舜舉往相其事,令括總兵以從,李稷主饋餉。禧言:「銀州雖據明堂川、無定河之會,而故城東南已為河水所吞,其西北又阻天塹,實不如永樂之形勢險厄。竊惟銀、夏、宥三州,陷沒百年,一日興復,於邊將事功,實為俊偉,軍鋒士氣,固已百倍;但建州之始,煩費不貲。若選擇要會,建置堡柵,名雖非州,實有其地,舊來疆塞,乃在腹心。已與沉括議築砦堡各六。砦之大
 者周九百步,小者五百步,堡之大者二百步,小者百步,用工二十三萬。」遂城永樂,十四日而成。禧、括、舜舉還米脂。明日,夏兵數千騎趨新城,禧亟往視之。或說禧曰:「初被詔相城,御寇,非職也。」禧不聽,與舜舉、稷俱行,括獨守米脂。先是,種諤還自京師,極言城永樂非計,禧怒變色,謂諤曰:「君獨不畏死乎?敢誤成事。」諤曰:「城之必敗,敗則死,拒節制亦死;死於此,猶愈於喪國師而淪異域也。」禧度不可屈,奏諤跋扈異議,詔諤守延州。夏兵二十萬屯
 涇原北,聞城永樂,即來爭邊。人馳告者十數,禧等皆不之信,曰:「彼若大來,是吾立功取富貴之秋也。」禧亟赴之,大將高永享曰:「城小人寡,又無水,不可守。」禧以為沮眾,欲斬之,既而械送延獄。比至,夏兵傾國而至,永享兄永能請及其未陳擊之。禧曰:「爾何知,王師不鼓不成列。」禧執刀自率士卒拒戰。夏人益眾,分陣迭攻抵城下。曲珍兵陳於水際,官軍不利,將士皆有懼色。珍白禧曰:「今眾心已搖,不可戰,戰必敗,請收兵入城。」禧曰:「君為大將,奈
 何遇敵不戰,先自退邪?」俄夏騎卒度水犯陳。鄜延選鋒軍最為驍銳,皆一當百,銀槍錦襖,光彩耀日,先接戰而敗,奔入城,蹂後陳。夏人乘之,師大潰,死及棄甲南奔者幾半。珍與殘兵入城,崖峻徑窄,騎卒緣崖而上,喪馬八千匹,遂受圍。水砦為夏人所據,掘井不及泉,士卒渴死者太半。夏人蟻附登城,尚扶創拒鬥。珍度不可敵,又白禧,請突圍而南;永能亦勸李稷盡捐金帛,募死士力戰以出,皆不聽。戊戌夜大雨,城陷,四將走免,禧、舜舉、稷死
 之,永能沒於陳。



 初,括奏夏兵來逼城,見官兵整,故還。帝曰:「括料敵疏矣,彼來未出戰,豈肯遽退邪、必有大兵在後。」已而果然。帝聞禧等死,涕泣悲憤,為之不食。贈禧金紫光祿大夫、吏部尚書,謚曰忠愍。官其家二十人。稷工部侍郎,官其家十二人。



 禧疏曠有膽略,好談兵,每云西北可唾手取,恨將帥怯爾。呂惠卿力引之,故不次用。自靈武之敗,秦、晉困棘,天下企望息兵,而沉括、種諤陳進取之策。禧素以邊事自任,狂謀輕敵,猝與強虜遇,至於
 覆沒。自是之後,帝始知邊臣不可信倚,深自悔咎,遂不復用兵,無意於西伐矣。子俯自有傳。



 李稷,字長卿,邛州人。父絢,龍圖閣直學士。稷用蔭歷管庫,權河北西路轉運判官,修拓深、趙、邢三州城,役無愆素,然峭刻嚴忍。察訪使者以為言,都水丞程昉亦訴其越職。詔令件析。御史周尹又論稷父死二十年不葬,僅徙東路,俄提舉蜀部茶場。甫兩歲,羨課七十六萬緡,擢鹽鐵判官。詔推揚其功以勸在位,遂為陜西轉運使、制
 置解鹽。秦民作舍道傍者,創使納「侵街錢」,一路擾怨,與李察皆以苛暴著稱。時人語曰:「寧逢黑殺,莫逢稷、察。」



 種諤起興、靈議,稷聞之亦上言:「可令邊面諸將各出兵撓之,使不得耕種,則其國必困,國困眾離,取可決也。」及出境,稷督餉,民苦折運,多散逸,稷令騎士執之,斷其足筋,宛轉山谷間,凡數十人,累日乃得死。始,稷受旨得斬郡守以下,於是上下相臨以峻法,雖小吏護丁夫,亦顓戮不請。軍食竟不繼。諤謀斬稷,客呂大鈞引義責之,復使
 還取糧。既集,諤猶宣言稷乏軍興,致大功不就,坐削兩秩,貶為判官。



 永樂既城,稷悉輦金、銀、鈔、帛充牣其中,欲誇示徐禧,以為城甫就而中已實。積金既多,故受圍愈急,而稷守之不敢去,以及於難。李舜舉別有傳。



 高永能,字君舉,世為綏州人。初,伯祖文□不舉州來歸,即拜團練使,已而棄之北遷,其祖文玉獨留居延州,至永能始家青澗。少有勇力,善騎射,由行伍補殿侍,稍遷供奉官。種諤取綏州,發永能兵六千先驅入囉兀,五戰皆
 捷,轉供備庫副使。治綏德城,闢地四千頃,增戶千三百,即知城事。



 元豐初,為鄜延都監。秋,大稔,夏人屯二千騎於大會平,將取稼。永能簡精騎突過其營,騎卒驚潰,獲鈐轄二人。轉六宅使。夏人患之,令曰:「有得高六宅者,賞金等其身。」經略使呂惠卿行邊,永能伏騎穀中,以備侵軼。邊騎果至,馳出擊走之。夏兵二萬犯當川堡,永能以千騎與相遇,度不能支,依險設疑兵,且鬥且卻,而令後騎揚塵,若援兵至者,奮而前,遂解去。擢本路鈐轄。



 四年,
 西討,永能為前鋒,圍米脂城。邊人十萬來援,永能謂弟永亨曰:「彼恃眾集易吾軍,營當大川,宜嚴陳待其至,張左右翼擊之,可破也。」詰旦,鏖戰於無定河,斬首數千級,得馬三千、橐駝牛羊萬計。城猶未下,密遣諜說降其東壁守將,衣以文錦,導以鼓吹,耀諸城下,酋令介訛遇乃出降。進東上閣門使、寧州刺史,以年請老,不許,又進四方館使、榮州團練使。



 永樂之役,獻謀皆不用。城既陷,其孫昌裔欲援之從間道出,永能嘆曰:「吾結發從事西羌,
 戰未嘗挫,今年已七十,受國大恩,恨無以報,此吾死所也。」顧易一卒敝衣,戰而死。其子世亮與昌裔求得尸以歸。詔贈房州觀察使,錄世亮為忠州刺史,諸孫皆侍禁殿直。



 永能家世州將,所領多故部曲,拊之有恩惠,遇敵則身先之。下有傷者,載以己副馬,故能得士死力。遠近喜言其事,稱之曰「老高」。及死,邊人無不痛惜。嘗過其遠祖唐綏州刺史思祥淘沙川廟,得畫像及神道碑上之,詔即所在賜田三十頃,以奉祭祀。



 永能之亡,延州
 將皇城使寇偉亦力戰而沒,贈均州防禦使。



 沈起,字興宗,明州鄞人。進士高第,調滁州判官,與監真州轉般倉。聞父病,委官歸侍,以喪免,有司劾其擅去。終喪,薦書應格當遷用,帝謂輔臣曰:「觀過知仁。今由父疾而致罪,何以厚風教而勸天下之為人子者。」乃特遷之,知海門縣。



 縣負海地卑,間歲海潮至,冒民田舍,民徙以避,棄其業。起為築堤百里,引江水灌溉其中,田益闢,民相率以歸,至立祠以報。御史中丞包拯舉為監察御
 史。吏部格,選吏以贓私絓法,無輕重終身不遷。起論其情可矜者,可限年敘用,遂著為令。立縣令考課法,設河渠司領諸道水政,乞採漢故事,擇卿大夫子弟入宿衛,選賢良文學高第給事宮省,勿專任宦官,宗室袒免親令補外官,復府兵,汰冗卒,書數十上。以論興國鐵官事不合,出通判越州,改知蘄、楚二州。



 京東歲饑盜起,除提點刑獄。至,則開首贖法攜其伍,盜內自睽疑,轉相束縛唯恐後。改開封府判官,為湖南轉運使。凡羽毛、筋革、舟楫、
 竹箭之材,多出所部,取於民無制,吏挾為奸。起會其當用,自與商人貿易,所省什六七。召為三司鹽鐵副使,直舍人院。



 熙寧三年,韓絳使陜西,加起集賢殿修撰、陜西都轉運使。慶州軍變,將寇長安,起率兵討平之。會韓絳城綏州不利,起亦罷知江寧府。入知吏部流內銓。奉使契丹,至王庭,其位著乃與夏使等,起曰:「彼陪臣爾,不當與王人齒。」辭不就列,遂升東朝使者,自是為定制。六年,拜天章閣待制、知桂州。



 自王安石用事,始求邊功,王韶
 以熙河進,章惇、熊本亦因此求奮。是時,議者言交址可取,朝廷命蕭注守桂經略之。注蓋造謀者也,至是,復以為難。起言:「南交小醜,無不可取之理。」乃以起代注,遂一意事攻討。妄言密受旨,擅令疆吏入溪洞,點集土丁為保伍,授以陳圖,使歲時肄習。繼命指使因督餫鹽之海濱,集舟師寓教水戰。故時交人與州縣貿易,悉禁止之。於是交址益貳,大集兵丁謀入寇。



 蘇緘知邕州,以書抵起,請止保甲,罷水運,通互市。起不聽,劾緘沮議,起坐邊
 議罷。命劉彞代之以守廣,日遏絕其表疏,於是交人疑懼,率眾犯境,邊陷廉、白、欽、邕四州,死者數十萬人。事聞,貶起團練使,安置郢州,徙越,又徙秀而卒。



 起生平喜談兵,嘗以兵法謁範仲淹,仲淹器其材,注孫武書以自見,卒用此敗。



 劉彞,字執中,福州人。幼介特,居鄉以行義稱。從胡瑗學,瑗稱其善治水,凡所立綱紀規式,彞力居多。第進士,為邵武尉,調高郵簿,移朐山令。治簿書,恤孤寡,作陂池,教
 種藝,平賦役,抑奸猾,凡所以惠民者無不至。邑人紀其事,目曰「治範」。



 熙寧初,為制置三司條例官屬,以言新法非便罷。神宗擇水官,以彞悉東南水利,除都水丞。久雨汴漲,議開長城口,彞請但啟楊橋斗門,水即退。為兩浙轉運判官。知虔州,俗尚巫鬼,不事醫藥。彞著《正俗方》以訓,斥淫巫三千七百家,使以醫易業,俗遂變。加直史館,知桂州。禁與交人互市,交址陷欽、廉、邕三州,坐貶均州團練副使,安置隨州。又除名為民,編隸涪州,徙襄州。元
 祐初,復以都水丞召還,病卒於道,年七十。著《七經中義》百七十卷,《明善集》三十卷,《居陽集》三十卷。



 論曰:兵,兇器也,雖聖人猶曰未學。輕敵寡謀,鮮有不自焚者。永樂之陷,安南之畔,死者百萬,罹禍甚慘,良由數人者不自量度,以開邊釁。禧、稷、永能之死,宜矣。起執議益堅,妄意輕舉,雖貶官莫贖其責。彞不能行所學,而規規然蹈前車之轍,以濟其過,焉得無罪?



 熊本,字伯通,番陽人。兒時知學,郡守範仲淹異其文。進
 士上第,為撫州軍事判官,稍遷秘書丞、知建德縣。縣令頃包魚池為圭田,本弛以與民。



 熙寧初,上書言:「陛下師用賢傑,改修法度,得稷、離、皋、夔之佐。」由是提舉淮南常平、檢正中書禮房事。



 六年,瀘州羅、晏夷叛,詔察訪梓、夔,得以便宜治夷事。本嘗通判戎州,習其俗,謂:「彼能擾邊者,介十二村豪為鄉導爾。」以計致百餘人,梟之瀘川,其徒股慄,願矢死自贖。本請於朝,寵以刺史、巡檢之秩,明示勸賞,皆踴躍順命,獨柯陰一酋不至。本合晏州十九
 姓之眾,發黔南義軍強弩,遣大將王宣、賈昌言率以進討。賊悉力旅拒,敗之黃葛下,追奔深入。柯陰窘,乞降,盡籍丁口、土田及其重寶善馬,歸之公上,受貢職。於是烏蠻羅氏鬼主諸夷皆從風而靡,願世為漢官奴。遷刑部員外郎、集賢殿修撰、同判司農寺。神宗勞之曰:「卿不傷財,不害民,一旦去百年之患,至於檄奏詳明,近時鮮儷焉。」賜三品服。西南用兵蠻中始此。



 蔡京時為秀州推官,本言其學行純茂,練習新法,薦為乾當公事。河、湟初復,
 本為秦鳳路都轉運使。熙河法禁闊略,蓄積不支歲月,本奏省冗官百四十員,歲減浮費數十萬。



 渝州南川獠木鬥叛,詔本安撫。本進營銅佛壩,抗其尤,焚積聚,以破其黨。木鬥氣索,舉溱州地五百里來歸,為四砦九堡,建銅佛壩為南平軍。初,熟獠王仁貴以木鬥親系獄,本釋其縛置麾下,至是推鋒先登。大臣議加本天章閣待制,帝曰:「本之文,朕所自知,當典書命。」遂知制誥。帝數稱其文有體,命院吏別錄以進。



 又上疏云:「天下之治,有因有
 革,期於趣時適治而已。議者猥用持盈守成之說,文茍簡因循之治,天下之吏因以安常習故為俗,奮言納忠者,悠悠之徒相與蹙額盱衡而詆罵之。陛下出大號,發大政,可謂極因革之理。然改制之始,安常習故之群圜視四起,交歡而合噪,或諍於廷,或謗於市,或投劾引去者,不可勝數。陛下燭見至理,獨立不奪,今雖少定,彼將伺隙而逞。願陛下深念之,勿使噪歡之眾有以窺其間,而終萬世難就之業,天下幸甚。」本之意,專以媚王安石
 也。



 範子淵創浚河之役,文彥博爭之,命本行視,議如彥博。安石白出本分司西京。居三年,起知滁州,改廣州,召為工部侍郎。宜州蠻擾邊,道除龍圖閣待制、知桂州。至則諭溪洞酋長,戒邊吏勿生事,請選將練兵代戍,益市馬以足騎兵,宜州遂無事。民蔡寶□全扇龍蕃與峒戶相仇殺,欲引兵致討以為功。本質之,色動,縛而投之海。蠻夷以為神。



 諜告交人明年將入寇,使者實其言,詔訪,本曰:「使者在道,安得此?藉使有謀,何自先知之?」已而果妄。
 是時,既以順州賜李乾德,疆畫未正,交人緣是輒暴勿陽地而逐儂智會。智會來乞師,本檄問狀,乾德斂兵謝本,因請以宿桑八洞不毛之地賜之,南荒遂安。



 轉運判官許彥先議通湖南鹽於西廣,計口授民,度可得息三十萬。本言:「桂管民貧地瘠,恐不堪命。」議遂格。入為吏部侍郎。逾年,力請外,仍請制、知洪州。言者謂本棄八洞為失謀,奪一官,徙杭州、江寧府,再知洪州。召還,卒於道。有文集、奏議共八十卷。



 蕭注,字巖夫,臨江新喻人。磊落有大志,尤喜言兵。常言:「四方有事,吾將兵數萬,鼓行其間,戰必勝,攻必取,豈不快哉!」



 舉進士,攝廣州番禺令。儂智高圍州數月,方舟數百攻城南,勢危甚。注自圍中出,募海濱壯士,得二千人,乘大舶集上流,因颶風起,縱火焚賊舟,破其眾。即日發縣門納援兵,民持牛酒、芻糧相繼入,城中人始有生意。自是每戰以勝歸。蔣偕上其功,擢禮賓副使、廣南駐泊都監。賊還據邕管,餘靖患其嘯誘諸洞,以屬注。注挺身
 入蠻中,施結恩信。狄青師次賓州,召會諸將,疑注倚賊聲勢為奸利,欲誅之。注覺,托為游辭,不肯往。賊破,青始聞注前功,以知邕州。



 智高走大理國,母與二弟寓特磨道。注帥師往討,獲一裨將。引致臥內,與之語,具得賊情,悉擒送闕下。拜西上閣門副使。募死士使入大理取智高,至則已為其國所殺,函首歸獻。轉為使。



 居邕數年,陰以利啖廣源群蠻,密繕兵甲,乃上疏曰:「交址雖奉朝貢,實包禍心,常以蠶食王土為事。往天聖中,鄭天益為轉
 運使,嘗責其擅賦云河洞。今雲河乃落蠻數百里,蓋年侵歲吞,馴致於是。臣已盡得其要領,周知其要害。今不取,異日必為中國憂。願馳至京師,面陳方略。」未報,而甲洞申紹泰犯西平,五將被害。諫官論注不法致寇,罷為荊南鈐轄、提點刑獄。李師中又劾其沮威嗜利,略智高閹民為奴,發洞丁採黃金無帳籍可考。中使按驗頗有實,貶泰州團練副使。淮南轉運使言:「注椎牛屠狗,招集游士,部勒為兵,教之騎射,請徙大州以縻之。」詔改鎮南
 軍節度副使。



 近臣有訟注廣州功者,起為右監門將軍、邠州都監。熙寧初,以禮賓使知寧州。環慶李信之敗,列城皆堅壁,注獨啟關夜宴如平時。復閣門使,管幹麟府軍馬。辭云:「身本書生,差長拊納,不閑戰鬥,懼無以集事。」時有言「交人挫於占城,眾不滿萬,可取也」。遂以注知桂州。



 入覲,神宗問攻取之策,對曰:「昔者臣有是言,是時溪洞之兵,一可當十;器甲堅利,親信之人皆可指呼而使。今兩者不如昔,交人生聚教訓十五年矣,謂之『兵不滿
 萬』,妄也。」既至桂,種酋皆來謁。注延訪山川曲折,老幼安否,均得其歡心,故李乾德動息必知之。然有獻征南策者,輒不聽。會沈起以平蠻自任,帝使代注而罷,注歸,卒於道,年六十一。詔優錄其子,賻絹三百。



 注有膽氣,嗜殺,而能相人。自陜西還,帝問注:「韓絳為安撫使,施設何如?」對曰:「廟算深遠,臣不能窺。然知絳當位極將相。」帝喜曰:「果如卿言,絳必成功。」問王安石,曰:「安石牛目虎顧,視物如射,意行直前,敢當天下大事。然不如絳得和氣為多,
 惟氣和能養萬物爾。」王韶為建昌參軍,注曰:「君他日類孫沔,但壽不及。」後皆如其言。



 陶弼,字商翁,永州人。少俶儻,放宕吳中。行山間,有雙鯉戲溪水上,佇觀之。傍一老父顧曰:「此龍也,行且鬥,君宜亟去。」去百步許,雷大震而雨,岸圮木拔。又出大雲,倉卒遇風暴怒,二十七艘同時溺,獨弼舟得濟,人以是異之。一見丁謂,謂妻以宗女,因從學兵法,能持論縱橫。慶歷中,楊畋討湖南猺,弼上謁,畋授之兵使往襲,大破之。以
 功得陽朔主簿。



 儂智高犯南海,畋為安撫使,闢參軍謀。使下英江會諸將議擊,未至,智高解去。弼舍舟,從其徒數十人,間關步出赴畋。次臨賀,大將蔣偕適戰死,餘眾畏亡將被誅,多降賊。弼數與之遇,亟矯畋命揭榜道上,諭使歸,許以不死,凡得千五百人。府罷,調陽朔令。課民植木官道旁,夾數百里,自是行者無夏秋暑暍之苦,它郡縣悉效之。攝興安令。移書說桂守蕭固浚靈渠以通漕,不聽;至李師中,卒浚之。師征安南,饋餉於是乎出,大
 為民利。



 知賓、容、欽三州,換崇儀副使,遷為使,知邕州。邕經儂寇,井隧蕩然,人不樂其生。弼綏輯惠養,至忘其勤。諸峒獻土物求內附,弼降意撫答,謝其贄,皆感悅無犯邊者。邕地卑下,水易集,夏大雨彌月,弼登城以望,三邊皆漫為陂澤,亟窒垠江三門,諭兵民即高避害。俄而水大至,弼身先版歃,召僚吏賦役,為土囊千餘置道上,水果從竇入,隨塞之。城雖不壞,而人皆乏食,則為發廩以振於內,方舟以饁於外,水不及女墻者三板,旬有五日
 乃退,公私一無所失亡。自橫、潯以東數州皆沒。弼久於邕,請便郡,徙鼎州。章惇經理五溪蠻事,薦為辰州,遷皇城使。降北江彭師宴,授忠州刺史。



 郭逵南征,轉弼康州團練使,復知邕州。民再罹禍亂,散匿山谷,弼率百騎深入左江峒,民知其至,扶老攜幼以歸。逵帥官軍臨富良江,使弼殿。交人納款,逵欲班師,恐為所襲。乃以計夜起,軍不整,騎步相蹈藉亂行。賊隔江陰伺覘,知弼殿,弗敢追。弼申令帳下毋動,遲明,結隊徐行,逵賴以善還。建所
 得廣源峒為順州,桄榔為縣。進弼西上閣門使,留知順州。



 州去邕二千里,多毒草瘴霧,戍卒死者什七八,弼亦疾甚,然蚤莫勞軍,視其良苦,意氣激揚,士莫不感泣,強奮起為用。交人襲取桄榔,揚聲欲圖州,獨難弼。弼素得人心,賊動息皆先知。獲間諜不殺,諭以逆順,縱之去,恩威兩施,以是終弼在不敢犯。加東上閣門使,未拜而卒。詔錄其家五人。



 弼能為詩,好士樂施,所得俸祿,悉以與人,家至貧不恤也。既死,妻在鄉里,僦屋以居。



 林廣,萊州人。以捧日軍卒為行門,授內殿崇班,從環慶蔡挺麾下。李諒祚寇大順城,廣射中之。李信敗於荔原,廣引兵西入,破十二盤,攻白豹、金湯,皆先登。夜過洛河,夏人來襲,廣揚聲選強弩列岸側,實卷甲疾趨,夏人疑不敢渡。嘗護中使臨邊,將及烏雞川,遽率眾循山行。道遇熟羌以險告,廣不答,夏人果伏兵於川,計不行而去。告者乃諜也。



 夏人圍柔遠城,廣止守,戒士卒即有變毋得輕動。火夜起積薪中,眾屯守自若。明日,敵至馬平川,
 大持攻具來。廣被甲啟他門鼓而出,若將奪其馬,敵舍城救馬,廣復入,益修守備,夜募死士斫其營。夏人數失利,始引退。累遷禮賓使。韓絳奏為本道將。



 慶兵據北城叛,廣在南城,望其眾進退不一,曰:「是不舉軍亂也。」挺身縋城出其後,諭以逆順,皆投兵聽命。出者財三百人,廣語餘眾曰:「亂者去矣,汝曹事我久,能聽命,不唯得活,仍有功。」得百餘人。激厲要束,使反攻城下兵,禽戮皆盡,遂平北城。出追亂者,至石門山與之遇,諭之不肯降;縱兵
 尾擊,敵知不得免,始請命。廣曰:「不從吾言,今窘而就死,非降也。」悉斬之。遷本路都監。詔入對,神宗獎金湯、石門之功,慰賜甚厚,將使開熙河。辭以不習洮、隴事,乃遷鈐轄使,還徙鄜延。攻踏白城,功最,遷皇城使。進討洮羌,加帝御器械、環慶副都總管。安南用師,詣闕請行。帝曰:「南方卑濕。知卿病足,西邊方開拓,宜復歸。」擢龍神衛四廂都指揮使、英州刺史。邊臣或言:「往者劉平因救鄰道戰沒,今宜罷援兵。」廣曰:「此乃制賊長計也。使賊悉力寇一
 路,而他道不救,雖古名將亦無能為已。平之所以敗,非出援罪。」乃止。



 再轉步軍都虞候。韓存寶討瀘蠻乞弟,逗撓不進,詔廣代之。廣至,閱兵合將,搜人材勇怯,三分之,日夕肄習,間椎牛享犒,士心皆奮。遣使開曉乞弟,仍索所亡卒。乞弟歸卒七人,奏書降而身不至。乃決策深入,陳師瀘水,率將吏東鄉再拜。誓之曰:「朝廷以存寶用兵亡狀,使我代之,要以必禽渠魁。今孤軍遠略,久駐賊境,退則為戮,冒死一戰,勝負未可知。縱死,猶有賞,愈於退
 而死也。與汝等戮力而進,可乎?」眾皆踴躍。廣挾所得渠帥及質子在軍,而令以次酋護餉,以是入箐道而無鈔略之患。師行有二途,從納溪抵江門近而險,從寧遠抵樂共壩遠而平。蠻意官軍必出江門,盛兵阻隘;而師趨樂共,蠻不能支,皆遁去。廣分兵繞帽溪,掩江門後,破其險,水際皆通行,益前進,每戰必捷。次落婆遠,乞弟遣叔父阿汝約降求退舍,又約不解甲。廣策其有異,除阜為壇,距中軍五十步,且設伏。明日,乞弟擁千人出降,匿弩
 士氈裘,猶豫不前謝恩。廣發伏擊之,蠻奔潰,斬阿汝及大酋二十八人。乞弟以所乘馬授弟阿字,大將王光祖追斬之,軍中爭其尸,乞弟得從江橋下脫走。得其種落三萬,進次歸徠州,窮探巢穴,發故酋甫望個恕□塚。天寒,士多墮指,而乞弟竟不可得。監軍先受密詔,聽引兵還,遂班師。



 拜衛州防禦使、馬軍都虞候。西兵未解,上疏求面陳方略。及入見,言:「韓存寶雖有罪,功亦多,以今日朝廷待諸將,存寶不至死。」廣還部,至閿鄉,疽發斷頸卒,年
 四十八。



 廣為人有風義,輕財好施,學通《左氏春秋》。臨事持重,長於料敵,以智損益《八陳圖》,又撰約束百餘條列上,邊地頗推行之。其名聞於西夏。秉常母梁氏將內侮,論中國將帥,獨畏廣,聞其南征,乃舉兵。然在瀘以敕書招蠻,既降而殺之,此其短也。遄被惡疾死,或以為殺降之報云。



 論曰:宋太宗既厭兵,一意安邊息民,海內大治。真宗、仁宗深仁厚澤,涵煦生民,然仁文有餘,義武不足,蓋是時
 中國之人,不見兵革之日久矣。於是契丹、西夏起為邊患,乃不吝繒帛以成和好。神宗撫承平之運,銳焉有為,積財練兵,志在刷恥,故一時材智之士,各得暴其所長,以興立事功,若熊本、蕭注、陶弼、林廣實然。本、注起身科第,弼能詩好士,廣學通《左氏春秋》。昔孫權勸呂蒙學,文武豈二致哉!本上書以媚時相,廣之征蠻,發塚殺降,君子疵之。



\end{pinyinscope}