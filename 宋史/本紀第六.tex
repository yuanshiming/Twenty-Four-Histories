\article{本紀第六}

\begin{pinyinscope}

 真宗一



 真宗應符稽古神功讓德文明武定章聖元孝皇帝,諱恆,太宗第三子也。母曰元德皇后李氏。初,乾德五年,五星從鎮星聚奎。明年正月,後夢以裾承日,有娠,十二月
 二日生於開封府第,赤光照室,左足指有文成「天」字。幼英睿,姿表特異,與諸王嬉戲,好作戰陣之狀,自稱元帥。太祖愛之,育於宮中。嘗登萬歲殿,升御榻坐,太祖大奇之,撫而問曰:「天子好作否?」對曰:「由天命耳。」比就學受經,一覽成誦。初名德昌,太平興國八年,授檢校太保、同中書門下平章事,封韓王,改名元休。端拱元年,封襄王,改元侃。淳化五年九月進封壽王,加檢校太傅、開封尹。至道元年八月立為皇太子,改今諱,仍判府事。故事,殿廬
 幄次在宰相上,宮僚稱臣,皆推讓弗受。見賓客李至、李沆,必先拜,迎送降階及門。開封政務填委,帝留心獄訟,裁決輕重,靡不稱愜,故京獄屢空,太宗屢詔嘉美。



 三年三月,太宗崩,奉遺制即皇帝位於柩前。



 夏四月乙未,尊皇后為皇太后,赦天下,常赦所不原者咸除之。丙申,群臣請聽政,表三上,從之。戊戌,始見群臣於崇政殿西序,尋賜器幣。癸卯,門下侍郎兼兵部尚書、平章事呂端加右僕射。弟越王元份進封雍王,吳王元傑進封兗王,並
 兼中書令。徐國公元偓進封彭城郡王,涇國公元偁進封安定郡王,並同平章事。元儼封曹國公。侄閬州觀察使惟吉為武信軍節度使。侍衛馬步軍都虞候傅潛、殿前都指揮使王超、侍衛馬軍都指揮使李繼隆、侍衛步軍都指揮使高瓊並領諸軍節度。駙馬都尉王承衍、石保吉、魏咸信並為諸軍節度使。甲辰,宣徽北院使、知樞密院事趙鎔加南院使,左丞李至、禮部侍郎李沆並參知政事。丁未,中外群臣進秩一等。罷鹽鐵、度支、戶部副
 使。癸丑,置鎮戎軍。乙卯,靜海軍節度使、交址郡王黎桓加兼侍中,進封南平王。



 五月丁卯,詔求直言。庚午,命兩制議豐盈之術以聞。甲戌,戶部侍郎、參知政事李昌齡責授忠武行軍司馬。甲申,放宮人給事歲久者。丙戌,以鎮安軍節度使李繼隆同平章事。封姊秦國、晉國二公主並為長公主,齊國公主改許國長公主,妹宣慈、賢懿、壽昌、萬壽四公主並為長公主。丁亥,立秦國夫人郭氏為皇后。



 六月乙未,以太宗墨跡賜天下名山。戊戌,追復
 涪王廷美西京留守兼中書令、秦王,贈兄魏王德昭太傅、岐王德芳太保。己亥,上大行皇帝謚曰神功聖德文武皇帝,廟號太宗。辛丑,詔罷獻祥瑞。甲辰,復封兄元佐為楚王。乙巳,追冊莒國夫人潘氏為皇后,謚莊懷。以工部侍郎、同知樞密院事錢若水為集賢院學士。贈弟元億為代國公。



 秋七月乙丑,詔轉運使更迭赴闕,訪以民事。癸酉,詔訪孔子嫡孫。乙亥,以殿前都虞候範廷召領河西軍節度使,葛霸保順軍節度使,王漢忠威塞軍節
 度使,康保裔彰國軍節度使,王昭遠保靜軍節度使。甲申,以範廷召、葛霸為定州、鎮州駐泊都部署,王漢忠為高陽關行營都部署,康保裔為並、代州都部署。



 八月丙申,罷鹽井役。己亥,以鎮海軍節度使曹彬為樞密使,知樞密院事趙鎔為壽州觀察使,同知樞密院事李惟清為御史中丞,戶部侍郎向敏中、給事中夏侯嶠並為樞密副使。庚子,命以十二月二日為承天節。戊申,太白犯太微。己酉,封乳母齊國夫人劉氏為秦國延壽保聖夫
 人。先是,帝以漢、唐封乳母為夫人縣君故事付中書,已乃有是命。戊午,熒惑入東井。庚申,西川廣武卒劉旴逐巡檢使韓景祐,掠蜀、漢等州,招安使上官正、鈐轄馬知節討平之。



 九月丁丑,二星隕西南。戊寅,以孔子四十五世孫延世為曲阜縣令,襲封文宣公。



 冬十月,夏人寇靈州,合河都部署楊瓊擊走之。己酉,葬太宗於永熙陵。丁巳,賜山陵使而下銀帛有差。歲星入氐。



 十一月甲子,祔太宗神主於太廟,以懿德皇后配,祔莊懷皇后於別廟。
 丙寅,詔兩京死罪以下遞減一等,緣山陵役民賜租有差。己巳,詔工部侍郎錢若水修《太宗實錄》。己卯,賜帛西鄙餫餉士卒。閱騎射,擢精銳者十人遷職。乙酉,廢理檢院。十二月癸巳,承天節,群臣上壽於崇德殿。丙申,追尊母賢妃李氏為皇太后。辛丑,詔諸路轉運使申飭令長勸農。甲辰,以銀州觀察使趙保吉為定難軍節度使。



 咸平元年春正月辛酉,詔改元。丙寅,上皇太后李氏謚曰元德。丁丑,召學官崔頤正講《書》,因命宰臣選明經術
 者以聞。戊寅,閱御龍直。辛巳,僧你尾尼等自西天來朝,稱七年始達。甲申,彗出營室北。



 二月癸巳,呂端等言彗出之應當在齊、魯分。帝曰:「朕以天下為憂,豈直一方耶?」甲午,詔求直言,避殿減膳。乙未,慮囚,老幼疾病,流以下聽贖,杖以下釋之。丁酉,彗滅。



 三月己巳,置太平州。壬申,賜進士孫僅等宴瓊林。辛巳,以趙保吉歸順,遣使諭陜西,縱綏、銀流民還鄉,家給米一斛。



 夏四月,旱。壬辰,禱白鹿山。壬寅,趙保吉遣弟繼瑗入謝。己酉,遣使按天下吏
 民逋負,悉除之。



 五月戊午朔,日有食之。甲子,幸大相國寺祈雨,升殿而雨。



 六月辛卯,詔近臣舉常參官才堪轉運使者。丙辰,以旱,免開封二十五州軍田租。



 秋七月甲子,詔民供億山陵者賜租什二。己巳,詔沿淮諸州藏瘞遺骸。



 八月癸卯,禁新小錢。己酉,幸諸王宮。



 九月己巳,詔呂端、錢若水重修《太祖實錄》。壬申,賜終南隱士種放粟帛緡錢。己卯,以左衛上將軍張永德為太子太師。



 冬十月丙戌朔,日有食之。戊子,呂端為太子太保,戶部尚書
 張齊賢、參知政事李沆並平章事,李至為武勝軍節度使。己丑,參知政事溫仲舒罷為禮部尚書,樞密副使夏侯嶠罷為戶部侍郎、翰林侍讀學士,以樞密副使向敏中為兵部侍郎、參知政事,翰林學士楊礪、宋湜並為樞密副使。丙午,許群臣著述詣閣獻,令兩制銓簡。



 十一月丙辰,龍缽貢馬二千騎。甲子,詔葺歷代帝王陵廟。十二月庚寅,幸許國長公主第視疾。癸卯,令三司判官舉才堪知州者各一人。是歲,溪峒、吐蕃諸族、勒浪十六府大
 首領、甘州回鶻、西南蕃黎州山後蠻來貢。定州苞傷稼,遣使振恤,除是年租。



 二年春正月甲子,詔尚書丞、郎、給、舍,舉升朝官可守大郡者各一人。丙子,定諸司使以下至三班使臣有罪比品聽贖。



 二月丙申,以趙普配饗太祖廟庭。詔群臣迎養父母,蠲天下逋負,釋系囚。己酉,戒百官比周奔競,有弗率者,御史臺糾之。



 三月丙辰,江、浙發廩振饑。戊辰,置荊湖南路轉運使。壬申,王漢忠為涇、原、邠、寧、靈、環都部署。



 閏月丁亥,以久不雨,帝諭宰相曰:「凡政有闕失,宜相規以道,毋惜直言。」詔天下系囚非十惡、枉法及己殺人者,死以下減一等。幸許國長公主第視疾,又幸北宅視德願疾。詔兩京諸路收瘞暴骸,營塞破塚。戊子,幸太一宮、天清寺祈雨。己丑,上皇太后宮名曰萬安。庚寅,罷有司營繕之不急者。詔中外臣直言極諫。從弟德願卒。壬辰,雨。辛丑,江南轉運使言宣、歙竹生米,民採食之。丙午,詔江、浙饑民入城池漁採勿禁。



 夏四月丙寅,許國長公主
 薨。



 五月丁亥,嚴服用之制。乙巳,幸曹彬第視疾。



 六月丁巳,宰臣進《重修太祖實錄》。戊午,曹彬薨。庚辰,大食國遣使來貢。



 七月甲申,以傅潛為鎮、定、高陽關行營都部署,張昭允為都鈐轄。給外任官職田。己丑,以橫海軍節度使王顯為樞密使。壬寅,制《聖教序》賜傳法院。甲辰,幸國子監,召學官崔偓佺講《尚書?大禹謨》。還,幸崇文院,賜秘書監、祭酒以下器幣。丙午,置翰林侍讀學士,以兵部侍郎楊徽之等為之;置翰林侍講學士,以國子祭酒邢昺為
 之。



 八月辛亥,御文德殿,文武百官入閣。乙卯,群臣上尊號曰崇文廣武聖明仁孝皇帝。丁巳,大宴崇德殿,始作樂。戊午,社,宴近臣於中書。丙寅,大閱於東北郊。癸酉,楊礪卒。乙亥,以太師贈濟陽郡王曹彬配饗太祖廟庭,司空贈太尉中書令薛居正、忠武軍節度使贈中書令潘美、右僕射贈侍中石熙載配饗太宗廟庭。



 九月庚辰朔,日有食之。戊子,召宗室宴射後苑。甲午,奉安太宗聖容於啟聖院新殿,帝拜而慟,左右皆掩泣。賜修殿內侍緡
 錢。癸卯,幸騏驥院,賜從官馬,還,宴射後苑。鎮、定都部署言敗契丹兵於廉良路,殺獲甚眾。



 冬十月壬子,宜州執溪峒蠻酋三十餘人詣闕,詔釋其罪,遣還。癸丑,放澧州蠻界歸業民租。戊午,置福建路惠民倉。



 十一月壬午,詔親王領大都督府節鎮者勿兼長史。乙酉,饗太廟。丙戌,祀天地於圜丘,以太祖、太宗配,大赦天下,錄功臣子孫之無祿者。御朝元殿,受尊號冊。丁亥,賜群臣帶服、鞍馬、器幣有差。庚寅,大宴含光殿。壬辰,張齊賢加門下侍郎,李
 沆加中書侍郎,中外臣悉加恩。甲午,以左神武軍大將軍德恭為左衛大將軍,左衛大將軍德彞為左神武軍大將軍。乙未,詔:幸河北,所次頓舍給用,毋泛及州縣。以周瑩為駕前軍都部署,石保吉為行營先鋒都部署。己亥,狩近郊。辛丑,賜京城父老衣帛。戊申,以魏咸信為貝、冀行營都部署。己酉,以李沆為東京留守。十二月辛亥,賜近臣戎服、廄馬。甲寅,駕發京師,次陳橋。王昭遠卒。戊午,駐蹕澶州。冀州言敗契丹兵於城南,殺千餘人,奪馬
 百餘匹。辛酉,宴從臣於行宮。以王超等督先鋒,仍示以陣圖,俾識部分。壬戌,賜近臣甲冑、弓劍。幸浮橋,登臨河亭,賜澶州父老錦袍、茶帛。甲子,次大名,躬御鎧甲於中軍。契丹攻威虜軍,本軍擊敗之,殺其酋帥。府州言官軍入契丹五合川,拔黃太尉砦,殲其眾,焚其車帳,獲馬牛萬計。丁卯,召見大名府父老,勞賜之。是歲,沙州蕃族首領、邛部川蠻、西南蕃、占城、大食國來貢。江、浙、廣南、荊湖旱,嵐州春霜害稼,分使發粟振之。



 三年春正月己卯朔,駐蹕大名府。詔並、代都部署高瓊等分屯冀州、邢州。辛巳,臨視樞密副使宋湜疾。癸未,以葛霸為貝、冀、高陽關前軍行營都部署。萊州防禦使田紹斌凡十人以功進秩。契丹犯河間,高陽關都部署康保裔死之。乙酉,流忠武軍節度使傅潛於房州、都鈐轄張昭允於通州,並削奪官爵。丁亥,幸紫極宮,還,登子城閱騎射。高陽關、貝、冀路都部署範廷召等追契丹至莫州,斬首萬餘級。庚寅,赦河北及淄、齊州罪人,非持杖劫
 盜、謀故殺、枉法贓、十惡至死者並釋之。錄將吏死事者子孫,民被焚掠者復其租。罷緣邊二十三州軍榷酤。令諸州舉吏民有武藝及材力過人者。壬辰,宋湜卒。甲午,發大名府。益州軍變,害鈐轄符昭壽,逐知州牛冕等,推都虞候王均為首作亂。詔戶部使雷有終為廬州觀察使,帥師會李惠等討之,均閉城門固守。庚子,至自大名府。戊申,幸呂端第視疾。



 二月庚申,宴含光殿。辛酉,詔:「近臣並知雜御史、尚書省五品及帶館閣三司職者,各舉
 升朝官有武乾堪邊任一人。」癸亥,以周瑩為宣徽南院使,王繼英為北院使,並知樞密院事。王旦為給事中,同知樞密院事。乙丑,以王顯為定州路行營都部署,王超為鎮州路行營都部署。丁卯,益州王均開城偽遁,雷有終等入城,為所敗,退保漢州,李惠死之。戊辰,京畿旱,慮囚。癸酉,大雨。甲戌,置靜樂軍。丙子,賞花苑中,召從臣宴射。



 三月戊寅朔,日有食之。甲午,御崇政殿試禮部貢舉人。



 夏四月戊申朔,賜進士陳堯咨等袍笏。庚戌,呂端薨。
 甲寅,閱河北防城舉人康克勤等擊射。乙卯,葬元德皇太后。丁巳,以葛霸為邠、寧、環、慶都部署。壬申,前知益州牛冕、西川轉運使張適並削籍,冕流儋州,適為連州參軍。



 五月丁卯,詔天下死罪減一等,流以下釋之,十惡至死、謀故劫殺、坐贓枉法者論如律。幸玉津園觀刈麥。己丑,幸金明池觀水嬉,遂幸瓊林苑宴射。壬寅,御試河北舉人。河決鄆州,詔徙州城。



 六月己未,太白晝見。丁卯,以向敏中為河北、河東宣撫使,按巡郡國,存慰士民。



 秋七
 月己亥,以翰林侍讀學士夏侯嶠、侍講邢昺為江、浙巡撫使。



 八月辛亥,京東水災,遣使安撫。



 九月庚辰,賜契丹降人蕭肯頭名懷忠,為右領軍衛將軍、嚴州刺史;招鶻名從化,為右監門衛將軍;蟲哥名從順,為千牛衛將軍。壬辰,幸大相國寺,遂宴射玉津園。壬寅,衛國公張永德薨。



 冬十月甲辰,雷有終大敗賊黨,復益州,殺三千餘人。壬子,綿、漢都巡檢、澄州刺史張思鈞削籍流封州。乙卯,幸元份宮視疾。令諸州兼群牧。己未,濱州防禦使王榮
 削籍流均州。己丑,雷有終追斬王均於富順監,禽其黨六千餘人。詔原川峽路系囚雜犯死罪以下。雷有終等以功進秩有差。丙寅,以翰林學士王欽若、知制誥梁顥分為川、峽安撫使。延州言破大盧、小盧等十族,獲人畜二十萬。



 十一月甲戌,環、慶副部署徐興削籍配郢州。乙亥,靈州副部署孫進責授復州團練副使。鄆州決河塞。戊寅,均畿內田稅。壬午,詔群臣盡言無諱,常參官轉對如故事,未預次對者聽封事以聞。辛卯,日南至,御朝元
 殿受朝。丙申,張齊賢罷為兵部尚書。十二月戊申,狩近郊,以親獲禽獻太廟。甲寅,大宴含光殿。乙卯,幸元份宮視疾。丁巳,閱武藝,遂宴射苑中。庚申,罷京畿均田稅。育吾蕃部貢嫠牛。甲子,契丹稅木監使黃顒等率屬內附,賜冠帶。丙寅,開封府奏獄空,詔嘉之。丁卯,詔河東、北緣邊吏民斬邊寇首一級支錢五千,禽者倍之,獲馬者給帛二十匹。是歲,高麗、大食國、高州蠻來貢。畿內、江南、荊湖旱,果、閬州水,並振之。



 四年春正月甲戌朔,詔天下系囚死罪己下減一等,杖罪釋之。辛巳,幸範廷召第視疾。甲申,命樞密直學士馮拯、陳堯叟詳中外封事。詔應益州軍民因城亂殺傷劫盜,除官吏外,皆釋不問。乙酉,命收瘞西川遺骸。丁亥,幸開寶寺,還,禦乾天門觀燈。庚子,謁啟聖院太宗神御殿。



 二月丁未,祈雨。戊申,交州黎桓貢馴犀象。癸丑,決天下獄。丁巳,幸大相國寺、上清宮祈雨。戊午,雨,帝方臨軒決事,沾服不御蓋。壬戌,詔群臣子弟奏補京官者試一經。
 甲子,釋逋負官物者二千六百餘人,蠲逋負物二百六十餘萬。已納而非理者以內府錢還之,沒者給其家。丙寅,詔學士、兩省御史臺五品、尚書省諸司四品以上,舉賢良方正直言敢諫一人。己巳,置永利監。



 三月甲戌,撫水州蠻酋蒙瑛等來納兵器、毒藥箭,誓不復犯邊。乙亥,詔史館韓瑗等舉御史臺推勘官。丁丑,風雪,帝謂宰相曰:「霾曀頗甚,卿等思闕政,以佐予治。」李沆等乞免官,不許。辛巳,分川峽轉運使為益、利、梓、夔四路。召終南隱士
 種放,辭疾不至。庚寅,左僕射呂蒙正、兵部侍郎向敏中並平章事,中書侍郎、平章事李沆加門下侍郎。高瓊為殿前都指揮使,葛霸為侍衛馬軍都指揮使,王漢忠為殿前副都指揮使,並領節度。司天監進《儀天歷》。辛卯,以參知政事王化基為工部尚書,同知樞密院事王旦為工部侍郎、參知政事,樞密直學士馮拯、陳堯叟並為右諫議大夫、同知樞密院事。



 夏四月丙午,葛霸為並、代行營都部署。壬子,詔親老無兼侍者特與近任。回鶻可汗
 祿勝貢玉勒鞍、名馬、寶器,願以兵助討繼遷。丙辰,審官院引對京朝官,閱殿最而黜陟之。己未,以王欽若為左諫議大夫、參知政事。庚申,幸元份宮視疾,遂幸諸王宮。辛未,御試制科舉人。



 五月壬申朔,禦乾元殿受朝。京畿系囚罪流以下遞減一等,杖罪釋之。癸酉,以元儼為平海軍節度使。甲申,工部侍郎致仕朱昂對便殿,賜器幣。戊子,亳州貢白兔,還之。乙未,大同軍留後桑贊為侍衛步軍副都指揮使,領河西軍節度。



 六月癸卯,有司言
 減天下冗吏凡十九萬五千餘人。丁巳,詔東川民田先為江水所害者除其租。丁卯,詔州縣學校及聚徒講誦之所,並賜《九經》。戊申,出陣圖標宰相,命督將練士,以備北邊。



 秋七月庚午,以河朔饋運勞民,詔轉運使減徭役存恤。己卯,邊臣言契丹謀入寇。以王顯為鎮、定、高陽關三路都部署,王超為副都部署,王漢忠為都排陣使。



 八月辛丑,張齊賢為涇、原等州安撫經略使。戊申,出環慶至靈州地圖險要示宰相,議戰守方略。己酉,御試制科舉
 人。壬子,幸開寶寺。又幸御龍營閱武藝,賜緡錢有差。遂觀稼北郊,宴射於含芳園。丁卯,遣使巴蜀,廉察風俗、官吏能否。戊辰,社,宴宰相於中書。



 九月,慶州地震。李繼遷陷清遠軍。



 冬十月,曹璨以蕃兵邀李繼遷輜重於唐龍鎮。己未,張斌破契丹於長城口。



 十一月壬申,知階州竇玭獻白鷹,還之。王顯奏破契丹,戮二萬人,獲統軍鐵林等。癸未,京城民獲金牌,有「趙為君萬年」字。庚寅,畋近郊。甲午,龜茲國來貢。十二月丁未,詔蜀賊王均既平,除追
 捕亡命,餘詿誤之民並釋不問。訛言動眾者,有司斬以聞。丙寅,太白晝見南斗。丁卯,詔罷三路都部署兼河北轉運使。



 閏月己巳,幸大相國寺。丁丑,邠、寧副都部署楊瓊等七將流嶺南。戊寅,李繼遷蕃族訛遇等歸順。己卯,以兵部尚書張齊賢為右僕射。壬午,靈州言河外砦主李瓊等以城降西夏。上念其力屈就禽,特釋其親屬。乙酉,李繼遷部族訛豬等率屬來附。庚寅,河北饑,蠲賦減役,發廩振之。是歲,龜茲、丹眉流、宜高上溪撫水州蠻來
 貢。梓州水,遣使振恤。



 五年春正月壬寅,李繼遷部將臥浪己等內附,給田宅。壬戌,環、慶部署張凝襲諸蕃,焚族帳二百餘,斬首五千級,降九百餘人。



 二月乙酉,詔邊士疾病戰沒者,冬春衣聽給其家。己丑,幸上清宮。以王漢忠為邠寧、環、慶路都部署。



 三月丁酉,李繼遷陷靈州,知州裴濟死之。庚戌,比部員外郎洪湛削籍流儋州,工部尚書趙昌言責授安遠軍司馬,知雜御史範正辭滁州團練副使。己未,御試
 禮部舉人。



 夏四月壬申,詔陜西民挽送緣邊芻糧者,賜租之半。壬午,命三司歲較戶口。丙戌,賜深、霸九州民租有差。癸巳,復雄州榷場。



 五月庚子,減河北冗官。壬寅,知榮州褚德臻坐盜取官銀,棄市。癸卯,置憲州。代州進士李光輔善擊劍,詣闕。帝曰:「若獎用之,民悉好劍矣。」遣還。甲辰,詔申明內侍養一子制。乙巳,蠲天下逋負。丙午,以王顯為河陽三城節度使。



 六月癸酉,繼遷圍麟州,曹璨請濟師,詔發並、代、石、隰州兵援之。乙亥,以侍衛馬軍都
 虞候王超為定州路駐泊行營都部署。己卯,以宣徽南院使、知樞密院事周瑩為永清軍節度使。己酉,詔益兵八千分屯環慶、涇原。知麟州衛居實言繼遷以眾二萬來攻城,兵出擊走之,殺傷過半。是月,都城大雨,壞廬舍,民有壓死者,振恤其家。



 秋七月甲午朔,日有食之。戊戌,幸啟聖院、太平興國寺、上清宮致禱,雨霽,遂幸龍衛營視所壞垣室,勞賜有差。乙巳,召終南隱士種放。疏丁岡河。癸丑,詔許高州蠻田彥伊子承寶等入朝,賜器帛、冠
 帶。乙卯,募河北丁壯。壬戌,契丹大林砦使王昭敏等來降。戎人寇洪德砦,守將擊走之。癸亥,增川峽官奉錢。



 八月,群臣三表上尊號,不允。丙子,沙州曹宗壽遣使入貢,以宗壽為歸義軍節度使。乙酉,石、隰部署言河西蕃族拽浪南山等四百人來歸。



 九月戊申,種放對於便殿,授左司諫、直昭文館。乙卯,賜種放第宅。



 冬十月己巳,遣使繼藥賜鎮戎軍將士。戊寅,詔河西戎人歸順者,給內地閑田處之。又詔諸州長吏與佐職官同錄問大闢罪人。
 辛巳,涇原部署系內屬蕃族數叛者九十一人,請誅之,詔釋其罪。丁亥,平章事向敏中罷為戶部侍郎,右僕射張齊賢為太常卿。庚寅,修豐州城。



 十一月壬辰,詔麟州給復一年。甲午,六穀首領潘羅支等貢馬,第給其直。辛丑,享太廟。壬寅,祀天地於圜丘,大赦。丁未,白州民黃受百餘歲,賜粟帛。己酉,封子玄祐為信國公。庚戌,呂蒙正加司空,李沆加右僕射,楚王元佐為右羽林軍上將軍,雍王元份守太傅,兗王元傑守太保,曹國公元儼同平
 章事。十二月壬午,賜京城百歲老人祝道巖爵一級。癸未,遷麟州內屬人於樓煩。是歲,河北、鄭、曹、滑州饑,振之。



\end{pinyinscope}