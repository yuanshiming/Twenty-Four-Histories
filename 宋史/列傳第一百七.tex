\article{列傳第一百七}

\begin{pinyinscope}

 傅楫沉畸蕭服附徐績張汝明黃葆光石公弼張克公附毛注洪彥升鐘傅陶節夫毛漸王祖道張莊趙遹



 傅楫,字符通,興化軍仙游人。少自刻厲,從孫覺、陳襄學。第進士,調揚州司戶參軍,攝天長令,發擿隱伏,奸猾屏跡。轉福清丞,知龍泉縣。孫覺為御史中丞,語之曰:「朝廷欲用君,盍少留?」楫曰:「仕宦所以樂居中者,免外臺督責耳,今俯首權門,觡外臺奚擇?且外官,己所當得也。」遂去不顧。



 道除太學博士,居四年,未嘗一跡大臣門。既滿,徑赴銓曹。楫丞福清時,受知郡守曾鞏,鞏弟布方執政,由是薦為太常博士。徽宗以端王就資善堂學,擇師傅為說
 書,升楫記室參軍,進侍講、翊善,中人輗事於府者,多與宮僚狎,楫獨漠然不可親,一府嚴憚之。五年不遷。鄒浩得罪貶,楫以贐行免官。



 徽宗即位,召為司封員外郎,歷監察御史、國子司業、起居郎,拜中書舍人。時曾布當國,自以于楫有汲引恩,冀為之用。楫略無所傾下,凡命令有不當,用人有未厭,悉極論之,雖屢卻不為奪,布大失望。帝以舊學故,多所延訪,楫每以遵祖宗法度、安靜自然為言。他日,李清臣勸帝清心省事,帝曰:「近臣中唯
 傅楫嘗道此。」



 楫在朝歲餘,見時事浸異,竊嘆曰:「禍其始此乎!」聞者甚之,楫笑曰:「後當信吾言。」遂上疏丐去,以龍圖待制知亳州。卒,年六十一。帝念其蕃邸舊臣,賜絹三百匹。



 沉畸,字德侔,湖州德清人。第進士,歷官州、縣。崇寧中,為尚書議禮編修官,召對,擢監察御史。畸至臺,欲有所論建,而六察無言事法,乃詣匭上十事,言花石擾民,土木弊國,冗費多,恩澤濫,議論異同,下情睽隔。其論當十、夾
 錫錢最為剴當,略曰:「小錢之便於民,久矣。古者軍興用之,或以一當百,至於當千,此權時之術,非可行於無事之世。今當十之議,固足紓目前,然使游手鼓鑄,無故有倍稱之息,何憚而不為?雖日加斷斬,勢不可止。恐未能期歲,東南小錢輕,錢輕則物重,物重則民愈困,此盜賊所由起也。陜西舊無銅錢,故以夾錫為貴,一切改鑄,則猶前日鐵錢耳。今東南方私鑄,又將使西北效之,是導民犯法也。」



 進殿中侍御史。嘗經國子監門,有小內侍從數
 騎絕道突過,騶卒追問不為止,臺檄諸司捕之不獲。畸曰:「風憲之地,可但已乎?」入言之,徽宗下內省跡治,竟抵罪。



 蔡京興蘇州錢獄,欲陷章綖兄弟,遣開封尹李孝壽、御史張茂直鞫之。株逮至千百,強抑使承盜鑄罪,死者甚眾,京猶以為緩。帝獨意其非辜,遣畸及御史蕭服往代。京將啖以顯仕,白為左正言,及擢侍御史。畸至蘇,即日決釋無左證者七百人,嘆曰:「為天子耳目司,而可傅會權要,殺人以茍富貴乎?」遂閱實平反以聞。京大怒,削
 畸三秩,貶監信州酒稅,未幾,卒。既而獄事竟,復羈管明州。使者持敕至家,將發棺驗實,畸子浚泣訴,乃止。建炎初,贈龍圖閣直學士。浚官至右正言。



 蕭服,字昭甫,廬陵人。第進士,調望江令,治以教化為本。訪古跡,得王祥臥冰池、孟宗泣筍臺,皆為築亭。又刻唐縣令鞠信陵文於石,俾民知所向。已而邑人朱氏女刲股愈母疾,人頌傳之,以為治化所致。知高安縣,尉獲兇盜,獄具矣,服審其辭,疑之,且視其刀室不與刃合,頃之
 而殺人者得,囚蓋平民也。徙知康州,未行,改親賢宅教授。提舉淮西常平,召為將作少監。



 以使事得入對,論人主聽言之要,以謂唐、虞盛世,猶畏巧言而SW讒說。纚纚數百言,徽宗謂有爭臣風,擢監察御史。奉詔作《崇寧備官記》,帝稱善,詔輔臣曰:「服文辭勁麗,宜居翰苑。朕愛其鯁諤,顧臺諫中何可闕此人?」俄偕沉畸使鞫獄,坐羈管處州,逾歲得歸。張商英當國,引為吏部員外郎。送遼使,得疾於道,遂致仕。既愈,還舊職,以父老,得請知蘄州。卒,
 年五十六。



 徐績,字符初,宣州南陵人。舉進士,調吳江尉,選桂州教授。王師討交址,轉運使檄績從軍。餉路瘴險,民當役者多避匿,捕得千餘人,使者使績杖之,績曰:「是固有罪,然皆饑羸病乏,不足勝杖,姑涅臂以戒,亦可已。」使者怒,欲並劾績,績力爭不變,使者不能奪。郭逵宿留不進,績謂副使趙離曰:「師出淹時,而主帥無討賊意,何由成功?」因具蠻人情狀疏於朝,謂斷者人主之利器,今諸將首鼠
 不進,惟斷自上意而已。既而逵、離果皆以無功貶。



 舒但聞其名,將以御史薦,績惡但為人,辭不答。求知建平縣,入為諸王宮教授,通判通州。瀕海有捍堤,廢不治,歲苦漂溺。績躬督防卒護築之,堤成,民賴其利。復教授廣陵、申王院,改諸王府記室參軍。哲宗見其文,諭獎之,欲俟滿歲以為左右史,未及用。



 徽宗立,擢寶文閣待制兼侍講,遷中書舍人,修《神宗史》。時紹聖黨與尚在朝,人懷異意,以沮新政。帝謂績曰:「朕每聽臣僚進對,非詐則諛;惟
 卿鯁直,朕所倚賴。」因論擇相之難,雲已召范純仁、韓忠彥。績頓首賀曰:「得人矣!」詔與蔡京同校《五朝寶訓》。績不肯與京聯職,固辭,奏京之惡,引盧杞為喻。遷給事中、翰林學士。上疏陳六事:曰時要,曰任賢,曰求諫,曰選用,曰破朋黨,曰明功罪。



 國史久不成,績言:「《神宗正史》,今更五閏矣,未能成書。蓋由元祐、紹聖史臣好惡不同,範祖禹等專主司馬光家藏記事,蔡京兄弟純用王安石《日錄》,各為之說,故論議紛然。當時輔相之家,家藏記錄,何得
 無之?臣謂宜盡取用,參討是非,勒成大典。」帝然之,命績草詔戒史官,俾盡心去取,毋使失實。



 帝之初政,銳欲損革新法之害民,曾布始以為然,已乃密陳紹述之說。帝不能決,以問績,績曰:「聖意得非欲兩存乎?今是非未定,政事未一,若不考其實,姑務兩存,臣未見其可也。」又因論棄湟州,請「自今勿妄興邊事,無邊事則朝廷之福,有邊事則臣下之利。自古失於輕舉以貽後悔,皆此類也。」



 績與何執中偕事帝於王邸,蔡京以宮僚之舊,每曲意
 事二人,績不少降節。謁歸視親病,或言翰林學士未有出外者,帝曰:「績謁告歸爾,非去朝廷也,奈何輕欲奪之!」俄而遭憂。京入輔,執中亦預政,擿績行章惇詞,以為詆先烈。服闋,以主管靈仙觀,入黨籍中。起知江寧府,言者復論為元祐奸朋,必不能推行學政,罷歸。



 大觀三年,知太平州。召入覲,極論茶鹽法為民病,帝曰:「以用度不足故也。」對曰:「生財有道,理財有義,用財有法。今國用不足,在陛下明詔有司,推講而力行之耳。」帝曰:「不見卿久,今
 日乃聞嘉言。」加龍圖閣直學士,留守南京。



 蔡京自錢塘召還,過宋見績,微言撼之曰:「元功遭遇在伯通右,伯通既相矣。」



 績笑曰:「人各有志,吾豈以利祿易之哉?」京慚不能對,績亦終不復用。以疾,除顯謨閣學士致仕。卒,年七十九。贈資政殿學士、正奉大夫。績挺挺持正,尤為帝所禮重,而不至大用,時議惜之。



 張汝明,字舜文,世為廬陵人,徙居真州。兄侍御史汝賢,元豐中以論尚書左丞王安禮,與之俱罷。未幾,卒。汝明
 少嗜學,刻意屬文,下筆輒千百言。入太學,有聲一時。國子司業黃隱將以子妻之,汝明約無飾華侈,協力承親歡,然後受室。



 登進士第,歷衛真、江陰、宜黃、華陰四縣主簿,杭州司理參軍,亳州鹿邑丞。母病疽,更數醫不效,汝明刺血調藥,傅之而愈。江陰尉貧且病,市物不時予直,部使者欲繩以法,汝明為鬻橐中裝,代償之。華陰修岳廟,費鉅財窘,令以屬汝明。汝明嚴與為期,民德其不擾,相與出力佐役,如期而成。他廟非典祀、妖巫憑以惑眾
 者,則毀而懲其人。滯州縣二十年,未嘗出一語干進,故無薦者。



 大觀中,或言其名,召置學制局,預考貢士,去取皆有題品。值不悅者誣以背王氏學,詔究其事,得所謂《去取錄》,徽宗覽之曰:「考校盡心,寧復有此?」特改宣教郎。



 擢監察御史。嘗攝殿中侍御史,即日具疏劾政府市恩招權,以蔡京為首。帝獎其介直。京頗憚之,徙司門員外郎,猶虞其復用,力排之,出通判寧化軍。地界遼,文移數往來,汝明名觸其諱,遼以檄暴於朝。安撫使問故,眾欲
 委罪於吏,汝明曰:「詭辭欺君,吾不為也。」坐責監壽州麻步場。遇赦,簽書漢陽判官。田法行,受牒按境內。時主者多不親行,汝明使四隅日具官吏所至,而躬臨以閱實,雖雨雪不渝,以故吏不得通賄謝,而稅均於一路最。晚知嶽州,屬邑得古編鐘,求上獻。汝明曰:「天子命我以千里,懼不能仰承德意,敢越職以幸賞乎?」卒於官,年五十四。



 汝明事親孝,執喪,水漿不入口三日,日飯脫粟,飲水,無醯鹽草木之滋。浸病羸,行輒踣。夢父授以服天南星
 法,用之,驗,人以為孝感。汝明學精微,研象數,貫穿經史百家,所著書不蹈襲前人語,有《易索書》、《張子卮言》、《大究經》傳於世。



 黃葆光,字符暉,徽州黟人。應舉不第,以從使高麗得官,試吏部銓第一,賜進士出身。由徐州司理參軍為太學博士,遷秘書省校書郎,擢監察御史、左司諫。始蒞職,即言:「三省吏猥多,如遷補、升轉、奉入、賞勞之類,非元豐舊制者,其大弊有十,願一切革去。」徽宗即命厘正之,一時
 士論翕然。而蔡京怒其異己,密白帝,請降御筆云:「當豐亨豫大之時,為衰亂減損之計。」徙葆光符寶郎。省吏醵錢入寶菉宮,作十道齋報上恩,帝思其忠,明年,復拜侍御史。



 遼人李良嗣來歸,上《平夷書》規進用,擢秘書丞。葆光論其五不可,大概言「良嗣兇黠忿鷙,犯不赦之罪於鄰國,逃命逭死,妄作《平夷》等書,萬一露洩,為患不細。中秘圖書之府,豈宜以罪人為之?宜厚其祿賜,置諸畿甸之外。」又言:「君尊如天,臣卑如地。剛健者君之德,而其道
 不可屈;柔順者臣之常,而其分不可亢。茍致屈以求合,則是傷仁,非所以馭下也;茍矯亢以求伸,則是犯分,非所以尊君也。」帝感悟,命近臣讀其奏於殿中。



 自崇寧後,增朝士,兼局多,葆光以為言。乃命蔡京裁定,京陽請一切廢罷,以激怒士大夫。葆光言:「如禮制局詳議官至七員、檢討官至十六員,制造局至三十餘員,豈不能省去一二,上副明天子之意?」時皆壯之。



 政和末,歲旱,帝以為念。葆光上疏曰:「陛下德足以動天,恩足以感人,檢身治
 事,常若不及,而不能感召和氣,臣所以不能無疑也。蓋人君有屈己逮下之心,而人臣無歸美報上之意者,能致陰陽之變;人君有慈惠惻怛之心,而人臣無將順奉承之意者,能致陰陽之變。陛下恭儉敦樸以先天下,而太師蔡京侈大過制,非所以明君臣之分;陛下以紹述為心,而京所行乃背元豐之法,強悍自專,不肯上承德意。太宰鄭居中、少宰餘深依違畏避,不能任天下之責。此天氣下而地不應,大臣不能尚德以應陛下之所求
 者如此。」疏入不報。且欲再上章,京權勢震赫,舉朝結舌,葆光獨出力攻之。京懼,中以它事,貶知昭州立山縣。又使言官論其附會交結,洩漏密語,詔以章揭示朝堂,安置昭州。京致仕,召為職方員外郎,改知處州。州當方臘殘亂之後,盡心收養,民列上其狀。加直秘閣,再任,卒,年五十八,州人祠之。



 葆光善論事,會文切理,不為橫議所移,時頗推重。本出鄭居中門,故極論蔡京無所顧,然其它不能不迎時好,方作神霄萬壽宮,溫州郭敦實、泗州
 葉點皆坐是得罪。葆光遂疏建昌軍陳並、秀州蔡崇、岳州傅惟肖、祁門令葛長卿不即奉行制書,存留僧寺形勝、佛像,及決罰道流,乞第行竄黜,遂悉坐停廢,議者尤之。



 石公弼,字國佐,越州新昌人。登進士第,調衛州司法參軍。淇水監牧馬逸,食人稻,為田主所傷。圉者訟至密,郡守韓宗哲欲坐以重闢。公弼謂此人無罪,宗哲曰:「人傷官馬,奈何無罪?」公弼曰:「禽獸食人食,主者安得不御,御
 之豈能無傷?使上林虎豹出而食人。可無殺乎?今但當懲圉者,民不可罪。」宗哲委,以屬吏。既而使者來慮囚,如公弼議。獲嘉民甲與乙斗,傷指;病小愈,復與丙斗,病指流血死。郡吏具獄,兩人以他物傷人,當死。公弼以為疑,駁而鞫之,乃甲指血流傷,因而丙發,指脫瘕中風死,非由擊傷也。兩人皆得免。



 章惇求太學官,或薦公弼,使往見。謝曰:「丞相素侮人,見者阿意茍容,所不忍也。」再調漣水丞。供奉高公備綱舟行淮,以溺告。公弼曰:「數日無風,安有是?」使尉
 核其所載,錢失百萬。呼舟人物色之,乃公備與寓客妻通,殺其夫,畏事覺,所至竊官錢賂其下,故詭為此說。即收捕窮治,皆服辜。



 知廣德縣,召為宗正寺主簿。入見,言:「朝廷比日所為,直詞罕聞,頌聲交至,未有為陛下廷爭可否者。願崇忠正以銷諛佞,通諫爭以除壅蔽。」徽宗善之。擢監察御史,進殿中侍御史。三舍法行,士子計等第,頗事告訐。公弼言:「設學校者,將以仁義漸摩,欲人有士君子之行。顧使之相告訐,非所以建學本意也。」又言:「刪
 定敕令官、寺監丞簿等,皆以執政近臣子弟為之,未有資考,不習政事。請一切汰遣,以開寒畯之路。」從之。



 由右正言改左司諫。論東南軍政之敝,以為「有兵之籍,無兵之技。以太半之賦,養無用之兵,異日懼有未然之患。」其後睦盜起,如其言。太史保章正朱汝楫冒奉得罪,而內侍失察者皆不坐。公弼言:「是皆矯稱詔旨,安得勿論?請自今中旨雖不當覆者,亦令有司審奏。」



 遷侍御史。蘇杭造作局工盛,公弼陳擾民之害,請革技巧之靡麗者,稍罷
 進奉,帝納之。蔡京始與公弼有連,故因得進用,至是,意浸異,京恚焉。徙太常少卿,遷起居郎,兼定王、嘉王記室。故事,初至宮,例得金繒之賜二百萬,公弼辭不受。



 大觀二年,拜御史中丞。執政言:「國朝未有由左史為中執法者。」帝曰:「公弼嘗為侍御史矣。」時斥賣元豐庫縑帛,賤估其直,許朝士分售,皆有定數,從官至二千匹。公弼得券,上還之。宰相有已取萬匹者,即日反其故。



 水官趙霆建開直河議,謂自此無水憂,已而決壞鉅鹿,法當斬。霆善
 交結,但削一官,猶為太僕少卿。公弼論為失刑,霆坐貶。京西轉運使張徽言欲因方田籍增立汝、襄、鄧三州稅,公弼以為「方田之制,奠天下之地征,正欲均其賦耳,而徽言掊克重斂,民何以堪?」詔罷之。遂劾蔡京罪惡,章數十上,京始罷。又言吏員猥冗,戾元豐舊制。於是堂選歸吏部者數千員,罷宮廟者千員、都水知埽六十員,縣非大郡悉省丞,在京茶事歸之戶部,諸道市舶歸之轉運司,仕塗為清。



 京雖上相印,猶提舉修《實錄》。公弼復言:「京
 盤旋京師無去志,其餘威震於群臣。願持必斷之決,以消後悔。」又因星變言之,竟出京杭州。及劉逵主國柄,公弼復論其廢紹述良法,啟用元祐邪黨學術,人以是知其非一意於正者。進兵部尚書兼侍讀。上疏言:「崇寧以來,臣下專務生事,開邊興利,營繕徭役,蹶民根本,因之饑饉。汴西挽運花石,農桑廢業,徒弊所有,以事無用。宜使之休息,以承天意。」



 張商英入相,欲引為執政,何執中、吳居厚交沮之。以樞密直學士知揚州。群不逞為俠於
 閭里,自號「亡命社」。公弼取其魁桀痛治,社遂破散。江賊巢穴菰蘆中,白晝出剽,吏畏不敢問。公弼嚴賞罰督捕,盡除之。改述古殿直學士、知襄州。蔡京再輔政,羅致其罪,責秀州團練副使,臺州安置。逾年,遇赦歸。卒,年五十五。後三歲,復其官。



 公弼初名公輔,徽宗以與楊公輔同名,改為公弼云。



 張克公,字介仲,穎昌陽翟人。起進士。大觀中,為監察御史,遷殿中侍御史。蔡京再相,克公與中丞石公弼論其
 罪,京罷,克公徙起居舍人。逾月,進中書舍人,改右諫議大夫。京猶留京師,會星文變,克公復論之,中其隱慝,語在京傳。京致仕,張商英為相,與鄭居中不合。克公由兵部侍郎拜御史中丞,治堂吏訟,歸曲商英,且疏其罪十。商英罷,京復召,銜克公弗置。徽宗知之,為徙吏部尚書。京欲以銓綜稽違中克公,既又擿其知貢舉事,帝以為所取得人,不問也。居吏部六年,卒,贈資政殿學士。



 毛注,字聖可,衢州西安人。舉進士,知南陵、高苑、富陽三
 縣,皆以治辦稱。大觀中,御史中丞吳執中薦為御史,詔賜對,未及而執中罷,注辭焉。徽宗固命之,既見,謂曰:「今士大夫方寡廉鮮恥,而卿獨知義命,故特召卿。」即以為主客員外郎,俄擢殿中侍御史。



 蔡京免相留京師,注疏其擅持威福,動搖中外,以葉夢得為腹心,交植黨與。帝為逐夢得,而遷注侍御史。遂極論京「受孟翊妖奸之書,與逆人張懷素游處,引兇朋林攄置政府,用所親宋喬年尹京。其門人播傳,咸謂陛下恩眷不衰,行且復用。」於是
 論者相繼,京遂致仕。



 四年,彗再見,注又言:「臣累論蔡京罪積惡大,天人交譴,雖罷相致政,猶怙恩恃寵,偃居賜第,以致上天威怒。推原其咎,實在於京。考京之罪,蓋不可以縷數:陛下去《黨碑》以開自新之路,京疾其異己而別為防禁;陛下頒明詔以來天下之言,京惡其議己而重致於法;以嚴刑峻罰脅持海內,以美官重祿交結人心,錢鈔屢更而商賈不行,邊事數易而國力大匱。聲焰所震,中外憤疾,宜早令去國,消弭災咎。」奏上,京始出居
 錢塘。



 注復採當世之急務,曰省邊事,足財用,收士心,禁技巧。大概謂:「近年以來,邊民僥幸茍得:昔所入貢者,今必城為郡縣;昔所羈縻者,今盡納其土疆。以內地金帛,而事窮荒不可計之費。今黔南已有處分,如夔、渝新邊,宜在裁省。運鹽昔主於漕計,今移於它司;常平昔積於外州,今輸於都下。經費安得不匱,財貨何以轉移?願詔有司,悉講復元豐舊制。湯之遭旱,以士失職為辭。今學校養士,蓋有常額,額外之人,不復可預教養,歲貢之餘,
 略無可進之地。願留貢籍三分,暫存科舉,以待學外之士,使無失職。東南造作奇玩、花石綱舟,與後苑工徒、京城營繕,並宜暫罷,以抑末敦本。凡此,皆聖政之所當先,人心悅則天意解矣。」注所論切於世務類此。



 遷左諫議大夫。張商英為相,言者攻之力,注亦言其無大臣體,然訖以與之交通,罷提舉洞霄宮,居家數歲,卒。建炎末,追復諫議大夫。



 洪彥升,字仲達,饒州樂平人。登第,調常熟尉。奉母之官,
 既至,前尉欲申期三月以規薦,而中分奉入。彥升處僧舍,卻奉不納,如約,始交印。歷郴州判官,簽書鎮東軍節度判官。



 彥升嘗闢廣西經略府,或稱其才,擢提舉常平。御史中丞石公弼薦新提舉廣西學事幸義可御史,及陛辭,適與同日,徽宗兩留之,遂為監察御史,遷殿中侍御史。彥升孤立,任言責閱五年,論:「蔡京再居元宰,假紹述之名,一切更張,敗壞先朝法度,朋奸誤國,公私困弊。既已上印,而偃蹇都城,上憑眷顧之恩,中懷跋扈之志。
 願早賜英斷,遣之出京。」「何執中緣潛邸之舊,德薄位尊,當軸處中,殊不事事,見利忘義,唯貨殖是圖。願解其機政,以全晚節。」「呂惠卿與張懷素厚善,序其所注《般若心經》云『我遇公為黃石之師。』且張良師黃石之策,為漢祖定天下,惠卿安得輒以為比?」他如鄧洵仁、蔡薿、劉拯、李孝稱、許光凝、許幾、盛章、李譓、任熙明之流,皆條摭其過,一不為回隱。



 右僕射張商英與給事中劉嗣明爭曲直,事下御史。彥升蔽罪商英,商英去。又累疏言郭天信以
 談命進用,交結竄斥;因請禁士大夫毋語命術,毋習釋教。



 先是,詔諸道監司具法令未備,若未便於民者,久而弗上。彥升言:「吏狃於勢,隨時俯仰,不能上承德音,因緣為奸者眾。有因追科而欲害熙寧保伍之法,因身丁而故搖崇寧學校之政,省事原情,當有勸沮。宜遣官編匯,辨其邪正,以行賞罰。」皆從之。遷給事中。嘗謁告一日,而張商英復官之旨經門下,言者以為顧避封駁,出知滁州。尋加右文殿修撰,進徽猷閣待制,知吉州。久之,知潭
 州,未行,卒,年六十三。贈太中大夫。



 論曰:蔡京用事,炎焰熾然,其勢莫敢遏。此數子者,乃力數其罪而連攻之,似矣。然葆光、克公主鄭居中,公弼、注朋張商英,皆非端直士也。若楫先見、畸、服不阿,汝明不欺,彥升孤立,其賢乎!唯績宮邸舊學,人望攸屬,而不使躋政地;至京則暫罷亟起,始終倚任焉。善善而不能用,惡惡而不能去,徽宗以之,此齊桓公所以嗤於郭亡也。



 鐘傅,字弱翁,饒州樂平人。本書生,用李憲薦,為蘭州推
 官。坐對獄不實,羈管郴州。紹聖中,章惇興邊事,奏還其官。得入對,為哲宗言:「兵貴智而不貴力,夏眾伙而勇,難以一舉滅。但當擇城險要,以正不朝削地之法,坐待其斃。」帝然之,命乾當熙河、涇原、秦鳳三路公事。



 夏人陷金明,渭帥毛漸出兵攻其沒煙砦,傅合擊破之,又與熙州王文鬱進築安西城,論功加秘閣校理。章楶帥渭,命傅所置將苗履統眾會涇原之靈平,夏人悉力來拒,傅步騎二萬,出不意造河梁以濟師,遂作金城關,又獻白草
 原捷,連進集賢殿修撰、知熙州。傅自始仕至此,僅再歲。遂擅帥熙、秦騎四萬出塞,無功而還。惇方主其議,不加罪。



 初,傅請合三路兵從青南訥心或顛耳關築天都城,以包淺井、□□囉、和市。工既集,復言水源不壯,不可興役。朝論以所奏乖異,將罷傅,曾布為言,但褫職。俄而白草原詐增首虜事覺,責監永州稅,再貶連州別駕。崇寧中,復起知河中府,歷鄆、瀛、渭三州,擢顯謨閣待制。建言:「河南要地,靈武為根本。其西十五州,六為王土。其東由清
 遠距羅山走靈州不及百里,夏以五監軍統焉。若選將簡師先擊之,以趨韋州,可斷其右臂。徐當拊納離畔,漸規進取,訖城蕭關,可斷其左臂。」乃條上十四事,未報。



 詔諸道進討,傅遣將折可適領銳騎出蕭關,至靈州川,有功。進龍圖閣直學士。會別將高永年沒於西,而可適遇雨失道,為虜所乘,乃班師。傅以稽違逗撓,黜知汝州,奪學士。未幾,復為杭州、真定、永興、太原、延安府,以故職卒。贈端明殿學士。傅從布衣致通顯,所行事大氐欺妄,故
 屢起屢僨云。



 陶節夫字子禮,饒州鄱陽人,晉大司馬侃之裔也。第進士,起家為廣州錄事參軍。楊元寇暴山谷間,捕系獄,屢越以逸,且不承為盜,既累年。節夫詰以數語,元即吐服,將適市,與諸囚訣曰:「陶公長者,雖死可無憾。」知新會縣,廣守章楶重其材。楶帥涇原,闢入府。



 崇寧初,為講議司檢討官,進虞部員外郎,遷陜西轉運副使,徙知延安府。以招降羌有功,加集賢殿修撰。築石堡等四城。石堡以
 天澗為隍,可趨者唯一路,夏人窖粟其間,以千數。既為宋有,其酋驚曰:「漢家取我金窟堝!」亟發鐵騎來爭。節夫分部將士遮御之,斬獲統軍以下數十百人。夏人度不可得,斂兵退。連擢顯謨閣待制、龍圖閣直學士。



 方議城銀州,諜告夏人已東。節夫料必西趨涇原,官屬不肯從,節夫曰:「吾計之熟矣。」乃遣裨將耿端彥疾驅至銀州,五日城成,夏人果從涇原至,則城備已固,遂遁去。進樞密直學士。



 節夫在延安日久,蔡京、張康國從中助之,故唯
 京意是徇。夏人欲款塞,拒弗納。放牧者執殺之,夏人怨怒,大入鎮戎軍,殺鹵數萬口。節夫尋領經制環慶、涇原、河東邊事,言:「今既得石堡,又城銀州,西夏洪、宥皆在吾顧盼中。橫山之地,十有七八,興州巢穴淺露,直可以計取。」遂陳取興、靈之策。加龍圖閣學士。會朝廷罷經制司,且棄所城地,節夫乃求內郡。徙洪州,改江寧府,歷青、秦二州、太原府。



 群盜李勉起遼州、北平之間,河東、河北騷動,兩路帥臣、憲臣皆罪去,至出臺郎督捕之。節夫請悉
 罷所遣兵,卒以計獲勉。坐上疏乞留本道兵勿移戍,降為待制、知永興軍,數月,卒。追復龍圖閣學士。



 毛漸,字正仲,衢州江山人。第進士,知寧鄉縣。熙寧經理五溪,漸條利害以上察訪使,使者諉以區畫,遂建新化、安化二縣。漸用是得著作佐郎、知安化縣,召為司農丞,提舉京西南路常平。



 元祐初,知高郵軍,遷廣東轉運判官。渠陽蠻擾邊,近臣言漸習知蠻事,徙荊湖北路轉運判官。時朝廷議棄地,漸曰:「蠻徭畔服不常,非稍威以兵,
 未易懷德。今一犯邊即棄地,非計也。」不報。渠陽既棄,蠻復大入鈔略,覆官軍,荊土為大擾。



 漸歷提點江西刑獄、江東、兩浙轉運副使。浙部水溢,詔賜緡錢二百萬以振之。漸言:「數州被害即捐二百萬,儻仍歲如之,將何以繼?」乃案錢氏有國時故事,起長安堰至鹽官,徹清水浦入於海;開無錫蓮蓉河,武進廟堂港,常熟疏涇、梅裏入大江;又開昆山七耳、茜涇、下張諸浦,東北道吳江,開大盈、顧匯、柘湖,下金山小官浦以入海。自是水不為患。



 加集
 賢校理,入為吏部右司郎中。以秘閣校理為陜西轉運使。攝渭、秦、熙三州。未幾,復攝帥涇原。日夜治兵,乘夏人犯邊,遣將搗其虛,遂破沒煙砦。進直龍圖閣、知渭州,命下,卒,年五十九。優贈龍圖閣待制。



 王祖道,字若愚,福州人。第進士,又舉制科,會罷,調韓城尉,知松陽、白馬二縣。為司農丞、監察御史。數言事,以論樞密承旨張誠一試補吏挾私、延州呂惠卿遣禁卒饋徐禧公使物非是,改司封員外郎、知汀、泉、福三州。歷使
 諸路,入為戶部、吏部員外郎,左司諫。言陜西兵未可減,徽宗謂其論事無足行,依阿茍容,出知海州。拜秘書少監,再為福州。加直龍圖閣、知桂州。



 蔡京開邊,祖道欲乘時徼富貴,誘王江酋楊晟免等使納士,誇大其辭,言:「向慕者百三十峒、五千九百家、十餘萬口,其旁通江洞之眾,尚未論也。王江在諸江合流之地,山川形勢,據諸峒要會,幅員二千里。宜開建城邑,控制百蠻,以武臣為守,置溪峒司主之。」詔以為懷遠軍,且頒諸司使至殿侍軍
 將告命,使第補其首領。置二砦,為立學。



 又言:「黎人為患六十年,道路不通。今願為王民,得地千五百里。」遂以安口隘為允州,中古州地為格州,增提舉溪峒官三員。又言羈縻知地州羅文誠、文州羅更晏、蘭州韋晏鬧、那州羅更從皆內附,請於黎母山心立鎮州,為下都督府,賜軍額曰靜海,知州領海南安撫都監,徙萬安軍於水口。南丹州莫公佞獨拒命,發兵討擒之,遂築懷遠軍為平州,格州為從州,南丹為觀州,並允、地、文、蘭、那五州置黔
 南路。擢祖道顯謨閣待制,進龍圖閣直學士。



 召為兵部尚書,未行,與融州張莊謀,使莊奏言海南一千二十峒皆已團結,所未得者百七十峒,今黎人款化,則未得者才十之一耳。於是徭、黎渠帥不勝忿,蜂起侵剽,圍新萬安軍及觀州,殺官吏。初,祖道徙城時,言黎人伐木助役。及是詔問,不能對。京芘之,猶除端明殿學士、知福州,復以刑部尚書召。大觀二年,卒,贈宣奉大夫。



 祖道在桂四年,厚以官爵金帛挑諸夷,建城邑,調兵鎮戍,輦輸內地
 錢布、鹽粟,無復齊限。地瘴癘,戍者歲亡什五六,實無尺地一民益於縣官。蔡京既自以為功,至謂:「混中原風氣之殊,當天下輿圖之半。」祖道用是超取顯美。張商英為相,治其誕罔,追貶昭信軍節度副使。京再輔政,復還之。然其所創名州縣,不旋踵皆罷。是後龐恭孫、張莊、趙遹、程鄰皆以拓地受上賞,大抵皆規模祖道。祖道起冗散,驟取美官,而朝廷受其敝云。



 張莊,應天府人也。元豐三年,擢進士第。歷提舉司、講議
 司檢討官,出提舉荊湖、夔州等路香鹽事。改提舉荊湖北路常平、本路提點刑獄,進龍圖閣直學士、廣南西路轉運副使。



 王祖道既請立朱崖諸州縣,徙萬安軍,詔莊按覆相度,實與祖道相表裏。祖道召為兵部尚書,授莊集賢殿修撰、知桂州。祖道既留,以莊知融州。已而祖道徙福州,莊復知桂州。奏:「安化上三州一鎮地土,及恩廣監洞蒙光明、落安知洞程大法、都丹團黃光明等納土,共五萬一千一百餘戶,二十六萬二千餘人,幅員九千
 餘里。」尋又奏:「寬樂州、安沙州、譜州、四州、七源等州納土,計二萬人,一十六州、三十三縣、五十餘峒,幅員萬里。」蔡京帥百官表賀,進莊兼黔南路經略安撫使、知靖州。



 王子武者,惠恭皇后族子也。靖州界接平、允、從三州,子武欲通之,因請復元祐所棄渠陽軍。渠陽既城,乃上言:「湖北至廣西,繇湖南則迂若弓背,自渠陽而往,猶弓弦耳。」因以利啖諸蠻使納土,立里堠。莊忌之,且欲蠻之多屬廣西為己功,因誘復水蠻石盛唐毀其烽表、橋梁。渠陽
 蠻酋楊惟聰請討之,子武以聞。朝議謂其生事,罷子武。



 未幾,安化蠻納土,莊遣黃忱往築州城。忱,蠻將也,知蠻情偽,力言不可。莊怒,遣忱護築溪州,別遣胡超、儂昌等築安化城,果為蠻所掩,超等沒者幾千人。中書舍人宇文粹中言:「祖道及莊擅興師旅,啟釁邀功,妄言諸蠻效順,納款得地。當時柄臣攬為綏撫四夷之功。奏賀行賞,張皇其事。自昔欺君,無大於此。」朝廷既追貶祖道,莊責舒州團練副使,永州安置,再貶連州,移和州。



 起知荊南
 府,徙江寧。復進徽猷閣直學士,歷知渭、毫、襄州、鎮江東平府。宣和六年,坐繕治東平城不加功輒復摧圮,降兩官,提舉嵩山崇福宮。卒,贈宣奉大夫。



 趙遹,開封人。大觀初,以發運司勾當公事為梓州路轉運司判官。滬、戎諸夷納土,命遹相置,以建立純州縣、砦勞,加直秘閣。升轉運副使,俄授龍圖閣直學士,為正使。



 政和五年,晏州夷酋卜漏反,陷梅嶺堡,知砦高公老遁。公老之妻,宗女也,常出金玉,器飲卜漏等酒漏心艷之。
 會滬帥賈宗諒以斂竹木擾夷部,且誣致其酋鬥個旁等罪,夷人咸怨。漏遂相結,因上元張燈襲破砦,虜公老妻及其器物,四出剽掠。遹行部昌州,聞之。倍道趣瀘州。賊分攻樂共城、長寧軍、武寧縣,宗諒皆遣將拒卻之。已而樂共城監押潘虎誘殺羅始黨族首領五十人,其族蠻憤怒,合漏等復攻樂共城。遹並劾之,詔斬虎,罷宗諒,代以康延魯,而聽遹節制。遹陰有專討意,兵端益大矣。於是詔發陜西軍、義軍、土軍、保甲三萬人,以遹為瀘南
 招討使。遹與別將馬覺、張思正分道出,期會於晏州。思峨州近而固,遹遣王育先破之,村囤諸落相繼而克,因其積穀食士卒。



 既抵晏州,覺、思正各以兵來會。漏據輪縛大囤,其山崛起數百仞,林箐深密,夷奔潰者悉赴之,乃壘石為城,外樹木柵,當道穿坑阱,僕巨蘗,布渠答,夾以守障,俯瞰官軍。矢石所中皆靡碎,遹軍不能進。間從巡檢種友直、田祐恭按視,其旁山崖壁特峭絕,賊恃之無守備。遹欲襲取,命友直、祐恭軍其下,而身當賊沖,番
 軍迭攻之。未旦,鼓而進,迨夕則止,賊並力拒戰,不得息。友直所部多思、黔土丁,習山險,而山多生猱,遹遣土丁捕之。伐去蒙密,緣崩石挽藤葛而上,得猱數十頭,束麻作炬,灌以膏蠟,縛於猱背。暮夜,復遣土丁負繩梯登崖顛,乃縋梯引下,人人銜枚,挈猱蟻附而上。比雞鳴,友直、祐恭與其眾悉登,擁刀斧穿箐入。及賊柵,出火然炬,猱熱狂跳,賊廬舍皆茅竹,猱竄其上,火輒發,賊號呼奔撲,猱益驚,火益熾。官軍鼓噪破柵,遹望見火,麾軍躡雲梯
 攻其前。兩軍相應,賊擾亂,不復能抗,赴火墮崖死者不可計,俘斬數千人。卜漏突圍走,至輪多囤,追獲之。晏州平,諸夷落皆降,拓地環二千里。遹為建城砦,畫疆畝,募人耕種,且習戰守,號曰「勝兵」。詔置沿邊安撫司,以轉運副使孫羲叟為安撫使。高公老妻不辱而死,詔贈節義族姬。



 加遹龍圖閣直學士、熙河蘭湟經略安撫使。遹以疾請祠,不許。既入對,賜上舍出身,拜兵部尚書。遹與童貫有隙,力請去,以提舉醴泉觀兼詳定一司敕令。六年,
 出知成德軍,拜延康殿學士,賜其子永裔上舍出身、秘書省校書郎。



 淶水人董才得罪亡命,因聚眾為賊,攻敗城邑,遼人不能制。中山帥府陰與才通,誘使來歸,才尋為遼所破,遂上書請取全燕以自效。王黼、童貫大喜,將許之,遹言不可。客或以沮朝廷密謀止遹,遹曰:「帥臣所部,封境雖異,事無異也。且論思獻納,侍從之職,遹今以侍從備帥臣,而真定、中山邊接,隙茍一開,吾境得無事乎?」疏奏,上然之,乃斥還才書。才窮蹙,轉入河東。詔以問
 遹,遹復具疏極論其害。洎遹徙熙州,黼等卒納才,又慮遹過闕入見有所陳,趣使便道赴鎮。諸蕃聞遹至,相賀曰:「吾父來,朝廷真欲無事矣!」爭出鋤耨,牛價為頓高。



 時議更陜西大鐵錢,價與銅錢輕重等。遹上言曰:「銅重鐵輕,自然之理,今反其理,民誰信之?以人奪天,雖厲其禁,終不可行也。」居數月,以疾乞致仕,命提舉嵩山崇福宮。起知中山、順昌、應昌府。金人舉兵,召遹赴闕,尋卒。



 永裔歷知眉州。言者論遹欺罔朝廷以軍功,永裔遂放罷。



 論曰:夏人時蹈窾,逐之使出則已。章惇、蔡京故撓之用兵,塗邊人肝腦於地,以幸己功,不亦顛乎?諸蠻溪峒,茅瘴非人域,鴆虺與居,況無敢闖吾圉。京乃使祖道、張莊之徒鑿空為功,舉中國重貲,棄諸不毛,而文飾奸慝,鋪張表賀,徽宗亦偃然受其欺,好大黷武之心一侈,而燕朔之謀作矣。《詩》曰:「池之竭矣,不云自頻;泉之竭矣,不云自中。」徽之耗內貪外,馴召禍敗,跡所從來,此其本也。嗚呼,可不戒哉!



\end{pinyinscope}