\article{列傳第二十一}

\begin{pinyinscope}

 李穀昝
 居潤竇貞固李濤弟浣孫仲容王易簡趙上交子□嚴張錫張鑄邊歸讜劉溫叟子燁孫幾劉濤邊光範劉載程羽



 李穀,字惟珍,穎州汝陰人。身長八尺,容貌魁偉。少勇力善射,以任俠為事,頗為鄉人所困,發憤從學,所覽如宿習。年二十七,舉進士,連闢華、泰二州從事。



 晉天福中,擢監察御史。少帝領開封尹,以穀為太常丞,充推官。晉祖幸鄴,少帝居守,加穀虞部員外郎,仍舊職。少帝為廣晉尹,穀又為府推官。及即位,拜職方郎中,俄充度支判官,轉吏部郎中,罷職。天福九年春,少帝親征契丹,詔許扈從,充樞密直學士,加給事中。為馮玉、李彥韜所排。會帝
 再幸河北,改三司副使,權判留司三司事。



 開運二年秋,出為磁州刺史、北面水陸轉運使。契丹入汴,少帝蒙塵而北,舊臣無敢候謁者,穀獨拜迎於路,君臣相對泣下。穀曰:「臣無狀,負陛下。」因傾囊以獻。會契丹主發使至州,穀禽斬之,密送款於漢祖,潛遣河朔酋豪梁暉入據安陽,契丹主患之,即議北旋。



 會有告契丹以城中虛弱者,契丹還攻安陽,陷其城,穀自郡候契丹,遂見獲。契丹主先設刑具,謂之曰:「爾何背我歸太原?」穀曰:「無之。」契丹主
 因引手車中,似取所獲文字,而穀知其詐,因請曰:「如實有此事,乞顯示之。」契丹國制,人未伏者不即置死。自後凡詰穀者六次,穀詞不屈。契丹主病,且曰:「我南行時,人云爾謂我必不得北還,爾何術知之?今我疾甚,如能救我,則致爾富貴。」穀曰:「實無術,蓋為人所陷耳。」穀氣色不撓,卒寬之。



 俄而德光道殂,永康繼立,署穀給事中。時契丹將麻答守真定,而李崧、和凝與家屬皆在城中。會李筠、何福進率兵逐麻答,推護聖指揮使白再榮權知留
 後。再榮利崧等家財,令甲士圍其居以求賂,既得之,復欲殺崧等滅口。穀遽見再榮謂之曰:「今國亡主辱,公輩握勁兵,不能死節,雖逐一契丹將,城中戰死者數千人,非獨公之力也。一朝殺宰相,即日中原有主,責公以專殺,其將何辭以對?」再榮甚懼,崧等獲免。



 漢初,入拜左散騎常侍。舊制,罷外郡歸本官,至是進秩,獎之也。俄權判開封府。時京畿多盜,中牟尤甚,穀誘邑人發其巢穴。有劉德輿者,梁時屢攝畿佐,居中牟,素有幹材,穀即署攝
 本邑主簿。浹旬,穀請侍衛兵數千佐德輿,悉禽賊黨,其魁一即縣佐史,一御史臺吏。搜其家,得金玉財貨甚眾,自是行者無患。俄遷工部侍郎。



 周祖西征,為西南面行營水陸轉運使。關右平,改陳州刺史。會有內難,急召赴闕。周祖兵入汴,命權判三司。廣順初,加戶部侍郎。未幾,拜中書侍郎、平章事,仍判三司。初,漢乾祐中,周祖討河中,穀掌轉運,時周祖已有人望,屬漢政紊亂,潛貯異志,屢以諷穀,穀但對以人臣當盡節奉上而已。故開國之
 初,倚以為相。是歲,淮陽吏民數千詣闕請立生祠,許之,穀懇讓得止。



 先是,禁牛革法甚峻,犯者抵死。穀乃校每歲用革之數,凡田十頃歲出一革,余聽民私用。又奏罷屯田務,以民隸州縣課役,盡除宿弊。穀父祖本居河南洛陽,經巢之亂,園廬蕩盡,穀生於外。既貴,訪得舊地,建蘭若,又立垣屋,凡族人之不可仕者分田居之。詔改清風鄉高陽里為賢相鄉勛德里。



 二年,晨起僕階下,傷右臂,在告,旬中三上表辭相位,周祖不允,免朝參,視事本
 司,賜白藤肩輿,召至便殿勉諭。穀不得已,起視事。征兗州,為東京留守、判開封府事。



 顯德初,加右僕射、集賢殿大學士。從世宗征太原,遇賊於高平,匿山谷中,信宿而出,追及乘輿,世宗慰撫之。世宗將趨太原,命穀先調兵食,又代符彥卿判太原行府事。師還,進位司空、門下侍郎,監修國史。穀以史氏所述本於起居注,喪亂以來遂廢其職,上言請令端明、樞密直學士編記言動,為內廷日歷,以付史官。是歲,河大決齊、鄆,發十數州丁壯塞之,
 命穀領護,刻期就功。



 二年冬,議伐南唐,以穀為淮南道行營前軍都部署,兼知廬、壽等州行府事,忠武軍節度王彥副之,韓令坤以下十二將率從。穀領兵自正陽渡淮,先鋒都將白延遇敗吳軍數千於來遠,又破千餘人於山口鎮,進攻上窯,又敗千餘眾,獲其小校數十人,長圍壽春。南唐遣大將劉彥貞來援,穀召將佐謀曰:「今援軍已過來遠,距壽陽二百里,舟棹將及正陽。我師無水戰之備,萬一斷橋梁,隔絕王師,則腹背受敵矣。不如退
 守浮梁,以待戎輅之至。」初,世宗至圉鎮,已聞此謀,亟走內侍乘驛止之。穀已退保正陽,仍焚芻糧,回軍之際,遞相掠奪,淮北役夫數百悉陷於壽春。世宗聞之怒,亟命李重進率師伐之,以穀判壽州行府。是秋,詔歸闕,得風痺疾,告滿百日,累表請致仕,優詔不允。每軍國大事,令中使就第問之。



 四年春,吳人壁紫金山,築甬道以援壽春,不及者數里。師老無功,時請罷兵為便,世宗令範質、王溥就穀謀之。穀手疏請親征,有必勝之利者三,世宗
 大悅,用其策。及淮南平,賞賜甚厚。出穀疏,令翰林學士承旨陶穀為贊以賜之。是夏,世宗還,穀扶疾見便殿,詔令不拜,命坐御坐側。以抱疾既久,請辭相位。世宗怡然勉之,謂曰:「譬如家有四子,一人有疾,棄而不養,非父之道也。朕君臨萬方,卿處輔相之位,君臣之間,分義斯在,奈何以祿奉為言。」穀愧謝而退。俄以平壽州,敘功加爵邑。是秋,穀抗表乞骸骨,罷相,守司空,加邑封,令每月肩輿一詣便殿,訪以政事。



 五年夏,世宗平淮南回,賜穀錢
 百萬、米麥五百斛、芻粟薪炭等。恭帝即位,加開府儀同三司,進封趙國公。求歸洛邑,賜錢三十萬,從其請。太祖即位,遣使就賜器幣。建隆元年,卒,年五十八。太祖聞之震悼,贈侍中。



 穀為人厚重剛毅,深沉有城府,雅善談論,議政事能近取譬,言多詣理,辭氣明暢,人主為之聳聽。人有難必救,有恩必報。好汲引寒士,多至顯位。與韓熙載善,熙載將南渡,密告穀曰:「若江東相我,我當長驅以定中原。」穀笑曰:「若中原相我,下江南探囊中物耳。」穀後
 果如其言。李昉嘗為穀記室,在淮上被病求先歸。穀視之曰:「子他日官祿當如我。」昉後至宰相、司空。



 周顯德中,扈載以文章馳名,樞密使王樸薦令知制誥。除書未下,樸詣中書言之。穀曰:「斯人薄命,慮不克享耳。」樸曰:「公在衡石之地,當以材進人,何得言命而遺才。」載遂知制誥,遷翰林學士,未幾卒。世謂樸能薦士,穀能知人。穀歸洛中,昭義李筠以穀周朝名相,遺錢五十萬,他物稱是,穀受之。既而筠叛,穀憂恚而終。子吉至補闕,拱至太子中
 允。



 昝居潤,博州高唐人。善書計。後唐長興中,隸樞密院為小吏,以謹願稱。晉初,出掌滑州廩庾,遂補牙職。會景延廣留守西洛,署為右職。延廣卒,居潤往依陜帥白文珂,文珂致仕,乃表薦居潤於周祖。



 時世宗尹京,詔以補府中要職。即位,擢為軍器庫使。從征高平,以功遷客省使,知青州。從向拱西征,為行營都監,秦、鳳平,以居潤為秦州,歷知鳳翔、河中府。顯德三年秋,遷內客省使,代王樸
 知開封府。四年,再幸壽州,命為副留守。十月,幸淮上,以居潤為宣徽北院使兼副留守。五年夏,南征還,復判開封府。六年,征關南,為東京副留守。及吳廷祚出塞河,命居潤權知開封府事。廷祚為樞密使,真判開封府,改左領軍衛上將軍。恭帝嗣位,加檢校太傅。



 太祖立,加檢校太尉。及徵澤、潞,命赴澶州巡警。師還,權知鎮州,加左領軍衛上將軍。建隆二年,又權知澶州。八月,拜義武軍節度,在鎮數年,得風痺,詔還京師。乾德四年,卒,年五十九,
 贈太師。



 居潤性明敏,有節概,篤於行義。初,晉室將亡,景延廣委其族自洛赴難,至則為遼人所執。遼人在洛者遽欲恣摽掠,延廣僚吏部曲悉遁,獨居潤力保護,其家以安。居潤與太祖同事世宗,情好款浹,嘗薦沉倫於太祖,以為純謹可用,後至宰相,世稱其知人。



 子惟質至內園使,弟居濟至水部員外郎。大中祥符三年,錄其孫建中為三班借職。



 竇貞固,字體仁,同州白水人。父專,後唐左諫議大夫。貞
 固幼能屬文,同光中舉進士,補萬全主簿。丁內艱去官,服除,授河東節度推官。時晉祖在藩,以貞固廉介,甚重之。及即位,擢為戶部員外郎、翰林學士,就拜中書舍人。



 天福三年,詔百僚各上封事,貞固疏曰:「臣聞舉善為明,知人則哲。聖君在位,藪澤豈有隱淪;昭代用材,政理固無紊亂。求賢若渴,從諫如流,鄭所以譽子皮;囗囗囗囗,囗囗囗囗,魯所以譏文仲。為國之要,進賢是先。陛下方樹丕基,宜求多士。乞降詔百僚,令各司議定一人,有何能識,堪何職官,朝廷
 依奏用之。若能符薦引,果謂當才,所奏之官,望加獎賞;如乖其舉,或涉徇私,所奏之官,宜加黜罰。自然官由德序,位以才升。三人同行,尚聞擇善;十目所視,必不濫知。臣職在論思,敢陳狂狷。」書奏,帝深嘉之,命所司著為令典。明年,改御史中丞,與太常卿崔梲、刑部侍郎呂琦、禮部侍郎張允同詳定正冬朝會禮節、樂章及二舞行列。歷刑部、門下二侍郎。



 少帝即位,拜工部尚書。遷禮部尚書,知貢舉。舊制,進士夜試,繼以三燭。長興二年改令晝
 試,貞固以晝晷短,難盡士材,奏復夜試。擇士平允,時論稱之。改刑部尚書,出為穎州團練使。歲餘,復拜刑部尚書。



 漢祖入汴,貞固與禮部尚書王松率百官見於滎陽西,漢祖駐駕,勞問久之。初營宗廟,帝以姓自漢出,遂襲國號,尊光武為始祖,並親廟為五。詔群臣議,貞固上言曰:「按《王制》:『天子七廟,諸侯五,大夫三,士一。』《正義》曰:『周之制七廟者,太祖及文王、武王之祧與親廟四也。』又曰:『七廟者,據周也。有其人則七,無其人則五。』至光武中興,及
 魏、晉、宋、齊、隋、唐,或立六廟,或立四廟,蓋建國之始,未盈其數也。《禮》曰『德厚者流光』,此天子可以祀六世之義也。今陛下大定寰區,重興漢祚,旁求典禮,用正宗祧,伏請立高、曾、祖、檷四親廟。及自古聖王祖有功、宗有德、更立始祖在四廟之外,不拘定數,所以或五廟或七廟。今請尊高皇帝、光武皇帝為始祖,法文王、武王不遷之制,用歷代六廟之規,庶合典禮。」漢祖從之。論者以天子建國,各從其所起,堯自唐侯,禹生大夏是也。立廟皆祖其有
 功,商之契,周之後稷,魏之武帝,晉之三廟是也。高祖起於晉陽,而追嗣兩漢,徒以同姓為遠祖,甚非其義;貞固又以四親匹庶,上合高、光,失之彌遠矣。但援立親廟可也,畬皆非禮。俄遷吏部尚書。



 初,帝與貞固同事晉祖,甚相得。時蘇逢吉、蘇禹珪自霸府僚佐驟居相位,思得舊臣冠首,以貞固持重寡言,有時望,乃拜司空、門下侍郎、平章事、弘文館大學士。貞固少時中蠱,若贅在喉中,常鯁閡。及為相日,因大吐,有物狀蜥蜴落銀盤中,毒氣沖
 盤,焚於中衢,臭聞百步外,人皆異之。隱帝即位,加司徒,改本貫永安鄉為賢相鄉,班瑞里為勛貴里。楊邠、史弘肇、王章樹黨恣橫,專權凌上,貞固但端莊自持,不能規救。



 周祖兵起,貞固與蘇逢吉奉隱帝兵次於野,敗。逢吉倉黃自殺,貞固遂詣周祖。周祖稱太后制,委貞固與蘇禹珪、王峻同掌軍國政事。周祖登位,加兼侍中。會以馮道為首相,改監修國史。俄罷相,守司徒,封沂國公。世宗即位,以範質為司徒,貞固遂歸洛陽,輸課役,齒為編民。
 貞固不能堪,訴於留守向拱,拱不聽。



 宋初,以前三公赴闕陪位,詣範質,求任東宮三少,預朝請,質不為奏。乃還洛,放曠山水,與布衣輩攜妓載酒以自適。開寶二年病困,自為墓志,卒,年七十八。



 李濤,字信臣,京兆萬年人。唐敬宗子郇王瑋十世孫。祖鎮,臨濮令。父元,將作監。朱梁革命,元以宗室懼禍,挈濤避地湖南,依馬殷,署濤衡陽令。濤從父兄鬱仕梁為閣門使,上言濤父子旅湖湘,詔殷遣歸京師,補河陽令。



 後
 唐天成初,舉進士甲科,自晉州從事拜監察御史,遷右補闕。宋王從厚鎮鄴,以濤為魏博觀察判官。歲餘,入為起居舍人。



 晉天福初,改考功員外郎、史館修撰。晉祖幸大梁,張從賞以盟津叛,陷洛陽,扼虎牢。故齊王全義子張繼祚者實黨之,晉祖將族其家。濤上疏曰:「全義歷事累朝,頗著功效。當巢、蔡之亂,京師為墟,全義手披荊棘,再造都邑,垂五十年,洛民賴之。乞以全義之故,止罪繼祚妻子。」從之。嘗奉詔為宋州括田使,前雄州刺史袁正
 辭繼束帛遺濤,以田園為托,濤表其事,晉祖嘉之。正辭坐降一階,濤遷浚儀令。改比部郎中、鹽判官,改刑部郎中。



 涇帥張彥澤殺記室張式,奪其妻,式家人詣闕上訴。晉祖以彥澤有軍功,釋其罪。濤伏閣抗疏,請置於法。晉祖召見諭之,濤植笏叩階,聲色俱厲,晉祖怒叱之,濤執笏如初。晉祖曰:「吾與彥澤有誓約,恕其死。」濤厲聲曰:「彥澤私誓,陛下不忍食其言;範延光嘗賜鐵券,今復安在?」晉祖不能答,即拂衣起,濤隨之,諫不已。晉祖不得已,召
 式父鐸、弟守貞、子希範等皆拜以官,罷彥澤節制。濤歸洛下,賦詩自悼,有「三諫不從歸去來」之句。先是,範延光據鄴叛,晉祖賜鐵券許以不死,終亦不免,故濤引之。晉祖崩,濤坐不赴臨,停。未幾,起為洛陽令,遷屯田職方郎中、中書舍人。



 會契丹入汴,彥澤領突騎入京城,恣行殺害,人皆為濤危之。濤詣其帳,通刺謁見。彥澤曰:「舍人懼乎?」濤曰:「今日之懼,亦猶足下昔年之懼也。向使先皇聽僕言,寧有今日之事。」彥澤大笑,命酒對酌,濤神氣自若。



 漢祖起義至洛,濤自汴奉百官表入對,漢祖問京師財賦,從契丹去後所存幾何,濤具對稱旨,漢祖嘉之。至汴,以為翰林學士。杜重威據鄴叛,高祖命高行周、慕容彥超討之,二帥不協。濤密疏請親征。高祖覽奏,以濤堪任宰輔,即拜中書侍郎兼戶部尚書、平章事。



 隱帝即位,楊邠、周祖共掌機密,史弘肇握兵柄,與武德使李鄴等中外爭權,互作威福。濤疏請出邠等藩鎮,以清朝政。隱帝不能決,白於太后,太后召邠等諭之。反為所構,免相歸
 第。時中書廚釜鳴者數四,濤晝寢閣中,夢嚴飾廳事,群吏趨走,雲迎新宰相帶諸司使,既寤,心異之。數日濤罷,以邠為相兼樞密使。及周祖舉兵,太后倉皇涕泣曰:「不用李濤之言,宜其亡也。」



 周初,起為太子賓客,歷刑部、戶部二尚書。世宗宴駕,為山陵副使。恭帝即位,封莒國公。



 宋初,拜兵部尚書。建隆二年,濤被病。有軍校尹勛董浚五丈河,陳留丁壯夜潰,勛擅斬隊長陳琲等十人,丁夫七十人皆杖一百,刵其左耳。濤聞之,力疾草奏,請斬勛
 以謝百姓。家人謂濤曰:「公久病,宜自愛養,朝廷事且置之。」濤憤言曰:「人孰無死,但我為兵部尚書,坐視軍校無辜殺人,烏得不奏?」太祖覽奏嘉之,詔削奪勛官爵,配隸許州。濤卒,年六十四,贈右僕射。



 濤慷慨有大志,以經綸為己任。工為詩,筆札遒媚,性滑稽,善諧謔,亦未嘗忤物,居家以孝友聞。景德三年,其孫惟勤詣闕自陳,詔授許州司士參軍。子承休至尚書水部郎中,承休子仲容。



 濤弟浣,字日新。幼聰敏,慕王、楊、盧、駱為文章。後唐長興
 初,吳越王錢騑卒,詔兵部侍郎楊凝式撰神道碑,令浣代草,凡萬餘言,文彩遒麗,時輩稱之。秦王從榮召至幕中,從榮敗,勒歸田里。久之,起為校書郎、集賢校理。晉天福中,拜右拾遺,俄召為翰林學士。會廢學士院,出為吏部員外郎,遷禮部郎中、知制誥。復置翰林,遷中書舍人,再為學士。時濤在西掖,縉紳榮之。



 契丹入汴,浣與同職徐臺符俱陷塞北。永康王兀欲襲位,置浣宣政殿學士。兀欲死,述律立,以其妻族蕭海貞為幽州節度使。海貞
 與浣相善,浣乘間諷海貞以南歸之計,海貞納之。



 周廣順二年,浣因定州孫方諫密表言契丹衰微之勢,周祖嘉焉,遣諜者田重霸繼詔慰撫,仍命浣通信。浣復表述契丹主幼弱多寵,好擊鞠,大臣離貳,若出師討伐,因與通好,乃其時也,請速行之。屬中原多事,不能用其言。



 浣在契丹嘗逃歸,為其所獲,防御彌謹。契丹應歷十二年六月卒,時建隆三年也。濤收浣文章編之為《丁年集》。浣二子,承確主客郎中,承續職方郎中。



 仲容字儀父,舉進士甲科,除大理評事、知三原縣。累擢監察御史,為殿試進士考官。真宗問題義,對稱旨,詔試中書,擢左司諫、直史館。天聖中,以起居郎為知制誥,累遷右諫議大夫。在西掖八年,次當補學士,而不為宰相張士遜所喜,罷為給事中、集賢院學士、判史館、司農寺,復知制誥。及石中立、張觀補學士,始以為翰林侍讀學士。久之,兼龍圖閣學士,至戶部侍郎卒。



 仲容性醇易,喜飲酒,不與物忤,與人言,未嘗及勢利。三弟早卒,字其諸
 孤十餘人如己子,當世稱其長者。然於吏事非所長。自集制草為《冠鳳集》十二卷。



 王易簡,字國寶,京兆萬年人。性介特寡合。曾祖朏,唐劍州刺史。祖遠,連州刺史。父貫,唐州刺史。易簡少好學,工詩。會僖宗幸蜀,長安兵亂,避地山谷。梁乾化中,邵王友誨鎮陜,易簡舉進士,詣府拔解,友誨贈錢二十萬。明年遂擢第,復隱華山。邠帥韓恭闢觀察支使。府罷,華帥李保衡復闢從事。逾年,尹皓代保衡,易簡仍在幕府。



 會朱
 友謙以河中叛歸莊宗,攻華州甚急,城中危懼,咸請築月城以自固。皓恃勇不聽,下令曰:「有敢復言者斬。」易簡固請,乃許。板築始畢,外城果壞,軍民賴之。會夜不能攻,友謙遂遁去。皓卒,易簡歸田里。久之,召為著作郎,數月棄去。復召為右拾遺,上書忤旨,出為鄧州節度推官。



 後唐同光中,遣魏王繼岌伐蜀,以宰相郭崇韜為招討使,闢易簡為巡官,改魏王都督府記室參軍。明宗即位,周帥羅周恭闢為掌書記。府罷,退居華陰,作《小隱詩》二十
 首並序以見志,好事者多傳誦。秦王從榮聞而重之,謂宰相馮道、李愚曰:「易簡有才,豈宜久居外地。」即召為祠部員外郎,改水部郎中、知制誥,拜中書舍人。



 晉初,賜金紫,判弘文館、史館事。晉祖為治務求速效,易簡上《漸治論》以諫之,詔書褒答,以論付史館。及廢翰林學士,易簡兼知內制,又拜御史中丞,歷右丞、吏部侍郎、左丞、判吏部銓。嘗上言:「選門格敕條件具存,藩府官僚習熟者少,凡給文解,未曉規程,以致選人詣都,親求解樣,往來跋
 涉,重可傷閔,傳寫少差,旋復驗放。乞自今委南曹詳定解樣,兼錄長定格取解條,下諸州,板置州院門,每取解時,準條式遵行。」從之。晉祖在大梁,臺省湫隘,易簡奏舉故事,一歲得元省錢二百萬,繕治省署及造器物,號為舉職。



 周朝諱「簡」,易止名易。廣順初,遷禮部尚書。是冬,合三銓為一,令易簡權判,俄改刑部尚書。周祖將親郊,命判兵部,會冊四廟,命為副使。周祖晏駕,為山陵副使。顯德四年,告老,以太子少保致仕,歸鄉里。



 宋初,召加少傅。
 所居華陰,構一鳴堂、二品樓,優游自適。建隆四年四月,無疾卒,年七十九。子景讓,進士及第,至尚書郎。



 趙上交,涿州範陽人。本名遠,字上交,避漢祖諱,遂以字稱。祖光鄴,鄂州錄事參軍。父簡章,涿州司馬。上交身長七尺,黑色,美風儀,善談論,負才任氣,為鄉里所推。



 後唐同光中,嘗詣中山幹王都。有和少微者亦在都門下,忌之,頗毀訾上交,都遂不為禮。上交不得志,因南游洛陽,與中官驃騎大將軍馬紹宏善。紹宏領北面轉運制置
 大使,表為判官,遷殿中丞。秦王從榮開府兼判軍衛,以上交為虞部員外郎,充六軍諸衛推官。李浣、張沆、魚崇遠皆白衣在秦府,悉與上交友善。累遷司封郎中,充判官。從榮素豪邁,不遵禮法,好暱群小。上交從容言曰:「王位尊崇,當修令德以慰民望,王忍為此,獨不見恭世子、戾太子之事乎?」從榮怒出之,歷涇、秦二鎮州節度判官。從榮及禍,僚屬皆坐斥,上交由是知名。



 晉初,召為左司郎中、度支判官,歷右諫議大夫。會廢翰林學士,以上交
 為中書舍人、知內外制,遷刑部侍郎。嘗上言:「伏睹長興中詔書:『州縣官在任詳讞刑獄、昭雪人命者,不限歲月赴選,許令超資注官,仍賜章服。諸道州府給付公驗,躬赴行部投狀,隨給優牒,庶絕欺罔,以存激勸。』載詳元詔,止言州縣,未該內外職司。乞自今但能雪活冤獄,不限中外官,並加旌賞。諸道州縣委長吏抄案以聞。俟本人考滿,即詣刑部投狀,毋得隔越年歲,庶使內外同律。」詔從之。俄遷戶部侍郎,拜御史中丞,彈舉無所阿避。



 契丹
 入汴,立明宗幼子許王從益為帝,以禮部尚書王崧為左丞相,上交為右丞相。契丹去,上交請去偽號,稱梁王。漢祖將至,從益遣上交馳表獻款,授檢校禮部尚書、太僕卿,遷秘書監。周祖監國,命太師馮道迎湘陰公於徐州,以上交副之。



 廣順初,拜禮部侍郎。會將試貢士,上交申明條制,頗為精密,始復糊名考校。擢扈載甲科,及取梁周翰、董淳之流,時稱得士。轉戶部侍郎。明年再知舉,謗議紛然。時樞密使王峻用事,常薦童子,上交拒之。峻
 怒,奏上交選士失實,貶商州司馬,朝議以為太重,會峻貶乃止,但坐所取士李觀、侯璨賦落韻,改太子詹事。



 顯德初,遷賓客。二年,拜吏部侍郎,多請告不朝,時出游別墅。世宗因問陶穀曰:「上交豈衰老乎?」穀對曰:「上交昔掌貢舉,放鬻市家子李觀及第,受所獻名園,多植花卉,優游自適。」世宗怒,免其官。



 宋初,起為尚書右丞。建隆二年正月,卒,年六十七。上交所蒞官以干聞,當時稱有公輔器。尤好吟詠,有集二十卷,張昭為序。



 子曮,字可畏。七歲
 喪母,過哀。十二能屬文,與兄晙同舉進士,未成名而兄夭,遂以蔭補千牛備身,歷秘書郎、殿中丞、著作郎。卒,年二十六。有集十卷,太宗嘗取以入內。



 張錫,福州閩縣人。梁末,劉君鐸任棣州刺史,闢為軍事判官。棣為鄆之屬郡,郡有曲務,鄆以牙將主之,頗橫恣,民有犯曲三斤,牙將欲置於死,君鐸力不能救。既而牙將盜麥百斛私造曲,事覺,錫判曰:「曲犯三斤,求生不克,麥盜百斛,免死誠難。」時郡吏以使府牙將乞免,錫不允,
 固置於法。



 同光末,趙在禮舉兵於鄴,瀕河諸州多構亂,錫權知州事,即出省錢賞軍,皆大悅,一郡獨全,棣人賴之。後為淄川令,不畏強御,專務愛民,刺史有所征,不答,由是銜之。及代,白其事於宰相馮道。道知錫介直,即奏召為監察御史,出為陜、虢觀察判官。晉開運二年,拜右補闕,歷起居郎、刑部員外郎、開封府判官、浚儀令、司門駕部二郎中,並以清節聞。周顯德中,以老疾求解官,授右諫議大夫致政。



 宋初,改給事中。錫無子,宰相範質嘗
 兄事之,館於別墅。錫以執政之門,不欲久處,往依鄉人鄧州觀察判官黃保緒。建隆二年六月,卒於穰下。



 張鑄,字司化,河南洛陽人。性清介,不事生產。曾祖居卿,祖裼,父文蔚,在唐俱舉進士。裼至翰林學士承旨、天平軍節度、檢校吏部尚書。文蔚,中書侍郎、平章事,《五代史》有傳。



 鑄,梁貞明三年舉進士,補福昌衛、集賢校理,拜監察御史,遷殿侍御史。仕後唐,歷起居郎、金部員外郎,賜緋,改右司員外郎。



 明宗初,轉金部郎中,賜金紫。嘗上言
 曰:「國家以務農為本,守令以勸課為先,廣闢田疇,用實倉廩。竊見所在鄉村浮戶,方事墾闢,甫成生計,種田未至二頃,植木未及十年,縣司以定色役,民畏責斂,舍之而去,殊乖撫恤之方,徒設招攜之令。望令諸州應有荒田縱民墾蒔,俟及五頃已上,三年外始聽差科。」從之。使兩浙還,遷考功郎中。



 晉天福初,福州王延義奉表稱藩,遣鑄持節冊為閩國王。少帝即位,改河南令。開運二年,召為太常少卿,避曾祖諱不拜,改秘書少監、判太常寺
 事。逾年,轉右庶子,分司西京。周廣順初,入為左諫議大夫、給事中,使朗州。顯德三年,授檢校禮部尚書、光祿卿,又以祖名請避,改秘書監、判光祿寺。宋初,加檢校刑部尚書。建隆四年,卒,年七十二。



 鑄美姿儀,善筆札,老能燈下細書如蠅頭。由晉以來,天地、宗廟及上徽號、封拜王公冊文,皆詔鑄書之。及卒,身無兼衣,家人鬻其服馬、園圃,得錢十萬以葬。



 邊歸讜,字安正,幽州薊人,父退思,檀州刺史。歸讜弱冠
 以儒學名。後唐末,客游並、邠。晉祖鎮太原,召置門下,表為河東節度推官、試秘書省校書郎,改太原府推官、試大理評事。



 天福初,拜監察御史。歷殿中侍御史、禮部員外郎,充戶部判官。遷水部郎中,賜金紫,拜比部郎中、知制誥。歷右諫議大夫、給事中。嘗上言:「使臣經過州縣,券料外妄自征需,以豐傔從,多索人驢,用遞行李。挾命為勢,凌下作威,供億稍遲,即加鞭棰,吏民受辱,寧免怨嗟。欲望察訪得情,嚴示懲戒。」從之。俄遷右散騎常侍。



 漢初,
 歷禮部、刑部二侍郎。時史弘肇怙權專殺,閭里告訐成風。歸讜言曰:「邇來有匿名書及言風聞事,構害善良,有傷風化,遂使貪吏得以報復私怨,讒夫得以肆其虛誕。請明行條制,禁遏誣罔。凡顯有披論,具陳姓名。其匿名書及風聞事者並望止絕。」論者韙之。



 周廣順初,遷兵部、戶部二侍郎。世宗聞其亮直,擢為尚書右丞、樞密直學士,以備顧問。就轉左丞,世宗以累朝以來憲綱不振,命為御史中丞。



 歸讜雖號廉直,而性剛介,言多忤物。顯德
 三年冬,大宴廣德殿,歸讜酒酣,揚袂言曰:「至於一杯而已。」世宗命黃門扶出之。歸讜回顧曰:「陛下何不決殺趙守微。」守微者,本村民,因獻策擢拾遺,有妻復娶,又言涉指斥,坐決杖配流,故歸讜語及之。翌日,伏閣請罪,詔釋之,仍於閣門復飲數爵,以愧其心。五年秋,歸讜與百官班廣德殿門外,忽厲聲聞於帝,詔奪一季奉。



 宋初,遷刑部尚書。建隆三年,告老,拜戶部尚書致仕。乾德二年,卒,年五十七。子定,雍熙二年進士及第。



 劉溫叟,字永齡,河南洛陽人。性重厚方正,動遵禮法。唐武德功臣政會之後。叔祖崇望,相昭宗。父岳,後唐太常卿。溫叟七歲能屬文,善楷隸。岳時退居洛中,語家人曰:「吾兒風骨秀異,所未知者壽耳。今世難未息,得與老夫皆為溫、洛之叟足矣。」故名之溫叟。以蔭補國子四門助教,河南府文學。清泰中,為左拾遺、內供奉。以母老乞歸就養,改監察御史,分司。時臺署廢弛,溫叟作新之。未幾,召為右補闕。



 晉初,王松權知青州,表為判官,加朝散階。
 入為主客員外郎。少帝領開封尹,奏為巡官,命典文翰,又改廣晉府巡官。少帝即位,拜刑部郎中,賜金紫。改都官郎中,充翰林學士。初,岳仕後唐,嘗居內署,至是溫叟復居斯任,時人榮之。溫叟既受命,歸為母壽,候立堂下。須臾聞樂聲,兩青衣舉箱出庭,奉紫袍、兼衣,母命卷簾見溫叟曰:「此即爾父在禁中日內庫所賜者。」溫叟拜受泣下,退開影堂列祭,以文告之。母感愴累日,不忍見溫叟。歲滿,加知制誥。



 契丹入汴,溫叟懼,隨契丹北遷,與承
 旨張允共上表求解職。契丹主怒,欲出允等為縣令。趙延壽曰:「若學士才不稱職求解者,守本官可也,不可加貶出。」遂得罷職出院。漢祖南下,溫叟自洛從至鄭州,稱疾不行。及入汴,溫叟久之方至,授駕部郎中。



 周初,拜左諫議大夫,逾年,改中書舍人,加史館修撰,判館事。顯德初,遷禮部侍郎、知貢舉,得進士十六人。有譖於帝者,帝怒,黜十二人,左遷太子詹事。溫叟實無私,後數年,其被黜者相繼登第。溫叟與張昭同修漢隱帝及周祖實錄,
 恭帝即位,遷工部侍郎兼判國子祭酒事。



 宋初,改刑部。建隆九年,拜御史中丞。丁內艱,退居西洛,旋復本官。三年,兼判吏部銓。因上言曰:「伏見兩京百司,漸乏舊人,多隳故事。雖檢閱具存於往冊,而舉行須在於攸司。蓋因年限得官,歸司者例與減選;冬集赴調,授任者尋又出京。兼有裁滿初官,不還舊局,但稱前資,用圖免役。又有嘗因停任,切欲歸司,而元敕不該,無由復職。遂使在司者失於教習,歷事者難於追還。伏望自今諸司職掌,除
 官勒留及歸司者,如理減外欠三選以下,仍須在司執行公事,及三十月即許赴集;如理選外欠三選以上,及在官不成資考者,即準元敕處分。若在任停官及在司停職者,經恩後於刑部出給雪牒,卻勒歸司,如無闕員,即令守闕,餘依敕格處分。



 一日晚歸由闕前,太祖方與中黃門數人偶登明德門西闕,前騶者潛知之,以白溫叟。溫叟令傳呼如常過闕。翌日請對,具言:「人主非時登樓,則近制咸望恩宥,輦下諸軍亦希賞給。臣所以呵導
 而過者,欲示眾以陛下非時不登樓也。」太祖善之。憲府舊例,月賞公用茶,中丞受錢一萬,公用不足則以贓罰物充。溫叟惡其名不取。任臺丞十二年,屢求代。太祖難其人,不允。開寶四年被疾,太祖知其貧,就賜器幣,數月卒,年六十三。



 溫叟事繼母以孝聞,雖盛暑非冠帶不敢見。五代以來,言執禮者惟溫叟焉。立朝有德望,精賞鑒,門生中尤器楊徽之、趙鄰幾,後皆為名士。範杲幼時,嘗以文贄溫叟,大加稱獎,以女妻之。



 太宗在晉邸,聞其清
 介,遣吏遺錢五百千,溫叟受之,貯廳西舍中,令府吏封署而去。明年重午,又送角黍、執扇,所遣吏即送錢者,視西舍封識宛然,還以白太宗。太宗曰:「我錢尚不用,況他人乎?昔日納之,是不欲拒我也;今周歲不啟封,其苦節愈見。」命吏輦歸邸。是秋,太宗侍宴後苑,因論當世名節士,具道溫叟前事,太祖再三賞嘆。



 雍熙初,子照罷徐州觀察推官待選,以貧詣登聞求注官。及引對,太宗問誰氏子,照以溫叟對。太宗愀然,召宰相語其事,且言當今
 大臣罕有其比。因問:「照當得何官?」宰相言:「免選以為厚恩。」帝曰:「其父有清操,錄其子登朝,庶足示勸。」擢照太子右贊善大夫,歷判三司理欠、憑由司,江南轉運司,入朝為司封郎中。炳、燁並進士及第。



 燁字耀卿,進士及第。積官秘書省著作郎。知龍門縣,群盜殺人,燁捕得之,將械送府,恐道亡去,皆斬之。眾服其果。通判益州,召還,時王曙治蜀,或言其政苛暴。真宗問:「曙治狀與凌策孰愈?」燁曰:「策在蜀,歲豐事簡,故得以寬
 假民。比歲小歉,盜賊竊發,非誅殺不能禁。然曙所行,亦未嘗出陛下法外。」帝善之。



 天禧元年,始置監官。帝謂宰相曰:「諫官御史,當識朝廷大體。」於是以燁為右正言。會歲薦饑,河決滑州,大興力役,饑殍相望。燁請策免宰相,以應天變。都城東南有泉出,民爭傳可以已疾,詔即其地建祥源觀。燁言其詭妄不經,且亢旱,不可興土木以營不急;又請罷提點刑獄,禁民棄父母事佛老者。皆不報。



 表請補外,帝以燁屢言事,乃以判三司戶部勾院,出
 安撫京西。還,直集賢院,同修起居注,遷右司諫。以尚書工部員外郎兼侍御史知雜事,權判吏部流內銓。請京朝官遭父母憂,官司毋得奏留,故事當起復者如舊。因詔益、梓、利、夔路長吏,仍舊奏裁,餘乞免持服者論其罪。改三司戶部副使,擢龍圖閣待制,提舉諸司庫務,權發遣開封府事。累遷刑部郎中、龍圖閣直學士、知河南府,徙河中府,卒。



 初,王曙坐寇準貶官,在朝無敢往見者。燁嘆曰:「友朋之義,獨不行於今歟?」往餞之,經宿而還。嘗善
 河中處士李瀆,瀆死,為陳其高行,詔以著作郎贈之。



 唐末五代亂,衣冠舊族多離去鄉里,或爵命中絕而世系無所考。惟劉氏自十二代祖北齊中書侍郎環俊以下,仕者相繼,而世牒具存焉。子幾。



 幾字伯壽,以燁任為將作監主簿。生而豪俊,長折節讀書,第進士。



 從範仲淹闢,通判邠州。邠地鹵,民病遠汲,幾浚渠引水注城中。役興,客曰:「自郭汾陽城此州,茍外水可釃,何待今日?無為虛費勞人也!」幾不答。未幾,水果至,
 鑿五池於通逵,民大便利。



 孫沔薦其才堪將帥,換如京使、知寧州。俗喜巫,軍校仗妖法結其徒,亂有日。幾使他兵伏壘門以伺,夜半盡禽之。加本路兵馬鈐轄、知邠州。



 儂智高犯嶺南,幾上書願自效,以為廣東、西捉殺。道聞蔣偕、張忠戰沒,疾馳至長沙,見狄青曰:「賊若退守巢穴,瘴毒方興,當班師以俟再舉。若恃勝求戰,此成擒耳。」賊果悉眾來,大戰於歸仁鋪。前鋒孫節死,幾以右軍搏斗,自辰至巳,勝負未決。幾言於青,出勁騎五千,張左右翼
 搗其中堅,賊駭潰。



 進皇城使、知涇州。陛見,辭以母老,丐覆文階歸養。仁宗諭之曰:「涇,內地也,將母莫便焉。」命特賜冠帔。領循州刺史,遷西上閣門使,再歸郎中班。曾公亮薦之,復以嘉州團練使為太原、涇原路總管。



 夏人寇周家堡,轉運使陳述古攝渭帥,幾移文索援兵,不聽,率諸將偕請,又不聽,乃趣以手書。述古怒,移幾為鳳翔,且劾生事。朝廷以總管非轉運使所得徙置,遣御史出按,述古黜,幾亦改鄜州。召判三班院。邊吏告夏人趨大順,
 英宗問幾。幾曰:「大順天險,非夏人可得近,正恐與趙明為仇爾。」帝曰:「明之子奔馬入城,幾為所掩,卿料敵一何神也。」以為秦鳳總管。



 神宗即位,轉四方館使、知保州,治狀為河北第一。逾六年,即請老,還為秘書監致仕。元豐三年,祀明堂,大臣言幾知音,詔詣太常定雅樂。幾曰:「古樂備四清聲,沿五季亂離廢,請增之。」樂成,予一子官。



 幾得謝二十年,放曠嵩、少間,遇唐末異人靖長官者得養生訣,故益老不衰。間與人語邊事,謂張耒曰:「比見詔書
 禁邊吏夜飲。此曹一旦有急,將使輸其肝腦,此平日禁其為樂,為今役者不亦難乎?夫椎牛釃酒,豐犒而休養之,非欲以醉飽為德,所以增士氣也。」耒敬識其語。再加通議大夫,卒,年八十一。



 幾篤於風義,推父遺恩官從兄,已得任子,必先兄弟子之孤者。其議樂律最善,以為:「律主於人聲,不以尺度求合。古今異時,聲亦隨變,猶以古冠服加於今人,安得而稱。儒者泥古,致詳於形名度數間,而不知清濁輕重之用,故求於器雖合,而考於聲則
 不諧。」嘗游佛寺,聞鐘聲,曰:「聲澌而悲,主者且不利。」是夕,主僧死。在保州,聞角聲,曰:「宮微而商離,至秋,守臣憂之。」及期,幾遇疾。然所學頗雜鄭、衛雲。



 劉濤,字德潤,徐州彭城人,後唐天成中,舉進士,釋褐為鳳翔掌書記,拜右拾遺,賜緋。時太常丞史在德上章,詞理鄙俗,仍犯廟諱。濤上言請正其罪,雖不允,時論是之。出為山南東道節度判官,召為左補闕,遷起居舍人。



 晉天福初,改司勛員外郎、史館修撰,遷工部郎中,賜金紫。
 歷度支、職方二郎中,掌左藏庫。時少帝奢侈,常以銀易金,廣其器皿。李崧判三司,令上庫金之數。及崧以元簿較之,少數千鎰。崧責曰:「帑庫通式,一曰不受虛數,毫厘則有重典。」濤曰:「帑司常有報不盡數,以備宣索。」崧令有司劾濤,濤事迫,以情告樞密使桑維翰,乃止罰一月奉。漢初,宰相蘇禹珪薦為中書舍人。



 周廣順中,坐令子監察御史頊代草誥命,左遷少府少監,分司西京;頊亦貶復州司戶。顯德初,就改太常少卿,俄拜右諫議大夫。四
 年,再知貢舉。樞密使王樸嘗薦童子劉譜於濤,濤不納,樸銜之。時世宗南征在迎鑾,濤引新及第人赴行在。樸時留守上都,飛章言濤取士不精。世宗命翰林學士李昉覆試,黜者七人。濤坐責授太子右贊善大夫。恭帝即位,遷右詹事。濤性剛毅不撓,素與宰相範質不協,常鬱鬱不得志,遂退居洛陽之清化里,杜門以書史自娛。



 太祖素知濤履行,開寶二年召赴闕,以老病求退,授秘書監致仕。年七十二卒。



 清泰初,中書舍人盧導受詔主文,
 將鎖宿,濤力薦薛居正,以為文章器業必至臺輔,導取之,後果為相。世稱其知人。



 頊子晟,晟子訥、譚,並進士及第。晟至屯田員外郎,訥為殿中侍御史。



 邊光範、字子儀,並州陽曲人。性謙退和雅,有吏材。父仁嗣,忠武軍節度副使。光範,後唐天成二年,起家榆次令,召為殿中丞,賜緋。長興四年,改太常丞。丁內艱。晉天福初,服闋,授檢校戶部員外郎、北京留守判官兼侍御史。二年,拜太府少卿。上書曰:「臣聞唐太宗有言:『朕居深宮
 之中,視聽不能及遠,所委者惟都督、刺史。』則知此官實系治亂,必須得人。今則刺史或因緣世祿,或貢奉家財,或微立軍功,或但循官序。實恐撫民無術,御吏無方,以此牧民,而民受其賜鮮矣。望選能吏以蘇民瘼,用致升平。」奏入,留中不出。俄為冊秦王李從□嚴副使。張從恩以外戚為河南尹,奏授判官。遷秘書監兼御史中丞,入拜大理少卿。



 少帝尹京,改衛尉少卿,充開封府判官,又改光祿少卿,廣晉府判官,賜金紫。少帝即位,拜右諫議大
 夫,權知開封府事,遷給事中。會蝗災,遣使亳州括借軍糧,稱為平允。時與契丹失歡,河朔連兵,命光範出使修好。會契丹復南入,光範行至趙州,召還。開運元年,權知鄭州,拜左散騎常侍。二年,入為樞密直學士。少帝以光範藩邸舊僚,待遇尤厚。因游宴,見光範位翰林學士下,即日拜尚書禮部侍郎、知制誥,充翰林學士,仍直樞密院。



 漢初,改檢校刑部尚書、衛尉卿。上言:「伏見朝廷除刺史,不限年月,或未及期年,又聞除代。往來跋涉,豈暇撫
 懷。望慎選良牧,立定年限,以責輯綏之效。」疏入,不報。乾祐二年,連使宋州虞城、汝州襄城,按視民田之傷稼者。是冬,為吳越加恩使。



 周廣順初,出知陳州,遷秘書監,俄召拜御史中丞,賜襲衣、銀器、繒彩、鞍勒馬,復為禮部侍郎。時禮部侍郎於貢部或掌或否,光範拜官,將及秋試,乃言於執政曰:「單門偶進,何言名第。若他曹公事,光範不敢辭;若處文衡,校閱名賢,品藻優劣,非下走所能。」執政曰:「公晉末為翰林、樞密直學士,勿避事也。」及期,光節
 辭疾不出,乃以翰林學士承旨徐臺符掌之,時論多其自知。



 世宗即位,改刑部侍郎、權知開封府,俄遷戶部。顯德三年,命往大名檢民田。五年,遣使普均租稅,光範詣宋州。時韓通掌禁兵,領宋師修汴堤,訪郡民,皆言光範均平之狀,乃具以聞,世宗嘉之。



 宋初,徵澤、潞,命光範為前軍轉運,計度鄭、洛、汝、孟、懷芻糧。秋,拜太常卿。時張昭為吏部尚書,朝議以其耆老,令光範簽判選事。



 建隆四年,襄州節度慕容延釗征湖南,以光範權知州事,路當
 沖會,餉饋無闕。是冬郊祀,召還。會延釗卒,復知襄州。大軍數萬由陜路討蜀,出漢上,光範復當供億,人不知勞。嘗舉本鎮判官李楫為殿中侍御史,後坐事除籍,光範左遷太子賓客,仍知襄州。



 五年,兼橋道使,朝廷遣使督治道,常六七輩,一使所調發民皆數百人,吏緣為奸,多私取民課,所發不充數,而道益不修。光範計其工,以州卒代民,官給器用,役不淹久,人以無擾。詔書褒美。開寶四年,復判吏部銓曹。御史中丞劉溫叟卒,以光範判
 御史臺事,數月,真拜中丞。六年,以疾解銓曹任。卒,年七十三。



 光範性至孝,謙退和易,雅有吏乾。母病疽,光範嘗吮之。景德中,錄其孫易從同學究出身。



 劉載,字德輿,涿州範陽人。唐盧龍節度濟之六世孫。父昭,下蔡令。載,後唐清泰中舉進士。晉初,解褐校書郎,遷著作佐郎,賜緋,拜左拾遺、集賢殿直學士。漢初,為殿中侍御史,丁內艱,服闋,復拜舊官。判西京留臺,改倉部員外郎。嘗著五論,曰《為君》、《為相》、《為將》、《去讒》、《納諫》,頗為文士
 所稱。



 周世宗初,擢知制誥。顯德三年,拜右諫議大夫,與右拾遺鄭起、尚書博士李寧同校道書。遷給事中,使許州定田租。俄賜金紫,為魏王符彥卿加恩國信使。



 宋初,浚五丈河,自陳橋達曹州之西境,命護其役。建隆四年,貝州節度使張光翰來朝,遣載權知州事。光翰歸鎮,載還,知貢舉。乾德初,掌建安榷貨務。六年,就為江南國主生辰使,召還,令知鎮州。



 開寶四年,坐與何繼筠不協,改山南東道行軍司馬。十年不召,嘗受詔權點檢州事。太
 平興國初,復入為給事中。三年,出知襄州,六年,代還。告老,改工部侍郎致仕,乃賜一子出身。八年,卒,年七十一。



 載尤好學,博通史傳,善屬文。嘗受詔撰明憲皇后謚冊文,又作《吊戰國賦》萬餘言行於世。雅信釋典,敦尚名節。



 子宗言,至比部郎中。宗望,景德二年進士及第。大中祥符四年,其孫介以載文集來獻,以為試將作主簿。



 程羽,字沖遠,深州陸澤人。少好學,能屬文。晉天福中,擢進士第,授陽穀主簿。歷虞鄉、醴泉、新都令,皆有政績。開
 寶中,選為兩使判官,入對,太祖詢以時事,敷奏稱旨,擢著作郎,出知興州。逾年,改知興元府。囗囗囗囗。八年,詔歸闕,以本官領開封府判官。



 羽性淳厚,蒞事恪謹。時太宗尹京,頗以長者待之。及即位,拜給事中,知開封府。未幾,出知成都府,為政寬簡,蜀人便之。入朝,拜禮部侍郎。上欲優以清職。故事,端明殿設學士二員,居翰林學士上,專備顧問,馮道、趙鳳始居是職,累朝因之。及是,即殿名以羽為文明殿學士,位在樞密副使下,且即泰寧坊
 營第以賜之。



 太平興國五年,典試貢士,御試得人居多。六年,以老疾求解職,拜兵部侍郎,未幾致仕,仍給全奉。雍熙元年,卒,年七十二。贈禮部尚書。



 子希振,以蔭至尚書虞部員外郎。大中祥符元年卒。其子適,賜同學究出身。從孫琳,別傳。



 論曰:五季為國,不四、三傳輒易姓,其臣子視事君猶傭者焉,主易則他役,習以為常。故唐方滅即北面於晉,漢甫稱禪已相率下拜於周矣。君子傷之,此《雜臣傳》所繇
 立也。李穀、邊歸讜、竇貞固、李濤輩,或在廟堂,或侍帷幄,世主之所寵任,社稷之所倚賴,而更事異姓,不能以名節生死,倫義廢矣。且穀以籌策自名,乃不能料藝祖有容人之量,及受李筠饋遺,懼其見殺,遂以憂死,又何繆耶?嗚呼,魏範粲、齊顏見遠,宜見褒於前史也。



\end{pinyinscope}