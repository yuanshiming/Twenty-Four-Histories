\article{志第六十一 禮十一(吉禮十一)}

\begin{pinyinscope}

 時享
 薦新加上祖宗謚號廟諱



 時享。太祖乾德六年十月,判太常寺和峴上言:「按《禮閣新儀》,唐天寶五年,詔享太廟宜祭料外,每室加常食一
 牙盤。將來享廟,欲每室加牙盤食,禘祫、時享亦準此制。」



 太宗太平興國六年十二月,太常禮院言:「今月二十三日,臘享太廟。緣孟冬已行時享,冬至又嘗親祀。按禮每歲五享,其禘祫之月即不行時享,慮成煩數,有爽恭虔。今請罷臘日薦享之禮,其孝惠別廟即如式。」從之。



 淳化三年十月八日,太常禮院言:「今年冬至,親祀南郊,前期朝享太廟,及奏告宣祖、太祖室。常例,每遇親祀,設朔、望兩祭,乃是十一月內三祭,太廟兩室又行奏告之禮,煩
 則不恭。又十一月二十日,皇帝朝享,去臘享日月相隔,未為煩數。欲望權停是月朔、望之祭,其臘享如常儀。」從之。



 真宗景德三年正月,畫日乙卯孟享太廟。其日以鄆王外攢,改用辛酉。十月十日,孟冬薦享。其月,明德皇后園陵,有司言:「故事,大祠與國忌日同日者,其樂備而不作,今請如例。」從之。四年七月,以莊穆皇后祔享,權停孟享。



 大中祥符三年十二月,帝謂王旦等言:「來年正月十一日孟享太廟,而有司擇八日宴,已在享廟致齋中。又
 七日上辛,祀昊天上帝。」王欽若言:「若移宴日避祀事,即自天慶節以來皆有所妨。」馮拯言:「上辛不可移,薦享宗廟是有司擇日,於禮無嫌。」帝曰:「當詢禮官。」終以契丹使發有常期,又將西巡,故不及改。



 八年,兼宗正卿趙安仁言:「準詔以太廟朔望上食品味,令臣詳定。望自今委御廚取親享廟日所上牙盤例,參以四時珍膳,選上局食手十人,赴廟饌造,上副聖心,式表精愨。」詔:所上食味,委宮闈令監造訖,安仁省視之。



 神宗元豐三年十月,詳定
 郊廟奉祀禮文所言:「祠禴嘗蒸之名,春夏則物未成而祭薄,秋冬則物成而禮備。今太廟四時雖有薦新,而孟享禮料無祠禴蒸嘗之別。伏請春加韭、卵,夏加麥、魚,秋加黍、豚,冬加稻、雁,當饋熟之節,薦於神主。其籩豆於常數之外,別加時物之薦,豐約各因其時,以應古禮。」從之。



 六年十一月,帝親祠南郊。前期三日,奉仁宗、英宗徽號冊寶於太廟。是日,齋於大慶殿。翌日,薦享於景靈宮。禮畢,帝服通天冠、絳紗袍,乘玉輅至太廟,宰臣、百官班迎
 於廟門。侍中跪請降輅,帝卻乘輿,步入廟,趍至齋宮。翌日,帝服靴袍至大次。有司奏中嚴、外辦,禮儀使跪奏請行事。帝服袞冕以出,至東門外,殿中監進大圭,帝執以入,宮架樂作,升東階,樂止。登歌樂作,至位,樂止。太祝、宮闈令奉諸室神主於坐,禮儀使贊曰:「有司謹具,請行事。」帝再拜,詣罍洗,登歌樂作,降階,樂止。宮架樂作,至洗南,北向,樂止。帝搢圭,盥帨,洗瓚、拭瓚訖,執圭。宮架樂作,升堂,樂止。登歌樂作,殿中監進鎮圭。帝搢大圭,執鎮圭,詣
 僖祖室,樂止。登歌奏《瑞安》之曲。至神坐前,北向跪,奠鎮圭於繅藉,執大圭跪,三上香,執瓚裸地,奠瓚,奉幣。奠訖,執圭,俯伏,興,出戶外,北向再拜。內侍舉鎮圭以授殿中監。至次室行事,皆如前儀。帝還位,登歌樂作,至位,樂止。宮架《興安》之樂作,文舞九成,止。禮部、戶部尚書以次官奉逐室俎豆,宮架《豐安》樂作,奠訖,樂止。帝再詣罍洗,登歌樂作,降階,樂止。宮架樂作,至洗南,北向立,樂止。帝搢圭,盥帨,洗爵、拭爵訖,執圭。宮架樂作,帝升東階,樂止。登
 歌樂作,至僖祖室,樂止。宮架樂作,帝搢圭跪,受爵,祭酒,三奠爵,執圭,俯伏,興,出戶外,北向立,樂止。太祝讀冊文,帝再拜。詣次室,皆如前儀。帝還位,登歌樂作,至位,樂止。文舞退,武舞進,宮架《正安》之樂作,亞獻以次行事如前儀,樂止。帝詣飲福位,登歌樂作,至位,樂止。宮架《僖安》樂作,帝再拜,搢圭跪,受爵,祭酒,三啐酒,奠爵,受俎,奠俎,受摶黍,奠黍豆,再受爵,飲福酒訖,奠爵,執圭,俯伏,興,再拜,樂止。帝還位,登歌樂作,至位,樂止。太常博士遍祭七祀、
 配享功臣。戶部、禮部尚書徹俎豆,登歌《豐安》樂作,徹訖,樂止。禮直官曰「賜胙」,行事、陪祠官皆再拜,宮架《興安》樂作,一成,止。太祝、宮闈令奉神主入諸祏室。禮儀使跪奏禮畢,登歌樂作,帝降階,樂止。宮架樂作,出東門,殿中監受大圭,歸大次,樂止。有司奏解嚴,轉仗赴南郊。



 初,國朝親享太廟,儀物有制。熙寧以來,率循舊典,元豐命官詳定,始多損益。元年,詳定郊廟禮文所言:「古者納牲之時,王親執鸞刀,啟其毛,而祝以血毛詔於室。今請改正儀
 注,諸太祝以毛血薦於神坐訖,徹之而退。唐崔沔議曰:『毛血盛於盤。』《開元》、《開寶通禮》及今儀注皆盛以豆。禮以豆盛菹醢,其薦毛血當盛以盤。」又言:「三牲骨體俎外,當加牛羊腸胃、豕膚俎各一。又古者祭祀無迎神、送神之禮,其初祭及末,皆不當拜。又宜令戶部陳歲貢以充庭實,如古禮,仍以龜為前,金次之,玉帛又次之,餘居後。又《周禮》大宗伯之職,凡享,蒞玉鬯。今以門下侍郎取瓚進皇帝,侍中酌鬯進瓚,皆未合禮。請命禮部尚書奉瓚臨
 鬯,禮部侍郎奉盤,以次進,皇帝酌鬯裸地訖,侍郎受瓚並盤而退。」又言:「皇帝至阼階,乃令太祝、宮闈令始奉神主置於坐,行禮畢,皇帝俟納神主,然後降階。」並從之。



 又言:「神坐當陳於室之奧東面。當行事時,皇帝立於戶內西向,即拜於戶內。有司攝事,晨稞饋食,亦立於戶內西向,更不出戶而拜。其堂上薦腥,則設神坐於扆前南向,皇帝立於中堂北向。有司攝事同此。」詔俟廟制成取旨。



 又請:「諸廟各設莞筵紛純,加繅席畫純,於戶內之東西
 面,皇帝行三獻禮畢,於此受嘏。」又言:「每室所用幾席,當如《周禮》,改用莞筵紛純,加繅席畫純,加次席黼純,左右玉幾。凡祭祀,皆繅次各加一重,並莞筵一重為五重。」又言:「古者宗廟九獻,皇及後各四,諸臣一。自漢以來為三獻,後無入廟之事,沿襲至今。若時享則有事於室,而無事於堂;禘祫則有事於堂,而無事於室。室中神位不在奧,堂上神位不當扆,有饋食而無朝踐。度今之宜,以備古九獻之意,請室中設神位於奧東面,堂上設神位於
 戶外之西南面,皇帝立於戶內西南,稞鬯為一獻;出戶立於扆前,北向,行朝踐薦腥之禮為再獻;皇帝立於戶內西面,行饋食薦熟之禮為三獻。」詔並候廟制成取旨。



 又請:「三年親祠,並祫享及有司攝事,每室並用太牢及制幣。宗廟堂上□□蕭以求陽,而有司行事□□茅香,宜易用蕭。灌鬯於地以求陰,宜束茅沃酒以象神之飲。凡幣皆埋於西階東,冊則藏有司之匱。」又請:「除去殿下板位及小次,而設皇帝板位於東階之上,西向。」又請:「凡奏告、
 祈禱、報謝,用牲牢祭饌,並出帝後神主,以明天地一體之義。又古者祭祀,兼薦上古、中古及當世之食,唐天寶中,始詔薦享每室加常食一牙盤,議者以為宴私之饌可薦寢宮,而不可瀆於太廟,宜罷之。古者吉祭必以其妃配,不特拜,請奠副爵無特拜。《儀禮》曰:『嗣舉奠。』請皇帝祭太廟,既稞之後,太祝以斝酌奠於鉶之南,俟正祭嘏訖,命皇子舉奠而飲。」



 又請:「命刑部尚書一員以奉大牲,兵部尚書一員奉魚十有五。仍令腥熟之薦,朝享、四孟及
 臘享,皆設神位於戶內南向。其稞將於室,朝踐於堂,饋熟於室,則於奧設莞筵紛純,加繅席畫純,加次席黼純,左右玉幾。當筵前,設饋食之豆八,加豆八,以南為上。鉶三,設於豆之南。南陳牛鉶居北,羊鉶在牛鉶之南,豕鉶在羊鉶之南。羞豆二,曰酏食、糝食,設於薦豆之北。大羹湆盛以登,設於羞豆之北。九俎設於豆之東,三三為列,以南為上。肵俎一,當臘俎之北,縱設之。牲首俎在北牖下,簠簋設於俎南,西上。籩十有八,設於簠簋之南,北上。
 戶外之東設尊彞,西上,南肆。胙階之東設六罍,其三在西,以盛玄酒,其三在東,以盛三酒。堂下陳鼎之位,在東序之南,居洗之西,皆西面北上。匕皆加於鼎之東,俎皆設於鼎之西,西肆。肵俎在北,亦西肆。若廟門外,則陳鼎於東方,各當其鑊,而在其鑊之西,皆北面北上。」



 又請:「既晨稞,諸太祝入,以血毛奠神坐。太官令取肝,以鸞刀制之,洗於鬱鬯,貫以膋,燎於爐炭。祝以肝膋入,詔神於室,又出以隋祭於戶外之左,三祭於茅菹。當饋熟之時,祝
 取菹擩於醢,祭於神坐前,豆間三。又取黍稷肺祭,祭如初,藉以白茅。既祭,宮闈令束而瘞之於西階東。若郊祀天地,則當進熟之時,祝取菹及黍稷肺,祭於正配神坐前,各三祭,畢,郊社令束茅菹而燔瘞之。祀天燔,祭地瘞,縮酒之茅,或燔或瘞,當與隋祭之菹同。」又言:「古者吉祭有配,皆一尸。其始祝洗酌奠,奠於鉶南,止有一爵。及主人獻尸,主婦亞獻,賓長三獻,亦止一爵。請罷諸室奠副爵。其祫享別廟,皇后自如常禮。應祠告天地、宗廟、社稷,
 並用牲幣。如唐置太廟局令,以宗正丞充,罷攝知廟少卿,而宮闈令不預祠事。」又言:「晨稞之時,皇帝先搢大圭,上香、稞鬯、復位,候作樂饋食畢,再搢大圭,執鎮圭,奠於繅藉。次奠幣、執爵,庶禮神並在降神之後。」從之。



 八年,太常寺言:「故事,山陵前,宗廟輟祭享,朔望以內臣行薦食之禮,俟祔廟畢仍舊。今景靈宮神御殿已行上食,太廟朔望薦食自當請罷。」從之。



 元祐七年,詔復用牙盤食。舊制,並於禮饌外設,元豐中罷之,禮官呂希純建議曰:「先
 王之祭,皆備上古、中古及今世之食。所設禮饌,即上古、中古之食,牙盤常食,即今世之食。議者乃以為宗廟牙盤原於秦、漢陵寢上食,殊不知三代以來,自備古今之食。請依祖宗舊制,薦一牙盤。」從之,乃更其名曰薦羞。希純又請:「帝後各奠一爵,後爵謂之副爵。今帝後惟奠一爵共享,瀆禮莫甚。請設副爵,亦如其儀。」



 大觀四年,議禮局言:「太廟每享,各設太尊二,則是以追享、朝享之尊,施之於禴祠蒸嘗,失禮尤甚。請今四時之享,不設太尊。」又
 言:「圭瓚之制,親祀以塗金銀瓚,有司行事以銅瓚,其大小長短之制皆不如禮,請改以應古制。」又言:「太廟圭瓚、別廟璋瓚,舊用鈱石,請改用玉。」又言:「新定太廟陳設之儀,盡依周制,籩豆各用二十有六,簠簋各八。以籩二十有六為四行,以右為上,羞籩二為第一行,朝事籩八次之,饋食籩八又次之,加籩八又次之。豆二十有六為四行,以左為上,羞豆二為第一行,朝事豆八次之,饋食豆八又次之,加豆八又次之。簠八為二行,在籩之外,簋八
 為二行,在豆之外。籩豆所實之物,悉如《周禮》籩人、醢人之制,惟簠以稻粱,簋以黍稷,而茅菹以FF,蚳醢以蜂子代之。」又言:「宗廟之祭用太牢而三鉶,實牛、羊、豕之羹,固無可論者。至於太羹止設一登,以《少牢饋食禮》考之,則少牢者羊、豕之牲也。佐食羞兩鉶,司士進湆二豆。三牲之祭,鉶既設三,則登亦如其數。請太廟設三登,實牛、羊、豕之湆以為太羹,明堂亦如之。」



 高宗建炎三年,奉安神主於溫州,權用酒脯。紹興五年,臨安府建太廟,始用特
 羊,十年改用少牢。其廟享之禮,七年祀明堂於建康,以徽宗之喪,太常少卿吳表臣援熙寧故事,謂當英宗喪未除,不廢景靈宮、太廟之禮。翰林學士朱震以為不然,謂:「《王制》:『喪三年不祭,惟天地、社稷越紼行事。』孰謂三年之喪,而可以見宗廟行吉禮乎?」吏部尚書孫近等言:「按《春秋》:『君薨,卒哭而祔,祔而作主,特祀於寢,蒸嘗禘於廟。』杜預謂:「新主既特祀於寢,則宗廟常祀,自當如舊。』又熙寧元年,神宗諒暗,用景德故事,躬行郊廟之禮。今明
 堂大禮,已在以日易月服除之後,皇帝合享太廟,所有鹵簿、鼓吹及樓前宮架、諸軍音樂皆備而不作。」



 三十二年,孝宗即位,擇日朝享太廟。禮部言:「牲牢、禮料、酒、齊等物,並如五享行之。」紹熙五年,寧宗即位,時有孝宗之喪。閏十月,浙東提舉李大性言:「自漢文帝以來,皆即位而謁廟。陛下龍飛已閱三月,未嘗一至宗廟行禮。鑾輿屢出,過太廟門而不入,揆之人情,似為闕典。乞早擇日,恭謁太廟。」詔乃遵用三年之制。吏部員外郎李謙請以來
 年正月上日躬行告廟之禮。禮寺以為俟皇帝從吉,討論施行。理宗即位,行三年之喪,初行明堂朝享,以大臣攝事,即吉後,始行親享之禮。



 薦新。太宗雍熙二年十一月,宗正寺言:「準詔,送兔十頭充享太廟。按《開寶通禮》,薦新之儀,詣僖祖室戶前,盥洗酌獻訖,再拜,次獻諸室如上禮。」遂詔曰:「夫順時搜狩,禮有舊章,非樂畋游,將薦宗廟,久隳前制,闕孰甚焉。爰遵時令,暫狩近郊,既躬獲禽,用以薦俎。其今月十一日畋
 獵,親射所獲田禽,並付所司,以備太廟四時薦享,著為令。」



 景祐二年,宗正丞趙良規言:「《通禮》著薦新凡五十餘物,今太廟祭享之外唯薦冰,其餘薦新之禮,皆寢不行。宜以品物時新,所司送宗正,令尚食簡擇滋味與新物相宜者,配以薦之。」於是禮官、宗正條定:「逐室時薦,以京都新物,略依時訓,協用典章。請每歲春孟月薦蔬,以韭以菘,配以卵。仲月薦冰,季月薦蔬以筍,果以含桃。夏孟月嘗麥,配以彘,仲月薦果,以瓜以來禽,季月薦果,
 以芡以菱。秋孟月嘗粟嘗穄,配以雞,果以棗以梨,仲月嘗酒嘗稻,蔬以茭筍,季月嘗豆嘗蕎麥。冬孟月羞以兔,果以慄,蔬以藷藇,仲月羞以雁以獐,季月羞以魚。凡二十八種,所司烹治。自彘以下,令御廚於四時牙盤食烹饌,卜日薦獻,一如《開寶通禮》。」又太常禮院言:「自來薦冰,惟薦太廟逐室帝主,後主皆闕。謹按朔望每室牙盤食,帝後同薦。又按《禮》:『有薦新如朔奠。』詳此獻祀,帝後主別無異等之義。今後前廟逐室後主,欲乞四時薦新,並如朔望
 牙盤例,後廟、奉慈廟如太廟之禮。」



 皇祐三年,太常寺王洙言:「每內降新物,有司皆擇吉日,至涉三四日,而物已損敗。自今令禮部預為關報,於次日薦之,更不擇日。」



 元豐元年,宗正寺奏:「據太常寺報,選日薦新兔、藷藇、慄黃。今三物久粥於市,而廟猶未薦,頗違禮意。蓋節序有蚤晏,品物有後先,自當變通,安能齊一?又唐《開元禮》,薦新不出神主。今兩廟薦新,及朔望上食,並出神主。請下禮官參定所宜。」



 詳定所言:「古者薦新於廟之寢,無尸,不卜
 日,不出神主,奠而不祭。近時擇日而薦,非也。天子諸侯,物熟則薦,不以孟仲季為限。《呂氏·月令》,一歲之間八薦新物,《開元禮》加以五十餘品。景祐中,禮官議以《呂紀》簡而近薄,唐令雜而不經,於是更定四時所薦凡二十八物,除依《詩》、《禮》、《月令》外,又增多十有七品。雖出一時之議,然歲時登薦,行之已久。依於古則太略,違於經則無法。今欲稍加刊定,取其間先王所嘗享用膳羞之物,見於經者存之,不經者去之。請自今孟春薦韭以卵,羞以葑,
 仲春薦冰,季春薦筍,羞以含桃;孟夏嘗麥以彘,仲夏嘗雛以黍,羞以瓜,季夏羞以芡以菱;孟秋嘗粟與稷,羞以棗以梨,仲秋嘗麻嘗稻,羞以蒲,季秋嘗菽,羞以兔以慄;孟冬羞以雁,仲冬羞以麇,季冬羞以魚。今春不薦鮪,誠為闕典。請季春薦鮪,無則闕之。舊有林檎、蕎麥、藷藇之類,及季秋嘗酒,並合刪去。凡新物及時出者,即日登獻,既非正祭,則不當卜日。《漢儀》嘗韭之屬,皆於廟而不在寢,故《韋玄成傳》以為廟歲二十五祠,而薦新在焉。自漢
 至於隋、唐,因仍其失,薦新雖在廟,然皆不出神主。今出神主,失禮尤甚。請依《五禮精義》,但設神座,仍候廟成,薦新於寢。」詔依所定,如鮪闕,即以魴鯉代之。既而知宗正丞趙彥若言:「禮院以仲秋茭萌不經,易以蒲白。今仲秋蒲無白,改從春獻。」



 大觀,禮局亦言:「薦新雖系以月,如櫻、筍三月當進,或萌實未成,轉至孟夏之類,自當隨時之宜,取新以薦。」政和四年,比部員外郎何天衢言:「祭不欲數,數則煩,祭不欲疏,疏則怠。先王建祭祀之禮,必得疏
 數之中,未聞一日之間,遂行兩祭者也。今太廟薦新,有與朔祭同日者。夫朔祭行於一月之首,不可易也。若夫薦新,則未嘗卜日,一月之內,皆可薦也。新物未備,猶許次月薦之,亦何必同朔日哉?」自是薦新偶與朔祭同日,詔用次日焉。中興仍舊制。



 加上祖宗謚號。太祖建隆元年九月,太常禮院言:「謹按唐大中初,追尊順宗、憲宗謚號,皇帝於宣政殿授玉冊,遣宰臣以下持節奉冊赴太廟。授冊日,帝既御殿,百僚
 拜訖,降階跪授冊於太尉,候太尉奉冊出宣政門,然後升殿。凡皇帝行禮,皆太常卿贊導奉引。」奏可。是月二十七日,帝御崇元殿,備禮遣使奉冊上四廟謚號。皇帝高祖府君冊曰:「孝曾孫嗣皇帝臣某,再拜稽首上言,伏以昊天有命,皇宋勃興,括厚載以開階,宅中區而撫運,夷夏蠻貊,罔不獻誠,山川鬼神,罔不受職。非臣否德,肇此丕圖,實賴先正儲休,上玄降鑒,既虔膺於大寶,乃眇覿於遐源,敢遵歷代之規,式薦配天之號。謹遣使司空
 兼門下侍郎同中書門下平章事王溥、副使兵部尚書李濤奉寶冊,上尊謚曰文獻皇帝,廟號僖祖,皇帝高祖母崔氏曰文懿皇后。」皇曾祖府君冊曰:「伏以天命匪忱,惟歸於有德,人文設教,必始於貽謀。乘時既肇於興王,報本敢稽於尊祖。非隆徽稱,則大享何以配神,非鏤良鈱,則洪烈何由垂世?方作《猗那》之頌,永嚴昭穆之容。謹遣使王溥、副使李濤奉冊寶,上尊謚曰惠元皇帝,廟號順祖,皇曾祖母桑氏曰惠明皇后。」皇祖驍衛府君冊曰:「伏
 以人瞻烏止,運葉龍飛。非發源之長,析派不能通上漢;非積基之厚,嗣孫不能有中區。今人紀肇修,孝思罔極,酌百王之損益,薦四廟之蒸嘗。謹遣使王溥、副使李濤奉寶冊,上尊謚曰簡恭皇帝,廟號翼祖,皇祖母京兆郡太夫人劉氏曰簡穆皇后。」聖考太尉府君冊曰:「昔者流火開祥,周發薦文王之號,黃星應運,曹丕揚魏祖之功。咸因致孝之誠,式展尊親之義,爰遵大典,亟上尊稱。謹遣使王溥、副使李濤奉冊寶,上尊謚曰昭武皇帝,廟號
 宣祖。」禮畢,群臣進表奉慰。



 太宗太平興國二年正月甲戌,上太祖英武聖文神德皇帝。真宗大中祥符元年十一月二十七日,帝於朝元殿備禮,奉祖宗尊謚冊寶,再拜授攝太尉王旦奉之以出,安太祖冊寶於玉輅,太宗冊寶於金輅,詣太廟,奉上太祖曰啟運立極英武聖文神德玄功大孝皇帝,太宗曰至仁應道神功聖德文武大明廣孝皇帝。禮畢,親行朝享之禮。天禧元年正月九日,加上六室尊謚二字:僖祖曰文獻睿和皇帝,順祖曰
 惠元睿明皇帝,翼祖曰簡恭睿德皇帝,宣祖曰昭武睿聖皇帝,太祖曰啟運立極英武睿文神德聖功至明大孝皇帝,太宗曰至仁應道神功聖德睿烈大明廣孝皇帝。禮畢,群臣拜表稱賀。十一日,帝行朝享之禮。



 仁宗天聖二年十一月二十五日,加上真宗謚曰文明武定章聖元孝皇帝。慶歷七年十一月二十五日,加上真宗謚曰膺符稽古成功讓德文明武定章聖元孝皇帝。



 神宗元豐六年五月,改加上尊謚作奉上徽號。十一月二日,
 奉上仁宗徽號曰體天法道極功全德神文聖武睿哲明孝皇帝,又上英宗徽號曰體乾膺歷隆功盛德憲文肅武睿神宣孝皇帝。



 哲宗紹聖二年正月,帝謂輔臣曰:「祖宗謚號,各加至十六字。神宗皇帝今止初謚,尚未增加,宜考求典故以聞。」宰臣章惇等對曰:「祖宗加謚,歲月不定。真廟初加八字,是天聖二年。今神宗祔廟已十年,故事加徽號必在南郊前,謹如聖旨討閱以聞。」四月二十七日,詔加上神宗皇帝徽號,於大禮前三日行禮。九
 月十六日,奉上冊寶曰神宗紹天法古運德建功英文烈武欽仁聖孝皇帝。



 徽宗崇寧三年十一月二十三日,更定神宗徽號曰體元顯道帝德王功英文烈武欽仁聖孝皇帝,又奉哲宗徽號曰憲元繼道顯德定功欽文睿武齊聖昭孝皇帝。大觀元年九月,加上僖祖徽號為十六字,曰立道肇基積德起功懿文憲武睿和至孝皇帝。政和三年十一月五日,加上神宗、哲宗徽號。前二日,皇帝御大慶殿,奉神宗冊寶授太師、魯國公蔡京,載以
 玉輅,奉哲宗冊寶授少師、太宰何執中,載以金輅,並詣太廟幄殿,奉安以俟。四日,皇帝詣景靈宮行禮,赴太廟宿齋。五日,服袞冕,恭上神宗冊寶於本室,曰體元顯道法古立憲帝德王功英文烈武欽仁聖孝皇帝,又上哲宗冊寶於本室,曰憲元繼道世德揚功欽文睿武齊聖昭孝皇帝。次行朝享,禮畢,赴南郊青城宮。



 紹興十二年十一月,詔議加上徽宗徽號曰體神合道駿烈遜功聖文仁德憲慈顯孝皇帝。十三年正月九日,皇帝御文德
 殿,命宰臣秦檜奏請太廟。十日,內殿宿齋,文武百僚集於發冊寶殿門幕次,次禮儀使、閣門官、太常博士、禮直官分立御幄前,次分引百僚入就殿下,東西相向立定,禮直官引奉冊寶使、侍中、中書令、舉寶舉冊官詣殿下西階之西,東向立。俟齋室簾降,皇帝服通天冠、絳紗袍,禮部侍郎奏中嚴外辦。次禮直官、太常博士引禮儀使當幄前俯伏跪奏:「禮儀使臣某言,請皇帝行奉上徽宗皇帝發冊寶之禮。」奏訖,俯伏,興。簾卷,前導官前導皇帝
 出幄,執大圭,詣冊寶幄東褥位,西向立,禮儀使奏請再拜,皇帝再拜,三上香,再拜,在位官皆再拜。前導還褥位,西向立,侍中、中書令、舉冊舉寶官升殿,入冊寶幄。舉冊寶官俱搢笏跪,舉冊寶與侍中、中書令奉冊寶進行,皇帝後從,降自西階,至殿下褥位,南向立。禮儀使奏皇帝再拜,舉冊官奉冊,舉寶官奉寶,皇帝搢大圭,跪奉受冊寶使,皇帝執大圭再拜,在位官皆再拜。持節者持節導冊寶進行,出殿正門。禮儀使奏禮畢。皇帝釋大圭,升自
 東階,入齋室。禮部郎中奏解嚴。次冊寶出北宮門,奉冊寶使以下騎從,至太廟靈星門外下馬,步從至太廟南神門外。次日,文武百僚集於太廟幕次,分引詣殿下再拜,冊寶使詣各室行奠獻禮。次贊者引舉冊官舉冊。,舉寶官舉寶,禮直官引侍中、中書令前導冊寶入自南正門,至殿西階下權置定,各再拜。次詣徽宗室,冊寶使俯伏跪奏稱:「嗣皇帝臣某,謹遣臣等奉徽號冊寶。」奉訖,俯伏,興。舉冊官舉冊進,中書令跪讀冊文,舉寶官舉寶進,
 侍中跪讀寶文,冊寶使以下各再拜,至冊寶幄安奉。禮畢,以次退。次文武百僚奉表稱賀。



 紹熙二年八月,詔上高宗徽號曰受命中興全功至德聖神武文昭仁憲孝皇帝。慶元三年,上孝宗徽號曰紹統同道冠德昭功哲文神武明聖成孝皇帝。嘉泰三年,上光宗徽號曰循道憲仁明功茂德溫文順武聖哲慈孝皇帝。寶慶三年,上寧宗徽號曰法天備道純德茂功仁文哲武聖睿恭孝皇帝。咸淳二年,上理宗徽號曰建道備德大功復興烈
 文仁武聖明安孝皇帝。並如紹興十三年儀注。



 廟諱。紹興二年十一月,禮部、太常寺言:「淵聖皇帝御名,見於經傳義訓者,或以威武為義,或以回旋為義,又為植立之象,又為亭郵表名,又為圭名,又為姓氏,又為木名,當各以其義類求之。以威武為義者,今欲讀曰『威』;以回旋為義者,今欲讀曰『旋』;以植立為義者,今欲讀曰『植』;若姓氏之類,欲去『木』為『亙』。又緣漢法,『邦』之字曰『國』,『盈』之字曰『滿』,止是讀曰『國』、曰『滿』,其本字見於經傳者未嘗改
 易。司馬遷,漢人也,作《史記》,曰:『先王之制,邦內畿服,邦外侯服。』又曰:『盈而不持則傾。』於『邦』字、『盈』字亦不改易。今來淵聖皇帝御名,欲定讀如前外,其經傳本字,即不當改易,庶幾萬世之下,有所考證,推求義類,別無未盡。」三十二年正月,禮部、太常寺言:「欽宗祔廟,翼祖當遷。於正月九日,先遷翼祖皇帝、簡穆皇后神主奉藏於夾室。所有以後翼祖皇帝諱,依禮不諱。」詔恭依。



 紹熙元年四月,詔:「今後臣庶命名,並不許犯祧廟正諱。如名字見有犯祧
 廟正諱者,並合改易。」



 嘉定十三年十月,司農寺丞岳珂言:「孝宗舊諱從『伯』從『玉』從『宗』。考國朝之制,祖宗舊諱二字者,皆著令不許並用。」又言「欽宗舊諱二字,其一從『■』從『旦』,其一從『火』從『亙』,皆合回避。乞並下禮、寺討論,頒降施行。」既而禮、寺討論:「所有欽宗、孝宗舊諱,若二字連用,並合回避,宜從本官所請,刊入施行。」從之。



\end{pinyinscope}