\article{列傳第二百三十一姦臣二}

\begin{pinyinscope}

 蔡
 京,字元長,興化仙遊人。登熙寧三年進士第,調錢塘尉、舒州推官,累遷起居郎。使遼還,拜中書舍人。時弟卞
 已為舍人,故事,入官以先後為序,卞乞班京下。兄弟同掌書命,朝廷榮之。改龍圖閣待制、知開封府。



 元豐末,大臣議所立,京附蔡確,將害王珪以貪定策之功,不克。司馬光秉政,復差役法,為期五日,同列病太迫,京獨如約,悉改畿縣雇役,無一違者。詣政事堂白光,光喜曰:「使人人奉法如君,何不可行之有!」已而臺諫言京挾邪壞法,出知成德軍,改瀛州,徙成都。諫官范祖禹論京不可用,乃改江淮荊浙發運使,又改知揚州。歷鄆、永興軍,遷龍
 圖閣直學士,復知成都。



 紹聖初,入權戶部尚書。章惇復變役法,置司講議,久不決。京謂惇曰:「取熙寧成法施行之爾,何以講為?」惇然之,雇役遂定。差雇兩法,光、惇不同。十年間京再蒞其事,成於反掌,兩人相倚以濟,識者有以見其姦。



 卞拜右丞,以京為翰林學士兼侍讀,修國史。文及甫獄起,命京窮治,京捕內侍張士良,令述陳衍事狀,即以大逆不道論誅,並劉摯、梁燾劾之。衍死,二人亦貶死,皆錮其子孫。王巖叟、范祖禹、劉安世復遠竄。京覬
 執政,曾布知樞密院,忌之,密言卞備位承轄,京不可以同升,但進承旨。



 徽宗即位,罷為端明、龍圖兩學士,知太原,皇太后命帝留京畢史事。踰數月,諫官陳瓘論其交通近侍,瓘坐斥,京亦出知江寧,頗怏怏,遷延不之官。御史陳次升、龔夬、陳師錫交論其惡,奪職,提舉洞霄宮,居杭州。



 童貫以供奉官詣三吳訪書畫奇巧,留杭累月,京與遊,不舍晝夜。凡所畫屏幛、扇帶之屬,貫日以達禁中,且附語言論奏至帝所,由是帝屬意京。又太學博士范
 致虛素與左街道錄徐知常善,知常以符水出入元符后殿,致虛深結之,道其平日趣向,謂非相京不足以有為。已而宮妾、宦官合為一詞譽京,遂擢致虛右正言,起京知定州。崇寧元年,徙大名府。韓忠彥與曾布交惡,謀引京自助,復用為學士承旨。徽宗有意修熙、豐政事,起居舍人鄧洵武黨京,撰《愛莫助之圖》以獻,徽宗遂決意用京。忠彥罷,拜尚書左丞,俄代曾布為右僕射。制下之日,賜坐延和殿,命之曰:「神宗創法立制,先帝繼之,兩遭
 變更,國是未定。朕欲上述父兄之志,卿何以教之?」京頓首謝,願盡死。二年正月,進左僕射。



 京起於逐臣,一旦得志,天下拭目所為,而京陰託「紹述」之柄,箝制天子,用條例司故事,即都省置講議司,自為提舉,以其黨吳居厚、王漢之十餘人為僚屬,取政事之大者,如宗室、冗官、國用、商旅、鹽澤、賦調、尹牧,每一事以三人主之。凡所設施,皆由是出。用馮澥、錢遹之議,復廢元祐皇后。罷科舉法,令州縣悉仿太學三舍考選,建辟雍外學于城南,以待四
 方之士。推方田於天下。榷江、淮七路茶,官自為市。盡更鹽鈔法,凡舊鈔皆弗用,富商巨賈嘗齎持數十萬緡,一旦化為流丐,甚者至赴水及縊死。提點淮東刑獄章縡見而哀之,奏改法誤民,京怒,奪其官。因鑄當十大錢,盡陷縡諸弟。御史沈畸等用治獄失意,羈削者六人。陳瓘子正彚以上書黥置海島。



 南開黔中,築靖州。辰溪猺叛,殺漵浦令,京重為賞,募殺一首領者賜之絹三百,官以班行,且不令質究本末。荊南守馬珹言:「有生猺,有省地猺,
 今未知叛者為何種族,若計級行賞,俱不能無枉濫。」蔣之奇知樞密院,恐忤京意,白言珹不體國,京罷珹,命舒亶代之,以勦絕群猺為期。西收湟川、鄯、廓,取䍧牱、夜郎地。



 擢童貫領節度使,其後揚戩、藍從熙、譚稹、梁師成皆踵之。凡寄資一切轉行,祖宗之法蕩然無餘矣。又欲兵柄士心皆歸己,建澶、鄭、曹、拱州為四輔,各屯兵二萬,而用其姻昵宋喬年、胡師文為郡守。禁卒于掫月給錢五百,驟增十倍以固結之。威福在手,中外莫敢議。累轉司
 空,封嘉國公。



 京既貴而貪益甚,已受僕射奉,復創取司空寄祿錢,如粟、豆、柴薪與傔從糧賜如故,時皆折支,亦悉從真給,但入熟狀奏行,帝不知也。



 時元祐群臣貶竄死徙略盡,京猶未愜意,命等其罪狀,首以司馬光,目曰姦黨,刻石文德殿門,又自書為大碑,遍班郡國。初,元符末以日食求言,言者多及熙寧、紹聖之政,則又籍范柔中以下為邪等,凡名在兩籍者三百九人,皆錮其子孫,不得官京師及近甸。五年,進司空、開府儀同三司、安遠
 軍節度使,改封魏國。



 時承平既久,帑庾盈溢,京倡為豐、亨、豫、大之說,視官爵財物如糞土,累朝所儲掃地矣。帝嘗大宴,出玉琖、玉卮示輔臣曰:「欲用此,恐人以為太華。」京曰:「臣昔使契丹,見玉盤琖,皆石晉時物,持以誇臣,謂南朝無此。今用之上壽,於禮無嫌。」帝曰:「先帝作一小臺財數尺,上封者甚眾,朕甚畏其言。此器已就久矣,倘人言復興,久當莫辨。」京曰:「事苟當於理,多言不足畏也。陛下當享天下之奉,區區玉器,何足計哉!」



 五年正月,彗出
 西方,其長竟天。帝以言者毀黨碑,凡其所建置,一切罷之。京免為開府儀同三司、中太乙宮使。其黨陰援於上,大觀元年,復拜左僕射。以南丹納土,躐拜太尉,受八寶,拜太師。



 三年,臺諫交論其惡,遂致仕。猶提舉修《哲宗實錄》,改封楚國,朝朔望。太學生陳朝老追疏京惡十四事,曰瀆上帝、罔君父、結奧援、輕爵祿、廣費用、變法度、妄制作、喜導諛、箝臺諫、熾親黨、長奔競、崇釋老、窮土木、矜遠略。乞投畀遠方,以禦魑魅。其書出,士人爭相傳寫,以為
 實錄。四年五月,彗復出奎、婁間,御史張克公論京輔政八年,權震海內,輕錫予以蠹國用,託爵祿以市私恩,役將作以葺居第,用漕船以運花石。名為祝聖而修塔,以壯臨平之山;託言灌田而決水,以符「興化」之讖。法名退送,門號朝京。方田擾安業之民,圜土聚徙郡之惡。不軌不忠,凡數十事。先是,御史中丞石公弼、侍御史毛注數劾京,未允,至是,貶太子少保,出居杭。



 政和二年,召還京師,復輔政,徙封魯國,三日一至都堂治事。京之去也,中
 外學官頗有以時政為題策士者。提舉淮西學士蘇棫欲自售,獻議請索五年間策問,校其所詢,以觀向背,於是坐停替者三十餘人。初,國制,凡詔令皆中書門下議,而後命學士為之。至熙寧間,有內降手詔不由中書門下共議,蓋大臣有陰從中而為之者。至京則又患言者議己,故作御筆密進,而丐徽宗親書以降,謂之御筆手詔,違者以違制坐之。事無巨細,皆託而行,至有不類帝劄者,群下皆莫敢言。由是貴戚、近臣爭相請求,至使中
 人楊球代書,號曰「書楊」,京復病之而亦不能止矣。



 既又更定官名,以僕射為太、少宰,自稱公相,總治三省。追封王安石、蔡確皆為王,省吏不復立額,至五品階以百數,有身兼十餘奉者。侍御史黃葆光論之,立竄昭州。拔故吏魏伯芻領榷貨,造料次錢券百萬緡進入,徽宗大喜,持以示左右曰:「此太師與我奉料也。」擢伯芻至徽猷閣待制。



 京每為帝言,今泉幣所積贏五千萬,和足以廣樂,富足以備禮,於是鑄九鼎,建明堂,修方澤,立道觀,作《大
 晟樂》,制定命寶。任孟昌齡為都水使者,鑿大伾三山,創天成、聖功二橋,大興工役,無慮四十萬。兩河之民,愁困不聊生,而京僴然自以為稷、契、周、召也。又欲廣宮室求上寵媚,召童貫輩五人,風以禁中逼側之狀。貫俱聽命,各視力所致,爭以侈麗高廣相誇尚,而延福宮、景龍江之役起,浸淫及于艮嶽矣。



 子攸、儵、翛,攸子行,皆至大學士,視執政。鞗尚茂德帝姬。帝七幸其第,賚予無算。命坐傳觴,略用家人禮。廝養居大官,媵妾封夫人,然公論益
 不與,帝亦厭薄之。



 宣和二年,令致仕。六年,以朱勔為地,再起領三省。京至是四當國,目昏眊不能事事,悉决于季子絛。凡京所判,皆絛為之,且代京入奏。每造朝,侍從以下皆迎揖,呫囁耳語,堂吏數十人,抱案後從,由是恣為姦利,竊弄威柄,驟引其婦兄韓梠为户部侍郎,媒糵密謀,斥逐朝士,創宣和庫式貢司,四方之金帛與府藏之所儲,盡拘括以實之,為天子之私財。宰臣白時中、李邦彥惟奉行文書而已,既不能堪。兄攸亦發其事,上怒,
 欲竄之,京力丐免,特勒停侍養,而安置韓梠黄州。未幾,褫絛侍讀,毀賜出身敕,而京亦致仕。方時中等白罷絛以撼京,京殊無去意。帝呼童貫使詣京,令上章謝事,貫至,京泣曰:「上何不容京數年,當有相讒譖者。」貫曰:「不知也。」京不得已,以章授貫,帝命詞臣代為作三表請去,乃降制從之。



 欽宗即位,邊遽日急,京盡室南下,為自全計。天下罪京為六賊之首,侍御史孫覿等始極疏其姦惡,乃以秘書監分司南京,連貶崇信、慶遠軍節度副使,衡
 州安置,又徙韶、儋二州。行至潭州死,年八十。



 京天資凶譎,舞智御人,在人主前,顓狙伺為固位計,始終一說,謂當越拘攣之俗,竭四海九州之力以自奉。帝亦知其姦,屢罷屢起,且擇與京不合者執政以柅之。京每聞將退免,輒入見祈哀,蒲伏扣頭,無復廉恥。燕山之役,京送攸以詩,陽寓不可之意,冀事不成得以自解。見利忘義,至於兄弟為參、商,父子如秦、越。暮年即家為府,營進之徒,舉集其門,輸貨僮隸得美官,棄紀綱法度為虛器。患失
 之心無所不至,根株結盤,牢不可脫。卒致宗社之禍,雖譴死道路,天下猶以不正典刑為恨。



 子八人,儵先死,攸、翛伏誅,絛流白州死,鞗以尚帝姬免竄,餘子及諸孫皆分徙遠惡郡。



 卞,字元度,與京同年登科,調江陰主簿。王安石妻以女,因從之學。元豐中,張璪薦為國子直講,加集賢校理、崇政殿說書,擢起居舍人,歷同知諫院、侍御史。居職不久,皆以王安石執政親嫌辭。拜中書舍人兼侍講,進給事
 中。



 哲宗立,遷禮部侍郎。使於遼,遼人頗聞其名。卞適有寒疾,命載以白馳車,典客者曰:「此,君所乘,蓋異禮也。」使還,以龍圖閣待制知宣州,徙江寧府,歷揚、廣、越、潤、陳五州。廣州寶貝叢湊,一無所取。及徙越,夷人清其去,以薔薇露灑衣送之。



 紹聖元年,復為中書舍人,上疏言:「先帝盛德大業,卓然出千古之上,發揚休光,正在史策。而實錄所紀,類多疑似不根,乞驗索審訂,重行刊定,使後世考觀,無所迷惑。」詔從之。以卞兼國史修撰。初,安石且死,
 悔其所作《日錄》,命從子防焚之,防詭以他書代。至是,卞即防家取以上,因芟落事實,文飾姦偽,盡改所修實錄、正史,於是呂大防、范祖禹、趙彥若、黃庭堅皆獲深譴。遷翰林學士。



 四年,拜尚書左丞,專託「紹述」之說,上欺天子,下脅同列。凡中傷善類,皆密疏建白,然後請帝親劄付外行之。章惇雖鉅姦,然猶在其術中。惇輕率不思,而卞深阻寡言,論議之際,惇毅然主持,卞或噤不啟齒。一時論者以為惇跡易明,卞心難見。



 徽宗即位,諫官陳瓘、任
 伯雨、御史龔夬疏其兄弟姦惡,瓘并数卞尊私史以厭宗廟之罪,伯雨言:「卞之惡有過於惇。去年封事,數千人皆乞斬惇、卞,公議於此可見矣。」遂陳其大罪有六,曰:「誣罔宣仁聖烈保祐之功,欲行追廢,一也;凡紹聖以來竄逐臣僚,皆卞啟而後行,二也;宮中厭勝事作,哲宗方疑,未知所處,惇欲召禮法官通議,卞云:『既犯法矣,何用禮法官議?』皇后以是得罪,三也;編排元祐章牘,萋菲語言,被罪者數千人,議自卞出,四也;鄒浩以言忤旨,卞激怒
 哲宗,致之遠謫,又請治其親故送別之罪,五也;蹇序辰建看詳訴理之議,章惇遲疑未應,卞即以二心之言迫之,惇默不敢對,即日置局,士大夫得罪者八百三十家,凡此皆卞謀之而惇行之,六也。願亟正典刑,以謝天下。」詔以資政殿學士知江寧府,連貶少府少監、分司池州。



 才逾歲,起知大名府,徙揚州,召為中太乙宮使,擢知樞密院。時京居相位,卞禮辭,不許。帝謀復湟、鄯,問于卞,卞以王厚、高永年對。與京合謀,竭府藏以事邊,募商人運
 糧,不復問其直貴賤。鄯、廓至斗米錢四千,束芻錢千二百,秦中騷困。及取三州,進金紫光祿大夫,永年竟為帳下執去以降。自是西方交兵,連年不息,追仇任伯雨所言,曲自辦理。至欲會獄證治,諸人坐貶。



 卞居心傾邪,一意以婦公王氏所行為至當。兄晚達而位在上,致己不得相,故二府政事時有不合。京以中旨用童貫為陝西制置使,卞言不宜用宦者,右丞張康國引李憲故事以對,卞曰:「用憲已非美事,憲猶稍習兵,貫略無所長,異時
 必誤邊計。」帝令中書行之。京于帝前詆卞,卞求去,以資政殿大學士知河南。



 妖人張懷素敗,卞素與之遊,謂其道術通神,嘗識孔子、漢高祖,至稱為大士,坐降職。旋加觀文殿學士,拜昭慶軍節度使,入為侍讀,進檢校少保、開府儀同三司,易節鎮東。



 政和末,謁歸上塚,道死,年六十。贈太傅,諡曰「文正」。高宗即位,追責為寧國軍節度副使。紹興五年,又貶單州團練副使。



 攸,字居安,京長子也。元符中,監在京裁造院。徽宗時為
 端王,每退朝,攸適趨局,遇諸途,必下馬拱立,王問左右,知為蔡承旨子,心善之。及即位,記其人,遂有寵。



 崇寧三年,自鴻臚丞賜進士出身,除秘書郎,以直秘閣、集賢殿修撰編修《國朝會要》,二年間至樞密直學士。京再入相,加龍圖閣學士兼侍讀,詳定《九域圖志》,修《六典》,提舉上清寶箓宫、秘書省兩街道錄院、禮制局。道、史官僚合百人,多三館雋遊,而攸用大臣子領袖其間,懵不知學,士論不與。初置宣和殿,命為大學士,賜毬文方團金帶,改
 淮康軍節度使。



 帝將去京,先逐其黨劉昺、劉煥等,使御史中丞王安中劾之。攸通籍禁庭,聞其事,亟請間百拜以懇,帝意遂解。其後與京權勢日相軋,浮薄者復間之,父子各立門戶,遂為仇敵。攸別居賜第,嘗詣京,京正與客語,使避之,攸甫入,遽起握父手為胗視狀,曰:「大人脈勢舒緩,體中得無有不適乎?」京曰:「無之。」攸曰:「禁中方有公事。」即辭去。客竊窺見,以問京,京曰:「君固不解此,此兒欲以為吾疾而罷我也。」閱數日,京果致仕。以季弟絛鍾
 愛于京,數請殺之,帝不許。



 攸歷開府儀同三司、鎮海軍節度使、少保,進見無時,益用事,與王黼得預宮中秘戲,或侍曲宴,則短衫窄褲,塗抹青紅,雜倡優侏儒,多道市井淫媟謔浪語,以蠱帝心。妻宋氏出入禁掖,子行領殿中監,視執政,寵信傾其父。帝留意道家者說,攸獨倡為異聞,謂有珠星璧月、跨鳳乘龍、天書雲篆之符,與方士林靈素之徒爭證神變事。於是神霄、玉清之祠遍天下,咎端自攸興矣。



 童貫伐燕,以攸副宣撫,攸童騃不習事。
 謂功業可唾手致。入辭之日,二美嬪侍上側,攸指而請曰:「臣成功歸,乞以是賞。」帝笑而弗責。涿州留守郭藥師擁所部八千人舉涿、易二州降,進攸少傅。王師入燕,進少師,封英國公。還,領樞密院。王黼罷政,帝欲大用攸,既而悔之,但進太保,徙封燕。帝欲內禪,親書「傳位東宮」字授李邦彥,邦彥卻立不敢承,遂以付攸。攸退,屬其客給事中吳敏,議遂定。



 靖康元年,從上皇南下。及還都,始責為大中大夫,繼而安置永州,連徙潯、雷。京死,御史言攸
 罪不減乃父,燕山之役禍及宗社,驕奢淫泆載籍所無,當竄諸海島。詔置萬安軍,尋遣使者隨所至誅之。



 翛初以恩澤為親衛郎、秘書丞,至保和殿學士。宣和中,拜禮部尚書兼侍講。時翛弟兄亦知事勢日異,其客傅墨卿、孫傅等復語之曰:「天下事必敗,蔡氏必破,當亟為計。」翛心然之,密與攸議,稍持正論,故與京異。然皆蓄縮不敢明言,遂引吳敏、李綱、李光、楊時等用之,以挽物情。尋加大學士,提舉醴泉觀。



 欽宗立,翛上募兵陝西策,自
 請行,又勸西幸,帝頗採納,俾知京兆府。計垂就,攸忌其功成,會金破濬州,徽宗南幸,攸假徽宗旨,請翛守鎮江,改資政殿大學士。或謂翛前計已乖,宜勿行。翛幸得去,不復辭。流言至京師,謂將復辟於鎮江。帝趣迎上皇還,而責翛昭信軍節度副使。



 攸之誅也,御史陳述且行,帝取詔批其尾曰:「翛亦然。」於是並誅。



 崈者,京族子也。性矯妄,善談鬼神事。當承門蔭,固推與庶兄,宗族稱為賢。崇寧初,京黨以學行修飭聞諸朝,與泉州布衣呂注皆著
 道士服。召入謁,累官拜給事中兼侍讀。



 京去位,為言者所攻,以顯謨閣待制提舉崇福宮。言者復論其不學無文,結豪民,規厚利,持道家吐納之說以為論思,侍立集英瞑目自若為不恭,遂奪職。陳正彚上京變事,置獄京師,具陳在杭州時,日聞崈盛言京有後福,獄上,詔削其籍。京復相,徽宗戒毋得用崈但復集英殿修撰,旋還待制,提點洞霄宮。宣和中,卒。



 趙良嗣,本燕人馬植,世為遼國大族,仕至光祿卿。行汙
 而內亂,不齒於人。政和初,童貫出使,道盧溝,植夜見其侍史,自言有滅燕之策,因得謁。童貫與語,大奇之,載與歸,易姓名曰李良嗣。薦諸朝,即獻策曰:「女真恨遼人切骨,而天祚荒淫失道。本朝若遣使自登、萊涉海,結好女真,與之相約攻遼,其國可圖也。」議者謂祖宗以來,雖有此道,以其地接諸蕃,禁商賈舟船不得行,百有餘年矣。一旦啟之,懼非中國之利。徽宗召見,問所來之因,對曰:「遼國必亡,陛下念舊民遭塗炭之苦,復中國往昔之疆,
 代天譴責,以治伐亂,王師一出,必壺漿來迎。萬一女真得志,先發制人,後發制於人,事不侔矣。」帝嘉納之,賜姓趙氏,以為秘書丞,圖燕之議自此始。遷直龍圖閣,提點萬壽觀,加右文殿修撰。



 宣和二年二月,使于金國,見其主阿骨打,議取燕、雲。使還,進徽猷閣待制。自是將命至六七,頗能緩頰盡心,與金爭議,進龍圖閣直學士。既得燕山,又加延康殿學士、提舉上清宮,官至光祿大夫。



 良嗣言:「頃在北國,與燕中豪士劉範、李奭及族兄柔吉三
 人結義同心,欲拔幽、薊歸朝,瀝酒於北極祠下,祈天為約,俟他日功成,即掛冠謝事,以表本心,初非取功名而徼富貴也。賴陛下威靈,今日之事幸而集,顧前日之約豈可欺哉?願許臣致仕,使得買田歸耕,令有識者曰:『此平燕首謀之人,得請閑退,天下美事也。』不然,則臣為敢欺神明,何所不至?」凡三上章,詔不許。既而朝廷納張覺,良嗣爭之云:「國家新與金國盟,如此必失其歡,後不可悔。」不聽。坐奪職,削五階。



 靖康元年四月,御史胡舜陟論
 其結成邊患,敗契丹百年之好,使金寇侵陵,禍及中國,乞戮之於市。時已竄柳州,詔廣西轉運副使李昇之即所至梟其首,徙妻子于萬安軍。



 張覺,平州義豐人也。在遼國第進士,為遼興軍節度副使。鎮民殺其節度使蕭諦里,覺拊定亂者,州人推領州事。燕王淳死,覺知遼必亡,籍丁壯五萬人,馬千匹,練兵為備。蕭后遣時立愛來知州,拒弗納。



 金人入燕,訪覺情狀於遼故臣康公弼,公弼言彼何能為,當示以不疑,乃
 以為臨海軍節度使,任知平州。遼相左企弓等將歸東,粘罕欲先遣兵擒覺,公弼曰:「如此是趣之叛也,我請使焉而觀之。」遂往見覺。覺曰:「契丹八路皆陷,今獨平州存,敢有異志?所以未釋甲者,防蕭幹耳。」厚賂公弼使還。公弼道其語,粘罕信之,升平州為南京,加覺同中書門下平章事。企弓、公弼與曹勇義、虞仲文皆東遷。



 時燕民盡徙,流離道路。或詣覺訴:「公弼、企弓等不能守燕,致吾民如是。能免我者,非公而誰?」覺召僚屬議,皆曰:「近聞天祚
 復振於松漠,金人所以急趨山西者,畏契丹議其後也。公能仗大義,迎故主以圖興復,責企弓等之罪而殺之,縱燕人歸燕,南朝宜無不納。儻金人西來,內用營、平之兵,外藉南朝之援,何所懼乎?」覺又訪于翰林學士李石,亦以為然。乃殺企弓等四人,復稱保大三年,繪天祚像於廳事,每事告而後行。呼父老諭曰:「女真,仇也,豈可從?」指其像曰:「此非汝主乎,豈可背?當相約以死,必不得已則歸中國。」燕人尚義,皆景從。於是悉遣徙民歸。



 石更名
 安弼,偕故三司使高黨往燕山說王安中曰:「平州自古形勝之區,地方數百里,帶甲十餘萬,覺文武全才,若為我用,必能屏翰王室。苟為不然,彼西迎天祚,北通蕭幹,將為吾肘腋患矣。」安中深然之,具奏於朝,願以身任其責,令安弼、黨詣京師。徽宗以手劄付詹度曰:「本朝與金國通好,信誓甚重,豈當首違?金人昨所以不即討覺者,以兵在關中而覺抗榆關故也。今既已東去,他日西來,則覺蕞爾數城,恐未易當。為今之計,姑當密示羈縻足
 矣。」而度數誘致之,諷令內附。



 宣和五年六月,覺遣書至安撫司云:「金虜恃虎狼之強,驅徙燕京富家巨室,止留空城以塞盟誓,緬想大朝,亦非得已。遺民假道當管,冤痛之聲,盈于衢路。州人不忍,僉謂宜抗賊命,以存生靈,使復父母之邦,且為大朝守禦之備,已盡遣其人過界,謹令掌書記張鈞、參謀軍事張敦固詣安撫司聽命。」



 金人聞覺叛,遣闍母國王將三千騎來討,覺帥兵迎拒之于營州,闍母以兵少,不交鋒而退,大書於門,有「今冬復
 來」之語。覺遂妄以大捷聞,朝廷建平州為泰寧軍,拜覺節度使,以安弼、黨、鈞、敦固皆為徽猷閣待制,宣撫司犒以銀絹數萬。詔命至,覺喜,遠出迎。金人諜知,舉兵來,覺不得返,同其弟挾所被詔敕奔燕。母妻先寓營州,為金人所得,弟聞之,亟往降,獻其詔敕。金人圍平州,覺之從弟及侄固守,金人以納叛為責,且求餉糧,凡攻擊數月,州民數千潰圍走,莫肯降。



 金人既平二州,始來索覺,王安中諱之。索愈急,乃斬一人貌類者去。金人曰:「此非覺
 也。覺匿于王宣撫甲仗庫,若不與我,我自以兵取之。」安中不得已,引覺出,數其過,使行刑,覺語殊不遜。既死,函首送之,燕之降將及常勝軍皆泣下,郭藥師曰:「若來索藥師,當奈何?」自是解體,金人終用是啟釁云。



 郭藥師,渤海鐵州人也。遼之將亡,燕王淳募遼東饑民為兵,使之報怨於女真,目曰「怨軍」,藥師為之渠首。明年,其兩營叛,藥師殺叛者羅青。都統蕭幹留二千人為四營,以藥師及張令徽、劉舜仁、甄五臣為將。淳建號于燕,
 改「怨軍」為「常勝軍」,擢藥師至諸衛上將軍、涿州留守。淳死,蕭后立,蕭幹專,國人貳。



 宣和四年九月,藥師擁所部八千人奉涿、易二州來歸,詔以為恩州觀察使。王師北討,劉延慶與幹軍于盧溝,藥師曰:「幹以全師抗我,燕城必虛,選勁騎襲之,可得也。」延慶遣藥師與諸將帥兵六千,夜半渡河,倍道而進。質明,甄五臣領五千騎奪迎春門以入,大軍繼至,下令納燕人降而盡殺契丹雜虜。藥師遣人諭蕭后,使趣降,后密詔蕭幹還戰於三市,藥師
 失馬,幾為所擒,遂以敗還,猶進安遠軍承宣使。十二月,拜武泰軍節度使。五年正月,加檢校少保,同知燕山府。



 詔入朝,徽宗禮遇甚厚,賜以甲第姬妾。張水嬉于金明池,使觀之,命貴戚大臣更互設宴。又召對於後苑延春殿,藥師拜廷下,泣言:「臣在虜,聞趙皇如在天上,不謂今日得望龍顏。」帝深褒稱之,委以守燕,對曰:「願效死。」又令取天祚以絕燕人之望,變色而言曰:「天祚,臣故主也,國破出走,臣是以降。陛下使臣畢命他所,不敢辭,若使反
 故主,非所以事陛下,願以付他人。」因涕泣如雨。帝以為忠,解所御珠袍及二金盆以賜。藥師出,諭其下曰:「此非吾功,汝輩力也。」即剪盆分給之。加檢校少傅,歸鎮。



 蕭幹犯塞,藥師破其眾於峰山,生擒阿魯太師,獲耶律德光尊號寶劍檢、塗金印,幹尋為部下所殺。策勳加檢校太傅。



 初,王安中知燕山府,詹度與藥師同知,藥師自以節鉞,欲居度上。度稱御筆所書有序,藥師不從。加以常勝軍肆橫,藥師右之,度不能制,告於朝廷。慮其交惡,命度
 與河間蔡靖兩易。靖至,坦懷待之,藥師亦重靖,稍為抑損,安中但諂事之,朝廷亦曲徇其意,所請無不從。良械精甲,多遣部曲貿易他道,為奇巧之物以奉權貴宦侍,於是譽言日聞。專制一路,增募兵號三十萬,而不改左衽,朝論頗以為慮。亟拜太尉,召入朝,辭不至。



 帝令童貫行邊,陰察其去就,不然,則挾之偕來。貫至燕,藥師迎于易州,再拜帳下,貫避之,曰:「汝今為太尉,位視二府,與我等耳,此禮何為?」藥師曰:「太師,父也。藥師唯拜我父,焉知
 其他?」貫釋然。遂邀貫視師,至於迥野,略無人跡,藥師下馬,當貫前掉旗一揮,俄頃,四山鐵騎耀日,莫測其數。貫眾皆失色。歸為帝言,藥師必能抗虜,蔡攸亦從中力主之。金使賀天寧節歸,送伴使見藥師兵,遇之于道,金使為之斂馬引避。鄉兵或持矛揭取其羊羜,皆不敢爭,奏言藥師威聲遠振,攸益謂其可倚,故內地不復防制。屢有告變及得其通金國書,輒不省。



 七年十二月,詹度言:「藥師瞻視不常,趣向懷異,蜂目烏喙,怙寵恃功,逆節已
 萌,兇橫日甚。今聞與金人交結,背負朝廷,興禍不遠,願早為之慮。」始詔遣官究實,而金兵已南下破檀、薊,至玉田。蔡靖遣藥師、張令徽、劉舜仁帥師出禦,其夕,令徽遁歸,靖與部使者詣藥師計事,藥師欲降,靖曰:「靖誓死報國,此何言邪?」引佩刀將自剄,藥師抱持之,並諸使者悉鎖于家。斡離不及郊,藥師率軍官迎拜,遂從以南。叛報至,帝猶秘其事,議封為燕王,割地與之,使世守,而已無及。



 斡離不至慶源,聞天子內禪,欲回軍,藥師曰:「南朝未
 必有備,不如姑行。」其後趦趄京城,詰索宮省與邀取寶器服玩,皆藥師導之也。



\end{pinyinscope}