\article{志第七十一 禮二十一(賓禮三)}

\begin{pinyinscope}

 朝儀
 班序百官轉對百官相見儀制



 朝儀班序。太祖建隆三年三月,有司上合班儀:太師,太傅,太保,太尉,司徒,司空,太子太師、太傅、太保,嗣王,郡王,
 左、右僕射,太子少師、少傅、少保,三京牧,大都督,大都護,御史大夫,六尚書,常侍,門下、中書侍郎,太子賓客,太常、宗正卿,御史中丞,左右諫議大夫,給事中,中書舍人,左、右丞,諸行侍郎,秘書監,光祿、衛尉、太僕、大理、鴻臚、司農、太府卿,國子祭酒,殿中、少府、將作監,前任節度使,開封、河南、太原尹,太子詹事,諸王傅,司天監,五府尹,國公,郡公,中都督,上都護,下都督,太子左右庶子,五大都督府長史,中都護,下都護,太常、宗正少卿,秘書少監,光祿等
 七寺少卿,司業,三少監,三少尹,少詹事,左右諭德、家令、率更令、僕,諸王府長史、司馬,司天少監,起居舍人,侍御史,殿中侍御史,左右補闕、拾遺,監察御史,郎中、員外郎,太常博士,五府少尹,五大都督府司馬,通事舍人,國子博士,五經博士,都水使者,四赤令,太常、宗正、秘書丞,著作郎,殿中丞,尚食、尚藥、尚舍、尚乘、尚輦奉御,大理正,太子中允、贊善、中舍、洗馬,諸王友、諮議參軍,司天五官正。凡雜坐者,以此為準。詔曰:「尚書中臺,萬事之本,而班位
 率次兩省官;節度使出總方面,古諸侯也,又其檢校兼守官多至師傅三公,而位居九寺卿監之下,甚無謂也。其給事、諫議、舍人宜降於六曹侍郎之下,補闕次郎中,拾遺、監察次員外郎,節度使升於六曹侍郎之上、中書侍郎之下,餘悉如故。」



 乾德元年閏十二月,詔:「自今一品致仕官曾帶平章事者,朝會宜綴中書門下班。」二年二月,詔復位內外官儀制。有司請令上將軍在中書侍郎之下,大將軍在少卿監之下,諸衛率、副率在東宮五品
 之下,內客省使視太卿,客省使視太監,引進使視庶子,判四方館事視少卿,閣門使視少監,諸司使視郎中,客省引進、閣門副使視員外郎,諸司副使視太常博士,通事舍人從本品,供奉官視諸衛率,殿直視副率,樞密承旨視四品朝官,兼南班官諸司使者從本品,副承旨視寺監丞,諸房副承旨視南省都事。凡視朝官者本品下,視京官在其上。



 開寶六年九月,詔曰:「周之宗盟,異姓為後,此先王所以睦九族而和萬邦也。晉王親賢莫二,位
 望俱崇,方資夾輔之功,俾先三事之列,宜位宰相上。」九年十一月,詔齊王廷美、武功郡王德昭位在宰相上。



 大中祥符元年正月,有司上酺宴班位。駙馬都尉、宮僚、員僚、皇親大將軍已下,行門、宰臣、樞密使已下,穎王、皇親郡王、侍衛馬軍都指揮使已下,皇親使相、皇親節度使、皇親觀察留後已下,皇親防禦、團練、刺史三班合為一;節度使、觀察留後已下,防禦、團練、刺史三班合為一,並重行異位。詔依所定。既而武康軍節度使李端願言:「
 使相亦當合為一班,不當獨行尊異。」詔令閣門再定,而閣門引儀制及以前議為是。端願復伸其議,自劾妄言。乃詔太常禮院與御史臺同詳定。禮院言:「常朝起居班次,緣祖宗舊制,不宜並合。」從之。



 四年閏三月,太常禮院、閣門言:「準詔同詳定閣門使李端愨所奏閣門儀制,宰臣與親王立班坐位分左右各為班首,宰臣、樞密使帶使相,或帶郡王並使相作一行,總為中書門下班。其親王獨行一班者,準封爵令。兄弟皇子皆封國,謂之親王,
 所以他官不可參綴。檢會坐次圖,直將宗室使相輒綴親王,蓋更張之時未見親王,遂致失於講求。近見朝拜景靈宮,東陽郡王顥亦綴親王班,竊恐未安。今取到閣門儀制,其合班宰臣、使相在東,親王在西,分班立。又祥符元年宴坐次圖,宰臣王旦與使相石保吉在東,寧王元偓、舒王元偁、廣陵郡王元儼、節度使惟吉在西,分班坐。其元儼、惟吉是郡王與節度使,許綴親王班,竊慮當時出自特旨。今來檢尋元初文字不見,在先朝只依祥
 符元年宴坐次圖子,親王及帶使相郡王在西為一班。臣等參詳,請依閣門儀制,親王在西,獨為一班,宗室郡王帶使相許綴親王立班坐次,即系臨時特旨。」從之。



 熙寧二年四月,國信所言:「大遼賀同天節左番使耶律奭赴文德殿拜表,言南使到北朝綴翰林學士班,今來卻在節度使之下。館伴者諭之,始就班。時下御史臺、閣門同詳定,奏稱人使不知本朝翰林學士班自在節度使之下,如遇合班即節度使在翰林學士之西差前,別為
 一班立,俱不相壓。欲且依久來儀制體例。」詔依所定。是月,編修閣門儀制所言:「慶歷中,改文明殿學士為觀文殿學士,又置大學士。按文明殿即今文德殿,乃正衙前殿也,後唐始置學士,序位樞密副使之下,每遇紫宸殿坐朝,則升殿侍立。蓋文德、紫宸通謂之前殿,故學士侍立為宜。其觀文殿深在禁中,乃與資政、端明殿相類,而資政、端明學士並不侍立。竊詳慶歷所改職名,雖用舊之班著,而殿之次序與舊義理不同。其觀文殿大學士
 自今遇紫宸殿坐朝,請更不升殿侍立。」從之。



 元祐元年五月,詔:「太師平章軍國重事文彥博,已降旨令獨班起居。自今赴經筵、都堂同三省、樞密院奏事,並序位在宰臣之上。」



 百官轉對。自建隆詔內殿起居日,令百官以次轉對,限以二人。其封章於閣門通進,復鞠躬自奏,宣徽使承旨宣答,拜舞而出,著為閣門儀制。



 淳化二年,詔:自今內殿起居日,復令常參官二人次對,閣門受其章。



 大中祥符末,罷不復行。



 景德三年,復詔:「群臣轉對,其在外京
 官內殿崇班以上,候得替,先具民間利害實封,於閣門上進,方得朝見。」



 治平中,命御史臺每遇起居日,令百官轉對。御史臺言:「舊制,起居日,輪兩省及文班秩高者二員轉對。若兩省宮有充學士、待制,則綴樞密班起居,內朝臣僚不與。」尋詔遇轉對日,增二員。



 熙寧初,閣門言:「舊制,中書省、樞密院奏事退,再引三班,假日則兩班,或再御後殿引對,多及午刻,遇開經筵,即至申末,恐久勞聖躬。請遇經筵日,自二府奏事外,止引一班,或有急奏及
 言事官請對即取旨,俟罷經筵日仍舊。」又言:「假日御崇政殿,每遇辰時,則隔班過延和殿再引,不待進食,至巳刻隔班取旨,尚許引對。請自今隔班過延和殿,俟已進食再引。遇寒暑、大風雨雪即令次日引對。」詔:「自今授外任者許令轉對訖朝辭。」監察御史裏行張戩、程顥言:「每欲奏事,必俟朝旨,或朝政有闕及聞外事而機速後時,則已無所及;況往復俟報,必由中書,萬一事干政府,則或致阻格。請依諫官例,牒閣門求對,或有急奏,即許越
 次登對,庶幾遇事入告,無憂失時。」又以編修閣門儀制所言,三衙有急奏,許於後殿登對,若別有奏陳,則報閣門如常制,或假日御崇政殿,則於已得旨對班後續引,且許兩制以上同班奏事。



 元豐中,詔:「尚書侍郎同郎官一員奏事,郎中、員外郎番次隨之,不許獨留身。侍郎以下,亦不許獨請奏事。其左右選非尚書通領者,聽侍郎以上郎官自隨。秘書、殿中省、諸寺監長官視尚書,貳丞以下視侍郎。」又詔:「三省、樞密院獨班奏事日。無得過三
 班。若三省俱獨班,則樞密院當請奏事。其見任官召對訖,次日即朝辭回任聽旨。」



 元祐中,宰臣呂大防言:「昨垂簾聽政,惟許臺諫以二人同對,故不正之言無得以入。今陛下初見群臣,請對者必眾。既人人得進,則善惡相雜,故於採納尤難。」帝曰:「人君以納諫為上,然邪正則不可不辨。」遂詔上殿班當直牒及帥臣、國信使副,許依元豐八年以前儀制。



 紹聖初,臣僚言:「文德殿視朝輪官轉對,蓋襲唐制,故祖宗以來,每遇轉對,侍從之臣亦皆與
 焉。元祐間因言者免侍從官轉對,續詔職事官權侍郎以上並免,自此轉對止於卿、監、郎官而已。請自今視朝轉對依元豐以前條制。」又詔:「自今三省、樞密院進擬在京文臣開封府推判官、武臣橫行使副、在外文臣諸路監司藩郡知州、武臣知州軍已上,取旨召對。」臣僚言:「每緣職事請對,待次旬日,遇有急奏,深恐失事。請自今後許依六曹、開封例,先次挑班上殿,仍不隔班。」又言:「諸路監司,朝廷所選,以推行法令,省問風俗,朝辭之日,當令
 上殿。」六曹尚書如有職事奏陳,許獨員上殿。其群臣請對,雖遇休假,特御便殿聽納。既又詔:「應節鎮郡守往令陛辭,歸許登對,不特審觀人材,亦所以重外任也。可於監司不許免對條下,增入節鎮郡守依此。」



 重和元年,臣僚言:「比年以來,二三大臣奏對留身,讒疏善良,請求相繼,甚非至公之體。」詔:「自今惟蔡京五日一朝許留身,餘非除拜、遷秩、因謝及陳乞免罷,並不許獨班奏事,令閣門報御史臺彈劾。」又詔:「寺監職事上部,部上省,故得上
 下維持,綱紀所出。今後雖系兩制,職司寺監不許獨對。」臣僚言:「祖宗舊制,有五日一轉對者,今惟月朔行之,有許朝官轉對者,今惟待制以上預焉。自明堂行視朔禮,歲不過一再,則是畢歲而論思者無幾。請遇不視朔,即令具章投進,以備覽觀。」又:「諸路監司未經上殿者,雖從外移,先赴闕引對,方得之官。」並從其議。



 百官相見儀制。乾德二年,詔曰:「國家職位肇分,軌儀有序,冀等威之斯辨,在品式之惟明。矧著位之庶官及內
 司之諸使,以至軒墀引籍,州縣命官,凡進見於宰相,或參候於長吏,既為總攝,合異禮容,稽於舊儀,且無定法。或傳晉天福、周顯德中,以廷臣、內職、賓從、將校,比其品數,著為綱條,載於刑統,未為詳悉。宜令尚書省集臺省官、翰林、秘書、國子司業、太常博士等詳定內外群臣相見之儀。」



 翰林學士承旨陶穀等奏:



 兩省官除授、假使出入,並參宰相,起居郎以下參同舍人。五品以上官,遇於途,斂馬側立,須其過。常侍以下遇三公、三師、尚書令,引
 避;其值僕射,斂馬側立。御史夫地、中丞皆分路行。起居郎以下避僕射,遇大夫,斂馬側立;中丞,分路。尚書丞郎、郎中、員外並參三師、三公、令、僕,郎中、員外兼參左右丞、本行尚書、侍郎及本轄左右司郎中、員外。御史大夫以下參三師、三公、尚書令,中丞兼參大夫,知雜事參中丞,三院御史兼參知雜及本院之長。大夫避尚書令以上,遇僕射,斂馬側立而避。大夫遇尚書丞郎、兩省官諸司三品以上、金吾大將軍、統軍上將軍,皆分路。餘官遇中丞,
 悉引避。知雜兼避中丞,遇左右丞斂馬側立,餘皆分路。郎中及少卿監、大將軍以下,皆避知雜。三院同行,如知雜之例。少卿監並參本司長官,丞參少卿。諸司三品遇僕射於途,皆引避。諸衛大將軍參本衛上將軍。東宮官參隔品。凡參者若遇於途,皆避。



 公參之禮,列拜堂上,位高受參者答焉。四赤縣令初見尹,趨庭,受拜後升廳如客禮。內客省使謁宰相、樞密使以客禮,閣門使以上列拜,皆答,客省副使至通事舍人、諸司使、樞密承旨不答
 焉。自樞密使副、宣徽使皆差降其禮,供奉官、殿直、教坊使副、辭令官、伎術官並趨庭,倨受。諸司副使參大使,通事舍人參閣門使,防禦、團練、刺史謁本道節帥,節度、防禦、團練副使謁本使,並具軍容趨庭,延以客禮。少尹、幕府於本院長官悉拜。防禦、團練判官謁本道節帥,並趨庭。上佐、州縣官見宰相、樞密使及本屬長官,並拜於庭天長、雄武等軍使見宰相、樞密亦知之。參本府賓幕官及曹掾,縣簿、尉參令,皆拜。王府官見親王如賓職見使長,府縣官兼三館
 職者見大尹同。赤縣令、六品以下未嘗參官,見宰相、樞密及本司長官,並拜階上。流外見流內品官,並趨庭。



 諸司非相統攝,皆稱移牒。分路者不得籠街及占中道,依秩序以分左右。遇於驛舍,非相統攝及名位縣隔,先至者居之。臺省官當通官呵止者,如舊式。文武官不得假借呼稱,以紊朝制。當避路者,若被宣召及有所捕逐,許橫度焉。



 又令:「諸司使、副使、通事舍人見宰相、樞密使,升階、連姓通名展拜,不答拜。其見樞密副使、參知政事、宣
 徽使,以客禮展拜。」



 太平興國以後,又制京朝官知令錄者,見本使州長吏以客禮,三司判官、推官、主判官見本如郎中、員外見尚書丞郎之儀。



 咸平中,又詔:開封府左右軍巡使、京官知司錄及諸曹參軍到畿縣見京尹,並趨庭設拜。六年,命翰林學士梁顥等詳定閣門儀制,成六卷,因上言:「三司副使序班、朝服比品素無定列,至道中,筵會在知制誥後、郎中前。今請同諸司、少卿監,班位在上。如官至給諫、卿監者,自如本品,朝會大宴隨判使
 赴長春殿起居引駕。其朝會引駕至前殿,與諸司使同退。」



 大中祥符五年,復命翰林學士李宗諤等詳定儀制:文武百官遇宰相、樞密使、參知政事,並避。起居郎以下遇給、舍以上,斂馬。御史大夫遇東宮三師、尚書丞郎、兩省侍郎,分路而行。中丞遇三師、三少、太常卿、金吾上將軍,並分路而行。知雜御史遇尚書侍郎、諸司三品、金吾大將軍、統軍、諸衛上將軍,分路而行。三院同行如知雜例,不同行,遇左右丞則避。尚書丞郎、郎中、員外遇三師、
 三公、尚書令,則避。郎中、員外遇丞郎,則避。太常博士以下朝官遇本司長官、三師、三公、僕射、尚書丞郎、大夫、中丞、知雜御史,並避,權知判者不避,遇兩省給舍以上,斂馬。京官遇丞郎、給舍、大卿監、祭酒以上及本寺少監卿、司業,並避。諸軍衛大將軍以下遇上將軍、統軍,亦避。詹事遇上臺官,如卿監之例。庶子、少詹事至太子僕遇東宮三師、三少,並避;遇上臺官,如少卿監例。中允以下遇東宮三師、三少,並避;遇賓客、詹事,斂馬;遇上臺官,如太
 常博士例。應合避尚書者,並避三司使。其權知開封府如本官品避。其臺省官雖不合避,而職在統臨者,並避。武班、內職並依此品。



 大觀二年,定王、嘉王府侍講沉錫等奏:「二王出就外學,其初見及侍王禮儀、講說疏數之節,請如故事。」手詔:「按祥符故事,記室翊善見諸王,皆下拜。真宗特以張士遜為王友,命王答拜,以示賓禮。今講讀輔翊之官,職在訓道,亦王友傅也,可如例,令王答拜。」群臣赴臺參、謝、辭者新授、加恩、出使者,尚書侍郎則三院御史
 各一員、中丞、大夫皆對拜三院仍班迎,不坐班即不赴。節度使、賓客、太常宗正卿則御史一員、中丞、大夫皆對拜。兩使留後至刺史、秘書監至五官正、上將軍至郎將、四廂都指揮使及內職軍校遙郡以上、樞密都承旨及內職帶正員官者、四赤縣令、三京司錄、節度行軍至團練副使、幕職官任憲銜者,皆御史一員對拜,中丞、大夫對揖亦令揖訖進言,得參風憲,再揖而退。若曾任中書、門下及左右丞皆不赴。加階勛、食邑、章服,館閣三司、開封府職事及內職轉使額、軍額,
 亦不赴臺謝。僕射過正衙日,臺官大夫以下與百官,並詣幕次致賀文官一品、二品曾任中書、樞密院者不赴。大夫、中丞則郎中、少卿監、大將軍以下亦然本官約止則不赴,僕射赴上都省者罷此儀。



\end{pinyinscope}