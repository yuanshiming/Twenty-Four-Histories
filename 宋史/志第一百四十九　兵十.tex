\article{志第一百四十九 兵十}

\begin{pinyinscope}

 遷補之制屯戍之制



 遷補之制自殿前、侍衛馬步軍校,每遇大禮後,各以次遷,謂之「轉員」。轉員至軍都指揮使,又遷則遙領刺史,又遷為廂都指揮使,遙領團練使。員溢,即從上罷軍職,為
 正團練使、刺史之本任,或有他州總管、鈐轄。其老疾若過失者,為御前忠佐馬軍都軍頭、副都軍頭,隸軍頭司。其黜,則為外州馬步軍都指揮使。凡軍主闕,以軍都指揮使遞遷;餘闕,以諸軍都虞候、指揮使、副指揮使、行首、軍使、副行首、副兵馬使、十將遞遷。凡將校,一軍營止補十人,其廂都指揮使、軍都指揮使、都虞候、指揮使,營主其一,即闕其三。殿前左右班都虞候遙領剌史,即與捧日軍都指揮使通,以次遷捧日、龍衛廂都指揮使,仍遙領
 團練使。若員溢,即為正刺史補外,他如諸軍例遞遷。



 凡列校轉補,有司先閱走躍、上下馬;次出指二十步,掩一目試之,左右各五占數為見物。武藝,弓射五斗,弩擴一石五斗,槍刀手稍練。負罪不至徒,年未高,或雖年高而無疾、精力不耗者,並取之。



 凡諸軍轉員後,取殿前指揮使長入祗候填行門,取東西班長入祗候、殿侍、諸班直充諸班押班、諸軍將校者,皆親閱。前一日,命入內都知或押班一人、勾當御藥院內侍一人,同軍頭引見司較
 定弓弩鬥力,標志之。凡弓弩藝等者,人占其一。至日,引見,弓弩列置殿前,命取一以射。軍頭引見司專視喝箭以奏。如喝失當,即奏改正。入內都知或押班同勾當御藥院內侍殿上察視,如引見司不覺舉,亦奏改正。槍刀手竭勝負,若喝不以實,並引見司失覺舉,並劾其罪。



 太平興國九年,上詣崇政殿轉改諸軍將校,自軍都指揮使以下、員僚以上,皆按名籍驗勞績而升陟之,凡數日而畢。內外感悅。乃謂宰臣等曰:「朕遷轉軍員,先取其循
 謹能御下者,武勇次之。若不自謹飭,則其下不畏憚,雖有一夫之勇,亦何所用!」



 咸平三年五月,上御便殿遷補軍職,凡十一日而畢。自神衛右第二軍都指揮使、恩州剌史周訓而下,遞遷者千三十一人。



 四年十二月,帝謂呂蒙正曰:「選眾求才,誠非易事。朕常孜孜詢訪,冀有所得。向求於軍校中,超擢八九人,委以方任,其間王能、魏能頗甚宣力,陳興、張禹珪亦有能名。」蒙正等曰:「才難求備。今拔十得五,有以見陛下知臣之明也。」



 五年,帝謂知
 樞密院周瑩曰:「國朝之制,軍員有闕,但權領之,三歲一遷補。未及期以功而授,止奉朝請而已。今闕員處則乏人部轄。須當例與轉補。」於是召瑩等至便殿,按軍籍次補,其屯戍於外及軍額在下素不該恩例,亦溥及之。凡再旬方畢。



 景德二年四月,帝曰:「殿前諸班、侍衛馬步諸軍及軍頭司諸軍員,因衰病或以他事出補外職,率皆臨事奏裁,殊無定制,可條其所入職名類例以聞。」又曰:「近累有諸處立功指揮使,未可別加遷擢,皆特補本軍
 都虞候。舊無此職名,蓋權宜加置,若後有闕,不須復補。」又曰:「內外諸軍所闕小校,儻以名次遷補,或慮不能盡得武干之士,自今並令閱試武藝,選擢為之。」



 大中祥符四年七月,詔曰:「自來轉補軍員,皆是議定降宣命訖,方引見轉補。其間有老病不任職者,臨時易之,無由整齊。經汾陰大禮,應殿前馬步軍諸班諸軍員,並分作甲次於崇政殿逐人唱名引見,朕自視之。有不任職者,當於不系禁軍處優與安排,免轉員之際,旋議改易。」八月,詔:「
 殿前、侍衛馬軍步軍司所管內外禁軍軍員,自來補轉,體例不一,未得均平。朕夙夜思之。今來該汾陰轉員,可立定久遠規制。其馬軍、步軍,自指揮使以下,各別轉補,皆令自下而升。仍將殿前、侍衛馬軍步軍司所轄軍分,各袞同轉補。如馬軍軍員自近下補至拱聖,即雙取之,以分補捧日、龍衛,其近下軍分有闕,即卻自捧日、龍衛雙取,升一員資補填。其步軍有闕,填補並準此。」又詔:「所議改更轉補軍員職名,恐諸軍未喻,可降宣命云:殿前、
 侍衛馬步軍司自來多是龍衛更轉入捧日,並神衛更入天武之類,是致難得出職,久成沉滯。今來轉員,出自朕意,並各與分兩頭遷改,其龍衛更不入捧日,並神衛更不入天武。其捧日、龍衛闕,於拱聖內隔間取人,分頭充填。其拱聖闕,即將驍騎、雲騎分頭轉入。其天武、神衛闕,於神勇內隔間取人,分頭充填。其神勇闕,即將宣武充填。其宣武闕,取殿前、步軍司虎翼充填。已上如取盡指定軍員,即轉已次軍員充填。所有寧朔軍分次第請
 受並轉員出入,今後並特與依驍勝體例施行。」



 六年十月,詔:「諸班直並馬步軍事軍員,其諸班、捧日、龍衛、天武、神衛五頭下出人外,其御龍諸直作一處轉;員僚直、拱聖、驍騎、雲騎、驍勝、武騎、寧朔、神騎已上軍額軍員,作一處挨排遞遷;水軍神勇、宣武,殿前司虎翼、衛聖,步軍司虎翼、奉節、廣勇、神射已上軍額軍員,作一處挨排轉補;事內殿前指揮使押班至都知只本班轉,其神衛、廣勇、神射已下至軍使、都頭,即逐指揮內遞遷。內有年及六
 十已下者並勾押赴闕,令殿前司看驗聞奏,當議相度安排。所有副兵馬使、副都頭員闕,仍取捧日、龍衛、神勇十將充填,餘並從之。內神衛水軍第一指揮,令立充神衛水軍指揮;殿前司上虎翼第二、步軍司上虎翼第一,並立充虎翼水軍指揮,依舊系逐司管押。其神衛水軍見管軍員,先自奉節補入,多不會舟楫,並一齊轉上外,卻將虎翼水軍兩指揮會水軍員與神衛水軍共三指揮一處袞轉。如轉至神衛水軍指揮使,除年老病患依
 例出職安排外,更不轉上。」



 天禧元年十月,以御前忠佐郭豐等六人並受將軍。初,軍頭司定年老負犯者將黜之。帝以其久居武列,命寘環衛,其帶遙郡者與大將軍,不帶遙郡者與將軍。



 天聖六年,將轉員,樞密院奏:「諸軍將校有因循不敢戢士者,請諭殿前、馬步軍司密以名聞。」八年,詔殿前、侍衛司同定內外諸軍排立資次。



 景祐二年,詔緣邊就糧兵有員闕,奏以舊人次遷。



 康定元年,詔三路就糧將校半以次遷,半遣自京師。又詔陜西土
 兵校長遣自京師,情不諳達,自今悉就本路通補。



 慶歷四年,詔捧日、天武選退將校超三資,餘超二資,悉補外職。五年,真定府、定州路都總管司奏:「奉詔閱教軍士,選補階級,弓射九斗至一石,距堋七十步至百步,射最親者為第一等。其閱教時,弓不必引滿,力競即發,務在必中。伏緣舊例軍中揀節級,以挽強引滿為勝。今一旦取射親者為第一等,其弓力止九斗、一石,箭留三兩指,而退素習挽強引滿之士,於理未便。」詔諸軍選節級用舊
 例,遇閱教即如近制。



 皇祐元年,詔:諸路就糧兵闕將校,須轉補滿三年聽遷。又詔:將帥麾下兵,非有戰功,毋得請遷隸上軍。



 嘉祐二年,詔:京東教閱本城、騎射、威邊、威勇、壯武,自初募置,即給鼓旗閱教以代禁軍,如有員闕,聽遞遷至副指揮使止,轉補後滿三歲,闕三分已上即舉行。其指揮使闕,即步軍司補之。



 至和三年,詔親從官入殿滿八年者補節級,從樞密院之請也。



 治平元年,遷諸班直長行洎禁軍副兵馬使已上有材武者,得七十
 人,帝臨軒親閱。喻天武右第三軍都指揮使王秀曰:「爾武藝雖不中格,而有戰功,且能恪守法度,其以爾為正刺史,務勤乃職,無負朕之委寄也。」又喻散直都虞候胡從、內殿直副都知張思曰:「爾能勤以持身,忠以事上,治軍又皆整肅,其以從為內園使,思為崇儀副使。」自餘擢遷有差。



 二年,詔:「廣南教閱忠敢、澄海,一營者即本營遞遷,兩營已上者,營三百人補五人,二百人至三百人補三人,二百人以下補二人,百人以下補一人,止於副指
 揮使。凡遞遷滿三歲,五階闕二、三階闕一即補。」四年,詔:「自今一營及二百五十人已上置校十人,闕三人即補。二百五十人已下置校七人,闕二人即補。京師非轉員並諸道就糧並準此令。」



 凡軍頭、十將、節級轉補,謂之「排連」,有司按籍閱試,如列校轉員法。弓射六斗、弩擴一石七斗、槍刀手稍練並取之。如舊不試武技者,即遞遷。其不教閱廂軍節級,則其半遞遷,其半取伉健未嘗犯徒刑、角力勝者充。



 治平四年,有司言:「軍士闕額多而將校
 眾,請以實領兵數制將校額,第其遷補,並通領五都之事。」乃詔:「二百五十人以上,補指揮使十人,以下七人,闕二人者以次補。補十將者,馬軍四十人,步軍如馬軍之數而加其一焉。百五十人以上者三十人,闕五人者以次補。不及百五十人者,如舊格,補單將二十人。」



 熙寧二年,樞密院請:「自今捧日、龍衛、天武、神衛廂都指揮使闕,無當次遷者,並虛之。其諸軍都指揮使、都虞候當遷者,闕多則間一名補轉,兼以次職事。吐渾等軍都指揮使、
 都虞候闕者,虛其闕。」六月,詔:「河東、陜西就糧軍士將校,其間材效之人,孤遠無由自達,有司審度其有軍功驍勇者以名聞,當擢寘班行,以備本路任使。」



 四年,詔:「諸班直嘗備宿衛,病告滿尚可療者,殿前指揮使補外牢城指揮使,餘以為捧日、天武第五軍押營,奉錢三千者予五百,二千以下者予三百。」



 六年,詔:「軍校老而諳部轄者優假之;雖疾不至罷癃,或未七十猶堪任事者勿罷。即法雖當留而不能部轄者以聞,當議處之廂軍。」二月,詔:「
 軍士選為節級,取兩嘗有功者,功等以先後,又等以重輕,又等以傷多者為上。」



 七年,詔:「十將以下當轉資而不欲者,凡一資,以功者賜帛十五匹,技優者十匹。」六月,詔:「在京轉員諸軍都虞候已上至軍都指揮使,以軍功當遷而願以授子孫者聽,視其秩有差。」



 八年,轉員,帝親閱,凡三日。舊制,捧日都虞候四人,至是,補者五人,而馬軍都指揮使闕驍騎二人,以捧日一人補驍騎軍主,餘四人如故,則次軍皆不得遷,乃補四人者皆為馬步軍副
 都軍頭。舊龍衛、拱聖、驍騎、武騎、寧朔、神騎為一百三十一營,今省五十營,而馬軍指揮以下已補八十一營,補外尚有溢員,乃詔所省營未移並者凡四十三,每營權置下名指揮使、副指揮使各一,軍使三,以便遞遷。



 九年,將轉員,樞密院奏:「換官稍優,軍校由行伍有功,不久乃至團練使。」帝曰:「祖宗以來,軍制固有意。凡隸在京殿前、馬步軍司所統諸營,置軍都指揮使、都虞候分領之。凡軍事,止責分領節制之人。責之既嚴,則遇之不得不
 優。至若諸路,則軍校不過各領一營,不可比也。」吳充等以本大末小對,帝然之。因言:「周室雖盛,成、康之後,浸以衰微。本朝太平百有餘年,由祖宗法度具在,豈可輕改也?」



 元豐元年,詔禁軍排連者三分其人,以其一取立功額外人,二分如令簡試。十二月,詔諸軍軍使、都頭以下並充兵額,正副指揮使以上置於額外,軍行則分押諸隊。又詔:「內殿直以下諸班直闕,按籍闕二分者虛其闕四之一,二分以上亦如之,不及二分補其半,餘並闕之。」



 四年,詔:「五路袞轉土軍與諸路不袞轉禁軍法,十將、副都頭、副兵馬使、都頭、軍使並如令。自副都指揮使至都虞候嘗轉資者,間以賜帛,已賜帛乃遷。」



 五年,詔以諸路教閱廂軍為下禁軍,排連如禁軍法。



 七年,樞密院言:「騎軍諸營、諸班直以年勞升至軍使者甚眾,無闕可補。」詔捧日、龍衛、拱聖、驍騎、雲騎、驍勝權置下名軍使,凡二百四十員,拱聖、驍騎、雲騎權置副兵馬使,凡九十員以處之。



 元祐元年,樞密院奏:「諸軍將年七十,若有疾,假滿百
 日不堪療者,諸廂都指揮使除諸衛大將軍致仕;諸軍都指揮使、諸班直都虞候帶遙郡除諸衛將軍致仕;諸班直、上四軍除屯衛,拱聖以下除領軍衛:仍並以有功勞者為左,無功勞者為右。」從之。



 二年,樞密院言:「舊例,行門對禦呈試武藝,並臨時特旨推恩,前期未嘗按試,至日旋乞增加斗力,或涉唐突,因以抵罪,請於轉員前一日按定鬥力。」從之。四月,樞密院言:「舊例,諸班直長行補諸軍員僚,並取入班及轉班二十年、年四十以上人。迨元
 豐四年,以闕額數多,乃特詔減五年,系一時之命。今諸軍員僚溢額,儻不定制,即異時遷補不行;若便依限年舊法,又慮未有合該出職之人。請於三次漸次增及舊例年限。」從之。



 五年,樞密院言:「轉員馬軍指揮使以下至副兵馬使,人數溢額,轉遷不行。」詔權置下名軍使一百七十人,副兵馬使一百七十五人。又言:「禁軍大閱,請以匹帛、銀楪支賜,罷轉資。」從之。六年,又言:「應排連長行充承局、押官者,先取年五十五以下、有戰功公據者,仍
 以戰功多少、得功先後、傷中輕重為次,事等而俱無傷中,則以事藝營名為次。」從之。



 紹聖二年,詔:「將來轉員換前班人,並從元豐轉員令,仍不得過一百二十人。元祐所限人數比試家狀指揮勿用。」



 三年,樞密院進呈轉員及行門試武藝、換前班、留住等條例。曾布言:「國初以來,皆面問其所欲,察相人才,或換官,或遷將校,或再任,此則威福在人主。以至唐突,或放罪,或行法,亦視其情狀而操縱之。元祐改法,乃令大閹與三司、軍頭司先指試
 定,但對御引呈,依拍定等第推恩,殊失祖宗馭眾之法。不許唐突,例坐徒罪兼決責人員,皆非舊法。唐突人雖有理,亦不施行。緣情輕者放罪,重者取旨,自有舊格。先朝燕達、林廣嘗唐突當降配,先帝釋之,後皆為名將。至情重則杖脊配嶺表者,有王明者住留叫呼,云:『若不得換前班,乞納命。』管軍賈逵乞重配,先帝亦貸之,但降一等,與換外官。如此,故人知恩威皆自人主出,豈可一切付之有司?」帝悅,詔令並依元豐以前條例施行。



 五年,馬
 步軍司言:「三路袞轉軍員,請依元豐七年詔,『應三月一日後續有得功嵌補升名並改轉名職自充下名者,並依先補名次,各理降宣月日以為高下,審會給據,候再經袞轉,即依嵌補升轉名次高下轉那。』自今三路軍員袞轉亦如之。」詔侍衛馬、步軍司,自今開具合轉補職名申樞密院降宣,餘並從之。七月,軍頭司引見殿前、馬步軍司揀到御龍諸直人材事藝應格,並補逐直將虞候,賜杖子。一名開弓偃身不應法,黜之。



 八月,樞密院言:



 《轉
 員旁通格》:「捧日、天武不帶遙刺軍都指揮使,換左藏庫使,仍除遙刺;殿前班不帶遙刺都虞候,換左藏庫使。」看詳殿前班帶遙郡都虞候,系與捧日帶遙郡軍都指揮使理先後相壓轉遷;其不帶遙郡殿前班都虞候、捧日軍都指揮使換官班,合一等推恩。欲殿前班不帶遙郡都虞候,依捧日不帶遙郡軍都指揮使換官。



 又拱聖、神勇與驍騎已下軍分有異,其逐軍都虞候、指揮使理難一等換官。欲拱聖、神勇都虞候依舊換供備庫使外,
 驍騎、雲騎、宣武都虞候換左藏庫副使,拱聖、神勇指揮使換內殿承制。捧日、天武、神、龍衛指揮使皆系上四軍,其捧日、天武換西京左藏庫副使,龍、神衛換內殿承制,比捧日、天武隔兩官,理有未均,欲神、龍衛指揮使換供備庫副使。



 又殿前班上名副都知換供備庫副使,下名副都知換內殿承制,自來以左右第一、第二班為資次,欲第一班換供備庫副使,第二班換內殿承制。



 又:「換前班差遣,州總管以下,並以五路緣邊為優,諸路為次。正團
 練使,州總管;正刺史,州鈐轄;諸司使副,都巡檢使、駐泊都監;內殿承制、崇班,巡檢、州都監;供奉官至借職,教押軍隊指使。」看詳諸司使、副已上差遣,見依格施行外,承制以下,欲依今來轉員所差遣例。



 又:「拱聖、神勇、驍騎、雲騎、宣武軍都指揮使換文思,仍除遙刺,已帶者依舊;御龍直都虞候,文思使,帶遙刺者依舊;內殿直兩次都虞候換左藏庫使,一次文思使,帶遙刺者依舊。」看詳拱聖、神勇與驍騎以下軍分有異,兼御龍直都虞候遇轉員
 合次神勇軍都指揮使轉行,及系環衛諸直人員最上名人,兼內殿直都虞候以次殿前班,及轉員無闕,合隨龍衛軍都指揮使轉行,理難於驍騎、雲騎、宣武軍都指揮使之下換官。欲御龍直、內殿直都虞候依格合換官外,並除遙刺;驍騎、雲騎、宣武軍都指揮使止與換文思使,更不除遙郡刺史,內已帶遙刺者並依舊。內殿前班副都知並與換供備庫副使。



 今馬步軍諸指揮事藝高強十將引見,取揀充員僚,內弓箭手短一指箭人合降
 一軍安排;弩手括不發,事體頗同,並弩手墜箭與括不發亦同,欲並降一軍安排。



 從之。



 十一月,樞密院言:「《轉員旁通冊》內御龍直都虞候至都頭副換官,惟指揮使上兩直與文思副使系降兩資,餘止降一資,散員至金槍都知、副都知皆換內殿承制,不惟職名有差,自副都知約六遷方轉都知;兼東西班、散直、鈞容直系近下班分,副都知亦降都知一等換內殿崇班。其東西班、散直押班與副都知職名不等,兩經轉遷,方入近下班分副都
 知,理難與都知一等換內殿崇班。又散指揮至鈞容直指揮使並換供備庫副使,緣東西班、散直、鈞容直遇轉員,止是遷入上班,亦難一等換官。」詔:「御龍下兩直指揮使換左藏庫副使,散員、散指揮、散都頭、散祗候,金槍都知換供備庫副使,東西班、散直押班換東頭供奉官,東西班指揮使換官依舊外,散直、鈞容直指揮使換左藏庫副使。」緣《轉員旁通冊》內未載雲、武騎軍都指揮使轉遷換官並恩例等,詔並依驍騎軍都指揮使格。



 四年二
 月,軍頭司引見捧日等兵試藝,帝於行間召邢斌、韓扆問曰:「開弓猶有餘力乎?」各對願增二石二斗弓。遣內待監定鬥力授之。射皆應法,並特充殿前指揮使,賜緡錢。



 元符元年七月,樞密院言:「將校、軍頭、十將各轉補者,委本將體量,不掩眼試五次,二十步見,若一次不同,減五步,掩一眼再試。但兩眼共見二十步,或一眼全不見二十步,仍試上下馬。如無病切,弓射五斗,弩踏一石五斗,槍刀、標手各不至生疏,並與轉補。即有病切,或精
 神尪悴,或將校年六十九,或經轉補後犯奸盜贓罪情罪重以上雖該降,並隔下奏聽旨。如差出者勾赴本將體量,在別州者,報所在州體量。排連長行充承局、押官者,先取年五十五以下、有兩次以上戰功人填闕,六人更取一名;餘取年四十以下、武藝高強、無病切人,試兩眼各五次,二十步見者選拍。內步軍以闕六分為率,先取弓手一分,次取弩手三分,次取槍牌刀手二分,更有零分者依六分為率,資次取揀,周而復始。長行犯徒經決
 及二年,或軍人因犯移配杖罪經三年、徒罪經四年,或已升揀軍分又經一年,各無過犯,並聽排連。不應充軍人,已投狀後,審會取放逐便,雖未給公憑,其請給差使並罷,有違犯,加凡人二等。不應充軍人,於法許逐便者,並追納元請投軍例物訖,報合屬去處,給公憑放逐便。如非品官之家,無例物回納,願依舊充軍者聽。」從之。



 三月,禮部言:「檢會故事,臣僚申請諸州軍府管押進奉衙校等,祖宗以來,並加散官。自更官制,階散並罷。既罷階
 散,若與轉資,似屬太優。欲每轉一資,支賜絹二十匹。如一名管押兩處,只許就一處支給。或一州一軍差二人同押,亦共與上件支賜。若一員官兩處進奉,只隨本官合推恩處從一支給。今押進奉皇帝登寶位禮物衙校等,欲依故例施行。」並從之。



 宣和七年十一月,南郊,制:「應軍員送軍頭司未得與差遣者,如後來別無過犯,卻與差遣。應廂軍人員補職及十五年未經遷補者,令所屬保明聞奏。應禁軍、廂軍因一犯濫情重不得補充人員
 及遞遷資給者,若經斷及五年不曾再犯,及不曾犯贓,委所在候排連日審實,特與不礙遷補。」



 建炎、紹興之間,排連、轉員屢嘗損益,而大率因於舊制。



 乾道六年,主管侍衛馬軍司公事李顯忠言:「本司諸兵將官有闕,自來擇眾所推者,不以次序上聞升遷。比年須自訓練官充準備將,準備將及二年升副將,副將及二年升正將,正將及三年升統領官,再及三年升統制官,竊恐無以激揚士氣。請今後兵將官有闕,不以年為限,許本司銓量
 人材膽勇服從上聞補用。」詔從其請。此誠砥礪兵將之良法也。



 嘉定中,樞密院言:



 諸軍轉員遷補,務在均一。如內諸班直循舊格排連,積習既久,往往超躐升轉,後名反居前列,高下不倫,甚失公平之意。



 今參酌前後例格,均次資序:其一曰,內殿直左第一班副都知轉東西班西第二都知,內殿直左第二班副都知轉散直左班都知。其二曰,散員左第二班副都知升內殿直左第一班副都知,散員右第一班副都知升內殿直左第一班副
 都知。其三曰,散員右第一班副都知升內殿直右第一班副都知,散中左第二班副都知升內殿直右第二班副都知。其四曰,散指揮左第一班副都知升散員左第一班副都知,散指揮右第一班副都知升散員右第一班副都知。其五曰,散指揮左第二班副都知升散員左第二班副都知,散指揮右第三班副都知升右第二班副都知。其六曰,散都頭左班副都知升散指揮左第一班副都知,散都頭右班副都知升散指揮右第一班都
 知。其七曰,散祗候左班副都知升散指揮左第一班副都知,散祗候右班副都知升散指揮右第二班副都知。其八曰,內殿直左第一班押班遷轉東西班西第一班副都知,內殿直右第一班押班轉東西班西第三班副都知。



 以上各系升四名外,御龍直御龍左第一直十將轉御龍弓箭直副都頭,御龍直右第一直十將轉御龍弩直副都頭,御龍骨朵子直左第一直十將升御龍左第一直十將,御龍弩直左第一直十將升御龍弓箭左
 第三直十將,系各升六名。



 於是超躐積習之弊盡革,而為定制焉。



 淳祐十一年,御史臺條奏軍功賞格違法之弊:「在法,邊戍獲捷、奇功、暴露、撤戍者,制閫、軍帥舉奏授官,必其人身親行陣,有戰御功。今自守闕進勇副尉至承信郎、承節郎者,其弊尤多,乃以奉權要,酬私恩,或轉售於人。方等第功賞之初,即竄名其中,朝廷審核,動涉歲年,已無稽考。甚至承受、廳吏、廝卒之流,足跡未嘗出都門,而沾親冒矢石、往來軍旅之恩,授以名器。請申嚴
 帥閫,令立功人親授告身,庶革冒濫。」



 寶祐五年,樞密院言:「應從軍職事,必立戰功,並隊伍中人曾經拍試武藝;若訓練官以遞而升者,或年限未及仍帶『權』字,俟年及方升正統制,此定法也。近年任子、雜流冒授者,才無差遣,便請從軍,繇統領至總管,曾幾何時,超躐而進。甫得總管,卻恥軍職,輒稱私計不便,或托父母老疾,巧計離軍,又以筋力未衰,求差正任,甚非法意。」



 至咸淳中,大將若呂文德、夏貴、孫虎臣、範文虎輩,矜功怙寵,慢上殘下,
 行伍功賞,視為己物,私其族姻故舊,俾戰士身膏於草莽,而奸人坐竊其勛爵矣。



 屯戍之制凡遣上軍,軍頭司引對,賜以裝錢。代還,亦入見,犒以飲食,簡拔精銳,退其癃老。至於諸州禁、廂軍亦皆戍更,隸州者曰駐泊。戍蜀將校,不遣都虞候,當行者易管他營。凡屯駐將校帶遙郡者,以客禮見長吏,餘如屯駐將校。凡駐泊軍,若捍禦邊寇,即總管、鈐轄共議,州長吏等毋預。事涉本城,並屯駐在城兵馬,即知州、都監、
 監押同領。若州與駐泊事相關者,公牒交報。凡戍更有程:京東西、河北、河東、陜西、江、淮、兩浙、荊湖、川峽、廣南東路三年,廣南西路二年,陜西城砦巡檢並將領下兵半年。



 景祐元年,詔:「若聞陜西戍卒,多為大將選置麾下,及偏裨臨陣,鮮得精銳自隨。自今以全軍隸逐將,毋得選占。」三年,詔廣、桂、荊、潭、鼎、澧六州各置雄略一營,與歸遠軍更戍嶺外。



 康定元年,頒銅符、木契、傳信牌。銅符上篆刻曰「某處發兵符」,下鑄虎豹為飾,而中分之。右符五,左
 旁作虎豹頭四;左符五,右旁為四竅,令可勘合。又以篆文相向側刻十干字為號:一甲己,二乙庚,三丙辛,四丁壬,五戊癸。左符刻十幹半字,右符止刻甲己等兩半字。右五符留京師,左符降總管、鈐轄、知州軍官高者掌之。凡發兵,樞密院下符一至五,周而復始。指揮三百人至五千人用一虎一豹符,五千人已上用雙虎豹符。樞密院下符以右符第一為始,內匣中,緘印之,命使者繼宣同下,雲下第一符,發兵與使者,復緘右符以還,仍疾置
 聞。所在籍下符資次日月及兵數,毋得付所司。



 其木契上下題「某處契」,中剖之,上三枚中為魚形,題「一、二、三」,下一枚中刻空魚,令可勘合,左旁題云「左魚合」,右旁題云「右魚合」。上三枚留總管、鈐轄官高者掌之,下一枚付諸州軍城砦主掌之。總管、鈐轄發兵馬,百人已上,先發上契第一枚,貯以韋囊,緘印之,遣指揮繼牒同往。所在驗下契與上契合,即發兵,復緘上契以還,仍報總管、鈐轄。其發第二、第三契亦如之。掌契官籍發契資次日月及
 兵數以為驗。



 傳信牌中為池槽,藏筆墨紙,令主將掌之。每臨陣傳命,書紙內牌中,持報兵官,復書事宜內牌中而還。主將密以字為號驗,毋得漏洩軍中事。



 呂夷簡言:「自元昊反,被邊城砦各為自守計,萬一賊有奔沖,即關輔驚擾。雖夏竦等屯永興,其實兵少。自永興距鄜延、環慶諸路,皆數百里,設有急緩,內外不能相救。請募勇敢士三萬,訓以武技,分置十隊,以有謀勇者三人將之,分營永興。西寇至,則舉烽相應,或乘勢討擊,進退不以地
 分,並受夏竦等節制。」詔從之。初,趙元昊反,以夏竦、陳執中知永興軍,節度陜西諸軍,久之無功。乃析秦鳳、涇原、環慶、鄜延為四路,以秦、渭、慶、延知州分領本路馬步軍。是歲,罷銅符、木契。詔曰:「陜西屯重兵,罄本路租稅,益以內庫錢帛,並西川歲輸,而軍儲猶不足。宜度隙地為營田務,四路總管、轉運悉兼領使。」



 慶歷二年,詔:「已發士三萬戍永興,委總管司部分閱教。歲以八月遣萬五千人戍涇、原、儀、渭州、鎮戎軍,十二月以萬五千人代,至二月
 無警即還,歲以為常。」葛懷敏等喪師,命範仲淹、韓琦、龐籍復統四路,軍期中覆不及者,以便宜從事。四年,夏人已納款,乃罷。四月,帝謂輔臣曰:「湖廣擊蠻吏士,方夏瘴熱,而罹疾者眾,宜遣醫往為胗視。」



 六年,詔:「騎軍以盛夏出戍,馬多道死。自今以八月至二月遣發。」又詔:「廣南方春瘴癘,戍兵在邊者權休善地。其自嶺外戍回軍士,予休兩月。」李昭亮上言:「舊制,調發諸軍先引見,試以戰陣,遷補校長。今或不暇試戰陣,請選強壯有武技者,每十
 人引見轉資後遣。」詔可。



 時契丹使來議關南地,朝廷經制河北武備,議者欲增兵屯。程琳自大名府徙安撫陜西,上言曰:「河朔地方數千里,連城三十六,民物繁庶,川原坦平。自景德以前,邊數有警,官軍雖眾,罕有成功。蓋定州、真定府、高陽關三路之兵,形勢不接,召發之際,交錯非便。況建全魏以制北方,而兵隸定州、真定府路,其勢倒置。請以河朔兵為四路,以鎮、定十州軍為一路,合兵十萬人;高陽關十一州軍為一路,合兵八萬人;滄、霸
 七州軍為一路,合兵四萬人;北京九州軍為一路,合兵八萬人。其駐泊鈐轄、都監各掌訓練,使士卒習聞主將號令,急緩即成部分。」



 天子下其章,判大名府夏竦奏:「鎮、定二路當內外之沖,萬一有警,各籍重兵,控守要害,迭為應援。若合為一,則兵柄太重,減之則不足以備敵。又滄州久隸高陽關,道裏頗近,瀕海斥鹵,地形沮洳,東北三百里,野無民居,非賊蹊徑。萬一有警,可決漳、御河東灌,塘澱隔越,賊兵未易奔沖,不必別建一路。惟北京為
 河朔根本,宜宿重兵,控扼大河南北,內則屏蔽王畿,外則聲援諸路。請以大名府、澶、懷、衛、濱、棣、德、博州、通利軍建為北京路。四路各置都總管、副都總管一人,鈐轄二人,都監四人。平時只以河北安撫使總制諸路,有警,即北京置四路行營都總管,擇嘗任兩府重臣為之。」



 議未決,竦入為樞密使,賈昌朝判大名府,復命規度。昌朝請如竦議,惟保州沿邊巡檢並雄、霸、滄州界河二司兵馬,國初以來,拓邊最號強勁,今未有所隸,請立沿邊巡檢
 司隸定州路,界河司隸高陽關路。



 於是下詔分河北兵為四路:北京、澶、懷、衛、德、博、濱、棣州、通利保、順軍合為大名府路;瀛、莫、雄、霸、貝、冀、滄州、永靜、乾寧、保定、信安軍合為高陽關路;鎮、邢、洺、相、趙、磁州合為真定府路;定、保、深、祁州、北平、廣信、安肅、順安、永寧軍合為定州路。凡兵屯將領,悉如其議。韓琦謂兵勢大分,請合定州、真定府為一,高陽關、大名府為一。朝廷以更寘甫新,不報。詔四路兵依陜西遣部將往來按閱。又詔自今兵戍回,揀充捧日、龍
 衛、天武、神衛等軍。



 皇祐元年,發禁兵十指揮戍京東,以歲饑備盜。詔陜西邊警既息,土兵可備守御,東軍屯戍者徙內郡,以省餉饋。二年,詔:「如聞河北諸屯將校,有老疾廢事而不知退,有善部勒著勞效而不得進,帥臣、監司審察,密以名聞。」



 四年,詔:「戍兵歲滿,有司按籍,遠者前二月,近者前一月遣代,戍還本管聽休。」五年,又詔:「廣西戍兵及二年而未得代者罷歸,鈐轄司以土兵歲一代之。」自儂智高之亂,戍兵逾二萬四千,至是聽還,而令土
 兵代戍。



 至和元年,詔陳、許、鄭、滑、曹州各屯禁兵三千。嘉祐五年,用賈昌朝奏,京北路置都監三人,駐扎許、蔡、鄭州,分督近畿屯兵。七年,詔陜西土兵番戍者毋出本路。



 治平二年,發兵指揮二十,分戍永興軍、邠州、河中府,仍遣官專掌訓練。三年,詔員僚直、龍衛毋出戍,神衛嘗留十指揮在營。又詔:「頃以東兵戍嶺南,冒犯瘴癘,得還者十無五六。自今歲滿,以江、淮教閱忠節、威果代之。」



 神宗嗣位,軍政多所更革。熙寧初,嘗與輔臣論河北守備。韓
 絳等曰:「漢、唐重兵皆在京師,其邊戍裁足守備而已。故邊無橫費,強本弱末,其勢亦順。開元後,有事四夷,權臣皆節制一方,重兵在西北。天寶之亂,由京師空虛,賊臣得以肆志也。」帝曰:「邊上老人亦謂今之邊兵過於昔時,其勢如倒植浮圖。朕亦每以此為念也。」三年,詔:「諸路戍兵,畸零不成部伍,致乖紀律,或互遣郡兵,更相往來,道路艱梗,宜悉罷之,易以上番全軍或就糧兵為戍;當遣者並隸總管司,以詔令從事。」



 舊制,河北軍馬不出戍,帝
 慮其驕惰,五年,始命河北、河東兵更戍,減其一歲以優之。其年,詔徙河州軍馬駐熙州,熙州軍馬駐通遠軍,追召易集,可省極邊軍儲。帝嘗曰:「窮吾國用者,冗兵也。其議徙軍於內郡,以弓箭手代之,冀省邊費。」



 九年,詔:「京師兵比留十萬,餘以備四方屯戍,數甚減少。自今戍兵非應發京師者勿遣。」其後,言者屢請損河北冗兵,詔立額止留禁兵七萬,而京東增置武衛軍四十二營,訓練精銳,皆以分隸河北,而以三千人散戍東南杭、揚、江寧諸
 州,以備盜賊。嶺外惟廣、韶、南雄州常有戍兵千人,桂林以瘴癘,間徙軍於全、永。元豐中,或請遣陜西路騎軍五七百戍桂林者,詔遣在京軍馬以戍之。



 元祐元年六月,右諫議大夫孫覺言:「將兵之禁,宜可少解,而責所在守臣與州郡兵官,可令乘時廣行召募,稍補前日之額。循祖宗之法,使屯駐三邊及川、廣、福建諸道州軍,往來道路,足以服習勞苦,南北番戍,足以均其勞佚。」詔:「陜西、河東、廣南將兵不輪戍他路,河北輪近裏一將赴河東,府
 界、諸路逐將與不隸將兵,並更互差撥出戍別路。赴三路者差全將或半將,餘路聽全指揮分差,仍不過半將。」



 十月,樞密院言:「東南一十三將,自團將以來,未曾均定出戍路分,及不隸將兵內有出戍窠名數少、所管指揮數多去處,未得均當。欲除廣南東、西兩路駐扎三將只充本路守御差使,虔州第六將、全、永州第九將準備廣南東、西路緩急勾抽策應,並不差戍他路外,餘八將及不隸將兵依均定路分都鈐轄司駐泊,分擘差使。內將
 兵、不隸將兵路分,卻於自京差撥步軍前去補戍,候將兵回日,卻行勾抽。」從之。



 十二月,廣西經略安撫使、都鈐轄司言:「乞除桂、宜、融、欽、廉州系將、不系將馬步軍輪差赴邕州極邊水土惡弱砦鎮監柵及巡防並都同巡檢等處,並乞依邕州條例,一年一替;其餘諸州差往邕州永平、古萬、太平、橫山、遷隆砦鎮及左、右江溪洞巡檢並欽州如昔峒駐扎抵棹砦,並二年一替;其諸州巡檢下,一年一替。」從之。



 二年,河東經略安撫使曾布言:「河外
 上番四將,每將內抽減步軍赴嵐、石州,分擘沿河等處差使,代開封府界等五將兵馬歸營;及赴岢嵐、火山軍駐扎,代東兵兩指揮赴太原府就食。」從之。是月,樞密院言:「昨為熙河蘭會路戍兵數多,尋以年滿,二千餘人節次抽減歸營,兼本路即目見管戍兵比額尚多一千三百餘人。今朝旨令熙河蘭會路都總管司遇本路緩急闕人,許於秦鳳路勾抽一將應副。緣本路即目事宜,慮向秋闕人防守,欲熙河蘭會路都總管司遇本路緩急
 闕人,聽全勾抽秦鳳路九將應副差使,從京東差步軍五指揮赴永興軍、商、虢州權駐扎,以備秦鳳路勾抽。」從之。



 紹聖四年,樞密院備呂惠卿所言:「『比緣邊牒報,西界點集本路叛卒。見闕守禦人兵,兼土兵未填闕額,並蕃兵弓箭手比元豐元年少二千二百有餘,東兵馬步軍比元豐四年、七年少十六指揮。乞於東步兵人內差撥一十六指揮添助防守。』兼本路自去歲泛差過軍馬三十六指揮,比之他路,已是倍多,即今戍兵二萬六千餘
 人,比之元豐四年人數,亦不至闕少,自可那融使喚。」詔:「鄜延路都總管司詳此照會,如遇賊兵犯塞,或本路舉兵,委是闕人,其年滿人指揮兵級,令相度事宜,權留三兩月,候事宜稍息遣還。」是月,詔:「河東路總管司那融替換上番兵馬,無令戍邊日久,致有勞弊。如無人替換,候春月事宜稍息,即先後上番四將抽減一番兵馬歸營。」



 元符二年閏九月,遣秦鳳戍兵十指揮應副熙河新邊戍守。十一月,以呂惠卿奏,減鄜延戍兵五十指揮。三年
 八月,詔遣虎翼軍六千戍熙河路,令代蕃兵及弓箭手還家休息。十二月,詔邊帥減額外戍兵。



 崇寧四年,詔:「廣南瘴癘之鄉,東西雖殊,氣候無異。西路戍兵二年一代,而東路獨限三年,代不如期,有隕於瘴癘者,朕甚惻然。其東路亦令二年一替,前期半年差人,如違,以違制論。」



 大觀二年六月,詔:「陜西諸路,自罷兵以來,數年於此,兵未曾徹。蓋緣邊將怯懦,坐費邊儲,戍卒勞苦。可除新邊的確人外,餘並依元豐罷邊事日戍額人數外,餘並直
 抽歸營。有司不得占吝,如違,以違制論。」又詔:「東南除見兵額外,帥府別屯二千人,望郡一千人。帥府置奉錢五百一指揮,以威捷為名;望郡奉錢四百一指揮,以威勝為名;帥府三指揮、望郡一指揮各奉錢三百,以全捷為名:並以步軍五百人為額。」三年六月,詔:「國家承平百五十年,東南一方,地大人眾,已見兵寡勢弱,非持久之道。可除見今兵額外,帥府別屯兵士二千人,望郡一千人。」



 宣和二年,詔河北軍馬與陜西、河東更戍。



 三年正月,詔:「
 河北軍馬與陜西、河東更戍,非元豐法,遂罷其令。應拖後人並與免罪,依舊收管。」閏五月,江、浙、淮南等路宣撫使童貫奏:「勘會江南東路、兩浙東西路各有東南一將,平日未嘗訓練武藝,臨敵必誤驅策。昨睦寇初發,天兵未到已前,遣令上項將兵捕賊,遂致敗衄,亡失軍兵甚多。今睦賊討平之後,脅從叛亡者方始還業,非增戍兵鎮遏,無以潛消兇暴。臣今擬留戍兵二萬五千五百七十八人,分置江南東路、兩浙東西路州軍防把,一年滿
 替出軍一次,依平蠻故事,每月別給錢三百,歲給鞋錢一千。其兵並隸本路安撫司統轄訓練。」詔從之。是年,權知婺州楊應誠奏:「凡屯戍將兵,須隸守臣,使兵民之任歸一,則號令不二,然後可以立事。」詔從之。續有旨改從舊制。



 四年,臣僚言:「東軍遠戍四川,皆京師及府界有武藝無過之人。既至川路,分屯散處,多不成隊,而差使無時,委致勞弊。蓋四川土兵既有詔不得差使,則其役並著東軍,實為偏重。若令四川應有土兵、禁軍與東軍一
 同差使,不惟勞逸得均,抑亦不失熙、豐置東軍彈壓蜀人兼備蠻寇之意。」詔本路鈐轄、轉運兩司公同相度利害以聞。



 五年,制置所奏:「江、浙增屯戍後兵,相度節鎮增添兩指揮處,餘州各一指揮,各不隸將。內兩指揮處,將一指揮以威果為名,一指揮以全捷為名,餘州並以威果為名。」從之。



 七年三月,詔:「廣南東、西路地遠山險,盜賊間有竊發。內郡戍兵往彼屯守,多緣瘴癘疾病,不任捕盜,又不諳知山川道里、林壑曲折,故盜不能禁。可令每巡
 檢下招置土人健勇輕捷者,參戍兵之半,互相關防,易於擒捕。令樞密院行之。」



 靖康元年四月,以種師道為太尉,依前鎮洮軍節度使、河北河東宣撫使,後加同知樞密院事。時師道駐軍滑州,實無兵從行,請合山東、陜西、京畿兵屯於青、滄、滑、衛、河陽,預為防秋之計。徐處仁等謂:「金人重載甫還,豈能復來?不宜先自擾費,示之以弱。」議格不行。



 七月,河北東路宣撫使李綱奏:「臣兩具論,以七月七日指揮止諸路防秋之兵為不可,必蒙聖察。今
 宣撫司既無兵可差,不知朝廷既止諸路防秋之兵,將何應副。兼遠方人兵各已在路,又已借請數月,本路漕司、州縣又已預備半年、百日之糧,今一放散,皆成虛費,而實要兵用處無可摘那,深恐誤國大計。」詔依所奏。



 紹興之初,群盜四起,有若岳飛、劉光世諸大將領兵尤重,隨宜調發,屯泊要害,控制捍蔽,是亦權宜之利矣。厥後樞府、帥臣屢言久戍之弊,甚者或十年或二十年而不更,尤可閔念。蓋出戍者皆已老瘁,而諸州所留,類皆少
 壯及工匠,三司多以坐甲為名,占留違制,有終身未嘗一日戍者,於是命帥臣、鈐轄司置諸州尺籍,定其姓名,依期更戍。帥臣又言:「有如貴溪戍兵,三月一更,由貴溪至池州,往返一千五百里,即是一月在途,徒有勞費。願以一年終更。」



 今考紹興間邊境弗靖,故以大軍屯戍,而踐更之期,近者三月,遠者三年。逮和議既成,諸軍移屯者漸歸營矣,惟防秋仍用移屯更戍之法,沿邊備御亦倚重焉。乾道、淳熙、紹熙之際,一遵其制。開禧初,復議用
 兵,駐扎諸兵始復移屯。和議再成,邊地一二要郡雖循舊貫,其諸駐扎更戍之法不講,而常屯之兵益多。逮夫端平破川蜀,咸淳失襄樊、裂淮甸,疆宇蹙而兵法壞。叛將賣降,庸夫秉鉞,間有圖國忘死之士,則遙制於權奸,移屯更戍,靡有定方。於是戍卒疲於奔命,不戰而斃者眾矣。至若將校之部曲,諸軍之名號,士卒之眾寡,詳列於屯駐者,茲不重錄云。



\end{pinyinscope}