\article{列傳第一百五十八}

\begin{pinyinscope}

 ○鄭𣪝王庭秀附仇悆高登婁寅亮宋汝為



 鄭𣪝,字致剛,建州人。政和八年舉進士,授安陸府教授,權信陽縣尉,監南康酒稅。遂召為御史臺主簿。張邦昌
 之僭號也,挺身見高宗於濟州。既即位,擢監察御史,遷右司諫,升為諫議大夫。



 帝至杭州,
 
  
   
  
 
 奏曰:「陛下南渡出於倉卒,省臺寺監、百司之臣獲濟者鮮,當擢吳中之秀以為用。況天下賢俊多避地吳、越,宜令守臣體訪境內寄居待闕,及見任宮觀等京朝官以上,各具姓名以聞,簡拔任使,庶幾速得賢才以濟艱厄。」詔從之。



 苗傅、劉正彥等逆亂,
 
  
   
  
 
 庭立面折二兇,且謂逆賊兇焰熾甚,非請外援無可為者。乃上章待罪求去,退見呂頤浩,議興復
 計,太后降詔不允。朱勝非言
 
  
   
  
 
 面折二兇事,拜御史中丞。



 時二兇竊威福之柄,肆行殺戮,日至都堂侵紊機政。
 
  
   
  
 
 言:「黃門宦者之設,本以給事內庭,供掃除而已。俾與政事,則貪暴無厭,待以兵權,則慘毒無已,皆前世已行之驗也。故宦官用事於上,則生人受禍於下,匹夫力不能勝,則群起而攻之。是以靖康之初,群起而攻之者庶民也;睿聖皇帝南渡,駐蹕未安,群起而攻之者眾兵也。今當痛革前弊,並令選擇其人,曾經事任招權納寵者,
 屏之遠方,俾無浸淫以激眾怒,則賞罰之柄自朝廷出,國勢尊矣。仍諭軍法便宜,止行於所轄軍伍,其餘當聞之朝廷,付之有司,明正典刑,所以昭尊君之禮而全臣子忠義之節也。」疏留中不出。
 
  
   
  
 
 對,請付外行之。



 又論:「黃潛善、汪伯彥均於誤國,而潛善之罪居多,今同以散官竄謫湖南;錢伯言與黃願皆棄城,呂源與梁揚祖皆擁兵而逃,今願罷官,揚祖落職,而源、伯言未正典刑,非所以勸懲。」詔竄削有差。



 傅、正彥日至都堂議事,
 
  
   
  
 
 奏:「將帥
 之臣不可預政。」及聞以簽書樞密院召呂頤浩,以禮部尚書召張浚,分張俊兵以五百人歸陜西,而浚不受尚書之命,俊不肯分所部兵,遂謫浚居郴州,擢俊以節度知鳳翔。
 
  
   
  
 
 知出二兇奸謀,具章乞留頤浩知金陵,浚不當貶,不報。
 
  
   
  
 
 遂遣所親謝向變姓名,微服為賈人,徒步如平江見浚等,具言城中事,以為嚴設兵備,大張聲勢,持重緩進,使賊自遁,無驚動三宮,此上策也。浚等聞之,皆感激奮厲為赴難計。



 俄詔睿聖皇帝為皇太弟、天下
 兵馬大元帥,幼主為皇太侄,即與大臣進議,以為:「在庭公卿、百司、群吏皆昔之臣屬也,今則與之比肩事主矣。稽之於古,則無所法;行之於今,則實逆天。或者謂大元帥可以任軍旅之大事,臣竊以為不然。昔舜之禪禹也,猶命禹徂征有苗,則禹雖受禪,而征伐之事舜猶親之也。唐睿宗傳位皇太子,以聽小事,自尊為太上皇,以聽大事。如是無不可者,則稽之於古為有法,行之於今為得宜。」



 太后垂簾同聽政,以安人心。退與御史王庭秀上
 疏力爭。太后召
 
  
   
  
 
 與宰執同對簾前,
 
  
   
  
 
 乞召庭秀,太后諭曰:「今欲令睿聖皇帝總領兵馬爾。」
 
  
   
  
 
 奏曰:「臣不知其他,但人君位號豈容降改,聞之天下,孰不懷疑。雖前世衰亂分裂之時,固未有旬日之間易兩君,一朝降兩朝位號者也。」太后令
 
  
   
  
 
 至都堂,朱勝非出朱昞等所上書以示
 
  
   
  
 
 、庭秀,
 
  
   
  
 
 、庭秀力言昨日詔書不可宣布,必召變。勝非與執政顏歧、王孝迪、路允迪皆在坐,尚書左丞張澄獨曰:「事勢若此,豈爭此名位耶?」澄欲出,
 
  
   
  
 
 等共止之。



 
  
   
  
 
 與李邴並為端明殿學士、同簽書樞密院事。高宗復位,進簽書,執政甫百日而卒。高宗甚悼之,謂大臣:「朕喪元子,猶能自排遣,於
 
  
   
  
 
 殆不能釋也。」



 庭秀字潁彥,慈溪人。與黃庭堅、楊時游,其為學旁搜遠紹,不茍趣時好,造詣深遠,操植堅正,發為文辭,俊邁宏遠。登政和二年上舍第,歷官州縣。



 侍御史李光薦為御史臺檢法官。宣和、靖康時,進言皆發於忠義。御史中丞言:「偽楚時庶官中如虞謨、王庭秀者,初非疾病,毅然致
 為臣而歸,願褒擢之。」拜監察御史,奏:「乞威斷當出於人主,而所遣宣諭官,當令舉廉吏。」又言:「刑名有疑慮者,令州郡法官申憲司閱實具奏,以取裁決。」遷殿中侍御史,論黃潛善賣官售寵,罷之。



 既與鄭𣪝力爭降封高宗事,未幾出知瑞州,右正言呂祉奏:「朝廷今日緣論大臣移一言官,明日罷一言官,則後日大臣行事有失,誰敢言者。」遂召為吏部郎,改左司,言:「朝廷比來深疾貪吏,然州縣之間豈無廉介自將、沈於下僚者,宜命五使,所至以
 廉潔清修、可以師表吏民者,以名來上,參之公議,不次升擢,以厲士風。」從之。



 遷檢正中書門下省諸房公事,與宰相議多不合,不自安,引疾求去。詔直秘閣、主管崇道觀而歸。



 仇悆,字泰然,益都人。大觀三年進士,授邠州司法,讞獄詳恕,多所全活。為鄧城令,滿秩,耆幼遮泣不得去。徙武陟令,屬朝廷方調兵數十萬於燕山,悆饋餉畢給。時主將縱士卒過市掠物,不予直,他邑官逃避,悆先期趣備,
 申嚴約束,遂以不擾。已而悆送運餉於涿,值大軍潰於盧溝河,囊橐往往委以資敵,悆間關營護,無一豪棄失。



 調高密丞,俗尚囂訟,悆攝縣事,剖決如流,事無淹夕,民至懷餅餌以俟決遣。猾吏楊蓋每陰疏令過,脅持為奸,悆暴其罪黥之,無不悅服。州闕司錄,命悆攝事,既行,邑氓萬餘邀留,至擁歸縣廨,時天寒,皆然火警守,布滿後先,悆由它道得出,或追拜馬首曰:「公舍我去,我必使公復來。」它日,悆方白事郡牙,忽數千人徑奪以歸,守針弗
 能遏。劇寇起萊、密間,素聞悆名,戒其黨毋犯高密境,民賴以安。密卒閉關叛掠,害官吏幾盡,獨呼曰:「無驚仇公。」



 南遷,丁母憂。服除,知建昌軍,入為考功員外。時任者宛轉兵間,亡失告牒十常七八,而銓部無案籍,訴丐者甚多,真偽錯亂。悆親為考核,其可據者悉責保識,因上聞行之。



 遷右司及中書門下檢正諸房公事,俄為沿海制置使。明守與宰相厚善,紿言士卒將為變,致遣精兵密捕。統制官徐文覺之,初謀縱軍剽略,頃之泛海去,呼曰:「
 我以仇公故,不殺人,不焚屋廬。」一城晏然。猶坐削兩官,主管太平觀。



 以淮西宣撫知廬州。劉豫子麟合金兵大入,民情洶懼。宣撫司統制張琦者,冀乘危為亂,驅居民越江南走。欲先脅悆出,擁甲士數千突入,露刃登樓,揚白麾,左右驚潰,迫悆上馬。悆徐謂曰:「若輩無守土責,吾當以死徇國,寇未至而逃,人何賴焉。」堅不為動,神色無少異。琦等錯愕,遽散其徒,人心遂定。



 時金人出入近境,悆求援於宣撫司,不報。又遣其子自間道赴朝廷告急,
 雖旌其子以官,而援卒不至。帝方下詔親征,而詔亦不至淮甸,喧言將棄兩淮為保江計。悆錄詔語揭之郡縣,讀者至流涕,咸思自奮。監押閻僅死於賊,餘眾來歸,州帑匱竭,無以為賞,悆悉引班坐,犒以酒食,慰勞之,眾皆感勵。募廬、壽兵得數百,益鄉兵二千,出奇直抵壽春城下,敵三戰皆北,卻走度淮。其後麟復增兵來寇,悆復壽春,俘馘甚眾,獲旗械數千,焚糧船百餘艘,降渤海首領二人。



 初,金人圍濠州,旬日未下,屬天寒,馬多殭死,乃悉
 眾向淮東。樞密使張浚方視師金陵,悆以策說之曰:金重兵在淮東,師老食匱,若以精兵二萬一自壽陽,一自漢上,徑趨舊京,當不戰而退,繼以大軍尾擊,蔑有不濟者。昔人謂『一日縱敵,數世之患。』願無失時之悔。」浚不能用。



 麟復以步騎數千至合肥,諜言兀術為之殿,人心怖駭,不知所為。會京西制置使遣牛皋統兵適至,悆顧左右曰:「召牛觀察來擊賊。」皋既至,以忠義撼之,皋素勇甚,以二千餘騎馳出,短兵相接,所向披靡,敵稍懾,散而復
 集者三。其副徐慶忽墜馬,敵競赴之,皋掖以上,手刜數人,因免胄大呼曰:「我牛皋也,嘗四敗兀術,可來決死。」寇畏其名,遂自潰。以悆克復守御功,加徽猷閣待制。



 明年,宣撫司始遣大將王德來,時寇已去,德謂其伍曰:「當事急時,吾屬無一人渡江擊賊,今事平方至,何面目見仇公耶?」德麾下多女真、渤海歸附者,見悆像,不覺以手加額。



 初,宣撫司既不以一卒援諸郡,但令焚積聚,棄城退保,文移不絕於道,又請浚督行之。浚檄悆度其宜處之,
 悆謂:「殘破之餘,兵食不給,誠不能支敵。然帥臣任一路之責,誓當死守。今若委城,使金人有淮西,治兵艦於巢湖,必貽朝廷憂。」力陳不可,浚韙其言,而卒全活數州之眾。尋詔詣闕,軍民號送之。



 改浙東宣撫使、知明州,以挫豪強、獎善良為理。吏受賕,雖一錢不貸,奸猾斂跡。州罹兵火既毀,悆斥廚錢助其費,買田行鄉飲酒禮。歲饑,發官儲損其直,民無死徙。朝廷聞之,進秩一等。



 再召,進對,帝親加褒諭,欲留置近密。言者以悆在郡多黥胥吏為
 慘酷,請授外藩。時峒獠未息,乃進直學士,為湖南安撫使,禁盜鑄錢者,趣使為農,物價既平,商賈遂通。數月,召還,加寶文閣學士、陜西都轉運使。時金人無故歸侵疆,詭計叵測,悆力陳非策,固辭不行。秦檜方主和議,以為異己,落職,以左朝奉郎、少府少監分司西京,全州居住。



 起知河南府,未行,金人果復陷所歸郡邑,如悆言。乃復待制,再知明州,改知平江府,陛辭,言:「我軍已習戰,非復前日,故劉錡能以少擊眾,敵大挫衄,若乘已振之勢,鼓
 行而前,中原可傳檄而定。」上嘉之。以言罷,提舉太平觀。積官至左朝議大夫,爵益都縣伯。卒,贈左通議大夫。



 悆性至孝,母沒時,方崎嶇轉徙,居喪盡禮。沿海制置使陳彥文薦於朝,起復之,悆不就。悆端方挺特,自初官訖通顯,無所附麗。令鄧城時,丞相範宗尹方為邑子,以文謁悆。悆他日語其父:「是子公輔器也。」宗尹既當國,未嘗以私見。悆在明州,嘗欲薦一幕官,問曰:「君日費幾何?」對以「十口之家,日用二千」。悆驚曰:「吾為郡守費不及此,屬僚
 所費倍之,安得不貪。」遂止。



 高登,字彥先,漳浦人。少孤,力學,持身以法度。宣和間,為太學生。金人犯京師,登與陳東等上書乞斬六賊。廷臣復建和議,奪種師道、李綱兵柄,登與東再抱書詣闕,軍民不期而會者數萬。王時雍縱兵欲盡殲之,登與十人屹立不動。



 欽宗即位,擢吳敏、張邦昌為相,敏又雪前相李邦彥無辜,乞加恩禮起復之。登上書曰:「陛下自東宮即位,意必能為民興除大利害。踐阼之始,兵革擾攘,朝
 廷政事一切未暇,人人翹足以待事息而睹惟新之政,奈何相吳敏、張邦昌?又納敏黨與之言,播告中外,將復用李邦彥,道路之人無不飲恨而去。是陛下大失天下之望,臣恐人心自此離矣。太上皇久處邦彥等於政府,紀綱紊亂,民庶愁怨,方且日以治安之言誘誤上皇,以致大禍,倉皇南幸,不獲寧居。主辱臣死,此曹當盡伏誅,今乃偃然自恣,朋比為奸,蒙蔽天日。陛下從敏所請,天下之人將以陛下為不明之君,人心自此離矣。」再上書
 曰:「臣以布衣之微賤,臣言系宗社之存亡,未可忽也。」於是凡五上書,皆不報。因謀南歸,忽聞邦昌各與遠郡,一時小人相繼罷斥,與所言偶合者十七八,登喜曰:「是可以盡言矣。」復為書論敏未罷,不報。



 初,金人至,六館諸生將遁去,登曰:「君在可乎?」與林邁等請隨駕,隸聶山帳中,而帝不果出。金人退師,敏遂諷學官起羅織,屏斥還鄉。



 紹興二年,廷對,極意盡言,無所顧避,有司惡其直,授富川主簿。憲董弅聞其名,檄讞六郡獄,復命兼賀州學事。
 學故有田舍,法罷歸買馬司,登請復其舊。守曰:「買馬、養士孰急?」登曰:「買馬固急矣,然學校禮義由出,一日廢,衣冠之士與堂下卒何異?」守曰:「抗長吏耶!」曰:「天下所恃以治者,禮義與法度爾,既兩棄之,尚何言!」守不能奪,卒從之。攝獄事,有囚殺人,守欲奏裁曰:「陰德可為。」登曰:「陰德豈可有心為之,殺人者死,而可幸免,則被死之冤何時而銷?」



 滿秩,士民丐留不獲,相率饋金五十萬,不告姓名,白於守曰:「高君貧無以養,願太守勸其咸受。」登辭之,不可,
 復無所歸,請置於學,買書以謝士民。歸至廣,會新興大饑,帥連南夫檄發廩振濟,復為糜於野以食之,願貸者聽,所全活萬計。歲適大稔,而償亦及數。民投牒願留者數百輩,因奏闢終其任。



 召赴都堂審察,遂上疏萬言及《時議》六篇,帝覽而善之,下六議中書。秦檜惡其譏己,不復以聞。



 授靜江府古縣令,道湖州,守汪藻館之。藻留與修《徽宗實錄》,固辭,或曰:「是可以階改秩。」登曰:「但意未欲爾。」遂行。廣西帥沈晦問登何以治縣,登條十餘事告之。
 晦曰:「此古人之政,今人詐,疑不可行。」對曰:「忠信可行蠻貊,謂不能行,誠不至爾。」豪民秦琥武斷鄉曲,持吏短長,號「秦大蟲」,邑大夫以下為其所屈。登至,頗革,而登喜其遷善,補處學職。它日,琥有請屬,登謝卻之,琥怒,謀中以危法。會有訴琥侵貸學錢者,登呼至,面數琥,聲氣俱厲,叱下,白郡及諸司置之法,忿而死,一郡快之。



 帥胡舜陟謂登曰:「古縣,秦太師父舊治,實生太師於此,盍祠祀之?」登曰:「檜為相亡狀,祠不可立。」舜陟大怒,摭秦琥事,移
 荔浦丞康寧以代登,登以母病去。舜陟遂創檜祠而自為記,且誣以專殺之罪,詔送靜江府獄。舜陟遣健卒捕登,屬登母死舟中,槁葬水次,航海詣闕上書,求納官贖罪,帝閔之。故人有為右司者,謂曰:「丞相云嘗識君於太學,能一見,終身事且無憂,上書徒爾為也。」登曰:「某知有君父,不知有權臣。」既而中書奏故事無納官贖罪,仍送靜江獄。登歸葬其母,訖事詣獄,而舜陟先以事下獄死矣,事卒昭白。



 廣漕鄭鬲、趙不棄闢攝歸善令,遂差考試,
 摘經史中要語命題,策閩、浙水災所致之由。郡守李仲文即馳以達檜,檜聞震怒,坐以前事,取旨編管容州。漳州遣使臣謝大作持省符示登,登讀畢,即投大作上馬,大作曰:「少入告家人,無害也。」登曰:「君命不敢稽。」大作愕然。比夜,巡檢領百卒復至,登曰:「若朝廷賜我死,亦當拜敕而後就法。」大作感登忠義,為泣下,奮劍叱巡檢曰:「省符在我手中,無它語也。汝欲何為,吾當以死捍之。」鬲、不棄亦坐鐫一官。



 登謫居,授徒以給,家事一不介意,惟聞
 朝廷所行事小失,則顰蹙不樂,大失則慟哭隨之,臨卒,所言皆天下大計。後二十年,丞相梁克家疏其事以聞。何萬守漳,言諸朝,追復迪功郎。後五十年,朱熹為守,奏乞褒錄,贈承務郎。



 登事其母至孝,舟行至封、康間,阻風,方念無以奉晨膳,忽有白魚躍於前。其學以慎獨為本,所著《家論》、《忠辨》等編,有《東溪集》行世。



 婁寅亮,字陟明,永嘉人。政和二年進士,為上虞丞。建炎四年,高宗至越,寅亮上疏云:「先正有言:『太祖舍其子而
 立弟,此天下之大公;周王薨,章聖取宗室育之宮中,此天下之大慮也。』仁宗感悟其說,詔英祖入繼大統。文子文孫,宜君宜王,遭罹變故,不斷如帶。今有天下者,獨陛下一人而已。屬者椒寢未繁,前星不耀,孤立無助,有識寒心。天其或者深戒陛下,追念祖宗公心長慮之所及乎?崇寧以來,諛臣進說,獨推濮王子孫以為近屬,餘皆謂之同姓,遂使昌陵之後,寂寥無聞,奔迸藍縷,僅同民庶。恐祀豐於暱,仰違天監,太祖在天莫肯顧歆,是以二
 聖未有回鑾之期,金人未有悔禍之意,中原未有息肩之日。臣愚不識忌諱,欲乞陛下於子行中遴選太祖諸孫有賢德者,視秩親王,俾牧九州,以待皇嗣之生,退處藩服,並選宣祖、太宗之裔,材武可稱之人,升為南班,以備環衛。庶幾上慰在天之靈,下系人心之望。」帝讀之感悟,樞密富直柔薦之。



 紹興元年,召赴行在,以其言宗社大計也。既入見,復上疏曰:「陛下轍跡所環,六年於外,險阻艱難,備嘗之矣。然而二聖未還,金人未滅,四方未靖
 者,何哉?天意若曰:天祚宋德,太祖不私其子而保之,不幸稈邪誤國而壞之,將使嗣聖念祖,思危而後獲之,乃所以申其永命也。臣誠狂妄,去歲上章,請陛下取太祖諸孫之賢者,視秩親王,使牧九州,誤蒙採聽,赦而不誅。茲蓋在天之靈發悟聖心,為社稷計,非愚臣之所及也。伏望宣告大臣行之,它日皇子之生,使之退處清暇,不過增一節度使爾。陛下以太祖之心,行章聖之慮,自然孝弟感通,兩宮回蹕,澤流萬世。」



 改令入官,擢監察御史。
 時相秦檜以其直柔所薦,惡之,諷言者論寅亮匿父喪不舉,下大理鞫問,無實,猶坐為族父冒占官戶罷職,送吏部,由是坐廢。



 宋汝為,字師禹,豐縣人。靖康元年,金人犯京師,闔門遇害。汝為思報國家及父兄之仇,建炎三年,金人再至,謁部使者陳邊事,遣對行在。高宗嘉納,特補修武郎,假武功大夫、開州刺史,奉國書副京東運判杜時亮使金。



 時劉豫節制東平,丞相呂頤浩因致書豫。汝為行次壽春,
 遇完顏宗弼軍,不克與時亮會,獨馳入其壁,將上國書。宗弼盛怒,劫而縛之,欲加僇辱。汝為一無懼色,曰:「死固不辭,然銜命出疆,願達書吐一辭,死未晚。」宗弼顧汝為不屈,遂解縛延之曰:「此山東忠義之士也。」命往見豫,汝為曰:「願伏劍為南朝鬼,豈忍背主不忠於所事。」力拒不行,乃至京師,瀕死者數四。



 豫僭號,汝為持頤浩書與之,開陳禍福,勉以忠義,使歸朝廷。豫悚而立曰:「使人!使人!使豫自新南歸,人誰直我,獨不見張邦昌之事乎?業已
 至此,夫復何言。」即拘留汝為。然以汝為儒士,乃授通直郎、同知曹州以誘之,固辭。遂連結先陷於北者凌唐佐、李亙、李儔為腹心,以機密歸報朝廷。唐佐等所遣僧及卒為邏者所獲,汝為所遣王現、邵邦光善達,朝廷皆官之。



 紹興十三年,汝為亡歸,作《恢復方略》獻於朝,且曰:「今和好雖定,計必背盟,不可遽馳。」時秦檜當國,置不復問。獨禮部尚書蘇符憐之,為言於朝,換宣教郎,添差通判處州。高宗憶其忠,特轉通直郎。



 汝為遂上丞相書,言:「用
 兵之道,取勝在於得勢,成功在乎投機。女真乘襲取契丹之銳,梟視狼顧,以窺中原,一旦長驅直搗京闕,升平既久,人不知兵,故彼得投其機而速發,由是猖獗兩河,以成盜據之功。既而關右、河朔豪傑士民避地轉鬥,從歸聖朝,將士戮力,削平群盜,破逐英雄,百戰之餘,勇氣萬倍。回思曩昔,痛自慚悔,人人扼腕切齒,願當一戰。加以金人兵老氣衰,思歸益切,是以去歲順昌孤壘,力挫其鋒。方其狼狽逃遁之際,此國家乘勝進戰之時也。惜
 乎王師遽旋,撫其機而不發,遂未能殄滅醜類,以成恢復之功。今聞其力圖大舉,轉輸淮北,其設意豈小哉!所慮秋冬復肆猖獗,兀術不死,兵革不休,雖欲各保邊陲,安可得也。今當乘去歲淮上破賊之勢,特降哀痛之詔,聲言親征,約諸帥長驅直搗,某月日各到東京,協謀並力,以俘馘兀術為急。」



 又言:「兀術好勇妄作,再起兵端,所共謀者,叛亡群盜而已。去夏諸帥各舉,金人奔命敗北之不暇,兀術深以為慮,故為先發制人之動,所恃者不
 過自能聚兵合勢,料王師以諸帥分軍爾。今計其步騎不過十萬,王師雲集,其眾數倍,合勢刻期,並進戮力,何憂乎不勝?若以諸帥難相統屬,宜除川、陜一路,專當撒離喝,權合諸帥為兩節制,公選大臣任觀軍容為宣慰之職,往來調和諸帥,使之上下同心,左右戮力,則勢既合不為賊所料矣。不然,分軍出陳、蔡,直搗東都,賊必首尾勢分,復以重兵急擊,然後以舟師自淮繇新河入鉅野澤,以步兵自洛渡懷、衛入太行山,以襲其內。舟師入
 鉅野,則齊魯搖,步兵入太行,則三晉應,賊勢雖欲合而不分,亦難乎為計矣。」



 久之,有告汝為於金人以蠟書言其機事者,大索不獲,尋知南歸。檜將械送金人,汝為變姓名為趙復,徒步入蜀。汝為身長七尺,疏眉秀目,望之如神仙。楊企道者,遇之溪上,企道曰:「必奇士也。」款留之,見其議論英發,洞貫古今,靖康間離亂事歷歷言之,企道益驚,遂定交,假僧舍居之。



 檜死,汝為曰:「朝廷除此巨蠹,中原恢復有日矣。」企道勸其理前事,汝為慨然太息曰:「吾
 結發讀書,奮身一出,志在為國復仇,收還土宇,頗為諸公所知,命繆數奇,軋於權臣,今老矣,新進貴人,無知我者。」汝為能知死期,嘗祭其先,終日大慟,將終,神氣不亂。



 汝為俶儻尚氣節,博物洽聞,飲酒至斗餘,未嘗見其醉,或歌或哭,涕淚俱下。其客蜀也,史載之、邵博、宇文亮臣、李燾相得甚歡,趙沂、王京魯、關民先、楊採、惠疇經紀其喪事。



 三十二年,其妻錢莫知汝為死,詣登聞鼓院以狀進,詔索之不得。隆興二年,其子南強以汝為之死哀訴
 於朝,參知政事虞允文,錢端禮以聞,特官一子。有《忠嘉集》行世。



 論曰:高宗播遷,復有苗、劉之變,此何時也,鄭𣪝、王庭秀正色立朝,以爭君臣之義,顧不韙哉!仇悆愷悌君子,遺澤在民。《易》曰「王臣蹇蹇」,高登有焉。婁寅亮請立太祖後為太子,能言人臣之所難言,而高宗亦慨然從之,君仁而臣直乎!宋汝為歸自金國,論事切直,與寅亮俱迕秦檜,一則誣以罪譴,一則逃遁以死,於乎悕矣!



\end{pinyinscope}