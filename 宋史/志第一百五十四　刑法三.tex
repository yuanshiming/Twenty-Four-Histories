\article{志第一百五十四 刑法三}

\begin{pinyinscope}

 天下疑獄,讞有不能決,則下兩制與大臣若臺諫雜議,視其事之大小,無常法,而有司建請論駁者,亦時有焉。



 端拱初,廣安軍民安崇緒隸禁兵,訴繼母馮與父知逸
 離,今奪資產與己子。大理當崇緒訟母,罪死。太宗疑之,判大理張佖固執前斷,遂下臺省雜議。徐鉉議曰:「今第明其母馮嘗離,即須歸宗,否即崇緒準法處死。今詳案內不曾離異,其證有四。況不孝之刑,教之大者,宜依刑部、大理寺斷。」右僕射李昉等四十三人議曰:「法寺定斷為不當。若以五母皆同,即阿蒲雖賤,乃崇緒親母,崇緒特以田業為馮強占,親母衣食不給,所以論訴。若從法寺斷死,則知逸何辜絕嗣,阿蒲何地托身?臣等議:田產
 並歸崇緒,馮合與蒲同居,供侍終身。如是,則子有父業可守,馮終身不至乏養。所犯並準赦原。」詔從昉等議,鉉、佖各奪奉一月。



 熙寧元年八月,詔:「謀殺已傷,按問欲舉,自首,從謀殺減二等論。初,登州奏有婦阿云,母服中聘於韋,惡韋醜陋,謀殺不死。按問欲舉,自首。審刑院、大理寺論死,用違律為婚奏裁,敕貸其死。知登州許遵奏,引律「因殺傷而自首,得免所因之罪,仍從故殺傷法」,以謀為所因,當用按問欲舉條減二等。刑部定如審刑、大
 理。時遵方召判大理,御史臺劾遵,而遵不伏,請下兩制議。乃令翰林學士司馬光、王安石同議,二人議不同,遂各為奏。光議是刑部,安石議是遵,詔從安石所議。而御史中丞滕甫猶請再選官定議,御史錢顗請罷遵大理,詔送翰林學士呂公著韓維、知制誥錢公輔復位。公著等議如安石,制曰「可」。於是法官齊恢、王師元、蔡冠卿等皆論奏公著等所議為不當。又詔安石與法官集議,反復論難。



 明年二月庚子,詔:「今後謀殺人自首,並奏聽敕
 裁。」是月,除安石參知政事,於是奏以為:「律意,因犯殺傷而自首,得免所因之罪,仍從故殺傷法;若已殺,從故殺法,則為首者必死,不須奏裁;為從者自有編敕奏裁之文,不須復立新制。」與唐介等數爭議帝前,卒從安石議。復詔:「自今並以去年七月詔書從事。」判刑部劉述等又請中書、樞密院合議,中丞呂誨、御史劉琦、錢顗皆請如述奏,下之二府。帝以為律文甚明,不須合議。而曾公亮等皆以博盡同異、厭塞言者為無傷,乃以眾議付樞密
 院。文彥博以為:「殺傷者,欲殺而傷也,即已殺者不可首。」呂公弼以為:「殺傷於律不可首。請自今已殺傷依律,其從而加功自首,即奏裁。」陳升之、韓絳議與安石略同。會富弼入相,帝令弼議,而以疾病,久之弗議,至是乃決,而弼在告,不預也。



 蘇州民張朝之從兄以槍戮死朝父,逃去,朝執而殺之。審刑、大理當朝十惡不睦,罪死。案既上,參知政事王安石言:「朝父為從兄所殺,而朝報殺之,罪止加役流,會赦,應原。」帝從安石議,特釋朝不問。更命呂
 公著等定議刑名,議不稱安石意,乃自具奏。初,曾公亮以中書論正刑名為非,安石曰:「有司用刑不當,則審刑、大理當論正;審刑、大理用刑不當,即差官定議;議既不當,即中書自宜論奏,取決人主。此所謂國體。豈有中書不可論正刑名之理?」



 三年,中書上刑名未安者五:



 其一,歲斷死刑幾二千人,比前代殊多。如強劫盜並有死法,其間情狀輕重有絕相遠者,使皆抵死,良亦可哀。若為從情輕之人別立刑,如前代斬右趾之比,足以止惡而
 除害。禁軍非在邊防屯戍而逃者,亦可更寬首限,以收其勇力之效。



 其二,徒、流折杖之法,禁綱加密,良民偶有抵冒,致傷肌體,為終身之辱;愚頑之徒,雖一時創痛,而終無愧恥。若使情理輕者復古居作之法,遇赦第減月日,使良善者知改過自新,兇頑者有所拘系。



 其三,刺配之法二百餘條,其間情理輕者,亦可復古徒流移鄉之法,俟其再犯,然後決刺充軍。其配隸並減就本處,或與近地。兇頑之徒,自從舊法。編管之人,亦迭送他所,量立
 役作時限,無得髡鉗。



 其四,令州縣考察士民,有能孝悌力田為眾所知者,給帖付身。偶有犯令,情輕可恕者,特議贖罰;其不悛者科決。



 其五,奏裁條目繁多,致淹刑禁,亦宜刪定。



 詔付編敕所詳議立法。



 初,韓絳嘗請用肉刑,曾布復上議曰:「無王之制刑罰,未嘗不本於仁,然而有斷肢體、刻肌膚以至於殺戮,非得已也。蓋人之有罪,贖刑不足以懲之,故不得已而加之以墨、劓、剕、宮、大闢,然審適輕重,則又有流宥之法。至漢文帝除肉刑而定笞
 棰之令,後世因之以為律。大闢之次,處以流刑,代墨、劓、剕、宮,不惟非先王流宥之意,而又失輕重之差。古者鄉田同井,人皆安土重遷。流之遠方,無所資給,徒隸困辱,以至終身。近世之民,輕去鄉井,轉徙四方,固不為患,而居作一年,即聽附籍,比於古亦輕矣。況折杖之法,於古為鞭撲之刑,刑輕不能止惡,故犯法日益眾,其終必至於殺戮,是欲輕而反重也。今大闢之目至多,取其情可貸者,處之以肉刑,則人之獲生者必眾。若軍士亡去應
 斬,賊盜贓滿應絞,則刖其足;犯良人於法應死,而情輕者處以宮刑。至於劓、墨,則用刺配之法。降此而後為流、徒、杖、笞之罪,則制刑有差等矣。」議既上,帝問可否於執政,王安石、馮京互有論辨,迄不果行。樞密使文彥博亦上言:「唐末、五代,用重典以救時弊,故法律之外,徒、流或加至於死。國家承平百年,當用中典,然猶因循有重於舊律者,若偽造官文書,律止流二千里,今斷從絞。近凡偽造印記,再犯不至死者,亦從絞坐。夫持杖強盜,本法
 重於造印,今造印再犯者死,而強盜再犯贓不滿五匹者不死,則用刑甚異於律文矣。請檢詳刑名重於舊律者,以敕律參考,裁定其當。」詔送編敕所。



 又詔審刑院、大理寺議重贓並滿輕贓法。審刑院言:「所犯各異之贓,不待罪等而累並,則於律義難通,宜如故事。」而大理寺言:「律稱,以贓致罪,頻犯者並累科;若罪犯不等者,即以重贓並滿輕贓各倍論;累並不加重者止從重。蓋律意以頻犯贓者,不可用二罪以上之法,故令累科;為非一犯,
 故令倍論。此從寬之一也。然六贓輕重不等,若犯二贓以上者,不可累輕以從重,故令並重滿輕滿輕。此從寬之二也。若以重並輕後加重,則止從一重,蓋為進則改從於輕法,退亦不至於容奸。而《疏議》假設之法,適皆罪等者,蓋一時命文耳。若罪等者盡數累並,不等者止科一贓,則恐知法者足以為奸,不知者但系臨時幸與不幸,非律之本意也。」帝是大理議,行之。八年,洪州民有犯徒而斷杖者,其餘罪會恩免,官吏失出,當劾。中書堂後官
 劉袞駁議,以謂「律因罪人,以致罪,罪人遇恩者,準罪人原法。洪州官吏當原。「又請自令官司出入人罪,皆用此令。而審刑院、大理寺以謂:「失入人罪,乃官司誤致罪於人,難用此令。其失出者,宜如袞議。」



 無豐三年,周清言:「審刑院、刑部奏斷妻謀殺案問自首,變從故殺法,舉輕明重,斷入惡逆斬刑。竊詳律意,妻謀殺夫,已殺,合入惡逆,以按問自首,變從故殺法,宜用妻毆夫死法定罪。且十惡條,謀與故鬥殺夫,方入惡逆,若謀而未殺,止當不睦。
 既用舉輕明重,宜從謀而未殺法,依敕當決重杖處死,恐不可入惡逆斬刑。」下審刑院、刑部參詳,如清議。



 邵武軍奏讞,婦與人奸,謀殺其夫,已而夫醉歸,奸者自殺之。法寺當婦謀殺為從,而刑部郎中杜紘議婦罪應死。又興元府奏讞,梁懷吉往視出妻之病,因寄粟,其子輒取食之,懷吉毆其子死。法寺以盜粟論,而當懷吉雜犯死罪,引赦原。而紘議出妻受寄粟,而其子輒費用,不入捕法。議既上,御史臺論紘議不當,詔罰金,仍展年磨勘。而
 侍郎崔臺符以下三人無所可否,亦罰金。



 八年,尚書省言:「諸獲盜,有已經殺人,及元犯強奸、強盜貸命斷配之人,再犯捕獲,有司例用知人欲告、或按問自首減免法。且律文自首減等斷遣者,為其情非巨蠹,有改過自新之心。至於奸、盜,與餘犯不同,難以例減。請強盜已殺人,並強奸或元犯強盜貸命,若持杖三人以上,知人欲告、按問欲舉而自首,及因人首告應減者,並不在減等例。」初,王安石與司馬光爭議按問自首法,卒從安石議。至
 是,光為相,復申前議改焉。乃詔:「強盜按問欲舉自首者,不用減等。」既而給事中范純仁言:「熙寧按問欲舉條並得原減,以容奸太多,元豐八年,別立條制。竊詳已殺人、強奸,於法自不當首,不應更用按問減等。至於貸命及持杖強盜,亦不減等,深為太重。按《嘉祐編敕》:『應犯罪之人,因疑被執,贓證未明,或徒黨就擒,未被指說,但詰問便承,皆從律按問欲舉首減之科。若已經詰問,隱拒本罪,不在首減之例。』此敕當理,當時用之,天下號為刑平。
 請於法不首者,自不得原減,其餘取《嘉祐編敕》定斷,則用法當情,上以廣好生之德,下則無一夫不獲之冤。」從之。



 又詔:「諸州鞫訊強盜,情理無可憫,刑名無疑慮,而輒奏請,許刑部舉駁,重行朝典,無得用例破條。」從司馬光之請也。光又上言:「殺人不死,傷人不刑,堯、舜不能以致治。刑部奏鈔兗、懷、耀三州之民有鬥殺者,皆當論死,乃妄作情理可憫奏裁,刑部即引舊例貸之。凡律、令、敕、式或不盡載,則有司引例以決。今鬥殺當死,自有正條,而
 刑部承例免死決配,是鬥殺條律無所用也。請自今諸州所奏大闢,情理無可憫,刑名無可疑,令刑部還之,使依法處斷。若實有可憫、疑慮,即令刑部具其實於奏鈔,先擬處斷,門下省審覆。如或不當,及用例破條,即駁奏取旨勘之。」



 元祐元年,純仁又言:「前歲四方奏讞,大闢凡二百六十四,死者止二十五人,所活垂及九分。自去年改法,至今未及百日,所奏按凡一百五十四,死者乃五十七人,所活才及六分已上。臣固知未改法前全活數
 多,其間必有曲貸,然猶不失『罪疑惟輕』之仁;自改法後,所活數少,其間必有濫刑,則深虧『寧失不經』之義。請自今四方奏大闢按,並令刑部、大理寺再行審覆,略具所犯及元奏因依,令執政取旨裁斷,或所奏不當,亦原其罪。如此則無冤濫之獄。」



 又因尚書省言,遠方奏讞,待報淹系,始令川、廣、福建、荊南路罪人,情輕法重當奏斷者,申安撫或鈐轄司酌情決斷乃奏。門下侍郎韓維言:「天下奏按,必斷於大理,詳議於刑部,然後上之中書,決之
 人主。近歲有司但因州郡所請,依違其言,即上中書,貼例取旨,故四方奏讞日多於前。欲望刑清事省,難矣。自今大理寺受天下奏按,其有刑名疑慮、情理可憫,須具情法輕重條律,或指所斷之法,刑部詳審,次第上之。」詔刑部立法以聞。



 崇寧五年,詔:「民以罪麗法,情有重輕,則法有增損。故情重法輕,情輕法重,舊有取旨之令。今有司惟情重法輕則請加罪,而法重情輕則不奏減,是樂於罪人,而難於用恕,非所以為欽恤也。自今宜遵舊法
 取旨,使情法輕重各適其中,否則以違制論。」宣和六年,臣僚言:「元豐舊法,有情輕法重,情重法輕,若入大闢,刑名疑慮,並許奏裁。比來諸路以大闢疑獄決於朝廷者,大理寺類以『不當』劾之。夫情理巨蠹,罪狀明白,奏裁以幸寬貸,固在所戒;然有疑而難決者,一切劾之,則官吏莫不便文自營。臣恐天下無復以疑獄奏矣。願詔大理寺並依元豐法。」從之。



 紹興初,州縣盜起,道不通,詔應奏裁者,權減降斷遣以聞。既而奏讞者多得輕貸,官無失
 入之虞,而吏有鬻獄之利,往往不應奏者,率奏之。



 三年,乃詔大闢應奏者,提刑司具因依繳奏。宣州民葉全二盜檀偕窖錢,偕令佃人阮授、阮捷殺全二等五人,棄尸水中,有司以「尸不經驗」奏。侍御史辛炳言偕系故殺,眾證分明,以近降法,不應奏。諸獄不當奏而奏者雖不論罪,今宣州觀望,欲並罪之。帝曰:「若宣州加罪,則實有疑者亦不復奏陳矣。」於是法寺、刑部止罰金。



 五年,給事中陳與義奏有司多妄奏出入人罪,帝為申嚴立法,終不
 悛。



 二十六年,右正言凌哲復上疏曰:「漢高入關,悉除秦法,與民約法三章耳。所謂殺人者死,實居其首。司馬光有言:『殺人者不死,雖堯、舜不能以致治。』斯言可謂至當矣。臣竊見諸路州、軍大闢,雖刑法相當者,類以可憫奏裁。自去歲郊後距今,大闢奏裁者五十餘人中,有實犯故殺、鬥殺常赦所不原者,法既無疑,情無可憫,刑、寺並皆奏裁貸減。彼殺人者可謂幸矣,被殺者銜恨九原,何時已邪?臣恐強暴之風滋長,良善之人莫能自保,其於
 刑政,為害非細。應今後大闢,情法相當、無可憫者,所司輒奏裁減貸者,乞令臺臣彈劾。」帝覽奏,曰:「但恐諸路滅裂,實有情理可憫之人,一例不奏,有失欽恤之意。」令刑部坐條行下。



 馴至乾道,讞獄之弊,日益滋甚。孝宗乃詔有司緣情引條定斷,更不奏裁。其後刑部侍郎方滋言:「有司斷罪,其間有情重法輕,情輕法重,情理可憫,刑名疑慮,命官犯罪,議親議故之類,難以一切定斷。今後宜於敕律條令,明言合奏裁事件,乞並依建隆三年敕文。」
 從之。



 六年,臣僚請:「今後大闢,只以為首應坐死罪者奏,為從不應坐死者,先次決遣。及流、徒罪,不許作情重取旨。不然,則坐以不應奏而奏之罪。」從之。



 至理宗時,往往讞不時報,囚多瘐死。監察御史程元鳳奏曰:「今罪無輕重,悉皆送獄,獄無大小,悉皆稽留。或以追索未齊而不問,或以供款未圓而不呈,或以書擬未當而不判,獄官視以為常,而不顧其遲,獄吏留以為利,而惟恐其速。奏案申牒既下刑部,遲延日月方送理寺。理寺看詳,亦復
 如之。寺回申部,部回申省,動涉歲月。省房又未遽為呈擬,亦有呈擬而疏駁者,疏駁歲月,又復如前。展轉遲回,有一二年未報下者。可疑可矜,法當奏讞,矜而全之,乃反遲回。有矜貸之報下,而其人已斃於獄者;有犯者獲貸,而干連病死不一者,豈不重可念哉?請自今諸路奏讞,即以所發月日申御史臺,從臺臣究省、部、法寺之慢。」從之。而所司延滯,尋復如舊。



 景定元年,乃下詔曰:「比詔諸提刑司,取翻異駁勘之獄,從輕斷決。而長吏監司多
 不任責,又引奏裁,甚者有十餘年不決之獄。仰提刑司、守臣審勘,或前勘未盡,委有可疑,除命官、命婦、宗婦、宗女及合用蔭人奏裁外,其餘斷訖以聞。官吏特免收坐一次。」



 凡應配役者傅軍籍,用重曲者黥其面。會赦,則有司上其罪狀,情輕者縱之,重者終身不釋。初,徒罪非有官當贖銅者,在京師則隸將作監役,兼役之宮中,或輸作左校、右校役。開寶五年,御史臺言:「若此者,雖有其名,無復
 役使。遇祠祭,供水火,則有本司供官。望令大理依格斷遣。」於是並送作坊役之。



 太宗以國初諸方割據,沿五代之制,罪人率配隸西北邊,多亡投塞外,誘羌為寇。乃詔:「當徒者,勿復隸秦州、靈武、通遠軍及緣邊諸郡。」時江、廣已平,乃皆流南方。先是,犯死罪獲貸者,多配隸登州沙門島及通州海島,皆有屯兵使者領護。而通州島中凡兩處官煮鹽,豪強難制者隸崇明鎮,懦弱者隸東州市。太平興國五年,始令分隸鹽亭役之,而沙門如故。端拱
 二年,詔免嶺南流配荷校執役。初,婦人有罪至流,亦執針配役。至是,詔罷免之。始令雜犯至死貸命者,勿流沙門島,止隸諸州牢城。舊制,僮僕有犯,得私黥其面。帝謂:「僮使受傭,本良民也。」詔:「盜主財者,杖脊、黥面配牢城,勿私黥之。十貫以上配五百里外,二十貫以上奏裁。」帝欲寬配隸之刑,祥符六年,詔審刑院、大理寺、三司詳定以聞。既而取犯茶鹽礬曲、私鑄造軍器、市外蕃香藥、挾銅錢誘漢口出界、主吏盜貨官物、夜聚為妖,比舊法咸從
 輕減



 乾興以前,州軍長吏往往擅配罪人。仁宗即位,首下詔禁止,且令情非巨蠹者,須奏待報。又詔諸路按察官取乾興赦前配隸兵籍者,列所坐罪狀以聞。自是赦書下,輒及之。初,京師裁造院募女工,而軍士妻有罪,皆配隸南北作坊。天聖初,特詔釋之,聽自便。婦人應配,則以妻窯務或軍營致遠務卒之無家者,著為法。時又詔曰:「聞配徒者,其妻子流離道路,罕能生還,朕甚憐之。自今應配者,錄具獄刑名及所配地里,上尚書刑部詳覆。」
 未幾,又詔應配者,須長吏以下集聽事慮問。後以奏牘煩冗,罷錄具獄,第以單狀上承進司。既又罷慮問焉。



 知益州薛田言:「蜀人配徒他路者,請雖老疾毋得釋。」帝曰:「遠民無知犯法,終身不得還鄉里,豈朕意哉?察其情可矜者許還。」後復詔罪狀獷惡者勿許。初,令配隸罪人皆奏待報,既而系獄淹久,奏請煩數。明道二年,乃詔有司參酌輕重,著為令。凡命官犯重罪,當配隸,則於外州編管,或隸牙校。其坐死特貸者,多杖、黥配遠州牢城,經恩
 量移,始免軍籍。天聖初,吏同時以贓敗者數人,悉竄之嶺南,下詔申儆在位。有平羌縣尉鄭宗諤者,受賕枉法抵死,會赦當奪官。帝問輔臣曰:「尉奉月幾何,豈祿薄不足自養邪?」王欽若對曰:「奉雖薄,廉士固亦自守。」特杖宗諤,配隸安州。其後數懲貪吏,至其末年,吏知以廉自飾,犯法者稍損於舊矣。



 罪人貸死者,舊多配沙門島,至者多死。景祐中,詔當配沙門島者,第配廣南地牢城,廣南罪人乃配嶺北。然其後又有配沙門島者。慶歷三年,既
 疏理天下系囚,因詔諸路配役人皆釋之。六年,又詔曰:「如聞百姓抵輕罪,而長吏擅刺隸他州,朕甚憫焉。自今非得於法外從事者,毋得輒刺罪人。」皇祐中,即赦,命知制誥曾公亮、李絢閱所配人罪狀以聞,於是多所寬縱。公亮請著為故事,且請益、梓、利、夔四路就委轉運、鈐轄司閱之。自後每赦命官,率以為常。配隸重者沙門島砦,其次嶺表,其次三千里至鄰州,其次羈管,其次遷鄉。斷訖,不以寒暑,實時上道。吳充建請:「流人冬寒被創,上
 道多凍死。請自今非情理巨蠹,遇冬月聽留役本處,至春月遣之。」詔可。



 熙寧二年,比部郎中、知房州張仲宣嘗檄巡檢體究金州金坑,無甚利。土人憚興作,以金八兩求仲宣不差官。及事覺,法官坐仲宣枉法贓應絞,援前比貸死,杖脊、黥配海島。知審刑院蘇頌言:「仲宣所犯,可比恐喝條。且古者刑不上大夫,仲宣官五品,有罪得乘車,今刑為徒隸,其人雖無足矜,恐污辱衣冠爾。」遂免杖、黥,流賀州。自是命官無杖黥法。



 六年,審刑院言:「登州沙
 門砦配隸,以二百人為額,餘則移置海外,非禁奸之意。」詔以三百人為額。廣南轉運司言:「春州瘴癘之地,配隸至者十死八九,願停配罪人。」詔:「應配沙門島者,許配春州,餘勿配。」既而諸配隸除兇盜外,少壯者並寘河州,止五百人。初,神宗以流人去鄉邑,疾死於道,而護送禁卒,往來勞費,用張誠一之議,隨所在配諸軍重役。後中丞黃履等言,罷之。凡犯盜,刺環於耳後:徒、流,方;杖,圓;三犯杖,移於面。徑不過五分。



 元祐六年,刑部言:「諸配隸沙門
 島,強盜殺人縱火,贓滿五萬錢、強奸毆傷兩犯致死,累贓至二十萬錢、謀殺致死,及十惡死罪,造蠱已殺人者,不移配。強盜徒黨殺人不同謀,贓滿二十五萬,遇赦移配廣南,溢額者配隸遠惡。餘犯遇赦移配荊湖南北、福建路諸州,溢額者配隸廣南。在沙門島滿五年,遇赦不該移配與不許給還而年及六十以上者,移配廣南。在島十年者,依餘犯格移配。篤疾或年及七十、在島三年以上,移配近鄉州軍。犯狀應移而老疾者同。其永不放
 還者,各加二年移配。」後又定令:「沙門島已溢額,移配瓊州、萬安軍、昌化、朱崖軍。」



 紹聖三年,刑部侍郎邢恕等言:「藝祖初定天下,主典自盜,贓滿者往往抵死。仁祖之初,尚不廢也。其後用法稍寬,官吏犯自盜,罪至極法,率多貸死。然甚者猶決刺配島,錢仙芝帶館職,李希甫歷轉運使,不免也。比朝廷用法益寬,主典人吏軍司有犯,例各貸死,略無差別。欲望進述祖宗故事,凡自盜,計贓多者,間出睿斷,以肅中外。」詔:「今後應枉法自盜,罪至死、贓
 數多者,並取旨。」



 或患加役流法太重,官有監驅之勞,而道路有奔亡之慮。蘇頌元豐中嘗建議:「請依古置圜土,取當流者治罪訖,髡首鉗足,晝則居作,夜則置之圜土。滿三歲而後釋,未滿歲而遇赦者,不原。既釋,仍送本鄉,譏察出入。又三歲不犯,乃聽自如。」時未果行。崇寧中,始從蔡京之請,令諸州築圜土以居強盜貸死者。晝則役作,夜則拘之,視罪之輕重,以為久近之限。許出圜土日充軍,無過者縱釋。行之二年,其法不便,乃罷。大觀元年,復行。
 四年,復罷。



 南渡後,諸配隸,《祥符編敕》止四十六條,慶歷中,增至百七十餘條。至於淳熙,又增至五百七十條,則四倍於慶歷矣。配法既多,犯者日眾,黥配之人,所至充斥。淳熙十一年,校書郎羅點言其太重,乃詔刑、寺集議奏聞。至十四年,未有定論。其後臣僚議,以為「若止居役,不離鄉井,則幾惠奸,不足以懲惡;若盡用配法,不恤黥刺,則面目一壞,誰復顧藉?強民適長威力,有過無由自新。檢照《元豐刑部格》,諸編配人自有不移、不放及移放
 條限;《政和編配格》又有情重、稍重、情輕、稍輕四等。若依仿舊格,稍加參訂,如入情重,則仿舊刺面,用不移不放之格;其次稍重,則止刺額角,用配及十年之格;其次稍輕,則與免黥刺,用不刺面、役滿放還之格;其次最輕,則降為居役,別立年限縱免之格。儻有從坐編管,則置之本城,減其放限。如此,則於見行條法並無抵牾,且使刺面之法,專處情犯兇蠹,而其它偶麗於罪,皆得全其面目,知所顧藉,可以自新。省黥徒,銷奸黨,誠天下之切務。」
 即詔有司裁定,其後迄如舊制。



 嘉泰四年,臣僚言:「配隸之人,蓋有兩等。其鄉民一時鬥毆殺傷,及胥吏犯贓貸命流配等,設使逃逸,未必能為大過,止欲從徒,配本州牢城重役,限滿給據,復為良民。至於累犯強盜,及聚眾販賣私商,曾經殺傷捕獲之人,非村民、胥吏之比,欲並配屯駐軍,立為年限,限滿改刺從正軍。」從之。其所配之地,自高宗來,或配廣南海外四州,或配淮、漢、四川,迄度宗之世無定法,皆不足紀也。



 凡內外所上刑獄,刑部、審刑院、大理寺參主之,又有糾察在京刑獄司以相審覆。官制即行,罷審刑、糾察,歸其職於刑部。四方之獄,則提點刑獄統治之。官司之獄:在開封,有府司、左右軍巡院;在諸司,有殿前、馬步軍司及四排岸;外則三京府司、左右軍巡院,諸州軍院、司理院,下至諸縣皆有獄。諸獄皆置樓牖,設漿鋪席,持具沐浴,食令溫暖,寒則給薪炭、衣物,暑則五日一滌枷杻。郡縣則所職之官躬行檢視,獄敝則修之使固。



 神宗即位初,
 詔曰:「獄者,民命之所系也。比聞有司歲考天下之奏,而多瘐死。深惟獄吏並緣為奸,檢視不明,使吾元元橫罹其害。《書》不云乎:『與其殺不辜,寧失不經。』其具為令:應諸州軍巡司院所禁罪人,一歲在獄病死及二人,五縣以上州歲死三人,開封府司、軍巡歲死七人,推吏、獄卒皆杖六十,增一人則加一等,罪止杖一百。典獄官如推獄,經兩犯即坐從違制。提點刑獄歲終會死者之數上之,中書檢察。死者過多,官吏雖已行罰,當更黜責。」



 未幾,復
 詔:「失入死罪,已決三人,正官除名編管,貳者除名,次貳者免官勒停,吏配隸千里。二人以下,視此有差。不以赦降、去官原免。未決,則比類遞降一等;赦降、去官,又減一等。令審刑院、刑部斷議官,歲終具嘗失入徒罪五人以上,京朝官展磨勘年,幕職、州縣官展考,或不與任滿指射差遣,或罷,仍即斷絕支賜。」以前法未備,故有是詔。又嘗詔:「官司失入人罪,而罪人應原免,官司猶論如法,即失出人罪。若應徒而杖,罪人應原免者,官司乃得用因
 罪人以致罪之律。」



 帝以國初廢大理獄非是,元豐元年詔曰:「大理有獄尚矣。今中都官有所劾治,皆寓系開封諸獄,囚既猥多,難於隔訊,盛夏疾疫,傳致瘐死,或主者異見,歲時不決,朕甚愍焉。其復大理獄,置卯一人,少卿二人,丞四人,專主鞫訊;檢法官二人,主簿一人。應三司、諸寺監吏犯杖、笞不俟追究者,聽即決,餘悉送大理獄。其應奏者,並令刑部、審刑院詳斷。應天下奏按亦上之。」五年,分命少卿左斷刑、右治獄。斷刑則評事、檢法詳斷,丞議,
 正審;治獄則丞專推劾,主簿掌按劾,少卿分領其事,而卿總焉。六年,刑部言:「舊詳斷官分公按訖,主判官論議改正,發詳議官復議。有差失問難,則書於檢尾,送斷官改正,主判官審定,然後判成。自詳斷官歸大理為評事、司直,議官為丞,所斷按草,不由長貳,類多差忒。」乃定制:「服評事、司直與正為斷司,丞與長貳為議司。凡斷公按,正先詳其當否,論定則簽印注日,移議司復議,有辨難,乃具議改正,長貳更加審定,然後判成錄奏。



 元祐初,三
 省言:「舊置糾察司,蓋欲察其違慢,所以謹重獄事,罷歸刑部,無復糾察之制。請以糾察職事委御史臺刑察兼之,臺獄則尚書省右司糾察之。」



 三年,罷大理寺獄。初,大理置獄,本以囚系淹滯,俾獄事有所統,而大理卿崔臺符等不能奉承德意,雖士大夫若命婦,獄辭小有連逮,輒捕系。凡邏者所探報,即下之獄。傅會鍛煉,無不誣服。至是,臺符等皆得罪,獄乃罷。



 八年,中書省言:「昨詔同外,歲終具諸獄囚死之數。而諸路所上,遂以禁系二十而
 死一者不具,即是歲系二百人,許以十人獄死,恐州縣弛意獄事,甚非欽恤之意。詔刑部自今不許輒分禁系之數。紹聖二年,戶部如三司故事,置推勘檢法官,應在京諸司事幹錢穀當追究者,從杖已下即定斷。



 三年,復置大理寺右治獄,官屬視元豐員,仍增置司直一員。大理卿路昌衡請:「分大理寺丞為左、右推,若有翻異,自左移右。再變,即命官審問,或御史臺推究。不許開封府互勘及地分探報,庶革互送挾仇之弊。徒已上罪移御史
 臺。命官追攝者,悉依條。若探報涉虛、用情托者,並收坐以聞。」



 初,法寺斷獄,大闢失入有罰,失出不坐。至是,以失出死罪五人比失入一人,失出徒、流罪三名,亦如之。著為令。元符三年,刑部言:「祖宗重失入之罪,所以恤刑。夫失出,臣下之小過;好生,聖人之大德。請罷失出之責,使有事讞議之間,務盡忠恕。」詔可。政和三年,臣僚言:「遠方官吏,文法既疏,刑罰失中,不能無冤。願委耳目之官,季一分錄所部囚禁,遇有冤抑,先釋而後以聞。歲終較所
 釋多寡,為之殿最。其徼功故出有罪者,論如法。」詔令刑部立法:諸入人徒、流之罪已結案,而錄問官吏能駁正,或因事而能推正者,累及七人,比大闢一名推賞。



 紹興六年,令諸鞫勘有情款異同而病死者,提刑司研究之,如冤,申朝廷取旨。十二年,令諸推究翻異獄,毋差初官、蔭子及新進士,擇曾經歷任人。二十七年,令監察御史每冬夏點獄,有鞫勘失實者,照刑部郎官,直行移送。二十九年,令殺人無證、尸不經驗之獄,具案奏裁,委提刑
 審問。如有可疑及翻異,從本司差官重勘,案成上本路,移他監司審定,具案聞奏。否則監司再遣官勘之,又不伏,復奏取旨。先是,有司建議:「外路獄三經翻異,在千里內者移大理寺。」三十一年,刑部以為非祖宗法,遂厘正之。乾道中,諸州翻異之囚,既經本州,次檄鄰路,或再翻異,乃移隔路,至有越兩路者。官吏旁午於道,逮系者困於追對。四年,乃令:「鞫勘本路累嘗差官猶稱冤者,惟檄鄰路,如尚翻異,則奏裁。」淳熙三年,令縣尉權縣事,毋自
 鞫獄,即令丞、簿參之。全闕,則於州官或鄰縣選官權攝之。



 金作贖刑,蓋以鞭撲之罪,情法有可議者,則寬之也。穆王贖及五刑,非法矣。宋損益舊制,凡用官蔭得減贖,所以尊爵祿、養廉恥也。乾德四年,大理正高繼申上言:「《刑統名例律》:三品、五品、七品以上官,親屬犯罪,各有等第減贖。恐年代已深,不肖自恃先蔭,不畏刑章。今犯罪身無官,須祖、父曾任本朝官,據品秩得減贖。如仕於前代,須有功惠及民、為時所推、歷官三品以上,乃得請。」從
 之。後又定:「流內品官任流外職,準律文,徒罪以上依當贖法。諸司授勒留官及歸司人犯徒流等罪,公罪許贖,私罪以決罰論。」淳化四年,詔諸州民犯罪,或入金贖,長吏得以任情而輕重之,自今不得以贖論。婦人犯杖以下,非故為,量輕重笞罰或贖銅釋之。



 仁宗深憫夫民之無知也,欲立贖法以待薄刑,乃詔有司曰:「先王用法簡約,使人知禁而易從。後代設茶、酒、監稅之禁,奪民厚利,刑用滋章。今之《編敕》,皆出律外,又數改更,官吏且不能
 曉,百姓安得聞之?一陷於理,情雖可哀,法不得贖。豈禮樂之化未行,而專用刑罰之弊與?漢文帝使天下人入粟於邊,以受爵免罪,幾於刑措。其議科條非著於律者,或冒利犯禁,奢侈違令,或過誤可憫,別為贖法。鄉民以穀麥,市人以錢帛,使民重穀麥,免刑罰,則農桑自勸,富壽可期矣。」詔下,論者以為富人得贖而貧者不能免,非朝廷用法之意。時命輔臣分總職事,以參知政事範仲淹領刑法,未及有所建明而仲淹罷,事遂寢。至和初,又
 詔:「前代帝王後,嘗仕本朝,官不及七品者,祖父母、父母、妻子罪流以下,聽贖。雖不仕而嘗被賜予者,有罪,非巨蠹,亦如之。」隨州司理參軍李父抃毆人死,抃上所授官以贖父罪,帝哀而許之。君子謂之失刑,然自是未嘗為此。而終宋之世,贖法惟及輕刑而已。



 恩宥之制,凡大赦及天下,釋雜犯死罪以下,甚則常赦所不原罪,皆除之。凡曲赦惟一路或一州,或別京,或畿內。凡德音,則死及流罪降等,餘罪釋之,間亦釋流罪。所被廣狹無常。又,天
 子歲自錄京師系囚,畿內則遣使,往往雜犯死罪以下第降等,杖、笞釋之,或徒罪亦得釋。若並及諸路,則命監司錄焉。



 初,太宗嘗因郊禮議赦,有秦再恩者,上書願勿赦,引諸葛亮佐劉備數十年不赦事。帝頗疑之。時趙普對曰:「凡郊祀肆眚,聖朝彞典,其仁如天,若劉備區區一方,臣所不取。」上善之,遂定赦。



 初,太祖將祀南郊,詔:「兩京、諸道,自十月後犯強竊盜,不得預郊祀之赦。所在長吏告諭,民無冒法。」是後將祀,必先申明此詔。天聖五年,馬
 亮言:「朝廷雖有是詔,而法官斷獄乃言終是會赦,多所寬貸,惠奸失詔旨。」遂詔:「已下約束而犯劫盜,及官典受贓,勿復奏,悉論如律。」七年春,京師雨,彌月不止。仁宗謂輔臣曰:「豈政事未當天心耶?」因言:「向者大闢覆奏,州縣至於三,京師至於五,蓋重人命如此。其戒有司,決獄議罪,毋或枉濫。」又曰:「赦不欲數,然舍是無以召和氣。」遂命赦天下。



 帝在位久,明於人之情偽,尤惡訐人陰事,故一時士大夫習為惇厚。久之,小人乘間密上書,疏人過失,
 好事稍相與唱和,又按人赦前事。翰林學士張方平、御史呂誨以為言,因下詔曰:「蓋聞治古,君臣同心,上下協穆,而無激訐之俗,何其德之盛也!朕竊慕焉。嘉與公卿大夫同底斯道,而教化未至,澆薄日滋。比者中外群臣,多上章言人過失,暴揚難驗之罪,或外托公言,內緣私忿,詆欺暖昧,茍陷善良。又赦令者,所以與天下更始,而有司多舉按赦前之事,殆非信命令,重刑罰,使人灑心自新之意也。今有上言告人罪,言赦前事者,訊之。至於言
 官,宜務大體,非事關朝政,自餘小過細故,勿須察舉。」



 神宗即位,又詔曰:「夫赦令,國之大恩,所以蕩滌瑕穢,納於自新之地,是以聖王重焉。中外臣僚多以赦前事捃摭吏民,興起獄訟,茍有詿誤,咸不自安,甚非持心近厚之義,使吾號令不信於天下。其內外言事、按察官,毋得依前舉劾,具按取旨,否則科違制之罪。御史臺覺察彈奏,法寺有此奏按,許舉駁以聞。」知諫院司馬光言曰:「按察之官,以赦前事興起獄訟,禁之誠為大善。至於言事之
 官,事體稍異。何則?御史之職,本以繩按百僚,糾摘隱伏。奸邪之狀,固非一日所為。國家素尚寬仁,數下赦令,或一歲之間至於再三,若赦前之事皆不得言,則其可言者無幾矣。萬一有奸邪之臣,朝廷不知,誤加進用,御史欲言則違今日之詔,若其不言,則陛下何從知之?臣恐因此言者得以借口偷安,奸邪得以放心不懼。此乃人臣之至幸,非國家之長利也。請追改前詔,刊去『言事』兩字。」光論至再,帝諭以「言者好以赦前事誣人」,光對曰:「若
 言之得實,誠所欲聞,若其不實,當罪言者。」帝命光送詔於中書。



 熙寧七年三月,帝以旱,欲降赦。時已兩赦,王安石曰:「湯旱,以六事自責曰:『政事不節與?』若一歲三赦,是政不節矣,非所以弭災也。」乃止。八年,編定《廢免人敘格》,常赦則郡縣以格敘用,凡三期一敘,即期未滿而遇非次赦者,亦如之。



 元祐元年,門下省言:「當官以職事墮曠,雖去官不免,猶可言。至於赦降大恩,與物更始,雖劫盜殺人亦蒙寬宥,豈可以一事差失,負罪終身?今刑部所
 修不以去官、赦降原減條,請更刪改。」



 徽宗在位二十五年,而大赦二十六,曲赦十四,德音三十七。而南渡之後,紹熙歲至四赦,蓋刑政紊而恩益濫矣。



 宋自祖宗以來,三歲遇郊則赦,此常制也。世謂三歲一赦,於古無有。景祐中,言者以為:「三王歲祀圜丘,未嘗輒赦。自唐興兵以後,事天之禮不常行,因有大赦,以蕩亂獄。且有罪者寬之未必自新,被害者抑之未必無怨。不能自新,將復為惡,不能無怨,將悔為善。一赦而使民悔善長惡,政教之大
 患也。願罷三歲一赦,使良民懷惠,兇人知禁。或謂未可盡廢,即請命有司,前郊三日理罪人,有過誤者引而赦之。州縣須詔到仿此。」疏奏,朝廷重其事,第詔:「罪人情重者,毋得以一赦免。」然亦未嘗行



\end{pinyinscope}