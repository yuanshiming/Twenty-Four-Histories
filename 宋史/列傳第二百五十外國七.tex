\article{列傳第二百五十外國七}

\begin{pinyinscope}

 流求國在泉州之東,有海島曰彭湖,煙火相望。其國塹柵三重,環以流水,植棘為藩,以刀槊弓矢劍鈹為兵器,
 視月盈虧以紀時。無他奇貨,商賈不通,厥土沃壤,無賦斂,有事則均稅。



 旁有毗舍邪國,語言不通,袒裸盱睢,殆非人類。淳熙間,國之酋豪嘗率數百輩猝至泉之水沃、圍頭等村,肆行殺掠。喜鐵器及匙箸,人閉戶則免,但刓其門圈而去。擲以匙箸則頫拾之,見鐵騎則爭剚其甲,駢首就戮而不知悔。臨敵用標槍,繫繩十餘丈為操縱,蓋惜其鐵不忍棄也。不駕舟楫,惟縛竹為筏,急則群舁之泅水而遁。
 國



 定安國本馬韓之種,為契丹所攻破,其酋帥糾合餘衆,保於西鄙,建國改元,自稱定安國。開寶三年,其國王烈萬華因女真遣使入貢,乃附表貢獻方物。太平興國中,太宗方經營遠略,討擊契丹,因降詔其國,令張犄角之勢。其國亦怨寇仇侵侮不已,聞中國用兵北討,欲依王師以攄宿憤,得詔大喜。



 六年冬,會女真遣使來貢,路由本國,乃托其使附表來上云:「定安國王臣烏玄明言:伏遇聖主洽天地之恩,撫夷貊之俗,臣玄明誠喜誠抃,頓
 首頓首。臣本以高麗舊壤,渤海遺黎,保據方隅,涉曆星紀,仰覆露鴻鈞之德,被浸漬無外之澤,各得其所,以遂本性。而頃歲契丹恃其強暴,入寇境土,攻破城砦,俘略人民。臣祖考守節不降,與衆避地,僅存生聚,以迄於今。而又扶餘府昨背契丹,並歸本國,災禍將至,無大於此。所宜受天朝之密畫,率勝兵而助討,必欲報敵,不敢違命。臣玄明誠懇誠願,頓首頓首。」其末題云:「元興六年十月日,定安國王臣玄明表上聖皇帝前。」



 上答以詔書曰:「
 敕定安國王烏玄明。女真使至,得所上表,以朕嘗賜手詔諭旨,且陳感激。卿遠國豪帥,名王茂緒,奄有馬韓之地,介於鯨海之表,強敵吞併,失其故土,沉冤未報,積憤奚伸。矧彼獯戎,尚搖蠆毒,出師以薄伐,乘夫天災之流行,敗衄相尋,滅亡可待。今國家已于邊郡廣屯重兵,只俟嚴冬,即申天討。卿若能追念累世之恥,宿戒舉國之師,當予伐罪之秋,展爾復仇之志,朔漠底定,爵賞有加,宜思永圖,無失良便。而況渤海願歸於朝化,扶余已背
 於賊庭,勵乃宿心,糾其協力,克期同舉,必集大勳。尚阻重溟,未遑遣使,倚注之切,鑒寐寧忘。」以詔付女真使,令齎以賜之。



 端拱二年,其王子因女真使附獻馬、雕羽鳴鏑。淳化二年,其五子太元因女真使上表,其後不復至。



 渤海本高麗之別種。唐高宗平高麗,徙其人居中國。則天萬歲通天中,契丹攻陷營府,高麗別種大祚榮走保遼東,睿宗以為忽汗州都督,封渤海郡王,因自稱渤海國,並有扶餘、肅慎等十餘國,歷唐、梁、後唐,朝貢不絕。



 後
 唐天成初,為契丹阿保機攻扶餘城下之,改扶餘為東丹府,命其子突欲留兵鎮之。阿保機死,渤海王復攻扶餘,不能克。歷長興、清泰,遣使朝貢。周顯德初,其酋豪崔烏斯等三十人來歸,其後隔絕不能通中國。



 太平興國四年,太宗平晉陽,移兵幽州,其酋帥大鸞河率小校李勳等十六人、部族三百騎來降,以鸞河為渤海都指揮使。六年,賜烏舍城浮渝府渤海琰府王詔曰:「朕纂紹丕構,奄有四海,普天之下,罔不率俾。矧太原封域,國之保
 障,頃因竊據,遂相承襲,倚限定為援,歷世逋誅。朕前歲親提銳旅,盡護諸將,拔並門之孤壘,斷匈奴之右臂,眷言吊伐,以蘇黔黎。蠢茲北戎,非理構怨,輒肆薦食,犯我封略。一昨出師逆擊,斬獲甚衆。今欲鼓行深入,席捲長驅,焚其龍庭,大殲丑類。素聞爾國密邇寇仇,迫於吞併,力不能制,因而服屬,困於率割。當靈旗破敵之際,是鄰邦雪憤之日,所宜盡出族帳,佐予兵鋒。俟其翦滅,沛然封賞,幽、薊土宇,復歸中原,朔漠之外,悉以相與。勖乃協力,
 朕不食言。」時將大舉征契丹,故降是詔諭旨。



 九年春,宴大明殿,因召大鸞河慰撫久之。上謂殿前都校劉延翰曰:「鸞河,渤海豪帥,束身歸我,嘉其忠順。夫夷落之俗,以馳騁為樂,候高秋戒候,當與駿馬數十匹,令出郊遊獵,以遂其性。」因以緡錢十萬並酒賜之。



 日本國者,本倭奴國也。自以其國近日所出,故以日本為名;或云惡其舊名改之也。其地東西南北各數千里,西南至海,東北隅隔以大山,山外即毛人國。自後漢始
 朝貢,歷魏、晉、宋、隋皆來貢,唐永徽、顯慶、長安、開元、天寶、上元、貞元、元和、開成中,並遣使入朝。



 雍熙元年,日本國僧奝然與其徒五六人浮海而至,獻銅器十餘事,並本國《職員今》、《王年代紀》各一卷。奝然衣綠,自云姓藤原氏,父為真連;真連,其國五品品官也。奝然善隸書,而不通華言,問其風土,但書以對云:「國中有《五經》書及佛經、《白居易集》七十卷,並得自中國。土宜五穀而少麥。交易用銅錢,文曰『乾文大寶』。畜有水牛、驢、羊,多犀、象。產絲蠶,多
 織絹,薄致可愛。樂有中國、高麗二部。四時寒暑,大類中國。國之東境接海島,夷人所居,身面皆有毛。東奧州產黃金,西別島出白銀,以為貢賦。國王以王為姓,傳襲至今王六十四世,文武僚吏皆世官。」



 其《年代紀》所記云:「初主號天御中主。次曰天村雲尊,其後皆以『尊』為號。次天八重雲尊,次天彌聞尊,次天忍勝尊,次瞻波尊,次萬魂尊,次利利魂尊,次國狹槌尊,次角龔魂尊,次汲津丹尊,次面垂見尊,次國常立尊,次天鑒尊,次天萬尊,次沫名
 杵尊,次伊奘諾尊,次素戔烏尊,次天照大神尊,次正哉吾勝速日天押穗耳尊,次天彥尊,次炎尊,次彥瀲尊,凡二十三世,並都於築紫日向宮。



 彥瀲第四子號神武天皇,自築紫宮入居大和州橿原宮,即位元年甲寅,當周僖王時也。次綏靖天皇,次安寧天皇,次懿德天皇,次孝昭天皇,次孝天皇,次孝靈天皇,次孝元天皇,次開化天皇,次崇神天篁,次垂仁天皇,次景行天皇,次成務天皇。次仲哀天皇,國人言今為鎮國香椎大神。次神功天皇,
 開化天皇之曾孫女,又謂之息長足姬天皇,國人言今為太奈良姬大神。次應神天皇,甲辰歲,始於百濟得中國文字,今號八蕃菩薩,有大臣號紀武內,年三百七歲。次仁德天皇,次履中天皇,次反正天皇,次允恭天皇,次安康天皇,次雄略天皇,次清甯天皇,次顯宗天皇,次仁賢天皇,次武烈天皇,次繼體天皇,次安開天皇,次宣化天皇。次天國排開廣庭天皇,亦名欽明天皇,即位十三年,壬申歲始傳佛法于百濟國,當此土梁承聖元年。



 次
 敏達天皇。次用明天皇,有子曰聖德太子,年三歲,聞十人語,同時解之,七歲悟佛法於菩提寺,講《聖曼經》,天雨曼陀羅華。當此土隋開皇中,遣使泛海至中國,求《法華經》。



 次崇峻天皇。次推古天皇,欽明天皇之女也。次舒明天皇,次皇極天皇。次孝德天皇,白雉四年,律師道照求法至中國,從三藏僧玄奘受經、律、論,當此土唐永徽四年也。次天豐財重日足姬天皇,令僧智通等入唐求大乘法相教,當顯慶三年。次天智天皇,次天武天皇,次持
 總天皇。次文武天皇,大寶三年,當長安元年,遣粟田真人入唐求書籍,律師道慈求經。次阿閉天皇,次皈依天皇。次聖武天皇,寶龜二年,遣僧正玄昉入朝,當開元四年。次孝明天皇,聖武天皇之女也,天平勝寶四年,當天寶中,遣使及僧入唐求內外經教及傳戒。次天炊天皇。次高野姬天皇,聖武天皇之女也。次白璧天皇,二十四年,遣二僧靈仙、行賀入唐,禮五臺山學佛法。次桓武天皇,遣騰元葛野與空海大師及延曆寺僧澄入唐,詣天
 臺山傳智者止觀義,當元和元年也。次諾樂天皇,次嵯峨天皇,次淳和天皇。次仁明天皇,當開成、會昌中,遣僧入唐,禮五台。次文德天皇,當大中年間。次清和天皇,次陽成天皇。次光孝天皇,遣僧宗睿入唐傳教,當光啟元年也。



 次仁和天皇,當此土梁龍德中,遣僧寬建等入朝。次醍醐天皇,次天慶天皇。次封上天皇,當此土周廣順年也。次冷泉天皇,今為太上天皇。次守平天皇,即今王也。凡六十四世。



 畿內有山城、大和、河內、和泉、攝津凡五
 州,共統五十三郡。東海道有伊賀、伊勢、志摩、尾張、三河、遠江、駿河、伊豆、甲斐、相模、武藏、安房、上總、常陸凡十四州,共統一百一十六郡。東山道有通江、美濃、飛驒、信濃、上野、下野、陸奧、出羽凡八州,共統一百二十二郡。北陸道有若狹、越前、加賀、能登、越中、越後、佐渡凡七州,共統三十郡。山陰道有丹波、丹彼、徂馬、因幡、伯耆、出雲、石見、隱伎凡八州,共統五十二郡。小陽道有播麼、美作、備前、備中、備後、安藝、周防、長門凡八州,共統六十九郡。南海道
 有伊紀、淡路、河波、贊耆、伊豫、土佐凡六州,共統四十八郡。西海道有築前、築後、豐前、豐後、肥前、肥後、日向、大隅、薩摩凡九州,共統九十三郡。又有壹伎、對馬、多礻執凡三島,各統二郡。是謂五畿、七道、三島,凡三千七百七十二都,四百一十四驛,八十八萬三千三百二十九課丁。課丁之外,不可詳見。皆奝然所記雲。



 按隋開皇二十年,倭王姓阿每,名自多利思比孤,遣使致書。唐永徽五年,遣使獻琥珀、馬腦。長安二年,遣其朝臣真人貢方物。開元
 初,遣使來朝。天寶十二年,又遣使來貢。元和元年,遣高階真人來貢。開成四年,又遣使來貢。此與其所記皆同。大中、光啟、龍德及周廣順中,皆嘗遣僧至中國,《唐書》中、《五代史》失其傳。唐咸亨中及開元二十三年、大曆十二年、建中元年,皆來朝貢,其記不載。



 太宗召見奝然,存撫之甚厚,賜紫衣,館于太平興國寺。上聞其國王一姓傳繼,臣下皆世官,因歎息謂宰相曰:「此島夷耳,乃世祚遐久,其臣亦繼襲不絕,此蓋古之道也。中國自唐季之亂,
 宇縣分裂,梁、周五代享曆尤促,大臣世胄,鮮能嗣續。朕雖德慚往聖,常夙夜寅畏,講求治本,不敢暇逸。建無窮之業,垂可久之範,亦以為子孫之計,使大臣之後世襲祿位,此朕之心焉。」



 其國多有中國典籍,奝然之來,復得《孝經》一卷、越王《孝經新義》第十五一卷,皆金縷紅羅標,水晶為軸。《孝經》即鄭氏注者。越王者,乃唐太宗子越王貞;《新義》者,記室參軍任希古等撰也。奝然復求詣五臺,許之,令所過續食;又求印本《大藏經》,詔亦給之。二年,隨
 台州甯海縣商人鄭仁德船歸其國。



 後數年,仁德還。奝然遣其弟子喜因奉表來謝曰:「日本國東大寺大朝法濟大師、賜紫、沙門奝然啟:傷鱗入夢,不忘漢主之恩;枯骨合歡,猶亢魏氏之敵。雖云羊僧之拙,誰忍鴻霈之誠。奝然誠惶誠恐,頓首頓首,死罪。奝然附商船之離岸,期魏闕於生涯,望落日而西行,十萬里之波濤難盡;顧信風而東別,數千里之山嶽易過。妄以下根之卑,適詣中華之盛。於是宣旨頻降,恣許荒外之跋涉;宿心克協,粗
 觀宇內之瑰奇。況乎金闕曉後,望堯雲於九禁之中;岩扃晴前,拜聖燈於五台之上。就三藏而稟學,巡數寺而優遊。遂使蓮華回文,神筆出於北闕之北;貝葉印字,佛詔傳於東海之東。重蒙宣恩,忽趁來跡。季夏解台州之纜,孟秋達本國之郊。爰逮明春,初到舊邑,緇素欣待,侯伯慕迎。伏惟陛下惠溢四溟,恩高五嶽,世超黃、軒之古,人直金輪之新。奝然空辭鳳凰之窟,更還螻蟻之封。在彼在斯,只仰皇德之盛;越山越海,敢忘帝念之深。縱粉
 百年之身,何報一日之惠。染筆拭淚,伸紙搖魂,不勝慕恩之至。謹差上足弟子傳燈大法師位嘉因、並大朝剃頭受戒僧祚乾等拜表以聞。」稱其本國永延二年歲次戊子二月八日,實也。



 又別啟,貢佛經,納青木函;琥珀、青紅白水晶、紅黑木槵子念珠各一連,並納螺鈿花形平函;毛籠一,納螺杯二口;葛籠一,納法螺二口,染皮二十枚;金銀蒔繪筥一合,納發鬘二頭,又一合,納參議正四位上藤佐理手書二卷、及進奉物數一卷、表
 狀一卷;又金銀蒔繪硯一筥一合,納金硯一、鹿毛筆、松煙墨、金銅水瓶、鐵刀;又金銀蒔繪扇筥一合,納檜扇二十枚、蝙蝠扇二枚;螺鈿梳函一對,其一納赤木梳二百七十,其一納龍骨十橛;螺鈿書案一、螺鈿書幾一;金銀蒔繪平筥一合,納白細布五匹;鹿皮籠一,納䝚裘一領;螺鈿鞍轡一副,銅鐵鐙、紅絲秋、泥障;倭畫屏風一雙;石流黃七百斤。



 ,建州海賈周世昌遭風飄至日本,凡七年得還,與其國人滕木吉至,上皆召見之。世昌
 以其國人唱和詩來上,詞甚雕刻膚淺無所取。詢其風俗,云婦人皆被髪,一衣用二三縑。又陳所記州名年號。上令滕木吉以所持木弓矢挽射,矢不能遠,詰其故,國中不習戰鬥。賜木吉時裝錢遣還。,其國僧寂昭等八人來朝,寂照不曉華言,而識文字,繕寫甚妙,凡問答並以筆劄。詔號圓通大師,賜紫方袍。天聖四年十二月,明州言日本國太宰府遣人貢方物,而不持本國表,詔卻之。其後亦未通朝貢,南賈時有傳其物貨至中
 國者。



 熙寧五年,有僧誠尋至台州,止天台國清寺,願留。州以聞,詔使赴闕。誠尋獻銀香爐,木槵子、白琉璃、五香、水精、紫檀、琥珀所飾念珠,及青色織物綾。神宗以其遠人而有戒業,處之開寶寺,盡賜同來僧紫方袍。是後連貢方物,而來者皆僧也。,使通事僧仲回來,賜號慕化懷德大師。明州又言得其國太宰府牒,因使人孫忠還,遣仲回等貢絁二百匹、水銀五千兩,以孫忠乃海商,而貢禮與諸國異,請自移牒報,而答其物直,付仲
 回東歸。從之。



 乾道九年,始附明州綱首以方物入貢。淳熙二年,倭船火兒滕太明毆鄭作死,詔械太明付其綱首歸,治以其國之法。三年,風泊日本舟至明州,衆皆不得食,行乞至臨安府者復百餘人。詔人日給錢五十文、米二升,俟其國舟至日遣歸。十年,日本七十三人復飄至秀州華亭縣,給常平義倉錢米以振之。紹熙四年,泰州及秀州華亭縣復有倭人為風所泊而至者,詔勿取其貨,出常平米振給而遣之。慶元六年至平江府,
 至定海縣,詔並給錢米遣歸國。



 党項,古析支之地,漢西羌之別種。後周世始強盛,有細風氏、費聽氏、往利氏、頗超氏、野亂氏、房當氏、來禽氏、拓拔氏最為強族。唐貞觀至上元間內附,散居西北邊。元和以後,頗相率為盜。會昌初,武宗置三使以統之:在邠、寧、延者為一使,在鹽、夏、長澤者為一使,在靈武、麟、勝者為一使。五代亦嘗入貢。今靈、夏、綏、麟、府、環、慶、豐州,鎮戎、天德、振武軍並其族帳。



 太祖建隆二年,代州刺史折乜
 埋來朝。乜埋,党項之大姓,世居河右,有捍邊之功,故授以方州,召令入覲而遣還。



 開寶元年,直蕩族首領啜佶等引並人寇府州,為王師所敗。詔內屬羌部十六府大首領屈遇與十二府首領羅崖領所部誅啜佶,啜佶懼,以其族歸順。以屈遇為檢校太保、歸德將軍,羅崖、啜佶並為檢校司徒、懷化將軍。



 太平興國二年二月,靈州部送歲市官馬,賂所過族帳物粗惡,羌人恚不受。知州、比部郎中張全操捕得十八人殺之,沒入其兵仗羊馬,戎
 人遂擾。上遣使齎金帛撫賜其族,與之盟,始定。召全操下有司鞫之,決杖流登州沙門島。是歲,靈州通遠軍界嗓咩族、折四族、吐蕃村族、柰㖞三家族、尾落族、柰家族、嗓泥族剽略官綱,詔靈州安守忠、通遠軍董遵誨討平之。六年,府州外浪族首領來都等來貢馬。七年,豐州大首領黃羅並弟乞蚌等來貢馬。又銀州羌部拓跋遇來訴本州賦役苛虐,乞移居內地,詔令各守族帳。又保細族結集扇動諸部,夏州巡檢使梁迥率兵討平之。



 雍熙
 初,諸族渠帥附李繼遷為寇,詔判四方館事田仁朗及閣門使王侁等相繼領兵討擊,並賜麟、府、銀、夏、豐州及日利、月利族敕書招諭之。



 二年四月,侁等於銀州北破悉利諸族,斬首三千六百餘級,生擒八十人,俘老小一千四百餘口,器甲一百八十六,梟偽署代州刺史折羅遇並弟埋乞,獲馬牛羊三萬計。五月,又于開光谷西杏子平破保寺、保香族,追奔二十餘里,斬首八百餘級,梟其首領埋乜已等五十七人,生擒四十九人,俘其老小
 三百餘人,獲牛羊馬驢凡四千余計。又破保、洗兩族,俘三千人,降五十五族,獲牛羊八千計。



 侁等又言,麟州及三族砦羌人二千餘戶皆降,酋長折御乜等六十四人獻馬首罪,願改圖自效,為國討賊,遂與部下兵入濁輪川,斬賊首五十級、酋豪二十人,李繼遷及三族砦監押折御乜皆遁去。旋命內客省使郭守文自三交乘驛亟往,與王侁等同領邊事。五月,王侁、李繼隆等又破銀州杏子平東北山谷內沒邵、浪悉訛等族,及濁輪川東、兔
 頭川西諸族,生擒七十八人,梟五十九人,俘二百三十六口,牛羊驢馬千二百六十,招降千四百五十二戶。



 六月,夏州尹憲等引兵至鹽城,吳移、越移等四族來降,憲等撫之。岌伽羅膩十四族拒命,憲等縱兵斬首千餘級,俘擒百人,焚千餘帳,獲馬牛羊七千計。又降銀麟夏等州、三族砦諸部一百二十五族,合萬六千一百八十九戶。酋豪折御乜窮蹙來歸,守文置之部下。又夏州咩嵬族魔病人乜崖在南山族結黨為寇,招懷不至,擒斬之,
 梟首徇衆,並滅其族。又府州女乜族首領來母崖男社正等內附,因遷居茗乜族中。



 七月,賜宥州界咩兀十族首領、都指揮使遇乜布等九人敕書,以安撫之。十一月,以勒浪族十六府大首領屈遇、名波族十二府大首領浪買當豐州路最為忠順,及兀泥三族首領佶移等、女女四族首領殺越都等歸化,並賜敕書撫之。



 端拱元年三月,火山軍言河西羌部直蕩族內附。二年四月,夏州趙保忠言:「臣准詔市馬,已獲三百匹,其宥州御泥布、囉
 樹等二族黨附繼遷,不肯賣馬,臣遂領兵掩殺二百餘人,擒百餘人,其族即降,各已安撫。」詔書獎諭之。十月,繼遷寇會州熟倉族,為其首領咩㗭率來離諸族擊走之。



 淳化元年,藏才三族都判啜尾卒,其子啜香來請命,乃令代其父。二年七月,以黃乜族降戶七百余散于銀、夏州舊地處之。八月,李繼遷居王庭鎮,趙保忠往襲之,繼遷奔鐵斤澤,貌奴、猥才二族奪其牛畜二萬餘。十一月,繼遷寇熟倉族,刺史咩㗭率來離諸族擊退之。先是,兀
 泥大首領泥中佶移內附,詔授慎州節度,俄復歸繼遷,其長子突厥羅與首領黃羅至是以千餘帳降,府州折御卿以聞,降詔慰諭之。趙保忠又襲破宥州御泥布、囉樹二族,尋各降之,以其朋附繼遷,來上。



 四年三月,直蕩族大首領啜尾、子河𣿭大首領馬一併來貢,詔以啜尾叔羅買為本族都監,又啜尾下首領十人、馬一下首領十二人皆賜錦袍、銀帶、器幣。是年,鄭文寶獻議禁青鹽,羌族四十四首領盟于楊家族,引兵騎萬三千餘人入
 寇環州石昌鎮,知環州程德玄等擊走之。因詔屯田員外郎、知制誥錢若水馳驛詣邊,馳其鹽禁,由是部族寧息。十二月,鹽州羌人酋長巢延渭為本州刺史。是年,藏才西族大首領羅妹來貢。



 五年正月,以綏州羌酋蘇移、山海㖡、母馱香三人並為懷化將軍,野利、嵬名乜屈、啜泥三人並為歸德郎將。四月,府州折御卿言:銀、夏州管勾生戶八千帳族悉來歸附,錄其馬牛羊萬計。邈二族大首領崖羅、藏才東族首領歲囉啜克各遣其子弟朝
 貢。六月,繼遷所驅脅內屬戎人橐駝路熟藏族首領乜遇率部族反攻繼遷,其弟力戰而死,既敗繼遷之衆,復來歸附。以遇為檢校司空,領會州刺史。是年,兀泥族首領黃羅內附,以為懷化將軍,領昭州刺史。



 至道元年四月,以勒浪嵬女兒門十六府大首領馬尾等內附,以馬尾為歸德大將軍、領恩州刺史,以勒浪樹李兒門首領沒崖為安化郎將,副首領遇兀為保順郎將。六月,賜慶州界首領順州刺史李奉明、澄州刺史李彥咩、鹽州刺
 史巢延渭、演州刺史李順忠、環州界首領會州刺史乜遇及靈州界並河外保安、保靖、臨河、懷遠、定遠五鎮等部敕書慰撫之。七月,睡泥族首領你乜逋令男詣靈州,言族內七百余帳為李繼遷劫略,首領𠵚逋一族奔往蕭關,你乜逋一族乞賜救助,詔賜以資糧。環州熟倉族癿遇略奪繼遷牛馬三十餘,繼遷令人招撫之,癿遇答云:「吾一心向漢,誓死不移。」詔以遇為會州刺史,賜帛五十匹、茶五十斤。



 二年三月,以府州界五族大首領折突
 厥移為安遠大將軍,父死來請命也。六月,勒浪族副首領遇兀等百九十三人歸附,貢馬七匹。遇兀舊隸契丹,淳化初,遷族帳於府州界,東至河百五十里,南至府州三百里,至是,始朝貢。上召問慰勞,賜錦袍銀帶。遇兀言部族多良馬,今始來朝,所貢未備。上曰:「吾嘉爾忠順之節,慕化來歸,固不以多馬為意也。」



 七月,李繼隆出討繼遷,賜麟府州兀泥巾族大首領突厥羅、女女殺族大首領越都、女女夢勒族大首領越移、女女忙族大首領越
 置、女女籰兒族大首領党移、沒兒族大首領莫末移、路乜族大首領越移、細乜族大首領慶元、路才族大首領羅保、細母族大首領羅保保乜凡十族敕書招懷之。閏七月,懷安鎮羌誘諸族寇慶州,監軍趙繼升率師擊敗之,斬首三百級,獲羊馬千計。



 三年二月,泥巾族大首領名悉俄,首領皆移、尹遇、崔保羅、沒佶,凡五人來貢馬。名悉俄等舊皆內屬,因李繼遷之叛,徙居河北,今復來貢。



 咸平元年三月,熟倉族癿遇來朝,真宗嘉其誠節,親見
 撫勞,賜以器幣。十月,兀泥族大首領、昭州刺史黃羅對於崇德殿。兀泥族在青岡嶺、三角城、龍馬川,領族帳千五百戶,初隸繼遷,俄投府州,淳化中數敗契丹,及與繼遷相攻擊。及繼遷內附,黃羅懼,北徙過黃河。今還舊地,遂入貢,且言繼遷既受朝命,不敢侵伐。上面加獎慰,賜齎甚厚。十二月,詔直蕩族大首領鬼啜尾于金家堡置渡,令諸族互市。



 二年正月,以咩逋族開道使泥埋領費州刺史。十月,以勒浪族十六府大首領、歸德大將軍、恩
 州刺史馬泥領本州團練使。十一月,藏才八族大首領皆賞羅等來獻名馬。四年七月,以會州刺史癿遇為保順郎將,蘇家族屈尾、鼻家族都慶、白馬族埋香、韋移族都香為安化郎將。九月,環州言,繼遷所掠羌族嵬逋等徙帳來歸,又繼遷諸羌族明葉示及撲咩、訛豬等首領率屬內附,並令給善地處之。其年,卑甯族首領喝鄰半祝貢名馬,自稱有精騎三萬,願備驅策。有詔慰獎,厚償其直。



 五年,咩逋族開道使、費州刺史泥埋遣子城逋入
 貢,上嘉泥埋數與繼遷戰鬥有勞,授錦州團練使,以其族弟屈子為懷化將軍充本族指揮使,城逋為歸德將軍充本族都巡檢使,余首領署軍主以下名識者凡十數人。又以黑山北莊郎族龍移為安遠大將軍,昧克為懷化將軍。八月,河西教練使李榮等向化。其年,羌寇抄金明縣,李繼周擊走之。



 十月,詔河西戎人歸投者遷內地,給以閒田。時勒厥麻等三族千五百帳以濁輪砦失守,越河內屬,分處邊境。邊臣屢言勒厥麻往來賊中,恐
 復叛去,乃徙置憲州樓煩縣,遣使賜金帛撫慰。十二月,咩逋族遣使來貢。上聞賀蘭山有小涼、大涼族甚盛,常恐與繼遷合勢為患,近知互有疑隙,輒相攻掠,朝廷欲遂撫之。乃召問咩逋使者,因其還特詔賜之,以激其立效。上又謂樞密使王繼英等曰:「邊臣言遷賊舉兵,屢為龍移、昧克所敗。此族在黃河北數萬帳,或號莊郎昧克,常以馬附藏才入貢,頗勤外禦。」六年,遂降詔獎慰之。二月,葉市族囉埋等持繼遷偽署牒率百餘帳來歸,以囉
 埋為本族指揮使,囉胡為軍使。邠寧部署言牛羊、蘇家等族殺繼遷族帳有功,上曰:「此族恃遠與險,久為賊援,屢遣邊吏招諭,近聞有志內附,尚疑其詐,果能格鬥立效。」詔厚賜首領等茶彩以獎激之。涇原部署言,者龍移卑陵山首領廝敦琶遣使稱已集本族騎兵,願隨軍討賊。



 三月,以咩逋族首領泥埋領鄯州防禦使,充靈州河外五鎮都巡檢使。時潘羅支已授河西節制,上以泥埋實與羅支犄角捍賊,故加恩寵。是月,綏州羌部軍使拽
 臼等百九十五口內屬。原州熟戶裴天下等請率族兵掩擊遷黨移湖等帳,來求策應,部署司不報。上以戎人宣力禦賊,不應沮之,即詔諭諸路以精甲策應。環州酋長蘇尚娘擊賊有勞,及屢告賊中機事,以為臨州刺史,賜錦袍銀帶。環慶部署張凝言:「內屬戎人興賊界錯居,屢為脅誘。臣領兵離木波鎮直湊八州原下砦,招降岑移等三十二族,又至分水嶺降麻謀等二十一族,柔遠鎮降巢迷等二十族,遂抵業樂,降𡗀樹羅家等一百族,
 合四千八十戶,第給袍帶物彩,慰遣還帳。」



 四月,繼遷寇洪德砦,酋長慶香與癿𡗀慶族合勢擊之,以砦兵策援,大敗繼遷,擒四十九人,墜崖死者甚衆,獲馬七十餘匹,旗鼓鎧甲數百計。上考陣圖以問入奏使,使者言砦兵拒賊千余步,慶香等親率部族與賊接戰,上曰:「慶香等假王師為援,而交鋒俘獲,乃其功也。」悉與所獲物,加賜銀彩,以慶香領順州刺史,癿𡗀慶領羅州刺史。河西內屬折勒厥麻等三族請以精兵千人、馬三百備征討,詔
 嵐州撫諭。環州白馬族與繼遷戰鬥,屢徙帳乏食,賜稟粟。又詔洪德砦歸附戎人,給內地土田,資以口糧。



 五月,唐龍鎮上言:鎮有貿易於府州者,為州人邀殺,盡奪資畜。乃詔府州自今許令互市,切加存撫。六月,瓦窯、沒劑、如羅、昧克等族濟河擊敗繼遷黨,優詔撫問。七月,補野狸族首領子阿宜為懷安將軍。八月,原、渭等州言本界戎人來附者八部二十五族,今詣吏納質。以環州蘇尚娘子孽娘為臨州刺史。府州八族都校明義等言,屢于
 麟州屈野川擊繼遷,及緣邊六七柵防遏,皆有克獲。詔獎齎之,仍令府州常以勁兵援助,勿失機便。



 景德元年正月,麟府路言:「附契丹戎人言泥族拔黃太尉率三百餘帳內屬。拔黃本大族,居黃河北古豐州,前數犯邊,阻市馬之路。其首領容貌甚偉,有智勇,桀黠難制,契丹結之,署為太尉,今悉衆款塞。」詔府州厚賜茶彩,給公田,依險居之,計口賦粟,且戒唐龍鎮無得侵擾。三月,宋師恭破羌賊于柳穀川,驅其帳族千餘人以還。六月,洪德砦
 言羌部羅泥天王等首領率屬來附。八月,野雞族侵掠環慶界,詔邊臣和斷,如其不從,則脅以兵威。九月,鎮戎軍言,先叛去熟魏族酋長茄羅、兀贓、成王等三族應詔撫諭,各率屬來歸。



 二年,熟戶旺家族擊夏兵,擒軍主一人以獻。環州言:「戎人入寇,擊走之,擒酋將慶𡗀送闕下,請斬於槁街。」上特貰死,配淮南。原州野狸族首領廝多逋丹卒,其子阿酌代為首領,且乞奉料。詔諭以立功則賜之。



 三年,府州折惟昌言兀泥族大首領名崖從父盛
 佶,為趙德明白池軍主,密遣使諭名崖雲,德明雖外托修貢之名,而點閱兵馬尤急,必恐劫掠山界,名崖以告。上嘉之,降詔撫諭,就賜錦袍銀帶。九月,秦州言野兒和尚族部落尤大,能稟朝命,凡諸族為寇盜者輒遏絕之,請加旌別。詔補三砦都首領。十一月,鎮戎軍曹瑋言叛去酋長蘇尚娘復求歸附。詔報瑋曰:「尚娘反覆無信,特恐狙詐,以誤邊吏,又使德明緣此為詞,不可納也。」



 四年,唐龍鎮羌族來美與其叔璘不葉,召契丹破之,來依府
 州。璘、美非大族,嘗持兩端,頃亦寇鈔近界,發兵趣之,則走河之東曰東壥,契丹加兵,則入河之西曰西壥,地極險阻,介卒騎兵所不能及。至是,上亦憫其窮而款塞,特優容之。會契丹使至,即令諭其事,仍還所掠璘、美人畜。其族人懷正又與璘互相讎劫,側近帳族不寧,詔遣使召而盟之,依本俗法和斷。



 大中祥符元年,鄜延鈐轄言,小湖臥浪族軍主最處近塞,往時出師皆命為前鋒,甚著誠節。詔補侍禁。二年六月,麟府鈐轄言杜慶族依援
 唐龍鎮,數侵別帳,請發熟戶兵擊之。上曰:「戎落皆吾民也,宜以道撫之。」不許。其年,兀泥族大首領名崖同府州折惟昌入貢,上親加撫問,特詔副都知張繼能賜射于瓊林苑。四年,藏才西族、中族首領奴移、橫全等並遣子來朝。五年,環慶熟戶有酗酒劫奪使臣馬櫻者,上怒,令部署司重罰之。



 六年,北界克山軍主率衆過大里河侵熟戶,為羅勒族都囉擊走之。詔以都囉為本族指揮使,且諭邊臣約飭族帳,謹守疆界,勿出境追襲。九月,夏州
 略去熟戶旺家族首領都子等來歸,隨而至者又三族,遣使存勞之。



 七年,涇原鈐轄曹瑋請署熟戶百帳以上大首領為本族軍主,次指揮使,又次副指揮使,百帳而下為本族指揮使,從之。五月,瑋言葉市族大首領豔奴歸順。七月,瑋又言北界萬子族謀鈔略,發兵逆之,大敗於天麻川,又為魏埋等族掩擊,殺其酋帥,斬首千餘級。八年,北界酋長、指揮使浪梅娘等來投,諭邊臣令追取熟戶亡入北界者,即遣還梅娘。



 九年,羌兵寇小力族,巡
 檢李文貞率兵奮擊,追斬籍遇太保首級,賜文貞錦袍銀帶。五月,北界毛屍族軍主浪埋、骨咩族酋長癿唱、巢迷族酋長馮移埋率其屬千一百九十口、牛馬雜畜千八百歸附,降詔撫之。



 天禧元年,環州言北界騎兵數千來剽熟戶,擊走之。二年,涇原路言樊家族九門都首領客廝鐸內屬,以廝鐸為軍主。三年,鄜延路言亡去熟戶委乞等六百九十五人,及骨咩、大門等族來歸。四年正月,又言宥州羌族臘兒率衆劫熟戶咩魏族,金明都監
 李士彬擊之,斬臘兒,梟七十二級,俘餘衆,獲甲馬三百餘。五月,小湖族都虞候喏嵬、巡檢胡懷節等擊賊有功,並進秩。環州七臼族軍主近膩納質歸化,以近膩領順州刺史,首領惹都等十五人補官有差。七月,撲咩族馬訛等率屬來附。十月,以淮安鎮六族都軍主乞埋為三班借職,充羌部巡檢。五年,北界羅骨等劫剽熟戶,環慶部署田敏追擊之,俘獲甚衆,詔獎敏等,賜器幣。



\end{pinyinscope}