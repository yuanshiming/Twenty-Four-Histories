\article{列傳第二百二十二外戚上}

\begin{pinyinscope}

 杜審琦
 弟審瓊審肇審進從子彥圭彥鈞孫守元曾孫惟序賀令圖楊重進附王繼勛劉知信子承宗劉文裕劉美子從德從廣孫永年馬季
 良附郭崇仁楊景宗符惟忠柴宗慶張堯佐



 自西漢有外戚之禍,歷代鑒之,崇爵厚祿,不畀事權,然而一失其馭,猶有肺附之變焉。宋法待外戚厚,其間有文武才住,皆擢而用之;怙勢犯法,繩以重刑,亦不少貸。仁、英、哲三朝,母后臨朝聽政,而終無外家干政之患,將法度之嚴,體統之正,有以防閑其過歟?抑母後之賢,自有以制其戚里歟?作《外戚傳》。



 杜審琦,定州安喜人,昭憲皇太后之兄。太后昆仲五人,審琦最長,其次審玉,次審瓊,次審肇,次審進。世居常山,以積善聞。審琦仕後唐,為義軍指揮使,天成二年卒,年三十五,審玉前一年卒,年二十二。太祖開國,贈審琦左神武軍大將軍,以其子彥超為西京作坊使。彥超卒,贈左領軍衛大將軍。



 審瓊,建隆初,授檢校國子祭酒。二年,拜左領軍衛將軍。三年,與其弟審肇、審進皆召赴闕。審瓊改左龍武軍大
 將軍,遷右衛大將軍。乾德初,領富州刺史。三年,以本官權判右金吾街仗事。四年春,步軍帥王繼勛坐事,詔審瓊兼點檢侍衛步軍司事。是秋,卒,年七十。太祖為廢朝三日,發哀成服,贈太保、寧國軍節度使,謚恭僖。



 審瓊性醇質,在公畏慎,宿衛勤謹,徼巡京邑,里閈清肅,人皆稱之。景德三年春,加贈審瓊太傅,妻吳氏陳留郡太夫人。是秋,改葬陪陵,又贈審瓊太師、中書令。子彥圭。



 審肇,建隆三年,起家授左武衛上將軍、檢校左僕射致
 仕,賜第於京師。乾德初,領濰州剌史。開寶二年,改左衛上將軍,仍致仕。三年,起為右驍衛上將軍,俄出知澶州,太祖以審肇未嘗歷郡務,乃命司封郎中姚恕通判州事,以左右之。未幾,河大決,東匯於鄆、濮數郡,民田罹水害。太祖怒其不即時上言,遣使案鞫,遂論恕棄市,審肇免官歸私第。俄復舊官,令致仕,特以濰州刺史月奉優給之。七年,卒,年七十二。太祖廢朝二日,素服發哀,贈太保、昭信軍節度,謚溫肅,遣中使護喪事。景德三年,加贈
 太傅,妻劉氏東海郡太夫人。子彥遵,至南作坊使。



 審進,建隆三年,起家授右神武大將軍,改右羽林大將軍。乾德元年,領賀州刺史。二年,知陜州。三年,就改保義軍節度觀察留後。五年,加本軍節度。太祖郊祀西洛,審進來朝,頒賚甚厚。太宗嗣位,加檢校太傅。太平興國二年,會許昌裔刺虢州,捃拾使州闕失事上訴,詔右拾遺李乾鞫之。乾因上言,請支郡不復隸藩鎮,皆得專達,從之。



 三年秋,以審進妻卒,廢朝。十一月郊禮畢,加檢校太
 尉。四年,上親征河東,審進與嵐州團練使周承晉、德州刺史孫方進、成州刺史慕容福起皆上言願率所部擊太原。上以審進耆年,不許。五年,來朝。是歲,契丹寇邊,出師捍禦。上幸大名勞軍,留審進警巡,都邑肅然。六年,復歸陜,親王宴餞,供帳甚盛。其年,就加檢校太師。九年夏,上以審進年高,不當煩以劇務,授右衛上將軍,奉給如故。



 雍熙四年,復授靜江軍節度。端拱元年,上親耕籍田,審進預其禮,恩賜彌渥,加開府儀同三司。是歲,卒,年七
 十九。上趣駕臨喪,哭之慟,廢朝三日,設次成服,親王公主以下並詣其第舉哀。贈中書令,謚恭惠。



 審進鎮陜二十餘年,勸農敦本,民庶便之。雖居位節制,無驕矜之色,人推其醇厚。景德三年,追封京兆郡王,妻趙氏南陽郡太夫人。後贈尚書令。子彥鈞、彥彬。彥彬至禮賓副使而卒。



 彥圭,起家六宅副使,遷翰林使。開寶五年,領信州剌史。六年,改領饒州團練使,俄加領本州防禦使。從征太原,
 與曹翰、孫繼業攻城西面。北征班師,命彥圭與孟玄喆、藥可瓊、趙延進率兵屯中山,坐市竹木矯制免算,責授洛苑使、饒州剌史,裁數日,牽復。餘年,遷沙州觀察使,出知定州。



 雍熙中北伐,命副米信為幽州西北道行營都部署。彥圭不容軍士晡食,設陣不整,以致亡失,坐左遷均州團練副使。雍熙二年,卒於貶所,年五十九,贈歸義軍節度。景德三年春,加贈中書令。是秋,又贈太師。子守元。



 彥鈞,起家補供奉官,累遷崇儀使。端拱初,加莊宅使,領羅州刺史。淳化四年,特置昭宣使,以彥鈞洎王延德、王繼恩為之。未幾,加領恩州防禦使。西鄙用兵,命為永興軍駐泊鈐轄。真宗嗣位,改領潁州防禦使,出知河中府,占謝便坐,求解內使之職,可之。歷知邠、慶、延、鳳四州。景德中,為天雄軍副都部署。車駕駐澶淵,為駕前東面貝冀路副都部署。契丹騎兵攻月城,彥鈞率兵擊走之,以勞優加封邑。召還,再任河中。



 彥鈞由戚里進,保位而已。
 會有言政事不舉者,徙西京水南北都巡檢使。大中祥符五年,復知莫州。馬知節為潁州防禦使,彥鈞換秦州。九年,拜密州觀察使,出為並代副都部署。天禧元年,卒,贈安化軍節度。錄其子贊文為供奉官,贊寧為殿直,孫宗壽為三班奉職。



 守元,開寶中,補左班殿直,得侍便殿,帶御器械,遷供奉官、莫州監軍。契丹入邊,與州將固守城壁,出兵邀擊,獲生口羊馬,以功加崇儀副使。未幾,改正使秩。歷如京、洛
 苑使。至道三年,領梧州剌史,連為並代、鎮定、高陽關鈐轄。大中祥符二年,副趙稹使契丹,復涖鎮定。頃之屬疾,詔遣其子殿直惟慶挾太醫乘驛診候,既至而卒,年五十八。



 惟序字舜功,自三班奉職累遷知惠州、莫州,以供備庫使為梓夔路鈐轄,徙環慶路,知邠州,又權慶州。會任福敗,以騎兵數千由懷安路破賊三砦,斬首數百級,獲牛馬千計。以功領忠州刺史,為涇原鈐轄,敕巡警邊州。



 久
 之,改六宅使、知雄州。時契丹勒兵燕、薊間,遣使求割地。未至,而惟序購得其草,先以聞。徙知滄州,又徙定州。再遷東上閣門使、知涇州。改四方館使、知瀛州,復知滄州。入朝,為祁州團練使,出知恩州,徙大名府路總管,改乾州團練使,卒。



 賀令圖,開封陳留人。父懷浦,孝惠皇后兄也,仕軍中為散指揮使。太平興國初,出為岳州刺史,領兵屯三交。雍熙三年,從楊業北征,死於陣。



 令圖少謹願,隸太宗左右,
 洎即位,補供奉官,改綾錦副使、知莫州,遷崇儀使、知雄州。雍熙二年,領平州剌史,充幽州行營壕砦使,以所部下固安、新城兩縣,克涿州。會父戰死,起家為六宅使,領本州團練使,護瀛州屯兵。



 先是,令圖握兵邊郡十餘年,恃藩邸舊恩,每歲入奏事,多言邊塞利害,及幽薊可取之狀。上信之,故有岐溝之舉。既而師敗,議者皆咎其貪功生事。



 令圖輕而無謀,契丹將耶律遜寧號於越者,使諜紿令圖曰:「我獲罪本國,旦夕願歸南朝,無路自拔,幸
 君侯少留意焉。」令圖不虞其詐,私遺以重錦十兩。是年十二月,於越率眾入寇,大將劉廷讓與戰於君子館,令圖為先鋒,被圍數重。於越傳言軍中「願得見雄州賀使君。」令圖嘗為所紿,意其來降而終獲大功,即引麾下數十騎逆之。將至其帳數步外,於越據床罵曰:「汝常好經度邊事,乃今送死來邪!」麾左右盡殺其從騎,反縛令圖而去。



 令圖與其父首謀北伐,一歲中父子皆陷焉。令圖時年三十九。是役也,武州防禦使、高陽關部署楊重進
 死之。



 重進,太原人。少有膂力,周祖鎮大名,以隸帳下,廣順初,補衛士。宋初,累遷至內殿直都虞候。太平興國初,改龍衛軍都校,領徐州剌史。從征太原,出為萊州剌史。隨曹彬北征,為右廂排陣使,改武州防禦使、高陽關部署。會契丹兵至,與之力戰,遂沒於陣。年六十五。



 王繼勛,彰德節度饒之子,孝明皇后同母弟也。生時,其母見一人赤發,狀貌怪異,入室中,遂生繼勛。及長,美風
 儀,性兇率無賴。以後故,為內殿供奉官、都知、溪州刺史。建隆二年,加領恩州團練使,又改龍捷右廂都指揮使,尋領永州防禦使。四年,收復湖南,改領彭州防禦使。是秋,將討西蜀,命繼勛戒期,將大閱。繼勛素與大校馬仁瑀不協,陰勒部下市白梃,將以相圖。太祖知之,為出仁瑀密州。俄遷保寧軍節度觀察留後、領虎捷左右廂都虞候、權侍衛步軍司事。



 繼勛所為多不法。會新募兵千餘隸雄武,將遣出征,多無妻室,太祖謂繼勛曰:「此必有
 願為婚者,不須備聘財,但酒炙可也。」繼勛不能諭上旨,縱令掠人子女,京城為之紛擾。上聞大驚,遣捕斬百餘人,人情始定。時後已崩,上追念後,故不之罪也。



 乾德四年,繼勛復為部曲所訟,詔中書鞫之。解兵柄,為彰國軍留後,奉朝請。繼勛自以失職,常怏怏,專以臠割奴婢為樂,前後多被害。一日,天雨墻壞,群婢突出,守國門訴冤。上大駭,命中使就詰之,盡得繼勛所為不法事。詔削奪官爵,勒歸私第,仍令甲士守之。俄又配流登州,未至,改
 右監門率府副率。



 開寶三年,命分司西京。繼勛殘暴愈甚,強市民家子女備給使,小不如意,即殺食之,而棺其骨棄野外。女儈及鬻棺者出入其門不絕,洛民苦之而不敢告。太宗在藩邸,頗聞其事。及即位,人有訴者,命戶部員外郎、知雜事雷德驤乘傳往鞫之。繼勛具伏,自開寶六年四月至太平興國二年二月,手所殺婢百餘人。乃斬繼勛洛陽市,及為強市子女者女儈八人、男子三人。長壽寺僧廣惠常與繼勛同食人肉,令折其脛而斬
 之。洛民稱快。



 其後家寓西洛潁陽,孫惟德不肖,不能自立,丐食以給。真宗聞而憫之,授惟德汝州司士參軍。



 劉知信字至誠,邢州人。父遷,晉天福末鳳翔帳前軍使,改滑州奉國軍校,從驍將皇甫暉御邊有功,早卒。母即昭憲太后之妹也,乾德初,封京兆郡太君,六年,進本郡太夫人,開寶三年十月卒。太祖廢朝發哀,追封齊國太夫人,陪葬安陵,贈遷太保。



 知信三歲而孤,宣祖憐其敏慧。建隆三年,起家授供奉官,丁內艱,轉六宅副使。開寶
 五年,遷軍器庫使,掌武德司。六年,領錦州刺史。屬郊祀西洛,為行宮使,駐洛中,又為西京武德、皇城、宮苑等使。車駕出郊,又充大內留守。



 太宗即位,進領本州團練使,拜武德使。從征河東,又為行宮使。太平興國五年,坐遣親信市竹木於秦、隴,矯制免所過算緡,入官多取其直,左授軍器庫使,領錦州刺史,俄復為武德使。會改武德為皇城司,即為皇城使。七年,坐秦王廷美事,改右衛將軍。是秋,出為靜難軍節度行軍司馬。九年,起為左衛將
 軍,領營州刺史。



 雍熙初,改左神武軍將軍,尋領檀州團練使,護屯兵於鎮州。會大舉北伐,與六宅使符昭壽為押陣都監。師還,諸將失道,知信獨整所部以歸。俄知定州兼兵馬鈐轄,押大陣右偏。一日,宴犒將士,契丹騎乘間至,知信不介而出,追之數十里,斬獲甚眾,以功就拜邕州觀察使。四年,召入,改並州路副都部署。端拱中,代還,知杭州。淳化四年,又知天雄軍府。太宗崩,充修奉永熙陵部署。咸平初,拜建武軍節度觀察留後,知永平軍
 府。契丹犯邊,復知天雄軍。真宗北巡,充駕前副都部署,歷知河陽、升州。景德元年,車駕幸澶淵,命為東京都巡檢使,復知定州。二年,以疾求還京,至鎮州卒,年六十三。廢朝,贈太尉、天平軍節度。



 知信以戚里致貴,尤被親任,中外踐歷,最為舊故。雖無顯赫稱,亦以循謹聞於時。子承宗、承渥。



 承宗,幼善射,兼習書數,以蔭補殿直,寄班祗候。咸平初,轉供奉官、鎮、定、高陽關三路承受公事,還,掌軍器庫。會
 真宗臨幸,見其整肅,面授閣門祗候。知信卒,轉內殿崇班。未幾,為河北緣邊安撫都監。大中祥符初,就加內殿承制,歷如京、文思二副使,徙河東緣邊安撫,又知保州。俄拜東染院使、知定州。副薛映使契丹,使還,歸本任,又兼鎮定路兵馬鈐轄,俄改宮苑使、知雄州、河北緣邊安撫使。在郡有治跡,詔書嘉獎,召歸,時靈昌決河初塞,擇守臣,以承宗為皇城使、知滑州。未幾,復代還。



 會西邊言吐蕃唃斯囉作文法,頗為邊患。命副龍圖閣直學士陳
 堯咨為鄜延、邠寧環慶、涇原儀渭、秦州路巡撫使,詔令堯咨等所至軍州犒官吏將校,諮訪民間利害、郡官使臣能否功過以聞。或有陳訴屈抑,經轉運、提點司區斷不當,即按鞫詣實,杖以下依法區理,徒以上驛聞,仍取系囚躬親錄問,催促論決。既行,就命堯咨知秦州,承宗為西上閣門使,充鈐轄。乾興初,進東上閣門使,徒鄜延都鈐轄而卒。中使護柩至京師,賜以葬地。



 承渥蔭補殿直,累任使,喜為條奏,至供奉官、閣門祗候。承宗子永釗,
 右侍禁、閣門祗候。



 劉文裕,字以寧,保州保塞人。祖正,晉幽州營田使兼平州刺史。父審奇,武牢關使。簡穆皇后即文裕祖姑也。審奇三子,長文遠,建隆中為供奉官,與並人戰萬善而沒。次即文裕,開寶四年,起家補殿直。八年,權管雲騎員僚直,預討江南,中弩矢,神色自若。太宗在藩邸,多得親接。太平興國二年,擢為內弓箭庫副使,特封其母張氏清河縣太君,出為秦、隴巡檢。



 有李飛雄者,太保致仕鏻之
 孫,秦州節度判官若愚之子。性兇險,不為其家所容,常往來京師、魏博間,與無賴惡少游處,縱酒蒲博為務。以其父故,盡知秦州倉庫所積,及地形險易、兵籍多少。又有妻父張季英為鳳翔盩厔尉,飛雄自京師往省之,因乘季英馬詐為使者,夜抵廄置呼卒索馬。卒秉炬出迎,飛雄以私市馬纓示之,卒不能辨,即授以馬。一卒乘一馬前導,以巡邊為名,因矯詔率巡驛殿直姚承遂,至隴州率監軍供奉官王守定,至吳山縣率縣尉盧贊,皆從
 行。先是,秦州內屬,羌人為寇,朝廷遣周承瑨、田仁朗、王侁、梁崇贊、韋韜、馬知節及文裕領兵屯清水縣,飛雄至,稱制盡縛之。承瑨等見姚承遂數輩同至,不覺其詐。仁朗獨號泣求詔書,飛雄叱之曰:「我受密旨,以若輩逗撓不用命,令盡誅。汝豈不聞封州殺李鶴邪?詔書汝豈得見!」先是,上即位,分命親信於諸道廉官吏善惡密以聞。嶺南使者言封州李鶴不奉法,誣奏軍吏謀反,詔即誅之。故飛雄引以為言。將械承瑨等詣秦州戮之,因據城
 叛,遂驅承瑨等行。



 初,飛雄詐宣制時,自言我上南府時親吏,文裕因哀告飛雄曰:「我亦嘗依晉邸,使者豈不營救之乎?」飛雄低語謂文裕曰:「爾能與我同富貴否?」文裕覺其詐,偽許之。飛雄即命左右釋文裕縛。文裕策馬前附耳語仁朗,仁朗佯墜馬,若卒中風眩狀。飛雄共前視之,又釋其縛。仁朗奮起搏飛雄,與文裕共擒之。飛雄尚呼云:「田仁郎等謀反殺使者。」送秦州獄鞫得實,飛雄、承遂、守定、贊坐要斬,夷飛雄家。捕先與飛雄善者何大舉
 等數輩,悉棄市,廄置卒亦夷其族。因下詔:中外臣庶之家,子弟或有乖檢,甚為鄉黨所知,雖加戒勖曾不悛改者,並許本家尊長具名聞,州縣遣吏錮送闕下,當配隸諸處。敢有藏匿不以名聞者,異時醜狀彰露,期功以上悉以其罪罪之。



 文裕後遷軍器庫使。四年,車駕征太原,命文裕與通事舍人王侁分兵控石嶺關。六年,領儒州刺史。明年,為高陽關都監。會契丹萬餘騎入,文裕與大將崔彥進擊卻之。雍熙初,徙屯三交,加領順州團練使。
 會李繼遷率折遇乜寇邊,初詔田仁朗與王侁等討之,仁朗坐逗遛,命文裕代仁朗。繼遷等遁去。



 從潘美北征,坐陷失驍將楊業,削籍,配隸登州,事具《業傳》。歲餘,上知業之陷由王侁,召文裕還。俄起為右領軍衛大將軍,領端州團練使,封其母清河郡太夫人,賜翠冠霞帔,授其弟文質殿直。逾月,文裕遷容州觀察使,出為鎮州兵馬部署。端拱元年,卒於屯所,年四十五。上甚悼惜,贈寧遠軍節度,命中使護喪歸葬京師。弟文暠至供奉官、閣門
 祗候,文質至內園使、連州刺史。



 劉美字世濟,並州人。四世祖質,絳州刺史。曾祖維嶽,不仕。祖延慶,右驍衛將軍。父通,宋初掌禁旅,從潘美徵廣南,又累戰北面,積勞至虎捷都指揮使,領嘉州刺史,太平興國中,扈蹕太原,卒於師,贈潁州防禦使。長女為真宗德妃,加贈定國軍節度兼侍中。大中祥符五年,德妃正位中宮,又贈維嶽忠正軍節度、檢校太傅,延慶彰德軍節度、檢校太尉,通永興軍節度兼中書令,追封曾祖
 母宋氏吳國太夫人,祖母河南縣君元氏許國太夫人,母龐氏徐國太夫人。初,通之卒,窆京城西。天禧二年,詔贈太師、尚書令,謚武懿,七月,遣升王府諮議參軍張士遜具鹵簿鼓吹,改葬於祥符鄧公原。皇后親臨奠,真宗御制祭文置靈坐右。



 美即後之兄也。初事真宗於藩邸,以謹力被親信,即位,補三班奉職,再遷右侍禁。咸平中,傅潛失律流房州,擇美監軍,及徙潛潁州,又為自京至陳、潁巡檢。石保吉在陳州大治廨舍,修城壁,不以聞,僮
 奴輩假威擾民。會有言者,遣美廉其狀,美曰:「保吉世受國恩,擁高貲,列藩閫,營繕過度,拙於檢下,誠或有之,自餘保無他患。」上意乃解。歸朝,充閣門祗候。



 大中祥符二年,護屯兵於漢州,歷遷供奉官,徒嘉州。士卒有病皆給醫藥,親察視撫循之。召還,改內殿崇班,提點在京倉場、東西八作司,以舉職聞,遷洛苑副使。八年,預修大內,以勞改南作坊使、同勾當皇城司。天禧初,遷洛苑使,領勤州刺史,與周懷政聯職。懷政奸恣,美未嘗阿附,懷政左
 右有過,必痛繩之。親從卒偵邏者多不時更易,美按籍分番次均使焉。上屢欲委之兵柄,以皇后懇讓故,中輟者數四。三年,授龍神衛四廂都指揮使,領昭州防禦使,改侍衛馬軍都虞候。五年,加武勝軍節度觀察留後。卒,年六十。廢朝三日,贈太尉、昭德軍節度,錄其子從德供備庫使,從廣內殿崇班,旁親遷補者數人,追封美亡妻宋氏河內郡夫人。



 仁宗嗣位,尊皇后為皇太后,贈維嶽鎮寧軍節度兼侍中,延慶建雄軍節度兼中書令,通彭
 城郡王,曾祖母宋氏陳國太夫人,祖母元氏衛國太夫人。母龐氏鄆國太夫人,美亦贈侍中。天聖二年,郊祀,加贈維嶽彰信軍節度兼中書令,延慶鎮安軍節度兼中書令,通鄭王,宋氏楚國太夫人,元氏韓國太夫人,龐氏魏國太夫人。五年,再郊,又贈維嶽天平軍節度、中書令兼尚書令,延慶彰化軍節度、許國公,通開府儀同三司、魏王,宋氏安國太夫人,元氏齊國太夫人,龐氏晉國太夫人,
 從德和州刺史,從廣內殿承制。有龔知進者,即通之友婿也,亦贈衛尉卿,其妻追封南安郡君。



 從德字復本,父美卒,年十四,自殿直遷至供備庫副使。弟從廣是歲始生,亦補西頭供奉官,遷內殿崇班。太后臨朝,從德以崇儀使真拜恩州刺史,改和州,又遷蔡州團練使,出知衛州,改恩州兵馬都總管,知相州。從德齒少無才能,特以外家故,恩寵無比。其在衛州,縣吏李熙輔者善事從德,乃薦其才於朝。太后喜曰:「兒能薦士,知所以為政矣。」即
 擢熙輔京官。從事鄭驤因緣從德,亦擢美官。從德妻,嘉州王蒙正女也。蒙正家豪右,以厚賂結納至郎官,為郡守。既而從德病,召還,道卒,年二十四。贈保寧軍節度使,封榮國公,謚康懷。太后悲憐之尤甚,錄內外姻戚門人及僮隸數十人。從德娣婿龍圖閣直學士馬季良、母越國夫人錢氏兄惟演子集賢校理曖及蒙正皆遷二官。尚書屯田員外郎戴融嘗佐從德衛州,以為三司度支判官。御史曹修古、楊偕、郭勸、推直官段少連上疏論之,
 皆坐貶。子永年。



 從廣字景元,少出入禁中,侍仁宗左右,太后愛之如家人子。太后崩,真拜崇州團練使。娶荊王元儼女。為滁州防禦使,時年十七。趙元昊反,從廣自言待罪行間,不能捍患疆埸,坐耗縣官,願上所給公使錢,帝嘉納之。為群牧都監,改副使。



 從廣自為防禦使十年不遷,特拜宣州觀察使、同勾當三班院,請補外自效,以知洺州。漳水溢,從廣穿隋故渠以殺水勢,洺人便之。徙邢州,籍鄉軍之
 罷老者聽引子弟自代,著為令。召還,復領三班院。出知襄州,徙真定府路馬步軍副都總管。卒,贈昭慶軍節度使,謚良惠。從廣性謹飭,然喜交士大夫,時頗稱之。



 永年字君錫,生四歲,授內殿崇班,許出入兩宮。仁宗使賦《小山詩》,有「一柱擎天」之語。帝誤投金杯瑤津亭下,戲謂左右曰:「能取之乎?」永年一躍持之而出,帝拊其首曰:「奇童子也。」常置內中,年十二,始聽出外,累遷廉州團練使,為陜州都監。郭邈山等為盜,永年密遣壯士夜渡河,
 殺其兇桀二十餘人,眾遂散。遷鈐轄,代還召見,問破賊狀,擢幹辦皇城司,改單州團練使、永興軍路總管。



 契丹遣使來請帝繪像,選副張昪報使。契丹以未得志,夜取巨石塞驛門,眾皆恐,永年素有力,手擲棄之,契丹驚以為神。



 出知涇州,帝賜詩寵之。郡兵歲以香藥為折支,三司不時輦致。振武卒素驕,突入通判聽事,請以他物代給,歡嘩語不遜。永年召至庭下數其罪,斬為首二人,餘不敢動。同提舉在京諸司庫務。凡三除防禦使,皆為言
 者所論而寢。



 知代州。契丹取西山木積十餘里,輦載相屬於路,前守不敢遏,永年遣人焚之,一夕盡。上其事,帝稱善。契丹移檄捕縱火盜,永年曰:「盜固有罪,然發在我境,何預汝事?」乃不敢復言。帝嘗問御戎策,對合旨,書「忠孝」字以賜。



 英宗立,遷沂州防禦使,復知代州。歷步軍馬軍殿前都虞候、太原定州路副都總管。王師征安南,永年請先士卒,度富良江取賊以獻,不許。遷邕州觀察使、步軍副都指揮使。卒,贈崇信軍節度使,謚曰莊恪。



 馬季良字元之,開封府尉氏人。家本茶商,娶劉美女。初補越州上虞尉,改秘書省校書郎,知明州鄞縣,入為刑部詳覆官。太后臨朝,遷光祿寺丞。頃之,擢秘閣校理、同判太常禮院,再遷太子中允、判三司度支勾院,以太常丞、直史館提舉在京諸司庫務,擢龍圖閣待制。三丞充近職,非故事也。遷尚書工部員外郎、龍圖閣直學士、同知審官院。劉從德卒,遺表季良遷二官,辭不就,而請以其子直方為館閣讀書。



 會江南旱,出為安撫使,再遷兵
 部郎中。太后崩,換濠州防禦使,赴本州。御史中丞範諷言季良徼幸得官,降屯衛將軍、滁州安置。開封府劾奏季良冒立券,庇占富民劉守謙免戶役,詔許季良自陳,以地給還。歲餘,徙壽州,致仕,還京師卒。



 季良因緣以進,無他行能,在禮院嘗建言,攝祠事官致齋三日無供帳飲食,非所以重祠事也。自是翰林、儀鸞司供帳,大官給食於祠所云。



 郭崇仁,字永年,守文之子,章穆皇后弟也,淳化四年,補
 左班殿直,遷東頭供奉官、閣門祗候。契丹入寇,齎密詔諭河北諸將,還奏稱旨,累遷崇儀副使兼閣門通事舍人。章穆崩,特除莊宅使、康州刺史,再遷宮苑使、昭州團練使。丁母憂,起復雲麾將軍,拜解州團練使,改蔡州,擢捧日天武四廂都指揮使、賀州防禦使、高陽關路馬步軍副都總管。以疾落軍職,改磁州防禦使。卒,贈彰德軍節度觀察留後。



 崇仁雖外戚,朝廷未嘗過推恩澤,其為解州團練使十年不遷,嘗除知相、衛二州,皆辭不行,蓋
 性慎靜,不樂外官也。



 楊景宗,字正臣,章惠皇太后從父弟,少蒲博無賴,客京師,以罪黜隸致遠務。章惠入宮為美人,奏補茶酒班殿侍,累遷西頭供奉官、閣門祗候,坐事降左侍禁、郢州兵馬都監。未久復官,累遷東染院副使。章惠為太后,進崇儀使,領連州刺史、揚州兵馬鈐轄。未幾,授秦州刺史,徙滑州鈐轄,遷舒州團練使,為兵馬總管。



 章惠崩,遷成州防禦使,坐入臨皇儀殿被酒歡噪,出為兗州總管,改天
 雄軍副都總管。時呂夷簡守魏,常以官屬禮飭戒之,而景宗肆志不悛,遂以不法奏。貶齊州都監,徒衛州,又徒鄆州鈐轄。召還,同勾當景靈宮、提舉四園苑。章獻、章懿二後升祔太廟,帝念章惠,故特拜景宗徐州觀察使,給留後奉。逾年,領軍頭引見司,出知磁州,為建寧軍節度觀察留後、知潞州,給節度使奉。領皇城司,坐衛士入禁中謀為亂,貶徐州觀察使、知濟州。還,提舉萬壽觀,復建寧軍留後,復領軍頭引見。又坐從卒王安挾刃入皇城,
 謫左監門衛大將軍、均州安置,起為汝州鈐轄。祀明堂覃恩,願還所改官,求為郡。帝謂輔臣曰:「景宗性貪虐,老而益甚,郡不可予也。」乃復以為建寧軍留後、提舉四園苑,改提舉在京諸司庫務。卒,贈安武軍節度使兼太尉,謚莊定。



 景宗起徒中,以外戚故至顯官,然暴戾,所至為人患。復使酒任氣,在滑州嘗毆通判王述僕地。帝深戒毋飲酒,景宗雖書其戒坐右,頃之輒復醉。其奉賜亦隨費無餘。始,宰相丁謂方盛,築第敦教坊,景宗為役卒負
 土第中,後謂敗,仁宗以其第賜景宗,居三十年乃終。



 符惟忠,字正臣,彥卿曾孫也。以外祖母賢靖大長公主蔭,為三班奉職,後擢閣門通事舍人、勾當東排岸司。三司使寇瑊繩下急,漕米數不足綱,吏卒率論以自盜。惟忠爭曰:「在法,欠不滿四百者不坐,若以自盜論,則計直八百即當坐徒矣。」瑊怒曰:「敢抗三司使邪?」惟忠曰:「職有當辨,非抗也。」瑊益怒,惟忠爭愈力,如所議乃已。



 以西染院副使權提舉倉草場、提點開封府界縣鎮公事。開封
 主簿樂誥,宰相王曾外孫也。或風使薦之。惟忠不從,曰:「誥無善狀,安可以勢使我。」既而誥果以贓敗。時吳奎為長垣尉,惟忠厚遇奎,白府共薦之。



 惠民河與刁河合流,歲多決溢,害民田,惟忠自宋樓鎮碾灣、橫隴村置二斗門殺水勢,以接鄭河、圭河,自是無復有水害。陜西用兵,除涇原路兵馬鈐轄兼知涇州。三司使鄭戩奏留都大管勾汴河使,建議以為渠有廣狹,若水闊而行緩,則沙伏而不利於舟,請即其廣處束以木岸。三司以為不便,
 後卒用其議。再遷西上閣門副使。契丹遣使求地,惟忠副富弼往報使,遷閣門使,至武強縣,疽發背卒。贈客省使、眉州防禦使。



 柴宗慶,字天祐,大名人。祖禹錫,鎮寧軍節度使。父宗亮,太子中舍。宗慶尚太宗女魯國長公主,升其行為禹錫子,拜左衛將軍,駙馬都尉,領恩州刺史。禹錫卒,真拜康州防禦使,改復州。



 舊制,諸公主宅皆雜買務市物,宗慶遣家僮自外州市炭,所過免算,至則盡鬻之,復市於務
 中。自是詔雜買務罷公主宅所市物。從祀汾陰,為行宮四面都巡檢,進泉州管內觀察使。又自言陜西市材木至京師,求蠲所過稅。真宗曰:「向諭汝毋私販以奪民利,今復爾邪!」既而河東提點刑獄劾宗慶私使人市馬不輸稅,貸不問。授武勝軍節度觀察留後,歷拜彰德軍節度使。



 仁宗即位,徙靜難軍,又徙永清、彰德軍,拜同中書門下平章事,徙節武成軍,出知澶州,未行,改陜州、潞州。後判鄭州,以縱部曲擾民,召還奉朝請,歲減公用錢四
 百萬。久之,出判濟州,用御史中丞賈昌朝言,留不遣,盡停本使公使錢。卒,贈中書令,謚曰榮密。主累封楚國大長公主,先宗慶沒。



 宗慶歷官多過失,性極貪鄙,積財鉅萬,而薄於自奉,甚至優人以為戲,宗慶雖知,莫能改也。無子。及終,願以貲產送官,仁宗以其女尚幼,不許。人謂宗慶選尚榮貴逾四十年,晚上積奉以裨軍用,蓋亦追補前過雲。



 張堯佐,字希元,河南永安人,溫成皇後世父也。舉進士,
 歷憲州、筠州推官。吉州有道士與商人夜飲,商人暴死,道士懼而遁,為邏者所獲,捕系百餘人。轉運使命堯佐覆治,盡得其冤。改大理寺丞、知汜水縣,遷殿中丞、知犀浦縣。犀浦地狹民繁,多田訟。堯佐正其疆界,條眾敝以曉之,訟遂簡。知開州,還,判登聞鼓院。



 時溫成方為脩媛,欲以門閥自表異,故堯佐稍進用,權開封府推官,又提點府界公事。諫官餘靖言:「用堯佐不宜太遽,頃者郭后之禍起於楊尚,不可不監。」未幾,遷三司戶部判官,又為
 副使。擢天章閣待制、吏部流內銓,累遷兵部郎中、權知開封府,加龍圖閣直學士,遷給事中、端明殿學士,拜三司使。



 明年,諫官包拯、陳升之、吳奎言:「比年以來,水冒城郭,地震河溢,蓋小人道盛。天下皆謂堯佐主大計,諸路困於誅求,內帑煩於借助,法制剚敞,實自堯佐。臣等竊惟親暱之私,聖人不免,惟處之有道,使不踐危機,斯為得矣。」仁宗祀明堂,改戶部侍郎,尋拜淮康軍節度使、群牧制置使、宣徽南院使、景靈宮使,賜二子進士出身。拯
 等復言:「陛下即位僅三十年,未有失道敗德之事,乃五六年來擢用堯佐,群口竊議,以謂其過不在陛下,在女謁、近習與執政大臣也。蓋女謁、近習知陛下繼嗣未立,既有所私,莫不潛有趨向;執政大臣不能規諫,乃從諛順旨,高官要職惟恐堯佐不滿其意,致陷陛下於私暱後宮之過。制下之日,陽精晦塞,氛霧蒙孛,宜斷以大義,亟命追寢。必不得已,宣徽、節度擇與一焉。如此,則合天意,順人情矣。」御史中丞王舉正留百官班,欲廷議,不許。
 乃詔曰:「近臺諫官乞罷堯佐三司,及言不可用為執政,若優與之官,於體為善,朕用其言,遂有是命。今復以為不可,前後反覆,於法當黜。其令中書戒諭之。自今言事官,相率上殿,先取旨。」是日,堯佐辭宣徽、景靈使,從之。



 未幾,復以宣徽使判河陽,舉正又抗章論之,至於三。時吳育判西京留臺,河陽民訟有不決者多詣育,育於狀尾判曲直。堯佐畏恐,即奉行之。召還,徒鎮天平軍。卒,贈太師,賜其家僦舍錢日三千。



 堯佐起寒士,持身謹畏,頗通
 吏治,曉法律,以戚里進,遽至崇顯,戀嫪恩寵,為世所鄙。子山甫,引進副使、樞密副都承旨。



 從弟堯封,孝謹好學,舉進士,為石州推官卒。次女,即溫成皇后也。累贈至中書令、清河郡王,謚曰景思。



\end{pinyinscope}