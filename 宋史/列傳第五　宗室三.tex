\article{列傳第五 宗室三}

\begin{pinyinscope}

 吳王顥益王頵吳王佖
 燕王俁楚王似獻愍太子茂鄆王楷肅王樞景王杞濟王栩徐王棣沂王□咢和王栻信王榛太子諶弟訓元懿太子信王璩莊文太子□魏王愷景獻太子詢鎮王竑



 英宗四子:長神宗,次吳榮王顥,次潤王顏,次益端獻王頵,皆宣仁聖烈高皇后出也。顏早亡,徽宗賜名追封。



 吳榮王顥字仲明,初名仲糾,自右內率府副率為和州防禦使,封安樂郡公,轉明州觀察使,進祁
 國公。治平元年,加檢校太
 傅、保寧軍節度使、同中書門下平章事,封東陽郡王。三年,出閣。神宗立,進封昌王;
 官
 制行,冊拜司空,徙王雍。哲宗嗣位,加太保,換成德、橫海二鎮,徙封揚王,賜贊拜不名,五日一謁禁中。帝致恭如家人禮。神宗祔廟,拜太傅,移鎮京兆、鳳翔。



 自熙寧以來,顥屢請居外,章上輒卻。至元祐初,乃賜咸宜坊第一區,榜曰「親賢」與弟頵對邸。車駕偕三宮臨幸,留宴終日。拜太尉,諸子皆命賜官,制曰:「先皇帝篤兄弟之好,以恩勝義,不許二叔
 出居於外,蓋武王待周公之意。太皇太后嚴朝廷之禮,以義制恩,始從其請,出就外宅,得孔子遠其子之意。二聖不同,同歸於道,皆可以為萬世法。朕承侍兩宮,按行新第,顧瞻懷思,潸然出涕。昔漢明帝問東平王:『在家何以為樂?』王言:『為善最樂。』帝大其言,因送列侯印十九枚,諸子五歲以上悉佩之,著之簡策,天下不以為私。今王諸子性於忠孝,漸於禮義,自勝衣以上,頎然皆有成人之風,朕甚嘉之。其各進一官,以助其為善之樂,尚勉之
 哉!毋忝父祖,以為邦家光。」徙封徐王,詔書不名。



 宣仁有疾,顥旦旦入問,因亦被病。宣仁祔廟,拜太師,徙王冀,賜入朝不趨。改淮南、荊南節度使,徙封楚王。病益篤,帝親挾醫視診,令晝夜具起居狀聞,小愈則喜。既而薨,年四十七。帝即臨哭,輟朝五日,成服苑中。贈尚書令兼中書令、揚荊冀三州牧、燕王,謚曰榮,陪葬永厚陵。徽宗即位,改封吳王。



 顥天資穎異,尤嗜學,始就外傅,每一經終,即遺講讀官以器幣服馬。工飛白,善射,好圖書,博求善本。
 神宗嘉其志尚,每得異書,亟馳使以示。嘗賜方團玉帶,俾服而朝,顥辭,乃為制玉魚以別之。是後親王遂踵為故實。初,居英宗喪,丐解官終制,以厭於至尊,不克遂。服慈聖光獻太后之服,易月當除,顥曰:「身為孫而情文缺然,若是可乎?請如心喪禮,須上禫除,即吉。」詔可。



 子孝騫嗣,終寧國軍節度使、晉康郡王;孝錫終嘉州團練使,贈永國公。



 益端獻王頵,初名仲恪,封大寧郡公,進鄮國公、樂安郡
 王、嘉王。所歷官賜,略與兄顥同。更武勝、山南西、保信、保靜、武昌、武安、武寧、鎮海、成德、荊南十節度,徙王曹、荊,位至太尉。元祐三年七月薨,年三十三,贈太師、尚書令、荊徐二州牧、魏王,謚端獻。徽宗改封益王。



 頵端重明粹,少好學,長博通群書,工飛白、篆籀。賓接宮僚,歲滿當去,輒奏留,久者至十餘年。頗好醫書,手著《普惠集效方》,且儲藥以救病者。



 子九人:孝哲,右驍衛將軍,早亡;孝奕,彰化軍節度觀察留後,贈司空、平原郡王;孝參,奉國軍節度
 使,改寧武、武勝,封豫章郡王;孝永,邢州觀察使,贈司空、廣陵郡王;孝詒、孝騭、孝悅、孝穎、孝願,皆至節度使。



 神宗十四子:長成王佾,次惠王僅,次唐哀獻王俊,次褒王伸,次冀王僩,次哲宗,次豫悼惠王價,次徐沖惠王倜,次吳榮穆王佖,次儀王偉,次徽宗,次燕王俁,次楚榮憲王似,次越王偲。八王皆早薨:佾、僅、伸、偉,徽宗賜名追封;俊、僩、倜、價,徽宗改封。



 吳榮穆王佖,帝第九子。初授山南東道節度使,封儀國
 公。哲宗立,加開府儀同三司、大寧郡王,進申王,拜司空。帝崩,佖於諸弟為最長,有目疾不得立。徽宗嗣位,以帝兄拜太傅,加殊禮,旋拜太師,歷京兆、真定尹,荊、揚、太原、興元牧,徙國陳。崇寧五年薨,輟視朝七日。贈尚書令兼中書令、徐州牧、燕王,謚榮穆。又加贈侍中,改封吳王。子有奕,武信軍節度使、和義郡王。



 燕王俁,帝第十子;越王偲,帝第十二子。母曰林婕妤。俁初授定武軍節度使、檢校太尉,封成國公;偲初授武成
 軍節度使、檢校太尉、祁國公。哲宗朝,俁加開府儀同三司,封咸寧郡王;偲加開府儀同三司,封永寧郡王。是後累換節鋮,歷任尹牧,俁進封莘王,偲封睦王。徽宗朝,俱歷太保、太傅,俁進封衛王、魏王、燕王,偲進封定王、鄧王、越王。靖康元年,同遷太師,俁授河東劍南西川節度使、成都牧,偲授永興成德軍節度使、雍州真定牧。



 二年,上皇幸青城,父老邀之不及,道遇二王,哭曰:「願與王俱死。」徐秉哲捕為首者戮之,益兵衛送二王於金營,北行至
 慶源境上,俁乏食薨,偲至韓州而薨。



 紹興初,有崔紹祖者至壽春府,稱越王次子,受上皇蠟詔為天下兵馬大元帥,興師恢復。鎮撫使趙霖以聞。召赴行在,事敗,送臺獄伏罪,斬於越州市。



 楚榮憲王似,帝第十三子。初為集慶軍節度使、和國公,進普寧郡王。元符元年出閣,封簡王。似於哲宗為母弟,哲宗崩,皇太后議所立,宰相章惇以似對。後曰:「均是神宗子,何必然。」乃立端王。徽宗定位,加司徒,改鎮武昌、武
 成,徙封蔡,拜太保,移鎮保平、鎮安,又改鳳翔、雄武。以王府史語言指斥,送大理寺驗治,似上表待罪。



 左司江諫江公望上疏,以為:「親隙不可開,開則言可離貳;疑跡不可顯,顯則事難磨滅。陛下之得天下也,章惇嘗持異議,已有隙跡矣。蔡王出於無心,年尚幼小,未達禍亂之萌,恬不以為恤。陛下一切包容,已開之隙復塗,已顯之跡復泯矣。恩意渥縟,歡然不失兄弟之情。若以曖昧無根之語,加諸至親骨肉之間,則有魏文『相煎太急』之譏,而忘
 大舜親愛之道,豈治世之美事邪。臣願陛下密詔有司,凡無根之言勿形案牘,倘有瑕可指,一人胸次,則終身不忘,跡不可泯,隙不可塗,則骨肉離矣。一有浸淫旁及蔡王之語,不識陛下將何以處之,陛下何顏見神考於太廟乎?」疏入,公望罷知淮陽軍。徽宗雖出公望,然頗思其言,止治其左右。



 崇寧中,徙鎮荊南、武寧。崇寧五年薨,贈太師、尚書令兼中書令、冀州牧、韓王,改封楚王,謚榮憲。



 子有恭,定國軍節度使、永寧郡王。



 哲宗一子:獻愍太子茂,昭懷劉皇后為賢妃時所生。帝未有子,而中宮虛位,後因是得立。然才三月而夭,追封越王,謚沖獻。崇寧元年,改謚獻愍。後之立也,鄒浩凡三上疏諫,隨削其稿。至是,或謂浩有「殺卓氏而奪其子,欺人可也,詎可以欺天乎」之語,徽宗昭暴其事,復竄浩昭州,而峻茂典冊。後上表謝,然浩蓋無是言也。



 徽宗三十一子:長欽宗,次袞王檉,次鄆王楷,次荊王楫,次肅王樞,次景王杞,次濟王栩,次益王棫,次高宗,次邠
 王材,次祁王模,次莘王植,次儀王樸,次徐王棣,次沂王□咢,次鄆王栱,次和王栻,次信王榛,次漢王椿,次安康郡王屋,次廣平郡王楗,次陳國公機,次相國公梃,次瀛國公樾,次建安郡王柍,次嘉國公椅,次溫國公棟,次英國公楒,次儀國公桐,次昌國公柄,次潤國公樅。檉、楫、材、栱、椿、機六王早薨。



 鄆王楷,帝第三子。初名煥。始封魏國公,進高密郡王、嘉王,歷奉寧、鎮安、鎮東、武寧、保平、荊南、寧江、劍南西川、鎮
 南、河東、寧海十一節度使。政和八年,廷策進士,唱名第一。母王妃方有寵,遂超拜太傅,改王鄆,仍提舉皇城司。出入禁省,不復限朝暮,於外第作飛橋復道以通往來。北伐之役,且將以為元帥,會白溝失利而止。欽宗立,改鎮鳳翔、彰德軍。靖康初,與諸王皆北遷。



 肅王樞,帝第五子。初封吳國公,進建安郡王、肅王,歷節度六鎮。靖康初,金人圍京城,要帝子弟為質,且求輸兩河。於是遣宰臣張邦昌從樞使斡離不軍,為金人所留,
 約俟割地畢遣還,而挾以北去。



 景王杞,初授武安軍節度使、檢校太尉,封冀國公。大觀二年,改授山南東道節度使,加開府儀同三司,封文安郡王。政和中,授檢校太保,尋遷太保,改授護國、武昌軍節度使,追封景王。靖康元年,授荊南、鎮東軍節度使,遷太傅。



 二年,遣詣金營充賀正旦使。既歸,又從上幸青城。及上皇出郊,杞日侍左右,衣不解帶,食不食肉,上皇制發願文,述祈天請命之意,以授杞。杞頓首泣。及北行,須
 發盡白。



 濟王栩,初授鎮洮軍節度使、檢校太尉,封魯國公。大觀二年,改授彰武軍節度使,加開府儀同三司,封安康郡王,政和中,授檢校太保,改荊南、清海軍節度使,進封濟王。靖康元年,授護國、寧海軍節度使,遷太傅。



 同景王杞為賀金人正旦使。既還,又與何桌為請命使,金帥紿栩曰:「自古有南即有北,不可相無,今所欲割地而已。」栩回以白上,且言金帥請與上皇相見,上曰:「豈可使上皇蒙
 塵。」遂自出,以栩從行。及索諸王家屬,栩夫人曹氏避難他出,徐秉哲捕而拘之,遂同北去。



 徐王棣,初授鎮江軍節度使、檢校太尉,封徐國公。政和中,授檢校太保。宣和中,改鎮南軍節度使,加開府儀同三司,封高平郡王。尋改山南東道、河陽三城節度使,進封徐王。後從淵聖北去。



 紹興二年,有萬州李勃者,偽稱祁王,內侍楊公謹與言徐王起居狀,勃遂改稱徐王。宣撫使張浚遣赴行在,上命王府故吏驗視,言非真,詔送
 大理,情得,棄市。



 沂王□咢,初授橫海軍節度使、檢校太尉、冀國公。政和中,授檢校太保。宣和中,改劍南西川節度使,加開府儀同三司,封河間郡王。尋改劍南東川、威武軍節度使,遷太保,進封沂王。



 後從淵聖出郊,至北方,與駙馬劉彥文告上皇左右謀變,金遣人按問,上皇遣莘王植、駙馬蔡鞗等對辨,凡三日,□咢、彥文氣折,金人誅之。



 和王栻,初授靜江軍節度使、檢校太尉、廣國公。三年,授
 檢校太保。尋改定武軍節度使,加開府儀同三司,封南康郡王。靖康元年,授瀛海、安化軍節度使、檢校太傅,追封和王。後從淵聖出郊。



 有遺女一人,高宗朝封樂平縣主,出適杜安石,命大宗正司主婚。



 信王榛,初授建雄軍節度使、檢校太尉,封福國公。三年,授檢校太保。宣和末,改安遠軍節度使,加開府儀同三司,封平陽郡王。靖康元年,授慶陽、昭化軍節度使,遷檢校太傅,進封信王。



 後從淵聖出郊,北行至慶源,亡匿真
 定境中。時馬廣與趙邦傑聚兵保五馬山砦,陰迎榛歸,奉以為主,兩河遺民聞風響應。



 榛遣廣詣行在奏之,其略曰:「邦傑與廣,忠義之心,堅若金石,臣自陷賊中,頗知其虛實。賊今稍惰,皆懷歸心,且累敗於西夏,而契丹亦出攻之。今山西諸砦鄉兵約十餘萬,力與賊抗,但皆苦窘,兼闕戎器。臣多方存恤,惟望朝廷遣兵來援,不然,久之恐反為賊用。臣於陛下,以禮言則君臣,以義言則兄弟,其憂國念親之心無異。願委臣總大軍,與諸砦鄉兵,
 約日大舉,決見成功。」廣既至,黃潛善、汪伯彥疑其非真,上識榛手書,遂除河外兵馬都元帥。潛善、伯彥終疑之,廣將行,密授朝旨,使幾察榛,復令廣聽諸路節制。廣知事不成,遂留於大名府不進。會有言榛將渡河入京,朝廷因詔擇日還京,以伐其謀。



 金人恐廣以援兵至,急發兵攻諸砦,斷其汲道,諸砦遂陷。榛亡,不知所在,或曰後興上皇同居五國城。



 紹興元年,鄧州有楊其姓者,聚千餘人,自稱信王。鎮撫使翟興覺詐,遣將斬之以聞。



 欽宗皇太子諶,朱皇后子也。政和七年生,為嫡皇孫,祖宗以來所未有,徽宗喜。蔡京奏除檢校少保、常德軍節度使,封崇國公,從之。會王黼得政,謀傾京,言其以東宮比人主,遂降為高州防禦使。靖康元年,遷檢校少保、昭慶軍節度使、大寧郡王。尋進檢校少傅、寧國軍節度使。四月,詔立為皇太子。



 二年,上幸青城,命密院同知孫傅兼太子少傅,吏部侍郎謝克家兼太子賓客,輔太子監國,稱制行事。未幾,金人請二帝諭太子出城。統制吳革
 力請留,欲以所募士微服衛太子潰圍以出。傅不許,乃謀匿民間,別求狀類太子者並宦者二人殺之,送金人,紿以宦者竊太子欲投獻,都人爭之,並傷太子。遲疑不決者五日。吳開、莫儔督脅甚急,範瓊恐變生,以危言讋衛士,遂擁太子與皇后共車以出。百官軍吏奔隨號哭,太學諸生擁拜車前,太子呼云:「百姓救我!」哭聲震天,已而北去。弟訓。



 訓乃北地所生。有碭山人留遇僧者,金人見之曰:「全似
 趙家少帝。」遇僧竊喜。紹興十年,三京路通,詔求宗室。遇僧自言少帝第二子,乃守臣遣赴行在,過泗州,州官孫守信疑之,白其守,請於朝。閣門言淵聖無第二子,詔寧信劾治。遇僧伏罪,黥隸瓊州。後有自北至者,曰:「淵聖小大王訓,見居五國城。」



 元懿太子諱敷,高宗子也,母潘賢妃。建炎元年六月,生於南京。拜檢校少保、集慶軍節度使,封魏國公。金人侵淮南,帝幸臨安,會苗傅、劉正彥作亂,逼帝禪位於敷,改
 元明受。既而傅等伏誅,帝復位,乃以敷為皇太子,從幸建康。太子立,屬疾,宮人誤蹴地上金爐有聲,太子驚悸,疾轉劇,薨,謚元懿。



 信王璩字潤夫,初名伯玖,藝祖七世孫,秉義郎子彥之子也。生而聰慧。



 初,伯琮以宗子被選入宮,高宗命鞠于婕妤張氏;吳才人亦請於帝,遂以伯玖命才人母之,賜名璩,除和州防禦使,時生七歲矣。伯琮以建國公就傅,璩獨居禁中。俄拜節度使,封吳國公,宰執趙鼎、劉大中、
 王庶等堅持之,命不果行。會秦檜專政,遂除保大軍節度使,封崇國公。尋詔赴資善堂聽讀。紹興十五年,加檢校少保,進封恩平郡王,出就外第。時伯琮己封普安郡王,璩官屬禮制相等夷,號東、西府。逾年,改武昌軍節度使。



 二十二年,子彥卒,璩去官持服,終喪,還舊官。顯仁太后崩,普安郡王始立為皇太子,璩因加恩稱皇侄,名位始定。遷開府儀同三司,判大宗正事,置司紹興府。



 孝宗即位,璩表請入賀,許之,特授少保,改靜江軍節度使。頃
 之,省紹興府宗正事,改判西外宗正司。璩累章乞閑,改醴泉觀使。淳熙中,除少傅。高宗崩,奔赴得疾,逾年而薨,年五十九,追封信王,累贈太保、太師。



 始,璩之入宮也,儲位未定者垂三十年,中外頗以為疑。孝宗既立,天性友愛,璩入朝,屢召宴內殿,呼以官,不名也,賜予無算。



 子四人:師淳歷忠州團練使、永州防禦使,師瀹、師淪、師路並補武翼大夫。孫希楙,特補保義郎。



 莊文太子諱□,孝宗嫡長子也,母郭皇后。初名愉,補右
 內率府副率,尋賜名□,除右監門衛大將軍、榮州刺史。孝宗為皇子時,□拜蘄州防禦使。及受禪,除少保、永興軍節度使,封鄧王。故事皇子出閣,封王,兼兩鎮,然後加司空。□自防禦使躐拜少保,章異數也。



 乾道元年,立為皇太子,冊廣國夫人錢氏為妃。詔增東宮從衛,太子謙讓。及奏捐月給雜物,從之。三年秋,太子病暍,醫誤投藥,病劇。上皇與帝親視疾,為赦天下。越三日薨,年二十四,謚莊文。



 太子賢厚,上皇與帝皆愛之。帝從禮官議服期,
 以日易月;文武百官服衰,服一日而除;東宮臣僚齊衰三月,臨七日而除。比葬,帝再至東宮,命宰臣奉謚冊,大小祥皆以執政官行禮。



 子挺,錢氏所生也,甫晬,除福州觀察使,封榮國公,乾道九年卒,贈武當軍節度使,追封豫國公。



 寧宗時,命宗子希琪為太子後。希琪,藝祖九世孫也,賜名搢,補右千牛衛將軍,置教授於府。開禧二年,除忠州防禦使。嘉定八年,更名思正。



 魏惠憲王諱愷,莊文同母弟也。初補右內率府副率,轉
 右監門衛大將軍、貴州團練使。孝宗受禪,拜雄武軍節度使、開府儀同三司,封慶王。



 莊文太子薨,愷次當立,帝意未決。既而以恭王英武類己,竟立之。加愷雄武、保寧軍節度使,進封魏王,判寧國府。妻華國夫人韋氏,特封韓、魏兩國夫人,以示優禮。賜黃金三千兩、白金一萬兩,命宰設祖於玉津園,王登車,顧謂虞允文曰:「更望相公保全。」比至鎮,奏朝天申節,許之。



 府長史上言,欲與司馬分治郡,俾王受成。愷奏曰:「臣被命判府,今專委長史、
 司馬,是處臣無用之地。況一郡置三判府,臣恐吏民紛競不一,徒見其擾。長史、司馬宜主錢穀、訟牒,俾擬呈臣依而判之,庶上下安,事益易治。」又請增士人貢額。朝廷悉從之。愷究心民事,築圩田之隤圮者,帝手詔嘉勞之。



 淳熙元年,徙判明州。輟屬邑田租以贍學。得兩歧麥,圖以獻,帝復賜手詔曰:「汝勸課藝植,農不游惰,宜獲瑞麥之應。」加愷荊南、集慶軍節度使,行江陵尹,尋改永興、成德軍節度使、揚州牧。七年,薨於明州,年三十五。帝素
 服發哀於別殿,贈淮南武寧軍節度使、揚州牧兼徐州牧,謚惠寧。



 王性寬慈,上皇雅愛之。雖以宗社大計出王於外,然心每念之,賜賚不絕。訃聞,帝滋然曰:「向所以越次建儲者,正為此子福氣差薄耳!」治二郡有仁聲,薨之日,四明父老乞建祠立碑,以紀遺愛。



 子二人。攄早卒。秉生於明州,母卜氏,信安郡夫人,王薨,還居行在。秉性早慧,帝愛之,將內禪,升耀州觀察使,封嘉國公。慶元間,封吳興郡王,領昭慶軍節度使。開禧二年薨,贈太保,封沂
 王,謚靖惠。



 子垓,三歲而夭。詔立宗室希瞿子為其後,更名均,領右千牛衛將軍,置教授於府。尋加福州觀察使。後更名貴和,即鎮王竑也。



 景獻太子諱詢,燕懿王後,藝祖十一世孫也。初名與願。寧宗既失袞王,從宰執京鏜等請,取與願養於宮中,年六歲,賜名□嚴,除福州觀察使。嘉泰二年,拜威武軍節度使,封衛國公,聽讀資善堂。



 開禧元年,時邊事益急,金人請誅首謀用兵者,□嚴用翊善史彌遠計,奏韓侂冑輕起
 兵端,上危宗社,宜賜黜罷,以安邊境。從之。



 曮立為皇太子,拜開府儀同三司,封榮王,更為幬。詔御朝太子侍立,宰執日赴資善堂會議。尋用天禧故事,宰輔大臣並兼師傅、賓客,太子出居東宮,更名詢。嘉定十三年薨,年二十九,謚景獻。



 鎮王竑,希瞿之子也。初,沂靖惠王薨,無嗣,以竑為之後,賜名均,尋改賜名貴和。太子詢薨,乃立貴和為皇子,賜名竑,授寧武軍節度使,封祁國公。嘉定十五年五月,加
 檢校少保,封濟國公。



 十七年六月辛未,竑生子,詔告天地、宗廟、社稷、宮觀。八月祭未,賜竑子名銓,授左千牛衛大將軍。丁亥,銓薨,贈復州防禦使,追封永寧侯。竑上表稱謝。



 竑好鼓琴,丞相史彌遠買美人善琴者,納諸御,而厚廩其家,使美人目閑竑,動息必以告。美人知書慧黠,竑嬖之。宮壁有輿地圖,竑指瓊崖曰:「吾他日得志,置史彌遠於此。」又嘗呼彌遠為「新恩」。以他日非新州則恩州也。彌遠聞之,嘗因七月七日進乞巧奇玩以覘之,竑乘
 酒碎於地。彌遠大懼,日夕思以處竑,而竑不知也。



 時沂王猶未有後,方選宗室希□子昀繼之。一日,彌遠為其父飯僧凈慈寺,獨與國子學錄鄭清之登惠日閣,屏人語曰:「皇子不堪負荷,聞後沂邸者甚賢,今欲擇講官,君其善訓迪之。事成,彌遠之坐即君坐也。然言出於彌遠之口,入於君之耳,若一語洩者,吾與君皆族矣。」清之拱手曰:「不敢。」乃以清之兼魏忠憲王府教授。清之日教昀為文,又購高宗書俾習焉。清之上謁彌遠,即以昀詩文
 翰墨以示,彌遠譽之不容口。彌遠嘗問清之:「吾聞其賢已熟,大要竟何如?」清之曰:「其人之賢,更僕不能數,然一言以斷之曰:不凡。」彌遠頷之再三,策立之意益堅。清之始以小官兼教授,其後累遷,兼如故。



 寧宗崩,彌遠始遣清之往,告昀以將立之之意。再三言之,昀默然不應。最後清之乃言曰:「丞相以清之從游之久,故使布腹心於足下。今足下不答一語,則清之將何以復命於丞相?」昀始拱手徐答曰:「紹興老母在。」清之以告彌遠,益相與嘆
 其不凡。



 竑跂足以需宣召,久而不至。彌遠在禁中,遣快行宣皇子,令之曰:「今所宣是沂靖惠王府皇子,非萬歲巷皇子,茍誤,則汝曹皆處斬。」竑不能自己,屬目墻壁間,見快行過其府而不入,疑焉。己而擁一人徑過,天已暝,不知其為誰,甚惑。



 昀既至,彌遠引入柩前,舉哀畢,然後召竑。竑聞命亟赴,至則每過宮門,禁衛拒其從者。彌遠亦引入柩前,舉哀畢,引出帷,殿帥夏震守之。既而召百官立班聽遺制,則引竑仍就舊班,竑愕然曰:「今日之事,
 我豈當仍在此班?」震紿之曰:「未宣制以前當在此,宣制後乃即位耳。」竑以為然。未幾,遙見燭影中一人已在御坐,宣制畢,閣門贊呼,百官拜舞,賀新皇帝即位。竑不肯拜,震捽其首下拜。皇后矯遺詔:竑開府儀同三司,進封濟陽郡王,判寧國府。帝因加竑少保,進封濟王。九月丁丑,以竑充醴泉觀使,令就賜第。



 寶慶元年正月庚午,湖州人潘壬與其弟丙謀立竑,竑聞變匿水竇中,壬等得之,擁至州治,以黃袍加身。竑號泣不從,不獲已,與之約
 曰:「汝能勿傷太后、官家乎?」眾許諾。遂發軍資庫金帛、會子犒軍,命守臣謝周卿率官屬入賀,偽為李全榜揭於門,數彌遠廢立罪,云:「今領精兵二十萬,水陸進討。」比明視之,皆太湖漁人及巡尉兵卒,不滿百人耳。竑知其謀不成,率州兵討之。遣王元春告於朝,彌遠命殿司將彭任討之,至則事平。彌遠令客秦天錫托召醫治竑疾,竑本無疾。丙戌,天錫詣竑,諭旨逼竑縊於州治。



 帝輟朝,賻銀絹各一千、會子萬貫,贈少師、保靜鎮潼軍節度使。給
 事中盛章、權直舍人院王既一再繳奏,詔從之。右正言李知孝累奏,追奪王爵,降封巴陵縣公。於是在廷之臣真德秀、魏了翁、洪咨夔、胡夢昱等每以竑為言,彌遠輒惡而斥遠之。



 端平元年,詔復官爵。妻吳氏為比丘尼,賜惠凈法空大師,月給缽錢百貫。景定五年,度宗降詔,追復元贈節度使。德祐元年,提領戶部財用兼修國史常楙請立竑後,試禮部侍郎兼中書舍人王應麟請更封大國,表墓錫謚,命大宗正司議選擇立後,迎善氣,銷惡
 運,莫先於此。下禮部議,贈太師、尚書令,依舊節度使,升封鎮王,謚昭肅。以田萬畝賜其家,遣應麟致祭



\end{pinyinscope}