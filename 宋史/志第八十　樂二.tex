\article{志第八十 樂二}

\begin{pinyinscope}

 景祐三年七月,馮元等上新修《景祐廣樂記》八十一卷,詔翰林學士丁度、知制誥胥偃、直史館高若訥、直集賢院韓琦取鄧保信、阮逸、胡瑗等鐘律,詳定得失可否以
 聞。



 九月,阮逸言:「臣等所造鐘磬皆本於馮元、宋祁,其分方定律又出於胡瑗算術,而臣獨執《周禮》嘉量聲中黃鐘之法及《國語》鈞鐘弦準之制,皆抑而不用。臣前蒙召對,言王樸律高而李照鐘下。竊睹禦制《樂髓新經歷代度量衡》篇,言《隋書》依《漢志》黍尺制管,或不容千二百,或不啻九寸之長,此則明班《志》已後,歷代無有符合者。惟蔡邕銅龠本得於《周禮》遺範,邕自知音,所以只傳銅龠,積成嘉量,則是聲中黃鐘而律本定矣。謂管有大小長
 短者,蓋嘉量既成,即以量聲定尺明矣。今議者但爭《漢志》黍尺無準之法,殊不知鐘有鈞、石、量、衡之制。況《周禮》、《國語》,姬代聖經,翻謂無憑,孰為稽古?有唐張文收定樂,亦鑄銅甌,此足驗周之嘉量以聲定律明矣。臣所以獨執《周禮》鑄嘉量者,以其方尺深尺,則度可見也;其容一釜,則量右見也;其重鈞,則衡可見也;聲中黃鐘之宮,則律可見也。既律、度、量、衡如此符合,則制管歌聲,其中必矣。臣昧死欲乞將臣見鑄成銅甌,再限半月內更鑄嘉
 量,以其聲中黃鐘之宮,乃取李照新鐘就加修整,務合周制鐘量法度。文字已編寫次,未敢具進。」詔送度等並定以聞。



 十月,度等言:「據鄧保信黍尺二,其一稱用上黨秬黍圓者一黍之長,累百成尺,與蔡邕合。臣等檢詳前代造尺,皆以一黍之廣為分,唯後魏公孫崇以一黍之長累為寸法,太常劉芳以秬黍中者一黍之廣即為一分,中尉元匡以一黍之廣度黍二縫以取一分,三家競不能決。而蔡邕銅龠,本志中亦不明言用黍長廣累尺。今
 將保信黃鐘管內秬黍二百粒以黍長為分,再累至尺二條,比保信元尺一長五黍,一長七黍,又律管黃鐘龠一枚,容秬黍千二百粒,以元尺比量,分寸略同。復將實龠秬黍再累者校之,即又不同。其龠、合、升、斗亦皆類此。又阮逸、胡瑗鐘律法黍尺,其一稱用上黨羊頭山秬黍中者累廣求尺,制黃鐘之聲。臣等以其大黍百粒累廣成尺,復將管內二百粒以黍廣為分,再累至尺二條,比逸等元尺一短七黍,一短三黍。蓋逸等元尺並用一等大
 黍,其實管之黍大小不均,遂致差異。又其銅律管十二枚,臣等據楚衍等圍九方分之法,與逸等元尺及所實龠秬黍再累成尺者校之,又各不同。又所制銅稱二量亦皆類此。臣等看詳其鐘、磬各一架,雖合典故,而黍尺一差,難以定奪。」又言:「太祖皇帝嘗詔和峴等用景表尺典修金石,七十年間,薦之郊廟,稽合唐制,以示詒謀。則可且依景表舊尺,俟天下有妙達鐘律之學者,俾考正之,以從周、漢之制。其阮逸、胡瑗、鄧保信並李照所用太
 府寺等尺及阮逸狀進《周禮》度量法,其說疏舛,不可依用。」



 五年五月,右司諫韓琦言:「臣前奉詔詳定鐘律,嘗覽《景祐廣樂記》,睹照所造樂不依古法,皆率己意別為律度,朝廷因而施用,識者非之。今將親祀南郊,不可重以違古之樂上薦天地、宗廟。竊聞太常舊樂見有存者,郊廟大禮,請復用之。」詔資政殿大學士宋綬、三司使晏殊同兩制官詳定以聞。七月,綬等言:「李照新樂比舊樂下三律,眾論以為無所考據。願如琦請,郊廟復用和峴所
 定舊樂,鐘磬不經鐫磨者猶存三縣奇七虡,郊廟、殿庭可以更用。」太常亦言:「舊樂,宮縣用龍鳳散鼓四面,以應樂節,李照廢而不用,止以晉鼓一面應節。舊樂,建鼓四,並鞞、應共十二面,備而不擊,李照以四隅建鼓與鎛鐘相應擊之。舊樂,雷鼓兩架各八面,止用一人考擊,李照別造雷鼓,每面各用一人椎鼓,順天左旋,三步一止,又令二人搖□以應之。又所造大竽、大笙、雙鳳管、兩儀琴、十二弦琴並行。今既復用舊樂,未審照所作樂器制度,
 合改與否?」詔:「悉仍舊制,其李照所作,勿復施用。」



 康定元年,阮逸上《鐘律制議》並圖三卷。皇祐二年五月,明堂禮儀使言:「明堂所用樂皆當隨月用律,九月以無射為均,五天帝各用本音之樂。」於是內出明堂樂曲及二舞名:迎神曰《誠安》;皇帝升降行止曰《儀安》;昊天上帝、皇地祇、神州地祇位奠玉幣曰《鎮安》,酌獻曰《慶安》;太祖、太宗、真宗位奠幣曰《信安》,酌獻曰《孝安》,司徒奉俎曰《饎安》;五帝位奠玉幣曰《鎮安》,酌獻曰《精安》,皇帝飲福曰《胙安》;退文
 舞、迎武舞、亞獻、終獻皆曰《穆安》,徹豆曰《歆安》,送神曰《誠安》歸大次曰《憩安》;文舞曰《右文化俗》,武舞曰《威功睿德》。又出御撰樂章《鎮安》、《慶安》、《信安》、《孝安》四曲,餘詔輔臣分撰。庚戌,詔:「御所撰樂曲名與常祀同者,更之。』遂更常所用圜丘寓祭明堂《誠安》之曲曰《宗安》,祀感生帝《慶安》之曲曰《光安》,奉慈廟《信安》之曲曰《慈安》。



 六月,內出御撰明堂樂八曲,以君、臣、民、事、物配屬五音,凡二十聲為一曲;用宮變、征變者,天、地、人、四時為七音,凡三十聲為一
 曲;以子母相生,凡二十八聲為一曲:皆黃鐘為均。又明堂月律五十七聲為二曲,皆無射為均;又以二十聲、二十八聲、三十聲為三曲,亦無射為均,皆自黃鐘宮入無射。如合用四十八或五十七聲,即依前譜次第成曲,其徹聲自同本律。及御撰鼓吹、警嚴曲、合宮歌並肄於太常。



 是月,翰林學士承旨王堯臣等言:



 奉詔與參議阮逸所上編鐘四清聲譜法,請用之於明堂者。竊以律呂旋宮之法既定以管,又制十二鐘準為十二正聲,以律計自
 倍半。說者云:「半者,準正聲之半,以為十二子聲之鐘,故有正聲、子聲各十二。」子聲即清聲也。其正管長者為均,自用正聲;正管短者為均,則通用子聲而成五音。然求聲之法,本之於鐘,故《國語》所謂「度律均鐘」者也。



 其編金石之法,則歷代不同,或以十九為一虡者,蓋取十二鐘當一月之辰,又加七律焉;或以二十一為一虡者,以一均聲更加濁倍;或以十六為一虡者,以一均清、正為十四,宮、商各置一,是謂「縣八用七」也;或以二十四為一虡,則
 清、正之聲備。故唐制以十六數為小架,二十四為大架,天地、宗廟、朝會各有所施。



 今太常鐘縣十六者,舊傳正聲之外有黃鐘至夾鐘四清聲,雖於圖典未明所出,然考之實有義趣。蓋自夷則至應鐘四律為均之時,若盡用正聲,則宮輕而商重,緣宮聲以下,不容更有濁聲。一均之中,宮弱商強,是謂陵僭,故須用子聲,乃得長短相敘。自角而下,亦循茲法。故夷則為宮,則黃鐘為角;南呂為宮,則大呂為角;無射為宮,則黃鐘為商、太簇為角;應
 鐘為宮,則大呂為商、夾鐘為角。蓋黃鐘、大呂、太簇、夾鐘正律俱長,並當用清聲,如此則音律相諧而無所抗,此四清聲可用之驗也。至他律為宮,其長短、尊卑自序者,不當更以清聲間之。



 自唐末世,樂文墜缺,考擊之法久已不傳。今若使匏、土、絲、竹諸器盡求清聲,即未見其法。又據大樂諸工所陳,自磬、簫、琴、和、巢笙五器本有清聲,塤、篪、竽、築、瑟五器本無清聲,五弦阮、九弦琴則有太宗皇帝聖制譜法。至歌工引音極唱,止及黃鐘清聲。



 臣等
 參議,其清、正二聲既有典據,理當施用。自今大樂奏夷則以下四均正律為宮之時,商、角依次並用清聲,自餘八均盡如常法。至於絲、竹等諸器舊有清聲者,令隨鐘石教習;本無清聲者,未可創意求法,且當如舊。惟歌者本用中聲,故夏禹以聲為律,明人皆可及。若強所不至,足累至和。請止以正聲作歌,應合諸器亦自是一音,別無差戾。其阮逸所上聲譜,以清濁相應,先後互擊,取音靡曼,近於鄭聲,不可用。



 詔可。



 七月,御撰明堂無射宮樂
 曲譜三,皆五十七字,五音一曲,奉俎用之;二變七律一曲,飲福用之;七律相生一曲,退文舞、迎武舞及亞獻、終獻、徹豆用之。



 是月,上封事者言:「明堂酌獻五帝《精安》之曲,並用黃鐘一均聲,此乃國朝常祀、五時迎氣所用舊法,若於親行大饗,即所未安。且明堂之位,木室在寅,火室在巳,金室在申,水室在亥,蓋木、火、金、水之始也;土室在西南,蓋土王之次也。既皆用五行本始所王之次,則獻神之樂亦當用五行本始月律,各從其音以為曲。其《
 精安》五曲,宜以無射之均;太簇為角,獻青帝;仲呂為征,獻赤帝;林鐘為宮,獻黃帝;夷則為商,獻白帝;應鐘為羽,獻黑帝。」詔兩制官同太常議,而堯臣等言:「大饗日迫,事難猝更。」詔俟過大禮,詳定以聞。



 九月,帝服靴袍,御崇政殿,召近臣、宗室、館閣、臺諫官閱雅樂,自宮架、登歌、舞佾之奏凡九十一曲遍作之,因出太宗琴、阮譜及御撰明堂樂曲音譜,並按習大樂新錄,賜群臣。又出新制頌塤、匏笙、洞簫,仍令登歌以八音諸器各奏一曲,遂召鼓吹
 局按警場,賜大樂、鼓吹令丞至樂工徒吏緡錢有差。帝既閱雅樂,謂輔臣曰:「作樂崇德,薦之上帝,以配祖考。今將有事於明堂,然世鮮知音,其令太常並加講求。」時言者以為鎛鐘、特磬未協音律,詔令鄧保信、阮逸、盧昭序同太常檢詳典禮,別行鑄造。太常薦太子中舍致仕胡瑗曉音,詔同定鐘磬制度。



 閏十一月,詔曰:「朕聞古者作樂,本以薦上帝、配祖考,三、五之盛,不相沿襲,然必太平,始克明備。周武受命,至成王時始大合樂;漢初亦沿舊
 樂,至武帝時始定泰一、后土樂詩;光武中興,至明帝時始改「大予」之名;唐高祖造邦,至太宗時孝孫、文收始定鐘律,明皇方成唐樂。是知經啟善述,禮樂重事,須三四世,聲文乃定。



 國初亦循用王樸、竇儼所定周樂,太祖患其聲高,遂令和峴減一律,真宗始議隨月轉律之法,屢加按核。然念《樂經》久墜,學者罕傳,歷古研覃,亦未究緒。頃雖博加訪求,終未有知聲、知經可信之人。嘗為改更,未適茲意。中書門下其集兩制及太常禮樂官,以天地、
 五方、神州、日月、宗廟、社蠟祭享所用登歌、宮縣,審定聲律是非,按古合今,調諧中和,使經久可用,以發揚祖宗之功德,朕何憚改為?但審聲、驗書,二學鮮並,互詆胸臆,無所援據,慨然希古,靡忘於懷。」



 於是中書門下集兩制、太常官,置局於秘閣,詳定大樂。王堯臣等言:天章閣待制趙師民博通今古,願同祥定,及乞借參知政事高若訥所校十五等古尺。並從之。



 三年正月,詔徐、宿、泗、耀、江、鄭、淮陽七州軍採磬石,仍令諸路轉運司訪民間有藏
 古尺律者上之。二月,詔兩制及禮官參稽典制,議定國朝大樂名,中書門下審加詳閱以聞。初,胡瑗請太祖廟舞用干戚,太宗廟兼用干、羽,真宗廟用羽、龠,以像三聖功德。然議者謂國朝七廟之舞,名雖不同,而干、羽並用,又廟制與古異。及瑗建言,止降詔定樂名而已。



 七月,堯臣等言:「按太常天地、宗廟、四時之祀,樂章凡八十九曲,自《景安》而下七十五章,率以『安』名曲,豈特本道德、政教嘉靖之美,亦緣神靈、祖考安樂之故。臣等謹上議,國朝樂
 宜名《大安》。」詔曰:「朕惟古先格王隨代之樂,亦既制作,必有稱謂,緣名以討義,由義以知德,蓋名者,德之所載,有行遠垂久之效焉。故《韶》以紹堯,《夏》以承舜,《濩》以救民,《武》以象伐,傳之不朽,用此道也。國家舉墜正失,典章交備,獨斯體大而有司莫敢易言之。朕憫然念茲,大懼列聖之休未能昭揭於天下之聽,是用申敕執事,還求博講而考定其衷。今禮官、學士迨三有事之臣,同寅一辭,以《大安》之議來復。且謂:藝祖之戡暴亂也,安天下之未安,
 其功大;二宗之致太平也,安天下之既安,其盛;洎朕之承烈也,安祖宗之所安,其仁厚。祇覽所議,熟復於懷。恭惟神德之造基,神功之戢武,章聖恢清凈之治,沖人蒙成定之業,雖因世之跡各異,而靖民之道同歸。以之播鐘球、文羽鑰、用諸郊廟、告於神明,曰『大』且『安』,誠得其正。」



 十二月,召兩府及侍臣觀新樂於紫宸殿,凡鎛鐘十二:黃鐘高二尺二寸半,廣一尺二寸,鼓六,鉦四,舞六,甬、衡並旋蟲高八寸四分,遂徑一寸二分,深一寸一厘,篆
 帶每面縱者四,橫者四,枚景挾鼓與舞,四處各有九,每面共三十六,兩欒間一尺四寸,容九斗九升五合,重一百六斤;大呂以下十一鐘並與黃鐘同制,而兩欒間遞減半分;至應鐘容九斗三升五合,而其重加至應鐘重一百四十八斤;並中新律本律。特磬十二:黃鐘、大呂股長二尺,博一尺,鼓三尺,博六寸九分寸之六,弦三尺七寸五分;太簇以下股長尺八寸,博九寸,鼓二尺七寸,博六寸,弦三尺三寸七分半,其聲各中本律。黃鐘厚二寸一
 分,大呂以下遞加其厚,至應鐘厚三寸五分。詔以其圖送中書。議者以為《周禮》:「大鐘十分其鼓間,以其一為之厚;小鐘十分其鉦間,以其一為之厚。」則是大鐘宜厚,小鐘宜薄。今大鐘重一百六斤,小鐘乃重一百四十八斤,則小鐘厚,非也。又:「磬氏為磬,倨句一矩有半,博為一,股為二,鼓為三。三分其股博,去其一以為鼓博;三分其鼓博,以其一為之厚。」今磬無博厚、無長短,亦非也。



 五年四月,命參知政事劉沆、梁適監議大樂。是月,知制誥王洙奏:「
 黃鐘為宮最尊者,但聲有尊卑耳,不必在其形體也。言鐘磬依律數為大小之制者,經典無正文,惟鄭康成立意言之,亦自云假設之法。孔穎達作疏,因而述之。據歷代史籍,亦無鐘磬依律數大小之說,其康成、穎達等即非身曾制作樂器。至如言『磬前長三律,二尺七寸;後長二律,一尺八寸,是磬有大小之制』者,據此以黃鐘為律。臣曾依此法造黃鐘特磬者,止得林鐘律聲。若隨律長短為鐘磬大小之制,則黃鐘長二尺二寸半,減至應鐘,則形制大
 小比黃鐘才四分之一。又九月、十月以無射、應鐘為宮,即黃鐘、大呂反為商聲,宮小而商大,是君弱臣強之象。今參酌其鎛鐘、特磬制度,欲且各依律數,算定長短、大小、容受之數,仍以皇祐中黍尺為法,鑄大呂、應鐘鐘磬各一,即見形制、聲韻所歸。」奏可。



 五月,翰林學士承旨王拱辰言:「奉詔詳定大樂,比臣至局,鐘磬已成。竊緣律有長短,磬有大小,黃鐘九寸最長,其氣陽,其象土,其正聲為宮,為諸律之首,蓋君德之象,不可並也。今十二鐘磬,
 一以黃鐘為率,與古為異。臣等亦嘗詢逸、瑗等,皆言『依律大小,則聲不能諧。』故臣竊有疑,請下詳定大樂所,更稽古義參定之。」是月,知諫院李兌言:「曩者紫宸殿閱太常新樂,議者以鐘之形制未中律度,遂斥而不用,復詔近臣詳定。竊聞崇文院聚議,而王拱辰欲更前史之義,王洙不從,議論喧嘖。夫樂之道廣大微妙,非知音入神,豈可輕議?西漢去聖尚近,有制氏世典大樂,但能紀其鏗鏘,而不能言其義。況今又千餘年,而欲求三代之音,
 不亦難乎?且阮逸罪廢之人,安能通聖明述作之事?務為異說,欲規恩賞。朝廷制樂數年,當國財匱乏之時,煩費甚廣。器既成矣,又欲改為,雖命兩府大臣監議,然未能裁定其當。請以新成鐘磬與祖宗舊樂參校其聲,但取諧和近雅者合用之。」



 六月,帝御紫宸殿,奏太常新定《大安》之樂,召輔臣至省府、館閣預觀焉,賜詳定官器幣有差。八月,詔:「南郊姑用舊樂,其新定《大安》之樂,常祀及朝會用之。」翰林學士胡宿上言:「自古無並用二樂之理,
 今舊樂高,新樂下,相去一律,難並用。且新樂未施郊廟,先用之朝會,非先王薦上帝、配祖考之意。」帝以為然。九月,御崇政殿,召近臣、宗室、臺諫、省府推判官觀新樂並新作晉鼓。乃以瑗為大理寺丞,逸復尚書屯田員外郎,保信領榮州防禦使,入內東頭供奉官賈宣吉為內殿承制,並以制鐘律成,特遷之。



 至和元年,言者多以陰陽不和由大樂未定。帝曰:「樂之不合於古久矣。水旱之來,系時政得失,豈特樂所召哉?」二年,潭州上瀏陽縣所得
 古鐘,送太常。初,李照斥王樸樂音高,乃作新樂,下其聲。太常歌工病其太濁,歌不成聲,私賂鑄工,使減銅齊,而聲稍清,歌乃協。然照卒莫之辨。又樸所制編鐘皆側垂,照、瑗皆非之。及照將鑄鐘,給銅于鑄瀉務,得古編鐘一,工人不敢毀,乃藏於太常。鐘不知何代所作,其銘云:「粵朕皇祖寶和鐘,粵斯萬年,子子孫孫永寶用。」叩其聲,與樸鐘夷則清聲合,而其形側垂。瑗後改鑄,正其鈕,使下垂,叩之弇,鬱而不揚。其鎛鐘又長甬而震掉,聲不和。
 著作佐郎劉羲叟謂人曰:「此與周景王無射鐘無異,上將有眩惑之疾。」嘉祐元年正月,帝御大慶殿受朝,前一夕,殿庭設仗衛、既具而大雨雪,至壓宮架折,帝於禁中跣而告天,遂暴感風眩,人以羲叟之言為驗。八月,禦制恭謝樂章。是月,詔恭謝用舊樂。



 四年九月,禦制祫享樂舞名:僖祖奏《大基》,順祖奏《大祚》,翼祖奏《大熙》,宣祖奏《大光》,太祖奏《大統》,太宗奏《大昌》,真宗奏《大治》,孝惠皇后奏《淑安》,孝章皇后奏《靜安》,淑德皇后奏《柔安》,章懷皇后奏《和
 安》,迎神、送神奏《懷安》,皇帝升降奏《肅安》,奠瓚奏《顧安》,奉俎、徹豆奏《充安》,飲福奏《禧安》,亞獻、終獻奏《祐安》,退文舞、迎武舞奏《顯安》,皇帝歸大次奏《定安》,登樓禮成奏《聖安》,駕回奏《採茨》;文舞曰《化成治定》,武舞曰《崇功昭德》。帝自制迎神、送神樂章,詔宰臣富弼等撰《大祚》至《採茨》曲詞十八。七年八月,禦制明堂迎神樂章,皆肄於太常。



 翰林學士王珪言:「昔之作樂,以五聲播於八音,調和諧合而與治道通,先王用於天地、宗廟、社稷,事於山川鬼神,使
 鳥獸盡感,況於人乎?然則樂雖盛而音虧,未知其所以為樂也。今郊廟升歌之樂,有金、石、絲、竹、匏、土、革而無木音。夫所謂柷吾文者,聖人用以著樂之始終,顧豈容有缺耶?且樂莫隆於《韶》,《書》曰『戛擊』,是柷、敔之用。既云下而擊□,知鳴球與柷吾文之在堂,故《傳》曰:『堂上堂下,各有柷敔也』。今陛下躬祠明堂,宜詔有司考樂之失而合八音之和。」於是下禮官議,而堂上始置柷敔。



 又秘閣校理裴煜奏:「大祠與國忌同者,有司援舊制,禮樂備而不作。忌日
 必哀,志有所至,其不有樂,宜也。然樂所以降格神祇非以適一己之私也。謹案開元中禮部建言,忌日享廟應用樂。裴寬立議,廟尊忌卑則作樂,廟卑忌尊則備而不奏。中書令張說以寬議為是。宗廟如此,則天地、日月、社稷之祠用樂明矣。臣以為凡大祠天地、日月、社稷與忌日同者,伏請用樂,其在廟則如寬之議。所冀略輕存重,不失其稱。」下其章禮官,議曰「《傳》稱祭天以禋為歆神之始,以血為陳饌之始;祭地以埋為歆神之始,以血為陳
 饌之始。宗廟以灌為歆神之始,以腥為陳饒之始。然則天地、宗廟皆以樂為致神之始,故曰大祭有三始,謂此也。天地之間虛豁而不見其形者,陽也。鬼神居天地之間,不可以人道接也。聲屬於陽,故樂之音聲號呼召於天地之間,庶幾神明聞之,因而來格,故祭必求諸陽。商人之祭,先奏樂以求神,先求於陽也;次灌地求神於陰,達於淵泉也。周人尚臭,四時之祭,先灌地以求神,先求諸陰也。然則天神、地祇、人鬼之祀不可去樂明矣。今七
 廟連室,難分廟忌之尊卑,欲依唐制及國朝故事:廟祭與忌同日,並縣而不作;其與別廟諸後忌同者,作之;若祠天地、日月、九宮、太一及蠟百神,並請作樂;社稷以下諸祠既卑於廟,則樂可不作。」翰林學士王珪等以為:「社稷,國之所尊,其祠日若與別廟諸後忌同者,伏請亦不去樂。」詔可。



 英宗治平元年六月,太常寺奏,仁宗配饗明堂,奠幣歌《誠安》,酌獻歌《德安》。二年九月,禮官李育上言:「南郊、太廟
 二舞郎總六十八,文舞罷,舍羽鑰,執干戚,就為武舞。臣謹按舊典,文、武二舞各用八佾,凡祀圜丘、祀宗廟,太樂令率工人以入,就位,文舞入,陳於架北,武舞立於架南。又文舞出,武舞入,有送迎之曲,名曰《舒和》,亦曰《同和》,凡三十一章,止用一曲。是進退同時,行綴先定,步武容體,各應樂節。夫《玄德升聞》之舞象揖讓,《天下大定》之舞象征伐,柔毅舒急不侔,而所法所習亦異,不當中易也。竊惟天神皆降,地祇皆出,八音克諧,祖考來格,天子親執
 珪幣,『相維闢公』,『嚴恭寅畏』,可謂極矣。而舞者紛然縱橫於下,進退取舍,蹙迫如是,豈明有德、像有功之誼哉?國家三年而躬一郊,同殿而享八室,而舞者闕如,名曰二舞,實一舞也。且如大朝會所以宴臣下,而舞者備其數;郊廟所以事天地、祖考,而舞者減其半:殊未為稱。事有近而不可邇,禮有繁而不可省,所系者大,而有司之職不敢廢也。伏請南郊、太廟文武二舞各用六十四人,以備帝王之禮樂,以明祖宗之功德。」奏可。



 四年八月,學士
 院建言:「國朝宗廟之樂,各以功德名舞。洪惟英宗,繼天遵業,欽明勤儉,不自暇逸。踐祚未幾,而恩行威立,固已超軼百王之上。今厚陵復土,祔廟有期,而樂名未立,亡以詔萬世。請上樂章及名廟所用舞曰《大英》之舞。自後禮官、御史有所建明,而詳定朝會及郊廟禮文官於樂節有議論,率以時考正之。」



 神宗熙寧九年,禮官以宗廟樂節而有請者三:



 其一、今祠太廟《興安》之曲,舉柷而聲已過,舉敔而聲不止,則始
 終之節未明。請祠祭用樂,一奏將終,則戛敔而聲少止,擊柷則樂復作,以盡合止之義。



 其二、大樂降神之樂,均聲未齊,短長不協,故舞行疾徐亦不能一。請以一曲為一變,六變用六,九變用九,則樂舞始終莫不應節。



 其三、周人尚臭,蓋先灌而後作樂;本朝宗廟之禮多從周,請先灌而後作樂。



 元豐二年,詳定所以朝會樂而有請者十:



 其一、唐元正、冬至大朝會,迎送王公用《舒和》,《開元禮》以初入門《舒和》之樂作,至位,樂止。蓋作樂所以待王公,
 今中書、門下、親王、使相先於丹墀上東西立,皇帝升御坐,乃奏樂引三品以上官,未為得禮。請侍從及應赴官先就立位,中書、門下、親王、使相、諸司三品、尚書省四品及宗室、將軍以上,班分東西入,《正安》之樂作,至位,樂止。



 其二、今朝會儀:舉第一爵,宮縣奏《和安》之曲,第二、第三、第四,登歌作《慶雲》、《嘉禾》、《靈芝》之曲。則是合樂在前、登歌在後,有違古義。請第一爵,登歌奏《和安》之曲,堂上之樂隨歌而發;第二爵,笙入奏《慶雲》之曲,止吹笙,餘樂不作;
 第三爵,堂上歌《嘉禾》之曲,堂下吹笙,《瑞木成文》之曲,一歌一吹相間;第四爵,合樂奏《靈芝》之曲,堂上下之樂交作。



 其三、定文舞、武舞各為四表,表距四步為酇綴,各六十四。文舞者服進賢冠,左執鑰,右秉翟,分八佾,二工執纛引前,衣冠同之。舞者進蹈安徐,進一步則兩兩相顧揖,三步三揖,四步為三辭之容,是為一成。餘成如之。自南第一表至第二表為第一成,至第三表為再成,至北第一表為三成,覆身卻行至第三表為四成,至第二表為
 五成,復至南第一表為六成,而武舞入。今文舞所秉翟羽,則集雉尾置於髹漆之柄,求之古制,實無所本。聶崇義圖,羽舞所執類羽葆幢,析羽四重,以結綬系於柄,此纛翳之謂也。請按圖以翟羽為之。



 其四、武舞服平巾幘,左執干,右執戈。二工執旌居前;執□、執鐸各二工;金錞二,四工舉;二工執鐲、執鐃;執相在左,執雅在右,亦各二工;夾引舞者,衣冠同之。分八佾於南表前,先振鐸以通鼓,乃擊鼓以警戒,舞工聞鼓聲,則各依酇綴總干正立
 定位,堂上長歌以詠嘆之。於是播□以導舞,舞者進步,自南而北,至最南表,以見舞漸。然後左右夾振鐸,次擊鼓,以金錞和之,以金鐲節之,以相而輔樂,以雅而陔步。舞者發揚蹈厲,為猛賁趫速之狀。每步一進,則兩兩以戈盾相向,一擊一刺為一伐,四伐為一成,成謂之變。至第二表為一變;至第三表為二變;至北第一表為三變;舞者覆身向堂,卻行而南,至第三表為四變;乃擊刺而前,至第二表回易行列,舂、雅節步分左右而跪,以右膝
 至地,左足仰起,像以文止武為五變;舞蹈而進,為兵還振旅之狀,振鐸、搖□、擊鼓,和以金錞,廢鐲鳴鐃,復至南第一表為六變而舞畢。古者,人君自舞《大武》,故服冕執干戚。若用八佾而為擊刺之容,則舞者執干戈。說者謂武舞戰象樂六奏,每一奏之中,率以戈矛四擊刺。戈則擊兵,矛則刺兵,玉戚非可施於擊刺,今舞執干戚,蓋沿襲之誤。請左執干,右執戈。



 其五、古之鄉射禮,三笙一和而成聲,謂三人吹笙,一人吹和。今朝會作樂,丹墀之上,
 巢笙、和笙各二人,其數相敵,非也。蓋鄉射乃列國大夫、士之禮,請增倍為八人,丹墀東西各三巢一和。



 其六、今宮縣四隅雖有建鼓、鞞、應,相傳不擊。乾德中,詔四建鼓並左右鞞、應合十有二,依李照所奏,以月建為均,與鎛鐘相應。鞞、應在建鼓旁,是亦朔鼙、應鼙之類。請將作樂之時,先擊鼙,次擊應,然後擊建鼓。



 其七、今樂縣四隅設建鼓,不擊,別施散鼓於樂縣內代之。乾德中,尹拙奏宜去散鼓,詔可,而樂工積習亦不能廢。李照議作晉鼓,以
 為樂節。請樂縣內去散鼓,設晉鼓以鼓金奏。



 其八、古者,瞽蒙、……,所以節一唱之終。請宮縣設□,以為樂節。



 其九、以天子禮求之,凡樂事播□,擊頌磬、笙磬,以鐘鼓奏《九夏》,是皆在庭之樂;戛擊則柷敔,球則玉磬,搏拊所以節樂,琴瑟所以詠詩,皆堂上樂也。磬本在堂下,尊玉磬,故進之使在上,若擊石拊石,則當在庭。後世不原於此,以春秋鄭人賂晉俟歌鐘二肆,遂於堂上設歌鐘、歌磬,蓋歌鐘則堂上歌之,堂下以鼓應之耳。歌
 必金奏相和,名曰歌鐘,則以節歌是已,豈堂上有鐘邪?歌磬之名,本無所出,晉賀循奏置登歌簨虡,採玉造小磬,蓋取舜廟鳴球之制。後周登歌,備錄鐘磬,隋、唐迄今,因襲行之,皆不應禮。請正、至朝會,堂上之樂不設鐘磬。



 其十、古者歌工之數:大射工六人,四瑟,則是諸侯鼓瑟以四人,歌以二人;天子八人,則瑟與歌皆四人矣。魏、晉以來,登歌五人,隋、唐四人,本朝因之,是循用周制也。《禮》「登歌下管」,貴人聲也,故《儀禮》瑟與歌工皆席於西階上。隋、唐相承,
 庭中磬虡之下,系以偶歌琴瑟,非所謂升歌貴人聲之義。今堂上琴瑟,比之周制,不啻倍蓰,而歌工止四人,音高下不相權。蓋樂有八音,所以行八風,是以舞佾與鐘磬俱用八為數。請罷庭中歌者,堂上歌為八,琴瑟之數放此,其箏、阮、築悉廢。



 太常以謂:「堂上鐘磬去之,則歌聲與宮縣遠。漢、唐以來,宮室之制浸廣,堂上益遠庭中,其上下樂節茍不相應,則繁亂而無序。況朝會之禮,起於西漢,則後世難以純用三代之制。其堂上鐘磬、庭中
 歌工與箏、築之器,從舊儀便。」遂如
 太常議。



\end{pinyinscope}