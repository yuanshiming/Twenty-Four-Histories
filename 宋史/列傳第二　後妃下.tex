\article{列傳第二 後妃下}

\begin{pinyinscope}

 神宗欽聖獻肅向皇后欽成朱皇后欽慈陳皇後林賢妃武賢妃
 哲宗昭慈孟皇后昭懷劉皇后徽宗顯恭王皇後鄭皇后王貴妃韋賢妃喬貴妃劉貴妃欽宗朱皇后高宗憲節邢皇後憲聖慈烈吳皇后潘賢妃張賢妃劉貴妃劉婉儀
 張貴妃孝宗成穆郭皇后成恭夏皇后成肅謝皇後蔡貴妃李賢妃光宗慈懿李皇后黃貴妃寧宗恭淑韓皇后恭聖仁烈楊皇后理宗謝皇后度宗全皇后楊淑妃



 神宗欽聖憲肅向皇后,河內人,故宰相敏中曾孫也。治平三年,歸於穎邸,封安國夫人。神宗即位,立為皇后。



 帝不豫,後贊宣仁後定建儲之議。哲宗立,尊為皇太后。宣仁命葺慶壽故宮以居後,後辭曰:「安有姑居西而婦處東,瀆上下之分。」不敢徙,遂以慶壽後殿為隆祐宮居之。帝將卜後及諸王納婦,後敕向族勿以女置選中。族黨有欲援例以恩換合職,及為選人求京秩者,且言有特旨,後曰:「吾族未省用此例,何庸以私情撓公法。」一不與。
 帝倉卒晏駕,獨決策迎端王。章惇異議,不能沮。



 徽宗立,請權同處分軍國事,後以長君辭。帝泣拜,移時乃聽。凡紹聖、元符以還,惇所斥逐賢大夫士,稍稍收用之。故事有如御正殿、避家諱、立誕節之類,皆不用。至聞賓召故老、寬徭息兵、愛民崇儉之舉,則喜見於色。才六月,即還政。



 明年正月崩,年五十六。帝追念不已,乃數加恩兩舅,宗良、宗回,皆位開府儀同三司,封郡王。而自敏中以上三世,亦追列王爵,非常典也。



 欽成朱皇后,開封人。父崔傑,早世;母李,更嫁朱士安。後鞠於所親任氏。熙寧初,入宮為御侍,進才人、婕妤,生哲宗及蔡王似、徐國公主,累進德妃。



 哲宗即位,尊為皇太妃。時宣仁、欽聖二太后皆居尊,故稱號未極。元祐三年,宣仁詔:《春秋》之義,「母以子貴」,其尋繹故實,務致優隆。於是輿蓋、仗衛、冠服,悉侔皇后。紹聖中,欽聖復命即合建殿,改乘車為輿,出入由宣德東門,百官上箋稱「殿下」,名所居為聖瑞宮。贈崔、任、朱三父皆至師、保。徽宗立,奉禮
 尤謹。



 崇寧元年二月薨,年五十一。追冊為皇后,上尊謚,陪葬永裕陵。



 欽慈陳皇后,開封人。幼穎悟莊重,選入掖庭,為御侍。生徽宗,進美人。帝崩,守陵殿,思顧舊恩,毀瘠骨立。左右進粥、藥,揮使去,曰:「得早侍先帝,願足矣!」未幾薨,年三十二。建中靖國元年,追冊為皇太后,上尊謚,陪葬永裕陵。



 林賢妃,南劍人,三司使特之孫,司農卿洙之女。幼選入宮,既長,遂得幸,封永嘉郡君,升美人。生燕王俁、越王偲、
 邢國公主,進婕妤。元祐五年薨。詔用一品禮葬,贈貴儀,又贈賢妃。



 武賢妃,始以選入宮。元豐五年,進才人。生吳王佖、賢和公主。歷美人、婕妤。徽宗即位,進昭儀、賢妃。大觀元年薨,乘輿臨奠,輟朝三日,謚曰惠穆。



 哲宗昭慈聖獻孟皇后,洺州人,眉州防禦使、馬軍都虞候、贈太尉元之孫女也。



 初,哲宗既長,宣仁高太后歷選世家女百餘入宮。後年十六,宣仁及欽聖向太后皆愛
 之,教以女儀。元祐七年,諭宰執:「孟氏子能執婦禮,宜正位中宮。」命學士草制。又以近世禮儀簡略,詔翰林、臺諫、給舍與禮官議冊後六禮以進。至是,命尚書左僕射呂大防攝太尉,充奉迎使,同知樞密院韓忠彥攝司徒副之;尚書左丞蘇頌攝太尉,充發策使,簽書樞密院事王巖叟攝司徒副之;尚書右丞蘇轍攝太尉,充告期使,皇叔祖同知大宗正事宗景攝宗正卿副之;皇伯祖判大宗正事高密郡王宗晟攝太尉,充納成使,翰林學士範百祿攝宗正卿副之;吏部尚書王存攝太尉,充納吉使,權戶部尚書劉奉世攝宗正卿副之;翰林學士梁燾攝太尉,充納採、問名使,御史中丞鄭雍攝宗正卿副之。帝親御文德殿冊為皇后。宣仁太后語帝曰:「得賢內助,非細事也。」進後父合門祗候在為宗儀使、榮州刺史,母王氏華原郡君。



 久之,劉婕妤有寵。紹聖三年,後朝景靈宮,訖事,就坐,諸嬪御立侍,劉獨背立簾下,後合中陳迎兒呵之,不顧,合中皆忿。冬至日,會朝欽聖太后於隆祐宮,
 後御坐未髹金飾,宮中之制,惟後得之。婕妤在他坐,有慍色,從者為易坐,制與後等。眾弗能平,因傳唱曰:「皇太后出!」後起立,劉亦起,尋各復其所,或已撤婕妤坐,遂僕於地。懟不復朝,泣訴於帝。內侍郝隨謂婕妤曰:「毋以此戚戚,願為大家早生子,此坐正當為婕妤有也。」



 會後女福慶公主疾,後有姊頗知醫,嘗已後危疾,以故出入禁掖。公主藥弗效,持道家治病符水入治。後驚曰:「姊寧知宮中禁嚴,與外間異邪?」令左右藏之;俟帝至,具言其故。
 帝曰:「此人之常情耳。」後即藝符於帝前。宮禁相汀傳,厭魅之端作矣。未幾,後養母聽宣夫人燕氏、尼法端與供奉官王堅為後禱祠。事聞,詔入內押班梁從政、管當御藥院蘇珪,即皇城司鞫之,捕逮宦者、宮妾幾三十人,搒掠備至,肢體毀折,至有斷舌者。獄成,命侍御史董敦逸覆錄,罪人過庭下,氣息僅屬,無一人能出聲者。敦逸秉筆疑未下,郝隨等以言脅之。敦逸畏禍及己,乃以奏牘上。詔廢後,出居瑤華宮,號華陽教主、玉清妙靜仙師,法名
 沖真。



 初,章惇誣宣仁後有廢立計,以後逮事宣仁,惇又陰附劉賢妃,欲請建為後,遂與郝隨構成是獄,天下冤之。敦逸奏言:「中宮之廢,事有所因,情有可察。詔下之日,天為之陰翳,是天不欲廢後也;人為之流涕,是人不欲廢後也。」且言:「嘗覆錄獄事,恐得罪天下後世。」帝曰:「敦逸不可更在言路。」曾布曰:「陛下本以皇城獄出於近習推治,故命敦逸錄問,今乃貶錄問官,何以取信中外?」乃止。帝久亦悔之,曰:「章惇誤我。」



 元符末,欽聖太后將復後位,
 適有布衣上書,以後為言者,即命以官;於是詔後還內,號元祐皇后,時劉號元符皇后故也。崇寧初,郝隨諷蔡京再廢後,昌州判官馮澥上書言後不得復。臺臣錢遹、石豫、左膚等連章論韓忠彥等信一布衣狂言,復己廢之後,以掠虛美,望斷以大義。蔡京與執政許將、溫益、趙挺之、張商英皆主其說。徽宗從之,詔依紹聖詔旨,復居瑤華宮,加賜希微元通知和妙靜仙師。



 靖康初,瑤華宮火,徙居延寧宮;又火,出居相國寺前之私第。金人圍汴,
 欽宗與近臣議再復後,尊為元祐太后。詔未下而京城陷。時六宮有位號者皆北遷,後以廢獨存。張邦昌僭位尊號為宋太后,迎居延福宮,受百官朝。胡舜陟、馬伸又言,政事當取後旨。邦昌乃復上尊號元祐皇后,迎入禁中,垂簾聽政。



 後聞康王在濟,遣尚書左右丞馮澥、李回及兄子忠厚持書奉迎。命副都指揮使郭仲荀將所部扈衛,又命御營前軍統制張俊逆於道。尋降手書,播告天下。王至南京,後遣宗室士人褭及內侍邵成章奉圭寶、
 乘輿、服御迎,王即皇帝位,改元,後以是日撤簾,尊後為元祐太后。尚書省言,「元」字犯後祖名,請易以所居宮名,遂稱隆祐太后。



 上將幸揚州,命仲荀衛太后先行,駐揚州州治。會張浚請先定六宮所居地,遂詔忠厚奉太后幸杭州,以苗傅為扈從統制。逾年,傅與劉正彥作亂,請太后聽政。又請立皇子。太后諭之曰:「自蔡京、王黼更祖宗法,童貫起邊事,致國家禍亂。今皇帝無失德,止為黃潛善、汪伯彥所誤,皆已逐矣。」傅等言必立皇太子,太后
 曰:「今強敵在外,我以婦人抱三歲小兒聽政,將何以令天下?」傅等泣請,太后力拒之。帝聞事急,詔禪位元子,太后垂簾聽政。朱勝非請令臣僚得獨對論機事,仍日引傅黨一人上殿,以釋其疑。太后從之,每見傅等,曲加慰撫,傅等皆喜。韓世忠妻梁氏在傅軍中,勝非以計脫之,太后召見,勉令世忠速來,以清巖陛。梁氏馳入世忠軍,諭太后意。世忠等遂引兵至,逆黨懼。朱勝非等誘以復闢,命王世修草狀進呈。太后喜曰:「吾責塞矣。」再以手札趣
 帝還宮,即欲撤簾。帝令勝非請太后一出御殿,乃命撤簾。是日,上皇太后尊號。



 太后聞張浚忠義,欲一見之,帝為召浚至禁中。承議郎馮楫嘗貽書苗傅勸復闢,上未之知,太后白其事,楫得遷秩。



 帝幸建康,命簽書樞密院事鄭玨衛太後繼發,比至,帝率群臣迎於郊。會防秋迫,命劉寧止制置江、浙,衛太後往洪州,百司非預軍事者悉從。仍命滕康、劉玨權知三省樞密院事從行,凡四方奏讞、吏部差注、舉闢、功賞之類,皆隸焉。復命四廂都
 指揮使楊惟忠,將兵萬人衛從。帝慮敵人來侵,密諭康、玨緩急取太后旨,便宜以行。過落星寺,舟覆,宮人溺死者十數,惟太后舟無虞。



 既至洪州,議者言:「金人自蘄、黃渡江,陸行二百餘里,即到洪州。」帝憂之,命劉光世屯江州。光世不為備,金人遂自大冶縣徑趣洪州。康、玨奉太后行,次吉州。金人追急,太后乘舟夜行。質明,至太和縣,舟人景信反,楊惟忠兵潰,失宮人一百六十,康、玨俱遁,兵衛不滿百,遂往虔州。太后及潘妃以農夫肩輿而行。
 帝慮太后徑入閩、廣,遣使歷詢後所在,及知在虔州,遂命中書舍人李正民來朝謁。



 時虔州府庫皆空,衛軍所給,惟得沙錢,市買不售,與百姓交鬥,縱火肆掠。土豪陳新率眾圍城,康、玨、惟忠弗能禁。惟忠步將胡友自外引兵破新於城下,新乃去。帝聞,罷康、玨,命盧益、李回代之。諭輔臣曰:「朕初不識太后,自迎至南京,愛朕不啻己出。今在數千里外,兵馬驚擾,當亟奉迎,以愜朕朝夕慕念之意。」遂遣御營司都統辛企宗、帶御器械潘永思迎歸。
 太后至越,帝親迎於行宮門外,遍問所過守臣治狀。



 入宮禁中,嘗微苦風眩。有宮人自言善符咒,疾良已。太后驚曰:「仁吾豈敢復聞此語耶!」立命出之。太后生辰,置酒宮中,從容謂帝曰:「宣仁太后之賢,古今母後未有其比。昔奸臣肆為謗誣,雖嘗下詔明辨,而國史尚未刪定,豈足傳信?吾意在天之靈,不無望於帝也。」帝聞之悚然。後乃更修《神宗》、《哲宗實錄》,始得其正,而奸臣情狀益著。



 帝事太后極孝,雖帷帳皆親視;或得時果,必先獻太后,然後
 敢嘗。宣教郎範燾與忠厚有憾,誣與太后密養欽宗子。帝曰:「朕於太后如母子,安得有此。」即治其罪。紹興五年春,患風疾,帝旦暮不離左右,衣弗解帶者連夕。



 四月,崩於行宮之西殿,年五十九。遺命擇地攢殯,俟軍事寧,歸葬園陵。帝詔曰:「朕以繼體之重,當從重服,凡喪祭用母后臨朝禮。」上尊號曰昭慈獻烈皇太后,推恩外家凡五十人。殯於會稽上皇村,附神主於哲宗室,位在昭懷皇后上。三年,改謚昭慈聖獻。



 後性節儉謙謹,有司月供千
 緡而止。幸南昌,斥賣私絹三千匹充費。尋詔文書應奏者避后父名,不許;群臣請上太皇太后號,亦不許。忠厚直顯謨閣,臺諫、給舍交章論列,後聞,即令易武,命學士院降詔,戒敕忠厚等不得預聞朝政、通貴近、至私第謁見宰執。以恩澤當得官者近八十員,後未嘗陳請。



 初,後受冊日,宣仁太后嘆曰:「斯人賢淑,惜福薄耳!異日國有事變,必此人當之。」後皆如所云。



 昭懷劉皇后,初為御侍,明艷冠後庭,且多才藝。由美人、
 婕妤進賢妃。生一子二女。有盛寵,能順意奉兩宮。時孟後位中宮,後不循列妾禮,且陰造奇語以售謗,內侍郝隨、劉友端為之用。孟后既廢,後竟代焉。右正言鄒浩上疏極諫,坐竄。徽宗立,冊為元符皇后。明年,尊為太后,名宮崇恩。帝緣哲宗故,曲加恩禮,後以是頗干預外事,且以不謹聞。帝與輔臣議,將廢之,而後已為左右所逼,即簾鉤自縊而崩,年三十五。



 徽宗顯恭王皇后,開封人,德州刺史藻之女也。元符二
 年六月,歸於端邸,封順國夫人。徽宗即位,冊為皇后。生欽宗及崇國公主。後性恭儉,鄭、王二妃方亢寵,後待之均平。巨閹妄意迎合,誣以暗昧。帝命刑部侍郎周鼎即秘獄參驗,略無一跡,獄止。後見帝,未嘗一語輒及,帝幡然憐之。大觀二年崩,年二十五。謚曰靜和,葬裕陵之次。紹興中,始附微宗廟室,改上今謚雲。



 鄭皇后,開封人也。父紳,始為直省官,以後貴,累封太師、樂平郡王。



 後本欽聖殿押班,徽宗為端王,每日朝慈德
 宮,欽聖命鄭、王二押班供侍。及即位,遂以二人賜之。後自入宮,好觀書,章奏能自制,帝愛其才。崇寧初,封賢妃,迂貴妃有異寵。徽宗多賚以詞章,天下歌之。



 王皇后崩,政和元年,立為皇后。將受冊,有司創制冠服,後言國用未足,冠珠費多,請命工改制妃時舊冠。又乞罷黃麾仗、小駕鹵簿等儀,從之。恩澤皆弗陳請。時族子居中在樞府,後奏:「外戚不當預國政,必欲用之,且令充妃職。」帝為罷居中。居中復用,後歸寧還言:「居中與父紳相往還,人
 皆言其招權市賄,乞禁絕,許御史奏劾。」後性端謹,善順承帝意。劉貴妃薨,帝思之不已,將追冊為後。後即奏妃乃其養子,乞別議褒崇之禮,帝大喜。



 欽宗受禪,尊為太上皇后,遷居寧德宮,稱寧德太后。從上皇幸南京,金師退,先歸。時用事者言,上皇將復闢於鎮江,人情危駭。或謂後將由端門直入禁中,內侍輩頗勸欽宗嚴備。帝不從,出郊迎後,於是兩宮歡甚洽。上皇聞之,即罷如洛之議。



 汴京破,從上皇幸青城。北遷,留五年,崩於五國城,年
 五十二。紹興七年,何蘇等使還,始知上皇及后崩,高宗大慟。詔立重成服,謚顯肅。後親族各遷官有差。祔主徽宗室,以聞哀日為大忌。梓宮歸,入境,承之以槨,納翬衣其中,與徽宗各攢於會稽永祐陵。



 先是,後至金營,訴於粘罕曰:「妾得罪當行,但妾家屬不預朝政,乞留不遣。」粘罕許之,故紳得歸。後既行,紳亦以是年薨,謚僖靖。家屬流寓江南,高宗憐之,詔所在尋訪賜官。有鄭藻者,後近屬也。紹興中帶御器械用後祔廟恩,拜隴州防禦使;凡
 四使金,歷官至保信軍節度使,加太尉。卒,追封榮國公,謚端靖。



 王貴妃,與鄭后俱為押班。徽宗立,封平昌郡君,進位至貴妃。生鄆王楷、莘王植、陳王機、惠淑康淑順德柔福沖懿帝姬。政和七年九月薨,謚曰懿肅。



 韋賢妃,開封人,高宗母也。初入宮,為侍御。崇寧末,封平昌郡君。大觀初,進婕妤,累遷婉容。高宗在康邸出使,進封龍德宮賢妃。從上皇北遷。建炎改元,遙尊為宣和皇
 后。封其父安道為郡王,官親屬三十人。由是遣使不絕。



 紹興七年,徽宗及鄭皇后崩聞至,帝號慟,諭輔臣曰:「宣和皇后春秋高,朕思之不遑寧處,屈己請和,正為此耳。」翰林學士朱震引唐建中故事,請遙尊為皇太后,從之。已而太常少卿吳表臣請依嘉祐、治平故事,俟三年喪畢,然後舉行。乃先降御札,播告天下。後三代俱追封王。



 帝以後久未歸,每顰蹙曰:「金人若從朕請,餘皆非所問也。」王倫使回,言金人許歸後。未幾,金人遣蕭哲來,亦言
 後將歸狀。遂豫作慈寧宮,命莫將、韓恕為奉迎使。十年,以金人猶未歸後,乃遙上皇太后冊寶於慈寧殿。是後,生辰、至、朔,皆遙行賀禮。



 洪皓在燕,求得後書,遣李微持歸。帝大喜曰:「遣使百輩,不如一書。」遂加微官。金人遣蕭毅、邢具瞻來議和,帝曰:「『朕有天下,而養不及親。徽宗無及矣!今立誓信,當明言歸我太后,朕不恥和,不然,朕不憚用兵。」毅等還,帝又語之曰:「太后果還,自當謹守誓約;如其未也,雖有誓約,徒為虛文。」



 命何鑄、曹勛報謝,召至
 內殿,諭之曰:「朕北望庭闈,無淚可揮。卿見金主,當曰:『慈親之在上國,一老人耳;在本國,則所系甚重。』以至誠說之,庶彼有感動。」鑄等至金國,首以後歸為請。金主曰:「先朝業已如此,豈可輒改?」勛再三懇請,金主始允。鑄等就館,館伴耶律紹文來信,金主許從所請。洪皓聞之,先遣人來報。鑄等還,具言其實。遂命參政王次翁為奉迎使。金人遣其臣高居安、完顏宗賢等扈從以行。



 十二年四月,次燕山,自東平舟行,由清河至楚州。既渡淮,命太后
 弟安樂郡王韋淵、秦魯國大長公主、吳國長公主迎於道。帝親至臨平奉迎,普安郡王、宰執、兩省、三衙管軍皆從。帝初見太后,喜極而泣。八月,至臨安,入居慈寧宮。



 先是,以梓宮未還,詔中外輟樂。至是,慶太后壽節,始用樂。謁家廟,親屬遷官幾二千人。



 太后聰明有智慮。初,金人許還三梓宮,太后恐其反復,呼役者畢集,然後起攢。時方暑,金人憚行,太后慮有他變,乃陽稱疾,須秋涼進發。已而稱貸於金使,得黃金三千兩以犒其眾,由是途中
 無間言。太后在北方,聞韓世忠名,次臨平,呼世忠至簾前慰勞。還宮,帝侍太后,或至夜分未去,太后曰:「且休矣,聽朝宜早,恐妨萬幾。」又嘗謂:「兩宮給使,宜令通用;不然,則有彼我之分,而佞人間言易以入也。」



 時皇后未立,太后屢為帝言,帝請降手書,太后曰:「我但知家事,外庭非所當預。」將行冊命,承平典禮,悉能記之。帝先意承志,惟恐不及,或一食稍減,輒不勝憂懼。常戒宮人曰:「太后年已六十,惟優游無事,起居適意,即壽考康寧;事有所闕,
 懼母令太后知,第來白朕。」



 十九年,太后年七十,正月朔,即宮中行慶壽禮,親屬各遷官一等。太后微恙,累月不出殿門,會牡丹盛開,帝入白,太后欣然步至花所,因留宴,竟日盡歡。忌日,以諭宰執。後苦目疾,募得醫皇甫坦,治即愈。



 二十九年,太后壽登八十,復行慶禮。親屬進官一等;庶人等九十、宗子女若貢士已上父母年八十者,悉官封之。九月,得疾,上不視朝,敕輔臣祈禱天地、宗廟、社稷,赦天下,減租稅。俄崩於慈寧宮,謚曰顯仁。攢於永
 祐陵之西,祔神主太廟徽宗室。親屬進秩者十四人,授官者三人。



 太后性節儉,有司進金唾壺,太後易,令用塗金。宮中賜予不過三數千,所得供進財帛,多積於庫。至是,喪葬之費,皆仰給焉。然好佛、老。初,高宗出使,有小妾言,見四金甲人執刀劍以衛。太后曰:「我祠四聖謹甚,必其陰助。」既北遷,常設祭;及歸,立祠西湖上。



 喬貴妃,初與高宗母韋妃俱侍鄭皇后,結為姊妹,約先貴者毋相忘。既而貴妃得幸徽宗,遂引韋氏,二人愈相
 得。二帝北遷,貴妃與韋氏俱。至是,韋妃將還,貴妃以金五十兩贈高居安,曰:「薄物不足為禮,願好護送姊還江南。」復舉酒酌韋氏曰:「姊善重保護,歸即為皇太后;妹無還期,終死於朔漠矣!」遂大慟以別。



 劉貴妃,其出單微。入宮,即大幸,由才人七遷至貴妃。生濟陽郡王棫、祁王模、信王榛。政和三年秋,薨。



 先是,妃手植芭蕉於庭曰:「是物長,吾不及見矣!」已而果然。左右奔千告帝,帝初以其微疾,不經意,趣幸之,已薨矣,始大悲惻。
 特加緊四字謚曰明達懿文。敘其平生,弦諸樂府。又欲踵溫成故事追崇,使皇后表請,因冊贈為後,而以明達謚焉。



 時又有安妃劉氏者,本酒保家女。初事崇轉宮,宮罷,出居宦者何聽家。內侍楊戩譽其美,復召入。妃以同姓養為女,遂有寵,為才人,進至淑妃。生建安郡王柍、嘉國公椅、英國公楒、和福帝姬。政和四年,加貴妃。朝夕得侍上,擅愛顓席,嬪御為之稀進。擢其父劉宗元節度使。



 妃天資警司,解迎意合旨,雅善塗飾,每制一服,外間即效
 之。林靈素以技進,目為九華玉真安妃,省其像於神霄帝君之左。宣和三年薨,年三十四。初謚明節和文,旋用明達近比,加冊贈為皇后,葬其園之北隅。帝悼之甚,後宮皆往唁,帝相與啜泣。崔妃獨左視無戚容,帝悲怒,疑其為厭蠱。卜者劉康孫緣妃以進,喜妄談休咎,捕送開封獄。醫曹孝忠竺疾無狀,閣內待王堯臣坐盜金珠及出金明池游宴事,人並鞫治。獄成,同日誅死。遂廢崔妃為庶人。崔生王椿及帝姬五人云。



 欽宗朱皇后,開封祥符人。父伯材,武康軍節度使。欽宗在東宮,徽宗臨軒備禮,冊為皇太子妃。欽宗即位,立為皇后。追封伯材為恩平郡王。後既北遷,不知崩聞。慶元三年上尊號,謚仁懷,祔於太廟欽宗室,推恩後家十五人。五年,奉安神御於景靈宮。



 兄二人:孝孫,靖康中以節鉞換授右金吾衛上將軍,卒贈開府含義同三同;孝章,一日孝莊,官至永慶軍承宣使,卒贈昭化軍節度使。



 高宗憲節邢皇后,開封祥符人。父煥,朝請郎。高宗居康
 邸,以歸聘之,封嘉國夫人。王出使,無人留居蕃衍宅。金人犯京師,夫人從三宮北遷。上皇遣曹勛歸,夫人脫所御金環,使內侍持付勛曰:「幸為吾白大王,願如此環,得早相見也。」王憐之。及即位,遙冊為皇后,宮後親屬二十五人。



 紹興九年,後崩於五國城,年三十四。金人秘之,高宗虛中宮以待者十六年。顯仁太后回鑾,始得崩聞。上為輟朝,行釋服之祭,謚懿節,祔主於別廟。



 紹興十二年八月,後梓宮至,攢於聖獻太后梓宮之西北。帝思後,殊
 慘不樂,皇后吳氏知帝意,乃請為其侄珣、琚婚邢氏二女,以慰帝心。淳熙末,改謚憲節,祔高宗廟。



 憲聖慈烈吳皇后,開封人。父近,以後貴,累官武翼郎,贈太師,追封吳王,謚宣靖。



 近嘗夢至一亭,扁曰「侍康」;傍植芍藥,獨放一花,殊妍麗可愛,花下白羊一,近寤而異之。後以乙未歲生,方產時,紅光徹戶外。年十四,高宗為康王,被選入宮,人謂「侍康」之徵。



 王即帝位,後常以戎服侍左右。後頗知書,從幸四明,衛士謀為變,入問帝所在,後
 紿之以免。未幾,帝航海,有魚躍入御舟,後曰:「此周人白魚之祥也。」帝大悅,封和義郡夫人。還越,進封才人。後益博習書史,又善翰墨,由是寵遇日至,與張氏並為婉儀,尋進貴妃。



 顯仁太后回鑾,亦愛後。憲節皇后崩聞至,秦檜等累表請立中宮,太后亦為言。紹興十三年,詔立貴妃為皇后。帝御文德殿授冊,後即穆清殿廷受之。追王三代,親屬由後官者三十五人。



 顯仁太后性嚴肅,後身承起居,順適其意。嘗繪《古列女圖》,置坐中為鑒;又取《詩
 序》之義,扁其堂曰「賢志」。



 初,伯琮以宗子召入宮,命張氏育之。後時為才人,亦請得育一子,於是得伯玖,更名璩。中外議頗籍籍。張氏卒,並育於後,後視之無間。伯琮性恭儉,喜讀書,帝與後皆愛之,封普安郡王。後嘗語帝曰:「普安,其天日之表也。」帝意決,立為皇子,封建王。出璩居紹興。



 高宗內禪,手詔後稱太上皇后,遷居德壽宮。孝宗即位,上尊號曰壽聖太上皇后。月朔,朝上皇畢,入見後如宮中儀。乾道七年,加號壽聖明慈。淳熙二年,以上皇
 行慶壽禮,復加壽聖齊明廣慈之號。十年,以後年七十,親屬推恩有差。十二年,加尊號曰備德。上皇崩,遺詔改稱皇太后。帝欲迎還大內,太后以上皇幾筵在德壽宮,不忍舍去,因名所御殿曰慈福,居焉。光宗即位,更號壽聖皇太后,以壽皇故,不稱太皇太后也。帝嘗言及用人,後「宜崇尚舊臣。」紹熙四年,後壽八十,帝乃覲後,奉冊禮,加尊號曰隆慈備福。五年正月,帝率群臣行慶壽禮,嘉王侍側,後勉以讀書辨邪正、立綱常為先。夏,孝宗崩,
 始正太皇太后之號。



 時光宗疾未平,不能執喪,宰臣請垂簾主喪事,後不可。已而宰執請如唐肅宗故事,群臣發喪太極殿,成服禁中,許之。後代行祭尊禮。尋用樞密趙汝愚請,於梓宮前垂簾,宣光宗手詔,立皇子嘉王為皇帝。翌日,冊夫人韓氏為皇后,撤簾。慶元元年,加號光祐,遷居重華宮。汝愚後以謫死,中書舍人汪義端目汝愚為李林甫,欲並逐其黨,太后聞而非之。



 三年十月,後寢疾,詔禱天地、宗廟、社稷,大赦天下,逾月而崩,年八十
 三。遺誥:「太上皇帝疾未痊愈,宜於宮中承重;皇帝服齊衰五月,以日易月。」詔服期年喪。謚曰憲聖慈烈,攢祔於永思陵。



 潘賢妃,開封人,元懿太子母也。父永壽,直翰林醫局官。高宗居康邸時納之,邢後北遷,妃未有位號,帝即位,將立為後,呂好問諫止之,立為賢妃。太子薨,從隆祐太后於江西,逾年還。紹興十八年薨。永壽,贈太子少師。



 張賢妃,開封人。建炎初,為才人,有寵,進婕妤。帝欲擇宗
 室子養禁中,輔臣問帝以宮中可付托者誰耶?帝曰:「已得之矣。」意在婕妤。已而伯琮入宮,年尚幼,婕妤與潘賢妃、吳才人方環坐,以觀其所向。時賢妃新失皇子,意忽忽不樂,婕妤手招之,遂向婕妤。帝因命婕妤母之,是為孝宗。尋遷婉儀,十二年卒,上為輟朝二日,贈賢妃。弟萃,閣門宣贊舍人,妃薨,遷秩二階。



 劉賢妃,臨安人。入宮為紅霞帔,遷才人,累遷婕妤、婉容,紹興二十四年進賢妃。頗恃寵驕侈,嘗因盛夏以水晶
 飾腳踏,帝見之,命取為枕,妃懼,撤去之。淳熙十四年薨。



 父懋,累官昭慶軍節度使。金人南侵,獻錢二萬緡以助軍興費。懋子允升,紹興末為和州防禦使、知閣門事。奉使還,遷蘄州防物使、福州觀察使。



 劉婉儀,初入宮,封宜春郡夫人。尋進才人,與劉婉容俱被寵,進婉儀。婉儀頗恃恩招,嘗遣人諷廣州蕃商獻明珠香藥,許以官爵。舶官林孝澤言於朝,詔止其獻。金人將叛盟,劉錡主戰,幸醫王繼先從中沮之,因謀誅錡,
 帝不懌。一日,在婉儀位,有憂色。婉儀陰訪得其言,以寬譬帝意。帝怪與繼先言合,詰之,婉儀急,具以實對。帝大怒,托以他過廢之。兄伉,累官和州防禦使、知閣門事,婉儀既廢,乃與祠罷歸。



 張貴妃,開封祥符人。初入宮,封永嘉郡夫人。乾道六年,進婉容。淳熙七年,封太上皇淑妃。十六年,進貴妃。紹熙元年薨。



 美人馮氏,才人韓氏、吳氏、李氏、王氏俱被寵幸,後皆廢。吳氏,中宮近屬也,紹興三十年,復故封。李氏、王
 氏俱明艷,淳熙末,上皇愛之。及崩,憲聖後見二才人,每感憤,孝宗即追告命,許自便。蓋非常制云。



 孝宗成穆郭皇后,開封祥符人。奉直大夫直卿之女孫,其六世祖為章穆皇后外家。孝宗為普安郡王時納郭氏,封咸寧郡夫人。生光守及莊文太子、魏惠憲王愷、邵悼肅王恪。紹興二十六年薨,年三十一,追封淑國夫人。三十一年,用明堂恩,贈福國夫人。既建太子,追封皇太子妃。及受禪,追冊為皇后,謚恭懷,尋改安穆。及營阜
 陵,又改成穆,祔孝宗廟。



 父瑊,累官昭慶軍承宣使,追封榮王。孝宗待郭氏恩禮彌厚,然不假外戚以官爵。後弟師禹、師元,官不過承宣使,師元不及建節而卒。將內禪,師禹始除節度使。光宗朝,官至太保,封永寧郡王。



 成恭夏皇后,袁州宜春人。曾祖令吉,為吉水簿。夏氏初入宮,為憲聖太後閣中侍御。普安郡王夫人郭氏薨,太后以夏氏賜王,封齊安郡夫人。即位,進賢妃。逾年,奉上皇命,立為皇后。乾道二年,謁家廟,親屬推恩十一人。三
 年崩,謚安恭。寧宗時,改謚成恭。



 初,後之生也,有異光穿室,父協奇之,及長,以姿納宮中。久之,父居益困,及歸,客袁之僧舍,號夏翁。翁亡,後始貴。訪得其弟執中,補承信郎、閣門祗候。未幾,遷右武郎、閣門宣贊舍人,累遷奉國軍節度使,提舉萬壽觀。寧宗即位,加少保。逾年,卒於家。



 初,執中與其微時妻至京,宮人諷使出之,擇配貴族,欲以媚後,執中弗為動。他日,後親為言,執中誦宋弘語以對,後不能奪。既貴,始從師學,作大字頗工,復善騎射。高
 宗行慶壽禮,近戚爭獻珍環,執中獨大書「一人有慶,萬壽無疆」以獻。高宗喜,錫賚甚渥。嘗為館伴副使,連射皆命中,金人駭服。孝宗聞其才,將召用之,謝曰:「他日無累陛下,保全足矣。」人以此益賢之。



 成肅謝皇后,丹陽人。幼孤,鞠於翟氏,因冒姓焉。及長,被選入宮。憲聖太后以賜普安郡王,封咸安郡夫人。王即位,進婉容。逾年,進貴妃。



 成恭皇后崩,中宮虛位。淳熙三年,妃侍帝,過德壽宮,上皇諭以立後意。尋遣張去為傳
 旨,立貴妃為皇后,復姓謝氏。親屬推恩者十人。光宗受禪,上尊號壽成皇后。孝宗崩,尊為皇太后。慶元初,加號惠慈。嘉泰二年,加慈祐太皇太后。三年崩,謚成肅,攢錡於永阜陵。



 後性儉慈,減膳羊,每食必先以進御。服汗濯衣,有數年不易者。弟淵,以後貴,授武翼郎。後嘗戒之曰:「主上化行恭儉,吾亦躬服汗濯,爾宜崇謙抑,遠驕侈。」後歷閣門宣贊舍人、帶御器械。光宗朝,遷果州團練使。寧宗立,轉萊州防禦使,擢知閣門事,仍幹辦皇城司。三遷
 至保信軍節度使,尋加太尉、開府儀同三司。成肅皇后崩,遺誥賜淵錢十萬緡、金二千兩、田十頃,僦緡日十千。後累升三少,封和國公。嘉定四年薨,贈太保。



 蔡貴妃,初入宮,為紅霞帔,封和義郡夫人,進婉容。淳熙十年冬,拜貴妃。十二年秋薨。父滂,宜春觀察使。



 李賢妃,初入宮,為典字,轉通義郡夫人,進婕妤。淳熙十年卒,贈賢妃。時李燾在經筵,嘗諫省後宮費。帝曰:「朕老矣,安有是?近葬李妃用三萬緡耳。」帝雖在位久,後宮寵
 幸,無著聞者。



 光宗慈懿李皇后,安陽人,慶遠軍節度使、贈太尉道之中女。初,後生,有黑鳳集道營前石上,道心異之,遂字後曰鳳娘。道帥湖北,聞道士皇甫坦善相人,乃出諸女拜坦。坦見後,驚不敢受拜,曰:「此女當母天下。」坦言於高宗,遂聘為恭王妃,封榮國夫人,進定國夫人。乾道四年,生嘉王。七年,立為皇太子妃。



 性妒悍,嘗訴太子左右於高、孝二宮,高宗不懌,謂吳後曰:「是婦將種,吾為皇甫坦所
 誤。」孝宗亦屢訓後:「宜以皇太后為法,不然,行當廢汝。」後疑其說出於太后。



 及太子即位,冊為皇后。光宗欲誅宦者,近習皆懼,遂謀離間三宮。會帝得習疾,孝宗購得良藥,欲因帝至宮授之。宦者遂訴於後曰:「太上合藥一大丸,俟宮車過即投藥。萬一有不虞,其奈宗社何?」後覘藥實有。心銜之。頃之,內宴,後請立嘉王為太子,孝宗不許。後曰:「妾六禮所聘,嘉王,妾親生也,何為不可?」孝宗大怒。後退,持嘉王泣訴於帝,謂壽皇有廢立意。帝惑之,遂不
 朝太上。



 帝嘗宮中浣手,睹宮人手白,悅之。他日,後遣人送食合於帝,啟之,則宮人兩手也。又黃妃有寵,因帝親郊,宿齋宮,後殺之,以暴卒聞。是夕風雨大作,黃壇燭盡滅,不能成禮。帝疾由是益增劇,不視朝,政事多決於後矣。後益驕奢,封三代為王,家廟逾制,衛兵多於太廟。後歸謁家廟,推恩親屬二十六人、使臣一百七十二人,下至李氏門客,亦奏補官。中興以來未有也。



 是時,帝久不朝太上,中外疑駭。紹熙四年九月重明節,宰執、侍從,
 臺諫連章請帝過宮。給事中謝深甫言:「父子至親,天理昭然。太上之愛陛下,亦猶陛下之愛嘉王。太上春秋高,千秋萬歲後,陛下何以見天下?」帝感悟,趣命駕朝重華宮。是日,百官班列俟帝出,至御屏,後挽留帝入,曰:「天寒,官家且飲酒。」百僚、侍衛相顧莫敢言。中書舍人陳傅良引帝裾請毋入,因至屏後,後叱曰:「此何地,爾秀才欲所頭邪?」傅良下殿慟哭,後復使人問曰:「此何理也。」傅良曰:「子諫父不聽,則號泣而隨之。」後益怒,遂傳旨罷還宮。其
 後孝宗崩,帝不能親執喪。



 宰相趙汝愚謀內禪,立寧宗,尊后曰太上皇后,上尊號曰壽仁。慶元六年崩,年五十六,謚慈懿。



 黃貴妃,淳熙末在德壽宮,封和義郡夫人。光宗為皇太子,傍無侍姬,上皇以夫人賜之,遂專寵。即位,拜貴妃。紹熙二年冬十一月,為皇后李氏所殺。帝聞而成疾。又有張貴妃,亦舊侍東宮,次婕妤符氏,後出嫁於民間。



 寧宗恭淑韓皇后,相州人,其六世祖為忠獻王琦。初,後
 與姊俱被選入宮,後能順適兩宮意,遂歸平陽郡邸,封新安郡夫人,進崇國夫人。王受禪,冊夫人為皇后。後父同卿,由知泰州升揚州觀察使;母莊氏,封安國夫人。



 慶元六年崩,謚恭淑。同卿累遷慶遠軍節度使,加太尉。慶元五年卒,贈太師,謚恭靖。



 同卿季父侂冑,自以有定策功,聲勢熏灼。同卿每懼滿盈,不敢干政。時天下皆知侂冑為後族,不知同卿乃後父也。同卿沒一年而後崩,侂冑竟敗,人始服其善遠權熱去。同卿子俟,後兄也,官至
 承宣使。



 恭聖仁烈楊皇后,少以姿容選入宮,忘其姓氏,或云會稽人。慶元元年三月,封平樂郡夫人。三年四月,進封婕妤。有楊次山者,亦會稽人,後自謂其兄也,遂姓楊氏。



 五年,進婉儀。六年,進貴妃。恭淑皇后崩,中宮未有所屬,貴妃與曹美人俱有寵。韓侂冑見妃任權術,而曹美人性柔順,勸帝立曹。而貴妃頗涉書史,知古今,性復機警,帝竟立之。



 次山客王夢龍知其謀,密以告後,後深銜之,與
 次山欲因事誅侂冑。會侂冑議用兵中原,俾皇子□嚴入奏:「侂冑再起兵端,將不利於社稷。」帝不答。後從傍贊之甚力,亦不答。恐事洩,俾次山擇廷臣可任者,與共圖之。禮部侍郎史彌遠,素與侂冑有隙,遂欣然奉命。參知政事錢象祖,嘗諫用兵貶信州,彌遠乃先告之。禮部尚書衛涇、著作郎王居安、前右司郎官張鎡皆預其謀。開禧三年十一月三日,侂冑方早朝,彌遠密遣中軍統制夏震伏兵六部橋側,率健卒擁侂冑至玉津園,槌殺之。復
 命彌遠。像祖等俱赴延和殿,以殛侂冑聞,帝不之信,越三日,帝猶謂其未死。蓋是謀悉出中宮及次山等,帝初不知也。



 後既誅侂冑,彌遠日益貴用事。嘉定十四年,帝以國嗣未定,養宗室子貴和,立為皇子,賜名竑。彌遠為丞相,既信任於後,遂專國政,竑漸不能平。初,竑好琴,彌遠買美人善琴者納之,而私厚美人家,令伺皇子動靜。竑嬖之,一日,竑指輿地圖標美人曰:「此瓊崖州也,他日必置史彌遠於此地。」美人以告彌遠。竑又書字於幾曰:「
 彌遠當決配八千里。」竑左右皆彌遠腹心,走白彌遠。彌遠大懼,陰蓄異志,欲立他宗室子昀為皇子,遂陰與昀通。



 十七年閏八月丁酉,帝大漸,彌遠夜召昀入宮,後尚未知也。彌遠遣後兄子穀及石以廢立事白後,後不可曰:「皇子先帝所立,豈敢擅變?」是夜,凡七往反,後終不聽。穀等乃拜泣曰:「內外軍民皆已歸心,茍不立之,禍變必生,則楊氏無唯類矣。」後默然良久,曰:「其人安在?」彌遠等召昀入,後拊其背曰:「汝今為吾子矣!」遂矯詔廢竑為濟
 王,立昀為皇子,即帝位,尊皇后曰皇太后,同聽政。



 寶慶二年十一月戊寅,加尊號壽明。紹定元年正月丙子,復加慈睿。四年正月,後壽七十,帝率百官朝慈明殿,加尊號壽明仁福慈睿皇太后。十二月辛巳,後不豫,詔禱祠天地、宗廟、社稷、宮觀,赦天下。五年十二月壬午,崩於慈明殿。壽七十有一,謚恭聖仁烈。



 次山官至少保,封永陽郡王。次山二子:谷封新安郡王,石永寧郡王。自有傳。侄孫鎮,尚理宗女周漢公主,官至左領軍衛將軍、附馬都
 統。宗族鳳孫等,皆任通顯云。



 理宗謝皇后,諱道清,天臺人。父渠伯,祖深甫。後生而黧黑,瞖一目。渠伯早卒,家產益破壞。後嘗躬親汲飪。



 初,深甫為相,有援立楊太后功,太后德之。理宗即位,議擇中宮,太后命選謝氏諸女。後獨在室,兄弟欲納入宮,諸父舉伯不可,曰:「即奉詔納女,當厚奉資裝,異時不過一老宮婢,事奚益?」會元夕,縣有鵲來巢燈山,眾以為后妃之祥。舉伯不能止,乃供送後就道。後旋病疹,良已,膚蛻,瑩
 白如玉;醫又藥去目瞖。時賈涉女有殊色,同在選中。及入宮,理宗意欲立賈。太后曰:「謝女端重有福,宜正中宮。」左右亦皆竊語曰:「不立真皇后,乃立假皇后邪!」帝不能奪,遂定立後。初封通義郡夫人,寶慶三年九月,進貴妃,十二月,冊為皇后。



 後既立,賈貴妃專寵;貴妃薨,閻貴妃又以色進。後處之裕如,略不介懷。太后深賢之,而帝禮遇益加焉。開慶初,大元兵渡江,理宗議遷都平江、慶元,後諫不可,恐搖動民心,乃止。



 理宗崩,度宗立。咸淳三年,
 尊為皇太后,號壽和聖福。進封三代:父渠伯,魏王;祖深甫、曾祖景之,皆魯王。宗族男女各進秩賜封賞賚有差。度宗崩,瀛國公即位,尊為太皇太后。太后年老且疾,大臣屢請垂簾同聽政,強之乃許。加封五代。



 太后以兵興費繁,痛自裁節,汰慈元殿提舉已下官,省泛索錢緡月萬。平章賈似道兵潰,陳宜中上疏請正其罪。太后曰:「似道勤勞三朝,豈宜以一旦罪而失遇大臣禮?」先削其官,後乃置法貶死。



 京朝官聞難,往往避匿遁去。太后命揭
 榜朝堂曰:「我國家三百年,待士大夫不薄。吾與嗣君遭家多難,爾小大臣不能出一策以救時艱,內則畔官離次,外則委印棄城,避難偷生,尚何人為?亦何以見先帝於地下乎?天命未改,國法尚存。凡在官守者,尚書省即與轉一次;負國逃者,御史覺察以聞。」



 德祐元年六月朔,日食既,太后削「聖福」以應天變。丞相王□龠老病,陳宜中、留夢炎庸懦無所長,日坐朝堂相爭戾。而張世傑兵敗於焦山,宜中棄官去。太後累召不至,遺書宜中母,使勉
 之。十月,始還朝。太后又親為書召夏貴等兵,曰:「吾母子不足念,獨不報先帝德乎?」貴等亦罕有至者。



 是月,大元兵破常州,太后遣陸秀夫等請和,不從。宜中即率公卿請遷都,太后不許,宜中痛哭固請,不得已從之。明日當啟行,而宜中倉卒失奏,於是宮車已駕,日且暮而宜中不至,太后怒而止。明年正月,更命宜中使軍中,約用臣禮。宜中難之,太后涕泣曰:「茍存社稷,臣,非所較也。」未幾,大元兵薄皋亭山,宜中宵遁,文武百官亦潛相引去。



 二
 月辛丑,大軍駐錢塘,宋亡。瀛國公與全後入朝,太后以疾留杭。是年八月,至京師,降封壽眷郡夫人。越七年終,年七十四,無子。



 兄奕,宋時封郡王。侄堂,兩浙鎮撫大使,尚榮郡公主;暨、□並節度使,端平初,頗干國政云。



 度宗全皇后,會稽人,理宗母慈憲夫人侄孫女也。略涉書史,幼從父昭孫知嶽州。開慶初,秩滿歸,道潭州。時大元兵自羅鬼入破全、衡、永、桂,圍潭州,人有見神人衛城者,已而潭獨不下。逾年事平,至臨安。



 會忠王議納妃。初,
 丁大全請選知臨安府顧巖女,已致聘矣;大全敗,巖亦罷去。臺臣論巖大全黨,宜別選名族以配太子。臣僚遂言全氏侍其父昭孫,往返江湖,備嘗艱險;其處貴富,必能盡警戒相成之道。理宗以母慈憲故,乃詔後入宮,問曰:「爾父昭孫,昔在寶祐間沒於王事,每念之,令人可哀。」後對曰:「妾父可念,淮、湖之民尤可念也。」帝深異之,詔大臣曰:「全氏女言辭甚令,宜配塚嫡,以承祭祀。」



 景定二年十一月,詔封永嘉郡夫人。十二月,冊為皇太子妃。弟永
 堅等補承信郎、直秘閣。



 度宗立,咸淳三年正月,冊為皇后。追贈三代,賜家廟、第宅。弟清夫、庭輝等一十五人,各轉一官。五年三月,後歸寧,推恩姻族五十六人,進一秩。咸平郡夫人全氏三十二人,各特封有差。



 後生子不育,次生瀛國公。十年,度宗崩,瀛國公立,冊為皇太后。宋亡,從瀛國公入朝於燕京。後為尼正智寺而終。



 楊淑妃,初選入宮為美人。咸淳三年,進封淑妃。推恩親屬幼節等三十四人進秩有差。生建國公是。宋亡,是走
 溫州,又走福州。眾推為主,冊妃為太后;封弟昺衛王。昺,修容俞氏所生也。



 至元十四年,大軍圍是於海上。明年四月,是卒,昺代立。十六年春二月,昺投海死,妃聞之大慟,曰:「我艱關忍死者,正為趙氏祭祀尚有可望爾,今天命至此,夫復何言!」遂赴海死。其將張世傑葬之海濱



\end{pinyinscope}