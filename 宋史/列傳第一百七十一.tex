\article{列傳第一百七十一}

\begin{pinyinscope}

 ○孟珙杜杲子庶王登楊掞張惟孝陳咸



 孟珙,字璞玉,隨州棗陽人。四世祖安,嘗從岳飛軍中有功。嘉定十年,金人犯襄陽,駐團山,父宗政時為趙方將,
 以兵御之。珙料其必窺樊城,獻策宗政由羅家渡濟河,宗政然之。越翼日,諸軍臨渡布陣,金人果至,半渡伏發,殲其半。宗政被檄援棗陽,臨陣嘗父子相失,珙望敵騎中有素袍白馬者,曰:「吾父也。」急麾騎軍突陣,遂脫宗政。以功補進勇副尉。



 十二年,完顏訛可步騎二十萬分兩路攻棗陽,環集城下,珙登城射之,將士驚服。宗政命珙取它道劫金人,破砦十有八,斬首千餘級,大俘軍器以歸,金人遁,以功升下班祗應。



 十四年,入謁制置使趙方,
 一見奇之,闢光化尉,轉進武校尉。十六年,以功特授承信郎。丁父憂,制置使起復之,珙辭,訖葬趣就職,又辭,轉成忠郎。理宗即位,特授忠翊郎,尋差峽州兵馬監押兼在城巡檢,京湖制置司差提督虎翼突騎軍馬,又闢京西第五副將,權管神勁左右軍統制。



 初,宗政招唐、鄧、蔡壯士二萬餘人,號「忠順軍」,命江海總之,眾不安,制置司以珙代海,珙分其軍為三,眾乃帖然。紹定元年,珙白制置司創平堰於棗陽,自城至軍西十八里,由八疊河經
 漸水側,水跨九阜,建通天槽八十有三丈,溉田十萬頃,立十莊三轄,使軍民分屯,是年收十五萬石。又命忠順軍家自畜馬,官給芻粟,馬益蕃息。二年,升京西第五正將、棗陽軍總轄,本軍屯駐忠順三軍。明年,差京西兵馬都監。丁母憂。又明年,起復京西兵馬鈐轄、棗陽軍駐札,仍總三軍。



 六年,大元將那顏倴盞追金主完顏守緒,逼蔡,檄珙戍鄂,討金唐、鄧行省武仙。仙時與武天錫及鄧守移剌瑗相掎角,為金盡力,欲迎守緒入蜀,犯光化,鋒
 剽甚。天錫者,鄧之農夫,乘亂聚眾二十萬為邊患。珙逼其壘,一鼓拔之,壯士張子良斬天錫首以獻。是役獲首五千級,俘其將士四百餘人,戶十二萬二十有奇,乃授江陵府副都統制,賜金帶。



 制置司檄珙問邊事,珙曰:「金人若向呂堰,則八千人不為少,然須木查、騰雲、呂堰等砦受節制乃可濟。」已而劉全、雷去危兩部與金人戰於夏家橋,小捷。有頃,金人犯呂堰,珙喜曰:「吾計得矣。」亟命諸軍追擊呂堰,進逼大河,退逼山險,砦軍四合,金人棄
 輜重走,獲甲士五十有二,斬首三千,馬牛橐駝以萬計,歸其民三萬二千有奇。瑗遣其部曲馬天章奉書請降,得縣五,鎮二十二,官吏一百九十三,馬軍千五百,步軍萬四千,戶三萬五千三百,口十二萬五千五百五十三。珙入城,瑗伏階下請死,珙為之易衣冠,以賓禮見。



 初,仙屯順陽,為宋軍所撓,退屯馬蹬。金順陽令李英以縣降,申州安撫張林以州降,珙言:「歸附之人,宜因其鄉土而使之耕,因其人民而立之長,少壯籍為軍,俾自耕自守,
 才能者分以土地,任以職使,各招其徒以殺其勢」制置司是之。七月己酉,仙愛將劉儀領壯士二百降,珙問仙虛實,儀陳:「仙所據九砦,其大砦石穴山,以馬蹬、沙窩、岵山三砦蔽其前;三砦不破,石穴未易圖也。若先破離金砦,則王子山砦亦破,岵山、沙窩孤立,三帥成禽矣。」珙翼日遣兵向離金,廬秀執黑旗帥眾入砦,金人不疑為宋軍,乃分據巷道,大呼縱火,掩殺幾盡。是夜,壯士楊青等搗王子山砦,護帳軍酣寢,王建入帳中,斬金將首囊佩
 之,平明視之,金小元帥也。



 丙辰,出師馬蹬,遣樊文彬攻其前門,成明等邀截西路,一軍圍訖石烈,一軍圍小總帥砦,火燭天,殺僇山積,餘逸去者復為成明伏軍所得,壯士老少萬二千三百來歸。師還,至沙窩西,與金人遇,大捷。是日,三戰三克。未幾,丁順等又破默候里砦。珙召儀曰:「此砦既破,板橋、石穴必震,汝能為我招之乎?」儀曰:「晉德與花腿王顯、金鎮撫安威故舊,招之必來。」乃遣德行,儀又請選婦人三百偽逃歸,懷招軍榜以向,珙從之。
 威見德,敘情好甚歡,介德往見顯,顯即日以書乞降。德復請珙遣劉儀候之。顯軍約五千,猶未解甲,珙令作栲栳陣;入陣,周視良久,乃去,如素所撫循;饗以牛酒,皆醉飽歌舞。珙料武仙將上岵山絕頂窺伺,令樊文彬詰旦奪岵山,駐軍其下,前當設伏,後遮歸路。已而仙眾果登山,及半,文彬麾旗,伏兵四起,仙眾失措,枕藉崖谷,山為之赬,殺其將兀沙惹,擒七百三十人,棄鎧甲如山。薄暮,珙進軍至小水河,儀還,具言仙不欲降,謀往商州依險
 以守,然老稚不願北去,珙曰:「進兵不可緩。」夜漏十刻,召文彬等受方略,明日攻石穴九砦。丙辰,蓐食啟行,晨至石穴。時積雨未霽,文彬患之,珙曰:「此雪夜擒吳元濟之時也。」策馬直至石穴,分兵進攻,而以文彬往來給事。自寅至巳力戰,九砦一時俱破,武仙走,追及於占魚砦,仙望見,易服而遁。復戰於銀葫蘆山,軍又敗,仙與五六騎奔。追之,隱不見,降其眾七萬人,獲甲兵無算。還軍襄陽,轉修武郎、鄂州江陵府副都統制。



 大元兵遣宣撫王楫
 約共攻蔡,制置使謀於珙,珙請以二萬人行,因命珙盡護諸將。金兵二萬騎繇真陽橫山南來,珙鼓行而前,金人戰敗,卻走,追至高黃陂,斬首千二百級。



 倴盞遣兔花忒、沒荷過出、阿悉三人來迓,珙與射獵,割鮮而飲,馳入其帳。倴盞喜,約為兄弟,酌馬湩飲之。金兵萬人自東門出戰,珙遮其歸路,掩入汝河,擒其偏裨八十有七人。得蔡降人,言城中饑,珙曰:「已窘矣,當盡死而守,以防突圍。」珙與倴盞約,南北軍毋相犯。決堰水,布虎落。倴盞遣萬
 戶張柔帥精兵五千人入城,金人鉤二卒以往,柔中流矢如蝟,珙麾先鋒救之,挾柔以出。撥發官宋榮不肅,將斬之,眾下馬羅拜以請,猶杖之。黎明,珙進逼石橋,鉤致生俘郭山,戰少卻。金人突至,珙躍馬入陣,斬山以徇,軍氣復張,殊死戰,進逼柴潭立柵,俘金人百有二,斬首三百餘級。翼日,命諸將奪柴潭樓。金人爭樓,諸軍魚貫而上。金人又飾美婦人以相蠱,麾下張禧等殺之,遂拔柴潭樓,俘其將士五百三十有七人。蔡人恃潭為固,外即
 汝河,潭高於河五六丈,城上金字號樓伏巨弩,相傳下有龍,人不敢近,將士疑畏。珙召麾下飲,再行,曰:「柴潭非天造地設,樓伏弩能及遠而不可射近,彼所恃此水耳,決而注之,涸可立待。」皆曰:「堤堅未易鑿。」珙曰:「所謂堅者,止築兩堤首耳,鑿其兩翼可也。」潭果決,實以薪葦,遂濟師攻城,擒其兩將斬之,獲其殿前右副點檢溫端,磔之城下,進逼土門。金人驅其老稚熬為油,號「人油砲」,人不堪其楚,珙遣道士說止之。



 端平元年正月辛丑,黑氣壓
 城上,日無光,降者言:「城中絕糧已三月,鞍靴敗鼓皆糜煮,且聽以老弱互食,諸軍日以人畜骨和芹泥食之,又往往斬敗軍全隊,拘其肉以食,故欲降者眾。」珙下令諸軍銜枚,分運雲梯布城下。己酉,珙帥師向南門,至金字樓,列雲梯,令諸將聞鼓則進,馬義先登。趙榮繼之,萬眾競登,大戰城上,降其丞相烏古論栲栳,殺其元帥兀林達及偏裨二百人。門西開,招倴盞入,江海執其參政張天綱以歸。珙問守緒所在,天綱曰:「城危時即取寶玉置
 小室,環以草,號泣自經,曰『死便火我』,煙焰未絕。」珙與倴盞分守緒骨,得金謚寶、玉帶、金銀印牌有差。還軍襄陽,特授武功郎、主管侍衛馬軍行司公事。擢建康府都統制兼權侍衛馬軍行司職事。



 太常寺簿朱楊祖、看班祗候林拓朝八陵,諜云大元兵傳宋來爭河南府,哨已及盟津,陜府、潼關、河南皆增屯設伏,又聞淮閫刻日進師,眾畏不前。珙曰:「淮東之師,由淮、泗溯汴,非旬餘不達,吾選精騎疾馳,不十日可竣事;逮師至東京,吾已歸矣。」於
 是晝夜兼行,與二使至陵下,奉宣御表,成禮而歸。制置司奏留珙襄陽兼鎮北軍都統制。鎮北軍者,珙所招中原精銳百戰之士萬五千餘人,分屯漅北、樊城、新野、唐、鄧間。俄令赴樞密院稟議,授帶御器械。二年,授主管侍衛馬軍司公事,時暫黃州駐札,朝辭,上曰:「卿名將之子,忠勤體國,破蔡滅金,功績昭著。」珙對曰:「此宗社威靈,陛下聖德,與三軍將士之勞,臣何力之有?」帝問恢復,對曰:「願陛下寬民力,蓄人材,以俟機會。」帝問和議,對曰:「臣介
 胄之士,當言戰,不當言和。」賜齎甚厚。兼知光州,又兼知黃州。



 三年,珙至黃,增埤浚隍,搜訪軍實,邊民來歸者日以千數,為屋三萬間居之,厚加賑貸。又慮兵民雜處,因高阜為齊安、鎮淮二砦,以居諸軍。創章家山、毋家山兩堡為先鋒、虎翼、飛虎營。兼主管管內安撫司公事,節制黃蘄光、信陽四郡軍馬。



 大元兵攻蘄州,珙遣兵解其圍;又攻襄陽,隨守張龜壽、荊門守朱楊祖、郢守喬士安皆委郡去,復州施子仁死之,江陵危急。詔沿江、淮西遣援,
 眾謂無逾珙者,乃先遣張順渡江,珙以全師繼之。大元兵分兩路:一攻復州,一在枝江監利縣編筏窺江。珙變易旌旗服色,循環往來,夜則列炬照江,數十里相接。又遣外弟趙武等共戰,躬往節度,破砦二十有四,還民二萬。嘉熙元年,封隨縣男,擢高州刺史,忠州團練使兼知江陵府、京西湖北安撫副使。未幾,授鄂州諸軍都統制。



 大元大將忒沒OA入漢陽境,大將口溫不花入淮甸,蘄守張可大、舒州李士達委郡去,光守董堯臣以州降。合
 三郡人馬糧械攻黃守王鑒,江帥萬文勝戰不利。珙入城,軍民喜曰:「吾父來矣。」駐帳城樓,指畫戰守,卒全其城,斬逗留者四十有九人以徇。御筆以戰功賞將士,特賜珙金碗,珙益以白金五十兩賜之諸將。將士彌月苦戰,病傷者相屬,珙遣醫視療,士皆感泣。



 二年春,授寧遠軍承宣使、帶御器械、鄂州江陵府諸軍都統制。珙以三軍賞典未頒,表辭。詔曰:「有功不賞,人謂朕何?三軍勛勞,趣其來上。封爵之序,自將帥始,卿奚辭焉?」未幾,授樞密副
 都承旨、京西湖北路安撫制置副使兼督視行府參謀官。未幾,升制置使兼知嶽州。乃檄江陵節制司搗襄、郢,於是張俊復郢州,賀順復荊門軍。十二月壬子,劉全戰於塚頭,戰於樊城,戰於郎神山,屢以捷聞。三年春正月,曹文鏞復信陽軍,劉全復樊城,遂復襄陽。授樞密都承旨、制置使兼知鄂州。全遣譚深復光化軍,息、蔡降,珙命以兵逆之,得壯士百餘,籍為忠衛軍。



 初,詔珙收復京、襄,珙謂郢然後可以通饋餉得荊門然後可以出奇
 兵,由是指授方略,發兵深入,所至以捷聞。珙奏略曰:「取襄不難而守為難,非將士不勇也,非車馬器械不精也,實在乎事力之不給爾。襄、樊為朝遷根本,今百戰而得之,當加經理,如護元氣,非甲兵十萬,不足分守。與其抽兵於敵來之後,孰若保此全勝?上兵伐謀,此不爭之爭也。」乃置先鋒軍,以襄、郢歸順人隸焉。



 庚寅,諜報大元兵欲大舉臨江,珙策必道施、黔以透湖湘,請粟十萬石以給軍餉,以二千人屯峽州,千人屯歸州。忠衛舊將晉德
 自光化來歸,珙獎用之。珙弟瑛以精兵五千駐松滋為夔聲援,遣於德興增兵守歸州隘口萬戶谷。大元兵自隨窺江,珙密遣劉全拒敵,遣伍思智以千人屯施州。大元大將塔海並禿雪帥師入蜀,號八十萬,珙增置營砦,分布戰艦,遣張舉提兵間道抵均州防遏。大元兵度萬州湖灘,施、夔震動,珙兄璟時為湖北安撫副使、知峽州,急以書謀備御。珙請於督府,帥師西上。璟調金鐸一軍迎拒于歸州大𤦩砦。劉義捷於巴東縣之清平村。珙弟
 璋選精兵二千駐澧州防施、黔路。四年,進封子。



 珙條上流備御宜為藩籬三層:乞創制副司及移關外都統一軍於夔,任涪南以下江面之責,為第一層;備鼎、澧為第二層;備辰、沅、靖、桂為第三層。峽州、松滋須各屯萬人,舟師隸焉,歸州屯三千人,鼎、澧、辰、沅、靖各五千人,郴、桂各千人,如是則江西可保。又遣楊鼎、張謙往辰、沅、靖三州,同守倅曉諭熟蠻,講求思、播、施、黔支徑,以圖來上。



 會諜知大元兵於襄樊隨、信陽招集軍民布種,積船材於鄧
 之順陽,乃遣張漢英出隨,任義出信陽,焦進出襄,分路撓其勢。遣王堅潛兵燒所積船材,又度師必因糧於蔡,遣張德、劉整分兵入蔡,火其積聚。制拜寧武軍節度使、四川宣撫使兼知夔州。招集麻城縣、巴河、安樂磯、管公店淮民三百五十有九人,皆沿邊經戰之士,號「寧武軍」,令璋領之。進封漢東郡侯兼京湖安撫制置使。



 回鶻愛里八都魯帥壯士百餘、老稚百一十五人、馬二百六十匹來降,創「飛鶻軍」,改愛里名艾忠孝,充總轄,乞補以官。
 四川制置使陳隆之與副使彭大雅不協,交章於朝。珙曰:「國事如此,合智並謀,猶懼弗克,而兩司方勇於私鬥,豈不愧廉、藺之風乎。」馳書責之,隆之、大雅得書大慚。



 厘蜀政之弊,為條班諸郡縣,曰差除計蜀,曰功賞不明,曰減克軍糧,曰官吏貪黷,曰上下欺罔。又曰:「不擇險要立砦柵,則難責兵以衛民;不集流離安耕種,則難責民以養兵。」乃立賞罰以課殿最,俾諸司奉行之。黎守閻師古言大理國請道黎、雅入貢,珙報大理自通邕、廣,不宜取
 道川蜀,卻之。兼夔路制置大使兼屯田大使。軍無宿儲,珙大興屯田,調夫築堰,募農給種,首秭歸,尾漢口,為屯二十,為莊百七十,為頃十八萬八千二百八十,上屯田始末與所減券食之數,降詔獎諭。靖州徭林賽良為亂,遣王瑀平之。



 淳祐二年,珙以京、襄死節死事之臣請於朝,建祠岳陽,歲時致祭,有旨賜名閔忠廟。淮東受兵,樞密俾珙應援,遣李得帥精兵四千赴之,珙子之經監軍。諜知京兆府也可那延以騎兵三千經商州取鶻嶺關,
 出房州竹山,遣王令屯江陵,尋進屯郢州,劉全屯沙市,焦進提千人自江陵、荊門出襄。檄劉全齎十日糧,取道南漳入襄,與諸軍合。



 大元兵至三川,珙下令應出戍主兵官,不許失棄寸土。權開州梁棟乏糧,請還司,珙曰:「是棄城也。」棟至夔州,使高達斬其首以徇。由是諸將稟令惟謹。大元兵至瀘,珙命重慶分司發兵應援,遣張祥屯涪州。拜檢校少保,進封漢東郡公。珙言:「沅之險不如辰,靖之險不如沅,三州皆當措置而靖尤急。今三州粒米
 寸兵無所從出,出京湖之憂一。江防上自秭歸,下至壽昌,亙二千里,自公安至峽州灘磧凡十餘處,隆冬水涸,節節當防,兵諱備多,此京湖之憂二。今尺籍數虧,既守灘磧,又守關隘,此京湖之憂三。陸抗有言:『荊州國之藩表,如其有虞,非但失一郡,當傾國爭之。若非增兵八萬並力備御,雖韓、白復生,無所展巧。』今日事勢大略相似,利害至重。」餘玠宣諭四川,道過珙,珙以重慶積粟少,餉屯田米十萬石,遣晉德帥師六千援蜀,之經為策應司
 都統制。四年,兼知江陵府。珙謂其佐曰:「政府未之思耳,彼若以兵綴我,上下流急,將若之何?珙往則彼搗吾虛,不往則誰實捍患。」識者是之。



 詔京湖調兵五千戍安豐,援壽春。珙遣劉全將以往。繼有命分兵三千備齊安,珙言:「黃州與壽昌三江口隔一水耳,須兵即度,何必預遣?先一日則有一日之費,無益有損,萬一上游有警,我軍已疲,非計之得也。」不從。五年,御筆以職事修舉,轉行兩官,許令回授。珙至江陵,登城嘆曰:「江陵所恃三海,不知
 沮洳有變為桑田者,敵一鳴鞭,即至城外。蓋自城以東,古嶺先鋒直至三水義,無所限隔。」乃脩復內隘十有一,別作十隘於外,有距城數十里者。沮、漳之水,舊自城西入江,因障而東之,俾繞城北入於漢,而三海遂通為一。隨其高下,為匱蓄洩,三百里間,渺然巨浸。土木之工百七十萬,民不知役,繪圖上之。



 珙以身鎮江陵,而兄璟帥武昌,故事,無兄弟同處一路者,乞歸田,不允。詔以兵五千援淮,珙使張漢英帥之。樞密調兵五千赴廣西,珙移書
 執政曰:「大理至邕,數千里部落隔絕,今當擇人分布數郡,使之分治生夷,險要形勢,隨宜措置,創關屯兵,積糧聚芻於何地,聲勢既張,國威自振。計不出此而聞風調遣,空費錢糧,無補於事。」不聽。大元大將大納至江陵,遣楊全伏兵荊門以戰,珙先期諜知,達於樞密,檄兩淮為備,兩淮不知也,後果如所報。珙奏:「襄、蜀蕩析,士無所歸,蜀士聚於公安,襄士聚於郢渚。臣作公安、南陽兩書院,以沒入田廬隸之,使有所教養。」請帝題其榜賜焉。



 初,珙
 招鎮北軍駐襄陽,李虎、王旻軍亂,鎮北亦潰,乃厚招之,降者不絕。行省範用吉密通降款,以所受告為質,珙白於朝,不從。珙嘆曰:「三十年收拾中原人,今志不克伸矣。」病遂革,乞休致,授檢校少師、寧武軍節度使致仕,終於江陵府治,時九月戊午也。是月朔,大星隕於境內,聲如雷。薨之夕,大風發屋折木。訃至,帝震悼輟朝,賻銀絹各千,特贈少師,三贈至太師,封吉國公,謚忠襄,廟曰威愛。



 珙忠君體國之念,可貫金石。在軍中與參佐部曲論事,
 言人人異,珙徐以片語折衷,眾志皆愜。謁士游客,老校退卒,壹以恩意撫接。名位雖重,惟建鼓旗、臨將吏而色凜然,無敢涕唾者。退則焚香掃地,隱幾危坐,若蕭然事外。遠貨色,絕滋味。其學邃於《易》,六十四卦各系四句,名《警心易贊》。亦通佛學,自號「無庵居士」。



 杜杲,字子昕,邵武人。父穎,仕至江西提點刑獄,故杲以任授海門買納鹽場,未上,福建提點刑獄陳彭壽檄攝閩尉。民有甲之子死,誣乙殺之,驗發中得沙,而甲舍旁
 有池沙類發中者,鞫問,子果溺死。



 江、淮制置使李玨羅致幕下。滁州受兵,檄杲提偏師往援,甫至,民蔽野求入避,滁守固拒,杲啟鑰納之。金人圍城數重,杲登陴中矢,益自奮厲,卒全其城。



 調江山丞,兩浙轉運使朱在闢監崇明鎮,崇明改隸淮東總領,與總領岳珂議不合,慨然引去。珂出文書一卷,曰:「舉狀也。」杲曰:「比而得禽獸,雖若丘陵,弗為。」珂怒,杲曰:「可劾者文林,不可強者杜杲。」珂竟以負蘆錢劾,朝廷察蘆無虧,三劾皆寢。



 淮西制置曾式
 中闢廬州節度推官。浮光兵變,杲單騎往誅其渠魁,守將爭餉金幣,悉封貯一室,將行,屬通判鄭準反之。安豐守告戍將扇搖軍情,且為變,帥欲討之,杲曰:「是激使叛也。」請與兩卒往,呼將諭之曰:「而果無他,可持吾書詣制府。」將即日行,一軍帖然。



 知六安縣,民有嬖其妾者,治命與二子均分。二子謂妾無分法,杲書其牘云:「《傳》云『子從父令』,律曰『違父教令』,是父之言為令也,父令子違,不可以訓。然妾守志則可,或去或終,當歸二子。」部使者季衍
 覽之,擊節曰:「九州三十三縣令之最也。」



 知定遠縣,會李全犯邊,衍時為淮帥,闢通判濠州,朝廷以杲久習邊事,擢知濠州。制置大使趙善湘謀復盱眙,密訪杲,杲曰:「賊恃外援,當斷盱眙橋梁以困之。」卒用其策成功。金眾數萬駐榆林阜請降,輜重甚富,或請誘而圖之。杲曰:「殺降不仁,奪貨不義,納之則有後患。諭而遣之。召奏事,差主管官告院,知安豐軍。善湘與趙範、範弟葵出師,遷淮西轉運判官。詔問守御策,杲上封曰:「沿淮旱蝗,不任徵役;
 中原赤立,無糧可因。若虛內事外,移南實北,腹心之地,必有可慮。」時在外諫出師者惟杲一人。及兵敗洛陽,人始服其先見。奉崇道祠,再知濠州,未行,改安豐。大元兵圍城,與杲大戰。明年,大兵復大至,又大戰。擢將作監,御書慰諭之。丞相李宗勉、參知政事徐榮叟曰:「帥淮西無逾杜杲者。」詔以安撫兼廬州,進太府卿、淮西制置副使兼轉運使。復與大元兵戰。累疏請老,不許。權刑部尚書。



 淳祐元年,乞去愈力,擢工部尚書,遂以直學士奉祠。帝
 欲起之帥廣西,以言者罷。帝曰:「杜杲兩有守功,若脫兵權,使有後禍,朕何以使人?」乃起知太平州。俄擢華文閣學士、沿江制置使、知建康府、行宮留守,節制安慶、和、無為三郡。



 杲罷楊林堡,以其費備歷陽,淮民寓沙上者護以師。首謁程顥祠。總領所即張栻宦游處,陳像設祀焉。置貢士莊,蠲民租二萬八千石。復與大元兵戰於真州。進敷文閣學士,遷刑部尚書,引見,帝加獎勞。乞歸不許,兼吏部尚書。杲隨資格通其礙,銓綜為精。梁成大子賂
 當國者求銓試,杲曰:「昔沈繼祖論朱文公,成大亦論真文忠公,皆得罪名教者,子孫宜廢錮,安得仕?」進徽猷閣,奉祀。請老,升寶文閣致仕。帝思前功,進龍圖閣而杲卒,遺表上,贈開府。



 杲淹貫多能,為文麗密清嚴,善行草急就章。晚歲專意理學,嘗言吾兵間無悖謀左畫,得於《四書》。子庶。



 庶字康侯,幼倜儻有大志,性剛勁,通宋典故,善為文。從父兵間,習邊事,未人仕已立戰功,明堂恩補官。大元兵
 圍安豐,兵將不相下,庶調護咸得其歡心,卒協力捍禦。杲帥淮西,闢書寫機宜文字。廬州圍解,庶白事廟堂,諸將饋金助上功費,皆受之,賞典行,歸悉反所饋。遷籍田令兼制機督幹。監呂文德、聶斌軍,與大元兵戰朱皋、白塚,遷將作監簿。



 杲在建康,庶通判和州,權知真州。郡素缺備,庶大修守御,具積排杉木殆十萬株。差知興化軍,奉祀鴻禧觀。起知邕州,改潮州,以言者寢命。赴淮東制司議幕,過闕,遷將作監丞。遷司農丞、知和州,陛辭,言:「今
 天時不可幸,地利不可恃,人和不可保,茍恃天幸,恃長江,恃清野,而付邊事於素不諳歷之人,未見其可。」帝嘉納。



 尋兼淮西提點刑獄,浚城濠,增守備,修學宮。知真州兼淮東提點刑獄,逾年,進直秘閣,移淮西兼廬州安撫副使,人歡迎如見慈父,治績甚多。就任加刑部郎中,升寶文閣,與大元兵戰於望仙、白沙城。升華文閣。開慶元年冬,進大理少卿、淮東轉運副使、兩淮制置司參謀官,特授兩淮制置使、知揚州。射陽湖饑民嘯聚,庶曰:「吾赤
 子也。」遣將招刺,得丁壯萬餘,戮止首惡數人。明年四月,火,抗章自劾,召赴行在。尋直寶文閣、知隆興府、江西轉運副使,卒。



 王登,字景宋,德安人。少讀書,喜古兵法,慷慨有大志,不事生產。出制置使孟珙幕府,久之,權知巴東縣。獻俘制置司,登念奮自書生,不拜,吏曰:「不拜則不敢上。」難之,竟棄功去。淳祐四年,舉進士,調興山主簿。總領賈似道檄修江陵城,條畫有法。明年,制置使李曾伯經理襄陽,登
 在行,以積功升,尋以母憂去。



 及吳淵為制置使,邊事甚亟,因憶弟潛盛言王登才略,具書幣招之。登方與客奕,發書,衣冠拜家廟,長揖出門,問牛幾何,可盡發犒師。淵慨然曰:「事亟矣,奈何?」登曰:「亟呼諸將共議。」眾至,歡躍曰:「景宋在此。」淵曰:「汝輩欲西門出,景宋欲從方城,如何?」眾曰:「惟命!」登曰:「用兵患不一,登書生,不過馮軾觀戰,請五大帥中擇一人為節制。」淵曰:「請監丞出,正謂此也。」即書銀牌曰:「監丞代某親行,將士用命不用命,賞罰畢具申。」
 登至沙市,椎牛釃酒,得七千人,誓曰:「登與諸將義同骨肉,今日之事,登不且命,諸將殺登以獻主帥;諸將有一不用命,登有制札在,不敢私也。」眾股慄聽命,竟立奇功於沮河。趙葵為制置使,見登握手曰:「景宋一身膽,惜相見晚也。」俾參宣撫司兼京西兩節。馬光祖為制置使,闢充參謀官,遷軍器少監、京西提點刑獄。



 登威聲日振。有餘思忠及徐制幾讒於光祖曰:「京湖知有王景宋,不知有馬制置,非久易位矣。」光祖疑焉,出登屯郢州,後以乾
 辦鐘蜚英調護,情好如初。侍御史戴慶炣劾思忠,其黨過元龍、沈翥在幕中,又傾之,以是議論不合,才略不能施,識者惜焉。



 開慶元年,登提兵援蜀,約日合戰,夜分,登經理軍事,忽絕倒,五藏出血。幕客唐舜申至,登尚瞪目視幾上文書,俄而卒。它日,舜申舟經漢陽,有蜀聲呼唐舜申者三,左右曰:「景宋聲也。」是夕,舜申暴卒。



 楊掞,字純父,撫州臨川人。少能詞賦,里陳氏館之教子,數月拂衣去。游襄、漢,既而代陳中選,陳謝之萬緡,輦以
 入倡樓,篋垂盡,夜忽自呼曰:「純父來此何為?」明日遂行。用故人薦,出淮閫杜杲幕,杲曰:「風神如許,它日不在我下。」由是治法徵謀多咨於掞。逾年,安豐被兵,掞慨然曰:「事亟矣,掞請行。」乃以奇策解圍,奏補七官。



 掞念置身行伍間,騎射所當工,夜以青布籍地,乘生馬以躍,初過三尺,次五尺至一丈,數閃跌不顧。制置使孟珙闢於幕,嘗用其策為「小子房」,與之茶局,周其資用。掞以本領錢數萬費之,總領賈似道稽數責償,珙以白金六百令掞償
 之,掞又散之賓客,酣歌不顧。似道欲殺之,掞曰:「漢高祖以黃金四萬斤付陳平,不問出入,公乃顧此區區,不以結豪傑之心邪?」似道始置之。珙嘗燕客,有將校語不遜,命斬之,掞從容曰:「斬之誠是,第方會客廣謀議,非其時非其地也。」珙大服。未幾,有大將立功,珙坐受其拜,掞為動色,因嘆曰:「大將立功,庭參納拜,信兜鍪不如毛錐子也。」於是謝絕賓客,治進士業,遂登第,調麻城尉。



 向士璧守黃州,檄入幕,尋以戰功升三官。無何,得心疾,曰:「我不
 可用矣。」遂調潭州節度推官。趙葵為京湖制置使,掞與偕行,王登迓於沙市,極談至夜分,掞退曰:「王景宋滿身是膽,惜欠沉細者,如掞副之,何事不可為也,但恐終以勇敗。」後登死,人以為知言。逾時,士璧守峽州,招之,病不果行而卒,贈架閣。



 張惟孝,字仲友,襄陽人。長六尺,通《春秋》,下第,乃工騎射。城中亂,爭出關,惟孝拔劍殺數人,趨白河,見一舟壯鉅甚,急登之,舟人不可,惟孝曰:「今日之事,非汝即我,能殺
 我者得此舟。」眾披靡,遂以舟達郢州。兵亂,奔沙洋,別之傑為帥,盡隘諸湖不洩水,惟孝令二人賈服前行,密窺隘兵,曰:「易與耳。」乃與十騎,衣黑袍,假為敵兵,曰:「後隊亟至。」守隘四五百人悉潰,舟趨藕池。



 開慶元年,卜居江陵,至沙市,眾舟大集,不可涉。頃有峨冠張蓋,從者數十,則宣撫姚希得之弟也,今曰:「敢有爭岸者投水中。」惟孝睥睨良久,提劍驅左右而出,舉白旗以麾,令眾船登岸,毋敢亂次。幹官鐘蜚英見而異之,以告唐舜申,舜申曰:「吾
 故人也。」具言惟孝平生。蜚英謂曰:「今日正我輩趨事赴功之秋。」惟孝不答,又叩之,則曰:「朝廷負人。」明日,蜚英導希得羅致之,宴仲宣樓,蜚英酒酣曰:「有國而後有家,天下如此,將安歸乎?」惟孝躍然曰:「從公所命。」乃請空名帖三十以還。逾旬,與三十騎俱擁甲士五千至,旗幟鮮明,部伍嚴肅,上至公安,下及墨山,游踏相繼。希得大喜,請所統姓名,惟孝曰:「朝廷負人,福難禍易,聊為君侯紓一時之難耳,姓名不可得也。」時鼎、澧五州危甚,於是擊鼓
 耀兵,不數日,眾至萬人,數戰俱捷,江上平。制使呂文德招之,不就而遁,物色之不可得,或云已趨淮甸,後不知所終。



 陳咸,字逢儒,監察御史升卿次子,為叔父巨卿後。登淳熙二年進士第,調內江縣尉。縣吏受賄,賦民不均,咸以聞於部使者,為下令聽民自陳利病,而委咸均其賦。改知果州南充縣,轉運司闢主管文字。歲旱,稅司免下戶兩稅,轉運使安節以為虧漕計,咸白安節曰:「茍利於民,
 違之不可。」因言:「今楮幣行於四川者幾虧三百萬,茍增印百萬,足以補放免之數。」安節從之。軍多濫請,咸每裁損,帥屬以為言,咸曰:「咸首可斷,濫請不可得。」蜀歲收激賞權輸絹錢,民以為病,咸白安節,核入節出,奏歲減二十餘萬緡。擢知資州,時久旱,咸被命即請帥臣發粟二千餘石以振。明年,東、西川皆旱,總制二司議蠲民賦而慮虧國課,咸請增印未補發引百有九萬以償所蠲,議遂決。大修學宮,政以最聞,改知普州。



 開禧元年,邊事興,
 四川宣撫使程松奇其才,闢主管機宜文字。咸首貽書論兵不可輕動,勸松搜人才,練軍實;考圖籍以疏財用之源,視險要以決攻守之計;約大將面會,以免疑忌之嫌;捐金帛募死士,以明間探之遠;出虛搗奇之策,審於當用;幸勝趨利之謀,寢而勿行。松復書深納,然實不能用。副使吳曦蔑視松,易置將兵,不關白正使。松務為簡貴,咸憂之,復說松收梁、洋以北義士為緩急用;據險厄,立關堡,杜支徑以備不虞。松又不能用。遷利路轉運判
 官。



 曦叛臣於金,關外四州繼沒,人情大駭。咸留大安軍督軍糧,檄其守楊震仲振流民,備奸盜,眾稍安。安丙密以曦反謀告咸,咸即遣人告松,松不之察。曦以咸蜀名士,欲首脅之以令其餘,檄咸議事。咸不往,遂之利州。抵城外,偽都運使徐景望已挾兵入居臺治。英宗諱日,景望大合樂以享,咸力拒之。



 初,咸自大安東下,遇偽將褚青與語,青有悔意。至是,以主管文字王釜、福艾可與共事,欲結二人誅景望,燒棧閣,絕曦援兵。既而釜棄官歸,
 咸以青不可保,謀遂沮。李道傳問咸:「計將安出?」咸曰:「事極不過一死耳,必不為吾蜀累也。」語家子欽曰:「咸受國厚恩,義當擊賊,恨無兵權,獨有下策,削發以全臣節。」會曦以書招之急,咸答書勸其稟命,既而欲親諭之,遂行,遇偽統領孟可道,知曦已僭亂,曰:「吾書不可用矣。」還至後金敖,入帳中以刀自斷其結,披緇而出。景望遣兵拘咸於岸,曦聞怒甚。吳睍勸曦召咸主武興寺,因殺之,安丙力為救解,乃得釋歸。曦既誅,咸語諸子曰:「吾不能討賊
 而棄官守,罪也。」上表自劾,安丙、楊輔等皆勉其出。丙尋奏以咸總蜀賦,從之。



 時僭亂後,帑藏赤立。咸至武興,與丙商榷利病,兵政財計,合為一家,請丙奏於朝。核諸司羨餘,移支常平廣惠米,鑄當五錢,榜賣官,並權截四路上供,汰弱兵二萬餘,規畫備至,故軍興增支之數八千七百五十餘萬,皆不取於民。咸總賦之始,贍軍帑廩緡不過一千四十五萬餘,糧不過九十一萬餘,料不過二萬餘。咸晝夜精勤,調度有方,不二歲,益昌大軍庫有楮
 引百八十萬,成都免引場樁撥二百一十餘萬,城下三倉軍糧四十餘萬石,預借米本一百一十餘萬,又別貯軍糧百四十九萬石,料七萬餘,而布帛絲綿、銅鐵錢與祠牒不預焉。



 劍外民久苦役調,或建議調東、西兩路及夔路丁壯共其勞。令始下,民憚行,馳訴於安丙,乞計直輸錢以免行,久而不克輸者十五餘萬,咸蠲之。蜀錢引舊約兩界五千餘萬,半藏於官,自軍興引皆散於民,宣、總二司增創三界通行八千餘萬,價日益落。咸捐一千
 二百餘萬緡以收十九界之半,又與丙議合茶馬司之力,再收九十一界,續造九十三界以兌之,於是引價復昂,糴價頓減。



 嘉陵江流忽淺,或云金人截上流,咸不動,疏而導之,自益昌至於魚梁,饋運無阻。金州地險,咸增饋米以實之,人皆曰:「金州之險,金人不可向,何益之為?」咸曰:「敵至而慮,無及矣。」未幾,金人犯上津,守賴以固。召為司農少卿,卒。丙列奏其功,賜謚勤節。初,宣諭使吳獵嘗表其節,詔進二秩,咸乞回贈所生父母焉。



 論曰:宋之辱於金久矣,值我國家興師討罪,聲震河朔,乃遣孟珙帥師夾攻,遂滅其國,以雪百年之恥。而珙說禮樂、敦詩書,誠寡與二。杜杲、王登、楊掞、張惟孝,思以功名自見,雖所立有小大,皆奇才也。陳咸不從逆曦,雖不能死,然理財於喪亂之餘,蜀賴以固守,豈不賢於匹夫而莫經溝瀆者哉!



\end{pinyinscope}