\article{志第七十六 禮二十六(兇禮二)}

\begin{pinyinscope}

 園陵濮安懿王園廟秀安僖王園廟莊文景獻二太子攢所上陵忌日



 皇后園陵。太祖建隆二年六月二日,皇太后杜氏崩於
 滋德殿。三日,百官入臨。明日大斂,攢於滋福宮,百官成服,中書、門下、文武百僚、諸軍副兵馬使以上並服布斜巾四腳、直領衣蘭衫,外命婦帕頭、帔、裙衫。九日,帝見百官於紫宸門。太常禮院言:「皇后、燕國長公主高氏、皇弟泰寧軍節度使光義、嘉州防禦使光美並服齊衰三年。準故事,合隨皇帝以日易月之制,二十五日釋服,二十七日禫除畢,服吉,心喪終制。」從之。



 七月,太常禮院言:「準詔議定皇太后謚,按唐憲宗母王太后崩,有司集議,以謚
 狀讀於太廟,然後上之。周宣懿皇后謚,即有司撰定奏聞,未常集議,制下之日,亦不告郊廟,修謚冊畢始告廟,還,讀於靈坐前。」詔從周制。於是,太常少卿馮吉請上尊謚曰明憲皇太后。九月六日,群臣奉冊寶告於太廟,翌日上於滋福宮。十月十六日,葬安陵。十一月四日,神主祔太廟宣祖室。



 乾德二年,改卜安陵於河南府鞏縣。三月二十五日,奉寶冊,改上尊謚曰昭憲皇太后,讀於陵次。二十六日,啟故安陵。二十七日,靈駕發引,命攝太尉、開
 封尹光義遣奠,讀哀冊。四月九日,掩皇堂。



 太祖孝明、孝惠二後。乾德元年十二月七日,皇后王氏崩。二十五日,命樞密承旨王仁贍為園陵使。時議改卜安陵於鞏,並以二後陪葬焉。皇堂之制,下深四十五尺,上高三十尺。陵臺再成,四面各長七十五尺。神墻高七尺五寸,四面各長六十五步。南神門至乳臺四十五步,高二丈三尺。吉仗用中宮鹵簿,兇仗名物悉如安陵而差減其數,孝惠又減孝明焉。



 二年三月二十七日,孝明
 皇后啟攢宮,群臣服初喪之服;明日,孝惠皇后自幄殿發引。皆設遣奠,讀哀冊。四月九日,葬孝惠於安陵之西北,孝明於安陵之北。二十六日,皆祔於別廟。其後,孝明升祔太祖室。



 太祖皇后宋氏,太宗至道元年四月二十八日崩。帝出次,素服舉哀,輟朝五日。六月六日,上謚曰孝章皇后。以歲在未,有忌,權攢於趙村沙臺。三年正月二十日,祔葬永昌陵之北。皇堂、陵臺、神墻、乳臺、鵲臺並如孝明園陵
 制度,仍以故許王及夫人李氏、魏王夫人王氏、楚王夫人馮氏、皇太子亡妻莒國夫人潘氏、將軍惟正之妻裴氏陪葬。二月二日,祔神主於別廟。莒國潘氏,至道三年六月追冊為莊懷皇后,陵曰保泰,神主祔後廟。



 太宗賢妃李氏,真宗至道三年十二月追尊為皇太后,謚曰元德,祔葬永熙陵。大中祥符六年,升祔太宗室。



 太宗明德皇后李氏,真宗景德元年三月十五日崩。十七日,群臣上表請聽政,凡五上始允。帝去杖、絰,服衰,即
 御坐,哀動左右。太常禮院言:「皇后宜準昭憲皇太后禮例,合隨皇帝以日易月之制。宗室雍王以下,禫除畢,吉服,心喪終制。」五月,詳定園陵,宜在元德皇太后陵西安葬。八月十二日,上謚。九月二十二日,遷坐於沙臺攢宮。十月七日,祔神主太宗室。三年十月十五日,帝詣攢宮致奠。十六日,發引。二十九日,掩皇堂。



 真宗章穆皇后郭氏,景德四年四月十五日崩。皇帝七日釋服,後改用十三日。群臣三日釋服。諸道、州、府官吏
 訃到日舉哀成服,三日而除。二十一日,司天監詳定園陵。帝令祔元德皇太后陵側,但可安厝,不必寬廣,其棺槨等事,無得鐫刻花樣,務令堅固。二十五日,殯於萬安宮之西階。詔兩制、三館、秘閣各撰挽詞。閏五月十三日,上謚曰莊穆。六月二十一日,葬永熙陵之西北。七月,有司奉神主謁太廟,祔享於昭憲皇后,享畢,祔別廟。大中祥符二年四月十五日,大祥。詔特廢朝,群臣奉慰。



 真宗宸妃李氏,仁宗明道元年二月二十六日薨。初葬
 洪福禪院之西北,命晏殊撰墓銘。二年四月六日,追冊為莊懿皇太后。十月五日,改葬永定陵之西北隅。十七日,祔神主於奉慈廟。



 真宗章獻明肅皇後劉氏,明道二年三月二十七日崩於寶慈殿,遷坐於皇儀殿。三十日,宣遺誥,群臣哭臨,見帝於殿之東廂奉慰。宗室削杖不散發。中書、樞密、使相比宗室,去斜巾、垂帽、首絰及杖。翰林學士至龍圖閣直學士已上、並節度使、文武二品已上,又去中單及褲。兩
 省、御史臺中丞文武百官以下,四腳幅巾、連裳、腰絰。館閣讀書、翰林待詔、伎術官並給孝服。宰相、百官朝晡臨三日,內外命婦朝臨三日。



 四月,遣使告哀遼、夏及賜遺留物。十日,司天監詳定山陵制度。皇堂深五十七尺。神墻高七尺五寸,四面各長六十五步。乳臺高一丈九尺,至南神門四十五步。鵲臺高二丈三尺,至乳臺四十五步。詔下宮更不修蓋,餘依。二十七日,以宰臣張士遜為山園使。是日,翰林學士馮元請上尊謚;九月四日,讀於
 靈坐。十月五日,葬永定陵之西北隅。十七日,祔神主於奉慈廟。



 真宗章惠皇后楊氏。景祐三年十一月五日,保慶皇太后崩。太常禮院言:「皇帝本服緦麻三月,皇帝、皇后服皆用細布,宗室皆素服、吉帶,大長公主以下亦素服,並常服入內,就次易服,三日而除。」詔以「保祐沖人,加服為小功,五日而除。」四年正月十六日,上謚。二月六日,葬永定陵之西北隅。十六日,升祔奉慈廟。



 仁宗慈聖光獻皇後曹氏。神宗元豐二年十月二十日,太皇太后崩於慶壽宮。是日,文武百官入宮,宰臣王珪升西階,宣遺誥已,內外舉哭,盡哀而出。二十六日大斂,命韓縝為山陵按行使。二十九日,皇帝成服。十一月,韓縝言:「永昭陵北稍西地二百十步內,取方六十五步,可為山陵。」上以迫隘,縝言:「若增十步,合征火相主及中五之數。」詔增十步。



 十二月,中書言:「先是,司天監選年月,遷祔濮安懿王三夫人。今大行太皇太后山陵,濮三夫人
 亦當舉葬。」於是詔宗室正任防禦使以上許從靈駕,已從濮安王夫人者,免從。



 三年正月十四日,上謚。太常禮院言:「大行太皇太后雖已有謚,然山陵未畢,俟掩皇堂,去『大行』,稱慈聖光獻太皇太后;祔廟題神主,仍去二『太』字。」



 秘閣校理何洵直言:「按禮,既葬,日中還,虞於正寢。蓋古者之葬,近在國城之北,故可以平旦而往,至日中即虞於寢,所謂葬日虞,弗忍一日離也。後世之葬,其地既遠,則禮有不能盡如古者。今大行太皇太后葬日至第
 六虞,自當行之於外,如舊儀;其七虞及九虞、卒哭,謂宜行之於慶壽殿。又按《春秋公羊傳》曰:『虞主用桑。』《士虞禮》曰:『桑主不文。』伏請罷題虞主。」太常言:「洵直所引,乃士及諸侯之禮。況嘉祐、治平並虞於集英殿,宜如故事。又嘉祐、治平,虞主已不書謚,當依所請。」



 太常禮院又言:「慈聖光獻皇后祔廟,前三日,告天地、社稷、太廟、皇后廟如故事。至日,奉神主先詣僖祖室,次翼祖、宣祖、太祖、太祖後。太宗皇帝、懿德皇后、明德皇后同一祝,次饗元德皇后。
 慈聖光獻皇后,異饌、異祝,行祔廟之禮。次真宗、仁宗、英宗室。禮畢,奉神主歸仁宗室。如此,則古者祔謁之禮及近代遍饗故事,並行不廢。」從之。三月十日,葬永昭陵。二十二日,祔於太廟。



 英宗宣仁聖烈皇后高氏,哲宗元祐八年九月三日崩於崇慶宮。遺誥:「皇帝成服,三日內聽政,群臣十三日,諸州長吏以下三日而除。釋服之後,勿禁作樂。園陵制度,務遵儉省。餘並如章獻明肅皇太后故事。」十四日,詔園
 陵依慈聖光獻太皇太后之制。紹聖元年正月二十八日,禮部言:「將題神主,謹按章獻明肅皇后神主書姓劉氏。」詔依故事。四月一日,葬永厚陵。



 神宗欽聖憲肅皇後向氏,建中靖國元年正月十三日崩。二月,太常寺言:「大行皇太后山陵一行法物,宜依元豐二年慈聖光獻皇后故事。皇堂之制,下深六十九尺,面方二丈五尺,石地穴深一丈,明高二丈一尺。鵲臺二,各高四丈一尺。乳臺二,各高二丈七尺。神墻高一
 丈三尺。」五月六日,葬永裕陵。二十六日,祔於神宗廟室。



 先是,元祐四年,美人陳氏薨,贈充儀,又贈貴儀。徽宗入繼大統,詔有司議追崇之典,上尊謚曰欽慈皇后,祔葬永裕陵,與欽聖同祔神宗室。崇寧元年二月,聖瑞皇太妃朱氏薨,制追尊為皇太后,遂上尊謚曰欽成皇后,五月祔葬永裕陵,祔神主於神宗室,皆備禮如故事。



 哲宗皇後劉氏,政和三年二月九日崩。詔:「崇恩太後合行禮儀,可依欽成皇后及開寶皇后故事,參酌裁定。」閏
 四月,上謚曰昭懷皇后。五月,葬永泰陵,祔神主於哲宗廟室。



 徽宗皇后王氏,大觀二年九月二十六日崩。尚書省言:「章穆皇后故事,真宗服七日。」從之。十月,太史局言:「大行皇后園陵斬草用十月二十四日斥,土用十一月十三日,葬用十二月二十七日。諸宗室合祔葬者,並依大行皇後月日時刻。」十一月,宰臣蔡京等請上謚曰靖和皇后。十二月,奉安梓宮於永裕陵之下宮,神主祔別廟。四年十二月,改謚曰惠恭。其後,高宗復改曰顯恭。



 哲宗昭慈聖獻皇後孟氏,紹興元年四月崩。詔以繼體之重,當承重服。以遺誥擇近地權殯,俟息兵歸葬園陵。梓取周身,勿拘舊制,以為他日遷奉之便。六月,殯於會稽上亭鄉。攢宮方百步,下宮深一丈五尺,明器止用鉛錫。置都監、巡檢各一員,衛卒百人。生日忌辰、旦望節序,排辦如天章閣儀。虞主還州,行祔廟禮。



 徽宗顯仁皇后韋氏,紹興二十九年崩,祔於永祐陵攢宮。



 高宗憲聖慈烈皇后吳氏,慶元三年崩。時光宗以太上皇承重,寧宗降服齊衰期。四年三月甲子,權攢於永思陵。



 孝宗成肅皇后夏氏,開禧三年崩,殯於永阜陵正北。吏部尚書陸峻言:「伏睹列聖在御,間有諸後上仙,緣無山陵可祔,是致別葬。若上仙在山陵已卜之後,無有不從葬者。其它諸後,葬在山陵之前,神靈既安,並不遷祔。惟元德、章懿二後,方其葬時,名位未正,續行追冊。其成穆
 皇后,孝宗登極即行追冊,改殯所為攢宮,典禮已備,與元德、章懿事體不同,所以更不遷祔。竊稽前件典禮,祇緣喪有前後,勢所當然,其於禮意,卻無隆殺。今來從葬阜陵,為合典故。」從之。



 寧宗恭聖仁烈皇后楊氏,紹定五年十二月崩,祔葬茂陵。



 濮安懿王園廟。治平三年,詔置園令一人,以大使臣為之。募兵二百人,以奉園為額。置柏子戶五十人。廟三間
 二廈,神門屋二所,及齋院、神廚、靈星門。其告祭濮安懿王及諸神祝文,並本宮教授撰。河南府給香幣、酒脯、禮物。太祝、奉禮則命永安縣尉、主簿攝,如闕官,以本府曹官。凡祭告及四仲饗,並依此制。奉安神主三獻,命西京差判官一員亞獻,朝臣一員終獻,攝。知園令出納神主。廟制用一品,夫人任氏墳域亦稱為園。



 元豐詔曰:「濮安懿王,先帝斟酌典禮,即園立廟,詔王子孫歲時奉祀,義協恩稱,後世無得議焉。今三夫人名位或未正,塋域或
 異處,有司置而不講,曷足以彰明先帝甚盛之德,仰承在天之志乎?三夫人可並稱曰『王夫人』,命主司擇歲月遷祔濮園,俾其子孫以時奉主與王合食,而致孝思焉。」禮官奏請,王夫人遷葬給鹵簿全仗,用鼓吹,至國門外減半。喪行與四時告享,並令嗣濮王主之。



 南渡後,主奉祠事,以嗣濮王為之;園令一員,以宗室為之;祠堂主管兼園廟香火官一員,以武臣為之。紹興二年九月,詔每歲給降福建度牒一十道,充祠堂仲饗、忌祭。五年二月,
 嗣濮王仲湜言:「被旨迎奉濮安懿王神主至行在,今已至紹興府,欲權就本處奉安。」從之。先是,神主、神貌在廬州,嗣濮王士從乞奉遷於穩便州郡安奉故也。



 十三年五月,知大宗正事、權主奉濮安懿王祠堂土□言:「濮安懿王神貌、神主權於紹興府光孝寺,仲享薦祭,其獻官、牲牢、禮料並多簡略。乞令有司討論舊制。」行下禮部、太常寺令參酌,欲令土□攝初獻,仍差士□子或從子二人攝亞、終獻。其合用牲牢,羊、豕各一;籩、豆各十,設禮料。
 初獻合服八旒冕,亞獻、終獻合服四旒冕,奉禮郎、太祝、太官令服無旒冕,並以舊制從事。從之。二十六年二月,嗣濮王士俴言:「濮安懿王祠堂,外無門牖,內闕龕帳,別無供具,望下紹興府置造修奉。」淳熙五年四月詔:「濮安懿王祠堂園廟,自今實及三年,令本堂牒紹興府檢計修葺。」從嗣濮王士輵請也。



 秀安僖王園廟。紹熙元年三月,詔秀王襲封等典禮。禮部、太堂寺乞依濮安懿王典禮,避秀安僖王名一字。詔
 恭依,仍置園廟。四月,詔:「皇伯滎陽郡王伯圭除太保,依前安德軍節度使,充萬壽觀使,嗣秀王,以奉王祀。」



 六月,禮部、太常寺言:「濮安懿王園廟制度,廟堂、神門宜並用獸。所安木主石埳,於室中西壁三分之一近南去地四尺開坎室,以石為之,其中可容神主趺匱。今來秀安僖王及夫人神主,欲乞並依上件典禮。四仲饗廟,三獻官並奉禮郎等,系嗣秀王充初獻,本位侄男攝亞、終獻,其奉禮郎等,乞湖州差官充攝。行禮合用牲羊、豕,湖州
 排辦;祭器、祭服,工部下文思院制造。每遇仲饗,本府前期牒報湖州排辦。所有行禮儀注,乞從太堂寺參照濮安懿王儀注修定。」並從之。其園廟差御帶霍漢臣同湖州通判一員相度聞奏。八月,霍漢臣暨通判湖州朱僎言奉詔相度園廟,以圖來上。十月,詔委通判一員,提督修造祠堂,如法修蓋。



 十一月,禮工部、太常寺言:「濮安懿王園廟三間二廈、神門屋二坐、齋院、神廚、靈星門,欲令湖州照應建造。」從之。三年正月一收,嗣秀王伯圭奏:「建
 造秀安僖王園廟,近已畢工,所有修制神主儀式,令所司檢照典故修制,委官題寫。」詔差權禮部尚書李巘題寫。二月,伯圭又奏:「秀安僖王祠堂園廟,乞從濮安懿王例,每三年一次,從本所移牒所屬州府檢計修造。」從之。



 莊文太子喪禮。乾道三年七月九日,皇太子薨。設素幄於太子宮正廳之東。皇帝自內常服至幄,俟時至,易服皂帕頭、白羅衫、黑銀帶、絲鞋,就幄發哀。是日,皇后服素詣宮,隨時發哀,如宮中之禮。合赴陪位官並常服、吉帶
 入麗正門,詣宮幕次,俟時至,常服、黑帶立班。俟發哀畢,易吉服,退。



 自發哀至釋服日,皇帝不視事,權禁行在音樂,仍命諸寺院聲鐘。其小斂、大斂合祭告,以本宮主管春坊官一員行禮;其餘祭告,以諸司官行禮。差護喪葬事一員,左藏庫出錢二萬貫、銀五千兩、絹五千匹。



 成服日,皇帝服期,次粗布帕頭、衣蘭衫、腰絰、絹襯衫、白羅鞋。六宮人不從服。皇太子妃及本宮人並斬
 衰三年。文武百官成服一日而除。其文武合赴官及御史臺、閣門、太常寺引班祗應人並服布帕頭、衣蘭衫,腰系布帶。本宮官僚並服齊衰三日服,臨七日而除,釋衰服後藏其服,至葬日服,葬畢而除。



 十二日,詔故皇太子攢所,就安穆皇后攢宮側近擇地。繼而都大主管所言:「太史局官等選到寶林院法堂堪充皇太子攢所。」從之。十三日,以皇太子薨告天地、宗廟、社稷、宮觀。十八日,賜謚莊文。閏七月一日,遣攝中書令、尚書右僕射魏杞奉謚
 冊、寶於皇太子靈柩前,百官常服入次,易黑帶,行禮畢,常服赴後殿門襪,進名奉慰。是夕,皇帝詣東宮行燒香之禮,如宮中之儀。



 二日,出葬,宰臣葉顒等詣靈柩前行燒香之禮。興靈訖,行事官陪位,親王、南班宗室、東宮官僚入班廳下,再拜,宰臣升詣香案前,上香、酹茶、奠酒訖,舉冊官舉哀冊,讀冊官跪讀,讀訖,宰臣再拜,各降階立。在位官皆再拜。靈柩進行,文武百僚奉辭於城外,親王、宗室並騎從至葬所。掩壙畢,辭訖,退。是日,百僚進名奉
 慰。



 四年五月,禮部、太常寺言:「國朝典故,即無皇太子小祥典禮。今參酌討論,將來莊文太子小祥日,乞皇帝前後殿特不視事。其日,先命侍從官一員常服詣太子神坐前行奠酹禮,令本宮官僚常服陪位,奠酹畢,退。次慶王、恭王常服赴神坐前奠酹畢,退。次太子妃並榮國公以下行家人禮。至大祥日,太子妃、榮國公以下及本宮人行禮畢,焚燒神帛,衰服,間月,妃及榮國公行禫祭家人禮。」從之。明年七月九日大祥,是日,皇帝不視事,差簽
 書樞密院事梁克家詣太子宮行奠酹禮,如前儀。



 景獻太子,嘉定十三年八月六日薨。其發哀制服,並如莊文太子之禮。九日,詔護喪視殯所於莊文太子攢宮之東,並依其制建造。九月十日,賜謚景獻,遣攝中書令、知樞密院事鄭昭先奉謚冊、寶於皇太子靈柩前,讀冊、讀寶如儀,訖,班退。至興靈日,宰臣詣皇太子柩前行禮畢,柩行。其宗室使相、南班官常服、黑帶,並赴陪位,騎從至葬所,俟掩攢畢,奉辭訖,退。其日,皇帝不視事,百司赴
 後殿門外立班,進名奉慰。十四年七月二日小祥,差知樞密院事鄭昭先充奠酹官。十五年八月六日大祥。九月十五日,詔景獻太子幾筵已徹,高平郡夫人傅氏可特封信國夫人,仍令主奉祭祀。



 上陵之禮。古者無墓祭,秦、漢以降,始有其儀。至唐,復有清明設祭,朔望、時節之祀,進食、薦衣之式。五代,諸陵遠者,令本州長吏朝拜,近者遣太常、宗正卿,或因行過親謁。宋初,春秋命宗正卿朝拜安陵,以太牢奉祠。乾德三
 年,始令宮人詣陵上冬服,歲以為常。開寶九年,太祖幸西京,過鞏縣,謁安陵奠獻。



 雍熙二年,宗正少卿趙安易言:「昨朝拜安陵、永昌陵,有司止設酒、脯、香,以未明行事,不設燭燎。又先赴永昌陵,後赴安陵,及帝後二位不遍拜,頗愆於禮。」事下有司,議曰:「按《開元禮》,春秋二仲月,司徒、司空巡陵,不設牲牢之祀。今請如宗廟薦享,少加裁減,除不設登鉶、牙盤食及太常登歌外,餘悉如大祠。朝拜日,有司豫於陵南百步道東設次,具翦除器以備酒
 掃。設宗正卿位於兆外之左,西向;陵官位於卿之東南,執事官又於其南,俱西向北上。設祭器、禮料、酒饌於兆門內。宗正卿以下各就位,再拜,盥手,奠酒,讀祝冊,再拜。先赴安陵,次永昌陵,次孝明、孝惠、懿德、淑德皇后陵。」從之。



 景德三年,真宗將朝諸陵,以宰臣王旦為朝拜諸陵大禮使。太常禮院言:「朝陵故事,合排小駕鹵簿。唐太宗朝獻陵,宿設黃麾仗,周衛陵寢。今請周設黃麾仗。又唐制:前一日,陵令以玉冊進御親書,近臣奉出,陵令受之。
 今請造竹冊四副,祝畢焚之。其百官位舊設陵所,從祝官及皇親、客使分於神道左右,貞觀中並陪列司馬門內。今望準舊儀施行。又舊儀,詣寢宮至大次之時,設百官位,奏請行禮。望令先入赴寢殿立班。貞觀中,皇帝至小次,素服乘馬。檢會今年正月,車駕朝拜明德攢宮,止服素白衣。當時皇帝在大祥之內,今既服除,望止服淡黃袍。又按貞觀、永徽故事,朝陵皆先親後尊,拜辭訖,出還大次,便進發,今望先朝永熙陵;行事及辭,皇帝皆兩
 次再拜,陪位官每陵亦各兩次再拜,今請皇帝詣安陵參辭,四度再拜,永昌、永熙陵各兩度設拜。舊儀,逐寢殿上食,備太牢之饌,珍羞庶品。近以羊豕代太牢。今請備少牢之祭,設奠、讀冊畢,復詣寢宮上珍羞庶品,別行致奠之禮。又舊儀,前發二日,太尉告太廟。今請依禮遍告六室。」詔特服素白衣,行事次序如告太廟,餘依所請。



 四年正月,車駕次鞏縣,罷鳴鞭及太常奏嚴、金吾傳呼。既至,齋於永安鎮行宮,太官進蔬膳。是夜,漏未盡三鼓,帝
 乘馬,卻輿輦傘扇,至安陵,素服步入司馬門行奠獻禮,諸陵亦然。又詣下宮。凡上宮用牲牢、祝冊,有司奉事;下宮備膳羞,內臣執事,百官陪位,又詣元德太后陵奠獻,別於陵西南設幄殿,祭如下宮。禮畢,遍詣孝明、孝章、懿德、淑德、明德、莊懷七後陵,遂單騎從內臣巡視陵闕,而親奠夔、魏、岐、鄆、安、周六王及恭孝太子諸墳。其三陵陪葬皇子、皇孫、公主之未出閣者,及諸王夫人之蚤亡者,各設位次諸陵下宮之東序。安陵百二十一墳,量
 設三十位,男子、女子共祝版二;昌陵十五墳,量設十位,熙陵八墳,量設五位,並祝版一以致祭焉。辰後,暫詣幄次更衣,復詣諸陵奉辭。有司以朝拜無辭禮,帝不忍,故復往。仍遣官祭一品皇親諸親墓。



 大中祥符四年正月,祀汾陰,經鞏縣,有司請于訾村王臺設幄殿,置三陵神坐,皇帝靴袍就幄,設香酒、時果、牙盤食奠獻,而命大臣以香幣、酒脯詣諸陵致告。駕還,復行親謁之禮,帝素服乘馬至永安縣,齋於行宮,夜漏未盡二鼓,詣三陵及元
 德太后、明德皇后陵奠獻,哀慟。未明,禮畢,復詣四陵奉辭,省視幾筵,奠獻如初禮。又遍詣諸後陵、諸王墳致奠。命中使遍祭皇親諸親墳及汝州秦王墳。



 是歲,命禮官定春秋二仲遣官朝陵儀注,以祭服行事,專差宗正卿一員朝拜三陵,別遣官二員分拜諸陵。又制長竿簷床二副,置陵表祝版,遣寬衣軍士三十二人輿送陵下。其後添差陵廟行禮官四員,選朝官、京官宗姓者充。



 翰林學士錢惟演言:「春秋朝陵,載於舊式,公卿親往,蓋表至
 恭。唐顯慶中,始詔三公行事,天寶以後,亦遣公卿巡謁,蓋取朝廷大臣,不必須同國姓。後參用太常、宗正卿。晉開運中,亦命吏部侍郎。近年以來,止遣宗正寺官,人輕位卑,實虧舊制,望自今於丞、郎、諸司三品內遣官,闕則差兩省諫、舍以上。所冀仰副追孝之心,以成稽古之美。」景祐初,滄州觀察使守節言:「寒食節例遣宗室拜陵,而十月令內司賓往,非所以致恭。」乃詔宗室正刺史以上一員朝拜。四年,減柏子戶,安陵、永昌、永熙各留四十戶,
 永定五十戶,會聖宮十戶。慶歷二年寒食、十月朔,宗室刺史以上,聽更往朝陵。



 皇祐三年,太常博士李壽朋奏:「帝後諸陵,薦饗皆有時,獨昭憲皇后以合葬安陵,不及時祭。」禮院言:「朝拜儀注,牲牢並如太廟常饗例,諸陵止奠一爵,而安陵奠兩爵,兩贊再拜,惟祭饌不兼設,蓋有司相承失之。」於是詔安陵昭憲皇后祝版、牲幣、御封香依太廟同室禮。更造諸陵祭器貯別庫。三陵皆置卒五百人,唯定陵以章獻太后故,別置一指揮。昭陵使甘昭
 吉引定陵例,請置守陵奉先兩指揮,京西轉運司請減定陵卒半以奉昭陵,詔選募一指揮,額五百人。



 初,永安縣官月朔朝定陵,望朝三陵。韓琦言:「昭陵未有朝日。」乃令縣官朔望分朝諸陵。熙寧中,詔文臣大兩省、武臣閣門使以上,經過陵下,並許朝拜。又詔:「自今臣僚朝拜諸陵,除見任、嘗任執政官許進湯,餘止奠獻、薦新,不特拜。」



 初,故事,車駕詣陵,謂之親謁。南渡之後,此禮不舉,故上陵或曰省視,或曰保護,或曰薦獻,或曰祭告,或曰致祭,
 或曰望祭,或曰修奉,悉遣官,不專於行禮也。建炎元年五月一日詔:「應永安軍祖宗陵寢,可差西京留守及臺臣一員躬親省視,如有合修奉去處,措置奏聞。」仍詔鄜延路副總管劉光世充省視陵寢使。又詔河南府鎮撫使翟興團結本處義兵,保護祖宗陵寢。四年六月,詔令禮部給降度牒一百道充祭告諸陵禮料,仍令翟興所差來人賚祭告表以行。



 紹興元年九月,起居郎陳與義言:「陛下躬履艱難之運,駐蹕東南,列聖陵邑,遠在洛師,
 顧瞻山川,未得時省。雖欲遣使,道路不通,聖懷日憤。近聞道路少通,差易前日,願詔執事每半年擇遣使臣兩員,往省諸陵。」詔令樞密院每半年差使臣兩員前去。三年正月,禮部、太常寺言:「春秋二仲,薦獻諸陵,乞於行在法惠寺設位,望祭行禮。」從之。自是每歲薦獻,率循此制。五月,詔令戶部支金一百兩付河南府鎮撫使司干辦公事任直清,充祭告永安軍諸陵。



 九年正月,上謂輔臣曰:「祖宗陵寢,久淪異域,今金國既割還故地,便當遣宗
 室使相與臣僚前去修奉灑掃。」尋命同判大宗正事士人褭、兵部侍郎張燾前云河南府祗謁修奉。六月,太常丞梁仲敏等言:「春秋二仲,遣宗室遙郡防禦使薦獻諸陵,太常少卿薦獻永祐陵,權宜於行在設位行禮。今道路既通,望依舊遣官前詣。」詔令西京留守司候仲秋就便選官前詣諸陵薦獻。士人褭、張燾回,言:「諸陵下石澗水,自兵興以來,涸竭幾十五年。二使到日,水即大至,父老驚嘆,以為中興之祥。」



 十年三月,禮部言:「池州銅陵縣丞呂
 和問進宮陵儀制,望付太常寺以備檢照。永安軍等處今已收復,遂委知軍詣諸陵逐位檢視,除永定、永昭、永厚、永裕、永泰陵園廟並無損動,內永安、永昌、永熙陵神臺璺裂,未敢一面擅行補飾。太常寺看詳若行補修,合就差所委修飾官奏告行禮。」詔令河南府委官,如法補飾,不得滅裂。其後兵部侍郎兼史館修撰張燾言:「伏見宣諭官方庭實有請,乞將來先帝山陵,一依永安陵等制度。臣區區愚忠,願明詔有司,異時永固陵凡金玉珍
 寶盡斥不用,播告天下,咸使聞知。如是,自然可保無虞。」上嘉納之。三十二年六月,詔祖宗陵寢,令本處招討使同本處官吏躬親朝謁,如法修奉,務在嚴潔,以稱孝思之意。



 乾道六年八月,詔承信郎劉湛特轉兩官,右迪功郎劉師顏特與右承務郎升擢差遣,秦世輔特轉一官,升充正將,以湛等歸正結義保護陵寢故也。



 端平元年正月,京西湖北安撫制置使史嵩之露布以滅金聞。二月,御筆:「國家南渡以後,八陵迥隔,常切痛心。今京湖帥
 臣以圖來上,恭覽再三,悲喜交集,凡在臣子,諒同此情。可令卿、監、郎官以上,詣尚書省恭視集議。」遂遣太常寺主簿朱揚祖、閣門祗候林拓朝謁八陵。



 紹興元年六月,太常寺言:「昭慈獻烈皇太后攢宮在越州會稽縣,合依四孟朝獻禮例,差宰執一員,前一日赴攢宮泰寧寺宿齋,至日,行朝拜之禮。」詔同知樞密院事李回行禮。二年三月,知紹興府張守言:「昭慈獻烈皇后攢宮近在府界,望許臣以時朝謁。」從之。自是守臣皆許朝謁。



 十七年十
 一月,殿中侍御史餘堯弼言:「望舉行舊制,於春秋二仲遣官詣永祐陵攢宮薦獻。」臣僚又言:「陵廟之祭,月有薦新,著在令典。方今宗廟久已遵奉,惟是永祐陵闕而未講,望令有司討論,舉而行之。」太常寺討論:「欲依《政和五禮》依典故,令兩攢宮遵依每月檢舉,差官行禮,其新物令逐宮預行關報紹興府排辦。」從之。



 二十七年六月,詔:「永祐陵及昭慈聖獻皇后攢宮檢察承受,以檢察宮陵所為名。」



 三十年九月,吏部言:「紹興府會稽知縣依仿陵
 臺令典故,於階銜內帶兼主管攢宮事務,量加優異。」



 淳熙元年正月,禮部、太常寺言:「春秋二仲,差太常少卿薦獻永祐陵攢宮,並周視陵域。如遇少卿有缺,乞從本寺前期取指揮。差本寺以次官充攝。所有今年仲春薦獻,即日見闕少卿。」詔差太常丞錢良臣。自後春秋遇少卿闕,率以為例。



 慶元元年六月,詔:「永阜陵孝宗皇帝攢宮,每歲秋季一就,令所差監察御史恭詣朝拜檢察。」從御史臺申請。諸陵亦如之。



 忌日,唐初始著罷樂、廢務及行香、修齋之文。基後,又朔望停朝,令天下上州皆準式行香。天祐初,始令百官詣閣奉慰。宋循其制,惟宣祖、昭憲皇后為大忌。前一日不坐,群臣詣西上閣門奉慰,移班奉慰皇太后,退,赴佛寺行香。凡大忌,中書悉集;小忌,差官一員赴寺。如車駕巡幸道遇忌日,皆不進名奉慰。留守自於寺院行香,仍不得在拜表之所。天下州府軍監亦如之。



 建隆二年,宣祖忌日,時明憲太后在殯,群臣止詣閣奉慰而罷行香。乾
 德二年,禘於太廟,其日,惠明皇后忌,有司言:「唐開成四年正月二十二日祀先農,與穆宗忌同日;太和七年十二月八日蠟百神,與敬宗忌同日。詔以近廟忌辰,作樂非便,宜令縣而不作。竊以農、蠟之祭,猶避廟忌而不作樂,況僖祖同廟連室而在諱辰,詎可輒陳金石之奏?伏望依禮縣而不作。」其後,宣祖、昭憲忌日,詔準太祖、太宗奉翼祖禮,前一日更不廢務。



 咸平中,有司將設春宴,金明池習水戲,開瓊林苑,縱都人游賞。帝以是月太宗忌
 月,命詳定故事以聞。史館檢討杜鎬等言:「按晉穆帝納後月,是康帝忌月,禮官荀訥議:『有忌日,無忌月;若有忌月,即有忌時、忌歲,益無所據。』當時從訥所議。唐武后神功元年,建安王攸宜破契丹,詣闕獻捷,軍人入城,例有軍樂,內史王及善以國家忌月,請備而不奏。鳳閣侍郎王方慶奏:『按《禮經》,有忌日而無忌月。』遂舉樂。憲宗時,太常博士韋公肅言:「《禮》無忌月禁樂,今太常教坊以正月為忌月,停郊廟饗宴之禮,中外士庶咸罷宴樂,竊恐乖
 宜。』時依公肅所奏。伏以忌日不樂,嘗載《禮經》;忌月徹縣,實無典故。況前代鴻儒,議論足據。其春宴及池苑,並合舉樂。」



 景德元年,北征凱旋京師,是日,以懿德皇后忌,詔徹鹵簿、鼓吹。禮官議曰:「班師振旅,國之大事,後之忌日,家之私事。今大賀凱旋,軍容宜肅。昔武王伐紂在諒冱中,猶前歌後舞。夫諒冱是重,遠忌是輕,以此而論,舉樂無爽。況《春秋》之義,不以家事辭王事,其還京日,法駕、鼓吹、音樂,並請振作。」



 尋詔:「自今宗廟忌日,西京及諸節鎮
 給錢十千,防禦、團練州七千,軍事州五千,以備齋設。」元德皇后忌日,舊制,樞密使依內諸司例,惟進名,不赴行香,知樞密院王欽若以為言。自是,三司使副、翰林樞密龍圖直學士並赴焉。真宗崩,元德、明德皇后忌日在禫制內,乃停進名行香。凡奉慰,宰相、樞密使各帥百官、內職共進名,節度使、留後、觀察使各進名。



 忌日前後,各禁刑三日如天慶節,釋杖以下情輕者,復斷屠宰,不視事前後各三日,禁樂各五日。其後,以歲月漸遠,禁刑、不視
 事各二日,禁樂各三日。章憲明肅太后忌辰,禮官請依章懿太后禮例,前後各二日不視事,一日禁屠宰,各三日禁樂。詔:應大忌日,行香,臣僚並素食。復立孝惠、孝章、淑德、章懷、章惠、溫成諸後為小忌,未幾,罷。神宗即位,太常禮院言:「僖祖及文懿皇后神主既祧,準禮不諱,忌日亦請依唐睿宗祧遷故事廢之。」



 初,神御殿酌獻,設皇帝位於庭下,而忌日兩府列於殿上;寺院行香,左右巡使、兩赤縣令於中門相向分立,俟宰臣至,立位前,直省官贊
 通揖,於禮無據。乃命行香群臣班殿下,宰相一員升殿跪爐,而罷通揖。又詔:大忌日不為假,執政官蚤出。禮部言:「順祖及惠明皇后既葬遷主,罷行香。忌日,請於永昌院佛殿之東張幄齋薦。」乃詔:「僖祖、翼祖並後六位忌日咸如之。」先是,翼祖、簡穆皇后神主奉藏夾室,依禮不忌。後復詔還本室,而忌日亦如舊焉。



 《政和新儀》:群臣進名奉慰,其日質明,文武朝參官入詣朝堂就次。御史臺先引殿中侍御史一員入就位,次西上閣門、御史臺分引朝
 參官及諸軍將校,次禮直官引三公以下在西上閣門南階下,每等重行異位,並北向東上。知西上閣門官於班前西向立,搢笏,執名紙,躬。三公以下文武百僚俱再拜,俊閣門官執笏、置名紙笏上、入西上閣門訖,退。群臣奉慰詣景靈宮,每等重行異位,並北向東上。禮直官揖班首以下再拜訖,引班首自東階升殿,舍人接引同升,詣香案前,搢笏,上香,跪奠茶訖,執笏興,降階復位,又再拜;次引班首以下分左右搢笏,行香,宰相、執政官分左
 右行香訖,執笏俱復位;次引班首升殿,詣香案前俯伏,跪,搢笏,執爐,俟讀疏畢,執笏俯伏,興,降階復位,又再拜,退。



 中興之制:忌日,百僚行香,在外州軍亦詣寺院行香,如在以日易月服制之內,並依禮例權停。大祥後次年,於歷日內箋注立忌辰,禁音樂一日。紹興元年二月,太常少卿蘇遲等以徽宗、欽宗留北,有朔望遙拜之禮,乃言:「凡遇祖宗帝後忌,前一日並忌日,皇帝自內先服紅袍遙拜訖,易服行禮。」從之。二年八月,詔:「應諸路州、軍見
 屯軍馬統兵官,每遇國忌,免行香。」



 十三年正月,御史臺言:「正月十三日,欽聖憲肅皇后忌,其日立春。準令,諸臣僚及將校立春日賜幡勝,遇稱賀等拜表、忌辰奉慰退即戴。欲乞候十三日忌辰行香退,即行戴插。」從之。三十一年六月,禮部侍郎金安節等言:「六月二十八日,欽慈皇后忌辰,系在淵聖皇帝以日易月釋服之外,百官行香,宜如常制。」詔依。三十二年正月,禮部、太常寺言:「已降旨:欽宗祔廟,翼祖當遷。於正月九日告遷翼祖皇帝、簡
 穆皇后神主奉藏於夾室,所有以後翼祖皇帝忌及諱、簡穆皇后忌,欲乞依禮不諱、不忌。」詔恭依。



 淳熙元年十一月詔:「文武百僚詣景靈宮國忌立班行香,自今如遇宰執俱致齋不及趁赴,於東班從上引官一員升殿跪爐行香,以次官一員詣西班行香。」先是,閣門得旨:國忌行香,宰執致齋不赴,其西壁武臣闕官押班,已降指揮,差使相或太尉、節度使等押班,可令文武班內班上一員東壁押班,止令西壁散香,今後準此。至是,禮部、太常
 寺重別指定來上,故有是命。



 四年十月,太常少卿齊慶冑言:「每遇國忌,文臣班列莫敢不肅,唯是武臣一班員數絕少,或以疾病在告,多不趁赴。」詔閣門、御史臺申嚴行下,如有違戾,彈劾聞奏。九年十月,侍御史張大經奏:「比來國忌行香日分,合赴官類多托疾在告,以免夙興拜跪之勞。乞自今如遇行香日,有稱疾托故不赴者,從本臺彈奏,乞置典憲。」從之。



 群臣私忌。開寶敕文:「應常參官及內殿起居職官等,
 自今刺史、郎中、將軍以下遇私忌,請準式假一日。忌前之夕,聽還私第。」其後有司言:「臣僚忌日恩賜,其間甚有無名者:如劉繼元、李煜、劉鋹之類,皆身為降俘,亡沒已久,而尚沾恩賜;及周朝忌日,尚有追薦;本朝亦有追尊皇后生日道場,並諸神祠亦有為生日者。請付禮官詳議,不經之物,一切省去。」詔周朝忌日仍舊,餘罷之。



\end{pinyinscope}