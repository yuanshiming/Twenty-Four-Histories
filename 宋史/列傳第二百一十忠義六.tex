\article{列傳第二百一十忠義六}

\begin{pinyinscope}

 ○趙良淳徐道隆姜才馬塈密祐張世傑陸秀夫徐應鑣陳文龍鄧得遇
 張玨



 趙良淳,字景程,居饒之餘干,太宗子恭憲王之後,丞相汝愚曾孫也。累世以學行名,號賢宗子。良淳少學於其鄉先生饒魯,知立身大節。及仕,所至以乾治稱,而未嘗干人薦舉。初以蔭為泰寧主簿,三遷至淮西運轄,浮湛冗官二十餘年。馬光祖、李伯玉、範丁孫交薦闢之,卒不振拔。考舉及格,改知分寧縣。分寧,江西劇邑,俗尚嘩訐,良淳治之,不用刑戮,不任吏胥,取民之敦孝者,身親尊
 禮之,至甚傑驁者,乃繩以法,俗為少革。秩滿,特差權江西安撫司機宜文字,詔除諸司審計院,督餉江西,升大理司直。



 咸淳末,廷臣議眾建宗室於內郡,以為屏翰,遂除良淳知安吉州。先是,知州李庚遁,百事隳廢。良淳至,日與僚吏論所以守禦之備,悉舉行之。時歲饑,民相聚為盜,所在蜂起。或請以兵擊之,良淳曰:「民豈樂為盜哉?時艱歲旱,故相率剽掠茍活耳。」命僚屬以義諭之,眾皆投兵散歸,其不歸者眾縛以獻。有掠人貨財詣其主謝
 過而還之者。良淳勸富人出粟振之,嘗語人曰:「使太守身可以濟民,亦所不惜也。」其言懇懇,足以動人,人皆倒囷以應之。朝議尋以徐道隆為浙西提刑,以輔良淳,加良淳直秘閣。



 文天祥去平江,潰兵四出剽掠,良淳捕斬數人,梟首市中,兵稍戢。已而範文虎遣使持書招降,良淳焚書斬其使。大兵迫獨松關,有旨趣道隆入衛。道隆既去,大兵至,軍其東西門。良淳率眾城守,夜就茇舍陴上,不歸。



 先是,朝廷遣將吳國定援宜興,宜興已危,不敢
 往,乃如安吉見良淳,願留以為輔。良淳見國定慷慨大言,意其可用也,請於朝,留戍安吉。已而國定開南門納外兵,兵入城呼曰:「眾散,元帥不殺汝。」於是眾號泣散去。良淳命車歸府,兵士止之曰:「事至此,侍郎當為自全計。」良淳叱去之。命家人出避,乃閉閣自經。有兵士解救之,復蘇,眾羅拜泣曰:「侍郎何自苦?逃之猶可求生。」良淳叱曰:「我豈逃生者邪?」眾猶環守不去,良淳大呼曰:「爾輩欲為亂邪?」眾涕泣出,復投繯而死。



 徐道隆字伯謙,婺州武義人。父煥,知南雄。道隆以任入官,累官潭州判官、權知全州。荊湖制置使汪立信奏闢道隆為參議官。立信遷兵部尚書,道隆與賓客十許人俱去江陵。趙孟傳為制置使,以道隆參其軍事,遂為提點刑獄。



 時文天祥既去平江,潰卒四出,為浙西患苦,安吉尤甚。有旨令道隆措置,乃梟其首亂者於市。牛監軍遁,範文虎、程鵬飛、管景模俱遺書誘降,道隆焚書斬使。



 大兵至臨平皋亭山,令間道入援,時水陸皆有屯軍,道
 絕不通,議由太湖經武康、臨安縣境勤王。即日乘舟出臨湖門,泊宋村。郡守趙良淳既縊死。德祐二年正月朔旦,追兵及道隆,江陵親從軍三百人殊死戰,矢盡槍槊折,一軍盡沒。道隆見執艦內,間守者少怠,赴水死,長子載孫亦赴水死。餘兵有脫歸者言於朝,命贈官賜謚,厚恤其家,立廟安吉,官其子孫。越三日宋亡。



 姜才,濠州人。貌短悍。少被掠入河朔,稍長亡歸,隸淮南兵中,以善戰名,然以來歸人不得大官,為通州副都統。
 時淮多健將,然驍雄無逾才。才知兵,善騎射,撫士卒有恩,至臨陣,軍律凜凜。其子當戰,回白事,才望見以為敗也,拔劍馳逐,幾殺之。



 賈似道出師,才以兵屬孫虎臣為先鋒,相拒於丁家洲。大軍設炮架彀車弩江濱,中流數千艘,旌旗聯亙,鼓行而下。才奮兵前接戰,鋒已交,虎臣遽過其妾所乘舟,眾見之,歡曰:「步帥遁矣。」於是諸軍皆潰,才亦收兵入揚州。大兵乘勝攻揚州,才為三疊陣逆之三里溝,戰有功。又與元帥戰揚子橋,日暮兵亂,流矢
 貫才肩,才拔矢揮刀而前,所向闢易。已而大軍築長圍,自揚子橋竟瓜洲,東北跨灣頭至黃塘,西北至丁村,務欲以久困之,時德祐元年也。



 明年正月,宋亡。二月,五奉使及一閣門宣贊舍人持謝太后詔來諭降,才發弩射卻之,復以兵擊五奉使於召伯堡,大戰而退。未幾,瀛國公至瓜洲,才與庭芝泣涕誓將士出奪之。將士皆感泣。乃盡散金帛犒兵,以四萬人夜搗瓜洲,戰三時,眾擁瀛國公避去。才追戰至浦子市,夜猶不退。阿術使人招之,
 才曰:「吾寧死,豈作降將軍邪!」四月,才以兵攻灣頭柵。五月,復攻之,騎旋濘而止,乃舍騎步戰,至四鼓,全師以歸。揚食盡,才時出運米真州、高郵以給兵。六月,護餉至馬家渡,萬戶史弼將兵擊奪之,才與戰達旦,弼幾殆,阿術馳兵來援,乃得免去。



 庭芝以在圍久。召才計事,屏左右,語久之,第聞才厲聲云:「相公不過忍片時痛耳。」左右聞之俱汗下。才自是以兵護庭芝第,期與俱死。



 七月,益王在福州,以龍神四廂都指揮使、保康軍承宣使召才,才
 與庭芝東至泰州,將入海。阿術以兵追及,圍泰州,使使者招之降,才不聽。阿術驅揚兵士妻子至城下,會才疽發脅不能戰,諸將遂開門降。都統曹安國入才臥內,執之以獻。阿術愛其忠勇,欲降而用之,才肆為慢言;阿術責庭芝不降,才曰:「不降者才也。」復憤憤不已,阿術怒,剮之揚州。才臨刑,夏貴出其傍,才切齒曰:「若見我寧不愧死邪?」



 有洪福者,夏貴家僮也,從貴積勞為鎮巢雄江左軍統制,鎮江北。貴降,福與子大淵、大源、下班祗候彭元亮
 結貴軍復之,加右武大夫、知鎮巢。貴既臣附,招福,不聽,使其從子往,福斬之。大兵攻城,久不拔,遣貴至城下,好語語福,請單騎入城。福信之,門發而伏兵起,執福父子,屠城中。貴涖殺,大源、大淵謔曰:「法止誅首謀,何至舉家為戮?」福叱曰:「以一命報宋朝,何至告人求活邪?」次及福,福大罵數貴不忠,請身南向死,以明不背國也。聞者流涕。



 馬塈,宕昌人也。一家父叔兄弟皆以忠勇為名將,而塈
 與其兄堃特顯。咸淳中,塈知欽州,徙知邕。邕地接六詔、安南,傍通諸溪峒,撫御少失宜,往往召亂。塈鎮撫諸蠻及治關隘,皆有條理,大理不敢越善闡,安南不敢入永平,諸峒皆上帳冊,邊陲晏然。廣西經略李興上其功,加閣門宣贊舍人。未幾,以左武衛將軍征入朝。已而宋亡,塈因留靜江,總屯戍諸軍,護經略司印守城。



 至元十四年,平章阿里海牙攻廣西。塈發所部及諸峒兵守靜江,而自將三千人守嚴關,鑿馬坑,斷嶺道。大兵攻嚴關不
 克,乃以偏師入平樂,過臨桂,夾攻塈。塈兵敗,退保靜江。平章使人招降,塈發弩射之。攻三月,塈夜不解甲,前後百餘戰,城中死傷相籍,訖無降意。城東隅稍卑,大軍陽攻西門,以精兵夜決水閘,攻東門,破其外城;塈閉內城城守,又破之。塈率死士巷戰,刀傷臂被執,殺之斷其首,猶握拳奮起,立逾時始僕。靜江破,邕守馬成旺及其子都統應麒以城降,獨塈部將婁鈴轄猶以二百五十人守月城不下。阿里海牙笑曰:「是何足攻。」圍之十餘日,婁
 從壁上呼曰:「吾屬饑,不能出降,茍賜之食,當聽命。」乃遺之牛數頭,米數斛。一部將開門取歸,復閉壁。大軍乘高視之,兵皆分米,炊未熟,生臠牛,啖立盡。鳴角伐鼓,諸將以為出戰也,甲以待。婁乃令所部入擁一火炮然之,聲如雷霆,震城土皆崩,煙氣漲天外,兵多驚死者。火熄入視之,灰燼無遺矣。



 密祐,其先密州人,後渡淮居廬州。祐為人剛毅質直,累官至廬州駐札、御前游擊中軍統領,改權江西路副總
 管。



 咸淳十年,以閣門宣贊舍人為江西都統。是冬,大元丞相伯顏下鄂州,留右丞阿里海牙守之,而將大兵東下。明年二月,朱祀孫遣高世傑取鄂州,阿里海牙以兵逆擊,執世傑荊江口,兵盡潰,半入江西。江西制置黃萬石招集之,且募寧都、廣昌、南劍義兵千餘人,盡以屬祐。十一月,大兵至隆興,劉槃兵敗,乃嬰城自守。萬石時移治撫州,將遁,懼祐不從,乃調祐兵援槃,且戒以勿戰。未至隆興,槃已降,都統夏驥率所部兵潰圍出。



 已而元帥張
 榮實、呂師夔提兵逼撫州,祐率眾逆之進賢坪,兵來呼曰:「降者乎?鬥者乎?」祐曰:「斗者也。」麾其兵突戰,進至龍馬坪,大兵圍之數重,矢下如雨。祐告其部曰:「今日死日也,若力戰,或有生理。」眾咸憤厲。自辰戰至日昃,祐面中矢,拔之復戰,又身被四矢三槍,眾皆死,僅餘數十人。祐乃揮雙刀斫圍南走,前渡橋,馬踏板斷,遂被執。眾見其勇,戒勿殺,輿歸隆興。元帥宋都OA曰:「壯士也。」欲降之,系之月餘,終不屈。嘗罵萬石為賣國小人,使我志不得伸。宋
 都OA命劉槃、呂師夔坐城樓,引祐樓下,以金符遺之,許以官,祐不受,語侵般、師夔,益不遜。又令祐子說之曰:「父死,子安之?」祐斥曰:「汝行乞於市,第云密都統子,誰不憐汝也。」怡然自解其衣請刑,遂死。觀者皆泣下。



 張世傑,範陽人。少從張柔戍杞,有罪,遂奔宋,隸淮兵中,無所知名。阮思聰見而奇之,言之呂文德,文德召為小校。累功至黃州武定諸軍都統制。攻安東州,戰疾力,與高達援鄂州有功,轉十官。尋從賈似道入黃州,戰萍草
 坪,奪還所俘,加環衛官,歷知高郵軍、安東州。



 咸淳四年,大軍築鹿門堡,呂文德請益兵於朝,調世傑與夏貴赴之。及呂文煥以襄陽降,命世傑將五千人守鄂州。世傑以鐵絙鎖兩城,夾以炮弩,其要津皆施杙,設攻具。大軍破新城,長驅而下,世傑力戰,不得前,遣人招之,不聽。丞相伯顏陽攻嚴山隘,潛自唐港蕩舟入漢,東攻鄂,鄂降。



 世傑提所部兵入衛,道復饒州,乃入朝。時方危急,徵諸將勤王多不至,獨世傑來,上下嘆異。自和州防禦使
 不數月累加至保康軍承宣使,總都督府兵。遣將四出,取浙西諸郡,復平江、安吉、廣德、溧陽諸城,兵勢頗振。七月,與劉師勇諸將大出師焦山,令以十舟為方,碇江中,非有號令毋發碇,示以必死。元帥阿術載彀士以火矢攻之,世傑兵亂,無敢發碇,赴江死者萬餘人。大敗,奔圌山。上疏請濟師,不報。尋擢龍、神衛四廂都指揮使。十月,進沿江招討使,改制置副使、兼知江陰軍。已而大軍至獨松關,召文天祥入衛,以世傑為保康軍節度使、知平
 江。尋亦召入衛,加檢校少保。



 二年正月,大軍迫臨安,世傑請移三宮入海,而與天祥合兵背城一戰。丞相陳宜中方遣人請和,不可,白太皇太后止之。未幾,和議亦沮。兵至皋亭山,世傑乃提兵入定海。石國英遣都統卞彪說之使降,世傑以為彪來從己俱南也,椎牛享之,酒半,彪從容為言,世傑大怒,斷其舌,磔之巾子山。



 四月,從二王入福州。五月,與宜中奉昰為主,拜簽書樞密院事。王世強導大軍攻之,世傑乃奉益王入海,而自將陳吊眼、
 許夫人諸畬兵攻蒲壽庚,不下。十月,元帥唆都將兵來援泉,遂解去。既而唆都遣人招益王,又遣經歷孫安甫說世傑,世傑拘安甫軍中不遣。招討劉深攻淺灣,世傑兵敗,移王居井澳,深復來攻井澳,世傑戰卻之,因徒碙洲。



 至元十五年正月,遣將王用攻雷州,用敗績。四月,益王殂,衛王昺立,拜世傑少傅、樞密副使。五月,遣瓊州安撫張應科攻雷州,三戰皆不利。六月,再決戰雷城下,應科死之。世傑以碙洲不可居,徙王新會之崖山。八月,封
 越國公。發瓊州粟以給軍。十月,遣凌震、王道夫襲廣州,震敗績。



 明年,元帥張弘範等兵至崖山,或謂世傑曰:「北兵以舟師塞海口,則我不能進退,盍先據海口。幸而勝,國之福也;不勝,猶可西走。」世傑恐久在海上有離心,動則必散,乃曰:「頻年航海,何時已乎?今須與決勝負。」悉焚行朝草市,結大舶千餘作水砦,為死守計,人皆危之。已而弘範兵至,據海口,樵汲道絕,兵茹乾糧十餘日,渴甚,下掬海水飲之,海咸,飲即嘔洩,兵大困。世傑率蘇劉義、
 方興日大戰。弘範得世傑甥韓,命以官,使三至招之,世傑歷數古忠臣曰:「吾知降,生且富貴,但為主死不移耳。」二月癸未,弘範等攻崖山,世傑敗,走衛王舟。大軍薄中軍,世傑乃斷維,以十餘艦奪港去。後還收兵崖山,劉自立擊敗之,降其將方遇龍、葉秀榮、章文秀等四十餘人。世傑復欲奉楊太妃求趙氏後而立之,俄颶風壞舟,溺死平章山下。



 劉師勇者,廬州人。以戰功歷環衛官。魯港師潰,賈似道欲東入海,師勇贊之入揚州圖再舉,似道
 然之。時姚訔復常州,似道命師勇以淮兵取呂城,朝廷加師勇和州防禦使,助訔守常,而以張彥守呂城,合兵拒大軍。戰失利,彥馬弱,陷淖中見執,呂城失守,常州勢益孤。大軍置彥城下招降,師勇以大義斥彥,彥慚而退。又遣範文虎來諭,師勇伏弩射走之。常受圍數月,援兵絕,有群鴟飛鳴繞城,眾惡為不祥,俄而城陷。師勇拔柵,戰且行,其弟馬墮塹,躍不能出,師勇舉手與訣而去。淮軍數千人皆鬥死。有婦人伏積尸下,窺淮兵六人反背
 相拄,殺敵十百人乃殪。師勇從二王至海上,見時事不可為,憂憤縱酒卒,葬於鼓山。



 陸秀夫,字君實,楚州鹽城人。生三歲,其父徙家鎮江。稍長,從其鄉人孟先生學,孟之徒恆百餘,獨指秀夫曰:「此非凡兒也。」景定元年,登進士第。李庭芝鎮淮南,聞其名,闢置幕中。時天下稱得士多者,以淮南為第一,號「小朝廷」。



 秀夫才思清麗,一時文人少能及之。性沉靜,不茍求人知,每僚吏至閣,賓主交歡,秀夫獨斂焉無一語。或時
 宴集府中,坐尊俎間,矜莊終日,未嘗少有希合。至察其事,皆治,庭芝益器之,雖改官不使去己,就幕三遷至主管機宜文字。咸淳十年,庭芝制置淮東,擢參議官。德祐元年,邊事急,諸僚屬多亡者,惟秀夫數人不去。庭芝上其名,除司農寺丞,累擢至宗正少卿兼權起居舍人。



 二年正月,以禮部侍郎使軍前請和,不就而反。二王走溫州,秀夫與蘇劉義追從之,使人召陳宜中、張世傑等皆至,遂相與立益王於福州。進端明殿學士、簽書樞密院
 事。宜中以秀夫久在兵間,知軍務,每事咨訪始行,秀夫亦悉心贊之,無不自盡。旋與議宜中不合,宜中使言者劾罷之。張世傑讓宜中曰:「此何如時,動以臺諫論人?」宜中皇恐,亟召秀夫還。



 時君臣播越海濱,庶事疏略,楊太妃垂簾,與群臣語猶自稱奴。每時節朝會,秀夫儼然正笏立,如治朝,或時在行中,淒然泣下,以朝衣拭淚,衣盡浥,左右無不悲動者。屬井澳風,王以驚疾殂,群臣皆欲散去。秀夫曰:「度宗皇帝一子尚在,將焉置之?古人有以
 一旅一成中興者,今百官有司皆具,士卒數萬,天若未欲絕宋,此豈不可為國邪?」乃與眾共立衛王。時陳宜中往占城,以與世傑不協,屢召不至。乃以秀夫為左丞相,與世傑共秉政。時世傑駐兵崖山,秀夫外籌軍旅,內調工役,凡有所述作,又盡出其手。雖匆遽流離中,猶日書《大學章句》以勸講。



 至元十六年二月,崖山破,秀夫走衛王舟,而世傑、劉義各斷維去,秀夫度不可脫,乃杖劍驅妻子入海,即負王赴海死,年四十四。



 翰林學士劉鼎孫亦
 驅家屬並輜重沉海,不死被執,搒掠無完膚,一夕得脫,卒蹈海。鼎孫字伯鎮,江陵人,進士也。



 方秀夫海上時,記二王事為一書甚悉,以授禮部侍郎鄧光薦曰:「君後死,幸傳之。」其後崖山平,光薦以其書還廬陵。大德初,光薦卒,其書存亡無從知,故海上之事,世莫得其詳云。



 徐應鑣,字巨翁,衢之江山人,世為衢望族。咸淳末,試補太學生。德祐二年,宋亡,瀛國公入燕,三學生百餘人皆從行。應鑣不欲從,乃與其子琦、崧、女元娘誓共焚,子女
 皆喜從之。



 太學故岳飛第,有飛祠,應鑣具酒肉祀飛曰:「天不祐宋,社稷為墟,應鑣死以報國,誓不與諸生俱北。死已,將魂魄累王,作配神主,與王英靈,永永無斁。」琦亦賦詩以自誓。祭畢,以酒肉餉諸僕,諸僕醉臥,應鑣乃與其子女入梯雲樓,積諸房書籍箱笥四周,縱火自焚。一小僕未寐,聞火聲,起至樓下穴牖視之,應鑣父子儼然坐立,如廟塑像。走報諸僕,壞壁入,撲滅火。應鑣不得死,與其子女怏怏出戶去,倉卒莫知所之,翌日得其尸祠
 前井中,皆殭立瞠目,面如生。諸僕為具棺斂,殯之西湖金牛僧舍。益王立福州,褒其節,贈朝奉郎、秘閣修撰。後十年,其同舍生劉汝鈞率儒者五十餘人收而葬之方家峪,私謚曰正節先生。



 陳文龍字君賁,福州興化人。丞相俊卿之後也。能文章,負氣節。初名子龍,咸淳五年廷對第一,度宗易其名文龍。



 丞相賈似道愛其文,雅禮重之。由鎮東軍節度判官、歷崇政殿說書、秘書省校書郎。數年,拜監察御史,皆出
 似道力。然自十數年,似道所置臺諫皆闒茸,臺中相承,凡有所建白,皆呈稿似道始行。至文龍為之,獨不呈稿,已忤似道。知臨安府洪起畏請行類田,似道主其說,文龍上疏以為不可,似道怒,寢其疏。襄陽久被圍,似道日恣淫樂,不少加意,時陽請督師,而陰使其黨留己,竟失襄陽。文龍上疏極言其失。範文虎總師無功,似道芘之,以知安慶,又除趙溍知建康,黃萬石知臨安。文龍言:「文虎失襄陽,今反見擢用,是當罰而賞也。溍乳臭小子,何
 以任大閫之寄?萬石政事怠荒,以為京尹,何以能治?請皆罷之。」似道大怒,黜文龍知撫州,旋又使臺臣李可劾罷之。未幾,呂文煥導大軍東下,範文虎首迎降,與文煥俱東。似道兵潰魯港,溍最先遁,以故列城從之皆遁,始悔不用文龍之言。起為左司諫,尋遷侍御史。



 時邊事甚急,王爚與陳宜中不能畫一策,而日坐朝堂爭私意。潛說友以平江降,臺臣請籍其家,爚以為可,宜中以為不可。張世傑諸將分四道出師,而大臣不監護,臺諫論之,
 爚請行邊,下公卿雜議,宜中請出督師,又下公卿雜議。文龍上疏曰:「《書》言『三后協心,同底於道。』北兵今日取某城,明日築某堡,而我以文相遜,以跡相疑,譬猶拯溺救焚,而為安步徐行之儀也。請詔大臣同心圖治,無滋虛議。」其後宜中與爚終不相能而去,至十月始來,事已不可為矣。



 是冬,累遷文龍至參知政事。未幾議降,文龍乃上章乞歸養,既出國門而悔之,復上疏求還,不報,乃歸。五月,益王稱制於福州,復以文龍參知政事。漳州叛,以
 文龍為閩、廣宣撫使討之。文龍以黃恮前守漳有恩信,闢為參謀官。按兵泉州,使恮入招撫之,恮至,民皆頓首謝罪。興化有石手軍者,能擲石中人,議者以其不足用罷之,石手軍亦叛,復命文龍為知軍,平之。



 已而降將王世強復導大軍入廣,建寧、泉、福皆降。知福州王剛中遣使徇興化,文龍斬之而縱其副以還,使持書責世強、剛中負國。遂發民兵自守,城中兵不滿千,大兵來攻不克,使其姻家持書招降之,文龍焚書斬其使。有風其納款
 者,文龍曰:「諸君特畏死耳,未知此生能不死乎?」乃使其將林華偵伺境上。華即降,且導兵至城下,通判曹澄孫開門降,執文龍與其家人至軍中,欲降之,不屈,左右凌挫之,文龍指其腹曰:「此皆節義文章也,可相逼邪?」強之,卒不屈,乃械系送杭州。文龍去興化即不食,至杭餓死。其母系福州尼寺中,病甚,無醫藥,左右視之泣下。母曰:「吾與吾兒同死,又何恨哉?」亦死。眾嘆曰:「有斯母,宜有是兒。」為收葬之。



 蒲壽庚以泉州降,告其民曰:「陳文龍非不
 忠義,如民何?」聞者笑之。大兵既歸,文龍之侄瓚復舉兵殺林華,據興化,未幾復破,瓚死之。



 鄧得遇,字達夫,邛州人。淳祐十年進士。調寧遠主簿,改知南昌縣,通判隆興府,監行在左藏庫,出知昭州,遷廣西提點刑獄,逾年攝經略事兼知靜江府。



 德祐元年,長沙被兵,得遇遣都統馬驥、馬應麒赴援。驥潛叛而還,得遇斬之,軍事悉委之應麒。未幾,馬塈代閫,議事不合。二年,移治蒼梧。



 靜江破,得遇朝服南望拜辭,書幅紙云:「宋室
 忠臣,鄧氏孝子。不忍偷生,寧甘溺死。彭咸故居,乃吾潭府。屈公子平,乃吾伴侶。優哉悠哉,吾得其所!」遂投南流江而死。



 張玨,字君玉,隴西鳳州人。年十八,從軍釣魚山,以戰功累官中軍都統制,人號為「四川虓將」。



 寶祐末,大兵攻蜀,破吉平隘,拔長寧,殺守將王佐父子。至閬州,降安撫楊奫,推官趙廣死之。至蓬州,降守將張大悅,運使施擇善死之。順慶、廣安諸郡,破竹而下。明年,合諸道兵圍合州,
 凡攻城之具無不精備。玨與王堅協力戰守,攻之九月不能下。景定初,合守王堅征入朝,以馬千代守合。四年,千子饋餉至虎相山,為東川兵所得,屢以書勸千降,朝廷乃以玨代千。玨魁雄有謀,善用兵,出奇設伏,算無遺策。其治合州,士卒必練,器械必精,御部曲有法,雖奴隸有功必優賞之,有過雖至親必罰不貸,故人人用命。



 自全汝楫失大良平,大兵築虎相山,駐兵兩城,時出攻梁山、忠萬開達,民不得耕,兵不得解甲而臥,每餉渠,竭數
 郡兵護送,死戰兩城之下始克入。咸淳二年十二月,玨遣其將史炤、王立以死士五十斧西門入,大戰城中,復其城。三年四月,平章賽典赤提兵入,壞重慶麥,道出合城下,玨碇舟斷江中為水城,大兵數萬攻之不克,遂引去。



 合州自餘玠用二冉生策,徙軍釣魚山,城壁甚固。然開、慶受兵,民凋弊甚,玨外以兵護耕,內教民墾田積粟,未再期,公私兼足。九年,叛將劉整復獻計,欲自青居進築馬鬃、虎頂山,扼三江口以圖合,匣刺統軍率諸翼兵
 以築之。左右欲出兵與之爭,玨不可,曰:「蕪菁平母德、彰城,汪帥勁兵之所聚也,吾出不意而攻之,馬鬃必顧其後,不暇城矣。」乃張疑兵嘉渠口,潛師渡平陽灘攻二城,火其資糧器械,越砦七十里,焚船場,統制周虎戰死,馬鬃城卒不就。



 十年,加寧江軍承宣使。德祐元年,升四川制置副使、知重慶府。五月,加檢校少保。徵其兵入衛,蜀道斷,不得達。六月,昝萬壽以嘉定及三龜、九頂降,守將侯都統戰死。已而瀘、敘、長寧、富順、開、達、巴、渠諸郡不一
 月皆下,合兵圍重慶,作浮梁三江中,斷援兵。自秋徂冬,援絕糧盡,玨屢以死士間入城,許以赴援,且為之畫守禦計。二年正月,遣其將趙安襲青居,執安撫劉才、參議馬嵩歸。二月,遣張萬以巨艦載精兵,斷內水橋,入重慶。四月,合重慶兵出攻鳳頂諸砦。玨結瀘士劉霖、先坤朋為內應。六月,遣趙安破神臂門,執梅應春殺之,復瀘州。重慶兵漸解去,圍瀘州。十二月,趙定應迎玨入重慶為制置。



 時陽立以涪州降,玨遣張萬攻走立,俘其僚屬馮
 巽午等。立復合兵來決戰,史進、張世傑戰死,萬不支,俘立妻子及安撫李端以歸。玨以都統程聰守涪。重慶兵盡退。玨聞二王立廣中,遣兵數百人求王所。調史訓忠、趙安等援瀘州。張萬入夔,連忠、涪兵拔石門及巴巫砦,獲將士百餘人,解大寧圍,攻破十八砦。明年六月,張德潤復破涪州,執守將程聰。先是,聰在重慶力主守城之議,玨入,不知也,使出守涪。聰至郡怏怏,不設備,至是被執。德潤以肩輿載聰歸,語之曰:「若子鵬飛為參政矣,旦
 晚可會聚也。」聰曰:「我執彼降,非吾子也。」



 是月,梁山軍袁世安降。十月,萬州破,殺守將上官夔。十一月,瀘州食盡,人相食,遂破之,安撫王世昌自經死。



 大兵會重慶,駐佛圖關,以一軍駐南城,一軍駐朱村坪,一軍駐江上。遣瀘州降將李從招降,玨不從。十二月,達州降將鮮汝忠破咸淳皇華城,執守將馬堃,軍使包申巷戰死。至元十五年春,玨遣總管李義將兵由廣陽,一軍皆沒。二月,大兵破紹慶府,執守將鮮龍,湖北提刑趙立與制司幕官趙
 酉泰



\end{pinyinscope}