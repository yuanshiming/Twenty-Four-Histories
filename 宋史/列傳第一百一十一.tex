\article{列傳第一百一十一}

\begin{pinyinscope}

 唐恪李邦彥餘深薛昂吳敏王安中王襄趙野曹輔耿南仲王宇附



 唐恪,字欽叟,杭州錢塘人。四歲而孤,聞人言其父,輒悲
 泣。以蔭登第,調郴尉。縣民有被害而尸不獲,吏執其鄰人,抑使自誣,令以為信。恪爭之,令曰:「否將為君累。」恪曰:「吾為尉而盜不能捕,更俾亡辜死平?」躬出訪求,夕,若有告者,旦而得尸,遂獲盜。知榆次,縣豪子雄於鄉,萃逋庇奸,不輸公賦,前後莫敢詰。恪以理善曉之,悟而自悔,折節為長者。最聞,擢提舉河東常平、江東轉運判官。



 大觀中,牂牁內附,召為屯田員外郎,持節招納夷人。夷始恫疑,衷甲以逆,恪盡去兵衛,從數十卒單行。夷望見歡呼,
 投兵聽命。以奉使稱職,遷右司員外郎、起居舍人。迎遼使還,言河北邊備弛廢,宜及今無事,以時治之。徽宗壯之,曰:「非卿誰宜為者。」命為都轉運使,加集賢殿修撰。中貴人稱詔有所市,恪不答,憤而歸,中以他事,降直龍圖閣、知梓州。



 歷五年,徙滄州。河決,水犯城下,恪乘城救理。都水孟昌齡移檄索船與兵,恪報水勢方惡,船當以備緩急;滄為極邊,兵非有旨不敢遣。昌齡怒,劾之,恪不為動,益治水。水去,城得全,詔書嘉獎。乃上疏請暫免保甲、
 保馬呈閱及復諸縣租,等第振貸,以寬被水之民。未報,悉便宜罷行之,民大悅。



 進龍圖閣待制、知揚州,召拜戶部侍郎。京師暴水至,汴且溢,付恪治之。或請決南堤以紓宮城之患,恪曰:「水漲堤壞,此亡可奈何,今決而浸之,是魚鱉吾民也。」亟乘小舟,相水源委,求所以利導之,乃決金堤注之河。浹旬水平,入對,帝勞之曰:「宗廟社稷獲安,卿之力也。」恪再拜,因上疏言:「水,陰類也,至犯京闕,天其或者以陰盛之沴儆告陛下乎?願垂意時事,益謹天
 戒。」



 宣和初,遷尚書,帝許以二府。為宰相王黼所陷,罷知滁州。言者論其治第歷陽,擾民逾制,提舉鴻慶宮。五年,起知青州;未行,召為吏部尚書,徙戶部。復請外,以延康殿學士知潭州,請往錢塘掃墓,然後之官,遂改杭州,



 靖康初,金兵入汴,李邦彥薦之,拜同知樞密院事,至則為中書侍郎。時進見者多論宣和間事,恪言於欽宗曰:「革弊當以漸,宜擇今日之所急者先之。而言者不顧大體,至毛舉前事,以快一時之憤,豈不傷太上道君之心哉。
 京、攸、黼、貫之徒既從竄斥,姑可已矣,他日邊事既定,然後白道君,請下一詔,與天下共棄之,誰曰不可。」帝曰:「卿論甚善,為朕作詔書,以此意布告在位。因賜東宮舊書萬卷,且用近比除子璟直秘閣,力辭之。



 八月,進拜少宰兼中書侍郎,帝注禮之甚渥。然恪為相,無濟時大略。金騎再來,邀割三鎮,恪集廷臣議,以為當與者十九,恪從之。使者既行,於是諸道勤王兵大集,輒諭止令勿前,皆反旆而去。洎金兵薄城下,始悔之,密言於帝曰:「唐自天
 寶而後屢失而復興者,以天子在外可以號召四方也。今宜舉景德故事,留太子居守而西幸洛,連據秦、雍,領天下親征,以圖興復。」帝將從其議,而開封尹何慄入見,引蘇軾所論,謂周之失計,未有如東遷之甚者。帝幡然而改,以足頓地曰:「今當以死守社稷。」擢慄門下侍郎,恪計不用。



 從帝巡城,為都人遮擊,策馬得脫,遂臥家求去。御史胡舜陟繼劾其罪,謂「恪之智慮不能經畫邊事,但長於交結內侍,今國勢日蹙,誠不可以備位。」乃以觀文
 殿大學士、中太一宮使兼侍讀罷,慄代為相。



 京城不守,車駕至金帥營,恪曰:「計失矣。一入,將不得還。」既而還宮,恪迎拜道左,請入覲,慄不可。二年正月,復幸,恪曰:「一之謂甚,其可再乎?」及金人逼百官立張邦昌,令吳開、莫儔入城取推戴狀,恪既書名,仰藥而死。



 李邦彥,字士美,懷州人。父浦,銀工也。邦彥喜從進士游,河東舉人入京者,必道懷訪邦彥。有所營置,浦亦罷工與為之,且復資給其行,由是邦彥聲譽弈弈。入補太學
 生,大觀二年,上舍及第,授秘書省校書郎,試符寶郎。



 邦彥俊爽,美風姿,為文敏而工。然生長閭閻,習猥鄙事,應對便捷;善謳謔,能蹴鞠,每輟街市俚語為詞曲,人爭傳之,自號李浪子。言者劾其游縱無檢,罷符寶郎,復為校書郎。俄以吏部員外郎領議禮局,出知河陽,召為起居郎。邦彥善事中人,爭薦譽之,累遷中書舍人、翰林學士承旨。



 宣和三年,拜尚書右丞;五年,轉左丞。浦死,贈龍圖閣直學士,謚曰宣簡。邦彥起復,與王黼不協,乃陰結蔡攸、
 梁師成等,讒黼罷之。明年,拜少宰,無所建明,惟阿順趨諂充位而已,都人目為「浪子宰相」。



 徽宗內禪,命為龍德宮使,升太宰。知眾議不與,外患日逼,抗疏丐宮祠。金人既薄都城,李綱、種師道罷,邦彥堅主割地之議。太學生陳東數百人伏宣德門上書,言邦彥及白時中、張邦昌、趙野、王孝迪、蔡懋、李乂之徒為社稷之賊,請斥之。邦彥退朝,群指而大詬,且欲毆之,邦彥疾驅得免。乃以特進、觀文殿大學士充太一宮使。不旬日,吳敏為請,復起
 為太宰。人皆駭愕,言者交論之。出知鄧州,遂請持餘服,提舉亳州明道宮。建炎初,以主和誤國,責建武軍節度副使,潯州安置。



 方蔡京、王黼用事,附麗者多援引入政府,若餘深、薛昂、吳敏、王安中、趙野,史皆逸其事,因附著於此云。



 餘深,福州人。元豐五年,進士及第。崇寧元年,為太常博士、著作佐郎,改司封員外郎,拜監察御史、殿中侍御史,試闢雍司業。



 累官御史中丞兼侍讀。治張懷素獄,事連
 蔡京,與開封尹林攄曲為掩覆,獄辭有及京者輒焚之。京遂力引深與攄驟至執政。大觀二年,以吏部尚書拜尚書左丞。三年,轉中書侍郎;四年,轉門下侍郎。京既致仕,深不自安,累疏請罷,乃以資政殿學士知青州。



 政和二年,京復赴都堂治事,於是深復入為門下侍郎。七年,拜少宰。宣和元年,為太宰,進拜少保,封豐國公。再封衛國,加少傅。時福建以取花果擾民,深為言之,徽宗不悅。遂請罷,出為鎮江軍節度使、知福州。靖康初,加恩特進、
 觀文殿大學士。故事,凡僕射、使相、宣徽使皆判州府,深以少傅、節度知福州,有司失之也。



 深諂附蔡京,結為死黨。京奸謀詭計得助多者,深為首,攄次之。言者累章劾深,深益懼,丐致仕。建炎二年,降中大夫,臨江軍居住。尋以渡江赦恩,還鄉里,卒。子日章,亦以言者罷徽猷閣待制。



 薛昂,杭州人,登元豐八年進士第。崇寧初,歷太學博士、校書郎、著作佐郎,為殿中侍御史,試起居郎,改中書舍
 人兼侍講,升給事中兼大司成。昂寡學術,士子有用《史記》、《西漢》語,輒黜之。在哲宗時,常請罷史學,哲宗斥為俗佞。拜翰林學士,以不稱職改刑部尚書,轉兵部。大觀三年,拜尚書左丞。明年,請補外,出知江寧,徙河南。久之,提舉嵩山崇福宮。



 政和三年,蔡京復用事,昂復自尚書右丞為左丞,遷門下侍郎。尋請罷,授彰化軍節度使、祐神觀使,改特進,充資政殿大學士、知應天府。昂與餘深、林攄始終附會蔡京,至舉家為京諱。或誤及之。輒加笞責,
 昂嘗誤及,即自批其口。靖康初,言者斥其罪,詔以金紫光祿大夫致仕。杭州軍亂,昂不請命領州事,責徽州居住。昂主王氏學,嘗在安石坐,圍棋賭詩,局敗,昂不能作,安石代之,時人以為笑云。



 吳敏,字符中,真州人。大觀二年,闢雍私試首選。蔡京喜其文,欲妻以女,敏辭。因擢浙東學事司干官,為秘書省校書郎,京薦之充館職。中書侍郎劉正夫以敏未嘗過省,不可,京乃請御筆特召上殿,除右司郎官。御筆自此
 始,違者以大不恭論,繇是權幸爭請御筆,而繳駁之任廢矣。升中書舍人、同修國史,改給事中。敏為蔡京所引,鄭居中方秉政,敏數言其失,居中銜之。坐駁盜當死者,罷為右文殿修撰、提舉南京鴻慶宮。久之,復為給事中、權直學士院兼侍講。



 徽宗將內禪,蔡攸探知上意,引敏入對。宰臣執政皆在,敏前奏事,且曰:「金人渝盟,舉兵犯順,陛下何以待?」上蹙然曰:「奈何!」時東幸計已定,命戶部尚書李梲先出守金陵。敏退,詣都堂言曰:「朝廷便為棄
 京師計,何理也?此命果行,須死不奉詔。」宰執以為言,梲遂罷行。皇太子除開封尹,上去意益決,敏因奏對得請,遂薦李綱。綱嘗語敏以上宜傳位,如唐天寶故事,故薦之,冀上或有所問也。明日,宰臣奏事,徽宗獨留李邦彥,語敏所對。命除門下侍郎,輔太子。敏駭曰:「臣既畫計,當從陛下巡幸。陛下且傳位,而臣受不次之擢,臣曷敢?」上曰:「不意卿乃爾敢言。」於是命敏草傳位詔。



 欽宗既立,上皇出居龍德宮,敏與蔡攸同為龍德宮副使,遷知樞密
 院事,拜少宰。敏主和議,與太宰徐處仁議不合,紛爭上前。御史中丞李回劾之,與處仁俱罷,為觀文殿大學士、醴泉觀使。頃之,言者論其芘蔡京父子,出知揚州,再貶崇信軍節度副使,涪州安置。建炎初,移柳州。俄用範宗尹薦,起知潭州,敏辭免,丐宮祠,乃提舉洞霄宮。紹興元年,復觀文殿大學士,為廣西、湖南宣撫使,卒於官。



 王安中,字履道,中山陽曲人。進士及第,調瀛州司理參軍、大名縣主簿,歷秘書省著作郎。政和間,天下爭言瑞
 應,廷臣輒箋表賀,徽宗觀所作,稱為奇才。他日,特出制詔三題使具草,立就,上即草後批:「可中書舍人。」未幾,自秘書少監除中書舍人,擢御史中丞。開封邏卒夜跡盜,盜脫去,民有驚出與卒遇,縛以為盜;民訟諸府,不勝考掠之慘,遂誣服。安中廉知之,按得冤狀,即出民,抵吏罪。



 有徐禋者,以增廣鼓鑄之說媚於蔡京,京奏遣禋措置東南九路銅事,且令搜訪寶貨。禋圖繪坑冶,增舊幾十倍,且請開洪州嚴陽山坑,迫有司承歲額數十兩。其所
 烹煉,實得銖兩而已。禋術窮,乃妄請得希世珍異與古之寶器,乞歸書藝局,京主其言。安中獨論禋欺上擾下,宜令九路監司覆之,禋竟得罪。



 時上方鄉神仙之事,蔡京引方士王仔昔以妖術見,朝臣戚里寅緣關通。安中疏請自今招延山林道術之士,當責所屬保任,宣召出入,必令察視其所經由,仍申嚴臣庶往還之禁;並言京欺君僭上、蠹國害民數事。上悚然納之。已而再疏京罪,上曰:「本欲即行卿章,以近天寧節,俟過此,當為卿罷京。」
 京伺知之,大懼,其子攸日夕侍禁中,泣拜懇祈。上為遷安中翰林學士,又遷承旨。



 宣和元年,拜尚書右丞;三年,為左丞。金人來歸燕,謀帥臣,安中請行。王黼贊於上,授慶遠軍節度使、河北河東燕山府路宣撫使、知燕山府,遼降將郭藥師同知府事。藥師跋扈,府事皆專行,安中不能制,第曲意奉之,故藥師愈驕。俄加檢校少保,改少師。時山後諸州俱陷,唯平州為張覺所據。金人入燕,以覺為臨海軍節度使。其後叛金,金人攻之,覺敗奔燕。金
 人來索急,安中不得已,縊殺之,函其首送金。郭藥師宣言曰:「金人欲覺即與,若求藥師,亦將與之乎?」安中懼,奏其言,因力求罷。藥師自是解體,金人終以是啟釁。安中以上清寶菉宮使兼侍讀召還,除檢校太保、建雄軍節度使、大名府尹兼北京留守司公事。



 靖康初,言者論其締合王黼、童貫及不幾察郭藥師叛命,罷為觀文殿大學士、提舉嵩山崇福宮;又責授朝議大夫、秘書少監、分司南京,隨州居住;又貶單州團練副使,像州安置。高宗
 即位,內徙道州,尋放自便。紹興初,復左中大夫。子闢章知泉州,迎安中往,未幾卒,年五十九。



 安中為文豐潤敏拔,尤工四六之制。徽宗嘗宴睿謨殿,命安中賦詩百韻以紀其事。詩成,嘗嘆不已,令大書於殿屏,凡侍臣皆以副本賜之。其見重如此。有《初寮集》七十六卷傳於世。



 王襄,初名寧,鄧州南陽人,擢進士第。崇寧二年,以軍器監主簿言事稱旨,擢庫部員外郎,改光祿少卿,出察訪陜西。還,為顯謨閣待制、權知開封府。府事浩穰,訟者株蔓
 千餘人,縲系滿獄。襄晝夜決遣,四旬俱盡;又閱月,獄再空。遷龍圖閣直學士、吏部侍郎,出知杭州;未至,改海州;又改應天府,徙鄆州。召為禮部尚書,移兵部,出知穎州,改永興軍。。蒲城妖賊王寧適同姓名,請更名宓。為左司諫石公弼所劾,徙汝州,俄奪學士,提舉南京鴻慶宮。



 大觀三年,以集賢殿修撰知潭州,改兵部侍郎,使高麗。還對稱旨,詔賜名襄。歷工部、吏部尚書,拜同知樞密院事。坐薦引近侍,以延康殿學士罷知亳州;又坐交通郭天
 信落職,提舉嵩山崇福宮。久之,起知郢州,復學士秩,尋加資政殿學士,徙知淮寧府。以言事忤王黼,復提舉崇福宮。



 宣和六年,起為河南尹。金人再入,出為西道都總管,張杲副之。高宗開大元帥府,襄以所部兵會於虞城縣。即位,命襄知河南府。襄初與趙野分總西北道諸軍,金人圍京師,徵兵入援,二人故迂道宿留。至是,降寧遠軍節度副使,永州安置,卒。



 趙野,開封人。登政和二年進士第。歷監察御史、殿中侍
 御史,試起居舍人兼太子舍人,俄遷中書舍人、給事中、大司成,拜刑部尚書、翰林學士。時蔡京、王黼更秉政,植黨相擠,一進一退,莫有能兩全者,野處之皆得其心,京、黼亦待之不疑。宣和七年,拜尚書右丞,升左丞。



 靖康初,為門下侍郎。徽宗東幸,詔野為行宮奉迎使。以左司諫陳公輔言,罷野行,出為北道都總管,顏岐副之。已而落職,提舉嵩山崇福宮。元帥府建,命與範訥為宣撫司,守東京,尋帥師屯宛亭,以待王師。王襄既責,野亦降安遠
 軍節度副使,邵州安置。



 建炎元年,復起知密州。時盜賊充斥山東,車駕如淮南,命令阻絕,野棄城去。軍校杜彥等乘間作亂,追野以歸。彥坐堂上數之曰:「汝知州而攜家先遁,此州之人,誰其為主?」野不能應,遂見殺。家屬悉為賊所分,唯子學老得免。



 曹輔,字載德,南劍州人。第進士。政和二年,以通仕郎中詞學兼茂科,歷秘書省正字。



 自政和後,帝多微行,乘小轎子,數內臣導從。置行幸局,局中以帝出日謂之有排
 當,次日未還,則傳旨稱瘡痍,不坐朝。始,民間猶未知。及蔡京謝表有「輕車小輦,七賜臨幸」,自是邸報聞四方,而臣僚阿順,莫敢言。輔上疏略曰:



 陛下厭居法宮,時乘小輿,出入廛陌之中、郊坰之外,極游樂而後反。道塗之言始猶有忌,今乃談以為帝某日由某路適某所,某時而歸;又云輿飾可辨而闢。臣不意陛下當宗廟社稷付托之重,玩安忽危,一至於此。夫君之與民,本以人合,合則為腹心,離則為楚、越,畔服之際在於斯須,甚可畏也。昔
 者仁祖視民如子。憫然惟恐其或傷。一旦宮闈不禁,衛士輒逾禁城,幾觸寶瑟。荷天之休,帝躬保祐。俚語有之,『盜憎主人』,主人何負於盜哉?況今革冗員,斥濫奉,去浮屠,誅胥吏,蚩愚之民,豈能一一引咎安分?萬一當乘輿不戒之初,一夫不逞,包藏禍心,發蜂蠆之毒,奮獸窮之計,雖神靈垂護,然亦損威傷重矣。又況有臣子不忍言者,可不戒哉!



 臣願陛下深居高拱,淵默雷聲,臨之以穹昊至高之勢,行之以日月有常之度。及其出也,太史擇日,
 有司除道,三衛百官,以前以後。若曰省煩約費,以便公私,則臨時降旨,存所不可闕,損所未嘗用。雖非祖宗奮制,比諸微服晦跡,下同臣庶,堂陛陵夷,民生奸望,不猶愈乎?



 上得疏,出示宰臣,令赴都堂審問。太宰餘深曰:「輔小官,何敢論大事?」輔對曰:「大官不言,故小官言之。官有大小,愛君之心,則一也。」少宰王黼陽顧左丞張邦昌、右丞李邦彥曰:「有是事乎?」皆應以不知。輔曰:「茲事雖里巷細民無不知,相公當國,獨不知邪?曾此不知,焉用彼相!」
 黼怒其侵已,令吏從輔受辭。輔操筆曰:「區區之心,一無所求,愛君而已。」退,待罪於家。黼奏不重責輔,無以息浮言,遂編管郴州。輔將言,知必獲罪,召子紳來,付以家事,乃閉戶草疏。夕有惡鳥鳴屋極,聲若紡輪,心知其不祥,弗恤也。處郴六年,黼當國不得移,輔亦怡然不介意。



 靖康元年,召為監察御史,守殿中侍御史,除左諫議大夫、御史中丞。不旬日,拜延康殿學士、簽書樞密院事。未幾,免簽書。金人圍汴都,要親王、大臣出盟,輔與尚書左丞
 馮澥出使粘罕軍。康王開元帥府於相州,金人請欽宗詔召之,乃遣輔往迓。至曹州,不見而復,遂從二帝留金軍中。張邦昌請歸輔,輔歸,乞奉祠,邦昌不從。康王次南京,邦昌遣輔來見。康王即位,輔仍舊職。未幾卒,詔厚恤其家。



 耿南仲,開封府人。與餘深同年登第,歷提舉兩浙常平,徙河北西路,改轉運判官、提點廣南東路及夔州路刑獄、荊湖江西兩路轉運副使,入為戶部員外郎、闢雍司業,坐事
 罷知衢州。政和二年,以禮部員外郎為太子右庶子,改定王、嘉王侍讀,俄試太子詹事、徽猷閣直學士,改寶文閣直學士。在東宮十年。



 欽宗辭內禪,得疾,出臥福寧殿,宰相百官班候,日暮不敢退。李邦彥曰:「皇太子素親耿南仲,可召之入。」南仲與吳敏至殿中侍疾。明日,帝即位,拜資政殿大學士、簽書樞密院事。未幾,免簽書。帝以南仲東宮舊臣,禮重之,賜宅一區,升尚書左丞、門下侍郎。



 金人再舉鄉京師,請割三鎮以和,議者多主戰守,唯南
 仲與吳幵堅欲割地。康王使軍前,請南仲偕。帝以其老,命其子中書舍人延禧代行。金人次洛陽,不復言三鎮,直請畫河為界。於是議遣大臣往,南仲以老辭,聶昌以親辭。上大怒,即令南仲出河東、昌出河北,議割地。



 初,南仲自謂事帝東宮,首當柄用,而吳敏、李綱越次進,位居己上,不能平。因每事異議,擯斥不附己者。綱等謂不可和,而南仲力沮之,惟主和議,故戰守之備皆罷。康王在相州,南仲偕金使王汭往衛州。鄉兵穀殺汭,汭脫去,南仲
 獨趣衛,衛人不納。走相州,以上旨喻康王,起河北兵入衛京師,因連署募兵榜揭之,人情始安。二帝北行,南仲與文武官吏勸進。



 高宗既即位,薄南仲為人,因其請老,罷為觀文殿大學士、提舉杭州洞霄宮。延禧以龍圖閣直學士知宣州。已而言者論其主和誤國罪,詔鐫學士秩,延禧亦落職與祠。尋責南仲臨江軍居住。御史中丞張澄又言:「南仲趣李綱往救河東,以致師潰,蓋不恤國事,用此報讎。」帝曰:「南仲誤淵聖,天下共知,朕嘗欲手劍
 擊之。」命降授別駕,安置南雄,行至吉州卒。建炎四年,復觀文殿大學士。



 王□字符忠,江州人。父易簡,資政殿大學士兼侍講。□歷校書郎、著作佐郎、度支員外郎兼充編修官、國子司業,為起居舍人,改中書舍人兼蕃衍宅直講。欽宗立,以給事中命兼邇英殿經筵侍講,轉吏部侍郎,升禮部尚書、翰林學士。



 康王之使金也,以□為尚書左丞副之。□憚行,假夢兆丐免,易簡亦上書以請。上震怒,追毀左丞
 命,降單州團練副使,新州安置,並易簡宮祠黜之。建炎四年,賊馬進破江州,易簡等三百人俱被害。



 論曰:三代之後,有天下而長久者,漢、唐、宋爾。漢、唐末世,朋黨相確,小人在位,然猶有君子扶持遷延,浸微浸滅;未有純用小人,至於主辱國播,如宋中葉之烈也。蔡京以紹述為羅,張端官、修士而盡之,上箝下錮,其術巧矣。徽宗亦頗悟,間用鄭居中、王黼、李邦彥輩,褫京柄權。以不肖易不肖,猶去野葛而代烏喙也,庸愈哉!當是時,王、
 蔡二黨,階京者芘京,締黼者右黼,援麗省臺,迭相指嗾,徼功挑患,汴、洛既震,則恇縮無策,茍生丐和。彼邦彥、安中、深、敏輩誤國之罪,當正其僇,而欽、高二君徒從竄典,信失刑矣。恪既預推戴,署狀乃死,無足贖者。輔以小臣劘上,面譙大臣,坐斥不變,獨終始無朋與,其賢矣乎。



\end{pinyinscope}