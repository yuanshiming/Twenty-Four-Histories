\article{列傳第六十二}

\begin{pinyinscope}

 張溫之魏瓘弟琰滕宗諒劉越附李防趙湘唐肅子詢張述黃震胡順之陳貫子安石範祥子育田京



 張溫之字景山。父秘,自有傳。溫之進士及第,補樂清尉,潤州觀察推官,校勘館閣書籍,遷集賢校理,通判常州,知溫州。蔡齊薦其材可用,擢提點淮南路刑獄。楊崇勛知亳州,恃恩為不法,誣蒙城知縣王申罪,械送獄。溫之廉得冤狀,乃出申,配奸吏若干人。徙廣南東路轉運使。夷人有犯,其酋長得自治而多慘酷,請一以漢法從事。權度支判官,為京西轉運使,加直史館,徙河北。被邊諸州發卒斬西山木,卒逃入契丹者歲數百人,敵既利其
 所開地,又得亡卒,故不急。溫之戒斬伐毋得深入北地,卒亦不敢逃。



 還,為鹽鐵副使,擢天章閣待制、河北都轉運按察使。保州、廣信、安肅軍自五代以來別領兵萬人,號緣邊都巡檢司,亦曰策先鋒,以知州、軍為使,置副二人,分所領卒為三部,使援鄰道。太祖嘗用之有功,詔每出巡別給糧錢以優之。其後州將不復出,內侍為副,數出巡,部卒偏得廩賜,軍中以為不均。通判保州石待舉言於溫之,請合三部兵更出入,季一出即別給錢糧,餘
 悉罷,仍請以武臣代內侍。時楊懷敏方任邊事,尤不悅巡檢司。雲翼卒惡石待舉,遂殺之以作亂。溫之自魏馳至城下,召諸將部分攻城,使人請懷敏曰:「不即來,當以軍法從事。」既至,又以兵自衛,溫之曰:諸將方集,獨敢以兵隨,將欲反邪!」叱去衛者。城開,田況潛殺降兵數百人,溫之預知其謀。除戶部副使,既而坐前事奪職,知虢州。



 王則反貝州,有言溫之在河北捕得妖人李教不殺,使得逸去,今乃為則主謀,事平,無其人。會冀州人段得政
 詣闕,自言「嘗為叔父屯田郎中曇賕免緣坐」,且言「曇以書屬溫之」,乃下御史按劾,雖不得書,猶奪三官,監鄂州稅。知漢陽軍,稍遷刑部郎中,復待制、知湖州,徙揚州。以光祿卿致仕,卒。溫之喜吏事,所至有聲。退居築家廟,率子弟歲時奉祠。



 魏瓘,字用之。父羽奏補秘書省校書郎、監廣積倉,知開封府倉曹參軍。持法精審,明吏事。上元起彩山,闕前張燈,與宦者護作,宦者挾氣,視瓘年少,輒誅索侵擾。瓘密
 以聞,詔杖宦者遣之。



 瓘門人魏綱上疏詆天書,流海島,瓘亦坐是停官。復監鄧州稅、鄂州茶,以大理寺丞知衡山縣,通判壽州,歷知循、隨、安州,提點廣南西路刑獄。邕州獠戶緣逋負沒婦女為傭者一千餘人,悉奏還其家。就除轉運使。劉鋹時計口以稅,雖舟居皆不免,至是而雷、化、欽、廉、高州猶未除,瓘為除之。減柳州無名役四百人。召權度支判官。尋以罪降知洪州,徙梓州路轉運使,還知蔡州、潭州,為京西轉運使,江、淮制置發運使,自主
 客郎中遷太常少卿,知廣州。築州城環五里,疏東江門,鑿東西澳為水閘,以時啟閉焉。拜右諫議大夫,再任臨江軍判官。



 史沆性險詖,嘗為瓘所劾免。會廣州封送貢餘椰子煎等餉京師,輒邀留之,飛奏指以為珍貨,詔遣內侍發驗無有,沆坐不實廢,瓘亦降知鄂州。未逾年,復為陜西轉運使,徙河北。以給事中知開封府,政事嚴明,吏民憚之。內東門索命婦車,得賂遺掖庭物,付府驗治,獄未上,內降釋罪。諫官吳奎言法當執奏,而瓘不即奏
 行,請以廢法論,降知越州。



 儂智高寇廣東、西,獨廣州城堅守不能下。於是論築城功,遷工部侍郎、集賢院學士,復知廣州,兼廣東經略安撫使,給禁卒五千,聽以便宜從事。屬狄青已破賊,召還,糾察在京刑獄。議者請開六塔河,塞商胡北流,宰相主其說,命瓘按視,還奏以為不可塞。下溪州蠻彭士羲叛,將發兵討除。進龍圖閣直學士、知荊南。瓘以為「五溪之險,師行鳥道,諸將貪功生事,於國家何所利?」因條上三策,以招徠為上,守御為下,功
 取為失。不報。後卒如瓘議。徙澶州、滑州。又徙鄧州,不行,請老,以吏部侍郎致仕,卒。



 瓘所至整辦,與人置對未嘗屈。史沆、王逵以善訟名天下,瓘既廢沆,又嘗奏抵逵罪,專任機數,不稱循吏。弟琰。



 琰字子浩,以父恩授秘書省正字,為吏強敏,名齊於瓘。嘗通判陳州,適歲饑,百姓相率強取人粟,坐死者甚眾,琰曰:「此迫於窮餓,豈得已者。」坐其首黥之。歷知壽、潤、滁、安州。壽州盜殺寺童子,有司執僧笞服,琰憫其非罪,命
 脫械縱去,一府爭以為不可,後數日得真盜。富人犯法當死而死獄中,琰曰:「是嘗欺匿異籍孤弱者財,所以自斃,覬不可窮治爾,其吏受賕而為之謀乎?」後有告者如琰所料。累官司農卿、知福州,徙廣州。以疾告,得知江寧府。晚昏眊,縱私人亂法,日笞撲無罪吏卒。監司劾奏,召判刑部,乃致仕,進衛尉卿,卒。



 滕宗諒,字子京,河南人。與範仲淹同年舉進士,其後仲淹稱其才,乃以泰州軍事推官召試學士院。改大理寺
 丞,知當塗、邵武二縣,遷殿中丞,代還。會禁中火,詔劾火所從起,宗諒與秘書丞劉越皆上疏諫。宗諒曰:「伏見掖庭遺燼,延熾宮闥,雖沿人事,實系天時。詔書亟下,引咎滌瑕,中外莫不感動。然而詔獄未釋,鞫訊尚嚴,恐違上天垂戒之意,累兩宮好生之德。且婦人柔弱,棰楚之下,何求不可,萬一懷冤,足累和氣。祥符中,宮掖火,先帝嘗索其類置之法矣,若防患以刑而止,豈復有今日之虞哉。況變警之來,近在禁掖,誠願修政以禳之,思患以防
 之。凡逮系者特從原免,庶災變可銷而福祥來格也。」疏奏,仁宗為罷詔獄。時章獻太后猶臨朝,宗諒言國家以火德王,天下火失其性由政失其本,因請太后還政,而越亦上疏。太后崩,擢嘗言還政者,越已卒,贈右司諫,而除宗諒左正言。



 劉越者字子長,大名人。少孤貧,有學行,亦宗諒同年進士。嘗知襄城、固始二縣,有能名。既贈官,又官其一子,賜其家錢十萬。



 宗諒後遷左司諫,坐言宮禁事不實,降尚
 書祠部員外郎、知信州。與範諷雅相善,及諷貶,宗諒降監池州酒。久之,通判江寧府,徙知湖州。元昊反,除刑部員外郎、直集賢院、知涇州。葛懷敏軍敗於定州,諸郡震恐,宗諒顧城中兵少,乃集農民數千戎服乘城,又募勇敢,諜知寇遠近及其形勢,檄報旁郡使為備。會範仲淹自環慶引蕃漢兵來援,時天陰晦十餘日,人情憂沮,宗諒乃大設牛酒迎犒士卒;又籍定州戰沒者於佛寺祭酹之,厚撫其孥,使各得所,於是邊民稍安。



 仲淹薦以自
 代,擢天章閣待制,徙慶州。上言:「朝廷既授範仲淹、韓琦四路馬步軍都總管、經略安撫招討使,而諸路亦帶招討稱號,非所宜。」詔罷之。御史梁堅劾奏宗諒前在涇州費公錢十六萬貫,及遣中使檢視,乃始至部,日以故事犒賚諸部屬羌,又間以饋遺游士故人。宗諒恐連逮者眾,因焚其籍以滅姓名。仲淹時參知政事,力救之,止降一官,知虢州。御史中丞王拱辰論奏不已,復徙岳州,稍遷蘇州,卒。



 宗諒尚氣,倜儻自任,好施與,及卒,無餘財。所
 蒞州喜建學,而湖州最盛,學者傾江、淮間。有諫疏二十餘篇。



 李防,字智周,大名內黃人。舉進士,為莫州軍事推官。隨曹彬入契丹,授忠武軍節度推官。括磁、相二州逃戶田,增租賦十餘萬。因請均定田稅,又請縣有破逃五十戶者令佐降下考,百戶殿三選,二百戶停所居官,能招攜者旌賞之。改秘書省著作佐郎、通判潞州,遷秘書丞。體量二浙民饑,建言逃戶田宜即召人耕種,使人不敢輕
 去甽畝,而官賦常在。又請京師置折中倉,聽人入粟,以江、浙、荊湖物償之。擢開封府推官,請與判官間三五日即府司軍巡院察冤獄。出為陜路轉運副使。先是沿江水遞,歲役民丁甚眾,頗廢農作,防悉以城卒代之。會分川、陜為四路,徙防梓州路轉運使,累遷尚書工部員外郎,為三司戶部判官。



 景德初,江南旱,詔與張知白分東、西路安撫。上言:「秦羲嘗增江、淮、兩浙、荊湖榷酤錢,民頗煩擾。江南以歲饑權罷,而淮南、荊湖未被德音。」詔悉罷
 之,仍詔羲等毋得復增榷酤之利。遂為江南轉運。淮南舊不禁鹽,制置司請禁鹽而官自鬻之,使兵夫輦載江上,且多漂失之患。防請令商人入錢帛京師,或輸芻糧西北邊,而給以鹽,則公私皆利,後採用之。徙知應天府,鑿府西障口為斗門,洩汴水,淤旁田數百畝,民甚利之。又徙興元府,入為三司鹽鐵判官,失舉免官。後起通判河南府,徙知宿、延、亳三州,為利州路轉運使,累遷兵部郎中、糾察刑獄,擢右諫議大夫、知永興軍,進給事中,復
 知延州,更耀、潞二州,卒。



 防好建明利害,所至必有論奏,朝廷頗施行之。其精力過人。防在江南,晏殊以童子謁見,防命賦詩,使還薦之,後至宰相。



 趙湘,字巨源,華州人。進士甲科,歷彰武、永興、昭武三軍節度推官,遷秘書省著作佐郎、知新繁縣。以吏最,命知商州,徙隴州、興元府,再遷太常博士。上《補政忠言》十篇,召判宗正寺,賜白金二百兩。久之,上書言:「元德李太后母育聖躬,請祔太宗廟室。」後用其說。冊趙德明,假尚書
 禮部員外郎,為官告副使。



 擢殿中侍御史,權判三司勾院,上言:「漢章帝以《月令》冬至之後有順陽助生之文,而無鞫獄斷刑之政,遂定令毋以十一月、十二月報囚。今季冬誕聖之月而決大闢不廢。願詔有司,自仲冬留大闢弗決,俟孟春臨軒閱視,情可矜惻者貸之,他論如法。」真宗曰:「此固善矣,然慮系囚益淹久,吏或因緣為奸爾。」湘又上書請封禪。未幾,命管勾南宮北宅事。東封泰山,為東京留守推官,禮成,遷侍御史。升州火,命湘往致祠,
 兼問民疾苦。還言轉運使劉照弛職不按部,知洪州馬景病不任事,皆罷黜之。



 糾察刑獄,改尚書刑部員外郎兼侍御史知雜事。湘又言:「舊制文武常參官日趨朝,並赴待漏院俟禁門闢,今則辰漏上始放外朝,故朝者多後時乃入。望敕正衙門主者察晚至,以懲其慢。若風雨寒暑托病不朝者罪之。」時帝親制五箴以自儆,湘因言:「宗室風化所本,宜有以訓厲,願特制銘以賜南北邸。」帝悅,為制宗室座右銘,賜寧王元偓以下並及湘,且諭之
 曰:「卿宗姓也,故賜卿。」



 祀汾陰,為考制度副使,請如《周官》置土訓,錄所過州縣山川與俗好惡,日上奏御。兼判宗正寺。歷三司戶部、度支副使。祀太清宮,管勾留司三司事。為鹽鐵副使,再遷工部郎中、直昭文館,出知河南府,徙河中府,為京西轉運使。又徙鳳翔府、延州,遷太常少卿、知襄州。又知應天府,進右諫議大夫,復知河南,為集賢院學士,以疾徙虢州,卒。



 唐肅,字叔元,杭州錢塘人。當錢俶時,始七歲,能誦《五經》,
 名聞其國中。後與孫何、丁謂、曹商游,學者慕之。舉進士,調郿縣主簿,徙泰州司理參軍。有商人寓逆旅,而同宿者殺人亡去,商人夜聞人聲,往視之,血沾商人衣,為捕吏所執,州趣獄具。肅探知其冤,持之,後數日得殺人者。後守雷有終就闢為觀察推官。遷秘書省著作佐郎,歷知聞喜、福昌縣,通判陜州。召拜監察御史。或薦肅為群牧判官,真宗曰:「朕欲別用肅。」遂提點梓州路刑獄。遷殿中侍御史,入為三司戶部判官,出知舒州。遷侍御史,為
 福建路轉運使,判三司開拆司。再遷工部郎中、知洪州。尋為江南東路轉運使,擢三司度支副使。奉使契丹,還,遷刑部。為龍圖閣待制、登聞檢院,知審刑院,卒。子詢。



 詢字彥猷,以父任為將作監主簿。天聖中,詔許天下士獻文章,應詔者百數,有司第其善者,詢數人而已,詔賜進士及第、知長興縣。



 後以太常博士知歸州,用翰林學士吳育薦為御史,未至,喪母。服除,育方參政事,宰相賈昌朝與詢有親嫌,育數與昌朝言,詢用故事當罷御史,
 昌朝欲留詢,不得已,以知廬州。凡官外徙者皆放朝辭,而詢獨不用,比入見,中丞張方平乃奏留詢,育爭不能得,詢由是怨育而附昌朝。昌朝雅不善育,詢希其旨上奏曰:「賢良方正、直言極諫、茂才異等科,漢、唐皆不常置。若天見災異,政有闕失,則詔在位薦之,不可與進士同時設科。若因災異,非時舉擢,宜如漢故事,親策當世要務,罷秘閣之試。」育亦奏言:「三代以來,取士之盛,莫如漢、唐。漢詔舉賢良文學直言極諫之士,非有災異而舉。唐
 制科之盛,固不專於災異也。況災異之出,或彌年所無,則此舉奚設?或頻歲而有,則於事太煩。令禮部進士數年一舉,因以制科隨之,則事與時宜。又從而更張之,使遺材絕望,非所以廣賢路也。」仁宗是育言,詔禮部:「自今制科隨進士貢舉,其著為令。」時育由制科進,帝以為得人,故詢力肆排詆,意在育不在制科也。



 育弟婦故駙馬都尉李遵勖妹,有六子而寡。詢又奏育弟婦久寡不使更嫁,欲用此附李氏自進。後詢終以故事罷御史,除尚
 書工部員外郎、直史館、知湖州,徙江西轉運使。



 會詔淮南、江、浙、荊湖六路轉運司移文發運使如所屬,詢爭以為不可,乃移福建路。還,為三司戶部判官,又判磨勘司,出為江東轉運使。上言:「執政純取科名顯者修起居注,非故事。」未幾,起居注闕人,帝特用詢,遂知制誥。以參知政事曾公亮親嫌,出知蘇州,徙杭、青二州,進翰林侍讀學士,累遷右諫議大夫。召還,勾當三班院,判太常寺,進給事中,卒,贈禮部侍郎。有集三十卷。



 詢少刻勵自修,已
 而不固所守,及知湖州,悅官妓取以為妾。好畜硯,客至輒出而玩之,有《硯錄》三卷。子坰,附王安石為監察御史裏行,自有傳。



 論曰:宋承平日久,吏多以嚴刻為治。溫之辨冤獄,配奸吏;瓘奏還婦女為傭者若干人;琰吏事不下於瓘,脫械縱囚,審知奸弊,何其明且決也。宗諒、劉越以孤生立朝,請太后還政。越年不逮用,聲名與宗諒同矣。防請罷榷酤,興水利,湘廉問疾苦,按不稱職者;肅明於獄訟:皆不
 多見也。然溫之以殺降而奪官,瓘以能置對而興謗,詢傅會喜進,竊非其據,雖列侍從,君子所不與也。



 張述,字紹明,遂州小溪人。舉進士,調咸陽縣主簿,改大理寺丞,遷太常博士。皇祐中,仁宗未有嗣,述上書曰:「生民之命,系於宗廟社稷,而繼嗣為之本。匹夫有百金之產,猶能定謀托後,事出於素,況有天下者哉。陛下承三聖之業,傳之千萬年,斯為孝矣。宗廟社稷未有托焉,此臣所以夙夜徬徨而為陛下憂也。謂宜慎擇宗親才而
 賢者,異其禮秩,試以職務,俾內外知聖心有所屬,則天下大幸。」至和元年,復上疏曰:「臣聞『明兩作離,大人以繼明照四方』。離為日,君象也。二明相繼故能久照,東升西沒,晝夜迭運,數之常也。陛下御天下且三紀矣,是日之正中也,而未聞以繼照為慮,臣竊疑之。歷觀前世或令出宮闈,或謀起閽寺,或奸臣首議,利幼主以專政,假後宮以盜權,安危之機發於頃刻。朝議恬然,曾不為計,此臣拳拳為陛下言也。」述前後七上疏,最後語尤激,仁宗
 終不以為罪。



 述慷慨喜論事,歷通判延州,知泗州,皆有政跡。後以尚書職方員外郎為江、浙、荊湖、福建、廣南路提點坑冶鐵錢事,行至萬州,道病卒。



 黃震,字伯起,建州浦城人。進士及第,累遷著作佐郎、通判遂州。嘗給兩川軍士緡錢,詔至西川,而東川獨不及,軍士謀為變。震白主者曰:「朝廷豈忘東川邪?殆詔書稽留爾。」即開州帑給錢如西川,眾乃定,明日詔至。累遷尚書都官員外郎、提點湖北路刑獄,還,判三司磨勘司,擢
 江、淮發運使。



 先是,李溥自三司小吏為發運使十餘年,奸贓狼籍,丁謂黨之,無敢言者。震將行,上書自陳,辭頗憤激,真宗知其意在溥也,諭之曰:「卿當與人和。」震對曰:「廉正公忠,臣職也。負陛下任使者,臣不敢與之和。」既至,發溥奸贓數十事,溥坐廢;而震亦為溥訟,奪一官。罷,畏謂權,不敢自直,及謂貶,乃復官,知饒州,徙廣東轉運使。廣南歲進異花數千本,至都下枯死者十八九,道路苦其煩擾,震奏罷之。震在真宗朝數論事,既卒,詔進其官
 一等。



 胡順之,字孝先,原州臨涇人。登進士第,試秘書省校書郎、知休寧縣。民有汪姓者豪橫,縣不能制,歲租賦常不入,適以訟逮捕,不肯出。順之曰:「令不行何以為政。」命積薪環而焚之,豪大駭,少長趨出,叩頭伏辜,推其長械送州,致之法。為青州從事。高麗入貢,中貴人挾以為重,使州官旅拜於郊。順之曰:「青,大鎮也。在唐押新羅、渤海,奈何卑屈如此?」獨不拜。大姓麻士瑤陰結貴侍,匿兵械,服
 用擬尚方,親黨僕使甚多,州縣被陵蔑,莫敢發其奸。會士瑤殺兄子溫裕,其母訴於州,眾相視曰:「孰敢往捕者?」順之持檄徑去,盡得其黨。有詔鞫問,士瑤論死,其子弟坐流放者百餘人。改著作佐郎、知常熟縣,遷秘書丞,分司南京。



 仁宗即位,遷太常博士。天聖、明道間,再上宰相書,乞太后還政,宰相匿不以聞。太后崩,順之附疾置自言,求其書,出宰相家。仁宗嘉其忠,特遷尚書屯田員外郎。其後數論朝廷事,仲淹愛其才,然挾術尚權,喜縱橫
 捭闔。以目失明廢,州里皆憚焉。



 陳貫,字仲通,其先相州安陽人,後葬其父河陽,因家焉。少倜儻,數上疏言邊事。舉進士,真宗識貫名,擢置高第。為臨安縣主簿,以秘書省著作佐郎為刑部詳覆官,改秘書丞,為審刑院詳議官,歷知衛州、涇州。督察盜賊,禁戢不肖子弟,簿書筦庫,賦租出入,皆自檢核。嘗謂僚屬曰:「視縣官物如己物,容有奸乎?」州人憚其嚴。擢利州路轉運使。歲饑,出職田粟賑饑者,又帥富民令計口占粟,
 悉發其餘。徙陜西,累遷尚書度支員外郎,入為三司鹽鐵判官。領河北轉運使,請疏徐、鮑、曹、易四水,興屯田。徙河東,歷三司戶部、鹽鐵副使,以刑部郎中直昭文館,知相州。還朝卒。



 貫喜言兵,咸平中,大將楊瓊、王榮喪師而歸,貫上書曰:「前日不斬傅潛、張昭允,使瓊輩畏死不畏法,請自今合戰而奔者,主校皆斬;大將戰死,裨校無傷而還,與奔軍同。軍衄城圍,別部力足救而不至者,以逗留論。」真宗嘉納之。又嘗上《形勢》、《選將》、《練兵論》三篇,大略
 言:



 地有六害。今北邊既失古北之險,然自威虜城東距海三百里,沮澤磽確,所謂天設地造,非敵所能輕入。由威虜西極狼山不百里,地廣平,利馳突,此必爭之地。凡爭地之利,先居則佚,後起則勞,宜有以待之。



 昔李漢超守瀛州,契丹不敢視關南尺寸地。今將帥大抵用恩澤進,雖謹重可信,卒與敵遇,方略何從而出邪?故敵勢益張,兵折於外者二十年。



 方國家收天下材勇以備禁旅,賴廩給賜予而已,恬於休息,久不識戰,可以衛京師,不
 可以戍邊境。請募土人隸本軍,籍丁民為府兵,使北捍契丹,西捍夏人。敵之情偽,地勢之險易,彼皆素知,可不戰而屈人之兵矣。



 後以疾卒。著《兵略》,世頗稱之。子安石。



 安石字子堅,以蔭鎖廳及第。嘉祐中,為夔、峽轉運判官。民蓄蠱毒殺人,捕誅其魁並得良藥圖,由是遇毒者得不死。提點陜西刑獄,攝帥鄜延,能用諜者,敵動靜輒先聞。嘗敕邊民戒嚴,既而數萬騎奄至,無所獲而去,璽書嘉之。歷使京西、河東、淮南、京東,知蘇州、邠州、河中府。戶
 部副使韓絳鎮太原,議行鹽法,與監司多不合,加安石集賢殿修撰,為河東都轉運使,議始定。謂其僚曰:「興事當有漸,急則擾。」乃出鹽付民而俾之券,使隨所得貿易,鬻畢而歸券,私販為減。進天章閣待制。



 官軍西征時,遣縣令佐督餉,安石謂文吏畏怯,武人邀功,乃但取敢行者。申約束以防眾潰,曰:「事不豫警,俟其犯而誅之,是罔民也。」王中正帥東師而西,報安石持四十日糧,而師駐白草平彌月。安石深念曰:「吾頓兵益久,而秦甲未至,倘
 不足於食,將以乏軍興罪我。」即擅發民再餉,乃以聞。李舜舉劾其專,詔置獄於潞,安石自麟州會逮,俄而他路饋糧多不繼,神宗察其無罪,赦之。



 尚書省初建,召為戶部侍郎。嘗與右曹李定同奏事,帝目留之曰:「卿豈非在淮南日不肯保李定持服者乎?」對曰:「詔問臣,臣不敢不以實奏。」帝曰:「以實事君,朕所與也。」進吏部侍郎。選人將改京官,須次久,臨當引對,率困於刑寺審問,或沮以微文,則一跌不復。安石則罷再問,以絕曩弊,遂為後法。出
 知永興軍、鄧、襄、陳、鄭州、河陽,至龍圖閣直學士。紹聖元年卒,年八十一。



 範祥,字晉公,邠州三水人。進士及第,自乾州推官稍遷殿中丞、通判鎮戎軍。元昊圍城急,詳帥將士拒退之。請築劉璠堡、定川砦,從之。歷知慶、汝、華三州,提舉陜西銀銅坑冶鑄錢。祥曉達財利,建議變鹽法,後人不敢易,稍加損益,人輒不便,語在《食貨志》。提點本路刑獄,制置解鹽,累遷度支員外郎,權轉運副使。古渭砦距秦州三百
 里,道經啞兒峽,邊城數請城之,朝廷以饋餉之艱不許。祥權領州事,驟請修築,未報,輒自興役。蕃部驚擾,青唐族羌攻破廣吳嶺堡,圍啞兒峽砦,官軍戰死者千餘人,坐削一官,知唐州。後復官,提舉陜西緣邊青、白鹽,改制置解鹽使,卒。



 嘉祐中,包拯言:「祥通陜西鹽法,行之十年,歲減榷貨務使緡錢數百萬,其勞可錄。」官其子孫景郊社齋郎。熙寧中,平洮、岷、疊、宕、河州數千里,置郡縣,以古渭為通遠軍。權陜西轉運副使張詵奏:「朝廷復洮、隴故
 地,自將帥至裨佐悉有功賞。臣見洮、渭父老言,皇祐中,轉運使祥因熟羌數被寇掠,其部族願輸土置城以為守御,乃即古渭為砦。祥此舉足以消沮邊隙,可謂知攻守之利矣。兵出少挫,身黜謀廢,臣竊悲之。冀推原舊功,少賜褒恤,使天下知祥死猶被恩,且舒祥忠義之氣。」詔贈秘書,錄一子未官者。子育。



 育字巽之,舉進士,為涇陽令。以養親謁歸,從張載學。有薦之者,召見,授崇文校書、監察御史裏行。神宗喻之曰:「《
 書》稱『SW讒說殄行』,此朕任御史之意也。」育請用《大學》誠意、正心以治天下國家,因薦載等數人。西夏入環慶,詔育行邊,還言:「寶元、康定間,王師與夏人三大戰而三北,今再舉亦然。豈中國之大,不足以支夏人數郡乎?由不察彼己,妄舉而驟用之爾。昨荔原之役,夏人聲言:『我自修壘,不與漢爭。』三犯之,然後掩殺,雖追奔亦不至境。由是觀之,其情大可見矣。」



 又使河東,論韓絳築囉兀二砦:「始調外郡稍遠邊城前後三十萬夫,遼州最為窮僻,然
 猶上戶配夫四百三十四,僦直計三千緡,下者十六人,其直十萬。輦運所經二十二驛,宣撫司不先告期,轉運使臨時督辦,致民皆破產,上下莫敢言。獨遼守李宏能約民力所勝,而饋不失期,顧以訴其實,翻令鞫罪。願貸被劾官吏,其芻糧在道者隨所至受之,使已困之民咸蒙德澤。」神宗皆從之。坐劾李定親喪匿服,罷御史,檢正中書戶房,固辭,乃知韓城縣。



 詔往鄜延議畫地界,育言:「保疆不如持約,持約不如敦信。前日疆埸嘗嚴矣,一旦
 約敗兵拏,鬥者跌於前,耕者侵於後,是封溝不足恃也。使人左去而兵革右興,金繒朝委而烽煙夕舉,是持約不足恃也。今我見利而加兵,當講好之後,復自立界,不亦愧乎!」安南行營郭逵、趙契以兵十萬伐交址,行及長沙,病死相屬,逵、契又不輯睦,育疏其不便,不從。久之,知河中府,加直集賢院,徙鳳翔,以直龍圖閣鎮秦州。



 元祐初,召為太常少卿,改光祿卿、樞密都承旨。劉安世暴其閨門不肅,出知熙州。時又議棄質孤、勝如兩堡,育爭之
 曰:「熙河以蘭州為要塞,此兩堡者蘭州之蔽也。棄之則蘭州危,蘭州危則熙河有腰膂之憂矣。」又請城李諾平、汝遮川,曰:「此趙充國屯田古榆塞之地也。」不報。入為給事中、戶部侍郎,卒。高宗紹興中,採其抗論棄地及進築之策,贈寶文閣學士。



 田京,字簡之,世居滄州,其後徙亳州鹿邑。舉進士,調蜀州司法參軍,自秦州觀察推官改秘書省著作佐郎,為大理寺詳斷官。



 趙元昊反,侍讀學士李仲容薦京知兵
 法,召試中書,擢通判鎮戎軍。夏守贇為陜西經略使,奏兼管勾隨軍糧料。入對,陳方略,賜五品服。尋為經略安撫判官。守贇既罷,以武略應運籌決勝科,及試秘閣,與他科偕試六論,京自以記誦非所長,引去。



 又參夏竦軍事。會遺翰林學士晁宗愨即軍中問攻守孰便,眾欲大舉入討,京曰:「夏人之不道久矣,未易破也。今欲驅不習之師,深入敵境,與之角勝負,此兵家所忌,師出必敗。」或曰:「不如講和。」京曰:「敵兵未嘗挫,安肯降我哉?」未幾,元昊
 使黃延德叩延州乞降,以奇兵出原、渭,敗大將任福。夏竦素不悅京,坐是改通判廬州,徙知邵武軍,提點河北路刑獄事。乃上言:「請擇要官守滄、衛,鑿西山石臼廢道以限戎馬,義勇聚教,復給糧,置卒守烽燧,用奇正法訓兵,徙戰馬內地以息邊費。」凡十餘事,仁宗頗嘉納之。



 入為開封府判官,坐械囚送獄道死,出知蔡州,徙相、邢二州,復提點河北刑獄事。王則據恩州反,京縋城趣南關,入驍健營撫士卒。保州振武兵焚民居欲應賊,京捕斬
 之乃定。賊遣其黨崔象偽出降,京以其持妖言惑眾,又斬以徇,由是營兵二十六指揮在外者皆懾服,不敢叛。州之南關,民眾多如城中,得不陷賊,京有功焉。京督士攻城甚力,賊系京妻子乘城迫使呼曰:「毋亟攻,城中將屠我輩矣。」京叱諸軍益進攻,注矢仰射,殺其家四人。賊知京無所顧,乃牽妻子去,恩州平。以不能預察賊,降監鄆州稅。



 先是,駐泊都監田斌亦以賊發不能捕,待罪兵間,及城破,從諸將入,以功遷宮苑副使,而京獨被謫。御
 史言失察賊過輕,忘家為國義獨重,不宜左遷,乃徙通判兗州。又徙知江陰軍,知密州,歷提點淮南刑獄事、京西轉運使,累遷兵部員外郎、直史館、知滄州轉運使。



 京能招輯流民,為之給田除稅租,凡增戶萬七千,特遷工部郎中。然傳者謂流民之數多不實,又強為人田非其所樂,侵民稅地,仿古屯田法,其後法不成,所給種錢牛價,民多不償,鞭笞督責,至累年不能平,公私皆患之。擢天章閣待制、陜西都轉運使,改兵部郎中,復知滄州,拜
 右諫議大夫,卒。



 京喜論議,然語繁而迂,頗通兵戰、歷算、雜家之術。為人尚氣節,少時與常山董士廉、汾陰郭京相友善,俱以倜儻聞。著《天人流術》、《通儒子》十數書,又有奏議十卷。



 論曰:人臣之職,當奮不顧身,而庸人怯夫於國事則噎喑而不言,若胡越肥瘠之不相干,如張述者其亦忠且果矣。黃震指李溥忤權臣,胡順之擊強宗,為眾人所不敢為;陳貫論兵事,範祥畫邊計,皆一時雋士。妖盜竊發,
 京出孤力保城南,置妻孥之憂,先登示賊,其勇蓋可壯也。



\end{pinyinscope}