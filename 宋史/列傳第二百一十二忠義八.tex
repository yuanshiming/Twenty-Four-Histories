\article{列傳第二百一十二忠義八}

\begin{pinyinscope}

 ○高永年鞠嗣復宋旅丁仲修項德附孫昭遠曾孝序趙伯振王士言祝公明附薛慶孫暉李靚楊照丁元附宋昌祚李政姜綬劉宣屈堅王琦韋永壽附鄭覃姚興張玘陳亨祖王拱劉泰孫逢李熙靖趙俊附姚邦基劉化源胡唐老王儔朱嗣孟附劉晏鄭振孟彥卿高談連萬夫謝皋附王大壽薛良顯唐敏求
 王師道



 高永年,河東蕃官也。為麟州都巡檢。王贍取青唐,永年總蕃兵為先鋒。贍入邈川,而宗哥叛,永年以千騎直抵其城,開省章峽路,擊走叛羌,結陣還青唐。羌攻甚急,復擊之去。會苗履、姚雄以援師至,戰溪蘭宗堡,履少卻,永年領勁騎斷羌為二,乃退。復與李克保敦谷,又戰於乾溝,單馬援矛,刺羌酋彪雞廝萬眾之中,斬其首,餘眾宵遁。已而隴拶自乾溝逼鄯州,永年佐贍拒守,及雄棄湟、
 鄯,皆以永年殿歸師。



 崇寧初,知岷州。蔡京議復兩州,王厚使永年帥兵二萬出京玉關,克安川堡,遂至湟,即知州事。自皇城副使進四方館使、利州刺史,為熙、秦兩路兵都統制,將前軍駐宗哥北。溪賒羅撒萃精勇據高阜,欲沖官軍,永年揮選鋒突陣,師乘之,羌大敗,遂平鄯州。遷賀州團練使,知其州。



 溪賒羅撒合夏國四監軍之眾,逼宣威城,永年出御之。行三十里,逢羌帳下親兵,皆永年昔所推納熟戶也。永年不之備,羌遽執永年以叛,遂
 為多羅巴所殺,探其心肝食之,謂其下曰:「此人奪我國,使吾宗族漂落無處所,不可不殺也。」是役也,王厚實主其事,而謀策皆出永年,乃劾永年信任降羌,坐受執縛,故贈恤不及雲。



 永年略知文義,范純仁嘗令贄所著書詣闕,作《元符隴右錄》,不以棄湟、鄯為是,故蔡京用之,雖成功,然竟以此死雲。



 鞠嗣復,不知何許人。宣和初,知歙州休寧縣。方臘黨破縣,欲逼使降,面斬二士以怖之,嗣復罵曰:「自古妖賊豈
 有長久者,爾當去逆從順,因我而歸朝,官爵尚可得,何為脅我使降?」嗣復知必死,不少懾,屢言何不速殺我,賊曰:「我,縣人也。明府宰邑有善政,我不忍殺。」乃委之而去。初,嗣復聞難,率吏民修城立門,眾赴功,守備略就。朝廷知之,進其官二等,加直秘閣,擢知睦州。嘗為賊所傷,自力度江乞師於宣撫使,未及行而卒。



 宋旅字庭實,莆田人。第進士,累官奉議郎、知剡縣。方臘既陷歙、睦、杭、衢、婺五州,且犯越,越盜亦起應之。縣吏多
 遁,旅遣妻子浮海歸閩,獨與民據守,以忠義激勸,部勒隊伍,為豫備計。俄而盜眾大至,射率壯銳,冒矢石,雖頗殺獲,終以力不敵,遂死之。越帥劉韐上其事,詔贈朝散郎,錄其四子。



 丁仲修字敏之,溫州人。方臘黨俞道安陷樂清,將渡江。巡檢陳華往捕,死之。先鋒將張理同、李振出南門迎敵,渡八接橋,橋斷馬蹶,溺死。賊至帆游,夏祥遣輔褒迎戰數十合,褒死之。仲修帥鄉兵禦諸樂灣,鄉兵失據而散,
 仲修以餘兵與賊戰,力屈乃死。



 項德,婺州武義人,郡之禁卒也。宣和間,盜發幫源,明年陷婺,而邑隨沒。德率敗亡百人破賊,因據邑之城隍祠。自二月訖五月,東抗江蔡,西拒董奉,北捍王國,大小百餘戰,出則居選鋒之先,入則殿後,前後俘馘不可勝計。賊目為「項鷂子」,聞其鉦則相率遁去。方謀復永康諸縣,而官兵至,德引其眾欲會合,賊盡銳邀之黃姑嶺下,德戰死。邑人哭聲震山谷,圖其像,歲時祭之。



 孫昭遠,字顯叔,其先眉州眉山人。元祐間進士,調長沙尉,闢河東經略司乾當公事。歷鳳翔府天興縣、河北山東撫諭盜賊乾當公事,尋擢河北、燕山府路轉運使。



 靖康元年,召為水部員外郎。金人圍太原,宋師多潰,欽宗遣折彥質乘傳同昭遠招集。會洛陽陷,西京留守、西道總管王襄徙治襄、漢,授昭遠西道總管。道收潰卒至京兆,遇永興路安撫範致虛會諸軍入援,昭遠督其進,且檄諸道使出師。環慶帥王似、熙河帥王倚各以師會,涇原
 帥席貢、秦鳳帥趙點、鄜坊使張深皆後師期,昭遠二十有八疏劾之。合諸道兵得十萬,命馬祐昌統之。昭遠與致虛同出關,祐昌與金人戰敗。京師陷,遣使至大元帥府。



 建炎元年,遷河南尹、西京留守、西道都總管。至洛收集散亡,得義兵萬餘人,柵伊陽,使民入保。其冬,金人來攻,昭遠遣將姚慶拒戰,軍敗,慶死。昭遠命將官王仔奉啟運諸殿神御,間道走行在。金兵益熾,昭遠戰不利,其下欲擁昭遠南還,昭遠罵曰:「若等平日衣食縣官,不以
 此時報國,南去何為!」叛兵怒,反擊昭遠,遂遇害。官屬無免者。四年,追贈徽猷閣待制。



 曾孝序,字逢原,泉州晉江人。以蔭補將作監主簿,監泰州海安鹽倉,因家泰州。累官至環慶路經略、安撫使。過闕,與蔡京論講議司事,曰:「天下之財貴於流通,取民膏血以聚京師,恐非太平法。」京銜之。時京方行結糴、俵糴之法,盡括民財充數,孝序上疏曰:「民力殫矣。民為邦本,一有逃移,誰與守邦?」京益怒,遣御史宋聖寵劾其私事,
 追逮其家人,鍛煉無所得,但言約日出師,幾誤軍期,削籍竄嶺表。遇赦,量移永州。京罷相,授顯謨閣待制、知潭州。復以論徭事與吳居厚不合,落職知袁州,尋復職,再知潭州。



 道州徭人叛,乘高恃險,機毒矢下射,官軍不得前,於兩山間僕巨木,橫累以守。孝序夜遣驍銳攀援而上,以大兵繼進,破平之。進顯謨閣直學士,遷龍圖閣直學士、知青州。繕修城池,訓練士卒,儲峙金谷,有數年之備,金人不敢犯。高宗即位,遷徽猷閣學士,升延康殿學
 士,召赴行在。既而青州民詣南都借留,許之。



 先是,臨朐土兵趙晟聚眾為亂,孝序付將官王定兵千人捕之,失利而歸。孝序責以力戰自贖,定乃以言撼敗卒,奪門斬關入,孝序出據廳事,瞋目罵之,遂與其子宣教郎訏皆遇害,年七十九。城無主,遂陷。



 知臨淄縣陸有常率民兵拒守,死於陣。知益都縣張侃、千乘縣丞丁興宗亦死之。後贈孝序五官,為光祿大夫,謚威愍;子訏承議郎。有常朝散郎,錄其家一人。贈侃、興宗二官,官二子。



 趙伯振,太祖八世孫。宣和六年進士。靖康末,為鄭州司錄,捍禦有功。上聞之,就遷直秘閣、通判州事。建炎元年,金人犯鄭州,守臣董庠棄城走。越八日城陷,伯振率兵巷戰,中流矢墜馬,遂遇害。事聞,贈朝請大夫,官其一子。



 王士言,武舉進士。累立戰功,西北服其威名。宣和初,擢河東廉訪使者。方臘為寇,詔擇材略之士,馮熙載薦為東南第三將,首解嘉興之圍。靖康元年,詔以浙西兵往河東防秋。金人攻澤州,畢力守御,金兵日增,士言分必
 死。他將力屈,城西南遂陷,乃使親卒持劍歸報,巷戰而死。康允之上其事,贈拱衛大夫、忠州團練使,官其後五人。



 祝公明,處州麗水人。太原府盂縣主簿。靖康間,金人犯河東,令棄官去,公明攝縣事,率保甲入援,圍守逾年,城陷不屈。子陶,為唐州司戶,中原失守,陶亦死官所。建炎中,贈公明承事郎。



 薛慶,起群盜,據高郵,兵數萬人,多驍雋敢鬥,能以少擊
 眾,附者日多。張浚聞慶無所系屬,欲歸麾下,親往招之。慶感服,因使守高郵,尋遷拱衛大夫、福州觀察使、承州天長軍鎮撫使。金人還自浙,屯天長、六合間,慶率眾劫之,得牛數百,悉賤估分畀民之力田者。



 金人欲自運河引舟北歸,而趙立在楚,慶在承,扼其沖不得進。金左監軍昌來見兀術,欲會兵攻楚州,真、揚鎮撫郭仲威聞之,約慶俱往迎敵。慶至揚州,仲威殊無行意,置酒高會。慶怒曰:「此豈縱酒時耶?我為先鋒,汝當繼後。」上馬疾馳
 去,平旦出揚州西門,從騎不滿百,轉戰十餘里,亡騎三人,仲威迄不至。慶與其下奔揚州,仲威閉門拒之,慶倉皇墜馬,為金追騎所獲。馬識舊路還,軍中見之曰:「馬還,太尉其死乎。」金人殺慶,承州陷。訃聞,贈保寧軍承宣使,官其家十人,封其妻碩人。



 孫暉,為泗州招信縣尉。建炎三年正月,金人陷泗州,州守呂元、閻瑾焚淮橋遁。金人由招信將渡淮,暉將射士民兵御之,沈其數舟。會大霧蔽日,金人莫測其多寡,相
 持逾半日,以疑兵縻暉,自上流渡兵。暉又戰且卻,城破,竟死於敕書樓。



 李靚字彥和,吉州龍泉人。幼孤,母督之學,不肯卒業,母詰之,辭曰:「國家遭女真之變,寓縣雲擾,士當捐軀為國勘大憝,安能呫囁章句間,效淺丈夫哉?」岳飛督師平虔寇,挺身從之,未行,奔母喪。服除,走淮南,以策干都督張浚,浚奇之,使隸淮西總管孫暉戲下。累功授承信郎。紹興十年,金遣其將翟將軍犯境,靚與部曲當其鋒,轉戰
 至西京天津橋南,俘翟將軍,乘勝逐北。會金兵大至,遂死之,年三十一。



 楊照者,濠州將官也。金人圍城急,照躍上角樓,刺賊之執黑旗者,洞腹抽腸而死。照俄中流矢,卒。有統領丁元者,遇金人十八里洲,被圍,元大呼其徒,勉以毋得負國。一舟二百人皆鬥死。詔並贈承信郎,錄其後。



 宋昌祚,和州鈐轄也。建炎三年,兀術犯和州,州人推昌祚權領軍事,率眾堅守,金人圍之數匝。禁軍左指揮使
 鄭立亦拳勇忠憤,共激士卒,晝夜備御不少怠。閱數日,軍士胡廣發弩中兀術左臂,兀術大怒,飛炮雨集,徑登弩發之地,城立破,金人入屠其城。昌祚與權倅唐璟、歷陽令蹇譽、司戶徐兟、縣尉邵元通及立、廣皆死譙樓上,磔裂以徇。軍士多不降,潰圍西出,保麻湖水砦,推鄉豪為統領。聞於朝,遂以趙霖為和州鎮撫使,昌祚、璟、譽、兟、元通各贈官,錄其子弟。



 李政,為雲騎第六指揮,在京東立戰功,補官授河北將
 官,冀州駐扎。靖康二年,知州權邦彥以兵赴元帥府勤王,金兵來攻,政守御有法,紀律嚴明,軍民皆不敢犯。金屢攻城,政皆卻之。夜搗其砦,所得財物盡散士卒,無纖毫入私家。號令明,賞罰信,由是人皆用命。俄攻城甚急,有登城者,火其門樓,與官兵相隔,政呼曰:「事急矣。有能躍火而過者,有重賞。」於是有十數人皆以濕氈裹身,持仗躍火而過,大呼力戰,金人驚駭,有失仗者,遂敗走。政大喜,皆厚賞之。未幾政死,城遂陷。權知州事單某者不降,自經死。



 姜綬,處州麗水人。金人再犯京師,內外不相聞。朝遷募忠勇士齎蠟書往南京總管司調兵赴援,綬以忠翊郎應募,乃刲股藏書,縋下南壁,為邏騎所獲,厲聲叱罵,遂被害。建炎中,州上其事,官其子特立承信郎。



 劉宣,為秦鳳路兵馬都監。金人入關、陜,宣遣蠟書密與吳玠相結,且率金將任拱等以所部歸朝。約日已定,有告之者,金人取宣縷擘之,其家屬配曹州。



 屈堅,為右武大夫、忠州防禦使。建炎二年,金人圍陜府,
 堅引所部救之。圍解,金人執堅,堅曰:「始吾所以來,為解圍也。城茍全,吾死何憾。」叱金人使速殺之。後贈三官,錄其家五人。



 王琦,為弓門砦巡檢。建炎四年,金人還自熙河,琦御之。金人立招降旗榜,改年號阜昌,眾皆拜,琦獨不屈。金人執而殺之。



 韋永壽者,紹興三十二年,以統制官與金人戰和州,子承節郎世堅救之,同死。張浚以言,贈中衛大夫、融州觀
 察使,世堅贈三官。



 鄭覃,字季厚,明州人。靖康二年貢於鄉。建炎四年春,金人陷明州,縱兵大掠,覃挈族闢難山谷間。金人追及,與兄章俱被執,脅以刃,曰:「予吾金,即貰死。」覃號泣指所瘞黃金釵遺之,遂見釋。而金兵相屬,覃拿小舟與其妻董同載去,顧謂章曰:「萬一不得脫,覃豈北面事異國者,兄勉主祭祀。」復為兵所劫去,迫使之降,覃厲辭罵不屈,躍入水中。董哭曰:「夫亡矣,與其受辱以生,不如死。」亦自沈。



 覃死後,孫、曾多舉進士,而清之最顯。覃累贈太師、秦國公,董秦國夫人。



 姚興,相州人。靖康中,以州校用。劫殺金人有功,借補承信郎。建炎初,張琪聚兵歸東京留守宗澤,興往從之,又從琪依劉洪道於池州。紹興元年,琪叛,掠饒州,呂頤浩招降之。琪既聽命而中變,執總管巨師古將殺之,興密諭所部,挾師古同其妻游騎而馳,夜歸頤浩。頤浩義之,請於朝,授武義郎,隸張俊軍中。復從劉錡守順昌,復宿、
 亳,下城父、永城、臨渙、蘄縣朱家村,遷武略大夫。戰淮壖有功,授右武大夫,累遷建康府駐扎御前破敵軍統制,充荊湖南路兵馬副都監。



 紹興三十一年,金人渝盟,興隸都統王權麾下,遇金兵五百騎於廬州之定林,與戰卻之,生得女直鶻殺虎。初,金主亮在壽春,江、淮制置使劉錡命權將兵迎敵,權怯懦不進,錡督戰益急,權不得已守廬州。及金兵渡淮,權遣興拒之,而退保和州。興與金人遇於尉子橋,金人以鐵騎進,興麾兵力戰,手殺數
 百人。權奔仙宗山,嚴兵自衛,興告急不應,統領戴皋帥馬軍引避。初,李二者,嘗有私恩於權,因得出入軍中,往來兩界貿易,間竊權旗幟遺金人。至是,金人立權旗幟以誤興,興往奔之,父子俱死焉。



 事聞,詔贈容州觀察使,又特官其後三人,即其砦立廟。既復淮西,又立廟戰所,賜額旌忠。開禧元年,戶部侍郎趙善堅言:「近守邊藩,詢訪故老,姚興以四百騎當金人十數萬,自辰至午,戰數十合,援兵不至,竟死於敵。金人相謂曰:『有如姚興者
 十輩,吾屬敢前乎?』興忠勇如此,宜超加爵謚。」於是賜謚忠毅。



 張玘字伯玉。世居河南澠池。建炎中,以家財募兵討金人,從者數千人。時翟興制置京西,玘以眾屬焉。金兵長驅渡河,玘御之白浪口,金人不得渡。積功補武翼大夫、成州刺史。董先為制置司前軍統制,玘佐之,每戰,冒矢石為諸軍先。



 紹興元年,金將高瓊率眾取商州。董先御之,玘乘銳奔擊,從騎不能屬,單馬至四皓廟,金兵數百
 騎至,玘瞋目大呼,挺刃突擊,金兵披靡莫敢向。是日,九戰九捷,追至試劍關,爭門,蹂踐死者百人。明年春,偕先繇藍田渡渭,規取長安。時偽齊經略使李諤屯渭北,與金將折合孛堇相為聲勢。玘陳兵華嚴川,俄白氣貫日,吏士歡奮,戰於興平、咸陽、渭河、石鱉穀。



 時劉豫據京師,先軍乏食,偽降豫,不挈家,玘事其夫人如舊。豫使人迎其妻,先密書報玘勿遣,且述必還意。王倚攝虢州,從偽意堅,玘患之。會別將董震自商州來,倚喜曰:「震與我善,
 今以兵來,天贊我也。」乃與震謀害玘。震陽許而陰以告。翼日,倚詣玘議事,玘叱下,責以大義,並推官祁宗儒斬之。先是,豫遣人持詔撫諭,以玘為商虢順州路兵馬都監、同統制軍馬,玘囚其使,至是並戮之。



 於是偽齊河南安撫孟邦雄、總管樊彥直據洛陽,兵直抵長水。玘遣將陳俊守白馬山,謝皋守船板山,梁進守錦屏山,盡匿精銳。金兵深入,玘戰東關,三砦響應,金兵潰。玘率精騎三千,一日夜馳三百里,黎明抵河南,邦雄就擒,彥直遁去。
 便宜升霸州防禦使。三年春,先自偽齊歸,玘還兵柄,退就位,時人義之。



 初,翟興既死,朝廷命其子琮襲,至是琮言於朝,真授玘武翼大夫、果州團練使、河南府孟汝唐州馬步軍副總管。擊金將閻銳於唐、鄧間,先登殺獲千餘人。未幾,詔先一行並聽神武後軍統制。玘從岳飛復京西六州,平湖賊鐘子義等,累功進拱衛大夫。入侍衛,始以誅王倚事聞,敕付史館,賜褒詔,進親衛大夫。



 三十二年,領御營宿衛前軍都統,屯泗州。時金人攻海州急,
 詔玘會鎮江都統制張子蓋赴之。賊環城數十匝,矢石如雨,玘戰於州北三里,麾精騎沖其陣,手殺數十人,殲其長,殺獲萬計,海州圍解。玘中流矢卒,子蓋上其功,特贈正任觀察使,官其後九人,廟號忠勇。孝宗即位,又命祠於戰所,贈清遠軍承宣使。



 子世雄,歿於符離之戰,贈武節大夫。



 陳亨祖者,淮寧大豪也。紹興末,官軍已復蔡州,亨祖遂領民兵據淮寧,執金知州完顏耶魯,以其城來歸。命為
 武翼大夫、忠州刺史、知淮寧府。金兵攻城,亨祖力戰死之,舉家五十餘人皆死。贈容州觀察使,立廟光州,賜額閔忠。



 王拱,建康府前軍統制。從都統邵宏淵收復虹縣,進取宿州,屢立奇功。隆興元年五月,與金人接戰,深入營中,自辰至申,力戰死。詔贈正任觀察使,官其家八人。許奏異姓,賜銀三百兩,即其砦立廟,賜額忠節。



 是役也,中亮大夫朱贇亦死之,贈承宣使。



 劉泰,樞密院忠義前軍正將也。慷慨好義,以私財募兵三百,糧儲器械一切不資於官。金人犯壽春,泰率所部赴援,轉戰累日,金人引去,泰身被數十創,一夕死。詔贈武翼郎,官其家三人。



 孫逢,眉山人。大觀四年進士,累官至太學博士。張邦昌僭立,有司趣百僚入賀,逢獨堅臥不起。夜既半,同僚強起之,不從,至垂泣與之訣。時祠部員外郎喻汝礪聞變,捫其膝曰:「不能為賊臣屈。」遂掛冠去。事畢,有司舉不至
 者,欲以逢與汝礪復於金人,邦昌以畢至告,乃免。逢聞之曰:「是必將肆赦遷官以重污我,



 我其可俟!」遂發疾而卒。



 李熙靖,晉陵人。提舉醴泉觀。邦昌使直學士院,熙靖固拒,因憂憤不食,疾且篤,謂友人曰:「百官何日再朝天乎?」泣數行下。邦昌又命禮部侍郎譚世績權直學士院,世績亦稱疾堅臥不起。熙靖尋卒。後並贈延康殿學士。



 趙俊字德進,南京宋城人。紹聖四年進士,官至朝奉郎。
 隱居杜門,雖鄉里不妄交。劉安世無恙時居河南,暇則獨一過之。徐處仁與俊厚善,及為丞相,鄉人多見用,俊未嘗往求,處仁亦忘之,獨不得官。



 建炎末,士大夫皆避地,俊獨不肯,曰:「但固吾所守爾,死生命也,避將安之?」衣冠奔踣於道者相繼,俊晏然不動。劉豫以俊為虞部員外郎,辭疾不受,以告畀其家,卒卻之,如是再三,豫亦不復強。凡家書文字,一不用豫僭號,但書甲子。後三年卒。



 承直郎
 姚邦基者,蜀人也。知尉氏縣,秩滿不復仕,屏居村落間,授徒自給。



 時宗室南渡不及者,尚散居民間,豫募人索之,承務郎閻琦匿不以聞,為人所告,豫杖之死。



 劉化源,耀州人。紹聖元年進士。建炎初,金人陷關陜,守令以城降者,金人因而命之。化源時知隴州,不肯降,城陷被執。金人使人守之,不得死,遂驅入河北,鬻蔬果、隱民間者十年,終不屈辱。



 有米璞者,與化源同鄉里,西人皆敬之。璞登政和二年進士第,時通判原州,劉豫欲官
 之,杜門謝病,卒不污偽命。



 有劉長孺者,亦耀州人。時簽書博州判官廳公事,與豫書,備陳祖宗德澤,勸以轉禍為福。豫怒,追其官,囚之百日,長孺終不屈。豫後復官之,不從。紹興九年,宣諭使周聿上之朝,詔赴行在,而簽書樞密院事樓炤言璞苦風痺,化源、長孺老病,遂命各轉兩官奉祠;又言新鳳翔教授陰晫守節不仕,詔特改令入官。其後金復渝盟,長孺知華陰縣,不屈而死。



 有李嚞者,開封人。宣和六年進士。建炎中,知彭陽縣,亦不降,與
 民移治境上。令執之以獻,金人欲官之,凡三辭。其後金人以為歸附,命為儒林郎,嚞言於所司曰:「昔為俘獲,不敢受歸附之賞。」還其牒。劉麟聞其賢,命張中孚以禮招致,嚞力拒之。紹興九年死原州。事聞,贈奉議郎,官其家一人。



 胡唐老,字俊明,樞密副使宿之曾孫也。崇寧間,與弟世將同登進士第。歷南京國子博士,知江陵縣,召為秘書省校書郎。靖康元年,擢殿中侍御史。金人再犯京師,攻
 圍日急,唐老請對曰:「城危矣。康王北使,為河朔士民留不得進,殆天意也。請就拜大元帥,俾召天下兵入援。」宰相何㮚是之,遂遣秦仔持蠟書詣相州,拜王河北兵馬大元帥。



 時朝廷趣西兵入衛,而不立帥。唐老疏:「乞命範致虛為宣撫使,節制諸路以進,不然必無功。」不聽。後致虛以孤軍與金人戰淆、澠間,它路兵不至,遂敗。



 京城破,金人搜括金銀,分命朝臣董之,以臺臣糾察,唐老預焉。出知無為軍。朝廷竄逐偽命之臣,坐降二官。先是,金人
 怒民間多匿金銀,杖唐老幾死,以疾得免稱臣於偽楚。至是,唐老不自言故,例從貶秩。



 三年,知衢州。苗傅敗走,以亂兵犯城,唐老拒之。會大雨雹,城上矢石俱發,賊不支,遂解去。以功擢秘閣修撰,未幾,進徽猷閣待制,充兩浙宣撫司參謀官,知鎮江府兼浙西安撫使。



 杜充降於金,建康失守,潰卒戚方等趣鎮江,城壁頹圮,兵不滿千,獨倚浙西制置韓世忠為重。世忠復去,唐老度力不敵,因撫之。無何,方欲犯臨安,妄言赴行在,請唐老部眾以
 行。唐老不從,諭以逆順禍福,方眾環脅之,唐老怒罵方,遂遇害。詔贈徽猷閣直學士,謚定愍。



 時安撫司機宜鄭凝之亦以兵死,詔官其家一人。凝之,戩孫也。



 王儔,以通判真州權通判廣德軍。建炎末,盜戚方既為劉晏所破,引兵欲趨宣城,道過廣德,入其郛。儔不屈,與權判官李唐俊、權司法潘偊、權知廣德縣韋績、權丞蔣夔皆死。後贈儔二官,唐俊等皆京秩,錄其家一人。



 朱嗣孟,饒州樂平人。宣和間進士,為廣德司戶兼司理。
 叛卒戚方破鎮江,犯廣德,守倉皇遣招安,無敢往者,奇嗣孟狀貌有膽略,遂以命焉。嗣孟雅自負,不復遜,直詣賊壘,問所以涉吾地何故,為陳逆順禍福,使自擇所處。方以迕己殺之。事聞,贈宣教郎,官其子。



 劉晏,字平甫,嚴州人。入遼,舉進士,為尚書郎。宣和四年,帥眾數百來歸,授通直郎。金人犯京師,以晏總遼東兵,號「赤心隊」。



 建炎初,從劉正彥擊淮西賊丁進。進黨頗眾,晏所提赤心騎才八百,乃為五色旗,使騎兵持之,循山
 而出,一色盡則以一色易之。賊見官軍累日不絕,顏色各異,遂不戰而降。遷朝散郎。正彥反,晏謂其部曲曰:「吾豈從逆黨者耶?」以眾歸韓世忠。世忠追正彥及苗傅於浦城,以晏騎六百為疑兵於浦山之陽,賊大駭,晏以所部力戰。正彥既擒,世忠上其功,遷一官。



 金人犯建康,杜充兵潰,世忠退保江陰,晏領赤心百五十騎屯青龍。群寇犯常州,郡守請晏為援,晏以精銳七千人出奇破之。進直龍圖閣。保馬跡山以捍寇,寇再至,晏選舟師迎戰,
 降其眾千五百人,郡人為晏立生祠。



 戚方圍宣城,急命晏往援,晏至城下,未立營壘,出不意直搗方帳下,方大驚卻走。晏欲生致方,單騎追之,方率其眾迎戰,晏不能敵,猶手殺數十人,為賊所害。事聞,贈龍圖閣待制,官其子四人,於死所立廟曰義烈,歲時祀之。



 鄭振,字亨叔,興化軍仙游人。建炎中,盜楊勍起,邑令檄振糾集民兵以御之。振力戰,賊眾披靡,一夕遁去。紹興十三年,群盜曾少龍、周老龍、何白旗、陳大刀眾至數萬,
 帥司檄振行,盜素聞振名,不戰自屈。十六年,盜詹鐵義者,入振井里,振帥眾拒之,殺數十人,遂遇害。廟食里中。



 有孫知微者,以朝請大夫通判舒州。紹興元年,賊劉忠入其境,執知微以去,知微不屈,忠怒,臠而食之。



 孟彥卿,忠厚從父也,頗知兵。通判潭州。建炎三年,潭城中叛卒焚掠,自東門出,帥臣向子諲命彥卿領兵追之。已而招安其眾。未幾,潰兵杜彥自袁州入瀏陽,遂犯善化、長沙二縣。彥卿率民兵拒之,手殺數人,賊勢挫,退還
 瀏陽。彥卿追與之戰。俄而民兵有自潰者,賊遂乘之,斬彥卿,持其首以告所掠民兵曰:「此善戰孟通判首也。」因支解以徇。



 添差通判趙民彥以民兵赴之,鏖戰瀏陽城南南流橋,依山為陣,殺傷甚眾。偶為間者折其陣中認旗,眾驚謂民彥已敗,遂潰,民彥為賊所得。邑士謝淳以才勇,眾推之帥民兵為前鋒,助民彥戰。淳手殺數十人,力屈亦被執。賊並殺之。事聞,彥卿、民彥並贈直龍圖閣,官其家各三人。淳字景祥,贈成忠郎,官其子晞古。朱熹
 帥湖南,請為彥卿、民彥立廟,以淳侑之。



 高談字景遂,邵武光澤人。紹定二年,旁郡盜作,諸子請避之,談曰:「昔楊子訓問避寇於胡文定公,語之曰:『往歲盜起燕山,則河北、關中可避;入關,則淮南、漢南可避;今惟二廣,寧保其無寇乎?吾惟存心以聽命爾。』小子識之,此格言也。今南去則汀、劍,西去則盱、贛,皆為盜區;東去富、沙,雖有城避,吾聞官吏例弗我納;北去廣信,防夫、守隸利人囊篋,指民為諜,數剽殺之。舍胡公之言未有他策也。」
 盜入,諸子又請,談曰:「有廟祏在,將焉之?」



 盜至,談出曰:「時和歲豐,何忍為此?」盜曰:「吏貪暴,民無所訴,我為直之。」談曰:「獨不能楇鼓上聞乎?民何辜而殺之。」盜怒,執諸庭。遺之牛酒,不釋;遺之金帛,不釋。談曰:「然則將何為?」盜曰:「我欲東破武陽,若得耆老如爾者,率是鄉子弟,吾其濟乎。」談曰:「斯言奚為至我。」唾賊大罵,遂遇害,而里人賴以免。



 談平居言動,必由禮法,故鄉人敬而附之。



 連萬夫,德安人,或曰南夫弟也。補將仕郎。建炎四年,群
 賊犯應山,萬夫率邑人數千保山砦,賊不能犯。寇浪子者以兵至,圍之三日,卒破之。賊知萬夫勇敢有謀,欲留為用,萬夫怒,厲聲罵賊,為所害。贈右承務郎,官其家一人。



 謝皋者,開封人,為鎮撫司統制官。李成陷虢州,欲降之,皋指腹示賊曰:「此吾赤心也。」自剖其心以死。



 王大壽,泉州人,為左翼隊將。紹定五年,海寇王子清犯圍頭,守真德秀遣大壽領卒百人防遏。猝與賊遇,奮前
 控弦,斃賊十餘,後無援者,遂沒。從死者五人。賊就俘,剖心祭之。事聞,贈官,恤其家。



 薛良顯,字貴勤,溫之瑞安人。登崇寧二年進士第,累官為大宗正丞,出為江東轉運使。江寧軍校周德作亂,良顯聞變,率眾與戰,斬十餘級,力不勝,死之。事聞,贈恤良渥。



 唐敏求,字好古,太平當塗人。宣和六年進士,調德化主簿。盜起,敏求挺身率眾捍賊,度力不能支,諭以禍福,賊
 憤詆觸,噪而前,遂遇害。事聞,加贈升朝官,仍補其子楠將仕郎。



 王師道,字居中,兗州人。為人沈勇。任吉州慄傳砦巡檢。紹興中,與盜戰於吳村,每射輒斃,追擊數里,遇賊有伏於民居者,挺身力戰,遂死。立廟其地。部使者以聞,官其二子。



 王輝者,青州人。亦嘗為慄傳砦巡檢。靖康初,詔起義兵,輝應募,立奇功,官至正使,寓吉州。淳熙二年,茶寇犯邑,郡以輝驍勇,檄之使行。至勝鄉,地險,輝勇於進,士
 卒不繼,為賊所得,以刃加頸欲全之,輝含血大罵,遂死。帥司以聞,贈忠州刺史,與恩澤二人,立廟羅陂。



 陳霖者,字傅容,泉州人。嘉定十三年進士,為瑞金尉。盜起江、閩,霖迎敵力戰,盜系之以去,不屈遇害。



\end{pinyinscope}