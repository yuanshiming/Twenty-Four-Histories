\article{志第五十 河渠七}

\begin{pinyinscope}

 東南諸水下



 淮郡諸水:紹興初,以金兵蹂踐淮南,猶未退師,四年,詔燒毀揚州灣頭港口閘、泰州姜堰、通州白莆堰,其餘諸堰,並令守臣開決焚毀,務要不通敵船;又詔宣撫司毀
 拆真、揚堰閘及真州陳公塘,無令走入運河,以資敵用。五年正月,詔淮南宣撫司,募民開浚瓜洲至淮口運河淺澀之處。



 乾道二年,以和州守臣言,開鑿姥下河,東接大江,防捍敵人,檢制盜賊。六年,淮東提舉徐子寅言:「淮東鹽課,全仰河流通快。近運河淺澀,自揚州灣頭港口至鎮西山光寺前橋垛頭,計四百八十五丈,乞發五千餘卒開浚。」從之。七年二月,詔令淮南漕臣,自洪澤至龜山淺澀之處,如法開撩。



 淳熙三年四月,詔築泰州月堰,
 以遏潮水。從守臣張子正請也。八年,提舉淮南東路常平茶鹽趙伯昌言:「通州、楚州沿海,舊有捍海堰,東距大海,北接鹽城,袤一百四十二里。始自唐黜陟使李承實所建,遮護民田,屏蔽鹽灶,其功甚大。歷時既久,頹圮不存。至本朝天聖改元,範仲淹為泰州西溪鹽官日,風潮泛溢,渰沒田產,毀壞亭灶,有請於朝,調四萬餘夫修築,三旬畢工。遂使海瀕沮洳瀉鹵之地,化為良田,民得奠居,至今賴之。自後浸失修治,才遇風潮怒盛,即有沖決
 之患。自宣和、紹興以來,屢被其害。阡陌洗蕩,廬舍漂流,人畜喪亡,不可勝數。每一修築,必請朝廷大興工役,然後可辦。望令淮東常平茶鹽司:今後捍海堰如有塌損,隨時修葺,務要堅固,可以經久。」從之。



 九年,淮南漕臣錢沖之言:「真州之東二十里,有陳公塘,乃漢陳登浚源為塘,用救旱饑。大中祥符間,江、淮制置發運置司真州,歲藉此塘灌注長河,流通漕運。其塘周回百里,東、西、北三面,倚山為岸,其南帶東,則系前人築壘成堤,以受啟閉。
 廢壞歲久,見有古來基趾,可以修築,為旱乾溉田之備。凡諸場鹽綱、糧食漕運、使命往還,舟艦皆仰之以通濟,其利甚博。本司自發卒貼築周回塘岸,建置斗門、石□達各一所,乞於揚子縣尉階銜內帶『兼主管陳公塘』六字,或有損壞,隨時補築,庶幾久遠,責有所歸。」



 十二年,和州守臣請於千秋澗置斗門,以防麻澧湖水洩入大江,遇歲旱灌溉田疇,實為民利。十四年,揚州守臣熊飛言:「揚州運河,惟藉瓜洲、真州兩閘瀦積。今河水走洩,緣瓜洲
 上、中二閘久不修治,獨潮閘一坐,轉運、提鹽及本州共行修整,然迫近江潮,水勢沖激,易致損壞;真州二閘,亦復損漏。令有司葺理上、下二閘,以防走洩。」從之。



 紹熙五年,淮東提舉陳損之言:「高郵、楚州之間,陂湖渺漫,茭葑彌滿,宜創立堤堰,以為瀦洩,庶幾水不至於泛溢,旱不至於乾涸。乞興築自揚州江都縣至楚州淮陰縣三百六十里,又自高郵、興化至鹽城縣二百四十里,其堤岸傍開一新河,以通舟船。仍存舊堤以捍風浪,載柳十餘
 萬株,數年後堤岸亦牢,其木亦可備修補之用。兼揚州柴墟鎮,舊有堤閘,乃泰州洩水之處,其閘壞久,亦於此創立斗門。西引盱眙、天長以來眾湖之水,起自揚州江都,經由高郵及楚州寶應、山陽,北至淮陰,西達於淮;又自高郵入興化,東至鹽城而極於海;又泰州海陵南至揚州泰興而徹於江:共為石□達十三,斗門七。乞以紹熙堰為名,鑱諸堅石。」淮田多沮洳,因損之築堤捍之,得良田數百萬頃。奏聞,除直秘閣、淮東轉運判官。



 浙江通大海,日受兩潮。梁開平中,錢武肅王始築捍海塘,在候潮門外。潮水晝夜沖激,版築不就,因命強弩數百以射潮頭,又致禱胥山祠。既而潮避錢塘,東擊西陵,遂造竹器,積巨石,植以大木。堤岸既固,民居乃奠。



 逮宋大中祥符五年,杭州言浙江擊西北岸益壞,稍逼州城,居民危之。即遣使者同知杭州戚綸、轉運使陳堯佐畫防捍之策。綸等因率兵力,籍梢楗以護其沖。七年,綸等既罷去,發運使李溥、內供奉官盧守懃經度,以為非便。
 請復用錢氏舊法,實石於竹籠,倚疊為岸,固以樁木,環亙可七里。斬材役工,凡數百萬,逾年乃成;而鉤末壁立,以捍潮勢,雖湍湧數丈,不能為害。



 至景祐中,以浙江石塘積久不治,人患墊溺,工部郎中張夏出使,因置捍江兵士五指揮,專採石修塘,隨損隨治,眾賴以安。邦人為之立祠,朝廷嘉其功,封寧江侯。



 及高宗紹興末,以錢塘石岸毀裂,潮水漂漲,民不安居,令轉運司同臨安府修築。孝宗乾道九年,錢塘廟子灣一帶石岸,復毀於怒潮。
 詔令臨安府築填江岸,增砌石塘,淳熙改元,復令有司:「自今江岸沖損,以乾道修治為法。」



 理宗寶祐二年十二月,監察御史兼崇政殿說書陳大方言:「江潮侵嚙堤岸,乞戒飭殿、步兩司帥臣,同天府守臣措置修築,留心任責,或有潰決,咎有攸歸。」



 三年十一月,監察御史兼崇政殿說書李衢言:「國家駐蹕錢塘,今逾十紀。惟是浙江東接海門,胥濤澎湃,稍越故道,則沖嚙堤岸,蕩析民居,前後不知其幾。慶歷中,造捍江五指揮,兵士每指揮以四
 百人為額。今所管才三百人,乞下臨安府拘收,不許占破。及從本府收買樁石,沿江置場樁管,不得移易他用。仍選武臣一人習於修江者,隨其資格,或以副將,或以路分鈐轄系銜,專一鈐束修江軍兵,值有摧損,隨即修補;或不勝任,以致江潮沖損堤岸,即與責罰。」



 臨安西湖周回三十里,源出於武林泉。錢氏有國,始置撩湖兵士千人,專一開浚。至宋以來,稍廢不治,水涸草生,漸成葑田。



 元祐中,知杭州蘇軾奏謂:「杭之為州,本江
 海故地,水泉咸苦,居民零落。自唐李泌始引湖水作六井,然後民足於水,井邑日富,百萬生聚,待此而食。今湖狹水淺,六井盡壞,若二十年後,盡為葑田,則舉城之人,復飲咸水,其勢必耗散。又放水溉田,瀕湖千頃,可無兇歲。今雖不及千頃,而下湖數十里間,茭菱穀米,所獲不貲。又西湖深闊,則運河可以取足於湖水,若湖水不足,則必取足於江潮。潮之所過,泥沙渾濁,一石五斗,不出三載,輒調兵夫十餘萬開浚。又天下酒官之盛,如杭歲
 課二十餘萬緡,而水泉之用,仰給於湖。若湖漸淺狹,少不應溝,則當勞人遠取山泉,歲不下二十萬工。」因請降度牒減價出賣,募民開治。禁自今不得請射、侵占、種植及臠葑為界。以新舊菱蕩課利錢送錢塘縣收掌,謂之開湖司公使庫,以備逐年雇人開葑撩淺。縣尉以「管勾開湖司公事」系銜。軾既開湖,因積葑草為堤,相去數里,橫跨南、北兩山,夾道植柳,林希榜曰「蘇公堤」,行人便之,因為軾立祠堤上。



 紹興九年,以張澄奏請,命臨安府招
 置廂軍兵士二百人,委錢塘縣尉兼領其事,專一浚湖;若包占種田,沃以糞土,重寘於法。十九年,守臣湯鵬舉奏請重開。乾道五年,守臣周淙言:「西湖水面唯務深闊,不容填溢,並引入城內諸井,一城汲用,尤在涓潔。舊招軍士止有三十餘人,今宜增置撩湖軍兵,以百人為額,專一開撩。或有種植茭菱,因而包占,增疊堤岸,坐以違制。」



 九年,臨安守臣言:「西湖冒佃侵多,葑茭蔓延,西南一帶,已成平陸。而瀕湖之民,每以葑草圍裹,種植荷
 花,駸駸不已。恐數十年後,西湖遂淤,將如越之鑒湖,不可復矣。乞一切芟除,務令凈盡,禁約居民,不得再有圍裹。」從之。



 臨安運河在城中者,日納潮水,沙泥渾濁,一汛一淤,比屋之民,委棄草壤,因循填塞。元祐中,守臣蘇軾奏謂:「熙寧中,通判杭州時,父老皆云苦運河淤塞,率三五年常一開浚。不獨勞役兵民,而運河自州前至北郭,穿闤闠中蓋十四五里,每將興工,市肆洶動,公私騷然。自胥吏、
 壕砦兵級等,皆能恐喝人戶,或云當於某處置土、某處過泥水,則居者皆有失業之憂。既得重賂,又轉而之他。及工役既畢,則房廊、邸舍,作踐狼籍,園圃隙地,例成丘阜,積雨蕩濯,復入河中,居民患厭,未易悉數。若三五年失開,則公私壅滯,以尺寸水行數百斛舟,人牛力盡,跬步千里,雖監司使命,有數日不能出郭者。詢其所以頻開屢塞之由,皆云龍山浙江兩閘,泥沙渾濁,積日稍久,便及四五尺,其勢當然,不足怪也。尋鏟刷捍江兵士及
 諸色廂軍,得一千人,七月之間,開浚茅山、鹽橋二河,各十餘里,皆有水八尺。自是公私舟船通利,三十年以來,開河未有若此深快者。然潮水日至,淤塞猶昔,則三五年間,前功復棄。今於鈐轄司前置一閘,每遇潮上,則暫閉此閘,候潮平水清復開,則河過闤闠中者,永無潮水淤塞、開淘騷擾之患。」詔從其請,民甚便之。



 紹興三年十一月,宰臣奏開修運河淺澀,帝曰:「可發旁郡廂軍、壯城、捍江之兵,至於廩給之費,則不當吝。」宰臣朱勝非等曰:「
 開河非今急務,而饋餉艱難,為害甚大。時方盛寒,役者良苦,臨流居人,侵塞河道者,悉當遷避;至於畚閘所經,沙泥所積,當預空其處,則居人及富家以僦屋取貲者皆非便,恐議者以為言。」帝曰:「禹卑宮室而盡力於溝洫,浮言何恤焉!」八年,又命守臣張澄發廂軍、壯城兵千人,開浚運河堙塞,以通往來舟楫。



 隆興二年,守臣吳芾言:「城裏運河,先已措置北梅家橋、仁和倉、斜橋三所作壩,取西湖六處水口通流灌入。府河積水,至望仙橋以南
 至都亭驛一帶,河道地勢,自昔高峻。今欲先於望仙橋城外保安閘兩頭作壩,卻於竹車門河南開掘水道,車戽運水,引入保安門通流入城,遂自望仙橋以南開至都亭驛橋,可以通徹積水,以備緩急。計用工四萬。」從之。



 乾道三年六月,知荊南府王炎言:「臨安居民繁伙,河港堙塞,雖屢開導,緣裁減工費,不能迄功。臣嘗措置開河錢十萬緡,乞候農暇,特詔有司,用此專充開河支費,庶幾河渠復通,公私為利。」上俞其請。四年,守臣周淙出公
 帑錢招集游民,開浚城內外河,疏通淤塞,人以治辦稱之。



 淳熙二年,兩浙漕臣趙磻老言:「臨安府長安閘至許村巡檢司一帶,漕河淺澀,請出錢米,發兩岸人戶出力開浚。」又言:「欲於通江橋置板閘,遇城中河水淺涸,啟板納潮,繼即下板,固護水勢,不得通舟;若河水不乏,即收閘板,聽舟楫往還為便。」



 七年,守臣吳淵言:「萬松嶺兩旁古渠,多被權勢及有司公吏之家造屋侵占,及內砦前石橋、都亭驛橋南北河道,居民多拋糞土瓦礫,以致填
 塞,流水不通。今欲分委兩通判監督,地分廂巡,逐時點檢,勿令侵占並拋揚糞土。秩滿,若不淤塞,各減一年磨勘;違,展一年,以示勸懲。」



 十四年七月,不雨,臣僚言:「竊見奉口至北新橋三十六里,斷港絕潢,莫此為甚。今宜開浚,使通客船,以平谷直。」從之。



 鹽官海水:嘉定十二年,臣僚言:「鹽官去海三十餘里,舊無海患,縣以鹽灶頗盛,課利易登。去歲海水泛漲,湍激橫沖,沙岸每一潰裂,嘗數十丈。日復一日,浸入鹵地,蘆
 州港瀆,蕩為一壑。今聞潮勢深入,逼近居民。萬一春水驟漲,怒濤奔湧,海風佐之,則呼吸蕩出,百里之民,寧不俱葬魚腹乎?況京畿赤縣,密邇都城。內有二十五里塘,直通長安閘,上徹臨平,下接崇德,漕運往來,客船絡繹,兩岸田畝,無非沃壤。若海水徑入於塘,不惟民田有咸水渰沒之患,而里河堤岸,亦將有潰裂之憂。乞下浙西諸司,條具築捺之策,務使捍堤堅壯,土脈充實,不為怒潮所沖。」從之。



 十五年,都省言:鹽官縣海塘沖決,命浙西
 提舉劉垕專任其事。既而垕言:



 縣東接海鹽,西距仁和,北抵崇德、德清,境連平江、嘉興、湖州;南瀕大海元與縣治相去四十餘里。數年以來,水失故道,早晚兩潮,奔沖向北,遂致縣南四十餘里盡淪為海。近縣之南,元有捍海古塘亙二十里。今東西兩段,並已淪毀,侵入縣兩旁又各三四里,止存中間古塘十餘里。萬一水勢沖激不已,不惟鹽官一縣不可復存,而向北地勢卑下,所慮咸流入蘇、秀、湖三州等處,則田畝不可種植,大為利害。



 詳
 今日之患,大概有二:一曰陸地淪毀,二曰咸潮泛溢。陸地淪毀者,固無力可施;咸潮泛溢者,乃因捍海古塘沖損,遇大潮必盤越流注北向,宜築土塘以捍咸潮。所築塘基址,南北各有兩處:在縣東近南則為六十里咸塘,近北則為袁花塘;在縣西近南亦曰咸塘,近北則為淡塘。



 亦嘗驗兩處土色虛實,則袁花塘、淡塘差勝咸塘,且各近裏,未至與海潮為敵。勢當東就袁花塘、西就淡塘修築,則可以御縣東咸潮盤溢之患。其縣西一帶淡
 塘,連縣治左右,共五十餘里,合先修築。兼縣南去海一里餘,幸而古塘尚存,縣治民居,盡在其中,未可棄之度外。今將見管樁石,就古塘稍加工築疊一里許,為防護縣治之計。其縣東民戶,日築六十里咸塘。萬一又為海潮沖損,當計用樁木修築袁花塘以捍之。



 上以為然。



 明州水:紹興五年,明州守臣李光奏:「明、越陂湖,專溉農田。自慶歷中,始有盜湖為田者,三司使切責漕臣,嚴立法禁。宣和以來,王仲薿守越,樓異守明,創為應奉,始廢
 湖為田,自是歲有水旱之患。乞行廢罷,盡復為湖。如江東、西之圩田,蘇、秀之圍田,皆當講究興復。」詔逐路轉運司相度聞奏。



 乾道五年,守臣張津言:「東錢湖容受七十二溪,方圓廣闊八百頃,傍山為固,疊石為塘八十里。自唐天寶三年,縣令陸南金開廣之。國朝天禧元年,郡守李夷庚重修之。中有四閘七堰,凡遇旱涸,開閘放水,溉田五十萬畝。比因豪民於湖塘淺岸漸次包占,種植菱荷,障塞湖水。紹興十八年,雖曾檢舉約束,盡罷請佃。歲
 久菱根蔓延,滲塞水脈,致妨蓄水;兼塘岸間有低塌處,若不淘浚修築,不惟浸失水利,兼恐塘埂相繼摧毀。乞候農隙趁時開鑿,因得土修治埂岸,實為兩便。」從之。



 鄞縣水:嘉定十四年,慶元府言:「鄞縣水自四明諸山溪澗會至他山,置堰小涇,下江入河。所入上河之水,專溉民田,其利甚博。比因淤塞,堰上山觜少有溪水流入上河。自春徂夏不雨,令官吏發卒開淘沙觜及浚港汊,又於堰上壘疊沙石,逼使溪流兗入上河。其它山水入府
 城南門一帶,有楔閘三所:曰烏金,曰積瀆,曰行春。烏金楔又名上水楔,昔因倒損,遂捺為壩,以致淤沙在河,或遇溪流聚湧,時復沖倒所捺壩,走洩水源。行春橋又名南石楔,楔面石板之下,歲久損壞空虛,每受潮水,演溢奔突,山於石縫,以致咸潮盡入上河。其縣東管有道士堰,至白鶴橋一帶,河港堙塞;又有朱賴堰,與行春等楔相連,堰下江流通徹大海。今春闕雨,上河乾淺,堰身塌損,以致咸潮透入上河,使農民不敢車注溉田。乞修砌
 上水、烏金諸處壩堰,仍選清強能幹職官,專一提督。」



 潤州水:紹興七年,兩浙轉運使向子諲言:「鎮江府呂城、夾岡,形勢高仰,因春夏不雨,官漕艱勤。尋遣官屬李澗詢究練湖本末,始知此湖在唐永泰間已廢而復興。今堤岸弛禁,致有侵佃冒決,故湖水不能瀦蓄,舟楫不通,公私告病。若夏秋霖潦,則丹陽、金壇、延陵一帶良田,亦被渰沒。臣已令丹陽知縣朱穆等增置二斗門、一石□達,及修補堤防,盡復舊跡,庶為永久之利。」乾道七年,以臣
 僚言:「丹陽練湖幅員四十里,納長山諸水,漕渠資之,故古語云:『湖水寸,渠水尺。」在唐之禁甚嚴,盜決者罪比殺人。本朝浸緩其禁以惠民,然修築嚴甚。春夏多雨之際,瀦蓄盈滿,雖秋無雨,漕渠或淺,但洩湖水一寸,則為河一尺矣。兵變以後,多廢不治,堤岸圮闕,不能貯水;強家因而專利,耕以為田,遂致淤澱。歲月既久,其害滋廣。望責長吏浚治堙塞,立為盜決侵耕之法,著於令。庶幾練湖漸復其舊,民田獲灌溉之利,漕渠無淺涸之患。」詔兩
 浙漕臣沉度專一措置修築。



 慶元五年,兩浙轉運、浙西提舉言:「以鎮江府守臣重修呂城兩閘畢,再造一新閘以固堤防,庶為便利。」從之。



 浙西運河,自臨安府北郭務至鎮江江口閘,六百四十一里。淳熙七年,帝因輔臣奏金使往來事,曰:「運河有淺狹處,可令守臣以漸開浚,庶不擾民。」至十一年冬,臣僚言:「運河之浚,自北關至秀州杉青,各有堰閘,自可瀦水。惟沿河上塘有小堰數處,積久低陷,無以防遏水勢,當
 以時加修治。兼沿河下岸涇港極多,其水入長水塘、海鹽塘、華亭塘,由六里堰下,私港散漫,悉入江湖,以私港深、運河淺也。若修固運河下岸一帶涇港,自無走洩。又自秀州杉青至平江府盤門,在太湖之際,與湖水相連;而平江閶門至常州,有楓橋、許墅、烏角溪、新安溪、將軍堰,亦各通太湖。如遇西風,湖水由港而入,皆不必浚。惟無錫五瀉閘損壞累年,常是開堰,徹底放舟;更江陰軍河港勢低,水易走洩。若從舊修築,不獨瀦水可以通舟,
 而無錫、晉陵間所有陽湖,亦當積水,而四傍田畝,皆無旱□之患。獨自常州至丹陽縣,地勢高仰,雖有奔牛、呂城二閘,別無湖港瀦水;自丹陽至鎮江,地形尤高,雖有練湖,緣湖水日淺,不能濟遠,雨晴未幾,便覺乾涸。運河淺狹,莫此為甚,所當先浚。」上以為然。



 至嘉定間,臣僚又言:「國家駐蹕錢塘,綱運糧餉,仰給諸道,所系不輕。水運之程,自大江而下至鎮江則入閘,經行運河,如履平地,川、廣巨艦,直抵都城,蓋甚便也。比年以來,鎮江閘口河
 道淤塞,不復通舟,乞令漕臣同淮東總領及本府守臣,公共措置開撩。」



 越州水:鑒湖之廣,周回三百五十八里,環山三十六源。自漢永和五年,會稽太守馬臻始築塘,溉田九千餘頃,至宋初八百年間,民受其利。歲月浸遠,浚治不時,日久堙廢。瀕湖之民,侵耕為田,熙寧中,盜為田九百餘頃。嘗遣廬州觀察推官江衍經度其宜,凡為湖田者兩存之,立碑石為界,內者為田,外者為湖。政和末,為郡守者務
 為進奉之計,遂廢湖為田,賦輸京師。自時奸民私占,為田益眾,湖之存者亡幾矣。紹興二十九年十月,帝諭樞密院事王綸曰:「往年宰執嘗欲盡乾鑒湖,云可得十萬斛米。朕謂若遇歲旱,無湖水引灌,則所損未必不過之。凡事須遠慮可也。」



 隆興元年,紹興府守臣吳芾言:「鑒湖自江衍所立碑石之外,今為民田者,又一百六十五頃,湖盡堙廢。今欲發四百九十萬工,於農隙接續開鑿。又移壯城百人,以備撩灑浚治,差強乾使臣一人,以『巡轄
 鑒湖堤岸』為名。」



 二年,芾又言:「修鑒湖,全藉斗門、堰閘蓄水,都泗堰閘尤為要害。凡遇約運及監司使命舟船經過,堰兵避免車拽,必欲開閘通放,以致啟閉無時,失洩湖水。且都泗堰因高麗使往來,宣和間方置閘,今乞廢罷。」其後芾為刑部侍郎,復奏:「自開鑒湖,溉廢田二百七十頃,復湖之舊。又修治斗門、堰閘十三所。夏秋以來,時雨雖多,亦無泛溢之患,民田九千餘頃,悉獲倍收,其為利較然可見。乞將江衍原立禁牌,別定界至,則堤岸自
 然牢固,永無盜決之虞。」



 紹興初,高宗次越,以上虞縣梁湖堰東運河淺澀,令發六千五百餘工,委本縣令、佐監督浚治。既而都省言,餘姚縣境內運河淺澀,壩閘隳壞,阻滯綱運,遂命漕臣發一萬七千餘卒,自都泗堰至曹娥塔橋,開撩河身、夾塘,詔漕司給錢米。



 蕭山縣西興鎮通江兩閘,近為江沙壅塞,舟楫不通。乾道三年,守臣言:「募人自西興至大江,疏沙河二十里,並浚閘里運河十三里,通使綱運,民旅皆利。復恐潮水不定,復有填淤,且
 通江六堰,綱運至多,宜差注指使一人,專以『開撩西興沙河』系銜,及發捍江兵士五十名,專充開撩沙浦,不得雜役,仍從本府起立營屋居之。」



 常州水:隆興二年,常州守臣劉唐稽言:「申、利二港,上自運河發流,經營回復,至下流析為二道,一自利港,一自申港,以達於江。緣江口每日潮汐帶沙填塞,上流游泥淤積,流洩不通;而申港又以江陰軍釘立標楬,拘攔稅船,每潮來,則沙泥為木標所壅,淤塞益甚。今若相度開
 此二河,但下流申、利二港,並隸江陰軍,若議定深闊丈尺,各於本界開淘,庶協力皆辦。又孟瀆一港在奔牛鎮西,唐孟簡所開,並宜興縣界沿湖舊百瀆,皆通宜興之水,藉以疏洩。近歲阻於吳江石塘,流行不快,而沿湖河港所謂百瀆,存者無幾。今若開通,委為公私之便。」至乾道二年,以漕臣姜詵等請,造蔡涇閘及開申港上流橫石,次浚利港以洩水勢。



 六年三月,又命兩浙運副劉敏士、浙西提舉芮輝於新涇塘置閘堰,以捍海潮;楊家港
 東開河置閘,通行鹽船。仍差閘官一人,兵級十五人,以時啟閉挑撩。五月,又以兩折轉運司並常州守臣言,填築五瀉上、下兩閘,及修築閘裏堤岸。仍於郭瀆港口舜郎廟側水聚會處,築捺硬壩,以防走洩運水。委無錫知縣主掌鑰匣,遇水深六尺,方許開閘,通放客舟。



 淳熙五年,以漕臣陳峴言,於十月募工開浚無錫縣以西橫林、小井及奔牛、呂城一帶地高水淺之處,以通漕舟。



 九年,知常州章沖奏:



 常州東北曰深港、利港、黃田港、夏港、五
 斗港,其西曰灶子港、孟瀆、泰伯瀆、烈塘,江陰之東曰趙港、白沙港、石頭港、陳港、蔡港、私港、令節港,皆古人開導以為溉田無窮之利者也;今所在堙塞,不能灌溉。



 臣嘗講求其說,抑欲不勞民,不費財,而漕渠旱不干,水不溢,用力省而見功速,可以為悠久之利者,在州之西南曰白鶴溪,自金壇縣洮湖而下,今淺狹特七十餘里,若用工浚治,則漕渠一帶,無乾涸之患;其南曰西蠡河,自宜興太湖而下,止開浚二十餘里,若更令深遠,則太湖水
 來,漕渠一百七十餘里,可免浚治之擾。至若望亭堰閘,置於唐之至德,而徹於本朝之嘉祐,至元祐七年復置,未幾又毀之。臣謂設此堰閘,有三利焉:陽羨諸瀆之水奔趨而下,有以節之,則當潦歲,平江三邑必無下流淫溢之患,一也。自常州至望亭一百三十五里,運河一有所節,則沿河之田,旱歲資以灌溉,二也。每歲冬春之交,重綱及使命往來,多苦淺涸;今啟閉以時,足通舟楫,後免車畝灌注之勞,三也。



 詔令相度開浚。



 嘉泰元年,守臣
 李玨言:



 州境北邊揚子大江,南並太湖,東連震澤,西據滆湖,而漕渠,界乎其間。漕渠兩傍曰白鶴溪、西蠡河、南戚氏、北戚氏、直湖州港,通於二湖;曰利浦、孟瀆、烈塘、橫河、五瀉諸港,通於大江,而中間又各自為支溝斷汊,曲繞參錯,不以數計。水利之源,多於他郡,而常苦易旱之患,何哉?



 臣嘗詢訪其故:漕渠東起望亭,西上呂城,一百八十餘里,形勢西高東下。加以歲久淺淤,自河岸至底,其深不滿四五尺。常年春雨連綿、江湖泛漲之時,河流
 忽盈驟減;連歲雨澤愆闕,江湖退縮,渠形尤亢;間雖得雨,水無所受,旋即走洩,南入於湖,北歸大江,東徑注于吳江;晴未旬日,又復乾涸,此其易旱一也。至若兩傍諸港,如白鶴溪、西蠡河、直湖、烈塘、五瀉堰,日為沙土淤漲,遇潮高水泛之時,尚可通行舟楫;若值小汐久晴,則俱不能通。應自余支溝別港,皆已堙塞,故雖有江湖之浸,不見其利,此其易旱二也。況漕渠一帶,綱運於是經由,使客於此往返。每遇水澀,綱運便阻;一入冬月,津送使
 客,作壩車水,科役百姓,不堪其擾;豈特溉田缺事而已。



 望委轉運、提舉常平官同本州相視漕渠,並徹江湖之處,如法浚治,盡還昔人遺跡,及於望亭修建上、下二閘,固護水源。



 從之。



 升州水:乾道五年,建康守臣張孝祥言:「秦淮之水流入府城,別為兩派:正河自鎮淮新橋直注大江;其為青溪,自天津橋出柵砦門,亦入於江。緣柵砦門地,近為有力者所得,遂築斷青溪水口,創為花圃。每水流暴至,則泛
 溢浸蕩,城內居民,尤被其害。若訪古而求,使青溪直達大江,則建康永無水患矣。」既而汪澈奏於西園依異時河道開浚,使水通柵門入。從之。



 先是,孝祥又言:「秦淮水三源,一自華山由句容,一自廬山由溧水,一自溧水由赤山湖,至府城東南,合而為一,縈回綿亙三百餘里,溪、港、溝、澮之水盡歸焉。流上水門,由府城入大江。舊上、下水門展闊,自兵變後,砌疊稍狹,雖便於一時防守,實遏水源,流通不快。兼兩岸居民填築河岸,添造屋宇。若禁
 民不許侵占,秦淮既復故道,則水不泛溢矣。又府東門號陳二渡,有順聖河,正分秦淮之水,每遇春夏天雨連綿,上源奔湧,則分一派之水,自南門外直入於江,故秦淮無泛濫之患。今一半淤塞為田,水流不通,若不惜數畝之田,疏導之以復古跡,則其利尤倍。」



 其後汪澈言:「水潦之害,大抵緣建康地勢稍低,秦淮既泛,又大江湍漲,其勢湓溢,非由水門窄狹、居民侵築所致。且上水門砌疊處正不可闊,闊則春水入城益多。自今指定上、下水
 門砌疊處不動,夾河居民之屋亦不毀除,止去兩岸積壞,使河流通快。況城中系行宮東南王方,不宜開鑿。」從之。



 嘉定五年,守臣黃度言:「府境北據大江,是為天險。上自採石,下達瓜步,千有餘里,共置六渡:一曰烈山渡,籍於常平司,歲有河渡錢額;五曰南浦渡、龍灣渡、東陽渡、大城堽渡、岡沙渡,籍於府司,亦有河渡錢額。六渡歲為錢萬餘緡。歷時最久,舟楫廢壞,官吏、篙工,初無廩給,民始病濟,而官漫不省。遂至奸豪冒法,別置私渡,左右旁
 午。由是官渡濟者絕少,乃聽吏卒苛取以充課。徒手者猶憚往來,而車簷牛馬幾不敢行,甚者扼之中流,以邀索錢物。竊以為南北津渡,務在利涉,不容簡忽而但求徵課。臣已為之繕治舟艦,選募篙梢,使遠處巡檢兼監渡官。於諸渡月解錢則例,量江面闊狹,計物貨重輕,斟酌裁減,率三之一或四之一;自人車牛馬,皆有定數,雕榜約束,不得過收邀阻。乞覓裒一歲之入,除烈山渡常平錢如額解送,其餘諸渡,以二分充修船之費,而以其
 餘給官吏、篙梢、水手食錢。令監渡官逐月照數支散,有餘則解送府司,然後盡絕私渡,不使奸民逾禁。」從之。



 秀州水:秀州境內有四湖:一曰柘湖,二曰澱山湖,三曰當湖,四曰陳湖。東南則柘湖,自金山浦、小官浦入於海。西南則澱山湖,自蘆歷浦入於海。西北則陳湖,自大姚港、朱裏浦入於吳松江。其南則當湖,自月河、南浦口、澉浦口亦達於海。支港相貫。



 乾道二年,守臣孫大雅奏請,於諸港浦分作閘或斗門,及張涇堰兩岸創築月河,置一
 閘,其兩柱金口基址,並以石為之,啟閉以時,民賴其利。



 十三年,兩浙轉運副使張叔獻言:「華亭東南枕海,西連太湖,北接松江,江北復控大海。地形東南最高,西北稍下。柘湖十有八港,正在其南,故古來築堰以御咸潮。元祐中,於新涇塘置閘,後因沙淤廢毀。今除十五處築堰及置石□達外,獨有新涇塘、招賢港、徐浦塘三處,見有咸潮奔沖,渰塞民田。今依新涇塘置閘一所,又於兩旁貼築咸塘,以防海潮透入民田。其相近徐浦塘,元系小派,
 自合築堰。又欲於招賢港更置一石□達。兼楊湖歲久,今稍淺澱,自當開浚。」上曰:「此閘須當為之。方今邊事寧息,惟當以民事為急。民事以農為重,朕觀漢文帝詔書,多為農而下。今置閘,其利久遠,不可憚一時之勞。」



 十五年,以兩浙路轉運判官吳坰奏請,命浙西常平司措置錢穀,勸諭人戶,於農隙並力開浚華亭等處沿海三十六浦堙塞,決洩水勢,為永久利。



 乾道七年,秀州守臣丘崇奏:「華亭縣東南大海,古有十八堰,捍禦咸潮。其十七久
 皆捺斷,不通里河;獨有新涇唐一所不曾築捺,海水往來,遂害一縣民田。緣新涇舊堰迫近大海,潮勢湍急,其港面闊,難以施工,設或築捺,決不經久。運港在涇塘向裏二十里,比之新涇,水勢稍緩。若就此築堰,決可永久,堰外凡管民田,皆無咸潮之害。其運港止可捺堰,不可置閘。不惟瀕海土性虛燥,難以建置;兼一日兩潮,通放鹽運,不減數十百艘,先後不齊,比至通放盡絕,勢必晝夜啟而不閉,則咸潮無緣斷絕。運港堰外別有港汊大
 小十六,亦合興修。」從之。



 八年,崇又言:「興築捍海塘堰,今已畢工,地理闊遠,全藉人力固護。乞令本縣知、佐兼帶『主管塘堰職事』系銜,秩滿,視有無損壞以為殿最。仍令巡尉據地分巡察。」詔特轉丘崇左承議郎,令所築華亭捍海塘堰,趁時栽種蘆葦,不許樵採。



 九年,又命華亭縣作監閘官,招收土軍五十人,巡邏堤堰,專一禁戢,將卑薄處時加修捺。令知縣、縣尉並帶『主管堰事』,則上下協心,不致廢壞。



 淳熙九年,又命守臣趙善悉發一萬工,修
 治海鹽縣常豐閘及八十一堰壩,務令高牢,以固護水勢,遇旱可以瀦積。十年,以浙西提舉司言,命秀州發卒浚治華亭鄉魚祈塘,使接松江太湖之水;遇旱,即開西閘堰放水入泖湖,為一縣之利。



 蘇州水:乾道初,平江守臣沉度、兩浙漕臣陳彌作言:「疏浚昆山、常熟縣界白茆等十浦,約用三百萬餘工。其所開港浦,並通徹大海。遇潮,則海內細沙,隨泛以入;潮退,則沙泥沉墜,漸致淤塞。今依舊招置闕額開江兵卒,次
 第開浚,不數月,諸浦可以漸次通徹。又用兵卒駕船,遇潮退,搖蕩隨之,常使沙泥隨潮退落,不致停積,實為久利。」從之。淳熙元年,詔平江府守臣與許浦駐扎戚世明同措置開浚許浦港,三旬訖工。



 黃巖縣水:淳熙十二年,浙東提舉勾昌泰言:「黃巖縣舊有官河,自縣前至溫嶺,凡九十里。其支流九百三十六處,皆以溉田。元有五閘,久廢不修。今欲建一閘,約費二萬餘緡,乞詔兩浙運司于窠名錢內支撥。」明年六月,昌
 泰復言:「黃巖縣東地名東浦,紹興中開鑿,置常豐閘。名為決水入江,其實縣道欲令舟船取徑通過,每船納錢,以充官費。一日兩潮,一潮一淤,才遇旱乾,更無灌溉之備。已將此閘築為平陸,乞戒自今永不得開鑿放入江湖,庶絕後患。」



 荊、襄諸水:紹興二十八年,監察御史都民望言:「荊南江陵縣東三十里,沿江北岸古堤一處,地名黃潭。建炎間,邑官開決,放入江水,設以為險阻以御盜。既而夏潦漲
 溢,荊南、復州千餘里,皆被其害。去年因民訴,始塞之。乞令知縣遇農隙隨力修補,勿致損壞。」從之。



 淳熙八年,襄陽府守臣郭杲言:「本府有木渠,在中廬縣界,擁漹水東流四十五里,入宜城縣。後漢南郡太守王寵,嘗鑿之以洩蠻水,謂之木里溝,可溉田六千餘頃。歲久堙塞,乞行修治。」既而杲又修護城堤以捍江流,繼築救生堤為二閘,一通於江,一達於濠。當水涸時,導之入濠;水漲時,入之於江。自是水雖至堤,無湍悍泛濫之患焉。十年五月,
 詔疏木渠,以渠旁地為屯田。尋詔民間侵耕者就給之,毋復取。



 慶元二年,襄陽守臣程九萬言:「募工修作鄧城永豐堰,可防金兵沖突之患,且為農田灌溉之利。」三年,臣僚言:「江陵府去城十餘里,有沙市鎮,據水陸之沖,熙寧中,鄭獬作守,始築長堤捍水。緣地本沙渚,當蜀江下流,每遇漲潦奔沖,沙水相蕩,摧圮動輒數十丈,見存民屋,岌岌危懼。乞下江陵府同駐扎副都統制司發卒修築,庶幾遠民安堵,免被墊溺。」從之。



 廣西水:靈渠源即離水,在桂州興安縣之北,經縣郭而南。其初乃秦史祿所鑿,以下兵於南越者。至漢,歸義侯嚴出零陵離水,即此渠也;馬伏波南征之師,饟道亦出於此。唐寶歷初,觀察使李渤立斗門以通漕舟。宋初,計使邊翊始修之。嘉祐四年,提刑李師中領河渠事重闢,發近縣夫千四百人,作三十四日,乃成。



 紹興二十九年,臣僚言:「廣西舊有靈渠,抵接全州大江,其渠近百餘里,自靜江府經靈川、興安兩縣。昔年並令兩知縣系銜『兼
 管靈渠』,遇堙塞以時疏導,秩滿無闕,例減舉員。兵興以來,縣道茍且,不加之意;吏部差注,亦不復系銜,渠日淺澀,不勝重載。乞令廣西轉運司措置修復,俾通漕運,仍俾兩邑今系銜兼管,務要修治。」從之。



\end{pinyinscope}