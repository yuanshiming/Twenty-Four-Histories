\article{列傳第十五}

\begin{pinyinscope}

 趙普弟安易



 趙普字則平,幽州薊人。後唐幽帥趙德鈞連年用兵,民力疲弊。普父回舉族徙常山,又徙河南洛陽。普沉厚寡言,鎮陽豪族魏氏以女妻之。



 周顯德初,永興軍節度劉
 詞闢為從事,詞卒,遺表薦普於朝。世宗用兵淮上,太祖撥滁州,宰相範質奏普為軍事判官。宣祖臥疾滁州,普朝夕奉藥餌,宣祖由是待以宗分。太祖嘗與語,奇之。時獲盜百餘,當棄市,普疑有無辜者,啟太祖訊鞫之,獲全活者眾。淮南平,調補渭州軍事判官。太祖領同州節度,闢為推官;移鎮宋州,表為掌書記。



 太祖北征至陳橋,被酒臥帳中,眾軍推戴,普與太宗排闥入告。太祖欠伸徐起,而眾軍擐甲露刃,喧擁麾下。及受禪,以佐命功,授右
 諫議大夫,充樞密直學士。



 車駕征李筠,命普與呂餘慶留京師,普願扈從,太祖笑曰:「若勝胃介乎?」從平上黨,遷兵部侍郎、樞密副使,賜第一區。建隆三年,拜樞密使、檢校太保。



 乾德二年,範質等三相同日罷,以普為門下侍郎、平章事、集賢殿大學士。中書無宰相署敕,普以為言,上曰:「卿但進敕,朕為卿署之可乎?」普曰:「此有司職爾,非帝王事也。」令翰林學士講求故實,竇儀曰:「今皇弟尹開封,同平章事,即宰相任也。」令署以賜普。既拜相,上視如
 左右手,事無大小,悉咨決焉。是日,普兼監修國史。命薛居正、呂餘慶參知政事以副之,不宣制,班在宰相後,不知印,不預奏事,水押班,但奉行制書而已。先是,宰相兼敕,皆用內制,普相止用敕,非舊典也。



 太祖數微行過功臣家,普每退朝,不敢便衣冠。一日,大雪向夜,普意帝不出。久之,聞叩門聲,普亟出,帝立風雪中,普惶懼迎拜。帝曰:「已約晉王矣。」已而太宗至,設重裀地坐堂中,熾炭燒肉。普妻行酒,帝以嫂呼之。因與普計下太原。普曰「太原
 當西北二面,太原既下,則我獨當之,不如姑俟削平諸國,則彈丸黑子之地,將安逃乎?」帝笑曰:「吾意正如此,特庫卿爾。」



 五年春,加右僕射、昭文館大學士。俄丁內艱,詔起復視事。遂勸帝遣使分詣諸道,征丁壯籍名送京師,以備守衛;諸州置通判,使主錢穀。由是兵甲精銳,府為充實。



 開寶二年冬,普嘗病,車駕幸中書。三年春,又幸其第撫問之。賜賚加等。六年,帝又幸其第。時錢王俶遣使致書於普,及海物十瓶,置於廡下。會車駕至,倉卒不及
 屏,帝顧問何物,普以實對。上曰:「海物必佳。」即命啟之。皆瓜子金也。普惶恐頓首謝曰:「臣未發書,實不知。」帝嘆曰:「受之無妨,彼謂國家事皆由汝書生爾!」



 普為政頗專,廷臣多忌之。時官禁私販秦、隴大木,普嘗遣親吏詣市屋材,聯巨筏至京師治第,吏因之竊貨大木,早稱普市貨鬻都下。權三司使趙玭廉得之以聞。太祖大怒,促令追班,將下制逐普,賴王溥奏解之。



 故事,宰相、樞密使每候對長春殿,同止盧中;上聞普子承宗娶樞密使李崇矩
 女,即令分異之。普又以隙地私易尚食蔬圃以廣其居,又營邸店規利。盧多遜為翰林學士,因召對屢攻其短。會雷有粼擊登聞鼓,訟堂後官胡贊、李可度受賕骫法及劉偉偽作攝牒得官,王洞嘗納賂可度,趙孚授西川官稱疾不上,皆普庇之。太祖怒,下御史府按問,悉抵罪,以有粼為秘書省正字。普恩益替,始詔參知政事與普更知印、押班、奏事,以分其權。未幾,出為河陽三城節度、檢校太傅、同平章事。



 太平興國初入朝,改太子少保,遷
 太子太保。頗為盧多遜所毀,奉朝請數年,鬱鬱不得志。會柴禹錫、趙鎔等告秦王廷美驕恣,將有陰謀竊發。帝召問,普言願備樞軸以察奸變,退又上書,自陳預聞太祖、昭憲皇太后顧托之事,辭甚切至。太宗感悟,召見慰諭。俄拜司徒兼侍中,封梁國公。先是,秦王廷美班在宰相上,至是,以普勛舊,再登元輔,表乞居其下,從之。及涪陵事敗,多遜南遷,皆普之力也。



 八年,出為武勝軍節度、檢校太尉兼侍中。帝作詩以餞之,普奉而泣曰:「陛下賜
 臣詩,當刻石,與臣朽骨同葬泉下。」帝為之動容。翌日,謂宰相曰:「普有功國家,朕昔與游,今齒發衰矣,不容煩以樞務,擇善地處之,因詩什以導意。普感激泣下,朕亦為之墮淚。」宋琪對曰:「昨日普至中書,執御詩涕泣,謂臣曰:『此生餘年,無階上答,庶希來世得效太馬力。』臣昨聞普言,今復聞宣諭,君臣始終之分,可謂兩全。」



 雍熙三年春,大軍出討幽薊,久未班師,普手疏諫曰:



 伏睹今春出師,將以收復關外,屢聞克捷,深快輿情。然晦朔屢更,薦臻
 炎夏,飛挽日繁,戰鬥未息,老師費財,誠無益也。



 伏念陛下自翦平太原,懷徠閩、浙,混一諸夏,大振英聲,十年之間,遂臻廣濟。遠人不服,自古聖王置之度外,何足介意。竊慮邪諂之輩,蒙蔽睿聰,致興無名之師,深蹈不測之地。臣載披典籍,頗識前言,竊見漢武時主父偃、徐樂、嚴安所上書及唐相姚無崇獻明皇十事,忠言至論,可舉而行。伏望萬機之暇,一賜觀覽,其失未遠,雖悔可追。



 臣竊念大發驍雄,動搖百萬之眾,所得者少,所喪者多。又
 聞戰者危事,難保其必勝;兵者兇器,深戒於不虞。所系甚大,不可不思。臣又聞上古聖人,心無固必,事不凝滯,理貴變通。前書有「兵久生變」之言,深為傑可慮,茍或更圖稽緩,轉失機宜。旬朔之間,時涉秋序,邊庭早涼,弓勁馬肥,我軍久困,切慮此際,或誤指蹤。臣方冒寵以守藩,曷敢興言而沮眾。蓋臣已日薄西山,餘沅無幾,酬恩報國,正在斯時。伏望速詔班師,無容玩敵。



 臣復有全策,願達聖聰。望陛下精調御膳,保養聖躬,挈彼疲氓,轉之富庶。
 將見邊烽不警,外戶不扃,率土歸仁,殊方異俗,相率響化,契丹獨將焉往?陛下計不出此,乃信邪謅之徒,謂契丹主少事多,所以用武,以中陛下之意。陛下樂禍求功,以為萬全,臣竊以為不可。伏願陛下審其虛實,究其妄謬,正奸臣誤國之罪,罷將士伐燕之師。非特多難興王,抑亦從諫則聖也。古之人尚聞尸諫,老臣未死,豈敢百諛為安身之計而不言哉?



 帝賜手詔曰:



 朕昨者興師選將,止令曹彬、米信等頓於雄、霸,裹糧坐甲以張軍聲。俟
 一兩月間山後平定,潘美、田重進等會兵以進,直抵幽州,然後控扼險固,恢復舊疆,此朕之志也。奈何將帥等不遵成算,各騁所見,領十萬甲士出塞遠門斗,速取其郡縣,更還師以援輜重,往復勞弊,為遼人所襲,此責在主將也。



 況朕踵百王之末,粗到承平,蓋念彼民陷於邊患,將救焚而拯溺,匪黷武以佳兵,卿當悉之也。疆場之事,已為之備,卿勿為憂。卿社稷元臣,忠言苦口,三復來奏,嘉愧實深。



 普表謝曰:



 昨以天兵久駐塞外,未克恢復,漸
 及炎蒸,事危勢迫,輒陳狂狷,甘俟憲章。陛下特鑒衷誠,親紆宸翰,密諭聖謀。臣竊審命師討罪,信為上策,將帥能遵成算,必可平定。惟其不副天心,由茲敗事。今既邊鄙有備,更復何虞。況陛下登極十年,坐隆大業,無一物之失所,見萬國之咸寧。所宜端拱穆清,嗇神和志,自可遠繼九皇,俯觀五帝。豈必窮邊極武,與契丹較勝負哉?臣素虧壯志,矧在衰齡,雖無功伐,願竭忠純。



 觀者咸嘉其忠。四年,移山南東道節度,自梁國公改封許國公。會
 詔下親耕籍田,普表求入覲,辭甚懇切。上惻然謂宰相曰:「普開國元臣,朕所尊禮,宜從其請。」既至,慰撫數四,普嗚咽流涕。陳王元僖上言曰:



 臣伏見唐太宗有魏玄成、房玄齡、杜如晦,明皇有姚崇、宗魏知古,皆任以輔弼,委之心膂,財成帝道,康濟九區,宗祀延洪,史策昭煥,良由登用得其人也。今陛下君臨萬方,焦勞庶政,宵衣旰食,以民為心。歷考前王,誠無所讓,而輔相之重,未偕曩賢。況為邦在於任人,任人在乎公正,公正之道莫先於
 賞罰,斯為政之大柄也。敬賞罰匪當,淑慝莫分,朝廷紀綱,漸致隳紊。必須公正之人典衡軸,直躬敢言,以辨得失,然後彞倫式序,庶務用康。



 伏見山南東道節度使趙普,開國元老,參謀締構,厚重有識,不妄希求恩顧以全祿位,不私徇人情以邀名望,此真聖朝之良臣也。竊聞之輩,朋黨比周,眾口嗷嗷,惡直醜正,恨不斥逐遐徼,以快其心。何者?蓋慮陛下之再用普也。然公讜之人,咸願陛下復委以政,啟沃君心,羽翼聖化。國有大事,
 使之謀之;朝有宏綱,使之舉之;四目未察,使之明之;四聰未至,使之達之」官人以材,則無竊祿,致君以道,則無茍容。賢愚洞分,玉石殊致,當使結朋黨以馳驁聲勢者氣索,縱巧佞以援引儕類者道消。沉冥廢滯得以進,名儒懿行得以顯,大政何患乎不舉,生民何患乎不康,匪窬期月之間,可臻清靜之治。臣知慮庸淺,發言魯直。伏望陛下旁採群議,俯察物情,茍用不失人,實邦國大幸。



 籍田禮畢,太宗欲相呂蒙正,以其新進,藉普舊德為之表
 率,冊拜太保兼侍中。帝謂之曰:「卿國之勛舊,朕所毗倚,古人恥其君不及堯、舜,卿其念哉。」普頓首謝。



 時樞密副使趙昌言與胡旦、陳象輿、董儼、梁顥厚善。會旦令翟馬周上封事,排毀時政,普深嫉之,奏流馬周,黜昌言等。鄭州團練使侯莫陳利用驕肆僭侈,大為不法,普廉得之,盡以條奏,利用坐流商州,普固請誅之。其嫉惡強直皆此類。



 李繼遷之擾邊,普建議以趙保忠復領夏臺故地,因令圖之。保忠反與繼遷同謀為邊患,時論歸咎於普,
 頗為同列所窺,不得專決。



 舊制,宰相以未時歸第,是歲大熱,特許普夏中至午時歸私第。明年,免朝謁,止日赴中書視事,有大政則召對。冬,被疾請告,車駕屢幸其第省之,賜予加等。普遂稱疾篤,三上表求致仕,上勉從之,以普為西京留守、河南尹,依前守太保兼中書令。普三表懇讓。賜手詔曰:「開國舊勛,惟卿一人,不同他等,無至固讓,俟首塗有日,當就第與卿為別。」普捧詔涕泣,因力疾請對,賜坐移晷,頗言及國家事,上嘉納之。普將發,車
 駕幸其第。



 淳化三年春,以老衰久病,令留守通判劉昌言奉表求致政,中使馳傳撫問,凡三上表乞骸骨。拜太師,封魏國公,給宰相奉料,令養疾,俟損日赴闕,仍遣其弟宗正少卿安易繼詔書賜之。又特遣使賜普詔曰:「卿頃屬微□彖,懇求致政,朕以居守之重,慮煩耆耋,維師之命,用表尊賢。佇聞有瘳,與朕相見。今賜羊酒如別錄,卿宜愛精神,近醫藥,強飲食,以副朕眷遇之意。」七月卒,年七十一。



 卒之先一歲,普生日,上遣其子承宗繼器幣、鞍
 馬就賜之。承宗復命,未幾卒。次歲,普已罷中書令。故事,無生辰之賜,特遣普侄婿左正言、直昭文館張秉賜之禮物。普聞之,因追悼承宗,秉未至而普疾篤。先是,普遣親吏甄潛詣上清太平宮致禱,神為降語曰:「趙普,宋朝忠臣,久被病,亦有冤累耳。」潛還,普力疾冠帶,出中庭受神言,涕泗感咽,是夕卒。



 上聞之震悼。謂近臣曰:「普事先帝,與朕故舊,能斷大事,響與朕嘗有不足,眾所知也。朕君臨以來,每優禮之,普亦傾竭自效,盡忠國家,真社稷
 臣也,朕甚惜之。」因出涕,左右感動。廢朝五日,為出次發哀。贈尚書令,追封真定王,賜謚忠獻。上撰神道碑銘,親八分書以賜之。遣右諫議大夫範杲攝鴻臚卿,護喪事。縛綃布各五百匹,米面各五百石。葬日,有司設鹵簿鼓吹如式。



 二女皆笄,普妻和氏言願為尼,太宗再三諭之,不能奪。賜長女名志願,號智果大師;次女名志英,號智圓大師。



 初,太祖側微,普從之游,既有天下,普屢以微時所不足者言之。太祖豁達,謂普曰:「若塵埃中可識天子、
 宰相,則人皆物色之矣。」自是不復言。普少習吏事,寡學術,及為相,太祖常勸以讀書。晚年手不釋卷,每歸私第,闔戶啟篋取書,讀之竟日。及次日臨政,處決如流。既薨,家人發篋視之,則《論語》二十篇也。



 普性深沈有岸谷,雖多忌克,而能以天下事為己任。宋初,在相位者多齷齪循默,普剛毅果斷,未有其比。嘗奏薦某人為某官,及祖不用。普明日復奏其人,亦不用。明日,普又以其人奏,太祖怒,碎裂奏牘擲地,普顏色不變,跪而拾之以歸。他日
 補綴舊紙,復奏如初。太祖乃悟,卒用其人。又有群臣當遷官,太祖素惡其人,不與。普堅以為請,太祖怒曰:「朕固不為遷官。卿若之何?」普曰:「刑以懲惡,賞以酬功,古今信道也。且刑賞天下之刑賞,非陛下之刑賞,豈得以喜怒專之。」太祖怒甚,起,普亦隨之。太祖入宮,普立於宮門,久之不去,竟得俞允。



 太宗入弭德超之讒,疑曹彬有軌,屬普再相,為彬辨雪保證,事狀明白。太宗嘆曰:「朕聽斷不明,幾誤國事。」即日竄逐德超,遇彬如舊。



 祖古守郡為奸
 利,事覺下獄,案劾,愛書未具。郊禮將近,太宗疾其貪墨,遣中使諭旨執政曰:「郊赦可特勿貸祖吉。」普奏曰:「敗官抵罪,宜正刑闢。然國家卜郊肆類,對越天地,告於神明,奈何以吉而隳陛下赦令哉?」太宗善其言,乃止。



 真宗咸平初,追封韓王。二年,詔曰:「故太師贈尚書令、追封韓王趙普,識冠人彞,才高王佐,翊戴興運,光啟鴻圖,雖呂望肆伐之勛,蕭何指縱之效,殆無以過也。自輔弼兩朝,周旋三紀,茂巖廊之碩望,分屏翰之劇權,正直不回,始終
 無玷,謀猷可復,風烈如生。宜預享於大丞,永同休於宗祏,茲為茂典,以答舊勛,其以普配饗太祖廟庭。」



 普子承宗,羽林大將軍,知潭、鄆二州,皆有聲;承煦,成州團練使。弟固、安易。固至都官郎中。



 安易字季和。建隆初,攝府州錄事參軍,節度使折德扆言其清幹,遂命即真。再遷河南府推官。會普居相位,十年不赴調。太平興國中,歷華、邢二鎮掌書記。部芻糧至太原城下,拜監察御史,知興元府;轉殿中,賜緋魚袋。先
 是,兩川民輸稅者以鐵錢易銅錢。安易言其非便,請許納鐵錢,詔從之。九年,起拜宗正少卿,知定州。會以曹璨知州,徙安易為通判,未幾代歸。又表求外任,命知耀州,留不遣,命按視北邊事。



 淳化中,嘗建議以蜀地用鐵錢,準銅錢數倍,小民市易頗為不便,請如劉備時令西川鑄大錢,以十當百。下都省集議,吏部尚書宋琪等言:「劉備時蓋患錢少,因而改作,今安易之請反患錢多,非經久計也。」而安易論請不已,仍募工鑄大錢百餘進之,極
 其精好,俄墜殿階皆碎,蓋熔鑠盡其精液矣。太宗不之詰,猶嘉其用心,賜以金紫,且遣其典鑄。既而大有虧耗,歲中裁得三千餘緡,眾議喧然,遂罷之。事具《食貨志》。



 歷知襄、廬二州,就遷宗正卿,歸朝,復領卿職。時屬籍未備,奏請纂錄,咸平初,乃命梁周翰與安易同修。安易略涉書傳,性強狠,好談世務,而疏闊不可用。初,太宗嘗問農政,安易請復井田之制。又以其家本燕薊,多訪以邊事。



 景德初,禮官詳定明德皇太后靈駕發引,於京師壬地
 權攢,依禮埋懸重,升祔神主。安易上言:



 《禮》云「既虞作主」,虞者,已葬設吉祭也。明未葬則未立虞主及神主。所以周制但鑿木為懸重,以主神靈。王後七月而葬,則埋懸重,掩玄堂,兇仗、轀輬車、龍輴之屬焚於柏城訖,始可立虞主。吉仗還京,備九祭,復埋虞主,然後立神主,升廟室。自曠古至皇朝,上奉祖宗陵廟行此禮,何以今日乃違典章,茍且升祔,方權攢妄立神主,未大葬輒埋懸重?且棺柩未歸園陵,則神靈豈入太廟?奈柏城未焚兇仗,則
 兇穢唐突祖宗。望約孝章近例,但於壬地權攢,未立神主升祔,兇儀一切祗奉。俟丙午年靈駕西去園陵,東回祔廟。如此則免於顛倒,不利國家。



 乃詔有司再加詳定。判禮院孫何等上言:



 按《晉書》羊太后崩,廢一時之祀,天地明堂,去樂不作。又按《禮》,王后崩,五祀之祭不行既殯而祭。所言五祀不行,則天地之祭不廢,遂議以園陵年月不便,須至變禮從宜。又緣先準禮文,候神主升祔畢,方行享祀。若俟丙午歲,則三年不祭宗廟,禮文有闕。況
 明德皇太后德配先朝,禮合升祔。遂與史館檢討同共參詳,以為廟未祔則神靈不至,伏恐祭祀難行。攢既畢則梓宮在郊,可以葬禮比附。遂按《禮》云「葬者藏也,欲人不得而見也。」既不欲穿壙動土,則龍輴、攢木、題湊,蒙槨上四柱如屋以覆,盡塗之。所合埋重,一依近例,便可升祔神主。安易妄言,以兇仗為兇穢,目群官為顛倒,指梓宮為棺柩,令百司分析園陵,浼瀆聖聽,誣罔臣下。



 安易又云「昔日睹群官盡公,奉二帝諸後,並先山陵,後祔廟;
 今日睹群官顛倒,奉明德皇太后,獨先祔廟,後園陵」者。今詳當時先山陵後祔廟,正為年月便順,別無陰陽拘忌。今則年月未便,理合從宜。未埋重則禮文不備,未升祔則廟祭猶闕,須從變禮,以合聖情。兼明德皇太后將赴權攢,而安易所稱「柏城未焚兇仗,則兇穢唐突祖宗。」按《檀弓》云:「喪之朝也,順死者之孝心也。」鄭玄注云,謂遷柩於廟。



 又云:「其哀離其室也,故至於祖考之廟而後行,商朝而殯於祖,周朝而遂葬。」今亦遙辭宗廟而後行,豈
 可以《禮經》所出目為顛倒,吉兇具儀謂之唐突哉?



 又云:「孝章皇后至道元年崩,亦緣有所嫌避,未赴園陵,出京權攢之時,不立神主入廟。直至至道三年,西去園陵,禮畢,然後奉虞主還京,易神主祔廟,以合典禮。」今詳當時文籍,緣孝章為太宗嫂氏,上仙之時,止輟五日視朝,百官不曾成服,與今不同。從初亦無詔命令住廟享。今明德皇太后母儀天下,主上孝極曾、顏,況上仙之初,即有遣命權停享祀。今按禮文,固合如此。安易荒唐庸昧,妄
 有援引,以大功之親,比三年之制,欺罔君上,乃至於斯。



 況安易以訐直自負,所詆者無非良善;以清要自高,所尚者無非鄙俗。名宦之志,老而益堅;詩書之文,懵而不習。本院所議,並明稱典故,旁考時宜,雖曰從權,粗亦稽古,請依無議施行。



 從之。安易又屢言陵廟事,詞多鄙俚。晚歲進趨不已,時論嗤之。二年卒,年七十六。贈工部尚書。錄其子承慶為國子博士,孫從政為太常寺奉禮郎。



 論曰:自古創業之君,其居潛舊臣,定策佐命,樹事建功,
 一代有一代之才,未嘗乏也。求其始終一心,休戚同體,貴為國卿,親若家相,若宋太祖之於趙普,可謂難矣。陳橋之事,人謂普及太宗先知其謀,理勢或然。事定之後,普以一樞密直學士立於新朝數年,範、王、魏三人罷相,始繼其位,太祖不亟於酬功,普不亟於得政。及其當揆,獻可替否,惟義之從,未嘗以勛舊自伐。偃武而修文,慎罰而薄斂,三百餘年之宏規,若平昔素定,一旦舉而措之。太原、幽州之役,終身以輕動為戒,後皆如其言。家人
 見其斷國大議,閉門觀書,取決方冊,他日竊視,乃《魯論》耳。昔傅說告商高宗曰:「學於古訓乃有獲,事不師古,以克永世,匪說攸聞。」普為謀國元臣,乃能矜式往哲,蓍龜聖模,宋之為治,氣象醇正,茲豈無助乎。晚年廷美、多遜之獄,大為太宗盛德之累,而普與有力焉。豈其學力之有限而猶有患失之心歟?君子惜之



\end{pinyinscope}