\article{列傳第一 後妃上}

\begin{pinyinscope}

 太祖母昭憲杜太后太祖孝惠賀皇后孝明王皇后孝章宋皇后
 太宗淑德尹皇后懿德符皇后明德李皇后元德李皇后真宗章懷潘皇後章穆郭皇後章獻明肅劉皇后李宸妃楊淑妃沉貴妃仁宗郭皇后慈聖光獻曹皇后張貴妃苗貴妃周貴妃楊德妃
 馮賢妃英宗宣仁聖烈高皇后



 周人尊祖之詩曰:「厥初生民,時維姜嫄。」蓋推本後稷之所自出,以為王跡之所由基也。宋之興,雖由先世積累,然至宣祖功業始大。昭憲杜後實生太祖、太宗,內助之賢,母範之正,蓋有以開宋世之基業者焉。觀其訓太祖以《無逸》治天下,至於豫定太宗神器之傳,為宗社慮,蓋益遠矣。厥後慈聖光獻曹後擁祐兩朝,宣聖烈高後
 垂簾聽政,而有無祐之治。南渡而後,若高宗之以母道事隆祐,孝宗奉明慈怡愉之樂,皆足以為百王法程。宋三百餘年,外無漢王氏之患,內無唐武、韋之禍,豈不卓然而可尚哉。昭憲垂裕之功,至是茂矣。舊史稱昭憲性嚴毅,有禮法。《易》之《家人》上九曰:「有孚,威如,終吉。」其是之謂歟。作《後妃傳》。



 太祖母昭憲杜太后,定州安喜人也。父爽,贈太師。母範氏,生五子三女,太后居長。既笄歸於宣祖。治家嚴毅有
 禮法。生邕王光濟、太祖、太宗、秦王廷美、夔王光贊、燕國陳國二長公主。



 周顯德中,太祖為定國軍節度使,封南陽郡太夫人。及太祖自陳橋還京師,人走報太后曰:「點檢已作天子。」太后曰:「吾兒素有大志,今果然。」太祖即位,尊為皇太后。太祖拜太后於堂上,眾皆賀。太后愀然不樂,左右進曰:「臣聞『母以子貴』,今子為天子,胡為不樂?」太后曰:「吾聞『為君難』,天子置身兆庶之上,若治得其道,則此位可尊;茍或失馭,求為匹夫不可得,是吾所以憂也。」
 太祖再拜曰:「謹受教。」



 建隆二年,太后不豫,太祖侍樂餌不離左右。疾亟,召趙普入受遺命。太后因問太祖曰:「汝知所以得天下乎?」太祖嗚噎不能對。太后固問之,太祖曰:「臣所以得天下者,皆祖考及太后之積慶也。」太后曰:「不然,正由周世宗使幼兒主天下耳。使周氏有長君,天下豈為汝有乎?汝百歲後當傳位於汝弟。四海至廣,萬幾至眾,能立長君,社稷之福也。」太祖頓首泣曰:「敢不如教。」太后顧謂趙普曰:「爾同記吾言,不可違也。」命普於榻
 前為約誓書,普於紙尾書「臣普書」。藏之金匱,命謹密宮人掌之。



 太后崩於滋德殿,年六十,謚曰明憲。葬安陵,神主祔享太廟。乾德二年,更謚昭憲,合祔安陵。



 太祖孝惠賀皇后,開封人。右千牛衛率府率景思長女也。性溫柔恭順,動以禮法。景思常為軍校,與宣祖同居護聖營。晉開運初,宣祖為太祖聘焉。周顯德三年,太祖為定國軍節度使,封會稽郡夫人。生秦國晉國二公主、魏王德昭。五年,寢疾薨,年三十。建隆三年四月,詔追冊
 為皇后。乾德二年三月,有司上謚曰孝惠。四月,葬安陵西北,神主享於別廟。神宗時,與孝章、淑德、章懷並祔太廟。



 孝明王皇后,邠州新平人。彰德軍節度饒第三女。孝惠崩,周顯德五年,太祖為殿前檢點校,聘後為繼室。後恭勤不懈,仁慈御下。周世宗賜冠帔,封瑯邪郡夫人。



 太祖即位,建隆元年八月,冊為皇后。常服寬衣,佐御膳,善彈箏鼓琴。晨起,誦佛書。事杜太后得驩心。生子女三人,皆
 夭。乾德元年十二月崩,年二十二。有司上謚,翰林學士竇儀撰哀冊文。二年四月,葬安陵之北。神主享於別廟。太平興國二年,祔享太廟。



 孝章宋皇后,河南洛陽人,左衛上將軍偓之長女也。母漢永寧公主。後幼時隨母入見,周太祖賜冠帔。乾德五年,太祖召見,復賜冠帔。時偓任華州節度,後隨母歸鎮。孝明后崩,復隨母來賀長春節。開寶元年二月,遂納入宮為皇后,年十七。性柔順好禮,每帝視朝退,常具冠帔
 候接,佐御饌。太祖崩,號開寶皇后。



 太平興國二年,居西宮。雍熙四年,移居東宮。至道元年四月崩,年四十四。有司上謚,權殯普濟佛舍。三年正月,祔葬永昌陵北。命吏部侍郎李至撰哀冊文,神主享於別廟。神宗時,升祔太廟。



 太宗淑德尹皇后,相州鄴人。滁州刺史廷勛之女。兄崇珂,保信軍節度。太宗在周時娶焉。早薨。及帝即位,詔追冊為皇后,並謚,葬孝明陵西北。神主享於別廟,後升祔
 太廟。



 懿德符皇后,陳州宛丘人。魏王彥卿第六女也。周顯德中,歸太宗。建隆初,封汝南郡夫人,進封楚國夫人。太宗封晉王,改越國。開寶八年薨,年三十四。葬安陵西北。帝即位,追冊為皇后,謚懿德,享於別廟。至道三年十一月,詔有司議太宗配,宰相請以後配,詔從之。奉神主升祔太廟,后姊,周世宗後也,淳化四年殂。



 明德李皇后,潞州上黨人。淄州刺史處耘第二女。開寶
 中,太祖為太宗聘為妃。既納幣,會太祖崩,至太平興國三年始入宮,年十九。雍熙元年十二月,詔立為皇后。後性恭謹莊肅,撫育諸子及嬪御甚厚。嘗生皇子,不育。至道二年,封後嫡母吳氏為衛國太夫人,後改封楚國,及封後母陳氏為韓國太夫人。



 太宗崩,真宗即位。至道三年四月,尊後為皇太后,居西宮嘉慶殿。咸平二年,宰相請別建宮立名,從之。四年宮成,移居之,仍上宮名曰萬安。景德元年崩,年四十五。謚明德。權殯沙臺。三年十月,
 祔葬永熙陵。禮官請以懿德、明德同祔太宗廟室,以先後為次,從之。



 李賢妃,真定人,乾州防禦使英之女也。



 太祖聞妃有容德,為太宗聘之。開寶中,封隴西郡君。太宗即位,進夫人。生皇女二人,皆早亡,次生楚王元佐。妃嘗夢日輪逼己,以裾承之,光耀遍休,驚而悟,遂生真宗。太平興國二年薨,年三十四。



 真宗即位,追封賢妃,又進上尊號為皇太后。有司上謚曰元德。咸平三年,祔葬永熙陵。以中書侍
 郎、平章事李沆為園陵使。車駕詣普安院攢宮,素服行禮,拜伏嗚咽。命駕部郎中、知制誥梁周翰撰哀冊。神主祔別廟。



 大中祥符元年,追贈後父英檢校太尉、安國軍節度、常山郡王,母魏國太夫人。大中祥符三年,禮官趙湘請以後祔太宗廟室。真宗曰:「此重事也,俟令禮官議之。」六年秋,宰相王旦與群臣表請後尊號中去「太」字,升祔太廟明德之次,從之。



 真宗章懷潘皇后,大名人,忠武軍節度美第八女。真宗
 在韓邸,太宗為聘之,封莒國夫人。端拱二年五月薨,年二十二。真宗即位,追冊為皇后,謚壯懷,葬永昌陵之側,陵名保泰。神主享於別廟,舊制後謚冠以帝謚。慶歷中,禮官言,「孝」字連太祖謚,「德」字連太宗謚。遂改「壯」為「章」,以連真宗謚雲。



 章穆郭皇后,太原人,宣徽南院便守文第二女。淳化四年,真宗在襄邸,太宗為聘之。封魯國夫人,進封秦國。真宗嗣位,立為皇后。景德四年,從幸西京還,以疾崩,年三
 十二。



 後謙約惠下,性惡奢靡。族屬入謁禁中,服飾華侈,必加戒勖。有以家事求言於上者,後終不許。兄子出嫁,以貧欲祈恩賚,但出裝具給之。上尤加禮重。



 及崩,上深嗟悼。禮官奏皇帝七日釋服,特詔增至十三日。太常上謚曰壯穆。靈駕發引,命翰林學士楊億撰哀冊。葬永熙陵之西北,神主享於別廟。以後弟崇儀副使崇仁為壯宅使、康州刺史,侄承慶、承壽皆遷官。大中祥符中,封後母高唐郡太夫人梁氏萊國太夫人。仁宗即位,升祔真
 宗廟室,改謚章穆。



 章獻明肅劉皇后,其先家太原,後徙益州,為華陽人。祖延慶,在晉、漢間為右驍衛大將軍;父通,虎捷都指揮使、嘉州刺史,從征太原,道卒。後,通第二女也。



 初,母龐夢月入懷,已而有娠,遂生後。後在襁褓而孤,鞠於外氏。善播□。蜀人龔美者,以鍛銀為業,摧之入京師。後年十五入襄邸,王乳母秦國夫人性嚴整,因為太宗言之,令王斥去。王不得已,置之王宮指使張耆家。太宗崩,真宗即位,
 入為美人。以其無宗族,乃更以美為兄弟,改姓劉。大中祥符中,為修儀,進德妃。



 自章穆崩,真宗欲立為皇后,大臣多以為不可,帝卒立之。李宸妃生仁宗,後以為己子,與楊淑妃撫禮甚至。後性警悟,曉書史,聞朝廷事,能記其本末。真宗退朝,閱天下封奏,多至中夜,後皆預聞。宮圍事有問,輒傅引故實以對。



 天禧四年,帝久疾居宮中,事多決於後。宰相上寇準密議奏請皇太子監國,以謀洩罷相,用丁謂代之。既而,入內都知周懷政謀廢後殺謂,
 復用準以輔太子。客省使楊崇勛、內殿承制楊懷吉詣謂告,謂夜乘犢車,挾崇勛、懷吉造樞密使曹利用謀。明日,誅懷政,貶準衡州司馬。於是詔皇太子開資善堂,引大臣決天下事,後裁制於內。



 真宗崩,遺詔尊後為皇太后,軍國重事,權取處分。謂等請太后御別殿,太后遣張景宗、雷允恭諭曰:「皇帝視事,當朝夕在側,何須別御一殿?」於是請帝與太后五日一御承明殿,帝位左,太后位右,垂簾決事。議已定,太後忽出手書,第欲禁中閱章奏,
 遇大事即召對輔臣。其謀出於丁謂,非太后意也。謂既貶,馮拯等三上奏,請如初議。帝亦以為言,於是始同御承明殿。百官表賀,太后哀慟。有司請制令稱「吾」,以生日為長寧節,出入御大安輦,鳴鞭侍衛如乘輿。令天下避太后父諱。群臣上尊號曰應元崇德仁壽慈聖太后,御文德殿受冊。



 天聖五年正旦,太后御會慶殿。群臣及契丹使者班廷中,帝再拜跪上壽。是歲郊祀前,出手書諭百官,毋請加尊號。禮成,帝率百官恭謝如元日。七年冬
 至,天子又率百官上壽,範仲淹力言其非,不聽。九月,詔長寧節百官賜衣,天下賜宴,皆如乾元節。



 明道元年冬至,復御文德殿。有司陳黃麾仗,設宮架、登歌、二舞。明年,帝親耕籍田,太后亦謁太廟,乘玉輅,服禕衣、九龍花釵冠,齋於廟。質明,服袞衣,十章,減宗彞、藻,去劍,冠儀天,前後垂珠翠十旒。薦獻七室,皇太妃亞獻,皇后終獻。加上尊號曰應天齊聖顯功崇德慈仁保壽太后。



 是歲崩,年六十五。謚曰章獻明肅,葬於永定陵之西北。舊制皇
 后皆二謚,稱制,加四謚自後始。追贈三世皆至太師、尚書令、兼中書令,父封魏王。



 初,仁宗即位尚少,太后稱制,雖政出宮闈,而號令嚴明,恩威加天下。左右近習亦少所假借,宮掖間未嘗妄改作。內外賜與有節,柴氏、李氏二公主入見,猶服髲剃。太后曰:「姑老矣。」命左右賜以珠璣帕首。時潤王元份婦安國夫人李氏老,發且落,見太后,亦請帕首。太后曰:「大長公主,太宗皇帝女,先帝諸妹也;若趙家老婦,寧可比耶?」舊賜大臣茶,有龍鳳飾,太后曰:「
 此豈人臣可得?」命有司別制入香京挺以賜之。賜族人御食,必易以扣器,曰:「尚方器勿使入吾家也。」常服絁繻練裙,侍者見仁宗左右簪珥珍麗,欲效之。太后戒曰:「彼皇帝嬪御飾也,汝安得學。」



 先是,小臣方仲弓上書,請依武后故事,立劉氏廟,而程琳亦獻《武后臨朝圖》,後擲其書於地曰:「吾不作此負祖宗事。」有漕臣劉綽者,自京西還,言在庾有出剩糧千餘斛,乞付三司。後問曰:「卿識王曾、張知白、呂夷簡、魯宗道乎?此四人豈因獻羨餘進哉!」



 後稱制凡十一年,自仁宗即位,乃諭輔臣曰:「皇帝聽斷之暇,宣詔名儒講習經史,以輔其德。」於是設幄崇政殿之西廡,而日命近臣侍講讀。



 丁謂、曹利用既以侮權貶竄,而天下惕然畏之。晚稍進外家,任內宮羅崇勛、江德明等訪外事,崇勛等以此勢傾中外。兄子從德死,姻戚、門人、廝役拜官者數十人。御史曹修古、楊偕、郭勸、段少連論奏,太后悉逐之。



 太后保護帝既盡力,而仁宗所以奉太后亦甚備。上春秋長,猶不知為宸妃所出,終太后
 之世無毫發間隙焉。及不豫,帝為大赦,悉召天下醫者馳傳詣京師。諸嘗為太后謫者皆內徙,死者復其宮。其後言者多追詆太后時事,範仲淹以為言,上曰:「此朕所不忍聞也。」下詔戒中外毋輒言。



 於是泰寧軍節度使錢惟演請以章獻、章懿與章穆並祔真宗室。詔三省與禮院議,皆以謂章穆皇后位崇中壺,已祔真宗廟室,自協一帝一後之文;章獻明肅處坤元之尊,章懿感日符之貴,功德莫與為比,謂宜崇建新廟,同殿異室,歲時薦饗,
 一用太廟之儀,仍別立廟名,以崇世享。翰林學士馮元等請以奉慈為名,詔依。慶歷五年,禮院言章獻、章懿二後,請遵國朝懿德、明德、元德三后同祔太宗廟室故事,遷祔真宗廟。詔兩制議,翰林學士王堯臣等議,請遷二後祔,序於章穆之次,從之。



 李宸妃,杭州人也。祖延嗣,仕錢氏,為金華縣主簿;父仁德,終左班殿直。初入宮,為章獻太后侍兒,壯重寡言,真宗以為司寢。既有娠,從帝臨砌臺,玉釵墜,妃惡之。帝心
 卜:釵完,當為男子。左右取以進,釵果不毀,帝甚喜。已而生仁宗,封崇陽縣君;復生一女,不育。進才人,後為婉儀。仁宗即位,為順容,從守永定陵。章獻太后使劉美、張懷德為訪其親屬,得其弟用和,補三班奉職。



 初,仁宗在襁褓,章獻以為己子,使楊淑妃保視之。仁宗即位,妃嘿處先朝嬪御中,未嘗自異。人畏太后,亦無敢言者。終太后世,仁宗不自知為妃所出也。



 明道元年,疾革,進位宸妃,薨,年四十六。



 初,章獻太后欲以宮人禮治喪於外,丞相
 呂夷簡奏禮宜從厚。太后遽引帝起,有頃,獨坐簾下,召夷簡問曰:「一宮人死,相公云云,何歟?」夷簡曰:「臣待罪宰相,事無內外,無不當預。」太后怒曰:「相公欲離間吾母子耶!」夷簡從容對曰:「陛下不以劉氏為念,臣不敢言;尚念劉氏,是喪禮宜從厚。」太后悟,遽曰:「宮人,李宸妃也,且奈何?」夷簡乃請治用一品禮,殯洪福院。夷簡又謂入內都知羅崇勛曰:「宸妃當以後服殮,用水銀實棺,異時勿謂夷簡未嘗道及。」崇勛如其言。



 後章獻太后崩,燕王為
 仁宗言:「陛下乃李宸妃有所生,妃死以非命。」仁宗號慟頓毀,不視朝累日,下哀痛之詔自責。尊宸妃為皇太后,謚壯懿。幸洪福院祭告,易梓宮,親哭視之,妃玉色如生,冠服如皇太后,以水銀養之,故不壞。仁宗嘆曰:「人言其可信哉!」遇劉氏加厚。陪葬永定陵,廟曰奉慈。又即景靈宮建神御殿,曰廣孝。慶歷中,改謚章懿,升祔太廟。拜用和為彰信軍節度使、檢校侍中,寵賚甚渥。既而追念不已,顧無以厚其家,乃以福康公主下嫁用和之子瑋。



 楊淑妃,益州郫人。祖□,父知儼,知儼弟知信,隸禁軍,為天武副指揮使。



 妃年十二入皇子宮。真宗即位,拜才人,又拜婕妤,進婉儀,仍詔婉儀升從一品,位昭儀上。帝東封、西祀,凡巡幸皆從。章獻太后為修儀,妃與之位幾埒。而妃通敏有智思,奉順章獻無所忤,章獻親愛之。故妃雖貴幸,終不以為己間,後加淑妃。真宗崩,遺制以為皇太后。



 始,仁宗在乳褓,章獻使妃護視,凡起居飲食必與之俱,所以擁祐扶持,恩意勤備。及帝即位,嘗召其侄永
 德見禁中,欲授以諸司副使。妃辭曰:「小兒豈勝大恩,小官可也。」更命為右侍禁。



 章獻遺誥尊為皇太后,居宮中,與皇帝同議軍國事。閣門趣百僚賀,御史中丞蔡齊目臺吏毋追班,乃入白執政曰:「上春秋長,習知天下情偽,今始親政事,豈宜使女後相繼稱制乎?」乃詔刪去遺誥「同議軍國事」語,第存後號。奉緡錢二萬助湯沐,後名其所居宮曰保慶,稱保慶皇太后。



 景祐三年,無疾而薨,年五十三。殯於皇儀殿。帝思其保護之恩,命禮官議加服
 小功。



 初,仁宗未有嗣,後每勸帝擇宗子近屬而賢者,養於宮中,其選即英宗也。英宗立,言者謂禮慈母於子祭,於孫止,請廢后廟,瘞其主園陵。英宗弗欲遽也,下有司議,未上,會帝崩,遂罷。後父祖皆累贈至一品,知信贈節度使。知信子景宗,見《外戚傳》。



 沈貴妃,宰相倫之孫,父繼宗,光祿少卿。大中祥符初,以將相家子被選。初為才人,歷美人、婕妤、充媛,至德妃。為人淑儉不華,帝亦以妃家世故,待之異眾。長秋虛位,帝
 欲立之,有從中沮之者,不果。嘉祐末,進貴妃。熙寧九年薨,年八十三。許出殯其家,車駕臨奠,輟視朝三日,謚昭靜。



 仁宗郭皇后,其先應州金城人。平盧軍節度使崇之孫也。天聖二年,立為皇后。



 初,帝寵張美人,欲以為後,章獻太後難之。後既立,而頗見疏。其後尚美人、楊美人俱幸,數與後忿爭。一日,尚氏於上前有侵後語,後不勝忿,批其頰,上自起救之,誤批上頸,上大怒。入內都知閻文應
 因與上謀廢後,且勸帝以爪痕示執政。上以示呂夷簡,且告之故,夷簡亦以前罷相怨後,乃曰:「古亦有之。」後遂廢。詔封為凈妃、玉京沖妙仙師,賜名清悟,居長樂宮。



 於是中丞孔道輔、諫官御史範仲淹段少連等十人伏閣言:「後無過,不可廢。」道輔等俱被黜責。景祐元年,出居瑤華宮,而尚美人亦廢於洞真宮入道,楊美人別宅安置。又賜后號金庭教主、沖靜元師。後帝頗念之,遣使存問,賜以樂府,後和答之,辭甚愴惋。帝嘗密令召入,後曰:「若
 再見召者,須百官立班受冊方可。」屬小疾,遣文應挾醫診視,數日,乃言後暴薨。中外疑閻文應進毒,而不得其實。上深悼之,追復皇后,而停謚冊祔廟之禮。



 慈聖光獻曹皇后,真定人,樞密使周武惠王彬之孫也。明道二年,郭后廢,詔聘入宮。景祐元年九月,冊為皇后。性慈儉,重稼穡,常於禁苑種穀、親蠶,善飛帛書。



 慶歷八年閏正月,帝將以望夕再張燈,後諫止。後三日,衛卒數人作亂,夜趙屋叩寢殿。後方侍帝,聞變遽起。帝欲出,後
 閉合擁持,趣呼都知王守忠使引兵入。賊傷宮嬪殿下,聲徹帝所,宦者以乳嫗歐小女子始奏,後叱之曰:「賊在近殺人,敢妄言耶!」後度賊必縱火,陰遣人挈水踵其後,果舉炬焚簾,水隨滅之。是夕,所遣宦侍,後皆親剪其發,諭之曰:「明日行賞,用是為驗。」故爭盡死力,賊即禽滅。合內妾與卒亂當誅,祈哀幸姬,姬言之帝,貸共死。後具衣冠見,請論如法,曰:「不如是,無以肅清禁掖。」帝命坐,後不可,立請,移數刻,卒誅之。



 張妃怙寵上僭,欲假後蓋出游。
 帝使自來請,後與之,無靳色。妃喜,還以告,帝曰:「國家文物儀章,上下有秩,汝張之而出,外廷不汝置。」妃不懌而輟。



 英宗方四歲,育禁中,後拊鞠周盡;迨入為嗣子,贊策居多。帝夜暴疾崩,後悉斂諸門鑰置於前,召皇子入。及明,宰臣韓琦等至,奉英宗即位,尊後為皇太后。



 帝感疾,請權同處分軍國事,御內東門小殿聽政。大臣日奏事有疑未決者,則曰「公輩更議之」,未嘗出己意。頗涉經史,多援以決事。中外章奏日數十,一一能紀綱要。檢柅曹
 氏及左右臣僕,毫分不以假借,宮省肅然。



 明年夏,帝疾益愈,即命撤簾還政,帝持書久不下,及秋始行之。敕有司崇峻典禮,以弟佾同中書門下平章事。神宗立,尊為太皇太后,名宮曰慶壽。帝致極誠孝,所以承迎娛悅,無所不盡,從行登玩,每先後策掖。後亦慈愛天至,或退朝稍晚,必自至屏扆候矚,間親持膳飲以食帝。外家男子,舊毋得入謁。後春秋高,佾亦老,帝數言宜使入見,輒不許。他日,佾侍帝,帝復為請,乃許之,因偕詣後合。少焉,帝
 先起,若令佾得伸親親意。後遽曰:「此非汝所當得留。」趣遣出。



 晚得水疾,侍醫莫能治。元豐二年冬,疾甚,帝視疾寢門,衣不解帶。旬日崩,年六十四。帝推恩曹氏,拜佾中書令,進官者四十餘人。



 初,王安石當國,變亂舊章,後乘間語神宗,謂祖宗法度不宜輕改。熙寧宗祀前數日,帝至後所,後曰:「吾昔聞民間疾苦,必以告仁宗,因赦行之,今亦當爾。」帝曰:「今無他事。」後曰:「吾聞民間甚苦青苗、助役,宜罷之。安石誠有才學,然怨之者甚眾,帝欲愛惜保
 全之,不若暫出之於外。」帝悚聽,垂欲止,復為安石所持,遂不果。



 帝嘗有意於燕薊,已與大臣定議,乃詣慶壽宮白其事。後曰:「儲蓄賜予備乎?鎧仗士卒精乎?」帝曰:「固已辦之矣。」後曰:「事體至大,吉兇悔吝生乎動,得之不過南面受賀而已;萬一不諧,則生靈所系,未易以言。茍可取之,太祖、太宗收復久矣,何待今日。」帝曰:「敢不受教。」



 蘇軾以詩得罪,下御史獄,人以為必死。後違預中聞之,謂帝曰:「嘗憶仁宗以制科得軾兄弟,喜曰:『吾為子孫得兩宰
 相。』今聞軾以作詩系獄,得非仇人中傷之乎?捃至於詩,其過微矣。吾疾勢已篤,不可以冤濫致傷中和,宜熟察之。」帝涕泣,軾由此得免。及崩,帝哀慕毀瘠,殆不勝喪。有司上謚,葬於永昭陵。



 張貴妃,河南永安人也。祖穎,進士第,終建平令。父堯封,亦舉進士,為石州推官卒。時堯封史堯佐補蜀官,堯封妻錢氏求挈孤幼隨之官,堯佐不收恤,以道遠辭。妃幼無依,錢氏遂納於章惠皇后宮寢。長得幸,有盛寵。妃巧
 慧多智數,善承迎,勢動中外。慶歷元年,封清河郡君,歲中為才人,遷修媛。忽被疾,曰:「妾姿薄,不勝寵名,原為美人。」許之。皇祐初,進貴妃。後五年薨,年三十一。仁宗哀悼之,追冊為皇后,謚溫成。追封堯封清河郡王,謚景思。而堯佐因緣僥幸,致位通顯云。



 苗貴妃,開封人。父繼宗。母許,先為仁宗乳保,出嫁繼宗。帝登位,得復通籍。妃以容德入侍,生唐王昕、福康公主。封仁壽郡君,拜才人、昭容、德妃。英宗育於禁中,妃擁祐
 頗有恩。既踐阼,疇其前勞,進貴妃。贈其父至太師、吳國公,母陳、楚國夫人。福康下嫁,當貤恩外家,抑不肯言。元祐六年薨,年六十九。哲宗輟朝,出奠,發哀苑中,謚曰昭節。



 周貴妃,開封人。生四歲,從其姑入宮,張貴妃育為女。稍長,遂得侍仁宗,生兩公主。帝崩,妃日一疏食,屏處一室,誦佛書,困則假寐,覺則復誦,晝夜不解衣者四十年。公主下嫁錢景臻、郭獻卿。連進至賢妃,徽宗立,加貴妃。歷
 五朝,勤約一致。啟壽藏於周氏塋南,傍建僧屋,費緡錢六萬,皆貯儲奉賜。郭公主先亡,詔許出外第,與親戚相往來。年九十三薨,謚昭淑。



 楊德妃,定陶人。天聖中,以章獻太后姻連,選為御侍,封原武郡君,進美人。端麗機敏,妙音律,組紃、書藝一過目如素習。父忠為侍禁,仁宗欲加獎擢,辭曰:「外官當積勞以取貴,今以恩澤徼幸,恐啟左右詖謁之端。」帝悅,命徙居肅儀殿。贈其祖貴州刺史,而官其叔弟五人。積與郭
 后不相能,後既廢,妃亦遣出。後復召為婕妤,歷修媛、修儀。熙寧五年薨,年五十四。贈德妃。



 馮賢妃,東平人。曾祖炳,知雜御史;祖起,兵部侍郎。妃以良家女,九歲入宮。及長,得侍仁宗,生邢、魯國二公主。封始平郡君。帝將登其品秩,力辭不拜。養女林美人得幸神宗,生二王而沒,王尚幼,妃保育如己子。累加才人、婕妤、修容。在禁掖幾六十年,始終五朝,動循禮度。薨,年七十七,贈賢妃。



 英宗宣仁聖烈高皇后,亳州蒙城人。曾祖瓊,祖繼勛,皆有勛王室,至節度使。母曹氏,慈聖光獻后姊也,故後少鞠宮中。時英宗亦在帝所,與後年同,仁宗謂慈聖,異日必以為配。既長,遂成昏濮邸。生神宗皇帝、岐王顥、嘉王頵、壽康公主。治平二年冊為皇后。



 後弟內殿崇班士林,供奉久,帝欲遷其官,後謝曰:「士林獲升朝籍,份量已過,豈宜援先後家比?」辭之。神宗立,尊為皇太后,居寶慈宮。帝累欲為高氏營大第,後不許。久之,但斥望春門外隙
 地以賜,凡營繕百役費,悉出寶慈,不調大農一錢。



 元豐八年,帝不豫,浸劇,宰執王珪等入問疾,乞立延安郡王為皇太子,太后權同聽政,帝頷之。珪等見太后簾下。後泣,撫王曰:「兒孝順,自官家服藥,未嘗去左右,書佛經以祈福,喜學書,已誦《論語》七卷,絕不好弄。」乃令王出簾外見珪等,珪等再拜謝且賀。是日降制,立為皇太子。初,岐、嘉二王日問起居,至是,令母輒入。又陰敕中人梁惟簡,使其妻制十歲兒一黃袍,懷以來,蓋密為踐阼倉卒備
 也。



 哲宗嗣位,尊為太皇太后。驛召司馬光、呂公著,未至,迎問今日設施所宜先。未及條上,已散遣修京城役夫,減皇城覘卒,止禁庭工技,廢導洛司,出近侍尤亡狀者。戒中外毋苛斂,寬民間保戶馬。事由中旨,王珪等弗預知。又起文彥博於既老,遣使勞諸途,諭以復祖宗法度為先務,且令亟疏可用者。



 從父遵裕坐西征失律抵罪,蔡確欲獻諛以固位,乞復其官。後曰:「遵裕靈武之役,塗炭百萬,先帝中夜得報,起環榻行,徹旦不能寐,聖情自
 是驚悸,馴致大故,禍由遵裕,得免刑誅,幸矣。先帝肉未冷,吾何敢顧私恩而違天下公議!」確悚心慄而止。



 光、公著至,並命為相,使同心輔政,一時知名士匯進於廷。凡熙寧以來政事弗便者,次第罷之。於是以常平舊式改青苗,以嘉祐差役參募役,除市易之法,逭茶鹽之禁舉邊砦不毛之地以賜西戎,而宇內復安。契丹主戒其臣下,復勿生事於疆場,曰:「南朝盡行仁宗之政矣。」



 蔡確坐《車蓋亭詩》謫嶺表,後謂大臣曰:「元豐之末,吾以今皇帝所
 書佛經出示人,是時惟王珪曾奏賀,遂定儲極。且以子繼父,有何間言?而確自謂有定策大功,妄扇事端,規為異時眩惑地。吾不忍明言,姑托訕上為名逐之耳。此宗社大計,奸邪怨謗所不暇恤也。」



 廷試舉人,有司請循天聖故事,帝後皆御殿,後止之。又請受冊寶於文德,後曰:「母後當陽,非國家美事,況天子正衙,豈所當禦?就崇政足矣。」上元燈宴,後母當入觀,止之曰:「夫人登樓,上必加禮,是由吾故而越黃制,於心殊不安。」但令賜之燈燭,遂
 歲以為常。



 侄公繪、公紀當轉觀察使,力遏之。帝請至再,僅遷一秩,終後之世不敢改。又以官冗當汰,詔損外氏恩四之一,以為宮掖先。臨政九年,朝廷清明,華夏綏定。



 宋用臣等既被斥,祈神宗乳媼入言之,冀得復用。後見其來,曰:「汝來何為?得非為用臣等游說乎?且汝尚欲如曩日,求內降乾撓國政耶?若復爾,吾即斬汝。」媼大懼,不敢出一言。自是內降遂絕,力行故事,抑絕外家俬恩。文思院奉上之物,無問鉅細,終身不取其一。人以……



\end{pinyinscope}