\article{志第五十四 禮四(吉禮四)}

\begin{pinyinscope}

 明堂



 明堂。宋初,雖有季秋大享之文,然未嘗親祠,命有司攝事而已。真宗始議行之,屬封岱宗、祀汾陰,故亦未遑。皇
 祐二年三月,仁宗謂輔臣:「今年冬至日,當親祀圜丘,欲以季秋行大享明堂禮。然自漢以來,諸儒各為論議,駁而不同。夫明堂者,布政之宮,朝諸侯之位,天子之路寢,乃今之大慶殿也。況明道初合祀天地於此,今之親祀,不當因循,尚於郊壝寓祭也。其以大慶殿為明堂,分五室於內。」仍詔所司詳定儀注以聞。禮院請依《周禮》,設五室於大慶殿。舊禮,明堂五帝位皆為幔室。今旁帷上幕,宜用青繒朱裏;四戶八牖,赤綴戶,白綴牖,宜飾以朱白
 繒。



 詔曰:「祖宗親郊,合祭天地,祖宗並配,百神從祀。今祀明堂,正當親郊之期,而禮官所定,祭天不及地祇,配坐不及祖宗,未合三朝之制。且移郊為大享,蓋亦為民祈福,宜合祭皇地祇,奉太祖、太宗、真宗並配,而五帝、神州亦親獻之。日、月、河、海諸神,悉如圜丘從祀之數。」禮官議諸神位未決,帝諭文彥博等曰:「郊壇第一龕者在堂,第二、第三龕設於左右夾廡及龍墀上,在壝內外者,列於堂東西廂及後廡,以象壇壝之制。仍先繢圖。」



 令輔臣、禮
 官視設神位。昊天上帝,堂下山罍各四。皇地祇,大尊、著尊、犧尊、山罍各二,在堂上室外神坐左;象尊二,壺尊二,山罍四,在堂下中陛東。三配帝、五方帝,山罍各二,於室外神坐左。神州,大尊、著尊、山罍各二,在堂上神坐左。牲各用一犢,毛不能如其方,以純色代。籩豆,數用大祠。日、月、天皇大帝、北極,大尊各二,在殿上神坐左。籩豆,數用中祠。五官,數用小祠。內官,象尊各二,每方岳、鎮、海、瀆,山尊各二,在堂左右。中官,壺尊各二,在丹墀、龍墀上。外官,
 每方丘陵、墳衍、原隰,概尊各二,眾星,散尊各二,在東西廂神坐左右。配帝席蒲越,五人帝莞,北極以上稿秸加褥,五官、五星以下莞不加褥,餘如南郊。景靈宮升降,置黃道褥位。致齋日,陳法駕鹵簿儀仗,壝門大次之後設小次。知廟卿酌奠七祠,文臣分享奉慈、後廟,近侍宿朝堂。行事及從升堂、百官分宿升龍門外,內庭省司宿本所,諸方客宿公館。設宿爟火於望燎位東南。牲增四犢,羊、豕依郊各十六,以薦從祀。帝謂前代禮有祭玉、燔玉,
 今獨有燔玉,命擇良玉為琮、璧。皇地祇黃琮、黃幣,神州兩圭有邸、黑幣,日月圭、璧,皆置神坐前,燔玉加幣上。五人帝、五官白幣,日月、內官以下,幣從方色。



 九月二十四日未漏上水一刻,百官朝服,齋於文德殿。明日未明二刻,鼓三嚴,帝服通天冠、降紗袍,玉輅,警蹕,赴景靈宮,即齋殿易袞圭,薦享天興殿畢,詣太廟宿齋,其禮具太廟。未明三刻,帝靴袍,小輦,殿門契勘,門下省奉寶輿先入。及大次,易袞圭入,至版位,樂舞作,沃盥,自大階升。禮儀
 使導入太室,詣上帝位,奠玉幣於神坐,次皇地祇、五方帝、神州,次祖宗。奠幣酌獻之敘亦然。皇帝降自中階,還版位,樂止。禮生引分獻官奉玉幣,祝史、齋郎助奠諸神坐,乃進熟。諸太祝迎上帝、皇地祇饌,升自中階;青帝、赤帝、神州、配帝、大明、北極、太昊、神農氏饌,升自東階;黃帝、白帝、黑帝、夜明、天皇大帝、軒轅、少昊、高陽氏饌,升自西階;內中官、五官、外官、五星諸饌,隨便升設。亞獻將升,禮生分引獻官俱詣罍洗,各由其階酌獻五人帝、日月、天
 皇、北極,下及左右夾廡、丹墀、龍墀、庭中五官、東西廂外官眾星坐。禮畢,帝還大次,解嚴,改服乘輦,御紫宸殿,百官稱賀。乃常服,御宣德門肆赦,文武內外官遞進官有差。宣制畢,宰臣百僚賀於樓下,賜百官福胙及內外致仕文武升朝官以上粟帛、羊酒。



 嘉祐七年七月,詔復有事於明堂,有司言:「皇祐參用南郊百神之位,不應祭法。宜如隋、唐舊制,設昊天上帝、五方帝位,以真宗配,而五人帝、五官神從祀,餘皆罷。又前一日,親享太廟,嘗停孟冬
 之薦,考詳典禮,宗廟時祭,未有因嚴配而輟者。今明堂去孟冬畫日尚遠,請復薦廟。前者祖宗並侑,今用典禮獨配;前者地祇、神州並享,今以配天而罷。是皆變於禮中之大者也。《開元》、《開寶》二禮,五帝無親獻儀。舊禮,先詣昊天奠獻,五帝並行分獻,以侍臣奠幣,皇帝再拜,次詣真宗神坐,於禮為允。」而帝欲盡恭於祀事,五方帝位並親獻焉。朝廟用犢一,羊七,豕七;昊天上帝、配帝犢各一,羊、豕各二;五方、五人帝共犢五,豕五,羊五;五官從祀共
 羊、豕十。



 英宗即位,禮官議仁宗配明堂,知制誥錢公輔等言:「《孝經》曰:『昔者周公郊祀后稷以配天,宗祀文王於明堂以配上帝。』又曰:『孝莫大於嚴父,嚴父莫大於配天,則周公其人也。』以周公言之則嚴父,以成王言之則嚴祖。方是時,政則周公,祭則成王,亦安在必嚴其父哉?《我將》之詩是也。真宗則周之武王,仁宗則周之成王,雖有配天之業,而無配天之祭,未聞成、康以嚴父之故,廢文王配天之祭而移之。以孔子之心推周公之祭,則嚴父
 也;以周公之心攝成王之祭,則嚴祖也,嚴祖、嚴父,其義一也。漢明始建明堂,以光武配,當始配之代,適符嚴父之說,章、安二帝亦弗之變,最為近古而合乎禮。唐中宗時,則以高宗配;在玄宗時,則以睿宗配;在永泰時,則以肅宗配。禮官不能推明經訓,務合古初,反雷同其論以惑時主,延及於今,牢不可破。仁宗嗣位之初,儻有建是論者,則配天之祭常在乎太宗矣。願詔有司博議,使配天之祭不膠於嚴父,而嚴父之道不專乎配天。」



 觀文殿
 學士孫抃等曰:「《易》稱『先王作樂崇德,薦之上帝,以配祖考』。蓋祖、考並可配天,符於《孝經》之說,不可謂必嚴其父也。祖、考皆可配郊與明堂而不同位,不可謂嚴祖、嚴父其義一也。雖周家不聞廢文配而移於武,廢武配而移於成,然《易》之配考,《孝經》之嚴父,歷代循守,不為無說。魏明帝祀文帝於明堂以配上帝,史官謂是時二漢之制具存,則魏所損益可知,亦不可謂章、安之後配祭無傳,遂以為未嘗嚴父也。唐至本朝講求不為少,所以不敢
 異者,舍周、孔之言無所本也。今以為《我將》之詩,祀文王於明堂而歌者,安知非孔子刪《詩》,存周全盛之《頌》被於管弦者,獨取之也?仁宗繼體守成,置天下於泰安四十二年,功德可謂極矣。今祔廟之始,抑而不得配帝,甚非所以宣章嚴父之大孝。」



 諫官司馬光、呂誨曰:「孝子之心,孰不欲尊其父?聖人制禮以為之極,不敢逾也。《詩》曰:『思文后稷,克配彼天。』又《我將》:『祀文王於明堂。』下此,皆不見於經。前漢以高祖配天,後漢以光武配明堂。以是觀之,
 自非建邦啟土、造有區夏者,皆無配天之文。故雖周之成、康,漢之文、景、明、章,德業非不美也,然而不敢推以配天,避祖宗也。孔子以周公有聖人之德,成太平之業,制禮作樂,而文王適其父,故引以證『聖人之德莫大於孝』答曾子,非謂凡有天下者皆當尊其父以配天,然後為孝也。近代祀明堂者,皆以其父配上帝,此乃誤釋《孝經》之義,而違先王之禮也。景祐中,以太祖為帝者之祖,比周之後稷,太宗、真宗為帝者之宗,比周之文、武,然則祀
 真宗於明堂以配上帝,亦未失古禮。仁宗雖豐功美德洽於四海,而不在二祧之位,議者乃欲舍真宗而以仁宗配,恐於祭法不合。」詔從抃議。



 御史趙鼎請遞遷真宗配雩祭,太宗配祈穀、神州,用唐故事。學士王珪等以為:「天地大祭有七,皆以始封受命創業之君配神作主,明堂用古嚴父之道配以近考,故在真宗時以太宗配,在仁宗時以真宗配,今則以仁宗配。仁宗始罷太宗明堂之配,太宗先已配雩祀、祈穀及神州之祭,本非遞遷。今
 明堂既用嚴父之道,則真宗配天之祭於禮當罷,不當復分雩祭之配也。」治平四年九月,大享明堂,以英宗配。



 元豐,詳定禮文所言:「祀帝南郊,以天道事之,則雖配帝用犢,《禮》所謂『帝牛不吉,以為稷牛』是也。享帝明堂,以人道事之,則雖天帝用太牢,《詩》所謂『我將我享,維羊維牛』是也。自梁用特牛,隋、唐因之,皆用特牲,非所謂以人道享上帝之意也。皇祐、熙寧所用犢與羊、豕,皆未應禮。今親祠上帝、配帝、五方帝、五人帝,請用牛、羊、豕各一。」太常
 禮院言:「今歲明堂,尚在慈聖光獻皇后三年之內,請如熙寧元年南郊故事,惟祀事用樂,鹵簿鼓吹、宮架、諸軍音樂皆備而不作,警場止鳴金鉦、鼓角而已。」自是,凡國有故皆用此制。



 六月,詔曰:「歷代以來,合宮所配,雜以先儒六天之說,朕甚不取。將來祀英宗皇帝於明堂,惟以配上帝,餘從祀群神悉罷。」詳定所言:「按《周禮》有稱昊天上帝,有稱上帝,有稱五帝者,一帝而已。將來祀英宗於明堂,合配昊天上帝及五帝,欲以此修入儀注。」並據知
 太常禮院趙君錫等狀:「按《周官》掌次職曰:『王大旅上帝,則張氈案;祀五帝,則設大次、小次。』又司服職曰:『祀昊天上帝則服大裘而冕,祀五帝亦如之。』明上帝與五帝異。則宗祀文王以配上帝者,非可兼五帝也。自鄭氏之學興,乃有六天之說,而事非經見。晉泰始初,論者始以為非,遂於明堂惟設昊天上帝一坐而已。唐《顯慶禮》亦然。請如詔祀英宗於明堂,惟配上帝,以稱嚴父之意。」又請:「以莞席代稿秸、蒲越,以玉爵代匏爵,其豆、登、簋、俎、尊、罍
 並用宗廟之器,第以不稞,不用彞瓚。罷爟火及設褥,上帝席以稿秸,配帝席以蒲越,皆加褥其上。飲福受胙,俟終三獻。」並從之。



 監察御史裏行王祖道言:「前詔以六天之說為非古,今復欲兼祀五帝,是亦六天也。禮官欲去四圭而廢祀神之玉,殊失事天之禮。望復舉前詔,以正萬世之失。」仍並詔詳定合用圭、璧。詳定所言:「宋朝祀天禮以蒼璧,則燎玉亦用蒼璧;禮神以四圭有邸,則燎玉亦用四圭有邸。而議者欲以蒼璧禮神,以四圭有邸從
 燎,義無所主。《開寶》、《開元禮》,祀昊天上帝及五帝於明堂,禮神燔燎皆用四圭有邸。今詔唯祀上帝,則四圭有邸,自不當設。宜如南郊,禮神燔燎皆用蒼璧。」又請:「宿齋於文德殿,祭之旦,服通天冠、絳紗袍,至大次,改祭服行事,如郊廟之禮。」



 先是,三省言:「按天聖五年南郊故事,禮畢行勞酒之禮,如元會儀。今明堂禮畢,請太皇太后御會慶殿,皇帝於簾內行恭謝禮,百僚稱賀訖,升殿賜酒。」太皇太后不許,詔將來明堂禮畢,更不受賀,百官並於內
 東門拜表。九月辛巳,大享於明堂。禮畢,詣景靈宮及諸寺觀行恭謝禮。元符元年,尚書左丞蔡卞言:「每歲大享明堂,即南郊望祭殿行禮,制度隘窄,未足以仰稱嚴事之意。今新作南郊齋宮端誠殿,實天子潔齋奉祠及見群臣之所,高明邃深,可以享神,即此行禮,於義為合。」



 初,元豐禮官以明堂寓大慶路寢,別請建立以盡嚴奉,而未暇講求。至是蔡京為相,始以庫部員外郎姚舜仁《明堂圖議》上,詔依所定營建。明年正月,以彗出西方,罷。大
 觀元年九月辛亥,大享於明堂,猶寓大慶殿。



 政和五年,詔:「宗祀明堂以配上帝,寓於寢殿,禮蓋雲闕。崇寧之初,嘗詔建立,去古既遠,歷代之模無足循襲。朕刺經稽古,度以九筵,分其五室,通以八風,上圓下方,參合先王之制。相方視址,於寢之南,僝工鳩材,自我作古,以稱朕昭事上帝率見昭考之心。」既又以言者「明堂基宜正臨丙方近東,以據福德之地」,乃徙秘書省宣德門東,以其地為明堂。



 又詔:「明堂之制,朕取《考工》互見之文,得其制作
 之本。夏后氏曰世室,堂修二七,廣四修一,五室三四步四三尺,九階,四旁兩夾窗。考夏后氏之制,名曰世室,又曰堂者,則世室非廟堂。修二七,廣四修一,則度以六尺之步,其堂修十四步,廣十七步之半。又曰五室三四步四三尺者,四步益四尺,中央土室也,三步益三尺,木、火、金、水四室也。每室四戶,戶兩夾窗,此夏制也。商人重屋,堂修七尋,崇三尺,四阿重屋,而又曰堂者,非寢也。度以八尺之尋,其堂修七尋。又曰四阿重屋,阿者屋之曲也,重
 者屋之復也,則商人有四隅之阿,四柱復屋,則知下方也。周人明堂度以九尺之筵。三代之制不相襲,夏曰世室,商曰重屋,周曰明堂,則知皆室也。東西九筵,南北七筵,堂崇一筵,五室,凡室二筵者,九筵則東西長,七筵則南北狹,所以象天,則知上圜也。名不相襲,其制則一,唯步、尋、筵廣狹不同而已。朕益世室之度,兼四阿重屋之制,度以九尺之筵,上圜象天,下方法地,四戶以合四序,八窗以應八節,五室以象五行,十二堂以聽十二朔。九
 階、四阿,每室四戶,夾以八窗。享帝嚴父,聽朔布政於一堂之上,於古皆合,其制大備。宜令明堂使司遵圖建立。」



 於是內出圖式,宣示於崇政殿,命蔡京為明堂使,開局興工,日役萬人。京言:「三代之制,修廣不相襲,夏度以六尺之步,商度以八尺之尋,而周以九尺之筵,世每近,制每廣。今若以二筵為太室,方一丈八尺,則室中設版位、禮器已不可容,理當增廣。今從周制,以九尺之筵為度,太室修四筵,三丈六尺。



 廣五筵,四丈五尺。



 共為九筵。木、火、金、水四
 室各修三筵,益四五,三丈一尺五寸。



 廣四筵,三丈六尺。



 共七筵,益四尺五寸。十二堂古無修廣之數,今亦廣以九尺之筵。明堂、玄堂各修四筵,三丈六尺。



 廣五筵,四丈五尺。



 左右個各修廣四筵。三丈六尺。



 青陽、總章各修廣四筵,三丈六尺。



 左右個各修四筵,三丈六尺。廣三筵,益四五。三丈一尺五寸。



 四阿各四筵,三丈六尺。



 堂柱外基各一筵,九尺。



 堂總修一十九筵,一十七丈一尺。廣二十一筵。一十八丈九尺。」



 蔡攸言:「明堂五門,諸廊結瓦,古無制度,漢、唐或蓋以茅,或蓋以瓦,或以木為瓦,以夾紵漆之。今酌古之制,
 適今之宜,蓋以素瓦,而用琉璃緣裏及頂蓋鴟尾綴飾,上施銅云龍。其地則隨所向甃以五色之石。欄楯柱端以銅為文鹿或闢邪象。明堂設飾,雜以五色,而各以其方所尚之色。八窗、八柱則以青、黃、綠相間。堂室柱門欄楯,並塗以朱。堂階為三級,級崇三尺,共為一筵。庭樹松、梓、檜,門不設戟,殿角皆垂鈴。」詔以「玄堂」犯祖諱,取「平在朔易」之義,改為平朔,門亦如之。仍改敷祐門曰左敷祐,左承天門曰右敷祐,右承天門曰平秩,更衣大次曰齋
 明殿。七年四月,明堂成,有司請頒常視朔聽朝。詔:「明堂專以配帝嚴父,餘悉移於大慶、文德殿。」群臣五表陳請,乃從之。



 禮制局言:「祀天神於冬至,祀地祇於夏至,乃有常日,無所事卜。季秋享帝,以先王配,則有常月而未有常日。禮不卜常祀而卜其日,所謂卜日者,卜其辛爾。蓋月有上辛、次辛,請以吉辛為正。」



 又言:「《周禮》:『祀昊天上帝,則大裘而冕,祀五帝亦如之。享先王則袞冕。』蓋於大裘舉正位以見配位,於袞冕舉配位以見正位,以天道事
 之,則舉卑明尊;大裘象道,袞冕象德,明堂以人道享上帝,請服袞冕。郊祀正位設蒲越,明堂正配位以莞,蓋取《禮記》所謂『莞簟之安』。請明堂正配位並用莞簟。又《周禮》:『以蒼璧禮天。』又曰:『四圭有邸,以祀天,旅上帝。』然說者謂禮神在求神之前,祀神在禮神之後。蓋一祭而並用也。夏祭方澤,兩圭有邸,與黃琮並用。明堂大享,蒼璧及四圭有邸亦宜並用。圜丘、方澤,執玄圭則搢大圭,執大圭則奠玄圭。《禮經》,祀大神祇,享先王,一如明堂親祠,宜如
 上儀。其正配二位,請各用籩二十六,豆二十六,簠八,簋八,登三,鉶三,柶盤、神位席、幣篚,祝篚、玉爵反坫、瑤爵、牛羊豕鼎各一,並局匕、畢茅、冪俎六,大尊、山尊、著尊、犧尊、象尊各二,壺尊六,皆設而弗酌。尊加冪。犧尊、象尊、壺尊、犧罍、象罍、壺罍各五,加勺、冪。御盤匜一,並篚、勺、巾。飲福受黍豆一,以玉飾。飯福受胙俎一。亞獻、終獻盥洗罍、爵洗罍並篚、勺、巾各一,神廚鸞刀一。」



 又言:「明堂用牲而不設庶羞之鼎。按元豐禮,明堂牲牢正配,各用牛一、
 羊一、豕一。宗祀止用三鼎而不設庶羞之鼎,其俎亦止合用六。宗廟祭祀五齊三酒,有設而弗酌者,若酒正所謂『以法共五齊三酒,以實八尊』是也。有設而酌者,若司尊彞所謂『醴齊縮酌,盎齊兌酌,凡酒修酌』是也。今太廟、明堂之用,請以大尊實泛齊,山尊實醴齊,著尊實盎齊,犧尊實緹齊,像尊實沉齊,壺尊實三酒,皆為弗酌之尊。又以犧尊實醴齊為初獻,像尊實盎齊為亞獻,並陳於阼階之上,犧在西,像在東。壺尊實清酒為終獻,陳於阼階之
 下,皆為酌尊。尊三,其貳以備乏匱。明堂雖嚴父,然配天與上帝,所以求天神而禮之,宜同郊祀,用禮天神六變之樂,以天帝為尊焉。皇祐以來,以大慶殿為明堂,奏請致齋於文德殿,禮成,受賀於紫宸殿。今明堂肇建,宜於大慶殿奏請致齋,於文德殿禮成受賀。宿齋奏嚴,本以警備。仁宗詔明堂直端門,故齋夕權罷。今明堂在寢東南,不與端門直,將來宗祀,大慶殿齋宿,皇城外不設鹵簿儀仗,其警場請列於大慶殿門之外。王者祀上帝於
 郊,配以祖,祀於明堂,配以檷。今有司行事,乃寓端誠殿,未盡禮意。請非親祀歲,有司行事,亦於明堂。改儀仗使曰禮衛,鹵簿使曰禮器,橋道頓遞使曰禮頓,大禮、禮儀二使仍舊制。又設季秋大享登歌,並用方士。」



 初,禮部尚書許光凝等議:「明堂五室祀五帝,而王安石以五帝為五精之君,昊天之佐,故分位於五室,與享於明堂。神宗詔唯以英宗配帝,悉去從祀群神。陛下肇新宏規,得其時制,位五帝於五室,既無以檷概配之嫌,止祀五帝,又
 無群神從祀之瀆,則神考絀六天於前,陛下正五室於後,其揆一也。」至是詔罷從祀,而親祠五室焉。尋詔每歲季秋大享,親祠明堂如孟月朝獻禮,罷有司攝事,及五使儀仗等。



 已而太常寺上《明堂儀》:皇帝散齋七日於別殿,致齋三日於內殿,有司設大次於齋明殿,設小次於明堂東階下。祀曰,行事、執事、陪祠官立班殿下,東西相向。皇帝服袞冕,太常卿、東上閣門官、太常博士前導。禮部侍郎奏中嚴外辦,太常卿奏請行禮。太常卿奏禮畢,
 禮部郎中奏解嚴。其禮器、牲牢、酒饌、奠獻、玉幣、升煙、燔首、祭酒、讀冊、飲福、受胙並樂舞等,並如宗祀明堂儀。其行事、執事、陪祠官,並前十日受誓戒於明堂。行事、執事官致齋三日,前一日並服朝服立班省饌,祀日並祭服。陪位官致齋一日。祀前二日仍奏告神宗配侑。自是迄宣和七年,歲皆親祀明堂。



 高宗紹興元年,禮部尚書秦檜等言:「國朝冬祀大禮,神位六百九十,行事官六百七十餘員,今鹵簿、儀仗、祭器、法物散失殆盡,不可悉行。宗
 廟行禮,又不可及天地。明堂之禮,可舉而行,乞詔有司討論以聞。」禮部、御史、太常寺言:「仁宗明堂以大慶殿為之,今乞於常御殿設位行禮。」乃下詔曰:「肇稱吉禮,已見於三歲之郊;載考彞章,當間以九筵之祀。因秋成物,輯古上儀,會天地以同禋,升祖宗而並配。」乃以九月十八日行事。



 四年,太常寺看詳、國子監丞王普言明堂有未合禮者十一事:其一,謂陶匏用於郊丘,玉爵用於明堂,今茲明堂實兼郊禮,宜用陶匏,他日正宗祀之禮,當奉
 玉爵。其二,《禮經》,太牢當以牛、羊、豕為序,今用《我將》之詩,遂以羊、豕、牛為序,所謂以辭害意,豈有用大牲作元祀,而反在羊、豕之後者?其三,陳設尊罍,宜仿《周官》司尊彞秋嘗之制。其四,泛齊醴齊,宜代以今酒而不易其名。其五、其六,祭器、冕服,當從古制。其七,皇帝未後詣齋室,則是致齋二日有半,乞用質明以成三日之禮。其八,齋不飲酒茹葷,乞罷官給酒饌,俾得專心致志,交於神明。其九,設神位版及升煙、奠冊,不當委之散吏。其十、十一,皆
 論樂。並從之。



 三十一年,以欽宗之喪,用元祐故事,皆前期朝獻景靈宮、朝享太廟,皆遣大臣攝事;唯親行大享之禮,禮畢宣赦,樂備不作。附廟畢如故事。享罷合祭,奉徽宗配。祀五天帝、五人帝於堂上,五官神於東廂,仍罷從祀諸神位,用熙寧禮也。



 孝宗淳熙六年,以群臣議,復合祭天地,並侑祖宗、從祀百神,如南郊。十五年九月,有事於明堂,上問宰執配位。周必大奏:「昨已申請,高宗幾筵未除,用徽宗故事未應配坐,且當以太祖、太宗並配。」
 留正亦言之。上曰:「有紹興間典故,可參照無疑。」



 嘉定十七年閏八月,理宗即位,大享當用九月八日,在寧宗梓宮未發之前,下禮官及臺諫、兩省詳議。吏部尚書羅點等言:「本朝每三歲一行郊祀,皇祐以來始講明堂之禮,至今遵行。稽之《禮經》,有『越紼行事』之文,『既殯而祭』之說,則雖未葬以前,可以行事。且紹熙五年九月,在孝宗以日易月釋服之後,未發引之前;慶元六年九月,亦在光宗以日易月釋服之後,未發引之前。今來九月八日,前
 祀十日,皇帝散齋別殿,百官各受誓戒,系在閏八月二十七日,即當在以日易月未釋服之內。乞下太史局,於九月內擇次辛日行禮,則在釋服之後,正與前史相同。」乃用九月二十八日辛卯。前二日,朝獻景靈宮,前一日,享太廟,遣官攝事。皇帝親行大享,禮成不賀。



 淳祐三年,將作少監、權樞密都承旨韓祥言:「竊以明堂之禮,累聖不廢嚴父配侑之典。南渡以來,事頗不同。高廟中興,徽宗北狩,當時合祭天地於明堂,以太祖、太宗配,非廢嚴
 父之祀,以父在故也。及紹興末,乃以徽廟配。孝宗在位二十八年,娛奉堯父,故無祀父之典,南郊、明堂,惟以太祖、太宗配,沿襲至今,遂使陛下追孝寧考之心有所未盡。」時朝散大夫康熙亦援倪思所著合宮嚴父為言。上曰:「三後並侑之說最當。」是後明堂以太祖、太宗、寧宗並侑。寶祐五年九月辛酉,復奉高宗升侑。於是明堂之禮,一祖三宗並配。度宗咸淳五年,明堂大享,又去寧宗,奉理宗與祖宗並配。



 先是,紹興初,權禮部尚書胡直孺等
 言:「國朝配祀,自英宗始配以近考,司馬光、呂誨爭之,以為詘祖進父,然卒不能奪王珪、孫抃之諂辭。其後,神宗謂周公守祀在成王之世,成王以文王為祖,則明堂非以考配明矣。王安石亦對以誤引《孝經》嚴父之說,惜乎當時無有辨正之者。今或者曰:後稷為周之祖,文王、武王是為二祧。高祖為漢之祖,孝文、孝武特崇兩廟。皆子孫世世所奉承者。太祖為帝者祖,太宗、真宗宜為帝者宗。皇祐以一祖二宗並配,議出於此。直孺等聞前漢以
 高祖配天,後漢以光武配明堂,蓋古之帝王非建邦啟土者,皆無配之祭。故雖周之成、康,漢之文、景、明、章,其德業非不美也,然而子孫不敢推以配天者,避祖宗也。有宋肇基創業之君,太祖是已。太祖則周之後稷,配祭於郊者也;太宗則周之文王,配祭於明堂者也。此二祭者,萬世不遷之法。皇祐宗祀,合祭天地,固宜以太祖、太宗配。當時蓋拘於嚴父,故配帝並及於真宗。今主上紹膺大統,自真宗至於神宗均為祖廟,獨躋則患在於無
 名,並配則幾同於袷享。今參酌皇祐詔書,請合祭昊天上帝、皇地祇於明堂,奉太祖、太宗以配,惟禮專而事簡,庶幾可以致力於神,萬世行之可也。」



 七年,徽宗哀聞。是歲九月,中書舍人傅崧卿援嚴父之說,不幸太上諱問奄至,而大享不及,理實未安。吏部尚書孫近等言:「元年以來,祖、宗並配,今論者乃欲祖、宗並配之外,增道君皇帝一位,不合典禮。」權禮部侍郎陳公輔言:「今梓宮未還,廟社未定,疆土未復,臣竊意祖宗、上皇神靈所望於陛
 下者,必欲興衰撥亂,恢復中原,迎還梓宮,歸藏陵寢,以隆我宋無疆之業。若如議者之言,以陛下貴為天子,上皇北狩十有一年,未獲天下之養,今不幸而崩,且欲因明堂之禮,追配上帝,謂是足以盡人子之孝,則於陛下之志,恐亦小矣。宜依故事合祭天地,祖、宗並侑。太上升配,似未可行。」至嘉定四年,遂以太祖、太宗、高宗、寧宗並侑,至度宗,復以太祖、太宗、高宗、理宗並配焉。



\end{pinyinscope}