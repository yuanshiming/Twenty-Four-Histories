\article{列傳第二百二十八宦者四}

\begin{pinyinscope}

 ○邵成章藍珪康履附馮益張去為陳源甘昪王德謙關禮董宋臣



 邵成章,欽宗朝內侍也。帝入青城,命成章衛皇太子赴宣德門稱制行事。太子北去,成章留於汴。康王將即位,元祐太后遣成章奉乘輿、服御至南京,從幸揚州。



 金人掠陜西、京東諸郡,群盜起山東,黃潛善、汪伯彥匿不以聞。及張遇焚真州,去行在六十里,帝亦不之知也。成章上疏條具潛善、伯彥之罪曰,必誤國,且申潛善等使聞之。帝怒,除名,南雄州編管。侍御史馬伸言:「成章緣上書得罪,今是何時,以言為諱?」



 久之,帝思成章忠直,召赴行
 在,其徒忌之,譖於帝曰:「邵九百來,陛下無歡樂矣!」遂止之於洪州。金人入洪,聞其名,訪求得之,謂之曰:「知公忠正,能事吾主,可坐享富貴。」成章不應,脅之以威,亦不從。金人曰:「忠臣也,吾不忍殺。」遺之金帛而去。



 藍珪、康履,初皆為康王府都監、入內東頭供奉官,嘗從康王使金人行營。及開元帥府,並主管機宜文字。朝廷遣人趣師入援,履等請王留相州,王叱之而行。既即位,二人俱恃恩用事,履尤妄作威福,大將如劉光世等多
 曲意事之。帝知之,詔內侍不許與統兵官相見,違者停官編隸。履終無所忌憚,與內侍曾擇凌忽諸將,或踞坐洗足,立諸將於左右,聲喏甚至馬前,故疾之者眾。俄遷內侍省押班、金州觀察使。



 帝在揚州,金兵卒至,帝馳馬出門,百官不戒備,從行者惟履等五六人。自是履等益自衒,愈有輕外朝心。及幸浙,道吳江,其黨競以射鴨為樂。比至杭州,江下觀潮,中官供帳,赫然遮道。統制苗傅等切齒曰:「此輩使天子至此,猶敢爾邪?」傅幕客王世修
 亦疾中官恣橫,以告武功大夫劉正彥,正彥曰:「會當共除之。」王淵躋樞筦,正彥以為由宦者所薦,愈不平,謀遂決。伏兵斬淵,遣兵圍履家,分捕中官,凡無須者皆殺之。



 履馳入白帝,傅等至,厲聲曰:「陛下信任中官,凡中官所主者皆得美官。王淵遇賊不戰,交康履得樞密。中官在外者已誅,更乞康履、藍珪、曾擇等誅之,以謝三軍。」帝不忍,除傅等官以安之。傅等曰:「欲遷官,第須控兩匹馬與內侍,何必至此!」帝問百官:「策安出?」主管浙西機宜文字
 時希孟曰:「中官之為患,至此極矣。不除之,天下之患未已。」軍器監葉宗諤言:「陛下何惜一康履,不以慰三軍?」帝不得已,遣人執履至,履望帝呼曰:「大家何獨殺臣?」遂以付傅,即腰斬之。梟其首。帝幸睿聖宮,傅等留內侍十五人奉左右。尋捕珪、擇等,皆編置遠州;擇,昭州,行一程,追還斬之。



 傅等誅,贈履官,謚榮節,召珪等還。中書舍人季陵言:「中官復召,其黨與相賀,氣焰益張,中外切齒。」不報。珪至,自武功大夫擢內侍省押班。慈寧宮建,命提點事
 務,尋升內侍省都知。及迎太后,命充都大主管。太后既還宮,珪奏應乾補授恩,乞聽慈寧宮施行。從之。珪初與履同進,而驕橫不及履,故幸以壽終。



 有安石者,與同姓,為內侍省副都知,至景福殿使、湖州觀察使。卒,贈保寧軍節度使,謚良恪。渡江後,中官贈謚自安石始。



 又有與履同姓者名住,為內侍省押班,亦親幸用事,與知閣門事藍公佐善,每邀公佐至其直舍,必縱飲大醉,薄莫乃歸,嘗漏洩禁中語。劉光遠被劾,住與內侍陳永錫受
 其金,力為營救。言官劾之,帝詔永錫與祠,住送吏部。後累官至均州觀察使。卒,贈保信軍節度使,謚忠定。



 馮益,康王邸舊人也。王即位,自入內東頭供奉官遷至乾辦御藥院,尋兼乾辦皇城司。恃舊恩驕恣。帝幸浙東,益與御前右軍都統制張俊爭渡,以語侵俊,且訴於帝。事下御史臺,侍御史趙鼎言:「明受之變,起於內侍,覆轍不可不戒。」事乃已。



 紹興三年,授武功大夫、康州防禦使、帶御器械。時帝用侍御史常同言,詔皇城司並隸臺察,
 益言非祖宗舊制,帝為追寢前詔。特遷宣政使。益自言藩邸舊吏,乞加恩,遂升明州觀察使。內廄舊有騏驥院官,益請別置御馬院,自領其事,又擅穿皇城便門。侍御史沈與求以為言,趙鼎等皆患之。



 會劉豫揭榜山東,言益遣人收買飛鴿,因有不遜語。張浚請斬益以釋謗,帝不許。鼎言事關國體,當解職加罰。帝喜曰:「聞益交關外事,漸不可長。」與祠放歸。浚意未息,鼎解之。益自是家居廩祠者十四年。



 先是,偽柔福帝姬之來,自稱為王貴妃
 季女,益自言嘗在貴妃閤,帝遣之驗視,益為所詐,遂以真告。及事覺,益坐驗視不實,送昭州編管,尋以與皇太后連姻得免。十九年,卒於家。



 張去為,內侍張見道養子也。初為韋太后宅提點官,累遷至安德軍承宣使、帶御器械,又遷內侍省押班。時見道為入內內侍省押班,父子並充景福殿使。去為浸有寵,請以一官回授見道,帝嘉而許之。其後見道以保康軍承宣使致仕,而去為與秦檜、王繼先俱用事,升延福
 宮使,累遷至入內內侍省都知,恃恩幹外朝謀議。



 金兵將至,遣使來,出慢言以相懼。去為陰沮用兵,進幸蜀之計,宰相陳康伯力非之,帝悟而止。侍御史杜莘老乞斬去為,以作士氣。先是,去為取御馬院西兵二百人,髡其頂發,都人駭之,莘老復劾其罪。帝不得已,令去為致仕,莘老亦出補外。



 及內禪,詔落致仕,提舉德壽宮,行移如內侍省,仍鑄印賜之。修宮有勞,又特遷安慶軍承宣使。初,安恭後入宮,去為實進之。後崩,上皇又遣去為傳旨,
 立謝貴妃為後,故亦貴重,然至死不復涉朝廷事。



 陳源,淳熙中提舉德壽宮,頗有寵。俄帶浙西副總管,給事中趙汝愚言:「內侍不當干軍政。」遂罷。源恃恩顓恣,本宮書史徐彥通者為源掌家務,不數歲,官至經武大夫;甄士昌,源廝役也,工理發,奏補承信郎;又補臨安府都吏李庚以官,使之窺伺府事。孝宗聞而惡之。十年春,詔源應奉日久,特落階官,與京祠。給事中宇文價封還錄黃,改外祠。臺官黃洽等又劾之,乃謫源建州居住,籍其
 貲進德壽宮。彥通除名、道州編管,士昌、庚皆抵罪。言者猶未已,移源郴州。源有園名小隱,其制視禁籞有加,高宗以賜王才人。



 光宗即位,復召還。紹熙四年,自拱衛大夫、永州防禦使除入內內侍省押班。帝以疾不朝重華宮,源與內侍楊舜卿、林億年數有間言。寧宗即位,命三人俱事光宗於泰安宮。御史章穎論其離間君親,乞行誅竄,以慰壽皇在天之靈。詔罷源等官,源撫州、億年常州居住,舜卿任便居住。慶元二年,以生皇子恩,源、億年
 許自便,舜卿與內祠。給事中汪義端駁之,乃移源婺州,億年湖州。義端再駁舜卿內祠,反坐外補,其後源等卒聽自便。億年養娼女以別業,源在貶所與妓濫,俱以淫媟聞,人疑其非宦者云。



 甘昪,內侍省押班澤之子。澤之死,昪累遷亦至押班。乾道中,帝頗親昪,昪以此用事。臨安尹胡與可為小官時,丐貸於臨安富民馬氏,不如欲,銜之。至是,馬以鬻官鹽逾格系獄,與可諷有司以私鹽論,御史陳升卿決獄,平
 反之。昪之子婦,與可女也,乃陰為與可地,譖升卿於帝前,謂為豪民馬請事,所得至萬緡。上疑,遂論罪,馬流嚴州,升卿由是罷去。



 時曾覿以使弼領京祠,王抃以知閤門兼樞密都承旨,昪為入內押班,相與盤結,士大夫無恥者爭附之。既而覿死抃逐,獨昪在,朱熹力言之,帝曰:「昪乃德壽宮所薦,謂有才耳。」熹曰:「奸人無才,何以動人主?」昪用事二十年,招權市賄,黃由對策,亦頗及之。後帝察其奸,遂抵之罪,籍其貲,竟以廢死。



 弟昺,淳熙末,乾辦
 內東門司、帶御器械。光宗朝,累遷至親衛大夫、保康軍承宣使、提舉祐神觀。慶元初,為內侍省都知。帝過壽康宮,昺有力焉。遷官二秩,頗貴寵。



 王德謙,初為嘉邸都監,頗親幸。孝宗大漸,光宗以疾久不朝重華宮。黃由時為王府贊讀,奏請嘉王詣重華宮問疾,既得旨,德謙固請覆奏,王斥之,遂行。孝宗崩,王在喪次,中外洶洶,王以告直講彭龜年。龜年以為建儲則人心安,須白中宮乃可。即諭德謙奏之皇太后,德謙不
 敢,強之,既而無報。



 王即位,德謙累遷昭慶軍承宣使、內侍省押班,賜居第。驕恣逾法,服食擬乘輿,出入或以導駕燈籠自奉。為人求官,贓以巨萬計,洩其事者禍立至,故外朝多附之。



 中書舍人吳宗旦事之尤謹,夜則易服造謁。德謙求為節度使,先薦宗旦為刑部侍郎、直學士院,將使草麻。宗旦先備草示之,引天寶、同光為比,德謙喜。制出,參政何澹不肯署,諫議大夫劉德秀率臺諫論列,宰相京鏜復以為言,命遂寢。



 韓侂胄與德謙爭用事,
 德謙屢以計勝,侂胄擠之,詔與外祠,臺諫又交章論駁。侍御史姚愈言吳宗旦嘗草德謙制,遂罷其官。愈又率同列力攻德謙,詔送廣德軍居住。尋以臨安尹劾其贓濫僭擬,詔降團練使、移居撫州,他事勿問。中書舍人高文虎請改為安置,臺諫復言其奸詭,乞自今不以赦移,雖特旨亦許執奏,帝用其言,德謙遂坐廢斥以死。



 關禮,高宗朝宦者。淳熙末,積官至親衛大夫、保信軍承宣使。孝宗頗親信之,後命提舉重華宮。



 孝宗崩,光宗
 疾,不能執喪,樞密趙汝愚等請建儲以安人心,光宗御批又有「念欲退閑」語,丞相留正懼,納祿去,人心愈搖。汝愚遣戚裡韓侂胄因內侍張宗尹以禪位之議奏,太皇太后曰:「此豈可易言!」明日,汝愚再遣侂胄附宗尹以奏,未獲命而侂胄退,與禮遇,禮知其意,問之,侂胄不以告。禮指天自誓不言,侂胄遂白其事,禮即入宮,泣告太后以時事可憂之狀,且曰:「留丞相已去,所恃者趙知院耳。今欲定大計而無太皇太后之命,亦將去矣。」太后驚曰:「知
 院,同姓也,事體與他人異。」禮曰:「知院未去,恃有太后耳。今有請不許,計無所出,亦惟有去而已。知院去,天下將若何?」太后悟,遂命禮傳旨侂胄以諭汝愚,約明日太后垂簾上其事。又明日,嘉王入行禫祭,汝愚即簾前進呈御批,太后遂命王即皇帝位。尋除禮入內內侍省都知,又差兼重華、慈福宮承受,充提舉皇城司,遷中侍大夫。



 禮不以功自居,乞致仕,不許;乞免推恩,又不許。南渡後,內侍可稱者惟邵成章與禮云。



 董宋臣,理宗朝宦者。淳祐中,以睿思殿祗候特轉橫行官。寶祐三年,兼乾辦祐聖觀。侍御史洪天錫劾之,不報,天錫坐左遷大理少卿。開慶初,大元兵駐江上,京師大震。宋臣贊帝遷幸寧海軍,簽判文天祥上疏乞誅宋臣,又不報。



 景定四年,自保康軍承宣使除入內內侍省押班,尋兼主管太廟、往來國信所,同提點內軍器庫、翰林院、編修敕令所、都大提舉諸司,提點顯應觀,主管景獻太子府事。會天祥以著作佐郎兼獻景府教授,義不與
 宋臣聯事,上書求去,天祥出知瑞州。



 言者論宋臣不置,帝曲為諭解庇之。秘書少監湯漢上封事,亦言:「宋臣十餘年來聲焰薰灼,其力能去臺諫、排大臣,至結兇渠以致大禍。中外惶惑切齒,而陛下方為之辨明,大臣方為之和解,此過計也。願收還押班等除命,不勝宗社之幸。」疏入,帝亦不之省。六月,命主管御前馬院及酒庫。既卒,帝猶命特轉節度使,其見寵愛如此。



\end{pinyinscope}