\article{列傳第一百五十一}

\begin{pinyinscope}

 ○趙汝愚子崇憲



 趙汝愚,字子直,漢恭憲王元佐七世孫,居饒之餘干縣。父善應,字彥遠,官終修武郎、江西兵馬都監。性純孝,親病,嘗刺血和藥以進。母畏雷,每聞雷則披衣走其所。嘗
 寒夜遠歸,從者將扣門,遽止之曰:「無恐吾母。」露坐達明,門啟而後入。家貧,諸弟未制衣不敢制,已制未服不敢服,一瓜果之微必相待共嘗之。母喪,哭泣嘔血,毀瘠骨立,終日俯首柩傍,聞雷猶起,側立垂涕。既終喪,言及其親,未嘗不揮涕,生朝必哭於廟。父終肺疾,每膳不忍以諸肺為羞。母生歲值卯,謂卯兔神也,終其身不食兔。聞四方水旱,輒憂形於色。江、淮警報至,為之流涕,不食累日;同僚會宴,善應悵然曰:「此寧諸君樂飲時耶!」眾為失
 色而罷。故人之孤女,貧無所歸,善應聘以為己子婦。有嘗同僚者死不克葬,子傭食他所,善應馳往哭之,歸其子而予之貲,使葬焉。道見病者必收恤之,躬為煮藥。歲饑,旦夕率其家人輟食之半,以飼饑者。夏不去草,冬不破壞,懼百蟲之游且蟄者失其所也。晉陵尤袤稱之曰:「古君子也。」既卒,丞相陳俊卿題其墓碣曰:「宋篤行趙公彥遠之墓。」



 汝愚早有大志,每曰:「丈夫得汗青一幅紙,始不負此生。」擢進士第一,簽書寧國軍節度判官,召試
 館職,除秘書省正字。孝宗方銳意恢復,始見,即陳自治之策,孝宗稱善,遷校書郎。知閣門張說擢簽書樞密院事,汝愚不往見,率同列請祠,未報。會祖母訃至,即日歸,因自劾,上不加罪。



 遷著作郎、知信州,易臺州,除江西轉運判官,入為吏部郎兼太子侍講。遷秘書少監兼權給事中。內侍陳源有寵於德壽宮,添差浙西副總管。汝愚言:「祖宗以童貫典兵,卒開邊釁,源不宜使居總戎之任。」孝宗喜,詔自今內侍不得兼兵職。舊制,密院文書皆經門
 下省,張說在西府,托言邊機不宜洩。汝愚謂:「東西二府朝廷治亂所關,中書庶政無一不由東省,何密院不然?」孝宗命如舊制。



 權吏部侍郎兼太子右庶子,論知閣王抃招權預政,出抃外祠。以集英殿修撰帥福建,陛辭,言國事之大者四,其一謂:「吳氏四世專蜀兵,非國家之利,請及今以漸抑之。」進直學士、制置四川兼知成都府。諸羌蠻相挻為邊患,汝愚至,悉以計分其勢。孝宗謂其有文武威風,召還。光宗受禪,趣召未至,殿中侍御史範處
 義論其稽命,除知潭州,辭,改太平州。進敷文閣學士,知福州。



 紹熙二年,召為吏部尚書。先是,高宗以宮人黃氏侍光宗於東宮,及即位為貴妃,後李氏意不能平。是年冬十一月郊,有司已戒而風雨暴至,光宗震懼,及齋宿青城,貴妃暴薨,駕還,聞之恚,是夕疾作。內侍馳白孝宗,孝宗倉卒至南內,問所以致疾之由,不免有所戒責。及光宗疾稍平,汝愚入對。上常以五日一朝孝宗於重華宮,至是往往以傳旨免,至會慶節上壽,駕不出,冬至朝
 賀又不出,都人以為憂。汝愚往復規諫,上意乃悟。汝愚又屬嗣秀王伯圭調護,於是兩宮之情通。光宗及後俱詣北內,從容竟日。



 四年,汝愚知貢舉,與監察御史汪義端有違言。汝愚除同知樞密院事,義端言祖宗之法,宗室不為執政,詆汝愚植黨沽名,疏上,不納。又論臺諫、給舍陰附汝愚,一切緘默,不報。論汝愚發策譏訕祖宗,又不報。汝愚力辭,上為徙義端軍器監。給事中黃裳言:「汝愚事親孝,事君忠,居官廉,憂國愛民,出於天性。義端實
 忌賢,不可以不黜。」上乃黜義端補郡,汝愚不獲已拜命。未幾,遷知樞密院事,辭不拜,有旨趣受告。汝愚對曰:「臣非敢久辭。臣嘗論朝廷數事,其言未見用,今陛下過重華,留正復相,天下幸甚。惟武興未除帥,臣心不敢安。」上遂以張詔代領武興軍,汝愚乃受命。



 光宗之疾生於疑畏,其未過宮也,汝愚數從容進諫,光宗出聞其語輒悟,入輒復疑。五年春,孝宗不豫,夏五月,疾日臻。光宗御後殿,丞相率同列入,請上詣重華宮侍疾,從臣、臺諫繼入,
 閣門吏以故事止之,不退。光宗益疑,起入內。越二日,宰相又請對,光宗令知閣門事韓侂胄傳旨云:「宰執並出。」於是俱至浙江亭俟命。孝宗聞之憂甚,嗣秀王簡丞相傳孝宗意,令宰執復入。侂胄奏曰:「昨傳旨令宰執出殿門,今乃出都門。」請自往宣押,汝愚等乃還第。



 六月丁酉,夜五鼓,重華大閹扣宰執私第,報孝宗崩,中書以聞,汝愚恐上疑,或不出視朝,持其札不上。次日,上視朝,汝愚以提舉重華宮關禮狀進,上乃許過北內,至日昃不出,
 宰相率百官詣重華宮發喪。壬寅,將成服,留正與汝愚議,介少傅吳琚請憲聖太后垂簾暫主喪事,憲聖不許。正等附奏曰:「臣等連日造南內請對,不獲。累上疏,不得報。今當率百恭官請,若皇帝不出,百官相與慟哭於宮門,恐人情騷動,為社稷憂。乞太皇太后降旨,以皇帝有疾,蹔就宮中成服。然喪不可無主,祝文稱『孝子嗣皇帝』,宰臣不敢代行。太皇太后,壽皇之母也,請攝行祭禮。」蓋是時正、汝愚之請垂簾也,以國本系乎嘉王,欲因簾前
 奏陳宗社之計,使命出簾幃之間,事行廟堂之上,則體正言順,可無後艱。而吳琚素畏慎,且以後戚不欲與聞大計,此議竟格。



 丁未,宰臣已下,待對和寧門,不報,乃入奏云:「皇子嘉王仁孝夙成,宜早正儲位以安人心。」又不報。越六日再請,御批云:「甚好。」明日,同擬旨以進,乞上親批付學士院降詔。是夕,御批付丞相云:「歷事歲久,念欲退閑。」留正見之懼,因朝臨佯僕於庭,密為去計。汝愚自度不得辭其責,念故事須坐甲以戒不虞,而殿帥郭杲
 莫有以腹心語者。



 會工部尚書趙彥逾至私第,語及國事,汝愚泣,彥逾亦泣,汝愚因微及與子意,彥逾喜。汝愚知彥逾善杲,因繆曰:「郭杲儻不同,奈何?」彥逾曰:「某當任之。」約明乃復命。汝愚曰:「此大事已出諸口,豈容有所俟乎?」汝愚不敢入私室,退坐屏後,以待彥逾之至。有頃,彥逾至,議遂定。明日,正以五更肩輿出城去,人心益搖,汝愚處之恬然。自吳琚之議不諧,汝愚與徐誼、葉適謀可以白意於慈福宮者,乃遣韓侂胄以內禪之意請於憲
 聖。侂胄因所善內侍張宗尹以奏,不獲命,明日往,又不獲命。侂胄逡巡將退,重華宮提舉關禮見而問之,侂胄具述汝愚意。禮令少俟,入見憲聖而泣。憲聖問故,禮曰:「聖人讀書萬卷,亦嘗見有如此時而保無亂者乎?」憲聖曰:「此非汝所知。」禮曰:「此事人人知之,今丞相已去,所賴者趙知院,旦夕亦去矣。」言與淚俱。憲聖驚曰:「知院同姓,事體與他人異,乃亦去乎?」禮曰:「知院未去,非但以同姓故,以太皇太后為可恃耳。今定大計而不獲命,勢不得
 不去。去,將如天下何?願聖人三思。」憲聖問侂胄安在,禮曰:「臣已留其俟命。」憲聖曰:「事順則可,令諭好為之。」禮報侂胄,且云:「來早太皇太后於壽皇梓宮前垂簾引執政。」侂胄復命,汝愚始以其事語陳騤、餘端禮,使郭杲及步帥閻仲夜以兵衛南北內,禮使其姻黨宣贊舍人傅昌朝密制黃袍。



 是日,嘉王謁告不入臨,汝愚曰:「禫祭重事,王不可不出。」翌日,示覃祭,群臣入,王亦入。汝愚率百官詣大行前,憲聖垂簾,汝愚率同列再拜,奏:「皇帝疾,未能執喪,
 臣等乞立皇子嘉王為太子,以系人心。皇帝批出有『甚好』二字,繼有『念欲退閑』之語,取太皇太后處分。」憲聖曰:「既有御筆,相公當奉行。」汝愚曰:「茲事重大,播之天下,書之史冊,須議一指揮。」憲聖允諾。汝愚袖出所擬太皇太后指揮以進,云:「皇帝以疾至今未能執喪,曾有御筆,欲自退閑。皇子嘉王擴可即皇帝位,尊皇帝為太上皇帝,皇后為太上皇后。」憲聖覽畢曰:「甚善。」汝愚奏:「自今臣等有合奏事,當取嗣君處分。然恐兩宮父子間有難處者,
 須煩太皇太后主張。」又奏:「上皇疾未平,驟聞此事,不無驚疑,乞令都知楊舜卿提舉本宮,任其責。」遂召舜卿至簾前,面喻之。憲聖乃命皇子即位,皇子固辭曰:「恐負不孝名。」汝愚奏:「天子當以安社稷、定國家為孝。今中外人人憂亂,萬一變生,置太上皇何地?」眾扶入素幄,披黃袍,方卻立未坐,汝愚率同列再拜。寧宗詣幾筵殿,哭盡哀。須臾,立仗訖,催百官班。帝衰服出就重華殿東廡素幄立,內侍扶掖乃坐。百官起居訖,行禫祭禮。汝愚即喪次,
 召還留正長百僚,命朱熹待制經筵,悉收召士君子之在外者。侍御史張叔椿請議正棄國之罰,汝愚為遷叔椿官。



 是月,上命汝愚兼權參知政事。留正至,汝愚乞免兼職,乃除特進、右丞相。汝愚辭不拜,曰:「同姓之卿,不幸處君臣之變,敢言功乎?」乃命以特進為樞密使,汝愚又辭特進。孝宗將欑,汝愚議欑宮非永制,欲改卜山陵,與留正議不合。侂胄因而間之,出正判建康,命汝愚為光祿大夫、右丞相。汝愚力辭至再三,不許。汝愚本倚正共
 事,怒侂胄不以告,及來謁,故不見,侂胄慚忿。簽書樞密羅點曰:「公誤矣。」汝愚亦悟,復見之。侂胄終不懌,自以有定策功,且依托肺腑,出入宮掖,居中用事。朱熹進對,以為言,又約吏部侍郎彭龜年同劾之,未果。熹白汝愚,當以厚賞酬勞,勿使預政,而汝愚謂其易制不為慮。



 右正言黃度欲論侂胄,謀洩,以內批斥去。熹因講畢,奏疏極言:「陛下即位未能旬月,而進退宰執,移易臺諫,皆出陛下之獨斷,大臣不與謀,給舍不及議。此弊不革,臣恐名
 為獨斷,而主威不免於下移。」疏入,遽出內批,除熹宮觀。汝愚袖批還上,且諫且拜,侂胄必欲出之,汝愚退求去,不許。吏部侍郎彭龜年力陳侂胄竊弄威福,為中外所附,不去必貽患。又奏:「近日逐朱熹太暴,故欲陛下亦亟去此小人。」既而內批龜年與郡,侂胄勢益張。



 侂胄恃功,為汝愚所抑,日夜謀引其黨為臺諫,以擯汝愚。汝愚為人疏,不虞其奸。趙彥逾以嘗達意於郭杲,事定,冀汝愚引與同列,至是除四川制置,意不愜,與侂胄合謀。陛辭
 日,盡疏當時賢者姓名,指為汝愚之黨,上意不能無疑。汝愚請令近臣舉御史,侂胄密諭中司,令薦所厚大理寺簿劉德秀,內批擢德秀為察官,其黨牽聯以進,言路遂皆侂胄之人。會黃裳、羅點卒,侂胄又擢其黨京鏜代點,汝愚始孤,天子益無所倚信。於是中書舍人陳傅良、監察御史吳獵、起居郎劉光祖各先後斥去,群憸和附,疾正士如仇讎,而衣冠之禍始矣。



 侂胄欲逐汝愚而難其名,或教之曰:「彼宗姓,誣以謀危社稷,則一綱無遺。」侂
 胄然之,擢其黨將作監李沐為正言。沐,彥穎之子也,嘗求節度使於汝愚不得,奏:「汝愚以同姓居相位,將不利於社稷,乞罷其政。」汝愚出浙江亭待罪,遂罷右相,除觀文殿學士、知福州。臺臣合詞乞寢出守之命,遂以大學士提舉洞霄宮。



 國子祭酒李祥言:「去歲國遭大戚,中外洶洶,留正棄相位而去,官僚幾欲解散,軍民皆將為亂,兩宮隔絕,國喪無主。汝愚以樞臣獨不避殞身滅族之禍,奉太皇太后命,翊陛下以登九五,勛勞著於社稷,精
 忠貫於天地,乃卒受黯黮而去,天下後世其謂何?」博士楊簡亦以為言。李沐劾祥、簡,罷之。太府丞呂祖儉亦上書訴汝愚之忠,詔祖儉朋比罔上,送韶州安置。太學生楊宏中、周端朝、張衜、林仲麟、蔣傅、徐範等伏闕言:「去歲人情驚疑,變在朝夕。當時假非汝愚出死力,定大議,雖百李沐,罔知攸濟。當國家多難,汝愚位樞府,本兵柄,指揮操縱,何向不可,不以此時為利,今上下安恬,乃獨有異志乎?」書上,悉送五百里外羈管。



 侂胄忌汝愚益深,謂
 不重貶,人言不已。以中丞何澹疏,落大觀文。監察御史胡紘疏汝愚唱引偽徒,謀為不軌,乘龍授鼎,假夢為符。責寧遠軍節度副使,永州安置。初,汝愚嘗夢孝宗授以湯鼎,背負白龍升天,後翼寧宗以素服登大寶,蓋其驗也,而讒者以為言。時汪義端行詞,用漢誅劉屈氂、唐戮李林甫事,示欲殺之意。迪功郎趙師召亦上書乞斬汝愚。汝愚怡然就道,謂諸子曰:「觀侂胄之意,必欲殺我,我死,汝曹尚可免也。」至衡州病作,為守臣錢鍪所窘,暴薨,
 天下聞而冤之,時慶元二年正月壬午也。



 汝愚學務有用,常以司馬光、富弼、韓琦、範仲淹自期。凡平昔所聞於師友,如張栻、朱熹、呂祖謙、汪應辰、王十朋、胡銓、李燾、林光朝之言,欲次第行之,未果。所著詩文十五卷、《太祖實錄舉要》若干卷、《類宋朝諸臣奏議》三百卷。汝愚聚族而居,門內三千指,所得廩給悉分與之,菜羹疏食,恩意均洽,人無間言。自奉養甚薄,為夕郎時,大冬衣布裘,至為相亦然。



 汝愚既歿,黨禁浸解,旋復資政殿學士、太中大
 夫,已而贈少保。侂胄誅,盡復元官,賜謚忠定,贈太師,追封沂國公。理宗詔配享寧宗廟庭,追封福王,其後進封周王。子九人,崇憲其長子也。



 崇憲,字履常,淳熙八年以取應對策第一,時汝愚侍立殿上,降,再拜以謝。孝宗顧近臣曰:「汝愚年幾何?已有子如此。」越三年,復以進士對策,擢甲科。上謂執政曰:「此汝愚子,豈即前科取應第一人者耶?」



 崇憲初仕為保義郎、監饒州贍軍酒庫,換從事郎、撫州軍事推官。汝愚帥蜀,
 闢書寫機宜文字,改江西轉運司干辦公事,監西京中嶽廟。汝愚既貶死,海內憤鬱,崇憲闔門自處。居數年,復汝愚故官職,多勸以仕。



 改奉議郎、知南昌縣事,奉行荒政,所活甚眾。升籍田令,制曰:「爾先人有功王室,中更讒毀,思其功而錄其子,國之典也。」崇憲拜命感泣,陳疏力辭,以為「先臣之冤未悉昭白,而其孤先被寵光,非公朝所以勸忠孝、厲廉恥之意。」俄改監行在都進奏院,復引陳瓘論司馬光、呂公著復官事申言之,乞以所陳下三
 省集議:「若先臣心跡有一如言者所論,即近日恩典皆為冒濫,先臣復官賜謚,與臣新命,俱合追寢。如公論果謂誣蔑,乞昭示中外,使先臣之讒謗既辨,忠節自明,而憲聖慈烈皇后擁祐之功德益顯。然後申飭史官、改正誣史,垂萬世之公。」



 又請正趙師召妄貢封章之罪,究蔡璉與大臣為仇之奸,毀龔頤正《續稽古錄》之妄。詔兩省史官考訂以聞。已而吏部尚書兼修國史樓鑰等請施行如章,從之。及誣史未正,復進言,其略謂:「前日史官徒
 以權臣風旨,刊舊史、焚元稿,略無留難。今詔旨再三,莫有慨然奮直筆者,何小人敢於為惡,而謂之君子者顧不能勇於為善耶?」聞者愧之。其後玉牒、日厲所卒以《重修龍飛事實》進呈,因崇憲請也。



 未幾,贈汝愚太師,封沂國公,擢崇憲軍器監丞,改太府監丞,遷秘書郎,辭,弗許。尋為著作佐郎兼權考功郎官。嘗因閔雨求言,乃上封事,謂:「今日有更化之名,無更化之實。人才,國之元氣,而忠鯁擯廢之士,死者未盡省錄,存者未悉褒揚。言論,國
 之風採,其間輸忠亡隱,有所規益者,豈惟獎激弗加,蓋亦罕見施用;偷安取容,無所建明者,豈惟黜罰弗及,或乃遂階通顯。」至若勉聖學以廣聰明,教儲貳以固根本,戒宰輔大臣同寅盡瘁以濟艱難,責侍從臺諫思職盡規以宣壅蔽,防左右近習竊弄之漸,察奸憸餘黨窺伺之萌,皆懇懇為上言之。



 請外,知江州。郡民歲苦和糴,崇憲疏於朝,永蠲之。且轉糴旁郡穀別廩儲之,以備歲儉。瑞昌民負茶引錢,新舊累積,為緡十七萬有奇,皆困不
 能償,死則以責其子孫猶弗貸。會新券行,視舊價幾倍蓰,崇憲嘆曰:「負茶之民愈困矣。」亟請以新券一償舊券二,詔從之。蓋受賜者千餘家,刻石以紀其事。修陂塘以廣溉灌,凡數千所。提舉江西常平兼權隆興府及帥漕司事,遷轉運判官仍兼帥事。



 初,汝愚捐私錢百餘萬創養濟院,俾四方賓旅之疾病者得藥與食,歲久浸移為它用。崇憲至,尋修復,立規約數十條,以愈疾之多寡為賞罰。棄兒於道者,亦收鞠之。社倉久敝,訪其利害而更
 張之。、



 以兵部郎中召,尋改司封,皆固辭,遂直秘閣、知靜江府、廣西經略安撫。靜江之屬邑十,地肥磽略等,而陽朔、修仁、荔浦之賦獨倍焉。自張栻奏減之餘,人猶以為病。崇憲請再加蠲減,詔遞損有差,三縣民立祠刻石。瓊守非才,激黎峒之變,乃劾去之,改闢能者代其任。蘿蔓峒者仍歲寇鈔為暴,實民何向父子陰誘導之。崇憲捐金繒付小校使系以來,置之法。因嚴民夷交通之禁,使邊民相什伍,寇至則鳴鼓召眾,先後掩擊,俘獲者賞,不
 至者有懲。先是,部內郡邑有警,輒移統府兵戍之,在宜州者百人,古縣半之。崇憲謂根本單虛,非所以窒奸萌,乃於其地各置兵如戍兵之數,而斂戍者以歸。邕為邊要害地,自狄青平儂智高,所以設韓捍防者甚至,歲久浸弛,而溪峒日強。崇憲條上其議,朝廷頗採其言,然未及盡用也。



 崇憲天性篤孝,居父喪,月餘始食食,小祥始茹果實,終喪不飲酒食肉,比御猶弗入者久之。



 論曰:自昔大臣處危疑之地,而能免於禍難者蓋鮮矣。
 昔者周成王立而幼沖,周公以王室懿親為宰輔,四國流言,而周公不免於居東之憂,非天降風雷之變,以彰周公之德而啟成王之衷,則所謂《金滕》之書,固無因而關於王之耳目,公之心果能以自明乎?公之心能自明,則天意之所以屬於周而綿八百載之丕祚者,實系於茲。不然,周其殆哉!



 趙汝愚,宋之宗臣也,其賢固不及周公,其位與戚又非若周公之尊且暱也。方孝宗崩,光宗疾,大喪無主,中外洶洶,一時大臣有畏難而去者矣。汝
 愚獨能奮不慮身,定大計於頃刻,收召明德之士,以輔寧宗之新政,天下翕然望治,其功可謂盛矣。然不幾時,卒為韓侂胄所構,一斥而遂不復返,天下聞而冤之。於此見天之所以眷宋者不如周,而宋之陵夷馴至於不可為,信非人力之所能也。



 汝愚父以純孝聞,而子崇憲能守家法,所至有惠政,亦可謂世濟其美者已。



\end{pinyinscope}