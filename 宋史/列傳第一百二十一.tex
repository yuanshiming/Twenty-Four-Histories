\article{列傳第一百二十一}

\begin{pinyinscope}

 朱勝非呂頤浩範宗尹範致虛呂好問



 朱勝非,字藏一,蔡州人。崇寧二年,上舍登第。靖康元年,為東道副總管,權應天府,金人攻城,勝非逃去。會韓世
 忠部將楊進破敵,勝非復還視事。逾年,詣濟州謂康王言,南京為藝祖興王之地,請幸之以圖大計。王即位南京。



 建炎改元,試中書舍人兼權直學士院。時方草創,勝非憑敗鼓草制,辭氣嚴重如平時。上疏言:「仁義者,天下之大柄,中國持之,則外夷服而諸夏尊;茍失其柄,則不免四夷交侵之患。國家與契丹結好,百有餘年,一旦乘其亂弱,遠交金人為夾攻計,是中國失其柄,而外侮所由招也。陛下即位,宜壹明正始之道,思其合於仁義者
 行之,不合者置之,則可以攘卻四夷,紹復大業矣。」上嘉之。總制使錢蓋進職,勝非言蓋為陜西制置使棄師誤國,封還貼黃,蓋遂罷。諫官衛膚敏坐論元祐太后兄子徙官,勝非言以外戚故去諫臣,非所以示天下。



 二年,除尚書右丞。時宰執蔭補多濫,勝非奏:「舊制,宰執子弟例不堂除,只就銓注,罷政不以罪,然後推恩。趙普子弟皆作武臣,普再相,長子授莊宅使;范純仁再相,子正平有文行,竟死選調;章惇子援及持皆高科,並為州縣、幕職、
 監當。惟夏竦子安期累作邊帥,授待制、直學士,王安石薦子雱為崇政殿說書,除待制。然安期猶有才幹,雱猶有學問。至蔡京子六人、孫四人,鄭居中、劉正夫子各二人,餘深、王黼、白時中、蔡卞、鄧洵仁洵武子各一人,並列從班。宣和末,諫官疏謂:『尚從竹馬之游,已造荷囊之列。』今不可以不戒。」遷中書侍郎。



 三年,上自鎮江南幸,留勝非經理。未幾,命為控扼使,已而拜宣奉大夫、尚書右僕射兼御營使。故事,命相進三官,勝非特遷五官。會王淵
 簽書樞密院事兼御營司都統制,內侍復用事恣橫,諸將不悅。於是苗傅、劉正彥與其徒王鈞甫、馬柔吉、王世修謀,誣淵結宦官謀反。正彥手斬淵,分捕中官,皆殺之,擁兵至行宮門外。勝非趨樓上,詰專殺之由。上親御樓撫諭,傅、正彥語頗不遜,勝非乃從皇太后出諭旨。傅等請高宗避位,太后抱皇子聽政,太后不可。傅顧勝非曰:「今日正須大臣果決,相公何無一言耶?」勝非還告上曰:「王鈞甫乃傅等腹心,適語臣云:『二將忠有餘,而學不足。』
 此語可為後圖之緒。」於是太后垂簾,高宗退居顯忠寺,號睿聖宮。勝非因請降赦以安傅等。又奏:「母後垂簾,須二臣同對,此承平故事。今日事機有須密奏者,乞許臣僚獨對,而日引傅徒二人上殿,以弭其疑。」太后語上曰:「賴相此人,若汪、黃在位,事已狼籍矣。」



 王鈞甫見勝非,勝非問:「前言二將學不足,如何?」鈞甫曰:「如劉將手殺王淵,軍中亦非之。」勝非因以言撼之曰:「上皇待燕士如骨肉,那無一人效力者乎?人言燕、趙多奇士,徒虛語耳。」鈞甫
 曰:「不可謂燕無人。」勝非曰:「君與馬參議皆燕中名人,嘗獻策滅契丹者。今金人所任,多契丹舊人,若渡江,禍首及君矣。盍早為朝廷協力乎!」鈞甫唯唯。王世修來見,勝非諭之曰:「國家艱難,若等立功之秋也。誠能奮身立事,從官豈難得乎。」世修喜,時往來道軍中情實。擢世修為工部侍郎。



 傅、正彥乞改年號及移蹕建康,勝非以白太后,因議恐盡廢其請,則倉卒變生,乃改元明受。以詔示世修曰:「已從若請矣。」傅等欲挾上幸徽、越,勝非諭之以
 禍福而止。傅聞韓世忠起兵,取其妻子為質。勝非紿傅曰:「今當啟太后召二人慰撫,使報知平江,諸君益安。」傅許諾。勝非喜曰:「二兇真無能為也。」諸將將至,傅等懼,勝非因謂之曰:「勤王之師未進者,使是間自反正耳。不然,下詔率百官六軍請上還宮,公等置身何地乎?」即召學士李邴、張守作百官章及太后手詔。



 四月朔,勝非率百官詣睿聖宮,親掖上乘馬還宮。苗傅請以王世修為參議,勝非曰:「世修已為從官,豈可復從軍?」上既復闢,勝
 非曰:「臣昔遇變,義當即死,偷生至此,欲圖今日之事耳。」乃乞罷政。上問誰可代者,勝非曰:「呂頤浩、張浚。」問孰優,曰:「頤浩練事而暴,浚喜事而疏。」上曰:「浚太年少。」勝非曰:「臣向被召,軍旅錢穀悉付浚,此舉浚實主之。」御史中丞張守論勝非不能預防,致賊猖獗,宜罷。不報。授觀文殿大學士、知洪州,尋除江西安撫大使兼知江州。



 紹興元年,馬進陷江州,侍御史沈與求論九江之陷,由勝非赴鎮太緩。降授中大夫,分司南京,江州居住。二年,呂頤浩
 薦兼侍讀,又薦都督江、淮、荊、浙諸軍事,給事中胡安國、侍御史江躋交章論罷之。頤浩力引其入,再除兼侍讀,尋拜尚書右僕射、同中書門下平章事。丁母憂去,起復右僕射兼知樞密院事,上《吏部七司敕令格式》一百八十卷。



 時員外郎江端友請營宗廟,議者非之,以為國家期於恢復,不常厥居,勝非方主和議,遂白上營宗廟於臨安。徐俯罷參政,勝非薦胡松年。侍御史常同劾松年乃王黼客,勝非徙同左史。莫儔謫曲江,其家蒼頭奴為
 勝非治疽而愈,奴為儔請,得復官。姻家劉式嘗言為兵官獲盜,勝非不以付部用,特旨改官。會久雨,勝非累章乞免,且自論當罷者十一事。魏矼亦劾其罪,遂罷。



 五年,慶詔言戰守四事,起知湖州,引疾歸。勝非與秦檜有隙,檜得政,勝非廢居八年,卒,謚忠靖。



 勝非,張邦昌友婿也。始,邦昌僭位,勝非嘗械其使,及金人過江,勝非請尊禮邦昌,錄其後以謝敵。苗、劉之變,保護聖躬,功居多。既去,力薦張浚。然李綱罷,勝非受黃潛善風旨草制,極言其
 狂妄。再相,忌趙鼎,鼎宣撫川、陜,欲重使名以制吳玠,勝非曰:「元樞出使,豈論此耶?」蓋因事出鼎而輕其權。人以此少之。及著《閑居錄》,亦多其私說云。



 呂頤浩,字符直,其先樂陵人,徙齊州。中進士第。父喪家貧,躬耕以贍老幼。後為密州司戶參軍,以李清臣薦,為邠州教授。除宗子博士,累官入為太府少卿、直龍圖閣、河北轉運副使,升待制徽猷閣、都轉運使。



 伐燕之役,頤浩以轉輸隨種師道至白溝。既得燕山,郭藥師眾二萬,
 契丹軍萬餘,皆仰給縣官,詔以頤浩為燕山府路轉運使。頤浩奏:「開邊極遠,其勢難守,雖窮力竭財,無以善後。」又奏燕山、河北危急五事,願博議久長之策。徽宗怒,命褫職貶官,而領職如故;尋復焉。進徽猷閣直學士。金人入燕,郭藥師劫頤浩與蔡靖等以降。敵退得歸,復以為河北都轉運使,以病辭,提舉崇福宮。



 高宗即位,除知揚州。車駕南幸,頤浩入見,除戶部侍郎兼知揚州,進戶部尚書。劇賊張遇眾數萬屯金山,縱兵焚掠。頤浩單騎與
 韓世忠造其壘,說之以逆順,遇黨釋甲降。進吏部尚書。



 建炎二年,金人逼揚州,車駕南渡鎮江,召從臣問去留。頤浩叩頭願且留此,為江北聲援;不然,敵乘勢渡江,事愈急矣。駕幸錢塘,拜同簽書樞密院事、江淮兩浙制置使,還屯京口。金人去揚州,改江東安撫、制置使兼知江寧府。



 時苗傅、劉正彥為逆,逼高宗避位。頤浩至江寧,奉明受改元詔赦,會監司議,皆莫敢對。頤浩曰:「是必有兵變。」其子抗曰:「主上春秋鼎盛,二帝蒙塵沙漠,日望拯救,
 其肯遽遜位於幼沖乎?灼知兵變無疑也。」頤浩即遣人寓書張浚曰:「時事如此,吾儕可但已乎?」浚亦謂頤浩有威望,能斷大事,書來報起兵狀。頤浩乃與浚及諸將約,會兵討賊。時江寧士民洶懼,頤浩乃檄楊惟忠留屯,以安人心。且恐苗傅等計窮挾帝繇廣德渡江,戒惟忠先為控扼備。俄有旨,召頤浩赴院供職。上言:「今金人乘戰勝之威,群盜有蜂起之勢,興衰撥亂,事屬艱難,豈容皇帝退享安逸?請亟復明闢,以圖恢復。」遂以兵發江寧,舉
 鞭誓眾,士皆感厲#



 將至平江,張浚乘輕舟迓之,相持而泣,咨以大計。頤浩曰:「頤浩曩諫開邊,幾死宦臣之手;承乏漕挽,幾陷腥膻之域。今事不諧,不過赤族,為社稷死,豈不快乎?」浚壯其言。即舟中草檄,進韓世忠為前軍,張俊翼之,劉光世為游擊,頤浩、浚總中軍,光世分軍殿後。頤浩發平江,傅黨托旨請頤浩單騎入朝。頤浩奏:所統將士,忠義所激,可合不可離。傅等恐懼,乃請高宗復闢。師次秀州,頤浩勉勵諸將曰:「今雖反正,而賊猶握兵居
 內。事若不濟,必反以惡名加我,翟義、徐敬業可監也。」次臨平,苗傅等拒戰。頤浩被甲立水次,出入行陣,督世忠等破賊,傅、正彥引兵遁。頤浩等以勤王兵入城,都人夾道聳觀,以手加額。



 朱勝非罷相,以頤浩守尚書右僕射、中書侍郎兼御營使,改同中書門下平章事。車駕幸建康,聞金人復入,召諸將問移蹕之地,頤浩曰:「金人謀以陛下所至為邊面,今當且戰且避,奉陛下於萬全之地,臣願留常、潤死守。」上曰:「朕左右不可以無相。」乃以韓世
 忠守鎮江,劉光世守太平。駕至平江,聞杜充敗績,上曰:「事迫矣,若何?」頤浩遂進航海之策。



 初,建炎御營使本以行幸總齊軍政,而宰相兼領之,遂專兵柄,樞府幾無所預。頤浩在位尤顓恣,趙鼎論其過。四年,移鼎為翰林學士、吏部尚書。鼎辭,且攻頤浩,章十數上,頤浩求去。除鎮南軍節度、開府儀同三司、醴泉觀使,詔以頤浩倡義勤王,故從優禮焉。



 奉化賊將璉乘亂為變,劫頤浩置軍中,高宗以頤浩故,赦而招之。尋除江東安撫、制置大使兼
 知池州。頤浩請兵五萬屯建康等處,又請王□燮、巨師古兵自隸。將之鎮,而李成遣將馬進圍江州。乃駐軍鄱陽,會楊惟忠兵,請與俱趨南康,遣師古救江州。賊眾鏖戰,頤浩、惟忠失利,師古敗奔洪州。頤浩乞濟師討李成,高宗曰:「頤浩奮不顧身,為國討賊,群臣所不及,但輕進,其失也。」詔王□燮以萬人速往策應。頤浩復軍左蠡,又得閣門舍人崔增之眾萬餘,軍勢復振。命□燮、增擊賊,敗之,乘勝至江州,則馬進已陷城矣。朝廷命張俊為招討使,俊
 既至,遂敗馬進。進遁,成以餘眾降劉豫。



 詔以淮南民未復業,須威望大臣措置,以頤浩兼宣撫,領壽春府、徐廬和州、無為軍。招降趙延壽於分寧,得其精銳五千,分隸諸將。張琪自徽犯饒州,有眾五萬。時頤浩自左蠡班師,帳下兵不滿萬人,郡人皇駭。頤浩命其將閻皋、姚端、崔邦弼列陣以待。琪犯皋軍,皋力戰,端、邦弼兩軍夾擊,大破之。拜少保、尚書左僕射、同中書門下平章事兼知樞密院事。



 二年,上自越州還臨安。時桑仲在襄陽,欲進取
 京城,乞朝廷舉兵為聲援。頤浩乃大議出師,而身自督軍北向。高宗諭頤浩、秦檜曰:「頤浩治軍旋,檜理庶務,如種、蠡分職可也。」二人同秉政,檜知頤浩不為公論所與,多引知名士為助,欲傾之而擅朝權。高宗乃下詔以戒朋黨,除頤浩都督江、淮、荊、浙諸軍事,開府鎮江。頤浩闢文武士七十餘人,以神武後軍及御前忠銳崔增、趙延壽二軍從行,百官班送。頤浩次常州,延壽軍叛,劉光世殲其眾;又聞桑仲已死,遂不進,引疾求罷。詔還朝,以知
 紹興府朱勝非同都督諸軍事。



 頤浩既還,欲傾秦檜,乃引勝非為助。給事中胡安國論勝非必誤大計,勝非復知紹興府,尋以醴泉觀使兼侍讀。安國持錄黃不下,頤浩持命檢正諸房文字黃龜年書行。安國以失職求去,罷之。檜上章乞留安國,不報。侍御史江躋、左司諫吳表臣皆以論救安國罷,程瑀、胡世將、劉一止、張燾、林待聘、樓照亦坐論檜黨斥,臺省一空,遂罷檜相。



 頤浩獨秉政,屢請興師復中原,謂:「太祖取天下,兵不過十萬,今有兵
 十六七萬矣。然自金人南牧,莫敢嬰其鋒。比年韓世忠、張俊、陳思恭、張榮屢奏,人有戰心,天將悔禍。又金人以中原付劉豫,三尺童子知其不能立國。願睿斷早定,決策北向。今之精銳皆中原人,恐久而消磨,他日難以舉事。」時盜賊稍息,頤浩請遣使循行郡國,平獄訟,宣德意。李綱宣撫湖南,頤浩言綱縱暴無善狀,請罷諸路宣撫之名,綱止為安撫使。時李光在江東,與頤浩書,言綱有大節,四夷畏服。頤浩稱光結黨,言者因論光,罷之。時方
 審量濫賞,頤浩時有縱舍,右司郎官王岡持不可,曰:「公秉國鈞,不平謂何。」



 頤浩再秉政凡二年,高宗以水旱、地震,下詔罪己求言,頤浩連章待罪。高宗一日謂大臣曰:「國朝四方水旱,無不上聞。近蘇、湖地震,泉州大水,輒不以奏,何也?」侍御史辛炳、殿中常同論其罪,遂罷頤浩為鎮南軍節度使、開府儀同三司、提舉洞霄宮,改特進、觀文殿大學士。五年,詔問宰執以戰守方略,頤浩條十事以獻,除湖南安撫、制置大使兼知潭州。時郴、衡、桂陽盜
 起,頤浩遣人悉平之。帝在建康,除頤浩少保、浙西安撫制置大使、知臨安府、行宮留守。明堂禮成,進封成國公。



 八年,上將還臨安,除少傅、鎮南定江軍節度使、江東安撫制置大使兼知建康府、行宮留守。頤浩引疾求去,除醴泉觀使。九年,金人歸河南地,高宗欲以頤浩往陜西,命中使召赴行在。頤浩以老病辭,且條陜西利害,謂金人無故歸地,其必有意。召趣赴闕,既至,以疾不能見,乃聽歸。未幾,卒,贈太師,封秦國公,謚忠穆。



 頤浩有膽略,善
 鞍馬弓劍,當國步艱難之際,人倚之為重。自江東再相,胡安國以書勸其法韓忠獻,以至公無我為先,報復恩仇為戒,頤浩不能用。時軍用不足,頤浩與朱勝非創立江、浙、湖南諸路大軍月樁錢,於是郡邑多橫賦,大為東南患云。



 範宗尹,字覺民,襄陽鄧城人。少篤學,工文辭。宣和三年,上舍登第。累遷侍御史、右諫議大夫。王云使北還,言金人必欲得三鎮。宗尹請棄之以紓禍,言者非之,宗尹罷
 歸。張邦昌僭位,復其職,遣同路允迪詣康王勸進。



 建炎元年,李綱拜右僕射,宗尹論其名浮於實,有震主之威。不報,出知舒州。言者論宗尹嘗污偽命,責置鄂州。既,召為中書舍人,遷御史中丞,拜參知政事。



 呂頤浩罷相,宗尹攝其位。時諸盜據有州縣,朝廷力不能制。宗尹言:「太祖收藩鎮之權,天下無事百五十年,可謂良法。然國家多難,四方帥守單寡,束手環視,此法之弊。今當稍復藩鎮之法,裂河南、江北數十州之地,付以兵權,俾蕃王室。
 較之棄地夷狄,豈不相遠?」上從其言。授宗尹通議大夫、守尚書右僕射、同中書門下平章事兼御營使,時年三十。近世宰相年少,未有如宗尹者。



 宗尹奏以京畿東西、淮南、湖北地並分為鎮,授諸將,以鎮撫使為名;軍興,聽便宜從事。然李成、薛慶、孔彥舟、桑仲輩起於群盜,翟興、劉位土豪,李彥光、郭仲威皆潰將,多不能守其地。宗尹請有司討論崇、觀以來濫賞,修書、營繕、應奉、開河、免夫、獄空之類,皆厘正之。宣靖執政、圍城、明受偽命之人,反
 用赦申雪;徐秉哲、吳幵、莫儔等並量移;吳敏、王孝迪、耿南仲、孫覿、蔡懋等並敘復。侍郎季陵希宗尹意,乞詔宰執於罪累中選真材實能,量付以事。沉與求劾陵,因及宗尹,宗尹求去。上為罷與求,宗尹乃復視事。



 初,宗尹廷對,詳定官李邦彥特取旨置宗尹乙科,宗尹德之,贈邦彥觀文殿大學士。樞密院副都承旨闕,宗尹擬刑煥、藍公佐、辛道宗三人,煥戚里,公佐管客省,道宗不知兵,人以此咎宗尹。密院計議官王佾結公佐,宗尹請除佾為
 宗正丞,侍御史張延壽劾之,上罷佾。



 紹興元年二月辛巳,日有黑子,宗尹以輔政無狀請免,上不許。魏滂為江東通判,諫官言其貪盜官錢,滂遂罷;李弼孺領營田,諫官言其媚事朱勉,弼孺亦罷:二人皆宗尹所薦。臺州守臣晁公為儲峙豐備,論者以為擾民,宗尹陰祐之。會公為妻受囚金事覺,上罷公為,宗尹不自安。時明堂覃恩,宗尹請舉行討論之事,上手札云:「朕不欲歸過君父,斂怨士大夫。」始,宗尹建此議,秦檜力贊之,及見上意堅,反擠
 宗尹。上亦惡其與辛道宗兄弟往來,遂罷。沉與求奏其罪狀,落職,未幾,命知溫州。退成天臺,卒,年三十七。



 宗尹有才智,當北敵肆行之沖,毅然自任,建議分鎮,以是得相位。然其置帥多授劇盜,又無總率統屬,且不遣援,不通餉,故諸鎮守鮮能久存者。及為政多私,屢為議者所詆云。



 範致虛,字謙叔,建州建陽人。舉進士,為太學博士。鄒浩以言事斥,致虛坐祖送獲罪,停官。徽宗嗣位,召見,除左
 正言,出通判郢州。崇寧初,以右司諫召,道改起居舍人,進中書舍人。蔡京建請置講議司,引致虛為詳定官,議不合,改兵部侍郎。自是入處華要,出典大郡者十五年。以附張商英,貶通州。政和七年,復官,入為侍讀、修國史,尋除刑部尚書、提舉南京鴻慶宮。



 初,致虛在講議司,延康殿學士劉昺嘗乘蔡京怒擠之。後王寀坐妖言系獄,事連昺論死,致虛爭之,昺得減竄,士論賢之。遷尚書右丞,進左丞。



 母喪逾年,起知東平府,改大名府。入見,時朝
 廷欲用師契丹,致虛言邊隙一開,必有意外之患。宰相謂其懷異。致虛乞終喪,從之。免喪,知鄧州,改河南府。中人規景華苑,欲奪故相富弼園宅。致虛言:「弼和戎有大功,使朝廷享百年之安,乃不保數畝之居邪?」弼園宅得不取。復移鄧州、提舉亳州明道宮。帝方好老氏,致虛希時好,營飭道宇,賜名煉真宮。



 靖康元年,召赴闕,道除知京兆府。時金人圍太原,聲震關中,致虛修戰守備甚力。朝廷命錢蓋節制陜西,除致虛陜西宣撫使。金人分道
 再犯京師,詔致虛會兵入援。錢蓋兵十萬至穎昌,聞京師破而遁,西道總管王襄南走。致虛獨與西道副總管孫昭遠合兵,環慶帥臣王似、熙河帥臣王倚以兵來會。致虛合步騎號二十萬,以右武大夫馬昌祐統之,命杜常將民兵萬人趨京師,夏俶將萬人守陵寢。



 兵有僧趙宗印者,喜談兵,席益薦之。致虛以便宜假官,俾充宣撫司參議官兼節制軍馬。致虛以大軍遵陸,宗印以舟師趨西京。金人破京師,遣人持登城不下之詔,以止入援之
 師,致虛斬之。初,金人守潼關,致虛奪之,作長城,起潼關迄龍門,所築僅及肩。宗印又以僧為一軍,號「尊勝隊」,童子行為一軍,號「凈勝隊」。致虛勇而無謀,委己以聽宗印。宗印徒大言,實未嘗知兵。至是,宗印舟師至三門津,致虛使整兵出潼關。金守臣高世由謂其帥粘罕曰:「致虛儒者,不知兵,遣斥候三千,自足殺之。」致虛軍出武關,至鄧州千秋鎮,金將婁宿以精騎沖之,不戰而潰,死者過半。杜常、夏俶先遁,致虛斬之。孫昭遠、王似、王倚等留
 陜府,致虛收餘兵入潼關。方致虛之鼓行出關也,裨將李彥仙曰:「行者利速,多為支軍,則舍不至淹,敗不至覆。若眾群聚而出殽、澠,一蹴於險,則皆潰矣。」致虛不聽,遂底於敗。



 高宗即位,言者論其逗撓不進,徙知鄧州。尋加觀文殿學士,復知京兆府;致虛力辭,而薦席益、李彌大、唐重自代。詔以重守京兆,致虛復知鄧州。次年,宗印領兵出武關,與致虛合。會金將銀朱兵壓境,致虛遁,宗印兵不戰走,轉運使劉汲力戰死焉。致虛坐落職,責授安
 遠軍節度副使,英州安置。高宗幸建康,召復資政殿學士、知鼎州。行至巴陵卒,贈銀青光祿大夫。



 呂好問,字舜徒,侍講希哲子也。以蔭補官。崇寧初,治黨事,好問以元祐子弟坐廢。兩監東嶽廟,司揚州儀曹。時蔡卞為帥,欲扳附善類,待好問特異。好問以禮自持,卞不得親。及卞得政,當時據屬拔擢略盡,獨好問留滯,卞諷之曰:「子少親我,即階顯列矣。」好問笑不答。



 靖康元年,以薦召為左司諫、諫議大夫,擢御史中丞。欽宗諭之曰:「
 卿元祐子孫,朕特用卿,令天下知朕意所向。」先是,徽宗將內禪,詔解黨禁,除新法,盡復祖宗之故。而蔡京黨戚根據中外,害其事,莫肯行。好問言:「時之利害,政之闕失,太上皇□旨備矣。雖使直言之士抗疏論列,無以過此,願一一施行之而已。」又言:「陛下宵衣旰食,有求治之意;發號施令,有求治之言。逮今半載,治效逾邈,良田左右前後,不能推廣德意,而陛下過於容養。臣恐淳厚之德,變為頹靡,且今不盡革京、貫等所為,太平無由可致。」欽
 宗鄉納。好問疏蔡京過惡,乞役海外,黜朋附之尤者以厲其餘。又建白削王安石王爵,正神宗配饗,褒表江公望,張庭堅、任伯雨、龔□等,除青苗之令,湔元符上書獲譴者,章前後疏十上。每奏對,帝雖當食,輒使畢其說。



 時金人既退,大臣不復顧慮,武備益弛。好問言:「金人得志,益輕中國,秋冬必傾國復來,禦敵之備,當速講求。今邊事經畫旬月,不見施設,臣僚奏請皆不行下,此臣所深懼也。」及邊警急,大臣不知所出,遣使講解。金人佯許而
 攻略自如,諸將以和議故,皆閉壁不出。好問言:「彼名和而實攻,朝廷不謀進兵遣將,何也?請亟集滄、滑、邢、相之戍,以遏奔沖,而列勤王之師於畿邑,以衛京城。」疏上不省。



 金人陷真定,攻中山,上下震駭,廷臣狐疑相顧,猶以和議為辭。好問率臺屬劾大臣畏懦誤國,出好問知袁州。欽宗憫其忠,下遷吏部侍郎。既而金人薄都城,欽宗思好問言,進兵部尚書。都城失守,召好問入禁中,軍民數萬斧左掖門求見天子,好問從帝御樓諭遣之。衛士
 長蔣宣帥其徒數百,欲邀乘輿犯圍而出,左右奔竄,獨好問與孫傅、梅執禮侍,宣抗聲曰:「國事至此,皆宰相信任奸臣,不用直言所致。」傅呵之。宣以語侵傅,好問曉之曰:「若屬忘家族,欲冒重圍衛上以出,誠忠義。然乘輿將駕,必甲乘無闕而後動,詎可輕邪?」宣詘服曰:「尚書真知軍情。」麾其徒退。



 帝再幸金營,好問實從,帝既留,遣好問還,尉拊都城。已而金人立張邦昌,以好問為事務官。邦昌入居都省,好問曰:「相公真欲立邪,抑姑塞敵意而徐
 為之圖爾?」邦昌曰:「是何言也?」好問曰:「相公知中國人情所向乎?特畏女真兵威耳。女真既去,能保如今日乎?大元帥在外,元祐皇太后在內,此殆天意,盍亟還政,可轉禍為福。且省中非人臣所處,宜寓直殿廬,毋令衛士俠陛。敵所遺袍帶,非戎人在旁,弛勿服。車駕未還,所下文書,不當稱聖旨。」以好問攝門下省。好問既系銜,仍行舊職。時邦昌雖不改元,而百司文移,必去年號,獨好問所行文書,稱「靖康二年」。吳幵、莫儔請邦昌見金使於紫宸、
 垂拱殿,好問曰:「宮省故吏驟見御正衛,必將憤駭,變且不測,奈何?」邦昌矍然止。王時雍議肆赦,好問曰:「四壁之外,皆非我有,將誰赦?」乃先赦城中。



 始,金人謀以五千騎取康王,好問聞,即遣人以書白王,言:「大王之兵,度能擊則邀擊之,不然,即宜遠避。」且言:「大王若不自立,恐有不當立而立者。」既,又語邦昌曰:「天命人心,皆歸大元帥,相公先遣人推戴,則功無在相公右者。若撫機不發,他人聲義致討,悔可追邪?」於是邦昌謀遣謝克家奉傳國寶
 往大元帥府,須金人退乃發。金將將還,議留兵以衛邦昌。好問曰:「南北異宜,恐北兵不習風土,必不相安。」金人曰:「留一勃堇統之可也。」好問曰:「勃堇貴人,有如觸發致疾,則負罪益深。」乃不復留兵。金人既行,好問趣遣使詣大元帥府勸進,請元祐太后垂簾,邦昌易服歸太宰位。太后自延福宮入聽政。



 高宗即位,太后遣好問奉手書詣行在所,高宗勞之曰:「宗廟獲全,卿之力也。」除尚書右丞。丞相李綱以群臣在圍城中不能執節,欲悉按其罪。
 好問曰:「王業艱難,政宜含垢,繩以峻法,懼者眾矣。」侍御史王賓論好問嘗污偽命,不可以立新朝。高宗曰:「邦昌僭號之初,好問募人繼白書,具道京師內外之事。金人甫退,又遣人勸進。考其心跡,非他人比。」好問自慚,力求去,且言:「邦昌僭號之時,臣若閉門潔身,實不為難。徒以世被國恩,所以受賢者之責,冒圍繼書於陛下。」疏入,除資政殿學士、知宣州、提舉洞霄宮,以恩封東萊郡侯。避地,卒於桂州。



 子本中、揆中、弸中、用中、忱中。孫祖謙、祖儉。
 本中、祖謙、祖儉別有傳。



 論曰:朱勝非、呂頤浩處苗、劉之變,或巽用其智,或震奮其威,其於復闢討賊之功,固有可言矣。然李綱、趙鼎當世之所謂賢者,而勝非、頤浩視之若冰炭然,其中之所存,果何如哉。範宗尹忍於污張邦昌之偽命,而誣李綱以震主之威,何其繆於是非也。範致虛佞附權臣,大誼已失,其總勤王之師,輕而寡謀,以底於敗,宜哉。若呂好問處艱難之際,其跡與宗尹同,而屈己就事,以規興復,
 亦若勝非之處苗、劉,其心有足亮云。



\end{pinyinscope}