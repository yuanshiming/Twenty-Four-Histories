\article{列傳第四十九}

\begin{pinyinscope}

 曹
 利用孫繼鄴附張耆子希一等楊崇勛夏守恩弟守贇子隨狄青張玉孫節附郭逵



 曹利用,字用之,趙州寧晉人。父諫,擢明經第,仕至右補闕,以武略改崇儀使。利用少喜談辯,慷慨有志操。諫卒,
 補殿前承旨,改右班殿直,遷為鄜延路走馬承受公事。



 景德元年,契丹寇河北,真宗幸澶州,射殺契丹大將撻覽,契丹欲收兵去,使王繼忠議和,擇可使契丹者。利用適奏事行在,樞密院以利用應選,帝曰:「此重事也,毋輕用人。」明日,樞密使王繼英又薦利用,遂授閣門祗候、崇儀副使,奉書詣契丹軍。帝語利用曰:「契丹南來,不求地則邀賂爾。關南地歸中國已久,不可許;漢以玉帛賜單于,有故事。」利用憤契丹,色不平,對曰:「彼若妄有所求,臣
 不敢生還。」帝壯其言。



 利用馳至軍中,耶律隆緒母見利用車上,車軛設橫板,布食器,召與飲食,其從臣重行坐。飲食畢,果議關南地,利用拒之。遣其臣韓杞來報命,利用再使契丹。契丹母曰:「晉德我,畀我關南地,周世宗取之,今宜還我。」利用曰:「晉人以地畀契丹,周人取之,我朝不知也。若歲求金帛以佐軍,尚不知帝意可否,割地之請,利用不敢以聞。」其政事舍人高正始遽前曰:「我引眾以來,圖復故地。若止得金帛歸,則愧吾國人矣。」利用曰:「
 子盍為契丹熟計,使契丹用子言,恐連兵結釁,不得而息,非國利也。」契丹度不可屈,和議遂定,利用奉約書以歸。擢東上閣門使、忠州刺史,賜第京師。契丹遣使來聘,遂命利用迎勞之。



 知宜州劉永規馭下殘酷,軍校乘眾怨,殺永規叛,陷柳城縣,圍象州,分兵掠廣州,嶺南騷動。帝謂輔臣曰:「向者司天占候當用兵,朕固憂遠方守將非其人,以起邊釁,今果然。曹利用曉方略,盡心於事,其以為廣南安撫使。」利用至嶺外,遇賊武仙縣。賊持健標,
 蒙採盾,衣甲堅利,鋒鏑不能入。利用使士持巨斧長刀破盾,遂斬首以徇。嶺南平,遷引進使。歷客省使、嘉州防禦使,出為鄜延路總管。大中祥符七年,拜樞密副使,加宣徽北院使、同知院事,進知院事,遂拜樞密使、同中書門下平章事。



 利用在位既久,頗恃功。天禧二年,輔臣丁謂、李迪爭論帝前,迪斥謂奸邪,因言利用與之為朋黨。利用曰:「以片文遇主,臣不如迪;捐軀以入不測之虜,迪不逮臣也。」迪坐是免,而利用以檢校太師兼太子少保
 為會靈觀使,進尚書右僕射。



 乾興初,加左僕射兼侍中、武寧軍節度使、景靈宮使,詔如曹彬給公使錢歲萬緡。契丹使者蕭從順桀驁,稱疾留館下,不時發。朝廷遣使問勞,相望於道。利用請一切罷之,從順乃引去。



 加司空。舊制,樞密使雖檢校三司兼侍中、尚書令,猶班宰相下。乾興中,王曾由次相為會靈觀使,利用由樞密使領景靈宮使,時重宮觀使,詔利用班曾上,議者非之。未幾,曾進昭文館大學士、玉清昭應宮使,將告謝,而利用猶欲
 班曾上,閣門不敢裁。帝與太后坐承明殿久之,遣押班趣班,閣門惶懼莫知所出,曾抗聲目吏曰:「但奏宰臣王曾等告謝。」班既定,而利用怏怏不平。帝使同列慰曉之,仍詔宰臣、樞密使序班如故事,而利用益驕,尚居次相張知白上。尋召張旻於河陽,為樞密使,利用疑代己,始悔懼焉。



 初,章獻太后臨朝,中人與貴戚稍能軒輊為禍福,而利用以勛舊自居,不恤也。凡內降恩,力持不予,左右多怨,太后亦嚴憚利用,稱曰「侍中」而不名。利用奏事
 簾前,或以指爪擊帶□,左右指以示太后曰:「利用在先帝時,何敢爾邪?」太后頷之。利用奏抑內降恩難屢卻,亦有不得已從之者。人揣知之,或紿太后曰:「蒙恩得內降輒不從,今利用家媼陰諾臣請,其必可得矣。」下之而驗,太后始疑其私,頗銜怒。



 內侍羅崇勛得罪,太后使利用召崇勛戒敕之,利用去崇勛冠幘,詬斥良久,崇勛恨之。會從子汭為趙州兵馬監押,而州民趙德崇詣闕告汭不法事。奏上,崇勛請往按治,遂窮探其獄。汭坐被酒衣
 黃衣,令人呼萬歲,杖死。初,汭事起,即罷利用樞密使,加兼侍中判鄧州。及汭誅,謫左千牛衛將軍、知隨州。又坐私貸景靈宮錢,貶崇信軍節度副使,房州安置,命內侍楊懷敏護送;諸子各奪二官,沒所賜第,籍其貲,黜親屬十餘人。宦者多惡利用,行至襄陽驛,懷敏不肯前,以語逼之,利用素剛,遂投繯而絕,以暴卒聞。



 後其家請居鄧州,帝惻然許之,命其子內殿崇班淵監本州稅。明道二年,追復節度兼侍中,後贈太傅,還諸子宮,賜謚襄悼,命
 學士趙概作神道碑,帝為篆其額曰「旌功之碑」,詔歸所沒舊產。



 利用性悍梗少通,力裁僥幸,而其親舊或有因緣以進者,故及於禍。然在朝廷忠藎有守,始終不為屈,死非其罪,天下冤之。



 孫繼鄴字符嗣,其先金陵人。祖謙,事李升為長劍都指揮使,南伐閩,援兵不至,戰死。父承睿時為小校,憤將兵者不如期,致其父沒,乃刺殺之,亡去,轉徙淮、楚間。久之,入京師,以策上太宗,授左班殿直,終左藏庫使。



 繼鄴初
 以三班奉職監涔陽酒稅。會宜州陳進反,曹利用闢以自隨,為前驅,破賊於象州大烏嶺。以功遷左侍禁、端州兵馬監押。徙秦州永寧砦,總徒城洛門,改西頭供奉官。晁迥薦為閣門祗候,上御戎策十數事。又用曹瑋薦,為鄜延路兵馬都監,徙知環州,累遷崇儀副使。會修築洪德砦,與總兵者論事不協,絀為冀州兵馬都監,起知保安軍,徙涇州。使契丹。



 樞密使曹利用欲用之,繼鄴惡其權盛,陰知利用將有禍,數以疾辭,遂除左龍武軍統軍
 致仕。利用貶,復為崇儀副使,遷供備庫使、知石州,徙保州,領恩州刺史、知雄州。累遷西上閣門使,擢為龍神衛四廂都指揮使、端州防禦使。出為環慶路副都總管,道改涇原路,兼知渭州。建言:「蕭關故道,前控大川,善水草,賊騎所從出也。誠得屬羌,與奉賜,且羈其酋領,使為藩籬,則可無西顧憂矣。」為步軍都虞候,徙真定路,卒。



 張耆,字符弼,開封人。年十一,給事真宗藩邸,及即位,授西頭供奉官。嘗與石知顒侍射苑中,連發中的,擢供備
 庫副使、帶御器械。



 咸平中,契丹犯邊,以功遷南作坊使、昭州刺史、天雄軍兵馬鈐轄。邊兵未解,徙鎮州行營鈐轄,又徙定州。契丹圍望都,耆與諸將從間道往援,比至,城已陷矣。耆與敵戰,身被數創,殺契丹梟將。遲明復戰,而王繼忠為契丹所執。耆還,因言天道方利先舉者,請大舉討之,及上興師出境之日。帝以問輔臣,以為不可。遷昭州團練使、並代州鈐轄。明年,契丹兵復入,旁欲親征,耆奏邊事十餘條,多論兵貴持重及所以取勝者。召
 還,入對,帝曰:「卿嘗請北伐,契丹入塞,與卿所請興師之日同,悔不用卿策。今領守澶州而未得人,如何?」耆請行。帝喜,命為駕前西面鈐轄,令至澶州候契丹遠近。耆馳騎往,改東面排陣鈐轄。



 事平,會曹州趙諫告耆受金,為人求薦禮部,貶供備庫使、潞州都監。久之,事稍辨,復官管勾皇城司。帝以耆歷河東,稔邊事,召耆至宣和閣,問地里險易狀。耆因言:「雲、應、蔚、朔四郡,間遣人以文移至並、代間,非覘邊虛實,即欲熟道路。宜密諭代州,使自雲、
 應、蔚至者由大石谷入,自朔至者由土墱入,餘間道皆塞之以示險。」景德罷兵,耆與曹璨、李神祐、岑保正閱軍籍,請汰罷癃者。遷英州防禦使、侍衛親軍馬軍都虞候。



 從帝東封,遷絳州防禦使、殿前都虞候。時建玉清宮,耆奏疏謂殫國財力,非所以承天意。遷相州觀察使、馬軍副都指揮使。從祀汾陰,授威塞軍節度使,進宣徽南院使兼樞密副使。罷,判河陽。丁父母憂,起復,徙武寧軍節度使,拜同中書門下平章事、判陳州。累遷鎮安軍、淮南
 節度使、判壽州。遣中書舍人張師德就賜告敕。尋召為樞密使兼群牧制置使、會靈觀使。



 先名旻,至是表改名耆。加尚書左僕射,歷河陽、泰寧、山南東道、昭德軍節度使,進兼侍中,封鄧國公。章獻太后崩,以左僕射、護國軍節度出判許州,移襄、鄧、孟、許、陳、壽六州,封徐國公。



 耆為人重密,有智數,真宗在東宮,嘗命授以《論語》、《左氏春秋》,後又賜《宸戒》二十條及《聖政記》、《冊府元龜》,故頗知傳記及術數之學,言象緯輒中。章獻太后微時嘗寓其家,耆
 事之甚謹。及太后預政,寵遇最厚,賜第尚書省西,凡七百楹,安佚富盛逾四十年。家居為曲闌,積百貨其中,與群婢相貿易。有病者親為診切,以藥儥之,欲錢不出也。所歷藩鎮,人頗以為擾。然御諸子嚴,日一見之,即出就外舍,論者亦以此多之。以太子太師致仕,卒,贈太師兼侍中,謚榮僖。



 子二十四人。得一,慶歷中守貝州,妖人王則作亂,不能死,又與之草禮儀,伏誅;可一,坐與群婢賊殺其妻,棄市;利一,團練使;誠一,客省使、樞密都承旨。



 希
 一字簡翁,以父耆任,累官引進使,在知冀、邢等九州。貝州叛,希一先引兵至,得其水門。猶絓兄得一累,監洪州鹽。復為河北緣邊安撫副使。請徙邊兵內地以寬糴費,每州歲為市平以糴邊穀,使人不能高下其價;戍卒之孥給糧,先軍士一日,使其家為伍保,坐以逃亡之累,皆著為法。徙成都利州路鈐轄、真定府路總管。



 累使遼及館客,遼人嘗以雄州不當禁漁界河、及役白溝兩屬民為言。希一曰:「界河之禁,起於大國統和年,今文移尚存。
 白溝本輸中國田租,我太宗特除之,自是大國侵牟立稅,故名兩屬,惡有中國不役之理?」遼人詞塞。以均州防禦使提舉集禧觀,卒。弟利一。



 利一字和叔。以蔭補供奉官、光州都監。提點京東、淮南刑獄,知莫、冀二州,為河北緣邊安撫都監兼閣門通事舍人、知廣信軍。



 諜告遼人宋元寇邊,利一置酒高會於譙門,元率眾遁去。徙知保州、雄州,累遷西上閣門使、嘉州團練使。遼人刺兩屬民為兵,民不堪其辱,利一綏
 來之。有大姓舉族南徙,慕而來者至二萬。利一發稟振恤,且移詰涿州,自是不敢復刺。



 巡檢趙用有罪,坐不察舉,改衛州鈐轄。久之,為定州路鈐轄,進馬步軍總管,徙真定、大名府路。歷知代、滄、澶、鄭、相州,終雄州團練使。



 楊崇勛,字寶臣,薊州人。祖守斌,事太祖為龍捷指揮使。父全美,事太宗為殿前指揮使。崇勛以父任為東西班承旨,事真宗於東宮。帝嘗曰:「聞若嗜學,吾授若書。」崇勛自是稍通兵法及前代興廢之事。真宗即位,遷左侍禁、
 西頭供奉官、寄班祗候。



 雷有終討王均,崇勛承受公事,以奏捷擢內殿崇班。累遷西上閣門使、群牧都監,改副使,以左衛大將軍、恩州刺史為樞密都承旨,尋提舉樞密諸房、通進銀臺司事。以英州防禦使為馬軍都虞候、並代州馬步軍副都總管,留為客省使、領群牧使。



 真宗久不豫,寇準罷。入內副都知周懷政謀奉帝為太上皇,傳位太子,復相準。嘗以謀訪崇勛,崇勛以變告。丁謂得其辭,夜造曹利用,共議發之。翌日,誅懷政,擢崇勛鄧州
 觀察使,不拜,乃以內客省使領桂州觀察使,復兼群牧使。初,群牧置使皆以文臣領之,崇勛曰:「馬者戰備,雖無事,可去邪?」



 仁宗即位,以彰德軍節度觀察留後知陳州,授殿前都虞候、真定府定州路副都總管、知定州,歷馬軍副都指揮使、殿前都指揮使、振武軍節度使,拜宣徽南院使兼樞密副使。宮中火,為修葺副使。又歷鎮南、定武軍、山南東道節度使。



 章獻與仁宗言,先帝最稱崇勛質信,可任大事,乃進樞密使。百官詣洪福院上章懿冊,
 退而立班奉慰,宰相張士遜過崇勛園飲,日中期不至。御史中丞範諷劾奏,與士遜俱罷,以同平章事、河陽三城節度使判許州。翌日,改陳州。景祐初,懷政家人訟冤,遂罷同平章事,知壽州,徙亳州,復知陳州。



 契丹將渝盟,朝廷擇將備邊,崇勛請行,復拜同平章事、判定州。既而老不任事,徙成德軍,又徙鄭州。坐其子宗誨納賕枉法,以左衛上將軍致仕,改太子太保,卒。贈太尉,溢恭密,尋改謚恭毅。



 崇勛性貪鄙,久任軍職。當真宗時,每對,輒肆
 言中外事,喜中傷人,人以是畏之。在藩鎮日,嘗役兵工作木偶戲人,塗以丹白,舟戴鬻於京師。



 夏守恩,字君殊,並州榆次人。父遇,為武騎軍校,與契丹戰,歿。時守恩才六歲。補下班殿侍,給事襄王宮,累遷西頭供奉官。



 真宗即位,四遷至北作坊使、普州刺史。帝幸澶淵,守恩從行,數見任使。遷博州刺史,歷龍神衛、捧日天武四廂都指揮使,泰州防禦使。帝不豫,中宮預政,以守恩領親兵,倚用之。擢殿前都虞候,以安遠軍節度使
 觀察留後管勾殿前馬步軍都指揮使事。



 天聖初,加步軍副都指揮使、威塞軍節度使,為永定陵總管。雷允恭、邢中和徙皇堂,穿地得水泉,土石相半,人疫,功不就。守恩以聞,允恭等伏誅。徙節河陽三城,歸本鎮,知澶、相、曹三州,並代路馬步軍都總管,歷天雄、泰寧、武寧節度使,為真定府定州路都總管。



 守恩所至,恃寵驕恣不法。其子元吉通賂遺,市物多不予直。定州通判李參發其贓,命侍御史趙及與大名府通判李鉞鞫問得實,法當死,
 帝命貸之,除名連州編管,卒貶所。



 守贇字子美。初,守恩給事襄王邸,王問其兄弟,守恩言守贇四歲而孤,日侍王邸,不得時撫養,心輒念之。王為動容,即日召入宮,而憐其幼,聽就外舍。後二年,復召入,王乳母齊國夫人使傅婢拊視之。



 稍長,習通文字。王為太子,守贇典工作事。及即位,授右侍禁。李繼遷叛,命使綏、夏伺邊釁,遷西頭供奉官、寄班祗候。帝幸大名,為駕前走馬承受。康保裔與賊戰,沒,部曲畏誅,聲言保裔降
 賊,密詔守贇往察之。守贇變服入營中,廉問得狀,還奏稱旨。詔恤保裔家,以守贇為真定路走馬承受公事。



 帝幸澶淵及祀汾陰,皆為駕前巡檢,累遷東綾錦副使。從幸亳州,命修行宮。轉崇儀使、提舉倉草場。帝甚親信之,遣中使問守贇曰:「欲管軍乎?為橫行使乎?」守贇曰:「臣得日近冕旒足矣。」尋遷西上閣門使、提舉諸司庫務,以右千牛衛大將軍、昭州刺史為樞密都承旨,兼領三班院。



 每契丹使至,與楊崇勛迭為館伴副使,凡十餘年。擢侍
 衛親軍步軍都虞候,改馬軍、並代州都總管。累遷步軍、馬軍殿前副都指揮使,建武、鎮東、保大軍節度使。俄以修大內勞,除殿前都指揮使,徙定國軍節度使。



 守恩坐贓廢,守贇亦以鎮海軍節度使罷管軍,之本鎮。逾年,徙定州路都總管,召知樞密院事。既入見,帝問西事,守贇言:「平時小障屯兵馬不及千餘,賊兵盛至,固守不暇,安能出門邪?宜並其兵以據沖要,伺便邀擊,功或可成。」帝然之。



 劉平、石元孫敗,人有以降賊誣告者。守贇頗辨其
 枉,引康保裔事為質,自請將兵擊賊。換宣徽南院使、陜西馬步軍都總管兼經略、安撫、緣邊招討使,命勾當御藥院張德明、黎用信掌御劍以隨之。然守贇性庸怯,寡方略,不為士卒所服。



 尋詔駐軍河中,居數月,徙屯鄜州。其子隨為陜西緣邊招討副使。時晏殊、宋綬知樞密院,又召守贇同知院事。隨卒,守贇請罷,以宣徽南院使、天平軍節度使判澶州,以疾徙相州。疾稍平,復為真定府定州等路都總管,未至,徙高陽關,就判瀛州。卒,贈太尉,
 謚忠僖。



 隨字君正,頗好儒術,多從士大夫游。以父蔭為茶酒班殿侍,遷右班殿直。仁宗在東宮,為率府副率兼春坊謁者。及即位,除內殿承制、閣門祗候,累遷西上閣門使,出為天雄軍兵馬鈐轄。以母疾召還,領三班院,再遷四方館使、營州刺史。出知衛州,真拜韶州團練使。徙邠州,遷泰州防禦使。



 元昊反,為鄜延路副都總管。隨本名元亨,與元昊有嫌,因奏改焉。尋徙環慶路,未幾,復還鄜延。元
 昊為書及錦袍、銀帶投境上,以遺金明李士彬,且約與同叛。候人得之,諸將皆疑士彬,獨隨曰:「此行間爾。士彬與羌世仇,若有私約,通贈遺,豈使眾知邪?」乃召士彬與飲,厚撫之。士彬感泣,後數日,果擊賊,斬首獲羊馬自效。



 及守贇知樞密院事,除耀州觀察使、知亳州。劉平、石元孫敗,以隨知河中府。守贇經略安撫陜西,留領會靈觀事。守贇還,復為陜西副都總管兼緣邊招討副使。帝曰:「朝廷方以邊事委卿,卿毋以父在機密為嫌。」時隨已病,次
 陜州,卒。贈昭信軍節度使,謚莊恪。隨在邊陲無多戰功,然慎重少過。



 論曰:「曹利用投身不測之淵,以口舌啖契丹,使河北七十年無鋒鏑之虞,勛業固偉矣。嶺南之戰,亦豈可少哉!恃功怙寵,祝萌而弗悟,可悲也已!耆、崇勛二夏奮闒茸,位將相,皆驕侈貪吝,恃私恩,違清議,君子所不取也。



 狄青,字漢臣,汾州西河人。善騎射。初隸騎禦馬直,選為散直。寶元初,趙元昊反,詔擇衛士從邊,以青為三班差
 使、殿侍、延州指使。時偏將屢為賊敗,士卒多畏怯,青行常為先鋒。凡四年,前後大小二十五戰,中流矢者八。破金湯城,略宥州,屠SV咩、歲香、毛奴、尚羅、慶七、家口等族,燔積聚數萬,收其帳二千三百,生口五千七百。又城橋子穀,築招安、豐林、新砦、大郎等堡,皆扼賊要害。嘗戰安遠,被創甚,聞寇至,即挺起馳赴,眾爭前為用。臨敵被發、帶銅面具,出入賊中,皆披靡莫敢當。



 尹洙為經略判官,青以指使見,洙與談兵,善之,薦於經略使韓琦、範仲淹
 曰:「此良將材也。」二人一見奇之,待遇甚厚。仲淹以《左氏春秋》授之曰:「將不知古今,匹夫勇爾。」青折節讀書,悉通秦、漢以來將帥兵法,由是益知名。以功累遷西上閣門副使,擢秦州刺史、涇原路副都總管、經略招討副使,又加捧日天武四廂都指揮使、惠州團練使。



 仁宗以青數有戰功,欲召見問以方略,會賊寇渭州,命圖形以進。元昊稱臣,徙真定路副都總管,歷侍衛步軍殿前都虞候、眉州防禦使,遷步軍副都指揮使、保大安遠二軍節度
 觀察留後,又遷馬軍副都指揮使。



 青奮行伍,十餘年而貴,是時面涅猶存。帝嘗敕青傅藥除字,青指其面曰:「陛下以功擢臣,不問門地,臣所以有今日,由此涅爾,臣願留以勸軍中,不敢奉詔。」以彰化軍節度使知延州,擢樞密副使。



 皇祐中,廣源州蠻儂智高反,陷邕州,又破沿江九州,圍廣州,嶺外騷動。楊畋等安撫經制蠻事,師久無功。又命孫沔、餘靖為安撫使討賊,仁宗猶以為憂。青上表請行,翌日入對,自言:「臣起行伍,非戰伐無以報國。願
 得蕃落騎數百,益以禁兵,羈賊首致闕下。」帝壯其言,遂除宣徽南院使、宣撫荊湖南北路、經制廣南盜賊事,置酒垂拱殿以遣之。時智高還據邕州,青合孫沔、餘靖兵次賓州。



 先是,蔣偕、張忠皆輕敵敗死,軍聲大沮。青戒諸將毋妄與賊鬥,聽吾所為。廣西鈐轄陳曙乘青未至,輒以步卒八千犯賊,潰於昆侖關,殿直袁用等皆遁。青曰:「令之不齊,兵所以敗。」晨會諸將堂上,揖曙起,並召用等三十人,按以敗亡狀,驅出軍門斬之。沔、靖相顧愕眙,諸
 將股慄。



 已而頓甲,令軍中休十日。覘者還,以為軍未即進。青明日乃整軍騎,一晝夜絕昆侖關,出歸仁鋪為陣。賊既失險,悉出逆戰。前鋒孫節搏賊死山下,賊氣銳甚,沔等懼失色。青執白旗麾騎兵,縱左右翼,出賊不意,大敗之,追奔五十里,斬首數千級,其黨黃師宓、儂建中智中及偽官屬死者五十七人,生擒賊五百餘人,智高夜縱火燒城遁去。遲明,青按兵入城,獲金帛鉅萬、雜畜數千,招復老壯七千二百嘗為賊所俘脅者,慰遣之。梟黃
 師宓等邕州城下,斂尸築京觀於城北隅。時賊尸有衣金龍衣者,眾謂智高已死,欲以上聞。青曰:「安知非詐邪?寧失智高,不敢誣朝廷以貪功也。」初,青之至邕也,會瘴霧昏塞,或謂賊毒水上流,土飲者多死,青殊憂之。一夕,有泉湧砦下,汲之甘,眾遂以濟。



 復為樞密副使,遷護國軍節度使、河中尹。還至京師,帝嘉其功,拜樞密使,賜第敦教坊,優進諸子官秩。初,青既行,帝每憂之曰:「青有威名,賊當畏其來。左右使令,非青親信者不可;雖飲食臥
 起,皆宜防竊發。」乃馳使戒之。及聞青已破賊,顧宰相曰:「速議賞,緩則不足以勸矣。」



 始,交址願出兵助討智高,餘靖言其可信,具萬人糧於邕、欽待之。詔以緡錢三萬賜交址為兵費,許賊平厚賞之。青既至,檄餘靖無通使假兵,即上奏曰:「李德政聲言將步兵五萬、騎一千赴援,非其情實。且假兵於外以除內寇,非我利也。以一智高而橫蹂二廣,力不能討,乃假兵蠻夷,蠻夷貪得忘義,因而啟亂,何以御之?請罷交址助兵。」從之。賊平,人服其有遠
 略。



 青在樞密四年,每出,士卒輒指目以相矜誇。又言者以青家狗生角,且數有光怪,請出青於外以保全之,不報。嘉祐中,京師大水,青避水徙家相國寺,行止殿上,人情頗疑,乃罷青為同中書門下平章事,出判陳州。明年二月,疽發髭,卒。帝發哀,贈中書令,謚武襄。



 青為人慎密寡言,其計事必審中機會而後發。行師先正部伍,明賞罰,與士同饑寒勞苦,雖敵猝犯之,無一士敢後先者,故其出常有功。尤喜推功與將佐。始,與孫沔破賊,謀一
 出青,賊既平,經制餘事,悉以諉沔,退若不用意者。沔始嘆其勇,既而服其為人,自以為不如也。尹洙以貶死,青悉力賙其家事。子諮、詠,並為閣門使。詠數有戰功。



 熙寧元年,神宗考次近世將帥,以青起行伍而名動夷夏,深沈有智略,能以畏慎保全終始,慨然思之,命取青畫像入禁中,禦制祭文,遣使繼中牢祠其家。



 張玉字寶臣,保定人。以六班散直隸狄青麾下,築青澗、招安砦。遇夏兵三萬,有馳鐵騎挑戰者,玉單持鐵簡出
 鬥,取其首及馬,軍中因號曰張鐵簡。以狀聞。仁宗曰:「真勇將也。」以為本路同巡檢。從征儂智高,抵歸仁驛,賊列三銳陳以逆官軍,軍小卻,玉率右廂突騎橫貫賊壘,賊大潰。帝召見,使作銳陳於殿廷下,觀破賊之勢。擢為廣西鈐轄,徙大名,進龍、神四廂都指揮使,為副都總管。



 諒祚攻大順城,玉以兵三千夜擊之,驚潰而去。累遷昭州防禦使,徙涇原。熙寧中,慶州卒叛,玉襲逐於石門,卒窮蹙請降,玉斬二百人,坐奪職,降為陵州團練使,居數月,
 復之。



 王韶開熙河,玉遷宣州觀察使,為副都總管。河北置三十七將,以玉為第一將。入為馬步軍都虞候,卒,贈建雄留後。



 孫節,開封人。少隸軍籍,以才勇補右侍禁。與狄青同在延州,數攻破敵砦有功,累遷西京左藏庫副使。及青討智高,闢隸麾下。至歸仁鋪,節為前鋒,直前搏戰,賊銳甚,節鏖山下,俄中槍而沒。特贈忠武軍節度留後,封其妻為仁壽郡君,官其子二人、從子三人,給諸司副使奉,終
 其喪。



 郭逵,字仲通,其先自邢徙洛。康定中,兄遵死於敵,錄逵為三班奉職,隸陜西範仲淹麾下。仲淹勉以問學。延安清剛社募兵誤殺熟羌,將論死,逵請而免之,活壯士十三人。方議取靈武,逵曰:「地遠而食不繼,城大而兵不多,未見其利。」未幾,涇原任福以全軍沒,人服其先見。



 陳執中安撫京東,奏為駐泊將。執中與賓佐論當今名將,共推葛懷敏。逵曰:「懷敏易與爾,他日必敗朝廷事。」執中始
 怒,居數日,問曰:「君何以知葛懷敏非名將而敗事邪?」曰:「喜功徼幸,徒勇無謀,可禽也。」執中嘆曰:「君真知兵,懷敏既覆師矣。」為真定兵馬監押。



 保州卒叛,田況遣逵往招之。逵與亂者侍其臻嘗同事範仲淹,馳至城下,示以舊所佩紫囊。臻識之,即與其黨韋貴、史克順皆再拜,邀逵登城。既見,申諭禍福,眾或疑不即下,曰:「若降,恐不免。」逵請以身為質,於是開城降。論功加閣門祗候、環慶兵馬都監。遭母憂,不得解官,凡三請乃許。慶帥杜杞贐以錢
 四十萬,謝弗受。卒喪,為涇原都監。拔古渭城,轉通事舍人,徙河北緣邊安撫都監。副吳奎使契丹,值其主受尊號,入觀禮。使還,黜為汾州都監。



 龐籍鎮河東,俾權忻州。契丹來求天池廟地,籍不能決,以諉逵。逵訪得太平興國中故牘,證為王土,檄報之,契丹愧伏。



 湖北溪蠻彭仕羲叛,加帶御器械,為路鈐轄兼知澧州。得蠻親信為鄉導,盡平諸隘,遂破其所居桃花州,仕羲棄城走,眾悉降。遷禮賓使,徙南路鈐轄、知邵州。武岡蠻反,逵討平之。累
 遷容州觀察使。仁宗山陵,以逵掌宿衛。遷殿前都虞候,出為涇原路副都部署。



 治平二年,以檢校太保同簽書樞密院,旋出領陜西宣撫使,判渭州。逵雖立軍功,而驟躋政地,議者不厭,諫官、御史交論之,不聽。神宗即位,遷靜難軍留後,召還。言者復力爭,乃改宣徽南院使、判鄆州。至鄆七日,徙鎮鄜延。



 種諤受嵬名山降,取綏州,夏人遂殺楊定。朝論以邊釁方起,欲棄綏。逵曰:「虜既殺王官,而又棄綏不守,見弱已甚。且名山舉族來歸,當何以處?」
 既而夏人欲以塞門、安遠二砦來易,朝廷許之。逵曰:「此正商於六百里之策也。非先交二砦,不可與。」遣其屬趙離、薛昌朝與夏使議,唯言砦基,離曰:「二砦之北,舊有三十六堡,且以長城嶺為界,西平王祥符所移書固在也。」虜使驚不能對,乃寢其請。初,詔焚棄綏州,逵匿而不下。至是,帝問大臣,皆莫知,逵始自劾向者違詔旨之罪,帝手詔褒答。



 夏人又求以亡命景詢易名山,逵曰:「詢,庸人也,於事何所輕重!受之則不得不還名山,恐自是蕃酋
 無復敢向化矣。」逵詗得殺楊定者首領姓名,諜告將斬之於境以謝罪,逵曰:「是且梟死囚以紿我。」報曰:「必執李崇貴、韓道喜來。」夏人言:「殺之矣。」逵命以二人狀貌物色詰問虜,情得,乃執獻之。加檢校太尉、雄武軍留後。



 韓絳主種諤計圖橫山,與逵議出兵。逵曰:「諤,狂生爾,朝廷徒以家世用之,必誤大事。」絳怒,以為沮撓,奏召逵還。明年,慶州亂,出判永興,徙秦州。王韶開熙河,逵案其不法。朝廷遣蔡確鞫之,謂逵誣罔,落宣徽使、知潞州。徙太原,復
 宣徽使。



 交址李乾德陷邕管,召為安南行營經略招討使兼荊湖、廣南宣撫使,請鄜延、河東舊吏士自隨。將行,宴於便殿,賜中軍旗章劍甲以示寵。次長沙,先遣將復邕、廉;至廣西,討拔廣源州,降守將劉應紀;又拔決里隘,乘勝取桄榔、門州,大戰富良江,斬偽王子洪真。乾德窮蹙,奉表歸命。時兵夫三十萬人,冒暑涉瘴地,死者過半。至是,與賊隔一水不得進,乃班師。坐貶左衛將軍,西京安置,屏處十年。哲宗立,復左屯衛大將軍致仕。起知潞
 州,進廣州觀察使、知河中。辭歸洛,改左武衛上將軍、提舉崇福宮,卒。輟視朝一日,贈雄武軍節度使。



 逵慷慨喜兵學,神宗嘗訪八陣遺法,對曰:「兵無常形,是特奇正相生之一法爾。」因為帝論其詳。在延安,使以教兵,久不就。逵擇諸校習金鼓屯營者六十四人,使人教一隊,頃刻而成。尤善用偏裨,每至所部,令人自言所能,暇日閱按之,故臨陣皆盡其技。



 李復圭治慶州之敗,既斬李信、劉甫,又欲罪鄜延都巡檢使白玉。玉見逵托以後事,且泣
 言不得終養母。逵哀之,不遣,申救甚力,得免。已而玉大捷於新砦,神宗謂逵曰:「白玉能以功補過,卿之力也。」每戰,先招懷,後戰鬥,愛惜士卒,不妄加誅戮。其殺賊婦女老弱者,皆不賞。雖坐征南無功久廢,猶隱然為一時宿將雲。



 論曰:宋至仁宗時,承平百年,武夫鷙卒遭時致位者雖有之,起健卒至政府,隱然為時名將,惟青與逵兩人爾。青在邊境凡二十五戰,無大勝,亦無大敗,最後昆侖
 一舉,頗著奇雋。考其識量,亦過人遠矣。逵料葛懷敏之敗,如燭照龜卜,一時最為知兵。雖南征無功,用違其長,又何尤焉



\end{pinyinscope}