\article{志第二十九 律歷九}

\begin{pinyinscope}

 皇祐渾儀



 堯敕羲和制橫簫以考察星度,其機衡用玉,欲其燥濕不變,運動有常,堅久而不能廢也。至於後世,鑄銅為圓儀,以法天體。自洛下閎造《太初歷》,用渾儀,及東漢孝
 和帝時,太史惟有赤道儀,歲時測候,頗有進退。帝以問典星待詔姚崇等,皆曰:「星圖有規法,日月實從黃道,今無其器,是以失之。」至永元十五年,賈逵始設黃道儀。桓帝延熹七年,張衡更制之,以四分為度。其後,陸績、王蕃、孔挺、斛蘭、梁令瓚、李淳風並嘗制作。五代亂亡,遺法蕩然矣。真宗祥符初,韓顯符作渾儀,但游儀雙環夾望筒旋轉,而黃、赤道相固不動。皇祐初,又命日官舒易簡、於淵、周琮等參用淳風、令瓚之制,改鑄黃道渾儀,又為漏刻、
 圭表,詔翰林學士錢明逸詳其法,內侍麥允言總其工。既成,置渾儀於翰林天文院之候臺,漏刻於文德殿之鐘鼓樓,圭表於司天監。帝為制《渾儀總要》十卷,論前代得失,已而留中不出。今具黃道游儀之法,著於此焉。



 第一重,名六合儀。



 陽經雙環:外圍二丈三尺二寸八分,直徑七尺七寸六分,闊六寸,厚六分。南北並立,兩面各列周天三百六十五度少強,北極出地三十五度少強。



 陰緯單環:外圍、徑、
 闊與陽經雙環等,外厚二寸五分,內厚一寸九分。上列十乾、十二支、八封方位,以正地形。上有池沿環流轉,以定平準。



 天常單環:外圍二丈四寸六分,直徑六尺八寸二分,闊、厚一寸二分。上列十乾、十二支、四維時刻之數,以測辰刻,與陽經、陰緯環相固,如卵之殼幕然。



 第二重,名三辰儀。



 璇璣雙環:外圍一丈九尺五寸六分,直徑六尺五寸二分,闊一寸四分,厚一寸。兩面各均周天三百六十五度
 少強,作二樞對兩極。



 赤道單環:外圍一丈九尺六寸八分,直徑六尺五寸六分,闊一寸一分,厚六分。上列二十八宿距度、周天三百六十五度少強,附於璇璣之上。



 黃道單環:外圍一丈九尺二分,直徑六尺三寸四分,闊一寸二分,厚一寸。上列周天三百六十五度少強,均分二十四氣、七十二候、六十四卦、三百六十策。出入赤道二十四度,與赤道相交,每歲退差一分有餘。



 白道單環:外圍一丈八尺六寸三分,直徑六尺二寸一分,闊一寸
 一分,厚五分。上列交度,置於黃道環中,入黃道六度,每一交終,退行黃道一度半弱,皆旋轉於六合之內。



 第三重,名四游儀。



 璇樞雙環:外圍一丈八尺二寸一分,直徑六尺七分,闊二寸,厚七分。兩面各列周天三百六十五度少強,挾直距以對樞軸,東西運轉於三辰儀內,以格星度。



 橫簫望筒:長五尺七寸,外方內圓,中通望孔,直徑六分,周於日輪,在璇樞直距之中,使南北游仰,以窺辰宿,無所不至。



 十字水平槽:長九尺四寸八分,首闊一尺二寸七分,身闊九寸二分、高七尺。水槽一寸,深八分,四柱各長六尺七寸八分,植於水槽之末,以輔天體,皆以銅為之。乃格七曜遠近盈縮,以知晝夜長短之效。其所測二十八舍距度,著於後;其周天星入宿去極所主吉兇,則具在《天文志》。



 角十二度,亢九度,氐十六度,房五度,心四度,尾十九度,箕十度,斗二十五度,牛七度,女十一度,虛十度,危十六
 度,室十七度,壁九度,奎十六度,婁十二度,胃十五度,昴十一度,畢十八度,觜一度,參十度,井三十四度,鬼二度,柳十四度,星七度,張十八度,翼十八度,軫十七度。



 皇祐漏刻



 自黃帝觀漏水,制器取則,三代因以命官,則挈壺氏其職也。後之作者,或下漏,或浮漏,或輪漏,或權衡,制作不一。宋舊有刻漏及以水為權衡,置文德殿之東廡。景祐三年,再加考定,而水有遲疾,用有司之請,增平水壺一、
 渴烏二、晝夜箭二十一。然常以四時日出傳卯正一刻,又每時正已傳一刻,至八刻已傳次時,即二時初末相侵殆半。皇祐初,詔舒易簡、於淵、周琮更造,其法用平水重壺均調水勢,使無遲疾。分百刻於晝夜;冬至晝漏四十刻,夜漏六十刻;夏至晝漏六十刻,夜漏四十刻;春秋二分晝夜各五十刻。日未出前二刻半為曉,日沒後二刻半為昏,減夜五刻以益晝漏,謂之昏旦漏刻。皆隨氣增損焉。冬至、夏至之間,晝夜長短凡差二十刻,每差一
 刻,別為一箭,冬至互起其首,凡有四十一箭。晝有朝、有禺、有中、有晡、有夕,夜有甲、乙、丙、丁、戊,昏旦有星中,每箭各異其數。凡黃道升降差二度四十分,則隨歷增減改箭。每時初行一刻至四刻六分之一為時正,終八刻六分之二則交次時。今列二十四氣、晝夜日出入辰刻、昏曉中星,以備參合。



 皇祐圭表



 觀天地陰陽之體,以正位辨方、定時考閏,莫近乎圭表。宋何承天始立表候日景,十年間,知冬至比舊用《景初歷》常後天三日。又唐一行造《大衍歷》,用
 圭表測知舊歷氣節常後天一日。今司天監圭表乃石晉時天文參謀趙延乂所建,表既欹傾,圭亦墊陷,其於天度無所取正。皇祐初,詔周琮、於淵、舒易簡改制之,乃考古法,立八尺銅表,厚二寸,博四寸,下連石圭一丈三尺,以盡冬至景長之數,面有雙水溝為平準,於溝雙
 刻尺寸分數,又刻二十四氣嶽臺晷景所得尺寸,置於司天監。候之三年,知氣節比舊歷後天半日。因而成書三卷,命曰《岳臺晷景新書》論前代測候是非、步算之法頗詳。既上奏,詔翰林學士範鎮為序以識。琮以謂二十四氣所得尺寸,比顯德《欽天歷》王樸算為密。今載氣之盈縮,備採用焉。



 小雪,皇祐元年己丑十月十九日戊寅



 新表測景長一丈一尺三寸五分,王樸算景長一
 丈一尺三寸九分,新法算景長一丈一尺三寸四分小分四十八。



 二年庚寅十月二十九日癸未雲陰不測。



 三年辛卯十月十日戊子



 新表測景長一丈一尺三寸,王樸算景長一丈一尺四寸七分,新法算景長一丈一尺二寸九分小分九十八。



 大雪,元年己丑十一月四日癸巳。雲陰不測。



 二年庚寅十一月十五日戊戌



 新表測景長一丈二尺四寸五分半,王樸算景長一
 丈二尺四寸五分,新法算景長一丈二尺四寸四分小分二十五。



 冬至,元年己丑十一月十九日戊申



 新表測景長一丈二尺八寸五分,王樸算景長一丈二尺八寸六分,新法算景長一丈二尺八寸五分。



 二年庚寅十一月三十日癸丑



 新表測景長一丈二尺八寸四分,王樸算景長一丈二尺八寸六分,新法算景長一丈二尺八寸五分。



 三年辛卯十一月十二日己未雲陰不測。



 小寒,元年己丑十二月四日癸亥



 新表測景長一丈二尺四寸,王樸算景長一丈二尺四寸八分,新法算景長一丈二尺四寸小分十五。



 二年庚寅閏十一月十五日戊辰雲陰不測。



 三年
 辛卯十一月二十七日甲戌



 新表測景長一丈二尺三寸七分,王樸算景長一丈二尺四寸八分小分二十六。



 大寒,元年己丑十二月十九日戊寅雲陰不測。



 二年庚寅十二月一日甲申



 新表測景長一丈一尺一寸七分,王樸算景長一丈一尺四寸四分,新法算景長一丈一尺一寸八分小分四十。



 三年辛卯十二月十二日己丑雲陰不測。



 立春,二年庚寅正月六日甲午雲陰不測。



 三年
 辛卯十二月十六日己亥雲陰不測。



 四年壬辰十二月二十七日甲辰



 新表測景長九尺六寸七分半,王樸算景長一丈一寸五分,新法算景長一丈六寸八分小分七



 雨水,二年庚寅正月二十一日己酉雲陰不測



 三年辛卯正月二日甲寅



 新表
 測景長八尺一寸半分,王樸算景長八尺五寸,新法算景長八尺九寸小分七十六



 四年壬辰正月十二日己未



 新表測景長八尺一寸二分半,王樸算景長八尺六寸一分,新法算景長八尺一寸二分小分一十八。



 驚蟄,二年
 庚寅二月七日甲子



 新表測景長六尺六寸三分,王樸算景長六尺八寸五分,新法算景長六尺六寸三分小分三十九。



 三年辛卯正月十七日己巳



 新表測景長六尺六寸五分,王樸算景長六尺八寸五分,新法算景長六尺六寸五分小分六十八



 四年壬辰正月二十八日乙亥雲陰不測



 春分,二年庚寅二月二十三日己卯



 新表測景長五尺三寸五分,王樸算景長五尺二寸七分,新法算景長五尺三寸四分小分七十七



 三年辛卯二月四日乙酉雲陰不測



 四年壬辰二月十四日庚寅



 新表測景長五尺三寸一分,五樸算景長
 五尺二寸七分,新法算景長五尺三寸小分七十二。



 清明,二年庚寅三月八日乙未



 新表測景長四尺二寸,王樸算景長三尺八寸九分,新法算景長四尺一寸八分小分六十一。



 三年辛卯二月十九日庚子雲陰不測。



 四年壬辰二月二十九日乙巳



 新表測景長四尺二寸二分,王樸算景長
 三尺九寸六分,新法算景長四尺二寸一分小分八十五。



 穀雨,二年庚寅三月二十三日庚戌雲陰不測



 三年辛卯三月四日乙卯



 新表測景長三尺三寸,王樸算景長二尺九寸六分,新法算景長三尺二寸九分小分八十六。



 四年壬辰三月十五日庚申



 新表測景長三尺三寸一分半,王樸算景長三尺一寸,新法算景長三尺三寸一分小分一十六。



 立夏,二年庚寅四月九日乙丑



 新表測景長二尺五寸七分,王樸算景長二尺三寸,新法算景長二尺五寸六分小分二十八。



 三年辛卯三月十九日庚午



 新表測景
 長二尺五寸七分半,王樸算景長二尺三寸,新法算景長二尺五寸七分小分四十二。



 四年壬辰三月三十日乙亥



 新表測景長二尺五寸八分半,王樸算景長二尺三寸四分,新法算景長二尺五寸八分小分四十四。



 小滿,二年庚寅四月二十四日庚辰



 新表測景長二尺三分,王樸算景長一尺
 八寸六分,新法算景長二尺三分小分五十一。



 三年辛卯四月五日乙酉



 新表測景長二尺三分半,王樸算景長一尺八寸六分,新法算景長二尺三分小分五十一。



 四年壬辰四月十六日辛卯雲陰不測。



 芒種,二年庚寅五月九日乙未



 新表測景長一尺六寸九分,王樸算景長一尺六寸,新法算景長一尺六寸半分小分九十七。



 三年辛卯四月二十一日辛丑



 新表測景長一尺六寸七分,王樸算景長一尺五寸九分,新法算景長一尺六寸七分小分八十四。



 四年壬辰五月二日丙午



 新表測景長一尺六寸八分半,王樸算景長一尺六寸,新法算景長一尺六寸八分小分二十。



 夏至,二年庚寅五月二十五日辛亥



 新表測景長一尺五寸七分半,王樸算景長一尺五寸一分,新法算景長一尺五寸七分。



 三年辛卯五月七日丙辰雲陰不測。



 四年壬辰五月十七日辛酉



 新表測景長一尺五寸七分,王樸算景長一尺
 五寸一分,新法算景長一尺五寸七分。



 小暑,二年庚寅六月十一日丙寅雲陰不測



 三年辛卯五月二十二日辛未



 新表測景長一尺六寸九分半,王樸算景長一尺六寸,新法算景長一尺六寸九分小分七十五。



 四年壬辰六月三日丙子
 雲陰不測。



 大暑,二年庚寅六月二十六日辛巳



 新表測景長二尺四寸,王樸算景長一尺八寸五分,新法算景長二尺四分小分九十七。



 三年辛卯六月七日丙戌。



 新表測景長二尺二分太,王樸算景長一尺八寸五分,新法算景長二尺四分小分二十四。



 四年壬辰六月十九日壬辰



 新表測景長二尺五分,王樸算景長一
 尺八寸七分,新法算景長二尺六分小分五十三。



 立秋,二年庚寅七月十一日丙申



 新表測景長二尺五寸九分,王樸算景長二尺二寸九分,新法算景長二尺五寸九分小分五十一。



 三年辛卯六月二十三日壬寅



 新表測景長二尺六寸一分半,王樸算景長二尺
 三寸三分,新法算景長二尺六寸二分小分七十三。



 處暑,二年庚寅七月二十七日壬子雲陰不測。



 三年辛卯七月九日丁巳



 新表測景長三尺三寸六分,王樸算景長三尺,新法算景長三尺三寸六分小分六十五。



 四年壬辰七月十九日壬戌雲
 陰不測。



 白露,二年庚寅八月十三日丁卯雲陰不測



 三年辛卯七月二十四日壬申雲陰不測



 四年壬辰八月五日丁丑雲陰不測



 秋分,二年庚寅八月二十八日壬午雲陰不測



 三年辛卯八月九日丁亥



 新表測景長五尺三寸八分,王樸算景長五尺二寸一分,新法算景長五尺三寸八分
 小分六十九。



 四年壬辰八月二十日壬辰雲陰不測



 寒露,二年庚寅九月十三日丁酉雲陰不測



 三年辛卯九月二十四日壬寅



 新表測景長六尺六寸七分,王樸算景長六尺八分,新法算景長六尺六寸七分小分八十八。



 四年壬辰九月六日戊申



 新表測景長六尺七寸三分半,王樸算景長六尺九寸一分,新法算景長六尺七寸四分小分八十四。



 霜降,二年庚寅九月
 二十八日壬子



 新表測景長八尺一寸六分,王樸算景長八尺四寸五分,新法算景長八尺一寸四分小分七十。



 三年辛卯九月十日戊午雲陰不測



 四年壬辰九月二十一日癸亥



 新表測景長八尺二寸,王樸算景長八尺五寸六分,新法算景長八尺一寸九分小分六十六。



 立冬,二年庚寅十
 月十四日戊辰



 新表測景長九尺八寸半分,王樸算景長一丈一寸,新法算景長九尺八寸一分小分二十五。



 三年辛卯九月二十五日癸酉



 新表測景長九尺七寸九分,王樸算景長一丈一寸,新法算景長九尺七寸八分小分六十
 三。



 四年壬辰十月六日戊寅



 新表測景長九尺七寸六分,王樸算景長一丈一寸,新法算景長九尺七寸六分小分
 一十。



 測景正加時早晚



 後漢熹平
 三年,《四分歷》志立冬中
 景長一丈,立春中景長九尺六寸。
 尋冬至南極,日晷最長,二氣去至日數既同,則中景應等,而前長後短,頓差四寸。此歷景冬至後天之驗也。二氣中景日差九分半弱,進退均調,略無盈縮,以率計之,二氣各退二日十二刻,則晷景之數,立冬更短,立春更長,並差二寸,二氣中景俱長九尺八寸矣,即立冬、立春之正日也。以此推之,歷置冬至後天亦二
 日十二刻也。熹平三年,時歷丁丑冬至,加時正在日中。以二日十二刻減之,定以乙亥冬至,加時在夜半後二十八刻。《宋志》大明五年十月十日,景一丈七寸七分半;十一月二十五日,景一丈八寸一分太。二十六日,一丈七寸五分強。折取其中,則中天冬至應在十一月三日求其早晚。令後二日景相減,則一日差率也,倍之為法。前二日減,以百刻乘之,為實。以法除實,得冬至加時在夜半後三十一刻,在《元嘉歷》後一日,天數之正也。量檢
 彌年,則加減均同。異歲相課,則遠近應率。觀二家之說,略而未通。熹平乃要取其中,而失於至前、至後之餘。大明則左右率,而失於為實、為法之數。若夫較景、定氣,歷家最為急務。觀古較驗,止以冬至前後數日之間,以定加時早晚。且景之差行,當二至前後,進退在微芒之間。又日有變行,盈縮稍異,若以為準,則加時相背。又晉、漢歷術,多以前後所測晷要取其中,此亦差過半日。今比歲較驗,在立冬、立春景移過寸,若較取加時,則宜以其
 相近者通計,半之為距至泛日;乃以其晷數相減,餘者以法乘之,滿其日晷差而一,為刻;乃以差刻求冬至,視其前晷,多則為減,少則為加,求夏至返之。



 加減距至泛日,為定日;仍加半日之刻,命從前距日辰,算外,即二至加時日辰及刻分。如此推求,則二至加時早晚可驗矣。



 皇祐嶽臺晷景法



 按《大衍》載日及《崇天》定差之率,雖號通密,然未能盡上下交應之理,則晷度無由合契。今立新法,使上符盈縮
 之行,下參句股之數,所算尺寸與天測驗,無有先後。其術曰:計二至後日數,乃減去二至約餘,仍加半日之分,即所求日午中積數,而置之以求進退差分,求進退差分者,置中積之數,如一象九十一日三十二分以下為在前;如一象以上,返減二至限一百八十二日六十一分,餘為在後。置前後度於上,列二百於下,以上減下,餘以下乘上,滿四千一百三十五除之為分,不滿,退除為小分。在冬至後即為進差,在夏至後即為退差。



 仍列初、末二限,求入初、末限者,置所求日午中積數,日在冬至後初限、夏至後末限之數四十五日六十二分以下,即為所求在初限;如在已上者,乃返減二至限,餘即為所求入末限。其冬至後末限、夏至後初限,以一百三十七日為率。



 用求午中晷
 數。求午中晷數者,視所求。如入冬至後初限、夏至後末限者,以入限日減一千九百三十七半,餘為泛差;仍以限日分乘其進退差,五因百約之,用減泛差,為定差;乃以入限日分自相乘,以乘定差,滿一百萬為尺,不滿為寸、為分及小分,以減冬至常晷一丈二尺八寸五分,餘為其日午中晷數。若所求入冬至後末限、夏至後初限者,乃三約入限日分,以減四百八十五少,餘為泛差;仍以進退差減極數,餘者若在春分後、秋分前者,直以四約之,以加泛差,為定差;若在春分前、秋分後者,乃以去二分日數及分乘之,滿六百而一,以減泛差,餘為定差,用以入限日分自相乘,以乘定差,滿一百萬為尺,不滿為寸、為分及小分,以加夏至常晷一尺五寸七分,即為其日午中晷數。若用周歲歷,直以其日晷景損益差分乘其日午中之餘,滿法約之,乃損益其下晷數,即其日午中定晷。



 如此推求,則上下通應之理,句股斜射之原,皆可視驗,
 乃具嶽臺晷景周歲算數。



\end{pinyinscope}