\article{志第一百四十一 兵二(禁軍下)}

\begin{pinyinscope}

 熙寧以後之制



 騎軍



 殿前指揮使左右班二。



 內殿直左右班四。



 散員左右班四。



 散指揮左右班四。



 散都頭左右班二。



 散祗候左右班二。



 金槍班左右二,元祐二年六月,密
 院言:「元豐七年,承旨司傳宣密院:殿前指揮使左右班槍手可各以五分為額,餘悉改充弓箭手。切詳先為在京馬軍全廢槍手,其諸班槍手有闕,無人揀填,遂有此宣旨。近因殿前馬步軍司奏,諸在京馬軍復置一分槍手,諸班槍手並依舊教閱。」詔:「元豐七年宣旨,更不施行。」



 東西班及弩手、龍旗直、招箭,總十一。中興後,東凡五班,西凡三班。



 散直左右四。熙寧九年,並南散直隸北散直。中興後,名招箭班散直。



 外殿直一。熙寧五年廢。



 銀槍班左右班二。中興置。



 茶酒舊班中興置。



 茶酒新班中興置。



 鈞容直國初一班。中興因之,後廢。已上為諸班。



 捧日並左射、金屈直、弩手、左第五軍,總三十五。京師三十三,雍丘、鄭各一。熙寧五年,捧日三十三並為二十二,廢弩手隸左射,餘留二十九。元第一,十月,以左射隸天武。二年,廢左射、金屈直。八月,廢第五軍,雍丘第二、南京第一並改為新立驍捷。九月,詔勿改,惟闕勿補,
 俟其少廢並。



 歸明渤海二。京師。元豐元年,撥填拱聖一,餘撥隸驍騎右四。



 拱聖二十一。京師。熙寧六年,並為十六,廢左射。中興後,副指揮一員。



 吐渾五。治平中,並為二。熙寧二年,並為一。元豐元年廢。中興後,屬步軍。



 驍騎二十二。京師。熙寧六年,並為十四,廢弩手、上驍騎。元豐元年,撥在京驍騎左第一隸神勇。



 驍勝十。熙寧三年廢。



 寧朔十。京師、尉氏各三,雍丘、滑、河陽、河陰各一。熙寧二年,並為七。元豐元年,在京第二第三並撥隸第一。



 龍猛八。熙寧三年,並為六。



 飛猛一。熙寧二年廢。



 契丹直三。咸平、棣昌、壽各二。熙寧九年廢。



 神騎十八。雍丘十三,咸平五。熙寧二年,並為十。中興後,副指揮一員。



 步斗六。尉氏、太康各一,蔡四。元豐元年,尉氏、太康各一、蔡州二皆撥隸步軍司虎翼。十一月,蔡州二改為新立驍捷,其第二充擒戎第四,等四,尉氏三、太康四第四充擒戎第五,太康一元豐元年並尉氏第三隸第
 一,太康第二改驍雄。二年,尉氏一勿填闕。



 吐渾直三。太原二,潞一。熙寧六年,廢潞州一。一年,廢太原二。元豐二年,太原、潞州各一,勿填闕。中興屬步軍。



 安慶直四。太原一,潞三。熙寧六年皆廢。



 三部落一。太原。熙寧三年廢。



 清朔四。西京二,穎昌、汝各一。



 擒戎五。西京、穎昌各二,汝一。元豐元年,蔡州置二。



 驍雄舊六,治平四年並為四。咸平、陳各二。熙寧初,以驍猛第四改充一。元豐六年,咸平、尉氏各一,闕勿補。



 其馬軍行司新軍目:



 選鋒中興置。



 神策選鋒軍、左翼軍、右翼軍、摧鋒軍、游奕軍、前軍、右軍、中軍、左軍、後軍、護聖馬步軍中興置。



 步軍



 御龍直左右二。



 御龍骨朵子直左右二。



 御龍弓箭直
 五。



 御龍弩直五。中興,左右班二。



 天武並寬衣、金屈直、左射,總三十四。京師三十三,咸平一。熙寧二年,並三十三為二十三。九年,廢左射。元豐元年,並陳留第七軍第一隸咸平第五軍第一。十月,廢寬衣天武。二年,廢第五軍,咸平第一改雄武弩手。九月,詔勿改,惟闕弗填。四年,廢金屈直。紹聖元年十一月,引進副使宋球言:「自立殿前司以來。有寬衣天武一指揮充駕出禁衛圍子,常守把在內諸門,熙寧中廢並,禁圍只差天武,皇城諸門更不差人。乞復置寬衣一指揮;或不欲添置,乞將天武本軍內以一指揮為寬衣天武。」詔:禁圍子合用天武人兵,令殿前司今後並選定四十已上、有行止無過犯、不系新招揀到人充,遇闕選填。



 神勇並上神勇二十一。京師。熙寧六年,並為十四,廢上神勇。孝宗初,改為護聖軍。



 廣勇四十三,每十為一軍。京師五,陳留二十二,咸平、東明、太康、胙城、南京各二,襄邑、陽武、鄆各一,滑三。熙寧九年,在京增置一。元祐二年八月,詔
 在京置左第三軍第一、右第三軍第一。



 神射五。陳留三,雍丘二。熙寧三年廢。



 龍騎二十,分三軍。京師四,尉氏、雍丘、咸平、鄭各二,南京、陳、蔡、河陽、棣、單、宿、白波各一。熙寧二年,並為十三。熙寧二年,在京第七隸第九。



 雄勇八。咸平三,鄆二,穎昌、鄭、滑各一。元豐元年,並咸平第二第三隸第一,鄆州第五隸第四,改曰雄威,並管城第七,白馬第八;穎昌一闕勿補。二年,咸平一闕勿補。



 宣威上下二。咸平、襄邑各一。熙寧三年,以咸平一隸廣捷,以襄邑一隸威猛,四年廢。



 廣捷五十六。陳留八,咸平六,雍丘四,襄邑、尉氏、穎昌各三,太康、扶溝、南京、亳、河陽、穎、寧陵各二,陳五,鄭、滑、曹、鄧、蔡、廣濟、穀熟、永城、襄城、萊各一。熙寧三年,亳州一並廣勇,永城縣一並隸亳州。元豐元年,並管城第四十隸本縣雄勇第七,並白馬縣第二十五隸本縣雄勇第八。



 廣德並揀中廣德,總十。咸平、尉氏、陽武、河陽、滄、鞏、白波各一,西京三。治平四年,並十四為八。熙寧六年,廢揀中廣德,
 尉氏揀中廣德第一、陽武第二改為廣德。



 雄威十。考城、襄邑、陳留各一,南京四,陳三。治平四年,並十、三為十。元豐元年,以南京第八分隸第三、第四、第七。二年,襄邑二闕勿補。



 勝捷、威勝、威捷建炎初置,隸殿前司。



 全捷、前軍、右軍、中軍、左軍、後軍自勝捷以下九軍,並中興後置。



 侍衛司侍衛親軍馬步軍都指揮使、副都指揮使、都虞候各一人。馬軍都指揮使、副都指揮使、都虞候各一人,步軍亦如之。自馬步軍都虞候以上,其員全闕,即馬軍、步軍都指揮使等各兼領其務。馬步軍有龍衛、神衛左右四廂都指揮使,龍衛、神衛左右廂各有都指揮使,每
 軍有都指揮使、都虞候,每指揮有指揮使、副指揮使,餘如殿前司之制。其所領騎步軍之額如左。



 騎軍



 員僚直左右四。京師二,恩、冀各一。熙寧二年,並左直為一,須人少撥隸如其軍省。五年,廢恩、冀州左右直弗補。六年,撥隸龍衛。元豐三年廢。



 龍衛並金屈直、左射、帶甲剩員四十四。京師三十八,雍丘、尉氏、河陽並揀中各一,澶二。熙寧元年,以澶州右第四軍第四隸第三,共並為一。九年,陳留並帶甲剩員二為一。熙寧元年,澶州、河陽、尉氏就糧四並隸別指揮。六年,三十九並為二十。八年,置帶甲剩員二。十年,廢亳州一。元豐元年,陳留帶甲剩員闕勿補。二年五月,廢金屈直、左射。八月,廢第十軍。十月,南京第十軍第一改新立驍捷左三。六年,廢帶甲剩員。中興,二十。



 忠猛一。定。熙寧五年廢。



 散員一。定。熙寧五年廢。



 驍捷
 二十六。尉氏新立及揀中各一,恩十四,冀十。熙寧元年,廢帶甲剩員。三年,廢揀中。五年,瀛州三撥隸本州雲翼,冀州十、恩州十四各並為五,莫州二並為一。十年,並冀、恩驍捷各五各為四。元豐元年,太康置新立驍捷一。



 雲騎十五。京師十一,陳留、南京各一,鞏縣二。熙寧二年,並十五為十。三年,第一至十二並為七。七月,第八撥隸第一第二。八年,置帶甲剩員一。元豐二年闕,選云捷第二軍補之。十月,雍丘帶甲剩員第一改為橫塞第十。中興,七。



 武騎二十一。京師、雍丘各六,尉氏三,陳留、考城、咸平、鄭各一,西京二。熙寧元年,廢咸平帶甲剩員為剩員。二年,並二十作十五。八年,置帶甲剩員一。九年,以雍丘帶甲剩員一隸雲騎帶甲剩員,共為一。十二月,在京四並為三,尉氏二並為一,考城一分隸雍丘寧朔,在京二並為一。十年,廢帶甲剩員。元豐元年,並帶甲剩員亳州第一。中興,三。



 驍銳四。莫三,冀一。熙寧五年,莫州三並為二,冀州第三虛其闕,以存者補捷。
 六年七月,莫州第一第二、冀州第三並改驍捷,是月廢。



 歸明神武馬一。尉氏。熙寧六年,改新立驍捷,七月,廢。



 飛捷四。雍丘。熙寧二年,並為二。元豐元年廢。



 驍武左右二十。北京七,真定三,定六,相、懷、洛、邢各一。熙寧元年,廢帶甲剩員。二年,北京七並為五。五年,真定府三並為二,定州六並為四,邢州、雲翼各一須人少並為一。十北京五並為四,定州四須人少並為三。元豐七年,以忠猛一分入驍武第七、第八、第九。



 廣銳總四十四。太原、代、並各三,汾五,石、嵐、岢嵐各二,晉、潞、慈、絳、澤、隰、憲、寧化、威勝、平定、火山各一,涇、原、鄜各二,秦、渭、環、邠、寧各一。元豐二年,忻、嵐州各一,闕勿補。三年,涇州二以下一補上一闕。五年,置蘭州二。中興,二十三。



 雲翼分左右廂,左三十四,右二十二,總五十六。真定、雄、瀛、深、趙、永寧各三,定、冀各六,保五,滄、北平、永靜、順安、保定各二,莫、邢、霸各一,廣信、安肅各四。熙寧五年,並滄州二為一,冀州六為三,真定府三為一,趙州三為
 二,定州六為四,順安軍二為一,永寧軍三為二,北平軍二須其闕並為一,安肅軍第一分隸第三,深州三為二,保州一分隸他軍。十年,莫州第十三分隸驍捷,真定府第八分隸驍武,定州四須其闕並為三,安肅軍三須其闕並為二,廣信軍四並為三。元祐元年,桂州二仍不廢。中興,三十三。



 有馬勁勇七。太原二,代、嵐各一,磁三。熙寧五年,磁三並為一。中興,五。



 騎捷五。瀛三,莫二。熙寧六年廢。



 廳子七。定二,相五。熙寧五年,並相廳子五為三,定廳子馬二為一。六年,相州廳子三並改廳子馬。十年,相州廳子馬第三分隸驍武廳子馬。中興,四。



 驍駿一。太原。熙寧六年廢。



 無敵六。定、北平各二,安肅、廣信各一。熙寧五年,北平二須人少並為一,撥隸雲翼三;廣信軍一撥隸雲翼。



 忠銳一。廣信。熙寧五年廢。



 威邊二。定、保各一。熙寧五年廢。



 克勝二。潞。



 飛騎二。麟。



 威遠二。府。



 克戎二。並。



 清塞一。延安。熙寧五年廢。



 武清一。晉。熙寧六
 年廢。



 萬捷七。相、翼、趙各二,滄一。熙寧五年,冀二並為一,以隸雲翼;相二須人少並為一。中興,七。



 雲捷十二。尉氏、咸平、西京、北京、澶各二,汝、懷各一。



 橫塞七。雍丘、咸平、考城、襄邑、寧陵各一,衛二。



 有馬安塞一。熙寧五年廢。



 蕃落八十三。環五,延、慶各四,秦並外砦十七,原、渭並外砦各十二,德順並外砦七,鎮戎並外砦十二,鳳翔、涇並外砦、儀、保安各二,隴一。熙寧三年,並外砦九為七。八月,涇原路以新砦所減蕃落隸在州蕃落,定額以三萬二千人。五年,隴州添置招馬軍蕃落一。九年,並陜西土蕃落渭州八為六,原州、秦州各五為四。元豐四年,環州下蕃落未排定指揮,並為禁軍。五年六月,葭蘆砦主乞置一。紹聖四年,詔:陜路增置馬軍十,各五百人為額,於永興、河中、鳳翔、同、華各置二。元符元年,詔:涇原路新築西安州置馬軍一,天都、臨羌砦各置馬軍一。六月,詔永興軍等路創置十指揮。二年,定邊城增置馬軍二,烏龍川、北嶺新砦各置馬
 軍一。崇寧五年,新築安邊城,置馬軍一。



 並州騎射一。熙寧六年,太原騎射第一改克戎。元豐七年,成都府置馬軍騎射一。中興後無。



 有馬雄略三。廣、桂、邕各一。熙寧三年,廣、桂、邕有馬雄略闕勿補,十年,以邕州住營兩指揮闕額移桂州,依舊置。紹聖元年,沅州增置有馬一。元符元年正月,詔荊湖南路、江南東路各增置有馬一。中興,二。



 崇捷崇寧三年,詔於京東、京西、河北、河東、開封府界創置馬步軍五萬人,計一百七指揮。馬軍三十五,步軍七十二,合三萬六千人。馬軍以崇捷、崇銳為名,步軍以崇武、崇威為名。



 崇銳崇寧三年,見上。以上二軍,中興後無。



 清澗騎射二。



 員僚剩員直以罪謫降者充立。



 前軍、右軍、中軍、左軍、後軍以上七軍,並中興後置。



 步軍



 神衛並水軍總三十一。京師。熙寧二年,並三十一為三十。三年,廢水軍。元豐二年,廢第九、
 第十,南京第一改雄武弩手。中興,四十六。



 虎翼九十六。京師九十,並水軍一,襄邑、東明、單各一,長葛二。熙寧二年,除水軍一外,並九十五為六十。六年,廢上虎翼。元豐四年,詔改差殿前虎翼右一四指揮為李憲親兵。



 奉節並上奉節五。京師。熙寧二年,殿上奉節。九月,上奉節兩指揮隸虎翼。六年十月,廢奉節。



 步武六。陳。



 武衛七十一。南京、真定、定、淄各四,北京、澶、相、邢、懷、趙、棣、洺、德、祁、通利、乾、廣濟各一,青五,鄆、徐、兗、曹、濮、沂、濟、單、萊、濰、登、淮陽、瀛、博各二,齊、密、滄各三。熙寧四年,帝諭文彥博等:「京東武衛軍素號精勇得力,不減陜西兵。」彥博曰:「京東之人沉鷙精悍,亦其性也。」五年,並滄三為二,真定府各四各為三,趙州、振武各一共為一。六年,詔岷州置一。元豐三年,河州武衛二為一。



 雄武並雄武弩手、床子弩雄武、揀中雄武、飛山雄武、揀中歸明雄武,總三十四。京十三,太原、尉氏、南京、鄭、汝、寧陵各二,咸平、東明、雍丘、襄邑、穎昌、曹、廣濟、穀熟、長葛各一。熙寧
 五年,廢揀中雄武。閏七月,並床子弩雄武、飛山雄武各五為二。六年,廢雄武。中興後,加「平海」字。



 飛虎三。陳留二,咸平一。熙寧三年廢。



 神銳二十六。太原六,潞、晉各三,澤、汾、隰、平定各二,代、絳、沂、遼、邢、威勝各一。元豐二年,潞州三,闕勿補。



 振武八十一。北京、澶、相、衛、霸、莫、祁、棣、趙、濱、洺、保安、永寧、通利、安肅、儀各一。真定、瀛、保、恩、邢、深、博、永寧、乾寧、慶、涇各二,延六,邠、隴各七,鄜、寧各五,磁四,滄、原各三。熙寧五年,瀛州二為一,滄州三為二,真定府二為一,邢州二以一分隸武衛、神銳、鎮武,磁州四為三。元豐三年,鄜州四為三,邠州五以一補上四指揮闕,隴州四為三。元祐七年,詔復置滄州第六十七、六十八。



 來化一。寧陵。熙寧七年廢。



 新立弩手二。廣濟。熙寧六年,定陶縣第二軍改雄武隊弩手。



 懷勇三。雍丘二,陳一。熙寧三年廢。



 威寧一。穎昌。熙寧二年廢。



 威猛上下十。襄邑四,咸平、穎昌、長葛各二。熙寧三年,宣威並入。



 雄勝三。陜、冀、濟各一。熙
 寧四年,分陜府雄勝隸他軍。中興,四。



 歸恩左右二。亳。熙寧三年,左第一並右第一。六年,第一改為雄勝。



 澄海弩手二。登。熙寧八年,廣西經略司選澄海赴桂州,以新澄海為名。中興,加「水軍」字。



 神虎二十六,永興六,鳳翔、河中、忻、隰、晉、威勝各二,太原、秦、延、鄜、華各一,潞三。熙寧九年,秦州一,闕勿補。



 保捷一百三十五。永興十二,同九,秦八,河中、邠、涇各七,滑、寧、耀各六,鳳翔、延、儀、華、隴、解、幹各五,陜、原、鄜各四,成三,慶、鳳、坊、晉、鎮戎各二,環、丹、商、虢、階、慶成、德順各一。熙寧五年,鳳翔府添置三。六年,添置一。元豐三年,並同州七為六,永興軍九為八。五年,蘭州置步軍二。紹聖四年,蘭州金城關置步軍四。元符元年,新築西安州,置步軍一,天都、臨羌砦各置步軍一;又詔於河北路大名府二十二州軍共創置馬步軍,步軍二十九指揮以保捷為名。二年,定邊城置步軍一。崇寧五年,安邊城置步軍一。中興後,增置一。



 捉生二。延。紹聖三年,環、慶州各置
 一。



 清邊弩手四十三。太原九,秦五,涇四,河中、隴各三,永興、代、潞、晉各二,慶、渭、環、同、坊、鎮戎、慈、丹、隰、汾、憲各一。熙寧六年,並鳳翔四為三。八年,吉陽並宣毅一來隸。九年,並秦州四為三。元豐三年,以河中清邊弩手將兵一隸本府保捷、清邊弩手。



 制勝九。永興、華各二,鳳翔、耀、同、乾、解各一。撥華一隸本州保捷、制勝,奉天一補其縣保捷闕。中興後增一。



 定功十。永興、秦、慶、原、渭、涇、儀、鄜、延、鎮戎各一。



 青澗二。中興後隸騎軍。



 平海二。登。



 建威一。秦。熙寧三年廢。



 效忠二十七。陳留三,太康、尉氏、襄邑、河陽、曹、合流各二,咸平、鄭、亳、衛、穎昌、單、澶、磁、廣濟、河陰、寧陵、白波各一。熙寧九年,磁、衛各一,須人少與武衛並為一。



 川效忠七。南京六,寧陵一。熙寧二年,南京六並隸上三。三年十二月,南京三並為二。



 宣毅一百七十四。隸京東西、河北、河東、淮南、江南、兩浙、荊湖、福建九路。京東路:南京、鄆、徐、曹、齊各二,青、兗、密、濮、沂、單、濟、淄、萊、濰、登、淮
 陽、廣濟各一。京西路:西京、滑、穎昌、河陽、陳、襄、鄭、穎、蔡、汝、隨、信陽各一,鄧二。河北路:真定、德、棣、博、邢、祁、恩、磁、深、定、洺、濱、通利、永靜、乾寧、永寧各一。河東路:太原、汾各六,晉四,澤、絳、石、代各三,潞、嵐、忻、遼、威勝、平定各二,慈、隰、憲、寧化各一。淮南路:揚、亳各二,廬、宿、壽、楚、真、泗、蘄、海、舒、泰、濠、和、光、黃、通、無為、高郵、漣水各一。江南路:江寧、江、洪、虔、吉、撫、袁、筠、建昌、南安各一。兩浙路:杭二,越、蘇、明、湖、婺、潤、溫、衢、常、處、秀各一。荊湖路:潭、全、鼎各二,荊南、邵、衡、永、郴、道、安、鄂、岳、澧、復、峽、歸、辰、荊門、漢陽、桂陽各一。福建路:二,福、泉、南劍、漳、汀、邵武、興化各一。熙寧三年,宿、揚、廬、壽、楚、真、泗、泰一並隸教閱忠節,各為一。蘄、海、舒、濠、和、光、黃、通、無為、高郵、漣水各一闕弗補。十二月,京東路三十三並為十三,荊湖南路道永衡各一、潭二撥隸威果,全二、邵一撥隸雄略,郴、桂陽各一不充額,荊南一撥隸威果,鼎二、澧岳安復鄂各一皆改教閱忠節,荊門、漢陽、歸、峽各一不充額,江南東路江寧、江南西路虔各一撥隸威果、雄
 略,洪、吉、撫、建昌各一皆改教閱忠節,筠、袁、南安各一不充額,福建路福一隸威果,建二並為一改威果,兩浙路杭二、越蘇潤各一皆改威果,湖、婺、溫、衢、常、處,秀各一不充額。熙寧五年,恩一、乾寧永靜真定邢洺磁定祁深永寧各一闕弗補。八年,吉鄉軍宣毅一隸清邊弩手,潞復置一。九年,定、邢、深、祁、磁、永寧、永靜、乾寧各一皆效忠。元豐元年,博二撥隸他州軍。



 宣毅床子弩炮手一。岢嵐。熙寧三年廢。



 建安二。府、嵐各一。



 威果二十五。荊南、江寧、杭、揚、廬、潭各三,洪、越、福各二,虔一。宣和三年,嚴州增置一。



 效順一。襄邑。熙寧六年,改雄武。



 揀中雄勇一。襄邑。



 懷順一。霸。



 歸聖一。雍丘。熙寧六年,改雄武。



 順聖一。鞏。中興已後無。



 懷恩三。荊南二,鄂一。



 揀中懷愛一。寧陵。熙寧六年廢。



 勇捷左右二十六。襄邑、北京、澶、陳、壽、汝、曹、宿各二,咸平、西京、南京、亳、寧陵、虹、河陰、鞏、長葛、韋城各一。熙寧三年,並十隸九,右十二
 並右二。元豐二年,唐、汝州各置土兵一。



 威武上下總十三。西京、河陽、鄭、鄆、澶、滑、濮、通利、鞏、河陰、永城各一,曹二。熙寧三年,廢下威武。九年,澶一隸效忠、勇捷。



 靜戎弩手四。河陽、澶、衛、通利各一。熙寧七年廢。



 平塞弩手並揀中平塞、新立平塞,總四。咸平、亳、河陰、白波各一。熙寧六年,廢弩手及新立、揀中平塞,亳平塞弩手及白波新立平塞、咸平揀中平塞並改下威武。



 忠勇三。成都。



 寧遠八。戎三,遂、梓、嘉、雅、江安各一。熙寧六年,瀘州增置一。



 忠節並川忠節、教閱忠節,總六十。雍丘、襄邑、寧陵各三,陳留、咸平、東明、亳、河陰、永城各二,南京五,太康、陽武、穎昌、江寧、揚、廬、宿、壽、楚、真、泗、泰、滁、岳、澧、池、歙、信、太平、饒、宣、洪、虔、吉、臨江、興國、廣濟、南康、廣德、長葛各一,合流四。熙寧三年,亳州第十四並勇捷,川忠節一並忠節。十二月,添置八。五年,蔡州置一。



 神威上下十三。陳留三,穎昌、鞏各二,雍丘、考城、咸平、河陽、廣濟、白波各一。



 歸遠十六。陳、穎
 昌、亳、壽、宿、鄧、襄、鼎各一,荊南、澧、潭、洪各二。元豐五年,成州置一。



 雄略二十五。荊南五,潭四,鼎、澧各三,廣、辰、桂各二,許、全、邵各一。熙寧三年,衡增置一,吉增置三百人及置部軍雄略一。崇寧三年,荊湖南路置四。



 招收十七。保四,霸、信安各三,定、軍城砦各二,廣信、安肅、順安各一。熙寧五年,霸、信安各二並為一,定二為一,安肅一、保二分隸振武、招收。八年,忻以保甲替罷揀充下禁軍。



 壯勇七。耀、解、滑各二,穎昌一。



 橋道並川橋道十八。襄邑、咸平、陽武各二,陳留、東明、尉氏、太康、西京、河陽、濮、鄆、鞏、河陰、白波、寧陵各一。熙寧三年,鄆川橋道改橋道,隸順化。



 清塞十二。曹二,鄭、鄆、滑、通利、鞏、河陰、白波、汜水各一,長葛二。



 崇武崇寧三年,置步軍京東西、河東北。



 崇威崇寧三年,置步軍京東西、河東北。



 敢勇元祐七年,詔河東、陜西路諸帥府募敢勇,以百人為額。宣和四年,詔越州招到敢勇三百人,撥充兩浙提刑司捉殺差使。



 靖安
 崇寧元年,詔荊湖北路添置禁軍五指揮,以靖安為名,隸侍衛步軍司



 廣固崇寧三年,詔添置廣固兵四指揮,以備京城工役。政和五年,詔於四指揮各增置五百人入額,自今更勿差客軍。



 通濟政和六年,詔增置通濟兵士二千人,牽挽御前綱運。自崇武至此六軍,中興後無。



 清衛宣和七年,減清衛等軍,令步軍司撥填一般軍分。



 刀牌手崇寧中立。廣西桂州。



 勁勇、壯武、靜江自勁勇以下三軍,舊隸廂軍。中興後,隸侍衛步軍。



 振華五百人為一軍。



 安遠、奉先園四。



 武寧、威勇、忠果、雄節、必勝六。



 前軍、右軍、中軍、左軍、後軍自振華以下十三軍,並中興後立。



 御前忠佐將校並與建隆以來制同。



 散員班一。



 馬直指揮一。



 步直指揮一。熙寧四年,馬步二直並廢,撥隸殿前、步軍司虎翼,其有馬者補雲
 騎。



 備軍一千九百六十人。熙寧二年,罷九百六十人。



 皇城司



 親從官指揮四。政和五年,創置第五指揮,以七百人為額。



 親事官指揮三。元豐五年增置一,守奉景靈宮。政和五年,西京大內官一,以五百五十人為額。又增置內園司一,以五百一十人為額。



 入內院子五百人。中興後,二百人。



 快行、長行中興後置,一百人。



 司圊三人。



 曹司中興置,三十人。



 將兵者,熙寧之更制也。先是,太祖懲藩鎮之弊,分遣禁旅戍守邊城,立更戍法,使往來道路,以習勤苦、均勞逸。故將不得專其兵,兵不至於驕惰。淳化、至道以來,持循
 益謹,雖無復難制之患,而更戍交錯,旁午道路。議者以為徒使兵不知將,將不知兵,緩急恐不可恃。神宗即位,乃部分諸路將兵,總隸禁旅,使兵知其將,將練其士,平居知有訓厲而無番戍之勞,有事而後遣焉,庶不為無用矣。



 熙寧七年,始詔總開封府畿、京東西、河北路兵分置將、副。由河北始,自第一將以下共十七將,在河北四路;自第十八將以下共七將,在府畿;自第二十五將以下共九將,在京東;自第三十四將以下共四將,在京西,
 凡三十有七。而鄜延、環慶、涇原、秦鳳、熙河又自列將焉。在鄜延者九,在涇原者十一,在環慶者八,在秦鳳者五,在熙河者九,凡四十有二。八年,又詔增置馬軍十三指揮,分為京東、西兩路。又募教閱忠果十指揮,在京西,額各五百人,其六在唐、鄧,其四在蔡、汝。



 元豐二年,又增置土兵勇捷兩指揮於京西,額各四百人,唐州方城為右第十一,汝州襄城為左第十二。凡馬軍十三指揮,忠果及土軍共十二指揮。四年,又詔團結東南路諸軍亦如
 京畿之法,共十三將:自淮南始,東路為第一,西路為第二,兩浙西路為第三,東路為第四,江南東路為第五,西路為第六,荊湖北路為第七,南路潭州為第八,全、邵、永州應援廣西為第九,福建路為第十,廣南東路為第十一,西路桂州為第十二,邕州為第十三。



 總天下為九十二將,而鄜延五路又有漢蕃弓箭手,亦各附諸將而分隸焉。凡諸路將各置副一人,東南兵三千人以下唯置單將。凡將、副皆選內殿崇班以上、嘗歷戰陳、親民者充,
 且詔監司奏舉。又各以所將兵多寡,置部將、隊將、押隊使臣各有差。又置訓練官次諸將佐。春秋都試,擇武力士,凡千人選十人,皆以名聞,而待旨解發,其願留鄉里者勿強遣。此將兵之法也。



 六年,熙河路經略制置李憲言:「本路雖有九將之名,其實數目多闕,緩急不給驅使。又蕃,漢雜為一軍,嗜好言語不同,部分居止悉皆不便,今未出戰,其害已多,非李靖所謂蕃、漢自為一法之意。若將本路九將並為五軍,各定立五軍將、副及都、同總
 領蕃兵將,使正兵合漢弓箭手自為一軍,其蕃兵亦各自為一軍。臨敵之際,首用蕃兵,繼以漢兵,必有成效,兼可減並將、副及部隊將員,於事為便。」詔從之。



 元祐元年,司馬光言:「近歲災傷,盜賊頗多,州郡全無武備。長吏侍衛單寡,禁旅盡屬將官,多與州郡爭衡,長吏勢力遠出其下。萬一有李順、王倫、王均、王則之寇乘間竊發,攻陷郡縣,豈不為朝廷憂!祖宗以來,諸軍少曾在營,常分番出戍。蓋欲使之勞筋骨,知艱難,輕去其家,習知山川險
 阻也。自置將以來,惟是全將起發,然後與將官偕行,其餘常在本營,飲食嬉游,養成驕惰,歲月滋久,不可復用。又每將下各有部隊將、訓練官等一二十人,而諸州又自有總管、鈐轄、都監、監押,設官重復,虛破廩祿。知兵者皆知其非。臣愚,欲乞盡罷諸路將官,其禁軍各委本州長吏與總管、鈐轄、都監、監押等,如未置將已前,使州郡平居武備有餘,然後緩急可責以守死。」



 諫議大夫孫覺亦以為言,於是詔陜西、河東、廣南將兵不出戍他路,其餘河
 北差近裏一將更赴河東,而諸路逐將與不隸將之兵並更互出戍,稍省諸路鈐轄及都監員,仍以將官兼州都監職事,卒不能盡罷將、副,如光等言。其年八月,樞密院言,近邊州軍及邊使經由道路,而減本處兵官,非是。於是邊州及人使經由道路,將官仍不兼都監。



 至紹聖間,樞密院言:「往時軍士犯法,將官得專決遣,故事無留滯。自州縣官預軍事以來,動多牽制,不得自裁。欲仍依舊法,及諸軍除轉排補,並隸將司,州縣無得輒預。其非
 屯駐所在,當俟將、副巡歷決之,餘委訓練官行焉。」詔從之。至是,州縣一無關預,兵愈驕,無復可用矣。



 元符元年,章楶又請增置涇原第十二將。



 宣和元年,詔非救護水火、收捕奸細妖人而輒差將兵者,坐之。後三年,知婺州楊應誠言:「諸路屯戍,當隸守臣,兵民之任一,然後號令不二。不然,將驕卒橫,侵漁細民,氣壓州郡,有不勝其憂者。」於是詔自今令隸守臣。無何,復詔曰:「將兵遵將官條教,除前隸守臣指揮。」其後,江、浙盜起,攻陷州邑,東南將
 兵,望風逃潰,無復能戰。事平,童貫奏言:「東南三將,類皆孱弱,全不知戰,虛費糧廩,驕隋自恣。平時主領占差營私,大半皆習工藝。遂致寇盜橫行,毒流一方,重費經畫。今事平之後,當添將增兵,鎮遏綏馭。然南人怯弱,素失訓練,終不堪戰。今欲於內郡別置三將,並隨京畿將分接續排置,使陜西軍更互戍守。庶幾東南可得實戰之士,於計為便。」詔從之。其後南渡諸屯駐大軍即舊將兵之類,而其駐扎之所則異於前矣。



 今摭建炎以後將兵列
 於屯駐大軍之次,而建炎水軍亦附見焉。



 建炎後諸屯駐大軍武鋒、精銳、敢勇、鎮淮、強勇、雄勝、武定、江都振武、泰熙振武、忠勇、游奕、淮陰前軍、副司左右軍、移戍左軍。



 淮東滁州:雄勝、安淮、青平小雄邊。



 淮東泰州:鎮江左軍。



 淮西廬州;強勇前軍、強勇右軍、武定、游奕、忠義、雄邊、全年。



 淮西濠州:武定選鋒軍、武定後軍、使效、威勝、游擊、義士諸軍、定遠武定。



 淮西安豐軍:武定前軍、武定右軍、防城戍軍、四色軍。



 淮西無為軍巢縣:池司
 右軍。



 淮西黃州:雄關飛虎軍。



 臨安府屯駐諸軍:雄節、威果、全捷、龍騎、歸遠。



 金州駐扎都統司兵。



 成都路安撫副司駐扎兵。



 四川大制司帳前飛捷軍。



 利州節制司諸軍。



 金州忠義軍。



 閬州節制司諸軍。



 潼川府制帳踏白軍。



 隆慶屯駐游奕軍。



 潼川安撫司忠定軍。



 夔州節制司軍。



 興元節制軍事利州都統司兵。



 四川制司帳前、信義兩軍。



 興元都統司屯駐合州軍、沔州乾道三年,三百人。



 沿江水軍建炎置。



 明州水軍紹興置。乾道元年,二千人,分左、右兩將。



 福州荻蘆、延祥砦紹興
 置,百五十人。乾道七年添招。凡五千人。



 鎮江駐扎御前水軍乾道三年,招三百人,淳熙五年增招千五百人。



 沿海水軍乾道六年置,一千人。



 潮州水軍乾道四年置,二百人。



 江陰水軍乾道四年置,三百人。



 廣東水軍乾道五年,增至二千人。



 平江許浦水軍乾道七年,七千人,淳熙五年,增五百人。



 江州水軍淳熙三年,招一千人。



 池州都統司水軍淳熙元年千人,嘉定中增至三千人。



 漳州水軍紹熙元年,漳、泉共六百人。



 泉州水軍見上。



 殿前澉浦水軍開禧元年,一千五百人。



 鄂州都統司水軍開禧十五年。



 太平州採石駐扎御前水軍嘉定十四年,五千人。



 建康都統司靖安水軍元隸都統司,嘉定中隸御前。



 馬軍行司唐灣水軍
 元隸馬軍行司,嘉定中隸御前。



 通州水軍乾道五年置。



 池州清溪雁汊控海水軍建炎四年置,百五十人。



 兩淮水軍紹興元年置,二千人。隆興元年,詔諸州斷配海賊刺隸。



\end{pinyinscope}