\article{列傳第一百三十三}

\begin{pinyinscope}

 張九成胡銓廖剛李迨趙開



 張九成,字子韶,其先開封人,徙居錢塘。游京師,從楊時學。權貴托人致幣曰:「肯從吾游,當薦之館閣。」九成笑曰:「
 王良尚羞與嬖奚乘,吾可為貴游客耶?」



 紹興二年,上將策進士,詔考官,直言者置高等。九成對策略曰:「禍亂之作,天所以開聖人也。願陛下以剛大為心,無以憂驚自沮。臣觀金人有必亡之勢,中國有必興之理。夫好戰必亡,失其故俗必亡,人心不服必亡,金皆有焉。劉豫背叛君親,委身夷狄,黠雛經營,有同兒戲,何足慮哉。前世中興之主,大抵以剛德為尚。去讒節欲,遠佞防奸,皆中興之本也。今閭巷之人皆知有父兄妻子之樂,陛下貴為
 天子,冬不得溫,夏不得清,昏無所定,晨無所省,感時遇物,淒惋於心,可不思所以還二聖之車乎?」又言:「閹寺聞名,國之不祥也,今此曹名字稍稍有聞,臣之所憂也。當使之安掃除之役,凡結交往來者有禁,干預政事者必誅。」擢置首選。楊時遺九成書曰:「廷對自中興以來未之有,非剛大之氣,不為得喪回屈,不能為也。」



 授鎮東軍簽判,吏不能欺。民冒鹺禁,提刑張宗臣欲逮捕數十人,九成爭之。宗臣曰:「此事左相封來。」九成曰:「主上屢下恤刑
 之詔,公不體聖意而觀望宰相耶?」宗臣怒,九成即投檄歸。從學者日眾,出其門者多為聞人。



 趙鼎薦於朝,遂以太常博士召。既至,改著作佐郎,遷著作郎,言:「我宋家法,曰仁而已。仁之發見,尤在於刑。陛下以省刑為急,而理官不以恤刑為念。欲詔理官,活幾人者與減磨勘。」從之。除浙東提刑,力辭,乃與祠以歸。



 未幾,召除宗正少卿、權禮部侍郎兼侍講,兼權刑部侍郎。法寺以大闢成案上,九成閱始末得其情,因請覆實,囚果誣服者。朝論欲以
 平反為賞,九成曰:「職在詳刑,可邀賞乎?」辭之。



 金人議和,九成謂趙鼎曰:「金實厭兵,而張虛聲以撼中國。」因言十事,彼誠能從吾所言,則與之和,使權在朝廷。鼎既罷,秦檜誘之曰:「且成檜此事。」九成曰:「九成胡為異議,特不可輕易以茍安耳。」檜曰:「立朝須優游委曲。」九成曰:「未有枉己而能直人。」上問以和議,九成曰:「敵情多詐,不可不察。」



 因在經筵言西漢災異事,檜甚惡之,謫守邵州。既至,倉庫虛乏,僚屬請督酒租宿負、苗絹未輸者,九成曰:「縱未
 能惠民,其敢困民耶?」是歲,賦入更先他時。中丞何鑄言其矯偽欺俗,傾附趙鼎,落職。



 丁父憂,既免喪,秦檜取旨,上曰:「自古朋黨畏人主知之,此人獨無所畏,可與宮觀。」先是,徑山僧宗杲善談禪理,從游者眾,九成時往來其間。檜恐其議己,令司諫詹大方論其與宗杲謗訕朝政,謫居南安軍。在南安十四年,每執書就明,倚立庭磚,歲久雙趺隱然。廣帥致籯金,九成曰:「吾何敢茍取。」悉歸之。檜死,起知溫州。戶部遣吏督軍糧,民苦之,九成移書痛
 陳其弊,戶部持之,九成即丐祠歸。數月,病卒。



 九成研思經學,多有訓解,然早與學佛者游,故其議論多偏。寶慶初,特贈太師,封崇國公,謚文忠。



 胡銓,字邦衡,廬陵人。建炎二年,高宗策士淮海,銓因御題問「治道本天,天道本民」,答云:「湯、武聽民而興,桀、紂聽天而亡。今陛下起干戈鋒鏑間,外亂內訌,而策臣數十條,皆質之天,不聽於民。」又謂:「今宰相非晏殊,樞密、參政非韓琦、杜衍、範仲淹。」策萬餘言,高宗見而異之,將冠之
 多士,有忌其直者,移置第五。授撫州軍事判官,未上,會隆祐太后避兵贛州,金人躡之,銓以漕檄攝本州幕,募鄉丁助官軍捍禦,第賞轉承直郎。丁父憂,從鄉先生蕭楚學《春秋》。



 紹興五年,張浚開督府,闢湖北倉屬,不赴。有詔赴都堂審察,兵部尚書呂祉以賢良方正薦,賜對,除樞密院編修官。



 八年,宰臣秦檜決策主和,金使以「詔諭江南」為名,中外洶洶。銓抗疏言曰:



 臣謹案,王倫本一狎邪小人,市井無賴,頃緣宰相無識,遂舉以使虜。專務詐
 誕,欺罔天聽,驟得美官,天下之人切齒唾罵。今者無故誘致虜使,以「詔諭江南」為名,是欲臣妾我也,是欲劉豫我也。劉豫臣事醜虜,南面稱王,自以為子孫帝王萬世不拔之業,一旦豺狼改慮,捽而縛之,父子為虜。商鑒不遠,而倫又欲陛下效之。夫天下者祖宗之天下也,陛下所居之位,祖宗之位也。奈何以祖宗之天下為金虜之天下,以祖宗之位為金虜藩臣之位!陛下一屈膝,則祖宗廟社之靈盡污夷狄,祖宗數百
 年之赤子盡為左衽,朝廷宰執盡為陪臣,天下士大夫皆當裂冠毀冕,變為胡服。異時豺狼無厭之求,安知不加我以無禮如劉豫也哉?



 夫三尺童子至無識也,指犬豕而使之拜,則怫然怒。今醜虜則犬豕也,堂堂大國,相率而拜犬豕,曾童孺之所羞,而陛下忍為之耶?倫之議乃曰:「我一屈膝則梓宮可還,太后可復,淵聖可歸,中原可得。」嗚呼!自變故以來,主和議者誰不以此說啖陛下哉!然而卒無一驗,則虜之情偽已可知矣。而陛下尚不覺悟,竭民膏血而不
 恤,忘國大仇而不報,含垢忍恥,舉天下而臣之甘心焉。就令虜決可和,盡如倫議,天下後世謂陛下何如主?況醜虜變詐百出,而倫又以奸邪濟之,梓宮決不可還,太后決不可復,淵聖決不可歸,中原決不可得,而此膝一屈不可復伸,國勢陵夷不可復振,可為痛哭流涕長太息矣!



 向者陛下間關海道,危如累卵,當時尚不忍北面臣虜,況今國勢稍張,諸將盡銳,士卒思奮。只如頃者醜虜陸梁,偽豫入寇,固嘗敗之於襄陽,敗之於淮上,敗之
 於渦口,敗之於淮陰,校之往時蹈海之危,固已萬萬,償不得已而至於用兵,則我豈遽出虜人下哉?今無故而反臣之,欲屈萬乘之尊,下穹廬之拜,三軍之士不戰而氣已索。此魯仲連所以義不帝秦,非惜夫帝秦之虛名,惜天下大勢有所不可也。今內而百官,外而軍民,萬口一談,皆欲食倫之肉。謗議洶洶,陛下不聞,正恐一旦變作,禍且不測。臣竊謂不斬王倫,國之存亡未可知也。



 雖然,倫不足道也,秦檜以腹心大臣而亦為之。陛下有堯、
 舜之資,檜不能致君如唐、虞,而欲導陛下為石晉,近者禮部侍郎曾開等引古誼以折之,檜乃厲聲責曰:「侍郎知故事,我獨不知!」則檜之遂非愎諫,已自可見,而乃建白令臺諫、侍臣僉議可否,是蓋畏天下議己,而令臺諫、侍臣共分謗耳。有識之士皆以為朝廷無人,籲,可惜哉!



 孔子曰:「微管仲,吾其被發左衽矣。」夫管仲,霸者之佐耳,尚能變左衽之區,而為衣裳之會。秦檜,大國之相也,反驅衣冠之俗,而為左衽之鄉。則檜也不唯陛下之罪人,
 實管仲之罪人矣。孫近傅會檜議,遂得參知政事,天下望治有如饑渴,而近伴食中書,漫不敢可否事。檜曰虜可和,近亦曰可和;檜曰天子當拜,近亦曰當拜。臣嘗至政事堂,三發問而近不答,但曰:「已令臺諫、侍從議矣。」嗚呼!參贊大政,徒取充位如此。有如虜騎長驅,尚能折沖禦侮耶?臣竊謂秦檜、孫近亦可斬也。



 臣備員樞屬,義不與檜等共戴天,區區之心,願斷三人頭,竿之蒿街,然後羈留虜使,責以無禮,徐興問罪之師,則三軍之士不戰
 而氣自倍。不然,臣有赴東海而死爾,寧能處小朝廷求活邪!



 書既上,檜以銓狂妄兇悖,鼓眾劫持,詔除名,編管昭州,仍降詔播告中外。給、舍、臺諫及朝臣多救之者,檜迫於公論,乃以銓監廣州鹽倉。明年,改簽書威武軍判官。十二年,諫官羅汝楫劾銓飾非橫議,詔除名,編管新州。十八年,新州守臣張棣訐銓與客唱酬,謗訕怨望,移謫吉陽軍。



 二十六年,檜死,銓量移衡州。銓之初上書也,宜興進士吳師古鋟木傳之,金人募其書千金。其謫廣
 州也,朝士陳剛中以啟事為賀。其謫新州也,同郡王延珪以詩贈行。皆為人所訐,師古流袁州,廷珪流辰州,剛中謫知虔州安遠縣,遂死焉。三十一年,銓得自便。



 孝宗即位,復奉議郎、知饒州。召對,言修德、結民、練兵、觀釁,上曰:「久聞卿直諒。」除吏部郎官。隆興元年,遷秘書少監,擢起居郎,論史官失職者四:一謂記注不必進呈,庶人主有不觀史之美;二謂唐制二史立螭頭之下,今在殿東南隅,言動未嘗得聞;三謂二史立後殿,而前殿不立,乞
 於前後殿皆分日侍立;四謂史官欲其直前,而閣門以未嘗預牒,以今日無班次為辭。乞自今直前言事,不必預牒閣門,及以有無班次為拘。詔從之。兼侍講、國史院編修官。因講《禮記》,曰:「君以禮為重,禮以分為重,分以名為重,願陛下無以名器輕假人。」



 又進言乞都建康,謂:「漢高入關中,光武守信都。大抵與人鬥,不搤其亢,拊其背,不能全勝。今日大勢,自淮以北,天下之亢與背也,建康則搤之拊之之地也。若進據建康,下臨中原,此高、光興
 王之計也。」



 詔議行幸,言者請紓其期,遂以張浚視師圖恢復,侍御史王十朋贊之。克復宿州,大將李顯忠私其金帛,且與邵宏淵忿爭,軍大潰。十朋自劾。上怒甚,銓上疏願毋以小衄自沮。



 時旱蝗、星變,詔問政事闕失,銓應詔上書數千言,始終以《春秋》書災異之法,言政令之闕有十,而上下之情不合亦有十,且言:「堯、舜明四目,達四聰,雖有共、鯀,不能塞也。秦二世以趙高為腹心,劉、項橫行而不得聞;漢成帝殺王章,王氏移鼎而不得聞;靈帝
 殺竇武、陳蕃,天下橫潰而不得聞;梁武信朱異,侯景斬關而不得聞;隋煬帝信虞世基,李密稱帝而不得聞;唐明皇逐張九齡,安、史胎禍而不得聞。陛下自即位以來,號召逐客,與臣同召者張燾、辛次膺、王大寶、王十朋,今燾去矣,次膺去矣,十朋去矣,大寶又將去,惟臣在爾。以言為諱,而欲塞災異之源,臣知其必不能也。」



 銓又言:「昔周世宗為劉旻所敗,斬敗將何徽等七十人,軍威大震,果敗旻,取淮南,定三關。夫一日戮七十將,豈復有將可
 用?而世宗終能恢復,非庸懦者去,則勇敢者出耶!近宿州之敗,士死於敵者滿野,而敗軍之將以所得之金賂權貴以自解,上天見變昭然,陛下非信賞必罰以應天不可。」其論納諫曰:「今廷臣以箝默為賢,容悅為忠。馴至興元之幸,所謂『一言喪邦』。」上曰:「非卿不聞此。」



 金人求成,銓曰:「金人知陛下銳意恢復,故以甘言款我,願絕口勿言『和』字。」上以邊事全倚張浚,而王之望、尹穡專主和排浚,銓廷責之。兼權中書舍人、同修國史。張浚之子栻賜
 金紫,銓繳奏之,謂不當如此待勛臣子。浚雅與銓厚,不顧也。



 十一月,詔以和戎遣使,大詢於庭,侍從、臺諫預議者凡十有四人。主和者半,可否者半,言不可和者銓一人而已,乃獨上一議曰:「京師失守自耿南仲主和,二聖播遷自何□主和,維揚失守自汪伯彥、黃潛善主和,完顏亮之變自秦檜主和。議者乃曰:『外雖和而內不忘戰。』此向來權臣誤國之言也。一溺於和,不能自振,尚能戰乎?」除宗正少卿,乞補外,不許。



 先是,金將蒲察徒穆、大周
 仁以泗州降,蕭琦以軍百人降,詔並為節度使。銓言:「受降古所難,六朝七得河南之地,不旋踵而皆失;梁武時侯景以河南來奔,未幾而陷臺城;宣、政間郭藥師自燕雲來降,未幾為中國患。今金之三大將內附,高其爵祿,優其部曲,以系中原之心,善矣。然處之近地,萬一包藏禍心,或為內應,後將噬臍,願勿任以兵柄,遷其眾於湖、廣以絕後患。」



 二年,兼國子祭酒,尋除權兵部侍郎。八月,上以災異避殿減膳,詔廷臣言闕政急務。銓以振災為
 急務,議和為闕政,其議和之書曰:



 自靖康迄今凡四十年,三遭大變,皆在和議,則醜虜之不可與和,彰彰然矣。肉食鄙夫,萬口一談,牢不可破。非不知和議之害,而爭言為和者,是有三說焉:曰偷懦,曰茍安,曰附會。偷懦則不知立國,茍安則不戒鴆毒,附會則覬得美官,小人之情狀具於此矣。



 今日之議若成,則有可吊者十;若不成,則有可賀者亦十。請為陛下極言之。何謂可吊者十?



 真宗皇帝時,宰相李沆謂王旦曰:「我死,公必為相,切勿與虜講
 和。吾聞出則無敵國外患,如是者國常亡,若與虜和,自此中國必多事矣。」旦殊不以為然。既而遂和,海內乾耗,旦始悔不用文靖之言。此可吊者一也。



 中原謳吟思歸之人,日夜引領望陛下拯溺救焚,不啻赤子之望慈父母,一與虜和,則中原絕望,後悔何及。此可吊者二也。



 海、泗今日之藩籬咽喉也,彼得海、泗,且決吾藩籬以瞰吾室,扼吾咽喉以制吾命,則兩淮決不可保。兩淮不保,則大江決不可守,大江不守,則江、浙決不可安。此可
 吊者三也。



 紹興戊午,和議即成,檜建議遣二三大臣如路允迪等,分往南京等州交割歸地。一旦叛盟,劫執允迪等,遂下親征之詔,虜復請和。其反復變詐如此,檜猶不悟,奉之如初,事之愈謹,賂之愈厚,卒有逆亮之變,驚動輦轂。太上謀欲入海,行朝居民一空,覆轍不遠,忽而不戒,臣恐後車又將覆也。此可吊者四也。



 紹興之和,首議決不與歸正人,口血未幹,盡變前議。凡歸正之人一切遣還,如程師回、趙良嗣等聚族數百,幾為蕭墻憂。今
 必盡索歸正之人,與之則反側生變,不與則虜決不肯但已。夫反側則肘腋之變深,虜決不肯但已,則必別起釁端,猝有逆亮之謀,不知何以待之。此可吊者五也。



 自檜當國二十年間,竭民膏血以餌犬羊,迄今府庫無旬月之儲,千村萬落生理蕭然,重以蝗蟲水潦。自此復和,則蠹國害民,殆有甚焉者矣。此可吊者六也。



 今日之患,兵費已廣,養兵之外又增歲幣,且少以十年計之,其費無慮數千億。而歲幣之外,又有私覿之費;私覿之外,又
 有賀正、生辰之使;賀正、生辰之外,又有泛使。一使未去,一使復來,生民疲於奔命,帑廩涸於將迎,瘠中國以肥虜,陛下何憚而為之。此其可吊者七也。



 側聞虜人嫚書,欲書御名,欲去國號「大」字,欲用再拜。議者以為繁文小節不必計較,臣切以為議者可斬也。夫四郊多壘,卿大夫之辱;楚子問鼎,義士之所深恥;「獻納」二字,富弼以死爭之。今醜虜橫行與多壘孰辱?國號大小與鼎輕重孰多?「獻納」二字與再拜孰重?臣子欲君父屈己以從之,則
 是多壘不足辱,問鼎不必恥,「獻納」不必爭。此其可吊者八也。



 臣恐再拜不已必至稱臣,稱臣不已必至請降,請降不已必至納土,納土不已必至銜壁,銜壁不已必至輿櫬,輿櫬不已必至如晉帝青衣行酒然後為快。此其可吊者九也。



 事至於此,求為匹夫尚可得乎?此其可吊者十也。



 竊觀今日之勢,和決不成,儻乾剛獨斷,追回使者魏杞、康湑等,絕請和之議以鼓戰士,下哀痛之詔以收民心,天下庶乎其可為矣。如此則有可賀者亦十:省
 數千億之歲幣,一也;專意武備,足食足兵,二也;無書名之恥,三也;無去「大」之辱,四也;無再拜之屈,五也;無稱臣之忿,六也;無請降之禍,七也;無納土之悲,八也;無銜璧、輿櫬之酷,九也;無青衣行酒之冤,十也。



 去十吊而就十賀,利害較然,雖三尺童稚亦知之,而陛下不悟。《春秋左氏》謂無勇者為婦人,今日舉朝之士皆婦人也。如以臣言為不然,乞賜流放竄殛,以為臣子出位犯分之戒。



 自符離之敗,朝論急於和戎,棄唐、鄧、海、泗四州與虜矣。金
 又欲得商、秦地,邀歲幣,留使者魏杞,分兵攻淮。以本職措置浙西、淮東海道。



 時金使僕散忠義、紇石烈志寧之兵號八十萬,劉寶棄楚州,王彥棄昭關,濠、滁皆陷。惟高郵守臣陳敏拒敵射陽湖,而大將李寶預求密詔為自安計,擁兵不救。銓劾奏之,曰:「臣受詔令範榮備淮,李寶備江,緩急相援。今寶視敏弗救,若射陽失守,大事去矣。」寶懼,始出師掎角。時大雪,河冰皆合,銓先持鐵錘錘冰,士皆用命,金人遂退。久之,提舉太平興國宮。



 乾道初,以
 集英殿修撰知漳州,改泉州。趣奏事,留為工部侍郎。入對,言:「少康以一旅復禹績,今陛下富有四海,非特一旅,而即位九年,復禹之效尚未赫然。」又言:「四方多水旱,左右不以告,謀國者之過也,宜令有司速為先備。」乞致仕。



 七年,除寶文閣待制,留經筵。求去,以敷文閣直學士與外祠。陛辭,猶以歸陵寢、復故疆為言,上曰:「朕志也。」且問今何歸,銓曰:「歸廬陵,臣向在嶺海嘗訓傳諸經,欲成此書。」特賜通天犀帶以寵之。



 銓歸,上所著《易》、《春秋》、《周禮》、《禮
 記解》,詔藏秘書省。尋復元官,升龍圖閣學士、提舉太平興國宮,轉提舉玉隆萬壽宮,進端明殿學士。六年,召歸經筵,銓引疾力辭。七年,以資政殿學士致仕。薨,謚忠簡。有《澹庵集》一百卷行於世。孫槻、矩,皆至尚書。



 廖剛,字用中,南劍州順昌人。少從陳瓘、楊時學。登崇寧五年進士第。宣和初,自漳州司錄除國子錄,擢監察御史。時蔡京當國,剛論奏無所避。以親老求補外,出知興化軍。欽宗即位,以右正言召。丁父憂,服闋,除工部員外
 郎,以母疾辭。



 紹興元年,盜起旁郡,官吏悉逃去,順昌民以剛為命。剛諭從盜者使反業,既而他盜入順昌,部使者檄剛撫定。剛遣長子遲諭賊,賊知剛父子有信義,亦散去。除本路提點刑獄。



 尋召為吏部員外郎,言:「古者天子必有親兵自將,所以備不虞而強主威,漢北軍、唐神策之類也。祖宗軍制尤嚴。願稽舊制,選精銳為親兵,居則以為衛,動則以為中軍,此強幹弱枝之道。」又言:「國家艱難已極,今方圖新,若會稽誠非久駐之地。請經營建
 康,親擁六師往為固守計,以杜金人窺伺之意。」遷起居舍人、權吏部侍郎兼侍講,除給事中。



 丁母憂,服闋,復拜給事中。剛言:「國不可一日無兵,兵不可一日無食。今諸將之兵備江、淮,不知幾萬,初無儲蓄,日待哺於東南之轉餉,浙民已困,欲救此患莫若屯田。」因獻三說,將校有能射耕,當加優賞,每耕田一頃,與轉一資;百姓願耕,假以糧種,復以租賦。上令都督府措置。



 時朝廷推究章惇、蔡卞誤國之罪,追貶其身,仍詔子孫毋得官中朝。至是
 章傑自崇道觀知婺州,章僅自太府丞提舉江東茶鹽事。剛封還詔書,謂即如此,何以示懲,乃並與祠。權戶部侍郎,尋遷刑部侍郎。求補外,除徽猷閣直學士、知漳州。



 七年二月,日有食之,詔內外官言事。剛言:「陛下有建國之封,所以承天意、示大公於天下後世者也,然而未遂正名者,豈非有所待耶?有所待,則是應天之誠未至也。願陛下昭告藝祖在天之靈,正建國儲君之號,布告中外,不匿厥旨。異時雖百斯男,不復更易,天下孰敢不服。」
 上讀之聳然,即召剛趣至闕,拜御史中丞。剛言:「臣職糾奸邪,當務大體,若捃摭細故,則非臣本心。」又奏經費不支,盜賊不息,事功不立,命令不孚,及兵驕官冗之弊。



 時徽宗已崩,上遇朔望猶率群臣遙拜淵聖,剛言:「禮有隆殺,兄為君則君之,己為君則兄之可也。望勉抑聖心,但歲時行家人禮於內庭。」從之。



 殿前司強刺民為兵,及大將恃功希恩,所請多廢法。剛知無不言,論列至於四五,驕橫者肅然。



 鄭億年與秦檜有連而得美官,剛顯疏其
 惡,檜銜之。金人叛盟,剛乞起舊相之有德望者,處以近藩,檜聞之曰:「是欲置我何地耶?」改工部尚書,而以王次翁為中丞。初,邊報至,從官會都堂,剛謂億年曰:「公以百口保金人,今已背約,有何面目尚在朝廷乎?」億年奉祠去。次翁與右諫議何鑄劾剛薦劉昉、陳淵,相為朋比,以徽猷閣直學士提舉亳州明道宮。明年致仕。以紹興十三年卒。



 子四人:遲、過、遂、遽,仕皆秉麾節,邦人號為「萬石廖氏」。



 李迨,東平人也。曾祖參,仕至尚書右丞。迨未冠入太學,因居開封。以蔭補官,初調渤海縣尉。



 時州縣團結民兵,民起田畝中,不閑坐作進退之節,或嘩不受令,迨立賞罰以整齊之,累月皆精練,部伍如法。部刺史按閱,無一人亂行伍者,遂薦之朝,改合入官。累遷通判濟州。



 時高宗以大元帥過濟,郡守自以才不及,遜迨行州事,迨應辦軍須無闕。會大元帥府勸進,乘輿儀物皆未備,迨諳熟典故,裁定其制,不日而辦。上深嘆賞,即除隨軍輦運。



 上即位於南京,授山東輦運,改金部郎。從駕至維揚,敵犯行在所,即取金部籍有關於國家經賦之大者載以行,及上於鎮江。時建炎三年二月也。宰相呂頤浩言於上,即日召見。



 未幾,丁父喪,詔起復,以中散大夫直龍圖閣,為御營使司參議官兼措置軍前財用。苗傅、劉正彥叛,呂頤浩、張浚集勤王之師,迨流涕謂諸將曰:「君第行,無慮軍食。」師行所至,食皆先具。事平,同趙哲等入對,上慰勞之。詔轉三官,辭不拜,除權戶部侍郎。



 四年,加顯謨
 閣待制,為淮南、江、浙、荊湖等路制置發運使。尋以軍旅甫定,乞持餘服,詔許之。紹興二年,知筠州。明年,移信州,尋提舉江州太平觀。



 五年十月,以舊職除兩浙路轉運使,言:「祖宗都大梁,歲漕東南六百餘萬斛,而六路之民無飛挽之擾,蓋所運者官舟,所役者兵卒故也。今駐蹕浙右,漕運地裡不若中都之遠,而公私苦之,何也?以所用之舟太半取於民間,往往鑿井沉船以避其役。如溫、明、虔、吉州等處所置造船場,乞委逐州守臣措置,募兵
 卒牽挽,使臣管押,庶幾害不及民,可以漸復漕運舊制。」詔工部措置。尋加徽猷閣直學士,升龍圖閣直學士,為四川都轉運使兼提舉成都等路茶事,並提舉陜西等路買馬。



 自熙、豐以來,始即熙、秦、戎、黎等州置場買馬,而川茶通於永興四路,故成都府、秦州皆有榷茶司。至是關陜既失,迨請合為一司,名都大提舉茶馬司,以省冗費,從之。逾年,詔迨以每歲收支之數具旁通驛奏,迨乃考其本末,具奏曰:



 紹興四年,所收錢物三千三百四十
 二萬餘緡,比所支闕五十一萬餘緡。五年,收三千六十萬緡,比所支闕一千萬餘緡。六年,未見。七年,所收三千六百六十萬餘緡,比所支闕一百六十一萬餘緡。自來遇歲計有闕,即添支錢引補助。紹興四年,添印五百七十六萬道。五年,添印二百萬道。六年,添印六百萬道。見今泛料太多,引價頓落,緣此未曾添印。兼歲收錢物內有上供、進奉等窠名一千五百九十九萬,系四川歲入舊額。其勸諭、激賞等項窠名錢物共二千六十八萬,系
 軍興後來歲入所增,比舊額已過倍,其取於民可謂重矣。



 臣嘗考《劉晏傳》,是時天下歲入緡錢千二百萬,而管榷居其半。今四川榷鹽榷酒歲入一千九十一萬,過於晏所榷多矣。諸窠名錢已三倍劉晏歲入之數,彼以一千二百萬贍中原之軍而有餘,今以三千六百萬貫贍川、陜一軍而不足。又如折估及正色米一項,通計二百六十五萬石。止以紹興六年朝廷取會官兵數,計六萬八千四百四十九人,決無一年用二百六十五萬石米
 之理。數內官員一萬一千七員,軍兵五萬七百四十九人,官員之數比軍兵之數約計六分之一。軍兵請給錢比官員請給不及十分之一,即是冗濫在官員,不在軍兵也。計司雖知冗濫,力不能裁節之,雖是寬剩,亦未敢除減,此朝廷不可不知也。



 蜀人所苦甚者,糴買、般運也。蓋糴買不科敷則不能集其事,茍科敷則不能無擾;般運事稍緩則船戶獨受其弊,急則稅戶皆被其害。欲省漕運莫如屯田,漢中之地約收二十五萬餘石,若將一
 半充不系水運去處歲計米,以一半對減川路糴買、般發歲計米,亦可少寬民力。兼臣已委官於興元、洋州就糴夏麥五十萬石,岷州欲就糴二十萬石,兼用營田所收一半之數十二萬石,三項共計五十七萬石。每年水運應付閬、利州以東計米五十八萬石,若得此三項,可盡數免川路糴買、般運,此乃恤民之實惠,守邊之良策也。



 降詔獎諭,以與吳玠不合,與祠。



 九年,金人歸我三京,命迨為京畿都轉運使。孟庾時為權東京留守,潛通北
 使。迨察其隱微,庾不能平,訟於朝,且使人告迨曰:「北人以兵至矣。」迨曰:「吾家食國家祿二百年,荷陛下重任,萬死不足報。吾老矣,豈能下穹廬之拜乎?首可斷而膝不可屈也。如果然,吾將極罵以死。」告者悚然而去。降聖節,庾失於行禮,為迨所持,庾自劾,迨因此求罷去,乃落職與祠歸,而庾以京師降於金人。



 迨尋復龍圖閣待制、知洪州。十六年,以疾丐祠。十八年卒。



 趙開,字應祥,普州安居人。登元符三年進士第。大觀二
 年,權闢NU正。用舉者改秩,即盡室如京師,買田尉氏,與四方賢俊游,因詗知天下利病所當罷行者。如是七年,慨然有通變救弊志。



 宣和初,除禮制局校正檢閱官。數月局罷,出知鄢陵縣。七年,除講議司檢詳官。開善心計,自檢詳罷,除成都路轉運判官,遂奏罷宣和六年所增上供認額綱布十萬匹,減綿州下戶支移利州水腳錢十分之三,又減蒲江六井元符至宣和所增鹽額,列其次第,謂之「鼠尾帳」,揭示鄉戶歲時所當輸折科等實
 數,俾人人具曉,鄉胥不得隱匿竄寄。



 嘗言:「財利之源當出於一,祖宗朝天下財計盡歸三司,諸道利源各歸漕計,故官省事理。並廢以還,漕司則利害可以參究,而無牽掣窒礙之患矣。」因指陳榷茶、買馬五害,大略謂:「黎州買馬,嘉祐歲額才二千一百餘。自置司榷茶,歲額四千,且獲馬兵逾千人,猶不足用,多費衣糧,為一害。嘉祐以銀絹博馬,價皆有定。今長吏旁緣為奸,不時歸貨,以空券給夷人,使待資次,夷人怨恨,必生邊患,為二害。初置司
 榷茶,借本錢於轉運司五十二萬緡,於常平司二十餘萬緡。自熙寧至今幾六十年,舊所借不償一文,而歲借乃準初數,為三害。榷茶之初,預俵茶戶本錢,尋於數外更增和買,或遂抑預俵錢充和買,茶戶坐是破產,而官買歲增。茶日濫雜,官茶既不堪食,則私販公行,刑不能禁,為四害。承平時,蜀茶之入秦者十幾八九,猶患積壓難售。今關、隴悉遭焚蕩,仍拘舊額,竟何所用?茶兵官吏坐縻衣糧,未免科配州縣,為五害。請依嘉祐故事,盡罷
 榷茶,仍令轉運司買馬,即五害並去,而邊患不生。如謂榷茶未可遽罷,亦宜並歸轉運司,痛減額以蘇茶戶,輕立價以惠茶商,如此則私販必衰,盜賊消弭,本錢既常在,而息錢自足。」



 朝廷是其言,即擢開都大提舉川、陜茶馬事,使推行之。時建炎二年也。於是大更茶馬之法,官買官賣茶並罷,參酌政和二年東京都茶務所創條約,印給茶引,使茶商執引與茶戶自相貿易。改成都舊買賣茶場為合同場買引所,仍於合同場置茶市,交易者
 必由市,引與茶必相隨。茶戶十或十五共為一保,並籍定茶鋪姓名,互察影帶販鬻者。凡買茶引,每一斤春為錢七十,夏五十,舊所輸市例頭子錢並依舊。茶所過每一斤徵一錢,住徵一錢半。其合同場監官除驗引、秤茶、封記、發放外,無得干預茶商、茶戶交易事。



 舊制買馬及三千匹者轉一官,比但以所買數推賞,往往有一任轉數官者。開奏:「請推賞必以馬到京實收數為格,或死於道,黜降有差。」比及四年冬,茶引收息至一百七十餘萬
 緡,買馬乃逾二萬匹。



 張浚以知樞密院宣撫川蜀,素知開善理財,即承制以開兼宣撫處置使司隨軍轉運使,專一總領四川財賦。開見浚曰:「蜀之民力盡矣,錙銖不可加,獨榷貨稍存贏餘,而貪猾認為己有,互相隱匿。惟不恤怨詈,斷而敢行,庶可救一時之急。」



 浚銳意興復,委任不疑,於是大變酒法,自成都始。先罷公使賣供給酒,即舊撲買坊場所置隔槽,設官主之,曲與釀具官悉自買,聽釀戶各以米赴官場自釀,凡一石米輸三千,並頭
 子雜用等二十二。其釀之多寡,惟錢是視,不限數也。明年,遂遍四路行其法。又法成都府法,於秦州置錢引務,興州鼓鑄銅錢,官賣銀絹,聽民以錢引或銅錢買之。凡民錢當入官者,並聽用引折納,官支出亦如之。民私用引為市,於一千並五百上許從便增高其直,惟不得減削。法既流通,民以為便。



 初,錢引兩料通行才二百五十萬有奇,至是添印至四千一百九十餘萬,人亦不厭其多,價亦不削。



 宣司獲偽引三十萬,盜五十人,浚欲從有
 司議當以死,開白浚曰:「相君誤矣。使引偽,加宣撫使印其上即為真。黥其徒使治幣,是相君一日獲三十萬之錢,而起五十人之死也。」浚稱善,悉如開言。



 最後又變鹽法,其法實視大觀東南、東北鹽鈔條約,置合同場鹽市,與茶法大抵相類。鹽引每一斤納錢二十五,土產稅有增添等共納九錢四分,所過每斤徵錢七分,住徵一錢五分,若以錢引折納,別輸稱提勘合錢共六十。初變榷法,怨詈四起,至是開復議更鹽法,言者遂奏其不便,乞
 罷之以安遠民,且曰:「如謂大臣建請,務全事體,必須更制,即乞札與張浚照會。」詔以其章示浚,浚不為變。



 時浚荷重寄,治兵秦川,經營兩河,旬犒月賞,期得士死力,費用不貲,盡取辦於開,開悉知慮於食貨,算無遺策,雖支費不可計,而贏貲若有餘。



 吳玠為四川宣撫副使,專治戰守,於財計盈虛未嘗問,惟一切以軍期趣辦,與開異趣。玠數以餉饋不繼訴於朝,開亦自劾老憊,丐去。朝廷未許,乃特置四川安撫制置大使之名,命席益為之。益
 前執政,詔位宣撫司上,朝論恐未安,仍詔張浚視師荊、襄、川、陜。



 六年,罷綿州宣撫司,玠仍以宣撫治兵事,軍馬聽玠移撥,錢物則委開拘收。尋除開徽猷閣待制,加玠兩鎮節鉞。復降旨,都轉運使不當與四路漕臣同系銜,成都、潼川兩路漕臣與都轉運使坐應副軍支錢物愆期,各貶二秩。朝廷故抑揚之,使之交解間隙、趣辦餉饋也。而開復與席益不和,抗疏乞將舊來宣撫司年計應副軍期,不許他司分擘支用。又指陳宣撫司截都漕運司錢,
 就果、閬糴米非是。又言應副吳玠軍須,紹興四年總為錢一千九百五十五萬七千餘緡,五年視四年又增四百二十萬五千餘緡。蜀今公私俱困,四向無所取給,事屬危急,實甚可憂,氣許以茶馬司奏計詣闕下,盡所欲言。



 朝廷既知開與玠及席益有隙,乃詔開赴行在,以李迨代之。會疾作不行,提舉江州太平觀。七年,復右文殿修撰、都大主管川陜茶馬。開已病,累疏丐去,詔從所乞,提舉太平觀。十一年卒。



 論曰:秦檜執國柄,其誤宋大計,固無以議為也。張九成之策,胡銓之疏,忠義凜然。廖剛請復用德望之人,豈茍阿時好者哉?李迨、趙開所謂可使治其賦也歟?



\end{pinyinscope}