\article{志第八十八 樂十(樂章四)}

\begin{pinyinscope}

 朝謁玉清昭應宮太清宮朝享景靈宮封禪禮汾陰奉天書祭九鼎



 真宗奉聖祖玉清昭應宮禦制十一首



 降聖,《真安》



 巍巍真宇,奕奕殊庭。規模太紫,炳煥丹青。



 元命祗答,大猷是經。多儀有踐,丕應無形。



 肆設金石,聲聞杳冥。佇回飆馭,永祐基扃。



 奉香,《靈安》



 芳氣上浹,飆馭下臨。紹承丕緒,永勵精明。



 氤氳成霧,蔥鬱垂陰。虔恭對越,介祉攸欽。



 奉饌,《吉安》



 發祥有自,介福無疆。紛綸丕應,保祐下方。



 嘉薦斯備,雅奏具揚。寅威洞達,監眄昭章。



 玉皇位酌獻,《慶安》



 無體之體,強名之名。監觀萬寓。統治九清。



 真期保祐,瑞命昭明。幹幹翼翼,祗答財成。



 聖祖位酌獻,《慶安》



 於昭靈貺,誕啟鴻源。功濟庶匯,慶流後昆。



 蘭肴登俎,桂酒盈尊。俯回飆駕,永庇雲孫。



 太祖位酌獻,《慶安》



 赫赫藝祖,受命高穹。威加海外,化浹區中。



 發祥宗祏,錫祐眇沖。欽承積德,勵翼精衷。



 太宗位酌獻,《慶安》



 明明文考,儲精上蒼。禮樂明備,溥率賓王。



 功德累洽,歷數會昌。孝思罔極,丕祐無疆。



 亞、終獻,《沖安》



 太初非有體,至道本無聲。降跡臨下土,成功陟上清。



 至仁敦動植,丕緒啟宗祊。紫禁承來格,
 鴻基保永寧。



 發祥垂誕告,致孝薦崇名。廣樂神欽奉,儲休固太平。



 飲福,《慶安》



 明明始祖,誕啟慶基。翼翼後嗣,虔奉孝思。



 精潔斯達,祉福咸宜。於以報貺,於以受厘。



 徹饌,《吉安》



 雕俎在御,飆駕聞聲。真游斯降,旨酒斯盈。



 大樂雲闋,大禮云成。徹彼常薦,罄此明誠。



 送聖,《真安》



 精心既達,真游允臻。禮容斯舉,福應惟醇。



 將整仙馭,言還上旻。永存嘉貺,用泰烝民。



 迎奉聖像四首並用《慶安》



 玉皇位



 玉虛上帝,金像睟容。宅真雲構,練日龜從。



 維皇對越,率禮寅恭。靈心丕應,福祿來崇。



 聖祖位



 總化在天,保昌厥緒。降格皇闈,瓊輪載御。



 藻仗星陳,睟容金鑄。祐我慶基,宅茲靈宇。



 太祖位



 烝哉大君,聿懷帝祖!鎔範真儀,奉尊靈宇。



 至感祥開,洪輝物睹。瞻謁盡恭,飛英率土。



 太宗位



 於顯神宗,德洽區中。祥金爍冶,範茲睟容。



 殊
 庭胥宇,備物致恭。明威有赫,降福來同。



 王清昭應宮上尊號三首



 奉告,《隆安》



 登隆妙號,欽翼淵宗。茂宣德禮,有恪其容。



 奉璋升薦,垂佩彌恭。揚休詠美,以間笙鏞。



 太初殿奉冊寶,《登安》



 皇靈垂祐,洪福彌隆。祗率綿寓,潔祀真容。



 嚴恭奉冊,對越清躬。睟容肅穆,懿號尊崇。



 禮盛樂舉,福祿來同。



 二聖殿奉絳紗袍,《登安》



 赫赫列聖,威德巍然。彤彤靈
 宇,睟儀在焉。



 奉以龍袞,被之象天。重慶宗稷,億萬斯年。



 太尉奉聖號冊寶,《真安》



 上旻降監,介祉實繁。邦家修報,妙道歸尊。



 增名霄極,奉冊靈軒。茂宣聖典,永祐黎元。



 寶冊升殿,《大安》



 圖書昭錫,典禮紹成。烝民何幸,教父儲靈。



 欽承景貺,祗奉崇名。臻虔寶冊,垂祐基扃。



 降神,《真安》



 猶龍之聖,降生厲鄉。教流清凈,道符混茫。



 大君肅謁,盛儀允臧。森羅羽衛,躬薦蕭薌。



 簪紱濟濟,鐘石洋洋。高真至止,介福誕祥。



 奉玉幣,《靈安》



 琳宮奕奕,黼坐煌煌。玉帛成禮,飆馭延祥。



 鴻儀有則,景福無疆。嘉應昭協,丕猶誕揚。



 奉饌,《吉安》



 金奏以諧,飆游斯格。靈監章明,皇心勵翼。



 肅奉雕俎,來升彩席。享德有孚,凝禧無斁。



 酌獻,《大安》



 欽崇至道,肅謁殊庭。順風而拜,明德惟馨。



 飆馭來格,尊酒斯盈。是酌是獻,心通杳冥。



 飲福,《大安》



 彼渦之壤,指李之區。千乘萬騎,來朝密都。



 躬陳芳薦,款接仙輿。飲酒受福,永耀鴻圖。



 亞、終獻,《正安》



 邈矣道祖,冥幾惚恍!常德不離,至真無象。



 引位清穹,降祥神壤。酌醴薦誠,控飆來享。



 送神,《真安》



 醴盞在戶,金奏在庭。籩豆有踐,黍稷非馨。



 義盡蠲潔,誠通杳冥。言旋風駟,祚我修齡。



 太極觀奉冊寶一首



 《登安》之曲



 薦號穹冥,登名祖檷。陟配陽郊,協宣典禮。



 感電靈區,誕聖鴻懿。冊寶斯陳,福祿來暨。



 景靈宮奉冊寶一首



 《登安》之曲



 穆穆真宗,錫羨蕃昌。飆輪臨貺,諄誨洞彰。



 虔崇懿號,祗答景祥。至誠致享,降福無疆。



 景祐元年親享景靈宮二首



 降真,《太安》



 真館奉幣,潔齊致馨。靈因斯格,社稷慶寧。



 送真,《太安》



 椒漿尊享,珍饌精祈。睟容杳邈,瑤輅霞飛。



 大觀三年朝獻景靈宮二首



 奉饌,《吉安》



 威靈洋洋,靡有常向。於惟欽承,來假來饗。



 博碩芬香,是烝是享。奉器有虔,載德無爽。



 爾牲既充,是烹是肆。爾肴既具,是羞是饋。



 非物之重,惟德之備。神之格思,歆我精意。



 高宗郊前朝獻景靈宮二十一首



 皇帝入門,《乾安》



 維皇齊居,承神其初。顒顒昂昂,龍步雲趨。



 景鐘鏗如,肅覲清都。肸蠁之交,神人用孚。



 升殿,《乾安》



 帝既臨享,罄茲精意。對越在天,爰升紫陛。



 孔容翼翼,保承丕緒。孝奉天儀,永錫爾類。



 降聖,《太安》



 惟德馨香,升聞八方。粵神臨之,來從帝鄉。



 萬靈景衛,有燁其光。監我精純,降福穰穰。



 盥洗,《乾安》



 齋居皇皇,瓊琚鏘鏘。承祭之初,其如在旁。



 挹彼注茲,儲禧迎祥。神之聽之,欣欣樂康。



 聖祖位,《乾安》



 涓選休辰,有事嘉薦。瑯瑯瓊佩,陟降巖殿。



 其陟伊何?幣玉斯奠。周旋中禮,千億儲羨。



 聖祖位奉玉幣,《靈安》



 上靈始祖,雲景元尊。嚴祀夙展,
 六樂朱軒。



 明玉之潔,豐帛之溫。暢乃繼序,承德不愆。



 還位,《乾安》



 我後臨饗,奠幣攸畢。式旋其趨,矩度有式。



 禮容齋莊,孝思純實。天休滋至,時萬時億。



 奉饌,《吉安》



 百職駿奔,來相於庭。奉盛以告,登茲芳馨。



 際天蟠地,默運三靈。神兮來歆,祚我休平。



 再盥洗,《乾安》



 有嚴大禮,對時休明。情文則粲,蠲潔必清。



 再臨觀盥,以專以精。真游來格,永觀厥成。



 再詣聖祖位,《乾安》



 於赫炎宋,十葉華耀。屬茲郊報,陟
 降在廟。



 其降伊何?椒漿桂酒。再拜斟酌,永御九有。



 聖祖位酌獻,《祖安》禦制



 瑤源誕啟,玉牒肇榮。覆育群有,監觀圓清。



 酒醴既洽,登薦惟誠。無有後艱,駿惠雲仍。



 還位,《乾安》



 奠鬯告成,式旋厥位。天步雍容,神人燕喜。



 九廟觀德,百靈薦祉。子孫其昌,垂千萬祀。



 文舞退、武舞進,《正安》



 於皇樂舞,進旅退旅。一弛一張,笙磬具舉。



 豈惟玩聲,像德是似。神鑒孔昭,福祿來予。



 亞、終獻,《沖安》



 五音飭奏,神既億康。澹其容與,薦此
 嘉觴。



 有來顯相,鋗玉鏘鏘。奉承若宥,罔不齊莊。



 飲福,《報安》



 嘉薦既終,神貺斯復。繼我思成。靈光下燭。



 孝孫承之,載祗載肅。敷錫庶民,亟蒙祉福。



 還位,《乾安》



 帝臨閟庭,逆厘上靈。神□安坐,肅若有承。



 嘉觴既申,德聞惟馨。靈光留俞,祚我億齡。



 徹饌,《吉安》



 普淖既薦,苾芬孔時。神嗜而顧,有來燕娭。



 饗矣將徹,載欽載祗。展詩以侑,益臻厥熙。



 送真,《太安》



 雍歌既徹,熙事備成。神夕奄虞,忽乘青冥。



 靈心回睠,監我精禋。誕降嘉祉,休德昭清。



 降殿,《乾安》



 我秩元祀,上推靈源。展事有侐,祲威肅然。



 丹戚既降,秉心益虔。荷天之休,於千萬年。



 望燎,《乾安》



 奕奕靈宮,有嚴毖祀。燔燎具揚,禮儀既備。



 帝心肅祗,天步旋止。對越在天,永膺蕃祉。



 還大次,《士安》



 帝將於郊,昭事上祀。爰茲畢觴,復即於此。



 飆游載旋,容旌沓騎。維皇嘉承,錫祚昌熾。



 高宗明堂前朝獻景靈宮十首



 降聖,《大安》



 德惟馨香,升聞八方。粵神之從,燦然有光。



 驂飛乘蒼,啾啾蹌蹌。逍遙從容,顧予不忘。



 升殿,《乾安》



 帝既臨享,龍馭華耀。孝孫承之,陟降在廟。



 誠意上交。慶陰下冒。天休駢至,千億克紹。



 聖祖位奠玉幣,《靈安》



 玉氣如虹,豐繒充笥。既奉既將,亦奠在位。



 有永群後,實相祀事。何以臨下?心意不貳。



 奉饌,《吉安》



 瓊琚鏘鏘,玄衣繡裳。薦嘉升香,粢盛芬芳。



 禮儀莫愆,鼓鐘喤喤。曾孫之常,綏福無疆。



 聖祖位酌獻,《祖安》



 裴回若留,靈其有喜。薦我馨香,挹茲酒醴。



 我祖在天,執道之紀。申祐無疆,奏神稱禮。



 文舞退、武舞進,《正安》



 進旅退旅,載執干戚。不愆於儀,容服有赫。



 式妥式侑,神保是格。靈鑒孔昭,孝思維則。



 亞、終獻,《沖安》用舊辭。



 飲福,《報安》



 於赫大神,總司元化。監我純精,威光來下。



 延昌之貺,千億馮藉。曾孫保之,丕平是迓。



 徹饌,《吉安》



 洋洋降臨,肅肅布列。熙事既成,嘉籩告徹。



 九天儲慶,垂祐無缺。浸明浸昌,綿綿瓜瓞。



 送真,《太安》



 高飛安翔,持御陰陽。幽贊圓穹,監觀四方。



 元精回復,奄虞孔良。畢觴降嘏,偃蹇於驤。



 望燎,《乾安》



 奕奕原祠,有嚴毖祀。禮儀孔宣,燔燎斯暨。



 帝心肅祗,天步旋止。熙事既成,永膺蕃祉。



 孝宗明堂前朝獻景靈宮八首



 盥洗,《乾安》



 合宮之饗,報本奉先。欽惟道祖,浚發璇源。



 駕言謁款,其盥惟虔。尚監精衷,錫祚綿綿。



 聖祖,《乾安》



 駿命有開,慶基無窮。祗率百闢,仰瞻睟容。



 鼓鐘斯和,黍稷斯豐。靈其居歆,福祿來崇。



 還位,《乾安》



 嘉玉既設,量幣即陳。徬佛靈游,來顧來寧。



 對越伊何?厥惟一純。祐我熙事,以迄於成。



 奉饌,《吉安》



 發祥仙源,流澤萬世。曷其報之?親饗三歲。



 相維列卿,潔粢是饋。匪物之尚,誠之為至。



 再詣盥洗,《乾安》



 華燈熒煌,瑞煙氤氳。威神如在,蠲潔必親。



 再盥於罍,再帨於巾。皇心肅祗,其敢憚勤。



 再詣聖祖位,《乾安》



 歲逢有年,月旅無祔。我將我饗,如幾如式。



 肅爾臣工,諧爾金石。本原休功,垂裕罔極。



 還位,《乾安》



 旨酒思柔,神具醉止。工祝既告,孝孫旋位。



 何以酢之?純嘏來備。燕及雲來,蕃衍無已。



 文舞退、武舞進,《正安》



 象德之成,有奕其舞。一弛一張,進旅退旅。



 □彗以管簫,和以鏞鼓。神其樂康,永錫多祜。



 寧宗郊前朝獻景靈宮二十四首



 皇帝入門,《乾安》



 閟幄邃深,雲景杳冥。天清日晬,展容
 玉庭。



 締基發祥,希夷降靈。神其來燕,是饗是聽。



 升殿,《乾安》



 帝居瑤圖,璇題玉京。日月經振,列宿上熒。



 桂簋飶芬,瑚器華精。夤承禋祀,用戒昭明。



 降神,《太安》六變



 圜鐘為宮



 四靈晨耀,五緯夕明。風雲晏和,天地粹清。



 靈兮來迎,靈兮來寧。啟我子孫,饗於純精。



 黃鐘為角



 芬枝揚烈,熉珠葉陶。闓珍闡符,展詩舞箾。



 神哉來下,神哉來翱。肅若有承,靈心招搖。



 太簇為征



 龍車既奏,鳳馭載翔。帝幄儜靈,天衢騰芳。



 神來留俞,神來蹇驤。禮鬯樂明,奏假孔將。



 姑洗為羽



 虹旌蜺旄,鸞旗翠蓋。星樞扶輪,月禦葉衛。



 靈至陰陰,靈般裔裔。來格來饗,福流萬世。



 盥洗,《乾安》



 禮文有俶,祀事孔明。將以潔告,允惟齊精。



 自盥而往,聿觀厥成。靈監下臨,天德其清。



 詣聖祖位,《乾安》



 維宋肖德,欽天顧右。於皇道祖,丕厘靈祐。



 葛藟殖繁,瓜瓞孕茂。克昌厥後,世世孝奏。



 聖祖位奉玉幣,《靈安》高宗御制,見前。



 皇帝還位,《乾安》



 桂宮耽耽,藻儀穆穆。天回袞彩,風韶璜玉。



 《咸》、《英》皦亮,容典炳煜。假我上靈,景命有僕。



 奉饌,《吉安》



 我簋斯盈,我簠斯實。或剝或烹,或燔或炙。



 有殽既將,為俎孔碩。禮儀卒度,永錫爾極。



 再盥洗,《乾安》



 觴澹初勺,禮戒重盥。假廟以《萃》,取象於《觀》。清明外暢,精肅中貫。我儀圖之,三靈攸贊。



 再詣聖祖位,《乾安》



 肇基駿命,鞏右鴻業。鼎玉龜符,垂固萬葉。



 靈貺具臻,神光燁燁。暉祚無疆,規重矩疊。



 聖祖位酌獻,《祖安》高宗御制,見前。



 還位,《乾安》



 皇帝瑞慶,長發其祥。纂系悠遠,逆源靈長。



 德之克明,休烈有光。配天作極,孝饗是將。



 文舞退、武舞進,《正安》



 持翟成象,秉朱就列。旄乘整溢,鳳儀諧節。



 揮舒皇文,歌蹈先烈。合好效歡,福流有截。



 亞獻,《沖安》



 光熉紫幄,神流玉房。秉文侑儀,嘉虞貳觴。



 震澹醉喜,徬佛迪嘗。璇源之休,地久天長。



 終獻,《沖安》



 靈輿蹇驤,畢觴泰筵。貳饗允穆,稞將克竣。



 垂恩儲祉,錫羨永年。將以慶成,燕及皇天。



 詣飲福位,《乾安》



 若木露英,清雲流霞。蔓蔓芝秀,馮馮桂華。



 綿瑞無疆,產嘏孔奢。皇則受之,鞏我帝家。



 飲福酒,《報安》



 旨酒惟蘭,勺漿惟椒。福流瓚斝,光燭琨瑤。



 拜貺清宮,凝輝慶霄。神其如在,徘徊招搖。



 還位,《乾安》



 烝哉我皇,繼天毓聖!逆厘元都,對越靈慶。



 如天斯久,如日斯盛。瑤圖浚邈,永隆駿命。



 徹饌,《吉安》



 房鉶陳列。室簋登奉。告饗具歆,展徹惟拱。



 祥光奕奕,嘉氣蒙□。受嘏不愆,燕天之寵。



 送真,《太安》



 雲車風馬,靈其來游。天門軼蕩,神其莫留。



 遣慶陰陰,祉發祥流。康我有宋,與天匹休。



 降殿,《乾安》



 璇庭爛景,紫殿流光。禮洽乾回,福應日昌。



 聖系厖鴻,景命溥將。德茂功成,率祀無疆。



 詣望燎位,《乾安》



 厥初生民,淵浚唯祖。芳薦既輟,明燎具舉。



 德馨升聞,靈貺蕃詡。懷濡上靈,祐周之祜。



 還大次,《乾安》



 帝假於宮,彞承清祀。天暉臨幄,宸衛森
 峙。



 行繇大室,旋趨紫畤。率禮不違,式敷靈祉。



 理宗明堂前朝獻景靈宮二首餘用舊辭



 升殿,登歌《乾安》



 我享我將,罄茲精意。陟降左右,維天與契。



 齋明乃心,祗肅在位。於萬斯年,百福來備。



 亞獻,宮架《沖安》



 慶雲鬱鬱,鳴璆瑯瑯。澹其容與,申薦貳觴。



 奉承若宥,神其樂康。錫以多祉,源深流長。



 大中祥符封禪十首餘同南、北郊



 山上圜臺降神,《高安》



 巖巖泰山,配德於天。奉符展採,
 翼翼幹幹。



 滌濯靜嘉,罔有弗蠲。上帝顧諟,冷風肅然。



 昊天上帝坐酌獻,《奉安》



 皇天上帝,陰騭下民。道崇廣覆,化洽鴻鈞。



 靈文誕錫,寶命惟新。增高欽事,式奉嚴禋。



 太祖配坐酌獻,《封安》



 於穆聖祖,肇開鴻業。我武惟揚,皇威有曄。



 四庾混同,百靈震疊。陟配高穹,明靈是接。



 太宗配坐酌獻,《封安》



 祗若封祀,神宗配天。禮樂明備,奠獻精虔。



 景靈來格,休祥藹然。於昭垂慶,億萬斯年。



 亞獻,《恭安》



 因高定位,禮修物備。薦鬯卜牲,虔恭寅畏。



 八音克諧,天神咸暨。降福穰穰,永錫爾類。



 終獻,《順安》



 浩浩元精,無臭無聲。臨下有赫,得一以清。



 備物致享,薦茲至誠。泰尊奠獻,夙夜齊明。



 社首壇降神,《靖安》



 至哉坤元,資生伊始。博厚稱德,沉潛柔止。



 降禪方位,聿修明祀,寅恭吉蠲,永錫蕃祉。



 皇地祗坐酌獻,《禪安》



 坤德直方,博厚無疆。秉陰得一,靜而有常。



 寶藏以發,乃育百昌。肅祗禪祭,錫祉穰
 穰。



 太祖配坐酌獻,《禪安》



 皇矣聖祖,丕赫神武。秉運宅中,威加九土。



 德厚功崇,頌聲載路。陟配方祗,對天之祜。



 太宗配坐酌獻,《禪安》



 毖祀柔祗,報功厚載。思文太宗,侑神嚴配。



 鐘石斯和,籩豆咸在。永錫坤珍。資生為大。



 汾陰十首



 降神,《靖安》



 茫茫坤載,粵惟太寧。資生光大,品物流形。



 瞻言汾曲,允宅神靈。聖皇躬享,明德惟馨。



 奠玉幣,登歌《嘉安》



 至誠旁達,柔祗格思。奉以琮幣,致
 誠在茲。



 奉俎,《豐安》



 博碩者牲,載純其色。體薦登俎,聿崇坤德。



 後土地祗坐酌獻,《博安》



 秉陰成德,敏樹宣功。應變審諦,神力無窮。



 沉潛剛克,流謙示中。潔茲奠獻,妙物玄通。



 太祖配坐酌獻,《博安》



 坤元茂育,植物成形。於穆聖祖,功齊三靈。



 嚴恭配侑,厚德攸寧。永懷錫羨,歆此惟馨。



 太宗配坐酌獻,《博安》



 報功厚載,祀事惟明。思文烈考,
 道濟群生。



 侑神定位,協德安平。馨潔並薦,享於克誠。



 飲福,《博安》



 寅威寶命,明祀惟虔。協神備物,罔不吉蠲。



 後祗格思,靈飆肅然。庭受景福,遐哉億年!



 亞、終獻,《正安》



 至哉柔祗,滋生蕃錫。滌濯靜嘉,寅恭夕惕。



 金奏純如,萬舞有奕。立我烝民,莫匪爾極。



 後土廟降神,《靖安》



 博厚流形,秉陰成德。柔順剛正,直方維則。



 明祗格思,素汾之側。祗載吉蠲,宸心翼翼。



 酌獻,《博安》



 至哉物祖,設象龍脽。動靜之德,翕闢攸宜。



 嘉慄以薦,精禱洪厘。茂宣陰貺,五穀蕃滋。



 祗奉天書六首



 朝元殿酌獻,《瑞文》



 妙道非常,神變無方。惟天輔德,靈貺誕章。



 玄文昭錫,寶歷彌昌。禮崇明祀,式薦馨香。



 含香園,《瑞文》



 運格熙盛,將封介丘。禮神之域,瑞命殊尤。



 靈文薦降,丕顯皇猷。聖心肅奉,永洽鴻休。



 泰山社首壇升降,《瑞文》



 玄穹眷懷,寶符申錫。垂露騰文,粲然靈跡。



 發祥吉圖,純熙寫奕。登薦欽崇,式昭天
 歷。



 奉香酌獻,《瑞安》



 謂天蓋高,惟皇合德。倬彼靈章,圖書是錫。



 眷命諄諄,被以遐歷。膺菉告成,虔恭欽翼。



 地屆興王,祥開圖菉。典禮昭成,祺祥交屬。



 大輅逶迤,卿雲紛鬱。祐我含靈,錫茲介福。祥符七年奉祀畢,天書回至應天府,有雲物之瑞,命制是曲,以紀休應。



 升降,《靈文》



 旻穹無聲,惟德是輔。降監錫符,垂文篆素。



 孝瑞紀封,英聲載路。既壽而昌,篤天之祜。



 祭九鼎十二首



 帝鼐土王日祀



 降神,《景安》



 日號丙丁,方號中央。德惟其時,蠲吉是將。



 夫何飲之?黃流玉瓚。夫何食之?有陳伊饌。



 奉饌,《豐安》



 粢盛既豐,牲牢既充。展茲熙事,溫溫其恭。



 惟明欣欣,燔炙芬芬。保乎天子,繁祉薦臻。



 亞、終獻,《文安》



 工祝致辭,黃流協鬯。爰登清歌,載期神享。



 噫予誠心,精禋是虔。嘉予陳祀,豐盈豆籩。



 春分,蒼鼎亞、終獻,《成安》



 法乾剛兮,鑄鼎奠方。涓嘉旦
 兮,齊明迎祥。



 胡為持幣?維箱及筥。胡為和羹?有錡維釜。



 立夏,岡鼎迎神,《凝安》



 我方東南,我日朱明。爰因其時,鼎以岡名。



 粢盛既馨,牲牷既盈。祐我皇家,巽令風行。



 亞、終獻,《成安》



 黃流在中,惟馨香祀。於薦於神,爰祗厥事。



 禮從多儀,以進為文。尊斝三獻,昭示孔勤。



 夏至,彤鼎酌獻,《成安》



 犧尊將將,徂基自堂。牲牷肥循,鼓鐘喤□。



 肆予醴齊,椒馨飶香。韋來歆顧,天祚永昌。



 立秋,阜鼎酌獻,《成安》



 明德崇享,磬筦鏘鏘。鏗兮佩舉,峨冠齊莊。



 肆陳有序,承箱是將。其牲伊何?籩豆大房。



 秋分,皛鼎亞、終獻,《成安》



 神宮巍巍,庭燎有輝。聲諧備樂,物陳豐儀。



 清酤既載,酌言獻之。惟神醉止,韋來蕃厘。



 立冬,魁鼎迎神,《凝安》



 時運而冬,乃神玄冥。陰陽相推,豐年以成。



 越陳嘉肅,牡牢粢盛。來享來依,監於明誠。



 酌獻,《成安》



 罍之初登,其儀昭陳。罍之既稞,其香升聞。



 神心嘉止,於焉欣欣。貽我有年,穰穰其仁。



 冬至,寶鼎奠幣,《明安》



 秉心齊明,奉牡博碩。匏絲鏗陳,冠佩儼飾。



 其肆其將,明神來格。執奠維何?猗歟幣帛!



\end{pinyinscope}