\article{列傳第三十}

\begin{pinyinscope}

 馬令琮杜漢徽張廷翰吳虔裕蔡審廷周廣張勛石曦張藏英陸萬友解暉李韜王晉卿郭廷謂子延浚從子延澤趙延進
 輔超



 馬令琮,本名令威,避周祖名改之,大名人。父全節,《五代史》有傳。全節歷橫海、定遠、昭義、彰德、定武、天雄六節度,皆署令琮為牙校,累授彰德牙內都指揮使、檢校尚書左僕射,領勤州刺史。令琮少善騎射,嘗從其父平安州及與鎮州安重榮戰,皆有功,由是知名。晉開運二年,全節卒,令琮起復,拜隰州刺史。漢祖開國,為西京巡檢使。周祖受命,改陳州刺史。征兗州,為京城四門外巡檢。世
 宗嗣位,移隨州。顯德二年,入為虎捷左第一軍都指揮使。六年,兼領建州刺史。



 太祖即位,出刺懷州。李筠叛,將親征,召三司張美餉兵食,美言河內密邇上黨,令琮日夜儲蓄以俟王師。太祖善之,命授團練使。執政言令琮方供億大軍,不可移他郡,故升懷州為團練,以令琮充使,又充先鋒都指揮使。澤、潞平,為昭義兵馬鈐轄。逾年被疾,詔許歸郡。乾德元年,卒,年三十九。太祖甚憐之,錄其子延恩為殿直。



 杜漢徽,京兆長安人。父阿孫,為太原威勝軍使。漢徽有膂力,善騎射,年十七,仕後唐武皇為廳直隊長。天成中,累遷護聖軍使。晉天福六年,與慕容鄴等討安州李金全,生禽指揮使孫厚,以功遷興順指揮使。八年,從征鎮州安重榮,改護聖指揮使,贈阿孫為左贊善大夫。開運二年,以所部戍深州,破契丹於樂壽,殺獲甚眾。漢初,從高行周討杜重威於鄴,屢為流矢所中,身被重創,猶力戰,觀者壯之。又率所部戍鎮州,破契丹於靈壽,獲車馬
 甚眾。周世宗征劉崇,漢徽有戰功,補龍捷左第五軍都虞候,移所部屯安平縣,破契丹於縣南,獲器甲車帳,遷本軍左第四軍都虞候。



 宋初,補本軍都校,領茂州刺史,改領潮州。從平李筠,又從平李重進,錄功居多。建隆三年,出為天長軍使,移雄武軍使、知屯田事。是冬,被病,即以符印授通判宋鸞,請告歸京。家人勸其求醫藥,漢徽笑曰:「我在戎行四十年,大小百餘戰,不死幸矣,安用藥為?」未幾,卒。



 張廷翰,冀州信都人。父慎圖,仕周為兵部郎中。廷翰少慷慨,有智略,善騎射。晉天福中,冀州刺史張建武召補牙校,其後刺史李沖署為本州牢城軍校。契丹入中原,署其黨何行通為刺史,契丹主道殂,州人共殺行通,推廷翰知州事。漢初,就拜刺史,廷翰盡捕殺行通者戮於市。為政寬厚簡易,民甚愛之。周廣順初,召赴闕,周祖見其貌魁偉,謂樞密使王峻曰:「冀州近邊,雖更擇人,亦無逾廷翰者。」即日遣還。在郡八年,契丹將高牟翰數擾邊,
 皆為廷翰擊走。廷翰家富於財,歲遣人繼金帛北入市善馬,常得數百匹,貢獻外悉遺貴近,甚獲美譽。顯德中,歷棣、海、沂三州團練使,屢率兵敗淮人,移萊州。



 宋初,又歷冀、亳二州。乾德二年卒,年四十七。



 吳虔裕,許州許田人。父徽,左屯衛將軍。虔裕少為郡吏,漢祖鎮許,愛其精謹,署以右職。及移鎮太原,以虔裕從。開國,擢為引進使,轉內客省使。時鎮州節度劉在明卒,遣虔裕率兵巡護。隱帝即位,召為宣徽北院使。



 周祖討
 三叛,以虔裕為河中行營都監,率護聖諸軍五千以往。李守貞出兵五千餘,設梯橋,分五路於長連城西北以御周祖。周祖令虔裕率大軍橫擊之,蒲人敗走,奪其梯橋,殺傷大半。師還,賜襲衣、玉帶。會樞密使楊邠上言求解職,隱帝遣人諭邠曰:「樞機之任,非卿不可,卿何聽間離而為此請耶?」使至而虔裕在坐,即揚言曰:「機要重地,非可久處,俾後來者迭居可也。」使還以白帝,帝怒,出虔裕為鄭州防禦使。乾祐末誅大臣,急詔入朝,命將兵守
 澶州。及留子陂戰敗,遂降周祖。廣順初,遣還,賜以襲衣、玉帶、鞍勒馬。從周祖討慕容彥超,破之。改汝州防禦使,歷右衛、左金吾衛二大將軍兼街仗使。



 太平興國六年,遷右千牛衛上將軍,仍判左街仗事。虔裕掌金吾三十餘年,端拱初卒,年八十八,贈太尉。



 虔裕性簡率,言多輕肆。右金吾上將軍王彥超告老,虔裕語人曰:「我縱殭僕殿階下,斷不學王彥超七十致仕。」人傳笑之。每朝會及從游宴,太宗憐其壽高,常慰撫之。子延彬,至儀鸞副使,
 延彬子仁美,至內殿崇班。



 蔡審廷,磁州武安人。曾祖凝,邢州別駕。祖綰,武安遠城三冶使。父顒,洺州長史。審廷少能騎射,晉初,應募補護聖散都頭。周顯德初,擢為殿前散員,轉鐵騎副兵馬使。從世宗戰高平有功,遷軍使。太祖為殿前都點檢,從世宗征淮南,審廷隸麾下,預戰紫金山,改副指揮使。



 宋初,授殿前散都頭指揮使。從征李筠,攻澤州先登,為飛石傷足,帝賜以良藥、美酒。及車駕還京,幸其官署問之,賜
 繼甚厚。尋轉內殿直都虞候,俄改伴飯都指揮使。建隆中,領富州刺史兼內外馬步軍副都軍頭。乾德初,授冀州刺史。徵太原時,為北面步軍都指揮使,屯兵易州。審廷訓練士卒甚整,太祖過鎮陽,見於行在所,賜名馬、寶劍,命為鎮州兵馬都鈐轄。開寶八年卒,年六十九。



 周廣,字大均,其先應州神武川人。父密,事晉,歷鄜、延、晉三鎮節度使。周廣順初,至太子太師致仕。廣幼從其父為牙校。漢初,授供奉官。未幾,擢左千牛衛將軍。周祖命
 將討慕容彥超於兗州,以廣為行營都監。賊平,錄功遷右武衛將軍。俄改右神武將軍,充鎮淮軍兵馬都監。從世宗征淮南。既得江北數州,即命廣勞來安集,民甚德之。因領常州刺史兼內外馬步軍都軍頭。淮南平,改眉州刺史。



 宋初,授隰州刺史。乾德三年,遷潘州團練使,令訓練雄武諸營。開寶二年,從征太原,為攻城樓櫓戰棹都部署,師還,加內外馬步軍副都軍頭。六年,改右屯衛大將軍,領郡如故。太平興國二年,卒。



 張勛,河南洛陽人。晉開運中,事留守景延廣為典客,延廣表為供奉官。周世宗將征淮南,以勛為申州緣淮巡檢。因採光州機事聞於朝廷,即命勛率兵同討平之,遂監光州軍,充內外巡檢。後攻黃州,敗吳人於麻城,復破柏業山砦,目中流矢。遷內園副使。及徵瀛、莫,以為霸州兵馬都監。



 初,徵李筠,勛從石守信董前軍,拔大會砦,及敗筠眾於太行,破澤州,皆預有功。太祖還京,命權知許州。未幾,李重進叛,又詔與石守信、李處耘先率兵進討。
 拔揚州,以勛為兵馬都監,遷氈毯使。討朗陵,充前軍兵馬都監。荊湖平,以功就拜衡州刺史。乾德初,克郴州及桂陽監,以勛為刺史兼監使。五年,代歸,至揚州卒,年六十八。太祖甚憐之,錄其子廷敏為殿直。勛性殘忍好殺,每攻破城邑,但揚言曰「且斬」,頗有橫罹鋒刃者。將赴衡州,州民皆涕泣相謂曰:「『張且斬』至矣,吾輩何以安乎!」



 石曦,並州太原人,晉祖弟韓王暉之子。天福中,以曦為右神武將軍。歷漢至周,為右武衛、左神武二將軍。恭帝
 即位,初為左衛將軍。會高麗王昭加恩,命曦副左驍衛大將軍戴交充使。



 建隆三年,再使高麗,遷左驍衛大將軍,護秦州屯兵。西人犯邊,曦率所領擊破之,斬渠帥十三人。太祖征晉,曦領兵二千人自澤、潞除道至太原,壅汾水灌其城,又益兵千人,部攻遼州。俄知雄州,代,為潭州鈐轄。開寶八年,領兵敗南唐軍二千餘於袁州,平梅山、板倉諸洞蠻寇,俘馘數千人。太平興國中,歷右神武、右羽林大將軍,連知孟、襄二州,遷領誠州刺史。雍熙四
 年,改知霸州兼部署。會陳廷山謀以平戎軍叛入北邊,曦察知之,與侯延濟定計,禽廷山以獻。錄其功,加領本州團練使、同知鎮州。淳化二年,移原州,遷右龍武軍大將軍。被病請告,詔特給全奉。四年,卒,年七十四,賵賻加等。



 張藏英,涿州範陽人,自言唐相嘉貞之後。唐末,舉族為賊孫居道所害。藏英年十七,僅以身免。後逢居道於豳州市,引佩刀刺之,不死,為吏所執。節帥趙德鈞壯之,釋
 而不問,以補牙職。藏英後聞居道避地關南,乃求為關南都巡檢使。至則微服攜鐵□,匿居道舍側,伺其出擊之,僕於地,嚙其耳啖之,遂禽歸。設父母位,陳酒肴,縛居道於前,號泣鞭之,臠其肉,經三日,刳其心以祭。即詣官首服,官為上請而釋之。燕、薊間目為「報仇張孝子」。契丹用為盧臺軍使兼榷鹽制置使,領坊州刺史。周廣順三年,率內外親屬並所部兵千餘人,及煮鹽戶長幼七千餘口,牛馬萬計,舟數百艘,航海歸周。至滄州,刺史李暉
 以聞。周祖頗疑之,令館於封禪寺,俄賜襲衣、銀帶、錢十萬、絹百匹、銀器、鞍勒馬。數月,世宗即位,授德州刺史。未幾召歸,對便殿,詢以備邊之策。藏英請於深州李晏口置砦,及誘境上亡命者以隸軍,願為主將,得便宜討擊。世宗悉從之。以為緣邊招收都指揮使,賜名馬、金帶。藏英遂築城李晏口,累月,募得勁兵數千人。會遣鳳翔節度王彥超巡邊,為契丹所圍,藏英率新募兵馳往擊之,轉戰十餘里,契丹解去。改濮州刺史,仍領邊任。契丹將
 高牟翰以精騎數千擾邊,藏英逆擊於胡盧河北,自旦至晡,殺傷甚眾。值暮收兵,契丹遁去。後因領兵巡樂壽,契丹幽州驍將姚內斌偵知藏英兵少,以精騎二千陣於縣之北,藏英率麾下擊之,自辰及申,士皆殊死戰,內斌遂解去。世宗降璽書褒美。征瓦橋關,為先鋒都指揮使,敗契丹騎數百於關北。下固安縣,又改關南排陣使。宋初,遷瀛州團練使,並護關南軍。建隆三年,卒於治所,年六十九。孫鑒,自有傳。



 陸萬友,蔚州靈丘人。少隸太原為裨校。漢祖起義,擢為護聖指揮使。隱帝即位,出為天雄軍馬軍都指揮使。周祖之起兵也,萬友預謀。及踐阼,擢為散員都指揮使,領獎州刺史。世宗嗣位,遷龍捷左第三軍都指揮使。轉控鶴右廂都校、領虔州團練使,改虎捷右廂、領閬州防禦使。恭帝嗣位,出為安州防禦使。



 宋初,歷沂、蘄二州防禦使。乾德四年,改汝州。開寶中,討南唐,造舟於採石磯以濟師,命萬友守之。江南平,為和州防禦使。太宗嗣位,以
 為晉、絳等州都巡檢使。帝徵太原,克汾、石二州,以萬友為石州都巡檢使,俄兼知石州,移巡警鳳翔、秦、隴。代歸,詔知瀛州,在郡二年,政務茍簡。雍熙二年,改右監門衛大將軍,充河陰兵馬都監,逾年卒,年七十三。萬友始業圬鏝,既貴達,不忘本,以銀為圬鏝器數十事示子孫。性猛暴,以武勇自任,所至無善政。太宗以其勛舊,恩遇不替,聘其次女為許王夫人。



 解暉,洺州臨洺人。父珪,應募為州兵,後唐天成中,西征
 至劍門,沒於陣。暉少有勇力,以父死戎事,得隸兵籍。戍雁門,與契丹接戰,斬首七級,獲酋長一人。以功遷奉國軍隊長。晉天福中,安重榮反鎮州,因舉兵向闕。至宋城,晉師逆戰,大破之。暉募軍中壯士百餘人夜搗賊壘,殺獲甚眾。暉頻中流失,而督戰自若,顏色不撓,以功遷本軍列校。周廣順初,劉崇與契丹侵晉州,暉從都部署、樞密使王峻等往援之。暉率敢死士三十餘,夜入契丹帳擊之,殺獲甚眾,遷本軍第五指揮使。從世宗征淮南,率
 所部下黃州,禽刺史高弼,遷虎捷第一軍都虞候。



 宋初,步軍都軍頭,從征澤州,力戰,目中流矢。師還,策勛為內外馬步軍副都軍頭。建隆四年,充湖廣道行營前軍戰棹都指揮使。潭州平,降璽書獎諭。偽統軍黃從志據岳州,暉率舟師討平之,生禽從志及將校十四人,俘斬數千,溺死者眾。改控鶴右第二軍都指揮使,領高州刺史。乾德六年,詔領所部軍屯上黨,從李繼勛略太原。開寶九年,破太原軍於境上,斬首千餘級,獲馬三十匹。改均
 州刺史。



 太平興國二年,詔於潞州北亂柳石圍中築城,名威勝軍,以暉為軍使。從征並州,與尚食使石彥斌率所部先下隆州,殺並州三百餘,禽招討使李詢等六人以獻於行在所,賜予有加。復令與彥贇督戰士隸城西行營,分攻太原。劉繼元降,太宗以太原宮女三人賜暉,俄以功遷本州團練使、知霸州。雍熙初,充雲、應、寰、朔、忻、代等州都巡檢使。三年,代歸本郡。淳化二年,被病,上章告老,改右千牛衛上將軍致仕。詔未至而卒,年八十。



 暉
 鷙猛木強,每受詔征伐,常身先之。人所憚者,暉視之若甚易,由是頻立戰功,金創遍體。時稱驍將。子守顒,至內殿崇班、閣門祗候。



 李韜,河朔人。有勇力膽氣,善用槊,為禁軍隊長。周祖征三叛,韜從白文珂攻河中,兵傅其城。文珂夜詣周祖議犒軍,留韜城下。時營柵未備,李守貞乘虛來襲,營中忽見火發,知賊聚至,惶怖失據。客省使閻晉卿率左右數十人,遇韜於月城側,謂韜曰:「事急矣,城中人悉被黃紙
 甲,為火光所照,色俱白,此殊易辨,奈軍士無鬥志何?」韜憤怒曰:「豈有食君祿而不為國致死耶!」即援槊而進,軍中死士十餘輩隨韜犯賊鋒。蒲有猛將躍馬持戈擬韜,韜刺之,洞胸而墜。又連殺數十人,蒲軍遂潰,因擊,大破之,守貞自是閉壘不敢出。俄驍將王三鐵降,城遂平,韜由此知名。累遷軍校,出為趙州刺史,移慈州。乾德六年卒。



 王晉卿,河朔人。少勇敢,為鄉里所推。周世宗在澶淵,晉
 卿以武藝求見,得隸帳下。及即位,補東頭供奉官。從戰高平,征淮甸,每遣宣傳密旨,甚親信之。洎北征,為先鋒都監,督戰有功,詔權控鶴都虞候。克關南,授軍器庫使。顯德四年,為龍捷右第一軍都指揮使,領彭州刺史。恭帝即位,出為濱州刺史。



 乾德中,為興州刺史。四年,移漢州。時蜀初平,寇盜充斥,晉卿嚴武備,設方略,禽捕剪滅,靡有遺漏,自是雖劇賊無敢窺其境。然以賄聞,太祖惜其才而不問。秩滿歸闕,以疾求頤養,改左監門衛將軍、
 奉朝請。貢重錦十匹、銀千兩以謝,詔不納,以其黷貨,愧之也。未幾,詔戍北邊,疆場清肅。開寶四年,復授莫州刺史。在郡謹斥候,善撫循,士卒皆樂為之用,邊民安堵。六年八月卒,年六十七。



 郭廷謂,字信臣,徐州彭城人。父全義,仕南唐為濠州觀察使。廷謂幼好學,工書,善騎射。補殿前承旨,改濠州中軍使,李景每令偵中朝機事入奏。全義卒,擢莊宅使、濠州監軍。周世宗攻淮右,南人屢敗,城中甚恐,廷謂與州
 將黃仁謙為固禦之計。周師遣諜以鐵券及其壘,廷謂拒之。城中負販之輩率不逞,廷謂慮其亡逸,籍置大寺,遣兵守之,給日食,俾制防城具,隨其所習,以故周師卒不得覘城中虛實。



 周師為浮梁渦口,命張從恩、焦繼勛守之,廷謂語仁謙曰:「此濠、壽之患也。彼以騎士勝,故利於陸;我以舟師銳,故便於水。今夏久雨,淮流泛溢,願假舟兵二千,斷其橋,屠其城,直抵壽春。」仁謙初沮其議,不得已從之,即輕棹銜枚抵其橋,麾兵斷笮,悉焚之。周師
 大衄,死者不可計,焚其資糧而還。以功授武殿使。周師退保定遠,又募壯士為負販狀入定遠,偵軍多寡及守將之名。還曰:「武行德、周務勍也。」廷謂曰:「是可圖也。」又籍鄉兵萬餘洎卒五千,日夕訓練,依山銜枚設伏以破之,周師大潰,行德單騎脫走。時有以玉帛子女餉廷謂者,悉拒之,唯取良馬二百匹以獻。以功為滁州刺史、上淮巡檢應援兵馬都監。及紫金山之戰,南唐諸將多歸降者,獨廷謂以全軍還守濠州,追不能及。時濠守欲棄城
 走,廷謂止之。俄加本州團練使,繕戈甲,治溝壘,常若敵至。是秋,周師復至,表於景請援,且言周兵四臨,乞卑辭請和,以固鄰好。夜出敢死士千餘襲周營,焚頭車洞屋,周師蹂躪死者甚眾。既而援兵不至,周師急擊,廷謂集諸軍壘門之外,南望大慟而降於周。至山陽,見世宗,特加宴勞,賜金帶、襲衣、良馬、器皿,拜亳州防禦使。以其弟本州馬步都校廷贊為和州刺史。命攻天長軍,降其將馬贇。又為樓櫓戰棹左右廂都監,俄歸譙郡。



 宋初,從征
 上黨,再知亳州。乾德二年代還,改絳州防禦使。兩川平,馮瓚知梓州,為僕夫所訟,召廷謂為靜江軍節度觀察留後以代之。州承舊政,有莊宅戶、車腳戶,皆隸州將,鷹鷂戶日獻雉兔,田獵戶歲入皮革;又有鄉將、都將、鎮將輩互擾閭里,廷謂悉除之。開寶五年,卒,年五十四。



 廷謂性恭謹,事母以孝聞,未嘗不束帶立侍。子延浚。廷謂兄廷諭,仕南唐為太子洗馬致仕,宋初至秘書監。廷諭子
 延澤。



 延浚字利川。幼謙和。廷謂為靜江軍節度使,延浚為桂州牙內都指揮使。廷謂卒,太祖錄延浚為供奉官,屢使西北,宣諭機事。



 太平興國初,以內庭宣索及殿前賜繼、移文庫務未有專領之者。乃置合同憑由印,命延浚與內藏庫副使劉蒙正掌之。又領八作司及督治汴河。



 雍熙三年,改崇儀使。詔與翟守素、田仁朗、王繼恩往河北,分路按行諸州城壘,發鎮兵葺之。端拱二年,詔建河北方田,命延浚等五人共往規畫,會罷其務而止。淳化
 四年,李順亂,改西京作坊使,充成都十州都巡檢使。時成都將陷,延浚單騎入城,與郭載議募亡卒退保劍門,賊數千來躡其後,擊破之。王繼恩率兵至,以延浚為先鋒壕砦使,即領兵倍道先進。賊出探騎數十,延浚悉禽之,盡得賊機事。延浚易旗變號,賊不知覺,斬關掩入,斬千餘級。繼恩又請延浚知漢州,州經兵燹,廨舍、橋梁、城砦悉毀。延浚募軍民葺之,又率州帑以應軍須。錄功,改洛苑使。又命率兵屯遂州,劍門鈐轄、轉運使劉錫言其
 勞,詔書嘉獎。真宗初,改內園使。代還,會河朔用兵,延浚馳往邊城,按視砦壘。咸平二年,疾卒。子有倫,為供奉官、閣門祗候。



 延澤字德潤,南唐試秘書省正字。乾德中,四遷著作佐郎,轉殿中丞、知建州。淳化二年,太宗聞延澤洎右贊善大夫董元亨皆好學,博通典籍,詔宰相召問經史大義,皆條對稱旨,命為史館檢討。歷國子《周易》博士、國子博士。咸平中求休退,授虞部員外郎致仕。居濠州城南,有
 小園以自娛,其詠牡丹千餘首。聚圖籍萬餘卷,手自刊校。範杲、韓丕皆與之游。景德初卒。元亨亦至虞部員外郎,嘗纘《玄門碑志》三十卷。



 趙延進,澶州頓丘人。父暉,周太子太師。暉為偏將時,趙在禮據鄴。延進頗親學,嘗與軍中少年入民家,競取財賄,延進獨持書數十編以歸,同輩哂之。



 漢末,暉領鳳翔節度,未赴鎮,王景崇據城反,命暉為都招討使擊之。延進年十八,屢當軍鋒。景崇平,延進奉捷奏以入,授鳳翔
 牙內指揮使,領貴州刺史。暉徙宋州,亦從為牙職,改領榮州刺史。睢陽有盜數百,各立酋帥,為民患。延進以父命,領牙兵千餘悉禽戮之,詔書褒美。丁外艱,表求持服。既終喪,周世宗征淮南,延進獻萬縑以助軍,仍請對,世宗召見之。時延進有從兄為虎捷都虞候、帳前橫沖指揮使,世宗指延進語之曰:「爾弟拳勇有謀,將授以禁軍大校。」延進自陳好讀書,不願也。翌日,授右千牛衛將軍、濠州兵馬鈐轄,從征瓦橋關,為隨駕金吾街仗使。



 宋初,
 遷右羽林軍將軍、濠州都監。會伐蜀,以襄州當川路津要,命為鈐轄、同知州務。蜀平,專領郡事。漢江水歲壞堤,害民田,常興工修護,延進累石為岸,遂絕其患。入為兩浙、漳泉國信使。開寶二年,授右龍武將軍、知靈州,以母老願留,得權判右金吾街仗使,歷知河中府、梓、相、青三州。



 太平興國中,大軍平並州,討幽薊,皆為攻城八作壕砦使。嘗詔督造炮具八百,期以半月,延進八日成。太宗親試之,大悅。又令主城北諸洞子。及班師,命與孟玄哲、
 藥可瓊留屯定州。遼人擾邊,命延進與崔翰、李繼隆將兵八萬御之,賜陣圖,分為八陣,俾以從事。師次滿城,遼騎坌至,延進乘高望之,東西亙野,不見其際。翰等方按圖布陣,陣去各百步,士眾疑懼,略無鬥志。延進謂翰等曰:「主上委吾等以邊事,蓋期於克敵爾。今敵眾若此,而我師星布,其勢懸絕,彼若持我,將何以濟!不如合而擊之,可以決勝。違令而獲利,不猶愈於辱國乎?」翰等曰:「萬一不捷,則若之何?」延進曰:「倘有喪敗,則延進獨當其責。」
 於是改為二陣,前後相副。士眾皆喜,三戰,大破之,獲人馬、牛羊、鎧甲數十萬。以功遷右監門衛大將軍、知鎮州。及代,吏民數千守闕借留,詔許留一年。俄改右領軍衛大將軍,出為高陽關、平戎軍都監兼緣邊巡檢,改鈐轄。知揚州,召入,授右屯衛大將軍,徙知相州。遷右驍衛大將軍,改知鄧州。淳化初,飛蝗不入境,詔褒之。還,判右金吾街仗事。至道二年,拜右金吾衛大將軍。咸平二年卒,年七十三,贈左武衛上將軍。



 延進姿狀秀整,涉獵經史,
 好作詩什,士流以此多之。延進妻即淑德皇后之妹,故在顯德、興國中,頗任以腹心。子昂,太平興國二年登進士第,至戶部郎中、直昭文館。



 輔超,忻州秀容人,家世業農。超少勇悍有力,晉開運中應募,隸澶州軍籍。漢乾祐中,趙思綰據永興叛,周祖護諸將討之,督兵攻城。超率驍勇十七人升雲梯,斫北門樓,樓壞而入,士卒繼進,城遂陷,以功補小校。顯德中,從太祖征淮南,常執銳前驅,定滁、泗,破淮陰,下揚州,以功
 轉日騎副兵馬使。



 宋初,從平上黨,再遷內直都知,太宗即位,以超為馬軍都軍頭。會親征太原,冒矢石攀堞先登,身被十三創,帝嘉其勇,賜錦袍、銀帶、帛五十段。詰朝,再乘城,中流矢者八,復加厚賜。大舉襲範陽,分兵三路,超隸偏將米信,為田重進先鋒,取飛狐、蔚州。遷馬步軍副都軍頭,俄出補曹州馬步軍都指揮使,領峰州刺史,改欒州。召歸,轉都軍頭。淳化三年,出為德州刺史,坐誣奏使者毆殺驛吏,責授右監門衛將軍,領誠州刺史。五
 年,復加都軍頭,領澄州刺史。真宗即位,加領獎州團練使,真拜萊州團練使,以年老願留京師,從之。景德元年卒,年七十七。



 論曰:太祖有天下,凡五代之臣,無不以恩信結之,既以安其反側,亦藉其威力,以鎮撫四方。故一時諸將吳虔裕、蔡審廷之徒,數從征討,咸有勞績焉。若馬令琮守河內,儲兵食以迎王師;解暉擊湖南,冒鋒鏑以禽敵將:此忠藎驍果,尤可稱者。漢徽之疾危辭藥,藏英之為親復
 仇,亦皆一節之美。惟張勛嗜殺,晉卿冒貨,雖立威著勤,所不取也



\end{pinyinscope}