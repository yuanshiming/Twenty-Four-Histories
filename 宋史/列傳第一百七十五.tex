\article{列傳第一百七十五}

\begin{pinyinscope}

 ○吳淵餘玠汪立信向士璧胡穎冷應澂曹叔遠從子豳王萬馬光祖



 吳淵,字道父,秘閣修撰柔勝之第三子也。幼端重寡言,
 苦志力學。五歲喪母,哭泣哀慕如成人。嘉定七年舉進士,調建德縣主簿,丞相史彌遠館留之,語竟日,大悅,謂淵曰:「君,國器也,今開化新置尉,即日可上,欲以此處君。」淵對曰:「甫得一官,何敢躁進,況家有嚴君,所當稟命。」彌遠為之改容,不復強。至官,就闢令。江東九郡之冤,訟於諸使者,皆乞送淵。改差浙東制置使司干辦公事。



 丁父憂,詔以前職起復,力辭,弗許,再辭,且貽書政府曰:「人道莫大於事親,事親莫大於送死,茍冒哀求榮,則平生大
 節已掃地矣,他日何以事君?」時丞相史嵩之方起復,或曰:「得無礙時宰乎?」淵弗顧,詔從之。服除,差浙東提舉茶鹽司干辦公事,尋改鎮江府節制司、沿江制置使司干辦公事。皆不就。知武陵縣,改揚子縣兼淮東轉運司干辦公事,添差通判真州。入為將作監丞,遷樞密院編修官兼刑部郎官,再遷秘書丞仍兼刑部郎官。以直煥章閣知平江府兼節制許浦水軍,提點浙西刑獄。



 會衢、嚴盜起,警報至,調遣將士招捕之,殲其渠魁,散其支黨,以
 功為樞密院檢詳諸房文字兼國史院編修官、實錄院檢討官兼左司。進右文殿修撰、樞密副都承旨兼右司兼檢正。適政府欲用兵中原、以據關守河為說,淵力陳其不可,大要謂「國家力決不能取,縱取之決不能守」,丞相鄭清之不樂而罷。出知江州,改江、淮、荊、浙、福建、廣南都大提點坑冶,都司袁商令御史王定劾淵,罷。侍御史洪咨夔不直之,劾定左遷。未幾,邊事果如淵言,清之致書引咎巽謝。差知鎮江府,定防江軍之擾,兼淮弄清總領,
 以功遷太府少卿,復以總領兼知鎮江,加集英殿修撰、知鎮江兼總領。進權工部侍郎,職任如舊。權兵部侍郎,權戶部侍郎,再為總領兼知鎮江。



 時淵造闕下入對,歷陳九事,甫下殿,御史唐璘擊之,璘蓋淵所薦者也。遂仍前職,提舉太平興國宮。久之,加寶章閣待制,再起知鎮江兼總領。未幾,以戶部侍郎兼知鎮江府,召赴行在。以寶章閣直學士知太平州,尋兼江東轉運使。



 時兩淮民流徙入境者四十餘萬,淵亟加慰撫而賙濟之,使之什
 伍,令土著人無相犯。旁郡流民焚劫無虛日,獨太平境內肅然無敢嘩者。以功加華文閣直學士、沿海制置使、知慶元府,不赴;以工部尚書、沿海制置副使知江州,亦不赴。升華文閣學士、知隆興府、江西安撫使兼轉運副使。會歲大侵,講行荒政,全活者七十八萬九千餘人。徙知潭州、湖南安撫使,不赴,加敷文閣學士,仍知隆興府,安撫、轉運副使如故。改知鎮江府兼都大提舉浙西沿海諸州軍、許浦、澉浦等處兵船,歲亦大侵,因淵全活者
 六十五萬八千餘人。右正言三疏劾淵,奪職。尋復職,提舉太平興國宮。未幾,改鴻慶宮。



 丁母憂,服除,進龍圖閣學士、江西安撫使兼知江州,尋為沿江制置副使兼提舉南康軍兵甲公事、節制蘄黃州、安慶府屯田使。湖南峒寇蔓入江右之境,破數縣,袁、洪大震,淵命將調兵,生禽其渠魁,亂遂平。遷兵部尚書、知平江府兼浙西兩淮發運使。尋兼知平江府,歲亦大侵,因淵全活者四十二萬三千五百餘人。兼浙西提點刑獄、知太平州兼提領
 兩淮茶鹽所,以功進端明殿學士、沿江制置使、江東安撫使兼知建康府、兼行宮留守、節制和州無為軍安慶府兼三郡屯田使。



 朝廷付淵以光、豐、蘄、黃之事,凡創司空山燕家山金剛臺三大砦、嵯峨山膺山什子山等二十二小砦,團丁壯置軍,分立隊伍,星聯棋布,脈絡貫通,無事則耕,有警則御。詔以淵興利除害所列二十有五事,究心軍民,拜資政殿大學士,職任如舊,與執政恩例,封金陵侯,復賜「錦繡堂」、「忠勤樓」大字。進爵為公,徙知福
 州、福建安撫使。改知平江府兼發運使。



 御史劉元龍劾淵,帝寢其奏,改知寧國府。累具辭免,且丐祠,以本官提舉洞霄宮。起知潭州、湖南安撫使,不赴。改知太平兼提領江、淮茶鹽所,轉荊湖制置大使、知江陵府兼夔路策應大使,兼京湖屯田大使,帶行京湖安撫制置大使。拜觀文殿學士,職任如舊,兼總領湖廣江西京西財賦、湖北京西軍馬錢糧。淵調兵二萬往援川蜀,其後力戰於白河、沮河、玉泉。寶祐五年正月朔,以功拜參知政事。越
 七日,卒,贈少師,賻銀絹以五百計。



 淵有材略,迄濟事功,所至興學養士,然政尚嚴酷,好興羅織之獄,籍入豪橫,故時有「蜈蚣」之謠。其弟潛亦數諫止之。所著《易解》及《退庵文集》、奏議。



 餘玠,字義夫,蘄州人。家貧落魄無行,喜功名,好大言。少為白鹿洞諸生,嘗攜客入茶肆,毆賣茶翁死,脫身走襄淮。時趙葵為淮東制置使,玠作長短句上謁,葵壯之,留之幕中。未幾,以功補進義副尉,擢將作監主簿、權發遣
 招進軍,充制置司參議官,進工部郎官。



 嘉熙三年,與大元兵戰於汴城、河陰有功,授直華文閣、淮東提點刑獄兼知淮安州兼淮東制置司參謀官。淳祐元年,玠提兵應援安豐,拜大理少卿,升制置副使。進對:「必使國人上下事無不確實,然後華夏率孚,天人感格。」又言:「今世胄之彥,場屋之士,田里之豪,一或即戎,即指之為粗人,斥之為噲伍。願陛下視文武之士為一,勿令偏有所重,偏必至於激,文武交激,非國之福。」帝曰:「卿人物議論皆不
 常,可獨當一面,卿宜少留,當有擢用。」乃授權兵部侍郎、四川宣諭使,帝從容慰遣之。



 玠亦自許當手挈全蜀還本朝,其功日月可冀。



 尋授兵部侍郎、四川安撫制置使兼知重慶府兼四川總領兼夔路轉運使。自寶慶三年至淳祐二年,十六年間,凡授宣撫三人,制置使九人,副四人,或老或暫,或庸或貪,或慘或繆,或遙領而不至,或開隙而各謀,終無成績。於是東、西川無復統律,遺民咸不聊生,監司、戎帥各專號令,擅闢守宰,蕩無紀綱,蜀日
 益壞。及聞玠入蜀,人心粗定,始有安土之志。



 玠大更敝政,遴選守宰,築招賢之館於府之左,供張一如帥所居,下令曰:「集眾思,廣忠益,諸葛孔明所以用蜀也。欲有謀以告我者,近則徑詣公府,遠則自言於郡,所在以禮遣之,高爵重賞,朝廷不吝以報功,豪傑之士趨期立事,今其時矣。」士之至者,玠不厭禮接,咸得其歡心,言有可用,隨其才而任之;茍不可用,亦厚遺謝之。



 播州冉氏兄弟璡、璞,有文武才,隱居蠻中,前後閫帥闢召,堅不肯起,聞
 玠賢,相謂曰:「是可與語矣。」遂詣府上謁,玠素聞冉氏兄弟,刺入即出見之,與分廷抗禮,賓館之奉,冉安之若素有,居數月,無所言。玠將謝之,乃為設宴,玠親主之。酒酣,坐客方紛紛競言所長,璡兄弟飲食而已。玠以微言挑之,卒默然。玠曰:「是觀我待士之禮何如耳。」明日更闢別館以處之,且日使人窺其所為。兄弟終日不言,惟對踞,以堊畫地為山川城池之形,起則漫去,如是又旬日,請見玠,屏人曰:「某兄弟辱明公禮遇,思有以少裨益,非敢
 同眾人也。為今日西蜀之計,其在徙合州城乎?」玠不覺躍起,執其手曰:「此玠志也,但未得其所耳。」曰:「蜀口形勝之地莫若釣魚山,請徙諸此,若任得其人,積粟以守之,賢於十萬師遠矣,巴蜀不足守也。」玠大喜曰:「玠固疑先生非淺士,先生之謀,玠不敢掠以歸己。」遂不謀於眾,密以其謀聞於朝,請不次官之。詔以璡為承事郎、權發遣合州,璞為承務郎、權通判州事。徙城之事,悉以任之。命下,一府皆喧然同辭以為不可。玠怒曰:「城成則蜀賴以
 安,不成,玠獨坐之,諸君無預也。」卒築青居、大獲、釣魚、雲頂、天生凡十餘城,皆因山為壘,棋布星分,為諸郡治所,屯兵聚糧為必守計。且誅潰將以肅軍令。又移金戎於大獲,以護蜀口。移沔戎於青居,興戎先駐合州舊城,移守釣魚,共備內水。移利戎於雲頂,以備外水。於是如臂使指,氣勢聯絡。又屬嘉定俞興開屯田於成都,蜀以富實。



 十年冬,玠率諸將巡邊,直搗興元,大元兵與之大戰。十二年,又大戰於嘉定。初,利司都統王夔素殘悍,號「王
 夜叉」,恃功驕恣,桀驁裯受節度,所至劫掠,每得富家,穴箕加頸,四面然箕,謂之「蟆蝕月」,以弓弦系鼻下,高懸於格,謂之「錯系喉」,縛人兩股,以木交壓,謂之「乾榨油」,以至用醋灌鼻、惡水灌耳口等,毒虐非一,以脅取金帛,稍不遂意,即死其手,蜀人患苦之。且悉斂部將倅馬以自入,將戰,乃高其估賣與之。朝廷雖知其不法,在遠不能詰也。大帥處分,少不嗛其意,則百計撓之,使不得有所為。玠至嘉定,夔帥所部兵迎謁,才贏弱二百人。玠曰:「久聞
 都統兵精,今疲敝若此,殊不稱所望。」夔對曰:「夔兵非不精,所以不敢即見者,恐驚從人耳。」頃之,班聲如雷,江水如沸,聲止,圓陣即合,旗幟精明,器械森然,沙上之人彌望若林立,無一人敢亂行者。舟中皆戰掉失色,而玠自若也。徐命吏班賞有差。夔退謂人曰:「儒者中乃有此人!」



 玠久欲誅夔,獨患其握重兵居外,恐輕動危蜀,謀於親將楊成,成曰:「夔在蜀久,所部兵精,前時大帥,夔皆勢出其右,意不止此也。視侍郎為文臣,必不肯甘心從令,今
 縱弗誅,養成其勢。後一舉足,西蜀危矣。」玠曰:「我欲誅之久矣,獨患其黨與眾,未發耳。」成曰:「侍郎以夔在蜀久,有威名,孰與吳氏?夔固弗若也。夫吳氏當中興危難之時,能百戰以保蜀,傳之四世,恩威益張,根本益固,蜀人知有吳氏而不知有朝廷。一旦曦為叛逆,諸將誅之如取孤豚。況夔無吳氏之功,而有曦之逆心,恃豨突之勇,敢慢法度,縱兵殘民,奴視同列,非有吳氏得人之固也。今誅之,一夫力耳,待其發而取之,難矣。」玠意遂決,夜召夔
 計事,潛以成代領其眾,夔才離營,而新將已單騎入矣,將士皆愕眙相顧,不知所為。成以帥指譬曉之,遂相率拜賀,夔至,斬之。成因察其所與為惡者數人,稍稍以法誅之。乃薦成為文州刺史。



 戎帥欲舉統制姚世安為代,玠素欲革軍中舉代之敝,以三千騎至雲頂山下,遣都統金某往代世安,世安閉關不納。且有危言,然常疑玠圖己。屬丞相謝方叔家子侄自永康避地雲頂,世安厚結之,求方叔為援。方叔因倡言玠失利戎之心,非我調
 停,且旦夕有變,又陰嗾世安密求玠之短,陳於帝前。於是世安與玠抗,玠鬱鬱不樂。寶祐元年,聞有召命,愈不自安,一夕暴下卒,或謂仰藥死。蜀之人莫不悲慕如失父母。



 玠自入蜀,進華文閣待制,賜金帶,權兵部尚書,進徽猷閣學士,升大使,又進龍圖閣學士、端明殿學士,及召,拜資政殿學士,恩例視執政。其卒也,帝輟朝,特贈五官。以監察御史陳大方言奪職。六年,復之。



 玠之治蜀也,任都統張實治軍旅,安撫王惟忠治財賦,監簿朱文炳
 接賓客,皆有常度。至於修學養士,輕徭以寬民力,薄征以通商賈。蜀既富實,乃罷京湖之餉;邊關無警,又撤東南之戍。自寶慶以來,蜀閫未有能及之者。惜其遽以太平自詫,進蜀錦蜀箋,過於文飾。久假便宜之權,不顧嫌疑,昧於勇退,遂來讒賊之口;而又置機捕官,雖足以廉得事情,然寄耳目於群小,虛實相半,故人多懷疑懼。至於世安拒命,玠威名頓挫,齎志以沒。有子曰如孫,取「當如孫仲謀」之義,遭論改師忠,歷大理寺丞,為賈似道所
 殺。



 汪立信,澈從孫也。立信曾大父智從澈宣諭湖北,道六安,愛其山水,因居焉。



 淳祐元年,立信獻策招安慶劇賊胡興、劉文亮等,借補承信郎。六年,登進士第,理宗見立信狀貌雄偉,顧侍臣曰:「此閫帥才也。」授烏江主簿,闢沿江制幕。知桐城縣,未上,闢荊湖制司干辦、通判建康府。荊湖制置趙葵闢充策應使司及本司參議官。葵去而馬光祖代之,立信是時猶在府也。



 鄂州圍解,賈似道既
 罔上要功,惡閫外之臣與己分功,乃行打算法於諸路,欲以軍興時支散官物為罪,擊去之。光祖與葵素有隙,且欲迎合似道,被旨即召吏稽勾簿書,卒不能得其疵。乃以開慶二年正月望夕,張燈宴設錢三萬緡為葵放散官物聞於朝。立信力爭之,謂不可,且曰:「方艱難時,趙公蒞事勤勞,而公以非理捃拾之。公一旦去此,後來者復效公所為,可乎?」光祖怒曰:「吾不才不能為度外事,知奉朝命而已。君他日處此,勉為之。」立信曰:「使某不為則
 已,果為之,必不效公所為也。」光祖益怒,議不行,立信遂投劾去。初,立信通判江陵府,葵制置荊湖,嘗以公事劾立信,及在沿江府,亦謀議寡諧,立信於葵蓋未嘗有一日之歡也。



 擢京西提舉常平,改知昭信軍、權淮東提刑。景定元年,差知池州、提舉江東常平、權知常州、浙西提點刑獄。明年冬,即嘉興治所講行荒政。尋改知江州,充沿江制置副使、節制蘄黃興國軍馬、提舉饒州南康兵甲,升江西安撫使。乞祠祿,差知鎮江,尋充湖南安撫使、
 知潭州。至官,供帳之物悉置官庫,所積錢連歲代納潭民夏稅,貧無告者予錢粟,病者加藥餌,雨雪旱潦軍民皆有給。興學校,士習為變。以潭為湖湘重鎮,創威敵軍,所募精銳數千人,後來者果賴其用。權兵部尚書、荊湖安撫制置、知江陵府。



 時襄陽被圍危急,立信上疏「請益安陸府屯兵,凡邊戍皆不宜抽減,黃州守臣陳奕素蓄異志,朝廷宜防之。」乃移書似道,謂:「今天下之勢十去八九,而君臣宴安不以為虞。夫天之不假易也,從古以然,
 此誠上下交修以迓續天命之幾,重惜分陰以趨事赴工之日也。而乃酣歌深宮,嘯傲湖山,玩歲心妻日,緩急倒施,卿士師師非度,百姓鬱怨非上,以求當天心,俯遂民物,拱揖指揮而折沖萬里者,不亦難乎!為今日之計者,其策有三。夫內郡何事乎多兵,宜盡出之江幹,以實外御。算兵帳見兵可七十餘萬人,老弱柔脆,十分汰二,為選兵五十餘萬人。而沿江之守,則不過七千里,若距百里而屯,屯有守將,十屯為府,府有總督,其尤要害處,輒
 參倍其兵。無事則泛舟長淮,往來游徼,有事則東西齊奮,戰守並用。刁鬥相聞,饋餉不絕,互相應援,以為聯絡之固。選宗室親王、忠良有乾用大臣,立為統制,分東西二府,以蒞任得其人,率然之勢,此上策也。久拘聘使,無益於我,徒使敵得以為辭,請禮而歸之,許輸歲幣以緩師期,不二三年,邊遽稍休,藩垣稍固,生兵日增,可戰可守,此中策也。二策果不得行,則天敗我也,若銜璧輿櫬之禮,則請備以俟。」似道得書大怒,抵之地,詬曰:「瞎賊狂
 言敢爾。」蓋以立信目微眇雲。尋中以危法廢斥之。



 咸淳十年,大元兵大舉伐宋,似道督諸軍出次江上,以立信為端明殿學士、沿江置使、江淮招討使,俾就建康府庫募兵以援江上諸郡。立信受詔不辭,即日上道,以妻子託愛將金明,執其手曰:「我不負國家,爾亦必不負我。」遂行。與似道遇蕪湖,似道拊立信背哭曰:「不用公言,以至於此。」立信曰:「平章、平章,瞎賊今日更說一句不得。」似道問立信何向?曰:「今江南無一寸乾凈地,某去尋一片
 趙家地上死,第要死得分明爾。」既至,則建康守兵悉潰,而四面皆北軍。立信知事不可成,嘆曰:「吾生為宋臣,死為宋鬼,終為國一死,但徒死無益耳,以此負國。」率所部數千人至高郵,欲控引淮漢以為後圖。



 已而聞似道師潰蕪湖,江漢守臣皆望風降遁。立信嘆曰:「吾今日猶得死於宋土也。」乃置酒召賓佐與訣,手為表起居三宮,與從子書,屬以家事。夜分起步庭中,慷慨悲歌,握拳撫案者三,以是失聲,三日扼吭而卒。以光祿大夫致仕,遺表
 聞,贈太傅。



 大元丞相伯顏入建康,金明以其家人免,或惡立信於伯顏,以其二策及其死告,且請戮其孥,伯顏嘆息久之,曰:「宋有是人,有是言哉!使果用,我安得至此。」命求其家厚恤之,曰:「忠臣之家也。」金明以立信之喪歸葬丹陽。



 立信子麟,內書寫機宜文字,在建康不肯從眾降,崎嶇走閩以死。



 初,立信之未仕也,家窶甚。會歲大侵,吳淵守鎮江,命為粥以食流民,使其客黃應炎主之。應炎一見立信,與語,心知其非常人,言於淵,淵大奇之,禮
 以上客,凡共張服御視應炎為有加,應炎甚怏怏。淵解之曰:「此君,吾地位人也,但遭時不同耳。君之識度志業,皆非其倫也,盍少下之。」是年,試江東轉運司,明年登第,後其踐歷略如淵而卒死於難,人謂淵能知人云。



 向士璧,字君玉,常州人。負才氣,精悍甚自好,紹定五年進士,累通判平江府,以臣僚言罷。起為淮西制置司參議官,又以監察御史胡泓言罷。起知高郵軍,制置使丘崇又論罷。起知安慶府、知黃州,遷淮西提點刑獄兼知
 黃州,加直寶章閣,仍舊職,奉鴻禧祠。特授將作監、京湖制置參議官,進直煥章閣、湖北安撫副使兼知峽州,兼歸峽施黔、南平軍、紹慶府鎮撫使,遷太府少卿、大理卿,進直龍圖閣。合州告急,制置使馬光祖命士璧赴援,數立奇功。帝亦語群臣曰:「士璧不待朝命,進師歸州,且捐家貲百萬以供軍費,其志足嘉。」進秘閣修撰、樞密副都承旨,仍舊職。



 開慶元年,涪州危,又命士璧往援,北兵夾江為營,長數十里,阻舟師不能進至浮橋。時朝廷自揚
 州移賈似道以樞密使宣撫六路,進駐峽州,檄士璧以軍事付呂文德,士璧不從,以計斷橋奏捷,具言方略。未幾,文德亦以捷聞。士璧還峽州,方懷傾奪之疑,尋闢為宣撫司參議官,遷湖南安撫副使兼知潭州,兼京西、湖南北路宣撫司參議官,加右文殿修撰,尋授權兵部侍郎、湖南安撫使兼知潭州。頃之,升湖南制置副使。大元將兀良哈OA兵自交址北還,前鋒至城下,攻圍急,士璧極力守御,聞後隊且至,遣王輔祐率五百人往覘之,以
 易正大監其軍,遇於南嶽市,一戰有功,潭州圍遂解。事聞,賜金帶,令服系,進兵部侍郎兼轉運使,餘依舊職。



 似道入相,疾其功,非獨不加賞,反諷監察御史陳寅、侍御史孫附鳳一再劾罷之,送漳州居住。又稽守城時所用金谷,逮至行部責償。幕屬方元善者,極意逢迎似道意,士璧坐是死,復拘其妻妾而征之。其後元善改知吉水縣,俄歸得狂疾,常呼士璧。時輔祐亦遠謫,及文天祥起兵召輔祐於謫所,則死矣。



 德祐元年三月,詔追復元官,
 仍還從官恩數,立廟潭州。明年正月,太府卿柳岳乞錄用其子孫,詔從之。



 胡穎,字叔獻,潭州湘潭人。父TS,娶趙方弟雍之女,二子,長曰顯,有拳勇,以材武入官,數有戰功,事見《趙範傳》。穎自幼風神秀異,機警不常,趙氏諸舅以其類己,每加賞鑒。成童即能倍誦諸經,中童子科,復從兄學弓馬,母不許,曰:「汝家世儒業,不可復爾也」。遂感勵苦學,尤長於《春秋》。



 紹定三年,範討李全,檄穎入幕,穎常微服行諸營,察
 眾志向,歸必三鼓。後全敗,遣穎獻俘於朝,以賞補官。五年,登進士第,即授京秩。歷官知平江府兼浙西提點刑獄,移湖南兼提舉常平,即家置司。性不喜邪佞,尤惡言神異,所至毀淫祠數千區,以正風俗。衡州有靈祠,吏民夙所畏事,潁撤之,作來諗堂奉母居之,嘗語道州教授楊允恭曰:「吾夜必瞑坐此室,察影響,咸無有。」允恭對曰:「以為無則無矣,從而察之。則是又疑其有也。」穎甚善其言。



 以樞密都承旨為廣東經略安撫使。潮州僧寺有大
 蛇能驚動人,前後仕於潮者皆信奉之。前守去,州人心疑焉,以為未嘗詣也;已而旱,咸咎守不敬蛇神故致此,後守不得已詣焉,已而蛇蜿蜒而出,守大驚得疾,旋卒。穎至廣州,聞其事,檄潮州令僧舁蛇至,至則其大如柱而黑色,載以闌檻,穎令之曰:「爾有神靈當三日見變怪,過三日則汝無神矣。」既及期,蠢然猶眾蛇耳,遂殺之,毀其寺,並罪僧。移節廣西,尋遷京湖總領財賦。咸淳間卒,贈四官。



 穎為人正直剛果,博學強記,吐辭成文,書判下
 筆千言,援據經史,切當事情,倉卒之際,對偶皆精,讀者驚嘆。臨政善斷,不畏強御。在浙西,榮王府十二人行劫,穎悉斬之。一日輪對,理宗曰:「聞卿好殺。」意在浙獄,穎曰:「臣不敢屈太祖之法以負陛下,非嗜殺也。」帝為之默然。



 冷應澄,字公定,隆興分寧人。寶慶元年進士,調廬陵主簿,即以廉能著。有愬事臺府者,必曰:「願下廬陵清主簿。」尤為楊長孺所識拔。調靜江府司錄參軍,治獄平恕,轉運使範應鈴列薦於朝。



 知萬載縣,大修學舍,招俊秀治
 其業,旌其通經飭行者以勸。歲歉,棄孩滿道,乃下令恣民收養,所棄父母不得復問,全活甚眾。葉夢得列其行事,風厲餘邑。通判道州。入監行在榷貨務,遷登聞鼓檢院。—



 景定元年,奉使督餉江上,還,知德慶府。前守政不立,縱豪吏漁獵,峒獠遂大為變,逼城六十里而營。應澄未入境,馳檄諭之曰:「汝等不獲已至此,新太守且上,轉禍為福,一機也。脅從影附,亦宜早計去就,不然不免矣。」獠感悟欲自歸,惑謀主不果,眾稍引去,應澄知其勢解,即
 厲士馬,出不意一鼓擒之,縱遣歸農,猶千餘人,乃請諸監司,歸郡之避難留幕府者,誅豪吏之激禍者。初經略雷宜中意應澄必以濟師來請,及是嘆服,亟上其事,薦應澄可大用。



 屬縣租賦,諉道阻久不至郡,應澄為之期曰:「首輸者與減分,末至則償所減。」民惟恐後,不一月訖事。凡諸綱官廩稍軍券,前政積不得者悉補還之,上下欣附。應澄亦極力摩撫,與為簡便。期年報政,奏罷抑配鹽法及乞用楮券折銀綱等五事,以紓民力,詔就升本
 道提舉常平兼轉運使,俾行其說。首劾守令貪橫不法十餘人,列郡肅然。最聞,加直秘閣。時經略使陳宗禮入為參知政事,帝問誰可代卿者,宗禮以應澄對,旋召為都官郎官,未行,就升直寶章閣、知廣州,主管廣南東路經略安撫司公事、馬步軍都總管,領漕、庾如故。



 五司叢劇,應澄即分時理務,不擾不倦,常曰:「治官事當如家事,惜官物當如己物。方今國計內虛,邊聲外震,吾等受上厚恩,安得清談自高以誤世。陶士行、卞望之吾師也。」自
 聞襄、樊受圍,日繕器械,裕財粟,以備倉卒,後卒賴其用,屢平大寇,未嘗輕殺,笞杖以降,亦加審慎,至其臨事輒斷,雖勢要不為撓奪。後卒於家。



 曹叔遠,字器遠,溫州瑞安人。少學於陳傅良。登紹熙元年進士第。久之,李壁薦為國子學錄,迕韓侂胄,罷。通判涪州,後守遂寧,營卒莫簡苦總領所侵刻,相率稱亂,勢張甚,入遂寧境,輒戢其徒無肆暴,曰:「此江南好官員也。」入朝,為工部郎,出知袁州。以太常少卿召,權禮部侍郎,
 遇事獻替,多所裨益。終徽猷閣待制,謚文肅。嘗編《永嘉譜》,識者謂其有史才。子觱,孫邰,皆登進士第。族子豳。



 豳字西士,少從錢文子學,登嘉泰二年進士第,授安吉州教授。調重慶府司法參軍,郡守度正欲薦之,豳辭曰:「章司錄母老,請先之。」正敬嘆。改知建昌縣,復故尚書李常山房,建齋舍以處諸生。擢秘書丞兼倉部郎官。出為浙西提舉常平,面陳和糴折納之敝,建虎丘書院以祀尹焞。移浙東提點刑獄,寒食放囚歸祀其先,囚感泣如
 期至。召為左司諫,與王萬、郭磊卿、徐清叟俱負直聲,當時號「嘉熙四諫」。上疏言:「立太子、厚倫紀,以弭火災」。又論餘天錫、李鳴復之過,迕旨,遷起居郎。進禮部侍郎,不拜,疏七上,進古詩以寓規正。久之,起知福州,再以侍郎召,為臺臣所沮而止。遂守寶章閣待制致仕,卒謚文恭。子愉老,亦登進士第。



 王萬,字處一,家世婺州,父游淮間,萬因生長濠州。少忠伉有大志,究心當世急務,尤精於邊防要害。登嘉定十
 六年進士第,調和州教授。端平元年,主管尚書吏部架閣文字,遷國子學錄。明年,添差通判鎮江府。



 時金初滅,當路多知其人豪也,咨問者旁午。鄭清之初謀乘虛取河洛,萬謂當急為自治之規。已而大元兵壓境。三邊震動,理宗下罪己詔,吳泳起草,又以咨萬,萬謂:「兵固失矣,言之甚,恐亦不可。今邊民生意如發,宜以振厲奮發,興感人心。」為條具沿邊事宜,遍告大臣要官,謂:「長淮千里,中間無大山澤為限,擊首尾應,正如常山蛇勢,首當並
 兩淮為一制閫之命是聽。兩淮惟濠州居中。濠之東為盱眙,為楚,以達鹽城,淮流深廣,敵所難度。濠之西為安豐,為光,以達信陽,淮流淺澀,敵每揭厲以涉之。法當調揚州北軍三千人,自淮東搗虛,常往來宿、亳間,使敵無意於東,而我並力淮西。淮西則又惟合肥居江、淮南北之中,法當建制置司合肥,而以濠梁、安豐、光州為臂,以黃岡為肘後緩急之助。又必令荊、襄每候西兵東來,輒尾之,使淮、襄之勢亦合,而後大規模可立。」



 論用兵,則謂:「
 當以五千人為屯,每屯一將、二長,一大將一路,又合一大將而並合於制置為總統。淮東可精兵三萬,光、黃可二萬,東西夾擊,而沿江制司會合肥兵共二萬,以牽制其中。行則給營陣,止則依城壘;行則齎乾糧,止則就食州縣。」論屯田,則謂:「當於新復州軍,東則海、邳,所依者水之險,西則唐、鄧,所依者山之險,畫此無地無田不耕,則歸附新軍流落餘民亦有固志。」



 又謂:「戎司舊分地戍守,殿步兵戍真、揚、六合,鎮江兵戍揚、楚、盱眙,建康馬司兵戍滁、
 濠、定遠,都統司兵戍廬、和、安豐,以至池司兵戍舒、蘄、巢縣,江司兵戍蘄、黃、浮光,地勢皆順,皆以統制部之出外,而皆常有帥臣居內,以本軍財賦葺營柵,撫士卒,備器械,以故軍事常整辦。遇警急則帥臣親統重兵以行。比乃有以建康馬帥而知黃州者,都統而知光州者,以池司都統而在楚州,以鎮江都統而在應天者,將不知兵,兵不屬將,往往以本軍之財,資他處之用,以致營柵壞而莫修,士卒貧而莫給,器械鈍而莫繕,宜與盡還舊制。」
 及請寬邊民,請團民兵,請援浮光,請邊民之能捍邊者,常厚其賞而小其官,使常得其力。其後兵興用窘,履畝之令行,則又言之廟堂曰:「今名更化,可反為故相之所不為乎?」其他敷陳,往往累數萬言,其自任之篤,切於當世如此。三年。授樞密院編修官。



 嘉熙六年,兼權屯田郎中,因轉對,言:「天命去留原於君心,陛下一一而思之,凡惻然有觸於心而未能安者,皆心之未能同乎天者也。天不在天,而在陛下之心,茍能天人合一,永永勿替,天
 命在我矣。」差知臺州,至郡日,惟蔬飯,終日坐廳事,事至立斷,吏無所售,往往改業散去,民亦化之不復訟,上下肅然,郡以大治。才五月,乞祠去。三年,遷屯田員外郎兼編修,轉對,言:「君臣上下盡克私心,以服人心,以回天心。」遷尚右郎官,尋兼崇政殿說書。



 四年,擢監察御史。首論史宅之,故相之子,曩者弄權,不當復玷從班。上命丞相再三諭旨,迄不奉詔。上不得已,出宅之知平江府。又論之,疏凡五上,史嵩之自江上董師入相,萬又首論之,謂
 其「事體迫遽,氣象傾搖,太學生欲趣其歸,則賄賂之跡已形。近或謂有族人發其私事,肆為醜詆者,以相國大臣而若此,非書之所謂大臣矣」。然當時論相之事已決,疏入,遷大理少卿。萬即日還常熟寓舍。遷太常少卿,辭。差知寧國府,辭。召赴行在奏事,出為福建提點刑獄,加直煥章閣、四川宣諭司參議官,皆力辭,乞休致。詔特轉朝奉郎,守太常少卿致仕,卒。嵩之罷相,眾方交論其非,上思萬先見,親賜御札,謂萬「立朝蹇諤,古之遺直,為郡
 廉平,古之遺愛。聞其母老家貧,朕甚念之,賜新會五千貫,田五百畝,以贍給其家。」



 初,萬之學專有得於「時習」之語,謂學莫先於言顧行,言然而行,未然者非言之偽也,習未熟也,熟則言行一矣。故終其身,行無不顧其言。發於設施論諫,皆根於中心。遺文有《時習編》及其他奏札及論天下事者凡十卷。



 馬光祖,字華父,婺州金華人。寶慶二年進士,調新喻主簿,已有能名。從真德秀學。改知餘干縣,差知高郵軍,遷
 軍器監主簿,差充督視行府參議官。奉雲臺祠。差知處州,監登聞鼓院,進太府寺丞兼莊文府教授、右曹郎官。出知處州,乞降僧道牒振濟,詔從之。加直秘閣,浙東提舉常平。移浙西提點刑獄,時暫兼權浙西提舉常平。起復軍器監、總領淮東軍馬錢糧兼知鎮江。進直徽猷閣、江西轉運副使兼知隆興府。以右正言劉漢弼言罷。後九年,起直徽猷閣、知太平州、提領江西茶鹽所。進直寶文閣,遷太府少卿,仍知太平州、提領江、淮茶鹽所。遷司
 農卿、淮西總領兼權江東轉運使。



 拜戶部尚書兼知臨安府、浙西安撫使。帝諭丞相謝方叔趣入覲,乞嚴下海米禁,歷陳京師艱食、和糴增價、海道致寇三害。加寶章閣直學士、沿江制置使、江東安撫使、知建康府兼行宮留守兼節制和州無為軍安慶府三郡屯田使,加煥章閣,尋加寶章閣學士。始至官,即以常例公用器皿錢二十萬緡支犒軍民,減租稅,養鰥寡孤疾無告之人,招兵置砦,給錢助諸軍昏嫁。屬縣稅折收絲綿絹帛,倚閣除
 免以數萬計。興學校,禮賢才,闢召僚屬,皆極一時之選。



 拜端明殿學士、荊湖制置、知江陵府,去而建康之民思之不已。帝聞,命以資政殿學士、沿江制置大使、江東安撫使再知建康,士女相慶。光祖益思寬養民力,興廢起壞,知無不為,蠲除前政逋負錢百餘萬緡,魚利稅課悉罷減予民,修建明道、南軒書院及上元縣學。手尊節費用,建平糴倉,貯米十五萬石,又為庫貯糴本二百餘萬緡,補其折閱,發糴常減於市價,以利小民。修飭武備,防拓
 要害,邊賴以安。其為政寬猛適宜,事存大體。



 公田法行,光祖移書賈似道言公田法非便,乞不以及江東,必欲行之,罷光祖乃可。進大學士兼淮西總領。召赴行在,遷提領戶部財用兼知臨安府、浙西安撫使。會歲饑,榮王府積粟不發廩,光祖謁王,辭以故,明日往,亦如之,又明日又往,臥客次,王不得已見焉。光祖厲聲曰:「天下孰不知大王子為儲君,大王不於此時收人心乎?」王以無粟辭;光祖探懷中文書曰:「某莊某倉若干。」王無以辭,得粟
 活民甚多。進同知樞密院事,尋差知福州、福建安撫使,以侍御史陳堯道言罷,以前職提舉洞霄宮。再以沿江制置、江東安撫使知建康,郡民為建祠六所。乞致仕,不許。咸淳三年,拜參知政事。五年,拜知樞密院事兼參知政事,以監察御史曾淵子言罷。給事中盧鉞復繳奏新命,以金紫光祿大夫致仕,卒,謚莊敏。



 光祖之在外,練兵豐財;朝廷以之為京尹,則剸治浩穰,風績凜然。三至建康,終始一紀,威惠並行,百廢無不修舉云。



 論曰:吳淵才具優長,而嚴酷累之。餘玠意氣豪雄,而志不克信。賈似道不用汪立信之策,殆天奪其魄矣。向士璧卒厄於似道,宋之不足圖存,蓋可知也。胡穎好毀淫祠,非其中之無慊,不能爾也。冷應澄安邊之才。曹叔遠、王萬皆正人端士。馬光祖治建康,逮今遺愛猶在民心,可謂能臣已。



\end{pinyinscope}