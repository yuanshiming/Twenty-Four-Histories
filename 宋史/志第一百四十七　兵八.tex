\article{志第一百四十七 兵八}

\begin{pinyinscope}

 揀選
 之
 制廩給之制



 揀選之制建隆初,令諸州召募軍士部送闕下,至則軍頭司覆驗等第,引對便坐,而分隸諸軍焉。其自廂軍而升禁兵,禁兵而升上軍,上軍而升班直者,皆臨軒親閱,
 非材勇絕倫不以應募,餘皆自下選補。



 咸平五年,於環、慶等州廂軍馬步軍六千餘人內選材勇者四千五百人,付逐砦屯防,以代禁兵。



 景德二年,宣示:「殿前、侍衛司諸禁軍中老疾者眾,蓋久從征戍,失於揀練,每抽替至京,雖量加閱視,亦止能去其尤者。今多已抽還,宜乘此息兵,精加選揀,雖議者恐其動眾,亦當斷在必行。昔太祖亦嘗患此,遂盡行揀閱,當時人情深以為懼,其後果成精兵。」樞密使王繼英等曰:「今兵革休息,不乘此時遴
 選,實恐冗兵徒費廩食。」帝曰:「然。近者契丹請盟,夏人納款,恐軍旅之情謂國家便謀去兵惜費。」乃命先於下軍選擇勇力者次補上軍。其老疾者,俟秋冬慎擇將臣令揀去之。



 三年正月,詔遣樞密都承旨韓崇訓等與殿前司、侍衛馬步軍司揀閱諸軍兵士,供備庫使、帶御器械綦政敏等分往京東、西路揀閱。八月,詔效順第一軍赴京揀閱,以補虎翼名闕。是軍皆河東人,帝念其累戍勞苦,故升獎焉。



 大中祥符二年四月,詔曰:「江南、廣東西路
 流配人等,皆以自抵憲章,久從配隸,念其遠地,每用軫懷。屬喬嶽之增封,洽溥天之大慶,不拘常例,特示寬恩。江南路宜差內殿崇班段守倫就升州、洪州,廣南東、西路差殿直、閣門祗候彭麟就桂州,與本路轉運使同勾抽諸州雜犯配軍,揀選移配淮南州軍牢城及本城。有少壯堪披帶者,即部送赴闕,當議近上軍分安排。如不願量移及赴闕者,亦聽。若地理遠處,即與轉運使同乘傳就彼,依此揀選。」



 五年正月,帝諭知樞密院王欽若等:「
 在京軍校差充外處人員,軍數不足,有妨訓練,可詔示殿前、侍衛馬步軍司簡補。禁軍逐指揮兵士內,捧日上三軍要及三百人,龍衛上四軍各二百五十人,拱聖、驍騎、驍勝、寧朔、神騎、云武騎各三百五十人,並於下次軍營升填,須及得本額等樣,及令軍頭司於諸處招揀到人內選填。營在京者引見分配,在外處者準此,仍委逐司擘畫開坐以聞。在京差出者,候替回揀選。」



 九年十一月,詔河北、河東、陜西諸州軍揀料本城兵,五百人以上
 升為一指揮,於本處置營教閱武藝,升為禁軍。



 天禧元年二月,遣使分往諸州軍揀廂軍驍壯及等者升隸上軍。六月,召選天下廂兵遷隸禁軍者,凡五千餘人。



 天聖間,嘗詔樞密院次禁軍選補法:



 凡入上四軍者,捧日、天武弓以九斗,龍衛、神衛弓以七斗,天武弩以二石七斗,神衛弩以二石三斗為中格。恩、冀員僚直、驍捷軍士選中四軍,則不復閱試。自餘招揀中者,並引對。凡員僚直闕,則以選中上軍及龍衛等樣、弓射七斗合格者充,仍
 許如龍衛例選補班直。



 凡選禁軍,自奉錢三百已上、弓射一石五斗、弩跖三石五斗、等樣及龍衛者,並親閱,以隸龍衛、神衛。凡騎禦馬直闕小底,則閱拱聖、驍騎少壯善射者充。凡弓手,內殿直以下選補殿前指揮使,射一石五斗;御龍弓箭直選補御龍直、御龍骨朵子直,東西班帶甲殿侍選補長入祗候,御龍諸直將虞候選補十將,射皆一石四斗;東西班、散直選補內殿直,捧日、員僚直、天武、龍衛、神衛親從選補諸班直,御龍骨朵子直、弓箭
 直將虞候選補十將,御龍直長行選補將虞候,射皆一石三斗;員僚、龍御、騎禦馬直小底選補散直,射皆一石二斗。凡弩手,東西班帶甲殿侍選補長騎祗候,射四石;御龍弩直將虞候選補十將,射三石八斗;長行選補將虞候,射三石五斗。其捧日、天武、龍衛親從選補弩手班、御龍弩直者,亦如之。其次別為一等,減二斗。自餘殿前指揮使、諸班直以歲久若上名出補外職者,所試弓弩鬥力皆遽減,弓自一石三斗至八斗,弩自三石二斗至
 五斗各有差。



 凡班直經上親閱隸籍者,有司勿復按試。其升軍額者,或取少壯拳勇,或旌邊有勞。至於河清遽補,牢城配軍亦間下詔選補,蓋使給役者有時而進,負罪者不終廢也。其退老疾,則以歲首,或出軍回;轉員皆揀汰,上軍以三歲。河北遇大閱亦如之。景祐元年,詔選教駿填拱聖諸軍,退其老疾為剩員,不任役者免為民。



 三年,詔選驍騎、雲騎、驍勝填拱聖,武騎、寧朔、神騎填驍騎。



 康定元年,選御輦官為禁軍。輦官二十六人遮輔臣
 喧訴,斬其首二人,餘黥隸嶺南,卒選如初。



 慶歷三年,詔韓琦、田況選京師奉錢五百已上禁軍武技精捷者,營取五人,樞密院籍記姓名,以備驅使。況因言:「今天下兵逾百萬,視先朝幾三倍,自昔養兵之冗,未有若是。且諸路宣毅、廣勇等軍孱弱眾甚,大不堪戰,小不堪役。宜分遣官選不堪戰者降為廂軍,不堪役者釋之。」上然其言。



 皇祐元年,揀河北、河東、陜西、京東西禁廂諸軍,退其罷癃為半分,甚者給糧遣還鄉里。系化外若以罪隸軍或
 嘗有戰功者,悉以剩員處之。



 三年,韓琦奏:「河北就糧諸軍願就上軍者,許因大閱自言。若等試中格,舊無罪惡,即部送闕,量材升補。」乃詔四路都總管司:「自今春秋閱,委主管選長五尺六寸已上、弓一石五斗、弩三石五斗者,並家屬部送闕。



 嘉祐二年,詔神衛水軍等以五年,諸司庫務役兵以三年一揀。五年,選京東西、陜西、河北、河東本城、牢城、河清、裝御、馬遞鋪卒長五尺三寸勝帶甲者,補禁軍。其嘗犯盜亡坐黥者,配外州軍歸遠、壯勇。



 八
 年,右正言王陶奏:「天下廂軍以歲首揀,至於禁軍雖有駐扎還日揀法,或不舉。臣竊惟調發禁軍本籍精銳,軍出之時尤當揀練。請下有司,凡調發禁軍,委當職官汰年六十已上,將校年六十五已上衰老者,如此則兵精而用省矣。」下其章。殿前、馬步軍司奏曰:「舊制,遣戍陜西、河北、河東、廣南被邊諸軍悉揀汰,餘路則無令。請自今諸軍調發,悉從揀法。」詔可。又詔:「凡選本城、牢城軍士以補龍猛等軍者,並案籍取嘗給奉錢五百及龍猛等者,
 以配龍猛;其不及等與嘗給奉錢四百以下,若百姓黥隸及龍騎等者,以配龍騎;其龍騎軍士戍還,即選填龍猛。自今本城、牢城悉三年一揀,著為令。」



 治平元年,閱親從官武技,得百二十人以補諸班直。乃詔:自今親從官,限年三十五以下者充。又詔:「如聞三路就糧兵,多老疾不勝鎧甲者,可勿拘時,揀年五十以上有子弟或異姓親屬等應樣者代之。如無,聽召外人。」是歲,詔京畿並諸路揀龍騎、壯勇、歸遠、本城、牢城、宣效六軍;河清、車營、致
 遠、窯務、鑄錢監、屯田務隸籍三十年勝鎧甲者,部送京師填龍猛等軍;其自廣南揀中者,就填江西、荊湖歸遠闕額。仍詔每三年以龍猛等軍闕數聞。又詔諸路,有步射引弓兩石、擴弩四石五斗已上者,奏遣詣闕。



 二年,詔京東教閱補禁軍。先是,京東教閱本城,自初置即番隸本路巡檢,久不選補。上聞其軍多勇壯可用者,欲示激勸,故有是詔。



 治平四年五月,揀選拱聖、神勇以下勇分,以補捧日、天武、龍、神衛闕數。



 元豐三年六月,權主管馬
 步軍司燕達言:「內外就糧退軍二十一指揮八千餘人,以禁軍小疾故揀退及武藝淺弱人配填,既不訓練,又免屯戍,安居冗食,耗蠹軍儲。若自今更不增補,庶漸銷減,候有闕,依禁軍選募,教習武世,不數年間,退軍可盡變銳士。內奉錢七百者減為五百,依五百奉錢軍等杖招揀。」從之。仍詔:「上四軍退軍改作五百奉錢軍額。」八月,殿前、步軍司虎翼十指揮出戍歸營,閔其勞苦,詔並升補為神勇指揮。廣西路經略司言:「雄略、澄海指揮闕額,
 請以諸路配送隸牢城卒所犯稍輕,及少壯任披帶者選補。」從之。



 四年四月,提舉河北義勇保甲狄諮言:「舊制,諸指揮兵給內有老疾年五十五已上、有弟侄子孫及等杖者,令承替名糧,其間亦有不堪征役者,乞年四十已上許令承替。」詔河北馬步諸軍依此。十二月,詔諸班直、上四軍,毋得簡常有罪改配人。



 元祐二年七月,詔諸路每歲於八月後解發試武藝人到闕,殿前司限次年正月,軍頭司限二月以前試驗推恩。呈試武藝人同。



 三
 年閏十二月,樞密院言:「在京諸軍兵額多闕,而京東、西路就糧禁軍往往溢額。」詔差官往逐路同長吏揀選發遣,以補其數。



 大觀元年四月,詔曰:「東南諸郡軍旅之事,久失訓齊,民雖浮弱,而阻山帶江,輕而易搖。安必慮危,誠不可忽。其諸軍事藝生疏精熟不同,非獨見將官訓練優劣,實亦系教頭能否。」樞密院請委逐路提舉訓練官妙選精熟教頭,二年一替,若能訓練精熟,然後推賞。從之。



 至若省並之法,凡軍各有營,營各有額。皇祐間,馬
 軍以四百、步軍以五百人為一營。承平既久,額存而兵闕,馬一營或止數十騎,兵一營或不滿一二百。而將校猥多,賜予廩給十倍士卒,遞遷如額不少損。帝患之,熙寧二年,始議並廢。陜西馬步軍營三百二十七,並為二百七十,馬軍額以三百人,步軍以四百人。其後凡撥並者,馬步軍營五百四十五並為三百五十五,而京師、府界、諸路及廂軍皆會總畸零,各足其常額。



 凡並營,先為繕新其居室,給遷徙費。軍校員溢,則以補他軍闕,或隨
 所並兵入各指揮,依職次高下同領。帝嘗謂輔臣曰:「天下財用,朝廷稍加意,則所省不可勝計。乃者銷並軍營,計減軍校、十將以下三千餘人,除二節賜予及傔從廩給外,計一歲所省,為錢四十五萬緡,米四十萬石,紬絹二十萬匹,布三萬端,馬蒿二百萬。庶事若此,邦財其可勝用哉!」



 初議並營,大臣皆以兵驕已久,遽並之必召亂,不可。帝不聽,獨王安石贊決之。時蘇軾言曰:「近者並軍搜卒之令猝然輕發,甚於前日矣,雖陛下不恤人言,持
 之益堅,而勢窮事礙,終亦必變。他日雖有良法美政,陛下能復自信乎?」樞密使文彥博曰:「近多更張,人情洶洶非一。」安石曰:「事合更張,豈憚此輩紛紛邪!」帝用安石言,卒並營之。自熙寧以至元豐,歲有並廢。



 元符二年,樞密院言:「已詔諸路並廢堡砦,減罷兵將,鄜延、秦鳳路已減並,餘路未見施行。」詔涇原、熙河蘭會、環慶、河東路速議以聞。



 三年,罷都護府,安撫使隸河、蘭州,以省饋運。詔邊帥減額外戍兵。



 建中靖國元年,減放秦鳳路土兵。



 大觀
 三年,詔:「昨降處分,措置東南利害,深慮事力未辦,應費不貲。其帥府、望郡添置禁軍,諸縣置弓手,並罷其壯城兵士,令帥府置一百人,餘望郡置五十人,舊多者自依舊。沿邊州軍除舊有外,罷增招壯城。帥府、望郡養馬並步人選充馬軍指揮,及支常平錢收糴封樁觔斗指揮,並罷。已添置路分鈐轄、路分都監,許令任滿。江南東西、兩浙各共差走馬承受內臣一員、帥府添置機宜文字去處,並罷。」



 四年,詔:「四輔州各減一將,其軍兵仰京畿轉
 運司將未足額並未有人,崇銳、崇威、崇捷、崇武內並廢四十四指揮已揀到人,隨等杖撥填四輔見闕禁軍。仍將逐輔系將、不系將軍兵,以住營遠近相度,重別分隸排定,及八將訓練駐扎去處,疾速開具以聞。河北、河東崇銳、崇威,河東十八指揮,河北不隸將十三指揮並廢,見管兵令總管司撥填本路禁軍闕額。河北路撥不盡人發遣上京,分填在京禁軍闕額。河東撥不盡人,並於本路禁軍額外收管。」



 宣和五年,詔:「兩浙盜賊寧息,其越
 州置捕盜指揮,可均填江東、淮東三路州軍闕額。」



 至神宗之世,則又有簡汰退軍之令。治平四年,詔揀拱聖、神勇以下軍補捧日、天武、龍衛、神衛兵闕。



 熙寧元年,詔諸路監司察州兵招簡不如法者按之,不任禁軍者降廂軍,不任廂軍者免為民。



 二年,從陳升之議,量減衛兵年四十以上稍不中程者請受。呂公弼及龍圖閣直學士陳薦皆言退軍不便。三年二月,司馬光亦曰:



 竊聞朝廷欲揀在京禁軍四十五以上微有呈切者,盡減請給,兼
 其妻子徙置淮南,以就糧食。若實有此議,竊謂非宜。何則?在京禁軍及其家屬,率皆生長京師,親姻聯布,安居樂業,衣食縣官,為日固久。年四十五未為衰老,微有呈切,尚任徵役,一旦別無罪負,減其請給,徙之淮南,是橫遭降配也。



 且國家竭天下之財養長征兵士,本欲備御邊陲。今淮南非用武之地,而多屯禁軍,坐費衣食,是養無用之兵,置諸無用之地。又邊陲常無事則已,異時或少有警急,主兵之臣必爭求益兵。京師之兵既少,必須
 使使者四出,大加召募,廣為揀選,將數倍多於今日所退之兵。是棄已教閱經戰之兵,而收市井畝之人,本欲減冗兵而冗更多,本欲省大費而費更廣,非計之得也。



 臣愚欲願朝廷且依舊法,每歲揀禁軍有不任征戰者減充小分,小分復不任執役者,放令自便在京居止,但勿使老病者尚占名籍,虛費衣糧。人情既安於所習,國家又得其力,冗兵既去,大費自省,此國家安危所系,不敢不言。



 右正言李常亦以為言。從之,是年,詔:「陜西就
 糧禁軍額十萬人,方用兵之初,其令陜西、河東亟募士補其闕。」



 四年,詔:「比選諸路配軍為陜西強猛,其以為禁軍,給賜視壯勇為優,隸步軍司,役於逐路都監、總管司。」詔廣東、福建、江西選本路配軍壯勇者,合所募兵萬人,以備征戍。三月,詔廣東路選雜犯配軍丁壯,每五百人為一指揮,屯廣州,號新澄海,如廣西之法。七月,手詔:「揀諸路小分年四十五以下勝甲者,升以為大分,五十已上願為民者聽。」舊制,兵至六十一始免,猶不即許。至是
 免為民者甚眾,冗兵由是大省。



 十年,遣官偕畿內,京東西、陜西、荊湖長吏簡募軍士,以補禁軍之闕。



 元豐元年,詔:以馬軍選上軍,上軍選諸班者,並馬射弓一石力。諸班直槍弩手闕,選親從、親事官,八並選捧日、龍衛弓箭手。



 二年,雲騎軍闕二千一百,以雲捷等軍補之。



 六年,騎兵年五十以下,教武技不成而才可以肄習者,並以為步軍。



 元祐四年,詔:「今後歲揀禁軍節級,筋力未衰者,年六十五始減充剩員。」



 八年,涇原路經略司奏:「揀選諸將
 下剩員,年六十以下精力不衰,仍充軍,以補闕額。」從之。陜西諸路如之。



 紹聖四年,樞密院言:「龍騎系雜犯軍額,闕數尚多。今欲將禁軍犯徒兵及經斷者,歲揀以填闕。」從之。



 元符元年又言:「就糧禁軍闕額,於廂軍內揀選年四十以下者填。」從之。



 宣和七年,詔京東西、淮南、兩浙帥司精選諸軍驍銳,發赴京畿輔郡兵馬制置使司。



 靖康元年,詔:「軍兵久失教習,當汰冗濫,精加揀擇。」然不能精也。方兵盛時,年五十已上皆汰為民,及銷並之久,軍額
 廢闕,則六十已上復收為兵,時政得失因可見矣。



 中興以後,兵不素練。自軍校轉補之法行,而揀選益精。大抵有疾患則選,有老弱則選,藝能不精則選,或由中軍揀補外軍,或揀外邊精銳以升禁衛。考《軍防令》,諸軍招簡等杖:天武第一軍五尺有八寸,捧日、天武第二軍、神衛五尺七寸三分,龍衛五尺有七寸,拱聖、神勇、勝捷、驍捷、龍猛、精朔五尺六寸五分,驍騎、雲騎、驍勝、宣武、殿前司虎翼、殿前司龍翼水軍五尺有六寸,武騎、寧朔、步軍司
 虎翼水軍、揀中龍衛、神騎、廣勇、龍騎、驍猛、雄勇、吐渾、擒戎、新立驍捷、驍武、廣銳、雲翼、有馬勁勇、步武、威捷、武衛、床子弩雄武、飛山雄武、神銳、振武、新招振武、新置振武、振華軍、雄武弩手、上威猛、廳子、無敵、上招收、冀州雄勝、澄海水軍弩手五尺五寸,廣捷、威勝、廣德、克勝、陜府雄勝、驍雄、雄威、神虎、保捷、清邊弩手、制勝、清澗、平海、雄武、龍德宮清衛、寧遠、安遠五尺四寸五分,克戎、萬捷、雲捷、橫塞、捉生、有馬雄略、效忠、宣毅、建安、威果、全捷、川效忠、
 揀中雄勇、懷順、忠勇、教閱忠節、神威、雄略、下威猛五尺四寸,亳州雄勝、飛騎、威遠、蕃落、懷恩、勇捷、上威武、下威武、忠節、靖安、川忠節、歸遠、壯勇、宣效五尺三寸五分、濟州雄勝、騎射、橋道、清塞、奉先、奉國、武寧、威勇、忠果、勁勇、下招收、壯武、雄節、靖江、武雄、廣節、澄海、懷遠、寧海、刀牌手、必勝五尺三寸,揀中廣效、武和、武肅、忠靖、三路廂軍五尺二寸。



 建炎三年,詔:「江南、江東、兩浙諸州軍正兵、土兵、除鎮江、越州,委守臣兵官巡檢,六分中選一分,部轄
 人年四十五以下,長行年三十五以下,合用器甲,候旨選擇赴行在。有軟弱不堪,年甲不應,或占庇不如數選發,其當職官有刑。」



 四年,詔:「神武義軍統制王□燮下閱到第三等軍兵一千六百六十人,填廂禁軍,其不任披帶者,分填嚴州新禁軍。」



 紹興二年,上謂輔臣曰:「邵青、單德忠、李捧三盜,招安至臨安日久,卿等其極揀汰。」呂頤浩、秦檜得旨與張俊同閱視,堪留者近七千人。詔命張俊選精銳,得兵五千人詣行在。



 二十年,樞密院言都統吳
 玠選中護衛西兵千人,詔隸殿司。又統制楊政選西兵三百二十五人,填步軍司。



 二十四年,詔:「御龍直見闕數,可以殿、步二司選拍試填諸班。」



 乾道二年,詔王琪選三百人充馬軍。



 慶元三年,殿司言:「正額效用萬一千五百九十二人,闕二百五十九人,於雄效內及效用帶甲拍試一石力弓、三石力弩合格人填闕額。」詔:「崇政殿祗候、親從填班直人數,特與免。其三衙舊司官兵及御馬直合揀班人,照闕額補。」



 嘉定十一年,臣僚言「今軍政所先,
 莫如汰卒。」謂「如千兵中有百人老弱,遇敵先奔,即千人皆廢矣。乞嚴敕中外將帥,務核其實。」



 其省並法自咸平始。建炎以後,臣僚屢言,軍額有闕,則並隸一等軍分,足其舊額,以便教閱,而指揮、制領、將佐之屬亦或罷或省,悉從其請。蓋當多事之秋,患兵之不足,望增補以壯軍容。事既寧息,患其有餘,必並省以核軍實,意則在乎少蘇民力也。



 嘉熙初,臣僚言:「今日兵貧若此,思變而通之。於卒伍中取強勇者,異其籍而厚其廩,且如百人之中
 揀十人,或二十,或三十,則是萬人中有三千兵矣。時試之弓弩,課之武藝,暇則馳馬擊球以為樂,秋冬使之校獵。其有材力精強,則厚賞賚之。又於其中拔其尤者,數愈少而廩愈厚,待之如子弟,倚之如腹心,緩急可用。蘇轍有言:『天子必有所私之將,將軍必有所私之士。』又必申命主帥、制領,鼓動而精擇之,假之統御之權,嚴其階級之法。將樂與士親,士樂為將用,則可以運動如意,不必別移一軍,別招新軍矣。」



 咸淳間,招兵無虛日,科降等
 下錢以萬計。奈何任非其人,白捕平民為兵,召募無法,揀選云乎哉!



 廩祿之制為農者出租稅以養兵,為兵者事徵守以衛民,其勢然也。唐以天下之兵分置藩鎮,天子府衛,中外校卒,不過十餘萬,而國用不見其有餘。宋懲五代之弊,收天下甲兵數十萬,悉萃京師,而國用不見其不足者,經制之有道,出納之有節也。國初,太倉所儲才支三、二歲。承平既久,歲漕江、淮粟六百萬石,而縑帛、貨貝、齒革
 百物之委不可勝用。其後軍儲充溢,常有餘羨。內外乂安,非偶然也。



 凡上軍都校,自捧日、天武暨龍衛、神衛左右廂都指揮使遙領團練使者,月俸錢百千,粟五十斛;諸班直都虞候、諸軍都指揮使遙領刺史者半之。自餘諸班直將校,自三十千至二千,凡十二等;諸軍將校,自三十千至三百,凡二十三等,上者有傔;廂軍將校,自十五千至三百五十,凡十七等,有食鹽;諸班直自五千至七百,諸軍自一千至三百,凡五等;廂兵閱教者,有月俸
 錢五百至三百,凡三等,下者給醬菜錢或食鹽而已。自班直而下,將士月給糧,率稱是為差;春冬賜衣有絹綿,或加紬布、緡錢。凡軍士邊外,率分口券,或折月糧,或從別給。其支軍食,糧料院先進樣,三司定倉敖界分,而以年月次之。國初,諸倉分給諸營,營在國城西,給糧於城東,南北亦然。相距有四十里者,蓋恐士卒習墮,使知負簷之勤。久之,有司乃取受輸年月界分,以軍次高下給之。



 凡三歲大祀,有賜賚,有優賜。每歲寒食、端午、冬至,有
 特支,特支有大小差,亦有非時給者。邊戍季加給銀、□奚鞋,邠、寧、環、慶緣邊難於爨汲者,兩月一給薪水錢,苦寒或賜絮襦褲。役兵勞苦,季給錢。戍嶺南者,增月奉。自川、廣戍還者,別與裝錢。川、廣遞鋪卒或給時服、錢、履。屯兵州軍,官賜錢宴犒將校,謂之旬設,舊止待屯泊禁軍,其後及於本城。



 天聖七年,法寺裁定諸軍衣裝,騎兵春冬衣各七事,步兵春衣七事、冬衣六事,敢質賣者重寘之法。



 景祐元年,三司使程琳上疏,論:「兵在精不在眾。河北、陜
 西軍儲數匱,而召募不已,且住營一兵之費,可給屯駐三兵,昔養萬兵者今三萬兵矣。河北歲費芻糧千二十萬,其賦入支十之三;陜西歲費千五百萬,其賦入支十之五。自餘悉仰給京師。自咸平逮今,凡二邊所增馬步軍指揮百六十。計騎兵一指揮所給,歲約費緡錢四萬三千,步兵所給,歲約費緡錢三萬二千,他給賜不預。合新舊兵所費,不啻千萬緡。天地生財有限,而用無紀極,此國用所以日屈也。今同、華沿河州軍,積粟至於紅腐
 而不知用;沿邊入中粟,價常踴貴而未嘗足。誠願罷河北、陜西募住營兵,勿復增置,遇闕即遷廂軍精銳者補之,仍漸徙營內郡,以便糧餉。無事時番戍於邊,緩急即調發便近。嚴戒封疆之臣,毋得侵軼生事以覬恩賞,違令者重寘之法。如此,則疆場無事,而國用有餘矣。」帝嘉納之。



 康定元年,詔戰場士卒給奉終其身。宰臣張士遜等言禁兵久戍邊。其家在京師者,或不能自給。帝召內侍即殿隅條軍校而下為數等,特出內藏庫緡錢十萬賜
 之。



 慶歷五年,詔:「湖南路發卒征蠻,以給裝錢者,毋得更予帶甲錢。」



 七年,帝因閱軍糧,諭倉官曰:「自今後當足數給之。」初,有司以糧漕自江、淮,積年而後支,惟上軍所給鬥升僅足,中、下軍率十得八九而已。



 嘉祐八年,殿前諸班請糧,比進樣異,輒不受散去。御史中丞王疇以為言。詔:「提點倉官自今往檢視,有不如樣,同坐之。軍士不時請及有喧嘩,悉從軍法。」



 皇祐二年,詔:「在外禁軍,凡郊賚折色,並給以實估之直。」



 五年,詔:「廣南捕蠻諸軍歲滿歸
 營,人賜錢二千,月增奉錢二百。度嶺陣亡及瘴癘物故者子孫或弟侄,不以等樣收一人隸本營者,支衣廩之半。」



 治平二年,詔:「涇原勇毅軍揀為三等,差給奉錢一千至五百為三等,勿復置營,以季集渭州按閱。」



 熙寧三年,帝手詔:「倉使給軍糧,例有虧減,出軍之家,侵牟益甚,豈朕所以愛養將士意哉!自今給糧毋損其數,三司具為令。」於是嚴河倉乞取減刻之事。



 四年,詔付趙離:「聞鄜延路諸軍數出,至鬻衣裝以自給,可密體量振恤之。」先是,
 王安石言:「今士卒極窘,至有衣紙而擐甲者,此最為大憂,而自來將帥不敢言振恤士卒,恐成姑息,以致兵驕。臣愚以為親士卒如愛子,故可與之俱死;愛而不能令,譬如驕子不可用也。前陛下言郭進事,臣案《進傳》,言進知人疾苦,所至人為立碑紀德;士卒小有違令,輒殺之。惟其能犒賞存恤,然後能殺違令者而人無怨。今宜稍寬牽拘將帥之法,使得用封樁錢物隨宜振恤,然後可以責將帥得士卒死力也。」



 四年,樞密院言:「不教閱廂軍
 撥並,各帶舊請外,今後招到者,並乞依本指揮新定請受。河北崇勝、河東雄猛、陜西保寧、京東奉化、京西壯武、淮南寧淮各醬菜錢一百,月糧二石,春衣絹二匹、布半匹、錢一千,冬衣絹二匹、紬半匹、錢一千、綿十二兩。兩浙崇節、江東西效勇、荊南北宣節、福建保節、廣東西清化除醬菜錢不支外,餘如六路。川四路克寧已上各小鐵錢一千,糧二石,春衣絹一匹、小鐵錢十千,冬衣絹一匹、紬一匹、綿八兩、小鐵錢五千。」並從之。



 七年,增橋道、清塞、
 雄勝諸軍奉滿三百。又詔:「今後募禁軍等賞給,並減舊兵之半。」



 十年,詔:「安南道死戰沒者,所假衣奉咸蠲除之。弓箭手、民兵、義勇等有貸於官者,展償限一年。又中外禁軍有定額,而三司及諸路歲給諸軍亦有常數。其闕額未補者,會其歲給並封樁,樞密承旨司簿其餘數,輒移用,論如法。」



 元豐二年,詔:「荊南雄略軍十二營南戍,瘴沒者眾,其議優恤之。軍校子孫降授職。有疾及不願為兵若無子孫者,加賜緡錢;軍士子孫弟侄收為兵,並給賻,
 除籍後仍給糧兩月;即父母年七十已上無子孫者,給衣廩之半,終其身。」



 哲宗即位,悉依舊制。



 徽宗崇寧四年,詔:「諸軍料錢不多,比聞支當十錢,恐難分用,自今可止給小平錢。」初,蔡京謀逐王恩,計不行,欲陰結環衛及諸士卒,乃奏皇城輔兵月給食錢五百者,日給一百五十。自是,每月頓增四貫五百。欲因以市私恩也。



 五年,樞密院言:「自熙寧以來,封樁隸樞密院,比因創招廣勇、崇捷、崇武十萬人,權撥封樁入尚書省。緣禁軍見闕數多,
 若專責戶部及轉運司應副,恐致誤事。」詔:「尚書省候極足十萬人外,理合撥還。自今應禁軍闕額封樁錢,仍隸樞密院。」



 宣和七年,詔:「國家養兵,衣食豐足。近歲以來,官不守法,侵奪者多;若軍司乞取及因事率斂,刻剝分數,反致不足。又官吏冗占猥多,修造役使,違法差借。雜役之兵,食浮於禁旅,假借之卒,役重於廂軍。近因整緝軍政,深駭聽聞。自今違戾如前者,重寘之法。」



 靖康元年,詔:「諸路州軍二稅課利,先行樁辦軍兵合支每月糧料、春
 衣、冬賜數足,方許別行支散官吏請給等。禁軍月糧,並免坐倉。



 自國初以來,內則三司,外則漕臺,率以軍儲為急務,故錢糧支賜,歲有定數。至於征戍調發之特支,將士功勞之犒賞,與夫諸軍闕額而收其奉廩以為上供之封樁者,雖無定數,而未嘗無權衡於其間也。封樁累朝皆有之,而熙寧為盛。其後雖有「今後再不封樁」之詔,然軍司告乏,則暫從其請,稍或優足,則封樁如舊。蓋宰執得人,則闕額用於朝遷;樞筦勢重,則闕額歸之密院。
 此政和軍政所以益不逮於崇寧、大觀之間者,由兩府之勢互有輕重,而不能恪守祖宗之法也。



 中興以後,多遵舊制。紹興四年,御前軍器所言:「萬全雜役額五百,戶部廩給有常法。比申明裁減,盡皆遁逃。若依部所定月米五斗五升,日不及二升;麥四斗八升,斗折錢二百,日餐錢百,實不足贍。」詔戶部裁定,月米一石七斗,增作一石九斗。



 五年,詔:「效用入資舊法,內公據、甲頭名稱未正,其改公據為守闕進勇副尉,日餐錢二百五十、米二升;
 甲頭為進勇副尉,日餐錢二百、米二升。非帶甲入隊人自依舊法。」宣撫使韓世忠言:「本軍調發,老幼隨行,緣效用內有不調月糧,不增給日請,軍兵米二升半、錢百,效用米二升、錢二百,乞日增給贍米一升半,庶幾戰士無家累後顧憂,齊心用命。」詔分屯日即陳請。



 十三年,詔:「殿司諸統領將官別無供給職田,日贍不足,差兵營運,浸壞軍政。可與月支供給:統制、副統制月一百五十千,統領官百千,正將、同正將五十千,副將四十千,準備將三
 十千,皆按月給。既足其家,可責後效。若仍前差兵負販,從私役禁軍法,所販物計贓坐之,必罰無赦。州縣知而不舉,同罪。」主管步軍司趙密言:「比定諸軍五等請給,招填闕額,要以屏革奸弊。第數內招收白身效用,填馬步軍使臣闕。其五等請給例內,馬軍效用依五人衙官例,步軍效用依三人衙官例。緣舊效用曾經帶甲出入,日止餐錢二百、米二升;有少壯善射者,既見初收效用廩給稍優,因逃他軍以希厚請。今擬五等招收白身效用
 與舊效用,不以馬步軍論,概增其給,人日支錢二百、米二升,填使臣闕。」



 隆興二年,殿前司言:「諸軍法,兵級年六十,將校年六十五,減充剩員給請,內有戰功亦止半給。比來年及不與減落,乞每營置籍,鄉貫、年甲、招刺日月悉書之,一留本營,一留戶部,一留總領,以備開落。」



 乾道八年,樞密院言:「二月為始,諸軍七人例以上,二分錢、三分銀、五分會子;五人例,三分錢、四分銀、三分會子。軍兵折麥餐錢,全支錢。使臣折麥、料錢,統制、軍佐供給分數
 仍舊。」



 淳熙三年,樞密院言:「兵部定請受格:效用一資守闕毅士,二資毅士,三資守闕效士,月各錢三千,折麥錢七百二十,米一石五升,春冬衣絹各二匹;四資效士,錢三千,折麥錢九百七十二,米一石一斗三升有奇,衣絹各二匹;五資守闕聽候使喚,錢四千五百,折麥錢一千八十,米一石二斗,絹三匹有半;六資聽候使喚,錢四千五百,折麥錢一千二百六十,米一石四斗七升,絹五匹;七資守闕聽候差使,八資聽候差使,錢四千五百,折麥
 錢一千四百四十,米一石六斗八升,絹各五匹;九資守闕準備使喚,十資準備差使,錢五千,折麥錢一千四百四十,米六石八升,絹各五匹。」



 紹熙元年,知常德府王銖言:「沿邊城砦之官,以備疆埸不虞,廩祿既薄,給不以時,孤寒小吏,何以養廉?致使熟視奸猾洩漏禁物,公私庇蓋,恬不加問,從而徇私受賕者有矣。弓手、士軍、戍卒傭直糧食,累月不支,迫於饑寒,侵漁蠻獠,小則致訟爭,大則啟邊釁。乞嚴敕州、軍按月廩給,如其未支,守倅即不
 得先請己奉。庶俾城砦官兵有以存濟,緩急之際,得其宣力。安邊弭盜,莫此為急。」



 厥後弊日以滋,迨至咸淳,軍將往往虛立員以冒稍食。以建康言之,有神策二軍,有游擊五軍,有親兵二軍,有制效二軍,有靖安、唐灣水軍,有游擊採石水軍,有精銳破敵軍,有效用、防江軍,原其初起,惟騎、戎兩司額耳。後仍各創軍分,額多而員少。一統制月請,以會子計之,則成一萬五百千,推之他軍,概可見矣。



 九年,四川制司有言:「戍兵生券,人月給會子六
 千,蜀郡物賈翔貴,請增人月給九千。」當是時財賦之出有限,廩稍之給無涯,浚民膏血,盡充邊費,金帛悉歸於二三大將之私帑,國用益竭,而宋亡矣。



 臣僚嘗言:「古者兵與農一,官無供億之煩,國有備御之責。後世兵與農二,竭國力以養兵,奉之若驕子,用之若傭人。今守邊急務,非兵農合一不可。其說者有二:曰屯田,曰民兵。川蜀屯田為先,民兵次之;淮、襄民兵為先,屯田次之。此足食足兵之良策也。」其言厄於權奸,竟不行。



\end{pinyinscope}