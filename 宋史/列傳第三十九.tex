\article{列傳第三十九}

\begin{pinyinscope}

 田紹斌王榮楊瓊錢守俊徐興王杲李重誨白守素張思鈞李琪王延範



 田紹斌,汾州人。仕河東劉鈞為佐聖軍使,戍遼州。周顯
 德四年,領五十騎來歸,鈞屠其父母家屬。世宗召補驍武副指揮使。



 宋初,隨崔彥進徵李筠,攻大會砦,破之,以功遷龍捷指揮使。又敗筠於澤州茶碾村,筠退保澤州,紹斌鑿濠圍守,流矢中左目,前軍部署韓令坤以其事聞。及太祖召見於潞州,紹斌殺晉軍益眾,奪其鎧甲。又從討李重進於揚州,壁城南,圍三日,城潰,斬首逾千級。賜袍帶、緡帛,尋補馬軍副都軍頭、龍衛指揮使。下荊湖,平嶺南,率皆從行。討蜀,隸大將劉延讓麾下。會全師雄
 寇神泉,紹斌率所部敗其黨數千,時漢、劍道梗,因賴以寧,太祖遣使孫晏繼詔賜SS有加。凡在蜀三歲,剽盜殄除。還,改龍捷都虞候。



 嘗盜官馬,貿直盡償博進,事發,獄具,有司引見講武殿,紹斌稱死罪。太祖知其驍勇,欲宥之,執於門外,遣內侍私謂之曰:「爾今死有餘責。」紹斌曰:「若恩貸臣死,當盡節以報。」俄復引見,釋之,且密賜白金。



 會征江南,擇諸軍借事得五百人,為步鬥軍,令紹斌領之,及率雲騎二千,抵升州城下,克獲居多。太祖親討河
 東,命紹斌從何繼筠扼契丹兵於北百井,奪賊鼓幟而還。



 太平興國初,擢龍衛軍指揮使、領江州刺史。二年,梅山洞蠻叛,命與翟守素分往擊之。至邵州,聞蠻酋苞漢陽死,去其居十里,大潰其眾,擒蠻二萬,令軍中取利劍二百斬之,餘五千遣歸諭諸洞,自是其黨帖服。太宗賜以金帛、緡錢、金帶、鞍馬。歷天武、日騎軍指揮使,改馬步都軍頭,出戍鎮、定、高陽關。



 曹彬之攻幽州也,命為先鋒指揮使,數遇契丹兵鬥,奪牛羊、器甲。師還,召見便殿,加領
 溪州團練使,復遣屯北面。端拱元年,拜冀州防禦使,尋改解州。



 淳化中,為河中、同、丹、坊、鄜、延、橫嶺蕃界都巡檢使。會鄭文寶議城席雞城砦為清遠軍,紹斌與文寶領其役。城畢,以文寶之請,命為知軍事。至道元年,拜會州觀察使,仍判解州,俄充靈州馬步軍部署。領徒入蕃討賊,斬首二千級,獲羊、馬、橐駝二萬計,馬以給諸軍之闕者。捷聞,手詔嘉諭之。數部金粟帛詣靈武、清遠,遠人讋服不擾。



 未幾,皇甫繼明、白守榮等督轉餉於靈州,紹斌
 率兵援接,抵咸井。賊逾三千餘來薄陣,且行且鬥,至耀德,凡殺千人。寇復尾後,紹斌為方陣,使被傷者居中,自將騎三百、步弩三百,與敵兵確於浦洛河,大敗之。



 初,守榮與紹斌為期,既而繼明卒,故後一日,遂為賊所圍。守榮等欲擊之,紹斌曰:「蕃戎輕佻,勿棄輜重與戰,當按轡結陣徐行。」守榮等忿曰:「若但率兵來迎,勿預吾事。」紹斌因率所部去輜重四五里。繼遷初見紹斌旌旗,不敢擊。守榮等自欲邀功,與戰。賊先伏兵,以羸騎挑戰,已而伏
 發,守榮等戰敗,丁夫愕眙遁,蹂踐至死者眾。紹斌率所部徐還,一無遺失。至清遠,與張延州會食。見濠中人裸而呼曰:「我白守榮也。」繩引而上,解衣遺之,遣內侍馬從順驛聞。太宗益嘉之,優詔褒美。



 時命李繼隆、範廷召討繼遷,就命紹斌為本州都部署兼內外都巡檢使。繼隆以浦洛之敗上聞,言紹斌握兵不顧,自言「靈武非我不能守」,欲圖方面,有異志。太宗怒曰:「此昔嘗背太原來投,今又首鼠兩端,真賊臣也。」即遣使捕系詔獄鞫問,貶右
 監門衛率府副率、虢州安置。



 真宗即位,召還,授右監門衛大將軍、領敘州刺史,尋改萊州防禦使,詔還其所籍居第,賜良馬十匹。調環、慶、靈州、清遠軍部署。慶州有野雞族,數為寇掠,道路患之。嘗有驍捷卒二十餘往邠州,為其掠奪,即馳告紹斌。紹斌召其酋帥三人,斷臂、馘、劓放還,寇感而化,帖服。紹斌素勇悍,與同職頗不葉。轉運使宋太初每按部靈州、清遠,多貿市,紹斌語發其私,太初心銜之,及還朝,言紹斌之過,尋赴召,直其事。



 咸平二
 年,北面寇警,復命為鎮、定、高陽關路押先鋒,隸傅潛。潛遣與石普並戍保州,普陰與知州楊嗣議出兵擊討之。及夜,普、嗣未還,紹斌疑其敗衄,即領兵援之。普、嗣果為賊所困,度嚴涼河,頗喪師眾。及紹斌至,即合兵疾戰,獲一百四十餘人,以勞遷邢州觀察使。潛屯中山,紹斌三馳書於潛,且言:「邊眾大至,但列兵唐河南,背城與戰,慎無窮追。」潛性巽懦,聞之益不敢出,賊眾益熾,焚劫城砦。車駕駐大名,召潛屬吏,詞逮紹斌,即遣使械系,下御史
 臺鞫問,免官,黜為左衛率府副率,送往上都,禁其出入。五年,授右千牛衛將軍致仕。



 景德初,起為左龍武軍將軍、永城兵馬都監。三年,遷左監門衛大將軍。帝以紹斌久失職,不宜在沖要,乃徙考城都監。大中祥符初,領長州刺史。從東封,朝覲壇就班,軍士建充庭旗,旗倒,壓紹斌僕地,遽起無傷。時紹斌已老,其壯健若此。遷左領軍衛大將軍、領康州團練使、鞏縣都監。二年,卒,年七十七。



 紹斌長兵間,習戰法,其後累以格鬥立功,然性暴戾,故
 屢被黜。子守信,為內殿崇班、閣門祗候。



 王榮,定州人。父洪嗣,仕晉為本州十縣游奕使。榮少有膂力,事瀛州馬仁瑀為廝役。太宗在藩邸,得隸左右。即位,補殿前指揮使,稍遷本班都知、員僚直都虞候。盜發棣州,州兵不能捕,榮往討擒之。加御前忠佐馬步軍都軍頭、領懿州刺史。坐受秦王廷美宴勞,出為濮州馬軍教練使。未行,馬仁瑀子告榮與秦王親吏善,因狂言「我不久當得節帥」,坐削籍,流海島。



 雍熙中召還,為副軍頭。
 端拱初,改員寮左右直都虞候兼都軍頭,復領懿州刺史。累遷龍衛都指揮使、領羅州團練使。率兵戍遂城,邊騎來寇,擊敗之,擒千餘人。召拜侍衛馬軍都虞候、峰州觀察使,出為定州行營都部署。榮粗率,所為不中理,侵取官地蒔蔬,吝惜公錢,不以勞將士,且母老不迎養,供給甚薄。太宗聞而怒曰:「忠臣出於孝子之門,榮事親若此,竄逐之餘,兇行弗悛,豈可復置左右,效晉帝養成張彥澤邪?」即詔罷,督責,授右驍衛大將軍。寄班供奉官張
 明護定州兵,睹榮不法,間嘗規正。榮護短,每疾其攻己。莊宅使王斌亦監軍是州,素與榮善,意明構榮之罪,因摭明以報怨。下樞密院問狀,皆不實。上怒,語左右曰:「張明起賤微中,以蹴鞠事朕,潔己小心,見於輩流。夫刑罰之加,必當其罪。今王斌以榮故而曲奏明罪,欲致刑憲,茍失其當,適足以快榮之心,而誣罔得以肆行矣。且榮凌轢同類,事君與親鮮竭其力。國家賞罰之柄,非所敢私,將帥之職,非裨校同。朕豈黨張明而棄王榮哉,奈何
 不求直於理之當也?」遂賜勞明緡錢、束帛,榮遷右羽林軍大將軍。



 真宗即位,領獎州刺史,尋授濱州防禦使,遷涇原儀渭駐泊部署。咸平二年,車駕北征,召為貝、冀行營副都部署。師旋,復還涇原。明年,援送靈武芻糧,疏於智略,不嚴斥候,至積石,夜為蕃寇所劫,營部大亂,眾亡殆盡。法當誅,恕死,除名配均州。六年,起為左衛將軍。



 景德初,權判左金吾街仗司事。上觀兵澶淵,契丹游騎涉河冰抵濮州境,命為黃河南岸都巡檢使,與鄭懷德自
 行在領龍衛兵追襲。時已詔滄州部署荊嗣先率所部屯淄、青,遣榮等合兵邀擊之。二年,遷左神武軍大將軍、領恩州刺史。郊祀,改左龍武軍、領達州團練使。大中祥符中,遷左衛大將軍、領昌州防禦使。六年,朝太清宮,命為河南府駐泊都監。九年,卒,年七十。官其一子。榮善射,嘗引強注屋棟,矢入木數寸,時人目為「王硬弓」。



 楊瓊,汾州西河人。幼事馮繼業,以材勇稱。太宗召置帳下。即位,隸御龍直,三遷神勇指揮使。從征太原,以勞補
 御龍直指揮使。雍熙初,改弩直都虞候兼御前忠佐馬步都軍頭、領顯州刺史。



 淳化中,李順叛蜀,瓊往夔、峽擒賊招安,領兵自峽上,與賊遇,累戰抵渝、合,與尹元、裴莊分路進討,克資、普二州、雲安軍,斬首數千級。詔書嘉獎,遣使即軍中真拜單州刺史。



 至道初,召還共職。明年,徙知霸州兼鈐轄。未幾,改防禦使,靈慶路副都部署、河外都巡檢使。賊累寇疆,瓊固捍有功。導黃河,溉民田數千頃。敗賊於合河鎮北,擒獲人畜居多。賊騎五百掠城下,
 擊破之,追北三十里。並賜詔嘉諭。



 咸平二年,命為涇原儀渭邠寧環慶清遠軍靈州路副都部署。尋徙鎮、定、高陽關三路押策先鋒,屯定州之北。明年,副王超為鎮州都部署,再遷環慶,徙定州。四年,召還,以鄜州觀察使充靈、環十州軍副都部署兼安撫副使。嘗遣使諭旨,賊若寇清遠及青岡、白馬砦,即合兵與戰。是秋,果長圍清遠,頓積石河。清遠屢走間使詣瓊請濟師,瓊將悉出兵為援,鈐轄內園使馮守規、都監崇儀使張繼能曰:「敵近,重
 兵在前,繼無以進,不可悉往。」乃止命副部署海州團練使潘璘、都監西京左藏庫劉文質率兵六千赴之,且曰:「伺我之繼至。」瓊逗遛不進,頓慶州。寇鼓兵攻南門,其子阿移攻北門,堙壕斷橋以戰。瓊遣鈐轄李讓督精卒六百往援,至則城陷矣。賊泊青岡城下,瓊與守規、繼能方緩行出師,及聞清遠之敗,益恇怯不前。順州刺史王瑰普謂瓊曰:「青岡地遠水泉,非屯師計,願棄之。」瓊合謀焚芻糧兵仗,驅老幼以出。瓊卻師,退保洪德砦,寇威浸
 熾,未嘗交一鋒。事聞於上,傳召瓊輩,悉系御史獄,治罪當死。兵部尚書張齊賢等議請如律,詔特貸命,削官,長流崖州,繼能、守規輩同坐,籍其家業。明年,移道州。



 景德初,起為右領軍衛將軍。分司西京。累遷左領軍衛大將軍、領賀州團練使、知兗州。有州卒自言得神術,能飛行空中,州人頗惑。瓊捕至,折其足,奏戮之。五年,卒,年六十七。錄其子舜臣為奉職。長子舜賓,內殿崇班、閣門祗候。



 錢守俊,濮州雷澤人。少勇鷙,嘗為盜陂澤中,稱「轉陂鶻」。
 周顯德中,應募為鐵騎卒。早事太祖,從征淮南,戰紫金山,下壽春,獲戰艦千餘艘。繼從克關南。



 宋初,補禁衛,隸散員直。乾德中,轉殿前班都知。尋徵太原,方戰,矢中左足,拔而復進,格鬥不已。還,改東西班指揮使,遷馬步軍副都軍頭。太平興國四年,命與張紹勍、李神祐、劉承珪率師屯定州,以備北邊。俄加秩領演州刺史,移屯趙州。又從征範陽,師還,道遇敵,戰於徐河,斬首千級,奪馬百匹。雍熙三年,命將北征,田重進出飛狐道,守俊以偏師
 為援,邊騎雲集,守俊按甲從容進戰,大敗之。連護屯兵於趙、定。代還,掌軍頭引見司。



 淳化三年,出為單州團練使。又明年,改遷齊州。時河西蕃部內擾,命以副都部署鎮其地。既而徙屯石州,數改官。時有言守俊病且老,握重兵不堪其職。召還,授左領軍衛大將軍、領潘州防禦使、權金吾街仗。大中祥符三年卒,年八十一。



 守俊累從軍征討,前後中三十六創。景德中,錄其子允慶為奉職。弟守信,官崇儀副使;守榮,內園使。



 徐興,青州人。以拳勇得隸兵籍。周顯德中,從太祖征淮右。宋初,隸御龍直。會平澤、潞,上其功,補控鶴軍使。征晉陽,部卒壅汾水灌並州城,益多其勞。還,遷本軍副指揮使。



 太平興國初,從潘美趣團柏谷,奮與賊鬥,有果敢氣,人莫能勝。生擒偽兵馬都監李美,身被重創,無所回撓。加指揮使。太宗征太原,討幽、薊,興從戰,屢中流矢,以著跡聞。補天武都虞候,累遷秩,出為洺州部署。初議建方田,命興董其事,尋復輟。端拱中,修鎮、定城,逾月訖工。改
 莫州防禦使、知靜戎軍,歷祁、博二州。



 咸平中,為涇、原、環、慶十州部署。詔督轉靈武芻糧,道積石,率掠於寇。興以步兵畏恧,戰不利,時王榮援兵不應,遂敗走。坐削籍,流郢州。會赦,入為右衛將軍,遷左監門衛大將軍。景德二年卒,年六十八。



 王杲,齊州人。周顯德中,應募為卒。從世宗收三關,隸先鋒。宋初,徵澤、潞,平揚州,杲應選從行,既獲戰功,乃拔遷散指揮使,累轉馬軍副都軍頭,屯並州。雍熙中,為龍衛
 右第二軍都虞候。會遣趙保忠還夏州,命杲引兵護送。及還,保忠以方物贐,杲拒不納。太宗知之,詔賜白金百兩。遷右第一軍,屯鎮州。



 契丹入寇,隸大將郭守文,捍城,杲守北關。寇退,命督餉蒿趣威虜軍。還抵徐河,時尹繼倫與寇戰,小衄,杲適遇賊河上,即按兵拒之,殺賊,奪所乘馬。守文上聞,得召見問狀,補都軍頭、領勤州刺史,命監河北,有能聲,尋命閱教定州諸軍騎射,入掌軍頭引見司。



 李順亂,與尹元並為西川招安使,敗賊,斬首萬
 級,以功真拜唐州刺史。時賊雖平,道路尚梗,餘黨或保山林以肆奸,杲與石普等追捕於彭州,於是始平。至道初,乃還。復遷靈州副部署,道環州,留改並州,徙知夏州。會趙保吉歸款,召還,次伏落津,移知石州,徙石、隰副部署。未幾,以轉餉河西失期,降右千牛衛大將軍。咸平五年,出為亳州永城縣都監。被召,將入見,以疾亟弗果,卒,年六十四。



 李重誨,應州金城人。祖高,後唐莊宅使、獎州刺史。父彥
 榮,仕契丹,署環州刺史,重誨嘗為其應州馬步軍都指揮使。太平興國五年,潘美出師禦寇,重誨從其節度使蕭咄李迎戰於代州北嶺,大敗。美斬咄李,擒重誨以獻。太宗召見,補鄧州馬步軍都指揮使。會趙普出鎮,奏監州軍。



 雍熙三年,召還,為武州刺史,出為忻州都巡檢、緣邊十八砦招安制置使,賜服帶、鞍馬。北兵寇邊,重誨以所部邀戰,敗之,獲羊馬、鎧甲甚眾,賜詔嘉美。會嶺蠻叛,改廣、桂、融、宜、柳州招安捉賊使,聽便宜從事。



 至道初,累
 遷涇原儀渭鎮戎軍鈐轄。咸平三年,徙邠寧環慶路。坐轉餉靈武不嚴斥候,至積石為虜騎掠於道,營部大亂,除名,流光州。五年,起為內殿崇班、鄜延駐泊都監,俄遷崇儀使。景德中,趙德明既納款,或言以麟、府謀有他志。上以涇原地要兵眾,慮有緩急,遂徙重誨為鈐轄。復遷益州,改皇城使。大中祥符六年,卒,年六十八。



 重誨純愨寡過。真宗悼其沒於遠土,命其子乘傳往護柩歸,聽止驛舍之別次。子禹謨,錄為將作監主簿。弟重睿,歷官澄
 州團練使。子禹偁,閣門祗候。



 白守素,開封人。祖延遇,仕周至鎮國軍節度。父廷訓,宋初為龍捷都指揮使、領博州刺史。守素以蔭補東班承旨。太平興國五年,遷補右班殿直,以善射,授供奉官、帶御器械,三遷至供備庫使。



 咸平三年春,契丹犯邊,命與王能戍邢州,俄又與麥守恩、石贊領先鋒御之。敵退,復與荊嗣督河北、京東捕賊。四年,命為鎮州行營鈐轄,領騎兵攝大陣西偏,屢當格鬥。俄改定州鈐轄,復徙鎮州。
 王繼忠之陷也,宋師還渡河,敵人乘之,守素據橋,有矢數百,每發必中,敵不敢近,遂引去。



 真宗與輔臣議三路御賊,咸曰:「威虜扼北道要害尤甚,請分騎兵六千屯之,命魏能為部署。」上曰:「能頗強愎,尤難共事,聞守素久練邊計,張銳性頗和善,參知戎務,庶克相濟。」乃命守素、銳為鈐轄,戍順安以貳之。



 景德元年,契丹侵長城口,守素與能發兵破之,追北過陽山,斬首級、獲器械甚眾,賜錦袍、金帶。俄徙屯冀州,轉運使劉綜舉其智勇,材任將帥,
 加領康州刺史。又提騎卒戍靜戎軍,兼蒞營田之役,俄為鎮、定鈐轄。是冬,契丹復內侵,守素敗其前鋒,獲車重,又入敵境,俘擒甚從。及請和,省邊戍之職,與曹璨留任鎮、定。追敘前勞,加合州團練使。



 大中祥符三年,命副李迪使契丹。守素居邊歲久,名聞北庭,頗畏伏之。上慮其不欲行,密遣內侍詢於守素,守素頓首感咽,即以崔可道代焉。再遷南作坊使。大中祥符五年,卒。上甚惜之,常賻外別SS錢五十萬,令護喪還京師,錄其一子官。



 張思鈞,邢州沙河人。祖中正,漢澤州刺史。思鈞少善擊劍、挽強,善博奕。初應募為卒,晉開運間,遷廣銳軍使。周廣順初,從聶知遇攻河東,破其眾三千餘。從向訓東征,為捉生將,擒小校張萬於江豬嶺。又從符彥卿與並人鬥代州,留為南北兩關巡檢。



 宋初,補龍衛指揮使。李繼勛下遼州,戰帶甲祠,斬首萬餘級,追奔至長城,擒其將莫山、鮑淑,掠人騎二百餘。俄屯潞州,合戰三十餘。乾德中,以勞秩遷都虞候。開寶三年,郭進、田欽祚戍三交,嘗
 從戰於石嶺關,斬首萬五千餘級。閣門祗候齊延琛、苗昶陷軍中,思鈞鼓勁騎突入,奪還。何繼筠入晉境,思鈞隸麾下,拔南橋徑度。大將之出,必闢為先鋒。太平興國初,屯定州,領兵援磁窯,戰敗其眾,身中五十創,奮不顧,乃逐賊,薄軍城,奪馬及鎧甲居多。未幾,邊人復攻,逆戰城下,斬首萬餘級。上嘉之,命賜服帶,領河州刺史。



 雍熙三年,邊人寇河間,劉廷讓會戰君子館,命思鈞翼從。時天大寒,弓不得彀,援兵不至,於是敗績,陷留軍中數年,
 役役不得還。端拱初,自契丹始逃歸,授澄州刺史、知齊州。思鈞以武進,素不知民政,僅逾月,即徙濮、鄆、濱、棣州巡檢。至道中,改鄜延巡檢使。會葺右堡砦,擊寇走之。未幾,寇逼保安軍,與曹璨往援,追躡五十餘里,至木場,寇乃遁去。



 真宗即位,徙益州鈐轄兼綿、漢九州都巡檢使。咸平中,以王均之亂,出兵保綿州。賊陷漢州,思鈞進攻,克之,斬偽刺史苗進,又與石普敗賊彌牟砦。巴西尉傅翱有善馬,思鈞求之,翱不與。思鈞平賊,心恃功居多,召
 翱至,責以轉餉後期,斬之。上聞其事,傳召付御史臺鞫治,罪當斬,特貸之,削籍流封州。



 六年,起為左司禦率府率、考城監軍。車駕幸澶淵,召詣行在,命李繼隆、石保吉同議兵事,賜服御有加。景德二年,為西京水北都巡檢使,俄分司西京。召對行在,上憫其老,授唐州防禦副使,徙鄭州。大中祥待二年,再遷左千牛衛將軍。四年七月,卒,年八十九。子承恩,為三班奉職。



 思鈞起行伍,征討稍有功。質狀小而精悍,太宗嘗稱其「樓羅」,自是人目為「小
 樓羅」焉。



 李琪,河南伊闕人。幼生長兵家,得給事宣祖,左右太祖,以材力稱,進備執御。及受禪,命補鎮職。太宗在京府,復令事之。由是累遷效忠都虞候、開封府馬步軍副都指揮使、領富州刺史。嘗請對,自言經事太祖,而京師無居宅,太宗以官第假之。



 琪性素鄙,歷事三朝,而行不加修。每分遣士卒守護關梁,必覬其贈遺,視所厚薄為重輕。太宗知之,遂改授屯衛大將軍,領郡如故,乃顧曰:「吾欲
 置琪於無過之地爾。」加左武衛大將軍。景德中,以老且病,表求五日一赴起居,俄為臺諫所糾,令赴常參。真宗念其舊,特賜給月奉以養。大中祥符元年卒,年八十四。



 王延範,江陵人。形貌奇偉,喜任俠,家富於財。父保義,為荊南高氏行軍司馬兼領武泰軍留後。高從誨奏署延範太子舍人。後隨從誨孫繼沖入覲,薦為大理寺丞、知泰州累遷司門員外郎。



 太平興國九年,為廣南轉運使。性豪率尚氣,尤好術數。嘗通判梓州,有杜先生以左道
 惑眾,謂延範曰:「汝意有所之,我常陰為之助。」延範心喜,敢為恣橫。後為江南轉運使,有劉昴賣卜於吉州市,其言多驗,謂延範曰:「公當偏霸一方。」又有徐肇為延範推九宮算法,得八少一,肇驚起曰:「君侯大貴不可言,當如江南李國主。」前戎城主簿田辨自言善相,謂延範曰:「君是坐天王形、頻伽眼、仙人鼻、雌龍耳、虎望,有大威德,猛烈富貴之相也。即日當乘四門輦。」至是,有豹入其公宇,噬傷數吏,從者皆恐心慄,不敢進,延範獨拔戟前逐,刺殺
 之,益以此自負。與廣州掌務殿直趙延貴、將作監丞雷說會宿,觀天象,延貴指西方一大星曰:「此所謂『火星入南斗,天子下殿走』者也。



 雷說出《星經》證之,乃太白行度經南斗,延貴謬為火星也。



 延範日夕與掌市舶陸坦議欲發兵,會坦代歸,延範寓書左拾遣韋務升為隱語,偵朝廷機事。延範奴視僚屬,峻刑多怨。會懷勇小將張霸給使轉運司,延範因事杖之,霸知延範與知廣州徐休復不協,詣休復,告延範將謀不軌及諸不法事。休復馳
 奏之。太宗遣高品閻承翰乘傳,會轉運副使李管暨休復雜治延範,具伏。與昴、辨、坦俱斬廣州市,籍沒延範家。務升除名配商州,延貴等皆抵罪,賜霸錢十萬。



 論曰:紹斌從征討,凡逾百戰,未嘗以為憚;屢被廢斥,未嘗以為慊。太祖宥盜馬罪,引見賜予,屈法使過,用能致其力也。榮薄事親,下詔督過。瓊折州卒足以釋妖惑。王杲辭贐於夏。思鈞拔身自歸,當斬而貸。琪以鄙稱。守俊、興輩以勇得備給使。守素久練邊計,人頗畏伏。重誨雖
 將略不足,亦有可稱。大抵武夫悍卒,不能無過,而亦各有所長。略其過而用其長,皆足以集事。至於一勝一負,兵家常勢,顧其大節何如耳。若榮也,薄其所生,大節虧矣,屢以罪黜,宜哉



\end{pinyinscope}