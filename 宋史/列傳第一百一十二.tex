\article{列傳第一百一十二}

\begin{pinyinscope}

 何慄孫傅陳過庭張叔夜聶昌張閣張近鄭僅宇文昌齡子常許幾程之邵龔原崔公度蒲卣



 何
 慄,字文赬,仙井人。政和五年進士第一,擢秘書省校書郎。逾年,提舉京畿學事,召為主客員外郎、起居舍人,遷中書舍人兼侍講。



 徽宗數從咨訪,欲付以言責。或論慄與蘇軾鄉黨,宗其曲學,出知遂寧府。已而留為御史中丞,論王黼奸邪專橫十五罪,黼既抗章請去,而猶豫未決。慄繼上七章,黼及其黨胡松年、胡益等皆罷,慄亦以徽猷閣待制知泰州。



 欽宗立,復以中丞召。閱月,為翰林學士,進尚書右丞、中書侍郎。會王云使金帥斡離不
 軍還,言金人怒割三鎮緩,卻禮幣弗納曰,兼旬使不至,則再舉兵。於是百官議從其請。慄曰:「三鎮,國之根本,奈何一旦棄之。況金人變詐罔測,安能保必信?割亦來,不割亦來。」宰相主割議,慄論辨不已,曰:「河北之民,皆吾赤子。棄地則並其民棄之,豈為父母意哉?」帝頗悟。慄請建四道總管,使統兵入援,以胡直孺、王襄、趙野、張叔夜領之。兵既響應,而唐恪、耿南仲、聶昌信和議,相與謀曰:「方繼好息民而調發不已,使金人聞之,奈何?」亟檄止之。



 慄
 解政事,俄以資政殿大學士領開封尹。金兵長驅傅城下,帝罷恪相,而拜慄為尚書右僕射兼中書侍郎,始復三省舊制。時康王在河北,信使不通,慄建議請以為元帥,密草詔稿上之。乃以康王充天下兵馬大元帥,陳遘充兵馬元帥,宗澤、汪伯彥充副元帥。京城失守,從幸金帥營,遂留不返。既而議立異姓,金人曰:「唯何慄、李若水毋得預議。」既陷朔庭,慄仰天大慟,不食而死,年三十九。



 建炎初,詔以為觀文殿大學士、提舉玉局觀使,祿其家。
 訃聞,贈開府儀同三司,議者指其誤國,不行。秦檜自北還,具道其死時狀,乃改贈大學士,官其家七人。



 孫傅,字伯野,海州人。登進士第,中詞學兼茂科,為秘書省正字、校書郎、監察御史、禮部員外郎。時蔡翛為尚書,傅為言天下事,勸其亟有所建,不然必敗。翛不能用。遷秘書少監,至中書舍人。



 宣和末,高麗入貢,使者所過,調夫治舟,騷然煩費。傅言:「索民力以妨農功,而於中國無絲毫之益。」宰相謂其所論同蘇軾,奏貶蘄州安置。給事
 中許翰以為傅論議雖偶與軾合,意亦亡他,以職論事而責之過矣,翰亦罷去。靖康元年,召為給事中,進兵部尚書。上章乞復祖宗法度,欽宗問之,傅曰:「祖宗法惠民,熙、豐法惠國,崇、觀法惠奸。」時謂名言。十一月,拜尚書右丞,俄改同知樞密院。



 金人圍都城,傅日夜親當矢石。讀丘浚《感事詩》,有「郭京楊適劉無忌」之語,於市人中訪得無忌,龍衛兵中得京。好事者言京能施六甲法,可以生擒二將而掃蕩無餘,其法用七千七百七十七人。朝廷
 深信不疑,命以官,賜金帛數萬,使自募兵,無問技藝能否,但擇其年命合六甲者。所得皆市井游惰,旬日而足。有武臣欲為偏裨,京不許,曰:「君雖材勇,然明年正月當死,恐為吾累。」其誕妄類此。敵攻益急,京談笑自如,云:「擇日出兵三百,可致太平,直襲擊至陰山乃止。」傅與何慄尤尊信,傾心待之。或上書見傅曰:「自古未聞以此成功者。正或聽之,姑少信以兵,俟有尺寸功,乃稍進任。今委之太過,懼必為國家羞。」傅怒曰:「京殆為時而生,敵中瑣
 微無不知者。幸君與傅言,若告他人,將坐沮師之罪。」揖使出。又有稱「六丁力士」、「天關大將」、「北斗神兵」者,大率皆效京所為,識者危之。京曰:「非至危急,吾師不出。」慄數趣之,徙期再三,乃啟宣化門出,戒守陴者悉下城,無得竊覘。京與張叔夜坐城樓上。金兵分四翼噪而前,京兵敗退,墮於護龍河,填尸皆滿,城門急閉。京遽白叔夜曰:「須自下作法。」因下城,引餘眾南遁。是日,金人遂登城。



 二年正月,欽宗詣金帥營,以傅輔太子留守,仍兼少傅。帝兼
 旬不返,傅屢貽書請之。及廢立檄至,傅大慟曰:「吾惟知吾君可帝中國爾,茍立異姓,吾當死之。」金人來索太上、帝後、諸王、妃主,傅留太子不遣。密謀匿之民間,別求狀類宦者二人殺之,並斬十數死囚,持首送之,紿金人曰:「宦者欲竊太子出,都人爭鬥殺之,誤傷太子。因帥兵討定,斬其為亂者以獻。茍不已,則以死繼之。」越五日,無肯承其事者。傅曰:「吾為太子傅,當同生死。金人雖不吾索,吾當與之俱行,求見二酋面責之,庶或萬一可濟。」傅寓
 直皇城司,其子來省,叱之曰:「使汝勿來,而竟來邪!吾已分死國,雖汝百輩來何益!」揮使速去。子亦泣曰:「大人以身徇國,兒尚何言。」遂以留守事付王時雍而從太子出。至南熏門,範TF力止之,金守門者曰:「所欲得太子,留守何預?」傅曰:「我宋之大臣,且太子傅也,當死從。」是夕,宿門下,明日,金人召之去。明年二月,死於朔廷。紹興中,贈開府儀同三司,謚曰忠定。



 陳過庭,字賓王,越州山陰人。中進士第,為館陶主簿、澶
 州教授、知中牟縣,除國子博士。何執中、侯蒙器其才,薦之,擢祠部、吏部、右司員外郎。使契丹,過庭初名揚庭,辭日,徽宗改賜今名。時人或傳契丹主苦風痺,又箭損一目,過庭歸證其妄,且勸帝以邊備為念。適太常少卿、起居舍人。宣和二年,進中書舍人;才七日,遷禮部侍郎;未盡一月,又遷御史中丞兼侍讀。睦寇竊發,過庭言:「致寇者蔡京,養寇者王黼,竄二人,則寇自平。又朱勉父子,本刑餘小人,交結權近,竊取名器,罪惡盈積,宜昭正典刑,
 以謝天下。」由是大與權貴迕,翻陷以不舉劾之罪,罷知蘄州。未半道,責海州團練副使,黃州安置。三年,得自便。



 欽宗立,以集英殿修撰起知潭州;未行,以兵部侍郎召,在道除中丞。初入見,帝諭以國家多難,每事當悉意盡言。於是節度使範訥丐歸環衛,過庭因言:「自崇寧以來,建旄鉞者多不由勛績,請除宗室及將帥立功者,餘並如訥例。」又乞辨宣仁後誣謗。姚古擁兵不援太原,陳其可斬之罪七,竄諸嶺表。進禮部尚書,擢右丞、中書侍郎。
 議遣大臣割兩河與金,耿南仲以老、聶昌以親辭,過庭曰:「主憂臣辱,願效死。」帝為揮涕嘆息,固遣南仲、昌。及城陷,過庭亦行,金人拘之軍中,因留不得還。建炎四年,卒於燕山,年六十,贈開府儀同三司,謚曰忠肅。



 張叔夜,字嵇仲,侍中耆孫也。少喜言兵,以蔭為蘭州錄事參軍。州本漢金城郡,地最極邊,恃河為固,每歲河冰合,必嚴兵以備,士不釋甲者累月。叔夜曰:「此非計也。不求要地守之,而使敵迫河,則吾既殆矣。」有地曰大都者,
 介五路間,羌人入寇,必先至彼點集,然後議所向,每一至則五路皆竦。叔夜按其形勢,畫攻取之策,訖得之,建為西安州,自是蘭無羌患。



 知襄城、陳留縣,蔣之奇薦之,易禮賓副使、通事舍人、知安肅軍,言者謂太優,還故官。獻所為文,知舒、海、泰三州。大觀中,為庫部員外郎、開封少尹。復獻文,召試制誥,賜進士出身,遷右司員外郎。



 使遼,宴射,首中的。遼人嘆詫,求觀所引弓,以無故事,拒不與。還,圖其山川、城郭、服器、儀範為五篇,上之。從弟克公
 彈蔡京,京遷怒叔夜,摭司存微過,貶監西安草場。久之,召為秘書少監,擢中書舍人、給事中。時吏惰不虔,凡命令之出於門下者,預列銜,使書名而徐填其事,謂之:「空黃」。叔夜極陳革其弊。進禮部侍郎,又為京所忌,以徽猷閣待制再知海州。



 宋江起河朔,轉略十郡,官軍莫敢嬰其鋒。聲言將至,叔夜使間者覘所向,賊徑趨海瀕,劫鉅舟十餘,載鹵獲。於是募死士得千人,設伏近城,而出輕兵距海,誘之戰。先匿壯卒海旁,伺兵合,舉火焚其舟。賊
 聞之,皆無鬥志,伏兵乘之,擒其副賊,江乃降。加直學士,徙濟南府。山東群盜猝至,叔夜度力不敵,謂僚吏曰:「若束手以俟援兵,民無□類,當以計緩之。使延三日,吾事濟矣。」乃取舊赦賊文,俾郵卒傳至郡,盜聞,果小懈。叔夜會飲譙門,示以閑暇,遣吏諭以恩旨。盜狐疑相持,至暮未決。叔夜發卒五千人,乘其惰擊之,盜奔潰,追斬數千級。以功進龍圖閣直學士、知青州。



 靖康改元,金人南下,叔夜再上章乞假騎兵,與諸將並力斷其歸路,不報。徙
 鄧州。四道置帥,叔夜領南道都總管。金兵再至,欽宗手札趣入衛。即自將中軍,子伯奮將前軍,仲熊將後軍,合三萬人,翌日上道。至尉氏,與金游兵遇,轉戰而前。十一月晦,至都,帝禦南熏門見之,軍容甚整。入對,言賊鋒方銳,願如唐明皇之避祿山,暫詣襄陽以圖幸雍。帝頷之。加延康殿學士。閏月,帝登城,叔夜陳兵玉津園,鎧甲光明,拜舞城下。帝益喜,進資政殿學士,令以兵入城,俄簽書樞密院。連四日,與金人大戰,斬其金環貴將二人。帝
 遣使繼蠟書,以褒寵叔夜之事檄告諸道,然迄無赴者。城陷,叔夜被創,猶父子力戰。車駕再出郊,叔夜因起居叩馬而諫,帝曰:「朕為生靈之故,不得不親往。」叔夜號慟再拜,眾皆哭。帝回首字之曰:「嵇仲努力!」



 金人議立異姓,叔夜謂孫傅曰:「今日之事,有死而已。」移書二帥,請立太子以從民望。二帥怒,追赴軍中,至則抗請如初,遂從以北。道中不食粟,唯時飲湯。既次白溝,馭者曰:「過界河矣。」叔夜乃矍然起,仰天大呼,遂不復語。明日,卒,年六十三。
 訃聞,贈開府儀同三司,謚曰忠文。



 聶昌,字幸遠,撫州臨川人。始繇太學上舍釋褐,為相州教授。用蔡攸薦,召除秘書郎,擢右司員外郎。時三省大吏階官視卿監者,立都司上,昌以名分未正,極論之。詔自今至朝請大夫止。以直龍圖閣為湖南轉運使,還為太府卿、戶部侍郎,改開封尹,復為戶部。昌本厚王黼,既而從蔡京,為黼所中,罷知德安府。又以鄉人訟,謫崇信軍節度副使,安置衡州。



 欽宗立,吳敏用事,以昌猛厲徑
 行為可助己,自散地授顯謨閣直學士、知開德府,道拜兵部侍郎,進戶部尚書,領開封府。昌遇事奮然不顧,敢誅殺。敏度不為用,始憚之,引唐恪、徐處仁共政,獨遺昌。



 李綱之罷,太學生陳東及士庶十餘萬人,撾鼓伏闕下,經日不退,遇內侍輒殺之,府尹王時雍麾之不去。帝顧昌俾出諭旨,即相率聽命。王時雍欲置東等獄,昌力言不可,乃止。



 昌再尹京,惡少年怙亂,晝為盜,入官民家攘金帛;且去,輒自縛黨中三兩輩,聲言擒盜,持仗部走
 委巷,乃釋縛,分所掠而去。人不奠居。昌悉彈治正法,而縱博弈不之問,或謂令所禁,昌曰:「姑從所嗜,以懈其謀,是正所以禁其為非爾。」昌舊名山,至是,帝謂其有周昌抗節之義,乃命之曰「昌」。



 京師復戒嚴,拜同知樞密院。入謝,即陳捍敵之策,曰:「三關四鎮,國家藩籬也,聞欲以畀敵,一朝渝盟,何以制之?願勿輕與,而檄天下兵集都畿,堅城守以遏其沖,簡禁旅以備出擊,壅河流以斷歸路。前有堅城,後有大河,勁兵四面而至,彼或南下,墮吾網
 中矣。臣願激合勇義之士,設伏開關,出不意掃其營以報。」帝壯之,命提舉守御,得以便宜行事。



 會金人再議和,割兩河,須大臣報聘。詔耿南仲及昌往,昌言:「兩河之人忠議勇勁,萬一不從,必為所執,死不瞑目矣。儻和議不遂,臣當分遣官屬,促勤王之師入衛。」許之。行次永安,與金將黏罕遇,其從者稱合門舍人,止昌徹傘,令用榜子贊名引見,昌不可,爭辨移時,卒以客禮見。昌往河東,至絳,絳人閉壁拒之。昌持詔抵城下,縋而登。州鈐轄趙子
 清麾眾害昌,抉其目而臠之,年四十九。



 建炎四年,始贈觀文殿大學士,謚曰忠愍。父用之,年九十,以憂死。



 昌為人疏雋,喜周人之急,然恩怨太明,睚眥必報。王黼之死,昌實遣客刺之,棄尸道旁。遂附耿南仲取顯位,左右其說以誤國,卒至禍變,而身亦不免焉。



 論曰:「何慄、孫傅、聶昌皆疏俊之士,而器質窳薄,使當重任于艱難之秋,宋事蓋可知矣,欽宗之再詣金營,慄實誤之,一死不足償也。傅匿太子之謀甚疏,昌河東之行
 尤謬,效死弗當,徒傷勇爾。過庭因方臘之亂,乞誅蔡京、王黼、朱勉以謝天下,庶幾有敢諫之風焉。



 張閣,字臺卿,河陽人。第進士。崇寧初,由衛尉主簿遷祠部員外郎;資閱淺,為掌制者所議,蔡京主之,乃止。俄徙吏部,遷宗正少卿、起居舍人,屬疾不能朝,改顯謨閣待制、提舉崇福宮。疾愈,拜給事中、殿中監,為翰林學士。



 河北諸帥以繕城訖役,降獎詔,有中貴人為之地,將繼此策賞。閣言:「此牧伯常職,若獎之,恐開邀功生事之路。」徽
 宗曰:「卿言是也。」格不下。嘗夜盛寒草制稿進,帝猶坐,賞其警敏,賜詩以為寵。京免相,閣當制,歷數其過,詞語遒拔,人士多傳誦之。



 京復相,以龍圖閣學士知杭州。浙部和買絹,杭獨居十三,戶有至數百匹者,閣請均之他郡。杭久闕守,閣經理有敘,去惡少年之為人害者,州以理聞。召拜兵部尚書兼侍讀,復為學士,上日特賜敕詔,且有意大用,未幾,卒,年四十六。閣初出守杭,思所以固寵,辭日,乞自領花石綱事,應奉由
 是滋熾云。



 張近,字幾仲,開封人。第進士,累遷大理正、發運使。呂溫卿以不法聞,近受詔鞫治,哲宗諭之曰:「此出朕命,卿毋畏惠卿。」對曰:「法之所在,雖陛下不能使臣輕重,何惠卿也?」溫卿謾不肯置對,近言:「溫卿所坐明白,儻聽其蔓詞,懼為株連者累。」詔以眾證定其罪。提舉河北東路常平、西路刑獄,入為刑部員外郎、大理少卿,以集賢殿修撰知瀛州。



 遼使為夏人請命,而宿兵以臨我,近請亦出秦甲戍北道,伐其謀。邊人呂懺兒入瓦橋為盜,吏執之,遼
 人因略宋民為質。近言:「朝廷方繼好息民,當使曲在彼。一偷之得失,不足為輕重,釋之便。」滄民漁於海,遼卒利其饒,而私舉網取魚。守兵與之斗,斬級三十二,州將請賞之。或言所殺乃平人,宜論如律,議弗決。近言:「邊人貪利喜功,遂賞之,則為國起怨;然彼挾兵涉吾地,謂之非盜可乎?如罪以擅興,他日將誰使御敵?願兩置賞刑,略而不問。」從之。



 出鎮高陽八年,累加顯謨閣待制、直學士,徙知太原府,以疾,提舉洞霄宮。先,承詔買馬三千給
 牧戶,近悉斂諸民而不予直,為御史所劾,失學士。二年而復之。卒,年六十五。



 鄭僅,字彥能,徐州彭城人。第進士,為大名府司戶參軍。留守文彥博以為材,部使者檄往他郡,彥博曰:「如鄭參軍詎可令數出?」奏改司法,遷冠氏令。河決府西,檄夜下調夫急,僅方閱保甲,盡籍即行,先他邑至,決遂塞。使者怒劾之,留守王拱辰爭於朝曰:「微冠氏,城民魚矣。」猶坐罰金。時河朔饑,盜起,獨冠氏無之,且不入境。他邑獲盜,
 詰治之,盜因言:「鄭冠氏仁,故相戒不犯爾。」知福昌縣,復值歲饑,悉意振貸,民不流亡。當第賞,不肯自列。



 提舉京東常平,入為戶部員外郎,至太府卿,加直龍圖閣,為陜西都轉運使。論饋餉河湟功,進集賢殿修撰、顯謨閣待制。僅請籍閉田為官莊,是歲,鎮戎、德順收穀十餘萬。會西寧高永年戰沒熙河,帥臣歸咎官莊奪屬羌地,致其怨畔,詔罷之,議者以為惜。



 改知慶州,諸軍多殺老弱,持首要賞。僅下令非強壯而能生致者,賞半之。有內附羌
 追寇,得老人,不忍殺,擒之,乃其父也,相持哭,一軍感動。時諸路爭進討奏捷,僅獨保境不生事,寇亦不犯。



 徙秦州,復為都轉運使,召拜戶部侍郎,改吏部侍郎、知徐州。以顯謨閣直學士、通議大夫卒,年六十七,贈光祿大夫,謚曰修敏。子望之,自有傳。



 宇文昌齡,字伯修,成都雙流人。進士甲科,調榮州推官。熊本經制梓夔,闢乾當公事。凡攻討招襲,建南平諸城砦,皆出其畫。遷大理丞。本歸闕,言其功,擢提舉秦鳳路
 常平,改兩浙。



 神宗患司農圖籍不肅,選官厘整,昌齡以使夔路入辭,留為寺主簿,遂拜監察御史。鄜延帥奏所部劉紹能與西羌通,將為患。帝察其不然,命昌齡即鄜州鞫之,果妄也。昌齡因請深戒守臣,毋生事徼賞,以靖邊人之心。使還,賜五品服。



 尚書省建,以為比部員外郎。時官曹更新,統紀未立,昌齡悉力從事,雖抵暮亦程吏不止。具所立綱要,請於朝而行之。三司故吏狃玩弛,多不便,思有以中之。擿邏卒糾其宿直遣小吏取衾服事,
 大臣欲論以私役,帝以職事修飭,釋不問。改吏部員外郎,出京西轉運副使,召為左司員外郎。



 送遼使至雄州,當宴,從者不待揖而坐,昌齡誚其使曰:「兩朝聘好百年矣,入境置宴,非但今日,揖而後坐,此禮渠可闕邪?」使者陽若不服,而心悟其非,卒成禮去。



 遷太常少卿,詔議郊祀合祭,論者不一。昌齡曰:「天地之數,以高卑則異位,以禮制則異宜,以樂舞則異數,至於衣服之章,器用之具,日至之時,皆有辨而不亂。夫祀者自有以感於無,自實
 以通於虛,必以類應類,以氣合氣,合然後可以得而親,可以冀其格。今祭地於圜丘,以氣則非所合,以類則非所應,而求高厚之來享,不亦難乎。」後竟用其議。改直秘閣、知梓州,歷壽州、河中府、鄧、鄆、青三州。



 徽宗立,召為刑部侍郎,徙戶部侍郎。陜西饋芻糧於邊,舊制令內郡轉給,為民病。昌齡建言止輸其州,而令量取道里費助邊糴,從之。歲省糴價五百萬,公私便之。以寶文閣待制知開封府,復為戶部侍郎,知青、杭、越三州。卒,年六十五,詔為
 封傅護送歸,官給其葬費。子常。



 常字權可。政和末,知黎州。有上書乞於大渡河外置城邑以便互市者,詔以訪常。常言:「自孟氏入朝,藝祖取蜀輿地圖觀之,畫大渡為境,歷百五十年無西南夷患。今若於河外建城立邑,虜情攜貳,邊隙寢開,非中國之福也。」尋提舉成都路茶馬。自熙、豐以來,歲入馬蕃多;至崇、觀間,其法始壞。提舉官歲以所入進羨餘,吏緣為奸,市馬裁十一二,且負其直,夷人皆怨。常盡革其弊,馬遂溢
 額。加直秘閣,改知夔州,進秘閣修撰。官累中大夫。卒。



 許幾,字先之,信州貴溪人。少以諸生謁韓琦於魏,琦勉入太學。擢第,調高安、樂平主簿,知南陵縣,還民之托僧尼為奸者數百人。



 提舉京西常平,為開封府推官,進至將作監。吏與匠比為奸欺,凡斫削、塗塈、丹雘之工當以次用,而始役即概給其稟,費亡藝而患不均。幾逆為之程,費省工倍。再遷太僕卿、戶部侍郎,以顯謨閣待制知鄆州。



 梁山濼多盜,皆漁者窟穴也。幾籍十人為保,使晨
 出夕歸,否則以告,輒窮治,無脫者。



 幾有吏乾,善理財,由是四入戶部至尚書。嘗以搖泉布法罷,又以治染院事失實,知婺州。進樞密直學士、河北都轉運使、徙知成德軍、知太原府。張商英裁損吏祿,幾預其議,貶永州團練副使,安置袁州。遇恩,復中大夫,卒。



 程之邵,字懿叔,眉州眉山人。曾祖仁霸,治獄有陰德。之邵以父蔭為新繁主簿。熙寧更募役法,常平使者欲概州縣民力,以羨乏相補。之邵曰:「此法乃成周均力遺意,
 當各以一邑之力供一邑之役,豈宜以此邑助他邑哉?」使者愧服,闢之邵為屬,聽其所為。熊本察訪蜀道歸,語諸朝曰:「役法初行,成都路為最詳,之邵力也。」詔召見,成都守趙抃奏留之。入為三司磨勘官,得隱匿數十萬緡。從副使蹇周輔計度江、嶺鹽,還,除廣東轉運判官。元祐初,提舉利、梓路常平,周輔得罪,亦罷知祥符縣。俄知泗州,為夔路轉運判官。夔守強狠不奉法,劾正其罪。大寧井鹽為利博,前議者輒儲其半供公上,餘鬻於民,使先
 輸錢,鹽不足給,民以病告。之邵盡發所儲與之,商賈既通,關征增數倍。除主管秦、蜀茶馬公事,革黎州買馬之弊,歲以仲秋為市,市四月止,以羨茶入熙、秦易戰騎,得良馬益多。



 知鳳翔府,民負債無以償,自焚其居,而紿曰遺火;有主藏吏殺四婢,人無知者。之邵發擿,岐人傳誦。徙鄭州。



 元符中復主管茶馬,市馬至萬匹,得茶課四百萬緡。童貫用師熙、岷,不俟報,運茶往博糴,發錢二十萬億佐用度。連加直龍圖閣、集賢殿修撰,三進秩,為熙河
 都轉運使。秦鳳出師,命之經制,即言已備十萬騎可食三百日矣。徽宗喜,擢顯謨閣待制。敵犯熙河,之邵攝帥事,屯兵行邊境,解去。俄得疾卒。方錄功轉太中大夫,不及拜,贈龍圖閣直學士,官護喪歸。子唐,至寶文閣學士。



 龔原,字深之,處州遂昌人。少與陸佃同師王安石。進士高第,元豐中為國子直講,以虞蕃訟失官。哲宗即位,詣訴理所得直,為國子丞、太常博士。方議祀北郊,原曰:「合祭,非理也。天子父天母地,既不為寒而廢祠,其可為暑
 而輟行?此漢儒陋說爾,願亟正之。」加秘閣校理,充徐王府記室,出為兩浙轉運判官。



 紹聖初,召拜國子司業,入對,帝問曰:「卿歷徐邸官,何為補外,得非大臣私意乎?」對曰:「臣出使鄉部,獲知民間事宜,臣素知如是,不知其因也。」旋兼侍講,遷秘書少監、起居舍人,權工部侍郎。為曾布所重,安惇論其直講時事,以集賢殿修撰知潤州。



 徽宗初,入為秘書監,進給事中。時除郎官五人,皆執政姻戚,悉舉駁之;又論郝隨得罪,不得居京師,鄧洵武不宜
 再入史院。朝論謂帝為哲宗服,當循開寶故事,為齊衰期。原曰:「三年之喪,自天子達於庶人,一也。」主議者斥其妄,黜知南康軍,改壽州。俄用三年之制,乃復修撰,知揚州。還朝,歷兵、工部二侍郎,除寶文閣待制、知廬州。陳瓘擊蔡京,原與瓘善,或謂原實使之,奪職居和州。起為亳州,命下而卒,年六十七。



 初,王安石改學校法,引原自助,原亦為盡力。其後,司馬光召與語,譏切王氏,原反復辨救不少衰。光嘆曰:「王氏習氣尚爾邪!」為司業時,請以安
 石所撰《字說》、《洪範傳》及子雱《論語》、《孟子義》刊板傳學者。故一時學校舉子之文,靡然從之,其敝自原始。



 崔公度,字伯易,高郵人。口吃不能劇談,而內絕敏,書一閱即不忘。劉沆薦茂才異等,辭疾不應命。用父任,補三班差使,非其好也,益閉戶讀書。歐陽修得其所作《感山賦》,以示韓琦,琦上之英宗,即付史館。授和州防禦推官,為國子直講,以母老辭。



 王安石當國,獻《熙寧稽古一法百利論》,安石解衣握手,延與語。召對延和殿,進光祿丞,
 知陽武縣。京官謁尹,故事當拜庭下,公度疑尹辱己,徑詣安石訴之,安石使鄧綰薦為御史。未幾,為崇文校書,刪定三司令式,於是誦言京官庭謁尹非宜,安石為下編敕所更其制。加集賢校理,知太常禮院。



 公度起布衣,無所持守,惟知媚附安石,晝夜造請,雖踞廁見之,不屑也。嘗從後執其帶尾,安石反顧,公度笑曰:「相公帶有垢,敬以袍拭去之爾。」見者皆笑,亦恬不為恥。請知海州。元祐、紹聖之間,歷兵禮部郎中、國子司業,除秘書少監、起
 居郎,皆辭不受。知穎、潤、宣、通四州,以直龍圖閣卒。



 蒲卣,字君錫,閬州人。母任知書,里中號「任五經」,卣幼以開敏聞。中進士第,歷利州司戶參軍、三泉主簿、知合江金水縣。通判文州,有獻議者欲開文州徑路達陜西,卣言:「洮、岷、積石至文為甚邇,自文出江油,鄧艾取蜀故道也。異時鬼章欲從此窺蜀,為其阻隘而止。夏人志此久矣,可為之信道乎?」議遂塞。



 為睦親宅教授,提舉湖北、京西常平。崇寧均田,轉運使以用不足,將度費以定稅,卣
 曰:「詔旨所以嘉惠元元爾,初不在增賦也。」宛、穰地廣沃,國初募民墾田,得為世業,令人毋輒訴,蓋百年矣,好訟者稍以易佃法搖之,卣一切禁止。有持獻於權貴而降中旨給賜者,卣言:「地盈千頃,戶且數百,傳子至孫久,一旦改隸,眾將不安。先朝明詔具在,不可易也。」朝廷是其議。



 提點湖南刑獄,知鼎、遼、隴、寧四州,復提舉潼川路刑獄。有議榷酤於瀘、敘間,云歲可得錢二十萬。卣言:「先朝念此地夷漢雜居,故弛其榷禁,以惠安邊人。今之所行,
 未見其利。」乃止。累官中大夫,卒,年七十二。



 論曰:《傳》曰:「尺有所不逮,寸有所不覃。」觀二張之理郡,鄭僅之守藩,宇文父子之便邊糶、革馬政,許幾、程之邵之經制財運,蒲卣之議稅榷,皆有可稱道。若閣之固寵於花石,而龔原、崔公度主王氏學以諂事安石,則搢紳所不齒也。



\end{pinyinscope}