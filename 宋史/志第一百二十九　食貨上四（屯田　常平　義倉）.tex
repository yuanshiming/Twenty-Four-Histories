\article{志第一百二十九 食貨上四(屯田 常平 義倉)}

\begin{pinyinscope}

 前代軍師所在,有地利則開屯田、營田,以省饋餉。宋太宗伐契丹,規取燕薊,邊隙一開,河朔連歲繹騷,耕織失業,州縣多閑田,而緣邊益增戍兵。自雄州東際於海,
 多積水,契丹患之,不得肆其侵突;順安軍西至北平二百里,其地平曠,歲常自此而入。議者謂宜度地形高下,因水陸之便,建阡陌,浚溝洫,益樹五稼,可以實邊廩而限戎馬。端拱二年,分命左諫議大夫陳恕、右諫議大夫樊知古為河北東、西路招置營田使,恕對極言非便。行數日,有詔令修完城堡,通導溝瀆,而營田之議遂寢。時又命知代州張齊賢制置河東諸州營田,尋亦罷。



 六宅使何承矩請於順安砦西引易河築堤為屯田。既而河朔
 連年大水,及承矩知雄州,又言宜因積潦蓄為陂塘,大作稻田以足食。會滄州臨津令閩人黃懋上書言:「閩地惟種水田,緣山導泉,倍費功力。今河北州軍多陂塘,引水溉田,省功易就,五三年間,公私必大獲其利。」詔承矩按視還,奏如懋言。遂以承矩為制置河北沿邊屯田使,懋為大理寺丞充判官,發諸州鎮兵一萬八千人給其役。凡雄莫霸州、平戎順安等軍興堰六百里,置斗門,引澱水灌溉。初年種稻,值霜不成。懋以晚稻九月熟,河北
 霜早而地氣遲,江東早稻七月即熟,取其種課令種之,是歲八月,稻熟。初,承矩建議,沮之者頗眾;武臣習攻戰,亦恥於營葺。既種稻不成,群議愈甚,事幾為罷。至是,承矩載稻穗數車,遣吏送闕下,議者乃息。而莞蒲、蜃蛤之饒,民賴其利。



 度支判官陳堯叟等亦言:「漢、魏、晉、唐於陳、許、鄧、穎暨蔡、宿、亳至於壽春,用水利墾田,陳跡具在。議選官大開屯田,以通水利,發江、淮下軍散卒及募民充役。給官錢市牛、置耕具,導溝瀆,築防堰。每屯十人,人給
 一牛,治田五十畝,雖古制一夫百畝,今且墾其半,俟久而古制可復也。畝約收三斛,歲可收十五萬斛,七州之間置二十屯,可得三百萬斛,因而益之,數年可使倉廩充實,省江、淮漕運。民田未闢,官為種植,公田未墾,募民墾之,歲登所取,並如民間主客之例。傅子曰:『陸田命懸於天,人力雖修,茍水旱不時,則一年之功棄矣。水田之制由人力,人力茍修,則地利可盡。』且蟲災之害亦少於陸田,水田既修,其利兼倍。」帝覽奏嘉之,遣大理寺丞皇
 甫選、光祿寺丞何亮乘傳按視經度,然不果行。



 至咸平中,大理寺丞王宗旦請募民耕穎州陂塘荒地凡千五百頃。部民應募者三百餘戶,詔令未出租稅,免其徭役。然無助於功利。而汝州舊有洛南務,內園兵人種稻,雍熙二年罷,賦予民,至是復置,命京朝官專掌。募民戶二百餘,自備耕牛,立團長,墾地六百頃,導汝水溉灌,歲收二萬三千石。襄陽縣淳河,舊作堤截水入官渠,溉民田三千頃;宜城縣蠻河,溉田七百頃;又有屯田三百餘頃。
 知襄州耿望請於舊地兼括荒田,置營田上、中、下三務,調夫五百,築堤堰,仍集鄰州兵每務二百人,荊湖市牛七百分給之。是歲,種稻三百餘頃。



 四年,陜西轉運使劉綜亦言:「宜於古原州建鎮戎軍置屯田。今本軍一歲給芻糧四十餘萬石、束,約費茶鹽五十餘萬,儻更令遠民輸送,其費益多。請於軍城四面立屯田務,開田五百頃,置下軍二千人、牛八百頭耕種之;又於軍城前後及北至水峽口,各置堡砦,分居其人,無寇則耕,寇來則戰。就
 命知軍為屯田制置使,自擇使臣充四砦監押,每砦五百人充屯戍。」從之。既而原、渭州亦開方田,戎人內屬者皆依之得安其居。



 是時兵費浸廣,言屯、營田者,輒詔邊臣經度行之。順安軍兵馬都監馬濟請於靖戎軍東壅鮑河,開渠入順安、威虜二軍,置水陸營田於其側。命莫州部署石普護其役,逾年而畢。知保州趙彬復奏決雞距泉,自州西至蒲城縣,分徐河水南流注運渠,廣置水陸屯田,詔駐泊都監王昭遜共成之。自是定州亦置屯
 田。五年,罷襄州營田下務。六年,耿望又請於唐州赭陽陂置務如襄州,歲種七十餘頃,方城縣令佐掌之,調夫耘耨。



 景德初,從京西轉運使張巽之請,詔止役務兵。二年,令緣邊有屯、營田州軍,長吏並兼制置諸營田、屯田事,舊兼使者如故。大中祥符九年,改定保州、順安軍營田務為屯田務,凡九州軍皆遣官監務,置吏屬。淮南、兩浙舊皆有屯田,後多賦民而收其租,第存其名。在河北者雖有其實,而歲入無幾,利在蓄水以限戎馬而已。天
 禧末,諸州屯田總四千二百餘頃,河北歲收二萬九千四百餘石,而保州最多,逾其半焉。



 襄、唐二州營田既廢,景德中,轉運使許逖復之。初,耿望借種田人牛及調夫耨獲,歲入甚廣。後張巽改其法,募水戶分耕,至逖又參以兵夫,久之無大利。天聖四年,遣尚書屯田員外郎劉漢傑往視,漢傑言:「二州營田自復至今,襄州得穀三十三萬餘石,為緡錢九萬餘;唐州得穀六萬餘石,為緡錢二萬餘。所給吏兵俸廩、官牛雜費,襄州十三萬餘緡,唐
 州四萬餘緡,得不補失。」詔廢以給貧民,頃收半稅。



 其後陜西用兵,詔轉運司度隙地置營田以助邊計,又假同州沙苑監牧地為營田,而知永興軍範雍括諸郡牛頗煩擾,未幾遂罷。右正言田況言:「鎮戎、原、渭,地方數百里,舊皆民田,今無復農事,可即其地大興營田,以保捷兵不習戰者分耕,五百人為一堡,三兩堡置營田官一領之,播種以時,農隙則習武事。」疏奏,不用。後乃命三司戶部副使夏安期等議並邊置屯田,迄不能成。



 治平三年,
 河北屯田三百六十七頃,得穀三萬五千四百六十八石。熙寧初,以內侍押班李若愚同提點制置河北屯田事。三年,王韶言:「渭原城而下至秦州成紀,旁河五六百里,良田不耕者無慮萬頃,治千頃,歲可得三十萬斛。」知秦州李師中論:「韶指極邊見招弓箭手地,恐秦州益多事。」詔遣王克臣等按視,復奏與師中同。再下沉起,起奏:「不見韶所指何地,雖實有之,恐召人耕種,西蕃驚疑。」侍御史謝景溫言:「聞沈起妄指甘谷城弓箭手地以塞韶
 妄。」而竇舜卿奏:「實止有閑田一頃四十三畝。」中書言:「起未嘗指甘谷城地以實韶奏,而師中前在秦州與韶更相論奏,互有曲直。」韶遂以妄指閑田自著作佐郎責保平軍節度推官,師中亦落待制。其後韓縝知秦州,乃言:「實有古渭砦弓箭手未請空地四千餘頃。」遂復韶故官,從其所請行之。明年,河北屯田司奏:「豐歲屯田,入不償費。」於是詔罷緣邊水陸屯田務,募民租佃,收其兵為州廂軍。



 時陜西曠土多未耕,屯戍不可撤,遠方有輸送之
 勤,知延州趙離請募民耕以紓朝廷憂,詔下其事。經略安撫使郭逵言:「懷寧砦所得地百里,以募弓箭手,無閑田。」離又言之,遂括地得萬五千餘頃,募漢蕃兵幾五千人,為八指揮,詔遷離官,賜金帛。而熙州王韶又請以河州蕃部近城川地招弓箭手,以山坡地招蕃兵弓箭手,每砦五指揮,以二百五十人為額,人給地一頃,蕃官二頃,大蕃官三頃。熙河多良田,七年,詔委提點秦鳳路刑獄鄭民憲興營田,許奏闢官屬以集事。



 樞密使吳充上
 疏曰:「今之屯田,誠未易行。古者一夫百畝,又受田十畝為公田,莫若因弓箭手仿古助田法行之。熙河四州田無慮萬五千頃,十分取一以為公田,大約中歲畝一石,則公田所得十五萬石。官無屯營牛具廩給之費,借用眾力而民不勞,大荒不收而官無所損,省轉輸,平糴價,如是者其便有六。」而提點刑獄鄭民憲言:「祖宗時屯、營田皆置務,屯田以兵,營田以民,固有異制。然襄州營田既調夫矣,又取鄰州之兵,是營田不獨以民也;邊州營
 屯,不限兵民,皆取給用,是屯田不獨以兵也;至於招弓箭手不盡之地,復以募民,則兵民參錯,固無異也。而前後施行,或侵占民田,或差借耨夫,或諸郡括牛,或兵民雜耕,或諸州廂軍不習耕種、不能水土,頗致煩擾。至於歲之所入,不償其費,遂又報罷。惟因弓箭手為助田法,一夫受田百畝,別以十畝為公田,俾之自備種糧功力,歲畝收一石,水旱三分除一,官無廩給之費,民有耕鑿之利,若可以為便。然弓箭手之招至,未安其業,而種糧
 無所仰給,又責其借力於公田,慮人心易搖,乞候稍稔推行。」九年,詔:「熙河弓箭手耕種不及之田,經略安撫司點廂軍佃之,官置牛具農器,人一頃,歲終參較弓箭手、廂軍所種優劣為賞罰。弓箭手逃地並營田召佃租課,許就近於本城砦輸納,仍免折變、支移。」



 元豐二年,改定州屯田司為水利司。及章惇築沅州,亦為屯田務,其後遂罷之,募民租佃,役兵各還所隸。五年,詔提舉熙河等路弓箭手、營田、蕃部共為一司,隸涇原路制置司。提舉
 熙河營田康識言:「新復土地,乞命官分畫經界,選知田廂軍,人給一頃耕之,餘悉給弓箭手,人加一頃,有馬者又加五十畝,每五十頃為一營。」「四砦堡見缺農作廂軍,許於秦鳳、涇原、熙河三路選募廂軍及馬遞鋪卒,願行者人給裝錢二千。」詔皆從之。



 知太原府呂惠卿嘗上《營田疏》曰:「今葭蘆、米脂里外良田,不啻一二萬頃,夏人名為『真珠山』、『七寶山』,言其多出禾粟也。若耕其半,則兩路新砦兵費,已不盡資內地,況能盡闢之乎?前此所不敢
 進耕者,外無捍衛也。今於葭蘆、米脂相去一百二十里間,各建一砦,又其間置小堡鋪相望,則延州之義合、白草與石州之吳堡、克明以南諸誠砦,千裏邊面皆為內地,而河外三州荒閑之地,皆可墾闢以贍軍用。凡昔為夏人所侵及蘇安靖棄之以為兩不耕者,皆可為法耕之。於是就糴河外,而使河內之民被支移者,量出腳乘之直,革百年遠輸貴糴,以免困公之弊。財力稍豐,又通葭蘆之道於麟州之神木,其通堡砦亦如葭蘆、米脂之
 法,而橫山膏腴之地,皆為我有矣。」



 七年,惠卿雇五縣耕牛,發將兵外護,而耕新疆葭蘆、吳堡間膏腴地號木瓜原者,凡得地五百餘頃,麟、府、豐州地七百三十頃,弓箭手與民之無力及異時兩不耕者又九百六十頃。惠卿自謂所得極厚,可助邊計,乞推之陜西。八年,樞密院奏:「去年耕種木瓜原,凡用將兵萬八千餘人,馬二千餘匹,費錢七千餘緡,穀近九千石,糗□近五萬斤,草萬四千餘束;又保甲守禦費緡錢千三百,米石三千二百,役耕
 民千五百,雇牛千具,皆強民為之;所收禾粟、蕎麥萬八千石,草十萬二千,不償所費。又借轉運司錢穀以為子種,至今未償,增入人馬防拓之費,仍在年計之外。慮經略司來年再欲耕種,乞早約束。」詔諭惠卿毋蹈前失。



 河東進築堡砦,自麟石、鄜延南北近三百里,及涇原、環慶、熙河蘭會新復城砦地土,悉募廂軍配卒耕種免役。已而營田司言諸路募發廂軍皆不閑田作,遂各遣還其州。



 紹興元年,知荊南府解潛奏闢宗綱、樊賓措置屯田,
 詔除宗綱充荊南府、歸峽州、荊門公安軍鎮撫使司措置五州營田官,樊賓副之。渡江後營田蓋始於此。其後荊州軍食仰給,省縣官之半焉。三年,德安府、復州、漢陽軍鎮撫使陳規放古屯田,凡軍士:相險隘,立堡砦,且守且耕,耕必給費,斂復給糧,依鋤田法,餘並入官。凡民:水田畝賦粳米一斗,陸田豆麥夏秋各五升,滿二年無欠,給為永業。兵民各處一方,流民歸業浸眾,亦置堡砦屯聚之。凡屯田事,營田司兼之;營田事,府、縣兼之。廷臣因
 規奏推廣,謂一夫授田百畝,古制也,今荒田甚多,當聽百姓請射。其有闕耕牛者,宜用人耕之法,以二人曳一犁。凡授田,五人為甲,別給蔬地五畝為廬舍場圃。兵屯以大使臣主之,民屯以縣令主之,以歲課多少為殿最。下諸鎮推行之。



 詔江東、西宣撫使韓世忠措置建康營田,如陜西弓箭手法。世忠言:「沿江荒田雖多,大半有主,難如陜西例,乞募民承佃。」都督府奏如世忠議,仍蠲三年租,滿五年,田主無自陳者,給佃者為永業。詔湖北、浙
 西、江西皆如之。其徭役科配並免。五年,詔淮南、川陜、荊襄屯田。



 六年,都督張浚奏改江、淮屯田為營田,凡官田逃田並拘籍,以五頃為一莊,募民承佃。其法:五家為保,共佃一莊,以一人為長,每莊給牛五具,耒耜及種副之,別給十畝為蔬圃,貸錢七十千,分五年償。命樊賓、王弗行之。尋命五大將劉光世、韓世忠、張俊、岳飛、吳玠及江淮、荊、襄、利路帥悉領營田使。遷賓司農少卿,提舉江、淮營田,置司建康,弗屯田員外郎副之。官給牛、種,撫存流
 移,一歲中收穀三十萬石有奇。殿中侍御史石公揆、監中獄李寀及王弗皆言營田之害,張浚亦覺其擾,請罷司,以監司領之,於是詔帥臣兼領營田。



 九月,以川陜宣撫吳玠治廢堰營田六十莊,計田八百五十四頃,歲收二十五萬石以助軍儲,賜詔獎諭焉。三十二年,督視湖北、京西軍馬汪澈言:「荊、湖兩軍屯守襄、漢,糧餉浩瀚。襄陽古有二渠,長渠溉田七千頃,木渠溉田三千頃,兵後堙廢。今先築堰開渠,募邊民或兵之老弱耕之,其耕牛、
 耒耜、種糧,令河北、京西轉運司措置,既省饋運,又可安集流亡。」從之。



 隆興元年,臣僚言州縣營田之實,其說有十,曰:擇官必審,募人必廣,穿渠必深,鄉亭必修,器用必備,田處必利,食用必充,耕具必足,定稅必輕,賞罰必行。且欲立賞格以募人,及住廣西馬綱三年以市牛。會有訴襄陽屯田之擾者,上欲罷之。工部尚書張闡言:「今日荊襄屯田之害,以其無耕田之民而課之游民,游民不足而強之百姓,於是百姓舍己熟田而耕官生田,或遠
 數百里征呼以來,或名雙丁而役其強壯,老稚無養,一方騷然,罷之誠是也。然自去歲以來,置耕牛農器,修長、木二渠,費已十餘萬,一旦舉而棄之,則荊襄之地終不可耕也。比見兩淮歸正之民,動以萬計,官不能續食,則老弱饑死,強者轉而之他。若使之就耕荊襄之田,非惟可免流離,抑使中原之民聞之,知朝廷有以處我,率皆襁負而至矣。異時墾闢既廣,取其餘以輸官,實為兩便。」詔除見耕者依舊,餘令虞允文同王玨措置。二年,江、淮
 都督府參贊陳俊卿言:「欲以不披帶人,擇官荒田,標旗立砦,多買牛犁,縱耕其中,官不收租,人自樂從。數年之後,墾田必多,穀必賤。所在有屯,則村落無盜賊之憂;軍食既足,則饋餉無轉運之勞。此誠經久守淮之策。」詔從之。



 乾道五年三月,四川宣撫使鄭剛中撥軍耕種,以歲收租米對減成都路對糴米一十二萬石贍軍。然兵民雜處村□,為擾百端;又數百里外差民保甲教耕,有二、三年不代者,民甚苦之。知興元府晁公武欲以三年所
 收最高一年為額,等第均數召佃,放兵及保甲以護邊。從之。八月,詔鎮江都統司及武鋒軍三處屯田兵並拘收入隊教閱。六年,罷和、揚州屯田。八年,復罷廬州兵屯田。



 淳熙十年,鄂州、江陵府駐扎副都統制郭杲言:「襄陽屯田,興置二十餘年,未能大有益於邊計。非田之不良,蓋人力有所未至。今邊陲無事,正宜修舉,為實邊之計。本司有荒熟田七百五十頃,乞降錢三萬緡,收買耕牛農具,便可施功。如將來更有餘力,可括荒田接續開墾。」
 從之。



 紹熙元年,知和州劉煒以剩田募民充萬弩手分耕。嘉定七年,以京西屯田募人耕種。十三年,四川宣撫安丙、總領任處厚言:「紹興十五年,諸州共墾田二千六百五十餘頃,夏秋輸租米一十四萬一千餘石,餉所屯將兵,罷民和糴,為利可謂博矣。乾道四年以後,屯兵歸軍教閱,而營田付諸州募佃,遂致租利陷失,驕將豪民乘時占據,其弊不可概舉。今豪強移徙,田土荒閑,正當拘種之秋,合自總領所與宣撫司措置。其逃絕之田,關
 內外亦多有之,為數不貲,其利不在營田之下,乞並括之。」初,玠守蜀,以軍儲不繼,治褒城堰為屯田,民不以為便。因漕臣郭大中言,約中其數,使民自耕。民皆歸業,而歲入多於屯田。



 端平元年八月,以臣僚言,屯五萬人於淮之南北,且田且守,置屯田判官一員經紀其事,暇則教以騎射。初弛田租三年,又三年則取其半。十月,知大寧監邵潛言:「昔鄭剛中嘗於蜀之關隘雜兵民屯田,歲收粟二十餘萬石。是後屯田之利既廢,糧運之費益
 增,宜詔帥臣縱兵民耕之,所收之粟計直以償之,則總所無轉輸之苦,邊關有儲峙之豐,戰有餘勇,守有餘備矣。」從之。



 嘉熙四年,令流民於邊江七十里內分田以耕,遇警則用以守江;於邊城三、五十里內亦分田以耕,遇警則用以守城;在砦者則耕四野之田,而用以守砦。田在官者免其租,在民者以所收十之一二歸其主,俟三年事定則各還元業。



 咸淳三年,詔曰:「淮、蜀、湖、襄之民所種屯田,既困重額,又困苛取,流離之餘,口體不充,及遇水
 旱,收租不及,而催輸急於星火,民何以堪!其日前舊欠並除之,復催者以違制論。」



 常平、義倉,漢、隋利民之良法,常平以平谷價,義倉以備兇災。周顯德中,又置惠民倉,以雜配錢分數折粟貯之,歲歉,減價出以惠民。宋兼存其法焉。



 太祖承五季之亂,海內多事,義倉浸廢。乾德初,詔諸州於各縣置義倉,歲輸二稅,石別收一斗。民饑欲貸充種食者,縣具籍申州,州長吏即計口貸訖,然後奏聞。其後以輸送煩勞,罷之。
 淳化三年,京畿大穰,分遣使臣於四城門置場,增價以糴,虛近倉貯之,命曰常平,歲饑即下其直予民。



 咸平中,庫部員外郎成肅請福建增置惠民倉,因詔諸路申淳化惠民之制。景德三年,言事者請於京東西、河北、河東、陜西、江南、淮南、兩浙皆立常平倉,計戶口多寡,量留上供錢自二三千貫至一二萬貫,令轉運使每州擇清幹官主之,領於司農寺,三司無輒移用。歲夏秋視市價量增以糴,糶減價亦如之,所減不得過本錢。而沿邊州郡
 不置。詔三司集議,請如所奏。於是增置司農官吏,創廨舍,藏籍帳,度支別置常平案。大率萬戶歲糴萬石,戶雖多,止五萬石。三年以上不糶,即回充糧廩,易以新粟。災傷州郡糴粟,鬥毋過百錢。後又詔當職官於元約數外增糴及一倍已上者,並與理為勞績。天禧四年,荊湖、川峽、廣南皆增置常平倉。五年,諸路總糴數十八萬三千餘斛,糶二十四萬三千餘斛。



 景祐中,淮南轉運副使吳遵路言:「本路丁口百五十萬,而常平錢粟才四十餘萬,
 歲饑不足以救恤。願自經畫增為二百萬,他毋得移用。」許之。後又詔:天下常平錢粟,三司轉運司皆毋得移用。不數年間,常平積有餘而兵食不足,乃命司農寺出常平錢百萬緡助三司給軍費。久之,移用數多,而蓄藏無幾矣。



 自景祐初畿內饑,詔出常平粟貸中下戶,戶一斛。慶歷中,發京西常平粟振貧民,而聚斂者或增舊價糴粟,欲以市恩;皇祐三年,詔誡之。淮南、兩浙體量安撫陳升之等言:「災傷州軍乞糴常平倉粟,令於元價上量添
 十文、十五文,殊非恤民之意。」乃詔止於元糴價出糶。五年,詔曰:「比者湖北歲儉,發常平以濟饑者,如聞司農寺復督取,豈朝廷振恤意哉?其悉除之。」



 明道二年,詔議復義倉,不果。景祐中,集賢校理王琪請復置:「令五等已上戶,隨夏秋二稅,二斗別輸一升,水旱減稅則免輸。州縣擇便地置倉貯之,領於轉運使。計以一中郡正稅歲入十萬石,則義倉可得五千石,推而廣之,則利博矣。明道中,饑歉,國家欲盡貸饑民則軍食不足,故民有流轉之
 患。是時,兼並之家出粟數千石則補吏,是豈以官爵為輕歟?特愛民濟物,不獲已為之爾。且兼並之家占田常廣,則義倉所入常多;中下之家占田常狹,則義倉所入常少。及水旱振濟,則兼並之家未必待此而濟,中下之民實先受其賜矣。」事下有司會議,議者異同而止。慶歷初,琪復上其議,仁宗納之,命天下立義倉,詔上三等戶輸粟,已而復罷。



 其後賈黯又言:「今天下無事,年穀豐熟,民人安樂,父子相保。一遇水旱,則流離死亡,捐棄道路,
 發倉廩振之則糧不給,課粟富人則力不贍,轉輸千里則不及事,移民就粟則遠近交困。朝廷之臣,郡縣之吏,倉卒不知所出,則民饑而死者過半矣。願放隋制立民社義倉,詔天下州軍遇年穀豐登,立法勸課蓄積,以備兇災。此所謂『樂歲粒米狼戾,多取之而不為虐』者也,況取之以為民耶?」下其說諸路以度可否,以為可行才四路,餘或謂賦稅之外兩重供輸,或謂恐招盜賊,或謂已有常平足以振給,或謂置倉煩擾。



 於是黯復上奏曰:「臣
 嘗判尚書刑部,見天下歲斷死刑多至四千餘人,其間盜賊率十六七,蓋愚民迫於饑寒,因之水旱,枉陷重闢。故臣請復民社義倉,以備兇歲。今諸路所陳,類皆妄議。若謂賦稅之外兩重供輸,則義倉之意,乃教民儲積以備水旱,官為立法,非以自利,行之既久,民必樂輸。若謂恐招盜賊,盜賊利在輕貨,不在粟麥,今鄉村富室有貯粟數萬石者,不聞有劫掠之虞。且盜賊之起,本由貧困。臣建此議,欲使民有貯積,雖遇水旱,不憂乏食,則人人
 自愛而重犯法,此正消除盜賊之原也。若謂有常平足以振給,則常平之設,蓋以準平谷價,使無甚貴甚賤之傷。或遇兇饑,發以振救,既已失其本意,而費又出公帑,今國用頗乏,所蓄不厚。近歲非無常平,小有水旱,輒流離餓莩,起為盜賊,則是常平果不足仰以振給也。若謂置倉廩,斂材木,恐有煩擾,則今州縣修治郵傳驛舍,皆斂於民,豈於義倉獨畏煩擾?人情可與樂成,不可與謀始,願自朝廷斷而行之。」然當時牽於眾論,終不果行。



 嘉
 祐二年,詔天下置廣惠倉。初,天下沒入戶絕田,官自鬻之。樞密使韓琦請留勿鬻,募人耕,收其租別為倉貯之,以給州縣郭內之老幼貧疾不能自存者,領以提點刑獄,歲終具出內之數上之三司。戶不滿萬,留田租千石,萬戶倍之,戶二萬留三千石,三萬留四千石,四萬留五千石,五萬留六千石,七萬留八千石,十萬留萬石。田有餘,則鬻如舊。四年,詔改隸司農寺,州選官二人主出納,歲十月遣官驗視,應受米者書名於籍。自十一月始,三日
 一給,人米一升,幼者半之,次年二月止。有餘乃及諸縣,量大小均給之。其大略如此。治平三年,常平入五十萬一千四十八石,出四十七萬一千一百五十七石。



 熙寧二年,制置三司條例司言:「諸路常平、廣惠倉錢穀,略計貫石可及千五百萬以上,斂散未得其宜,故為利未博。今欲以見在觔斗,遇貴量減市價糶,遇賤量增市價糴,可通融轉運司苗稅及錢斛就便轉易者,亦許兌換。仍以見錢,依陜西青苗錢例,願預借者給之。隨稅輸納
 觔斗,半為夏料,半為秋料,內有請本色或納時價貴願納錢者,皆從其便。如遇災傷,許展至次料豐熟日納。非惟足以待兇荒之患,民既受貸,則兼並之家不得乘新陳不接以邀倍息。又常平、廣惠之物,收藏積滯,必待年儉物貴然後出糶,所及者不過城市游手之人。今通一路有無,貴發賤斂,以廣蓄積,平物價,使農人有以赴時趨事,而兼並不得乘其急。凡此皆以為民,而公家無所利其入,是亦先王散惠興利、以為耕斂補助之意也。欲
 量諸路錢穀多寡,分遣官提舉,每州選通判幕職官一員,典乾轉移出納,仍先自河北、京東、淮南三路施行,俟有緒推之諸路。其廣惠倉除量留給老疾貧窮人外,餘並用常平倉轉移法。」詔可。



 既而條例司又言:「常平、廣惠倉條約,先行於河北、京東、淮南三路,訪問民間多願支貸,乞遍下諸路轉運司施行,當議置提舉官。」時天下常平錢穀見在一千四百萬貫石。詔諸路各置提舉官二員,以朝官為之,管當一員,京官為之,或共置二員,開封
 府界一員,凡四十一人。



 初,神宗既用王安石為參知政事,安石為帝言天下財利所當開闢斂散者,帝然其說,遂創立制置三司條例司。安石因請以著作佐郎編校集賢書箱呂惠卿為制置司檢詳文字,自是專一講求立為新制,欲行青苗之法。蘇轍自大名推官上書,召對,亦除條例司檢詳文字。安石出青苗法示之,轍曰:「以錢貸民,使出息二分,本非為利。然出納之際,吏緣為奸,雖有法不能禁;錢入民手,雖良民不免非理費用;及其納
 錢,雖富民不免違限。如此則鞭笞必用,州縣多事矣。唐劉晏掌國計,未嘗有所假貸。有尤之者,晏曰:『使民僥幸得錢,非國之福;使吏倚法督責,非民之便。吾雖未嘗假貸,而四方豐兇貴賤,知之未嘗逾時。有賤必糴,有貴必糶,以此四方無甚貴甚賤之病,安用貸為?』晏之言,漢常平法耳,公誠能行之,晏之功可立俟也。」安石自此逾月不言青苗。



 會河北轉運司乾當公事王廣廉召議事,廣廉嘗奏乞度僧牒數千道為本錢,於陜西轉運司私行
 青苗法,春散秋斂,與安石意合。至是,請施行之河北,於是安石決意行之,而常平、廣惠倉之法遂變而為青苗矣。蘇轍以議不合罷。而諸路提舉官往往迎合安石之意,務以多散為功。富民不願取,貧者乃欲得之,即令隨戶等高下品配,又令貧富相兼,十人為保首。王廣廉在河北,一等戶給十五千,等而下之,至五等猶給一千,民間喧然不以為便。廣廉入奏謂民皆歡呼感德,然言不便者甚眾。右正言李常、孫覺乞詔有司毋以強民。時提
 舉府界常平事侯叔獻屢督提點府界縣鎮呂景散錢,景以畿縣各有屯兵,歲入課利僅能贍給;又民戶嘗貸糧五十餘萬石,尚悉以聞;今條例司又以買陜西鹽鈔錢五十萬緡為青苗錢給散,恐民力不堪。詔送條例司,召提舉司官至中書戒諭之。王安石言:「若此,諸路必顧望,不敢推行新法,第令條例司指揮。」從之。



 三年,判大名府韓琦言:



 臣準散青苗詔書,務在惠小民,不使兼並乘急以要倍息,而公家無所利其入。今所立條約,乃自鄉
 戶一等而下皆立借錢貫陌,三等以上更許增借,坊郭戶有物業勝質當者亦依鄉戶例支借。且鄉村上等戶並坊郭有物業者,乃從來兼並之家,今令多借之錢,一千令納一千三百,則是官自放錢取息,與初詔絕相違戾。又條約雖禁抑勒,然須得上戶為甲頭以任之,民愚不慮久遠,請時甚易,納時甚難。故自制下以來,上下惶惑,皆謂若不抑散,則上戶必不願請;近下等第與無業客戶雖或願請,必難催納。將來必有行刑督索,及勒乾
 系書手、典押、耆戶長同保均陪之患。



 去歲河朔豐稔,米斗不過七八十錢,若乘時多斂,俟貴而糶,不唯合古制,無失陷,兼民被實惠,亦足收其羨贏。今諸倉方糴而提舉司已亟止之,意在移此糴本盡為青苗錢,則三分之息可為己功,豈暇更恤斯民久遠之患?若謂陜西嘗行其法,官有所得而民以為便,此乃轉運司因軍儲有闕,適自冬及春雨雪及時,麥苗滋盛,定見成熟,行於一時可也。今乃建官置司,以為每歲常行之法,而取利三分,
 豈陜西權宜之比哉?兼初詔且於京東、淮南、河北三路試行,俟有緒方推之他路。今三路未集,而遽盡於諸路置使,非陛下憂民、祖宗惠下之意。乞盡罷提舉官,第委提點刑獄官依常平舊法施行。



 帝袖出琦奏示執政曰:「琦真忠臣,朕始謂可以利民,不意乃害民如此。且坊郭安得青苗,而使者亦強與之?」安石勃然進曰:「茍從其所欲,雖坊郭何害?」因難琦奏,曰:「陛下修常平法以助民,至於收息,亦周公遺法也。如桑弘羊籠天下貨財以奉人
 主私用,乃可謂興利之臣;今抑兼並,振貧弱,置官理財,非所以佐私欲,安可謂興利之臣乎?」曾公亮、陳升之皆言坊郭不當俵錢,與安石論難久之而罷。帝終以琦說為疑,安石遂稱疾不出。



 帝諭執政罷青苗法,公亮、升之欲即奉詔,趙抃獨欲俟安石出自罷之,連日不決。帝更以為疑,因令呂惠卿諭旨起安石,安石入謝。既視事,志氣愈悍,面責公亮等,由是持新法益堅。詔以琦奏付制置條例司,條例司疏列琦奏而辨析其不然。琦復上疏
 曰:



 「制置司多刪去臣元奏要語,唯舉大概,用偏辭曲難,及引《周禮》「國服為息」之說,文其謬妄,上以欺罔聖聽,下以愚弄天下。臣竊以為周公立太平之法,必無剝民取利之理,但漢儒解釋或有異同。《周禮》「園廛二十而稅一,唯漆林之徵二十而五」,鄭康成乃約此法,謂:「從官貸錢若受園廛之地,貸萬錢者出息五百。」賈公彥廣其說,謂:「如此則近郊十一者,萬錢期出息一千,遠郊二十而三者,萬錢期出息一千五百,甸、稍、縣、都之民,萬錢期出息
 二千。」如此,則須漆林之戶取貸,方出息二千五百,當時未必如此。今放青苗錢,凡春貸十千,半年之內便令納利二千,秋再放十千,至歲終又令納利二千,則是貸萬錢者,不問遠近,歲令出息四千。《周禮》至遠之地止出息二千,今青苗取息過《周禮》一倍,制置司言比《同禮》取息已不為多,是欺罔聖聽,且謂天下之人不能辨也。



 且古今異宜,《周禮》所載有不可施於今者,其事非一。若謂泉府一職今可施行,則制置司何獨舉注疏貸錢取息一
 事,以詆天下之公言哉?康成又注云:「王莽時貸以治產業者,但計所贏受息,無過歲什一。」公彥疏云:「莽時雖計本多少為定,及其催科,唯所贏多少。假令萬錢歲贏萬錢催一千,贏五千催五百,餘皆據利催什一。」若贏錢更少,則納息更薄,比今青苗取利尤為寬少。而王莽之外,上自兩漢,下及有唐,更不聞有貸錢取利之法。今制置司遇堯、舜之主,不以二帝、三王之道上裨聖政,而貸錢取利更過莽時,此天下不得不指以為非,而老臣不可
 以不辨也。



 況今天下田稅已重,固非《周禮》什一之法,更有農具、牛皮、鹽曲、□奚錢之類,凡十餘目,謂之雜錢。每夏秋起納,官中更以綢絹觔斗低估,令民以此雜錢折納。又歲散官鹽與民,謂之蠶鹽,折納絹帛。更有預買、和買綢絹,如此之類,不可悉舉,皆《周禮》田稅什一之外加斂之物,取利已厚,傷農已深,奈何又引《周禮》「國服為息」之說,謂放青苗錢取利乃周公太平已試之法?此則誣污聖典,蔽惑睿明,老臣得不太息而慟哭也!



 制置司又謂
 常平舊法亦糶與坊郭之人。坊郭有物力戶未嘗零糴常平倉觔斗,此蓋欲多借錢與坊郭有業之人,以望收利之多,妄稱《周禮》以為無都邑鄙野之限,以文其曲說,唯陛下詳之。」



 樞密使文彥博亦數言不便,帝曰:「吾遣二中使親問民間,皆云甚便。」彥博曰:「韓琦三朝宰相,不信,而信二宦者乎?」先是,王安石陰結入內副都知張若水、押班藍元震,帝因使二人潛察府界俵錢事,還言民皆情願,無抑配者,故帝益信之。初,群臣進讀邇英畢,帝問:「
 朝廷每更一事,舉朝洶洶,何也?」司馬光曰:「青苗出息,平民為之,尚能以蠶食下戶至饑寒流離,況縣官法度之威乎?」呂惠卿曰:「青苗法願則取之,不願不強也。」光曰:「愚民知取債之利,不知還債之害,非獨縣官不強,富民亦不強也。」帝曰:「陜西行之久,民不以為病。」光曰:「臣陜西人也,見其病不見其利。朝廷初不許,有司尚能以病民,況法許之乎!」及拜官樞密副使,光上章力辭至六七,曰:「帝誠能罷制置條例司,追還提舉官,不行青苗、助役等法,
 雖不用臣,臣受賜多矣。不然,終不敢受命。」竟出知永興軍。



 當是時,爭青苗錢者甚眾,翰林學士範鎮言:「陛下初詔云公家無所利其入,今提舉司以戶等給錢,皆令出三分之息,物議紛紜,皆云自古未有天子開課場者。民雖至愚,不可不畏。」後以言不行致仕。臺諫官呂公著、孫覺、李常、張戩、程顥等皆以論青苗罷黜。知亳州富弼、知青州歐陽修繼韓琦論青苗之害,且持之不行,亦坐移鎮。知陳留縣姜潛之官才數月,青苗令下,潛即榜於縣
 門,又移之鄉村,各三日無人至,遂撤榜付吏曰:「民不願矣!」府、寺疑潛壅令,使其屬按驗,無違令者。潛知不免,即移疾去。



 知山陰縣陳舜俞不肯奉行,移狀自劾曰:「方今小民匱乏,願貸之人往往有之。譬如孺子見飴蜜,孰不染指爭食?然父母疾止之,恐其積甘足以生病。故耆老戒其鄉黨,父兄誨其子弟,未嘗不以貸貰為不善治生。今乃官自出舉,誘以便利,督以威刑,非王道之舉也。況正月放夏料,五月放秋料,而所斂亦在當月,百姓得錢
 便出息輸納,實無所利。是使民一取青苗錢,終身以及世世一歲嘗兩輸息錢,乃別為一賦以弊生民也。」坐謫南康軍鹽酒稅。陜西轉運副使陳繹止環、慶等六州毋散青苗錢,且留常平倉物以備用,條例司劾其罪,詔釋之。五月,制置三司條例司罷歸中書,以常平新法付司農寺,命集賢校理呂惠卿同判寺,兼領田役水利。七年,帝患俵常平官吏多違法,王安石請縣專置一主簿,主給納役錢及常平,不過五百員,費錢三十萬貫耳。從之。



 帝以久旱為憂,翰林學士承旨韓維言:「畿縣近督青苗甚急,往往鞭撻取足,民至伐桑為薪以易錢。旱災之際,重罹此苦。」帝頗感悟。太皇太后亦嘗為帝言:「聞民間甚苦青田、助役錢,盍罷之!」會百姓流離,帝憂見顏色,益疑新法不便,欲罷之。安石不悅,屢求去,四月,出知江寧府。然安石薦韓絳代相,仍以呂惠卿佐之,於安石所為遵守不變。既而詔諸路常平錢穀常留一半外,方得給散。兩經倚閣常平錢人力,不得支借。民間非時闕乏,許以
 物產為抵,依常平限輸納。當輸錢而願輸穀若金帛者,官立中價示民。物不盡其錢,足以錢;錢不盡其物者,還其餘直。又聽民以金帛易穀,而有司少加金帛之直。六年,戶部言:「準詔諸路常平可酌三年斂散中數,取一年為格,歲終較其增虧。今以錢銀穀帛貫、石、匹、兩定年額:散一千一百三萬七千七百七十二,斂一千三百九十六萬五千四百五十九。比元豐三年散增二百一十四萬八千三百四十二,斂增一百三萬四千九百六十三;
 四年散增二百七十九萬九千九百六十四,斂虧一百九十八萬六千五百一十五。」詔三年四年散多斂少及散斂俱少之處,戶部下提舉司具析以聞。



 十年,詔開封府界先自豐稔畿縣立義倉法。明年,提點府界諸縣鎮公事蔡承禧言:「義倉之法,以二石而輸一斗,至為輕矣。乞今年夏稅之始,悉令舉行。」詔可,仍以義倉隸提舉司。京東西、淮南、河東、陜西路義倉以今年秋料為始,民輸稅不及鬥免輸,頒其法於川峽四路。元豐二年,詔威、茂、
 黎三州罷行義倉法,以夷夏雜居,歲賦不多故也。八年,並罷諸路義倉。



 元祐元年,詔:「提舉官累年積蓄錢穀財物,盡樁作常平錢物,委提點刑獄交管,依舊常平倉法行之。罷各縣專置主簿。」四月,再立常平錢穀給斂出息之法,限二月或正月以散及一半為額,民間絲麥豐熟,隨夏稅先納所輸之半,願伴納者止出息一分。左司諫王巖叟、監察御史上官均、右正言王覿、右司諫蘇轍、御史中丞劉摯交章論復行青苗之非。八月,司馬光奏:「先
 朝散青苗,本為利民,並取情願。後提舉官速要見功,務求多散,或舉縣追呼,或排門抄扎;亦有無賴子弟謾昧尊長,錢不入家;亦有他人冒名詐請,莫知為誰,及至追催,皆歸本戶。今朝廷深知其弊,故悉罷提舉官,不復立額考校,訪聞人情安便。欲下諸路提點刑獄,申嚴州縣抑配之禁。」詔從之。



 中書舍人蘇軾不書錄黃,奏曰:「熙寧之法,未嘗不禁抑配,而其害至此。民家量入為出,雖貧亦足,若令分外得錢,則費用自廣。況子弟欺謾父兄,人
 戶冒名詐請,似此本非抑配。臣謂以散及一半為額,與熙寧無異。今許人願請,未免設法罔民,使快一時非理之用,而不慮後日催納之患。二者皆非良法,相去無幾。今已行常平糶糴之法,惠民之外,官亦稍利,何用二分之息,以賈無窮之怨?」於是王巖叟、蘇轍、朱光庭、王覿等復言:「臣等屢有封事,乞罷青苗,皆不蒙付外。願盡付三省,公議得失。」初,同知樞密院范純仁以國用不足,建請復散青苗錢,四月之詔,蓋純仁意也。時司馬光以疾在
 告,已而臺諫皆言其非,不報。光尋奏乞約束州縣抑配,蘇軾又繳奏,乞盡罷之。光始大悟,遂力疾入對。尋詔:「常平錢穀,止令州縣依舊法趁時糴糶,青苗錢更不支俵。除舊欠二分之息,元支本錢驗見欠多少,分料次隨二稅輸納。」



 紹聖元年,詔除廣南東、西路外,並復置義倉,自來歲始,放稅二分已上免輸,所貯專充振濟,輒移用者論如法。二年,戶部尚書蔡京首言:「承詔措置財利,乞檢會熙、豐青苗條約,參酌增損,立為定制。」淮南轉運司副
 使莊公嶽謂:「自元祐罷提舉官後,錢穀為他司侵借,所存無幾。欲乞追還給散,隨夏秋稅償納,勿立定額,自無抑民失財之患。」奉議郎鄭僅、朝奉郎郭時亮、承議郎許幾董遵等皆言:「青苗最為便民,願戒抑配,止收一分之息。」詔並送詳定重修敕令所。三年,舊欠常平錢穀人戶,仍許請給。



 宣和五年,令州縣歲散常平錢穀畢,即揭示請人名數,逾月斂之,庶革偽冒之弊。先是,諸路災傷,截撥上供年額米斛數多,致闕中都歲計,令京東、江南、兩
 浙、荊湖路義倉穀各留三分,餘並起發赴京,補還截撥之數。六年,詔罷之。



 高宗紹興元年,並提舉常平司於提刑司。明年,以臣僚言復常平官,講補肋之政以廣儲蓄。九年,用宗正丞鄭鬲言,以常平錢於民輸賦未畢之時,悉數和糴。二十八年,以趙令詪請,糶州縣義倉米之陳腐者。



 孝宗隆興二年,遣司農少卿陳良弼點檢浙東常平等倉。乾道六年,知衢州胡堅奏廣糴常平。福建轉運副使沉樞奏,水旱州郡請留轉運司和糴米以續常平,
 上即為之施行。八年,戶部侍郎楊倓奏:「義倉在法夏秋正稅鬥輸五合,不及鬥者免輸,凡豐熟縣九分以上即輸一升。令諸路州縣歲收苗米六百餘萬石,其合收義倉米數不少,間有災傷,支給不多。訪聞諸州軍皆擅用,請稽之。」



 寧宗慶元元年,詔戶部右曹專領義倉。十一年,臣僚言:「紹興初,臺臣嘗請通一縣之數,截留下戶苗米,輸之於縣,別儲以備振濟,使窮民不至于艱食;惟負郭義倉,則就州輸送。至於屬縣之義倉,則令、丞同主之,每
 歲終,令、丞合諸鄉所入之數上之守、貳,守、貳合諸縣所入之數上之提舉常平,提舉常平合一道之數上之朝廷,考其盈虧,以議殿最。」從之。



 寶慶三年,侍御史李知孝言:「郡縣素無蓄積,緩急止仰朝廷,非立法本意。曩淮東總領岳珂任江東轉運判官,以所積經常錢糴米五萬石,樁留江東九郡,以時濟、糴,諸郡皆蒙其利。其後史彌忠知饒州,趙彥悈知廣德軍,皆自積錢糴米五千石。以是推之,監司、州郡茍能節用愛民,即有贏羨。若立之規
 繩,加以黜陟,所糴至萬石者旌擢,其不收糴與擾民及不實者鐫罰,庶幾郡縣趨事,蓄積歲增,實為經久之利。」有旨從之。



 景定元年九月,赦曰:「諸路已糶義米價錢,州郡以低價抑令上戶補糴,正稅逃閣,義米用虧,常平司責縣道陪納,縣道遂敷吏貼、保正長、攬戶等人均納。自今視時收糴,見系吏貼等人陪納之錢並與除放。」五年,監察御史程元嶽奏:「隨粳帶義,法也。今粳糯帶義之外,又有所謂外義焉者,絹、綢、豆也,豈有絹、綢、豆而可加之
 義乎?縱使違法加義,則絹加絹,綢加綢,豆加豆,猶可言也;州縣一意椎剝,一切理苗而加一分之義,甚者赦恩已蠲二稅,義米依舊追索。貧民下戶所欠不過升合,星火追呼,費用不知幾百倍。破家蕩產,鬻妻子,怨嗟之聲,有不忍聞。望嚴督監司,止許以粳帶義,其餘盡罷。其有循習病民者重其罰。」從之。咸淳二年,以諸路景定三年以前常平義倉米二百餘萬石,減時直糶之。



\end{pinyinscope}