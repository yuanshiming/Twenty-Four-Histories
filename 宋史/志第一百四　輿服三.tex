\article{志第一百四 輿服三}

\begin{pinyinscope}

 天子之服皇太子附后妃之服命婦附



 天子之服,一曰大裘冕,二曰袞冕,三曰通天冠,絳紗袍,四曰履袍,五曰衫袍,六曰窄袍,天子祀享、朝會、親耕及親事、燕居之服也,七曰御閱服,天子之戎服也。中興之
 後則有之。



 大裘之制。神宗元豐四年,詳定郊廟奉祀禮文所言:「《周禮·司裘》『掌為大裘,以供王祀天之服』;《司服》『王祀昊天上帝,則服大裘而冕,祀五帝亦如之。享先王則袞冕』。而《禮記》云:『郊祭之日,王被袞以象天,戴冕璪十有二旒,則天數也。』王肅據《家語》,以為臨燔柴,脫袞冕,著大裘。則是《禮記》被袞,與《周禮》大裘,郊祀並用二服,事不相戾,但服之有先後耳。是以《開寶通禮》:皇帝服袞冕出赴行宮,祀日,
 服袞冕至大次;質明,改服大裘而冕出次。蓋袞冕盛服而文之備者,故於郊之前期被之,以至大次。既臨燔柴,則脫袞冕服裘,以明天道至質,故被裘以體之。今儀注,車駕赴青城,服通天冠、絳紗袍。祀之日,乃服靴袍至大次,服袞冕臨祭,非尚質之義。乞並依《開寶通禮》。」詔詳定所參議。



 又言:「臣等詳大裘之制,本以尚質,而後世反以尚文,故冕之飾大為不經。而禮書所載,上有垂旒加飾,又異『大裘不裼』之說。今參考諸說,大裘冕無旒,廣八寸,
 長一尺六寸,前圜後方,前低寸二分,玄表朱裏,以繒為之。玉笄以朱組為紘,玉瑱以玄紞垂之。為裘以黑羔皮,領袖以黑繒,纁裳朱紘而無章飾。佩白玉,玄組綬。革帶,博二寸,玉鉤,以佩紘屬之。素帶,朱裏,絳純其外,上朱下綠。白紗中單,皂領,青褾、□、裾。朱襪,赤舄,黑絇、繶、純。乞下所屬制造。其當暑奉祠之服,乞降梁陸瑋議以黑繒為裘,及《唐輿服志》以黑羔皮為緣。」詔重詳定。



 光祿寺丞、集賢校理陸佃言:「臣詳冕服有六。《周官》弁師云『掌王之
 五冕』,則大裘與袞同冕。故《禮記》云『郊之日,王被袞以象天』。又曰『服之襲也,充,美也』;『禮不盛,服不充,故大裘不裼』。此明王服大裘,以袞衣襲之也。先儒或謂周祀天地皆服大裘,而大裘之冕無旒,非是。蓋古者裘不徒服,其上必皆有衣,故曰『緇衣羔裘』,『黃衣狐裘』,『素衣麑裘』。如郊祀徒服大裘,則是表裘以見天地。表裘不入公門,而乃欲以見天地,可乎?且先王之服,冬裘夏葛以適寒暑,未有能易之者也。郊祀天地,有裘無袞,則夏祀赤帝與至日
 祭地祇,亦將被裘乎?然則王者冬祀昊天上帝,中裘而表袞,明矣。至於夏祀天神地祇,則去裘服袞,以順時序。《周官》曰『凡四時之祭祀,以宜服之』,明夏不必衣裘也。或曰,祭天尚質,故徒服大裘,被袞則非尚質。臣以為尚質者,明有所尚而已,不皆用質也。今欲冬至禋祀昊天上帝,服裘被袞,其餘祀天及祀地祇,並請服袞去裘,各以其宜服之。」



 於是詳定所言:「裘不可徒服。《禮記》曰『大裘不裼』,則襲可知,所謂大裘之襲者,袞也,與袞同冕。伏請冬
 祀昊天與黑帝,皆服大裘,被以袞。其餘非冬祀天及夏至祭地,則皆服袞。」



 六年,尚書禮部言:「經有大裘而無其制,近世所為,惟梁、隋、唐為可考。請緣隋制,以黑羔皮為裘,黑繒為領袖及里、緣,袂廣可運肘,長可蔽膝。按皇侃說,祭服之下有袍繭,袍繭之下有中衣。朝服,裼衣之下有裘,裘之下有中衣。然則今之親郊,中單當在大裘之下,其袂之廣狹,衣之長短,皆當如裘。伏乞改制。」於是神宗始服大裘,而加袞冕焉。



 哲宗元祐元年,禮部言:「元豐所造
 大裘,雖用黑羔皮,乃作短袍樣,襲於袞衣之下,仍與袞服同冕,未合典禮。」下禮部、太常寺共議。上官均、吳安詩、常安民、劉唐老、襲原、姚勉請依元豐新禮,丁騭請循祖宗故事,王愈請仿唐制,朱光庭、周秩請以玄衣襲裘。獨禮部員外郎何洵直在元豐中嘗預詳定,以陸佃所議有可疑者八:



 按《周禮·節服氏》「掌祭祀朝覲,袞冕六人,惟王之太常」;「郊祀,裘冕二人」。既云袞冕,又雲裘冕,是袞與裘各有冕。乃雲裘與袞同冕,當以袞襲之。裘既無冕,又
 襲於袞,中裘而表袞,何以示裘袞之別哉?古人雖質,不應以裘為夏服,蓋冬用大裘,當暑則以同色繒為之。《記》曰:「郊祭之日,王被袞以象天。」若謂裘上被袞,以被為襲,則《家語》亦有「被裘象天」之文。諸儒或言「臨燔柴,脫袞冕,著大裘」,或云「脫裘服袞」,蓋裘袞無同冕兼服之理。今乃以二服合為一,可乎?



 且大裘,天子吉服之最上,若大圭、大路之比,是裘之在表者。《記》曰:「大裘不裼。」說者曰,無別衣以裼之,蓋他服之裘褻,故表裘不入公門。事天以報
 本復始,故露質見素,不為表襮,而冕亦無旒,何必假他衣以藩飾之乎?凡裘上有衣謂之裼,裼上有衣謂之襲,襲者,裘上重二衣也。大裘本不裼,《鄭志》乃云:「裘上有玄衣,與裘同色。」蓋趙商之徒,附會為說,不與經合。襲之為義,本出於重沓,非一衣也。



 古者齋祭異冠,齋服降祭服一等。祀昊天上帝、五帝,以裘冕祭,則袞冕齋。故鄭氏云:「王齋服袞冕。」是袞冕者,祀天之齋服也。唐《開元》及《開寶禮》始以袞冕為齋服,裘冕為祭服,兼與張融「臨燔柴脫
 袞服裘」之義合。請從唐制,兼改制大裘,以黑繒為之。



 佃復破其說曰:



 夫大裘而冕,謂之裘冕,非大裘而冕,謂之袞冕。則裘冕必服袞,袞冕不必服裘。今特言裘冕者,主冬至言之。《周禮·司裘》:「掌為大裘,以供王祀天之服。」則祀地不服大裘,以夏日至,不可服裘故也。今謂大裘當暑,以同色繒為之,尤不經見。



 兼裼襲,一衣而已,初無重沓之義。被裘而覆之則曰襲,袒而露裘之美則曰裼。所謂「大裘不裼」,則非袞而何?《玉藻》曰:「禮不盛,服不充,故大裘
 不裼。」則明不裼而襲也,充,美也。鄭氏謂大裘之上有玄衣,雖不知覆裘以袞,然尚知大裘不可徒服,必有玄衣以覆之。《玉藻》有尸襲之義。《周禮》裘冕注云:「裘冕者,從尸服也。」夫尸服大裘而襲,則王服大裘而襲可知。且裘不可以徒服,故被以袞,豈借袞以為飾哉?



 今謂祭天用袞冕為齋服,裘冕為祭服,此乃襲先儒之謬誤。後漢顯宗初服日、月、星辰十二章,以祀天地。自魏以來,皆用袞服。則漢、魏祭天,嘗服袞矣,雖無大裘,未能盡合於禮,固未
 嘗有表裘而祭者也。且裘,內服也,與袍同。袍褻矣,而欲襌以祭天,以明示質,是欲衣義衣以見上帝也。洵直復欲為大裘之裳,纁色而無章飾。夫裘安得有裳哉?請從先帝所志。



 其後詔如洵直議,去黑羔皮而以黑繒制焉。



 政和議禮局上:大裘,青表纁里,黑羔皮為領、褾、□,朱裳,被以袞服。冬至祀昊天上帝服之,立冬祀黑帝、立冬後祭神州地祇亦如之。中興之後,無有存者。



 紹興十三年,禮部侍郎王賞等言:「郊祀大禮,合依《禮經》,皇帝服大裘被
 袞行禮。據元豐詳定郊廟禮文,何洵直議以黑繒創作大裘如袞,惟領袖用黑羔。乞如洵直議。」詔有司如祖宗舊制,以羔制之。禮部又言:「關西羊羔,系天生黑色。今有司涅白羔為之,不中禮制,不如權以繒代。又元祐中,有司欲為大裘,度用百羔。哲宗以為害物,遂用黑繒。請依太常所言。」從之。遂以袞襲裘,冕亦十二旒焉。



 袞冕之制。宋初因五代之舊,天子之服有袞冕,廣一尺二寸,長二尺四寸,前後十二旒,二纊,並貫真珠。又有翠
 旒十二,碧鳳御之,在珠旒外。冕版以龍鱗錦表,上綴玉為七星,旁施琥珀瓶、犀瓶各二十四,周綴金絲網,鈿以真珠、雜寶玉,加紫雲白鶴錦里。四柱飾以七寶,紅綾里。金飾玉簪導,紅絲絳組帶。亦謂之平天冠。袞服青色,日、月、星、山、龍、雉、虎蜼七章。紅裙,藻、火、粉米、黼、黻五章。紅蔽膝,升龍二並織成,間以雲朵,飾以金鈒花鈿窠,裝以真珠、琥珀、雜寶玉。紅羅襦裙,繡五章,青褾、□、裾。六採綬一,小綬三,結玉環三。素大帶朱裏,青羅四神帶二,繡四神
 盤結。



 綬帶飾並同袞服。



 白羅中單,青羅抹帶,紅羅勒帛。鹿盧玉具劍,玉鏢首,鏤白玉雙佩,金飾貫真珠。金龍鳳革帶,紅襪赤舄,金鈒花,四神玉鼻。祭天地宗廟,朝太清宮、饗玉清昭應宮景靈宮、受冊尊號、元日受朝、冊皇太子則服之。



 太祖建隆元年,太常禮院言:「準少府監牒,請具袞龍衣、絳紗袍、通天冠制度令式。袞冕,垂白珠十有二旒,以組為纓,色如其綬,黈纊充耳,玉簪導。玄衣纁裳,十二章:八章在衣,日、月、星辰、山、龍、華蟲、火、宗彞;四章在裳,藻、粉
 米、黼、黻。衣褾領如上,為升龍,皆織就為之。山、龍以下,每章一行,重以為等,每行十二。白紗中單,黼領,青褾、□、裾。蔽膝加龍、山、火三章。革帶,玉鉤。大帶,素帶朱裏,紕其外,上朱下綠,紐約用組。鹿盧玉具劍,大珠鏢首,白玉雙佩,玄組。雙大綬六採,玄、黃、赤、白、縹、綠,純玄質,長二丈四尺五寸,首廣一尺。小雙綬長二尺六寸,色同大綬,而首半之,間施三玉環。朱襪赤舄,加金飾。」詔可。



 二年,太子詹事尹拙、工部尚書竇儀議:「謹按《周禮》:『弁師掌王之五冕,
 皆玄冕朱裏延紐,五採繅,十有二就,皆五採玉十有二,玉笄朱紘。諸侯之繅旒九就,鈱玉三採,其餘如王之事,繅斿皆就,玉瑱、玉笄。』疏云:『王不言玉瑱,於此言之者,王與諸侯互相見為義。是以王言玄冕、朱裏延紐及朱紘,明諸侯亦有之。諸公言玉瑱,明王亦有之。』詳此經、疏之文,則是本有充耳。今請令君臣袞冕以下並畫充耳,以合正文。」從之。



 乾德元年閏十二月,少府監楊格、少監王處訥等上新造皇帝冠冕。先是,郊祀冠冕,多飾以珠玉,帝以華
 而且重,故命改制之。



 仁宗景祐二年,又以帝後及群臣冠服,多沿唐舊而循用之,久則有司浸為繁文,以失法度。詔入內內侍省、御藥院與太常禮院詳典故,造冠冕,蠲減珍華,務從簡約,俾圖以進。續詔通天冠、絳紗袍更不修制。由是改制袞冕。天版元闊一尺二寸,長二尺四寸,今制廣八寸,長一尺六寸。減翠旒並鳳子,前後二十四珠旒並合典制。天板頂上,元織成龍鱗錦為表,紫雲白鶴錦為里,今制青羅為表,採畫出龍鱗,紅羅為里,採
 畫出紫雲白鶴。所有犀瓶、琥珀瓶各二十四,今減不用。金絲結網子上,舊有金絲結龍八,今減四,亦減絲令細。天板四面花墜子、素墜子依舊,減輕造。冠身並天柱,元織成龍鱗錦,今用青羅,採畫出龍鱗;金輪等七寶,元真玉碾成,今更不用,如補空卻,以雲龍細窠。分旒玉鉤二,今減去之。天河帶、組帶、款慢帶依舊,減輕造。納言,元用玉制,今用青羅,採畫出龍鱗錦。金棱上棱道,依舊用金,即減輕制。黈纊,玉簪。袞服八章,日、月、星辰、山、龍、華蟲、火、
 宗彞,青羅身,紅羅□,繡造。所有雲子,相度稀稠補空,更不用細窠,亦不使真珠裝綴。中單,依舊皂白制造。裙用紅羅,繡出藻、粉米、黼、黻,周回花樣仍舊,減稀制之。蔽膝用紅羅,繡升龍二,雲子補空,減稀制之,周回依舊,細窠不用。六採綬依舊,減絲織造。所有玉環亦減輕。帶頭金葉減去,用銷金。四神帶不用。劍、佩、梁、帶、襪、舄並依舊。



 嘉祐元年,王洙奏:「天子法服,冕旒形度重大,華飾稍繁,願集禮官參定。」詔禮院詳典禮上聞,而禮院繪圖以進。因
 敕御藥院更造,其後,冕服稍增侈如故。



 英宗治平二年,知太常禮院李育奏曰:



 郊廟之祭,本尚純質,袞冕之飾,皆存法象,非事繁侈、重奇玩也。冕則以《周官》為本,凡十二旒,間以採玉,加以紘、綖、笄、瑱之飾。袞則以《虞書》為始,凡十二章,首以辰象,別以衣裳繪繡之採。東漢至唐,史官名儒,記述前制,皆無珠翠、犀寶之飾,何則?鷸羽蚌胎,非法服所用;琥珀犀瓶,非至尊所冠;龍錦七星,已列採章之內;紫雲白鶴,近出道家之語,豈被袞戴璪、象
 天則數之義哉!自大裘之廢,顓用袞冕,古樸稍去,而法度尚存。夫明水大羹,不可以眾味和;《雲門》《咸池》,不可以新聲間;袞冕之服,不宜以珍怪累也。若魏明之用珊瑚,江左之用翡翠,侈靡衰播之餘,豈足為聖朝道哉!



 且太祖建隆元年少府監所造冕服,及二年博士聶崇義所進《三禮圖》,嘗詔尹拙、竇儀參校之,皆仿虞、周、漢、唐之舊。至四年冬服之,合祭天地於圜丘,用此制也。太宗亦嘗命少府制於禁中,不聞改作。及真宗封泰山,禮官請服袞冕。
 帝曰:「前王服羔裘,尚質也。今則無羔裘而有袞冕,可從近制。」是豈有意於繁飾哉。蓋後之有司,率意妄增,未嘗確議,遂相循而用。故仁宗嘗詔禮官章得像等詳議之,其所減過半,然不經之飾,重者多去,輕者尚存,不能盡如詔書之意。故至和三年,王洙復議去繁飾,禮官畫圖以獻,漸還古禮,而有司所造,復如景祐之前。



 又按《開寶通禮》及《衣服令》,冕服皆有定法,悉無寶錦之飾。夫太祖、太宗富有四海,豈乏寶玩,顧不可施之郊廟也。臣竊謂,
 陛下肇祀天地,躬饗祖檷,服周之冕,觀古之象,願復先王之制,祖宗之法。其袞冕之服,及□、綬、佩、舄之類,與《通禮》、《衣服令》、《三禮圖》制度不同者,宜悉改正。



 詔太常禮院、少府參定,遂合奏曰:



 古者冕服之用,郊廟殊制。唐興,天子之服有二等,而大裘尚存。顯慶初,長孫無忌等採《郊特牲》之說,獻議廢大裘。自是郊廟之祭,一用袞冕,然旒章之數,止以十二為節,亦未聞有餘飾也。國朝冕服,雖仿古制,然增以珍異巧縟,前世所未嘗有。夫國之大事,
 莫大於祀,而祭服違經,非以肅祀容、尊神明也。臣等以謂宜如育言,參酌《通禮》、《衣服令》、《三禮圖》及景祐三年減定之制,一切改造之。



 孔子曰:「麻冕,禮也,今也純儉,吾從眾。」純者,絲也,變麻用絲,蓋已久矣。則冕服之制,宜依舊以羅為之。冕廣一尺二寸,長二尺二寸,約以景表尺,前圓後方,黝上朱下,以金飾版側,以白玉珠為旒,貫之以五採絲繩。前後各十二旒,旒各十二珠,相去一寸,長二尺。朱絲組為纓,黈纊充耳,金飾玉簪導。青衣纁裳,十二
 章:八章繪之於衣,日、月、星辰、山、龍、華蟲、火、宗彞也;四章繡之於裳,藻、粉米、黼、黻也。錦龍褾、領,織為升龍。山、龍而下,一章為一行,重以為等,行十二。別制大帶,素表朱裏,朱綠終闢。□、紱、舄,大小綬,亦去珠玉、鈿窠、琥珀、玻璃之飾。其中單、革帶、玉具劍、玉佩、朱襪之制,已中禮令,無復改為,則法服有稽,祭禮增重。



 復詔禮院再詳以聞。而內侍省奏謂:「景祐中已裁定,可因而用也。」從之。



 神宗元豐元年,詳定郊廟禮文所言:



 凡冕版廣八寸,長尺六寸,
 與古制相合,更不復議。今取少府監進樣,如以青羅為表,紅羅為里,則非《弁師》所謂「玄冕朱裏』者也。上用金棱天板,四周金絲結網,兩旁用真珠、花素墜之類,皆不應禮。伏請改用朱組為紘,玉笄、玉瑱,以玄紞垂瑱,以五採玉貫於五色藻為旒,以青、赤、黃、白、黑五色備為一玉,每一玉長一寸,前後二十四旒,垂而齊肩,以合孔子所謂純儉之義。



 又古者祭服、朝服之裳,皆前三幅,後四幅,前為陽以象奇,後為陰以象偶。惟深衣、中襌之屬連衣裳,而
 裳復不殊前後,然以六幅交解為十二幅,像十二月。其制作莫不有法,故謂之法服。今少府監袞服,其裳乃以八幅為之,不殊前後,有違古義。伏請改正祭服之裳,以七幅為之,殊其前後。以今太常周尺度之,幅廣二尺二寸,每幅兩旁各縫殺一寸,謂之削幅,腰間闢積無數。裳側有純,謂之綼;裳下有純,謂之金易。綼、緆之廣各寸半,表裏合為三寸。群臣祭服之裳,仿此。從之。



 政和議禮局更上皇帝冕服之制:冕版廣八寸,長一尺六寸,前高八寸
 五分,後高九寸五分。青表朱裏,前後各十有二旒,五採藻十有二就,就間相去一寸。青碧錦織成天河帶,長一丈二尺,廣二寸。朱絲組帶為纓,黈纊充耳,金飾玉簪導,長一尺二寸。袞服,青衣八章,繪日、月、星辰、山、龍、華蟲、火、宗彞;纁裳四章,繡藻、粉米、黼、黻。蔽膝隨裳色,繡升龍二。白羅中單,皂褾、□,紅羅勒帛,青羅襪帶。緋白羅大帶,革帶,白玉雙佩。大綬六採,赤、黃、黑、白、縹、綠,小綬三色,如大綬,間施玉環三。朱襪,赤舄,緣以黃羅。



 中興仍舊制,延,以
 羅衣木,玄表朱裏,長尺有六寸,前低一寸二分,四旁緣以金,覆於卷武之上,繅以五色絲貫五色玉,前後各十二,凡用二百八十有八。玉笄,充耳用黃綿,紘以朱組,以其一屬於左笄上垂下,又屈而屬於右笄,系之而垂其餘。玄衣,八章,升龍於山,繪。裳纁,四章,繡。幅前三後四,斷而不屬,兩旁殺縫,腰闢積,綼緆之廣皆如舊。大帶以緋白羅合而紩之,以朱綠飾其側,上朱下綠,其束處以組為紐約,下垂三尺。通天冠、絳紗袍亦如之。白羅中單,領、
 褾、襈以黻,服裘則以皂。絳紗袍則衣用白紗,領、褾、襈以朱。綬大小各一,大綬織以六採,青、黃、黑、白、縹、綠,下垂青絲網,上有結,垂玉環三;小綬制如大綬,惟三色。大裘、絳紗袍皆用之。革帶,博二寸,革為里,緋羅為表,飾以玉銙,鈕以下鉤。通天冠、絳紗袍亦用之。韍從裳色,上有紕,下有純,去上五寸,繪以山、龍、火,上接革帶系之。佩有衡,有琚瑀,有沖牙,系於革帶,左右各一。上設衡,衡下垂三帶,貫以蠙珠。次則中有金獸面,兩旁夾以雙璜,又次設
 琚瑀。下則沖牙居中央,兩旁有玉滴子,行則擊牙而有聲。舄有絇,有純,有繶,有綦,以緋羅為之,首加金飾。服通天冠、絳紗袍則用黑舄,以烏皮為之。常服則用白舄,以絲為之。襪,羅表繒里,施靿著綦以系之,赤舄以朱,黑舄以白,白舄同。



 通天冠。二十四梁,加金博山,附蟬十二,高廣各一尺。青表朱裏,首施珠翠,黑介幘,組纓翠緌,玉犀簪導。絳紗袍,以織成雲龍紅金條紗為之,紅裡,皂褾、□、裾,絳紗裙,蔽
 膝如袍飾,並皂褾、□。白紗中單,朱領、褾、□、裾。白羅方心曲領。白襪,黑舄,佩綬如袞。大祭祀致齋、正旦冬至五月朔大朝會、大冊命、親耕籍田皆服之。



 仁宗天聖二年,南郊,禮儀使李維言:「通天冠上一字,準敕回避。」詔改承天冠。中興之制,冠高九寸,服用並同。



 乾道九年,又用履袍。袍以絳羅為之,折上巾,通犀金玉帶。系履,則曰履袍;服靴,則曰靴袍。履、靴皆用黑革。四孟朝獻景靈宮、郊祀、明堂,詣宮、宿廟、進胙,上壽兩宮及端門肆赦,並服之。大禮
 畢還宮,乘平輦,服亦如之。若大輦,則服通天、絳紗如常儀。



 衫袍。唐因隋制,天子常服赤黃、淺黃袍衫,折上巾,九還帶,六合靴。宋因之,有赭黃、淡黃袍衫,玉裝紅束帶,皂文鞞,大宴則服之。又有赭黃、淡黃衣癸袍,紅衫袍,常朝則服之。又有窄袍,便坐視事則服之。皆皂紗折上巾,通犀金玉環帶。窄袍或御烏紗帽。中興仍之。初,高宗踐祚於南都,隆祐太后命內臣上乘輿服御,有小冠。太后曰:「祖宗
 閑居之所服也,自神宗始易以巾。願即位後,退朝止戴此冠,庶幾如祖宗時氣象。」後殿早講,皇帝服帽子,紅袍,玉束帶,講讀官公服系□奚。晚講,皇帝服頭巾,背子,講官易便服。此嘉定四年講筵之制也。



 御閱服。以金裝甲,乘馬大閱則服之。



 圭。宋初,凡大祭祀、大朝會,天子皆執圭。元豐二年,詳定儀注所言:「《周禮》:『王執鎮圭。』釋者曰:『祭天地宗廟及朝日、夕月,則執之。若朝覲,諸侯授玉於王,王受玉,撫玉而已。』《
 考工記》:『天子執冒四寸,以朝諸侯。』蓋天子以冒圭邪刻之處,冒諸侯之圭,以齊瑞信也。未有臨臣子而執鎮圭者。《唐六典》殿中監掌服御之事,凡大祭祀,則搢大圭,執鎮圭;若大朝會,止進爵。《開寶通禮》始著元會執圭,出自西房。淳化中,上壽進酒,又令內侍奉圭,於周制、唐禮皆不合。其元會受朝賀,請不執鎮圭上壽。」詔可。



 三年,詔議大圭尺度,詳定所言:「《考工記》:『鎮圭尺有二寸,天子守之。』『大圭長三尺,杼上終葵首,天子服之』。後魏以降,以白玉
 為之,長尺有二寸,西魏以來皆然。方而不折,雖非古制,蓋後世以所得之玉,隨宜為之。今請揆玉之有無制之。」



 又言:「唐禮,親祀天地神祇,皆搢大圭,執鎮圭。有事宗廟,則執鎮圭而已。王涇《郊祀錄》曰:『大圭,質也,事天地之禮質,故執而搢之。鎮圭,文也,宗廟之禮亦文,故無兼執之義。』不知大圭,天子之笏也,通用於郊廟。請自今皇帝親祠郊廟,搢大圭,執鎮圭。奉祀之時,既接神再拜,則奠鎮圭為摯,大圭為笏。」



 又言:「《開元》及《開寶通禮》,皇帝升輅,不言
 執圭。祀日,質明,至中壝門外,殿中監進大圭,尚衣奉御,又以鎮圭授殿中監以進。於是始搢大圭,執鎮圭。今皇帝乘玉輅,執鎮圭,赴景靈宮及太廟、青城,皆乘輅執圭,殊不應禮。請自今乘輅不執圭,還內御大輦亦如之。」



 詳定所又言大圭中必之制,請制薦玉繅藉,以木為乾,廣袤如玉,以韋衣之,韋上畫五採文,前後垂之。又制約圭繅藉長尺,上玄下絳,為地五採五就,因以為飾。每奠圭,則以薦玉之繅陳於地,執圭,則以約圭之繅備失墜,因
 垂之為飾。況大圭搢之紳帶之間,不可無中必,明矣。俟明堂服大圭,宜依鎮圭所約之組,令可系之。



 哲宗元祐元年,禮部言:「元豐新禮,皇帝祀天,搢大圭,其制圓首前詘,於禮未合。今欲仿西魏、隋、唐玉笏之制,方而不折,上下皆博三寸,長尺二寸,其厚以鎮圭為約。」從之。



 政和二年,宦者譚稹獻玄圭。其制,兩旁刻十二山,若古山尊,上銳下方。上有雷雨之文,下無□彖飾,外黑內赤,中一小好,可容指,其長尺有二寸。詔付廷議。議官以為周王執鎮
 圭,緣飾以四鎮之山,其中有好,為受組之地,其長尺有二寸,周人仿古為之,而王執以鎮四方也。徽宗乃以是歲冬御大慶殿受圭焉。



 三年,又詔曰:「先王以類而求祀,圜丘以象形,蒼玉以象色,冬日以至取其時,大裘而冕法其幽,而未有以體其道,天玄而地黃,今大圭內赤外黑,於以體之,冬祀可搢大圭,執玄圭,永為定制。」中興仍舊制,大祭祀則執大圭以為笏,上太上皇、皇太后冊寶亦如之。



 皇太子之服。一曰袞冕,二曰遠游冠、朱明衣,三曰常服。袞冕:青羅表、緋羅紅綾里、塗金銀鈒花飾,犀簪導,紅絲組,前後白珠九旒,二纊貫水晶珠。青羅衣,繡山、龍、雉、火、虎蜼五章;紅羅裳,繡藻、粉米、黼、黻四章。紅羅蔽膝,繡山、火二章。白紗中單,青褾、□、裾。革帶,塗金銀鉤,瑜玉雙佩。四採織成大綬,結二玉環,金塗銀鈒花飾。青羅襪帶,紅羅勒帛。玉具劍,金塗銀鈒花,玉鏢首。白羅襪,朱履,金塗銀扣。從祀則服之。遠游冠:十八梁,青羅表,金塗銀鈒
 花飾,犀簪導,紅絲組為纓,博山,政和加附蟬。朱明服:紅花金條紗衣,紅紗里,皂褾、□。紅紗裳,紅紗蔽膝,並紅紗里。白花羅中單,皂褾、□,白羅方心曲領。羅襪,黑舄,革帶,劍,佩,綬。餘同袞服。襪帶,勒帛。執桓圭。受冊、謁廟、朝會則服之。常服:皂紗折上巾,紫公服,通犀金玉帶。



 太宗至道元年,太常禮院言:「南郊,皇太子充亞獻,合著祭祀服。準制度,袞冕以組為纓,色如其綬,青纊充耳,玄衣纁裳,凡九章,每章一行,重以為等,皆織為之。白紗中單,黻領,青
 褾、□、裾。革帶,金鉤。大帶,素帶不朱裏,亦紕以朱綠,紐約用組。黻隨裳色,二章。朱組,雙大綬四採,赤白縹紺,純朱質,長一丈八尺,三百二十首,廣九寸。小雙綬,長二尺六寸,色同大綬,而首半之,間施二玉環。朱襪赤舄,舄加金飾,餘同舊制。侍從祭祀及竭廟、加元服、納妃則服之。」詔依上制造。政和議禮局更上皇太子服制,袞冕惟青纊充耳,餘並同國初之制。加元服、從祀、納妃、釋奠文宣王服之。中興並同。



 其皇子之服,紹興三十二年十月,禮
 官言:「皇子鄧、慶、恭三王,遇行事服朝服,則七梁額花冠,貂蟬籠巾,金塗銀立筆,真玉佩,綬,金塗銀革帶,烏皮履。若服祭服,則金塗銀八旒冕,真玉佩,綬,緋羅履襪。」詔文思院制造。



 後妃之服。一曰禕衣,二曰朱衣,三曰禮衣,四曰鞠衣。皇后首飾花一十二株,小花如大花之數,並兩博鬢。寇飾以九龍四鳳。禕之衣,深青織成,翟文赤質,五色十二等。青紗中單,黼領,羅縠褾□,蔽膝隨裳色,以緅為領緣,用翟為章,三等。大帶隨衣色,朱裏,紕其外,上以朱錦,下以綠錦,紐約用青組,革帶以青衣之,白玉雙佩,黑組,雙大綬,小綬三,間施玉環三,青襪、舄,舄加金飾。受冊、
 朝謁景靈宮服之。鞠衣,黃羅為之,蔽膝、大帶、革舄隨衣色,餘同禕衣,唯無翟文,親蠶服之。妃首飾花九株,小花同,並兩博鬢,冠飾以九翬、四鳳。褕翟,青羅繡為搖翟之形,編次於衣,青質,五色九等。素紗中單,黼領,羅縠褾□,蔽膝隨裳色,以緅為領緣,以搖翟為章,二等。大帶隨衣色,不朱裏,紕其外,餘仿皇后冠服之制,受冊服之。



 皇太子妃首飾花九株,小花同,並兩博鬢。褕翟,青織為搖翟之形,青質,五色九等。素紗中單,黼領,羅縠褾示巽,皆
 以朱色,蔽膝隨裳色,以緅為領緣,以搖翟為章,二等。大帶隨衣色,不朱裹,紕其外,上以朱錦,下以綠錦,紐約用青組。革帶以青衣之,白玉雙佩,純朱雙大綬,章採尺寸與皇太子同。受冊、朝會服之。鞠衣,黃羅為之,蔽膝、大帶、革帶隨衣色,餘與褕翟同,唯無翟,從蠶服之。



 中興,仍舊制。其龍鳳花釵冠,大小花二十四株,應乘輿冠梁之數,博鬢,冠飾同皇太后,皇后服之,紹興九年所定也。花釵冠,小大花十八株,應皇太子冠梁之數,施兩博鬢,去龍
 鳳,皇太子妃服之,乾道七年所定也。其服,後惟備禕衣、禮衣,妃備褕翟,凡三等。其常服,後妃大袖,生色領,長裙,霞帔,玉墜子;背子、生色領皆用絳羅,蓋與臣下不異。



 命婦服。政和議禮局上:花釵冠,皆施兩博鬢,寶鈿飾。翟衣,青羅繡為翟,編次於衣及裳。第一品,花釵九株,寶鈿準花數,翟九等;第二品,花釵八株,翟八等;第三品,花釵七株,翟七等;第四品,花釵六株,翟六等;第五品,花釵五株,翟五等。並素紗中單,黼領,朱褾、□,通用羅縠,蔽膝隨
 裳色,以緅為領緣,加文繡重雉,為章二等。



 二品以下準此。



 大帶,革帶,青襪、舄,佩,綬。受冊、從蠶服之。七年,臣僚言:「今文臣九品,殊以三品之服,至於命婦,已厘八等之號,而服制未有名稱。詔有司視其夫之品秩,而定其服飾。」詔送禮制局定之。其儀闕焉。



\end{pinyinscope}