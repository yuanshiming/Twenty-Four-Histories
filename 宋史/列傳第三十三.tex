\article{列傳第三十三}

\begin{pinyinscope}

 王贊張保續趙玭盧懷忠王繼勛丁德裕張延通梁迥史珪田欽祚侯贇王文寶翟守素王侁劉審瓊



 王贊,澶州觀城人。少為小吏,累遷本州馬步軍都虞候。周世宗鎮澶淵,每旬決囚,贊引律令辨析中理,問之,知其嘗事學問,即署右職。及即位,補東頭供奉官,累遷右驍衛將軍、三司副使。時張美為使,世宗問:「京城衛兵歲廩幾何?」美不能對,贊代奏甚析,美因是銜之。及征關南,言於世宗,以贊為客省使,領河北諸州計度使。五代以來,姑息藩鎮,有司不敢繩以法。贊所至,發擿奸伏,無所畏忌,振舉綱領,號為稱職,由是邊臣切齒。師還,復為三
 司副使。



 建隆初,始平李重進,太祖素知贊材幹,可委以完葺,即令知揚州。既行,舟覆於閶橋下,溺死,親屬隨沒者三人。上甚嗟悼,謂左右曰:「溺吾樞密使矣!」蓋將大用也。賻其家絹三百匹,米、麥各二百斛。



 張保續字嗣光,京兆萬年人。父洪,唐左武衛上將軍,保續以蔭補太廟齋郎。梁貞明中,調補臨濟尉,選充四方館通事舍人。後唐天成初,領瓜州官告國信副使。郊祀,改右贊善大夫。晉天福中,歷太府、光祿二少卿,職同正,
 領通事舍人。開運二年,契丹入寇,杜重威、李守貞、符彥卿等率兵御之。命保續馳騎往來軍中諭機事。既而大破敵於陽城,使還,以本官充西上閣門副使。明年,使荊南,復命轉東上閣門副使。契丹犯闕,被驅北徙,留範陽,歲餘逃歸。



 漢乾祐初,出為隴州防禦使。周祖革命,召為東上閣門副使,從平慕容彥超。累遷引進副使、知閣門事。世宗即位,授西上閣門使。明年,進秩東上閣門使。從上征淮南,會壽州納款,遣保續先往慰撫,及劉仁贍率
 將卒出降,以功遷判四方館事,就遷客省使。從平瓦橋關,奉使吳越。



 宋初,遷衛尉卿,判四方館、客省、閣門事。保續性介直,好儉素,在閣門前後四十年,善宣贊辭,令聽者傾聳。累使藩國不辱命。歷事六朝,未嘗有過。從征李筠,以足疾留河內,後歸京師。建隆三年,卒,年六十四。



 趙玭,澶州人。家富於財。晉天福中,以納粟助邊用,補集賢小史,調濮州司戶參軍。刺史白重進以其年少,欲試以事,因以滯獄授之。玭為平決,悉能中理。重進移刺虢、
 成二州,連闢為從事。會契丹構難,秦帥何重建獻地於蜀,孟知祥署高彥儔秦州節度,成為支郡,因署玭秦、成、階等州觀察判官。



 周顯德初,命王景帥兵討秦鳳。彥儔出兵救援,未至,聞軍敗,因潰歸。□比閉門不納,召官屬諭之曰:「今中朝兵甲無敵於天下,自用師西征,戰無不勝。蜀中所遣,將皆武勇者,卒皆驍銳者,然殺戮遁逃之外,幾無孑遺。我輩安忍坐受其禍?去危就安,當在今日。」眾皆俯伏聽命。玭遂以城歸朝。世宗欲命以藩鎮,宰相範
 質不可,乃授郢州刺史,歷汝、密、澤三州刺史。



 建隆中,入為宗正卿。乾德初,出為泰州刺史。二年,改左監門衛大將軍、判三司。玭狂躁幸直,多忤上旨,太祖頗優容之。嘗廉得宰相趙普私市秦、隴大木事,潛以奏白,然懼普知,因稱足疾求解職。五年春,罷使,守本官。自是累獻密疏,皆留中不出,常疑普中傷。六年,詣闕,納所授告命,詔勒歸私第。又請退居鄆州,不許。玭不勝忿,逾年,伺普入朝,馬前揚言其短。上聞之,召玭及普於便殿,面質其事。玭
 大言詆普販木規利,上怒,促令集百官逐普,且諭其事。王溥等奏玭誣罔大臣,普事得解。上詰責玭,命武士撾之,令御史鞫於殿庭。普為營救,得寬其罰,黜為汝州牙校。太平興國三年卒,年五十八。



 盧懷忠,瀛州河間人。少有膂力,善騎射。漢乾祐初,寓居河中,值李守貞之叛,周祖圍其城,懷忠夜逾城出見,陳攻取便宜。河中平,奏補供奉官。從征慕容彥超於兗州。顯德初,監沂州軍,以所部破海州,功居多。世宗議北征,
 先遣懷忠按視出師道路。三關平,遷如京副使。



 宋初,遷內酒坊副使。會朗州軍亂,太祖將出師致討,遣懷忠使荊南,因謂曰:「江陵人情去就,山川向背,我欲盡知之。」懷忠使還,奏曰:「繼沖甲兵雖整,而控弦不過三萬;年穀雖登,而民苦於暴斂。南邇長沙,東距金陵,西迫巴蜀,北奉朝廷。觀其形勢,蓋日不暇給矣。」太祖召宰相範質等謂曰:「江陵四分五裂之國,今出師湖南,假道荊渚,因而下之,萬全策也。」即以懷忠為前軍步軍都監。荊湖平,以功遷
 內酒坊使。



 乾德二年,改判四方館事,知江陵府。四年,王師伐蜀。江陵當峽、江會沖,以供億之勞,遷客省使。又明年,使江南還,中途遇疾,肩輿歸京師。太祖遣醫丸艾以賜之,未幾卒,年四十九。大中祥符四年,錄其子熙為校書郎。



 王繼勛,陜州平陸人。隸河中府為牙校。李守貞之叛,令繼勛據潼關,為郭從義所破,走還河中。俄白文珂、劉詞領兵至城下,守貞又遣繼勛與其愛將聶知遇夜出攻
 河西砦,復為漢兵所敗,被創而遁。繼勛度守貞必敗,遂逾城出降,周祖奏補供奉官。廣順初,領汾州刺史,充晉、磁、隰等州緣邊巡檢,歷憲、麟、石、磁四州刺史。



 宋初,遷磁州團練使,坐境上用兵失律、荊罕儒陷陣,責授右監門衛率。初平荊襄,命權知道州,未幾,授本州刺史。州境與廣南接,劉鋹屢引兵入寇,繼勛因上言嶺表可圖之狀。及王師南伐,以為賀州道行營馬步軍都監。繼勛有武勇,在軍陣,常用鐵鞭、鐵槊、鐵□,軍中目為「王三鐵」。



 丁德裕,洺州臨洺人。父審琦,彰武軍節度。周廣順初,以蔭補供奉官。宋初,歷通事舍人、西上閣門副使。建隆三年,遷東上閣門使。從慕容延釗平荊湖,以功授引進使。又與潘美、尹崇珂克郴州,遷客省使。乾德五年,遷內客省使。時成都初平,群寇大起,用為西川都巡檢使,與閣門副使張延通同率師討之,擒賊帥康祚,磔於市。歲餘,盡平其黨。頗與延通不協,歸朝,告其陰事,延通坐棄市。又奏轉運使、禮部郎中李鉉嘗醉酒,言涉指斥。上怒,驛
 召鉉下御史案之。鉉言德裕在蜀日屢以事請求,多拒之,皆有狀。御史以聞。太祖悟,止坐鉉酒失,責授左贊善大夫。



 未幾,德裕亦出知潞州。會征江南,遣德裕為常州行營兵馬都監,領吳越兵,助主帥進討。常州平,命權知州事。又改升州東南路行營都監,敗潤州軍五千餘於城下。及拔潤州,移領常、潤等州經略巡檢使。德裕以傾險為眾所惡,恃勢剛狠,不恤士卒,黷貨無厭,越人苦之。錢俶奏其事,貶房州刺史,卒。



 張延通,潞州潞城人。父彥成,周右金吾衛上將軍。延通性穎悟,有才幹,蔭補供奉官。宋初,歷通事舍人,遷東上閣門副使。開寶中,為西川兵馬都監。太祖以蜀寇未平,命同內客省使丁德裕、引進副使王班、內臣張嶼領兵屯蜀部。德裕頗專恣,延通面質其短,德裕銜之。又與張嶼不協,延通亦為和解之,德裕疑延通與嶼為黨,益不悅。會太祖征太原,有使自行在至,備言太祖當盛暑躬冒矢石,勞頓萬狀。延通曰:「主上勤勞若此,而吾輩日享
 安樂。」蓋言不自安也。德裕不答。會張嶼先歸闕,太祖賜予甚厚。延通、德裕繼至,則召延通顧問,而待德裕稍薄。德裕頗疑懼,遂奏延通嘗對眾言涉指斥,且多不法事,指嶼為黨。太祖怒,即收延通、張嶼及王班下御史臺鞫之,延通等引伏。太祖始欲舍之,及引問,延通抗對不遜,遂斬之。嶼、班並內臣王仁吉並仗脊,嶼配流沙門島,班許州,仁吉西窯務,時開寶二年也。



 梁迥,博州聊城人。少為吏部小史。周世宗在藩邸日,得
 給事左右。及嗣位,補殿直,改供奉官,四遷至左藏庫使。



 太祖將討西蜀,以迥監秦州戍兵。蜀平,改監霸州兵,轉宮苑使。從征太原還,會命蜀州刺史聶章為沁州兵馬部署,以迥監其軍。無何,並人入寇,迥與閻彥進同率兵擊敗之,以功遷東上閣門使。開寶五年,命為廣南道兵馬都監兼諸州巡檢。



 八年,奉使江南。迥素貪冒,外務矯飾,初若嚴毅不可犯,雖饋食亦不受,江南人頗憚之。既而奉以貲貨,殆直數萬緡,迥即大喜過望,登舟縱酒,繼
 日宴樂。及歸,戀戀不發,人多笑之。暨王師伐金陵,命迥與潘美、劉遇率步兵先赴荊南。且以迥護行營步兵及左廂戰棹,與吳人戰採石,殺獲甚眾。江南平,以功領順州團練使。



 太宗即位,判四方館事,領禁軍戍澤州。太平興國三年,錢俶來朝,命往淮、泗迎勞。夏,汴水大決,詔迥發畿內丁男三千護塞汴口。四年,徵太原,以迥為行營前軍馬步軍都監,督軍攻城,中流矢四。車駕還,命與孟玄哲、崔翰率兵屯定州,以功遷引進使。五年,受詔與潘
 美城並州於三交,及築緣邊堡障。七年,李繼遷寇邊,以迥領兵護銀、夏州。八年,召歸,授唐州防禦使,令赴職。



 雍熙二年,繼遷誘殺都巡檢使曹光實,乘勢數寇邊。復召迥為銀、夏都巡檢使,赴邊捍禦之。三年夏,卒於銀州官舍,年五十九。



 迥性粗率,尤不喜文士,故事,節帥出鎮及來朝,便殿宴勞,翰林學士皆預坐。開寶中,迥為閣門使,白太祖曰:「陛下宴犒將帥,安用此輩預坐?」自是罷之。至淳化中,翰林學士蘇易簡白於太宗,始復預焉。大中祥
 符八年,錄迥子廷翰為奉職。



 史珪,河南洛陽人。父暉,晉嚴衛指揮使。珪少以武勇隸軍籍,周顯德中,遷小校。太祖領禁衛,以珪給事左右。及受禪,用為御馬直隊長,四遷馬步軍副都軍頭兼控鶴、弓弩、大劍都指揮使。開寶六年,加都軍頭,領毅州刺史。



 太祖初臨御,欲周知外事,令珪博訪。珪廉得數事白於上,驗之皆實,由是信之,後乃漸肆威福。民有市官物不當價者,珪告其欺罔,當置法,列肆無不側目。上聞之,因
 下詔曰:』古人以獄市為寄者,蓋知小民唯利是從,不可盡法而繩之也。況先甲之令,未嘗申明。茍陷人於刑,深非理道。將禁其二價,宜示以明文,自今應市易官物,有妄增價直欺罔官錢者,案鞫得實,並以枉法論。其犯在詔前者,一切不問。」自是珪不復敢言。



 時德州刺史郭貴知邢州,國子監丞梁夢升知德州,貴族人親吏之在德州者頗為奸利,夢升以法繩之。貴素與珪善,遣人以其事告珪,圖去夢升。珪悉記於紙,將伺便言之。一日,上因
 言:「爾來中外所任,皆得其人。」珪遽曰:「今之文臣,亦未必皆善。」乃探懷中所記以進,曰:「只如知德州梁夢升欺蔑刺史郭貴,幾至於死。」上曰:「此必刺史所為不法。夢升,真清強吏也。」因以所記紙付中書曰:「即以夢升為贊善大夫。」既又曰:「與左贊善。」珪以譖不行,居常怏怏。九年,坐漏洩禁中語,出為光州刺史。會歲饑,淮、蔡民流入州境,珪不待聞,即開倉減價以糶,所全活甚眾,吏民詣闕請植碑頌德者數百人。



 太平興國初,以為揚、楚等九州都巡
 檢使。四年,徵太原,命珪與彰信軍節度劉遇攻城北面。從征幽州,坐所部逗撓失律,責授定武行軍司馬。數月,召為右衛將軍、領平州刺史。督浚惠民河,自尉氏達京九十里,數旬而畢,民咸便之。會江、淮民曲謀首等數十百人聚為盜,命珪率龍猛騎兵五百往捕,悉獲之。六年,遷隰州刺史,知保州、靜戎軍。上緣邊便宜十五事,皆從之。



 雍熙中,從曹彬徵幽州,為押陣部署,以所部下涿州。師還,卒,年六十一。珪多智數,好以甘言小惠取譽於人,
 故所至不忍其去云。



 田欽祚,穎州汝陰人。父令方,漢虢州團練使。帳下伶人靖邊庭妻有美色,令方私之,邊庭不勝忿。會陜西三叛連衡,關輔間人情大擾。邊庭率其徒數人夜縋入州廨,害令方,因掠郡民投趙思綰,至潼關,與守關使者戰,遂敗散。朝廷錄欽祚為殿直,改供奉官。周世宗征淮南,為前軍都監。從征關南還,會塞澶淵決河,命欽祚領禁兵護役,因令督治澶州城。淮人寇高密,刺史王萬威求濟
 師,命欽祚領州兵援之,既至,圍解。



 宋初,遷閣門通事舍人。乾德二年冬,討蜀,為北路先鋒都監,令乘傳往來宣達機事。孟昶降,奉捷書馳奏,遷西上閣門副使。蜀土寇亂,又遣欽祚率師討平之。四年春,並人寇樂平,從羅彥瑰拒之,獨以所部三千人破寇,擒副將一人,俘獲甚眾,以功遷西上閣門使。開寶二年,又與何繼筠破賊兵於石嶺關,領賀州刺史,判四方館使。三年,契丹寇中山,以欽祚為定州路兵馬都部署。與戰遂城,自旦及晡,殺傷
 甚眾。欽祚馬中流矢踣,騎士王超授欽祚以馬,軍復振,敵解去。朝廷將議討江表,遣欽祚覘之,還奏合旨,江南所得寶貨直三千萬,悉以賜欽祚。會興師,首命欽祚與曹彬、李漢瓊率騎軍先赴江陵,就命為升州西南路行營馬軍兼左廂戰棹都監。領兵敗吳軍萬餘於溧水,斬其主帥李雄等五人,擒裨將二人。進圍金陵,為南面攻城部署。既平,以功加領汾州防禦使。



 太平興國初,遷引進使,為晉州都鈐轄。太原驍將楊業率眾寇洪洞縣,欽
 祚擊敗之,斬首千餘級,獲馬數百。太宗賜欽祚白金五千兩,令市宅。四年,從征太原,護前鋒騎兵,屯石嶺關以捍契丹。



 欽祚性剛戾負氣,多所忤犯,與主帥郭進不協,進戰功高,屢為欽祚所陵,心不能甘,遂自縊死。初,賊兵奄至,進出戰,欽祚但閉壁自守,既去,又不追。所受月奉芻粟,多販鬻規利,為部下所訴,責授睦州團練使。車駕北巡,以為幽州西路行營壕砦都監。六年秋,改房州團練使,逾年,又改柳州。嶺外多瘴氣,因遘疾,累表乞生還
 闕下。上憐之,遷郢州團練使。在郡二年,入覲,欽祚見上,涕泣不已。以為銀、夏、綏、宥都巡檢使,俄召還。會征幽州,命欽祚與宣徽南院使郭守文為排陣使。時欽祚已被病,受詔不勝喜,一夕,卒。



 欽祚性陰狡,尤不喜儒士,好狎侮同列,人多惡之。子承誨,仕至供奉官、閣門祗候;承說,至崇儀副使。



 侯贇,並州太原人。父義,漢遼州刺史。贇以蔭補殿前承旨。周顯德中,再遷至供奉官,使江南,復命領三門、集津發
 運使。



 宋初,為諸衛將軍。先是,朝廷歲仰關中穀麥以給用,贇掌其事歷三十年,國用無闕。累遷至右武衛將軍。開寶中,歷知建安軍、揚、徐二州,皆有善政。太宗即位,移知福州,改右衛將軍。太平興國二年,錢俶初納土,詔贇馳往兩浙諸州閱視軍儲芻茭,累遷右衛大將軍。七年,知靈州,按視蕃落,宴犒以時,得邊士心,部內大治,遷左衛。在朔方凡十餘年,上念久次,求可代者而難其人。淳化二年,卒於官,年七十四,贈本衛上將軍。



 王文寶,開封陽武人,以任子補殿直。太平興國初,累遷至軍器庫使。嘗使契丹。會陳洪進獻漳、泉地,以文寶監泉州兵。群盜大起,文寶與轉運使楊克讓、知州喬惟岳共討平之。以功領媯州刺史,加內弓箭庫使。二年,京西轉運使程能議開新河,自襄、漢至京師,引白河水注焉,以通湘、潭之漕。詔發唐、鄧、汝、穎、許、蔡、陳、鄭丁夫數萬赴其役,又發諸州兵萬人助之。命文寶與六宅使李繼隆、作坊副使李仁祐、劉承珪分往護作。既而地高水下,不能
 通,卒堙廢焉。雍熙四年,改東上閣門使,歷知涇、延二州。會遼人寇通遠軍,命文寶率師致討,還遷判四方館事。



 文寶歷內職三十年,雅好言外事,太祖、太宗頗信任之,中外咸畏其口。出為高陽關兵馬鈐轄,淳化二年,卒於官。



 翟守素,濟州任城人。父溥,晉左司禦率府率。守素以父任為殿直,歷漢、周,遷供奉官,領承天軍使。乾德中,為引進副使,從王全斌伐蜀,以往來馳告軍事為職。蜀平,擢
 判四方館事。以兩川餘寇未殄,慮致騷動,再令守素入蜀經略諸郡,分兵以防遏之。



 開寶中,會麟、府內屬戎人爭地不決,因致擾亂,命守素馳往撫喻。守素辨其曲直,戎人悅服。從征太原,命海州刺史孫方進圍汾州,守素監其軍,轉引進使。



 開寶三年,命為劍南十州都巡檢使,東上閣門使郭崇信副之。賜守素錢五百萬,入謝日,復遣為岐帥符彥卿官告使。守素辭以錫繼優厚,不敢更當奉使之詔,上不許。九年,吳越國王錢俶來朝,命守素護
 諸司供帳,迎勞郊外。並壘未下,詔與洺州防禦使郭進率兵深入其境,蹈藉禾稼,守素多所虜獲。太宗即位,遷客省使,領憲州刺史。



 太平興國三年夏,河決滎陽,詔守素發鄭之丁夫千五百人,與卒千人領護塞之。是秋,梅山洞蠻恃險叛命,詔遣守素率諸州屯兵往擊之。值霖雨彌旬,弓弩解弛,不堪用,明日,將接戰,守素一夕令削木為弩。及旦,賊奄至,交射之,賊遂敗。乘勝逐北,盡平其巢穴。先是,數郡大吏、富人多與賊帥包漢陽交通,既而
 得其書訊數百封,守素並焚之,反側以定。俄而錢俶獻浙右之地,詔守素為兩浙諸州兵馬都監,安撫諸郡,人心甚悅,即以知杭州。歲滿,為西京巡檢使。秦王廷美以事勒歸私第,以守素權知河南府兼留守司事,屬洛陽歲旱艱食,多盜,上憂之。守素既至,漸以寧息。未幾,遷商州團練使。



 雍熙二年,改知延州。自劉廷讓敗於君子館,河朔諸州城壘多圮。四年,詔守素與田仁朗、王繼恩、郭延浚分路按行,發諸州鎮兵增築,護其役。賜白金三十
 兩,留充天雄軍兵馬鈐轄、知大名府,改知潞州。會建方田,命為代北方田都部署、並州兵馬鈐轄,從屯夏州,改知鳳翔府。



 淳化中,夏帥趙保忠上言,其弟繼遷誘戎人為寇,且求援師。詔守素率兵復屯夏州,未幾,又徙石州,以老病上疏求歸本郡,從之。三年,卒,年七十一。



 守素逮事四朝,綿歷內職五十餘年。性謹慎,寬仁容眾,所至有治績。凡斷大闢獄,雖罪狀明白,仍遍詢僚寀,僉同而後決;屬吏有過不面折,必因公宴援往事之相類者言其
 獲咎,以微警之。新進後生多至節帥,而守素久次不遷,殊無隕獲意,時論以此多之。



 王侁,字秘權,開封浚儀人。父樸,周樞密使,侁以父任太僕寺丞。樸卒,世宗幸其第,召見諸孤,以侁為東頭供奉官。開寶中,征江南,命侁率師戍桐城。王師渡江,與樊若水同知池州,領兵敗江南軍四千餘於宣州。金陵平,加閣門祗候。



 太平興國初,預討梅山洞蠻。契丹使來貢,詔侁送於境上。還,使靈州、通遠軍。及旋,言主帥所留牙兵
 率與邊人交結,頗桀黠難制,歲久當慮,請悉代之。太宗因遣侁調內郡卒往代之。戍者聞代,多不願還。侁察其中旅拒者斬之以徇,眾皆悚息,遂將以還。一歲中數往來西邊,多奏便宜,上多聽用,遷通事舍人。



 四年,從征太原,以侁護陽曲、塌地、石嶺關諸屯,賜廄馬介冑。五月,即城下轉東上閣門副使。晉陽平,留為嵐、憲巡檢。九年,代還,遷西上閣門使,賜錢百萬。河西三族首領折遇乜叛入李繼遷,侁帥師討擒之,以功領蔚州刺史。王師北征,
 命為並州駐泊都監,又為雲、應等州兵馬都監。



 侁性剛愎,以語激楊業,業因力戰陷於陣,侁坐除名,配隸金州,事載《楊業傳》。會赦,移均州團練副使。淳化五年召還,道病,至京師卒。



 弟僎,供奉官、閣門祗候,坐徵交址軍敗誅;備、偃並進士及第,偃至太常博士。



 樸弟格,宋初為右補闕、直史館,至都官員外郎、廣南轉運使。格子侗,太平興國進士,至都官員外郎。



 劉審瓊,涿州範陽人。家素貧。漢乾祐中,湘陰公鎮彭門,
 審瓊始隸帳下。周祖受命,遁去,依永興軍節度劉詞,頗委任之。詞卒,屬太祖節鎮,給事左右。及受禪,補殿直。從平澤、潞,改供奉官。



 開寶中,累遷至軍器庫使。會樞密使李崇矩門人鄭伸擊登聞鼓,誣告崇矩受太原席羲叟黃金,私結翰林學士扈蒙,以甲科私羲叟,引審瓊為證。上怒,召審瓊詰問,審瓊具言其誣枉,得解,遂出知鎮州。七年,太宗征河東,駐蹕月餘,儲彳侍無闕,遷領檀州刺史、知潭州。州素多火,日調民積水為防,民甚勞之。審瓊至,
 悉罷之,以為民便。徙知河陽。淳化三年,受代歸,陳衰老,乞正受郡符。上閔其舊人,授坊州刺史。至道三年,卒於官。



 審瓊嘗給事外諸侯,雅善酒令博鞠,年八十餘,筋力不衰,髭發黳黑。孫爽,進士及第,後為祠部員外郎、秘閣校理。



 論曰:王贊奮跡小校,有奉公之節,繩奸列郡,不畏強御。保續單車出使,不辱君命。懷忠識荊渚之將危。繼勛知番禺之可取。侯贇久治邊郡。文寶數護屯兵:斯各一
 時之效也。德裕、梁迥、欽祚、王侁皆練習戎旅,頗著勛勞,然率強戾而乏溫克,以速於戾,斯乃明哲之所戒。玭以剛險蒙悔吝,珪以發擿肆威福,其不逞者歟!守素不事躁競,審瓊克享期頤。《易》曰:「視履考祥,其旋元吉。」此之謂也



\end{pinyinscope}