\article{志第四十九 河渠六}

\begin{pinyinscope}

 東南諸水上



 開寶間,議征江南。詔用京西轉運使李符之策,發和州丁夫及鄉兵凡數萬人,鑿橫江渠於歷陽,令符督其役。渠成,以通漕運,而軍用無闕。



 八年,知瓊州李易上言:「州
 南五里有度靈塘,開修渠堰,溉水田三百餘頃,居民賴之。」



 初,楚州北山陽灣尤迅急,多有沉溺之患。雍熙中,轉運使劉蟠議開沙河,以避淮水之險,未克而受代。喬維岳繼之,開河自楚州至淮陰,凡六十里,舟行便之。



 天禧元年,知升州丁謂言:「城北有後湖,往時歲旱水竭,給為民田,凡七十六頃,出租錢數百萬,蔭溉之利遂廢。令欲改田除租,跡舊制,復治岸畔,疏為塘陂以畜水,使負郭無旱歲,廣植浦芡,養魚鱉,縱貧民漁採。」又明州請免濠
 池及慈溪、鄞縣陂湖年課,許民射利。詔並從之。



 二年,江、淮發運使賈宗言:「諸路歲漕,自真、揚入淮、汴,歷堰者五,糧載煩於剝卸,民力罷於牽挽,官私船艦,由此速壞。今議開揚州古河,繚城南接運渠,毀龍舟、新興、茱萸三堰,鑿近堰漕路,以均水勢。歲省官費十數萬,功利甚厚。」詔屯田郎中梁楚、閣門祗候李居中按視,以為當然。明年,役既成,而水注新河,與三堰平,漕船無阻,公私大便。



 四年,淮南勸農使王貫之導海州石闥堰水入漣水軍,溉
 民田;知定遠縣江澤、知江陰軍崔立率民修廢塘,浚古港,以灌高仰之地。並賜詔獎焉。



 神宗熙寧元年十月,詔:「杭之長安、秀之杉青、常之望亭三堰,監護使臣並以『管幹河塘』系銜,常同所屬令佐,巡視修固,以時啟閉。」從堤舉兩浙開修河渠胡淮之請也。



 二年三月甲申,先是,凌民瞻建議廢呂城堰,又即望亭堰置閘而不用。及因浚河,隳敗古涇函、石閘、石□達,河流益阻,百姓勞弊,至是,民瞻等貶降有差。



 六年五月,杭州於潛縣令郟但言:「蘇州
 環湖地卑多水,沿海地高多旱,故古人治水之跡,縱則有浦,橫則有塘,又有門堰、涇瀝而棋布之。今總二百六十餘所。欲略循古人之法,七里為一縱浦,十里為一橫塘,又因出土,以為堤岸,度用夫二十萬。水治高田,旱治下澤,不過三年,蘇之田畢治矣。」十一月,命但興修水利。然措置乖方,民多愁怨,僅及一年,遂罷兩浙工役。又數月,中書檢正沉括復言:「浙西涇濱淺涸,當浚;浙東堤防川瀆堙沒,當修。請下司農貸緡募役。」從之,仍命括相度
 兩浙水利。



 九年正月壬午,劉瑾言:「揚州江都縣古鹽河、高郵縣陳公塘等湖、天長縣白馬塘沛塘、楚州寶應縣泥港射馬港、山陽縣渡塘溝龍興浦、淮陰縣青州澗、宿州虹縣萬安湖小河、壽州安豐縣芍陂等,可興置,欲令逐路轉運司選官覆按。」從之。



 元豐五年九月,淮南監司言:「舒州近城有大澤,出灊山,注北門外。比者,暴水漂居民,知州楊希元築捍水堤千一百五十丈,置洩水斗門二,遂免淫潦入城之患。」並璽書獎諭。



 六年正月戊辰,開
 龜山運河,二月乙未告成,長五十七里,闊十五丈,深一丈五尺。初,發運使許元自淮陰開新河,屬之洪澤,避長淮之險,凡四十九里。久而淺澀,熙寧四年,皮公弼請復浚治,起十一月壬寅,盡明年正月丁酉而畢,人便之。至是,發運使羅拯復欲自洪澤而上,鑿龜山裏河以達於淮,帝深然之。會發運使蔣之奇入對,建言:「上有清汴,下有洪澤,而風浪之險止百里淮,邇歲溺公私之載不可計。凡諸道轉輸,涉湖行江,已數千里,而覆敗於此百里
 間,良為可惜。宜自龜山蛇浦下屬洪澤,鑿左肋為復河,取淮為源,不置堰閘,可免風濤覆溺之患。」帝遣都水監丞陳祐甫經度。祐甫言:「往年田棐任淮南提刑,嘗言開河之利。其後淮陰至洪澤,竟開新河,獨洪澤以上,未克興役。今既不用閘蓄水,惟隨淮面高下,開深河底,引淮通流,形勢為便。但工費浩大。」帝曰:「費雖大,利亦博矣。」祐甫曰:「異時,淮中歲失百七十艘。若捐數年所損之費,足濟此役。」帝曰:「損費尚小,如人命何。」乃調夫十萬開治,既
 成,命之奇撰記,刻石龜山,後至建中靖國初,之奇同知樞密院,奏:「淮水浸淫,沖刷堤岸,漸成墊缺,請下發運司及時修築。」自是,歲以為常。



 是年,將作監主簿李湜言:「鼎、澧等州,宜開溝洫,置斗門,以便民田。」詔措置以聞。七年十月,浚真、楚運河。



 哲宗元祐四年,知潤州林希奏復呂城堰,置上下閘,以時啟閉。其後,京口、瓜州、奔牛皆置閘。是歲,知杭州蘇軾浚茆山、鹽橋二河,分受江潮及西湖水,造堰閘,以時啟閉。初,杭近海,患水泉咸苦,唐刺史
 李泌始導西湖,作六井,民以足用。及白居易復浚西湖,引水入運河,復引溉田千頃。湖水多葑,自唐及錢氏後廢而不理。至是,葑積二十五萬餘丈,而水無幾。運河失湖水之利,取給於江潮,潮水淤河,泛溢闤闠,三年一浚,為市井大患,故六井亦幾廢。軾既浚二河,復以餘力全六井,民獲其利。



 十二月,京東轉運司言:「清河與江、浙、淮南諸路相通,因徐州呂梁、百步兩洪湍淺險惡,多壞舟楫,由是水手、牛驢、手牽戶、盤剝人等,邀阻百端,商賈不行。朝
 廷已委齊州通判滕希靖、知常州晉陵縣趙竦度地勢穿鑿。今若開修月河石堤,上下置閘,以時開閉,通放舟船,實為長利。乞遣使監督興修。」從之。



 紹聖二年,詔「武進、丹陽、丹徒縣界沿河堤岸及石□達、石木溝,並委令佐檢察修護,勸誘食利人戶修葺。任滿,稽其勤惰而賞罰之。」從工部之請也。



 四年四月,水部員外郎趙竦請浚十八里河,令賈種民相度呂梁、百步洪,添移水磨。詔發運並轉運司同視利害以聞。



 元符元年正月,知潤州王悆建
 言:「呂城閘常宜單水入澳,灌注閘身以濟舟。若舟沓至而力不給,許量差牽駕兵卒,並力為之。監官任滿,水無走洩者賞,水未應而輒開閘者罰,守貳、令佐,常覺察之。」詔可。



 三月甲寅,工部言:「淮南開河所開修楚州支家河,導漣水與淮通。」賜名通漣河。



 二年閏九月,潤州京口、常州奔牛澳閘畢工。先是,兩浙轉運判官曾孝蘊獻澳閘利害,因命孝蘊提舉興修,仍相度立啟閉日限之法。



 三年二月,詔:「蘇、湖、秀州,凡開治運河、港浦、溝瀆,修壘堤岸,
 開置斗門、水堰等,許役開江兵卒。」



 徽宗崇寧元年十二月,置提舉淮、浙澳閘司官一員,掌杭州至揚州瓜洲澳閘,凡常、潤、杭、秀、揚州新舊等閘,通治之。



 崇寧二年初,通直郎陳仲方別議浚吳松江,自大通浦入海,計工二百二十二萬七千有奇,為緡錢、糧斛十八萬三千六百,乞置乾當官十員。朝廷下兩浙監司詳議,監司以為可行。時又開青龍江,役夫不勝其勞,而提舉常平徐確謂:「三州開江兵卒千四百人,使臣二人,請就令護察已開之
 江,遇潮沙淤澱,隨即開淘;若他役者,以違制論。」確與監司往往被賞,人以為濫。



 十二月,詔淮南開修遇明河,自真州宣化鎮江口至泗州淮河口,五年畢工。



 明年三月,詔曰:「昨二浙水災,委官調夫開江,而總領無法,役人暴露,飲食失所,疾病死亡者眾。水仍為害,未嘗究實按罪,反蒙推賞,何以厭塞百姓怨咨。」乃下本路提刑司體量。提刑司言:「開浚吳松、青龍江,役夫五萬,死者千一百六十二人,費錢米十六萬九千三百四十一貫石,積水至
 今未退。」於是元相度官轉運副使劉何等皆坐貶降。



 四年正月,以倉部員外郎沈延嗣提舉開修青草、洞庭直河。



 大觀元年五月,中書舍人許光凝奏:「臣向在姑蘇,遍詢民吏,皆謂欲去水患,莫若開江浚浦。蓋太湖在諸郡間,必導之海然後水有所歸。自太湖距海,有三江,有諸浦,能疏滌江、浦,除水患猶反掌耳。今境內積水,視去歲損二尺,視前歲損四尺,良由初開吳松江,繼浚八浦之力也。吳人謂開一江有一江之利,浚一浦有一浦之利。
 願委本路監司,與諳曉水勢精強之吏,遍詣江、浦,詳究利害,假以歲月,先為之備。然後興夫調役,可使公無費財,而歲供常足;人不告勞,而民食不匱,是一舉而獲萬世之利也。」詔吳擇仁相度以聞,開江之議復興矣。



 十一月,詔曰:「《禹貢》:『三江既導,震澤底定。』今三江之名,既失其所,水不趨海,故蘇、湖被患。其委本路監司,選擇能臣,檢按古跡,循導使之趨下,並相度圩岸以聞。」於是復詔陳仲方為發運司屬官,再相度蘇州積水。



 二年八月,詔:「常、
 潤歲旱河淺,留滯運船,監司督責浚治。」三年,兩浙監司言:「承詔案古跡,導積水,今請開淘吳松江,復置十二閘。其餘浦閘、溝港、運河之類,以次增修。若田被水圍,勸民自行修治。」章下工部,工部謂:「今所具三江,或非禹跡;又吳松江散漫,不可開淘洩水。」遂命諸司再相度以聞。



 四年八月,臣僚言:「有司以練湖賜茅山道觀,緣潤州田多高仰,及運渠、夾岡水淺易涸,賴湖以濟,請別用天荒江漲沙田賜之,仍令提舉常平官考求前人規畫修築。」從
 之。十月,戶部言:「乞如兩浙常平司奏,專委守、令籍古瀦水之地,立堤防之限,俾公私毋得侵占。凡民田不近水者,略仿《周官》遂人、稻人溝防之制,使合眾力而為之。」詔可。



 政和元年,知陳州霍端友言:「陳地污下,久雨則積潦害稼。比疏新河八百里,而去淮尚遠,水不時洩。請益開二百里,起西華,循宛丘,入項城,以達於淮。」從之



 政和元年十月,詔蘇、湖、秀三州治水,創立圩岸,其工費許給越州鑒湖租賦。已而升蘇州為平江府,潤州為鎮江府。



 二
 年七月,兵部尚書張閣言:「臣昨守杭州,聞錢塘江自元豐六年泛溢之後,潮訊往來,率無寧歲。而比年水勢稍改,自海門過赭山,即回薄巖門、白石一帶北岸,壞民田及鹽亭、監地,東西三十餘里,南北二十餘里。江東距仁和監止及三里,北趣赤岸口二十里。運河正出臨平下塘,西入蘇、秀,若失障御,恐他日數十里膏腴平陸,皆潰於江,下塘田廬,莫能自保,運河中絕,有害漕運。」詔亟修築之。



 四年二月,工部言:「前太平州判官盧宗原請開
 修自江州至真州古來河道湮塞者凡七處,以成運河,入浙西一百五十里,可避一千六百里大江風濤之患;又可就土興築自古江水浸沒膏腴田,自三百頃至萬頃者凡九所,計四萬二千餘頃,其三百頃以下者又過之。乞依宗原任太平州判官日已興政和圩田例,召人戶自備財力興修。」詔沉鏻等相度措置。



 六年閏正月,知杭州李偃言:「湯村、巖門、白石等處並錢塘江通大海,日受兩潮,漸至侵嚙。乞依六和寺岸,用石砌疊。」乃命劉既
 濟修治。



 八月,詔:「鎮江府傍臨大江,無港澳以容舟楫,三年間覆溺五百餘艘。聞西有舊河,可避風濤,歲久湮廢,宜令發運司浚治。」



 是年,詔曰:「聞平江三十六浦內,自昔置閘,隨潮啟閉,歲久堙塞,致積水為患。其令守臣莊徽專委戶曹趙霖講究利害,導歸江海,依舊置閘。」於是,發運副使應安道言:「凡港浦非要切者,皆可徐議。惟當先開昆山縣界茜涇塘等六所;秀之華亭縣,欲並循古法,盡去諸堰,各置小斗門;常州、鎮江府、望亭鎮,仍舊置閘。」
 八月,詔戶曹趙霖相度役興,而兩浙擾甚。七年四月己未,尚書省言:「盧宗原浚江,慮成搔擾。」詔權罷其役,趙霖別與差遣。



 重和元年二月,前發運副使柳庭俊言:「真揚楚泗、高郵運河堤岸,舊有斗門水閘等七十九座,限則水勢,常得其平,比多損壞。」詔檢計修復。六月,詔:「兩浙霖雨,積水多浸民田,平江尤甚,由未浚港浦故也。其復以趙霖為提舉常平,措置救護民田,振恤人戶,毋令流移失所。」八月,詔加霖直秘閣。



 宣和元年二月,臣僚言:「江、淮、
 荊、漢間,荒瘠彌望,率古人一畝十鐘之地,其堤閼、水門、溝澮之跡猶存。近絳州民呂平等詣御史臺訴,乞開浚熙寧舊渠,以廣浸灌,願加稅一等。則是近世陂池之利且廢矣,何暇復古哉。願詔常平官,有興修水利功效明白者,亟以名聞,特與褒除,以勵能者。」從之。



 八月,提舉專切措置水利農田所奏:「浙西諸縣各有陂湖、溝港、涇濱、湖濼,自來蓄水灌溉,及通舟楫,望令打量官按其地名、丈尺、四至,並鐫之石。」從之。



 三月,趙霖坐增修水利不當,
 降兩官。六月,詔曰:「趙霖興修水利,能募被水艱食之民,凡役工二百七十八萬二千四百有奇,開一江、一港、四浦、五十八瀆,已見成績,進直徽猷閣,仍復所降兩官。」



 宣和二年九月,以真、揚等州運河淺澀,委陳亨伯措置。三年春,詔發運副使趙億以車畎水運河,限三月中三十綱到京。宦者李琮言:「真州乃外江綱運會集要口,以運河淺澀,故不能速發。按南岸有洩水斗門八,去江不滿一里。欲開斗門河身,去江十丈築軟壩,引江潮入河,然
 後倍用人工車畎,以助運水。」從之。



 四月,詔曰:「江、淮漕運尚矣。春秋時,吳穿邗溝,東北通射陽湖,西北至末口。漢吳王濞開邗溝,通運海陵。隋開邗溝,自山陽至揚子入江。雍熙中,轉運使劉蟠以山陽灣迅急,始開沙河以避險阻。天禧中,發運使賈宗始開揚州古河,繚城南接運渠,毀三堰以均水勢。今運河歲淺澀,當詢訪故道,及今河形勢與陂塘瀦水之地,講究措置悠久之利,以濟不通。可令發運使陳亨伯、內侍譚稹條具措置以聞。」



 八月,
 臣僚言:「比緣淮南運河水澀逾半歲,禁綱舟篙工附載私物,今河水增漲,其令如舊。」



 初,淮南連歲旱,漕運不通,揚州尤甚,詔中使按視,欲浚運河與江、淮平。會兩浙有方臘之亂,內侍童貫為宣撫使,譚稹為制置使,貫欲海運陸輦,稹欲開一河,自盱眙出宣化。朝廷下發運司相度,陳亨伯遣其屬向子諲視之。子諲曰:「運河高江、淮數丈,自江至淮,凡數百里,人力難浚。昔唐李吉甫廢閘置堰,治陂塘,洩有餘,防不足,漕運通流。發運使曾孝蘊嚴
 三日一啟之制,復作歸水澳,惜水如金。比年行直達之法,走茶鹽之利,且應奉權幸,朝夕經由,或啟或閉,不暇歸水。又頃毀朝宗閘,自洪澤至召伯數百里,不為之節,故山陽上下不通。欲救其弊,宜於真州太子港作一壩,以復懷子河故道,於瓜州河口作一壩,以復龍舟堰,於海陵河口作一壩,以復茱萸、待賢堰,使諸塘水不為瓜洲、真、泰三河所分,於北神相近作一壩,權閉滿浦閘,復朝宗閘,則上下無壅矣。」亨伯用其言,是後滯舟皆通利
 云。



 三年二月,詔:「趙之鑒湖,明之廣德湖,自措置為田,下流堙塞,有妨灌溉,致失常賦,又多為權勢所占,兩州被害,民以流徙。宜令陳亨伯究實,如租稅過重,即裁為中制;應妨下流灌溉者,並馳以予民。」



 五年三月,詔:「呂城至鎮江運河淺澀狹隘,監司坐視,無所施設,兩浙專委王復,淮南專委向子諲,同發運使呂淙措置車水,通濟舟運。」



 四月,又命王仲閎同廉訪劉仲元、漕臣孟庾專往來措置常、潤運河。又詔:「東南六路諸閘,啟閉有時。比聞綱
 舟及命官妄稱專承指揮,抑令非時啟版,走洩河水,妨滯綱運,誤中都歲計,其禁止之。」



 五月,詔:「以運河淺涸,官吏互執所見,州縣莫知所從。其令發運司提舉等官同廉訪使者,參訂經久利便列奏。」是月,臣僚言:「鎮江府練湖,與新豐塘地理相接,八百餘頃,灌溉四縣民田。又湖水一寸,益漕河一尺,其來久矣。今堤岸損缺,不能貯水,乞候農隙次第補葺。」詔本路漕臣並本州縣官詳度利害,檢計工料以聞。



 六年九月,盧宗原復言:「池州大江,乃
 上流綱運所經,其東岸皆暗石,多至二十餘處;西岸則沙洲,廣二百餘里。諺云『拆船灣』,言舟至此,必毀拆也。今東岸有車軸河口沙地四百餘里,若開通入杜湖,使舟經平水,徑池口,可避二百里風濤拆船之險,請措置開修。」從之。



 七年九月丙子,又詔宗原措置開浚江東古河,白蕪湖由宣溪、溧水至鎮江,渡揚子,趨淮、汴,免六百里江行之險,並從之。



 靖康元年三月丁卯,臣僚言:「東南瀕江海,水易洩而多旱,歷代皆有陂湖蓄水。祥符、慶歷
 間,民始盜陂湖為田,後復田為湖。近年以來,復廢為田,雨則澇,旱則涸。民久承佃,所收租稅,無計可脫,悉歸御前,而漕司之常賦有虧,民之失業無算。可乞盡括東南廢湖為田者,復以為湖,度幾凋瘵之民,稍復故業。」詔相度利害聞奏。



 八月辛丑,戶部言:「命官在任興修農田水利,依元豐賞格,千頃以上,該第一等,轉一官,下至百頃,皆等第酬獎;紹聖亦如之。緣政和續附常平格,千頃增立轉兩官,減磨勘三年,實為太優。」詔依元豐、紹聖舊格。



\end{pinyinscope}