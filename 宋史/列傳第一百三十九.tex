\article{列傳第一百三十九}

\begin{pinyinscope}

 何鑄王次翁範同楊願樓照勾龍如淵薛弼羅汝楫子願附蕭振



 何鑄,字伯壽,餘杭人。登政和五年進士第,歷官州縣,入為諸王宮大小學教授、秘書郎。御史中丞廖剛薦鑄操
 履勁正,可備拾遺補闕之選。即命對。鑄首陳:「動天之德莫大於孝,感物之道莫過於誠。誠孝既至,則歸梓宮於陵寢,奉兩宮於魏闕,紹大業,復境土,又何難焉。」帝嘉納之。



 拜監察御史,尋遷殿中侍御史。上疏論:「士大夫心術不正,徇虛以掠名,托名以規利。言不由中而首尾向背,行險自售而設意相傾者,為事君之失。懷險□戲之謀,行刻薄之政,輕儇不莊,慢易無禮者,為行己之失。乞大明好惡,申飭中外,各務正其心術,毋或欺誕。」蓋有所
 指也。時遷溫州諸宮殿神像於湖州,有司迎奉,所過騷然。鑄言:「孝莫大於寧神,寧神莫大於得四海之歡心。浙東旱荒,若加勤動,恐道路怨咨。乞務從簡約,不得過為騷擾。」疏奏,其事遂已。擢右諫議大夫。論:「中興之功,在於立志,天下之事濟與否,在於思與不思。願陛下事無大小,精思熟慮,求其至當而行。如是,則事無過舉矣。」尋拜御史中丞。



 先是,秦檜力主和議,大將岳飛有戰功,金人所深忌,檜惡其異己,欲除之,脅飛故將王貴上變,逮飛系大
 理獄,先命鑄鞫之。鑄引飛至庭,詰其反狀。飛袒而示之背,背有舊涅「盡忠報國」四大字,深入膚理。既而閱實俱無驗,鑄察其冤,白之檜。檜不悅曰:「此上意也。」鑄曰:「鑄豈區區為一岳飛者,強敵未滅,無故戮一大將,失士卒心,非社稷之長計。」檜語塞,改命萬俟契。飛死獄中,子雲斬於市。



 檜銜鑄。時金遣蕭毅、邢具瞻來議事,檜言:「先帝梓宮未反,太后鑾輿尚遷朔方,非大臣不可祈請。」乃以鑄為端明殿學士、簽書樞密院事為報謝使。鑄曰:「是行猶
 顏真卿使李希烈也,然君命不可辭。」既返命,檜諷萬俟契使論鑄私岳飛為不反,欲竄諸嶺表,帝不從,止謫徽州。



 時有使金者還,言金人問鑄安在,曾用否。於是復使知溫州。未幾,以端明殿學士提舉萬壽觀兼侍讀,召赴行在,力辭。乃再遣使金,使事秘而不傳。既歸報,帝復許以大用,又力請祠,除資政殿學士、知徽州。居數月,提舉江州太平興國宮。卒,年六十五。



 鑄孝友廉儉。既貴,無屋可居,止寓佛寺。其辨岳飛之冤,亦人所難。然紹興己未
 以後,遍歷臺諫,所論如趙鼎、李光、周葵、範沖、孫近諸人,未免迎望風旨,議者以此少之。至於慈寧歸養,梓宮復還,雖鑄祈請之力,而金謀蓋素定矣。



 先是,金諸將皆已厭兵欲和,難自己發,故使檜盡室航海而歸,密有成約。紹興以後,我師屢捷,金欲和益堅。至是,遣鑄銜命,蓋檜之陰謀,以鑄嘗爭岳飛之獄,而飛竟死,使金知之而其議速諧也。



 鑄死四十餘年,謚通惠,其家辭焉。嘉定初,改謚恭敏。



 王次翁,字慶曾,濟南人。聚徒授業,齊、魯多從游者,入太學,貧甚,夜持書就旁舍借燈讀之。禮部別頭試第一,授恩州司理參軍,歷婺州教授、闢雍博士,出知道州。



 燕雲之役,取免夫錢不及期,輒以乏興論。次翁檄取屬邑丁籍,視民產高下以為所輸多寡之數,約期受輸,不擾而集。除廣西轉運判官。時劇盜馬友、孔彥舟、曹成更據長沙,帥檄漕司預鳩糧芻三十萬以備調發,次翁即以具報,吏愕眙,次翁曰:「兵未必發,先擾民可乎?吾以一路常
 平上供計之,不啻三十萬。」已而賊不犯境。召對,論事不合,出知處州,乞祠,歸寓於婺。



 呂頤浩帥長沙,闢為參謀官。頃之,力乞致仕。秦檜召還,道出婺,次翁見之。樓照言:「頤浩與次翁同郡,頤浩再相,次翁貧困至此。」檜笑曰:「非其類也。」檜居朝,遂以為吏部員外郎,遷秘書少監,除起居舍人,遷中書舍人。劉光世除使相,奏以文資蔭其子,次翁執奏繳還。



 除工部侍郎兼侍講。蜀闕帥,宰執擬次翁以聞。帝以次翁明經術,留兼資善堂翊善。改御史中
 丞。論趙鼎不法,罷知泉州。部差李泗為鄂州巡檢,而湖北宣撫使不可,次翁言:「法令沮於下,而不知朝廷之尊,漸不可長。」帝令詰宣撫司。宣贊舍人陳諤、孫崇節即閣門受旨升轉,次翁言:「閣門徑自畫旨,不由三省,非祖宗法。」寢弗命。呼延通因內教出不遜語,次翁乞斬通以肅軍,且言:「著令,寸鐵入皇城者有常刑。」遂罷內教。



 韓世忠與劉光世、張俊與劉錡皆不相能,次翁言:「世忠於光世因言議有隙,俊於錡由措置有睽。竊恐錡保一孤壘,光
 世軍處窮,獨俊與世忠不肯急援。願遣使切責,因用郭子儀、李光弼以忠義泣別相勉者感動之。」



 金人敗盟入侵,次翁為秦檜言於帝曰:「前日國是,初無主議,事有小變則更用他相,後來者未必賢於前人,而排斥異黨,收召親故,紛紛非累月不能定,於國事初無補。願陛下以為至戒,無使小人異議乘間而入。」檜德之。先是,檜兄子與其內兄王



 □奐皆以恩幸得官,檜初罷政,二人擯斥累年。至是,次翁希檜旨,言:「吏部之有審量,皆暴揚君父過
 舉,得無傷陛下孝治。乞悉罷建炎、紹興前後累降指揮。」由是二人驟進。



 初,次翁既論罷趙鼎,鼎歸會稽,上書言時政。檜忌鼎復用,乃令次翁又言之,乞顯置於法。且言:「特進乃宰相階官,鼎雖謫降,而階官如故,是未嘗罷相也。」遂降散官,謫居興化軍。右諫議大夫何鑄又論鼎罪重罰輕,降朝奉大夫,移漳州。檜意猶未厭,次翁又論:「鼎聞邊警,喜見顏色。繩以漢法,當伏不道之誅;責以《春秋》,當坐誅意之罰。雖再行貶責,然朝奉大夫視中大夫品
 秩不相遼,漳州比興化尤為善地,以此示罰,人將玩刑。」再移潮州安置。



 次翁除參知政事。兩浙轉運司牒試,主司觀望,檜與次翁子侄預選者數人,士論大駭。金人敗於柘皋,帝曰:「將帥成不戰劫敵之功,乃輔弼奇謀指縱之力。」除一子職名。



 檜召三大將論功行賞,岳飛未至。檜與次翁謀,以明日率世忠、俊置酒湖上,欲出,則語直省官曰:「姑待岳少保來。」益令堂廚豐其燕具,如此展期以待者六七日。飛既至,皆除樞密使,罷兵柄。次翁歸語其
 子伯庠曰:「吾與秦相謀之久矣。」



 太后回鑾,次翁為奉迎扈從禮儀使。初,太后貸金於金使以犒從者,至境,金使責償乃入。次翁以未得檜命,且懼檜疑其私相結納,欲攘其位,堅不肯償,相持境上凡三日,中外憂慮,副使王□奐裒金與之。太后歸,泣訴於帝曰:「王次翁大臣,不顧國家利害,萬一有變,則我子母不相見矣。」帝震怒,欲暴其罪誅之。次翁先白檜謂所以然者,以未嘗稟命,故不敢專。檜大喜,力為營救,奏為報謝使以避帝怒。



 使還,帝立
 中宮,奏為冊寶副使,帝終惡之。檜諭次翁辭位,遂以資政殿學士奉祠,引年歸,居明州。檜憐之,饋問不絕。十九年,卒,年七十一,贈宣奉大夫,諸子婿親戚族人添差浙東者又數人,皆檜為開陳也。檜擅國十九年,凡居政府者,莫不以微忤出去,終始不二者,惟次翁爾。



 範同,字擇善,建康人。登政和五年第,再中宏詞科,累官至吏部員外郎。與秦檜力主和議。紹興八年,假太常少卿接伴金使蕭哲、張通古入境,同北向再拜,問金主起
 居,軍民見者多流涕。除中書門下省檢正諸房公事,權吏部侍郎兼實錄院修撰,遷給事中。



 十一年,檜再主和議,患諸將難制,同獻計於檜,請皆除樞府,罷其兵權。檜喜,乃密奏以柘皋之捷,召三大將赴行在,論功行賞。同入對,帝命與林待聘分草三制,世忠、俊樞密使,飛副使,並宣押赴樞府治事。張俊與檜意合,且覺朝廷欲罷兵權,即首納所統兵。帝召同入對,復以同為翰林學士,俄拜參知政事兼修實錄。



 同始贊和議,為檜所引,及在政
 府,或自奏事,檜忌之。萬俟契因論:「同貳政之初,首為遷葬之議,自建康至信州,調夫治道,怨嗟籍籍。近朝廷收天下兵柄,歸之宥密,同輒於稠人中貪天功以為己有。」遂罷與祠。檜意未已,契再論,責授左朝奉郎、秘書少監,謫居筠州。



 十四年,復朝奉大夫,提舉江州太平觀,移池州。十八年,復太中大夫、知太平州。卒,年五十二。



 楊願,字原仲。宣和末,補太學錄。二帝北遷,金人聞願名,索之,願匿民間。上書執政,請迎復元祐皇后。又奔濟州
 元帥府勸進,闢為屬。



 高宗即位,以元帥府結局恩,授修職郎,御營司闢機宜文字。歷新昌縣丞、越州判官。秦檜薦之,召改樞密院編修官。登紹興二年進士第,遷計議官。召試館職,罷。主管崇道觀,復除秘書郎。議者謂外任未終,故通判明州。



 檜既專政,召為秘書丞。未幾,拜監察御史。臺長言願資淺,當先歷郎官,改司封員外郎,遷右司,起居舍人兼權中書舍人。初修玉牒,特以命願,願言:「玉牒當載靖康推戴趙氏事,以秦檜建議本末書之。」



 十
 三年,權直學士院,充金國賀正旦接伴使。金使完顏曄入境,猶欲據主席,中使傳宣,曄不迎拜,願以禮折之,皆聽服。及還,就充送伴使。十四年,為御史中丞。逾月,升端明殿學士、簽書樞密院事兼參知政事,仍兼修玉牒。



 十五年罷,提舉太平觀。初,願與張擴並居西掖,一時書命,藉擴潤色。擴詠《二毫筆詩》,願以為誚己,訴於檜,訹御史李文會劾之。高閌侍經筵,帝問張九成安否,翌日,又問檜,檜曰:「九成以唱異惑眾,為臺臣所論,予郡,乃力乞祠。
 觀其意,終不為陛下用。」帝曰:「九成清貧,不可無祿。」檜疑閌薦之,以語願,願又嗾文會攻閌去。藤州守臣言遷客李光作詩諷刺時政,願在中司,傅會其說,謂:「光縱橫傾險,子弟賓客往來吳、越,誘人上書,動搖國是。」光再移謫瓊海。文會既升西府,願覘檜意稍厭,即數其害政,罷之。後二日,願遂補其處。帝與檜論事,因曰:「朕謂進用士大夫,一相之責也。一相既賢,則所薦皆賢。」願曰:「陛下任相如此,蓋得治道之要。」又論史事,檜曰:「靖康圍城中,失節
 者相與作私史,公肆擠排。」帝曰:「卿不推異姓,宜其不容。」願曰:「檜非獨是時不肯雷同,宣和間耿延禧為學官,以其父在東宮,勢傾一時,士皆靡然從之,以徼後福,獨檜守正不易。」蓋自檜再居相位,每薦執政,必選世無名譽、柔佞易制者。願希檜意迎合,附下罔上,至是斥去,天下快之。



 又三年,起知宣州。玉牒書成,加資政殿學士,移建康府。二十二年,卒,年五十二。



 初,願守宣城,表弟王炎調蘄水令,過之,醉中謂願曰:「嘗於呂丞相處得公頃歲所
 通書,其間頗及秦丞相之短,尚記憶否?」願聞之,色如死灰,遂留炎不聽去。會願移守金陵,宴監司,大合樂,守卒皆怠,炎即青溪得客舟以行,願憂撓而卒。



 樓照,字仲暉,婺州永康人。登政和五年進士第,調大名府戶曹,改西京國子博士、闢雍錄、淮寧府司儀曹事,改尚書考功員外郎。



 帝在建康,照謂:「今日之計,當思古人量力之言,察兵家知己之計。力可以保淮南,則以淮南為屏蔽,權都建康,漸圖恢復。力未可以保淮南,則因長
 江為險阻,權都吳會,以養國力。」於是移蹕臨安。擢右司郎中。時銓曹患員多闕少,自倅貳以下多添差。照言:「光武並省吏員,今縱未能損其所素有,安可置其所本無乎?」



 紹興二年,秦檜罷相,照亦以言者論去。六年,召為左司員外郎,尋遷殿中侍御史。明年,遷起居郎。言:「今暴師日久,財用匱乏。考唐故事,以宰相領鹽鐵轉運使,或判戶部,或兼度支。今宰相之事難行,若參仿唐制,使戶部長貳兼領諸路漕權,何不可之有?內則可以總大計之
 出入,外則可以制諸道之盈虛,如劉晏自按租庸,以知州縣錢谷利病。」詔三省相度措置,卒施行之。又言:「監司、郡守,系民甚切。乞令侍從官各舉通判資序或嘗任監察御史以上可任監司、郡守者一二人。」詔從之,命中書、門下置籍。



 七年,宰相張浚之兄滉賜出身與郡,中書舍人張燾封還,乃命照行,照又封還,而竟為權起居舍人何掄書黃行下,於是燾與照皆請補外,以秘閣修撰知溫州。未幾,除中書舍人,與勾龍如淵並命。如淵入對,帝
 謂之曰:「卿與樓照皆朕所親擢。」尋遷給事中兼直學士院。



 九年,以金人來和傳敕,照草其文,曰:「乃上穹開悔禍之期,而大金報許和之約。割河南之境土,歸我輿圖;戢宇內之干戈,用全民命。」尋兼侍讀,除端明殿學士、簽書樞密院事。繼命往陜西宣諭德意。照奏:「京城統制吳革、知環州田敢、成忠郎盧大受皆以節義,革為範瓊所害,敢、大受為劉豫所殺,乞賜褒恤。」又奏:「陜西諸路陷劉豫,郡縣有不從偽之人,所籍貲產,並令勘驗給還。」照至東
 京,檢視宮室,尋詣永安軍謁陵寢,遂至長安。



 會李世輔自夏國欲歸朝,照以書招之,世輔以二千人赴行在。尋至鳳翔,以便宜命郭浩帥鄜延,楊政帥熙河蘭鞏,吳璘帥鳳翔。照欲盡移川口諸軍於陜西,璘曰:「金人反復難信,今移軍陜右,則蜀口空虛。金若自南山搗蜀,要我陜右軍,則我不戰自屈。當依山為屯,控守要害。」於是璘、政二軍獨屯內地。照又會諸路監司於鳳翔,皆言蜀邊屯駐大軍之久,坐困四川民力,乃下其議,語在《胡世將傳》。



 照還朝,以親老求歸省於明州,許之,命給假迎侍,仍賜以金帶。十四年,以資政殿學士知紹興府,過闕入見,除簽書樞密院事兼權參知政事。尋為李文會、詹大方所劾,與祠。久之,除知宣州,徙廣州,未行而卒,年七十三。後謚襄靖。



 照早附蔡京改秩,為臺諫所論。其後立朝至位二府,皆與秦檜同時。其宣諭陜西,妄自尊大,或者論其好貨失將士心雲。



 勾龍如淵,字行父,永康軍導江人。勾姓本出古勾芒,高
 宗即位,避御名,更勾龍氏。政和八年,登上舍第。沉浮州縣二十年,以張浚薦,召試館職。



 紹興六年,除秘書省校書郎。歷著作佐郎、祠部員外兼禮部、起居舍人。嘗進所為文三十篇,帝曰:「卿文極高古,更令平易盡善。」後因進對,帝復言:「文章平易者多淺近,淵深者多艱澀,惟用意淵深而造語平易,此最難者。」



 八年,兼給事中、同知貢舉,除中書舍人兼侍讀,兼直學士院。面命草趙鼎罷相制,如淵言:「陛下既罷鼎,則用人才須聳動四方,當速召君
 子,顯黜小人。」帝曰:「君子謂誰?」曰:「孫近、李光。」「小人謂誰?」曰:「呂本中。」先是,祠臣曾開以老病辭不草國書,帝欲用如淵代之,而趙鼎薦本中,故如淵憾之。



 又言:「臣觀朝廷事,非君臣情通,未易能濟。大臣於事稍有過差,陛下訓飭之可也。陛下所欲為,勢有未可,大臣亦當明白辯論。然必陛下先與大臣言及此意,若不先言,即大臣論一事不從,尚未之覺,至再至三,遂以為陛下疏之,或疑他人有以間之。既以懷疑,即不能盡誠,陛下察其不誠,又從
 而疑之,安有君臣之間,動相疑間而能久於其位者?願陛下明諭之。」帝曰:「前此未常有以此告朕者,卿見秦檜亦宜語此。」時檜方得君,如淵猶恐委檜未專,故及之。除御史中丞。



 先是,檜力主和,執政、侍從及內外諸臣皆以為非是,多上書諫止者,檜患之。如淵為檜謀曰:「相公為天下大計,而邪說橫起,盍不擇人為臺諫,使盡擊去,則相公之事遂矣。」檜大喜,即擢如淵中司。



 如淵言:「凡事必有初,及其初而為之則易,無其端而發之則難。陛下即
 位,一初也;渡江,二初也;移蹕建康,三初也;自建康復還臨安,四初也。自趙鼎相,劉大中、王庶相繼去,今復獨任一相,召一二名士,凡事有當行而弊有當去者,又一初也,臣願以正紀綱、辨邪正、明賞罰、謹名器、審用度、厚風俗、去文具七者為獻。」



 又言:「孟庾召節在途,士論不與。」帝曰:「朕欲遣令使金國,在廷莫更有小人否?」對曰:「如趙鼎為相,盡隳紀綱,乃竊賢相之名而去。王庶在樞府,盡用奸計,乃以和議不合,賣直而去。劉大中以不孝得罪,乃
 竊朝廷美職而去。」帝曰:「卿胡不論?」對曰:「目今士論見孟庾之召,王庶之去,已有『一解不如一解』之語。願陛下不惜孟庾一人,以正今日公論,其它容臣一一為陛下別白之。」於是出庾知嚴州。又連論庶、大中,皆罷之。



 金國遣二使來議和,許歸河南地。使者踞甚,議受書之禮不決,外議洶洶。如淵建議取其書納禁中。於是同諫長請對,又呼臺吏問:「朝廷有大議論,許臺諫見宰執商議乎?」吏曰:「有。」遂赴都堂與宰執議取書事,宰執皆以為然。帝親
 筆召如淵、李誼入對。明日,詔宰執就館見金使,受其書納入,人情始安。



 九年,奏召還曾開、範同,而罷施庭臣、莫將,以謂:「開、同之出,雖曰語言之過,而其心實出於愛君;庭臣、將之遷,雖曰議論之合,而其跡終近於希進。今國論既定,好惡黜陟,所宜深謹。」又論張邦昌時偽臣因赦復職非是。帝曰:「卿言是也,朕亦欲置此數匹夫不問。」對曰:「將恐無以示訓。」其後卒不行。



 忽一日,如淵言:「和議之際,臣粗自效,如臣到都堂,若不遏朝廷再遣使之議,則和
 議必至於壞,而宣對之日,稍有將順,則遂至於屈。臣於二者,粗有報國之忠。臣親老,願求歸。」帝不許。如淵疑帝有疏之之意,又奏曰:「臣向薦君臣腹心之論,陛下大以為然。其後秦檜在和議可否未決之間欲求去,陛下頗罪之,臣再三為檜辨析。今陛下與檜君臣如初,而臣反若有讒訴於其間者。」帝曰:「朕素不喜讒,卿其勿疑。」如淵嘗與施庭臣忿爭,庭臣謂如淵有指斥語,帝謂秦檜曰:「以朕觀之,庭臣之罪小,如淵之罪大。」檜請斥庭臣而徙
 如淵,待其求去然後補外。帝不可,於是與庭臣皆罷。



 初,如淵與莫將及庭臣皆力主和議,如淵緣此擢中司,而將及庭臣緣此皆峻用。張燾、晏敦復上疏專以三人為言。如淵入言路,即劾二人,至是與庭臣俱罷。其後檜擬如淵知遂寧府,帝曰:「此人用心不端。」遂已。兩奉祠,卒,年六十二。



 如淵始以張浚薦召,而終乃翼秦檜擠趙鼎,仇呂本中,逐劉大中、王庶,心跡固可見矣。子佃、僎、似。



 薛弼,字直老,溫州永嘉人。登政和二年進士第,調懷州
 刑曹、杭州教授。初頒《五禮》《新書》,定著釋奠先聖誤用下丁,弼據禮是正,州以聞,詔從其議。監左藏東庫。內侍王道使奴從旁禮絹美惡,多取之,弼白版曹窮治,人嚴憚之。



 靖康初,金兵攻汴京,李綱定議堅守,眾不悅。弼意與綱同,圍解,遷光祿寺丞。嘗言:「姚平仲不可恃。」未幾而敗。綱救太原,弼言:「金必再至,綱不當去,宜先事河北。」金人果再入。始命刑部侍郎宋伯友提舉河防,弼以點檢糧草從之,為計畫甚切,皆不能用,乃乞罷歸,改三門、白波
 輦運,尋主管明道宮,提舉淮東鹽事,改湖南運判。



 楊么據洞庭,寇鼎州,王□燮久不能平,更命岳飛討之。么陸耕水戰,樓船十餘丈,官軍徒仰視不得近。飛謀益造大舟,弼曰:「若是,則未可以歲月勝矣。且彼之所長,可避而不可斗也。今大旱,湖水落洪,若重購舟首,勿與戰,逐筏斷江路,蒿其上流,使彼之長坐廢,而精騎直搗其壘,則破壞在目前矣。」飛曰:「善。」兼旬,積寇盡平,進直秘閣。時道殣相望,弼以聞,帝惻然,命給錢六萬緡、廣西常平米六萬
 斛、鄂州米二十萬斛振之,且使講求富弼青州荒政,民賴以蘇。



 王彥自荊移襄,遷延不即赴。彥所將八字軍皆中原勁卒,朝廷患其恣橫,以弼直徽猷閣代之。彥殊不意,弼徑入府受將吏謁,大駭。弼曲折譬曉,彥感悟,即日出境。



 除岳飛參謀官。飛母死,遁於廬山,張宗元攝飛事。飛將張憲移疾,部曲洶洶,生異語。弼謂諸將曰:「太尉力乞張公,而詔使隨至,岳軍素整,今而嘩哄,是汝曹累太尉也。」諸將以諗憲,憲佯悟曰:「相公腹心,惟參謀知之。」眾
 乃定。除戶部郎官,再知荊南。



 桃源劇盜伍俊既招安,復謀叛,提點刑獄萬俟契不能制,乃以委弼,弼許俊以靖州。俊喜曰:「我得靖,則地過桃源遠矣。」俊至,則斬以徇。遷秘閣修撰、陜西轉運使,以左司郎官召知虔州,移黃州。



 時福州大盜有號「管天下」、「伍黑龍」、「滿山紅」之屬,其眾甚盛,鈐轄李貴為賊所獲。,民作山砦自保。守臣莫將議委漳、泉、汀、建,募強壯游手各千人為效用,與殿司統制張淵同措置。未及行,詔升弼集英殿修撰,與將兩易。弼至
 郡,漕臣以游手易聚難散,恐為他日患,聞於朝。事下弼議,弼謂:「昔守章貢,有武夫周虎臣、陳敏者,丁壯各數百,皆能戰,視官軍可一當十。」乃奏虎臣為副將,敏為巡檢,選丁壯千人,號「奇兵」,日給糗糧,責以滅賊。自是歲費錢三萬六千餘緡、米九千石,凡四年而賊平。弼知廣州,擢敷文閣待制。卒,年六十三。



 初,秦檜居永嘉,弼游其門。弼在湖北除盜,歸功於萬俟契。檜誣岳飛下吏,契以中司鞫獄,飛父子及憲皆死。朱芾、李若虛亦坐嘗為飛謀議,
 奪職,惟弼得免,且為檜用,屢更事任,通籍從官,世以此少之。



 羅汝楫,字彥濟,徽州歙縣人。登政和二年進士第,監登聞鼓院,遷大理丞、刑部員外郎。奏命官犯公罪,勿取特旨以終惠臣子,又戶口凋耗,宜少寬養子之禁。



 拜監察御史。未逾月,遷殿中侍御史。與中丞何鑄交章論岳飛,罷其樞筦。朱芾、李若虛嘗為飛議曹,主帥有異意而不能諫;又言,飛獄具,寺官聚斷,咸謂死有餘罪,寺丞何彥
 猷、李若樸獨喧然以眾議為非,欲從輕典。皆坐黜。王庶謫道州,郡丞孫行儉以官廨居之,汝楫劾其無忌憚當斥,且令庶徙居。劉子羽知鎮江,上言:「和好非久遠計,宜及閑暇為備。」檜怒,風汝楫論罷之。



 時撫州有兩陳四系獄,誤論輕罪者死,汝楫誦其冤,且言:「獨罪獄官而守卒不坐,非祖宗法。」於是詔天下斷死刑,守以下引囚問姓名、鄉里然後決。又言:「國家駐蹕臨安,淮南不可置度外,當重防海之寄,守長江之要,革竄名賞籍以勸有功。」



 遷
 起居郎兼侍講。帝問:「或謂《春秋》有貶無褒,此誼是否?」對曰:「《春秋》上法天道,春生秋殺,若貶而無褒,則天道不具矣。」帝稱善,嘗曰:「自王安石廢《春秋》學,聖人之旨浸以不明。近世得其要者,惟胡安國與卿耳。」兼權中書舍人,除右諫議大夫。



 有南雄守奏對:「太后之歸,和議之力也,當盡按前言和不便者。」時相是之,驟用為臺官,中外悚懼,多束裝待遣。汝楫言:「皆不當罪,宜以崇寧事黨為戒。」議遂寢。



 遷御史中丞。舊例,中丞、侍御史不並置,乃更侍御
 史。汝楫求去益力,遷吏部尚書,充國信使。除龍圖閣學士、知嚴州。秩滿,請祠,居喪未終而卒,年七十。累贈開府儀同三司。子顥、籲、頡、頌、願、□,皆有文。



 願字端良,博學好古。法秦、漢為詞章,高雅精煉,朱熹特稱重之。有《小集》七卷,《爾雅翼》二十卷。知鄂州,有治績,以父故不敢入岳飛廟。一日,自念吾政善,姑往祠之,甫拜,遽卒於像前。人疑飛之憾不釋云。



 蕭振,字德起,溫州平陽人。幼莊重,不好弄。稍長,能自謀
 學。嘗奉父命董農役隴畝,手不釋卷,其師謂其父曰:「此兒遠大器也。」未冠,游郡庠,既冠,升太學。時有號「三賢」者,推振為首。登政和八年進士第,調信州儀曹。



 時州郡奉神霄宮務侈靡,振不欲費財勞民,與守議不合。會方臘寇東南,距信尤近,守欲危振,檄振攝貴溪、弋陽二邑。既而王師至衢,又檄振督軍餉,振治辦無闕。大將劉光世見而喜之,欲以軍中俘馘授振為賞,振辭曰:「豈可不冒矢石而貪人之功乎!」諸邑盜未息,守復檄振如初。振悉
 意區處,許其自新,賊多降者。守以贓去,振獨為辦行,守愧謝之。



 調婺州兵曹兼功曹。時振婦翁許景衡以給事中召,振祝之曰:「公至朝幸勿見薦。」景衡詢其故,振曰:「今執政多私其親,願為時革弊。」景衡然之。



 時盜賊所在猖獗,婺卒揚言欲叛以應賊,官吏震恐。振選諸邑士兵強勇者幾千人,日習武以備,蓄異謀者稍懼。有一兵官素得軍士心,守疑而罷之,群卒數百人被甲挺刃,斬儀門入。振聞即往,群卒皆羅拜呼曰:「某等屈抑,願兵曹理之。」
 振使之言,厲色叱曰:「細事耳。車駕南巡,大兵咫尺,汝速死耶!可急釋械,當為汝言。」眾拜謝而去。郡守由是益相信,事悉與謀。嘗議城守,振請以錢數萬緡庸工板築,未數月,城壘屹然,一毫無擾。任滿歸,告其親曰:「家世業農,幸有田可力以奉甘旨,振不願仕。」或薦於朝,授婺州教授,改秩,乞祠。



 以執政薦召對,敷奏數事,皆中時病,帝大喜,拜監察御史。明年冬,以親老乞補外,章七上,不許。面奏曰:「臣事親之日短,事陛下之日長。」指心自誓:「今日之
 事父母,乃他日之事陛下也。」遂除提點浙西刑獄,尋召為宗正少卿,俄擢侍御史。



 振本趙鼎所薦,後因秦檜引入臺,時劉大中與鼎不主和議,振遂劾大中以搖鼎。大中既出,振謂人曰:「如趙丞相不必論,盍自為去就。」鼎遂罷。



 後振知紹興府,改兵部,除徽猷閣待制、知湖州。陛辭,奏曰:「國家講和,恐失諸將心,宜遣使撫諭,示以朝廷息兵寬民意。雖兩國通好,戰御之備宜勿弛。」帝曰:「卿欲奉親求便,豈不知朕有親哉?」振曰:「「臣之親所系者一夫也,
 陛下之親所系者天下也。陛下以天下為心,聖孝愈光矣。」帝嘆其忠。將行,白檜曰:「宰相如一元氣,不可有私,私則萬物為之不生。」檜不悅。



 振至州,檜欲取羨餘,振遺檜書,謂:「財用在天下,如血氣之在一身,移左以實右,則病矣。」檜屬以私事,又不克盡從。以親老乞祠,提舉太平觀。後知臺州。海寇勢張,振至,克之。二十二年,以楊煒在獄供涉,鐫徽猷待制,謫居池州。



 初,煒將上書,責李光徇秦檜議和。時振為侍御史,煒見振道書意,振然其言。及振
 知臺州,而煒治邑有聲,每大言無顧忌,振擊節稱善,遂薦煒改秩,又移書於檜從子秦昌時,俾同薦之。屬吏密語振曰:「煒嘗以書責李參政及太師,昌時義不當舉,待制亦不可舉。」振曰:「吾業已許之,豈可中輟。」遂因煒獄中供前事而貶。



 明年,詔除敷文閣待制、知成都府、安撫制置使。軍儲適闕,倉吏以窘告,振奏留對糴米八萬斛以足軍食,以其直歸計所。總計者利在掊克,即先告檜,謂振唱為闕乏之語,風御史劾振要譽,復謫池陽。而總計
 者以譖得蜀帥,既而專用羅織掊克其民,民益思振。



 檜死,語得聞,帝大感悟。亟遣振還成都,父老歡呼蜀道。振至,一切以寬治。或問其故,振曰:「承縱弛,革之當嚴,今繼苛劾,非寬則民力瘁矣。」帝嘉振治行,謂宰臣沉該、湯思退曰:「四川善政,前有胡世將,今有蕭振。」進秩四等,加敷文閣學士。卒於成都府治,年七十二。振兩為蜀守,威行惠孚,死之日,民無老稚,相與聚哭於道。遺表至,帝悼惜之,賻銀五百兩、絹五百匹,贈四官。



 振好獎善類,端人正
 士多所交識,其間有卓然拔出者,迄為名臣。振居瀕江,自父微時,見過客與掌渡者爭,多溺死。振造大舟,傭工以濟,人感其德,相與名其江為蕭家渡云。有文集二十卷。子諴、忱。



 論曰:何鑄、王次翁以下數人者,附麗秦檜,斥逐忠良,以饕富貴,而次翁尤為柔媚,故檜獨憐之,其在位最久。孔子所謂鄙夫患得患失無所不至者,此輩是已。鑄能伸岳飛之枉,雖為可尚,然又為之使金而通問焉,蓋墮其
 術而不悟者,檜之計深哉。



\end{pinyinscope}