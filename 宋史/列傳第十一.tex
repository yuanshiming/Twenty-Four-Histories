\article{列傳第十一}

\begin{pinyinscope}

 王景子廷義王晏郭
 從義曾孫承祐李洪信弟洪義武行德楊承信侯章



 王景,
 萊州掖人,家世力田。景少
 倜儻,善騎射,不事生業,結裏中惡少為群盜。梁大將王檀鎮滑臺,以景隸麾下,與後唐莊宗戰河上,檀有功,景嘗左右之。莊宗入汴,景來降,累遷奉聖都虞候。清泰末,從張敬達圍晉陽,會契丹來援,景以所部歸晉祖。



 天福初,授相州刺史。範延光據鄴叛,屬郡多為所脅從,景獨分兵拒守,晉祖嘉之,遷耀州團練使。及代,曾晉祖幸,留為京城巡檢使,改洺州團練使。開運初,授侍衛馬軍左廂都校。二年,契丹南侵,少帝幸澶淵,恙高行周等大破契丹眾於戚城,遷侍衛馬軍都指揮使、領鄭州防禦使,出為晉州巡檢使、知州事,拜橫海軍節度。契丹至汴,以其黨代景。景歸次常山,聞契丹主殂樂城,即間道歸鎮,斬關而入,契丹遁去。



 漢乾祐初,加同平章事。會契丹饑,幽州民多度關求食,至滄州境者五千餘人,景善懷撫,詔給田處之。



 周祖微時與景善,及即位,加兼侍中。景起身行伍,素無
 智略,
 然臨政不尚刻削,民有訟必面詰之,不至大過即諭而釋去,不為胥吏所搖,由是部民便之。廣順初入朝,民周環等數百人遮道留之不獲,有截景馬鐙者,俄以景為護國軍節度,歲餘,遷鎮鳳翔。顯德
 初,
 封褒國公,加開府階。世宗即位,加兼中書令。先是,秦、鳳陷蜀,州旁蕃漢戶詣闕請收復,世宗命景與向拱率兵出大荼關進討,連陷砦柵,遂命景為西面行營都部署,大破蜀軍於上邽,斬首數萬級。是秋,秦州降。逾年,徙景鎮秦州兼西面緣邊都部署。恭帝即位,進封涼國公。



 宋初,加守太保,封太
 原郡王。建隆二年春來朝,太祖宴賜加等,復以為鳳翔節度、西面緣邊都部署。四年,卒,年七十五。贈太傅,追封岐王,謚元靖。



 初,景之奔晉也,妻坐戮,二子逃獲免。晉這厚,賞萬計,嘗問景所欲,對曰:「臣自歸國,受恩隆厚,誠無所欲。」固問之,景稽顙再拜曰:「臣昔為卒,嘗負胡床從隊長出入,屢過官妓侯小師家,意甚慕之。今妻被誅,誠得小師為妻足矣。」晉祖大笑,即以小師賜景。景甚寵嬖之,後累封楚國夫人。侯氏嘗盜景金數百兩,私遺
 舊人,景知而不責。



 性謙退,折節下士,每朝廷使至,雖卑位必降階送迎,周旋盡禮。左右或曰:「王位尊崇,無自謙抑。」景曰:「人臣重君命,固當如是,我惟恐不謹耳。」初封郡王,朝廷以吏部尚書張昭將命,景尤加禮重,以萬餘緡遺昭。左右或言其過厚,景曰:「我在行伍間,即聞張尚書名,今使於我,是朝廷厚我也,豈可以往例為限耶?」



 景子廷義、廷睿、廷訓。廷訓至驍衛大將軍致仕。



 廷義起家供奉官,改如京副使,以善騎射,周世宗擢為虎捷都虞候,
 遷龍捷右第二軍都校、領珍州刺史。宋初,改內外馬步軍副都軍頭。乾德四年,與韓重贇率師護治滑州靈河新堤。六年,增治京城,又命廷義董其役。開寶二年,加領橫州團練使,從征太原。廷義性勇敢,親彭士乘城,獨免冑,矢中其腦而顛,經宿卒,年四十七。太祖甚惜之,優詔贈建雄軍節度。廷義性驕傲,好誇誕,每言:「我當代王景之子。」聞者感笑之,因目為「王當代」。



 王宴,徐州滕人,家世力田。晏少壯勇無賴,嘗率君冠行
 攻劫。梁末,徐方大亂,屬邑皆為他盜所剽,惟晏鄉里恃晏獲全。



 後唐同光中,應募隸禁軍,累遷奉國小校。



 晉開運末,與本軍都校趙暉、忠衛都校侯章等戍陜州。會契丹至汴,遣其將劉願據陜,恣行暴虐,晏與暉等謀曰:「今契丹南侵,天下洶洶,英雄豪傑固當乘時自奮。且聞太原劉公威德遠被,人心歸服,若殺願送款河東,為天下唱首,則取富貴如反掌耳。」暉等然之。晏乃率敢死士數人夜逾城,入府署,劫庫兵給其徒,遲明,斬願首級府門
 外。眾請暉為帥,章為本城副指揮使、內外巡檢使兼都虞候;遣其子漢倫奉表晉陽。時漢祖雖建號,威聲未振得晏等來歸,甚喜,即日以暉為保平軍節度,章為鎮國軍節度,晏為降州防禦使,仍領舊職。既而暉等表晏始謀功為第一,遷建雄軍節度。漢祖入汴,加同平章事。



 周祖即位,加兼侍中。廣順元年,劉崇侵晉州,晏閉關不出,設伏城上。並人以為怯,競攀堞而登,晏麾伏兵擊之,顛死者甚眾,遂橋遁。遣漢倫追北數埂,斬首百餘級,
 擢漢倫濱州刺史。八月來朝,周祖以家彭城,授武寧軍節度,俾榮其鄉里。三年,周祖征袞州,次張康鎮,晏來朝,獻馬七匹,賜襲衣、金帶。親郊畢,封滕國公,加開府階。世宗即位,加兼中書令。



 初,晏至鎮,悉召故時同為盜者遺以金帛,從容置酒語之曰:「吾鄉素多盜,我與諸君昔嘗為之。後來者固當出諸君之下,為我告諭,令不復為,若不能改,吾必盡滅其族。」由是人安靜,吏民詣闕舉留,請為晏立衣錦碑。世宗初,復請立德政碑。世宗命比
 部郎中、知制誥張正撰文賜之,詔改其鄉里為使相鄉勛德時,私門立戟。未幾,改河南尹、西京留守。顯德三年,移鳳翔節度。六年,從世宗北征,為益津關一路馬車都部署,韓令坤副焉,遂平三關。



 太祖即位,進封趙國公。從征李筠,師還,改安遠軍節度。乾德元年,進封韓國公,上章請老,拜太子太師致仕。每朝會,令綴中書門下班。俄歸洛陽別墅。四年冬,卒,年七十七。廢朝三日,贈中書令。



 初,晏為軍校,與平陸人王興善,其妻亦相為娣姒。晏既
 貴,乃薄興,興不能平。晏妻病,興語人曰:「吾能治之。」晏遽訪興,興曰:「我非能醫,但以公在陜時止一妻,今妓妾甚眾,得非待糟糠之薄,致夫人怏怏成疾耶?若能斥去女侍,夫人之疾可立愈。」晏以為謗己,乃誣以他事,悉案誅其夫妻。



 守西洛日,白重贊鎮河陽,時世宗征淮南,重贊虎並人乘間為寇,因葺城壘,且約晏為援。晏意欲兼有三城,即與漢倫同率兵赴之。重贊聞其來,拒不納,遣人語之曰:「公在陜州已立大功,河陽小城不煩枉駕。」慚不
 能對,遂引兵還。



 郭從義,其先沙陀部人。父紹古,事後唐武皇忠謹,特見信任,賜姓李氏。紹古卒,從義才丱角,莊宗畜於宮中,與諸子齒。明宗與紹古同事武皇,情好款狎,即位,以從義補內職,累遷內園使。



 晉天福初,始復姓郭氏。坐事出為宿州團練副使。丁內艱北歸,遂家太原。漢祖在鎮,表為馬步軍都虞候,屢率師破破契丹於代北。及建大號,從義首贊其謀,擢鄭州防禦使,充東南道行營都虞候,領首
 軍自太行路渡河。



 漢祖入汴,以為河北都巡檢使。杜重威據大名叛,以為行營諸軍都虞候,重威降,為鎮寧軍節度。趙思綰之叛,為行營都部署,賜戎裝、器仗、金帶。師至永興,圍其城,即以從義為永興軍節度。思綰糧盡,城中人相食,從義第書矢上射入城中,說思綰令降,仍表於朝廷,許以華州節制。隱帝從其計,即遣使諭思綰,思綰開門納款。翌日,從義具軍容入城,憩候館中,思綰入謁,即令武士執之,並其黨三百餘人悉斬於市,以功加
 同平章事。周廣順初,加兼侍中,移鎮許州。顯德初,親郊,加檢校太師。世宗將征劉崇,從義適來朝,因請扈從,世宗甚悅,改天平軍節度,即令從符彥卿破契丹於忻口。師還,以功加兼中書令。四年,從征淮南,移鎮徐州。及世宗自迎鑾至泗州,見於行在。恭帝即位,加開府階。



 宋初,加守中書令。太祖征揚州,從義迎謁於路,願扈從,不允。乾德二年,又為河中尹、護國國節度。六年,以疾歸京師。開寶二年,改左金吾衛上將軍。逾年,上章請老,拜太子
 太師致仕。四年,卒,年六十三,贈中書令。



 從義性重厚,有謀略,多技藝,尤善飛白書。初,思綰之叛也,巡檢使喬守溫遁去,姬妾遁事,坐棄市,人皆冤之。從義善擊球,嘗侍太祖於便殿,命擊之。從義易衣跨驢,馳驟殿庭,周旋擊拂,曲盡其妙。既罷,上賜坐,謂之曰:「卿技固精矣,然非將相所為。」從義大慚。



 子守忠、守信。守忠至閑廄副使。守信字寶臣,頗知書,與士
 大夫游,至東上閣門使、知邢州,卒。子世隆為比部員外郎。世隆子昭祐、承祐。昭祐為閣門祗候。



 承祐字天錫,娶舒王元戴女,授西頭供奉官。仁宗為皇太子,承祐補左清道率府率、春坊左謁者,真宗為玉石小牌二,勒銘為戒飭之。帝即位,遷西院副使兼閣道通事舍人、勾當翰林司,遷西上閣門副使。坐盜御酒及用尚方金器除名,岳州編管,徙許州別駕。起為率府率,遷西京作坊使、勾當右騏驥院。院之大校試路馬者,前
 鳴鞭擁禦蓋,承祐代試之,其狂僭如此。進六宅使、象州團練使。承祐性狡獪,緣東宮恩,又憑借王邸親,既廢復用,乃僭言事,或指切人過失,同謂之「武諫官」。真授衛州刺史、知相州,入為群牧副使,改濰州團練使,歷知曹、鄭、澶、鄆、貝州。徙澶州兵馬總管,役卒有異謀者,廉得不待奏,捕斬之。再知澶州,會中使過,遽延入問管軍闕補何人,使者曰:「聞朝廷方擇才武者。」承祐起挽強自玄,左右皆笑。



 入為龍、神衛四廂都指揮使。以父喪,起復真定
 府、定州等路副都總管。諫官歐陽修、餘靖論其非才,改知相州,尋徙大名府副都總管。樞密使杜衍惡承祐驕恣,奏罷軍,為相州觀察使、永興軍副都總管,改知邢州,徙河陽兵馬總管。衍去位,復進為殿前都虞候、並代州副都總管兼知代州,徙邢州。諫官錢明逸言承祐無廉守,邢民素厭苦之,改相州,徙秦鳳路副總管。累遷建武軍節度使、殿前副都指揮使。



 尋以宣徽南院使判應天府,府壁壘不完,盜至卒無以御,承祐始城南關,浚沙、
 濉、盟三河。徙亳州。諫官言承祐祐在應天府給糧不以次,且擅留糧綱,批宣頭,不發戍還兵,越法杖配輕罪,借用翰林器,出入擁旗槍,以禁兵同周衛,體涉狂僭,無人臣禮。罷宣徽南院使,許州都總管,徙節保靜軍、知許州。



 轉運使蘇舜元薦承祐有將帥才,政事如龔、黃。帝謂輔臣曰:「彼庸人,監司乃龔、黃比之,何所取信哉。」改知鄭州,未行,暴疾卒。贈太尉,謚曰密。承祐所至,多興作為煩擾,百姓苦之。



 李洪信,並州晉陽人,漢聖太后弟也。後弟六人,洪信居長,少善騎射。後唐明宗在藩時,隸帳下,及即位,愛將朱弘實總領捧聖軍,弘實擢洪信為爪牙,漸遷小校。應順中,潞王舉兵,少帝弘實而東奔,捧聖軍數百唑行,洪信預焉。及次衛州,少帝與晉高祖遇,因有疑貳,謀害晉祖,其從兵皆亂。時漢祖方護晉祖,洪信以兵應之,獲免。清泰中,又為雍王重美牙校。



 晉初,為興順左廂都指揮使。漢祖統禁軍,遷鎮太原,奏隸麾下。漢祖領陳州
 刺史、左護聖左廂都指揮使,俄加岳州防禦使。從漢祖降鄴,以警扈之勞,授侍衛馬軍都指揮使、領武信軍節度。



 乾祐中,以群小用事,心懷憂懼,白太后求解軍職,出為鎮寧軍節度。歲餘,遷保義軍節度。初,楊邠以元從功臣為方鎮者不諳政務,令三司擇軍將分補諸鎮都押牙、孔目官,或恃以朝選,藩帥難制。洪信聞內難,即召馬步軍都校聶召,奉國軍校楊德、王建、黃全武、楊進、翟本,右牙都校任溫、武,護聖都校康審澄及判官路濤、掌
 書記張洞、都押牙楊昭勍、孔目官魏守恭,悉殺之,誣奏謀逆。



 周廣順初,加同平章事。洪信常以此妄殺自歉,及革命,內不自安。周祖猶以漢太后之故,移鎮京兆。本城兵不滿千,王峻西征至陜州,以援晉州為辭,又取去數百人。及劉崇北遁,遣禁兵千餘屯京兆,洪信益懼,即請入朝,懇辭藩鎮,拜左武衛上將軍。世宗即位,遷左驍衛上將軍。顯德五年,改右龍武軍統軍,從世宗北征,為合流口部署。



 乾德五年,改左驍衛上將軍。開寶五年請老,
 以本官致仕。八年,卒,年七十四。



 洪信無他才術,徒以外戚致位將相。斂財累鉅萬,而吝嗇尤甚。時節鎮皆廣置帳下親兵,惟洪信最寡少。弟洪義。



 洪義本名洪威,避周祖名改焉。漢祖鎮太原,補親校。開國,授護聖左廂都校、領岳州防禦使,遷侍衛馬軍都指揮使、領武信軍節度。



 少帝即位,改鎮寧軍節度。會誅楊邠、史弘肇等,時侍衛步軍都指揮使王殷屯澶州,即遣供奉官孟業繼密詔令洪義殺之,又令護聖都指揮使
 郭崇等害周祖於鄴。洪義素怯懦,慮殷覺,遷延不敢發,遽引業見殷,殷乃錮業,送密詔於周祖。洎周祖起兵,少帝又詔洪義扼河橋,及周祖兵至,洪義就降。漢室之亡,由洪義也。



 廣順初,權知宋州節度,未幾,真拜歸德軍節度,加同平章事,權知許州。歲餘,改鎮安州。顯德初,加檢校太師。世宗即位。中兼侍中,未幾,徙青州。六年夏,遷京兆尹、永興軍節度。恭帝嗣位,加開府階。



 宋初,加兼中書令,移鄜州。乾德五年,代歸。卒年五十九,贈太師。



 武行德,並州榆次人,身長九尺餘,材貌奇偉,家甚貧,常採樵鬻之自給。晉祖鎮並門,暇日,從禽郊外,值行德負薪趨拱於道左,晉祖見其魁岸,又所負薪異常,令力士更舉之,俱不能舉,頗奇之,因留帳下。



 晉天福初,授奉國都頭,遷指揮使,改控鶴指揮使、寧國軍都虞候。開運中,契丹至汴,行德被獲,乃偽請於契丹以自效。契丹信之,方具舟數十艘載鎧甲,令行德率將校軍卒送歸其國。激汴至河陰,行德謂諸將曰:「我輩受國厚恩,而受制於
 契丹,與其離鄉井、投邊塞,為異域之鬼,曷若與諸君驅逐兇黨,共守河陽,姑俟契丹兵退,視天命所屬歸之,建功業,定禍亂,以圖富貴可乎?」眾素服行威名,皆曰:「所向惟命,不敢愛死。」行德即殺契丹監使,分授器甲,由汜水倍道抵河陽。契丹節度使崔廷勛出兵來拒,行德麾眾逆擊,自旦及午殊死戰,廷勛大敗,棄城走。行德遂據河陽,盡以府庫分給將士,因推行德知州事。時契丹兵尚充斥,行德厲士卒,繕甲兵,據上游,士氣益奮,人望歸
 之。



 聞漢祖起太原,即自稱河陽都部署,遣其弟行友間道奉表勸進,漢祖覽奏喜甚,即授行德河陽三城節度。漢祖由晉、絳至洛,行德迎候境上,以所部兵翼至京師,還河陽。



 乾祐中,加同平章事,移真定尹、成德軍節度。廣順實,加兼侍中,俄改忠武軍節度,遷河南尹、西京留守。時禁鹽入城,犯者法至死,告者給厚賞。洛陽民家嫗將入城鬻蔬,俄有僧從嫗買蔬,就筥翻視,密置鹽筥中,少答其直,不買而去。嫗持入城,抱關者搜得鹽,擒以詣府。
 行德見盛鹽補非村嫗所有,疑而詰之,嫗言:「適有僧自城外買蔬,取視久之而去。」即捕僧訊治之,具伏與關吏同誣以希賞。行德釋嫗,斬僧及抱關吏數輩。人畏之若神明,部下凜然。三年,丁外艱,起復。



 顯德初,加開府階,進封譙國公。世宗即位,兼中書令。初,世宗處河東還,次河,以洛陽城頭缺,令葺之。行德率部民萬餘完其城,封邢國公。是秋,代王晏為武寧軍節度,與晏兩換其任。先是,唐末楊氏據淮甸,自甬橋東南決汴,匯為污澤。二
 年,將議南征,遣行德率所部丁壯於古堤疏導之,東達於泗上。及親征,以行德為濠州行營都部署,破淮軍二千餘人於郡境。俄遣率師屯定遠以逼其城,為吳所敗,死者數百人,行德以身免,左授右衛上將軍。五年,下淮南,復授行德保大軍節度兼中書令。恭帝嗣位,進封宋國公。



 宋初,加中書令,進封韓國公,再授忠武軍節度,改封魏國公。乾德二年冬,移鎮安州,加開府儀同三司。開寶二年,入為太子太傅。太平興國三年,以本官致仕。
 四年,卒,年七十二,贈太師。



 楊承信,字守真,其先沙陀部人。父光遠,仕晉至太師、壽王。承信,光遠第二子,幼以父任,自義武軍節院使領蘭州刺史,歷宣武、平盧二軍牙校。



 開運初,光遠以青州叛,少帝遣李守貞等討之,食盡勢窮,承信兄承勛劫其父以降,青州平,光遠死。承信與弟承祚詣闕請死,詔釋之,以承信為右羽林將軍,承祚為右驍衛將軍,放歸,服喪私第,尋安置鄭州。初,光遠送款契丹求援,兵未至而光
 遠降。及契丹來寇,承勛昌為鄭州防禦使,召數其罪殺之。以承信為平戶軍節度,繼父職。仁漢歷安、鄜二州節度,累加檢校太師。



 周廣順初,加同平章事。諸將西討劉崇,承信表求預行。以郊祀恩加開府階,封杞國公。世宗即位,進韓國公。顯德初,征淮南,為濠州攻城副都部署,改壽州北砦都部署兼知行府事。壽州平,累戰功,擢忠正軍節度、同平章事。時徙州治下蔡,承信既增文其城,又遣監軍薛友柔敗淮人六百餘於廬州北。恭帝即位,
 進封魯國公。



 宋初,加兼侍中,來朝,會征李筠,命為澤州西面都部署,筠平,移鎮河中。乾德元年,進封趙國公。二年,卒,年四十四,贈中書令。



 承信身長八尺,美信表,善持論,且多藝能,雖叛臣之子,然累歷藩鎮,刻勵為政而不苛,故能始終富貴。其卒也,蒲民表乞祠之,則其遺愛之在人者可知矣。景德四年,錄其孫松為奉職。



 侯章,並州榆次人。初在並門事後唐莊宗為隊長,明宗朝遷小校。晉開運末,為忠衛指揮使,屯兵陜州,為內外
 馬步軍都指揮使兼三城巡檢使。



 會契丹入中原,與趙暉、王晏謀斬契丹將劉願,送款於漢祖。漢祖入汴,擢為鎮國軍節度。乾祐初,加同平章事,尋移鎮邠州。章居鎮無善政,傲上剝下,以貪猥聞,用見戶為逃,擅其租賦,乃矯奏貧民數千戶負稅租,久禁系不能輸,願以己奉代。時方姑息,詔褒之。副使趙彥鐸有良馬,章欲之不與,誣彥鐸謀逆之,殺之,亦置而不問。俄加檢校太師。



 周初,加兼侍中。廣順二年入朝,獻銀帛,請開宴,周祖謂左右曰:「諸
 侯來朝,天子自當錫宴,以申愷樂,豈俟其貢奉為之耶?」使復賜之。仍令有司自今藩鎮有進奉者勿受。俄賜宴廣政殿,章又獻銀千兩、馬七匹上壽,復不納。三年,授鄧州節度。周祖親郊,加開府階,封申國公。世宗即位,加兼中書令。世宗親征壽陽,命章為攻城水砦都部署,右衛大將軍王璨副之。俄徙西北水砦都部署,再為武勝軍節度。



 建隆元年八月,授太子太師,封楚國公。既罷節鎮,居常怏怏。一日於朝堂與故舊言晉、漢間事,時有輕忽
 章者,章厲聲曰:「當遼主疾作謀歸,有上書請避暑嵩山者,我粗人,以戰鬥取富貴,若此諛佞,未嘗為之。」坐中有慚者。乾德五年卒。



 諭曰:王京輩微時,或至為盜、負薪,遭五代之亂,奮身戎功,重據邊要。宋興,稽顙北響,太祖待以誠信,宜無不自安者。景趨利改圖,乃至滅族。王晏、郭從義遷怒肆忿,誣人以死。侯章在藩邸有剝下之名,李洪義狃於肺腑之戚,而無外凜之志,咎孰甚焉。斯皆亂世之習,有不能盡
 去之者。武行德守洛邑,辯究欺罔,民用畏服,顧不優於諸人耶?



\end{pinyinscope}