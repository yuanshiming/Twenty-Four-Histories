\article{列傳第一百九十四儒林五}

\begin{pinyinscope}

 ○範沖朱震胡安國子寅宏寧



 範沖,字元長,登紹聖進士第。高宗即位,召為虞部員外郎,俄出為兩淮轉運副使。



 紹興中,隆祐皇后誕日,上置
 酒宮中,從容語及前朝事,後曰:「吾老矣,有所懷為官家言之。吾逮事宣仁聖烈皇后,聰明母儀,古今未見其比。曩因奸臣誣謗,有玷聖德,建炎初雖下詔辨明,而史錄未經刪定,無以傳信後世,而慰在天之靈也。」上悚然,亟詔重修神、哲兩朝《實錄》,召沖為宗正少卿兼直史館。沖父祖禹,元祐中嘗修《神宗實錄》,盡書王安石之過,以明神宗之聖。其後安石婿蔡卞惡之,祖禹坐謫死嶺表。至是復以命沖,上謂之曰:「兩朝大典,皆為奸臣所壞,故以
 屬卿。」沖因論熙寧創置,元祐復古,紹聖以降弛張不一,本末先後,各有所因。又極言王安石變法度之非,蔡京誤國之罪。上嘉納之,遷起居郎。



 俄開講筵,升兼侍讀。上雅好《左氏春秋》,命沖與朱震專講。沖敷衍經旨,因以規諷,上未嘗不稱善。會皇子建國公瑗出就傅,首命沖以徽猷閣待制提舉建隆觀,為資善堂翊善,而朱震兼贊讀。詔曰:「朕為宗廟社稷大計,不敢私於一身,選於屬籍,得藝祖七世孫鞠之宮中。茲擇剛辰,出就外傅,宜有端
 良之士以充輔導之官,博觀在廷,無以易汝沖,德行文學,為時正人。乃祖發議嘉祐之初,乃父納忠元祐之際,敷求是似,尚有典刑。顧資善之開,史館經筵,姑仍厥舊。朕方求多聞之益,爾實兼數器之長,施及童蒙,綽有餘裕。蔽自朕志,宜即安之。」時張浚在長沙,亦薦沖、震可備訓導。沖、震皆一時名德老成,極天下之選,上命建國公見翊善、贊讀,皆納拜。俄遷翰林學士兼侍讀,沖力辭,改翰林侍讀學士,用其父故事也。尋以龍圖閣直學士奉
 祠。卒,年七十五。



 沖之修《神宗實錄》也,為《考異》一書,明示去取,舊文以墨書,刪去者以黃書,新修者以朱書,世號「朱墨史」。及修《哲宗實錄》,別為一書,名《辨誣錄》。沖性好義樂善,司馬光家屬皆依沖所,沖撫育之。為光編類《記聞》十卷奏御,請以光之族曾孫宗召主光祀。又嘗薦尹焞自代云。



 朱震,字子發,荊門軍人。登政和進士第,仕州縣以廉稱。胡安國一見大器之,薦於高宗,召為司勛員外郎,震稱
 疾不至。會江西制置使趙鼎入為參知政事,上諮以當世人才,鼎曰:「臣所知朱震,學術深博,廉正守道,士之冠冕,使位講讀,必有益於陛下。」上乃召之。既至,上問以《易》、《春秋》之旨,震具以所學對。上說,擢為祠部員外郎,兼川、陜、荊、襄都督府詳議官。震因言:「荊、襄之間,沿漢上下,膏腴之田七百餘里,若選良將領部曲鎮之,招集流亡,務農種穀,寇來則御,寇去則耕,不過三年,兵食自足。又給茶鹽鈔於軍中,募人中糴,可以下江西之舟,通湘中之
 粟。觀釁而動,席卷河南,此以逸待勞,萬全計也。」



 遷秘書少監兼侍經筵,轉起居郎。建國公出就傅,以震為贊讀,仍賜五品服。遷中書舍人兼翊善。時郭千里除將作監丞,震言:「千里侵奪民田,曾經按治,願寢新命。」從之。轉給事中兼直學士院,遷翰林學士。是時,虔州民為盜,天子以為憂,選良太守往慰撫之。將行,震曰:「使居官者廉而不擾,則百姓自安,雖誘之為盜,亦不為矣。願詔新太守到官之日,條具本郡及屬縣官吏有貪墨無狀者,一切
 罷去,聽其自擇慈祥仁惠之人,有治效者優加獎勸。」上從其言。故事,當喪無享廟之禮。時徽宗未祔廟,太常少卿吳表臣奏行明堂之祭。震因言:「《王制》:『喪三年不祭,惟天地社稷為越紼而行事。』《春秋》書『夏五月乙酉,吉,禘於莊公』,《公羊傳》曰:『譏始不三年也。』國朝景德二年,真宗居明德皇后喪,既易月而除服,明年遂享太廟,合祀天地於圜丘。當時未行三年之喪,專行以日易月之制可也,在今日行之則非也。」詔侍從、臺諫、禮官參議,卒用御史
 趙渙、禮部侍郎陳公輔言,大饗明堂。七年,震謝病丐祠,旋知禮部貢舉,會疾卒。



 震經學深醇,有《漢上易解》云:「陳摶以《先天圖》傳種放,放傳穆修,穆修傳李之才,之才傳邵雍。放以《河圖》、《洛書》傳李溉,溉傳許堅,許堅傳範諤昌,諤昌傳劉牧。穆修以《太極圖》傳周惇頤,惇頤傳程顥、程頤。是時,張載講學於二程、邵雍之間。故雍著《皇極經世書》,牧陳天地五十有五之數,惇頤作《通書》,程頤著《易傳》,載造《太和》、《參兩》篇。臣今以《易傳》為宗,和會雍、載之論,上
 採漢、魏、吳、晉,下逮有唐及今,包括異同,庶幾道離而復合。」蓋其學以王弼盡去舊說,雜以莊、老,專尚文辭為非是,故其於象數加詳焉。其論《圖》、《書》授受源委如此,蓋莫知其所自云。



 胡安國,字康侯,建寧崇安人。入太學,以程頤之友朱長文及潁川靳裁之為師。裁之與論經史大義,深奇重之。三試於禮部,中紹聖四年進士第。初,廷試考官定其策第一,宰職以無詆元祐語,遂以何昌言冠,方天若次之,
 又欲以宰相章惇子次天若。時發策大要崇復熙寧、元豐之制,安國推明《大學》,以漸復三代為對。哲宗命再讀之,注聽稱善者數四,親擢為第三。為太學博士,足不躡權門。



 提舉湖南學事,有詔舉遺逸,安國以永州布衣王繪、鄧璋應詔。二人老不行,安國請命之官,以勸為學者。零陵簿稱二人黨人范純仁客,而流人鄒浩所請托也。蔡京素惡安國與己異,得簿言,大喜,命湖南提刑置獄推治,又移湖北再鞫,卒無驗,安國竟除名。未幾,簿以他
 罪抵法,臺臣直前事,復安國元官。



 政和元年,張商英相,除提舉成都學事。二年,丁內艱,移江東。父沒終喪,謂子弟曰:「吾昔為親而仕,今雖有祿萬鐘,將何所施?」遂稱疾不仕,築室墓傍,耕種取給,蓋將終身焉。宣和末,李彌大、吳敏、譚世勣合薦,除屯田郎,辭。



 靖康元年,除太常少卿,辭;除起居郎,又辭。朝旨屢趣行,至京師,以疾在告。一日方午,欽宗亟召見,安國奏曰:「明君以務學為急,聖學以正心為要。心者萬事之宗,正心者揆事宰物之權。願擢
 名儒明於治國平天下之本者,虛懷訪問,深發獨智。」又言:「為天下國家必有一定不可易之計,謀議既定,君臣固守,故有志必成,治功可立。今南向視朝半年矣,而紀綱尚紊,風俗益衰,施置乖方,舉動煩擾。大臣爭競,而朋黨之患萌;百執窺覦,而浸潤之奸作。用人失當,而名器愈輕;出令數更,而士民不信。若不掃除舊跡,乘勢更張,竊恐大勢一傾,不可復正。乞訪大臣,各令展盡底蘊,畫一具進。先宣示臺諫,使隨事疏駁。若大臣議絀,則參用臺
 諫之言;若疏駁不當,則專守大臣之策。仍集議於朝,斷自宸衷,按為國論,以次施行。敢有動搖,必罰無赦。庶幾新政有經,可冀中興。」欽宗曰:「比留詞掖相待,已命召卿試矣。」語未竟,日昃暑甚,汗洽上衣,遂退。



 時門下侍郎耿南仲倚攀附恩,凡與己不合者,即指為朋黨。見安國論奏,慍曰:「中興如此,而曰績效未見,是謗聖德也。」乃言安國意窺經筵,不宜召試。欽宗不答。安國屢辭,南仲又言安國不臣。欽宗問其狀,南仲曰:「往不事上皇,今又不事
 陛下。」欽宗曰:「渠自以病辭,初非有向背也。」每臣僚登對,欽宗即問識胡安國否,中丞許翰曰:「自蔡京得政,士大夫無不受其籠絡,超然遠跡不為所污如安國者實鮮。」欽宗嘆息,遣中書舍人晁說之宣旨,令勉受命,且曰:「他日欲去,即不強留。」既試,除中書舍人,賜三品服。南仲諷臺諫論其稽命不恭,宜從黜削。疏奏不下,安國乃就職。



 南仲既傾宰相吳敏、樞密使李綱,又謂許景衡、晁說之視大臣升黜為去就,懷奸徇私,並黜之。安國言:「二人為去就,
 必有陳論。懷奸徇私,必有實跡。乞降付本省,載諸詞命。」不報。



 葉夢得知應天府,坐為蔡京所知,落職奉祠。安國言:「京罪已正,子孫編置,家財沒入,已無蔡氏矣。則向為京所引者,今皆朝廷之人,若更指為京黨,則人才見棄者眾,黨論何時而弭!」乃除夢得小郡。



 中書侍郎何𣓨建議分天下為四道,置四都總管,各付一面,以衛王室、捍強敵。安國言:「內外之勢,適平則安,偏重則危。今州郡太輕,理宜通變。一旦以二十三路之廣,分為四道,事得專
 決,財得專用,官得闢置,兵得誅賞,權恐太重。萬一抗衡跋扈,何以待之?乞據見今二十三路帥府,選擇重臣,付以都總管之權,專治軍旅。或有警急,即各率所屬守將應援,則一舉兩得矣。」尋以趙野總北道,安國言魏都地重,野必誤委寄。是冬,金人大入,野遁,為群盜所殺,西道王襄擁眾不復北顧,如安國言。



 李綱罷,中書舍人劉玨行詞,謂綱勇於報國,數至敗衄。吏部侍郎馮澥言玨為綱游說,玨坐貶。安國封還詞頭,以為「侍從雖當獻納,至
 於彈擊官邪必歸風憲。今臺諫未有緘默不言之咎,而澥越職,此路若開,臣恐立於朝者各以好惡脅持傾陷,非所以靖朝著。」南仲大怒,何𣓨從而擠之,詔與郡。𣓨以安國素苦足疾,而海門地卑濕,乃除安國右文殿修撰、知通州。



 安國在省一月,多在告之日,及出必有所論列。或曰:「事之小者,盍姑置之?」安國曰:「事之大者無不起於細微,今以小事為不必言,至於大事又不敢言,是無時而可言也!」



 安國既去逾旬,金人薄都城。子寅為郎,在城
 中,客或憂之,安國愀然曰:「主上在重圍中,號令不出,卿大夫恨效忠無路,敢念子乎!」敵圍益急,欽宗亟召安國及許景衡,詔竟不達。



 高宗即位,以給事中召。安國言:「昨因繳奏,遍觸權貴,今陛下將建中興,而政事弛張,人才升黜,尚未合宜,臣若一一行其職守,必以妄發,干犯典刑。」黃潛善諷給事中康執權論其托疾,罷之。三年,樞密張浚薦安國可大用,再除給事中。賜其子起居郎寅手札,令以上意催促。既次池州,聞駕幸吳、越,引疾還。



 紹興
 元年,除中書舍人兼侍講,遣使趣召,安國以《時政論》二十一篇先獻之。論入,復除給事中。二年七月入對,高宗曰:「聞卿大名,渴於相見,何為累詔不至?」安國辭謝,乞以所進二十一篇者施行。其論之目,曰《定計》、《建都》、《設險》、《制國》、《恤民》、《立政》、《核實》、《尚志》、《正心》、《養氣》、《宏度》、《寬隱》。論《定計》略曰:「陛下履極六年,以建都,則未有必守不移之居;以討賊,則未有必操不變之術;以立政,則未有必行不反之令;以任官,則未有必信不疑之臣。舍今不圖,後悔何及!」
 論《建都》謂:「宜定都建康以比關中、河內,為興復之基。」論《設險》謂:「欲固上流,必保漢、沔;欲固下流,必守淮、泗;欲固中流,必以重兵鎮安陸。」論《尚志》謂:「當必志於恢復中原,祗奉陵寢;必志於掃平仇敵,迎復兩宮。」論《正心》謂:「戡定禍亂,雖急於戎務,而裁決戎務,必本於方寸。願選正臣多聞識、有志慮、敢直言者置諸左右,日夕討論,以宅厥心。」論《養氣》謂:「用兵之勝負,軍旅之強弱,將帥之勇怯,系人君所養之氣曲直何如。願強於為善,益新厥德,使信
 於諸夏、聞於夷狄者,無曲可議,則至剛可以塞兩間,一怒可以安天下矣。」安國嘗謂:「雖諸葛復生,為今日計,不能易此論也。」



 居旬日,再見,以疾懇求去。高宗曰:「聞卿深於《春秋》,方欲講論。」遂以《左氏傳》付安國點句正音。安國奏:「《春秋》經世大典,見諸行事,非空言比。今方思濟艱難,《左氏》繁碎,不宜虛費光陰,耽玩文採,莫若潛心聖經。」高宗稱善。尋除安國兼侍讀,專講《春秋》。時講官四人,援例乞各專一經。高宗曰:「他人通經,豈胡安國比。」不許。



 會除
 故相朱勝非同都督江、淮、荊、浙諸軍事,安國奏:「勝非與黃潛善、汪伯彥同在政府,緘默附會,循致渡江。尊用張邦昌結好金國,淪滅三綱,天下憤鬱。及正位塚司,苗、劉肆逆,貪生茍容,辱逮君父。今強敵憑陵,叛臣不忌,用人得失,系國安危,深恐勝非上誤大計。」勝非改除侍讀,安國持錄黃不下,左相呂頤浩特令檢正黃龜年書行。安國言:「『有官守者,不得其職則去』。臣今待罪無補,既失其職,當去甚明。況勝非系臣論列之人,今朝廷乃稱勝非
 處苗、劉之變,能調護聖躬。昔公羊氏言祭仲廢君為行權,先儒力排其說。蓋權宜廢置非所施於君父,《春秋》大法,尤謹於此。建炎之失節者,今雖特釋而不問,又加選擢,習俗既成,大非君父之利。臣以《春秋》入侍,而與勝非為列,有違經訓。」遂臥家不出。



 初,頤浩都督江上還朝,欲去異己者,未得其策。或教之指為朋黨,且曰:「黨魁在瑣闈,當先去之。」頤浩大喜,即引勝非為助,而降旨曰:「胡安國屢召,偃蹇不至,今始造朝,又數有請。初言勝非不可
 同都督,及改命經筵,又以為非,豈不以時艱不肯盡瘁,乃欲求微罪而去,其自為謀則善,如國計何?」落職,提舉仙都觀。是夕,彗出東南。右相秦檜三上章乞留之,不報,即解相印去。侍御史江躋上疏,極言勝非不可用,安國不當責。右司諫吳表臣亦言安國扶病見君,欲行所學,今無故罪去,恐非所以示天下。不報。頤浩即黜給事中程瑀、起居舍人張燾及躋等二十餘人,雲應天變除舊布新之象。臺省一空,勝非遂相,安國竟歸。



 五年,除徽猷
 閣待制、知永州,安國辭。詔以經筵舊臣,重閔勞之,特從其請,提舉江州太平觀,令纂修所著《春秋傳》。書成,高宗謂深得聖人之旨,除提舉萬壽觀兼侍讀。未行,諫官陳公輔上疏詆假托程頤之學者。安國奏曰:「孔、孟之道不傳久矣,自頤兄弟始發明之,然後知其可學而至。今使學者師孔、孟,而禁不得從頤學,是入室而不由戶。本朝自嘉祐以來,西都有邵雍、程顥及其弟頤,關中有張載,皆以道德名世,公卿大夫所欽慕而師尊之。會王安石、
 蔡京等曲加排抑,故其道不行。望下禮官討論故事,加之封爵,載在祀典,比於荀、楊、韓氏,仍詔館閣裒其遺書,校正頒行,使邪說者不得作。」奏入,公輔與中丞周秘、侍御史石公揆承望宰相風旨,交章論安國學術頗僻。除知永州,辭,復提舉太平觀,進寶文閣直學士,卒,年六十五。詔贈四官,又降詔加賻,賜田十頃恤其孤,謚曰文定,蓋非常格也。



 安國強學力行,以聖人為標的,志於康濟時艱,見中原淪沒,遺黎塗炭,常若痛切於其身。雖數以
 罪去,其愛君憂國之心遠而彌篤,每有君命,即置家事不問。然風度凝遠,蕭然塵表,視天下萬物無一足以嬰其心。自登第迄謝事,四十年在官,實歷不及六載。



 朱震被召,問出處之宜,安國曰:「子發學《易》二十年,此事當素定矣。世間惟講學論政,不可不切切詢究,至於行己大致,去就語默之幾,如人飲食,其饑飽寒溫,必自斟酌,不可決諸人,亦非人所能決也。吾平生出處皆內斷於心,浮世利名如蠛蠓過前,何足道哉!」故渡江以來,儒者進
 退合義,以安國、尹焞為稱首。侯仲良言必稱二程先生,他無所許可。後見安國,嘆曰:「吾以為志在天下,視不義富貴真如浮雲者,二程先生而已,不意復有斯人也。」



 安國所與游者,游酢、謝良佐、楊時皆程門高弟。良佐嘗語人曰:「胡康侯如大冬嚴雪,百草萎死,而松柏挺然獨秀者也。」安國之使湖北也,時方為府教授,良佐為應城宰,安國質疑訪道,禮之甚恭,每來謁而去,必端笏正立目送之。



 自王安石廢《春秋》不列於學官,安國謂:「先聖手所
 筆削之書,乃使人主不得聞講說,學士不得相傳習,亂倫滅理,用夏變夷,殆由乎此。」故潛心是書二十餘年,以為天下事物無不備於此。每嘆曰:「此傳心要典也。」



 安國少欲以文章名世,既學道,乃不復措意。有文集十五卷、《資治通鑒舉要補遺》一百卷。三子,寅、宏、寧。



 寅字明仲,安國弟之子也。寅將生,弟婦以多男欲不舉,安國妻夢大魚躍盆水中,急往取而子之。少桀黠難制,父閉之空閣,其上有雜木,寅盡刻為人形。安國曰:「當有
 以移其心。」別置書數千卷於其上,年餘,寅悉成誦,不遺一卷。游闢雍,中宣和進士甲科。



 靖康初,以御史中丞何𣓨薦,召除秘書省校書郎。楊時為祭酒,寅從之受學。遷司門員外郎。金人陷京師,議立異姓,寅與張浚、趙鼎逃太學中,不書議狀。張邦昌偽立,寅棄官歸,言者劾其離次,降一官。



 建炎三年,高宗幸金陵,樞密使張浚薦為駕部郎官,尋擢起居郎。金人南侵,詔議移蹕之所,寅上書曰:



 昨陛下以親王、介弟出師河北,二聖既遷,則當糾合
 義師,北向迎請。而遽膺翊戴,亟居尊位,斬戮直臣,以杜言路。南巡淮海,偷安歲月,敵入關陜,漫不捍禦。盜賊橫潰,莫敢誰何,元元無辜,百萬塗地。方且制造文物,講行郊報,自謂中興。金人乘虛直搗行在,匹馬南渡,淮甸流血。迨及返正寶位,移蹕建康,不為久圖,一向畏縮遠避。此皆失人心之大者也。



 自古中興之主所以能克復舊物者,莫不本於憤恥恨怒,不能報怨,終不茍已。未有乘衰微闕絕之後,固陋以為榮,茍且以為安,而能久長無
 禍者也。黃潛善與汪伯彥方以乳嫗護赤子之術待陛下,曰:「上皇之子三十人,今所存惟聖體,不可不自重愛。」曾不思宗廟則草莽湮之,陵闕則畚鍤驚之,堂堂中華戎馬生之,潛善、伯彥所以誤陛下、陷陵廟、蹙土宇、喪生靈者,可勝罪乎!本初嗣服,既不為迎二聖之策,因循遠狩,又不為守中國之謀。以致於今德義不孚,號令不行,刑罰不威,爵賞不勸。若不更轍以救垂亡,則陛下永負孝悌之愆,常有父兄之責。人心一去,天命難恃,雖欲羈
 棲山海,恐非為自全之計。



 願下詔曰:「繼紹大統,出於臣庶之諂,而不悟其非;巡狩東南,出於僥幸之心,而不虞其禍。金人逆天亂倫,朕義不共天,志思雪恥。父兄旅泊,陵寢荒殘,罪乃在予,無所逃責。」以此號召四海,聳動人心,決意講武,戎衣臨陣。按行淮、襄,收其豪英,誓以戰伐。天下忠義武勇,必雲合響應。陛下凡所欲為,孰不如志?其與退保吳、越,豈可同年而語哉!



 自古中國強盛如漢武帝、唐太宗,其得志四夷,必並吞掃滅,極其兵力而後
 已。中國禮義所自出也,恃強凌弱且如此。今乃以仁慈之道、君子長者之事,望於兇頑之粘罕,豈有是理哉!今日圖復中興之策,莫大於罷絕和議,以使命之幣,為養兵之資。不然,則僻處東南,萬事不競。納賂則孰富於京室?納質則孰重於二聖?反復計之,所謂乞和,決無可成之理。



 夫大亂之後,風俗靡然,欲丕變之,在於務實效,去虛文。治兵擇將,誓戡大憝者,孝弟之實也;遣使乞和,冀幸萬一者,虛文也。屈己求賢,信用群策者,求賢之實也;
 外示禮貌,不用其言者,虛文也。不惟面從,必將心改,茍利於國,即日行之者,納諫之實也;和顏泛受,內惡切直者,虛文也。擢智勇忠直之人,待御以恩威,結約以誠信者,任將之實也;親厚庸奴,等威不立者,虛文也。汰疲弱,擇壯勇,足其衣食,申明階級,以變其驕悍之習者,治軍之實也;教習兒戲,紀律蕩然者,虛文也。遴選守刺,久於其官,痛刈奸贓,廣行寬恤者,愛民之實也;軍須戎具,徵求取辦,蠲租赦令,茍以欺之者,虛文也。若夫保宗廟、陵
 寢、土地、人民,以此六實者行乎其間,則為中興之實政也。陵廟荒圮,土宇日蹙,衣冠黔首,為血為肉,以此六虛者行乎其間,則為今日虛文。陛下戴黃屋,建幄殿,質明輦出房,雉扇金爐夾侍兩陛,仗馬衛兵儼分儀式,贊者引百官入奉起居,以此度日。彼粘罕者,晝夜厲兵,跨河越岱,電掃中土,遂有吞吸江湖,蹂踐衡霍之意。吾方擁虛器,茫然未知所之。



 君子小人,勢不兩立。仁宗皇帝在位,得君子最多。小人亦時見用,然罪者則斥;君子亦或
 見廢,然忠顯則收。故其成當世之功,貽後人之輔者,皆君子也。至王安石則不然,斥絕君子,一去而不還;崇信小人,一任則不改。故其敗當時之政,為後世之害者,皆小人也。仁宗皇帝所養之君子,既日遠而銷亡矣。安石所致之小人,方蕃息而未艾也。所以誤國破家,至毒至烈,以致二聖屈辱,羿、莽擅朝,伏節死難者不過一二人。此浮華輕薄之害,明主之所畏而深戒者也。



 古之稱中興者曰:「撥亂世,反之正。」今之亂亦云甚矣,其反正而興
 之,在陛下;其遂陵遲不振,亦在陛下。昔宗澤一老從官耳,猶能推誠感動群賊,北連懷、衛,同迎二聖,克期密應者,無慮數十萬人。何況陛下身為子弟,欲北向而有為,將見舉四海為陛下用,期以十年,必能掃除妖沴,遠迓父兄,稱宋中興。其與惕息遁藏,蹈危負恥如今日,豈不天地相絕哉!



 疏入,宰相呂頤浩惡其切直,除直龍圖閣、主管江州太平觀。



 二年五月,詔內外官各言省費、裕國、強兵、息民之策,寅以十事應詔,曰修政事、備邊陲、治軍
 旅、用人才、除盜賊、信賞罰、理財用、核名實、屏諛佞、去奸慝。疏上不報,尋命知永州。



 紹興四年十二月,復召為起居郎,遷中書舍人,賜三品服。時議遣使入雲中,寅上疏言:



 女真驚動陵寢,殘毀宗廟,劫質二聖,乃吾國之大仇也。頃者,誤國之臣遣使求和,以茍歲月,九年於茲,其效如何?幸陛下灼見邪言,漸圖恢復,忠臣義士聞風興起,各思自效。今無故蹈庸臣之轍,忘復仇之義,陳自辱之辭,臣切為陛下不取也。



 若謂不少貶屈,如二聖何?則自
 丁未以至甲寅,所為卑辭厚禮以問安迎請為名而遣使者,不知幾人矣,知二聖之所在者誰歟?聞二聖之聲音者誰歟?得女真之要領而息兵者誰歟?臣但見丙午而後,通和之使歸未息肩,而黃河、長淮、大江相繼失險矣。夫女真知中國所重在二聖,所懼在劫質,所畏在用兵,而中國坐受此餌,既久而不悟也。天下謂自是必改圖矣,何為復出此謬計邪?



 當今之事,莫大於金人之怨。欲報此怨,必殄此仇。用復仇之議,而不用講和之政,使
 天下皆知女真為不共戴天之仇,人人有致死之心,然後二聖之怨可平,陛下人子之職舉矣。茍為不然,彼或願與陛下歃盟泗水之上,不知何以待之?望聖意直以世仇無可通之義,寢罷使命。



 高宗嘉納,云:「胡寅論使事,詞旨剴切,深得獻納論思之體。」召至都堂諭旨,仍降詔獎諭。既而右僕射張浚自江上還,奏遣使為兵家機權,竟反前旨。寅復奏疏言:「今日大計,只合明復仇之義,用賢修德,息兵訓民,以圖北向。儻或未可,則堅守待時。若
 夫二三其德,無一定之論,必不能有所立。」寅既與浚異,遂乞便郡就養。



 始,寅上言:「近年書命多出詞臣好惡之私,使人主命德討罪之詞,未免玩人喪德之失,乞命詞臣以飾情相悅、含怒相訾為戒。」故寅所撰詞多誥誡,於是忌嫉者眾。朝廷辨宣仁聖烈之誣,行遣章惇、蔡卞,皆宰臣面授上旨,令寅撰進。除徽猷閣待制、知邵州,辭。改集英殿修撰,復以待制改知嚴州,又改知永州。



 徽宗皇帝、寧德皇后訃至,朝廷用故事以日易月,寅上疏言:「禮:
 仇不復則服不除。願降詔旨,用喪三年,衣墨臨戎,以化天下。」尋除禮部侍郎、兼侍講兼直學士院。丁父憂,免喪,時秦檜當國,除徽猷閣直學士、提舉江州太平觀。俄乞致仕,遂歸衡州。



 檜既忌寅,雖告老,猶憤之,坐與李光書譏訕朝政落職。右正言章復劾寅不持本生母服不孝,諫通鄰好不忠,責授果州團練副使、新州安置。檜死,詔自便,尋復其官。紹興二十一年卒,年五十九。



 寅志節豪邁,初擢第,中書侍郎張邦昌欲以女妻之,不許。始,安國
 頗重秦檜之大節,及檜擅國,寅遂與之絕。新州謫命下,即日就道。在謫所著《讀史管見》數十萬言,及《論語詳說》,皆行於世。其為文根著義理,有《斐然集》三十卷。



 宏字仁仲,幼事楊時、侯仲良,而卒傳其父之學。優游衡山下餘二十年,玩心神明,不舍晝夜。張栻師事之。



 紹興間上書,其略曰:



 治天下有本,仁也。何謂仁?心也。心官茫茫,莫知其鄉,若為知其體乎?有所不察則不知矣。有所顧慮,有所畏懼,則雖有能知能察之良心,亦浸消亡而
 不自知,此臣之所大憂也。夫敵國據形勝之地,逆臣僭位於中原,牧馬駸駸,欲爭天下。臣不是懼,而以良心為大憂者,蓋良心充於一身,通於天地,宰制萬事,統攝億兆之本也。察天理莫如屏欲,存良心莫如立志。陛下亦有朝廷政事不干於慮,便嬖智巧不陳於前,妃嬪佳麗不幸於左右時矣。陛下試於此時沉思靜慮,方今之世,當陛下之身,事孰為大乎?孰為急乎?必有歉然而餒,惻然而痛,坐起徬徨不能自安者,則良心可察,而臣言可
 信矣。



 昔舜以匹夫為天子,瞽叟以匹夫為天子父,受天下之養,豈不足於窮約哉?而瞽叟猶不悅。自常情觀之,舜可以免矣,而舜蹙然有憂之,舉天下之大無足以解憂者。徽宗皇帝身享天下之奉幾三十年。欽宗皇帝生於深宮,享乘輿之次,以至為帝。一旦劫於仇敵,遠適窮荒,衣裘失司服之制,飲食失膳夫之味,居處失宮殿之安、妃嬪之好,動無威嚴,辛苦墊隘。其願陛下加兵敵國,心目睽睽,猶饑渴之於飲食。庶幾一得生還,父子兄弟
 相持而泣,歡若平生。引領東望,九年於此矣。夫以疏賤,念此痛心,當食則嗌,未嘗不投箸而起,思欲有為,況陛下當其任乎?而在廷之臣,不能對揚天心,充陛下仁孝之志,反以天子之尊,北面仇敵。陛下自念,以此事親,於舜何如也?



 且群臣智謀淺短,自度不足以任大事,故欲偷安江左,貪圖寵榮,皆為身謀爾。陛下乃信之,以為必持是可以進撫中原,展省陵廟,來歸兩宮,亦何誤耶!



 萬世不磨之辱,臣子必報之仇,子孫之所以寢苫枕戈,弗
 與共天下者也;而陛下顧慮畏懼,忘之不敢以為仇。臣下僭逆,有明目張膽顯為負叛者,有協贊亂賊為之羽翰者,有依隨兩端欲以中立自免者,而陛下顧慮畏懼,寬之不敢以為討。守此不改,是祖宗之靈,終天暴露,無與復存也;父兄之身,終天困辱,而求歸之望絕也;中原士民,沒身塗炭,無所赴訴也。陛下念亦及此乎?



 五安石輕用己私,紛更法令,棄誠而懷詐,興利而忘義,尚功而悖道,人皆知安石廢祖宗法令,不知其並與祖宗之道
 廢之也。邪說既行,正論屏棄,故奸諛敢挾紹述之義以逞其私,下誣君父,上欺祖宗,誣謗宣仁,廢遷隆祐。使我國家君臣父子之間,頓生疵癘,三綱廢壞,神化之道泯然將滅。遂使敵國外橫,盜賊內訌,王師傷敗,中原陷沒,二聖遠棲於沙漠,皇輿僻寄於東吳,囂囂萬姓,未知攸底,禍至酷也。



 若猶習於因循,憚於更變,亡三綱之本性,昧神化之良能,上以利勢誘下,下以智術干上。是非由此不公,名實由此不核,賞罰由此失當,亂臣賊子由此
 得志,人紀由此不修,天下萬事倒行逆施,人欲肆而天理滅矣。將何以異於先朝,求救禍亂而致升平乎?



 末言:



 陛下即位以來,中正邪佞,更進更退,無堅定不易之誠。然陳東以直諫死於前,馬伸以正論死於後,而未聞誅一奸邪,黜一諛佞,何摧中正之力,而去奸邪之難也?此雖當時輔相之罪,然中正之士乃陛下腹心耳目,奈何以天子之威,握億兆之命,乃不能保全二三腹心耳目之臣以自輔助,而令奸邪得而殺之,於誰責而可乎?臣
 竊痛心,傷陛下威權之不在己也。



 高閌為國子司業,請幸太學,宏見其表,作書責之曰:



 太學,明人倫之所在也。昔楚懷王不返,楚人憐之,如悲親戚。蓋忿秦之以強力詐其君,使不得其死,其慘勝於加之以刃也。太上皇帝劫制於強敵,生往死歸,此臣子痛心切骨,臥薪嘗膽,宜思所以必報也。而柄臣乃敢欺天罔人,以大仇為大恩乎?



 昔宋公為楚所執,及楚子釋之,孔子筆削《春秋》,乃曰:「許侯盟於薄,釋宋公。」不許楚人制中國之命也。太母,天
 下之母,其縱釋乃在金人,此中華之大辱,臣子所不忍言也。而柄臣乃敢欺天罔人,以大辱為大恩乎?



 晉朝廢太后,董養游太學,升堂嘆曰:「天下之理既滅,大亂將作矣。」則引遠而去。今閣下自睹忘仇滅理,北面敵國,以茍宴安之事,猶偃然為天下師儒之首。既不能建大論,明天人之理以正君心;乃阿諛柄臣,希合風旨,求舉太平之典,又為之詞云云,欺天罔人孰甚焉!



 宏初以蔭補右承務郎,不調。秦檜當國,貽書其兄寅,問二弟何不通書,
 意欲用之。寧作書止敘契好而已。宏書辭甚厲,人問之,宏曰:「政恐其召,故示之以不可召之端。」檜死,宏被召,竟以疾辭,卒於家。



 著書曰《知言》。張栻謂其言約義精,道學之樞要,制治之蓍龜也。有詩文五卷、《皇王大紀》八十卷。



 寧字和仲,以蔭補官。秦檜當國,召試館職,除敕令所刪定官。秦熺知樞密院事,檜問寧曰:「熺近除,外議云何?」寧曰:「外議以為相公必不為蔡京之所為也。」遷太常丞、祠部郎官。



 初,以寧父兄故召用,及寅與檜忤,乃出寧為夔
 路安撫司參議官。除知澧州,不赴。主管臺州崇道觀,卒。



 安國之傳《春秋》也,修纂檢討盡出寧手。寧又著《春秋通旨》,以羽翼其書云。



\end{pinyinscope}