\article{志第九十四 樂十六(鼓吹下)}

\begin{pinyinscope}

 高宗郊祀大禮五首



 《導引》



 聖皇巡狩,清蹕駐三吳,十世嗣瑤圖。邊塵不動干戈戢,文德溥天敷。灰飛緹室氣潛噓,郊見紫壇初。
 歸來赦令樓前下,喜氣溢寰區。



 《六州》



 雙鳳落,佳氣藹龍山。澄江左,清湖右,日夜海潮翻。因吉地,卜築圜壇。宏基隆陛級,神位周環。邊陲靜,掛起橐鞬,奠枕海隅安。三年親祀,一陽初動,虔修大報,高處紫煙燔。看鳴鑾,鉤陳肅,天仗轉,朔風寒。孤竹管,雲和瑟,樂奏徹天關。嘉籩薦,玉奠璵璠。奉神歡。九霄瑞氣起祥煙,來如風馬欻然還,留福已滋繁。回龍馭,升丹闕,布皇澤,春色滿人間。



 《
 十二時》



 日將旦,陰曀潛消,天宇扇祥飆。邊陲靜謐,夜熄鳴刁,文教普旁昭。興太學,多士舒翹。奉宗祧,新廟榜宸毫,配侑享於郊。慈寧萬壽,四海仰東朝。男女正,中壺致《桃夭》。年屢稔,漕舟銜尾伙,高廩接楹饒。廟堂自有擎天一柱,功比漢庭蕭。多少群工同德,俊乂旁招。吉祥諸福集,燮理四時調。三年郊見,六變奏《咸》、《韶》。望雲霄,降福與唐堯。



 《奉禋歌》



 蒼蒼天色是還非,視下應疑亦若斯。統元氣,
 覆無私。四時寒暑推移,物蕃滋,造化有誰知!嚴大報,反本始,禮重祀神祇。律管灰吹,黃宮動,陽來復,景長時。車陳法駕,仗列黃麾,帝心祗。紫霄霽,霜華薄,星爛明垂。祥煙起,紛敷浮袞冕,六變笙鏞迭奏,一誠幣玉交持。宮漏聲遲,千官顯相多儀。百神嬉,風馬雲車,來止來綏,誕降純禧。受神策,萬年無極,歌頌《昊天成命》周詩。



 《降仙臺》



 升煙既罷,良夜未曉,天步下神丘。鏘鏘
 鳴玉佩,煒煒照金蓮,杳靄雲裘。彩仗初轉,回龍馭,旌旆悠悠。星影疏動與天流,漏盡五更籌。大明升,東海頭。杲杲靈曜,倒影射旗旒。輦路具修,鬱蔥瑞光浮。歸來雙闕,看禦樓,有仙鶴銜書赦囚。萬方喜氣,均祉福,播歌謳。



 孝宗郊祀大禮五首



 《導引》



 重華天子,長至奉神虞,九奏會軒、朱。星暉雲潤東方曉,拜貺竹宮初。歸來千乘護皇輿,瑞景集金鋪。
 雞竿高唱恩書下,惠露匝中區。



 《六州》



 嚴更永,今夕是何年?玉衡正,鉤陳燦,天宇起祥煙。協風應,江海安瀾。重規仍疊矩,聖主乘乾。舜授禹,盛事光前,稱壽玉卮邊。三年親祀,一陽回律,八鄉承宇,觚陛紫為壇。仰天顏,齋居寂,誠心肅,禮容專。鳴鐘石,擁輿衛,五輅列駢闐。聽金鑰,虎旅無眠。儼千官,須期顯相嘉籩。一人儉德動天淵,費減大農錢。神示格,宗祧燕,人民悅,祉福正綿綿。



 《
 十二時》



 庭有燎,疊鼓鳴鼉,更問夜如何?信星彪列,天象森羅。虞旦閟宮,畢觴清廟,漿柘樽犧繼猗那,嘉頌可同科。扈聖萬肩摩。飭躬三宿,泰畤縟儀多,丘澤合,岳瀆從羲、和。神光燭,雲車風馬,芝作蓋,玉為珂。奉瑄成禮,燔柴竣事,休嘉砰隱,丹闕湛恩波。共願乾坤隤祉,邊鄙投戈。覆盂連瀚海,洗甲挽天河。欣欣喜色,長遇六龍過。奏雲和,三春薦嘉禾。



 《奉禋歌》



 吹葭緹鑰氣潛分,云採宜書壤效珍。長日至,
 一陽新。四時玉燭和均,物欣欣,化轉洪鈞。郊之祭,孤竹管,六變舞《雲門》。自古嚴禋,犧牲具,粢盛潔,豆籩陳。袞龍陟降,幣玉紛綸,徹高閽。靈之斿,神哉沛,排歷昆侖。《九歌》畢,盈郊瞻□□燎,鬥轉參橫將旦,天開地闢如春。清蹕移輪,闐然鼓吹相聞。□祥雲,驩臚八階,厘逆三神。聖矣吾君!華封祝,慈宮萬壽,椒掖多男,六合同文。



 《降仙臺》



 漏殘柝靜,雞聲遠到,高燎入層霄。雲裘蟠瑞
 靄,天步下嘉壇,旗旆飄搖。黃麾列仗貔貅整,氣壓江潮。導前從後盛官僚,玉佩間金貂。望扶桑,日漸高,陰霾霜雪,底處不潛消!輦路祥飆,披拂絳紗袍。雲間端闕仰岧嶢,挾春澤,喜浹黎苗。禮成大慶鰲三抃,受昕朝。



 寧宗郊祀大禮四首



 《六州》



 皇撫極,明德貫乾坤。信星列,卿雲爛,輝亙紫微垣。思報貺,明詔祠官,練時搜曠典,紫畤觚壇。昭孝德,
 親御和鑾,振鷺玉珊珊。精純謁款,膋蕭爌煬,黃流湛澹,百末布生蘭。扣天閽,延飛駕,相徬佛,降雲端。神光集,嘉響應,靄靄萬衣冠。竣熙事,清曉輕寒。恣榮觀,華衣霧縠般般。乾坤並貺慶君歡,翹首聖恩寬。遵皇極,沛天澤,靈心懌,龜鼎永尊安。



 《十二時》



 宵景霽,河漢清夷,曠典講明時。合祛升侑,孝德爰熙。陳稞閟宮,澹觴太室,來奏天儀。駟蒼螭,玉輅馭蕤綏。觚陛展躬祠。長梢飾玉,翠羽秀金支。華始倡,
 雅韻出宮垂。神來下,雲車風馬,繽晻藹,宴棲遲。畢觴流胙,柴煙竣事,棠梨回謁,宣室受蕃厘。盛德無心專饗,端為民祈。云恩有截,雨澤霈無涯。君王愉樂,和氣溢瑤卮。壽天齊,長擁神基。



 《奉禋歌》



 葭飛璇鑰初陽,雲絕清臺薦景祥。風應律,日重光。歲功順,底金穰。壽而康,庭壺樂無疆。皇展報,新禮樂,觚陛詠賓鄉,珠幄熉黃。登瑞絲巢,陳俎豆,澹嘉觴。袞衣輝煥,寶佩琳瑯,奠椒漿。慶陰陰,神來下,鳳翥
 龍驤。靈燕喜,錫符仍降嘏,鏞管琳瑯歡亮。神之出,祓蘭堂。輦路天香,輕煙半襲旗常,祉滂洋。受厘宣室,返馭齋房,恩與風翔。華封祝,皇來有慶,八荒同壽,寶歷無疆。



 《降仙臺》



 星芒收採,雲容放曉,羲馭漸揚明。觚壇竣事霽,風襲袞衣輕,鑾路塵清。甘泉鹵簿祲威肅,回軫旋衡。千官導從粲簪纓,鈞奏間《韶》、《英》。瞻龍闈,近鳳城。都人雲會,芬茀夾道歡迎。宸極尊榮。卮玉慶熙成,瓊樓
 天上起和聲。布春澤,洪暢寰瀛。嵩呼萬歲鰲三抃,頌升平。



 明堂大禮四首



 《合宮歌》



 聖明朝,曠典乘秋舉,大饗本仁祖。九室八牖四戶,敕躬齊戒格堪輿。盛牲實俎,並侑總稽古。玉露乍肅天宇,冰輪下照金鋪。燎煙噓,鬱尊香,《雲門》舞。徬佛翔坐,靈心咸嘉娛。眾星俞,美光屬,照熉珠。清曉禦丹儀,湛恩遍浹率溥,歡聲雷動岳鎮呼。徐命法駕,萬
 騎花盈路。萬姓齊祝,壽同天地,事超唐、虞。看平燕云,從此興文偃武,待重會諸侯舊東都。



 《六州》



 商秋肅,嘉會協中辛。涓路寢,修禋祀,聖德昭清。端志慮,罄竭齋精。錦繡排天仗,羽衛繽紛。朝太室,返中宸,被袞接神明。時平天地俱清晏,兼金行萬寶,物盛藹清馨。瞻熉座,舂容娭燕三靈。奠瑤爵,薦量幣,清思窈冥冥。望昆侖,輸嘉祥,塞絪縕。誠殫禮洽慶休成,潤澤被生民。端門肆眚,昕庭稱賀,俱將戩穀萬壽祝
 明君。



 《十二時》



 炎圖鞏,天祚昌期,聖德茂重離。英明經遠,浚哲昭微。寶儉更深慈。觀萬國累洽重熙。對時報禮秩神祇,玉帛湊華夷。肅雍顯相,百闢盡欽祗。奄嘉虞,英璧奠華滋。神安坐,景氣澄虛極,光焰燭長麗。展詩應律,萬舞逶遲,三獻洽皇儀。垂靈祲,慶祐來宜,禮無違。鳴鑾臨帝闕,飛鳳下天倪。清和寰宇,霈澤一朝馳。醇化無為,萬祀鞏丕基。



 《
 導引》



 合宮親饗,青女肅長空,精意與天通。後皇臨顧誰為侑?文祖暨神功。函蒙祉福歲常豐,聲教被華戎。兩宮眉壽同榮樂,戩穀永來崇。



 乾道發太上皇帝、太上皇后冊寶一首



 《導引》



 重華真主,晨夕奉庭闈,禋祀慶成時。乾元坤載同歸美,寶冊兩光輝。斑衣何似赭黃衣,此事古今稀。都人歡樂嵩呼震,聖壽總天齊。



 淳熙發太上皇帝、太上皇后冊寶一首



 《
 導引》



 新陽初應,樂事起彤庭,和氣滿吳京。家來慶東皇壽,西母共長生。金書玉篆粲龍文,前導沸歡聲。修齡無極名無盡,一歲一回增。



 加上太上皇帝、太上皇后冊寶一首



 《導引》



 皇家多慶,親壽與天長,德業播輝光。焜煌寶冊來清禁,玉篆映金相。庭闈尊奉會明昌,佳氣溢康莊。洪禧申輯名增衍,億載頌無疆。



 恭上壽聖皇太后、至尊壽皇聖帝、壽成皇后尊號冊
 寶一首



 《導引》



 皇家盛事,三殿慶重重,聖主極推崇。瑤編寶列相輝映,歸美意何窮。鈞《韶》九奏度春風,彩仗煥儀容。歡聲和氣彌寰宇,皇壽與天同。



 加上壽聖皇太后尊號冊寶一首



 《導引》



 重親萬壽,八帙衍新元,禮典備文孫。溫溫和氣迎長日,寶冊煥瑤琨。徽音顯號自堯門,德行已該存。更期昌算齊箕翼,愈久愈崇尊。



 嘉泰二年加上壽成太皇太后冊寶一首



 《導引》



 思齊文母,盛德比姜、任,擁祐極恩深。湯孫歸美熙鴻號,鏤玉更繩金。虞廷萬闢萃華簪,法仗儼天臨。層闈慶典年年舉,千古播徽音。



 親耕籍田四首



 《導引》



 春融日暖,四野瑞煙浮,柳菀更桑柔。土膏脈起條風扇,宿雪潤田疇。金根轂轉如雷動,羽衙擁貔貅。扶攜老稚康衢滿,延跂望凝旒。斗移星轉,一氣又環
 周,六府要時修。務農重穀人胥勸,耕籍禮殊尤。壇壝岳峙文明地,黛耜駕青牛。雍容南畝三推了,玉趾更遲留。



 《六州》



 昭聖武,不戰屈人兵。干戈戢,烽燧息,海宇清寧。民豐業,歌詠升平。願咸歸畎畝,力穡為氓。經界正,東作西成。農務軫皇情,躬親耒耜,相勸深耕。人心感悅,擊壤沸歡聲。乘鑾輅,羽旗彩仗鮮明。傳清蹕,行黃道,緹騎出重城。仰瞻日表映朱紘,環佩更鏘鳴。百執公
 卿,不辭染屨意專精,準擬奉粢盛。田多稼,風行遐邇,家家給足,胥慶三登。



 《十二時》



 臨寰宇,恭己巖廊,屬意在耕桑。愛民利物,德邁陶唐,躋俗盡淳厖。開千畝,帝籍神倉。舉彞章,祗祓壇場,為農事祈祥。涓辰行禮,節物值春陽。罄齊莊,明德薦馨香。宮禁邃,嬪妃並御侍,穜稑獻君王。中闈表率,陰教逾光。帳殿靄熉黃,梐枑設,翠幕高張,慶雲翔。尊罍陳酒醴,金石奏宮商。神靈感格,歲歲富倉箱。慶
 明昌,行旅不繼糧。



 《奉禋歌》



 吾皇端立太平基,奉祀肅雍格神祇。撫御耦,降嘉種,何辭手攬洪縻。命太史視日,祗告前期。驗穹象,天田入望更光輝。掌禮陳儀,搜鉅典,迎春令,頒宣溫詔,遍九圍,人盡熙熙。仰明時,儼垂衣,佳氣氤氳表厖禧。豐年屢,大田生異粟,含滋吐秀,九種傳圖,盡來丹闕,瑞應昌時。亨運正當攝提,佇見詠京坻。躬稼穡,重耘耔。盛禮興行先百姓,崇本業,憂勤如禹、稷,播在
 聲詩。



 顯仁皇后上仙發引三首



 《導引》



 長樂晚,彩戲萊衣,奄忽夢報仙期。帝鄉渺渺乘鸞去,啼紅嬪御不勝悲,蒼梧煙水杳難追。腸斷處,過江時。銀濤千萬疊,不知何處是瑤池。



 《六州》



 中興運,孝治格升平。回騩馭。弭鳳駕,冊寶初上鴻名。龍樓問寢候雞鳴,更翻來戲彩衣輕。坤躔夜照老人星,金觴上壽,長願燕慈寧。乘雲何處去!愁斷紫
 簫聲。追思金殿,椒壁丹楹。又誰知勤儉仁明,風行化被宮庭。祐聖主,底明時,陰功暗及生靈。離宮晚,花卉娉婷。甲觀高,潮海崢嶸。往事回頭,忽飄零。空留嬪御,掩泣望霓旌。會稽山翠,永祐陵高,而今便是蓬、瀛。



 《十二時》



 炎圖景運正延鴻,文思坐深宮。慈寧大養,樂事時奏宸聰。皇齡永,恩霈下遍寰中。君王垂彩服,嬪御上瑤鐘。年年誕節,就盈吉月,交慶流虹。歡洽意方濃,不覺仙游渺邈,但號泣蒼穹。追慕念音容,詩書慈
 儉,配古追□從。躬行四德,誰知繼《二南》風。移盼俄空,寶金監脂澤塵封。清都遠,帝鄉遙,杳難通。想雲軿還上瀛、蓬。稽山何在?當年禹宅,萬古蔥蔥。歸難堪,潮頭定,海波融。



 顯仁皇后神主祔太廟一首



 《導引》



 返虞長樂,猶是億賓天,何事駕仙軿。簫笳儀衛辭宮闕,移仗入雲煙。於皇清廟敞華筵,昭穆謹承先。千秋長奉烝嘗孝,永享中興年。



 欽宗皇帝一首



 《導引》



 鼎湖龍遠,九祭畢嘉觴,遙望白雲鄉。簫笳淒咽離天闕,千仗儼成行。聖神昭穆盛重光,寶室萬年藏。皇心追慕思無極,孝饗奉烝嘗。



 安穆皇后一首



 《導引》



 鳳簫聲斷,縹緲溯丹丘,猶是憶河洲。熒煌寶冊來天上,何處訪仙游?蔥蔥鬱鬱瑞光浮,嘉酌侑芳羞。琱輿繡幰歸新廟,百世與千秋。



 景靈宮奉安神御三首



 徽宗皇帝《導引》



 中興復古,孝治日昭鴻,原廟飾瑰宮。金璧千門磻萬□,楹桷競穹崇。亭童芝蓋擁旌龍,列聖儼相從。共錫神孫千萬壽,龜鼎亙衡、嵩。



 顯仁皇后《導引》



 坤儀厚載,遺德滿寰中,歸御廣寒宮。玉容如在飆輿遠,長樂起悲風。霓旌絳節下層空,雲闕曉曈曨。真游千載安原廟,聖孝與天通。



 欽宗皇帝《導引》



 深仁厚德,流澤自無窮,仙馭倏
 賓空。衣冠未返蒼梧遠,遙望鼎湖龍。人間仿佛認天容,縹緲五雲中。帝城猶有遺民在,垂淚向西風。



 安恭皇后上仙發引一首



 金殿晚,愁結坤寧。天下母,忽仙升。雲山浩浩歸何處?但聞空際彩鸞聲。紫簫斷後無蹤跡,煙靄夜澄澄。曉夢到瑤城,當時花木正冥冥。



 高宗梓宮發引三首



 《導引》



 寒日短,草露朝晞。仙鶴下,夢雲歸。大椿亭畔蒼
 蒼柳,悵無由挽住天衣。昭陽深,暝鴉飛。愁帶箭,戀恩棲。笳簫三疊奏,都人悲淚袂成帷。



 《六州》



 堯傳舜,盛事千古難並。回龍馭,辭鳳掖,北內別有蓬、瀛。為天子父,冊鴻名,萬年千歲福康寧,春秋不說楚冥靈。萊衣彩戲,漢殿玉卮輕,宸游今不見,煙外落霞明。前回丁未,霧塞神京。正同符光武中興,擎天獨力扶傾。定宗廟,保河山,乾坤整頓庚庚。功成了,脫屣遺榮。訪崆峒,容與丹庭。笑挹塵寰,不留行。吾皇哀
 戀,淚血灑神旌。腸斷濤江渡,明日稽山,暮雲東望元陵。



 《十二時》



 璧門雙闕轉蒼龍,德壽儼祗宮。軒屏正坐,天子親拜天公,儀紳笏,羅鵷鷺,粲庭中。仙家歡不盡,人世壽無窮。誰知雲路,玉京成就,催返璇穹,轉手萬緣空。見說煙霄好處,不與下方同。塵合霧迷蒙,笙簫寥亮,樓閣玲瓏。中興大業,巍巍稽古成功。事去孤鴻,忍聽宵柝晨鐘!靈轝駕,素幃低,杳厖茸。浙江潮,萬神護,
 川後滋恭。因山祗事,崔嵬禹穴,此日重逢。柏城封,愁長夜,起悲風。歌《清廟》,千古誦高宗。



 虞主赴德壽宮一首



 《導引》



 上皇天大,華旦煥堯文,鴻福浩無垠,羽龍俄駕靈輴去,空金巢鼎湖雲。稽山翠擁浙江濆,歸旆卷繽紛。仙游指日嚴升祔,萬載頌高勛。



 祔廟一首



 《導引》



 虞觴奉主,仙馭返皇宮,禮典極欽崇。雲旗前導
 開清廟,龍管咽熏風。巍巍堯父告神功,追慕孝誠通。千秋萬歲中興統,宗祀與天同。



 淳熙十六年高宗神御奉安一首



 《導引》



 中興揖遜,功德仰兼隆,仁澤被華戎。鼎湖俄痛遺弓墮,如日想威容。柔儀懿範與堯同,飆馭儼相從。靈宮真館偕來燕,垂裕永無窮。



 紹熙五年孝宗皇帝虞主還宮一首



 《導引》



 孝宗純孝,前聖更何加!高蹈處重華。丹成仙
 去龍輴遠,越岸暮山遐。波神先為卷寒沙,來往護靈槎。九虞禮舉神祇樂,萬世祐皇家。



 祔廟一首



 《導引》



 吾皇盡孝,宗廟務崇尊,鉅典備彌文。巍巍東向開基主,七世祔神孫。追思九閏整乾坤,寰宇慕洪恩。從今密邇高宗室,千載事如存。



 慶元六年光宗皇帝發引一首



 笳鼓發,雲慘寒空。丹旐去,卷悲風。憂勤六載親幾務,
 有巍巍聖德仁功。褰裳尊處大安宮,荊鼎就,遽遺弓。仙游攀不及,臣民號慟訴蒼穹。



 神御奉安一首



 《導引》



 龜書畀姒,歷數在皇躬,揖遜仰高風。鼎湖龍去遺弓墮,冠劍金巢深宮。塗山齊德翊成功,仙魄蚤賓空。珍臺閑館棲神地,獻饗永無窮。



 寧宗皇帝發引三首



 《導引》



 三弄曉,雲黯天低。攀六引,轉悲淒。儉慈孝哲鐘
 天性,深仁厚澤遍群黎。東西南北傒商霓。功甫就,別宸閨。臣民千古恨,幾時羽衛帶潮歸!



 《六州》



 明天子,昔日丕纂鴻圖。躬道德,崇學問,稽古訓,訪群儒。日親廣廈論唐、虞,講求政治想都俞,君臣一德志交孚。外夷效順,猶自選車徒。仁恩沾四國,固結滿寰區。千年宗社,萬歲規摹。重新天命出乾符,老癃策杖相扶,願觀德化遍方隅。幸無死須臾,謂宜聖壽等嵩呼。遽登雲輿上龍湖,宸居幽寂紫雲孤。宸章寶
 畫,但與日星俱。龍帷鳳翣已載途,忍聽笳鼓嗟籲!



 《十二時》



 弋綈革舄最仁賢,儉德自躬全。尤勤庶政,三十餘年。金風肅,秋漸老,攝調愆。忱恂遍群祀,號泣訴旻天。綴衣將出,神凝玉幾,一夜登仙,弓墮隔蒼煙。七月有來同軌,引綍動靈輇。淒愴淚潸然,行號巷哭,《薤露》聲傳。東城去路,驚濤忍見江船!憔悴山川,不禁簫鼓咽。山陰處,茂林修竹芊芊。望陵宮,應弗遠,金粟堆前。人徒慕戀,百神警侍,盤翥驅先。戴鴻恩,空痛慕,淚
 珠連。千秋歲,功德寄華編。



 神主祔廟一首



 《導引》



 中興四葉,休德繼昭清,王度日熙平。氣調玉燭金穰應,八表頌聲騰。中原圖籍入宸廷,列聖慰真靈。袞龍登廟游仙闕,億萬載尊承。



 寶慶三年奉上寧宗徽號一首



 《導引》



 中興五葉,天子肇明禋,一德格高旻。寧皇至聖功超古,萬國慕深仁。徽稱顯號又還新,功德粲雕鈱。
 乾坤繪畫終難盡,遺澤在斯民。



 莊文太子薨一首



 《導引》



 秋月冷,秋鶴無聲。清禁曉,動皇情。玉笙忽斷今何在?不知誰報玉樓成。七星授轡驂鸞種,人不見,恨難平。何以返霓旌?一天風露苦淒清。



 景獻太子薨一首



 《導引》



 霜月苦,宮鼓冬冬。霓旐啟,鶴闈空。洞簫聲斷知何處,海山依約五雲東。玉符龍節參神閟,昭聖眷,慘
 天容。千古恨無窮,遍山松柏撼悲風。



\end{pinyinscope}