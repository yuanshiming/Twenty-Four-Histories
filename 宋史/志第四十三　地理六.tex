\article{志第四十三 地理六}

\begin{pinyinscope}

 廣南東路廣南西路燕山府路



 廣南東路。府一:肇慶。州十四:廣,韶,循,潮,連,梅,南雄,英,賀,封,新,康,南恩,惠。縣四十三。南渡後,府三:肇慶,德慶,英德。州十一:廣,韶,循,潮,連,封,新,南恩,梅,雄,惠。紹興三十二年,
 戶五十一萬三千七百一十一,口七十八萬四千七百七十四。



 廣州,中,都督府,南海郡,清海軍節度。開寶五年,廢咸寧、番禺、蒙化、游水四縣。大觀元年,升為帥府。舊領廣南東路兵馬鈐轄,兼本路經略、安撫使。元豐戶一十四萬三千二百六十一。貢胡椒、石發、糖霜、檀香、肉豆蔻、丁香母子、零陵香、補骨脂、舶上茴香、沒藥、沒石子。元豐貢沉香、甲香、詹糖香、石斛、龜殼、水馬、鼊皮、藤簟。縣八:南海,望。隋縣。後
 改常康,開寶五年復。



 番禺,上。開寶中,廢入南海。皇祐三年復置。有銀爐鐵場。



 增城,中



 清遠,中。有大富銀場、靜定鐵場、錢糾鉛場。



 懷集,中。有大利銀場。



 東筦,中下。開寶五年,廢入增城。六年復置。有桂角等三銀場,靜康等三鹽場,海南、黃田等三鹽柵。



 新會,下。有千歲錫場、海晏等六鹽場。



 信安。下。本義寧縣,開寶五年,廢入新會。六年,復置。太平興國初,改信安。熙寧五年,省為鎮,入新州新興縣。元祐元年復為縣。紹聖元年,復省為鎮,後復為縣,還隸廣州。



 南渡後,無信安,增縣一:香山,紹興二十二年,以東筦香山鎮為縣。



 韶州,中,始興郡,軍事。元豐戶五萬七千四百三十八。貢絹、鐘乳。縣五:曲江,望。有永通錢監、靈源等三銀場,中子銅場。



 翁源,望。有大湖銀場,大富
 鉛場。



 樂昌,中。有黃坑等二銀場、太平鉛場。仁化,中。開寶五年,廢入樂昌。咸平三年,復置。有大眾、多田二鐵場、多寶鉛場。



 建福。宣和三年,以岑水場析曲江、翁源地置縣。南渡後,無建福,增縣一:乳源。乾道二年,析曲江之崇信、樂昌依化鄉,於洲頭津置。



 監一:永通。



 循州,下,海豐郡,軍事。元豐戶四萬七千一百九十二。貢絹、藤盤。縣三:龍川,望。有大有鉛場。宣和三年,改龍川曰雷鄉。紹興元年復舊。



 興寧,望。晉縣。天禧三年,移治長樂。有夜明銀場。



 長樂。上。熙寧四年,析興寧縣置。有羅翊等四錫場。



 潮州,下,潮陽郡,軍事。元豐戶七萬四千六百八十二。貢蕉布、甲香、鮫魚皮。縣三:海陽,望。有海門等三砦、三河口鹽場、豐濟銀場、橫衡等二
 錫場。



 潮陽,中下。本海陽縣地。紹興二年,廢入海陽。八年復。



 揭陽。宣和三年,割海陽三鄉置揭陽縣。紹興二年,廢入海陽。八年復,仍移治吉帛村。是謂「三陽」。



 連州,下,連山郡,軍事。元豐戶三萬六千九百四十三。貢苧布、官桂。元豐貢鐘乳。縣三:桂陽,望。有同官銀場。



 陽山,中。有銅坑銅場。



 連山。中。紹興六年廢為鎮。十八年復。



 梅州,下,軍事。本潮州程鄉縣。南漢置恭州,開寶四年改,熙寧六年廢,元豐五年復。宣和二年,賜郡名義安。紹興六年,廢州為程鄉縣,仍帶程鄉軍事。十四年。復為州。元
 豐戶一萬二千三百七十。貢銀、布。縣一:程鄉。中。有樂口銀場、石坑鉛場、龍坑鐵場。



 南雄州,下,本雄州,軍事。開寶四年,加「南」字。宣和二年,賜郡名保昌。元豐戶二萬三百三十九。貢絹。縣二:保昌,望。



 始興。中。舊隸韶州,開寶四年來隸。



 英德府,下,本英州,軍事。宣和二年,賜郡名曰真陽。慶元元年,以寧宗潛邸,升府。元豐戶三千一十九。貢紵布。縣二:真陽,望。有鐘峒銀場、禮平銅場。



 涵光。上。開寶四年,自廣州隸連州。六年,自連州來隸。有賢德等
 三銀場。



 賀州,下,臨賀郡,軍事。開寶四年,廢蕩山、封陽、馮乘三縣。本屬東路,大觀二年五月,割屬西路。戶四萬二百五。貢銀。縣三:臨賀,緊。有太平銀場。



 富川,上。



 桂嶺。中。



 南渡後,屬廣西路。



 封州,望,臨封郡,軍事。本下郡,大觀元年,升為望郡。紹興七年,省州,以二縣隸德慶府。十年,復舊。元豐戶二千七百七十九。貢銀。縣二:封川,下。



 開建。下。開寶五年,廢入封川。六年,復置。



 肇慶府,望,高要郡,肇慶軍節度。本端州,軍事。元符三
 年,升興慶軍節度。大觀元年,升下為望。重和元年,賜肇慶府名,仍改軍額。元豐戶二萬五千一百三。貢銀、石硯。縣二:高要,中。有沙利銀場、浮蘆鐵場。



 四會。中。舊隸廣州,熙寧六年來屬。有金場、銀場。



 新州,下,新興郡,軍事。開寶五年,廢平興縣。元豐戶一萬三千六百四十一。貢銀。縣一:新興。中。咸平六年,移治州城西。



 德慶府,望。本康州,晉康郡,軍事。開寶五年,廢州及悅城、晉康、都城並入端溪,以隸端州,尋復為州。大觀四年,升為望郡。紹興元年,以高宗潛邸,升為府。十四年,置永慶
 軍節度。元豐戶八千九百七十九。貢銀。縣二:端溪,下。有雲烈錫場。



 瀧水。下。舊隸瀧州,州廢,以縣來隸。有羅磨、護峒二銀場。



 南恩州,下,恩平郡,軍事。舊恩州。開寶五年,廢恩平、杜陵二縣。慶歷八年以河北路有恩州,乃加「南」字。元豐戶二萬七千二百一十四。貢銀。縣二:陽江,中。有海口、海陵、博臘、遂訓等四砦,有鉛場。



 陽春。下。熙寧六年廢春州,並銅陵縣入陽春來隸。有欖徑鐵場。



 惠州,下,軍事。宣和二年,賜郡名博羅。元豐戶六萬一千一百二十一。貢甲香、藤箱。縣四:歸善,中。有阜民錢監,酉平、流坑二銀場,永
 吉、信上、永安三錫場,三豐鐵場,淡水鹽場。



 海豐,下。有雲溪、楊安、勞謝三錫場,古龍、石橋二鹽場。



 河源,緊。有立溪、和溪、永安三錫場。



 博羅。中。有鐵場。



 廣南西路。大觀元年,割融、柳、宜及平、允、從、庭、孚、觀九州為黔南路,融州為帥府,宜州為望郡。三年,以黔南路並入廣西,以廣西黔南路為名。四年,依舊稱廣南西路。州二十五:桂,容,邕,融,像,昭,梧,藤,龔,潯,柳,貴,宜,賓,橫,化,高,雷,欽,白,鬱林,廉,瓊,平,觀。軍三:昌化,萬安,朱崖。縣六十五。南渡後,府二:靜江,慶遠。州二十:容,邕,像,融、昭,梧,藤,潯,貴,柳,
 賓,橫,化,高,雷,欽,廉,賀,瓊,鬱林。軍三:南寧,萬安,吉陽。紹興二十二年,戶四十八萬八千六百五十五,口一百三十四萬一千五百七十二。



 靜江府。本桂州,始安郡,靜江軍節度。大觀元年,為大都督府,又升為帥府。舊領廣南西路兵馬鈐轄,兼本路經略、安撫使。紹興三年,以高宗潛邸,升府。寶祐六年,改廣西制置大使,後四年廢,復為廣西路經略、安撫使。元豐戶四萬六千三百四十三。貢銀、桂心。縣十一:臨桂,緊。嘉祐六
 年,廢慕化縣入焉。



 興安,望。唐全義縣。晉置溥州。乾德元年,州廢。太平興國初,改今名。



 靈川,望。



 荔浦,望。



 永福。下。



 修仁,中。熙寧四年,廢修仁縣為鎮入荔浦。元豐元年復。



 義寧,中下。本義寧鎮,馬氏奏置。開寶五年,廢入廣州新會。六年復置。理定,下。



 古,下。



 永寧。中。唐豐水縣。熙寧四年,廢為鎮入荔浦。元祐元年復。



 南渡後,無永寧縣。



 容州,下,都督府,普寧郡,寧遠軍節度。開寶五年,廢欣道、渭龍、陵城三縣。元豐戶一萬三千七百七十六。貢銀、珠砂。縣三:普寧,上。開寶五年,廢繡州,以常林、阿林、羅繡三縣並入焉。



 陸川,中。開寶五年,廢順州,省龍豪、溫水、龍化、南河四縣入焉。九年,移治公平,淳化五年,又徙治於舊溫水縣。



 北流。中。開寶五年廢
 禺州,以峨石、扶萊、羅辨、陵城四縣地入焉。



 邕州,下,都督府,永寧郡,建武軍節度。開寶五年,廢朗寧、封陵、思龍三縣。大觀元年,升為望郡。紹興三年,置司市馬於橫山砦,以本路經略、安撫總州事,同提點買馬,專任武臣;隆興後文武通差。寶祐元年,兼邕、宜、欽、融鎮撫使。元豐戶五千二百八十八。貢銀。縣二:宣化,下。景祐二年,廢如和縣入焉。



 武緣。下。景祐二年,廢樂昌縣入焉。



 砦一:太平。舊領永平、太平、古萬、橫山四砦,《元豐九域志》止存太平一砦。



 金場一:鎮乃。熙寧六年置。



 羈縻州四十四,縣五,
 洞十一。忠州、凍州、江州、萬丞州、思陵州、左州、思誠州、譚州、渡州、龍州、七源州、思明州、西平州、上思州、祿州、石西州、思浪州、思同州、安平州、員州、廣源州、勤州、南源州、西農州、萬崖州、覆利州、溫弄州及五黎縣、羅陽、陀陵縣、永康縣,武盈洞、古甑洞、憑祥洞、鐏峒、卓峒、龍英洞、龍聳洞、徊洞、武德洞、古佛洞、八洞:並屬左江道。思恩州、鶼州、思城州、勘州、歸樂州、武峨州、倫州、萬德州、蕃州、昆明州、婪鳳州、侯唐州、歸恩州、田州、功饒州、歸城州、武籠州及龍川縣:並屬右江道。初,安平州曰波州,皇祐元年改。元祐三年,又改懷化洞為州。



 融州,融水郡,清遠軍節度。本軍事州,大觀二年,升為帥府。三年,罷帥府,剛軍額,又升為下都督府。崇寧元年,置武陽砦、羅城堡。二年,置樂善砦,廢羅城堡。四年,即融水
 縣王口砦置平州。政和元年,廢平州,仍為王口砦,與融江、文村、潯江、臨溪四堡砦來隸,尋復故。紹興四年,復廢平州為王口砦,觀州為高峰砦。元豐戶五千六百五十八。貢金、桂心。縣一:融水。中。開寶五年,置羅城縣。熙寧七年,廢武陽、羅城二縣為鎮來隸。



 砦一:融江。南渡後,增縣一:懷遠。下。紹興四年州廢,復為砦來隸;十四年,復為縣。有臨溪、文村、潯江三堡,高峰砦。



 羈縻州一:樂善州。



 象州,下,像郡,景德上年,升防禦。景定三年,徙治來賓縣之蓬萊。元豐戶八千七百一十七。貢金、藤器、患子。縣四:
 陽壽。中下。



 來賓,中下。舊隸嚴州,州廢來屬。開寶七年,又以廢嚴州之歸化入焉。



 武化,下。熙寧七年,廢武化縣入來賓。元祐元年復。



 武仙。下。



 南渡後,無武化縣。



 昭州,下,平樂郡,軍事。開寶五年,廢永平縣。元豐戶一萬五千八百八十。貢銀。縣四:平樂,中。大中祥符元年,移治州城東。



 立山,中。熙寧五年廢蒙州,以東區、蒙山二縣入焉。



 龍平,中。開寶五年廢富州,以縣來隸,又以思勤、馬江入焉。熙寧八年,又隸梧州。元豐八年復來隸。宣和中改昭平。淳熙六年復今名。恭城。下。太平興國元年,徙治於北鄉龍渚市。景定五年復舊。



 梧州,下,蒼梧郡,軍事。元豐戶五千七百二十。貢銀、白石
 英。縣一:蒼梧,下。熙寧四年,省戎城縣為鎮,入蒼梧。



 藤州,下,感義郡,軍事。開寶三年,廢寧風、感義、義昌三縣。元豐戶六千四百二十二。貢銀。縣二:鐔津,中。



 岑溪。下。熙寧四年,廢南儀州為縣,隸州。



 龔州,下,臨江郡,軍事。開寶五年,廢陽川、武林、隨建、大同四縣。政和元年,州廢,隸潯州,三年,復。紹興六年,復廢,仍隸潯州。元豐戶八千三十九。貢銀。縣一:平南。中。開寶五年,以思明州之武郎來屬。嘉祐二年,廢武郎縣入焉。



 潯州,下,潯江郡,軍事。開寶五年,廢皇化縣,俄又廢州,以桂平隸貴州。六年,復置。元豐戶六千一百四十一。貢銀。縣一:桂平。下



 柳州,下,龍城郡,軍事。咸淳元年,徙治柳城縣之龍江。元豐戶八千七百三十。貢銀。縣三:馬平,中。



 洛容,中。嘉祐四年,廢象縣入洛容。



 柳城。中。梁龍城縣。景德三年改。



 貴州,下,懷澤郡,軍事。元豐戶七千四百六十。貢銀。縣一:鬱林。中下。隋鬱平縣。開寶四年改。



 慶遠府,下。本宜州,龍水郡,慶遠軍節度。舊軍事州。景祐三年,廢崖山縣。宣和元年,賜軍額。河池縣,不詳何年並省。咸淳元年,以度宗潛邸,升慶遠府。元豐戶一萬五千八百二十三。貢生豆蔻、草豆蔻。元豐貢銀。縣四:龍水,上。淳化五年,以柳州洛曹來隸;嘉祐七年,廢入龍水。熙寧八年二月,廢羈縻懷遠軍古陽縣為懷遠砦迷昆縣為鎮,並思立砦並入焉。有懷遠、思立二砦。後改宜山。



 天河,下。大觀元年六月,以天河縣並德謹砦、堰江堡隸融州。靖康元年九月,復來隸。有德謹一砦。



 忻城,中下。慶歷三年,以羈縻芝忻、歸恩、絲虧等州地置縣。



 思恩。下。熙寧八年,自環州來隸,徙治帶溪砦,省鎮寧州禮丹縣入焉。元豐六年,復徙舊治。有普義、帶溪、鎮寧三
 砦。



 南渡後,增縣一:河池。下。有銀場。



 羈縻州十,軍一,監二。溫泉州、環州、鎮寧州,領縣二。蕃州、金城州、文州、蘭州,領縣三。安化州,領縣四。迷昆州、智州,領縣五。懷遠軍,領縣一。又有富仁、富安二監。舊領思順、歸化二州,慶歷四年,並入柳州馬平縣。



 賓州,下,安城郡,軍事。開寶五年,廢州、瑯琊保城二縣,以嶺方隸邕州。六年,以嶺方復置州。元豐戶七千六百二十。貢銀、藤器。縣三:嶺方,下。遷江,中。本邕州羈縻州,天禧四年置。



 上林。中下。開寶五年,自邕州來屬,廢澄州止戈、賀水、無虞入焉。



 橫州,下,寧浦郡,軍事。開寶五年,廢樂山、從化二縣,又以
 廢巒州永定來屬。元豐戶三千四百五十一。貢銀。縣二:寧浦,下。



 永定。下。開寶六年,廢巒州武靈、羅竹二縣入焉。熙寧四年,省入寧浦。元祐三年復置,後更名永淳。



 化州,下,陵水郡,軍事。本辯州,太平興國五年改。開寶中,廢陵羅縣。元豐戶九千三百七十三。貢銀、高良姜。縣二:石龍,下。



 吳川。下。本屬羅州,州廢,開寶五年來隸。



 南渡後,增縣一:石城。乾道三年,析吳川西鄉置,因石城岡為名。



 高州,下,高涼郡,軍事。開寶五年,廢良德縣。景德元年,並
 入竇州,移治茂名。三年,復置,以二縣還隸。元豐戶一萬一千七百六十六。貢銀。縣三:電白,下。



 信宜。中下。唐信儀縣。太平興國初改信宜。熙寧四年廢竇州,以信宜縣來隸。有銀場。



 茂名。下。開寶五年,自潘州來隸。



 雷州,下,海康郡,軍事,開寶五年,廢徐聞、遂溪二縣。元豐戶一萬三千七百八十四。貢良姜。元豐貢斑竹。縣一:海康。下。有冠頭砦。



 南渡後,復二縣:遂溪,紹興十九年復置。



 徐聞。乾道七年復置。



 欽州,下,寧越郡,軍事。開寶五年,廢遵化、欽江、內亭三縣。天聖元年,徙州治南賓砦。元豐戶一萬五百五十二。貢
 高良姜、翡翠毛。縣二:靈山,望。有咄步砦。



 安遠。下。唐保京縣。宋初改安京,景德中,改今名。有如洪、如昔二砦。



 白州,下,南昌郡,軍事。開寶五年,廢南昌、建寧、周羅三縣。政和元年廢州,以其地隸鬱林,三年復。南渡後,復廢入鬱林。元豐戶四千五百八十九。貢銀、縮砂。縣一:博白。中。南渡後,隸鬱林州。



 鬱林州,下,鬱林郡,軍事州。開寶中,廢鬱平、興德二縣。州初治興業,至道二年,徙今治。政和元年,廢白州,博白來
 隸。三年,復置白州,以博白還舊隸。南渡後,廢白州,以博白來隸。元豐戶三千五百六十四。貢縮砂。元豐貢銀。縣二:南流,中下。舊隸牢州,州廢來隸,又以廢牢州之定川、宕川,黨州容山、懷義、撫康、善牢入焉。



 興業。下。以廢鬱平、興德入焉。



 廉州,下,合浦郡,軍事。開寶五年,廢封山、蔡龍、大廉三縣,移州治於長沙場,置石康縣。太平興國八年,改太平軍,移治海門鎮。咸平元年復。元豐戶七千五百。貢銀。縣二:合浦,上。有二砦。



 石康。下。本常樂州,宋並為縣。



 瓊州,下,瓊山郡,靖海軍節度。本軍事州。大觀元年,以黎母山夷峒建鎮州,賜軍額為靖海。政和元年,鎮州廢,以其地及軍額來歸。元豐戶八千九百六十三。貢銀、檳榔。縣五:瓊山,中。熙寧四年,省舍城入焉。有感恩、英田場二柵。



 澄邁。下。開寶五年廢崖州,與舍城、文昌並來隸。



 文昌,下。



 臨高,下。紹興初,移於莫村。



 樂會。下。唐置,環以黎洞,寄治南管。大觀三年,割隸萬安軍,後復來屬。



 南寧軍,舊昌化軍,同下州。本儋州,熙寧六年,廢州為軍。紹興六年,廢昌化、萬安、吉陽三軍為縣,隸瓊州。十三年,
 為軍使,十四年復為軍,以屬縣還隸本軍。後改今名。元豐戶八百五十三。貢高良姜。元豐貢銀。縣三:宜倫,下。隋義倫縣。太平興國初改。



 昌化,下。熙寧六年省,元豐三年復。有昌化砦。感恩。下。熙寧六年省,元豐四年復。



 萬安軍,同下州。舊萬安州,萬安郡。熙寧七年,廢為軍。紹興六年,廢軍為萬寧縣,以軍使兼知縣事,隸瓊州。十三年,復為軍。元豐戶二百七十。貢銀。縣二:萬寧,下。後復名萬安。



 陵水。下。熙寧七年為鎮,元豐三年復。紹興六年隸瓊州。十三年,復來隸。



 吉陽軍,同下州。本朱崖軍,即崖州。熙寧六年,廢為軍。紹興六年,廢軍為寧遠縣。十三年復。後改名吉陽軍。元豐戶二百五十一。貢高良姜。鎮二:臨川,藤橋。熙寧六年,省寧遠、吉陽二縣為臨川、藤橋二鎮。寧遠即臨川。



 南渡後,縣二:寧遠,下。紹興六年復縣,隸瓊州。十三年,復來屬。



 吉陽。下。熙寧六年,廢為藤橋鎮,隸瓊州。紹興六年復。



 平州。崇寧四年三月,王江古州蠻戶納土,於王口砦建軍,以懷遠為名,割融州融江、文村、潯江、臨溪四堡砦並隸軍。尋改懷遠軍為平州,仍置倚郭懷遠縣。又置百萬
 砦及萬安砦,又於安口隘置允州及安口縣,又於中古州置格州及樂古縣。五年,改格州為從州。政和元年,廢平州,依舊為王口砦;並融江、文村、潯江、臨溪四堡砦並依舊隸融州,廢懷遠縣。又廢從州為樂古砦,並通靖、鎮安、百萬砦並撥隸允州。又廢允州,權留平州,又權置百萬砦。宣和二年,賜平州郡名曰懷遠。紹興四年,廢平州仍為王口砦,隸融州。十四年,復以王口砦為懷遠縣。



 從州。廢置具
 平州。



 允州。廢置同上。



 庭州。大觀元年,以宜州河池縣置庭州,倚郭縣曰懷德。又於南丹州中平縣置砦曰靖南,尋撥隸庭州。大觀二年,置安遠砦。大觀四年,廢庭州,移靖南砦於廢孚州。宣和五年,移安遠砦於平安山置。



 孚州。大觀元年,以地州建隆縣置孚州,倚郭縣曰歸仁。四年,廢孚州及歸仁縣為靖南砦。先於南丹州中平縣置靖南砦,今移置此。政和七年,復置孚州及歸仁縣,仍
 移靖南砦歸舊處。宣和三年,復廢孚州及歸仁縣,置靖南砦。大觀四年,隸觀州。紹興四年,廢靖南砦。



 溪州。大觀元年,以宜州思恩縣帶溪砦置溪州。四年,廢。



 鎮州。大觀元年,置鎮州於黎母山心,倚郭縣以鎮寧為名,升鎮州為都督府,賜靜海軍額。政和元年,廢鎮州,以靜海軍額為瓊州。



 延德軍。崇寧五年,初置延德縣於朱崖軍黃流、白沙、側浪之間。大觀元年,改為軍,又置倚郭縣曰通遠。政和元
 年,廢延德軍為感恩縣,昌化軍通遠縣為通遠鎮,隸朱崖軍。政和六年,置延德砦,又以通遠鎮為砦。



 地州。崇寧五年納土。大觀元年,以地州建隆縣置孚州。



 文州。崇寧五年納土。大觀元年,置綏南砦。紹興四年廢。



 蘭州。崇寧五年納土。



 那州。崇寧五年納土。



 觀州。大觀元年,克南丹州,以南丹州為觀州,置倚郭縣。大觀四年,以南丹州還莫公晟,復於高峰砦置觀州。紹興
 四年,廢觀州為高峰砦,存留木門、馬臺、平洞、黃泥、中村等堡砦。



 隆州。



 兌州。政和四年,置隆州、兌州並興隆縣、萬松縣。宣和三年,廢隆州及興隆縣為威遠砦,兌州及萬松縣為靖遠砦。二州先置思忠、安江、鳳麟、金斗、朝天等五砦並廢,各隸新砦,仍並隸邕州。



 廣南東、西路,蓋《禹貢》荊、揚二州之域,當牽牛、婺女之分。南濱大海,西控夷洞,北限五嶺。有犀象、玳瑁、珠璣、銀銅、
 果布之產。民性輕悍。宋初,以人稀土曠,並省州縣。然歲有海舶貿易,商賈交湊,桂林邕、宜接夷獠,置守戍。大率民婚嫁、喪葬、衣服多不合禮。尚淫祀,殺人祭鬼。山林翳密,多瘴毒,凡命官吏,優其秩奉。春、梅諸州,炎癘頗甚,許土人領任。景德中,令秋冬赴治,使職巡行,皆令避盛夏瘴霧之患。人病不呼醫服藥。儋、崖、萬安三州,地狹戶少,常以瓊州牙校典治。安南數郡,土壤遐僻,但羈縻不絕而已。



 燕山府路。府一:燕山。州九、涿,檀,平,易,營,順,薊,景,經。縣二十。宣和四年,詔山前收復州縣,合置監司,以燕山府路為名,山後別名雲中府路。



 燕山府。唐幽州,範陽郡,盧龍軍節度。石晉以賂契丹,契丹建為南京,又改號燕京。金人滅契丹,以燕京及涿、易、檀、順、景、薊六州二十四縣來歸。宣和四年,改燕京為燕山府,又改郡曰廣陽,節度曰永清軍,領十二縣。五年,童貫、蔡攸入燕山。七年,郭藥師以燕山叛,金人復取之。
 析津,宛平,都市,賜名廣寧。



 昌平,良鄉,潞,武清,安次,永清,玉河,香河,賜名清化。



 漷陰。



 涿州。唐置,石晉以賂契丹。宣和四年,金將郭藥師以州降,賜郡名曰涿水,升威行軍節度。縣四:範陽,歸義,固安,新城。賜名威城。



 檀州。隋置,石晉以賂契丹。宣和四年,金人以州來歸,賜郡名曰橫山,升鎮遠軍節度。七年,金人復破之。縣二:密雲,行唐。賜名威塞。



 平州。隋置,後唐時為契丹所陷,改遼興府,以營、灤二州隸之。宣和四年,賜郡名漁陽,升撫寧軍節度。五年,遼將張覺據州來降,尋為金所破。縣三:盧龍,賜名盧城。



 石城,賜名臨關。



 馬城。賜名安城。



 易州。唐置,雍熙四年,陷於契丹。宣和四年,金人以州來歸,賜郡名曰遂武,防禦。縣三:易水,淶水,容城。



 營州。隋置,後唐時為契丹所陷。宣和四年,賜郡名曰平盧,防禦。縣一:柳城。賜名鎮山。



 順州。唐置,石晉以賂契丹。宣和四年,金人以州來歸,賜郡名曰順興,團練。縣一:懷柔。



 薊州。唐置,石晉以賂契丹。宣和四年,金人以州來歸,賜郡名曰廣川,團練。七年,金人破之。縣三:漁陽,賜名平盧。



 三河,玉田。



 景州。契丹置。宣和四年,金人以州來歸,賜郡名曰灤川,軍事。縣一:遵化。



 經州。本薊州玉田縣。宣和六年,建為州。七年,陷於金。



 雲中府路。



 雲中府,唐雲州,大同軍節度。石晉以賂契丹,契丹號為西京。宣和三年,始得雲中府、武、應、朔、蔚、奉、聖、歸、化、儒、媯等州,所謂山後九州也。



 武州。唐置,石晉以賂契丹。宣和五年,金人以州來歸。六年,築固疆堡。尋復為金人所取。



 應州。故屬大同軍節度,後唐置彰國軍,石晉以賂契丹。宣和五年,契丹將蘇京以州來降。金人尋遂京,復取之。



 朔州。唐置,後唐為振武軍,石晉以賂契丹。宣和五年,守將韓正以州來降。金人尋逐正,復取之。



 蔚州。唐置,石晉以賂契丹。宣和五年,守將陳翊以州來降。六年,翊為金人所殺,復取之。



 奉聖州。唐新州,後唐置威塞軍節度,石晉以賂契丹。在雲中府之東,契丹改為奉聖州。



 歸化州。舊毅州,後唐改為武州,石晉以賂契丹,契丹改為歸化州。



 儒州。唐置,石晉以賂契丹。



 媯州。唐置,石晉以賂契丹,契丹改為可汗州。



\end{pinyinscope}