\article{列傳第二百一十六隱逸上}

\begin{pinyinscope}

 ○戚同文陳摶種放萬適李瀆魏野邢敦林逋高懌徐復孔旼何群



 中古聖人之作《易》也,於《遁》之上九曰「肥遁,無不利」,《蠱》之上九曰「不事王侯,高尚其事」。二爻以陽德處高地,而皆以隱逸當之。然則隱德之高於當世,其來也遠矣。巢、由雖不見於經,其可誣哉。五季之亂,避世宜多。宋興,巖穴弓旌之招,疊見於史,然而高蹈遠引若陳摶者,終莫得而致之,豈非二卦之上九者乎?種放之徒,召對大廷,亹亹獻替,使其人出處,果有合於《艮》之君子時止時行,人何譏焉。作《隱逸傳》。



 戚同文,字同文,宋之楚丘人。世為儒。幼孤,祖母攜育於外氏,奉養以孝聞。祖母卒,晝夜哀號,不食數日,鄉里為之感動。



 始,聞邑人楊愨教授生徒,日過其學舍,因授《禮記》,隨即成誦,日諷一卷,愨異而留之。不終歲畢誦《五經》,愨即妻以女弟。自是彌益勤勵讀書,累年不解帶。時晉末喪亂,絕意祿仕,且思見混一,遂以「同文」為名字。愨嘗勉之仕,同文曰:「長者不仕,同文亦不仕。」愨依將軍趙直家,遇疾不起,以家事托同文,即為葬三世數喪。直復厚
 加禮待,為築室聚徒,請益之人不遠千里而至。登第者五六十人,宗度、許驤、陳象輿、高象先、郭成範、王礪、滕涉皆踐臺閣。



 同文純質尚信義,人有喪者力拯濟之,宗族閭裏貧乏者周給之。冬月,多解衣裘與寒者。不積財,不營居室,或勉之,輒曰:「人生以行義為貴,焉用此為!」由是深為鄉里推服。有不循孝悌者,同文必諭以善道。頗有知人鑒,所與游皆一時名士。樂聞人善,未嘗言人短。與宗翼、張昉、滕知白為友。生平不至京師。長子維任隨州
 書記,迎同文就養,卒於漢東,年七十三。好為詩,有《孟諸集》二十卷。楊徽之嘗因使至郡,一見相善,多與酬唱。徽之嘗云陶隱居號堅白先生,先生純粹質直,以道義自富,遂與其門人追號堅素先生。



 二子維、綸。維,建隆二年,以屯田員外郎為曹王府翊善,累官職方郎中,致仕,卒,年八十一。綸自有傳。



 大中祥符二年,府民曹城即同文舊居旁造舍百餘區,聚書數千卷,延生徒講習甚盛。詔賜額為本府書院,命綸子奉禮郎舜賓主之,署誠府助
 教,委本府幕官提舉之。



 楊愨者,虞城人。力學勤志,不求聞達。



 宗翼者,蔡州上蔡人。父為虞城主簿,因家焉。篤孝恭謹,負米養母。好學強記,經籍一見即能默寫。歐陽、虞、柳書皆得其楷法。能屬文。隱而不仕,家無鬥粟,怡怡如也,未嘗以貧窶干人。市物不評價,市人知而不欺。嘗言「晝夜者,昏曉之辨也」,故既暝未曙,皆不出戶。見鄰里小兒,待之如成人,未嘗欺紿。同文嘗謂翼曰:「子勞謙有古人風,真吾友也。」卒,年八十餘。子度,舉進士,至侍御史,歷
 京西轉運使,預修《太祖實錄》。



 張昉有史材,歷知雜御史、省郎,至殿中少監致仕。子信,自有傳。



 滕知白善為詩,至刑部員外郎、河北轉運使。子涉,為給事中。



 高象先父凝祐,刑部郎中,以強乾稱。象先,淳化中三司戶部副使,卒於光祿少卿。



 郭成範最有文,為倉部員外郎,掌安定公書記。辭疾,以司封員外郎致仕,卒。



 王礪事母甚謹,太平興國五年進士,至屯田郎中。子渙、瀆、淵、沖、泳。渙子稷臣,瀆子堯臣,並進士及第。渙子夢臣,進士出身。



 陳摶,字圖南,亳州真源人。始四五歲,戲渦水岸側,有青衣媼乳之,自是聰悟日益。及長,讀經史百家之言,一見成誦,悉無遺忘,頗以詩名。後唐長興中,舉進士不第,遂不求祿仕,以山水為樂。自言嘗遇孫君仿、獐皮處士二人者,高尚之人也,語摶曰:「武當山九室巖可以隱居。」摶往棲焉。因服氣闢穀歷二十餘年,但日飲酒數杯。移居華山雲臺觀,又止少華石室。每寢處,多百餘日不起。



 周世宗好黃白術,有以摶名聞者,顯德三年,命華州送至
 闕下。留止禁中月餘,從容問其術,摶對曰:「陛下為四海之主,當以致治為念,奈何留意黃白之事乎?」世宗不之責,命為諫議大夫,固辭不受。既知其無他術,放還所止,詔本州長吏歲時存問。五年,成州刺史朱憲陛辭赴任,世宗令齎帛五十匹、茶三十斤賜摶。



 太平興國中來朝,太宗待之甚厚。九年復來朝,上益加禮重,謂宰相宋琪等曰:「摶獨善其身,不干勢利,所謂方外之士也。摶居華山已四十餘年,度其年近百歲。自言經承五代離亂,幸
 天下太平,故來朝覲。與之語,甚可聽。」因遣中使送至中書,琪等從容問曰:「先生得玄默修養之道,可以教人乎?」對曰:「摶山野之人,於時無用,亦不知神仙黃白之事,吐納養生之理,非有方術可傳。假令白日沖天,亦何益於世?今聖上龍顏秀異,有天人之表,博達古今,深究治亂,真有道仁聖之主也。正君臣協心同德、興化致治之秋,勤行修煉,無出於此。」琪等稱善,以其語白上。上益重之,下詔賜號希夷先生,仍賜紫衣一襲,留摶闕下,令有司
 增葺所止雲臺觀。上屢與之屬和詩賦,數月放還山。



 端拱初,忽謂弟子賈德升曰:「汝可於張超谷鑿石為室,吾將憩焉。」二年秋七月,石室成,摶手書數百言為表,其略曰:「臣摶大數有終,聖朝難戀,已於今月二十二日化形於蓮花峰下張超谷中。」如期而卒,經七日支體猶溫。有五色雲蔽塞洞口,彌月不散。



 摶好讀《易》,手不釋卷。常自號扶搖子,著《指玄篇》八十一章,言導養及還丹之事。宰相王溥亦著八十一章以箋其指。摶又有《三
 峰寓言》及《高陽集》、《釣潭集》,詩六百餘首。



 能逆知人意,齋中有大瓢掛壁上,道士賈休復心欲之,摶已知其意,謂休復曰:「子來非有他,蓋欲吾瓢爾。」呼侍者取以與之,休復大驚,以為神。有郭沆者,少居華陰,夜宿雲臺觀。摶中夜呼令趣歸,沆未決;有頃,復日曰:「可勿歸矣。」明日,沆還家,果中夜母暴得心痛幾死,食頃而愈。



 華陰隱士李琪,自言唐開元中郎官,已數百歲,人罕見者;關西逸人呂洞賓有劍術,百餘歲而童顏,步履輕疾,頃刻數百里,世以為神仙。皆
 數來摶齋中,人咸異之。大中祥符四年,真宗幸華陰,至雲臺觀,閱摶畫像,除其觀田租。



 又有許瓊者,開封鄢陵人。開寶五年,子永罷盧縣尉,詣匭上言:「臣年七十五,父瓊年九十九,長兄年八十一,次兄年七十九,欲乞近地一官,以就榮養。」上覽奏,召永訊之,即命迎其父赴闕。瓊得對於講武殿,上顧問久之,悉能奏對,而詞氣不衰,言唐末以來事,歷歷可聽。上悅其父子俱享遐壽,賜襲衣、犀帶、銀鞍勒馬、帛三十匹、茶二十斤,授永鄢城令。是
 時,澶密齊沂、萊江吉萬州、江陰梁山軍,各奏八十已上呂繼美等二十九人,並賜爵公士。真宗時,凡老人年百歲已上者,州縣以名聞,皆詔賜衣帛、米麥,長吏存撫之。



 種放,字明逸,河南洛陽人也。父詡,吏部令史,調補長安主簿。放沉默好學,七歲能屬文,不與群兒戲。父嘗令舉進士,放辭以業未成,不可妄動。每往來嵩、華間,慨然有山林意。未幾父卒,數兄皆干進,獨放與母俱隱終南豹林谷之東明峰,結草為廬,僅庇風雨。以請習為業,從學
 者眾,得束脩以養母,母亦樂道,薄滋味。



 放得闢谷術,別為堂於峰頂,盡日望雲危坐。每山水暴漲,道路阻隔,糧糗乏絕,止食芋慄。性嗜酒,嘗種秫自釀,每曰空山清寂,聊以養和,因號雲溪醉侯。幅巾短褐,負琴攜壺,溯長溪,坐磐石,採山藥以助飲,往往終日。值月夕或至宵分,自豹林抵州郭七十里,徒步與樵人往返。性不喜浮圖氏,嘗裂佛經以制帷帳。所著《蒙書》十卷及《嗣禹說》、《表孟子上下篇》、《太一祠錄》,人頗稱之。多為歌詩,自稱「退士」,嘗作
 傳以述其志。



 淳化三年,陜西轉運宋惟乾言其才行,詔使召之。其母恚曰:「常勸汝勿聚徒講學。身既隱矣,何用文為?果為人知而不得安處,我將棄汝深入窮山矣。」放稱疾不起。其母盡取其筆硯焚之,與放轉居窮僻,人跡罕至。太宗嘉其節,詔京兆賜以緡錢使養母,不奪其志,有司歲時存問。咸平元年母卒,水漿不入口三日,廬於墓側。翰林學士宋湜、集賢院學士錢若水、知制誥王禹偁言其貧不克葬,詔賜錢三萬、帛三十匹、米三十斛以
 助其喪。



 四年,兵部尚書張齊賢言放隱居三十年,不游城市十五載,孝行純至,可勵風俗,簡樸退靜,無謝古人。復詔本府遣官詣山,以禮發遣赴闕,齎裝錢五萬,放辭不起。明年,齊賢出守京兆,復條陳放操行,請加旌賁。即賜詔曰:「汝隱居丘園,博通今古,孝悌之行,鄉里所推,慕古人之遺榮,挹君子之常道。屢覽守藩之奏,彌彰遁世之風,載渴來儀,副予延佇。今遣供奉官周旺齎詔,召汝赴闕,賜帛百匹、錢十萬。」九月,放至,對崇政殿,以幅巾見,
 命坐與語,詢以民政邊事。放曰:「明王之治,愛民而已,惟徐而化之。」餘皆謙讓不對。即日授左司諫、直昭文館,賜巾服簡帶,館於都亭驛,大官供膳。翌日,表辭恩命。上知放舊與陳堯叟游,令堯叟諭意;又謂宰相曰:「朕求茂異,以廣視聽,資治道。如放終未樂仁,亦可遂其請也。」中書傳詔,放曰:「病居山林,天恩累加禮聘,巖猿溪鳥之性,固不敢以祿仕為意。然主上虛懷待士,旰食憂人之心,亦不敢以羈束為念。」遂詔不聽其讓。數日,復召見,賜緋衣、
 象簡、犀帶、銀魚,禦制五言詩寵之,賜昭慶坊第一區,加帷帳什物,銀器五百兩,錢三十萬。中謝日,賜食學士院,自是屢得召對。六年春,再表謝暫歸故山,詔許其請。將行,又遷起居舍人,命館閣官宴餞于瓊林苑,上賜七言詩三章,在席皆賦。十月,遣使就山撫問,圖其林泉居處以獻,優詔趣其入覲,放以疾未平為請。



 景德元年十月,來朝,言歸山之久,請計月不受奉,詔特給之。嘗因觀書賦詩,上曰:「放體格高古。聞其歸,私居終日,默坐一室。山
 水之樂,亦天性也。每所詢問,皆據經以對,頗多裨益。朕優待之,蓋以激浮競也。」放每至京師,秦雍生徒多就而受業。二年,擢為右諫議大夫。表乞嵩少養疾,許之,令河南府檢校。召對資政殿,曲宴學士院,王欽若洎當直學士、舍人、待制悉預。既罷,又賜宴於欽若直廬。表乞免都門置餞之禮。屢遣中使勞問,賜以茶藥。是冬,復來朝。三年,以兄喪請告歸終南營葬,復召宴賜詩。



 放山居草舍五六區,啖野蔬蕎麥。表求太宗御書及經史音疏,悉給
 焉。十月,復至,上謂宰相曰:「放比來高尚其事,每所詢問,頗有可採。朝廷雖加爵秩,而未能大用,即物議未厭,所慮放卷而懷之。」即遣內侍任文慶齎詔諭之曰:「朕臨御寰區,憂勤旰昃,詳延茂異,物色隱淪,思訪話言,用熙庶績。以卿棲心巖竇,屏跡囂塵,躡綺皓之遐蹤,有曾、顏之至行,特舉賁園之典,果符前席之心。每所諮詢,備詳理道,載觀敷納,蔚有材謀,深簡朕懷,頗思大用。然以群情未悉,成命是稽。今四隩來同,萬區思乂,方崇政本,庶厚
 時風。卿必能酌斟化源,丹青王度,恢富國強兵之術,陳制禮作樂之規。返樸還淳,措刑息訟,輔予不逮,馴至太平,登用機衡,弼成寡昧。卿宜體茲眷遇,罄乃誠明,敘經國之大猷,述致君之遠略,盡形奏牘,以沃朕心。副涼德之倚毗,褰外朝之觀聽,乃司樞務,式洽至公。」



 放上言曰:「臣讀書業文,實自父師之誨,學古嗜退,本求山水之樂。思率天性以奉至道,豈有意於麋鹿,蓋無心於紱冕。其所幸者,邦家化成,疆場兵偃,群黎鼓舞,庶匯胥悅。蒲帛
 之聘,寵渙巖谷,君命薦及,肅聽祗受。既朝象魏之下,但愧巖林之賤。奉聖顏於咫尺,聆德音之教論。列跡侍從,峨冠諫諍。雖愚者之慮,竭忠規而屢陳;而大君之明,懼瞽言之無補。今又訪以禮樂之制,詢其刑政之方,且小器微材,欲加大用。蓋念沿革之攸宜,歷三五而既異,弛張之體,豈一二而可述。國家謀建皇極,躋納富壽,惟二聖之光宅,總百王之闕漏,豈伊葑菲,敢預論述。方今德義宣明,鸞驥戾止,如臣之才,儼爾駢列。伏望洞知臣之
 鑒,憐守節之志,俾泛駕無覆壓之害,使為器免溢蕩之咎,寢此過聽,遂其夙心。況臣首獻納之行,不為無位;預清閑之對,不為疏隔。又安敢碌碌而依違,嘿嘿而曠素?願且齒於諫署,庶少觀於朝制,斯亦否能有適,名器無假。唯茲保全之惠,仰醫仁聖之賜。」



 時先俾陳堯叟諭旨,堯叟手筆審其意,放云:「自被聘召,及遷諫垣,無所補報,為幸多矣。今主上聖明,朝無闕政,處之顯位,則是重增其過。」及覽表,上曰:「放能守分懇讓,益可嘉也。」大中祥符
 元年,命判集賢院,從封泰山,拜給事中。二年四月,求歸山,宴餞於龍圖閣,命學士即席賦詩,制序。上作詩,卒章云:「我心虛佇日,無復醉山中。」初,放作詩嘗有「溪上醉眠都不知」之句,故及之。三年正月,復召赴闕,表乞賜告,手詔優答之。作歌賜之,乃齎衣服、器幣,令京兆府每季遣幕職就山存問。四年正月,復來朝,從祠汾陰,拜工部侍郎。



 放屢至闕下,俄復還山,人有詒書嘲其出處之跡,且勸以棄位居巖谷,放不答。放終身不娶,尤惡囂雜,故京
 城賜第為擇僻處。然祿賜既優,晚節頗飾輿服。於長安廣置良田,歲利甚博,亦有強市者,遂致爭訟,門人族屬依倚恣橫。王嗣宗守京兆,放嘗乘醉慢罵之。嗣宗屢遣人責放不法,仍條上其事。詔工部郎中施護推究,會赦恩而止。四月,求歸山,又賜宴遣之。所居山林,細民多縱樵採,特詔禁止。放遂表徙居嵩山天封觀側,遣內侍就興唐觀基起第賜之。假逾百日,續給其奉。然猶往來終南,按視田畝。每行必給驛乘,在道或親詬驛吏,規算糧
 具之直。時議浸薄之。



 嘗曲宴令群臣賦詩,杜鎬以素不屬辭,誦《北山移文》以譏之。上嘗語近臣曰:「放為朕言事甚眾,但外廷無知者。」因出所上《時議》十三篇,其目曰:《議道》、《議德》、《議刑》、《議器》、《議文武》、《議制度》、《議教化》、《議賞罰》、《議官司》、《議軍政》、《議獄訟》、《議征賦》、《議邪正》。



 八年十一月乙丑,晨興,忽取前後章疏稿悉焚之,服道士衣,召諸生會飲於次,酒數行而卒。訃聞,上甚嗟悼,親制文遣內侍朱允中致祭。歸葬終南,贈工部尚書,錄其侄世雍同學究出身。



 萬適,字縱之,陳州宛丘人,自號遣玄子。六七歲即為詩。及長,喜學問,精於《道德經》。與高錫族子冕及韓伾交游,酬唱多有警句。不求仕進,專以著述為務,有《狂簡集》百卷、《雅書》三卷、《志苑》三卷、《雍熙詩》二百首,《經籍擿科討論》計四十卷。



 淳化中,伾任翰林學士,因召對,上問曰:「卿早在嵩陽,當時輩流頗有遺逸否?」伾以適及楊璞、田誥為對,上悉令召至闕下。詔書下而誥卒。璞既至,對於便殿,不願仕進,上賜以束帛,與一子出身,遣還故郡。適最後
 至,特授慎縣主簿。適素康強無疾,詔下日已病,猶勉強赴朝謝,舉止山野,人皆笑之,後數日卒。



 田誥者,歷城人。好著述,聚學徒數百人,舉進士至顯達者接踵,以故聞名於朝,宋惟翰、許袞皆其弟子也。誥著作百餘篇傳於世,大率迂闊。每構思必匿深草中,絕不聞人聲,俄自草中躍出,即一篇成矣。



 楊璞字契玄,鄭州新鄭人。善歌詩,士大夫多傳誦。與畢士安尤相善,每乘牛往來郭店,自稱東里遺民。嘗杖策入嵩山窮絕處,構思為歌詩,凡數
 年得百餘篇。璞既被召,還,作《歸耕賦》以見志。真宗朝諸陵,道出鄭州,遣使以茶帛賜之。卒,年七十八。



 李瀆,河南洛陽人也。六世祖坦,馮翊令。坦生仲芳,大理司直。仲芳生玄初,福建觀察推官。玄初生鄑,即瀆之曾祖也,字堯封,仕梁,歷滑、魏、宋三鎮留後,拜崇政使、禮部尚書。後唐天成中,以太子少傅致仕,卒,贈太保。祖延昭,殿中丞。父瑩字正白,善詞賦,廣順進士,蒲帥張鐸闢為記室,因家河中。乾德初,右補闕蘇德祥薦為殿中侍御
 史、度支判官。使江南,坐受李從善賂遺,責授右贊善大夫,卒。



 初,瑩禱河祠而生瀆,故名瀆字河神,後改字長源。淳澹好古,博覽經史。十六丁外艱,服闕,杜門不復仕進。家世多聚書畫,頗有奇妙。王祐典河中,深加禮待,自是多聞於時。往來中條山中,不親產業,所居木石幽勝。談唐室已來衣冠人物,歷歷可聽。罕著文。前後州將皆厚遇之。王旦、李宗諤與之世舊,每勸其仕,瀆皆不答。所乘馬,嘗為宗人借,憩於廛間。人有見者以語瀆,瀆即鬻之,
 其惡囂如此。州閭化其儉德。



 真宗祀汾陰,直史館孫冕言其隱操,請加搜採,陳堯叟復薦之。命使召見,辭足疾不起。遣內侍勞問,令長吏歲時存撫。明年,又遣使存問,瀆自陳世本儒墨習靜避世之意。素嗜酒,人或勉之,答曰:「扶羸養疾,舍此莫可。從吾所好,以盡餘年,不亦樂乎!」嘗語諸子曰:「山水足以娛情,茍遇醉而卒,吾之願也。吾將與爾永訣,爾輩當常在左右。」即設外寢,與諸子同處。一日,忽曰:「適有人至床下,誦詩云:『行到水窮處,未知天
 盡時。』言訖不見,吾當逝矣。」亟取瑩集七十編洎書畫付諸子,促家人置酒。頃之,卒。時天禧三十年十二月三日也,年六十三。



 四年春,詔曰:「故河中府處士李瀆,簪纓傳緒,儒雅踐方,曠逸自居,恬智交養。迨茲晚節,彌邵清猷,奄及淪亡,良深軫惻。特行賁典,式慰營魂。惟蓬閣之司文,乃儒林之美秩。仍示歸生之賻,兼推給復之恩。申飭守臣,優恤其後。豈獨旌於泉壤,亦足厚於民風。可特贈秘書省著作佐郎,賜其家帛二十匹,米三十斛,州縣常加
 存恤,二稅外蠲其差役。」



 魏野,字仲先,陜州陜人也。世為農。母嘗夢引袂於月中承兔得之,因有娠,遂生野。及長,嗜吟詠,不求聞達。居州之東郊,手植竹樹,清泉環繞,旁對雲山,景趣幽絕。鑿土袤丈,曰樂天洞,前為草堂,彈琴其中,好事者多載酒肴從之游,嘯詠終日。前後郡守,雖武臣舊相,皆所禮遇,或親造謁。趙昌言性尤倨傲,特署賓次,戒閽吏野至即報。野不喜巾幘,無貴賤,皆紗帽白衣以見,出則跨白驢。過
 客居士往來留題命話,累宿而去。野為詩精苦,有唐人風格,多警策句。所有《草堂集》十卷,大中祥符初契丹使至,嘗言本國得其上帙,願求全部,詔與之。



 祀汾陰歲,與李瀆並被薦,遣陜令王希招之,野上言曰:「陛下告成天地,延聘巖藪,臣實愚戇,資性慵拙,幸逢聖世,獲安故里,早樂吟詠,實匪風騷,豈意天慈,曲垂搜引。但以嘗嬰心疾,尤疏禮節,麋鹿之性,頓纓則狂,豈可瞻對殿墀,仰奉清燕。望回過聽,許令愚守,則畎畝之間,永荷帝力。」詔州
 縣長吏常加存撫,又遣使圖其所居觀之。五年四月,復遣內侍存問。天禧三年十二月,無疾而卒,年六十。州上其狀。



 四年正月,詔曰:「國家舉旌賞之命,以輝丘園,申恤贈之恩,用慰泉壤,所以褒逸民而厚風俗也。故陜州處士魏野,服膺儒素,刻意篇章,顧詞格之清新,為士流之推許,而能篤淳古之行,慕肥遁之風。頃屬時巡,嘗加聘召,懇陳誠志,願遂《考槃》。及此淪亡,載深嗟悼!蘭臺清秩,追飾幽扃,厚其賻助之資,寬以復除之命。諒惟優禮,式
 顯令名。魂而有知,歆此殊渥。可特贈秘書省著作郎,賻其家帛二十匹,米三十斛,州縣常加存恤,二稅外免其差徭。」



 瀆即野中表兄也。瀆卒訃至,野哭之慟,謂其子曰:「吾不可去,去必不至。」第遣其子赴之,裁六日而野亦卒,時甚異焉。



 邢敦,字君雅,不知何許人。家於雍丘,與宋準、趙昌言交游甚厚。太平興國初,嘗舉進士不第,慨然有隱遁意。性介僻,不妄交友。耽玩經史,精於術數,工繪畫,頗嗜酒。或
 游市廛,過客詢以休咎者,多不之語。里中號邢夫子。大中祥符七年,真宗幸亳回,邑人列上其事,王曾為考制度使,以名聞。詔曰:「敦早預詞場,勤修天爵,超然處退,亦既累年。屬覽公車之言,俾參郡學之職,用精儒業,以寵耆年。可許州助教。」敦讓而不受。乾興元年,無疾而卒,年七十四。



 林逋,字君復,杭州錢塘人。少孤,力學,不為章句。性恬淡好古,弗趨榮利,家貧衣食不足,晏如也。初放游江、淮間,
 久之歸杭州,結廬西湖之孤山,二十年足不及城市。真宗聞其名,賜粟帛,詔長吏歲時勞問。薛映、李及在杭州,每造其廬,清談終日而去。嘗自為墓於其廬側。臨終為詩,有「茂陵他日求遺稿,猶喜曾無《封禪書》」之句。既卒,州為上聞,仁宗嗟悼,賜謚和靖先生,賻粟帛。



 逋善行書,喜為詩,其詞澄浹峭特,多奇句。既就稿,隨輒棄之。或謂:「何不錄以示後世?」逋曰:「吾方晦跡林壑,且不欲以詩名一時,況後世乎!」然好事者往往竊記之,今所傳尚三百餘
 篇。



 逋嘗客臨江,時李諮方舉進士,未有知者,逋謂人曰:「此公輔器也。」及逋卒,諮適罷三司使為州守,為素服,與其門人臨七日,葬之,刻遺句內壙中。



 逋不娶,無子,教兄子宥,登進士甲科。宥子大年,頗介潔自喜,英宗時,為侍御史,連被臺移出治獄,拒不肯行,為中丞唐介所奏,降知蘄州,卒於官。



 高懌,字文悅,荊南高季興四世孫。幼孤,養於外家。十三歲能屬文,通經史百家之書。聞種放隱終南山,乃築室
 豹林谷,從放受業。放奇之,不敢處以弟子行。與同時張蕘、許勃號「南山三友」。



 會詔舉沈淪草澤,知長安寇準聞其名薦之,辭不起。景祐中,錄國初侯王後,懌推其弟忻得官。及範雍建京兆府學,召懌講授諸生,席間常數十百人。杜衍嘗請賜處士號,乃命為大理評事,懌固辭。仁宗嘉其守,號安素處士。詔州縣歲時禮遇之,給良田五百畝。文彥博表其經術該通,有高世之行,可以勵風俗,詔賜第一區。嘉祐中,就除光祿寺丞,復固辭。夢道士持
 素書聘為白鹿洞主,卒。



 有韓退者,稷山人。亦師事種放。母死,負土成墳,徒跣終喪,去隱嵩山。吳遵路,石延年論其高節。詔賜粟帛,號安逸處士,以壽終。



 徐復,字復之,建州人。初游京師,舉進士不中。退而學《易》,通流衍卦氣法,自筮知無祿,遂亡進取意。游學淮、浙間數年,益通陰陽、天文、地理、遁甲、占射諸家之說。他日聽其鄉人林鴻範說《詩》,且言《詩》之所以用於樂者,忽若有得。因以聲器求之,遂悟大樂,於七音、十二律清濁次序
 及鐘磬侈弇、匏竹高下制度皆洞達。方仁宗留意於樂,詔天下求知樂者,大臣薦胡瑗,瑗作鐘磬,大變古法。復笑曰:「聖人寓器以聲,今不先求其聲而更其器,其可用乎!」後瑗制作皆不效。



 範仲淹過潤州,見復問曰:「今以衍卦占之,四夷無變異乎?」復克西方當用兵,推其月日,後無少差。慶歷初,與布衣郭京俱召見,帝問天時人事,復對曰:「以京房《易》卦推之,今年所配年月日時,當小過也。剛失位而不中,其在強君德乎?」帝又問:「明年主何卦?」復
 曰:「《乾》卦用事。」說至九五盡而止。帝又問:「前年京師黑風,何所應?」復曰:「其兆在內,豫王喪其應也。」明日,命為大理評事,固以疾辭,乃賜號沖晦處士,補其子發試秘書省校書郎。復性高潔,而處世未嘗自異,後居杭州十數年卒。



 郭京者,少任俠,不事家產,平居好言兵。範仲淹、滕宗諒數薦之。



 孔旼,字寧極,孔子四十六代孫。隱居汝州龍興縣龍山之蚩陽城。性孤潔,喜讀書。有田數百畝,賦稅常為鄉里
 先。遇歲饑,分所餘賙不足者,未嘗計有無。聞人之善若出於己,動止必依禮法。環所居百餘里,人皆愛慕之,見旼於路,輒斂衽以避。葬其父,廬墓三年,臥破棺中,日食米一溢。壁間生紫芝數十本。州以行義聞,賜粟帛,又給復其家。近臣列薦,授秘書省校書郎致仕。居數年,召為國子監直講,辭不赴,即遷光祿寺丞。頃之,起知龍興縣,復辭。卒,贈太常丞。



 盜嘗入旼家,發其廩粟,旼避之,縱其所取。嘗逢羸弱者為盜掠奪其貲,旼追盜與語,責之以
 義,解金畀之,使歸所掠。居山未嘗逢毒蛇虎豹,或謂之曰:「子毋夜行,此亦可畏。」旼曰:「無心則無所畏。」晚年惟玩《周易》、《老子》,他書亦不復讀。為《太玄圖》張壁上,外列方州部家,而規其中心,空之無所書。曰:「《易》所謂寂然不動者,與此無異也。」



 何群,字通夫,果州西充人。嗜古學,喜激揚論議,雖業進士,非其好也。慶歷中,石介在太學,四方諸生來學者數千人,群亦自蜀至。方講官會諸生講,介曰:「生等知何群
 乎?群日思為仁義而已,不知饑寒之切己也。」眾皆注仰之。介因館群於其家,使弟子推以為學長。群愈自克厲,著書數十篇,與人言未嘗下意曲從,同舍目群為「白衣御史」。



 群嘗言:「今之士,語言說易,舉止惰肆者,其衣冠不如古之嚴也。」因請復古衣冠。又上書言:「三代取士,皆舉於鄉里而先行義。後世專以文辭就,文辭中害道者莫甚於賦,請罷去。」介贊美其說。會諫官御史亦言以賦取士無益治道,下兩制議,皆以為進士科始隋歷唐數百
 年,將相多出此,不為不得人,且祖宗行之已久,不可廢也。群聞其說不行,乃慟哭,取平生所為賦八百餘篇焚之。講官視群賦既多且工,以為不情,絀出太學。群徑歸,遂不復舉進士。



 嘉祐中,龍圖閣直學士何剡表其行義,賜號安逸處士。群既死,趙抃守益州,奏群遺稿有益時政,願詔果州錄上之,云:「非若茂陵書起天子侈心也。」寢不下。



\end{pinyinscope}