\article{樂志第九十 樂十二(樂章六)}

\begin{pinyinscope}

 祭太
 社太稷祭風雨雷師祭先農先蠶親耕藉田



 蠟祭釋奠文宣王武成王祭祚德廟祭司中司命



 景德祭社稷三首



 降神,《靜安》



 百穀蕃滋,麗乎下土。聿崇明祀,垂之千古。



 育物惟茂,粒民斯普。報本攸宜,國章咸睹。



 奠玉幣酌獻,《嘉安》



 於穆大祀,功利相宣。靈壇美報,歷代昭然。



 介以蕃祉。祚以豐年。土爰稼穡,允協民天。



 送神,《靜安》



 制幣犧齊,正辭無愧。樂以送之,畢其精意。



 景祐祀社稷三首



 迎神,《寧安》



 五紀之本,百貨何極?道著開闢,惠周動植。



 國崇美穀,民資力穡。奠獻惟寅,神靈來格。



 初獻升降,《正安》;太社、后土、太稷、后稷奠玉幣,並《嘉安》;奉俎,《豐安》。同前。



 亞、終獻,《文安》;送神,《寧安》神之來兮,降茲下土。神之去兮,杳無處所。壇壝肅然,瘞幣徹俎。乃粒之功,冠於萬古。



 奉俎,《豐安》神州地祇、皇地祇與社稷通用。



 禮崇明禋,維馨斯酒,潔粢豐盛,殺時犉牡。



 齊莊嚴祗,升燎於□□。其報伊何?如山如阜。



 大觀祀社稷九首



 迎神,《寧安》



 黃鐘二奏



 惟土之尊,民食資焉。陰祀昭格,牲牢腥膻。



 有功於民,告其吉蠲。神之來享,雲車翩翩。



 太簇角二奏



 惟穀之神,函育無窮。百嘉蕃殖,民依厥功。



 嚴飭壇壝,威儀肅雍。神之來享,祈於登豐。



 姑洗徵二奏



 猗歟那歟,生養斯民!家給人足,時底熙純。



 祗嚴明禋,於薦苾芬。粢盛豐潔,神乃有聞。



 南呂羽二奏



 籩豆斯陳,三牲告幽。報本之禮,答神之
 休。



 來歆芬香,豐登於秋。倉箱千萬,治符成周。



 初獻升降,《正安》



 崇崇廣壇,嚴恭祀事。威儀孔時,周旋進止。



 鏘若環佩,誠通於幽。相於農植,邦其咸休。



 奠幣,《嘉安》



 於嘻陰祀,封土惟崇。於時之吉,歆予鼓鐘。



 柔靜化光,人賴其功。陳茲量幣,百貨是隆。



 酌獻,《嘉安》



 坤元生物,功利相宣。蠲茲祀事,美報致虔。



 清酤芬如,靈壇巋然。酌尊奠觴,神其格焉!



 亞、終獻,《文安》



 薦嘉但時,洋洋來格。載登茲壇,齊明維
 敕。



 神用居歆,順成農穡。其崇若墉,其比如櫛。



 送神,《寧安》



 尊罍芬香,威儀肅雍。靈心嘉止,洋洋交通。



 神歸降禧,年斯屢豐。倉箱千萬,慰予三農。



 紹興祀太社太稷十七首



 迎神用《寧安》



 函鐘為宮春社用。



 五祀之本,社稷有嚴。芟柞伊始,夫敢不虔。



 吉日惟戊,式薦豆籩。神其來格,用介有年!



 函鐘為宮秋社、臘用。



 功烈在民,誕受露雨。《良耜》既歌,乃揚
 帗舞。



 是奉是尊,厚禮斯舉。相其豐年,多稌多黍。



 太簇為角



 是尊是奉,茲率舊章。樂音純繹,薦溢圓方。



 情文備矣,神其迪嘗!永觀錫羨,多穡穰穰。



 姑洗為征



 穀資土養,民賴穀生。功利之博,莫之與京。



 式嚴祠壇,因物薦誠。禮具樂奏,惟神顧歆。



 南呂為羽



 國主社稷,時祀有常。肅若舊典,報本不忘。



 粢盛豐潔,歌吟青黃。尊神倏來,百物賓將。



 盥洗,《正安》



 祭重齊肅,神格專精。沃洗於阼,涓潔著誠。



 清明鬯矣,熙事備成。以似以續,如坻如京。



 升壇,《正安》



 神地之道,粒食有先。歲謹祈報,禮嚴豆籩。



 降登裸薦。罔或不虔。以似以續,宜屢豐年。



 太社位奠玉幣,《嘉安》春秋太稷、土正、后稷通用。



 土發而祭,農祥是祈。籩豆加篚,典禮有彞。



 惟茲珪幣,用告肅祗。神靈降鑒,錫我繁厘。



 太社位奠玉幣,《嘉安》秋臘太稷、土正、后稷通用。



 赫赫媼神,稼穡是司。方是藉斂,報本攸宜。



 嘉壇建祀,玉帛陳儀。明靈昭
 格,以介蕃厘。



 還位,《正安》



 國主太祀,地道聿神。稷司百穀,利毓惟均。



 練日新吉,粢盛苾芬。神燕娭矣,福此下民。



 捧俎,《豐安》



 嘉承天和,黍稷翼翼。默相農功,繄神之德。



 俎實犧牲,舊章是式。嗣有豐年,我瘐維億。



 太社位酌獻,《嘉安》春社太稷、土正、后稷通用。



 封土崇祀,有烈在民。千載不昧,福此人群。



 洗爵奠斝,有酒其芬。神具醉止,愷樂欣欣。



 太社位酌獻,《嘉安》秋社臘太稷、正土、后稷通用。



 葉氣嘉生,年穀順成。萬億及秭,如坻如京。



 奉時櫜牡,告於神明。歌此《良耜》,於昭德馨。



 亞、終獻,《文安》



 風雨時若,自天降康。稼穡滋殖,自神發祥。



 穀我婦子,豐年穰穰。報本嚴祀,齊明允臧。



 徹豆,《娭安》



 報本之禮,載於甲令。靈壇昭告,神既來聽。



 徹彼豆籩,精誠斯罄。實惟豐年,農夫之慶。



 送神,《寧安》



 乃粒烝民,功昭萬古。國有常祀,薦獻式敘。



 肅肅雍雍,舊章咸舉。神保聿歸,介我稷黍。



 望瘞,《正安》



 地載萬物,民資乃功。報本稱祀,太稷攸同。



 禮樂既備,訖埋愈恭。神其降嘏,時和歲豐。



 熙寧祭風師五首



 迎神,《欣安》



 飄颻而來,淅瀝而下。爰張其旗,爰整其駕。



 有豆有登,有兆有壇。弭旌柅輈,降止且安。



 升降,《欽安》



 盥帨於下,有盤有匜。饋酌於上,有登有彞。



 服容柔止,進退優止。即事寅恭,神其休止。



 奠幣,《容安》



 育我嘉生,神惠是仰。載致斯幣,庶幾用享。



 鼓之舞之,式繄爾神。錫福無疆,祐此下民。



 亞、終獻,《雍安》



 慄慄壇坫,載是豆觴。醇烈氤氳,普薦芬芳。



 酌之維宜,獻之維時。民有報侑,靈用安之。



 送神,《欣安》



 奠獻紛紛,靈心欣欣。超然而返,眾御如云。



 其旋伊何?多黍多稌。其祥伊何?不愆厥敘。



 大觀祭風師六首



 降神,《欣安》



 羽旗雲車,飄颻自天。猗歟南箕,歆嘉升煙!



 牲餼粢盛,俎簋鉶籩。維神戾止,從空冷然。



 初獻升降,《欽安》



 明昭惟馨,威儀孔時。鏘鏘鳴佩,欽薦牲犧。



 惟恭惟祗,無愆無違。周旋中禮,肅恭委蛇。



 奠幣,《容安》



 吹噓於喁,披拂氤氳。眾竅咸作,潛運化鈞。



 恩大功豐,酬神維恭。嘉贈盈箱,於物有容。



 酌獻,《雍安》



 犧尊斯陳,清酤盈中。芬芬苾苾,馨香交通。



 明靈來思,歆我精衷。維千萬祀,品物芃芃。



 亞、終獻,《雍安》



 清酤洋洋,虔恭注茲。條鬯敷宣,神用歆
 之。



 尊罍靜嘉,金奏諧熙。於皇肆祀,休我群黎。



 送神,《欣安》



 窈冥無窮,肸蠁斯融。來終嘉薦,歸返遙空。



 惟神之歸,欣安導和。惟神之澤,於彼滂沱。



 雨師五首



 迎神,《欣安》



 神之無象,亦可思索。維雲陰陰,維風莫莫。



 降止壇宇,來顧芳馨。侑以鼓歌,薦此明誠。



 升降,《欽安》



 佩玉璆如,黼黻示詹如。承神不懈,訖獲嘉虞。



 聖皇命祀,臣敢弗恭。凡爾在位,翼翼雍雍。



 奠幣,《容安》



 崇崇壇階,靈既降止。有嚴執奠,承祀茲始。



 明靈在天,式顧庶察。澤潤以時,永拂荒札。



 酌獻,亞、終獻,《雍安》



 寅恭我神,惟上之使。俾成康年,民徯休祉。



 折俎既登,□酒既盈。匪薦是專,配以明誠。



 送神,《欣安》



 牲俎告徹,嘉樂休成。卒事有嚴,燕虞高靈。



 蕃我民人,育我稷黍。萬有千祀,承神之祐。



 紹興祭風師六首



 迎神,《欣安》



 夫物絪縕,神氣撓之。誰歟其司?維南之箕。



 俶哉明庶,我祀維時!我心孔勞,神其下來!



 初獻升降、盥洗,《欽安》



 神哉沛矣,厥靈載揚!揚靈如何?剡剡皇皇。



 我其承之,繩繩齊莊。往從鬱人,爰俠挾芳。



 奠幣,《容安》



 物之流形,甚畏瘥癘。八風平矣,嘉生以遂。



 絲縷之積,有量斯幣。惟本之報,匪物之貴。



 酌獻,《雍安》



 我求於神,無臭無聲。神之燕享,惟時專精。



 大磬在列,□□燎在庭。侑我桂酒,娭其以聽。



 亞、終獻



 禮有三祀,儀物視帝。神臨消搖,疇敢跛倚!



 重
 觴載申,百味孔旨。神兮樂康,答我以祉。



 送神曲同迎神。



 荃其止乎?禗禗其容。奄橫四海,蹇莫之窮。時不驟得,禮焉有終。荃其行乎?余心心蟲□。



 雨師雷神七首



 迎神,《欣安》



 眾萬之托,動之潤之。昭格孔時,維神之依。



 冷然後先,肆我肯顧。是耶非耶?紛其來下。



 初獻盥洗、升降,《欽安》



 言言祠宮,爰考我禮。維西有罍,維東有洗。



 爰潔爰滌,載薦其醴。神在何斯?匪遠具邇。



 奠幣,《容安》



 霈兮隱兮,蹶其陰威。相我有終,胡寧不知!



 我幣有陳,我邸斯珪。豈維有陳,於以奠之。



 雨師位酌獻,《雍安》



 山川出雲,裔裔而縷。載霪載蒙,其德乃溥。



 自古有年,胡然莫祖!無簡我觴,無怠我俎。



 雷神位酌獻曲同初獻。



 瞻彼南山,有虺其出。維蟄之奮,維癘之息。



 眷焉顧饗,在夏之日。觴豆匪報,皇忍忘德。



 亞、終獻曲同雨師。



 作解之德,形聲一兮。爰展獻侑,酌則三兮。



 我興有假,云胡有私!下土是冒,庶其遠而。



 送神曲同迎神。



 陰旄載旋,鼓車其鞭。問神安歸?冥然而天。



 皇有正命,祀事孔蠲。其臨其歸,億萬斯年。



 雍熙享先農六首餘同祈穀。



 降神,《靜安》



 先農播種,九穀務滋。靈壇致享,《良耜》陳儀。



 吉日惟亥,運屬純熙。樂之作矣,神其格思。



 奠玉幣,《敷安》



 親耕展祀,明靈來格。九有駿奔,百司庇職。



 獻奠肅肅,登降翼翼。祈彼豐穰,福流萬國。



 奉俎,《豐安》



 肅陳《韶》舞,祗薦犧牲。乃逆黃道,以率躬耕。



 亞獻,《正安》



 祀惟古典,食乃民天。歆茲潔祀,以應祈年。



 終獻,《正安》



 式陳芳薦,爰致虔誠。神其降鑒,永福黎氓。



 送神,《靜安》



 明禋紺壇,靈風肅然。登歌已闋,神馭將旋。



 道光帝籍,禮備公田。鑒茲躬稼,永賜豐年。



 明道親享先農十首



 迎神,《靜安》



 稼政之本,民食惟天。《甫田》兆歲,後稷其先。



 靈壇既祀,黛耜攸虔。乃聖能享,億萬斯年。



 皇帝升降,《隆安》



 冕服在御,壇壝有儀。陟降左右,天惟
 顯思。



 奠玉幣,《嘉安》



 將躬黛耜,先陟靈壇。嘉玉量幣,樂舉禮殫。



 神既至止,福亦和安。乾斯積詠,萬國多歡。



 奉俎,《豐安》



 將迎景福,乃薦嘉牲。籍於千畝,用此精誠。



 皇帝初獻,《禧安》



 雲罍已實,玉爵有舟。薦於靈籍,佇乃神休。



 飲福,《禧安》



 神既至享,福亦來酬。申錫純嘏,旨酒維柔。



 思文后稷,貽我來牟。子孫千億,丕荷天休。



 退文舞、進武舞,《正安》



 羽葆有奕,文武交相。周旋合度,福祿無疆。



 亞獻,《正安》



 豆籩雖薦,黍稷非馨。惠我豐歲,歆茲至誠。



 終獻,《正安》



 歆我嘉薦,錫我蕃禧。多黍多稌,如京如坻。



 送神,《靜安》



 獻終豆徹,禮備樂成。祠容肅肅,風馭冥冥。



 三時務本,一伐躬耕。人祗胥悅,祉福是膺。



 景祐享先農五首



 迎神,《凝安》



 在昔神農,首茲播殖。無有污萊,盡為稼穡。



 乃粒斯民,實惟帝力。嘉薦令芳,佇瞻來格。



 升降,《同安》



 居德之厚,厥祀攸陳。土膏初脈,農事先春。



 鏗然金奏,儼若華紳。陟降於阼,福祿惟神。



 奠幣,《明安》



 農為政本,食乃民天。苾芬明祀,藨□良田。



 陳茲量幣,望彼豐年。茂介福祉,來欽吉蠲。



 酌獻,《成安》



 農祥晨正,平秩東作。倬彼大田,庤乃錢鎛。



 酒醴盈尊,金璆合樂。期茲萬年,充於六幕。



 送神,《凝安》



 務嗇之本,恤祀惟馨。神斯至止,降福攸寧。



 崇茲稼政,合於禮經。俎徹樂闋,邈仰回靈。



 先蠶六首



 迎神,《明安》



 生民之朔,衣皮而群。惟聖有作,被冒以文。



 禮樂以成,貴賤以分。欲報之德,金石諧均。



 升降,《翊安》



 掩抑笙簫,鏗谹金石。神來宴娭,嘉我休德。



 奉祀之臣,洗心翊翊。錫茲福禧,以惠四國。



 奠幣,《娭安》



 皇天降物,屢化若神。聖實先識,躬以教民。



 功被天下,為萬世文。幣以達志,庶幾徹聞。



 酌獻,《美安》



 □哉聖神,成功微妙!乃袞乃裳,以供郊廟。



 百末旨酒,嘉觴自照。靈徠宴饗,不嚬以笑。



 亞、終獻,《惠安》



 神之徠,駕蹌蹌。紫壇熙,燭夜光。會竽瑟,



 鳴球瑯。薦旨酒,雜蘭芳。祐明德,賜百祥。



 送神,《祥安》



 神之功兮,四海所宗。占五帝兮,莫與比崇。



 倏往來兮,旌旗容容。恭明祀兮,萬世無窮。



 紹興享先農十一首



 皇帝入內壝盥洗,《隆安》



 大事在祀,齊潔為先。既盥而
 升,奉以周旋。



 下觀而化,無敢不蠲。惟神降格,監厥精虔。



 迎神,《靜安》



 猗歟田祖,粒食之宗!世世仰德,青壇載崇。



 時惟后稷,躬稼同功。作配並祀,以詔無窮。



 神農、后稷位奠幣,《嘉安》



 制為量幣,厚意是將。求之以類,各因其方。



 於以奠之,精誠允彰。神其享止,惠我無疆。



 尚書奉俎,《豐安》



 柔毛剛鬣,或剝或烹。為俎孔碩,登薦
 厥誠。



 酌獻,《禧安》



 蠲滌醆斝,巾帨而升。挹彼注茲,酒醴維清。



 洋洋在上。享於克誠。神其孚祐,以厚民生。



 文舞退、武舞進,《正安》



 羽毛幹戚,張弛則殊。進旅退旅,匪棘匪舒。



 亞獻,《正安》



 顯相祀事,濟濟鏘鏘。舉斝酌醴,神其允臧。



 終獻,《正安》



 殽核維旅,酒醴維馨。於再於三,禮則有成。



 飲福,《禧安》



 幽明位異,施報理同。克恭明神,降福乃豐。



 我膺受之,來燕來崇。豈伊專享,於彼三農。



 徹豆,《歆安》



 莫重於祭,非禮不成。籩豆有踐,爾殽既馨。



 神具醉止,薦以齊明。贊徹孔時,厘事斯成。



 送神,《靜安》



 神之來止,風駛雲翔。神之旋歸,有迎有將。



 歌以送之,磬管鏘鏘。何以惠民?豐年穰穰。



 親耕藉田七首



 皇帝出大次,《乾安》



 勤勞稼穡,必躬必親。為藉千畝,以教導民。



 帝出乎震,時惟上春。天顏咫尺,望之如云。



 親耕



 元辰既擇,禮備樂成。洪縻在手,祗飾專精。



 三推一伐,端冕朱紘。靡辭染屨,以示黎氓。



 升壇



 方壇屹立,陛級而登。玉色下照,臨觀耦耕。



 萬目咸睹,如日之升。成規成矩,百祿是膺。



 公卿耕藉



 群公顯相,奉事齋莊。率時農夫,舉耜載揚。



 播厥百穀。以祐我皇。多黍多稌,丕應農祥。



 群官耕藉



 畟畟良耜,我田既臧。土膏其動,春日載陽。



 執事有恪,於此中邦。農夫之慶,棲畝餘糧。



 降壇



 肇新帝藉,率我農人。三推終畝,祗事咸均。



 陟降孔時,粲然有文。受天之祜,多稼如云。



 歸大次



 教民稼穡,不令而行。進退有度,琚瑀鏘鳴。



 言還熉幄,禮則告成。帝命率育,明德惟馨。



 紹興祀先農攝事七首



 迎神,《凝安》



 青陽開動,土膏脈起。日練吉亥,為農祈祉。



 典秩增峻,儀物具美。幄光熉黃,庶幾戾止。



 初獻升殿,《同安》盥洗同。



 率職咸蒞,禮容睟然。澡身端意,
 陟降靡愆。



 神心嘉虞,享茲潔蠲。敷錫純祐,屢登豐年。



 奠幣,《明安》



 靈斿載臨,見光陳贄。有嚴篚實,式將純意。



 肸蠁既接,禮行有次。神兮安留,歆我禋祀。



 神農位酌獻,《成安》



 耒耜之教,帝實開先。致養垂利,古今民天。



 嘉薦報本,於以祈年。誠格和應,神娭福延。



 後稷位酌獻,《成安》



 有周膺歷,實起後稷。相時神功,率由稼穡。



 振古稱祀,先農並食,阜我昌我,時萬時億。



 亞、終獻,《同安》



 旨具百味,酌備三疇。貳觴既畢,禮洽意
 周。



 庶幾嘉享,格神之幽。相我穡事,錫以有秋。



 送神,《凝安》



 熙事成兮,始終潔齊。籩豆徹兮,撙節靡垂。



 靈有嘉兮,降福孔皆。飄然逝兮,我心孔懷。



 祀先蠶六首



 迎神,《明安》



 功被寰宇,人蠡蟲之靈。有神司之,以生以成。



 典禮有初,祀事講明。孔蓋翠旌,降集於庭。



 初獻盥洗、升殿,《翊安》降同。



 靈修戾止,詔以毛血。既盥而帨,尊爵蠲潔。



 金石諧宛,登降有節。宜顧享鄉,情文不
 越。



 奠幣,《娛安》



 化日初長,時當暮春。蠶事方興,惟後惟嬪。



 絲纊御冬,殘生濟人。敢忘報本,篚幣是陳。



 酌獻,《美安》



 盛服承祀,出自公桑。衣不羽皮,利及萬方。



 百味旨酒,有飶其香。神其歆止,洋洋在傍。



 亞、終獻,《惠安》



 日吉辰良,禮備樂作。精誠內孚,俎豆交錯。



 升歌清越,侑此三爵。黎民不寒,幽顯同樂。



 送神,《祥安》



 神之來矣,靈風肅然。云胡不留?歸旐有翩。



 乃舉舊典,歲以告虔。降福我邦,於萬斯年。



 景德蠟祭百神三首



 降神,《高安》



 百物蕃阜,四方順成。通其八蠟,合乃嘉平。



 旨酒斯醇,大庖孔盈。萬靈來格,威儀以成。



 奠玉幣酌獻,《嘉安》



 肅肅靈壇,昭昭上天,潔粢豐盛,以享以虔。



 百神咸萃,六樂斯縣。介茲景福,期於百年。



 送神,《高安》



 來顧來享,禮成樂備。靈馭翩翩,雲行雨施。



 熙寧蠟祭十三首



 東、西郊降神,《熙安》



 天錫康年,四方順成。乃通蠟祭,索享於明。



 金石四作,羽旄翠旌。神來宴娭,澤被群生。



 升降,《肅安》



 惟蠟有祭,報神之功。合聚萬物,來享來宗。



 承詔攝事,不忘肅雍。靈之格思,福祿來崇。



 奠幣,《欽安》



 穰穰豐年,繄侯休德。帥承天和,欽象古則。



 嘉玉量幣,奠容翼翼。靈施暨民,罔有終極。



 奉俎,《承安》



 禮崇明祀,必先成民。奉牲以告,備腯其均。



 炮炙芬芬。俎豆莘莘。錫之純嘏,以祐斯人。



 酌獻,《懌安》



 秩秩禮文,為壇四方。嘉慄旨酒,百神迪嘗。



 敷與萬物,既阜既昌。伊樂厥福,傳世無疆。



 亞、終獻,《慶安》



 禮文備矣,肅肅無嘩。金石諧節,圭璧光華。



 粢以告豐,醴以告嘉。錫茲福祉,以澤幽遐。



 送神,《宣安》



 靈之來下,擴景乘光。靈之回禦,景龍以驤。



 鑒我休德,降嘏產祥。大田多稼,以惠無疆。



 南、北方迎神,《簡安》



 美若休德,民和歲豐。稼穡云施,其積如墉。



 惠我四國,先嗇之功。祭之百種,來享來宗。



 升降,《穆安》



 皇皇靈德,經緯萬方。承詔攝事,陟降以莊。



 高冠岌峨,長佩鏘洋。嘉承神貺,令聞不忘。



 奠幣,《嘉安》



 於穆明祀,莫如報功。靈之利澤,惠我無窮。



 齋以滌志,幣以達衷。撫寧四極,永錫登豐。



 酌獻,《禔安》



 英英禮文,既備而全。嚴嚴四郊,屹屹紫壇。



 百末旨酒,其馨若蘭。何以畀民?既壽而安。



 亞、終獻,《曼安》



 林林生民,含哺而嬉。教之稼穡,實神之為。



 圖報厚德,萬祀無期。以假以享,錫我繁禧。



 送神,《成安》



 嘉薦芳美,靈來宴娭。斿車結雲,若風馬馳。



 既至而喜,錫我蕃禧。嘉承天貺,曼壽無期。



 大觀蠟祭二首



 東郊亞、終獻,《慶安》



 震乘春陽,仁司生殖。錫我歲豐,襄我民力。



 誰其尸之?宗子先嗇。億萬斯年,懷神罔極。



 南郊升降,《穆安》



 穆如熏風,敷舒文藻。氣蒸消除,豐予黍稻。



 神之聽之,鐘鼓咸考。於萬斯年,惟皇之報。



 紹興以後蠟祭四十二首



 東方百神降神,《熙安》



 圜鐘為宮



 玄冥凌厲,歲聿其周。天地閉藏,農且息休。



 古為蠟禮,伊耆肇修。爰薦飶馨,以迓飆斿。



 黃鐘為角



 惟大明尊,實首三辰。功赫庶物,光被廣輪。



 歲方索享,咸秩群神。靈斿來下,尸此明禋。



 太簇為征



 三時不害,四方順成。酬功報始,以我齋明。



 《豳》頌土鼓,樂此嘉平。降祥幅員,惠於函生。



 姑洗為羽



 日昱乎晝,容光必照。肸蠁之交,惟人所召。



 有監在下,視茲升燎。肅若其承,云駢星曜。



 初獻升降,《肅安》



 禮儀告具,心儼容莊。工歌屢奏,聲和義章。



 崇壇陟降,濟濟蹌蹌。靈光共仰,嘉薦芬芳。



 大明位奠玉幣,《欽安》



 晨曦未融,天宇澄穆。有虔秉誠,將以幣玉。



 如在左右,罔不祗肅。神兮安留,錫以祉福。



 帝神農氏位奠幣曲同大明。



 農為政本,食乃民天。神農氏作,民始力田。



 先嗇之配,禮報則然。有幣將之,維以告虔。



 後稷氏位奠幣曲同大明。



 播種之功,時惟后稷。推以配天,莫匪爾極。



 崇侑清祀,是為司嗇。陳幣奠將,永祚王國。



 奉俎,《承安》



 享以精禋,馨非稷黍。工祝致告,孔碩為俎。



 執事駿奔,繩繩具舉。神之嘉虞,介福是與。



 大明位酌獻,《擇安》



 肇禋備祀,教民美報,時和歲豐,奉醴以告。



 惟照臨功,等於載燾。酌獻雲初,明神所勞。



 神農位酌獻曲同大明。



 惟酒欣欣,惟神冥冥。是顧是享,來燕來寧。



 耒耜之利,神所肇興。萬世永賴,無斁其承。



 後稷位酌獻曲同大明。



 釋之蒸之,為酒為醴。推本所由,於焉洽禮。



 周邦開基,邰家是啟。獻茲嘉觴。拜下首稽。



 亞、終獻,《慶安》



 申以貳觴,百味且旨。禮告三終,神具醉止。



 旌容騎沓,揚光紛委。降福穰穰,被大豐美。



 送神,《宣安》



 禮樂既成,神保聿歸。言歸何所?地紀天維。



 豈惟屢豐,嗣歲所祈。億萬斯年,神來燕娭。



 西方百神降神,《熙安》



 圜鐘為宮



 玄冬肇祀,始於伊耆。歲事聿成,庸答蕃厘。



 眷言西顧,匪神司之。歸功爾神,
 翩其下來。



 黃鐘為角



 魄生自西,照望太陽。下暨諸神,貺施萬方。



 節適風雨,富我囷箱。共承嘉祀,惟以迪嘗。



 太簇為征



 神罔小大,奠方茲土。祭列坊墉,禮迨貓虎。



 有功斯民,祀乃其所。非稷馨香,厥福周溥。



 姑洗為羽



 豐年穰穰,美芳職職。籩豆方圓,其儀孔碩。



 風馬在御,雲車載飭。來顧來享,維俟休德。



 初獻升降,《肅安》



 盥獻恭莊,燎煙芬酷,載陟載降,禮容
 可度。



 欽惟爾神,上下肅肅。成我稷黍,鑒此牲玉。



 夜明位奠玉幣,《欽安》



 穆穆太陰,禮嚴姊事。璧玉華光,推以槥對。



 十二周天,歲乃有終。盡我備物,莫報元功。



 神農位奠幣曲同夜明。



 耒耜肇興,白神農氏。稼穡滋殖,為農者始。



 作配明祀,奠以告虔。萬世佃漁,帝功卓然。



 後稷位奠幣曲同夜明。



 明明周祖,惟民之恤。播種為教,下民乃粒,



 曾是索饗,而匪先公。萬物難報,阡陌之功。



 奉俎,《承安》



 時和歲登,物亡疾瘯。實俎間膏,報神之福。



 匪神之福,曷成且豐!肥腯咸有,惟神之功!



 夜明位酌獻,《擇安》



 除壇西郊,坎其擊鼓。百靈至止,結璘作主。



 秬鬯湛淡,玉斝觩。是謂嘉德,神其安留。



 神農位酌獻曲同夜明。



 蕩蕩鴻明,稱秩群祀。配以昔帝,式重農事。



 潔我圭瓚,黃流在中。靈其鑒茲。肸蠁豐融。



 後稷位酌獻曲同夜明。



 歲十二月,祀有常典。登列司嗇,言反其本。



 酌彼泰尊,百末蘭生。承神嘉虞。繄此德馨。



 亞、終獻,《慶安》



 歌磬臚驩,膋蕭激香。飆禦奄留,申以貳
 觴。



 相與震澹,告靈其醉。庶幾聽之,成我熙事。



 送神,《宣安》



 禮備樂成,澹然將歸。其留消搖,像輿已轙。



 偃蹇欲驤,羽毛紛委。忽乘杳冥,遺此福祉。



 南方百神迎神,《簡安》



 維物之精,散乎太空。維索之饗,合聚而同。



 乃擊土鼓,於歲之終。格彼幽矣,肸蠁其通。



 初獻盥洗、升降,《穆安》



 有帨其新,有匜其潔。言念清祀,弗簡弗褻。



 誠意既交,品物斯列。是用告虔。靡神不說。



 奠幣,《吉安》



 百室機杼,衣褐具宜。民以卒歲,神實惠之。



 言舉祀典,答神之厘。有篚斯陳。振古如茲。



 神農位酌獻,《穆安》



 肇降生民,有不粒食。維時神農,乃為先嗇。



 爾耒爾耨,云誰之因。酌以污尊,我思古人。



 後稷位酌獻,《穆安》



 維后之功,配天其大。祀而稷之,萬世如在。



 黃冠野服,駿奔皇皇。自古有年,神其降康。



 亞、終獻,《曼安》



 豐年孔多,百禮以洽。匪極神歡,何以昭答!



 載酌之酒,用申其勤。神具醉止,與物交欣。



 送神,《成安》



 卒爵樂闋,禮儀告備。神保聿歸,敢以辭致。



 順成之方,其蠟乃通。自今以始,八方攸同。



 北方百神迎神,《簡安》



 蕩蕩閫決,氣清泬寥。徬佛象輿,麗於穹霄。



 蹇其來下,肅然風飄。神乎安留,於焉消搖。



 初獻盥洗、升降,《穆安》



 齊誠揭虔,敬恭祀事。維儼之容,維潔之器。



 雍雍樂成,肅肅禮備。神其燕娭,錫祉庶類。



 奠幣,《吉安》配位同。



 神宅於幽,眑□沉沉。至和塞明,考我德音。



 神聽靜嘉,儼乎若臨。幣以薦誠,敢有弗欽。



 神農氏位酌獻,《禔安》



 先嗇之功,神實稱首。以耜以耒,
 俶載南畝。



 列籍皇墳,億世是守。何以為報?爰潔茲酒。



 後稷氏位酌獻,《禔安》



 煌煌后稷,實配於天。司嗇作稼,民以有年。



 匪神之私,歲以醴告。酌彼泰尊,於德之報。



 亞、終獻,《曼安》



 蘭生百末,申以貳觴。神具醉止,爛其容光。



 遺我豐年,萬億及秭。俾民驩康,以洽百禮。



 送神,《成安》



 靈之來兮,虯龍沓沓。下土光景,憑陵閶闔。



 靈之旋兮,羽衙委蛇。偃蹇高驤,遺此蕃厘。



 景祐祭文宣王廟六首



 迎神,《凝安》



 大哉至聖,文教之宗!紀綱王化,丕變民風。



 常祀有秩,備物有容。神其格思,是仰是崇。



 初獻升降,《同安》



 右文興化,憲古師今。明祀有典,吉日惟丁。



 豐犧在俎,雅奏來庭。周旋陟降,福祉是膺。



 奠幣,《明安》



 一王垂法,千古作程。有儀可仰,無德而名。



 齊以滌志,幣以達誠。禮容合度,黍稷非馨。



 酌獻,《成安》



 自天生聖,垂範百王。恪恭明祀,陟降上庠。



 酌彼醇旨,薦此令芳。三獻成禮,率由舊章。



 飲福,《綏安》



 犧像在前,豆籩在列。以享以薦,既芬既潔。



 禮成樂備,人和神悅。祭則受福,率遵無越。



 兗國公配位酌獻,《成安》哲宗朝增此一曲。



 無疆之祀,配侑可宗。事舉以類,與享其從。



 嘉慄旨酒,登薦惟恭。降此遐福,令儀肅雍。



 送神,《凝安》



 肅肅庠序,祀事惟明。大哉宣父,將聖多能!



 歆馨肸蠁,回馭凌兢。祭容斯畢,百福是膺。



 大觀三年釋奠六
 首



 迎神,《凝安》



 仰之彌高,鉆之彌堅。於昭斯文,被於萬年。



 峨峨膠庠,神其來止。思報無窮,敢忘於始。



 升降,《同安》



 生民以來,道莫與京。溫良恭儉,惟神惟明。



 我潔尊罍,陳茲芹藻。言升言旋,式崇斯教。



 奠幣,《明安》



 於論鼓鐘,於茲西雍。粢盛肥碩,有顯其容。



 其容洋洋,咸瞻像設。幣以達誠,歆我明潔。



 酌獻,《成安》



 道德淵源,斯文之宗。功名糠秕,素王之風。



 碩兮斯牲,芬兮斯酒。綏我無疆,與天為久。



 配位酌獻,《成安》



 儼然冠纓,崇然廟庭。百王承祀,涓辰惟丁。



 於牲於醑,其從予享。與聖為徒,其德不爽。



 送神,《凝安》



 肅莊紳緌,吉蠲牲犧。於皇明祀,薦登惟時。



 神之來兮,肸蠁之隨。神之去兮,休嘉之貽。



 大晟府擬撰釋奠十四首



 迎神,《凝安》



 黃鐘為宮



 大哉宣聖,道德尊崇!維持王化,斯民是宗。



 典祀有常,精純並隆。神其來格,於昭盛容。



 大呂為角



 生而知之,有教無私。成均之祀,威儀孔時。



 維茲初丁,潔我盛粢。永適其道,萬世之師。



 太簇為征



 巍巍堂堂,其道如天。清明之象,應物而然。



 時維上丁,備物薦誠。維新禮典,樂諧中聲。



 應鐘為羽



 聖王生知,闡乃儒規。《詩》、《書》文教,萬世昭垂。



 良日惟丁,靈承不爽。揭此精虔,神其來享。



 初獻盥洗,《同安》



 右文興化,憲古師經。明祀有典,吉日惟丁。



 豐犧在俎,雅奏在庭。周旋陟降,福祉是膺。



 升殿,《同安》



 誕興斯文,經天緯地。功加於民,實千萬世。



 笙鏞和鳴,粢盛豐備。肅肅降登,歆茲秩祀。



 奠幣,《明安》



 自生民來,誰底其盛!惟王神明,度越前聖。



 粢幣具成,禮容斯稱。黍稷非馨,惟神之聽。



 奉俎,《豐安》



 道同乎天,人倫之至。有饗無窮,其興萬世。



 既潔斯牲,粢明醑旨。不懈以忱,神之來暨。



 文宣王位酌獻,《成安》



 大哉聖王,實天生德!作樂以崇,時祀無斁。



 清酤惟馨,嘉牲孔碩。薦羞神明,庶幾昭格。



 兗國公位酌獻,《成安》



 庶幾屢空,淵源深矣。亞聖宣猷,
 百世宜祀。



 吉蠲斯辰,昭陳尊簋。旨酒欣欣,神其來止。



 鄒國公位酌獻,《成安》



 道之由興,於皇宣聖。惟公之傳,人知趨正。



 與享在堂,情文實稱。萬年承休,假哉天命。



 亞、終獻用《文安》



 百王宗師,生民物軌。瞻之洋洋,神其寧止。



 酌彼金罍,惟清且旨。登獻惟三,於嘻成禮。



 徹豆,《娛安》



 犧像在前,豆籩在列。以饗以薦,既芬既潔。



 禮成樂備,人和神悅。祭則受福,率遵無越。



 送神,《凝安》



 有嚴學宮,四方來宗。恪恭祀事,威儀雍雍。



 歆茲惟馨,飆馭旋復。明禋斯畢,咸膺百福。



 景祐釋奠武成王六首



 迎神,《凝安》



 維師尚父,四履分封。靈神峻密,祀事寅恭。



 蕭薌祗薦,飆馭排空。如幾如式,福祿來崇。



 太尉升降,《同安》



 上公攝事,袞服斯皇。禮容濟濟,佩響鏘鏘。



 靈斿惚恍,嘉薦令芳。神具醉止,降福穰穰。



 奠幣,《明安》



 四岳之裔,涼彼武王。發揚蹈厲,周室用昌。



 追封廟食,簡冊增芳。升幣以奠,磬管鏘鏘。



 酌獻,《成安》



 獵渭之陽,理冥嘉應。非龍非虎,聿求元聖。



 平易近民,五月報政。祀典之宗,於斯為盛。



 飲福,《綏安》



 神機經武,隆周之寓。表海分封,邁燕超魯。



 耽耽廟貌,俎豆有序。薦福邦家,維師尚父。



 送神,《凝安》



 聖朝稽古,崇茲武經。禮交樂舉,於神之庭。



 嘉慄旨酒,既饗芳馨。永嚴列象舄,劍簪纓。



 熙寧祀武成王一首



 初獻升降,《同安》



 武德洸洸,日靖四方。百王所祀,休有
 烈光。



 命官攝事,佩玉鏘鏘。思皇多祜,以惠無疆。



 大觀祀武成王一首



 酌獻,《成安》



 涼彼周王,君臣相遇。終謀其成,諸侯來許。



 洋洋神靈,尊載酒醑。新聲為侑,笙簫備舉。



 紹興釋奠武成王七首



 迎神,《凝安》姑洗為宮



 於赫烈武,光昭古今。載嚴祀事,敕備惟欽。



 既潔其牲,既諧其音。神之格思,來顧來歆。



 初獻升殿,《同安》



 肅肅廟中,有嚴階戚。匪棘匪徐,進退
 可則。



 冕服是儀,環佩有節。神之鑒觀,率履不越。



 奠幣,《明安》



 祀率舊典,禮崇駿功。齊明衷正,肸蠁豐融。



 量幣肅備,周旋鞠躬。神其昭受,幽贊無窮。



 正位酌獻,《成安》



 赫赫尚父,時維鷹揚。神潛韜略,襟抱帝王。



 談笑致主,竹帛流芳。國有嚴祀,載稽典常。



 留侯位酌獻



 眷彼留侯,奇籌贊漢。依乘風雲,勒成功旦。



 克配明禋,儀刑有煥。英氣如生,來格來衎。



 亞、終獻,《正安》



 道助文德,言為世師。功名不泯,祀事無
 遺。



 旨酒惟馨,具醉在茲。有嘉累獻,神其燕娭。



 送神



 日惟上戊,神顧精純。禮備三獻,樂成七均。



 奄留洋洋,流福無垠。言還恍惚,空想如存。



 紹興祀祚德廟八首



 迎神,《凝安》姑洗為宮



 匿孤立後,惟義惟忠。昔者神考,追錄乃功。



 祀典載加,進爵錫公。神兮降格,尚鑒褒崇。



 初獻升降,《同安》



 廟宇更新,輪奐豐敞。神靈如在,英姿颯爽。



 執事進趨,降升俯仰。威儀翼翼,虔祈歆饗。



 奠幣,《明安》



 牲薦碩大,幣致精純。聿升祀事,茲用兼陳。



 箱篚既實,奠獻惟寅。饗我至意,福祿來成。



 強濟公位酌獻,《成安》



 以身托孤,實惟死友。撫嫗長之,若父若母。



 潛授於韓,克興厥後。崇廟以獻,德侈報厚。



 英略公位酌獻,《成安》



 立孤固難,死亦匪易。義輕一身,開先趙嗣。



 肅穆廟貌,烈有餘氣。式旋嘉薦,昭哉祀事!



 啟祐公位酌獻,《成安》



 於皇時宋,永祚有基。始繄覆護,扶而立之。



 敢忘昭答,牲分酒釃。靈其燕饗,益相本支。



 亞、終獻用《正安》



 眑□靈宇,神安且翔。三哲鼎峙,中薦嘉觴。



 凜若義氣,千載彌光。猗其祐之,錫羨無疆。



 送神,《凝安》



 禮樂云備,畢觴爾神。翊翊音送,轙輿若聞。



 駕言歸兮,靈斿結雲。祚我千億,介福來臻。



 司中司命五首



 迎神,《欣安》



 冠峨峨兮,服章蕤蕤。靈來下兮,進止委蛇。



 我涓我壇,我潔我俎。降輿卻旌,於茲享御。



 升降,《欽安》



 紳緌舒舒,佩環鏗鏗。陟降上下,壇燎光明。



 有盥於罍,有帨於巾。不吳不敖,庶以安神。



 奠幣,《容安》



 我誠既潔,我豆既豐。神來降期,有儼其容。



 薦此嘉幣,肅肅雍雍。何以侑之?於樂鼓鐘。



 酌獻,《雍安》



 酌茲旨酒,既盈且芬。式用來歆,衎衎熏熏。



 何以寧神?薦有嘉籩。何以錫民?曰惟豐年。



 送神,《欣安》



 雲兮飄飄,風兮棱棱。飆馭返空,杲日來升。



 歸旆揚揚,眾樂鏘鏘。我神式歡,惠我嘉祥。



 五龍六首



 迎神,《禧安》



 神之智兮,躍漢潛幽。欲豢擾兮,無董與劉。



 陳金石兮,佐侑牢羞。庶燕享兮,澤應民求。



 升降,《雅安》



 靈之至兮,逸駕騰驤。噓雲吸氣,承祀日光。



 展詩鳴律,肅莊琳瑯。何以膺神?貺惠無疆。



 奠幣,《文安》



 維靈德兮,變化不常。沛天澤兮,周流八荒。



 奠嘉幣兮,肅雍不忘。永祐民兮,錫以豐穰。



 酌獻,《愷安》



 練吉日兮,進神之堂。牲既陳兮,粢盛既香。



 奠桂酒兮,容與嘉觴。靈安留兮,錫我福祥。



 亞、終獻,《嘉安》



 明明天子,禮文咸秩。矧神之功,橫被九域。



 雲施稱民,物產滋殖。嘉承惠和,罔有終極。



 送神,《登安》



 靈之來下,以雨先驅。靈之旋馭,五雲結車。



 操環應夏,發匣瑞虞。真人在御,來獻珍符。



\end{pinyinscope}