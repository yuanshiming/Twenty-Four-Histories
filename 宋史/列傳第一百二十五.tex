\article{列傳第一百二十五}

\begin{pinyinscope}

 劉錡吳玠吳璘子挺



 劉錡,字信叔,德順軍人,滬川軍節度使仲武第九子也。美儀狀,善射,聲如洪鐘。嘗從仲武征討,牙門水斛滿,以箭射之,拔箭水注,隨以一矢窒之,人服其精。宣和間,用
 高俅薦,特授閣門祗候。



 高宗即位,錄仲武後,錡得召見,奇之,特授閣門宣贊舍人,差知岷州,為隴右都護。與夏人戰屢勝,夏人兒啼,輒怖之曰:「劉都護來!」張浚宣撫陜西,一見奇其才,以為涇原經略使兼知渭州。浚合五路師潰於富平,慕洧以慶陽叛,攻環州。浚命錡救之,留別將守渭,自將救環。未幾,金攻渭,錡留李彥琪捍洧,親率精銳還救渭,已無及,進退不可,乃走德順軍。彥琪遁歸渭,降金。錡貶秩知綿州兼沿邊安撫。



 紹興三年復官,為
 宣撫司統制。金人攻拔和尚原,乃分守陜、蜀之地。會使者自蜀歸,以錡名聞。召還,除帶御器械,尋為江東路副總管。六年,權提舉宿衛親軍。帝駐平江,解潛、王彥兩軍交鬥,俱罷,命錡兼將之。錡因請以前護副軍及馬軍,通為前、後、左、右、中軍與游奕,凡六軍,每軍千人,為十二將。前護副軍,即彥八字軍也。於是錡始能成軍,扈從赴金陵。七年,帥合肥;八年,戍京口。九年,擢果州團練使、龍神衛四廂都指揮使,主管侍衛馬軍司。



 十年,金人歸三京,
 充東京副留守,節制軍馬。所部八字軍才三萬七千人,將發,益殿司三千人,皆攜其孥,將駐於汴,家留順昌。錡自臨安溯江絕淮,凡二千二百里。至渦口,方食,暴風拔坐帳,錡曰:「此賊兆也,主暴兵。」即下令兼程而進,未至,五月,抵順昌三百里,金人果敗盟來侵。



 錡與將佐舍舟陸行,先趨城中。庚寅,諜報金人入東京。知府事陳規見錡問計,錡曰:「城中有糧,則能與君共守。」規曰:「有米數萬斛。」錡曰:「可矣。」時所部選鋒、游奕兩軍及老稚輜重,相去尚
 遠,遣騎趣之,四鼓乃至。及旦得報,金騎已入陳。



 錡與規議斂兵入城,為守禦計,人心乃安。召諸將計事,皆曰:「金兵不可敵也,請以精銳為殿,步騎遮老小順流還江南。」錡曰:「吾本赴官留司,今東京雖失,幸全軍至此,有城可守,奈何棄之?吾意已決,敢言去者斬!」惟部將許清號「夜叉」者奮曰:「太尉奉命副守汴京,軍士扶攜老幼而來,今避而走,易耳。然欲棄父母妻子則不忍;欲與偕行,則敵翼而攻,何所逃之?不如相與努力一戰,於死中求生也。」
 議與錡合。錡大喜,鑿舟沉之,示無去意。置家寺中,積薪於門,戒守者曰:「脫有不利,即焚吾家,毋辱敵手也。」分命諸將守諸門,明斥堠,募土人為間探。於是軍士皆奮,男子備戰守,婦人礪刀劍,爭呼躍曰:「平時人欺我八字軍,今日當為國家破賊立功。」



 時守備一無可恃,錡於城上躬自督厲,取偽齊所造癡車,以輪轅埋城上;又撤民戶扉,周匝蔽之;城外有民居數千家,悉焚之。凡六日粗畢,而游騎已涉穎河至城下。壬寅,金人圍順昌,錡豫於城
 下設伏,擒千戶阿黑等二人,詰之,云:「韓將軍營白沙渦,距城三十里。」錡夜遣千餘人擊之,連戰,殺虜頗眾。既而三路都統葛王褎以兵三萬,與龍虎大王合兵薄城。錡令開諸門,金人疑不敢近。



 初,錡傅城築羊馬垣,穴垣為門。至是,與清等蔽垣為陣,金人縱矢,皆自垣端軼著於城,或止中垣上。錡用破敵弓翼以神臂、強弩,自城上或垣門射敵,無不中,敵稍卻。復以步兵邀擊,溺河死者不可勝計,破其鐵騎數千。特授鼎州觀察使、樞密副都承
 旨、沿淮制置使。



 時順昌受圍已四日,金兵益盛,乃移砦於東村,距城二十里。錡遣驍將閻充募壯士五百人,夜斫其營。是夕,天欲雨,電光四起,見辮發者輒殲之。金兵退十五里。錡復募百人以往,或請銜枚,錡笑曰:「無以枚也。」命折竹為嘂,如市井兒以為戲者,人持一以為號,直犯金營。電所燭則皆奮擊,電止則匿不動,敵眾大亂。百人者聞吹聲即聚,金人益不能測,終夜自戰,積尸盈野,退軍老婆灣。



 兀朮在汴聞之,即索靴上馬,過淮寧留一
 宿,治戰具,備糗糧,不七日至順昌。錡聞兀朮至,會諸將於城上問策,或謂今已屢捷,宜乘此勢,具舟全軍而歸。錡曰:「朝廷養兵十五年,正為緩急之用,況已挫賊鋒,軍聲稍振,雖眾寡不侔,然有進無退。且敵營甚邇,而兀朮又來,吾軍一動,彼躡其後,則前功俱廢。使敵侵軼兩淮,震驚江、浙,則平生報國之志,反成誤國之罪。」眾皆感動思奮,曰:「惟太尉命。」



 錡募得曹成等二人,諭之曰:「遣汝作間,事捷重賞,第如我言,敵必不汝殺。今置汝綽路騎中,
 汝遇敵則佯墜馬,為敵所得。敵帥問我何如人,則曰:『太平邊帥子,喜聲伎,朝廷以兩國講好,使守東京圖逸樂耳。』」已而二人果遇敵被執,兀朮問之,對如前。兀朮喜曰:「此城易破耳。」即置鵝車炮具不用。翌日,錡登城,望見二人遠來,縋而上之,乃敵械成等歸,以文書一卷系於械,錡懼惑軍心,立焚之。



 兀朮至城下,責諸將喪師,眾皆曰:「南朝用兵,非昔之比,元帥臨城自見。」錡遣耿訓以書約戰,兀朮怒曰:「劉錡何敢與我戰,以吾力破爾城,直用靴
 尖趯倒耳。」訓曰:「太尉非但請與太子戰,且謂太子必不敢濟河,願獻浮橋五所,濟而大戰。」兀朮曰:「諾。」乃下令明日府治會食。遲明,錡果為五浮橋於穎河上,敵由之以濟。



 錡遣人毒穎上流及草中,戒軍士雖渴死,毋得飲於河者;飲,夷其族。敵用長勝軍嚴陣以待,諸酋各居一部。眾請先擊韓將軍,錡曰:「擊韓雖退,兀朮精兵尚不可當,法當先擊兀朮。兀朮一動,則餘無能為矣。」



 時天大暑,敵遠來疲敝,錡士氣閑暇,敵晝夜不解甲,錡軍皆番休更
 食羊馬垣下。敵人馬饑渴,食水草者輒病,往往困乏。方晨氣清涼,錡按兵不動,逮未、申間,敵力疲氣索,忽遣數百人出西門接戰。俄以數千人出南門,戒令勿喊,但以銳斧犯之。統制官趙撙、韓直身中數矢,戰不肯已,士殊死鬥,入其陣,刀斧亂下,敵大敗。是夕大雨,平地水深尺餘。乙卯,兀朮拔營北去,錡遣兵追之,死者萬數。



 方大戰時,兀朮被白袍,乘甲馬,以牙兵三千督戰,兵皆重鎧甲,號「鐵浮圖」;戴鐵兜牟,周匝綴長簷。三人為伍,貫以韋索,
 每進一步,即用拒馬擁之,人進一步,拒馬亦進,退不可卻。官軍以槍標去其兜牟,大斧斷其臂,碎其首。敵又以鐵騎分左右翼,號「拐子馬」,皆女真為之,號「長勝軍,專以攻堅,戰酣然後用之。自用兵以來,所向無前;至是,亦為錡軍所殺。戰自辰至申,敵敗,遽以拒馬木障之,少休。城上鼓聲不絕,乃出飯羹,坐餉戰士如平時,敵披靡不敢近。食已,撤拒馬木,深入斫敵,又大破之。棄尸斃馬,血肉枕藉,車旗器甲,積如山阜。



 初,有河北軍告官軍曰:「我輩
 元是左護軍,本無鬥志,所可殺者兩翼拐子馬爾。」故錡兵力擊之。兀朮平日恃以為強者,什損七八,至陳州,數諸將之罪,韓常以下皆鞭之,乃自擁眾還汴。捷聞,帝喜甚,授錡武泰軍節度使、侍衛馬軍都虞候、知順昌府、沿淮制置使。



 是役也,錡兵不盈二萬,出戰僅五千人。金兵數十萬營西北,亙十五里,每暮,鼓聲震山谷,然營中喧嘩,終夜有聲。金遣人近城竊聽,城中肅然,無雞犬聲。兀術帳前甲兵環列,持燭照夜,其眾分番假寐馬上。錡以
 逸待勞,以故輒勝。時洪皓在燕密奏:「順昌之捷,金人震恐喪魄,燕之重寶珍器,悉徙而北,意欲捐燕以南棄之。」故議者謂是時諸將協心,分路追討,則兀朮可擒,汴京可復;而王師亟還,自失機會,良可惜也。



 七月,命為淮北宣撫判官,副楊沂中,破敵兵於太康縣。未幾,秦檜請令沂中還師鎮江,錡還太平州,岳飛以兵赴行在,出師之謀寢矣。



 十一年,兀朮復簽兩河兵,謀再舉。帝亦測知敵情,必不一挫遂已,乃詔大合兵於淮西以待之。金人攻
 廬、和二州,錡自太平渡江,抵廬州,與張俊、楊沂中會。而敵已大入,錡據東關之險以遏其沖,引兵出清溪,兩戰皆勝。行至柘皋,與金人夾石梁河而陣。河通巢湖,廣二丈,錡命曳薪壘橋,須臾而成,遣甲士數隊路橋臥槍而坐。會沂中、王德、田師中、張子蓋之軍俱至。



 翌日,兀朮以鐵騎十萬分為兩隅,夾道而陣。德薄其右隅,引弓射一酋斃之,因大呼馳擊,諸軍鼓噪。金人以拐子馬兩翼而進。德率眾鏖戰,沂中以萬兵各持長斧奮擊之,敵大敗;
 錡與德等追之,又敗於東山。敵望見曰:「此順昌旗幟也。」即退走。



 錡駐和州,得旨,乃引兵渡江歸太平州。時並命三帥,不相節制。諸軍進退多出於張俊,而錡以順昌之捷驟貴,諸將多嫉之。俊與沂中為腹心,而與錡有隙,故柘皋之賞,錡軍獨不與。



 居數日,議班師,而濠州告急。俊與沂中、錡趨黃連埠援之,距濠六十里,而南城已陷。沂中欲進戰,錡謂俊曰:「本救濠,今濠已失,不如退師據險,徐為後圖。」諸將曰:「善。」三帥鼎足而營,或言敵兵已去,錡
 又謂曰:「敵得城而遽退,必有謀也,宜嚴備之。」俊不從,命沂中與德將神勇步騎六萬人,直趨濠州,果遇伏敗還。



 遲明,錡軍至藕塘,則沂中軍已入滁州,俊軍已入宣化。錡軍方食,俊至,曰:「敵兵已近,奈何?」錡曰:「楊宣撫兵安在?」俊曰:「已失利還矣。」錡語俊:「無恐,錡請以步卒禦敵,宣撫試觀之。」錡麾下皆曰:「兩大帥軍已渡,我軍何苦獨戰?」錡曰:「順昌孤城,旁無赤子之助,吾提兵不滿二萬,猶足取勝;況今得地利,又有銳兵邪?」遂設三覆以待之。俄而
 俊至,曰:「諜者妄也,乃戚方殿後之軍爾。」錡與俊益不相下。



 一夕,俊軍士縱火劫錡軍,錡擒十六人,梟首槊上,餘皆逸。錡見俊,俊怒謂錡曰:「我為宣撫,爾乃判官,何得斬吾軍?」錡曰:「不知宣撫軍,但斬劫砦賊爾。」俊曰:「有卒歸,言未嘗劫砦。」呼一人出對。錡正色曰:「錡為國家將帥,有罪,宣撫當言於朝,豈得與卒伍對事?」長揖上馬去。已,皆班師,俊、沂中還朝,每言岳飛不赴援,而錡戰不力。秦檜主其說,遂罷宣撫判官,命知荊南府。岳飛奏留錡掌兵,不
 許,詔以武泰之節提舉江州太平觀。



 錡鎮荊南凡六年,軍民安之。魏良臣言錡名將,不當久閑。乃命知潭州,加太尉,復帥荊南府。江陵縣東有黃潭,建炎間,有司決水入江以御盜,由是夏秋漲溢,荊、衡間皆被水患。錡始命塞之,斥膏腴田數千畝,流民自占者幾千戶。詔錡遇大禮許奏文資,仍以其侄汜為江東路兵馬副都監。



 三十一年,金主亮調軍六十萬,自將南來,彌望數十里,不斷如銀壁,中外大震。時宿將無在者,乃以錡為江、淮、浙西
 制置使,節制逐路軍馬。八月,錡引兵屯揚州,建大將旗鼓,軍容甚肅,觀者嘆息。以兵駐清河口,金人以氈裹船載糧而來,錡使善沒者鑿沉其舟。錡自楚州退軍召伯鎮,金人攻真州,錡引兵還揚州,帥劉澤以城不可守,請退軍瓜洲。金萬戶高景山攻揚州,錡遣員琦拒於皂角林,陷圍力戰,林中伏發,大敗之,斬景山,俘數百人。捷奏,賜金五百兩、銀七萬兩以犒師。



 先是,金人議留精兵在淮東以御錡,而以重兵入淮西。大將王權不從錡節制,
 不戰而潰,自清河口退師揚州,以舟渡真、揚之民於江之南,留兵屯瓜洲。錡病,求解兵柄,留其侄汜以千五百人塞瓜洲渡,又令李橫以八千人固守。詔錡專防江,錡遂還鎮江。



 十一月,金人攻瓜洲,汜以克敵弓射卻之。時知樞密院事葉義問督師江、淮,至鎮江,見錡病劇,以李橫權錡軍。義問督鎮江兵渡江,眾皆以為不可,義問強之。汜固請出戰,錡不從,汜拜家廟而行。金人以重兵逼瓜洲,分兵東出江皋,逆趨瓜洲。汜先退,橫以孤軍不能
 當,亦卻,失其都統制印,左軍統制魏友、後軍統制王方死之,橫、汜僅以身免。



 方諸軍渡江而北也,錡使人持黃、白幟登高山望之,戒之曰:「賊至舉白幟;合戰舉二幟,勝則舉黃幟。」是日二幟舉,逾時,錡曰:「黃幟久不舉,吾軍殆矣。」錡憤懣,病益甚。都督府參贊軍事虞允文自採石來,督舟師與金人戰。允文過鎮江,謁錡問疾。錡執允文手曰:「疾何必問。朝廷養兵三十年,一技不施,而大功乃出一儒生,我輩愧死矣!」



 召詣闕,提舉萬壽觀。錡假都亭驛
 居之。金之聘使將至,留守湯思退除館以待,遣黃衣諭錡徙居別試院,錡疑汜累己,常懼有後命。三十二年閏二月,錡發怒,嘔血數升而卒。贈開府儀同三司,賜其家銀三百兩,帛三百匹。後謚武穆。



 錡慷慨深毅,有儒將風。金主亮之南也,下令有敢言錡姓名者,罪不赦。枚舉南朝諸將,問其下孰敢當者,皆隨姓名其答如響,至錡,莫有應者。金主曰:「吾自當之。」然錡卒以病不能成功。世傳錡通陰陽家行師所避就,錡在揚州,命盡焚城外居屋,
 用石灰燼白城壁,書曰:「完顏亮死於此。」金主多忌,見而惡之,遂居龜山,人眾不可容,以致是變雲。



 吳玠,字晉卿,德順軍隴干人。父葬水洛城,因徙焉。少沉毅有志節,知兵善騎射,讀書能通大義。未冠,以良家子隸涇原軍。政和中,夏人犯邊,以功補進義副尉,稍擢隊將。從討方臘,破之;及擊河北群盜,累功權涇原第十將。靖康初,夏人攻懷德軍,玠以百餘騎追擊,斬首百四十級,擢第二副將。



 建炎二年春,金人渡河,出大慶關,略秦
 雍,謀趨涇原。都統制曲端守麻務鎮,命玠為前鋒,進據青溪嶺,逆擊大破之,追奔三十里,金人始有憚意。權涇原路兵馬都監兼知懷德軍。金人攻延安府,經略使王庶召曲端進兵,端駐邠州不赴,且曰:「不如蕩其巢穴,攻其必救。」端遂攻蒲城,命玠攻華州,拔之。



 三年冬,劇賊史斌寇漢中,不克,引兵欲取長安,曲端命玠擊斬之,遷忠州刺史。宣撫處置使張浚巡關陜,參議軍事劉子羽誦玠兄弟才勇,浚與玠語,大悅,即授統制,弟璘掌帳前親
 兵。



 四年春,升涇原路馬步軍副總管。金帥婁宿與撒離喝長驅入關,端遣玠拒於彭原店,而擁兵邠州為援。金兵來攻,玠擊敗之,撒離喝懼而泣,金軍中目為「啼哭郎君」。金人整軍復戰,玠軍敗績。端退屯涇原,劾玠違節度,降武顯大夫,罷總管,復知懷德軍。張浚惜玠才,尋以為秦鳳副總管兼知鳳翔府。時兵火之餘,玠勞來安集,民賴以生。轉忠州防禦使。



 九月,浚合五路兵,欲與金人決戰,玠言宜各守要害,須其弊而乘之。及次富平,都統制
 又會諸將議戰,玠曰:「兵以利動,今地勢不利,未見其可。宜擇高阜據之,使不可勝。」諸將皆曰:「我眾彼寡,又前阻葦澤,敵有騎不得施,何用他徙?」已而敵驟至,輿柴囊土,藉淖平行,進薄玠營。軍遂大潰,五路皆陷,巴蜀大震。



 玠收散卒保散關東和尚原,積粟繕兵,列柵為死守計。或謂玠宜退守漢中,扼蜀口以安人心。玠曰:「我保此,敵決不敢越我而進,堅壁臨之,彼懼吾躡其後,是所以保蜀也。」玠在原上,鳳翔民感其遺惠,相與夜輸芻粟助之。玠
 償以銀帛,民益喜,輸者益多。金人怒,伏兵渭河邀殺之,且令保伍連坐;民冒禁如故,數年然後止。



 紹興元年,金將沒立自鳳翔,別將烏魯折合自階、成出散關,約日會和尚原。烏魯折合先期至,陣北山索戰,玠命諸將堅陣待之,更戰迭休。山谷路狹多石,馬不能行,金人舍馬步戰,大敗,移砦黃牛,會大風雨雹,遂遁去。沒立方攻箭筈關,玠復遣將擊退之,兩軍終不得合。



 始,金人之入也,玠與璘以散卒數千駐原上,朝問隔絕,人無固志。有謀劫
 玠兄弟北去者,玠知之,召諸將歃血盟,勉以忠義。將士皆感泣,願為用。張浚錄其功,承制拜明州觀察使。居母喪,起復,兼陜西諸路都統制。



 金人自起海角,狃常勝,及與玠戰輒北,憤甚,謀必取玠。婁宿死,兀朮會諸道兵十餘萬,造浮梁跨渭,自寶雞結連珠營,壘石為城,夾澗與官軍拒。十月,攻和尚原。玠命諸將選勁弓強弩,分番迭射,號「駐隊矢」,連發不絕,繁如雨注。敵稍卻,則以奇兵旁擊,絕其糧道。度其困且走,設伏於神坌以待。金兵至,伏
 發,眾大亂。縱兵夜擊,大敗之。兀朮中流矢,僅以身免。張浚承制以玠為鎮西軍節度使,璘為涇原路馬步軍副總管。兀朮既敗,遂自河東歸燕山;復以撒離喝為陜西經略使,屯鳳翔,與玠相持。



 二年,命玠兼宣撫處置使司都統制,節制興、文、龍三州。金久窺蜀,以璘駐兵和尚原扼其沖,不得逞,將出奇取之。時玠在河池,金人用叛將李彥琪駐秦州,睨仙人關以綴玠;復令游騎出熙河以綴關師古,撒離喝自商於直搗上津。三年正月,取金州。
 二月,長驅趨洋、漢,興元守臣劉子羽急命田晟守饒風關,以驛書招玠入援。



 玠自河池日夜馳三百里,以黃柑遺敵曰:「大軍遠來,聊用止渴。」撒離喝大驚,以杖擊地曰:「爾來何速耶!」遂大戰饒風嶺。金人被重鎧,登山仰攻。一人先登則二人擁後;先者既死,後者代攻。玠軍弓弩亂發,大石摧壓,如是者六晝夜,死者山積而敵不退。募敢死士,人千銀,得士五千,將夾攻。會玠小校有得罪奔金者,導以祖溪間路,出關背,乘高以闞饒風。諸軍不支,遂
 潰,玠退保西縣。敵入興元,劉子羽退保三泉,築潭毒山以自固,玠走三泉會之。



 未幾,金人北歸,玠急遣兵邀於武休關,掩擊其後軍,墮澗死者以千計,盡棄輜重去。金人始謀,本謂玠在西邊,故道險東來,不虞玠馳至。雖入三郡,而失不償得。進玠檢校少保,充利州路、階成鳳州制置使。



 四年二月,敵復大入,攻仙人關。先是,璘在和尚原,餉饋不繼;玠又謂其地去蜀遠,命璘棄之,經營仙人關右殺金平,創築一壘,移原兵守之。至是,兀朮、撒離喝
 及劉夔率十萬騎入侵,自鐵山鑿崖開道,循嶺東下。玠以萬人當其沖。璘率輕兵由七方關倍道而至,與金兵轉戰七晝夜,始得與玠合。



 敵首攻玠營,玠擊走之。又以雲梯攻壘壁,楊政以撞竿碎其梯,以長矛刺之。璘拔刀畫地,謂諸將曰:「死則死此,退者斬!」金分軍為二,兀朮陣於東,韓常陣於西。璘率銳卒介其間,左縈右繞,隨機而發。戰久,璘軍少憊,急屯第二隘。金生兵踵至,人被重鎧,鐵鉤相連,魚貫而上。璘以駐隊矢迭射,矢下如雨,死者
 層積,敵踐而登。撒離喝駐馬四視曰:「吾得之矣。」翌日,命攻西北樓,姚仲登樓酣戰,樓傾,以帛為繩,挽之復正。金人用火攻樓,以酒缶撲滅之。玠急遣統領田晟以長刀大斧左右擊,明炬四山,震鼓動地。明日,大出兵。統領王喜、王武率銳士,分紫、白旗入金營,金陣亂。奮擊,射韓常,中左目,金人始宵遁。玠遣統制官張彥劫橫山砦,王俊伏河池扼歸路,又敗之。以郭震戰不力,斬之。是役也,金自元帥以下,皆攜孥來。劉夔乃豫之腹心。本謂蜀可圖,
 既不得逞,度玠終不可犯,則還據鳳翔,授甲士田,為久留計,自是不妄動。



 捷聞,授玠川、陜宣撫副使。四月,復鳳、秦、隴三州。七月,錄仙人關功,拜檢校少師、奉寧保定軍節度使,璘自防禦使升定國軍承宣使,楊政以下遷秩有差。六年,兼營田大使,易保平、靜難節。七年,遣裨將馬希仲攻熙州,敗績,又失鞏州,玠斬之。



 玠與敵對壘且十年,常苦遠餉勞民,屢汰冗員,節浮費,益治屯田,歲收至十萬斛。又調戍兵,命梁、洋守將治褒城廢堰,民知灌溉
 可恃,願歸業者數萬家。九年,金人請和。帝以玠功高,授特進、開府儀同三司,遷四川宣撫使,陜西階、成等州皆聽節制。遣內侍奉親札以賜,至,則玠病已甚,扶掖聽命。帝聞而憂之,命守臣就蜀求善醫,且飭國工馳視,未至,玠卒於仙人關,年四十七。贈少師,賜錢三十萬。



 玠善讀史,凡往事可師者,錄置座右,積久,墻牖皆格言也。用兵本孫、吳,務遠略,不求小近利,故能保必勝。御下嚴而有恩,虛心詢受,雖身為大將,卒伍至下者得以情達,故士
 樂為之死。選用將佐,視勞能為高下先後,不以親故、權貴撓之。



 玠死,胡世將問玠所以制勝者,璘曰:「璘從先兄有事西夏,每戰,不過一進卻之頃,勝負輒分。至金人,則更進迭退,忍耐堅久,令酷而下必死,每戰非累日不決,勝不遽追,敗不至亂。蓋自昔用兵所未嘗見,與之角逐滋久,乃得其情。蓋金人弓矢,不若中國之勁利;中國士卒,不及金人之堅耐。吾常以長技洞重甲於數百步外,則其沖突固不能相及。於是選據形便,出銳卒更迭撓
 之,與之為無窮,使不得休暇,以沮其堅忍之勢。至決機於兩陣之間,則璘有不能言者。」



 晚節頗多嗜欲,使人漁色於成都,喜餌丹石,故得咯血疾以死。方富平之敗,秦鳳皆陷,金人一意睨蜀,東南之勢亦棘,微玠身當其沖,無蜀久矣。故西人至今思之。謚武安,作廟於仙人關,號思烈。淳熙中,追封涪王。子五人:拱、扶、摠、擴、摠。拱亦握兵云。



 吳璘,字唐卿,玠弟也。少好騎射,從玠攻戰,積功至閣門
 宣贊舍人。紹興元年,箭筈關之戰,斷沒立與烏魯折合兵,使不得合,金人遁,璘功居多,超遷統制和尚原軍馬,於是玠駐師河池,璘專守原。及兀朮大入,玠兄弟以死守之。敵陣分合三十餘,璘隨機而應,至神坌伏發,金兵大敗,兀朮中流矢遁。張浚承制以璘為涇原路馬步軍副都總管,升康州團練使。



 三年,遷榮州防禦使、知秦州,節制階、文。是歲,玠敗於祖溪嶺,時璘猶在和尚原,玠命璘岔棄原別營仙人關,以防金人深入。四年,兀朮、撒離
 喝果以大兵十萬至關下,璘自武、階路入援。先以書抵玠,謂殺金平地闊遠,前陣散漫,須後陣阻隘,然後可以必勝。玠從之,急修第二隘。璘冒圍轉戰,會於仙人關。敵果極力攻第二隘,諸將有請別擇形勝以守者,璘奮曰:「兵方交而退,是不戰而走也,吾度此敵去不久矣,諸君第忍之。」震鼓易幟,血戰連日。金兵大敗,二酋自是不敢窺蜀者數年。



 露布獻捷,遷定國軍承宣使、熙河蘭廓路經略安撫使、知熙州。六年,新置行營兩護軍,璘為左護
 軍統制。九年,升都統制,尋除秦鳳路經略安撫使、知秦州。玠卒,授璘龍、神衛四廂都指揮使。



 時金人廢劉豫,歸河南、陜西地。樓照使陜,以便宜欲命三帥分陜而守,以郭浩帥鄜延,楊政帥熙河,璘帥秦鳳,欲盡移川口諸軍於陜西。璘曰:「金人反復難信,懼有他變。今我移軍陜右,蜀口空虛,敵若自南山要我陜右軍,直搗蜀口,我不戰自屈矣。當且依山為屯,控其要害,遲其情見力疲,漸圖進據。」照從之,命璘與楊政兩軍屯內地保蜀,郭浩一軍
 屯延安以守陜。



 既而胡世將以四川制置權宣撫司事,至河池,璘見之曰:「金大兵屯河中府,止隔大慶一橋爾,騎兵疾馳,不五日至川口。吾軍遠在陜西,緩急不可追集,關隘不葺,糧運斷絕,此存亡之秋也。璘家族固不足恤,如國事何!」時朝廷恃和忘戰,欲廢仙人關。於是世將抗奏謂:「當外固歡和,內修守御。今日分兵,當使陜、蜀相接,近兵宮賀仔諜知撒離喝密謀曰:『要入蜀不難,棄陜西不顧,三五歲南兵必來主之,道路吾已熟知,一發取
 蜀必矣。』敵情如是,萬一果然,則我當為伐謀之備,仙人關未宜遽廢,魚關倉亦宜積糧。」於是璘僅以牙校三隊赴秦州,留大軍守階、成山砦,戒諸將毋得撤備。世將尋真除宣撫,置司河池。



 十年,金人敗盟,詔璘節制陜西諸路軍馬。撒離喝渡河入長安,趨鳳翔,陜右諸軍隔在敵後,遠近震恐。時楊政在鞏,郭浩在鄜延,惟璘隨世將在河池。世將急召諸將議,惟涇原帥田晟與楊政同至,參謀官孫渥謂河池不可守,欲退保仙人原,璘厲聲折之
 曰:「懦語沮軍,可斬也!璘請以百口保破敵。」世將壯之,指所居帳曰:「世將誓死於此!」乃遣渥之涇原,命田晟以三千人迎敵。璘又遣姚仲拒於石壁砦,敗之。詔同節制陜西諸路軍馬。



 璘以書遺金將約戰,金鶻眼郎君以三千騎沖璘軍,璘使李師顏以驍騎擊走之。鶻眼入扶風,復攻拔之,獲三將及女真百十有七人。撒離喝怒甚,自戰百通坊,列陣二十里。璘遣姚仲力戰破之,授鎮西軍節度使,升侍衛步軍都虞候。十一年,與金統軍胡盞戰剡
 家灣,敗之,復秦州及陜右諸郡。



 初,胡盞與習不祝合軍五萬屯劉家圈,璘請討之。世將問策安出,璘曰:「有新立疊陣法:每戰,以長槍居前,坐不得起;次最強弓,次強弩,跪膝以俟;次神臂弓。約賊相搏至百步內,則神臂先發;七十步,強弓並發;次陣如之。凡陣,以拒馬為限,鐵鉤相連,俟其傷則更代之。遇更代則以鼓為節。騎,兩翼以蔽於前,陣成而騎退,謂之『疊陣』。」諸將始猶竊議曰:「吾軍其殲於此乎?」璘曰:「此古束伍令也,軍法有之,諸君不識爾。
 得車戰餘意,無出於此,戰士心定則能持滿,敵雖銳,不能當也。及與二酋遇,遂用之。



 二酋老於兵,據險自固,前臨峻嶺,後控臘家城,謂我必不敢輕犯。先一日,璘會諸將問所以攻,姚仲曰:「戰於山上則勝,山下則敗。」璘以為然,乃告敵請戰,敵笑之。璘夜半遣仲及王彥銜枚截坡,約二將上嶺而後發火。二將至嶺,寂無人聲,軍已畢列,萬炬齊發。敵駭愕曰:「吾事敗矣。」習不祝善謀,胡盞善戰,二酋異議。璘先以兵挑之,胡盞果出鏖戰。璘以疊陣法
 更休迭戰,輕裘駐馬亟麾之,士殊死鬥,金人大敗。降者萬人,胡盞走保臘家城,璘圍而攻之。城垂破,朝廷以驛書詔璘班師,世將浩嘆而已。明年,竟割和尚原以與敵。撤戍割地,皆秦檜主之也。



 十二年,入覲,拜檢校少師、階成岷鳳四州經略使,賜漢中田五十頃。十四年,朝議析利州路為東西路,以璘為西路安撫使,治興州,階、成、西和、鳳、文、龍、興七州隸焉。時和議方堅,而璘治軍經武,常如敵至。十七年,徙奉國軍節度使,改行營右護軍為御前
 諸軍都統制,安撫使如故。二十一年,以守邊安靜,拜少保。二十六年,領興州駐扎御前諸軍都統制職事,改判興州。渡江以來未有使相為都統制者,時璘已為開府儀同三司,故改命之。



 三十一年,金主亮叛盟,拜四川宣撫使。秋,亮渡淮,遣合喜為西元帥,以兵扼大散關,游騎攻黃牛堡。璘即肩輿上殺金平,駐軍青野原,益調內郡兵分道而進,授以方略。制置使王剛中來會璘計事,璘尋移檄契丹、西夏及山東、河北,聲金人罪以致討。未幾,
 兼陜西、河東招討使。璘以病還興州,總領王之望馳書告執政,謂璘多病,猝有緩急,蜀勢必危。請移璘侄京襄帥拱歸蜀,以助西師。凡五書未報。璘已力疾,復上仙人關。



 三十二年,璘遣姚仲取鞏,王彥屯商、虢、陜、華,惠逢取熙河。或久攻不下,或既得復失,竟無成功。金人據大散關六十餘日,相持不能破。仲舍鞏攻德順已逾四旬,璘以知夔州李師顏代之,遣子挺節制軍馬。挺與敵戰於瓦亭,敗之。璘自將至城下,守陴者聞呼「相公來」,觀望咨
 嗟,矢不忍發。璘按行諸屯,預治黃河戰地,斬不用命者,先以數百騎嘗敵。敵一鳴鼓,銳士空壁躍出突璘軍。璘軍得先治地,無不一當十。至暮,璘忽傳呼「某將戰不力」,人益奮搏,敵大敗,遁入壁。黎明,師再出,敵堅壁不動。會天大風雷,金人拔營去,凡八日而克。璘入城,市不改肆,父老擁馬迎拜不絕。璘尋還河池。



 四月,原州受圍,璘命姚仲以德順之兵往援,璘自趨鳳翔視師。諸將雖力戰,敵攻益急,增兵至七萬。五月,仲與敵戰於原州之北嶺,
 仲敗績。初,仲自德順至原,由九龍泉上北嶺,令諸軍持滿引行。以盧士敏兵為前陣,所統軍六千為四陣,姚仲兵為後拒。隨地便利以列,與敵鏖戰,開合數十。會輜重隊隨陣亂行,敵兵沖之,軍遂大潰,失將三十餘人。始,璘出師,王之望嘗言:「此行士卒銳氣,不及前時,仲年來數奇,不可委以要地。」及仲至原,璘亦貽仲書,謂原圍未即解,且還德順。書未達而仲敗,璘亦無功還。尋奪仲兵,欲斬之,或勸而止,械系河池獄。



 孝宗受禪,賜璘札,命兼陜
 西、河東路宣撫招討使。璘策金人必再爭德順,亟馳赴城下,而完顏悉烈等兵十餘萬果來攻。萬戶豁豁復領精兵自鳳翔繼至。璘築堡東山以守,敵極力爭之,殺傷太半,終不能克。時議者以為兵宿於外,去川口遠,恐敵襲之,欲棄三路。遂詔璘退師。敵乘其後,璘將士死亡者甚眾,三路復為敵有。拜少傅。隆興二年冬,金人侵岷州,璘提兵至祁山,金人聞之,退師,遣使來告曰:「兩國已講和矣。」會詔至,俱解去。



 沉介為四川安撫、制置使,與璘議
 不協,兵部侍郎胡銓上書,語頗及璘。璘抗章請朝,上親札報可。未半道,請罷宣撫使及致仕,皆不允。乾道元年詣闕,遣中使勞問,召對便殿,許朝德壽宮。高宗見璘,嘆曰:「朕與卿,老君臣也,可數入見。」璘頓首謝。兩宮存勞之使相踵,又命皇子入謁。拜太傅,封新安郡王。越數日,詔仍領宣撫使,改判興元府。及還鎮,兩宮宴餞甚寵。璘入辭德壽宮,泣下。高宗亦為之悵然,解所佩刀賜之,曰:「異時思朕,視此可矣。」



 璘至漢中,修復褒城古堰,溉田數千
 頃,民甚便之。三年,卒,年六十六。贈太師,追封信王。上震悼,輟視朝兩日,賻贈加等。高宗復賜銀千兩。初,璘病篤,呼幕客草遺表,命直書其事曰:「願陛下毋棄四川,毋輕出兵。」不及家事,人稱其忠。



 璘剛勇,喜大節,略苛細,讀史曉大義。代兄為將,守蜀餘二十年,隱然為方面之重,威名亞於玠。高宗嘗問勝敵之術,璘曰:「弱者出戰,強者繼之。」高宗曰:「此孫臏三駟之法,一敗而二勝也。」



 嘗著《兵法》二篇,大略謂:「金人有四長,我有四短,當反我之短,制彼
 之長。四長曰騎兵,曰堅忍,曰重甲,曰弓矢。吾集蕃漢所長,兼收而並用之,以分隊制其騎兵;以番休迭戰制其堅忍;制其重甲,則勁弓強弩;制其弓矢,則以遠克近,以強制弱。布陣之法,則以步軍為陣心、左右翼,以馬軍為左右肋,拒馬布兩肋之間;至帖撥增損之不同,則系乎臨機。」知兵者取焉。



 王剛中嘗談劉錡之美,璘曰:「信叔有雅量、無英概,天下雷同譽之,恐不能當逆亮,璘竊憂之。」剛中不以為然,錡果無功,以憂憤卒。璘選諸將率以功。
 有薦才者,璘曰:「兵官非嘗試,難知其才。以小善進之,則僥幸者獲志,而邊人宿將之心怠矣。」子挺。



 挺字仲烈,以門功補官。從璘為中郎將,部西兵詣行在。高宗問西邊形勢、兵力與戰守之宜,挺占對稱旨,超授右武郎、浙西都監兼御前祗候,賜金帶。尋差利路鈐轄,改利州東路前軍同統制,繼改西路。



 紹興三十一年,金人渝盟,璘以宣撫使總三路兵御之,挺願自力軍前,璘以為中軍統制。王師既復秦州,金將合喜孛堇介叛將
 張中彥以兵來爭,挺破其治平砦。已而南市城賊亦掎角為援,轉戰竟日。挺令前軍統制梅彥麾眾直據城門,眾弗喻,彥亦懼力不敵。挺督之,彥出兵殊死戰,挺率背嵬騎盡易黃旗繞出敵後,憑高突之。敵嘩曰:「黃旗兒至矣!」遂驚敗。挺不自為功,狀彥第一,士頗多之。璘亦引嫌,並匿其功。擢榮州刺史,尋拜熙河經略、安撫使。



 明年,挺被檄與都統制姚仲率東西路兵攻德順。金左都監空平涼之眾以援合喜,又遣精兵數萬自鳳翔來會。仲駐
 軍六盤,挺獨趨瓦亭,身冒矢石,眾從之。金人舍騎操短兵奮斗,挺遣別將盡奪其馬,金眾遂潰。挺勒兵追之,禽千戶耶律九斤、孛堇等百三十七人。



 金人懲前衄,悉兵趨德順。璘自秦州來督師,先壁於險,且治夾河戰地。金人果大至,挺誘致之,至所治戰地,盛兵蹙之,敵不能支,一夕遁去。鞏州久不下,挺以選鋒至城下,諸將咸曰:「西北坡陀地易攻,若分兵各當一面,宜得利。」挺曰:「西北雖卑而土堅,東南並河多沙礫善圮。且兵分則少,以少當
 堅城,可得而下乎?」乃命悉眾擊東南陬。不二日,樓櫓俱盡。夜半,其將雷千戶約降,黎明,城破。以功授團練使,又以瓦亭功授郢州防禦使。



 孝宗即位,加璘兼陜西、河東路招討宣撫使。璘慮敵必再爭德順,至自河池,金人果合兵十餘萬列柵以拒。有大酋引騎數千睨東山,璘命挺領騎迎擊,卻之。遂據東山,築堡以守。敵不能爭,乃益修攻具,為大車匿戰士其中,將填隍而進。挺命掄大木植中道,車至不得前。拜武昌軍承宣使,尋加龍神衛四
 廂都指揮使、熙河路經略安撫使中軍統制,時年二十五。會朝廷主議和,詔西師解嚴,父子遂旋軍。



 乾道元年,升本軍都統制。三年,以父命入奏,拜侍衛親步軍指揮使,節制興州軍馬。璘卒,起復金州都統、金房開達安撫使,改利州東路總管。挺力求終喪,服除,召為左衛上將軍。朝廷方議置神武中軍五千人以屬御前,命挺為都統制。挺力陳不當輕變祖宗法,事遂寢。拜主管侍衛步軍司公事。



 挺每燕見從容,嘗論兩淮形勢曠漫,備多力
 分,宜擇勝地扼以重兵,敵仰攻則不克,越西南又不敢,我以全力乘其弊,蔑不濟者。帝頗嘉納。淳熙元年,改興州都統,拜定江軍節度使。初,軍中自置互市於宕昌,以來羌馬,西路騎兵遂雄天下。自張松典榷牧,奏絕軍中互市,自以馬給之,所得多下駟。挺至,首陳利害以聞,乞歲市五百匹,詔許七百匹。



 始,武興所部就餉諸郡,漫不相屬。挺奏以十軍為名,自北邊至武興列五軍,曰踏白、摧鋒、選鋒、策選鋒、游奕;武興以西至綿為左、右、後三軍;
 而駐武興者前軍、中軍。營部於是始井井然。四年,入覲,除知興州、利州西路安撫使。密修皂郊堡,增二堡,繕戎器,儲於兩庫,敵終不覺。



 十年冬,特加檢校少保。成州、西和歲大侵,挺力為振恤,諭總賦者分軍儲以佐之,全活殆數千萬。蜀自諸軍宿師,凡廩賜,官率糴三之一,視價高下給之,名曰「折估」,隨所屯地相為乘除。歲久屯他徙,廩賜不易舊,至有同部伍而廩相倍蓰者,挺裒為中制上之。



 光宗即位,御筆獎勞。而西和、階、成、鳳、文、龍六州器
 械弗繕,挺節冗費,屯工徒,悉創為之。御軍雖嚴,而能時其緩急,士以不困。郡東北有二穀水,挺作二堤以捍之。紹熙二年,水暴發入城。挺既振被水者,復增築長堤,民賴以安。詔問備邊急務,即建增儲之策,由是糧糗不乏。四年春,以疾乞致仕,詔加太尉。卒,年五十六。贈少師、開府儀同三司。



 挺少起勛閥,弗居其貴,禮賢下士,雖遇小官賤吏,不敢怠忽。拊循將士,人人有恩。璘故部曲拜於庭下,輒降答之,即失律,誅治無少貸。璘嘗對孝宗言,諸
 子中惟挺可任。孝宗亦曰:「挺是朕千百人中選者。」歲時問勞不絕,被遇尤深厚。光宗賜內府珍奇,以示殊禮。子五人,曦,其次也。曦仕至太尉、昭信軍節度使,以叛誅,見別傳。



 論曰:劉錡神機武略,出奇制勝,順昌之捷,威震敵國,雖韓信泜上之軍,無以過焉。或謂其英概不足,雅量有餘,豈其然乎?吳玠與弟璘智勇忠實,戮力協心,據險抗敵,卒保全蜀,以功名終,盛哉!挺累從征討,功效甚著,有父
 風矣。然玠晚頗荒淫,璘多喪敗,豈狃於常勝,驕心侈歟!抑三世為將,釀成逆曦之變,覆其宗祀,蓋有由焉。



\end{pinyinscope}