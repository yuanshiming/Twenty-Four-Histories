\article{列傳第二百一十一忠義七}

\begin{pinyinscope}

 ○高敏張吉景思忠弟思立王奇蔣興祖郭滸朱友恭附吳革李翼阮駿趙士嶐士醫士真士遒士跂陳自仁叔皎叔憑訓之聿之
 陳淬黃友郝仲連劉惟輔高子孺韓青附牛皓魏彥明劉士英翟興弟進朱蹕朱良方允武龔楫李亙凌唐佐楊粹中強霓康傑李伸郭僎郭贊王迸吳從龍司馬夢求林空齋黃介孫益王仙吳楚材李成大陶居仁



 高敏,登州人。為涇原指使,數與西夏戰,遭重傷。範仲淹、韓琦皆薦之,為閣門祗候,歷利州路、邠寧環慶都監,主蕃部事。



 羌圍大順城,偏將趙懷德力戰,其下以銀買級,主帥李復圭以所部不整欲治之。敏言懷德善用人,戰必勝,當略其小過,且蕃官難強以漢法,復圭乃止。羌人聲言將出鄜延,敏屢白復圭曰:「兵家之事,聲東擊西,環慶嘗破白豹、金湯,結釁已深,不可不備。」已而果以兵三十萬來寇。



 總管楊遂駐兵大義,以敏為先鋒將。夏人攻
 奪大順水砦,敏出通路,自寅及午,且戰且前,多所斬獲。次榆林,援兵不至,中流矢死,年五十七。官止東頭供奉官。詔贈嘉州刺史,錄其三子為侍禁、殿直。



 張吉者,慶州卒也,為淮安鎮守烽。夏人寇東谷,掠得之,脅以兵,使呼城中曰:「淮安諸砦已破,宜速降。」吉反其辭曰:「努力!諸砦無虞,賊糧盡且去矣,毋庸降。」賊怒,害之。詔贈內殿崇班,又錄其子。



 景思忠,字進之,普州安岳人。以父西上閣門使泰蔭,累
 官西京左藏庫使,為遂州駐泊都監。夷人寇淯井,鈐轄張承祐出兵救之,思忠部卒五百為前鋒。夷乘險薄官軍,官軍戰不利,死者十之六。左右勸思忠引避,不聽,奮劍疾戰而死。走馬使張宗望為言,詔察訪熊本考實,得其事,神宗憫之,官思忠及同死者之子七人,餘皆賜其家錢帛。



 弟思立,以蔭主渭州治平砦。囉兀用兵,韓絳使攝保安軍。夏人寇順寧,思立擅領兵赴援,諸將敗,一軍獨全。以
 功知德順軍,策應王韶取熙州,過洮,築當川堡,克羌香子、珂諾城,遂定河州。嘗與羌力戰,斬不用命者數人,軍聲大振。韶言其臨事忠勇,進如京副使、通事舍人,再擢東上閣門使、河州刺史,賜繡旗、朱甲。又遷四方館使、河州團練使,知其州。神宗知思立母老而未有官舍,命其弟思誼為秦州判官以便養。



 青宜結鬼章舉兵襲殺伐木卒,害小校七人,以書抵思立,詞不遜。思立不能忍,帥兵六千攻之於踏白城。鈐轄韓存寶、蕃將瞎藥交止之,
 不聽。自將中軍,使存寶及魏奇為先鋒,王存將左,賈翊將右。鬼章眾二萬,分三砦以抗官軍。戰數十合,羌從山下圍中軍。他將王寧、李元凱沒於陣,思立、存寶潰圍出,諸將多傷,議曰:「日暮兵疲,宜移屯東岡以自固。」思立以魏奇創重,獨徙其軍,方遣之而殿後兵亂,前人望見,亦皆潰。思立且鬥且退,曰:「我適以百騎走羌數千人,無助我者,今敗矣,當自剄以謝朝廷。」眾止之。少頃再戰,遂死。時已除忠州防禦使,會其死,不及拜。帝以其輕敵致敗,
 不復贈官。



 王奇,汾州人,武舉中第。章惇經營湖北溪洞,以為將領,降其酋舒光貴,縛元猛,平懿、洽等州。累遷如京副使,為湖南都監,徙廣西。宜州蠻寇邊,奇領兵至天河縣,期旦日會戰。裨將費萬夜以眾竊出河泥隘,戰沒。經略使移書迫奇,奇不能堪。後數日,蠻萬人驟集,奇輕出,遂敗。麾下猶數百人,勸策馬逃去,奇罵曰:「大丈夫當盡節以報國,何走為!」戰而死。詔贈皇城使、忠州防禦使,官其家六
 人,仍賜金帛。



 蔣興祖,常州宜興人,之奇之孫也。以蔭累調饒州司錄。睦州盜起,旁郡皆震,興祖白州將糾吏卒,絹戰具,盜不敢謀。以功遷官,知開封府陽武縣。陽武,古博浪沙地,土脈脆惡,大河薄其南。嘗積雨泛溢,埽且潰,興祖躬救護,露宿其上,彌四旬,堤以不壞。治為畿邑最,使者交薦之。靖康初,金兵犯京師,道過縣,或勸使走避,興祖曰:「吾世受國恩,當死於是。」與妻子留不去。監兵與賊通,斬以徇。金
 數百騎來攻,不勝,去。明日師益至,力不敵,死焉,年四十二。妻及長子相繼以悸死。詔贈朝散大夫。



 郭滸,德順中安堡人。從軍,積官至武經郎,為涇原第八副將。金人犯陜西,渭帥以下叛降,獨滸義不許,稱病去。帥惡忌之,傅致以罪,下之獄,脅使俱降。滸奮而呼曰:「大丈夫今得死所矣!終不能受污。叛逆大惡,天地所不容,吾雖死,誓不爾貸,當訴於地下耳。」眾醜其語,即殺之。建炎三年,贈武翼大夫、忠州刺史。



 同死者朱友恭,西安人。
 以忠翊郎為涇原第一副將。部兵捍金人於華亭,數有功。會金兵大集,友恭赴敵力戰,為所得。渭帥既降,誘以甘言,許優進官秩,不肯從,更詆辱之。帥不勝忿,斷其脛以徇,經日乃斬之。後贈敦武郎。



 吳革,字義夫,華州華陽人,國初勛臣廷祚七世孫也。少好學,喜談兵。再試禮部不中,乃從涇原軍,以秉義郎干辦經略司公事。



 金人南牧,帥兵解遼州之圍。使粘罕軍,見之庭,揖不拜,責其貪利敗約,詞直氣勁。粘罕少屈,為
 追回威勝諸屯兵,授書使歸。欽宗問割地與不割地利害,對曰:「金人有吞噬之意,願悉起關中士馬赴都為備。」詔以為武功大夫、閣門宣贊舍人,持節諭陜西。行至朱遷,聞金人犯京師,復還。與張叔夜同入城,請於帝,乞幸秦州;又乞出城劫之,使不敢近;又乞諸門同出兵牽制、沖突、尾襲、應援,可一戰而勝。時眾言已入,皆不果。後金兵攻安上門,填道度壕,革言之守將,使洩蔡河水以灌之,不聽。及填道將合,欲用前議,則水已涸矣。



 車駕幸金
 營,革以為墮其詐,往請叔夜,欲身見其大酋計事。叔夜問其故,曰:「茲行有三說:一則天子還內,二則金騎歸國,三則革死。」叔夜為言之,不報。上皇、妃、後、太子出郊,革白孫傅乞留之,不得。乃為傅謀,於啟聖僧院置振濟局,募士民就食。一日之間至者萬計,陰以軍法部勒,將攻金營。久之,遷於同文館,所合已至數萬,多兩河驍悍之士。



 既而有立張邦昌之議,革謀先誅範瓊輩,以三月八日起兵。謀既定,前期二日,有班直甲士數百人排闥入言:「
 邦昌以七日受冊,請亟起事。」革乃被甲上馬,至咸豐門,四面皆瓊黨,紿革入帳,即執之,脅以從逆。革罵之極口,引頸受刃,顏色不變。其麾下百人皆同死。



 李翼,麟州新秦人。宣和末,為代州西路都巡檢使,屯崞縣。金人取代,執守將嗣本,遣來諭降,翼射卻之,帥士卒堅守。義勝軍統領崔忠殺都監張洪輔,夜引金兵入城,翼挺身搏戰達旦,力不敵被執。酋粘罕欲臣之,怒罵不屈,與縣令李聳、丞王唐臣、尉劉子英、監酒閻誠、將官折
 可與同死之。



 阮駿者,興化軍人。紹聖元年進士,為河南府少尹。金人犯京師,率所隸兵擁護神御殿,抱神御,罵聲不絕口,卒被害。特贈朝議大夫。



 趙士嶐字景瞻,太宗之後。生五歲,補右班殿直。既長,游庠序,月試數居前列。一日,投筆嘆曰:「昔賢有不願為章句儒,出玉門關、佩侯印者,彼何人哉!」遂不復事科舉。去為郡縣吏,累遷至淮南西路兵馬鈐轄,駐壽春。



 劇賊丁一箭眾號十萬,來攻城。郡守不知兵,凡備御之策悉委
 士嶐。賊三旬不退,士嶐募軍中敢死士與之謀。有張宣者應募,獨持槊縋城下,擊殺數十人,賊眾披靡。乃選壯士數百人夜開城門,出其不意擊走之,追奔數十里。以功遷三官,秩滿,授江東路鈐轄。



 李成叛,據江、淮六七郡,連兵數萬,遣其黨馬進圍九江,守臣姚舜明與士嶐及副鈐轄劉紹先御之。進攻城益急,士嶐竭力捍守。江東帥呂頤浩屯鄱陽,既復南康,與建武節度使楊惟忠兵會,遣統制巨師古援江州,未至,遇伏敗。紹興元年正月,
 詔張俊為江、淮招討使,入辭,頗言成兵眾。高宗責以立功,俊悚懼受命。未至,城已陷。



 時守城罷卒僅數千,捍賊百餘日,城中食盡。舜明、紹先議縱火,因棄城去,士嶐毅然獨糾合部曲餘民守城。城破,眾號呼曰:「無殺我趙鈐轄。」賊入城大掠。成素服士嶐之義,欲以為偽安撫使,士嶐怒罵曰:「賊欲屈我耶!」陰裂帛以書使示諸子曰:「賊不殺我,義不茍活,汝輩得出,為我雪恥。」遂仰藥而卒,年五十二。賊怒,並害其家數十口。事聞,上嘉悼,贈武功大夫,
 官其孫二人。


士嶐六子,皆有文行:不惉、不
 \gezhu{
  乂心}
 、不愆、不恧、不UC、不隱。是役也,不
 \gezhu{
  乂心}
 、不UC、不隱死焉。



 又宗子有士醫、士真、士遒,皆以死事聞。



 士醫,任秀州兵馬都監。建炎四年,兀術入州,士醫乘城拒戰,城陷死之。後贈武翼大夫,官其二子。



 士真,權知信陽軍。寇劉滿至,士真拒之。兵潰,滿執之去荊門,遇害。後贈右朝奉大夫。官其一子。



 士遒,以武翼大夫守官江州。紹興五年,馬進寇江州,士
 遒遇害。贈武德大夫,官其家二人。



 士跂,濮王曾孫也。靖康末,為右監門衛大將軍、吉州團練使。金人驅宗室北行,士跂得間道遁去。居邢州,結土豪將舉事。有告者,金人執而殺之。事聞,贈保寧軍節度使,謚忠果。



 叔皎,秦悼王四世孫。元豐中,為右班殿直,累遷至德州兵馬都監。自靖康以來,劉順、呂拱、劉亨相繼謀叛,叔皎皆設方略捕擒之。建炎二年,金人圍城,郡檄叔皎率兵
 御之,前後六戰。圍急,有江喆者,與郡守宗諒謀以城降,叔皎斬喆以徇。金人登城,叔皎猶力戰,勢窮被執,怒罵不屈,遂遇害。



 叔憑,建炎間,任陜州都監,累官武翼大夫,就遷通守。金人圍陜州既久,援兵不至,城危。時叔憑子官盧氏,遺以蠟丸書曰:「人臣當死國難,況吾以近屬,其可辱命耶?死固其所也。」遂死之。時通判王滸,職官劉效、陳思道、馮經、李嶽、杜開,縣令張玘,將佐盧亨等五十一人俱死,無降
 者。



 訓之字誨道,秦悼王五世孫。父叔侯,官至惠州防禦使。訓之登政和二年進士,調東平儀曹,知平江府吳縣。朱勔怙勢役州縣,訓之不為屈。勔嘗執數輩詣縣請治,訓之悉縱之。懺勔,遂移疾去。



 宣和末,盜起河北,訓之屢與人言:「契丹舊盟未可渝,金人新好未可恃。」未幾,金人犯京師,訓之居揚州,率大姓募士勤王,聞都城失守,乃止。



 建炎三年,知吉州永豐縣。孟太后避地虔州,護衛統制
 杜彥與其麾下叛,後軍楊世雄應之,將犯永豐。訓之與尉陳自仁簡兵分為二,一取間道繞賊後,一據地利匿其精兵以誘賊。賊至伏發,殲其眾。會賊別校繼至,官兵未成列,訓之率數十輩拒戰,厲聲罵賊,與自仁俱被害。事聞,詔贈訓之朝散郎、直秘閣,謚忠果,自仁通直郎,官其子,邑人為立祠。



 太后之發吉州也,至太和,眾皆潰。從事郎、三省樞密院乾辦官劉德老為金人追騎所殺。官其家一人。



 是年,金人過江,陳淬戰死,岳飛等兵皆引去。
 上元丞趙壘之帥鄉兵迎敵,死之。贈奉議郎,官其家一人。



 聿之,安定郡王叔東子也。建炎中,為成忠郎。金人圍潭州,帥臣向子諲率眾守城,聿之隸東壁。子諲循城,顧聿之曰:「君宗室,不可效他人茍簡。」聿之感慨流涕。金兵登城縱火,子諲率官吏突門遁去,城遂陷,聿之巷戰,大罵而死。將官武經郎劉玠亦死之。事聞,贈聿之左監門衛大將軍,玠武經大夫,皆官其家。其後朱熹為請立廟,賜
 號忠節。



 陳淬,字君銳,興化軍莆田人。紹聖初,下第,挾策西游。時呂惠卿帥鄜延,淬戎服往見,惠卿問相見何事,淬曰:「大丈夫求見大丈夫,又何事?」惠卿器之,補三班奉職。與西人接戰於烏原,手殺十餘人,擒其砦主。奏為左班殿直、鄜延路兵馬都監,累遷武經郎。丁外艱。



 宣和四年,召赴闕,授真定路分都監兼知北砦、河北第一將,尋拜忠州團練使、真定府路馬步副總管。七年,金人入真定,淬以
 孤軍御之,妻孥八人皆遇害。



 建炎元年,闢諸軍統制,宗澤命擊金人於南華,敗之。兼大名府路都總管兵馬鈐轄,擢知恩州。王善者,金之種落也。擁眾十萬,長驅兩河,遂襲恩。淬與長子仲剛拒戰,賊飛刃及淬,仲剛以身蔽刃,死之。明年,善復圍陳州,淬大敗善兵,拜宿州安撫使。李成叛,詔以淬為御營使、六軍都統、淮南招撫使討之,三戰三捷。未幾,金人犯採石,又檄淬回援建康。淬將中軍,戚方將前,王𤫉將後。淬曰:「彼眾雖多,然止有二十艘,
 一艘不越五十人,每至不過千人。吾伏兵葭蘆翳薈間,俟其旋濟旋獲,前後不相知,訖濟,當盡獲矣。」杜充不從,金兵遂犯板橋,諸軍皆潰,淬獨與戰,勢窮力盡,據胡床大罵,刃交於胸而色不動,與其從子仲敏俱死。詔贈拱衛大夫、明州觀察使,官其一子一婿。



 黃友,字龍友,溫州平陽人。少不羈,十五入太學,語同輩曰:「大丈夫不能為國立功,亦造化中贅物耳。」因投筆西游。邊帥劉法一見奇之,延致門下。會西鄙軍哄,都護高
 永年戰沒,友作七詩哀其忠。其後幕府奏功,沒永年之實,恤典不及。其子以友詩進,徽宗覽之惻然,遂加贈謚。友亦免省試,登進士第,調永嘉、瑞安二縣主簿,攝華陰令,有政聲。



 方臘竊發,友同諸將收復,所至披靡。婺寇復作,守留友攝兵曹,為殄滅計。友請往諭之,既次浦江,賊望風解去。復單騎次武義,賊眾持釘一榼置其前,友正色叱之曰:「汝等何速死耶?」賊首李德壯之,亟麾退,一境貼然,婺人圖像祀之。



 通判檀州。會金人敗盟,郭藥師以
 常勝軍叛,燕土響應,友獨領數千人與之戰,躬冒矢石,破裂唇齒。欽宗即位,制置使詹度奏友久服武事,籌略過人。丞相何𬃄從而薦之,召對,問友唇齒破裂狀,為之稱嘆,齎予甚渥。



 進直徽猷閣、制置司參謀官,同種師中解太原圍。友遣兵三千奪榆次,得糧萬餘斛。明日,大軍進榆次十里而止,友亟白師中:「地非利,將三面受敵。」論不合,友仰天嘆曰:「事去矣!」迨曉,兵果四合,矢石如雨,敵益以鐵騎,士卒奔潰。敵執友謂曰:「降則赦汝。」友厲聲曰:「
 男兒死耳!」遂遇害。帝書「忠節傳家」四字旌其閭,官其後八人。



 友體貌英偉,膽雄萬夫,謀畫機密,出人意表。嘗語子弟曰:「天下承平日久,武事玩弛,萬一邊書告警,馬革裹尸,乃吾素志。他日收吾骸,足心黑子為識也。」其忠誠許國根於天性如此。



 郝仲連,昌元人。建炎元年,金人犯河中,守臣席益遁去。仲連時為貴州防禦使,宣撫範致虛遣節制河東軍馬,屯河中,就權府事。金將婁宿以重兵壓城,仲連率眾力
 戰,外援不至,度不能守,先自殺其家人,城陷不屈,及其子皆遇害。後贈中侍大夫、明州觀察使。



 劉惟輔,涇州人。以同州觀察使為熙河馬步軍副總管。金人既得秦州,經略使張深遣惟輔將三千騎禦之。金前軍逾鞏州,距熙才百里,惟輔留軍熟羊城,以千八百騎夜趨新店。黎明軍進,短兵相接,殺傷大當。惟輔舞槊刺其先鋒將孛堇黑鋒,洞胸墮馬死,敵為奪氣退。深檄隴右都護張嚴往追之,至鳳翔境上,惟輔不欲聽嚴節
 制,乃自別道由吳山出寶雞,獲金游騎。嚴擁大兵及金人於五里坡,金人知之,伏兵坡下,嚴與曲端期而不至,徑前,遇伏死之。惟輔自石鼻砦遁歸。



 金人略熙河,惟輔將去,顧熙河尚有積粟,恐金人因之以守,急出悉焚之。金人追及,所部皆走,惟輔與親信數百匿山寺中,遣人詣夏國求附,夏國不受。其親信軍詣金人降,金人執惟輔,誘之百方,終不言。金人怒,捽以出,惟輔奮首曰:「死犬!斬即斬,吾頭豈汝捽也。」顧坐上客曰:「國家不負汝,一旦
 遽降敵耶?」即閉口不復言而死。張浚聞之,承制贈昭化軍節度使,賻金帛布以二百計,官子孫十二人,立廟成州,號忠烈。



 有高子孺,狄道人。知蘭州龕穀砦,聞惟輔尚存,固守以待。及城陷,先刃其家而後死。韓青為熙河馬步軍第六將,間行從惟輔,為金人所擒,亦罵不絕口而死。



 牛皓,福津人。為武功大夫、川陜宣撫後軍中部將。紹興五年,金右都監撒離曷與其熙河經略使慕洧欲犯秦
 川,宣撫副使吳玠遣諸校分道伺之。皓至瓦吾谷,與金將虎山遇,皓所部步卒不滿二百,乃下與戰,謂其從曰:「吾所以舍馬者,欲與若等同死也。」金人見皓異於他人,欲招之,皓力戰死。



 有承信郎高萬,且罵且戰,與熙河路部將任安、宣撫司隊官秦元、薛琪、張亨皆死於陣。金人相謂:「真健兒也。」後皓、安皆贈翊衛大夫,官其家五人,贈萬等三官,錄其子。



 魏彥明,開封人。通判延安府。建炎二年,金人陷府東城,
 而西城猶堅守。金人並兵入鄜延,王庶自當鄜州來路,遣統制官龐世才當延安來路。天大雪,世才戰敗,自是金兵專圍西城。初受圍時,彥明與權府事劉選分地而守,彥明當東壁,空家貲以賞戰士,金人不敢犯。王庶子之道未弱冠,率老弱乘城。金人晝夜攻城,閱十有三日城陷,彥明坐子城樓上,金人並其家執之,諭使速降。彥明曰:「吾家食宋祿,犬輩使背吾君乎?」婁宿怒殺之。詔贈中大夫,官一子。



 劉士英,宣和間為溫州教授。方臘陷處州,州人爭具舟欲遁,士英奮謂不當避。自郡將而下皆排沮之,士英獨身任責,推郡茂才石礪為謀主,治兵峙糧,籍保伍,分其地為八隅,委官統率,以鐘為約,令民聞鐘聲則趨所守堞。未幾,賊來攻,拒守凡四十餘日,官軍繼至,賊潰去。



 靖康初,通判太原府。金人入境,帥臣張孝純欲避之,士英率通判方笈、將官王稟力止孝純。及城陷,稟赴火死,士英持短兵接戰,死之。笈在金,因講和使附書言二人死
 節,後刻石於溫、衢二州。



 翟興,字公祥,河南伊陽人。少以勇聞。劇賊王伸起,興與弟進應募擊賊,號大翟、小翟。金人犯京師,西道總管王襄檄興統領在城軍馬。以保護陵寢功補承信郎,闢京西北路兵馬副鈐轄,為陜西宣撫司前軍統制。高世由以澤州降金,金以為西京留守。興與進提步卒數百,卷甲夜趨洛陽,擒世由等斬之。



 群盜冀德、韓清出沒汝、洛間,興以輕騎追襲,德就擒,清僅以身免。會進為叛將楊
 進所害,賊乘勢擊敗官軍,興帥餘眾拒賊,保伊川。明年,訴進死事於朝,以興代進為京西北路安撫制置使兼京西北路招討使,兼知河南府。楊進屯鳴皋山北,興與子琮帥鄉兵時出擾之,進懼,棄輜重南走,興邀擊於魯山縣,進中流矢死,餘眾潰去,西京平。



 賊王俊據汝州,興引兵攻之,俊棄城去,退保繖蓋山。興進攻,免胄大呼曰:「賊識我乎?我翟總管也。」眾皆披靡,遂破之。



 金人犯河陽、鞏縣、永安軍,興遣子琮與搏戰,屢捷,追至澠池。詔授河
 南孟、汝、唐州鎮撫使兼知河南府,轉武略大夫兼閣門宣贊舍人,寓治伊陽。時河東、北雖陷,土豪聚眾保險,興遣蠟書結約之,向密、王簡、王英輩皆願受節制。奏上,高宗嘉之,授河東、北路軍馬使,遍檄山砦,由是汾、澤、潞、懷、衛間山砦首領皆應命。



 金人入陜右,興遣將邀擊,俘五十餘人,又遣子琮生擒金河東都統保骨,遂復陽城縣,乘勝取絳之垣曲,進至米糧川。紹興元年春,金重兵犯河南,時興軍乏糧,就食諸道,僅存親兵自衛,人情震恐。
 興授將彭玘方略,設伏於井首,俟敵至陽遁,金眾果追玘,伏發,金帥就擒。鄧州人楊某擁眾河北,偽稱「信王」,興遣將董先追獲於商州殺之。進武功大夫、忠州團練使。



 劉豫將遷汴,以興屯伊陽,憚之,遣蔣頤持書誘興以王爵。興斬頤焚其書,豫計不行,乃陰遣人啖裨將楊偉以利,偉殺興,攜其首奔豫。或云:賂偉為內應,以兵徑犯中軍,興奮擊墜馬死。事聞,贈保信軍節度使。



 興威貌魁偉,每怒,須輒張。軍食不繼,士以菽粟雜藜藿食之,激以忠義,無不奮厲。在河南累
 年,金人不敢犯諸陵。詔賜軍名「忠護」。



 子琮,沈勇有父風,繼興為鎮撫使;琳,閣門祗候。



 進字先之。以捕盜勞補下班殿侍,累功充京西第一將。坐熙河帥劉法涇原戰失利,降官停任,尋敘復。女真歸故地,改河北第四將。往至遂城,會契丹兵奄至,都統制劉延慶以進為先鋒,與契丹戰於幽州石料岡、盧溝河皆捷。又與契丹大將遇於峰山,力戰彌日,契丹潰去。



 金人犯京師,朝廷密詔西道總管王襄會兵三萬赴京城,
 至葉縣,襄欲引兵而南,進諫止之,因分軍遣進持書而西。時經略使範致虛已合五路軍馬次潼關,以進統河南民兵,收復西京。進至福昌,遣兵襲金營。時金游騎往來外邑,進設伏擒之。金人逼靈山砦,進父子兄弟與之戰,潰圍至高都,集鄉兵七百人,夜行晝伏,五日至洛城,夜半破關入,擒高世由。再捷於伊陽白草塢。都總管孫昭遠至洛陽,以進戍澠池界,授武義大夫、閣門宣贊舍人。



 金人犯白浪隘,將渡河,進破之。未幾,洛陽再陷,進在
 伊陽,裒散亡才千人。金人犯薛封,進選精銳三百人,夜縱火斫其營,焚死者甚眾。又戰於驢道堰,生擒金將翟海,追至梅花穀。賊冀德、韓清嘯聚南陽,進間道擊之,德降,繼斬清於艾蒿平。勒兵抵龍門,屢與金人夾河戰,乘勝入洛陽。或曰:「彼砦尚固,城未可守。」不聽。金人聚懷、衛、蒲、孟數州之眾薄城下,斧諸門入,進率士卒巷戰,次子亮死之。遷武功大夫、閣門宣贊舍人,充京西北路兵馬都鈐轄,尋授馬步軍副總管,升本路制置使,兼知河
 南府。



 會東京留守杜充所招巨寇楊進號「沒角牛」者,擁兵數萬,殘害汝、洛間。進謂其兄興欲力除之。會楊進遣數百騎絕水犯進營,進乘半渡擊之,追賊數十里,破賊四砦,馬驚墜塹,為賊所害。贈左武大夫、忠州刺史,官其後五人。



 朱蹕,湖州安吉人,知錢塘縣。建炎三年,金人陷杭州,初犯餘杭,守臣康允之退保赭山。蹕白允之率弓手、士軍前路拒敵,使杭民為逃死計。行二十里,遇金兵,蹕兩中
 流矢,左右掖至天竺山,猶能率鄉兵禦敵。後數日遇害。時兀術自安吉進兵,過獨松關,曰:「南朝若以羸兵數百守此,吾豈能遽度哉!」



 朱良者,字良伯,吳郡人。世儒科。建炎中,為海鹽縣尉。金兵入境,良謂僚友曰:「今日乃忠臣義士死國之時也。」被甲執戈,集所部百餘人奮而前,擊金兵數人死,眾為披靡,然力不敵,竟死。事聞,官其子思,後守漢陽。



 方允武者,衢州人。武學上舍,補官為常州宜興巡檢。建
 炎三年,金人入縣之金泉鄉,允武率土軍、鄉民迎敵,殺獲數級,奪弓箭與旗。後遇金兵梅嶺村,力戰而沒。詔贈兩官,官其家二人。



 龔楫字濟道,兵部侍郎原之孫,世以儒學顯。楫懦如不勝衣。建炎初,聞金人陷郡縣,輒忿恚不食,念有以自見而不可得。兀術據和州,以偏師萬人築堡新塘,遏絕濡須之路。楫率家僮百餘人襲之,鄉里從者三千餘人,獲千戶二,系累者數百人,輜重稱是。縱遣所掠州民父母
 妻子,將歸於滁、和鎮撫司。遇金兵大至,乃取道圩上,金騎兵據其沖,不得前,眾多赴水死。楫麾其眾曰:「今日鬥死亦足稱義士,自棄溝瀆無益也。」戰敗,為金人所獲,猶挺劍刺其一人,罵不絕口,金人臠割之。年二十二。



 金人初至新塘,有蔣子春者,教授里中。金人見其挾書,又人物秀整,喜之,欲命以官,子春怒罵,乃殺之。



 李亙者,字可大,兗州乾封人。少好學,有知慮。大觀二年進士。徐處仁當國,擢尚書郎官。建炎末,金人犯淮南,亙
 不及避,劉豫使守大名。與凌唐佐謀,密陳豫可取狀告於朝。募卒劉全、宋萬、僧惠欽輩十餘,往返事洩,全、萬、惠欽為邏者所得,亙坐死。後贈官,立祠曰愍忠。



 又有武顯大夫孫安道,為應天府兵馬鈐轄。城陷不得歸,謀挺身還朝,為人所告而死。後贈忠州刺史。



 凌唐佐字公弼,徽州休寧人。元符三年進士。建炎初,提點京畿刑獄,加直秘閣,知南京。南京陷,劉豫因使為守。唐佐與宋汝為密疏其虛實,遣人持蠟書告於朝。江、淮
 都督呂頤浩過常州,得唐佐從孫憲,授保義郎、閣門祗候,俾持帛書遺之。憲至睢陽,事洩,豫捕唐佐並其家,憲脫歸。唐佐見豫,責以大義,豫怒,斬唐佐境上。李橫復潁昌,言於朝,詔贈徽猷閣待制。



 楊粹中,真定府人。建炎二年,金人大入,時粹中知濮州,固守不下。粘罕以濮小郡,易之,將官姚端乘其不意,夜搗其營,直犯中軍,粘罕跣足走,僅以身免。遂急攻城,凡三十三日而陷,端率死士突出。粘罕入其城,粹中登浮
 圖不下,粘罕嘉其忠義,許以不死,乃以粹中歸。粹中竟不屈而死,守御官杜績亦死之。贈粹中徽猷閣待制。



 強霓,自金歸宋,為武功大夫、閣門宣贊舍人、知環州、環慶路統制軍馬兼沿邊安撫使。隆興間,金兵圍環州,與其弟武經大夫、環慶路統領沿邊忠義軍馬震堅守孤城,招誘使降,不屈,城陷死焉。興州駐扎御前諸軍統制吳挺言於朝,並贈觀察使,立廟西和州,賜額旌忠。



 康傑者,權知扶風縣,與金將馮宣戰,宣愛而欲招之,傑
 奮曰:「吾今也當死於陣,不能降敵。」宣殺之。



 李伸者,知天興縣,堅守不下,城陷,曰:「吾豈使敵殺我。」遂自殺。



 郭僎,字同升,開封祥符縣人。以父任調海州東海縣尉,權祥符縣尉。時童貫子師閔死,敕葬邑境,僎任道途之役。貫命徹民屋之當道者,僎先籍童氏屋數十間欲毀之,貫遽令勿毀,由是民屋得免。



 再調濱州招安丞,又為亳州蒙城丞。令以鹽科邑民,僎爭之不可。郡守以僎丞
 鹿邑,中貴人楊逢周率軍士二百人,以捕寇為名入邑境,所至騷動。僎檄逢周取所受文書,逢周不與,僎令尉譏察之。逢同歸,訴於徽宗,詔逮僎赴開封府獄,獄以狀聞,乃使還任。



 闢權咸平縣丞。靖康初,勤王兵有剽掠邑界者,僎率民兵擊之,得犯者斬以徇。會金人大至,力不敵,其僚欲降之,僎走南京從趙野乞師,不從,慟哭而歸。尋知宣城縣。苗傅、劉正彥之變,呂頤浩傳檄諸郡,僎說郡守劉玨,請募勇士倍道赴難,揭榜復用建炎年號,人
 皆韙之。



 通判全州,權饒州浮梁宰,未行,時有賊張頂花者已逼縣境,眾止之,僎曰:「安逸則就,艱危則辭,非我所學。」徑就道。至縣,約束吏士,誓以死戰。賊聞之,偽降,入邑為變,邑官竄伏,僎曰:「吾為宰,義不可去。」端坐公署,賊徒責僎,僎大罵不絕口,遂遇害。詔贈承議郎,錄其後二人。



 郭贊者,汝陽縣丞也。建炎二年,金人陷蔡州,守臣閻孝忠聞之,先遣其家,獨聚軍民守城。金人陷城,孝忠為所執,見其貌陋且侏儒,乃令荷擔,因乘間而逃。獨贊朝服詬
 叱不肯降,遂見殺。



 王迸字純父,饒州樂平人。鄉舉恩免,為固始簿,攝邑。紹定中,金兵犯淮,守令望風遁,迸度力不能御,懷印自投於井而死。



 吳從龍字子云,官至武功郎、建康府統制。紹定兵難,為先鋒,援不至,被擒,使至泰州城下誘降,終不屈,死之。廟祀揚、泰二州,賜額褒忠。官其弟從虎,至武經大夫。



 司馬夢求,敘州人,溫國公光之後也。母程,歸及門,夫死,
 誓不它適,旌其門曰「節婦」。夢求,其族子,取以為後。景定三年,舉進士。咸淳末,調江陵沙市監鎮。沙市距城才十五里,南阻蜀江,北倚江陵,地勢險固,為舟車之會,恃水為防。德祐元年,湖水忽涸,北兵橫遏中道,乘南風縱火,都統程文亮逆戰於馬頭岸,制置使高達束手不援,文亮降。夢求朝服望闕再拜,自經死。



 林空齋,永福人,失其名。父同,官至監丞。空齋舉進士,歷知縣,解官家居。益王立,張世傑圍泉州,乃率鄉人黃必
 大、劉仝祖即其家開忠義局,起義兵,復永福縣。時王積翁以福安送款世傑,然實密約北兵。兵至,屠永福,必大、仝祖等走它邑。空齋盛服坐堂上,嚙指血書壁云:「生為忠義臣,死為忠義鬼。草間雖可活,吾不忍為爾。諸君何為者,自古皆有死。」俄見執,不屈而死。



 黃介,字剛中,隆興分寧人。意氣卓越,喜兵法。制置使朱祀孫帥蜀,介上攻守策,祀孫愛之,以自隨。夏貴闢充廣濟簿尉,平反死囚,尹不能抗。錢真孫復闢入幕,及與真
 孫別,誦「南八,男兒死爾」語以勉之。後家居,帥鄉民登龍安山為保聚計。德祐元年,北兵至砦,眾奔潰,介堅守不去,且射且詬,面中六矢不為動,顧謂家僮陳力曰:「爾盡力勿走。」力曰:「主在,死生同之。」介身被鏃如蝟,面頸復中十三矢,倚柵而死,力亦死。



 妻劉被掠,子用中逃,得不死。乃壯,求母四方,逾十年,得於京師以歸,州里稱為黃孝子云。



 孫益,揚州泰興人。少豪俠。紹定中,李全犯楊州,游騎薄
 泰興城下,縣令王爚募人守御,益起從之。俄賊兵大至,益率眾拒之。眾見賊勢盛,且前且卻,益厲聲呼曰:「王令君募我來,將以守護城邑也。今賊至城下,我輩不為一死,復何面目見令君乎?」遂身先赴敵,死之。



 同時顧緒、顧珣俱戰死。事聞,贈益保義郎,緒、珣承節郎,各官其子一人。



 王仙,蜀都統也。守涪州,北兵攻圍無虛日,勢孤援絕。宋亡之二年,城始破,仙自刎,斷其亢不殊,以兩手自摘其首墜死。



 曹琦,蜀進士也。知南平軍,亦被執,脫身南歸,制置闢主管機宜文字。聞都統趙安以城降,就守禦地自經死。



 吳楚材名炎,以字行,建昌南城人。德祐元年,建昌降。明年春,楚材還其鄉領村,糾集民兵。時江西制置使黃萬石走邵武,遂繇邵武守黎靖德請於萬石,乞濟師,萬石不許,而授楚材迪功郎、權制置司計議官以安之,且戒勿興兵。楚材不聽,二月己亥,自領村率眾,晨炊蓐食,將攻城。鉦鼓震動,甫至近郊之龜湖,北兵三道蹴之,奪其
 長梯鐵鉤,因進攻領村,拒以木柵,不得入。事聞,益王元帥府承制遷楚材宣義郎、帶行太社令、知建昌軍,俾聚兵圖再舉。萬石匿其命。



 楚材既失利,且乏援,大元兵誘降,其眾多解去。楚材走光澤,為人所執,及其子應登以獻。郡遣錄事婁南良訊之曰:「汝何為錯舉?」楚材抗聲曰:「不錯,不錯。如府錄所為,乃大錯爾。府錄受宋官爵,今乃為敵用事,還思身上綠袍自何而得?吾一鄙儒,特為忠義所激,為國出力,事雖不成,正不錯也。」南良愧而語塞。
 及吳浚為江西制置、招討使,斬楚材父子,傳首諸邑。益王立於福州,聞而哀之,贈官朝奉郎,即邵武境上?



\end{pinyinscope}