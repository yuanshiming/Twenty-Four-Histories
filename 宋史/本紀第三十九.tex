\article{本紀第三十九}

\begin{pinyinscope}

 寧宗三



 嘉定元年春正月戊寅,右諫議大夫葉時等請梟韓侂冑首於兩淮以謝天下,不報。辛巳,下詔求言。壬午,王柟還自河南,持金人牒,求韓侂冑首。丙戌,葉時等復請梟
 侂冑首於兩淮。戊子,安定郡王伯栩薨。壬辰,以史彌遠知樞密院事,以許奕為金國通謝使。



 二月戊申,追復趙汝愚觀文殿大學士,謚忠定。詔史官改紹熙以來韓侂冑事跡。壬子,詔臨安府振給流民。戊午,責授程松果州團練副吏、賓州安置。是月,郴州黑風峒寇羅世傳作亂,招降之。



 三月癸酉,以毛自知首論用兵,奪進士第一人恩例。戊子,下詔戒飭內外群臣。復秦檜王爵、贈謚。己丑,王柟自軍前再還行在,議以韓侂冑函首易淮、陜侵地。
 辛卯,詔梟侂冑首於兩淮。是春,子□生。



 夏四月丙辰,詔後省科別群臣奏疏可行者以聞。贈彭龜年寶謨閣直學士,落李沐寶文閣學士。戊午,再責授陳自強復州團練副使、雷州安置,仍籍其家。



 閏月辛未,置拘榷安邊錢物所。壬申,雨雹。癸未,子□薨,追封肅王,謚沖靖。詔大理、三衙、臨安府及諸路闕雨州縣決系囚,釋杖以下。甲申,詔自今視事,令皇太子侍立。乙酉,以錢象祖兼太子少傅,衛涇、雷孝友、林大中並兼太子賓客。辛卯,以旱,禱於
 天地、宗廟、社稷。癸巳,減常膳。乙未,蠲兩浙闕雨州縣貧民逋賦。命大理、三衙、臨安府、兩浙州縣決系囚。丙申,幸太乙宮、明慶寺禱雨。丁酉,以旱,詔求言。



 五月辛酉,賜禮部進士鄭自成以下四百二十有六人及第、出身。甲子,太白經天。乙丑,以飛蝗為災,減常膳。丁卯,詔侍從、臺諫疏奏闕政,監司、守令條上民間利害。



 六月庚午,金人歸大散關。辛未,金人歸濠州。乙亥,衛涇罷。丙子,遣鄒應龍賀金主生辰。甲申,林大中薨。乙未,以蝗,禱於天地、社稷。
 丙戌,詔侍從、兩省、臺諫舉沿邊守臣。辛卯,以史彌遠兼參知政事。



 秋七月辛丑,詔呂祖泰特補上州文學。癸丑,以丘崇同知樞密院事。壬戌,以飛蝗為災,詔三省疏奏寬恤未盡之事。



 八月戊辰朔,發米振貧民。辛未,丘崇卒。甲戌,命侍從、臺諫、兩省詳議會子折閱利害。辛巳,以禮部尚書婁機同知樞密院事,吏部尚書樓鑰簽書樞密院事。丙戌,詔禮部侍郎許奕、起居舍人曾從龍考訂監司、守令所陳民間利害,擇可行者以聞,其未上者趣之。
 甲午,發米二十萬,振糶江、淮流民。



 九月辛丑,金使完顏侃、喬宇入見。壬子,出安邊所錢一百萬緡,命江、淮制置大使司糴米振饑民。己未,詔以和議成諭天下。甲子,遣曾從龍使金賀正旦。乙丑,大風。赦沿邊諸州。



 冬十月丙子,以錢象祖為左丞相,史彌遠為右丞相。雷孝友知樞密院事仍兼參知政事,婁機參知政事,樓鑰同知樞密院事。己卯,褒錄慶元上書楊宏中等六人。庚辰,封伯柷為安定郡王。辛巳,蔡璉除名,配贛州牢城。癸未,金遣使
 來賀瑞慶節。



 十一月丙辰,金主璟殂。戊午,史彌遠以母憂去位。十二月戊辰,錢象祖罷。庚午,四川初行當五大錢。升嘉興府為嘉興軍。再奪李沐三官、信州居住。戊寅,改命曾從龍使金吊祭。己卯,黎州蠻畜卜寇邊。己丑,遣宇文紹彭使金賀即位。辛卯,蠲兩淮州軍二稅一年。是歲,江、淮制置司汰雄淮軍歸農,淮東揀刺八千餘人以補鎮江大軍及武鋒軍之闕,淮西揀刺二萬六千餘人以為御前定武軍。



 二年春正月庚子,詔內外有司疏陳節用之事。辛丑,金遣裴滿正來告哀。丁己,以樓鑰參知政事,御史中丞章良能知樞同密院事,吏部尚書宇文紹節簽書樞密院事。庚申,金遣蒲察知剛來獻遺留物。詔侍從、兩省、臺諫各舉監司、郡守治行尤異者二三人。



 二月己巳,金遣使來告即位。庚午,黎州蠻寇邊。壬午,以會子折閱日甚,詔侍從、兩省以下各疏奏所見。丁亥,罷法科試經義,復六場舊法。戊子,大風。



 三月丙申,雨雹。巳酉,詔民以減會子
 之直籍沒家財者,有司立還之。戊午,禁兩淮官吏私買民田。庚申,命浙西及沿江諸州給流民病者藥。辛酉,罷漳、泉、福三州、興化軍賣廢寺田。壬戌,出內庫錢十萬緡為臨安貧民棺槥費。



 夏四月乙丑,詔諸路監司督州縣捕蝗。戊辰,江、淮制置司言,放廬、濠二州忠義軍歸農。甲申,賜臨安諸軍死者棺錢。戊子,賜楊震仲謚曰節毅。



 五月丙申,史彌遠起復。丁酉,以旱,詔諸路監司決系囚,劾守令之貪殘者,戊戌,借補訓武郎羅日願謀為變,伏誅。
 庚子,詔侍從、兩省、臺諫各舉監司、郡守有政績才望者二人,以補郎官之闕。辛丑,申命州縣捕蝗。癸卯,詔兩淮、荊襄守令以戶口多寡為殿最。乙卯,釋大理、三衙、臨安府、兩浙州縣杖以下囚。除茶鹽賞錢。己未,以旱,詔群臣上封事。庚申,禱於天地、宗廟、社稷。



 六月癸亥朔,命浙西諸州諭民種麻豆,毋督其租。詔臺省及諸路監司速決滯獄。戊辰,奉安成肅皇后神御於景靈宮。己巳,遣俞應符賀金主生辰。乙酉,復禱雨於天地、宗廟、社稷。己丑,命
 江西、福建、二廣豐稔諸州糴運以給臨安,仍償其費。辛卯,京湖制置司言,放諸州新軍及忠義人歸農。



 秋七月癸巳,命有司舉行寬恤之政五條。乙未,詔荒歉州縣七歲以下男女聽異姓收養,著為令。己亥,蠲信陽、荊門、漢陽軍民賦。壬寅,命兩淮轉運司給諸州民麥種。癸卯,募民以振饑免役。



 八月甲子,聽兩淮諸州民行鐵錢於沿江八州。乙丑,以安丙為四川制置大使,罷宣撫司。甲戌,冊皇太子。丁丑,皇太子謁於太廟。戊寅,詔皇太子更名
 詢。己卯,黎州蠻復寇邊。丙戌,發米十萬石振兩淮饑民。



 九月己亥,朝獻於景靈宮。庚子,朝享於太廟。辛丑,合祭天地於明堂,大赦。丙午,增太學內舍生十員。癸丑,命吏部郎官劉□龠等審定中外所陳會子利害,上於朝。己未,遣費培使金賀正旦。



 冬十月丁卯,命京湖制置司募逃卒及放散忠義以補廂、禁軍闕。丁丑,金遣使來賀瑞慶節。己丑,命兩淮轉運司給諸州民稻種。減公私房廊白地錢什之三。



 十一月辛卯朔,沔州統制張林等謀作亂,
 事覺,貸死除名、廣南羈管。甲午,詔浙西監司募饑民修水利。乙未,以歲饑罷雪宴。是月,郴州黑風峒寇李元礪作亂,眾數萬,連破吉、郴諸縣,詔遣荊、鄂、江、池四州軍討之。十二月甲子,四川制置大使司調官軍討黎州蠻,敗績。己巳,賜朱熹謚曰文。乙亥,詔諸州毋糴職田租。丙戌,金遣使來賀明年正旦。是歲,諸路旱蝗,揚、楚、衡、郴、吉五州、南安軍盜起。



 三年春正月甲辰,下詔招諭群盜。又詔戒飭監司、郡守。
 丙午,雨土。



 二月辛酉,黎州蠻復寇邊。庚午,詔楚州武鋒軍歲給累重錢,如大軍例。壬午,以工部侍郎王居安知隆興府,督捕峒寇。



 三月丁酉,蠲都城及荒歉諸州民間逋負。己亥,以湖南轉運判官曹彥約知潭州,督捕峒寇。庚子,賜彭龜年謚曰忠肅。甲寅,誅楚州渠賊胡海。丙辰,以久雨,釋兩浙州縣系囚。



 夏四月癸亥,李元礪犯南雄州,官軍大敗。乙丑,決臨安系囚,釋杖以下。丙寅,詔監司、守臣安集泰、吉二州民經賊蹂踐者。戊辰,出內庫錢二
 十三萬緡賜臨安軍民。己巳,詔臨安府給細民病死者棺櫬。



 五月乙未,淮東賊悉平,詔寬恤殘破州縣。甲辰,以去歲旱蝗,百官應詔封事,命兩省擇可行者以聞。乙巳,命沿海諸州督捕海寇。戊申,經理兩淮屯田。庚戌,以江陵忠勇軍為御前忠勇軍。癸丑,以久雨,發米振貧民。



 六月丁己朔,日有食之。壬戌,命有司舉行寬恤之政十有九條。癸亥,遣黃中賀金主生辰。己卯,加楊次山少保,封永陽郡王。詔三衙、江上、四川諸軍主帥核實軍籍,期冒
 者以贓論。是月,池州副都統許俊、江州副都統劉元鼎與李元礪戰於江西,皆不利。知潭州曹彥約又與賊戰,亦為所敗,賊勢愈熾。



 秋七月辛卯,申嚴圍田增廣之禁。癸卯,定南班為三十員。



 八月乙亥,大風拔木。是月,臨安府蝗。



 九月丙戌朔,詔三衙、江上諸軍,升差將校必以材藝年勞,其徇私者,臺諫及制置、總領劾之。癸丑,遣錢仲彪使金賀正旦。



 冬十月壬申,雷。金遣使來賀瑞慶節。丁丑,推南雄州戰歿將士恩。



 十一月癸巳,賞楚州平賊功。
 乙巳,遣朝臣二人往兩浙路與提舉官議收浮鹽。是月,李元礪迫贛州、南安軍,詔以重賞募人討之。十二月丙辰,詔江、淮諸司嚴飭守令安集流民。戊午,婁機罷。丙寅,湖南賊羅世傳縛李元礪以降,峒寇悉平。辛巳,金遣使來賀明年正旦。黎州蠻請降。是歲,臨安、紹興二府、嚴、衢二州大水,振之,仍蠲其賦。



 四年春正月己丑,敘州蠻攻嘉定府利店砦,陷之。甲辰,以四川鹽擔錢對減激賞絹一年。丙午,詔湖南、江西諸
 州經賊蹂踐者,監司、守臣考縣令安集之實,第其能否以聞。



 二月乙卯,李元礪伏誅。壬戌,羅世傳補官,尋復叛。辛巳,罷廣西諸州牛稅。



 閏月丁未,大風。辛亥,詔諸路帥臣、監司、守令格朝廷振恤之令及盜發不即捕者,重罪之。



 三月己未,臨安府振給病民,死者賜棺錢。丙子,沔州將劉世雄等謀據仙人原作亂,伏誅。夏四月甲申,禁兩浙、福建州縣科折鹽酒。己丑,以吳曦沒官田租代輸關外四州旱傷秋稅。丙午,賜黑風峒名曰效忠。戊申,出內
 庫錢瘞疫死者貧民。是月,四川制置大使司置安邊司以經制蠻事,命成都路提刑李□、潼川路安撫許奕共領之。



 五月乙亥,賜禮部進士趙建大以下四百六十有五人及第、出身。



 六月丁亥,遣余嶸賀金主生辰,會金國有難,不至而還。減京畿囚罪一等,釋杖以下。辛丑,更定四川諸軍軍額。



 秋七月壬戌,太白晝見。丙寅,詔四川官吏嘗受偽命者,自今毋得敘用。丁丑,詔軍興以來爵賞冒濫者聽自陳,除其罪。



 九月辛酉,敘州蠻寇邊。乙亥,羅
 世傳為其黨所殺。丁丑,遣程卓使金賀正旦。詔附會開邊得罪之人,自今毋得敘用。



 冬十月甲辰,以金國有難,命江淮、京湖、四川制置司謹邊備。



 十一月己酉朔,日有食之。癸丑,賞平峒寇功。甲戌,申嚴諸軍升差之制。十二月辛巳,奉議郎張鎡坐扇搖國本除名、象州羈管。癸未,以會子折閱不行,遣官體訪江、浙諸州。乙巳,金遣使來賀明年正旦。是歲,金國有難,賀生辰使不至。



 五年春正月己巳,詔諸路通行兩浙倍役法,著為令。壬
 申,賜李好義謚曰忠壯。



 二月壬午,罷兩淮軍興以來借補官。



 三月庚戌,四川制置司遣兵分道討敘州蠻,其酋米在請降。戊辰,以久雨,詔大理、三衙、臨安府、兩浙州縣決系囚。甲戌,以廣東、湖南、京西盜平,監司、帥臣進職有差。夏五月癸酉,安南國王李龍[A147]卒,以其子昊旵為安南國王。詔州縣見役人毋納免役錢,役滿復輸。



 六月癸未,遣傅誠賀金主生辰。乙酉,禁銅錢過江。



 秋七月庚申,賞降敘州蠻功。戊辰,以雷雨毀太廟屋,避正殿減膳。



 八
 月甲戌朔,御後殿,復膳。



 九月丙午,太白晝見。己酉,有司上《續編中興禮書》。庚戌,遵義砦夷楊煥來獻馬。辛未,罷沿海諸州海船錢。遣應武使金賀正旦。



 冬十月辛巳,詔諸路總領官歲舉堪將帥者二三人,安撫、提刑舉可備將材者各二人。戊子,金遣使來賀瑞慶節。戊戌,雷。遣使吊祭安南。



 十一月庚申,朝獻於景靈宮。辛酉,朝饗於太廟。壬戌,祀天地於圜丘,大赦。十二月丁丑,再蠲濠州租稅一年。壬午,詔蠲州縣橫增稅額。己亥,金遣使來賀明
 年正旦。



 六年春正月庚申,宇文紹節卒。詔侍從、臺諫、兩省官、帥守、監司各舉實才二三人。



 二月丁丑,太白晝見。丙戌,有司上《嘉定編修吏部條法總類》。乙未,詔宗室毋與胥吏通姻,著為令。



 三月癸亥,樓鑰罷。



 夏四月丙子,以章良能參知政事。甲午,復法科試經義法,雜流進納人不預。



 五月丁卯,以旱,命大理、三衙、臨安府決系囚。戊辰,修慶元六年以來寬恤詔令。



 六月乙亥,詔刑部歲終上諸州未
 決之獄於尚書省,擇其最久者罪之。丁丑,遣董居誼賀金主生辰,會金國亂,不至而還。丁亥,復監司臧否守令及監司、郡守舉廉吏所知法。丙申,詔三衙、江上諸軍主帥各舉堪將帥者二三人。



 八月己巳朔,詔諸路監司、帥臣舉所部官吏之才行卓絕、績用章著者。庚午,知思州田宗範謀作亂,夔州路安撫司遣兵討平之。是月,金人弒其主允濟。



 九月甲辰,蠲京、湖諸州逋負二十八萬餘緡。



 閏月戊辰朔,詔御史臺置考課監司簿。丙戌,以金主
 新立,命四川謹邊備。己丑,詔湖北監司、守令振恤旱傷。癸巳,雷。甲午,史彌遠等上《三祖下七世仙源類譜》、《高宗寶訓》、《皇帝玉牒》、《會要》。乙未,大雷。丙申,以雷發非時,下罪己詔。



 冬十月丁酉朔,申嚴互送之禁。戊申,遣真德秀賀金主即位,會金國亂,不至而還。庚戌,遣李□使金賀正旦,亦不至而還。甲子,金遣使來告即位。



 十一月癸未,虛恨蠻寇嘉定府之中鎮砦。十二月壬寅,蠲瓊州丁鹽錢。癸亥,金遣使來賀明年正旦。是歲,兩浙諸州大水,振之。



 七年春正月丁卯朔,四川制置司遣提舉皂郊博馬務何九齡率諸將及金人戰於秦州城下,敗還。丁丑,章良能薨。壬午,沔州都統王大才斬何九齡,梟首境上,以其事聞。



 三月丁卯,以安丙同知樞密院事,成都府路安撫使董居誼為四川制置使。庚辰,金國來督二年歲幣。戊子,金人來止賀正旦使。



 夏四月癸卯,蠲福建沿海諸州貧民納鹽。



 五月丁丑,太白經天。乙酉,賜禮部進士袁甫以下五百四人及第、出身。



 六月辛丑,以旱,命諸路州軍
 禱雨。甲辰,詔諸路監司、守臣速決滯訟。丙午,蠲兩浙路諸州贓賞錢。壬子,釋大理、三衙及兩浙路杖以下囚。丁巳,置嘉定府邊丁二千人以備蠻。



 秋七月甲子朔,以左諫議大夫鄭昭先簽書樞密院事兼權參知政事。戊辰,詔省吏毋授參議官。乙亥,金人來告遷於南京。庚寅,以起居舍人真德秀奏,罷金國歲幣。是月,夏人以書來四川,議夾攻金人,不報。



 八月癸巳朔,罷關外四州所增方田稅。乙未,罷四川宣制司所補官。癸卯,復建宗學。置博
 士、諭各一人,弟子員百人。金國復來督歲幣。乙巳,太白經天。禁州縣沮壞義役。戊申,詔以安丙為觀文殿學士、知潭州。



 九月壬戌朔,日有食之,太白晝見。乙丑,史彌遠等上《高宗中興經武要略》。戊寅,調殿前司兵增戍天長縣。丙戌,以久雨,釋大理、三衙、臨安府杖以下囚。庚寅,釋兩浙路杖以下囚。除茶鹽賞錢。



 冬十月壬辰朔,出內帑錢振臨安府貧民。



 十一月辛酉朔,遣聶子述使金賀正旦,刑部侍郎劉□龠等及太學諸生上章言其不可,不報。
 丙戌,命浙東監司發常平米振災傷州縣。罷四川制置大使司所開鹽井。十二月甲午,復罷同安監鑄錢。丁巳,金遣使來賀明年正旦。是歲,黎州蠻畜卜始降。



 八年春正月辛未,命師禹嗣秀王。詔侍從、兩省、臺諫各舉將材三人。己卯,遣丁煜賀金主生辰。戊子,申嚴銷金鋪翠之禁。



 二月丙午,雷孝友罷。壬子,蠲平江等五郡逋負米,釋其系囚。己未,雨土。



 三月辛酉,詔大郡歲舉廉吏二人,小郡一人。乙亥,以旱,命諸路州縣禱雨。丙子,蠲臨
 安府茶鹽賞錢。釋兩浙諸州系囚。辛巳,應賢良方正能直言極諫科何致坐妄造事端、營惑眾聽,配廣西牢城。癸未,安定郡王伯柷薨。丙戌,釋江、淮闕雨州郡杖以下囚。



 夏四月乙未,幸太一宮、明慶寺禱雨。辛丑,避正殿,減膳。壬寅,禱雨於天地、宗廟、社稷。癸卯,詔中外臣民直言時政得失。乙巳,減臨安及諸路雜犯死罪以下囚,釋杖以下。



 五月辛未,雨。己卯,命利州路安撫司招刺忠義人。辛巳,御正殿,復膳。癸未,復命有司禱雨。甲申,詔贓吏毋
 得減年參選,著為令。乙酉,發米振糶臨安府貧民。



 六月丙辰,詔兩浙、江、淮路諭民雜種粟麥麻豆,有司毋收其賦,田主毋責其租。



 秋七月辛酉,以鄭昭先參知政事,禮部尚書曾從龍簽書樞密院事。壬戌,詔四川立楊巨源廟,名曰褒忠。戊辰,蠲兩淮諸州今年秋稅並極邊五州明年夏稅。癸酉,蠲臨安、紹興二府貧民夏稅。丙子,發米三十萬石振糶江東饑民。庚辰,詔弟搢更名思正,侄均更名貴和。甲申,詔職田蠲放如民田,違者坐之。



 八月己
 丑,賜張栻謚曰宣。庚子,申嚴宗子訓名法。丁未,權罷旱傷州縣比較賞罰。己酉,禁州縣遏糴。是月,蘭州盜程彥暉求內附,四川制置使董居誼卻之。



 九月己巳,朝獻於景靈宮。庚午,朝饗於太廟。辛未,合祭天地於明堂,大赦。乙亥,申嚴兩浙圍田之禁。甲申,罷四川法科試。



 冬十月乙未,命六部各類赦書寬恤事,下諸路監司推行。壬寅,金遣使來賀瑞慶節。



 十一月丙辰朔,封伯澤為安定郡王。癸亥,遣施累使金賀正旦。十二月己丑,詔楊巨源、李
 好義子孫各進一官。辛亥,金遣使來賀明年正旦。是歲,兩浙、江東西路旱蝗。



 九年春正月乙丑,賜呂祖謙謚曰成。置馬軍司水軍。乙亥,遣留筠賀金主生辰。丙子,命諸州招填軍籍。辛巳,罷諸路旱蝗州縣和糴及四川關外科糴。



 二月甲申朔,日有食之。辛亥,東西兩川地大震。



 三月乙卯,又震。甲子,又震,馬湖夷界山崩八十里,江水不通。丁卯,又震。壬申,又震。丁丑,詔侍從、臺諫、兩省舉堪監司者各二人。



 夏四月
 戊戌,秦州人唐進與其徒何進等引眾十萬來歸,四川制置使董居誼拒卻之。



 五月癸酉,太白晝見。



 六月辛卯,西川地震。壬辰,又震。乙未,又震,黎州山崩。戊申,振恤浙西被水州縣,寬其租稅。



 秋七月戊辰,詔邊縣擇才不拘常法,其餘並遵三年之制。



 九月甲申,詔兩浙、江東監司核州縣被水最甚者,蠲其租。



 冬十月癸亥,西川地震。甲子,又震。丙寅,金遣使來賀瑞慶節。



 十一月庚寅,遣陳伯震使金賀正旦。癸卯,以程彥暉攻圍鞏州,迫及川界,命
 利州副都統劉昌祖移駐西和州以備之。十二月丁巳,再給諸軍雪寒錢。乙亥,金遣使來賀明年正旦。



\end{pinyinscope}