\article{志第一百一十七 職官四}

\begin{pinyinscope}

 御史
 臺秘書省殿中省太常寺宗正寺大宗正司



 附:光祿寺衛尉寺大僕寺



 御史臺掌糾察官邪,肅正綱紀。大事則廷辨,小事則奏彈。其屬有三院:一曰臺院,侍御史隸焉;二曰殿院,殿中
 侍御史隸焉;三曰察院,監察御史隸焉。凡祭祀、朝會,則率其屬正百官之班序。咸平四年,以御史二人充左右巡使;分糾不如法者。文官,右巡主之,武官,左巡主之;分其職掌,糾其違失,常參班簿、祿料、假告皆主之。祭祀則兼監祭使,掌受誓戒致齋,檢視糾劾。又有廊下使,專掌入閣監食;又有監香使,掌國忌行香,二使臨時充。通稱曰五使。元豐正官名,於是使名悉罷。



 御史大夫宋初不除正員,止為加官。檢校官帶憲銜,
 有至檢校御史大夫者。元豐官制行,亦並除去。



 中丞一人,為臺長,舊兼理檢使。凡除中丞而官未至者,皆除右諫議大夫權。熙寧五年,以知雜御史鄧綰為中丞,初除諫議大夫,王安石言礙近制,止以綰為龍圖閣待制權,御史中丞不遷諫議大夫自綰始。九年,鄧潤甫自正言知制誥為中丞,以宰相屬官不可長憲府,於是復遷右諫議大夫權。元豐五年,以承議郎徐禧為知制誥權中丞。禧言:「中丞糾彈之任,赴舍人院行詞,疑若未
 安。」會官制行,罷知制誥職,乃以本官試中丞。南渡初除官最多,隆興後被擢浸少。淳熙十年,始除黃洽,又三年再除蔣繼周。臺諫例不兼講讀,神宗命呂正獻,亦止命時赴講筵。中興兼者二人,萬俟離、羅汝楫皆以秦檜意。慶元後,司諫以上無不預經筵者矣。



 侍御史一人,掌貳臺政。



 殿中侍御史二人,掌以儀法糾百官之失。凡大朝會及朔望、六參,則東西對立,彈其失儀者。



 監察御史六人,掌分察六曹及百司之事,糾其謬誤,大事則奏劾,小事則舉正。迭監祠祭。歲詣三省、樞密院以下輪治。凡六察之事,稽其多寡當否,歲終條具殿最,以詔黜陟。百官應赴臺參謝辭者,以拜跪、書札體驗其老疾。凡事經郡縣、監司、省曹不能直者,直牒閣門,上殿論奏。官卑而入殿中監察御史者,謂之「里行」。治平四年,中丞王陶言:「奉詔舉臺官,而才行可舉者多以資淺不應格。」乃詔舉三任以上知縣為里行。熙寧二年詔:「御史闕,
 委中丞奏舉,毋拘官職高下兼權。」三年,孫覺薦秀州軍事推官李定,對稱旨,為太子中允權監察御史裏行,由選人為御史自定始。於是知制誥宋敏求、蘇頌、李大臨以定資淺,封還詞頭,不草制,相繼罷去。



 元豐八年,裁減察官兩員,餘許盡兼言事。紹聖二年復置。



 元祐元年,詔臺諫官許二人同上殿。又令六曹差除更改事,畫黃到,即報臺。又改六察旬奏為季奏。四年,詔:「應臺察事已彈舉而稽違窬月者,遇赦不得原減。」元符二年詔吏部:「守令課績最
 優者關臺考察,不實者重行黜責。」崇寧二年,都省申明:「臺官職在繩愆糾謬,自宰臣至百官,三省至百司,不循法守,有罪當劾,皆得糾正。」政和六年,詔在京職事官與外任按察官,雖未升朝,並赴臺參謝辭。七年,中丞王安石奏:「以本臺覺察彈奏事刊為一書,殿中侍御史以上錄本給付。」從之。



 靖康元年,監察御史胡舜陟言:「監察御史自唐至本朝,皆論政事、擊官邪,元豐、紹聖著於甲令,崇寧大臣欲其便己,遂更成憲。乞令本臺增入監察御
 史言事之文。」詔依祖宗法。又詔宰執不得薦舉臺諫官。舊《臺令》,御史上下半年分詣三省、樞密院點檢諸房文字,輪詣尚書六曹按察;奉行稽違,付受差失,咸得彈糾。渡江後,稍闊不舉。紹興三年,始復其舊。是年十一月,殿中侍御史常同言:「元豐始置六察,上自諸部、寺監,下至廩庫、場務,無不分隸,以詔廢置。而乃有寅緣申請,乞不隸臺察者,恐非法意,宜遵舊制。」從之。乾道二年詔:「自今非曾經兩任縣令,不得除監察御史。」慶元二年,侍御史
 黃黼言:「監察御史高宗時嘗置六員,孝宗時置三員,今分按之任止二人,乞增置一員。」自後常置三員。



 檢法一人,掌檢詳法律。主簿一人,掌受事發辰,勾稽簿書。宋初置推直官二人,專治獄事。凡推直有四:曰臺一推,曰臺二推,曰殿一推,曰殿二推。咸平中,置推勘官十員。元豐官制行,定員分職,里行、推直等官悉罷。紹興初,詔檢法、主簿特令殿中侍御史奏闢。紹熙中,侍御史林大中以論事不合去,所奏闢檢法官李謙、主簿彭龜年
 亦乞同罷。嘉定元年,劉矩除檢法官,範之柔除主簿,以後二職皆闕。乾道並省吏額,前司主管班次二人,正副引贊官二人,入品知班三人,知班五人,書令史四人,驅使官四人,法司二人,六察書吏九人,貼司五人,通引官三人。



 三京留司御史臺管勾臺事各一人,舊曰判臺。以朝官以上充。掌拜表行香,糾舉違失。令史二人,知班、驅使倌、書吏各一人,中興以後不置。



 秘書省監少監丞各一人,監掌古今經籍圖書、國史實錄、天文歷數之事,少監為之貳,而丞參領之。其屬有五:著作郎一人,著作佐郎二人,掌修纂日歷;秘書郎二人,掌集賢院、史館、昭文館、秘閣圖籍,以甲、乙、丙、丁為部,各分其類;校書郎四人,正字二人,掌校仇典籍,判正訛謬,各以其職隸於長貳。惟日歷非編修官不預。歲於仲夏曝書,則給酒食費,尚書、學士、侍郎、待制、兩省諫官、御史並
 赴。遇庚伏,則前期遣中使諭旨,聽以早歸。大典禮,則長貳預集議。所以待遇儒臣,非他司比。宴設錫予,率循故事。



 宋初,置三館長慶門北,謂之西館。太平興國初,於升龍門東北,創立三館書院。三年,賜名崇文院,遷西館書貯焉。東廊為集賢書庫,西廊分四部,為史館書庫。大中祥符八年,創外院於右掖門外。天禧初,令以三館為額,置檢討、校勘等員。檢討以京朝官充,校勘自京朝、幕職至選人皆得備選。以內侍二人為勾當官,通掌三館圖
 籍事,孔目官、表奏官、掌舍各一人。又有監書庫內侍一人。兼監秘閣圖籍孔目官一人。



 秘閣系端拱元年就崇文院中堂建閣,以三館書籍真本並內出古畫墨跡等藏之。淳化元年,詔次三館置直閣、以朝官充。



 校理,以京朝官充。以諸司三品、兩省五品以上官一人判閣事。直閣、校理通掌閣事,掌繕寫秘閣所藏。供御人、裝裁匠十二人。元豐五年,職事官貼職悉罷,以崇文院為秘書省官屬,始立為定員,分案四,置
 吏八。



 崇文院,太平興國三年置。端拱元年,建秘閣於院中。昭文館、史館、集賢院皆沿唐制立名,但有書庫寓於崇文院廡下。三館、秘閣、崇文院各置貼職官。又有集賢殿修撰、直龍圖閣、校勘,通謂之館職。初,英宗謂輔臣曰:「館閣所以育雋材,比選數人出使,無可者,豈乏材耶?」歐陽修曰:「今取材路狹,館閣止用選人編校書籍,故進用稍遲。」上曰:「卿等各舉數人,雖親戚世家勿避。」於是宰相琦、公亮,參知政事修、概各薦五人,未及試,神宗登極,先召十人試以詩賦,而開封府界提點陳汝義別以奏封稱旨預試。於是御史吳申言:「試館職者請策以經史及世務,毋用辭賦。」遂詔:「自今試館職專用策論。」熙寧二年,置崇文校書,始除河南府永安主簿邢恕。乃詔自今應選舉可用人並除校書,候二年取旨除館職官。五年,以隸秘書省。



 元祐初,復置直集賢院、校理。自校理而上,職有六等,內外官並許帶,恩數
 仍舊。又立試中人館職法,選人除正字,京官除校書郎。



 校書郎供職二年,除集賢校理。秘書郎、著作佐郎比集賢校理。著作郎比直集賢院、直秘閣。丞及三年除秘閣校理。三年二月,詔御試唱名日,秘書丞至正字升殿侍立。九月,復試賢良於閣下。五年,置集賢院學士並校對黃本書籍官員。紹聖初,罷校對,以編修日歷選本省,易集賢院學士為殿修撰,直院為直秘閣,集賢校理為秘書校理。十二月,詔禮部,本省長貳定校仇之課,月終具奏。



 入伏午時減半,過渡伏依舊,從蘇軾之請。



 又罷本省官任滿除館職法。元符二年,詔職事官罷帶館職,悉復元豐官制崇寧
 五年,詔館閣並除進士出身人。政和五年四月,詔秘書省殿以右文為名,改集賢殿修撰為右文殿修撰。是月,駕詣景靈宮朝獻,還幸秘書省。詔曰:「延見多士,歷覽藏書之府,祖宗遺文在焉,屋室淺狹,甚非稱太平右文之盛,宜重行修展。」八月,詔秘書省移於新左藏庫,以其地為堂。七年,詔類集所訪遺書,名曰《秘書總目》。宣和二年,立定秘書省員額:監、少監、丞並依元豐舊制,著作郎以四員為額,校書郎二員,正字四員。



 渡江後,制作未遑。紹興
 元年,始詔置秘書省,權以秘監或少監一員,丞、著作郎佐各一員,校書、正字各二員為額。續又參酌舊制,校書郎、正字召試學士院而後命之。自是採求闕文,補綴漏逸,四庫書略備。即秘書省復建史館,以修《神宗》、《哲宗實錄》,選本省官兼檢討、校勘,以侍從官充修撰。五年,效唐人十八學士之制,監、少、丞外,置著作郎佐、秘書郎各二人,校書郎、正字通十二人。又移史館於省之側,別為一所,以增重其事。九年,詔著作局惟修日歷,遇修國史則
 開國史院,遇修實錄則開實錄院,以正名實。十三年,詔復每歲曝書會。是冬,新省成,少監游操援政和故事,乞置提舉官,遂以授禮部侍郎秦熹,令掌求遺書,仍鑄印以賜。置編定書籍官二人,以校書郎、正字充。



 孝宗即位,詔館職儲養人才,不可定員。乾道九年,正字止六員;淳熙二年,監、少並置,皆前所未有。除少監、丞外,以七員為額,尋復詔不立額。紹熙二年,館職闕人,上令召試二員,謹加審擇,取學問議論平正之人。自是,監、少、丞外,多止
 除二員,是時,陳傅良上言:「請以右文、秘閣修撰並舊館閣校勘三等為史官。自校勘供職,稍遷秘閣修撰,又遷右文。在院三五年,如有勞績,就遷次對,庶幾有專官之效,無冷局之嫌。」時論韙之,然不果行。中興分案四:曰經籍,曰祝版,曰知雜,曰太史。吏額:都、副孔目官二人,四庫書直官二人,表奏官、書庫官各一人,守當官二人,正名楷書五人,守闕一人,正貼司及守闕各六人,監門官一人以武臣充,專知官一人。



 日歷所隸秘書省,以著作郎、著作佐郎掌之。以宰執時政記、左右史起居注所書會集修撰為一代之典。舊於門下省置編修院,專掌國史、實錄,修纂日歷。元豐元年詔:「宣徽院等供報修注事,自今更不供起居院,直供編修院日歷所。」四年十一月,廢編修院歸史館。官制行,屬秘書省國史案。六年,詔秘書省長、貳毋得預著作修纂日歷事,進書即擊銜,以防漏洩,如舊編修院法焉。八年,詔吏部郎中曾肇、禮部郎中林希兼著作。職事官兼職
 自此始。元祐五年,移國史案置局,專掌國史、實錄,編修日歷,以國史院為名,隸門下省,更不隸秘書省。紹聖二年,詔日歷還秘書省。宣和二年,詔罷在京修書諸局,惟秘書省日歷所系元豐國史案,除著作郎官專管修纂日歷之事無定員外,其分案編修日歷書庫官吏,並依元豐法。紹興元年,初修皇帝日歷,詔以修日歷所為名,本省長、貳通行修纂。三年,詔宰臣提舉,侍從官修撰,十一月,詔以修國史日歷所為名。四年,詔以史館為名。十
 年,詔依舊制並歸秘書省國史案,以著作郎、佐修纂,舊史館官罷歸元官。尋復詔以國史日歷所為名,續並修《神宗》、《哲宗寶訓》。隆興元年,詔編類聖政所並歸日歷所,依舊宰臣提領,仍令日歷所吏充行遣。



 會要所以省官通任其事。紹興九年,詔秘書省官仇校《國朝會要》,逐官添給茶湯錢。乾道四年,詔尚書右僕射陳俊卿兼提舉編修《國朝會要》,每遇提舉官開院過局,就本省道山堂聚呈文字,提舉諸司官、承受官、主管諸
 司官,並令國史日歷所官兼。五年,令本省再加刪定,以續修《國朝會要》為名。九年,秘書少監陳騤言:「編類建炎以後會要成書,以《中興會要》為名。」並從之。其後接續修纂,並隸秘書省。



 國史實隸院提舉國史監修國史提舉實錄院修國史同修國史史館修撰、同修撰實錄院修撰、同修撰直史館編修官檢討官校勘、檢閱、校正、編校官初,紹興三年,詔置國史院,重修《神宗》、《哲宗
 實錄》,以從官充修撰,續以左僕射呂頤浩提舉國史,右僕射朱勝非監修國史。四年,置直史館及檢討、校勘各一員。五年,置修撰官二員,校勘官無定員。是時,國史、實錄皆寓史館,未有置此廢彼之分。九年,修《徽宗實錄》,詔以實錄院為名,仍以宰臣提舉,以從官充修撰、同修撰,餘官充檢討,無定員。明年,以未修正史,詔罷史館官吏並歸實錄院。二十八年,實錄書成,詔修《三朝正史》,復置國史院,以宰臣監修,侍從官兼同修,餘官充編修。明年,
 詔國史院以宰臣提舉置修國史、同修國史共二員。編修官二員,又置都大提舉諸司官、承受官、諸司官各一員。以內侍省官充。隆興元年,以編類聖政所並歸國史院,命起居郎胡銓同修國史。二年,參政錢端禮權監修國史;乾道元年,參政虞允文權提舉國史:皆前所未有。二年,詔置實錄院,修《欽宗實錄》,其修撰、檢討官以史院官兼領。四年,實錄告成,詔修《欽宗正史》。以右僕射蔣芾提舉《四朝國史》,詔增置編修官二員,續又增置三員。淳
 熙三年,特命李燾以秘書監權同修國史、權實靈院同修撰。四年,罷實錄院,專置史院。十五年,《四朝國史》成書,詔罷史院,復開實錄院修《高宗實錄》。慶元元年,開實錄院修纂《孝宗實錄》。六年,詔實錄院同修撰以四員、檢討官以六員為額。嘉泰元年,開實錄院修纂《光宗實錄》。二年,復開國史院,自是國史與實錄院並置矣。實錄院吏兼行國史院事,點檢文字一人,書庫官八人,楷書四人



 太史局掌測驗天文,考定歷法。凡日月、星辰、風雲、氣候、
 祥眚之事,日具所占以聞。歲頒歷於天下,則預造進呈。祭祀、冠昏及大典禮,則選所用日。其官有令,有正,有春官、夏官、中官、秋官、冬官正,有丞,有直長,有靈臺郎,有保章正。其判局及同判,則選五官正以上業優考深者充。保章正五年、直長至令十年一遷,惟靈臺郎試中乃遷,而挈壺正無遷法。其別局有天文院、測驗渾儀刻漏所,掌渾儀臺晝夜測驗辰象。



 鐘鼓院,掌文德殿鐘鼓樓刻漏進牌之事。



 印歷所,掌雕印歷書。南渡後,並同隸秘書省,長、貳、丞、郎輪季點檢。



 算學元豐七年,詔四選命官通算學者,許於吏部就試,其合格者,上等除博士,中、次為學諭。元祐元年初,議者謂:「本監雖準朝旨造算學,元未興工,其試選學官亦未有應格。竊慮徒有煩費,乞罷修建。」崇寧三年,遂將元豐算學條制修成敕令。五年,罷算學,令附於國子監。十一月,從薛昂請,復置算學。大觀三年,太常寺考究,以黃帝
 為先師,自常先、力牧至周王樸以上從祀,凡七十人。四年,以算學生並入太史局。宣和二年,詔並罷官吏。



 殿中省監少監監、丞各一人,監掌供奉天子玉食、醫藥、服御、幄帟、輿輦、舍次之政令,少監為之貳,丞參領之。凡總六局:日尚食,掌膳羞之事;日尚藥,掌和劑診候之事;曰尚醞,掌酒醴之事;曰尚衣,掌衣服冠冕之事;曰尚舍,掌次舍幄帟之事;曰尚輦,掌輿輦之事。



 六尚各有典御二人,奉御六人或四人,監門二人或一人。又尚食有膳工,尚藥有醫師,尚醞有酒工,尚衣有衣徒,尚舍有幕士,
 尚輦有正供等,皆分隸其局。



 又置提舉六尚局及管幹官一員。舊殿中省判省事一人,以無職事朝官充。雖有六尚局,名別而事存,凡官隨局而移,不領於本省。所掌唯郊祀、元日、冬至天子御殿,及禘袷後廟、神主赴太高,供具傘扇;而殿中監視秘書監,為寄祿官而已。元豐中,神宗欲復建此官,而度禁中未有其地,但詔御輦院不隸省寺,令專達焉。初,權太府卿林顏因按內藏庫,見乘輿服御雜貯百物中,乃乞復殿中省六尚,以嚴奉至尊。於是徽宗乃
 出先朝所度《殿中省圖》,命三省行之,而其法皆左正言姚祐所裁定,是歲崇寧二年也。三年,蔡京上修成《殿中省六尚局供奉庫務敕令格式》並《看詳》凡六十卷,仍冠以「崇寧」為名。政和元年,殿中省高伸上編定《六尚供奉式》。靖康元年,詔六尚局並依祖宗法。又詔:「六尚局既罷。格內歲貢品物萬數,尚為民害,非祖宗舊制,其並除之。」



 御藥院勾當官無常員,以入內內侍充。掌按驗秘方、以時劑和藥品,以進御及供奉禁中之用。



 舊制,勾當御藥院遷官至遙領
 團練、防御者,謂之暗轉,干冒恩澤,浸不可止。嘉祐五年,詔御藥院內臣如當轉出而特留者,俟其出,計所留歲月優遷之,更不許累計所遷資序。非勾當御藥院而留者,其出更不推恩。



 典八人,藥童十一人,匠七人。崇寧二年,並入殿中省。



 尚衣庫使副使舊曰內衣庫,大中祥符三年改。監官二人,以內侍、三班充,掌駕頭服御傘扇之名物。凡御殿、大禮前一日,請乘輿袞冕、鎮圭、袍服於禁中以待進御,事已,復還內庫。典一人,匠四人,掌庫十人。



 內衣物庫在文德殿後,太平興國二年,置受納匹段庫,受納綾、錦,西川鹿胎、綾、羅、絹、匹段。大中祥符元年並入。



 監官二人,以京朝官並內侍充,舊三人,以諸司使、副及三班、內侍充。



 掌受納錦綺、綾羅、色帛、銀器、腰束帶料。造年支,準備衣服,以待頒賜諸王、宗室、文武近臣禁軍將校時服,並給宰臣、親王、皇親、使相生日器幣,兩府臣僚、百官、皇親轉官中謝、朝辭特賜,及大遼諸外國人使辭見銀器、射弓、衣帶。典八人,掌庫三十一人。



 新衣庫在太平坊。



 監官二人,以諸司使副、三班及內侍充。掌受錦綺、雜帛、衣服之物,以備給賜及邦國儀注之用,並受納衣服以賜諸司丁匠、諸
 軍。監門二人,以三班使臣充。典十人,掌庫五十五人。



 朝服、法物庫太平興國二年置,後分三庫:「一在天安殿後,一在右掖門內北廊,一在正陽門外。



 監官二人,以諸司使、副及三班、內侍充,掌百官朝服、諸司儀仗之名物。典三人,掌庫三十人。已上崇寧二年並入殿中省。



 舊有裁造院、針線院、雜賣場,後省並之。



 太常寺卿少卿丞各一人博士四人主簿、協律郎、奉禮郎、太祝各一人卿掌禮樂、郊廟、社稷、壇壝、陵寢之事,少卿為之貳,丞參領之。禮之名有五:曰吉禮,
 曰賓禮,曰軍禮,曰嘉禮,曰兇禮。皆掌其制度儀式。祭祀有大祠,有小祠。其犧牲、幣玉、酒醴、薦獻、器服各辨其等;掌樂律、樂舞、樂章以定宮架、特架之制,祭祀享則分樂而序之。凡親祠及四孟月朝獻景靈宮、郊祀告享太廟,掌贊相禮儀升降之節。歲時朝拜陵寢,則視法式辨具以授祠官。凡祠事,差官、卜日、齋戒皆檢舉以聞。初獻用執政官,則卿為終獻用卿,則少卿為亞獻;博士為終獻;闕則以次互攝。郊祀已,頒御札則撰儀以進。宮架、鼓吹、
 警場,率前期按閱即習。餘祀及朝會、宴享、上壽、封冊之儀物亦如之。若禮樂有所損益,及祀典、神祀、爵號與封襲、繼嗣之事當考定者,擬上於禮部。凡太醫之政令,以時頒行。



 宋初,舊置判寺無常員,以兩制以上充,丞一人,以禮官久次官高者充。別置太常禮院,雖隸本寺,其實專達。有判院、同知院四人,寺與禮院事不相兼。康定元年,置判寺、同判寺,始並兼禮院事。元豐正名,始專其職。公案五,置吏十有一。無祐三年,詔太常寺置長貳,他寺
 監則互置。紹聖中,復舊制。大觀元年,應太常寺所被旨及施行典禮事,季輪博士銓次成籍,以備討論。政和四年令,祠事監察御史闕,則以六曹郎官及館職攝充。宣和三年,令本寺因革禮五年一檢,舉接續編修。建炎初,並省冗職,惟太常、大理不並。詔太常少卿一員兼宗正少卿,罷丞、簿,惟置博士一員。紹興三年,復置丞。九年,臣僚言:「元豐正名,太常主議論者博士四人,乞參稽舊典,添置博士,以稱朝廷搜補闕軼、緝熙彌文之意。」詔添博
 士一員。十年,置簿一員。十五年,詔太常討論置籍田令,續置太社令。隆興元年,並省博士一員,主簿一員,又以光祿寺並歸太常,罷丞。明年,詔丞、簿並依舊制。



 分案九:曰禮儀,掌討論大慶典禮、神祠道釋、襲封定謚、檢舉忌辰。曰祠祭,掌大中小祠祀差行事官並酒齊、幣帛、蠟燭、禮料。曰壇廟,掌行室壇、廟域、陵寢。曰大樂,掌大樂教習樂舞、鼓吹、警場。曰法物,掌給納朝、祭服。曰廩犧,掌歲中祠祭牲牢羊豕滌室。曰太醫,掌臣僚陳乞醫人,補充太
 醫助教等。曰掌法,曰知雜,並掌本寺條制雜務。裁減吏額,贊引使二人,正禮直官二人,副禮直官二人,正名贊者七人,守闕贊者七人,私名贊者七人,胥吏一人,胥佐四人,貼司一人,書表司一人,祠祭局供官十二人,祭器司供官十人,樂正三人,鼓吹令一人,本寺天樂祭器庫專知官一人、庫子二人,圓壇大樂禮器庫專知官一人、庫子一人。



 博士掌講定五禮儀式,有改革則據經審議。凡於法
 應謚者,考其行狀,撰定謚文。有祠事,則監視儀物,掌凡贊導之事。



 主簿,掌稽考簿書。



 協律郎,掌律、呂以和陰陽之聲,正宮架、特架樂舞之位。大祭祀享宴用樂,則執麾以詔作止之節,舉麾、鼓柷而樂作;偃麾、戛敔而樂止。凡樂,掌其序事。



 奉禮郎,掌奉幣帛授初獻官,大禮則設親祠板位。



 太祝,掌讀冊辭,授摶黍以嘏告,飲福則進爵,酌酒受其虛爵。



 郊社會,掌巡視四郊及社稷。



 壇壝,掌凡掃除之事,祭祀則省牲。



 太廟令,掌宗廟
 薦新七祀及功臣從享之禮。



 籍田令,掌帝籍耕耨出納之事,植五穀蔬果,藏冰以待用。



 宮闈令,率其屬以汛灑廟庭,凡修治潔除之事。



 提點管幹郊廟祭器所南郊太廟祭器庫提點朝服法物庫所朝服法物庫南郊什物庫太廟什物庫掌藏其器服,以待祭祀、朝會之用。凡冠服,視其等而頒於執事之臣。



 教坊及鈐轄教坊所掌宴樂閱習,以待宴享之用,考其藝而進退之。



 諸陵祠墳所掌先世后妃之墳園而以時獻
 享。



 太醫局有丞,有教授,有九科醫生額三百人。歲終則會其全失而定其賞罰。太醫局,熙寧九年置,以知制誥熊本提舉,大理寺丞單驤管幹。後詔勿隸太常寺,置提舉一、判局二,判局選知醫事者為之。科置教授一,選翰林醫官以下與上等學生及在外良醫為之。學生常以春試,取合格者三百人為額。太學、律學、武學生、諸營將士疾病,輪往治之。各給印紙,書其狀,歲終稽其功緒,為三等第補之:上等月給錢十五千,毋過二十人;中等十千,毋過三十人;下等五千,毋過五十人。失多者罰黜之。受兵校錢物者,論如監臨強乞取法。三學生原預者聽受,而禁邀求者。又官制行,隸太常禮部,自政和以後,隸醫學,詳見《選舉志》。孝宗隆興元年,省並醫官而罷局生。續以虞允文請,依舊存留醫學科,逐舉附試省試
 別試所,更不置局,權令太常寺掌行。紹熙二年,復置太醫局,局生以百員為額,餘並依未罷局前體例,仍隸太常寺。



 大晟府以大司樂為長,典樂為貳。次曰大樂令,秩比丞。次曰主簿、協律郎。又有按協聲律、制撰文字、運譜等官,以京朝官、選人或白衣士人通樂律者為之。又以武臣監府門及大樂法物庫,以侍從及內省近侍官提舉。所典六案;曰大樂,曰鼓吹,曰宴樂,曰法物,曰知雜,曰掌法。國朝禮、樂掌於奉常。崇寧初,置局議大樂;樂
 成,置府建官以司之,禮、樂始分為二。五年二月,因省冗員,並之禮官;九月,復舊。大觀四年,以官徒廩給繁厚,省樂令一員、監官二員,吏祿並視太常格。宣和二年,詔以大晟府近歲添置冗濫徼幸,罷不復再置。



 宗正寺卿少卿丞主簿各一人。卿掌敘宗派屬籍,以別昭穆而定其親疏,少卿為之貳,丞參領之。凡修纂牒、譜、圖、籍,其別有五:曰玉牒,以編年之體敘帝系而記其歷數,凡政令賞罰、封域戶口、豐兇祥瑞之事載
 焉,曰各籍,序同姓之親而第其服紀之戚疏遠近。曰宗藩慶系錄,辨譜系之所自出,序其子孫而列其名位品秩。曰仙源積慶圖,考定世次枝分派別而系以本宗。曰仙源類譜,序男女宗婦族姓婚姻及官爵遷敘而著其功罪、生死。凡錄以一歲,圖以三歲、牒、譜、籍以十歲修纂以進。宋初,舊置判寺事二人,以宗姓兩制以上充,闕則以宗姓朝官以上知丞事。掌奉諸廟諸陵薦享之事,司皇族之籍。主簿一員,以京官充。舊自丞、簿以上,皆宗姓為之,通署寺事。初置卿、
 少,率命常參官判寺事。大中祥符八年,以兵部侍郎趙安易兼卿,判寺趙世長改為知寺事。九年,始定丞、郎以上兼卿,給、舍以下兼少卿,郎中以下兼丞,京官兼主簿。其卿闕,則丞以下行寺事而無知、判之名。元豐官制行,詔宗正長貳不專用國姓,蓋自有大宗正司以統皇族也。渡江後,卿不常置,少卿一人,以太常兼。紹興三年,復置少卿一人。五年,復置丞;十年,置主簿;隆興元年並省。次年,詔丞、簿復舊制。嘉定九年,詔以宗學改隸宗正寺,自此寺官又預校試之事。分案二;曰屬籍,曰知雜。吏額,胥長一人,胥史一人,胥佐二人,楷書二人,貼書二
 人。



 大宗正司景祐三年始制司,以皇兄寧江軍節度使濮王知大宗正事,皇侄彰化軍節度觀察留後守節同知大宗正事,元豐正名,仍置知及同知官各一人,選宗室團練、觀察使以上有德望者充;丞二人,以文臣京朝官以上充。掌糾合族屬而訓之以德行、道藝,受其詞訟而糾正其愆違,有罪則先劾以聞;法例不能決者,同上殿取裁。若宮邸官因事出入,日書於籍,季終類奏。歲錄存
 亡之數報宗正寺。凡宗室服屬遠近之數及其賞罰規式,皆總之。



 官屬有記室一人,掌箋奏;講書、教授十有二人,分位講授,兼領小學之事。舊制,擇宗室賢者為知大宗正事,次一人為同知;其後,位高屬尊者為判。熙寧三年,始以異姓朝臣二員知丞事,置局為睦親、廣親宅。是歲省管幹睦親、廣親宅及提舉郡、縣主等它官,以其事歸宗正。自熙寧中置丞,始以都官員外郎張稚圭為之。神宗疑用異姓,王安石言:前代宗正固有用庶姓者,乃錄春秋時公侯大夫事。神宗曰:「此雖無前代故事,行之何害?」安石曰:「聖人創法,不必皆循前代所已行
 者。」於是召稚圭對而命之。



 分案五,置吏十有一。元豐五年,詔大宗正司不隸六曹,其丞屬中書省奏差。元祐四年,詔宗室越本司訴事者罪之。六年,詔宗正按熙寧敕諸院建小學,自八歲至十四歲,首檢舉入學。紹聖元年,詔袒免外兩世孤遺貧乏者,驗實廩給之。四年,詔宗室若婦女自外還京,並報宗正,崇寧三年,詔大宗正及外宗正司將條貫事跡關宗正寺,修纂圖牒。政和三年,詔以知大宗正事仲忽提舉宗子學事。



 崇寧三年,置南外宗正司於南
 京,西外宗正司於西京,各置敦宗院,初,講議司言:「宗室疏屬原居兩京輔郡者,各置敦宗院,其兩京各置外宗正司。」從之。仍詔各擇宗室之賢者一人為知宗,掌外居宗室,詔復定宗學博士、正錄員數。大觀四年罷,政和二年復舊。又詔敦宗院宗子有文藝、行實眾所共知者,許外宗正官考察以聞。



 中興後,以位高屬尊者為判大宗正事,其知及同知如舊制。又置知大宗正丞一員,以文臣充,掌糾合宗室而檢防訓飭之。凡南班宗室磨勘、遷
 轉、襲封、請給,核其當否;嫁娶房奩、分析財產,酌厚薄多寡而訂其議。凡宗室除合該賜名外,皆大宗正定名而後報宗正寺。其餘遷授官資、支給錢米,考核以詔予奪。其不率教者以法拘之,歲久知悔,則除其過名。復直南外宗正司、西外宗正司,以處宗室之在外者。」各仍舊制設敦宗院,皆設知宗,所在通判職官兼丞、簿,其糾合、檢防、訓飭如大宗正司。西、南外兩司闕知宗,間令大宗正司選擇保明而後授之。又各置教授以課其行藝。南
 渡初,先徙宗室於江、淮,於是大宗正司移



 江寧,南外移鎮江,西外移揚州。其後屢徙,後西外止於福州,南外止於泉州;又置紹興府宗正司,蓋初隨其所寓而分管轄之。乾道七年,嘗欲移紹興府宗司於蜀,不果,後並歸行在。嘉定間,用臣僚言,乞凡除授知宗,須擇老成更練之人。詔知宗正丞照百司例每日入局所,以示增重宗盟之意。



 玉牒所淳化六年,始設局置官,詔以《皇宋玉牒》為名,建
 玉牒殿。咸平初,命趙安易、梁周翰編屬籍,始創規制。大中祥符六年,以知制誥劉筠、夏竦為修玉牒官,自後置一員或二員。元豐官制行,分隸宗正寺官。寺丞王鞏奏:「玉牒十年一進,並以學士典領。自熙寧中範鎮進書之後,《神宗玉牒》至今未修。仙源類譜自慶歷中張方平修進之後,僅五十年,並無成書。乞別立法,其修玉牒及類譜官,每二年一具草繳進。」從之。紹聖三年,應宗室賜名,三祖下各隨祖宗之支子而下,雖兄弟數多,並為一字
 相連。南渡後,紹興十二年,始建玉牒所。提舉一人或二人,以宰相執政為之,以侍從官一人兼修,宗正卿、少而下同修纂。先是,宗正寺丞邵大受奏:「講求宗正寺舊掌之書,曰皇帝玉牒,曰仙源積慶圖,曰宗藩慶系錄,曰宗支屬籍。南渡四書散失,今重加修纂《仙源慶系屬籍總要》,合圖、錄、屬籍三者而一之,既無愧於昔矣;獨玉牒一書未修,宜搜訪討論,以正九族,以壯本支。」於是始置官如舊制,分案五,置吏十。乾道八年,詔玉牒殿主管香火,
 差內侍三員、武臣一員充,並改作乾辦玉牒所殿。



 光祿寺卿少卿丞主簿各一人。卿掌祭祀、朝會、宴鄉酒醴膳羞之事,修其儲備而謹其出納之政,少卿為之貳,丞參領之。凡祭祀,共五齊、三酒、牲牢、鬱鬯及尊彞、籩豆、簠簋、鼎俎、鉶登之實,前期飭有司辦具牲鑊,視滌濯,奉牲則告充告各,共其明水火焉。禮畢,進胙於天子而頒於百執事之人。分案五,置吏十。元祐三年,詔長、貳互置。政和六年二月,監察御史王桓奏:「祭祀牢醴
 之具掌於光祿,而寺官未嘗臨視,請大祠以長貳、朔祭及中祠以丞簿監視宰割,禮畢頒胙,有故及小祠,聽以其屬攝。」從之。舊置判寺事一人,以朝官以上充。光祿卿、少,皆為寄祿。元豐制行,始歸本寺。中興後,廢並入禮部。



 太官令掌膳羞割烹之事。凡供進膳羞,則辨其名物,而視食之宜,謹其水火之齊。祭祀共明水、明火,割牲取毛血牲體,以為鼎俎之實。朝會宴享,則供其酒膳。凡給賜,視其品秩而為之等。元祐初,罷太官令。二年復置。



 崇寧
 三年,置尚食局,太官令惟掌祠事。



 法酒庫內酒坊掌以式法授酒材,視其厚薄之齊,而謹其出納之政。若造酒以待供進及祭祀,給賜,則法酒庫掌之;凡祭祀,供五齊三酒,以實尊罍。內酒坊惟造酒,以待餘用。



 太官物料庫掌預備膳食薦羞之物,以供太官之用,辨其名數而會其出入。



 翰林司掌供果實及茶茗湯藥。



 牛羊司、牛羊供應所掌供大中小祀之牲牷及太官宴享膳羞之用。



 乳酪院掌供造酥酪。



 油醋庫掌供油及鹽胾。



 外物料庫掌
 收儲米、鹽、雜物以待膳食之須。凡百司頒給者取具焉。



 衛尉寺卿少卿丞主簿各一人。卿掌儀衛兵械、甲冑之政令,少卿為之貳,丞參領之。凡內外作坊輸納兵器,則辨其名數、驗其良窳以歸於武庫,不如式者罰之。時其曝涼而封籍其數,若進御及頒給,則按籍而出之。每季委官檢視,歲終上計帳於兵部。掌凡幄帟之事,大禮設帷宮,張大次、小次,陳鹵簿儀仗。長貳晝夜巡徼,察其不如儀者,押仗官則前期稟差。凡仗衛,供羽儀、
 節鉞、金鼓、棨戟,朝宴亦如之。宴享賓客,供幕帟、茵席,視其敝者移少府、軍器監修焉。舊制,判寺事一人,以郎官以上充。凡武庫、武器歸內庫,守宮歸儀鸞司,本寺無所掌。元豐官制行,始歸本寺。分案四,置吏十。元祐三年、詔長貳互置。所隸官司十有三:內弓箭庫、南外庫、軍器弓槍庫、軍器弩劍箭庫,掌藏兵杖、器械、甲冑,以備軍國之用。儀鸞司,掌供幕帟供帳之事。軍器什物庫、宣德樓什物庫,掌收貯
 什物,給用則按籍而頒之。左右金吾街司、左右金吾仗司、六軍儀仗司,掌清道、徼巡、排列,奉引儀仗以肅禁衛。凡儀物以時修飭,選募人兵而校其遷補之事。中興後,衛尉寺廢,並入工部。



 太僕寺卿少卿丞主簿各一人。卿掌車輅、廄牧之令,少卿為之貳,丞參領之。國有大禮,供其輦輅、屬車,前期戒有司教閱象馬。凡儀仗既陳,則巡視其行列。後妃、親王、公主、執政官應給車乘者,視品秩而頒之。總國
 之馬政,籍京都坊監、畿甸牧地畜馬之數,謹其飼養,察其治療,考蕃息損耗之實,而定其賞罰焉,死則斂其□□尾、筋革入於官府。凡閱馬,差次其高下,應給賜則如格。歲終鉤覆帳籍,以上駕部。若有事於南北郊,侍中請降輿升輅,則卿授綏。舊置判寺事一人。以朝官以上充。凡邦國廄牧、車輿之政令,分隸群牧司、騏驥院諸坊監,本寺但掌天子五輅、屬車,後妃、王公車輅,給大中小祀羊。元豐官制行,始歸本寺。分案五,置吏十有八,總局十有
 二。元祐二年,詔外監事,令本寺依群牧司舊法施行;應內外馬事專隸太僕,直達樞密院,更不經尚書省及駕部。三年,詔省主簿一員。崇寧二年,詔太僕寺依舊制不治外事,歸尚書駕部;應馬事,上樞密院所隸官司。



 車輅院掌乘輿、法物,凡大駕、法駕、小駕供輦輅及奉引屬車,辨其名數與陳列先後之序。



 左、右騏驥院左、右天駟監掌國馬,別其駑良,以待軍國之用。



 鞍轡庫應奉御馬鞍勒,及以韉轡給賜臣下。



 養象所掌調御馴象。



 駝坊車營致遠務掌分養雜畜,以供負載般運。



 牧養上下監掌治療病馬及申駒數,有耗失則送皮剝所。元豐末,廢畿內牧馬監。元祐初,置左、右天廄坊,聽民間承佃牧地。紹聖元年,依元豐法置孳生監。中興後,廢太僕寺,並入兵部。



 群牧司制置使一人,景德四年置,以樞密使、副為之。至道三年,罷而復置。使一人,咸平三年置,以兩省以上官充;副便一人,以閣門以上及內侍都知充。都監二人,
 以諸司使以上充。判官二人,以京朝官充。掌內外廄牧之事,周知國馬之政,而察其登耗焉。凡受宣詔、文牒,則以時下於院、監。大事則制置使同簽署,小事則專遣其副使,都監多不備置,判官、都監每歲更出諸州巡坊監,點印國馬之蕃息者。又有左、右廂提點,隸本司。都勾押官一人,勾押官一人,押司官一人。



 鞍轡庫使副使監官二人,以諸司副使及三班使臣、內侍充。掌御馬金玉鞍勒,及給賜王公、群臣、外國
 使並國信韉轡之名物。勾管一人,典五人,掌庫十四人。元豐並入太僕寺。



\end{pinyinscope}