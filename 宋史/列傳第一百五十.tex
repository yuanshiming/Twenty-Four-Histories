\article{列傳第一百五十}

\begin{pinyinscope}

 ○周必大留正胡晉臣



 周必大,字子充,一字洪道,其先鄭州管城人。祖詵,宣和中卒廬陵,因家焉。父利建,太學博士。必大少英特,父死,鞠於母家,母親督課之。



 紹興二十年,第進士,授徽州戶
 曹。中博學宏詞科,教授建康府。除太學錄,召試館職,高宗讀其策,曰:「掌制手也。」守秘書省正字。館職復召試自此始。兼國史院編修官,除監察御史。



 孝宗踐祚,除起居郎。直前奏事,上曰:「朕舊見卿文,其以近作進。」上初御經筵,必大奏:「經筵非為分章析句,欲從容訪問,裨聖德,究治體。」先是,左右史久不除,並記注壅積,必大請言動必書,兼修月進。乃命必大兼編類聖政所詳定官,又兼權中書舍人。侍經筵,嘗論邊事,上以蜀為憂,對曰:「蜀民久
 困,願詔撫諭,事定宜寬其賦。」應詔上十事,皆切時弊。



 權給事中,繳駁不闢權幸。翟婉容位官吏轉行礙止法,爭之力,上曰:「意卿止能文,不謂剛正如此。」金索講和時舊禮,必大條奏,請正敵國之名,金為之屈。



 曾覿、龍大淵得幸,臺諫交彈之,並遷知閣門事,必大與金安節不書黃,且奏曰:「陛下於政府侍從,欲罷則罷,欲貶則貶,獨於二人委曲遷就,恐人言紛紛未止也。」明日宣手詔,謂:「給舍為人鼓扇,太上時小事,安敢爾!」必大入謝曰:「審爾,則是
 臣不以事太上者事陛下。」退待罪,上曰:「朕知卿舉職,但欲破朋黨、明紀綱耳。」旬日,申前命,必大格不行,遂請祠去。



 久之,差知南劍州,改提點福建刑獄。入對,願詔中外舉文武之才,區別所長為一籍,藏禁中,備緩急之用。除秘書少監、兼直學士院,兼領史職。鄭聞草必大制,上改竄其末,引漢宣帝事。必大因奏曰:「陛下取漢宣帝之言,親制贊書,明示好惡。臣觀西漢所謂社稷臣,乃鄙樸之周勃,少文之汲黯,不學之霍光。至於公孫弘、蔡義、韋賢,
 號曰儒者,而持祿保位,故宣帝謂俗儒不達時宜。使宣帝知真儒,保至雜伯哉?願平心察之,不可有輕儒名。」上喜其精洽,欲與之日夕論文。



 德壽加尊號,必大曰:「太上萬壽,而紹興末議文及近上表用嗣皇帝為未安。按建炎遙拜徽宗表,及唐憲宗上順宗尊號冊文,皆稱皇帝。」議遂定。趙雄使金,齎國書,議受書禮。必大立具草,略謂:「尊卑分定,或較等威;叔侄親情,豈嫌坐起!」上褒之曰:「未嘗諭國書之意,而卿能道朕心中事,此大才也。」



 兼權兵
 部侍郎。奏請重侍從以儲將相,增臺諫以廣耳目,擇監司、郡守以補郎官。尋權禮部侍郎、兼直學士院,同修國史、實錄院同修撰。



 一日,詔同王之奇、陳良翰對選德殿,袖出手詔,舉唐太宗、魏徵問對,以在位久,功未有成,治效優劣,苦不自覺,命必大等極陳當否。退而條陳:「陛下練兵以圖恢復而將數易,是用將之道未至;擇人以守郡國而守數易,是責實之方未盡。諸州長吏,倏來忽去,婺州四年易守者五,平江四年易守者四,甚至秀州一
 年而四易守,吏奸何由可察,民瘼何由可蘇!」上善其言,為革二弊。江、湖旱,請捐南庫錢二十萬代民輸,上嘉之。



 兼侍講,兼中書舍人。未幾,辭直學士院,從之。張說再除簽書樞密院,給事中莫濟封還錄黃,必大奏曰:「昨舉朝以為不可,陛下亦自知其誤而止之矣。曾未周歲,此命復出。貴戚預政,公私兩失,臣不敢具草。」上批:「王曮疾速撰入。濟、必大予宮觀,日下出國門。」說露章薦濟、必大,於是濟除溫州,必大除建寧府。濟被命即出,必大至豐城
 稱疾而歸,濟聞之大悔。必大三請祠,以此名益重。



 久之,除敷文閣待制兼侍讀、兼權兵部侍郎、兼直學士院。上勞之曰:「卿不迎合,無附麗,朕所倚重。」除兵部侍郎,尋兼太子詹事。奏言:「太宗儲才為真宗、仁宗之用,仁宗儲才為治平、元祐之用。自章、蔡沮士氣,卒致裔夷之禍。秦檜忌刻,逐人才,流弊至今。願陛下儲才於閑暇之日。」



 上日御球場,必大曰:「固知陛下不忘閱武,然太祖二百年天下,屬在聖躬,願自愛。」上改容曰:「卿言甚忠,得非虞銜橛
 之變乎?正以仇恥未雪,不欲自逸爾。」升兼侍讀,改吏部侍郎,除翰林學士。



 久雨,奏請減後宮給使,寬浙郡積逋,命省部議優恤。內直宣引,論:「金星近前星,武士擊球,太子亦與,臣甚危之。」上俾語太子,必大曰:「太子人子也,陛下命以驅馳,臣安敢勸以違命,陛下勿命之可也。」



 乞歸,弗許。上欲召人與之分職,因問:「呂祖謙能文否?」對曰:「祖謙涵養久,知典故,不但文字之工。」除禮部尚書兼翰林學士,進吏部兼承旨。詔禮官議明堂典禮,必大定圜丘
 合宮互舉之議。被旨撰《選德殿記》及《皇朝文鑒序》。必大在翰苑幾六年,制命溫雅,周盡事情,為一時詞臣之冠。或言其再入也,實曾覿所薦,而必大不知。



 除參知政事,上曰:「執政於宰相,固當和而不同。前此宰相議事,執政更無語,何也?」必大曰:「大臣自應互相可否。自秦檜當國,執政不敢措一辭,後遂以為當然。陛下虛心無我,大臣乃欲自是乎?惟小事不敢有隱,則大事何由蔽欺。」上深然之。久旱,手詔求言。宰相謂此詔一下,州郡皆乞振濟,
 何以應之,約必大同奏。必大曰:「上欲通下情,而吾儕阻隔之,何以塞公論。」



 有介椒房之援求為郎者,上俾諭給舍繳駁,必大曰:「臺諫、給舍與三省相維持,豈可諭意?不從失體,從則壞法。命下之日,臣等自當執奏。」上喜曰:「肯如此任怨耶?」必大曰:「當予而不予則有怨,不當予而不予,何怨之有!」上曰:「此任責,非任怨也。」除知樞密院。上曰:「每見宰相不能處之事,卿以數語決之,三省本未可輟卿也。」



 山陽舊屯軍八千,雷世方乞止差鎮江一軍五千,
 必大曰:「山陽控扼清河口,若今減而後增,必致敵疑。揚州武鋒軍本屯山陽者,不若歲撥三千,與鎮江五千同戍。」郭杲請移荊南軍萬二千永屯襄陽,必大言:「襄陽固要地,江陵亦江北喉襟。」於是留二千人。上諭以「金既還上京,且分諸子出鎮,將若何?」必大言:「敵恫疑虛喝,正恐我先動。當鎮之以靜,惟邊將不可不精擇。」



 拜樞密使。上曰:「若有邊事,宣撫使惟卿可,他人不能也。」上諸軍升差籍,時點召一二察能否,主帥悚激,無敢容私。創諸軍點
 試法,其在外解發而親閱之。池州李忠孝自言正將二人不能開弓,乞罷軍。上曰:「此樞使措置之效也。」金州謀帥,必大曰:「與其私舉,不若明揚。」令侍從、管軍薦舉。或傳大石林牙將加兵於金,忽魯大王分據上京,邊臣結約夏國。必大皆屏不省,勸上持重,勿輕動。既而所傳果妄。上曰:「卿真有先見之明。」



 淳熙十四年二月,拜右丞相。首奏:「今內外晏然,殆將二紀,此正可懼之時,當思經遠之計,不可紛更欲速。」秀州乞減大軍總制錢二萬,吏請勘
 當,必大曰:「此豈勘當時耶?」立蠲之。封事多言大臣同異,必大曰:「各盡所見,歸於一是,豈可尚同?陛下復祖宗舊制,命三省覆奏而後行,正欲上下相維,非止奉行文書也。」



 高宗升遐,議用顯仁例,遣三使詣金。必大謂:「今昔事殊,不當畏敵曲徇。」止之。賀正使至,或請權易淡黃袍御殿受書,必大執不可,遂為縞素服,就帷幄引見。十五年,思陵發引,援熙陵呂端故事,請行,乃攝太傅,為山陵使。明堂加恩,封濟國公。



 十一月,留身乞去,上獎勞再三。忽
 宣諭:「比年病倦,欲傳位太子,須卿且留。」必大言:「聖體康寧,止因孝思稍過,何遽至倦勤。」上曰:「禮莫大於事宗廟,而孟饗多以病分詣;孝莫重於執喪,而不得自至德壽宮。欲不退休,得乎?朕方以此委卿。」必大泣而退。十二月壬申,密賜紹興傳位親札。辛卯,命留身議定。二月壬戌,又命預草詔,專以奉幾筵、侍東朝為意。拜左丞相、許國公。參政留正拜右丞相。壬子,上始以內禪意諭二府。二月辛酉朔,降傳位詔。翼日,上吉服御紫宸殿。必大奏:「陛
 下巽位與子,盛典再見,度越千古。顧自今不得日侍天顏。」因哽噎不能言,上亦泫然曰:「正賴卿等協贊新君。」



 光宗問當世急務,奏用人、求言二事。三月,拜少保、益國公。李巘草二相制,抑揚不同,上召巘令帖麻改定,既而斥巘予郡。必大求去。



 何澹為司業,久不遷,留正奏選之。澹憾必大而德正,至是為諫長,遂首劾必大。詔以觀文殿大學士判潭州。澹論不已,遂以少保充醴泉觀使。判隆興府,不赴,復除觀文殿學士、判潭州,復大觀文。坐所舉
 官以賄敗,降滎陽郡公。復益國公,改判隆興,辭,除醴泉觀使。



 寧宗即位,求直言,奏四事:曰聖孝,曰敬天,曰崇儉,曰久任。慶元元年,三上表引年,遂以少傅致仕。



 先是,布衣呂祖泰上書請誅韓侂胄,逐陳自強,以必大代之。嘉泰元年,御史施康年劾必大首唱偽徒,私植黨與,詔降為少保。自慶元以後,侂胄之黨立偽學之名,以禁錮君子,而必大與趙汝愚、留正實指為罪首。



 二年,復少傅。四年,薨,年七十有九。贈太師,謚文忠。寧宗題篆其墓碑曰「
 忠文耆德之碑。」



 自號平園老叟,著書八十一種,有《平園集》二百卷。嘗建三忠堂於鄉,謂歐陽文忠修、楊忠襄邦乂、胡忠簡銓皆廬陵人,必大平生所敬慕,為文記之,蓋絕筆也。一子,綸。



 留正,字仲至,泉州永春人。六世祖從效,事太祖,為清遠軍節度使,封鄂國公。紹興十三年,第進士,授南恩州陽江尉、清海軍節度判官。



 龔茂良守番禺,正言:「在法:劫盜臟滿五貫死,海盜加等。小民餌利,率身陷重闢。請鏤梓
 海上,使戶知之。」民始知避。用茂良薦,赴都堂審察。宰相虞允文奇之,薦於上。得對,正言:「國家右文而略武備,祖宗以天下全力用於西夏,承平日久,邊不為備,至敵人長驅而不能支。今當改轍,使文武並用。」孝宗嘉嘆,書札中要語下三省施行。



 知循州,陛辭,言:「士大夫名節不立,國家緩急無所倚仗。靖康金人犯闕,死義者少,因亂謀利者多。今欲恢復,當崇尚名節。」上益喜,明日諭輔臣:「留正奏事,議論耿耿,可與職事官。」除軍器監簿,歷官考功
 郎官。太常謚葉義問「恭簡」,正覆謚,言:「義問將兵出疆,不知敵人情偽,及金犯邊,督視寡謀,幾至敗事。」下太常更議,時論韙之。



 擢起居舍人,尋權中書舍人。光宗自東宮朝,顧見正,謂左右曰:「修整如此,其人可知。」乃請於上,兼太子左諭德。正言:「記注進御,非設官本意。乞自今免奏御。」詔從之。



 為中書舍人兼侍講,兼權兵部侍郎,除給事中。張說子薦往視鎮江戰艦,挾勢游觀,沉舟溺卒,除知閣門事、樞密副承旨,正封對還詞頭。洪邦直除御史,正言:「
 邦直為邑人所訟,不宜任風憲。」



 兼權吏部尚書,言:「用人莫先論相。陛下志在恢復,而相位不能任輔贊。望精選人才,與圖大計。」時相益不樂,以顯謨閣直學士出知紹興府。



 侍御史範仲芑劾前帥臟六十萬,有詔核責。正明其非辜,御史怒,並劾正,降顯謨合待制、提舉玉隆萬壽宮。尋復職。知贛州,奏減上供米,不報。及為相,蠲一萬八千石。知隆興府。



 進龍圖閣直學士、四川制置使,兼知成都府。平四蜀折租價,歲減酒課三十八萬。乾道初,羌酋
 奴兒結越大渡河,據安靜砦,侵漢地幾百里。正密授諸將方略,擒奴兒結以歸,盡俘其黨,羌平。進敷文閣學士,尋詔赴行在。正在蜀以簡素化民,歸裝僅書數簏,人服其清。



 除端明殿學士、簽書樞密院事,參知政事,同知樞密院事。孝宗密諭內禪意,拜右丞相。一日奏事,皇太子參決侍立,上顧謂太子曰:「留正純誠可托。」



 光宗受禪,主管左右春坊姜特立隨龍恩擢知閣門事,聲勢浸盛。正列其招權預政狀,乞斥逐,上意猶未決。會副參闕,特立
 謁正曰:「上以丞相在位久,欲遷左相,葉翥、張枃當擇一人執政,未知孰先?」正奏之,上大怒,詔特立提舉興國宮。孝宗聞之,曰:「真宰相也。」



 紹熙元年,進左丞相。正謹法度,惜名器,豪發不可干以私。引趙汝愚首從班,卒與之共政。用黃裳為皇子嘉王翊善,世號得人。嘉王感疾,正言:「陛下只有一子,隔在宮墻外非便,乃令蚤正元良之位,入居東宮,則朝夕相見甚順。」又奏:「太子,天下本。《傳》曰:『豫建太子,所以重宗廟社稷』。漢文帝即位,即建太子。本朝
 皇子居塚嫡,有未出閣而正儲位者。皇子嘉王既居塚嫡,出合已久,宜早正儲位,以定天下本。」再月不報。檢《漢文帝紀》及本朝真宗立仁宗典故,並呂誨、張方平兩奏,節其要語繳奏。



 上不豫,外議洶洶,正與同列間至福寧殿奏事,處分得宜,人情以安。進封申國公。上疾浸平,正乞歸政,不許。



 初,正帥蜀,慮吳氏世將,謀去之。至是,朝廷議更蜀帥,正言:「西邊三將,惟吳氏世襲兵柄,號為『吳家軍』,不知有朝廷。」遂以戶部侍郎丘崈行。及吳挺死,韓侂
 胄為吳氏地,使吳曦世襲。正力請留曦環衛,遣張詔代挺。後數歲,曦入蜀,卒稔變。



 壽皇聖政成,進少保,封衛國公。李端友以椒房親,手詔除郎,正繳還,上不納,復執奏曰:「昔館陶公主為子求郎,明帝不許。今端友依憑內援,恐累聖德。」姜特立除浙東副總管,尋召赴行在,正引唐憲宗召吐突承璀事,乞罷相。上批:「成命已行,朕無反汗,卿宜自處。」正待罪六和塔,奏言:「陛下近年,不知何人獻把定之說,遂至每事堅執,斷不可回。天下至大,機務至
 煩,事出於是,則人無異詞,可以固執;事出於非,則眾論紛起,必須惟是之從。臣恐自此以往,事無是非,陛下壹持把定之說,言路遂塞。」因繳進前後錫齎及告敕,待罪範村,乞歸田里,不許。



 壽聖太后將以冬至上尊號冊寶,以正為禮儀使,攝太傅。於是上遣左司徐誼諭旨,正復入都堂視事。是行也,待罪凡一百四十日。冊寶禮成,拜不傅,封魯國公。正力辭。



 五年正月,孝宗疾革,正數請車駕過宮。一日,上拂衣起,正引裾泣諫,隨至福寧殿門。正
 退上疏,言極激切。六月戊戌,孝宗崩,光宗以疾未能執喪,正率同列屢奏,乞早正嘉王儲位,又擬指揮付學士院降詔。尋有手詔:「朕歷事歲久,念欲退閑。」正得之始懼,請對,復不報。即出國門,上表請老,末曰:「願陛下速回淵鑒,追悟前非,漸收人心,庶保國祚。」



 正始議以上疾未克主喪,宜立皇太子監國;若終喪未倦勤,當復闢。設議內禪,太子可即位。時從臣鄭湜奏與正同。既而趙汝愚以內禪請於憲聖,正謂:「建儲詔未下,遽及此,他日必難處。」
 論既違,以肩輿逃去。及嘉王即位,尊皇帝為太上皇帝,以正為大行攢宮總護使,寧宗即位。入謝,復出。憲聖命速宣押,時汝愚亦以為請,上親札,遣使召正還。



 侍御史張叔椿請議正棄國之罰,乃徙叔椿吏部侍郎,而正復相。入賀,且請車駕一出,慰安都人心;及定壽康宮南向,撤去新增禁旅。詔悉從之。進少傅,屢辭不拜,奏言:「陛下勉徇群情,以登大寶,當遇事從簡,示天下以不得已之意,實非頒爵之時。」



 韓侂胄浸謀預政,數詣都堂,正使省
 吏諭之曰:「此非知閣日往來之地。」侂胄怒而退。會經筵晚講賜坐,正執奏以為非,上不懌。侍御史黃度論馬大同罪,正擬度補外,上知其情,除度右正言。正請推恩隨龍人,上曰:「朕未見父母,可恩及下人耶?」積數事失上意,侂胄從而間之。八月,手詔正以少師、觀文殿大學士判建康府。尋又以諫議大夫張叔椿言,落職。



 慶元元年六月,詔正以上皇付正手詔八字進入,宣付史館。復觀文殿大學士。



 初,劉德秀自重慶入朝,未為正所知,謁正客
 範仲黼請為言,正曰:「此人若留之班行,朝廷必不靜。」乃除大理簿,德秀憾之。至是為諫議大夫,論正四大罪,褫職,自是彈劾無虛歲。以張釜言,責授中大夫、光祿卿,分司西京,邵州居住。明年,令自便。給事中謝源明封還錄黃,量移南劍州,再許自便。



 復光祿大夫、提舉洞霄宮。上章乞納祿,詔復元官職致仕。又以御史林採言,依舊官光祿大夫致仕。俄復觀文殿學士、金紫光祿大夫。嘉泰元年,進封魏國公,復少師、觀文殿大學士。開禧二年七
 月,薨,年七十八。贈太師。



 正出處大致如紹熙去國,恥與姜特立並位而待罪近郊,五月復入,議者猶惜其去之不勇。首發大議,蚤正嘉王儲位,遂致言者深文,指為棄國,豈弘毅有所不足耶?或問範仲黼:「留、趙二公處變不同如何?」仲黼曰:「趙,同姓之卿也;留則異姓之卿,反復之而不聽,則去。」聞者以為名言。



 有《詩文》、《奏議》、《外制》二十卷行於世。寶慶三年,謚忠宣。子恭、丙、端,皆為尚書郎。孫元英,工部侍郎;元剛,起居舍人。



 胡晉臣,字子遠,蜀州人。登紹興二十七年進士第,為成都通判。制置使範成大以公輔薦諸朝,孝宗召赴行在。入對,疏當今士俗、民力、邊備、軍政四弊。試學士院,除秘書省校書郎,遷著作佐郎兼右曹郎官。



 輪對,論三事:一,無忽講讀官,以仁宗為法;二,責諫官以糾官邪,責宰相以抑奔競;三,廣聽納、通下情,以銷未形之患。又極論近幸,上覽奏色動。晉臣口陳甚悉,至論及兩稅折變,天威稍霽,首肯久之。



 趙雄時秉政,手詔下中書問近幸姓名。
 晉臣翼日至中書,執政詰其故,晉臣曰:習招權,丞相豈不知之?」即條具大者以聞。上感悟,自是近習嚴憚。



 晉臣以親年高,求外補,知漢州,除潼川路提點刑獄,以憂去。服除再召,以五事見,曰:「選將帥,廣常平,治渠堰,更銓法,通楮幣。上謂輔臣曰:「胡晉臣言可行。」



 除度支郎,累遷侍御史。朱熹除兵部郎官,以病足未供職。侍郎林慄與熹論《易》不合,因奏熹不即受印為傲慢。晉臣上疏留熹而排慄,物論歸重。



 光宗嗣位,遷工部侍郎,除給事中,每
 以裁濫恩、惜名器為重,內降持不下,上嘉其有守,拜端明殿學士、簽書樞密院事。正謝日,上命條上軍政利害。既而朝重華宮,孝宗謂曰:「嗣君擢任二三大臣,深愜朕意,聞外庭亦無異詞。」晉臣拜謝。



 除參知政事兼同知樞密院事。上自南郊後久不御朝,晉臣與丞相留正同心輔政,中外帖然。其所奏陳,以溫凊定省為先,次及親君子、遠小人、抑僥幸、消朋黨,啟沃剴切,彌縫縝密,人無知者。未幾,薨於位,贈資政殿學士,謚文靖。



 論曰:謀大事,決大議,非凝定有立者不能也。周必大、留正一時俱以相業稱,然必大純篤忠厚,能以善道其君,光、寧禪受之際,懼禍而去,其可為有立乎哉?若胡晉臣爭論朱熹,則侃侃有守者也。



\end{pinyinscope}