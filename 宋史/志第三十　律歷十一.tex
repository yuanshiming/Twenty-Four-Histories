\article{志第三十 律歷十一}

\begin{pinyinscope}

 步交會



 交終分:三十二萬七千三百六十一、秒九千九百四十四。



 交終日:二十七、餘二千五百五十一、秒九千九百四十四。



 交中日:一十三、餘七千二百九十、秒九千九百七十二。



 朔差日:二、餘三千八百三十一、秒五十六。



 望策:一十四、餘九千二百六、秒五千。



 後限日:一、餘一千九百一十五、秒五千二十八。



 前限日:一十二、餘五千三百七十五、秒四千九百四十四。



 以上秒母同一萬。



 交率:一百八十三。



 交數:二千三百三十一。



 交終度:三百六十三分七十六。



 交中度:一百八十一分八十八。



 交象度:九十分九十四。



 半交象度:四十五分四十七。



 陽曆食限:四千九百,定法四百九十。



 陰曆食限:七千九百,定法七百九十。



 求天正十一月經朔加時入交泛日:置天正十一月經朔加時積分,以交終分及秒去之,不盡,滿統法為日,不
 滿為餘秒,即天正十一月經朔加時入交泛日及餘秒。



 求次朔及望加時入交泛日:置天正經朔加時入交泛日及餘秒,求次朔,以朔差加之。求望,以望策加之,滿交終日及餘秒去之。即次朔及望加時入交泛日及餘秒。若以經朔小餘減之,餘為夜半入交泛日。



 求定朔望夜半入交泛日:置經朔、望夜半入交泛日,若定朔、望大餘有進退者,亦進退交日,否則因經為定,即定朔望夜半入交泛日及餘秒。



 求次朔夜半入交泛日:置定朔夜半入交泛日及餘秒,大月加二日,小月加一日,餘皆加九千四百七十八、秒五十六,求次日,累加一日,滿交終日及餘秒去之,即次定朔及每日夜半入交泛日及餘秒。



 求朔望加時入交常日:置經朔、望入交泛日及餘秒,以其朔、望入盈縮限朏朒定數朏減朒加之,即朔、望加時入交常日及餘秒。



 求朔望加時入交定日:置其朔、望入轉朏朒定數,以交
 率乘之,交數而一,所得,以朏減朒加入交常日及余秒,滿與不足,進退其日,即朔、望加時入交定日及餘秒。



 求月行入陰陽曆:置其朔、望入交定日及餘秒,在交中已下為月行陽曆;已上去之,餘為月行陰曆。



 求朔望加時月行入陰陽曆積度:置月行入陰陽曆日及餘秒,以統法通日,內餘,九而一為分,分滿百為度,即朔望加時月行入陰陽曆積度及分。



 求朔望加時月去黃道度:置入陰陽曆積度及分,如交
 象已下為入少象;已上,覆減交中度,餘為入老象。皆列於上,下列交中度,相減相乘,進位,如一百三十八而一,為泛差。又視入老、少象度,如半交象已下為初;已上去之,餘為末。皆二因,退位,初減末加泛差,滿百為度,即朔、望加時月去黃道度及分。



 求日月食甚定餘:置定朔小餘,如半統法已下,與半統法相減相乘,如三萬六千九十而一為時差,以減。如半統法已上減去半統法,餘亦與半統法相減相乘,如一
 萬八千四十五而一為時差,午前以減,午後以加,皆加、減定朔小餘,為日食甚小餘。與半法相減,餘為午前、後分。其月食者,以定望小餘為月食甚小餘。



 求日月食甚辰刻:各置食甚小餘,倍之,以辰法除之為辰數,不滿,五因,滿刻法而一為刻,不滿為分。其辰數命子正,算外,即食甚辰刻及分。若加半辰,即命起子初。



 求氣差:置其朔盈、縮限度及分,自相乘,進二位,盈初、縮末一百九十七而一,盈末、縮初二百一十九而一,皆用
 減四千一十,為氣泛差。以乘午前、後分,如半晝分而一,所得,以減泛差,為定差。春分後,交初以減,交中以加;秋分後,交初以加,交中以減。如食在夜,反用之。



 求刻差:置其朔盈、縮限度及分,與半周天相減相乘,進二位,二百九而一,為刻泛差。以乘午前、後分,如三千七百半而一,為定差。冬至後午前、夏至後午後,交初以加,交中以減。冬至後午後、夏至後午前,交初以減,交中以加。



 求日入食限交前後分:置朔入交定日及餘秒,以氣、刻、
 時三差各加減之,如交中日已下為不食;已上去之,如後限已下為交後分;前限已上覆減交中日,餘為交前分。



 求日食分:置交前後分,如陽曆食限已下為陽曆食定分;已上,用減一萬二千八百,餘為陰曆食定分。如不足減者,日不食。



 各如定法而一為大分,不盡,退除為小分。小分半已上為半強,已下為半弱。命大分以十為限,即得日食之分。



 求日食泛用分:置日食定分,退二位,列於上,在陽曆列九十八於下,在陰曆列一百五十八於下,各相減相乘,陽以二百五十而一,陰以六百五十而一,各為日食泛用分。



 求月入食限交前後分:置望月行入陰陽曆日及餘秒,如後限已下為交後分。前限已上覆減交中日,餘為交前分。



 求月食分:置交前後分,如三千七百已下,為食既;已上,
 覆減一萬一千七百,不足減者為不食。



 餘以八百而一為大分,不盡,退除為小分。小分半已上為半強,已下為半弱。命大分以十為限,即得月食之分。



 求月食泛用分:置望交前、後分,自相乘,退二位,交初以一千一百三十八而一,用減一千二百三,交中以一千二百六十四而一,用減一千八十三,各為月食泛用分。



 求日月食定用分:置日月食泛用分,以一千三百三十七乘之,以定朔、望入轉算外轉定分而一,所得,為日月
 食定用分。



 求日月食虧初復滿小餘:置日月食甚小餘,以定用分減之,為虧初;加之,為復滿:即各得所求小餘。若求辰刻,依食甚術入之。



 求月食更籌法:置望辰分,四因,退位,為更法;五除之,為籌法。



 求月食入更籌:置虧初、食甚、復滿小餘,在晨分已下加晨分,昏分已上減去昏分,皆以更法除之為更數,不盡,
 以籌法除之為籌數。其更、籌數命初更,算外,即各得所入更、籌。



 求日月食甚宿次:置朔、望之日晨前夜半黃道日度及分,以統法約日月食甚小餘,加之,內月食更加半周天,各依宿次去之,即日月食甚所在宿次。



 求月食既內外刻分:置月食交前、後分,覆減三千七百,如不足減者,為食不既。



 退二位,列於上,下列七十四,相減相乘,進位,如三十七而一,所得以定用分乘之,如泛用分而一,
 為既內分;以減定用分,餘為既外分。



 求日月帶食出入所見之分:各以食甚小餘與日出、入分相減,餘為帶食差。其帶食差在定用分已上,為不帶食出入。



 以乘所食之分,滿定用分而一,若月食既者,以既內分減帶食差,餘乘所食之分,如既外分而一,所得,以減既分,如不足減者,為帶食既出入。



 以減所食之分,餘為帶食出、入所見之分。



 求日食所起:日在陽曆,初起西南,甚于正南,復滿東南;日在陰曆,初起西北,甚於正北,復滿東北。其食八分已
 上者,皆起正西,復滿正東。此據午地而論之,當審黃道斜正可知。



 求月食所起:月在陽曆,初起東北,甚於正北,復滿西北;月在陰曆,初起東南,甚于正南,復滿西南。其食八分已上者,皆起正東,復滿正西。此據午地而論之,當審黃道斜正可知。



 步五星



 五星曆策:一十五度、約分二十一、秒九十。



 木星周率:四百七十九萬八千五百二十六、秒九十二。?周日:三百九十八、余一萬五百八十六、秒九十二。



 歲差:一百一十六、秒七十二。



 伏見度:一十三半。



 木星盈縮曆



 火星周率:九百三十八萬二千五百六十、秒七十六。



 周日:七百七十九、余一萬一千一百九十、秒七十六。



 歲差:一百一十六、秒一十三。



 伏見度:一十八。



 火星盈縮曆



 土星周率:四百五十四萬八千四百三十一、秒八十五。



 周日:三百七十八、餘一千九十一、秒八十五。



 歲差:一百一十六、秒三十。



 伏見度:一十六半。



 土星盈縮曆



 金星周率:七百二萬四千三百二十一、秒三十四。



 周日:五百八十三、余一萬八百三十一、秒三十四。



 歲差:一百一十六、秒六十九。



 伏見度:一十一半。金星盈縮曆



 水星周率:一百三十九萬四千二、秒七。



 周日:一百一十五、余一萬五百五十二、秒七。



 歲差:一百一十六、秒四十。



 夕見晨伏度:一十五。



 晨見夕伏度:二十一。



 水星盈縮曆



 求五星天正冬至後平合中積中星:置天正冬至氣積
 分,各以其星周率去之,不盡,用減周率,余滿統法約之為度,不滿,退除為分秒,命之為平合中積。因而重列之為平合中星,各以前段變日加平合中積,又以前段變度加平合中星,其經退行者即減之,各得五星諸變中積中星。



 求五星入曆:各以其星歲差乘所求積年,滿周天分去之,不盡,以統法約之為度,不滿,退除為分秒,以減平合中星,為平合入曆度及分秒。求諸變者,各以前段限度
 累加之,為五星諸變入曆度及分秒。



 求五星諸變盈縮定差:各置其星其變入曆度及分秒,如半周天已下為盈,已上去之為縮。以五星曆策度除之為策數,不盡,為入策度及分秒。以其策下損益率乘之,如曆策而一為分,分滿百為度,以損益其下盈縮積度,即五星諸段盈縮定差。



 求五星平合及諸變定積:各置其星其變中積,以其段盈縮定差盈加縮減之,即其段定積日及分。以天正冬
 至大餘及約分加之,滿統法去之,不盡,命甲子,算外,即定日辰及分。



 求五星諸變入所在月日:各置其星其變定積,以天正閏日及約分加之,滿朔策及約分除之為月數,不盡,為入月已來日數。命月數起天正十一月,算外,即其星其段入其月經朔日數及分。乃以其朔日、辰相距,即所在月、日。



 求五星平合及諸變加時定星:各置其星其變中星,以
 盈縮定差盈加縮減之,內金倍之,水三之,然後加減,即五星諸段定星。以天正冬至加時黃道日度加時命之,即其星其段加時所在宿度及分秒。五星皆因留為後段初日定星,餘依術算。



 求五星諸變初日晨前夜半定星:各以其段初行率乘其段加時分,百約之,以順減退加其日加時定星,即為其星其段初日晨前夜半定星。加命如前,即得所求。



 求諸變日率度率:各以其段日辰距至後段日辰為其
 段日率;以其段夜半定星與後段夜半定星相減,余為其段度率。



 求諸變平行分:各置其段度率,以其段日率除之,為其段平行度及分秒。



 求諸變總差:各以其段平行分與後段平行分相減,餘為泛差。並前段泛差,四因,退一位,為總差。若前段無平行分相減為泛差者,因後段初日行分與其段平行分相減,為半總差,倍之,為總差。若後段無平行分相減為
 泛差者,因前段末日行分與其段平行分相減,為半總差,倍之,為總差。其在再行者,以本段平行分十四乘之,十五而一,為總差。內金星依順段術求之。



 求初末日行分:各半其段總差,加減其段平行分,後行分少,加之為初,減之為末;後行分多,減之為初,加之為末。退行者,前段減之為初,加之為末;後段加之為初,減之為末。



 為其星其段初、末日行分。



 求每日晨前夜半星行宿次:置其段總差,減日率一以除之,為日差;累損益初日行分,後行分少,損之;後行分多,益之。



 為每日
 行度及分秒;乃順加退減其星其段初日晨前夜半定星,命之,即每日夜半星行所在宿次。



 徑求其日宿次:置所求日,減一,半之,以日差乘而加減初日行分,後行分少,減之;後行分多,加之算。



 以所求日乘之,為積度;以順加退減其星其段初日夜半宿次,即所求日夜半宿次。



 求五星合見伏行差:木、火、土三星,以其段初日星行分減太陽行分,為行差。金、水二星順行者,以其段初日太
 陽行分減星行分,為行差。金、水二星退行者,以其段初日星行分並太陽行分,為行差。內水星夕伏、晨見,直以太陽行分為行差。



 求五星定合見伏泛用積:木、火、土三星,各以平合晨疾、夕伏定積,便為定合見、伏泛用積。金、水二星各置其段盈縮定差,內水星倍之,以其段行差除之為日,不滿,退除為分,在平合夕見、晨伏者,盈減縮加定積,為定合見、伏泛用積;在退合夕伏、晨見者,盈加縮減定積,為定合
 見、伏泛用積。



 求五星定合積定星:木、火、土三星,以平合行差除其日盈縮分,為距合差日。以盈縮分減之,為距合差度。以差日、差度盈減縮加其星定合泛用積,為其星定合定積、定星。金、水二星順合者,以平合行差除其日盈縮分,為距合差日。以盈縮分加之,為距合差度;以差日、差度盈加縮積其星定合泛用積,為其星定合定積、定星。金、水二星退合者,以平合行差除其日盈縮分,為距合差日;
 以減盈縮減之分,為距合差度;以差日盈減縮加,以差度盈加縮減再定合泛用積,為其星再定合定積、定星。各以天正冬至大餘及約分加定積,滿統法去之,命甲子,算外,即得定合日辰。以天正冬至加時黃道日度加定星,依宿次去之,即得定合所在宿次。



 求五星定見伏定積:木、火、土三星以泛用積晨加、夕減一象,如半周天已下自相乘,已上,覆減一周天,餘亦自相乘,七十五而一,所得,以其星伏見度乘之,十五而一
 為差,如其段行差除之為日,不滿,退除為分,見加伏減泛用積,為其星定見、伏定積。金、水二星以行差除其日盈縮分為日,在夕見、晨伏,盈加縮減泛用積,為常用積;夕伏、晨見,盈減縮加泛用積,為常用積;如常用積在半周天已下為冬至後;已上去之,餘為夏至後。各在一象已下自相乘,已上,覆減一周天,餘亦自相乘,冬至後晨、夏至後夕,以十八而一;冬至後夕、夏至後晨,以七十五而一,所得,以其星伏見度乘之,十五而一為差,如其段
 行差除之為日,不滿,退除為分,冬至後晨見、夕伏,夏至後夕見、晨伏,以加常用積,為其星定見、伏定積;冬至後夕見、晨伏,夏至後晨見、夕伏,以減常用積,為其星定見、伏定積。加命如前,即得定見、伏日辰。



\end{pinyinscope}