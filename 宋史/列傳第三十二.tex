\article{列傳第三十二}

\begin{pinyinscope}

 李進卿子延渥楊美何繼筠子承矩李漢超子守恩郭進牛思進附李謙溥子允正姚內斌董遵誨賀惟忠馬仁瑀



 李進卿,並州晉陽人。少以驍勇隸護聖軍。晉天福中,杜重威帥師敗安重榮於宗城,進卿力戰有功,擢為興順軍校。周祖開國,命領所部兵戍靈壽,久之,遷龍捷指揮使。顯德初,從世宗戰高平,改鐵騎指揮使,歷散員左射都校,改鐵騎及內殿直都虞候。



 宋初,領貴州刺史,三遷鐵騎左廂都指揮使,領乾州團練使。乾德初,遷控鶴左廂都指揮使,改漢州團練使。二年,轉虎捷左廂都指揮使,領澄州團練使。是歲冬,伐蜀,以進卿為歸州路行營
 步軍都指揮使,拔巫山砦,下夔、萬二州。蜀平,錄功拜侍衛親軍步軍都虞候,領保順軍節度。開寶二年,太祖親征河東,留進卿為在京都巡檢,穎州刺史常暉、淄州刺史韓光願分為河南、北巡檢。及還,改親軍馬軍都虞候。六年,遷步軍都指揮使,領靜江軍節度,卒,年五十九,贈侍中。子延渥、延信。延信至內殿崇班。



 延渥以蔭補供奉官,尋為閣門祗候,三遷至西京左藏庫使。咸平初,歷知平戎、寧邊、順安軍、保州、威虜軍鈐
 轄,又知冀州。六年,徙瀛州。



 景德初,契丹大舉擾邊,經胡盧河,逾關南,十月,抵城下。晝夜鼓噪,四面夾攻。旬日,其勢益張,唯擊鼓伐木之聲相聞,驅奚人負板秉燭乘墉而上。延渥率州兵強壯,又集巡檢史普所部乘城,發礧石巨木擊之,皆累累而墜,殺傷甚眾。翌日,契丹主與其母親鼓眾急擊,發矢如雨。延渥分兵拒守益堅,契丹遁去,死者三萬餘,傷者倍之,獲鎧甲、兵矢、竿牌數百萬,驛書以聞。賜延渥錦袍、金帶,將士緡錢,遷延渥本州團練使。
 以通判、太子中允陸元凱為國子博士,賜緋;推官李翔為太子中允;錄事參軍蔡亨為右贊善大夫;侍禁、兵馬監押王誨,殿直、貝冀同巡檢史普為內殿崇班充職如故。



 初,戍棚垂板護城才數寸許,契丹射之,矢集其上凡二百餘。及請葺城,詔取板視之,真宗頗稱其勞。又聞城守之際,陸元凱流矢中面,史普勇敢不避敵,復遷元凱屯田員外郎、普尚食副使。普尋卒,又錄其子昭度為右侍禁,昭儉為奉職。



 二年,延渥徙知邢州,歷天雄軍、貝州
 副都部署,知冀、貝、博三州。大中祥符八年入朝,以疾,連賜告,換右領軍衛大將軍,領演州團練使。明年,從其請,以左武衛大將軍致仕。天禧初,卒。子宗禹,為內殿崇班。



 楊美,並州文水人。本名光美,避太宗舊名改焉。美狀貌雄偉,武力絕人,以豪俠自任。漢乾祐中,周祖征三叛,美杖策詣軍門求見,周祖召與語,壯之,留帳下。廣順初,累遷禁軍大校,從世宗征淮南,以功擢鐵騎都指揮使,領白州刺史。



 太祖與美有舊,即位,以為內殿直都知。建隆
 三年,升青州北海縣為軍,以美為軍使。為政尚簡易,民皆德之。乾德二年,召還,北海民數百詣闕乞留,詔諭之不去,笞為首者始罷。遷馬步軍都頭。會討蜀,以美為歸州路戰棹左右廂都指揮使。蜀平,遷內外馬步軍副都軍頭,領恩州團練使。開寶二年,改領端州防禦使。六年,加都軍頭,領宣州觀察使。俄授虎捷左右廂都指揮使,領河西軍節度。會遣黨進、潘美徵太原,命美為行營馬軍都虞候。太平興國二年冬,出為保靜軍節度。三年夏,
 以疾求解官歸京師,尋醫藥,詔遣內侍與道士馬志視之。未幾,卒,年四十八,贈侍中。命中使護葬。美為人任氣好施,凡得予賜及奉祿,盡賙給親戚故舊。死之日,家無餘財,人多嘆息之。



 何繼筠,字化龍,河南人。父福進,歷事後唐至周,累官忠武、成德、天平三節度。繼筠幼時與群兒戲,必分行伍為戰陣之象。晉初,補殿直。周祖討三叛,表繼筠從行。賊平,改供奉官。廣順初,福進鎮真定,署衙內都校,嘗領偏師
 出土門,與並人戰,斬首數千級,以功領欽州刺史。契丹將高模翰率二千騎擾深、冀,以葦□伐度胡盧河。繼筠與虎捷都指揮使劉誠誨率兵拒之,至武強,獲老稚千餘人,模翰遁去。俄隨福進入朝,為內殿直都知。福進卒,起復,為濮州刺史,領兵戍靜安軍。契丹內侵,繼筠逆擊敗之,改棣州刺史。世宗征瓦橋關,命繼筠以所部兵出百井道,破並人數千眾。恭帝即位,以為西北面行營都監。



 建隆二年,升棣州為團練,以繼筠充使。三年,命為關南
 兵馬都監。乾德四年,加本州防禦使。開寶元年秋,命昭義節度李繼勛等征太原,以繼筠為先鋒部署。至渦河,與並人遇,擊走之,奪汾河橋,敗其眾於城下,獲馬五百匹,擒其將張環、石贇以獻。二年春,太祖親征晉陽,契丹來援。繼筠時屯兵陽曲縣,驛召至行在所,授以方略,命將精騎數千赴石嶺關拒契丹,謂之曰:「翌日亭午,俟卿來奏捷也。」至期,帝御北臺以俟。見一騎自北來,亟遣逆問之,乃繼筠子承睿來獻捷。生擒刺史二人,獲生口百
 餘,斬首千餘級,馬七百餘匹,器甲甚眾。初,並人恃契丹為聲援,及捷奏,太祖命以所獲首級、鎧甲示城下,並人喪氣。繼筠以功拜建武軍節度、判棣州。三年,來朝,詔賜鞍馬、戎杖,令戍邊。四年秋,來朝,疽發背。車駕幸其第,錫繼甚厚。未幾,卒,年五十一。帝親臨之,為之流涕,從容謂侍臣曰:「繼筠捍邊有功,朕不早授方鎮者,慮其數奇耳。今才領節制,果至淪沒,良可惜也。」贈侍中,賻絹五百匹,中使護喪,令以生平所佩劍及介冑同葬。



 繼筠深沉有
 智略,前後備邊二十年,與士卒同甘苦,得其死力。善揣邊情,邊人畏伏,多畫像祠之。子承矩。



 承矩字正則。幼為棣州衙內指揮使,從繼筠討劉崇,擒其將胡澄以獻。開寶四年,授閑廄副使。太平興國三年,漳、泉陳洪進納士,詔承矩乘傳監泉州兵。會仙游、莆田、百丈寇賊嘯聚,承矩與喬維嶽、王文寶討平之,以功就遷閑廄使。疏為政之害民者數十事上之,悉被容納。會改使名,即為崇儀使。五年,知河南府。時調丁男百十輩
 轉送上供綱,承矩以為橫役,奏罷其事。徙知潭州,凡六年,囹圄屢空,詔嘉獎之。入為六宅使。端拱元年,領潘州刺史,命護河陽屯兵。



 米信知滄州,以其不習吏事,命承矩知節度副使,實專郡治。時契丹撓邊,承矩上疏曰:「臣幼侍先臣關南征行,熟知北邊道路、川源之勢。若於順安砦西開易河蒲口,導水東注于海,東西三百餘里,南北五七十里,資其陂澤,築堤貯水為屯田,可以遏敵騎之奔軼。俟期歲間,關南諸泊悉壅闐,即播為稻田。其緣
 邊州軍臨塘水者,止留城守軍士,不煩發兵廣戍。收地利以實邊,設險固以防塞,春夏課農,秋冬習武,休息民力,以助國經。如此數年,將見彼弱我強,彼勞我逸,此御邊之要策也。其順安軍以西,抵西山百里許,無水田處,亦望選兵戍之,簡其精銳,去其冗繆。夫兵不患寡,患驕慢而不精;將不患怯,患偏見而無謀。若兵精將賢,則四境可以高枕而無憂。」太宗嘉納之。



 屬霖雨為災,典者多議其非便。承矩引援漢、魏至唐屯田故事,以折眾論,務
 在必行。乃以承矩為制置河北緣邊屯田使,俾董其役。事具《食貨志》。由是自順安以東瀕海,廣袤數百里,悉為稻田,而有莞蒲蜃蛤之饒,民賴其利。



 淳化四年,擢為西上閣門使、知滄州,逾年,徙雄州。御書印紙錄其功最,仍賜以弓劍。承矩推誠御眾,同其甘苦。邊民有告機事者,屏左右與之款接,無所猜忌,故契丹動息皆能前知。至道元年,契丹精騎數千夜襲城下,伐鼓縱火,以逼樓堞。承矩整兵出拒,遲明,列陣酣戰久之,斬馘甚眾,擒其酋
 所謂鐵林相公者,契丹遁去。是年春,府州嘗敗契丹眾,承矩條殺獲以諭州民,或揭於市,契丹愧忿,故有是役。太宗意其輕率致寇,復命與滄州安守忠兩換其任。魏廷式使河北,得雄州功狀,抗表上言。又遣內侍劉勍核實,及麾下士有功者千餘人,皆進擢繼賜。



 真宗嗣位,復遣知雄州,賜承矩詔曰:「朕嗣守鴻業,惟懷永固,思與華夷共臻富壽。而契丹自太祖在位之日,先帝繼統之初,和好往來,禮幣不絕。其後克復汾、晉,疆臣貪地,為國生
 事,信好不通。今者聖考上仙,禮當訃告。汝任居邊要,洞曉詩書,凡有事機,必能詳究,輕重之際,務在得中。」承矩貽書契丹,諭以懷來之旨,然未得其要。



 咸平二年,契丹南侵,屢遣內侍以密詔問御遏之計,密封以獻。嘗詔聽邊民越拒馬河塞北市馬。承矩上言曰:「緣邊戰棹司自淘河至泥姑海口,屈曲九百餘里,此天險也。太宗置砦一十六,鋪百二十五,廷臣十一人,戍卒三千餘,部舟百艘,往來巡警,以屏奸詐,則緩急之備,大為要害。今聽公私
 貿市,則人馬交度,深非便宜,且砦、鋪皆為虛設矣。」疏奏,即停前詔,屢被手札褒飭。三年,召還,拜引進使。州民百餘詣闕貢馬,乞借留承矩,詔書嘉獎,復遣之。承矩上言曰:



 契丹輕而不整,貪而無親,勝不相讓,敗不相救。以馳騁為容儀,以弋獵為耕釣。櫛風沐雨,不以為勞;露宿草行,不以為苦。復恃騎戰之利,故頻年犯塞。臣聞兵有三陣:日月風雲,天陣也;山陸水泉,地陣也;兵車士卒,人陣也。今用地陣而設險,以水泉而作固,建設陂塘,綿亙滄
 海,縱有敵騎,安能折沖?昨者契丹犯邊,高陽一路,東負海,西抵順安,士庶安居,即屯田之利也。今順安西至西山,地雖數軍,路才百里,縱有丘陵岡阜,亦多川瀆泉源,因而廣之,制為塘埭,自可息邊患矣。



 今緣邊守將多非其才,不悅詩書,不習禮樂,不守可疆界。制御無方,動誤國家,雖提貔虎之師,莫遏犬羊之眾。臣按兵法,凡用兵之道,校之以計而索其情,謂將孰有能,天地孰得,法令孰行,兵眾孰強,士卒孰練,賞罰孰明,此料敵制勝之道
 也。知此而用戰者必勝,否則必敗。夫惟無慮而易敵者必擒於人也。伏望慎擇疆吏,出牧邊民,厚之以奉祿,使悅其心,借之以威權,使嚴其令。然後深溝高壘,秣馬厲兵,為戰守之備。修仁立德,布政行惠,廣安輯之道。訓士卒,闢田疇,勸農耕,畜芻粟,以備兇年。完長戟,修勁弩,謹烽燧,繕保戍,以防外患。來則御之,去則備之,如此則邊城按堵矣。



 臣又聞古之明王,安集吏民,順俗而教,簡募良材,以備不虞。齊桓、晉文皆募兵以服鄰敵,故強國之
 君,必料其民有膽勇者聚為一卒,樂進戰效力以顯忠勇者聚為一卒,能逾高赴遠、輕足善鬥者聚為一卒,此三者兵之練銳,內出可以決圍,外入可以屠城。況小大異形,強弱異勢,險易異備。卑身以事強,小國之形也。以蠻夷伐蠻夷,中國之形也。故陳湯統西域而郅支滅,常惠用烏孫而邊部寧。且聚膽勇、樂戰、輕足之徒,古稱良策,請試行之。且邊鄙之人,多負壯勇,識外邦之情偽,知山川之形勝。望於邊郡置營召募,不須品度人才,止求
 少壯有武藝者萬人。俟契丹有警,令智勇將統而用之,必顯成功,乃中國之長算也。



 又如榷場之設,蓋先朝從權立制,以惠契丹,縱其渝信犯盟,亦不之廢,似全大體。今緣邊榷場,因其犯塞,尋即停罷。去歲以臣上言,於雄州置場賣茶,雖貲貨並行,而邊氓未有所濟。乞延訪大臣,議其可否,或文武中有抗執獨議,是必別有良謀。請委之邊任,使施方略,責以成功。茍空陳浮議,上惑聖聰,□氐如靈州,足為證驗,況茲契丹又非夏州之比也。



 四年
 十月,建議選銳兵於乾寧軍,挽刀魚船自界河直趣平州境,以牽西面之勢。五年,詔兼領制置屯田使。始建榷場,或者謂承矩意在繼好,然契丹無厭,未足誠信,徒使公行窺伺。會契丹有殺斥候卒者,復罷之。時契丹數窺邊城,大浚渠,頗撓其役。詔承矩握兵深入其境,以分其勢。承矩以無騎兵,第遣數千卒出混泥城,襲之而還。



 景德元年,入朝,進領英州團練使。真宗謂宰相曰:「承矩讀書好名,以才能自許,宜擇善地處之。」冬,出知澶州。承矩
 自守邊以來,嘗欲朝廷懷柔遠人,為息兵之計。及是,車駕按巡本部,卒與契丹和,益加嘆賞。韓杞之至也,命郊勞之。明年春,復知雄州。是歲,契丹始遣使奉幣。承矩以朝廷待邊人之禮悠久可行者,悉疏以聞。手詔嘉納,仍聽事有未盡者便宜裁處。三年,真拜雄州團練使。時邊兵稍息,農政未修。又置緣邊安撫使,命承矩為之,且詔邊民誘其復業。承矩曰:「契丹聞之,必謂誘其部屬也。」乃易詔文為水旱流民之意。王欽若時知樞密,援漢蟲達、
 周仲居改詔,請罪承矩。帝曰:「承矩任邊有功,當優假之。」第詔自今朝旨未便者,奏稟進止。



 承矩頗有識鑒,典長沙日,李沆、王旦為佐,承矩厚待之,以為有公輔器。善推步,自知冥數,乃以老疾求僻郡。詔自擇其代,承矩以李允則為請。乃授承矩齊州團練使,遣之任,至郡裁七日,卒,年六十一。特贈相州觀察使,賻錢五十萬,絹五百匹,中使護葬。



 以其子龜齡為侍禁;昌齡、九齡為殿直;遐齡為齋郎。緣邊洎涿、易州民,聞承矩卒,皆相率詣雄州發
 哀飯僧。昌齡娶齊王女太和縣主,至內殿崇班。昌齡子象中,為閣門祗候。



 李漢超,雲州雲中人。始事鄴帥範延光,不為所知。又事鄆帥高行周,亦不見親信。會周世宗鎮澶淵,漢超遂委質焉。即位,補殿前指揮使,三遷殿前都虞候。



 宋初,改散指揮都指揮使,領綿州刺史,累遷控鶴左廂都校,領恩州團練使。從平李重進,尋遷齊州防禦使兼關南兵馬都監。漢超在關南,人有訟漢超強取其女為妾及貸而
 不償者,太祖召而問之曰:「汝女可適何人?」曰:「農家也。」又問:「漢超未至關南,契丹如何?」曰:「歲苦侵暴。」曰:「今復爾耶?」曰:「否。」太祖曰:「漢超,朕之貴臣也,為其妾不猶愈於農婦乎?使漢超不守關南,尚能保汝家之所有乎?」責而遣之。密使諭漢超曰:「亟還其女並所貸,朕姑貰汝,勿復為也。不足於用,何不以告朕耶?」漢超感泣,誓以死報。在郡十七年,政平訟理,吏民愛之,詣闕求立碑頌德。太祖詔率更令徐鉉撰文賜之。



 霸州監軍馬仁瑀嘗兄事漢超,多
 自肆,擅發麾下卒入遼境,剽奪人口、羊馬,由是二將交惡。太祖慮其生變,遣中使賜漢超、仁瑀金帛,令和解之。太平興國初,遷應州觀察使、判齊州,仍為關南巡檢。二年八月,卒於屯所。太宗甚悼惜,贈太尉、忠武軍節度,中使護葬。漢超善撫士卒,與之同甘苦,死之日,軍中皆流涕。子守恩。



 守恩,少驍果善戰,有父風。初補齊州牙職。開寶二年,太祖親征太原,漢超為北面行營都監,守恩從父軍中。會
 契丹遣兵援河東,至定州西嘉山,將入土門,守恩領牙兵數千騎戰敗之。斬首三千級,獲戰馬、器甲甚眾,擒首領二十七人。隨漢超見於行在,賜戎服金帶、器幣、緡錢,太祖謂左右曰:「此稚子能若是,他日將帥才也。」漢超卒,擢為驍猛軍校,累官至隴州刺史、知靈州。與轉運使陳緯部芻糧過瀚海,為賊所邀,守恩及子廣文助教象之、隴州衙內指揮使望之、弟寄班守忠皆沒。真宗聞之震悼,特賜守恩洪州觀察使。次子祐之、順之、用之、潤之、慶
 之、成之、藏之。



 郭進,深州博野人。少貧賤,為鉅鹿富家傭保。有膂力,倜儻任氣,結豪俠,嗜酒蒲博。其家少年患之,欲圖殺進,婦竺氏陰知其謀,以告進,遂走晉陽依漢祖。漢祖壯其材,留帳下。晉開運末,契丹擾邊。漢祖建號太原。契丹主道殂,漢祖將入汴,進請以奇兵間道先趨洺州,因定河北諸郡。累遷乾、坊二州刺史。少帝即位,改磁州。



 周廣順初,移淄州。二年,吏民詣觀察使舉留。是秋,遷登州刺史。會
 群盜攻劫居民,進率鎮兵平之,部內清肅,民吏千餘人詣闕請立《屏盜碑》,許之。顯德初,移衛州。衛、趙、邢、洺間多亡命者,以汲郡依山帶河,易為出沒,伺間椎剽,吏捕之輒遁去,故累歲不能絕其黨類。進備知其情狀,因設計發擿之,數月間剪滅無餘,郡民又請立碑記其事。改洺州團練使,有善政,郡民復詣闕請立碑頌德,詔左拾遺鄭起撰文賜之。進嘗於城四面植柳,壕中種荷芰蒲□,後益繁茂。郡民見之有垂涕者,曰:「此郭公所種也。」



 建隆
 初,太祖親征澤、潞,遷本州防禦使,充西山巡檢。嘗與曹彬、王全斌入太原境,獲數千人。開寶二年,太祖親征河東,以進為行營前軍馬軍都指揮使。九年,命將征河東,以進為河東道、忻、代等州行營馬步軍都監,招徠山後諸州民三萬七千餘口。太平興國初,領雲州觀察使、判邢州,仍兼西山巡檢,賜京城道德坊第一區。



 四年,車駕將征太原,先命進分兵控石嶺關,為都部署,以防北邊。契丹果犯關,進大破之,又攻破西龍門砦,俘馘來獻,自是
 並人喪氣。時田欽祚護石嶺軍,恣為奸利諸不法事,進雖力不能禁,亦屢形於言。進武人,性剛烈,戰功高,欽祚以他事侵之,心不能甘,自經死,年五十八,欽祚以暴卒聞。太宗悼惜久之,贈安國軍節度,中使護葬。後頗聞其事。因罷欽祚內職,出為房州團練使。



 進有材幹,輕財好施,然性喜殺,士卒小違令,必寘於死,居家御婢僕亦然。進在西山,太祖遣戍卒,必諭之曰:「汝輩謹奉法。我猶貸汝,郭進殺汝矣。」其御下嚴毅若此。然能以權道任人,嘗
 有軍校自西山詣闕誣進者,太祖詰知其情狀,謂左右曰:「彼有過畏罰,故誣進求免爾。」遣使送與進,令殺之。會並人入寇,進謂誣者曰:「汝敢論我,信有膽氣。今舍汝罪,能掩殺並寇,即薦汝於朝;如敗,可自投河東。」其人踴躍聽命,果致克捷。進即以聞,乞遷其職,太祖從之。



 初,開寶中,太祖令有司造宅賜進,悉用筒瓦。有司言:舊制,非親王公主之第不可用。帝怒曰:「進控扼西山十餘年,使我無北顧憂。我視進豈減兒女耶?亟往督役,無妄言。」太平
 興國初,又賜宅一區。



 牛思進者,祁州無極人。少從軍,以膂力聞。嘗取強弓絓於耳,以手引之令滿。又負墻立,力士二人撮其乳曳之,嶷不動,軍中咸異之。太平興國四年,知平定軍,從征河東,石嶺關部署郭進卒,命思進代之。師還,以功改本州團練使。七年,授右千牛衛上將軍致仕,卒。



 李謙溥,字德明,並州盂人。性慷慨,重然諾。父蕘,後唐清泰中,晉祖鎮並門,署為參謀。天福初,為開封府推官,使
 契丹還,上言:「屈節外國,非久長策。」時晉祖方父事契丹,不悅其言,出為汝州魯山令,卒官。



 謙溥少通《左氏春秋》。從晉祖入汴,補殿直,奉使契丹。少帝即位,改西頭供奉官,漢初,遷東頭。周祖討三叛及守鄴都,謙溥往來宣密命,周祖愛之。廣順初,遷供備庫副使。世宗征劉崇,遼州刺史張乙堅壁不下,遣謙溥單騎說之,乙以城降,以功改閑廄使。師還,留為晉州兵馬都監。以偏師入河東境,頻致克捷,世宗詔褒美之。會隰州刺史孫義卒,時世宗親
 征淮南,謙溥謂節帥楊廷璋曰:「大寧,咽喉要地,不可闕守。且車駕出征,若俟報,則孤城陷矣。」廷璋即署謙溥權隰州事。至郡,亟命浚城隍,嚴兵備,凡八日,並人果以數千騎來寇。時盛暑,謙溥單衣持扇,從二小吏登城,徐步按視戰具。並人退舍,後旬餘,大發沖車攻城。謙溥募敢死士,得百餘人,短兵堅甲,銜枚夜縋出城。會廷璋兵至,合勢夾攻,掩其不及。並人大擾,悉眾遁去。追北數十里,斬首千餘級,時顯德四年也。明年五月,攻破孝義縣,以
 功領衢州刺史、監軍如故。世宗北征,召赴行在。恭帝即位,為澶州巡檢使,詔城莫州,數旬而就。改丹州刺史。



 建隆四年,移慈州,兼晉、隰緣邊都巡檢,行石州事,以興同砦為治所。冬,將有事於南郊。太祖命四路進兵,略地太原。鄭州刺史孫延進、絳州刺史沉繼深、通事舍人王睿等師出陰地,以謙溥為先鋒,會霍邑。謙溥因畫攻取之策,繼深等共沮之,延進不能用。軍還,出白璧關,次穀口,謙溥語諸將曰:「王師深入敵境,今既退軍,彼必乘我,諸
 君當備之。」諸將不答,謙溥獨令所部擐甲。俄追騎果至,延進等倉皇走穀中,獨謙溥麾兵拒之,並人引退。未幾,移隰州刺史。



 開寶元年,命李繼勛等征太原,以謙溥為汾州路都監。太祖征晉陽,為東砦都監。前軍副部署黨進遣謙溥伐木西山以給軍用,未至,聞鼓聲,乃並人逼西砦。大將趙贊御之,並眾未退,謙溥麾所部赴之。太祖遽至觀戰,怪其赴援者非精甲,問之,乃謙溥也,帝甚喜。謙溥在州十年,敵人不敢犯境。有招收將劉進者,勇力
 絕人,謙溥撫之厚,藉其死力,往來境上,以少擊眾。並人患之,為蠟丸書以間進,佯遺書道中,晉帥趙贊得之以聞。太祖令械進送闕下,謙溥詰其事,進伏請死。謙溥曰:「我以舉宗四十口保汝矣。」即上言進為並人所惡,此乃反間也。奏至,帝悟,遽令釋之,賜以禁軍都校戎帳、服具。進感激,願擊敵自效。



 開寶三年,召謙溥為濟州團練使。後邊將失律,復為晉、隰緣邊巡檢使,邊民聞之喜,爭相迎勞於道左。六年,領兵入太原,連拔七砦。八年,以疾求
 歸,肩輿抵洛,太祖遣中使領太醫就視之。至京師,疾篤,累上章辭祿,不許。明年春,卒,年六十二。太祖甚痛惜之,賻贈有加,葬事官給。



 謙溥與宣祖同里閉,弟謙升與太祖為布衣交。其母閻嘗厚待太祖,及即位,數迎入宮中,使左右掖之,不令拜,命坐飲食,話及舊故,賜繼優厚。雍熙中,太宗為許王納謙升女為夫人,以謙升為如京副使。謙溥子允則、允正,允則至寧州防禦使。從子允恭為內殿崇班、閣門祗候。



 允正字修己,以為蔭補供奉官。太平興國中,掌左藏庫,屢得升殿奏事,太宗頗記憶其舊故。雍熙中,與張平同掌三班,俄為閣門祗候。四年,遷閣門通事舍人。時女弟適許王,以居第質於宋偓,太宗詰之曰:「爾父守邊二十餘年,止有此第耳,何以質之?」允正具以奏,即遣內侍輦錢贖還,縉紳咸賦詩頌美。



 淳化中,命討戎、瀘州叛蠻。遷西上閣門副使。太宗慮京城獄囚淹系,命允正提總之。嘗請詔御史臺給開封府司錄司、左右軍巡、四排岸司印
 紙作囚簿,署禁系月日,條其罪犯,歲滿較其殿最。詔從其請。逾年,開封府上言:「京師浩穰,禁系尤眾,御史府考較之際,胥吏奔命,有妨推鞫,況無欺隱,不煩推校。」卒罷之。允正又提點左右藏,屢乘傳北面,經度邊要。五年,為衛州修河部署。會建清遠軍積石砦,命詣瀚海部分其役。還,拜西上閣門使、並州駐泊鈐轄。俄代張永德知州事,徙代州。



 咸平初,使西蜀詢訪民事,還,進秩東上閣門使,歷知鎮、莫二州。又為並、代馬步軍鈐轄。契丹擾邊,
 車駕駐大名,允正與高瓊率太原軍出土門路來會,召見便殿。所部有廣銳騎士數百,皆素練習,命允正引以入,賜緡錢。遣屯邢州,與石保吉逐遼人,遼人遁去。俄以兵會大名,復還並代。五年,合涇原儀渭、邠寧環慶兩路為一界,命王漢忠為都部署,驛召允正為鈐轄兼安撫都監,即日上道。又命與錢若水同詣洪德、懷安沿邊諸砦經度邊事,加領誠州刺史。七月,罷兩路之職,復任並代鈐轄。每錢若水按巡邊壘,即詔權蒞州事。進四方館使,代
 馬知節為鄜延部署、兼知延州,改客省使、知定州兼鎮定都鈐轄。



 大中祥符三年,累表求還。至京師,將祀汾陰,以疾難於扈從,命為河陽部署以便養。會張崇貴卒,趙德明頗逾軼,亟詔徙允正為鄜延部署,內侍密詔存諭。禮成,領河州團練使。允正頗知書,性嚴毅,疏財,喜自修飭。素病佝僂,以是罕在要近,累典邊任,多殺戮。是秋,徙知永興軍,卒,年五十一。



 姚內斌,平州盧龍人。仕契丹為關西巡檢、瓦橋關使。周
 顯德六年,太祖從世宗北征,兵次瓦橋關,內斌率眾五百人以城降。世宗以為汝州刺史,吏民詣闕舉留,恭帝詔褒之。內斌本名犯宣祖諱下一字,遂改今名。從平李筠,改虢州刺史。西夏數犯西鄙,以內斌為慶州刺史兼青、白兩池榷鹽制置使。在郡十數年,西夏畏伏,不敢犯塞,號內斌為「姚大蟲」,言其武猛也。



 初,內斌降,其妻子皆在契丹。乾德四年,子承贊密自幽州來歸。五年,幽州民田光嗣等又以內斌兒女六人間道來歸,太祖並召見,
 賜以衣服、緡錢、鞍馬,令中使護送還內斌。開寶四年,召赴闕,上待之甚厚,遣歸治所。七年春,暴得疾卒,年六十四。遣中使護喪,歸葬洛陽,常賻外,賜其子田三十頃。承贊為供奉官、閣門祗候,死於陣;承鑒至殿中丞。



 董遵誨,涿州範陽人。父宗本,善騎射,隸契丹帥趙延壽麾下,嘗以事說延壽,不能用。及延壽被執,舉族南奔。漢祖得之,擢拜隨州刺史,署遵誨隨州牙校。周顯德初,世宗北征,大將高懷德,遵誨之舅也,表遵誨從行。師次高
 平,與晉人遇。將接戰,晉兵未成列,懷德命遵誨先出奇兵擊之,晉人潰,大軍繼進,遂敗之。二年,討秦、鳳,大將韓通又表遵誨自隨。與賊戰於唐倉,先登陷陣,擒蜀招討使王鸞以獻,克秦、鳳二州。師還,錄其前後功,補東西班押班,又遷驍武指揮使。四年,從世宗征淮南,攻合肥,下之。六年,從韓通平雄、霸二州。



 太祖微時,客游至漢東,依宗本,而遵誨憑借父勢,太祖每避之。遵誨嘗謂太祖曰:「每見城上紫雲如蓋,又夢登高臺,遇黑蛇約長百尺餘,
 俄化龍飛騰東北去,雷電隨之,是何祥也?」太祖皆不對。他日論兵戰事,遵誨理多屈,拂衣而起。太祖乃辭宗本去,自是紫雲漸散。及即位,一日,便殿召見,遵誨伏地請死,帝令左右扶起,因諭之曰:「卿尚記往日紫雲及龍化之夢乎?」遵誨再拜呼萬歲。俄而部下有軍卒擊登聞鼓,訴其不法十餘事,太祖釋不問。遵誨益惶愧待罪,太祖召而諭之曰:「朕方赦過賞功,豈念舊惡耶?汝可勿復憂,吾將錄用汝。」遵誨再拜感泣。又問遵誨:「母安在?」遵誨
 奏曰:「母氏在幽州,經患難睽隔。」太祖因令人賂邊民,竊迎其母,送與遵誨。遵誨遣外弟劉綜貢馬以謝,太祖解其所服真珠盤龍衣,命繼賜之。綜曰:「遵誨人臣,豈敢當此。」太祖曰:「吾方委以方面,不此嫌也。」



 會李筠叛澤、潞,令遵誨從慕容延釗討之,遷馬軍都軍頭,因留之鎮守。三年,召歸,再遷為散員都虞候。乾德六年,以西夏近邊,授通遠軍使。遵誨既至,召諸族酋長,諭以朝廷威德,刲羊釃酒,宴犒甚至,眾皆悅服。後數月,復來擾邊,遵誨率兵深
 入其境,擊走之,俘斬甚眾,獲羊馬數萬,夷落以定。太祖嘉其功,就拜羅州刺史,使如故。太宗即位,兼領靈州路巡檢。



 遵誨不知書,豁達無崖岸,多方略,能挽強命中,武藝皆絕人。在通遠軍凡十四年,安撫一面,夏人悅服。嘗有剽略靈武進奉使鞍馬、兵器者,遵誨部署帳下欲討之。夏人懼,盡歸所略,拜伏請罪,遵誨即慰撫令去。自是各謹封略,秋毫不敢犯。歷太祖、太宗朝,委遇始終不替,許以便宜制軍事。太平興國六年,卒,年五十六。帝軫悼久
 之,遣中使護葬,賵賻加等,錄其子嗣宗、嗣榮為殿直。



 賀惟忠,忻州定襄人。少勇敢,善騎射。周祖將兵討三叛,惟忠謁於道左,自陳其有武藝,周祖悅之,即留置所部。洎開國,得隸世宗帳下,奏補供奉官,不辭,輒入朝。世宗怒之,及嗣位,終不遷擢。



 初授儀鸞副使,令知易州,捍邊有功,尋遷正使。開寶二年,太祖駐常山,以惟忠為本州刺史兼易、定、祁等州都巡檢使。嘗中流矢,六年,金瘡發而卒。太祖聞之嗟悼,即以其子昭度為供奉官。



 惟忠性
 剛果,知書,洞曉兵法,有方略。在易州繕完亭障,撫士卒,得其死力,每乘塞用兵,所向必克,威名震北邊,故十餘年間契丹不敢南牧。昭度至西京作坊使。淳化中,知通遠軍,有罪當棄市,減死流商州。



 馬仁瑀,大名夏津人。十餘歲時,其父令就學,輒逃歸。又遣於鄉校習《孝經》,旬餘不識一字。博士笞之,仁瑀夜中獨往焚學堂,博士僅以身免。常集里中群兒數十人,與之戲,為行陣之狀,自稱將軍,日與之約,鞭其後期者,群
 兒皆畏伏。又市果均給之,益相親附。及長,善射,挽弓二百斤。



 漢乾祐中,周祖鎮鄴,仁瑀年十六,願隸帳下,周祖素聞其勇,既見,甚喜,留置左右。廣順初,補內殿直。世宗嗣位,命衛士習射苑中,仁瑀弓力最勁,而所發多中,賜錦袍、銀帶。會太原劉崇入寇,世宗親征至高平,周師不利,諸將多引退。仁瑀謂眾曰:「主辱臣死,安用我輩!」乃控弦躍馬,挺身出陣射賊,斃者數十人,士氣益振,大軍乘之,崇遂敗績。世宗至上黨,諸將坐失律誅者七十餘人。
 擢仁瑀為弓箭控鶴直指揮使,及還京,又遷散指揮使。從征淮南,至楚州,攻水砦。砦中建飛樓高百尺餘,世宗觀之,相去殆二百步,樓上望卒厲聲嫚罵,世宗怒甚,命左右射之,遠莫能及。仁瑀引滿,應弦而顛。及淮南平,身被數十創,賜以良藥,遷內殿直都虞候。又從平三關。恭帝嗣位,詔從太祖北伐。



 初以佐命功授散員都指揮使,領貴州刺史,俄遷鐵騎右廂都指揮使,又為虎捷左廂都指揮使,領扶州團練使。從平澤、潞,以功領常州防禦
 使,改龍捷左廂都指揮使。建隆二年,改領岳州防禦使,俄又移領漢州。



 初,詔仁瑀等領荊湖諸郡,不數歲,復其地。至是,將征蜀,又詔領川、峽諸郡,遂平之。先是,薛居正知貢舉,仁瑀私囑所與者,榜出,無其人。聞喜宴日,仁瑀酒酣,攜所囑者詣居正切責之。為御史中丞劉溫叟所劾,帝優容之。王繼勛以後族驕恣,凌蔑將帥,人皆引避。獨仁瑀詞氣不相下,嘗攘臂欲毆之。會帝將講武郊外,遂欲相圖,各勒所部兵私市白梃。太祖密知之,詔罷講武,
 出仁瑀為密州防禦使。



 太祖征晉陽,命仁瑀率師巡邊,至上谷、漁陽。契丹素聞仁瑀名,不敢出,因縱兵大掠,俘生口、牛羊數萬計。駕還,仁瑀歸治所。明年,群盜起兗州,賊首周弼、毛襲甚勇悍,材貌奇偉,弼號曰「長腳龍」。監軍討捕數不利,詔仁瑀掩擊。仁瑀率帳下十餘卒入泰山,擒弼,盡獲其黨,魯郊遂寧。



 開寶四年,遷瀛州防禦使。兄子嘗因醉誤殺平民,系獄當死。民家自言非有宿憾,但過誤爾,願以過失殺傷論。仁瑀曰:「我為長吏,而兄子殺
 人,此怙勢爾,非過失也。豈敢以私親而亂國法哉?」遂論如律,給民家布帛為棺斂具。太平興國初,移知遼州。四年,車駕征太原,命仁瑀與成州刺史慕容超、飛龍使白重貴、八作使李繼升分兵攻城。及徵範陽,命仁瑀率禁兵擊契丹於盧龍北,契丹兵奔潰。師還,遷朔州觀察使,判瀛州事。七年,卒,年五十。贈河西軍節度,葬事官給。



 論曰:宋初,交、廣、劍南、太原各稱大號,荊湖、江表止通貢奉,契丹相抗,西夏未服。太祖常注意於謀帥,命李漢超
 屯關南,馬仁瑀守瀛州,韓令坤鎮常山,賀惟忠守易州,何繼筠領棣州,以拒北敵。又以郭進控西山,武守琪戍晉州,李謙溥守隰州,李繼勛鎮昭義,以御太原。趙贊屯延州,姚內斌守慶州,董遵誨屯環州,王彥升守原州,馮繼業鎮靈武,以備西夏。其族在京師者,撫之甚厚。郡中筦榷之利,悉以與之。恣其貿易,免其所過徵稅,許其召募亡命以為爪牙。凡軍中事皆得便宜,每來朝,必召對命坐,厚為飲食,錫繼以遣之。由是邊臣富貲,能養死士,
 使為間諜,洞知敵情;及其入侵,設伏掩擊,多致克捷,二十年間無西北之憂。以至命將出師,平西蜀,拓湖湘,下嶺表,克江南,所向遂志,蓋能推赤心以馭群下之所致也。



 若李進卿、楊美亦專師西征,而美居北海,以樂易結民心,誠得為政之本。延渥、承矩、守恩、允正皆紹先業,以勛名著。承矩議屯田,贊和好,其謀甚遠。守恩以果敢死事。宋之武功,於斯為盛焉



\end{pinyinscope}