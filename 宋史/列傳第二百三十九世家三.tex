\article{列傳第二百三十九世家三}

\begin{pinyinscope}

 吳越錢氏



 吳越錢俶,字文德,杭州臨安人。本名弘俶,以犯宣祖偏諱,去之。祖鏐,因黃巢之亂,據有吳越,昭宗授以杭、越兩
 藩節制,封彭城郡王。歷梁、後唐,加吳越國王,卒,子元瓘嗣。元瓘卒,子佐嗣。佐卒,弟倧嗣,為其大將胡進思所廢,遂迎立俶,事具《舊五代史/卷133錢鏐
 
 五代史》。俶即元瓘之第九子也,母吳越國恭懿夫人吳氏。



 晉開運中,為台州刺史。數月,有僧德詔語俶曰:「此地非君為治之所,當速歸,不然不利。」俶從其言,即求歸國,未幾,有進思之變。



 漢乾祐初,授東南面兵馬都元帥、鎮海鎮東軍節度使、開府儀同三司、檢校太師兼中書令、杭越等州大都督、吳越國王,賜號翊聖廣
 運同德保定功臣,賜以金印、玉冊。三年,江南遣其將查文徽攻福州,俶發兵擒文徽,獻捷,加尚書令。



 周廣順初,授諸道兵馬元帥。二年,授天下兵馬元帥,改賜推誠保德安邦致治忠正功臣。六月,丁母憂,起復。世宗即位,授天下兵馬都元帥。顯德三年,世宗征淮南,令俶以所部分路進討。俶遣偏將吳程圍毗陵,陷關城,擒刺史趙仁澤;路彥銖圍宣城。俄俶軍戰敗,復失常州。會李景上表求割地內附,詔俶班師。五年夏四月,杭州災,府舍悉為
 煨燼,將延及倉庾,俶命酒祝曰:「食為民天,若盡焚之,民命安仰!」火遂止。世宗聞之,遣內侍齎詔恤問。是歲,淮南內屬,遣翰林學士陶穀、司天監趙修己使俶,賜羊馬橐駝,自是以為常。七月,又遣閤門使曹彬賜俶兵甲、旗幟。六年,恭帝嗣位,賜崇仁昭德宣忠保慶扶天翊亮功臣。



 建隆元年,授天下兵馬大元帥。俶舅寧國軍節度吳延福有異圖,左右勸俶誅之,俶曰:「先夫人同氣,安忍置於法?」言訖鳴咽流涕,但黜延福於外,終全母族。自太祖受
 命,俶貢奉有加常數。二年,遣使賜俶戰馬二百、羊五千、橐駝三十。乾德元年,以白金萬兩、犀牙各十株、香藥一十五萬斤、金銀真珠玳瑁器數百事來貢,改賜承家保國宣德守道忠正恭順功臣。是冬,郊祀,遣其子惟濬入貢。



 開寶五年,改賜開吳鎮越崇文耀武宣德守道功臣,封其妻孫氏為賢德順穆夫人。未幾,遣幕吏黃夷簡入貢,上謂之曰:「汝歸語元帥,常訓練兵甲,江南強倔不朝,我將發師討之,元帥當助我,無惑人言云『皮之不存,毛
 將安傅』」。特命有司造大第於薰風門外,連亘數坊,棟宇宏麗,儲偫什物無不悉具。因召進奉使錢文贄,謂之曰:「朕數年前令學士承旨陶穀草詔,比來城南建離宮,令賜名『禮賢宅』,以待李煜及汝主,先來朝者以賜之。」詔以草示文贄,遂遣文贄賜俶戰馬及羊,諭旨於俶。



 七年五月,賜俶襲衣、玉帶、玉鞍勒馬、金器二百兩、銀器三千兩、錦綺千段。是冬,討江南。遣內客省使丁德裕齎詔,以俶為昇州東面招撫制置使,賜戰馬二百匹、旌旗劍甲;令
 德裕以禁兵步騎千人為俶前鋒,盡護其軍。李煜貽書於俶,其略曰:「今日無我,明日豈有君?一旦明天子易地酬勳,王亦大梁一布衣耳。」俶不答,以書來上。



 八年,俶率兵拔常州,加守太師,詔俶歸國。俶遣大將沈承禮等率兵水陸隨王師平潤州,遂進討金陵。上嘗召進奏使任知果,令諭旨於俶曰:「元帥克毗陵有大功,俟平江南,可暫來與朕相見,以慰延想之意。即當遣還,不久留也。朕三執圭幣以見上帝,豈食言乎?」江南平,論功以俶大將
 沈承禮、孫承祐並為節度使,為防禦使者一人、刺史六人。



 九年二月,俶與其妻孫氏、子惟濬、平江軍節度使孫承祐來朝,上遣皇子興元尹德昭至睢陽迎勞。俶將至,車駕先幸禮賢宅,按視供帳之具。及至,詔俶居之。對於崇德殿,貢白金四萬兩、絹五萬匹,賜襲衣、玉帶、金器千兩、白金器三千兩、羅綺三千段、玉勒馬。即日宴長春殿,俶又貢白金二萬兩、絹三萬匹、乳香二萬斤。賀平江左,貢白金五萬兩、錢十萬貫、綿百八十萬兩、茶八萬五千
 斤、犀角象牙二百株、香藥三百斤。車駕幸其第,又貢白金十萬兩、絹五萬匹、乳香五萬斤,以助郊祭。



 三月庚午,詔曰:「古者宗工大臣特被隆眷,或劍履上殿,或書詔不名,率由豐功,待以殊禮。今我兼其命數,用獎勳賢,輝映古今,允為優異。咨爾吳越國王錢俶,德隆宏茂,器識深遠,撫奧區於吳會,勒洪伐於宗彝。昨以江表不庭,王師致討,委方面之兵柄,克常、潤之土宇,輔翼帝室,震疊皇靈。而乃執圭來庭,垂紳就列,罄事君之誠愨,為群后之
 表儀。爰峻徽章,以旌元老。可特賜劍履上殿,書詔不名。」以俶妻賢德順穆夫人孫氏為吳越國王妃,令惟濬齎詔賜之。宰相以為異姓諸侯王妻無封妃之典,太祖曰:「行自我朝,表異恩也。」俶獻白金六萬兩、絹六萬匹為謝。



 太祖數詔俶與其子惟濬宴射苑中,惟諸王預坐。每宣諭俶,俶拜謝,多令內侍掖起,俶感泣。又嘗一日召宴,獨太宗、秦王侍坐,酒酣,太祖令俶與太宗、秦王敘昆仲之禮,俶伏地叩頭,涕泣固讓,乃止。會將以四月幸西京,親
 雩祀,俶懇請扈從,不許,留惟濬侍祠,令俶歸國。太祖宴餞於講武殿,賜窄衣、玉束帶、玉鞍勒馬、玳瑁鞭、金銀錦綵二十餘萬、銀裝兵八百事,謂俶曰:「南北風土異宜,漸及炎暑,卿可早發。」俶涕泣言願三歲一朝,太祖曰:「川陸迂遠,當俟詔旨,即來覲也。」俶將發京師,特賜導從儀衛之物,率皆鮮麗,令自禮賢宅陳列至迎春苑。自俶之至,逮於歸國,太祖所賜金器萬兩、白金器又數萬兩、白金十餘萬兩、錦綺綾羅紬絹四十餘萬匹、馬數百匹,他物不
 可勝計。俶既歸國,嘗視事功臣堂,一日命坐於東偏,謂左右曰:「西北者神京在焉,天威不違顏咫尺,俶豈敢寧居乎?」



 太宗即位,加食邑五千戶。俶貢御衣,通天犀帶,絹萬匹,金器、玳瑁器百餘事,金銀釦器五百事,塗金銀香臺、龍腦檀香床、銀假果、水晶花凡數千計,價直鉅萬;又貢犀角象牙三十株、香藥萬斤、乾薑五萬斤、茶五萬斤。俶又請歲增常貢,詔不許。太平興國二年正月,孫氏卒,遣給事中程羽吊祭。九月,上言乞所賜詔書呼名,不許。



 三年三月,來朝,遣判四方館事梁迥至泗州迎勞;惟濬先在闕下,上遣至睢陽候俶。俶先遣孫承祐入奏事,上即遣承祐護諸司供帳勞俶於郊,又命齊王廷美宴俶於迎春苑。俶至,對於崇德殿,賜襲衣、玉帶、金銀器、玉鞍勒馬、錦彩萬匹、錢千萬;賓佐崔仁冀等賜金銀帶、器幣、鞍馬有差。即日宴俶長春殿,令劉鋹、李煜預坐。俶貢白金五萬兩、錢萬萬,絹十萬匹、綾二萬匹、綿十萬,屯茶十萬斤、建茶萬斤、乾薑萬斤,越器五萬事,錦緣席千,金銀
 畫舫三、銀飾龍舟四,金飾烏樠木御食案、御床各一,金樽罍盞斝各一、金飾玳瑁器三十事、金釦藤盤二、金釦雕象俎十,銀假果樹十事、翠毛真珠花三叢,七寶飾食案十、銀樽罍十、盞斝副焉,金釦越器百五十事、雕銀俎五十,密假果、剪羅花各二十樹,銀釦大盤十,銀裝鼓二、七寶飾胡琴五弦箏各四、銀飾箜篌方響羯鼓各四、紅牙樂器二十二事,乳香萬斤、犀角象牙各一百株,香藥萬斤、蘇木萬斤。上又嘗召俶及其子惟濬宴後苑,泛舟
 池中,上手酌酒以賜俶,俶跪飲之。其恩待如此。



 四月,會陳洪進納土,俶上言曰:「臣伏有懇誠,貯於肺腑,幸因入覲,輒敢上聞。蓋虞神道之害盈,必冀天慈之從欲。臣近蒙朝廷賜以劍履上殿,詔書不名,仍以本道領募卒徒,嘗營戈甲,特建國王之號,俾增師律之嚴,皆所以假其寵名,託於鄰敵。方今幅員開外,名數洞分,豈可冒居,自罹公議?合從省罷,以正等威。除本道軍士、器甲臣已曾奏納外,其所封吳越國王及天下兵馬大元帥職名,望
 皆許解罷。凡頒詔命,願復名呼,庶聖朝無虛授之恩,微臣免疾顛之禍。」優詔不許。



 五月乙酉,俶再上表:「臣慶遇承平之運,遠修肆覲之儀,宸眷彌隆,寵章皆極。斗筲之量實覺滿盈,丹赤之誠輒茲披露。臣伏念祖宗以來,親提義旅,尊戴中京,略有兩浙之土田,討平一方之僭逆。此際蓋隔朝天之路,莫諧請吏之心。然而稟號令於闕庭,保封疆於邊徼,家世承襲,已及百年。今者幸遇皇帝陛下嗣守丕基,削平諸夏,凡在率濱之內,悉歸輿地之
 圖。獨臣一邦僻介江表,職貢雖陳於外府,版籍未歸於有司,尚令山越之民,猶隔陶唐之化。太陽委照,不及蔀家,春雷發聲,兀為聾俗,則臣實使之然也,罪莫大焉。不勝大願,願以所管十三州獻於闕下執事,其間地里名數別具條析以聞。伏望陛下念奕世之忠勤,察乃心之傾向,特降明詔,允茲至誠。」



 詔答曰:「卿世濟忠純,志遵憲度,承百年之堂構,有千里之江山。自朕纂臨,聿修覲禮,睹文物之全盛,喜書軌之混同,願親日月之光,遽忘江
 海之志。甲兵樓櫓既悉上於有司,山川土田又盡獻於天府,舉宗效順,前代所無,書之簡編,永彰忠烈。所請宜依。」



 丁亥,詔曰:「漢寵功臣,聿著帶河之誓;周尊元老,遂分表海之邦。其有奄宅勾吳,早綿星紀,包茅入貢,不絕於累朝,羽檄起兵,備嘗於百戰。適當輯瑞而來勤,爰以提封而上獻。宜遷內地,別錫爰田,彌昭啟土之榮,俾增書社之數。吳越國王錢俶天資純懿,世濟忠貞,兆積德於靈源,書大勳於策府。近者慶沖人之踐阼,奉國珍而來
 朝,齒革羽毛既修其常貢,土田版籍又獻於有司,願宿衛於京師,表乃心於王室。眷茲誠節,宜茂寵光。是用列西楚之名區,析長淮之奧壤,建茲大國,不遠舊封,載疏千里之疆,更重四征之寄。疇其爵邑,施及子孫,永夾輔於皇家,用對揚於休命,垂厥百世,不其偉歟!其以淮南節度管內封俶為淮海國王,仍改賜寧淮鎮海崇文耀武宣德守道功臣,即以禮賢宅賜之。」惟濬為節度使兼侍中,惟治為節度使,惟演為團練使,惟願暨侄郁、昱並
 為刺史,弟儀、信並為觀察使,將校孫承祐、沈承禮並為節度使。體貌隆盛,冠絕一時。



 是歲七月中元,京城張燈,令有司於俶宅前設燈山、陳聲樂以寵之。八月,令兩浙發俶緦麻以上親及管內官吏悉歸朝,凡舟一千四十四艘,所過以兵護送。杭州貢俶樂人凡八十有一人,詔以三十六人還杭州,四十五人賜俶。俶上表謝,上親畫「付中書送史館」。



 四年二月宴苑中,俶被病拜不能起,上命以銀裝肩輿送歸,因以賜之。四月,從征太原,賜羊三
 百、酒十斛。俶小心謹恪,每晨趨行闕,人未有至者,必先至,假寐以待旦。上知之,謂俶曰:「卿已中年,宜避風冷,自今入謁不須太早也。」特輟御前二大燭以賜之,令先赴前頓。上嘗賜從臣食於中路頓,並賜衛士羊臂臑、卮酒,觀其飲啖。上見其雄壯,因顧俶,俶進曰:「所謂『如虎如貔、如熊如羆』者也。」會劉繼元降,上御連城臺誅軍中先亡命太原者,顧謂俶曰:「卿能保全一方以歸於我,不致血刃,深可嘉也。」俶頓首謝。俶中途被足疾,車駕親臨問,
 令太醫然艾以灸,疾尋愈。還京策勳,宰相進擬加食邑萬戶、實封千戶。上即改白麻,倍加食邑二萬戶、實封二千戶。



 五年八月,俶被病,上臨問,賜白金萬兩、錢千萬、絹萬匹、金器千兩,賜其子惟濬、惟治白金各萬兩。是冬,車駕幸大名府,詔俶乘肩輿即路。六年,又被病,賜告久之,上遣中使賜俶文楸棋局、水精棋子,乃諭旨曰:「朕機務之餘,頗曾留意,以卿在假,可用此遣日。」



 八年十二月,上言曰:「臣以蕞爾之軀,蒙被恩寵,賦祿百萬,兼職數四。元
 帥之任實本於兵權,國王之號蓋屏於帝室,尚書總百揆之重,中書掌八柄之繁,維師冠於上台,開府當於極品,臣之孱瑣,罔克負荷。邦國之制式著等威,名器之間固有涯分,徒速罪戾,以取顛隮。伏望聖旨特從省罷。」不許。表三上,下詔曰:「 分茅胙土,所以彰世及之榮;大輅繁纓,所以表名器之重。至若褒寵勳德,度越典常,咨於舊章,爰推異數。乃有體好謙之德,形固讓之辭,敦諭再三,確乎不拔,用曲至公之論,式光知止之風。淮海國王錢
 俶方岳炳靈,風雲通感,奄有勾吳之地,不忘象魏之心。掃境來朝,舉宗宿衛,籍其土宇,入於朝廷,式昭職員,胙之淮海,居天子二老之任,啟真王萬戶之封,並加寵章,用答忠順。而乃屢形表疏,願避官榮,發於深衷,誠不可奪。若以靈臺偃伯,武庫橐兵,天下一家書軌之無外,五侯九伯征伐之不行。願寢元帥之名,勉徇由衷之請。其乃世祚明德,存於帶礪之盟;帝齎良弼,寵以台輔之任。極馭貴之爵,增衍食之封,非足酬庸,適以昭德,勉膺渥
 澤,克副眷懷。可罷天下兵馬大元帥,餘如故。」



 雍熙元年,改封漢南國王。四年春,出為武勝軍節度,改封南陽國王。俶久被病,詔免入辭。將發,賜玉束帶、金唾壺、碗盎等。俶四上表讓國王,改封許王。端拱元年春,徙封鄧王。會朝廷遣使賜生辰器幣,與使者宴飲至幕,有大流星墮正寢前,光燭一庭,是夕暴卒,年六十。



 俶以天成四年八月二十四日生,至是八月二十四日卒,復與父元瓘卒日同,人皆異之。上為廢朝七日,追封秦國王,諡「忠懿」,仍
 正衙備禮發冊曰:



 :皇帝若曰:昊穹眷祐,賢哲挺生,稟象緯之純精,負經綸之盛業,作民父母,為國翰垣。其存也冠中臺而長諸侯,其沒也峻徽章而崇禮命。咨爾故安時鎮國崇文耀武宣德守道功臣、武勝軍節度、鄧州管內觀察處置等使、開府儀同三司、守太師、尚書令兼中書令、使持節鄧州諸軍事、行鄧州刺史、上柱國、鄧王、食邑九萬七千戶、食實封一萬六千九百戶、賜劍履上殿、詔書不名錢俶,嗣祖考之令德,奠東南之奧區,開國承
 家,本仁祖義;以忠孝而保社稷,以廉讓而化人民;勤翊戴於累朝,克惠綏於一境,世傳威略,志慕聲明。



 :當武庫戢兵,洞閱詩書之府;洎秣陵問罪,雄張犄角之師。致區宇之同文,賴忠良之協力。逮於纂紹,益享崇高,蘊明哲而保身,務傾輸而竭節,盡獻土壤,來歸闕庭,予嘉乃功,薦錫殊寵。而道隆簡退,志尚謙沖,屢辭郤縠之權,難奪范宣之讓。朕深惟勳舊,俾就養頤,爰出殿於大邦,庶聿臻於眉壽,式繄元老,永輔眇躬。



 :何天道之難諶,而梁木
 之斯壞!長沙既往,空存甲令之勳;征虜云亡,但見雲臺之像。賵賻從於異等,嗟悼廢於臨朝;寧酬柱石之勳,未極君臣之分。庸加典則,以厚始終。



 :今遣使太中大夫、尚書工部侍郎、上柱國、汾陽郡開國侯、食邑一千戶、賜紫金魚袋郭贄持節冊贈爾為秦國王。嗚呼!德無不報,予敢忘於格言;魂而有知,爾尚欽於天命。嗚呼哀哉!



 命中使護其喪歸葬洛陽。自鏐至俶世有吳越之地僅百年,管內諸州皆子弟,將校授任而後請命於朝,有至使相
 者。俶任太師、尚書令兼中書令四十年,為元帥三十五年。及歸朝卒,子惟演、惟濟皆童年,召見慰勞,並起家諸衛將軍。善始令終,窮極富貴,福履之盛,近代無比。



 然甚儉素,自奉尤薄,常服大帛之衣,幃帳茵褥皆用紫絁,食不重味。頗知書,雅好吟詠。在吳越日,自編其詩數百首為《正本集》,因陶穀奉使至杭州,求為之序。性謙和,未嘗忤物。在藩日,每朝廷使至,接遇勤厚。所上乘輿、服物、器玩,製作精妙,每遣使修貢,必羅列於庭,焚香再拜,其恭
 謹如此。崇信釋氏,前後造寺數百,歸朝又以愛子為僧。善草書,上一日遣使謂曰:「聞卿善草聖,可寫一二紙進來。」俶即以舊所書絹圖上之,詔書褒美,因賜玉硯金匣一,紅綠象牙管筆、龍鳳墨、蜀箋、盈丈紙皆百數。



 屬久病家居,有黃門趙海被酒造其第求見,因出藥數丸謂俶曰:「此頗療目疾,願王即餌之。」俶即餌焉。既去,家人皆惶駭不測,俶曰:「此但醉耳,又何疑哉?」後數日,上聞大驚,捕海繫獄,決杖流海島。



 初,俶為胡進思所立,廢其兄倧,徙
 越州,資給豐厚。進思屢請除之,恐為後患,俶泣曰:「若殺吾兄,吾終不忍,汝欲行其志,吾當退避賢路。」進思慚而退。俶慮進思害倧,遣親將薛溫為倧守衛,戒之曰:「委汝以保全廢王,苟有非常,汝當以死捍之。」溫至越旬餘,有二卒夜持刃逾垣入,倧闔戶拒之,呼聲達於外,溫領徒而入,斃二卒於庭中,乃進思之所遣也。進思因憂懼,疽發背,卒。從左右屢有以倧為言,俶終拒之。倧居越州二十餘年卒。



 俶自建隆已來貢奉不絕,及用兵江左,所貢
 數十倍。先是鏐與戰士多賜己姓,後俶歸朝,皆稱同宗。淳化三年,詔令復本姓。又浙中劉氏避鏐諱,改為金氏,亦令還故。景德中,有司請以禮賢宅為司天監,真宗以先朝所賜,不許。大中祥符八年,子惟演等復表上之,詔賜錢五萬貫,仍各賜第一區。



 子惟濬、惟治、惟渲、惟演、惟灝、惟溍、惟濟。惟渲至韶州團練使,惟灝賀州團練使,惟溍至左龍武將軍、獎州刺史。惟演自有。



 惟濬,字禹川,俶嫡子也。裁數歲,俶表授鎮海鎮東兩軍
 節度副大使、檢校太保、鈐轄兩浙管內土客諸軍事。建隆元年,加檢校太傅。三年,領建武軍節度。乾德初,加檢校太尉。是年冬,來朝,因侍祠南郊。六年,復來朝,侍郊祀,命兵部員外郎、知制誥盧多遜迎勞之。開寶二年,授鎮東等軍節度、浙江東西道觀察處置、兩浙制置營田發運等使。未幾,來朝,太祖召宴苑中,令黃門奏《簫韶》樂,與諸王同席而坐。賜白玉帶、珠綴衣、水精鞍勒御馬,賜賚鉅萬計。月餘遣歸,辭日,又賜襲衣、玉帶、金鞍勒馬。四年,
 又來朝,因侍祠南郊,寵待殊等。及大兵征金陵,惟濬從父下毗陵,以功加平章事。九年,隨俶入朝,俶先歸,留惟濬扈從郊祀西洛。



 太宗即位,加兼侍中。太平興國二年,丁母妃孫氏憂,起復,加鎮東大將軍、右金吾衛大將軍,員外置同正。俶將入朝,惟濬先奉方物來貢,詔戶部郎中侯涉至泗州迎勞之,賜賚無算,並增其食邑。三年,隨俶來朝,俶盡獻浙右之地,改封淮海國王,徙惟濬淮南節度。是冬,郊祀恩,加檢校太師。從平太原及從征幽薊,
 又從幸大名。雍熙元年,郊祀,改山南東道節度。四年,徙鎮安州。惟濬雖再移鎮,常留京師。端拱初,籍田,封蕭國公。俄俶薨,起復,加兼中書令。



 惟濬與俶諸子共進錢金、綾羅、犀玉帶笏、犀角、象牙、丁香、金玉馬腦鞍勒、金玉珠翠首飾、樂器、博具、器皿什物、馬橐駝牛驢車凡數十萬計。俶妻俞氏又進金銀十餘萬、犀二十株、通犀赬犀玉帶二十二條、水晶佛像十二事。惟濬又進女樂十人,上不納,各賜錦綵三十段遣還之。淳化初,杭州以錢氏家
 廟所藏唐、梁以來累朝所賜玉冊竹冊各三副、鐵券一來上,上悉以賜惟濬。明年春,得疾暴卒,年三十七。廢朝二日,追封邠王,諡「安僖」,中使典喪事。



 子守吉、守讓。守吉至西京作坊使。



 守讓,字希仲,以蔭累遷供備庫使,天禧四年,錄諸國之後,加領榮州刺史,改東染院使,卒。守讓頗勤學為文章,退居多閉關讀書,屢獻歌頌,真宗優詔褒獎。有集二十卷。子恕,娶曹王元偁女長安縣主。



 惟治,字和世,廢王倧之長子。倧初遷於越而惟治生,俶
 愛之,養為己子。幼好讀書。八歲授兩浙牙內諸軍指揮使,判軍糧營田事,又改德化軍使,遷檢校太保、台州團練使。乾德四年四月,制授寧遠軍節度、檢校太傅,仍兼衙職,與惟濬節旄同日而至,國人榮之。



 王師討江南,惟治從俶率兵下常州,策勳改奉國軍節度。俶入朝,命惟治權發遣軍國事。俶還,令奉幣入貢,撫諭命賜甚厚。惟治又獻塗金銀香師子、香鹿鳳鶴孔雀、寶裝髹合、釦金瓷器萬事,吳繚綾千匹。辭日,賜襲衣玉帶、塗金鞍勒馬、
 金銀器、繒綵踰萬計。



 太宗嗣位,進檢校太尉。太平興國三年,俶再入覲,又權國事。一夕廄中火,惟治率兵臨高下視,令親信十數輩仗劍申令,敢後顧者斬,頃之火息。妻族有隸帳下者恃親犯法,惟治命杖背於府門。俶既納土,朝廷命考功郎中范旻知杭州,惟治奉兵民圖籍、帑廩管籥授旻,與其弟惟渲、惟灝歸朝。次近郊,遣內侍護諸司供帳迎勞至京師,即日召對長春殿,賜衣服、金帶、鞍勒馬、器幣,改領鎮國軍節度。五年八月,車駕幸俶
 第,召見惟治,賜白金萬兩。



 惟治善草隸,尤好二王書,嘗曰:「心能御手,手能御筆,則法在其中矣。」家藏書帖圖書甚眾,太宗知之,嘗謂近臣曰:「錢俶兒侄多工草書。」因命翰林書學賀丕顯詣其第,遍取視之,曰:「諸錢皆效浙僧亞栖之跡,故筆力軟弱,獨惟治為工耳。」惟治嘗以鍾繇、王羲之、唐玄宗墨跡凡七軸為獻,優詔褒答。



 雍熙三年,大出師征幽州,命惟治知真定軍府兼兵馬都部署。前一日曲宴內殿,惟治獻詩,帝覽之悅,酒半,遣小黃門密
 諭北面之寄。至則訓兵享士,頗勤政務,設廚饌於城門以待使傳。



 初,惟濬雖俶嫡嗣,然俶以其放蕩無檢,故器惟治,再俾權國務。嘗一夕俶暴疾,孫妃悉斂符籥付惟治,後惟濬知之,甚恚恨。洎入朝,惟濬止奉朝請,而委惟治藩任焉。俶薨召還,起復檢校太師。移疾就第百日,有司請罷奉,特詔續給。累上表請罷節鎮,優詔不許。



 惟治既病,心恍惚,家事不肅。咸平初,僮奴以姦私殺人於庭,事連閨閫。真宗為停按鞫,止授右監門衛上將軍,其子
 駕部員外郎丕責授郢州團練副使。晚年頗貧匱。景德中,其弟惟演獻文,上對宰相稱其公王之後,能苦心翰墨,令記其名,因曰:「錢氏繼世忠順,子孫可念,如聞惟治頗貧乏,尤可軫惻。」特轉右武衛上將軍,月給奉十萬。累加左驍衛上將軍、左神武統軍。大中祥符七年七月,卒,年六十六,贈太師。初,有司援統軍陳承昭、孟玨例,當贈東宮保傅。上以俶奉土歸國,優其贈典。又聞群臣家貧乏者不欲官給喪事,為罷詔葬。錄其四子官,及外弟、子
 婿、親校並甄擢之。



 惟治好學,聚圖書萬餘卷,多異本。慕皮、陸為詩,有集十卷。書跡多為人藏秘,晚年雖病廢,猶或揮翰。真宗嘗語惟演曰:「朕知惟治工書,然以疾不欲遣使往取,卿為求數輻進來。」翌日,寫聖製詩數十章以獻,賜白金千兩。



 初鎮四明,嘗夢神人披甲,自稱「西嶽神」,謂惟治曰:「公面有缺文」,即捧土培之。後領華州節鉞二十年。



 子丕,字簡之,幼好學。雍熙中,俶上言欲求舉進士,太宗以其世家子,特召試內署,授秘書丞,賜金紫,累遷
 駕部郎中。嘗知新淦縣,又知衡州。惟治卒,以將作少監起復,俄為三司戶部判官,卒於光祿少卿。



 惟濟,字巖夫。生七歲,俶封漢南國王,奏補本府元從指揮使,歷諸衛將軍,領恩州刺史,改東染院使,真拜封州刺史。真宗祀汾陰還,燕近臣苑中,命惟濟射,一發中的。故事,刺史射不解箭,帝賜解之,且賜襲衣、金帶。



 其後請試郡,命知絳州。民有條桑者,盜奪桑不能得,乃自創其臂,誣桑主欲殺人,久繫不能辨。惟濟取盜與之食,視之,盜
 以左手舉匕箸,惟濟曰:「以右手創人者上重下輕,今汝創特下重,正用左手傷右臂,非爾自為之邪?」辭遂服。帝聞之,謂宰相向敏中曰:「惟濟試守郡輒明辨,後必為能吏矣!」



 徙潞州。民相驚有外寇,奔城而仆者相枕藉,惟濟從容行視,從騎甚省,民乃安。遷永州團練使,改知成德軍。仁宗即位,加檢校司空。民有偽作白金質取緡錢者,其家來告,惟濟曰:「第聲言被盜,示以重購,質者當來責餘直,即得之矣。」已而果然,乃杖配之。以吉州防禦使留
 再任,遷虔州觀察使,知定州。有婦人待前妻子不仁,至燒銅錢灼臂,惟濟取婦人所生兒置雪中,械婦人往視兒死。其慘毒多此類。遷武昌軍節度觀察留後,改保靜軍留後。



 惟濟喜賓客,豐宴犒,家無餘貲,帝賜白金二千兩,所負公使錢七百餘萬。卒,贈平江節度使,諡「宣惠」。遣使護葬事,賜賻錢二百萬、絹千匹。有《玉季集》二十卷。惟濟有吏幹,能戢下而性苛忍,所至牽蔓滿獄。重囚棄市,或斷手足,探肝膽,用以威眾。觀者色動,而惟濟自若也。



 儼,字誠允,俶之異母弟也。本名信,淳化初改焉。幼為沙門,及長,頗謹慎好學。俶襲國封,命為鎮東軍安撫副使。周顯德四年,奏署衢州刺史。太祖平揚州,俶遣儼入賀,命閤門副使武懷節齎詔迎勞,賜賚甚厚。及歸,又賜玉帶、名馬、錦綵、器皿。開寶三年,代兄偡知湖州,充宣德軍安撫使。俶奉詔攻毗陵,命儼督漕運。太平興國二年,從俶之請,授新、媯、儒等州觀察使,仍知湖州,儼兄儀為慎、瑞、師等州觀察使。入朝,以儼為隨州觀察使,儀為金州
 觀察使。侍祠郊宮,特召升儼班於節度使之次。儀卒,儼換金州。常從幸天駟監,會賜從官馬,太宗敕有司曰:「錢儼儒者,宜擇馴馬給之。」未幾,出判和州,在職十七年。咸平六年,卒,年六十七,贈昭化軍節度。



 儼嗜學,博涉經史。少夢人遺以大硯,自是樂為文辭,頗敏速富贍,當時國中詞翰多出其手。歸京師,與朝廷文士遊,歌詠不絕。淳化初,嘗獻《皇猷錄》,咸平又獻《光聖錄》,並有詔嘉答。所著有前集五十卷、後集二十四卷、《吳越備史》十五卷、《備史
 遺事》五卷、《忠懿王勳業志》三卷,又作《貴溪叟自敘傳》一卷。



 善飲酒,百卮不醉,居外郡嘗患無敵,或言一軍校差可倫擬,儼問其狀,曰:「飲益多,手益恭。」儼曰:「此亦變常,非善飲也。」



 昱,字就之,忠獻王佐之長子。佐薨,昱尚幼,國人立倧,遂以昱為咸寧、大安二宮使。俶嗣國,承制授秀州刺史。太祖受禪,俶遣昱入貢,與江南使同侍宴射於後苑。江南使先中的,令昱解之,昱應弦而中,賜以玉帶。及平蜀,復
 來賀。歸國,為台州刺史。俶得福州,命昱守之。王師討江南,為東面水陸行營應援使。從俶入朝,授白州刺史。



 昱好學,多聚書,喜吟詠,多與中朝卿大夫唱酬。嘗與沙門贊寧談竹事,迭錄所記,昱得百餘條,因集為《竹譜》三卷。俄獻《太平興國錄》。求換臺省官,令學士院召試制誥三篇,改秘書監,判尚書都省。時新葺省署,昱撰記奏御,又嘗以鍾、王墨跡八卷為獻,有詔褒美。



 出知宋州,改工部侍郎,歷典壽、泗、宿三州,率無善政。至道中,郊祀,當進秩,
 太宗曰:「昱貴家子無檢操,不宜任丞郎。」以為郢州團練使。咸平二年,表入朝,以病不及陛見,卒,年五十七。



 昱善筆劄,工尺牘,太祖嘗取觀賞之,賜以御書金花扇及《急就章》。昱聰敏能覆棋,工琴畫,飲酒至斗餘不亂。善諧謔,生平交舊終日談宴,未曾犯一人家諱。有集二十卷。然貪猥縱肆,無名節可稱。生子百數。涉,雍熙中進士及第。絳,至內殿承制、閤門祗候,累典郡,頗以幹力稱。



 俶之群從又有台州刺史仰之子昭序,字著明,好學喜聚書,書
 多親寫。知通利軍,以勤幹聞,至如京副使。衢州刺史偓之子昭度,字九齡,至供奉官。俊敏工為詩,多警句,有集十卷,蘇易簡為序行於世。



 孫承祐,杭州錢塘人。俶納其姊為妃,因擢處要職,累遷浙江東道鹽鐵副使、鎮海鎮東兩軍節度副使、知靜海軍節度事。



 開寶初,隨俶子惟濬入貢,詔授光祿大夫、檢校太保、鎮東鎮海等軍行軍司馬。俶又私署中吳軍節度。七年,俶復遣承祐入貢,賜襲衣、玉帶、鞍勒馬、黃金器
 五百兩、銀器三千兩、雜綵五千匹,且令諭旨於俶,將有事於江表。及王師渡江,命內客省使丁德裕率步騎一千,詔俶以所部與德裕會攻常、潤。承祐從俶克毗陵,功居多,詔改中吳軍為平江軍,真授承祐節。太平興國中,俶來朝,盡獻其地,徙承祐泰寧軍節度使。五年,從幸大名,留知府事。雍熙二年,改知滑州,數月卒,贈太子太師,中使護葬。



 承祐在浙右日,憑藉親寵,恣為奢侈,每一飲宴,凡殺物命千數,常膳亦數十品方下箸。所居室中,爇
 龍腦日不下數兩。從車駕北征,以橐駝負大斛貯水養魚自隨。至幽州南村落間,日已旰,西京留守石守信與其子駙馬都尉保吉及近臣十數人尚未朝食,適遇承祐,即延所止幕舍中,膾魚具食,窮極水陸,人皆異之。



 承祐少時,嘗夢人以蓍草一本,增其一而授之。既寤,以語所親曰:「『大衍之數五十,其用四十有九』,今增其一,我壽止於此乎。」果五十而卒。



 子誘,至駕部郎中,出為淮南節度行軍司馬。



 沈承禮,湖州烏程人。錢鏐辟置幕府,署處州刺史。鏐子元瓘以女妻之,署為府中右職,出為台州刺史。元瓘卒,子佐嗣,以承禮掌親兵。俶襲位,命知威武軍節度事,充兩浙都鈐轄使。



 王師征江南,俶遣承禮率水陸數萬人助平毗陵,因攻潤州。城中兵夜出焚外柵,諸將皆欲馳救,承禮曰:「古人有言,擊東南而備西北者,此之謂也。」命士皆擐甲蓐食,堅壁不動。他壘不設備者悉驚擾,獨承禮所部敵人不敢窺。丹陽平,遂率兵抵建業。李煜歸朝,
 錄其功,真授福州節制。太平興國初,俶盡獻浙右地,徙承禮鎮密州。八年,卒,年六十七。廢朝二日,贈太子太師,中使護葬。



 初,秦王廷美之敗也,有司按驗,俶、惟濬、孫承祐及陳洪進皆嘗有贈遺,獨承禮無焉。



\end{pinyinscope}