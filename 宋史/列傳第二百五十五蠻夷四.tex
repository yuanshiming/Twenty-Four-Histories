\article{列傳第二百五十五蠻夷四}

\begin{pinyinscope}

 西南諸夷,
 漢訸訫郡地。武帝元鼎六年,定西南夷,置訸訫郡。唐置費、珍、莊、琰、播、郎、訸、夷等州。其地北距充州百五十里,東距辰州二千四百里,南距交州一千五百里,西距昆明九百里。無城郭,散居村落。土熱,多霖雨,稻粟皆再熟。無徭役,將戰征乃屯聚。刻木為契。其法,劫盜者,償其主三倍;殺人者,出牛馬三十頭與其家以贖死。病疾無醫藥,但擊銅鼓、銅沙鑼以祀神。風俗與東謝蠻同。隋大業末,首領謝龍羽據其地,勝兵數萬人。唐末,
 王建據西川,由是不通中國。後唐天成二年,訸訫清州刺史宋朝化等一百五十人來朝。其後孟知祥據西川,復不通朝貢。



 乾德三年,平孟昶。五年,知西南夷南寧州蕃落使龍彥曁舀等遂來貢,詔授彥曁舀歸德將軍、南寧州刺史、蕃落使,又以順化王武才為懷化將軍,武才弟若啟為歸德司階,武龍州部落王子若溢、東山部落王子若差、羅波源部落王子若台、訓州部落王子若從、雞平部落王子若冷、戰洞部落王子若磨、羅母殊部落王子
 若母、石人部落王子若藏並為歸德司戈。開寶二年,武才等一百四十人又來貢,以武才為歸德將軍。來人乞賜武才鈿函手詔,以舊制所無,不許。四年,其國人詣涪州,言南寧州蕃落使龍彥曁舀卒,歸德將軍武才及八刺史狀請以彥曁舀子漢瑭為嗣,詔授漢瑭南寧州刺史兼蕃落使。八年,三十九部順化王子若發等三百七十七人來貢馬百六十匹、丹砂千兩。



 太平興國五年,夷王龍瓊琚遣其子羅若從並諸州蠻七百四十四人以方物、
 名馬來貢。六年,保州刺史董奇死,以其子紹重繼之。雍熙二年八月,奉化王子以慈等三百五十人以方物來貢。夷王龍漢鸸自稱權南寧州事兼蕃落使,遣訸訫諸州酋長趙文橋率種族百餘人來獻方物、名馬,並上蜀孟氏所給符印。授漢鸸歸德將軍、南寧州刺史,以文橋等並為懷化司戈。端拱二年,漢鸸又貽書五溪都統向通漢,約以入貢。淳化元年,漢鸸遣其弟漢興來朝。三年,夷王龍漢興及都統龍漢曁堯、刺史龍光顯、龍光盈及順
 化王雨滯等各貢馬、朱砂。



 至道元年,其王龍漢曁堯遣其使龍光進率西南訸訫諸蠻來貢方物。太宗召見其使,詢以地裏風俗,譯對曰:「地去宜州陸行四十五日。土宜五穀,多種粳稻,以木弩射獐鹿充食。每三二百戶為一州,州有長。殺人者不償死,出家財以贖。國王居有城郭,無壁壘,官府惟短垣。」光進之說,與前書所記小異,故並敘之。上因令作本國歌舞,一人吹瓢笙如蚊蚋聲,良久,數十輩連袂宛轉而舞,以足頓地為節。詢其曲,則名曰《
 水曲》。其使十數輩,從者千餘人,皆蓬發,面目黧黑,狀如猿猱。使者衣虎皮氈裘,以虎尾插首為飾。詔授漢曁堯甯遠大將軍,封歸化王;又以歸德將軍羅以植為安遠大將軍,保順將軍龍光盈、龍光顯並為安化大將軍,光進等二十四人並授將軍、郎將、司階、司戈。其本國使從者,有甲頭王子、刺史、判官、長史、司馬、長行、傔人七等之名。



 咸平元年,其王龍漢曁堯遣使龍光腆又率訸訫諸蠻千余人來貢,詔授光腆等百三十人官。三年,都部署張文
 黔來貢。五年,漢曁堯又遣牙校率部蠻千六百人、馬四百六十匹並藥物布帛等來貢,賜冠帶於崇德殿,厚齎遣還。六年,知全州錢絳請招誘溪峒名豪,上以生事,寢其奏不報。



 景德元年,詔西南訸訫諸國進奉使親至朝廷者,今廣南西路發兵援之,勿抑其意。先是,龍光進等來朝,上矜其道遠,人馬多斃,因詔宜州自今可就賜恩物。至是,懇請詣闕,從之。二年,詔羈縻保、霸州刺史董紹重、董忠義歲賜紫綾錦袍。四年,西南蠻羅甕井都指揮使
 顏士龍等來貢。士龍種落遐阻,未嘗來朝,今始至,詔館餼賜予如高、溪州。



 大中祥符元年,瀘州言江安縣夷人殺傷內屬戶,害巡檢任賽,既不自安,遂為亂。詔遣閣門祗候侍其旭乘傳招撫。旭至,蠻人首罪,殺牲為誓。未幾,復叛。旭因追斬數十級,擒其首領三人,又以衣服糸由布誘降蠻鬥婆行者,將按誅其罪。上以旭召而殺之,違招安之實,即降詔戒止;且令篤恩信,設方略制禦,無尚討伐以滋驚擾。二年,旭言夷人恃岩險,未即歸服。詔文思
 副使孫正辭等為都巡檢使,乃分三路入其境,脅以兵威,皆震懾伏罪。三年,正辭言夷人安集,降詔嘉獎。先有蠻羅忽余甚忠順,防援井監,捕殺違命者不已。上遣內臣郝昭信褒慰之,且諭以赦蠻黨前罪,勿復邀擊。



 四年,茂州夷族首領、耆老,刑牛犬於三溪,誓不侵擾州界。又峽路鈐轄執為亂夷人王群體等至闕下,上曰:「蠻夷不識教義,向之為亂,亦守臣失於綏撫。」並免死,分隸江、浙遠地。其年,霸州董哲為其巡檢使董延早所殺。五年,黎
 洞夷人互相殺害,巡檢使發兵掩捕。上聞而切責之曰:「蠻夷相攻,許邊吏和斷,安可擅發兵甲,或致擾動?」即令有司更選可任者代之。



 六年,晏州多剛縣夷人鬥望、行牌率眾劫淯井監,殺駐泊借職平言,大掠資畜。知瀘州江安縣、奉職文信領兵趨之,遇害。民皆驚擾,走保戎州。轉運使寇曁咸即令諸州巡檢會江安縣,集公私船百餘艘,載糧甲,張旗幟,擊銅鑼,鼓吹,自蜀江下抵清浮壩,樹營柵,招安近界夷族,諭以大兵將至,勿與望等同惡。未
 幾,納溪、藍順州刺史史個松,生南八姓諸團,烏蠻獨廣王子界南廣溪移、悅等十一州刺史李紹安,山后高、鞏六州及江安界娑婆村首領,並來乞盟,立竹為誓門,刺貓狗雞血和酒飲之,誓同力討賊。曁咸乃署榜,許以官軍至不殺其老幼,給賜衣幣酒食。上遣內殿崇班王懷信乘傳與曁鹹等議綏撫方略,歒言鬥望等屢為寇鈔,恃寬赦不悛惡,今請發嘉、眉屯兵捕剪,以震懼之。



 六年九月,詔懷信為嘉、眉、戎、瀘等州水陸都巡檢使,閣門祗候康訓、
 符承訓為都同巡檢使,及發虎翼、神虎等兵三千餘人,令懷信與曁咸商度進討。上因謂樞密使陳堯叟曰:「往時孫正辭討蠻,有虎翼小校率眾冒險者三人,朕志其姓名,今以配懷信。正辭嘗料簡鄉丁號'白艻子兵',以其識山川險要,遂為鄉導,今亦令懷信召募。又使臣宋賁屢規畫溪洞事,適中機要,以賁知江安縣與懷信等議事。」歒乃點集昌、瀘、富順監白艻子弟得六千餘人。十一月,懷信、康訓分領,緣溪入合灘,至生南界鬥滿村遇夷賊
 二千餘人,擊之,殺傷五百人,奪梭槍藤牌。會暮,收眾保砦。夷党三千餘人分兩道,張旗喊呼來逼砦柵,懷信出擊,皆潰散。進壁娑婆,遇夷二千于羅固募村,又破之。追至鬥行村上屏風山,連破四砦。一日三戰,俘馘百餘人,奪資糧五千石、槍刀什器萬數,焚羅固募鬥引等三十餘村、庵舍三千區。懷信又引兵至鬥行村追擊過盧羅,射僕二百餘人,椚其欄柵千數。分遣部下于羅個頰羅能落運等村及龍峨山掩殺,大獲戎具,斬首級及重傷投
 崖死者頗眾,燒舍千區及積谷累萬。兩路兵會於涇灘置砦,遣康訓部壕砦卒修涇灘路,以渡大軍。俄為夷賊所邀,戰不利,訓顛於崖,死之。懷信引兵急擊,大敗之,追斬至涇灘。懷信夾砦于晏江口,曁咸與符承訓偵知賊諜欲乘夜擊晏江,馳報懷信,即自涇灘拔砦赴之。比至晏江北山,夷眾萬余已自東南合勢逼懷信砦,懷信彀強弩環砦射賊,曁鹹等整眾乘高策援,夷人大懼而卻,合擊破之,死傷千餘人。



 七年正月,其酋鬥望三路分眾來鬥,
 又為官軍大敗,射殺數百人,溺江水死者莫計。夷人震澁,詣軍首服,納牛羊、銅鼓、器械,曁鹹等依詔撫諭。二月,還軍淯井,夷首鬥望及諸村首領悉赴監自陳,願貸死,永不寇盜邊境。因殺三牲盟誓,辭甚懇苦。即犒以牢酒,感悅而去。歒、懷信等上言夷人寧息,請置淯井監壕柵,並許近界市馬。從之。



 八年,夔州路上言黔州西南密州夷族張聲進遣使進奉,為南寧州蕃落使龍漢曁堯邀奪,仇劫不已,乞降敕書安撫。



 天聖四年龍光凝、景祐三年龍光辨、康定元年龍光饧、
 慶曆五年龍以特、皇祐二年龍光澈等,繼以方物來貢獻。與以特俱至者七百十九人。是年,以安遠將軍、知蕃落使龍光辨為甯遠軍大將軍,甯遠將軍知靜蠻軍節度使龍光凝、承宣武甯大將軍龍異豈並為安遠大將軍,承宣奉化大將軍龍異魯為武甯大將軍。至和中,龍以烈、龍異靜、首領張漢陛、王子羅以崇等皆入貢,命其首領而下九十三人為大將軍至郎將。嘉祐中,以烈復至。大率龍姓諸部族地遠且貧,熙寧中來見,賜
 以袍帶等物,刺其數於背。又有張玉、石自品者,嘉祐中來貢,而鶼州亦遣人貢馬。有董氏世知保州曰仲元者,襲是州二十餘年矣,至是益州鈐轄司表其善拊蠻夷,命為本州刺史。鶼州、保州皆西南邊地也。又有夷在瀘州部,亦西南邊地,所部十州:曰鞏、曰定、曰高、曰奉、曰淯、曰宋、曰納、曰晏、曰投附、曰長寧,皆夷人居之,依山險,善寇掠。淯井監者,在夷地中,朝廷置吏領之,以拊禦夷眾,或不得人,往往生事。



 慶曆四年四月,夷人攻三江砦,詔
 秦鳳路總管司發兵千人選官馳往捕擊。既而瀘州教練使、生南招安將史愛誘降夷賊鬥敖等,詔並補三班差使、殿侍、淯井監一路招安巡檢。未幾,夷眾復寇三江砦,指使王用等擊走之。



 皇祐元年二月,夷眾萬餘人復圍淯井監,水陸不通者甚久。初,監戶負晏州夷人錢而歐傷鬥落妹,其眾憤怒,欲報之。知瀘州張昭信勸諭,既已聽服,而氵肓育井監復執婆然村夷人細令等,殺長寧州落占等十人,故激成其亂。詔知益州田況發旁郡士卒,命
 梓夔路兵馬鈐轄宋定往援之。於是兩路合官軍洎白艻子弟幾二萬人與戰,兵死者甚眾,饑死又千餘人,數月然後平。賜況及轉運使敕書,褒獎宋定而下十三人,進秩有差。後況還朝,乃奏夷眾連年為亂,繇主者非其人,請令轉運、鈐轄司舉官為知監、監押,代還日,特遷一資。從之。



 嘉祐二年,三里村夷鬥還等百五十人復謀內寇。有黃土坎夷鬥蓋,長寧州人也,先以其事來告。淯井監引兵趨之,捕斬七千餘級。鈐轄司上聞,詔賜鬥蓋錢
 三十萬、錦袍、銀帶。明年,又補鬥蓋長寧州刺史。



 瀘州部舊領姚州廢已久,有烏蠻王子得蓋者來居其地,部族最盛,數遣人詣官,自言願得州名以長夷落。事聞,因賜號姚州,鑄印予之。得蓋又乞敕書一通以遺子孫,詔從其請。



 夔州路又有溱、南二州夷,頗盛強,皇祐初,詔自今歲遣使者存問之。



 雅州西山野川路蠻者,亦西南夷之別種也,距州三百里,有部落四十六,唐以來皆為羈縻州。太平興國三年,首領馬令膜等十四人以名馬、犎牛、
 虎豹皮、麝臍來貢,並上唐朝敕書告身凡七通,咸賜以冠帶,其首領悉授官以遣之。紹聖二年,以碉門砦蠻部王元壽襲懷化司戈云。



 黎州諸蠻,凡十二種:曰山后兩林蠻,在州南七日程;曰邛部川蠻,在州東南十二程;曰風琶蠻,在州西南一千一百里;曰保塞蠻,在州西南三百里;曰三王蠻,亦曰部落蠻,在州西百里;曰西箐蠻,有彌羌部落,在州西三百里;曰淨浪蠻,在州南一百五十里;曰白蠻,在州東南一百
 里;曰烏蒙蠻,在州東南千里;曰阿宗蠻,在州西南二日程。凡風琶、兩林、邛部皆謂之東蠻,其餘小蠻各分隸焉。邛部于諸蠻中最驕悍狡譎,招集蕃漢亡命,侵攘他種,閉其道以專利。曰大雲南蠻,曰小雲南蠻,即唐南詔,今名大理國,自有傳。夷俗尚鬼,謂主祭者鬼主,故其酋長號都鬼主。



 山后兩林蠻,後唐天成間始來貢。開寶二年六月壬子,勿兒遣部落將軍離魚以狀白黎州,期十月內入貢,成都府以聞,詔嘉答之。至是來朝,賜以器幣。由黎州南行七
 日而至其地,又一程,至溋州。溋州今廢,空城中但有浮圖一。又二程,至建昌城。又十七程,至雲南。三年七月,又朝貢。六年四月,邛部川歸德將軍阿伏上言,為山后兩蠻勿兒率眾侵掠堡砦。八年,懷化將軍勿尼等六十余人來貢,詔以勿尼為歸德將軍,又以兩林蠻大鬼主蘇吠為懷化將軍。



 太平興國二年,遣使王子卑彩、副使牟蓋、鬼主還祖等七十八人以名馬來貢,乞頒正朔。下詔曰:「山后兩林要蠻主歸德將軍勿尼、懷化將軍勿兒等
 克慕聲明,遠修職貢,並增環衛之秩,俾為夷落之榮。勿尼可特授歸德大將軍,勿兒可特授懷化大將軍。」是冬,又遣使離魚貢犀二株、馬九匹,來賀登極。四年,勿兒與都鬼主又遣王子祚遇以名馬來貢。八年,蠻主弟牟昂及王子牟蓋、摩忙、卑愧、副使牟計等二百三十九人來貢。詔以牟昂為懷化大將軍,牟蓋等三人為歸德郎將,牟計等百二十人並為懷化司戈。



 雍熙三年,勿尼等及其王子李奉恩復來貢馬。淳化元年,王子離魚、副使卑
 都、卑諭、鬼主曵皮禮等百二十八人來貢。詔授離魚歸德將軍,卑都保順郎將,卑諭歸德司戈,卑熱等五十四人懷化司戈。



 天禧二年,山后兩林百蠻都鬼主李阿善遣將軍卑熱等一百五十人來貢。



 邛部川蠻,亦曰大路蠻,亦曰勿鄧,居漢越溋郡會無縣地。其酋長自稱「百蠻都鬼主」。開寶二年六月,都鬼主阿伏白黎州,期以十月令王子入貢,成都府以聞,詔嘉納之。四年,黎州定遠兵士構叛,聚居鹿角溪,阿伏令弟遊
 擊將軍卑吠等率眾平之。詔賜阿伏銀帶、錦袍,並賜其眾銀帛各百,以為歸德將軍。六年,阿伏與山后兩林蠻主勿兒言語相失,勿兒率兵侵邛部川,頗俘殺部落。黎州以聞,並賜詔慰諭,令各守封疆,勿相侵犯。



 太平興國四年,首領牟昂、諸族鬼主副使離襪等各以方物來貢。



 雍熙二年,都鬼主諾驅並其母熱免遣王子阿有等百七十二人以方物、名馬來貢。詔以諾驅為懷化將軍,並賜其母銀器。



 端拱二年,遣弟少蓋等三百五十人來賀
 籍田,貢禦馬十四匹、馬二百八十匹、犀角二、象牙二、莎羅毯一、合金銀飾蠻刀二、金飾馬鞍勒一具、敎原羊十、撐牛六。詔以少蓋為歸德郎將。



 淳化元年,諾驅自部馬二百五十匹至黎州求互市,詔增給其直。諾驅令譯者言更入西蕃求良馬以中市。二年,復遣子牟昂、叔離襪以方物、良馬、嫠牛來貢,仍乞加恩。詔授諾驅懷化大將軍,少蓋懷化將軍,牟昂歸德將軍,離襪懷化司戈;又封諾驅母歸德郡太君熱免甯遠郡太君,弟離遮、小男阿醉
 都判官,任彥德等一百九十一人為懷化司戈。



 至道元年,李順亂西川,王繼恩討平之。遣嘉州牙校辛顯使,諾驅奉淳化二年所授官告、敕書及日曆為信,因言與賊樊秀等接戰,敗之,復請朝覲,通嘉州舊路。繼恩上言:「通嘉州路非便,只令于黎州賣馬。」詔不允。其入覲王子一十九人並加官,鬼主三十六人並賜敕書以撫之。至道三年,遣王子阿醉來朝。



 真宗咸平二年,遣王子部的等來貢文犀、名馬,賜衣帶、器幣有差。又乞給印,以「大渡河
 南山前、後都鬼主」為文,從之。五年,又遣王子離歸等二百余人入貢。六年,黎州言邛部川都蠻王諾驅卒,其子阿遒立。



 景德二年,阿遒遣王子將軍百九十二人來貢。詔授阿遒安遠將軍,阿遒叔懷化將軍,阿育為歸德將軍,離歸為懷化將軍,大判官懷化司候任彥德、王子將軍部的並為懷化郎將,判官任惟慶為懷化司候。大中祥符元年,遣將軍趙勿娑等獻名馬、犀角、象齒、娑羅毯,會于泰山。禮畢,阿遒加恩。勿娑等厚賜遣還。



 天聖八年
 十月,邛部川都蠻王黎在遣卑郎、離滅等來貢方物。時占城、龜茲、沙州亦皆入貢,至以家自隨。晏殊因請圖其人物衣冠,並訪道裏風俗以上史官,詔可。九年三月,命黎在為保義將軍,又命其部族為郎將、司戈、司候,凡三十餘人。明道元年,黎州言黎在請三歲一貢,詔諭以道路遐遠,聽五年一至。景祐初,黎州復言邛部蠻請歲入貢,詔如明道令。寶元元年,百蠻都王忙海遣將軍卑蓋等貢方物,且請三歲一貢,不許。



 慶曆四年,邛部川山前、
 山后百蠻都鬼主牟黑遣將軍阿濟等三百三十九人獻馬二百一十、嫠牛一、大角羊四、犀株一、莎羅毯一。慶曆間,有都鬼主弁黑等入貢。未幾,其王咩墨擾邊,知黎州孫固使其首領苴克殺之。



 熙寧三年,苴克遣使來賀登寶位,自稱「大渡河南邛部川山前、山后百蠻都首領」,賜敕書、器幣、襲衣、銀帶。是年,苴克死,詔以其子韋則為懷化校尉、大渡河南邛部川都鬼主。九年,遣其將軍卑郎等十四人入貢。



 乾道元年,詔以崖鹁蔑襲兄蒙備金紫
 光祿大夫、懷化校尉、都鬼主如故。淳熙元年,吐蕃寇西邊,崖鹁蔑率眾掩擊,詔嘉其功。二年五月,兩林蠻王弟籠畏及酋長崖來率部義等攻邛部川之籠甕城,不克,大掠而去。崖鹁蔑追之,不及。制置使范成大檄黎州嚴加備禦。八年,崖鹁蔑死,其侄墨崖襲職。詔黎州屯戍土軍、禁軍及西兵,遇有邊事並聽本州守臣節制。



 嘉定九年,邛部川逼于雲南,遂伏屬之。其族素效順,捍禦邊陲,既折歸雲南,失西南一藩籬矣。



 風琶蠻,咸平初,其王曩䓾䓾遣使烏柏等貢馬五十七匹,素地紅花娑羅毯二,來賀即位。詔授曩䓾及進奉使等官,優賜遣之。景德三年,又遣烏柏來貢,詔授曩䓾歸德將軍,烏柏等四十六人弟遷郎將、司階、司戈。



 保塞蠻,開寶間,其蠻七十餘人由大渡河來歸,時時來貨其善馬。紹興二十七年,川、秦都大司言:「漢地民張太二姑率眾劫殺市馬蠻客崖遇等,恐啟邊釁,已加慰諭,並償其直矣。」詔免知州唐栾及通判陳伯強官,抵首賊
 法。



 部落蠻,有劉、楊、郝、趙、王五姓。淳熙七年十月,黎州五部落蠻貢馬三百匹求內附,詔許通互市,卻其所獻馬。



 彌羌部落。乾道九年,吐蕃青羌以知黎州宇文紹直不仇其馬價,憤怨為亂。詔帥憲撫安之,紹直罷免。青羌首領奴兒結等市馬黎州,大肆虜掠,權州事王窭多給金帛,亟遣還。宣撫使虞允文言窭貪功,恐他部效尤,漸啟邊釁。詔降窭兩官。十月,黎州吐蕃復寇邊,攻虎掌砦。詔四川宣
 撫司檄成都府調兵二千人戍黎州以禦之。



 淳熙二年,奴兒結還所虜生口三十九人。黎州與之盟,復聽其互市,給賞歸之。制置使范成大言:「所虜未盡歸我,豈可復與通好?」詔謫宇文紹直,編管千里外。成大增黎州五砦,籍強壯五千人為戰兵;吐蕃入寇之徑凡十有八,皆築堡戍之。奴兒結率眾二千扣安靜砦。成大調飛山卒千人赴之,度其三日必遁,戒勿追。已而果然。



 青羌奴兒結為邊害者十餘年,其後制置使留正以計禽殺之,盡
 殲其黨。淳熙十二年,趙汝愚代為制置使,或謂殺降不祥,必啟邊患,汝愚不為動,但分守險要,嚴備以待之。明年,奴兒結弟三開果入寇,邊備完固,三開不能攻,走歸。汝愚縣重賞以間群蠻,三開不能孤立,遂以憂死。時虛恨蠻族最強,破小路蠻,並其地,與黎州接壤,請通互市。汝愚以黎州三面被邊,若更通虛恨蠻,恐重貽他日之憂,不若拒之為便。帝以其知大體,從之。尋汝愚以定青羌功加龍圖閣直學士。



 嘉定元年十二月,彌羌蓄卜由惡
 水渡河,寇黎州,破碉子砦。初,蓄卜弟悶巴至三沖為人所殺,又徙白水村渡於安靜砦,羌人患之。蓄卜遂與青羌詣邛部川,欲假道女兒城以入寇。守臣楊子謨諜知之,數以貲遺其都王母,俾毋假道,時時饋米以濟其饑,蠻人德之。會趙公庀代為郡,靳不與,蓄卜遂得假道渡河,攻茆坪砦,掠三松、蠶砂、橫山、三增、白羊諸村。郡遣西兵將党壽禦之,失利,復遣統領王光世往。羌人由茆坪以革船渡河,光世憚之,留屯三沖不敢進。羌人焚掠既
 盡,渡河而歸。二年二月,復寇黎州良溪砦,官軍敗績。八年二月,蓄卜降。蓄卜連年入寇,皆青羌曳失索助之,守臣袁郐冉遣安靜砦總轄杜軫招降之。



 他如浮浪蠻、白蠻、烏蒙蠻、阿宗蠻,則其地各有所服屬云。



 敘州三路蠻:西北曰董蠻,正西曰石門部,東南曰南廣蠻。



 董蠻在馬湖江右,韈侯國也。唐羈縻馴、騁、浪、商四州之地。其酋董氏,宋初有董舂惜者貢馬,自稱「馬湖路三十七部落都王子」。其地北近犍為之沭川賴因砦。砦厄
 蠻險,蠻數寇抄。熙寧、紹聖中,朝廷皆為徙賴因監押駐榮丁砦,而以縣吏控截。政和五年,始改差監押充知砦事,蠻寇掠如故。



 南廣蠻在敘州慶符縣以西,為州十有四。大觀三年,有夷酋羅永順、楊光榮、李世恭等各以地內屬,詔建滋、純、祥三州,後皆廢。



 石門蕃部與臨洮土羌接,唐曲、播等十二州之地。俗椎髻、披氈、佩刀,居必欄棚,不喜耕稼,多畜牧。其人精悍善戰鬥,自馬湖、南廣諸族皆畏之。蓋古浪稽、魯望諸部也。



 威州保霸蠻者,唐保、霸二州也。天寶中所置,後陷沒。酋董氏,世有其地,與威州相錯,因羈縻焉。



 保州有董仲元、霸州有董永錫者,嘉祐及熙寧中皆嘗請命於朝。政和三年,知成都龐恭孫始建言開拓,置官吏。於是以董舜咨保州地為祺州,董彥博霸州地為亨州,授舜咨刺史,彥博團練使。舜咨尋遷觀察使;彥博留後,遂為節度使。詔成都給居第、田十二頃。二州經費歲用錢一萬二千一百緡,米麥一萬四千七百石,絹二千八百五十匹,糸
 由布、綾綿、茶、鹽、銀等不預焉。後皆為砦。



 茂州諸部落,蓋、塗、靜、當、直、時、飛、宕、恭等九州蠻也。蠻自推一人為州將,治其眾,而常詣茂州受約束。茂州居群蠻之中,地不過數十里,宋初無城隍,惟植鹿角自固。蠻乘夜屢入寇,民甚苦之。熙寧八年,相率詣州請築城,知州事范百常實主是役。蠻以為侵其地,率眾奄至,百常擊走之,乃合靜、時等蠻來寇。百常拒守凡七十日。詔遣王中正將陝西兵來援,入恭州、宕州,誅殺頗眾,蠻乃降。



 政和五年,有直州將郅永壽、湯延俊、董承有等各以地內屬,詔以永壽地建壽寧軍,延俊、承有地置延寧軍。時威州亦建亨、祺二州,然亨至威才九十里,壽寧距茂才五里,在大早江之外,非扼控之所,未幾皆廢。



 七年,塗、靜、時、飛等州蠻復反茂州,殺掠千餘人。知成都周燾遣兵馬鈐轄張永鐸等擊之,畏懦不敢進,皆坐黜。以孫羲叟節制綿、茂軍,於是中軍將種友直等破其都祿板舍原諸族,蠻敗散。其酋旺烈等詣茂州請降,乃班師。授旺烈
 官,月給茶彩。自後蠻亦驕。



 宣和五年,宕、恭、直諸部落入寇。六年,塗、靜蠻復犯茂州云。



 渝州蠻者,古板蚒七姓蠻,唐南平獠也。其地西南接烏蠻、昆明、哥蠻、大小播州,部族數十居之。



 治平中,熟夷李光吉、梁秀等三族據其地,各有眾數千家。間以威勢脅誘漢戶,有不從者屠之,沒入土田。往往投充客戶,謂之納身,稅賦皆裏胥代償。藏匿亡命,數以其徒偽為生獠動邊民,官軍追捕,輒遁去,習以為常,密賂黠民覘守令
 動靜,稍築城堡,繕器甲。遠近患之。



 熙寧三年,轉運使孫固、判官張詵使兵馬使馮儀、弁簡、杜安行圖之,以禍福開諭,因進兵,復賓化砦,平蕩三族。以其地賦民,凡得租三萬五千石,絲綿一萬六千兩。以賓化砦為隆化縣,隸涪州;建榮懿、扶歡兩砦。



 其外銅佛壩者,隸渝州南川縣,地皆膏腴。自光吉等平,他部族據有之。朝廷因補其土人王才進充巡檢,委之控扼。才進死,部族無所統,數出盜邊。朝廷命熊本討平之,建為南平軍,以渝州南川、涪
 州隆化隸焉。



 元豐四年,有楊光震者,助官軍破乞弟,殺其党阿訛。大觀二年,木攀首領趙泰、播州夷族楊光榮各以地內屬,詔建溱、播二州,後皆廢。



 黔州、涪州徼外有西南夷部,漢訸訫郡,唐南寧州、訸訫、昆明、東謝、南謝、西趙、充州諸蠻也。其地東北直黔、涪,西北接嘉、敘,東連荊楚,南出宜、桂。俗椎髻、左衽,或編發;隨畜牧遷徙亡常,喜險阻,善戰鬥。部族共一姓,雖各有君長,而風俗略同。宋初以來,有龍蕃、方蕃、張蕃、石蕃、羅蕃
 者,號「五姓蕃」,皆常奉職貢,受爵命。



 治平四年十二月,知靜蠻軍、蕃落使、守天聖大王龍異閣等入見,詔以異閣為武甯將軍,其屬二百四十一人各授將軍及郎將。



 熙甯元年,有方異曁兄,三年,有張漢興各以方物來獻,授異曁兄靜蠻軍,漢興捍蠻軍,並節度使。六年,龍蕃、羅蕃、方蕃、石蕃八百九十人入覲,貢丹砂、氈、馬,賜袍帶、錢帛有差。其後,比歲繼來。龍蕃眾至四百人,往返萬里,神宗憫其勤,詔五姓蕃五歲聽一貢,人有定數,無輒增加,及別立
 首領,以息公私之擾。命宋敏求編次《諸國貢奉錄》,客省、四方館撰儀,皆著為式。



 元豐五年,張蕃乞添貢奉人至三百,詔故事以七十人為額,不許。七年,西南程蕃乞貢方物,願依五姓蕃例注籍。從之。



 元祐二年,西南石蕃石以定等齎表,自稱「西平州武聖軍」。禮部言元豐著令以五年一貢為限,今年限未及。詔特令入貢。五年,八年,紹聖四年,龍蕃皆貢方物。龍氏于諸姓為最大,其貢奉尤頻數,使者便衣布袍,至假伶人之衣入見,蓋實貧陋,所
 冀者恩賞而已。故事,蠻夷入貢,雖交詾、於闐之屬皆御前殿見之,獨此諸蕃見於後殿,蓋卑之也。



 元符二年,又有牟韋蕃入貢,詔以進奉人韋公憂、公市、公利等為郎將。



 諸蕃部族數十,獨五姓最著,程氏、韋氏皆比附五姓,故號「西南七蕃」云。



 施州蠻者,夔路徼外熟夷,南接訸訫諸蠻,又與順、富、高、溪四州蠻相錯,蓋唐彭水蠻也。



 咸平中,施蠻嘗入寇,詔以鹽與之,且許其以粟轉易,蠻大悅,自是不為邊患。後
 因饑,又以金銀倍實直質于官易粟,官不能禁。熙甯六年,詔施州蠻以金銀質米者,估實直;如七年不贖,則變易之。著為令。



 熊本經制淯井事,蠻酋田現等內附,夔路轉運判官董鉞、副使孫珪、知施州寇平,皆以招納功被賞。



 施、黔比近蠻,子弟精悍,用木弩藥箭,戰鬥し捷,朝廷嘗團結為忠義勝軍。其後,瀘州、淯井、石泉蠻叛,皆獲其用。



 高州蠻,故夜郎也,在涪州西南。宋初、其酋田景遷以地內
 附,賜名珍州,拜為刺史。景遷以郡多火災,請易今名。大觀二年,有駱解下、上族納土,復以珍州名云。



 瀘州西南徼外,古羌夷之地,漢以來王侯國以百數,獨夜郎、滇、邛都、溋、昆明、徙、●都、冉垆ζ、白馬氐為最大。夜郎,在漢屬訸訫郡,今涪州之西,溱、播、珍等州封域是也;滇,在漢為益州郡,今姚州善闡之地是也;邛都,溋州會同川與吐蕃接,今邛部川蠻所居也;溋,今溋州;昆明,在黔、瀘徼外,今西南蕃部所居也;徙,今雅州嚴道地;●都,在
 黎州南,今兩林及野川蠻所居地是也;冉垆ζ,今茂州蠻、汶山夷地是也;白馬氐,在漢為武都郡,今階州、汶州,蓋羌類也:此皆巴蜀西南徼外蠻夷也。



 自黔、恭以西,至涪、瀘、嘉、敘,自階又折而東,南至威、茂、黎、雅,被邊十餘郡,綿亙數千里,剛夷惡獠,殆千萬計。自治平之末訖于靖康,大抵皆通互市,奉職貢,雖時有剽掠,如鼠竊狗偷,不能為深患。參考古今,辨其封域,以見琛贐之自至,梯航之所及者爾。若夫邊荊楚、交廣,則系之溪峒云。



 淯水夷者,
 羈縻十州五囤蠻也,雜種夷獠散居溪穀中。慶曆初,瀘州言:「管下溪峒十州,有唐及本朝所賜州額,今烏蠻王子得蓋居其地。部族最盛,旁有舊姚州,廢已久,得蓋願得州名以長夷落。」詔復建姚州,以得蓋為刺史,鑄印賜之。得蓋死,其子竊號「羅氏鬼主」。鬼主死,子僕射襲其號,浸弱不能令諸族。



 烏蠻有二酋領:曰晏子,曰斧望個恕,常入漢地鬻馬。晏子所居,直長寧、寧遠以南,斧望個恕所居,直納溪、江安以東,皆僕夜諸部也。晏子距漢地絕
 近,猶有淯井之阻。斧望個恕近納溪,以舟下瀘不過半日。二酋浸強大,擅劫晏州山外六姓及納溪二十四姓生夷。夷弱小,皆相與供其寶。



 熙寧七年,六姓夷自淯井謀入寇,命熊本經制之。景思忠戰沒,本將蜀兵,募土丁及夷界黔州弩手,以毒矢射賊,賊驚潰。於是山前後、長寧等十郡八姓及武都夷皆內附。提點刑獄范百祿作文以誓之曰:



 蠢茲夷醜,淯溪之滸。為虺為豺,憑負固圉。殺人於貨,頭顱草莽。莫慘燔炙,莫悲奴虜。狃刦熟慝,胡
 可悉數。疆吏苟玩,噤不敢語。



 奮若之歲,曾是強禦。躑躅嘯聚,三壕、羅募。僨我將佐,戕我士伍。西南繹騷,帝赫斯怒。帝怒伊何?神聖文武。民所安樂,惟曰慈撫。民所疾苦,惟曰砭去。乃用其良,應變是許。粥熊裔孫,爰馭貔虎。殲其渠酋,判其黨與。既奪之心,復斷右股。



 攝提孟陬,徂征有敘。背孤擊虛,深入厥阻。兵從天下,鐵首其舉。紛紜騰遝,莫敢嬰牾。火其巢穴,及其亘貯。暨其貲畜,墟其林VE。殺傷系縲,以百千數。涇灘望風,悉力比附。丁為帝民,地
 曰王土。投其器械,籍入官府。百死一贖,莫保銅鼓。



 歃盟神天,視此狗鼠。敢忘誅絕,以幹罪罟。乃稱上恩,俾復故處。殘醜厥角,泣血訴語:「天子之德,雨撸覆護。三五噍類,請比涇仵。」



 大邦有令,其戒警汝:天既汝貸,汝勿予侮。惟十九姓,往安汝堵。吏治汝責,汝力汝布。吏時汝耕,汝稻汝黍。懲創於今,無忄太往古。小有堡障,大有城戍。汝或不聽,汝擊汝捕。尚有刦將,突騎強旅。傅此黔軍,毒矢勁弩。天不汝容,暴汝居所。不汝遺育,悔于何取!



 立石于武寧
 砦。



 熊本言二酋桀黠,不羈縻之則諸蠻未易服,遂遣人說誘招納。於是晏子、斧望個恕及僕夜皆願入貢,受王命。晏子未及命而死,乃以個恕知歸來州,僕夜知姚州,以個恕之子乞弟、晏子之子沙取祿路並為把截將、西南夷部巡檢。



 八年,俞州獠寇南州,獠酋阿訛率其黨奔個恕。熊本重賞檄斬訛。訛桀黠,習知邊境虛實,個恕匿不殺,詭降於納溪。訛得不死,甚德個恕,為伺邊隙。會個恕老厭兵,以事屬乞弟,遂與訛侵諸部。



 十年,羅苟夷犯
 納溪砦。初,砦民與羅苟夷競魚笱,誤毆殺之,吏為按驗。夷已忿,謂:「漢殺吾人,官不嘗我骨價,反暴露之。」遂叛。提點刑獄穆曁向言:「納溪去瀘一舍,羅苟去納溪數里,今托事起端,若不加誅,則烏蠻觀望,為害不細。」乃詔涇原副總管韓存寶擊之。存寶召乞弟等犄角,討蕩五十六村,十三囤蠻乞降,願納土承賦租。乃詔罷兵。



 元豐元年,乞弟率晏州夷合步騎六千至江安城下,責平羅苟之賞。城中守兵才數百,震恐不能授甲,蠻數日乃引去。知瀘
 州喬敘要欲與盟,遣梓夔都監王宣以兵二千守江安,仍奏以乞弟襲歸來州刺史。韓運遣小校楊舜之召乞弟拜敕,乞弟不出;遣就賜之,亦不見;而令小蠻從舜之取敕以去。喬敘因沙取祿路以賄招乞弟,乃肯來。



 三年,盟於納溪。蠻以為畏己,益悖慢。盟五日,遂以眾圍羅個牟族。羅個牟,熊本所團結熟夷也。王宣往救之,蠻解圍,合力拒官軍。宣與一軍皆沒,事遂張,ㄞ召存寶授方略,統三將兵萬八千趨東川。存寶怯懦不敢進,乞弟送款
 紿降,存寶信之,遂休兵於綿、梓、遂、資間。



 四年,詔以環慶副總管林廣代存寶,按寶逗撓,誅之。熟夷楊光震殺阿訛,詔林廣與光震同力討賊。乞弟恐,復送款。帝以其前後反覆,無真降意,督廣進師。廣遂破樂共城,至鬥蒲村,斬首二千五百級。次落婆,乞弟乃納降。廣盛陳兵以受之,對語良久,乞弟疑有變,引眾遁。廣帥兵深入,會大雨雪,浹旬始次老人山,山形劍立。度黑崖,至鴉飛不到山。五年正月,次歸來州,天大寒,然桂為薪,軍士皆凍墮指。
 留四日,求乞弟不可得。內侍麥文暼丙問廣軍事,廣曰:「賊未授首,當待罪。」文暼丙乃出所受密詔曰:「大兵深入討賊,期在梟獲元惡。如已破其巢穴,雖未得乞弟,亦聽班師。」軍中皆呼萬歲,曰:「天子居九重,明見萬里外。」乃以眾還。自納溪之役,師行凡四十日。築樂共城、江門砦、梅嶺席帽溪堡,西達淯井,東道納溪,皆控制要害。捷書聞,赦梓州路,以歸來州地賜羅氏鬼主。乞弟既失土,窮甚,往來諸蠻間,無所依。帝猶欲招來之,命知瀘州王光祖開
 諭,許以自新。會其死,於是羅始黨、鬥然、鬥更等諸酋請依十九姓團結,新收生界八姓、兩江夷族請依七姓團結,皆為義軍。從之。自是瀘夷震懾,不復為邊患。沙取祿路死,子鱉弊承襲。



 政和五年,晏州夷卜漏叛,砦將高公老遁,招討使趙瘇討平之,授鱉弊西南夷界都大巡檢。事見《趙瘇傳》。



\end{pinyinscope}