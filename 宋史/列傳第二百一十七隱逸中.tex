\article{列傳第二百一十七隱逸中}

\begin{pinyinscope}

 ○王樵張愈黃晞周啟明代淵陳烈孫侔劉易姜潛連庶章詧俞汝尚
 陽孝本鄧考甫宇文之邵吳瑛松江漁翁杜生順昌山人南安翁張舉



 王樵,字肩望,淄州淄川人。居縣北梓桐山。博通群書,不治章句,尤善考《易》。與賈同、李冠齊名,學者多從之。咸平中,契丹游騎度河,舉家被掠。樵即棄妻,挺身入契丹訪父母,累年不獲,還東山。刻木招魂以葬,立祠畫像,事之如生,服喪六年,哀動行路。又為屬之尊者次第成服,北
 望嘆曰:「身世如此,自比於人可乎!」遂與俗絕,自稱贅世翁,唯以論兵擊劍為事。一驢負裝,徒步千里,晚年屢游塞下。畫策干何承矩、耿望,求滅遼復仇,不用。乃於城東南隅累磚自環,謂之「繭室」。銘其門曰:「天生王樵,薄命寡智,材不濟時,道號『贅世』。生而為室,以備不虞,死則藏形,不虞乃備。」病革,入室自掩戶卒。治平末,職方郎中向宗道知淄州,訪繭室,已構屋為民居。得樵甥牟氏子,乃知改葬。因而即其地復作繭室及祠堂,刻石以記之。



 張愈字少愚,益州郫人,其先自河東徙。愈雋偉有大志,游學四方,屢舉不第。寶元初,上書言邊事,請使契丹,令外夷相攻,以完中國之勢,其論甚壯。用使者薦,除試秘書省校書郎,願以授父顯忠而隱於家。文彥博治蜀,為置青城山白雲溪杜光庭故居以處之。丁內艱,鹽酪不入口。再期,植所持柳杖於墓,忽生枝葉,後合抱。六召不應。喜奕棋。樂山水,遇有興,雖數千里輒盡室往。遂浮湘、沅,觀浙江,升羅浮,入九疑,買石載鶴以歸。杜門著書,未
 就,卒。



 妻蒲氏名芝,賢而有文,為之誄曰:「高視往古,哲士實殷,施及秦、漢,餘烈氛氳。挺生英傑,卓爾逸群,孰謂今世,亦有其人。其人伊何?白雲隱君。嘗曰丈夫,趨世不偶,仕非其志,祿不可茍,營營末途,非吾所守。吾生有涯,少實多艱,窮亦自固,困亦不顛。不貴人爵,知命樂天,脫簪散發,眠雲聽泉。有峰千仞,有溪數曲,廣成遺趾,吳興高躅。疏石通逕,依林架屋,麋鹿同群,晝游夜息。嶺月破雲,秋霖灑竹,清意何窮,真心自得,放言遺慮,何榮何辱?孟
 春感疾,閉戶不出,豈期遂往,英標永隔。抒詞哽噎,揮涕汍瀾,人誰無死,惜乎材賢。已矣吾人,嗚呼哀哉!」



 黃晞,字景微,建安人。少通經,聚書數千卷,學者多從之游,自號聱隅子。著《歔欷瑣微論》十卷,以謂聱隅者枿物之名,歔欷者嘆聲,瑣微者述辭也。石介在太學,遣諸生以禮聘召,晞走匿鄰家不出。樞密使韓琦表薦之,以為太學助教致仕。受命一夕卒。



 周啟明字昭回,其先金陵人,後占籍處州。初以書謁翰
 林學士楊億,億攜以示同列,大見嘆賞,自是知名。四舉進士皆第一。景德中,舉賢良方正科,既召,會東封泰山,言者謂此科本因災異訪直言,非太平事,遂報罷。於是歸,教弟子百餘人,不復有仕進意,里人稱為處士。轉運使陳堯佐表其行義於朝,賜粟帛。仁宗即位,除試助教,就加廩給。久之,特遷秘書省秘書郎。改太常丞,卒。啟明篤學,藏書數千卷,多手自傳寫,而能口誦之。有古律詩、賦、箋、啟、雜文千六百餘篇。



 代淵,字蘊之,本代州人。唐末,避地導江,家世為吏,有陰德。淵性簡潔,事親以孝聞。受學於李畋、張達。年四十,鄉人更勸,舉進士甲科,得清水主簿。嘆曰:「祿不及親,何所為耶?」還家教授,坐席常滿。安撫使舉鳳州團練推官,不就。知益州楊日嚴又薦之,遂以太子中允致仕。謝絕諸生,著《周易旨要》、《老佛雜說》數十篇。田況上其書,自太常丞改祠部員外郎。晚年日菜食,巾褐山水間,自號虛一子。長吏歲時致問,澹然與對,略不及私。嘉祐二年九月,
 有疾,召術士擇日,云「丙申吉」,頷之,是日沐浴而絕。



 陳烈字季慈,福州候官人。性介僻,篤於孝友。居親喪,勺飲不入於口五日,自壯及老,奉事如生。學行端飭,動遵古禮,平居終日不言,御童僕如對賓客。里中人敬之,冠昏喪祭,請而後行。從學者常數百。賢父兄訓子弟,必舉烈言行以示之。



 嘗以鄉薦試京師不利,即罷舉。或勉之求仕,則曰:「伊尹守道,成湯三聘以幣;呂望既老,文王載之俱歸。今天子仁聖好賢,有湯、文之心,豈無先覺如伊、
 呂者乎?」仁宗屢詔之,不起。人問其故,應曰:「吾學未成也。」公卿大夫、郡守、鄉老交章稱其賢。嘉祐中,以為本州教授,歐陽修又言之,召為國子直講,皆不拜。



 已而福建提刑王陶言其為妻林氏所訟,因詆烈貪詐,乞奪所受恩。司馬光為諫官,率同列爭曰:「臣等每患士無名檢,故舉烈以厲風俗。烈平生操守,出於誠實,雖有迂闊不合中道,猶為守節之士,當保而全之。若夫婦不相諧,則聽之離絕,毋使節行之士為橫辱所挫。」陶說遂不行。



 元祐初,
 部使者申薦之,詔從其尚,以宣德郎致仕。明年,復教授本州。,在職不受廩奉,鄉里問遺絲毫無所受;家租有餘,則推以濟貧乏。卒,年七十六。



 孫侔,字少述,與王安石、曾鞏游,名傾一時。早孤,事母盡孝。志於祿養,故屢舉進士。及母病革,自誓終身不求仕。客居江、淮間,士大夫敬畏之。



 劉敞知揚州,言其孝弟忠信,足以扶世矯俗,求之朝廷,呂公著、王安石之流也。詔以為揚州教授,辭。敞守永興,闢入幕府,亦辭。英宗時,沈
 遘及王陶、韓維連薦之,授忠武軍推官、常州推官,皆不赴。



 少與安石友善,安石為相,過真州與相見,侔待之如布衣交。卒,年六十六。



 初,王回、王令、常秩與侔皆有盛名,回、令不壽,秩為隱不竟,唯侔以不仕始終。



 劉易,忻州人。性介烈,博學好古,喜談兵。韓琦知定州,上其所著《春秋論》,授太學助教、並州州學說書。不能屈志仕進,寓居於虢之盧氏,習闢谷術。趙抃復薦其行誼,賜號退安處士。易作詩,琦每為書之石,或不可其意輒滌
 去,琦亦再書之。尹洙帥渭,延致尊禮,狄青代洙,遇之亦厚。治平末,卒,琦作文祭之云:「剛介之性,天下能合者有幾?淵源之學,古人不到者甚多。」其敬之如此。熙寧察訪定戶役,詔易家用處士如七品恩,得減半,示優禮云。



 姜潛,字至之,兗州奉符人。從孫復學《春秋》。用田況舉召試學士院,為明州錄事參軍。以母思鄉求致仕,敕過門下,知封駁司吳奎封還之,而與韓絳共上章以薦,徙兗州錄事參軍。從奎闢鄆州教授,奎升堂拜其母,又薦為
 國子直講、韓王宮伴讀。謁宗正允弼,吏引趨庭,潛不答,呼馬欲去,遂以客禮見。



 熙寧初,詔舉選人淹滯者與京官凡三十七人,潛在選中。神宗聞其賢,召對延和殿,訪以治道何以致之,對曰:「有《堯》、《舜》二《典》在,顧陛下致之之道何如。」知陳留縣,至數月,青苗令下,潛出錢,榜其令於縣門,已,徙之鄉落,各三日無應者。遂撤榜付吏曰:「民不願矣!」錢以是獨得不散。司農、開封疑潛沮格,各使其屬來驗,皆如令。而條例司劾祥符住散青苗錢,潛知且不
 免,移疾去,縣人詣府請留之,不得。家居卒,年六十六。



 連庶字居錫,安州應山人。舉進士,調商水尉、壽春令。興學,尊禮秀民,以勸其俗;開瀕淮田千頃,縣大治。淮南王舊壘在山間,會大水,州守議取其甓為城,庶曰:「弓矢舞衣傳百世,藏於王府,非為必可用,蓋以古之物傳於今,尚有典刑也。」壘因是得存。以母老乞監陳州稅。嘗送客出北門,見日西風塵,而冠蓋憧憧不已,慨然有感,即日求分司歸。久之,翰林學士歐陽修、龍圖閣直學士祖無
 擇言庶文學行義,宜在臺閣。以知昆山縣,辭不行。累遷職方員外郎,卒。



 庶始與弟庠在鄉里,時宋郊兄弟、歐陽修皆依之。及二宋貴達,不可其志,退居二十年。守道好修,非其人不交,非其義秋毫不可污也。庶既死,宋郊之孫義年為應山令,緣邑人之意,作堂於法興僧舍,繪二宋及庶、庠之像祠事之。庠亦登科,敏於政事,號良吏,終都官郎中。



 章詧字,隱之,成都雙流人。少孤,鞠於兄嫂,以所事父母
 事之。博通經學,尤長《易》、《太玄》,著《發隱》三篇,明用蓍索道之法,知以數寓道之用、三摹九據始終之變。蜀守蔣堂、楊察、張方平、何郯、趙抃咸以逸民薦,一賜粟帛,再命州助教,不就。嘉祐中,賜號沖退處士。王素時為州,因更其所居之鄉曰處士,里曰通儒,坊曰沖退。詧由是益以道自裕,尊生養氣,憂喜、是非亦不以撓其心形。



 嘗訪里人範百祿,謂曰:「子闢穀二十餘年,今強力尚足,子亦嘗知以氣治疾之說乎?」百祿因從扣《太玄》,詧為解述大旨,再
 復《摛》詞曰:「『人之所好而不足者,善也;所醜而有餘者,惡也。君子能強其所不足,而拂其所有餘,《太玄》之道幾矣。』此子云仁義之心,予之於《太玄》也,述斯而已。若苦其思,艱其言,迂溺其所以為數而忘其仁義之大,是惡足以語夫道哉?」熙寧元年,卒,年七十六。子祀,亦好古學,嘗應行義敦遣詔。仍世有隱德,其所居猶存。



 俞汝尚,字退翁,湖州烏程人。少時讀書於鄣南之昆山。為人溫溫有禮,議論不茍。不可於意,有所不言,言之未
 嘗妄也。不肯料理生事,不以貧乏撓其懷,淡於勢利。聞人善言善行,記之不忘,時時為人道之。擢進士第,涉歷州縣,無少營進取之心。嘗知導江縣,新繁令卒,使者使承其乏,將資以公田,辭,不許,至則悉以周舊令之家。熙寧初,簽書劍南西川判官。趙抃守蜀,以簡靜為治,每旦退坐便齋,諸吏莫敢至,唯汝尚來輒排闥徑入,相對清談竟暮。



 王安石當國,患一時故老不同己,或言汝尚清望,可置之御史,使以次彈擊。驛召詣京師,既知所以薦
 用意,力辭,章再上得免。親故有責以不能與子孫為地者,汝尚笑曰:「是乃所以為其地也。」還家苦貧,未能忘祿養。又從趙抃於青州,遂以屯田郎中致仕。蘇軾、蘇轍、孫覺、李常皆賦詩文嘆美之。



 優游數年,當六月徂暑,寢室不可居,出舍於門,妻黃就視之,汝尚曰:「人生七十者希,吾與夫人皆過之,可以行矣。」妻應曰:「然則我先去。」後三日卒。汝尚庀其喪,為作銘,召諸子告曰:「吾亦從此逝矣。」隱幾而終,相去才十日。孫侔,紹興中敷文閣直學士。



 陽孝本,字行先,虔州贛人。學博行高,隱於城西通天巖。蘇頌、蒲宗孟皆以山林特起薦之。蘇軾自海外歸,過而愛焉,號之曰玉巖居士。嘗直造其室,知其不娶,戲以為元德秀之流。孝本自言為陽城之裔,故軾詩有云:「眾謂元德秀,自稱陽道州。」嘉之也。隱遁二十年,一時名士多從之游。崇寧中,舉八行,解褐為國子錄,再轉博士。以直秘閣歸,卒,年八十四。



 鄧考甫,字成之,臨川人。第進士,歷陳留尉、萬載永明令、
 知上饒縣,積官奉議郎,提點開封府界河渠,坐事去官,遂閉戶著書,不復言仕。



 元符末,詔求直言。考甫年八十一,上書云:「亂天下者,新法也,末流之禍,將不可勝言。今宜以時更化,純法祖宗。」因論熙寧而下,權臣迭起,欺世誤國,歷指其事而枚數其人。蔡京嫉之,謂為詆訕宗廟,削籍羈筠州。崇寧去黨碑,釋逐臣,同類者五十三人,其五十人得歸,惟考甫與範柔中、封覺民獨否,遂卒於筠。且死,命幼孫名世執筆,口占百餘言,其略曰:「予自謂山
 中宰相,虛有其才也;自謂文昌先生,虛有其詞也。不得大用於盛世,亦無憾焉,蓋有天命爾。」所論述有《卜世大寶龜》、《伊周素蘊》、《義命雜著》、《太平策要》等,凡二百五十餘篇。



 宇文之邵,字公南,漢州綿竹人。舉進士,為文州曲水令。轉運以輕縑高其價,使縣鬻於民。之邵言:「縣下江上山,地狹人貧,耕者亡幾,方歲儉饑,羌夷數入寇,不可復困之以求利。」運使怒。



 會神宗即位求言,乃上疏曰:「天下一
 家也。祖宗創業、守成之法具在。陛下方居諒陰,諂諛奸佞之人屏伏未動,正可念五聖之功德,常若左右前後。京師者,諸夏之視效,俗宜敦厚,而勿憸薄浮侈是尚。公卿大夫,民之表也,宜以名節自勵,而勢利合雜是先。願以節義廉恥風導之,使人知自重。千里之郡,有利未必興,有害未必除者,轉運使、提點刑獄制之也。百里之邑,有利未必興,有害未必除者,郡制之也。前日赦令,應在公逋負一切蠲除,而有司操之益急,督之愈甚,使上澤不
 下流,而細民益困。如擇賢才以為三司之官,稍假郡縣以權,則民瘼除矣。然後監番、棸、蹶、楀之盛以保安外戚,考《棠棣》、《角弓》之義以親睦九族,興墜典,拔滯淹,遠誇毗,來忠讜。凡所建置,必與大臣共議以廣其善,號令威福則專制之。如此,則天下之人思見太平可拱而俟也。」



 疏奏不報。喟然曰:「吾不可仕矣。」遂致仕,以太子中允歸,時年未四十。自強於學,不易其志,日與交友為經史琴酒之樂,退居十五年而終。司馬光曰:「吾聞志不行,顧祿位
 如錙銖;道不同,視富貴如土芥。今於之邵見之矣。」範鎮亦曰:「之邵位下而言高,學富而行篤,少我二十一歲而先我掛冠,使吾慊然。」其為兩賢所推尚如此。



 吳瑛,字德仁,蘄州蘄春人。以父龍圖閣學士遵路任補太廟齋郎,監西京竹木務,簽書淮南判官,通判池州、黃州,知郴州,至虞部員外郎。治平三年,官滿如京師,年四十六,即上書請致仕。公卿大夫知之者相與出力挽留之,不聽,皆嘆服以為不可及,相率賦詩飲餞於都門,遂
 歸。



 蘄有田,僅足自給。臨溪築室,種花釀酒,家事一付子弟。賓客至必飲,飲必醉,或困臥花間,客去亦不問。有臧否人物者,不酬一語,但促奴益行酒,人莫不愛其樂易而敬其高。嘗有貴客過之,瑛酒酣而歌,以樂器扣其頭為節,客亦不以為忤。視財物如糞土,妹婿輒取家財數十萬貸人,不能償,瑛哀之曰:「是人有母,得無重憂!」召而焚其券。門生為治田事歷歲,忽謝去,曰:「聞有言某簿書為欺者,誼不可留。」瑛命取前後文書示之,蓋未嘗發封
 也。盜入室,覺而不言,且取其被,乃曰:「他物唯所欲,夜正寒,幸舍吾被。」其真率曠達類此。



 哲宗朝有薦之者,召為吏部郎中,就知蘄州,皆不起。崇寧三年感疾,即閉閤謝醫藥,至垂絕不亂。卒,年八十四。



 松江漁翁者,不知其姓名。每棹小舟游長橋,往來波上,扣舷飲酒,酣歌自得。紹聖中,閩人潘裕自京師調官回,過吳江,遇而異焉,起揖之曰:「予視先生氣貌,固非漁釣之流,願丐緒言,以發蒙陋。」翁瞪視曰:「君不凡,若誠有意,
 能過小舟語乎?」裕欣然過之。翁曰:「吾厭喧煩,處閑曠,遁跡於此三十年矣。幼喜誦經史百家之言,後觀釋氏書,今皆棄去。唯飽食以嬉,尚何所事?」裕曰:「先生澡身浴德如此。今聖明在上,盍出而仕乎?」笑曰:「君子之道,或出或處,吾雖不能棲隱巖穴,追園、綺之蹤,竊慕老氏曲全之義。且養志者忘形,養形者忘利,致道者忘心,心形俱忘,其視軒冕如糞土耳,與子出處異趣,子勉之。」裕曰:「裕也不才,幸聞先生之高義,敢問舍所在。」曰:「吾姓名且不欲
 人知,況居室耶!」飲畢,長揖使裕反其所,鼓枻而去。



 杜生者,潁昌人。不知其名,縣人呼為杜五郎。所居去縣三十里,有屋兩間,與其子並居,前有空地丈餘,即為籬門,生不出門者三十年。



 黎陽尉孫軫往訪之。其人頗灑落,自陳村人無所能,官人何為見顧。軫問所以不出門之因,笑曰:「以告者過也。」指門外一桑曰:「憶十五年前,亦曾納涼其下,何謂不出?但無用於時,無求於人,偶自不出耳,何足尚哉。」問所以為生,曰:「昔時居邑之南,有田五
 十畝,與某兄同耕。迨兄子娶婦,度所耕不足贍,乃盡以與兄,而攜妻子至此,蒙鄉人借屋,遂居之。唯與人擇日,又賣醫藥以給飦粥,亦有時不繼。後子能耕,荷長者見憐,與田三十畝使之耕,尚有餘力,又為人傭耕,自此食足。鄉人貧,以醫術自業者多。念己食既足,不當更兼他利,由是擇日賣藥,一切不為。」問常日何所為,曰:「端坐耳。」「頗觀書否?」曰:「二十年前,曾有人遺一書策,無題號,其間多說浮名經,當時極愛其議論,今忘之,並書亦不知所
 在矣。」時盛寒,布袍草屩,室中枵然,而氣韻閑曠,言詞精簡,。蓋有道之士也。問其子之為人,曰:「村童也,然性質甚淳厚,不妄言,不敢嬉。唯間一至縣買鹽酪,可數行跡以待其歸,徑往徑還,未嘗旁游一步也。」軫嗟嘆,留連久之,乃去。後至延安幕府,為沈括言之。括時理軍書,迨夜半,疲極未臥,聞軫談及此,及頓忘其勞。



 順昌山人。靖康末,有避亂於順昌山中者,深入得茅舍,主人風裁甚整,即之語,士君子也。怪而問曰:「諸君何事
 挈妻孥能至是耶?」因語之故。主人曰:「亂何自而起耶?」眾爭為言,主人嗟惻久之,曰:「我父為仁宗朝人也,自嘉祐末卜居於此,因不復出。以我所聞,但知有熙寧紀年,亦不知於今幾何年矣。」



 南安翁者。漳州陳元忠客居南海日,嘗赴省試過南安,會日暮,投宿野人家,茅茨數椽,竹樹茂密可愛。主翁雖麻衣草屨,而舉止談對宛若士人。幾案間有文籍散亂,視之皆經、子也。陳叩之曰:「翁訓子讀書乎?」曰:「種園為生
 耳。」「亦入城市乎?」曰:「十五年不出矣。」問:「藏書何用?」曰:「偶有之耳。」因雜以他語。少焉,風雨暴作,其二子歸,舍鉏揖客,人物不類農家子。翁進豆羹享客,不復共談,遲明別去。



 陳以事留城中,翌日,見翁倉遑而行,陳追詰之曰:「翁云十五年不出城,何為到此?」曰:「吾以急事不容不出。」問之,乃大兒於關外鬻果失稅,為關吏所拘。陳為謁監征,至則已捕送郡。翁與小兒偕詣庭下,長子當杖,翁懇白郡守曰:「某老鈍無能,全藉此子贍給。若渠不勝杖,則翌日
 乏食矣。願以身代之。」小兒曰:「大人豈可受杖,某願代兄。」大兒又以罪在己,甘心焉,三人爭不決。小兒來父耳旁語,若將有所請,翁叱之,兒必欲前。郡守疑之,呼問所以,對曰:「大人元系帶職正郎,宣和間累典州郡。」翁急拽其衣使退,曰:「兒狂,妄言。」守詢誥敕在否,兒曰:「見作一束置甕中,埋於山下。」守立遣吏隨兒發取,果得之,即延翁上坐,謝而釋其子。次日,枉駕訪之,室已虛矣。



 張UF字子厚,常州人。登進士甲科。以無他兄弟,獨養其
 親,不忍斯須去左右。親友強之仕,乃調青溪主簿,亦不之官。閉戶讀書四十年,手校數萬卷,無一字舛。窮經著書,至夜分不寐。元豐中,近臣薦其高行。至於元祐,大臣復薦之,起教授潁州,辭不就。於是孫覺、胡宗愈、範祖禹交章言曰:「UF且死草萊,後世必以為朝廷失士。」蘇軾言之尤切。詔拜秘書省校書郎,敕郡縣致禮敦遣,竟不出。



 UF孝弟修於家,忠信行於友,聲名聞於人,蹈中守常,從容不迫,為當時名流所慕,以不造門為恥。崇寧四年,卒。
 明年,詔以UF隱德丘園,聲聞顯著,賜謚曰正素先生。



\end{pinyinscope}