\article{列傳第二百六忠義二}

\begin{pinyinscope}
霍安國李涓李邈劉翊徐揆陳遘趙不試趙令
 \gezhu{
  山成}
 唐重郭忠孝程迪徐徽言
 向子韶楊邦乂



 霍安國,不知何許人。燕山之復,以直秘閣為轉運判官。宣和末,知懷州。靖康元年,路允迪奉使至懷,表其治狀,加直龍圖閣。歲中,進右文、集英殿修撰,徙知隆德府,未行復留。金騎再至,遂被圍,安國捍禦不遺力,鼎、澧兵亦至,相與共守。拜徽猷閣待制,然竟以閏十一月城陷。將官王美投壕死。粘罕引安國以下分為四行,使夷官問不降者為誰,安國曰:「守臣安國也。」問餘人,通判州事直
 徽猷閣林淵,兵馬鈴轄、濟州防禦使張彭年,都監趙士詝、張諶、於潛,鼎、澧將沈敦、張行中及隊將五人,同辭對曰:「淵等與知州一體,皆不肯降。」酋令引於東北鄉,望其國拜降,皆不屈,乃解衣面縛,殺十三人而釋其餘。安國一門無噍類。明年,贈延康殿學士。



 李涓,字浩然,駙馬都尉遵勖曾孫也。以蔭為殿直,召試中書,易文階,至通直郎,知鄂州崇陽縣。靖康元年,京城被圍,羽檄召天下兵。鄂部縣七,當發二千九百人,皆未
 集,涓獨以所募六百銳然請行。或謂:「盍徐之,以須他邑。」涓曰:「事急矣,當持一信報天子,為東南倡。」而募士多市人,不能軍,涓出家錢買牛酒激犒之。令曰:「吾固知無益,然世受國恩,唯直死耳。若曹知法乎,『失將者死』,鈞之一死,死國留名,男兒不朽事也。」眾皆泣。即日,引而東,北過淮,蒲圻、嘉魚二縣之兵始至,合而前。至蔡,天大雪,蔡人忽噪而奔,曰:「敵至矣。」即結陣以待。少焉,游騎果集。涓馳馬先犯其鋒,下皆步卒,蒙鹵盾徑進,頗殺其騎,且走。涓
 乘勝追北十餘里,大與敵遇,飛矢蝟集,二縣兵亟舍去。涓創甚,猶血戰,大呼叱左右負己,遂死焉,年五十三。士卒死者六七。上官有忌涓者,脅亡卒誣已遁。明年,金兵去,蔡人以其尸歸。朝廷錄其忠,贈朝奉郎,官其三子。



 李邈,字彥思,臨江軍清江人。唐宗室宰相適之之後。少有才略,精悍敏決,見事風生。以父任為太廟齋郎。初調安州司理,監潤州酒務。用薦改京官,監在京竹木務,擢提轄環慶路糧草,通判河間府。



 以迕蔡京、童貫,換右列,
 由承議郎換莊宅副使,知信安軍,遷知霸州,為遼國賀正副使。還,貫將連金人夾攻契丹,呼邈至私第,以語動之,使附己。邈言契丹人未厭其主,貫懼邈有異議,即奏不俟對,令復任。邈上書言:「契丹不可滅,茍誤機事,願誅臣以謝邊吏。」都轉運使沈積中捃邈罪五十有三條,鞠治一無所得,乃以建神霄宮不如詔,免官。



 久之,監在京染院,進都大提舉京西汴河堤岸。盜起浙東,改江、淮、兩浙制置司管當公事,改知嚴州,代還。貫欲以西師入燕,
 邈復語貫曰:「方臘小醜,一呼屠七州四十餘縣,竭數路之力而後能平之,殆天以此警公也,何可遽移之北乎?」因密教貫陰佐契丹以圖金人,貫不能用,乃乞致仕。貫收復燕山,奏邈知涿州,改易州,皆辭不赴。嘆曰:「國家禍亂自茲始矣!」



 金人犯京師,詔趣入見,邈慨然復起就道。既至,會姚平仲戰不利,京師震動,上不以時賜對,問御敵奈何?邈言:「勝負兵家之常勢,陛下無過憂,第古未有和戰不定而能成功者。」因言:「種師道宿將,有重名,二敵
 所畏。朝廷自主和議,而盡以諸道兵畀師道,視敵為進退。將在軍中,君命有所不受,使見可擊而進,勝固社稷之福;不勝,亦足使敵知吾將帥有以國為任者。」上稱善,而耿南仲方主和議,不合,乃換右文殿修撰、京畿轉運使,辭不拜。



 金人猶駐毛駝崗,乃以邈為京城西壁守禦使。邈言:「姚平仲敗績,而敵猶不敢留,是畏我也。不以種師道再戰,已失機會;尚可尾其行,及河半渡擊之,猶足為後戒。」議復格。三上章致仕,不允。改主管馬軍公事、權
 樞密副都承旨,出為河北西路制置使。以措置山西塘灣、屯田、弓箭手事。邈論塘灣不可為,奪制置使,下遷提舉保甲,仍領措置司。又論不已,再奪觀察使,則金兵將及境矣。遂復舊官,守真定。後二日,落階,拜青州觀察使,仍知府事。



 邈始視事,兵不滿二千,錢不滿二百萬,自度無以拒敵,乃諭民出財,共為死守。民恃邈為固,不數日,得錢十三萬貫、粟十一萬石,募民為勇敢亦數千人。而新集之兵皆無鬥志,金人至,邈乞師於宣撫副使劉韐,
 且間道走蠟書上聞,皆不報。城被圍,且戰且守,相持四旬。城破,邈巷戰不克,將赴井,左右持之不得入。斡離不脅邈拜,不拜,以火燎其須眉及兩髀,亦不顧,乃拘於燕山府。



 金人問曰:「集民兵擊我,謂我為賊,何也?」邈曰:「汝負盟,所至掠吾金帛子女,何諱吾言敵?」不能屈。久之,欲以邈知滄州,笑而不答。且說之曰:「天下強弱之勢安有常,特吾中國適逢其隙耳。汝不以此時歸二帝及兩河地,歲取重幣如契丹,以為長利,強尚可恃乎?」金人諱其言,
 命邈被發左衽,邈憤,詆毀甚力,金人撾其口,猶吮血噀之。翼日,自去發為浮屠,金人大怒,遂遇害。將死,顏色不變,南向再拜,端坐就戮,燕人為之流涕。高宗贈昭化軍節度使,謚曰忠壯。



 劉翊,靖康元年,以吉州防禦使為真定府路都鈴轄。金人攻廣信、保州不克,遂越中山而攻真定。翊率眾晝夜搏戰城上。金兵初攻北壁,翊拒之,乃偽徙攻東城,宣撫使李邈復趣翊往應;越再宿,潛移攻具還薄北城,眾攀
 堞而上,城遂陷。邈就執,翊猶集左右巷戰,已而稍亡去,翊顧其弟曰:「我大將也,其可受賊戮乎!」挺身潰圍欲出,諸門已為敵所守,乃之孫氏山亭中,解絳自縊死。



 徐揆,衢州人。游京師,入太學。靖康元年,試開封府進士,為舉首,未及大比而遭國難。欽宗詣金營不歸,揆帥諸生扣南薰門,以書抵二酋,請車駕還闕。其略曰:「昔楚莊王入陳,欲以為縣,申叔時諫,復封之。後世君子,莫不多叔時之善諫,楚子之從諫,千百歲之下,猶想其風採。本
 朝失信大國,背盟致討,元帥之職也;郡城失守,社稷幾亡而存,元帥之德也;兵不血刃,市不易肆,生靈幾死而活,元帥之仁也;雖楚子存陳之功,未能有過。我皇帝親屈萬乘,兩造轅門,越在草莽,國中喁喁,跂望屬車之塵者屢矣。道路之言,乃謂以金銀未足,故天子未返,揆竊惑之。今國家帑藏既空。編民一妾婦之飾,一器用之微,無不輸之公上。商賈絕跡,不來京邑,區區豈足以償需索之數。有存社稷之德,活生靈之仁,而以金帛之故,留
 質君父。是猶愛人之子弟,而辱其父祖,與不愛無擇,元帥必不為也。願推惻隱之心,存始終之惠,反其君父,班師振旅,緩以時日,使求之四方,然後遣使人奉獻,則楚封陳之功不足道也。」二酋見書,使以馬載揆至軍詰難,揆厲聲抗論,為所殺。建炎二年,追錄死節,詔贈宣教郎,而官其後。



 陳遘,字亨伯,其先自江寧徙永州。登進士第。知莘縣,為治有績,魏尹蔣之奇、馮京、許將交薦之。知雍丘縣,徽宗
 將以為御史,而遭父祐甫憂。畢喪,為廣西轉運判官。蔡京啟蠻徭地,建平、從、允三州,遘言:「蠻人幸安靜,輕擾以兆釁,不可。」京惡之,以他事罷歸。



 旋知商州、興元府,入為駕部、金部員外郎。張商英得政,用為左司員外郎。俄擢給事中,會商英免相。蔡薿攝封駁,力沮止之,遘懼,請外。以直秘閣為河北轉運使,加直龍圖閣,徙陜西。召還京師,而蔡京復相,再使河北,徙淮南。帝將易置發運使,命選諸道計臣有閥閱者,執政以遘言,京曰:「職卑不可用,
 願更選。」帝曰:「可除集英殿修撰使往。」京乃不敢言。遂為副使,未幾,升為使。朝廷方督綱餉,運渠壅澀,遘使決呂城、陳公兩塘達於渠。漕路甫通,而朱勔花石綱塞道,官舟不得行。遘捕系其人,而上章自劾。帝為黥勔人,進遘徽猷閣待制。



 宣和二年冬,方臘亂,詔以屬遘。遘言:「臘始起青溪,眾不及千,今脅從已過萬,又有蘇州石生、歸安陸行兒,皆聚黨應之。東南兵弱勢單,士不習戰,必未能滅賊。願發京畿兵、鼎澧槍盾手,兼程以來,庶幾蜂起愚
 民。不至滋蔓。」帝悉行其言。



 加龍圖閣直學士,經制七路,治於杭。時縣官用度百出,遘創議度公私出納,量增其贏,號「經制錢」。其後總制使翁彥國仿其式,號「總制錢」。於是天下至今有「經總制錢」名,自兩人始也。



 又言:「妖賊陵暴州縣,唯搜求官吏,恣行殺戮。往往斷截支體,探取肺肝,或熬以鼎油,或射以勁矢,備極慘毒,不廠怨心。蓋貪污嗜利之人,倚法侵牟騷動,不知藝極。積有不平之氣,結於民心,一旦乘勢如此,可為悲痛!此風不除,必更生
 事。臣願採摭官吏奸贓尚仍舊習者,按治以聞,乞重置於理。」許之。



 又進學士,凡所施置,以御筆先下。於是劾越州王仲薿糾市民造金茶器,減直買軍糧券,而以私錢取之,仲薿坐黜。杭經巨寇後,河渠堙窒,邦人以水潦為病。前守數請於朝,皆以勞費輟役。遘以冬月檄真、揚、潤、楚諸郡,凡守閘綱卒,悉集治所。先是,當閉閘,群卒無以食,率凍餓不自聊。聞命,相率呼舞以來者二千人,用其力治河,不兩月畢,杭人利焉。



 徙河北都轉運使,進延康
 殿學士,歷知中山、真定、河間府。欽宗立,加資政殿學士,積官至光祿大夫。復為真定,又徙中山。金人再至,遘冒圍入城,堅壁拒守。詔康王領天下大元帥,命遘為兵馬元帥。受圍半年,外無援師。京都既陷,割兩河求和。遘弟光祿卿適至中山,臨城諭旨,遘遙語之曰:「主辱臣死。吾兄弟平居以名義自處,寧當賣國家為囚孥乎?」適泣曰:「兄但盡力,勿以弟為念。」



 遘呼總管使盡括城中兵擊賊,總管辭,遂斬以徇。又呼步將沙振往。振素有勇名,亦固
 辭,遘固遣之。振怒且懼,潛衷刃入府。遘妾定奴責其輒入,振立殺之,遂害遘於堂,及其子錫並僕妾十七人。長子鉅以官淮南獲免。振出,帳下卒噪而前曰:「大敵臨城,汝安得殺吾父?」執而捽裂之,身首無餘。城中無主,乃開門出降。金人入見其尸曰:「南朝忠臣也。」斂而葬諸鐵柱寺。建炎初,贈特進。



 遘性孝友,為人寬厚長者。任部刺史二十年,每出行郡邑,必焚香祈天,願不逢貪濁吏。嘗薦王安中、呂頤浩、張愨、謝克家、何鑄,後皆至公輔,世以為
 知人。



 適由開封少尹、衛尉少卿至光祿卿。是役也,金人執之以北。後十年,死於雲中。



 趙不試,太宗六世孫。宣和末,通判相州,尋權州事兼主管真定府路經略安撫公事。建炎元年,知相州。初,汪伯彥既去相,金人執其子似,遣來割地,似至相,不試固守不下。明年,金人大入。州久被圍,軍民無固志,不試謂之曰:「今城中食乏,外援不至。不試,宗子也,義不降,計將安出?」眾不應。不試知事不可為,遂登城與金人約勿殺,許
 之。既啟門,乃納其家井中,然後以身赴井,命提轄官實以土。州人皆免於死。


趙令
 \gezhu{
  山成}
 ,燕懿王玄孫,安定郡王令衿兄也。初名令裨。建炎初,仕至鄂州通判,領兵戍武昌。賊閻瑾犯黃州,縱掠而去。令
 \gezhu{
  山成}
 渡江存撫之,黃人乃安。李綱言於上,擢直龍圖閣、知黃州,賜今名。奉詔修城,凡六月而畢。賊張遇過城下,招令
 \gezhu{
  山成}
 。度不能拒,出城見之,遇飲以酒,一舉而盡,曰:「固知飲此必死,願勿殺軍民。」遇驚曰:「先以此試公耳。」
 更取毒酒沃地,地裂有聲,乃引軍去。未幾,丁進、李成兵迭至,俱擊卻之。叛將孔彥舟又引兵圍城,率民兵固守,凡六日乃解。


三年,以內艱去,詔起復。時金人聞孟太后在南昌,欲邀之,徑犯黃州。令
 \gezhu{
  山成}
 已還在道,郡卒得金人木笴鑿頭箭,浮江告急。令
 \gezhu{
  山成}
 疾趨,夜半入城。金人力攻,翼日城陷。金人欲降之,大罵不屈,酌以酒,揮之不肯飲,又衣以戰袍,曰:「我豈當服!」金人曰:「趙使君何堅執膝?」曰:「但當拜祖宗,豈能拜犬彘!」金人怒鞭之,流血被面,罵不絕
 口而死。事聞,贈徽猷閣待制,謚曰愍忠。州人乞立廟,從之。初,城破,都監王達、判官吳源、巡檢劉卓,皆以不屈死焉。



 唐重,字聖任,眉州彭山人。少有大志。大觀三年進士。徽宗親策士,問以制禮作樂,重對曰:「事親從兄,為仁義禮樂之實。陛下以神考為父,哲宗為兄,盍亦推原仁義之實而已,何以制作為?」授蜀州司理參軍,改成都府府學教授,知懷安軍金堂縣,授闢雍錄。



 先是,朝廷以拓土為
 功,邊帥爭興利以徼賞,凡蜀東西、夔峽路及荊湖、廣南,皆誘近邊蕃夷獻其地之不可耕者,謂之納土,因置州縣,所至騷然。重以其利害白之宰相,因是薦之,召對。遷吏部員外郎、左司郎官、起居舍人。



 金人入京師,重言:「開邊之禍,起於童貫,故金人以貫為禍首。若斬貫首,遣人傳送於金,尚可緩兵。」或獻議遠避,重聞衛士語,以告於朝,始定守城之計。擢右諫議大夫。時宰執各主和戰二議,重上疏乞命其廷辨得失。金人要求金帛,中書侍郎
 王孝迪下令,有匿金銀者死,許人告。重曰:「如此,則子得以告父,弟得以告兄,奴婢得以告主矣,豈初政所宜?」即與御史抗論,乃止。又累疏乞斬蔡京父子以謝天下。尋遷中書舍人,詞命多所繳奏。又言:「近世不次用人,其間致身宰輔,有未嘗一日出國門者。乞先補外,以為之倡。」上開納,而宰相執奏以為不可。明日,臺諫皆得罪,重落職知同州。



 金人已陷晉、絳,將及同。重度不能守,乃開門縱州人使出,自以殘兵數百守城,以示必死。金人疑有
 備,不復渡河而返。降詔獎諭,擢天章閣待制。先是,陜西宣撫使範致虛提五路兵勤王,至陜州。重遺致虛書,言:「中都倚秦兵為爪牙,諸夏恃京師為根本。今京城圍久,人無鬥志,若五路之師逡巡未進,則所以為爪牙者不足恃,而根本搖矣。然潰卒為梗,關中公私之積已盡;又聞西夏侵掠鄜延,為腹背患。今莫若移檄蜀帥及川峽四路,共資關中守禦之備,合秦、蜀以衛王室。」致虛銳於出師,由澠池屯千秋鎮,為金將所敗,軍皆潰,退保潼關,
 而五路之力益耗矣。重募人間道走京城歸報。二帝既北行,重即移檄川、秦十路帥臣,各備禮物往軍前迎奉。



 未幾,高宗即位,重上疏論今急務有四,大患有五。所謂急務者,以車駕西幸為先,次則建藩鎮、封宗子,通夏國之好,繼青唐之後,使相犄角,以緩敵勢。所謂大患者,法令滋彰,朝綱委靡,軍政敗壞,國用竭,民心離。欲救此者,宜守祖宗成憲,登用忠直,大正賞刑,誠今日之急務。



 長安謀帥,劉岑自河東使還,上亦詢可守關中者,岑以重
 對,乃以天章閣直學士知京兆府,尋兼京兆府路經略制置使。



 重前在同州,凡三疏上大元帥府,乞早臨關中以符眾望。且畫三策:一謂鎮撫關中以固根本,然後營屯於漢中,開國於西蜀,此為策之上;若駐節南陽,控楚、吳、越、齊、趙、魏之師,以臨秦、晉之墟,視敵強弱為進退,選宗親賢明者開府於關中,此為策之次;儻因都城,再治城池汴、洛之境,據成皋、崤函之險,悉嚴防守,此策之下;若引兵南度,則國勢微弱,人心離散,此最無策。暨至永
 興,又六上疏,皆以車駕幸關中為請。並條奏關中防河事宜,大意謂:虢、陜殘破,解州、河中已陷,同、華州沿河與金人對壘,邊面亙六百餘里。本路無可戰之兵,乞增以五路兵馬十萬以上,委漕臣儲偫以守關中。



 章凡七八上,朝廷未有所處。重復上疏曰:「關中百二之勢,控制陜西六路,捍蔽川峽四路。今蒲、解失守,與敵為鄰,關中固,則可保秦、蜀十路無虞。緣逐路帥守、監司各有占護,不相通融。昨範致虛會合勤王之師,非不竭力,而將帥各
 自為謀,不聽節制。乞選宗親賢明者充京兆牧,或置元帥府,令總管秦、蜀十道兵馬以便宜從事,應帥守、監司並聽節制。緩急則合諸道之兵以衛社稷,不惟可以禦敵,亦可以救郡縣瓦解之失。」又乞節制五路兵,俱不報。



 金將婁宿渡河陷韓城縣,時京兆餘兵皆為經制使錢蓋調赴行在。重度勢不可支,以書別其父克臣曰:「忠孝不兩立,義不茍生以辱吾父。」克臣報之曰:「汝能以身徇國,吾含笑入地矣。」及金人入境,重遺書轉運使李唐孺曰:「
 重平生忠義,不敢辭難。始意迎車駕入關,居建瓴之勢,庶可以臨東方。今車駕南幸矣,關陜又無重兵,雖竭智力何所施,一死報上不足惜。」



 及金兵圍城,城中兵不滿千,固守逾旬,外援不至。而經制副使傅亮以精銳數百奪門出降,城陷,重以親兵百人血戰。諸將扶重去,重曰:「死吾職也。」戰不已,眾潰,重中流矢死。初,唐孺以其書聞,俄以死節報。上哀悼之,贈資政殿學士,後謚恭愍。



 郭忠孝,字立之,河南人,簽書樞密院事逵之子。受《易》、《中
 庸》於程頤。少以父任補右班殿直,遷右侍禁。登進士第,換文資,授將作監主簿。年逾三十,不忍去親側,多仕於河南筦庫間。宣和間,為河東路提舉。解梁、猗氏與河東接壤,盜販鹽者數百為群,歲起大獄,轉相告引,抵罪者眾。忠孝止治其首,餘悉寬貸。宰相王黼怒之,坐廢格鹽法免。



 靖康初,召為軍器少監。入對,以和議為非是,力陳追擊之策,謂:「兵家忌深入,金人自燕薊興兵,逾河朔,犯都城,其鋒不可當,今銳氣且衰,又顧子女玉帛之獲,故
 議和以款我師。今諸道之師集矣,宜乘其惰擊之,若不能擊其歸,他日安能御其來。」上命與宰相吳敏、樞密李綱議,忠孝復條上戰守利害、士馬分合之策十餘事。主和者眾,卒不用其策。改永興軍路提點刑獄,措置保甲。初,議者請擇保甲十萬刺為義勇,分隸河朔諸郡。忠孝曰:「保甲歲久,死亡者眾,擇三萬人守都城可也,河朔騎兵之地,非保甲所宜。」上從之。忠孝亟走關陜,得勝兵三萬,分隸十將,擇一將統之。繼遣兵趨澤、潞,聽宣撫司節
 制。



 金人再犯京師,永興帥範致虛率諸軍繇淆、澠入援,忠孝曰「金人深入,而河東無守備,願分兵走太行,扼其歸路,彼必來戰,城下之圍可緩。」致虛以為然。檄河中守席益、馮翊守唐重與忠孝同出河東,為牽制之舉,大軍盡出函谷。忠孝獨以蒲、解軍三千至猗氏,遇金人,破之。逾絳州,破太平砦,斬首數百級。攻平陽,入其郛。會大軍失利淆、澠間,乃引還。



 及金人犯永興,兵寡,或勸忠孝以監司出巡,可以避禍。忠孝不答,與經略唐重分城而守。
 忠孝主西壁,唐重主東壁。金人陳城下,忠孝募人以神臂弓射之,敵不得前。已而攻陷城東南隅,忠孝與重及副總管楊宗閔、轉運副使桑景詢、判官曾謂、經略主管機宜文字王尚、提舉軍馬武功大夫程迪俱死之。朝廷贈忠孝大中大夫。子雍,別有傳。



 程迪,字惠老,開封人。父博古,部鄜延兵戰死永樂。迪以門蔭得官。宣和中,從楊惟中徵方臘有功,加武功大夫、榮州團練使、瀘南潼川府路走馬承受公事。



 諸使合薦
 迪忠義謀略,可任將帥,召赴行在。經略制置使唐重以敵迫近,留迪提舉軍馬,措置民兵以為備。金人已自同州渡河,或勸迪還蜀,迪思有以報國,不從。乃詣種氏諸豪,謀率眾保險,俟其勢稍衰,出奇擊之。轉運使桑景詢知其謀,以告唐重,揭榜許民擇險自固。會前河東經制使傅亮建議當守不當避,重從之,以亮為制置副使,去者悉還。



 既而金兵益迫,重乃以迪提舉永興路軍馬,措置民兵,令迪行視南山諸谷,將運金帛徙治其中。因召
 土豪,集民兵以補軍籍。會應募者眾,亮語重曰:「人心如此,假以旬日,守備且具,奈何望風棄去。」重大然之,即檄諸司聽亮節制。金人近城,迪又欲選兵迎戰,使老稚得趣險,尚可以活十萬人。亮執議城守,金人四面急攻,外無援兵,迪率諸司及統制偏裨以下東鄉會盟:「危急必以死相應,誓不與敵俱生。」慷慨嗚咽,同盟皆感泣。城破,乃自亮所分地始。亮先出降,眾潰。迪率其徒行徇於眾曰:「敵仇我矣,降亦死,戰亦死!」努力與鬥,憤怒大呼,口流
 血,士皆感奮,多所斬殺。迪冒飛矢,持短兵接戰數十合,身被創幾遍,絕而復蘇,猶厲聲叱戰不已,遂死之。麾下士舁置空室中,比屋皆燼,室獨不火,及斂,容色如生。詔贈明州觀察使,謚恭愍。子昌諤。



 徐徽言,字彥猷,衢之西安人。少為諸生,泛涉書傳。負氣豪舉,有奇志,喜談功名事。大觀二年,詔求材武士,韓忠彥、范純粹、劉仲武以徽言應詔,召見崇德殿,賜武舉絕倫及第。



 歷保德軍監押,以邊功加閣門祗候、平陽府軍
 馬鈴轄,權知保德軍。改總領河西軍馬,以討西夏功,累遷秉義郎。宣和四年,將伐燕,命太原帥張孝純招河西帳族,遣徽言入其地。帳族拒而射之,徽言迎戰破之,遂定天德、雲內兩城。宣撫使童貫嫉其功,檄太原不得違節度。復棄去。孝純先定朔、武二州,亦不能守。改知火山軍兼統制河西軍馬,徙赴石州。



 靖康初,遷武翼郎、閣門宣贊舍人。金人圍太原,分兵絕饟道,自隰、石以北,命令不通者累月。徽言以三十人渡河,一戰破之。遷武經郎、
 知晉寧軍兼嵐石路沿邊安撫使。



 金人再犯京師,陜西制置使範致虛糾合五路兵赴難,檄徽言守河西。欽宗割兩河以紓禍,同知樞密院事聶昌出河東,為金人所劫,以便宜割河西三州隸西夏。晉寧軍民大恐,曰:「棄麟、府、豐,晉寧豈能獨存!」徽言曰:「此使人矯詔耳。三郡在河西,設有詔,猶當執奏,況無之耶!」遂率兵復取三州,夏人所置守長皆出降,徽言慰遣之。又並取嵐、石凳州,教戈舡卒乘羊皮渾脫亂流以掩敵。金人益備克胡砦、吳堡
 津,遣守領為九州都統,與晉寧對壘。徽言出奇兵襲逐之。時河東郡縣淪沒,遺民日徯王師之至。徽言陰結汾、晉土豪數十萬,約復故地則奏官為守長,聽世襲。條其事以聞,俟報可,即身率精甲搗太原,徑取雁門,留兵戍守;且曰:「定全晉則形勝為我有,中原當指期克復,投機一時,會不可失。」奏上,詔徽言聽王庶節制,議遂格。



 金人忌徽言,欲速拔晉寧以除患。建炎二年冬,自蒲津涉河圍之。先是徽言移府州,約折可求夾攻金人。可求降,金
 將婁宿挾至城下以招徽言。徽言故與可求為姻,乃登陴以大義噍數之。可求仰曰:「君於我胡大無情?」徽言攝弓厲言曰:「爾於國家不有情,我尚於爾何情?寧惟我無情,此矢尤無情。」一發中之,可求走,因出兵縱擊,遂斬婁宿孛堇之子。當是時,環河東皆已陷,獨晉寧屹然孤墉,橫當強敵,勢相百不抗。徽言堅壁持久,撫摩疲傷,遣沒人泅河,召民之逃伏山谷者幾萬眾,浮筏西渡,與金人鏖河上,大小數十戰,所俘殺過當。晉寧號天下險,徽言
 廣外城,東壓河,下塹不測,譙堞雄固,備械甚整。命諸將畫隅分守,敵至則自致死力,以勁兵往來為游援。



 金進攻數敗,不得志,圍之益急。晉寧俗不井飲,寄汲於河。金人載茭石湮壅支流,城中水乏絕,儲偫浸罄,鎧仗空敝,人人惴憂,知殞亡無日。徽言能得眾心,奮枵餓傷夷之餘,裒折槊斷刃,以死固守。既自度不支,取炮機、篦格,凡守具悉火之,曰:「無以遺敵。」遣人間道馳書其兄昌言曰:「徽言孤國恩死矣,兄其勉事君。」一夕,裨校李位、石贇系
 帛書飛笴上,陰約婁宿啟外郭納金兵。徽言與太原路兵馬都監孫昂決戰門中,所格殺甚眾,退嬰牙城以守。金人攻之不已,徽言置妻子室中,積薪自焚。仗劍坐堂上,慷慨語將士:「我天子守土臣,義不見蔑敵手。」因拔佩刀自擬,左右號救持之急,金兵猥至,挾徽言以去,然猶憚其威名。



 婁宿得徽言所親說徽言:「盍具冠韍見金帥。」徽言斥曰:「朝章,覲君父禮,以入穹廬可乎?汝污偽官,不即愧死,顧以為榮,且為敵人搖吻作說客耶?不急去,吾
 力猶能搏殺汝。」婁宿就見徽言,語曰:「二帝北去,爾其為誰守此?」徽言曰:「吾為建炎天子守。」婁宿曰:「我兵己南矣,中原事未可知,何自苦為?」徽言怒曰:「吾恨不尸汝輩歸見天子,將以死報太祖、太宗地下,庸知其他!」婁宿又出金制曰:「能小屈,當使汝世帥延安,舉陜地並有之。」徽言益怒,罵曰:「吾荷國厚恩,死正吾所,此膝詎為汝輩屈耶?汝當親刃我,不可使餘人見加。」婁宿舉戟向之,覬其懼狀。徽言披衽迎刃,意象自若。飲以酒,持杯擲婁宿曰:「我
 尚飲汝酒乎?」慢罵不已。金人知不可屈,遂射殺之。粘罕聞其死,怒婁宿曰:「爾粗狠,何專殺義人以逞爾私?」治其罪甚慘。



 初,徽言與劉光世束發雅故。光世被命援太原,次吳堡津,輒頓不進。徽言移書趣行,未聽;又諭以太原危不守,旦暮望救,總管承詔赴急,不宜稽固取方命罪,光世猶前卻。徽言即露章劾其逗撓,封副與之,光世惶遽引道。



 宣撫使張浚與諸使者相繼以死節事聞,高宗撫幾震悼,顧謂宰相曰:「徐徽言報國死封疆,臨難不屈,
 忠貫日月,過於顏真卿、段秀實遠矣。不有以寵之,何以勸忠,昭示來世。」乃贈晉州觀察使,謚忠壯。再贈彰化軍節度。



 孫昂,亦引刀欲自刺,金人擁至軍前,不屈而死,至是贈成忠郎、圍練使。徽言子岡既同死事,而從孫適亦以守安豐死。昂父翊,宣和末知朔寧府,救太原,死於陣。各世著忠義云。



 向子韶,字和卿,開封人,神宗後再從侄也。年十五入太學,登元符三年進士第。特恩改承事郎,授荊南府節度
 判官,累官至京東轉運副使。屬郡郭奉世進萬緡羨餘,戶部聶昌請賞之以勸天下。子韶劾奉世,且言近臣首開聚斂之端,浸不可長,士論韙之。以父憂免,起復,知淮寧府。



 建炎二年,金人犯淮寧,子韶率諸弟城守,諭士民曰:「汝等墳墓之國,去此何之,吾與汝當死守。」時有東兵四千人,第三將岳景綬欲棄城率軍民走行在,子韶不從,景綬引兵迎敵而死。金人晝夜攻城,子韶親擐甲胄,冒矢石,遣其弟子率赴宗澤乞援兵,未至,城陷。子韶率
 軍民巷戰,力屈為所執。金人坐城上,欲降之,酌酒於前,左右抑令屈膝,子韶直立不動,戟手責罵,金人殺之。其弟新知唐州子褒、朝請郎子家等與闔門皆遇害,惟一子鴻六歲得存。事聞,再贈通議大夫,官其家六人,後謚忠毅。初,金人至淮寧府,楊時聞之曰:「子韶必死矣。」蓋知其素守者云。



 楊邦乂,字晞稷,吉州吉水人。博通古今,以舍選登進士第,遭時多艱,每以節義自許。歷婺源尉、蘄廬建康三郡
 教授,改秩知溧陽縣。會叛卒周德據府城,殺官吏。邦乂立縣獄囚趙明於庭,欲誅之,因諭之曰:「爾悉里中豪傑,誠能集爾徒為邑人誅賊,不惟宥爾罪,當上功畀爵。」明即請行,邦乂飲之卮酒,使自去。越翼日,討平之。



 建炎三年,金人至江上。高宗如浙西,留右僕射杜充為御營使,駐扎建康,命劉光世、韓世忠、王𤫉諸將悉聽充節制。充性酷而無謀,士心不附。渡碙沙,充遣陳淬、岳飛等及金人戰於馬家渡。自辰至未,戰數合,勝負未決。𤫉擁兵弗
 救,淬被擒,𤫉兵遁,充率麾下數千人降。金人濟江,鼓行逼城。時李棁以戶部尚書董軍餉,陳邦光以顯謨閣直學士守建康,皆具降狀,逆之十里亭。金帥完顏宗弼既入城,棁、邦光率官屬迎拜,惟邦乂不屈膝,以血大書衣裾曰:「寧作趙氏鬼,不為他邦臣。」宗弼不能屈。翼曰,遣人說邦乂,許以舊官。邦乂以首觸柱礎流血,曰:「世豈有不畏死而可以利動者?速殺我。」翼日,宗弼等與棁、邦光宴堂上,立邦乂於庭,邦乂叱棁、邦光曰:「天子以若捍城,敵
 至不能抗,更與共宴樂,尚有面目見我乎?」有劉團練者,以幅紙書「死活」二字示邦乂曰:「若無多云,欲死趣書『死』字。」邦乂奮筆書「死」字,金人相顧動色,然未敢害也。已而宗弼再引邦乂,邦乂不勝憤,遙望大罵曰:「若女真圖中原,天寧久假汝,行磔汝萬段,安得污我!」宗弼大怒,殺之,剖取其心,年四十四。事聞,贈直秘閣,賜田三頃,官為斂葬,即其地賜廟褒忠,謚忠襄,官其四子。



 邦乂少處郡學,目不視非禮,同舍欲隳其守,拉之出,托言故舊家,實娼
 館也。邦乂初不疑,酒數行,娼女出,邦乂愕然,疾趨還舍,解其衣冠焚之,流涕自責。紹興七年,樞密院言邦乂忠節顯著,上曰:「顏真卿異代忠臣,朕昨已官其子孫,邦乂為朕死節,不可不厚褒錄,以為忠義之勸。」加贈徽猷閣待制,增賜田三頃。



\end{pinyinscope}