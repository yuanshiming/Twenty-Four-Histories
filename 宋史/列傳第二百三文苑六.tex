\article{列傳第二百三文苑六}

\begin{pinyinscope}

 ○黃庭堅晁補之弟詠之秦觀張耒陳師道李廌劉恕王無咎蔡肇李格非呂南公郭祥正
 米芾劉詵倪濤李公麟周邦彥朱長文劉弇



 黃庭堅字魯直,洪州分寧人。幼警悟,讀書數過輒成誦。舅李常過其家,取架上書問之,無不通,常驚,以為一日千里。舉進士,調葉縣尉。熙寧初,舉四京學官,第文為優,教授北京國子監,留守文彥博才之,留再任。蘇軾嘗見其詩文,以為超軼絕塵,獨立萬物之表,世久無此作,由是聲名始震。知太和縣,以平易治。時課頒鹽筴,諸縣爭
 占多數,太和獨否,吏不悅,而民安之。



 哲宗立,召為校書郎、《神宗實錄》檢討官。逾年,遷著作佐郎,加集賢校理。《實錄》成,擢起居舍人。丁母艱。庭堅性篤孝,母病彌年,晝夜視顏色,衣不解帶。及亡,廬墓下,哀毀得疾幾殆。服除,為秘書丞,提點明道宮兼國史編修官。紹聖初,出知宣州,改鄂州。章惇、蔡卞與其黨論《實錄》多誣,俾前史官分居畿邑以待問,摘千餘條示之,謂為無驗證。既而院吏考閱,悉有據依,所餘才三十二事。庭堅書「用鐵龍爪治河,
 有同兒戲」,至是首問焉。對曰:「庭堅時官北都,嘗親見之,真兒戲耳。」凡有問,皆直辭以對,聞者壯之。貶涪州別駕、黔州安置,言者猶以處善地為法。以親嫌,遂移戎州。庭堅泊然,不以遷謫介意。蜀士慕從之游,講學不倦,凡經指授,下筆皆可觀。



 徽宗即位,起監鄂州稅,簽書寧國軍判官,知舒州,以吏部員外郎召,皆辭不行。丐郡,得知太平州,至之九日,罷主管玉隆觀。庭堅在河北與趙挺之有微隙,挺之執政,轉運判官陳舉承風旨,上其所作《
 荊南承天院記》,指為幸災,復除名、羈管宜州。三年,徙永州,未聞命而卒,年六十一。



 庭堅學問文章,天成性得,陳師道謂其詩得法杜甫,學甫而不為者。善行、草書,楷法亦自成一家。與張耒、晁補之、秦觀俱游蘇軾門,天下稱為四學士,而庭堅於文章尤長於詩,蜀、江西君子以庭堅配軾,故稱「蘇、黃」。軾為侍從時,舉以自代,其詞有「瑰偉之文,妙絕當世,孝友之行,追配古人」之語,其重之也如此。初,游灊皖山谷寺、石牛洞,樂其林泉之勝,因自號山
 谷道人云。



 晁補之,字無咎,濟州鉅野人,太子少傅迥五世孫,宗愨之曾孫也。父端友,工於詩。補之聰敏強記,才解事即善屬文,王安國一見奇之。十七歲從父官杭州,稡錢塘山川風物之麗,著《七述》以謁州通判蘇軾。軾先欲有所賦,讀之嘆曰:「吾可以閣筆矣!」又稱其文博辯雋偉,絕人遠甚,必顯於世。由是知名。



 舉進士,試開封及禮部別院,皆第一。神宗閱其文曰:「是深於經術者,可革浮薄。」調澶州
 司戶參軍,北京國子監教授。元祐初,為太學正,李清臣薦堪館閣,召試,除秘書省正字,遷校書郎,以秘閣校理通判揚州,召還,為著作佐郎。章惇當國,出知齊州,群盜晝掠途巷。補之默得其姓名、囊橐皆審,一日宴客,召賊曹以方略授之,酒行未竟,悉擒以來,一府為徹警。坐修《神宗實錄》失實,降通判應天府、亳州,又貶監處、信二州酒稅。徽宗立,復以著作召。既至,拜吏部員外郎、禮部郎中,兼國史編修、實錄檢討官。黨論起,為諫官管師仁所
 論,出知河中府,修河橋以便民,民畫祠其像。徙湖州、密州、果州,遂主管鴻慶宮。還家,葺歸來園,自號歸來子,忘情仕進,慕陶潛為人。大觀末,出黨籍,起知達州,改泗州,卒,年五十八。



 補之才氣飄逸,嗜學不知倦,文章溫潤典縟,其凌麗奇卓出於天成。尤精《楚詞》,論集屈、宋以來賦詠為《變離騷》等三書。安南用兵,著《罪言》一篇,大意欲擇仁厚勇略吏為五管郡守,及修海上諸郡武備,議者以為通達世務。從弟詠之。



 詠之字之道,少有異材,以蔭入官。調揚州司法參軍,未上。時蘇軾守揚州,補之倅州事,以其詩文獻軾,軾曰:「有才如此,獨不令我一識面邪?」乃具參軍禮入謁,軾下堂挽而上,顧坐客曰:「奇才也!」復舉進士,又舉宏詞,一時傳誦其文。為河中教授,元符末,應詔上書論事,罷官。久之,為京兆府司錄事,秩滿,提點崇福宮,卒,年五十二,有文集五十卷。



 秦觀,字少游,一字太虛,揚州高郵人。少豪雋,慷慨溢於
 文詞,舉進士不中。強志盛氣,好大而見奇,讀兵家書與己意合。見蘇軾於徐,為賦黃樓,軾以為有屈、宋才。又介其詩於王安石,安石亦謂清新似鮑、謝。軾勉以應舉為親養,始登第,調定海主簿、蔡州教授。元祐初,軾以賢良方正薦於朝,除太學博士,校正秘書省書籍。遷正字,而復為兼國史院編修官,上日有硯墨器幣之賜。



 紹聖初,坐黨籍,出通判杭州。以御史劉拯論其增損實錄,貶監處州酒稅。使者承風望指,候伺過失,既而無所得,則以
 謁告寫佛書為罪,削秩徙郴州,繼編管橫州,又徙雷州。徽宗立,復宣德郎,放還。至藤州,出游華光亭,為客道夢中長短句,索水欲飲,水至,笑視之而卒。先自作挽詞,其語哀甚,讀者悲傷之。年五十三,有文集四十卷。



 觀長於議論,文麗而思深。及死,軾聞之嘆曰:「少游不幸死道路,哀哉!世豈復有斯人乎!」弟覿字少章,覯字少儀,皆能文。



 張耒,字文潛,楚州淮陰人。幼穎異,十三歲能為文,十七時作《函關賦》,已傳人口。游學於陳,學官蘇轍愛之,因得
 從軾游,軾亦深知之,稱其文汪洋沖澹,有一倡三嘆之聲。



 弱冠第進士,歷臨淮主簿、壽安尉、咸平縣丞。入為太學錄,范純仁以館閣薦試,遷秘書省正字、著作佐郎、秘書丞、著作郎、史館檢討。居三館八年,顧義自守,泊如也。擢起居舍人。紹聖初,請郡,以直龍圖閣知潤州。坐黨籍,徙宣州,謫監黃州酒稅,徙復州。徽宗立,起為通判黃州,知兗州,召為太常少卿,甫數月,復出知潁州、汝州。崇寧初,復坐黨籍落職,主管明道宮。初,耒在潁,聞蘇軾訃,為舉
 哀行服,言者以為言,遂貶房州別駕,安置於黃。五年,得自便,居陳州。



 耒儀觀甚偉,有雄才,筆力絕健,於騷詞尤長。時二蘇及黃庭堅、晁補之輩相繼沒,耒獨存,士人就學者眾,分日載酒肴飲食之。誨人作文以理為主,嘗著論云:「自《六經》以下,至於諸子百氏騷人辯士論述,大抵皆將以為寓理之具也。故學文之端,急於明理,如知文而不務理,求文之工,世未嘗有也。夫決水於江、河、淮、海也,順道而行,滔滔汨汨,日夜不止,沖砥柱,絕呂梁,放於
 江湖而納之海,其舒為淪漣,鼓為波濤,激之為風飆,怒之為雷霆,蛟龍魚鱉,噴薄出沒,是水之奇變也。水之初,豈若是哉!順道而決之,因其所遇而變生焉。溝瀆東決而西竭,下滿而上虛,日夜激之,欲見其奇,彼其所至者,蛙蛭之玩耳。江、河、淮、海之水,理達之文也,不求奇而奇至矣。激溝瀆而求水之奇,此無見於理,而欲以言語句讀為奇,反覆咀嚼,卒亦無有,文之陋也。」學者以為至言。作詩晚歲益務平淡,效白居易體,而樂府效張籍。



 久於
 投閑,家益貧,郡守翟汝文欲為買公田,謝不取。晚監南嶽廟,主管崇福宮,卒,年六十一。建炎初,贈集英殿修撰。



 陳師道,字履常,一字無己,彭城人。少而好學苦志,年十六,摎以文謁曾鞏,鞏一見奇之,許其以文著,時人未之知也,留受業。熙寧中,王氏經學盛行,師道心非其說,遂絕意進取。鞏典五朝史事,得自擇其屬,朝廷以白衣難之。元祐初,蘇軾、傅堯俞、孫覺薦其文行,起為徐州教援,又用梁燾薦,為太學博士。言者謂在官嘗越境出南京見
 軾,改教授潁州。又論其進非科第,罷歸。調彭澤令,不赴。家素貧,或經日不炊,妻子慍見,弗恤也。久之,召為秘書省正字,卒,年四十九,友人鄒浩買棺斂之。



 師道高介有節,安貧樂道。於諸經尤邃《詩》、《禮》,為文精深雅奧。喜作詩,自云學黃庭堅,至其高處,或謂過之,然小不中意,輒焚去,今存者才十一。世徒喜誦其詩文,至若奧學至行,或莫之聞也。嘗銘黃樓,曾子固謂如秦石。



 初,游京師逾年,未嘗一至貴人之門,傅堯俞欲識之,先以問秦觀,觀曰:「是
 人非持刺字、俯顏色、伺候乎公卿之門者,殆難至也。」堯俞曰:「非所望也,吾將見之,懼其不吾見也,子能介於陳君乎?」知其貧,懷金欲為饋,比至,聽其論議,益敬畏,不敢出。章惇在樞府,將薦於朝,亦屬觀延致。師道答曰:「辱書,諭以章公降屈年德,以禮見招,不佞何以得此,豈侯嘗欺之耶?公卿不下士,尚矣,乃特見於今而親於其身,幸孰大焉。愚雖不足以齒士,猶當從侯之後,順下風以成公之名。然先王之制,士不傳贄為臣,則不見於王公,所
 以成禮而其敝必至自鬻,故先王謹其始以為之防,而為士者世守焉。師道於公,前有貴賤之嫌,後無平生之舊,公雖可見,禮可去乎?且公之見招,蓋以能守區區之禮也,若昧冒法義,聞命走門,則失其所以見招,公又何取焉。雖然,有一於此,幸公之他日成功謝事,幅巾東歸,師道當御款段,乘下澤,候公於東門外,尚未晚也。」及惇為相,又致意焉,終不往。官潁時,蘇軾知州事,待之絕席,欲參諸門弟子間,而師道賦詩有「向來一瓣香,敬為曾南
 豐」之語,其自守如是。



 與趙挺之友婿,素惡其人,適預郊祀行禮,寒甚,衣無綿,妻就假於挺之家,問所從得,卻去,不肯服,遂以寒疾死。



 李廌,字方叔,其先自鄆徙華。廌六歲而孤,能自奮立,少長,以學問稱鄉里。謁蘇軾於黃州,贄文求知。軾謂其筆墨瀾翻,有飛沙走石之勢,拊其背曰:「子之才,萬人敵也,抗之以高節,莫之能御矣。」廌再拜受教。而家素貧,三世未葬,一夕,撫枕流涕曰:「吾忠孝焉是學,而親未葬,何以
 學為!」旦而別軾,將客游四方,以蕆其事。軾解衣為助,又作詩以勸風義者。於是不數年,盡致累世之喪三十餘柩,歸窆華山下,範鎮為表墓以美之。益閉門讀書,又數年,再見軾,軾閱其所著,嘆曰:「張耒、秦觀之流也。」



 鄉舉試禮部,軾典貢舉,遺之,賦詩以自責。呂大防嘆曰:「有司試藝,乃失此奇才耶!」軾與範祖禹謀曰:「廌雖在山林,其文有錦衣玉食氣,棄奇寶於路隅,昔人所嘆,我曹得無意哉!」將同薦諸朝,未幾,相繼去國,不果。軾亡,廌哭之慟,曰:「吾
 愧不能死知己,至於事師之勤,渠敢以生死為間!」即走許、汝間,相地卜兆授其子,作文祭之曰:「皇天后土,鑒一生忠義之心;名山大川,還萬古英靈之氣。」詞語奇壯,讀者為悚。中年絕進取意,謂潁為人物淵藪,始定居長社,縣令李佐及里人買宅處之。卒,年五十一。



 廌喜論古今治亂,條暢曲折,辯而中理。當喧溷倉卒間如不經意,睥睨而起,落筆如飛馳。元祐求言,上《忠諫書》、《忠厚論》並獻《兵鑒》二萬言論西事。朝廷擒羌酋鬼章,將致法,廌深論
 利害,以為殺之無益,願加寬大,當時韙其言。



 劉恕,字道原,筠州人。父渙字凝之,為潁上令,以剛直不能事上官,棄去。家於廬山之陽,時年五十。歐陽修與渙,同年進士也,高其節,作《廬山高》詩以美之。渙居廬山三十餘年,環堵蕭然,饘粥以為食,而游心塵垢之外,超然無戚戚意,以壽終。



 恕少穎悟,書過目即成誦。八歲時,坐客有言孔子無兄弟者,恕應聲曰:「以其兄之子妻之。」一坐驚異。年十三,欲應制科,從人假《漢》、《唐書》,閱月皆歸之。
 謁丞相晏殊,問以事,反覆詰難,殊不能對。恕在鉅鹿時,召至府,重禮之,使講《春秋》,殊親帥官屬往聽。未冠,舉進士,時有詔,能講經義者別奏名,應詔者才數十人,恕以《春秋》、《禮記》對,先列注疏,次引先儒異說,末乃斷以己意,凡二十問,所對皆然,主司異之,擢為第一。他文亦入高等,而廷試不中格,更下國子試講經,復第一,遂賜第。調鉅鹿主簿、和川令,發強擿伏,一時能吏自以為不及。恕為人重意義,急然諾。郡守得罪被劾,屬吏皆連坐下獄,
 恕獨恤其妻子,如己骨肉,又面數轉運使深文峻詆。



 篤好史學,自太史公所記,下至周顯德末,紀傳之外至私記雜說,無所不覽,上下數千載間,鉅微之事,如指諸掌。司馬光編次《資治通鑒》,英宗命自擇館閣英才共修之。光對曰:「館閣文學之士誠多,至於專精史學,臣得而知者,唯劉恕耳。即召為局僚,遇史事紛錯難治者,輒以諉恕。恕於魏、晉以後事,考證差繆,最為精詳。



 王安石與之有舊,欲引置三司條例。恕以不習金穀為辭,因言天子
 方屬公大政,宜恢張堯、舜之道以佐明主,不應以利為先。又條陳所更法令不合眾心者,勸使復舊,至面刺其過,安石怒,變色如鐵,恕不少屈。或稠人廣坐,抗言其失無所避,遂與之絕。方安石用事,呼吸成禍福,高論之士,始異而終附之,面譽而背毀之,口順而心非之者,皆是也。恕奮厲不顧,直指其事,得失無所隱。



 光出知永興軍,恕亦以親老,求監南康軍酒以就養,許即官修書。光判西京御史臺,恕請詣光,留數月而歸。道得風攣疾,右手
 足廢,然苦學如故,少間,輒修書,病亟乃止。官至秘書丞,卒,年四十七。



 恕為學,自歷數、地里、官職、族姓至前代公府案牘,皆取以審證。求書不遠數百里,身就之讀且抄,殆忘寢食。偕司馬光游萬安山,道旁有碑,讀之,乃五代列將,人所不知名者,恕能言其行事始終,歸驗舊史,信然。宋次道知亳州,家多書,恕枉道借覽。次道日具饌為主人禮,恕曰:「此非吾所為來也,殊廢吾事。」悉去之。獨閉閣,晝夜口誦手抄,留旬日,盡其書而去,目為之翳。著《五
 代十國紀年》以擬《十六國春秋》,又採太古以來至周威烈王時事,《史記》、《左氏傳》所不載者,為《通鑒外紀》。



 家素貧,無以給旨甘,一毫不妄取於人。自洛南歸,時方冬,無寒具。司馬光遺以衣襪及故茵褥,辭不獲,強受而別,行及潁,悉封還之。尤不信浮屠說,以為必無是事,曰:「人如居逆旅,一物不可乏,去則盡棄之矣,豈得齎以自隨哉?」好攻人之惡,每自訟平生有二十失、十八蔽,作文以自警,亦終不能改也。



 死後七年,《通鑒》成,追錄其勞,官其子羲
 仲為郊社齋郎。次子和仲有超軼材,作詩清奧,刻厲欲自成家,為文慕石介,有俠氣,亦摎死。



 王無咎,字補之,建昌南城人。第進士,為江都尉、衛真主簿、天臺令,棄而從王安石學,久之,無以衣食其妻子,復調南康主簿,已又棄去。好書力學,寒暑行役不暫釋,所在學者歸之,去來常數百人。王安石為政,無咎至京師,士大夫多從之游,有卜鄰以考經質疑者。然與人寡合,常閉門治書,惟安石言論莫逆也。安石上章薦其才行該
 備,守道安貧,而久棄不用,詔以為國子直講,命未下而卒,年四十六。



 蔡肇,字天啟,潤州丹陽人。能為文,最長歌詩。初事王安石,見器重。又從蘇軾游,聲譽益顯。第進士,歷明州司戶參軍、江陵推官。元祐中,為太學正,通判常州,召為衛尉寺丞,提舉永興路常平。徽宗初,入為戶部員外郎,兼編修國史,言者論其學術反覆,提舉兩浙刑獄。張商英當國,引為禮部員外,進起居郎,拜中書舍人。前此,試三題,
 率以宰相上馬為之候,肇援筆立就,不加潤飾,商英讀之擊節。才逾月,以草御史幸義責詞不稱,罷為顯謨閣待制、知明州,言者又論其包藏異意,非議闢雍以為不當立,奪職,提舉洞霄宮。會赦,復之,卒。



 李格非,字文叔,濟南人。其幼時,俊警異甚。有司方以詩賦取士,格非獨用意經學,著《禮記說》至數十萬言,遂登進士第。調冀州司戶參軍,試學官,為鄆州教授,郡守以其貧,欲使兼他官,謝不可。入補太學錄,再轉博士,以文
 章受知於蘇軾。常著《洛陽名園記》,謂「洛陽之盛衰,天下治亂之候也」。其後洛陽陷於金,人以為知言。紹聖立局編元祐章奏,以為檢討,不就,戾執政意,通判廣信軍。有道士說人禍福或中,出必乘車,氓俗信惑,格非遇之途,叱左右取車中道士來,窮治其奸,杖而出諸境。召為校書郎,遷著作佐郎、禮部員外郎,提點京東刑獄,以黨籍罷,卒,年六十一。



 格非苦心工於詞章,陵轢直前,無難易可否,筆力不少滯。嘗言:「文不可以茍作,誠不著焉,則不
 能工。且晉人能文者多矣,至劉伯倫《酒德頌》、陶淵明《歸去來辭》,字字如肺肝出,遂高步晉人之上,其誠著也。」



 妻王氏,拱辰孫女,亦善文。女清照,詩文尤有稱於時,嫁趙挺之之子明誠,自號易安居士。



 呂南公,字次儒,建昌南城人。於書無所不讀,於文不肯綴緝陳言。熙寧中,士方推崇馬融、王肅、許慎之業,剽掠補拆臨摹之藝大行,南公度不能逐時好,一試禮闈不偶,退築室灌園,不復以進取為意。益著書,且借史筆以
 褒善貶惡,遂以「袞斧」名所居齋。嘗謂士必不得已於言,則文不可以不工,蓋意有餘而文不足,則如吃人之辨訟,必未始不虛,理未始不直,然而或屈者,無助於辭而已。觀書契以來,特立之士,未有不善於文者。士無志於立則已,必有志焉,則文何可以卑賤而為之?故毅然盡心,思欲與古人並。



 元祐初,立十科薦士,中書舍人曾肇上疏,稱其讀書為文,不事俗學,安貧守道,志希古人,堪充師表科,一時廷臣亦多稱之。議欲命以官,未及而卒。
 遺文曰《灌園先生集》,傳於世。



 郭祥正,字功父,太平州當塗人,母夢李白而生。少有詩聲,梅堯臣方擅名一時,見而嘆曰:「天才如此,真太白後身也!」舉進士,熙寧中,知武岡縣,簽書保信軍節度判官。時王安石用事,祥正奏乞天下大計專聽安石處畫,有異議者,雖大臣亦當屏黜。神宗覽而異之,一日問安石曰:「卿識郭祥正乎?其才似可用。」出其章以示安石,安石恥為小臣所薦,因極口陳其無行。時祥正從章惇察訪
 闢,聞之,遂以殿中丞致仕。後復出,通判汀州。知端州,又棄去,隱於縣青山,卒。



 米芾,字元章,吳人也。以母侍宣仁後藩邸舊恩,補浛光尉。歷知雍丘縣、漣水軍,太常博士,知無為軍,召為書畫學博士,賜對便殿,上其子友仁所作《楚山清曉圖》,擢禮部員外郎,出知淮陽軍。卒,年四十九。



 芾為文奇險,不蹈襲前人軌轍。特妙於翰墨,沈著飛翥,得王獻之筆意。畫山水人物,自名一家,尤工臨移,至亂真不可辨。精於鑒
 裁,遇古器物書畫則極力求取,必得乃已。王安石嘗摘其詩句書扇上,蘇軾亦喜譽之。冠服效唐人,風神蕭散,音吐清暢,所至人聚觀之。而好潔成癖,至不與人同巾器。所為譎異,時有可傳笑者。無為州治有巨石,狀奇醜,芾見大喜曰:「此足以當吾拜!」具衣冠拜之,呼之為兄。又不能與世俯仰,故從仕數困。嘗奉詔仿《黃庭》小楷作周興嗣《千字韻語》。又入宣和殿觀禁內所藏,人以為寵。



 子友仁字元暉,力學嗜古,亦善書畫,世號小米,仕至兵部
 侍郎、敷文閣直學士。



 劉詵,字應伯,福州福清人。中進士第,歷莆田主簿、知廬江縣。崇寧中,為講議司檢討官,進軍器、大理丞,大晟府典樂。詵通音律,嘗上歷代雅樂因革及宋制作之旨,故委以樂事。又言:「《周官》大司樂禁淫聲、慢聲,蓋孔子所謂放鄭聲者。今燕樂之音,失於高急,曲調之詞,至於鄙俚,恐不足以召和氣。宋,火德也,音尚徵,徵調不可闕。臣按古制,旋十二宮以七聲,得正徵一調,惟陛下才取。」徽宗
 曰:「卿言是也,五聲闕一不可,《徵招》、《角招》為君臣相說之樂,此朕所欲聞而無言者,卿宜為朕典司之。」他日,禁中出古鐘二,詔執政召詵按於都堂,詵曰:「此與今太簇、大呂聲協。」命取大晟鐘扣之,果應。又曰:「鐘擊之無餘韻,不如石聲,《詩》所云『依我磬聲』者,言其清而定也。復取以合之,聲益諧。歷宗正、鴻臚、衛尉、太常寺少卿,纂《續因革禮》,卒。



 詵居母喪盡禮,有雙芝生墓側,人以為孝感。



 倪濤,字巨濟,廣德軍人。丱角能屬文,博學強記。年十五,
 試太學第一,遂擢進士,調廬陵尉、信陽軍教授。入為太學正,秘書省校書郎、著作佐郎,司勛、左司員外郎。朝廷議有事燕云,大臣爭先決策,為固位計,皆心知不可,無敢一出口,濤獨言其非。且曰:「景德以來,遼守約不犯邊,盟誓固在,不可渝也。天下久平,士不習戰,軍儲又屈,毋輕議以詒後患。」王黼怒曰:「君敢沮軍事邪!」於是言者論其鼓唱撰造,貶監朝城縣酒稅,再徙茶陵船場,卒,年三十九。死之明年,金人犯闕,朝廷憶濤言,官其一子。有《雲陽
 集》傳於世。



 李公麟,字伯時,舒州人。第進士,歷南康、長垣尉,泗州錄事參軍,用陸佃薦,為中書門下後省冊定官、御史檢法。好古博學,長於詩,多識奇字,自夏、商以來鐘、鼎、尊、彞,皆能考定世次,辨測款識,聞一妙品,雖捐千金不惜。紹聖末,朝廷得玉璽,下禮官諸儒議,言人人殊。公麟曰:「秦璽用藍田玉,今玉色正青,以龍蚓鳥魚為文,著『帝王受命之符』,玉質堅甚,非昆吾刀、蟾肪不可治,琱法中絕,此真秦
 李斯所為不疑。」議由是定。



 元符三年,病痺,遂致仕。既歸老,肆意於龍眠山巖壑間。雅善畫,自作《山莊圖》,為世寶。傳寫人物尤精,識者以為顧愷之、張僧繇之亞。襟度超軼,名士交譽之,黃庭堅謂其風流不減古人,然因畫為累,故世但以藝傳云。



 周邦彥,字美成,錢塘人。疏雋少檢,不為州里推重,而博涉百家之書。元豐初,游京師,獻《汴都賦》餘萬言,神宗異之,命侍臣讀於邇英閣,召赴政事堂,自太學諸生一命
 為正,居五歲不遷,益盡力於辭章。出教授廬州,知溧水縣,還為國子主簿。哲宗召對,使誦前賦,除秘書省正字。歷校書郎、考功員外郎,衛尉、宗正少卿,兼議禮局檢討,以直龍圖閣知河中府。徽宗欲使畢禮書,復留之。逾年,乃知隆德府,徙明州,入拜秘書監,進徽猷閣待制、提舉大晟府。未幾,知順昌府,徙處州,卒,年六十六,贈宣奉大夫。



 邦彥好音樂,能自度曲,製樂府長短句,詞韻清蔚,傳於世。



 朱長文,字伯原,蘇州吳人。年未冠,舉進士乙科,以病足不肯試吏,築室樂圃坊,著書閱古,吳人化其賢。長吏至,莫不先造請,謀政所急,士大夫過者以不到樂圃為恥,名動京師,公卿薦以自代者眾。元祐中,起教授於鄉,召為太學博士,遷秘書省正字。元符初,卒。哲宗知其清,賻絹百。



 有文三百卷,《六經》皆為辨說。又著《琴史》而序其略曰:「方朝廷成太平之功,制禮作樂,比隆商、周,則是書也,豈虛文哉!」蓋立志如此。



 劉弇,字偉明,吉州安福人。兒時警穎,日誦萬餘言。登元豐二年進士第,繼中博學宏詞科。歷官知嘉州峨眉縣,改太學博士。元符中,有事於南郊,弇進《南郊大禮賦》,哲守覽之動容,以為相如、子雲復出,除秘書省正字。徽宗即位,改著作佐郎、實錄院檢討官,以疾卒於官。



 弇少嗜酒,不事拘檢。為文辭剷剔瑕纇,卓詭不凡。有《龍雲集》三十卷,周必大序其文,謂「廬陵自歐陽文忠公以文章續韓文公正傳,遂為一代儒宗,繼之者弇也」。其相推重如此
 云。



\end{pinyinscope}