\article{列傳第十}

\begin{pinyinscope}

 韓令坤父倫慕容延劍子德豐從子德琛符彥卿子昭願昭壽



 韓令坤,磁州武安人。



 父倫,少以勇敢隸成德軍兵籍,累遷徐州下邳鎮將兼守禦指揮使。世宗以令坤貴,擢陳
 州行軍司馬,及令坤領陳州,徙倫許州。罷職,復居宛丘,多以不法乾郡政,私酤求市利,掊斂民財,公私患之。項城民武詣闕訴其事,命殿中侍御史率汀按之。倫詐報汀雲被詔赴闕,汀奏之。世宗怒,追劾具伏,法當棄市。令坤泣請於世宗,遂免死流海島。顯德六年,為左驍衛中郎將,遷左監門衛將軍。宋初,拜磁州刺史,轉亳州團練使。乾德四年,改本州防禦使,卒。



 令坤少隸周祖帳下,廣順初,歷鐵騎散員都虞候,控鶴右第一軍都校、領和
 州刺史。世宗即位,授殿前都虞候。俄賞高平之功,為龍捷左廂都虞候、領容州團練使,進本廂都指揮使、領泗州防禦使。徵太原,為行營前軍都校。未幾,為侍衛馬軍都指揮使、領定武軍節度。



 世宗使宰相郴穀將兵征淮南,俾令坤等十二將以從。谷退保正陽,為吳人所乘。令坤與宣祖、李重進合兵擊之,大敗吳人。世宗親征,聞揚州無備,遣令坤及宣祖、白延遇、趙晁等襲之。令坤先令延遇以精騎數百遲明馳入,城中不之覺。令坤繼
 至撫之,民皆按堵。南唐東都副留守馮延魯為僧匿寺中,令坤求獲之,送行在,遂以令坤知州事。由是泰州懼,以城降。



 時錢俶受詔攻常、潤,圍毗陵,反為南唐所敗。南唐乘勝遣將陸孟俊逼泰州,周師不能守,孟俊遂進軍蜀罔,逼揚州,令坤棄其城。世宗怒,命太祖與張永德領兵趨六合援之。令坤聞援至,復入城守,與孟俊兵戰,大敗之,擒孟俊,敗其將馬貴於楚州灣頭堰,擒漣州刺史秦進崇。俄命向拱為緣江招討使,以令坤副之,下壽州。
 歸朝,加檢校太尉、領鎮安軍節度使。世宗乃復幸淮右,次楚州,遣令坤率兵先入揚州,命權知軍府事。揚州城為吳人所毀,詔發丁壯別築新城,命令坤為修城都部署。



 六年春,命令坤以汴、亳民導汴水入於蔡。三月,世宗將北征,命率龍捷、虎捷、驍武兵先赴大名,又副王晏為益津關一路都部署,俄為霸州都部署,率所部兵之。恭帝即位,加檢校太尉、侍衛馬步軍都虞候。冬,詔防北邊。



 宋初,移領天平軍,加侍衛馬步軍都指揮使、同平章
 事。太祖親征李筠,詔令坤率兵屯河陽。及澤、潞平,還京,錫宴令坤等於禮賢講武殿,賜襲衣、器幣、鞍勒馬有差,以功加兼侍中。又從討李重進。建隆二年,改成德軍節度,充北面緣邊兵馬都部署。將赴鎮,上於別殿置酒餞之,因勖其為治。



 乾德六年,疽發背卒,年四十六。太祖素服發哀於講武殿,錄其子慶朝為閑廄使,慶雄為閑廄副使。令坤有才略,識治道,與太祖同事周室,情好親密。鎮常山凡七年,北邊以寧。聞其卒,甚悼惜之。



 初,南唐遣
 邊鎬破湖南,以馬希崇分司揚州,及令坤克取之,希崇以妓稀薄獻,令坤甚嬖之。會擒陸孟俊,將械送行在所,楊氏於簾間窺見之,即拊膺慟哭。令坤怪問之,楊氏曰:「孟俊往年入潭州,殺我家二百口,惟妾為希崇所匿得免,願甘心焉。」令坤以詰孟俊,孟俊具伏,令坤乃殺之。



 慕容延釗,太原人。父章,襄州馬步軍都校、領開州刺。延釗少以勇干聞。漢祖之興也,周祖為其佐命,以延釗隸帳下。周廣順初,補西頭供奉官,歷尚食副使、鐵騎都
 虞候。



 世宗即位,為殿前散指揮使都校、領溪州刺史。高平之戰,督左先鋒,以功授虎捷左廂都指揮使、領本州團練使;遷殿前都虞候、領睦州防禦使。從征淮南,改龍捷左廂都校、沿江馬軍陪署。歸朝,復為殿前都虞候,出為鎮淮軍都部署。顯德五年,世宗在迎鑾江口,聞吳人舟數艘泊東水布洲,即命延釗與右神武統軍宋延渥討之。延釗以驍騎由陸進,延渥督舟師沿江繼進,大破之。淮南平,遷殿前副都指揮使、領淮南節度。恭帝即
 位,改鎮寧軍節度,充殿前副都點檢,復為北面行營馬步軍都虞候。



 太祖即位,延釗方握重兵屯真定,帝遣使諭旨,許以便宜從事。延釗與坤率所部兵按治邊境,以鎮靜聞。太祖嘉之,加殿前都點檢、同中書門下二品,避其父名故也。郴筠叛,初命與王全斌由東路會兵進討,俄為行營都部署、知潞州行府事;及平,加兼侍中,詔還澶州。



 建隆二年,長春節來朝,賜宅一區。表解軍職,徙為山南東道節度、西南面兵馬都部署。是冬大寒,遣
 中使賜貂裘、百子氈帳。四年春,命師南征,以延釗為湖南道行營前軍都部署。時延釗被病,詔令肩輿即戎事。賊將汪端與眾數千擾朗州,延早擒之,磔於市。荊、湘既平,加檢校太尉。是冬,卒,年五十一。



 初,延釗與太祖友善,顯德末,太祖任殿前都點檢,延釗為副,常兄事延釗;及即位,每遣使勞問,猶以兄呼之。洎寢疾,御封藥以賜,聞其卒,慟哭久之。贈中書令,追封河南郡王,錄其子弟授官者四人。



 子德業、德豐、德鈞。德業至衛州刺史,德鈞至
 尚食副使。延釗弟延忠,歷內殿直、供奉西頭官都知,至磁州刺史;延卿至虎捷軍都指揮使。延卿子德琛。



 德豐字日新,幼聰悟,延釗愛之,嘗曰:「興吾門者必此子。」八歲,補山南東道衙內指揮使。延釗卒,授如京使。



 開寶中,從征太原,領御砦南面巡檢。又為揚州都監。征南唐,為洞子都監。城既下,命為升州都監。市厘安靜,澤國富饒,使者多裒聚金帛,德豐獨以廉潔聞。俄領蔚州刺史。



 太平興國二年,知慶州兼邠、寧都巡檢。嘗破小遇族,奪
 名馬數十匹,詔書褒諭。居任九年,以簡靜為治,邊鎮安之。



 雍熙四年,使登、萊閱強壯,及還,拜西上閣門使。是冬,出為定遠軍鈐轄,命領後陣中隊,別將萬騎以御邊害。



 淳化二年,進秩東上,知邢州。三年,改判四方館事,出知延州。時侯延廣知靈武,或言其得西夏情,倔強難制,命德豐代之,就賜白金三千兩。會建使名,改為四方館使。未幾,以所部不治,徙知慶州,俄又改靈州兼部署。穀價湧貴,德豐出私廩賑饑民,全活者眾。轉引進使。賊入境,
 德豐率兵擊走,獲羊馬甚眾。



 咸平二年,遷客省使,知鎮州,召對便坐,撫慰甚至。昌冬,遼人南侵,德豐繕兵固守,饋饋不絕,詔獎之。三年,改滄州。德豐輕財好施,厚享將士。在西邊時,母留京師,妻孥寓長安,貧甚,真宗憫之,特詔給團練使奉。逾年,進穎州團練使,知貝、瀛二州。五年,卒,年五十五。家無餘財,談者善之。



 子惟素,至殿內承制。



 德琛權延釗蔭補供奉官,累遷內殿崇班、知夔州。李順之亂,賊猷張餘領眾十萬餘、舟千艘來寇。與順戰龍山,
 斬首千餘級;又與白繼贇擊賊,斬二萬科,悉焚其舟。賊剽開州,圍雲安,德琛往援之,又斬百餘級。累詔褒諭。歷西京作坊、左藏二副使。咸平二年,轉崇儀副使、荊湖北路鈐轄。蠻擾澧、鼎境上,德琛戰於北水義,奪耕牛、鎧甲,斬馘以歸。徙峽路鈐轄,未至,復知夔州。景德中,領梧州刺史,復任峽路,再遷莊宅使,又為並、代鈐轄,知憲州。天禧初,改右監門衛大將軍。



 符彥卿字冠侯,陳州宛丘人。父存審,後唐宣武軍節度、
 蕃漢為步軍都總管中書令。彥卿年十三,能騎射。事莊宗於太原,以謹願稱,出入臥內,及長,以為親從指揮使。入汴,遷散員指揮使。郭從謙之亂,莊宗左右皆引去,惟彥卿力戰,射殺十數人,俄矢集乘輿,遂慟哭而去。天成三年,以龍武都虞候、吉州刺史討王都於定州,大破契丹於嘉山。明年克其城,授耀州團練使。改慶州刺。奉詔築堡方渠北烏侖山口,以招黨項。清泰初,改易州,兼領北面騎軍,賜戎服、介冑、戰馬。嘗射獵遂城鹽臺澱,
 一日射獐、彘、狼、狐、兔四十二,觀者神之。晉天福初,授同州節度。兄彥饒亦鎮滑臺。俄而彥饒叛,彥卿上表待罪,乞田里,晉祖釋不問。改左羽林統軍,俄兼領右羽林,改鎮鄜延。



 少帝幼與彥卿狎,即位,召還,出鎮河陽三城。遼人南侵,詔彥卿率所部拒戰澶州。契丹騎兵數萬圍高行周於鐵丘,諸將莫敢當其鋒,彥卿獨自變量百騎擊之,遼人遁去,行周得免。又李守貞討平青州楊光遠,移鎮許州,封祁國公。



 開花運二年,與杜重威、李守貞經略
 北鄙。契丹主率眾十餘萬圍晉盱陽城,軍中乏水,鑿井輒壞,爭絞泥吮之,人馬多渴死,時晉師居下風,將戰,弓弩莫施。彥卿謂張彥澤、皇甫遇曰:「與其束手就擒,曷若死戰,然未必死。」彥澤等然之。遂潛兵尾其後,順風擊之,契丹大敗,其主乘橐駝以遁,獲其器甲、旗仗數萬以。少帝嘉之,改武寧軍節度、同平章事。



 為左右所間,會再出師河朔,彥卿不預,易其行伍,配以羸師數千,戍荊州口。及杜重威以大軍降于滹水,急詔彥卿與高
 領禁兵屯澶淵。會彥澤引遼兵入汴,彥卿與行周遂歸遼。遼主以陽城之敗詰彥卿,彥卿對曰:「臣事晉王,不敢愛死,今日之事,死生唯命。」遼主笑而釋之。



 會徐、宋寇盜蜂起,遼主即遣彥卿歸鎮。行次甬橋,賊魁李仁恕擁眾數萬攻徐州。彥卿守習縋而出,大呼賊中曰:「相公當為國討賊,何故入虎口,乃助賊攻城?我雖父子,今為仇敵,當死戰,城不可
 入。」賊惶愧羅拜彥卿前,乞免罪,彥卿為設誓,乃解去。



 漢祖入汴,彥卿自徐州來朝,改鎮袞州,加兼侍中。乾祐中,加兼中書令,封魏國公,拜守太保,移鎮青州。及殺楊邠輩,召促赴闕下。



 周祖即位,封淮陽王。劉銖誅,以其京城第宅賜彥卿。及徵袞州,彥卿朝行在,獻馬及錦彩、軍糧萬石,連被賜繼。俄移鎮鄆州。會召魏府王殷,欲以彥卿代鎮。俄遼人起兵,留殷控扼,故彥卿不入朝。殷得罪,即以彥卿為大名尹、天雄軍節度,進封衛王。



 世宗初,並人
 擾潞州,潞兵敗,命彥卿領兵從磁州固鎮路壓其背。及帝親征,命為行營一行都部署知太原行府事,領步騎二萬亓討。



 初,彥卿之行也,世宗以並人雖敗,朝廷饋運不繼,未議攻擊,且令觀兵城下,徐圖進取。及周師入境,汾、晉吏民望風款接,皆以久罹虐政,願輸軍須以資兵力,世宗從之。而連下數州,彥卿等皆以芻糧未備,欲旋軍。世宗不之省,乃調山東近郡挽軍食濟之。及世宗至城下,命與郭從義、向訓、白重贊、史彥超率萬騎屯忻
 口,以拒北援,又下盂縣。



 遼人駐忻北,游騎及近郊,史彥超以二千騎當其鋒,左右馳擊,彥超死之;敗遼眾二千餘,遼騎遁走。先知為遼人所掩,重傷數百人,諸將論議矛盾,師故不振。世宗乃班師,數賜彥卿繒彩、鞍勒馬,遣歸本鎮。還京,拜彥卿太傅,改封魏王。恭帝即位,加守太尉。



 太祖即位,加守太師。建隆四年春,來朝,賜襲衣、玉帶。宴射於金鳳園,太祖七發皆中的,彥卿貢名馬稱賀。



 開寶二年六月,移鳳翔節度,被病肩輿赴鎮,至西京,上言
 疾亟,請主不醫洛陽,從之。假滿百日,猶請其奉,為御史所劾,下留司御史臺。太祖以姻舊特免推鞫,止罷其節制。八年六月,卒,年七十八。喪事官給。



 彥卿將家子,勇略有謀,善用兵。存審之第四子,軍中謂之「符第四」。前後賞賜鉅萬,悉分給帳下,故士卒樂為效死。遼人自陽城之敗,尤畏彥卿,或馬病不飲齕,必唾而咒曰:「此中豈有符王邪?晉少主既陷契丹,德光之母問左右曰:「彥卿安在?」或對曰:「聞其已遣歸徐州矣。」德光母曰:「留此人中原,何失
 策之甚!」其威名如此。



 鎮大名餘十年,政委牙校劉思遇。思遇貪黠,怙勢斂貨財,公府之利多入其家,彥卿不之覺。時藩鎮率遣親吏受民租,概量增溢,公取其餘羨,而魏郡尤甚。太祖聞之,遣常參官主其事,由是斛量始平。詔以羨零部餘粟賜彥卿,以愧其心。



 彥卿酷好鷹犬,吏卒有過,求名鷹犬以獻,雖盛怒必貰之。性不飲酒,頗謙恭下士,對賓客終日談笑,不及世務,不伐戰功。居洛陽七八年,每春月,乘小駟從家僮一二游僧寺名園,優游自適。



 周世宗宣懿皇后、太宗懿德皇后,皆彥卿女也。自恭帝及太祖兩朝,賜詔書不名。子昭信、昭願、昭壽。昭信,天雄軍衙內都指揮使、領賀州刺史。周顯德初,卒,贈檢校太保、閬州防禦使。



 昭願字致恭,謹厚謙約,頗讀書好事。周廣順中,以蔭補天雄軍牙職,俄領興州刺史。



 開寶中,改領恩州。彥卿養疾居洛,入補供奉官。四年,改領羅州刺史。七年,遷西京作坊副使。俄授尚食使,出護陳、許、蔡、穎等州巡兵。從征
 太原,為御營四面巡檢使。及攻幽州,命與定國軍節度偓率兵萬餘,置砦城南。師還,真拜蔡州刺史,知並澶二州。不逾月,復移並門兼副部署。丁內艱,起復,為本州團練使,連知永興軍、梓滑二州。



 咸平初,又為天雄軍、邢州二鈐轄。三年,以疾求歸京師,詔遣中使、尚醫馳傳診視。既還,帝賜以名方御藥,拜本州防禦使。四年,卒,年五十七。車駕臨哭,贈鎮東軍節度。子承煦,為左千牛衛將軍。



 昭壽,初補供奉官。開寶七年,改西京作坊副使。歷遷六宅副使、領蘭州刺。雍熙二年冬,命與劉知信護鎮州屯兵。命遣將北征,又與知信為押隊都監,轉尚食使,真拜光州刺史。端拱二年,知洪州。淳化四年,改定州。咸平初,遷鳳州團練使、益州鈐轄。



 昭壽以貴家子日事游宴,簡倨自恣,常紗帽素氅衣,偃息後圃,不理戎務,有所裁決,即令家人傳道。多集錦工就廨舍織纖麗綺帛,每有所須,取給於市,餘半歲方給其直,又令部曲私邀取之。
 廣糴黍稻,未及成熟者亦取之,悉貯寺觀中,久之損敗,即勒道釋價之。縱其下凌忽軍校。



 劍南自李順平後,人心洶洶,知州牛冕緩馳無政,昭壽又不能御軍,人皆怨憤。神衛卒趙延順等八人謀欲害昭壽,未敢發。三年正旦,中使自峨眉山還京,昭壽戒馭吏具鞍馬將送之,延順等悉解廄中馬□疆,奔逸庭下,陽逐喧呼,登廳執昭壽殺之,並殺二僕,據甲仗庫,取兵器。都監王澤聞之,急召本軍都虞候王均率兵擒捕。延順左執昭壽首,右操劍,
 徬徨無所適,卒見均至,即與眾推均為帥,合驍猛、威武兵為亂。牛冕洎轉運使張適奔漢州。是秋,官兵討平之。見《雷有終傳》。



 昭壽子承諒,娶齊王女嘉興縣主,至內殿承制。



 諭曰:五季之亂,內則權臣擅命,外則藩鎮握兵。宋興,內外廓清,若天去其疾,或納節以備宿衛,或請老以奉朝請。雖太祖善御,諸臣知機,要亦否極而泰之象也。彥卿一門二後,累朝襲寵,有謀善戰,聲振殊俗,與時進退,其
 名將之賢者歟?令坤、延釗素與太祖親善,平荊、湘則南服底定,鎮常山則北邊載寧可,未嘗恃舊與功以啟嫌隙。創業君臣有過人者,類如是夫



\end{pinyinscope}