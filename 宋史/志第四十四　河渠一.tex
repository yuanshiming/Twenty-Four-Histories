\article{志第四十四 河渠一}

\begin{pinyinscope}

 黃河上



 黃河自昔為中國患,《河渠書》述之詳矣。探厥本源,則博望之說,猶為未也。大元至元二十七年,我世祖皇帝命學士蒲察篤實西窮河源,始得其詳。今西蕃朵甘思南
 鄙曰星宿海者,其源也,四山之間,有泉近百泓,匯而為海,登高望之,若星宿布列,故名。流出復瀦,曰哈刺海,東出曰赤賓河,合忽闌、也裏術二河,東北流為九渡河,其水猶清,騎可涉也。貫山中行,出西戎之都會,曰闊即、曰闊提者,合納憐河,所謂「細黃河」也,水流已濁。繞昆侖之南,折而東注,合乞里馬出河,復繞昆侖之北,自貴德、西寧之境,至積石,經河州,過臨洮,合洮河,東北流至蘭州,始入中國。北繞朔方、北地、上郡而東,經三受降城、豐東
 勝州,折而南,出龍門,過河中,抵潼關。東出三門、集津為孟津,過虎牢,而後奔放平壤。吞納小水以百數,勢益雄放,無崇山巨磯以防閑之,旁激奔潰,不遵禹跡。故虎牢迤東距海口三二千里,恆被其害,宋為特甚。始自滑臺、大伾,嘗兩經泛溢,復禹跡矣。一時奸臣建議,必欲回之,俾復故流,竭天下之力以塞之。屢塞屢決,至南渡而後,貽其禍於金源氏,由不能順其就下之性以導之故也。



 若江,若淮,若洛、汴、衡漳,暨江、淮以南諸水,皆有舟楫溉
 灌之利者,歷敘其事而分紀之。為《河渠志》。



 河入中國,行太行西,曲折山間,不能為大患。既出大岯,東走赴海,更平地二千餘里,禹跡既湮,河並為一,特以堤防為之限。夏秋霖潦,百川眾流所會,不免決溢之憂,然有司所以備河者,亦益工矣。



 自周顯德初,大決東平之楊劉,宰相李谷監治堤,自陽谷抵張秋口以遏之,水患少息。然決河不復故道,離而為赤河。



 太祖乾德二年,遣使案行,將治古堤。議者以舊河不可卒復,力役且大,
 遂止。但詔民治遙堤,以御沖注之患。其後赤河決東平之竹村,七州之地復罹水災。三年秋,大雨霖,開封府河決陽武,又孟州水漲,壞中水單橋梁,澶、鄆亦言河決,詔發州兵治之。四年八月,滑州河決,壞靈河縣大堤,詔殿前都指揮使韓重贇、馬步軍都軍頭王廷義等督士卒丁夫數萬人治之,被泛者蠲其秋租。



 五年正月,帝以河堤屢決,分遣使行視,發畿甸丁夫繕治。自是歲以為常,皆以正月首事,季春而畢。是月,詔開封大名府、鄆澶滑孟
 濮齊淄滄棣濱德博懷衛鄭等州長吏,並兼本州河堤使,蓋以謹力役而重水患也。



 開寶四年十一月,河決澶淵,泛數州。官守不時上言,通判、司封郎中姚恕棄市,知州杜審肇坐免。五年正月,詔曰:「應緣黃、汴、清、御等河州縣,除準舊制種藝桑棗外,委長吏課民別樹榆柳及土地所宜之木。仍案戶籍高下,定為五等:第一等歲樹五十本,第二等以下遞減十本。民欲廣樹藝者聽,其孤、寡、煢、獨者免。是月,澶州修河卒賜以錢、鞋,役夫給以茶。三
 月,詔曰:「朕每念河渠潰決,頗為民患,故署使職以總領焉,宜委官聯佐治其事。自今開封等十七州府,各置河堤判官一員,以本州通判充;如通判闕員,即以本州判官充。」五月,河大決濮陽,又決陽武。詔發諸州兵及丁夫凡五萬人,遣穎州團練使曹翰護其役。翰辭,太祖謂曰:「霖雨不止,又聞河決。朕信宿以來,焚香上禱於天,若天災流行,願在朕躬,勿延於民也。翰頓首對曰:「昔宋景公諸侯耳,一發善言,災星退舍。今陛下憂及兆庶,懇禱如是,
 固當上感天心,必不為災。」



 六月,下詔曰:「近者澶、濮等數州,霖雨薦降,洪河為患。朕以屢經決溢,重困黎元,每閱前書,詳究經瀆。至若夏后所載,但言導河至海,隨山浚川,未聞力制湍流,廣營高岸。自戰國專利,堙塞故道,小以妨大,私而害公,九河之制遂隳,歷代之患弗弭。凡搢紳多士、草澤之倫,有素習河渠之書,深知疏導之策,若為經久,可免重勞,並許詣闕上書,附驛條奏。朕當親覽,用其所長,勉副詢求,當示甄獎。」時東魯逸人田告者,
 纂《禹元經》十二篇,帝聞之,召至闕下,詢以治水之道,善其言,將授以官,以親老固辭歸養,從之。翰至河上,親督工徒,未幾,決河皆塞。



 太宗太平興國二年秋七月,河決孟州之溫縣、鄭州之滎澤、澶州之頓丘,皆發緣河諸州丁夫塞之。又遣左衛大將軍李崇矩騎置自陜西至滄、棣,案行水勢。視堤岸之缺,亟繕治之;民被水災者,悉蠲其租。三年正月,命使十七人分治黃河堤,以備水患。滑州靈河縣河塞復決,命西上閣門使郭守文率卒塞之。七
 年,河大漲,蹙清河,凌鄆州,城將陷,塞其門,急奏以聞。詔殿前承旨劉吉馳往固之。



 八年五月,河大決滑州韓村,泛澶、濮、曹、濟諸州民田,壞居人廬舍,東南流至彭城界入於淮。詔發丁夫塞之。堤久不成,乃命使者按視遙堤舊址。使回條奏,以為「治遙堤不如分水勢。自孟抵鄆,雖有堤防,唯滑與澶最為隘狹。於此二州之地,可立分水之制,宜於南北岸各開其一,北入王莽河以通於海,南入靈河以通於淮,節減暴流,一如汴口之法。其分水河,
 量其遠邇,作為斗門,啟閉隨時,務乎均濟。通舟運,溉農田,此富庶之資也。」不報。時多陰雨,河久未塞,帝憂之,遣樞密直學士張齊賢乘傳詣白馬津,用太牢加璧以祭。十二月,滑州言決河塞,群臣稱賀。



 九年春,滑州復言房村河決,帝曰:「近以河決韓村,發民治堤不成,安可重困吾民,當以諸軍代之。」乃發卒五萬,以侍衛步軍都指揮使田重進領其役,又命翰林學士宋白祭白馬津,沉以太牢加璧,未幾役成。



 淳化二年三月,詔:「長吏以下及巡
 河主埽使臣,經度行視河堤,勿致壞隳,違者當寘於法。」四年十月,河決澶州,陷北城,壞廬舍七千餘區,詔發卒代民治之。是歲,巡河供奉官梁睿上言:「滑州土脈疏,岸善隤,每歲河決南岸,害民田。請於迎陽鑿渠引水,凡四十里,至黎陽合大河,以防暴漲。」帝許之。五年正月,滑州言新渠成,帝又案圖,命昭宣使羅州刺史杜彥鈞率兵夫,計功十七萬,鑿河開渠,自韓村埽至州西鐵狗廟,凡十五餘里,復合於河,以分水勢。



 真宗咸平三年五月,河
 決鄆州王陵埽,浮鉅野,入淮、泗,水勢悍激,侵迫州城。命使率諸州丁男二萬人塞之,逾月而畢。始,赤河決,擁濟、泗,鄆州城中常苦水患。至是,霖雨彌月,積潦益甚,乃遣工部郎中陳若拙經度徙城。若拙請徙於東南十五里陽鄉之高原,詔可。是年,詔:「緣河官吏,雖秩滿,須水落受代。知州、通判兩月一巡堤,縣令、佐迭巡堤防,轉運使勿委以他職。」又申嚴盜伐河上榆柳之禁。



 景德元年九月,澶州言河決橫□埽;四年,又壞王八埽,並詔發兵夫完
 治之。大中祥符三年十月,判河中府陳堯叟言:「白浮圖村河水決溢,為南風激還故道。」明年,遣使滑州,經度西岸,開減水河。九月,棣州河決聶家口,五年正月,本州請徙城,帝曰:「城去決河尚十數里,居民重遷。」命使完塞。既成,又決於州東南李民灣,環城數十里民舍多壞,又請徙於商河。役興逾年,雖捍護完築,裁免決溢,而湍流益暴,壖地益削,河勢高民屋殆逾丈矣,民苦久役,而終憂水患。八年,乃詔徙州於陽信之八方寺。



 著作佐郎李垂
 上《導河形勝書》三篇並圖,其略曰:



 臣請自汲郡東推禹故道,挾御河,較其水勢,出大伾、上陽、太行三山之間,復西河故瀆,北注大名西、館陶南,東北合赤河而至於海。因於魏縣北析一渠,正北稍西徑衡漳直北,下出邢、洺,如《夏書》過洚水,稍東注易水、合百濟、會朝河而至於海。大伾而下,黃、御混流,薄山障堤,勢不能遠。如是則載之高地而北行,百姓獲利,而契丹不能南侵矣。《禹貢》所謂「夾右碣石入於海」,孔安國曰:「河逆上此州界。」



 其始作自
 大伾西八十里,曹公所開運渠東五里,引河水正北稍東十里,破伯禹古堤,徑牧馬陂,從禹故道,又東三十里轉大伾西、通利軍北,挾白溝,復西大河,北徑清豐、大名西,歷洹水、魏縣東,暨館陶南,入屯氏故瀆,合赤河而北至於海。既而自大伾西新發故瀆西岸析一渠,正北稍西五里,廣深與汴等,合御河道,逼大伾北,即堅壤析一渠,東西二十里,廣深與汴等,復東大河。兩渠分流,則三四分水,猶得注澶淵舊渠矣。大都河水從西大河故瀆
 東北,合赤河而達於海,然後於魏縣北發御河西岸析一渠,正北稍西六十里,廣深與御河等,合衡漳水;又冀州北界、深州西南三十里決衡漳西岸,限水為門,西北注滹沱,潦則塞之,使東漸渤海,旱則決之,使西灌屯田,此中國御邊之利也。



 兩漢而下,言水利者,屢欲求九河故道而疏之。今考圖志,九河並在平原而北,且河壞澶、滑,未至平原而上已決矣,則九河奚利哉。漢武舍大伾之故道,發頓丘之暴沖,則濫兗泛齊,流患中土,使河朔
 平田,膏腴千里,縱容邊寇劫掠其間。今大河盡東,全燕陷北,而御邊之計,莫大於河。不然,則趙、魏百城,富庶萬億,所謂誨盜而招寇矣。一日伺我饑饉,乘虛入寇,臨時用計者實難;不如因人足財豐之時,成之為易。



 詔樞密直學士任中正、龍圖閣直學士陳彭年、知制誥王曾詳定。中正等上言:「詳垂所述,頗為周悉。所言起滑臺而下,派之為六,則緣流就下,湍急難制,恐水勢聚而為一,不能各依所導。設或必成六派,則是更增六處河口,悠久
 難於堤防。亦慮入滹沱、漳河,漸至二水淤塞,益為民患,又築堤七百里,役夫二十一萬七千,工至四十日,侵占民田,頗為煩費。」其議遂寢。



 七年,詔罷葺遙堤,以養民力。八月,河決澶州大吳埽,役徒數千,築新堤,亙二百四十步,水乃順道。八年,京西轉運使陳堯佐議開滑州小河分水勢,遣使視利害以聞。及還,請規度自三迎陽村北治之,復開汊河於上游,以洩其壅溢。詔可。



 天禧三年六月乙未夜,滑州河溢城西北天臺山旁,俄復潰於城西
 南,岸摧七百步,漫溢州城,歷澶、濮、曹、鄆,注梁山泊;又合清水、古汴渠東入於淮,州邑罹患者三十二。即遣使賦諸州薪石、楗橛、芟竹之數千六百萬,發兵夫九萬人治之。四年二月,河塞,群臣入賀,上親為文,刻石紀功。



 是年,祠部員外郎李垂又言疏河利害,命垂至大名府、滑衛德貝州、通利軍與長吏計度。垂上言:



 臣所至,並稱黃河水入王莽沙河與西河故瀆,注金、赤河,必慮水勢浩大,蕩浸民田,難於防備。臣亦以為河水所經,不無為害。今
 者決河而南,為害既多,而又陽武埽東、石堰埽西,地形污下,東河洩水又艱。或者云:「今決處漕底坑深,舊渠逆上,若塞之,旁勞必復壞。」如是,則議塞河者誠以為難。若決河而北,為害雖少,一旦河水注御河,蕩易水,徑乾寧軍,入獨流口,遂及契丹之境。或者云:「因此搖動邊鄙。」如是,則議疏河者又益為難。臣於兩難之間,輒畫一計:請自上流引北載之高地,東至大伾,瀉復於澶淵舊道,使南不至滑州,北不出通利軍界。



 何以計之?臣請自衛州東
 界曹公所開運渠東五里,河北岸凸處,就岸實土堅引之,正北稍東十三里,破伯禹古堤,注裴家潭,徑牧馬陂,又正東稍北四十里,鑿大伾西山,釃為二渠:一逼大伾南足,決古堤正東八里,復澶淵舊道;一逼通利軍城北曲河口,至大禹所導西河故瀆,正北稍東五里,開南北大堤,又東七里,入澶淵舊道,與南渠合。夫如是,則北載之高地,大伾二山脽股之間分酌其勢,浚瀉兩渠,匯注東北,不遠三十里,復合於澶淵舊道,而滑州不治自涸
 矣。



 臣請以兵夫二萬,自來歲二月興作,除三伏半功外,至十月而成。其均厚埤薄,俟次年可也。



 疏奏,朝議慮其煩擾,罷之。



 初,滑州以天臺決口去水稍遠,聊興葺之,及西南堤成,乃於天臺口旁築月堤。六月望,河復決天臺下,走衛南,浮徐、濟,害如三年而益甚,帝以新經賦率,慮殫困民力,即詔京東西、河北路經水災州軍,勿復科調丁夫,其守捍堤防役兵,仍令長吏存恤而番休之。五年正月,知滑州陳堯佐以西北水壞,城無外御,築大堤,又
 疊埽於城北,護州中居民;復就鑿橫木,下垂木數條,置水旁以護岸,謂之「木龍」,當時賴焉。復並舊河開枝流,以分導水勢,有詔嘉獎。



 說者以黃河隨時漲落,故舉物候為水勢之名:自立春之後,東風解凍,河邊入候水,初至凡一寸,則夏秋當至一尺,頗為信驗,故謂之「信水」。二月、三月桃華始開,冰泮兩積,川流猥集,波瀾盛長,謂之「桃華水」。春末蕪菁華開,謂之「菜華水」。四月末壟麥結秀,擢芒變色,謂之「麥黃水」。五月瓜實延蔓,謂之「瓜蔓水」。朔野
 之地,深山窮谷,固陰冱寒,冰堅晚泮,逮乎盛夏,消釋方盡,而沃蕩山石,水帶礬腥,並流於河,故六月中旬後,謂之「礬山水」。七月菽豆方秀,謂之「豆華水」。八月菼亂華,謂之「荻苗水」。九月以重陽紀節,謂之「登高水」。十月水落安流,復其故道,謂之「復槽水」。十一月、十二月斷冰雜流,乘寒復結,謂之「蹙凌水」。水信有常,率以為準;非時暴漲,謂之「客水」。


其水勢,凡移谼橫注,岸如刺毀,謂之「扎岸」。漲溢逾防,謂之「抹岸」。埽岸故朽,潛流漱其下,謂之「塌岸」。浪勢
 旋激,岸土上隤,謂之「淪卷」。水侵岸逆漲,謂之「上展」。順漲,謂之「下展」。或水乍落,直流之中,忽屈曲橫射,謂之「徑
 \gezhu{
  穴叫}
 」。水猛驟移,其將澄處,望之明白,謂之「拽白」,亦謂之「明灘」。湍怒略渟,勢稍汩起,行舟值之多溺,謂之「薦浪水」。水退淤澱,夏則膠土肥腴。初秋則黃滅土,頗為疏壤,深秋則白滅土,霜降後皆沙也。



 舊制,歲虞河決,有司常以孟秋預調塞治之物,梢芟、薪柴、楗橛、竹石、茭索、竹索凡千餘萬,謂之「春料」。詔下瀕河諸州所產之地,仍遣使會河渠
 官吏,乘農隙率丁夫水工,收採備用。凡伐蘆荻謂之「芟」,伐山木榆柳枝葉謂之「梢」,辮竹糾芟為索。以竹為巨索,長十尺至百尺,有數等。先擇寬平之所為埽場。埽之制,密布芟索,鋪梢,梢芟相重,壓之以土,雜以碎石,以巨竹索橫貫其中,謂之「心索」。卷而束之,復以大芟索擊其兩端,別以竹索自內旁出,其高至數丈,其長倍之。凡用丁夫數百或千人,雜唱齊挽,積置於卑薄之處,謂之「埽岸」。既下,以橛臬閡之,復以長木貫之,其竹索皆埋巨木於
 岸以維之,遇河之橫決,則復增之,以補其缺。凡埽下非積數疊,亦不能遏其迅湍,又有馬頭、鋸牙、木岸者,以蹙水勢護堤焉。



 凡緣河諸州,孟州有河南北凡二埽,開封府有陽武埽,滑州有韓房二村、憑管、石堰、州西、魚池、迎陽凡七埽,舊有七里曲埽,後廢。



 通利軍有齊賈、蘇村凡二埽,澶州有濮陽、大韓、大吳、商胡、王楚、橫隴、曹村、依仁、大北、岡孫、陳固、明公、王八凡十三埽,大名府有孫杜、侯村二埽,濮州有任村、東、西、北凡四埽,鄆州有博陵、張秋、關山、子路、
 王陵、竹口凡六埽,齊州有採金山、史家渦二埽,濱州有平河、安定二埽,棣州有聶家、梭堤、鋸牙、陽成四埽,所費皆有司歲計而無闕焉。



 仁宗天聖元年,以滑州決河未塞,詔募京東、河北、陜西、淮南民輸薪芻,調兵伐瀕河榆柳,賙溺死之家。二年,遣使詣滑、衛行視河勢。五年,發丁夫三萬八千,卒二萬一千,緡錢五十萬,塞決河,轉運使五日一奏河事。十月丙申,塞河成,以其近天臺山麓,名曰天臺埽。宰臣王曾率百官入賀。十二月,浚魚池埽減
 水河。



 六年八月,河決於澶州之王楚埽,凡三十步。八年,始詔河北轉運司計塞河之備,良山令陳曜請疏鄆、滑界糜丘河以分水勢,遂遣使行視遙堤。明道二年,徙大名之朝城縣於杜婆村,廢鄆州之王橋渡、淄州之臨河鎮以避水。



 景祐元年七月,河決澶州橫隴埽。慶歷元年,詔權停修決河。自此久不復塞,而議開分水河以殺其暴。未興工而河流自分,有司以聞,遣使特祠之。三月,命築堤于澶以捍城。八年六月癸酉,河決商胡埽,決口廣
 五百五十七步,乃命使行視河堤。



 皇祐元年三月,河合永濟渠注乾寧軍。二年七月辛酉,河復決大名府館陶縣之郭固。四年正月乙酉,塞郭固而河勢猶壅,議者請開六塔以披其勢。至和元年,遣使行度故道,且詣銅城鎮海口,約古道高下之勢。二年,翰林學士歐陽修奏疏曰:



 朝廷欲俟秋興大役,塞商胡,開橫隴,回大河於古道。夫動大眾必順天時、量人力,謀於其始而審於其終,然後必行,計其所利者多,乃可無悔。比年以來,興役動眾,
 勞民費財,不精謀慮於厥初,輕信利害之偏說,舉事之始,既已蒼皇,群議一搖,尋復悔罷。不敢遠引他事,且如河決商胡,是時執政之臣,不慎計慮,遽謀修塞。凡科配梢芟一千八百萬,騷動六路一百餘軍州,官吏催驅,急若星火,民庶愁苦,盈於道途。或物已輸官,或人方在路,未及興役,尋已罷修,虛費民財,為國斂怨,舉事輕脫,為害若斯。今又聞復有修河之役,三十萬人之眾,開一千餘里之長河,計其所用物力,數倍往年。當此天災歲旱、
 民困國貧之際,不量人力,不順天時,知其有大不可者五:



 蓋自去秋至春半,天下苦旱,京東尤甚,河北次之。國家常務安靜振恤之,猶恐民起為盜,況於兩路聚大眾、興大役乎?此其必不可者一也。



 河北自恩州用兵之後,繼以兇年,人戶流亡,十失八九。數年以來,人稍歸復,然死亡之餘,所存者幾,瘡痍未斂,物力未完。又京東自去冬無雨雪,麥不生苗,將逾暮春,粟未布種,農心焦勞,所向無望。若別路差夫,又遠者難為赴役;一出諸路,則兩
 路力所不任。此其必不可者二也。



 往年議塞滑州決河,時公私之力,未若今日之貧虛;然猶儲積物料,誘率民財,數年之間,始能興役。今國用方乏,民力方疲,且合商胡塞大決之洪流,此一大役也。鑿橫隴開久廢之故道,又一大役也。自橫隴至海千餘里,埽岸久已廢,頓須興緝,又一大役也。往年公私有力之時,興一大役,尚須數年,今猝興三大役於災旱貧虛之際。此其必不可者三也。



 就令商胡可塞,故道未必可開。鯀障洪水,九年無功,
 禹得《洪範》五行之書,知水潤下之性,乃因水之流,疏而就下,水患乃息。然則以大禹之功,不能障塞,但能因勢而疏決爾。今欲逆水之性,障而塞之,奪洪河之正流,使人力斡而回注,此大禹之所不能。此其必不可者四也。



 橫隴湮塞已二十年,商胡決又數歲,故道已平而難鑿,安流已久而難回。此其必不可者五也。



 臣伏思國家累歲災譴甚多,其於京東,變異尤大。地貴安靜而有聲,巨嵎山摧,海水搖蕩,如此不止者僅十年,天地警戒,宜不
 虛發。臣謂變異所起之方,尤當過慮防懼,今乃欲於兇艱之年,聚三十萬之大眾於變異最大之方,臣恐災禍自茲而發也。況京東赤地千里,饑饉之民,正苦天災。又聞河役將動,往往伐桑毀屋,無復生計。流亡盜賊之患,不可不虞。宜速止罷,用安人心。



 九月,詔:「自商胡之決,大河注金堤,浸為河北患。其故道又以河北、京東饑,故未興役。今河渠司李仲昌議欲納水入六塔河,使歸橫隴舊河,舒一時之急。其令兩制至待制以上、臺諫官,與河
 渠司同詳定。」



 修又上疏曰:



 伏見學士院集議修河,未有定論。豈由賈昌朝欲復故道,李仲昌請開六塔,互執一說,莫知孰是。臣愚皆謂不然。言故道者,未詳利害之原;述六塔者,近乎欺罔之繆。今謂故道可復者,但見河北水患,而欲還之京東。然不思天禧以來河水屢決之因,所以未知故道有不可復之勢,臣故謂未詳利害之原也。若言六塔之利者,則不待攻而自破矣。今六塔既已開,而恩、冀之患,何為尚告奔騰之急?此則減水未見其
 利也。又開六塔者云,可以全回大河,使復橫隴故道。今六塔止是別河下流,已為濱、棣、德、博之患,若全回大河,顧其害如何?此臣故謂近乎欺罔之繆也。



 且河本泥沙,無不淤之理。淤常先下流,下流淤高,水行漸壅,乃決上流之低處,此勢之常也。然避高就下,水之本性,故河流已棄之道,自古難復。臣不敢廣述河源,且以今所欲復之故道,言天禧以來屢決之因。



 初,天禧中,河出京東,水行於今所謂故道者。水既淤澀,乃決天臺埽,尋塞而復
 故道;未幾,又決於滑州南鐵狗廟,今所謂龍門埽者。其後數年,又塞而復故道。已而又決王楚埽,所決差小,與故道分流,然而故道之水終以壅淤,故又於橫隴大決。是則決河非不能力塞,故道非不能力復,所復不久終必決於上流者,由故道淤而水不能行故也。及橫隴既決,水流就下,所以十餘年間,河未為患。至慶歷三、四年,橫隴之水,又自海口先淤,凡一百四十餘里;其後游、金、赤三河相次又淤。下流既梗,乃決於上流之商胡口。然
 則京東、橫隴兩河故道,皆下流淤塞,河水已棄之高地。京東故道,屢復屢決,理不可復,不待言而易知也。



 昨議者度京東故道功科,但云銅城已上乃特高爾,其東比銅城以上則稍低,比商胡已上則實高也。若云銅城以東地勢斗下,則當日水流宜決銅城已上,何緣而頓淤橫隴之口,亦何緣而大決也?然則兩河故道,既皆不可為,則河北水患何為而可去?臣聞智者之於事,有所不能必,則較其利害之輕重,擇其害少者而為之,猶愈害
 多而利少,何況有害而無利,此三者可較而擇也。



 又商胡初決之時,欲議修塞,計用梢芟一千八百萬,科配六路一百餘州軍。今欲塞者乃往年之商胡,則必用往年之物數。至於開鑿故道,張奎所計工費甚大,其後李參減損,猶用三十萬人。然欲以五十步之狹,容大河之水,此可笑者,又欲增一夫所開三尺之方,倍為六尺,且闊厚三尺而長六尺,自一倍之功,在於人力,已為勞苦。云六尺之方,以開方法算之,乃八倍之功,此豈人力之所
 勝?是則前功既大而難興,後功雖小而不實。



 大抵塞商胡、開故道,凡二大役,皆困國勞人,所舉如此,而欲開難復屢決已驗之故道,使其虛費,而商胡不可塞,故道不可復,此所謂有害而無利者也。就使幸而暫塞,以紓目前之患,而終於上流必決,如龍門、橫隴之比,此所謂利少而害多也。



 若六塔者,於大河有減水之名,而無減患之實。今下流所散,為患已多,若全回大河以注之,則濱、棣、德、博河北所仰之州,不勝其患,而又故道淤澀,上流
 必有他決之虞,此直有害而無利耳,是皆智者之不為也。今若因水所在,增治堤防,疏其下流,浚以入海,則可無決溢散漫之虞。



 今河所歷數州之地,誠為患矣;堤防歲用之夫,誠為勞矣。與其虛費天下之財,虛舉大眾之役,而不能成功,終不免為數州之患,勞歲用之夫,則此所謂害少者,乃智者之所宜擇也。



 大約今河之勢,負三決之虞:復故道,上流必決;開六塔,上流亦決;河之下流,若不浚使入海,則上流亦決。臣請選知水利之臣,就其
 下流,求入海路而浚之;不然,下流梗澀,則終虞上決,為患無涯。臣非知水者,但以今事可驗者較之耳。願下臣議,裁取其當焉。



 預議官翰林學士承旨孫抃等言:開故道,誠久利,然功大難成;六塔下流,可導而東去,以紓恩、冀金堤之患。



 十二月,中書上奏曰:「自商胡決,為大名、恩冀患。先議開銅城道,塞商胡,以功大難卒就,緩之,而憂金堤泛溢不能捍也。願備工費,因六塔水勢入橫隴,宜令河北、京東預完堤埽,上河水所居民田數。」詔下中書
 奏,以知澶州事李璋為總管,轉運使周沆權同知潭州,內侍都知鄧保吉為鈐轄,殿中丞李仲昌提舉河渠,內殿承制張懷恩為都監。而保吉不行,以內侍押班王從善代之。以龍圖閣直學士施昌言總領其事,提點開封府界縣鎮事蔡挺、勾當河渠事楊緯同修河決。修又奏請罷六塔之役,時宰相富弼尤主昌議,疏奏亦不省。



 嘉祐元年四月壬子朔,塞商胡北流,入六塔河,不能容,是夕復決,溺兵夫、漂芻蒿不可勝計。命三司鹽鐵判官
 沈立往行視,而修河官皆謫。宦者劉恢奏:「六塔之役,水死者數千萬人,穿土干禁忌;且河口乃趙徵村,於國姓、御名有嫌,而大興臿斫,非便。」詔御史吳中復、內侍鄧守恭置獄於澶。劾仲昌等違詔旨,不俟秋冬塞北流而擅進約,以致決潰。懷恩、仲昌仍坐取河材為器,懷恩流潭州,仲昌流英州,施昌言、李璋以下再謫,蔡挺奪官勒停。仲昌,垂子也。由是議者久不復論河事。



 五年,河流派別於魏之第六埽,曰二股河,其廣二百尺。自二股河行一
 百三十里,至魏、恩、德、博之境,曰四界首河。七月,都轉運使韓贄言:「四界首古大河所經,即《溝洫志》所謂『平原、金堤,開通大河,入篤馬河,至海五百餘里』者也。自春以丁壯三千浚之,可一月而畢。支分河流入金、赤河,使其深六尺,為利可必。商胡決河自魏至於恩冀、乾寧入於海,今二股河自魏、恩東至於德、滄入於海,分而為二,則上流不壅,可以無決溢之患。」乃上《四界首二股河圖》。七年七月戊辰,河決大名第五埽。



 英宗治平元年,始命都水
 監浚二股、五股河,以紓恩、冀之患。初,都水監言:「商胡堙塞,冀州界河淺,房家、武邑二埽由此潰,慮一旦大決,則甚於商胡之患。」乃遣判都水監張鞏、戶部副使張燾等行視,遂興工役,卒塞之。



 神宗熙寧元年六月,河溢恩州烏欄堤,又決冀州棗強埽,北注瀛。七月,又溢瀛州樂壽埽。帝憂之,顧問近臣司馬光等。都水監丞李立之請於恩、冀、深、瀛等州,創生堤三百六十七里以御河,而河北都轉運司言:「當用夫八萬三千餘人,役一月成。今方災
 傷,願徐之。」都水監丞宋昌言謂:「今二股河門變移,請迎河港進約,簽入河身,以紓四州水患。」遂與屯田都監內侍程昉獻議,開二股以導東流。於是都水監奏:「慶歷八年,商胡北流,於今二十餘年,自澶州下至乾寧軍,創堤千有餘里,公私勞擾。近歲冀州而下,河道梗澀,致上下埽岸屢危。今棗強抹岸,沖奪故道,雖創新堤,終非久計。願相六塔舊口,並二股河導使東流,徐塞北流。」而提舉河渠王亞等謂:「黃、御河帶北行入獨流東砦,經乾寧軍、
 滄州等八砦邊界,直入大海。其近海口闊六七百步,深八九丈,三女砦以西闊三四百步,深五六丈。其勢愈深,其流愈猛,天所以限契丹。議者欲再開二股,漸閉北流,此乃未嘗睹黃河在界河內東流之利也。」



 十一月,詔翰林學士司馬光、入內內侍省副都知張茂則,乘傳相度四州生堤,回日兼視六塔、二股利害。二年正月,光入對:「請如宋昌言策,於二股之西置上約,擗水令東。俟東流漸深,北流淤淺,即塞北流,放出御河、胡盧河,下紓恩、冀、
 深、瀛以西之患。」



 初,商胡決河自魏之北,至恩、冀、乾寧入於海,是謂北流。嘉祐五年,河流派於魏之第六埽,遂為二股,自魏、恩東至於德、滄,入於海,是謂東流。時議者多不同,李立之力主生堤,帝不聽,卒用昌言說,置上約。



 三月,光奏:「治河當因地形水勢,若強用人力,引使就高,橫立堤防,則逆激旁潰,不惟無成,仍敗舊績。臣慮官吏見東流已及四分,急於見功,遽塞北流。而不知二股分流,十里之內,相去尚近,地勢復東高西下。若河流並東,一
 遇盛漲,水勢西合入北流,則東流遂絕;或於滄、德堤埽未成之處,決溢橫流。雖除西路之患,而害及東路,非策也。宜專護上約及二股堤岸。若今歲東流止添二分,則此去河勢自東,近者二三年,遠者四五年,候及八分以上,河流沖刷已闊,滄、德堤埽已固,自然北流日減,可以閉塞,兩路俱無害矣。」



 會北京留守韓琦言:「今歲兵夫數少,而金堤兩埽,修上、下約甚急,深進馬頭,欲奪大河。緣二股及嫩灘舊闊千一百步,是以可容漲水。今截去八
 百步有餘,則將束大河於二百餘步之間,下流既壅,上流蹙遏湍怒,又無兵夫修護堤岸,其沖決必矣。況自德至滄,皆二股下流,既無堤防,必侵民田。設若河門束狹,不能容納漲水,上、下約隨流而脫,則二股與北流為一,其患愈大。又恩、深州所創生堤,其東則大河西來,其西則西山諸水東注,腹背受水,兩難捍禦。望選近臣速至河所,與在外官合議。」帝在經筵以琦奏諭光,命同茂則再往。



 四月,光與張鞏、李立之、宋昌言、張問、呂大防、程昉
 行視上約及方鋸牙,濟河,集議於下約。光等奏:「二股河上約並在灘上,不礙河行。但所進方鋸牙已深,致北流河門稍狹,乞減折二十步,令近後,仍作蛾眉埽裹護。其滄、德界有古遙堤,當加葺治。所修二股,本欲疏導河水東去,生堤本欲捍禦河水西來,相為表裏,未可偏廢。」帝因謂二府曰:「韓琦頗疑修二股。」趙抃曰:「人多以六塔為戒。」王安石曰:「異議者,皆不考事實故也。」帝又問:「程昉、宋昌言同修二股如何?」安石以為可治。帝曰:「欲作簽河甚
 善。」安石曰:「誠然。若及時作之,使決河可東,北流可閉。」因言:「李立之所築生堤,去河遠者至八九十里,本計以御漫水,而不可御河南之向著,臣恐漫水亦不可御也。」帝以為然。五月丙寅,乃詔立之乘驛赴闕議之。



 六月戊申,命司馬光都大提舉修二股工役。呂公著言:「朝廷遣光相視董役,非所以褒崇近職、待遇儒臣也。」乃罷光行。



 七月,二股河通快,北流稍自閉。戊子,張鞏奏:「上約累經泛漲,並下約各已無虞,東流勢漸順快,宜塞北流,除恩、冀、
 深、瀛、永靜、乾寧等州軍水患。又使御河、胡盧河下流各還故道,則漕運無壅遏,郵傳無滯留,塘泊無淤淺。復於邊防大計,不失南北之限,歲減費不可勝數,亦使流移歸復,實無窮之利。且黃河所至,古今未嘗無患,較利害輕重而取舍之可也。惟是東流南北堤防未立,閉口修堤,工費甚伙,所當預備。望選習知河事者,與臣等講求,具圖以聞。」乃復詔光、茂則及都水監官、河北轉運使同相度閉塞北流利害,有所不同,各以議上。



 八月己亥,光
 入辭,言:「鞏等欲塞二股河北流,臣恐勞費未易。或幸而可塞,則東流淺狹,堤防未全,必致決溢,是移恩、冀、深、瀛之患於滄、德等州也。不若俟三二年,東流益深闊,堤防稍固,北流漸淺,薪芻有備,塞之便。」帝曰:「東流、北流之患孰輕重?」光曰:「兩地皆王民,無輕重;然北流已殘破,東流尚全。」帝曰;「今不俟東流順快而塞北流,他日河勢改移,奈何?」光曰:「上約固則東流日增,北流日減,何憂改移。若上約流失,其事不可知,惟當並力護上約耳。」帝曰:「上約
 安可保?」光曰:「今歲創修,誠為難保,然昨經大水而無虞,來歲地腳已牢,復何慮。且上約居河之側,聽河北流,猶懼不保;今欲橫截使不行,庸可保乎?」帝曰:「若河水常分二流,何時當有成功?」光曰:「上約茍存,東流必增,北流必減;借使分為二流,於張鞏等不見成功,於國家亦無所害。何則?西北之水,並於山東,故為害大,分則害小矣。鞏等亟欲塞北流,皆為身謀,不顧國力與民患也。」帝曰:「防捍兩河,何以供億?」光曰:「並為一則勞費自倍,分二流則
 勞費減半。今減北流財力之半,以備東流,不亦可乎?」帝曰:「卿等至彼視之。」



 時二股河東流及六分,鞏等因欲閉斷北流,帝意向之。光以為須及八分乃可,仍待其自然,不可施功。王安石曰:「光議事屢不合,今令視河,後必不從其議,是重使不安職也。」庚子,乃獨遣茂則。茂則奏:「二股河東傾已及八分,北流止二分。」張鞏等亦奏:「丙午,大河東徙,北流淺小。戊申,北流閉。」詔獎諭司馬光等,仍賜衣、帶、馬。



 時北流既塞,而河自其南四十里許家港東決,
 泛濫大名、恩、德、滄、永靜五州軍境。三年二月,命茂則、鞏相度澶、滑州以下至東流河勢、堤防利害。時方浚御河,韓琦言:「事有緩急,工有後先,今御河漕運通駛,未至有害,不宜減大河之役。」乃詔輟河夫卒三萬三千,專治東流。



\end{pinyinscope}