\article{志第八十五 樂七(樂章一)}

\begin{pinyinscope}

 郊祀祈穀雩祀五方帝感生帝



 建隆郊祀八曲



 降神,《高安》在國南方,時維就陽。以祈帝祉,式致民康。
 豆籩鼎俎,金石絲簧。禮行樂奏,皇祚無疆。



 皇帝升降,《隆安》步武舒遲,升壇肅祗。其容允若,於禮攸宜。



 奠玉幣,《嘉安》嘉玉制幣,以通神明。神不享物,享於克誠。



 奉俎,《豐安》笙鏞備樂,繭慄陳牲。乃迎芳俎,以薦高明。



 酌獻,《禧安》丹雲之爵,金龍之杓。挹於尊罍,是曰清酌。



 飲福,《禧安》潔茲五齊,酌彼六尊。致誠斯至,率禮彌敦。
 以介景福,永隆後昆。重熙累洽,帝道攸尊。



 亞獻、終獻,《正安》謂天蓋高,其聽孔卑。聞樂歆德,介以福禧。



 送神,《高安》倏兮而來,忽兮而回。雲馭杳邈,天門洞開。



 咸平親郊八首



 降神,《高安》圜丘何方?在國之陽。禮神合祭,運啟無疆。祖考來格,籩豆成行。其儀肅肅,降福穰穰。



 皇帝升降,《隆安》禮備樂成,乾健天行。帝容有穆,佩玉
 鏘鳴。



 奠玉幣,《嘉安》定位毖祀,告於神明。嘉玉量幣,享於克誠。



 奉俎,《豐安》有牲斯純,有俎斯陳。進於上帝,昭報深仁。



 酌獻,《禧安》大報於帝,盛德升聞。醴齊良潔,粢盛苾芬。



 飲福,《禧安》祀帝圜丘,九州獻力。禮行於郊,百神受職。靈祗格思,享我明德。天鑒孔章,玄祉昭錫。



 亞獻、終獻,《正安》羽鑰雲罷,干戚載揚。接神有恪,錫羨
 無疆。



 送神,《高安》神駕來思,風舉雲飛。神馭歸止,天空露晞。



 景祐親郊,三聖並侑二首



 奠幣,《廣安》千齡啟運,三後在天。嘉壇並侑,億萬斯年。



 酌獻,《彰安》皇基締構,帝系靈長。躬薦鬱鬯,子孫保昌。



 常祀二首



 太祖配位奠幣,《定安》翕受駿命,震疊群方。侑祀上帝,德厚流光。



 酌獻,《英安》誕受靈符,肇基丕業。配享潔尊,永隆萬葉。



 元符親郊五首



 餘同咸平,凡闕者皆用舊詞。



 降神,《景安》六變辭同。無為靡遠,深厚廣圻。祭神恭在,弁冕袞衣。粢盛豐美,明德馨輝。以祥以祐,非眇專祈。



 升降,《乾安》罍洗、飲福並奏。神靈擁衛,景從雲隨。玉色溫粹,天步舒遲。周旋陟降,皇心肅祗。千靈是保,百福攸宜。



 退文舞、迎武舞,《正安》左手執鑰,右手秉翟。進旅退旅,萬舞有奕。



 徹豆,《熙安》陟彼郊丘,大祀是承。其豆孔庶,其香始升。上帝時歆,以我齊明。卒事而徹,福祿來成。



 送神,《景安》馨遺八尊,器空二簋。至祝至虔,穹祗貺祉。



 政和親郊三首



 皇帝升降,《乾安》因山為高,爰陟其首。玉趾躩如,在帝左右。帝謂我王,予懷仁厚。眷言顧之,永綏九有。



 配位酌獻,《大寧》於穆文祖,妙道九德。默契靈心,肇基王跡。啟祐後人,垂裕罔極。合食昭薦,孝思維則。



 於
 皇順祖,積德累祥。發源深厚,不耀其光。基天明命,厥厚克昌。是孝是享,申錫無疆。



 高宗建炎初,國步尚艱,乃詔有司,天帝地祗及他大祀,先以時舉。太常尋奏,近已增募樂工,乾、羽、簨、虡亦備,始循舊禮,用登歌樂舞。其祀昊天上帝。



 降神用《景安》



 圜鐘為宮,三奏搜講上儀,式修毖祀。日吉辰良,禮成樂備。風馭雲旗,聿來歆止。嘉我馨德,介茲繁祉。



 黃鐘為角,一奏我將我享,涓選休成。執事有恪,惟寅惟清。樂既六變,肅雍和鳴。高高在上,庶幾是聽。



 太簇為征,一奏禮崇禋祀,備物薦誠。昭格穹昊,明德惟馨。風馬雲車,肸蠁居歆。申錫無疆,繼我思成。



 姑洗為羽,一奏惟天為大,物始攸資。恭承禋祀,以報以祈。神不可度,日監在茲。有馨明德,庶其格思。



 皇帝盥洗,《正安》靈承上帝,厲意專精。設洗於阼,罍水以清。盥以致潔,感通神明。無遠弗屆,其饗茲誠。



 升壇,《正安》皇矣上帝,神格無方!一陽肇復,典祀有常。豆登豐潔,薦德馨香。棐忱居歆,降福穰穰。



 上帝位奠玉幣,《嘉安》治極發聞,不瑕有芬。嘉玉陳幣,神屆欣欣。誠心昭著,欽恭無文。以安以侑,篤祜何垠。



 太祖位奠幣,《安定》茫茫蒼穹,孰知其紀!精意潛通,雖遠而邇。量幣薦誠,有實斯篚。睠然顧之,永錫繁祉。



 皇帝還位,《正安》典祀有常,昭事上帝。奉以告虔,逮迄奠幣。鐘鼓既設,禮儀既備。神之格思,恭承貺賜。



 捧俎,《豐安》祀事孔明,禮文惟楙。爰潔犧牲,載登俎豆。或肆或將,無聲無臭,精祲潛通,永綏我後。



 上帝酌獻,《嘉安》氣萌黃鐘,萬物資始。欽若高穹,吉蠲時祀。神筴泰元,增授無已。群生熙熙,函蒙繁祉。



 太祖位酌獻,《英安》赫赫翼祖,受命於天。德邁三代,威加八埏。陟配上帝,明禋告虔。流光垂裕,於萬斯年。



 文舞退、武舞進,《正安》大德曰生,陰陽寒暑。樂舞形容,干戚鑰羽。一弛一張,退旅進旅。神安樂之,祉錫綿宇。



 亞、終獻,《文安》惟聖普臨,順皇之德。典禮有彞,享祀不忒。籩豆靜嘉,降登肸飭。神具醉止,景貺咸集。



 徹豆,《肅安》內心齊誠,外物蠲潔。神來迪嘗,俎豆既徹。燕及群生,靡或夭閼。降福穰穰,時萬時億。



 送神,《景安》於赫上帝,乘龍御天。惟聖克事,明饗斯虔。薦豆雲徹,靈猋且旋。載錫休祉,其惟有年。



 望燎,《正安》靈承上帝,精意感通。馨香旁達,粢盛既豐。登降有儀,祀備樂終。神之聽之,福祿來崇。



 紹興十三年,初舉郊祀,命學士院制宮廟朝獻及圜壇行禮、登門肆赦樂章,凡五十有八。至二十八年,以臣僚有請改定,於是禦制樂章十有三及徽宗元禦制仁宗廟樂章一,共十有四篇。餘則分命大臣與兩制儒館之士,一新撰述,並懿節別廟樂曲凡七十有四,俱匯見焉。其祀圜丘。



 皇帝入中壝,《乾安》



 帝出於震,巽惟齊明。律曰姑洗,以示潔清。



 我交於神,蠲意必精,既盥而往,祈鑒斯誠。



 降神,《景安》



 陽動黃宮,日旋南極。天門蕩蕩,百神受職。



 爰熙紫壇,熉黃殊色。神哉沛來,蓋親有德。



 盥洗,《乾安》



 帝顧明德,監於克誠。齊戒滌濯,式示潔清。



 郊丘合祛,享意必精。既盥而薦,熙事備成。



 升壇,《乾安》



 帝監崇壇,媼神其從。稽古合祛,並侑神宗。



 升階奠玉,誠意感通。貺施鼎來,受福無窮。



 昊天上帝位奠玉幣,《嘉安》禦制



 上穹昊天,日星垂曜。照臨下土,王國是保。



 維玉與帛,寅恭昭報。永左右之,欽
 若至道。



 皇地祗位奠玉幣,《嘉安》禦制



 至哉坤厚,隤然止靜。柔載動植,資始成性。



 玉光幣色,璨若其映。式恭禋祀,有邦之慶。



 太祖皇帝位尊幣,《廣安》禦制



 明明翼祖,並侑泰壇。肇造綿宇,王業孔艱。



 表正封略,上際下蟠。躬以大報,亦止於燔。



 太宗皇帝位奠玉幣,《化安》禦制。



 赫赫巍巍,及時純熙。昊
 天成命,後則受之。



 登邁邃古,光被聲詩。有幣陟配,孫謀所貽。



 降壇,《乾安》



 躬展盛儀,天步逡巡。樂備禮交,嘉玉既陳。



 神方安坐,薦祉紛綸。陟降有容,皇心載勤。



 還位,《乾安》



 克昭王業,命成昊天。泰畤禋燎,八陛惟圜。



 肅然威儀,登降周旋。是謂精享,神監吉蠲。



 奉俎,《豐安》



 至大惟天,云何稱德!展誠致薦,牲用博碩。



 誠以牲寓,帝由誠格。居歆降祥,時萬時億。



 再詣盥洗,《乾安》



 帝出於震,巽惟潔齊。神明其德,乃稱禋柴。



 惟茲吉蠲,昭事聿懷。重盥而祀,敷錫孔皆。



 再升壇與初升同,惟易奠玉作奠酌。



 昊天上帝位酌獻,《禧安》禦制



 謁款壇陛,祗祀泰禋。丘圜自然,可格至神。



 桂尊登酌,嘉薦方新。靡福菲眇,敷祐下民。



 皇地祗位酌獻,《光安》禦制



 厚德光大,承元之明。茲潛莩吹,升於昭清。



 冰天桂海,咸資化成。恭酌彞醪,報本惟
 精。



 太祖皇帝位酌獻,《彰安》禦制



 於赫皇祖,創業立極。肅肅靈命,蕩蕩休德。



 嘉觴精潔,雅奏金石。丕顯神謨,惟後之則。



 太宗皇帝位酌獻,《韶安》禦制



 丕鑠帝宗,復受天命。群陰猶黷,一戎大定。



 奠鬯斯馨,功歌在詠。祐啟後人,文軌蚤正。



 還位,《乾安》



 肆類上帝,懷柔百神。稿秸既設,珪幣既陳。



 精誠潛交,已事而竣。祐我億載,基圖日新。



 入小次,《乾安》



 恭展美報,聿修上儀。禮樂和節,登降適宜。



 德焉斯親,神靡不娭。海內承福,式固邦基。



 文舞退、武舞進,《正安》



 泰元尊臨,富媼繁祉。於皇祖宗,既昭格止。



 奏舞象功,靈其有喜。永言孝思,盡善盡美。



 亞獻,《正安》



 陽丘其高,神祇並位。即奠厥玉,既奉厥醴。



 亦有嘉德,克相毖祀。旨酒載爵,以成熙事終獻同,止易
 再酌為三酌。



 出小次位,《乾安》



 爰熙紫壇,天地並貺。來燕來寧,畢陳鬱鬯。



 承神至尊,精意所鄉。告靈饗奠,祉福其暢。



 詣飲福位,《乾安》



 帝臨崇壇,媼神其從。祖宗並歆,福祿攸同。



 兵寢刑措,時和歲豐。其膺受之,將施無窮降壇同,止易「將」作「以」。



 飲福,《禧安》



 八音克諧,降神出祗。風馬雲車,陟降在茲。



 錫我純嘏,我應受之。一人有慶,燕及群黎。



 還位,《乾安》



 帝出於震,孝奏上儀。燔燎膻薌,神徠燕娭。



 肅若舊典,罔或不祗。既右饗之,翕受蕃厘。



 徹豆,《熙安》



 燎薌既升,爇膋以潔。於豆於登,焄紐蒿有飶。



 紫幄熉黃,神其安悅。將以慶成,薄言盍徹。



 送神,《景安》



 九霄眇邈,神不可求。何以降之?監德之修。



 三獻備成,神不可留。何以送之?保天之休。



 望燎,《乾安》



 謂天蓋高,陽噓而生。日月列宿,皆天之神。



 肆求厥類,與陽俱升。視燎於壇,以終其勤。



 望瘞,《乾安》



 謂地蓋厚,陰翕而成。社稷群望,皆地之靈。



 肆求厥類,與陰俱凝。視瘞於坎,以終其勤。



 還大次,《乾安》



 舞具八佾,樂備六成。大矣孝熙,厲意專精。



 已事而竣,回軫還衡。我應受之,以莫不增。



 還內,《採茨》



 五輅鳴鑾,八神警蹕。天官景從,莫不祗慄。



 祲威盛容,昭哉祖述。祚我無疆,葉氣充溢。



 寧宗郊祀二十九首



 皇帝入中壝,《乾安》



 合祀丘澤,登侑祖宗。顧諟惟精,靈承惟恭。



 有嚴皇儀,有莊帝容。監於克誠,肅肅雍雍。



 降神,《景安》



 圜鐘為宮



 天門蕩蕩,雲車陰陰。百神咸秩,三靈顧歆。



 神哉來娭,神哉溥臨。饗時宋德,翼翼小心。



 黃鐘為角



 華蓋既動,紫微洞開。星樞周旋,日車徘徊。



 靈兮顧祐,靈兮沛來。載燕載娭,式時壇垓。



 太簇為征



 泰尊媼厘,祖功宗德。辰躔陪營,岳瀆受職。



 神哉來下,神哉來格。饗德惟馨,留虞嘉席。



 姑洗為羽



 金石宣昭,羽旄紛綸。潔火夕照,明水夜陳。



 娭哉惟靈,娭哉惟神。風馬招搖,惟德之親。



 皇帝盥洗,《乾安》



 皇帝儉勤,盥用陶瓦。禮神頌祗,奠幣獻斝。



 月鑒陰肅,醴液融冶。挹彼注茲,禮無違者。



 升壇,《乾安》



 崇臺穹窿,高靈下墮。慶陰徬佛,從坐嶪峨。



 宵升於丘,時通權火。維天之命,百祿是荷。



 降壇



 帝饗於郊,一精二純。紫觚陟降,嘉玉妥陳。



 神方留娭,瑞貺紛綸。申錫無疆,螽斯振振。



 還位



 肅肅禮度,鋗□宮奏。天行徐謐,皇儀昭懋。



 光連重璧,物備簋豆。於皇以饗,無聲無臭。



 尚書奉俎



 列俎孔陳,嘉籩維實。鼎煁陽燧,玉流星液。



 我牲既碩,我薦既苾。神監下昭,安坐翔吉。



 再詣盥洗



 帝澄初觴,禮嚴再盥。精明顯昭,齊顒洞貫。



 靈娭留俞,神光炳煥。我宋受福,永壽於萬。



 再升壇



 紫壇岳立,神光夜燭。有儼旒採,有鸞佩玉。



 霄垠顧祐,祖宗熙穆。對越不忘,俾爾戩穀。



 降壇,《乾安》



 天容澄謐,景氣晏和。瓚斝薦醇,鋗璆葉歌。



 帝降庭止,夜其如何?神助之休,宜爾眾多。



 還位,《乾安》



 甘露流英,卿雲舒採。靈俞有喜,神光晻暖。



 穆穆來蒞,洋洋如在。帝用居歆,澤及四海。



 入小次,《乾安》



 聽惟饗德,監惟棐忱。顧諟思明,靈承思欽。



 永言端蒞,肅對下臨。上帝是皇,毋貳爾心。



 文舞退、武舞進,《正安》



 羽鑰陳容,干戚按節。德閑而泰,功勞而決。



 虞我神祇,揚我謨烈。盡美盡善,福流有截。



 亞獻,《正安》



 帝臨中壇,神從八陛。花玉展瑞,明馨薦醴。



 亦有嘉德,克相盛禮。獻茲重觴,降福彌彌。



 終獻,《正安》



 敬事天地,升侑祖宗。陳盥於三,介觴之重。



 秉德翼翼,有來雍雍。相予祀事,福嘏日溶。



 出小次,《乾安》



 孝奏展成,熙儀畢薦。光流桂俎,祥衍椒奠。



 風管晨凝,雲容天轉。拜貺於郊,右序詒燕。



 詣飲福位,《乾安》



 所饗惟清,所欽惟馨。靈喜留俞,天景窈冥。



 福祿來成,福祿來寧。皇用時斂,壽我慈庭。



 飲福,《禧安》,



 瓚斝觩,觥罍氤氳。有醴惟香,有酒惟欣。



 肸蠁豐融,懿懿芬芬。我龍受之,如川如云。



 降壇,《乾安》



 天錫多祉,皇受五福。言瞻瑤壇,迄奉瑄玉。



 昭星炳耀,元氣回復。帝儀載旋,有嘉穆穆。



 還位,《乾安》



 璇圖天深,鼎文日輝。慶流皇家,像炳紫微。



 乾回冕旒,云煥袞衣。何千萬年,式於九圍。



 尚書徹豆,《熙安》



 蘭豆既升,簠簋既登。禮備俎實,饗貴牲□烝。



 時乃告徹,器用畢興。祚我皇基,介福是膺。



 送神,《景安》



 神輔有德,來燕來娭。禮薦熙成,三靈逆厘。



 神饗有道,言旋言歸。福祉咸蒙,百世本支。



 詣望燎位,《乾安》



 莫神乎天,陽噓而生。日月星辰,皆乾之精。



 肆求厥類,與陽俱升。視燎於壇,展也大成。



 詣望瘞位,《乾安》



 地載萬物,陰翕而成。山岳河瀆,皆坤之靈。



 克肖其象,與陰俱凝。視瘞於坎,思求厥成。



 還大次,《乾安》



 福方流胙,祈方欽柴。鹵簿載肅,球架允諧。



 帝祉具臨,皇靈允懷。遄御於次,降福孔皆。



 還內,《乾安》



 八福呵蹕,千官景從。回軫還衡,昆威盛容。



 妥飾芝鳳,御朝雲龍。歸壽慈闈,敷時民雍。



 景祐上辛祈穀,仁宗御制二首



 太宗配位奠幣,《仁安》



 天祚有開,文德來遠。祈彀日辛,侑神禮展。



 酌獻,《紹安》



 於穆神宗,惟皇永命。薦醴六尊,聲歌千詠。



 紹興祈穀三首



 降神、盥洗、升壇、還位及上帝奠玉幣、奉俎,並同圜丘。



 太宗位奠幣,《宗安》



 於穆思文,克配上帝。涓選休成,遵揚嚴衛。



 祗薦明誠,肅陳量幣。享茲吉蠲,申錫來裔。



 上帝位酌獻,《嘉安》



 三陽肇新,萬物資始。精誠祈天,其聽斯邇。



 願均雨暘,田疇之喜。如坻如京,以備百禮。



 太宗位酌獻,《德安》



 天錫勇智,允惟太宗。功隆德盛,與帝比崇。



 禮嚴陟配,誠達精衷。尚其錫祉,歲以屢豐。



 孟夏雩祀,仁宗御制二



 太祖配坐奠幣,《獻安》



 昊天蓋高,祀事為大。嚴配皇靈,億福來介。



 酌獻,《感安》



 龍見而雩,神之來格。犧象精良,威靈赫奕。



 紹興雩祀一首



 上帝位酌獻,《嘉安》



 蒼蒼昊穹,覆臨下土。欽惟歲事,民所依怙。



 爰竭精虔,禮典斯舉。甘澤以時,介我稷黍。



 冬至、孟春、孟夏、季秋四祀,上公攝事七首



 降神,《景安》二章



 天何言哉,至清而健!默定幽贊,降祥福善。



 夙設圜壇,恭陳嘉薦。貞馭下臨,儲休錫羨。



 生物之祖,興益之宗。於國之陽,以禋昊穹。



 六變降神,於論鼓鐘。親德享道,錫羨無窮。



 太尉行,《正安》



 禮經之重,祭典為宗。上公攝事,登降彌恭。



 庶品豐潔,令儀肅雍。百神萃止,惟吉之從。



 司徒奉俎,《豐安》



 禮崇禋祀,神鑒孔明。牲牷博腯,以炰以烹。



 馨香蠲潔,品物惟精。錫以純嘏,享茲至誠。



 飲福,《廣安》



 簠簋既陳,吉蠲登薦。洗心防邪,肅祗祭典。



 陟降惟寅,籩豆有踐。百福咸宜,淳耀丕顯。



 亞、終獻,《文安》



 秩秩禮文,肅肅嚴祀。仰洽神休,式協民紀。



 灌獻有容,敘其俎簋。明德惟馨,以介丕祉。



 送神,《景安》



 帝臨中壇,肅恭禋祀。靈景舒光,飛龍旋軌。



 送神有章,神心具醉。輔德惟仁,永錫元祉。



 景德以後祀五方帝十六首



 青帝降神,《高安》六變



 四序伊始,三陽肇新。氣迎東郊,蟄戶咸春。



 功宣播殖,澤被生民。祝史正辭,昭事惟寅。



 奠玉幣、酌獻,並用《嘉安》



 條風始至,盛德在木。平秩東作,種獻穜穋。



 律應青陽,氣和玉燭。惠彼兆民,以介景福。



 送神,《高安》



 備物致用,薦羞神明。禮成樂舉,克享克禋。



 酌獻,《祐安》



 條風斯應,候歷維新。陽和啟蟄,呂物皆春。



 篪簧協奏,簠簋畢陳。精羞豐薦,景福攸臻。



 赤帝降神,《高安》



 長嬴戒序,候正南訛。功資蕃育,氣應清和。



 鼎實嘉俎,樂備登歌。神其來享,降福孔多。



 奠玉幣、酌獻,《嘉安》景祐用《祐安》,辭亦不同



 象分離位,德配炎精。景風協律,化神含生。



 百嘉茂育,乃順高明。神無常享,享於克誠。



 送神,《高安》



 籩豆有踐,黍稷惟馨。禮終三獻,神歸杳冥。



 黃帝降神,《高安》



 坤輿厚載,黃裳元吉。宅中居正,含章抱質。



 分王四季,其功靡秩。育此群生,首茲六律。



 奠玉幣、酌獻,《嘉安》景祐用《祐安》,辭亦不同



 中央定位,厚德惟新。五行攸正,四氣爰均。



 笙鏞以間,簠簋斯陳。為民祈福,肅奉明禋。



 送神,《高安》



 土德居中,方輿配位。樂以送神,式申昭事。



 白帝降神,《高安》



 西顥騰晶,天地始肅。盛德在金,百嘉
 茂育,



 擴弩射牲,築場登穀。明靈格思,旌罕紛屬。



 尊玉幣、酌獻,《嘉安》景祐用《祐安》,辭亦不同



 博碩肥腯,以炰以烹。嘉慄旨酒,有彌斯盈。



 肴核惟旅,肅肅烝烝。吉蠲備物,享於克誠。



 送神,《高安》



 飆輪戾止,景燭靈壇。金奏繹如,白露漙漙。



 黑帝降神,《高安》



 隆冬戒序,歲歷順成。一人有慶,萬物由庚。



 有旨斯酒,有碩斯牲。報功崇德,正直聰明。



 奠玉幣、酌獻,《嘉安》景祐用《祐安》,辭亦不同



 大儀斡運,星紀環周。
 三時不害,黍稷盈疇。



 克誠致享,品物咸羞。禮成樂變,錫祚貽休。



 送神,《高安》



 管磬咸和,禮獻斯畢。靈歟言旋,神降之吉。



 紹興以後祀五方帝六十首



 青帝降神,《高安》



 圜鐘宮三奏



 於神何司,而德於木?肅然顧歆,則我斯福。



 我祀孔時,我心載祗。匪我之私,神來不來。



 黃鐘為角,一奏



 神兮焉居?神在震方。仁以為宅,秉天
 之陽。



 神之來矣,道修以阻。望神未來,使我心苦。



 太簇為征,一奏



 神在途矣,習習以風。百靈後先,敢一不恭!



 奔走癘疫,祓除菑兇。顧瞻下方,逍遙從容。



 姑洗羽一奏



 溫然仁矣,熙然春矣。龍駕帝服,穆將臨矣。



 我酒清矣,我肴烝矣。我樂備矣,我神顧矣。



 升殿,《正安》



 在國之東,有壇崇成。節以和樂,式降式登。



 潔我佩服,璆琳鏘鳴。匪壇斯高,曷妥厥靈?



 青帝奠玉幣,《嘉安》



 物之熙熙,胡為其然。蒙神之休,乃
 敢報旃。



 有邸斯珪,有量斯幣。於以奠之,格此精意。



 太昊氏位尊幣,《嘉安》



 卜歲之初,我迎春祗。孰克侑饗,曰古宓戲,萬世之德。



 再拜稽首,敢愛斯璧。



 奉俎,《豐安》



 靈兮安留,煙燎既升。有碩其牲,有俎斯承。



 匪牲則碩,我德惟馨。緩節安歌,庶幾是聽。



 青帝酌獻,《祐安》



 百末布蘭,我酒伊旨。酌以匏爵,洽我百禮。



 帝居青陽,顧予嘉觴。右我天子,宜君宜王。



 太昊酌獻,《祐安》



 五德之王,誰實始之?功括造化,與天
 無期。



 酌我清酤,盥獻載飭。神鑒孔饗,天子之德。



 亞、終獻,《文安》



 貳觴具舉,承神嘉虞。神具醉止,眷焉此都。



 我歲方新,我畝伊殖。時暘時雨,繄神之力。



 送神,《高安》



 忽而來兮,格神鴻休。忽而往兮,神不予留。



 神在天兮,福我壽我。千萬春兮,高靈下墮。



 赤帝降神,《高安》



 圜鐘為宮



 離明御正,德協於火。有感其生,維帝是何。



 帝圖炎炎,貽福錫我。鑒於妥虔,高靈下墮。



 黃鐘為角



 赤精之君,位於朱明。茂育萬物,假然長嬴。



 我潔我盛,我蠲我誠。神其下來,雲車是承。



 太簇為征



 八卦相蕩,一氣散施。隆熾恢臺,職神尸之。



 肅肅飆禦,神戾於天。於昭神休,天子萬年。



 姑洗為羽



 燁燁其光,炳炳其靈。窅其如容,欻其如聲。



 扇以景風,導以朱斿。我德匪類,神其安留。



 升殿,《正安》



 除地國南,有基崇祟。載陟載降,式虔式恭。



 燎煙既燔,黻冕斯容。神如在焉,肆予幽通。



 赤帝奠玉幣,《嘉安》



 太微呈祥,炎德克彰。祐我基命,格於明昌。



 一純二精,有嚴典祀。於以奠之,以介繁祉。



 神農氏奠幣,《嘉安》



 練以纁黃,有篚將之。肸蠁斯答,有神昭之。



 維神於民,實始貨食。歸德報功,敢怠王國。



 奉俎,《豐安》



 有牲在滌,從以騂牡。或肆或將,有潔其俎。



 神嗜飲食,飶飶芬芬。莫腆於誠,神其顧歆!



 赤帝酌獻,《祐安》



 四月維夏,兆於重離。帝執其衡,物無癘疵。



 於皇帝功,思樂旨酒,奠爵既成,垂福則有。



 神農氏酌獻,《祐安》



 猗歟先農,肇茲黍稷!既殖既播,有此粒食。



 秬鬯潔清,彞樽疏冪。竭我瑤斝,莫報嘉績。



 亞、終獻,《文安》



 盥爵奠斝,載虔載恭。籩豆靜嘉,於樂鼓鐘。



 禮備三獻,神具醉止。孰顯神德?揚光紛委。



 送神,《高安》



 神來何從?馺然靈風。神去何之?杳然幽蹤。



 伊神去來,霧散雲烝。獨遺休祥,山崇川增。



 黃帝降神,《高安》



 圜鐘為宮



 維帝奠位,乃咸於時。孰主張是,而樞紐之?



 穀我腹我,比予於兒。告我冠服,迨其
 委蛇。



 黃鐘角



 蓀無不在,日輿我居。孰不可來?肸蠁斯須。



 象服龍駕,淵淵鼓桴。蓀不汝多,多汝意孚。



 太簇征



 樂哉帝居,逝留無常!爾信我宅,爾中我鄉。



 乃眷茲土,於赫君王。翩然下來,去未遽央。



 姑洗羽



 澹兮撫琴,啾兮吹笙。神之未來,肅穆以聽。



 繽紛羽旄,姣服在中。神既來止,亦無惰容。



 升殿,《正安》



 民生地中,動作食息。輿我周旋,莫匪爾極。



 捕鰈東海,搴茅南山。彼勞如何,矧升降間!



 黃帝奠玉幣,《嘉安》



 萬櫝之寶,一絇之絲。孕之育之,誰為此施?



 歸之後神,神曰何為?不宰之功,蕩然四垂。



 有熊氏位奠幣,《嘉安》



 維有熊氏,以土勝王。其後皆沿,茲德用壯。



 黼黻幅舄,裳衣是創。幣之元纁,對此昭亮。



 奉俎,《豐安》



 王曰欽哉,無愛斯牲!登我元祀,亦有皇靈。



 以將以享,或剝或烹。大夫之俎,天子之誠。



 黃帝酌獻,《祐安》



 黍以為翁,鬱以為婦。以侑元功,以酌
 大斗。



 伊誰歆之?皇皇帝後。伊誰嘏之?天子萬壽。



 有熊氏酌獻,《祐安》



 昔在綿邈,有人公孫。登政撫辰,節用良勤。



 所蓄既大,所行宜遠。載其華樽,從以簫管。



 亞、終獻,《文安》



 羽觴更陳,厥味清涼。飲之不煩,又有蔗漿。



 夜未艾止,明星浮浮。願言妥靈,靈兮淹留。



 送神,《高安》



 靈不肯留,沛兮將歸。玉節猋逝,翠旗並馳。



 顧瞻佇立,悵然佳期。蹇千萬年,無斁人斯。



 白帝降神。《高安》



 圜鐘為宮



 白藏啟序,庶匯向成。有嚴
 禋祀,用答幽靈。



 風馬雲車,來燕來寧。洋洋在上,休福是承。



 黃鐘角



 素精肇節,金行固藏。氣沖炎伏,明河翻霜。



 功收有年,禮薦有章。祗越眇冥,鴻基永昌。



 太簇征



 昊天之氣,揫斂萬匯。涓日潔齊,有嚴厥祀。



 有牲維肥,有酒維旨。神之燕娭,錫茲福祉。



 姑洗羽



 執矩斯兌,實惟素靈。受職儲休,萬寶以成。



 饗於西郊,奠玉陳牲。侑以雅樂,來歆克誠。



 升殿,《正安》



 素猋諧律,西顥墮靈。肇復元祀,晨煬肅清。



 下土層陔,嘉薦芳馨。以禦蕃祉,介我西成。



 白帝奠玉幣,《嘉安》



 惟時素秋,肇舉元祀。禮備樂作,降登有數。



 洋洋在上,神既來止。神之格思,錫我繁祉。



 少昊氏位奠幣,《嘉安》



 西顥肅清,群生茂遂。有嚴報典,孔明祀事。



 珪幣告虔,神靈燕喜。繼我豐年,以錫民祉。



 奉俎,《豐安》



 洽禮既陳,諧音具舉。有滌斯牲,孔碩為俎。



 維帝居歆,介我稷黍。樂哉有秋,繄神之祜!



 白帝酌獻,《祐安》



 徂商肇祀,靈蓋孔饗。恭承嘉禧,湛澹秬鬯。



 監此馨香,靈其安留。疇惠下民,匪靈之休。



 少昊氏位酌獻,《祐安》



 沆碭西顥,功載萬世。乘金宅兌,侑我明祀。



 嘉觴布蘭,牲玉潔精。神之燕虞,肅用有成。



 亞、終獻,《文安》



 肅成萬物,泬寥其秋。惟茲祀事,戾止靈斿。



 酌獻具舉。典禮是求。冀福斯民,黍稷盈疇。



 送神,《高安》



 沆碭白藏,順成萬寶。有來德馨,於昭神妥。



 露華晨晞,飆馭聿還。介我嗣歲,澤均幅員。



 黑帝降神,《高安》



 圜鐘為宮



 吉日壬癸,律中應鐘。國有故常,北郊迎冬。



 乃蕆祀事,必祗必恭。明默雖異,感而遂通。



 黃鐘為角



 良月盈數,四氣推遷。帝於是時,典司其權。



 高靈下墮,降祉幅員。神之聽之,祀事罔愆。



 太簇為征



 北方之神,執權司冬。三時務農,於焉告功。



 禮備樂作,歸功於神。風馬來游,永錫斯民。



 姑洗為羽



 天地閉塞,盛德在水。黑精之君,降福羨祉。



 洋洋在上,若或見之。齊莊承祀,其敢斁思。



 升殿,《正安》



 昧爽昭事,煌煌露光。滌溉蠲潔,容儀肅莊。



 牲肥酒旨,薦此芬芳。降陟有序,禮無越常。



 黑帝奠玉幣,《嘉安》



 晨曦未升,天宇肅穆。祗若元祀,將以幣玉。



 神之格思,三獻茅縮。明靈懌豫,下土是福。



 高陽氏位奠幣,《嘉安》



 飆馭雲蓋,神之顧歆。丕昭禮容,發揚樂音。



 祀事既舉,仰當神心。申以嘉幣,式薦誠諶。



 奉俎,《豐安》



 辰牡孔碩,奉牲以告。秘祝非祈,豐年宜報。



 至意昭徹,交乎神明。降福穰穰,用燕群生。



 黑帝酌獻,《祐安》



 赫赫神游,周流八極。德馨上聞,於焉來格。



 不腆酒醴,用伸悃愊。神其歆之!民用響德。



 高陽氏酌獻,《祐安》



 十月納禾,民務藏蓋。不有神休,民罔攸賴。



 孟冬之吉,禮行不昧。神降百祥,昭著蓍蔡。



 亞、終獻,《文安》



 萬匯揫斂,時惟冬序。蠢爾黎氓,人此室處。



 酌獻告神,禮以時舉。賴此陰騭,民有所怙。



 送神,《高安》



 神之戾止,天門夜開。禮備告成,雲軿亟回。



 旗纛晻靄,萬靈喧豗。獨遺祉福,用澤九垓。



 乾德以後祀感生帝十首



 降神,《大安》



 和均玉管,政協璇衡。四序資始,萬物含生。



 皇猷允洽,至德惟明。為民祈福,克致精誠。



 太保行,《保安》



 衣冠儼若,步武有容。公卿濟濟,率禮惟恭。



 罍洗,《正安》



 昊天降康,云何以報?斯謀斯惟,雍雍灌鬯。



 身之潔兮,神斯來止。神之享兮,民斯福矣。



 奠玉幣,《慶安》



 籩豆有踐,玉帛斯陳。神無常享,享於精純。



 奉俎,《咸安》



 俎實具列,明德惟馨。肅容祗薦,神其降靈。



 酌獻,《崇安》



 樂調鳳律,酒浥犧尊。至靈斯禦,盛德彌敦。



 飲福,《廣安》



 三陽戒律,萬匯騰精。既蘇昆蟲,畢達勾萌。



 具陳犧象,式薦誠明。錫以蕃祉,永保咸平。



 亞、終獻,《文安》



 大君有命,祀典咸修。薦獻式敘,淑慎優柔。



 徹豆,《肅安》以下二首政和中制



 奉承明祀,惟羊惟牛。卬盛於豆,備陳庶羞。



 鐘鼓喤□,神具醉止。其徹嘉籩,永綏福祉。



 送神,《普安》



 既臨下土,復歸於天。神之報貺,受福無邊。



 景祐祀感生帝二首



 宣祖配位奠幣,《皇安》



 浚發長源,粵惟始祖。五運協圖,萬靈來護。



 酌獻,《肅安》



 龍德而隱,源流則長。宜乎億祀,侑享彌昌。



 元符祀感生帝五首



 降神,《大安》六變



 二儀交泰,七政順行。四序資始,萬物含生。



 皇朝創業,盛德致平。為民祈福,潔此精誠。



 初獻升降,《保安》



 冕旒儼若,步武有容。公卿濟濟,《韶》、《濩》邕邕。



 帝位酌獻樂和鳳律,酒奠犧尊。神明斯享,禮盛難論。



 亞、終獻,《文安》



 大君有命,闕典咸修。帝歆明祀,祐聖千秋。



 送神,《普安》



 俯臨下土,回復上天。觸類而長,荷福無邊。



 帝位奠玉幣同前《慶安》,禧祖奠幣同景祐《皇安》,酌獻同景祐宣祖《肅安》,奉俎同熙寧《咸安》。



 紹興以後祀感生帝十六首



 降神,《大安》



 圜鐘為宮



 炎精之神,飛軿碧落。駕以浮雲,丹書赤雀。



 禮備豆籩,樂諧簫勺。神具醉止,祐我景鑠。



 黃鐘為角



 宋德惟火,神實司之。上儀申蕆,迎方重離。



 瑤幣告潔,秀華金支。啾啾神龍,來介繁禧。



 太簇為征



 於物司火,於方峙南。璇霄來下,羽衛毿毿。



 祠官祝厘,聊佩合簪。本支有衍,則百斯男。



 姑洗為羽



 惟神之安,方解羽鑾。赤旗霞曳,從以炎官。



 居歆嘉薦,肸蠁靈壇。神之格矣,民訖多盤。



 盥洗,《保安》



 沖牙鏘鳴,肅容專精。交神之義,罔敢弗誠。



 設洗於阼,罍水惟清。盥以致潔,感通神明。



 升殿,《保安》



 三陽交泰,日新惟良。大建厥祀,茲報興王。



 禮嚴陟降,德薦馨香。聿懷嘉慶,降福穰穰。



 感生帝位奠玉幣,《光安》



 肅肅嚴祀,神幽必聞。騁駕臨
 饗,將歆飶芬。



 嘉玉陳幣,欽恭無文。永綏多祜,國祚何垠。



 僖祖位奠幣,《皇安》



 於穆文獻,景炎發祥。啟茲皇運,垂慶無疆。



 篚幣有陳,式昭肅莊。神之格思,如在洋洋。



 奉俎,《咸安》



 籩豆大房,秩秩在列。奉牲以告,既全既潔。



 樂均無爽,牲醴攸設。神兮燕娭。霓旌孑孓。



 感生帝位酌獻,《崇安》



 盛德在火,相我炎祚。典祀有常,牲玉維具。



 風馬雲車,翩翩來顧。式蕃帝祉,後昆有裕。



 僖祖位酌獻,《肅安》



 皇矣文獻,開國有先。德配感生,對越在天。



 練日得辛,來止靈壇。神其錫羨,瑞應猗蘭。



 文舞退、武舞進,《正安》



 苾苾芬芬,神具醉止。笙磬鏗鏘,干旄旖旎。



 鬷假無言,神靈惟喜。申錫蕃厘,暨我孫子。



 亞、終獻,《文安》



 偉炎厥初,緣感而系。慶衍式崇,昭融有契。



 樂功既諧,觴獻斯繼。歆類不違,克昌百世。



 徹豆,《肅安》



 潔陳斯備,昭格惟禋。神歆以飫,宰徹其餕。



 清歌振曉,葉氣流春。永錫祚嗣,以渥烝民。



 送神,《大安》



 豐祀孔飾,肅來自天。蘭尊既徹,飆馭載遄。



 騎雲縹緲,聆樂流連。惟邁惟顧,降福綿綿。



 望燎,《普安》



 禮文既洽,熏燎聿升。嘉氣四塞,丹誠上騰。



 惟類之應,惟福之興。永熾天統,億載靈承。



\end{pinyinscope}