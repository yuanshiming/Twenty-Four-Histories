\article{本紀第八}

\begin{pinyinscope}

 真宗
 三



 四年春正月辛巳,詔執事汾陰懈怠者,罪勿原。乙酉,習祀后土儀。丁亥,將祀汾陰,謁啟聖院太宗神御殿、普安院元德皇后聖容。丙申,詔以六月六日天書再降日為
 天貺節。丁酉,奉天書發京師。日上有黃氣如匹素,五色雲如蓋,紫氣翼仗。庚子,右僕射、判河陽張齊賢見於汜水頓。陳堯叟獻白鹿。辛丑,陳幄殿于訾村,望拜諸陵。甲辰,至慈澗頓,賜道傍耕民茶荈。



 二月戊申,賜扈駕諸軍緡錢。華州獻芝草。東京獄空。壬子,出潼關,渡渭河,遣近臣祠西嶽。癸丑,次河中府。丁巳,黃雲隨天書輦。次寶鼎縣奉祗宮。戊午,登後圃延慶亭。己未,瀵泉湧,有光如燭。辛酉,祀后土地祗。是夜,月重輪,還奉祗宮,紫氣四塞。
 幸開元寺,作大寧宮。壬戌,甘州回鶻、蒲端、三麻蘭、勿巡、蒲婆、大食國、吐蕃諸族來貢。大赦天下,常赦不原者咸赦除之。文武官並遷秩,該敘封欲回授祖父母者聽。四品以上,迨事太祖、太宗潛藩或嘗更邊任家無食祿者,錄其子孫。建寶鼎縣為慶成軍。建隆佐命及公王將相丘塚,所在致祭。給西京分司官實奉三分之一。令法官慎刑名,有情輕法重者以聞。賜天下酺三日。大宴群臣於穆清殿,賜父老酒食衣幣。作《汾陰配饗銘》、《河瀆四海贊》。
 召草澤李瀆、劉巽,瀆以疾辭,授巽大理評事。乙丑,觀酺。加號西嶽。詔葺夷齊祠。丁卯,賜寧王元偓服帶、鞍馬有加。乙巳,次華州,幸雲臺觀。庚午,宴宣澤亭,紫雲如龍,起嶽上。召見隱士鄭隱、李寧,賜茶果、束帛。辛未,次閿鄉縣,召見道士柴又玄,問以無為之要。壬申,宴虢州父老於湖城行宮。



 三月甲戌,次陜州,召草澤魏野,辭疾不至。乙亥,賜運船卒時服。己卯,次西京。庚辰,罷河北緣邊工役。壬午,幸上清宮。甲申,幸崇法院,移幸呂蒙正第,賜服御、
 金幣。丙戌,大宴大明殿。丁亥,詔葺所經歷代帝王祠廟。己丑,御五鳳樓觀酺。壬辰,詔朝陵自西京至鞏縣不舉樂。癸巳,禁扈從人踐田稼。甲午,發西京。丙申,謁安陵、永昌諸陵。壬寅,幸列子廟,表潘孝子墓。



 夏四月甲辰朔,上至自汾陰。壬子,幸元偁宮視疾。駙馬都尉李遵勖責授均州團練副使。峽路鈐轄執為亂夷人王群體等,帝憫其異俗,免死配隸。丙辰,大宴含光殿。己未,錢種放歸終南。甲子,王旦加右僕射,元佐為太尉,元偓進封相王。乙
 丑,幸元偓宮視疾。葺尚書省。加王欽若吏部尚書,陳堯叟戶部尚書,馮拯工部尚書。丙寅,以張齊賢為左僕射。丁卯,許國公呂蒙正薨。



 五月丙子,加交址郡王李公蘊同平章事。癸未,廬、宿、泗等州麥自生。辛卯,幸北宅視德存疾。京兆旱,詔振之。癸巳,詔州城置孔子廟。乙未,加上五岳帝號,作《奉神述》。丁酉,慮囚,死罪流徒降等,杖以下釋之。辛丑,視德存疾。



 六月丙午,太白晝見。亳州二龍見禹祠。德存卒。丙寅,遣使安撫江、淮南水災,許便宜從事。
 詔授交、甘等州、大食、蒲端、三麻蘭、勿巡國奉使官。



 秋七月壬申朔,除閩、浙、荊湖、廣南歲丁錢四十五萬。壬午,韓國、吳國、隋國長公主進封衛國、楚國、越國長公主。鎮、眉、昌等州地震。己丑,詔先蠲濱、棣州水災田租十之三,今所輸七分更除其半。丙申,江、洪、筠、袁江漲,沒民田。



 八月乙巳,太白晝見。丙午,幸南宮視惟敘疾。詔除纁田租。庚戌,曲宴諸王、宰相。癸丑,賜青州孤老煢獨民帛。惟敘卒。丙辰,錄唐長孫無忌、段秀實等孫,授官。丁巳,詔文武官
 有言刑政得失、邊防機事者並賜對。癸亥,甘州回紇可汗夜落紇奉表詣闕。乙丑,刻御制《大中祥符頌》於左承天祥符門。河決通利軍,合御河,壞州城及傷田廬,遣使發粟振之。九月丁丑,涇原鈐轄曹瑋言籠竿川熟戶蕃部以閑田輸官,請於要害地募兵以居,從之。戊子,幸太乙宮祈晴。辛卯,向敏中等為五嶽奉冊使。癸巳,禦乾元樓觀酺。



 冬十月戊申,御朝元殿發五岳冊。丁巳,定江、淮鹽酒價,有司慮失歲課,帝曰:「茍便於民,何顧歲入也。」十一
 月庚午,占城國貢獅子。丙子,御試服勤詞學、經明行修貢舉人。



 十二月乙巳,詔楚、泰州潮害稼,復租。沒溺人賜千錢、粟一斛。是歲,西涼府、夏、豐、交州、甘州、諸溪峒蠻來貢。畿內蝗。河北、陜西、劍南饑。吉州、臨江軍江水溢,害民田舍。兗州孑□蟲,不為災。



 五年正月乙亥,賜處州進士周啟明粟帛。戊寅,雨木冰。壬午,幸元偁宮視疾。河決棣州。



 二月庚戌,詔貢舉人公罪聽贖。丙寅,詔官吏安撫濱、棣被水農民。



 三月己丑,御
 試禮部舉人。丁未,峒酋田仕瓊等貢溪布。庚戌,王旦等並加特進、功臣。丁巳,免濱、棣民物入城市者稅一年。



 夏四月戊申,以向敏中為平章事。有司請違法販茶者許同居首告,帝謂以利敗俗非國體,不許。壬子,除通、泰、楚州鹽亭戶積負丁額課鹽。乙丑,樞密直學士邊肅責授岳州團練副使。



 五月辛未,江、淮、兩浙旱,給占城稻種,教民種之。戊寅,修儀劉氏進封德妃。丁亥,免棣州租十之三。戊子,賜近臣金華殿所種麥。



 六月庚申,賜杭州草澤
 林逋粟帛。壬戌,詔常參官舉幕職、州縣官充京官。癸亥,賜邵武軍被水者錢粟。



 秋七月戊辰,作保康門。



 八月丙申朔,日有食之。丁酉,禁周太祖葬冠劍地樵採。戊戌,張齊賢為司空致仕。甲辰,詔樞密直學士限置六員。庚戌,淮南旱,減運河水灌民田,仍寬租限,州縣不能存恤致民流亡者罪之。己未,作五岳觀。



 九月辛未,張齊賢入對。壬申,觀新作延安橋。幸大相國寺、上清宮。射於宜春苑。癸酉,徙澄海三指揮屯嶺北州郡。戊子,王欽若、陳堯叟
 並為樞密使、同平章事,丁謂為戶部侍郎、參知政事。庚寅,幸故鄆王、兗王宮。



 冬十月戊午,延恩殿道場,帝瞻九天司命天尊降。己未,大赦天下,賜致仕官全奉。辛酉,作《崇儒術論》,刻石國學。



 閏月己巳,上聖祖尊號。辛未,謝太廟。壬申,立先天、降聖節,五日休沐、輟刑。乙亥,詔上聖祖母懿號,加太廟六室尊謚。丙子,群臣上尊號曰崇文廣武感天尊道應真祐德上聖欽明仁孝皇帝。丁丑,出舒州所獲瑞石,文曰「志公記」。戊寅,建景靈宮太極觀於壽
 丘。辛巳,建安軍鑄聖像。龍見雲中。戊子,禦制配享樂章並二舞名,文曰《發祥流慶》,武曰《降真觀德》。



 十一月丙申,親祀玉皇於朝元殿。甲辰,加王旦門下侍郎,向敏中中書侍郎,楚王元佐太師,相王元偓太傅,舒王元偁太保。內外官加恩。置玉清昭應宮使。以王旦為之。丁未,作《汴水發願文》。庚戌,詔允言朝參。乙卯,罷獻珍禽異獸。十二月甲子,置景福殿使。戊辰,作景靈宮。京師大寒,鬻官炭四十萬,減市直之半以濟貧民。壬申,改謚玄聖文宣王
 曰至聖文宣王。戊寅,溪峒張文喬等八百人來朝。己卯,知天雄軍寇準言獄空,詔獎之。乙酉,振泗州饑。丙戌,詔天慶等節日,民犯罪情輕者釋之。丁亥,立德妃劉氏為皇后。是歲,交州、甘州、西涼府、溪峒蠻來貢。京城、河北、淮南饑,減直鬻穀以濟流民。



 六年春正月癸巳朔,上御朝元殿受朝。司天監言五星同色。庚子,詔減配隸法十二條。戊申,禁內臣出使預民政。己酉,賜京師酺五日。辛亥,進封衛國、楚國、越國長公
 主三人為徐國、邠國、宿國。庚申,置淑儀、淑容、順儀、順容、婉儀、婉容,在昭儀上。置司宮令,在尚宮上。以婕妤楊氏為婉儀,貴人戴氏為修儀,美人曹氏為婕妤。辛酉,詔宗正寺以帝籍為玉牒。



 二月戊辰,觀酺。己亥,泰州言海陵草中生聖米,可濟饑。



 三月丁未,詔沙門島流人罪輕者徙近地。乙卯,建安軍鑄玉皇、聖祖、太祖、太宗尊像成,以丁謂為迎奉使。



 夏四月庚辰,詔淮南給饑民粥,麥登乃止。壬午,太白晝見。癸未,幸元偁宮視疾。丙戌,詔諸州死
 罪可疑者詳審以聞。



 五月壬辰,詔伎術官未升朝特賜緋紫者勿佩魚。甲辰,聖像至。丙午,詔聖像所經郡邑減系囚死罪,流以下釋之。升建安軍為真州。乙卯,謁聖像,奉安於玉清宮。丁巳,遣使奏告諸陵。



 六月壬戌,惟和卒。趙州黑龍見。丁卯,壽丘獻紫莖金芝。癸酉,保安軍雨,河溢,兵民溺死,遣使振之。丙子,詔翰林學士陳彭年等刪定《三司編敕》。丁丑,崇飾諸州黃帝祠廟。



 秋七月癸巳,上清宮道場獲龍於香合中。己亥,中書門下表請元德皇
 后祔廟。庚子,行配祔禮。癸卯,詔天下勿稅農器。己酉,亳州官吏父老三千三百人詣闕請謁太清宮。



 八月庚申,詔來春親謁亳州太清宮。辛酉,以丁謂為奉祀經度制置使。丙寅,禁太清宮五里內樵採。庚午,加號太上老君混元上德皇帝。置禮儀院。



 九月庚寅,幸元偁宮視疾。丁酉,出玉宸殿種占城稻示百官。



 冬十月辛酉,元德皇后祔廟。甲子,亳州太清宮枯檜再生。真源縣菽麥再實。癸酉,謁玉清昭應宮。己卯,作《步虛詞》付道門。壬午,降聖節
 賜會如先天節儀。



 十一月辛亥,幸元偁宮視疾。癸丑,賜御史臺《九經》、諸史。甲寅,判亳州丁謂獻芝草三萬七千本。乙卯,龜茲遣使來貢。十二月戊午朔,日有食之。庚申,涇原鈐轄曹瑋言發兵討原州界撥藏族違命者,捕獲甚眾。回鶻遣使來貢。己巳,天書扶侍使趙安仁等上奉天書車輅、鼓吹、儀仗。壬申,獻天書於朝元殿,遂告玉清昭應宮及太廟。乙亥,幸開寶寺、上清宮。己卯,幸太一宮。戎、瀘蠻寇平。是歲,西蕃、高州蠻、龜茲來貢。



 七年春正月辛丑,慮囚。壬寅,車駕奉天書發京師。丙午,次奉元宮。判亳州丁謂獻白鹿一,芝九萬五千本。戊申,王旦上混元上德皇帝冊寶。己酉,朝謁太清宮。天書升輅,雨雪倏霽,法駕繼進,佳氣彌望。是夜,月重輪,幸先天觀、廣靈洞霄宮。曲赦亳州及車駕所經流以下罪。升亳州為集慶軍節度,減歲賦十之二。改奉元宮為明道宮。太史言含譽星見。庚戌,御均慶樓,賜酺三日。壬子,詔所過頓遞侵民田者,給復二年。丙辰,建南京歸德殿,赦境
 內及京畿車駕所過流以下罪。追贈太祖幕府元勛僚舊,錄常參官逮事者並進秩,欲授子孫者聽。作鴻慶宮。



 二月戊午,次襄邑縣,皇子來朝。庚申,夏州趙德明遣使詣行闕朝貢。辛酉,至自亳州。丙寅,詔天地壇非執事輒臨者斬。辛未,饗太廟。壬申,恭謝天地,大赦天下。乙亥,益州鑄大鐵錢。



 三月,城淯井監。癸己,雄州甲仗庫火。甲午,制加宰相王旦、向敏中、楚王元佐、相王元偓、舒王元偁、榮王元儼樞密使、同平章事。乙未,宴翔鸞閣。辛丑,發粟
 振儀州饑。復諸州觀察使兼刺史。甲辰,幸元偁宮視疾。丁未,封皇子慶國公。青州民趙嵩百一十歲,詔存問之。



 夏四月丁巳,西涼府廝鐸督遣使來貢。己未,賜淮南諸州民租十之二。癸亥,河南府獄空,有鳩巢其戶,生二雛。甲子,以歸義軍留後曹賢順為歸義軍節度使。丙子,舒王元偁薨。



 五月壬辰,王旦為兗州景靈宮朝修使,乙未,又為天書刻玉使。涇原言葉施族大首領艷般率族歸順。



 六月乙卯,禁文字斥用黃帝名號故事。丙辰,眉州通
 判董榮受賕鬻獄,長安知縣王文龜酗酒濫刑,並投荒裔。戊午,戒州縣官吏決罪逾法。壬申,封婉儀楊氏為淑妃。乙亥,樞密使王欽若罷為吏部尚書,陳堯叟為戶部尚書。以寇準為樞密使、同平章事。丙子,詔棣州經水,流民歸業者給復三年。



 秋七月辛丑,交州李公蘊敗鶴柘蠻,獻捷。癸卯,太白晝見。甲辰,以同州觀察使王嗣宗、內客省使曹利用並為樞密副使。



 八月甲寅,置景靈宮使,以向敏中為之。乙卯,除江、淮、兩浙被災民租。丁巳,楊光
 習坐擅領兵出砦,又誣軍中謀殺司馬張從吉,配隸鄧州。乙丑,給河東沿邊將士皮裘氈襪。甲戌,河決澶州。丁丑,命內臣奉安太祖、太宗聖像於鴻慶宮。辛巳,詔嶺南戍兵代還日,人給裝錢五百。



 九月丙戌,含譽星再見。辛卯,尊上玉皇聖號曰太上開天執符御歷含真體道玉皇大天帝。戊戌,御試服勤詞學、經明行修舉人。辛丑,幸五岳觀。冬十一月乙酉,濱州河溢。玉清昭應宮成,詔減諸路系囚罪流以下一等。己丑,加王旦司空、修宮使。壬
 辰,禦乾元門觀酺。



 十二月癸丑朔,日當食不虧。丙辰,詔王欽若等五人各舉京朝、幕職、州縣官詳練刑典、曉時務、任邊寄者二人。丁巳,詔川、峽、閩、廣轉運、提點刑獄官察屬吏貪墨慘刻者。己未,作元符觀。庚申,契丹使蕭延寧等辭歸國。辛酉,加楚王元佐尚書令,相王元偓太尉,榮王元儼兼中書令,忠武軍節度使魏咸信同平章事,餘並進秩。涇原路請築籠竿城。是歲,夏州、西涼府、高麗、女真來貢。淮南、江、浙饑,除其租。天下戶九百五萬五千
 七百二十九,口二千一百九十七萬六千九百六十五。



 八年春正月壬午朔,謁玉清昭應宮,奉表告尊上玉皇大天帝聖號,奉安刻玉天書於寶符閣,還御崇德殿受賀,赦天下,非十惡、枉法贓及己殺人者咸除之。文武官滿三歲者,有司考課以聞。乙酉,詔環州緣邊卒人賜薪水錢。庚寅,置清衛二指揮奉宮觀。乙未,皇女入道。戊戌,徙棣州城。庚戌,詔王欽若等舉供奉官至殿直有武乾者一人。



 二月,泗州周憲百五歲,詔賜束帛。甲寅,宗正寺
 火。丙辰,唃廝囉、立遵貢名馬。丙寅,以元佐為天策上將軍、興元牧,賜劍履上殿,詔書不名。丁卯,遣使巡撫淮、浙路。癸酉,祈雨。丙子,詔進士六舉、諸科九舉者許奏名。庚辰,大雨。



 三月乙酉,幸元偓宮視疾。戊戌,宴宗室,射於苑中。壬寅,御試禮部貢舉人。



 夏四月辛酉,賜宰相《五臣論》。壬戌,以寇準為武勝軍節度使、同平章事,王欽若、陳堯叟並為樞密使、同平章事。戊辰,德彞卒。壬申,榮王元儼宮火,延及殿閣內庫。癸酉,詔求直言。命丁謂為大內修
 葺使。戊寅,王膺坐應詔言事乖繆貶。



 五月壬午,榮王元儼罷武信軍節度使,降封端王。庚寅,熒惑犯軒轅。壬辰,廢內侍省黃門。禁金飾服器。庚子,放宮人一百八十四人。



 六月己酉朔,日有食之。辛未,詔諸州以《御制七條》刻石。乙亥,惟忠卒。



 閏月己卯,赦天下。庚辰,王欽若上《彤管懿範》。



 七月丙辰,以諸州牛疫免牛稅一年。戊午,王嗣宗為大同軍節度使。丙寅,幸相王元偓新宮。以宮城火,詔諸王徙宮於外。丙子,幸瑞聖園觀稼,宴射於水心殿。



 八
 月己卯,大理少卿閻允恭、開封判官韓允坐枉獄除名。戊戌,詔京兆、河中府、陜、同、華、虢等州貸貧民麥種。



 九月,注輦國貢土物、珍珠衫帽。甲寅,唃廝囉聚眾數十萬,請討平夏人以自效。丁卯,宴宗室,射於後苑。己巳,賜注輦使袍服、牲酒。



 冬十月乙巳,王欽若上《聖祖先天紀》。戊申,回鶻呵羅等來貢。



 十一月辛酉,相王元偓加兼中書令,端王元儼進封彭王。癸亥,高麗使同東女真來貢。十二月戊寅,皇子冠。丁亥,侍禁楊承吉使西蕃還,以地理圖
 進。辛卯,皇子慶國公封壽春郡王。是歲,占城、宗哥族及西蕃首領來貢。坊州大雨,河溢。陜西饑。



 九年春正月丙辰,置會靈觀使,以丁謂為之,加刑部尚書。壬申,以張士遜、崔遵度為壽春郡王友。



 二月丁亥,王旦等上《兩朝國史》。戊子,加旦守司徒,修史官以下進秩、賜物有差。甲午,詔以皇子就學之所名資善堂。延州蕃部饑,貸以邊穀。



 三月丙午,除雷州無名商稅錢。秦州曹瑋撫捍蕃境得宜,詔嘉之。己酉,王欽若上《寶文統錄》。辛
 酉,以西蕃宗哥族李立遵為保順軍節度使。壬戌,詔舉官必擇廉能。癸亥,置修玉牒官。乙丑,著作佐郎高清以贓賄仗脊,配沙門島。



 夏四月庚辰,周伯星見。丙申,賜天下酺。振延州蕃族饑。庚子,幸陳堯叟第視疾。壬寅,以唐相元稹七世孫為臺州司馬。



 五月乙巳,邠寧環慶部署王守斌言夏州蕃騎千五百來寇慶州,內屬蕃部擊走之。癸丑,幸南宮視惟憲疾。甲寅,惟憲卒。乙卯,毛尸等三族蕃官馮移埋率屬來歸,詔撫之。丙辰,詔天下系囚死
 罪減等,流以下釋之。丁巳,向敏中為宮觀慶成使。甲子,左天廄草場火。庚午,太白晝見。



 六月戊寅,幸會靈觀,宴祝禧殿,癸未,京畿蝗。



 秋七月,撫水蠻寇宜州,廣南西路請便宜掩擊,許之。丁未,增築京師新城。丙辰,開封府祥符縣蝗附草死者數里。戊午,停京城工役。癸亥,以畿內蝗,下詔戒郡縣。甲子,詔京城禁樂一月。丁卯,幸太乙宮、天清寺。



 八月壬申,知秦州曹瑋言伏羌砦蕃部廝雞波與宗哥族連結為亂,以兵夷其族帳。丙子,令江、淮發運
 司留上供米五十萬以備饑年。磁、華、瀛、博等州蝗,不為災。丙戌,制玉皇聖號冊文。以陳堯叟為右僕射。戊子,以旱,罷秋宴。壬辰,群臣請受尊號冊寶,表五上,從之。



 九月癸卯,雄、霸河溢。甲辰,以丁謂為平江軍節度使。丙午,陳彭年、王曾、張知白並參知政事。丁未,曹瑋言宗哥唃廝囉、蕃部馬波叱臘、魚角蟬等寇伏羌砦,擊敗之,斬首千餘級。庚戌,以不雨,罷重陽宴。利州水,漂棧閣。甲寅,雨。督諸路捕蝗。丁巳,詔以旱蝗得雨,宜務稼省事及罷諸營
 造。戊午,禁諸路貢瑞物。戊辰,青州飛蝗赴海死,積海岸百餘里。己巳,詔民有出私廩振貧乏者,三千石至八千石,第授助教、文學、上佐之秩。



 冬十月己卯,王欽若表上《翊聖保德真君傳》。壬申,詔馮拯等各舉殿直以上武乾者一人。壬辰,置直龍圖閣。



 十一月,會靈觀甘露降。乙巳,詔河、陜諸路州簡禁軍五百人。丁未,河西節度使石普坐妄言災異,除名流賀州。丁卯,以唐裴度孫坦為鄭州助教。是歲,西蕃宗哥族、邛部山後蠻、夏州、甘州來貢。諸
 州有隕霜害稼及水災者,遣使振恤,除其租。



 天禧元年春正月辛丑朔,改元。詣玉清昭應宮薦獻,上玉皇大天帝寶冊、袞服。壬寅,上聖祖寶冊。己酉,上太廟謚冊。庚戌,享六室。辛亥,謝天地於南郊,大赦,御天安殿受冊號。乙卯,宰相讀天書於天安殿,遂幸玉清昭應宮,作《欽承寶訓述》示群臣。壬戌,詔以四月旦日為天祥節。丙寅,命王旦為兗州太極觀奉上冊寶使。



 二月庚午,詔振災,發州郡常平倉。壬申,御正陽門觀酺。丁丑,置諫官、
 御史各六員,每月一員奏事,有急務聽非時入對。戊寅,王旦加太保、中書侍郎、平章事,向敏中加吏部尚書。楚王元佐領雍州牧;相王元偓加尚書令兼中書令,進封徐王;彭王元儼加太保;壽春郡王禎兼中書令。王欽若加右僕射,趙德明加太傅,中外官並加恩。辛巳,考課京朝官改秩及考者。壬午,定宗室子授官之制。庚寅,進封李公蘊為南平王。秦州神武軍破宗歌族、馬波叱臘等於野吳谷,多獲人馬。己亥,陳彭年卒。



 三月辛丑,以不
 雨,禱於四海。壬寅,不雨,罷上巳宴。庚申,免潮州逋鹽三百七十萬有奇。辛酉,令作淖糜濟懷、衛流民。



 夏四月庚辰,陳堯叟卒。戊子,邵州野竹生實,以食饑。



 五月戊戌,詔所在安恤流民。戊申,以王旦為太尉、侍中,五日一入中書,旦懇辭不拜。己酉,熒惑犯太微。乙卯,縱歲獻鷹犬。己未,奉太祖聖容於西京應天院,向敏中為禮儀使。諸路蝗食苗,詔遣內臣分捕,仍命使安撫。



 六月壬申,赦西京系囚,死罪減一等,流以下釋之。父老年八十者賜茶帛,
 除其課役。戊寅,除升州後湖租錢五十餘萬,聽民溉田。陜西、江、淮南蝗,並言自死。庚辰,盜發後漢高祖陵,論如律,並劾守土官吏,遣內侍王克讓以禮治葬,知制誥劉筠祭告。因詔州縣申前代帝王陵寢樵採之禁。乙酉,免大食國蕃客稅之半。龜茲國使張復延等貢玉勒鞍馬,令給其直。己丑,王旦對於崇政殿。



 秋七月丁未,霖雨,放朝。己未,幸魏咸信第視疾。甲子,魏咸信卒。



 八月庚午,以王欽若為左僕射兼中書侍郎、平章事。壬申,向敏中加
 右僕射兼門下侍郎。王旦對於便殿。丙子,詔京城禁圍草地聽民耕牧。丁丑,禁採狨。戊寅,免牛稅一年。



 九月癸卯,以參知政事王曾為禮部侍郎,李迪為參知政事,馬知節知樞密院事,曹利用、任中正、周起並同知樞密院事。丙午,幸王旦第視疾。戊申,以蝗,罷秋宴。己酉,王旦薨。甲寅,詔能拯救汴渠覆溺者給賞,或溺者貧者,以官錢給之。丁未,教衛士騎射。



 冬十月辛未,詔閣門自今審官、三班院、流內銓,後殿日引公事,勿過兩司。壬申,諭諸州
 非時災沴不以聞者論罪。己卯,罷京東上貢物。辛卯,賜壽春郡王及王友張士遜等詩。



 十一月己亥,詔曲宴日輟後殿視事。辛丑,曹瑋平鬼留家族。壬寅,詔淮、浙、荊湖治放生池,禁漁採。乙卯,幸太一宮,大雪,帝謂宰相曰:「雪固豐稔之兆,第民力未充,慮失播種。卿等其務振勸,毋遺地利。」壬戌,契丹使耶律準來賀承天節。高麗使徐訥率女真首領入對崇政殿,獻方物。十二月丙寅,京城雪寒,給貧民粥,並瘞死者。乙亥,罷京城工役。丙子,嚴寒,放
 朝。丁丑,放逋負,釋系囚。己卯,女真國人歸,給裝錢。高麗使徐訥賜射瑞聖園。辛卯,詔陜西緣邊鬻穀者勿算。壬辰,遣使緣汴河收瘞流尸。是歲,三佛齊、龜茲國來貢。諸路蝗,民饑。鎮戎軍風雹害稼,詔發廩振之,蠲租賦,貸其種糧。



 二年春正月乙未,真游殿芝草生。壬寅,振河北、京東饑。辛亥,賜壽春郡王《恤民歌》。戊午,王欽若等上《天禧大禮記》四十卷。己未,遣使諭京東官吏安撫饑民,又命諸路
 振以淖糜。



 二月丙寅,甘州來貢。丁卯,壽春郡王加太保,進封升王。詔近臣舉常參官堪任御史者。庚午,右正言劉燁請自今言事許升殿,從之。庚辰,振京西饑。乙酉,幸徐王元偓宮視疾。



 三月辛丑,修京城。丙辰,先貸貧民糧種止勿收。



 夏四月戊子,幸飛山雄武教場,宴賜從臣將士。庚寅,赦天下,死罪減一等,流以下釋之。



 閏月,辰州討下溪州蠻,斬首六十餘級,降千餘人。己亥,詔戶部尚書馮拯等舉幕職、令錄堪充京官者各二人。癸卯,馬知節
 為彰德軍留後。丁未,靈泉出京師,飲者愈疾。作祥源觀。壬子,幸徐王元偓宮視疾。



 五月壬戌,詔長吏恤孝弟力田者。甲子,徐王元偓薨。丁卯,釋下溪州蠻彭儒猛罪。丙戌,西京訛言妖如帽,夜蜚,民甚恐。



 六月壬辰,詔三班使臣經七年者考課遷秩。己亥,詔諸州上佐、文學、參軍謫降十年者,聽還鄉。乙巳,訛言帽妖至京師,民夜叫噪達曙,詔捕嘗為邪法人耿概等棄市。辛亥,彗出北斗魁。



 秋七月壬申,以星變赦天下,流以下罪減等,左降官羈管
 十年以上者放還京師,京朝官丁憂七年未改秩者以聞。丁亥,彗沒。



 八月庚寅,群臣請立皇太子,從之。壬寅,下溪州彭儒猛納所掠漢生口、器甲等,詔賜袍帶。甲辰,立皇子升王為皇太子。大赦天下,宗室加恩,群臣賜勛一轉。戊申,黎州山後兩林百蠻都王李阿善遣使來貢。壬子,彭王元儼進封通王。以李迪兼太子賓客。癸丑,作《元良箴》賜皇太子。甲寅,楚王元佐加興元牧,徐國長公主進封福國,邠國長公主進封建國,宿國長公主進封鄂
 國。乙卯,詔畎索河水入金水河。丙辰,以德雍、德文、惟政並為諸州防禦使,允成、允升、允寧並為諸州團練使。



 九月丁卯,冊皇太子。庚午,詔全給外戍諸軍物。庚辰,禦乾元門觀酺。



 冬十月庚子,御玉宸殿,召近臣觀刈占城稻,遂宴安福殿。



 十二月辛丑,以張旻為武寧軍節度使、同平章事。是歲,占城國、甘州、溪峒、黎州山後蠻來貢。陜西旱,振之。江陰軍蝻,不為災。



 三年春正月癸亥,貢舉人郭□貞等見崇政殿。□貞冒喪赴
 舉,命典謁詰之,即引咎,殿三舉。



 二月乙未,河南府地震。



 三月戊午朔,日有食之。遣呂夷簡體訪陜、亳民訛言。丙寅,御試禮部貢舉人。癸未,翰林學士、工部尚書錢惟演等坐知舉失實,降一官。甲申,穎州石隕出泉,飲之愈疾。



 夏四月甲午,西上閣門使高繼勛坐市馬虧直削官。



 五月丁巳,大食國來貢。乙丑,左諫議大夫戚綸坐訕上,貶岳州副使。辛未,慮囚。



 六月癸未,浚淮南漕渠,廢三堰。甲午,王欽若為太子太保。河決滑州。戊戌,以寇準為中書
 侍郎兼吏部尚書、平章事,丁謂為吏部尚書、參知政事。滑州決河,泛澶、濮、鄆、齊、徐境,遣使救被溺者,恤其家。



 秋七月壬申,曹璨卒。群臣表上尊號曰體元禦極感天尊道應真寶運文德武功上聖欽明仁孝皇帝。



 八月丁亥,大赦天下。普度道釋童行。滑州龍見,河決。辛卯,太白晝見。己亥,慶州亡去熟戶委乞等來歸。庚戌,遣使撫恤京東西、河北水災。



 九月乙丑,慶州骨咩、大門等族歸附。辛巳,遣中官存問高麗貢使之被溺者。冬十一月己巳,謁
 景靈宮。庚午,饗太廟。辛未,祀天地於圜丘,大赦天下。選兩任五考無責罰者試身、言、書、判。丁丑,御天安殿受尊號冊。



 十二月丙戌,富州蠻酋向光澤表納土,詔卻之。辛卯,向敏中加左僕射、中書侍郎兼禮部尚書、平章事,寇準加右僕射,通王元儼進封涇王,曹利用、丁謂並為樞密使,百官加恩。癸巳,以任中正、周起並為樞密副使。是歲,高麗、女真來貢。江、浙及利州路饑,詔振之。



 四年春正月乙丑,以華州觀察使曹瑋為鎮國軍留後、
 僉樞密院事。丙寅,開揚州運河。己巳,幸元符觀。庚午,贈處士魏野著作郎,賜其家粟帛。



 二月,帝不豫。癸未,遣使安撫淮南、江、浙、利州饑民。滑州決河塞。辛丑,發唐、鄧八州常平倉振貧民。



 三月戊午,以淄州民饑,貸牛糧。甲子,振蕃部粟。庚午,詔川峽致仕官聽還本貫。癸酉,川、廣舉人勿拘定額。己亥,振益、梓民饑。己卯,向敏中薨。



 夏四月丁亥,大風,晝晦。庚寅,分江南轉運使為東、西路。丙申,杖殺前定陶縣尉麻士瑤於青州。



 五月丁巳,發粟振秦、隴。



 六月丙申,以寇準為太子太傅、萊國公。河決滑州。壬寅,御試禮部奏名舉人九十三人。



 秋七月丁巳,太白晝見。辛酉,京城大雨,水壞廬舍大半。丙寅,以李迪為吏部侍郎兼太子少傅、平章事,馮拯為樞密使、吏部尚書、同平章事。以霖雨壞營舍,賜諸軍緡錢。庚午,以丁謂為平章事,曹利用同平章事。癸酉,入內副都知周懷政伏誅。丁丑,太子太傅寇準降授太常卿,翰林學士盛度、樞密直學士王曙並罷職。



 八月,永興軍都巡檢使朱能殺中使
 叛。乙酉,以任中正、王曾並參知政事。詔利、夔路置常平倉。丙戌,朱能自殺。壬寅,寇準貶道州司馬。甲辰,賜諸軍器幣。入內押班鄭志誠坐交朱能,削兩任、配隸房州。



 九月己酉,分遣近臣張知白、晁迥、樂黃目等各舉常參官,諸路轉運及勸農使各舉堪京官、知縣者二人,知制誥、知雜御史、直龍圖閣各舉堪御史者一人。丙辰,始御崇德殿視事,治朱能黨,死、流者數十人。己未,久雨,放朝。壬戌,給事中朱巽、工部郎中梅詢坐不察朱能奸謫官。丁
 卯,赦天下。己巳,遣使安撫永興軍。壬申,賜京城酺。



 冬十月戊寅,命依唐制,雙日不視事。壬午,幸正陽門觀酺。帝自不豫,浸少臨行,至是人情大悅。壬辰,以王欽若為資政殿大學士。甲辰,減水災州縣秋租。丙午,召皇子、宗室、近臣玉宸殿觀稻,賜宴。



 十一月戊午,召近臣於龍圖閣觀禦制文詞,帝曰:「朕聽覽之暇,以翰墨自娛,雖不足垂範,亦平生游心於此。」宰臣丁謂請鏤板宣布。庚申,內出禦制七百二十二卷付宰臣。丙寅,丁謂加門下侍郎兼
 太子太傅,李迪加中書侍郎兼尚書左丞,依前少傅。迪、謂忿爭於帝前。戊辰,罷謂為戶部尚書,迪為戶部侍郎。任中正、王曾、錢惟演並兼太子賓客,張士遜、林特並兼太子詹事,晏殊為太子左庶子。己巳,詔謂赴中書視事如故。庚午,詔自今除軍國大事仍舊親決,餘皆委皇太子同宰相、樞密使等參議行之。太子上表陳讓,不允。以丁謂兼太子少師,馮拯兼少傅,曹利用兼少保。辛未,詔自今群臣五日於長春殿起居,餘只日視朝於承明殿。
 甲戌,丁謂等請作天章閣奉安御集。十二月乙酉,皇太子親政,詔內臣傳旨須覆奏。丁亥,龜茲、甘州回鶻遣使來貢。己丑,王欽若加司空。庚寅,議事資善堂,命張景宗侍皇太子。丁酉,以王欽若為山南東道節度使、同平章事。



 閏月丁卯,以唃廝囉為邊患,詔陳堯咨等巡撫。庚午,京城穀貴,減直發常平倉。乙亥,帝不豫,力疾御承明殿,賜手書宰相,諭以輔導儲貳之意。是歲,京西、陜西、江、淮、荊湖諸州稔。



 五年春正月己丑,帝疾愈,出幸啟聖院。癸巳,詔天下死罪降,流以下釋之。乙未,遣使撫京東水災。丁酉,以張士遜為樞密副使。己亥,宴近臣承明殿。



 二月甲寅,審刑院言天下無斷獄。丙寅,賜天下酺。庚午,以孔子四十七世孫聖祐襲封文宣公。



 三月辛巳,御正陽門觀酺。辛丑,京東、西水災,賜民租十之五。壬寅,丁謂加司空,馮拯加左僕射,曹利用加右僕射,任中正工部尚書。



 夏四月丙辰,客星出軒轅。



 五月乙亥,慮囚,降天下死罪。



 六月丙午,太
 白晝見。



 秋七月甲戌朔,日有食之。戊寅,新作景靈宮萬壽殿。



 八月壬戌。熒惑犯南斗。



 九月戊寅,唃廝囉請降。



 冬十月癸卯,蠲京東西、淮、浙被災民租。壬子,依漢、唐故事,五日一受朝,遇慶會,皇太子押班。



 十一月戊子,王欽若以山南東道節度使坐擅赴闕,降司農卿、分司南京。是歲,高麗遣使來貢。京東、河北、兩川、荊湖稔。



 乾興元年春正月辛未朔,改元。丁亥,御東華門觀燈。戊戌,蠲秀州水災民租。



 二月庚子,大赦天下。癸卯,上尊號
 曰應天尊道欽明仁孝皇帝。詔蘇、湖、秀州民饑,貸以廩粟。甲辰,制封丁謂為晉國公,馮拯為魏國公,曹利用為韓國公。庚戌,詔徐州振貧民。甲寅,對宰相於寢殿。帝不豫增劇,禱於山川神祇。戊午,帝大漸,遺詔皇太子於柩前即皇帝位。尊皇后為皇太后,權處分軍國事,淑妃為皇太妃。帝是日崩於延慶殿,年五十五,在位二十六年。十月己酉,葬永定陵。己未,祔太廟。天聖二年十一月,上尊謚曰文明武定章聖元孝皇帝,廟號真宗。慶歷七年,
 加謚膺符稽古神功讓德文明武定章聖元孝皇帝。



 贊曰:真宗英悟之主。其初踐位,相臣李沆慮其聰明,必多作為,數奏災異以杜其侈心,蓋有所見也。及澶洲既盟,封禪事作,祥瑞沓臻,天書屢降,導迎奠安,一國君臣如病狂然,籲,可怪也。他日修《遼史》,見契丹故俗而後推求宋史之微言焉。宋自太宗幽州之敗,惡言兵矣。契丹其主稱天,其後稱地,一歲祭天不知其幾,獵而手接飛雁,鴇自投地,皆稱為天賜,祭告而誇耀之。意者宋之諸
 臣,因知契丹之習,又見其君有厭兵之意,遂進神道設教之言,欲假是以動敵人之聽聞,庶幾足以潛消其窺覦之志歟?然不思修本以制敵,又效尤焉,計亦末矣。仁宗以天書殉葬山陵,嗚呼賢哉!



\end{pinyinscope}