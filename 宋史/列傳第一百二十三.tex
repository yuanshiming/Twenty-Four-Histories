\article{列傳第一百二十三}

\begin{pinyinscope}

 韓世忠子彥直



 韓世忠,字良臣,延安人。風骨偉岸,目瞬如電。早年鷙勇絕人,能騎生馬駒。家貧無產業,嗜酒尚氣,不可繩檢。日者言當作三公,世忠怒其侮己,毆之。年十八,以敢勇應
 募鄉州,隸赤籍,挽強馳射,勇冠三軍。



 崇寧四年,西夏騷動,郡調兵捍禦,世忠在遣中。至銀州,夏人嬰城自固,世忠斬關殺敵將,擲首陴外,諸軍乘之,夏人大敗。既而以重兵次蒿平嶺,世忠率精銳鏖戰,解去。俄復出間道,世忠獨部敢死士珠死鬥,敵少卻,顧一騎士銳甚,問俘者,曰:「監軍駙馬兀□移也。」躍馬斬之,敵眾大潰。經略司上其功,童貫董邊事,疑有所增飾,止補一資,眾弗平。從劉延慶築天降山砦,為敵所據,世忠夜登城斬二級,割護城
 氈以獻。繼遇敵佛口砦,又斬數級,始補進義副尉。至藏底河,斬三級,轉進勇副尉。



 宣和二年,方臘反、江、浙震動,調兵四方,世忠以偏將從王淵討之。次杭州,賊奄至,勢張甚,大將惶怖無策。世忠以兵二千伏北關堰,賊過,伏發,眾蹂亂,世忠追擊,賊敗而遁。淵嘆曰:「真萬人敵也。」盡以所隨白金器賞之,且與定交。時有詔能得臘首者,授兩鎮節鉞。世忠窮追至睦州清溪峒,賊深據巖屋為三窟,諸將繼至,莫知所入。世忠潛行溪谷,問野婦得徑,即
 挺身仗戈直前,渡險數里,搗其穴,格殺數十人,禽臘以出。辛興宗領兵截峒口,掠其俘為己功,故賞不及世忠。別帥楊惟忠還闕,直其事,轉承節郎。



 三年,議復燕山,調諸軍,至則皆潰。世忠往見劉延慶,與蘇格等五十騎俱抵滹沱河。逢金兵二千餘騎,格失措,世忠從容令格等列高岡,戒勿動。屬燕山潰卒舟集,即命艤河岸,約鼓噪助聲勢。世忠躍馬薄敵,回旋如飛。敵分二隊據高阜,世忠出其不意,突二執旗者,因奮擊,格等夾攻之,舟卒鼓
 噪,敵大亂,追斬甚眾。時山東、河北盜賊蜂起,世忠從王淵、梁方平討捕,禽戮殆盡,積功轉武節郎。



 欽宗即位,從梁方平屯浚州。金人壓境,方平備不嚴,金人迫而遁,王師數萬皆潰。世忠陷重圍中,揮戈力戰,突圍出,焚橋而還。欽宗聞,召對便殿,詢方平失律狀,條奏甚悉。轉武節大夫。詔諸路勤王兵領所部入衛,會金人退,河北總管司闢選鋒軍統制。



 時勝捷軍張師正敗,宣撫副使李彌大斬之,大校李復鼓眾以亂,淄、青之附者合數萬人,山
 東復擾。彌大檄世忠將所部追擊,至臨淄河,兵不滿千,分為四隊,布鐵蒺藜自塞歸路,令曰:「進則勝,退則死,走者命後隊剿殺。」於是莫敢返顧,皆死戰,大破之,斬復,餘黨奔潰。乘勝逐北,追至宿遷,賊尚萬人,方擁子女椎牛縱酒。世忠單騎夜造其營,呼曰:「大軍至矣,亟束戈卷甲,吾能保全汝,共功名。」賊駭粟請命,因跪進牛酒。世忠下馬解鞍,飲啖之盡,於是眾悉就降。黎明,見世忠軍未至,始大悔失色。以功遷左武大夫、果州團練使。



 詔入朝,授
 正任單州團練使,屯滹沱河。時真定失守,世忠知王淵守趙,遂亟往。金人至,聞世忠在,攻益急,糧盡援絕。人多勉其潰圍去,弗聽。會大雪,夜半,以死士三百搗敵營。敵驚亂,自相擊刺,及旦盡遁。後有自金國來者,始知大酋是日被創死,故眾不能支。遷嘉州防禦使。



 還大名,趙野闢為前軍統制。時康王如濟州,世忠領所部勸進。金人縱兵逼城,人心忷懼,世忠據西王臺力戰,金人少卻。翌日,酋帥率眾數萬至,時世忠戲下僅千人,單騎突入,斬
 其酋長,遂大潰。康王即皇帝位,授光州觀察使、帶御器械。世忠請移都長安,下兵收兩河,時論不從。初建御營,為左軍統制。是歲,命王淵、張俊討陳州叛兵,劉光世討黎驛叛兵,喬仲福討京東賊李昱,世忠討單州賊魚臺。世忠已破魚臺,又擊黎驛叛兵,敗之,皆斬以獻。於是群盜悉平,入備宿衛。而河北賊丁順、楊進等皆赴招撫司,宗澤收而用之。



 建炎二年,升定國軍承宣使。帝如揚州,世忠以所部從。時張遇自金山來降,抵城下,不解甲,人
 心危懼,世忠獨入其壘,曉以逆順,眾悉聽命。李民眾十萬亦降,比至,有反復狀。王淵遣世忠諭旨,世忠知其黨劉彥異議,即先斬彥,驅李民出,縛小校二十九人,送淵斬之。事定,授京西等路捉殺內外盜賊。



 金人再攻河南,翟進合世忠兵夜襲悟室營,不克,反為所敗。會丁進失期,陳思恭先遁,世忠被矢如棘,力戰得免。還汴,詰一軍之先退者皆斬,左右懼。進由是與世忠有隙,尋以叛誅。召世忠還,授鄜延路副總管,加平寇左將軍,屯淮陽,會
 山東兵拒敵。粘罕聞世忠扼淮陽,乃分兵萬人趨揚州,自以大軍迎世忠戰。世忠不敵,夜引歸,敵躡之,軍潰於沐陽,合門宣贊舍人張遇死之。



 三年,帝召諸將議移蹕,張俊、辛企宗請往湖南,世忠曰:「淮、浙富饒,今根本地,詎可舍而之他?人心懷疑,一有退避,則不逞者思亂,重湖、閩嶺之遙,安保道路無變乎?淮、江當留兵為守,車駕當分兵為衛,約十萬人,分半扈江、淮上下,止餘五萬,可保防守無患乎?」在陽城收合散亡,得數千人,聞帝如錢塘,
 即繇海道赴行在。



 苗傅、劉正彥反,張浚等在平江議討亂,知世忠至,更相慶慰,張俊喜躍不自持。世忠得俊書,大慟,舉酒酹神曰:「誓不與此賊共戴天!」士卒皆奮。見浚曰:「今日大事,世忠願與張俊身任之,公無憂。」欲即進兵。浚曰:「投鼠忌器,事不可急,急則恐有不測,已遣馮轓甘言誘賊矣。」



 三月戊戌,以所部發平江。張俊慮世忠兵少,以劉寶兵二千借之。舟行載甲士,綿互三十里。至秀州,稱病不行,造雲梯,治器械,傅等始懼。初,傅、正彥聞世忠
 來,檄以其兵屯江陰。世忠以好語報之,且言所部殘零,欲赴行在。傅等大喜,許之,至矯制除世忠及張俊為節度使,皆不受。時世忠妻梁氏及子亮為傅所質,防守嚴密。朱勝非紿傅曰:「今白太后,遣二人慰撫世忠,則平江諸人益安矣。」於是召梁氏入,封安國夫人,俾迓世忠,速其勤王。梁氏疾驅出城,一日夜會世忠於秀州。未幾,明受詔至,世忠曰:「吾知有建炎,不知有明受。」斬其使,取詔焚之,進兵益急。傅等大懼。次臨平,賊將苗翊、馬柔吉負
 山阻河為陣,中流植鹿角,梗行舟。世忠舍舟力戰,張俊繼之,劉光世又繼之。軍少卻,世忠復舍馬操戈而前,令將士曰:「今日當以死報國,面不被數矢者皆斬。」於是士皆用命。賊列神臂弩持滿以待,世忠瞋目大呼,挺刃突前,賊闢易,矢不及發,遂敗。傅、正彥擁精兵二千,開湧金門以遁。世忠馳入,帝步至宮門,握世忠手慟哭曰:「中軍吳湛佐逆為最,尚留朕肘腋,能先誅乎?」世忠即謁湛,握手與語,折其中指,戮於市,又執賊謀主王世修以屬吏。
 詔授武勝軍節度使御營左軍都統制。請於帝曰:「賊擁精兵,距甌、閩甚邇,儻成巢窟,卒未可滅,臣請討之。」於是以為江、浙制置使,自衢、信追擊,至漁梁驛,與賊遇。世忠步走挺戈而前,賊望見,咋曰:「此韓將軍也!」皆驚潰。擒正彥及傅弟翊送行在,傅亡建陽,追禽之,皆伏誅。世忠初陛辭,奏曰:「臣誓生獲賊,為社稷刷恥,乞殿前二虎賁護俘來獻。」至是,卒如其言。帝手書「忠勇」二字,揭旗以賜。授檢校少保、武勝昭慶軍節度使。



 兀朮將入侵,帝召諸將
 問移蹕之地,張俊、辛企宗勸自鄂、岳幸長沙,世忠曰:「國家已失河北,山東,若又棄江、淮,更有何地?」於是以世忠為浙西制置使,守鎮江。既而兀朮分道渡江,諸屯皆敗,世忠亦自鎮江退保江陰。杜充以建康降敵,兀朮自廣德破臨安,帝如浙東。世忠以前軍駐青龍鎮,中軍駐江灣,後軍駐海口,俟敵歸邀擊之。帝召至行在,奏:「方留江上截金人歸師,盡死一戰。」帝謂輔臣曰:「此呂頤浩在會稽,嘗建此策,世忠不謀而同。」賜親札,聽其留。會上元節,
 就秀州張燈高會,忽引兵趨鎮江。及金兵至,則世忠軍已先屯焦山寺。金將李選降,受之。兀朮遣使通問,約日大戰,許之。戰將十合,梁夫人親執桴鼓,金兵終不得渡。盡歸所掠假道,不聽;請以名馬獻,又不聽。撻辣在濰州,遣孛堇太一趨淮東以援兀朮,世忠與二酋相持黃天蕩者四十八日。太一孛堇軍江北,兀朮軍江南,世忠以海艦進泊金山下,預以鐵綆貫大鉤授驍健者。明旦,敵舟噪而前,世忠分海舟為兩道出其背,每縋一綆,則曳
 一舟沉之。兀朮窮蹙,求會語,祈請甚哀。世忠曰:「還我兩宮,復我疆土,則可以相全。」兀朮語塞。又數日求再會,言不遜,世忠引弓欲射之,亟弛去,謂諸將曰:「南軍使船欲如使馬,奈何?」募人獻破海舟策。閩人王某者,教其舟中載土,平版鋪之,穴船版以棹槳,風息則出江,有風則勿出。海舟無風,不可動也。又有獻謀者曰:「鑿大渠接江口,則在世忠上流。」兀朮一夕潛鑿渠三十里,且用方士計,刑白馬,剔婦人心,自割其額祭天。次日風止,我軍帆弱不
 能運,金人以小舟縱火,矢下如雨。孫世詢、嚴允皆戰死,敵得絕江遁去。世忠收餘軍還鎮江。初,世忠謂敵至必登金山廟,觀我虛實。乃遣兵百人伏廟中,百人伏岸滸,約聞鼓聲,岸兵先入,廟兵合擊之。金人果五騎闖入,廟兵喜,先鼓而出,僅得二人。逸其三,中有絳袍玉帶、既墜而復馳者,詰之,乃兀朮也。是役也,兀朮兵號十萬,世忠僅八千餘人。帝凡六賜札,褒獎甚寵。拜檢校少保、武成感德軍節度使,神武左軍都統制。



 建安範汝為反,辛企
 宗等討捕未克,賊勢愈熾。以世忠為福建、江西、荊湖宣撫副使,世忠曰:「建居閩嶺上流,賊沿流而下,七郡皆血肉矣。」亟領步卒三萬,水陸並進。次劍潭,賊焚橋,世忠策馬先渡,師遂濟。賊盡塞要路拒王師,世忠命諸軍偃旗僕鼓,徑抵鳳凰山,俯瞰城邑,設雲梯火樓,連日夜並攻,賊震怖叵測。五日城破,汝為竄身自焚,斬其弟岳、吉以徇,禽其謀主謝向、施逵及裨將陸必強等五百餘人。世忠初欲盡誅建民,李綱自福州馳見世忠曰:「建民多無
 辜。」世忠令軍士馳城上毋下,聽民自相別,農給牛谷,商賈馳征禁,脅從者汰遣,獨取附賊者誅之。民感更生,家為立祠。捷聞,帝曰:「雖古名將何以加。」賜黃金器皿。



 世忠因奏江西、湖南寇賊尚多,乞乘勝討平。廣西賊曹成擁餘眾在郴、邵。世忠既平閩寇,旋師永嘉,若將就休息者。忽由處、信徑至豫章,連營江濱數十里,群賊不虞其至,大驚。世忠遣人招之,成以其眾降,得戰士八萬,遣詣行在。遂移師長沙。時劉忠有眾數萬,據白面山,營柵相望。
 世忠始至,欲急擊,宣撫使孟庾不可,世忠曰:「兵家利害,策之審矣,非參政所知,請期半月效捷。」遂與賊對壘,弈棋張飲,堅壁不動,眾莫測。一夕,與蘇格聯騎穿賊營,候者呵問,世忠先得賊軍號,隨聲應之,周覽以出,喜曰:「此天錫也。」夜伏精兵二千於白面山,與諸將拔營而進,賊兵方迎戰,所遣兵已馳入中軍,奪望樓,植旗蓋,傳呼如雷,賊回顧驚潰,麾將士夾擊,大破之,斬忠首,湖南遂平。授太尉,賜帶、笏,仍敕樞密以功頒示內外諸將。師還建
 康,置背嵬軍,皆勇鷙絕倫者。九月,為江南東、西路宣撫使,置司建康。



 三年三月,進開府儀同三司,充淮南東、西路宣撫使,置司泗州。時聞李橫進師討偽齊,議遣大將,以世忠忠勇,故遣之。仍賜廣馬七綱,甲千副,銀二萬兩,帛二萬匹;又出錢百萬緡,米二十八萬斛,為半歲之用。命戶部侍郎姚舜明詣泗州,總領錢糧;倉部郎官孫逸如平江府、常秀饒州,督發軍食。李橫兵敗還鎮,世忠不果渡淮。



 四年,以建康、鎮江、淮東宣撫使駐鎮江。是歲,金
 人與劉豫合兵,分道入侵。帝手札命世忠飭守備,圖進取,辭旨懇切。世忠受詔,感泣曰:「主憂如此,臣子何以生為!」遂自鎮江濟師,俾統制解元守高郵,候金步卒;親提騎兵駐大儀,當敵騎,伐木為柵,自斷歸路。會遣魏良臣使金,世忠撤炊爨,紿良臣有詔移屯守江,良臣疾馳去。世忠度良臣已出境,即上馬令軍中曰:「視吾鞭所向。」於是引軍次大儀,勒五陣,設伏二十餘所,約聞鼓即起擊。良臣至金軍中,金人問王師動息,具以所見對。聶兒孛
 堇聞世忠退,喜甚,引兵至江口,距大儀五里;別將撻孛也擁鐵騎過五陣東。世忠傳小麾鳴鼓,伏兵四起,旗色與金人旗雜出,金軍亂,我軍迭進。背嵬軍各持長斧,上揕人胸,下斫馬足。敵被甲陷泥淖,世忠麾勁騎四面蹂躪,人馬俱斃,遂擒撻孛也等二百餘人。所遣董旼亦擊金人於天長縣之鴉口,擒女真四十餘人。解元至高郵,遇敵,設水軍夾河陣,日合戰十三,相拒未決。世忠遣成閔將騎士往援,復大戰,俘生女真及千戶等。世忠復親
 追至淮,金人驚潰,相蹈藉,溺死甚眾。捷聞,群臣入賀,帝曰:「世忠忠勇,朕知其必能成功。」沉與求曰:「自建炎以來,將士未嘗與金人迎敵一戰,今世忠連捷以挫其鋒,厥功不細。」帝曰:「第憂賞之。」於是部將董旼、陳桷、解元、呼延通等皆峻擢有差。論者以此舉為中興武功第一。



 時撻辣屯泗州,兀朮屯竹塾鎮,為世忠所扼,以書幣約戰,世忠許之,且使兩伶人以橘、茗報聘。會雨雪,金饋道不通,野無所掠,殺馬而食,蕃漢軍皆怨。兀術夜引軍還,劉麟、
 劉猊棄輜重遁。



 五年,進少保。六年,授武寧安化軍節度使、京東淮東路宣撫處置使,置司楚州。世忠披草萊,立軍府,與士同力役。夫人梁親織薄為屋。將士有怯戰者,世忠遺以巾幗,設樂大宴,俾婦人妝以恥之,故人人奮厲。撫集流散,通商惠工,山陽遂為重鎮。劉豫兵數入寇,輒為世忠所敗。



 時張浚以右相視師,命世忠自承、楚圖淮陽。劉豫方聚兵淮陽,世忠即引軍渡淮,旁符離而北,至其城下。為賊所圍,奮戈一躍,潰圍而出,不遺一鏃。呼
 延通與金將牙合孛堇搏戰,扼其吭而禽之,乘銳掩擊,金人敗去。既而圍淮陽,賊堅守不下,約曰:「受圍一日,則舉一烽。」至是,六烽具舉,兀朮與劉猊皆至。世忠求援於張俊,俊以世忠有見吞意,不從。世忠勒陣向敵,遣人語之曰:「錦衣驄馬立陣前者,韓相公也。」或危之,世忠曰:「不如是,不足以致敵。」敵果至,殺其導戰二人,遂引去。尋詔班師,復歸楚州,淮陽之民,從而歸者以萬計。



 三月,除京東、淮東宣撫處置使兼節制鎮江府,仍楚州置司。四月,
 賜號「揚武翊運功臣」,加橫海、武寧、安化三鎮節度使。九月,帝在平江,世忠自楚州來朝。



 十月,邊報急,劉光世欲棄廬州還太平,張俊亦請益兵。都督張浚曰:「今日之事,有進擊,無退保。」於是世忠引兵渡淮,與金將訛里也力戰。劉猊將寇淮東,為世忠兵扼,不得進。七年,築高郵城,民益安之。



 初,世忠移屯山陽,遣間結山東豪傑,約以緩急為應,宿州馬秦及太行群盜,多願奉約束者。金人廢劉豫,中原震動,世忠謂機不可失,請全師北討,招納歸
 附,為恢復計。會秦檜主和議,命世忠徙屯鎮江。世忠言:「金人詭詐,恐以計緩我師,乞留此軍蔽遮江、淮。」又力陳和議之非,願效死節,率先迎敵;若不勝,從之未晚。又言王倫、藍公佐交河南地界,乞令明具無反復文狀為後證。章十數上,皆慷慨激切,且請單騎詣闕面奏,帝率優詔褒答。後金果渝盟,咸如其言。



 金使蕭哲之來,以詔諭為名,世忠聞之,凡四上疏言:「不可許,願舉兵決戰,兵勢最重處,臣請當之。」又言:「金人欲以劉豫相待,舉國士大
 夫盡為陪臣,恐人心離散,士氣凋沮。」且請馳驛面奏,不許。既而伏兵洪澤鎮,將殺金使,不克。



 九年,授少師。十年,金人敗盟,兀朮率撒離曷、李成等破三京,分道深入。八月,世忠圍淮陽,金人來救,世忠迎擊於泇口鎮,敗之。又遣解元擊金人於潭城,劉寶擊於千秋湖,皆捷。親隨將成閔從統制許世安奪淮陽門而入,大戰門內。世安中四矢,閔被三十餘創,復奪門出。世忠奏其功,擢武德大夫,閔由是知名。世忠進太保,封英國公,兼河南、北諸路招
 討使。



 十一年,兀朮恥順昌之敗,復謀再入,詔大合兵於淮西以待。既而金敗於柘皋,復圍濠州。世忠受詔救濠,以舟師至招信縣,夜以騎兵擊金人於聞賢驛,敗之。金人攻濠州,五日而破。破三日,世忠至,楊沂中軍已南奔。世忠與金人戰於淮岸,夜遣劉寶溯流將劫之,金人伐木塞赤龍洲,扼其歸路,世忠知之,全師而還。金人自渦口渡淮北去,自是不得入侵。世忠在楚州十餘年,兵僅三萬,而金人不敢犯。



 秦檜收三大將權,四月,拜樞密使,
 遂以所積軍儲錢百萬貫,米九十萬石,酒庫十五歸於國。世忠既不以和議為然,為檜所抑。及魏良臣使金,世忠又力言:「自此人情消弱,國勢委靡,誰復振之?北使之來,乞與面議。」不許,遂抗疏言檜誤國。檜諷言者論之,帝格其奏不下。世忠連疏乞解樞密柄,繼上表乞骸。十月,罷為醴泉觀使、奉朝請,進封福國公,節鉞如故。自此杜門謝客,絕口不言兵,時跨驢攜酒,從一二奚童,縱游西湖以自樂,平時將佐罕得見其面。



 十二年,改潭國公。顯
 仁皇后自金還,世忠詣臨平朝謁。後在北方聞其名,慰問者良久。十三年,封咸安郡王。十七年,改鎮南、武安、寧國節度使。二十一年八月薨,進拜太師,追封通義郡王。孝宗朝,追封蘄王,謚忠武,配饗高宗廟庭。



 世忠初得疾,敕尚醫視療,將吏臥內問疾,世忠曰:「吾以布衣百戰,致位王公,賴天之靈,保首領沒於家,諸君尚哀其死邪?」及死,剛朝服、貂蟬冠、水銀、龍腦以斂。



 世忠嘗戒家人曰:「吾名世忠,汝曹毋諱『忠』字,諱而不言,是忘忠也。」性戇直,勇
 敢忠義,事關廟社,必流涕極言。岳飛冤獄,舉朝無敢出一語,世忠獨攖檜怒,語在《檜傳》。又抵排和議,觸檜尤多,或勸止之,世忠曰:「今畏禍茍同,他日瞑目,豈可受鐵杖於太祖殿下?」時一二大將,多曲徇檜茍全,世忠與檜同在政地,一揖外未嘗與談。



 嗜義輕財,錫繼悉分將士,所賜田輸租與編戶等。持軍嚴重,與士卒同甘苦,器仗規畫,精絕過人,今克敵弓、連鎖甲、狻猊鍪,及跳澗以習騎,洞貫以習射,皆其遺法也。嘗中毒矢入骨,以強弩括取
 之,十指僅全四,不能動,刀痕箭瘢如刻畫。然知人善獎用,成閔、解元、王勝、王權、劉寶、岳超起行伍,秉將旄,皆其部曲云。解兵罷政,臥家凡十年,澹然自如,若未嘗有權位者。晚喜釋、老,自號清涼居士。



 子彥直、彥質、彥古,皆以才見用。彥古戶部尚書。



 彥直字子溫。生期年,以父任補右承奉郎,尋直秘閣。六歲,從世忠入見高宗,命作大字,即拜命跪書「皇帝萬歲」四字。帝喜之,拊其背曰:「他日,令器也。」親解孝宗丱角之
 繻傅其首,賜金器、筆研、監書、鞍馬。年十二,賜三品服。



 紹興十七年,中兩浙轉運司試。明年,登進士第,調太社令。二十一年,世忠薨,服除,秦檜素銜世忠不附和議,出彥直為浙東安撫司主管機宜文字。檜死,拜光祿寺丞。二十九年,遷屯田員外郎兼權右曹郎官、工部侍郎。張浚都督江、淮軍馬,檄權計議軍事。督府罷,奉祠。



 乾道二年,遷戶部郎官、主管左曹,總領淮東軍馬錢糧。會大軍倉給糧,徑乘小輿往察之,給米不如數,捕吏寘於理。初,代
 者以乏興罷,交承,為緡錢僅二十萬,明年奏計乃四倍,且以其贏獻諸朝。帝嘉之。拜司農少卿,進直龍圖閣、江西轉運兼權知江州。



 時朝廷還岳飛家貲產多在九江,歲久業數易主,吏緣為奸。彥直搜剔隱匿,盡還岳氏。復為司農少卿,總領湖北、京西軍馬錢糧,尋兼發運副使。會時相不樂,密啟換武,授利州觀察使、知襄陽府,充京西南路安撫使。



 七年,授鄂州駐札御前諸軍都統制。條奏軍中六事,乞備器械、增戰馬、革濫賞、厲奇功、選勇略、
 充親隨等,朝廷多從之。先是,軍中騎兵多不能步戰,彥直命騎士被甲徒行,日六十里,雖統制官亦令以身帥之,人人習於勞苦,馳騁如飛。事聞,詔令三衙、江上諸軍仿行之。



 八年,丐歸文班,乃授左中奉大夫,充敷文閣待制、知臺州。丐祠養親,提舉祐神觀、奉朝請。進對言:「頃自岳飛為帥,身居鄂渚,遙領荊襄,田師中繼之,始分鄂渚為二軍,乞復舊。」又乞並京西、湖北轉運為一司,分官置司襄陽,可一事體,帝善之。遷刑部侍郎。



 明年,兼工部侍
 郎,同列議:大闢三鞫之弗承,宜令以眾證就刑,欲修立為令。彥直持不可,白丞相梁克家曰:「若是,則善類被誣,必多冤獄。且笞杖之刑,猶引伏方決,況人命至重乎?」議卒格。以議奪吳名世改正過名不當,降兩官。



 會當遣使於金,在廷相顧莫肯先,帝親擇以往,聞命慨然就道。方入境,金使蒲察問接國書事,論難往復數十,蒲察理屈,因笑曰:「尚書能力為主。」既至,幾罹禍者數,守節不屈,金卒禮遣之,帝嘉嘆。遷吏部侍郎,尋權工部尚書,復中大
 夫,改工部尚書兼知臨安府。方控辭,以言罷,提舉太平興國宮,尋提舉祐神觀、奉朝請。



 尋知濕州,首捕巨猾王永年窮治之,杖徙他州。奏免民間積逋,以郡餘財代輸之,然以累欠內帑坊場錢不發,鐫一官。海寇出沒大洋劫掠,勢甚張,彥直授將領土豪等方略,不旬日,生禽賊首,海道為清。樞密奏功,進敷文閣學士,以弟彥質為兩浙轉運判官,引嫌易泉府。丐祠奉親,差提舉祐神觀,仍奉朝請,特令佩魚,示異數也。



 入對,乞搜訪靖康以來死
 節之士,以勸忠義。又上薦舉乞選人已經關升、實歷六考、無贓私罪犯者,雜試以經術法律,限其員額,定其高下,俾孤寒者得以自達,定為改官之制。又乞令州郡守臣任滿日,開具本州實在財賦數目,具公移與交代者,並達臺省,庶可核實,以戢奸弊,帝悉嘉納。



 淳熙十年夏旱,應詔言,邇者濫刑,為致旱之由。明年,入對,論三衙皆所以拱扈宸居,而司馬乃遠在數百里外,乞令歸司。久之,再為戶部尚書。會歲旱,乞廣糴為先備。又乞追貶部
 曲曾誣陷岳飛者,以慰忠魂。以言降充敷文閣學士。帝追感世忠元勛,遣使諭彥直,且謂彥直有才力,言者誣之。彥直感泣奏謝。尋提舉萬壽觀,有疾,帝賜之藥。進顯謨閣學士、提舉萬壽觀。



 嘗摭宋朝事,分為類目,名《水心鏡》,為書百六十七卷。禮部尚書尤袤修國史,白於朝,下取是書以進,光宗覽之,稱善。進龍圖閣學士、提舉萬壽觀,轉光祿大夫致仕。卒,特贈開府儀同三司,賜銀絹九百,爵至蘄春郡公。



 論曰:古人有言:「天下安,注意相;天下危,注意將。」宋靖康、建炎之際,天下安危之機也,勇略忠義如韓世忠而為將,是天以資宋之興復也。方兀朮渡江,惟世忠與之對陣,以閑暇示之。及劉豫廢,中原人心動搖,世忠請乘時進兵,此機何可失也?高宗惟奸檜之言是聽,使世忠不得盡展其才,和議成而宋事去矣。暮年退居行都,口不言兵,部曲舊將,不與相見,蓋懲岳飛之事也。昔漢文帝思頗、牧於前代,宋有世忠而不善用,惜哉!



\end{pinyinscope}