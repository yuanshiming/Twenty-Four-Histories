\article{列傳第一百九十三儒林四}

\begin{pinyinscope}

 ○劉子翬呂祖謙蔡元定子沉陸九齡兄九韶陸九淵薛季宣陳傅良葉適戴溪蔡幼學楊泰之



 劉子翬,字彥沖,贈太師韐之仲子。以父任授承務郎,闢真定府幕屬。韐死靖康之難,子翬痛憤,幾無以為生,墓三年。服除,通判興化軍。寇楊勍犯閩境,子翬與郡將張當世畫計備御,如素服戎事者,賊不敢犯。事聞,詔因任。



 子翬始執喪致羸疾,至是以不堪吏責,辭歸武夷山,不出者凡十七年。間走其父墓下,瞻望徘徊,涕泗嗚咽,或累日而返。妻死不再娶,事繼母呂氏及兄子羽盡孝友。子羽之子珙,幼英敏嗜學,子翬教之不懈,珙卒有立。



 與籍溪胡憲、白水劉勉之交相得,每見,講學外無雜言。它所與游,皆海內知名士,而期以任重致遠者,惟新安朱熹而已。初,熹父松且死,以熹托子翬。及熹請益,子翬告以《易》之「不遠復」三言,俾佩之終身,熹後卒為儒宗。子翬少喜佛氏說,歸而讀《易》,即渙然有得。其說以為學《易》當先《復》,故以是告熹焉。



 一日,感微疾,即謁家廟,泣別母,與親朋訣,付珙家事,指葬處,處親戚孤弱之無業者,訓學者修身求道數百言。後二日卒,年四十七。學者稱屏
 山先生。珙,別有傳。



 呂祖謙,字伯恭,尚書右丞好問之孫也。自其祖始居婺州。祖謙之學本之家庭,有中原文獻之傳。長從林之奇、汪應辰、胡憲游,既又友張栻、朱熹,講索益精。



 初,蔭補入官,後舉進士,復中博學宏詞科,調南外宗教。丁內艱,居明招山,四方之士爭趨之。除太學博士,時中都官待次者例補外,添差教授嚴州,尋復召為博士兼國史院編修官、實錄院檢討官。輪對,勉孝宗留意聖學。且言:「恢復
 大事也,規模當定,方略當審。陛下方廣攬豪傑,共集事功,臣願精加考察,使之確指經畫之實,孰為先後,使嘗試僥幸之說不敢陳於前,然後與一二大臣定成算而次第行之,則大義可伸,大業可復矣。」



 召試館職。先是,召試者率前期從學士院求問目,獨祖謙不然,而其文特典美。嘗讀陸九淵文,喜之,而未識其人。考試禮部,得一卷,曰:「此必江西小陸之文也。」揭示,果九淵,人服其精鑒。父憂,免喪,主管臺州崇道觀。



 越三年,除秘書郎、國史院
 編修官、實錄院檢討官。以修撰李燾薦,重修《徽宗實錄》。書成,進秩。面對,言曰:「夫治道體統,上下內外不相侵奪而後安。鄉者,陛下以大臣不勝任而兼行其事,大臣亦皆親細務而行有司之事,外至監司、守令職任,率為其上所侵而不能令其下。故豪猾玩官府,郡縣忽省部,掾屬凌長吏,賤人輕柄臣。平居未見其患,一旦有急,誰與指麾而伸縮之邪?如曰臣下權任太重,懼其不能無私,則有給、舍以出納焉,有臺諫以救正焉,有侍從以詢訪
 焉。儻得端方不倚之人分處之,自無專恣之慮,何必屈至尊以代其勞哉?人之關鬲脈絡少有壅滯,久則生疾。陛下於左右雖不勞操制,茍玩而弗慮,則聲勢浸長,趨附浸多,過咎浸積,內則懼為陛下所遣而益思壅蔽,外則懼為公議所疾而益肆詆排。願陛下虛心以求天下之士,執要以總萬事之機。勿以圖任或誤而謂人多可疑,勿以聰明獨高而謂智足遍察,勿詳於小而忘遠大之計,勿忽於近而忘壅蔽之萌。」



 又言:「國朝治體,有遠過
 前代者,有視前代為未備者。夫以寬大忠厚建立規模,以禮遜節義成就風俗,此所謂遠過前代者也。故於俶擾艱危之後,駐蹕東南逾五十年,無纖毫之虞,則根本之深可知矣。然文治可觀而武績未振,名勝相望而幹略未優,故雖昌熾盛大之時,此病已見。是以元昊之難,範、韓皆極一時之選,而莫能平殄,則事功之不競從可知矣。臣謂今日治體視前代未備者,固當激厲而振起。遠過前代者,尤當愛護而扶持。」



 遷著作郎,以末疾,請
 祠歸。先是,書肆有書曰《聖宋文海》,孝宗命臨安府校正刊行。學士周必大言:《文海》去取差謬,恐難傳後,盍委館職銓擇,以成一代之書?孝宗以命祖謙。遂斷自中興以前,崇雅黜浮,類為百五十卷,上之,賜名《皇朝文鑒》。



 詔除直秘閣。時方重職名,非有功不除,中書舍人陳爓駁之。孝宗批旨云:「館閣之職,文史為先。祖謙所進,採取精詳,有益治道,故以寵之,可即命詞。」爓不得已草制。尋主管沖祐觀。明年,除著作郎兼國史院編修官。卒,年四十五。謚
 曰成。



 祖謙學以關、洛為宗,而旁稽載籍,不見涯涘。心平氣和,不立崖異,一時英偉卓犖之士皆歸心焉。少卞急,一日,誦孔子言:「躬自厚而薄責於人」,忽覺平時忿懥渙然冰釋。朱熹嘗言:「學如伯恭,方是能變化氣質。」其所講畫,將以開物成務,既臥病,而任重道遠之意不衰。居家之政,皆可為後世法。修《讀詩記》、《大事記》,皆未成書。考定《古周易》、《書說》、《閫範》、《官箴》、《辨志錄》、《歐陽公本末》,皆行於世。晚年會友之地曰麗澤書院,在金華城中,既歿。郡人即
 而祠之。子延年。



 蔡元定,字季通,建州建陽人。生而穎悟,八歲能詩,日記數千言。父發,博覽群書,號牧堂老人,以程氏《語錄》、邵氏《經世》、張氏《正蒙》授元定,曰:「此孔、孟正脈也。」元定深涵其義。既長,辨析益精。登西山絕頂,忍饑啖薺讀書。



 聞朱熹名,往師之。熹扣其學,大驚曰:「此吾老友也,不當在弟子列。」遂與對榻講論諸經奧義,每至夜分。四方來學者,熹必俾先從元定質正焉。太常少卿尤袤、秘書少監楊萬
 里聯疏薦於朝,召之,堅以疾辭。築室西山,將為終焉之計。



 時韓侂胄擅政,設偽學之禁,以空善類。臺諫承風,專肆排擊,然猶未敢誦言攻朱熹。至沈繼祖、劉三傑為言官,始連疏詆熹,並及元定。元定簡學者劉礪曰:「化性起偽,烏得無罪!」未幾,果謫道州。州縣捕元定甚急,元定聞命,不辭家即就道。熹與從游者數百人餞別蕭寺中,坐客興嘆,有泣下者。熹微視元定,不異平時,因喟然曰:「友朋相愛之情,季通不挫之志,可謂兩得矣。」元定賦詩曰:「
 執手笑相別,無為兒女悲。」眾謂宜緩行,元定曰:「獲罪於天,天可逃乎?」杖屨同其子沉行三千里,腳為流血,無幾微見言面。



 至舂陵,遠近來學者日眾,州士子莫不趨席下以聽講說。有名士挾才簡傲、非笑前修者,亦心服謁拜,執弟子禮甚恭。人為之語曰:「初不敬,今納命。」愛元定者謂宜謝生徒,元定曰:「彼以學來,何忍拒之?若有禍患,亦非閉門塞竇所能避也。」貽書訓諸子曰:「獨行不愧影,獨寢不愧衾,勿以吾得罪故遂懈。」一日,謂沉曰:「可謝客,
 吾欲安靜,以還造化舊物。」閱三日卒。侂胄既誅,贈迪功郎,賜謚文節。



 元定於書無所不讀,於事無所不究。義理洞見大原,下至圖書、禮樂、制度,無不精妙。古書奇辭奧義,人所不能曉者,一過目輒解。熹嘗曰:「人讀易書難,季通讀難書易。」熹疏釋《四書》及為《易》、《詩》傳、《通鑒綱目》,皆與元定往復參訂。《啟蒙》一書,則屬元定起稿。嘗曰:「造化微妙,惟深於理者能識之,吾與季通言而不厭也。」及葬,以文誄之曰:「精詣之識,卓絕之才,不可屈之志,不可窮之
 辯,不復可得而見矣。」學者尊之曰西山先生。



 其平生問學,多寓於熹書集中。所著書有《大衍詳說》、《律呂新書》、《燕樂》、《原辯》、《皇極經世》、《太玄潛虛指要》、《洪範解》、《八陣圖說》,熹為之序。



 子淵、沉,皆躬耕不仕。淵有《周易訓解》。



 沉字仲默,少從朱熹游。熹晚欲著《書傳》,未及為,遂以屬沉。《洪範》之數,學者久失其傳,元定獨心得之,然未及論著,曰:「成吾書者沉也。」沉受父師之托,沈潛反復者數十年,然後成書,發明先儒之所未及。其於《洪範》數,謂:「體天地之撰者《
 易》之象,紀天地之撰者《範》之數。數始於一奇,象成於二偶。奇者數之所以立,偶者數之所以行。故二四而八,八卦之象也;三三而九,九疇之數也。由是八八而又八八之為四千九十六,而象備矣;九九而又九九之為六千五百六十一,而數周矣。《易》更四聖而象已著,《範》錫神禹而數不傳。後之作者,昧象數之原,窒變通之妙,或即象而為數,或反數而擬象,牽合傅會,自然之數益晦焉。」



 始,從元定謫道州,跋涉數千里,道楚、粵窮僻處,父子相對,
 常以理義自怡悅。元定沒,徒步護喪以還。有遺之金而義不可受者,輒謝卻,之曰:「吾不忍累先人也。」年僅三十,屏去舉子業,一以聖賢為師。隱居九峰,當世名卿物色將薦用之,沉不屑就。次子抗,別有傳。



 陸九齡,字子壽。八世祖希聲,相唐昭宗。孫德遷,五代末,避亂居撫州之金溪。父賀,以學行為里人所宗,嘗採司馬氏冠昏喪祭儀行於家,生六子,九齡其第五子也。幼穎悟端重,十歲喪母,哀毀如成人。稍長,補郡學弟子員。



 時秦檜當國,無道程氏學者,九齡獨尊其說。久之,聞新博士學黃、老,不事禮法,慨然嘆曰:「此非吾所願學也。」遂歸家,從父兄講學益力。是時,吏部員外郎許忻有名中朝,退居臨川,少所賓接,一見九齡,與語大說,盡以當代文獻告之。自是九齡益大肆力於學,翻閱百家,晝夜不倦,悉通陰陽、星歷、五行、卜筮之說。



 性周謹,不肯茍簡涉獵。入太學,司業汪應辰舉為學錄。登乾道五年進士第。調桂陽軍教授,以親老道遠改興國軍,未上,會湖南茶
 寇剽廬陵,聲搖旁郡,人心震攝。舊有義社以備寇,郡從眾請,以九齡主之,門人多不悅。九齡曰:「文事武備,一也。古者有征討,公卿即為將帥,比閭之長,則五兩之率也。士而恥此,則豪俠武斷者專之矣。」遂領其事,調度屯御皆有法。寇雖不至,而郡縣倚以為重。暇則與鄉之子弟習射,曰:「是固男子之事也。」歲惡,有剽劫者過其門,必相戒曰:「是家射多命中,無自取死。」



 及至興國,地濱大江,俗儉嗇而鮮知學。九齡不以職閑自佚,益嚴規矩,肅衣冠,
 如臨大眾,勸綏引翼,士類興起。不滿歲,以繼母憂去。服除,調全州教授。未上,得疾。一日晨興,坐床上與客語,猶以天下學術人才為念。至夕,整襟正臥而卒。年四十九。寶慶二年,特贈朝奉郎、直秘閣,賜謚文達。



 九齡嘗繼其父志,益修禮學,治家有法。闔門百口,男女以班各供其職,閨門之內嚴若朝廷。而忠敬樂易,鄉人化之,皆遜弟焉。與弟九淵相為師友,和而不同,學者號「二陸」。有來問學者,九齡從容啟告,人人自得。或未可與語,則不發。嘗
 曰:「人之惑有難以口舌爭者,言之激,適固其意;少需,未必不自悟也。」



 廣漢張栻與九齡不相識,晚歲以書講學,期以世道之重。呂祖謙常稱之曰:「所志者大,所據者實。有肯綮之阻,雖積九仞之功不敢遂;有毫厘之偏,雖立萬夫之表不敢安。公聽並觀,卻立四顧,弗造於至平至粹之地,弗措也。」兄九韶。



 九韶字子美。其學淵粹。隱居山中,晝之言行,夜必書之。其家累世義居,一人最長者為家長,一家之事聽命焉。
 歲遷子弟分任家事,凡田疇、租稅、出內、庖爨、賓客之事,各有主者。九韶以訓戒之辭為韻語,晨興,家長率眾子弟謁先祠畢,擊鼓誦其辭,使列聽之。子弟有過,家長會眾子弟責而訓之,不改,則撻之,終不改,度不可容,則言之官府,屏之遠方焉。九韶所著有《梭山文集》、《家制》、《州郡圖》。



 陸九淵,字子靜。生三四歲,問其父天地何所窮際,父笑而不答。遂深思,至忘寢食。及總角,舉止異凡兒,見者敬
 之。謂人曰:「聞人誦伊川語,自覺若傷我者。」又曰:「伊川之言,奚為與孔子、孟子之言不類?近見其間多有不是處。」初讀《論語》,即疑有子之言支離。他日讀古書,至「宇宙」二字,解者曰「四方上下曰宇,往古來今曰宙」,忽大省曰:「宇宙內事乃己分內事,己分內事乃宇宙內事。」又嘗曰:「東海有聖人出焉,此心同也,此理同也。至西海、南海、北海有聖人出,亦莫不然。千百世之上有聖人出焉,此心同也,此理同也。至於千百世之下有聖人出,此心此理,亦
 無不同也。」



 後登乾道八年進士第。至行在,士爭從之游。言論感發,聞而興起者甚眾。教人不用學規,有小過,言中其情,或至流汗。有懷於中而不能自曉者,為之條析其故,悉如其心。亦有相去千里,聞其大概而得其為人。嘗曰:「念慮之不正者,頃刻而知之,即可以正。念慮之正者,頃刻而失之,即為不正。有可以形跡觀者,有不可。以形跡觀人,則不足以知人。必以形跡繩人,則不足以救之。」初調隆興靖安縣主簿。丁母憂,服闋,改建寧崇安縣。
 以少師史浩薦,召審察,不赴。侍從復薦,除國子正,教諸生無異在家時。除敕令所刪定官。



 九淵少聞靖康間事,慨然有感於復仇之義。至是,訪知勇士,與議恢復大略。因輪對,遂陳五論:一論仇恥未復,願博求天下之俊傑,相與舉論道經邦之職;二論願致尊德樂道之誠;三論知人之難;四論事當馴致而不可驟;五論人主不當親細事。帝稱善。未幾,除將作監丞,為給事中王信所駁,詔主管臺州崇道觀。還鄉,學者輻湊,每開講席,戶外屨滿,
 耆老扶杖觀聽。自號象山翁,學者稱象山先生。嘗謂學者曰:「汝耳自聰,目自明,事父自能孝,事兄自能弟,本無欠闕,不必它求,在乎自立而已。」又曰:「此道與溺於利欲之人言猶易,與溺於意見之人言卻難。」或勸九淵著書,曰:「《六經》注我,我注《六經》。」又曰:「學茍知道,《六經》皆我注腳。」



 光宗即位,差知荊門軍。民有訴者,無早暮,皆得造於庭,復令其自持狀以追,為立期,皆如約而至,即為酌情決之,而多所勸釋。其有涉人倫者,使自毀其狀,以厚風俗。
 唯不可訓者,始置之法。其境內官吏之貪廉,民俗之習尚善惡,皆素知之。有訴人殺其子者,九淵曰:「不至是。」及追究,其子果無恙。有訴竊取而不知其人,九淵出二人姓名,使捕至,訊之伏辜,盡得所竊物還訴者,且宥其罪使自新。因語吏以某所某人為暴,翌日有訴遇奪掠者,即其人也,乃加追治。吏大驚,郡人以為神。申嚴保伍之法,盜賊或發,擒之不逸一人,群盜屏息。



 荊門為次邊而無城。九淵以為:「郡居江、漢之間,為四集之地,南捍江陵,北
 援襄陽,東護隨、郢之肋,西當光化、夷陵之沖,荊門固則四鄰有所恃,否則有背肋腹心之虞,由唐之湖陽以趨山,則其涉漢之處已在荊門之肋;由鄧之鄧城以涉漢,則其趨山之處已在荊門之腹。自此之外,間道之可馳,漢津之可涉,坡陀不能以限馬,灘瀨不能以濡軌者,所在尚多。自我出奇制勝,徼敵兵之腹肋者,亦正在此。雖四山環合,易於備御,而城池闕然,將誰與守?」乃請於朝而城之,自是民無邊憂。罷關市吏譏察而減民稅,商賈
 畢集,稅入日增。舊用銅錢,以其近邊,以鐵錢易之,而銅有禁,復令貼納。九淵曰:「既禁之矣,又使之輸邪?」盡蠲之。故事,平時教軍伍射,郡民得與,中者均賞,薦其屬不限流品。嘗曰:「古者無流品之分,而賢不肖之辨嚴;後世有流品之分,而賢不肖之辨略。」每旱,禱即雨,郡人異之。逾年,政行令修,民俗為變,諸司交薦。丞相周必大嘗稱荊門之政,以為躬行之效。



 一日,語所親曰:「先教授兄有志天下,竟不得施以沒。」又謂家人曰:「吾將死矣。」又告僚屬
 曰:「某將告終。」會禱雪,明日,雪。乃沐浴更衣端坐,後二日日中而卒。會葬者以千數,謚文安。



 初,九淵嘗與朱熹會鵝湖,論辨所學多不合。及熹守南康,九淵訪之,熹與至白鹿洞,九淵為講君子小人喻義利一章,聽者至有泣下。熹以為切中學者隱微深痼之病。至於無極而太極之辨,則貽書往來,論難不置焉。門人楊簡、袁燮、舒璘、沈煥能傳其學云。



 薛季宣,字士龍,永嘉人。起居舍人徽言之子也。徽言卒
 時,季宣始六歲,伯父敷文閣待制弼收鞠之。從弼宦游,及見渡江諸老,聞中興經理大略。喜從老校、退卒語,得岳、韓諸將兵間事甚悉。年十七,起從荊南帥闢書寫機宜文字,獲事袁溉。溉嘗從程頤學,盡以其學授之。季宣既得溉學,於古封建、井田、鄉遂、司馬法之制,靡不研究講畫,皆可行於時。



 金兵之未至也,武昌令劉錡鎮鄂渚。季宣白錡,以武昌形勢直淮、蔡,而兵寡勢弱,宜早為備,錡不聽。及兵交,稍稍資季宣計畫。未幾,汪澈宣諭荊襄,
 而金兵趨江上,詔成閔還師入援。季宣又說澈以閔既得蔡,有破竹之勢,宜守便宜勿遣,而令其乘勝下潁昌,道陳、汝,趨汴都,金內顧且驚潰,可不戰而屈其兵矣。澈不聽。



 時江、淮仕者聞金兵且至,皆預遣其奴而系馬於庭以待。季宣獨留家,與民期曰:「吾家即汝家,即有急,吾與汝偕死。」民亦自奮。縣多盜,季宣患之,會有伍民之令,乃行保伍法,五家為保,二保為甲,六甲為隊,因地形便合為總,不以鄉為限,總首、副總首領之。官族、士族、富族
 皆附保,蠲其身,俾輸財供總之小用。諸總必有圃以習射,禁蒱博雜戲,而許以武事角勝負,五日更至庭閱之,而賞其尤者;不幸死者予棺,復其家三年。鄉置樓,盜發,伐鼓舉烽,瞬息遍百里。縣治、白鹿磯、安樂口皆置戍。復請於宣諭司,得戰艦十,甲三百,羅落之。守計定,訖兵退,人心不搖。



 樞密使王炎薦於朝,召為大理寺主簿,未至,為書謝炎曰:「主上天資英特,群臣無將順緝熙之具,幸得遭時,不能格心正始,以建中興之業,徒僥幸功利,誇
 言以眩俗,雖復中夏,猶無益也。為今之計,莫若以仁義紀綱為本。至於用兵,請俟十年之後可也。」



 時江、湖大旱,流民北渡江,邊吏復奏淮北民多款塞者,宰相虞允文白遣季宣行淮西,收以實邊。季宣為表廢田,相原隰,復合肥三十六圩,立二十二莊於黃州故治東北,以戶授屋,以丁授田,頒牛及田器穀種各有差,廩其家,至秋乃止。凡為戶六百八十有五,分處合肥、黃州間,並邊歸正者振業之。季宣謂人曰:「吾非為今日利也。合肥之圩,邊
 有警,因以斷柵江,保巢湖。黃州地直蔡沖,諸莊輯則西道有屏蔽矣。」光州守宋端友招集北歸者止五戶,而雜舊戶為一百七十,奏以幸賞,季宣按得其實而劾之。時端友為環列附托難撼,季宣奏上,孝宗怒,屬大理治,端友以憂死。



 季宣還,言於孝宗曰:「左右之人進言者,其情不可不察也。托正以行邪,偽直以售佞,薦退人物,曾非誦言,游揚中傷,乃自不意。一旦號令雖自中出,而其權已歸私門矣。故齊威之霸,不在阿、即墨之誅賞,而在毀
 譽者之刑。臣觀近政,非無阿、即墨之誅賞,奈何毀譽之人自若乎?」帝曰:「朕方圖之。」



 季宣又進言曰:「日城淮郡,以臣所見,合肥板乾方立,中使督視,卒卒成之。臣行過郡,一夕風雨,墮樓五堵。歷陽南壁闕,而居巢庳陋如故,乃聞有靡錢鉅萬而成城四十餘丈者。陛下安取此!然外事無足道,咎根未除,臣所深憂。左右近侍,陰擠正士而陽稱道之,陛下儻因貌言而聽之,臣恐石顯、王鳳、鄭注之智中也。」又言:「近或以好名棄士大夫,夫好特為臣
 子學問之累。人主為社稷計,唯恐士不好名,誠人人好名畏義,何鄉不立?」帝稱善,恨得季宣晚,遂進兩官,除大理正。



 自是,凡奏請論薦皆報可。以虞允文諱闕失,不樂之。居七日,出知湖州,會戶部以歷付場務,錙銖皆分隸經總制,諸郡束手無策,季宣言於朝曰:「自經總制立額,州縣鑿空以取贏,雖有奉法吏思寬弛而不得騁。若復額外征其強半,郡調度顧安所出?殆復巧取之民,民何以勝!」戶部譙責愈急,季宣爭之愈強,臺諫交疏助之,乃
 收前令。



 改知常州,未上,卒,年四十。季宣於《詩》、《書》、《春秋》、《中庸》、《大學》、《論語》皆有訓義,藏於家。其雜著曰《浪語集》。



 陳傅良,字君舉,溫州瑞安人。初患科舉程文之弊,思出其說為文章,自成一家,人爭傳誦,從者雲合,由是其文擅當世。當是時,永嘉鄭伯熊、薛季宣皆以學行聞,而伯熊於古人經制治法,討論尤精,傅良皆師事之,而得季宣之學為多。及入太學,與廣漢張栻、東萊呂祖謙友善。祖謙為言本朝文獻相承條序,而主敬集義之功得於
 栻為多。自是四方受業者愈眾。



 登進士甲科,教授泰州。參知政事龔茂良才之,薦於朝,改太學錄。出通判福州。丞相梁克家領帥事,委成於傅良,傅良平一府曲直,壹以義。強御者不得售其私,陰結言官論罷之。



 後五年,起知桂陽軍。光宗立,稍遷提舉常平茶鹽、轉運判官。湖湘民無後,以異姓以嗣者,官利其貲,輒沒入之。傅良曰:「絕人嗣,非政也。」復之幾二千家。轉浙西提點刑獄。除吏部員外郎,去朝十四年,至是而歸,須鬢無黑者,都人聚觀
 嗟嘆,號「老陳郎中」。



 傅良為學,自三代、秦、漢以下靡不研究,一事一物,必稽於極而後已。而於太祖開創本原,尤為潛心。及是,因輪對,言曰:「太祖皇帝垂裕後人,以愛惜民力為本。熙寧以來,用事者始取太祖約束,一切紛更之。諸路上供歲額,增於祥符一倍。崇寧重修上供格,頒之天下,率增至十數倍。其它雜斂,則熙寧以常平寬剩、禁軍闕額之類別項封樁,而無額上供起於元豐,經制起於宣和,總制、月樁起於紹興,皆迄今為額,折帛、和
 買之類又不與焉。茶引盡歸於都茶場,鹽鈔盡歸於榷貨務,秋苗斗斛十八九歸於綱運,皆不在州縣。州縣無以供,則豪奪於民,於是取之斛面、折變、科敷、抑配、贓罰,而民困極矣。方今之患,何但四夷?蓋天命之永不永,在民力之寬不寬耳,豈不甚可畏哉?陛下宜以救民窮為己任,推行太祖未泯之澤,以為萬世無疆之休。」



 且言:「今天下之力竭於養兵,而莫甚於江上之軍。都統司謂之御前軍馬,雖朝廷不得知;總領所謂之大軍錢糧,雖版曹
 不得與。於是中外之勢分,而事權不一,施行不專,雖欲寬民,其道無由。誠使都統司之兵與向者在制置司時無異,總領所之財與向者在轉運司時無異,則內外為一體。內外一體,則寬民力可得而議矣。」帝從容嘉納,且勞之曰:「卿昔安在?朕不見久矣。其以所著書示朕。」退,以《周禮說》十三篇上之,遷秘書少監兼實錄院檢討官、嘉王府贊讀。



 紹熙三年,除起居舍人。明年,兼權中書舍人。初,光宗之妃黃氏有寵,李皇后妒而殺之。光宗既聞之,
 而復因郊祀大風雨,遂震懼得心疾,自是視章疏不時。於是傅良奏曰:「一國之勢猶身也,壅底則致疾。今日遷延某事,明日阻節某人,即有奸險乘時為利,則內外之情不接,威福之柄下移,其極至於天變不告,邊警不聞,禍且不測矣!」帝悟,會疾亦稍平,過重華宮。而明年重明節,復以疾不往,丞相以下至於太學諸生皆力諫,不聽,而方召內侍陳源為內侍省押班,傅良不草詞,且上疏曰:「陛下之不過宮者,特誤有所疑而積憂成疾,以至此
 爾。臣嘗即陛下之心反覆論之,竊自謂深切,陛下亦既許之矣。未幾中變,以誤為實,而開無端之釁;以疑為真,而成不療之疾。是陛下自貽禍也。」書奏,帝將從之。百官班立,以俟帝出。至御屏,皇后挽帝回,傅良遂趨上引裾,後叱之。傅良哭於庭,後益怒,傅良下殿徑行。詔改秘閣修撰仍兼贊讀,不受。



 寧宗即位,召為中書舍人兼侍讀、直學士院、同實錄院修撰。會詔朱熹與在外宮觀,傅良言:「熹難進易退,內批之下,舉朝驚愕,臣不敢書行。」熹於
 是進寶文閣待制,與郡。御史中丞謝深甫論傅良言不顧行,出提舉興國宮。明年察官交疏,削秩罷。嘉泰二年復官,起知泉州,辭。授集英殿修撰,進寶謨閣待制,終於家,年六十七。謚文節。



 傅良著述有《詩解詁》、《周禮說》、《春秋後傳》、《左氏章指》行於世。



 葉適,字正則,溫州永嘉人。為文藻思英發。擢淳熙五年進士第二人,授平江節度推官。丁母憂。改武昌軍節度判官。少保史浩薦於朝,召之不至,改浙西提刑司干辦
 公事,士多從之游。參知政事龔茂良復薦之,召為太學正。



 遷博士,因輪對,奏曰:「人臣之義,當為陛下建明者,一大事而已。二陵之仇未報,故疆之半未復,而言者以為當乘其機,當待其時。然機自我發,何彼之乘?時自我為,何彼之待?非真難真不可也,正以我自為難,自為不可耳。於是力屈氣索,甘為退伏者,於此二十六年。積今之所謂難者陰沮之,所謂不可者默制之也。蓋其難有四,其不可有五。置不共戴天之仇而廣兼愛之義,自為虛
 弱,此國是之難一也。國之所是既然,士大夫之論亦然。為奇謀秘畫者止於乘機待時,忠義決策者止於親征遷都,深沉慮遠者止於固本自治,此議論之難二也。環視諸臣,迭進迭退,其知此事本而可以反覆論議者誰乎?抱此志意而可以策勵期望者誰乎?此人才之難三也。論者徒鑒五代之致亂,而不思靖康之得禍。今循守舊模,而欲驅一世之人以報君仇,則形勢乖阻,誠無展足之地。若順時增損,則其所更張動搖,關系至重,此法
 度之難四也。又有甚不可者,兵以多而至於弱,財以多而至於乏,不信官而信吏,不任人而任法,不用賢能而用資格:此五者,舉天下以為不可動,豈非今之實患歟!沿習牽制,非一時矣。講利害,明虛實,斷是非,決廢置,在陛下所為耳。」讀未竟,帝蹙額曰:「朕比苦目疾,此志已泯,誰克任此,惟與卿言之耳。」及再讀,帝慘然久之。



 除太常博士兼實錄院檢討官。嘗薦陳傅良等三十四人於丞相,後皆召用,時稱得人。會朱熹除兵部郎官,未就職,為
 侍郎林慄所劾。適上疏爭曰:「慄劾熹罪無一實者,特發其私意而遂忘其欺矣!至於其中『謂之道學』一語,利害所系不獨熹。蓋自昔小人殘害忠良,率有指名,或以為好名,或以為立異,或以為植黨。近創為『道學』之目,鄭丙倡之,陳賈和之,居要津者密相付授,見士大夫有稍慕潔修者,輒以道學之名歸之,以為善為玷闕,以好學為己愆,相與指目,使不得進。於是賢士惴慄,中材解體,銷聲滅影,穢德垢行,以避此名。慄為侍從,無以達陛下之
 德意志慮,而更襲用鄭丙、陳賈密相付授之說,以道學為大罪,文致語言,逐去一熹,自此善良受禍,何所不有!伏望摧折暴橫,以扶善類。」疏入,不報。



 光宗嗣位,由秘書郎出知蘄州。入為尚書左選郎官。是時,帝以疾不朝重華宮者七月,事無鉅細,皆廢不行。適見上力言:「父子親愛出於自然。浮疑私畏,似是而非,豈有事實?若因是而定省廢於上,號令愆於下,人情離阻,其能久乎!」既而帝兩詣重華宮,都人歡悅。適復奏:「自今宜於過宮之日,令
 宰執、侍從先詣起居。異時兩宮聖意有難言者,自可因此傳致,則責任有歸。不可復近習小人增損語言,以生疑惑。」不報。而事復浸異,中外洶洶。



 及孝宗不豫,群臣至號泣攀裾以請,帝竟不往。適責宰相留正曰:「上有疾明甚。父子相見,當俟疾瘳。公不播告,使臣下輕議君父,可乎?」未幾,孝宗崩,光宗不能執喪。軍士籍籍有語,變且不測。適又告正曰:「上疾而不執喪,將何辭以謝天下?今嘉王長,若預建參決,則疑謗釋矣。」宰執用其言,同入奏
 立嘉王為皇太子,帝許之。俄得御批,有「歷事歲久,念欲退閑」之語,正懼而去,人心愈搖。知樞密院趙汝愚憂危不知所出,適告知閣門事蔡必勝曰:「國事至此,子為近臣,庸坐視乎?」蔡許諾,與宣贊舍人傅昌朝、知內侍省關禮、知閣門事韓侂胄三人定計。侂胄,太皇太后甥也。會慈福宮提點張宗尹過侂胄,侂胄覘其意以告必勝。適得之,即亟白汝愚。汝愚請必勝議事,遂遣侂胄因張宗尹、關禮以內禪議奏太皇太后,且請垂簾,許之,計遂定。
 翌日禫祭,太皇太后臨朝,嘉王即皇帝位,親行祭禮,百官班賀,中外晏然。凡表奏皆汝愚與適裁定,臨期,取以授儀曹郎,人始知其預議焉。遷國子司業。



 汝愚既相,賞功將及適,適曰:「國危效忠,職也。適何功之有?」而侂胄恃功,以遷秩不滿望怨汝愚。適以告汝愚曰:「侂胄所望不過節鉞,宜與之。」汝愚不從。適嘆曰:「禍自此始矣!」遂力求補外。除太府卿、總領淮東軍馬錢糧。及汝愚貶衡陽,而適亦為御史胡紘所劾,降兩官罷,主管沖祐觀,差知衢
 州,辭。



 起為湖南轉運判官,遷知泉州。召入對,言於寧宗曰:「陛下初嗣大寶,臣嘗申繹《卷阿》之義為獻。天啟聖明,銷磨黨偏,人才庶幾復合。然治國以和為體,處事以平為極。臣欲人臣忘己體國,息心既往,圖報方來可也。」帝嘉納之。初,韓侂胄用事,患人不附,一時小人在言路者,創為「偽學」之名,舉海內知名士貶竄殆盡。其後侂胄亦悔,故適奏及之,且薦樓鑰、丘崈、黃度三人,悉與郡。自是禁網漸解矣。



 除權兵部侍郎,以父憂去。服除,召至。時有
 勸侂胄立蓋世功以固位者,侂胄然之,將啟兵端。適因奏曰:「甘弱而幸安者衰,改弱而就強者興。陛下申命大臣,先慮預算,思報積恥,規恢祖業,蓋欲改弱以就強矣。竊謂必先審知強弱之勢而定其論,論定然後修實政,行實德,弱可變而為強,非有難也。今欲改弱以就強,為問罪驟興之舉,此至大至重事也。故必備成而後動,守定而後戰。今或謂金已衰弱,姑開先釁,不懼後艱,求宣和之所不能,為紹興之所不敢,此至險至危事也。且所
 謂實政者,當經營瀕淮沿漢諸郡,各為處所,牢實自守。敵兵至則阻於堅城,彼此策應,而後進取之計可言。至於四處御前大軍,練之使足以制敵,小大之臣,試之使足以立事,皆實政也。所謂實德者,當今賦稅雖重而國愈貧,如和買、折帛之類,民間至有用田租一半以上輸納者。況欲規恢,宜有恩澤。乞詔有司審度何名之賦害民最甚,何等橫費裁節宜先。減所入之額,定所出之費。既修實政於上,又行實德於下。此其所以能屢戰而不
 屈,必勝而無敗也。」



 除權工部侍郎。侂胄欲藉其草詔以動中外,改權吏部侍郎兼直學士院,以疾力辭兼職。會詔諸將四路出師,適又告侂胄宜先防江,不聽。未幾,諸軍皆敗,侂胄懼,以丘崈為江、淮宣撫使,除適寶謨閣待制、知建康府兼沿江制置使。適謂三國孫氏嘗以江北守江,自南唐以來始失之,建炎、紹興未暇尋繹。乃請於朝,乞節制江北諸州。



 及金兵大入,一日,有二騎舉旗若將渡者,淮民倉皇爭斫舟纜,覆溺者眾,建康震動。適謂
 人心一搖,不可復制,惟劫砦南人所長,乃募市井悍少並帳下願行者,得二百人,使採石將徐緯統以往。夜過半,遇金人,蔽茅葦中射之,應弦而倒。矢盡,揮刀以前,金人皆錯愕不進。黎明,知我軍寡來追,則已在舟中矣。復命石跋、定山之人劫敵營,得其俘馘以歸。金解和州圍,退屯瓜步,城中始安。又遣石斌賢渡宣化,夏侯成等分道而往,所向皆捷。金自滁州遁去。時羽檄旁午,而適治事如平時,軍須皆從官給,民以不擾。淮民渡江有舟,次
 止有寺,給錢餉米,其來如歸。兵退,進寶文閣待制、兼江、淮制置使,措置屯田,遂上堡塢之議。



 初,淮民被兵驚散,日不自保。適遂於墟落數十里內,依山水險要為堡塢,使復業以守,春夏散耕,秋冬入堡,凡四十七處。又度沿江地創三大堡:石跋則屏蔽採石,定山則屏蔽靖安,瓜步則屏蔽東陽、下蜀。西護歷陽,或連儀真,緩急應援,首尾聯絡,東西三百里,南北三四十里。每堡以二千家為率,教之習射。無事則戍,以五百人一將。有警則增募新
 兵及抽摘諸州禁軍二千人,並堡塢內居民,通為四千五百人,共相守戍。而制司於每歲防秋,別募死士千人,以為劫砦焚糧之用。因言堡塢之成有四利,大要謂:「敵在北岸,共長江之險,而我有堡塢以為聲援,則敵不敢窺江,而士氣自倍,戰艦亦可以策勛。和、滁、真、六合等城或有退遁,我以堡塢全力助其襲逐,或邀其前,或尾其後,制勝必矣。此所謂用力寡而收功博也。」三堡就,流民漸歸。而侂胄適誅,中丞雷孝友劾適附侂胄用兵,遂奪
 職。自後奉祠者凡十三年,至寶文閣學士、通議大夫。嘉定十六年,卒,年七十四。贈光祿大夫,謚文定。



 適志意慷慨,雅以經濟自負。方侂胄之欲開兵端也,以適每有大仇未復之言重之。而適自召還,每奏疏必言當審而後發,且力辭草詔。第出師之時,適能極力諫止,曉以利害禍福,則侂胄必不妄為,可免南北生靈之禍。議者不能不為之嘆息焉。



 戴溪,字肖望,永嘉人也。少有文名。淳熙五年,為別頭省
 試第一。監潭州南嶽廟。紹熙初,主管吏部架閣文字,除太學錄兼實錄院檢討官。正錄兼史職自溪始。升博士,奏兩淮當立農官,若漢稻田使者,括閑田,諭民主出財,客出力,主客均利,以為救農之策。除慶元府通判,未行,改宗正簿。累官兵部郎官。



 開禧時,師潰於符離,溪因奏沿邊忠義人、湖南北鹽商皆當區畫,以銷後患。會和議成,知樞密院事張巖督師京口,除授參議軍事。數月,召為資善堂說書。



 由禮部郎中凡六轉為太子詹事兼秘
 書監。景獻太子命溪講《中庸》、《大學》,溪辭以講讀非詹事職,懼侵官。太子曰:「講退便服說書,非公禮,毋嫌也。」復命類《易》、《詩》、《書》、《春秋》、《論語》、《孟子》、《資治通鑒》,各為說以進。權工部尚書,除華文閣學士。嘉定八年,以宣奉大夫、龍圖閣學士致仕。卒,贈特進、端明殿學士。理宗紹定間,賜謚文端。



 溪久於宮僚,以微婉受知春官,然立朝建明,多務秘密,或議其殊乏骨鯁云。



 蔡幼學,字行之,溫州瑞安人。年十八,試禮部第一。是
 時,陳傅良有文名於太學,幼學從之游。月書上祭酒芮燁及呂祖謙,連選拔,輒出傅良右,皆謂幼學之文過其師。孝宗聞之,因策士將置首列。而是時外戚張說用事,宰相虞允文、梁克家皆陰附之。幼學對策,其略曰:「陛下資雖聰明而所存未大,志雖高遠而所趨未正,治雖精勤而大原不立。即位之始,冀太平旦暮至。奈何今十年,風俗日壞,將難扶持;紀綱日亂,將難整齊;人心益搖,將難收拾;吏慢兵驕,財匱民困,將難正救。」又曰:「陛下恥名相
 之不正,更制近古,二相並進,以為美談。然或以虛譽惑聽,自許立功;或以緘默容身,不能持正。」蓋指虞允文、梁克家也。又曰:「漢武帝用兵以來,大司馬、大將軍之權重而丞相輕。公孫弘為相,衛青用事,弘茍合取容,相業無有。宣、元用許、史,成帝用王氏,哀帝用丁、傅,率為元始之禍。今陛下使姨子預兵柄,其人無一才可取。宰相忍與同列,曾不羞恥。按其罪名,宜在公孫弘上。」蓋指張說也。帝覽之不懌,虞允文尤惡之。遂得下第,教授廣德軍。



 丁
 父憂,再調潭州。執政薦於朝,帝許之,且問:「年幾何矣?何以名幼學?」參政施師點舉《孟子》「幼學壯行」之語以對。上佇思,慨然曰:「今壯矣,可行也。」遂除敕令所刪定官。首言:「大恥未雪,境土未復,陛下睿知神武,可以有為。而茍且之議,委靡之習,顧得以緩陛下欲為之心。」孝宗喜曰:「解卿意,欲令朕立規模爾。」尋以母憂去。



 光宗立,以太學錄召,改武學博士。逾年,遷太學,擢秘書省正字兼實錄院檢討官,遷校書郎。時光宗以疾不朝重華宮,幼學上封
 事曰:「陛下自春以來,北宮之朝不講。比者壽皇愆豫,侍從、臺諫叩陛請對,陛下拂衣而起,相臣引裾,群臣隨以號泣。陛下退朝,宮門盡閉,大臣累日不獲一對清光。望日之朝,都人延頸,遷延至午,禁衛飲恨。市廛軍伍,謗誹籍籍,旁郡列屯,傳聞疑怪,變起倉卒,陛下實受其禍。誠思身體發膚壽皇所與,宗社人民壽皇所命,則疇昔慈愛有感乎心,可不獨出聖斷,復父子之歡,弭宗社之禍!」疏入,不報。



 寧宗即位,詔求直言。幼學又奏:「陛下欲盡為
 君之道,其要有三:事親、任賢、寬民,而其本莫先於講學。比年小人謀傾君子,為安靖和平之說以排之。故大臣當興治而以生事自疑,近臣當效忠而以忤旨擯棄,其極至於九重深拱而群臣盡廢,多士盈庭而一籌不吐。自非聖學日新,求賢如不及,何以作天下之才!自熙寧、元豐而始有免役錢,有常平積剩錢,有無額上供錢;自大觀、宣和而始有大禮進奉銀絹,有贍學糴本錢,有經制錢;自紹興而始有和買折帛錢,有總制錢,有月樁大
 軍錢;至於茶鹽酒榷、稅契、頭子之屬,積累增多,較之祖宗無慮數十倍,民困極矣。」



 幼學既論列時政,其極歸之聖學。帝稱善,將進用之。時韓侂胄方用事,指正人為「偽學」,異論者立黜。幼學遂力求外補,特除提舉福建常平。陛辭,言:「今除授命令徑從中出,而大臣之責始輕;諫省、經筵無故罷黜,而多士之心始惑。或者有以誤陛下至此耶!」侂胄聞之不悅。既至官,日講荒政。時朱熹居建陽,幼學每事咨訪,遂為御史劉德秀劾罷,奉祠者凡八年。



 起知黃州,改提點福建路刑獄,未行。有勸侂胄以收召海內名士者,乃召幼學為吏部員外郎。入見,言:「高宗建炎間減婺州和買絹折羅事,因諭輔臣曰:『一日行得如此一事,一年不過三百六十事而已。』陛下除兩浙丁錢,視高宗無間,然而兵事既開,諸路罹鋒鏑轉餉之艱,江、湖以南有調募科需之擾,惟陛下以愛惜邦本為念。」遷國子司業、宗正少卿,皆兼權中書舍人。



 侂胄既誅,餘黨尚塞正路,幼學次第彈繳,竄黜尤眾,號稱職。遷中書舍
 人兼侍講。故事,閣門、宣贊而下,供職十年,始得路都監若鈐轄。侂胄壞成法,率五六年七八年即越等除授,有已授外職猶通籍禁闥者,幼學一切厘正。



 嘉定初,同樓鑰知貢舉。時正學久錮,士專於聲律度數,其學支離。幼學始取義理之文,士習漸復於正。兼直學士院,內外制皆溫醇雅厚得體,人多稱之。除刑部侍郎,改吏部,仍兼職。趙師UA除知臨安府,UA辭。故事,當有不允詔。幼學言:「師UA以媚權臣進官,三尹京兆,狼籍無善狀,詔必出褒
 語,臣何辭以草?」命遂寢。改兼侍讀,師UA命乃下。



 除龍圖閣待制、知泉州,徙建康府、福州,進福建路安撫使。政主寬大,惟恐傷民。福建下州,例抑民買鹽,以戶產高下均賣者曰產鹽,以交易契紙錢科敷者曰浮鹽,皆出常賦外,久之遂為定賦。幼學力請蠲之,不報。提舉司令民以田高下藏新會子,不如令者籍其貲。幼學曰:「罔民而可,吾忍之乎!惟有去而已。」因言錢幣未均,秤提無術,力求罷去。遂升寶謨閣直學士、提舉萬壽宮。召權兵部尚書
 兼修玉牒官,尋兼太子詹事。



 先是,朝廷既遣歲幣入金境,適值其有難,不果納,則遽以兵叩邊索之。中外洶洶,皆言當亟與。幼學請對,言:「玉帛之使未還,而侵軼之師奄至,且肆其侮慢,形之文辭。天怒人憤,可不伸大義以破其謀乎!」於是朝論奮然,始詔與金絕。幼學因請「固本根以弭外虞,示意向以定眾志,公汲引以合材謀,審懷附以一南北。」帝稱善。一夕感異夢,星隕於屋西南隅,遂卒,年六十四。



 幼學早以文鳴於時,而中年述作,益窮根
 本,非關教化之大、由情性之正者不道也。器質凝重,莫窺其際,終日危坐,一語不妄發。及辨論義理,縱橫闔闢,沛然如決江河,雖辯士不及也。嘗續司馬光《公卿百官表》,《年歷》、《大事記》、《備忘》、《辨疑》、《編年政要》、《列傳舉要》,凡百餘篇,傳於世。



 楊泰之,字叔正,眉州青神人。少刻志於學,臥不設榻幾十歲。慶元元年類試,調滬川尉,易什邡,再調綿州學教授、羅江丞,制置司檄置幕府。吳獵諭蜀,泰之貽書曰:「使
 吳曦為亂,而士大夫不從,必有不敢為;既亂,而士大夫能抗,曦猶有所憚。夫亂,曦之為也;亂所以成,士大夫之為也。」



 改知嚴道縣,攝通判嘉定。白厓砦將王塤引蠻寇利店,刑獄使者置塤於法,又罥絓餘人當坐死。泰之訪知夷都實邇利店,夷都蠻稱亂,不需引導,固請釋之,不聽。乃去官。宣撫使安丙薦之曰:「蜀中名儒楊虞仲之子,當逆臣之變,勉有位者毋動。言不用,拂衣而去。使得尺寸之柄,必能見危致命。」召泰之赴都堂審察,以親老辭。
 差知廣安軍,未上,丁父憂。免喪,知富順監。去官,以祿稟數千緡予鄰里,以千緡為義莊。知普州,以安居、安岳二縣受禍尤慘,泰之力白丙盡蠲其賦。丙復薦於朝,召赴行在,固辭。知果州。踦零錢病民,泰之以一年經費儲其贏為諸邑對減,上尚書省,按為定式。民歌之曰:「前張後楊,惠我無疆。」張謂張義,實自發其端,而泰之踵行之。



 理宗即位,趣入對,言:「法天行健,奮發英斷,總攬威權,無牽於私意,無奪於邪說,以救蠱敝,以新治功。本朝德澤,邇來
 斫喪無餘,民無恆心,何以為國?陛下以直言求人,而以直言罪之,使天下以言為戒。臣恐言路既梗,士氣益消,循循默默,浸成衰世之風,為國者何便於此?」上奇其對,以為工部郎中。其後言事者相繼,無所避忌,自泰之發之。遷軍器少監、大理少卿。



 紹定元年入對,謂:「風雨為暴,水潦潰溢,此陰盛陽微之證。而臺臣諉曰霅川水患之慘,桀之餘烈也。」後又言:「巴陵追降之命,重於違群臣,輕於絕友愛。陛下居天位之至逸,則當思天倫之大痛。秦邸
 歿於房陵,既行封謚,又錄用其子。今乃曰『不當為之後,以貽它日憂』,何示人之不廣乎?」又曰:「今日不言,後必有言之者。與其追恤於後,固不若舉行於今也。」是日,詔直寶謨閣、知重慶府。為書以別丞相曰:「宰相職事,無大於用人有道,去自私之心,恢容人之度,審取舍之理而已。」至官,俗用大變。主管千秋鴻禧觀,卒。



 所著《克齋文集》、《論語解》、《老子解》、《春秋列國事目》、《公羊》、《穀梁類》、《詩類》、《詩名物編》、《論》、《孟類》、《東漢三國志南北史唐五代史類》、《歷代通鑒
 本朝長編類》、《東漢名物編》、《詩事類》、《大易要言》、雜著,凡二百九十七卷。



\end{pinyinscope}