\article{列傳第四十三}

\begin{pinyinscope}

 陳
 堯佐兄堯叟弟堯咨從子漸宋庠弟祁



 陳堯佐,字希元,其先河朔人。高祖翔,為蜀新井令,因家焉,遂為閬州閬中人。父省華字善則,事孟昶為西水尉。蜀平,授隴城主簿,累遷櫟陽令。縣之鄭白渠為鄰邑強
 族所據,省華盡去壅遏,水利均及,民皆賴之,徙樓煩令。端拱三年,太宗親試進士,伯子堯叟登甲科,占謝,辭氣明辨,太宗顧左右曰:「此誰子?」王沔以省華對。即召省華為太子中允,俄判三司都憑由司,改鹽鐵判官,遷殿中丞。河決鄆州,命省華領州事。俄為京東轉運使,超拜祠部員外郎、知蘇州,賜金紫。時遇水災,省華復流民數千戶,殍者悉瘞之,詔書褒美。歷戶部、吏部二員外郎,改知潭州。省華智辨有吏乾,入掌左藏庫,判吏部南曹,擢鴻
 臚少卿。景德初,判吏部銓,權知開封府,轉光祿卿。舊制,卿監坐朵殿,太宗以省華權蒞京府,別設其位,升於兩省五品之南。省華以府事繁劇,請禁賓友相過,從之。未幾,因疾求解任,拜左諫議大夫,再表乞骸骨,不許,手詔存問,親閱方藥賜之。三年,卒,年六十八,特贈太子少師。



 堯佐進士及第,歷魏縣、中牟尉,為《海喻》一篇,人奇其志。以試秘書省校書郎知朝邑縣,會其兄堯叟使陜西,發中人方保吉罪,保吉怨之,誣堯佐以事,降本縣主簿。徙
 下邽,遷秘書郎、知真源縣,開封府司錄參軍事,遷府推官。坐言事忤旨,降通判潮州。修孔子廟,作韓吏部祠,以風示潮人。民張氏子與其母濯於江,鱷魚尾而食之,母弗能救。堯佐聞而傷之,命二吏拏小舟操網往捕。鱷至暴,非可網得,至是,鱷弭受網,作文示諸市而烹之,人皆驚異。



 召還,直史館、知壽州。歲大饑,出奉米為糜粥食餓者,吏人悉獻米至,振數萬人。徙廬州,以父疾請歸,提點開封府界事,後為兩浙轉運副使。錢塘江篝石為堤,堤
 再歲輒壞。堯佐請下薪實土乃堅久,丁謂不以為是,徙京西轉運使,後卒如堯佐議。徙河東路,以地寒民貧,仰石炭以生,奏除其稅。又減澤州大廣冶鐵課數十萬。徙河北,母老祈就養,召糾察在京刑獄,為御試編排官,坐置等誤降官,監鄂州茶場。



 天禧中,河決,起知滑州,造木龍以殺水怒,又築長堤,人呼為「陳公堤」。初營永定陵,復徙京西轉運使,入為三司戶部副使,徙度支,同修《真宗實錄》。不試中書,特擢知制誥兼史館修撰,知通進、銀臺
 司。進樞密密直學士、知河南府,徙並州。每汾水暴漲,州民輒憂擾,堯佐為築堤,植柳數萬本,作柳溪,民賴其利。



 召同修《三朝史》,代弟堯咨同知開封府,累遷右諫議大夫,為翰林學士,遂拜樞密副使。祥符知縣陳詁治嚴急,吏欲罪詁,乃空縣逃去,太后果怒。而詁連呂夷簡親,執政以嫌不敢辨。事下樞密院,堯佐獨曰:「罪詁則奸吏得計,後誰敢復繩吏者?」詁由是得免。以給事中參知政事,遷尚書吏部侍郎。



 太后崩,執政多罷,以戶部侍郎知永興
 軍。過鄭,為郡人王文吉以變事告,下御史中丞範諷劾治,而事乃辨。改知廬州,徙同州,復徙永興軍。初,太后遣宦者起浮圖京兆城中,前守姜遵盡毀古碑碣充磚甓用,堯佐奏曰:「唐賢臣墓石,今十亡七八矣。子孫深刻大書,欲傳之千載,乃一旦與瓦礫等,誠可惜也。其未毀者,願敕州縣完護之。」徙鄭州。會作章惠太后園陵,州供張甚嚴,賜書褒諭。既而拜同中書門下平章事、集賢殿大學士。以災異數見,罷為淮康軍節度使、同中書門下平
 章事、判鄭州。以太子太師致仕,卒,贈司空兼侍中,謚文惠。



 堯佐少好學,父授諸子經,其兄未卒業,堯佐竊聽已成誦。初肄業錦屏山,後從種放於終南山,及貴,讀書不輟。善古隸八分,為方丈字,筆力端勁,老猶不衰。尤工詩。性儉約,見動物,必戒左右勿殺,器服壞,隨輒補之,曰:「無使不全見棄也。」號「知餘子」。自志其墓曰:「壽八十二不為夭,官一品不為賤,使相納祿不為辱,三者粗可歸息於父母棲神之域矣。」陳摶嘗謂其父曰:「君三子皆當將相,
 惟中子貴且壽。」後如摶言。有《集》三十卷,又有《潮陽編》、《野廬編》、《愚丘集》、《遣興集》。



 堯叟字唐夫,解褐光錄寺丞、直史館,與省華同日賜緋,遷秘書丞。久之,充三司河南東道判官。時宋、亳、陳、穎民饑,命堯叟及趙況等分振之。再遷工部員外郎、廣南西路轉運使。嶺南風俗,病者禱神不服藥,堯叟有《集驗方》,刻石桂州驛。又以地氣蒸暑,為植樹鑿井,每三二十里置亭舍,具飲器,人免暍死。會加恩黎桓,為交州國信使。
 初,將命者必獲贈遺數千緡,桓責賦斂於民,往往斷其手及足趾。堯叟知之,遂奏召桓子,授以朝命,而卻其私覿。又桓界先有亡命來奔者,多匿不遣,因是海賊頻年入寇。堯叟悉捕亡命歸桓,桓感恩,並捕海賊為謝。



 先是,歲調雷、化、高、藤、容、白諸州兵,使輦軍糧泛海給瓊州。其兵不習水利,率多沉溺,咸苦之。海北岸有遞角場,正與瓊對,伺風便一日可達,與雷、化、高、太平四州地水路接近。堯叟因規度移四州民租米輸於場,第令瓊州遣蜑
 兵具舟自取,人以為便。



 咸平初,詔諸路課民種桑棗,堯叟上言曰:「臣所部諸州,土風本異,田多山石,地少桑蠶。昔雲八蠶之綿,諒非五嶺之俗,度其所產,恐在安南。今其民除耕水田外,地利之博者惟麻苧爾。麻苧所種,與桑柘不殊,既成宿根,旋擢新幹,俟枝葉裁茂則刈獲之,周歲之間,三收其苧。復一固其本,十年不衰。始離田疇,即可紡績。然布之出,每端止售百錢,蓋織者眾、市者少,故地有遺利,民艱資金。臣以國家軍須所急,布帛為先,
 因勸諭部民廣植麻苧,以錢鹽折變收市之,未及二年,已得三十七萬餘匹。自朝廷克平交、廣,布帛之供,歲止及萬,較今所得,何止十倍。今樹藝之民,相率競勸;杼軸之功,日以滋廣。欲望自今許以所種麻苧頃畝,折桑棗之數,諸縣令佐依例書歷為課,民以布赴官賣者,免其算稅。如此則布帛上供,泉貨下流,公私交濟,其利甚博。」詔從之。代還,加刑部員外郎,充度支判官。



 未幾,會撫水蠻酋蒙令國殺使臣擾動,命堯叟為廣南東、西兩路安
 撫使,賜金紫遣之。事平,遷兵部,拜主客郎中、樞密直學士、知三班兼銀臺通進封駁司、制置群牧使。



 河決澶州王陵口,詔往護塞之,遂與馮拯同為河北、河東安撫副使。時中外上封奏者甚眾,命與拯詳定利害,及與三司議減冗事。俄與拯並拜右諫議大夫、同知樞密院事。有言三司官吏積習依違,文牒有經五七歲不決者,吏民抑塞,水旱災沴,多由此致。請委逐部判官檢覆判決,如復稽滯,許本路轉運使聞奏,命官推鞫,以警弛慢。乃詔
 堯叟與拯舉常參官干敏者,同三司使議減煩冗,參決滯務。堯叟請以秘書丞直史館孫冕同領其事,凡省去煩冗文帳二十一萬五千餘道,又減河北冗官七十五員。



 五年,郊祀,進給事中。會王繼英為樞密使,以堯叟簽署院事,奉秩恩例悉同副使,遷工部侍郎。真宗幸澶淵,命乘傳先赴北砦按視戎事,許以便宜。景德中,遷刑部、兵部二侍郎,與王欽若並知樞密院事。真宗朝陵,權東京留守。每裁剸刑禁,雖大闢亦止面取狀,亟決遣之,以
 故獄無系囚。真宗曰:「堯叟素有裁斷,然重事宜付有司按鞫而詳察之。」因密加詔諭。俄兼群牧制置使。始置使,即以堯叟為之,及掌樞密,即罷其任。至是,以國馬戎事之本,宜得大臣總領,故又委堯叟焉。自是多立條約。又著《監牧議》,述馬政之重。預修國史。



 大中祥符初,東封,加尚書左丞。詔撰《朝覲壇碑》,進工部尚書,獻《封禪聖制頌》,帝作歌答之。祀汾陰,為經度制置使、判河中府。禮成,進戶部尚書。時詔王欽若為《朝覲壇頌》,表讓堯叟,不許。別
 命堯叟撰《親謁太寧廟頌》,加特進,賜功臣。又以堯叟善草隸,詔寫途中禦制歌詩刻石。



 五年,與欽若並以本官檢校太傅、同平章事,充樞密使,加檢校太尉。從幸太清宮,加開府儀同三司。未幾,與欽若罷守本官,仍領群牧。明年,復與欽若以本官檢校太尉、同平章事,充樞密使。堯叟素有足疾,屢請告。九年夏,帝臨問,勞賜加等。疾甚,表求避位,遣閣門使楊崇勛至第撫慰,以詢其意。堯叟詞志頗確,優拜右僕射、知河陽。肩輿入辭,至便坐,許三
 子扶掖升殿,賜詩為餞,又賜仲子希古緋服。



 天禧初,病亟,召其子執筆,口占奏章,求還輦下,詔許之。肩輿至京師,卒,年五十七。廢朝二日,贈侍中,謚曰文忠,錄其孫知言、知章為將作監主簿。長子師古賜進士出身,後為都官員外郎。希古至太子中舍,坐事除籍。



 堯叟偉姿貌,強力,奏對明辨,多任知數。久典機密,軍馬之籍,悉能周記。所著《請盟錄》三集二十卷。



 母馮氏,性嚴。堯叟事親孝謹,怡聲侍側,不敢以貴自處。家本富,祿賜且厚,馮氏不許
 諸子事華侈。景德中,堯叟掌樞機,弟堯佐直史館,堯咨知制誥,與省華同在北省,諸孫任官者十數人,宗親登科者又數人,榮盛無比。賓客至,堯叟兄弟侍立省華側,客不自安,多引去。舊制登樞近者,母妻即封郡夫人。堯叟以父在朝,母止從父封,遂以妻封表讓於母,朝廷援制不許。父既卒,帝欲褒封其母,以問王旦。旦曰:「雖私門禮制未闕,公朝降命亦無嫌也。」乃封上黨郡太夫人,進封滕國,年八十餘無恙,後堯叟數年卒。



 堯咨字嘉謨,舉進士第一,授將作監丞、通判濟州,召為秘書省著作郎、直史館、判三司度支勾院,始合三部勾院兼總之。擢右正言、知制誥。崇政殿試進士,堯咨為考官,三司使劉師道屬弟幾道以試卷為識驗,坐貶單州團練副使。復著作郎、知光州。尋復右正言、知制誥,知荊南。改起居舍人,同判吏部流內銓。舊格,選人用舉者數遷官,而寒士無以進,堯咨進其可擢者,帝特遷之。改右諫議大夫、集賢院學士,以龍圖閣直學士、尚書工部郎
 中知永興軍。長安地斥鹵,無甘泉,堯咨疏龍首渠注城中,民利之。然豪侈不循法度,敞武庫,建視草堂,開三門,築甬道,出入列禁兵自衛。用刑慘急,數有仗死者。嘗以氣凌轉運使樂黃目,黃目不能堪,求解去,遂徙堯咨知河南府。既而有發堯咨守長安不法者,帝不欲窮治,止削職徙鄧州,才數月,復知制誥。



 堯咨性剛戾,數被挫,忽忽不自樂。堯叟進見,帝問之,對曰:「堯咨豈知上恩所以保祐者,自謂遭讒以至此爾!」帝賜詔條其事切責,乃皇
 恐稱謝。還,判登聞檢院,復龍圖閣直學士。坐失舉,降兵部員外郎。喪母,起復工部郎中、龍圖閣直學士、會靈觀副使。邊臣飛奏唃廝囉立文法召蕃部欲侵邊,以為陜西緣邊安撫使。再遷右諫議大夫、知秦州,徙同州,以尚書工部侍郎權知開封府。入為翰林學士,以先朝初榜甲科,特詔班舊學士蔡齊之上。



 換宿州觀察使、知天雄軍,位丞郎上。堯咨內不平,上章固辭,皇太后特以只日召見,敦諭之,不得已,拜命。自契丹修好,城壁器械久不
 治,堯咨葺完之。然須索煩擾,多暴怒,列軍士持大梃侍前,吏民語不中意,立至困僕。以安國軍節度觀察留後知鄆州。建請浚新河,自魚山至下杷以導積水。拜武信軍節度使、知河陽,徙澶州,又徙天雄軍。所居棟摧,大星霣于庭,散為白氣。已而卒,贈太尉,謚曰康肅。



 堯咨於兄弟中最為少文,然以氣節自任。工隸書。善射,嘗以錢為的,一發貫其中。兄弟同時貴顯,時推為盛族。子述古,太子賓客致仕;博古,篤學能文,為館閣校勘,早卒。



 從子漸字鴻漸,少以文學知名於蜀。淳化中,與其父堯封皆以進士試廷中,太宗擢漸第,輒辭不就,願擢其父,許之。至咸平初,漸始仕,為天水縣尉。時學者罕通揚雄《太玄經》,漸獨好之,著書十五篇,號《演玄》,奏之。召試學士院,授儀州軍事推官。舉賢良方正科,不中,復調隴西防禦推官,坐法免歸,不復有仕進意,蜀中學者多從之游。堯咨不學,漸心薄之。堯咨後貴顯,與漸益不同,因言漸罪戾之人,聚徒太盛,不宜久留遠方。即召漸至京師,授
 穎州長史。丁謂等知其無他,得改鳳州團練推官,遷耀州節度推官。卒,有文集十五卷,自號金龜子。



 宋庠,字公序,安州安陸人,後徙開封之雍丘。父杞,嘗為九江掾,與其妻鐘禱於廬阜。鐘夢道士授以書曰:「以遺爾子。」視之,《小戴禮》也,已而庠生。他日見許真君像,即夢中見者。



 庠天聖初舉進士,開封試、禮部皆第一,擢大理評事、同判襄州。召試,遷太子中允、直史館,歷三司戶部判官,同修起居注,再遷左正言。郭皇后廢,庠與御史伏
 閣爭論,坐罰金。久之,知制誥。時親策賢良、茂才等科,而命與武舉人雜視。庠言:「非所以待天下士,宜如本朝故事,命有司設次具飲膳,斥武舉人令別試。」詔從之。



 兼史館修撰、知審刑院。密州豪王澥私釀酒,鄰人往捕之,澥紿奴曰:「盜也。」盡使殺其父子四人。州論奴以法,澥獨不死。宰相陳堯佐右澥,庠力爭,卒抵澥死。改權判吏部流內銓,遷尚書刑部員外郎。仁宗欲以為右諫議大夫、同知樞密院事,中書言故事無自知制誥除執政者,乃詔
 為翰林學士。帝遇庠厚,行且大用矣。



 庠初名郊,李淑恐其先己,以奇中之,言曰:「宋,受命之號;郊,交也。合姓名言之為不祥。」帝弗為意,他日以諭之,因改名庠。寶元中,以右諫議大夫參知政事。庠為相儒雅,練習故事,自執政,遇事輒分別是非。嘗從容論及唐入閣儀,庠退而上奏曰:



 入閣,乃有唐只日於紫宸殿受常朝之儀也。唐有大內,又有大明宮,宮在大內之東北,世謂之東內,高宗以後,天子多在。大明宮之正南門曰丹鳳門,門內第一
 殿曰含元殿,大朝會則御之;第二殿曰宣政殿,謂之正衙,朔望大冊拜則御之;第三殿曰紫宸殿,謂之上閣,亦曰內衙,只日常朝則御之。天子坐朝,須立伏於正衙殿,或乘輿止禦紫宸,即喚仗自宣政殿兩門入,是謂東、西上閣門也。



 以本朝宮殿視之:宣德門,唐丹鳳門也;大慶殿,唐含元殿也;文德殿,唐宣政殿也;紫宸殿,唐紫宸殿也。今欲求入閣本意,施於儀典,須先立仗文德庭,如天子止禦紫宸,即喚仗自東、西閣門入,如此則差與舊儀合。
 但今之諸殿,比於唐制南北不相對爾。又按唐自中葉以還,雙日及非時大臣奏事,別開延英殿,若今假日御崇政、延和是也。乃知唐制每遇坐朝日,即為入閣,其後正衙立仗因而遂廢,甚非禮也。



 庠與宰相呂夷簡論數不同,凡庠與善者,夷簡皆指為朋黨,如鄭戩、葉清臣等悉出之,乃以庠知揚州。未幾,以資政殿學士徙鄆州,進給事中。參知政事範仲淹去位,帝問宰相章得像,誰可代仲淹者,得像薦宋祁。帝雅意在庠,復召為參知政事。
 慶歷七年春旱,用漢災異策免三公故事,罷宰相賈昌朝,輔臣皆削一官,以庠為右諫議大夫。帝嘗召二府對資政殿,出手詔策以時事,庠曰:「兩漢對策,本延巖穴草萊之士,今備位政府而比諸生,非所以尊朝廷,請至中書合議條奏。」時陳執中為相,不學少文,故夏竦為帝畫此謀,意欲困執中也。論者以庠為知體。



 明年,除尚書工部侍郎,充樞密使。皇祐中,拜兵部侍郎、同中書門下平章事、集賢殿大學士。享明堂,遷工部尚書。嘗請復群臣
 家廟,曰:「慶歷元年赦書,許文武官立家廟,而有司終不能推述先典,因循顧望,使王公薦享,下同委巷,衣冠昭穆,雜用家人,緣偷襲弊,甚可嗟也。請下有司論定施行。」而議者不一,卒不果復。



 三年,祁子與越國夫人曹氏客張彥方游。而彥方偽造敕牒,為人補官,論死。諫官包拯奏庠不戢子弟,又言庠在政府無所建明,庠亦請去。乃以刑部尚書、觀文殿大學士知河南府,後徙許州,又徙河陽,再遷兵部尚書。入覲,詔綴中書門下班,出入視其
 儀物。以檢校太尉、同平章事充樞密使,封莒國公。數言:「國家當慎固根本,畿輔宿兵常盈四十萬,羨則出補更戍,祖宗初謀也,不茍輕改。」既而與副使程戡不協,戡罷,而御史言庠昏惰,乃以河陽三城節度、同平章事判鄭州,徙相州。以疾召還。



 英宗即位,移鎮武寧軍,改封鄭國公。庠在相州,即上章請老,至是請猶未已。帝以大臣故,未忍遽從,乃出判亳州。庠前後所至,以慎靜為治,及再登用,遂沉浮自安。晚愛信幼子,多與小人游,不謹。御史呂
 晦請敕庠不得以二子隨,帝曰:「庠老矣,奈何不使其子從之。」至亳,請老益堅,以司空致仕。卒,贈太尉兼侍中,謚元獻。帝為篆其墓碑曰「忠規德範之碑」。



 庠自應舉時,與祁俱以文學名擅天下,儉約不好聲色,讀書至老不倦。善正訛謬,嘗校定《國語》,撰《補音》三卷。又輯《紀年通譜》,區別正閏,為十二卷。《掖垣叢志》三卷,《尊號錄》一卷,別集四十卷。天資忠厚,嘗曰:「逆詐恃明,殘人矜才,吾終身不為也。」沉邈嘗為京東轉運使,數以事侵庠。及庠在洛,邈子
 監曲院,因出借縣人負物,杖之,道死實以他疾。而邈子為府屬所惡,欲痛治之以法,庠獨不肯,曰:「是安足罪也!」人以此益稱其長者。弟祁。



 祁字子京,與兄庠同時舉進士,禮部奏祁第一,庠第三。章獻太后不欲以弟先兄,乃擢庠第一,而置祁第十。人呼曰「二宋」,以大小別之。釋褐復州軍事推官。孫奭薦之,改大理寺丞、國子監直講。召試,授直史館,再遷太常博士、同知禮儀院。有司言太常舊樂數增損,其聲不和。詔
 祁同按試。李照定新樂,胡瑗鑄鐘磬,祁皆典之,事見《樂志》。預修《廣業記》成,遷尚書工部員外郎、同修起居注、權三司度支判官。方陜西用兵,調費日蹙,上疏曰:



 兵以食為本,食以貨為資,聖人一天下之具也。今左藏無積年之鏹,太倉無三歲之粟,尚方冶銅匱而不發。承平如此,已自凋困,良由取之既殫、用之無度也。朝廷大有三冗,小有三費,以困天下之財。財窮用褊,而欲興師遠事,誠無謀矣。能去三冗、節三費,專備西北之屯,可曠然高枕
 矣。



 何謂三冗?天下有定官無限員,一冗也;天下廂軍不任戰而耗衣食,二冗也;僧道日益多而無定數,三冗也。三冗不去,不可為國。請斷自今,僧道已受戒具者姑如舊,其它悉罷還為民,可得耕夫織婦五十餘萬人,一冗去矣。天下廂軍不擇孱小尪弱而悉刺之,才圖供役,本不知兵,又且月支廩糧,歲費庫帛,數口之家,不能自庇,多去而為盜賊,雖廣募之,無益也。其已在籍者請勿論,其它悉驅之南畝,又得力耕者數十萬,二冗去矣。國家
 郡縣,素有定官,譬以十人為額,常以十二加之,即遷代、罪謫,隨取之而有。今一官未闕,群起而逐之,州縣不廣於前,而官五倍於舊,吏何得不茍進,官何得不濫除?請詔三班審官院內諸司、流內銓明立限員,以為定法。其門蔭、流外、貢舉等科,實置選限,稍務擇人,俟有闕官,計員補吏,三冗去矣。



 何謂三費?一曰道場齋醮,無有虛日,且百司供億,至不可貲計。彼皆以祝帝壽、奉先烈、祈民福為名,臣愚以為此主者為欺盜之計爾。陛下事天地、
 宗廟、社稷、百神,犧牲玉帛,使有司端委奉之、歲時薦之,足以竦明德、介多福矣,何必希屑屑之報哉?則一費節矣。二曰京師寺觀,或多設徒卒,添置官府,衣糧率三倍他處。居大屋高廡,不徭不役,坐蠹齊民,其尤者也。而又自募民財,營建祠廟,雖曰不費官帑,然國與民一也,舍國取民,其傷一焉,請罷去之,則二費節矣。三曰使相節度,不隸藩要。夫節相之建,或當邊鎮,或臨師屯,公用之設,勞眾而饗賓也。今大臣罷黜,率叨恩除,坐靡邦用,莫
 此為甚。請自今地非邊要、州無師屯者,不得建節度;已帶節度,不得留近藩及京師,則三費節矣。



 臣又聞之,人不率則不從,身不先則不信。陛下能躬服至儉,風示四方,衣服起居,無逾舊規,後宮錦繡珠玉,不得妄費,則天下響應,民業日豐,人心不搖,師役可舉,風行電照,飲馬西河。蠢爾戎首,在吾掌中矣!



 徙判鹽鐵勾院,同修禮書。次當知制誥,而庠方參知政事,乃以為天章閣待制,判太常禮院、國子監,改判太常寺。庠罷,祁亦出知壽州,徙
 陳州。還,知制誥、權同判流內銓,以龍圖閣直學士知杭州,留為翰林學士。提舉諸司庫務,數厘正弊事,增置勾當公事官,其屬言利害者,皆使先稟度可否,而後議於三司,遂著為令。徙知審官院兼侍讀學士。庠復知政事,罷祁翰林學士,改龍圖學士、史館修撰,修《唐書》。累遷右諫議大夫,充群牧使。庠為樞密使,祁復為翰林學士。



 景祐中,詔求直言,祁奏:「人主不斷是名亂。《春秋》書:『殞霜,不殺菽。』天威暫廢,不能殺小草,猶人主不斷,不能制臣下。」
 又謂:「與賢人謀而與不肖者斷,重選大臣而輕任之,大事不圖而小事急,是謂三患。」其意主於強君威,別邪正,急先務,皆切中時病。



 會進溫成皇后為貴妃。故事,命妃皆發冊,妃辭則罷冊禮。然告在有司,必俟旨而後進。又凡制詞,既授閣門宣讀,學士院受而書之,送中書,結三少銜,官告院用印,乃進內。祁適當制,不俟旨,寫誥不送中書,徑取官告院印用之,亟封以進。後方愛幸,覬行冊禮,得告大怒,擲於地。祁坐是出知許州。甫數月,復召為
 侍讀學士、史館修撰。祀明堂,遷給事中兼龍圖閣學士。坐其子從張彥方游,出知亳州。兼集賢殿修撰。



 歲餘,徙知成德軍,遷尚書禮部侍郎。請弛河東、陜西馬禁,又請復唐馱幕之制。居三月,徙定州,又上言:



 天下根本在河北,河北根本在鎮、定,以其扼賊沖,為國門戶也。且契丹搖尾五十年,狼態猘心,不能無動。今垂涎定、鎮,二軍不戰,則薄深、趙、邢、洺,直搗其虛,血吻婪進,無所顧藉。臣竊慮欲兵之強,莫如多穀與財;欲士訓練,莫如善擇將帥;
 欲人樂鬥,莫如賞重罰嚴;欲賊顧望不敢前,莫如使鎮重而定強。夫恥怯尚勇,好論事,甘得而忘死:河北之人,殆天性然。陛下少勵之,不憂不戰。以欲戰之士,不得善將,雖斗猶負。無穀與財,雖金城湯池,其勢必輕。



 今朝廷擇將練卒,制財積糧,乃以陜西、河東為先,河北為後,非策也。西賊兵銳士寡,不能深入,河東天險,彼憚為寇。若河北不然,自薊直視,勢同建瓴,賊鼓而前,如行莞衽。故謀契丹者當先河北,謀河北者舍鎮、定無議矣。臣願先
 入谷鎮、定,鎮、定既充,可入谷餘州。列將在陜西、河東有功狀者,得遷鎮、定,則鎮、定重。天下久平,馬益少,臣請多用步兵。夫雲奔飆馳,抄後掠前,馬之長也;強弩巨梃,長槍利刀,什伍相聯,大呼薄戰,步之長也。臣料朝廷與敵相攻,必不深入窮追,毆而去之,及境則止,此不特馬而步可用矣。臣請損馬益步,故馬少則騎精,步多則鬥健,我能用步所長,雖契丹多馬,無所用之。



 夫鎮、定一體也,自先帝以來為一道,帥專而兵不分,故定揕其胸,則鎮
 搗其肋,勢自然耳。今判而為二,其顯顯有害者,屯砦山川要險之地裂而有之,平時號令文移不能一,賊脫叩營壘,則彼此不相謀,尚肯任此責邪!請合鎮、定為一路,以將相大臣領之,無事時以鎮為治所,有事則遷治定,指授諸將,權一而責有歸,策之上也。陛下當居安思危,熟計所長,必待事至而後圖之,殆矣。



 河東馬強,士習善馳突,與鎮、定若表裏,然東下井陘,不百里入鎮、定矣。賊若深入,以河東健馬佐鎮、定兵,掩其惰若歸者,萬出萬
 全,此一奇也。臣聞事切於用者,不可以文陳,臣所論件目繁碎,要待刀筆吏委曲可曉,臣已便俗言之,輒別上擇將畜財一封,乞下樞密院、三司裁制之。



 又上《御戎論》七篇。加端明殿學士,特遷吏部侍郎、知益州。尋除三司使。右司諫吳及嘗言祁在定州不治,縱家人貸公使錢數千緡,在蜀奢侈過度。既而御史中丞包拯亦言祁益部多游燕,且其兄方執政,不可任三司。乃加龍圖閣學士、知鄭州。《唐書》成,遷左丞,進工部尚書。以羸疾,請便醫
 藥,入判尚書都省。逾月,拜翰林學士承旨,詔遇入直,許一子主湯藥。復為群牧使,尋卒。遺奏曰:「陛下享國四十年,東宮虛位,天下系望,人心未安。為社稷深計,莫若擇宗室賢材,進爵親王,為匕鬯之主。若六宮有就館之慶,聖嗣蕃衍,則宗子降封郡王,以避正嫡,此定人心、防禍患之大計也。」



 又自為志銘及《治戒》以授其子:「三日斂,三月葬,慎無為流俗陰陽拘忌也。棺用雜木,漆其四會,三塗即止,使數十年足以臘吾骸、朽衣巾而已。毋以金銅
 雜物置塚中。且吾學不名家,文章僅及中人,不足垂後。為吏在良二千石下,勿請謚,勿受贈典。塚上植五株柏,墳高三尺,石翁仲他獸不得用。若等不可違命。若等兄弟十四人,惟二孺兒未仕,以此諉莒公。莒公在,若等不孤矣。」後贈尚書。



 祁兄弟皆以文學顯,而祁尤能文,善議論,然清約莊重不及庠,論者以祁不至公輔,亦以此云。修《唐書》十餘年,自守亳州,出入內外嘗以稿自隨,為列傳百五十卷。預修《籍田記》、《集韻》。又撰《大樂圖》二卷,文集
 百卷。祁所至,治事明峻,好作條教。其子遵《治戒》不請謚,久之,學士承旨張方平言祁法應得謚,謚曰景文。



 論曰:咸平、天聖間,父子兄弟以功名著聞於時者,於陳堯佐、宋庠見之。省華聲聞,由諸子而益著。堯佐相業雖不多見,世以寬厚長者稱之。堯叟出典方州,入為侍從,課布帛,修馬政,減冗官,有足稱者。庠明練故實,文藻雖不逮祁,孤風雅操,過祁遠矣。君子以為陳之家法,宋之友愛,有宋以來不多見也,嗚呼賢哉!



\end{pinyinscope}