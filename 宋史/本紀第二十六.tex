\article{本紀第二十六}

\begin{pinyinscope}

 高宗
 三



 四年春正月甲辰朔,御舟碇海中。乙巳,金人犯明州,張俊及守臣劉洪道擊卻之。丙午,帝次臺州章安鎮。己酉,遣小校自海道如虔州問安太后。庚戌,金人再犯明州,
 張俊引兵去,浙東副總管張思政及劉洪道繼遁。癸丑,貶郭仲荀汝州團練副使、廣州安置。丙辰,詔原兩浙州郡降金官吏。丁巳,婁宿陷陜州,守臣李彥仙死之。己未,金人陷明州夜,大雨震電,乘勝破定海,以舟師來襲御舟,張公裕以大舶擊退之。辛酉,發章安鎮。壬戌,雷雨又作。甲子,泊溫州港口。乙丑,以中書舍人李正民為兩浙、湖南、江西撫諭使,詣太后問安。丁卯,臺州守臣晁公為棄城遁。虔州衛兵及鄉兵相殺,縱火肆掠三日。劉可轉
 寇京西,屢為桑仲所敗,至是為其黨所殺,復推劉超據荊門軍。戊辰,滕康、劉玨罷,仍奪職。己巳,換給僧道度牒,人輸錢十千。辛未,命臣僚條具兵退之後措置之策、駐蹕之所。是月,金人攻楚州,守臣趙立拒之。金人犯邠州,曲端遣涇原路副總管吳玠拒戰,敗之於彭原;又陷同州。張浚遣謝亮使夏國,至則其主乾順已稱制,遂還。



 二月甲戌朔,酈瓊率眾降於劉光世。叛將傅選詣虔州乞降。乙亥,奉安祖宗神御於福州。詔復以盧益為資政殿
 學士,李回端明殿學士,並權知三省、樞密院事。金人陷潭州,將吏王暕、劉價、趙聿之戰死,向子諲率兵奪門亡去,金兵大掠,屠其城。丙子,金人自明州引兵還臨安。癸未,虔州鄉兵首領陳新率眾數萬圍城,叛將胡友亦犯虔州,與新戰,破之,新乃去。甲申,禁逃卒投刺別軍。丙戌,金人自臨安退兵,命劉光世率兵追之。丁亥,金人陷汴京,權留守上官悟出奔,為盜所殺。庚寅,帝次溫州。浙東防遏使傅崧卿入越州。辛卯,金人陷秀州。甲午,知蔡州
 程昌寓棄城南歸。鼎州民鐘相作亂,自稱楚王。乙未,杜充罷。丙申,以金兵退,肆赦。張浚承制以陜西制置使王似知成都府。罷諸路武臣提點刑獄。李成入舒州。金游騎至平江,周望奔太湖,守臣湯東野亦遁。茶陵縣軍賊二千餘人犯郴州永興縣。戊戌,金人入平江,縱兵焚掠。辛丑,白虹貫日。鐘相陷澧州,殺守臣黃宗。權湖北制置使傅雱招諭孔彥舟,彥舟聽命,因以為湖南、北捉殺使。荊南守臣唐愨棄城去。金人陷醴州,守臣王淑棄城去。
 是月,張浚自秦州引兵入援。



 三月癸卯朔,孔彥舟入鼎州。金人去平江,統制陳思恭以舟師邀敗其後軍於太湖。呂頤浩請幸浙西。丙午,趙鼎言金兵去未遠,遂緩其行。丁未,命發運司說諭兩浙富民助米,以備巡幸。辛亥,遣兵部員外郎馮康國等撫諭荊湖南北、廣南諸路。壬子,金人入常州,守臣周杞棄城去。甲寅,遣盧益及御營都統制辛企宗奉迎太后東還。丙辰,金人犯終南縣,經略使鄭恩戰敗,死之。丁巳,金人至鎮江府,韓世忠屯焦
 山寺邀擊之。詔侍從官各舉可充監司者一二人。辛酉,御舟發溫州。宣撫司節制軍馬李允文部兵至鄂州。御營前軍將楊勍叛。甲子,張浚請便宜闢官不許動改。戊辰,孔彥舟擊敗鐘相,禽相及其子子昂,檻送行在。己巳,戚方陷廣德軍,殺權通判王儔。



 夏四月癸酉,蠲江西州縣兵盜賤破民家夏稅。戊寅,吳玠及金人戰於邠州彭原店,敗績,部將楊晟死之。己卯,以觀文殿學士朱勝非為江西、湖南北宣撫使。是日,張浚引兵至房州,知金兵
 退,乃還。癸未,帝駐越州。甲申,下詔親征,巡幸浙西。韓世忠駐軍揚子江,要金人歸路,屢敗之,兀□引軍走建康。乙酉,以御史中丞趙鼎為翰林學士,鼎固辭不拜。戚方圍宣州。劉光世遣統制王德誘誅劉文舜於饒州。丙申,用趙鼎劾奏,呂頤浩罷為鎮南軍節度使、醴泉觀使。命三省、樞密院同班奏事。韓世忠及兀□再戰江中,金人乘風縱火,世忠敗績。兀□渡江,屯六合縣。丁酉,復以趙鼎為御史中丞。戊戌,振明州被兵民家。己亥,以張俊為
 浙西、江東制置使。辛丑,王德破妖賊王宗石於信州貴溪縣,執其渠帥,諸縣悉平。是月,金人犯江西者自荊門軍北歸,留守司同統制牛皋潛軍寶豐擊敗之。



 五月甲辰,以範宗尹為尚書右僕射兼御營使。辛亥,統領赤心隊軍馬劉晏及戚方戰於宣州,敗死。壬子,金人焚建康府,執李梲、陳邦光而去。淮南宣撫司統制岳飛邀擊於靜安鎮,敗之。是夜,紫微垣內有赤雲亙天,白氣貫其中。癸丑,詔臺諫等官各舉所知二人。以張守參知政事、趙
 鼎簽書樞密院事。以白金三萬兩賜韓世忠軍,贈戰歿將孫世詢、嚴永吉、張淵等官。甲寅,金人陷定遠縣,執閭勍去,勍不屈,死之。巨師古擊戚方於宣州,數敗之,方引去。乙卯,王綯罷。丁巳,命劉光世移軍捕戚方。楊勍犯婺州。戊午,復置權尚書六部侍郎。癸亥,詔中原、淮南流寓士人,聽所在州郡附試。甲子,周望罷,尋分司、衡州居住。置京畿、淮南、湖北、京東西路鎮撫使。乙丑,升高郵軍為承州。以翟興、孟汝、趙立、劉位、趙霖、李成、吳翊、李彥先、薛
 慶並為鎮撫使:興河南府,唐州,立楚、泗州、漣水軍,位滁、濠州,霖和州、無為軍,成舒、蘄,翊光、黃州,彥先海州、淮陽軍,慶,承州、天長軍。丁卯,慶及金人戰於承州城下,累敗之。戊辰,命江、浙州縣祭戰死兵民。分江東、西為鄂州、江州、池州三路,置安撫使。罷諸路帥臣兼制置使、諸州守臣兼管內安撫使。是月,劉超據荊南,分兵犯峽州,又合叛將彭筠犯復州。淮西敗將崔增陷焦湖水砦。河東、北經制使王俊舉兵及金人戰於襄城縣,敗之,復穎昌
 府。張浚承制以金、房州隸利路。



 六月辛未朔,蠲紹興府三縣湖田米。詔侍從、臺諫、諸將集議駐蹕事宜。楊勍犯處州。癸酉,遣統制陳思恭討勍。合江南兩路轉運為都轉運使。再貶周望昭化軍節度副使、連州安置。甲戌,罷御營司。以範宗尹兼知樞密院事。乙亥,王□燮遣統領林閏等追襲楊勍於東陽縣,軍敗,裨將李在死之。丁丑,以劉光世部兵為御前巡衛軍,光世為都統制。楊勍等焚建州。戚方犯湖州安吉縣,詔張俊捕之。戊寅,更御前五
 軍為神武軍,御營五軍為神武副軍。以知建康府權邦彥為淮南等路制置發運使。滁、濠鎮撫使劉位為賊張文孝所殺,命其子綱襲職。庚辰,置鎮撫使六人:陳規,德安府、復州、漢陽軍;解潛,荊南府、歸、峽州、荊門、公安軍;程昌寓,鼎、澧州;陳求道,襄陽府、鄧、隨、郢州;範之才,金、均、房州;馮長寧,淮寧、順昌府、蔡州。辛巳,慮囚。申命有司,討論厘正崇寧以來濫賞。罷諸州添差通判職官。癸未,召劉光世赴行在。甲申,岳飛破戚方於廣德軍。乙酉,鐘相偽
 將胡源引兵入慈利縣,執其黨陳誠來降。丙戌,以呂頤浩為建康路安撫大使,劉光世為兩浙路安撫大使,朱勝非為江州路安撫大使,郭仲威為真、揚州鎮撫使。戚方詣張俊降。庚寅,召韓世忠率兵赴行在。辛卯,妖賊王宗石等伏誅。壬辰,權密州都巡檢徐文率部兵泛海來歸。甲午,置樞密院乾辦官四員。乙未,郭仲威犯鎮江府,遣岳飛擊之。是月,兀□聞張浚在秦州,將舉兵北伐,自六合引兵趨陜西。



 秋七月癸卯,劉光世援宣撫使例,乞
 便宜行事,不許。詔軍興以來諸州得便宜指揮者並罷。乙巳,馮長寧復順昌府。張浚罷曲端都統制。丁未,以劉光世為集慶軍節度使、開府儀同三司。戊申,以孔彥舟為辰、沅、靖州鎮撫使。張浚獻黃金萬兩助軍用。宣撫司遣統制官呂世存、王俊復鄜州,其餘州縣多迎降。後軍將王闢叛,陷歸州,鈐轄田祐恭擊敗之。己酉,王闢犯房州,守臣韋知幾棄城走。庚戌,楊勍受劉光世招安,尋復叛去,迫泉州。癸丑,崔增犯太平州,守臣郭偉拒卻之。乙
 卯,金人徙二帝自韓州之五國城。劉光世乞移司平江,不許。丙辰,張俊合諸將戚方等兵萬餘赴行在。丁巳,申命元祐黨人子孫於州郡自陳,盡還當得恩數。韓世忠、張俊並罷。己未,禁閩、廣、淮、浙海舶商販山東,慮為金人鄉導。詔江、浙、福建州縣,諭豪右募民兵據險立柵,防遏外寇。庚申,以岳飛為通、泰州鎮撫使。辛酉,建州民範汝為作亂,命統制李捧捕之。乙丑,復李邦彥以下十九人官職,聽自便。復李綱銀青光祿大夫,許翰、顏岐端明殿
 學士。張浚貶曲端階州居住。丁卯,金人立劉豫為帝,國號齊。戊辰,罷提領措置茶鹽司。己巳,詔王□燮部兵屯信州。程昌寓遣將杜湛禽李合戎於松滋縣。是月,張用據漢陽軍,沿江措置副使李允文招降之,以便宜徙鄂州路副總管,以右軍統制馬友知漢陽軍。



 八月辛未朔,以禮部尚書謝克家參知政事。壬申,李成請降於江州,詔撫納之,張浚停程千秋官、文州編管。癸酉,選神武中軍親兵六百人番直禁中。甲戌,詔侍從官日一員輪直,進
 故事關治體者。丁丑,以韓世忠為檢校少師、武成感德軍節度使,張俊檢校少保、寧武昭慶軍節度使。贈監察御史常安民、左司諫江公望為左諫議大夫,錄其後二人。庚辰,太后至自虔州。薛慶及金人戰於揚州城下,死之。郭仲威奔興化縣。辛巳,侍御史沈與求、戶部侍郎季陵以論宰相範宗尹,皆黜,宗尹復視事。癸未,盧益罷。張浚復永興軍,再貶曲端海州團練副使、萬州安置。甲申,陳萬信餘黨雷進作亂。乙酉,焚慈利、石門二縣。以御營司
 參議官王擇仁權河東制置使,山砦首領韋忠佺為都統制,宋用臣、馮賽同都統制。丙戌,命李成、吳翊捍禦上流,翊棄城去,以成為四州鎮撫使。命李捧便道過信州招捕靳賽。戊子,以饒、信妖賊平,赦二州徒以下囚,蠲民今年役錢。貶滕康永州、劉玨衡州,並居住。己丑,詔岳飛救楚州,仍命劉光世遣兵往援。辛卯,杜湛渡江討群賊,復石首等五縣。壬辰,盜入梅州,殺守臣沉同之,大掠而去。癸巳,命福建安撫使程邁會兵討範汝為。甲午,知虢
 州邵興遣統制閻興及金人戰於解州東,屢破之。金人陷承州。命陳思恭屯兵明州,以防海道。劉光世遣王德、酈瓊以輕兵渡江。乙未,遇金游騎於召伯埭,敗之。戊戌,以桑仲為襄陽、鄧隨郢州鎮撫使。是月,罷提舉廣西峒丁。孔彥舟入潭州,宣撫司參議官王以寧率兵拒之,以寧敗,遁去。宣撫司主管機宜文字傅雱在彥舟軍中,承制以彥舟權湖南副總管。劉綱以乏食,率兵奔溧陽。



 九月辛丑,呂頤浩入見,請益兵,命王□燮、巨師古、顏孝恭兵
 隸之,分屯境內。壬寅,詔諸路決囚。甲辰,徽宗皇後鄭氏崩於五國城。戊申,命秦鳳將關師古領兵赴行在。劉豫僭位於北京。庚戌,禁宣撫司僚屬便宜行事,及京西、湖南北路勿隸川、陜宣撫司節制。癸丑,涇原同統制李彥琦及金人戰於洛河車渡,敗之。乙卯,罷中書門下省檢正官。桑仲陷均、房州,進犯白土關。丙辰,復增左右司郎官為四員。金人攻楚州,趙立死之。丁巳,趙霖復和州。李成遣馬進犯興國軍。戊午,荊、襄賊趙延壽犯德安府,陳
 規拒卻之。己未,金、均、房安撫使王彥及桑仲戰於平麗縣,敗之。王闢詣彥降。辛酉,李捧擊範汝為於建州,官軍皆潰,捧遁去。金人犯揚州,統制靳賽逆戰於港河,敗之。金人陷延安府,執呂世存,又陷保安軍。癸亥,張浚遣都統制劉錫統五路兵及金將婁宿戰於富平縣,浚駐邠州督戰,官軍敗績。丙寅,給劉光世犒軍銀二萬兩、絹二萬匹。戊辰,趙延壽焚郢州。金人陷楚州,鎮撫使李彥先求救,兵敗死之。



 冬十月庚午朔,張浚斬環慶經略使趙
 哲於邠州,貶劉錫合州安置,命諸將各領兵歸本路。浚退保秦州,陜西大震。辛未,秦檜自楚州金將撻懶軍中歸於漣水軍丁祀水砦。壬申,命楊惟忠、王□燮討李成。丙子,以孔彥舟為鼎、澧、辰、沅、靖州鎮撫使。戊寅,鐘相餘黨楊華舉兵圍桃源縣。己卯,馬進犯江州。癸未,程昌寓入鼎州,擊楊華,破之。甲申,趣劉光世救楚州。丁亥,以李回同知樞密院事。庚寅,遣前御史臺檢法官謝向招範汝為。召張浚以兵入援。追復李邦彥觀文殿大學士。辛卯,
 虔州賊李敦仁及弟世雄舉兵破虔州石城縣。甲午,命楊惟忠率兵屯江州。乙未,岳飛破金人於承州。丙申,詔劉光世節制諸鎮,守禦通、泰州,伺便襲金人過淮。是月,馮長寧棄城去,尋以淮寧附劉豫。江東賊張琪犯建康府,劉洪道招降之。環慶路統制慕洧叛附於夏國。涇原統制張中彥、經略司乾辦趙彬叛降金人。劉忠據岳州平江縣白面山。王善餘黨祝友擁眾為亂,屯滁州龔家城。



 十一月癸卯,慕洧遂引金人圍環州。呂頤浩復南康
 軍。甲辰,趙鼎罷。乙巳,秦檜入見。丙午,岳飛棄泰州渡江。丁未,金人犯泰州,飛退保江陰軍沙上。以御史中丞富直柔簽書樞密院事,秦檜為禮部尚書。李允文殺岳州守臣袁植。呂頤浩會楊惟忠與馬進戰南康軍,不利。戊申,頤浩遣巨師古救江州,為進所敗,師古奔洪州。金人陷涇原,經略使劉錡退屯瓦亭。己酉,以孔彥舟為湖南副總管,部兵屯潭州。庚戌,命神武副軍都統制辛企宗討範汝為。壬子,日南至,率百官遙拜二帝,乙卯,改樞密
 院乾辦官為計議官,丙辰,金人陷泰州。丁巳,通州守臣呂伸棄城去。王彥攻桑仲於黃水,破之,房州平。張浚以彥為金、均、房州鎮撫使。崔增犯池州,劉洪道遣統制李貴擊走之,增以兵萬餘詣呂頤浩降。甲子,詔諸路守臣節制管內軍馬。丙寅,金、房州賊郭希犯歸州,田祐恭擊卻之。命王□燮部兵萬人速援呂頤浩。祝友渡江大掠。是月,張浚退軍興州,秦鳳副總管吳玠收餘兵保大散關東和尚原。詔諸路轉運司括借寺觀田租蘆場三年。



 十
 二月庚午,安南請入貢,卻之。辛未,遣度支員外郎韓球括饒、信諸州錢糧,凡江、湖、川、廣上供皆拘之。壬申,命孔彥舟援江州。丙子,禁節制軍馬守臣便宜行事。丁丑,馬進分兵犯洪州。乙丑,李敦仁犯撫州崇仁縣,命李山、張忠彥討之。壬辰,金人犯熙州,總管劉惟輔戰敗之,殺五千餘人。甲午,再犯熙州,惟輔軍潰被執,死之。乙未,以張俊為江南招討使,討李成。丁酉,範汝為降,詔補民兵統領。是月,張浚承制復海州團練副使曲端左武大夫,興
 州居住。是歲,宣撫處置司始令四川民歲輸激賞絹三十三萬匹有奇。



 紹興元年春正月己亥朔,帝在越州,帥百官遙拜二帝,不受朝賀。下詔改元,釋流以下囚,復賢良方正直言極諫科,蠲兩浙夏稅、和買紬絹絲綿,減閩中上供銀三分之一。戊申,改命張俊為江淮路招討使。復江、池路為江東、西路,分荊湖江南諸州為荊湖東、西路,置安撫司,治池、江、鄂、鼎州。江南東、西路各置轉運司,荊湖東、西路轉
 運司通掌兩路財賦。以呂頤浩為江東路安撫大使,朱勝非江西路安撫大使。馬進陷江州,守臣姚舜明棄城走,端明殿學士王易簡等二百人皆遇害。己酉,岳飛引兵之洪州。金人犯揚州。謝向率範汝為討平建陽賊劉時舉。金人犯秦州,吳玠擊敗之。庚戌,又犯西寧州,守臣俱重迎降。辛亥,謝克家罷。壬子,詔京官、知縣並堂除,內外侍從各舉可任縣令者二人,犯贓連坐。自今不歷縣令者勿除監司、郎官,不歷外任者勿除侍從,著為令。張
 中孚以原州叛降於金。癸丑,李敦仁圍建昌軍,蔡延世率鄉兵擊退之。賊曹成入漢陽軍,李允文招之,成入鄂州,復趨江西。丁巳,呂頤浩遣王□燮、崔增擊賊於湖口,大敗之。頤浩及楊惟忠引兵趨江州。辛酉,詔:「太祖創業垂統,德被萬世。神宗詔封子孫一人為安定郡王,世世勿絕。自宣和末至今未舉。有司其上應襲封人名,依故事舉行。」金人再圍環州。是月,張浚復曲端榮州刺史、提舉江州太平觀、閬州居住,尋移恭州。



 二月戊辰朔,宜章縣
 民李冬至二作亂,犯英、連、韶、郴諸州。祝友降,劉光世分其軍,以友知楚州。庚午,改行宮禁衛所為行在皇城司。李成黨邵友犯筠州,守臣王庭秀棄城去。辛未,犯臨江軍,守臣康倬遁。壬申,初定歲祀天地、社稷,如奏告之禮。癸酉,桑仲自棗陽引兵還襄陽。丁丑,鄜延將李永琦叛,犯慶陽府。戊寅,禁州郡統兵官擅招安亂軍盜賊。己卯,日中有黑子,四日乃沒。以辛企宗為福建制置使。辛巳,以秦檜參知政事。壬午,水賊張榮入通州。癸未,詔辛企
 宗及謝向罷遣範汝為兵,汝為不聽命。甲申,詔王□燮、張俊掎角討捕馬進等賊。丙戌,復置秘書省。己丑,命孔彥舟、呂頤浩、張俊會兵討李成。壬辰,雨雹。癸巳,邵青寇宣州。丙申,詔諸路提刑司以八月類省試。張浚亦以便宜合川、陜舉人即置司類省試。丁酉,宣教郎範燾坐誣訟孟忠厚,且及太后,除名、潮州編管。是月,李敦仁犯汀州。馬友遣其黨犯鄂州,總管張用拒卻之。李允文以友權湖南招捉公事,友大掠漢陽而去,過岳州,守臣吳錫遁,
 友據之。



 三月戊戌朔,以嚴、衢二州守臣柳約、李處勱有治效,各進職一等。呂頤浩遣崔增、王□燮合兵擊李成於湖口,大敗之。庚子,張浚以富平之敗上疏待罪,詔免。壬寅,禁諸路遏糴。丙午,張俊、楊沂中、岳飛渡江擊馬進,大敗之。孔彥舟焚掠潭州,趨衡州。己酉,李成犯饒州。庚戌,張俊、楊沂中復擊馬進於筠河,敗之,復筠州,進奔江州。男子崔紹祖詐稱越王中子,受上皇詔為天下兵馬大元帥,趙霖以聞。辛亥,詔赴行在。命劉光世兼淮南、京東
 路宣撫使,治揚州,經畫屯田。光世迄不行。甲寅,罷諸州免行錢。乙卯,金人破階州。庚申,劉超犯澧州,統制杜湛率兵拒之。甲子,始下詔罪李成,募人禽斬,赦脅從者。張俊追馬進至江州,進戰敗,遁去。乙丑,俊復江州,楊沂中、趙密引兵追擊進,又大敗之。成奔蘄州。振淮南、京東西流民。荊湖東路安撫使向子諲說降馬友,與共討李冬至二,平之。是月,金人攻張榮縮頭湖水砦,榮擊敗之,來告捷,劉光世以榮知泰州。金人迫興州,張浚退保閬州,
 以端明殿學士張深為四川制置使,及參議軍事劉子羽趨益昌。參謀官王庶為龍圖閣待制、知興元府兼利、夔兩路制置使,節制陜西諸路。桑仲以其黨李道知隨州。



 夏四月己巳,張浚承制分利、閬、劍、文、政五州為利州路,置經略安撫使。庚午,張琪復叛,犯當塗縣。金將撻懶渡淮,屯宿遷縣馬樂湖。壬申,太白晝見。乙亥,劉光世復楚州。階州統領杜肇復階州。馬友引兵入潭州。戊寅,杜琪棄澧州,劉超入據之。己卯,金涇原帥趙彬犯耀州,守
 臣趙澄擊走之。淮賊寇宏犯濠州。庚辰,隆祐皇太后崩。癸未,桑仲陷鄧州,守將譚兗棄城走,河東招捉使王俊引兵來援,仲執斬之,以其黨李橫知州事。乙酉,為太后制期年服。辛卯,群臣三上表,始聽政。癸巳,命向子諲發兵及廣西安撫許中同扼險要,防孔彥舟入廣,仍許脅從自新以招諭。是月,京西賊李忠陷商州,守臣楊伯孫棄城走。呂頤浩遣統制閻皋、通判建昌軍蔡延世襲擊李敦仁,禽其弟世雄、世臣。



 五月丙申朔,蠲江西路被
 賊州縣賦稅。丁酉,詔呂頤浩、朱勝非、劉光世並兼淮南諸州宣撫使。始奪李成官。戊戌,以張用為舒、蘄鎮撫使。癸卯,作「大宋中興寶」成。金人犯和尚原,吳玠擊敗之。丙午,初復召試館職之制。劉光世遣統制王德襲揚州,執郭仲威以獻,伏誅。辛亥,水軍統制邵青叛,圍太平州。趙彬及金人合兵圍慶陽府,守臣楊可升擊敗之。甲寅,命知南外宗正事令選年幼宗子,將育於宮中。詔收耆戶長役錢。己未,詔州縣因軍期徵取民財物者,立式榜
 示,禁過數催擾。庚申,孔彥舟引眾過潭州,馬友迎擊,大敗之。彥舟趨岳州,犯鄂州。李允文以彥舟為湖東副總管,屯漢陽。辛酉,以直秘閣宗綱為荊南鎮撫司措置營田官,樊賓為副。壬戌,劉光世招降邵青。趙延壽據分寧縣,呂頤浩招降之。是月,張俊及李成戰於黃梅縣,殺馬進,成敗,遁歸劉豫。李忠、譚兗各率兵歸張浚,浚命王庶分其兵。張用復叛,寇江西,岳飛招降之。湖州進士吳木上書論宰執,送徽州編管。



 六月己巳,始鬻承直、修武郎
 以下官。壬申,冊謚皇太后曰昭慈獻烈。甲戌,張琪犯餘杭,又犯宣州。乙亥,月犯心。庚辰,湖賊楊華、楊廣犯鼎州,程昌寓拒卻之。上虞縣丞婁寅亮上書,請選立繼嗣。壬午,權攢昭慈獻烈皇后於越州。張琪犯徽州,守臣郭東棄城去,琪入據之。癸未,張浚引大兵至瑞昌縣之丁家洲,李允文自鄂部兵歸浚,浚並其兵,護允文赴行在。邵青率舟師至鎮江,甲申,復叛去。丁亥,崇安民廖公昭合範汝為餘黨熊志寧作亂,眾既散,志寧復與建陽民
 丁朝佐合兵陷二縣。戊子,慮囚。己丑,邵青犯江陰軍之福山,遣海州鎮撫使李進彥、中軍統制耿進率舟師會劉光世討之。南安賊吳忠、宋破壇、劉洞天作亂。庚寅,江西提刑司遣官討之,破壇、洞天皆伏誅,忠遁去。癸巳,熙河統制關師古、洮東安撫郭玠同討熙州叛兵,連敗之。甲午,廣賊鄧慶、龔富圍南雄州,守臣鄭成之率兵民以拒。蠲建、劍、汀州、邵武軍租。是月,知虢州邵興屯盧氏縣,為河南統制董先所破,走興元,先遂取商、虢二州。張浚
 承制以吳玠為陜西諸都統制。時關隴六路盡陷,止餘階、成、岷、鳳、洮五郡、鳳翔之和尚原、隴州之方山原。粘罕既得陜西地,悉與偽齊。



 秋七月乙未朔,以馬友權荊湖東路副總管,趣討孔彥舟。統制潘逵、後軍將胡江等叛、破玉山、弋陽、永豐三縣,遣樞密院準備將領徐文討之。戊戌,吳錫復入邵州。庚子,以岳飛為神武右副軍統制,留軍洪州,彈壓盜賊。辛丑,封伯右武衛大將軍令話為安定郡王。壬寅,虔州賊陳顒作亂,命趣捕之。甲辰,詔
 秘書省長貳通修日歷。丙午,劉光世遣將喬仲福擊邵青於常熟,為所敗。撻懶自宿遷北歸。戊申,韓世清追襲張琪,復祁門縣。庚戌,張俊執傅雱赴行在。張浚以曲端屬吏,以武臣康隨提點夔路刑獄,與王庶雜治之。辛酉,召呂頤浩赴行在。張琪犯饒州,頤浩遣閻皋擊敗之。琪黨姚興降,琪走徽州。癸亥,範宗尹罷。是月,濠州守臣李玠棄城去。王彥數擊敗李忠。趙彬來歸,張浚承制以彬為陜西轉運使,又以涇原兵馬都監李彥琪為本路副
 總管,彥琪尋叛去。



 八月丙寅,以孔彥舟為蘄、黃鎮撫使。丁卯,以知潭州吳敏為荊湖東西、廣南路宣撫使。張浚殺曲端於恭州獄。張用部兵至瑞昌歸張浚,浚以用為本軍統制。戊辰,張守等上《紹興重修敕令格式》。癸酉,復以汪伯彥為江東安撫大使。乙亥,呂頤浩遣將李鑄復舒州。丁丑,祔昭慈獻烈皇后神主於溫州太廟。戊寅,張守罷。以李回參知政事,富直柔同知樞密院事。庚辰,杜湛及劉超戰於彭山,為所敗。辛巳,超及楊華、楊廣合兵
 復寇鼎州,程昌寓遣湛率舟師擊敗之。遣辛企宗移軍福州,討熊志寧、胡江等諸賊。韓世清及張琪戰,世清敗,琪復入祁門縣。壬午,命張俊遣兵捕之。鑄紹興錢。癸未,詔許邵青、張琪脅從徒黨自新。乙酉,以李成在順昌,恐復謀亂,遣使繼蠟書諭淮寧、蔡州將士,立賞格,募人禽斬成。丁亥,以秦檜為尚書右僕射、同中書門下平章事兼知樞密院事。庚寅,復李綱資政殿大學士。募人往京東、河南伺察金、齊動止,仍繼詔慰撫忠義保聚之人。蔡
 州鎮撫使範福棄城去,以土豪李祐代之。辛卯,蠲徽州被賊民家夏稅。壬辰,置三省、樞密院賞功房。是月,知郢州曹成掠湖西,犯沅州,與知復州李宏合屯濟陽,既而攻宏,宏奔潭州。



 九月甲午朔,張琪黨李捧犯宣州,守臣李彥卿及韓世清擊卻之。詔江東、西路安撫使復治建康府、洪州。以王□燮知池州,楊惟忠知江州,並兼管內安撫使,率部兵赴官。丙申,斬李世臣。己亥,以資政殿學士葉夢得為江南東路安撫大使,兼壽春等六州宣撫使。
 庚子,張琪復陷宣州,已乃遁去。辛丑,命王□燮討琪。丁未,詔歲再遣使省謁諸陵,因撫問河南將士。命馬友移屯鄂州。庚戌,命宗室右監門衛大將軍士芑朝饗溫州太廟。辛亥,合祭天地於明堂,太祖、太宗並配,大赦。罷諸州守臣節制軍馬。錄用元符末上書人子孫。癸丑,復以呂頤浩為尚書左僕射、同中書門下平章事兼知樞密院事。丁巳,王彥破李忠於秦郊店,忠奔歸劉豫。戊午,禁福建轉運司抑民出助軍錢。落範宗尹觀文殿學士。己未,
 初措置河南諸鎮屯田。以戶部尚書孟庾為江東西、湖東等路宣諭制置使。辛酉,詔四方有建策能還兩宮者,實封以聞,有效者賞以王爵。壬戌,遣御史胡世將督捕福建盜賊。是月,長星見。



 冬十月乙丑,詔蔡京、王黼門人實有才能者,公舉敘擢。李回罷。丙寅,朱勝非分司、江州居住。丁卯,以李允文恣睢專殺,賜死大理獄。己巳,王德招邵青,降之。庚午,以孟庾參知政事,徽猷閣直學士湯東野為江、淮發運使。劉洪道招降李捧、華旺。壬申,置行
 在大宗正司。癸酉,兀□攻和尚原,吳玠及弟璘力戰,大敗之,兀□僅以身免。丁丑,增置諸路武尉。戊寅,以張俊為太尉,移屯婺州。壬午,初置見錢關子,招人入中,以給軍食。範汝為復叛,入建州,守臣王浚明棄城走,辛企宗退屯福州。甲申,劉超請降,以超守光州。戊子,崔紹祖伏誅。詔邵青以舟師赴行在。己丑,升越州為紹興府。李成軍正李雱伏誅。知承州王林禽張琪於楚州,檻送行在。壬辰,錄程頤孫易為分寧令。癸巳,範汝為犯邵武軍,守
 臣吳必明、統制李山率兵拒之,眾潰,退保光澤縣。關師古復秦州,獲郭振。是月,劉豫遣將王世沖寇廬州,守臣王亨大破之,斬世沖。曹成及馬友戰於潭州,成敗,還攸縣。王才遣將丁順圍濠州,劉光世遣兵攻橫澗山,順解圍去。



 十一月乙未,葉夢得至建康,以詔招王才,降之。丙申,遣內侍撫問孔彥舟、桑仲。丁酉,榜諭福建、江東群盜,赦其脅從者。戊戌,詔移蹕臨安。以孟庾為福建、江西、荊湖宣撫使,神武左軍都統制韓世忠副之,仍命械謝向、
 陸棠赴行在。己亥,以婁寅亮為監察御史。範汝為犯光澤縣,李山走信州。辛丑,續編《紹興太常因革禮》。桑仲請正劉豫惡逆之罪,詔進幸荊南。乙巳,以右司諫韓璜黨富直柔,責監潯州稅。張琪伏誅。庚戌,富直柔罷。荊湖、廣西宣撫使吳敏始受命置司柳州。辛亥,升康州為德慶府。壬子,詔內外侍從各舉所知三人。丙辰,程昌寓遣杜湛擊楊華,敗之。命張俊遣使持詔招曹成,以所部赴行在。己未,楊華請降。辛酉,命吏部侍郎李光節制臨安府
 內外諸軍。壬戌,曹成犯安仁縣,執安撫使向子諲,進攻道州。是月,前知廓州李惟德以岷州來歸。吳玠始遣人通書夏國。


十二月乙丑,吳敏罷。丙寅,復置樞密院都承旨。範汝為遣葉澈寇南劍州,守臣
 \gezhu{
  □角}
 拒戰,大破之。己巳,遣吏部侍郎傅崧卿為淮東宣諭使。甲戌,遣江東安撫司統制郝晸、顏孝恭討建昌軍賊。乙亥,辛企宗罷,仍追三官,率兵赴軍前自效。丁丑,蠲諸路在官積欠。詔官戶名田過制者與民均科。以岳飛為神武副軍都統制,
 部兵屯洪州。曹成陷道州,守臣向子忞棄城走。戊寅,以彗出,求直言。增行在職事官職錢。遣駕部員外郎李願撫諭川、陜。己卯,詔兩浙分東、西路,置提點刑獄。庚辰,桑仲遣兵寇復州,守臣俎遹棄城去。辛巳,復置廣西提舉茶鹽司。知海州薛安靖殺偽都巡檢使王企中,率軍民以城來歸。增諸路酒錢,以備軍費。甲申,知龍州範綜、統制雷仲舉兵復水洛城。己丑,起復陜西都統制吳玠為鎮西軍節度使。詔江西安撫司趣兵討捕吳忠。是月,劉
 豫遣將王彥充攻壽春府。桑仲遣李橫復寇金州,王彥拒戰於馬郎嶺,大破之,均州平。蔡州褒信縣弓手許約叛,據光州。階州安撫孫注復洮州。龔富等圍南劍州。



\end{pinyinscope}