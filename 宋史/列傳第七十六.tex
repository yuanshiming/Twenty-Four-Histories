\article{列傳第七十六}

\begin{pinyinscope}

 邵亢從父必馮京錢惟演從弟易易子彥遠明逸諸孫景諶勰即



 邵亢,字興宗,丹陽人。幼聰發過人,方十歲,日誦書五千言。賦詩豪縱,鄉先生見者皆驚偉之。再試開封,當第一,以賦失韻,弗取。範仲淹舉亢茂才異等,時布衣被召者
 十四人,試崇政殿,獨亢策入等,除建康軍節度推官。或言所對策字少,不應式,宰相張士遜與之姻家,故得預選,遂報罷。而士遜子實娶它邵,與亢同姓耳。士遜既不能與直,亢亦不自言。



 趙元吳叛,亢言:「用兵在於擇將,今天下久不知戰,而所任多儒臣,未必能應變。武人得長一軍,又已老,詎能身先矢石哉?間起故家恩幸子弟,彼安識攻守之計?況將與卒素不相附,又亡堅甲利兵之御。此不待兩軍相當,而勝敗之機,固已形矣。」因獻《兵說》
 十篇。



 召試秘閣,授穎州團練推官。晏殊為守,一以事諉之。民稅舊輸陳、蔡,轉運使又欲覆折緡錢,且多取之。亢言:「民之移輸,勞費已甚。方仍歲水旱,又從而加取,無乃不可乎?」遂止。入為國子監直講、館閣校勘、同知太常禮院。張貴妃薨,立園陵,禁京城樂一月,亢累疏罷之。進集賢校理。仁宗繼嗣未立,亢言:「國之外患在邊圉,然御之之術,不過羈縻勿絕而已。內患則不然,系社稷之安危,不可不蚤定也。」提點開封縣鎮公事。比有縱火者,一不
 獲則主吏坐罪,民或自燔其居以中吏。亢請非延及旁舍者,雖失捕,得勿坐。徙為府推官,改度支判官。



 契丹遣使賀乾元節,未至,仁宗崩。議者謂宜卻,或欲俟其及國門而諭使之還,亢請令奉書至柩前,使見嗣君。從之。選為穎王府翊善,加直史館。召對群玉殿,英宗訪以世事,稱之曰:「學士真國器也。」擢同修起居注。建言:「陛下初政,欲治國者先齊家,穎王且授室,願採用古昏禮。公主下降,不宜厭舅姑之尊。」帝深納之。他日,諭王曰:「以翊善端
 直樸厚,輟為諫官矣。」王出道帝語,遂以知制誥知諫院。東宮建,為右庶子。



 神宗立,遷龍圖閣直學士。有譖之者曰:「先帝大漸時,亢嘗建垂簾之議。」御史吳申即論之。帝知其妄,置不問。亢自訴曰:「方先帝不豫,群臣莫得進見,臣無由面陳,必有章奏。乞索之禁中,若得之,臣當伏誅;不然,則讒臣者,豈宜但已,願下獄考實。」帝不許。時待制以上為帥、守,每他徙必遷職秩,亢請未滿兩歲者勿推恩。王陶劾韓琦,吳奎與之辨。亢詆奎所言顛倒,失大臣
 體,蓋欲人並撼琦。琦與奎竟同日去。



 進樞密直學士、知開封府。亢遇事敏密,吏操辭牘至前,皆反復閱之。人或以為勞,亢曰:「決是非於須臾,正當爾。初雖煩,後乃省也。」籍裏閭惡少年與吏之廢停者,一有所犯,皆遷處之,畿下鬥訟為之衰止。拜樞密副使。



 夏人誘殺知保安軍楊定,朝廷謀西討。亢曰:「天下財力殫屈,未宜用兵,唯當降意撫納,俟不順命,則師出有名矣。」因條上其事。詔報之曰:「中國民力,大事也。兵興之後,不無倍率,人心一搖,安危
 所系。今動自我始,先違信誓,契丹聞之,將不期而自合,茲朕所深憂者。當悉如卿計。」未幾,夏主諒祚死,國人執殺定者來請和。或欲乘此更取塞門地,亢以為幸人之喪,非義也,乃止。



 亢在樞密逾年,無大補益,帝頗厭之,嘗與諫官孫覺言,欲以陳升之代亢,而使守長安。覺遽劾亢薦升之,帝怒其希指,黜覺,亢亦引疾辭,以資政殿學士知越州。歷鄭、鄆、亳三州。薨,年六十一。贈吏部尚書,即其鄉賜以居宅,謚曰安簡。從父
 必。



 必字不疑。舉進士,為上元主簿。國子監立石經,必善篆隸,召充直講。選為《唐書》編修官。必以史出眾手,非古人撰述之體,辭不就。進集賢校理、同知太常禮院。天子且親祠,執事者習禮壇下。必言:「《周官·大宗伯》:『凡王之禱祠,肄儀為位。』鄭康成釋云:『若今肄司徒府。』古禮如此。今即祠所習之,為不敬。」乃徙於尚書省。張貴妃受冊,禮官議命婦入賀儀未決,或曰:「妃為修媛時,命婦已不敢亢禮,況今日乎?」必曰:「宮省事秘不可知。既下有司議,惟有外
 一品南省上事百官班見之儀,然禮無不答。」眾議乃定。



 出知常州,召為開封府推官。坐在常州日杖人至死,責監邵武稅,然杖者實不死。久之,知高郵軍,提點淮南刑獄,為京西轉運使。必居官震厲風採,始至郡,惟一赴宴集;行部,但一受酒食之饋。以為數會聚則人情狎,多受饋則不能行事,非使者體也。入修起居注、知制誥。



 雄州種木道上,契丹遣人夜伐去,又數漁界河中。事聞,命必往使,必以理折契丹,屈之。還,知諫院。編《仁宗御集》成,遷
 寶文閣直學士、權三司使,加龍圖閣學士、知成都。卒於道,年六十四。遣中使護其喪歸。



 馮京,字當世,鄂州江夏人。少雋邁不群,舉進士,自鄉舉、禮部以至廷試,皆第一。時猶未娶,張堯佐方負宮掖勢,欲妻以女。擁至其家,束之以金帶,曰:「此上意也。」頃之,宮中持酒殽來,直出奩具目示之。京笑不視,力辭。出守將作監丞、通判荊南軍府事。還,直集賢院、判吏部南曹,同修起居注。吳充以論溫成皇后追冊事,出知高郵,京疏
 充言是,不當黜。劉沆請人並斥京,仁宗曰:「京亦何罪?」但解其記注,旋復之。



 試知制誥。避婦父富弼當國嫌,拜龍圖閣待制、知揚州。改江寧府,以翰林侍讀學士召還,糾察在京刑獄。為翰林學士、知開封府。數月不詣丞相府,韓琦語弼,以京為傲。弼使往見琦,京曰:「公為宰相,從官不妄造請,乃所以為公重,非傲也。」出安撫陜西,請城古渭,通西羌唃氏,畀木征官,以斷夏人右臂。除端明殿學士、知太原府。



 神宗立,復為翰林學士,改御史中丞。王安石
 為政,京論其更張失當,累數千百言,安石指為邪說,請黜之。帝以為可用,擢樞密副使。河東麟、府、豐三州,城壘兵械不治,官吏皆受譴。京以先帥本道,上章自劾曰:「使諸路帥臣,知其雖一時脫去,後能僥竊名位者,猶必行法,將不敢復偷惰曠職。」優詔不聽。進參知政事。數與安石論辨,又薦劉分文、蘇軾掌外制。安石令保甲養馬,京謂必不可行。會選人鄭俠上書言時政,薦京可相,呂惠卿因是譖京與俠通,罷知亳州。未幾,以資政殿學士知渭
 州。茂州夷叛,徙知成都府。蕃部何丹方寇雞粽關,聞京兵至,請降。議者遂欲蕩其巢窟,京請於朝,為禁侵掠,給稼器,餉糧食,使之歸。夷人喜,爭出犬豕割血受盟,願世世為漢藩。惠卿告安石罪,發其私書,有曰「勿令齊年知」,齊年謂京也,與安石同年生。帝以安石為欺,復召京知樞密院。京以疾未至,帝中夕呼左右語曰:「適夢馮京入朝,甚慰人意。」乃賜京詔,有「渴想儀刑,不忘夢寐」之語。及入見,首以所夢告焉。頃之,以觀文殿學士知河陽。



 哲宗
 即位,拜保寧軍節度使、知大名府,又改鎮彰德。於是範祖禹言:「京再執政,初與王安石不合,後為呂惠卿所傾,其中立不倚之操,為先帝稱挹。且昭陵學士,獨京一人存,若付以樞密,必允公論。」時京已老,乃以為中太一宮使兼侍講,改宣徽南院使,拜太子少師,致仕。紹聖元年,薨,年七十四。帝臨奠於第,贈司徒,謚曰文簡。



 始,京鄉居,受恩通判南宮成,迨貴,以郊恩官其子。嘗過外兄朱適,出侍妾,詢知為同年進士妻,亟請而嫁之。其為郡守,諸
 縣公事至,即歷究之,茍與縣牘合而處斷麗於法者,呼法吏決罪,不以侍獄。報下捷疾,一無壅滯,人服其敏云。



 錢惟演,字希聖,吳越王俶之子也,少補牙門將,從俶歸朝,為右屯衛將軍。歷右神武軍將軍。博學能文辭,召試學士院,以笏起草立就,真宗稱善。改太僕少卿,獻《咸平聖政錄》。命真秘閣,預修《《冊府元龜》,詔與楊億分為之序。除尚書司封郎中、知制誥,再遷給事中、知審官院。大中祥符八年,為翰林學士,坐私謁事罷之。尋遷尚書工部
 侍郎,再為學土、會靈觀副使。又坐貢舉失實,降給事中。復工部侍郎,擢樞密副使、會靈觀使兼太子賓客,更領祥源觀。累遷工部尚書。



 仁宗即位,進兵部。王曾為相,以惟演嘗位曾上,因拜樞密使。故事,樞密使必加檢校官,惟演止以尚書充使,有司之失也。初,惟演見丁謂權盛,附之,與為婚。謂逐寇準,惟演與有力焉。及序樞密題名,獨刊去準,名曰「逆準」,削而不書。謂禍既萌,惟演慮並得罪,遂擠謂以自解。宰相馮拯惡其為人,因言:「惟演以妹
 妻劉美,乃太后姻家,不可與機政,請出之。」乃罷為鎮國軍節度觀察留後,即日改保大軍節度使、知河陽。逾年,請入朝,加同中書門下平章事、判許州。未即行,冀復用,侍御史鞠詠奏劾之,惟演乃亟去。天聖七年,改武勝軍節度使。明年來朝,上言先□在洛陽,願守宮鑰。即以判河南府,再改泰寧軍節度使。



 惟演雅意柄用,抑鬱不得志。及帝耕籍田,求侍祠,因留為景靈宮使。太后崩,詔還河南。惟演不自安,請以莊獻明肅太后、莊懿太后並配
 真宗廟室,以希帝意。惟演既與劉美親,又為其子曖娶郭后妹,至是,又欲與莊懿太后族為婚。御史中丞範諷劾惟演擅議宗廟,且與後家通婚姻。落平章事,為崇信軍節度使,歸本鎮。未幾,卒,特贈侍中。太常張瑰按,《謚法》敏而好學曰「文」,貪而敗官曰「墨」,請謚文墨。其家訴於朝,詔章得像等復議,以惟演無貪黷狀,而晚節率職自新,有惶懼可憐之意,取《謚法》追悔前過曰「思」,改謚曰思。慶歷間,二太后始升祔真宗廟室,子曖復訴前議,乃改謚
 曰文僖。



 惟演出於勛貴,文辭清麗,名與楊億、劉筠相上下。於書無所不讀,家儲文籍侔秘府。尤喜獎厲後進。初,真宗謚號稱「文」,惟演曰「真宗幸澶淵禦契丹,盟而服之,宜兼謚『武』。下有司議,乃加謚「武定」。所著《典懿集》三十卷,又著《金坡遺事》、《飛白書敘錄》《逢辰錄》、《奉藩書事》。惟演嘗語人曰:「吾平生不足者,惟不得於黃紙上押字爾。」蓋未嘗歷中書故也。子曖、晦、暄,從弟易。



 晦字明叔,以大理評事娶獻穆大長公主女,累遷東上閣門使、貴州團練使。
 王守忠領兩使留後,移閣門定朝立燕坐位,晦因言:「天子大朝會,令宦者齒士大夫坐殿上,必為外夷所笑。」守忠更欲以禮服進酒,晦又以為不可。勾當三班院、群牧都監,授忠州防禦使、知河中府。帝因戒曰:「陜西方罷兵,民困久矣。卿為朕愛撫,毋縱酒樂,使人呼為貴戚子弟也。」晦頓首謝。改穎州防禦使,為秦鳳路馬步軍總管。復還三班院,同提舉集禧觀。歷霸州防禦使,為群牧副使,卒。



 暄字載陽,以父蔭累官駕部郎中、知撫州,移臺州。臺
 城惡地下,秋潦暴集,輒圮溺,人多即山為居。暄為增治城堞,壘石為臺,作大堤捍之。進少府監、權鹽鐵副使。暄鉤考諸路逋租,兩浙轉運使負課當坐,暄上言:「浙部仍歲饑,故租賦不登籍,今使者獲罪,必亟斂於民,民不堪矣。」神宗即詔釋之。官制行,為光祿卿,出知鄆州,拜寶文閣待制,卒。子景臻,尚秦、魯國大長公主。景臻子忱,在《外戚傳》。



 易字希白。始,父倧嗣吳越王,為大將胡進思所廢,而立
 其弟俶。俶歸朝,群從悉補官,易與兄昆不見錄,遂刻志讀書。昆字裕之,舉進士,為治寬簡便民,能詩,善草隸書,累官右諫議大夫,以秘書監於家。



 易年十七,舉進士,試崇政殿,三篇,日未中而就。言者惡其輕俊,特罷之。然自此以才藻知名。太宗嘗與蘇易簡論唐世文人,嘆時無李白。易簡曰:「今進士錢易,為歌詩殆不下白。」太宗驚喜曰:「誠然,吾當自布衣召置翰林。」值盜起劍南,遂寢。真宗在東宮,圖山水扇,會易作歌,賞愛之。



 易再舉進士,就開
 封府試第二。自謂當第一,為有司所屈,乃上書言試《朽索之馭六馬賦》,意涉譏諷。真宗惡其無行,降第二。明年,第二人中第,補濠州團練推官。召試中書,改光祿寺丞、通判蘄州。奏疏曰:「堯放四罪而不言殺,彼四者之兇,尚惡言殺,非堯仁之至乎?古之肉刑者劓、椓、黥、刖皆非死,尚以為虐。近代以來,斷人手足,鉤背烙筋,身見白骨而猶視息,四體分落乃方絕命。以此示人,非平世事也。今四方長吏競為殘暴,婺州先斷賊手足,然後斬之以聞。
 壽州巡檢使磔賊於闤闠之中,其旁猶有盜物者。使嚴刑可誡於眾,則秦之天下無叛民矣。臣以謂非法之刑,非所以助治,惟陛下除之。」帝嘉納其言。



 景德中,舉賢良方正科,策入等,除秘書丞、通判信州。東封泰山,獻《殊祥錄》,改太常博士、直集賢院。祀汾陰,幸亳州,命修《車駕所過圖經》,獻《宋雅》一篇,遷尚書祠部員外郎。坐發國子監諸科非其人,降監穎州稅。數月,召還。久之,判三司磨勘司。上言:「官物在籍,而三司移文厘正,或其數細微,輒歷
 年不得報,徒擾州縣。自今官錢百、穀斗、帛二尺以下,非欺紿者除之。」真宗雅眷詞臣,其典掌誥命,皆躬自柬拔。擢知制誥、判登聞鼓院、糾察在京刑獄。累遷左司郎中,為翰林學士,



 儤直未滿,卒。仁宗憐之,召其妻盛氏至禁中,賜以冠帔。



 易才學瞻敏過人,數千百言,援筆立就。又善尋尺大書行草,及喜觀佛書,嘗校《道藏經》,著《殺生戒》,有《金閨》、《瀛州》、《西垣制集》一百五十卷,《青雲總錄》、《青雲新錄》《南部新書》、《洞微志》一百三十卷。子彥遠、明逸,相繼皆
 以賢良方正應詔。宋興以來,父子兄弟制策登科者,錢氏一家而已。



 彥遠字子高,以父蔭補太廟齋郎,累遷大理寺丞。舉進士第,以殿中丞為御史臺推直官。通判明州,遷太常博士。舉賢良方正能直言極諫科,擢尚書祠部員外郎、知潤州。上疏曰:



 陛下即位以來,內無聲色之娛,外無畋漁之樂,而前歲地震,雄、霸、滄、登,旁及荊湖,幅員數千里,雖往昔定襄之異,未甚於此。今復大旱,人心嗷嗷,天其或
 者以陛下備寇之術未至,牧民之吏未良,天下之民未安,故出譴告以示之。茍能順天之戒,增修德業,宗社之福也。



 今契丹據山後諸鎮,元昊盜靈武、銀、夏,衣冠車服,子女玉帛,莫不有之。往時,元昊內寇,出入五載,天下騷然。及納款賜命,則被邊長吏,不復銓擇,高冠大裾,恥言軍旅,一日契丹負恩,乘利入塞,豈特元昊之比耶?湖、廣蠻獠劫掠生民,調發督斂,軍須百出,三年於今,未聞分寸之效。惟陛下念此三方之急,講長久之計,以上答天
 戒。



 時旱蝗,民乏食,彥遠發常平倉賑救之。部使者詰其專且搉價,彥遠不為屈。召為右司諫,請勿數赦,擇牧守,增奉入以養廉吏,息土木以省功費。遷起居舍人、直集賢院、知諫院。會諸路奏大水,彥遠言陰氣過盛,在《五行傳》「下有謀上之象」,請嚴宮省宿衛。未幾,有挾刃犯謻門者。特賜五品服。又上疏曰:



 農為國家急務,所以順天養財,御水旱,制蠻夷之原本也。唐開元戶八百九十餘萬,而墾田一千四百三十餘萬頃。今國家戶七百三十餘
 萬,而墾田二百一十五萬餘頃,其間逃廢之田,不下三十餘萬,是田疇不闢,而游手者多也。勸課其可不興乎?



 本朝轉運使、提點刑獄、知州、通判,皆帶勸農之職,而徒有虛文,無勸導之實。宜置勸農司,以知州為長官,通判為佐,舉清強幕職、州縣官為判官。先以墾田頃畝及戶口數、屋塘、山澤、溝洫、桑柘,著之於籍,然後設法勸課,除害興利。歲終農隙,轉運司考校之,第其賞罰。



 楊懷敏妄言契丹主宗真死,乃除入內副都知;內侍黎用信以罪
 竄海島,赦歸,遽得環衛官致仕;許懷德、慎鏞高年未謝事;楊景宗、郭承祐闟冗小人,宜廢不用:歷舉劾之,多見聽納。彥遠性豪邁,其任言職,數有建明。卒於官。



 明逸字子飛。繇殿中丞策制科,轉太常博士。為呂夷簡所知,擢右正言。首劾範仲淹、富弼:「更張綱紀,紛擾國經。凡所推薦,多挾朋黨。乞早罷免,使奸詐不敢效尤,忠實得以自立。」疏奏,二人皆罷;其夕,杜衍亦免相。明逸蓋希章得象、陳執中意也。



 石元孫與夏人戰沒,以死事褒贈,
 既而生歸,朝廷釋不問。明逸請正其僨軍之罪,乃竄之遠方而奪其恩。進同修起居注、知制誥,擢知諫院,為翰林學士。自登科至是,才五年。加史館修撰、知開封府。妄人冷青自稱皇子,捕至府,明逸方正坐,青□七曰:「明逸安得不起?」明逸為起,坐尹京無威望;又獄吏榜婦人酇氏墮足死,罷為龍圖閣學士、知蔡州。歷揚、青、鄆、曹州、應天府,還,判流內銓、知通進銀臺司,復出知成德軍、渭州。加端明殿學士、知秦州。



 先是,於闐入貢,道邈川,唃廝囉留
 不遣。會其妻亡,前帥張方平請因而恤之,且誘其般次入貢,詔賻絹千匹。明逸言:「朝廷撫唃氏至厚,頃以招馬為名,賂繒縕;邀請六事,既徇其五,而猶觖望。今壅遏荒服之貢,固有罪矣,豈可復加賜以辱國體?」從之。而於闐使與般次亦皆至。廝囉有子質於秦,別子木征居河州。殿侍程從簡私與之盟,令過洮河,許以官,且歸其質子。事不驗,木征怒,留貢使。明逸械從簡往詰,因斬之。木征惶懼,悉遣所留者。



 治平初,復為翰林學士。神宗立,御史
 論其傾險憸薄,頃附賈昌朝、夏竦以陷正人,文辭淺繆,豈應冒居翰院?乃罷學士。久之,知永興軍。熙寧四年,卒,年五十七。贈禮部尚書,謚曰修懿。



 藻字醇老,明逸之從子也。幼孤,刻厲為學。第進士,又中賢良方正科,為秘閣校理。



 慈聖後臨朝,藻三上書乞還政。同修起居注、知制誥。加樞密直學士、知開封府。平居樂易無崖岸,而居官獨立守繩墨,為政簡靜有條理,不肯徇私取顯。數求退,改翰林侍讀學士、知審官東院。卒,年六十一。神宗知其
 貧,賻錢五十萬,贈太中大夫。



 景諶,景臻之從兄也。繇殿直巡轄兩京馬遞,中進士第。初赴開封解試,時王安石得其文,以為知道者。既薦送之,又推譽於公卿間,自是執弟子禮。安石提點府界,景諶為屬主簿,又以文薦之。執喪居許,聞安石得政,喜,因事來京師謁之。方盛夏,安石與僧智緣臥於地,一最親者袒坐其側。顧景諶褫服脫帽,未及它語,卒然問曰:「青苗、助役如何?」景諶曰:「利少害多,異日必為民患。」又問:「孰
 為可用之人?」曰:「居喪不交人事,而知人尤難事也。」遂辭出。



 後調官復來,安石已作相,又往詣之。安石令先與弟安國相見。安國亦與之善,謂景諶曰:「相君欲以館閣相處而任以事。」景諶曰:「百事皆可為,所不知者新書、役法耳。」及見安石,安石欲令治峽路役書,且委以戎、滬蠻事。景諶曰:「峽路民情,僕固不能知;而戎、滬用兵,系朝廷舉動、一路生靈休戚,願擇知兵愛人者。」安石大怒,坐上客數十人,皆為之懼。退就謁舍,賞激之與詆以為矯者參
 半。景諶笑曰:「自古以來,好利者眾,而顧義者寡,故天下萬事,皆由人而不在於己。茍為利所動,而由於人,則盜亦可為也。夫盜之所以為盜者,利勝於義,而不知所以為之者耳。吾又何憾焉?」遂與安石絕。熙寧末,從張景憲闢知瀛州,終身為外官,僅至朝請郎而卒。



 勰字穆父,彥遠之子也。生五歲,日誦千言。十三歲,制舉之業成。熙寧三年試應,既中秘閣選,廷對入等矣,會王安石惡孔文仲策,遷怒罷其科,遂不得第。以蔭知尉氏
 縣,授流內銓主簿。判銓陳襄嘗登進班簿,神宗稱之。襄曰:「此非臣所能,主薄錢勰為之耳。」明日召對,將任以清要官。安石使弟安禮來見,許用為御史。勰謝曰:「家貧母老,不能為萬里行。」安石知不附己,命權鹽鐵判官,歷提點京西、河北、京東刑獄。元豐定官制,勰方居喪。帝於左司郎中格自書其姓名,須終制日授之。



 奉使吊高麗,外意頗謂欲結之以北伐。勰入請使指,帝曰:「高麗好文,又重士大夫家世,所以選卿,無他也。」乃求呂端故事以行,
 凡饋餼非故所有者皆弗納。歸次紫燕島,王遣二吏追餉金銀器四千兩。勰曰:「在館時既辭之矣,今何為者?」吏泣曰:「王有命,徒歸則死,且左番已受。」勰曰:「左右番各有職,吾唯例是視,汝可死,吾不可受。」竟卻之。還,拜中書舍人。



 元祐初,遷給事中,以龍圖閣待制知開封府。老吏畏其敏,欲困以事,導人訴牒至七百。勰隨即剖決,簡不中理者,緘而識之,戒無復來。閱月聽訟,一人又至,呼詰之曰:「吾固戒汝矣,安得欺我?」其人讕曰:「無有。」勰曰:「汝前訴
 云云,吾識以某字。」啟緘示之,信然,上下皆驚吒。宗室、貴戚為之斂手,雖丞相府謁吏幹請,亦械治之。積為眾所憾,出知越州,徙瀛州。召拜工部、戶部侍郎,進尚書,加龍圖閣直學士,復知開封,臨事益精。蘇軾乘其據案時遺之詩,勰操筆立就以報。軾曰:「電掃庭訟,響答詩筒,近所未見也。」



 哲宗蒞政,翰林缺學士,章惇三薦林希,帝以命勰,仍兼侍讀。以嘗行惇謫詞,懼而求去。帝曰:「豈非『鞅鞅非少主之臣,硜硜無大臣之節』者乎?朕固知之,毋庸避
 也。」嘗侍經幄,帝留與之語曰:「臺臣論徐邸事,其辭及鄭、雍,小人離間骨肉如此。若雍有請,當付卿以美詔慰安之。」既而雍章至,勰答詔云:「弗容群枉,規欲動搖,朕察其厚誣,力加明辨,夫何異趣,乃爾乞身。」帝見之,謂能道所欲言者。惇因是極意排詆,諷全臺攻之,言不已。罷知池州,卒於官,年六十四。訃未至,帝猶即其從弟景臻問安否。元符末,追復龍圖閣學士。



 即字中道,吳越王諸孫也。第進士,為睦州推官。部使者
 有獄在衢,啖即以薦牘,使往治。即曰:「吾寧老冗選中,豈忍以數十人易一薦乎?」至,則平反之。闢鄜延幕府。崇寧中,為陜西轉運判官。王師復銀州,轉餉最。徽宗召對,問曰:「靈武可取乎?」對曰:「夏人去來飄忽,不能持久,是其所短;然其民皆兵,居不縻飲食,動不勤轉餉,願敕邊臣先為不可勝以待釁,庶可得志。」帝曰::「大砦泉可取否?」對曰:「是所謂瀚海也。臣聞其地皆舄鹵,無水泉,或以飲馬,口鼻皆裂,正得之無所用。」帝然之。



 除直龍圖閣、知慶州。至
 鎮,築安邊城,歸德堡,包地萬頃,縱耕其中,歲得粟數十萬。徙知延安府,加集賢殿修撰,又進徽猷閣待制、顯謨閣直學士。在延五年,童貫宣撫陜西,得便宜行事。時長安百物踴貴,錢幣益輕,貫欲力平之,計司承望風旨,取市價率減什四,違者重置於法,民至罷市。徐處仁爭之,得罪。又行均糴法,賤入民粟,而高金帛估以賞,下至蕃兵、射士之授田者,咸被抑配,關內騷然,幾於生變。即亦屢抗章,極陳其害,貶永州團練副使,然糴害亦寢。



 數月,
 還待制、知興仁府,從太原,以童貫宣撫本道辭,不許。居二年,以疾提舉洞霄宮,復真學士。睦寇作,起知宣州。即自力上道,至則悉意應軍須。貫上其功,進龍圖閣學士。貫遂引為河北、河東參謀,以老固辭,乃轉正奉大夫致仕。卒,贈金紫光祿大夫,謚曰忠定。



 論曰:進士自鄉舉至廷試皆第一者才三人,王曾、宋庠為名宰相,馮京為名執政,風節相映,不愧其科名焉。邵亢知太常,裁損張貴妃恤典,穎王授室、公主下嫁,請用
 古典,可謂不愧其官守矣。邵必亦習禮者也,預修《唐書》而能力辭,以為史出眾手,非古人撰述之體,豈非名言乎?錢惟演敏思清才,著稱當時,然急於柄用,阿附希進,遂喪名節。錢氏三世制科,易、明逸皆掌書命,時人榮之。惜乎易以輕俊,明逸以傾。



\end{pinyinscope}