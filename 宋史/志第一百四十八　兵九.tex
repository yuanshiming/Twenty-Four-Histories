\article{志第一百四十八 兵九}

\begin{pinyinscope}

 訓練之制



 訓練之制禁軍月奉五百以上,皆日習武技。三百以下,或給役,或習技。其後別募廂兵,亦閱習武技,號教閱廂軍。戍川、廣者舊不訓練,嘉祐以後稍習焉。凡諸日習之
 法,以鼓聲為節,騎兵五習,步兵四習,以其坐作進退非施於兩軍相當者然。自宋初以來,中外諸軍皆用之。



 明道二年,樞密使王曙言:「天下廂軍止給役而未嘗習武技,宜取材勇者訓肄,升補禁軍。」上可其奏。



 康定元年,帝御便殿閱諸軍陣法。議者謂諸軍止教坐作進退,雖整肅可觀,然臨敵難用,請自今遣官閱陣畢,令解鐙以弓弩射。營置弓三等,自一石至八斗;弩四等,自二石八斗至二石五斗,以次閱習。詔行之陜西、河東、河北路。是
 歲,詔:「教士不衽帶金革,緩急不足以應敵。自今諸軍各予鎧甲十、馬甲五,令迭披帶。」又命諸軍班聽習雜武技,勿輒禁止。



 慶歷元年,徙邊兵不教者於內郡,俟習武技即遣戍邊。



 二年,諸軍以射親疏為賞罰,中的者免是月諸役,仍籍其名。闕校長,則按籍取中多者補。樞密直學士楊偕請教騎兵止射九斗至七斗三等弓,畫的為五暈,去的二十步,引滿即發,射中者,視暈數給錢為賞。騎兵佩劈陣刀,訓肄時以木桿代之。奏可。



 四年,詔:「騎兵帶甲
 射不能發矢者,奪所乘馬與本營藝優士卒。」韓琦言:「教射唯事體容及強弓,不習射親不可以臨陣。臣至邊,嘗定弓弩挽強、跖硬、射親格,願行諸軍立賞肄習。歲以春秋二時各一閱,諸營先上射親吏卒之數,命近臣與殿前、馬步軍司閱之。其射親入第四至第七等,量先給賜;入第三等已上及挽強、跖硬中格,悉引對親閱;等數多者,其正副指揮使亦第賜金帛。」詔以所定格班教諸軍。四年,遣官以陜西陣法分教河北軍士。



 五年,密詔益、利、
 梓、夔路鈐轄司,以弓弩習士卒,候民間觀聽浸熟,即便以短兵日教三十人,十日一易。知並州明鎬言:「臣近籍諸營武藝之卒,使帶甲試充奇兵外,為三等,庶幾主將悉知軍中武技強弱,臨敵可用。」詔頒其法三路。範仲淹請以帶甲射一石充奇兵,餘自九斗至七斗第為三等,射力及等即升之。詔著為令。



 六年,詔諸軍夏三月毋教弓弩,止習短兵。又詔:「以春秋大教弓射一石四斗、弩擴三石八斗、槍刀手勝三人者,立為武藝出眾格。中者,本
 營闕階級即以次補。」



 至和元年,詔:「諸軍選將校,武藝鈞,以射親為上。」韓琦又言:「奉詔,軍士弩擴四石二斗並弓箭、槍手應舊規選中者,即給挺補守闕押官,然則排連舊制為虛文矣。請三路兵遇春秋大教,武技出眾者優給賞物,免本營他役,候階級闕,如舊制選補。」奏可。



 治平二年,詔:「河北戰卒三十萬一千、陜西四十五萬九百並義勇等,委總管司訓練,毋得冗占。」



 熙寧元年,詔曰:「國家置兵以備戰守,而主兵之官冗占者眾,肄習弗時,或誤
 軍事。帥臣、安撫、監司其察所部有占兵不如令者以聞。」十月,樞密院請陜西、河東選三班使臣及士人任殿侍者,以為河北諸路指使,教習騎軍。或言河朔兵有教閱之名而無其實,請班教法於其軍,久而弗能者,罷為廂軍。奏可。



 二年,帝常語執政:「並邊訓練士卒,何以得其精熟?」安石對曰:「京東所教兵已精強,願陛下推此法以責邊將,間詔其兵親臨閱試。訓練簡閱有不如詔者罰之,而賞其能者。賞不遣賤,罰不避貴,則法行而將吏加勸,
 士卒無不奮勵矣。」九月,選置指使巡教諸軍,殿前司四人,馬、步軍司各三人。



 三年,帝親閱河東所教排手,進退輕捷,不畏矢石。遂詔殿前司,步軍指揮當出戍者,內擇槍刀手伉健者百人,教如河東法,藝精者免役使,以優獎之。



 五年四月,詔在京殿前馬步諸軍巡教使臣,並以春秋分行校試。射命中者第賜銀楪,兵房置籍考校,以多少定殿最。五月,詔以涇原路蔡挺衙教陣隊於崇政殿引見,仍頒諸路。其法:五伍為隊,五隊為陣,陣橫列,騎
 兵二隊亦五伍列之。其出皆以鼓為節,束草像人而射焉,中者有賞。馬步皆前三行槍刀,後二行弓弩,附隊以虎蹲弩、床子弩各一,射與擊刺迭出,皆聞金即退。預籍人馬之強者隱於隊中,遇可用,則別出為奇。帝以其點閱周悉,常有出野之備,故令頒行。



 六年,詔:「河北四路承平日久,重於改作,茍遂因循,益隳軍制。其以京東武衛等六十二營隸屬諸路,分番教習,餘軍並分遣主兵官訓練。」九月,詔:「自今巡教使臣校殿最,雖以十分為率,其事
 藝第一等及九分已上,或射親及四分,雖殿,除其罰;第二等事藝及八分,或射親不及三分,雖最,削其賞。」十月,選涇原士兵之善射者,以教河朔騎軍馳驟野戰。帝曰:「裁並軍營,凡省軍員四千餘人,此十萬軍之資也。儻訓練精勇,人得其用,不惟勝敵,亦以省財。」安石等曰:「陛下頻年選擇使臣,專務訓練,間御便殿躬親試閱,賞罰既明,士卒皆奮。觀其技藝之精,一人為數夫之敵,此實國家安危所系也。」是時,帝初置內教法,旬一御便殿閱武,
 校程其能否而勸沮之,士無不爭勸者。



 七年,詔教閱戰法,主將度地之形,隨宜施行。二月,詔:「自今歲一遣使,按視五路安撫使以下及提舉教閱諸軍、義勇、保甲官,課其優劣以聞而誅賞之。」



 八年,詔:「在京諸軍營屯迫隘,馬無所調習。比創四教場,益寬大,可以馳騁。其令騎軍就教者,日輪一營,以馬走驟閱習。」五月,臧景陳馬射六事:一、順□□直射,二、背射,三、盤馬射,四、射親,五、野戰,六、輪弄,各為說以曉射者。詔依此教習。八月,帝令曾孝寬視教
 營陣。大閱八軍陣於荊家陂,訖事大賞。



 元豐元年十月,詔立在京校試諸軍技藝格,第為上中下三等。步射,六發而三中為一等,二中為二等,一中為三等。馬射,五發驟馬直射三矢、背射二矢,中數、等如步射法。弩射,自六中至二中,床子弩及炮自三中至一中,為及等。並賞銀有差。槍刀並標排手角勝負,計所勝第賞。其弓弩墜落,或縱矢不及堋,或挽弓破體,或局而不張,或矢不滿,或弩跖不上牙,或擭不發,或身倒足落,並為不合格。即射
 已中賞,餘箭不合格者,降一等,無可降者罷之。



 是月,賈逵、燕達等言:「近者增損東南排弩隊法,與東南所用兵械不同,請止依東南隊法,以弩手代小排。若去敵稍遠則施箭,近則左手持弩如小排架隔,右手執刀以備斬伐,與長兵相參為用。」詔可,其槍手仍以標兼習。十一月,京西將劉元言:「馬軍教習不成,請降步軍,又不成,降廂軍。」乃下令諸軍,約一季不能學者,如所請降之。十二月,詔:「開封府界、京東西將兵,十人以一人習馬射,受教於
 中都所遣教頭。在京步軍諸營弓箭手,亦十人以一人習馬射,受教於教習馬軍所。藝成,則展轉分教於其軍。」



 二年四月,遣內侍石得一閱視京西第五將所教馬軍。五月,得一言其教習無狀,詔本將陳宗等具折。宗等引罪,帝責曰:「朝廷比以四方驕悍為可虞,選置將臣分總禁旅,俾時訓肄,以待非常。至於部勒規模,悉經朕慮,前後告戒已極周詳。使宗等稍異木石,亦宜略知人意。尸祿日久,既頑且慵,茍遂矜寬,實難勵眾,可並勒停。」是
 月,詔殿前、步軍司兵各置都教頭掌隸教習之事,弩手五營、弓箭手十營、槍刀標排手五營各選一人武藝優者奏補。逐司各舉散直一人為指使,罷巡教使臣。是日,詔河東、陜西諸路:「舊制,馬軍自十月一日馳射野戰,至穀雨日止。塞上地涼,自今教起八月,止五月一日。」七月,詔諸路教閱禁軍毋過兩時。九月,內出教法格並圖象頒行之。步射執弓、發矢、運手、舉足、移步,及馬射、馬使蕃槍、馬上野戰格鬥,步用標排,皆有法象,凡千餘言,使軍士
 誦習焉。



 四年五月,詔東南諸路轉運、提點刑獄司,體量將兵自降教閱新法之後,軍士有所倍費以聞。蓋自團立將兵以來,軍人日新教閱,舊資技藝以給私費者,悉無暇為故也。



 六年,從郭忠紹之請,步軍弩手第一等者,令兼習神臂弓。



 七年八月,詔開封府界、京東西路專選監司提舉教閱。神宗留心武備,既命立武學、校《七書》以訓武舉之士,又頒兵法以肄軍旅,微妙淵通,取成於心,群臣莫望焉。



 元祐元年四月,右司諫蘇轍上言:「諸道禁
 軍自置將以來,日夜按習武藝,將兵皆蚤晚兩教,新募之士或終日不得休息。今平居無事,朝夕虐之以教閱,使無遺力以治生事,衣食殫盡,憔悴無聊,緩急安得其死力?請使禁軍,除新募未習之人,其餘日止一教。是月,朝請郎任公裕言:「軍中誦習新法,愚懵者頗以為苦。夫射志於中,而擊刺格鬥期於勝,豈必盡能如法?」樞密院亦以為元降教閱新法自合教者指授,不當令兵眾例誦。詔從之。九月,樞密院奏:「異時馬軍教御陣外,更教馬
 射。其法:全隊馳馬皆重行為『之』字,透空發矢,可迭出,最便利。近歲專用順鬃直射、抹秋背射法,止可輕騎挑戰,即用眾乃不能重列,非便。請自今營閱排日,馬軍『之』字射與立背射,隔日互教。」詔可。



 三年五月,罷提舉教習馬軍所。



 六年六月,三衙申樞密院,乞近伏七十日依令式放諸軍教。王嚴叟白韓忠彥曰:「景德故事,皆內侍省檢舉傳宣,今但歲舉為常,則不復見朝廷恩意。」忠彥以為然,又開陳太皇太后。曰:「如此則為常事,待處分內侍
 省。」遂詔:「今後入狀,遣中侍傳宣諸軍住教。」



 紹聖元年三月,樞密院言:「禁軍春秋大教賞法,每千人增取二百一十人,給賞有差。」從之。



 二年二月,樞密院言:「馬軍自九月至三月,每十日一次出城水率渲,教習回答野戰走驟向背施放,遇風雪假故權住。」從之。



 三年五月,詔在京、府界諸路禁軍格鬥法,自今並依元豐條法教習。七月,詔選弩手兼習神臂弓。八月,詔:「殿前、馬步軍司見管教頭,別選事藝精強、通曉教像體法者,展轉教習。其弓箭手馬、步
 射射親,用點藥包指及第二指知鏃,並如元豐格法。」是月,又詔復神臂弓射法為百二十步。



 元符元年十月,曾布既上巡教使臣罰格,因言:「祖宗以來,御將士常使恩歸人主,而威令在管軍。凡申嚴軍政,豈待朝廷立法而後施行耶?是管軍失職矣。」帝深以為然。



 政和元年二月,詔:「春秋大教,諸軍弓弩鬥力,並依元豐舊制。」



 四年五月,臣僚上言:「神臂弓垛遠百二十步,給箭十隻,取五中為合格,軍中少得該賞,恐惰於習射。送殿前、
 馬步軍司勘會,將中貼箭數並改為上垛,其一中貼此兩上垛。」從之。



 五年三月,詔:「自今敢占留將兵,不赴教閱,並以違御筆論。不按舉者,如其罪。」十一月,臣僚言:「春秋大教,諸軍弓弩上取鬥力高強,其射親中多者,激賞太薄,無以為勸。」詔依元豐法。



 八年,詔州郡禁軍出戍外,常留五分在州教閱,從毛友之請也。



 重和元年正月,而兵部侍郎宇文粹中進對,論禁軍訓練不精,多充雜役。帝曰:「祖宗軍旅之法最為密致,神考尤加意訓習,近來兵官浸以弛慢。
 古者春振旅,夏茇舍,秋治兵,冬大閱,皆於農隙以講事,大司馬教戰之法,大宗伯大田之禮,細論周制,大抵軍旅之政,六卿無有不總之者。今士人作守倅,任勸農事,不以勸耕稼為職;管軍府事,不以督訓練為意。自今如役使班直及禁衛者,當差人捉探懲戒。更候日長,即親御教閱激賞。」尋以粹中所奏參照條令行之。



 宣和三年四月,立騎射賞法,其背射上垛中貼者,依步射法推賞。



 靖康元年二月,詔:「軍兵久失教習,當汰冗濫。今三衙與諸將
 招軍,惟務增數希賞,但及等杖,不問勇怯。招收既不精當,教習又不以時,雜色占破,十居三四。今宜招兵之際,精加揀擇,既系軍籍,專使教習,不得以雜色拘占。又神臂弓、馬黃弩乃中國長技,宜多行教習,以捍邊騎。仍令間用衣甲教閱,庶使習熟。」四月,詔復置教場,春秋大閱,及復內教法以激賞之。



 陣法熙寧二年十一月,趙離乞講求諸葛亮八陣法,以授邊將,使之應變。詔郭逵同離講求,相度地形,定為陣圖聞奏。



 五年四月,詔蔡挺先
 進教閱陣圖。帝嘗謂:「今之邊臣無知奇正之體者,況奇正之變乎!且天地五行之數不過五,五陣之變,出於自然,非強為之。」宰相韓絳因請諸帥臣各具戰陣之法來上,取其所長,立以為法。從之。帝患諸將軍行無行陣之法,嘗曰:「李靖結三人為隊必有意。星書,羽林皆以三人為隊,靖深曉此,非無據也。」乃令賈逵、郭固試之。十二月,知通遠軍王韶請降合行條約,詔賜禦制《攻守圖》、《行軍環珠》、《武經總要》、《神武秘略》、《風角集占》、《四路戰守約束》各
 一部,餘令關秦鳳路經略司抄錄。



 六年,詔諸路經略司,結隊並依李靖法,三人為一小隊,九人為一中隊,賞罰俟成序日取裁。其隊伍及器甲之數,依涇原路牙教法。九月,趙離言:「欲自今大閱漢蕃陣隊,且以萬二千五百人為法,旌旗麾幟各隨方色。戰國時,大將之旗以龜為飾,蓋取前列先知之義。令中軍亦宜以龜為號。其八隊旗,別繪天、地、風、雲、龍、虎、鳥、蛇。天、地則象其方園,風、雲則狀其飛揚,龍、虎則狀其猛厲,鳥、蛇則狀其翔盤之勢,以備大閱。」樞密院以
 為陣隊旗號若繪八物,應士眾難辨,且其間亦有無形可繪者。遂詔止依方色,仍異其形制,令勿雜而已。



 七年,又命呂惠卿、曾孝寬比校三五結隊法。十月,以新定結隊法並賞罰格及置陣形勢等,遣近侍李憲付趙離曰:「陣法之詳已令憲面諭,今所圖止是一小陣,卿其從容析問,憲必一一有說。然置陣法度,久失其傳,今朕一旦據意所得,率爾為法,恐有未盡,宜無避忌,但具奏來。」繼又詔曰:「近令李憲繼新定結隊法並賞罰格付卿,同議
 可否,因以團立將官,更置陣法,卿必深悉朝廷經畫之意。如日近可了,宜令李憲繼赴闕。」離奏曰:



 置陣之法,以結隊為先。李靖以五十人為一隊,每三人自相得者結為一小隊,合三小隊為一中隊,合五中隊為一大隊,餘押官、隊頭、副隊頭、左右傔旗五人即充五十,並相依附。今聖制:每一大隊合五中隊,五十人為之;中隊合三小隊,九人為之;小隊合三人為之,亦擇心意相得者。又選壯勇善槍者一人為旗頭,令自擇如己藝、心相得者二
 人為左右傔;次選勇悍者人為引戰;又選軍校一人執刀在後,為擁隊。凡隊內一人用命,二人應援;小隊用命,中隊應援;中隊用命,大隊應援;大隊用命,小隊應援。如逗撓觀望不即赴救,致有陷失者,本隊委擁隊軍校,次隊委本轄隊將,審觀不救所由,斬之。其有不可救,或赴救不及,或身自受敵,體被重創,但非可救者,皆不坐。其說雖與古同,而用法尤為精密。此蓋陛下天錫勇智,不學而能也。



 然議者謂四十五人而一長,不若五人而一
 長之密。且以五人而一長,即五十人而十長也,推之於百千萬,則為長者多,而統制一也。至如周制:五人為伍,屬之比長;五伍為兩,屬之閭胥;四兩為卒,屬之族師;五卒為旅,屬之黨正;五旅為師,屬之州長;五師為軍,屬之命卿。此猶今之軍制,百人為都,五都為營,五營為軍,十軍為廂。自廂都指揮使而下,各有節級,有員品,亦昔之比長、閭胥、族師、黨正之任也。



 議者謂什伍之制,於都法為便,然都法恐非臨陣對敵決勝之術也。況八陣之
 法,久失其傳,聖制一新,稽之前聞,若合符節。夫法一定,易以致人。敵好擊虛,吾以虛形之;敵好背實,吾以實形之。然而所擊者非其虛,所背者非其實,故逸能勞之,飽能饑之,此所謂致人而不致於人也。



 七年七月,詔諸路安撫使各具可用陣隊法,及訪求知陣隊法者以聞。九月,崇儀使郭固以同詳定古今陣法賜對,於是內出《攻守圖》二十五部付河北。



 八年二月,帝批:「見校試七軍營陣,以分數不齊,前後抵牾,難為施用。可令見校試官摭
 其可取者,草定八軍法以聞。」初,詔樞密院曰:「唐李靖兵法,世無全書,雜見《通典》,離析訛舛。又官號物名與今稱謂不同,武人將佐多不能通其意。令樞密院檢詳官與王震、曾收、王白、郭逢原等校正,分類解釋,令今可行。」又命樞密院副都承旨張誠一、入內押班李憲與震、逢原行視寬廣處,用馬步軍二千八百人教李靖營陣法。以步軍副都指揮使楊遂為都大提舉,誠一、憲為同提舉,震、逢原參議公事,夏元象、臧景等為將副、部隊將、乾當
 公事,凡三十九人。



 誠一等初用李靖六花陣法,約受兵二萬人為率,為七軍,內虞候軍各二千八百人,取戰兵千九百人為七十六隊,戰兵內每軍弩手三百,弓手三百,馬軍五百,跳蕩四百,奇兵四百,輜重每軍九百,是為二千八百人。帝諭近臣曰:



 黃帝始置八陣法,敗蚩尤於涿鹿。諸葛亮造八陣圖於魚復平沙之上,壘石為八行。晉桓溫見之,曰:「常山蛇勢。」此即九軍陣法也。至隋韓擒虎深明其法,以授其甥李靖。靖以時遇久亂,將臣通
 曉者頗多,故造六花陣以變九軍之法,使世人不能曉之。大抵八陣即九軍,九軍者,方陣也。六花陣即七軍,七軍者,圓陣也。蓋陣以圓為體,方陣者內圓而外方,圓陣即內外俱圓矣。故以方圓物驗之,則方以八包一,圓以六包一,此九軍六花陣之大體也。六軍者,左右虞候軍各一,為二虞候軍;左右廂各二,為四廂軍;與中軍共為七軍。八陣者,加前後二軍,共為九軍。開國以來,置殿前、馬步軍三帥,即中軍、前後軍帥之別名;而馬步軍都虞候是
 為二虞候軍,天武、捧日、龍神衛四廂是為四廂軍也。中軍帥總制九軍,即殿前都虞候,專總中軍一軍之事務,是其名實與古九軍及六花陣相符而不少差也。今論兵者俱以唐李筌《太白陰經》中陣圖為法,失之遠矣。



 朕嘗覽近日臣僚所獻陣圖,皆妄相眩惑,無一可取。果如其說,則兩敵相遇,必須遣使豫約戰日,擇寬平之地,夷阜塞壑,誅草伐木,如射圃教場,方可盡其法爾。以理推之,其不可用決矣。今可約李靖法為九軍營陣之制。然
 李筌圖乃營法,非陣法也。朕採古之法,酌今之宜,曰營曰陣,本出於一法,特止曰營,行曰陣;在奇正言之,則營為正、陣為奇也。



 於是以八月大閱八軍陣於城南荊家陂。已事,賜遂而下至指使、馬步軍銀絹有差。



 八年,詔諸路權住教五軍陣,止教四御陣。



 九年四月,帝於輔臣論營陣法,謂:「為將者少知將兵之理,且八軍、六軍皆大將居中,大將譬則心也,諸軍,四體也。運其心智,以身使臂,以臂使指,攻左則右救,攻右則左救,前後亦然,則軍何
 由敗也!」



 元豐四年,以九軍法一軍營陣按閱於城南好草陂,已事,獎諭。



 七年,詔:「已降五陣法,令諸將教習,其舊教陣法並罷。」蓋九軍營陣為方、圓、曲、直、銳,凡五變,是為五陣。



 元祐元年,高翔言,乞以御陣與新陣法相兼教閱,從之。蓋元豐七年,詔專用五陣法,而舊教御陣遂廢;至是,復令互教。



 紹聖三年,復罷教御陣。



 大觀二年,詔以五陣法頒行諸路。



 靖康元年,監察御史胡舜陟奏:「通直郎秦元所著兵書、陣圖、師律三策、大八陣圖一、小圖
 二,皆酌古之法,參今之宜,博而知要,實為可用。」詔令賜對。當時君臣雖無雄謀遠略,然猶切切焉以經武為心。



 高宗建炎元年,始頒樞密院教閱法,專習制御摧鋒破敵之藝、全副執帶出入、短樁神臂弓、長柄刀、馬射穿甲、木挺。每歲擬春秋教閱法,立新格。神臂弓日給箭二十,射親去垛百二十步。刀長丈二尺以上,氈皮裹之,引斗五十二次,不令刀頭至地。每營選二十人閱習,經兩閱者五十人為一隊,教習分合,隨隊多少,分隸五軍。每軍
 各置旗號,前軍緋旗,飛鳥為號;後軍皂旗,龜為號;左軍青旗,蛟為號;右軍白旗,虎為號;中軍黃旗,神人為號。又別以五色物號制招旗、分旗。舉招旗,則五軍以旗相應,合而成陣;舉分旗,則五軍以旗相應,分而成隊。左右前卻,或分藏為伏,或分出為奇,皆舉旗為號。更鳴小金、應鼓,備瞻望不及者。豫約伏藏之所,緩鳴小金即止,急鳴應鼓即奇兵出陣趍戰,急鳴小金即伏兵出。其春秋大教推賞,依海行格法。



 李綱言:「水戰之利,南方所宜。沿河、
 淮、海、江帥府、要郡,宜效古制造戰船,以運轉輕捷安穩為良。又習火攻,以焚敵舟。」詔命楊觀復往江、浙措置,河、淮別委官。三年,親閱水軍於鎮江登雲門外。



 紹興四年,詔內殿按閱神武中軍官兵推賞。



 二十四年,臣僚言:「州郡禁卒,遠方縱馳,多不訓練,春秋教閱,臨時備數,乞申嚴舊制。」



 三十一年,詔:「比聞諸路州廂、禁軍、土軍,有司擅私役,妨教閱。帥府其嚴責守兵勤兵歸營,訓練精熟,以備點視。」



 孝宗乾道二年,幸候潮門外,次幸白石閱兵,三
 衙率將佐道駕,射生官兵就御輦下獻所獲。是日,有數將獨手運大刀,上曰:「刀重幾何?」李舜舉奏:「刀皆重數十斤。」有旨:「卿等教閱精明。」又諭陳敏曰:「軍馬衣裝整肅如此。」特錫賚鞍馬、金帶,士卒推賞有差。



 四年,幸茅灘教閱。舉黃旗,連三鼓,變方陣;五鼓,舉白旗,變圓陣;次二鼓,舉赤旗,變銳陣;青旗,變直陣。畢事,上大悅,賞賚加倍。兵分東西,呈大刀、火炮,上問李舜舉:「按閱比曩時如何?」舜舉奏:「今日之兵,陛下親訓練,撫以深恩,錫以重賞,忠勇倍常。」



 乾道中,詔弓箭手元射一石四斗力升加三斗,元射一石力升加五斗,弩手元射四石力升加五斗,元射兩石七斗力升加八斗,進秩推賞有差。宰執進射親賞格,虞允文曰:「拍試以鬥力升請給,今用射親定賞,恐不加意鬥力。」上曰:「然。他日雖強弓弩可以取勝,若止習射親,則鬥力不進。此賞格不須行。」



 淳熙間,立槍手及射鐵簾格。上謂輔臣曰:「聞射鐵簾,諸軍鼓躍奮厲。」周必大曰:「兵久不用,此輩無進取,自然氣惰。今陛下激勸告戒,人人皆
 勝兵。」於是殿前、步軍司諸軍及馬軍舊司弓弩手,射鐵簾合格兵共一千八百四十餘。詔中垛簾弓箭手一石二斗力十箭,弩手四石力八箭,依格進兩秩,各賜錢百緡;弓箭手一石力十箭以上,弩手三石力八箭,各進兩秩。詔中外諸軍賞格亦如之。



 紹熙元年,詔殿司:「許浦水軍並江上水軍歲春、秋兩教外,每月輪閱習。沿海水軍準是。」知徽州徐誼言:「諸路禁軍,近法以十分為率,二分習弓,六分習弩,餘二分習槍、牌。習弓者聽兼習弩,鬥
 力可以觀其進退,射親可以察其能否。勤惰之實,人有稽考。」詔下諸路遵守之。執政胡晉臣言:「比年用射鐵簾推賞,往往獲遷秩,是亦足以作成人才。」上曰:「射鐵簾不難,此賞格太濫,其專以武藝精熟為尚。」



 二年,樞密院言:「殿、步司諸軍弓箭手,帶甲六十步射,一石二斗力,箭十二,六箭中垛為本等。弩手,帶甲百步射,四石力,箭十二,五箭中垛為本等。槍手,駐足舉手攛刺,以四十攛為本等。主帥委統制、統領較其藝。本等外取升加多者,每軍五
 千五百人以上弓、弩、槍手各十五人,詣主帥審實,上樞密院覆試。各擇優等二人升轉兩秩,餘人給錢五緡,俟將來再試。」



 慶元二年,幸候潮門外大閱。



 嘉泰二年,詔將按閱諸軍,賞賚依慶元二年增給。



 寶慶二年,莫澤言:「州郡禁軍,平時則以防寇盜,有事則以備戎行,實錄於朝廷,非州郡可得私役。比年州郡軍政隳廢,吝於廩給,闕額恆多。郡官、主兵官有窠占,寓公有借事,存留者不什一。當教閱時,鈐、總、路分雖號主兵,僅守虛籍,莫敢號召。
 入教之次,坐作進退殆同兒戲。守臣利虛券不招填,主兵受厚賂改年甲。且一兵請給,歲不下百緡,以小計之,一郡占三百人,是虛費三萬緡也。私役禁軍,素有常憲。守帥闢園池,建第宅,不給餐錢;寓公去城遼絕,類得借兵,擾害鄉閭。近而輔郡至有寓公占四五百兵者。良由兵官之權輕,而私占之禁弛也。乞嚴戒監司、守倅等,止許借廂軍,仍不得妨教閱,餘官雖廂軍亦勿借。」



 淳祐十一年,臺臣條陳軍匠不閑閱習之弊:「按舊制,禁兵毋私
 役。比歲凡州軍屯營駐扎之處,多循舊習,每一州軍匠無慮數百,官無小大各占破,而雕鏤、組繡、攻金、設色之事靡所不有。工藝雖精,擊刺不習,設有小警,何能授甲?乞申嚴帥守及統兵官,應軍匠聽歸營伍閑習訓練,勿競作無益,虛糜廩稍,以妨軍實。」



 咸淳初,臣僚言:「諸軍統領、統制、正將、副將正欲在軍訓練,閑於武事,一有調用,令下即行,士悉將智,將悉士勇,所向無敵。今江南州郡、沿江制閫置帳前官,專任營運,不為軍計,實為家謀,絕
 無戰陣新功,率從帳前升差。大略一軍僅二三千,而使臣至五六百,以供雜役。」



 九年,臣僚言:「比者招募軍兵,一時徒取充數,以覬賞格。涅刺之後,更不教閱。主兵官苦以勞役,日夜罔休,一或少違,即罹囹圄,榜掠之酷,兵不堪命,而死者逃者接踵也。今請以新招軍分隸諸隊,使之熟紀律,習事藝,或旬或月上各郡閱試。」蓋弊至於此,而訓練之制大壞矣。



\end{pinyinscope}