\article{志第六十六 禮十六(嘉禮四)}

\begin{pinyinscope}

 宴
 饗游觀賜酺



 宴饗之設,所以訓恭儉、示惠慈也。宋制,嘗以春秋之季仲及聖節、郊祀、籍田禮畢,巡幸還京,凡國有大慶皆大
 宴,遇大災、大札則罷。天聖後,大宴率於集英殿,次宴紫宸殿,小宴垂拱殿,若特旨則不拘常制。凡大宴,有司預於殿庭設山樓排場,為群仙隊仗、六番進貢、九龍五鳳之狀,司天雞唱樓於其側。殿上陳錦繡帷帟,垂香球,設銀香獸前檻內,藉以文茵,設御茶床、酒器於殿東北楹,群臣盞斝於殿下幕屋。設宰相、使相、樞密使、知樞密院、參知政事、樞密副使、同知樞密院、宣徽使、三師、三公、僕射、尚書丞郎、學士、直學士、御史大夫、中丞、三司使、給、
 諫、舍人、節度使、兩使留後、觀察、團練使、待制、宗室、遙郡團練使、刺史、上將軍、統軍、軍廂指揮使坐於殿上,文武四品以上、知雜御史、郎中、郎將、禁軍都虞候坐於朵殿,自餘升朝官、諸軍副都頭以上、諸蕃進奉使、諸道進奉軍將以上分於兩廡。宰臣、使相坐以繡墩;曲宴行幸用杌子。



 參知政事以下用二蒲墩,加罽兟;曲宴,樞密使、副並同。



 軍都指揮使以上用一蒲墩;自朵殿而下皆緋緣氈條席。殿上器用金,餘以銀。其日,樞密使以下先起居訖,當侍立者升殿。
 宰相率百官入,宣徽、閣門通唱,致辭訖,宰相升殿進酒,各就坐,酒九行。每上舉酒,群臣立侍,次宰相、次百官舉酒;或傳旨命酹,即搢笏起飲,再拜。曲宴多令不拜。



 或上壽朝會,止令滿酌,不勸。中飲更衣,賜花有差。宴訖,蹈舞拜謝而退。



 建隆元年,大宴於廣德殿,酒九行而罷。乾德元年十一月,南郊禮成,大宴廣德殿,謂之飲福。是後三年,開寶三年、五年、六年、七年、八年,並設秋宴於大明殿,以長春節在二月故也。太平興國之後,止設春宴,在大明者十
 一,在含光者六。宴日,親王、樞密使副、宣徽、三司使、駙馬都尉皆侍立,軍校自龍武四廂都指揮使以上立於庭。其宴契丹使亦於崇德殿,但近臣及刺史、郎中而上預焉。



 淳化四年正月,以南郊禮成,大宴含光殿,直史館陳靖上言:「古之饗宴者,所以省禍福而觀威儀也。故宴以禮成,賓以賢序,《風》、《雅》之作,茲為盛焉。伏見近年內殿賜宴,群臣當坐於朵殿、兩廊者,拜舞方畢,趨馳就席,品列之序,糾紛無別。及至尊舉爵,群臣起立,先後不整,俯仰
 失節。欲望自今令有司預依品位告諭,其有逾越班次、拜起失節、喧嘩過甚者,並令糾舉。又惟飫賜之典,以寵武夫,大烹之餘,故為盛饌。計一飯所費,可數人之屬厭,而將校輩或至終宴之時,尚有欲炙之色,蓋執事者失於察視,不及潔豐而使然也。伏望並申嚴制。」至道元年三月,御史中丞李昌齡亦言:「廣宴之設,以均飫賜,得齒高會,宜乎盡禮。而有位之士,鮮克致恭,當糾其不恪。又供事禁庭,當定員數,籍姓名以謹其出入。酒肴之司,或
 虧精潔,望分命中使巡察。」並從之。



 咸平三年二月,大宴含光殿,自是始備設春秋大宴。五年,御史臺言:「大宴,起居舍人、司諫、正言、三院使、御史並坐於殿廊,望自今移升朵殿,自餘依舊。」十二月,詔凡內宴,宗正卿令升殿坐,班次依合班儀。翰林學士梁顥請以春秋大宴、小宴、賞花、行幸次為四圖,頒下閣門遵守。從之。



 景德二年九月,詔曰:「朝會陳儀,衣冠就列,將以訓上下、彰文物,宜慎等威,用符紀律。況屢頒於條令,宜自顧於典刑。稍歷歲時,
 漸成懈慢。特申明制,以儆具僚。自今宴會,宜令御史臺預定位次,各令端肅,不得喧嘩。違者,殿上委大夫、中丞,朵殿委知雜御史、侍御史,廊下委左右巡使,察視彈奏;內職殿直以上赴起居、入殿庭行私禮者,委閣門彈奏;其軍員,令殿前侍衛司各差都校一人提轄,但虧失禮容,即送所司勘斷訖奏。仍令閣門、宣徽使互相察舉,敢蔽匿者糾之。」



 大中祥符元年十二月,詔宣徽院、御史臺、閣門、殿前馬步軍司,凡內宴臣僚、軍員並祗候使臣等,
 並以前後儀制曉諭,務令遵稟,違者密具名聞。其軍員有因酒言詞失次及醉僕者,即先扶出,或遣殿前司量添巡檢軍士護送歸營。又詔臣僚有托故請假不赴宴者,御史臺糾奏。天禧四年,直集賢院祖士衡言:「大宴將更衣,群臣下殿,然後更衣,更衣後再坐,則群臣班於殿庭,候上升坐,起居謝賜花,再拜升殿。」



 仁宗天聖三年,監察御史朱諫言:「伏見大宴,宗室先退,允為得禮。尚有文武臣僚父子、兄弟者,皆預再坐,欲望自今內宴,百官有
 父子、兄弟、叔侄同赴,再坐時卑者先退。」慶歷七年,御史言:「凡預大宴並御筵,其所賜花,並須載歸私第,不得更令僕從持戴,違者糾舉。」



 熙寧二年正月,閣門言:「準詔裁定集英殿宴入殿人數:中書二十二人,樞密院三十人,宣徽院八人,親王八人,昭德軍節度使、兼侍中曹佾三人,皇親使相三人,皇親正刺史已上至節度使並駙馬都尉各一人,翰林司一百七十八人,御廚六百人,儀鸞司一百五十人,祗候庫二十人,內衣物庫七人,新衣庫
 七人,內弓箭庫三人,鈐轄教坊所三人,鐘鼓樓一十六人,御藥院八人,內物料庫九人,法酒庫一十六人,內酒坊八人,入內內侍省前後行、親事官共五人,皇城司職員手分二人,御史臺知班一十一人,灑掃親從官人員已下一百人,兩廊覷步親從官四十二人,提舉司勾押官手分三人,提舉火燭巡檢人員一十人,快行親從官一十一人,支散兩省花後苑造作所工匠等四人,客省承授行首八人,四方館職掌二人,閣門承受行首已下
 一十八人。」是歲十一月,以皇子生,宴集英殿。



 七年九月,詔:「自今大宴,親王、皇親使相、樞密使副使、宣徽使、駙馬都尉並於殿門外幕次就賜酒食。」舊制,會食集英西廊之廡下,喧卑為甚,權發遣宣徽院吳充奏其事,故有是命。



 元豐七年三月,大宴集英殿,命皇子延安郡王侍立。宰相王珪等率百官廷賀。詔曰:「皇家慶事,與卿等同之。」珪等再拜稱謝。久之,王乃退。時王未出閣,帝特令侍宴,以見群臣。九年,閣門言:「大宴不用兩軍妓女,只用教坊
 小兒之舞。」王拱辰請以女童代之。元祐八年,詔罷獨看。故事,大宴前一日,御殿閱百戲,謂之獨看。修國史範祖禹言:「是日進《神宗紀》草,陛下覽先帝史冊甫畢,即觀百戲,理似未安,故請罷之。」



 元祐二年九月,經筵講《論語》徹章,賜宰臣、執政、經筵官宴於東宮,帝親書唐人詩分賜之。三年六月,罷春宴。八月,罷秋宴,以魏王出殯,翰林學士蘇軾不進教坊致語故也。是後以時雨未足,集英殿試舉人,尚書省火,禁中祈禳,邠國公主未菆,皆罷宴。凡大宴
 有故而罷,賜賜預宴官酒饌於閣門朝堂,升殿官雖假故不從游宴,亦遣中使就第賜焉。親王、中書、樞密、宣徽、三司使副、學士、步軍都虞候以上、三師、三公、東宮三師三公以下、曾任中書門下致仕者,亦同。



 凡外國使預宴者,祥符中宴崇德殿,夏使於西廊南赴坐,交使以次歇空,進奉、押衙次交州,契丹舍利、從人則於東廊南赴坐。四年,又升甘州、交州於朵殿,夏州押衙於東廊南頭歇空坐。七年,龜茲進奉人使歇空坐於契丹舍利之下。其
 後又令龜茲使副於西廊南赴坐,進奉、押衙重行於後,瓜州、沙州使、副亦於西廊之南赴坐,其餘大略以是為準。



 大觀三年,議禮局上集英殿春秋大宴儀:



 其日,預宴文武百僚詣殿庭,東西相向立。皇帝出御需雲殿,閣門、內侍、管軍等起居。皇帝降坐,御集英殿,鳴鞭,殿中監已下通班起居。殿中監、少監升殿,通喚閣門官升殿。攝左右軍巡使靴笏起居訖,系鞋執杖侍立,餘非應奉官分出。次鐘鼓樓節級就位,四拜起居。



 次舍人通喚訖,分引
 群官橫行北向,東上閣門官贊大起居,班首出班俯伏,跪,致辭訖,俯伏,興,復位。群官再拜舞蹈,又再拜,贊各就坐,再拜,舍人分引升殿,席前相向立,朵殿、兩廡官立於席後。有遼使則舍人引大遼舍利西入大起居,贊各就坐,贊再拜,贊就坐,引升西廊。次舍人傳事引從人分入,四拜起居,謝坐,並同舍利儀。教坊使以下通班大起居,看盞人謝,升殿再拜。內侍進御茶床,殿侍酹酒訖,次贊天武門外祗候。東上閣門官詣御坐,奏班首姓名以下
 進酒。



 舍人分引殿上臣僚橫行北向,贊再拜。舍人引班首稍前,東上閣門官接引詣御坐,東北向,搢笏,殿中監奉盤盞授班首,少監啟盞,以酒注之。班首奉詣御前進訖,少退,虛跪,興,以盤授殿中監,出笏,東上閣門官引退,舍人接引復位,贊再拜。舍人引班首稍前,殿上臣僚席前相向立,東上閣門官接引詣御坐,東北向,搢笏,殿中監授盤,奉詣御前,西向立,樂作,皇帝飲訖。舍人分引殿上臣僚橫行北向,東上閣門引班首接盞,退,虛跪,興,授
 盞殿中監,出笏,引退,舍人接引復位,贊再拜,贊各賜酒,群臣再拜,贊各就坐,群官皆立席後,復贊就坐。



 酒初行,群官搢笏受酒,先宰相,次百官,皆作樂。皇帝再舉酒,並殿中監、少監進。



 群臣俱立席後,樂作,飲訖,贊各就坐。復行群臣酒,飲訖。皇帝三舉酒,皆如第一之儀。尚食典奉御進食,太官設群臣食,樂作。賜祗應臣僚酒食,贊謝拜訖,復位。皇帝四舉酒,並典御進酒。



 樂工致語,群官皆立席後,致語訖,贊百官再拜,就坐,樂作。皇帝五舉酒,樂工奏樂,庭下舞
 隊致詞,樂作,舞隊出。



 東上閣門奏再坐時刻。俟放隊訖,內侍舉御茶床,皇帝降坐,鳴鞭,群臣退。賜花,再坐。前二刻,御史臺、東上閣門催班,群官戴花北向立,內侍進班齊牌,皇帝詣集英殿,百官謝花再拜,又再拜就坐。內侍進御茶床,皇帝舉酒,殿上奏樂,庭下作樂。皇帝再舉酒,殿上奏樂,庭下舞隊前致語,樂作,出。皇帝三舉酒、四舉酒皆如上儀。若宣示盞,即隨所向,閣門官以下揖稱宣示盞,躬贊就坐。若宣勸,即立席後躬飲訖,贊再拜。內侍
 舉御茶床,舍人引班首以下降階再拜舞蹈,又再拜訖,分班出。閣門官側奏無公事,皇帝降坐,鳴鞭。



 集英殿飲福大宴儀。初,大禮畢,皇帝逐頓飲福,餘酒封進入內。宴日降出,酒既三行,泛賜預坐臣僚飲福酒各一盞,群臣飲訖,宣勸,各興,立席後,贊再拜謝訖,復坐飲,並如春秋大宴之儀。



 紹興十三年三月三日,詔宴殿陳設止用緋、黃二色,不用文繡,令有司遵守,更不制造。五月,閣門修立集英殿大宴儀注。



 乾道八月十二月,詔今後前宰相
 到闕,如遇赴宴賜茶,其合會墩杌,非特旨,並依官品。又行門、禁衛諸色祗應人,依紹興例,並賜絹花。自是惟正旦、生辰、郊祀及金使見辭各有宴,然大宴視東京時則亦簡矣。



 曲宴。凡幸苑囿、池禦,觀稼、畋獵,所至設宴,惟從官預,謂之曲宴。或宴大遼使、副於紫宸殿,則近臣及刺史、正郎、都虞候以上預。暮春後苑賞花、釣魚,則三館、秘閣皆預。



 太祖建隆元年七月,親征澤、潞,宴從臣於河陽行宮,又宴韓令坤已下於禮賢講武殿,賜襲衣、器幣、鞍
 馬,以賞澤、潞之功也。四年四月,宴從臣於玉津園。乾德三年七月六日,詔皇弟開封尹、宰相、樞密使、翰林學士、中書舍人泛舟後苑新池,張樂宴飲,極歡而罷。是歲重陽,宴近臣於長春殿。



 太宗太平興國九年三月十五日,詔宰相、近臣賞花於後苑,帝曰:「春氣暄和,萬物暢茂,四方無事。朕以天下之樂為樂,宜令侍從詞臣各賦詩。」帝習射於水心殿。雍熙二年四月二日,詔輔臣、三司使、翰林、樞密直學士、尚書省四品兩省五品以上、三館學士
 宴於後苑,賞花、釣魚,張樂賜飲,命群臣賦詩習射。賞花曲宴自此始。三年十二月一日,大雨雪,帝喜,御玉華殿,召宰臣及近臣謂曰:「春夏以來,未嘗飲酒,今得此嘉雪,思與卿等同醉。」又出禦制《雪詩》,令侍臣屬和。後凡曲宴不盡載。



 真宗咸平元年二月二十二日,宴群臣於崇德殿,不作樂。二年八月七日,再宴,用樂。三年二月晦,賞花,宴於後苑,帝作《中春賞花釣魚詩》,儒臣皆賦,遂射於水殿,盡歡而罷。自是遂為定制。四年十一月二十日,御龍圖閣
 曲宴,詔近臣觀太宗草、行、飛白、篆、籀、八分書及畫。景德二年十二月五日,宴尚書省五品、諸軍都指揮使以上、契丹使於崇德殿,不舉樂,以明德太后喪制故也。時契丹初來賀承天節,擇膳夫五人繼本國異味,就尚食局造食,詔賜膳夫衣服、銀帶、器帛。大中祥符六年七月二十九日,詔輔臣觀粟於後苑禦山子,觀禦制文閣禦書及《嘉禾圖》,賜飲。是日,皇子從游。天禧四年七月十一日,詔近臣及寇準、馮拯觀內苑谷,遂宴於玉宸殿。十月二十九日,詔皇太
 子、宗室、近臣、諸帥赴玉宸殿翠芳亭觀稻,賜宴,仍以稻分賜之。



 仁宗天聖二年,既禫除,百官五表請聽樂,而秋燕用樂之半。詔輔臣曰:「昨日宴宮中,朕數四上勉皇太后聽樂。」王欽若以聞太后,太后曰:「自先帝棄天下,吾終身不欲聽樂。皇帝再三為請,其可重違乎!」明年上元節,乃朝謁景靈上清宮、啟聖院、相國寺,還御正陽門,宴從官,觀燈。次日,太后召命婦臨觀。及春秋大宴,歲為常。夏,觀南御莊刈麥,秋,瑞聖園刈穀,並宴從官,或射,不為常。
 皇祐五年,後苑寶政殿刈麥,謂輔臣曰:「朕新作此殿,不欲植花,歲以種麥,庶知穡事不易也。」自是幸觀谷、麥,惟就後苑,春夏賞花、釣魚則歲為之。嘉祐七上十二月,特召兩府、近臣、三司副使、臺諫官、皇子、宗室、駙馬都尉,管軍臣僚至龍圖、天章閣,觀三聖御書,及寶文閣為飛白分賜,下逮館閣官,制《觀書詩》,賜韓琦以下和進,遂宴群玉殿,傳詔學士王珪撰詩序,刊石於閣。數日,再會天章閣,觀三朝瑞物,復宴群玉殿,酒行,上曰:「天下久無事,今
 日之樂,與卿等共之,宜盡醉,勿復辭。」因召韓琦至御榻前,別賜一大卮。出禁中名花,金盤貯香藥,令各持歸,莫不沾醉,至暮而罷。



 熙寧元年四月,御史中丞滕甫言:「臣聞君命召,不俟駕,此臣子所以恭其上也。今賜宴而有托詞不至者,甚非恭上之節也。請自今宴設,群臣非大故與實有疾病,無得托詞,仍令御史臺察舉。」二年八月,《實錄》書成,皆宴垂拱殿。十月,修定閣門儀制所言:「垂拱殿曲宴,當直翰林學士與觀文、資政、龍圖、寶文、樞密、直
 龍圖、天章、寶文閣直學士並赴坐,而翰林學士兼他職者不預,考之官制,似未齊一。請自今曲宴,翰林學士與雜學士並赴。」從之。元豐五年七月,以《兩朝國史》書成,宴於垂拱殿。十一月,宴景靈宮祠官於紫宸殿。



 元祐二年九月,經筵講《論語》徹章,賜宰臣、執政、經筵官宴於東宮,帝親書唐人詩賜之。紹聖三年十一月,以進《神宗皇帝實錄》畢,曲宴,宰臣、執政、文臣試侍郎、武臣觀察使以上並修圖史官赴坐。元符元年五月,詔受寶畢,宴於紫宸
 殿,宰臣以下,文臣職事官、六曹員外郎、監察御史以上,武臣郎將、諸軍副指揮使以上預坐。



 政和二年三月,上巳禦筵,詔令移用他日,以國有故,宰臣請罷宴故也。大觀三年,議禮局上垂拱殿曲宴儀:



 皇帝視事畢,東上閣門進呈坐圖,舍人奏閣門無公事,皇帝降坐,鳴鞭,入殿後閣。



 諸司排設備,東上閣門附內侍奏班齊,皇帝出閣升坐,鳴鞭。三公、直學士以上、親王、使相至觀察使以上,分東西入,詣殿庭,橫行北向立定。班首奏聖躬萬福,舍
 人贊各就坐,再拜訖,分引詣東西階升殿,席前相向立。次教坊使以下常起居,次看盞人謝,升殿,次內侍進御茶床,殿侍酹酒訖,閣門詣御坐,躬奏班首姓名以下進酒。舍人分引殿上臣僚,橫行北向,贊再拜。班首奉酒躬進,樂作,皇帝飲訖。舍人贊各賜酒,群官俱再拜。贊各就坐,群官皆立席後,復贊就坐。



 酒初行,先宰相,次百官,皆作樂。後準此。



 尚食興,奉御進食,太官令設群官食。酒五行,若宣示盞,即隨所向,閣門揖稱宣示盞,躬,贊就坐。若宣
 勸,即立席後躬飲,贊再拜。內侍舉御茶床,舍人引班首以下降階橫行,北向再拜,分班出。皇帝降坐。



 止巳、重陽賜宴儀:



 其日,預宴官以下並赴宴所就次,諸司排設備,預宴官以下詣庭中望闕位立。次中使詣班首之左,稍前立,中使宣曰「有敕」,在位官皆再拜訖。中使宣曰「賜卿等禦筵」,在位官皆再拜,搢笏舞蹈,又再拜。中使退,預宴官分東西升階就坐。



 酒行,樂作。食訖、食畢,樂止。酒五行,預宴官並興就次,賜花有差。少頃,戴花畢,與宴官詣望
 闕位立,謝花,再拜訖,復升就坐。酒行,樂作。飲訖、食畢,樂止。酒四行而退。



 游觀。天子歲時游豫,則上元幸集禧觀、相國寺,御宣德門觀燈;首夏幸金明池觀水嬉,瓊林苑宴射;大祀禮成,則幸太一宮、集禧觀、相國寺恭謝,或詣諸寺觀焚香,或至近郊閱武、觀稼,其事蓋不一焉。



 太祖建隆元年四月,幸玉津園。是後凡十三臨幸。九月,幸宜春苑。是後觀習水戰者二十有八,幸大相國寺、封禪寺者各五,龍興寺
 及皇弟開封尹園各三,幸太清觀、建隆觀者再,崇夏寺、廣化寺、等覺寺者各一,觀水磑者八,閱炮車、觀水櫃、觀稼、幸飛龍院、幸開封府、幸都亭驛、幸禮賢院、幸茶庫染院、幸河倉、幸金鳳園,皆一再至焉。



 太宗太平興國二年二月,幸新鑿池,賜役卒錢布有差,六月,幸飛龍院。是後凡四幸。三年四月,觀刈麥。九年正月六日,幸景龍門外水磑,帝臨水而坐,召從臣觀之,因謂曰:「此水出於山源,清澄甘潔。近河之地,水味皆甘,豈河潤所及乎?」宋琪等
 曰:「亦猶人性善惡,染習致然。」帝曰:「卿言是也。」四月,幸金明池習水戰,帝御水殿,召近臣觀之,謂宰相曰:「水戰,南方之事也。今其地已定,不復施用,時習之,示不忘戰耳。」因幸講武臺,閱諸軍都試,軍中之絕技者遞加賜賚。遂登瓊林苑樓,陳百戲,擲金錢,令樂人爭之,極歡而罷。五月二日,出南熏門觀稼,召從臣列坐田中,令民刈麥,咸賜以錢帛。回幸玉津園觀漁,張樂、習射,既宴而歸。明年五月,幸城南觀麥,賜田夫布帛有差。雍熙四年四月,幸
 金明池觀水嬉,賜從官飲。帝曰:「雨霽天涼,中外無事,宜勿惜醉。」因登苑中樓,盡歡而罷。淳化三年三月,幸金明池,命為競渡之戲,擲銀甌於波間,令人泅波取之。因御船奏教坊樂,岸上都人縱觀者萬計。帝顧視高年皓首者,就賜白金器皿。九月,幸潛龍園,駐輦池東岸,臨水謂近臣曰:「朕不至此已十年,昔尹京日,無事常痛飲池上,今池邊之木已成林矣。」因顧教坊使郭守忠等數人曰:「汝等前日以樂童從我,今亦皓首,光陰迅速如此。」嗟嘆
 久之。帝親引滿舉白,群臣盡醉。



 真宗咸平元年八月,幸諸王宮。二年九月,幸開寶寺、福聖院。是後,二寺臨幸者凡十有四。三年五月,幸金明池觀水戲,揚旗鳴鼓,分左右翼,植木系彩,以為標識,方舟疾進,先至者賜之。移幸瓊林苑,登露臺,鈞容直奏樂,臺下百戲競集,從臣皆醉。自是凡四臨幸。九月,幸大相國寺。是後再幸者九。幸上清宮者十有二,幸玉津園者十,幸太一宮、玉清昭應宮各六,餘不盡載。大中祥符八年正月十九日,中書門下
 上言:「伏睹今月十四日,皇帝詣諸宮寺焚香,總三十餘處,過百拜以上。臣等侍從,倍增憂灼,昨崇政殿已面奏陳。臣聞尊事萬靈,固先精意;登用百禮,乃貴時中。在經久之從宜,必裁正而惟允。伏望特命攸司,載詳定式。自今車駕幸諸宮、觀、寺、院,正殿再拜;及諸殿,令群臣以下分拜。庶垂億載,允葉通規。」乃詔禮儀院詳定差減焉。



 仁宗景祐三年,詔閣門詳定車駕幸宮、觀、寺、院支賜茶絹等第。



 哲宗紹聖四年三月八日,詔自今遇車駕出新城,
 令殿前馬、步軍司取旨,權差馬、步軍赴新城外四面巡檢下祗應,每壁馬軍二百人,步軍三百人,並於城外巡警。



 三元觀燈,本起於方外之說。自唐以後,常於正月望夜,開坊市門然燈。宋因之,上元前後各一日,城中張燈,大內正門結彩為山樓影燈,起露臺,教坊陳百戲。天子先幸寺觀行香,遂御樓,或御東華門及東西角樓,飲從臣。四夷蕃客各依本國歌舞列於樓下。東華、左右掖門、東
 西角樓、城門大道、大宮觀寺院,悉起山棚,張樂陳燈,皇城雉堞亦遍設之。其夕,開舊城門達旦,縱士民觀。後增至十七、十八夜。



 太祖建隆二年上元節,御明德門樓觀燈,召宰相、樞密、宣徽、三司使、端明、翰林、樞密直學士、兩省五品以上官、見任前任節度觀察使飲宴,江南、吳越朝貢使預焉。四夷蕃客列坐樓下,賜酒食勞之,夜分而罷。三年正月十三夜然燈,罷內前排場戲樂,以昭憲皇太后喪制故也。



 太平興國二年七月中元節,御東角樓
 觀燈,賜從官宴飲。五年十月下元節,依中元例,張燈三夜。雍熙五年上元節,不觀燈,躬耕籍田故也。後凡遇用兵及災變、諸臣之喪,皆罷。



 真宗景德元年正月十四日,賜大食、三佛齊、蒲端諸國進奉使緡錢,令觀燈宴飲。大中祥符元年十一月二十五日,詔天慶節聽京城然燈一晝夜。六年四月十六日,先天降聖節亦如之。天聖二年六月,罷降聖節然燈。



 政和三年正月,詔放燈五日。五年十二月二十九日,詔景龍門預為元夕之具,實欲觀
 民風、察時態、黼飾太平、增光樂國,非徒以游豫為事。特賜公、師、宰執以下宴,及御制詩四韻賜太師蔡京。六年正月七日,御筆:「今歲閏餘候晚,猶未春和。晷短氣寒,於宴集無舒緩之樂。景靈宮朝獻,移十四日東宮、十五日西宮,畢,詣上清儲祥宮燒香。十六日詣醴泉觀等處燒香。上元節移於閏正月十四日為始。」宣和六年十二月二十四日,賜太師蔡京以下應兩府赴睿謨殿宴,景龍門觀燈。續有旨,宣太傅王黼赴宴。七年正月十八日,宴
 輔臣,觀燈。



 賜酺。自秦始。秦法,三人以上會飲則罰金,故因事賜酺,吏民會飲,過則禁之。唐嘗一再舉行。



 太宗雍熙元年十二月,詔曰:「王者賜酺推思,與眾共樂,所以表升平之盛事,契億兆之歡心。累朝以來,此事久廢,蓋逢多故,莫舉舊章。今四海混同,萬民康泰,嚴禋始畢,慶澤均行。宜令士庶之情,共慶休明之運,可賜酺三日。」二十一日,御丹鳳樓觀酺,召侍臣賜飲。自樓前至朱雀門張樂,作山車、
 旱船,往來御道。又集開封府諸縣及諸軍樂人列於御街,音樂雜發,觀者溢道,縱士庶游觀,遷市肆百貨於道之左右。召畿甸耆老列坐樓下,賜之酒食。明日,賜群臣宴於尚書省,仍作詩以賜。明日,又宴群臣,獻歌、詩、賦、頌者數十人。



 真宗景德三年九月,詔許群臣、士庶選勝宴樂,御史臺、皇城司毋得糾察。四年二月甲申,上御五鳳樓觀酺,宗室、近臣侍坐。樓前露臺奏教坊樂,召父老五百人列坐,賜飲於樓下。後二日,上復御樓,賜宗室、文武
 百官宴於都亭驛,賜諸班、諸軍將校羊酒。大中祥符元年正月,詔應致仕官並令赴都亭驛酺宴,御樓日合預坐者亦聽。又詔朝臣已辭、未見,並聽赴會。凡酺,命內諸司使三人主其事,於乾元樓前露臺上設教坊樂。又駢系方車四十乘,上起彩樓者二,分載鈞容直、開封府樂。復為棚車二十四,每十二乘為之,皆駕以牛,被之錦繡,縈以彩紖,分載諸軍、京畿伎樂,又於中衢編木為欄處之。徙坊市邸肆對列禦道,百貨駢布,競以彩幄鏤版為飾。
 上禦乾元門,召京邑父老分番列坐樓下,傳旨問安否,賜以衣服、茶帛。若五日,則第一日近臣侍坐,特召丞、郎、給、諫,上舉觴,教坊樂作,二大車自升平橋而北,又有旱船四挾之以進,輣車由東西街交騖,並往復日再焉。東距望春門,西連閶闔門,百戲競作,歌吹騰沸。宗室親王、近列牧伯洎舊臣、宗室官,為設彩棚於左右廊廡。士庶縱觀,車騎填溢,歡呼震動。第二日宴群臣百官於都亭驛、宗室於親王宮。第三日宴宗室內職於都亭驛、近臣
 於宰相第。第四日宴百官於都亭驛、宗室於外苑。第五日復宴宗室內職於都亭驛、近臣於外苑。上多作詩,賜令屬和,及別為勸酒詩。禁軍將校日會於殿前馬、步軍之廨。



 是歲,東封泰山,所過州府,上御子城門樓,設山車、彩船載樂,從臣侍坐,本州父老、進奉使、蕃客悉預。兗州駐蹕,仍賜群臣會於延壽寺。所在改賜門名,兗州曰「回鑾覃慶」,鄆州曰「升中延福」,濮州曰「告成延慶」。澶州以行宮迫隘,當衢結彩為殿,名曰「延禧」。幸汾陰、亳州,皆如東
 封路。河中府門名曰「詔畢宣恩」,陜州曰「霈澤惠民」,鄭州曰「回鑾慶賜」。西京將議改五鳳樓名,上曰:「此太祖所建,因瑞應,不可更也。」華陰就行宮宴父老,賜驛亭名曰「宣澤」。至鄭州,以太宗忌日甫過,罷會,賜與如例。亳州曰「奉元均慶」,南京曰「重熙頒慶」。



 天禧五年,以畿縣追集、老人疲勞之故,止召兩赤縣、坊縣父老預會,其不預名亦聽,給以賜物。天下賜酺,各令州、府會官屬父老,邊州或遣中使就賜。又詔開封府:「賜酺日,罪人酗酒而不傷人者,
 咸釋之。再犯,論如法。」後賜酺皆準此。宋之繁庶,於斯為盛,後遂為定制云。



\end{pinyinscope}