\article{列傳第八十}

\begin{pinyinscope}

 鄭獬陳襄錢公輔孫洙豐稷呂誨劉述劉琦錢顗鄭俠



 鄭獬,字毅夫,安州安陸人。少負俊材,詞章豪偉峭整,流
 輩莫敢望。進士第一。通判陳州,入直集賢院、度支判官、修起居注、知制誥。



 英宗郎位,治永昭山陵,悉用乾興制度。獬言:「今國用空乏,近者賞軍,已見橫斂,富室嗟怨,流聞京師。先帝節儉愛民,蓋出天性,凡服用器玩,極於樸陋,此天下所共知也。而山陵制度,乃欲效乾興最盛之時,獨不傷儉德乎?願飭有司,損其名數。」又言:「天子初即位,郡國馳表稱賀,例官其人,此出五代餘習,因仍未改。今庶官猥眾,充溢銓曹。況前日群臣進官,已布維新之
 澤,不須復行此恩,以開僥幸。」皆不報。又上疏言:「陛下初臨御,恭默不言,所與共政者七八大臣而已,焉能盡天下之聰明哉?願申詔中外,許令盡言,有可採錄,召與之對。至於臣下進見,訪以得失,虛心求之,必能有益治道。」帝嘉納之。時詔諸郡敦遣遺逸之士,至則試之秘閣,命以官。頗有謬舉者,眾論喧嘩,旋即廢罷。獬言:「古之薦士,以謂拔十得五,猶得其半;況今所失未至十五,而遽以浮言廢之,可乎?願復此科,使豪俊無遺滯之嘆。」未及行,
 出知荊南。治平中,大水求言,獬上疏曰:「陛下側身思咎,念有以消復之,不知求忠言者,將欲用之邪,抑但舉故事邪?觀前世之君,因變異以求諫者甚眾,及考其實,則能用其言而載於行事者,蓋亦鮮矣。今詔發天下忠義之士,必有極其所韞,以薦諸朝,一日萬機,勢未能盡覽,不過如平時下之中書、密院,至於無所行而後止。如是則與前世之為空言者等爾。謂宜選官置屬,掌所上章,與兩府近臣從容講貫,可則行之,否則罷之,有疑焉,則
 廣詢而決之。群臣得而眾事舉,此應天之實也。天下之進言也甚難,而上之受言也常忽。願陛下採群臣之章疏,容而聽之,史冊大書,以為某年大水,詔求直言,用某人之辭而求某事,以出夫前世之為空言者,無令徒掛墻壁為虛文而已。」還,判三班院。



 神宗初,召獬夕對內東門,命草吳奎知青州及張方平、趙抃參政事三制,賜雙燭送歸舍人院,外廷無知者。遂拜翰林學士。朝廷議納橫山,獬曰:「兵禍必起於此。」已而種諤取綏州,獬言:「臣竊
 見手詔,深戒邊臣無得生事。今乃特尊用變詐之士,務為掩襲,如戰國暴君之所尚,豈帝王大略哉!諤擅興,當誅。」又請因諒祚告哀,遣使立其嗣子,識者韙之。



 權發遣開封府。民喻興與妻謀殺一婦人,獬不肯用按問新法,為王安石所惡,出為侍讀學士、知杭州。御史中丞呂誨乞還之,不聽。未幾,徙青州。方散青苗錢,獬言:「但見其害,不忍民無罪而陷憲網。」引疾祈閑,提舉鴻慶宮,卒,年五十一。家貧子弱,其柩蒿殯僧屋十餘年,滕甫為安州,乃
 克葬。



 陳襄,字述古,福州侯官人。少孤,能自立,出游鄉校,與陳烈、周希孟、鄭穆為友。時學者沉溺於雕琢之文,所謂知天盡性之說,皆指為迂闊而莫之講。四人者始相與倡道於海濱,聞者皆笑以驚,守之不為變,卒從而化,謂之「四先生」。



 襄舉進士,調浦城主簿,攝令事。縣多世族,以請托肋持為常,令不能制。襄欲稍革其俗,每聽訟,必使數吏環立於前。私謁者不得發,老奸束手。民有失物者,賊
 曹捕偷兒至,數輩相撐拄,襄語之曰:「某廟鐘能辨盜,犯者捫之輒有聲,餘則否。」乃遣吏先引以行,自率同列詣鐘所祭禱,陰塗以墨,而以帷蔽之。命群盜往捫,少焉呼出,獨一人手無所污,扣之,乃為盜者;蓋畏鐘有聲,故不敢觸,遂服罪。



 知河陽縣,始教民種稻。富弼為郡守,一見即禮遇之。襄留意教化,進縣子弟於學。或讒之於弼,謂其誘邑子以資過客,弼疑焉。人勸毀學舍以塞謗,不聽。久之,弼以語襄,襄曰:「自反而縮,雖千萬人往矣。公茍有
 惑志,何名知己,」益講說不少懈。弼由是愈益奇之,及入相,薦為秘閣校理、判祠部。譯經僧死,遺表度十僧,列子廟三年度一道士,皆抑不行。



 知常州,運渠橫遏震澤,積水不得北入江,為常、蘇二州病。襄度渠之丈尺與民田步畝,定其數,授以浚法。未幾,遂削望亭古堰,水不復積。入為開封府推官、鹽鐵判官。神宗立,奉使契丹,以設席小異於常,不即坐。契丹移檄疆吏,坐出知明州。明年,同修起居注,知諫院,改侍御史知雜事。論青苗法不便,曰:「
 臣觀制置司所議,莫非引經以為言,而其實則稱貸以取利,事體卑削,貽中外譏笑。是特管夷吾、商鞅之術,非聖世所宜行。望貶斥王安石、呂惠卿以謝天下。」又乞罷韓絳政府,以杜大臣爭利而進者,且言韓維不當為中丞,劉述、范純仁等無罪,宜復官。皆不聽,而召試知制誥。襄以言不行,辭不肯試,願補外。安石欲以為陜西轉運使,帝惜其去,留修起居注。襄懇辭,手詔諭之,乃就職。逾年,為知制誥,安石又欲出之,帝不許。尋直學士院,安石
 益忌之,擿其書詔小失,出知陳州,徙杭州,以樞密直學士知通進、銀臺司兼侍讀,判尚書都省。卒,年六十四,贈給事中。



 襄蒞官所至,必務興學校。平居存心以講求民間利病為急。既亡,友人劉尋視其篋,得手書累數十幅,盈紙細書,大抵皆民事也。在經筵時,神宗顧之甚厚,嘗訪人材之可用者。襄以司馬光、韓維、呂公著、蘇頌、范純仁、蘇軾至於鄭俠三十三人對,謂光、維、公著皆股肱心膂之臣,不當久外;謂俠愚直敢言,發於忠義,投竄瘴癘,
 朝不謀夕,願使得生還。帝不能盡用。



 錢公輔,字君倚,常州武進人。少從胡翼之學,有名吳中。第進士甲科。通判越州,為集賢校理、同判吏部南曹。歷開封府推官、戶部判官、知明州。衙前法以三等差次勞勤,應格者聽指酒場以自補,富者足欲而貧得日困,充募益鮮;額有不足,至役鄉民,破產不供費。公輔取酒場官鬻之,分輕重以給役者,不復調民。同修起居注,進知制誥。



 英宗即位,陳《治平十議》,大要言採民政,分吏課,擇
 守宰,置二府官屬。又作《帝問》一篇上之。王疇為翰林學士未久,擢副樞密。公輔謂疇素望淺,不草制。帝以初政用大臣,而公輔格詔,謫為滁州團練使。議者以為重,呂誨等上章救之,不得。逾年,起知廣德軍。神宗立,拜天章閣待制、知鄧州,復知制誥。入見,帝勞苦之,使錄《十議》以進,命知諫院。嘗至中書白事,富弼謂曰:「上求治如饑渴,正賴君輩同心以濟。」公輔曰:「朝廷所為是,天下誰敢不同!所為非,公輔欲同之,不可得已。」



 王安石雅與之善,既
 得志,排異己者,出滕甫鄆州。公輔數於帝前言甫不當去。薛向更鹽法,安石主其議,而公輔謂向當黜,遂拂安石意,罷諫職,旋出知江寧府。明年,帝欲召還,安石言其助小人為異議,不宜在左右,但徙揚州。以病乞越,改提舉崇福觀,卒,年五十二。



 孫洙,字臣源,廣陵人。羈丱能文,未冠擢進士。包拯、歐陽修、吳奎舉應制科,進策五十篇,指陳政體,明白剴切。韓琦讀之,太息曰:「慟哭流涕,極論天下事,今之賈誼也。」再
 遷集賢校理、知太常禮院。



 治平中求言,以洙應詔疏時弊要務十七事後多施行,兼史館檢討、同知諫院,乞增諫員以廣言路。凡有章奏,輒焚其稿,雖親子弟不得聞。王安石主新法,多逐諫官御史,洙知不可,而鬱鬱不能有所言,但力求補外,得知海州。免役法行,常平使者欲加斂緡錢,以取贏為功,洙力爭之。方春旱,發運使調民浚漕渠以通鹽舸,洙持之不下,三上奏乞止其役。旱蝗為害,致禱於朐山,澈奠,大雨,蝗赴海死。



 尋乾當三班院。
 三班員過萬數,功罪籍不明,前後抵牾,吏左右出入,公為欺奸。洙革其甚者八事,定為令。同修起居注,進知制誥。先是,百官遷敘,用一定之詞,洙建言:「群臣進秩,事理各異,而同用一詞;至或一門之內,數人拜恩,名體散殊,而格以一律。茍從簡便,非所以暢王言、重命令也。」詔自今封贈蔭補,每大禮一易,他皆隨等撰定。



 元豐初,兼直學士院。澶州河平,作靈津廟,詔洙為之碑,神宗獎其文。擢翰林學士,才逾月,得疾。時參知政事闕,帝將用之,數
 遣中使、尚醫勞問。入朝期日,洙小愈,在家習肄拜跽,僨不能興,於是竟卒,年四十九。帝臨朝嗟惜,常賻外賜錢五十萬。



 洙博聞強識,明練典故,道古今事甚有條理。出語皆成章,雖對親狎者,未嘗發一鄙語。文詞典麗,有西漢之風。士大夫共以丞輔期之,不幸早世,一時憫傷焉。



 豐稷,字相之,明州鄞人。登第,為穀城令,以廉明稱。從安燾使高麗,海中大風,檣折,舟幾覆,眾惶擾莫知所為,稷獨神色自若。燾嘆曰:「豐君未易量也。」知封丘縣,神宗召
 對,問:卿昔在海中遭風波,何以不畏?」對曰:「巨浸連天,風濤固其常耳,憑仗威靈,尚何畏!」帝悅,擢監察御史。治參知政事章惇請托事,無所移撓,出惇陳州。徒著作佐郎、吏部員外郎,提點利州、成都路刑獄。



 入為殿中侍御史。上疏哲宗曰:「陛下明足以察萬事之統,而不可用其明;智足以應變曲當,而不可用其智。順考古道,二帝所以聖;儀刑文王,成王所以賢。願以《洪範》為元龜,祖訓為寶鑒,一動一言,思所以為則於四海,為法於千載,則教化
 行,習俗美,而中國安矣。」劉奉世冊立夏國嗣子乾順,而乾順來賀坤成節,奉世遽出境,稷劾之,奉世以贖論,遷右司諫。揚、荊二王為天子叔父,尊寵莫並,密令蜀道織錦茵。稷於正衙論曰:「二聖以儉先天下,而宗王僭侈,官吏奉承,皆宜糾正。」既退,御史趙□幾謂曰:「聞君言,使□幾汗流浹背。」改國子司業、起居舍人,歷太常少卿、國子祭酒。車駕幸太學,命講《書·無逸篇》,賜四品服,除刑部侍郎兼侍講。元祐八年春,多雪,稷言:「今嘉祥未臻,沴氣交作,豈
 應天之實未充,事天之禮未備,畏天之誠未孚歟?宮掖之臣,有關預政事,如天聖之羅崇勛、江德明,治平之任守忠者歟?願陛下昭聖德,祗天戒,總正萬事,以消災祥。」帝親政,召內侍居外者樂士宣等數人。稷言:「陛下初親萬機,未聞登進忠良,而首召近幸,恐上累大德。」



 以集賢院學士知穎州、江寧府,拜吏部侍郎,又出知河南府,加龍圖閣待制。章惇欲困以道路,連歲亟徙六州。徽宗立,以左諫議大夫召,道除御史中丞。入對,與蔡京遇,京越
 班揖曰:「天子自外服召公中執法,今日必有高論。」稷正色答曰:「行自知之。」是日,論京奸狀,既而陳瓘、江公望皆言之,未能動。稷語陳師錫等曰:「京在朝,吾屬何面目居此?」擊之不已,京遂去翰林。又乞辨宣仁誣謗之禍,且言:「史臣以王安石《日錄》亂《神宗實錄》,今方修《哲宗實錄》,願申飭之。」時宦官漸盛,稷懷《唐書·仇士良傳》讀於帝前,讀數行,帝曰:「已諭。」稷為若不聞者,讀畢乃止。



 曾布得助嬖暱,將拜相,稷約其僚共論之。俄轉工部尚書兼侍讀,布
 遂相。稷謝表有佞臣之語,帝問為誰,對曰:「曾布也。陛下斥之外郡,則天下事定矣。」改禮部。論宋用臣不當賜美謚,不為書敕。哲宗升祔,議功臣配享,稷以為當用司馬光、呂公著。或謂二人嘗得罪,不可用。稷曰:「止論其有功於時爾,如唐五王豈非得罪於中宗,何嫌於配享?」又言:「陛下以『建中靖國』紀元,臣謂尊賢納諫,舍己從人,是謂『建中』;不作奇技淫巧,毋使近習招權,是謂『靖國』。以副體元謹始之義。」禁內織錦緣宮簾為地衣,稷言:「仁宗衾褥
 用黃絁,服御用縑繒,宜守家法。」詔罷之。



 稷盡言守正,帝待之厚,將處之尚書左丞,而積忤貴近,不得留,竟以樞密直學士守越。蔡京得政,修故怨,貶海州團練副使、道州別駕,安置臺州。除名徙建州,稍復朝請郎。卒,年七十五。建炎中,追復學士,謚曰清敏。



 初,文彥博嘗品稷為人似趙抃,及賜謚,皆以「清」得名。稷三任言責,每草疏,必密室,子弟亦不得見。退多焚稿,未嘗以時政語人。所薦士如張庭堅、馬涓、陳瓘、陳師錫、鄒浩、蔡肇,皆知名當世云。



 論曰:熙寧行新法,輕進少年爭趨競進,老成知務者逡巡引退,何其見幾之明耶?獬議論剴切,精練民事,青苗法行,獬獨幡然求去,至窘迫不堪,弗恤也。襄奮起海隅,屢折不變,學者卒從而化,乃心民事,死猶不已。公輔以忤安石見黜,洙為諫官不能言,至免役取贏,洙方力爭,所謂不揣其本者歟!稷劾蔡京,論司馬光、呂公著當配享廟庭,蓋亦名侍從也。



 呂誨,字獻可,開封人。祖端,相太宗、真宗。誨性純厚,家居
 力學,不妄與人交。進士登第,由屯田員外郎為殿中侍御史。時廷臣多上章訐人罪,誨言:「臺諫官許風聞言事,蓋欲廣採納以補闕政。茍非職分,是為侵官。今乃詆斥平生,暴揚曖昧,刻薄之態浸以成風,請下詔懲革。」樞密副使程戡結貴幸,致位政地,誨疏其過,以宣徽使判延州。復上言:「戡以非才罷,不宜更委邊任;宣徽使地高位重,非戡所當得也。」兗國公主薄其夫,夜開禁門入訴。誨請並劾閽吏,且治主第宦者罪,悉逐之。御藥供奉官四
 人遙領團練使,御前忠佐當汰復留,誨劾樞密使宋庠陰求援助,徇私紊法。詔罷庠而用陳升之為副使,誨又論之。升之既去,誨亦出知江州,時嘉祐六年也。



 上疏請蚤建皇嗣,曰:「竊聞中外臣僚,以聖嗣未立,屢有密疏請擇宗人。唯陛下思忠言,奮獨斷,以遏未然之亂。又聞太史奏,彗躔心宿,請備西北。按《天文志》,心為天王正位,前星為太子,直則失勢,明則見祥。今既直且暗,而妖彗乘之,臣恐咎證不獨在西北也。自夏及秋,雨淫地震,陰盛
 之沴,固有冥符。近者宗室之中,訛言事露,流傳四方,人心駭惑,窺覦之志,可不防其漸哉!願為社稷宗廟計,審擇親賢,稽合天意,宸謀已定,當使天下共知。萬一有奸臣附會其間,陽為忠實,以緩上心,此為患最大,不可不察也。」仁宗以誨章付中書韓琦,由此定議。



 召為侍御史,改同知諫院。英宗不豫,誨請皇太后日命大臣一員,與淮陽王視進藥餌。都知任守忠用事久,帝之立非守忠意,數間諜東朝,播為惡言,內外洶懼。誨上兩宮書,開陳
 大義,詞旨深切,多人所難言者。帝疾小愈,屢言乞親萬幾。太后歸政,誨言於帝曰:「後輔佐先帝歷年,閱天下事多矣。事之大者,宜關白咨訪然後行,示弗敢專。」遂論守忠平生罪惡,並其黨史昭錫竄之南方。內臣王昭明等為陜西四路鈐轄,專主蕃部。誨言:「自唐以來,舉兵不利,未有不自監軍者。今走馬承受官品至卑,一路已不勝其害,況鈐轄乎?」卒罷之。



 治平二年,遷兵部員外郎,兼侍御史知雜事。上言:「臺諫者,人主之耳目,期補益聰明,以
 防壅蔽。舊三院御史,常有二十員,而後益衰減,蓋執政者不欲主上聞中外之闕失。今臺闕中丞,御吏五員,惟三人在職,封章十上,報聞者八九。諫官二人,一他遷,一出使,言路壅塞,未有如今日之甚者。竊為陛下羞之。」帝覽奏,即命邵必知諫院。



 於是濮議起,侍從請稱王為皇伯,中書不以為然,誨引義固爭。會秋大水,誨言:「陛下有過舉而災沴遽作,惟濮王一事失中,此簡宗廟之罰也。」郊廟禮畢,復申前議,七上章,不聽;乞解臺職,亦不聽。遂
 劾宰相韓琦不忠五罪,曰:「昭陵之土未幹,遽欲追崇濮王,使陛下厚所生而薄所繼,隆小宗而絕大宗。言者論辨累月,琦猶遂非,不為改正,中外憤鬱,萬口一詞。願黜居外藩,以慰士論。」又與御史范純仁、呂大防共劾歐陽修「首開邪議,以枉道說人主,以近利負先帝,陷陛下於過舉」。皆不報。已而詔濮王稱親,誨等知言不用,即上還告敕,居家待罪,且言與輔臣勢難兩立。帝以問執政,修曰:「御史以為理難並立,若臣等有罪,當留御史。」帝猶豫
 久之,命出御史,既而曰:「不宜責之太重。」乃下遷誨工部員外郎、知蘄州。



 神宗立,徙晉州,加集賢殿修撰、知河中府。召為鹽鐵副使,擢天章閣待制,復知諫院,拜御史中丞。初,中旨下京東買金數萬兩,又令廣東市真珠,傳云將備宮中十閣用度。誨言:「陛下春秋富盛,然聰明睿知,以天下為心,必不留神於此,願亟罷之。」



 王安石執政,時多謂得人。誨言其不通時事,大用之,則非所宜。著作佐郎章闢光上言,岐王顥宜遷居外邸。皇太后怒,帝令治
 其離間之罪。安石謂無罪。誨請下闢光吏,不從,遂上疏劾安石曰:「大奸似忠,大佞似信,安石外示樸野,中藏巧詐,陛下悅其才辨而委任之。安石初無遠略,惟務改作立異,罔上欺下,文言飾非,誤天下蒼生,必斯人也。如久居廟堂,必無安靜之理。闢光之謀,本安石及呂惠卿所導。闢光揚言:『朝廷若深罪我,我終不置此二人。』故力加營救。願察於隱伏,質之士論,然後知臣言之當否。」帝方注倚安石,還其章。誨求去,帝謂曾公亮曰:「若出誨,恐安
 石不自安。」安石曰:「臣以身許國,陛下處之有義,臣何敢以形跡自嫌,茍為去就。」乃出誨知鄧州。蘇頌當制,公亮謂之曰:「闢光治平四年上書時,安石在金陵,惠卿監杭州酒稅,安得而教之?」故制詞云:「黨小人交譖之言,肆罔上無根之語。」制出,帝以咎頌,以公亮之言告,乃知闢光治平時自言他事,非此也。誨之將有言也,司馬光勸止之,誨曰:「安石雖有時名,然好執偏見,輕信奸回,喜人佞己。聽其言則美,施於用則疏;置諸宰輔,天下必受其禍。
 且上新嗣位,所與朝夕圖議者,二三執政而已,茍非其人,將敗國事。此乃腹心之疾,救之惟恐不逮,顧可緩耶?」誨既斥,安石益橫。光由則服誨之先見,自以為不及也。



 明年,改知河南,命未下而寢疾矣。旋提舉崇福宮,以疾表求致仕曰:「臣本無宿疾,醫者用術乖方,妄投湯劑,率任情意,差之指下,禍延四支。一身之微,固無足恤,奈九族之托何!」蓋以身疾諭朝政也。



 誨三居言責,皆以彈奏大臣而去,一時推其鯁直。居病困,猶旦夕憤嘆,以天下
 事為憂。既革,司馬光往省之,至則目已瞑。聞光哭,蹶然而起,張目強視曰:「天下事尚可為,君實勉之。」光曰:「更有以見屬乎?」曰:「無有。」遂卒,年五十八,海內聞者痛惜之。



 元祐初,呂大防、范純仁、劉摯表其忠,詔贈通議大夫,以其子由庚為太常寺太祝。自誨罷去,御史劉述、劉琦、錢顗皆以言安石被黜。



 劉述字孝叔,湖州人。舉進士,為御史臺主簿,知溫、耀、真三州,提點江西刑獄,累官都官員外郎,六年不奏考功
 課。知審官院胡宿言其沉靜有守,特遷兵部員外郎,改荊湖南北、京西路轉運使,再以覃恩遷刑部郎中。



 神宗立,召為侍御史知雜事,又十一年不奏課。帝知其久次,授吏部郎中。嘗言去奢當自後宮始,章闢光宜誅,高居簡宜黜,張方平不當參大政,王拱辰不當除宣徽使。皆不報。滕甫為中丞,述將論之。甫聞,先請對。甫退,述乃言甫為言官無所發明,且擿其隱慝。帝曰:「甫遇事輒爭,裨益甚多,但外人不知耳。甫談卿美不輟口,卿無言也。」



 王
 安石參知政事,帝下詔專令中丞舉御史,不限官高卑。趙抃爭之,弗得。述言:「舊制,舉御史官,須中行員外郎至太常博士,資任須實歷通判,又必翰林眾學士與本臺丞雜互舉。蓋眾議僉舉,則各務盡心,不容有偏蔽私愛之患。今專委中丞,則愛憎在於一己。若一一得人,猶不至生事;萬一非其人,將受權臣屬托,自立黨援,不附己者得以中傷,媒薛誣陷,其弊不一。夫變更法度,其事不輕,而止是參知政事二人,同書札子。且宰相富弼暫謁
 告,曾公亮已入朝,臺官今不闕人,何至急疾如此!願收還前旨,俟弼出,與公亮同議,然後行之。」弗聽。



 述兼判刑部,安石爭謀殺刑名,述不以為是。及敕下,述封還中書,奏執不已。安石白帝,詔開封府推官王克臣劾述罪。於是述率御史劉琦、錢顗共上疏曰:「安石執政以來,未逾數月,中外人情囂然胥動。蓋以專肆胸臆,輕易憲度,無忌憚之心故也。陛下任賢求治,常若饑渴,故置安石政府。必欲致時如唐、虞,而反操管、商權詐之術,規以取媚。
 遂與陳升之合謀,侵三司利柄,取為己功;開局設官,用八人者分行天下,驚駭物聽,動搖人心。去年因許遵文過飾非,妄議自首按問之法,安石任一偏之見,改立新議,以害天下大公。章闢光獻岐邸遷外之說,疏間骨肉,罪不容誅。呂誨等連章論奏,乞加竄逐。陛下雖許其請,安石獨進瞽言,熒惑聖聽。陛下以為愛己,隱忍不行。先朝所立制度,自宜世世子孫,守而勿失;乃欲事事更張,廢而不用。安石自應舉歷官,尊尚堯、舜之道,以倡率學
 者,故士人之心靡不歸向,謂之為賢。陛下亦聞而知之,遂正位公府。遭時得君如此之專,乃首建財利之議,務為容悅,言行乖戾,一至於此。剛狠自任,則又甚焉。奸許專權之人,豈宜處之廟堂,以亂國紀!願早罷逐,以慰安天下元元之心。曾公亮位居丞弼,不能竭忠許國,反有畏避之意,陰自結援以固寵,久妨賢路,亦宜斥免。趙抃則括囊拱手,但務依違大臣,事君豈當如是!」



 疏上,安石奏先貶琦、顗監處、衢州鹽務。公亮疑太重,安石曰:「蔣之
 奇亦降監,當從之。」司馬光乃上疏曰:「臣聞孔子曰:『守道不如守官。』孟子曰:『有言責者,不得其言則去。』此古今通義,人臣之大節也。彼謀殺已傷自首刑名,天下皆知其非。朝廷既違眾議而行之,又以守官之臣而罪之,臣恐失天下之心也。夫紲食鷹鸇者,求其鷙也,鷙而烹之,將安用哉!今琦、顗所坐,不過疏直,乃以迕犯大臣,猥加譴謫,恐臣下自此以言為諱。乞還其本資,以靖群聽。」不報。



 開封獄具,述三問不承。安石欲置之獄,光又與范純仁爭
 之,乃議貶為通判。帝不許,以知江州。逾歲,提舉崇禧觀。卒,年七十二,紹興初,贈秘閣修撰。



 劉琦,字公玉,宣城人。博學強覽,立志峻潔。以都官員外郎通判歙州。召為侍御史,建言:「自城綏州,數致羌寇,宜棄之。」浙西開漕渠,役甚小,使者張大其事,以功遷官。言者論其非,詔琦就劾,官吏人人惴恐。琦但按首謀二人而已。既貶,通判鄧州而卒,年六十一。



 錢顗,字安道,常州無錫人。初為寧海軍節度推官,守孫
 沔用威嚴為治,屬吏奔走聽命。顗當官而行,無所容撓,遇不可,必爭之,由是獨見器重。知贛、烏程二縣,皆以治行聞。



 治平末,以金部員外郎為殿中侍御史裏行。許遵議謀殺案問刑名,未定而入判大理,顗以為:「一人偏詞,不可以汨天下之法,遵所見迂執,不可以當刑法之任。」不從。二年而貶,將出臺,於眾中責同列孫昌齡曰:「平日士大夫未嘗知君名,徒以昔官金陵,媚事王安石,宛轉薦君,得為御史。亦當少思執國,奈何專欲附會以求美
 官?



 顗今當遠竄,君自謂得策邪?我視君犬彘之不如也。」即拂衣上馬去。



 後自衢徙秀州。家貧母老,至丐貸親舊以給朝晡,而怡然無謫官之色。蘇軾遺以詩,有「烏府先生鐵作肝」之句,世因目為「鐵肝御史」。卒,年五十三。



 鄭俠,字介夫,福州福清人。治平中,隨父官江寧,閉戶苦學。王安石知其名,邀與相見,稱獎之。進士高第,調光州司法參軍。安石居政府。凡所施行,民間不以為便。光有疑獄,俠讞議傅奏,安石悉如其請。俠感為知己,思欲盡
 忠。



 秩滿,徑入都。時初行試法之令,選人中式者超京官,安石欲使以是進,俠以未嘗習法辭。三往見之,問以所聞。對曰:「青苗、免役、保甲、市易數事,與邊鄙用兵,在俠心不能無區區也。」安石不答。俠退不復見,但數以書言法之為民害者。久之,監安上門。安石雖不悅,猶使其子雱來,語以試法。方置修經局,又欲闢為檢討,更命其客黎東美諭意。俠曰:「讀書無幾,不足以辱檢討。所以來,求執經相君門下耳。而相君發言持論,無非以官爵為先,所
 以待士者亦淺矣。果欲援俠而成就之,取其所獻利民便物之事,行其一二,使進而無愧,不亦善乎?」



 是時,免役法出,民商咸以為苦,雖負水、舍發、擔粥、提茶之屬,非納錢者不得販鬻。稅務索市利錢,其末或重於本,商人至以死爭,如是者不一。俠因東美列其事。未幾,詔小夫裨販者免征,商之重者十損其七,他皆無所行。



 是時,自熙寧六年七月不雨,至於七年之三月,人無生意。東北流民,每風沙霾曀,扶攜塞道,羸瘠愁苦,身無完衣。並城民
 買麻糝麥麩,合米為糜,或茹木實草根,至身被鎖械,而負瓦楬木,賣以償官,累累不絕。俠知安石不可諫,悉繪所見為圖,奏疏詣閣門,不納。乃假稱密急,發馬遞上之銀臺司。其略云:「去年大蝗,秋冬亢旱,麥苗焦枯,五種不入,群情懼死;方春斬伐,竭澤而漁,草木魚鱉,亦莫生遂。災患之來,莫之或御。願陛下開倉廩,賑貧乏,取有司掊克不道之政,一切罷去。冀下召和氣,上應天心,延萬姓垂死之命。今臺諫充位,左右輔弼又皆貪猥近利,使夫
 抱道懷識之士,皆不欲與之言,陛下以爵祿名器,駕馭天下忠賢,而使人如此,甚非宗廟社稷之福也。竊聞南征北伐者,皆以其勝捷之勢、山川之形,為圖來獻,料無一人以天下之民質妻鬻子,斬桑壞舍,流離逃散,遑遑不給之狀上聞者。臣謹以逐日所見,繪成一圖,但經眼目,已可涕泣。而況有甚於此者乎!如陛下行臣之言,十日不雨,即乞斬臣宣德門外,以正欺君之罪。」疏奏,神宗反復觀圖,長籲數四,袖以入。是夕,寢不能寐。翌日,命開封
 體放免行錢,三司察市易,司農發常平倉,三衛具熙河所用兵,諸路上民物流散之故。青苗、免役權息追呼,方田、保甲並罷,凡十有八事。民間歡叫相賀。又下責躬詔求言。越三日,大雨,遠近沾洽。輔臣入賀,帝示以俠所進圖狀,且責之,皆再拜謝。



 安石上章求去,外間始知所行之由,群奸切齒,遂以俠付御史,治其擅發馬遞罪。呂惠卿、鄧綰言於帝曰:「陛下數年以來,忘寐與食,成此美政,天下方被其賜;一旦用狂夫之言,罷廢殆盡,豈不惜哉?」
 相與環泣於帝前,於是新法一切如故。



 安石去,惠卿執政,俠又上疏論之。仍取唐魏徵、姚崇、宋璟、李林甫、盧祀傳為兩軸,題曰《正直君子邪曲小人事業圖跡》。在位之臣暗合林甫輩而反於崇、璟者,各以其類,復為書獻之。並言禁中有被甲、登殿等事。惠卿奏為謗訕,編管汀州。御史臺吏楊忠信謁之曰:「御史緘默不言,而君上書不已,是言責在監門而臺中無人也。」取懷中《名臣諫疏》二帙授俠曰:「以此為正人助。」惠卿暴其事,且嗾御史張琥
 並劾馮京為黨與。俠行至太康,還對獄,獄成,惠卿議致之死。帝曰:「俠所言非為身也,忠誠亦可嘉,豈宜深罪?」但徙英州。既至,得僧屋將壓者居之,英人無貧富貴賤皆加敬,爭遣子弟從學,為築室以遷。



 哲宗立,始得歸。蘇軾、孫覺表言之,以為泉州教授。元符七年,再竄於英。徽宗立,赦之,仍還故官,又為蔡京所奪,自是不復出。布衣糲食,屏處田野,然一言一話,未嘗忘君。宣和元年卒,年七十九。里人揭其閭為鄭公坊,州縣皆祀之於學。紹熙初,
 詔贈朝奉郎。官其孫嘉正為山陰尉。



 論曰:誨以言三黜,述、琦、顗窮厄至死,皆充然無悔,身雖不偶,而聲名則昭著於天下後世矣。俠以區區小官,雖未信而諫,能以片言悟主,殃民之法幾於一舉而空之,功雖不成,而此心亦足以白於天下後世。呂惠卿、鄧綰之罪,可勝誅哉!



\end{pinyinscope}