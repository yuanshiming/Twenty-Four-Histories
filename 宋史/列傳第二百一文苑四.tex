\article{列傳第二百一文苑四}

\begin{pinyinscope}

 ○穆修石延年劉潛附蕭貫蘇舜欽尹源黃亢黃鑒楊蟠顏太初郭忠恕



 穆修,字伯長,鄆州人。幼嗜學,不事章句。真宗東封,詔舉齊、魯經行之士,修預選,賜進士出身,調泰州司理參軍。負才,與眾齟齬,通判忌之,使人誣告其罪,貶池州。中道亡至京師,叩登聞鼓訴冤,不報。居貶所歲餘,遇赦得釋,迎母居京師,間出游匄以給養。久之,補潁州文學參軍,徙蔡州。明道中,卒。



 修性剛介,好論斥時病,詆誚權貴,人欲與交結,往往拒之。張知白守亳,亳有豪士作佛廟成,知白使人召修作記,記成,不書士名。士以白金五百遺
 修為壽,且求載名於記,修投金庭下,俶裝去郡。士謝之,終不受,且曰:「吾寧糊口為旅人,終不以匪人污吾文也。」宰相欲識修,且將用為學官,修終不往見。母死,自負櫬以葬,日誦《孝經》、《喪記》,不用浮屠為佛事。



 自五代文敝,國初,柳開始為古文。其後,楊億、劉筠尚聲偶之辭,天下學者靡然從之。修於是時獨以古文稱,蘇舜欽兄弟多從之游。修雖窮死,然一時士大夫稱能文者必曰穆參軍。



 慶歷中,祖無擇訪得所著詩、書、序、記、志等數十首,集為
 三卷。



 石延年,字曼卿,先世幽州人。晉以幽州遺契丹,其祖舉族南走,家於宋城。延年為人跌宕任氣節,讀書通大略,為文勁健,於詩最工而善書。



 累舉進士不中,真宗錄三舉進士,以為三班奉職,延年恥不就。張知白素奇之,謂曰:「母老乃擇祿耶?」延年不得已就命。後以右班殿直改太常寺太祝,知金鄉縣,有治名。用薦者通判乾寧軍,徙永靜軍,為大理評事、館閣校勘,歷光祿、大理寺丞,上書
 章獻太后,請還政天子。太后崩,範諷欲引延年,延年力止之。後諷敗,延年坐與諷善,落職通判海州。久之,為秘閣校理,遷太子中允,同判登聞鼓院。



 嘗上言天下不識戰三十餘年,請為二邊之備。不報。及元昊反,始思其言,召見,稍用其說。命往河東籍鄉兵,凡得十數萬,時邊將遂欲以捍賊,延年笑曰:「此得吾粗也。夫不教之兵勇怯相雜,若怯者見敵而動,則勇者亦牽而潰矣。今既不暇教,宜募其敢行者,則人人皆勝兵也。」又嘗請募人使唃
 廝囉及回鶻舉兵攻元昊,帝嘉納之。



 延年喜劇飲,嘗與劉潛造王氏酒樓對飲,終日不交一言。王氏怪其飲多,以為非常人,益奉美酒肴果,二人飲啖自若,至夕無酒色,相揖而去。明日,都下傳王氏酒樓有二仙來飲,已乃知劉、石也。延年雖酣放,若不可攖以世務,然與人論天下事,是非無不當。



 初,與天章閣待制吳遵路同使河東,及卒,遵路言於朝廷,特官其一子。



 劉潛字仲方,曹州定陶人。少卓逸有大志,好為古文,以
 進士起家,為淄州軍事推官。嘗知蓬萊縣,代還,過鄆州,方與曼卿飲,聞母暴疾,亟歸。母死,潛一慟遂絕,其妻復撫潛大號而死。時人傷之,曰:「子死於孝,妻死於義。」



 同時以文學稱京東者,齊州歷城有李冠,舉進士不第,得同《三禮》出身,調乾寧主簿,卒。有《東皋集》二十卷。



 蕭貫,字貫之,臨江軍新喻人。俊邁能文,尚氣概。舉進士甲科,為大理評事,通判安、宿二州,遷太子中允、直史館。仁宗即位,進太常丞、同判禮院。歷吏部南曹、開封府推
 官、三司鹽鐵判官,為京東轉運使。



 時提舉捉賊劉舜卿善捕盜,號「劉鐵彈」,恃功為不法,前後畏其兇悍,莫敢治。貫至,發之,廢為民。徙江東,改知洪州,累遷尚書刑部員外郎。坐前使江東不察所部吏受賕,降知饒州。



 有撫州司法參軍孫齊者,初以明法得官,以其妻杜氏留里中,而紿娶周氏入蜀。後周欲訴於官,齊斷發誓出杜氏。久之,又納倡陳氏,挈周所生子之撫州。未逾月,周氏至,齊捽置廡下,出偽券曰:「若傭婢也,敢爾邪!」乃殺其所生子。
 周訴於州及轉運使,皆不受。人或告之曰:「得知饒州蕭史君者訴之,事當白矣。」周氏以布衣書姓名,乞食道上,馳告貫。撫非所部,而貫特為治之。更赦,猶編管齊、濠州。遷兵部員外郎,召還,將試知制誥,會營建獻、懿二皇太后陵,未及試而卒。



 貫臨事敢為,不茍合於時。初,感疾,夢綠衣中人召至帝所,賦《禁中曉寒歌》,詞語清麗,人以比唐李賀。



 蘇舜欽,字子美,參知政事易簡之孫。父耆,有才名,嘗為
 工部郎中、直集賢院。舜欽少慷慨有大志,狀貌怪偉。當天聖中,學者為文多病偶對,獨舜欽與河南穆修好為古文、歌詩,一時豪俊多從之游。



 初以父任補太廟齋郎,調滎陽縣尉。玉清昭應宮災,舜欽年二十一,詣登聞鼓院上疏曰:



 烈士不避鈇鉞而進諫,明君不諱過失而納忠,是以懷策者必吐上前,蓄冤者無至腹誹。然言之難不如容之難,容之難不如行之難,有言之必容之行之,則三代之主也,幸陛下留聽焉。



 臣觀今歲自春徂夏,霖雨
 陰晦,未嘗少止,農田被災者幾於十九。臣以謂任用失人、政令多過、賞罰弗中之所召也。天之降災,欲悟陛下,而大臣歸咎於刑獄之濫,陛下聽之,故肆赦天下以為禳救。如此則是殺人者不死,傷人者不抵罪,而欲以合天意也。古者斷決滯訟以平水旱,不聞用赦,故赦下之後,陰雨及今。



 前志曰:「積陰生陽,陽生則火災見焉。」乘夏之氣發洩於玉清宮,震雨雜下,烈焰四起,樓觀萬疊,數刻而盡,非慢於火備,乃天之垂戒也。陛下當降服、減膳、
 避正寢,責躬罪己,下哀痛之詔,罷非業之作,拯失職之民,察輔弼及左右無裨國體者罷之,竊弄權威者去之;念政刑之失,收芻蕘之論,庶幾所以變災為祐。



 浹日之間,未聞為此,而將計工役以圖修復,都下之人聞者駭惑,聚首橫議,咸謂非宜。皆曰章聖皇帝勤儉十餘年,天上富庶,帑府流衍,乃作斯宮,及其畢功,海內虛竭。陛下即位及十年,數遭水旱,雖征賦咸入,而百姓困乏。若大興土木,則費知紀極,財力耗於內,百姓勞於下,
 內耗下勞,何以為國!況天災之,己違之,是欲競天,無省己之意。逆天不祥,安己難任,欲祈厚貺,其可得乎!今為陛下計,莫若求吉士,去佞人,修德以勤至治,使百姓足給而征稅寬減,則可以謝天意而安民情矣。



 夫賢君見變,修道除兇,亂世無象,天不譴告。今幸天見之變,是陛下修己之日,豈可忽哉!昔漢元帝三年,茂陵白鶴館災,詔曰:「乃者火災降於孝武園館,朕戰心慄恐懼,不燭變異,罪在朕躬。群有司又不肯極言朕過,以至於斯,將何寤
 焉!」夫茂陵不及上都,白鶴館大不及此宮,彼尚降詔四方,以求己過,是知帝王憂危念治,汲汲如此。



 臣又按《五行志》:賢佞分別,官人有敘,率由舊章,禮重功勛,則火得其性。若信道不篤,或耀虛偽,讒夫昌,邪勝正,則火失其性,自上而降。及濫炎妄起,燔宗廟,燒宮室,雖興師徒而不能救。魯成公三年,新宮災,劉向謂成公信三桓子孫之讒、逐父臣之應。襄公九年春,宋火,劉向謂宋公聽讒、逐其大夫華弱奔魯之應。今宮災豈亦有是乎?願陛下
 拱默內省而追革之,罷再造之勞,述前世之法,天下之幸也。



 又上書曰:



 歷觀前代聖神之君,好聞讜議,蓋以四海至遠,民有隱慝,不可以遍照,故無間愚賤之言而擇用之。然後朝無遺政,物無遁情,雖有佞臣,邪謀莫得而進也。



 臣睹乙亥詔書,戒越職言事,播告四方,無不驚惑,往往竊議,恐非出陛下之意。蓋陛下即位以來,屢詔群下勤求直言,使百僚轉對,置匭函,設直言極諫科。今詔書頓異前事,豈非大臣雍蔽陛下聰明,杜塞忠良之口,
 不惟虧損朝政,實亦自取覆亡之道。夫納善進賢,宰相之事,蔽君自任,未或不亡。今諫官、御史悉出其門,但希旨意,即獲美官,多士盈庭。噤不得語。陛下拱默,何由盡聞天下之事乎?



 前孔道輔、範仲淹剛直不撓,致位臺諫,後雖改他官,不忘獻納。二臣者非不知緘口數年,坐得卿輔,蓋不敢負陛下委注之意。而皆罹中傷,竄謫而去,使正臣奪氣,鯁士咋舌,目睹時弊而不敢論。



 昔晉侯問叔向曰:「國家之患孰為大?」對曰:「大臣持祿而不極諫,小
 臣畏罪而不敢言,下情不得上通,此患之大者。」故漢文感女子之說而肉刑是除,武帝聽三老之議而江充以族。肉刑古法,江充近臣,女子三老,愚耄疏隔之至也。蓋以義之所在,賤不可忽,二君從之,後世稱聖。況國家班設爵位,列陳豪英,故當責其公忠,安可教之循默?賞之使諫,尚恐不言;罪其敢言,孰肯獻納?物情閉塞,上位孤危,軫念於茲,可為驚怛!覬望陛下發德音,寢前詔,勤於採納,下及芻蕘,可以常守隆平,保全近輔。



 尋舉進士,改
 光祿寺主簿,知長垣縣,遷大理評事,監在京店宅務。康定中,河東地震,舜欽詣匭通疏曰:



 臣聞河東地大震裂,湧水壞屋廬城堞,殺民畜幾十萬,歷旬不止。始聞惶駭疑惑。竊思自編策所紀前代衰微喪亂之世,亦未嘗有此大變。今四聖接統,內外平寧,戎夷交歡,兵革偃息,固與夫衰微喪亂之世異,何災變之作反過之耶?且妖祥之興,神實尸之,各以類告,未嘗妄也。天人之應,古今之鑒,大可恐懼。豈王者安於逸豫、信任近臣而不省政事乎?
 廟堂之上,有非才茍祿、竊弄威福而侵上事者乎?又豈施設之政有不便民者乎?深宮之中,有陰教不謹以媚道進者乎?西北羌夷有背盟犯順之心乎?臣從遠方來,不知近事,心疑而口不敢道也。所怪者,朝廷見此大異,不修闕政,以厭天戒、安民心,默然不恤,如無事之時。諫官、御史不聞進牘鋪白災害之端,以開上心。然民情洶洶,聚首橫議,咸有憂悸之色。



 臣以世受君祿,身齒國命,涵濡惠澤,以長此軀,目睹心思,驚怛流汗,欲盡吐肝膽,以
 拜封奏。又見範仲淹以剛直忤奸臣,言不用而身竄謫,降詔天下,不許越職言事。臣不避權右,必恐橫罹中傷,無補於國,因自悲嗟,不知所措。



 既而孟春之初,雷震暴作,臣以謂國家闕失,眾臣莫敢為陛下言者,唯天丁寧以告陛下。陛下果能沛發明詔,許群臣皆得獻言,臣初聞之踴躍欣抃。旬日間頗有言事者,其間豈無切中時病,而未聞朝廷舉而行之,是亦收虛言而不根實效也。臣聞唯誠可以應天,唯實可以安民,今應天不以誠,安民不
 以實,徒布空文,增人太息耳,將何以謝神靈而救弊亂也!豈大臣蒙塞天聽,不為陛下行之?豈言事迂闊無所取,不足行也?臣竊見綱紀隳敗,政化闕失,其事甚眾,不可概舉,謹條大者二事以聞:



 一曰正心。夫治國如治家,治家者先修己,修己者先正心,心正則神明集而萬務理。今民間傳陛下比年稍邇俳優賤人,燕樂逾節,賜予過度。燕樂逾節則蕩,賜予過度則侈。蕩則政事不親,侈則用度不足。臣竊觀國史,見祖宗日視朝,旰昃方罷,猶
 坐於後苑,門有白事者,立得召對,委曲詢訪,小善必納。真宗末年不豫,始間日視事。今陛下春秋鼎盛,實宵衣旰食求治之秋,而乃隔日御殿,此政事不親也。又府庫匱竭,民鮮蓋藏,誅斂科率,殆無虛日。計度經費,二十倍於祖宗時,此用度不足也。政事不親,用度不足,誠國大憂。臣望陛下修己以御人,洗心以鑒物,勤聽斷,舍燕安,放棄優諧近習之纖人,親近剛明鯁直之良士。因此災變,以思永圖,則天下幸甚。



 其二曰擇賢。夫明主勞於求
 賢而逸於任使,然盈庭之士不須盡擇,在擇一二輔臣及御史、諫官而已。陛下用人尚未慎擇。昨王隨自吏部侍郎遷門下侍郎平章事,超越十資,復為上相。此乃非常之恩,必待非常之才,而隨虛庸邪諂,非輔相之器,降麻之後,物論沸騰。故疾纏其身,災仍於國,此亦天意愛惜我朝,陛下鑒之哉!且石中立頃在朝行,以詼諧自任,士人或有宴集,必置席間,聽其語言,以資笑噱。今處之近輔,不聞嘉謀,物望甚輕,人情所忽,使災害屢降而朝
 廷不尊,蓋近臣多非才者。陛下左右尚如此,天下官吏可知也。實恐遠人輕笑中國,宜即行罷免,別選賢才。又張觀為御史中丞,高若訥為司諫,二人者皆登高第,頗以文詞進,而溫和軟懦,無剛鯁敢言之氣。斯皆執政引拔建置,欲其慎默,不敢舉揚其私,時有所言,則必暗相關說,旁人窺之,甚可笑也。故御史、諫官之任,臣欲陛下親擇之,不令出執政門下。臺諫官既得其人,則近臣不敢為過,乃馭下之策也。



 臣以謂陛下身既勤儉,輔弼、臺
 諫又皆得人,則天下何憂不治,災異何由而生?惟陛下少留意焉。



 範仲淹薦其才,召試,為集賢校理,監進奏院。舜欽娶宰相杜衍女,衍時與仲淹、富弼在政府,多引用一時聞人,欲更張庶事。御史中丞王拱辰等不便其所為。會進奏院祠神,舜欽與右班殿直劉巽輒用鬻故紙公錢召妓樂,間夕會賓客。拱辰廉得之,諷其屬魚周詢等劾奏,因欲搖動衍。事下開封府劾治,於是舜欽與巽俱坐自盜除名,同時會者皆知名士,因緣得罪逐出四
 方者十餘人。世以為過薄,而拱辰等方自喜曰:「吾一舉網盡矣。」



 舜欽既放廢,寓於吳中,其友人韓維責以世居京師而去離都下,隔絕親交。舜欽報書曰:



 蒙聞責以兄弟在京師,不以義相就,獨羈外數千里,自取愁苦。予豈無親戚之情,豈不知會合之樂也?安肯舍安逸而甘愁苦哉!



 昨在京師,不敢犯人顏色,不敢議論時事,隨眾上下,心志蟠屈不開,固亦極矣。不幸適在嫌疑之地,不能決然早自引去,致不測之禍,捽去下吏,人無敢言,友仇
 一波,共起謗議。被廢之後,喧然未已,更欲置之死地然後為快。來者往往鉤賾言語,欲以傳播,好意相恤者幾希矣。故閉戶不敢與相見,如避兵寇。偷俗如此,安可久居其間!遂超然遠舉,羈泊於江湖之上,不唯衣食之累,實亦少避機阱也。



 況血屬之多,資入之薄,持國見之矣。常相團聚,可乏衣食乎?不可也。可閉關常不與人接乎?不可也。與人接必與之言,與之言必與之還往,使人人皆如持國則可,不迨持國者必加釀惡言,喧布上下,使
 僕不能自明,則前日之事未為重也。



 都無此事,亦終日勞苦,應接之不暇,寒暑奔走塵土泥淖中,不能了人事,羸馬餓僕,日棲棲取辱於都城,使人指背譏笑哀閔,亦何顏面,安得不謂之愁苦哉!



 此雖與兄弟親戚相遠,而伏臘稍足,居室稍寬,無終日應接奔走之勞,耳目清曠,不設機關以待人,心安閑而體舒放。三商而眠,高舂而起,靜院明窗之下,羅列圖史琴樽以自愉悅,有興則泛小舟出盤、閶二門,吟嘯覽古於江山之間。渚茶、野釀足
 以銷憂,菁鱸、稻蟹足以適口。又多高僧隱君子,佛廟勝絕,家有園林,珍花奇石,曲池高臺,魚鳥留連,不覺日暮。



 昔孔子作《春秋》而夷吳,又曰:「吾欲居九夷。」觀今之風俗,樂善好事,知予守道好學,皆欣然願來過從,不以罪人相遇,雖孔子復生,是亦必欲居此也。以彼此較之,孰為然哉!人生內有自得,外有所適,固亦樂矣,何必高位厚祿,役人以自奉養,然後為樂?今雖僑此,亦如仕宦南北,安可與親戚常相守耶!予窘迫,勢不得如持國意,必使
 我尸轉溝洫,肉餧豺虎,而後以為安所義,何其忍耶!《詩》曰:「凡今之人,莫如兄弟。」謂兄弟以恩,急難必相拯救。後章曰:「喪亂既平,既安且寧,雖有兄弟,不如友生。」謂友朋尚義,安寧之時,以禮義相琢磨。予於持國,外兄弟也。急難不相救,又於未安寧之際,欲以義相琢刻,雖古人所不能受,予欲不報,慮淺吾持國也。



 二年,得湖州長史,卒。舜欽數上書論朝廷事,在蘇州買水石作滄浪亭,益讀書,時發憤懣於歌詩,其體豪放,往往驚人。善草書,每酣酒
 落筆,爭為人所傳。及謫死。世尤惜之。妻杜氏,有賢行。



 兄舜元,字才翁,為人精悍任氣節,為歌詩亦豪健,尤善草書,舜欽不能及。官至尚書度支員外郎、三司度支判官。



 尹源,字子漸,少博學強記,與弟洙皆以文學知名,洙議論明辨,果於有為。源自晦,不矜飾,有所發即過人。初以祖蔭補三班借職,稍遷殿直。舉進士,為奉禮郎,累遷太常博士,歷知芮城、河陽、新鄭三縣,通判涇州。時知滄州劉渙坐專斬部卒,降知密州。源上書言:「渙為主將,部卒
 有罪不伏,笞輒呼萬歲,渙斬之不為過。以此謫渙,臣恐邊兵愈驕,輕視主將,所系非輕也。」渙遂獲免。



 嘗作《唐說》及《敘兵》十篇上之。其《唐說》曰:



 世言唐所以亡,由諸侯之強,此未極於理。夫弱唐者,諸侯也。唐既弱矣,而久不亡者,諸侯維之也。燕、趙、魏首亂唐制,專地而治,若古之建國,此諸侯之雄者,然皆恃唐為輕重。何則?假王命以相制則易而順,唐雖病之,亦不得而外焉。故河北順而聽命,則天下為亂者不能遂其亂;河北不順而變,則奸雄
 或附而起。德宗世,朱泚、李希烈始遂其僭而終敗亡,田悅叛於前,武俊順於後也。憲宗討蜀、平夏、誅蔡、夷鄆,兵連四方而亂不生,卒成中興之功者,田氏稟命、王承宗歸國也。武宗將討劉稹之叛,先正三鎮,絕其連衡之計,而王誅以成。如是二百年,奸臣逆子專國命者有之,夷將相者有之,而不敢窺神器,非力不足,畏諸侯之勢也。



 及廣明之後,關東無復唐有,方鎮相侵伐者,猶以王室為名。及梁祖舉河南,劉仁恭輕戰而敗,羅氏內附,王鎔
 請盟,於時河北之事去矣。梁人一舉而代唐有國,諸侯莫能與之爭,其勢然也。向使以僖、昭之弱,乘巢、蔡之亂,而田承嗣守魏,王武俊、朱滔據燕、趙,強相均,地相屬,其勢宜莫敢先動,況非義舉乎?如此雖梁祖之暴,不過取霸於一方耳,安能強禪天下?故唐之弱者,以河北之強也;唐之亡者,以河北之弱也。



 或曰:「諸侯強則分天子之勢,子何議之過乎?」曰:「秦、隋之勢無分於諸侯,而亡速於唐,何如哉?」或曰:「唐之亡其由君失道乎?」曰:「君非失道,而
 才不至焉爾,其亡也,臣實主之。請極其說:唐太宗起艱難有天下,其用臣也,聽其言而盡其才,故君臣相親而至治安。以及後世,視太宗由茲而興,雖其聖不及,而任臣納諫之心一也。君有太宗之心,臣非太宗之臣,上聽其下,或不能辨其奸,下惑其上,無所不至,所以敗也。何哉?夫君一而臣眾,大聖之君不相繼而出,大奸之臣則世有之。大聖在上,則奸無所容,其臣莫不賢。茍君之才不能勝臣之奸,則雖有賢者不能進矣。如是,然未至於失道,
 猶失道也。明皇非不欲天下如貞觀之治,而馭臣之才不能勝林甫之奸,於是有祿山之禍。德宗非不欲平暴亂、安四方,而君人之術不能勝盧杞之邪,於是有朱泚之變。以至於僖、昭,其心皆欲去亂而即治也,而才不逮於明皇、德宗,輔臣之奸邪或過於林甫、盧杞,求國不亡,安可得已!然跡其事,君豈有失道乎?於時天下非無賢,由君不能主聽也。故至賢之主與夫失道之主,其興其亡,皆自取之,此系乎君者也。中才之主,其臣正勝邪則
 治而安,邪勝正則亂而亡,此系乎臣者也。然則唐之亡非君之為,臣之為也。」



 其《敘兵》曰:



 唐杜牧當會昌中河朔用兵,嘗為文數篇,上論歷代軍事利害,繼以本朝制兵、用將之得失,下參以當時事機。牧,儒者,位不顯,其術未嘗試,然識者謂牧知兵,雖古名將不能過。今觀牧所著,大要究極當世之務,不專狃古法,使時君可行而易為功,此其善也。



 今兵之利鈍所以與唐世異者,唐自中世以來,諸侯皆自募兵訓練,出攻入守,上下一志,故討淮
 西、青、冀、滄德、澤潞之叛,以至四征夷狄,大率假外兵以集事,朝廷所出神策禁軍,不過為聲援而已,故所至多有功。



 今則不然,國家患前世藩鎮之強,凡天下所募驍勇,一萃於京師。雖濱塞諸郡,大者籍兵不逾數千,每歲防秋,則戍以禁兵,將師任輕而勢分,軍事往往中御。愚謂此可以施於無事時,鎮中國,服豪傑心,茍戎夷侵軼,未必能取勝也。何則?兵主於外則勇,主於內則驕,勇生於勞,驕生於逸。夫外兵所習尚皆疆埸戰鬥勞苦之事,死
 生之命制之於將,故勇,勇而使之戰則多利;內兵居京師,日享安逸,加之以賞賚,未嘗服甲胄、荷戈戟,不知將帥號令之嚴,故驕,驕而勞之則怨,以之戰則多鈍。



 若唐之失,失於諸侯之不制,非失於外兵之強,故有驕將,罕聞有驕兵。今之失,失於將太輕,而外兵不足以應敵,內兵鮮得其用,故有驕兵,不聞有驕將。且唐之所失者勢也,今之所失者制也。勢也者。不得已也,制也者,可為而不為也。



 然則為今之計當如何?曰:「稍革舊制,大募豪勇,
 益外兵之籍,俾足以戰敵。以內兵為聲勢,重邊將之任,使專一軍之事,而不得連州郡之勢,斯可以獲近利而亡後害也。



 餘文多不錄。



 趙元昊寇定川堡,葛懷敏發涇原兵救之,源是時通判慶州,遺懷敏書曰:「賊舉國而來,其利不在城堡,而兵法有不得而救者,宜駐兵瓦亭,擇利而後動。」懷敏不聽,以敗。範仲淹、韓琦薦其才,召試學士院。源素不喜賦,請以論易賦,主試者方以賦進,不悅其言,第其文下,除知懷州,卒。



 黃亢,字清臣,建州浦城人也。母夢星殞於懷,掬而吞之,遂有娠。少奇穎過人,年十五,以文謁翰林學士章得象,得象奇之。游錢塘,以詩贈處士林逋,逋尤激賞。時王隨知杭州,奏禁西湖為放生池,亢作詩數百言以諷,士人爭傳之。亢為人侏儒,不飾小節,對人野率,如不能言。然嗜學強記,為文詞奇偉。卒,鄉人類其文為十二卷,號《東溪集》。



 黃鑒,字唐卿,與亢同鄉里,少敏慧過人。舉進士,補桂陽
 監判官,為國子監直講。同郡楊億尤善其文詞,延置門下,由是知名。累遷太常博士,為國史院編修官。嘗詔館閣官後苑賞花,而鑒特預召。國史成,擢直集賢院。以母老,出通判蘇州,卒。



 楊蟠,字公濟,章安人也。舉進士,為密、和二州推官。歐陽修稱其詩。蘇軾知杭州,蟠通判州事,與軾倡酬居多。平生為詩數千篇,後知壽州,卒。



 顏太初,字醇之,徐州彭城人,顏子四十七世孫。少博學,
 有雋才,慷慨好義。喜為詩,多譏切時事。天聖中,亳州衛真令黎德潤為吏誣構,死獄中,太初以詩發其冤,覽者壯之。文宣公孔聖祐卒,無子,除襲封且十年。是時有醫許希以針愈仁宗疾,拜賜已,西向拜扁鵲曰「不敢忘師也!」帝為封扁鵲神應侯,立祠城西。太初作《許希詩》,指聖祐事以諷在位,又致書參知政事蔡齊,齊為言於上,遂以聖祐弟襲封。山東人範諷、石延年、劉潛之徒喜豪放劇飲,不循禮法,後生多慕之,太初作《東州逸黨詩》,孔道
 輔深器之。太初中進士後,為莒縣尉,因事忤轉運使,投劾去。久之,補閬中主簿。時範諷以罪貶,同黨皆坐斥,齊與道輔薦太初,上其嘗所為詩,召試中書,言者以為此嘲譏之辭,遂報改臨晉主簿。



 前此有太常博士宋武通判同州,與守爭事,恚死,守憾之,捃構其子以罪,發狂亦死,父子寓骨僧舍。時守方貴顯,無敢為直冤,太初因事至同州,葬武父子,蘇舜欽表其事於墓左。後移應天府戶曹參軍、南京國子監說書,卒。著書號《洙南子》,所居在
 鳧、繹兩山之間,號鳧繹處士。有集十卷,《淳曜聯英》二十卷。



 子復,嘉祐中,本郡敦遣至京師,召試舍人院,為奉議郎。



 郭忠恕,字恕先,河南洛陽人。七歲能誦書屬文,舉童子及第,尤工篆籀。弱冠,漢湘陰公召之,忠恕拂衣遽辭去。周廣順中,召為宗正丞兼國子書學博士,改《周易》博士。



 建隆初,被酒與監察御史符昭文競於朝堂,御史彈奏,忠恕叱臺吏奪其奏,毀之,坐貶為乾州司戶參軍。乘醉
 毆從事範滌,擅離貶所,削籍配隸靈武。其後,流落不復求仕進,多游岐、雍、京、洛間,縱酒跅弛,逢人無貴賤輒呼「苗」。有佳山水即淹留,浹旬不能去。或逾月不食。盛暑暴露日中,體不沾汗,窮冬鑿河水而浴,其傍凌澌消釋,人皆異之。



 尤善畫,所圖屋室重復之狀,頗極精妙。多游王侯公卿家,或待以美醞,豫張紈素倚於壁,乘興即畫之,茍意不欲而固請之,必怒而去,得者藏以為寶。太宗即位,聞其名,召赴闕,授國子監主簿,賜襲衣、銀帶、錢五萬,
 館於太學,令刊定歷代字書。



 忠恕性無檢局,放縱敗度,上憐其才,每優容之。益使酒,肆言謗讟,時擅鬻官物取其直,詔減死,決杖流登州。時太平興國二年。已行至齊州臨邑,謂部送吏曰:「我今逝矣!」因掊地為穴,度可容其面,俯窺焉而卒,稾葬於道側。後累月,故人取其尸將改葬之,其體甚輕,空空然若蟬蛻焉。所定《古今尚書》並《釋文》並行於世。



\end{pinyinscope}