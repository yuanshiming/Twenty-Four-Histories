\article{列傳第二百五十二蠻夷一}

\begin{pinyinscope}

 古者帝王之勤遠略,耀兵四裔,不過欲安內而捍外爾,非所以求逞也。西南諸蠻夷,重山復嶺,雜廁荊、楚、巴、黔、
 巫中,四面皆王土。乃欲揭上腴之征以取不毛之地,疲易使之眾而得梗化之氓,誠何益哉!樹其酋長,使自鎮撫,始終蠻夷遇之,斯計之得也。然無經久之策以控馭之,狌鼯之性便於跳梁,或以仇隙相尋,或以饑饉所逼,長嘯而起,出則衝突州縣,入則負固山林,致煩興師討捕,雖能殄除,而斯民之荼毒深矣。宋恃文教而略武衛,亦豈先王制荒服之道哉!



 西南溪峒諸蠻皆盤瓠種,唐虞為要服。周世,其眾彌盛,宣王命方叔伐之。楚莊既霸,
 遂服於楚。秦昭使白起伐楚,略取蠻夷,置黔中郡,漢改為武陵。後漢建武中,大為寇鈔,遣伏波將軍馬援等至臨沅擊破之,渠帥饑困乞降。曆晉、宋、齊、梁、陳,或叛或服。隋置辰州,唐置錦州、溪州、巫州、敘州,皆其地也。唐季之亂,蠻酋分據其地,自署為刺史。晉天福中,馬希範承襲父業,據有湖南,時蠻徭保聚,依山阻江,殆十餘萬。至周行逢時,數出冠邊,逼辰、永二州,殺掠民畜無寧歲。



 太祖既下荊、湖,思得通蠻情、習險厄、勇智可任者以鎮撫之。
 有辰州徭人秦再雄者,長七尺,武健多謀,在行逢時,屢以戰鬥立功,蠻黨伏之。太祖召至闕下,察其可用,擢辰州刺史,官其子為殿直,賜予甚厚,仍使自辟吏屬,予一州租賦。再雄感恩,誓死報效。至州日訓練士兵,得三千人,皆能被甲渡水,曆山飛塹,捷如猿猱。又選親校二十人分使諸蠻,以傳朝廷懷來之意,莫不從風而靡,各得降表以聞。太祖大喜,復召至闕,面加獎激,改辰州團練使,又以其門客王允成為辰州推官。再雄盡瘁邊圉,五
 州連袤數千里,不增一兵,不費帑庾,終太祖世,邊境無患。又有溪州刺史彭士愁等以溪、錦、獎州歸馬氏,立銅柱為界。



 建隆四年,知溪州彭允林、前溪州刺史田洪贇等列狀歸順,詔以允林為溪州刺史,洪贇為萬州刺史。允林卒,以其子師皎代為刺史。四月,水鬥都虞候林抱義上辰、敘二州圖。



 乾德二年四月,溪、敘、獎等州民相攻劫,遣殿直牛允齎詔諭之,乃定。三年七月,珍州刺史田景遷內附。五溪團練使、洽州刺史田處崇上言:「湖南節
 度馬希範建敘州潭陽縣為懿州,署臣叔父萬盈為刺史。希範卒,其弟希萼襲位,改為洽州,願復舊名。」詔從其請。十二月,詔溪州宜充五溪團練使,刻印以賜之。四年,南州進銅鼓內附,下溪州刺史田思遷亦以銅鼓、虎皮、麝臍來貢。五年冬,以溪州團練使彭允足為濮州牢城都指揮使,溪州義軍都指揮使彭允賢為衛州牢城都指揮使,珍州錄事參軍田思曉為博州牢城都指揮使。允足等溪峒酋豪據山險,持兩端,故因其入朝而置之
 內地。



 開寶元年,珍州刺史田景遷言,本州連歲災沴,乞改為高州,從之。八年,景遷卒,其子衙內都指揮使彥伊來請命,即以為刺史。九年,獎州刺史田處達以丹砂、石英來貢。



 太平興國二年,懿州刺史、五溪都團練使田漢瓊以其子、弟、女夫、大將、五溪統軍都指揮使田漢度而下十二人來貢,詔並加檢校官以獎之。三年,夷州蠻任朗政等來貢。七年,詔辰州不得移部內馬氏所鑄銅柱。溪州刺史彭允殊上言:「刺史舊三年則為州所易,望朝
 廷禁止。」賜敕書安撫之。八年,錦、溪、敘、富四州蠻相率詣辰州,言願比內郡輸租稅。詔長吏察其謠俗情偽,並按視山川地形圖畫來上,卒不許。懿州刺史田漢瓊、錦州刺史田漢希上言,願兩易其地,詔從之。又以知敘州舒德郛為刺史。



 雍熙元年,黔南言溪峒夷獠疾病,擊銅鼓、沙鑼以祀神鬼,詔釋其銅禁。



 淳化二年,知晃州田漢權言,本管砂井步夷人粟忠獲古晃州印一鈕來獻。因請命以漢權為晃州刺史。又以五溪諸州統軍、鶴州刺史
 向通漢為富州刺史,從其請也。是年,荊湖轉運使言,富州向萬通殺皮師勝父子七人,取五藏及首以祀魔鬼。朝廷以其遠俗,令勿問。三年,晁州刺史田漢權、錦州刺史田保全遣使來貢。五年,以舒德言為元州刺史。獎、晃、敘、懿、元、錦、費、福等州皆來貢,上親視器幣以賜之。



 至道元年,高州、溪州並來貢。二年,上親祀南郊,富州刺史向通漢上言:「聖人郊祀,恩浹天壤,況五溪諸州連接十洞,控西南夷戎之地。惟臣州自昔至今,為辰州牆壁,障
 護辰州五邑,王民安居。臣雖僻處遐荒,洗心事上,伏望陛下察臣勤王之誠,因茲郊禮,特加真命。」詔加通漢檢校司徒,進封河內郡侯。



 咸平元年,通漢又言請定租賦,真宗以荒服不征,弗之許。其年,古州刺史向通展以芙蓉朱砂二器、馬十匹、水銀千兩來獻,詔有司鑄印以賜通展。二年,以下溪州刺史彭允殊為右千牛衛將軍致仕,以其侄文勇為刺史。三年,高州刺史田彥伊遣子貢方物及輸兵器。四年,其酋向君猛又遣弟君泰來朝。上
 溪州刺史彭文慶來貢水銀、黃蠟。



 五年正月,天賜州蠻向永豐等二十九人來朝。夔州路轉運使丁謂言:「溪蠻入粟實緣邊砦柵,頓息施、萬諸州饋餉之弊。臣觀自昔和戎安邊,未有境外轉糧給我戍兵者。」先是,蠻人數擾,上召問巡檢使侯廷賞,廷賞曰:「蠻無他求,唯欲鹽爾。」上曰:「此常人所欲,何不與之?」乃詔諭丁謂,謂即傳告陬落,群蠻感悅,因相與盟約,不為寇鈔,負約者,眾殺之。且曰:「天子濟我以食鹽,我願輸與兵食。」 自是邊穀有三年之
 積。七月,高州刺史田彥伊子承寶等百二十二人來朝,賜巾服、器幣,以承寶為山河使、九溪十峒撫諭都監。



 六年四月,丁謂等言,高州義軍務頭角田承進等擒生蠻六百六十餘人,奪所略漢口四百餘人。初,益州軍亂,議者恐緣江下峽,乃集施、黔、高、溪蠻豪子弟捍禦,群蠻因熟漢路,寇略而歸。謂等至,即召與盟,令還漢口。既而有生蠻違約,謂遣承進率眾及發州兵擒獲之,焚其室廬,皆震懾伏罪。謂乃置尖木砦施州界,以控扼之,自是寇
 鈔始息,邊溪峒田民得耕種。七月,南高州義軍指揮使田彥強、防虞指揮使田承海來貢,施州叛蠻譚仲通等三十餘人來歸。



 景德元年,高州五姓義軍指揮使田文鄯來貢。富州刺史向通漢遣使潭州營佛事,以報朝廷存恤之惠。二年,夔州路降蠻首領皆自署職名,請因而命之,上不許,第令次補牙校。是歲,辰州諸蠻攻下溪州,為其刺史彭儒猛擊走之,擒酋首以獻,詔賜儒猛錦袍、銀帶。儒猛自陳母老,願被恩典,詔特加邑封。十二月,荊
 湖北路言,溪峒團練使彭文綰送還先陷漢口五十人,詔授文綰檢校太子賓客,知中彭州。其年,懿州刺史田漢希卒,以其子漢能為刺史。三年,高州新附蠻酋八十九人來貢。五溪都防禦使向通漢表求追贈父母,從之。溪州刺史彭文慶率溪峒群蠻來朝。又高州諸名豪百餘人入貢。四年五月,以高州刺史田彥伊子承寶為寧武郎將,高州土軍都指揮使田思欽為安化郎將。其年,宜州軍亂,朝廷恐宜、融溪峒因緣侵擾,因降詔約勒首
 領,皆奉詔,部分種族,無敢輒動。



 大中祥符元年,夔州路言,五團蠻嘯聚,謀劫高州,欲令暗利砦援之。上以蠻夷自相攻,不許發兵。三月,知元州舒君強、知古州向光普並加銀青光祿大夫、檢校太子賓客。八月,黔州言,磨嵯、洛浦蠻首領龔行滿等率族二千三百人歸順。十月,溪峒諸蠻獻方物於泰山。三年,澧州言,慈利縣蠻相仇劫,知州劉仁霸請率兵定之。上恐深入蠻境,使其疑懼,止令仁霸宣諭詔旨,遂皆感服。四年,安、遠、順、南、永寧、濁水
 州蠻酋田承曉等三百七十三人來貢。五年,詔:「昨許溪峒蠻夷歸先劫漢口及五十人者,特署職名,仍聽來貢。如聞緣此要利,輒掠邊民充數,所在切辨察之。」其年,夔蠻千五百人乞朝貢,上慮其勞費,不許。又詔:施州溪蠻朔望犒以酒肴。閏十月,五溪蠻向貴升及磨嵯、洛浦蠻來貢。六年,夔州蠻彭延暹、龔才晃等來貢。辰州溪峒都指揮使魏進武率山徭數百人數寇城砦,朝廷不欲發兵窮討,乃降詔招諭。七年,進武詣吏請罪,署為三班借
 職,監房州稅,仍賜裝錢。八年,詔中彭州彭文綰歲賜錦袍。



 天禧元年,溪州蠻寇擾,遣兵討之。二年,辰州都巡檢使李守元率兵入白霧團,擒蠻寇十五人,斬首百級,降其酋二百餘人。知辰州錢絳等入下溪州,破砦柵,斬蠻六十餘人,降老幼千餘。刺史彭儒猛亡入山林,執其子仕漢等赴闕。詔高州蠻,捕儒猛來獻者厚加賞典。其年,儒猛因順州蠻田彥晏上狀本路,自訴求歸。轉運使以聞,上哀憐之,特許釋罪。儒猛乃奉上所略民口、器甲,詔
 辰州通判劉中象召至明灘,與歃血要盟,遣之。詔以仕漢為殿直,儒霸、儒聰為借職,賜冠帶、緡帛。富州刺史向通漢率所部來朝,貢名馬、丹砂、銀裝劍槊、兜鍪、彩牌等物。詔賜襲衣、金帶、鞍勒馬,並其子光澤以下器幣有差,特許通漢五日一朝。逾月,通漢上《五溪地理圖》,願留京師,上嘉美之,特授通漢檢校太傅、本州防禦使,還賜疆土,署其子光澤等三班職名。通漢再表欲留京師,不允,乃為光澤等求內地監臨,及言歲賜衣,願使者至本任,
 並從之。既辭,又賜以襲衣、金帶。三年,通漢卒,以其子光憲知州事。其後,光澤不為親族所容,上表納土,上察其意,不許。四年,知古州向光普遣使鼎州營僧齋,以祝聖壽。



 初,北江蠻酋最大者曰彭氏,世有溪州,州有三,曰上、中、下溪,又有龍賜、天賜、忠順、保靜、感化、永順州六,懿、安、遠、新、給、富、來、寧、南、順、高州十一,總二十州,皆置刺史。而以下溪州刺史兼都誓主,十九州皆隸焉,謂之誓下。州將承襲,都誓主率群酋合議,子孫若弟、侄、親黨之當
 立者,具州名移辰州為保證,申鈐轄司以聞,乃賜敕告、印符,受命者隔江北望拜謝。州有押案副使及校吏,聽自補置。



 彭氏自允殊、文勇、儒猛相繼為下溪州刺史,至仕漢為殿直,留西京,後輒遁歸。天聖初,以狀白辰州,自言父老兄亡,潛歸本道,願放還家屬。詔徙其家京師,舍以官第。未幾,儒猛言仕漢逃歸,誘群蠻為亂,遣別子仕端等殺之。朝廷嘉其忠,降詔獎諭。時儒猛為檢校尚書右僕射,特遷左僕射。又以仕端為檢校國子祭酒,知溶州,
 加賜鹽三百斤、彩三十匹。彭氏有文綰者,知中彭州,即忠順州也。三年,儒猛攻殺文綰,其子儒索率其黨九十二人來歸,補儒索復州都知兵馬使,餘官為稟給。五年,儒猛死,仕端以名馬來獻,詔還其馬,命知下溪州,賜以袍帶。七年,遂以其弟仕羲貢方物。明道初,仕端死,復命仕羲為刺史,累遷檢校尚書右僕射。自允殊至仕羲五世矣。



 仕羲有子師寶,景祐中知忠順州。慶曆四年,以罪絕其奉貢。蓋自咸平以來,始聽二十州納貢,歲有常賜,
 蠻人以為利,有罪則絕之。其後,師寶數自訴,請知上溪州。皇祐二年,始從其請,朝貢如故。既而師寶妻為仕羲取去,師寶忿恚。至和二年,與其子知龍賜州師黨舉族趨辰州,告其父之惡;且言仕羲嘗殺誓下十三州將,奪其符印,並有其地,貢奉賜予悉專之,自號如意大王,補置官屬,將起為亂。於是知辰州宋守信與通判賈師熊、轉運使李肅之合議,率兵數千,深入討伐,以師寶為鄉導。兵至而仕羲遁入他峒,不可得,俘其孥及銅柱,而官
 軍戰死者十六七,守信等皆坐貶。



 自是,蠻獠數入寇鈔,邊吏不能制。朝廷姑欲無事,間遣吏諭旨,許以改過自歸,裁損五七州貢奉歲賜。初輒不聽,後遣三司副使李參、文思副使竇舜卿、侍御史朱處約、轉運使王綽經制,大出兵臨之,且馳檄招諭。而仕羲乃陳本無反狀,其僭稱號、補官屬,特遠人不知中國禮義而然。守信等輕信師寶之譖,擅伐無辜,願以二十州舊地復貢奉內屬。朝廷又遣殿中丞雷簡夫往視之。嘉祐二年,仕羲乃歸所
 掠兵丁五十一人、械甲千八百九事,率蠻眾七百飲血就降,辰州亦還其孥及銅柱。時師寶已死,遣師黨歸知龍賜州,戒勿殺。



 自是,仕羲歲奉職貢。然黠驁,數盜邊,即辰州界白馬崖下喏溪聚眾據守,朝廷數招諭,令歸侵地,不聽。熙寧三年,為其子師彩所弑。師彩專為暴虐,其兄師晏攻殺之,並誅其黨,納誓表於朝,並上仕羲平生鞍馬、器服,仍歸喏溪地,乃命師晏襲州事。五年,復以馬皮、白峒地來獻。詔進為下溪州刺史,賜母妻封邑。章惇
 經制南、北江,湖北提點刑獄李平招納師晏,誓下州峒蠻張景謂、彭德儒、向永勝、覃文猛、覃彥霸各以其地歸版籍,師晏遂降。詔修築下溪州城,並置砦於茶灘南岸,賜新城名會溪,新砦名黔安,戍以兵,隸辰州,出租賦如漢民。遣師晏詣闕,授禮賓副使、京東州都監,官其下六十有四人。



 元豐八年,湖北轉運司言辰州江外生蠻覃仕穩等願內附,詔不許招納。其後彭仕誠者復為都誓主。元祐三年,羅家蠻寇鈔,詔召仕誠及都頭覃文懿等
 至辰州約敕之。四年,知誓下保靜州彭儒武、知永順州彭儒同、知謂州彭思聰、知龍賜州彭允宗、知藍州彭土明、知吉州彭儒崇,各同其州押案副使進奉興龍節及冬至、正旦溪布有差。



 初,熙寧中,天子方用兵以威四夷,湖北提點刑獄趙鼎言峽州峒首刻剝亡度,蠻眾願內屬。辰州布衣張翹亦上書言南、北江利害,遂以章惇察訪湖北,經制蠻事。而南江之舒氏、北江之彭氏、梅山之蘇氏、誠州之楊氏相繼納土,創立城砦,使之比內地為
 王民。北江彭氏已見前。南江諸蠻自辰州達於長沙、邵陽,各有溪峒:曰敘、曰峽、曰中勝、曰元,則舒氏居之;曰獎、曰錦、曰懿、曰晃,則田氏居之;曰富、曰鶴、曰保順、曰天賜、曰古,則向氏居之。舒氏則德郛、德言、君疆、光銀,田氏則處達、漢瓊、漢希、漢能、漢權、保金,向氏則通漢、光普、行猛、永豐、永晤,皆受朝命。自治平末,光銀入貢。故事,南江諸蠻亦隸辰州,貢進則給以驛券,光銀援以為請,詔以券九道給之。其後有峽州舒光秀者,以刻剝其眾不附。



 張
 翹言:「南江諸蠻雖有十六州之地,惟富、峽、敘僅有千戶,餘不滿百,土廣無兵,加以薦饑。近向永晤與繡、鶴、敘諸州蠻自相仇殺,眾苦之,咸思歸化。願先招富、峽二州,俾納土,則餘州自歸,並及彭師晏之孱弱,皆可郡縣。」詔下知辰州劉策商度,策請如翹言。熙寧五年,乃遣章惇察訪。未幾,策卒,乃以東作坊使石鑒為湖北鈐轄兼知辰州,且助惇經制。明年,富州向永晤獻先朝所賜劍及印來歸順,繼而光銀、光秀等亦降。獨田氏有元猛者,頗桀驁
 難製,異時數侵奪舒、向二族地。惇遣左侍禁李資將輕兵往招諭。資,辰州流人,曩與張翹同獻策者也。褊宕無謀,褻慢夷獠,遂為懿、洽州蠻所殺。惇進兵破懿州,南江州峒悉平,遂置沅州,以懿州新城為治所,尋又置誠州。



 元祐初,傅堯俞、王岩叟言:「沅、誠州創建以來,設官屯兵,布列砦縣,募役人,調戍兵,費钜萬,公私騷然,荊湖兩路為之空竭。又自廣西融州創開道路達誠州,增置潯江等堡,其地無所有,湖、廣移賦以給一方,民不安業,願斟
 酌廢置。」朝廷以沅州建置至是十五年,蠻情安習已久,但廢誠州為渠陽軍,而沅州至今為郡。元祐初,諸蠻復叛,朝廷方務休息,痛懲邀功生事。廣西張整、融州溫嵩坐擅殺蠻人,皆置之罪。詔諭湖南、北及廣西路曰:「國家疆理四海,務在柔遠。頃湖、廣諸蠻近漢者無所統壹,因其請吏,量置城邑以撫治之。邊臣邀功獻議,創通融州道路,侵逼峒穴,致生疑懼。朝廷知其無用,旋即廢罷;邊吏失於撫遏,遂爾扇搖。其叛酋楊晟台等並免追討,諸
 路所開道路、創置堡砦並廢。 」自後,五溪郡縣棄而不問。



 崇寧以來,開邊拓土之議復熾,於是安化上三州及思廣洞蒙光明、樂安峒程大法、都丹團黃光明、靖州西道楊再立、辰州覃都管罵等各願納土輸貢賦。又令廣西招納左、右江四百五十餘峒。宣和中,議者以為「招致熟蕃,接武請吏,竭金帛、繒絮以啖其欲,捐高爵、厚奉以侈其心。開闢荒蕪,草創城邑,張皇事勢,僥幸賞恩。入版圖者存虛名,充府庫者亡實利。不毛之地,既不可耕;狼子
 野心,頑冥莫革。建築之後,西南夷獠交寇,而溪峒子蠻亦復跳梁。士卒死於干戈,官吏沒於王事,肝腦塗地,往往有之。以此知納土之義,非徒無益,而又害之所由生也。莫若俾帥臣、監司條具建築以來財用出入之數,商較利病,可省者省,可並者並,減戍兵漕運,而夷狄可撫,邊鄙可亡患矣!」乃詔悉廢所置初郡。其餘諸蠻,自乾興以來,或叛或服,其類不一,各以歲月次之。



 乾興初,順州蠻田彥晏率其黨田承恩寇施州暗利砦,縱火而去,夔
 州發兵擊之,俘獲甚眾。彥晏在真宗朝為歸德將軍、檢校太子賓客、知順州;承恩者,知保順州田彥曉子也。明年,彥晏款邊上誓狀,願還所掠金帛、器械,且輸粟二千石自贖。詔拒其粟,舍其所負金帛,第令歸掠去戶口。仍加彥晏寧遠將軍、檢校工部尚書,承恩檢校國子祭酒兼監察御史,皆知州如故。後又有田忠顯者,與其黨百九人入貢。



 天聖二年,知古州向光普自言,嘗創佛寺,請名報國,歲度僧一人,許之。四年,歸順等州蠻田思欽等
 以方物來獻,時來者三百一人,而夔州路轉運司不先以聞,詔劾之。既而又詔安、遠、天賜、保順、南、順等州蠻貢京師,道里遼遠而離寒暑之苦,其聽以貢物留施州,所賜就給之。願入貢者十人,聽三二人至闕下,首領聽三年一至。七年,黔州蠻、舒延蠻、繡州蠻向光緒皆來貢。九年,施州屬蠻覃彥綰等寇永寧砦。景祐中,澧州屬蠻五百餘人入寇。時州將崔承祐畏避不以聞,為荊湖鈐轄司所奏,詔劾罷之。寶元二年,辰州狤獠三千餘人款附,
 以州將張昭懿招輯有功,進一官。



 慶曆三年,桂陽監蠻徭內寇,詔發兵捕擊之。蠻徭者,居山谷間,其山自衡州常寧縣屬於桂陽、郴連賀韶四州,環紆千餘里,蠻居其中,不事賦役,謂之徭人。初,有吉州巫黃捉鬼與其兄弟數人皆習蠻法,往來常寧,出入溪峒,誘蠻眾數百人盜販鹽,殺官軍,逃匿峒中,既招出而殺之,又徙山下民他處。至是,其黨遂合五千人,出桂陽藍山縣華陰峒,害巡檢李延祚、潭州都監張克明。事聞,擢楊畋提點刑獄,督
 攻討事,久之不克。遂詔湖南轉運使郭輔之等招撫之,始於湖南置安撫司。蠻所至殺掠居民,縱火劫財物,被害者甚眾。詔被害者並入山捕蠻,土兵蠲復有差。初,發兵捕蠻,至或誤殺良民,仁宗命訪之,口給絹五匹,仍拊其家。時蠻勢方熾,又遣殿中侍御史王絲、三司度支副使徐的經制。降敕書委知潭州劉沆招諭,能自歸者第錄以官。沆大發兵臨之,以敕書從事,降二千餘人,使散居所部,錄其首領鄧文誌、黃文晟、黃士元皆為三班奉
 職。又以內殿承製亓贇、崇班胡元嘗在石硋峒捕殺有勞,進贇莊宅副使,元禮賓副使,時四年冬也。



 五年二月,餘黨唐和等復內寇,乃詔湖南安撫、轉運、提點刑獄便宜從事。又特賜官兵土丁錢有差。於是沆檄楊畋等八路入討,覆蕩桃油平、能家源等,皆其巢穴,捕斬首級甚眾。詔官兵有功者九百餘人第遷一資,錄其應募討擊者道州進士十四人,並官之。然唐和等猶未平。又詔:「如聞賊黨欲降,其罷出兵,逃匿者諭使歸復,州縣拊存之。」
 是冬,蠻復入寇,與胡元及右侍禁郭正、趙鼎、殿侍王孝先戰於華陰峒隘口,元等死之,劉沆、楊畋皆坐黜。以劉夔代沆為安撫使,夔言:「唐和等既敗官軍,殺將吏,聚眾益自疑,恐浸為邊患,願以詔書招安,就補溪峒首領。」詔可。



 是時,湖湘騷動,兵不得息。六年夏,仁宗顧謂輔臣曰:「官軍久戍南方,夏秋之交,瘴癘為虐,其令太醫定方和藥,遣使給之。」自是繼賜緡錢。未幾,夔言敗唐和於銀江源。轉運使周沆亦言指揮辛景賢招降賊黨五十六戶
 二百五十九人,錄其首領,戒所部拊存之。先是,命三司戶部判官崔嶧為體量安撫,往議討除、招安二策,既而知桂陽監宋守信奏:「唐和嘯聚千餘眾為盜,五六年卒未能克者,朝廷不許窮討故也。今衡州監酒黃士元頗習溪峒事,願得敢戰士二千、引路土丁二百,優給金帛,使之逐捕,必得然後已,並敕亓贇等合力以進。彼既勢窮,必將款附。」詔用其策,於是大發兵討之。其眾果懼,遁入郴州黃莽山,由趙峒轉寇英、韶州,依山自保。是冬,帝
 閔士卒暴露,復諭執政密戒主帥安恤。



 七年,唐和遣其子執要領詣官,自言願貸糧米,居所保峒中。時楊畋復為湖南鈐轄,詔趨連、韶州山下,與廣南東、西轉運使共告諭之,使以兵械上官,質其親屬。詔補唐和、盤知諒、房承映、承泰、文運等五人為峒主,授銀青光祿大夫、檢校國子祭酒兼監察御史、武騎尉。知諒等,蓋唐和黨也。至冬,其眾悉降。



 皇祐五年,邵州蠻舒光銀因湖南安撫司官自陳捍禦之勞,願於峒中置中勝州,詔可。嘉祐二年,羅
 城峒蠻寇澧州,發兵擊走之。三年,以施州蠻向永勝所領州為安定州。五年,以邵州蠻楊光倩知徽州。光倩,通漢之子也。通漢,慶曆初嘗入貢,既死,光倩襲之。舊制,溪峒知州卒,承襲者許進奉行州事,撫遏蠻人,及五年,安撫司為奏給敕告。至是,光倩行州事七年,無他過,故命之。



\end{pinyinscope}