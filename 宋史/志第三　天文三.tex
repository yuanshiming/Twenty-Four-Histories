\article{志第三 天文三}

\begin{pinyinscope}

 二十八舍上二十八舍



 東方



 角宿二星,為天關,其間天門也,其內天庭也。故黃道經其中,七曜之所行也。左角為天田,為理,主刑。其南為太陽道。右角為將,主兵。其北為太陰道。蓋天
 之三門,猶房之四表。星明大,吉,王道太平,賢者在朝;動搖、移徙,王者行;左角赤明,獄平;暗而微小,王道失。陶隱居曰:「左角天津,右角天門,中為天關。」日食角宿,王者惡之;暈於角內,有陰謀,陰國用兵得地,又主大赦。月犯角,大臣憂獄事,法官憂黜。又占憂在宮中。月暈,其分兵起;右角,右將災;左亦然。或曰主水;色黃,有大赦。月暈三重,入天門及兩角,兵起,將失利。歲星犯,為饑。熒惑犯之,國衰,兵敗;犯左角,有赦;右角,兵起;守之,讒臣進,政事急;居陽,有喜。填星犯角為喪,一曰兵起。太白犯角,群臣有異謀。辰
 星犯,為小兵;守之,大水。客星犯,兵起,五穀傷;守左角,色赤,為旱;守右角,大水。彗星犯之,色白,為兵;赤,所指破軍;出角,天下兵亂。星孛於角,白,為兵;赤,軍敗;入天市,兵、喪。流星犯之,外國使來;入犯左角,兵起。雲氣黃白入右角,得地;赤入左,有兵;入右,戰勝;黑白氣入於右,兵將敗。



 按漢永元銅儀,以角為十三度;而唐開元游儀,角二星十二度。舊經去極
 九
 十一度,今測九十三度半。距星正當赤道,其黃道在赤道南,不經角中;今測角在赤道南二度半,黃道復經角中,即與天象合。景祐測驗,角二星十二度,距南星去極九十七度,在赤道外六度,與《乾象新書》合,今從《新書》為正。



 南門二星,在庫樓南,天之外門也,主守兵禁。星明,
 則遠方來貢;暗,則夷叛;中有小星,兵動。客、
 彗守之,兵起。



 庫樓十星,六大
 星庫也,南四星樓也,在角宿南。一曰天庫,兵車之府也。旁十五星,三三而聚者柱也,中央四小星衡也。芒角,兵起;星亡,臣下逆;動,則將行;實,為吉;虛,乃兇。歲星犯之,主兵。熒惑
 犯之,為兵、旱。月入庫樓,為兵。彗、孛入,兵、饑。客星入,夷兵起。流星入,兵盡出。赤雲氣入,內外不安。天庫生角,有兵。



 平星二星,在庫樓北,角南,主平天下法獄,廷尉之象。正,則獄訟平;月暈,獄官憂。熒惑犯之,兵起,有赦。彗星犯,政不行,執法者黜。



 平道二星,在角宿間,主平道之官。武密曰:「天子八達之衢,主轍軾。」明正,吉;動搖,法駕有虞。歲星守之,天下治。熒惑、太白守,為亂。客星守,車駕出行。流星守,去賢用奸。



 天田二星,在角北,主畿內封域。武密曰:「天子籍田也。」歲星守之,穀稔。熒惑
 守之,為旱。太白守,穀傷。辰星守,為水災。客星守,旱、蝗。



 天門二星,在平星北。武密云:「在左角南,朝聘待客之所。」星明,萬方歸化;暗,則外兵至。月暈其外,兵起。熒惑入,關梁不通;守之,失禮。太白守,有伏兵。客
 星
 犯,有謀上者。



 進
 賢一星,在平
 道西,主卿相舉逸材。明,則賢人用;暗,則邪臣進。太陰、歲星犯之,大臣死。熒惑犯,為喪,賢人隱。太白犯之,賢者退。歲星、太白、填星、辰星合守之,其占為天子求賢。黃白紫氣貫之,草澤賢人出。



 周鼎三星,在角宿上,主流亡。星明,國安;不見,則運不昌;動搖,國將移。《乾象新書》引郟鄏定鼎事,以周衰秦無道,鼎淪泗水,其精上為星。李太異曰:「商巫咸《星圖》已有周
 鼎,蓋在秦前數百年矣。」



 按《步天歌》,庫樓十星,柱十五星,衡四星,平星、平道、天田、天門各二星,進賢一星,周鼎三星,俱屬角宿。而《晉志》以左角為天田,別不載天田二星,《隋志》有之。平道、進賢、周鼎,《晉志》皆屬太微垣,庫樓並衡星、柱星、南門、天門、平星皆在二十八宿之外。唐武密及景祐書乃與《步天歌》合。



 亢宿四星,為天子內朝,總攝天下奏事。聽訟、理獄、錄功。
 一曰疏廟,主疾疫。星明大,輔忠民安;動,則多疾。為天子正坐,為天符。秋分不見,則穀傷糴貴。太陽犯之,諸侯謀國,君憂。日暈,其分大臣兇,多雨,民饑、疫。月犯之,君憂或大臣當之;左為水,右為兵。月暈,其分先起兵者勝;在冬,大人憂。歲星犯之,有赦,穀有成;守之,有兵,人多病;留三十日以上,有赦;又曰:「犯則逆臣為亂。」熒惑犯,居陽,為喜;陰,為憂;有芒角,大人惡之;守之久,民憂,多雨水,又為兵。填星犯,穀傷,民亡;逆行,女專政,逆臣為謀;守之,有兵。太
 白犯之,國亡,民災;逆行,為兵亂;有芒角,貴臣戮;守之,有水旱災,或為喪。辰星犯之,為水,又為大兵;守之,米貴,民疾,歲旱,盜起,民相惡。客星犯,國不安;色赤為兵、旱,黃為土功;青黑,使者憂;守之穀傷。一云有赦令;黑,民流。彗犯,國災;出,則有水、兵、疫、臣叛;白,為喪。孛星犯,國危,為水,為兵;入,則民流;出,則其國饑。流星入,外國使來,穀熟;出,為天子遣使,赦令出。李淳風曰:「流星入亢,幸臣死。」雲氣犯之,色蒼,民疫;白,為土功;黑,水;赤,兵。一云:白,民虐疾;黃,土
 功。



 右亢宿四星,漢永元銅儀十度,唐開元游儀九度。舊去極八十九度,今九十一度半。景祐測驗,亢九度,距南第二星去極九十五度。



 大角一星,在攝提間,天王坐也。又為天棟,正經紀也。光明潤澤,為吉;青,為憂;赤,為兵;白,為喪;黑,為疾;色黃而靜,民安;動,則人主好游。月犯之,大臣憂,王者惡之。月暈,其分人主有服。五星犯之,臣謀主,有兵。太白守之,為兵。彗
 星出,其分主更改,或為兵。天子失仁則守之。孛星犯,為兵;守之,主憂。客星犯、守,臣謀上;出,則人主受制。流星入,王者惡之;犯之,邊兵起。雲氣青,主憂;白,為喪;黃氣出,有喜。



 折威七星,在亢南,主斬殺,斷軍獄。月犯之,天子憂。五星犯,將軍叛。彗、孛犯,邊將死。雲氣犯,蒼白,兵亂;赤,臣叛主;黃白,為和親;出,則有赦;黑氣入,人主惡之。



 攝提六星,左右各三,直斗杓南,主建時節,伺禨祥。其星
 為盾,以夾擁帝坐,主九卿。星明大,三公恣,主弱;色溫不明,天下安;近大角,近戚有謀。太陰入,主受制。月食,其分主惡之。熒惑、太白守,兵起,天下更主。彗、孛入,主自將兵;出,主受制。流星入,有兵;出,有使者出;犯之,公卿不安。雲氣入,赤,為兵,九卿憂;色黃,喜;黑,大臣戮。



 陽門二星,在庫樓東北,主守隘塞,御外寇。五星入,五兵藏。彗星守之,外夷犯塞、兵起。赤雲氣入,主用兵。



 頓頑二星,在折威東南,主考囚情狀,察詐偽也。星明,無
 咎;暗,則刑濫。彗星犯之,貴人下獄。



 按《步天歌》,大角一星,折威七星,左、右攝提總六星,頓頑、陽門各二星,俱屬角宿。而《晉志》以大角、攝提屬太微垣,折威、頓頑在二十八宿之外。陽門則見於《隋志》,而《晉史》不載。武密書以攝提、折威、陽門皆屬角、亢。《乾象新書》以右攝提屬角,左攝提屬亢,餘與武密書同。《景祐》測驗,乃以大角、攝提、頓頑、陽門皆屬於亢,其說不同。



 氐宿四星,為天子舍室,後妃之府,休解之房。前二星,適也;後二星,妾也。又為天根,主疫。後二星大,則臣奉度,主安;小,則臣失勢;動,則徭役起。日食,其分,卿相有讒諛,一曰王者後妃惡之,大臣憂。日暈,女主恣,一曰國有憂,日下興師。月食其宿,大臣兇,後妃惡之,一曰糴貴。月暈,大將兇,人疫;在冬,為水,主危,以赦解之。月犯,左右郎將有誅,一曰有兵、盜。犯右星,主水;掩之,有陰謀,將軍當之。歲星犯,有赦,或立後;守之,地動,年豐;逆行,為兵。熒惑犯之,
 臣僭上,一云將軍憂;守,有赦。填星犯,左右郎將有誅;守之,有赦;色黃,後喜,或冊太子;留舍,天下有兵;齊明,赦。太白犯之,郎將誅;入,其分疾疫;或云犯之,拜將;乘右星,水災。辰星犯,貴臣暴憂;守之,為水,為旱,為兵;入守,貴人有獄;乘左星,天子自將。客星犯,牛馬貴;色黃白,為喜,有赦,或曰邊兵起,後宮亂;五十日不去,有刺客。彗星犯,有大赦,糴貴;滅之,大疫;入,有小兵,一云主不安。孛星犯,糴貴;出,則有赦;入,為小兵;或云犯之,臣乾主。流星犯,秘閣官
 有事;在冬夏,為水、旱;《乙巳占》,後宮有喜;色赤黑,後宮不安。雲氣入,黃為土功;黑主水;赤為兵;蒼白為疾疫;白,後宮憂。



 按漢永元銅儀、唐開元游儀,氐宿十六度,去極九十四度。景祐測驗與《乾象新書》皆九十八度。



 天乳一星,在氐東北,當赤道中。明,則甘露降。彗、客入,天雨。



 將軍一星,騎將也,在騎官東南,總領車騎軍將、部陣行
 列。色動搖,兵外行。太白、熒惑、客星犯之,大兵出,天下亂。



 招搖一星,在梗河北,主北兵。芒角、變動,則兵大行;明,則兵起;若與棟星、梗河、北斗相直,則北方當來受命中國。又占:動,則近臣恣;離次,則庫兵發;色青,為憂;白,為君怒;赤,為兵;黑,為軍破;黃,則天下安。彗星犯,北邊兵動;出,其分夷兵大起。孛犯,蠻夷亂。客星出,蠻夷來貢,一云北地有兵、喪。流星出,有兵。雲氣犯,色黃白,相死;赤,為內兵亂;色黃,兵罷;白,大人憂。



 帝席三星,在大角北,主宴獻酬酢。星明,王公災;暗,天下安;星亡,大人失位;動搖,主危。彗犯,主憂,有亂兵。客星犯,主危。



 亢池六星,在亢宿北。亢,舟也;池,水也。主渡水,往來送迎。微細,兇;散,則天下不通;移徙不居其度中,則宗廟有怪。五星犯之,川溢。客星犯,水,蟲多死。武密云:「主斷軍獄,掌棄市殺戮。」與舊史異說。



 騎官二十七星,在氐南,天子虎賁也,主宿衛。星眾,天下
 安;稀,則騎士叛;不見,兵起。五星犯,為兵。客星守之,將出有憂,士捽發。流星入,兵起,色蒼白,將死。



 梗河三星,在帝席北,天矛也。一曰天鋒,主北邊兵,又主喪,故其變動應以兵、喪。星亡,國有兵謀。彗星犯之,北兵敗。客星入,兵出,陰陽不和。一云北兵侵中國。流星出,為兵。赤雲氣犯,兵敗;蒼白,將死。



 車騎三星,在騎官南,總車騎將,主部陣行列。變色動搖,則兵行。太白、熒惑、客星犯之,大兵出,天下亂。



 陣車三星,在氐南,一云在騎官東北,革車也。太白、熒惑守之,主車騎滿野,內兵無禁。



 天輻二星,在房西斜列,主乘輿,若《周官》巾車官也。近尾,天下有福。五星、客、彗犯之,則輦轂有變。一作天福。



 按《步天歌》,已上諸星俱屬氐宿。《乾象新書》以帝席屬角,亢池屬亢;武密與《步在歌》合,皆屬氐,而以梗河屬亢。《占天錄》又以陣車屬於亢,《乾象新書》屬氐,餘皆與《步天歌》合。



 房宿四星,為明堂,天子布政之官也,亦四輔也。下第一星,上將也;次,次將也;次,次相也;上星,上相也。南二星君位,北二星夫人位。又為四表,中為天衢、為天關,黃道之所經也。南間曰陽環,其南曰太陽;北間曰陰環,其北曰太陰。七曜由乎天衢,則天下和平,由陽道,則旱、喪;由陰道,則水、兵。亦曰天駟,為天馬,主車駕。南星曰左驂,次左服,次右服,次右驂。亦曰天廄。又主開閉,為畜藏之所由。星明,則王者明;驂星大,則兵起;離,則民流;左驂、服亡,則
 東南方不可舉兵;右亡,則西北不可舉兵。日食,其分為兵,大臣專權。日暈,亦為兵,君臣失政,女主憂。月食其宿,大臣憂,又為王者昏,大臣專政。月暈,為兵;三宿,主赦,及五舍不出百日赦。太陰犯陽道,為旱;陰道,為雨;中道,歲稔。又占上將誅。當天門、天駟,穀熟。歲星犯之,更政令,又為兵,為饑,民流;守之,大赦,天下和平,一云良馬出。熒惑犯,馬貴,人主憂;色青,為喪;赤,為兵;黑,將相災;白芒,火災;守之,有赦令;十日勾巳者,臣叛。填星犯之,女主憂,勾巳,
 相有誅;守之,土功興,一曰旱、兵,一曰有赦令。太白犯,四邊合從;守之,為土功;出入,霜雨不時。辰星犯,有殃;守之,水災。一云北兵起,將軍為亂。客星犯,歷陽道,為旱;陰道,為水,國空,民饑;色白,有攻戰;入,為糴貴。彗星犯,國危,人亂,其分惡之。孛星犯,有兵,民饑,國災。流星犯之,在春夏,為土功;秋冬,相憂;入,有喪。《乙巳占》:出,其分天子恤民,下德令。雲氣入,赤黃,吉;如人形,後有子;色赤,宮亂;蒼白氣出,將相憂。



 按漢永元銅儀、唐開元游儀,房宿五度。舊去極百八度,今百十度半。景祐測驗,房距南第二星去極百十五度,在赤道外二十三度。《乾象新書》在赤道外二十四度。



 鍵閉一星,在房東北,主關鑰。明,吉;暗,則宮門不禁。月犯之,大臣憂,火災。歲星守之,王不宜出。填星占同。太白犯,將相憂。熒惑犯,主憂。彗星、客星守之,道路阻,兵起,一云兵滿野。



 鉤鈐二星,在房北,房之鈐鍵,天之管鑰。王者至孝則明。又曰明而近房,天下同心。房、鉤鈐間有星及疏折,則地動,河清。月犯之,大人憂,車駕行。月食,其分將軍死。歲星守之,為饑;去其宿三寸,王失政,近臣起,亂。熒惑守之,有德令。太白守,喉舌憂。填星守,王失土。彗星犯,宮庭失業。客星、流星犯,王有奔馬之敗。



 東咸、西咸各四星,東咸在心北,西咸在房西北,日、月、五星之道也。為房之戶,以防淫泆也。明,則信吉。東咸近鉤
 鈐,有讒臣入。西咸近上及動,有知星者入。月、五星犯之,有陰謀,又為女主失禮,民饑。熒惑犯之,臣謀上。與太白同犯,兵起。歲星、填星犯之,有陰謀。流星犯,後妃恣,王有憂。客星犯,主失禮,後妃恣。



 罰三星,在東、西咸正南,主受金罰贖。曲而斜列,則刑罰不中。彗星、客星犯之,國無政令,憂多,枉法。



 日一星,在房宿南,太陽之精,主昭明令德。明大,則君有德令。月犯之,下謀上。歲星守,王得忠臣,陰陽和,四夷賓,
 五穀豐。太白、熒惑犯之,主有憂。客星、彗星犯之,主失位。



 從官二星,在房宿西南,主疾病巫醫。明大,則巫者擅權。彗、孛犯之,巫臣作亂。雲氣犯,黑,為巫臣戮;黃,則受爵。



 按《步天歌》,以上諸星俱屬在房。日一星,《晉》、《隋志》皆不載,以他書考之,雖在房宿南,實入氐十二度半。武密書及《乾象新書》惟以東咸屬心,西咸屬房,與《步天歌》不同,餘皆吻合。



 心宿三星,天王正位也。中星曰明堂,天子位,為大辰,主
 天下之賞罰;前星為太子;後星為庶子。星直,則王失勢。明大,天下同心;天下變動,心星見祥;搖動,則兵離民流。日食,其分刑罰不中,將相疑,民饑,兵、喪。日暈,王者憂之。月食其宿,王者惡之,三公憂,下有喪。月暈,為旱,穀貴,蟲生,將兇。與五星合,大兇。太陰犯之,大臣憂;犯中央及前後星,主惡之;出心大星北,國旱;出南,君憂,兵起。歲星犯之,有慶賀事,穀豐,華夷奉化;色不明,有喪,旱。熒惑犯之,大臣憂;貫心,為饑;與太白俱守,為喪。又曰:熒惑居其陽,
 為喜;陰,為憂。又曰:守之,主易政;犯,為民流,大臣惡之;守星南,為水;北,為旱;逆行,大臣亂。填星犯之,大臣喜,穀豐;守之,有土功;留舍三十日有赦;居久,人主賢;中犯明堂,火災;逆行,女主干政。太白犯,糴貴,將軍憂,有水災,不出一年有大兵;舍之,色不明,為喪;逆行環繞,大人惡之。辰星犯明堂,則大臣當之,在陽為燕,在陰為塞北,不則地動、大雨;守之,為水,為盜。客星犯之,為旱;守之,為火災;舍之,則糴貴,民饑。彗星犯之,大臣相疑;守之而出,為蝗、
 饑,又曰為兵。星孛,其分有兵、喪,民流。流星犯,臣叛;入之,外國使來;色青,為兵,為憂;黃,有土功;黑,為兇。雲氣人,色黃,子孫喜;白,亂臣在側;黑,太子有罪。



 按漢永元銅儀、唐開元游儀,心三星皆五度,去極百八度。景祐測驗,心三星五度,距西第一星去極百十四度。



 積卒十二星,在房西南,五營軍士之象,主衛士掃除不祥。星小,為吉;明,則有兵;一星亡,兵少出;二星亡,兵半出;
 三星亡,兵盡出。五星守之,兵起;不則近臣誅。彗星、客星守之,禁兵大出,天子自將。雲氣犯之,青赤,為大臣持政,欲論兵事。



 按《步天歌》,積卒十二星屬心,《晉志》在二十八宿之外,唐武密書與《步天歌》合。《乾象新書》乃以積卒屬房宿為不同,今兩存其說。



 尾宿九星,為天子後宮,亦主后妃之位。上第一星,後也;次三星,夫人;次星,嬪妾也。亦為九子。均明,大小相承,則
 後宮有序,子孫蕃昌。明,則後有喜,穀熟;不明,則後有憂,穀荒。日食,其分將有疾,在燕風沙,兵、喪,後宮有憂,人君戒出。日暈,女主喪,將相憂。月食,其分貴臣犯刑,後宮有憂。月暈,有疫,大赦,將相憂,其分有水災,後妃憂。太陰犯之,臣不和,將有憂。歲星犯,穀貴;入之,妾為嫡,臣專政;守之,旱,火災。熒惑犯之,有兵;留二十日,水災;留三月,客兵聚;入之,人相食,又云宮內亂。填星犯之,色黃,後妃喜;入,為兵、饑、盜賊;逆行,妾為女主;守之而有芒角,更姓易政。
 太白犯、入,大臣起兵;久留,為水災;出、入、舍、守,糴貴,兵起,後宮憂;失行,軍破城亡。辰星犯守,為水災,民疾,後宮有罪者,兵起;入,則萬物不成,民疫。客星犯、入,宮人惡之;守之,賤女暴貴;出,則為風、為水,後宮惡之,兵罷,民饑多死。彗星犯,後惑主,宮人出,兵起,宮門多土功;出入,貴臣誅,有水災。孛犯,多土功,大臣誅;守之,宮人出;出,為大水,民饑。流星入、犯,色青,舊臣歸;在春夏,後宮有口舌;秋冬,賢良用事;出,則後宮喜,有子孫;色白,後宮妾死;出入,風雨
 時,穀熟,入,後族進祿;青黑,則後妃喪。雲氣入,色青,外國來降;出,則臣有亂。赤氣入,有使來言兵。黑氣入,有諸侯客來。



 按漢永元銅儀,尾宿十八度,唐開元游儀同。舊去極百二十度,一云百四十度;今百二十四度。景祐測驗,亦十八度,距西行從西第二星去極百二十八度,在赤道外二十二度。《乾象新書》二十七度。



 神宮一星,在尾宿第三星旁,解衣之內室也。



 天江四星,在尾宿北,主太陰。明動,為水,兵起;星不具,則津梁不通;參差,馬貴。月犯,為兵,為臣強,河津不通。熒惑犯,大旱;守之,有立主。太白犯,暴水。彗星犯,為大兵。客星入,河津不通。流星犯,為水,為饑。赤雲氣犯,車騎出;青,為多水;黃白,天子用事,兵起;入,則兵罷。



 傅說一星,在尾後河中,主章祝官也,一曰後宮女巫也,司天王之內祭祀,以祈子孫。明大,則吉,王者多子孫,輔佐出;不明,則天下多禱祠;亡,則社稷無主;入尾下,多祝
 詛。《左氏傳》「天策焞輸」,即此星也。彗星、客星守之,天子不享宗廟。赤雲氣入,巫祝官有誅者。



 魚一星,在尾後河中,主陰事,知雲雨之期。明大,則河海水出;不明,則陰陽和,多魚;亡,則魚少;動搖,則大水暴出;出,則河大魚多死。月暈或犯之,則旱,魚死。熒惑犯其陽,為旱;陰,為水。填星守之,為旱。赤雲氣犯出,兵起,將憂;入,兵罷;黃白氣出,兵起。



 龜五星,在尾南,主卜,以占吉兇。星明,君臣和;不明,則上
 下乖。熒惑犯,為旱;守,為火。客星入,為水,憂。流星出,色赤黃,為兵;青黑,為水,各以其國言之。赤雲氣出,卜祝官憂。



 按神宮、傅說、魚各一星,天江四星,龜五星,《步天歌》與他書皆屬尾。而《晉志》列天江於天市垣,以傅說、魚、龜在二十八宿之外,其說不同。



 箕宿四星,為後宮妃后之府,亦曰天津,一曰天雞。主八風,又主口舌,主蠻夷。星明大,穀熟;不正,為兵;離徙,天下不安;中星眾亦然,糴貴。凡日月宿在箕、壁、翼、軫者,皆為
 風起;舌動,三日有大風。日犯或食其宿,將疾,佞臣害忠良,皇后憂,大風沙。日暈,國有妖言。月食,為風,為水、旱,為饑,後惡之。月暈,為風,穀貴,大將易,又王者納後。月犯,多風,糴貴,為旱,女主憂,君將死,後宮干政。歲星入,宮內口舌,歲熟,在箕南,為旱;在北,為有年;守之,多惡風,穀貴,民饑死。熒惑犯,地動;入,為旱;出,則有赦;久守,為水;逆行,諸侯相謀,人主惡之。填星犯,女主憂;久留,有赦;守之,後喜,有土功;色黃光潤,則太后喜;又占:守,有水;守九十日,人
 流,兵起,蝗。太白犯,女主喜;入,則有赦;出,為土功,糴貴;守之,為旱,為風,民疾;出入留箕,五穀不登,多蝗。辰星犯,有赦;守,則為旱;動搖、色青,臣自戮,又占:為水溢、旱、火災、穀不成。客星入犯,有土功,宮女不安,民流;守之,為饑;色赤,為兵;守其北,小熟;東,大熟;南,小饑;西,大饑;出,其分民饑,大臣有棄者;一云守之,秋冬水災。彗星犯守,東夷自滅;出,則為旱,為兵,北方亂。孛犯,為外夷亂,糴貴;守之,外夷災;出,為穀貴,民死,流亡;春夏犯之,金玉貴;秋冬,土功興;
 入,則多風雨;色黃,外夷來貢。雲氣出,色蒼白,國災除;入,則蠻夷來見;出而色黃,有使者;出箕口,斂,為雨;開,為多風少雨。



 按漢永元銅儀,箕宿十度,唐開元游儀十一度。舊去極百十八度,今百二十度。景祐測驗,箕四星十度,距西北第一星去極百二十三度。



 糠一星,在箕舌前,杵西北。明,則豐熟;暗,則民饑,流亡。



 杵三星,在箕南,主給庖舂。動,則人失釜甑;縱,則豐;橫,則
 大饑;亡,則歲荒;移徙,則人失業。熒惑守,民流。客星犯、守,歲饑。彗、孛犯,天下有急兵。



 按《晉志》,糠一星、杵三星在二十八宿之外。《乾象新書》與《步天歌》皆屬箕宿。



 北方



 南斗六星,天之賞祿府,主天子壽算,為宰相爵祿之位,傳曰:天廟也。丞相太宰之位,褒賢進士,稟受爵祿,又主兵。一曰天機。南二星魁,天梁也。中央二星,天相也。北二星,天府廷也。又謂南星者,魁星也;北星,杓也,第一
 星曰北亭,一曰天開,一曰鈇鍎。石申曰:「魁第一主吳,二會稽,三丹陽,四豫章,五廬江,六九江。」星明盛,則王道和平,帝王長齡,將相同心;不明,則大小失次;芒角動搖,國失忠臣,兵起,民愁。日食在斗,將相憂,兵起,皇後災,吳分有兵。日暈,宰相憂,宗廟不安。月食,其分國饑,小兵,後、夫人憂。月暈,大將死,五穀不生。月犯,將臣黜,風雨不時,大臣誅;一歲三入,大赦;又占:入,為女主憂,趙、魏有兵;色惡,相死。歲星犯,有赦;久守,水災,穀貴;守及百日,兵用,大臣
 死。熒惑犯,有赦,破軍殺將,火災;入二十日,糴貴;四十日,有德令;守之,為兵、盜;久守,災甚;出鬥上行,天下憂;不行,臣憂;入,內外有謀;守七日,太子疾。填星犯,為亂;入,則失地;逆行,地動;出、入、留二十日,有大喪;守之,大臣叛。又占;逆行,先水後旱;守之,國多義士。太白犯之,有兵,臣叛;留守之,破軍殺將;與火俱入,白爍,臣子為逆;久,則禍大。辰星犯,水,穀不成,有兵;守之,兵、喪。客星犯,兵起,國亂;入,則諸侯相攻,多盜,大旱,宮廟火,穀貴;七日不去,有赦。彗星
 犯,國主憂;出,則其分有謀,又為水災,宮中火,下謀上,有亂兵;入,則為火,大臣叛。孛犯入,下謀上,有亂兵;出,則為兵,為疾,國憂。流星入,蠻夷來貢;犯之,宰相憂,在春天子壽,夏為水,秋則相黜,冬大臣逆;色赤而出鬥者,大臣死。雲氣入,蒼白,多風;赤,旱;出,有兵起,宮廟火;入,有兩赤氣,兵;黑,主病。



 按漢永元銅儀,斗二十四度四分度之一,唐開元游儀,二十六度。去極百十六度,今百十九度。景祐測驗,
 亦二十六度,距魁第四星去極百二十二度。



 鱉十四星,在南斗南,主水族,不居漢中,川有易者。熒惑守之,為旱。辰星守,為火。客星守,為水。流星出,色青黑,為水;黃,為旱。雲氣占同。一曰有星守之,白衣會,主有水。



 天淵十星,一曰天池,一曰天泉,一曰天海,在鱉星東南九坎間,又名太陰,主灌溉溝渠。五星守之,大水,河決。熒惑入,為旱。客星入,海魚出。彗星守之,川溢傷人。



 狗二星,在南斗魁前,主吠守,以不居常處為災。熒惑犯
 之,為旱。客星入,多土功,北邊饑;守之,守禦之臣作亂。



 建六星,在南斗魁東北,臨黃道,一曰天旗,天之都關。為謀事,為天鼓,為天馬。南二星,天庫也。中二星,市也,鈇鍎也。上二星,為旗跗。斗建之間,三光道也,主司七曜行度得失,十一月甲子天正冬至,大歷所起宿也。星動,人勞役。月犯之,臣更天子法;掩之,有降兵。月食,其分皇后娣侄當黜。月暈,大將死,五穀不成,蛟龍見,牛馬疫。月與五星犯之,大臣相譖有謀,亦為關梁不通,大水。歲星守,為
 旱,糴貴,死者眾,諸侯有謀;入,則有兵。熒惑守之,臣有黜者,諸侯有謀,糴貴;入,則關梁不通,馬貴;守旗跗三十日,有兵。填星守之,王者有謀。太白守,外國使來。辰星守,為水災,米貴,多病。彗、孛、客星犯之,王失道,忠臣黜。客星守之,道路不通,多盜。流星入,下有謀;色赤,昌。



 天弁九星弁一作辨,在建星北,市官之長,主列肆、闤闠、市籍之事,以知市珍也。明盛,則萬物昌;不明及彗、客犯之,糴貴;久守之,囚徒起兵。



 天雞二星,在牛西,一在狗國北,主異鳥,一曰主候時。熒惑舍之,為旱,雞多夜鳴。太白、熒惑犯之,為兵。填星犯之,民流亡。客星犯,水旱失時;入,為大水。



 狗國四星,在建星東南,主三韓、鮮卑、烏桓、玁狁、沃且之屬。星不具,天下有盜;不明,則安;明,則邊寇起。月犯之,烏桓、鮮卑國亂。熒惑守之,外夷兵起。太白守之,鮮卑受攻。客星守,其王來中國。



 天鑰八星,在南斗杓第二星西,主開閉門戶。明,則吉;不
 備,則關鑰無禁。客星、彗星守之,關梁閉塞。



 農丈人一星,在南斗西南,老農主稼穡者,又主先農、農正官。星明,歲豐;暗,則民失業;移徙,歲饑。客星、彗星守之,民失耕,歲荒。



 按《步天歌》,已上諸星皆屬南斗。《晉志》以狗國、天雞、天弁、天鑰、建星皆屬天市垣,餘在二十八宿之外。《乾象新書》以天鑰、農丈人屬箕,武密又以天鑰屬尾,互有不同。



 牛宿六星,天之關梁,主犧牲事。其北二星,一曰即路,一曰聚火。又曰:上一星主道路,次二星主關梁,次三星主南越。明大,則王道昌,關梁通,牛貴;怒,則馬貴;動,則牛災,多死;始出而色黃,大豆賤;赤,則豆有蟲;青,則大豆貴;星直,糴賤;曲,則貴。日食,其分兵起;暈,為陰國憂,兵起。月食,有兵;暈,為水災,女子貴,五穀不成,牛多暴死,小兒多疾。月暈在冬三月,百四十日外有赦;暈中央大星,大將被戮。月犯之,有水,牛多死,其國有憂。歲星入犯,則諸侯失
 期;留守,則牛多疫,五穀傷;在牛東,不利小兒;西,主風雪;北,為民流;逆行,宮中有火;居三十日至九十日,天下和平,道德明正。熒惑犯之,諸侯多疾,臣謀主;守,則穀不成,兵起;入或出守斗南,赦。填星犯之,有土功;守之,雨雪,民人、牛馬病。太白犯之,諸侯不通;守,則國有兵起;入,則為兵謀,人多死。辰星犯,敗軍移將,臣謀主。客星犯守之,牛馬貴,越地起兵;出,牛多死,地動,馬貴。彗星犯之,吳分兵起;出,為糴貴,牛死。孛犯,改元易號,糴貴,牛多死,吳、越兵
 起,下當有自立者。流星犯之,王欲改事;春夏,穀熟;秋冬,穀貴;色黑,牛馬昌,關梁入貢。雲氣蒼白橫貫,有兵、喪;赤,亦為兵;黃白氣入,牛蕃息;黑,則牛死。



 按漢永元銅儀,以牽牛為七度,唐開元游儀八度。舊去極百六度,今百四度。景祐測驗,牛六星八度,距中央大星去極百十度半。



 天田九星,在斗南,一曰在牛東南,天子畿內之田。其占與角北天田同。客星犯之,天下憂。彗、孛犯守之,農夫失
 業。



 河鼓三星,在牽牛西北,主天鼓,蓋天子及將軍鼓也。一曰三鼓,主天子三將軍,中央大星為大將軍,左星為左將軍,右星為右將軍。左星,南星也,所以備關梁而拒難也,設守險阻,知謀徵也。鼓欲正直而明,色黃光澤,將吉;不正,為兵、憂;星怒,則馬貴;動,則兵起;曲,則將失計奪勢;有芒角,將軍兇猛象也;亂搖,差度亂,兵起。月犯之,軍敗亡。五星犯之,兵起。彗星、客星犯,將軍被戮。流星犯,諸侯作
 亂。黃白雲氣入之,天子喜;赤,為兵起;出,則戰勝;黑,為將死。青氣入之,將憂;出,則禍除。



 左旗九星,在河鼓左旁,右旗九星,在牽牛北、河鼓西南,天之鼓旗旌表也。主聲音、設險、知敵謀。旗星明大,將吉。五星犯守,兵起。



 織女三星,在天市垣東北,一曰在天紀東,天女也,主果蓏、絲帛、珍寶。王者至孝,神祇咸喜,則星俱明,天下和平;星怒而角,布帛貴。陶隱居曰:「常以十月朔至六七日晨
 見東方。」色赤精明者,女工善;星亡,兵起,女子為候。織女足常向扶筐,則吉;不向,則絲綿大貴。月暈,其分兵起。熒惑守之,公主憂,絲帛貴,兵起。彗星犯,後族憂。星孛,則有女喪。客星入,色青,為饑;赤,為兵;黃,為旱;白,為喪;黑,為水。流星入,有水、盜,女主憂。雲氣入,蒼白,女子憂;赤,則為女子兵死;色黃,女有進者。



 漸臺四星,在織女東南,臨水之臺也,主晷漏、律呂事。明,則陰陽調而律呂和;不明,則常漏不定。客星、彗星犯之,
 陰陽反戾。



 輦道五星,在織女西,主王者游嬉之道。漢輦道通南北宮,其象也。太白、熒惑守之,御路兵起。



 九坎九星,在牽牛南,主溝渠、導引泉源、疏瀉盈溢,又主水旱。星明,為水災;微小,吉。月暈,為水;五星犯之,水溢。客星入,天下憂。雲氣入,青,為旱;黑,為水溢。



 羅堰三星,在牽牛東,拒馬也,主堤塘,壅蓄水源以灌溉也。星明大,則水泛溢。



 天桴四星,在牽牛東北橫列,一曰在左旗端,鼓桴也,主漏刻。暗,則刻漏失時。武密曰:「主桴鼓之用。」動搖,則軍鼓用;前近河鼓,若桴鼓相直,皆為桴鼓用。太白、熒惑守之,兵鼓起。客星犯之,主刻漏失時。



 按《步天歌》,已上諸星俱屬牛宿。《晉志》以織女、漸臺、輦道皆屬太微垣,以河鼓、左旗、右旗、天桴屬天市垣,餘在二十八宿之外。武密以左旗屬箕屬鬥,右旗亦屬鬥,漸臺屬鬥,又屬牛,餘與《步天歌》同。《乾象新書》則又
 以左旗、織女、漸臺、輦道、九坎皆屬於斗。



 須女四星,天之少府,賤妾之稱,婦職之卑者也,主布帛裁制、嫁娶。星明,天下豐,女巧,國富;小而不明,反是。日食在女,戒在巫祝、后妃禱祠,又占越分饑,後妃疾。日暈,後宮及女主憂。月食,為兵、旱,國有憂。月暈,有兵謀不成;兩重三重,女主死。月犯之,有女惑,有兵不戰而降,又曰將軍死。歲星犯之,後妃喜,外國進女;守之,多水,國饑,喪,糴貴,民大災,熒惑犯之,大臣、皇后憂,布帛貴,民大災;守
 之,土人不安,五穀不熟,民疾,有女喪,又為兵;入則糴貴;逆行犯守,大臣憂;居陽,喜;陰,為憂。填星犯守,有苛政,山水出,壞民舍,女謁行,後專政,多妖女;留五十日,民流亡。太白犯之,布帛貴,兵起,天下多寡女;留守,有女喪,軍發。辰星犯,國饑,民疾;守之,天下水,有赦,南地火,北地水,又兵起,布帛貴。客星犯,兵起,女人為亂;守之,宮人憂,諸侯有兵,江淮不通,糴貴。彗星犯,兵起,女為亂;出,為兵亂,有水災,米鹽貴。星孛,其分兵起,女為亂,有奇女來進;出入,國
 有憂,王者惡之。流星犯,天子納美女,又曰有貴女下獄;抵須女,女主死。《乙巳占》:出入而色黃潤,立妃後;白,為後宮妾死。雲氣入,黃白,有嫁女事;白,為女多病;黑,為女多死;赤,則婦人多兵死者。



 按漢永元銅儀,以須女為十一度。景祐測驗,十二度,距西南星去極百五度,在赤道外十四度。



 十二國十六星,在牛女南,近九坎,各分土居列國之象。九坎之東一星曰齊,齊北二星曰趙,趙北一星曰鄭,鄭
 北一星曰越,越東二星曰周,周東南北列二星曰秦,秦南二星曰代,代西一星曰晉,晉北一星曰韓,韓北一星曰魏,魏西一星曰楚,楚南一星曰燕,有變動,各以其國占之。陶隱居曰:「越星在婺女南,鄭一星在越北,趙二星在鄭南,周二星在越東,楚一星在魏西南,燕一星在楚南,韓一星在晉北,晉一星在代北,代二星在秦南,齊一星在燕東。」



 離珠五星,在須女北,須女之藏府,女子之星也。又曰:主
 天子旒珠、後夫人環佩。去陽,旱;去陰,潦。客星犯之,後宮有憂。



 奚仲四星,在天津北,主帝車之官。凡太白、熒惑守之,為兵祥。



 天津九星,在虛宿北,橫河中,一曰天漢,一曰天江,主四瀆津梁,所以度神通四方也。一星不備,津梁不通;明,則兵起;參差,馬貴;大,則水災;移,則水溢。彗、孛犯之,津敗,道路有賊。客星犯,橋梁不修;守之,水道不通,船貴。流星出,
 必有使出,隨分野占之。赤雲氣入,為旱;黃白,天子有德令;黑,為大水;色蒼,為水,為憂;出,則禍除。



 敗瓜五星,在匏瓜星南,主修瓜果之職,與匏瓜同占。



 匏瓜五星一作瓠瓜,在離珠北,天子果園也,其西觜星主後宮,不明,則後失勢;不具或動搖,為盜;光明,則歲豐;暗,則果實不登。彗、孛犯之,近臣僭,有戮死者。客星守之,魚鹽貴,山谷多水;犯之,有游兵不戰。蒼白雲氣入之,果不可食;青,為天子攻城邑;黃,則天子賜諸侯果;黑,為天子食
 果而致疾。



 扶筐七星,為盛桑之器,主勸蠶也,一曰供奉後與夫人之親蠶。明,吉;暗,兇;移徙,則女工失業。彗星犯,將叛。流星犯,絲綿大貴。



 按《步天歌》,已上諸星俱屬須女,而十二國及奚仲、匏瓜、敗瓜等星,《晉志》不載,《隋志》有之。《晉志》又以離珠、天津屬天市垣,扶筐屬太微垣。《乾象新書》以周、越、齊、趙屬牛,秦、代、韓、魏、燕、晉、楚、鄭屬女。武密以離珠、匏瓜屬
 牛又屬女,以奚仲屬危。《乾象新書》以離珠、匏瓜屬牛,敗瓜屬斗又屬牛,以天津西一星屬鬥,中屬牛,東五星屬女。



 虛宿二星,為虛堂,塚宰之官也,主死喪哭泣,又主北方邑居、廟堂祭祀祝禱事。宋均曰:「危上一星高,旁兩星下,似蓋屋也。」蓋屋之下,中無人,但空虛似乎殯宮,主哭泣也。明,則天下安;不明,為旱;欹斜上下不正,享祀不恭;動,將有喪。日食,其分其邦有喪。日暈,民饑,後妃多喪。月食,
 主刀劍官有憂,國有喪。月暈,有兵謀,風起則不成,又為民饑。月犯之,宗廟兵動,又國憂,將死。歲星犯,民饑;守之,失色,天王改服;與填星同守,水旱不時。熒惑犯之,流血滿野;守之,為旱,民饑,軍敗;入,為火災,功成見逐;或勾巳,大人戰不利。填星犯之,有急令;行疾,有客兵;入,則有赦,穀不成,人不安;守之,風雨不時,為旱,米貴,大卜欲危宗廟,有客兵。太白犯,下多孤寡,兵,喪;出,則政急;守之,臣叛君;入,則大臣下獄。辰星犯,春秋有水;守之,亦為水災,在
 東為春水,南為夏水,西為秋水,北冬有雷雨、水。客星犯,糴貴;守之,兵起,近期一年,遠則二年,有哭泣事;出,為兵、喪。彗星犯之,國兇,有叛臣;出,為野戰流血;出入,有兵起,芒焰所指國必亡。星孛其宿,有哭泣事;出,則為野戰流血,國有叛臣。流星犯,光潤出入,則塚宰受賞,有赦令;色黑,大臣死;入而色青,有哭泣事;黃白,有受賜者;出,則貴人求醫藥。雲氣黃入,為喜;蒼,為哭;赤,火;黑,水;白,有幣客來。



 按漢永元銅儀,以虛為十度,唐開元游儀同。舊去極百四度,今百一度。景祐測驗,距南星去極百三度,在赤道外十二度。



 司命二星,在虛北,主舉過、行罰、滅不祥,又主死亡。逢星出司命,王者憂疾,一曰宜防祅惑。



 司祿二星,在司命北,主增年延德,又主掌功賞、食料、官爵。



 司危二星,在司祿北,主矯失正下,又主樓閣臺榭、死喪、
 流亡。



 司非二星,在司危北,主司候內外,察愆尤,主過失。《乾象新書》:命、祿、危、非八星主天子已下壽命、爵祿、安危、是非之事。明大,為災;居常,為吉。



 哭二星,在虛南,主哭泣、死喪。月、五星、彗、孛犯之,為喪。



 泣二星在哭星東,與哭同占。



 天壘城一十三星,在泣南,圜如大錢,形若貫索,主鬼方、北邊丁零類,所以候興敗存亡。熒惑入守,夷人犯塞。客
 星入,北方侵。赤雲氣掩之,北方驚滅,有疾疫。



 離瑜三星,在十二國東,《乾象新書》在天壘城南。離,圭衣也;瑜,玉飾,皆婦人見舅姑衣服也。微,則後宮儉約;明,則婦人奢縱。客星、彗星入之,後宮無禁。



 敗臼四星,在虛、危南,兩兩相對,主敗亡、災害。石申曰:「一星不具,民賣甑釜;不見,民去其鄉。」五星入,除舊布新。客星、彗星犯之,民饑,流亡。黑氣入,主憂。



 按《步天歌》,已上諸星俱屬虛宿。司命、司祿、司危、司非、
 離瑜、敗臼,《晉志》不載,《隋志》有之。《乾象新書》以司命、司祿、司危、司非屬須女;泣星、敗臼屬危。武密書與《步天歌》合。



 危宿三星,在天津東南,為天子宗廟祭祀,又為天子土功,又主天府、天市、架屋、受藏之事。不明,客有誅,土功興;動或暗,營宮室,有兵事。日食,陵廟摧,有大喪,有叛臣。日暈,有喪。月食,大臣憂,有喪,宮殿圮。月暈,有兵、喪,先用兵者敗。月犯之,宮殿陷,臣叛主,來歲糴貴,有大喪。歲星犯
 守,為兵、役徭,多土功,有哭泣事,又多盜。熒惑犯之,有赦;守之,人多疾,兵動,諸侯謀叛,宮中火災;守上星,人民死,中星諸侯死,下星大臣死,各期百日十日;守三十日,東兵起,歲旱,近臣叛;入,為兵,有變更之令。填星守之,為旱,民疾,土功興,國大戰;犯之,皇后憂,兵,喪;出、入、留、舍,國亡地,有流血;入,則大亂,賊臣起。太白犯之,為兵,一曰無兵兵起,有兵兵罷,五穀不成,多火災;守之,將憂,又為旱,為火;舍之,有急事。辰星犯之,大臣誅,法官憂,國多災;守之,
 臣下叛,一云皇后疾,兵、喪起。客星犯,有哭泣,一曰多雨水,穀不收;入之,有土功,或三日有赦;出,則多雨水,五穀不登;守之,國敗,民饑。彗星犯之,下有叛臣兵起;出,則將軍出國,易政,大水,民饑。孛犯,國有叛者兵起。流星犯之,春夏為水災,秋冬為口舌;入,則下謀上;抵危,北地交兵。《乙巳占》:流星出入色黃潤,入民安,穀熟,土功興;色黑,為水,大臣災。雲氣入,蒼白,為土功;青,為國憂;黑,為水,為喪;赤,為火;白,為憂,為兵;黃出入,為喜。



 按漢永元銅儀,以危為十六度;唐開元游儀十七度。舊去極九十七度,距南星去極九十八度,在赤道外七度。



 虛梁四星,在危宿南,主園陵寢廟、禱祝。非人所處,故曰虛梁。一曰宮宅屋幃帳寢。太白、熒惑犯之,為兵。彗、孛犯,兵起,宗廟改易。



 天錢十星,在北落師門西北,主錢帛所聚,為軍府藏。明,則庫盈;暗,為虛。太白、熒惑守之,盜起。彗、孛犯之,庫藏有
 賊。



 墳墓四星,在危南,主山陵、悲慘、死喪、哭泣。大曰墳,小曰墓。五星守犯,為人主哭泣之事。



 杵三星,在人星東,一云在臼星北,主舂軍糧。不具,則民賣甑釜。



 臼四星,在杵星下,一云在危東。杵臼不明,則民饑;星眾,則歲樂;疏,為饑;動搖,亦為饑;杵直下對臼,則吉;不相當,則軍糧絕;縱,則吉;橫,則荒;又臼星覆,歲饑;仰,則歲熟。彗
 星犯之,民饑,兵起,天下急。客星守之,天下聚會米粟。



 蓋屋二星,在危宿南九度,主治宮室。五星犯之,兵起。彗、孛犯守,兵災尤甚。



 造父五星,在傳舍南,一曰在騰蛇北,御官也。一曰司馬,或曰伯樂,主御營馬廄、馬乘、轡勒。移處,兵起,馬貴;星亡,馬大貴。彗、客入之,僕御謀主,有斬死者,一曰兵起;守之,兵動,廄馬出。



 人五星,在虛北,車府東,如人形,一曰主萬民,柔遠能邇;
 又曰臥星,主夜行,以防淫人。星亡,則有詐作詔者,又為婦人之亂;星不具,王子有憂。客、彗守犯,人多疾疫。



 車府七星,在天津東,近河,東西列,主車府之官,又主賓客之館。星光明,潤澤,必有外賓,車駕華潔。熒惑守之,兵動。彗、客犯之,兵車出。



 鉤九星,在造父西河中,如鉤狀。星直,則地動;他星守,占同。一曰主輦輿、服飾。明,則服飾正。



 按《步天歌》,已上諸星俱屬危宿。《晉志》不載人星、車府,《
 隋志》有之。杵、臼星,《晉》、《隋志》皆無。造父、鉤星,《晉志》屬紫微垣,蓋屋、虛梁、天錢在二十八宿外。《乾象新書》以車府西四星屬虛,東三星屬危。武密書以造父屬危又屬室,餘皆與《步天歌》合。按《乾象新書》又有天綱一星,在危宿南,入危八度,去極百三十二度,在赤道外四十一度。《晉》、《隋志》及諸家星書皆不載,止載危、室二宿間與北落師門相近者。近世天文乃載此一星,在鬼、柳間,與外廚、天紀相近。然《新書》兩天綱雖同在危度,
 其說不同,今姑附於此。



 營室二星,天子之宮,一曰玄宮,一曰清廟,又為軍糧之府,主土功事。一曰室一星為天子宮,一星為太廟,為王者三軍之廩,故置羽林以衛;又為離宮閣道,故有離宮六星在其側。一曰定室,《詩》曰「定之方中」也。星明,國昌;不明而小,祠祀鬼神不享;動,則有土功事;不具,憂子孫;無芒、不動,天下安。日食在室,國君憂,王者將兵,一曰軍絕糧,土卒亡。日暈,國憂,女主憂黜。月食,其分有土功,歲饑。
 月暈,為水,為火,為風。



 月犯之,為土功,有哭泣事。歲星犯之,有急而為兵;入,天子有赦,爵祿及下;舍室東,民多死;舍北,民憂;又曰守之,宮中多火災,主不安,民疫。熒惑犯,歲不登;守之,有小災,為旱,為火,糴貴;逆行守之,臣謀叛;入,則創改宮室;成勾巳者,主失宮。填星犯,為兵;守之,天下不安,人主徙宮,後、夫人憂,關梁不通,貴人多死;久守,大人惡之,以赦解,吉;逆行,女主出入恣;留六十日,土功興。太白犯五寸許,天子政令不行;守,則兵大忌之,以赦
 令解;一曰太子、后妃有謀;若乘守勾巳、逆行往來,主廢后妃,有大喪,宮人恣;去室一尺,威令不行;留六十日,將死;入,則有暴兵。辰星犯之,為水;入,則後有憂,諸侯發動於西北。客星犯入,天子有兵事,軍饑,將離,外兵來;出於室,兵先起者敗。彗星出,占同;或犯之,則弱不能戰;出入犯之,則先起兵者勝,一曰出室為大水。孛犯或出入,先起兵者勝;出,有小災,後宮亂。武密曰:「孛出,其分有兵、喪;道藏所載,室專主兵。」流星犯,軍乏糧,在春夏將軍貶,秋
 冬水溢。《乙巳占》曰:「流星出入色黃潤,軍糧豐,五穀成,國安民樂。」雲氣入,黃,為土功;蒼白,大人惡之;赤,為兵,民疫;黑,則大人憂。



 按漢永元銅儀,營室十八度,唐開元游儀十六度。舊去極八十五度。景祐測驗,室十六度,距南星去極八十五度,在赤道外六度。



 雷電六星,在室南,明動,則雷電作。



 離宮六星,兩兩相對為一坐,夾附室宿上星,天子之別
 宮也,主隱藏止息之所。動搖,為土功;不具,天子憂。太白、熒惑入,兵起;犯或勾巳環繞,為后妃咎。彗星犯之,有修除之事。



 壘壁陣十二星一作壁壘,在羽林北,羽林之垣壘,主天軍營。星明,國安;移動,兵起;不見,兵盡出,將死。五星入犯,皆主兵。太白、辰星,尤甚。客星入,兵大起,將吏憂。流星入南,色青,後憂;入北,諸侯憂;色赤黑,入東,後有謀;入西,太子憂;黃白,為吉。



 騰蛇二十二星,在室宿北,主水蟲,居河濱。明而微,國安;移向南,則旱;向北,大水。彗、孛犯之,水道不通。客星犯,水物不成。



 土功吏二星,在壁宿南,一曰在危東北,主營造宮室,起土之官。動搖,則版築事起。



 北落師門一星,在羽林軍南,北宿在北方,落者,天軍之藩落也,師門猶軍門。長安城北門曰「北落門」,像此也。主非常以候兵。星明大,安;微小、芒角,有大兵起。歲星犯之,
 吉。熒惑入,兵弱不可用。客星犯之,光芒相及,為兵,大將死;守之,邊人入塞。流星出而色黃,天子使出;入,則天子喜;出而色赤,或犯之,皆為兵起。雲氣入,蒼白,為疾疫;赤,為兵;黃白,喜;黑雲氣入,邊將死。



 八魁九星,在北落東南,主捕張禽獸之官也。客、彗入,多盜賊,兵起。太白、熒惑入守,占同。



 天綱一星,在北落西南,一曰在危南,主武帳宮舍,天子游獵所會。客、彗入,為兵起,一云義兵。



 羽林軍四十五星,三三而聚散,出壘壁之南,一曰在營室之南,東西布列,北第一行主天軍,軍騎翼衛之象。星眾,則國安;稀,則兵動;羽林中無星,則兵盡出,天下亂。月犯之,兵起。歲星入,諸侯悉發兵,臣下謀叛,必敗伏誅。太白入,兵起。填星入,大水。五星入,為兵。熒惑、太白經過,天子以兵自守。熒惑入而芒赤,興兵者亡。客星入,色黃白,為喜;赤,為臣叛。流星入南,色青,後有疾;入北,諸侯憂;入東而赤黑,後有謀;入西,太子憂。雲氣蒼白入南,後有憂;
 北,諸侯憂;黑,太子、諸侯忌之;出,則禍除;黃白,吉。



 斧鉞三星,在北落師門東,芟刈之具也,主斬芻稿以飼牛馬。明,則牛馬肥腯;動搖而暗,或不見,牛馬死。《隋志》、《通志》皆在八魁西北,主行誅、拒難、斬伐奸謀。明大,用兵將憂;暗,則不用;移動,兵起。月入,大臣誅。歲星犯,相誅。熒惑犯,大臣戮。填星入,大臣憂。太白入,將誅。客、彗犯,斧鉞用;又占:客犯,外兵被擒,士卒死傷,外國降;色青,憂;赤,兵;黃白,吉。



 按《步天歌》,已上諸星皆屬營室。雷電、土功吏、斧鉞,《晉志》皆不載,《隋志》有之。壘壁陣、北落師門、天綱、羽林軍,《晉志》在二十八宿外,騰蛇屬天市垣。武密書以騰蛇屬營室,又屬壁宿。《乾象新書》以西十六星屬尾、屬危,東六星屬室;羽林軍西六星屬危,東三十九星屬室;以天綱屬危,斧鉞屬奎。《通占錄》又以斧鉞屬壁、屬奎,說皆不同。



 壁宿二星,主文章,天下圖書之秘府。明大,則王者興,道
 術行,國多君子;星失色,大小不同,王者好武,經術不用,圖書廢;星動,則有土功。日食於壁,陽消陰壞,男女多傷,國不用賢。日暈,名士憂。月食,其分大臣憂,文章士廢,民多疫。月暈,為風、水,其分有憂。月犯之,國有憂,為饑,衛地有兵。歲星犯之,水傷五穀;久守或凌犯、勾巳,有兵起。熒惑犯之,衛地憂;守之,國旱,民饑,賢不用;一占:王有大災。填星犯守,圖書興,國王壽,天下豐,國用賢;一占:物不成,民多病;逆行成勾巳者,有土功;六十日,天下立王。太白
 犯之一二寸許,則諸侯用命;守之,文武並用,一曰有軍不戰,一曰有兵喪,一曰水災,多風雨;一曰犯之多火災。辰星犯,國有蓋藏保守之事,王者刑法急;守之,近臣憂,一曰其分有喪,有兵,奸臣有謀;逆行守之,橋梁不通。客星犯之,文章士死,一曰有喪;入,為土功,有水;守之,歲多風雨;舍,則牛馬多死。彗星犯之,為兵,為火,一曰大水,民流。孛犯,為兵,有火、水災。流星犯,文章廢;《乙巳占》曰:「若色黃白,天下文章士用。」赤雲氣入之,為兵;黑,其下國破;黃,
 則外國貢獻,一曰天下有烈士立。



 按漢永元銅儀,東壁二星九度。舊去極八十六度。景祐測驗,壁二星九度,距南星去極八十五度。



 天廄十星,在東壁之北,主馬之官,若今驛亭也,主傳令置驛,逐漏馳騖,謂其急疾與晷漏競馳也。月犯之,兵馬歸。彗星入,馬廄火。客星入,馬出行。流星入,天下有驚。



 霹靂五星,在雲雨北,一曰在雷電南,一曰在土功西,主陽氣大盛,擊碎萬物。與五星合,有霹靂之應。



 雲雨四星,在雷電東,一云在霹靂南,主雨澤,成萬物。星明,則多雨水。辰星守之,有大水;一占:主陰謀殺事,孳生萬物。



 鈇鍎五星,在天倉西南,刈具也,主斬芻飼牛馬。明,則牛馬肥;微暗,則牛馬饑餓。



 按《步天歌》,壁宿下有鈇鍎五星,《晉》、《隋志》皆不載。《隋志》八魁西北三星曰鈇鍎,又曰鈇鉞,其占與《步天歌》室宿內斧鉞略同,恐即是此誤重出之。霹靂五星、雲雨
 四星,《晉志》無之,《隋志》有之。武密書以雲雨屬室宿。天廄十星,《晉志》屬天市垣,其說皆不同。



\end{pinyinscope}