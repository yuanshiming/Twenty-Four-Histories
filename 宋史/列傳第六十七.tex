\article{列傳第六十七}

\begin{pinyinscope}

 上官正盧斌周審玉裴濟李繼宣張旦張煦張佶



 上官正,字常清,開封人。少舉《三傳》,後為鄜州攝官。雍熙中,召授殿前承旨,屢遣鞫獄,遷供奉官、閣門祗候、天雄
 監軍。淳化中,轉作坊副使、劍門都監。李順之亂,分其黨趨劍門,時疲兵數百人,正奮勵士氣以御之。會成都監軍宿翰領兵投劍門,與正兵合,因迎擊,大破賊數千眾,斬馘殆盡。奏至,太宗嘉之,詔書獎飭,並賜襲衣、金帶,超正為六宅使、劍州刺史、充劍門部署,翰自供奉官擢崇儀使、領昭州刺史。數月,正被疾,請尋醫,至闕。疾愈,入對,上勞問久之,復遣還任所,賜以金丹、良藥、衣帶、白金千兩、馬三匹,授以方略,令招撫殘孽,慰勉遣之。



 初,川賊甚
 盛,朝議深以棧路為憂,正以孤軍力戰挫賊鋒,自是閣道無壅,王師得以長驅而入。賊眾三百餘,敗歸成都,順怒其驚眾,盡斬之,然自此沮氣矣。後賊既誅,餘寇匿山谷,恃險結集,剽劫為患。王繼恩百計召誘不至,正諭以朝廷恩信,皆相率出降。未幾,加峰州團練使,與雷有終並為西川招安使,代王繼恩。



 正木強好凌人,自謂平賊有勞,受人主知,無所顧忌。數面攻兩川官吏之短而暴揚之,眾積怨怒,多上章訴其不法者。太宗謂近臣曰:「人
 臣可任用者,朕常欲保全。正婞直而失於謙和,每謗書至,朕雖力與明辯,然眾怒難犯,恐其不能自全。」乃賜手札戒諭曰:「言者,君子之樞機,樞機之發,榮辱之主,不可不慎也。夫遇事輒發,悔不可及。儻自恃無瑕,而好面攻人之短,豈謂喜怒不形於色耶?當以和輯遠民為念,斯盡善矣。」正上表謝。



 真宗即位,改莊宅使。是秋,廣武叛卒劉旴嘯聚數千輩,逐都巡檢使韓景祐,略漢蜀邛州、懷安永康軍。正與鈐轄馬知節領兵趨新津,抵方井,擊敗
 之,斬旴,平其黨。遷南作坊使,賜錦袍、金帶。咸平初,召還,擢拜東上閣門使、勾當軍頭引見司,俄權戶部使。二年,出知滄州,徙高陽關副都部署,真拜洺州團練使。車駕北巡,以為行營先鋒鈐轄。



 尋知青州,未行,會王均叛蜀,命為峽路都鈐轄,移知梓州。又歷滄、瀛、鎮、貝四州,高陽關部署。以足疾,求知磁州,手詔慰勉。會邢州地震,民居不安,徙正典之。移潞州。景德中,以河北新經兵革,慎擇守臣,以正知貝州,遷洺州防禦使,復知滄州,移同州。再表
 引年,授左龍武軍大將軍、平州防禦使,分司西京。尋以本官致仕,賜全俸,仍以見緡給之。四年,卒,年七十五。子璨至內殿崇班。



 盧斌,開封人。以筆札事晉邸,太宗即位,補殿直。雍熙中,領兵屯霸州。會大舉北伐,令以五千騎隨曹彬抵祁溝。時契丹據河,王師乏水,斌請以千弩斫砦,契丹遁去,遂移軍夾河。既克涿州,令斌以萬人戍守,會食盡,大兵將還,斌因懇言:「涿州深在北境,外無援兵,內無資糧,丁籍
 殘失,守之無利。今若還師,必須結陣而去,以一陣之役,比於固守,其利百矣。」復慮遼人乘便剽襲,宜為之備。彬以為然,遂令斌擁城中老幼,並狼山南還易州。彬之旋也,無復行伍,果為契丹所乘。諸將皆以失律被譴,斌亦下樞密院問狀,太宗聞其嘗建議棄涿州,遂釋不問。以為霸州破虜軍緣邊巡檢。



 端拱中,又為永舉軍、華州巡檢。時大賊侯和尚、劉渥劫興平、櫟陽,殺捕賊官二人。斌率兵掩襲,且追且鬥,薄南山,渡渭水,抵鳳翔,復至耀州,
 擒錐並盡。以勞,改供奉官。召還,面加獎慰,授閣門祗候,又賜白金、緡錢、衣帶。尋為梓、遂十二州都巡檢使,太宗諭之曰:「川陜人情易搖,設有寇攘,雖他境亦當襲逐,仍許便宜從事,不須中覆。」淳化二年,賊任誘等寇昌、合州。斌率兵頓昌州南牛斗山,偵知賊在龍水鎮,值大雨,斌馳馬四十里,騎從數十人,遂斬誘等百餘級,賊眾悉平。



 三年,富順監蠻掠榮州,斌晨夜倍道以赴,得州兵千人,署隨軍糧料以張其勢。蠻乃遁,追至地頭鎮東南八
 十里,樹柵,招其酋甫羌一阿奴綱,諭以朝旨,歃血刻石為盟而遣之。俄而榮、戎、資州、富順監賊十五隊鈔鄉邑,斌擒三百人,部送闕下,餘悉臨敵斬戮。



 四年,賊王盡復起榮、資,斌擊滅之,盡縛以獻。遷內殿崇班。是冬,李順為亂,斌即率兵六百抵成都,鬥戰連月,殺數萬人。明年,成都不守,斌還梓州,集十州兵赴援,知州張雍委以監護之任。會江水泛溢,毀子城。斌勸諭州民,翌日,畚鍤大集,自城西大濠中掘塹深丈,決西河水,注之以環城。二月,賊
 渠相里貴眾二十一萬傅城下,城中兵裁三千。斌曰:「軍法倍兵不戰,然狂丑烏合,非訓練之師,以吾仗天子威靈,必可殲蕩。」即感厲士伍,負土塞南北門,為固守之計。又突出與賊戰,擊刺三十餘合,賊稍卻。俄復大設機石、連弩、沖車、雲梯,四面鼓噪乘城,矢石亂下,斌與州將隨機設備。長圍八十日,會王繼恩令石知顒率兵來援,斌出東門迎勞王師,賊不戰而潰。斌乘勝追斬及納降二萬餘。五月,賊數萬圍閬州,斌領千兵赴之,斬首五千,圍
 遂解。又至蓬州老雅山,賊眾三千為陣拒斌,斌擊敗之,至城下,賊復大集,斬三千級。蓬州平,斌傳詔安撫蓬、閬、渠、達四州,擢授西京作坊使,領成州刺史。



 斌在川峽六年,以孤軍御寇,累立戰功,表求入奏。太宗遣使諭之曰:「俟妖孽盡殄,當召汝。」既而賊黨集梓、綿、漢三州境上,斌往平之。未幾,代還,太宗親加勞問。拜東上閣門使、檢校左僕射,加食邑三百戶,賜白金千兩、袍笏、金帶。上言:「葭萌路出師討賊,可直入利州。若寇焚棧道,劍門之險不
 足固也,請置砦柵。」從之。



 尋命為銀、夏兵馬鈐轄,遣與李繼隆等五路出師討李繼遷。斌求對,懇言曰:「羌夷之族,馬驕兵悍,往來無定,敗則走他境,疾戰沙漠,非天兵所利。不若堅保靈州,於內地多積芻糧,以師援送。茍其至也,會兵首尾擊之,庶幾無枉費,而不失固圉之策矣。」時業已出師,不從其議。改授靈環路鈐轄,領兵二萬為前鋒,令於烏、白池與諸軍會。斌謂李繼隆曰:「靈州抵烏、白池,月餘方至。若自環州橐駝路,裁十日程。」即不俟詔而
 往,與諸將失期,不見賊而還。俄徙屯寧州,以疾召歸,勾當軍頭引見司。咸平初,卒,年五十。子文質殿中丞。



 周審玉,開封人。父勛,以親校事唐明宗,累立戰功,太平興國中,至隰州團練使。周顯德初,審玉蔭補殿直,從世宗平瓦橋關,甚見親信。太祖受禪,為供奉官,未幾,加閣門祗候。累遷崇儀、洛苑副使,西京作坊使。雍熙中,契丹犯塞,潘美屯師定州,審玉為監軍。嘗與敵戰,而先鋒劉緒陷賊,審玉躍馬趣擊,拔緒而還,以勇敢聞。



 淳化中,知
 貝州。有驍捷卒戍州者三十七人,同謀殺審玉,劫庫兵而叛,推虞候趙咸雍為首。審玉覺之,與轉運使王嗣宗率兵悉擒其黨,斬十五級,磔咸雍於市。先是,咸雍父鏻,晉天福中,嘗誘契丹屠州城。至是五十年,而其子戮於都市,舊老猶記其事,咸異之。審玉以功領順州刺史。



 至道初,徙並州鈐轄。咸平初,知鳳翔府。有桑門乘傳而西,以市木為名,威動府縣。審玉曰:「此有所倚而為也。」因按詰之,盡得其奸狀,杖其背,械送闕下。以目疾,代還,奉朝
 請,俄丁內艱。既而謂親友曰:「僕齒發遲暮,而未能辭祿仕者,良以慰母心爾,今可行其志矣。」乃拜章請老,得千牛衛大將軍致仕。三年,卒,年七十四。審玉晚年,好讀《神農本草》,留意方術。少長兵間,習知攻守之法。真宗嘗召至便坐,示以攻戰器。方奏對,疾作,詔遣使就第,賜白金慰恤之。子允迪,為虞部員外郎。



 裴濟字仲溥,絳州聞喜人。唐相耀卿八世孫,後徙家河中。濟少事晉邸,同輩有忮悍者,濟屢糾其過失,被譖,出
 補太康鎮將。未幾,譖濟者坐法。太宗知濟可任,會即位,補殿直,為天威軍兵馬監押。及平太原,徵幽薊,濟迎謁陪扈,令監軍易州,契丹攻城不能下。以勞,遷西頭供奉官。



 太平興國末,江表盜起,命為巡檢,遷崇儀副使。召還,遷崇儀使。監戍兵於威虜軍,塗次鎮州,夜有賊騎扣城門,大呼曰:「官軍至矣。」州將然之,促守吏開關,濟遽止之曰:「此必妄也。」及旦,果有敵兵遁去。太宗嘉之,遷西上閣門使、定州都監,就加行營鈐轄,尋知定州。契丹三萬騎
 來攻,濟逆擊於徐河,斬數千級,獲牛馬、鎧仗甚眾。



 淳化初,與周瑩同判四方館,未幾,為鎮州行營鈐轄。又與李繼隆擊賊於唐河,濟短兵陷陣,賊大敗走,優詔褒美。初,繼隆以濟性剛,不悅之;及是役,撫濟恨相知之晚。改四方館使,復知定州,徙天雄軍鈐轄。遷客省使,復知定州。至道二年,改內客省使、知鎮州。立春日,出土牛以祭,酌奠始畢,有卒挾牛去。濟察其舉止,知欲為變,亟命擒之,果有竊發者數十人,已劫廛間矣,悉搜捕腰斬之,軍民
 肅然。濟在鎮、定凡十五年,威績甚著。召還,知天雄軍。



 咸平初,李繼遷叛,以濟領順州團練使、知靈州兼都部署。至州二年,謀緝八鎮,興屯田之利,民甚賴之。其年,清遠軍陷,夏人大集,斷餉道,孤軍絕援,濟刺指血染奏,求救甚急,兵不至,城陷,死之。上聞嗟悼,特贈鎮江軍節度。三子並優進秩。濟在諸使中甚有聲望,及沒,夏人皆惜之。景德中,濟妻永泰郡君景氏卒,特詔追封平陽郡夫人,諸子給奉終喪。



 子德穀虞部郎中,德基至如京使,德豐
 殿中丞。濟兄麗澤、弟麗正,並進士及第。麗澤至右補闕,麗正至金部員外郎。麗正子德輿,為殿中丞。



 李繼宣,開封浚儀人。乾德中,補右班殿直,令與御帶更直,裁十七歲。嘗命往陜州捕虎,殺二十餘,生致二虎、一豹以獻。太平興國初,掌南作坊使,改供奉官,出為邠、寧、慶三州巡檢、都監。繼宣本名繼隆,與明德皇后兄同姓名。至是,太宗為改焉。



 五年,召還,承受定州路奏事。奉詔修長城口、平塞威虜靜戎軍、保州,又領兵入敵境,獲老幼
 千餘,牛畜數百。又率兵捍契丹於乾寧泥姑海口。契丹寇靜戎軍,從崔彥進過拒馬河接戰,自午至申,大敗之。又為貝州監軍。



 雍熙三年,曹彬北征,繼宣從先鋒李繼隆至方城,力戰三日,大軍繼至,遂克固州。進壁涿州東,又與敵鬥,乘勝攻北門,克之。日領輕騎度涿河,覘敵勢,又將五千騎援米信,因率勁騎追至新城北,大敗之,斬其酋賀恩相公,繼宣亦中流矢。大軍還雄州取芻糧,遇契丹新城,疾戰至暮,繼宣中十創,劍及兜鍪。明日復戰,
 繼隆為敵所邀,繼宣以所部拔之,且戰且行,奪涿河,數日,乃至涿州。及棄州保歧溝關,又戰拒馬上,追奔至孤山,契丹乃引去。留屯滿城,俄還貝州。召入,以功超授崇儀使,代王繼恩為易州駐泊都監,賜錢五十萬,白金五百兩。又領騎兵五千戍北平,押大陣東偏,受田重進節度,屯長城口。敵至大溝,繼宣進滿城。敵至定州,奪唐河橋,重進召繼宣洎田紹斌赴援,紹斌為敵所敗,繼宣獨按部轉鬥入定州。敵兵北去,重進命將五千騎躡其後,
 抵拒馬河。及敵據楊□,繼宣徑掩擊之,遂焚廬舍而遁。



 雍熙四年,為高陽關行營都監。端拱初,契丹騎至瀛、鎮,繼宣率步騎萬人入敵境,抵勝務,焚聚落,獲生口,契丹乃引還。時易州候騎不至,繼宣於易州、平塞軍、長城口、威虜靜戎順安軍至高陽,為望櫓七所,舉烽以候警急。二年,為鎮、定、高陽關三路排陣都監,押大陣西偏。與李繼隆部芻糧抵威虜,還度徐河,為敵追襲。繼宣駐軍與鬥,殺獲甚眾。又領騎二千,敗契丹於保州西射城,追薄
 西山,有詔褒美。



 淳化三年,徙知保州,又轉莊宅使。築關城,浚外濠,葺營舍千五百區;造船一百艘,入雞距泉以運糧,人咸便之。數月,徙定州行營都監,戍深州,改高陽關行營都監。課軍中勁弩,為入陣之備。五年,領高州刺史。會契丹泛海劫千乘縣,繼宣請於海口置砦以御之。



 至道三年,遷北作坊使,俄召還,加南作坊使,出為鎮州行營鈐轄。契丹寇定州,命主無地分馬。敵至懷德橋,繼宣領兵三千掩襲之。至則契丹已壞橋,繼宣橫木而度,
 追奔五十餘里。契丹焚鎮州中渡、常山二橋,繼宣領兵趣之,契丹保豐隆山砦,繼宣伐木治常山橋,契丹聞之,大懼,拔砦遁走。



 繼宣銳於追襲,傅潛為部署,繼宣詣潛請行,頗為所抑。及召潛屬吏,詔繼宣與高瓊同主軍事,逐敵越拒馬河,復為鎮州鈐轄。受詔按視緣邊城砦,權知威虜軍,敵騎至城下,屢出兵設伏,斬獲甚眾。俄還鎮州。



 咸平四年,拜西上閣門使,領康州刺史,為前陣鈐轄,與秦翰、楊延昭、楊嗣分屯靜戎、威虜。敵至,會師於威虜,
 延昭、嗣輕騎先赴羊山,繼宣與翰、分左右隊客整所部,翰全軍亦往,繼宣留壁赤虜,止以二騎繼進。至,則延昭、嗣適為敵所乘。繼宣即召赤虜之師,與翰師合勢大戰,敵走上羊山。繼宣逐之,環山麓至其陰。繼宣馬連中矢斃,凡三易騎,進至牟山谷,大克捷。延昭、嗣、翰之師,初頓赤虜,既而退保威虜,繼宣以所部獨與敵角,薄暮,始至威虜。詔書稱獎,特加檢校官及食邑。



 明年,徙定州鈐轄,捍契丹於唐河。會緣邊都巡檢使楊延昭、楊嗣禦敵師
 敗,詔繼宣與內殿崇班王汀代之。望都之敗,敵騎剽郡縣,繼宣壁徐河,契丹數十隊薄威虜,威虜魏能與戰,走之,久而繼宣始至。又寇靜戎,汀請分兵自將襲契丹,繼宣拒之,雖日出游騎偵敵勢,屢徙砦而未嘗出戰。為能、汀所發,召還,令樞密院問狀,降為如京副使。



 景德初,加如京使、鎮州鈐轄。契丹乘秋來攻,時桑贊病足,鄭誠赴定州,繼宣獨主鎮州全師,歷屯邢、趙。及與契丹和,命為高陽關鈐轄。是冬,復為西上閣門使,領康州刺史。三年,
 兼知瀛州。繼宣罕識字,上以河間郡事繁,慮獄訟有枉,命高繼勛代之,止為鈐轄。



 大中祥符初,徙鎮、定兩路鈐轄,進秩東上閣門使。召還,改鄆州部署,加四方館使。以疾,授西京水南都巡檢使,每夕罕巡警,為留司所舉,特詔增巡檢一員,專主夜巡。六年,疾甚,求至京師尋醫,卒,年六十四。子守忠,左侍禁、閣門祗候。



 張旦,趙州人。勇敢善射,以經學中第,至國子博士。淳化中,知陵州。時李順構亂,連下城邑。賊黨數萬攻陵州,州
 兵不滿三百,舊不設城塹。旦修完戰具,置鹿角砦,驅市人進戰,大敗之,殺五千餘人,獲器械萬計。詔書褒之,特遷水部員外郎,賜緋魚,由是知名。數月,西川招安使上官正言:「雅州密邇蠻蜑,在於鎮撫須得其人,伏見水部員外郎張旦,前守陵州,以孤軍抗群寇,保全壁壘,至今劍外伏其威名。望改授諸司使,令知州事。」上以省郎之重,不欲換他職,乃授刑部員外郎,賜金紫。乘傳之任,寇不敢犯。



 真宗即位,遷兵部員外郎,改尚食使、知德清軍。
 景德中,契丹入寇,陷軍壁。旦與其子利涉率眾奮擊,並戰沒。上聞之驚悼,特贈左衛大將軍、深州團練使,利涉崇儀副使。錄其四子官。時有上封事者,言朝廷宜優加恩典,以勸忠臣。詔以恤旦事告諭天下。



 又虎翼都虞候胡福戍軍城,率兵力戰,金創遍體,猶奮劍轉鬥,矢無虛發,麾下已盡,獨挺刃殺數十人。副指揮使尚祚能運大撾,所斬首拉肋者,亦百餘人,眾寡不敵,遂與指揮使張睿、劉福、都頭輔能等四人並死之。真宗嘉嘆其忠勇,遣
 使訪遺骸,唯得福尸,命其子厚葬之。贈福洺州團練使,祚濱州刺史,睿演州刺史,劉福臨州刺史,能等並贈諸衛率府副率。又邯鄲令李晦辭赴任,值道梗,留德清同拒敵;侍禁夏承皓部兵至大名界遇敵,皆戰沒。贈晦辭工部員外郎,承皓崇儀使。時又贈受事河朔而沒者,殿直劉超供備庫使,入內高班內品李知順為六宅副使,奉職胡度等三人為內殿崇班,仍各錄其子,及賜其家金帛。



 張煦,字輔暘,開封人。開寶末,補府中牙職。雍熙二年,自陳太宗尹京嘗事左右,命為殿前承旨,遷殿直、歙州監軍。兇人黃行達弟坐法抵死,行達誣州將故入其罪,詔宣州通判姚鉉與煦鞫之,即日決遣。還擢供奉官、閣門祗候。占謝日,又改內殿崇班、鎮定、邢、趙、山西、土門路都巡檢使。契丹騎兵剽境上,煦以所部斬首數十,走之。葛霸、周瑩、李繼宣稱其幹舉,有詔嘉獎。代還,拜供備庫副使,權知環州。數月,改岢嵐軍使,又知保安軍。



 咸平中,王
 均亂蜀,以煦為綿、漢、劍門路都巡檢使。又與雷有終進攻成都,煦主東砦,焚其郛及樓堞,均突圍而遁。賊平,以功就遷正使,徙益州都監,與知州宋太初同提總本路諸軍事。有戰艦卒將謀擾動,煦即日斬之。



 夏人寇邊,改涇原儀渭都鈐轄。又為邠寧環慶路鈐轄兼巡檢、安撫都監,累躡寇入賊中,掩殺甚眾,有詔嘉獎。會遣王超、張凝、秦翰援靈武,命煦為西路行營都監。至鎮戎,聞靈武已陷,復還本任。與張凝入西夏境,出白豹鎮,至柔遠川,
 夏人七百餘邀戰,煦與慶州監軍張綸擊殺甚眾。清遠故城有酋長,請以甲騎三萬來降。煦與凝曰:「此詐也。」亟嚴兵以待之,果然。凝按部歸環州,道為敵所邀。煦聞之,領所部銳兵自慶州赴之,一昔與凝會,射殺其大將,與凝同還。



 景德元年,加領賀州刺史,復為涇原儀渭鎮戎軍鈐轄,再知環州。四年,宜州戍卒陳進反,命副曹利用為廣東西路安撫使。賊眾擁判官宜州盧均,僭號南平王,圍象州,煦以兵會利用斬之。初與利用同署紙,人持
 百枚,備給立功將士。及破賊,利用在前軍無所給,煦在後而所給過半,真宗謂其太過。賊平,改如京使,知懷州。



 東封歲,權河陽鈐轄,遷文思使、知曹州。會江、淮災歉,分命大藩長吏綏撫,以煦為江南西路安撫都監。俄還濟陰,加北作坊使,又徙滄州,就轉宮苑使,領康州刺史。大中祥符九年,加領昭州團練使、知鄜州。未幾,復知滄州。天禧三年,拜西上閣門使,徙並代鈐轄。以老疾求近郡,得知磁州。四年,卒,年七十三。煦明術數,善相宅,時稱其
 妙。



 張佶,字仲雅,本燕人,後徙華州渭南。初名志言,後改焉。父昉,殿中少監。佶少有志節,始用蔭補殿前承旨,以習儒業,獻文求試,換國子監丞。遷著作佐郎、監三白渠、知涇陽縣。端拱初,為太子右贊善大夫。曹州民有被誣殺人者,詔往按之,發擿奸伏,冤人得雪。尋通判忻州,遷殿中丞,兼御河督運。



 至道中,通判陜州,再部送芻糧赴靈武,就改國子博士。咸平初,擢為陜西轉運副使,賜緋魚。
 至延安,遇夏人入寇,親督兵擊敗之。三年,徙西川轉運副使。時詔討王均,以饋餉之勞,遷虞部員外郎。賊平,分川峽為四路,以佶為利州路轉運使。有薦其武幹者,召還,授如京使、涇原鈐轄兼知鎮戎軍。徙麟府路鈐轄,夏人來寇,佶率兵與戰,親射殺酋帥,俘獲甚眾,餘黨遁去。詔書褒之,賜錦袍、金帶。景德中,徙益州鈐轄,加宜州刺史,遷文思使。佶御軍撫民,甚有威惠,蜀人久猶懷之。



 大中祥符四年,車駕祀汾陰,以為西京舊城巡檢、鈐轄。禮
 成,加授北作坊使,充趙德明官告使。又為鄜延鈐轄。會秦州李浚暴卒,上語近臣曰:「天水邊要,宜速得人。」馬知節稱佶可任,上然之,遂改左騏驥使,就命知秦州。至州,置四門砦,開拓疆境,邊部頗怨。又臨渭置採木場,戎人不之爭,移帳而去。佶不甚存撫,亦不奏加賚賜,邊人追悔,引眾劫掠,佶深入掩擊,敗走之。議者又欲加恩宗哥、立遵等族,以扼平夏,佶請拒絕之,事具《吐蕃傳》。朝廷始務寧邊,以佶輕信易事,徙邠寧路鈐轄。天禧初,召為契丹
 國信副使,再任邠寧,兼知邠州,遷宮苑使。未逾月,擢拜西上閣門使,復為涇原鈐轄。四年,卒,年六十九。



 佶涉獵書史,好吟詠,勇敢善射,有方略,其總戎護塞,以威名自任。子宗象,兵部員外郎、直史館、度支判官。



 論曰:自古盛德之世,未嘗無邊圉之患,要在得果毅之臣以捍禦之。昔人有言「誰能去兵」,漢祖亦云「安得猛士」,蓋為此也。李順叛蜀,攻陷郡邑,正捍劍門,斌守梓潼,其績最多。契丹入寇,審玉、繼宣,拔陷將於重圍之中,固有
 餘勇,佶、煦宣力西南,勤幹威惠,亦皆可取。濟、旦以孤城捍強寇,援絕戰死,一代死事之表表者,其可泯諸。



\end{pinyinscope}