\article{列傳第一百九十二儒林三}

\begin{pinyinscope}

 ○邵伯溫喻樗洪興祖高閌程大昌林之奇林光朝楊萬里



 邵伯溫,字子文,洛陽人,康節處士雍之子也。雍名重一
 時,如司馬光、韓維、呂公著、程頤兄弟皆交其門。伯溫入聞父教,出則事司馬光等,而光等亦屈名位輩行,與伯溫為再世交,故所聞日博,而尤熟當世之務。光入相,嘗欲薦伯溫,未果而薨。後以河南尹與部使者薦,特授大名府助教,調潞州長子縣尉。



 初,蔡確之相也,神宗崩,哲宗立,邢恕自襄州移河陽,詣確謀造定策事。及司馬光子康詣闕,恕召康詣河陽,伯溫謂康曰:「公休除喪未見君,不宜枉道先見朋友。」康曰:「已諾之。」伯溫曰:「恕傾巧,或
 以事要公休,若從之,必為異日之悔。」康竟往。恕果勸康作書稱確,以為他日全身保家計。康、恕同年登科,恕又出光門下,康遂作書如恕言。恕蓋以康為光子,言確有定策功,世必見信。既而梁燾以諫議召,恕亦要燾至河陽,連日夜論確功不休,且以康書為證,燾不悅。會吳處厚奏確詩謗朝政,燾與劉安世共請誅確,且論恕罪,亦命康分折,康始悔之。康卒,子植幼。宣仁後憫之。呂大防謂康素以伯溫可托,請以伯溫為西京教授以教植。伯
 溫既至官,則誨植曰:「溫公之孫,大諫之子,賢愚在天下,可畏也。」植聞之,力學不懈,卒有立。



 紹聖初,章惇為相。惇嘗事康節,欲用伯溫,伯溫不往。會法當赴吏部銓,程頤為伯溫曰:「吾危子之行也。」伯溫曰:「豈不欲見先公於地下耶?」至則先就部擬官,而後見宰相。惇論及康節之學,曰:「嗟乎,吾於先生不能卒業也。」伯溫曰:「先君先天之學,論天地萬物未有不盡者。其信也,則人之仇怨反覆者可忘矣。」時惇方興黨獄,故以是動之。惇悚然。猶薦之於
 朝,而伯溫願補郡縣吏,惇不悅,遂得監永興軍鑄錢監。時元祐諸賢方南遷,士鮮訪之者。伯溫見範祖禹於咸平,見范純仁於潁昌,或為之恐,不顧也。會西邊用兵,復夏人故地,從軍者得累數階,伯溫當行,輒推同列。秩滿,惇猶在相位。伯溫義不至京師,從外臺闢環慶路帥幕,實避惇也。



 徽宗即位,以日食求言。伯溫上書累數千言,大要欲復祖宗制度,辨宣仁誣謗,解元祐黨錮,分君子小人,戒勞民用兵,語極懇至。宣仁太后之謗,伯溫既辨
 之,又著書名《辨誣》。後崇寧、大觀間,以元符上書人分邪正等,伯溫在邪等中,以此書也。



 出監華州西嶽廟,久之,知陜州靈寶縣,徙芮城縣。丁母憂,服除,主管永興軍耀州三白渠公事。童貫為宣撫使,士大夫爭出其門,伯溫聞其來,出他州避之。除知果州,請罷歲輸瀘南諸州綾絹、絲綿數十萬以寬民力。除知興元府、遂寧府、邠州,皆不赴。擢提點成都路刑獄,賊史斌破武休,入漢、利,窺劍門,伯溫與成都帥臣盧法原合謀守劍門,賊竟不能入,
 蜀人德之。除利路轉運副使,提舉太平觀。紹興四年,卒,年七十八。初,邵雍嘗曰:「世行亂,蜀安,可避居。」及宣和末,伯溫載家使蜀,故免於難。



 伯溫嘗論元祐、紹聖之政曰:「公卿大夫,當知國體,以蔡確奸邪,投之死地,何足惜!然嘗為宰相,當以宰相待之。範忠宣有文正餘風,知國體者也,故欲薄確之罪,言既不用,退而行確詞命,然後求去,君子長者仁人用心也。確死南荒,豈獨有傷國體哉!劉摯、梁燾、王巖叟、劉安世忠直有餘,然疾惡已甚,不知
 國體,以貽後日縉紳之禍,不能無過也。」



 趙鼎少從伯溫游,及當相,乞行追錄,始贈秘閣修撰。嘗表伯溫之墓曰:「以學行起元祐,以名節居紹聖,以言廢於崇寧。」世以此三語盡伯溫出處云。



 著書有《河南集》、《聞見錄》、《皇極系述》、《辨誣》、《辨惑》、《皇極經世序》、《觀物內外篇解》近百卷。三子:溥、博、傅。



 喻樗,字子才,其先南昌人。初,俞藥仕梁,位至安州刺史,武帝賜姓喻,後徙嚴,樗其十六世孫也。少慕伊、洛之學,
 中建炎三年進士第,為人質直好議論。趙鼎去樞筦,居常山,樗往謁,因諷之曰:「公之事上,當使啟沃多而施行少。啟沃之際,當使誠意多而語言少。」鼎奇之,引為上客。鼎都督川陜、荊襄,闢樗為屬。



 紹興初,高宗親征,樗見鼎曰:「六龍臨江,兵氣百倍,然公自度此舉,果出萬全乎?或姑試一擲也?」鼎曰:「中國累年退避不振,敵情益驕,義不可更屈,故贊上行耳。若事之濟否,則非鼎所知也。」樗曰:「然則當思歸路,毋以賊遺君父憂。」鼎曰:「策安出?」樗曰:「張
 德遠有重望,居閩。今莫若使其為江、淮、荊、浙、福建等路宣撫使,俾以諸道兵赴闕,命下之日,府庫軍旅錢穀皆得專之。宣撫來路,即朝廷歸路也。」鼎曰:「諾。」於是入奏曰:「今沿江經畫大計略定,非得大臣相應援不可。如張浚人才,陛下終棄之乎?」帝曰:「朕用之。」遂起浚知樞密院事。浚至,執鼎手曰:「此行舉措皆合人心。」鼎笑曰:「子才之功也。」樗於是往來鼎、浚間,多所裨益。頃之,以鼎薦,授秘書省正字兼史館校勘。



 初,金既退師,鼎、浚相得歡甚。人知
 其將並相,樗獨言:「二人宜且同在樞府,他日趙退則張繼之。立事任人,未甚相遠,則氣脈長。若同處相位,萬有一不合,或當去位,則必更張,是賢者自相背戾矣。」後稍如其言。又嘗曰:「推車者遇艱險則相詬病,及車之止也,則欣然如初。士之於國家亦若是而已。」



 先是,樗與張九成皆言和議非便。秦檜既主和,言者希旨,劾樗與九成謗訕。樗出知舒州懷寧縣,通判衡州,已而致仕。檜死,復起為大宗正丞,轉工部員外郎、出知蘄州。孝宗即位,用
 為提舉浙東常平,以治績聞。淳熙七年,卒。



 初,樗善鑒識,宣和間,謂其友人沈晦試進士當第一。建炎初,又謂今歲進士張九成當第一,凌景夏次之。會風折大槐,樗以作二簡遺之,後果然。趙鼎嘗以樊光遠免舉事訪樗,樗曰:「今年省試不可無此人。」於是光遠亦第一。初,樗二女方擇配,富人交請婚,不許。及見汪洋、張孝祥,乃曰「佳婿也。」遂以妻之。



 洪興祖,字慶善,鎮江丹陽人。少讀《禮》至《中庸》,頓悟性命
 之理,績文日進。登政和上舍第,為湖州士曹,改宣教郎。高宗時在揚州,庶事草創,選人改秩軍頭司引見,自興祖始。召試,授秘書省正字,後為太常博士。



 上疏乞收人心,納謀策,安民情,壯國威。又論國家再造,一宜以藝祖為法。紹興四年,蘇、湖地震。興祖時為駕部郎官,應詔上疏,具言朝廷紀綱之失,為時宰所惡,主管太平觀。



 起知廣德軍,視水原為陂塘六百餘所,民無旱憂。一新學舍,因定從祀:自十哲曾子而下七十有一人,又列先儒左
 丘明而下二十有六人。擢提點江東刑獄。知真州。州當兵沖,瘡痍未瘳。興祖始至,請復一年租,從之。明年再請,又從之。自是流民復業,墾闢荒田至七萬餘畝。



 徙知饒州,先夢持六刀,覺曰:「三刀為益,今倍之,其饒乎?」已而果然。是時秦檜當國,諫官多檜門下,爭彈劾以媚檜。興祖坐嘗作故龍圖閣學士程瑀《論語解序》,語涉怨望,編管昭州,卒,年六十有六。明年,詔復其官,直敷文閣。



 興祖好古博學,自少至老,未嘗一日去書。著《老莊本旨》、《周易通
 義》、《系辭要旨》、《古文孝經序贊》、《離騷楚詞考異》行於世。



 高閌,字抑崇,明州鄞縣人。紹興元年,以上舍選賜進士第。執政薦之,召為秘書省正字。時將賜新進士《儒行》、《中庸》篇,閌奏《儒行》詞說不醇,請止賜《中庸》,庶幾學者得知聖學淵源,而不惑於他說,從之。



 權禮部員外郎兼史館校勘。面對,言:「《春秋》之法,莫大於正名。今樞密院號本兵柄,而諸路軍馬盡屬都督,是朝廷兵柄自分為二。又周六卿,其大事則從其長,小事官屬猶得專達。今一切拘
 以文法,雖利害灼然可見,官長且不敢自決,必請於朝,故廟堂之事益繁,而省曹官屬乃與胥吏無異。又政事之行,給、舍得繳駁,臺諫得論列,若給、舍以為然,臺諫以為不然,則不容不改。祖宗時有繳駁臺諫章疏不以為嫌者,恐其得於風聞,致朝廷之有過舉。然此風不見久矣,臣恐朝廷之權反在臺諫。且祖宗時,監察御史許言事,靖康中嘗行之。今則名為臺官,實無言責,此皆名之未正也。」



 尋遷著作佐郎,以言者論罷,主管崇道觀。召為
 國子司業。時興太學,閌奏宜先經術,帝曰:「士習詩賦已久,遽能使之通經乎?」閌曰:「先王設太學,惟講經術而已。國初猶循唐制用詩賦,神宗始以經術造士,遂罷詩賦,又慮不足以盡人才,乃設詞學一科。今宜以經義為主,而加詩賦。」帝然之。閌於是條具以聞。其法以《六經》、《語》、《孟》義為一場,詩賦次之,子史論又次之,時務策又次之。太學課試及郡國科舉,盡以此為法,且立郡國士補國學監生之制。中興已後學制,多閌所建明。



 閌又言建學之
 始,宜得老成以誘掖後進。乃薦全州文學師維藩,詔除國子錄。維藩,眉山人,精《春秋》學,林慄其高第也,故首薦之。新學成,閌奏補試者六千人,且乞臨雍,繼率諸生上表以請。於是帝幸太學,秦熺執經,閌講《易。泰卦》,賜三品服。胡寅聞之,以書責閌曰:「閣下為師儒之首,不能建大論,明天人之理,乃阿諛柄臣,希合風旨,求舉太平之典,欺天罔人孰甚焉!平生志行掃地矣。」



 閌少宗程頤學。宣和末,楊時為祭酒,閌為諸生。胡安國至京師,訪士於時,
 以閌為首稱,由是知名。閌除禮部侍郎,帝因問閌張九成安否,明日,復以問秦檜,檜疑閌薦,中丞李文會承檜旨劾閌,出知筠州,不赴,卒。初,秦棣嘗使姚孚請婚,閌辭之。其著述有《春秋集傳》行於世。



 程大昌,字泰之,徽州休寧人。十歲能屬文,登紹興二十一年進士第。主吳縣簿,未上,丁父憂。服除,著十論言當世事,獻於朝,宰相湯思退奇之,擢太平州教授。明年,召為太學正,試館職,為秘書省正字。



 孝宗即位,遷著作佐
 郎。當是時,帝初政,銳意事功,命令四出,貴近或預密議。會詔百官言事,大昌奏曰:「漢石顯知元帝信己,先請夜開宮門之詔。他日,故夜還,稱詔啟關,或言矯制,帝笑以前詔示之。自是顯真矯制,人不復言。國朝命令必由三省,防此弊也。請自今被御前直降文書,皆申省審奏乃得行,以合祖宗之規,以防石顯之奸。」又言:「去歲完顏亮入寇,無一士死守,而兵將至今策勛未已。惟李寶捷膠西,虞允文戰採石,實屠亮之階。今寶罷兵,允文守夔,此
 公論所謂不平也。」帝稱善,選為恭王府贊讀。遷國子司業兼權禮部侍郎、直學士院。帝問大昌曰:「朕治道不進,奈何?」大昌對曰:「陛下勤儉過古帝王,自女真通和,知尊中國,不可謂無效。但當求賢納諫,修政事,則大有為之業在其中,不必他求奇策,以幸速成。」又言:「淮上築城太多,緩急何人可守?設險莫如練卒,練卒莫如擇將。」帝稱善。



 除浙東提點刑獄。會歲豐,酒稅逾額,有挾朝命請增額者,大昌力拒之,曰:「大昌寧罪去,不可增也。」徙江西轉
 運副使,大昌曰:「可以興利去害,行吾志矣。」會歲歉,出錢十餘萬緡,代輸吉、贛、臨江、南安夏稅折帛。清江縣舊有破坑、桐塘二堰,以捍江護田及民居,地幾二千頃。後堰壞,歲罹水患且四十年,大昌力復其舊。



 進秘閣修撰,召為秘書少監,帝勞之曰:「卿,朕所簡記。監司若人人如卿,朕何憂?」兼中書舍人。六和塔寺僧以鎮潮為功,求內降給賜所置田產仍免科徭,大昌奏:「僧寺既違法置田,又移科徭於民,奈何許之!況自修塔之後,潮果不嚙岸乎?」寢
 其命。權刑部侍郎,升侍講兼國子祭酒。大昌言:「闢以止闢,未聞縱有罪為仁也。今四方讞獄例擬貸死,臣謂有司當守法,人主察其可貸則貸之。如此,則法伸乎下,仁歸乎上矣。」帝以為然。兼給事中。江陵都統制率逢原縱部曲毆百姓,守帥辛棄疾以言狀徙帥江西。大昌因極論「自此屯戍州郡,不可為矣」!逢原由是坐削兩官,降本軍副將。累遷權吏部尚書。言:「今日諸軍,西北舊人日少,其子孫伉健者,當教之戰陣。不宜輕聽離軍。且禁衛之
 士,祖宗非獨以備宿衛而已,南征北伐,是嘗為先鋒。今率三年輒補外,用違其長,既有征行,無人在選。奈何始以材武擇之,而終以庸常棄之乎?願留三衙勿遣。」



 會行中外更迭之制,力請郡,遂出知泉州。汀州賊沈師作亂,戍將蕭統領與戰死,閩部大震。漕檄統制裴師武討之。師武以未得帥符不行,大昌手書趣之曰:「事急矣,有如帥責君,可持吾書自解。」當是時,賊謀攻城,而先使諜者衷甲縱火為內應。會師武軍至,復得諜者,賊遂散去。遷
 知建寧府。光宗嗣位,徙知明州,尋奉祠。紹熙五年,請老,以龍圖閣學士致仕。慶元元年卒,年七十三,謚文簡。



 大昌篤學,於古今事靡不考究。有《禹貢論》、《易原》、《雍錄》、《易老通言》、《考古編》、《演繁露》、《北邊備對》行於世。



 林之奇,字少穎,福州候官人。紫微舍人呂本中入閩,之奇甫冠,從本中學。時將試禮部,行次衢州,以不得事親而反。學益力,本中奇之,由是學者踵至。中紹興二十一年進士第,調莆田簿,改尉長汀,召為秘書省正字,轉校
 書郎。



 會朝廷欲令學者參用王安石《三經義》之說,之奇上言:「王氏三經,率為新法地。晉人以王、何清談之罪,深於桀、紂。本朝靖康禍亂,考其端倪,王氏實負王、何之責。在孔、孟書,正所謂邪說、詖行、淫辭之不可訓者。」或傳金人欲南侵,之奇作書抵當路,以為「久和畏戰,人情之常。金知吾重於和,故常以虛聲喝我,而示我以欲戰之意,非果欲戰,所以堅吾和。欲與之和,宜無憚於戰,則其權在我」。又言:「戰之所須不一,而人才為先。必求可與共患
 難者,非得如龐士元所謂俊傑者不可也。」



 以痺疾乞外,由宗正丞提舉閩舶,參帥議,遂以祠祿家居,自稱拙齋。東萊呂祖謙嘗受學焉。淳熙三年卒,年六十有五。



 有《書》《春秋》《周禮說》、《論》、《孟》《楊子講義》、《道山記聞》等書行於世。



 林光朝,字謙之,興化軍莆田人。再試禮部不第,聞吳中陸子正嘗從尹焞學,因往從之游。自是專心聖賢踐履之學,通《六經》,貫百氏,言動必以禮,四方來學者亡慮數百人。南渡後,以伊、洛之學倡東南者,自光朝始。然未嘗
 著書,惟口授學者,使之心通理解。嘗曰:「道之全體,全乎太虛。《六經》既發明之,後世注解固已支離,若復增加,道愈遠矣。」



 孝宗隆興元年,光朝年五十,以進士及第。調袁州司戶參軍。乾道三年,龍大淵、曾覿以潛邸恩幸進,臺諫、給舍論駁不行。張闡自外召為執政,銳欲去之,覺其不可拙,遂以老疾力辭不拜。而光朝及劉朔方以名儒薦對,頗及二人罪,由是光朝改左承奉郎、知永福縣。而大臣論薦不已,召試館職,為秘書省正字兼國史編修、
 實錄檢討官,歷著作佐郎兼禮部郎官。八年,進國子司業兼太子侍讀,史職如故。是時,張說再除簽書樞密院事,光朝不往賀,遂出為廣西提點刑獄,移廣東。



 茶寇自荊、湘剽江西,薄嶺南,其鋒銳甚。光朝自將郡兵,檄摧鋒統制路海、本路鈐轄黃進各以軍分控要害。會有詔徙光朝轉運副使,光朝謂賊勢方張,留屯不去,督二將遮擊,連敗之,賊驚懼宵遁。帝聞之,喜曰:「林光朝儒生,乃知兵耶。」加直寶謨閣,召拜國子祭酒兼太子左諭德。四年,
 帝幸國子監,命講《中庸》,帝大稱善,面賜金紫。不數日,除中書舍人。是時,吏部郎謝廓然由曾覿薦,賜出身,除殿中侍御史,命從中出。光朝愕曰:「是輕臺諫、羞科目也。」立封還詞頭。天子度光朝決不奉詔,改授工部侍郎,不拜,遂以集英殿修撰出知婺州。光朝老儒,素有士望。在後省未有建明,或疑之,及聞繳駁廓然,士論始服。光朝因引疾提舉興國宮,卒,年六十五。



 楊萬里,字廷秀,吉州吉水人。中紹興二十四年進士第,
 為贛州司戶,調永州零陵丞。時張浚謫永,杜門謝客,萬里三往不得見,以書力請,始見之。浚勉以正心誠意之學,萬里服其教終身,乃名讀書之室曰誠齋。



 浚入相,薦之朝。除臨安府教授,未赴,丁父憂。改知隆興府奉新縣,戢追胥不入鄉,民逋賦者揭其名市中,民言雚趨之,賦不擾而足,縣以大治,會陳俊卿、虞允文為相,交薦之,召為國子博士。侍講張栻以論張說出守袁,萬里抗疏留栻,又遺允文書,以和同之說規之,栻雖不果留,而公論偉
 之。遷太常博士,尋升丞兼吏部侍右郎官,轉將作少監、出知漳州,改常州,尋提舉廣東常平茶鹽。盜沈師犯南粵,帥師往平之。孝宗稱之曰「仁者之勇」,遂有大用意,就除提點刑獄。請於潮、惠二州築外砦,潮以鎮賊之巢,惠以扼賊之路。俄以憂去。免喪,召為尚左郎官。



 淳熙十二年五月,以地震,應詔上書曰:



 臣聞:言有事於無事之時,不害其為忠;言無事於有事之時,其為奸也大矣。南北和好逾二十年,一旦絕使,敵情不測。而或者曰:彼有五
 單于爭立之禍。又曰:彼有匈奴困於東胡之禍,既而皆不驗。道塗相傳,繕汴京城池,開海州漕渠,又於河南、北簽民兵,增驛騎,制馬櫪,籍井泉,而吾之間諜不得以入,此何為者耶?臣所謂言有事於無事之時者一也。



 或謂金主北歸,可為中國之賀。臣以中國之憂,正在乎此。此人北歸,蓋懲創於逆亮之空國而南侵也。將欲南之,必固北之。或者以身填撫其北,而以其子與婿經營其南也。臣所謂言有事於無事之時者二也。



 臣竊聞論者或
 謂緩急,淮不可守,則棄淮而守江,是大不然。昔者吳與魏力爭而得合肥,然後吳始安。李煜失滁、揚二州,自此南唐始蹙。今曰棄淮而保江,既無淮矣,江可得而保乎?臣所謂言有事於無事之時者三也。



 今淮東、西凡十五郡,所謂守帥,不知陛下使宰相擇之乎,使樞廷擇之乎?使宰相擇之,宰相未必為樞廷慮也;使樞廷擇之,則除授不自己出也。一則不為之慮,一則不自己出,緩急敗事,則皆曰:非我也。陛下將責之誰乎?臣所謂言有事於
 無事之時者四也。



 且南北各有長技,若騎若射,北之長技也;若舟若步,南之長技也。今為北之計者,日繕治其海舟,而南之海舟則不聞繕治焉。或曰:吾舟素具也,或曰:舟雖未具而憚於擾也。紹興辛巳之戰,山東、採石之功,不以騎也,不以射也,不以步也,舟焉而已。當時之舟,今可復用乎?且夫斯民一日之擾,與社稷百世之安危,孰輕孰重?事固有大於擾者也。臣所謂言有事於無事之時者五也。



 陛下以今日為何等時耶?金人日逼,疆場
 日擾,而未聞防金人者何策,保疆場者何道?但聞某日修某禮文也,某日進某書史也,是以鄉飲理軍,以干羽解圍也。臣所謂言有事於無事之時者六也。



 臣聞古者人君,人不能悟之,則天地能悟之。今也國家之事,敵情不測如此,而君臣上下處之如太平無事之時,是人不能悟之矣。故上天見災異,異時熒惑犯南斗,邇日鎮星犯端門,熒惑守羽林。臣書生,不曉天文,未敢以為必然也。至於春正月日青無光,若有兩日相摩者,茲不曰大
 異乎?然天猶恐陛下不信也,至於春日載陽,復有雨雪殺物,茲不曰大異乎?然天猶恐陛下又不信也,乃五月庚寅,又有地震,茲又不曰大異乎?且夫天變在遠,臣子不敢奏也,不信可也;地震在外,州郡不敢聞也,不信可也。今也天變頻仍,地震輦轂,而君臣不聞警懼,朝廷不聞咨訪,人不能悟之,則天地能悟之。臣不知陛下於此悟乎,否乎?臣所謂言有事於無事之時者七也。



 自頻年以來,兩浙最近則先旱,江淮則又旱,湖廣則又旱,流徙
 者相續,道殣相枕。而常平之積,名存而實亡;入粟之令,上行而下慢。靜而無事,未知所以振救之;動而有事,將何以仰以為資耶?臣所謂言有事於無事之時者八也。



 古者足國裕民,惟食與貨。今之所謂錢者,富商、巨賈、閹宦、權貴皆盈室以藏之,至於百姓三軍之用,惟破楮券爾。萬一如唐涇原之師,因怒糲食,蹴而覆之,出不遜語,遂起朱泚之亂,可不為寒心哉!臣所謂言有事於無事之時者九也。



 古者立國必有可畏,非畏其國也,畏其人
 也。故苻堅欲圖晉,而王猛以為不可,謂謝安、桓沖江左之望,是存晉者,二人而已。異時名相如趙鼎、張浚,名將如岳飛、韓世忠,此金人所憚也。近時劉珙可用則早死,張栻可用則沮死,萬一有緩急,不知可以督諸軍者何人,可以當一面者何人,而金人之所素憚者又何人?而或者謂人之有才,用而後見。臣聞之《記》曰:「茍有車必見其式,茍有言必聞其聲。」今曰有其人而未聞其可將可相,是有車而無式,有言而無聲也。且夫用而後見,非臨
 之以大安危,試之以大勝負,則莫見其用也。平居無以知其人,必待大安危、大勝負而後見焉。成事幸矣,萬一敗事,悔何及耶?昔者謝玄之北禦苻堅,而郗超知其必勝;桓溫之西伐李勢,而劉倓知其必取。蓋玄於履屐之間無不當其任,溫於蒱博不必得則不為,二子於平居無事之日,蓋必有以察其小而後信其大也,豈必大用而後見哉?臣所謂言有事於無事之時者十也。



 願陛下超然遠覽,昭然遠寤。勿矜聖德之崇高,而增其所未能;
 勿恃中國之生聚,而嚴其所未備。勿以天地之變異為適然,而法宣王之懼災;勿以臣下之苦言為逆耳,而體太宗之導諫。勿以女謁近習之害政為細故,而監漢、唐季世致亂之由;勿以仇讎之包藏為無他,而懲宣、政晚年受禍之酷。責大臣以通知邊事軍務如富弼之請,勿以東西二府異其心;委大臣以薦進謀臣良將如蕭何所奇,勿以文武兩途而殊其轍,勿使賂宦者而得旄節如唐大歷之弊,勿使貨近幸而得招討如梁段凝之敗。
 以重蜀之心而重荊、襄,使東西形勢之相接;以保江之心而保兩淮,使表裏唇齒之相依。勿以海道為無虞,勿以大江為可恃。增屯聚糧,治艦扼險。君臣之所咨訪,朝夕之所講求,姑置不急之務,精專備敵之策。庶幾上可消於天變,下不墮於敵奸。



 然天下之事,有本根,有枝葉。臣前所陳,枝葉而已。所謂本根,則人主不可以自用。人主自用,則人臣不任責,然猶未害也。至於軍事,而猶曰「誰當憂此,吾當自憂」。今日之事,將無類此?《傳》曰:「木水有
 本原。」聖學高明,願益思其所以本原者。



 東宮講官闕,帝親擢萬里為侍讀。宮僚以得端人相賀。他日讀《陸宣公奏議》等書,皆隨事規警,太子深敬之。王淮為相,一日問曰:「宰相先務者何事?」曰:「人才。」又問:「孰為才?」即疏朱熹、袁樞以下六十人以獻,淮次第擢用之。歷樞密院檢詳,守右司郎中,遷左司郎中。



 十四年夏旱,萬里復應詔,言:「旱及兩月,然後求言,不曰遲乎?上自侍從,下止館職,不曰隘乎?今之所以旱者,以上澤不下流,下情不上達,故天
 地之氣隔絕而不通。」因疏四事以獻,言皆懇切。遷秘書少監。會高宗崩,孝宗欲行三年喪,創議事堂,命皇太子參決庶務。萬里上疏力諫,且上太子書,言:「天無二日,民無二王。一履危機,悔之何及?與其悔之而無及,孰若辭之而不居。願殿下三辭五辭,而必不居也。」太子悚然。高宗未葬,翰林學士洪邁不俟集議,配饗獨以呂頤浩等姓名上。萬里上疏詆之,力言張浚當預,且謂邁無異指鹿為馬。孝宗覽疏不悅,曰:「萬里以朕為何如主!」由是以
 直秘閣出知筠州。



 光宗即位,召為秘書監。入對,言:「天下有無形之禍,僭非權臣而僭於權臣,擾非盜賊而擾於盜賊,其惟朋黨之論乎!蓋欲激人主之怒莫如朋黨,空天下人才莫如朋黨。黨論一興,其端發於士大夫,其禍及於天下。前事已然,願陛下建皇極於聖心,公聽並觀,壞植散群,曰君子從而用之,曰小人從而廢之,皆勿問其某黨某黨也。」又論:「古之帝王,固有以知一己攬其權,不知臣下竊其權。大臣竊之則權在大臣,大將竊之則
 權在大將,外戚竊之則權在外戚,近習竊之則權在近習。竊權之最難防者,其惟近習乎!非敢公竊也,私竊之也。始於私竊,其終必至於公竊而後已。可不懼哉!」



 紹熙元年,借煥章閣學士為接伴金國賀正旦使兼實錄院檢討官。會《孝宗日歷》成,參知政事王藺以故事俾萬里序之,而宰臣屬之禮部郎官傅伯壽。萬里以失職力丐去,帝宣諭勉留。會進《孝宗聖政》,萬里當奉進,孝宗猶不悅,遂出為江東轉運副使,權總領淮西、江東軍馬錢糧。
 朝議欲行鐵錢於江南諸郡,萬里疏其不便,不奉詔,忤宰相意,改知贛州,不赴,乞祠,除秘閣修撰,提舉萬壽宮,自是不復出矣。



 寧宗嗣位,召赴行在,辭。升煥章閣待制、提舉興國宮。引年乞休致,進寶文閣待制致仕。嘉泰三年,詔進寶謨閣直學士,給賜衣帶。開禧元年召,復辭。明年,升寶謨閣學士,卒,年八十三,贈光祿大夫。



 萬里為人剛而褊。孝宗始愛其才,以問周必大,必大無善語,由此不見用。韓侂胄用事,欲網羅四方知名士相羽翼,嘗築
 南園,屬萬里為之記,許以掖垣。萬里曰:「官可棄,記不可作也。」侂胄恚,改命他人。臥家十五年,皆其柄國之日也。侂胄專僭日益甚,萬里憂憤,怏怏成疾。家人知其憂國也,凡邸吏之報時政者皆不以告。忽族子自外至,遽言侂胄用兵事。萬里慟哭失聲,亟呼紙書曰:「韓侂胄奸臣,專權無上,動兵殘民,謀危社稷,吾頭顱如許,報國無路,惟有孤憤!」又書十四言別妻子,筆落而逝。



 萬里精於詩,嘗著《易傳》行於世。光宗嘗為書「誠齋」二字,學者稱誠齋
 先生,賜謚文節。子長孺。



\end{pinyinscope}