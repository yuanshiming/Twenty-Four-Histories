\article{志第一百一十六 職官三}

\begin{pinyinscope}

 吏部戶部禮部兵部刑部工部六部監門六部架閣



 吏部掌文武官吏選試、擬注、資任、遷敘、蔭補、考課之政令,封爵、策勛、賞罰殿最之法。凡文階官之等三十,武
 選官之等五十有六,幕職州縣官之等七,散官之等九,皆以左右高下分屬於四選。曰尚書左選,文臣京朝官以上及職任非中書首除授者悉掌之。曰尚書右選,武臣升朝官以上及職任非樞密院除授者悉掌之。自初任至幕職州縣官,侍郎左選掌之。自副尉以上至從義郎,侍郎右選掌之。若文武官雖不隸左右選,而職任系中書省、樞密院除授者,其制命誥敕,皆本部奉行。凡應注擬、升移、敘復、蔭補及酬賞、封贈者,所隸審驗格法上尚書省,法例可否不決應取裁者,亦如之。若中散大夫、左右武大夫以上合命詞者,列其遷敘資級、歲月、功過上中書省、樞密院畫旨給告,通書
 本部長、貳及所隸郎官。其屬有曰司封,曰司勛,曰考功。凡官十有三:尚書一人;侍郎一人;郎中、員外郎,尚書選二人,侍郎選各
 一人,司封、司勛、考功各一人。



 舊制有三司,尚書主其一,侍郎二員各主其一,分銓注擬事。其後,但存尚書銓,餘東西銓印存而事廢。淳化中,又置考課院,磨勘幕府州縣功過,引對黜陟。至道二年,以其事歸流內銓。判流內銓事二人,以御
 史知雜以上充。掌節度判官以下州府判司、諸縣令佐擬注對揚、磨勘功過之事。判部事二人,以帶職京朝官或無職事朝官充。凡文吏班秩品命令一出於中書,而小選院即不復置,本曹但掌京朝官敘服章、申請攝官、訃吊祠祭,及幕府州縣官格式闕簿、辭謝,拔萃舉人兼南曹甲庫之事。流外銓,掌考試附奉諸司人吏而已。南曹掌考驗選人殿最成狀,而送流內銓關試、勾黃、給歷之事。甲庫掌受制敕黃,關給簽符優牒,選人改
 名廢置之事。初,淳化三年,置磨勘京朝官院。四年,改。太平興國中,置差遣院,至是並入審官院。置知院二人,以御史知雜以上充。



 舊以朝官充。



 掌考校京朝官殿最,敘其爵秩而詔於朝,分擬內外任使而奏之。



 元豐官制行,六曹尚書、侍郎為長貳,郎官理郡守以上資任者為郎中,通判以下資序者為員外郎。除授皆視寄祿官,高一品以上者為「行」,下一品者為「守」,下二品以下者為「試」,品同者不用行、守、試,餘職準此。元祐初,置權尚書,奉賜依守侍
 郎,班序在試尚書之下,雜壓在左、右常侍之下。又置權侍郎,如未歷給事中、中書舍人及待制以上者,並帶「權」字,祿賜比諫議大夫。郎官雖理知州資序,未曾實歷知州及監司、開封府推官者,止除員外郎。又詔,職事官除去「行」字一等。又以六曹職事閑劇不等,減定員數,事簡者他司兼領,司封、司勛各減郎官一員。紹聖初,詔元豐法以行、守、試制祿三等。元符元年,吏部言:「元祐法,小使臣只降宣扎,但務從簡,於理未安,請自借職而上依元
 豐法給告。」從之。崇寧元年,詔:「大宗正丞,大理正,諸寺監丞,太學、武學、律學博士,太學正、錄,諸宮院、諸州教授,堂除外,其吏部闕不許占差已授未赴及初到任人。」二年,詔:「十年不到部者,依《長定格》與降一官;二十年以上,則除其籍。」靖康元年七月,詔以吏部四選逐曹條例編集板行。八月,臣僚言:「祖宗時未有宗室參部之法,神宗時,始選擇差注一二。崇寧初,立法大優,宗室參選之日在本部名次之上,既壓年月深遠勞效顯著之人,復著名
 州大郡優便豐厚之處。議者頗欲懲革,不注郡守縣令,與在部人通理名次。」從之。



 尚書掌文武二選之法而奉行其制命。凡序位有品,寓祿有階,列爵有等,賜勛有給,分任有職,選官有格,考其功過,計其歲月,辨其位秩,而以序進之。凡文臣自京朝官,武臣自大使臣以上,舊內殿崇班以上。



 選授、封爵、功賞、課最之事,所隸官分掌其事,兼總於尚書,驗實而後判成。以天下職事員闕具注于籍,月取其應選者揭而書之,集
 官注擬,考閥閱以定其可否。若有疑不能決,小事則申請,大事則稟議於尚書省,應論奏者與郎官同請對。大祭祀則奉玉幣以授左僕射,執爵以授左丞。舊,尚書為所遷官名,班左丞上。自厘正百司,吏部以金紫光祿大夫,戶、禮、兵、刑、工部以銀青光祿大夫換授,而任六曹尚書者始實領職事。左選分案八,置吏三十;右選分案六,置吏十有六。曰主事、令史,曰書令史,曰守當官。二十四司亦如之。南渡初,諸曹長、貳互置,惟吏部備官。紹興八
 年,依元祐制,六曹皆置權尚書,以處未應資格之人。其屬有侍郎二人,分左、右選。尚書左、右選各置郎中一人,侍郎左、右選各置郎中一人,司封、司勛、考功各一人。郎官分掌其事,而兼總於尚書。左選,掌考校京朝官以上殿最,敘其爵秩,擬內外任使而奏授之。分案十二:曰六品,曰七品,曰八品,曰九品,曰注擬,曰名籍,曰掌闕,曰催驅,曰甲庫,曰檢法,曰知雜,曰奏薦賞功司。吏額,主事一人,令史二人,書令史九人,守當官一十一人,正貼司一
 十六人,私名一十二人,楷書二人,法司一人。官告院六部監門隸焉。右選,掌大使臣以上差注,材武人有格二十一,及破格出闕,較量功過,奏薦諸軍賞功。分案十:曰大夫,曰副使,曰修武,曰注擬掌闕,曰奏薦賞功,曰開拆,曰名籍,曰甲庫,曰法司,曰知雜。吏額,主事一人,令史二人,書令史九人,守當官一十二人,正貼司八人,私名一十人,法司一人。紹熙三年,左司諫謝源明言:「乾道九年詔旨:『六部應承三省、密院批送勘當文字,並令本部郎
 官、長貳按法裁決可否,申上朝廷施行。』即不得持兩端。如或事有疑難,及生創無條例者,令長貳據所見申明將上取旨。乞明詔六曹遵守。」從之。



 侍郎分左右選:左選,掌文臣之未改官者。凡始命而未應參部者,皆試而後選。若應格,則具歲月歷任功罪及所舉官員數,同郎官引見於便殿,稟奏改官。右選,掌武臣之未升朝者。



 舊自供奉官以上。



 其職任自親民官至部隊將、監當官,皆掌其選授注擬之法。凡初仕而試不中等,及
 已入官而未應選者,皆勿注正闕。官制行,尚書、侍郎通治曹事,奏事則同班,惟吏部分領四選。大祭祀則舉玉幣置諸案,薦饌則進搏黍,進熟則執匏爵以授右丞,飲福則奉爵,視朝則執文武班簿對立,以待顧問。左選分案十五,置吏四十有三,右選分案八,置吏四十有七。



 紹興四年,吏部侍郎葉祖洽言:「侍郎左選,準元豐朝旨,類姓置簿。左右選理宜一體,右選亦乞置簿拘轄功過。」從之。建炎四年五月,詔六曹復置權侍郎,如元祐故事,滿二年為真。



 補外者除待制,未滿,除修撰。



 左選,掌承直郎以下擬注州府
 判司、諸縣令佐、監當及磨勘功過之事,分案十三。乾道裁減吏額,共置三十五人。右選,掌校副尉以上較試、擬官、行賞、換官,考其殿最,分案十五。乾道裁減吏額,共置四十八人。舊制,吏部除侍郎二員,分典左、右選,總稱吏部侍郎。間命官兼攝,惟稱左選侍郎或右選而已。紹熙三年,謝深甫、張叔椿兼攝,始有侍左侍郎、侍右侍郎之稱。既而林大中、沉揆擢貳尚書,則「侍左」「侍右」徑入除目。相承不改。



 郎中員外郎尚左尚右侍左侍右



 舊主判二人,以朝官充。元豐官制行,置吏部郎中,主管尚書左、右選及侍郎左、右選各一員,參掌選事而分治之。凡郎官,並用知府資序以上人充,未及者為員外郎。建炎四年,詔權攝、添差郎官並罷。初進擬,第云吏部郎官;及擬告身細御,始直書尚書吏部郎中或員外郎,主管尚書某選,主管侍郎某選。紹興八年,呂希常以監六部門兼權侍右郎官。紹興三十一年,李端明正除尚右郎官,既而何輔、楊倓、費
 行之除吏部郎官,皆有侍左、侍右、尚左、尚右之稱。自此相承不改。淳熙十六年,光宗即位,詔四選通差,用尚書顏師魯之請也。先是,乾道元年詔:「今後非曾任監司、守臣,不除郎官,著為令。」自是館學、寺監臣,拘礙資格,遷除不行。郎曹闕員,但得兼攝,旋即外補;間有不次擢用者,則自二著躐升二史,以至從列。其自外召至為郎,則資級已高,曾不數月,必序進卿、少,而郎有正員者益少矣。



 司封郎中員外郎掌官封、敘贈、承襲之事。凡三師、
 三公以下至升朝官褒贈祖考、母妻,親王、郡王、內外命婦以下保任宗屬、封爵諸親,皆因其位敘而為之等。凡宗室當賜名訓,具抄擬官。凡庶姓孔氏、柴氏、折氏之後應承襲者,辨其嫡庶。列爵九等:曰王,曰郡王,曰國公,曰郡公,曰縣公,曰侯,曰伯,曰子,曰男。分國三等:大國二十七,次國二十,小國二百二十。內命婦之品五:曰貴妃、淑妃、德妃、賢妃,曰大儀、貴儀、淑儀、淑容、順儀、順容、婉儀、婉容、昭儀、昭容、昭媛、修儀、修容、修媛、充儀、充容、充媛,曰婕
 妤,曰美人,曰才人、貴人。外內命婦之號十有四:曰大長公主,曰長公主,曰公主,曰郡主,曰縣主,曰國夫人,曰郡夫人,曰淑人,曰碩人,曰令人,曰恭人,曰宜人,曰安人,曰孺人。敘贈之制:三公、宰臣、執政、節度使三代,金紫、銀青光祿大夫二代,餘官一代,皆辨其位序以進之。加食邑實封,則視其品之高下,以為戶數多寡之節。凡事之可否,與司勛通決於長貳。分案三,設吏六。元祐元年,中書後省言:「臣僚封贈父母,仍舊制命詞,太中大夫觀察使
 以上用專詞,餘用海詞。」二年,詔:「父及嫡母存,不得請所生母封贈。所生母未封,亦不許先及其妻。」紹聖元年,詔:「宗室換授文官身亡者,通直郎以上贈三官。」



 元符元年,以元祐間封贈紊前制,詔並依元豐法。



 二年,詔:「寺監官雜壓在通直郎之上者,雖系宣教郎,遇大禮封贈。」政和二年,詔:「封母則隨所封五等,謂如封南陽縣開國男,則隨其爵稱南陽縣男令人,封魏國公,則稱魏國公夫人之類。



 應婦人不因夫、子得封號,



 謂命官非升朝而母年九十以上,或士庶人婦女年百歲,並特旨若回授者。



 或因子孫得封贈,其夫至升朝或非升朝應封贈
 者,並孺人。」宣和二年,臣僚言:「近年有京官任校書郎、正字者得封贈,今則監丞未升朝者亦乞依例,蓋緣監丞雜壓在校書郎之上,故引以為請,甚無謂也。不獨此爾,又有小使臣偶因薄勞或磨勘轉官,遂乞回授封贈父母,實為太濫。望降旨,今後封贈並依舊法,敢有擅更陳乞紊亂典章者,置之典刑,庶幾僥幸者息而名分正矣。」從之。建炎以後並同。



 司勛郎中員外郎參掌勛賞之事。凡勛級十有二:曰
 上柱國,正二品;曰柱國,從二品;曰上護軍,正三品;曰護軍,從三品;曰上輕車都尉,正四品;曰輕車都尉,從四品;曰上騎都尉,正五品;曰騎都尉,從五品;曰驍騎尉,正六品;曰飛騎尉,從六品;曰雲騎尉,正七品;曰武騎尉,從七品。率三歲一遷,必因其除授以加之。凡賞有格。若事應賞,從其所隸之司考實以報,則必審核其狀,以格覆之,謂之「有法酬賞」;非格所載,參酌輕重擬定,以上尚書省,謂之「無法酬賞」。若功賞未醉而賞格改易者,輕從舊格,
 重從新格。錄用前代帝系及勛臣之後,則考其族系而奉行其制命。分案四,置吏十有九。



 元祐元年,吏部言:「諸色人援引徼求,入流太冗。應工匠伎藝之屬無法入官者,雖有勞績,並止比類支賜,未經酬獎者亦如之。」紹聖二年,戶部言:「元豐官制,司勛覆有法式酬賞,無法式者定之。元祐中,有法式者止令所屬勘驗,自後應幹錢穀,本部指定關司勛,則是戶部兼司勛之職,請依舊制。」從之。四年,應川峽人任本路差遣者,酬獎減半。政和四年,
 詔:「司勛行下所屬,將一司一路條制,參照《酬獎格法》,類集參用。」又詔以詳定國朝勛德臣僚職位姓名送吏部。用工部尚書鄭允中所編傳也。隆興元年省並,以司封郎官兼領。淳熙元年,復以司農寺丞範仲芭兼司勛,未幾改除,復省。裁減吏額,主事一人,令史一人,書令史四人,守當官三人,正貼司四人,私名三人。



 考功郎中員外郎掌文武官選敘、磨勘、資任、考課之政令。凡命官,隨所隸遷,以其職事具注于歷,給之於其
 屬州若司,歲書其功過。應升遷授者,驗歷按法而敘進之;有負殿,則正其罪罰。以七事考監司:一曰舉官當否,二曰勸課農桑、增墾田疇,三曰戶口增損,四曰興利除害,五曰事失案察,六曰較正刑獄,七曰盜賊多寡。以四善、三最考守令:德義有聞、清謹明著、公平可稱、恪勤匪懈為四善;獄訟無冤、催科不擾為治事之最,農桑墾殖水利興修為勸課之最,屏除奸盜、人獲安處、振恤困窮、不致流移為撫養之最。通善、最分三等:五事為上,二事
 為中,餘為下。若能否尤著,則別為優劣,以詔黜陟。凡內外官,計在官之日,滿一歲為一考,三考為一任。



 磨勘之法,文選官之等四:銀青光祿大夫至朝議大夫,進士理八年,非進士理十年;通直郎至太中大夫充諫議大夫、待制以上職任者,理三年;朝散大夫至承務郎,理四年。武選官之等六:遙郡團練使、刺史、合門舍人轉左武、右武郎,理十年;武功大夫以下,理七年;橫行武德大夫以下至校尉,理五年;合門祗候初補從義郎以下至承節
 郎、承信郎充隨行指使,理四年;承信郎以功補授及宗室觀察使以下祗應校尉,理三年;宗室承宣使以下祗應校尉,理二年。幕職州縣官之等三:進士第一、第二、第三名及第者,一任回改京官;自留守、府判官至縣令,理六考;自軍巡判官至縣尉,理七考。率以法計其歷任歲月、功過而序遷之。凡改服色者以年勞計之。執政官、節度使、銀青光祿大夫以上應縊者,覆太常所定行狀,報尚書省官集議以聞。



 紹聖四年,河東提刑司徐君平奏:「乞凡將集議,前期三日,持考功狀
 遍示當議之官,使先紬繹而後集於都堂以詢之,庶幾有所見者得以自申,以稱朝廷博謀盡下之意。」從之。



 凡立碑碣名額之事,掌之。舊制,考課院其定殿最皆有考辭。元豐官制行,悉罷。分案十有七,置吏六十有八。



 元祐三年,詔:「知州考課法,吏部上其事於尚書省,送中書省取旨賞罰。劣等應罰而已沖降者,仍從沖降法。縣令以下,本部專行。」六年,樞密院言:「元豐末,堂除知州軍三年為任,武任依此。元祐初,以成資為任,武臣未曾立法。」詔武臣任六等差遣,川廣成資餘並三十個月為任。建
 炎以後並同。應文武臣磨勘、關升、資任、較考,定其殿最,別其優劣,以詔黜陟予奪;沒則謚,審覆而參定之。凡特恩賜謚,命詞給告,餘給敕。分案十一:曰六品,曰七品,曰八品,曰曹掾,曰令丞,曰從義,曰成忠,曰資任,曰檢法,曰知雜,曰開拆。裁減吏額,主事二人,令史四人,書令史八人,守當官十三人,正貼司三人,私名一十人。



 淳熙十三年,再共減三人。



 官告院主管官一員,以京朝官充。舊制,提舉一人,以知制誥充;判院一人,以
 帶職京朝官充。



 掌吏、兵、勛、封官告,以給妃嬪、王公、文武品官、內外命婦及封贈者,各以本司告身印印之。文臣用吏部,武臣用兵部,王公及命婦用司封,加勛用司勛。官制行,四選皆用吏部印,惟蕃官則用兵部印記。凡綾紙幅數、褾軸名色,皆視其品之高下,應奏鈔畫聞者給之。令史十五人。



 元豐五年,官制所復位《制授敕授奏授告身式》,從之。紹聖元年,吏部言:「元豐法,凡入品者給告身,無品者給黃牒。元祐中,以內外差遣並職事官本等內改易
 或再任者,並給黃牒,乃與無品人等。」詔:「今後帥臣監司待制以上知州,並給告,餘依舊。」三年,詔:「職事官監察御史以上因事罷,並給告。」元符元年,吏部言:「元祐法,小使臣只降宣扎,乞自承信郎而上依舊給告。」宣和元年,詔:「官告院立條,凡制造告身法物,應用綾錦,私輒放效織造及貿販服用者,立賞許告。」



 大抵官告之制,自乾德四年,詔定告身綾紙褾軸,其制闕略。咸平、景德中,兩加潤澤,至皇祐始備。神宗即位,循用皇祐舊格,逮元豐改制,
 名號雖異,品秩則同,故亦未遑別定。徽宗大觀初,乃著為新格,凡褾帶綱軸等飾,始加詳矣。



 凡文武官綾紙五種,分十二等。



 色背銷金花綾紙二等。



 一等一十八張,滴粉縷金花大犀軸,八荅暈錦褾韜,色帶。三公、三少、侍中、中書令用之。一等一十七張,滴粉縷金花中犀軸,天下樂錦褾犀軸,色帶。左右僕射、使相、王用之。



 白背五色綾紙二等。



 一等一十七張,滴粉縷金花,翠毛獅子錦褾韜,玳瑁軸,色帶。知樞密院,兩省侍郎,尚書左右丞,同知、簽書樞密院事,嗣王,郡王,特進,觀文殿大學士,太尉,東宮三少,冀、袞、青、徐、揚、荊、豫、梁、雍州牧,御史大夫,宗室節度使至率府副率之帶皇字者用之。一等一十七張,暈錦褾韜,玳瑁軸,色帶。觀文殿學士,資政殿大學士,六尚書,金紫光祿、銀青光祿、光祿大夫,左、右金吾衛,左、右
 衛上將軍,節度、承宣、觀察,並用之。



 大綾紙四等。



 一等一十五張,暈錦褾,兩面撥花穗草大牙軸,色帶。宣奉、正奉大夫,翰林學士,資政、端明殿學士,龍圖、天章、寶文、顯謨、徽猷閣學士,左、右散騎常侍,御史中丞,開封尹,六曹侍郎,樞密直學士,龍圖天章、寶文、顯謨、徽猷閣直學士,正議、通奉大夫,諸衛上將軍,太子賓客,詹事,侯,用之。一等十二張,法錦褾,兩面撥花細牙軸,色帶。給事中,中書舍人,通議大夫,司成,左、右諫議大夫,龍圖、天章、寶文、顯謨、徽猷閣待制,太中大夫,秘書、殿中監,伯,用之。一等一十張,法錦褾,撥花常使大牙軸,色帶。中大夫,七寺卿,京畿、三路轉運使,發運使,中奉、中散大夫,通侍大夫,樞密都承旨,祭酒,太常、宗正少卿,秘書、殿中少監,正侍、中侍大夫,入內內侍省內侍省、都知,諸州刺史,中亮、中衛大夫,防禦、團練使,太子左、右庶子,諸衛大將軍,附馬都尉,典樂,子,用之。一等八張,盤球錦褾,大牙軸,色帶。七寺少卿,朝議、奉直大夫,左、右司郎中,司業,開封
 少尹,少府、將作、軍器監,都水使者,拱衛大夫,太子詹事,左、右諭德,左武、右武大夫,入內內侍省、內侍省副都知,樞密承旨、副都承旨,諸房副承旨,起居郎、舍人,侍御史,左、右司員外郎,六曹郎中,朝請、朝散、朝奉大夫,京畿、三路轉運副使,諸路轉運使、副使,知上州,提舉三路保甲,入內內侍省、內侍省押班,武功至武翼大夫,開封左、右司錄事,蕃官使臣,殿中侍御史,左右司諫、正言,監察御史,和安大夫至翰林良醫,男,用之。內殿中侍御史、監察御史用九張,蕃官使臣用大錦褾,背帶,此其小異者也。



 中綾紙二等。



 一等七張,中錦褾,中牙軸,青帶。諸司員外郎,朝請朝散、朝奉郎,少府、將作、軍器少監,諸衛將軍,太子侍讀、侍講,中亮、中衛,左武、右武郎中,知下州,諸路提點刑獄,發運判官,提點鑄錢,承議郎,武功至武翼郎,太子中允、舍人,親王府翊善、贊讀、侍讀,符寶郎,太常、中正、秘書、殿中丞,六尚奉御,大理正,著作郎,通事舍人,太子諸率府率,直龍圖閣,開封府諸曹事,大
 晟府樂令,直秘合,崇政殿說書,和安郎至翰林醫正,用之。一等六張,中錦褾,中牙軸,青帶。奉議郎,七寺丞,秘書郎,太常博士,著作佐郎,國子、少府、將作、軍器、都水監承,國子博士,大理司直評事,修武、敦武郎,通直郎,內常侍,轉運判官,提舉學士,諸州通判,御史臺檢法官、主簿,九寺主簿,親王記室,閣門祗候,樞密院逐房副承旨,從義、秉義郎,太學、武學博士,開封諸曹掾,陵臺令,兩赤縣令,忠訓、忠翊郎,節度、防禦、團練副使,行軍司馬,太醫正,太史局令、正、丞、五官正,翰林醫官,闢NU博士,太子諸率府副率,用之。



 小綾紙二等。



 一等五張,黃花錦褾,角軸,青帶。校書郎,正字,宣教郎,太常寺協律、奉禮郎,太祝,郊社、太官令,律學博士,國子、少府、將作、軍器、都水監主簿,宣幾郎,保義、成忠郎,太學正、錄,律學,承事、承奉、承務、承信、承、節郎,門下、中書省錄事,尚書省都事,三省、樞密院主事,闢NU正、錄,用之。一等五張,黃花錦褾,次等角軸,青帶。幕職、州縣官,三省樞密院令史、書史,流外
 官,諸州別駕、長史、司馬、文學、司士、助教,技術官,用之。



 凡宮掖至外命婦羅紙七種,分十等:



 遍地銷金龍五色羅紙二等。



 一等一十八張,韜帶,兩面銷金雲鳳褾,紅絲綱子,金樣鈒花塗□錔,滴粉縷金花鳳大犀軸。大長公主、長公主、公主用之。一等一十七張,韜帶,兩面銷金雲鳳褾,紅絲網子,金樣鈒花塗□錔,滴粉縷金花鳳子中犀軸。貴儀、淑儀、淑容、順儀、順容、婉儀、婉容、內宰用之。



 遍地銷金鳳子五色羅紙二等。



 一等一十五張,韜帶,銷金鳳子褾,紅絲綱子,金塗銀□錔,滴粉縷金雲鳳玳瑁軸。昭儀、昭容、昭媛、修儀、修容、修媛、充儀、充容、充媛、副宰用之。一等一十二張,韜帶,銷金盤鳳標,紅絲網子,金塗銀□錔,滴粉金雲鳳玳瑁軸。婕妤、才人、貴人、美人用之。



 銷金團窠花五色羅紙二等。



 一等一十張,八荅暈錦逯韜,色帶,紫絲網子,銀□錔,滴粉縷金葵花
 玳瑁褾軸。尚儀,尚服,尚食,尚寢,尚功,宮正,內史,宰相曾祖母、祖母、母、妻,親王妻,用之。一等八張,翠色獅子錦標韜,色帶,紫絲網子,銀□錔,滴粉縷金梔子花玳瑁軸。郡主,縣主,國夫人,內命婦,郡夫人,執政官祖母、母、妻,用之。



 銷金大花五色羅紙一等。



 七張,雲雁錦褾韜,色帶,紫絲網子,銀□錔,滴粉縷金玳瑁軸。寶林御女,採女,二十四司典掌,尚書省掌籍、掌樂,主管仙韶,用之。



 金花五色羅紙一等。



 七張,法錦褾韜,色帶,紫絲網子,銀□錔,縷金玳瑁軸。郡夫人,郡君,宗室妻,朝奉大夫、遙郡刺史以上母妻,升朝官母,諸班直都虞候、指揮使、禁軍都虞候、軍都虞候、御前忠佐母,蕃官母妻,諸神廟夫人,用之。



 五色素羅紙一等。



 七張,錦褾韜,色帶,紫絲綱子,銀□錔,大牙軸。宗室女,升朝官妻,諸班直都虞候、指揮使、禁軍都虞候、軍都指揮使、忠佐妻,用之。



 凡內外軍校封贈綾紙三種,分四等:



 大綾紙二等。



 一等七張,法錦褾,大牙軸,青帶。遙郡刺史以上用之。一等七張,大錦褾,大牙軸,青帶。藩方指揮使、御前忠佐馬步軍都副都軍頭、馬步軍都軍頭、藩方馬步軍都指揮使用之。內帶遙郡者,法錦褾,色帶。



 中陵紙一等。



 五張,中錦褾,中牙軸,青帶。都虞候以上諸班指揮使,御前忠佐馬步軍副都軍頭,藩方馬步軍副都指揮使、都虞候,用之。內加至爵邑者,用大綾紙,大牙軸,大錦褾。



 小綾紙一等。五張,黃花錦褾,次等角軸,青帶。諸軍指揮使以下用之。如加至爵邑者,同上。



 凡封蠻夷酋長及蕃長綾紙兩種,各一等:



 五色銷金花綾紙一等。一十八張,翠色獅子錦褾,法錦韜,紫絲網子,銀□錔,滴粉縷金牡丹花玳瑁軸,色帶。南平、占城、真臘、闍婆國王用之。



 中綾紙一等。七張,法錦褾,中牙軸,青帶。藩蠻官承襲、轉官用之。



 大觀並歸尚書省,政
 和仍歸吏部。差主管官。



 建炎元年,詔:「文臣太中大夫、武臣正任觀察使及宗室南班官以上給告,以下並給敕。」三年,詔逐等依舊給告。紹興二年,詔:「四品以下官及職事官監察御史以上,官告並用錦褾外,其餘官並封贈權用纈羅代充。」十四年,始盡用錦。其後,又詔內外命婦、郡夫人以上,乃得用網袋及銷金,其餘則否。至二十六年,詔內外文武臣僚告敕並依大觀格式制造。裁減吏額,共置二十九人。



 淳熙十三年又減五人。



 戶部國初,以天下財計歸之三司,本部無職掌,止置判部事一人,以兩制以上充,以受天下上貢,元會陳於庭。元豐正官名,始並歸戶部。掌天下人戶、土地、錢穀之政令,貢賦、征役之事。以版籍考戶口之登耗,以稅賦持軍國之歲計,以土貢辨郡縣之物宜,以征榷抑兼並而佐調度,以孝義婚姻繼嗣之道和人心,以田務券責之理直民訟,凡此歸於左曹。以常平之法平豐兇、時斂散,以免役之法通貧富、均財力,以伍保之法聯比閭、察資
 賊,以義倉振濟之法救饑饉、恤艱扼,以農田水利之政治荒廢、務稼穡,以坊場河渡之課酬勤勞、省科率,凡此歸於右曹。尚書置都拘轄司,總領內外財賦之數,凡錢穀帳籍,長貳選吏鉤考。其屬三:曰度支,曰金部,曰倉部。



 熙寧中,以知樞密院陳升之、參知政事王安石制置條例,建官設屬,取三司條例看詳,具所行事付之。三年,罷歸中書,以常平、免役、農田、水利新法歸司農,以冑案歸軍器監,修造歸將作監,推勘公事歸大理寺,帳司、理欠
 司歸比部,衙司歸都官,坑冶歸虞部,而三司之權始分矣。元豐官制行,罷三司歸戶部左、右曹,而三司之名始泯矣。凡官十有三:尚書一人,侍郎二人,郎中、員外郎,左右曹各二人,度支、金部、倉部各二人。



 元祐初,門下侍郎司馬光言:「天下錢穀之數,五曹各得支用,戶部不知出納見在,無以量入為出。乞令尚書兼領左、右曹,錢穀財用事有散在五曹、寺監者,並歸戶部,使尚書周知其數,則利權歸一;若選用得人,則天下之財庶幾可理。」詔尚書
 省立法。三年,三省言:、大理寺右治獄並罷,依三司舊例,戶部置推勘檢法官,治在京官司凡錢穀事,增置乾當公事二員。」紹聖元年,罷戶部乾當公事,置提舉、管幹官,復行免役、義倉,厘正左、右曹職,依元定官制。三年,右曹令侍郎專領,尚書不與。建中靖國元年,復乾當公事官二員。政和二年五月,詔依神宗官制,委右曹侍郎專主行常平,自今許本部直達奏裁。又詔依熙、豐舊制,本部置都拘轄司,總領戶、度、金、倉四部財賦。宣和六年,詔戶
 部闢官依元豐法。



 尚書侍郎掌軍國用度,以周知其出入盈虛之數。凡州縣廢置,戶口登耗,則稽其版籍;若貢賦征稅,斂散移用,則會其數而頒其政令焉。凡四司所治之事,侍郎為之貳,郎中、員外郎參領之,獨右曹事專隸所掌侍郎。若事屬本曹,郡縣監司不能直者,受其訟焉。大饗祀薦饌,則尚書奉俎,飲福則徹之。朝會則奏貢物。左曹分案五,置吏四十;右曹分案五,置吏五十有六。建炎兵興,嘗以
 知樞密院張愨提領措置戶部財用,後遷中書侍郎,仍兼之。五年,復以參知政事孟庾提領措置。後罷,專委戶部長貳。左曹分案三:曰戶口,掌凡諸路州縣戶口升降,民間立戶分財,科差人丁,典賣屋業,陳告戶絕,索取妻男之訟。曰農田,掌農田及田訟務限,奏豐稔,驗水旱蟲蝗,勸課農桑,請佃地土,令佐任滿賞罰,繳奏諸州雨雪,檢按災傷逃絕人戶。曰檢法,掌凡本部檢法之事,設科有三:曰二稅,掌受納、驅磨、隱匿、支移、折變。曰房地,掌諸
 州樓店務房廊課利,僧道免丁錢及土貢獻物。曰課利,掌諸軍酒課,比較增虧,知、通等職位姓名,人戶買撲鹽場酒務租額酒息,賣田投納牙契。外有開拆、知雜司。右曹分案六:曰常平,掌常平、農田水利及義倉振濟,戶絕田產,居養鰥、寡、孤、獨之事。曰免役,曰坊場,曰平準,各隨其名而任其事。曰檢法,曰知雜。裁減吏額,左曹四十人,右曹三十人。淳熙十年,詔左藏南庫撥隸戶部。舊制,戶部侍郎二人,中興初,止除長貳、各一員,或止除尚書若
 侍郎一員。紹興四年七月,詔戶部侍郎二員,通治左、右曹,自此相承不改。



 郎中左曹右曹



 員外郎掌分曹治事。建炎三年,詔省並郎曹,惟戶部五司以職事煩劇不並,仍各置一員。紹興中,專置提舉帳司,總天下帳狀,以戶部左曹郎官兼之。右曹歲具常平錢物總數,每秋季具冊以聞。初置主管左、右曹,總稱戶部郎官。紹興七年,閻彥昭以太府寺丞兼左曹郎官。紹興三十二年,徐康正除左曹郎官,自是相
 承不改。是年,又詔:「戶部事有可疑難裁決者,許長貳與眾郎官聚議,文字皆令連書,有定議,然後付本曹行遣。」



 度支郎中員外郎參掌計度軍國之用,量貢賦稅租之入以為出。凡軍須邊備,會其盈虛而通其有無。若中外祿賜及大禮賞給,皆前期以辨。歲終,則會諸路財用出入之數奏於上,而以其副申尚書省。凡小事則擬畫,大事諮其長貳;應申請更改舉行勘審者,則先檢詳供具。分案六,置吏五十有一。凡上供有額,封樁有數,科買
 有期,皆掌之。有所漕運,則計程而給其直。凡內外支供及奉給驛券,賞賜衣物錢帛,先期擬度,時而予之。分案五:曰度支,曰發運,曰支供,曰賞賜,曰知雜。乾道四年,置會稽都籍,度支掌之。裁減吏額,置五十人。淳熙十三年,又減四人。



 金部郎中員外郎參掌天下給納之泉幣,計其歲之所輸,歸於受藏之府,以待邦國之用。勾考平準、市舶、榷易、商稅、香茶、鹽礬之數,以周知其登耗,視歲額增虧而
 為之賞罰。凡綱運不濡滯及負折者,計程帳催理。凡造度、量、權、衡,則頒其法式。合同取索及奉給、時賜,審覆而供給之。分案六:曰左藏,日右藏,曰錢帛,曰榷易,曰請給,曰知雜。裁減吏額,共置六十人。淳熙十三年,又減四人。



 倉部郎中員外郎參掌國之倉庾儲積及其給受之事。凡諸路收糴折納,以時舉行;漕運上供封樁,以時催理;應供輸中都而有登耗,則比較以聞。歲以應用芻粟前期報度支,均定支移、折變之數。其在河北、陜西、河東
 路者,書其所支歲月,季一會之。若內外倉場帳籍供申愆期,則以法究治。分案六,置吏二十有四。元祐元年四月,省郎官一員,十月復置。分案六:曰倉場,曰上供,曰糶糴,曰給納,曰知雜,曰開拆。建炎三年,罷司農寺歸倉部。紹興四年復舊。裁減吏額,共置二十五人,續又減二人。



 禮部掌國之禮樂、祭祀、朝會、宴饗、學校、貢舉之政令。祭之名有三:天神曰祀,地祇曰祭,宗廟曰饗。又有大祀、中祀、小祀之別。幣玉、牲牢、器服,各從其等。凡雅樂,以六律、
 六同合陰陽之聲為樂律,金、石、絲、竹、匏、土、革、木為樂器,宮架八佾,特架六佾,分武文先後之序為樂舞,其所歌為樂章。若有事於南北效、明堂,籍田、禘祫太廟,薦享景靈宮,酌獻陵園,及行朝貢、慶賀、宴樂之禮,前期飭有司辨具,閱所定儀注,以舊章參考其當否,上尚書省,冊寶及封冊命禮亦如之。凡禮樂制度有所損益,小事則同太常寺,大事則集侍從官、秘書省長貳或百官,議定以聞。凡天下選士,具注于籍,三歲貢舉,與夫學校試補三
 舍生。掌后妃、親王以下推恩,公主下嫁,宗室冠、婚、喪、葬之制,及賜旌節、章服、冠帔、門戟,旌表孝行之法。若印記、圖書、表疏之事皆掌焉。大祥瑞,則朝參官以上詣閣門表賀,餘於歲終條奏。



 舊屬禮儀院,判院一人,以樞密院使、參知政事充;知院,以諸司三品以上充。主吏無定數,擇三司京朝百司胥史充。禮部止設判部一人,掌科舉,補奏太廟郊社齋郎、室長、掌坐,都省集議,百官謝賀章表,諸州申祥瑞,出入內外牌印之事。兼領貢院,掌受
 諸州解發進士諸科名籍及其家保狀、文卷,考驗戶籍、舉數、年齒而藏之。若朝廷遣官知舉,則主判官罷,事畢,以知舉官卑者一員主判。元豐官制行,悉歸禮部。其屬三:曰祠部,曰主客,曰膳部。設官十:尚書、侍郎各一人,郎中、員外郎四司各一人。元祐初,省祠部郎官一員,以主客兼膳部。紹聖改元,主客、膳部互置郎官兼領。建炎以後並同。



 尚書掌禮樂、祭祀、朝會、宴享、學校,貢舉之政令,侍郎為
 之貳,郎中、員外郎參領之。凡講議制度,損益儀物,則審覆有司所定之式,以次諮決,而質於尚書省。大祭祀則省牲,鼎鑊視滌濯,薦腥則奉籩豆、簠簋,及飲福徹之,稞則奉瓚臨鬯。凡天地、宗廟、陵園之祀,後妃、親王、將相封冊之命,皇子加封,公主降嫁,稽其彞章以詔上下而舉行之。朝廷慶會宴樂,宗室冠、婚、喪、祭,蕃使去來宴賜,與夫經筵、史館、賜書、修書之禮,例皆同奉常講求參酌,而定其儀節。三歲貢舉,學校試補諸生,皆總其政。旌節章
 服之頒,祥瑞表奏之進,凡關於禮樂者,皆掌之。建炎三年,詔鴻臚、光祿寺並歸於禮部,太常、國子監亦隸焉。分案五:曰禮樂,曰貢舉,曰宗正奉使帳,曰封冊表奏,曰檢法。各隨其名而治其事。裁減吏額,四十五人。



 續又減四人。



 侍郎奏中嚴外辦,同省牲及視饌腥熟之節。祼,受瓚奉盤。歲祀昊天上帝,祭皇地祇,與尚書迭為亞獻。祭太社、太稷、神州地祇,則迭為初獻。祀九宮貴神、五帝、感生帝、朝日、夕月、蠟祭東西方亦如之。大朝會,則尚書奏藩國
 貢物。凡慶賀若謝,則郎中、員外郎分撰表文。祠事,與太常少卿、祠部官迭為終獻或亞獻。親郊,自景靈宮朝獻太廟朝享至望燎禮畢,乘輿還內,皆奏解嚴。分案十,置吏三十有五。南渡,諸曹長、貳互置。紹興七年,禮部置侍郎二員。隆興元年,詔:「除尚書不常置外,禮部侍郎置一員。」



 郎中員外郎元豐,郎官、員外郎參領禮樂、祭祀、朝會、宴享、學校、貢舉之事。有所損益,則審訂以次諮決。凡慶
 會若謝,掌撰表文。與祠部、主客、膳部並列為四。建炎三年,並省郎曹,禮部領主客,祠部領膳部。隆興元年,復詔禮部、祠部一員兼領,自是並行四司之事矣。通置吏五十四人。



 祠部郎中員外郎掌天下祀典、道釋、祠廟、醫藥之政令。月奏祠祭、國忌、休暇之日。每歲大祀忌日,大忌前一日,皆不坐。元日、冬至、寒食假各七日。天慶、先天、降聖節各五日。誕聖節、正七月望、夏至、臘各三日。天祺、天貺節、
 人日、中和、二社、上巳、端午、三伏、七夕、授衣、重九、四立、春秋分及每旬假各一日。若神祠封進爵號,則覆太常所定以上尚書省。凡宮觀、寺院道釋,籍其名額,應給度牒,若空名者毋越常數。初補醫生,令有司試藝業,歲終校全失而賞罰之。分案五,置吏二十有一。



 主客郎中員外郎掌以賓禮待四夷之朝貢。凡郊勞、授館、宴設、賜予,辨其等而以式頒之。至則圖其衣冠,書其山川風俗。有封爵禮命,則承詔頒付。掌嵩、慶、懿陵祭
 享,崇義公承襲之事。分案四,置吏七。



 元祐六年七月,兵部言:「《兵部格》,掌蕃夷官授官;《主客令》,蕃國進奉人陳乞轉授官職者取裁。即舊應除轉官者,報所屬看詳。舊來無例,創有陳乞,曹部職掌未一,久遠互失參驗,自今不以曾未貢及例有無,應緣進奉人陳乞,授官加恩,令主客關報兵部。」從之。



 膳部郎中員外郎掌牲牢、酒醴、膳羞之事。凡所用物,前期計度,以關度支。若祭祀、朝會、宴享,則同光祿寺官視其善否,酒成則嘗而後進。季冬命藏冰,春分啟之,以待供賜。分案七,置吏九。



 兵部掌兵衛、儀仗、鹵簿、武舉、民兵、廂軍、土軍、蕃軍,四夷
 官封承襲之事,輿馬、器械之政,天下地土之圖。凡儀衛,大朝會用黃麾大仗;文德殿視朝及冊命王公大臣,用黃麾半仗;紫宸殿受外國使朝,用黃麾角仗;文德殿發冊,用黃麾細仗。鹵簿有大駕、法駕、小駕,皆掌其數及行列先後之儀,為圖以授有司。凡武選之制,仿貢舉之法。凡聯其什伍而教之以戰為民兵,材不中禁衛而足以執役為廂軍,就其鄉井募以御盜為土軍,以老疾而裁其功力之半為剩員。團結以御戎為洞丁,為義軍、弩手;
 屬羌分隸邊將為蕃兵。籍其名數而頒其禁令。大將出征,奏捷則告於廟,破賊則露布以聞。凡招置廂、禁軍及州郡屯營,三衙遷補,守戍軍吏轉補,文武官白直、宣借,皆掌之。其屬三:曰職方,曰駕部,曰庫部。舊判部事一人,以兩制充。掌三駕儀仗、鹵簿圖、春秋釋奠武成王廟及武舉,歲終以義軍、弓箭手戶數上於朝。國初,掌千牛備身,殿中省進馬籍。元豐設官十,尚書、侍郎各一,四司郎中、員外郎各一。元祐初,省駕部郎中一員,以職方兼庫
 部。紹興改元,詔職方、庫部互置郎官一員兼。



 尚書掌兵衛、武選、車輦、甲械、廄牧之政令。以天下郡縣之圖而周知其地域。凡陳鹵簿,設仗衛,飭官吏整肅,蕃夷除授,奉行其制命。凡軍兵以名籍統隸者,閱習按試,選募遷捕,及武舉、校試之事,皆總之。侍郎為之貳,郎中、員外郎參掌之。大禮,則尚書充鹵簿使;大祀,奉魚牲及俎;視朝,則侍郎執班簿對立;小祀,則郎中、員外郎薦俎並徹。分案九,置吏四十有七。凡蕃夷屬戶授官、封襲之
 事皆掌之。建炎三年,並衛尉寺隸焉。分案十:曰賞功,曰民兵衛,曰廂兵,曰人從看詳,曰帳籍告身,曰武舉,曰蕃官,曰開拆,曰知雜,曰檢法。乾道裁減吏額,共置三十人。續詔:「將下班祗應並進義校尉、守闕進義副尉、進武校尉、守闕進武副尉並隸兵部,許於殿前司抽差下班祗應,文字人吏六名,赴部行遣。」



 侍郎掌貳尚書之事。南渡,長貳互置,續置侍郎二員,紹興常置一員。



 郎中員外郎參掌本部長貳之事。建炎三年,詔兵部兼職方,駕部兼庫部。隆興元年,詔駕部、兵部郎官共一員兼領,自是四司合為一矣。厥後間或並置,若從軍或將命於外,則假以為寵焉。



 職方郎中員外郎掌天下圖籍,以周知方域之廣袤,及郡邑、鎮砦道里之遠近。凡土地所產,風俗所尚,具古今興廢之因,州為之籍,遇閏歲造圖以進。四夷歸附,則分隸諸州,度田屋錢糧之數以給之。分案三,置吏五。舊
 判司事一人,以無職事朝官充,掌受閏年圖經。國初,令天下每閏年造圖納儀鸞司。淳化四年,令再閏一造;咸平四年,令上職方。轉運畫本路諸州圖,十年一上。紹熙三年,職方、駕部吏額通入兵部、庫部,並作四十二人。



 駕部郎中員外郎掌輿輦、車馬、驛置、廄牧之事。大禮,戒有司具五輅。凡奉使之官赴闕,視其職治給馬如格。官文書則量其遲速以附步馬急遞。總內外監牧,籍其租入多寡、孳產登耗。凡市馬於四夷者,溢歲額則賞之。
 分案六,置吏十有三。建炎三年,並太僕寺隸焉。



 庫部郎中員外郎掌鹵簿、儀仗、戎器、供帳之事,國之武庫隸焉。凡內外甲仗器械,造作繕修,皆有法式。若御大慶、文德殿,應用鹵簿名數,前期以戒有司。祭祀、喪葬,則給以等差。總衛尉寺金吾仗司兵匠之數,考其功罪、歲月而以法升降之。分案四,置吏九。



 刑部掌刑法、獄訟、奏讞、赦宥、敘復之事。凡斷獄本於律,律所不該,以敕、令、格式定之。凡律之名十有二:曰名例,曰禁
 衛,曰職制,曰戶婚,曰廄庫,曰擅興,曰盜賊,曰鬥訟,曰詐偽,曰雜律,曰捕亡,曰斷獄。禁於未然之謂令,施於已然之謂敕,設於此而使彼至之之謂格,設於此而使彼效之之謂式。其一司一路海行所不該者,折而為專法。若情可矜憫而法不中情者讞之,皆閱其案狀,傳例擬進。應詔獄及案劾命官,追命奸盜,以程督之。審覆京都闢囚,在外已論決者,摘案檢察。凡大理、開封、殿前馬步司獄,糾正其當否;有辯訴,以情法與奪、赦宥、降放、敘
 雪。若命官牽復,則以期數定之。其屬三:曰都官,曰比部,曰司門。設官十有三:尚書一人,侍郎二人:郎中、員外郎,刑部各二人,都官、比部、司門各一人。



 國初,以刑部覆大闢案。淳化二年,增置審刑院,知院事一人,以郎官以上至兩省充,詳議官以京朝官充,掌詳讞大理所斷案牘而奏之。凡獄具上,先經大理,斷讞既定,報審刑,然後知院與詳議官定成文草,奏記上中書,中書以奏天子論決。大中祥符二年,置糾察刑獄司,糾察官二人,以兩制
 以上充。凡在京刑禁,徒以上實時以報;若理有未盡或置淹恤,追覆其案,詳正而駁奏之。凡大闢,皆錄問。熙寧三年,詔:「詳議、詳斷、詳覆官,初入以三年為任,次以三十月為任,欲出者聽前任滿半年指闕注官,滿三任者堂除。」八年,罷詳議、詳斷官親書節案,止令節略付吏,仍減議官一、斷官二。元豐二年,知院安燾言:「天下奏案,益多於往時。自熙寧八年減議官、斷官,力既不足,故事多疏謬。」增詳議官一,刑部增詳斷官一。三年八月,詔:「省審刑
 院歸刑部。以知院官判刑部,掌詳議、詳覆司事。刑部主判官為同判刑部,掌詳斷司事,審刑議官為刑部詳議官。」官制行,悉罷歸刑部。



 元祐元年,省比部郎官一員,以都官兼司門。五月,三省言:「舊制,糾察在京刑獄以察違慢,自罷歸刑部,無復申明糾舉之制,請以御史臺刑察兼領。其御史臺刑獄,令尚書省右司糾察。」從之。刑部舊有詳覆案,自官制行,歸諸路提刑司,至是復置。四年,並制勘、體量為一案。紹聖元年,詔都官、司門互置郎官一員。
 崇寧二年十二月,詔:「刑部尚書通治左右曹,侍郎一治左曹,一治右曹,如獨員,即通治,餘並依官制格令。」



 尚書掌天下刑獄之政令。凡麗於法者,審其輕重,平其枉直,而侍郎為之貳。應定奪、審覆、除雪、敘復、移放,則尚書專領之;制勘、體量、奏讞、糾察、錄問,則長貳治之;而郎中、員外郎分掌其事。有司更定條法,則復議其當否。凡聽訟獄或輕重失中,有能駁正,詔其賞罰。若頒赦宥,則糾官吏之稽違者;大祀,則尚書蒞誓,薦熟則奉牲。大禮
 肆赦,則侍郎授赦書付有司宣讀,承旨釋囚。分案十二,置吏五十有二。紹興後,分案十三:曰制勘,掌凡根勘諸路公事;曰體量,掌凡體究之事;曰定奪,掌訴雪除落過名;曰舉敘,掌命官敘復;曰糾察,掌審問大闢;曰檢法,掌供檢條法;曰頒降,掌頒條法降赦;曰追毀,掌斷罰追毀宣敕;曰會問,掌批會過犯;曰詳覆,掌諸路大闢帳狀;曰捕盜;曰帳籍,掌行在庫務、理欠帳籍;曰進擬,掌進斷案刑名文書。裁減吏額,置三十五人。



 侍郎舊制,應定奪、審覆、除雪、敘復、移放,尚書專領之。若制勘、體量、奏讞、糾察、錄問,長貳通治之。南渡,長貳互置。隆興常置一員。淳熙十六年,依崇寧專法,奏獄及法令事,請大理寺官赴部共議之,用侍郎吳博古之說也。



 郎中員外郎各二人,分左右廳,掌詳覆、敘雪之事。建炎三年,刑部郎官以二員為額,關掌職事,初無分異。紹興二十六年,詔依元豐舊法,分廳治事。先是,右司汪應辰言:「刑部郎官分為左右,左以詳覆,右以敘雪,同僚異
 事,祖宗有深意。倘初無分異,則有不當於理者,孰為追改?乞遵用舊制,要使官各有守,人各有見,參而用之,以稱欽恤之意。」從之,仍令今後仿此。



 都官郎中員外郎掌徒流,配隸。凡天下役人與在京百司吏職皆有籍,以考其役放及增損廢置之數。若定差副尉,舊為軍大將。



 則計其所歷,而以役之輕重均其勞逸,給印紙書其功過,展減磨勘歲月。元祐八年,以綱運差使關歸吏部,省副尉員三百。紹聖間,復其額,及元豐押
 綱法,歸都官。崇寧二年二月,復配隸案。先是,元豐中,都官有吏籍、配隸案,元祐中,罷之。因刑部有請,乃詔如舊。六月,侍郎劉賡奏:「副尉差遣有立定優重等第,都官條雖特旨亦許執奏,乞申嚴其禁。」從之。分案四,置吏十有八。建炎三年,詔比部兼司門。隆興元年,詔都官、比部共置一員。自此都官兼比部司門之事。分案五:曰差次,曰磨勘,曰吏籍,曰配隸,曰知雜,各因其名而治其事。裁減吏額,置十二人。



 淳熙十三年,減三人。



 比部郎中員外郎掌勾覆中外帳籍。凡場務、倉庫出納在官之物,皆月計、季考、歲會,從所隸監司檢察以上比部,至則審覆其多寡登耗之數,有陷失,則理納。鉤考百司經費,有隱昧,則會問同否而理其侵負。舊帳案隸三司,自治平中至熙寧初,凡四年帳未鉤考者已逾十有二萬,錢帛、芻粟積虧不可勝計。五年十一月,曾布奏以四方財賦當有簿書文籍,以鉤考其給納登耗多寡。遂置提舉帳司,選人吏二百人,驅磨天下帳籍,並選官
 吏審覆。七年二月,詔帳司每歲具天下財用日出入數以聞。元豐初年,詔:「諸路財賦出入,自今三年一供,著為令。」官制行,厘其事歸比部。元祐元年七月,用司馬光奏,悉總於戶部。三年,厘正倉部,勾覆、理欠、憑由案及印發鈔引事歸比部。政和六年,詔:「寺監先期檢舉,如庫務監官所造文帳委無未備,方許批書,違者御史臺奏劾。」用郎官梅執禮之請也。分案五,置吏百有一。建炎以後,或以都官兼比部、司門之事。



 司門郎中員外郎掌門關、津梁、道路之禁令,及其廢置移復之事。應官吏、軍民、輦道商販,譏察其冒偽違縱者。凡諸門啟閉之節及關梁餘禁,以時舉行。分案二,置吏五。



 工部掌天下城郭、宮室、舟車、器械、符印、錢幣,山澤、苑囿、河渠之政。凡營繕,歲計所用財物,關度支和市;其工料,則飭少府、將作監檢計其所用多寡之數。凡百工,其役有程,而善否則有賞罰。兵匠有關,則隨以緩急招募。籍
 坑冶歲入之數,若改用錢寶,先具模制進御請書。造度、量、權、衡則關金部。印記則關禮部。凡道路、津梁,以時修治。舊制,判部事一人,以兩制以上充。元豐並歸工部。其屬三:曰屯田,曰虞部,曰水部。設官十。尚書、侍郎各一人,工部、屯田、虞部、水部郎中、員外郎各一人。元祐元年,省水部郎官一員。紹聖元年,詔屯田、虞部互置郎官一員兼領。



 尚書掌百工水土之政令,稽其功緒以詔賞罰。總四司
 之事,侍郎為之貳。若制作、營繕、計置、採伐所用財物,按其程序以授有司,郎中、員外郎參掌之。應官吏、兵民緣本曹事有功賞罪罰,則審實以上尚書省。大祭祀,則尚書薦俎與徹。若諸監鼓鑄錢寶,按年額而課其數,因其登耗以詔賞罰。凡車輦、飭器、印記之造,則少府監、文思院隸焉。甲兵器械之制,則軍器所隸焉。有合支物料工價,則申於朝,以屬戶部。建炎並將作、少府、軍器監並歸工部。是時營繕未遑,惟戎器方急。紹興二年,詔於行在
 別置作院造器甲,令工部長貳提點,郎官逐旬點檢。少府監既歸工部,文思院上下界監官並從本部闢差。又詔御前軍器所隸工部,自是營造稍廣。宰臣議:「戶部以給財為務,工部以辦事為能,誠非一體。」欲令戶、工部兼領其事,卒未能合。隆興以後,宮室、器甲之造浸稀,且各分職掌,部務益簡,特提其綱要焉。分案六:曰工作,曰營造,曰材料,曰兵匠。曰檢法,曰知雜。又專立一案,以御前軍器案為名。裁減吏額,共置四十二人。



 侍郎掌貳尚書之事。南渡初,長、貳互置,隆興詔各置一員。



 郎中員外郎舊制,凡制作、營繕、計置、採伐材物,按程序以授有司,則參掌之。建炎三年,詔:「工部郎官兼虞部,屯田郎官兼水部。隆興元年,詔工部、屯田共一員兼領,自此四司合為一矣。淳熙九年,以趙公暠為屯田員外郎,自是不復省。



 屯田郎中員外郎掌屯田、營田、職田、學田、官莊之政
 令,及其租入、種刈、興修、給納之事。凡塘濼以時增減,堤堰以時修葺,並有司修葺種植之事,以賞罰詔其長貳而行之。分案三,置吏八。



 虞部郎中員外郎掌山澤、苑囿、場冶之事,辨其地產而為之厲禁。凡金、銀、銅、鐵、鉛、錫、鹽、礬,皆計其所入登耗以詔賞罰。分案四,置吏七。



 水部郎中員外郎掌溝洫、津梁、舟楫、漕連之事。凡堤防決溢,疏導壅底,以時約束而計度其歲用之物。修治
 不如法者,罰之;規畫措置為民利者,賞之。分案六,置吏十有三。紹興累減吏額,四司通置三十三人。



 軍器所隸工部。



 提點官二員,紹興三十二年,詔於邊臣內差。



 提轄、監造官各二員,乾辦、受給、監門官各一員。掌鳩工聚材、制造戎器之政令。舊就軍器監置,別差提舉官,以內侍領之。紹興中,改隸工部,罷提舉官,日輪工部郎官、軍器監官前去本所點驗監視;後復以中人典領。工部侍郎黃中以為言,請復隸屬。從之。孝宗即位,有旨增置提點官,以內
 省都知李綽為之,改稱提舉,免隸工部。後以御史張震力爭,復隸工部。後改隸步軍司,尋復舊。紹熙元年,減省員額,如上制。



 文思院隸工部。



 提轄官一員,監官三員,內置一員文臣,京朝官充。



 監門官一員。掌金銀、犀玉工巧及採繪、裝鈿之飾。凡儀物、器仗、權量、輿服所以供上方、給百司者,於是出焉。沿革附見榷貨務都茶場提轄官。



 六部監門六部監門官一員,掌司門鑰。紹興二年置。
 選升朝文臣有才力人充,仍令六部踏逐奏差。序位、請給依寺、監丞,郎官有闕得兼之。初從吏部尚書沈與求之請也。



 主管架閣庫掌儲藏帳籍文案以備用。擇選人有時望者為之。舊有管幹架閣庫官,宣和罷之,紹興十五年復置,吏、戶部各差一員,禮、兵部共差一員,刑、工部共差一員,以主管尚書某部架閣庫為名,從大理寺丞周楙請也。喜定八年,又置三省、樞密院架閣官。



\end{pinyinscope}