\article{志第八十三 樂五}

\begin{pinyinscope}

 高宗南渡,經營多難,其於稽古飾治之事,時靡遑暇。建炎元年,首詔有司曰:「朕承祖宗遺澤,獲托臣民之上,扶顛持危,夙夜痛悼。況於聞樂以自為樂,實增感於朕心。」
 二年,復下詔曰:「朕方日極憂念,屏遠聲樂,不令過耳。承平典故,雖實廢名存,亦所不忍,悉從減罷。」是歲,始據光武舊禮,以建武二載創立郊祀,乃十一月壬寅祀天配祖,敕東京起奉大樂登歌法物等赴行在所,就維揚江都築壇行事。凡鹵簿、樂舞之類,率多未備,嚴更警場,至就取中軍金鼓,權一時之用。



 紹興元年,始饗明堂。時初駐會稽,而渡江舊樂復皆毀散。太常卿蘇遲等言:「國朝大禮作樂,依儀合於壇殿上設登歌,壇殿下設宮架。今
 親祠登歌樂器尚闕,宣和添用鑰色,未及頒降,州郡無從可以創制,宜權用望祭禮例,止設登歌,用樂工四十有七人。」乃訪舊工,以備其數。



 四年,再饗,國子丞王普言:「按《書·舜典》,命夔曰:『詩言志,歌永言,聲依永,律和聲。』蓋古者既作詩,從而歌之,然後以聲律協和而成曲。自歷代至於本朝,雅樂皆先制樂章而後成譜。祟寧以後,乃先制譜,後命詞,於是詞律不相諧協,且與俗樂無異。乞復用古制。又按《周禮》奏黃鐘、歌大呂以祀天神。黃鐘,堂下
 之樂;大呂,堂上之樂也。郊祀之禮,皇帝版位在午階下,故還位之樂當奏黃鐘;明堂版位在阼階上,則還位當歌大呂。今明堂禮不下堂,而襲郊祀還位例,並奏黃鐘之樂,於義未當。」尋皆如普議。



 先是,帝嘗以時難備物,禮有從宜,敕戒有司參酌損益,務崇簡儉。仍權依元年例,令登歌通作宮架,其押樂、舉麾官及樂工器服等,蠲省甚多。既而國步漸安,始以保境息民為務,而禮樂之事浸以興矣。



 十年,太常卿蘇攜言:「將來明堂行禮,除登歌
 大樂已備,見闕宮架、樂舞,諸路州軍先有頒降登歌大樂,乞行搜訪應用。」丞周執羔言:「大樂兼用文、武二舞,今殿前司將下任道,系前大晟府二舞色長,深知舞儀,宜令赴寺教習。」卿陳桷言:「前期五使,例合按閱,仍詔應侍祠執事朝臣,並作樂教習。」禮儀博士周林復言:「神位席地陳設,至尊親行酌獻,堂上下皆地坐作樂,而鐘磬工乃設木小榻,當教習日,使立以考擊,庶革循習簡陋之弊。



 初,上居諒闇,臣僚有請罷明堂行禮奏樂、受胙等事,
 上諭禮官詳定。太常寺檢照景德、熙、豐親郊典故,除郊廟、景靈宮並合用樂,其鹵簿、鼓吹及樓前宮架、諸軍音樂,皆備而不作。每處警場,止鳴金鉦、鼓角而已,即無去奏樂、受胙之文。大饗為民祈福,為上帝、宗廟而作樂,禮不敢以卑廢尊。《書》「斂五福,錫庶民」,況熙寧禮尤可考,其赦文有曰「六樂備舞,祥祉來臻」是也。於是詔遵行之。其後,禮部侍郎施坰奏:「禮經蕃樂出於荒政,蓋一時以示貶抑。昨內外暫止用樂,今徽考大事既畢,慈寧又已就
 養,其時節上壽,理宜舉樂,一如舊制。」禮部尋言:「太母還宮,國家大慶,四方來賀。自今冬至、元正舉行朝賀之禮,依國朝故事,合設大仗及用樂舞等,庶幾明天子之尊,舊典不至廢墜。」有詔俟來年舉行。



 十有三年,郊祀,詔以祐陵深弓劍之藏,長樂遂晨昏之養,昭答神天,就臨安行在所修建圓壇。於是有司言:「大禮排設備樂,宮架樂辦一料外,登歌樂依在京夏祭例,合用兩料。其樂器,登歌則用編鐘、磬各一架,柷、敔二,搏拊、鼓二,琴五色,自一、
 三、五、七至九絲玄各二,瑟四,笛四,塤、篪、簫並二,巢笙、和笙各四;並七星、九曜、閏餘匏笙各一,麾幡一。宮架則用編鐘、編磬各十二架,柷、敔二,琴五色,各十,瑟二十六;巢笙及簫並一十四,七星、九曜、閏餘匏笙各一,竽笙十,塤一十二,篪一十八,笛二十,晉鼓一,建鼓四,麾幡一。」乃從太常下之兩浙、江南、福建州郡,又下之廣東西、荊湖南北,括取舊管大樂,上於行都,有闕則下軍器所制造,增修雅飾,而樂器浸備矣。其樂工,詔依太常寺所請,選擇
 行止畏謹之人,合登歌、宮架凡用四百四十人,同日分詣太社、太稷、九宮貴神。每祭各用樂正二人,執色樂工、掌事、掌器三十六人,三祭共一百一十四人,文舞、武舞計用一百二十八人,就以文舞番充。其二舞引頭二十四人,皆召募補之。樂工、舞師照在京例,分三等廩給。其樂正、掌事、掌器,自六月一日教習;引舞、色長、文武舞頭、舞師及諸樂工等,自八月一日教習。於是樂工漸集。



 十四年,太常寺言:「將來大禮,見闕玉磬十六枚。其所定聲
 律,系於玉分厚薄,取聲高下。正聲凡十有二,黃鐘厚八分,進而為大呂、太簇、夾鐘、姑洗、仲呂、蕤賓、林鐘、夷則、南呂、無射、應鐘,每律增一分,至應鐘一寸九分而止。清聲夾鐘厚二寸三分,退而為太簇、大呂、黃鐘,共四清聲,各減一分,至黃鐘二寸而止。」乃下之四川茶馬司,寬數增分,市易以供用。太常博士張晟又言:「大樂所用武舞之飾,以干配刀,《周禮·司兵》『祭祀,授舞者兵』,先儒謂『授以朱干、玉戚』,《郊特牲》『朱干、玉戚,冕而舞大武』。」乃從所請,仿《三
 禮圖》,令造玉戚,以配舞干。



 是歲,始上徽宗徽號,特制《顯安》之樂。至於奉皇太后冊、寶於慈寧宮,樂用《聖安》;皇后受冊、寶於穆清殿,樂用《坤安》,亦皆先後參次而舉。《顯安》以無射、夾鐘為宮,周大司樂饗先王,奏無射而歌夾鐘,「夾鐘之六五,上生無射之上九。夾鐘,卯之氣,二月建焉,而辰在降婁;無射,戌之氣,九月建焉,而辰在大火。」無射,陽律之終,夾鐘實為之合,蓋取其相親合而萃祖考之精神於假廟也。《聖安》純用大呂,《坤安》純用中呂。大呂,陰
 律之首,崇母儀也;中呂,陰律之次,明婦順也。



 明年正旦朝會,始陳樂舞,公卿奉觴獻壽。據元豐朝會樂:第一爵,登歌奏《和安》之曲,堂上之樂隨歌而發;第二爵,笙入,乃奏瑞曲,惟吹笙而餘樂不作;第三爵,奏瑞曲,堂上歌,堂下笙,一歌一吹相間;第四爵,合樂仍奏瑞曲,而上下之樂交作。今悉仿舊典,首奏《和安》,次奏《嘉木成文》、《滄海澄清》、《瑞粟呈祥》三曲,其樂專以太簇為宮。太簇之律,生氣湊達萬物,於三統為人正,於四時為孟春,故元會用之。



 時給事中段拂等討論景鐘制度,按《大晟樂書》:「黃鐘者,樂所自出,而景鐘又黃鐘之本,故為樂之祖,惟天子郊祀上帝則用之,自齋宮詣壇則擊之,以召至陽之氣。既至,聲闋,眾樂乃作。祀事既畢,升輦,又擊之。蓋天者,群物之祖,今以樂之祖感之,則天之百神可得而禮。音韻清越,拱以九龍,立於宮架之中,以為君圍;環以四清聲鐘、磬、鎛鐘、特磬,以為臣圍;編鐘、編磬以為民圍。內設寶鐘球玉,外為龍虡鳳琴。景鐘之高九尺,其數九九,實高八
 尺一寸。垂則為鐘,仰則為鼎。鼎之大,中於九斛,退藏實八斛有一焉。」內出皇祐大樂中黍尺,參以太常舊藏黃鐘律編鐘,高適九寸,正相吻合,遂遵用黍尺制造。



 鐘成,命左僕射秦檜為之銘。其文曰:「皇宋紹興十六年,中興天子以好生大德,既定寰宇,乃作樂以暢天地之化,以和神人。維茲景鐘,首出眾樂,天子專用禋祀,謹拜手稽首而獻銘。其銘曰:德純懿兮舜、文繼。躋壽域兮孰內外?薦上帝兮偉茲器。聲氣應兮同久視。貽子孫兮彌萬世。」
 旋又命禮局造鎛鐘四十有八、編磬一百八十七、特磬四十八及添制編鐘等,命軍器所造建鼓八、雷鼓二、晉鼓一、雷□二、柷敔各四。尋制金鐘、玉磬二架。



 初,元豐本虞庭鳴球及晉賀循採玉造磬之義,命榮咨道肇造玉磬。元祐親祠,嘗一用之,久藏樂府。至政和加以磨礱,俾協音律,並造金鐘,專用於明堂。蓋堂上之樂,歌鐘居左,歌磬居右。金玉稟氣於乾,純精至貴,故鐘必以金,磬必以玉,始備金聲玉振之全,此中興所以繼作也。於是帝
 諭輔臣,以鐘磬音律,其餘皆和,惟黃鐘、大呂猶未應律,宜熟加考究。詔禮官以鑄造鎛鐘,更須詳審,令聲和而律應,乃可奉祀。命太常前期按閱,仍用皇祐進呈雅樂禮例。皇帝御射殿,召宰執、侍從、臺諫、寺監、館閣及武臣刺史以上,閱視新造景鐘及禮器。皇帝即御坐,撞景鐘,用正旦朝會三曲,奏宮架之樂,其制造官推恩有差。添置景鐘樂正一、鎛鐘樂工十有二,特磬樂工亦如之。次降下古制銅錞一,增造其二;古銅鐃一,增造其六。改造
 登歌夷則律玉磬,降到長笛二十有四,並付太常寺掌之,專俟大禮施用。



 既而刑部郎官許興古奏:「比歲休祥協應,靈芝產於廟楹,瑞麥秀於留都。昔乾德六年,嘗詔和□見作《瑞木》、《馴象》及《玉烏》、《皓雀》四瑞樂章,以備登歌。願依典故,制為樂章,登諸郊廟。」詔從其請,命學士沈虛中作歌曲,以薦於太廟、圜丘、明堂。尋又內出禦制郊祀大禮天地、宗廟樂章,及詔宰執、學士院、兩省官刪修郊祀大禮樂章,付太常肄習。



 天子親祀南郊,圜鐘為宮,三奏,
 樂凡六成,歌《景安》,用《文德武功》之舞;饗明堂,夾鐘為宮,三奏,樂凡九成,歌《誠安》,用《祐文化俗》、《威功睿德》之舞。前二日,朝獻景靈宮,圜鐘為宮,三奏,凡六成,所奏樂與南郊同,歌《興安》,用《發祥流慶》、《降真觀德》之舞。前一日,朝饗太廟,黃鐘為宮,三奏,樂凡九成,歌《興安》,所用文、武二舞與南郊同。僖祖廟用《基命》之樂舞,翼祖廟用《大順》之樂舞,宣祖廟用《天元》之樂舞,太祖廟用《皇武》之樂舞》,太宗廟用《大定》之樂舞。真宗、仁宗廟樂舞曰《熙文》、曰《美成》,英
 宗、神宗廟樂舞曰《治隆》、曰《大明》,哲宗、徽宗、欽宗廟樂舞曰《重光》、曰《承元》、曰《端慶》,皆以無射宮奏之。



 每歲祀昊天上帝者凡四:正月上辛祈穀,孟夏雩祀,季秋饗明堂,冬至祀圜丘是也。圜鐘為宮,樂奏六成,與南郊同,乃用《景安》之歌、《帝臨嘉至》、《神娭錫羨》之舞。祀地祗者二:夏至祀皇地祗,樂奏八成,乃用《寧安》之歌、《儲靈錫慶》、《嚴恭將事》之舞;立冬後祀神州地祗,樂奏八成,歌《寧安》,與祀皇地祗同名而異曲,用《廣生儲祐》、《厚載凝福》之舞。孟春上辛祀感
 生帝,其歌《大安》,其樂舞則與歲祀昊天同。三年一袷及時饗太廟,九成之樂、《興安》之歌,與大禮前事朝饗同,而用《孝熙昭德》、《禮洽儲祥》之舞。太社、太稷用《寧安》,八成之樂,與歲祀地祗同。至於親制贊宣聖及七十二弟子,以廣崇儒右文之聲;親視學,行酌獻,定釋奠為大祀,用《凝安》,九成之樂。郡邑行事,則樂止三成。他如親饗先農、親祀高禖,則敞壇壝、奏樂舞,按習於同文館、法惠寺。親耕籍田,則據宣和舊制,陳設大樂,而引呈耒耜、護衛耕根
 車、儀仗鼓吹至以二千人為率。先農樂用《靜安》;高禖樂用《景安》;皇帝親行三推禮,樂用《乾安》。其補苴軼典、搜講彌文者至矣。先朝凡雅樂皆以『安』名,中興一遵用之。



 南郊樂,其宮圜鐘;明堂樂,其宮夾鐘。圜鐘即夾鐘也。夾鐘生於房、心之氣,實為天帝之堂,故為天宮。祭地祗,其宮函鐘,即林鐘也。林鐘生於未之氣,未為坤位,而天社、地神實在東井、輿鬼之外,故為地宮。饗宗廟,其宮用黃鐘。黃鐘生於虛、危之氣,虛、危為宗廟,故為人宮。此三者,各
 用其聲類求之。然天宮取律之相次:圜鐘為陰聲第五,陰將極而陽生,故取黃鐘為角。黃鐘,陽聲之首也。太簇,陽聲之第二,故太簇為征。姑洗,陽聲之第三,故姑洗為羽。天道有自然之秩序,乃取其相次者以為聲。地宮取律之相生:函鐘上生太簇,故太簇為角;太簇下生南呂,南呂上生姑洗,故南呂為征,姑洗為羽。地道資生而不窮,乃取其相生者以為聲。人宮取律之相合:黃鐘子,大呂丑,故黃鐘為宮、大呂為角,子合醜也;太簇寅,應鐘亥,
 故太簇為徵、應鐘為羽,寅合亥也。人道以合而相親,乃取其合者以為聲。周之降天神、出地示、禮人鬼,樂之綱要實在於此。獨商聲置而不用,蓋商聲剛而主殺,實鬼神之所畏也。樂奏六成者,即仿周之六變,八成、九成亦如之。



 文、武二舞皆用八佾。國初,始改《崇德》之舞曰《文德》,改《象成》之舞曰《武功》。其《發祥流慶》、《降真觀德》則祥符所制,以薦獻聖祖;其《祐文化俗》、《威功睿德》則皇祐所制,以奉明禋。其祀帝,有司行事,以《帝臨嘉至》、《神娭錫羨》,與夫
 獻太廟以《孝熙昭德》、《禮洽儲祥》,則制於元豐。其《廣生儲祐》、《厚載凝福》以祀方澤,則制於宣和。至紹興祀皇地祗,易以《儲靈錫慶》、《嚴恭將事》,而用宣和所制舞以分祀神州地祗,轉相緝熙,樂舞浸備。至中興而賡續裁定,實集其成。中祀而下,多有樂而無舞,則在《禮》「凡小祭祀不興舞」之義也。



 紹興三十一年,有詔:教坊日下蠲罷,各令自便。蓋建炎以來,畏天敬祖,虔恭祀事,雖禮樂煥然一新,然其始終常以天下為憂,而未嘗以位為樂,有足稱者。



 孝宗初踐大位,立班設仗於紫宸殿,備陳雅樂。禮官尋請車駕親行朝饗,用登歌、金玉大樂及彩繪宮架、樂舞;仗內鼓吹,以欽宗喪制不用。迨安穆皇后祔廟,禮部侍郎黃中首言:「國朝故事,神主升祔,系用鼓吹導引,前至太廟,乃用樂舞行事。宗廟薦享雖可用樂,鼓吹施於道路,情所未安,請備而不作。」續下給、舍詳議,謂:「薦享宗廟,為祖宗也,故以大包小,則別廟不嫌於用樂。今祔廟之禮為安穆而行,豈可與薦享同日語?將來祔禮,謁祖宗
 諸室,當用樂舞;至別廟奉安,宜停而不用。蓋用樂於前殿,是不以欽宗而廢祖宗之禮;停樂於別廟,是安穆為欽宗喪禮而屈也。如此,則於禮順,於義允。」遂俞其請。既而右正言周操上言:「祖宗前殿,尊無二上,其於用樂,無復有嫌。然用之享廟行禮之日則可,而用於今日之祔則不可。蓋祔禮為安穆而設,則其所用樂是為安穆而用,雖曰停於別廟,而為祔後用樂之名猶在也。孰若前後殿樂俱不作為無可議哉?」詔從之。



 隆興元年天申節,
 率群臣詣德壽宮上壽,議者以欽宗服除,當舉樂。事下禮曹,黃中復奏曰:「臣事君,猶子事父也。《春秋》,賊未討,不書葬,以明臣子之責。況欽宗實未葬,而可遽作樂乎?」事遂寢。



 乾道改元,始郊見天地。太常洪適奏:「聖上踐阼,務崇乾德,郊丘講禮,專以誠意交於神明。竊謂古今不相沿樂,金石八音不入俗耳,通國鮮習其藝,而聽之則倦且寐,獨以古樂嘗用之郊廟爾。昔者,竽工、鼓員不應經法,孔光、何武嘗奏罷於漢代,前史是之。今樂工為數甚
 伙,其鹵簿六引、前後鼓吹,有司已奏明,詔三分減一,惟是肄習尚逾三月之淹。夫驅游手之人振金擊石,安能盡中音律,使鳳儀而獸舞?而日給虛耗,總為緡錢近二百萬。若從裁酌,用一月教習,自可應聲合節,不至闕事。」於是詔郊祀樂工,令肄習一月。



 太常寺復言:「郊祀合用節奏樂工、登歌宮架樂工、引舞舞工,其分詣社稷及別廟,並番輪應奉,更不添置。」尋以禮官裁減壇下宮架二百七人,省十之一;琴二十人,瑟十二人,各省其半;笙、簫、
 笛可省者十有八人;篪、塤可省者十人。其分詣給祠凡一百十四,止用八十人。鐘、磬凡四十八架,止設三十有二人,其宮架鐘、磬仍舊。排殿閑慢樂色量省人數,悉報如章。



 禮部郎官蕭國梁又言:「議禮者嘗援紹興指揮,時饗亞獻既入太室,即引終獻行事,雖便於有司侍祠,免至跛倚,而其流將至於簡。宗廟用之郊饗尤為非宜。蓋有獻必有樂,卒爵而後樂闋。今亞、終獻樂舞雖同,而其作有始,其成有終,不可亂也。若使之相繼行事,雜然於
 酌獻之間,則其為樂舞者,不知亞獻之樂耶,終獻之樂耶?」詔從其請訂定。



 淳熙六年,始舉明堂禋禮,命五使按雅樂並嚴更、警場於貢院,奉詔將樂器依堂上、堂下儀制排設,五使及應赴官僚從旁立觀按閱,仍聽往來察視。時大禮使趙雄言:「前例,閱樂至皇帝詣飲福位一曲,即五使以下皆立,而每閱奠玉幣及酌獻等樂,皆坐自如,於禮未盡,不當襲用前例。」故有是詔。既而禮官討論,自紹興以來,凡五饗明堂,禮畢還輦,並未經用樂,即無
 作樂節次可考。乃參酌禮例,成禮稱賀及肆赦用樂導駕,並用皇祐大饗典故施行。其南郊、明堂儀注,實述紹興成憲,又命有司兼酌元豐、大觀舊典,為後世法程。其用樂作止之節,粲然可觀:



 前三日,太常設登歌樂於壇上,稍南,北向,設宮架於壇南內壝之外,立舞表於酇綴之間明堂登歌設於堂上前楹間,宮架設於庭中。前一日,設協律郎位二:一於壇上樂虡西北,一於宮架西北。押樂官位二:太常丞於登歌樂虡北,太常卿於宮架北。省牲之夕,押樂太常卿
 及丞入行樂架,協律郎展視樂器。



 祀之日,樂正帥工人、二舞以次入。皇帝乘輿,自青城齋殿出,樂正撞景鐘,降輿入大次,景鐘止明堂不用景鐘。服大裘袞冕,自正門入,協律郎跪,俯伏,舉麾,興。工鼓柷,宮架《乾安》之樂作,凡升降、行止皆奏之明堂奏《儀安》。至午階版位,西向立,協律郎偃麾戛敔,樂止明堂至阼階下,樂止。凡樂,皆協律郎舉麾而後作,偃麾而後止。禮儀使奏請行事,宮架作《景安之樂》。



 明堂作《誠安》。



 文舞進,左丞相等升,詣神位前,樂作,六成止。皇帝執大圭再
 拜,內侍進御匜帨,宮架樂作,帨手畢,樂止。禮儀使前導升壇,宮架樂作,至壇下,樂止。升自午階,明堂並升自阼階。



 登歌樂作,至壇上,樂止。登歌《嘉安》之樂作明堂至堂上作《鎮安》。奠鎮圭、奠玉幣於上帝,樂止。詣皇地祗、太祖、太宗神位前,如上儀。禮儀使導還版位,登歌樂作,降階,樂止明堂降自阼階。宮架樂作,至版位,樂止。奉俎官入正門,宮架《豐安》之樂作明堂作《禧安》。跪,奠俎訖,樂止。內侍以御匜帨進,宮架樂作,帨手拭爵,樂止。禮儀使導升壇,宮架樂作,至午階,樂止。升自
 午階,登歌樂作,至壇上,樂止明堂無升壇。登歌《禧安》之樂作明堂作《慶安》,詣神位前,三祭酒,少立,樂止。讀冊,皇帝再拜。每詣神位並如之。禮儀使導還版位,登歌樂作,降階,樂止。宮架樂作,至版位,樂止。奏請還小次,宮架樂作,入小次,樂止。



 武舞進,宮架《正安》之樂作明堂作《穆安》。舞者立定,樂止。亞獻,升,詣酌尊所,西向立,宮架《正安》之樂作明堂皇太子為亞獻,作《穆安》。三祭酒,以次酌獻如上儀,樂止。終獻亦如之。奏請詣飲福位,宮架樂作,至午階,樂止。升自午階,登歌樂作,
 將至位,樂止。登歌《禧安》之樂作明堂作《胙安》。飲福,禮畢,樂止。禮儀使導還版位,登歌樂作,降階,樂止。宮架樂作,至版位,樂止明堂不降階。徹豆,登歌《熙安》之樂作明堂作《歆安》。送神,宮架《景安》之樂作,一成止明堂作《誠安》。詣望燎、望瘞位,宮架樂作,至位,樂止明堂有燎無瘞。燎、瘞畢,還大次,宮架《乾安》之樂作明堂作《憩安》。至大次,樂止。皇帝乘大輦出大次,樂正撞景鐘明堂不用景鐘,鼓吹振作,降輦還齋殿,景鐘止。百官、宗室班賀於端誠殿,奏請聖駕進發,軍樂導引,至麗正門,大樂正
 令奏《採茨》之樂,入門,樂止明堂就賀於紫宸殿,不奏《採茨》。



 乃御麗正門肆赦。前期,太常設宮架樂於門之前,設鉦鼓於其西,皇帝升門至御閣,大樂正令撞黃鐘之鐘,右五鐘皆應,《乾安》之樂作,升御坐,樂止。金雞立,太常擊鼓,囚集,鼓聲止。宣制畢,大樂正令撞蕤賓之鐘,左五鐘皆應,皇帝還御幄,樂止。乘輦降門,作樂,導引至文德殿,降輦,樂止。



 按大禮用樂,凡三十有四色:歌色一,笛色二,塤色三,篪色四,笙色五,簫色六,編鐘七,編磬八,鎛鐘九,特磬十,琴十一,
 瑟十二,柷、敔十三,搏拊十四,晉鼓十五,建鼓十六,鞞、應鼓十七,雷鼓祀天神用。



 十八,雷□鼓同上



 一十九,靈鼓祭地祗用



 二十,靈□鼓同上



 二十一,露鼓饗宗廟用



 二十二,露□鼓同上



 二十三,雅鼓二十四,相鼓二十五,單□鼓二十六,旌纛二十七,金鉦二十八,金錞二十九,單鐸三十,雙鐸三十一,鐃鐸三十二,奏坐三十三,麾幡三十四。此國樂之用尤大者,故具載於篇。



 初,紹興崇建皇儲,詔有司備禮冊命,然在欽宗恤制,未及制樂。乾道初元,詔立皇太子,命禮部、
 太常寺討論舊禮以聞。受冊日,陳黃麾仗於大慶殿,設宮架樂於殿庭,皇帝升御坐,作《乾安》之樂,升,用黃鐘宮,降,用蕤賓宮。皇太子入殿門,作《明安》之樂,受冊出殿門亦如之,皆用應鐘宮。至七年,易應鐘而奏以姑洗。古者,太子生則太師吹管以度其聲,觀所協之律。有虞典樂教冑子,自天子之元子皆以樂為教,所以養其性情之正,蕩滌邪穢,消融查滓而和順於道德,則陳金石雅奏,以重元良。冊拜宜仿古誼,式昭盛禮。由唐季世,儲貳罕
 定,國家益多故而禮廢樂闕。至於建隆定樂,雖詔皇太子出入奏《良安》,至道始冊皇太子,有司言:「太子受冊,宜奏《正安》之樂。百年曠典,至是舉行,中外胥悅。至天禧冊命,禮儀院復奏改《正安》之樂。乾道之用《明安》,實祖述天禧,而以姑洗為宮,則唐東宮軒垂奏樂舊貫雲。



 孝宗素恭儉,每賀正使赴宴作樂,多遇上辛齋禁,有司條治平用樂典故以進。及生辰使上壽,適親郊散齋,樞密副使陳俊卿請以禮諭北使,毋用樂。不得已,則上壽之日設
 樂,而宣旨罷之,及宴使人,然後用之,庶存事天之誠。上可其奏,且曰:「宴殿雖進御酒,亦勿用。」宰相葉顒、魏杞方主用樂之議,以為樂奏於紫宸,乃使客之禮。俊卿獨奏曰:「適奉詔旨,仰見聖學高明,過古帝王遠甚。彼初未嘗必欲用樂,而我乃望風希意,自為失禮以徇之,他日輕侮,何所不至?」尋詔:「垂拱上壽止樂,正殿猶為北使權用。」後三年,賀使當朝辭,復值散齋,上乃諭館伴以決意去樂及議所以處之者,如使人必以作樂為言,則移茶酒
 就驛管領,遂有更不用樂之詔。



 其後因雨澤愆期,分禱天地、宗廟,精修雩祀。按禮,大雩,帝用盛樂。而唐開元祈雨雩壇,謂之特祀,乃不以樂薦。於是太常朱時敏言:「《通典》載雩禮用舞僮歌《雲漢》,晉蔡謨議謂:『《雲漢》之詩,興於宣王,歌之者取其修德禳災,以和陰陽之義。』乞用舞僮六十四人,衣玄衣,歌《雲漢》之詩。」詔亟從之。



 淳熙二年,詔以上皇加上尊號,立春日行慶壽禮。有司尋言:「乾道加尊號,用宮架三十六,樂工共一百一十三人。今來加號
 慶壽,事體尤重,合依大禮例,用四十八架,樂正、樂工用一百八十八人,庶得禮樂明備。」仍令分就太常寺、貢院前五日教習。前期,太常設宮架之樂於大慶殿,協律郎位於宮架西北,東向;押樂太常卿位於宮架之北,北向;皇太子及文武百僚,並位於宮架之北,東西相向,又設宮架於德壽殿門外,協律郎、太常卿位如之。及發冊寶日,儀仗、鼓吹列於大慶殿門,樂正、師二人以次入。贊者引押樂太常卿、協律郎入,就位,奏中嚴外辦訖,禮儀使
 奏請皇帝恭行發冊寶之禮,太常卿導冊寶,《正安》之樂作。中書令奉寶、侍中奉冊進行,《禮安》之樂作。發寶冊畢,鼓吹振作,儀衛等以次從行。皇帝自祥曦殿輦至德壽宮行禮,冊寶入殿門,作《正安》之樂。上皇出宮,作《乾安》之樂;升御坐,奉上冊寶,作《聖安》之樂;降御坐,作《乾安》之樂。太后冊寶進行,用《正安》;出閣升坐,用《坤安》;降坐入閣,復作《坤安》之樂。禮部尚書趙雄等言:「國朝舊制,車駕出,奏樂。今慶典之行,亙古未有,自非禮儀祥備,無以副中外
 歡愉之心。請慶壽行禮日,聖駕往還並用樂及簪花。」詔從之。既而太常又言:「郊禋禮成,宜進胙慈闈,行上壽飲酒禮。所有上壽合辦仙樓仍用樂,某樂人照天申節禮例。」凡上詣德壽宮,或恭請上皇游幸,或至南內,或上皇命同宴游,或時序賞適、過宮侍宴,或聖節張樂、珥花、奉玉卮為上皇壽,率從容竟日,隆重養至樂,備極情文。



 及高宗之喪,孝宗力行三年之制,有司雖未嘗別設樂禁,而過期不忍聞樂。金使以會慶節來賀,稽之舊典,引對使
 人或許上壽,惟輟樂不舉。孝宗斷以禮典,卻其書幣,就館遣行。次年再至,始用紹興故事,移宴於館而不作樂。高宗升祔,太常言:「祔饗行禮,當設登歌、宮架、樂舞,晨稞饋食,其用樂如朝饗之制。」於是,高宗廟昉奏《大德》之樂舞。禮部言:「今虞祔之行,純用古禮,導引神主,自有衛仗及太常鼓吹,而雜用道、釋,於禮非經,乞行蠲免。」詔從其請。



 即而大享明堂,起居舍人鄭僑奏:「祭祀於事為大,禮樂於用為急,然先王處此,有常變之不同,各務當其禮
 而已。昔舜居堯喪,三載遏密,後世既用漢文以日易月之文,又用漢儒越紼行事之制,循習既久,不特用禮而又用樂,去古愈遠。聖主躬服通喪,有司請舉大禮,屈意從之。且大饗之禮,祭天地也,聖主身親行之,行禮作樂,似不可廢。其它官分獻與夫先期奏告例用樂者,權宜蠲寢,不亦可乎?今若因明堂損益而裁定之,亦足為將來法。」乃命太常討論,始詔除降神、奠玉幣、奉俎、酌獻、換舞、徹豆、送神依曲禮作樂外,所有皇帝及獻官盥洗、登
 降等樂皆備而不作雲。



\end{pinyinscope}