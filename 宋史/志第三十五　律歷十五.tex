\article{志第三十五 律歷十五}

\begin{pinyinscope}

 乾道四年,禮部員外郎李燾言:「《統元歷》行之既久,與天不合,固宜;《大衍歷》最號精微,用之亦不過三十餘年,後之欲行遠也難矣。抑歷未差,無以知其失;未驗,無以知
 其是。仁宗用《崇天歷》,天聖至皇祐四年十一月日食,二歷不效,詔以唐八歷及宋四歷參定,皆以《景福》為密,遂欲改作。而劉羲叟謂:「《崇天歷》頒行逾三十年,所差無幾,詎可偶緣天變,輕議改移?」又謂:「古聖人歷象之意,止於敬授人時,雖則預考交會,不必吻合辰刻,或有遲速,未必獨是歷差。」乃從羲叟言,復用《崇天歷》。羲叟歷學為宋第一,歐陽修、司馬光輩皆遵用之。《崇天歷》既復用,又十三年,治平二年,始改用《明天歷》,歷官周琮皆遷官。後三年,
 驗熙寧三年七月月食不效,乃詔復用《崇天歷》,奪琮等所遷官。熙寧八年,始更用《奉元歷》,沉括實主其議。明年正月月食,遽不效,詔問修歷推恩者姓名,括具奏辨,得不廢。識者謂括強辨,不許其深於歷也。然後知羲叟之言然。願申飭歷官,加意精思,勿執今是。益募能者,熟復討論,更造密度,補治新歷。」緣燾嘗承詔監視測驗,值新歷太陰、熒惑之差,恐書成所差或多,見譏能者,乃詔諸道訪通歷者。久之,福州布衣阮興祖上言新歷差謬,荊
 大聲不以白部,即補興祖為局生。



 初,新歷之成也,大聲、孝榮共為之。至是,大聲乃以太陰九道變赤道別演一法,與孝榮立異於後。秘書少監、崇政殿說書兼權刑部侍郎汪大猷等言:「承詔於御史臺監集局官,參算明年太陰宿度,箋注御覽詣實。今大聲等推算明年正月至月終九道太陰變赤道,限十二月十五日以前具稿成,至正月內,臣等召歷官上臺,用渾儀監驗疏密。」從之。



 五年,國子司業兼權禮部侍郎程大昌、侍御史單時,秘書
 丞唐孚、秘書郎李木言:「都省下靈臺郎充歷算官蓋堯臣、皇甫繼明、宋允恭等言:『厥今更造《乾道新歷》,朝廷累委官定驗,得見日月交食密近天道,五星行度允協躔次,惟九道太陰間有未密。搜訪能歷之人補治新歷,半年未有應詔者,獨荊大聲別演一法,與劉孝榮《乾道歷》定驗正月內九道太陰行度。今來二法皆未能密於天道,《乾道》太陰一法與諸歷比較,皆未盡善。今撮其精微,撰成一法,其先推步到正月內九道太陰正對在赤道
 宿度,願委官與孝榮、大聲驗之。如或精密,即以所修九道經法,請得與定驗官更集孝榮、大聲等同赴臺,推步明年九道太陰正對在赤道宿度,點定月分定驗,從其善者用之。』大昌等從大聲、孝榮所供正月內太陰九道宿度,已赴太史局測驗上中旬畢,及取大聲、孝榮、堯臣等三家所供正月下旬太陰宿度,參照覽視,測驗疏密,堯臣、繼明、允恭請具今年太陰九道宿度。欲依逐人所請,限一月各具今年太陰九道變黃道正對赤道其宿
 某度,依經具稿,送御史臺測驗官不時視驗,然後見其疏密。」



 裴伯壽上書言:



 孝榮自陳預定丁亥歲四月朔日食、八月望月食,俱不驗。又定去年二月望夜二更五點月食九分以上,出地復滿。臣嘗言於宰相,是月之食當食既出地,《紀元歷》亦食既出地,生光在戌初二刻,復滿在戌正三刻。是夕,月出地時有微雲,至昏時見月已食既,至戌初三刻果生光,即食既出地可知;復滿在戌正三刻,時二更二點:臣所言卒驗。孝榮言見行歷交食先
 天六刻,今所定月食復滿,乃後天四刻,新歷謬誤為甚。



 其一曰步氣朔,孝榮先言氣差一日,觀景表方知其失,此不知驗氣者也。臣之驗氣,差一二刻亦能知之。《紀元》節氣,自崇寧間測驗,逮今六十餘載,不無少差,茍非測驗,安知其失?凡日月合朔,以交食為驗,今交食既差,朔亦弗合矣。



 其二曰步發斂,止言卦候而已。



 其三曰步日躔,新歷乃用《紀元》二十八宿赤道度,暨至分宮,遽減《紀元》過宮三十餘刻,殊無理據。而又赤道變黃道宿度,婁、
 胃二宿頓減《紀元》半度。在術則婁、胃二宿合二十八度,婁當十二度太,今新歷婁作十二度半,乃棄四分度之一。室、軫二宿虛收復多,少數變宿,分宮既訛,是以乾道己丑歲太陽過宮差誤。



 其四曰步晷漏,新歷不合前史。唐開元十二年測景於天下,安南測夏至午中晷在表南三寸三分,新歷算在表北七寸;其鐵勒測冬至午中晷長一丈九尺二寸六分,新歷算晷長一丈四尺九寸九分,乃差四尺二寸七分,其謬蓋若此。



 其五曰步月離,
 諸歷遲疾、朏朒極數一同,新歷朏之極數少朒之極數四百九十三分,疾之極數少遲之極數二十分,不合歷法。



 其六曰步交會,新歷妄設陽準、陰準等差,蓋欲茍合已往交食,其間復有不合者,則遷就天道,所以預定丁亥、戊子二歲日月之食,便見差違。



 其七曰步五星,以渾儀測驗新歷星度,與天不合。蓋孝榮與同造歷人皆不能探端知緒,乃先造歷,後方測驗,前後倒置,遂多差失。夫立表驗氣,窺測七政,然後作歷,豈容掇拾緒餘,超接舊
 歷,以為新術,可乎?



 新歷出於五代民間《萬分歷》,其數朔餘太強,明歷之士往往鄙之。今孝榮乃三因萬分小歷,作三萬分為日法,以隱萬分之名。三萬分歷即萬分歷也。緣朔餘太強,孝榮遂減其分,乃增立秒,不入歷格。前古至於宋諸歷,朔餘並皆無秒,且孝榮不知王處訥於萬分增二,為《應天歷》日法,朔餘五千三百七,自然無秒,而去王樸用秒之歷。



 臣與造《統元歷》之後,潛心探討復三十餘年,考之諸歷,得失曉然。誠假臣演撰之職,當與
 太史官立表驗氣,窺測七政,運算立法,當遠過前歷。



 詔送監視測驗官詳之,達於尚書省。



 時談天者各以技術相高,互相詆毀。諫議大夫單時、秘書少監汪大猷、國子司業權禮部侍郎程大昌、秘書丞唐孚、秘書郎李木言:「《乾道新歷》,荊大聲、劉孝榮同主一法,自初測驗以至權行施用,二人無異議。後緣新歷不密,詔訪求通歷者,孝榮乃訟阮興祖緣大聲補局生,自是紛紛不已。大聲官以判局提點歷書為名,乃言不當責以立法起算。不知
 起歷授時,何所憑據。且正月內五夜,比較孝榮所定五日並差,大聲所定五日內三日的中,兩日稍疏。繼伯壽進狀獻術,時等將求其歷書上臺測驗,務求至當,而大聲等正居其官,乃飾辭避事,測驗弗精。且大聲、孝榮同立新法,今猶反復,茍非各具所見,他日歷成,大聲妄有動搖,即前功盡廢。請令孝榮、大聲、堯臣、伯壽各具乾道五年五月已後至年終,太陰五星排日正對赤道躔度,上之御史臺,令測驗官參考。」詔從之。



 六年,日官言:「比詔
 權用《乾道歷》推算,今歲頒歷於天下,明年用何歷推算?」詔亦權用《乾道歷》一年。秋,成都歷學進士賈復自言,詔求推明熒惑、太陰二事,轉運使資遣至臨安,願造新歷畢還蜀,仍進《歷法九議》。孝宗嘉其志,館於京學,賜廩給。太史局李繼宗等言:「十二月望,月食大分七、小分九十三。賈復、劉大中等各虧初、食甚分夜不同。」詔禮部侍郎鄭聞監李繼宗等測驗。是夜,食八分。秘書省言,靈臺郎宋允恭、國學生林永叔、草澤祝斌、黃夢得、吳時舉、陳彥
 健等各推算日食時刻、分數異同。乃詔諫議大夫姚憲監繼宗等測驗五月朔日食。憲奏時刻、分數皆差舛,繼宗、澤、大聲削降有差。



 太史局春官正、判太史局吳澤等言:「乾道十年頒賜歷日,其中十二月已定作小盡,乾道十一年正月一日注:癸未朔,畢乾道十一年正月一日。《崇天》、《統元》二歷算得甲申朔,《紀元》、《乾道》二歷算得癸未朔,今《乾道歷》正朔小餘,約得不及進限四十二分,是為疑朔。更考日月之行,以定月朔大小,以此推之,則當是
 甲申朔。今歷官弗加精究,直以癸未注正朔,竊恐差誤,請再推步。於是俾繼宗監視,皆以是年正月朔當用甲申。兼今歲五月朔,太陽交食,本局官生瞻視到天道日食四分半:虧初西北,午時五刻半;食甚正北,未初二刻;復滿東北,申初一刻。後令永叔等五人各言五月朔日食分數並虧初、食甚、復滿時刻皆不同。並見行《乾道歷》比之,五月朔天道日食多算二分少強,虧初少算四刻半,食甚少算三刻,復滿少算二刻已上。又考《乾道歷》比
 之《崇天》、《紀元》、《統元》三歷,日食虧初時刻為近;較之《乾道》,日食虧初時刻為不及。繼宗等參考來年十二月系大盡,及十一年正月朔當用甲申,而太史局丞、同判太史局荊大聲言《乾道歷》加時系不及進限四十二分,定今年五月朔日食虧初在午時一刻。今測驗五月朔日食虧初在午時五刻半,《乾道歷》加時弱四百五十分,茍以天道時刻預定乾道十二年正月朔,已過甲申日四百五十分。大聲今再指定乾道十一年正月合作甲申朔,
 十年十二月合作大盡,請依太史局詳定行之。」五月,詔歷官詳定。



 淳熙元年,禮部言:「今歲頒賜歷書,權用《乾道新歷》推算,明年復欲權用《乾道歷》。」詔從之。十一月,詔太史局春官正吳澤推算太陽交食不同,令秘書省敕責之,並罰造歷者。三年,判太史局李繼宗等奏:「令集在局通算歷人重造新歷,今撰成新歷七卷,《推算備草》二卷,校之《紀元》、《統元》、《乾道》諸歷,新歷為密,願賜歷名。」於是詔名《淳熙歷》,四年頒行,令禮部、秘書省參詳以聞。



 淳熙四
 年正月,太史局言:「三年九月望,太陰交食。以《紀元》、《統元》、《乾道》三歷推之,初虧在攢點九刻,食二分及三分已上;以新歷推之,在明刻內食大分空,止在小分百分中二十七。是夜,瞻候月體盛明,雖有雲而不翳,至旦不見虧食,於是可見《紀元》、《統元》、《乾道》三歷不逮新歷之密。今當預期推算淳熙五年歷,蓋舊歷疏遠,新歷未行,請賜新歷名,付下推步。」



 禮部驗得孟邦傑、李繼宗等所定五星行度分數各有異同。繼宗云:六月癸酉,木星在氐宿三度一
 十九分。邦傑言:夜昏度瞻測得木星在氐宿三度半,半系五十分,雖見月體,而西南方有雲翳之。繼宗云:是月戊寅,木星在氐宿三度四十一分;邦傑言:四望有云,雖雲間時露月體,所可測者木星在氐宿三度太,太系七十五分。繼宗云:庚辰土星在畢宿三度二十四分,金星在參宿五度六十五分,火星在井宿七度二十七分;邦傑言:五更五點後,測見土星入畢宿二度半,半系五十分,金星入參宿六度半,火星入井宿八度多三分。繼宗
 云:七月辛丑,太陰在角宿初度七十一分,木星在氐宿五度七十六分;邦傑言:測見昏度太陰入軫宿十六度太,太系七十五分,木星入氐宿六度少,少系二十五分。孝宗曰:「自古歷無不差者,況近世此學不傳,求之草澤,亦難其人。」詔以《淳熙歷》權行頒用一年。



 五年,金遣使來朝賀會慶節,妄稱其國歷九月庚寅晦為己丑晦。接伴使、檢詳丘辨之,使者辭窮,於是朝廷益重歷事。李繼宗、吳澤言:「今年九月大盡,系三十日,於二十八日早晨
 度瞻見太陰離東濁高六十餘度,則是太陰東行未到太陽之數。然太陰一晝夜東行十三度餘,以太陰行度較之,又減去二十九日早晨度太陰所行十三度餘,則太陰尚有四十六度以上未行到太陽之數,九月大盡,明矣。其金國九月作小盡,不當見月體;今既見月體,不為晦日。乞九月三十日、十月一日差官驗之。」詔遣禮部郎官呂祖謙。祖謙言:「本朝十月小盡,一日辛卯朔,夜昏度太陰躔在尾宿七度七十分。以太陰一晝夜平行十
 三度三十一分,至八日上弦日,太陰計行九十一度餘。按歷法,朔至上弦,太陰平行九十一度三十一分,當在室宿一度太。金國十月大盡,一日庚寅朔,夜昏度太陰約在心宿初度三十一分。太陰一晝夜亦平行十三度三十一分,自朔至本朝八日為金國九日,太陰已行一百四度六十二分,比之本朝十月八日上弦,太陰多行一晝夜之數。今測見太陰在室宿二度,計行九十二度餘,始知本朝十月八日上弦,密於天道。」詔祖謙復測驗。
 是夜,邦傑用渾天儀法物測驗,太陰在室宿四度,其八日上弦夜所測太陰在室宿二度。按歷法,太陰平行十三度餘,行遲行十二度。今所測太陰,比之八日夜又東行十二度,信合天道。



 十年十月,詔:甲辰歲歷字誤,令禮部更印造,頒諸安南國。繼宗、澤及荊大聲削降有差。



 十二年九月,成忠郎楊忠輔言:「《淳熙歷》簡陋,於天道不合。今歲三月望,月食三更二點,而歷在二更二點;數虧四分,而歷虧幾五分。四月二十三日,水星據歷當夕伏,而
 水星方與太白同行東井間,昏見之時,去濁猶十五餘度。七月望前,土星已伏,而歷猶注見。八月未弦,金已過氐矣,而歷猶在亢。此類甚多,而朔差者八年矣。夫守疏敝之歷,不能革舊,其可哉!忠輔於《易》粗窺大衍之旨,創立日法,撰演新歷,不敢以言者,誠懼太史順過食布非。恃刻漏則水有增損、遲疾,恃渾儀則度有廣狹、斜正。所賴今歲九月之交食在晝,而《淳熙歷》法當在夜,以晝夜辨之,不待紛爭而決矣。輒以忠輔新歷推算,淳熙十二年
 九月定望日辰退乙未,太陰交食大分四、小分八十五,晨度帶入漸進大分一、小分七;虧初在東北,卯正一刻一十一分,系日出前;食甚在正北,辰初一刻一十分;復滿在西北,辰正初刻,並日出後。其日日出卯正二刻後,與虧初相去不滿一刻。以地形論之,臨安在嶽臺之南,秋分後晝刻比嶽臺差長,日當先歷而出,故知月起虧時,日光已盛,必不見食。以《淳熙歷》推之,九月望夜,月食大分五、小分二十六,帶入漸進大分三、小分四十七;虧
 初在東北,卯初三刻,系攢點九刻後;食甚在正北,卯正三刻後;復滿在西北,辰正初刻後,並在晝。」禮部乃考其異同,孝宗曰:「日月之行有疏數,故歷久不能無差,大抵月之行速,多是不及,無有過者。可遣臺官、禮部官同驗之。」詔遣禮部侍郎顏師魯。其夜戌正二刻,陰雲蔽月,不辨虧食。師魯請詔精於歷學者與太史定歷,孝宗曰:「歷久必差,聞來年月食者二,可俟驗否?」



 十三年,右諫議大夫蔣繼周言,試用民間有知星歷者,遴選提領官,以重
 其事,如祖宗之制。孝宗曰:「朝士鮮知星歷者,不必專領。」乃詔有通天文歷算者,所在州、軍以聞。八月,布衣皇甫繼明等陳:「今歲九月望,以《淳熙歷》推之,當在十七日,實歷敝也。太史乃注于十六日之下,徇私遷就,以掩其過。請造新歷。」而忠輔乞與歷官劉孝榮及繼明等各具己見,合用歷法,指定今年八月十六日太陰虧食加時早晚、有無帶出、所見分數及節次、生光復滿方面、辰刻、更點同驗之,仰合乾象,折衷疏密。再請今年八月二十九
 日驗月見東方一事,茍見月餘光,則其日不當以為晦也。又今年九月十六日驗月未盈一事,茍見月體東向之光猶薄,則其日不當為望也。知晦望之差,則朔之差明矣。必使氣之與朔無毫發之差,始可演造新歷。付禮部議,各具先見,指定太陰虧食分數、方面、辰刻,定驗折衷。詔師魯、繼周監之。既而孝榮差一點,繼明等差二點,忠輔差三點,乃罷遣之。



 十四年,國學進士會稽石萬言:



 《淳熙歷》立元非是,氣朔多差,不與天合。按淳熙十四年
 歷,清明、夏至、處暑、立秋四氣,及正月望、二月十二月下弦、六月八月上弦、十月朔,並差一日。如卦候、盈、虛、沒、滅、五行用事,亦各隨氣朔而差。南渡以來,渾儀草創,不合制度,無圭表以測日景長短,無機漏以定交食加時,設欲考正其差,而太史局官尚如去年測驗太陰虧食,自一更一點還光一分之後,或一點還光二分,或一點還光三分以上,或一點還光三分以下,使更點乍疾乍徐,隨景走弄,以肆欺蔽。若依晉泰始、隋開皇、唐開元課歷
 故事,取《淳熙歷》與萬所造之歷各推而上之於千百世之上,以求交食,與夫歲、月、日、星辰之著見於經史者為合與否,然後推而下之,以定氣朔,則與前古不合者為差,合者為不差,甚易見也。



 然其差謬非獨此耳,冬至日行極南,黃道出赤道二十四度,晝極短,故四十刻,夜極長,故六十刻;夏至日行極北,黃道入赤道二十四度,晝極長,故六十刻,夜極短,故四十刻;春、秋二分,黃、赤二道平而晝夜等,故各五十刻。此地中古今不易之法。至王
 普復位刻漏,又有南北分野、冬夏晝夜長短三刻之差。今《淳熙歷》皆不然,冬至晝四十刻極短、夜六十刻極長,乃在大雪前二日,所差一氣以上;自冬至之後,晝當漸長,夜當漸短,今過小寒,晝猶四十刻,夜猶六十刻,所差七日有餘;夏至晝六十刻極長、夜四十刻極短,乃在芒種前一日,所差亦一氣以上;自夏至之後,晝當漸短,夜當漸長,今過小暑,晝猶六十刻,夜猶四十刻,所差亦七日有餘;及晝、夜各五十刻,又不在春分、秋分之下。



 至於
 日之出入,人視之以為晝夜,有長短,有漸,不可得而急與遲也,急與遲則為變。今日之出入增減一刻,近或五日,遠或三四十日,而一急一遲,與日行常度無一合者。請考正《淳熙歷》法之差,俾之上不違於天時、下不乖於人事。



 送秘書省、禮部詳之。



 皇甫繼明、史元寔、皇甫迨、龐元亨等言:「石萬所撰《五星再聚歷》,乃用一萬三千五百為日法,特竊取唐末《崇元》舊歷而婉其名爾。《淳熙歷》立法乖疏,丙午歲定望則在十七日,太史知其不可,遂注
 望於十六日下,以掩其過。臣等嘗陳請於太史局官對辨,置局更歷,迄今未行。今考《淳熙歷經》則又差於將來。戊申歲十一月下弦則在二十四日,太史局官必俟頒歷之際,又將妄退於二十三日矣。法不足恃,必假遷就,而朔望二弦,歷法綱紀,茍失其一,則五星盈縮、日月交會、與夫昏旦之中星、晝夜之晷刻,皆不可得而正也。渾儀、景表,壺漏之器,臣等私家無之,是以歷之成書,猶有所待。國朝以來,必假創局而歷始成,請依改造大歷故
 事,置局更歷,以祛太史局之敝。」事上聞,宰相王淮奏免送後省看詳,孝宗曰:「使秘書省各司同察之,亦免有異同之論。」六月,給事中兼修玉牒官王信亦言更歷事,以為歷法深奧,若非詳加測驗,無以見其疏密。乞令繼明與萬各造來年一歲之歷,取其無差者。詔從之。十二月,進所造歷。淮等奏:「萬等歷日與淳熙十五年歷差二朔,《淳熙歷》十一月下弦在二十四日,恐歷法有差。」孝宗曰:「朔豈可差?朔差則所失多矣。」乃命吏部侍郎章森、秘書
 丞宋伯嘉參定以聞。



 十五年,禮部言:「萬等所造歷與《淳熙歷》法不同,當以其年六月二日、十月晦日月不應見而見為驗,兼論《淳熙歷》下弦不合在十一月二十四日,是日請遣官監視。」詔禮部侍郎尤袤與森監之。六月二日,森奏:「是夜月明,至一更二點入濁。」十月晦,袤奏:「晨前月見東方。」孝宗問:「諸家孰為疏密?」周必大等奏:「三人各定二十九日早,月體尚存一分,獨忠輔、萬謂既有月體,不應小盡。」孝宗曰:「十一月合朔在申時,是以二十九日
 尚存月體耳。」



 十六年,承節郎趙渙言:「歷象大法及《淳熙歷》,今歲冬至並十二月望,月食皆後天一辰,請遣官測驗。」詔禮部侍郎李巘、秘書省鄧馹等視之。巘等請用太史局渾儀測驗,如乾道故事,差秘書省提舉一員專監之。詔差秘書丞黃艾、校書郎王叔簡。



 紹熙元年八月,詔太史局更造新歷頒之。二年正月,進《立成》二卷、《紹熙二年七曜細行歷》一卷,賜名《會元》,詔巘序之。



 紹熙四年,布衣王孝禮言:「今年十一月冬至,日景表當在十九日壬
 午,《會元歷》注乃在二十日癸未,系差一日。《崇天歷》癸未日冬至加時在酉初七十六分,《紀元歷》在丑初一刻六十七分,《統元歷》在丑初二刻二分,《會元歷》在丑初一刻二百四十分。迨今八十有七年,常在丑初一刻,不減而反增。《崇天歷》寔天聖二年造,《紀元歷》崇寧五年造,計八十二年。是時測景驗氣,如冬至後天乃減六十七刻半,方與天道協。其後陳得一造《統元歷》,劉孝榮造《乾道》、《淳熙》、《會元》三歷,未嘗測景。茍弗立表測景,莫識其差。乞遣
 官令太史局以銅表同孝禮測驗。」朝遷雖從之,未暇改作。」



 慶元四年,《會元歷》占候多差,日官、草澤互有異同,詔禮部侍郎胡紘充提領官,正字馮履充參定官,監楊忠輔造新歷。右諫議大夫兼侍講姚愈言:「太史局文籍散逸,測驗之器又復不備,幾何而不疏略哉!漢元鳳間,言歷者十有一家,議久不決,考之經籍,驗之帝王錄,然後是非洞見。元和間,以《太初》違天益遠,晦朔失實,使治歷者修之,以無文證驗,雜議蜂饗起,越三年始定。此無他,不
 得儒者以總其綱,故至於此也。《周官》馮相氏、保章氏志日月星辰之運動,而塚宰實總之。漢初,歷官猶宰屬也。熙寧間,司馬光、沉括皆嘗提舉司天監,故當是時歷數明審,法度嚴密。乞命儒臣常兼提舉,以專其責。」



 五年,監察御史張巖論馮履唱為詖辭,罷去。詔通歷算者所在具名來上。及忠輔歷成,宰臣京鏜上進,賜名《統天》,頒之,凡《歷經》三卷,《八歷冬至考》一卷,《三歷交食考》三卷,《晷景考》一卷,《考古今交食細草》八卷,《盈縮分損益率立成》二
 卷,《日出入晨昏分立成》一卷,《岳臺日出入晝夜刻》一卷,《赤道內外去極度》一卷,《臨安午中晷景常數》一卷,《禁漏街鼓更點辰刻》一卷,《禁漏五更攢點昏曉中星》一卷,《將來十年氣朔》二卷,《己未庚申二年細行》二卷,總三十二卷。慶元五年七月辛卯朔,《統天歷》推日食,雲陰不見。六年六月乙酉朔,推日食不驗。



 嘉泰二年五月甲辰朔,日有食之,詔太史與草澤聚驗於朝,太陽午初一刻起虧,未初刻復滿。《統天歷》先天一辰有半,乃罷楊忠輔,詔草
 澤通曉歷者應聘修治。



 開禧三年,大理評事鮑澣之言:「歷者,天地之大紀,聖人所以觀象明時,倚數立法,以前民用而詔方來者。自黃帝以來,至於秦、漢,六歷具存,其法簡易,同出一術。既久而與天道不相符合,於是《太初》、《三統》之法相繼改作,而推步之術愈見闊疏,是以劉洪,祖沖之之減破斗分,追求月道,而推測之法始加詳焉。至於李淳風、一行而後,總氣朔而合法,效乾坤而擬數,演算之法始加備焉。故後世之論歷,轉為精密,非過於
 古人也,蓋積習考驗而得之者審也。試以近法言之:自唐《麟德》、《開元》而至於五代所作者,國初《應天》而至於《紹熙》、《會元》,所更者十二書,無非推求上元開闢為演紀之首,氣朔同元,而七政會於初度。從此推步,以為歷本,未嘗敢輒為截法,而立加減數於其間也。獨石晉天福間,馬重績更造《調元歷》,不復推古上元甲子七曜之會,施於當時,五年輒差,遂不可用,識者咎之。今朝廷自慶元三年以來,測驗氣景,見舊歷後天十一刻,改造新歷,賜
 名《統天》,進歷未幾,而推測日食已不驗,此猶可也。但其歷書演紀之始,起於唐堯二百餘年,非開闢之端也。氣朔五星,皆立虛加、虛減之數;氣朔積分,乃有泛積、定積之繁。以外算而加朔餘,以距算而減轉率,無復強弱之法,盡廢方程之舊。其餘差漏,不可備言。以是而為術,乃民間之小歷,而非朝廷頒正朔、授民時之書也。漢人以謂歷元不正,故盜賊相續,言雖迂誕,然而歷紀不治,實國家之重事。願詔有司選演撰之官,募通歷之士,置局
 討論,更造新歷,庶幾並智合議,調治日法,追迎天道,可以行遠。」



 澣之又言:「當楊忠輔演造《統天歷》之時,每與議論歷事,今見《統天歷》舛近,亦私成新歷。誠改新歷,容臣投進,與太史、草澤諸人所著之歷參考之。」七月,澣之又言:「《統天歷》來年閏差,願以諸人所進歷,令秘書省參考頒用。」



 秘書監兼國史院編修官、實錄院檢討官曾漸言:「改歷,重事也,昔之主監事者,無非道術精微之人,如太史公、洛下閎、劉歆、張衡、杜預、劉焯、李淳風、一行、王樸等,
 然猶久之不能無差。其餘不過遞相祖述,依約乘除,舍短取長,移疏就密而已,非有卓然特達之見也。一時偶中,即復舛戾。宋朝敝在數改歷法。《統天歷》頒用之初,即已測日食不驗,因仍至今置閏遂差一月,其為當改無疑。然朝廷以一代鉅典責之專司,必其人確然著論,破見行之非,服眾多之口,庶幾可見。按乾道、淳熙、慶元,凡三改歷,皆出劉孝榮一人之手,其後遂為楊忠輔所勝。久之,忠輔歷亦不驗,故孝榮安職至今。紹熙以來,王孝
 禮者數以自陳,每預測驗,或中或不中;李孝節、陳伯祥本皆忠輔之徒;趙達,卜筮之流;石如愚獻其父書,不就測驗晷景,止定月食分數,其術最疏;陳光則並與交食不論,愈無憑依。此數人者,未知孰為可付,故鮑澣之屢以為請。今若降旨開局,不過收聚此數人者,和會其說,使之無爭。來年閏差,其事至重。今年八月,便當頒歷外國,而三數月之間急遽成書,結局推賞,討論未盡,必生詆訾。今劉孝榮、王孝禮、李孝節、陳伯祥所擬改歷,及澣
 之所進歷,皆已成書,願以眾歷參考,擇其與天道最近且密者頒用,庶幾來年置閏不差。請如先朝故事,搜訪天下精通歷書之人,用沉括所議,以渾儀、浮漏、圭表測驗,每日記錄,積三五年,前後參較,庶幾可傳永久。」



 漸又言:「慶元三年以後,氣景比舊歷有差,至四年改造新歷未成時,當頒五年歷,乃差官以測算晷景、氣朔加時辰刻附《會元歷》頒賜。今若頒來年氣朔,既有去年十月以後、今年正月以前所測晷景,已見天道冬至加時分數,
 來年置閏,比之《統天歷》亦已不同,兼諸所進歷並可參考。請速下本省,集判局官於本省參考,使澣之覆考,以最近之歷推算氣朔頒用。」於是詔漸充提領官,澣之充參定官,草澤精算造者、嘗獻歷者與造《統天歷》者皆延之,於是《開禧》新歷議論始定。詔以戊辰年權附《統天歷》頒之。既而婺州布衣阮泰發獻《渾儀十論》,且言《統天》、《開禧》歷皆差。朝廷令造木渾儀,賜文解罷遣之。



 嘉定三年,鄒淮言歷書差忒,當改造。試太子詹事兼同修國史、實
 錄院同修撰兼秘書監戴溪等言,請詢漸、澣之造歷故事。詔溪充提領官,澣之充參定官,鄒淮演撰,王孝禮、劉孝榮提督推算官生十有四人,日法用三萬五千四百。四年春,歷成,未及頒行,溪等去國,歷亦隨寢。韓侂冑當國,或謂非所急,無復敢言歷差者,於是《開禧歷》附《統天歷》行於世四十五年。



 嘉泰元年,中奉大夫、守秘書監俞豐等請改造新歷。監察御史施康年劾太史局官吳澤、荊大聲、周端友循默尸祿,言災異不及時,詔各降一
 官。臣僚言:「頒正朔,所以前民用也。比歷書一日之間,吉兇並出,異端並用,如土鬼、暗金兀之類,則添注於兇神之上猶可也,而其首則揭九良之名,其末則出九曜吉兇之法、勘昏行嫁之法,至於《周公出行》、《一百二十歲宮宿圖》,凡閭閻鄙俚之說,無所不有。是豈正風俗、示四夷之道哉!願削不經之論。」從之。二年五月朔,日食,太史以為午正,草澤趙大猷言午初三刻半日食三分。詔著作郎張嗣古監視測驗,大猷言然,歷官乃抵罪。



 嘉定四年,秘
 書省著作郎兼權尚左郎丁端祖請考試司天生。十三年,監察御史羅相言:「太史局推測七月朔太陽交食,至是不食。願令與草澤新歷精加討論。」於是澤等各降一官。



 淳祐四年,兼崇政殿說書韓祥請召山林布衣造新歷。從之。五年,降算造成永祥一官,以元算日食未初三刻,今未正四刻,元算虧八分,今止六分故也。



 八年,朝奉大夫、太府少卿兼尚書左司郎中兼敕令所刪修官尹渙言:「歷者,所以統天地、侔造化,自昔皆擇聖智典司其
 事。後世急其所當緩,緩其所當急,以為利吾國者,惟錢穀之務;固吾圉者,惟甲兵是圖,至於天文、歷數,一切付之太史局,荒疏乖謬,安心為欺,朝士大夫莫有能詰之者。請召四方之通歷算者至都,使歷官學焉。」



 十一年,殿中侍御史陳垓言:「歷者,天地之大紀,國家之重事。今淳祐十年冬所頒十一年歷,稱成永祥等依《開禧》新歷推算,辛亥歲十二月十七日立春在酉正一刻,今所頒歷乃相師堯等依《淳祐》新歷推算,到壬子歲立春日在申
 正三刻。質諸前歷,乃差六刻,以此頒行天下,豈不貽笑四方!且許時演撰新歷,將以革舊歷之失。又考驗所食分數,《開禧》舊歷僅差一二刻,而李德卿新歷差六刻二分有奇,與今頒行前後兩歷所載立春氣候分數亦差六刻則同。由此觀之,舊歷差少,未可遽廢;新歷差多,未可輕用。一旦廢舊歷而用新歷,不知何所憑據。請參考推算頒行。」



 十二年,秘書省言:「太府寺丞張湜同李德卿算造歷書,與譚玉續進歷書頗有抵牾,省官參訂兩歷
 得失疏密以聞。其一曰:玉訟德卿竊用《崇天歷》日法三約用之。考之《崇天歷》用一萬五百九十為日法,德卿用三千五百三十為日法,玉之言然。其二曰:玉訟積年一億二千二十六萬七千六百四十六,不合歷法。今考之德卿用積年一億以上。其三曰:玉訟壬子年六月,癸丑年二月、六月、九月,丙辰年七月置閏皆差一日。今秘書省檢閱林光世用二家歷法各為推算。其四曰:德卿歷與玉歷壬子年立春、立夏以下十五節氣時刻皆同,雨水、
 驚蟄以下九節氣各差一刻。其五曰:德卿推壬子年二月乙卯朔日食,帶出已退所見大分八;玉推日食,帶出已退所見大分七。辰當壁宿度,同。其六曰:德卿歷斗分作三百六十五日二十四分二十八秒,玉歷斗分作三百六十五日二十四分二十九秒,二歷斗分僅差一秒。惟二十八秒之法,起於齊祖沖之,而德卿用之。使沖之之法可久,何以歷代增之?玉既指其謬,又多一秒,豈能必其天道合哉!請得商□鶴推算,合眾長而為一,然後
 賜名頒行。」十二年,歷成,賜名《會天》,寶祐元年行之,史闕其法。



 咸淳六年十一月三十日冬至,至後為閏十一月。既已頒歷,浙西安撫司準備差遣臧元震言:



 歷法以章法為重,章法以章歲為重。蓋歷數起於冬至,卦氣起於《中孚》,十九年謂之一章,一章必置七閏,必第七閏在冬至之前,必章歲至、朔同日。故《前漢志》云:「朔旦冬至,是謂章月。」《後漢志》云:「至、朔同日,謂之章月。」「積分成閏,閏七而盡,其歲十九,名之曰章。」《唐志》曰:「天數終於九,地數終於
 十,合二終以紀閏餘。」章法之不可廢也若此。



 今所頒庚午歲歷,乃以前十一月三十日為冬至,又以冬至後為閏十一月,莫知其故。蓋庚午之閏,與每歲閏月不同;庚午之冬至,與每歲之冬至又不同。蓋自淳祐壬子數至咸淳庚午,凡十九年,是為章歲,其十一月是為章月。以十九年七閏推之,則閏月當在冬至之前,不當在冬至之後。以至、朔同日論之,則冬至當在十一月初一日,不當在三十日。今以冬至在前十一月三十日,則是章
 歲至、朔不同日矣。若以閏月在冬至後,則是十九年之內止有六閏,又欠一閏。且一章計六千八百四十日,於內加七閏月,除小盡,積日六千九百四十日或六千九百三十九日,約止有一日。今自淳祐十一年辛亥章歲十一月初一日章月冬至後起算,十九年至咸淳六年庚午章歲十一月初一日當為冬至,方管六千八百四十日。今算造官以閏月在十一月三十日冬至之後,則此一章止有六閏,更加六閏除小盡外,實積止六千九百
 十二日,比之前後章歲之數,實欠二十八日。歷法之差,莫甚於此。況天正冬至乃歷之始,必自冬至後積三年餘分,而後可以置第一閏。今庚午年章歲丙寅日申初三刻冬至,去第二日丁卯僅有四分日之一,且未正日,安得遽有餘分?未有餘分,安得遽有閏月?則是後一章之始不可推算,其謬可知矣。今欲改之,有簡而易行之說。蓋歷法有平朔,有經朔,有定朔。一大一小,此平朔也;兩大兩小,此經朔也;三大三小,此定朔也。今正以定朔
 之說,則當以前十一月大為閏十月小,以閏十一月小為十一月大,則丙寅日冬至即可為十一月初一,以閏十一月初一之丁卯為十一月初二日,庶幾遞趲下一日置閏,十一月二十九日丁未始為大盡。然則冬至既在十一月初一,則至、朔同日矣;閏月既在至節前,則十九年七閏矣。此昔人所謂晦節無定,由時消息,上合履端之始,下得歸餘於終,正謂此也。



 夫歷久未有不差,差則未有不改者。後漢元和初歷差,亦是十九年不得七
 閏,歷雖已頒,亦改正之。顧今何靳於改之哉!元震謂某儒者,豈欲與歷官較勝負?既知其失,安得默而不言邪!



 於是朝廷下之有司,遣官偕元震與太史局辨正,而太史之詞窮,元震轉一官,判太史局鄧宗文、譚玉等各降官有差。因更造歷,六年,歷成,詔試禮部尚書馮夢得序之;七年,頒行,即《成天歷》也。



 德祐之後,陸秀夫等擁立益王,走海上,命禮部侍郎鄧光薦與蜀人楊某等作歷,賜名《本天歷》,今亡。



\end{pinyinscope}