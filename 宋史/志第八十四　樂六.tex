\article{志第八十四 樂六}

\begin{pinyinscope}

 光宗受禪,崇上壽皇聖帝、壽成皇后暨壽聖皇太后尊號,壽皇樂用《乾安》,壽聖、壽成樂用《坤安》,三殿慶禮,在當時侈為盛儀。尋以禮部、太常寺言:「國朝歲饗上帝,太祖
 肇造王業,則配冬饗於圜丘;太宗混一區宇,則春祈穀、夏大雩、秋明堂俱配焉。高宗身濟大業,功德茂盛,所宜奉侑,仰繼祖宗,以協先儒嚴祖之議,以彰文祖配天之烈。」乃季秋升侑於明堂,奠幣用《宗安》之樂,酌獻用《德安》之樂,並登歌作大呂宮。及加上高宗徽號,奉冊、寶以告,用《顯安》之樂。



 紹熙元年,始行中宮冊禮,發冊於文德殿:皇帝升降御坐用《乾安》之樂,持節展禮官出入殿門用《正安》之樂。受冊於穆清殿:皇后出就褥位用《坤安》,至位
 用《承安》,受冊寶,用《成安》,受內外命婦賀就坐,用《和安》,內命婦進行賀禮,用《惠安》,外命婦進行賀禮用《咸安》,皇后降坐用《徽安》,歸閣用《泰安》冊、寶入殿門用《宜安》。宋初立後,自景祐始行冊命之禮。元祐納後,典章彌盛,而六禮發制書日,樂備不作,惟皇后入宣德門,朝臣班迎,鳴鐘鼓而已。崇寧中,乃陳宮架,用女工,皇后升降行止,並以樂為節。至紹興復制樂,以重禕翟,詔執色勿用女工,令太常止於門外設樂。隆興冊禮時,則國樂未舉,淳熙始
 遵用之,而紹熙敷賁舊典,於此特加詳備。紹興樂奏仲呂宮,仲呂為陰;紹熙樂奏太簇宮,太簇為陽:用樂同而揆律異焉。



 明年郊祀,太常耿秉奏:「致敬鬼神,以禮樂為本,樂欲其備,音欲其和。今所用雷鼓之屬,正所以祀天致神,而皮革虛緩,聲不能振應;登歌、大樂樂器及樂舞工人冠服,有積歲久而損弊者,宜葺新之。太常在籍樂工,不給於役,召募百姓,罕能習熟。郊祀事重,其樂工親扈乘輿,和樂雅奏,期以接天地、享祖宗,請優其日廩,以
 籍田司錢給之,樂藝稍精,仍加賞勸。其緣托權要、送名充數者,嚴戢絕之。」又言:「大禮前期,皇帝朝饗太廟,別廟內安穆、安恭皇后二室,前此系大臣分詣行事。今既親詣室稞,其酌獻、升殿所奏樂曲,恐不相協,宜命有司更制。」皆從之。



 寧宗即位,孝宗升祔,祧僖祖,立別廟,禮官言:「僖祖既仿唐興聖立為別廟,遇袷則即廟以饗,孟冬袷饗日,合先詣僖祖廟室行禮。其樂舞欲依每歲別廟五饗設樂禮例,於僖祖添設登歌樂。如僖廟行禮,就廟殿
 依次作登歌樂,其宮架樂則於太廟殿上通作。」詔從之。



 既而臣僚言:「皇帝因重明聖節,詣壽康宮上壽舉樂,仰體聖主事親盡孝之志,俯遂臣子尊君親上之忱,此國家典禮之大者也。檢照典故,天申節賜御筵,在上壽次日。今乃於前一日賜文武百僚宴,重明上壽,用樂攸始,而臣下聽樂乃在君父之先,義有未安。」遂命改用次日。凡奉上冊寶於慈福、壽康宮者,再備樂行禮,一用乾道舊制。尋御文德殿制冊皇后,有司請設宮架之樂,依儀
 施行。慶元六年瑞慶節,金使至,以執光宗、慈懿皇后喪,詔就驛賜御筵,並不作樂。



 嘉定二年,明堂大饗,禮部尚書章穎奏:「太常工籍闕少,率差借執役。當親行薦饗,或容不根游手出入殿庭,非所以肅儀衛、嚴禁防也。乞申紹興、開禧已行禁令,不許用市井替名,顯示懲戒,庶俾駿奔之人小大嚴潔,以稱精禋。」臣僚又奏:「郊祀登歌列於壇上,簉於上龕,蓋在天地祖宗之側也。宮架列於午階下,則百神所同聽也。夫樂音莫尚於和,今絲、竹、管、弦
 類有闕斷,拊搏、佾舞,賤工、窶人往往垢玩猱雜,宜申嚴以肅祀事。」皆俞其請。至十四年,詔:「山東、河北連城慕義,殊俗郊順,奉玉寶來獻,其文曰『皇帝恭膺天命之寶』,實惟我祖宗之舊。」乃明年元日,上御大慶殿受寶,用鼓吹導引,備陳宮架大樂,奏詩三章:一曰《恭膺天命》,二曰《舊疆來歸》,三曰《永清四海》,並奏以太簇宮。



 理宗享國四十餘年,凡禮樂之事,式遵舊章,未嘗有所改作。先是,孝宗廟用《大倫》之樂,光宗廟用《大和》之樂;至是,寧宗祔廟,用《
 大安》之樂。紹定三年,行中宮冊禮,並用紹熙元年之典。及奉上壽明仁福慈睿皇太后冊寶,始新制樂曲行事。當時中興六七十載之間,士多嘆樂典之久墜,類欲搜講古制,以補遺軼。於是,姜夔乃進《大樂議》於朝。夔言:



 紹興大樂,多用大晟所造,有編鐘、鎛鐘、景鐘,有特磬、玉磬、編磬,三鐘三磬未必相應。塤有大小,簫、篪、笛有長短,笙、竽之簧有厚薄,未必能合度,琴、瑟弦有緩急燥濕,軫有旋復,柱有進退,未必能合調。總眾音而言之,金欲應石,
 石欲應絲,絲欲應竹,竹欲應匏,匏欲應土,而四金之音又欲應黃鐘,不知其果應否。樂曲知以七律為一調,而未知度曲之義;知以一律配一字,而未知永言之旨。黃鐘奏而聲或林鐘,林鐘奏而聲或太簇。七音之協四聲,各有自然之理。今以平、入配重濁,以上、去配輕清,奏之多不諧協。



 八音之中,琴、瑟尤難。琴必每調而改弦,瑟必每調而退柱,上下相生,其理至妙,知之者鮮。又琴、瑟聲微,常見蔽於鐘、磬、鼓、簫之聲;匏、竹、土聲長,而金石常不
 能以相待,往往考擊失宜,消息未盡。至於歌詩,則一句而鐘四擊,一字而竽一吹,未協古人槁木貫珠之意。況樂工茍焉占籍,擊鐘磬者不知聲,吹匏竹者不知穴,操琴瑟者不知弦。同奏則動手不均,迭奏則發聲不屬。比年人事不和,天時多忒,由大樂未有以格神人、召和氣也。



 宮為君、為父,商為臣、為子,宮商和則君臣父子和。徵為火,羽為水,南方火之位,北方水之宅,常使水聲衰、火聲盛,則可助南而抑北。宮為夫,徵為婦,商雖父宮,實
 征之子,常以婦助夫、子助母,而後聲成文。徵盛則宮唱而有和,商盛則徵有子而生生不窮,休祥不召而自至,災害不祓而自消。聖主方將講禮郊見,願詔求知音之士,考正太常之器,取所用樂曲,條理五音,隱括四聲,而使協和。然後品擇樂工,其上者教以金、石、絲、竹、匏、土、歌詩之事,其次者教以戛、擊、乾、羽、四金之事,其下不可教者汰之。雖古樂未易遽復,而追還祖宗盛典,實在茲舉。



 其議雅俗樂高下不一,宜正權衡度量:



 自尺律之法亡於
 漢、魏,而十五等尺雜出於隋、唐正律之外,有所謂倍四之器,銀字、中管之號。今大樂外有所謂下宮調,下宮調又有中管倍五者。有曰羌笛、孤笛,曰雙韻、十四弦以意裁聲,不合正律,繁數悲哀,棄其本根,失之太清;有曰夏笛、鷓鴣,曰胡盧琴、渤海琴,沉滯抑鬱。腔調含糊,失之太濁。故聞其聲者,性情蕩於內,手足亂於外,《禮》所謂「慢易以犯節,流湎以忘本,廣則容奸,狹則思欲」者也。家自為權衡,鄉自為尺度,乃至於此。謂宜在上明示以好惡。凡
 作樂制器者,一以太常所用及文思所頒為準。其它私為高下多寡者悉禁之,則斯民「順帝之則」,而風俗可正。



 其議古樂止用十二宮:



 周六樂奏六律、歌六呂,惟十二宮也。「王大食,三侑。」注云:「朔日、月半。」隨月用律,亦十二宮也。十二管各備五聲,合六十聲;五聲成一調,故十二調。古人於十二宮又特重黃鐘一宮而已。齊景公作《征招》、《角招》之樂,師涓、師曠有清商、清角、清徵之操。漢、魏以來,燕樂或用之,雅樂未聞有以商、角、征、羽為調者,惟迎氣
 有五引而已,《隋書》云「梁、陳雅樂,並用宮聲」是也。若鄭譯之八十四調,出於蘇祗婆之琵琶。大食、小食、般涉者,胡語;《伊州》、《石州》、《甘州》、《婆羅門》者,胡曲;《綠腰》、《誕黃龍》、《新水調》者,華聲而用胡樂之節奏。惟《瀛府》、《獻仙音》謂之法曲,即唐之法部也。凡有催袞者,皆胡曲耳,法曲無是也。且其名八十四調者,其實則有黃鐘、太簇、夾鐘、仲呂、林鐘、夷則、無射七律之宮、商、羽而已,於其中又闕太簇之商、羽焉。國朝大樂諸曲,多襲唐舊。竊謂以十二宮為雅樂,周
 制可舉;以八十四調為宴樂,胡部不可雜。郊廟用樂,咸當以宮為曲,其間皇帝升降、盥洗之類,用黃鐘者,群臣以太簇易之,此周人王用《王夏》、公用《驁夏》之義也。



 其議登歌當與奏樂相合:



 《周官》歌奏,取陰陽相合之義。歌者,登歌、徹歌是也;奏者,金奏、下管是也。奏六律主乎陽,歌六呂主乎陰,聲不同而德相合也,自唐以來始失之。故趙慎言云:祭祀有下奏太簇、上歌黃鐘,俱是陽律,既違禮經,抑乖會合。」今太常樂曲,奏夾鐘者奏陰歌陽,其合
 宜歌無射,乃或歌大呂;奏函鐘者奏陰歌陽,其合宜歌蕤賓,乃或歌應鐘;奏黃鐘者奏陽歌陰,其合宜歌大呂,乃雜歌夷則、夾鐘、仲呂、無射矣。茍欲合天人之和,此所當改。



 其議祀享惟登歌、徹豆當歌詩:



 古之樂,或奏以金,或吹以管,或吹以笙,不必皆歌詩。周有《九夏》,鐘師以鐘鼓奏之,此所謂奏以金也。大祭祀登歌既畢,下管《象》、《武》。管者,簫、篪、笛之屬。《象》、《武》皆詩而吹其聲,此所謂吹以管者也。周六笙詩,自《南陔》皆有聲而無其詩,笙師掌之以
 供祀饗,此所謂吹以笙者也。周升歌《清廟》,徹而歌《雍》詩,一大祀惟兩歌詩。漢初,此制未改,迎神曰《嘉至》,皇帝入曰《永至》:皆有聲無詩。至晉始失古制,既登歌有詩,夕牲有詩,饗神有詩,迎神、送神又有詩。隋、唐至今,詩歌愈富,樂無虛作。謂宜仿周制,除登歌、徹歌外,繁文當刪,以合於古。



 其議作鼓吹曲以歌祖宗功德:



 古者,祖宗有功德,必有詩歌,《七月》之陳王業是也。歌於軍中,周之愷樂、愷歌是也。漢有短簫鐃歌之曲凡二十二篇,軍中謂之騎
 吹,其曲曰《戰城南》、《聖人出》之類是也。魏因其聲,制為《克官渡》等曲十有二篇;晉亦制為《征遼東》等曲二十篇;唐柳宗元亦嘗作為鐃歌十有二篇,述高祖、太宗功烈。我朝太祖、太宗平僭偽,一區宇;真宗一戎衣而卻契丹;仁宗海涵春育,德如堯、舜;高宗再造大功,上儷祖宗。願詔文學之臣,追述功業之盛,作為歌詩,使知樂者協以音律,領之太常,以播於天下。



 夔乃自作《聖宋鐃歌曲》:宋受命曰《上帝命》,平上黨曰《河之表》,定維揚曰《淮海濁》,取湖
 南曰《沅之上》,得荊州曰《皇威暢》,取蜀曰《蜀山邃》,取廣南曰《時雨霈》,下江南曰《望鐘山》,吳越獻國曰《大哉仁》,漳、泉獻土曰《謳歌歸》,克河東曰《伐功繼》,徵澶淵曰《帝臨墉》,美仁治曰《維四葉》,歌中興曰《炎精復》,凡十有四篇,上於尚書省。書奏,詔付太常。然夔言為樂必定黃鐘,迄無成說。其議今之樂極為詳明,而終謂古樂難復,則於樂律之原有未及講。



 其後,朱熹深悼先王制作之湮泯,與其友武夷蔡元定相與講明,反復參訂,以究其歸極。熹在慶
 元經筵,嘗草奏曰:「自秦滅學,禮樂先壞,而樂之為教,絕無師授。律尺短長,聲音清濁,學士大夫莫知其說,而不知其為闕也。望明詔許臣招致學徒,聚禮樂諸書,編輯別為一書,以補六藝之闕。」後修禮書,定為《鐘律》、《樂制》等篇,垂憲言以貽後人。



 蓋宋之樂議,因時迭出,其樂律高下不齊,俱有原委。建隆初用王樸樂,藝祖一聽,嫌其太高,近於哀思,詔和峴考西京表尺,令下一律,比舊樂始和暢。至景祐、皇祐間,訪樂、議樂之詔屢頒,於是命李照
 改定雅樂,比樸下三律。照以縱黍累尺,雖律應古樂,而所造鐘磬,才中太簇,樂與器自相矛盾。阮逸、胡瑗復定議,止下一律,以尺生律,而黃鐘律短,所奏樂聲復高。元豐中,以楊傑條樂之疵,召範鎮、劉幾參定。幾、傑所奏,下舊樂三律,範鎮以為聲雜鄭、衛,且律有四厘六毫之差,太簇為黃鐘,宮商易位,欲求真黍以正尺律,造樂來獻,復下李照一律。至元祐廷奏,而詔獎之。初,鎮以房庶所得《漢書》,其言黍律異於他本,以大府尺為黃帝時尺,司
 馬光力辨其不然。鎮以周釜、漢斛為據,光謂釜本《考工》所記,斛本劉歆所作,非經不足法。鎮以所收開元中笛及方響合於仲呂,校太常樂下五律,教坊樂下三律。光謂此特開元之仲呂,未必合於後夔,力止鎮勿奏所為樂。光與鎮平生大節不謀而同,惟鐘律之論往返爭議,凡三十餘年,終不能以相一。



 是時,濂、洛、關輔諸儒繼起,遠溯聖傳,義理精究。周惇頤之言樂,有曰:「古者聖王制禮法、修教化,三綱正,九疇敘,百姓大和,萬物咸若,乃作
 樂以宣八風之氣。樂聲淡而不傷,和而不淫。淡則欲心平,和則躁心釋。德盛治至,道配天地,古之極也。後世禮法不修,刑政苛紊,代變新聲,導欲增悲,故有輕生敗倫不可禁者矣。樂者,古以平心,今以助欲;古以宣化,今以長怨。不復古禮,不變今樂,而欲至治者,遠哉!」



 程頤有曰:「律者,自然之數。先王之樂,必須律以考其聲。尺度權衡之正,皆起於律。律管定尺,以天地之氣為準,非秬黍之比也。律取黃鐘,黃鐘之聲亦不難定,有知音者,參上下
 聲考之,自得其正。」



 張載有曰:「聲音之道與天地通,蠶吐絲而商弦絕,木氣盛則金氣衰,乃此理自相應。今人求古樂太深,始以古樂為不可知,律呂有可求之理,惟德性深厚者能知之。」此三臣之學,可謂窮本知變,達樂之要者矣。



 熹與元定蓋深講於其學者,而研覃真積,述為成書。元定先究律呂本原,分其篇目,又從而證辨之。



 其黃鐘篇曰:



 天地之數始於一,終於十:其一、三、五、七、九為陽,九者,陽之成也;其二、四、六、八、十為陰,十者,陰之成也。
 黃鐘者,陽聲之始,陽氣之動也,故其數九。分寸之數,具於聲氣之先,不可得而見。及斷竹為管,吹之而聲和,候之而氣應,而後數始形焉。均其長,得九寸;審其圍,得九分;積其實,得八百一十分。長九寸,圍九分,積八百一十分,是為律本,度量權衡於是而受法,十一律由是損益焉。



 其《證辨》曰:古者考聲候氣,皆以聲之清濁、氣之先後求黃鐘也。夫律長則聲濁而氣先至,律短則聲清而氣後至,極長極短則不成聲而氣不應。今欲求聲氣之中,而莫適為準,莫若且多截竹以擬黃鐘之管,或極其短,或極其長,長短之內,每差一分而為一管,皆即以其長權為九寸,而度圍徑如黃鐘之法焉。更迭以吹,則中
 聲可得;淺深以列,則中氣可驗。茍聲和氣應,則黃鐘之為黃鐘者信矣。黃鐘信,則十一律與度量權衡者得矣。後世不知出此,而惟尺之求。晉氏而下,多求之金石;梁、隋以來,又參之秬黍;至王樸專恃累黍,金石亦不復考。夫金石真偽固難盡信,而秬黍長短小大不同,尤不可恃。古人謂『子穀秬黍,中者實其鑰』,是先得黃鐘而後度之以黍,以見周徑之度,以生度量權衡之數而已,非律生於黍也。百世之下,欲求百世之前之律者,亦求之聲氣之元而毋必之於秬黍,斯得之矣。」



 《黃鐘生十一律篇》曰:



 子、寅、辰、午、申、戌六陽辰皆下生,醜、卯、巳、未、酉、亥六陰辰皆上生。陽數以倍者,三分本律而損其一也;陰數以四者,三分本律而增其一也。六陽辰當位,自得六陰位以居其沖。其林鐘、
 南呂、應鐘三呂在陰,無所增損;其大呂、夾鐘、仲呂三呂在陽,則用倍數,方與十二月之氣相應,蓋陰陽自然之理也。



 其《證辨》曰:「按《呂氏》、《淮南子》,上下相生,與司馬氏《律書》、《漢前志》不同,雖大呂、夾鐘、仲呂用倍數則一,然《呂氏》、《淮南》不過以數之多寡為生之上下,律呂陰陽錯亂而無倫,非其本法也。」



 《十二律篇》曰:



 按十二律之實,約以寸法,則黃鐘、林鐘、太簇得全寸;約以分法,則南呂、姑洗得全分;約以厘法,則應鐘、蕤賓得全厘;約以毫法,則大呂、夷則得全毫;約以絲法,則夾鐘、無射得全絲。約至仲呂之實十三萬一千七十二,以三分之,不
 盡二算,其數不行,此律之所以止於十二也。



 其《證辨》曰:「黃鐘為十二律之首,他律無大於黃鐘,故其正聲不為他律役。至於大呂之變宮、夾鐘之羽、仲呂之徵、蕤賓之變徵、夷則之角、無射之商,自用變律半聲,非復黃鐘矣。此其所以最尊而為君之象,然亦非人所能為,乃數之自然,他律雖欲役之而不可得也。此一節最為律呂旋宮用聲之綱領也。」



 《變律篇》曰:



 十二律各自為宮,以生五聲二變。其黃鐘、林鐘、太簇、南呂、姑洗、應鐘六律,則能具足。至蕤賓、大呂、夷則、夾鐘、無射、仲呂六律,則取黃鐘、林鐘、太簇、南呂、姑洗、應鐘六律之聲,少下,不和,故有變律。律之當變者有六:黃鐘、林鐘、太簇、南呂、姑
 洗、應鐘。變律者,其聲近正律而少高於正律,然後洪纖、高下不相奪倫。變律非正律,故不為宮。



 其《證辨》曰:「十二律循環相生,而世俗不知三分損益之數,往而不返。仲呂再生黃鐘,止得八寸七分有奇,不成黃鐘正聲。京房覺其如此,故仲呂再生,別名執始,轉生四十八律。不知變律之數止於六者,出於自然,不可復加。雖強加之,亦無所用也。房之所傳出於焦氏,焦氏卦氣之學,亦去四而為六十,故其推律必求合此數。不知數之自然,在律不可增,於卦不可減也。何承天、劉焯譏房之病,乃欲增林鐘已下十一律之分,使至仲呂反生黃鐘,還得十七萬七千一百四十七之數,則是惟黃鐘一律成律,他十一律皆不應三分損益之數,其失又甚於房。



 《律生五聲篇》曰:



 宮聲八十一,商聲七十二,角聲六十四,徵聲五十四,
 羽聲四十八。按黃鐘之數九九八十一,是為五聲之原,三分損一以下生徵,征三分益一以上生商,商三分損一以下生羽,羽三分益一以上生角。至角聲之數六十四,以三分之,不盡一算,數不可行,此聲之數所以止於五也。



 其《證辨》曰:「《通典》曰:『黃鐘為均,用五聲之法以下十一辰,辰各有五聲,其為宮商之法亦如之。辰各有五聲,合為六十聲,是十二律之正聲也。』夫黃鐘一均之數,而十一律於此取法焉。以十二律之宮長短不同,而其臣、民、事、物、尊卑,莫不有序而不相亂,良以是耳。沉括不知此理,乃以為五十四在黃鐘為徵、在夾鐘為角、在仲呂為商者,其亦誤矣。俗樂之有清聲,略知此意。但不知仲呂反生黃鐘,黃鐘又自林鐘再生太簇,皆為變律,
 已非黃鐘、太簇之清聲耳。胡瑗於四清聲皆小其圍徑,則黃鐘、太簇二聲雖合,而大呂、夾鐘二聲又非本律之半。且自夷則至應鐘四律,皆以次而小其徑圍以就之,遂使十二律、五聲皆有不得其正者。李照、範鎮止用十二律,則又未知此理。蓋樂之和者,在於三分損益;樂之辨者,在於上下相生。若李照、範鎮之法,其合於三分損益者則和矣,自夷則已降,其臣、民、事、物,豈能尊卑有辨而不相凌犯乎?晉荀勖之笛,梁武帝之通,皆不知而作者也。」



 《變聲篇》曰:



 變宮聲四十二,變徵聲五十六。五聲宮與商、商與角、徵與羽相去各一律,至角與征、羽與宮相去乃二律。相去一律則音節和,相去二律則音節遠。故角、徽之間,近徵收一聲,比徵少下,故謂之變徵;羽、宮之間,
 近宮收一聲,少高於宮,故謂之變宮。角聲之實六十有四,以三分之,不盡一算,既不可行,當有以通之。聲之變者二,故置一而兩,三之得九,以九因角聲之實六十有四,得五百七十六。三分損益,再生變徵、變宮二聲,以九歸之,以從五聲之數,存其餘數,以為強弱。至變徵之數五百一十二,以三分之,又不盡二算,其數又不行,此變聲所以止於二也。變宮、變徵,宮不成宮,征不成徵,《淮南子》謂之「和謬」,所以濟五聲之不及也。變聲非正聲,故不
 為調。



 其《證辨》曰:「宮、羽之間有變宮,角、征之間有變徵,此亦出於自然,《左氏》所謂『七音』,《漢前志》所謂「七始」是也。然五聲者,正聲,故以起調、畢曲,為諸聲之綱。至二變聲,則不比於正音,但可濟其所不及而已。然有五聲而無二變,亦不可以成樂也。」



 《八十四聲篇》曰:



 黃鐘不為他律役,所用七聲皆正律,無空、積、忽、微。自林鐘而下,則有半聲:大呂、太簇一半聲,夾鐘、姑洗二半聲,蕤賓、林鐘四半聲,夷則、南呂五半聲,無射、應鐘為六半聲。中呂為十二律之窮,三半聲也。自蕤賓而下則有變律:蕤賓一變律,大呂二變律,夷則三變律,夾鐘四變律,無射五變律,中呂六變律
 也。皆有空、積、忽、微,不得其正,故黃鐘獨為聲氣之元。雖十二律八十四聲皆黃鐘所生,然黃鐘一均,所謂純粹中之純粹者也。八十四聲:正律六十三,變律二十一。六十三者,九七之數也;二十一者,三七之數也。



 《六十調篇》曰:



 十二律旋相為宮,各有七聲,合八十四聲。宮聲十二,商聲十二,角聲十二,徵聲十二,羽聲十二,凡六十聲,為六十調,其變宮十二,在羽聲之後、宮聲之前;變征十二,在角聲之後、徵聲之前:宮徵皆不成,凡二十四聲,不可
 為調。黃鐘宮至夾鐘羽,並用黃鐘起調、黃鐘畢曲;大呂宮至姑洗羽,並用大呂起調、大呂畢曲;太簇宮至仲呂,並用太簇起調、太簇畢曲;夾鐘宮至蕤賓羽,並用夾鐘起調、夾鐘畢曲;姑洗宮至林鐘羽,並用姑洗起調、姑洗畢曲;仲呂宮至夷則羽,並用仲呂起調、仲呂畢曲;蕤賓宮至南呂羽,並用蕤賓起調、蕤賓畢曲;林鐘宮至無射羽,並用林鐘起調、林鐘畢曲;夷則宮至應鐘羽,並用夷則起調、夷則畢曲;南呂宮至黃鐘羽,並用南呂起調、
 南呂畢曲;無射宮至大呂羽,並用無射起調、無射畢曲;應鐘宮至太簇羽,並用應鐘起調、應鐘畢曲,是為六十調。六十調即十二律也,十二律即一黃鐘也。黃鐘生十二律,十二律生五聲二變。五聲各有紀綱,以成六十調,六十調皆黃鐘損益之變也。宮、商、角三十六調,老陽也;其徵、羽二十四調,老陰也。調成而陰陽備也。



 或曰:「日辰之數由天五、地六錯綜而生,律呂之數由黃鐘九寸損益而生,二者不同。至數之成,則日有六甲、辰有五子為
 六十日;律呂有六律、五聲為六十調,若合符節,何也?」曰:「即所謂調成而陰陽備也。」夫理必有對待,數之自然也。以天五、地六合陰與陽言之,則六甲、五子究於六十,其三十六為陽,二十四為陰。以黃鐘九寸紀陽不紀陰言之,則六律、五聲究於六十,亦三十六為陽,二十四為陰。蓋一陽之中,又自有陰陽也。非知天地之化育者,不能與於此。



 其《證辨》曰:「《禮運》:『五聲、六律、十二管還相為宮。』孔氏疏曰:『黃鐘為第一宮,至中呂為第十二宮,各有五聲,凡六十聲。』聲者,所以起調、畢曲,為諸聲之綱領,正《禮運》所謂『還相為宮』也。《周禮·大司樂》,祭祀不用商,惟
 宮、角、征、羽四聲。古人變宮、變徵不為調,《左氏傳》曰:『中聲以降,五降之後,不容彈矣。』以二變聲之不可為調也。後世以變宮、變徵參而為八十四調,其亦不考矣。」



 《候氣篇》曰:



 以十二律分配節氣,按歷而俟之。其氣之升,分、毫、絲、忽,隨節各異。夫陽生於《復》,陰生於《姤》,如環無端。今律呂之數,三分損益,終不復始,何也?曰:「陽之升始於子,午雖陰生,而陽之升於上者未已,至亥而後窮上反下;陰之升始於午,子雖陽生,而陰升於上亦未已,至巳而後窮上反下。律於陰則不書,故終不復始也。是以升,陽之數,自子至巳差強,在律為
 尤強,在呂為差弱;自午至亥漸弱,在律為尤弱,在呂為差強。分數多寡,雖若不齊,然而絲分毫別,各有條理,此氣之所以飛灰,聲之所以中律也。」



 或曰:「《易》以道陰陽,而律不書陰,何也?」曰:「《易》盡天下之變,善惡無不備,律致中和之用,止於至善者也,以聲言之,大而至於雷霆,細而至於蠛蠓,無非聲也。《易》則無不備也,律則寫其所謂黃鐘一聲而已。雖有十二律六十調,然實一黃鐘也。是理也,在聲為中聲,在氣為中氣,在人則喜怒哀樂未發與
 發而中節,此聖人所以一天人、贊化育之道也。」



 其《證辨》曰:「律者,陽氣之動,陽聲之始,必聲和氣應,然後可以見天地之心。今不此之務,乃區區於秬黍之縱橫、古錢之大小,其亦難矣。然非精於歷數,則氣節亦未易正。」



 至於審度量、謹權衡,會粹古今,辨析尤詳,皆所以參伍而定黃鐘為中聲之符驗也。朱熹深好其書,謂國家行且平定,中原必將審音協律,以諧神人。受詔典領之臣,宜得此書奏之,以備東都郊廟之樂。



 熹定《鐘律》、《詩樂》、《樂制》、《樂舞》等篇,匯分於所修禮書中,皆聚古樂之根源,簡約可觀。而《鐘律》分前後篇,其前篇
 為條凡七:一曰十二律陰陽、辰位相生次第之圖,二曰十二律寸、分、厘、毫、絲、忽之數,三曰五聲五行之象、清濁高下之次,四曰五聲相生、損益、先後之次,五曰變宮、變徽二變相生之法,六曰十二律正變、倍半之法,七曰旋宮八十四聲、六十調之圖。其後篇為條凡六:一曰明五聲之義,二曰明十二律之義,三曰律寸舊法,四曰律寸新法,五曰黃鐘分寸數法,六曰黃鐘生十一律數。大率採元定所著,更互演繹,尤為明邃。其《樂制》匯於王朝禮,
 其《樂舞》匯於祭禮,上下千載,旁搜遠紹,昭示前聖禮樂之非迂,而將期古樂之復見於今,熹蓋深致意焉。其《詩樂篇》別系於後。



\end{pinyinscope}