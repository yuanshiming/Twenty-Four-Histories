\article{志第一百五十七 藝文三}

\begin{pinyinscope}

 衛宏《漢舊儀》三卷



 應劭《漢官儀》一卷



 蔡質《漢官典儀》一
 卷



 《漢制拾遺》一卷不知何人編



 蕭嵩《唐開元禮》一百五十卷一云王立等作



 又《開元禮儀鏡》五卷



 韋彤《開元禮儀釋》二十卷



 《開元禮儀鏡略》十卷



 《開元禮百
 問》二卷



 《開元禮教林》一卷



 《開元禮類釋》十二卷



 並不知作者



 顏真卿《歷古創置儀》五卷



 柳珵《唐禮纂要》六卷



 韋公肅《禮閣新儀》三十卷



 王彥威一本作「崔靈恩」



 《續曲臺禮》三十卷



 王涇《大唐郊祀錄》十卷



 李隨《吉兇五服儀》一卷



 《紅亭紀吉
 儀》一卷獨孤儀及陸贄撰



 孟詵《家祭禮》一卷



 徐閏《家祭儀》一
 卷



 鄭正則《祠享儀》一卷



 又《家祭儀》一卷



 賈頊《家薦儀》一卷



 範傳式《寢堂時饗儀》一卷



 孫日用《仲享儀》一卷



 袁郊《服飾變古元錄》三卷



 裴茞《書儀》三卷



 劉岳《吉兇書儀》二卷



 陳致雍《曲臺奏議集》



 又《州縣祭禮儀、五禮儀鏡》六卷



 《寢祀儀》一卷



 朱熹《二十家古今祭禮》二卷



 《政和五禮新儀》二百
 四十卷鄭居中、白時中、慕容彥逢、強淵明等撰



 杜衍《四時祭享儀》一卷



 劉溫叟《開寶通禮》二百卷



 盧多遜《開寶通禮儀纂》一百卷



 賈昌朝《太常新禮》四十卷



 沿情子《新禮》一卷不知名



 《大中祥符封禪記》五十卷丁謂、李宗諤等撰



 《大中祥符祀汾陰記》
 五十卷丁謂等撰



 張知白《御史臺儀制》六卷



 宋綬《天聖鹵
 簿記》十卷



 文
 彥
 博、高若
 訥《大饗明堂記》二十卷



 文彥博《大饗明堂記要》二卷



 歐陽修《太常因革禮》一百卷



 韓琦《參用古今家祭式》無卷



 許洞《訓俗書》一卷



 王安石《南郊式》一百十卷



 李德芻《聖朝徽名錄》十卷



 《國朝祀典》一卷不知作者



 陳襄《郊廟奉祀禮文》三十卷



 《諸州釋奠文宣王儀注》一卷元豐間重修



 司馬光《書儀》八卷



 又《涑水祭儀》一卷



 《居家雜議》一卷



 範祖禹《祭儀》一卷



 《幸太學儀》一卷元祐六年儀



 《納后儀》一卷元祐七年儀



 呂大防、大臨《家祭儀》一卷



 《橫渠張氏祭儀》一卷張載撰



 《釋奠祭
 器圖》及《諸州軍釋奠儀注》一卷崇寧中頒行



 《藍田呂氏祭說》一卷呂大均撰



 《伊川程氏祭儀》一卷程頤撰



 《宣和重修鹵簿圖記》三十五卷蔡攸等撰



 李沇《皇宋大典》三卷



 夏休《辨太常禮官儀定章九冕服》一卷



 《紹興太常初定儀注》三卷



 範寅賓《五祀新儀撮要》十五卷



 鄭樵《鄉飲禮》三卷



 又《鄉飲禮圖》三卷



 史定之《鄉飲酒儀》一卷



 《中興禮書》二卷淳熙中禮部、太常寺編



 《歷代明堂事跡》一卷



 《儀物志》三卷



 《祀祭儀式》一卷



 《太常圖》一卷



 並不知作者



 葉克刊《南劍鄉飲酒儀》一卷



 汪楫《
 鄉飲規約》一卷



 《淳熙編類祭祀儀式》一卷齊慶冑所撰



 張維《釋奠通祀圖》一卷



 李□《公侯守宰士庶通禮》三十卷



 趙師《熙朝盛典詩》二卷



 趙希蒼《趙氏祭錄》二卷



 朱熹《釋奠儀式》一卷



 又《四家禮範》五卷



 《家禮》一卷



 李宗思《禮範》一卷



 韓挺《服制》一卷



 張叔椿《五禮新儀》十五卷



 高閌《送終禮》一卷



 陳孔碩《釋奠儀禮考正》一卷



 周端朝《冠婚喪祭禮》二卷集司馬氏、程氏、呂氏禮



 管銳《嘗聞錄》一卷



 吳仁傑《廟制罪言》二卷



 又《郊祀贅說》二卷



 潘徽《江都集禮》一百四卷
 本百二十卷,今殘缺



 和峴《秘閣集》二十卷



 王皞《禮閣新編》六十三卷



 黃廉《大禮式》二十卷



 何洵直、蔡確《禮文》三十卷



 《唐吉兇禮儀禮圖》三卷



 龐元英《五禮新編》五十卷



 《大觀禮書賓軍等四禮》五百五卷《看詳》十二卷



 《大觀新編禮書古禮》二百三十二卷《看詳》十七卷



 歐陽修《太常禮院祀儀》二十四卷



 和峴《禮神志》十卷



 孫奭《大宋崇祀錄》二十卷



 賈昌朝《慶歷祀儀》六十三卷



 《朱梁南郊儀注》一卷



 《吳南郊圖記》一卷



 王涇一作「浮」



 《祠儀》一卷



 陳繹《南郊附式條貫》一
 卷



 向宗儒《南郊式》十卷



 陳昉《北郊祀典》三十卷



 蔣猷《夏祭敕令格式》一部卷亡



 宋郊《明堂通儀》二卷



 《明堂袷饗大禮令式》三百九十三卷元豐間



 《明堂大饗視朔頒朔布政儀範敕令格式》一部宣和初。卷亡



 王欽若《天書儀制》五卷



 又《鹵簿記》三卷



 馮宗道《景靈宮供奉敕令格式》六十卷



 《景靈宮四孟朝獻》二卷



 《諸陵薦獻禮文儀令格式並例》一百五十一冊紹聖間。卷亡



 張諤《熙寧新定祈賽式》二卷



 張傑《春秋車服圖》五卷



 劉孝孫《二儀實錄衣服名義》二卷



 《
 祭服制度》十六卷



 《祭服圖》三冊卷亡



 《五服志》三卷



 裴茞《五服儀》二卷



 劉筠《五服年月「年月」一作「用」



 敕》一卷



 《喪服加減》一卷



 李至《正辭錄》三卷



 《朝會儀注》一卷元豐間



 《大禮前天興殿儀》二卷元豐間



 葉均《徽號冊寶儀注》一卷



 宋綬《內東門儀制》五卷



 李淑《閣門儀制》十二卷



 又《王後儀範》三卷



 梁顥《閣門儀制》十二卷



 又並《目錄》十四卷



 《閣門集例》並《目錄》、《大臣特恩》三十卷



 《閣門儀制》四卷



 《閣門令》四卷



 《蜀坤儀令》一卷



 《皇后冊禮儀範》八冊大觀間。卷亡



 《帝系後妃吉禮》
 並《目錄》一百一十卷重和元年



 王巖叟《中宮儀範》一部卷亡



 王與之《祭鼎儀範》六卷



 高中《六尚供奉式》二百冊卷亡



 王叡《雜錄》五卷



 《營造法式》二百五十冊元祐間。卷亡



 張直方《打球儀》一卷



 李詠《打球儀注》一卷



 《高麗入貢儀式條令》三十卷元豐間



 《高麗女真排辨式》一卷元豐間



 《諸蕃進貢令式》十六卷董氈、鬼章一,闍婆一,占城一,層檀一,大食一,勿巡一,注輦一,羅、龍、方、張、石蕃一,於闐,拂菻一,交州一,龜茲、回鶻一,伊州、西州、沙州一,三佛齊一,丹眉流一,大食陀婆離一,大俞盧和地一



 王晉《使範》一卷



 李商隱《使範》一卷



 《家範》十卷



 盧僎《家範》一
 卷



 司馬光《家範》一卷



 孟說《家祭儀》一卷



 周元陽《祭錄》一卷



 賈氏《葬王播儀》一卷



 鄭洵瑜《書儀》一卷



 杜有晉《書儀》二卷



 鄭餘慶《書儀》三卷



 右儀注類一百七十一部,三千四百三十八卷



 《律》十二卷



 《律疏》三十卷唐長孫無忌等撰



 《唐式》二十卷



 李林甫《開元新格》十卷



 又《令》三十卷



 《唐律令事類》四十卷



 《度支長行旨》五卷



 《大和格後敕》四十卷



 元泳《式苑》四卷



 宋璟《旁通開元格》一卷



 蕭旻《開元禮律格令要訣》一卷



 裴光
 庭《開元格令科要》一卷



 狄兼謨《開成刑法格》十卷



 《開成詳定格》十卷



 張戣《大中統類》十二卷



 《大中刑法總要》六十卷



 《大中已後雜敕》三卷



 《大中後雜敕》十二卷



 《梁令》三十卷



 《梁式》二十卷



 《梁格》十卷



 《天成長定格》一卷



 《天成雜敕》三卷



 《天福編敕》三十一卷



 張昭《顯德刑統》二十卷



 姜虔嗣《江南刑律統類》十卷



 《江南格令條》八十卷



 《蜀雜制敕》三卷



 盧紓《刑法要錄》十卷



 黃克升《五刑纂要錄》三卷



 《刑法纂要》十二卷



 《斷獄立成》三卷



 黃懋《刑法要例》八卷



 張員《法鑒》八卷



 田晉《章程體要》二卷



 王行先一作「仙」



 《令律手鑒》二卷



 張履冰《法例六贓圖》二卷



 張伾《判格》三卷



 盛度《沿革制置敕》三卷



 王皞《續疑獄集》四卷



 趙綽《律鑒》一卷



 《法要》一卷



 《外臺秘要》一卷



 《百司考選格敕》五卷



 《憲問》十卷



 《建隆編敕》四卷



 《開寶長定格》三卷



 《太平興國編敕》十五卷



 蘇易簡《淳化編敕》三十卷



 柴成務《咸平編敕》十二卷



 丁謂《農田敕》五卷



 陳彭年《大中祥符編敕》四十卷



 又《轉運司編敕》三十卷



 韓琦《端拱以來宣敕札子》六十
 卷



 又《嘉祐編敕》十八卷《總例》一卷



 晁迥《禮部考試進士敕》一卷



 呂夷簡《一司一務敕》三十卷



 賈昌朝《慶歷編敕》十二卷《總例》一卷



 《貢舉條制》十二卷至和二年



 吳奎《嘉祐錄令》十卷



 又《驛令》三卷



 《審官院編敕》十五卷



 王珪《在京諸司庫務條式》一百三十卷



 《銓曹格敕》十四卷



 孫奭《律音義》一卷



 王誨《群牧司編》十二卷



 張稚圭《大宗正司條》六卷



 王安禮《重修開封府熙寧編》十卷



 沈立《新修審官西院條貫》十卷又《總例》一卷



 《支賜式》十二卷



 《支賜式》二
 卷



 《官馬俸馬草料等式》九卷



 《熙寧新編大宗正司敕》八卷



 陳繹《熙寧編三司式》四百卷



 又《隨酒式》一卷



 《馬遞鋪特支式》二卷



 《熙寧新定諸軍直祿令》二卷



 曾肇《將作監式》五卷



 蒲宗孟《八路敕》一卷



 李承之《禮房條例》並《目錄》十九冊卷亡



 章惇《熙寧新定孝贈式》十五卷



 又《熙寧新定節式》二卷



 《熙寧新定時服式》六卷



 《熙寧新定皇親祿令》十卷



 《司農寺敕》一卷《式》一卷



 《熙寧將官敕》一卷



 吳充《熙寧詳定軍馬敕》五卷



 沉括《熙寧詳定諸色人廚料式》一卷



 《熙
 寧新修凡女道士給賜式》一卷



 《諸敕式》二十四卷



 《諸敕令格式》十二卷



 又《諸敕格式》三十卷



 張敘《熙寧葬式》五十五卷



 範鏜《熙寧詳定尚書刑部敕》一卷



 張誠一《熙寧五路義勇保甲敕》五卷《總例》一卷



 又《學士院等處敕式交並看詳》二十卷



 《御書院敕式令》二卷



 許將《熙寧開封府界保甲敕》二卷



 《申明》一卷



 沉希顏《元豐新近定在京人從敕式三等》卷亡



 李定《元豐新修國子監大學小學元新格》十卷



 又《令》十三卷



 賈昌朝《慶歷編敕》、《律學武學敕式》
 共二卷



 《武學敕令格式》一卷元豐間



 《明堂赦條》一卷元豐間



 曾伉《新修尚書吏部式》三卷



 蔡碩《元豐將官敕》十二卷



 《貢舉醫局龍圖天章寶文閣等敕令儀式》及《看詳》四百一十卷元豐間



 《宗室及外臣葬敕令式》九十二卷元豐間



 《皇親祿令並厘修敕式》三百四十卷



 吳雍《都提舉市易司敕令》並《厘正看詳》二十一卷、《公式》二卷元豐間



 《水部條》十九卷元豐間



 朱服《國子監支費令式》一卷



 元絳《讞獄集》十三卷



 崔臺符《元豐編敕令格式》並《赦書德音》、《申明》八十
 一卷



 《吏部四選敕令格式》一部元祐初。卷亡



 《元豐戶部敕令格式》一部元祐初。卷亡



 《六曹條貫》及《看詳》三千六百九十四冊元祐間。卷亡



 《元祐諸司市務敕令格式》二百六冊卷亡



 《六曹敕令格式》一千卷元祐初



 《紹聖續修武學敕令格式看詳》並《凈條》十八冊建中靖國初。卷亡



 《樞密院條》二十冊《看詳》三十冊元祐間。卷亡



 《紹聖續修律學敕令格式看詳》並《凈條》十二冊建中靖國初。卷亡



 《諸路州縣敕令格式》並《一時指揮》十三冊卷亡



 《六曹格子》十冊卷亡



 《中書省官制事目格》一百二十卷



 《
 尚書省官制事目格參照卷》六十七冊卷亡



 《門下省官制事目格》並《參照參卷舊文凈條厘析總目目錄》七十二冊卷亡



 《徽宗崇寧國子監算學敕令格式》並《對修看詳》一部卷亡



 《崇寧國子畫書學敕令格式》一部卷亡



 沈錫《崇寧改修法度》十卷



 《諸路州縣學法》一部大觀初。卷亡



 《大觀新修內東門司應奉禁中請給敕令格式》一部卷亡



 《國子大學闢雍並小學敕令格式申明一時指揮目錄看詳》一百六十八冊卷亡



 鄭居中《政和新修學法》一百三十卷



 李圖南《
 宗子大小學敕令格式》十五冊卷亡



 何執中《政和重修敕令格式》五百四十八冊卷亡



 《政和祿令格》等三百二十一冊卷亡



 《宗祀大禮敕令格式》一部政和間。卷亡



 張動《直達綱運法》並《看詳》一百三十一冊卷亡



 王韶《政和敕令式》九百三卷



 白時中《政和新修御試貢士敕令格式》一百五十九卷



 孟昌齡《政和重修國子監律學敕令格式》一百卷



 《接送高麗敕令格式》一部宣和初。卷亡



 《奉使高麗敕令格式》一部宣和初。卷亡



 《明堂敕令格式》一千二百六冊宣和初。卷亡



 《兩浙福
 建路敕令格式》一部宣和初。卷亡



 薛昂《神霄宮使司法令》一部卷亡



 劉次莊《青囊本旨論》一卷



 王晉《使範》一卷



 和凝《疑獄集》三卷



 竇儀《重詳定刑統》三十卷



 盧多遜《長定格》三卷



 呂夷簡《天聖編敕》十二卷



 《天聖令文》三十卷呂夷簡、夏竦等撰



 《八行八刑條》一卷大觀元年禦制



 《崇寧學制》一卷徽宗學校新法



 《附令敕》十八卷慶歷中編,不知作者



 《五服敕》一卷劉筠、宋綬等撰



 張方平《嘉祐驛令》三卷



 又《嘉祐祿令》十卷



 王安石《熙寧詳定編敕》等二十五卷



 《新編續降並敘法條貫》一卷編治平、熙寧詔旨並官吏
 犯罪敘法、條貫等事



 曾布《熙寧新編常平敕》二卷



 《審官東院編敕》二卷熙寧七年編



 張大中《編修入國條貫》二卷



 又《奉朝要錄》二卷



 範鏜《熙寧貢舉敕》二卷



 《八路差官敕》一卷編熙寧總條、審官東院條、流內銓條



 《熙寧法寺斷例》十二卷



 《熙寧歷任儀式》一卷不知作者



 蔡確《元豐司農敕令式》十七卷



 李承之《江湖淮浙鹽敕令賞格》六卷



 曾伉《元豐新修吏部敕令式》十五卷



 崔臺符《元豐敕令式》七十二卷



 呂惠卿《新史吏部式》二卷



 又《縣法》十卷



 程龜年《五服相犯法纂》三卷



 孫奭《律令
 釋文》一卷



 《續附敕令》一卷慶歷中編,不知作者



 《三司條約》一卷慶歷中纂集



 陸佃《國子監敕令格式》十九卷



 曾旼《刑名斷例》三卷



 章惇《元符敕令格式》一百三十四卷



 鄭居中《學制書》一百三十卷



 蔡京《政和續編諸路州縣學敕令格式》十八卷



 白時中《政和新修貢士敕令格式》五十一卷



 李元弼《作邑自箴》一卷



 張守《紹興重修敕令格式》一百二十五卷



 《紹興重修六曹寺監庫務通用敕令格式》五十四卷秦檜等撰



 《紹興重修吏部敕令格式》並《通用格式》一
 百二卷朱勝非等撰



 《紹興重修常平免役敕令格式》五十四卷秦檜等撰



 《紹興重修貢舉敕令格式申明》二十四卷紹興中進



 《紹興參附尚書吏部敕令格式》七十卷陳康伯等撰



 《紹興重修在京通用敕令格式申明》五十六卷紹興中進



 《大觀告格》一卷



 鄭克《折獄龜鑒》三卷



 《乾道重修敕令格式》一百二十卷虞允文等撰



 《淳熙重修吏部左選敕令格式申明》三百卷龔茂良等撰



 《諸軍班直錄令》一卷



 鄭至道《諭俗編》一卷



 趙緒《金科易覽》一卷



 劉高夫《金科玉律總括詩》三卷



 《金科
 玉律》一卷



 《金科類要》一卷



 《刑統賦解》一卷



 並不知作者



 韓琦《嘉祐詳定編敕》三十卷



 王日休《養賢錄》三十二卷



 《淳熙重修敕令格式》及《隨敕申明》二百四十八卷



 《淳熙吏部條法總類》四十卷淳熙二年敕令所編



 《慶元重修敕令格式》及《隨敕申明》二百五十六卷慶元三年詔重修



 《慶元條法事類》八十卷嘉泰元年敕令所編



 《開禧重修吏部七司敕令格式申明》三百二十三卷開禧元年上



 《嘉定編修百司吏職補授法》一百三十三卷嘉定六年上



 《嘉定編修吏部條法總類》五十卷嘉定中詔修



 趙仝《疑獄集》三卷



 《九族五服圖制》一卷不知何人編



 《大宗正司敕令格式申明》及《目錄》八十一卷紹興重修



 《編類諸路茶監敕令格式目錄》一卷



 右刑法類二百二十一部,七千九百五十五卷。



 吳兢《西齊書目錄》一卷



 毋煚《古今書錄》四十卷



 李肇《經史釋文》三卷



 朱遵度《群書麗藻目錄》五十卷



 《隆安西庫書目》二卷不知作者



 《唐秘閣四部書目》四卷



 《唐四庫搜訪圖書目》一卷



 《梁天下郡縣目》一卷



 《後唐統類目》一卷



 杜
 鎬《龍圖閣書目》七卷



 又《十九代史目》二卷



 《太清樓書目》四卷



 《王宸殿書目》四卷



 韋述《集賢書目》一卷



 《學士院雜撰目》一卷



 歐陽伸一作「坤」



 《經書目錄》十一卷



 楊九齡《經史書目》七卷



 楊松珍《歷代史目》十五卷



 宗諫注《十三代史目》十卷



 商仲茂《十三代史目》一卷



 《河南東齋一作「齊」



 史書目》三卷



 曾氏《史監》三卷



 孫玉汝《唐列聖實錄目》二十五卷



 《唐書敘例目錄》一卷



 沉建《樂府詩目錄》一卷



 蔣彧《書目》一卷



 劉德崇《家藏龜鑒目》十卷



 田鎬、尹植《文樞密要
 目》七卷



 劉沆《書目》二卷



 《禁書目錄》一卷學士院、司天監同定



 王堯臣、歐陽修《崇文總目》六十六卷



 《沉氏萬卷堂目錄》二卷



 歐陽修《集古錄》五卷



 李淑《邯鄲書目》十卷



 吳秘《家藏書目》二卷



 《秘閣書目》一卷



 《史館書新定書目錄》四卷不知作者



 李德芻《邯鄲再集書目》三十卷



 崔君授《京兆尹金石錄》十卷



 《國子監書目》一卷



 《荊州田氏書總目》三卷田鎬編



 劉涇《成都府古石刻總目》一卷



 趙明誠《金石錄》三十卷



 又《諸道石刻目錄》十卷



 徐士龍《求書補闕》一卷



 董逌《廣川
 藏書志》二十六卷



 鄭樵《求書闕記》七卷



 又《求書外記》十卷



 《集古系時錄》一卷



 《圖譜有無記》二卷



 《群玉會記》三十六卷



 陳貽範《穎川慶善樓家藏書目》二卷



 《遂初堂書目》二卷尤袤集



 《徐州江氏書目》二卷



 《呂氏書目》二卷



 《三川古刻總目》一卷



 《鄱陽吳氏籯金堂書目》三卷



 《孫氏群書目錄》二卷



 《紫雲樓書目》一卷



 《川中書籍目錄》二卷



 《秘書省書目》二卷



 陳騤《中興館閣書目》七十卷《序例》一卷



 石延慶、馮至游校勘《群書備檢》三卷



 晁公武《讀書志》四卷



 張攀《中興館閣續書目》三十卷



 《諸州書目》一卷



 滕強恕《東湖書自志》一卷



 右目錄類六十八部,六百七卷。



 何承天《姓苑》十卷



 林寶《姓苑》三卷



 又《姓史》四卷



 《元和姓纂》十卷



 《五姓證事》二十卷



 竇從一《系纂》七卷



 陳湘《姓林》五卷



 李利涉《姓氏秘略》三卷



 又《編古命氏》三卷



 《五聲類氏族》五卷



 孔平《姓系氏族》一卷



 《姓略》六卷



 崔日用《姓苑略》一卷



 魏子野《名字族》十卷



 《同姓名譜》六卷



 《尚書血脈》
 一卷



 《春秋氏族譜》一卷



 《春秋宗族謚譜》一卷



 《帝王歷記譜》二卷



 《帝系圖》一卷



 李匡文《天潢源派譜說一作「統」》一卷



 又《唐皇室維城錄》一卷



 又《李氏房從譜》一卷



 李茂嵩一作「高」



 《唐宗系譜》一卷



 《唐書總記帝系》三卷



 《宋玉牒》三十三卷



 《仁宗玉牒》四卷



 《英宗玉牒》四卷



 李衢《皇室維城錄》一卷



 宋敏求《韻類次宗室譜》五十卷



 司馬光《宗室世表》三卷



 《臣寮家譜》一卷



 黃恭之《孔子系葉傳》三卷



 《文宣王四十二一作「三」



 代家狀》一卷



 《闕里譜系》一卷



 趙異世《趙氏大
 宗血脈譜》一卷



 《趙氏龜鑒血脈圖錄記》一卷



 令狐峘《陸氏宗系碣》一卷



 陸師儒《陸氏英賢記》三卷



 《蔣王惲家譜》一卷



 王方慶《王氏譜》一卷



 《唐汭家譜》一卷



 劉復禮《劉氏大宗血脈譜》一卷



 《劉與家譜》一卷



 王僧孺《徐義倫家譜》一卷



 《李用休家譜》二卷



 《徐商徐詵家譜》四卷



 《周長球家譜》一卷



 《費氏家譜》一卷



 《錢氏集錄》三卷



 陸景獻《吳郡陸氏宗系譜》一卷



 毛漸《毛氏世譜》一部卷亡



 曾肇《曾氏譜圖》一部卷亡



 洪興祖《韓愈年譜》一卷



 周文《汝南周氏家譜》一卷



 崔
 班《歐陽家譜》一卷



 梁元帝《古今同姓名錄》二卷



 竇澄之《扶風竇氏血脈家譜》一卷



 李林甫《唐室新譜》一卷



 又《天下郡望姓氏族譜》一卷



 《唐相譜》一卷不知作者



 孔至《姓氏古今雜錄》一卷



 陶茇麟《陶氏家譜》一卷



 李匡文《元和縣主昭穆譜》一卷



 又《皇孫郡王譜》一卷



 《玉牒行樓》一卷



 《偕日譜》一卷



 邢曉《帝王血脈小史記》五卷



 又《帝王血脈圖小史後記》五卷



 韋述《百家類例》三卷



 韋述、蕭穎士《宰相甲族》一卷



 裴揚休《百氏譜》五卷



 曹大宗《姓源韻譜》一卷



 杜
 信《京兆杜氏家譜》一卷



 劉沆《劉氏家譜》一卷



 《唐顏氏家譜》一卷



 《韓吏部譜錄》二卷



 《李氏郇王家譜》一卷



 並不知作者



 唐邴《唐氏譜略》一卷



 楊侃《家譜》一卷



 《宋仙源積慶圖》一卷起僖祖迄哲宗



 《宗室齒序圖》一卷



 《天源類譜》一卷



 《祖宗屬籍譜》一卷



 《向敏中家譜》一卷向緘撰



 邵思《姓解》三卷



 錢惟演《錢氏慶系譜》二卷



 王回《清河崔氏譜》一卷



 孫秘《尊祖論世錄》一卷



 蘇洵《蘇氏族譜》一卷



 錢明逸《熙寧姓纂》六卷



 魏予野《古今通系圖》一卷



 李復《南陽李英公家譜》一
 卷



 成鐸《文宣王家譜》一卷



 吳逵《帝王系譜》一卷



 黃邦俊《群史姓纂韻》六卷



 顏嶼《袞國公正枝譜》一卷



 手採真子《千姓編》一卷



 《符彥卿家譜》一卷符承宗撰



 《建陽陳氏家譜》一卷



 《萬氏譜》一卷



 《趙郡東祖李氏家譜》二卷



 《鮮于氏血脈圖》一卷



 《長樂林氏家譜》一卷



 並不知作者



 丁維皋《百族譜》三卷



 鄧名世《古今姓氏書辨證》四十卷



 李燾《晉司馬氏本支一卷》



 又《齊梁本支》一卷



 徐筠《姓氏源流考》七十八卷



 李氏《歷代諸史總括姓氏錄》一
 卷



 右譜牒類一百十部,四百三十七卷。



 桑欽《水經》四十卷酈道元注



 《城塚記》一卷按序,魏文帝三年,劉裕得此記



 葛洪《關中記》一卷



 雷次宗《豫章古今記》三卷



 沈懷遠《南越志》五卷



 梁元帝《職貢圖》一卷



 楊衒之《洛陽伽藍記》三卷



 《煬帝開河記》一卷不知作者



 魏王泰《坤元錄》十卷



 沙門辨機《大唐西域圖記》十二卷



 梁載言《十道四蕃志》十五卷



 韋述《兩京新記》五卷



 達奚弘通《西南海蕃行記》一卷



 馬溫之《鄴都故事》二卷



 李吉甫《元和郡國圖志》四十卷



 元結《九
 疑山圖記》一卷



 賈耽《皇華四達》十卷



 又《貞元十道錄》四卷



 《國要圖》一卷



 《方志圖》二卷



 《三代地理志》六卷



 《地理論》六卷



 劉之推《文括九土一作「州」



 要略》三卷



 樂史《坐知天下記》四十卷



 王曾《九域圖》三卷



 王洙《皇祐方域記》三十卷



 《要覽》一卷



 韓鬱《十道四蕃引》一卷



 趙珣《開元分野圖》一卷



 又《十道記》一卷



 《十八路圖》一卷《圖副》二十卷熙寧間天下州府車監縣鎮圖



 李德芻《元豐郡縣志》三十卷《圖》三卷



 沉括《天下郡縣圖》一部卷亡



 陳坤臣《郡國人物志》一百五十卷



 歐
 陽忞《巨鰲記》五卷



 孫結《唐國鑒圖》一卷



 曹璠《國照》十卷



 又《元和國計圖》十卷



 韋澳《諸道山河地名要略》九卷一名《處分語》,一名《新集地理書》



 陳延禧《隋朝洛都記》一卷



 又《蜀北路秦程記》一卷



 《北征雜記》一卷



 姜嶼《明越風物志》七卷



 元廣之《金陵地記》六卷



 劉公鉉《鄴城新記》三卷



 李璋《太原事跡》十四卷



 盧求《襄陽故事》十卷



 《湘中記》一卷



 餘知古《渚宮故事》十卷



 張周封《華陽風俗錄》一卷



 韓昱《江州事跡》三卷張密注



 韋宙一作「寅」



 《零陵錄》一卷



 楊備恩《蜀都故事》二
 卷



 許嵩《六朝宮苑記》二卷



 邢昺《景德朝陵地理記》三十卷



 韋齊一作「濟」



 休《雲南行記》二卷



 馬敬寔《諸道行程血脈圖》一卷



 陳隱之《續南荒錄》一卷



 韋皋一作「阜」



 《西南夷事狀》二十卷



 《西戎記》二卷



 張建章《渤海國記》三卷



 顧愔《新羅國記》一卷



 達奚洪一作「通」



 《海外三十六國記》一卷



 《雲南風俗錄》十卷



 辛怡顯《至道雲南錄》三卷



 李德裕《黠戛斯朝貢圖》一卷



 崔峽《列國入貢圖》二十卷



 郭璞《山海經贊》二卷



 元結《諸山記》一卷



 《岳瀆福地圖》一卷



 盧鴻《嵩岳記》一
 卷



 《華山記》一卷



 《衡山記》一卷



 《峨眉山記》二卷



 僧法琳《廬山記》一卷



 陸鴻漸《顧渚山記》一卷



 令狐見堯《玉笥山記》一卷



 沈立《蜀江志》十卷



 《宣和編類河防書》一百九十二卷



 東方朔《十洲記》一卷



 張華《異物評》二卷



 劉恂《嶺表錄異》三卷



 《嶺表異物志》一卷



 孟管《嶺南異物志》一卷



 《南海異事》五卷



 鄭虔《天寶軍防錄》一卷



 林特《會稽錄》三十卷



 盛度《庸調租賦》三卷



 陳傳《歐冶拾遺》一卷



 毛漸《地理五龍秘法》一部卷亡



 林住《閩中記》十卷



 盧肇《海潮賦》一卷



 僧
 應物《九華山記》二卷



 又《九華山舊錄》一卷



 盧求《成都記》五卷



 樊綽《雲南志》十卷



 又《南蠻記》十卷



 李居一《王屋山記》一卷



 徐云虔《南詔錄》三卷



 韋莊《蜀程記》一卷



 又《峽程記》一卷



 莫休符《桂林風土記》一卷



 章僚《海外使程廣記》三卷



 張建章《戴鬥諸蕃記》一卷



 曹璠《須知國鏡》二卷



 王權《大梁夷門記》一卷



 吳從政《襄沔雜記》三卷



 竇滂《雲南別錄》一卷



 陸廣微《吳地記》一卷



 曹大宗《郡國志》二卷



 韋瑾《域中郡國山川圖經》一卷



 《唐夷狄貢》一卷



 《兩京道里記》
 三卷不知作者



 張修《九江新舊錄》三卷



 張氏《燕吳行役記》二卷不知作者



 羅含《湘中山水記》三卷



 平居誨《于闐國行程錄》一卷



 胡嶠《陷虜記》一卷



 王德璉《鄱陽縣記》一卷



 徐鍇《方輿記》一百三十卷



 範子長《皇州郡縣志》一百卷



 司馬儼《峽山履平集》一卷



 潘子韶《峽江利涉集》一卷



 杜光庭《續成都記》一卷



 範旻《邕管雜記》三卷



 李昉《歷代宮殿名》一卷



 樂史《太平寰宇記》二百卷



 魏羽《吳會雜錄》一卷



 張參《江左記》三卷



 陶岳《零陵總記》十五卷



 李宗諤《圖經》九十
 八卷



 又《圖經》七十七卷



 《越州圖經》九卷



 《陽明洞天圖經》十五卷



 李垂《導河形勝書》一卷



 王曾《契丹志》一卷



 楊備《恩平郡譜》一卷



 劉夔《武夷山記》一卷



 林世程《重修閩中記》十卷



 郭之美《羅浮山記》一卷



 周衡《湘中新記》七卷



 陳倩《茅山記》一卷



 僧文政《南嶽尋勝錄》一卷



 李上交《豫章西山記》二卷



 《廣西郡邑圖志》一卷張維序。



 王靖《廣東會要》四卷



 張田《廣西會要》二卷



 劉昌詩《六峰志》十卷



 薛常州《地理叢考》一卷



 李和篪《輿地要覽》二十三卷



 《重修徐州
 圖經》三卷嘉定中撰



 《離(崔十)志》十卷



 《雁山行記》一卷不知何人編



 王日休《九丘總要》三百四十卷



 餘哲《聖域記》二十五卷



 程大昌《雍錄》十卷



 錢景衎《南嶽勝概》一卷



 曾洵《句曲山記》七卷



 周淙《臨安志》十五卷



 談鑰《吳興志》二十卷



 潘廷立《富川圖志》六卷



 韓挺《儀真志》七卷



 劉浩然《合肥志》十卷



 李說《黃州圖經》五卷



 童宗說《盱江志》十卷



 姜得平又《續志》十卷



 袁震《臨江軍圖經》七卷



 李伸《重修臨江志》七卷



 雷孝友《瑞州郡縣志》十九卷



 田渭《辰州風土記》六卷



 袁
 觀《潼川府圖經》十一卷



 張津《四明圖經》十二卷



 史正志《建康志》十卷



 江文叔《桂林志》一卷



 蔡戡《靜江府圖志》十二卷



 熊克《鎮江志》十卷



 葛元騭《武陽志》十卷



 宋宜之《無為志》三卷



 胡兆《秋浦志》八卷



 羅願《新安志》十卷



 汪師孟《黃山圖經》一卷



 範成大《桂海虞衡志》三卷



 韋楫《昭潭志》二卷



 晁百揆《潯陽志》十二卷



 吳蕓《沅州圖經》四卷



 《安南土貢風俗》一卷乾道中安南入貢,客省承詔具其風俗及貢物名數



 程九萬《歷陽志》十卷



 蘇思恭《曲江志》十二卷



 毛憲《信安志》十六卷



 《臨
 賀郡志》一卷不知作者



 蕭玠《晉康志》七卷



 周端朝《桂陽志》五卷



 劉子登《武陵圖經》十四卷



 鄭昉《都梁志》二卷



 《赤城志》四十卷陳耆卿序



 陸游《會稽志》二十卷



 王中行《潮州記》一卷



 《莆陽人物志》三卷鄭僑序



 王震《閬苑記》三十卷



 冉木《潛藩武泰志》十四卷



 趙抃《成都古今集記》三十卷



 張朏《齊記》一卷



 《南北對鏡圖》一卷



 《混一圖》一卷



 《西南蠻夷朝貢圖》一卷



 《巨鰲記》六卷



 《交廣圖》一卷



 《平江府五縣正圖經》二卷



 並不知作者



 李華《湟川開峽志》五卷



 宋敏求《長安志》一十
 卷



 又《東京記》二卷



 《河南志》二十卷



 陳舜俞《廬山記》二卷



 謝頤素《海潮圖論》一卷



 王瓘《北道刊誤志》十五卷



 林須《霍山記》一卷



 檀林《甌治拾遺》一卷



 又《大理國行程》一卷



 陳冠《熙河六州圖記》一卷



 王向弼《龍門記》三卷



 王存《九域志》十卷



 孟猷《上饒志》十卷



 滕宗諒《九華山新錄》一卷



 朱長文《吳郡圖經續記》三卷



 王正倫《古今洛城事類》二卷



 王得臣《江夏辨疑》一卷



 譚掞《邕管溪洞雜記》一卷



 李洪《鎮洮補遺》一卷



 李獻父《隆慮洞天錄》一卷



 林□票《永陽
 志》三十五卷



 曾旼《永陽郡縣圖志》四卷



 劉拯《濠上摭遺》一卷



 蘇氏《夏國樞要》二卷



 左文質《吳興統記》十卷



 孫穆《雞林類事》三卷



 馬子嚴《岳陽志》二卷



 程演《職方機要》四十卷



 範致明《岳陽風土記》一卷



 又《池陽記》一卷



 歐陽忞《輿地廣記》三十八卷



 虞剛簡《永康軍圖志》二十卷



 錢紳《同安志》十卷



 徐兢《宣和奉使高麗圖經》四十卷



 吳致堯《九疑考古》二卷



 洪芻《豫章職方乘》三卷



 董棻《嚴州圖經》八卷



 厲居正《齊安志》二十卷



 洪遵《東陽志》十卷



 許靖夫《
 齊安拾遺》一卷



 環中《汴都名實志》三卷



 陳哲夫《李渠志》一卷



 《續修宜春志》十卷



 唐稷《清源人物志》十三卷



 李盛《章貢志》十二卷



 曾賁《括蒼續志》十卷



 陳柏朋《括蒼志》一卷



 趙彥勵《莆陽志》十五卷



 陸琰《莆陽志》七卷



 李獻父《相臺志》十二卷



 《江行圖志》一卷沉該訂正,不知作者



 《同安後志》十卷



 《大禹治水玄奧錄》一卷



 《三輔黃圖》一卷



 《高麗日本傳》一卷



 《南劍州圖經》一卷



 《地裡圖》一卷



 《指掌圖》二卷



 《南海錄》一卷



 《福建地理圖》一卷



 《泉南錄》二卷



 《吳興雜錄》七卷



 《南朝
 宮苑記》一卷



 《廬山事跡》三卷



 並不知作者



 李常《續廬山記》一卷



 《東京至益州地裡圖》卷亡



 《四明山記》一卷



 《地裡圖》一卷



 《南岳衡山記》一卷



 《考城圖經》一卷



 《常州風土記》一卷



 《清溪山記》一卷



 《水山記》一卷



 《茅山新記》一卷



 《青城山記》一卷



 《契丹國土記、契丹疆宇圖》二卷



 《契丹地裡圖》一卷



 並不知作者



 李幼傑《莆陽比事》七卷



 何友諒《武陽志》二十七卷



 陳謙《永寧編》十五卷



 黃以寧《惠陽志》十卷



 劉牧《建安志》二十四卷



 又《建安續志類編》二卷



 鄒孟卿《寧武志》十五
 卷



 李皋《汀州志》八卷



 林英發《景陵志》十四卷



 楊彥為《保昌志》八卷



 傅巖《鄖城志》十二卷



 楊泰之《普州志》三十卷



 孫祖義《高郵志》三卷



 宇文紹奕《臨邛志》二十卷



 又《補遺》十卷



 林晡《姑熟志》五卷



 王招《蕪湖圖志》九卷



 楊□□《臨漳志》十卷



 方傑《清漳新志》十卷



 章穎《文州古今記》十二卷



 杜孝嚴《文州續記》四卷



 孫楙《舂陵圖志》十卷



 張貴謨《臨汝圖志》十五卷



 徐自明《零陵志》十卷



 又《浮光圖志》三卷



 梁克家《長樂志》四十卷



 張埏《零陵志》十卷



 陸峻、丁光遠《
 蘄春志》十卷



 段子游《均州圖經》五卷



 李韋之《邵陽圖志》三卷



 黃汰《邵陽紀舊》一卷



 鞏嶸《邵陵類考》二卷



 孫顯祖《靖州圖經》四卷



 黃曄《龜山志》三卷



 李震《彭門古今集志》二十卷



 蔡畤《續同安志》一卷



 程叔達《隆興續職方乘》十卷



 項預《吳陵志》十四卷



 朱端章《南康記》八卷



 又《廬山拾遺》二十卷



 練文《廬州志》十卷



 吳機《吉州記》三十四卷



 錢之望、吳莘《楚州圖經》二卷



 劉宗《襄陽志》四十卷



 劉清之《衡州圖經》三卷



 趙甲《隆山志》三十六卷



 鄒補之《毗陵志》
 十二卷



 王銖《荊門志》十卷



 張孝曾《富水志》十卷



 王棨《重修荊門志》十卷



 徐得之《郴江記》八卷



 史本《古沔志》一卷



 周夢祥《贛州圖經》卷亡



 閻蒼舒《興元志》二十卷



 許開《南安志》二十卷



 孫昭先《淮南通川志》十卷



 余元一《清湘志》六卷



 鄭少魏《廣陵志》十二卷



 褚孝錫《長沙志》十一卷



 鄭紳《桂陽圖志》六卷



 黃疇若《龍城圖志》十卷



 胡至《重修龍城圖志》十卷



 陳宇《房州圖經》三卷



 虞太中《臨封志》三卷



 曹叔達《永嘉志》二十四卷



 周澄《永嘉志》七卷



 鄭應申《江陰
 志》十卷



 梁希夷《新昌志》一卷



 馬景修《通川志》十五卷



 黃環《夷陵志》六卷



 馬導《夔州志》十三卷



 《四明風俗賦》一卷不在何人撰



 丁介《武陵郡離合記》六卷



 史定之《番陽志》三十卷



 楊潛《雲間志》三卷



 徐筠《修水志》十卷



 張元成《嘉禾志》四卷



 鄧樞《鶴山叢志》十卷



 王寬夫《古涪志》十七卷



 李棣《浮光圖志》二十卷



 林仁伯《古歸志》十卷



 趙興清《歷陽志補遺》十卷



 王知新《合淝志》十卷



 霍篪《澧陽圖志》八卷



 劉伋《陵水圖志》三卷



 胡槻《普寧志》三卷



 王寅孫《沈黎志》二
 十三卷



 趙汝廈《程江志》五卷



 又《瓊管圖經》十六卷



 劉灝《清源志》七卷



 沉作賓、趙不跡《會稽志》二十卷



 邵笥《括蒼慶元志》一卷



 趙善贛《通義志》三十五卷



 張士牷《西和州志》十九卷



 李修己《同谷志》十七卷



 李錡《續同谷志》十卷



 義太初《高涼圖志》七卷



 趙師岌《潮州圖經》二卷



 鄭鄖《洋州古今志》十六卷



 張心達《甘泉志》十五卷



 陳峴《南海志》十三卷



 趙伯謙《韶州新圖經》十二卷



 俞聞中《敘州圖經》三十卷



 黎伯巽《靜南志》十二卷



 任逢《墊江志》三十卷



 劉德
 禮《夔州圖經》四卷



 馬紆《續廬山記》四卷



 《江州圖經》一卷



 《宕渠志》二卷



 《吉陽軍圖經》一卷



 《忠州圖經》一卷



 《珍州圖經》三卷



 《衢州圖經》一卷



 《沅州圖經》四卷



 《復州圖經》三卷



 《果州圖經》五卷



 《思州圖經》一卷



 《南平軍圖經》一卷



 《大寧監圖經》六卷



 並不知作者



 右地理類四百七部,五千一百九十六卷。



 《越絕書》十五卷或云子貢所作



 趙曄《吳越春秋》十卷



 司馬彪《九州春秋》九卷



 常璩《華陽國志》十二卷



 和苞《漢趙記》一卷



 範亨《燕書》二十卷



 蕭方等《三十國春秋》三十卷



 《三十國春秋鈔》一卷不知作者



 吳信都鎬《淝上英雄小錄》二卷



 《吳錄》二十卷徐鉉、高遠、喬舜、潘佐等撰



 《南唐書》十五卷不知作者



 王顏《南唐烈祖開基志》十卷



 李昊《蜀書》二十卷



 蔣文懌《閩中實錄》十卷



 林仁志《王氏解除運圖》三卷



 毛文錫《前蜀王氏記事》二卷



 《吳越備史》十五卷吳越錢儼托名範坰、林禹撰



 錢儼《備史遺事》五卷



 王保衡《晉陽見聞要錄》一卷



 董淳《後蜀孟氏記事》三卷



 徐鉉、湯悅《江南錄》十卷



 路振《九國志》五十一卷



 又《楚
 書》五卷



 鄭文寶《南唐近事集》一卷



 又《江表志》二卷



 陳彭年《江南別錄》四卷



 龍袞《江南野史》二十卷



 曾顏《渤海行年記》十卷



 胡賓王《劉氏興亡錄》一卷



 陶岳《荊湘近事》十卷



 周羽沖《三楚新錄》三卷



 曹衍《湖湘馬氏故事》二十卷



 王舉《天下大定錄》十卷



 盧臧《楚錄》五卷



 張唐英《蜀檮杌》十卷



 劉恕《十國紀年》四十卷



 《閩王事跡》一卷



 《高氏世家》十卷



 《湖南故事》十三卷



 《十國載記》三卷



 《江南餘載》二卷



 《高皇帝過江事實》一卷



 《廣王事跡》一卷



 並不知作者



 錢惟演《
 家王故事》一卷



 右霸史類四十四部,四百九十八卷。



 凡史類二千一百四十七部,四萬三千一百九卷



\end{pinyinscope}