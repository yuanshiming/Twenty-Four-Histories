\article{志第一百一十五 職官二}

\begin{pinyinscope}

 樞密院宣徽院三司使翰林學士院侍讀侍講崇政殿說書諸殿學士諸閣學士諸修撰直閣東宮官王府官



 樞密院掌軍國機務、兵防、邊備、戎馬之政令,出納密命,
 以佐邦治。凡侍衛諸班直、內外禁兵招募、閱試、遷補、屯戍、賞罰之事,皆掌之。以升揀、廢置揭帖兵籍;有調發更戍,則遣使給降兵符。除授內侍省官及武選官,將領路分都監、緣邊都巡檢使以上。大事則稟奏,其付授者用宣;小事則擬進,其付授者用扎。先具所得旨,關門下省審覆。面得旨者為錄白,批奏得畫者為畫旨,並留為底。惟以白紙錄送,皆候報施行。其被御寶批旨者,即送門下省繳覆。應給誥者,關中書省命詞。即事干大計,造作、
 支移軍器,及除都副承旨、三衙管軍、三路沿邊帥臣、太僕寺官,文臣換右職,仍同三省取旨。



 宋初,循唐、五代之制,置樞密院,與中書對持文武二柄,號為「二府」。院在中書之北,印有「東院」、「西院」之文,共為一院,但行東院印。而職事條目頗多。神宗初政,乃省其務之細者歸之有司,而增置審官西院,專領閣門祗候以上至諸司使差遣。官制行,隨事分隸六曹,專以本兵為職,而國信、民兵、牧馬總領,仍舊隸焉。舊分四房,曰兵,曰吏,曰戶,曰禮,至是
 厘正,凡分房十。其後,又增支馬、小吏二房,凡房十有二:曰北面房,掌行河北、河東路吏卒,北界邊防、國信事。曰河西房,掌行陜西路、麟、府、豐、嵐、石、隰州、保德軍吏卒,西界邊防、蕃官。曰支差房,掌行調發軍,湖北路邊防及京東、京西、江、淮、廣南東路吏卒,遷補殿侍,選親事官。曰在京房,掌行殿前步軍司事,支移兵器,川陜路邊防及畿內、福建路吏卒,軍頭、皇城司衛兵。曰教閱房,掌行中外校習,封樁闕額請給,催督驛遞及湖南路邊防。曰廣西
 房,掌行招軍捕盜賞罰,廣南西路邊防及兩浙路吏卒。而禁軍轉員,則各隨其房之所領兵額治之。曰兵籍房,掌行諸路將官差發禁兵、選補衛軍文書。曰民兵房,掌行三路保甲、弓箭手。曰吏房,掌行差將領武臣知州軍、路分都監以上及差內侍官文書。曰知雜房,掌行雜務。曰支馬房,掌行內外馬政並坊院監牧吏卒、牧馬、租課。曰小吏房,掌行兩省內臣磨勘功過敘用,大使臣已上歷任事狀及校尉以上改轉遷遣。吏三十有八:逐房副
 承旨三人,主事五人,守闕主事二人,令史十三人,書令史十五人。元祐既創支馬、小吏二房,增令史為十四人,書令史十九人,創正名貼房十八人。大觀增逐房副承旨為五人,創守闕書令史三人,增正名二十八人。



 中書、密院既稱「二府」,每朝奏事,與中書先後上殿。慶歷中,二邊用兵,知制誥富弼建言,邊事系國安危,不當專委樞密。仁宗以為然,即詔中書同議。諫官張方平亦言中書宜知兵事,乃以宰相呂夷簡、章得像並兼樞密使。熙寧
 初,滕甫言:「中書、密院議邊事,多不合。趙明與西人戰,中書賞功,而密院降約束;郭逵修堡柵,密院方詰之,而中書以下褒詔。願大臣凡戰守、除帥,議同而後下。」神宗善之。元祐四年,知樞密院安燾以母憂去職,樞密院官偶獨員。諫議大夫梁燾、司諫劉安世言:「國朝革五代之弊,文、武二柄,未嘗專付一人,乞依故事命大臣兼領。」靖康元年,知樞密院事李綱言:「在祖宗之時,樞密掌兵籍、虎符,三衙管諸軍,率臣主兵柄,各有分守,所以維持軍政,
 萬世不易之法。自童貫以領樞密院事為宣撫使,既主兵權,又掌兵籍、虎符,今日不可不戒。乞將團結到勤王正兵並付制置使,行營司兵付三衙。」從之。



 樞密使知院事同知院事樞密副使簽書院事同簽書院事樞密使知院事,佐天子執兵政,而同知、副使、簽書為之貳。凡邊防軍旅之常務,與三省分班稟奏;事干國體,則宰相、執政官合奏;大祭祀則迭為獻官。



 國初,官無定制,有使則置副,有知院則置同知院,
 資淺則用直學士簽書院事。熙寧元年,文彥博、呂公弼為使,韓維、邵亢為副使。時陳升之三至樞府,神宗欲稍異其禮,乃以為知院事。於是知院與使、副並置。元豐五年,將改官制,議者欲廢密院歸兵部。帝曰:「祖宗不以兵柄歸有司,故專命官以統之,互相維制,何可廢也?」於是得不廢。帝又以樞密聯職輔弼,非出使之官,乃定置知院、同知院二人,使、副悉罷。元祐初,復置簽書院事,仍以樞密直學士充。同簽書樞密院事,治平末,以殿前都虞
 候郭逵為之,又以逵判渭州。帝初即位,中丞王陶、御史呂景等皆言之。逵歸,改除宣徽南院使、知鄆州,自是不復置。政和六年,以內侍童貫權簽書樞密院河西、北面房事。七年,貫宣撫陜西、河東北三路,帶同簽書樞密院。既而詔元豐官制即無同簽書樞密院事,改為權領樞密院。然簽書院事,元豐亦未嘗置。宣和元年,詔童貫領樞密院事,後復以鄭居中為之。



 建炎初,置御營司,以宰相為之使。四年,罷,以其事歸樞密院機速房,命宰相範
 宗尹兼知樞密院。紹興七年,詔:「樞密本兵之地,事權宜重。可依故事置樞密使,以宰相張浚兼之。」又詔立班序立依宰相例。其後或兼或否。至開禧,以宰臣兼使,遂為永制。使與知院,同知、副使,亦或並除,其簽書、同簽書並為端明殿學士,恩數特依執政,或以武臣為之,亦異典也。



 都承旨副都承旨掌承宣旨命,通領院務。若便殿侍立,閱試禁衛兵校,則隨事敷奏,承所得旨以授有司,蕃
 國入見亦如之。檢察主事以下功過及遷補之事。都承旨,舊用院吏遞遷。熙寧三年,始以東上合門使李評為之,又以皇城使李綬為之副,更用士人自評、綬始。是月,詔都承旨、副都承旨見樞密使、副如合門使禮。五年,以同修起居注曾孝寬兼都承旨,參用儒臣自孝寬始。元豐四年,客省使張誠一為都承旨。都承旨復用武臣,自誠一始。元祐初,復以文臣為都承旨。其後以待制充。元符三年,王師約為都承旨,左司諫陳瓘言:「神考以文臣
 為都承旨,其副則參求外戚武臣之可用者。今師約未歷邊任,擢置樞屬掾文臣之位,甚非神考設官之意。」至崇寧以後,專用武臣。



 建炎四年,高宗在會稽,以武臣辛道宗為都承旨,頗用事。紹興元年,道宗既免,乃詔依元祐職制,都承旨以兩制為之。如未曾任侍從之人,即依權侍郎法,又或加學士、待制、修撰貼職。乾道初,再用武臣,自張說始。淳熙九年,都承旨復用士人,自蕭燧始。副都承旨文、武通除。



 檢詳官熙寧四年置,視中書檢正官。元豐初,定以三員,及改官制,罷之。建炎三年,復置檢詳兩員,敘位在左、右司之下。紹興二年減一員。



 計議官四員。建炎四年,罷御營使司,並歸樞密院為機速房。隨司減罷屬官,置乾辦官四員,詔並改為計議官。至紹興十一年減罷。



 編修官隨事置,無定員,以本院官兼者,不入銜。熙寧三年,以王存、顧臨等同編修《經武要略》,兼刪定諸房例冊。
 初擬都、副承旨提舉,神宗謂存等皆館職,不欲令承旨提舉,詔改為管幹。紹聖四年,編修刑部、軍馬司事,令都、副承旨兼領。政和七年,編修《北邊條例》,又別置詳覆官。



 講議司崇寧元年,以尚書省講議武備房歸樞密院置,以知院蔡卞提舉。三年,卞奏武備本院諸房可行,不必專局,乃罷之。紹興置編修官二員。



 監三省、樞密院門舊系差小使臣及內侍官充。嘉定六年,詔以曾經作縣、通判資序人充。小使臣省罷,內侍官改以三省、樞密院門
 機察官系銜。



 主管三省、樞密院架閣文字一員,嘉定八年置,以選人、京朝官通差。



 三省、樞密院激賞庫三省、樞密院激賞酒庫監官各二人。初以武臣,嘉泰末,始易以選人。



 二庫並因紹興用兵,創以備邊;後兵罷,專以備堂、東兩廚應幹宰執支遣。若朝廷軍期急速錢物金帶,以備激犒;諸軍將帥告命綾紙,以備科撥調遣等用。省、院、府吏胥之給,亦取具焉。



 御營使提舉修政局制國用使都督諸路軍馬中興,多以宰相兼領兵政、財用之事,而執政同預焉。因事創名,未久遄罷,可以不書。以其關宰相設施,因記其名稱本末附見焉。



 建炎元年,置御營司,以宰相為之使,仍以執政官兼副使。其屬有參贊軍事,以侍從官兼;提舉一行事務,以大將兼。其將佐有都統制及五軍統制以下官。初以總齊行在軍中之政。三年,詔御營使司止管行在五軍營砦事務,其餘應乾邊防措置等事,厘正
 歸三省、樞密院。四年,詔自今宰相兼知樞密院事,罷御營使。時臣僚言:「宰相之職,無所不統。本朝沿五代之制,政事分為兩府,兵權付於樞密,比年又置御營使,是政出於三也。請罷御營司,以兵權付之密院,而以宰相兼知,庶幾可以漸議兵政。」故罷使及官屬,以其事歸密院,為機速房。至紹興二十九年九月,詔:「祖宗舊制,樞密院即無機速房,合行減罷。」紹興三十一年,金主亮來攻,帝將臨江視師。其冬,以和義郡王楊存中為御營宿衛使,兵罷復免。明,孝宗即位,又以御營使命之。然但自名一司,掌殿前忠勇等軍,非復建
 炎之比,未幾而。存中非宰執,附見於此。



 紹興二年,詔置修政局,令百官條具修車馬、備器械,命右相秦檜提舉,參知政事同領之。其下有參詳官一人,侍從為之,參議官二人,檢討官四人,卿郎為之;如講議司故事。三月而罷局。



 乾道四年,詔:「理財之要,裕財為重,自今宰相可帶兼制國用使,參政可同知國用事。」



 先是,臣僚言:「近以宰相兼樞密使,蓋欲使宰相知兵也。宰相今雖知兵,而財穀出入之原,宰相猶未知也。望法李唐之制,委宰相兼領三司使職事,財穀出納之大綱,宰相領之於上,而戶部治其凡。」故有是命。五年二月,罷國用司。八年,詔:「官制已定,丞相
 事無不統,所有國用一司,與參知政事並不兼帶。」嘉泰四年,詔遵孝宗典故,宰相兼國用使,參知政事同知國用事,仍於侍從、卿監中擇二人充屬官。右丞相陳自強兼國用使,參知政事兼知樞密院事費士寅、參知政事張嚴兼同知國用事。



 以兵部侍郎薛叔似兼參計官,太府卿陳景思同參計官。先是,臣僚言:「今日財計,非錢穀不足可憂,而滲漏日滋之為可慮也。周家以塚宰制國用,而唐亦以宰相兼領度支,是知財賦國家之大計,其出入之數有餘、不足,為大臣者皆所當知,庶可節以制度,關防欺隱。宜略仿祖宗遺意,命大臣兼提領天下財賦。」從之。陳自強罷,亦廢。



 紹興五年,制以左通
 議大夫、尚書左僕射、同中書門下平章事兼知樞密院事趙鼎,左政奉大夫、尚書右僕射、同中書門下平章事兼知樞密院事張浚都督諸路軍馬。未幾,浚暫往江上措置邊防,至七年秋廢罷。其餘宰臣、執政開府於外者,別載於篇。



 編修敕令所提舉宰相兼。



 同提舉執政兼。



 詳定侍從官兼。刪定官就職事官內差兼?



 掌裒集詔旨,纂類成書。紹興十二年罷。乾道六年,復置詳定一司敕令所,以右丞相虞允
 文提舉,參知政事梁克家同提舉。淳熙十五年省罷,紹熙二年復置局。慶元二年,復置提舉,以右丞相餘端禮兼,同提舉以參知政事京鏜兼,仍以編修敕令所為名。



 宣徽院宣徽南院使北院使掌總領內諸司及三班內侍之籍,郊祀、朝會、宴饗供帳之儀,應內外進奉,悉檢視其名物。舊制,以檢校為使,或領節度及兩使留後,闕則樞密副使一人兼領二使,亦有兼樞密副使、簽書樞密院
 者。南院資望比北院頗優,然皆通掌,止用南院印,二使共院而各設廳事。其吏史則有都勾押官、勾押官各一人,前行三人,後行十二人,分掌四案:一曰兵案,二曰騎案,主賜群臣新史,及掌諸司使至崇班、內侍供奉官、諸司工匠兵卒之名籍,及三班而下遷補、假故、鞫劾之事。



 三曰倉案,掌春秋及聖節大宴、節度使迎授恩賜、上元張燈、四時祠祭及契丹朝貢、內廷學士赴上,並督其供帳,內外進奉視其名物,教坊伶人歲給衣帶,專其奏覆。四曰冑案。掌郊祀、御殿、朝謁聖容、賜酺國忌供帳之事,諸司使副、三班使臣別籍分產,司其條制,頒諸司工匠休假之。故事,與參知政事、樞密副使、同知樞密院事以先後入敘位。熙
 寧四年,詔位參政、樞副、同知下,著為令。九年,詔:「今後遇以職事侍殿上,或中書、樞密院合班問聖體,及非次慶賀,並特序二府班。官制行,罷宣徽院,以職事分隸省、寺,而使號猶存。



 初,吏部尚書王拱辰治平中知大名府,神宗即位,拜太子少保。明年,檢校太傅,改宣徽北院使,尋遷南院,立班序位視簽樞。元豐六年,拱辰除武安軍節度使再任,自此遂罷使名不復除。獨太子少師張方平許依舊領南院使致仕。哲宗即位,始遷太子太保而
 罷使名。元祐三年,復置南、北院使,儀品恩數如舊制。六年,以馮京為南院使,而方平亦復使名。中書舍人韓川言:「祖宗設此官,禮均二府,以待勛舊,未嘗帶以致仕。且宣徽,武官也;宮保,文官也,不宜混並。」不聽。方平亦固辭不拜。七年,馮京亦以使致仕。紹聖三年,議者言官名雖復,而無所治之事,乃罷之。南渡以後,不復再置。



 三司使使副使判官鹽鐵使度支使戶部使
 三部副使三部判官三司之職,國初沿五代之制,置使以總國計,應四方貢賦之入,朝廷之預,一歸三司。通管鹽鐵、度支、戶部,號曰計省,位亞執政,目為計相。其恩數廩祿,與參、樞同。太平興國八年,分置三使。淳化四年,復置使一員,總領三部。又分天下為十道:曰河南,河東,關西,劍南,淮南,江南東、西,兩浙,廣南。在京東曰左計,京西曰右計,置使二員分掌。俄又置總計使判左、右計事,左、右計使判十道事,凡
 干涉計度者,三使通議之。五年,罷十道左右計使,復置三部使。咸平六年,罷三部使,復置三司一員。關正使,則以給、諫以上權使事。



 使一人,以兩省五品以上及知制誥、雜學士、學士充。亦有輔臣罷政出外,召還充使者。使闕,則有權使事;又闕,則有權發遣公事。掌邦國財用之大計,總鹽鐵、度支、戶部之事,以經天下財賦而均其出入焉。凡奏事及大事悉置案,奏牒常事止署案。太平興國初,以賈琰為三司副使,七年,以侯陟、王明同判三司,遂省副使。鹽鐵,掌天下山澤之貨,關市、河渠、軍器之事,以
 資邦國之用。度支,掌天下財賦之數,每歲均其有無,制其出入,以計邦國之用。戶部,掌天下戶口、稅賦之籍,榷酒、工作、衣儲之事,以供邦國之用。



 副使以員外郎以上歷三路轉運及六路發運使充。



 判官以朝官以上曾歷諸路轉運使、提點刑獄充。



 三部副使各一人,通簽逐部之事。舊以員外郎以上充。端拱初,省。淳化三年復置,又省。至道初,又置。真宗即位,副使遷官,遂罷之。咸平六年復置。



 三部判官各三人,分掌逐案之事。



 舊以朝官充。國初承舊制,每部判官一人。乾德四年,三部各置推官一人。太平興國三年,諸案置推官或巡官,以朝
 官充。四年,三司止置判官一人、推官三人。及分十道,二計各置判官一人。五年,廢十道,三部各置判官二人。



 三部各有孔目官一人,都勾押官一人,勾覆官四人。



 鹽鐵分掌七案:一曰兵案,掌衙司軍將、大將、四排岸司兵卒之名籍,及庫務月帳,吉兇儀制,官吏宿直,諸州衙吏、胥吏之遷補,本司官吏功過,三部胥吏之名帳及刑獄,造船、捕盜、亡逃絕戶資產、禁錢。景德二年,並度支案為刑案。二曰冑案,掌修護河渠、給造軍器之名物,及軍器作坊、弓弩院諸務諸季料籍。



 三曰商稅案,四曰都鹽案,五曰茶案,六曰鐵案,掌金、銀、銅、鐵、朱砂、白礬、綠礬、石炭、錫、鼓鑄。



 七曰設案。掌旬設節料齋錢、餐錢、羊豕、米面、薪炭、陶器等物。



 度支分掌八案:



 一曰賞給案掌諸給賜、賻贈例物、口食、內外春冬衣、時服、綾、
 羅、紗、縠、綿、布、鞋、席、紙、染料,市舶、權物務、三府公吏。



 二曰錢帛案掌軍中春冬衣、百官奉祿、左藏錢帛、香藥榷易。



 三曰糧料案掌三軍糧料、諸州芻粟給受、諸軍校口食、御河漕運、商人飛錢。



 四曰常平案掌諸州平糴。大中祥符七年,置主吏七人。



 五曰發運案掌汴河廣濟蔡河漕運、橋梁、折斛,三稅。



 六曰騎案,掌諸坊監院務飼養牛羊、馬畜及市馬等。



 七曰觔斗案掌兩京倉廩BW積,計度東京糧料,百官祿;粟廚料。



 八曰百官案。掌京朝幕職官奉料、祠祭禮物、諸州驛料。



 戶部分掌五案:一曰戶稅案掌夏稅。



 二曰上供案掌諸州上供錢帛。



 三曰修造案掌京城工作及陶瓦八作、排岸作坊、諸庫簿張,勾校諸州營壘、官廨、橋梁、竹木、排筏。



 四曰曲案掌榷酤、官曲。



 五曰衣糧案掌勾校百官諸軍諸司奉料、春
 冬衣、祿粟、茶、鹽、□奚醬、傔糧等。三部諸案,並與本部都孔目官以下分掌。



 三部勾院判官各一人,以朝官充。掌勾稽天下所申三部金穀百物出納帳籍,以察其差殊而關防之。鹽鐵院、度支院、戶部院勾覆官各一人。



 都磨勘司,端拱九年置。判司官一人,以朝官充。掌覆勾三部帳籍,以驗出入之數。



 都主轄支收司,淳化三年置。



 判司官以判磨勘司官兼。掌官物已支未除之數,候至所受之處,附籍報所由司而對除之。天下上供物至京,即日奏之,納畢,取其鈔
 以還本州。



 拘收司,咸平四年置。



 以判磨勘司兼掌。凡支收財利未結絕者,籍其名件而督之。



 都理欠司,雍熙二年,三部各置理欠,有勾簿司,景德四年廢。



 判司官一人,以朝官充。掌理在京及天下欠負官物之籍,皆立限以促之。



 都憑由司,以判都理欠司官兼,掌在京官物支破之事。凡部支官物,皆覆視無虛謬,則印署而還之,支訖,復據數送勾而銷破之。



 開拆司,判司官一人,以朝官充。掌受宣敕及諸州申牒之籍,發放以付三部,兼掌發放、勾鑿、催驅、受事。



 發放司,掌受三司帖牒而下之。太平興國年中置。



 勾鑿司,掌勾校三部公事簿帳。



 催驅司,掌督京城諸司庫務末帳,京畿倉場庫務月帳憑由送勾,及三部支訖內外奉祿之事。



 受事司,掌諸處解送諸色名籍,以發付三部。



 衙司管轄官二人,以判開拆司官及內侍都知、押班充。掌大將、軍將名籍,第其勞而均其役使。



 勾當公事官二員,以朝官充。掌分左右廂檢計、定奪、點檢、覆驗、估剝之事。



 三司推勘公事一人,以京朝官充。掌推劾諸部
 公事。



 勾當諸司、馬步軍糧料院官各一人,以京朝官充。掌文武官諸司、諸軍給受奉料,批書券歷,諸倉庫案驗而稟賦之。



 勾當馬步軍專勾司官一人,以京朝官充。舊以三班。



 掌諸軍兵馬逃亡收並之籍,諸司庫務給受之數,審校其欺詐,批歷以送糧料院。



 以上並屬三司使。元豐官制行,罷三司使並歸戶部。



 翰林學士院翰林學士承旨翰林學士知制誥直學士院
 翰林權直學士院權直掌制誥、詔、令撰述之事。凡立后妃,封親王,拜宰相、樞密使、三公、三少,除開府儀同三司、節度使,加封,加檢校官,並用制;賜大臣太中大夫、觀察使以上,用批答及詔書;餘官用敕書;布大號令用御札;戒勵百官、曉諭軍民用敕榜;遣使勞問臣下,口宣。凡降大赦、曲赦、德音,則先進草;大詔命及外國書,則具本取旨,得畫亦如之。



 凡拜宰相及事重者,晚漏上,天子御內東門小殿,宣召面諭,給筆札書所得旨。稟奏歸院,
 內侍鎖院門,禁止出入。夜漏盡,具詞進入;遲明,白麻出,合門使引授中書,中書授舍人宣讀。其餘除授並御札,但用御寶封,遣內侍送學士院鎖門而已。至於赦書、德音,則中書遣吏持送本院,內侍鎖院如除授焉。凡撰述皆寫畫進入,請印署而出,中書省熟狀亦如之。若已畫旨而未盡及舛誤,則論奏貼正。凡宮禁所用文詞皆掌之。乘輿行幸,則侍從以備顧問,有獻納則請對,仍不隔班。凡奏事用榜子,關白三省、樞密院用諮報,不名。



 凡初
 命為學士,皆遣使就第宣詔旨召入院。上日,敕設會從官,宥以樂。元豐中,始命佩魚,自蒲宗孟始。見執政議事則系□奚,蓋與侍從異禮也。政和三年,強淵明請以前後所被旨及案例,修為本院敕令格式。五年,御書《摛文堂》榜賜學士院。靖康元年,吳幵等奏:「大禮鎖院,麻三道以上,系雙學士宿直分撰,乞依故事。」從之。



 承旨,不常置,以學士久次者為之。凡他官入院未除學士,謂之直院;學士俱闕,他官暫行院中文書,謂之權直。自國初至元
 豐官制行,百司事失其實,多所厘正,獨學士院承唐舊典不改。乾道九年,崔敦詩初以秘書省正字兼翰林權直。淳熙五年,敦詩再入院,議者以翰林乃應奉之所,非專掌制誥之地,更為學士院權直。後復稱翰林權直,然亦互除不廢,權、正或至三人。



 翰林侍讀學士太宗初,以著作佐郎呂文仲為侍讀。真宗咸平二年,以楊徽之、夏侯嶠並為翰林侍讀學士,始建學士之職。其後,馮元為翰林侍讀,不帶學士;又以
 高若訥為侍讀,不加別名,但供職而已。天禧三年,張知白為刑部侍郎,充翰林侍讀學士、知天雄軍府,侍讀學士外使自知白始。元豐官制,廢翰林侍讀、侍講學士不置,但以為兼官。然必侍從以上,乃得兼之,其秩卑資淺則為說書。歲春二月至端午日,秋八月至長至日,遇只日入侍邇英閣,輪官講讀。元祐七年,復增學士之號,元符元年省去。建炎元年,詔可特差侍從官四員充講讀官,遇萬機之暇,令三省取旨,就內殿講讀。



 充宮觀兼侍
 讀:元豐八年五月,資政殿大學士呂公著兼侍讀,提舉中太乙宮兼集禧觀公事。七月,韓維兼侍讀,提舉中太乙宮。元祐元年,端明殿學士範鎮致仕,提舉中太乙宮兼集禧觀公事,兼侍讀,不赴。六年,馮京兼侍讀,充太乙宮使。未幾,乞致仕,不允,仍免經筵進讀。中興以來,如朱勝非、張浚、謝克家、趙鼎、萬俟離並以萬壽觀使兼侍讀。隆興元年,張燾以萬壽觀、湯思退以醴泉觀並侍讀。乾道五年,劉章以祐神觀兼焉。



 臺諫兼侍讀:自慶歷以
 來,臺丞多兼侍讀,諫長未有兼者。紹興十二年春,萬俟離以中丞、羅汝楫以諫議始兼侍讀,自後每除言路,必兼經筵矣。



 翰林侍講學士咸平二年,國子祭酒邢昺為侍講學士。其後,又以馬宗元為侍講,不加別名,但供職而已。景德四年,以翰林侍講學士邢昺知曹州,侍講學士外使自昺始。故事,自兩省、臺端以上兼侍講,元祐中,司馬康以著作佐郎兼侍講,時朝議以文正公之賢,故特有是
 命。紹興五年,範沖以宗卿、朱震以秘少並兼,蓋殊命也。乾道六年,張栻始以吏部員外郎兼。蓋中興後,庶官兼侍講者,惟此三人。若紹興二十五年張扶以祭酒、隆興二年王佐以檢正、乾道七年林憲以宗卿入經筵,亦兼侍講者。蓋扶本以言路兼說書就升其秩,佐時攝版曹,憲嘗為右史且有舊例,故稍優之。



 臺諫兼侍講:慶歷二年,召御史中丞賈昌朝侍講邇英閣。故事,臺丞無在經筵者,仁宗以昌朝長於講說,特召之。神宗用呂正獻,
 亦止命時赴講筵去學士職。中興後,王賓為御史中丞,見請復開經筵,遂命兼講。自後十五年間,繼之者惟王唐、徐俯二人,皆出上意。紹興十二年,則萬俟離、羅汝楫,紹興二十五年,則正言王鈱、殿中侍御史董德元,並兼侍講。非臺丞、諫長而以侍講為稱,又自此始。其後,猶或兼說書,臺官自尹穡,隆興二年五月;諫官自詹元宗,乾道九年十二月。後並以侍講為稱,不復兼說書矣。



 宮觀兼侍講:國初自元豐以來,多以宮觀兼侍讀。乾道七
 年,寶文待制胡銓除提舉祐神觀兼侍講。是日,以宰執進呈,虞允文奏曰:「胡銓早歲士節甚高,不宜令其遽去朝廷。」帝曰:「銓固非他人比,且除在京宮觀,留侍經筵。」故有是命。



 崇政殿說書掌進讀書史,講釋經義,備顧問應對。學士侍從有學術者為侍講、侍讀,其秩卑資淺而可備講說者則為說書。仁宗景祐元年正月,命賈昌朝、趙希言、王宗道、楊安國並為崇政殿說書,日輪二員祗候。初,侍
 講學士孫奭年老乞外,因薦昌朝等。至是,特置此職以命之。慶歷二年,以趙師民預講官,復為崇政殿說書,不兼侍講。元祐間,程頤以布衣為之。然範祖禹乃以著作佐郎兼侍講,司馬康又嘗以著作佐郎兼侍講,前此未有也。崇寧中,初除說書二人,皆以隱逸起,蔡崇、呂瓘,仍遂其性,詔以士服隨班朝謁入侍。



 渡江後,尹焞初以秘書兼之,中間王十朋、範成大皆以郎官兼,亦殊命也。近事,侍從以上兼經筵則曰侍講,庶官則曰崇政殿說書,
 故左史兼亦曰侍講。紹興十二年,萬俟離、羅汝楫並兼講讀。蓋秦梓時已兼說書,便於傳道,秦熹復繼之。每除言路,必預經筵,檜死始罷。慶元後,臺丞、諫長暨副端、正言、司諫以上,無不預經筵者。正言兼說書自端明巫伋始,副端兼說書自端明餘堯弼始,察官兼說書自少卿陳夔始,修注兼說書自朱學震始。修注官多得兼侍講。開禧三年十一月,王簡卿知諫院為左史,仍兼崇政殿說書。言者以為不可,罷之。



 觀文殿大學士學士之職,資望極峻,無吏守,無職掌,惟出入侍從備顧問而已。觀文殿即舊延恩殿,慶歷七年更名。皇祐元年,詔:「置觀文殿大學士,寵待舊相,今後須曾任宰相,乃得除授。」時賈昌朝由使相右僕射、觀文殿大學士判尚書都省。觀文殿置大學士,自昌朝始。三年,詔班在觀文殿學士之前六尚書之上。自是曾任宰相者,出必為大學士。熙寧中,韓絳宣撫陜西、河東,得罪罷守本官。四年,用明堂赦,授觀文殿學士。宰相不為大
 學士,自絳始。中興後,非宰相而除者,自紹興二十年秦熹始。熹知樞密院、郊祀大禮使,禮成,以學士遷,且視儀揆路,非典故也。乾道四年,汪澈舊以樞密使為學士遷。九年,王炎以樞密使為西川安撫使除。至慶元間,趙彥逾自工部尚書為端明殿學士,直以序遷至焉。曾為宰相而不為大學士者,自紹興元年範宗尹始。



 觀文殿學士觀文殿本隋煬帝殿名,國初,為文明殿學士。慶歷七年,宋庠言:「文明殿學士稱呼正同真宗謚
 號,兼禁中無此殿額,其學士理自當罷,乞擇見今正朝或秘殿以名學士易之。」乃詔改為紫宸殿學士,以參知政事丁度為之。時學士多以殿名為官稱,丁遂稱曰「丁紫宸」。八年,御史何郯以為紫宸不可為官稱,於是改延恩殿為觀文殿,即殿名置學士,仍以度為之。自後非曾任執政者弗除。熙寧中,王韶以熙河功,元豐中,王陶以宮僚,雖未歷二府,亦除是職,蓋異恩也。然韶猶兼端明殿、龍圖學士云。



 資政殿大學士資政殿在龍圖閣之東序。景德二年,王欽若罷參政,真宗特置資政殿學士以龐之,在翰林學士下。十二月,復以欽若為資政殿大學士,班文明殿學士之下,翰林學士承旨之上。資政殿置大學士,自欽若始。自欽若班翰林承旨上,一時以為殊寵。祥符初,向敏中以前宰相再入為東京留守,復加此職。自是訖天聖末,二十餘年不以除人。明道元年,李迪知河陽召還,始再命之。景祐四年,王曾罷相,復除。三十年間除三人,
 皆前宰相也。宋庠罷參知政事,仁宗眷之厚,因加此職。自欽若後,非宰相而除者,惟庠一人。康定二年,右正言梁適請遵先朝故事,定以員數。於是詔大學士置二員,學士三員。紹興十年,鄭億年歸自偽齊,除資政殿,二年加大學士,許出入如二府儀。億年未嘗秉政。十五年,秦熹自翰林學士承旨為資政,詔立班恩數同執政。十六年,秦檜弟梓以端明卒於湖州,進大資致仕,恤典同參政。是後,從臣自端明視政府而序進者,遂為常矣。



 端明殿學士端明殿即西京正衙殿也。後唐天成元年,明宗即位之初,四方書奏,命樞密使安重誨進讀,懵於文義。孔循獻議,始置端明殿學士,命馮道、趙鳳俱以翰林學士充,班在翰林學士上。後有轉改,止於翰林學士內選任。初如三館例,職在官下;趙鳳轉侍郎,諷任圜特移職在官上,後遂為故事。宋太宗初,以程羽為之,後隨殿名改為文明殿學士。慶歷中,改為紫宸,後又改為觀文。明道二年,改承明殿為端明殿,復置端明殿學士,
 以翰林侍讀學士宋綬為之,在翰林學士之下。自明道訖元豐,無前執政為之者,僅以待學士之久次者。元豐中,以前執政為之,自曾孝寬始;以見任執政為之,自王安禮始。政和中,嘗改為延康殿。建炎二年,都省言:延康殿學士舊系端明殿學士。詔依舊。後拜簽樞者多領焉。



 總閣學士直學士宋朝庶官之外,別加職名,所以厲行義、文學之士。高以備顧問,其次與論議、典校讎。得之為榮,選擇尤精。元豐中,修三省、寺監之制,其職並罷,滿
 歲補外,然後加恩兼職。直龍圖閣、省、寺監掌貳補外,或領監司、帥臣則除之;待制、雜學士、給諫以上補外則除之。系一時恩旨,非有必得之理。元祐二年,詔復增館職及職事官並許帶職,尚待二年加直學士,中丞、侍郎、給舍、諫議通及一年加待制。紹聖三年,詔職事官罷帶職,非職事之官仍舊。中興後,學士率以授中司、列曹尚書、翰林學士之輔外者,權尚書、給諫、侍郎則帶直學士、待制焉。



 龍圖閣學士直學士待制大中祥符中建。在會慶殿西偏,北連禁中,閣東曰資政殿,西曰述古殿。閣上以奉太宗御書、御制文集及典籍、圖畫、寶瑞之物,及宗正寺所進屬籍、世譜。有學士、直學士、待制、直閣等官。學士,大中祥符三年置,以杜鎬為之,班在樞密直學士上。六年,詔結銜在本官之上。直學士,景德四年置,以杜鎬為之,班在樞密直學士下。祥符六年,詔結銜在本官之上。待制,景德元年置,以杜鎬、戚綸為之,並依舊
 充職。四年,詔班在知制誥下,並赴內殿起居。自改官制,為學士初復之職,或知制誥平出除之。


天章閣學士直學士待制天禧四年建。在會慶殿之西,龍圖閣之北。明年,仁宗即位,修天章閣畢,以奉安真宗御制。東曰群玉殿,西曰
 \gezhu{
  蕊木}
 珠殿,北曰壽昌殿,南曰延康殿。內以桃花文石為流桮之所。以在位受天書祥符,改曰天章,取為章於天之義。天聖八年置待制。慶歷七年,又置學士、直學士。又有侍講。學士,慶歷七年初
 置,在龍圖閣學士之下。學士罕以命人,迄仁宗世,才王贄一人。秦堪自顯謨閣進直天章閣,以稱呼非便辭。詔改龍圖,自是天章不為帶職。直學士,慶歷七年,初置天章閣直學士,在龍圖閣直學士之下。待制,天聖八年初置。寓直於秘閣,與龍圖遞宿,尋命範諷鞠詠充職。中興後,圖籍、符瑞、寶玩之物,若國史、宗正寺所進屬籍,獨藏於天章閣,祖宗御容、潛邸旌節亦安奉焉。



 寶文閣學士直學士待制閣在天章閣之東西
 序,群玉、蕊珠殿之北。即舊壽昌閣,慶歷改曰寶文。嘉祐八年,英宗即位,詔以仁宗御書、御集藏於閣,命王珪撰記立石。治平四年,神宗即位,始置學士、直學士、待制,恩賜如龍圖。英宗御書附於閣。學士,治平四年初置,以呂公著兼。直學士,治平四年初置,以邵必為之。待制,治平四年初置。



 顯謨閣學士直學士待制元符元年,曾布、鄧洵仁各申請建閣。詔翰林學士、中書舍人撰閣名五以聞,
 遂建閣藏神宗御集,以顯謨為名。徽宗建中靖國元年,詔以顯謨閣為熙明閣,仍置學士、直學士、待制;續奉旨,仍以顯謨為額。崇寧元年,詔顯謨閣學士、直學士、待制如三閣故事,序位在寶文閣學士、直學士、待制之下。學士、直學士、待制,並建中靖國元年置。



 徽猷閣學士直學士待制大觀二年,初建徽猷閣,以藏哲宗御集。置學士、直學士、待制等官。



 敷文閣學士直學士待制紹興十年置。藏徽宗
 聖制,置學士等官。



 煥章閣學士直學士待制淳熙初建。藏高宗御制。十五年,置學士等官。



 華文閣學士直學士待制慶元二年置。藏孝宗御制,置學士等官。



 寶謨閣學士直學士待制嘉泰二年置。藏光宗御制,置學士等官。



 寶章閣學士直學士待制寶慶二年置。藏寧
 宗御制,置學士等官。



 顯文閣學士直學士待制咸淳元年置。藏理宗御制,置學士等官。



 集英殿修撰國初,有集賢殿修撰、直龍圖閣、直秘閣三等。政和六年,始置集英殿修撰、右文殿修撰、秘閣修撰。舊制,貼職無雜壓,至是因增置,乃定為雜壓。其集英修撰,中興後以寵六曹權侍郎之補外者,下待制一等。



 右文殿修撰元祐元年,許內外官帶貼職。紹聖二年,
 詔職事官罷帶職,易集賢殿學士為修撰。政和六年,以集賢院無此名,其見任集賢院修撰並改為右文殿修撰,次於集英殿修撰,為貼職之高等。



 秘閣修撰政和六年置,以待館閣之資深者,仍多由直龍圖閣遷焉。



 直龍圖閣祥符九年,以馮元為太子中允、直龍圖閣,直閣之名始此。凡館閣之久次者,必選直龍圖閣,皆為擢待制之基也。中興後,凡直閣為庶官任藩閫、監司者
 貼職,各隨高下而等差之。



 直天章閣至直顯文閣,並同。



 直秘閣國初,以史館、昭文館、集賢院為三館,皆寓崇文院。太宗端拱元年,詔就崇文院中堂建秘閣,擇三館真本書籍萬餘卷及內出古畫、墨跡藏其中,以右司諫直史館宋泌為直秘閣。直館、直院則謂之館職,以他官兼者謂之貼職。元豐以前,凡狀元、制科一任還,即試詩賦各一而入,否則用大臣薦而試,謂之入館。官制行,廢
 崇文院為秘書監,建秘閣於中,自監少至正字列為職事官。罷直館、直院之名,獨以直秘閣為貼職,皆不試而除,蓋特以為恩數而已。故事,外官除館職如秘閣校理、直秘閣者,必先移書在省執事,敘同僚之好,乃即館設盛會宴之。自崇寧以來,外官除館職既多,此禮浸廢。



 東宮官太子太師太傅太保太子少師少傅少保國初,師傅不常設。仁宗升儲,置三少各一人。參政李
 昉兼掌賓客。及升首相,遂進少傅,此宰相兼宮僚之始也。丁謂兼少師,馮拯兼少傅,曹利用兼少保,是時實為東宮官,餘多以前宰執為致仕官。若太子太師、太傅、太保,以待宰相官未至僕射者,及樞密使致仕,亦隨本官高下除授。太子少師、少傅、少保,以待前執政,惟少師非經顧命不除。若因遷轉,則遞進一官,至太師即遷司空。天禧末,皇太子同聽政,乃以首相兼少師。自後神宗、欽宗、孝宗、光宗在東宮,皆不置。開禧三年,史彌遠自詹事
 入樞府,乃進兼賓客。已而太子侍立,遂以丞相錢象祖兼太子少傅。明年,景獻太子立,象祖兼少師,彌遠以右相兼少傅。未幾,彌遠丁內艱,像祖亦去位。又明年,彌遠起復,遂兼進少師。景定元年,度宗升儲,以賈似道為少師。



 太子賓客至道元年建儲,初置賓客二人,以他官兼。天禧四年,參政任中正、樞副錢惟演、參政王曾並兼太子賓客,執政兼東宮官始此。中興後不置。開禧三年,景
 獻太子立,始以執政兼賓客,後復省。景定元年,度宗升儲,以朱熠、皮龍榮、沉炎並兼賓客。



 太子詹事仁宗升儲,置詹事二人。神宗、欽宗升儲,並置二人,皆以他官兼,登位後省。乾道元年,莊文太子立,置詹事二人。逾月,詔太子詹事遇東宮講讀日,並往陪侍。七年,光宗正儲位,以敷文閣直學士王十朋、敷文閣待制陳良翰為太子詹事,不兼他官,非常制也。景定元年,度宗升儲,以楊棟兼詹事。



 太子左庶子右庶子左諭德右諭德舊制不常設。儲闈之建,隨宜制官,以備僚寀,多以他官兼領。仁宗、神宗升儲,庶子、諭德各置二人。欽宗升儲,置一人。紹興三十二年,孝宗以建王立為皇太子,置庶子、諭德各一人,除右虛左。乾道元年及七年,各置一人。開禧三年,景獻太子立,初除左虛右,明年,左右始並置。



 太子侍讀侍講神宗升儲,始置各一人。乾道、淳熙、開禧,各依故事並置。乾道七年,禮部太常寺言:「討論東
 宮開講並節朔賀慶、辭謝禮儀。宮僚講讀,無已行故事,當依放講筵,少殺其禮。每遇講讀,詹事以下至進讀官上堂,並用賓禮參見,依官職序坐。皇太子正席,講讀官迭起如延英儀,講罷復位。節朔不受宮僚參賀;元日、冬至,詹事以下箋賀。謝辭,初如常見之禮。後離位致詞,復位就坐,茶湯罷。詹事初上,參見皇太子,拜,皇太子答拜。庶子等初上,參見,皇太子受拜。庶子、諭德及講讀官雖有坐受之禮,止是《五禮新儀》所載;兼逐日致拜之禮,近
 例皆已不行,或遇合致拜日,更合參酌天禧、至道故事施行。」



 按天禧二年九月五日,左庶子張士遜等言:「臣等日詣資善堂參見皇太子,得令升階列拜,然後跪受,望令皇太子坐受參見。」詔不許。至道元年,皇太子每見太子賓客,必先拜,迎送常降階及門。



 並從之。



 太子中舍人舍人至道、天禧各置一人。神宗、欽宗升儲,並如舊置。嘉定初,除二人。慶元以中舍人在舍人上。



 資善堂翊善贊讀直講說書皇太子宮小學教授資善堂小學教授翊善、贊讀、直講皆舊制。
 說書而下,中興以後增置。資善堂自仁宗為皇子時,為肄業之所,每皇子出就外傅,選官兼領。元豐八年,哲宗初開講筵,詔講讀官日赴資善堂,以雙日講讀,仍輪一員宿直。又詔三省、樞密院、講讀、修注官錫宴於資善堂。政和元年,定王、嘉王出就資善堂聽讀,詔宰執就見。靖康元年,詔皇太子出就外傅,就資善堂置學舍,令國子監供監書。紹興五年,孝宗封建國公,出就資善堂聽講。先是,宰臣趙鼎得旨於宮門內造書院,至是始成,以
 為資善堂。命儒臣為直講、翊善,悉如資善故事。尋用趙鼎言,以左史範沖充翊善,右史朱震充贊讀,時稱極選。帝曰:「朕令國公見沖、震必設拜,蓋尊重師傅,不得不如此。」紹興十二年,建國公出就外第。及紹興三十年,由普安郡王為皇子,進封建王。時皇孫皆就傅,以校書郎王十朋為小學教授。紹興三十二年,孝宗即位,詔三皇子位各置說書官一員,又置贊讀、直講一員。淳熙七年,皇孫英國公始就傅,詔置皇太子宮小學教授一員。十六
 年,光宗即位,皇子進封嘉王,置王府贊讀、翊善、直講各一員。慶元六年,景獻太子為福州觀察使,詔令資善堂授書,置小學教授二員。開禧元年,進封榮王,仍開資善堂,置贊讀、直講、說書官各一員,又置翊善一員。度宗升儲,並置翊善、贊讀等官。



 主管左、右春坊事二人,以內臣兼;同主管左、右春坊事二人,以武臣兼;承受官一人,以內侍充。仁宗、神守升儲,並置。中興後,置官並同。



 太子左、右衛率府率副率左、右司禦率府率副率左、右清道率府率副率左右監門率府率副率左、右內率府率副率官存而無職司。至道元年,東宮置左清道率府率、副率兼左春坊謁者,主贊引。三年,真宗即位而省。天禧二年,又以左清道率郭承慶、左右監門副率夏元亨兼左右春坊謁者,仁宗即位復省。中興後不置,惟以監門率府副率為壞衛階官。



 親王府傅長史司馬諮議參軍友記室參軍王府教授小學教授傅及長史、司馬,有其官而未嘗除。太平興國八年,諸王出閣,楚王府置諮議參軍二員,翊善一員;陳王府置諮議、翊善各一員;韓王、冀王、益王置翊善各一員。後又置記室及諸王府侍講一員。並以常參官兼充。



 其後,多不置諮議,翊善、記室或止一員。



 大中祥符九年,仁宗初封壽春郡王,置友二員,亦以常參官兼充。天禧二年,進封升王,友遷諮議,仍置記室一員。又皇侄
 皇孫侍教、南北伴讀無定數。



 至道初,太宗以皇親子孫就講學,欲置侍講之職,中書言:「按唐太宗改諸王侍讀為奉諸王講讀,今皇孫、皇侄皆環衛之職,請以教授為名。」從之。選京朝官通經者充。其後又令王府記室、翊善、侍講分兼南北宅教授。大中祥符二年,又有侍教之名,自是南北院或有伴讀。



 凡諸宮皆有教授,初無定員。是年,英宗以宗室自率府副率已上八百餘人,奉朝請者四百餘人,而教官才六員,乃詔增置教授官:凡皇族年三十已上者百一十三人,置講書四員;年二十以上者百十三人,置講書四員;年十五已上者三百九人,增置教授五員;年十四已下
 者,別置小學教授十二員;並舊六,為二十七員,以分教之。其子弟不率教,俾教授官、本位尊長具名申大宗正司,量行戒責。教授官不職,大宗正司密訪以聞。舊制,親賢宅置講書,紹興十二年,改為府教授,掌教親賢宅南班宗子。淳熙十二年,詔建魏惠憲王府,置小學教授二員,以館職兼充,掌訓皇孫。既長,趨朝謁,則不以小學名,而講習如故。自後皇侄、皇孫皆置教授。



\end{pinyinscope}