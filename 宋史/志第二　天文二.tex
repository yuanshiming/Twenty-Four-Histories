\article{志第二 天文二}

\begin{pinyinscope}

 紫微垣太微垣天市垣



 紫微垣



 紫微垣東蕃八星,西蕃七星,在北斗北,左右環列,翊衛之象也。一曰大帝之坐,天子之常居也,主命、主度也。東蕃
 近閶闔門第一星為左樞,第二星為上宰,三星曰少宰,四星曰上弼一曰上輔,五星為少弼一曰少輔,六星為上衛,七星為少衛,八星為少丞或曰上丞。其西蕃近閶闔門第一星為右樞,第二星為少尉,第三星為上輔,
 第四星為少輔,第五星為上衛,第六星為少衛,第七星為上丞。其占,欲均明,大小有常,則內輔盛;垣直,天子自將出征;門開,兵起宮垣。兩蕃正南開如門,曰閶闔。有流星自門出四野者,當有中使御命,視其所往分野論之;不依門出入者,
 外蕃國使也。太陰、歲星犯紫微垣,有喪。太白、辰星犯之,改世。熒惑守宮,君失位。客星守,有不臣,國易政。國皇星,兵。彗星犯,有異王立。流星犯之,為兵、喪,水旱不調。使星入北方,兵起石氏云:東西兩蕃總十六星,西蕃亦八星,一右樞,二上尉,三少尉,四上輔,五少輔,六上衛,七少衛,八少丞。上宰一星,上輔二星,三公也。少宰一星,少輔二星,三孤也。此三公、三孤在朝者也。左右樞、上少丞,疑丞輔弼,四鄰之謂也。尉二星,衛四星,六軍大副尉,四衛將軍也。



 北極五星,在紫微宮中,北辰最尊者也,其紐星為天樞,天運無窮,三光迭耀,而極星不移,故曰「居其所而眾星
 共之」。樞星在天心,四方去極各九十一度。賈逵、張衡、蔡邕、王蕃、陸績皆以北極紐星之樞,是不動處。在紐星末猶一度有餘。今清臺則去極四度半。第一星主月,太子也;二星主日,帝王也,亦太一之坐,謂最赤明者也;第三星主五行,庶子也。《乾象新星書》曰:「第三星主五行,第四星主諸王,第五星為後宮。」閎云:「北極五星,初一曰帝,次二曰後,次三曰妃,次四曰太子,次五曰庶子。」四曰太子者,最赤明者也。後四星勾曲以抱之者,帝星也。太公望以為北辰,以為耀魄寶,以為帝極者是也。或以勾陳口中一星為耀魄寶者,非是。北極中星不明,主不用事;右星不明,太子憂;左星不明,庶子
 憂;明大動搖,主好出游;色青微者,兇。客星入,為兵、喪。彗星入,為易位。流星入,兵起地動。



 北斗七星在太微北,杓攜龍角,衡殷南斗,魁枕參首,是為帝車,運於中央,臨制四海,以建四時、均五行、移節度、定諸紀,乃七政之樞機,陰陽之元本也。魁第一星曰天樞,正星,主天。又曰樞為天,主陽德,天子象。其分為秦,《漢志》主徐州。《天象占》曰:「天子不恭宗廟,不敬鬼神,則不明,變色。」二曰璇,法星,主地。又曰璇為地,主陰刑,女主象。
 其分為楚,《漢志》主益州。《天象占》曰:「若廣營宮室,妄鑿山陵,則不明,變色。」三曰璣,為人,主火,為令星,主中禍。其分為梁,《漢志》主冀州。若王者不恤民,驟徵役,則不明,變色。四曰權,為時,主水,為伐星,主天理,伐無道。其分為吳,《漢志》主荊州。若號令不順四時,則不明,變色。五曰玉衡,為音,主土,為殺星,主中央,助四方,殺有罪。其分為燕,《漢志》主兗州。若廢正樂,務淫聲,則不明,變色。六曰闓陽,為律,主木,為危星,主天倉、五穀。其分為趙,《漢志》主揚州。若
 不勸農桑,峻刑法,退賢能,則不明,變色。七曰搖光,為星,主金,為部星,為應星,主兵。其分為齊,《漢志》主豫州。王者聚金寶,不修德,則不明,變色。又曰一至四為魁,魁為璇璣;五至七為杓,杓為玉衡:是為七政,星明其國昌。第八曰弼星,在第七星右,不見,《漢志》主幽州。第九曰輔星,在第六星左,常見,《漢志》主並州。《晉志》,輔星傅乎闓陽,所以佐鬥成功,丞相之象也。其色在春青黃,在夏赤黃,秋為白黃,冬為黑黃。變常則國有兵殃,明則臣強。斗旁欲多
 星則安,斗中星少則人恐。太陰犯之,為兵、喪、大赦。白暈貫三星,王者惡之。星孛於北斗,主危。彗星犯,為易主。流星犯。主客兵。客星犯,為兵。五星犯之,國亂易主。



 按:北斗與輔星為八,而《漢志》云九星,武密及楊維德皆採用之。《史記索隱》云:「北斗星間相去各九千里。其二陰星不見者,相去八千里。」而丹元子《步天歌》亦云九星,《漢書》必有所本矣。



 勾陳六星,在紫宮中,五帝之後宮也,太帝之正妃也,大
 帝之帝居也。《樂緯》曰:「主後宮。」巫咸曰:「主天子護軍。」《荊州占》:「主大司馬。」或曰主六軍將軍。或曰主三公、三師,為萬物之母。六星比陳,像六宮之化,其端大星曰元始,餘星乘之曰庶妾,在北極配六輔甘氏曰:勾陳在辰極左,是為鉤陳衛六軍將軍。或以為後宮,非是。勾陳口中一星為陽德,天皇大帝內坐。或即以為天皇大帝,非是。其占,色不欲甚明,明即女主惡之。星盛,則輔強;主不用諫,佞人在側,則不見。客星入之,色蒼白,將有憂;白,為立將;赤黑,將死。客星出而色赤,戰有功;守之,後宮有女使欲謀。彗星犯之,
 後宮有謀,近臣憂。流星入,為迫主。青氣入,大將憂。



 天皇大帝一星,在勾陳口中,其神曰耀魄寶,主御群靈,執萬神圖,大人之象也。客星犯之,為除舊布新。彗、孛犯,大臣叛。流星犯,國有憂。雲氣入之,潤澤,吉。黃白氣入,連大帝坐,臣獻美女;出天皇上者,改立王。



 四輔四星,又名四弼,在極星側,是曰帝之四鄰,所以輔佐北極,而出度授政也。去極星各四度。閎云:「四輔一名中斗。」或以為後宮,非是。武密曰:「光浮而動,兇;明小,吉;暗,則不理。」客星犯之,
 大臣憂。彗、孛犯,權臣死。流星犯,大臣黜。黃、白氣入,四輔有喜。白氣入,相失位。



 五帝內坐五星,在華蓋下,設敘順,帝所居也。色正,吉;變色,為災。客星犯紫宮中坐,占為大臣犯主。彗、孛犯之,民饑,大臣憂,三年有兵起。流星犯,為兵起、臣叛;出,為有誅戮。雲氣入,色黃,太子即位,期六十日,赤黃,人君有異。



 六甲六星,在華蓋杠旁,主分陰陽,配節候,故在帝旁,所以布政教、授農時也。明,則陰陽和;不明,則寒暑易節;星
 亡,水旱不時。客星犯之,色赤,為旱;黑,為水;白,則人多疫。彗、孛犯,女主出政令。流星犯,為水旱,術士誅。雲氣犯,色黃,術士興。蒼白,史官受爵。



 柱史一星,在北極東,主記過,左右史之象。一云在天柱前,司上帝之言動。星明,為史官得人;不明,反是。客星犯之,史官有黜者。彗、孛犯,太子憂,或百官黜。流星犯,君有咎。雲氣犯,色黃,史有爵祿。蒼白氣入,左右史死。



 女史一星,在柱史北,婦人之微者,主傳漏。



 天柱五星,在東垣下,一云在五帝左稍前,主建政教。一曰法五行,主晦朔、晝夜之職。明正,則吉,人安,陰陽調;不然,則司歷過。客星犯之,國中有賊。彗、孛犯,宗廟不安,君憂,一曰三公當之。雲氣赤黃,君喜;黑,三公死。



 女御四星,在大帝北,一云在勾陳腹,一云在帝坐東北,御妻之象也。星明,多內寵。客星犯之,後宮有誅,一云自戮。孛、彗犯,後宮有誅。流星犯,後宮有出者。一雲外國進美女。雲氣化黃,為後宮有子,喜;蒼白,多病。



 尚書五星,在紫微東蕃內,大理東北,《晉志》在東南維,一云在天柱右稍前,主納言,夙夜咨謀,龍作納言之象。彗、孛犯之,官有叛,或太子憂。流星若出,則尚書出使;犯之,諫官黜,八坐憂。雲氣入,黃,為喜;黃而赤,尚書出鎮;黑,尚書有坐罪者。



 大理二星,在宮門左,一云在尚書前,主平刑斷獄。明,則刑憲平;不明,則獄有冤酷。客星犯之,貴臣下獄;色黃,赦;白,受戮;赤黃,無罪;守之,則刑獄冤滯,或刑官有黜。彗犯,
 獄官憂;流星,占同。雲氣入,黃白,為赦;黑,法官黜。



 陰德二星,巫咸圖有之,在尚書西,甘氏云:「陰德外坐在尚書右,陽德外坐在陰德右,太陰太陽入垣翊衛也。」《天官書》則以「前列直斗口三星,隨北端銳,若見若不見,曰陰德。」謂施德不欲人知也。主周急振撫。明,則立太子,或女主治天下。客星犯之,為旱,饑;守之,發粟振給。彗、孛犯,後宮有逆謀。流星犯,君令不行。雲氣入,黃,為喜;青黑,為憂。



 天床六星,在紫微垣南門外,主寢舍解息燕休。一曰在二樞之間,備幸之所也。陶隱居云:「傾則天王失位。」客星入宮中,有刺客,或內侍憂。彗、孛犯之,主憂,大臣失位。流星犯,後妃叛,女主立,或人君易位。雲氣入,色黃,天子得美女,後宮喜有子;蒼白,主不安,青黑,憂;白,兇。



 華蓋七星,杠九星,如蓋有柄下垂,以覆大帝之坐也,在紫微宮臨勾陳之上。正,吉;傾,則兇。客星犯之,王室有憂,兵起。彗、孛犯,兵起,國易政。流星犯,兵起宮內,以赦解之;
 貫華蓋,三公災。雲氣入,黃白,主喜;赤黃,侯王喜。



 傳舍九星,在華蓋上,近河,賓客之館,主北使入中國。客星犯,邦有憂;一曰客星守之,備奸使;亦曰北地兵起。彗、孛犯,守之,亦為北兵。黑雲氣入,北兵侵中國。



 八穀八星,在華蓋西、五車北,一曰在諸王西。武密曰:「主候歲豐儉,一稻、二黍、三大麥、四小麥、五大豆、六小豆、七粟、八麻。」甘氏曰:「八穀在宮北門之右,司親耕,司候歲,司尚食。」星明,吉;一星亡,一穀不登;八星不見,大饑。客星入,
 穀貴。彗星入,為水。黑雲氣犯之,八穀不收。



 內階六星,在文昌東北,天皇之階也。一曰上帝幸文館之內階也。明,吉;傾動,憂。彗、孛、客、流星犯之,人君遜避之象。



 文昌六星,在北斗魁前、紫微垣西,天之六府也,主集計天道。一曰上將、大將軍,建威武;二曰次將、尚書,正左右;三曰貴相、大常,理文緒;四曰司祿、司中、司隸,賞功進;五曰司命、司怪、太史,主滅咎;六曰司寇、大理,佐理寶。所謂
 一者,起北斗魁前近內階者也。明潤色黃,大小齊,天瑞臻,四海安;青黑微細,則多所殘害;動搖,三公黜。月暈其宿,大赦。歲星守之,兵起。熒惑守之,將兇。太白守、入,兵興。填星守,國安。客星守,大臣叛。彗、孛犯,大亂。流星犯,宮內亂。



 三公三星,在北斗杓南及魁第一星西,一云在斗柄東,為太尉、司徒、司空之象。在魁西者名三師,占與三公同,皆主宣德化、調七政、和陰陽之官也。移徙,不吉;居常,則
 安;一星亡,天下危;二星亡,天下亂;三星不見,天下不治。客星犯,三公憂。彗、孛及流星犯之,三公死。



 天牢六星,在北斗魁下,貴人之牢也,主繩愆禁暴。甘氏云:「賤人之牢也。」月暈入,多盜。熒惑犯之,民相食,國有敗兵。太白、歲星守,國多犯法。客星、彗星犯之,三公下獄,或將相憂。流星犯之,有赦宥之令。



 勢四星,在太陽守西北,一曰在璣星北。勢,腐形人也,主助宣王命,內常侍官也。以不明為吉,明則閹人擅權。



 天理四星,在北斗魁中,貴人之牢也。星不欲明,其中有星則貴人下獄。客星犯,多獄。彗、孛犯之,國危。赤雲氣犯之,兵大起,將相行兵。



 相一星,在北斗第四星南,總百司,集眾事,掌邦典,以佐帝王。一曰在中鬥文昌之南,在朝少師行大宰者。明,吉;暗,兇;亡,則相黜。



 太陽守一星,在相星西北、斗第三星西南,大將、大臣之象,主設武備以戒不虞。一曰在下臺北,太尉官也,在朝
 少傅行大司馬者。明,吉;暗,兇。客、彗、孛犯之,為易政,將相憂,兵亂。雲氣入,黃,為喜;蒼白,將死;赤,大臣憂。



 內廚二星,在紫微垣西南外,主六宮之內飲食及后妃夫人與太子燕飲。彗、孛或流星犯之,飲食有毒。



 天廚六星,在扶筐北,一曰在東北維外,主盛饌,今光祿廚也。星亡,則饑;不見,為兇。客星、流星犯之,亦為饑。



 天一一星,在紫微宮門右星南,天帝之神也,主戰鬥,知吉兇。明,則陰陽和,萬物盛,人君吉;亡,則天下亂。客星犯,
 五穀貴。彗、孛犯之,臣叛。流星犯,兵起,民流。雲氣犯,黃,君臣和;黑,宰相黜。



 太一一星,在天一南相近一度,亦天帝神也,主使十六神,知風雨、水旱、兵革、饑饉、疾疫、災害所在之國也。明,吉;暗,兇;離位,有水旱。客星犯,兵起,民流,火災,水旱,饑饉。彗、孛犯,兵、喪。流星犯,宰相、史官黜。雲氣犯,黃白,百官受賜;赤為旱、兵;蒼白,民多疫。



 天槍三星,在北斗杓東。一曰天鉞,天之武備也,故在紫
 微宮左右,所以禦難也。明,吉;暗、小,兵敗;芒角動,兵起。客星、彗星、流星犯,皆為兵、饑。



 天棓五星,在女床北,天子先驅也,主分爭與刑罰藏兵,亦所以禦難,備非常也。一星不具,其國兵起;明,有憂;細微,吉。客星入,兵、喪。彗星守,兵起。流星犯,諸侯多爭。雲氣犯,蒼白、黑,為兇。



 天戈一星,又名玄戈,在招搖北,主北方。芒角動搖,則北兵起。客星守之,北兵敗。彗、孛、流星犯之,占同。雲氣犯,黑,
 為北兵退;蒼白,北人病。



 太尊一星,在中臺北,貴戚也。不見,為憂。客、彗、流星犯之,並為貴戚將敗之徵。



 按《步天歌》載,中宮紫微垣經星常宿可名者三十五坐,積數一百六十有四。而《晉志》所載太尊、天戈、天槍、天棓皆屬太微垣,八穀八星在天市垣,與《步天歌》不同。



 太微垣



 太微垣十星,《漢志》曰:「南宮朱鳥,權、衡。」《晉志》曰:「天子庭也,五帝之坐也,十二諸侯之府也。其外蕃,九卿也。一曰太微為衡,衡主平也;又為天庭,理法平辭,監升授德,列宿受符,諸神考節,舒情稽疑也。南蕃中二星間曰端門。東曰左執法,廷尉之象。西曰右執法,御史大夫之象。執法所以舉刺兇邪。左執法東,左掖門也。右執法西,右掖門也。東蕃四星:南第一曰上相,其北,東太陽門也;第二曰次相,其北,中華東門也;第三曰次將,其北,東太陰門也;第
 四曰上將,所謂四輔也。西蕃四星:南第一曰上將,其北,西太陽門也;第二曰次將,其北,中華西門也;第三曰次相,其北,西太陰門也;第四曰上相,亦曰四輔也。」《漢志》:「環衛十二星,蕃臣:西,將;東,相;南四星,執法;中,端門;左右,掖門。」《乾象新書》:十星,東西各五,在翼、軫北。其西蕃北星為上相,南門右星為右執法。東西蕃有芒及動搖者,諸侯謀上。執法移,刑罰尤急。月、五星入太微軌道,吉;其所犯中坐,成刑。月犯太微垣,輔臣惡之,又君弱臣強,四方兵
 不制;犯執法,《海中占》云:「將相有免者期三年。」月入東西門、左右掖門,而南出端門,有叛臣,君憂;入西門,出東門,君憂,大臣假主威。月中犯乘守四輔,為臣失禮,輔臣有誅。月暈,天子以兵自衛。一月三暈太微,有赦。月食太微,大臣憂,王者惡之。歲星入,有赦;犯之,執法臣有憂;入東門,天下有急兵;守之,將、相、執法憲臣死;入端門,守天庭,大禍至;入南門,出東門,旱;入南門,逆出西門,國有喪;逆行入東門,出西門,國破亡。填星、熒惑犯之,逆行入,為兵、
 喪;犯上將,上將憂;守端門,國破亡,或三公謀上,有戮臣;犯西上將,天子戰於野,上相死;入太微,色白無芒,天下饑;退行不正,有大獄;犯太微門,左右將死;入天庭在屏星南,出左掖門左將死,右掖門右將死,直出端門無咎;入太微,凌犯、留止,為兵,入二十日,廷尉當之,留天庭十日有赦;犯太微東南陬,歲饑,執法大臣憂;犯上相,大臣死。填星犯入太微,有德令,女主執政。若逆行執法、四輔,守之,有憂;守太微,國破;守西蕃,王者憂。太白犯入太微,為兵,
 大臣相殺;留守,有兵、喪;與填星犯太微中,王者惡之;入右掖門,從端門出,貴人奪勢;晝見太微,國有兵、喪。月掩太白於端門外,國受兵。辰星犯太微,天子當之,有內亂;入天庭,後宮憂,大水;守左右執法,入,兵起,有赦;入西門,後宮災,大水;入西門,出東門,為兵、喪、水災。客星犯入太微,色黃白,天子喜;出入端門,國有憂;左掖門,旱;右掖門,國亂;出天庭,有苛令,兵起;入太微三十日,有赦;犯四輔,輔臣兇。彗星犯太微,天下易;出太微,宮中憂,火災;犯執
 法,執法者黜;犯天庭,王者有立;孛于翼,近太微上將,為兵、喪;孛於西蕃,主革命;孛五帝,亡國殺君。流星出太微,大臣有外事;出南門甚眾,貴人有死者;縱橫太微宮,主弱臣強;由端門入翼,光照地有聲,有立王。雲氣出入,色微青,君失位。青白黑雲氣入左右掖,為喪;出,無咎。赤氣入東掖門,內兵起。黃白雲氣入太微垣,人主喜,年壽長。入左右掖門,天子有德令。黑及蒼白氣入,天子憂,出則無咎。黑氣如蛇入垣門,有喪。



 內五帝坐五星,內一星在太微中,黃帝坐,含樞紐之神也。天子動得天度,止得地意,從容中道則明以光,不明則人主當求賢以輔法;不則奪勢。四帝星夾黃帝坐,四方各去二度。東方,蒼帝靈威仰之神也。南方,赤帝赤熛怒之神也。西方,白帝白招拒之神也。北方,黑帝葉光紀之神也。黃帝坐明,天子壽,威令行;小,則反是,勢在臣下;若亡,大人當之。月出坐北,禍大;出坐南,禍小;出近之,大臣誅,或饑;犯黃帝坐,有亂臣。抵帝坐,有土功事。月暈帝
 坐,有赦。《海中占》:月犯帝坐,人主惡之。五星守黃帝坐,大人憂。熒惑、太白入,有強臣。歲星犯,有非其主立。熒惑犯,兵亂;入天庭,至帝坐,有赦。太白入之,兵在宮中。填逆行,守黃帝坐,亡君之戒。五星入,色白,為亂。客星色黃白抵帝坐,臣獻美女。彗星入,宮亂;抵帝坐,或如粉絮,兵、喪並起。流星犯之,大臣憂;抵四帝坐,輔臣憂,人多死。蒼白氣抵帝坐,天子有喪;青赤,近臣欲謀其主;黃白,天子有子孫喜。月犯四帝,天下有喪,諸侯有憂。五星犯四帝,為憂。



 太子一星,在帝坐北,帝儲也。儲有德,則星明潤。雲氣入,黃為喜,黑為憂。太白、熒惑、客星、流星守、犯,皆為憂。一云金、火守之,或入,太子不廢則為篡逆之事。



 內五諸侯五星,在九卿西,內侍天子,不之國也。《乾象新書》:在郎位南,闢雍禮得,則星明;亡,則諸侯黜。



 從官一星,在太子北,侍臣也。以不見為安,一曰不見則帝不安,如常則吉。



 幸臣一星,在帝坐東北,常侍太子,以暗為吉。《新書》:在太
 子東,青、赤氣入之,近臣謀君不成。



 內屏四星,在端門內,近右執法。屏者,所以擁蔽帝庭也。



 左右執法各一星,在端門兩旁,左為廷尉之象,右為御史大夫之象,主舉刺兇奸。君臣有禮,則光明潤澤。《乾象新書》:在中臺南,明,則法令平。月、五星及客星犯守,則君臣失禮、輔臣黜。熒惑、太白入,為兵。流星犯之,尚書憂。



 郎位十五星,在帝坐東北,一曰依烏郎府也。周之元士,漢之光祿、中散、諫議、議郎、郎中是其職,主衛守也。其星
 不具,後妃災,幸臣誅。星明大,或客星入之,大臣為亂,元士憂。彗、孛犯,郎官失勢。彗星、枉矢出其次,郎佐謀叛。熒惑守之,兵、喪。赤氣入,兵起;黃白,吉;黑,兇。



 郎將一星,在郎位北,主閱具,以為武備也。若今之左、右中郎將。《新書》曰:在太微垣東北。明,大臣叛。客星犯、守,郎將誅。黃、白氣入,則受賜。流星犯,將軍憂。



 常陳七星,如畢狀,在帝坐北,天子宿衛虎賁之士,以設強禦也。星搖動,天子自出將;明,則武兵用;微,則弱。客星
 犯,王者行誅。



 九卿三星,在三公北,主治萬事,今九卿之象也。《乾象新書》:在內五諸侯南,占與天紀同。



 三公三星,在謁者東北,內坐朝會之所居也。《乾象新書》:在九卿南,其占與紫微垣三公同。



 謁者一星,在左執法東北,主贊賓客、辨疑惑。《乾象新書》:在太微垣門內,左執法北。明盛,則四夷朝貢。



 三臺六星,兩兩而居,起文昌,列抵太微。一曰天柱,三公
 之位也。在人曰三公,在天曰三臺,主開德宣符。西近文昌二星,曰上臺,為司命,主壽;次二星曰中臺,為司中,主宗室;東二星曰下臺,為司祿,主兵,所以昭德塞違也。又曰三臺為天階,太一躡以上下。一曰泰階,上階上星為天子,下星為女主;中階上星為諸侯三公,下星為卿大夫;下階上星為士,下星為庶人,所以和陰陽而理萬物也。又曰上臺上星主兗、豫,下星主荊、揚;中臺上星主梁、雍,下星主冀;下臺上星主青,下星主徐。人主好兵,則
 上階上星疏而色赤。修宮廣囿,肆聲色,則上階合而橫。君弱則上階迫而色暗。公侯皆叛,率部動兵,則中階上星赤。外夷來侵,邊國騷動,則中階下星疏而橫,色白。卿大夫廢正向邪,則中階下星疏而色赤。民不從令,犯刑為盜,則上階下星色黑。去本就末,奢侈相尚,則下階上星闊而橫,色白。君臣有道,賦省刑清,則上階為之戚。諸侯貢聘,公卿盡忠,則中階為之比。庶人奉化,徭役有敘,則下階為之密。若主奢欲,數奪民時,則上階為之奪。諸
 侯僭強,公卿專貪,則中階為之疏。士庶逐末,豪傑相凌,則下階為之闊。三階平,則陰陽和,風雨時,穀豐世泰;不平,則反是。三臺不具,天下失計。色明齊等,君臣和而政令行;微細,反是。一曰天柱不見,王者惡之。司命星亡,春不得耕。司中不具,夏不得耨。司祿不具,秋不得獲。一曰三臺色青,天下疾;赤,為兵;黃潤,為德;白,為喪;黑,為憂。月入,君憂,臣亂,公族叛。月入而暈,三公下獄。客星入之,貴臣賜爵邑;出而色蒼,臣奪爵;守之,大臣黜,或貴臣多病。
 彗星犯,三公黜。流星入,天下兵將憂;抵中臺,將相憂,人主惡之。雲氣入,蒼白,民多傷;黃白潤澤,民安君喜;黃,將相喜;赤,為憂;青黑,憂在三公;蒼白,三公黜。



 按上臺二星在柳北,其北星入柳六度。中臺二星,其北入張二度。下臺二星在太微垣西蕃北,其北星入翼二度。武密書:三臺屬鬼,又屬柳、屬張。《乾象新書》:上臺屬柳,中臺屬張,下臺屬翼。



 長垣四星,在少微星南,主界域,及北方。熒惑入之,北人
 入中國。太白入,九卿謀,邊將叛。彗、孛犯之,北地不安。流星入,北方兵起,將入中國。



 少微四星,在太微西,士大夫之位也。一名處士,亦天子副主,或曰博士官,一曰主衛掖門。南第一星處士,第二星議士,第三星博士,第四星大夫。明大而黃,則賢士舉。月五星犯守處士,女主憂,宰相易。歲犯,小人用,忠臣危。火犯,賢德退。土犯,宰相易,女主憂。金犯,大臣誅,又曰以所居主占之。客星、孛星犯之,王者憂,奸臣眾。彗星犯,
 功臣有罪,一曰法令臣誅。流星出,賢良進,道術用。雲氣入,色蒼白,賢士憂,大臣黜。



 靈臺三星,在明堂西,神之精明曰靈,四方而高曰臺,主觀雲物,察符瑞,候災變也。武密曰:與司怪占同。



 虎賁一星,在下臺星南,一曰在太微西蕃北、下臺南,靜室旄頭之騎官也。明,則臣順,與車騎星同占。



 明堂三星,在太微西南角外,天子布政之宮。明吉,暗兇。五星、客星及彗犯之,主不安其宮。



 右上元太微宮常星一十九坐,積數七十有八,而《晉志》所載,少微、長垣各四星,屬天市垣,與《步天歌》不同。



 天市垣



 天市垣二十二星,在氐、房、心、尾、箕、斗內宮之內。東蕃十一星:南一曰宋,二曰南海,三曰燕,四曰東海,五曰徐,六曰吳越,七曰齊,八曰中山,九曰九河,十曰趙,十一曰魏。西蕃十一星:南一曰韓,二曰楚,三曰梁,四曰巴,五曰蜀,六曰秦,七曰周,八曰鄭,九曰晉,十曰河間,十一曰河中。
 像天王在上,諸侯朝王,王出皋門大朝會,西方諸侯在應門左,東方諸侯在應門右。其率諸侯幸都市也亦然。一曰在房、心東北,主權衡,主聚眾。又曰天旗庭,主斬戮事。《乾象新書》曰:市中星眾潤澤,則歲實。熒惑守之,戮不忠之臣。彗星掃之,為徙市易都。客星入,為兵起;出,為貴喪。《天文錄》曰:天子之市,天下所會也。星明大,則市吏急,商人無利;小,則反是;忽然不明,糴貴;中多小星,則民富。月入天市,易政更弊,近臣有抵罪,兵起。月守其中,女主
 憂,大臣災。五星入,將相憂,五官災;守之,主市驚更弊。又曰:五星入,兵起。熒惑守,大饑,火災。或芒角色赤如血,市臣叛。填星守,糴貴。太白入,起兵,糴貴。辰星守,蠻夷君死。客星守,度量不平;星色白,市亂;出天市,有喪。彗星守,穀貴;出天市,豪傑起,徙易市都;掃帝坐,出天市,除舊布新。流星入,色蒼白,物貴;赤,火災,民疫。一曰出天市,為外兵。雲氣入,色蒼白,民多疾;蒼黑,物貴;出,物賤;黃白,物賤;黑,為嗇夫死。



 帝坐一星,在天市中,天皇大帝外坐也。光而潤澤,主吉,威令行;微小,大人憂。月犯之,人主憂。五星犯,臣謀主,下有叛;熒惑,尤甚。客星入,色赤,有兵;守之,大臣為亂。彗、孛犯,人民亂,宮廟徙。流星犯,諸侯兵起,臣謀主,貴人更令。



 候一星,在帝坐東北候,一作後,主伺陰陽也。明大,輔臣強;細微,國安;亡,則主失位;移,則不安居。太陰犯之,輔臣憂。客、彗守之,輔臣黜。孛犯,臣謀叛。



 宦者四星,在帝坐西南侍,主刑餘之臣也。星微,吉;失常,
 宦者有憂。



 斗五星,在宦者南,主平量。《乾象新書》:在帝坐西,覆則歲熟,仰則荒。客、彗犯,為饑。



 斛四星,在斗南,主度量、分銖、算數。其星不明,兇;亡,則年饑。一曰在市樓北,名天斛。



 列肆二星,在斛西北,主貨金、玉、珠、璣。



 屠肆二星,在帛度東北,主屠宰、烹殺。《乾象新書》:在天市垣內十五度。



 車肆二星,在天市門中,主百貨。星不明,則車蓋盡行;明,則吉。客星、彗星守之,天下兵車盡發。《乾象新書》:在天市垣南門偏東。



 宗正二星,在帝坐東南,宗大夫也。武密曰:主囗司宗得失之官。《乾象新書》:在宗人西。彗星守之,若失色,宗正有事。客星守之,更號令也;犯之,主不親宗廟。星孛其分,宗正黜。



 宗人四星,在宗正東,主錄親疏享祀。宗族有序,則星如
 綺文而明正;動,則天子親屬有變。客星守之,貴人死。



 宗星二星,在候星東,宗室之象,帝輔血脈之臣。《乾象新書》:在宗人北。客星守之,宗支不和;暗,則宗支弱。



 帛度二星,在宗星東北,主度量買賣平貨易者。《乾象新書》:在屠肆南。星明大,尺量平,商人不欺。客星、彗星守之,絲綿大貴。



 市樓六星,在天市中,臨箕星之上,市府也,主市賈律度。其陽為金錢,陰為珠玉,變見,各以其所占之。《乾象新書》:
 主闤闠,度律制令,在天市中。星明,吉;暗,則市吏不理。彗星、客星守之,市門多閉。



 七公七星,在招搖東,為天相,三公之象也,主七政。明,則輔佐強;大而動,為兵;齊政,則國法平;戾,則獄多囚;連貫索,則世亂;入河中,糴貴,民饑。太白守之,天下亂,兵起。客星守,歲饑,主危。流星出其分,主將黜。



 貫索九星,在七公星前,賤人之牢也。一曰連索,一曰連營,一曰天牢,主法律,禁強暴。牢口一星為門,欲其開也。
 星在天市垣北。星皆明,天下獄繁;七星見,小赦;五星、六星,大赦;動,則斧鍎用;中空,改元。石申曰:一星亡,則有賜爵;三星亡,大赦,遠期八十日;入河中,為饑;中星眾,則囚多。辰星犯之,主水,米貴。彗星出,其分中外豪傑起。客星入,有枉死者;色黃,諸侯獻地;青,為憂;赤,為兵;白,乃為吉。流星入,女主憂,或赦;出,則貴女死。雲氣入,色蒼白,天子亡地;青,兵起;黑,獄多枉死;白,天子喜。



 天紀九星,在貫索東,九卿之象,萬事綱紀,主獄訟。星明,
 則天下多訟;亡,則政理壞,國紀亂;散絕,則地震山崩;與女床合,則君失禮,女謁行。客星守之,主危,民饑。客星犯,諸侯舉兵。彗、孛犯之,地震。客星、彗星合守,天下獄訟不理。



 女床三星,在天紀北,後宮御女侍從官也,主女事。明,則宮人恣;舒,則妾代女主;不動,則吉;不見,女子多疾。客星、彗星守之,宮人謀上。客星入,女子憂,後宮恣動,女謁行。雲氣出,色黃,後宮有福;白,為喪;黑,兇;青,女多疾。



 右天市垣常星可名者一十七坐,積數八十有八。而市樓、天斛、列肆、車肆、斗、帛度、屠肆等星,《晉志》皆不載,《隋志》有之,屬天市垣,與《步天歌》合。又貫索,七公、女床、天紀,《晉志》屬太微垣。按《乾象新書》:天紀在天市垣北,女床屬箕宿,貫索屬房宿,七公屬氐宿。武密以七公屬房,又屬尾;貫索屬房,又屬氐、屬心;女床屬於尾、箕。說皆不同。



\end{pinyinscope}