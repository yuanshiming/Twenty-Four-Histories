\article{列傳第一百七十九}

\begin{pinyinscope}

 ○王伯大鄭寀應人繇徐清叟李曾伯王野蔡抗張磻馬天驥朱熠饒虎臣戴慶炣皮龍榮沈炎



 王伯大,字幼學,福州人。嘉定七年進士。歷官主管戶部
 架閣,遷國子正、知臨江軍,歲饑,振荒有法。遷國子監丞、知信陽軍,改知池州兼權江東提舉常平。久之,依舊直秘閣、江東提舉常平,仍兼知池州。端平三年,召至闕下,遷尚右郎官,尋兼權左司郎官,遷右司郎官、試將作監兼右司郎中,兼提領鎮江、建寧府轉般倉,兼提領平江府百萬倉,兼提領措置官田。進直寶謨閣、樞密副都承旨兼左司郎中。進對,言:



 今天下大勢如江河之決,日趨日下而不可挽。其始也,搢紳之論,莫不交口誦詠,謂太平之
 期可矯足而待也;未幾,則以治亂安危之制為言矣;又未幾,則置治安不言而直以危亂言矣;又未幾,則置危亂不言而直以亡言矣。嗚呼,以亡為言,猶知有亡矣,今也置亡而不言矣。人主之患,莫大乎處危亡而不知;人臣之罪,莫大乎知危亡而不言。



 陛下親政,五年於茲,盛德大業未能著見於天下,而招天下之謗議者何其籍籍而未已也?議逸欲之害德,則天下將以陛下為商紂、周幽之人主;議戚宦近習之撓政,則天下將以朝廷為
 恭、顯、許、史、武、韋、仇、魚之朝廷;議奸儔佞朋之誤國,則天下又將為漢黨錮、元祐黨籍之君子。數者皆犯前古危亡之轍跡,忠臣懇惻而言之,志士憤激而和之。陛下雖日御治朝,日親儒者,日修辭飾色,而終莫能弭天下之議。言者執之而不肯置,聽者厭之而不憚煩,於是厭轉而為疑,疑增而為忿,忿極而為愎,則罪言黜諫之意藏伏於陛下之胸中,而凡迕己者皆可逐之人矣。彼中人之性,利害不出於一身,莫不破厓絕角以阿陛下之所
 好。其稍畏名義者,則包羞閔默而有跋前疐後之憂;若其無所顧戀者,則皆攘袂遠引,不願立於王之朝矣。



 陛下試反於身而自省曰:吾之制行,得無有屋漏在上、知之在下者乎?徒見嬖暱之多,選擇未已,排當之聲,時有流聞,則謂精神之內守,血氣之順軌,未可也。陛下又試於宮閫之內而加省曰:凡吾之左右近屬,得無有因微而入,緣形而出,意所狎信不復猜覺者乎?徒見內降干請,數至有司,裏言除臣,每實人口,則謂浸潤之不行,邪
 逕之已塞,未可也。陛下又試於朝廷政事之間而三省曰:凡吾之諸臣,得無有讒說殄行,震驚朕師,惡直醜正,側言改度者乎?徒見剛方峭直之士,昔者所進,今不知其亡,柔佞闒茸之徒,適從何來,而遽集於斯也,則謂舉國皆忠臣,聖朝無闕事,未可也。



 夫以陛下之好惡用舍,無非有招致人言之道;及人言之來,又復推而不受。不知平日之際遇信任者,肯為陛下分此謗乎?無也。陛下誠能布所失於天下,而不必曲為之回護,凡人言之所
 不貸者,一朝赫然而盡去之,務使蠹根悉拔,孽種不留,如日月之更,如風雷之迅,則天下之謗,不改而自息矣。陛下何憚何疑而不為此哉!



 又極言邊事,曲盡事情。



 以直寶謨閣知婺州。遷秘書少監,拜司農卿,復為秘書少監,進太常少卿兼中書門下檢正諸房公事。遷起居舍人,升起居郎兼權刑部侍郎。臣僚論罷,以集英殿修撰提舉太平興國宮。起,再知婺州,辭免,復舊祠。



 淳祐四年,召至闕,授權吏部侍郎兼權中書舍人。尋為吏部侍郎
 仍兼權中書舍人、兼侍讀。時暫兼權侍右侍郎,兼同修國史、實錄院同修撰。權刑部尚書,尋為真。七年,拜端明殿學士、簽書樞密院事兼權參知政事。八年,拜參知政事。以監察御史陳垓論罷,以資政殿學士知建寧府。寶祐元年,卒。



 鄭採,不詳何郡人。初歷官為秘書省校書郎兼國史編修、實錄檢討。遷著作佐郎兼權侍右郎官,升著作郎兼侍講。拜右正言,言:「丞相史嵩之以父憂去,遽欲起之,意
 甚厚也。奈何謗議未息,事關名教,有尼其行。」帝答曰:「卿言雖切事理,進退大臣豈易事也!」



 擢殿中侍御史。疏言:「臺諫以糾察官邪為職,國之紀綱系焉。比劉漢弼劾奏司農卿謝逵,陛下已行其言矣,未及兩月,忽復敘用,何其速也!漢弼雖亡,官不可廢。臣非為漢弼惜,為朝廷惜也。」又奏劾王瓚、龔基先、胡清獻,鐫秩罷祠,皆從之。三人者,不才臺諫也。



 遷侍御史。疏言:「比年以來,舊章寢廢。外而諸閫,不問勛勞之有無,而爵秩皆得以例遷;內而侍
 從,不問才業之憂劣,而職位皆可以例進。執政之歸休田里者,與之貼職可也,而凡補外者,皆授之矣。故自公侯以至節度,有同序補,自書殿以至秘閣,錯立周行。名器之輕,莫此為甚。無功者受賞,則何以旌有功之士;有罪者假寵,則何以服無罪之人。矧事變無窮,而名器有限,使名器常重於上,則人心不敢輕視於下,非才而罔功者不得覬幸於其間,則負慷慨之氣、懷功名之願者,陛下始可得而鼓舞之矣。」遷左諫議大夫。



 淳祐七年,拜
 端明殿學士、同簽書樞密院。以監察御史陳求魯論罷。淳祐九年五月,卒。採之居言路,嘗按工部侍郎曹豳、主管吏部架閣文字洪芹,則大傷公論云。



 應人繇,字之道,慶元府昌國人。刻志於學。嘉定十六年,試南省第一,遂舉進士,為臨江軍教授。入為國子學錄兼莊文府教授。遷太學博士,又遷秘書郎,請蚤建太子。入對,帝問星變,人繇請「修實德以答天戒」。帝問州縣貪風,人繇曰:「貪黷由殉色而起。成湯制官刑,儆有位,首及於巫風
 淫風者,有以也。」帝問藏書,人繇請「訪先儒解經注史」,因及程迥、張根所著書皆有益世教。帝善之。遷秘書省著作佐郎兼權尚左郎官、兼翰林權直。又遷著作郎,仍兼職,以言罷。



 淳祐二年,敘復奉祠。遷宗正寺丞兼權禮部郎官,兼國史編修、實錄檢討,以言罷。差知臺州,召兼禮部郎官、崇政殿說書。遷秘書少監,仍兼職,兼權直學士院。又遷起居舍人、權兵部侍郎,時暫兼權吏部侍郎兼直學士院,帝一夕召人繇草麻,夜四鼓,五制皆就,帝奇其才。
 遷吏部侍郎仍兼職。進翰林學士兼中書舍人。



 八年,授同知樞密院事兼參知政事。九年拜參知政事,封臨海郡侯,乞歸田里。以資政殿學士知平江府,提舉洞霄宮。寶祐三年,殿中侍御史丁大全論罷,尋卒。德祐元年,詔復元職致仕。



 徐清叟,字直翁,煥章閣學士應龍之子。嘉定七年進士。歷主管戶部架閣,遷籍田令。疏言:「邇者江右、閩嶠,盜賊竊發,監司帥守,未免少立威名,專行誅戮,此特以權濟
 事而已。而偏州僻壘,習熟見聞,轉相仿效,亦皆不俟論報,輒行專殺。欲望明行禁止,一變臣下嗜殺希進之心,以無墜祖宗立國仁厚之意。」遷軍器監主簿。入對,言:「太后舉哀之日,陛下以後服下同媵妾,令別置大袖一襲。文思院觀望,欲如後飾,再造其一以進,詔卻之。此真知嫡庶之辨者。請宣付史館,以垂法後世。」



 遷太常博士。入對,疏言:「陛下親政以來,精神少振而氣脈未復,條目畢舉而綱紀未張,公道若伸而私意之未盡克者,則亦風
 化之先務,勸戒之大權,與夫選用之要術,猶有闕略而未之講明者爾。何謂風化之先務?曰原人倫以釋群惑者是已。何謂勸戒之大權?曰惜名器以示正義者是已。何謂選用之要術?曰因物望而進人才者是已。」蓋欲請復皇子竑王爵,裁抑史彌遠恤典,召用真德秀、魏了翁也。



 兼崇政殿說書。遷秘書郎,升著作佐郎兼權司封郎官,遷軍器少監,皆兼職依舊。遷將作監,拜殿中侍御史兼侍講。遷太常少卿兼權戶部侍郎兼侍講。三疏丐外,
 給事中洪咨夔、起居舍人吳泳皆抗疏留之。尋權工部侍郎。以右文殿修撰知泉州,集英殿修撰知靜江府、廣西經略安撫使。遷侍右侍郎、主管雲臺觀。召赴闕,遷戶部侍郎,再為侍右侍郎。以寶章閣直學士知溫州,改知福建安撫使,改知婺州。以煥章閣直學士差知泉州,辭免。改知袁州,又改知紹興府、兩浙東路安撫使,辭免。改知潭州,尋知廣州兼廣東經略安撫使。



 召赴闕,權兵部尚書兼侍讀。淳祐九年,兼同修國史、實錄院同修撰,權
 吏部尚書,遷禮部尚書。拜端明殿學士、簽書樞密院事,進同知樞密院事,封晉寧郡公。奏修《四朝國史》志傳,五上章乞改機政,帝不許。十二年,拜參知政事。尋知樞密院事兼參知政事,監察御史朱應元論罷,以資政殿大學士提舉玉隆萬壽宮,改洞霄宮,復以監察御史朱熠論罷。久之,以舊職提舉洞霄宮。



 開慶元年,召赴闕,以舊職提舉祐神觀兼侍讀。出知泉州,復提舉祐神觀。景定三年,轉兩官致仕,卒,贈少師,謚忠簡。清叟父子兄弟皆
 以風節相尚,而清叟劾罷袁甫,於公論少貶云。



 李曾伯,字長孺,覃懷人,後居嘉興。歷官通判濠州,遷軍器監主簿,添差通判鄂州兼沿江制置副使司主管機宜文字。遷度支郎官,授左司郎官、淮西總領。尋遷右司郎官,太府少卿兼左司郎官,兼敕令所刪修官。遷太府卿、淮東制置使兼淮西制置使,詔軍事便宜行之。曾伯疏奏三事:答天心,重地勢,協人謀。又言:「邊餉貴於廣積,將材貴於素儲,賞與不可以不精,戰士不可以不恤。」又
 條上:「淮面舟師之所當戒,湖面險阻之所當治。」加華文閣待制,又加寶章閣直學士,進權兵部尚書。



 淳祐六年正月朔,日食。曾伯應詔,歷陳先朝因天象以謹邊備、圖帥材,乞早易閫寄,放歸田里。又請修浚泗州西城。加煥章閣學士,言者相繼論罷。



 九年,以舊職知靜江府、廣西經略安撫使,兼廣西轉運使。陳守邊之宜五事。進徽猷閣學士、京湖安撫制置使、知江陵府,兼湖廣總領,兼京湖屯田使,進龍圖閣學士。疏言:「襄陽新復之地,城池雖
 修浚,田野未加闢;室廬雖草創,市井未阜通。請蠲租三年。」詔從之。加端明殿學士兼夔路策應大使。進資政殿學士,制置四川邊面,與執政恩例。尋授四川宣撫使,特賜同進士出身。召赴闕,加大學士,知福州兼福建安撫使。辭免,以大學士提舉洞霄宮。



 起為湖南安撫大使兼知潭州,兼節制廣南,移治靜江。開慶元年,進觀文殿學士,以諫議大夫沈炎等論罷。景定五年,起知慶元府兼沿海制置使。咸淳元年,殿中侍御史陳宗禮論劾,褫職。
 德祐元年,追復元官。



 曾伯初與賈似道俱為閫帥,邊境之事,知無不言。似道卒嫉之,使不竟其用云。



 王野,字子文,寶章閣待制介之子也。以父陰補官,登嘉定十二年進士第。仕潭時,帥真德秀一見異之,延致幕下,遂執弟子禮。德秀欲授以詞學,野曰:「所以求學者,義理之奧也。詞科惟強記者能之。」德秀益器重之。



 紹定初,汀、邵盜作,闢議幕參贊,攝邵武縣,後復攝軍事。盜起唐石,親勒兵討之。後為樞密院編修兼檢討。襄、蜀事急,議
 遣使講和,時相依違不決。史嵩之帥武昌,首進和議。野言:「今日之事宜先定規模,並力攻守。」上疏言八事。繼為副都承旨,奏請「出師,絕和使,命淮東、西夾攻。不然,利害將深。」理宗深然之,令樞密院下三閫諭旨。嘉熙元年,輪對,採事系安危者四端,而專以司馬光仁、明、武推說。復推廣前所言八事,以孝宗講軍實激發帝意。



 淳祐初,自江西赴闕,奏祈天永命十事。嵩之起復,傾國爭之,野上疏乞聽終喪,後又言嵩之當顯絕而終斥,益嚴君子小
 人之限。拜禮部尚書,奏十事,終之曰:「陛下一心,十事之綱領也。」前後奏陳,皆明正剴切,鑿鑿可行。其為兩浙轉運判官,以察訪使出視江防,首嘉興至京口增修官民兵船守險備具。為江西轉運副使、知隆興府,繼有它命,時以米綱不便,就湖口造轉般倉,請事畢受代。



 知鎮江府,兼都大提舉浙西兵船。江面幾千里,調兵捍禦,以守江尤重於淮,瓜洲一渡甚狹,請免鎮江水軍調發,專一守江,置游兵如呂蒙所言「蔣欽將萬人巡江上」,增創水
 艦,就揚子江習水戰,登金山指麾之。是冬,揚子橋有警,急調湯孝信所領游兵救之而退。



 淳祐末,遷沿江制置使、江東安撫使、節制和州無為軍安慶府兼三郡屯田、行宮留守。巡江,引水軍大閱,舳艫相銜幾三十里。憑高望遠,考求山川險厄,謂要務莫如屯田。講行事宜,修飭行宮諸殿室,推京口法,創游擊軍萬二千,蒙沖萬艘,江上晏然。寶祐二年,拜端明殿學士、簽書樞密院事,封吳郡侯。與宰相不合,言者攻之,以前職主管洞霄宮。卒,贈
 七官,位特進。



 野因德秀知朱熹之學,凡熹門人高弟,必加敬禮。知建寧府,創建安書院,祠熹,以德秀配。有奏議、文集若干卷。野工於詩,書法祖唐歐陽詢,署書尤清勁。



 蔡抗,子仲節,處士元定之孫。紹定二年進士。其後差主管尚書刑、工部架閣文字。召試館職,遷秘書省正字。升校書郎兼樞密院編修官,遷諸王宮大小學教授。疏奏:「權奸不可復用,國本不可不早定。」帝善其言。遷樞密院編修官兼權屯田郎官。遷著作佐郎兼侍右郎官,兼樞
 密院編修官。尋兼國史院編修官、實錄檢討官。江東提點刑獄,加直秘閣,特授尚書司封員外郎,進直寶章閣,尋加寶謨閣,移浙東。召為國子司業兼資善堂贊讀,兼玉牒所檢討官,時暫兼侍立修注官。拜宗正少卿兼國子司業。進直龍圖閣、知隆興府。試國子祭酒兼侍立修注官。拜太常少卿,仍兼資善堂翊善。權工部侍郎兼國史院編修官、實錄院檢討官。



 遷工部侍郎,時暫兼禮部侍郎,兼權吏部尚書。加端明殿學士、同簽書樞密院事,
 差兼同提舉編修《經武要略》。同知樞密院事,拜參知政事。落職予祠,起居郎林存請加竄削,從之。未逾年,復端明殿學士、提舉洞霄宮。乞致仕。轉一官,守本官職致仕。卒,謚文簡,以犯祖諱,更謚文肅。



 張磻,字渭老,福州人。嘉定四年進士。歷官闢點檢贍軍激賞酒庫所主管文字,差主管尚書吏部架閣。遷太常博士、宗正丞兼權兵部郎官。遷國子祭酒,時暫兼權禮部侍郎,尋為真,兼國史編修、實錄檢討。加集英殿修撰,
 差知婺州。復為禮部侍郎、權兵部尚書,時暫兼權吏部尚書。以右補闕程元鳳論罷。寶祐三年,復權刑部尚書兼侍讀,拜端明殿學士、簽書樞密院事,升同知樞密院事兼參知政事。五年,拜參知政事。進封長樂郡公,轉三官,守參知政事致仕。九月,卒。遺表上,贈少師。



 馬天驥,字德夫,衢州人。紹定二年進士,補簽書領南判官廳公事。遷秘書省正字兼沂靖惠王府教授。遷秘書省校書郎,升著作佐郎。輪對,假司馬光五規之名,條上
 時敝,詞旨切直。遷考功郎官,入對,言:「周世宗當天下四分五裂之餘,一念振刷,猶能轉弱為強,陛下有能致之資,乘可為之勢,一轉移間耳。」



 遷秘書監、直秘閣、知吉州。遷宗正少卿,以秘閣修撰知紹興府,主管浙東安撫司公事兼提舉常平。權兵部侍郎,授沿海制置使,差知慶元府。改知池州兼江東提舉常平。改知廣州兼廣東經略安撫使。寶祐四年,遷禮部侍郎,兼直學士院,兼侍讀,兼國子祭酒。拜端明殿學士、同簽書樞密院事,封信安
 郡侯。五年,以殿中侍御史朱熠、右正言戴慶炣、監察御史吳衍翁應弼等論罷,依舊職提舉洞霄宮。景定元年,知衢州,以兵部侍郎章鑒論罷。有旨,依舊職予祠。起知福州、福建安撫使,以職事修舉,升大學士。改知平江府。又改知慶元府兼沿海制置使,提舉洞霄宮。褫職罷祠。咸淳三年,追奪執政恩數,送信州居住。四年,放令自便,後卒於家。



 朱熠,溫州平陽人。端平二年,武舉第一。遷閣門舍人,差
 知沅州,改橫州,復為閣門舍人、知雷州。入對,為監察御史陳垓論罷;臣僚復論,降一官。久之,授帶御器械兼乾辦皇城司,差知興國軍。遷度支郎官,拜監察御史兼崇政殿說書。擢右正言,殿中侍御史兼侍講,遷侍御史。寶祐六年,遷左諫議大夫。拜端明殿學士、簽書樞密院事,同知樞密院事。開慶元年,拜參知政事兼權知樞密院事。景定元年,知樞密院事兼參知政事,兼太子賓客。以舊職知慶元府、沿海制置使。奉祠。為監察御史胡用虎
 論罷。久之,監察御史張桂、常茂相繼糾劾,送處州居住。咸淳四年,詔令自便。五年,侍御史章鑒復以為言,驅之還鄉,尋卒。熠居言路彈劾最多,一時名士若徐清叟、呂中、尤焴、馬廷鸞,亦皆不免雲。



 饒虎臣,字宗召,寧國人。嘉定七年進士。歷官遷將作監主簿,差知徽州。遷秘書郎,升著作郎兼權右司郎官。遷兵部郎官兼權左司郎官,特授左司郎中。遷司農少卿兼左司,兼國史編修、實錄檢討。遷司農卿、直龍圖閣、福
 建轉運判官,浙東提點刑獄。拜太府卿兼中書門下檢正諸房公事。以秘閣修撰、兩浙轉運使權禮部侍郎,尋為真。時暫兼權侍右侍郎。寶祐六年,兼同修國史、實錄院同修撰,暫通攝吏部尚書。拜端明殿學士、同簽書樞密院事。開慶元年,同知樞密院事,兼權參知政事。景定元年,拜參知政事。殿中侍御史何夢然論罷,以資政殿學士提舉洞霄宮。夢然再劾之,褫職罷祠。四年,敘復元官,提舉太平興國宮。卒。德祐元年,禮部侍郎王應麟、右
 史徐宗仁乞追復元官,守資政殿學士致仕。



 戴慶炣,字彥可,溫州永嘉人。淳祐十年進士。歷官差主管戶部架閣文字。召試館職,遷秘書省正字兼史館校勘。升校書郎,遷右正言、左司諫、殿中侍御史。升侍御史。開慶元年,拜右諫議大夫。尋加端明殿學士、簽書樞密院事兼權參知政事,同知樞密院事兼參知政事。未幾,守本官致仕。卒,贈特進、資政殿大學士。



 皮龍榮,字起霖,一字季遠,潭州醴陵人。淳祐四年進士。
 歷官主管吏部架閣文字,遷宗學諭,授諸王宮大小學教授兼資善堂直講。入對,請「以改過之實,易運化之名,一過改而一善著,百過改而百善融。」遷秘書郎,升著作郎。入對,因及真德秀、崔與之廉,龍榮曰:「今天下豈無廉者,願陛下崇獎之以風天下,執賞罰之公以示勸懲。」帝以為然。兼兵部郎官、差知嘉興府。



 召赴闕,遷侍右郎官兼資善堂贊讀。又遷吏部員外郎兼直講。入對,言:「忠王之學,願陛下身教之於內。」帝嘉納。遷將作監兼尚右郎
 官,秘書少監兼吏部郎中,宗正少卿、起居郎兼權侍左侍郎,兼給事中,吏部侍郎兼贊讀,封醴陵縣男。遷集賢殿修撰、提舉太平興國宮。召見,進刑部侍郎,加寶章閣待制、荊湖南路轉運使,權刑部尚書兼翊善。景定元年四月,拜端明殿學士、簽書樞密院,進封伯。權參知政事兼太子賓客。二年,拜參知政事,仍兼太子賓客,封壽沙郡公。三年,罷為湖南安撫使,判潭州。四年,以資政殿大學士提舉洞霄宮。以右正言曹孝慶論罷。



 咸淳元年,以
 舊職奉祠。殿中侍御史陳宗禮、監察御史林拾先後論劾,削一官。它日,帝偶問龍榮安在,賈似道恐其召用,陰諷湖南提點刑獄李雷應劾之。雷應至官,謁龍榮,龍榮托故不出;既退,又斥罵之。或以語雷應,不能平,遂疏其罪,又謂「每對人言,有『吾擁至尊於膝上』之語。」詔徙衡州居住。湖南提刑治衡州,龍榮恐不為雷應所容,未至而歿。



 龍榮少有志略,精於《春秋》學,有文集三十卷。性伉直,似道當國,不肯降志。又以度宗舊學,卒為似道所擯。德
 祐元年,復其官致仕。二年,太府卿柳岳乞加贈謚,未及行而宋亡。



 沈炎,字若晦,嘉興人。寶慶二年進士。調嵊縣主簿,廣西經略司準備差遣,湖南安撫司干辦公事。討郴寇有功,改知金華縣,沿江制置司干官。通判和州,沿江制置主管機宜文字。監三省、樞密院門,樞密院編修官。為監察御史、右正言、左司諫、殿中侍御史、侍御史。景定元年,拜右諫議大夫。加端明殿學士、同簽書樞密院事兼太子
 賓客。二年,拜同知樞密院事,兼權參知政事,以資政殿學士提舉洞霄宮。三年,進大學士,致仕。卒,贈少保。炎居言路,嘗按劾福建轉運使高斯得、觀文殿學士李曾伯、沿江制置司參謀官劉子澄、左丞相吳潛。然論罷右丞相丁大全及其黨與,則為公論也。



 論曰:王伯大立朝直諒。鄭寀、沈炎居言路,不辨君子小人,皆彈拄之,吾不知其何說也。應人繇清慎沒世。徐清叟風採凜乎班行之間。李採伯之治邊,短於才者也。王野
 得名父師,而其學問益光。蔡抗號為君子,史闕其事。若張磻、馬天驥、饒虎臣未見卓然有可稱道者。戴慶炣、皮龍榮登第皆未久而位至執政,龍榮不附權臣,為所擯斥而死,猶為可取,慶炣無所稱述焉。朱熠在臺察如狂猘,遇人輒噬之云。



\end{pinyinscope}