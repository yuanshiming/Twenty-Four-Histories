\article{志第九十八 儀衛三}

\begin{pinyinscope}

 國初鹵簿



 國初鹵簿。太祖建隆四年,將郊祀,大禮使範質與鹵簿使張昭、儀仗使劉溫叟,同詳定大駕鹵簿之制,惟得唐長興《南郊鹵簿字圖》,校以令文,頗有闊略違戾者。禮儀
 使陶穀建議:「金吾及諸衛將軍導駕及押仗,舊服紫衣,請依《開元禮》各服本色繡袍。金吾以闢邪,左右衛以瑞馬,驍衛以雕虎,威衛以赤豹,武衛以瑞鷹,領軍衛以白澤,監門衛以師子,千牛衛以犀牛,六軍以孔雀為文。舊,執仗軍士悉衣五色畫衣,隨人數給之,無有準式,請以五行相生之色為次,黑衣先之,青衣次之,赤、黃、白又次之。大駕五輅,各有副車,近代浸廢,請依令文增造。又案明宗舊圖,導駕三引而儀仗法物人數多,周太祖鹵簿
 六引而人數少,請準令文用六引,其鹵簿各依本品以給。」從之。舊清游隊有甲騎具裝,亡其制度,穀以其所記造之。又作大輦,皆率意定其制。穀又取天文大角、攝提列星之象,作攝提旗及北斗旗、二十八宿旗、十二辰旗、龍墀十三旗、五方神旗、五方鳳旗、四瀆旗。時有貢黃鸚鵡、白兔,及馴象自來,又作金鸚鵡、玉兔、馴象旗。太祖又詔別造大黃龍負圖旗一,大神旗六,日旗一,月旗一,君王萬歲旗一,天下太平旗一,師子旗二,金鸞旗一,金鳳
 旗一,五龍旗五,凡二十一旗,皆有架,南郊用之。大黃龍負圖旗陳於明德門前,餘二十旗悉立於宿頓宮前,遇朝會冊禮,亦皆陳於殿庭。凡馬步儀仗,共一萬一千二百二十二人,悉用禁軍。大將軍、將軍以軍主、都虞候攝事,中郎將、都尉以指揮使、副指揮使攝事,校尉、主帥以軍使、副兵馬使、都頭、副都頭、十將攝事。



 乾德三年,蜀平,命左拾遺孫逢吉收蜀法物,其不中度者悉毀之。是歲,太祖親閱鹵簿。四年,始令改畫衣為繡衣,至開寶三年
 而成,謂之「繡衣鹵簿」。其後郊祀皆用之。軍衛羽儀,自是浸甚。每大祀,命大禮、禮儀、儀仗、鹵簿、橋道頓遞五使,鹵簿使專掌定字圖排列,儀仗使糾督之,大禮及餘使同按閱,致齋日巡仗。又命殿前大校管勾捧日、奉宸隊,侍衛大校勾當儀仗兵隊,捧日、天武廂主四人,編排捧日、奉宸隊及執仗人,內諸司使、副使三員同押儀仗,別二員編排導引官。六年,詔節度使已下,除在京巡檢及押儀仗外,並令服褲褶衣導引。



 太宗至道中,令有司以絹畫
 為圖,圖凡三幅,中幅車輅、六引及導駕官,外兩幅儀衛,其警場青城,又別為圖,圖成,以藏秘閣。凡仗內自行事官、排列職掌並捧日、奉宸、散手天武外,步騎一萬九千一百九十八人,此極盛也。



 真宗咸平五年,詔南郊儀仗引駕官,不得多帶從人。宰臣,親王,樞密、宣徽使,參知政事,樞密副使,三司使,各四人。尚書,節度使,翰林學士、侍讀、侍講學士,各三人。給事,諫議,知制誥,大卿監,金吾大將軍,樞密都承旨、副承旨,客省閣門使、副使,諸司使、副
 使至內殿崇班,各二人。少卿監,諸行郎中已下,閣門祗候已下,各一人。又詔南郊引駕官,中書、樞密院一行在東,親王一行在西,餘依官次。大中祥符元年,改小駕為鸞駕。



 自太祖易繡衣鹵簿後,太宗、真宗皆增益之。仁宗即位,儀典多襲前世,宋綬定鹵簿,為《圖記》十卷上之,詔以付秘閣。凡大駕,用二萬六十一人,大率以太僕寺主車輅,殿中省主輿輦、傘扇、御馬,金吾主纛、槊、十六騎、引駕細仗、牙門,六軍主槍仗,尚書兵部主六引諸隊、大角、
 五牛旗,門下省主寶案,司天臺主鐘漏,太常主鼓吹,朝服法物庫出旗器、名物、衣冠、幰蓋,軍器庫出箙、弩、矢,內弓箭庫出戎裝、雜仗。凡六引導駕、太僕卿、千牛將軍、殿中侍御史、司天監少府監僚佐局官、乘黃令、大將軍、金吾上將軍、將軍、六統軍,皆以京朝官內諸司使、副使以下攝事。仗內用禁軍諸班直:捧日、天武、拱聖、神勇、宣武、驍騎、武勝、寧朔、虎翼兵。大將軍、將軍以軍主、都虞候攝。中郎將、郎將、都尉以指揮使、副指揮使攝。校尉、主帥、旅
 帥、隊正以軍使、副兵馬使、都頭、副都頭、十將攝。餘法駕、鸞駕、黃麾仗,則遞減其數。



 景祐五年,賈昌朝言儀衛三事:



 一曰南郊鹵簿,車駕出宮詣郊廟日,執球杖供奉官,於導駕官前分列迎引,至於齋宮。夫球杖非古,蓋唐世尚之,以資玩樂。其執之者皆褻服,錦繡珠玉,過於侈麗,既不足以昭文物,又不可以備軍容。常時豫游,或宜施用。方今夙夜齋戒,親奉大祀,端冕顒昂,鼓吹不作,而乃陳戲賞之具,參簪紳之列,導迎法駕,入於祠宮。稽諸典
 儀,未為允稱。況導駕官兩省員數悉備,何煩更有此色供奉官,謂宜徹去球杖,俟禮畢還宮,鼓吹振作,即復使就列。



 二曰大駕鹵簿,有羊車前列。臣案羊車本漢、晉之代,乘於後宮。隋大業中,增金寶之飾,駕以小駟,馭以傳童,自是以來,遂為法從。唐制兼有輦車、副車之名,國朝因循,尚未改革。竊以郊祭天地,廟見祖宗,車服所陳,動必由禮。至於四望、耕根之屬,兼包歷代,皆或有因,豈容後宮所乘,參陪五輅。欲望大駕不用羊車,所冀肅恭,稽
 合典禮。



 三曰南郊大駕鹵簿,儀衛甚眾,有司雖依典禮,名物次第,兵杖數目,預先分布,及五使量行案閱。其如被差執掌吏員兵伍,素不閑習,行列先後,多失次序;所持名物,亦或差互。押當官但以行事為名,從便趨進,失其處守。竊謂三載親郊,國之大事,旁陳象物,仰法乾行,四方之人,觀禮於是,宜詳制度,以示光華。請大駕鹵簿前後仗衛次第,於致齋前命儀仗、鹵簿使令有司執簿籍率押當官暨諸衛、諸省執仗士卒將領者,自殿門至
 郊廟分列之處,詳視先後及器仗名品,無令差忒。



 詔禮儀使宋綬與太常禮院同詳定以奏。綬奏:「鹵簿內有諸司供奉,蓋資備物,以奉乘輿。今昌朝言宿齋之時,不可陳玩樂之具。請郊祀前一日,應供奉官等令宿幕次,俟皇帝行禮畢降壇,導至青城,由青城前導歸大內。後漢劉熙《釋名》曰:「騾車、羊車,各以所駕名之也。」隋《禮儀志》曰:「漢氏或以人牽,或駕果下馬。」此乃漢代已有,晉武偶取乘於後宮,非特為掖庭制也。況歷代載於《輿服志》,自唐
 至今,著之禮令,宜且仍舊。其鹵簿儀仗,遇南郊前,五使預閱素備,願依昌朝所奏,下儀仗、鹵簿使加點閱,使之齊肅。」



 皇祐二年,將享明堂,鹵簿使奏:「法駕減大駕三分之一,而兵部亡字圖故本,且文牘散逸,雖粗有名數,較之禮令,未有以裁其中。」詔禮官與兵部加考正,為圖以奏。及上圖,法駕鹵簿用萬有一千八十八人。嘉祐二年祫享,用禮儀使奏:「南郊仗,金吾上將軍、六統軍、左右千牛,皆服紫繡戎服,珂佩,騎而前;節度使亦衣褲褶導駕,
 如舊例。」是月,禮官奏:「南郊還,在禮當乘金輅,而或詔乘大輦,宜著於令,常以大輦從。」六年,幸睦親宅,內侍抱駕頭墮馬,駕頭壞。御史中丞韓絳奏請嚴儀衛,事下閣門、太常禮院議。遂合奏:「車駕出,請以閣門祗候及內侍各二員,扶駕頭左右,次扇筤,又以皇城親從兵二十人從其後。」



 神宗熙寧七年,詔太常看詳兵部大駕鹵簿字圖,遂奏言:「制器尚象,有其數者,必有其義。後世車駕儀仗,多雜秦、漢制度,當革其尤者。《周禮·車僕》:「凡師,共革車,
 各以其萃。」萃,副車也。諸輅之副,宜次正輅。羊車,前代宮中所乘;五牛旗,蓋古之五時副車也,以木牛載旗,用人輿之,失其本制:宜除去。」從之。



 元豐元年,詳定所言:「大駕輿輦、仗衛儀物,兼取歷代所用,其間情文訛舛甚眾。或規摹茍簡而因循已久;或事出一時而不足為法。」詔令更定。於是請去二十八宿、五星、攝提旗所繪人形,及龍、虎、仙童、大神、金鸚鵡、黃鸚鵡、網子、螣蛇、神龜等旗。舊制,親祠南郊,皇帝自大次至版位,內侍二人執翟羽前導,號
 曰「拂翟」。拂翟不出禮典,乃漢乾祐中宮中導從之物,不宜用諸郊廟。詔可。



 又禮文所言:



 近制,金輅不以金飾諸末,像輅不以象飾諸末,革輅不鞔,木輅不漆,請改飾四輅。太常則繪三辰,加升龍、降龍,大旗則繪交龍、大赤鳥隼、大白熊虎、大麾龜蛇而去其雲龍,使之應禮。又古者,五輅皆載旗,謂之「道德之車。」《考工記》車戟崇於殳,酋矛崇於戟,各四赤,戟矛皆插車騎,謂之「兵車」。戰國尚武,故增插四戟,謂之闟戟」。則知德車、武車,固異用矣。漢鹵簿,
 前驅有鳳凰闟戟,猶未施於五輅。江左以來,五輅乃加棨戟於車之右,韜以黼繡之衣。後周司輅,左建旗,右建闟戟,闟戟方六尺,而被之以黼,皆戾於古。請去五輅闟戟,以應「道德」之稱,而建太常於車後之中央,升輅則由左。



 又按《周禮》:「大馭,掌馭玉輅以祀。」則祀乘玉輅也。齋僕掌馭金輅,齋右充金輅之右,則齋乘金輅也。齋祀之車,異用而不相因。國朝親祠太廟,致齋文德殿,翌日即進玉輅,非制。請進金輅,俟太廟祠畢,翌日,御玉輅詣郊。



 又《
 周禮》戎右職曰:「會同,充革車。」《儀禮》曰:「貳車畢乘。」《禮記》曰:「乘君之乘車,不敢曠左,左必式。」蓋古者後車餘輅,不敢曠空,必使人乘之,所以別曠左之嫌也。自秦兼九國車服,西漢因之,大駕屬車八十一乘。《後漢志》云:「尚書、御史所載。」揚雄曰:「鴟夷國器,托於屬車。」則是漢之屬車,非獨載人,又以載物,亦《儀禮》所謂「畢乘」之義也。國朝鹵簿,車十二乘,虛設於法駕之後,實近曠左之嫌。請令尚書、御史乘之,或以載乘輿服御。



 又言:「法駕之行,必有共輿者,
 蓋以承清問。《周官》有太僕、齋僕、道僕,所以御車,至參乘,則其禮益重。故道德之車則有齋右、道右,武車則有戎右,皆以士大夫為之。國朝之制,乘輿有太僕而無參乘,請增近臣一員,立車右。」



 其後,詔增制五輅及參乘,玉輅建太常,金輅建大旗,像輅建大赤,革輅建大白,木輅建大麾。諸輅之副,各次正輅,仍存闟戟焉。時大駕鹵簿,仗下官一百四十六員,執仗、押引從軍員、職掌諸軍諸司二萬二千二百二十一人。初,玉輅自唐顯慶中傳之,號「
 顯慶輅」。神宗更制新玉輅,六年正月,御大慶殿受朝,先夕陳諸庭,夜半徹幕屋,壓焉。自是竟乘舊輅。



 徽宗建中靖國元年,太常寺狀具南郊儀仗,人兵二萬一千五百七十五人。政和四年,禮制局言:「鹵簿六引儀仗,信幡承以雙龍,大角黑漆畫龍,紫繡龍袋,長鳴、次鳴、大小橫吹、五色衣幡、緋掌畫交龍。按《樂令》,三品以上緋掌畫蹲豹。蓋唯乘輿器用,並飾以龍。今六引內系群臣鹵簿,而旗物通畫交龍,非便,合厘正。」七年,兵部尚書蔣猷請令有
 司取《天聖鹵簿圖記》,更加考正可否而因革之。詔如其請。宣和元年,蔡攸被旨改修,凡人物器服,盡從古制,飾以丹採,三十有三卷。



 高宗初至南京,孟太后以乘輿服御及御輦儀仗來進。建炎初,詔東京所屬起發祭器、法服、儀仗赴行在所。十一月,帝郊於揚州,儀仗用一千三百五十五人。倉卒渡江,皆為金兵所焚。紹興十二年,有司言:「天子起居,當備法駕,況太母回鑾,將奉郊迎。」遂令工部尚書莫將等
 檢會本朝文德、大慶殿舊儀,下太常定,用二千二百六十五人,於是始備黃麾仗,慶、冊、親饗皆用焉。是年冬,玉輅成。



 明年,郊,準國初大駕之數,一萬一千二百二十二人。內舊用錦襖子者以纈繒代,用銅革帶者以勒帛代。而指揮使、都頭仍舊用錦帽子、錦臂袖者,以方勝練鵲羅代;用絁者以綢代。禁衛班直服色,用錦繡、金銀、真珠、北珠者七百八十人,以頭帽、銀帶、纈羅衫代。旗物用繡者,以錯採代;車路院香鐙案、衣褥、睥睨,御輦院華蓋、曲蓋
 及仗內幢角等袋用繡者,以生色代。殿前司仗內金槍、銀槍、旗干,易以漆飾;而拂扇、坐褥以珠飾者去之。帝曰:「事天貴質,若惟事華麗,非初意矣。」十月,鹵簿器物及金象革木四輅、大安輦皆成。太常又奏,前後六引鼓吹八百八十四人,舊制騎。今路狹擁遏,欲止令步導。從之。十六年,始增捧日、奉宸隊,合一萬五千五十人。鹵簿之制備矣。三十一年九月,行明堂禮,儀物視郊祀省三之一,用一萬一千五人。



 孝宗隆興二年正月,以鹵簿勞民,乃
 令有司條具其可省者。次年郊祀,止用六千八百八十九人,蓋減紹興二十八年人數之半也。乾道六年之郊,雖仍備五輅、大安輦、六象,而人數則如舊焉。自後,終宋之世,雖微有因革,大抵皆如乾道六年之制。若明堂,則四輅、大安輦皆省,止用三千三百十九人。故事,祀前二日詣景靈宮,皆備大駕儀仗、乘輅。中興後,以行都與東都不同,前二日止乘輦。次日,自太廟詣青城,始登輅,設鹵簿。自紹興十三年始也。車駕遇雨,玉輅施障,從駕臣
 僚賜雨具,中道遇晴則撤。郊壇遇雨,則就青城放御仗,逍遙子還宮,導駕官免步導。



 大駕鹵簿。像六,中道,分左右。次六引,中道。第一,開封令;第二,開封牧;駕從餘州縣出者,所在刺史、縣令導駕,準此。



 第三,太常卿;第四,司徒;第五,御史大夫;第六,兵部尚書。以上各用本品鹵簿。



 次纛十二。每纛一人持,一人托,四人扯,騎二人押。



 次犦槊騎八,押衙四人騎引。



 左右金吾上將軍四人,將軍四人,大將軍各一人,折沖都尉一人。大將軍、都尉並夾以犦槊二,每槊一人執,二人夾,纛槊皆中道。



 次清游隊。左右道。



 白澤旗
 二,一人執,二人引,二人夾,左右金吾折沖都尉各一人領。



 弩八,弓箭三十二,槊四十。次左右金吾十六騎,左右道,主帥各一人分領。



 弩八,弓箭十二,槊十二。次夾道佽飛,騎。左右金吾果毅都尉各二人分領。



 虞候佽飛四十八人,鐵甲佽飛二十四人。



 次前隊殳仗。左右道。



 左右領軍衛將軍各一人,犦槊四人,主帥四人,殳八十,叉八十;相間。



 左右武衛屯衛主帥各四人,殳各五十人,叉各五十人;左右驍衛主帥四人,殳四十,叉四十。次朱雀旗一,中道,一人持,二人引,二人夾。



 弩四,弓箭十六。次龍旗十二。中道,並一人執,二人
 引,二人護後;副竿二,皆騎,左右金吾果毅都尉各一人領。



 風伯、雨師旗各一,雷公、電母旗各一,木、火、土、金、水星旗各一,左、右攝提旗各一,北斗旗一。次指南、記里鼓、白鷺、鸞旗、崇德、皮軒車。



 左右金吾衛果毅都尉各一人,來往檢校。次引駕十二重。中道,並騎。



 弩八,弓箭八,槊八。



 次太常前部鼓吹。令二人,府史四人從。



 □鼓十二在左,主帥四人騎領。



 金鉦十二在右,主帥四人騎領。



 大鼓百二十,主帥二十人騎領。



 長鳴百二十,主帥六人騎領。



 鐃鼓十二,主帥四人騎領。



 歌二十四,拱宸管二十四,簫二十四,笳二十四,大橫吹百二十,主帥十人騎領。



 節鼓二,
 笛二十四,簫二十四,觱篥二十四,笳二十四,桃皮觱篥二十四;□鼓十二在左,主帥二人騎領。



 金鉦十二在右,主帥二人騎領。



 小鼓百二十,主帥十人騎領。



 中鳴百二十,主帥六人騎領。



 羽葆鼓十二,主帥四人騎領。



 歌二十四,拱宸管二十四,簫二十四,笳二十四。



 次司天監一人,騎,引相風、刻漏,中道。令史一人,排列官二人,騎從。相風烏輿一,匠人一。



 交龍鉦、鼓各一,司晨、典事各一人騎從。



 鐘樓、鼓樓各一,行漏輿一,漏刻生四人從。



 清道二人,十二神輿一。司天官一人押。



 次持鈒前隊。中道。



 左右武衛果毅都尉各一人分領,校尉二人。絳
 引幡一,金節十二,罕一在左,畢一在右,朱雀幢一,叉一。



 青龍、白虎幢各一,分左右。叉各一。



 導蓋一。叉一。



 稱長一人,鈒戟二百八十人,分左右;左右武衛將軍各一人,校尉四人,分左右。次殿中侍御史二人,黃麾一。騎二夾。



 次前部馬隊。左右隊。



 第一隊,角宿、亢宿、斗宿、牛宿旗各一,執次同龍墀旗,角、亢在左,斗、牛在右,餘隊同此。



 左右金吾衛折沖都尉各一人分領,弩十,弓箭二十,槊四十;並分左右,餘隊皆同。



 第二隊,氐宿、房宿、女宿、虛宿旗各一,左右領軍衛果毅都尉各三人分領;兼第三、第四隊。



 第
 三隊心宿、危宿旗各一;第四隊尾宿、室宿旗各一;第五隊箕宿、壁宿旗各一,左右領軍衛折沖都尉各一人分領;第六隊奎宿、井宿旗各一,左右屯衛折沖都尉各一人分領;第七隊婁宿、鬼宿旗各一,左右武衛果毅都尉各三人分領;兼第八、第九隊。



 第八隊胃宿、柳宿旗各一;第九隊昴宿、星宿旗各一;第十隊畢宿、張宿旗各一,左右驍衛折沖都尉各三人分領;兼第十一、十二隊。



 第十一隊觜宿、翼宿旗各一;第十二隊參宿、軫宿旗各一。



 次步甲前隊。左右道。



 犦
 槊四,左右領軍衛將軍各一人檢校。第一隊,鶡雞旗二,引、執同馬隊。



 左右領軍衛折沖都尉各一人分領,赤鍪甲、弓箭六十;第二隊,貔旗二,左右領軍衛果毅都尉各一人分領,赤鍪甲、刀盾六十;第三隊,玉馬旗二,左右領軍衛折沖都尉各一人分領,青鍪甲、弓箭六十;第四隊,三角獸旗二,左右領軍衛果毅都尉各一人分領,青鍪甲、刀盾六十;第五隊,黃鹿旗二,左右屯衛折沖都尉各一人分領,黑鍪甲、弓箭六十;第六隊,飛麟旗二,左右屯衛
 果毅都尉各一人分領,黑鍪甲、刀盾六十;第七隊,駃騠旗二,左右武衛折沖都尉各一人分領,白鍪甲、弓箭六十;第八隊,鸞旗二,左右武衛果毅都尉各一人分領,白鍪甲、刀盾六十;第九隊,麟旗二,左右驍衛折沖都尉各一人分領,黃鍪甲、弓箭六十;第十隊,馴象旗二,左右驍衛果毅都尉各一人分領,黃鍪甲、刀盾六十;第十一隊,玉兔旗二,左右衛折沖都尉各一人分領,黃鍪甲、弓箭六十;第十二隊,闢邪旗二,左右衛果毅都尉各一人分
 領,黃鍪甲、刀盾六十。



 次前部黃麾仗。左右道。



 絳引幡二十;第一部,左右領軍衛大將軍各一人檢校,兼檢校第二部。



 折沖都尉各一人分領,主帥二人。



 龍頭竿赤氅二十,揭鼓二,儀鍠五色幢二十,龍頭竿小孔雀氅二十,小戟二十,揭鼓二,龍頭竿五色鵝毛氅二十,弓箭二十,龍頭雞毛氅二十,朱縢盾二十,龍頭竿繡氅二十,弓箭二十,槊二十,揭鼓二,綠縢盾二十;第二部,左右領軍衛折沖都尉各一人分領;主帥及氅鍠等並同第一部,餘準此。



 第三部,左右屯衛大將軍各一
 人檢校,果毅都尉各一人分領;第四部,左右武衛大將軍各一人檢校,折沖都尉各一人分領;第五部,左右驍衛大將軍各一人檢校;兼檢校第六部,折沖都尉各一人分領。第六部,左右衛果毅都尉各一人分領。



 次六軍儀仗。中道,在殿中黃麾後。



 左右神武軍統軍各一人,本軍旗二,一人執,一人引,二人夾,都頭各一人騎押。



 吏兵、力士旗各五,白乾槍五十,柯舒十,鐙仗八,相間。



 排闌旗二十,掩尾天馬旗二。左右羽林軍、左右龍武軍,並同神武軍。惟羽林用赤豹、黃熊旗各五,龍武用龍君、虎君旗各五。



 次引駕旗十六,中道,
 執人同六軍旗。



 十二辰旗各一,天王旗四。排仗通直官二人騎領。



 次龍墀旗十三,中道,各一人執,二人引,二人夾,排仗將二人騎領。



 天下太平旗一,青龍、赤龍、黃龍、白龍、黑龍旗各一,金鸞、金鳳旗各一,獅子旗二,日旗、月旗各一,君王萬歲旗一。



 次御馬二十四匹,中道,並以天武官二人執轡。



 尚乘奉御二人從。次日月合璧旗一,次苣文旗二,次五星連珠旗一,次祥雲旗二。以上並一人執,二人引,二人夾,佩橫刀,執弓箭。



 次長壽幢一。次青龍、白虎旗各一。左右道。左右衛果毅都尉各一人分領七十騎,弩八,弓箭二十二,槊四
 十。



 次班劍儀刀隊。左右道。



 左右衛將軍各一人,親衛郎將各二人,班劍二百二十,為第一、第二行;勛衛郎將各二人,班劍二百二十,為第三、第四行;翊衛郎將各三人,儀刀三百七十八,為第五、第六、第七行;左右驍衛翊衛郎將各一人,儀刀一百三十四,為第八行;左右武衛翊衛郎將各一人,儀刀一百三十八,為第九行;左右屯衛翊衛郎將各一人,儀刀一百四十二,為第十行;左右領軍衛翊衛郎將各一人,儀刀一百四十六,為第十一行;左
 右金吾衛翊衛郎將各一人,儀刀一百五十,為第十二行。



 次五仗。左右道。



 左右衛供奉中郎將各二人,親勛翊衛各二十四人,左右衛郎將各一人,散手翊衛各三十人,左右驍衛郎將各一人,翊衛各二十八人。



 次左右驍衛、翊衛三隊。第一隊,花鳳旗二,大將軍各一人,弩十,弓箭二十,槊四十;第二隊,飛黃旗二,將軍各一人,弩、弓箭、槊同第一隊,下準此。



 第三隊,吉利旗二,郎將各一人。



 次金吾細仗。殿中傘扇,千牛。中道。



 青龍、白虎旗各一,一人執,三人引,騎二人押當。



 五嶽神
 旗各一,五方神旗各一,五方龍旗二十五,五方鳳旗二十五,四瀆神旗各一。各一人執,二人引,二人夾,四旗屬兵部,每行次五方鳳旗。



 援寶三十二人,香案一,符寶郎一人,寶案一,寶輿一。輿士十二人。



 碧襴二十四人,騎,內十四人,執儀刀。



 方傘二,雉扇四,四色官六人,押仗二人,金甲天武官二人,進馬四人,千牛將軍一人,千牛八人,中郎將二人,長史二人,引駕官四人,天武官三百人。次球仗供奉官一百人。



 次左右衛夾轂隊。左右道。



 第一、第四隊,朱鍪甲、刀盾各六十,折沖都尉各一
 人檢校;第二、第五隊,白鍪甲、刀盾各六十,果毅都尉各一人檢校;第三、第六隊,黑鍪甲、刀盾各六十,果毅都尉各一人檢校。



 次捧日、奉宸隊。左右道。



 捧日三十五隊,隊四十人,騎;奉宸二十五隊,隊四十人。並五重相間。



 次導駕官。中道。



 通事舍人八人,分左右;侍御史二人,分左右;御史中丞二人,分左右;正言二人,分左右;司諫二人,分左右;起居郎二人在左,起居舍人二人在右;諫議大夫四人,分左右;給事中四人在左,中書舍人六人在右;散騎四人,分左
 右;門下侍郎二人在左,中書侍郎二人在右;侍中二人在左,中書令二人在右。次鳴鞭二。



 中道。



 次宮苑馬二。中道。



 次殿中省仗。大傘二,方雉尾扇四,腰輿一,排列官一人騎領。



 小雉尾扇四,方雉尾扇十二,華蓋二,香鐙一。



 次誕馬二,玉輅。皇帝升輅,則太僕卿御,千牛大將軍二人夾輅,將軍二人陪乘。前有誕馬二,教馬官二人。



 次諸司隨駕供奉。次大輦,掌輦四人導,尚輦奉御二人騎從。



 殿中少監二人,騎。本省供奉二人騎從。



 次御馬二十四。並以天武官二人執轡,尚輦直長二人騎從。



 次持鈒後隊。中道。



 左右武衛旅帥各一人,大傘二,大雉尾扇二夾。大雉尾
 扇四,小雉尾扇十二,朱團扇十二,華蓋二,叉二。



 睥睨十二,御刀六,玄武幢一,叉一。



 絳麾二,細槊十二。次大角百二十。左右金吾果毅都尉各一人騎從。



 次後部鼓吹。中道。



 鼓吹丞二人,騎。典事四人騎從。



 羽葆鼓十二,主帥四人騎從。



 歌二十四,拱宸管二十四,簫二十四,笳二十四;主帥二人騎領。



 鐃鼓十二,主帥四人騎領。



 歌二十四,簫二十四,笳二十四;小橫吹百二十,主帥八人騎領。



 笛二十四,簫二十四,觱篥二十四,笳二十四,桃皮觱篥二十四。



 次黃麾幡二,騎二夾。



 殿中侍御史二人,騎。令史四人騎從。



 次芳亭輦一,
 鳳輦一,小輿一,尚輦直長二人,騎,檢校。書令史四人騎從。



 次五牛旗輿各一,左右屯衛隊正各一人,騎,檢校。並執銀裝長刀。



 次乘黃令、丞二人。府史四人騎從。



 次金、象、革、木輅。次五副輅。次耕根車。次進賢、明遠、羊車。次屬車十二。次中書、門下、秘書、殿中省局官各一,騎。次黃鉞、豹尾車。



 次後部黃麾仗。左右道,與殿中黃麾相並。



 第一部,左右驍衛將軍各一人檢校,折沖都尉各一人分領;主帥氅鍠等並同前部,下皆準此。



 第二部,左右武衛將軍各一人檢校,折沖都尉各一人分領;第三部,左右屯
 衛將軍各一人檢校,折沖都尉各一人分領;第四部,左右領軍衛折沖都尉各一人分領;第五部,左右驍衛折沖都尉各一人分領;第六部,左右驍衛折沖都尉各一人分領,絳引幡二十,護後主帥二十人。



 次步甲後隊。左右道。



 第一隊,貔旗二,執、引並同前。



 左右衛果毅都尉各一人分領;鍪甲、弓盾同前隊第十二。



 第二隊,鶡雞旗二,左右衛折沖都尉各一人分領;鍪甲、弓箭同前隊第十一。



 第三隊,仙鹿旗二,左右驍衛果毅都尉各一人分領;鍪甲、刀盾同前隊第十。



 第四隊,金鸚鵡旗二,
 左右驍衛折沖都尉各一人分領;鍪甲、弓箭同前隊第九。



 第五隊,瑞麥旗二,左右武衛果毅都尉各一人分領;鍪甲、刀盾同前隊第八。



 第六隊,孔雀旗二,左右武衛折沖都尉各一人分領;鍪甲、弓箭同前隊第七。



 第七隊,野馬旗二,左右屯衛果毅都尉各一人分領;鍪甲、刀盾同前隊第六。



 第八隊,犛牛旗二,左右屯衛折沖都尉各一人分領;鍪甲、弓箭同前隊第五。



 第九隊,甘露旗二,左右領軍衛果毅都尉各一人分領;鍪甲、刀盾同前隊第四。



 第十隊,網子旗二,左右領軍衛折沖都尉各一人分領;鍪甲、弓箭同
 前隊第三。



 第十一隊,鶡雞旗二,左右領軍衛果毅都尉各一人分領;鍪甲、刀盾同前隊第二。



 第十二隊,貔旗二,左右領軍衛折沖都尉各一人分領。鍪甲、弓箭同前隊第一。



 次後部馬隊。左右道。



 第一隊,角端旗二,左右衛折沖都尉各三人分領;兼第二、第三隊。每隊弩、弓箭、槊並同前隊。



 第二隊,赤熊旗二;第三隊,兕旗二,左右驍衛果毅都尉各三人分領;兼第四隊。



 第四隊,太常旗二;第五隊,馴象旗二,左右武衛折沖都尉各三人分領;兼第六、第七隊。



 第六隊,鵔鸃旗二;第七隊,轆□蜀旗二;第八隊,騶牙旗二,
 左右屯衛果毅都尉各二人分領;第九隊,蒼烏旗二;第十隊,白狼旗二;第十一隊,龍馬旗二,左右領軍折沖都尉各二人分領;第十二隊,金牛旗二。



 次後隊殳仗。左右道。



 左右領軍衛主帥四人,殳八千,叉八十;左右武衛主帥四人,殳五十,叉五十;左右屯衛驍衛主帥各四人,殳四十,叉四十。次掩後隊。中道。



 左右屯衛折沖都尉各一人,大戟五十,刀盾五十,弓箭五十,槊五十。



 次真武隊。中道。



 金吾折沖都尉一人,仙童、螣蛇、真武、神龜旗各一,十人執,二人引,二人
 夾。



 槊二十五,弓箭二十,弩五。



 車駕至青城,則周衛行宮及壇內外。其青城坐甲布列三百三十六鋪:殿前指揮使二十四鋪,四百七十七人;內殿直一十鋪,一百四十一人;散員一十輔,一百四十二人;散指揮一十鋪,一百四十一人;散都頭一十鋪,一百四十三人;散祗候一十鋪,一百四十人;金槍一十鋪,一百五十人;銀槍一十鋪,一百五十人;東第一班三鋪,五十二人;東第二班三鋪,五十三人;東第三班六鋪,九
 十一人;東第四班五鋪,八十四人;東第五班二鋪,二十二人;下茶酒班一鋪,三十一人;散直一十鋪,一百四十九人;鈞容直一十鋪,二百人;御龍直二十二鋪,三百八十五人;御龍骨朵子直一十二鋪,二百一十二人;御龍弓箭直一十八鋪,二百九十六人;御龍弩直二十二鋪,三百五十六人;把天門天武一鋪,八人;駕頭扇筤天武一鋪,三十二人;禁衛天武六鋪,三百一十人;約攔天武三十鋪,三百一十人;方圍子親從三十四鋪,三百六十
 人;禁衛崇政殿親從四十鋪,並提舉人員共四百六十三人;行宮司親從一十二鋪,一百八十人;快行親從四鋪,八十六人。行宮殿門崇政殿親從四十六人,行宮殿門親從並提舉人員二百四十人,把街約攔親事官貼諸處齪門一十隊及提舉人員一百三人,殿前指揮使已下看守馬火甲隊一千一百七十一人,右禁衛諸班共六千七百二十有四人。



 圜壇東門外中道夾立諸班直主首引駕人員九人,御龍四直門旗六十人,御龍仗
 劍六人,天武把門長行八人。



 大次前外圍親從四隊三十八人,執燭親從八十六人,行宮殿門一十二人,御龍直四十人。大次後把街約攔執事官五十一人。大次兩壁快行六十九人,於禁衛外排立壇周圍,守踏道。裏圍親從十將、節級二十二人,壇從裏第二重方圍親從三百二十四人。大次及外壝外諸門行宮司共一百六十人,宮架及壇東幄幕、宰臣百官幕次共六十人。右自大次前外圍至百官幕次,共八百六十二人。凡詣小次行
 禮,不須隨從。大次前裏圍並攔前一百七十一人,執燭一百二十九人,外圍一百八十人,行宮門及快行二十四人。



 右自裏圍至行宮快行共五百四人。



 凡詣小次行禮,隨從祗應。



 圓壇從外壝下分作九重:從中第一重,殿前指揮使等七百四十四人;第二重,御龍直等六百九十五人;第三重,散員等六百四十二人;第四重,散都頭等七百一十人;第五重,天武骨朵大劍約攔五百八十一人;第六重,御營四面巡檢下步軍八百六十七人;第七重,御營四面並
 青城圓壇巡檢下步軍八百六十七人;第八重,御營四面巡檢下馬軍四百三十三人;第九重,御營四面巡檢及青城圓壇巡檢下馬軍四百三十四人。壇四門殿前指揮使行門三十五人,內人員一十五人,壇東門夾立擎鞭長行一十人。



 右自青城赴壇諸班親從文武及御營圓壇巡檢下,總七千四百六十七人。



 駕至太廟,環衛如郊壇,坐甲布列二百六十三鋪。殿前指揮使二十四鋪,四百七十七人;內殿直、散員、散指揮、散都頭、散祗候、散直各一十鋪,一百二十人,共六十
 鋪七百二十人;金槍一十鋪,一百五十人;銀槍一十鋪,一百五十人;東第一、第二班各二鋪,三十人,共四鋪,六十人;東第三、第四班各四鋪,六十人,共八鋪,一百二十人;東第五班二鋪,二十二人;下茶酒班一鋪,三十一人;御龍直八鋪,三百八十五人;御龍骨朵子直四鋪,二百一十二人,御龍弓箭直六鋪,二百九十六人;御龍弩直八鋪,三百五十六人;把行門天武一鋪,八人;駕頭扇筤天武一鋪,三十二人;禁衛天武六鋪,三百一十人;禁
 衛崇政殿親從四十鋪,並提舉人員共四百六十三人;行宮司親從一十二鋪,一百八十人;快行親從四鋪,八十六人;方圍親從二十四鋪,三百六十人;約攔天武三十鋪,三百一十人。



 行宮殿門崇政殿親從及提舉人員二百八十六人,把街約攔親事官貼諸處齪門一十二隊,並提舉人員一百三人,御營四面巡檢六員下步軍九百一十八人,親從四十人。青城內至圜壇巡檢下親從四十人。右禁衛諸班直等御營四面巡檢軍兵,及青城
 至圜壇巡檢下親從,總六千一百四十五人。



 左山商氏家藏宋人《青城》、《圜壇》、《太廟》三圖,其布置行列,極為詳備,因附鹵簿之後,庶覽之者可以考一代之制云。



 凡鹵簿內牙門旗,中道四,分二門;左右道各十,分五門。中道一門在金吾細仗前,一門在掩後隊後。左右廂第一門在步甲前隊第六後,第二門在前部黃麾仗前,第三門在後部黃麾仗前,第四門在黃麾仗後,第五門在步甲後隊第六后。每旗二人執,四人夾,並騎,分左右。每門監門校尉六人領。



 又大駕,郊祀、籍田、薦獻玉清昭應
 景靈宮用之。迎奉聖像亦用大駕,惟不設象及六引導駕官。法駕,減太常卿、司徒、兵部尚書,白鷺、崇德車,大輦、五副輅,進賢、明遠車,又減屬車四,餘並三分減一。泰山下、汾陰行禮,明堂、大慶殿恭謝用之,凡一萬一千八十八人。鸞駕,又減縣令、州牧、御史大夫,指南、記里、鸞旗、皮軒車,像輅、革略、木輅,耕根車、羊車、黃鉞車、豹尾車、屬車,小輦、小輿,餘並減半。朝陵,迎泰山天書,東封、西祀,朝謁太清宮,奏告玉清昭應宮,奉迎刻玉天書,躬謝太廟,皆
 用之。鸞駕舊用二千人,大中祥符五年,真宗告太廟,增至七千人。兵部黃麾仗,用太常鼓吹,太僕寺金玉輅,殿中省大輦,其制無定,然皆減於小駕。御樓、車駕親征或省方還京,迎禁中天書,五嶽上冊,建安軍迎奉聖像,太廟上冊皆用之。



\end{pinyinscope}