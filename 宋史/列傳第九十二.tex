\article{列傳第九十二}

\begin{pinyinscope}

 楊
 佐李兌從弟先沈立張掞張燾俞充劉瑾閻詢葛宮從子思書張田榮李載姚渙朱景子光庭李琮朱壽隆盧士宏單煦楊仲元餘良肱
 潘夙



 楊佐,字公儀,本唐靖恭諸楊後,至佐,家於宣。及進士第,為陵州推官。州有鹽井深五十丈,皆石也,底用柏木為乾,上出井口,垂綆而下,方能及水。歲久乾摧敗,欲易之,而陰氣騰上,入者輒死;惟天有雨,則氣隨以下,稍能施工,晴則亟止。佐教工人以木盤貯水,穴竅灑之,如雨滴然,謂之「雨盤」。如是累月,井幹一新,利復其舊。



 累遷河陰發運判官,乾當河渠司。皇祐中,汴水殺溢不常,漕舟不
 能屬。佐度地鑿瀆以通河流,於是置都水監,命佐以鹽鐵判官同判。京城地勢南下,涉夏秋則苦霖潦,佐開永通河,疏溝澮出野外,自是水患息。又議治孟陽河,議者謂不便。佐言:「國初歲轉京東粟數十萬,今所致亡幾,儻不浚復舊跡,後將廢矣。」乃從其策。



 出為江、淮發運使。孟陽之役,調民七、八千,夷丘墓百數,怨聲盈塞。詔開封鞫治,官吏獨舍佐不問。糾察刑獄劉敞請加貶黜,不聽。召為鹽鐵副使,拜天章閣待制,復判都水,知審官院,權發
 遣開封府。



 嘗使契丹,虜饋以方物,書獨稱名。英宗升遐,奉遺留物再往使,卒於道,年六十一。詔護喪歸,賻以黃金,恤其家。



 李兌,字子西,許州臨穎人。登進士第,由屯田員外郎為殿中侍御史。按齊州叛卒,獄成,有欲夜篡囚者,兌以便宜斬之,人服其略。



 張堯佐判河陽,兌言堯佐素無行能,不宜以戚里故用。改同知諫院。狄青宣撫廣西,入內都知任守忠為副,兌言以宦者觀軍容,致主將掣肘,非計。
 仁宗為罷守忠。太常新樂成,王拱辰以為十二鐘磬一以黃鐘為律,與古異,胡瑗及阮逸亦言聲不能諧。詔近臣集議,久而不決。兌言:「樂之道,廣大微妙,非知音入神,詎容輕議。願參新舊,但取諧和近雅者,合而用之。」進侍御史知雜事,擢天章閣待制、知諫院。轉運使制祿與郡守殊,時有用彈劾奪節及老疾請郡者,一切得仍奉稍。兌言非所以勸沮,乃詔悉依所居官格。兌在言職十年,凡所論諫,不自表襮,故鮮傳世。



 出知杭州,帝書「安民」二
 字以寵。徙越州,加龍圖閣直學士、知廣州,南人謂自劉氏納土後,獨兌著清節。還知河陽,帝又寵以詩。徙鄧州。富人榜僕死,系頸投井中而以縊為解。兌曰:「既赴井,復自縊,有是理乎?必吏受賕教之爾。」訊之果然。



 兌歷守名郡,為政簡嚴,老益精明。自鄧歸,泊然無仕宦意。對便殿,力丐退,英宗命無拜,以為集賢院學士、判西京御史臺。積官尚書右丞,轉工部尚書致仕。卒,年七十六,謚曰莊。從弟先。



 先字淵宗,起進士,為虔州觀察推官,攝吉州永新令。兩州俗尚訟,先為辨枉直,皆得其平。



 知信州、南安軍,撫楚州,歷利、梓、江東、淮南轉運使。壽春民陳氏施僧田,其後貧弱,往丐食僧所而僧逐之,取僧園中筍,遂執以為盜。先詰其由,奪田之半以還之。所至治官如家,人目以俚語:在信為「錯安頭」,謂其無貌而有材也;在楚為「照天燭」,稱其明也。楚有民迫於輸賦,殺牛鬻之。里胥白於官,先愍焉,但令與杖。通判孫龍舒以為徒刑,毀其桉。明日龍
 舒來,先引囚曰:「汝罪應杖,以通判貸汝矣。」遣之出。



 積官至秘書監致仕。兄兌尚無恙,事之彌篤。以子敘封,得太中大夫,閑居一紀卒,年八十三。子庭玉,年六十即棄官歸養。人賢其家法云。



 沉立,字立之,歷陽人。舉進士,簽書益州判官,提舉商胡埽。採摭大河事跡、古今利病,為書曰《河防通議》,治河者悉守為法。遷兩浙轉運使。蘇、湖水,民艱食,縣戒強豪民發粟以振,立亟命還之,而勸使自稱貸,須歲稔,官為責
 償。茶禁害民,山場、榷場多在部內,歲抵罪者輒數萬,而官僅得錢四萬。立著《茶法要覽》,乞行通商法,三司使張方平上其議。後罷榷法,如所請;立召為戶部判官。



 奉使契丹,適行冊禮,欲令從其國服,不則見於門。立折之曰:「往年北使講見儀,未嘗令北使易冠服,況門見邪?」契丹愧而止。



 遷京西北轉運使。都水方興六塔河,召與議,立請止修五股等河及漳河,分殺水勢以省役,從之。加集賢修撰、知滄州,進右諫議大夫、判都水監,出為江、淮發
 運使。居職辦治,加賜金,數詔嘉之。知越州、杭州、審官西院、江寧府。



 初,立在蜀,悉以公粟售書,積卷數萬。神宗問所藏,立上其目及所著《名山水記》三百卷。徙宣州,提舉崇禧觀。卒,年七十二。



 張掞,字文裕,齊州歷城人。父蘊,咸平初,監淄州兵。契丹入寇,游騎至淄、青間,州人將棄城,蘊拔刀遮止於門,力治守備,游騎為之引去。郡守愧,始謀掠為己功,反陷以罪,蘊受而不校。



 掞幼篤孝,蘊病,刲股肉以療。舉進士,知
 益都縣。當督賦租,置里胥弗用,而民皆以時入。石介獻《息民論》,請以益都為天下法。丁內艱,時隆寒,徒跣舉柩,叩首流血,與兄揆廬墓左。



 明道中,京東饑,盜起,以御史中丞範諷薦,知萊州掖縣。民訴旱於州,拒之,掞自薦奏聞,詔除登、萊稅。通判永興軍,為集賢校理,四遷為龍圖閣直學士、知成德軍。宦者閻士良為鈐轄,多撓帥權,用危法中軍校,掞直之,而劾士良。英宗登極,朝廷使來告,士良辭疾居家,宴客自若,奏抵其罪。入判太常、司農寺,
 累官戶部侍郎致仕。熙寧七年,卒,年八十。



 掞忠篤誠愨,既老益康寧。少從劉潛、李冠游,及其死,率里人葬之,置田贍其孥。事揆如父,理家必諮而行,為鄉黨矜式。



 張燾,字景元,樞密直學士奎之子也。舉進士,通判單州。州卒謀亂,期有日,燾得告者,徐詣營取首惡,置諸法。知沂、濰二州。沂產布,濰產絹,而有司科賦相反,燾始革之。濰多圭田,率計畝征絹,而蠲河役,燾不肯踵例,廢法還其役,入損於舊五之四,且命吏曰:「吾知守己而已,無妨
 後人,汝勿著為式。」



 提點河北刑獄,攝領澶州,七日而商胡決。燾拯溺救饑,所全活者十餘萬,猶坐免。數年,復提點河東、陜西、京西刑獄,為鹽鐵判官、淮南轉運使、江淮發運副使。泗州水,城且壞,燾悉力營護,詔寵其勞。入為戶部副使。京師賦曲於酒,人有常籍,毋問售不售,或蹶產以償。燾請罷歲額,嚴禁令,隨所用曲多寡以售,自是課增溢。官修睦親宅,議取民居,燾言:「芳林園有餘地,宗室足自處,無庸起民居。」從之。孝嚴殿成,請圖乾興以來
 文武大臣像於壁。



 遷天章閣待制、陜西都轉運使。蒲津浮橋壞,鐵牛皆沒水中,燾以策列巨木於岸以為衡,縋石其秒,挽出之,橋復其初。保安二土豪善騎射,為邊人所憚,故縱善馬誘使取之,而強以漢法。燾按得其狀,俱以隸軍。加龍圖閣直學士、知成都府。蜀人苦多盜,燾嚴保伍,使不得隱,而申其捕限。南蠻寇黎、雅,討走之,罷磨刀崖戍卒。改知瀛州。



 母喪服闋。故事,起執政以詔,近臣以堂帖;神宗特命賜詔。判太常寺,知鄧、許二州,復判太
 常,知通進、銀臺司,提舉崇福宮,由給事中易通議大夫。卒,年七十。



 燾才智敏給,常從範仲淹使河東。至汾州,民遮道數百趨訴,仲淹以付。燾方與客弈,局未終,處決已竟。英宗時,三司前奏事,帝詰鑄錢本末,皆不能對,燾悉論無隱。帝是之,顧左右識其姓名,後欲以為觀察使守邊,曰:「卿家世事也。」燾對曰:「臣叔父亢有大才,臣愚不可繼。」遂止。



 俞充字公達,明州鄞人。登進士第。熙寧中為都水丞,提
 舉沿汴淤泥溉田,為上腴者八萬頃。檢正中書戶房,加集賢校理、淮南轉運副使,遷成都路轉運使。茂州羌寇邊,充上十策御戎。神宗遣內侍王中正同經制,建三堡,復永康為軍,因詐殺羌眾以為中正功,與深相結,至出妻拜之。中正還闕,舉充可任。召判都水監,進直史館。中書都檢正御史彭汝礪論其媚事中正,命遂寢。



 河決曹村,充往救護,還,陳河防十餘事,概論「水衡之政不修,因循茍且,浸以成習。方曹村決時,兵之在役者僅十餘人,
 有司自取敗事,恐未可以罪歲也。」加集賢殿修撰、提舉市易,歲登課百四十萬。故事當賜錢,充曰:「奏課,職也,願自今罷賜。」詔聽之。



 擢天章閣待制、知慶州。慶陽兵驕,小繩治輒肆悖,充嚴約束,斬妄言者五人於軍門。聞有病苦則巡撫勞餉,死不能舉者出私財以周其喪,以故莫不畏威而懷惠。環州田與夏境犬牙交錯,每獲必遭掠,多棄弗理,充檄所部復以時耕植。慕家族山夷叛,舉戶亡入西者且三百,充遣將張守約耀兵塞上,夏人亟反
 之。



 充之帥邊,實王珪薦,欲以遏司馬光之入。充亦知帝有用兵意,屢倡請西征,後言:「夏酋秉常為母梁所戕,或云雖存而囚,不得與國政。其母宣淫兇恣,國人怨嗟,實為興師問罪之秋也。秉常亡,將有桀黠者起,必為吾患。今師出有名,天亡其國,度如破竹之易。願得乘傳入覲,面陳攻討之略。」詔令掾屬入議,未及行,充暴卒,年四十九。



 劉瑾,字符忠,吉州人,沆之子也。第進士,為館閣校勘。沆
 亡,得褒贈。知制誥張瑰草詞,語涉譏貶,瑾泣涕不能食,闔門衰絰,邀宰相自言。朝廷為改書命,黜瑰為州,瑾亦坐衰服入公門罷職。沒喪不就官,丐守墳墓。王素為請,以伸孝子之志。詔復職,遷集賢校理、通判睦州,為淮南轉運副使。召修起居注,加史館修撰、河北轉運使,拜天章閣待制、知瀛州。坐與世居通問,徙明州。未行,改鎮廣州。與樞密院論戍兵不合,改虔州。戰棹都監楊從先奉旨募兵不至,擅遣其子懋糾諸縣巡檢兵集郡下,瑾怒責
 之,遽發悖謬語,懋訴瑾於朝,遂廢於家。逾年,復待制、知江州,歷福州、秦州、成德軍,卒。



 瑾素有操尚,所蒞以能稱,然御下苛嚴,少縱舍,好面折人短,以故多致訾怨。



 閻詢,字議道,鳳翔天興人。少時以學問著聞,擢進士第,又中書判拔萃科。累遷秘書丞,為監察御史裏行。詔治王素獄,坐有姻嫌不以聞,降監河陽酒稅,累遷為鹽鐵判官。使契丹。詢頗諳北方疆理,時契丹在靴澱,迓者王惠導詢由松亭往,詢曰:「此松亭路也,胡不徑蔥嶺而迂
 枉若是,豈非誇大國地廣以相欺邪?」惠慚不能對。加直龍圖閣、知梓州。徙河東轉運使,言:三路土兵疲老者,聽其族以強壯者代。」從之。進集賢殿修撰、知河中府。大河漲,壞浮橋,詢易為長橋。拜天章閣待制、知廣州,不即赴,罷職知商州。神宗轉右諫議大夫,改邠、同二州,提舉上清太平宮,卒,年七十九。



 葛宮,字公雅,江陰人。舉進士,授忠正軍掌書記。善屬文,上《太平雅頌》十篇,真宗嘉之,召試學士院,進兩階。又獻《
 寶符閣頌》,為楊億所稱。知南充縣,東川饑,民艱食,部使者檄守資、昌兩州,以惠政聞。知南劍州。土豪彭孫聚黨數百,憑依山澤為盜,出害吏民,不可捕,宮遣沙縣尉許抗諭降之。並溪山多產銅、銀,吏挾奸罔利,課歲不登,宮一變其法,歲羨餘六百萬。三司使聞於朝,論當賞。宮曰:「天地所產,吾顧盜之,又可為功乎?」卒不言。



 徙知滁、秀二州,秀介江湖間,吏為關涇瀆上,以征往來,間有昏葬,趨期者多不克,宮命悉毀之。積官秘書監、太子賓客。治平
 中,轉工部侍郎。熙寧五年,卒,年八十一。宮性敦厚,恤錄宗黨,撫孤嫠,賴以存者甚眾。



 宮弟密,亦以進士為光州推官。豪民李新殺人,嫁其罪於邑民葛華,且用華之子為證。獄具,密得其情,出之。法當賞,密白州使勿言。仕至太常博士。天性恬靖,年五十,忽上章致仕,姻黨交止之,笑曰:「俟罪疾、老死不已而休官者,安得有餘裕哉。」即退居,號草堂逸老,年八十四乃終。平生為詩慕李商隱,有西昆高致。



 子書思,踵登第,調建德主簿。時密已老,欲迎
 以之官,密難之。書思曰:「曾子不肯一日去親側,豈以五斗移素志哉?」遂投劾歸養十年餘。近臣表其志行,以為泗州教授,弗就。密不得已,許以他日偕行,始乞監新市鎮。居父喪,哀毀骨立,盛暑不釋苴麻,終禫不忍去塚舍。累年,乃出仕,歷封丘主簿、漣水縣丞。時兄書元為望江令,同隸淮南監司,有舍兄而薦己者,移書乞改薦兄,不許,則封檄還之。其篤行類皆若此。仕至朝奉郎,亦告老,父子歸休皆不待年。卒,年七十三,特謚曰清孝。子勝仲,孫立
 方,皆以學業至侍從,世為儒家。勝仲自有傳。



 論曰:佐、立擅水衡之政,為時所稱。兌居官論諫,無所表襮,先克承之。掞之孝,燾之智,瑾之苛嚴,詢之辭令,皆著一時,自致顯官。俞充制軍禁暴,足為能臣,而希時相之意,倡請西征,使其不死,邊陲之禍,其可既乎?葛氏自宮以下,簪纓相繼,盛哉。



 張田,字公載,澶淵人。登進士第,知應天府司錄。歐陽修薦其才,通判廣信軍。夏竦、楊懷敏建策增七郡塘水,詔
 通判集議,田曰:「此非禦敵策也,壞良田,浸塚墓,民被其患,不為便。」因奏疏極論,謫監郢州稅。



 久之,通判冀州。內侍張宗禮使經郡,酣酒自恣,守貳無敢白者,田發其事,詔配西陵灑掃。攝度支判官。祫享太廟,又請自執政下差減賚費,唐介論其虧損上恩,出知蘄州。俄提點湖南刑獄,介與司馬光又狀其傾險,改知湖州,徙廬州,治有善跡。



 移桂州。異時蠻使朝貢假道,與方伯抗禮,田獨坐堂上,使引入拜於庭,而犒賄加腆。土豪劉紀、廬豹素為
 邊患,訖田去,不敢肆。京師禁兵來戍,不習風土,往往病於瘴癘,田以兵法訓峒丁而奏罷戍。或告交址李日尊兵九萬,謀襲特磨道,諸將請益兵,田曰:「交址兵不滿三萬,必其國有故,張虛聲以□赫我耳。」諜既得實,果其兄弟內相殘,懼邊將乘之也。宜州人魏利安負罪亡命西南龍蕃,從其使入貢,凡十反。,至是龍以烈來,復從之。田因其入謁,詰責之,梟其首,欲並斬以烈,叩頭流血請命。田曰:「汝罪當死,然事幸在新天子即位赦前,汝自從朝廷
 乞恩。」乃密請貸其死。



 熙寧初,加直龍圖閣、知廣州。廣舊無外郭,民悉野處,田始築東城,環七里,賦功五十萬,兩旬而成。初,役人相驚以白虎夜出,田跡知其偽,召戒邏者曰:「今夕有白衣人出入林間者,謹捕之。」如言而獲。城既就,東南微陷,往視之,暴卒,年五十四。



 田為人伉直自喜,好嫚罵,氣陵其下,故死無哀者。然臨政以清,女弟聘馬軍帥王凱,欲售珠犀於廣,顧曰:「南海富諸物,但身為市舶使,不欲自污爾。」作欽賢堂,繪古昔清刺史像,日夕
 師拜之。蘇軾嘗讀其書,以侔古廉吏。



 榮諲,字仲思,濟州任城人。父宗範,知信州鉛山縣。詔罷縣募民採銅,民散為盜,宗範請復如故。真宗嘉異,擢提點江、浙諸路銀銅坑冶,歷官九年。



 諲舉進士,至鹽鐵判官。晉州產礬,京城大豪歲輸鐵五萬緡,顓其利,諲請榷於官,自是數入四倍。為廣東轉運使。廣有板步古河路絕險,林箐瘴毒。諲開真陽峽,至洸口古徑,作棧道七十間抵清遠,趨廣州,遂為夷塗。



 復入為開封府判官。太康
 民事浮屠法,相聚祈禳,號「白衣會」,縣捕數十人送府。尹賈黯疑為妖,請殺其為首者而流其餘,諲持不從,各具議上之。中書是諲議,但流其首而杖餘人。加直史館、知澶州。



 改京東轉運使。萊陽產銀砂,民有私採者,事露,安撫使欲論以劫盜。諲曰:「山澤之利,人得有之,所盜者豈民財耶?」貸免甚眾。又使成都府路,召為戶部副使,以集賢殿修撰知洪州。以疾故,徙舒州,未至而卒。累官秘書監,年六十五。



 李載,字伯熙,黎陽人。少苦學,隆暑讀書,置足於水,雖得疾,不舍去。登進士第,調冀州推官。知大名冠氏縣,府守呂夷簡入相,薦其材,知齊州。鈐轄趙瑜使酒毆載,乃扃戶避逸。瑜得罪,載坐不舉劾,黜為信陽軍。安撫使錢明逸等為之申理,改常州。知祥符縣,有巫以井泉飲人,云可愈疾,趨者旁午,載杖巫,堙其井。歷知虢州、漣水軍。



 載性篤孝,侍母病不解帶,至病亟不能食,載亦不食,母知之,為強食。六為州,一以寬厚稱。以光祿卿提舉仙源觀,
 卒,年七十四。



 姚渙字虛舟,世家長安。隋開皇中,有景徹者,以討平瀘夷,策功為普州刺史,卒,子孫遂家普州。渙第進士,監益州交子務,發奸隱萬緡,主吏皆當死,渙曰:「戮人以乾澤,非吾志也,義不蔽奸而已。」請於使者,願不受賞,於是全活者眾。知峽州。宜都民為盜所殘,縣執囚訊服,以獄上。渙移劾於他有司,居亡何,真盜獲。大江漲溢,渙前戒民徙儲積、遷高阜,及城沒,無溺者。因相地形築子城、埽臺,
 為木岸七十丈,繚以長堤,楗以薪石,厥後江漲不為害,民德之。徙知涪州,賓化夷多犯境,渙施恩信拊納,酋豪爭羅拜廷下,訖渙去無警。終光祿卿,年六十七。



 朱景,字伯晦,河南偃師人。舉進士,調滎澤簿。西方用兵,詔侍從館閣舉縣令,景預選,知隴州汧源縣。累遷知汝州。葉驛道遠,隸囚為送者所虐,多死,俗傳為「葉家關」,景重禁以絕其患。擢知壽州,秩祿視提點刑獄。始至,亟發廩振給,以勸富者出積穀,所活數萬。城西居民三千室,
 建請築外郭環入之,公私稱便。再遷光祿卿。



 熙寧初,病革,自占遺表,呼其子光庭操筆書之。其略云:「切聞河北水災、地震,陛下當減膳避殿,齋居加省,召二府大臣朝夕咨訪闕失,思所以弭咎。」凡數百言,無一語求恩。卒,年七十一。詔加賻贈,錄其子以官。



 光庭字公掞,十歲能屬文。辭父蔭擢第,調萬年主簿。數攝邑,人以「明鏡」稱。歷四縣令。曾孝寬以才薦,神宗召見,問欲再舉安南之師。光庭對曰「願陛下勿以人類畜之。
 蓋得其地不可居,得其民不可使,何益於廣土闢地也。」又問治何經,對曰:「少從孫復學《春秋》。」又問:「今中外有所聞乎?」對曰:「陛下更張法度,臣下奉行或非聖意,故有便有不便。誠能去其不便,則天下受福矣。」帝以其言為疏闊,不用。簽書河陽判官,從呂大防於長安幕府。五路出師討西夏,雍為都會,事倚以辦,調發期會甚急,光庭每執不從。使者怒,將加以乏興罪,光庭求免去,大防為之解。



 哲宗即位,司馬光薦為左正言,首乞罷提舉常平官、
 保甲青苗等法。論蔡確為山陵使,而乃先靈駕而行,為臣不恭。又言章惇欺罔肆辯,韓縝挾邪冒寵,言甚切。宣仁後嘉其守正,諭令盡言,毋有所畏避。遷左司諫,又論「蘇軾試館職發策云:『今欲師仁祖之忠厚,而患百官有司不舉其職,或至於偷;欲法神考之厲精,而恐監司、守令不識其意,流入於刻。』臣謂仁宗難名之盛德,神考有為之善志,而不當以『偷』、『刻』為議論,望正其罪,以戒人臣之不忠者。」未幾,中丞傅堯俞、侍御史王巖叟相繼論列。
 宣仁後曰:「詳覽文意,是指今日百官有司、監司守令言之,非所以諷祖宗也。」遂止。



 河北饑,遣持節行視,即發廩振民;而議者以耗先帝積年兵食之蓄,改左司員外郎。遷太常少卿,拜侍御史。論蔡確怨謗之罪,確貶新州。拜右諫議大夫、給事中。乞補外,除集賢殿修撰、知亳州。數月召還,復為給事中。



 坐封還劉摯免相制,復落職守亳。歲餘,徙潞州,加集賢院學士。鄰境旱饑,流民入境者踵接,光庭日為食以食之,常至暮,自不暇食,遂感疾,猶自
 力視事。出禱雨,拜不能興,再宿而卒,年五十八。紹聖中,追貶柳州別駕。元符初,又停錮其諸子。



 光庭始學於胡瑗,瑗告以為學之本在於忠信,故終身行之。徽宗立,復其官。



 李琮,字獻甫,江寧人。登進士第,調寧國軍推官。州庾積穀腐敗,轉運使移州散於民,俾至秋償新者。守將行之,琮曰:「穀不可食,強與民責而償之,將何以堪。」持不下,守愧謝而止。



 呂公著尹開封,薦知陽武縣。役法初行,琮處
 畫盡理,旁近民相率撾登聞鼓,願視以為則。徽宗召對,擢利州路、江東轉運判官。行部至宣城,按民田詭稱逃絕者九千戶,他縣皆然。言於朝,命以戶部判官使江、浙,選強明吏立賞剔抉。吏幸賞,以多為功,琮亦因是希進,民患苦之,得緡錢百餘萬。進度支判官,頒職式於諸道。淮南賦入甲它部,以為轉運副使,徙梓州路。



 元祐初,言者論其括隱稅之害,黜知吉州。御史呂陶又言巴蜀科折已重,琮復強民輸稅,且無得以奇數並合,人尤咨怨。
 於是凡以括田受賞者悉奪之。歷相、洪、潞三州。潞有謀亂者,為書期日揭道上,部使者聞之,懼,檄索奸甚亟。琮置不問,以是日置酒高會,訖無他。入為太府卿,遷戶部侍郎,以寶文閣待制知杭州、永興軍、河南、瀛州。卒,年七十五。



 琮長於吏治,而所至主於掊克,為士論嗤鄙。子回,紹興初參知政事。



 朱壽隆,字仲山,密州諸城人。以蔭知九隴縣。吏告民一家七人以火死,壽隆曰:「寧有盡室就焚無一脫者,殆必
 有奸。」逾月獲盜,果殺其人而縱火也。知宿州,宿多劇盜,至白晝被甲剽攻,郡縣不能制。壽隆設方略耳目,捕斬千餘人。



 擢提點廣西刑獄。嶺外新經儂寇,修營城障,貴州虐用其人,不能聊生。壽隆馳詣州,械守送獄,奏黜之。老稚婦女遭亂,流轉不能自還者,檄所在資送其還。舊制,溪蠻侵暴羈縻州,雖殺人無得仇報,壽隆請聽相償,蠻始畏戢。



 歷鹽鐵度支判官、夔路轉運使。巴峽地隘,民困於役,免其不應法者千五百人。復為鹽鐵判官、京東
 轉運使,賜三品服。歲惡民移,壽隆諭大姓富室畜為田僕,舉貸立息,官為置籍索之,貧富交利。以少府監知揚州,卒,年六十八。



 壽隆為人和厚,接談怡怡,必當於理,而不屈於權貴。狄青討賊,欲殺裨將不用命者數人,壽隆極論罪不當死。孫沔在坐,曰:「儂賊害民萬計,此何足惜。」壽隆曰:「王師之來以除民害,顧可效賊為暴邪?」青感其言而止。



 盧士宏,字子高,新鄭人。以父任屢更州縣,所至著清名。
 知信陽軍。官捕為妖術者,餘黨懼及,群聚山谷間,士宏請減其罪招之,即相帥歸命。徙知漢州,校實民產,使力役不濫,人德之。又知洋州。先是,圭田多虛籍。士宏考校,令隨實以輸,自部使者而下,皆十損七八。文彥博、包拯以廉能薦,由三司開拆司擢夔州路轉運使,遂知廣州。或傳安南舟數百泊海中,將為寇,嶺徼驚搖。士宏灼其非,是日,從賓客宴游為樂,民賴以安。受代還,引疾丐便郡,知鄭州。未幾,以光祿卿致仕。卒,年七十三。凡衣衾棺
 槨之制,皆有遺命,戒諸子勿為銘志。



 單煦,字孟陽,平原人。舉進士,知洛陽縣。民以妖幻傳相教授,煦跡捕戮三十餘人,當得上賞,不肯言。轉知昌州,時詔城蜀治,煦以蜀地負山帶江,一旦毀籬垣而興板築,其費巨萬,非民力所堪,請但築子城。轉運使即移諸郡如其議。



 徙清平軍使。有二盜殺人,捕治不承,煦縱使之食,甲食之既,乙不下咽,執而訊之,果殺人者。為御史臺推直官,江南人誣轉運使呂昌齡以賄,中丞張忭訊
 而論之。鞫未就,敕煦往治,煦不肯阿其長,卒直昌齡。乞外遷,知濮、合二州。合居涪、漢間,夏秋患於淫潦,煦築東堤以御之。赤水縣鹽井涸,奏蠲其賦。累官光祿卿,卒,年七十七。



 煦友愛兄熙,兄嘗毆人至死,未有知者。煦曰:「家貧親老,仰兄以養,義當代之死。」即趨詣斗所以待捕。已而死者蘇,驚問之,煦以情告。其人感嘆,遂輟訟。



 楊仲元,字舜明,管城人。第進士,調宛丘主簿。民訴旱,守拒之,曰:「邑未嘗旱,狡吏導民而然。」仲元白之曰:「野無青
 草,公日宴黃堂,宜不能知,但一出郊可見矣。狡吏非他,實仲元也。」竟免其稅。知澤州沁水縣,民持物來輸者,視其價稍增之,餘則下其估。官有所須,不強賦民,聽以所有與官為入,度相當則止,率常先辦。河外用兵,督餫轉西界,夕宿洪谷口。仲元相其地,乃寇所由徑路,亟命去之。民以困乏為辭,不聽,寇果夜出劫諸部,沁水獨免。後二十年,其子過縣,父老拜泣曰:「河西之役,非公無今日矣。」



 初,軍期尚緩,而仲元督行良急。至則芻糧有不集者
 皆可賤市,後期者物數倍其價,民始知其為利。州買羊,斂民差出錢帛滋蔓,病民為甚,仲元更其令,戶才費錢百。又遣吏市羔於他所,明年以供州,不科一錢。徙知鄖鄉縣,宰相張士遜先塋隸境內,將屬之,召不往。至則按籍均役之,雖堂帖求免,不為減。



 歷知光、虔、虢三州,官光祿卿,改中散大夫。戒諸子曰:「吾入官五十年,未嘗以私怒加人,雖杖刑之微,茍有兩比,不敢與輕法,以是為報國耳。」卒,年七十五。



 餘良肱,字康臣,洪州分寧人。第進士,調荊南司理參軍。屬縣捕得殺人者,既自誣服,良肱視驗尸與刃,疑之曰:「豈有刃盈尺傷不及寸乎?」白府請自捕逮,未幾,果獲真殺人者。民有失財物逾十萬,逮平民數十人,方暑,搒掠號呼聞於外;或有附吏耳語,良肱陰知其為盜,亟捕詰之,贓盡得。



 改大理寺丞,出知湘陰縣。縣逋米數千石,歲責里胥代輸,良肱論列之,遂蠲其籍。通判杭州,江潮善溢,漂官民廬舍,良肱累石堤二十里障之,潮不為害。
 時王陶為屬官,常以氣犯府帥,吏或訴陶,帥挾憾欲按之,良肱不可曰:「使陶以罪去,是以直不容也。」帥遂已。後陶官於朝,果以直聞。知虔州,士大夫死嶺外者,喪車自虔出,多弱子寡婦。良肱悉力振護,孤女無所依者,出俸錢嫁之。以母老,得知南康軍。丁母憂,服除,為三司使判官。



 方關、陜用兵,朝議貸在京民錢,良肱力爭之,會大臣亦以為言,議遂格。內府出腐幣售三司,三司吏將受之,良肱獨曰:「若賦諸軍,軍且怨;不則貨諸民,民且病。請付
 文思,以奉帷幄。



 改知明州。朝廷方治汴渠,留提舉汴河司。汴水澱污,流且緩,執政主挾河議。良肱謂:「善治水者不與水爭地。方冬水涸,宜自京左浚治,以及畿右,三年,可使水復行地中。」弗聽。又議伐汴堤木以資挾河。良肱言:「自泗至京千餘里,江、淮漕卒接踵,暑行多病暍,藉蔭以休。又其根盤錯,與堤為固,伐之不便。」屢爭不能得,乃請不與其事。執政雖怒,竟不為屈。改太常少卿、知潤州,遷光祿卿、知宣州,治為江東最。請老,提舉洪州玉隆觀,
 卒,年八十一。七子,卞、爽最知名。卞字洪範,爽字荀龍,皆以任子恩試校書郎。



 卞博學多大略,累為唐州判官、湖北安撫司勾當機宜文字。討叛蠻有功,知沅州。蠻殺沿邊巡檢,卞設方略復平之,加奉議郎。先是,良肱為鼎州推官,五溪蠻叛,良肱運糧境上,周知其利害,上書言:「此彈丸地,不足煩朝廷費,不如棄與而就撫之。」當時是其議,未果棄也。及蠻叛,斷渠陽道,扼官軍不得進,卞適使湖北,帥唐義問即授卞節制諸將。陰選死士三千人,夜
 銜枚繞出賊背,伐山開道,漏未盡數刻,入渠陽。黎明整眾出,賊大駭,盡銳來戰,奮擊大破之。鼓行度險,賊七遇七敗,斬首數千級,蠻遂降。尋有詔廢渠陽軍為砦,盡拔居人護出之。紹聖初,治棄渠陽罪,免歸。徽宗即位,復奉議郎,管勾玉隆觀。未幾,復渠陽為靖州,又論前事免,終於家。



 爽尚氣自信,不少貶以合世。應元豐詔,上便宜十五事,言過剴切。元祐末,爽復極言請太皇太后還政事,章惇憾爽不附己,乃擿其言為謗訕,以瀛州防禦推官
 除名,竄封州。久之,起知明州,未行,以言者罷,監東嶽廟。崇寧中,與卞俱入黨籍。



 潘夙,字伯恭,鄭王美從孫也。天聖中,上書論時政,授仁壽主簿。久之,知韶州,擢江西轉運判官,提點廣西、湖北刑獄。邵州蠻叛,湖南騷動,遷轉運使,專制蠻事,親督兵破其團峒九十。徙知滑州,改湖北轉運便,知桂州。坐在湖北時匿名書誣判官韓繹,謫監隨州酒稅。起知光化軍。大臣以將帥才舉之,易端州刺史,再遷徙鄜州。召對,
 訪交、廣事稱旨,還司封郎中、直昭文館,復知桂州。



 交人敗於占城,偽表稱賀以為大捷,神宗詔之曰:「智高之難方二十年,中人之情,燕安忽事,直謂山僻蠻獠,無可慮之理。殊不思禍生於所忽,唐六詔為中國患,此前事之師也。卿本將家子,寄要蕃,宜體朕意,悉心經度。」夙遂上書陳交址可取狀,且將發兵。未報,而徙河北轉運使,歷度支、鹽鐵副使,知河中府。章惇察訪荊湖,討南、北江蠻猺,陳夙憂邊狀,以知潭州。再遷光祿卿,知荊南、鄂州,卒,
 年七十。



 論曰:士之官斯世,有一善可稱,致生民咸被其澤於無窮者,故州郡之寄為尤重,張田免禁兵毒於瘴厲,士宏考圭田出於實輸,朱景父子、諲、載、煦、渙、士宏、壽隆輩,皆有德在民。仲元不以私怒加人,良肱明於折獄,夙以將家子而能留心邊務,用當其材,舉能其官。若琮也雖長於吏治,而所至掊克,君子奚取焉。



\end{pinyinscope}