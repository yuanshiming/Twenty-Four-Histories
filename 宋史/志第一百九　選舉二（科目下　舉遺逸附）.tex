\article{志第一百九 選舉二(科目下 舉遺逸附)}

\begin{pinyinscope}

 高宗建炎初,駐蹕揚州,時方用武,念士人不能至行在,下詔:「諸道提刑司選官即轉運置司州、軍引試,使副或判官一人董之。河東路附京西轉運司。國子監、開封府
 人就試於留守司,命御史一人董之。國子監人願就本路試者聽。」二年,定詩賦、經義取士,第一場詩賦各一首,習經義者本經義三道,《語》、《孟》義各一道;第二場並論一道;第三場並策三道。殿試策如之。自紹聖後,舉人不習詩賦,至是始復,遂除《政和令》命官私相傳習詩賦之禁。又詔:「下第進士,年四十以上六舉經御試、八舉經省試,五十以上四舉經御試、五舉經省試者,河北、河東、陜西特各減一舉;元符以前到省,兩舉者不限年,一舉年五
 十五已上者;諸道轉運司、開封府悉以名聞,許直赴廷試。」



 是秋,四方士集行在,帝親策於集英殿,第為五等,賜正奏名李易以下四百五十一人進士及第、進士出身、同學究出身、同出身。第一人為左宣教郎,第二、第三人左宣義郎,第四、第五人左儒林郎。第一甲第六名以下並左文林郎,第二甲並左從事郎,第三甲以下並左迪功郎。特奏名第一人附第二甲,賜進士及第,第二、第三人賜同進士出身,余賜同學究出身。登仕郎、京府助教、
 上下州文學、諸州助教入五等者,亦與調官。川、陜、河北、京東正奏名不赴者一百三人,以龍飛特恩,即家賜第。故事,廷試上十名,內侍先以卷奏定高下。帝曰:「取士當務至公,豈容以己意升降,自今勿先進卷。」



 三年,詔:「過省進士赴御試不及者,令漕臣據元舉送狀申省,給敕賜同進士出身。其計舉者,賜下州文學,並釋褐焉。」左司諫唐輝言:「舊制,省試用六曹尚書、翰林學士知貢舉,侍郎、給事中同知貢舉,卿監、郎官參詳,館職、學官點檢,御史
 監視,故能至公厭人心。今諸道類試,專委憲臣,奸弊滋生,才否貿亂,士論囂然,甚不稱更制設科之意,請並還禮部。」遂罷諸道類試。四年,復川、陜試如故。



 紹興元年,當祀明堂,復詔諸道類試,擇憲、漕或帥守中文學之人總其事,使精選考官。於是四川宣撫處置使張浚始以便宜令川、陜舉人,即置司州試之。會侯延慶言:「兵興,太學既罷,諸生解散,行在職事及厘務官隨行有服親及門客,往往鄉貢隔絕,請立應舉法,以國子監進士為名。」令
 轉運司附試。又詔:「京畿、京東西、河北、陜西、淮南士人轉徙東南者,令於寓戶州軍附試,別號取放。」



 時諸道貢籍多毀於兵,乃詔轉運司令舉人具元符以後得解、升貢、戶貫、三代、治經,置籍於禮部,以稽考焉。應該恩免解舉人,值兵毀失公據者,召京官二員委保,所在州軍給據,仍申部注籍。侍御史曾統請取士止用詞賦,未須兼經,高宗亦以古今治亂多載於史,經義登科者類不通史,將從其議。左僕射呂頤浩曰:「經義、詞賦均以言取人,宜
 如舊。」遂止。



 二年,廷試,手詔諭考官,當崇直言,抑諛佞。得張九成以下二百五十九人,凌景夏第二。呂頤浩言景夏詞勝九成,請更寘第一。帝曰:「士人初進,便須別其忠佞,九成所對,無所畏避,宜擢首選。」九成以類試、廷策俱第一,命特進一官。時進士卷有犯御名者,帝曰:「豈以朕名妨人進取邪?」令寘本等。又命應及第人各進一秩。舊制,潛藩州郡舉人,必曾請舉兩到省已上乃得試。帝嘗封蜀國公,是年,蜀州舉人以帝登極恩,徑赴類省試,自
 是為例。



 五年,初試進士於南省,戒飭有司:「商搉去取,毋以絺繪章句為工,當以淵源學問為尚。事關教化、有益治體者,毋以切直為嫌。言無根柢、肆為蔓衍者,不在採錄。」「舉人程文,許通用古今諸儒之說,及出己意,文理優長為合格。」三月,御試奏名,汪應辰第一。初,考官以有官人黃中第一,帝訪諸沉應求,應求以沉遘與馮京故事對,乃更擢應辰為魁,遂為定制。



 舊制。御試初考既分等第,印封送覆考定之,詳定所或從初,或從覆,不許別自
 立等。嘉祐中廢。至是,知制誥孫近奏:「若遵舊制,則高下升黜,盡出詳定官,初、覆考為虛設。請自今初、覆考皆未當,始許奏稟別置等第。」諫議大夫趙霈請用《崇寧令》,凡隔二等、累及五人許行奏稟,從之。是年,川、陜進士止試宣撫司,特奏名則置院差官,試時務策一道,禮部具取放分數、推恩等第頒示之。



 舊法,隨侍見任守倅等官,在本貫二千里外,曰滿里子弟。試官內外有服親及婚姻家,曰「避親」。館於見任門下,曰「門客」。是三等許牒試,否則
 不預。間有背本宗而竄他譜,飛賕而移試他道者,議者病之。六年,詔牒試應避者,令本司長官、州守倅、縣令委保,詭冒者連坐。



 七年,命行在職事、厘務官並宗子應舉、取應及有官人,並於行在赴國子監試,始命各差詞賦、經義考官。八年,以平江府四經巡幸,其得解舉人援臨安、建康駐蹕例,各免文解一次。時聞徽宗崩,未及大祥,禮部言:故事,因諒闇罷殿試,則省試第一人為榜首,補兩使職官。帝特命為左承事郎,自此率以為常。九年,以
 陜西舉人久蹈北境,理宜優異,非四川比,令禮部別號取放。川、陜分類試額自此始。是歲,以科試、明堂同在嗣歲,省司財計艱於辦給,又患初仕待闕率四五年,若使進士、蔭人同時差注,俱為不便,增展一年,則合舊制。十年,遂詔諸州依條發解,十二年正月省試,三月御試,後皆準此。



 十三年,國子司業高閌言:「取士當先經術。請參合三場,以本經、《語》、《孟》義各一道為首,詩賦各一首次之,子史論一道、時務策一道又次之,庶幾如古試法。又《春
 秋》義當於正經出題。」並從之。初立同文館試,凡居行在去本貫及千里已上者,許附試於國子監。十五年,凡特奏名賜同學究出身者,舊京府助教今改將仕郎。是歲,始定依汴京舊制,正奏及特恩分兩日唱名。十七年,申禁程文全用本朝人文集或歌頌及佛書全句者。



 十八年,以浙漕舉人有勢家行賂、假手濫名者,諭有司立賞格,聽人捕告。十九年,詔:「自今鄉貢,前一歲,州軍屬縣長吏籍定合應舉人,以次年春縣上之州,州下之
 學,核實引保,赴鄉飲酒,然後送試院。及期投狀射保者勿受。」自神宗朝程顥、程頤以道學倡於洛,四方師之,中興盛於東南,科舉之文稍用頤說。諫官陳公輔上疏詆頤學,乞加禁絕;秦檜入相,甚至指頤為「專門」,侍御史汪勃請戒飭攸司,凡專門曲說,必加黜落;中丞曹筠亦請選汰用程說者:並從之。二十一年,御試得正奏名四百人,特奏名五百三十一人。中興以來,得人始盛。



 二十二年,以士習《周禮》、《禮記》,較他經十無一二,恐其學浸廢,遂
 命州郡招延明於《二禮》者,俾立講說以表學校,及令考官優加誘進。舊諸州皆以八月選日試舉人,有趁數州取解者。二十四年,始定試期並用中秋日,四川則用季春,而仲秋類省。初,秦檜專國,其子熹廷試第一,檜陽引降第二名。是歲,檜孫塤舉進士,省試、廷對皆首選,姻黨曹冠等皆居高甲,後降塤第三。二十五年,檜死,帝懲其弊,遂命貢院遵故事,凡合格舉人有有權要親族,並令覆試。仍奪塤出身,改冠等七人階官並帶「右」字,餘悉駁放。
 程、王之學,數年以來,宰相執論不一,趙鼎主程頤,秦檜主王安石。至是,詔自今毋拘一家之說,務求至當之論。道學之禁稍解矣。



 自經、賦分科,聲律日盛,帝嘗曰:「向為士不讀史,遂用詩賦。今則不讀經,不出數年,經學廢矣。」二十七年,詔復行兼經,如十三年之制。內第一場大小經義各減一道,如治《二禮》文義優長,許侵用諸經分數。時號為四科。



 舊蜀士赴廷試不及者,皆賜同進士出身。帝念其中有俊秀能取高第者,不宜例置下列,至是,遂
 諭都省寬展試期以待之。及唱名,閻安中第二,梁介第三,皆蜀士也,帝大悅。二十九年,孫道夫在經筵,極論四川類試請托之弊,請盡令赴禮部。帝曰:「後舉但當遣御史監之。」道夫持益堅,事下國子監,祭酒楊椿曰:「蜀去行在萬里,可使士子涉三峽、冒重湖邪?欲革其弊,一監試得人足矣。」遂詔監司,守人卒賓客力可行者赴省,餘不在遣中。是歲,四川類省試始從朝廷差官。



 初,類試第一人恩數優厚,視殿試第三人,賜進士及第;後以何耕對策
 忤秦檜,乃改禮部類試蜀士第一等人,並賜進士出身,自是無有不赴御試者。惟遇不親策,則類省試第一人恩數如舊,第二、第三人皆附第一甲,九名以上附第二甲焉。是年詔:「四川等處進士,路遠歸鄉試不及者,特就運司附試一次,仍別行考校,取旨立額。」



 三十一年,禮部侍郎金安節言:「熙寧、元豐以來,經義詩賦,廢興離合,隨時更革,初無定制。近合科以來,通經者苦賦體雕刻,習賦者病經旨淵微,心有弗精,智難兼濟。又其甚者,論既
 並場,策問太寡,議論器識,無以盡人。士守傳注,史學盡廢,此後進往往得志,而老生宿儒多困也。請復立兩科,永為成憲。」從之。於是士始有定向,而得專所習矣。既而建議者以為兩科既分,解額未定,宜以國學及諸州解額三分為率,二取經義,一取詩賦。若省試,則以累舉過省中數立為定額而分之。詔下其議,然竟不果行。



 孝宗初,詔川、廣進士之在行都者,令附試兩浙轉運司。隆興元年,御試第一人承事郎、簽書諸州節度判官,第
 二第三人文林郎、兩使職官,第四第五人從事郎、初等職官,第六人至第四甲並迪功郎、諸州司戶簿尉,第五甲守選。乾道元年,詔四川特奏名第一等第一名賜同學究出身,第二名至本等末補將仕郎,第二等至第四等賜下州文學,第五等諸州助教。二年,御試,始推登極恩,第一名宣義郎,第二名與第一名恩例,第三名承事郎;第一甲賜進士及第並文林郎,第二甲賜進士及第並從事郎,第三、第四甲進士出身,第五甲同進士出身;特奏
 名第一名賜進士出身,第二、第三名賜同進士出身。



 四年,裁定牒試法:文武臣添差官除親子孫外並罷,其行在職事官除監察御史以上,餘並不許牒試。六年,詔諸道試官皆隔一郡選差,後又令歷三郡合符乃聽入院,防私弊也。



 帝欲令文士能射御,武臣知詩書,命討論殿最之法。淳熙二年禦試,唱第後二日,御殿,引按文士詹騤以下一百三十九人射藝。翌日,又引文士第五甲及特奏名一百五十二人。其日,進士具襴笏入殿起居,易
 戎服,各給箭六,弓不限鬥力,射者莫不振厲自獻,多命中焉。天子甚悅。凡三箭中帖為上等,正奏第一人轉一官,與通判,餘循一資;二箭中為中等,減二年磨勘;一箭中帖及一箭上垛為下等,一任回不依次注官;上四甲能全中者取旨;第五甲射入上等注黃甲,餘升名次而已。特奏名五等人射藝合格與文學,不中者亦賜帛。



 四年,罷同文館試。又命省試簾外官同姓異姓親若門客,亦依簾內官避親法,牒送別院。五年,以階、成、西和、鳳州
 正奏名比附特奏名五路人例,特升一甲。六年,詔特奏名自今三名取一,寘第四等以前,餘並入第五等,其末等納敕者止許一次,潛藩及五路舊升甲者今但升名。其後又許納敕三次,為定制焉。



 十一年,進士廷試不許見燭,其納卷最後者降黜之。舊制,廷試至暮許賜燭,然殿深易暗,日昃已燭出矣。凡賜燭,正奏名降一甲,第五甲降充本甲末名;特奏名降一等,第五等與攝助教。凡試藝於省闈及國子監、兩浙轉運司者,皆禁燭,其它郡
 國,率達旦乃出。十月,太常博士倪思言:「舉人輕視史學,今之論史者獨取漢、唐混一之事,三國、六朝、五代為非盛世而恥談之,然其進取之得失,守禦之當否,籌策之疏密,區處兵民之方,形勢成敗之跡,俾加討究,有補國家。請諭春官:凡課試命題,雜出諸史,無所拘忌;考核之際,稍以論策為重,毋止以初場定去留。」從之。



 十四年,御試正奏名王容第一。時帝策士,不盡由有司,是舉容本第三,親擢為榜首。翰林學士洪邁言:「《貢舉令》:賦限三百
 六十字,論限五百字。今經義、論、策一道有至三千言,賦一篇幾六百言,寸晷之下,唯務貪多,累牘連篇,何由精妙?宜俾各遵體格,以返渾淳。」



 時朱熹嘗欲罷詩賦,而分諸經、子、史、時務之年。其《私議》曰:「古者大學之教,以格物致知為先,而其考校之法,又以九年知類通達、強立不反為大成。今《樂經》亡而《禮經》闕,二戴之《禮》已非正經,而又廢其一。經之為教已不能備,而治經者類皆舍其所難而就其易,僅窺其一而不及其餘。若諸子之學同出
 於聖人,諸史則該古今興亡治亂得失之變,皆不可闕者。而學者一旦豈能盡通?若合所當讀之書而分之以年,使之各以三年而共通其三四之一。凡《易》、《詩》、《書》為一科,而子年、午年試之;《周禮》、《儀禮》及二《戴記》為一科,而卯年試之;《春秋》及《三傳》為一科,而酉年試之。義各二道,諸經皆兼《大學》、《論語》、《中庸》、《孟子》義一道。論則分諸子為四科,而分年以附焉。諸史則《左傳》、《國語》、《史記》、《兩漢》為一科,《三國》、《晉書》、《南北史》為一科,《新舊唐書》、《五代史》為一科。時
 務則律歷、地理為一科,以次分年如經、子之法,試策各二道。又使治經者各守家法,答義者必通貫經文,條舉眾說而斷以己意,有司命題必依章句,如是則士無不通之經、史,而皆可用於世矣。」其議雖未上,而天下誦之。



 光宗初,以省試春淺,天尚寒,遂展至二月朔卜曰,殿試於四月上旬。紹熙元年,仍按射,不合格者罷賜帛。舊命官鎖廳及避親舉人同試。三年,始令分場,以革假人試藝者,於是四蜀皆然。



 寧宗慶元二年,韓侂冑襲秦檜餘論,
 指道學為偽學,臺臣附和之,上章論列。劉德秀在省闈,奏請毀除語錄。既而知貢舉吏部尚書葉翥上言:「士狃於偽學,專習語錄詭誕之說、《中庸》《大學》之書,以文其非。有葉適《進卷》、陳傅良《待遇集》,士人傳誦其文,每用輒效。請令太學及州軍學,各以月試合格前三名程文,上御史臺考察,太學以月,諸路以季。其有舊習不改,則坐學官、提學司之罪。」是舉,語涉道學者,皆不預選。四年,以經義多用套類,父子兄弟相授,致天下士子不務實學。遂
 命有司:六經出題,各於本經摘出兩段文意相類者,合為一題,以杜挾冊讎偽之計。



 嘉泰元年,起居舍人章良能陳主司三弊:一曰沮抑詞賦太甚,既暗削分數,又多置下陳。二曰假借《春秋》太過,諸處解榜,多寘首選。三曰國史、實錄等書禁民私藏,惟公卿子弟因父兄得以竊窺,冒禁傳寫,而有司乃取本朝故事,藏匿本末,發為策問,寒士無由盡知。命自今詩賦純正者寘之前例,《春秋》唯卓異者寘高等,餘當雜定,策題則必明白指問。四年,
 詔:「自今礙格、不礙格人試於漕司者,分院異題,永為定制。」



 開禧元年,詔:「禮部考試,以三場俱優為上,二場優次之,一場優又次之,俱劣為下。毋以片言只字取人。編排既定,從知舉審定高下,永為通考之法。」二年,以舉人奸弊滋多,命諸道漕司、州府、軍監,凡發解舉人,合格試卷姓名,類申禮部。候省試中,牒發御史臺,同禮部長貳參對字畫,關御藥院內侍照應,廷試字畫不同者,別榜駁放。



 舊制,秋貢春試,皆置別頭場,以待舉人之避親者。自
 緦麻以上親及大功以上婚姻之家,皆牒送。惟臨軒親試,謂之天子門生,雖父兄為考官,亦不避。嘉定元年,始因議臣有請,命朝官有親屬赴廷對者,免差充考校。十二年,命國子牒試,禁假托宗枝、遷就服屬,犯者必寘於罰。十五年,秘書郎何淡言:「有司出題,強裂句讀,專務斷章,離絕旨意,破碎經文。望令革去舊習,使士子考注疏而辨異同,明綱領而識體要。」從之。



 至理宗朝,奸弊愈滋。有司命題茍簡,或執偏見臆說,互相背馳,或發策用事
 訛舛,故士子眩惑,莫知適從,才者或反見遺。所取之士既不精,數年之後,復俾之主文,是非顛倒逾甚,時謂之繆種流傳。復容情任意,不學之流,往往中第。而舉人之弊凡五:曰傳義,曰換卷,曰易號,曰卷子出外,曰謄錄滅裂。迨寶慶二年,左諫議大夫朱端常奏防戢之策,謂:「試院監大門、中門官,乃一院襟喉切要,乞差有風力者。入試日,一切不許傳遞。門禁既嚴,則數弊自清。士人暮夜納卷,易於散失。宜令封彌官躬親封鐍卷匱,士人親書
 幕歷投匱中。俟舉人盡出院,然後啟封,分類抄上,即付謄錄所。明旦,申逐場名數於御史臺檢核。其撰號法,上一字許同,下二字各異,以杜訛易之弊。謄錄人選擇書手充,不許代名,具姓名字樣,申院覆寫檢實。傳義置窠之人,委臨安府嚴捕。其考官容情任意者,許臺諫風聞彈奏,重寘典憲。及出官錢,立賞格,許告捉懷挾、傳題、傳稿、全身代名入試之人。」帝悉從之,且命精擇考官,毋仍舊習。舊制,凡即位一降科詔,及大比之歲,二月一日一
 降詔,許發解,然後禮部遍牒諸路及四川州軍。至是,以四川鎖院改用二月二十一日,與降詔日相逼,遂改用正月十五日奏裁降詔。



 紹定元年,有言舉人程文雷同,或一字不差。其弊有二:一則考官受賂,或授暗記,或與全篇,一家分傳謄寫;一則老儒賣文場屋,一人傳十,十人傳百,考官不暇參稽。於是命禮部戒飭,前申號三日,監試會聚考官,將合取卷參驗互考,稍涉雷同,即與黜落。或仍前弊,以致覺察,則考官、監試一例黜退。初,省試
 奉敕差知貢舉一員,同知二員,內差臺諫官一員;參詳官若干員,內差監察御史一員。俾會聚考校,微寓彈壓糾察之意。韓侂冑用事,將鈐制士人,遂於三知舉外,別差同知一員,以諫官為之,專董試事,不復干預考校,參詳官亦不差察官。於是約束峻切,氣焰熏灼。嘉泰間,更名監試,其失愈甚,制造簿歷,嚴立程限。至是,復舊制,三知舉內差一臺諫,十參詳內差一御史,仍戒飭試官,精加考校,如日力不給,即展期限。



 二年,臣僚言考官之弊:
 詞賦命題不明,致士子上請煩亂;經義不分房別考,致士子多悖經旨。遂飭考官明示詞賦題意,各房分經考校。凡廷試,唯蜀士到杭最遲,每展日以待。會有言:「蜀士嗜利,多引商貨押船,致留滯關津。」自是,定以四月上旬廷試,更不移展。三年,臣僚請:「學校、場屋,並禁斷章截句,破壞義理,及《春秋經》越年牽合。其程文,本古注、用先儒說者取之,穿鑿撰說者黜落。」



 四年,臣僚甚言科場之弊,乞戒飭漕臣嚴選考官。地多經學,則博選通經者;地多
 賦學,則廣致能賦者。主文必兼經賦,乃可充其職。監試或倅貳不勝任,必別擇人。仍令有司量展揭封之期,庶考校詳悉,不致失士。於是命遍諭國子監及諸郡,恪意推行約束,違戾者彈劾治罪。初,四川類試,其事雖隸制司,而監試、考官共十員,唯大院別院監試、主文各一員從朝命,余聽制司選差。自安丙差四員之外,權委成都帥守臨期從近取具。是歲,始仍舊朝命四員,餘從制司分選。



 時場屋士子日盛,卷軸如山。有司不能遍睹,迫於
 日限,去取不能皆當。蓋士人既以本名納卷,或別為名,或易以字,一人而納二三卷。不禁挾書,又許見燭,閩、浙諸郡又間日引試,中有一日之暇,甚至次日午方出。於是經義可作二三道,詩賦可成五六篇。舉人文章不精,考官困於披閱。幸皆中選,乃以兄弟承之,或轉售同族,奸詐百端,真偽莫辨。乃命諸郡關防,於投卷之初,責鄉鄰核實,嚴治虛偽之罪、縱容之罰,其弊稍息。



 命官鎖廳及避親舉人,自紹熙分場各試,寒士憚之。緣避親人七
 人取一,其額太窄,咸以為窘;而朝士之被差為大院考官者,恐多妨其親,亦不願差。寒士於鄉舉千百取一之中,得預秋薦,以數千里之遠,辛勤赴省;而省闈差官,乃當相避。遂有隱身匿名不認親戚以求免者,憤懣憂沮狼狽旅邸者,彼此交怨,相視為仇。至是,言者謂:「除大院收試外,以漕舉及待補國子生到省者,與避親人同試於別院,亦將不下數百。人數既多,其額自寬,寒士可不怨其親戚,朝士可不憚於被差。」從之。既而以諸路轉運
 司牒試,多營求偽冒之弊,遂罷之。其實有妨嫌者收試,每百人終場取一人,於各路州軍解額窄者量與均添,庶士子各安鄉里,無復詐競。於是臨安、紹興、溫、臺、福、婺、慶元、處、池、袁、潮、興化及四川諸州府,共增解額一百七十名。未幾,又命止許牒滿裏親子孫及門客,召見任官二員委保,與有官礙格人各處收試,五十人取放一人。合牒親子孫別項隔截收試,不及五十人亦取一人。凡涉詐冒,並坐牒官、保官。



 初,唐、鄧二州嘗陷於金,金滅,復
 得其地,命仍舊類試於襄陽,但別號考校,以優新附士子。舊制,光州解額七名,渡江後為極邊,士子稀少,權赴試鄰州,淳熙間,本州自置科場,權放三名。至是,已五六十年,舉人十倍於前,遂命復還舊額。



 端平元年,以牒試已罷,解額既增,命增額州郡措置關防,每人止納一卷,及開貢院添差考官。時有言:門客及隨侍親子孫五十人取一,臨安府學三年類申人漕試七十取一,又令別試院分項異處收試,已為煩碎;兼兩項士人習賦習《書》
 之外,習他經者差少,難於取放。遂命將兩項混同收試考校,均作六十取一;京學見行食職事生員二百二十四名,別項發號考校,不限經賦,取放一名。



 侍御史李鳴復等條列建言,謂:「臺諫充知舉、參詳,既留心考校,不能檢柅奸弊,欲乞仍舊差臺諫為監試。懷挾之禁不嚴,皆為具文,欲乞懸賞募人告捉,精選強敏巡按官及八廂等人,謹切巡邏,有犯,則鐫黜官員。考校不精,多緣點檢官不時供卷,及開院日迫,試卷沓至,知舉倉卒不及,遂
 致遺才,欲乞試院隨房置歷程督,點檢官書所供卷數,逐日押歷考校。試卷不遵舊式,務從簡便,點檢、參詳穿聯為一,欲乞必如舊制,三場試卷分送三點檢、三參詳、三知舉,庶得詳審。試官互考經賦,未必精熟,欲乞前期約度試卷,經、賦凡若干,則各差試官若干,不至偏重。」並從之。



 嘉熙元年,罷諸牒試,應郎官以上監司、守倅之門客及姑姨同宗之子弟,與游士之不便於歸鄉就試者,並混同試於轉運司,各從所寓縣給據,徑赴司納卷,一
 如鄉舉之法。家狀各書本貫,不問其所從來,而定其名「寓試」,以四十名為額,就試如滿五十人,則臨時取旨增放。又罷諸路轉運司及諸州軍府所取待補國子生,自明年並許赴國子監混試。以士子數多,命於禮部及臨安轉運司兩試院外,紹興、安吉各置一院,從朝廷差官前詣,同日引試,分各路士人就試焉。同在京,不許見燭。是年,已失京西諸州軍,士多徙寓江陵、鄂州,命京湖制置司於江陵別立貢院,取德安府、荊門軍、歸峽復三州
 及隨、郢、均、房等京西七郡士人,別差官混試,用十二郡元額混取以優之。



 牒試既罷,又復冒求國子,士大夫為子弟計者,輒牒外方他族,利為場屋相資,或公然受價以鬻。命遍諭百官司知雜司等:如已準朝廷辨驗,批書印紙,批下國子監收試,即報赴試人躬赴監。一姓結為一保,每保不過十人,責立罪罰,當官書押,遞相委保,各給告示,方許投納試卷。冒牒官降官罷任,或一時失於參照,誤牒他族,計自陳悔牒一次。冒牒中選之人,限主
 保官、舉人一月自首,舉人駁放,主保官免罪;出限不首者,仍照前條罪之。凡類試卷,封彌作弊不一。至是,命前期於兩浙轉運司、臨安府選見役吏胥共三十人,差近上一名部轄入院,十名專管詩賦,餘分管諸經。各隨所管號,於引試之夕,分尋試卷,各置簿封彌,不許混亂;卻別差一吏將號置歷,發過謄錄所書寫。其簿、歷,封彌官收掌,不經吏手,不許謄錄人干預,以革其弊。



 二年,省試下第及游學人,並就臨安府給據,赴兩浙轉運司混試
 待補太學生。臣僚言:「國子牒試之弊,冒濫滋甚。在朝之士,有強認疏遠之親為近屬者,有各私親故換易而互牒者,有為權勢所軋、人情所牽應命而泛及者,有自揆子弟非才、牒同姓之雋茂利其假手者,有文藝素乏、執格法以求牒轉售同姓以謀利者。今後令牒官各從本職長官具朝典狀保明,先期取本官知委狀,仍立賞格,許人指實陳首。冒牒之官,按劾鐫秩;受牒之人,駁放殿舉;保官亦與連坐。專令御史臺覺察,都省勘會。類申門
 客、滿里子孫仍前漕試,六十人取一,較之他處雖甚優,而取無定額,士有疑心,就試者少。宜令額寬而試者眾,塗一而取之精。」遂依前例放行寓試,以四十名為定額,仍前待補;其類申門客、滿里子孫及附試並罷。



 淳祐元年,臣僚言:「既復諸路漕試,合國子試、兩項科舉及免舉人,不下千數。宜復撥漕舉、冑舉同避親人並就別院引試,使大院無卷冗之患,小院無額窄之弊。」從之。時淮南諸州郡歲有兵禍,士子不得以時赴鄉試,且漕司分差
 試官,路梗不可徑達。三年,命淮東州郡附鎮江府秋試,淮西州郡附建康試,蘄黃光三州、安慶府附江州試。三試所各增差試官二員,別項考校,照各州元額取放。是歲,兩浙轉運司寓試終場滿五千人,特命增放二名,後雖多不增,如不及五千人,止依元額。別院之試,大率士子與試官實有親嫌者,紹定間,以漕試、冑試無親可避者亦許試,或謂時相徇於勢要子弟故也;端平初,撥歸大院,寒雋便之;淳祐元年,又復赴別院,是使不應避親
 之人抑而就此,使天下士子無故析而為二,殊失別試之初意。至是,依端平厘正之,復歸大院。



 九年,以臣僚言:「士子又有免解偽冒入試者,或父兄沒而竊代其名,或同族物故而填其籍。」於是令自本貫保明給據,類其姓名先申禮部,各州揭以示眾,犯者許告捉,依鬻舉法治罪。十二年,廣南西路言:「所部二十五郡,科選於春官者僅一二,蓋山林質樸,不能與中土士子同工,請授兩淮、荊襄例別考。」朝廷從其請。自是,廣南分東、西兩路。



 寶祐
 二年,監察御史陳大方言:「士風日薄,文場多弊。乞將發解士人初請舉者,從所司給帖赴省,別給一歷,如命官印紙之法,批書發解之年及本名年貫、保官姓名,執赴禮部,又批赴省之年,長貳印署。赴監試者同。如將來免解、免省,到殿批書亦如之。如無歷則不收試。候出官日赴吏部繳納,換給印紙。應合免解、免省人,亦從先發解處照此給歷。如省、殿中選,將元歷發下御史臺考察,以憑注闕給告。士子得歷,可為據證;有司因歷,可加稽驗。
 日前偽冒之人,可不卻而自遁。」遂自明年始行之。



 鄉貢、監補、省試皆有復試,然銓擇猶未精,其間濫名充貢者,不可欺同舉之人,冒選橋門者,不逃於本齋之職事。遂命今後本州審察,必責同舉之聯保,監學簾引,必責長諭之證實,並使結罪,方與放行。中書復試,凡涉再引,非系雜犯,並先札報各處漕司,每遇詔舉,必加稽驗。凡復試,令宰執出題,不許都司干預,仍日輪臺諫一員,簾外監試。四年,命在朝之臣,除宰執、侍從、臺諫外,自卿監、郎
 官以下至厘務官,各具三代宗支圖三本,結立罪狀,申尚書省、御史臺及禮部,所屬各置簿籍,存留照應。遇屬子孫登科、發解、入學、奏補事故,並具申入鑿。後由外任登朝,亦於供職日後,具圖籍記如上法。遇冑試之年,照朝廷限員,於內牒能應舉人就試,以革冑牒冒濫之弊。



 景定二年,冑子牒試員:宰執牒緦麻以上親增作四十人,侍從、臺諫、給事中、舍人小功以上親增作二十七人,卿監、郎官、秘書省官、四總領小功以上親增作二十人,
 寺監丞簿、學官、二令大功以上親增作十五人,六院、四轄、省部門、史館校勘、檢閱大功以上親增作十人,臨安府通判牒大功以上親增作八人,餘應牒親子孫者,一仍舊制。



 度宗初,以雷同假手之弊,多由於州郡試院繼燭達旦,或至次日辰、巳猶未出院,其所以間日者,不惟止可以惠不能文之人,適足以害能文之士,遂一遵舊制,連試三日。時諸州郡以鄉貢終場人眾而元額少,自咸淳九年為始,視終場人多寡,每二百人取放一名。以
 士子數多,增參詳官二員,點檢試卷官六員。又以臣僚條上科場之弊,以大院別院參詳官、點檢試卷官兼考雷同,又監試兼專一詳定雷同試卷,不預考校。遂罷簾外點檢雷同官,國子監解試雷同官亦罷。



 先是,州郡鄉貢未有復試。會言者謂冒濫之弊,惟在鄉貢,遂命漕臣及帥守於解試揭曉之前,點差有出身倅貳或幕官專充復試。盡一日命題考校,解名多者,斟酌分日。但能行文不繆、說理優通、覺非假手即取,非才不通就與駁放。
 如將來省復不通,罪及元復試漕守之臣及考校官。十年,省試,命大院、別院監試官於坐圖未定之先,親監分布坐次,嚴禁書鋪等人,不許縱容士子拋離座案,過越廊分,為傳義假手之地。時成都已歸附我朝,殿試擬五月五日,以蜀士至者絕少,展至末旬。又因復試特奏名至部猶少,展作六月七日。近臣以隆暑為請,復命立秋後擇日。七月八日,度宗崩,竟不畢試。嗣君即位,下禮部討論,援引皆未當,既不可謂之亮陰,又不可不赴廷對,
 乃仿召試館職之制而行之。



 新進士舊有期集,渡江後置局於禮部貢院,特旨賜餐錢,唱第之三日赴焉。上三人得自擇同升之彥,分職有差。朝謝後拜黃甲,其儀設褥於堂上,東西相向,皆再拜。拜已,擇榜中年長者一人,狀元拜之,復擇最少者一人拜狀元。所以侈寵靈,重年好,明長少也。



 制舉無常科,所以待天下之才傑,天子每親策之。然宋之得才,多由進士,而以是科應詔者少。惟召試館職及
 後來博學宏詞,而得忠鯁文學之士。或起之山林,或取之朝著,召之州縣,多至大用焉。太祖始置賢良方正能直言極諫、經學優深可為師法、詳閑吏理達於教化凡三科,不限前資,見任職官,黃衣草澤,悉許應詔,對策三千言,詞理俱優則中選。乾德初,以郡縣亡應令者,慮有司舉賢之道或未至也,乃詔許士子詣闕自薦。四年,有司僅舉直言極諫一人,堪為師法一人,召陶穀等發策,帝親御殿臨視之,給硯席坐於殿之西隅。及對策,詞理
 疏闊,不應所問,賜酒饌宴勞而遣之。



 開寶八年,詔諸州察民有孝弟力田、奇才異行或文武材幹、年二十至五十可任使者,具送闕下,如無人塞詔,亦以實聞。九年,諸道舉孝弟力田及有才武者凡七百四十人,詔翰林學士李昉等於禮部試其業,一無可採。而濮州以孝悌薦名者三百七十人,帝駭其多,召對講武殿,率不如詔。猶自陳素習武事,復試以騎射,輒顛隕失次。帝紿曰:「是宜隸兵籍。」皆號呼乞免,乃悉罷去。詔劾本部濫舉之罪。



 咸
 平四年,詔學士、兩省御史臺五品、尚書省諸司四品以上,於內外京朝幕府州縣官、草澤中,各舉賢良方正一人,不得以見任轉運使及館閣職事人應詔。是年,策秘書丞查道等七人,皆入第四等。景德二年,增置博通墳典達於教化、才識兼茂明於體用、武足安邊、洞明韜略運籌決勝、軍謀宏遠材任邊寄等科,詔中書門下試察其才,具名聞奏,將臨軒親策之。自是應令者浸廣,而得中高等亦少。



 太宗以來,凡特旨召試者,於中書學士舍
 人院,或特遣官專試,所試詩、賦、論、頌、策、制誥,或三篇,或一篇,中格則授以館職。景德後,惟將命為知制誥者,乃試制誥三道。



 每道百五十字。



 東封及祀汾陰時,獻文者多試業得官,蓋特恩也。時言者以為:「兩漢舉賢良,多因兵荒災變,所以詢訪闕政。今國家受瑞登封,無闕政也,安取此?」乃罷其科,惟吏部設宏詞、拔萃、平判等科如舊制。



 仁宗初,詔曰:「朕開數路以詳延天下之士,而制舉獨久不設,意者吾豪傑或以故見遺也,其復置此科。」於是增其名,
 曰:賢良方正能直言極諫科,博通墳典明於教化科,才識兼茂明於體用科,詳明吏理可使從政科,識洞韜略運籌帷幄科,軍謀宏遠材任邊寄科,凡六,以待京、朝之被舉及起應選者。又置書判拔萃科,以待選人。又置高蹈丘園科,沉淪草澤科,茂材異等科,以待布衣之被舉者。其法先上藝業於有司,有司較之,然後試秘閣,中格,然後天子親策之。



 治平三年,命宰執舉館職各五人。先是,英宗謂中書曰:「水潦為災,言事者云『咎在不能進賢』,
 何也?」歐陽修曰:「近年進賢路狹,往時入館有三路,今塞其二矣。進士高科,一路也;大臣薦舉,一路也;因差遣例除,一路也。往年進士五人以上皆得試,第一人及第不十年有至輔相者,今第一人兩任方得試,而第二人以下不復試,是高科路塞矣。往時大臣薦舉即召試,今只令上簿候缺人乃試,是薦舉路塞矣。惟有因差遣例除者,半是年勞老病之人。此臣所謂薦舉路狹也。」帝納之,故有是命。韓琦、曾公亮、趙概等舉蔡延慶以下凡二十
 人,皆令召試,宰臣以人多難之。帝曰:「既委公等舉之,茍賢,豈患多也?先召試蔡延慶等十人,餘須後時。」神宗以進士試策,與制科無異,遂詔罷之。試館職則罷詩、賦,更以策、論。



 元祐二年,復制科。凡廷試前一年,舉奏官具所舉者策、論五十首奏上,而次年試論六首,御試策一道,召試、除官、推恩略如舊制。右正言劉安世建言:「祖宗之待館職也,儲之英傑之地以飭其名節,觀以古今之書而開益其聰明,稍優其廩,不責以吏事,所以滋長德器,
 養成名卿賢相也。近歲其選浸輕,或緣世賞,或以軍功,或酬聚斂之能,或徇權貴之薦。未嘗較試,遂獲貼職,多開幸門,恐非祖宗德意。望明詔執政,詳求文學行誼,審其果可長育,然後召試,非試毋得輒命,庶名器重而賢能進。」三年,乃詔:「大臣奏舉館職,並如舊召試、除授,惟朝廷特除,不用此令。」安世復奏曰:「祖宗時入館,鮮不由試。惟其望實素著,治狀顯白,或累持使節,或移鎮大藩,欲示優恩,方令貼職。今既過聽臣言,追復舊制,又謂『朝廷
 特除,不在此限』。則是人材高下,資歷深淺,但非奏舉,皆可直除,名為更張,弊源尚在。願仿故事,資序及轉運使,方可以特命除授,庶塞僥幸,以重館職之選。」



 紹聖初,哲宗謂:「制科試策,對時政得失,進士策亦可言。」因詔罷制科。既而三省言:「今進士純用經術。如詔誥、章表、箴銘、賦頌、赦敕、檄書、露布、誡諭,其文皆朝廷官守日用不可闕,且無以兼收文學博異之士。」遂改置宏詞科,歲許進士及第者詣禮部請試,如見守官則受代乃請,率以春試上
 舍生附試,不自立院也。試章表、露布、檄書用駢儷體,頌、箴銘、誡諭、序記用古體或駢儷,惟詔誥、赦敕不以為題。凡試二日四題,試者雖多,取毋過五人,中程則上之三省復試之,分上、中二等,推恩有差;詞藝超異者,奏取旨命官。大觀四年詔:「宏詞科格法未詳,不足以致文學之士,改立詞學兼茂科,歲附貢士院試,取毋過三人。」政和增為五人。不試檄書,增制誥,以歷代史事借擬為之,中格則授館職。宰臣執政親屬毋得試。宣和罷試上舍,乃
 隨進士試於禮部。



 紹興元年,初復館職試,凡預召者,學士院試時務策一道,天子親覽焉。然是時校書多不試,而正字或試或否。二年,詔舉賢良方正能直言極諫科,一遵舊制,自尚書兩省諫議大夫以上、御史中丞、學士、待制各舉一人。凡應詔者,先具所著策、論五十篇繳進,兩省侍從參考之,分為三等,次優以上,召赴秘閣,試論六首,於《九經》、《十七史》、《七書》、《國語》、《荀揚管子》、《文中子》內出題,學士兩省官考校,御史監之,四通以上為合格。仍分
 五等,入四等以上者,天子親策之。第三等為上,恩數視廷試第一人,第四等為中,視廷試第三人,皆賜制科出身;第五等為下,視廷試第四人,賜進士出身;不入等者與簿尉差遣,已仕者則進官與升擢。七年,以太陽有異。令中外侍從各舉能直言極諫一人。是冬,呂祉舉選人胡銓,汪藻舉布衣劉度,即除銓樞密院編修官,而度不果召。自是詔書數下,未有應者。



 孝宗乾道二年,苗昌言奏:「國初嘗立三科,真宗增至六科,仁宗時並許布衣應
 詔,於是名賢出焉。請參稽前制,間歲下詔,權於正文出題,不得用僻書注疏,追復天聖十科,開廣薦揚之路,振起多士積年委靡之氣。」遂詔禮部集館職、學官雜議,皆曰:「注疏誠可略,科目不必廣。天下之士,屏處山林,滯跡遐遠,侍從之臣,豈能盡知。」遂如國初之制,止令監司、守臣解送。



 七年,詔舉制科以六論,增至五通為合格,始命官、糊名、謄錄如故事。試院言:「文卷多不知題目所出,有僅及二通者。」帝命賜束帛罷之,舉官皆放罪。舊試六題,
 一明一暗。時考官命題多暗僻,失求言之意,臣僚請遵天聖、元祐故事,以經題為第一篇,然後雜出《九經》、《語》、《孟》內注疏或子史正文,以見尊經之意。從之。初,制科取士必以三年,十一年,詔:「自今有合召試者,舉官即以名聞。」明年春,李巘言:「賢良之舉,本求讜言以裨闕政,未聞責以記誦之學,使才行學識如晁、董之倫,雖注疏未能盡記,於治道何損?」帝以為然,乃復罷注疏。



 高宗立博學宏詞科,凡十二通,制誥、詔表、露布、檄、箴銘、記贊、頌序內雜
 出六題,分為三場,每場體制一古一今。遇科場年,應命官除歸明、流外、入貲及犯贓人外,公卿子弟之秀者皆得試。先投所業三卷,學士院考之,拔其尤者召試,定為三等。上等轉一官,選人改秩,無出身人賜進士及第,並免召試,除館職。中等減三年磨勘,與堂除,無出身人賜進士出身;下等減二年磨勘,無出身人賜同進士出身:並許召試館職。南渡以來所得之士,多至卿相、翰苑者。



 理宗嘉熙三年,臣僚奏:「詞科實代王言,久不取人,日就廢
 弛。蓋試之太嚴,故習之者少。今欲除博學宏詞科從舊三歲一試外,更降等立科。止試文辭,不貴記問。命題止分兩場,引試須有出身人就禮部投狀,獻所業,如試教官例。每一歲附銓闈引試,惟取合格,不必拘額,中選者與堂除教授,已系教官資序及京官不願就教授者,京官減磨勘,選人循一資。他時北門、西掖、南宮舍人之任,則擇文墨超卓者用之。其科目,則去『宏博』二字,止稱詞學科。」從之。淳祐初,罷。景定二年,復嘉熙之制。



 初,內外學官多朝廷特注,後稍令國子監取其舊試藝等格優者用之。熙寧八年,始立教授試法,即舍人院召試大義五道。元豐七年,令諸州無教官,則長吏選在任官上其名,而監學審其可者使兼之。元祐中,罷試法,已而論薦益眾,乃詔須命舉乃得奏。紹聖初,三省立格,中制科及進士甲第、禮部奏名在上三人、府監廣文館第一人、從太學上舍得第,皆不待試,餘召試兩經大義各一道,合格則授教官。元符中,增試三經。政和二年,臣僚
 言:「元豐召試學官六十人,而所取四人,皆知名之士,故學者厭服。近試率三人取一,今欲十人始取一人,以重其選。」從之。自是或如舊法,中書選注。又嘗員外添置八行應格人為大藩教官,不以蒞職,隨廢。或用元豐試法,更革無常。



 高宗初年,復教官試。紹興中,議者謂:「欲為人師,而自獻以求進,非禮也。」乃罷試而自朝廷選差。已而又復之,凡有出身者許應,先具經義、詩、賦各三首赴禮部,乃下省闈,分兩場試之。初任為諸州教官,由是為兩
 學之選。十五年,從國子監丞文浩所言,於《六經》中取二經,各出兩題,毋拘義式,以貫穿該贍為合格。其後,四川制置司遇類省試年,亦仿禮部附試,自嘉泰元年始。



 凡童子十五歲以下,能通經作詩賦,州升諸朝,而天子親試之。其命官、免舉無常格。真宗景德二年,撫州晏殊、大名府姜蓋始以童子召試詩賦,賜殊進士出身,蓋同學究出身。尋復召殊試賦、論,帝嘉其敏贍,授秘書正字。後或罷或復。自仁宗即位,至大觀末,賜出身者僅二十
 人。



 建炎二年,明舊制,親試童子,召見朱虎臣,授官賜金帶以寵之。後至者或誦經、史、子、集,或誦禦制詩文,或誦兵書、習步射,其命官、免舉,皆臨期取旨,無常格。淳熙中,王克勤始以程文求試。內殿引見,孝宗嘉其警敏,補從事郎,令秘閣讀書。會禮部言:「本朝童子以文稱者,楊億、宋綬、晏殊、李淑,後皆為賢宰相、名侍從。今郡國舉貢,問其所能,不過記誦,宜稍艱其選。」八年,始分為三等:凡全誦《六經》、《孝經》、《語》、《孟》及能文,如《六經》義三道、《語》《孟》義各一
 道、或賦一道、詩一首為上等,與推恩;誦書外能通一經,為中等,免文解兩次;止能誦《六經》、《語》、《孟》為下等,免文解一次。覆試不合格者,與賜帛。寧宗嘉定十四年,命歲取三人,期以季春集闕下。先試於國子監,而中書復試之,為永制焉。理宗後罷此科,須卓絕能文者,許諸郡薦舉。



 科目既設,猶慮不能盡致天下之才,或韜晦而不屑就也,往往命州郡搜羅,而公卿得以薦言。若治平之黃君俞,熙寧之王安國,元豐則程頤,元祐則陳師道,元符則
 徐積,皆卓然較著者也。熙寧三年,諸路搜訪行義為鄉里推重者,凡二十有九人。至,則館之太學,而劉蒙以下二十二人試舍人院,賜官有差,亦足以見幽隱必達,治世之盛也。其後,應詔者多失實,而朝廷亦厭薄之。



 高宗垂意遺逸,首召布衣譙定,而尹焞以處士入講筵。其後束帛之聘,若王忠民之忠節,張志行之高尚,劉勉之、胡憲之力學,則賜出身,俾教授本郡,或賜處士號以寵之。所以振清節,厲頹俗。如徐庭筠之不出,蘇云卿之晦跡,
 世尤稱焉。寧宗慶元間,蔡元定以高明之資,講明一代正學,以尤袤、楊萬里之薦召之,固以疾辭,竟以偽學貶死,眾咸惜之。理、度以後,國勢日迫,賢者肥遁,迄無聞焉。



\end{pinyinscope}