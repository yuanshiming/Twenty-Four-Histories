\article{列傳第一百六十一}

\begin{pinyinscope}

 ○陳敏張詔畢再遇安丙楊巨源李好義



 陳敏,字元功,贛之石城人,父皓,有才武,建炎末,以破贛賊李仁功,補官至承信郎。敏身長六尺餘,精騎射,積官
 至忠靖郎。以楊存中薦,擢閣門祗候。時閩地多寇,殿司兵往戍,率不習水土。至是,始募三千兵置左翼軍,以敏為統制,漳州駐扎。敏按諸郡要害,凡十有三處,悉分兵扼之,盜發輒獲。贛州齊述據城叛,嘯聚數萬,將棄城南寇。每聞之曰:「贛兵精勁,善走嶮,若朝廷發兵未至,萬一奔沖,江、湖、閩、廣騷動矣。」不俟命,領所部馳七日,徑抵贛圍其城。逾月,朝廷命李耕以諸路兵至,破之。累功授右武大夫,封武功縣男,領興州刺史。召赴闕,高宗見其狀
 貌魁岸,除破敵軍統制。尋丁母憂,詔起復,以所部駐太平州。



 紹興三十一年,金主亮來攻,成閔為京湖路招討使,以敏軍隸之,升馬司統制,軍於荊、漢間。敏說閔曰:「金人精騎悉在淮,汴都必無守備,若由陳、蔡徑搗大梁,潰其腹心,此救江、淮之術也。」不聽。從閔還駐廣陵,時金兵尚未渡淮,敏又說閔邀其歸師,復不聽。敏遂移疾歸姑孰。



 孝宗即位,張浚宣撫江、淮,奏敏為神勁軍統制。浚視師,改都督府武鋒軍都統制。朝廷遣李顯忠北伐,浚欲
 以敏偕行,敏曰:「盛夏興師非時,且金人重兵皆在大梁,我客彼主,勝負之勢先形矣。願少緩。」浚不聽,令敏屯盱眙。顯忠至符離,果失律,敏遂入泗州守之。金人議和,詔敏退守滁陽。敏請於朝,謂滁非受敵之所,改戍高郵,兼知軍事。與金人戰射陽湖,敗之,焚其舟,追至沛城,復敗之。



 乾道元年,遷宣州觀察使,召除主管侍衛步軍司公事。居歲餘,敏抗章曰:「久任周廬,無以效鷹犬,況敵情多詐,和不足恃。今兩淮無備,臣乞以故部之兵,再戍高郵。」
 仍請更築其城。乃落常階,除光州觀察使,分武鋒為四軍,升敏為都統制兼知高郵軍事,仍賜築城屯田之費。敏至郡,板築高厚皆增舊制。自寶應至高郵,按其舊作石䃮十二所,自是運河通洩,無沖突患。



 四年,北界人侍旺叛於漣水軍,密款本朝,稱結約山東十二州豪傑起義,以復中原。上以問敏,敏曰:「旺欲假吾國威以行劫爾,必不能成事,願勿聽。」適屯田統領官與旺交通,旺敗,金有間言,上知非敏罪,乃召敏為左驍衛上將軍。



 言事者
 議欲戍守清河口,敏言:「金兵每出清河,必遣人馬先自上流潛渡,今欲必守其地,宜先修楚州城池,蓋楚州為南北襟喉,彼此必爭之地。長淮二千餘里,河道通北方者五,清、汴、渦、潁、蔡是也;通南方以入江者,惟楚州運河耳。北人舟艦自五河而下,將謀渡江,非得楚州運河,無緣自達。昔周世宗自楚州北神堰鑿老鸛河,通戰艦以入大江,南唐遂失兩淮之地。由此言之,楚州實為南朝司命,願朝廷留意。」及是,再出守高郵,乃詔與楚州守臣
 左祐同城楚州,祐卒,遂移守楚州。北使過者觀其雉堞堅新,號「銀鑄城」。



 以歸正人二百家逃歸,降授忠州團練使,罷為福建路總管,改江西路總管,贛州駐札。月餘,朝廷命往福州揀軍,又命還豫章教閱江西團結諸郡人馬。俄提舉祐神觀,仍奉朝請,繼復蘄州防禦使,再除武鋒軍都統制兼知楚州,復光州觀察使,以疾卒。特贈慶遠軍承宣使。



 張詔字君卿,成州人。少隸張俊帳下,積功守和州。嘗被
 旨介聘,一日金人持所繪祐、獻二陵像至館中,皆北地服,詔向之再拜。館者問之,答曰:「詔雖不識其人,但龍鳳之姿,天日之表,疑非北朝祖宗也,敢不拜!」孝宗聞而喜之,由是驟用。



 紹熙五年,除興州都統制兼知興州,代吳挺。慶元二年,趙彥逾帥蜀,以關外去興元遠,緩急恐失事機,復請分東西為二帥,詔遂兼西路安撫司公事。先是,趙汝愚為從官時,每奏吳氏世掌蜀兵,非國家之利,請以張詔代領武興之軍。蓋汝愚之意欲以吳曦為文
 臣帥,以杜他日握兵之漸,而未及行也。汝愚既知樞密院,力辭不拜,白於光宗曰:「若武興朝除帥,則臣夕拜命。」上許之,乃以詔為成州團練使、興州諸軍都統制。詔在興州,甚得士心。六年卒,郭杲代之。



 畢再遇,字德卿,兗州人也。父進,建炎間從岳飛護衛八陵,轉戰江、淮間,積階至武義大夫。再遇以恩補官,隸侍衛馬司,武藝絕人,挽弓至二石七斗,背挽一石八斗,步射二石,馬射一石五斗。孝宗召見,太悅,賜戰袍、金錢。



 開
 禧二年,下詔北伐,以殿帥郭倪招撫山東、京東,遣再遇與統制陳孝慶取泗州。再遇請選新刺敢死軍為前鋒,倪以八十七人付之。招撫司克日進兵,金人聞之,閉榷場、塞城門為備。再遇曰:「敵已知吾濟師之日矣,兵以奇勝,當先一日出其不意。」孝慶從之。再遇饗士卒,激以忠義,進兵薄泗州。泗有東西兩城,再遇令陳戈旗舟楫於石屯下,如欲攻西城者,乃自以麾下兵從陟山徑趨東城南角,先登,殺敵數百,金人大潰,守城者開北門遁。西
 城猶堅守,再遇立大將旗,呼曰:「大宋畢將軍在此,爾等中原遺民也,可速降。」旋有淮平知縣縋城而下乞降,於是兩城皆定。郭倪來饗士,出御寶刺史牙牌授再遇,辭曰:「國家河南八十有一州,今下泗兩城即得一刺史,繼此何以賞之?且招撫得朝廷幾牙牌來?」固辭不受。尋除環衛官。



 倪調李汝翼、郭倬取宿州,復遣孝慶等繼之。命再遇以四百八十騎為先鋒取徐州,至虹,遇郭、李兵裹創旋,問之,則曰:「宿州城下大水,我師不利,統制田俊邁
 已為敵擒矣。」再遇督兵疾趨,次靈壁,遇孝慶駐兵於鳳凰山,將引還,再遇曰:「宿州雖不捷,然兵家勝負不常,豈宜遽自挫!吾奉招撫命取徐州,假道於此,寧死靈壁北門外,不死南門外也。」會倪以書抵孝慶,令班師,再遇曰:「郭、李軍潰,賊必追躡,吾當自御之。」金果以五千餘騎分兩道來,再遇令敢死二十人守靈壁北門,自領兵沖敵陣。金人見其旗,呼曰「畢將軍來也」。遂遁。再遇手揮雙刀,絕水追擊,殺敵甚眾,甲裳盡赤,逐北三十里。金將有持
 雙鐵簡躍馬而前,再遇以左刀格其簡,右刀斫其脅,金將墮馬死。諸軍發靈壁,再遇獨留未動,度軍行二十餘里,乃火靈壁。諸將問:「夜不火,火今日,何也?」再遇曰:「夜則照見虛實,晝則煙埃莫睹,彼已敗不敢迫,諸軍乃可安行無虞。汝輩安知兵易進而難退邪?」



 還泗州,以功第一,自武節郎超授武功大夫,除左驍衛將軍。於是丘崇代鄧友龍為宣撫使,檄倪還惟揚,尋棄泗州。命再遇還盱眙,遂知盱眙軍,尋改鎮江中軍統制,兼守如故。以鳳凰
 山功,授達州刺史。其冬,金人以騎步數萬、戰船五百餘艘渡淮,泊楚州、淮陰間,宣撫司檄再遇援楚,遣段政、張貴代之。再遇既去盱眙,政等驚潰,金人入盱眙;再遇復定盱眙,除鎮江副都統制。



 金兵七萬在楚州城下,三千守淮陰糧,又載糧三千艘泊大清河。再遇謀知之,曰:「敵眾十倍,難以力勝,可計破也。」乃遣統領許俊間道趨淮陰,夜二鼓銜枚至敵營,各攜火潛入,伏糧車間五十餘所,聞哨聲舉火,敵驚擾奔竄,生擒烏古倫師勒、蒲察元
 奴等二十三人。



 金人復自黃狗灘渡淮,渦口戍將望風遁,濠、滁相繼失守,又破安豐。再遇謂諸將曰:「楚城堅兵多,敵糧草已空,所慮獨淮西耳。六合最要害,彼必並力攻之。」



 乃引兵赴六合。尋命節制淮東軍馬。金人至竹鎮,距六合二十五里。再遇登城,偃旗鼓,伏兵南土門,列弩手土城上,敵方臨濠,眾弩俱發,宋師出戰,聞鼓聲,城上旗幟並舉,金人驚遁,追擊大敗之。金萬戶完顏薄辣都、千戶泥龐古等以十萬騎駐成家橋、馬鞍山,進兵圍城
 數重,欲燒壩木,決壕水,再遇令勁弩射退之。既而紇石烈都統合兵進攻益急,城中矢盡,再遇令人張青蓋往來城上,金人意其主兵官也,爭射之,須臾矢集樓墻如蝟,獲矢二十餘萬。紇石烈引兵退,已乃益增兵,環城四面營帳亙三十里。再遇令臨門作樂以示閑暇,而間出奇兵擊之。敵晝夜不得休,乃引退。再遇料其且復來,乃自提兵奪城東野新橋,出敵之背,金人遂遁去,追至滁,大雨雪,乃旋。獲騾馬一千五百三十一、鞍六百,衣甲旗
 幟稱是。授忠州團練使。



 三年,除鎮江都統制兼權山東、京東招撫司事。還至揚州,除驍衛大將軍。金圍楚州已三月,列屯六十餘里。再遇遣將分道撓擊,軍聲大振,楚圍解。兼知揚州、淮東安撫使。揚州有北軍二千五百人,再遇請分隸建康、鎮江軍,每隊不過數人,使不得為變。更造輕甲,長不過膝,披不過肘,兜鍪亦殺重為輕,馬甲易以皮,車牌易以木而設轉軸其下,使一人之力可推可擎,務便捷不使重遲。敢死一軍,本烏合亡命,再遇能
 駕馭得其用。陳世雄、許俊等皆再遇所薦。張健雄恃勇桀驁,再遇狀其罪於朝,命以軍法戮之,諸將懾服。



 嘉定元年,除左驍衛上將軍。和好成,累疏乞歸田里,賜詔不允,除保康軍承宣使,降詔獎諭,尋令帶職奏事,提舉祐神觀。六年,提舉太平興國宮,十年,以武信軍節度使致仕。卒,年七十。贈太尉,累贈太師,謚忠毅。



 再遇姿貌雄傑,早以拳力聞,屬時寢兵,無所自見。一旦邊事起,諸將望風奔衄,再遇威聲始著,遂為名將雲。



 安丙,字子文,廣安人。淳熙間進士,調大足縣主簿。秩滿詣闕,陳蜀利病十五事,言皆剴切。丁外艱,服除,闢利西安撫司干辦公事,調曲水丞。吳挺為帥,知其才,邀致之。改秩,知新繁縣。丁內艱,服除,知小溪縣。通判隆慶府,嘉泰三年,郡大水,丙白守張鼎,發常平粟振之。尋又鑿石徙溪,自是無水患。知大安軍,歲旱,民艱食,丙以家財即下流糴米數萬石以振。事聞,詔加一秩。



 開禧二年,邊事方興,程松為四川宣撫使,吳曦副之,丙陳十可憂於松。
 繼而松開府漢中,道三泉,夜延丙議。丙又為松言曦必誤國,松不省。蓋丙嘗為其父客,素知曦。既而曦奏丙為隨軍轉運司,居河池。時梁、洋義士方襲取和尚原,旋為金人所奪,守將棄甲而走。十一月戊子,金人攻湫池堡,破天水,繇西和入成州,師潰,曦置不問。金人肆掠關外四州,如踐虛邑,軍民莫知死所。曦已潛遣其客姚淮源交金人,至是曦還興州,留丙魚關,已而檄還武興。十二月丙寅,金人持其詔及金印至罝口,曦密受之,宣言使
 者欲得四州以和,馳書諷松去。癸酉,曦受金詔稱蜀王,榜諭四川。三年正月甲午,曦僭號建官,稱臣於金,以其月為元年,改興州為興德府,以丙為中大夫、丞相長史、權行都省事。



 先是,從事郎錢鞏之從曦在河池,嘗夢曦禱神祠,以銀杯為珓擲之,神起立謂曦曰:「公何疑?公何疑?後政事已分付安子文矣。」曦未省,神又曰:「安子文有才,足能辦此。」鞏之覺,心異其事,具以語曦。事既熾,丙不得脫,度徒死無益,陽與而陰圖之。遂與楊巨源、李好義
 等謀誅曦,語見《巨源》、《好義傳》。



 徐景望在利州,逐土人,擅財賦。丙遣弟煥往約諸將,相與拊定,及景望伏誅,軍民無敢嘩者。於是傳檄諸道,按堵如故。曦僭位凡四十一日。三月戊寅,陳曦所以反及矯制平賊便宜賞功狀,自劾待罪,函曦首級、違制法物與曦所受金人詔印及所匿庚牌附驛。



 朝廷初聞變,莫知所為。韓侂胄與曦書,亦謂「嗣頒茅土之封」,亟召知鎮江府宇文紹節問之,紹節曰:「安丙非附逆者,必能討賊。」於是密降帛書曰:「安丙素
 推才具,有志事功,今聞曦謀不軌,爾為所脅,諒以兇焰方張,恐重為蜀禍,故權且從之爾,豈一日忘君父者?如能圖曦報國,以明本心,即當不次推賞,雖二府之崇亦無所吝,更宜審度機便,務在成事,以副委屬之意。」帛書未至,露布已聞,上下動色交慶。辛丑,加丙端明殿學士、中大夫、知興州、安撫使兼四川宣撫副使,詔獎諭,恩數視執政,如帛書旨也。



 時都統孫忠銳由鳳州進攻大散關不克,統領強德等出奇道由松林堡破金砦,四月癸
 丑,克之。忠銳貪功吝財,賞罰迷繆,大失軍心,且速還鳳州,以關鑰付庸將陳顯。癸酉,大散關復陷。巨源自請收復,丙遣朱邦寧佐之。丙深惡忠銳,檄赴司議事,欲廢之。巨源至鳳,斬忠銳及其子揆,丙遂以忠銳附偽進表之罪聞於朝。先是,以誅曦功,巨源補朝奉郎,與通判差遣。巨源遣其親校傳檜訴功於朝,語見《巨源傳》。於是丙拜疏丐閑。至是,金人揭示境上,得丙首者與銀絹二萬匹兩,即授四川宣撫。



 時方議和,丙獨戒飭將士,恫疑虛喝,
 以攻為守,威聲甚著。詔以蜀平,遣吳獵撫諭四川。時沿邊關隘悉為金毀,丙遺時相書,謂:「西和一面,已修仇池,聚糧積芻,使軍民可守。若敵至,則堅壁不戰,彼欲攻則不可,欲越則不敢。若西和可守,成州之境自不敢犯。成州黑谷、南谷亦皆頓重兵。天水雖不可守,距天水十里所,見創白環堡,與西和相為掎角,又增堡雞頭山,咸以民卒守之,及修黃牛堡,築興趙原,屯千餘人。鳳州秋防原尤為險絕,紹興初,州治於此,宣撫吳玠嘗作家計砦,
 前即馬嶺堡,正扼鳳州之後。凡此數堡既堅,金人決不敢近。而河池、殺金平、魚關皆大軍屯聚,其他徑路,雖關之裏如大安,亦陰招民卒,授以器械,為掩擊之備矣。」又云:「見於關表廣結義士,月給以糧,俾各保田廬墳墓,逮事定,則系之尺籍而勸之耕,庶可經久。以丙所見,直為守計,則精選五萬人亦為有餘。」



 好義守西和,謂四州兵後,民不聊生,請蠲租以惠創痍。丙請於朝。又以沔州都統司所統十軍權太重,故自吳璘至挺、曦皆有尾大不
 掉之憂,乃請分置副都統制,各不相隸,以前右中左後五軍隸都統司,踏白、摧鋒、選鋒、策鋒、游奕五軍隸副司。詔皆從之。



 時方信孺使還,金人和意未決,且欲得首議興師之人,侂胄大怒。上手書賜丙,謂:「金人必再至,當激勵將士,戮力赴功。」侂胄既誅,賜丙金器百二十兩、細幣二十匹,進資政殿學士。和議成,還大散、隔牙關。丙分遣僚吏,經量洋、沔、興元、大安民田,別定租稅。



 右丞相史彌遠起復,丙移書曰:「昔仁宗起復富鄭公、文潞公,孝宗起
 復蔣丞相,皆力辭,名教所系,人言可畏,望閣下速辭成命,以息議者之口。」論者韙之。



 升大學士、四川制置大使兼知興元府。諜知金人遷汴,關輔豪傑款塞願降者眾。丙以為此正冉閔告晉之時,乃與宰臣書,謂當興問罪之師。朝論憂丙輕舉,乃詔丙益修守備。



 七年春,丙使所愛吏安蕃、何九齡合官軍夜襲秦州,敗歸。王大才執九齡等七人斬之,而訟丙於朝。三月,詔丙同知樞密院事兼太子賓客,賜手書召之。行次廣德軍,進觀文殿學士、
 知潭州、湖南安撫使。至官,留意學校,請於太常創大成樂。



 而政尚嚴酷,轉運判官章徠劾丙,不報。御史李安行並徠劾之,徠罷,丙授崇信軍節度使、開府儀同三司、萬壽觀使。譴閣門舍人聞人璵錫命,賜旌節、金印、衣帶、鞍馬。三辭,還蜀。



 董居誼帥蜀,大失士心。金人乘之,破赤丹、黃牛堡,入武休關,直搗梁、洋,至大安,宋師所至輒潰,散入巴山。十二年,聶子述代之。時丙之子癸仲知果州,子述即檄兼參議官。四月,紅巾賊張福、莫簡叛,入利州,子
 述遁去。總領財賦楊九鼎與賊遇,走匿民舍,賊追九鼎殺之。子述退保劍門,檄癸仲兼節制軍馬,任討賊之責。癸仲召戎帥張威等軍來會,賊自閬趨遂寧,所過無不殘滅。丙欲自持十萬緡偕子述往益昌募士,子述曰:「大臣非得上旨,未可輕出。」丙遂如果州。



 時四川大震,甚於曦之變。張方首奏,勛望如丙,今猶可用。魏了翁移書宰執,謂安丙不起,則賊未即平,蜀未可定,雖賊亦曰:「須安相公作宣撫,事乃定耳。」



 李壁、李𡌴時並鎮潼、遂,亦皆以
 國事勉丙。五月乙未,丙至果州,是日賊焚蓬溪縣。



 己酉,詔起丙為四川宣撫使,予便宜,尋降制授保寧軍節度使兼知興元府、利東安撫使。丙奏:「臣不辭老以報國,但事不任怨,難以圖成,將恐騰謗交攻,使臣獨抱赤心,無從上白。昔秦使甘茂攻宜陽,至質之以『息壤在彼』,魏使樂羊攻中山,至示之以謗書一篋。君臣之間,似不必爾。然自古及今,謗以疑間而成,禍以忌嫉而得;況臣已傷弓於既往,豈容不懲沸於方來。」詔曰:「昔唐太宗以西寇
 未平,詔起李靖,靖慷慨請行,不以老疾為解。代宗有朔方之難,圖任郭子儀,聞命引道,亦不以讒惎自疑。皆能乘時立功,焜燿竹帛,朕甚慕之。今蜀道俶擾,未寬顧憂,朕起卿燕閑,付以方面,而卿忠於報國,誼不辭難,朕之用人庶幾於唐宗,卿之事朕無愧於李、郭矣。勉圖雋功,以濟國事!」尋命丁焴改知興元府。



 甲申,發果州。丙戌,至遂寧,賊猶負固於普州之茗山。丙下令諸軍合圍,絕其樵汲之路以困之。未幾,張威、李貴俘獲張福等十七人
 以獻,丙命臠王大才以祭九鼎。七月庚子,盡俘餘黨千餘人,皆斬之。庚戌班師,乃移治利州,賜保寧軍節度使印。癸仲亦加三秩,進直華文閣,起復,主管宣撫司機宜文字。明年,進丙少保,賜衣帶鞍馬。



 丙以關表營田多遺利,命官括之。有文垓者方持母喪,以便宜起復,乾辦魚關糧料院,俾之措置,且以宣撫副使印假之。而馮安世者,又即利州置根括局。於是了翁遺丙書,謂:「幕府舉闢,當用經術信厚之士,不可用冒喪之人。且公八年鎮蜀,
 有恩則有怨,豈可人人而校,事事而理,自處甚狹,恐貽子孫賓客無窮之累。



 雖今日理財難拘故常,然告絕產、首白契、訐隱田、伺富民過失、糾鹽酒戶虧額,報怨挾憤、招權納賄者,必且紛然,而公任其怨。」丙復書曰:「關外糴買當用四百萬緡,而總所見緡止二十五萬,多方措置,非得已而不已。儻皆清流,何由辦事?



 蜀士中如令弟嘉父、李成之輩,清則清、高則高矣,其肯辦錢穀俗務乎。劉德修嘗雅責楊嗣勛不能舉義誅叛,嗣勛云:『德修特未
 當局耳。』丙於華父亦云。」其後,安世不法滋甚,近臣有以書抵丙,而安世之徒亦發其事,丙械送大安窮治之。



 先是,夏人來乞師並兵攻金人,丙且奏且行,分遣將士趨秦、鞏、鳳翔,委丁焴節制,師次於鞏。夏人以樞密使子寧眾二十餘萬,約以夏兵野戰,宋師攻城。既而攻鞏不克,乃已。



 丙卒,訃聞,以少傅致仕,輟視朝二日,贈少師,賻銀絹千計,賜沔州祠額為英惠廟。理宗親札賜謚忠定。丙所著有《皛然集》。



 楊巨源字子淵,其先成都人。父信臣,客益昌,因家焉。巨源倜儻有大志,善騎射,涉獵諸子百家之書。應進士不中,武舉又不中。劉光祖見而異之,薦之總領錢糧陳曄,以右職舉為鳳州堡子原倉官,馳騁射獵,傾財養士,沿邊忠義,咸服其才。分差魚關糧料院,移監興州合江贍軍倉。



 吳曦叛,巨源陰有討賊志,結義士三百人,給其錢糧。有游奕軍統領張林者,力能挽兩石弓,隊將朱邦寧身長六尺,勇力過人,皆為曦所忌,雖屢戰有功亦不加
 賞,林等憾之。時林在罝口,邦寧在合江,巨源因與深相締結,並集忠義人朱福、陳安、傅檜之徒。



 曦脅安丙為丞相長史,丙稱疾,眉士程夢錫見丙,丙嘆曰:「世事如此,世無豪傑!」夢錫因及巨源之謀。丙曰:「肯見我乎?」乃囑夢錫以書致巨源,延之臥所。巨源曰:「先生而為逆賊丞相長史耶?」丙號哭曰:「目前兵將,我所知,不能奮起。必得豪傑,乃滅此賊,則丙無復憂。」巨源曰:「先生之意決乎?」丙指天誓曰:「若誅此賊,雖死為忠鬼,夫復何恨!」巨源大喜,曰:「
 非先生不足以主此事,非巨源不足以了此事。」



 當是時,李好義、好問亦結李貴、楊君玉、李坤辰凡數十人,坤辰邀巨源與好義會。巨源又大喜曰:「吾與安長史議以三月六日邀曦謁廟,合勇士刺之。」好義曰:「彼出則齪巷,從衛且千人,事必難濟。聞熟食日祭東園,圖之此其時也。」



 巨源然之。好義願一見長史以為信。巨源曰:「吾今先為長史言之,來日偽宮,令長史問君先世是已。」巨源以告丙,明日,好義在偽宮見丙,揖之。丙曰:「鄉與尊父同僚,楊
 省乾盛談才略,旦夕以職事相委。」其謀乃決。



 君玉先屬其鄉人白子申擬詔,文不雅馴,巨源更為之,例用合江倉朱記。巨源、好義憂事浸洩,遂以二月乙亥未明,好義率其徒入偽宮,巨源持詔乘馬,自稱奉使,入內戶,曦啟戶欲逸,李貴執殺之。衛者始拒鬥,聞有詔皆卻。巨源、好義迎丙宣詔,以曦首徇。三軍推丙權四川宣撫使,巨源權參贊軍事。丙奏功於朝,以巨源第一,詔補承事郎。



 巨源謂丙曰:「曦死,賊膽以破,關外四州為蜀要害,盍乘勢
 復取。」好義亦以為言。丙慮軍無見糧,巨源力言四州不取,必有後患,自請為隨軍措置糧運。於是分遣好義復西和州,張林、李簡復成州,劉昌國復階州,孫忠銳復散關。俄詔巨源轉朝奉郎,與通判差譴,兼四川宣撫使司參議官。丙素惡忠銳,聞忠銳失守散關,檄其還,欲廢之,先命巨源偕邦寧以沔兵二千策應。巨源至鳳州,因忠銳出迎,伏壯士於幕後,突出斬之,並其子揆。丙遂以忠銳附偽賀表聞於朝,且待罪。



 先是,獎諭誅叛詔書至沔
 州,巨源謂人曰:「詔命一字不及巨源,疑有以蔽其功者。」俄報王喜授節度使,巨源彌不平。時趙彥吶以在夔誅祿禧得州通判,巨源曰:「殺祿禧與通判,殺吳曦亦與通判耶?」以啟謝丙曰:「飛矢以下聊城,深慕魯仲連之高誼;解印而去彭澤,庶幾陶靖節之清風。」又遣訴功於朝,而從興元都統制彭輅乞書遺韓侂胄,輅陽許而陰以白丙。或言巨源與其徒米福、車彥威謀為亂,丙命喜鞫之,福、彥威皆抵罪。正將陳安復告巨源結死士入關,欲焚
 沔州州治,俟丙出則殺之。丙積前事,因欲去巨源,然未有以發也。



 會巨源在鳳州以檄書遺金鳳翔都統使,其辭若用間者,且自稱宣撫副使而以參議官印印之。金以檄至丙。巨源方與金戰,敗於長橋,丙乃移書召巨源,巨源疑焉。



 有梁泉主簿高岳成者,巨源薦為隨軍撥運,來見巨源,贊其歸,巨源信之。



 時輅已至沔,六月壬申,巨源還幕府,丙密命輅收巨源。巨源殊不知,以為謁己也,語畢,輅起,巨源送之賓次。武士就挽其裾,巨源猶叱之,
 則已為驅至庭下。



 巨源大呼曰:「我何罪?」丙隔屏遣人謂之曰:「若為詐稱宣撫副使?」命械送閬州獄。巨源曰:「我一時用間,異時必有為我明其事。」丙餉以肴酒,巨源曰:「一身無愧,死且無憾;惟有妹未嫁,宣撫念之。」癸酉,巨源舟抵大安龍尾灘,將校樊世顯者呼於岸,巨源知將見殺,指其地而語之曰:「此好一片葬地。」世顯曰:「安有是?」舟行數步,謂曰:「宣參久渴,莫進杯酒?」巨源辭以不飲。又曰:「宣參荷械已久,盍少蘇?」巨源未及答,左右遽取利刀斷其
 頭,不絕者逾寸,遂以巨源自殪聞宣撫司。後數日,丙命瘞之。



 巨源死,忠義之士為之扼腕,聞者流涕,劍外士人張伯威為文以吊,其辭尤悲判。巨源之屬吏也,李壁在政府,聞之曰:「嘻,巨源其死矣!」丙以人情洶洶,封章求免。楊輔亦謂丙殺巨源必召變,請以劉甲代之。初,巨源與好義結官軍,而丙密為反正之計,各未相知,合巨源於好義者李坤辰,而合好義於丙者巨源也。巨源遺光祖書,述丙酬答之語,鋟梓競傳之,丙已弗樂,浸潤不已,積
 成此禍。



 成忠郎李珙投軌,獻所作《巨源傳》為之訟冤,朝廷亦念其功,賜廟褒忠,贈寶謨閣待制,官其二子。制置使崔與之請官給其葬,加贈寶謨閣直學士、太中大夫。



 嘉熙元年,理宗特賜謚忠愍。子履正終大理卿、四川制置副使。



 李好義,下邽人。祖師中,建炎間以白丁守華州,積官忠州團練使。父定一,興州中軍統制。好義弱冠從軍,善騎射,西邊第一。初以準備將討文州蕃部有功,開禧初,韓
 侂胄開邊,吳曦主師,好義為興州正將,數請出精兵襲金人,曦蓄異謀,不納。未幾,關外四州俱陷,金人長驅入散關,曦受金人說,以蜀叛。好義自青坊聞變亟歸,與其兄對哭,謀誅之。



 會曦遣李貴追殺宣撫程松,貴語其徒曰:「程宣撫朝廷重臣,不可殺。」好義知其赤心,可以所謀告之。貴遂約李彪、張淵、陳立、劉虎、張海等,好義又密結親衛軍黃術、趙亮、吳政等。女弟夫楊君玉亦與知,好義戒言曰:「此事誓死報國,救四蜀生靈,慎毋洩。」留其母以
 質。好義兄弟謀曰:「今日人皆可殺曦,皆可為曦,曦死後,若無威望者鎮撫,恐一變未息,一變復生。」欲至期立長史安丙以主事,蓋曦嘗授丙偽丞相,而丙托疾不往,故兄弟有是謀也。



 既而君玉與李坤辰者來,坤辰因言丙亦與合江倉楊巨源陰結忠義欲圖曦。好義遂遣君玉偕坤辰約巨源以報丙。丙大喜曰:「非統制李定一之子乎?此人既來,斷曦之臂矣。」遂與好義約二月晦舉事,見《巨源傳》。乃約彪、術、貴等七十有四人及士人路良弼、王
 芾。好義夜饗士,麾眾受甲,與好古、好仁及子姓拜決於家廟,囑妻馬氏曰:「日出無耗,當自為計,死生從此決矣。」馬氏叱之曰:「汝為朝廷誅賊,何以家為?我決不辱李家門戶。」馬氏之母亦曰:「行矣,勉之!汝兄弟生為壯夫,死為英鬼。」好義喜曰:「婦人女子尚念朝廷不愛性命,我輩當如何?」



 眾皆踴躍。既行,小將祿禕引十卒來助,各以黃帛為號。好義誓於眾曰:「入宮妄殺人、掠財物者死。」



 時偽宮門洞開,好義大呼而入曰:「奉朝廷密詔,安長史為宣撫,
 令我誅反賊,敢抗者夷其族。」曦護衛千兵皆棄梃而走,遂至偽殿東角小門,入世美堂,近曦寢室。曦聞外哄,倉皇而起,露頂徒跣,開寢戶欲遁,見貴復止,以手捍內戶,貴前爭戶,戶紐折。曦走,貴追及,手執其髻,舉刃中曦頰,曦素勇有力,撲貴僕於地不能起。好義急呼王換斧其腰者二,曦負痛手縱,貴起遂斫其首。引眾擁曦首出偽宮,亟馳告丙宣詔,軍民拜舞,歡聲動天地,持曦首撫定城中,市不改肆。



 好義請乘時取關外四州,巨源贊之,丙
 大喜。巨源輔行,王喜忌其能,沮之。



 好義曰:「西和乃腹心之地,西和下,則三州可不戰而復矣。今不圖,後悔無及。



 願得馬步千人,死士二百,齎十日糧可濟。」丙從其請,忠義響應,次獨頭嶺,進士王榮仲兄弟率民兵會合夾擊,金人死者蔽路。十戰至山砦高堡,七日至西和。好義率眾攻城,親犯矢石,人人樂死,以少擊眾,前無留敵。金西和節使完顏欽奔遁,好義整眾而入,軍民歡呼迎拜,籍府庫以歸於官。



 好義初欲乘勝徑取秦、隴以牽制淮寇,
 而宣撫司令謹守故疆,不得侵越,士氣皆沮。好義以中軍統制知西和州,卒。丙以勞績上於朝,特贈檢校少保,仍給田以贍其家。後吳獵為請謚曰忠壯。好義喜誦《孟子》及《左傳》,以為終身行此足矣。



 誅曦時,惟幼子植留家。迄事,人爭冒功賞,君玉欲注植名,好義指心曰:「惟此物不可欺。」



 曦既誅,好義集於丙家,王喜後至,心懷邪謀,欲刃好義,丙力救解,然日以殺好義為心。及好義守西和,喜遣其死黨劉昌國聽節制,好義與之酬酢,歡飲達旦,
 好義心腹暴痛洞瀉,而昌國遁矣。既殮,口鼻爪指皆青黑,居民莫不冤之,號慟如私親,摧鋒一軍幾至於變。既而昌國白日見好義持刃刺之,驚怖僕地,疽發而殂。



 喜,曦大將也,貪淫狠愎,誅曦之日不肯拜詔,遣其徒入偽宮虜掠殆盡,又取曦姬妾數人。其後欲戕好義為曦復仇,丙不能止,便宜處以節度使知興州,而恨猶未已。嘗出兵於船柵嶺,鋒未及交,棄軍先遁,金人遂由黑谷長驅入境。朝廷慮喜為變,授節度使移荊鄂都統制而死。



 論曰:陳敏善守,畢再遇善戰。張詔出使不辱國,為將得士心,趙汝愚薦為武興帥,以其才足以制曦也。曦之畔,向非安丙、楊巨源、李好義之謀,西方之憂莫大焉。然丙卒以是殺巨源,何其媢疾而殘賊也?李好義失於周防,竟為王喜所圖。



 宋知喜為曦黨,既不能罪,又以節鎮賞之,幾何而不為唐末之姑息以成藩鎮之禍乎?



\end{pinyinscope}