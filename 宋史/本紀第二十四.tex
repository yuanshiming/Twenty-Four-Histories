\article{本紀第二十四}

\begin{pinyinscope}

 高宗一



 高宗受命中興全功至德聖神武文昭仁憲孝皇帝,諱構,字德基,徽宗第九子,母曰顯仁皇后韋氏。大觀元年五月乙巳生東京之大內,赤光照室。八月丁丑,賜名,授
 定武軍節度使、檢校太尉,封蜀國公。二年正月庚申,封廣平郡王。宣和三年十二月壬子,進封康王。資性朗悟,博學強記,讀書日誦千餘言,挽弓至一石五斗。宣和四年,始冠,出就外第。



 靖康元年春正月,金人犯京師,軍於城西北,遣使入城,邀親王、宰臣議和軍中。朝廷方遣同知樞密院事李梲等使金,議割太原、中山、河間三鎮,遣宰臣授地,親王送大軍過河。欽宗召帝諭指,帝慷慨請行。遂命少宰張邦昌為計議使,與帝俱。金帥斡離不留
 之軍中旬日,帝意氣閑暇。二月,會京畿宣撫司都統制姚平仲夜襲金人砦不克,金人見責,邦昌恐懼涕泣,帝不為動,斡離不異之,更請肅王。癸卯,肅王至軍中,許割三鎮地。進邦昌為太宰,留質軍中,帝始得還。金兵退,復遣給事中王云使金,以租賦贖三鎮地。又以蠟書結遼降將耶律餘睹,為金人所得。八月,金帥粘罕復引兵深入,陷太原。斡離不破真定。冬十月,王云從吏自金先還,言金人須帝再至乃議和。雲歸,言金人堅欲得地,不然,
 進兵取汴都。十一月,詔帝使河北,奉袞冕、玉輅,尊金主為伯,上尊號十八字。被命,即發京師。以門下侍郎耿南仲主和議,請與俱,乃以其子中書舍人延禧為參議官偕行。帝由滑、浚至磁州,守臣宗澤請曰:「肅王去不返,金兵已迫,復去何益?請留磁。」磁人以雲將挾帝入金,遂殺雲。時粘罕、斡離不已率兵渡河,相繼圍京師。從者以磁不可留,知相州汪伯彥亦以蠟書請帝還相州。



 閏月,耿南仲馳至相,見帝致辭,以面受欽宗之旨,盡起河北兵
 入衛,帝乃同南仲募兵勤王。初,朝廷聞金兵渡河,欲拜帝為元帥。至是,殿中侍御史胡唐老復申元帥之議,尚書右僕射何□擬詔書以進,欽宗遣閣門祗候秦仔持蠟詔至相,拜帝為河北兵馬大元帥,知中山府陳亨伯為元帥,汪伯彥、宗澤為副元帥。仔於頂發中出詔,帝讀之嗚咽,兵民感動。



 十二月壬戌朔,帝開大元帥府,有兵萬人,分為五軍,命武顯大夫陳淬都統制軍馬。閣門祗候侯章繼蠟書至自京師,詔帝盡發河北兵,命守臣自
 將。帝乃下令諸郡守與諸將,議引兵渡河。乙亥,帝率兵離相州。丙子,履冰渡河。丁丑,次大名府。宗澤以二千人先諸軍至,知信德府梁揚祖以三千人繼至,張俊、苗傅、楊沂中、田師中皆在麾下,兵威稍振。會簽書樞密院事曹輔繼蠟詔至,云金人登城不下,方議和好,可屯兵近甸,毋輕動。汪伯彥等皆信和議,惟宗澤請直趨澶淵為壁,次第解京城之圍。伯彥、南仲請移軍東平。帝遂遣澤以萬人進屯澶淵,揚言帝在軍中。自是澤不復預府中
 謀議。帝決意趨東平。庚寅,帝發大名。



 建炎元年春正月癸巳,帝至東平。初,帝軍在相州,京城圍久,中外莫知帝處。及是,陳請四集,取決帥府。壬寅,高陽關路安撫使黃潛善、總管楊惟忠亦部兵數千至東平。命潛善進屯興仁,留惟忠為元帥都統制。金人聞帝在澶淵,遣甲士及中書舍人張澄來召。宗澤命壯士射之,澄乃遁。伯彥等請帝如濟州。二月庚辰,發東平。癸未,次濟州。時帥府官軍及群盜來歸者號百萬人,分屯濟、
 濮諸州府,而諸路勤王兵不得進。二帝已在金人軍中。三月丁酉,金人立張邦昌為帝,稱大楚。黃潛善以告,帝慟哭,僚屬欲奉帝駐軍宿州,謀渡江左,帝聞三軍籍籍,遂輒。承制以宗澤為徽猷閣待制。丁巳,斡離不退師,徽宗北遷。戊午,承制以汪伯彥為顯謨閣待制,充元帥;潛善為徽猷閣待制,充副元帥。夏四月,粘罕退師,欽宗北遷。癸亥,邦昌尊元祐皇后為宋太后,遣人至濟州訪帝,又遣吏部尚書謝克家來迎。耿南仲率幕僚勸進,帝避
 席流涕,遜辭不受。伯彥等引天命人心為請,且謂靖康紀元,為十二月立康之兆。帝曰:「當更思之。」以知淮寧府趙子崧為寶文閣學士、元帥府參議官、東南道總管,統東南勤王兵。邦昌遣閣門宣贊舍人蔣師愈等持書詣帝,自言從權濟事,及將歸寶避位之意。帝亦貽諸帥書,以未得至京,已至者毋輒入。聞資政殿大學士、領開封府事李綱在湖北,遣劉默持書訪之。又諭宗澤等,以受偽命之人義當誅討,然慮事出權宜,未可輕動。澤復書
 謂邦昌篡亂蹤跡已無可疑,宜早正天位,興復社稷,不可不斷。門下侍郎呂好問亦以蠟書來,言帝不自立,恐有不當立而立者。丁卯,謝克家以「大宋受命之寶」至濟州,帝慟哭跪受,命克家還京師,趣辦儀物。戊辰,濟州父老詣軍門,言州四旁望見城中火光屬天,請帝即位於濟。會宗澤來言,南京乃藝祖興王之地,取四方中,漕運尤易。遂決意趨應天。是夕,邦昌手書上延福宮太后尊號曰元祐皇后,入居禁中,以尚書左丞馮澥為奉迎使。
 皇后又遣兄子衛尉少卿孟忠厚持手書遺帝。皇后垂簾聽政。邦昌權尚書左僕射,率在京百官上表勸進,不許。甲戌,皇后手書告中外,俾帝嗣統。乙亥,百官再上表,又不許。丁丑,馮澥等至濟州,百官三上表,許以權聽國事。戊寅,命宗澤先勒兵分駐長垣、韋城等縣,以備非常。東道副總管朱勝非至濟州,宣撫司統制官韓世忠以兵來會。庚辰,帝發濟州,鄜延副總管劉光世自陜州來會,以光世為五軍都提舉。辛巳,次單州。壬午,次虞城縣。
 西道都總管王襄自襄陽來會。癸未,至應天府。皇后詔有司備法駕儀仗。乙酉,張邦昌至,伏地慟哭請死,帝慰撫之。承制以汪伯彥為顯謨閣直學士,黃潛善為徽猷閣直學士。權吏部尚書王時雍等奉乘輿服御至,群臣勸進者益眾,命有司築壇府門之左。



 五月庚寅朔,帝登壇受命,禮畢慟哭,遙謝二帝,即位於府治。改元建炎。大赦,常赦所不原者咸赦除之。張邦昌及應於供奉金國之人,一切不問。命西京留守司修奉祖宗陵寢。罷天下
 神霄宮。住散青苗錢。應死節及歿於王事者並推恩。奉使未還者,祿其家一年。應選人並循資,已系承直郎者,改次等京官。臣僚因亂去官者,限一月還任。潰兵、群資咸許自新。免系官欠負,蠲南京及元帥府常駐軍一月以上州縣夏稅。應天府特奏名舉人並與同進士出身,免解人與免省試。諸路特奏名三舉以上及宗室嘗預貢者,並推恩。應募兵勤王人以兵付州縣主兵官,聽赴行在。中外臣庶許言民間疾苦,雖詆訐亦不加罪。命官犯
 罪,更不取特旨裁斷。蔡京、童貫、朱勉、李彥、孟昌齡、梁師成、譚稹及其子孫,更不收敘。內外大臣,限十日各舉布衣有材略者一人。餘如故事。以黃潛善為中書侍郎,汪伯彥同知樞密院事。元祐皇后在東京,是日徹簾。辛卯,遙尊乾龍皇帝為孝慈淵聖皇帝,元祐皇后為元祐太后。詔史官辨宣仁聖烈皇后誣謗。築景靈宮於江寧府。壬辰,以張邦昌為太保、奉國軍節度使、同安郡王,五日一赴都堂參決大事。以河東、北宣撫使範訥為京城留
 守。癸巳,遙尊帝母韋賢妃為宣和皇后,遙立嘉國夫人邢氏為皇后。耿南仲罷。甲午,以李綱為尚書右僕射兼中書侍郎,趣赴行在,楊惟忠為建武軍節度使,主管殿前司公事。罷諸盜及民兵之為統制者,簡其士馬隸五軍。乙未,以生辰為天申節。馮澥罷,以兵部尚書呂好問為尚書右丞。命中軍統制馬忠、後軍統制張忭率兵萬人,趣河間府追襲金人。丙申,以呂好問兼門下侍郎。丁酉,以黃潛善兼御營使,汪伯彥副之,真定府路副總管
 王淵為都統制,鄜延路副總管劉光世提舉一行事務。王時雍黃州安置。命統制官薛廣、張瓊率兵六千人會河北山水砦義兵,共復磁、相。戊戌,以資政殿學士路允迪為京城撫諭使,龍圖閣學士耿延禧副之。贈吏部侍郎李若水觀文殿學士,謚忠愍。己亥,召太學生陳東赴行在。李綱至江寧,誅叛卒周德等。庚子,詔:以靖康大臣主和誤國,責李邦彥為建寧軍節度副使、潯州安置,徙吳敏柳州,蔡懋英州。李梲、宇文虛中、鄭望之、李鄴皆以
 使金請割地,責廣南諸州並安置。辛丑,詔張邦昌知幾達變,勛在社稷,如文彥博例,月兩赴都堂。壬寅,封後宮潘氏為賢妃。以江、淮發運使梁揚祖提領東南茶鹽事。癸卯,天申節,罷百官上壽。乙己,賜諸路勤王兵還營者錢,人三千。丙午,以誣謗宣仁聖烈皇后,追貶蔡確、蔡卞、邢怒、蔡懋官。以保靜軍節度使姚古知河南府。金人陷河中府,權府事郝仲連死之。丁未,徽宗至燕山府。庚戌,以宗澤為龍圖閣學士、知襄陽府。壬子,進張邦昌太傅。
 丙辰,罷監察御史張所,尋責江州安置。丁巳,詔成都、京兆、襄陽、荊南、江寧府、鄧、揚二州儲資糧,修城壘,以備巡幸。以簽書樞密院事張叔夜嘗援京城力戰,從徽宗北行,遙命為觀文殿大學士、醴泉觀使。戊午,右諫議大夫範宗尹罷。遣太常少卿周望使河北軍前通問二帝。西道總管王襄、北道總管趙野坐勤王稽緩,並分司,襄陽府、青州居住。尋責襄永州、野邵州,並安置。



 六月己未朔,李綱入見,上十議,曰國是、巡幸、赦令、僭逆、偽命、戰、守、本
 政、責成、修德。以前殿前副都指揮使王宗濋引衛兵遁逃,致都城失守,責官、邵州安置。徽猷閣直學士徐秉哲假資政殿學士,為大金通問使,秉哲辭。庚申,封靖康軍節度使仲湜嗣濮王。粘罕還屯雲中。辛酉,命新任郎官未經上殿者並引對。御史中丞顏岐罷。徐秉哲責官、梅州安置。詔河北、京、陜、淮、湖、江、浙州軍縣鎮募人修築城壁。壬戌,置登聞檢鼓院。癸亥,以黃潛善為門下侍郎兼權中書侍郎。張邦昌坐僭逆,責降昭化軍節度副使、潭
 州安置。及受偽命臣僚:王時雍高州,吳開永州,莫儔全州,李擢柳州,孫覿歸州,並安置。顏博文、王紹以下論罪有差。以知懷州霍安國、河東宣撫使劉韐死節,贈安國延康殿學士,韐資政殿大學士。甲子,命李綱兼御營使。乙丑,以龍神衛四廂都指揮使馬忠為河北經制使,措置民兵。洪芻罷左諫議大夫,下臺獄。丁卯,以祠部員外郎喻汝礪為四川撫諭,督漕計羨緡及常平錢物。罷開封、諸州、軍、府司錄曹掾官。州、軍通判二員者省其一。權
 減宰執奉賜三之一。省諸路提舉常平司、兩浙、福建提舉市舶司。賊李孝忠寇襄陽,守臣黃叔敖棄城遁。立格買馬。辛未,以子敷生,大赦。籍天下神霄宮錢穀充經費。拘天下職田錢隸提刑司。還元祐黨籍及上書人恩數。癸酉,詔陜西、山東諸路帥臣團結軍民,互相應援。乙亥,增諸縣弓手,置武尉領之。宗室叔向以所募勤王兵屯京師,或言為變,命劉光世捕誅之。戊寅,以汪伯彥知樞密院事。遣宣義郎傅雱使河東軍前,通問二帝。己卯,置
 沿河、沿淮、沿江帥府十有九,要郡三十九,次要郡三十八,帥守兼都總管,守臣兼鈐轄、都監,總置軍九十六萬七千五百人。別置水軍七十七將,造舟江、淮諸路。置三省、樞密院賞功司。東京留守範訥落節鉞、淄州居住。庚辰,以二帝未還,禁州縣用樂。辛巳,置沿河巡察六使。壬午,以戶部尚書張愨同知樞密院事兼提舉措置戶部財用。癸未,呂好問罷。甲申,並尚書戶部右曹所掌歸左曹,命尚書總領。乙酉,以宗澤為東京留守,杜充為北京
 留守,罷監司州郡職田。丙戌,詔陜西、河北、京東西路募兵合十萬人,更番入衛行在。命京東、西路造戰車。丁亥,以張所為河北西路招撫使。括買官民馬,勸出財助國。戊子,以錢蓋為陜西經制使,封趙懷恩為安化郡王,因召五路兵赴行在。



 秋七月己丑朔,以樞密副都承旨王□燮為河東經制使。庚寅,詔王淵、劉光世、統制官張俊、喬仲福、韓世忠分討陳州軍賊杜用、京東賊李昱及黎驛、魚臺潰兵,皆平之。辛卯,籍東南諸州神霄宮及贍學錢
 助國用。叔右監門衛大將軍、貴州團練使士珸以磁、洺義兵復洺州。乙未,以溫州觀察使範瓊為定武軍承宣使、御營司同都統制。丙申,賜諸路強壯巡社名為「忠義巡社」,專隸安撫司。戊戌,欽宗至燕山府。以忻州觀察使張忭為河北制置使。東都宣武卒杜林謀據成都叛,伏誅。己亥,詔臺省、寺監繁簡相兼,學官、館職減舊制之半。辛丑,復議吳開、莫儔等十一人罪,並廣南、江、湖諸州安置,餘遞貶有差。壬寅,詔:「奉元祐太后如東南,六宮及衛
 士家屬從行,朕當獨留中原,與金人決戰。」以延康殿學士許翰為尚書右丞。甲辰,以右諫議大夫宋齊愈當金人謀立異姓,書張邦昌姓名,斬於都市。乙巳,手詔:「京師未可往,當巡幸東南。」丙午,詔定議巡幸南陽。以觀文殿學士範致虛知鄧州,修城池,繕宮室,輸錢穀以實之。丁未,遣官詣京師迎奉太廟神主赴行在。己酉,罷四道都總管。以尚書虞部員外郎張浚為殿中侍御史。庚戌,徵諸道兵,期八月會行在。丙辰,徽宗自燕山密遣閣門宣
 贊舍人曹勛至,賜帝絹半臂,書其領曰:「便可即真,來援父母。」帝泣以示輔臣。張所、傅亮軍發行在。是月,關中賊史斌犯興州,僭號稱帝。



 八月戊午朔,洪芻等坐圍城日括金銀自盜,及私納宮人,芻及餘大均、陳沖貸死,流沙門島,餘五人罪有差。勝捷軍校陳通作亂於杭州,執帥臣葉夢得,殺漕臣吳昉。己未,元祐太后發京師。庚申,以劉光世為奉國軍節度使,韓世忠、張俊皆進一官。辛酉,右司諫潘良貴罷。壬戌,以李綱為尚書左僕射兼門下
 侍郎,黃潛善為右僕射兼中書侍郎,張愨兼御營副使。癸亥,命御營使、副大閱五軍。庚午,更號元祐太后為隆祐太后。辛未,罷傅亮經制副使,召赴行在。壬申,召布衣譙定赴行在。命御營統制辛道宗討陳通。是夕,東北方有赤氣。癸酉,以耿南仲主和誤國,南雄州安置。乙亥,用張浚言,罷李綱左僕射。丙子,隆祐太后發南京,命侍衛馬軍都指揮使郭仲荀護衛如江寧,兼節制江、淮、荊、浙、閩、廣諸州,制置東南盜賊。丁丑,以龍圖閣直學士錢伯
 言知杭州,節制兩浙、淮東將兵及福建槍杖手,討陳通。庚辰,降榜招諭杭州亂兵。壬午,用黃潛善議,殺上書太學生陳東、崇仁布衣歐陽澈。乙酉,遣兵部員外郎江端友等撫諭閩、浙、湖、廣、江、淮、京東西諸路,及體訪官吏貪廉、軍民利病。許翰罷。丁亥,博州卒宮儀作亂,犯萊州。



 九月己丑,建州軍校張員等作亂,執守臣張動,轉運副使毛奎、判官曹仔為所殺,嬰城自守。範瓊捕斬李孝忠於復州。壬辰,以金人犯河陽、汜水,詔擇日巡幸淮甸。鑄
 建炎通寶錢。命淮、浙沿海諸州增修城壁,招訓民兵,以備海道。甲午,命揚州守臣呂頤浩繕修城池。宗澤往河北視師,七日還。是夜,辛道宗兵潰於嘉興縣。丁酉,詔荊襄、關陜、江淮皆備巡幸。戊戌,罷買馬。己亥,以子敷為檢校少保、集慶軍節度使,封魏國公。詔內外官司參用嘉祐、元豐敕,以俟新書。庚子,二帝徙居□郡。辛丑,陳通劫提點刑獄周格營,殺格,執提點刑獄高士曈。壬寅,遣徽猷閣待制孟忠厚迎奉太廟神主赴揚州。以直秘閣王
 圭為招撫判官,代張所,尋責所廣南安置。乙巳,宗澤表請車駕還闕。戊申,河北招撫司都統制王彥渡河擊金人,破之,復新鄉縣。己酉,以諜報金人欲犯江、浙,詔暫駐淮甸捍禦,稍定即還京闕。募民入貲授官。軍賊趙萬入常州,執守臣何袞。罷諸路經制招撫使。庚戌,始通當三大錢於淮、浙、荊湖諸路。壬子,命湖南撫諭官馬伸持詔賜張邦昌死於潭州,並誅王時雍。癸丑,詔有敢妄議惑眾沮巡幸者,許告而罪之,不告者斬。乙卯,王彥及金人
 戰,敗績,奔太行山聚眾,其裨將岳飛引其部曲自為一軍。趙萬陷鎮江府,守臣趙子崧棄城渡江,保瓜洲。是秋,金人分兵據兩河州縣,惟中山、慶源府、保、莫、邢洺、冀、磁、絳、相州久之乃陷。



 冬十月丁巳朔,帝登舟幸淮甸。戊午,太后至揚州。己未,罷諸路勸誘獻納錢物。庚申,罷諸路召募潰兵忠義等人,及寄居官擅集勤王兵者。癸亥,募群盜能並滅賊眾者官之。甲子,以張浚論李綱不已,落綱觀文殿大學士,止奉宮祠。知秀州兼權浙西提點刑
 獄趙叔近入杭州招撫陳通。乙丑,罷帥府、要郡、次要郡新軍及水軍。丁卯,以王淵為杭州制置盜賊使,統制官張俊從行。庚午,次泗州,幸普照寺。甲戌,太白晝見。己卯,次楚州寶應縣。後軍將孫琦等作亂,逼左正言盧臣中墮水死。庚辰,命劉光世討鎮江叛兵。辛巳,以光世為滁和濠州、江寧府界招捉盜賊制置使,御營統制官苗傅為使司都統制。朝請郎李棫提舉廣西左、右兩江峒丁公事。癸未,至揚州,禁內侍統兵官相見。丙戌,王淵、張俊
 誘趙萬等,悉誅之。



 十一月戊子,李綱鄂州居住。真定軍賊張遇入池州,守臣滕祐棄城遁。己丑,詔雜犯死罪有疑及情理可憫者,撫諭官同提刑司酌情減降,先斷後聞。壬辰,遣王倫等為金國通問使。乙未,以張愨為尚書左丞,工部尚書顏岐同知樞密院事。丙申,曲赦應天府、毫、宿、揚、泗、楚州、高郵軍。丙午,以張愨為中書侍郎。戊申,以顏岐為尚書左丞兼權門下侍郎,御史中丞許景衡為右丞,刑部尚書郭三益同知樞密院事。權密州趙野
 棄城遁,軍校杜彥據州,追野,殺之。辛亥,命福建路增招弓手。金人陷河間府。是月,軍賊丁進圍壽春府,守臣康允之拒卻之。



 十二月丙辰朔,命從臣四員充講讀官,就內殿講讀。丁巳,詔諸路提刑司選官,即轉運司所在州類省試進士,以待親策。辛酉,王淵入杭州,執陳通等誅之。壬戌,青州敗將王定以兵作亂,殺帥臣曾孝序。癸亥,粘罕犯汜水關,西京留守孫昭遠遣將拒之,戰歿,昭遠將兵南遁,尋命部將王仔奉啟運宮神禦赴行在。甲子,
 改授後父徽猷閣待制邢煥為光州觀察使。乙丑,詔凡刑賞大政並經三省,其乾請墨敕行下者罪之。丙寅,張遇犯江州。戊辰,金人圍棣州,守臣姜剛之固守,金兵解去。甲戌,金人陷同州,守臣鄭驤死之。張遇犯黃州。己卯,金人陷汝州,入西京。庚辰,金人陷華州。辛巳,破潼關。河東經制使王□燮自同州引兵遁入蜀。丁進詣宗澤降。乙酉,增置廣西弓手以備邊。以戶部尚書黃潛厚為延康殿學士、同提舉措置財用。



\end{pinyinscope}