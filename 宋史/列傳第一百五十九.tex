\article{列傳第一百五十九}

\begin{pinyinscope}

 ○王信汪大猷袁燮吳柔勝游仲鴻李祥王介宋德之楊大全



 王信,字誠之,處州麗水人。既冠,入太學,登紹興三十年
 進士第,試中教官,授建康府學教授。丁父憂,服除,進所著《唐太宗論贊》及《負薪論》,孝宗覽之,嘉嘆不已,特循兩資,授太學博士。



 時須次者例徙外,添差溫州教授。郡饑疫,議遣官振救之,父老願得信任其事,守不欲以煩信,請益力,信聞之,欣然為行,遍至病者家,全活不可勝記。



 差敕令所刪定官,法令有不合人情,自相牴牾,吏得以傅會出入者,悉厘正之。



 轉對,言:「敵情不可測,和議不可恃,今日要當先為自備之策,以待可乘之機。」



 上以為是。
 又論:「太學正、錄掌規矩之官而員多,博士掌訓導之官而員少,請以正、錄兩員升為博士。」從之。論除官脞冗之敝,乞精選監司而擇籍名,郡將代半歲乃注人。上親以其章授宰臣行。



 權考功郎官。蜀人張公遷,初八年免銓,至是改秩,吏妄引言,復令柅之,信鉤考其故,吏怖服。有三蜀士實礙式,吏受賕為地,工部尚書趙雄,蜀人也,以屬信,信持弗聽,已而轉吏部閱審成牘,撫掌愧嘆,嗟激不已,以聞於上。



 它日,上謂尚書蔡洸曰:「考功得王信,銓
 曹遂清。」邏者私相語,指為神明。



 武臣給告不書年齒,磨轉蔭薦,肆為奸欺,不可控摶,為擿最者數事告宰相,付之大理獄。事連三衙,殿帥王友直銳爭之,上審知其非,沮之曰:「考功所言,公事也,汝將何為?」獄具,皆伏辜。因請置籍,以柅後患。



 授軍器少監,仍兼考功郎官。丁母憂,吏裒金殺牲禱神,願信服闋無再為考功。



 既起,知永州。入奏事,留為將作少監,復考功郎官,轉軍器少監兼右司郎官,升員外郎。四方有以疑獄來上者,信反復披覽,常
 至夜分。



 升左司員外郎,轉對,論士大夫趨向之敝:「居官者逃一時之責,而後之禍患有所不恤;獻言者求一時之合,而行之可否有所不計。集事者以趣辦為能,而不為根本之慮;謀利者以羨餘為事,而不究源流之實。持論尚刻薄,而浸失祖宗忠厚之意;革敝預煩碎,而不明國家寬大之體。因循玩習,恬不為怪。願酌古之道,當時之宜,示好惡於取舍之間,使天下靡然知鄉,而無復為目前茍且之徇。」又論:「朝廷有恤民之政,而州縣不能行
 恤民之實。近歲不登,陛下軫念元元,凡水旱州郡租賦,或蠲放,或倚閣住催。然倚閣住催之名可以並緣為擾,願明與減放。」又論豫備三說:收逃亡之卒,選忠順之官,嚴訓練之職。又言屯田利害。上皆納其說。



 兼玉牒所檢討官、提領戶部酒庫。久之,上諭信曰:「知朕意否?行用卿,慮書生不長於財賦,故以命卿,果能副朕所委。」



 為中書門下檢正諸房文字,遷太常少卿兼權中書舍人。假禮部尚書使於金,肄射都亭,連中其的,金人駴曰:「尚書得
 非黑王相公子孫耶?」謂王德用也。信得米芾書法,金人寶之。歸言金人必衰之兆有四,在我當備之策有二,上首肯之。



 太史奏仲秋日月五星會於軫,信言:「休咎之徵,史策不同,然五星聚者有之,未聞七政共集也。分野在楚,願思所以順天而應之。」因條上七事。又言:「陛下即位之初,經營中原之志甚銳,然功之所以未立者,正以所用之人不一。其人不一,故其論不一;其論不一,故其心不一。願豫求至當之論,使歸於一。鎖闈封駁,而右府所
 不下關中書,或斜封捷出,左於公論。統領官奴事內侍,坐謫遠州,幸蒙赦還而遽復故職。潛藩恩舊之隸徒,榷酤官而齒朝士。老禁校僥冀節鉞,詭計可得之,而奉稍恩典,與正不異。閣門多溢額祗候。妃嬪進封而冒指它姓為甥侄。既一一塗歸,有雖書讀而徐核其不當者,續爭救之。」上曰:「事有不可不問者,第言之,朕無有不為卿行者。」於是益抗志不回。



 宦者甘昪既逐遠之矣,屬高宗崩,用治喪事,人莫敢言。昪俄提舉德壽宮,信亟執奏,舉
 朝皆悚。翰林學士洪邁適入,上語之曰:「王給事論甘昪事甚當。朕特白太上皇后,聖訓以為:『今一宮之事異於向時,非我老人所能任,小黃門空多,類不習事,獨昪可任責,分吾憂。渠今已歸,居室尚不能有,豈敢蹈故態。』以是駁疏不欲行。卿見王給事,可道此意。」信聞之乃止。



 信遇事剛果,論奏不避權要,繇此人多嫉之,信亦力求去,提舉崇福宮。詔求言,信條十事以獻,其目曰:法戒輕變,令貴必行,寬州郡以養民力,修軍政以待機會,郡當
 分其緩急,縣當別其劇易嚴銅錢之禁,廣積聚之備,處歸附之人,收逃亡之卒。



 起知湖州,信未涉州縣,據桉剖析,敏如流泉。擢集英殿修撰、知紹興府、浙東安撫使。奏免甫官錢十四萬、絹七萬匹、綿十萬五千兩、米二千萬斛。山陰境有犬英犬茶湖,四環皆田,歲苦潦,信創啟斗門,導停瀦注之海,築十一壩,化匯浸為上腴。民繪象以祠,更其名曰王公湖。築漁浦堤,禁民不舉子,買學田,立義塚,眾職修理。加煥章閣待制,徙知鄂州,改池州。



 初,信扶其父
 喪歸自金陵,草屨徒行,雖疾風甚雨,弗避也,由是得寒濕疾。



 及聞孝宗遺詔,悲傷過甚,疾復作,至是浸劇,上章請老,以通議大夫致仕。有星隕於其居,光如炬,不及地數尺而散。數日,信卒,遺訓其子以忠孝公廉。所著有《是齋集》行世。



 汪大猷,字仲嘉,慶元府鄞縣人。紹興七年,以父恩補官,授衢州江山縣尉,曉暢吏事。登十五年進士第,授婺州金華縣丞,爭財者諭以長幼之禮,悅服而退。



 李椿年行
 經界法,約束嚴甚,檄大猷覆視龍游縣,大猷請不實者得自陳,毋遽加罪。改建德,遷知昆山縣。丁父憂,免喪,差總領淮西、江東錢糧幹官,改乾辦行在諸司糧料院。



 參知政事錢端禮宣諭淮東,闢乾辦公事,充參議官,遷大宗丞兼吏部郎官,又兼戶部右曹。入對,言:「總核名實,責任臣下。因才而任,毋違所長,量能授官,毋拘流品。」孝宗顧謂左右曰:「疏通詳雅而善議論,有用之才也。」除禮部員外郎。丞相洪適薦兼吏部侍郎,仍遷主管左選。



 莊文
 太子初建東宮,兼太子左諭德、侍講,兩日一講《孟子》,多寓規戒。太子嘗出龍大淵禁中所進侍燕樂章,諭宮僚同賦,大猷曰:「鄭、衛之音,近習為昌,非講讀官所當預。」白於太子而止。遷秘書少監,修《五朝會要》。金人來賀,假吏部尚書為接伴使。尋兼權刑部侍郎,又兼崇政殿說書,又兼給事中。



 孝宗清燕,每訪政事,嘗曰:「朕每厭宦官女子之言,思與卿等款語,欲知朝政闕失,民情利病,茍有所聞,可極論之。」大猷遂陳耆長雇直隸經總制司,並緣
 法意使里正兼催科之役,厲民為甚。又論:「亭戶未嘗煮鹽,居近場監,貸錢射利,隱寄田產,害及編氓,宜取二等以上充役。」又論:「賜田勛戚,豪奪相先,陵轢州縣,惟當賜金,使自求之。」又論:「沒入貲產,止可行於強盜、贓吏,至於倉庫綱運之負陷者,惟當即其業收租以償,既足則給還,使復故業。」轉對,言捕酒之害,及居官者不得鑄銅為器。上嘉獎曰:「卿前後所言,皆今日可行之事。」



 權刑部侍郎,升侍講,言:「有司率用新制,棄舊法,輕重舛牾,無所遵
 承,使舞文之吏時出,以售其奸,請明詔編纂。」書成上進,上大悅。



 尚書周執羔韓元吉、樞密劉珙以強盜率不處死,無所懲艾,右司林慄謂:「太祖朝強盜贓滿三貫死,無首從,不問殺傷。景祐增五貫,固從寬。今設六項法,非手刃人,例奏裁黥配,何所懲艾,請從舊法,贓滿三貫者斬。」大猷曰:「此吾職也。」遂具奏曰:「強盜烏可恕,用舊法而痛懲之,固可也。天聖以來,益用中典,浸失禁奸之意。今所議六項法,犯者以法行之,非此而但取財,惟再犯者死,
 可謂寬嚴適中。若皆置之死地,未必能禁其為盜,盜知必死,將甘心於事主矣,望稍開其生路。」乃奏用六項法則死者十七人,用見行法則十四人,舊法百七十人俱死。



 遂從大猷議。



 借吏部尚書為賀金國正旦使,至盱眙,得印榜云:「強盜止用舊法,罷六項法。」



 還朝自劾求去,上聞之,復行六項法。



 改權吏部侍郎兼權尚書。夜傳旨學士院,出唐沈既濟論選舉事,曰:「今日有此敝,可行與否,詰旦當面對。」即奏:「事與今異,敝雖似之,言則難行。」上
 曰:「卿言甚明。」既郊,差充鹵簿使,以言去,授敷文閣待制、提舉太平興國宮。



 起知泉州。毗舍邪嘗掠海濱居民,歲遣戍防之,勞費不貲。大猷作屋二百區,遣將留屯。久之,戍兵以真臘大買為毗舍邪犯境,大猷曰:「毗舍邪面目黑如漆,語言不通,此豈毗舍邪耶?」遂譴之。故事蕃商與人爭鬥,非傷折罪,皆以牛贖,大猷曰:「安有中國用島夷俗者,茍在吾境,當用吾法。」三佛齊請鑄銅瓦三萬,詔泉、廣二州守臣督造付之。大猷奏:「法,銅不下海。中國方禁
 銷銅,奈何為其所役?」卒不與。進敷文閣直學士,留知泉州。



 逾年,提舉太平興國宮,改知隆興府、江西安撫使。以大暑討永新禾山洞寇,不利,自劾,降龍圖閣待制,落職,南康軍居住,提舉太平興國宮。復龍圖閣待制,提舉上清太平宮。復敷文閣待制,升學士。沒,贈二官。



 大猷與丞相史浩同里,又同年進士,未嘗附麗以干進,浩深嘆美之。好周施,敘宗族外族為《興仁錄》,率鄉人為義莊二十餘畝以倡,眾皆欣勸。所著有《適齋稿》、《備忘》、《訓鑒》等
 書。



 袁燮,字和叔,慶元府鄞縣人。生而端粹專靜,乳媼置盤水其前,玩視終日,夜臥常醒然。少長,讀東都《黨錮傳》,慨然以名節自期。入太學,登進士第,調江陰尉。



 浙西大饑,常平使羅點屬任振恤。燮命每保畫一圖,田疇、山水、道路悉載之,而以居民分布其間,凡名數、治業悉書之。合保為都,合都為鄉,合鄉為縣,徵發、爭訟、追胥,披圖可立決,以此為荒政首。除沿海制屬。連丁家艱,寧宗即位,以
 太學正召。時朱熹諸儒相次去國,丞相趙汝愚罷,燮亦以論去,自是黨禁興矣。久之,為浙東帥幕、福建常平屬、沿海參議。



 嘉定初,召主宗正簿、樞密院編修官,權考功郎官、太常丞、知江州,改提舉江西常平、權知隆興。召為都官郎官,遷司封。因對,言:「陛下即位之初,委任賢相,正士鱗集,而竊威權者從旁睨之。彭龜年逆知其必亂天下,顯言其奸,龜年以罪去,而權臣遂根據,幾危社稷。陛下追思龜年,蓋嘗臨朝太息曰:『斯人猶在,必大用之。』固
 已深知龜年之忠矣。今正人端士不乏,願陛下常存此心,急聞剴切,崇獎樸直,一龜年雖沒,眾龜年繼進,天下何憂不治。」「臣昨勸陛下勤於好問,而聖訓有曰:『問則明』。臣退與朝士言之,莫不稱善。而側聽十旬,陛下之端拱淵默猶昔也,臣竊惑焉。夫既知如是而明,則當知反是而暗。明則輝光旁燭,無所不通;暗則是非得失,懵然不辨矣。」



 遷國子司業、秘書少監,進祭酒、秘書監。延見諸生,必迪以反躬切己,忠信篤實,是為道本。聞者悚然有得,
 士氣益振。兼崇政殿說書,除禮部侍郎兼侍讀。



 時史彌遠主和,燮爭益力,臺論劾燮,罷之,以寶文閣待制提舉鴻慶宮。起知溫州,進直學士,奉祠以卒。



 燮初入太學,陸九齡為學錄,同里沈煥、楊簡、舒璘亦皆在學,以道義相切磨。



 後見,九齡之弟九淵發明本心之指,乃師事焉。每言人心與天地一本,精思以得之,兢業以守之則與天地相似。學者稱之曰絜齋先生。後謚正獻。子甫自有傳。



 吳柔勝,字勝之,宣州人。幼聽其父講伊、洛書,已知有持
 敬之學,不妄言笑。



 長游郡泮,人皆憚其方嚴。登淳熙八年進士第,調都昌簿。丞相趙汝愚知其賢,差嘉興府學教授,將置之館閣,會汝愚去,御史湯碩劾柔勝嘗救荒浙右,擅放田租,為汝愚收人心,且主朱熹之學,不可為師儒官,自是閑居十餘年。



 嘉定初,主管刑、工部架閣文字,遷國子正。柔勝始以朱熹《四書》與諸生誦習,講義策問,皆以是為先。又於生徒中得潘時舉、呂喬年,白於長,擢為職事,使以文行表率,於是士知趨向,伊、洛之學,晦
 而復明。遷太學博士,又遷司農寺丞。



 出知隨州。時再議和好,尤戒開邊隙,旁塞之民事與北界相涉,不問法輕重皆殺之。郡民梁皋有馬為北人所盜,追之急,北人以矢拒皋,皋與其徒亦發二矢。北界以為言,郡下七人於獄,柔勝至,立破械縱之,具始末報北界而已。收土豪孟宗政、扈再興隸帳下,後宗政、再興皆為名將。築隨州及棗陽城,招四方亡命得千人,立軍曰「忠勇」,廩以總所闕額,營柵器械悉備。除京西提刑,領州如故。改湖北運判
 兼知鄂州。甫至,值歲歉,即乞糴於湖南,大講荒政,十五州被災之民,全活者不可勝計。



 改知太平州,除直秘閣,主管毫州明道宮。改直華文閣,除工部郎中,力辭,除秘閣修撰,依舊宮觀以卒,謚正肅。二子淵、潛,俱登進士,各有傳。



 游仲鴻,字子正,果之南充人。淳熙二年進士第,初調犍為簿。李昌圖總蜀賦,闢糴買官,奇其才,曰:「吾董餉積年,惟得一士。」昌圖召入,首薦之,擢四川制置司干辦公事。
 制置使趙汝愚一見即知敬之。



 敘州董蠻犯犍為境,憲將合兵討之,仲鴻請行。詰其釁端,以州負馬直也,乃使人諭蠻曰:「歸俘則還馬直,不然大兵至矣。」蠻聽命,仲鴻受其降而歸。改秩,知中江縣,總領楊輔檄置幕下。時關外營田凡萬四千頃,畝僅輸七升。仲鴻建議,請以兵之當汰者授之田,存赤籍,遲以數年,汰者眾,耕者多,則橫斂一切之賦可次第以減。輔然之,大將吳挺沮而止。趙汝愚移帥閩,舉仲鴻自代,制置使京鏜、轉運劉光祖亦
 交薦於朝。



 紹熙四年,赴召,趙汝愚在樞密,謂仲鴻直諒多聞,訪以蜀中利病。汝愚欲親出經略西事,仲鴻曰:「宥密之地,斡旋者易,公獨不聞呂申公『經略西事當在朝廷」之語乎?」汝愚悟而止。差乾辦諸司糧料院。



 光宗以疾久不朝重華宮,仲鴻遺汝愚書,陳宗社大計,書有「伊、周、霍光」



 語,汝愚讀之駭,立焚之,不答。又遺書曰:「大臣事君之道,茍利社稷,死生以之。既不死,曷不去?」汝愚又不答。孝宗崩,仲鴻泣謂汝愚曰:「今惟有率百官哭殿庭,以請
 親臨。」宰相留正以病去,仲鴻亟簡汝愚曰:「禫日不決,禍必起矣。」



 汝愚又不答。後三日,嘉王即位於重華宮。



 汝愚既拜右丞相,以仲鴻久游其門,闢嫌不用。初,汝愚之定策也,知閣韓侂胄頗有勞,望節鉞,汝愚不與。侂胄方居中用事,恚甚。汝愚跡已危,方益自嚴重,選人求見者例不許。仲鴻勸以降意容接,覬遏異論,而汝愚以淮東、西總賦積弊,奏遣仲鴻核實。仲鴻曰:「丞相之勢已孤,不憂此而顧憂彼耶?」改監登聞鼓院以行。



 會侍講朱熹以論
 事去國,仲鴻聞之,即上疏曰:「陛下宅憂之時,御批數出,不由中書。前日宰相留正之去,去之不以禮;諫官黃度之去,去之不以正;近臣朱熹之去,復去之不以道。自古未有舍宰相、諫官、講官而能自為聰明者也。願亟還熹,毋使小人得志,以養成禍亂。」



 監察御史胡紘希侂胄意,誣汝愚久蓄邪心,嘗語人以乘龍授鼎之夢,又謂朝士中有推其宗派,以為裔出楚王元佐正統所在者,指仲鴻也。初,欲直書仲鴻名,同臺張孝伯見之曰:「書其名則
 竄矣。凡阿附宰相,本冀官爵,此人沉埋六院且二年,心跡可察。」卒不書其名。



 慶元元年,汝愚罷相,仲鴻遷軍器監主簿,力丐外,除知洋州。朱熹聞其出,曰:「信蜀士之多奇也。」越三年,起知嘉定府。擢利路轉運判官,數忤宣撫副使吳曦,曦言仲鴻老病,朝命易他部。未幾,曦叛,宣撫司幕官薛紱訪仲鴻於果山,仲鴻對之泣,指案上一編書示紱曰:「開禧丁卯正月游某死。」謂家人曰:「曦逼吾死,即填其日。」



 時宣撫使程松已大棄其師遁,仲鴻以書
 勸成都帥楊輔討賊,輔不能用。至是松至果,仲鴻謂紱曰:「宣威肯留,則吾以積奉二萬緡犒兵,護宣威之成都。」松不顧而去。總賦劉崇之繼至,仲鴻遣其子似往見,以告松者告之,崇之復不聽。未幾,曦誅,參政李壁奏除利路提點刑獄,尋乞休致,予祠而歸,遷中奉大夫。



 嘉定八年卒,年七十八。劉光祖表其隧道曰:「於乎,慶元黨人游公之墓。」



 紹定五年,謚曰忠。子似,淳祐五年為右丞相,自有傳。



 李祥,字元德,常州無錫人。隆興元年進士,為錢塘縣主簿。時姚憲尹臨安,俾攝錄參。邏者以巧發為能,每事下有司,必監視鍛煉,囚服乃已。嘗誣告一武臣子謗朝政,鞫於獄,祥不使邏者入門。既而所告無實,具以白尹,尹驚曰:「上命無實乎?」祥曰:「即坐譴,自甘。」憲具論如祥意,上駭曰:「朕幾誤矣,卿吾爭臣也。」遂賜憲出身為諫大夫,祥調濠州錄事參軍。安豐守臣冒占民田,訟屢改而不決,監司委祥,卒歸之民。未幾,其人易守濠,以嫌換司理廬
 州;守出改官奏留之,不可。



 主管戶部架閣文字、太學博士、國子博士、司農寺丞、樞密院編脩官兼刑部郎官、大宗正丞、軍器少監。言:「忝朝跡八年,在外賢才不勝眾,願更出迭入由臣始。」出提舉淮東常平茶鹽、淮西運判。兩淮鐵錢比不定,祥疏乞官賜錢米銷濫惡者,廢定城、興國、漢陽監,更鑄紹熙新錢,從之,淮人以安。



 遷國子司業、宗正少卿、國子祭酒。丞相趙汝愚以言去國,祥上疏爭之,曰:「頃壽皇崩,兩宮隔絕,中外洶洶,留正棄印亡去,國
 命如發。汝愚不畏滅族,決策立陛下,風塵不搖,天下復安,社稷之臣也。奈何無念功至意,忽體貌常典,使精忠巨節怫鬱黯暗,何以示後世?」



 除直龍圖閣、湖南運副,言者劾罷之。於是太學諸生楊宏中、周端朝等六人上書留之,俱得罪。主沖祐觀,再請老,以直龍圖閣致仕。嘉泰元年八月卒,謚肅簡。



 王介,字元石,婺州金華人。從朱熹、呂祖謙游。登紹熙元年進士第,廷對陳時弊,大略言:「近者罷拾遺、補闕,有遠
 諫之意,小人唱為朋黨,有厭薄道學之名。」上嘉其直,擢居第三人。



 簽書昭慶軍節度判官廳公事,除為國子錄,上疏言:「壽皇親挈神器授之陛下,孝敬豈可久闕乎?」又言:「婦事舅姑如事父母,不可虧宮中之禮。」不報。孝宗崩,介又力請上過宮執喪,累疏言辭激切,人嘆其忠。



 寧宗即位,介上疏言:「陛下即位未三月,策免宰相,遷易臺諫,悉出內批,非治世事也。崇寧、大觀間事出御批,遂成北狩之禍。杜衍為相,常積內降十數封還,今宰相不敢封
 納,臺諫不敢彈奏,此豈可久之道。」遷太學博士。



 時韓侂胄居中潛弄威福之柄,猶未肆也,而文墨議論之士陰附之以希進,於是始無所憚矣。侂胄始疑介前封事詆己,且其弟仰胄嘗以舊識求自通,介拒絕之,侂胄怨益深。



 添差通判紹興府,尋知邵武軍。會學禁起,諫大夫姚愈劾介與袁燮皆偽學之黨,且附會前相汝愚,主管臺州崇道觀。久之,差知廣德軍。侂胄之隸人蘇師旦忿介不通謁,目為偽黨,並及甲寅廷對之語,以告侂胄。有勸
 其自明者,介曰:「吾發已種種,豈為鼠輩所使邪!」侂胄亦畏公議不敢發。以外艱去。



 免喪,知饒州,未赴,召為秘書郎,遷度支郎官。師旦已建節,介與同列謁政府,遇之於庭,客皆逾階而揖,介不顧。於是殿中侍御史徐柟劾介資淺立異,奉祠,除都大坑冶。



 侂胄誅,朝廷更化,介召還,除侍左郎官兼右司、太子舍人,改兵部郎官、國子司業、太子侍講兼國史院編修官、實錄院檢討官,除國子祭酒。會以不雨,詔百官指陳闕失,時宰相史彌遠以母喪
 起復,介手疏歷論時政,推本《洪範》僭恆晹若之證,謂:「羅日願為變,是下人謀上也。修好增幣,而金人猶觖望,是夷人亂華也。內批數出,是左右干政也。諫官無故出省,是小人間君子也。皆謂之僭。一僭已足以致天變,而況兼有之哉。」又言:「漢法天地降災,策免丞相,乞令彌遠終喪,擇公正無私者置左右,王、呂、蔡、秦之覆轍,可以為戒。」



 接送伴金國賀生辰使還,奏:「故事兩國通廟諱、御名,而本朝止通御名,高宗至光宗皆傳名而不傳諱,紹熙初,
 黃裳嘗以為言,而未及厘正。願正典禮,以尊宗廟。」



 除秘書監,升太子右諭德。其在春宮,篤意輔導,每遇講讀,因事規諫。太子嘗欲索館中圖畫,卻而弗與,及張燈設樂,則諫止之;且乞選配故家以正始,絕令旨以杜請謁,宮僚分日上直,以資見聞。



 遷宗正少卿兼權中書舍人,繳駁不避權貴。張允濟以閣職為州鈐,介謂此小事而用權臣例,破祖宗制,不可不封還詞頭。丞相語介曰:「此中宮意。」介曰:「宰相而逢宮禁意向,給舍而奉宰相風旨,朝
 廷紀綱掃地矣。」



 居數日,除起居舍人。介奏:「宰相以私請不行,而托威福於宮禁,權且下移,誰敢以忠告陛下者。」乞歸老,不許。言:「本朝循唐入閣之制,左右史不立前殿,若御後殿,則立朵殿下,何所聞見而修起居注乎?乞依歐陽修、王存、胡銓所請,分立殿上。」



 吏部侍郎許奕以言事去國,介奏曰:「陛下更化三年,而言事官去者五人,倪思、傅伯成既去,其後蔡幼學、鄒應龍相繼而出,今許奕復蹈前轍。此五臣者,四為給事,一為諫大夫,兩年之間,
 盡聽其去。或謂此皆宰相意,自古未有大臣因給舍論事而去之者,是大臣誤陛下也,將恐成孤立之勢。」疏奏,乞補外,以右文殿修撰知嘉興府。



 歲餘,升集英殿修撰、知襄陽府、京西安撫使。徙知慶元府兼沿海制置使,以疾奉祠。嘉定六年八月卒,年五十六。端平三年,郡守趙汝談請於朝,特贈中大夫、寶章閣待制,謚忠簡。子野,自有傳。



 宋德之,字正仲,其先京兆人。隋諫大夫遠謫彭山,子孫
 散居於蜀,遂為蜀州人。德之以應舉擢慶元二年外省第一,為山南道掌書記。召除國子正,遷武學博士。



 與諸生論八陣之象本乎八卦,皆動物也,奇正之變,往來相生而不窮,知此然後可以致勝。



 遷編修樞密院。時兵釁有萌,會赤眚見太陰,犯權星,未浹日,內北門鴟尾災,延及三省、六部,詔求言,德之奏:「離為火,為日,為甲胄;坎為水,為月,為盜,為隱伏。故火失其性,赤氣見,憂在甲兵;水失其性,太陰失度,憂在隱伏。」



 因疏七事,皆當今至切之
 患,乃曰:「人火小變不足慮,天象之變,臣竊危之。」



 他日,又對曰:「今敵未動,而輕變祖宗舊制,命武臣帥邊以自遺患。晉叛將、唐藩鎮之禍基於此矣。」時吳曦在西陲,皇甫斌在襄漢,郭倪、李爽在兩淮,德之預以為慮。



 除太常丞,出知閬州。會曦變,托跌足以避偽,事平,始赴閬。擢本路提點刑獄,制帥安丙奏:「德之傲視君命,不俟代者之來,徑用觀察使印領事。」詔降一官,改潼川路轉運判官、湖南路提刑,改湖北。



 召為兵部郎官。朝論有疑安丙意,丞相
 史彌遠首以問德之,德之對曰:「蜀無安丙,朝廷無蜀矣,人有大功,實不敢以私嫌廢公議。」忤時相意,遂罷。安丙深感德之,嘗謂人曰:「丙不知正仲,正仲知丙;丙負正仲,正仲不負丙。」請昏於德之,不許。論者益稱德之之賢。起知眉州,監特奏名試,得疾而卒。



 德之大父耕,性剛介,一朝棄官去,莫知所終。從父廉語德之曰:「吾昔至臨安府,有人言蜀有宋宣教者過浙江而去,吾適越求之,則入四明矣。」德之渡浙江尋訪,至雪竇,有蜀僧言:「聞諸耆老
 云:山後有爛平山,有二居士焉,其一宋宣教也。」德之躋攀至爛平,見丹灶,置祠其上而歸。



 楊大全,字渾甫,眉之青神人。乾道八年進士,調溫江尉,攝邑有政聲。紹熙三年,召除監登聞鼓院。五年,光宗以疾久,不克省重華宮,廷臣多論諫者。太學生汪安仁等二百餘人上書,而龔日章等百餘人以投軌上書為緩,必欲伏闕。大全謂:「院以登聞名,實明目達聰之地也,今乃使人視為具文,吾何顏以尸此職。」乃為書以諫,力請
 過宮,書上不報。大全於是三上疏,其略曰:臣之志於憂君者,不畏義死,不榮幸生,不以言而獲罪為恥,而以言不聽從為恥。自古諫之不效,其大者身膏斧鑕,其次亦流竄四裔,其小者猶罷免終身,未有若今日不勉於聽從,亦不加於黜逐,徒餌之以無所譴呵之恩,使皆饕富貴,甘豢養,以消靡其風節。平居皆貪祿懷奸之士,則臨難必無仗節死義之人。



 陛下自夏秋以來,執政從官之死者皆不信,卒之果然乎?不然乎?建康趙濟死,武興吳
 挺死,今尚不以為然,則事有幾微於朕兆者,可諫陛下乎?萬一變起蕭墻,禍生肘腋,陛下必將以為不信,坐受其危亡矣。



 盜滿山東而高、斯弄權,二世不知也。蠻寇成都而更奏捷,明皇不知也。此猶左右聾瞽爾。今在朝之士瀝忠以告,而陛下不聽,是陛下自壅蔽其聰明也。今外間傳聞,以為壽皇將幸越,幸吳興,此愛陛下之深,欲泯其跡也。陛下當亟圖所以解壽皇之憂。



 疏入,又不報。



 寧宗即位,遷宗正寺主簿。慶元元年,易太常寺主簿,遷
 司農寺丞。修《高宗實錄》,充檢討官。先是,韓侂胄用事,私臺諫之選為己羽翼,且欲得知名士,借其望以壓群言,一時之好進者,恨不預此選也。會御史虛位,有力薦大全者,屬大全一往見,且曰:「公朝見,除目夕下矣。」大全笑謝,決不往,明日遂丐外。時《實錄》將上矣,上必推恩,大全去不少待。於是除知金州,至姑蘇,以病卒。



 論曰:王信有文學,通政事。汪大猷敦厚老成。袁燮學有所本。吳柔勝、游仲鴻名在偽學。觀李祥訟趙汝愚,公論
 藉是以立。王介、楊大全直道而行。宋德之其知兵者歟?



\end{pinyinscope}