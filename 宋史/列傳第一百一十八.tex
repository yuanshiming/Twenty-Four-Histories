\article{列傳第一百一十八}

\begin{pinyinscope}

 李綱下



 紹興二年,除觀文殿學士、湖廣宣撫使兼知潭州。是時,荊湖江、湘之間,流民潰卒群聚為盜賊,不可勝計,多者至數萬人,綱悉蕩平之。上言:「荊湖、國之上流,其地數千
 里,諸葛亮謂之用武之國。今朝廷保有東南,控馭西北。加鼎、澧、岳、鄂若荊南一帶,皆當屯宿重兵,倚為形勢,使四川之號令可通,而襄、漢之聲援可接,乃有恢復中原之漸。」議未及行,而諫官徐俯、劉斐劾綱,罷為提舉西京崇福宮。



 四年冬,金人及偽齊來攻,綱具防禦三策,謂:「偽齊悉兵南下,境內必虛。儻出其不意,電發霆擊,搗穎昌以臨畿甸,彼必震懼還救,王師追躡,必勝之理,此上策也。若駐蹕江上,號召上流之兵,順流而下,以助聲勢,金
 鼓旌旗,千里相望,則敵人雖眾,不敢南渡。然後以重師進屯要害之地,設奇邀擊,絕其糧道,俟彼遁歸,徐議攻討,此中策也。萬一借親征之名,為順動之計,使卒伍潰散,控扼失守,敵得乘間深入,州縣望風奔潰,則其患有不可測矣。往歲,金人利在侵掠,又方時暑,勢必還師,朝廷因得以還定安集。今偽齊導之而來,勢不徒還,必謀割據。奸民潰卒從而附之,聲勢鴟張,茍或退避,則無以為善後之策。昔苻堅以百萬眾侵晉,而謝安以偏師破
 之。使朝廷措置得宜,將士用命,安知北敵不授首於我?顧一時機會所以應之者如何耳。望降臣章與二三大臣熟議之。」詔:綱所陳,今日之急務,付三省、樞密院施行。時韓世忠屢敗金人於淮、楚間,有旨督劉光世、張浚統兵渡河,車駕進發至江上勞軍。



 五年,詔問攻戰、守備、措置、綏懷之方,綱奏:



 願陛下勿以敵退為可喜,而以仇敵未報為可憤;勿以東南為可安,而以中原未復、赤縣神州陷於敵國為可恥;勿以諸將屢捷為可賀,而以軍政
 未修、士氣未振而強敵猶得以潛逃為可虞。則中興之期,可指日而俟。



 議者或謂敵馬既退,當遂用兵為大舉之計,臣竊以為不然。生理未固,而欲浪戰以僥幸,非制勝之術也。高祖先保關中,故能東向與項籍爭。光武先保河內,故能降赤眉、銅馬之屬。肅宗先保靈武,故能破安、史而復兩京。今朝廷以東南為根本,將士暴露之久,財用調度之煩,民力科取之困,茍不大修守備,痛自料理,先為自固之計,何以能萬全而制敵?



 議者又謂敵人
 既退,當且保據一隅,以茍目前之安,臣又以為不然。秦師三伐晉,以報殽之師;諸葛亮佐蜀,連年出師以圖中原,不如是,不足以立國。高祖在漢中,謂蕭何曰:『吾亦欲東。』光武破隗囂,既平隴,復望蜀。此皆以天下為度,不如是,不足以混一區宇,戡定禍亂。況祖宗境土,豈可坐視淪陷,不務恢復乎?今歲不征,明年不戰,使敵勢益張,而吾之所糾合精銳士馬,日以損耗,何以圖敵?謂宜於防守既固、軍政既修之後,即議攻討,乃為得計。此二者,守
 備、攻戰之序也。



 至於守備之宜,則當科理淮南、荊襄,以為東南屏蔽。夫六朝之所以能保有江左者,以強兵巨鎮,盡在淮南、荊襄間。故以魏武之雄,苻堅、石勒之眾,宇文、拓拔之盛,卒不能窺江表。後唐李氏有淮南,則可以都金陵,其後淮南為周世宗所取,遂以削弱。近年以來,大將擁重兵於江南,官吏守空城於江北,雖有天險而無戰艦水軍之制,故敵人得以侵擾窺伺。今當於淮之東西及荊襄置三大帥,屯重兵以臨之,分遣偏師,進守
 支郡,加以戰艦水軍,上運下接,自為防守。敵馬雖多,不敢輕犯,則藩籬之勢盛而無窮之利也。有守備矣,然後議攻戰之利,分責諸路,因利乘便,收復京畿,以及故都。斷以必為之志而勿失機會,則以弱為強,取威定亂於一勝之間,逆臣可誅,強敵可滅,攻戰之利,莫大於是。



 若夫萬乘所居,必擇形勝以為駐蹕之所,然後能制服中外,以圖事業。建康自昔號帝王之宅,江山雄壯,地勢寬博,六朝更都之。臣昔舉天下形勢而言,謂關中為上,今
 以東南形勢而言,則當以建康為便。今者,鑾輿未復舊都,莫若且於建康權宜駐蹕。願詔守臣治城池,修宮闕,立官府,創營壁,使粗成規模,以待巡幸。蓋有城池然後人心不恐,有官府然後政事可修,有營壘然後士卒可用,此措置之所當先也。



 至於西北之民,皆陛下赤子,荷祖宗涵養之深,其心未嘗一日忘宋。特制於強敵,陷於塗炭,而不能以自歸。天威震驚,必有結納來歸、願為內應者。宜給之土田,予以爵賞,優加撫循,許其自新,使陷
 溺之民知所依怙,莫不感悅,益堅戴宋之心,此綏懷之所當先也。



 臣竊觀陛下有聰明睿智之姿,有英武敢為之志,然自臨御,迨今九年,國不闢而日蹙,事不立而日壞,將驕而難御,卒惰而未練,國用匱而無贏餘之蓄,民力困而無休息之期。使陛下憂勤雖至,而中興之效,邈乎無聞,則群臣誤陛下之故也。



 陛下觀近年以來所用之臣,慨然敢以天下之重自任者幾人?平居無事,小廉曲謹,似可無過,忽有擾攘,則錯愕無所措手足,不過奉
 身以退,天下憂危之重,委之陛下而已。有臣如此,不知何補於國,而陛下亦安取此?夫用人如用醫,必先知其術業可以已病,乃可使之進藥而責成功。今不詳審其術業而姑試之,則雖日易一醫,無補於病,徒加疾而已。大概近年,閑暇則以和議為得計,而以治兵為失策,倉卒則以退避為愛君,而以進御為誤國。上下偷安,不為長久之計。天步艱難,國勢益弱,職此之由。



 今天啟宸衷,悟前日和議退避之失,親臨大敵。天威所臨,使北軍數十
 萬之眾,震怖不敢南渡,潛師宵奔。則和議之與治兵,退避之與進御,其效概可睹矣。然敵兵雖退,未大懲創,安知其秋高馬肥,不再來擾我疆埸,使疲於奔命哉?



 臣夙夜為陛下思所以為善後之策,惟自昔創業、中興之主,必躬冒矢石,履行陣而不避。故高祖既得天下,擊韓王信、陳豨、黥布,未嘗不親行。光武自即位至平公孫述,十三年間,無一歲不親征。本朝太祖、太宗,定維揚,平澤、潞,下河東,皆躬御戎輅;真宗亦有澶淵之行,措天下於大
 安。此所謂始憂勤而終逸樂也。



 若夫退避之策,可暫而不可常,可一而不可再,退一步則失一步,退一尺則失一尺。往時自南都退而至維揚,則關陜、河北、河東失矣;自維揚退而至江、浙,則京東、西失矣。萬有一敵騎南牧,復將退避。不知何所適而可乎?航海之策,萬乘冒風濤不測之險,此又不可之尤者也。惟當於國家閑暇之時,明政刑,治軍旅,選將帥,修車馬,備器械,峙糗糧,積金帛。敵來則御,俟時而奮,以光復祖宗之大業,此最上策也。臣願
 陛下自今以往,勿復為退避之計,可乎?



 臣又觀古者敵國善鄰,則有和親,仇讎之邦,鮮復遣使。豈不以釁隙既深,終無講好修睦之理故耶?東晉渡江,石勒遣使於晉,元帝命焚其幣而卻其使。彼遣使來,且猶卻之,此何可往?假道僭偽之國,其自取辱,無補於事,祗傷國體。金人造釁之深,知我必報,其措意為何如?而我方且卑辭厚幣,屈體以求之,其不推誠以見信,決矣。器幣禮物,所費不貲,使軺往來,坐索士氣,而又邀我以必不可從之事,
 制我以必不敢為之謀,是和卒不成,而徒為此擾擾也。非特如此,於吾自治自強之計,動輒相妨,實有所害。金人二十餘年,以此策破契丹、困中國,而終莫之悟。夫辨是非利害者,人心所同,豈真不悟哉?聊復用此以僥幸萬一,曾不知為吾害者甚大,此古人所謂幾何僥幸而不喪人之國者也。臣願自今以往,勿復遣和議之使,可乎?



 二說既定,擇所當為者,一切以至誠為之。俟吾之政事修,倉廩實,府庫充,器用備,士氣振,力可有為,乃議大
 舉,則兵雖未交,而勝負之勢已決矣。



 抑臣聞朝廷者根本也,藩方者枝葉也,根本固則枝葉蕃,朝廷者腹心也,將士者爪牙也,腹心壯則爪牙奮。今遠而強敵,近而偽臣,國家所仰以為捍蔽者在藩方,所資以致攻討者在將士,然根本腹心則在朝廷。惟陛下正心以正朝廷百官,使君子小人各得其分,則是非明,賞罰當,自然藩方協力,將士用命,雖強敵不足畏,逆臣不足憂,此特在陛下方寸之間耳。



 臣昧死上條六事:一曰信任輔弼,二曰
 公選人材,三曰變革士風,四曰愛惜日力,五曰務盡人事,六曰寅畏天威。



 何謂信任輔弼?夫興衰撥亂之主,必有同心同德之臣相與有為,如元首股肱之於一身,父子兄弟之於一家,乃能協濟。今陛下選於眾以圖任,遂能捍禦大敵,可謂得人矣。然臣願陛下待以至誠,無事形跡,久任以責成功,勿使小人得以間之,則君臣之美,垂於無窮矣。



 何謂公選人才?夫治天下者,必資於人才,而創業、中興之主,所資尤多。何則?繼體守文,率由舊章,
 得中庸之才,亦足以共治;至于艱難之際,非得卓犖瑰偉之才,則未易有濟。是以大有為之主,必有不世出之才,參贊翊佐,以成大業。然自昔抱不群之才者,多為小人之所忌嫉,或中之以黯暗,或指之為黨與,或誣之以大惡,或擿之以細故。而以道事君者,不可則止,難於自進,恥於自明,雖負重謗、遭深譴,安於義命,不復自辨。茍非至明之主,深察人之情偽,安能辨其非辜哉?陛下臨御以來,用人多矣,世之所許以為端人正士者,往往閑
 廢於無用之地;而陛下寤寐側席,有乏材之嘆,盍少留意而致察焉!



 何謂變革士風?夫用兵之與士風,似不相及,而實相為表裏。士風厚則議正而是非明,朝廷賞罰當功罪而人心服,考之本朝嘉祐、治平以前可知已。數十年來,奔競日進,論議徇私,邪說利口,足以惑人主之聽。元祐大臣,持正論如司馬光之流,皆社稷之臣也,而群枉嫉之,指為奸黨,顛倒是非,政事大壞,馴致靖康之變,非偶然也。竊觀近年士風尤薄,隨時好惡,以取世
 資,潝訿成風,豈朝廷之福哉?大抵朝廷設耳目及獻納論思之官,固許之以風聞,至於大故,必須核實而後言。使其無實,則誣人之罪,服讒搜慝,得以中害善良,皆非所以修政也。



 何謂愛惜日力?夫創業、中興,如建大廈,堂室奧序,其規模可一日而成,鳩工聚材,則積累非一日所致。陛下臨御,九年於茲,境土未復,僭逆未誅,仇敵未報,尚稽中興之業者,誠以始不為之規模,而後不為之積累故也。邊事粗定之時,朝廷所推行者,不過簿書期
 會不切之細務,至於攻討防守之策,國之大計,皆未嘗留意。夫天下無不可為之事,亦無不可為之時。惟失其時,則事之小者日益大,事之易者日益難矣。



 何謂務盡人事?夫天人之道,其實一致,人之所為,即天之所為也。人事盡於前,則天理應於後,此自然之符也。故創業、中興之主,盡其在我而已,其成功歸之於天。今未嘗盡人事,敵至而先自退屈,而欲責功於天,其可乎?臣願陛下詔二三大臣,協心同力,盡人事以聽天命,則恢復土宇,
 剪屠鯨鯢,迎還兩宮,必有日矣。



 何謂寅畏天威?夫天之於王者,猶父母之於子,愛之至,則所以為之戒者亦至。故人主之於天戒,必恐懼修省,以致其寅畏之誠。比年以來,熒惑失次,太白晝見,地震水溢,或久陰不雨,或久雨不霽,或當暑而寒,乃正月之朔,日有食之。此皆天意眷祐陛下,丁寧反復,以致告戒。惟陛下推至誠之意,正厥事以應之,則變災而為祥矣。



 凡此六者,皆中興之業所關,而陛下所當先務者。



 今朝廷人才不乏,將士足用,
 財用有餘,足為中興之資。陛下春秋鼎盛,欲大有為,何施不可?要在改前日之轍,斷而行之耳。昔唐太宗謂魏徵為敢言,徵謝曰:「陛下導臣使言,不然,其敢批逆鱗哉。」今臣無魏徵之敢言,然展盡底蘊,亦思慮之極也。惟陛下赦其愚直,而取其拳拳之忠。



 疏奏,上為賜詔褒諭。除江西安撫制置大使兼知洪州。有旨,赴行在奏事畢之官。六年,綱至,引對內殿。朝廷方銳意大舉,綱陛辭,言今日用兵之失者四,措置未盡善者五,宜預備者三,當善
 後者二。



 時宋師與金人、偽齊相持於淮、泗者半年,綱奏:「兩兵相持,非出奇不足以取勝。願速遣驍將,自淮南約岳飛為掎角,夾擊之,大功可成。」已而宋師屢捷,劉光世、張俊、楊沂中大破偽齊兵於淮、肥之上。



 車駕進發幸建康。綱奏乞益飭戰守之具,修築沿淮城壘,且言:「願陛下勿以去冬驟勝而自怠,勿以目前粗定而自安,凡可以致中興之治者無不為,凡可以害中興之業者無不去。要以修政事,信賞罰,明是非,別邪正,招徠人材,鼓作士
 氣,愛惜民力,順導眾心為先。數者既備,則將帥輯睦,士卒樂戰,用兵其有不勝者哉?」



 淮西酈瓊以全軍叛歸劉豫,綱指陳朝廷有措置失當者、深可痛惜者及當監前失以圖方來者凡十有五事,奏之。張浚引咎去相位,言者引漢武誅王恢為比。綱奏曰:「臣竊見張浚罷相,言者引武帝誅王恢事以為比。臣恐智謀之士卷舌而不談兵,忠義之士扼腕而無所發憤,將士解體而不用命,州郡望風而無堅城,陛下將誰與立國哉?張浚措置失當,
 誠為有罪,然其區區徇國之心,有可矜者。願少寬假,以責來效。」



 時車駕將幸平江,綱以為平江去建康不遠,徒有退避之名,不宜輕動。復具奏曰:



 臣聞自昔用兵以成大業者,必先固人心,作士氣,據地利而不肯先退,盡人事而不肯先屈。是以楚、漢相距於滎陽、成皋間,高祖雖屢敗,不退尺寸之地;既割鴻溝,羽引而東,遂有垓下之亡。曹操、袁紹戰於官渡,操雖兵弱糧乏,荀彧止其退避;既焚紹輜重,紹引而歸,遂喪河北。由是觀之,今日之事,
 豈可因一叛將之故,望風怯敵,遽自退屈?果出此謀,六飛回馭之後,人情動搖,莫有固志,士氣銷縮,莫有鬥心。我退彼進,使敵馬南渡,得一邑則守一邑,得一州則守一州,得一路則守一路;亂臣賊子,黠吏奸氓,從而附之,虎踞鴟張,雖欲如前日返駕還轅,復立朝廷於荊棘瓦礫之中,不可得也。



 借使敵騎沖突,不得已而權宜避之,猶為有說。今疆埸未有警急之報,兵將初無不利之失,朝廷正可懲往事,修軍政,審號令,明賞刑,益務固守。而
 遽為此擾擾,棄前功,挑後患,以自趨於禍敗,豈不重可惜哉!八年,王倫使北還,綱聞之,上疏曰:



 臣竊見朝廷遣王倫使金國,奉迎梓宮。今倫之歸,與金使偕來,乃以「詔諭江南」為名,不著國號而曰「江南」,不云「通問」而曰「詔諭」,此何禮也?臣請試為陛下言之。金人毀宗社,逼二聖,而陛下應天順人,光復舊業。自我視彼,則仇讎也;自彼視我,則腹心之疾也,豈復有可和之理?然而朝廷遣使通問,冠蓋相望於道,卑辭厚幣,無所愛惜者,以二聖在其
 域中,為親屈己,不得已而然,猶有說也。至去年春,兩宮兇問既至,遣使以迎梓宮,亟往遄返,初不得其要領。今倫使事,初以奉迎梓宮為指,而金使之來,乃以詔諭江南為名。循名責實,已自乖戾,則其所以罔朝廷而生後患者,不待詰而可知。



 臣在遠方,雖不足以知其曲折,然以愚意料之,金以此名遣使,其邀求大略有五:必降詔書,欲陛下屈體降禮以聽受,一也。必有赦文,欲朝廷宣布,班示郡縣,二也。必立約束,欲陛下奉藩稱臣,稟其號令,
 三也。必求歲賂,廣其數目,使我坐困,四也。必求割地,以江為界,淮南、荊襄、四川,盡欲得之,五也。此五者,朝廷從其一,則大事去矣。



 金人變詐不測,貪婪無厭,縱使聽其詔令,奉藩稱臣,其志猶未已也。必繼有號令,或使親迎梓宮,或使單車入覲,或使移易將相,或改革政事,或竭取租賦,或朘削土宇。從之則無有紀極,一不從則前功盡廢,反為兵端。以為權時之宜,聽其邀求,可以無後悔者,非愚則誣也。使國家之勢單弱,果不足以自振,不得
 已而為此,固猶不可,況土宇之廣猶半天下,臣民之心戴宋不忘,與有識者謀之,尚足以有為,豈可忘祖宗之大業,生靈之屬望,弗慮弗圖,遽自屈服,冀延旦暮之命哉?



 臣願陛下特留聖意,且勿輕許,深詔群臣,講明利害、可以久長之策,擇其善而從之。



 疏奏,雖與眾論不合,不上以為忤,曰:「大臣當如此矣。」



 九年,除知潭州、荊湖南路安撫大使,綱具奏力辭,曰:「臣迂疏無周身之術,動致煩言。今者罷自江西,為日未久,又蒙湔祓,畀以帥權。昔漢文
 帝聞季布賢,召之,既而罷歸,布曰:『陛下以一人之譽召臣,一人之毀去臣,臣恐天下有以窺陛下之淺深。』顧臣區區進退,何足少多。然數年之間,亟奮亟躓,上累陛下知人任使之明,實有系於國體。」詔以綱累奏,不欲重違,遂允其請。次年薨,年五十八。訃聞,上為軫悼,遣使賻贈,撫問其家,給喪葬之費。贈少師,官其親族十人。



 綱負天下之望,以一身用舍為社稷生民安危。雖身或不用,用有不久,而其忠誠義氣,凜然動乎遠邇。每宋使至燕
 山,必問李綱、趙鼎安否,其為遠人所畏服如此。綱有著《易傳》內篇十卷、外篇十二卷,《論語詳說》十卷,文章、歌詩、奏議百餘卷,又有《靖康傳信錄》、《奉迎錄》、《建炎時政記》、《建炎進退志》、《建炎制詔表札集》、《宣撫荊廣記》、《制置江右錄》。



 論曰:以李綱之賢,使得畢力殫慮於靖康、建炎間,莫或撓之,二帝何至於北行,而宋豈至為南渡之偏安哉?夫用君子則安,用小人則危,不易之理也。人情莫不喜安而惡危。然綱居相位僅七十日,其謀數不見用,獨於黃
 潛善、汪伯彥、秦檜之言,信而任之,恆若不及,何高宗之見,與人殊哉?綱雖屢斥,忠誠不少貶,不以用舍為語默,若赤子之慕其母,怒呵猶噭□焉挽其裳裾而從之。嗚呼,中興功業之不振,君子固歸之天,若綱之心,其可謂非諸葛孔明之用心歟?



\end{pinyinscope}