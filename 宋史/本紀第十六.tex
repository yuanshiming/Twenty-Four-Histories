\article{本紀第十六}

\begin{pinyinscope}

 神宗三



 三
 年春正月乙丑朔,以大行太皇太后在殯,不視朝。癸酉,升許州為穎昌府。丙子,降穎昌囚罪一等,徒以下釋之。戊寅,上太皇太后謚曰慈聖光獻。戊子,詔審刑院、刑
 部斷議官失入者,歲具數罰之。己丑,高麗國遣使來貢。白虹貫日。辛卯,於闐國大首領阿令顛顙溫等來貢。癸巳,白虹貫日。



 二月丙午,以翰林學士章惇參知政事。丙辰,始御崇政殿視朝。丁巳,命輔臣禱雨。



 三月乙丑,工部侍郎、同平章事吳充罷為觀文殿大學士、西太一宮使。癸酉,葬慈聖光獻皇后於永昭陵。丙子,南丹州入貢,以刺史印賜之。乙酉,祔慈聖光獻皇后神主於太廟。戊子,降兩京、河陽囚罪一等,民緣山陵役者,蠲其賦。己丑,以
 慈聖光獻皇后弟昭德軍節度使曹佾為司徒兼中書令,改護國軍節度使,餘親屬加恩有差。



 夏四月乙未,觀文殿大學士吳充薨。丁酉,封宗暉為濮陽郡王,濮安懿王子孫皆進官一等。己亥,遼遣耶律永芳等來賀同天節。乙巳,以瀘州夷乞弟侵擾,詔邊將討之。戊申,乞弟寇戎州,兵官王宣等戰歿。甲寅,罷群牧行司,復置提舉買馬監牧司。乙卯,令御史分案諸路監司。庚申,詔御史臺六察以糾劾多寡為殿最,任滿取旨升黜。辛酉,增國子
 監歲賜錢六千緡。



 五月乙丑,詔自今三伏內,五日一御前殿。辛巳,以穎昌進士劉堂上《制盜十策》,授徐州蕭縣尉。甲申,復命韓存寶經制瀘夷。詔改都大提舉導洛通汴司為都提舉汴河堤岸司。是月,青州臨朐、益都石化為面。



 六月甲午,日有五色雲。戊戌,詔省宗室教授,存十三員。丙午,詔中書詳定官制。罷兵部勾當公事官。詔河北、河東、陜西路各選文武官一員提舉義勇保甲。壬子,詔罷中書門下省主判官,歸其事於中書。是月,安州、臨
 江軍產芝及連理麥。



 秋七月庚午,河決澶州。甲戌,詔自今遇大禮罷上尊號。癸未,彗出太微垣。丙戌,避殿減膳,詔求直言。丁亥,罷群神從祀明堂。戊子,太白晝見。



 八月乙巳,罷省、寺、監官領空名者。癸丑,遣王存等賀遼主生辰、正旦。戊午,彗不見。九月壬戌,增宣祖定州東安墳地二十頃及守園戶。丙寅,御殿復膳。乙亥,正官名。以開府儀同三司易中書令、侍中、同平章事,特進易左、右僕射,自是以下至承務郎易秘書省校書郎、正字、將作監主
 簿有差,檢校僕射以下及階散憲銜並罷,詳在《職官志》。辛巳,大饗明堂,以英宗配,赦天下。癸未,薛向、孫固並為樞密副使。乙酉,詔即景靈宮作十一殿,以時王禮祠祖宗。以王安石為特進,改封荊國公。丙戌,進封岐王顥為雍王,嘉王頵為曹王,並為司空。文彥博為太尉。封曹佾為濟陽郡王,宗旦為華陰郡王。馮京為樞密使。薛向罷知穎州。丁亥,以呂公著為樞密副使。閏九月乙卯,加文彥博河東、永興軍節度使,以富弼為司徒。



 十一月己丑
 朔,日當食,雲陰不見。十二月甲辰,遼遣蕭偉等來賀正旦。



 四年春正月乙未,命步軍都虞候林廣代韓存寶經制瀘夷。庚子,詔試進士加律義。辛亥,於闐來貢。馮京罷知河陽。孫固知樞密院,龍圖閣直學士韓縝同知樞密院事。



 二月辛未,置秦州鑄錢監。己卯,分東南團結諸軍為十三將。



 三月乙未,詔在京官毋舉闢執政有服親。癸卯,章惇罷知蔡州。甲辰,以翰林學士張璪參知政事。乙巳,
 命官閱九軍營陣法於京城南。戊申,大閱。丙辰,董氈遣使來貢。



 夏四月癸亥,遼遣耶律祐等來賀同天節。御延和殿閱試保甲。己巳,詔罷南郊合祭天地,自今親祀北郊,如南郊儀,有故不行,則以上公攝事。壬申,慮囚。山陰縣主簿餘行之謀反,伏誅。乙酉,河決澶州小吳埽。



 五月丁酉,詔河東路提點刑獄劉定專振被水民。戊申,封晉程嬰為成信侯,公孫杵臼為忠智侯,立廟於絳州。



 六月戊午,河北諸郡蝗生。癸未,命提點開封府界諸縣公事
 楊景略、提舉開封府界常平等事王得臣督諸縣捕蝗。



 秋七月己丑,太白晝見。庚寅,西邊守臣言夏人囚其主秉常,詔陜西、河東路討之。甲午,鄜延、涇原、環慶、熙河、麟府路各賜金銀帶、錦襖、銀器、鞍轡、象笏。甲辰,韓存寶坐逗留無功伏誅。丁未,大軍進攻米脂砦。己酉,詔曾鞏充史館修撰,專典史事。詔內外官司舉官悉罷。令大理卿崔臺符同尚書吏部,審官東西、三班院議選格。



 八月乙卯朔,罷中書堂選,悉歸有司。丙辰,詔蠲河北東路災傷
 州軍今年夏料役錢。辛酉,夏人寇臨川堡,詔董氈會兵伐之。以金州刺史燕達為武康軍節度使。己巳,復置滑州。丁丑,熙河經制李憲敗夏人於西市新城,獲酋首三人、首領二十餘人。庚辰,又襲破於女遮谷,斬獲甚眾。辛巳,司馬光、趙彥若上所修《百官公卿年表》十卷,《宗室世表》三卷。



 九月乙酉,董氈遣使來貢,且言已遣首領洛施軍篤喬阿公等將兵三萬會擊夏。



 ……
 延經略副使種諤率眾擊破之。辛亥,種諤又敗夏人於無定川。



 十月丁巳,米脂砦降。己未,拂菻國來貢。庚申,熙河兵至女遮谷,與夏人遇,戰敗之。乙丑,涇原兵至磨哆隘,遇夏人,與其統軍梁大王戰,敗之,追奔二十里,斬大首領沒囉
 臥沙、監軍使梁格鬼等十五級,獲首領統軍侄訖多埋等二十二人。己巳,入銀州。庚午,環慶行營經略使高遵裕復清遠軍。種諤遣曲珍等領兵通黑水安定堡,路遇夏人,與戰,破之,斬獲甚眾。癸酉,復韋州。乙亥,李憲敗夏人於屈吳山。丁丑,曲珍與夏人戰於蒲桃山,敗之。戊寅,種諤入夏州。詔諸將存撫降人。辛巳,史館修撰曾鞏乞收採名臣高士事跡遺文,詔從之。涇原節制王中正入宥州。



 十一月癸未朔,日有食之。丁亥,諸軍合攻靈州,種
 諤敗夏人於黑水。己丑,李憲敗夏人於囉逋川。辛卯,種諤降橫河平人戶,破石堡城,斬獲甚眾。辛丑,師還。癸卯,種諤至夏州索家平,兵眾三萬人,以無食而潰。丙午,高遵裕以師還,夏人來追,遂潰。十二月辛未,林廣破乞弟於納江。乙亥,慈聖光獻皇后禫祭,宰臣王珪等上表請聽樂,不許,自是五表,乃從之。戊寅,遼遣蕭福全等來賀正旦。



 五年春正月癸未朔,不受朝。丙申,御宣德門觀燈。己亥,
 白虹貫日。庚子,責授高遵裕郢州團練副使、本州安置。乙巳,作新渾儀、浮漏。辛亥,詔再議西討,以熙河經制李憲為涇原、熙河蘭會安撫制置使,李浩權安撫副使。



 二月癸丑朔,頒三省、樞密、六曹條制。詔鄜延軍士病不能歸者,賜其家絹十匹。丙辰,以乞弟平,班師。辛酉,詔:董氈首領結鄰死,其朝辭物給其子董訥支藺氈,增賜絹百匹。癸亥,華陰郡王宗旦薨。丁卯,封武昌軍節度觀察留後宗惠為江夏郡王。癸酉,以出師,赦梓州路,減囚罪一
 等,民緣軍事役者蠲其賦。封董氈為武威郡王。丙子,渤泥來貢。



 三月壬辰,親策進士。甲午,策武舉。己亥,以日當食,避殿減膳,赦天下,降死罪一等,流以下原之。詔杭州歲修吳越王墳廟。壬寅,鄜延路副總管曲珍敗夏人於金湯。乙巳,賜進士、諸科出身千四百二十八人。丙午,雨土。



 夏四月壬子朔,日食不見。甲寅,御殿復膳。丁巳,遼遣耶律永端等來賀同天節。己未,沉括奏遣曲珍將兵綏德城,應援討葭蘆寨左右見聚羌落,詔從之。乙丑,以直
 龍圖閣徐禧知制誥、權御史中丞。癸酉,官制成。以王珪為尚書左僕射兼門下侍郎,蔡確為尚書右僕射兼中書侍郎。甲戌,太中大夫章惇為門下侍郎,張璪為中書侍郎,翰林學士薄宗孟為尚書左丞,翰林學士王安禮為尚書右丞。錄唐段秀實後,復其家。丁丑,同知樞密院呂公著罷知定州。



 五月辛已朔,行官制。丁亥,賞蠻將士有差。癸巳,豐州卒張世矩等作亂,伏誅。其黨王安以母老,詔特原之。作尚書省。戊戌,詔兩省官人舉可任御
 史者各二人。甲辰,遣給事中徐禧治鄜延邊事。



 六月辛亥朔,環慶經略司遣將與夏人戰,破之,斬其統軍嵬名妹精嵬、副統軍訛勃遇。甲寅,王珪上《兩朝史》。戊午,詔修《兩朝寶訓》。詔以成都路供給瀘州邊事,曲赦,免二稅。甲子,改翰林醫官院為醫官局。壬申,交址獻馴犀二。癸酉,豫章郡王宗諤薨。戊寅,曲珍等敗夏人於明堂川。作天源河。秋七月辛巳,廣西經略司言知宜州王奇與賊戰,敗績。壬午,詔罷大理寺官赴中書省讞案。戊子,詔御史
 中丞舒但舉任言事或察官十人。辛卯,詔尚書考功員外郎蔡京編手詔。庚子,以蔡京為起居郎,仍同詳定官制。丁未,垂拱殿宴修史官。己酉,始建雩壇,祀上帝,以太宗配。



 八月庚戌朔,封御侍武氏為才人。壬子,進封均國公傭為延安郡王。以昭容朱氏為賢妃。庚申,帝有疾。詔歲以四孟月朝獻景靈宮。辛未,遣韓忠彥等賀遼主生辰、正旦。鳳州團練使種諤以行軍迂道,降授文州刺史。壬申,詔罷增減幕職、州縣官奉。甲戌,城永樂。戊寅,河決
 原武。



 九月丁亥,夏人三十萬眾寇永樂,曲珍戰不利,裨將寇偉等死之,夏人遂圍城。己丑,帝以疾愈,降京畿囚罪一等,徒以下釋之。壬辰,遣使行視畿縣民被水患者。乙未,詔張世矩等將兵救永樂砦。戊戌,永樂陷,給事中徐禧、內侍李舜舉、陜西轉運判官李稷死之。己亥,詔客省、引進、四方館、東西上閣門各置使、副等職。庚子,安化蠻寇宜州,知州王奇死之,詔贈忠州防禦使。辛丑,賞董氈將士有差。癸卯,滑州河水溢。



 冬十月辛亥,洛口、廣武
 大河溢。甲寅,知延州沉括以措置乖方,責授均州團練副使、隨州安置;鄜延路副都總管曲珍以城陷敗走,降授皇城使。丙辰,修定景靈宮儀。乙丑,詔贈永樂死事臣徐禧金紫光祿大夫、吏部尚書,李舜舉昭化軍節度使,並賜謚忠愍,李稷朝奉大夫、工部侍郎,入內高品張禹勤皇城使,各推恩賜贈有差。癸酉,貶知太原府、資政殿大學士呂惠卿知單州。



 十一月戊寅朔,罷御史察諸路。壬午,景靈宮成,告遷祖宗神御。癸未,初行酌獻禮。乙酉,
 以奉安神御赦天下,官與享大臣子若孫一人。庚寅,紫宸殿宴侍祠官。十二月丁巳,新樂成。以賢妃周氏為德妃。辛酉,塞原武決河。丙寅,休日御延和殿,引進對官十人。辛未,西南龍蕃來貢。壬申,遼遣耶律儀等來賀正旦。丙子,錄永樂死事將皇城使寇偉等十三人及東上閣門副使景思誼等九十人,贈賜有差。



 六年春正月丁丑朔,御大慶殿受朝,始用新樂。儀鸞司徹幕屋壞,毀玉輅。甲申,白虹貫日。丁亥,朝獻景靈宮。己
 丑,層檀入貢。庚寅,御宣德門觀燈。癸巳,詔御史六察罷上下半年更易法。乙未,詔修周、漢以來陵廟。乙巳,御崇政殿閱武士。丙午,封楚三閭大夫屈平為忠潔侯。



 二月丁未,夏人數十萬眾攻蘭州,鈐轄王文鬱率死士七百餘人擊走之。丙辰,以夏人犯蘭州,貶熙河經略使李憲為經略安撫都總管,以王文鬱為西上閣門使、知蘭州,副使李浩為四方館使。甲子,詔供備庫使高遵治、西京左藏庫副使張壽各降一官。



 三月辛卯,夏人寇蘭州,副
 總管李浩以衛城有功,復隴州團練使。乙未,休日御延和殿,引進對官八人。丙申,河東將薛義敗夏人於葭蘆西嶺。戊戌,以檢校太尉、上柱國、太原郡開國公王拱辰為武安軍節度使。麟、府州將郭忠詔等敗夏人於乜離抑部,詔行賞有差。己亥,河東將高永翼敗夏人於真卿流部。



 夏四月己酉,朝獻景靈宮。辛亥,遼遣蕭固等來賀同天節。甲子,禮部郎中林希上《兩朝寶訓》。李浩敗夏人於巴義溪。辛未,雨土。壬申,御邇英閣,蔡卞進講《周禮》。



 五
 月丙子朔,於闐入貢。甲申,以時暑,趣決開封、大理獄。庚寅,以旱,慮囚。甲午,夏人寇蘭州,右侍禁韋定死之。癸卯,詔賜資州孝子支漸粟帛。是月,夏人寇麟州,知州訾虎敗之。六月乙巳朔,詔御史臺六察各置御史一員。癸丑,詔御史中丞、兩省官各舉可任言事或監察御史五人。



 閏月乙亥朔,夏主秉常請修貢,許之。戊寅,詔陜西、河東毋輒出兵。丙戌,詔內外文武各舉應武舉一人。汴水溢。丙申,太師、守司徒、韓國公富弼薨,謚文忠。



 秋七月乙卯,
 祔孝惠、孝章、淑德、章懷皇后於廟。丙辰,以四后祔廟,降京畿囚罪一等,流以下原之。孫固罷知河陽。以同知樞密院韓縝知樞密院,戶部尚書安燾同知樞密院。戊午,朝獻景靈宮。



 八月丙子,賜升祔陪祠官宴於尚書省。己卯,太白晝見。乙酉,遣蔡京等賀遼主生辰、正旦。辛卯,蒲宗孟罷,王安禮為尚書左丞,吏部尚書李清臣為尚書右丞。



 九月癸卯朔,日有食之。



 冬十月癸酉朔,秉常遣使上表,請復修職貢,乞還舊疆。戊子,封孟軻為鄒國公。癸
 巳,會稽郡王世清薨。庚子,尚書省成。辛丑,封馬援為忠顯王。



 十二月癸卯,加上仁宗謚曰體天法道極功全德神文聖武睿哲明孝皇帝,英宗曰體乾應歷隆功盛德憲文肅武睿神宣孝皇帝。甲辰,朝獻景靈宮。乙巳,朝享太廟。丙午,祀昊天上帝於圜丘,赦天下。甲寅,文彥博以太師致仕。乙卯,以觀文殿大學士韓絳為建雄軍節度使。庚申,幸尚書省,官執政五服內未仕者一人,進尚書以下官一等。



 七年春正月丙午,封洺州防禦使世準為安定郡王。癸丑,夏人寇蘭州,李憲等擊走之。甲寅,以賢妃朱氏為德妃。



 二月甲戌,太師文彥博入覲,置酒垂拱殿。癸未,進封濮陽郡王宗暉為嗣濮王,封宗晟為高密郡王,宗綽為建安郡王,宗隱為安康郡王,宗瑗為漢東郡王,宗愈為華原郡王。



 三月辛丑,賜文彥博宴於瓊林苑,帝制詩以賜之。庚申,御崇政殿大閱。壬戌,詔賜鬼章寫經紙,還其所獻馬。癸亥,白虹貫日。



 夏四月辛未,大食國來貢。乙亥,
 遼遣蕭浹等來賀同天節。丁丑,賜饒州童子朱天錫五經出身。丙戌,景靈宮天元殿門生芝草六本。壬辰,朝獻景靈宮。癸巳,夏人寇延州安塞堡,將官呂真敗之。



 五月壬子,慮囚,降死罪一等,杖以下釋之。辛酉,白虹貫日。壬戌,以孟軻配食文宣王,封荀況、楊雄、韓愈為伯,並從祀。詔諸路帥臣、監司等舉大使臣為將領。



 六月丙子,夏人寇德順軍,巡檢王友死之。辛卯,江夏郡王宗惠薨。



 秋七月甲辰,伊、洛溢,河決元城。丙午,遣使振恤,賜溺死者家
 錢。壬子,朝獻景靈宮。甲寅,王安禮罷。



 八月庚午,詔王光祖遣人招諭乞弟,許出降免罪補官。是歲,乞弟死。辛巳,遣陳睦等賀遼主生辰、正旦。



 九月壬寅,西南龍蕃來貢。乙巳,三佛齊來貢。乙丑,夏人圍定西城,熙河將秦貴敗之。



 冬十月乙亥,夏人寇熙河。庚辰,饒州童子朱天申對於睿思殿,賜五經出身。辛巳,朝獻景靈宮。戊子,詔分畫交址界,以六縣二峒賜之。乙未,夏人寇靜邊砦,涇原將彭孫敗之。



 十一月丁酉朔,寇清邊砦,隊將白玉、李貴死
 之。甲辰,夏國主秉常遣使來貢。乙卯,太白晝見。十二月戊辰,端明殿學士司馬光上《資治通鑒》,以光為資政殿學士,降詔獎諭。庚寅,詔門下、中書外省官同舉言事御史。辛卯,遼遣耶律襄等來賀正旦。是歲,河東饑。河北水,壞洺州廬舍,蠲其稅。



 八年春正月戊戌,帝不豫。甲辰,赦天下。乙巳,使輔臣代禱景靈宮。乙卯,分遣群臣禱於天地、宗廟、社稷。



 二月辛巳,開寶寺貢院火。丁亥,命禮部鎖試別所。癸巳,上疾甚,
 遷御福寧殿,三省、樞密院入見,請立皇太子及請皇太后權同聽政,許之。



 三月甲午朔,立延安郡王傭為皇太子,賜名煦,皇太后權同處分軍國事。乙未,赦天下,遣官告於天地、宗廟、社稷、諸陵。丁酉,皇太后命吏部尚書曾孝寬為冊立皇太子禮儀使。戊戌,上崩於福寧殿,年三十有八。皇太子即皇帝位,尊皇太后為太皇太后,皇后為皇太后,德妃朱氏為皇太妃。太皇太后權同處分軍國事。



 九月己亥,上大行皇帝謚曰英文烈武聖孝皇帝,
 廟號神宗。



 十月乙酉,葬於永裕陵。



 贊曰:帝天性孝友,其入事兩宮,必侍立終日,雖寒暑不變。嘗與岐、嘉二王讀書東宮,侍講王陶講諭經史,輒相率拜之,由是中外翕然稱賢。其即位也,小心謙抑,敬畏輔相,求直言,察民隱,恤孤獨,養耆老,振匱乏。不治宮室,不事游幸,歷精圖治,將大有為。未幾,王安石入相。安石為人,悻悻自信,知祖宗志吞幽薊、靈武,而數敗兵,帝奮然將雪數世之恥,未有所當,遂以偏見曲學起而乘之。
 青苗、保甲、均輸、市易、水利之法既立,而天下洶洶騷動,慟哭流涕者接踵而至。帝終不覺悟,方斷然廢逐元老,擯斥諫士,行之不疑。卒致祖宗之良法美意,變壞幾盡。自是邪佞日進,人心日離,禍亂日起。惜哉!



\end{pinyinscope}