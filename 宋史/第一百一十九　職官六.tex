\article{第一百一十九 職官六}

\begin{pinyinscope}

 殿前司侍衛親軍環衛官皇城司三衛官客省引進四方館東西上閣門帶御器械人內內侍省內侍省開封府臨安府河南應天府次府節度使承宣觀察防禦等使



 殿前司都指揮使、副都指揮使、都虞候各一人。
 掌殿前諸班直及步騎諸指揮之名籍,凡統制、訓練、番衛、戍守、遷補、賞罰,皆總其政令。而有都點檢、副都點檢之名,在都指揮之上,後不復置,入則侍衛殿陛,出則扈從乘輿,大禮則提點編排,整肅禁衛鹵簿儀仗,掌宿衛之事,都指揮使以節度使為之。而副都指揮使、都虞候以刺史以上充。資序淺則主管本司公事,馬步軍亦如之。備則通治,闕則互攝。凡軍事皆行以法,而治其獄訟。若情不中法,則稟奏聽旨。



 騎軍有殿前指揮使、內殿
 直、散員、散指揮、散都頭、散祗候、金槍班、東西班、散直、鈞容直及捧日以下諸軍指揮,步軍有御龍直、骨朵子直、弓箭直、弩直及天武以下諸軍指揮。諸班有都都虞候指揮使、都軍使、都知、副都知、押班。御龍諸直,有四直都虞候,本直各有虞候、指揮使、副指揮使、都頭、副都頭、十將、將虞候。騎軍、步軍,有捧日、天武左右四廂都指揮使,捧日、天武左右廂各有都指揮使。每軍有都指揮使、都虞候,每指揮有指揮使、副指揮使,每都有軍
 使、副兵馬使、十將、將虞候、承局、押官,各以其職隸於殿前司。



 元祐七年,簽書樞密院王嚴叟言:「祖宗以來,三帥不曾闕兩人,若殿帥闕,難於從下超補,姚麟系殿前都虞候,合升作步軍副都指揮使。」紹聖三年,詔:「殿前指揮使金鎗弩手班、龍旗直所減人額及排定班分,並依元豐詔旨。」政和四年,詔:「殿前都指揮使在節度使之上,殿前副都指揮使在正任承宣使之上,殿前都虞候在正任防禦使之上。」



 渡江後,都指揮間虛不除,則以主管殿
 前司一員任其事。其屬有幹辦公事、主管禁衛二員,淮備差遣、淮備差使、點檢醫藥飯食各一員,書寫機宜文字一員。本司掌諸班直禁旅扈衛之事,捧日、天武四廂隸焉。訓齊其眾,振飭其藝,通輪內宿,並宿衛親兵並聽節制。其下有統制、統領、將佐等分任其事。凡諸軍班直功賞、轉補,行門拍試、換官,閱實排連以詔於上;諸殿侍差使年滿出職,祗應參班,核其名籍;以時教閱,則謹鞍馬、軍器、衣甲之出入;軍兵有獄訟,則以法鞫治。初,渡江
 草創,三衙之制未備,稍稍招集,填置三帥。資淺者,各有主管某司公事之稱。又別置御營司,擢王淵為都統制。其後外州駐扎,又有御前諸軍都統制之名。又並入神武軍,以舊統制、統領改充殿前司統制、統領官。



 乾道中,臣僚言:「三衙軍制名稱不正,以舊制論之,軍職大者凡八等,除都指揮使或不常置外,曰殿前副都指揮使、馬軍副都指揮使、步軍副都指揮使。次各有都虞候,次有捧日、天武四廂都指揮使。龍、神衛四廂都指揮使。秩秩
 有序,若登第然。降此而下,則分營、分廂各置副都指揮使。邊境有事,命將討捕,則旋立總管、鈐轄、都監之名,使各將其所部以出,事已復初。今以宿衛虎士而與在外諸軍同其名,以統制、統領為之長,又使遙帶外路總管、鈐轄,皆非舊典。所當法祖宗之舊,正三衙之名,改諸軍為諸廂,改統制以下為都虞候、指揮使,要使宿衛之職,預有差等,士卒之心,明有所系,異時拜將,必無一軍皆驚之舉。」時不果行。淳熙以後,四廂之職多虛,而殿司職、
 司有權管幹,有時暫照管之號,愈非乾道以前之比矣。



 侍衛親軍馬軍都指揮使、副都指揮使、都虞候各一人,掌馬軍諸指揮之名籍,凡統制、訓練、番衛、戌守、遷補、賞罰,皆總其政今;侍衛扈從,及大禮宿衛,所掌如殿前司官。所領馬軍,自龍衛而下有左右四廂都指揮使,龍衛左右廂各有都指揮使。每軍有都指揮使、都虞候,每指揮有指揮使、副指揮使,每都有軍使、副兵馬使、十將、將虞候、承勾、押官,各以其職隸於馬軍司。政和四
 年,詔以馬軍都指揮使、馬軍副都指揮使在正任觀察使之上,馬軍都虞候在正任防禦使之上。



 中興後。置主管侍衛馬軍司一員。其屬有乾辦公事、淮備差遣、點檢醫藥飯食各一員。掌出戌建康,差主管機宜文字一員,掌馬軍之政令。凡出入扈衛、守宿以奉上,開收閱習、轉補以勵下,如殿前司。凡名籍核其在亡,過則以法繩之,有巡防敕應,則糾率差撥龍衛四廂隸焉。



 侍衛親軍步軍都指揮使、副都指揮使、都虞候
 各一人。掌步軍諸指揮之名籍,凡統制、訓練、番衛、戌守、遷補、賞罰,皆總其政令;侍衛扈從,及大禮宿衛,如殿前司。所領步軍、自神衛而下有左右四廂都指揮使,左右廂各有都指揮使。每軍有都指揮使、都虞候,每指揮有指揮使、副指揮使、每都有都頭、副都頭、十將、將、虞候、承勾、押官,各以其職隸於步軍司。政和四年,詔以步軍都指揮使、步軍副都指揮使在正任觀察使之上,虞候在正任防禦使之上。



 中興後,置主管侍衛步軍司一員。其屬有乾辦公事二員。淮備差
 遣、點檢醫藥飯食各一員,掌步軍之政令。凡出入扈衛、守宿以奉上,開收閱習、轉補以勵下,如殿前司。凡名籍校其在亡,過則以法繩之,有巡防敕應,則糾率差撥神衛四廂隸焉。



 環衛官左、右金吾衛上將軍大將軍將軍中郎將郎將



 左、右衛上將軍大將軍將軍中郎將郎將



 左、右驍衛上將軍大將軍將軍



 左、右武衛上將軍大將軍將軍



 左、右屯衛上將軍大將軍將軍



 左、右領軍衛上將軍大將軍將軍



 左、右監門衛上將軍大將軍將軍



 左、右千牛衛上將軍大將軍將軍中郎將郎將



 諸衛上將軍、大將軍、將軍,並為環衛官,無定員,皆
 命宗室為之,亦為武臣之贈典。大將軍以下,又為武官責降散官。政和中,改武臣官制,而環衛如故,蓋雖有四十八階,別無所領故也。靖康元年,詔以武安軍節度使錢景臻等為左金吾衛上將軍,保信軍節度使劉敷等為右金吾衛上將軍,用御史中丞陳過庭言,遵藝祖開寶初罷王彥超、武行德等歸環衛故事也。其禁兵分隸殿前及侍衛兩司,所稱十二衛將軍,皆空官無實,中興多不除授。隆興中,始命學士洪遵等討論典故,復置十六衛,
 號環衛官。其法:節度使則領左、右金吾衛上將軍,承宣使則領左、右衛上將軍,在內則兼帶,在外則不帶;正任為上將軍,遙郡為大將軍,正親兄弟子孫試充。又詔祖宗諸後自明肅至欽慈諸後及后妃嬪御之家,各具本宗堪充諸衛官以名銜聞。又詔三衛郎為三衛侍郎。又詔博士並差文臣。崇寧四年二月置,五年正月罷。



 皇城司乾當官七人,以武功大夫以上及內侍都知、押班充。掌宮城出入之禁令,凡周廬宿衛之事、宮門啟
 閉之節皆隸焉。每門給銅符二、鐵牌一,左符留門,右符請鑰,鐵牌則請鑰者自隨,以時參驗而啟閉之。總親從、親事官名籍,辨其宿衛之地,以均其番直。人物偽冒不應法,則譏察以聞。凡臣僚朝覲,上下馬有定所,自宰相、親王以下,所帶人從有定數,揭榜以止其喧哄。元豐六年,詔乾當皇城司,除兩省都知、押班外,取年深者減罷。止留十員。元祐元年,詔乾當官閱三年無過者遷秩一等,再任滿者減磨勘二年。元符元年,詔:「應宮城出入請
 納官物,呈稟公事,傳送文書,並御廚、翰林、儀鸞司非次祗應,聽於便門出入,即不由所定門者,論如蘭入律。應差辦人物入內,及內諸司差人往他所應奉,並前一日具名數與經歷諸門報皇城司。」二年,詔皇城司任滿酬獎依熙寧五年指揮,再任滿無遺闕,取旨。政和五年,詔皇城司可創置親從弟五指揮,以七百人為額,親從官舊有四指揮,元客共二千二百七十人,



 仍以五尺九寸一分六厘使為將軍,副使為中郎將,使臣以下為左、右郎將,通以十員為額,宗
 室不在此例。除管軍則解,或領閣門、皇城之類則仍帶,雖戚里子弟,非戰功人不除,批書印紙屬殿前司。是時,帝諭宰相,以為如文臣館閣儲才之地。紹熙初,嘗欲留闕以儲將才,循初意也。嘉泰中,復申明隆興之詔,屏除貪得妄進,以重環尹之官,嘉定二年,復因臣僚言,專以曾為兵將有功績及名將子孫之有才略者充。通前後觀之,可以見環衛儲才之意。



 三衛官三衛郎一員,秩比太中大夫。中郎為之貳,文
 武各一員,秩比朝議大夫。博士二員,主簿一員。親衛府郎十員,中郎十員;勛衛府郎十員,中郎十員;翊衛府郎二十員,中郎二十員;文武各四十員。三衛郎治其府之事。率其屬日直於殿陛,長在左,立起居郎之前;貳分左右,文東武西,立都承旨之後,仗退,治事於府。博士掌孝道,校試三衛所習文武之藝。親衛立於殿上兩旁,勛衛立於朵殿,翊衛立於兩階衛士之前。三衛郎依給、舍,中郎依少卿,餘依寺丞。親衛官以後妃嬪御之
 家有服親,及翰林學士並管軍正任觀察使以上子孫;勛衛官以勛臣之世、賢德之後有服親,太中大夫以上及正任團練使、遙郡觀察使以上;翊衛官以卿監、正任刺史、遙郡團練使以上,並以為等。其將校、十將、節級等應合行事件,比第四指揮及見行條貫。六年三月,應臣僚輒帶售雇人入宮門,罪賞並依宗室法,將帶過數止坐本官,若兼領外局,所定人從非隨本官輒入者,依闌入法。十一月,詔嘉王楷差提舉皇城司整肅隨駕
 禁衛所。靖康元年,詔應入皇城門,依法服本色,輒衣便服及不裹頭帽入出者並科罪。所隸官屬一:冰井務,掌藏冰以薦獻宗廟、供奉禁庭及邦國之用。若賜予臣下,則以法式頒之。



 中興初,為行營禁衛所,差主管官,掌出入皇城宮殿門等敕號,察其假冒,車駕行幸則糾察導從。紹興元年,改稱行在皇城司。提舉官一員,提點官二員,乾當官五員,以諸司副使、內侍都知押班充。掌皇城宮殿門,給三色牌號,稽驗出入。凡親從,親事官五指
 揮,入內院子、守闕入內院子指揮,總其名籍,均其勞役,察其功過而賞罰之。凡諸門必謹所守;蠲潔齊肅,郊祀大禮,則差撥隨從守衛;有宴設,則守門約闌。每年春秋,按賞親從逐指揮、親事官第一指揮、長行三色武藝、弓弩槍手。皇城周回或有墊陷,移文修整。嘉定間,臣僚言:「皇城一司,總率親從,嚴護周盧,參錯禁旅,權亞殿嚴,乞專以知閣、御帶兼領。仍立定親從員額,以革泛監。」並從之。



 客省、引進使客省使、副使各二人。掌國信使見辭宴賜及四方進奉、四夷朝覲貢獻之儀,受其幣而賓禮之,掌其饔餼欽食,還則頒詔書,授以賜予。宰臣以下節物,則視其品秩以為等。若文臣中散大夫、武臣橫行刺史以上還闕朝覲,掌賜酒饌。使闕,則引進、四方館、合門使副互權。大觀元年,詔客省、四方館不隸臺察。政和二年,改定武選新階,乃詔客省、四方館、引進司、東、西上合門所掌職務格法。並令尚書省具上。又詔高麗已稱國信,
 改隸客省。靖康元年,詔客省、引進司、四方館、西上合門為殿庭應奉,與東上合門一同隸中書省,不隸臺察。



 引進司使、副各二人。掌臣僚、蕃國進奉禮物之事,班四方館上。使闕,則客省、四方館互兼。



 四方館使二人。掌進章表,凡文武官朝見辭謝、國忌賜香,及諸道元日、冬至、朔旦慶賀起居章表,皆受而進之。郊祀大朝會,則定外國使命及致仕、未升朝官父老陪位之版,進士、道釋亦如之。掌凡護葬、賻贈、朝拜之事。
 客省、四方館,建炎初並歸東上合門,皆知合總之。



 東、西上合門東上合門、西上合門使各三人,副使各二人,宣贊舍人十人,舊名通事合人,政和中改。祗候十有二人。掌朝會宴幸、供奉贊相禮儀之事,使、副承旨稟命,舍人傳宣贊謁,祗候分佐舍人。凡文武官自宰臣、宗室自親王、外國自契丹使以下朝見謝辭皆掌之,視其品秩以為引班、敘班之次,贊其拜舞之節而糾其違失。若慶禮奉表,則東上合門掌之;慰禮進名,則西上合門掌之。月進班
 簿,歲終一易,分東西班揭貼以進。自客省而下,因事建官,皆有定員。遂立積考序遷之法,聽其領職居外,增置看班祗候六人,由看班遷至使皆五年,使以上七年,遇闕乃遷,無闕則加遙郡。



 元豐七年,詔客省、四方館使、副領本職外,官最高者一員兼領合門事。元祐元年,詔客省、四方館、合門並以橫行通領職事。紹聖三年,詔看班祗候有闕,令吏部選定,尚書省呈人材,中書省取旨差。崇寧四年,詔合門依元豐法隸門下省。大觀元年,詔合
 門依殿中省例,不隸臺察。政和六年,詔宣贊播告,直誦其辭。靖康元年,詔合門並立員額。監察御史胡舜陟奏:「合門之職,祖宗所重:宣贊不過三五人,熙寧間,通事舍人十三員。祗候六人,當時議者猶以為多。今舍人一百八員,祗候七十六員,看班四員,內免職者二百三員,由宦侍恩幸以求財,朱勉父子賣尤多,富商豪子往往得之。真宗時,諸王夫人因聖節乞補合門,帝曰:『此職非可以恩澤授。』不許。神宗即位之初,用宮邸直省官郭昭選為合門祗候,司馬光言:『此祖宗以蓄養賢才,在文臣為館職。』其重如此,今豈可賣以求財,乞賜裁省。」故有是詔。



 舊制有東、西上合門,多以處外戚勛貴。建炎初元,並省為一,其引進司、四方館並歸合門,客省循舊法,非橫行不許知
 合門。紹興元年,帝以朱錢孫藩邸舊人,稍習儀注,命轉行橫行一官,主管合門。又曰:「藩邸舊人,自內侍及使臣皆不與行在職任,止與外任,籛孫以合門無諳練人,故留之。」五年,詔右武大夫以上並稱知合門事兼客省、四方館事,官未至者,即稱同知合門事同兼客省、四方館事,以除授為序,稱同知者在知合門之下。宣贊舍人任傳宣引贊之事,與合門祗候並為合職,間帶點檢合門薄書公事。紹興中,許令供職,注授內外合入差遣,闕到
 然後免供職。其後供職舍人員數稍冗,裁定以四十員為額。



 乾道六年,上欲清合門之選,除宣贊舍人、合門祗候仍舊通掌贊引之職外,置合門舍人十員,以待武舉之入官者。掌諸殿覺察失儀兼侍立,駕出行幸亦如之。六參、常朝,後殿引親王起居。仿儒臣館閣之制,召試中書省,然後命之。又許轉對如職事官,供職滿三年與邊郡。淳熙間,置看班祗候,令忠訓郎以下充,秉義郎以上,始除合門祗候。又增重薦舉合門祗候之制,必廉幹有方
 略、善弓馬、兩任親民無遺闕及曾歷邊任者充。紹熙以來,立定員額。慶元初,申嚴合門長官選擇其屬之令,非右科前名之士不預召試,蓋以為右列清選云。



 帶御器械宋初,選三班以上武干親信者佩櫜鞬、御劍,或以內臣為之,止名:御帶」。咸平元年,改為帶御器械。景祐二年,詔自今無得過六人。慶歷元年,詔遇闕員,曾歷邊任有功者補之。中興初,諸將在外多帶職,蓋假禁近之名,為軍旅之重。紹興七年,樞密院言:「帶御器械官
 當帶插。」帝曰:「此官本以衛不虞,今乃佩數笴□箭,不知何用。方承平時,至飾以珠玉,車駕每出,為觀美而已。他日恢復,此等事當盡去之。」二十九年,詔中外舉薦武臣,無闕可處,增置帶御器械四員。然近侍亦或得之。乾道發來,詔立班樞密院檢詳文之上。淳熙間,凡正除軍中差遣或外任者,不許銜內帶行,又須供職一年,方與解帶恩例,於是屬鞬之職益加重焉。



 入內內侍省內侍省宋初,有內中高品班院,化
 五年,改入內內班院,又改入內黃門班院,又改內侍省入內內侍班院。景德三年,詔:「東門取索司可並隸內東門司,餘入內都知司;內東門都知司、內侍省入內內侍班院可立為入內內侍省,以諸司隸之。」宋初,有內班院,淳化五年,改為黃門,九月,又改內侍省。



 入內內侍省與內侍省號為前後省,而入內省尤為親近。通侍禁中、役縣褻近者,隸入內內侍省。拱侍殿中、備灑掃之職、役使雜品者,隸內侍省。入內內侍省有都都知、都知、副都知、
 押班、內東頭供奉官、內西頭供奉官、內侍殿頭、內侍高品、內侍高班、內侍黃門。內侍省有左班都知、副都知、押班、內東頭供奉官、內西頭供奉官、內侍殿頭、內侍高品,內侍高班、內侍黃門。自供奉官至黃門,以二百八十人為定員。凡內侍初補曰小黃門,經恩遷補則為內侍黃門。後省官闕,則以前省官補。押班次遷副都知,次遷都都知,遂為內臣之極品。



 熙寧中,入內內侍省內侍省都知、押班遂省,各以轉入先後相壓,永
 為定式。其官稱,則有內客省使、延福宮使、宣政使、宣慶使、昭宣使。元豐議改官制,張誠一欲易都知、押班之名,置殿中監以易內侍省。既而宰執進呈,神宗曰:「祖宗為此名有深意,豈可輕議?」政和二年,始遂改焉。以通侍大夫易內客省使,正侍大夫易延福宮使,中侍大夫易景福殿使,中亮大夫易宣慶使,中衛大夫易宣政使,拱衛大夫易昭宣使,供奉官易內東頭供奉官,左侍禁易內西頭供奉官,右侍禁易內侍殿頭,左班殿直易內侍高
 品,右班殿直易內侍高班,而黃門之名如故。



 其屬有:御藥院,勾當官四人,以入內內侍充,掌按驗方書,修合藥劑,以待進御及供奉禁中之用。內東門司勾,當官四人,以入內內侍充,掌宮禁人物出入,周知其名數而譏察之。合同憑由司,監官二人,掌禁中宣索之物,給其要驗,凡特旨賜予,皆具名數憑由,付有司淮給。管勾往來國信所,管勾官二人,以都知、押班充,掌契丹使介交聘之事。後苑勾當官,無定員,以內侍充,掌苑
 囿、池沼、臺殿種藝雜飾,以備游幸。造作所,掌造作禁中及皇屬婚娶之名物。龍圖、於昌、寶文閣,勾當四人,以入內內侍充,掌藏祖宗文章、圖籍及符瑞寶玩之物,而安像設以崇奉之。軍頭引見司,勾當官五人,以內侍省都知、押班及合門宣贊舍人以上充,掌供奉便殿禁衛諸軍入見之事,及馬,步兩直軍員之名。翰林院,勾當官一員,以內侍押班、都知充,總天文、書藝、圖晝、醫官四局,凡執伎以事上者皆在焉。



 中興以來,深懲內侍
 用事之弊,嚴前後省使臣與兵將官往來之禁,著內侍官不許出謁及接見賓客之令。紹興三十年,詔內侍省所掌職務不多,徒有冗費,可廢並歸入內內侍省。舊制,內侍遇聖節許進子,年十二試以墨義,即中程者,候三年引見供職。三十二年,殿中侍御史張震言宦者員眾,孝宗即命內侍省具見在人數,免會慶節進子,仍定以二百人為額。乾道間,以差赴德壽宮應奉闕人,增置二百五十人。紹熙三年,依宰臣奏,中官只令承受宮禁中
 事,不許預聞他事。嘉定初,詔內侍省陳乞恩例,親屬充寄班祗候,以十年為限。



 開封府牧、尹不常置,權知府一人,以待制以上充。掌尹正畿甸之事,以教法導民而勸課之。中都之獄訟皆受而聽焉,小事則專決,大事則稟奏。若承旨已斷者,刑部、御史臺無輒糾察。屏除寇盜,有奸伏則戒所隸官捕治。凡戶口、賦役、道釋之占京邑者,頒其禁令,會其帳籍。大禮,橋道頓遞則為之使,仗內奉引則差官攝牧。



 其屬有
 判官、推官四人,日視推鞫,分事以治。而佐其長,領南司者一人,督察使院,非刑獄訟訴則主行之。司錄參軍一人,折戶婚之訟,而通書六曹之案牒。功曹、倉曹、戶曹、兵曹、法曹、士曹參軍各一人,視其官曹分職蒞事。左右軍巡使、判官各二人,他掌京城爭鬥及推鞫之事。左右廂公事乾當官四人,掌檢覆推問,凡鬥訟事輕者聽論決。領縣十有八,鎮二十有四,令佐、訓練、征榷、監臨、巡警之官,知府事者率統隸焉。分案六,置吏六百。



 開封典
 司轂下,自建隆以來,為要劇之任。至熙寧間,增給吏錄,禁其受賕,省衙前役以寬民力,厘折獄訟歸於廂官,而治事視前日損去十四。元祐元年,詔府界捕盜官吏隸本府,與都大提舉司同管轄而掌其賞罰。置新城內左、右二廂。三年,以罷大理寺獄,置軍巡院判官一員。四年,罷新置二廂。六年,王嚴叟言:「左、右廳推官公事詞狀,初無通治明文,請事擊朝省及奏請通治外,餘並據號分治。」從之。紹聖元年,知府事錢勰言:「自祖宗以來,並分左
 右廳置推官各一員。近年止除推官,元祐中,並令分治。請依故事分左右廳,各除推官一員,作兩廳共治職事。」又言:「熙寧中,置舊城左右廂,元祐初,增置於新城內,四年,罷增置兩廂,今請復置。」從之。三年,詔開封、祥符知縣事自今選秩通判人充。四年,詔開封府所薦推、判官,並召對取旨。



 崇寧三年,蔡京奏:「乞罷權知府,置牧一員、尹一員,專總府事;少尹二員,分左右,貳府之政事。牧以皇子領之。尹以文臣充,在六曹尚書之下、侍郎之上。少尹
 在左右司郎官之下、列曹郎官之上。以士、戶、儀、兵、刑、工為六曹次序,司錄二員,六曹各二員,參軍事八員。開封、祥符兩縣置案仿此。易胥吏之稱,略依《唐六典》制度。」又請移開封府治所於舊尚書省,從之。



 太宗、真宗嘗任府尹,自到道後,知府者必帶「權」字,蔡京乃以潛邸之號處臣下,建置曹官以上凡十六員,比舊增要官十一員,



 五年,詔開封府屬官參軍待並依舊員額。大觀元年,要孝壽乞增置府學博士一員。從之。詔:「開封六職閑劇不同,如士曹之官,唯主到罷批書,而刑、戶事繁,自今凡士之婚田鬥
 訟皆在士曹,餘曹仿此。」二年,詔皇領牧,錄令如執政官,又詔天下州郡並依開封府分曹置掾。政和二年,復置開封府學錢糧官一員。五年,盛章奏:乞依尚書六部置架閣主管官一員。宣和元年,聶山奏:司錄、六曹官乞依省部少監封敘。詔修入條令。



 監安府舊為杭州,領浙西兵馬鈴轄,建炎三年,詔改為臨安府,其守臣令帶浙西同安撫使。時置帥在鎮江府,紹興駐驆安遂正稱安撫使置知府一員、通判二
 員,簽書節度判官廳公事、節度推官、觀察推官、觀察判官、錄事參軍、左司理參軍、右司理參軍、司戶參軍、司法參軍各一員。



 本府掌畿甸之事,籍其戶口,均其賦役,頒其禁令。城外內分南北左右廂,各置廂官,以聽民之訟訴。



 廂官許奏闢京朝官親民資序人充,後以臣僚言,罷城內兩廂官,惟城外置焉。



 分使臣十員,以緝捕在城盜賊。立五酒務,置監官以裕財。分六都監界分,差兵一百四十八鋪以巡防煙火。置兩總轄,承受御前朝旨文字。凡御寶、御批、實封有所取索,則供進;凡
 省,臺、寺、監、監司符牒及管下諸縣及倉場等申到公事,則受而理之;凡大禮及國信,隨事應辦,祠祭共其禮料,會聚陳其幄帟,人使往來,辨其舟楫,皆先期飭於有司。



 領縣九,分士、戶、儀、兵、刑、工六案。內戶案分上、中、下案,外有免役案、常平案、上下開拆司、財賦司、大禮局、國信司、排辦司、修造司,各治其事。置吏:點檢文字、都孔目官、副孔目官、節度孔目官、觀察孔目官各一名,磨勘司主押官、正開拆官、副開拆官各一人,下名開拆官二名,押司
 官八人,前後行守分二十一人,貼司三十人。



 乾道七年,皇太子領尹事,廢臨安府通判、簽判職官。置少尹一員,日受民詞以白太子,間日率僚屬詣宮稟事。置判官二員、推官三員。有旨,少伊比仿知府,判官比通判,推官比幕職官,其統臨職分,並照從來條例。九年,皇太子解尹事,臨安府知、通、簽判、推判官並依舊置。既據保義郎趙禮之狀:「臨安府依條合置兵馬監押一員。經任監當四員,初任監當闕一員,昨皇太子領府尹更不差注,今既
 辭免,乞將宗室添差員闕衣舊。」從之。淳熙三年,詔罷備攝官,惟緝捕使臣十二員、聽候差使六員許令闢置。嘉泰四年,詔臨安府添差不厘務總管路鈐二十員。州鈐轄、路分都監、副都監二十員。正、副將十五員。安撫司淮備將領十五員,州都監以下十員。共以八十員為額。尋減總管路鈴五員。開禧三年,復省罷總管、路分共六員。



 河南應天府牧尹少尹司錄戶曹法曹士曹尹以下掌同開封會,尹闕則置知府事一
 人,以郎中以上充,二品以上曰判府。次府及節度州淮此。



 通判一人,以朝官充。



 判官、推官各一人,或以京朝官簽書。



 使院牙職、左右軍巡悉同開封,而主、典以下差減其數。戶曹通掌府院戶籍、考課、稅賦,法曹專掌讞議,士曹或蔭敘起家,不常置。



 諸州府同。至道初,罷司理院,州置司士,取官吏強慢者為,給簿、尉奉。



 助教有特恩而受者,不厘務。



 次府牧尹少尹司錄戶曹法曹士曹司理文學助教牧、尹以下所掌並同開封,大中祥符八年,以楚王為興元牧,其後又為京兆、江陵牧,自餘無為者。



 尹闕則知府事一人,
 以朝官及刺史以上或諸司使充。



 通判一人,以京朝官充。乾德初,諸州置通判,統治軍、州之政,事得專達,與長吏均禮。大藩或置兩員。戶少事簡有不置者,正刺史以上州知州,雖小處亦特置。



 使院牙職事並同前。



 節度使宋初無所掌,其事務悉歸本州知州、通判兼總之,亦無定員。恩數與執政同。初除,鎖院降麻,其禮尤異,以待宗室近屬、外戚、國婿年勞久次者。若外任,除殿帥始授此官,亦止於一員;或有功勛顯著,任帥守於外,及前宰執拜者,尤不輕授。又遵唐制,以節度使兼中書
 令、或侍中、或中書門下平章事,皆謂之使相,以待勛賢故老及宰相久次罷政者;隨其舊職或檢校官加節度使出判大藩,通謂之使相。元豐以新制,始改為開府儀同三司。舊制,敕出中書門下,故事之大者使相系銜。至是,皆南省奉行,而開府不預。



 八年,鎮江軍節度使、檢校太傅韓絳為開府儀同三司、判大名府。元祐五年,太師、平章軍國重事文彥博為開府儀同三司、守太師、充護國軍山南西道節度使致仕。自崇寧五年司空、左僕射
 蔡京為開府儀同三司、安遠軍節度使、中太一宮使,其後故相而除則有劉正夫、餘深,前執政則有蔡攸、梁子美,外戚則有向宗回、宗良、鄭紳、錢景臻,殿帥則有高俅,內侍則有童貫、梁師成。宣和末,節度使至六十人,議者以為濫。



 親王、皇子二十六人,宗室十一人,前執政二人,大將四人,外戚十人,宦者恩澤計七人。



 中興,諸州升改節鎮凡十有二。是時,諸將勛名有兼兩鎮、三鎮者,實為希闊之典。



 宋朝元臣拜兩鎮節度使者才三人:韓琦、文彥博、中興後呂頤浩是也。三公卒辭之。而諸大將若韓、張、呂、岳、楊劉之流,率至兩鎮節度使,其後加到三鎮者三人:韓世忠
 鎮南、武安、寧國,張俊靜江、寧武、靜海,劉錡護國、寧武、保靜。



 其後相承,宰執從官及后妃之族拜者不一。然自建炎至嘉泰,宰相特拜者六人,呂頤浩、張浚、虞允文皆以勛,史浩以舊,趙雄、葛邲以恩。執政一人,葉右丞夢得。



 從官二人而已。張端明澄、楊敷學倓。



 惟紹興中曹勛、韓公裔,乾道中曾覿,嘉泰中姜特立、譙令譙,皆以攀附恩澤,亦累官至焉,非常制也。



 承宣使無定員,舊名節度觀察留後。政和七年,詔:「觀察留後乃五季藩鎮官以所親信留充後務之稱,不可
 循用,可冠以軍名,改為承宣使。」唐有留後,五代因之,宋初,留後、觀察皆不得本州刺史。大中祥符七年,令有司檢討故事,始復帶之。



 觀察使無定員。初沿唐制置諸州觀察使。凡諸衛將軍及使遙領者,資品並止本官敘,政和中,詔承宣、觀察使仍不帶持節等。



 防禦使團練使諸州刺史無定員。靖康元年,臣僚言:「遙郡、正任恩數遼絕,自遙郡適正任者,合次第轉
 行。今自遙郡與落階官而授正任,直超轉本等正官,是皆奸巧希進躐取。乞應遙郡承宣使有功勞除正任者,止除正任刺史。」從之。凡未落階官者為遙郡,除落階官者為正任。朝謁禦宴,惟正任預焉。遙郡並止本官敘,正任復次第轉行,考之舊制,梯級有差。中興以後,節度移鎮浸少,後有一定不易徑遷太尉;承宣、觀察徑作一官,及遙郡落階官久就除正任。紹興末,臣僚以為言,雖復置檢校官,餘未盡改。



\end{pinyinscope}