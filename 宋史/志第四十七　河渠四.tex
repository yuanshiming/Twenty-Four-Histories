\article{志第四十七 河渠四}

\begin{pinyinscope}

 汴
 河
 下洛河蔡河廣濟河金水河白溝河京畿溝渠白河三白渠鄧許諸渠附



 元豐元年五月,西頭供奉官張從惠復言:「汴口歲開閉,修堤防,通漕才二百餘日。往時數有建議引洛水入汴,
 患黃河嚙廣武山,須鑿山嶺十數丈,以通汴渠,功大不可為。去年七月,黃河暴漲,水落而稍北,距廣武山麓七里,退灘高闊,可鑿為渠,引洛入汴。」範子淵知都水監丞,畫十利以獻。又言:「汜水出玉仙山,索水出嵩渚山,合洛水,積其廣深,得二千一百三十六尺,視今汴流尚贏九百七十四尺。以河、洛湍緩不同,得其贏餘,可以相補。猶慮不足,則旁堤為塘,滲取河水,每百里置木閘一,以限水勢。兩旁溝、湖、陂、濼,皆可引以為助,禁伊、洛上源私引
 水者。大約汴舟重載,入水不過四尺,今深五尺,可濟漕運。起鞏縣神尾山,至土家堤,築大堤四十七里,以捍大河。起沙谷至河陰縣十里店,穿渠五十二里,引洛水屬於汴渠。」疏奏,上重其事,遣使行視。



 二年正月,使還,以為工費浩大,不可為。上復遣入內供奉宋用臣,還奏可為,請「自任村沙谷口至汴口開河五十里,引伊、洛水入汴河,每二十里置束水一,以芻楗為之,以節湍急之勢,取水深一丈,以通漕運。引古索河為源,注房家、黃家、孟家
 三陂及三十六陂,高仰處瀦水為塘,以備洛水不足,則決以入河。又自汜水關北開河五百五十步,屬於黃河,上下置閘啟閉,以通黃、汴二河船筏。即洛河舊口置水□,通黃河,以洩伊、洛暴漲。古索河等暴漲,即以魏樓、滎澤、孔固三斗門洩之。計工九十萬七千有餘。仍乞修護黃河南堤埽,以防侵奪新河」。從之。



 三月庚寅,以用臣都大提舉導洛通汴。四月甲子興工,遣禮官告祭。河道侵民塚墓,給錢徙之,無主者,官為瘞藏。六月戊申,清汴成,
 凡用工四十五日。自任村沙口至河陰縣瓦亭子,並汜水關北通黃河,接運河,長五十一里。兩岸為堤總長一百三里,引洛水入汴。七月甲子,閉汴口,徙官吏、河清卒於新洛口。戊辰,遣禮官致祭。十一月辛未,詔差七千人,赴汴口開修河道。



 三年二月,宋用臣言:「洛水入汴至淮,河道漫闊,多淺澀,乞狹河六十里,為二十一萬六千步。」以四月興役。五月癸亥,罷草屯浮堰。五年三月,宋用臣言:「金水河透水槽阻礙上下汴舟,宜廢撤。」從之。十月,狹
 河畢工。



 六年八月,範子淵又請「於武濟山麓至河岸並嫩灘上修堤及壓埽堤,又新河南岸築新堤,計役兵六千人,二百日成。開展直河,長六十三里,廣一百尺,深一丈,役兵四萬七千有奇,一月成。」從之。十月,都提舉司言:「汴水增漲,京西四斗門不能分減,致開決堤岸。今近京惟孔固斗門可以洩水下入黃河。若孫賈斗門雖可洩入廣濟,然下尾窄狹,不能盡吞。宜於萬勝鎮舊減水河、汴河北岸修立斗門,開淘舊河,創開生河一道,下合入
 刁馬河,役夫一萬三千六百四十三人,一月畢工。」詔從其請,仍作二年開修。七年四月,武濟河潰。八月,詔罷營閉,縱其分流,止護廣武三埽。



 哲宗元祐元年閏二月辛亥,右司諫蘇轍言:「近歲京城外創置水磨,因此汴水淺澀,阻隔官私舟船。其東門外水磨,下流汗漫無歸,浸損民田一二百里,幾敗漢高祖墳。賴陛下仁聖惻怛,親發德音,令執政共議營救。尋詔畿縣於黃河春夫外,更調夫四萬,開自盟河,以疏洩水患,計一月畢工。然以水磨
 供給京城內外食茶等,其水止得五日閉斷,以此工役重大,民間每夫日顧二百錢,一月之費,計二百四十萬貫。而汴水渾濁,易至填淤,明年又須開淘,民間歲歲不免此費。聞水磨歲入不過四十萬貫,前戶部侍郎李定以此課利,惑誤朝聽,依舊存留。且水磨興置未久,自前未有此錢,國計何闕?而小人淺陋,妄有靳惜,傷民辱國,不以為愧。況今水患近在國門,而恬不為怪,甚非陛下勤恤民物之意。而又減耗汴水,行船不便。乞廢罷官磨,
 任民磨茶。」



 三月,轍又乞「令汴口以東州縣,各具水匱所占頃畝,每歲有無除放二稅,仍具水匱可與不可廢罷,如決不可廢,當如何給還民田,以免怨望。」八月辛亥,轍又言:「昨朝旨令都水監差官,具括中牟、管城等縣水匱,元浸壓者幾何,見今積水所占幾何,退出頃畝幾何。凡退出之地,皆還本主。水占者,以官地還之;無田可還,即給元直。聖恩深厚,棄利與民,所存甚遠。然臣聞水所占地,至今無可對還,而退出之田,亦以迫近水匱,為雨水
 浸淫,未得耕鑿。知鄭州岑象求近奏稱:『自宋用臣興置水匱以來,元未曾取以灌注,清汴水流自足,不廢漕運。』乞盡廢水匱,以便失業之民。」十月,遂罷水匱。



 四年冬,御史中丞梁燾言:



 嘗求世務之急,得導洛通汴之實,始聞其說則可喜,及考其事則可懼。竊以廣武山之北,即大河故道,河常往來其間,夏秋漲溢,每抵山下。舊來洛水至此,流入於河。後欲導以趨汴渠,乃乘河未漲,就嫩灘之上,峻起東西堤,闢大河於堤北,攘其地以引洛水,中
 間缺為斗門,名通舟楫,其實盜河以助洛之淺涸也。洛水本清,而今汴常黃流,是洛不足以行汴,而所以能行者,附大河之餘波也。增廣武三埽之備,竭京西所有,不足以為支費,其失無慮數百萬計。從來上下習為欺罔,朝廷惑於安流之說,稅屋之利,恬不為慮。而不知新沙疏弱,力不能制悍河,水勢一薄,則爛熳潰散,將使怒流循洛而下,直冒京師。是甘以數百萬日增之費,養異時萬一之患,亦已誤矣。夫歲傾重費以坐待其患,何若折
 其奔沖,以終除其害哉。



 為今之計,宜復為汴口,仍引大河一支,啟閉以時,還祖宗百年以來潤國養民之賜,誠為得策。汴口復成:則免廣武傾注,以長為京師之安;省數百萬之費,以紓京西生靈之困;牽大河水勢,以解河北決溢之災;便東南漕運,以蠲重載留滯之弊;時節啟閉,以除蹙凌打凌之苦;通江、淮八路商賈大船,以供京師之饒。為甚大之利者六,此不可忽也。惟拆去兩岸舍屋,盡廢僦錢,為害者一而甚小,所謂損小費以成大利
 也。臣之所言,特其大略爾。至於考究本末,措置纖悉,在朝廷擇通習之臣付之,無牽浮議,責其成功。又言:



 臣聞開汴之時,大河曠歲不決,蓋汴口析其三分之水,河流常行七分也。自導洛而後,頻年屢決,雖洛口竊取其水,率不過一分上下,是河流常九分也。猶幸流勢臥北,故潰溢北出。自去歲以來,稍稍臥南,此其可憂,而洛口之作,理須早計。竊以開洛之役,其功甚小,不比大河之上,但闢百餘步,即可以通水三分,即永為京師之福,又減
 河北屢決之害;兼水勢既已牽動,在於回河尤為順便,非獨孫村之功可成,澶州故道,亦有自然可復之理。望出臣前章,面詔大臣與本監及知水事者,按地形水勢,具圖以聞。



 不報。至五年十月癸巳,乃詔導河水入汴。



 紹聖元年,帝親政,復召宋用臣赴闕。七月辛丑,廣武埽危急。壬寅,帝語輔臣:「埽去洛河不遠,須防漲溢下灌京師。」明日,乃詔都水監丞馮忱之相度築欄水簽堤。丁巳,帝諭執政曰:「河埽久不修,昨日報洛水又大溢,注于河,若
 廣武埽壞,河、洛為一,則清汴不通矣,京都漕運殊可憂。宜亟命吳安持、王宗望同力督作,茍得不壞,過此須圖久計。」丙寅,吳安持言:「廣武第一埽危急,決口與清汴絕近,緣洛河之南,去廣武山千餘步,地形稍高。自鞏縣東七里店至今洛口不滿十里,可以別開新河,導洛水近南行流,地里至少,用功甚微。」詔安持等再按視之。



 十一月,李偉言:「清汴導溫洛貫京都,下通淮、泗,為萬世利。自元祐以來屢危急,而今歲特甚。臣相視武濟山以下二
 十里名神尾山,乃廣武埽首所起,約置刺堰三里餘,就武濟河下尾廢堤、枯河基址增修疏導,回截河勢東北行,留舊埽作遙堤,可以紓清汴下注京城之患。」詔宋用臣、陳祐甫覆按以聞。



 十二月甲午,戶部尚書蔡京言:「本部歲計,皆藉東南漕運。今年上供物,至者十無二三,而汴口已閉。臣責問提舉汴河堤岸司楊琰,乃稱自元豐二年至元祐初,八年之間,未嘗塞也。」詔依元豐條例。明年正月庚戌,用臣亦言:「元豐間,四月導洛通汴,六月放
 水,四時行流不絕。遇冬有凍,即督沿河官吏,伐冰通流。自元祐二年,冬深輒閉塞,致河流涸竭,殊失開導清汴本意。今欲卜日伐冰,放水歸河,永不閉塞。及凍解,止將京西五斗門減放,以節水勢,如惠民河行流,自無壅遏之患。」從之。



 三年正月戊申,詔提舉河北西路常平李仲罷歸吏部。仲在元祐中提舉汜水輦運,建言:「西京、鞏縣、河陽、汜水、河陰縣界,乃沿黃河地分,北有太行、南有廣武二山,自古河流兩山之間,乃緣禹跡。昨自宋用臣創
 置導洛清汴,於黃河沙灘上,節次創置廣、雄武等堤埽,到今十餘年間,屢經危急。況諸埽在京城之上,若不別為之計,患起不測,思之寒心。今如棄去諸埽,開展河道,講究興復元豐二年以前防河事,不惟省歲費、寬民力,河流且無壅遏決溢之患。望遣諳河事官相視施行。」又乞復置汴口,依舊以黃河水為節約之限,罷去清汴閘口。



 四年閏二月,楊琰乞依元豐例,減放洛水入京西界大白龍坑及三十六陂,充水匱以助汴河行運。詔賈種
 民同琰相度合占頃畝,及所用功力以聞。五月乙亥,都提舉汴河堤岸賈種民言:「元豐改汴口為洛口,名汴河為清汴者,凡以取水於洛也。復匱清水,以備淺澀而助行流。元祐間,卻於黃河撥口,分引渾水,令自□上流入洛口,比之清洛,難以調節。乞依元豐已修狹河身丈尺深淺,檢計物力,以復清汴,立限修浚,通放洛水。及依舊置洛斗門,通放西河官私舟船。」從之。帝嘗謂知樞密院事曾布曰:「先帝作清汴,又為天源河,蓋有深意。元祐中,
 幾廢。近賈種民奏:『若盡復清汴,不用濁流,乃當世靈長之慶。」布對曰:「先帝以天源河為國姓福地,此眾人所知,何可廢也。」十二月,詔:「京城內汴河兩岸,各留堤面丈有五尺,禁公私侵牟。」



 元符三年,徽宗即位,無大改作,汴渠稍湮則浚之。大觀中,言者論:「胡師文昨為發運使,創開泗州直河,及築簽堤阻遏汴水,尋復淤澱,遂行廢拆。然後並役數郡兵夫,其間疾苦竄歿,無慮數千,費錢穀累百萬計。狂妄生事,誣奏罔功,官員冒賞至四十五人。」師
 文由是自知州降充宮觀。



 宣和元年五月,都城無故大水,浸城外官寺、民居,遂破汴堤,汴渠將溢,諸門皆城守。起居郎李綱奏:「國家都汴,百有六十餘載,未嘗少有變故。今事起倉猝,遐邇驚駭,誠大異也。臣嘗躬詣郊外,竊見積水之來,自都城以西,漫為巨浸。東拒汴堤,停蓄深廣,湍悍浚激,東南而流,其勢未艾。然或淹浸旬時,因以風雨,不可不慮。夫變不虛發,必有感召之因。願詔廷臣各具所見,擇其可採者施行之。」詔:「都城外積水,緣有司
 失職,堤防不修,非災異也。」罷綱送吏部,而募人決水下流,由城北注五丈河,下通梁山濼,乃已。



 七月壬子,都提舉司言:「近因野水沖蕩沿汴堤岸,及河道淤淺,若止役河清,功力不勝,望俟農隙顧夫開修。」從之。五年十二月庚寅,詔:「沿汴州縣創添欄河鎖柵歲額,公私不以為便,其遵元豐舊制。」



 靖康而後,汴河上流為盜所決者數處,決口有至百步者,塞久不合,乾涸月餘,綱運不通,南京及京師皆乏糧。責都水使者措置,凡二十餘日而水復
 舊,綱運沓來,兩京糧始足。又擇使臣八員為沿汴巡檢,每兩員各將兵五百人,自洛口至西水門,分地防察決溢云。



 洛水貫西京,多暴漲,漂壞橋梁。建隆二年,留守向拱重修天津橋成。甃巨石為腳,高數丈,銳其前以疏水勢,石縱縫以鐵鼓絡之,其制甚固。四月,具圖來上,降詔褒美。開寶九年,郊祀西京,詔發卒五千,自洛城菜市橋鑿渠抵漕口三十五里,饋運便之。其後導以通汴。



 蔡河貫京師,為都人所仰,兼閔水、洧水、水異水以通舟。閔水自尉氏歷祥符、開封合於蔡,是為惠民河。洧水自許田注鄢陵東南,歷扶溝合於蔡。水異水出鄭之大隗山,注臨穎,歷鄢陵、扶溝合於蔡。凡許、鄭諸水合堅白雁、丈八溝,京、索合西河、褚河、湖河、雙河、欒霸河皆會焉。猶以其淺涸,故植木橫棧;棧為水之節,啟閉以時。



 太祖建隆元年四月,命中使浚蔡河,設斗門節水,自京距通許鎮。二年,詔發畿甸、陳、許丁夫數萬浚蔡水,南入穎川。乾德二
 年二月,令陳承昭率丁夫數千鑿渠,自長社引潩水至京師,合閔水。渠成,潩水本出密縣大隗山,歷許田。會春夏霖雨,則泛溢民田。至是渠成,無水患,閔河益通漕焉。



 太宗淳化二年,以水異水泛溢,浸許州民田,詔自長葛縣開小河,導水異水,分流二十里,合於惠民河。



 真宗咸平五年七月,京師霖雨,溝洫壅,惠民河溢,泛道路,壞廬舍,知開封府寇準治丁岡古河洩導之。大中祥符元年六月,開封府言:「尉氏縣惠民河決。」遣使督視完塞。二年四月,陳州
 言:「州地洿下,苦積潦,歲有水患,請自許州長葛縣浚減水河及補棗村舊河,以入蔡河。」從之。九年,知許州石普請於大流堰穿渠,置二斗門,引沙河以漕京師。遣使按視。四月,詔遣中使至惠民河,規畫置壩子,以通舟運。



 仁宗天聖二年二月,崇儀副使、巡護惠民河田承說獻議:重修許州合流鎮大流堰斗門,創開減水河通漕,省迂路五百里。詔遣使按視以聞。五年八月,都大巡護惠民河王克基言:「先準宣惠民、京、索河水淺小,緣出源西京、
 鄭、許州界,惠民河下合橫溝、白雁溝、京、索河下合西河、湖河、雙河、欒霸河、丈八溝,各為民間裁水蒔稻灌園,宜令州縣巡察。」七年,王克基言:「按舊制,蔡河斗門棧板須依時啟閉,調停水勢。」嘉祐三年正月,開京城西葛家岡新河,以有司言:「至和中,大水入京城,請自祥符縣界葛家岡開生河,直城南好草陂,北入惠民河,分注魯溝,以紓京城之患。」



 神宗熙寧四年七月,程昉請開宋家等堤,畎水以助漕運。八月,三班借職楊琰請增置上下壩閘,
 蓄水備淺涸。詔琰掌其事。六年九月戊辰,將作監尚宗儒言:「議者請置蔡河木岸,計功頗大。」詔修固土岸。八年,詔京西運米於河北,於是侯叔獻請因丁字河故道鑿堤置閘,引汴水入於蔡,以通舟運。河成,舟不可行,尋廢。十月,詔都水監展惠民河,欲便修城也。九年七月,提轄修京城所請引霧澤陂水至咸豐門,合京、索河,由京、索簽入副堤河,下合惠民。都水監謂:「不若於順天門外簽直河身,及於染院後簽入護龍河,至咸豐門南復入
 京、索河,實為長利。」從之。



 徽宗崇寧元年二月,都水監言:惠民河修簽河次下硬堰畢工。詔立捕獲盜洩賞。大觀元年十二月,開水異河入蔡河,從京畿都轉運使吳擇仁之請也。政和元年十月己酉,詔差水官同京畿監司視蔡河堤防及淤淺者,來春並工治之。



 廣濟河導菏水,自開封歷陳留、曹、濟、鄆,其廣五丈,歲漕上供米六十二萬石。



 太祖建隆二年正月,遣使往定陶規度,發曹、單丁夫數萬浚之。三月,幸新水門觀放水入
 河。先是,五丈河泥淤,不利行舟。遂詔左監門衛將軍陳承昭於京城之西,夾汴水造斗門,引京、索、蔡河水通城濠入斗門,俾架流汴水之上,東進於五丈河,以便東北漕運。公私咸利。三年正月,遣右龍武統軍陳承昭護修五丈河役,車駕臨視,賜承昭錢二十萬。乾德三年,京師引五丈河造西水磑。



 太宗太平興國三年正月,命發近縣丁夫浚廣濟河。



 真宗景德二年六月,開封府言:「京西沿汴萬勝鎮,先置斗門,以減河水,今汴河分注濁水入
 廣濟河,堙塞不利。」帝曰:「此斗門本李繼源所造,屢詢利害,以為始因京、索河遇雨即泛流入汴,遂置斗門,以便通洩。若遽壅塞,復慮決溢。」因令多用巨石,高置斗門,水雖甚大,而餘波亦可減去。三年,內侍趙守倫建議:自京東分廣濟河由定陶至徐州入清河,以達江、湖漕路。役既成,遣使覆視,繪圖來上。帝以地有隆阜,而水勢極淺,雖置堰埭,又歷呂梁灘磧之險,非可漕運,罷之。



 仁宗天聖六年七月,尚書駕部員外郎閻貽慶言:「五丈河下接
 濟州之合蔡鎮,通利梁山濼。近者天河決蕩,溺民田,壞道路,合蔡而下,漫散不通舟,請治五丈河入夾黃河。」因詔貽慶與水官李守忠規度,計功料以聞。



 神宗熙寧七年,趙濟言:「河淺廢運,自此物賤傷農,宜議興復,以便公私。」詔張士澄、楊琰修治。八月,都提舉汴河堤岸司言:「欲於通津門汴河岸東城裏三十步內開河,下通廣濟,以便行運。」從之。八年,又遣琰同陳祐甫因汴河置滲水塘,又自孫賈斗門置虛堤八,滲水入西賈陂,由減水河注
 霧澤陂,皆為河之上源。九年,詔依元額漕粟京東,仍修壩閘,為啟閉之節。九年三月,詔遣官修廣濟河壩閘。元豐五年三月癸亥,罷廣濟輦運司,移上供物自淮陽軍界入汴,以清河輦運司為名,命張士澄都大提舉。七月,御史王植言:「廣濟安流而上,與清河溯流入汴,遠近險易較然,廢之非是。」詔監司詳議。七年八月,都大提舉汴河堤岸司言:「京東地富,穀粟可漕,獨患河澀。若因修京城,令役兵近汴穴土,使之成渠,就引河水注之廣濟,則
 漕舟可通,是一舉而兩利也。」從之。



 哲宗元祐元年,詔斥祥符霧澤陂募民承佃,增置水匱。又即宣澤門外仍舊引京、索源河,置槽架水,流入咸豐門。皆以為廣濟淺澀之備。三月,三省言:「廣濟河輦運,近因言者廢罷,改置清河輦運,迂遠不便。」詔知棣州王諤措置興復。都水監亦言:「廣濟河以京、索河為源,轉漕京東歲計。今欲依舊,即令於宣澤門外置槽架水,流入咸豐門裡,由舊河道復廣濟河源,以通漕運。」從之。



 金水河一名天源,本京水,導自滎陽黃堆山,其源曰祝龍泉。



 太祖建隆二年春,命左領軍衛上將軍陳承昭率水工鑿渠,引水過中牟,名曰金水河,凡百餘里,抵都城西,架其水橫絕於汴,設斗門,入浚溝,通城濠,東匯於五丈河。公私利焉。乾德三年,又引貫皇城,歷後苑,內庭池沼,水皆至焉。開寶九年,帝步自左掖,按地勢,命水工引金水由承天門鑿渠,為大輪激之,南注晉王第。真宗大中祥符二年九月,詔供備庫使謝德權決金水,自天波
 門並皇城至乾元門,歷天街東轉,繚太廟入後廟,皆甃以礱甓,植以芳木,車馬所經,又累石為間梁。作方井,官寺、民舍皆得汲用。復引東,由城下水竇入於濠。京師便之。



 神宗元豐五年,金水河透水槽阻礙上下汴舟,遣宋用臣按視。請自板橋別為一河,引水北入於汴,後卒不行,乃由副堤河入於蔡。以源流深遠,與永安青龍河相合,故賜名曰天源。先是,舟至啟槽,頗滯舟行。既導洛通汴,遂自城西超字坊引洛水,由咸豐門立堤,凡三千三
 十步,水遂入禁中,而槽廢。然舊惟供灑掃,至徽宗政和間,容佐請於七里河開月河一道,分減此水,灌溉內中花竹。命宋升措置導引,四年十一月,畢工。重和元年六月,復命藍從熙、孟揆等增堤岸,置橋、槽、壩、閘,浚澄水,道水入內。內庭池□既多,患水不給,又於西南水磨引索河一派,架以石渠絕汴,南北築堤,導入天源河以助之。



 白溝無山源,每歲水潦甚則通流,才勝百斛船,逾月不雨即竭。



 至道二年三月,內殿崇班閻光澤、國子博士邢
 用之上言:「請開白溝,自京師抵彭城呂梁口,凡六百里,以通長淮之漕。」詔發諸州丁夫數萬治之,以光澤護其役。議者非之。會宋州通判王矩上表,極陳其不可,且言:「用之田園在襄邑,歲苦水潦,私幸渠成。」遂罷其役。咸平六年,用之為度支員外郎,又令自襄邑下流治白溝河,導京師積水,而民田無害。



 神宗熙寧六年,都水監丞侯叔獻請儲三十六陂及京、索二水為源,仿真、楚州開平河置閘,則四時可行舟,因廢汴渠。帝曰:「白溝功料易耳,
 第汴渠歲運甚廣,河北、陜西資焉。又京畿公私所用良材,皆自汴口而至,何可遽廢?」王安石曰:「此役茍成,亦無窮之利也。當別為漕河,引黃河一支,乃為經久。」馮京曰:「若白溝成,與汴、蔡皆通漕,為利誠大,恐汴終不可廢。」帝然之,詔劉璯同叔獻覆視。八月,都水監言:「白溝自濉河至於淮八百里,乞分三年興修。其廢汴河,俟白溝畢功,別相視。仍請發穀熟淤田司並京東汴河所隸河清兵赴役。」從之。七年正月,都水監言:「自盟河畎導汴南諸水,
 近者失於疏浚,為害甚大。」於是輟夫修治,而白溝之役廢。



 初,王安石欲罷白溝、修汴南水利,帝曰:「人多以白溝不可為,而卿獨見可為?」安石曰:「果不可為,罷之誠宜;若可為,即俟時為之,何必計校人言也。」



 徽宗政和二年十月,都水監丞孟昌齡言開浚含暉門外白溝河,開堰放水,仍舊通流。



 京畿溝洫:汴都地廣平,賴溝渠以行水潦。真宗景德二年五月,詔開京城濠以通舟楫,毀官水磑三所。三年,分
 遣入內內侍八人,督京城內外坊里開浚溝渠。先是,京都每歲春浚溝讀,而勢家豪族,有不即施工者。帝聞之,遣使分視,自是不復有稽遲者,以至雨潦暴集,無所雍遏,都人賴之。大中祥符三年,遣供備庫使謝德權治溝洫,導太一宮積水抵陳留界,入亳州渦河。五年三月,帝宣示宰臣曰:「京師所開溝渠,雖屢鈐轄,仍令內侍分察吏擾。」



 仁宗天聖元年八月,東西八作司與內殿承制、閣門祗候劉永崇等言:「內外八廂創置八字水口,通流兩
 水入渠甚利,慮所置處豪富及勢要阻抑,乞下令巡察。」從之。二年七月,內殿崇班、閣門祗候張君平等言:「準敕按視開封府界至南京、宿、亳諸州溝河形勢,疏決利害凡八事:一、商度地形,高下連屬,開治水勢,依尋古溝洫浚之,州縣計力役均定,置籍以主之。二、施工開治後,按視不如元計狀及水壅不行、有害民田者,按官吏之罪,令償其費。三、約束官吏,毋斂取夫眾財貨入己。四、縣令佐、州守倅,有能勸課部民自用工開治不致水害者,敘
 為勞績,替日與家便官;功績尤多,別議旌賞。五、民或於古河渠中修築堰堨,截水取魚,漸至澱淤,水潦暴集,河流不通,則致深害,乞嚴禁之。六、開治工畢,按行新舊廣深丈尺,以校工力。以所出土,於溝河岸一步外築為堤埒。七、凡溝洫上廣一丈,則底廣八尺,其深四尺,地形高處或至五六尺,以此為率。有廣狹不等處,折計之,則畢工之日,易於覆視。八、古溝洫在民田中,久已淤平,今為賦籍而須開治者,據所占地步,為除其賦。」詔令頒行。



 神
 宗熙寧元年三月,都水監言:「畿內溝河至多,而諸縣各役人夫開淘,十才二三,須二三年方可畢工。請令府界提點司選官,與縣官同定緊慢功料,據合差夫數,以五分夫,役十分工,依年分開淘,提點司通行點校。」從之。二年閏十一月,詔以府界道路積水,妨民輸納,命都水監差官溝畎。元豐五年,詔開在京城濠,闊五十步,深一丈五尺,地脈不及者,至泉止。



 徽宗大觀元年七月,以京城霖雨,水浸居民,道路不通,遣官分督疏導。是月又詔:「自
 京至八角鎮,積水妨行旅。轉運司選官疏導,修治橋梁,毋使病涉。」



 白河在唐州,南流入漢。太平興國三年正月,西京轉運使程能獻議,請自南陽下向口置堰,回水入石塘、沙河,合蔡河達於京師,以通湘潭之漕。詔發唐、鄧、汝、穎、許、蔡、陳、鄭丁夫及諸州兵,凡數萬人,以弓箭庫使王文寶、六宅使李繼隆、內作坊副使李神祐、劉承珪等護其役。塹山堙谷,歷博望、羅渠、少柘山,凡百餘里,月餘,抵方城,地
 勢高,水不能至。能獻復多役人以致水,然不可通漕運。會山水暴漲,石堰壞,河不克就,卒堙廢焉。



 端拱元年,供奉官閣門祗候閻文遜、苗忠俱上言:「開荊南城東漕河,至師子口入漢江,可通荊、峽漕路至襄州;又開古白河,可通襄、漢漕路至京。」詔八作使石全振往視之,遂發丁夫治荊南漕河至漢江,可勝二百斛重載,行旅者頗便,而古白河終不可開。



 三白渠在京兆涇陽縣。淳化二年秋,縣民杜思淵上書
 言:「涇河內舊有石翣以堰水入白渠,溉雍、耀田,歲收三萬斛。其後多歷年所,石翣壞,三白渠水少,溉田不足,民頗艱食。乾德中,節度判官施繼業率民用梢穰、笆籬、棧木,截河為堰,壅水入渠。緣渠之民,頗獲其利。然凡遇暑雨,山水暴至,則堰輒壞。至秋治堰,所用復取於民,民煩數役,終不能固。乞依古制,調丁夫修疊石翣,可得數十年不撓。所謂暫勞永逸矣。」詔從之,遣將作監丞周約己等董其役,以用功尤大,不能就而止。



 至道元年正月,度
 支判官梁鼎、陳堯叟上《鄭白渠利害》:「按舊史,鄭渠元引涇水,自仲山西抵瓠口,並北山東注洛,三百餘里,溉田四萬頃,畝收一鐘。白渠亦引涇水,起谷口,入櫟陽,注渭水,長二百餘里,溉田四千五百頃。兩渠溉田凡四萬四千五百頃,今所存者不及二千頃,皆近代改修渠堰,浸隳舊防,繇是灌溉之利,絕少於古矣。鄭渠難為興工,今請遣使先詣三白渠行視,復修舊跡。」於是詔大理寺丞皇甫選、光祿寺丞何亮乘傳經度。



 選等使還,言:



 周覽鄭
 渠之制,用功最大。並仲山而東,鑿斷岡阜,首尾三百餘里,連亙山足,岸壁頹壞,堙廢已久。度其制置之始,涇河平淺,直入渠口。暨年代浸遠,涇河陡深,水勢漸下,與渠口相懸,水不能至。峻崖之處,渠岸摧毀,荒廢歲久,實難致力。其三白渠溉涇陽、櫟陽、高陵、雲陽、三原、富平六縣田三千八百五十餘頃,此渠衣食之源也,望令增築堤堰,以固護之。舊設節水斗門一百七十有六,皆壞,請悉繕完。渠口舊有六石門,謂之「洪門」,今亦隤圮,若復議興
 置,則其功甚大,且欲就近度其岸勢,別開渠口,以通水道。歲令渠官行視,岸之缺薄,水之淤填,實時浚治。嚴豪民盜水之禁。



 涇河中舊有石堰,修廣皆百步,捍水雄壯,謂之「將軍翣」,廢壞已久。杜思淵嘗請興修,而功不克就。其後止造木堰,凡用梢樁萬一千三百餘數,歲出於緣渠之民。涉夏水潦,木堰遽壞,漂流散失,至秋,復率民以葺之,數斂重困,無有止息。欲令自今溉田既畢,命水工拆堰木置於岸側,可充二三歲修堰之用。所役緣渠之
 民,計田出丁,凡調萬三千人。疏渠造堰,各獲其利,固不憚其勞也。選能吏司其事,置暑於涇陽縣側,以時行視,往復甚便。



 又言:



 鄧、許、陳、穎、蔡、宿、亳七州之地,有公私閑田凡三百五十一處,合二十二萬餘頃,民力不能盡耕。皆漢、魏以來,召信臣、杜詩、杜預、任峻、司馬宣王、鄧艾等立制墾闢之地。內南陽界鑿山開道,疏通河水,散入唐、鄧、襄三州以溉田。又諸處陂塘防埭,大者長三十里至五十里,闊五丈至八丈,高一丈五尺至二丈。其溝渠,大
 者長五十里至百里,闊三丈至五丈,深一丈至一丈五尺,可行小舟。臣等周行歷覽,若皆增築陂堰,勞費頗甚,欲堤防未壞可興水利者,先耕二萬餘頃,他處漸圖建置。



 時著作佐郎孫冕總監三白渠,詔冕依選等奏行之。後自仲山之南,移治涇陽縣。其七州之田,令選於鄧州募民耕墾,皆免賦入。復令選等舉一人,與鄧州通判同掌其事。選與亮分路按察,未幾而罷。



 景德三年,鹽鐵副使林特、度支副使馬景盛陳關中河渠之利,請遣官行
 鄭、白渠,興修古制。乃詔太常博士尚賓乘傳經度,率丁夫治之。賓言:「鄭渠久廢不可復,今自介公廟回白渠洪口直東南,合舊渠以畎涇河,灌富平、櫟陽、高陵等縣,經久可以不竭。」工既畢而水利饒足,民獲數倍。



\end{pinyinscope}