\article{本紀第十}

\begin{pinyinscope}

 仁宗二



 明道元年春二月癸卯,呂夷簡上《三朝寶訓》。丙午,詔仕廣南者毋過兩任,以防貪黷。庚戌,以張士遜同中書門下平章事、集賢殿大學士。戊午,錄故宰臣孫,並試將作
 監主簿。甲子,詔員外郎以上致仕者,錄其子校書郎,三丞以上齋郎。丁卯,以真宗順容李氏為宸妃,是日妃薨。



 三月戊子,頒《天聖編敕》。戊戌,以江、淮旱,遣使與長吏錄系囚,流以下減一等,杖笞釋之。己亥,除婺、秀州丁身錢。



 夏四月丙午,錄系囚。戊午,知棣州王涉坐冒請官地為職田,配廣南牢城。



 五月癸酉,遣使點檢河北城池器甲,密訪官吏能否。壬午,廢杭、秀二州鹽場。



 秋七月丙申,詔諸路轉運使舉國子監講官。丁酉,王曙罷。太白晝見,彌
 月乃滅。



 八月辛丑,以晏殊為樞密副使。丙午,晏殊參知政事。甲寅,以楊崇勛為樞密副使。辛酉,授唃廝囉為寧遠大將軍、愛州團練使。壬戌,大內火,延八殿。癸亥,移御延福宮。甲子,以呂夷簡為修內使。乙丑,詔群臣直言闕失。丁卯,大赦。



 九月庚寅,重作受命寶。丙申,皇太后出金銀器易左藏緡錢二十萬,以助修內。



 冬十月庚子,黃白氣五貫紫微垣。丁巳,詔漢陽軍發廩粟以振饑民。



 十一月甲戌,以修內成,恭謝天地於天安殿,謁太廟,大赦,改
 元,百官進秩,優賞諸軍。是日還宮。己卯,冬至,率百官賀皇太后於文德殿,御天安殿受朝。壬辰,延州言夏王趙德明卒。癸巳,以德明子元昊為定難軍節度使、西平王。十二月壬寅,以楊崇勛為樞密使。戊午,詔獲劫盜者奏裁,毋擅殺。壬戌,西北有蒼白氣亙天。是歲,京東、淮南、江東饑。



 二年春正月己卯,詔發運使以上供米百萬斛振江、淮饑民,遣使督視。



 二月戊戌,含譽星見東北方。庚子,詔江、
 淮民饑死者,官為之葬祭。乙巳,皇太后服袞衣、儀天冠饗太廟,皇太妃亞獻,皇后終獻。是日,上皇太后尊號曰應元齊聖顯功崇德慈仁保壽皇太后。丁未,祀先農於東郊,躬耕籍田,大赦。百官上尊號曰睿聖文武體天法道仁明孝德皇帝。



 三月庚午,加恩百官。丁亥,祈雨於會靈觀、上清宮、景德開寶寺。庚寅,以皇太后不豫,大赦,除常赦所不原者。乾興以來貶死者復官,謫者內徙。甲午,皇太后崩,遺詔尊皇太妃為皇太后。呂夷簡為山陵使。



 夏四月丙申朔,出大行皇太后遺留物賜近臣。壬寅,追尊宸妃李氏為皇太后,至是帝始知為宸妃所生。甲辰,以大行皇太后山陵五使並兼追尊皇太后園陵使。戊申,聽政於崇政殿西廂。己酉,罷乾元節上壽。壬子,詔臣僚、宗戚、命婦毋得以進獻祈恩澤,及緣親戚通表章。罷創修寺觀。帝始親政,裁抑僥幸,中外大悅。癸丑,召還宋綬、範仲淹。丙辰,內侍江德明等並坐交通請謁黜。己未,呂夷簡、張耆、夏竦、陳堯佐、範雍、趙稹、晏殊皆罷。以張士遜為
 昭文館大學士,李迪同中書門下平章事、集賢殿大學士,王隨參知政事,李諮為樞密副使,王德用簽書樞密院事。壬戌,御紫宸殿,以張士遜為山陵使兼園陵使。癸亥,上大行太后謚曰莊獻明肅,追尊宸妃李氏為皇太后,謚曰莊懿。



 五月戊辰,詔禮部貢舉。癸酉,詔中外勿輒言皇太后垂簾日事。乙亥,罷群牧制置使。丙子,命宰臣張士遜撰《謝太廟》及《躬耕籍田記》。檢討宋祁言,皇太后謁廟非後世法,乃止撰《籍田記》。戊寅,錄系囚。



 六月甲午
 朔,日有食之。壬寅,錄周世宗及高季興、李煜、孟昶、劉繼元、劉鋹後。癸卯,命審刑、大理詳定配隸刑名。戊午,減天下歲貢物。



 秋七月丁丑,詔知耀州富平縣事張龜年增秩再任,以其治行風告天下。戊子,詔以蝗旱,去尊號「睿聖文武」四字,以告天地宗廟,仍令中外直言闕政。



 八月甲午朔,契丹使來吊慰祭奠。壬寅,作奉慈廟。甲辰,詔中外毋避莊獻明肅太后父諱。丁巳,置端明殿學士。



 九月甲戌,幸洪福院,臨莊懿太后梓宮。丙子、壬午,臨如之。



 冬
 十月癸巳朔,太白犯南斗。甲午,禁登州民採金。丁酉,祔葬莊獻明肅皇太后、莊懿皇太后於永定陵。甲辰,詔以兩川歲貢綾錦羅綺紗,以三之二易為絲由絹,供軍須。己酉,祔莊獻明肅太后、莊懿太后神主於奉慈廟。癸丑,下德音,降東、西京囚罪一等,徒以下釋之。緣二太后陵應奉民戶,免租賦科役有差。丙辰,贈周王祐為皇太子。戊午,張士遜、楊崇勛罷,以呂夷簡為門下侍郎、同中書門下平章事、昭文館大學士,王曙為樞密使,王德用為樞
 密副使,宋綬參知政事,蔡齊為樞密副使。



 十一月癸亥朔,薛奎罷。詔增宗室奉。太白犯南斗。乙丑,追冊美人張氏為皇后。甲戌,贈寇準為中書令。十二月丙申,復置提點刑獄。丁酉,詔諸路轉運使、副,歲遍歷所部,仍令州軍具所至月日以聞。甲辰,以京東饑,出內藏絹二十萬代其民歲輸。乙巳,詔修周廟。丁未,詔臺官非中丞、知雜保薦者勿任。戊申,出宮人二百。乙卯,廢皇后郭氏為凈妃、玉京沖妙仙師,居長寧宮。御史中丞孔道輔率諫官、御
 史,大呼殿門請對,詔宰相告以皇后當廢狀。丙辰,出道輔及諫官範仲淹,仍詔臺諫自今毋相率請對。丁巳,詔明年改元。禁邊臣增置堡砦。是歲,畿內、京東西、河北、河東、陜西蝗,淮南、江東、兩川饑,遣使安撫,除民租。注輦國來貢。



 景祐元年春正月甲子,發江、淮漕米振京東饑民。丙寅,詔開封府界諸縣作糜粥以濟饑民,諸災傷州軍亦如之。戊辰,詔三司鑄景祐元寶錢。甲戌,詔執政大臣議兵
 農可更制者以聞。詔募民掘蝗種,給菽米。癸未,詔禮部所試舉人十取其二,進士三舉、諸科五舉嘗經殿試,進士五舉年五十、諸科六舉年六十,及曾經先朝御試者,皆以名聞。甲申,淮南饑,出內藏絹二十萬代其民歲輸。丁亥,置崇政殿說書。庚寅,詔停淮南上供一年。



 二月乙未,罷書判拔萃科。辛丑,詔禮部貢院:諸科舉人七舉者,不限年,並許特奏名。甲辰,權減江、淮漕米二百萬石。戊申,詔麟、府州振蕃、漢饑民。



 三月壬午,免諸路災傷州軍
 今年夏稅。癸未,詔解州畦戶逋鹽蠲其半。是月,賜禮部奏名進士、諸科及第出身七百八十三人。



 夏四月丁酉,開封府判官龐籍言,尚美人遣內侍稱教旨免工人市租。帝為杖內侍,仍詔有司自今宮中傳命,毋得輒受。癸丑,詔置殿中侍御史、監察御史裏行。



 五月辛酉,出布十萬端,易錢糴河北軍儲。丁卯,禁民間織錦刺繡為服飾。西川歲織錦上供亦罷之。癸酉,詔臺諫未曾歷郡守者與郡。壬午,錄系囚。是月,契丹主宗真之母還政於子,出
 居慶陵。



 六月壬辰,交州民六百餘人內附。庚子,免畿內被災民稅之半。己酉,策制舉、武舉人。乙卯,詔州縣官非理科決罪人至死者,並奏聽裁。



 閏月甲子,泗州淮、汴溢。己巳,常州無錫縣大風發屋。乙亥,毀天下無額寺院。壬午,罷造玳瑁、龜筒器。



 秋七月丙申,賜壽州下蔡縣被溺之家錢有差。己亥,樞密使王曙加同平章事。辛丑,詔文武提刑毋得互相薦論。壬子,詔轉運使與長吏,舉所部官專領常平倉粟。甲寅,河決澶州橫隴埽。



 八月庚申,薛
 奎卒。壬戌,有星孛於張、翼。癸亥,王曙卒。甲子,月犯南斗。戊辰,帝不豫。庚午,以王曾為樞密使。辛未,以星變,大赦,避正殿,減常膳,輔臣奏事延和殿閣。壬申,詔凈妃郭氏出居於外,美人尚氏入道,楊氏安置別宅。



 九月壬辰,百官請只日御前殿,如先帝故事,詔可。丁酉,帝康復,御正殿,復常膳。甲辰,詔立皇後曹氏。丙午,熒惑犯南斗。



 冬十月庚申,罷淮南、江、浙、荊湖制置發運使,詔淮南轉運兼發運事。乙亥,作郊廟《景安》、《興安》、《祐安》之曲。



 十一月己丑,
 冊曹氏為皇后。癸丑,作《大安》之曲以饗聖祖。十二月癸酉,賜西平王趙元昊佛經。是歲,南平王李德政獻馴像二,詔還之。開封府、淄州蝗。



 二年春正月癸丑,置邇英、延義二閣,寫《尚書·無逸》篇於屏。



 二月戊午,御延福宮觀大樂。癸亥,舊給事資善堂者皆推恩。戊辰,李迪罷,以王曾為門下侍郎、同中書門下平章事、集賢殿大學士,王隨、李諮知樞密院事,蔡齊、盛度參知政事,王德用、韓億同知樞密院事。



 三月戊申,出
 內庫珠賜三司,以助經費。



 夏四月庚午,詔天下有知樂者,所在薦聞。



 五月甲午,獠寇雷、化州,詔桂、廣會兵討之。丙申,錄系囚。庚子,議太祖、太宗、真宗廟並萬世不遷。南郊升侑上帝,以太祖定配,二宗迭配。丙午,降天下系囚罪一等,杖以下釋之。丁未,廣西言鎮寧州蠻入寇。



 六月丁巳,詔幕職官初任未成考毋薦。乙亥,頒《一司一務及在京敕》。鎮寧蠻請降。



 秋七月戊申,廢西京採柴務,以山林賦民,官取十之一。



 八月壬子朔,詔輕強盜法。甲寅,
 宴紫宸殿,初用樂。甲戌,幸安肅門炮場閱習戰。己卯,置提點銀銅坑冶鑄錢官。



 九月壬寅,按新樂。己酉,作睦親宅。命中丞杜衍等汰三司胥吏。宋綬上《中書總例》。



 冬十月辛亥朔,復置朝集院。癸亥,復群牧制置使。丁卯,詔諸路歲輸緡錢,福建、二廣易以銀,江東以帛。庚午,熒惑犯左執法。



 十一月戊子,廢後郭氏薨。癸巳,朝饗景靈宮。甲午,饗太廟、奉慈廟。乙未,祀天地於圜丘,大赦。錄五代及諸國後。宗室任諸司使以下至殿直者,換西班官。百官上尊
 號曰景祐體天法道欽文聰武聖神孝德皇帝。丁未,加恩百官。十二月壬子,加唃廝囉為保順軍留後。丙子,詔長吏能導民修水利闢荒田者賞之。是歲,以鎮戎軍薦饑,貸弓箭手粟麥六萬石。



 三年春正月壬辰,追復郭氏為皇后。丁酉,葬皇后郭氏。



 二月丙辰,命官較太常鐘律。壬戌,詔兩制、禮官詳定京師士民服用、居室之制。甲子,以廣南兵民苦瘴毒,為置醫藥。丁卯,修陜西三白渠。



 三月癸巳,復商賈以見錢
 算請官茶法。乙未,觀新定鐘律。戊戌,詔兩省、卿監、刺史、閣門以上致仕,給奉如分司官,長吏歲時勞賜之。改維州為威州。夏五月庚辰,購求館閣逸書。丙申,錄系囚。丙戌,天章閣待制範仲淹坐譏刺大臣,落職知饒州。集賢校理餘靖、館閣校勘尹洙、歐陽修並落職補外。詔戒百官越職言事。



 六月壬申,虔、吉州水溢,壞城郭、廬舍,賜被溺家錢有差。



 秋七月丁亥,禁民間私寫編敕、刑書。乙未,置大宗正司。庚子,大雨震電。太平興國寺災。辛丑,降三京
 罪囚一等,徒以下釋之。



 八月己酉,班民間冠服、居室、車馬、器用犯制之禁。乙卯,月犯南斗。



 九月庚辰,幸睦親宅宴宗室。癸巳,熒惑犯南斗。是月,定申心喪解官法。



 冬十月丁未,命章得像等考課諸路提刑。甲寅,作朝集院。



 十一月戊寅,保慶皇太后楊氏崩。辛卯,上保慶太后謚曰莊惠。十二月丙寅,李諮卒。丁卯,王德用知樞密院事,章得像同知樞密院事。是歲,南平王李德政、西南蕃來貢。南丹州莫淮戟內附。



 四年春正月壬午,詔均諸州解額。



 二月己酉,葬莊惠皇太后於永定陵。己未,祔神主於奉慈廟。庚申,德音:降東、西京及靈駕所過州縣囚罪一等,徒以下釋之。乙丑,置赤帝像於宮中祈嗣。



 三月甲戌,置天章閣侍講。戊寅,詔禮部貢舉。



 夏四月乙巳,呂夷簡上《景祐法寶新錄》。甲子,呂夷簡、王曾、宋綬、蔡齊罷,以王隨為門下侍郎、同中書門下平章事、昭文館大學士,陳堯佐同中書門下平章事、集賢殿大學士,盛度知樞密院事,韓億、程琳、石中立參
 知政事,王鬷同知樞密院事。



 五月庚戌,皇子生,錄系囚,降死罪一等,流以下釋之。是日,皇子薨。乙卯,以旱,遣使決三京系囚。丙寅,芝生化成殿楹。



 六月乙亥,杭州江潮壞堤,遣使致祭。戊子,出《神武秘略》賜邊臣。己丑,奉安太祖御容於揚州建隆寺。



 秋七月丁未,詔河東、河北州郡密嚴邊備。戊申,有星數百西南流至壁東,大者其光燭地,黑氣長丈餘,出畢宿下。



 八月甲戌,越州水,賜被溺民家錢有差。甲午,詔三司、轉運司毋借常平錢穀。冬十一
 月癸亥,罷登、萊賣金場。



 十二月甲申,並、代、忻州並言地震,吏民壓死者三萬二千三百六人,傷五千六百人,畜擾死者五萬餘。遣使撫存其民,賜死傷之家錢有差。是歲,滑州民蠶成被,長二丈五尺。唃廝囉、龜茲、沙州來貢。



 寶元元年春正月甲辰,雷。丙辰,以地震及雷發不時,詔轉運使、提舉刑獄按所部官吏。除並、代、忻州壓死民家去年秋糧。



 二月壬申,詔復日御前殿。甲午,安化蠻寇宜、融州。



 三月戊戌朔,王隨、陳堯佐、韓億、石中立罷,以張士
 遜為門下侍郎、同中書門下平章事、昭文館大學士,章得像同中書門下平章事、集賢殿大學士,王鬷、李若穀並參知政事,王博文、陳執中同知樞密院事。己亥,發邵、澧、潭三州駐泊兵討安化州蠻。是月,賜禮部奏名進士、諸科及第出身七百二十四人。



 夏四月癸酉,王博文卒。乙亥,以張觀同知樞密院事。壬辰,除宜、融州夏稅。



 五月乙巳,錄系囚。



 六月戊寅,罷舉童子。己卯,建州大水,壞民廬舍,賜死傷家錢有差,其無主者,官葬祭之。甲申,詔天
 下諸州月上雨雪狀。



 秋七月壬戌,策制舉人。癸亥,策武舉人。



 八月丁卯,復淮南、江、浙、荊湖制置發運使。庚辰,熒惑犯南斗。



 九月戊申,詔應祀事,己受誓戒而失虔恭者,毋以赦原。賜宜、融州討蠻兵緡錢。



 冬十月丙寅,詔戒百官朋黨。



 十一月甲辰,詔廣西鈐轄進兵討安化蠻。乙巳,詔宜、融州民嘗從軍役者,免今夏稅,運糧者免其半。戊申,朝饗景靈宮。己酉,饗太廟及奉慈廟。庚戌,祀天地於圜丘,大赦,改元。百官上尊號曰寶元體天法道欽文聰
 武聖神孝德皇帝。乙卯,復奏舉縣令法。王曾薨。十二月癸亥朔,加恩百官。甲子,京師地震。丙寅,鄜延路言趙元昊反。甲戌,禁邊人與元昊互市。己卯,奉寧軍節度使、知永興軍夏竦兼涇原、秦鳳路安撫使,振武軍節度使、知延州範雍兼鄜延、環慶路安撫使。是歲,達州大水,黎州蠻來貢。



 二年春正月己酉,王隨卒。辛亥,安化蠻平。癸丑,趙元昊表請稱帝、改元。



 三月丁未,鑄皇宋通寶錢。乙卯,閱試衛
 士。戊午,賜陜西緣邊軍士緡錢。



 夏四月癸亥,授唃廝囉二子瞎氈、磨氈角團練使。乙丑,放宮女二百七十人。壬申,免昭州運糧死蠻寇者家徭二年、賦租一年。丁亥,募河東、陜西民入粟實邊。



 五月癸巳,詔近臣舉方略材武之士各二人。己亥,禁皇族及諸命婦、女冠、尼等非時入內。癸卯,命近臣同三司議節省浮費。丙午,遣使體量安撫陜西、河東。己酉,錄系囚。壬子,王德用罷,以夏守贇知樞密院事。



 六月壬戌,詔省浮費,自乘輿服御及宮掖所
 須,宜從簡約,若吏兵祿賜,毋概行裁減。戊辰,詔諸致仕官嘗犯贓者,毋推恩子孫。丁丑,益州火,焚廬舍三千餘區。壬午,削趙元昊官爵,除屬籍。



 秋七月丁巳,詔宗室遇南郊及乾元節恩,許官一子,餘五歲授官。戊午,以夏竦知涇州兼涇原、秦鳳路沿邊經略安撫使、涇原路馬步軍都總管,範雍兼鄜延、環慶路沿邊經略安撫使、鄜延路馬步軍都總管。八月丁卯,以篳篥城唃廝波補本族軍主。甲戌,皇子生。丙子,降三京囚罪一等,徒以下釋之。
 辛巳,命輔臣報祠高禖。



 九月壬寅,詔河北轉運使兼都大制置營田屯田事。乙卯,出內庫銀四萬兩易粟,振益、梓、利、夔路饑民。



 十月庚午,賜麟、府州及川、陜軍士緡錢。甲申,詔兩川饑民出劍門關者勿禁。



 十一月戊子朔,出內庫珠,易緡錢三十萬糴邊儲。丁酉,盛度、程琳罷,出御史中丞孔道輔。壬寅,以王鬷知樞密院事,宋庠參知政事。十二月庚申,詔審刑院、大理寺、刑部毋通賓客。壬申,詔:「御史闕員,朕自擇舉。」是歲,曹、濮、單州蝗。



 康定元年春正月丙辰朔,日有食之。壬戌,賜國子監學田五十頃。是月,元昊寇延州,執鄜延、環慶兩路副都總管劉平、鄜延副都總管石元孫。詔陜西運使明鎬募強壯備邊。



 二月丁亥,以夏守贇為宣徽南院使、陜西馬步軍都總管、經略安撫使。詔潼關設備。辛卯,月、太白俱犯昴。壬辰,夏守贇兼沿邊招討使。出內藏緡錢十萬賜戍邊禁兵之家。知制誥韓琦安撫陜西。白氣如繩貫日。甲午,括畿內、京東西、淮南、陜西馬。丙申,詔諸路轉運使、提
 刑訪知邊事者以聞。丁酉,詔樞密院同宰臣議邊事。辛丑,出內藏緡錢八十萬付陜西市糴軍儲。丙午,德音:釋延州、保安軍流以下罪,寇所攻掠地除今夏稅,戍兵及戰死者賜其家緡錢。是日改元,去尊號「寶元」字,許中外臣庶上封章言事。丁未,詔陜西量民力,蠲所科芻糧。癸丑,降範雍為尚書吏部侍郎、知安州。甲寅,出內庫珠償民馬直。



 三月丙辰,詔大臣條陜西攻守策。癸亥,命韓琦治陜西城池。乙丑,閱虎翼軍習戰。辛未,詔延州錄戰沒軍
 士子孫,月給糧。丙子,大風,晝暝,是夜有黑氣長數丈,見東南。丁丑,罷大宴。詔中外言闕政。戊寅,王鬷、陳執中、張觀罷,以晏殊、宋綬知樞密院事,王貽永同知樞密院事。詔按察官舉才堪將帥者。庚辰,詔參知政事同議邊事。辛巳,德音:降天下囚罪一等,徒以下釋之。賜京師、河北、陜西、河東諸軍緡錢,蠲陜西夏稅十之二,減河東所科粟。



 夏四月丙戌,省陜西沿邊堡砦。癸巳,詔諸戍邊軍月遣內侍存問其家,病致醫藥,死為斂葬之。甲午,遣使籍
 陜西強壯軍。乙未,契丹國母復遣使來賀乾元節。乙巳,增補河北強壯軍。丙午,鄜延路兵馬都監黃德和坐棄軍要斬。丁未,贈劉平、石元孫官,錄其子孫。辛亥,築延州金明栲栳砦。



 五月甲寅朔,詔前殿奏事毋過五班,餘對後殿。命大官賜食。壬戌,張士遜致事,呂夷簡為門下侍郎、同中書門下平章事、昭文館大學士。癸酉,詔夏守贇進屯鄜州。戊寅,以夏竦為陜西馬步軍都總管兼招討使。是月,元昊陷塞門砦,兵馬監押王繼元死之,又陷安
 遠砦。



 六月丙戌,詔假日御崇政殿視事如前殿。丁亥,以夏守贇同知樞密院事。甲午,降三京囚罪一等,徒以下釋之。乙未,南京鴻慶宮神御殿火。壬寅,遣使體量安撫京東、西。甲辰,增置陜西、河北、河東、京東西弓手。



 秋七月乙丑,遣使以討元昊告契丹。庚午,閱諸軍習戰。戊寅,皇子昕為忠正軍節度使,封壽國公。



 八月戊戌,禁以金箔飾佛像。癸卯,遣尚書屯田員外郎劉渙使邈川。戊申,夏守贇罷,以杜衍同知樞密院事。辛亥,詔範仲淹、葛懷敏
 領兵驅逐塞門等砦蕃騎出境,仍募弓箭手,給地居之。



 九月甲寅,滑州河溢。戊午,李若谷罷,以宋綬、晁宗愨參知政事,鄭戩同知樞密院事。戊辰,以晏殊為樞密使,王貽永、杜衍、鄭戩並樞密副使。甲戌,詔使臣、諸班、諸軍有武藝者自陳。辛巳,閱諸軍習戰。是月,元昊寇三川砦,都巡檢楊保吉死之。又圍師子、定川堡,戰士死者五千餘人,遂陷乾溝、乾河、趙福三堡。環慶路兵馬副都總管任福破白豹城。



 冬十月乙未,制銅符、木契、傳信牌。甲辰,錄方
 略士六十一人,授官有差。



 十一月壬戌,有大星流西南,聲如雷者三。十二月癸未,出內藏庫絹一百萬助糴軍儲。詔南京祠大火。丙戌,詔以常平緡錢助糴軍儲。癸卯,宋綬卒。戊申,鑄當十錢權助邊費。



\end{pinyinscope}