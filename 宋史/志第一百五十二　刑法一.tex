\article{志第一百五十二 刑法一}

\begin{pinyinscope}

 夫天有五氣以育萬物,木德以生,金德以殺,亦甚盭矣,而始終之序,相成之道也。先王有刑罰以糾其民,則必溫慈惠和以行之。蓋裁之以義,推之以仁,則震仇殺戮
 之威,非求民之死,所以求其生也。《書》曰:「士制百姓於刑之中,以教祗德。」言刑以弼教,使之畏威遠罪,導以之善爾。唐、虞之治,固不能廢刑也。惟禮以防之,有弗及,則刑以輔之而已。王道陵遲,禮制隳廢,始專任法以罔其民。於是作為刑書,欲民無犯,而亂獄滋豐,由其本末無序,不足相成故也。



 宋興,承五季之亂,太祖、太宗頗用重典,以繩奸慝,歲時躬自折獄慮囚,務底明慎,而以忠厚為本。海同悉平,文教浸盛。士初試官,皆習律令。其君一以
 寬仁為治,故立法之制嚴,而用法之情恕。獄有小疑,覆奏輒得減宥。觀夫重熙累洽之際,天下之民咸樂其生,重於犯法,而致治之盛於乎三代之懿。元豐以來,刑書益繁,已而憸邪並進,刑政紊矣。國既南遷,威柄下逮,州郡之吏亦頗專行,而刑之寬猛系乎其人。然累世猶知以愛民為心,雖其失慈弱,而祖宗之遺意蓋未泯焉。今摭其實,作《刑法志》。



 宋法制因唐律令、格式,而隨時損益,則有《編敕》,一司、一
 路、一州、一縣又別有《敕》。建隆初,詔判大理寺竇儀等上《編敕》四卷,凡一百有六條,詔與新定《刑統》三十卷並頒天下,參酌輕重為詳,世稱平允。太平興國中,增《敕》至十五卷,淳化中倍之。咸平中增至萬八千五百五十有五條,詔給事中柴成務等芟其繁亂,定可為《敕》者二百八十有六條,準律分十二門,總十一卷。又為《儀制令》一卷。當時便其簡易。大中祥符間,又增三十卷,千三百七十四條。又有《農田敕》五卷,與《敕》兼行。



 仁宗嘗問輔臣曰:「或
 謂先朝詔令不可輕改,信然乎?」王曾曰:「此憸人惑上之言也。咸平之所刪,太宗詔令十存一二,去其繁密以便於民,何為不可?」於是詔中外言《敕》得失,命官修定,取《咸平儀制令》及制度約束之在《敕》者五百餘條,悉附《令》後,號曰《附令敕》。天聖七年《編敕》成,合《農田敕》為一書,視《祥符敕》損百有餘條。其麗於法者,大闢之屬十有七,流之屬三十有四,徒之屬百有六,杖之屬二百五十有八,笞之屬七十有六。又配隸之屬六十有三,大闢而下奏聽
 旨者七十有一。凡此,皆在律令外者也。既頒行,因下詔曰:「敕令者,治世之經,而數動搖則眾聽滋惑,何以訓迪天下哉?自今有司毋得輒請刪改。有未便者,中書、樞密院以聞。」然至慶歷,又復刪定,增五百條,別為《總例》一卷。後又修《一司敕》二千三百十有七條,《一路敕》千八百二十有七條,《一州》、《一縣敕》千四百五十有一條。其麗於法者,大闢之屬總三十有一,流之屬總二十有一,徒之屬總百有五,杖之屬總百六十有八,笞之屬總十有二。又
 配隸之屬總八十有一,大闢而下奏聽旨者總六十有四。凡此,又在《編敕》之外者也。



 嘉祐初,因樞密使韓琦言,內外吏兵奉祿無著令,乃命類次為《祿令》。三司以驛料名數,著為《驛令》。琦又言:「自慶歷四年,距嘉祐二年,敕增至四千餘條,前後抵牾。請詔中外,使言《敕》得失,如天聖故事。」七年,書成。總千八百三十四條,視《慶歷敕》,大闢增六十,流增五十,徒增六十有一,杖增七十有三,笞增三十有八。又配隸增三十,大闢而下奏聽旨者增四十有
 六。又別為《續附令敕》三卷。



 神宗以律不足以周事情,凡律所不載者一斷以敕,乃更其目曰敕、令、格、式,而律恆存乎敕外。熙寧初,置局修敕,詔中外言法不便者集議更定,擇其可恆採者賞之。元豐中,始成書二十有六卷,復下二府參訂,然後頒行。帝留意法令,每有司進擬,多所是正。嘗謂:「法出於道,人能體道,則立法足以盡事。」又曰:「禁於已然之謂敕,禁於未然之謂令,設於此以待彼之謂格,使彼效之之謂式。修書者要當識此。」於是凡入
 笞、杖、徒、流、死,自名例以下至斷獄,十有二門,麗刑名輕重者,皆為敕。自品官以下至斷獄三十五門,約束禁止者,皆為令。命官之等十有七,吏、庶人之賞等七十有七,又有倍、全、分、厘之級凡五等,有等級高下者皆為格。表奏、帳籍、關牒、符檄之類凡五卷,有體制模楷者皆為式。



 元祐初,中丞劉摯言:「元豐編修敕令,舊載敕者多移之令,蓋違敕法重,違令罪輕,此足以見神宗仁厚之德。而有司不能推廣,增多條目,離析舊制,因一言一事,輒立
 一法,意苛文晦,不足以該事物之情。行之幾時,蓋已屢變。宜取慶歷、嘉祐以來新舊敕參照,去取刪正,以成一代之典。」右諫議孫覺亦言煩細難以檢用。乃詔摯等刊定。哲宗親政,不專用元祐近例,稍復熙寧、元豐之制。自是用法以後沖前,改更紛然,而刑制紊矣。



 崇寧元年,臣僚言:「有司所守者法,法所不載,然後用例。今引例破法,非理也。」乃令各曹取前後所用例,以類編修,與法妨者去之。尋下詔追復元豐法制,凡元祐條例悉毀之。



 徽宗
 每降御筆手詔,變亂舊章。靖康初,群臣言:「祖宗有一定之法,因事改者,則隨條貼說,有司易於奉行。蔡京當國,欲快己私,請降御筆,出於法令之外,前後抵牾,宜令具錄付編修敕令所,參用國初以來條法,刪修成書。」詔從其請,書不果成。



 高宗播遷,斷例散逸,建炎以前,凡所施行,類出人吏省記。三年四月,始命取嘉祐條法與政和敕令對修而用之。嘉祐法與見行不同者,自官制、役法外,賞格從重,條約從輕。紹興元年,書成,號《紹興敕令格式》,
 而吏胥省記者亦復引用。監察御史劉一止言:「法令具在,吏猶得以為奸,今一切用其所省記,欺蔽何所不至?」十一月,乃詔左右司、敕令所刊定省記之文頒之。時在京通用敕內,有已嘗沖改不該引用之文,因大理正張柄言,亦詔刪削。十年,右僕射秦檜上之。然自檜專政,率用都堂批狀、指揮行事,雜入吏部續降條冊之中,修書者有所畏忌,不敢刪削,至與成法並立。吏部尚書周麟之言:「非天子不議禮,不制度,不考文。」乃詔削去之。



 至乾
 道時,臣僚言:「紹興以來,續降指揮無慮數千,抵牾難以考據。」詔大理寺官詳難,定其可否,類申刑部,以所隸事目分送六部長貳參詳。六年,刑部侍郎汪大猷等上其書,號《乾道敕令格式》,八年,頒之。當是時,法令雖具,然吏一切以例從事,法當然而無例,則事皆泥而不行,甚至隱例以壞法,賄賂既行,乃為具例。



 淳熙初,詔除刑部許用乾道刑名斷例,司勛許用獲盜推賞例,並乾道經置條例事指揮,其餘並不得引例。既而臣僚言:「乾道新書,
 尚多抵牾。」詔戶部尚書蔡洸詳定之,凡刪改九百餘條,號《淳熙敕令格式》。帝復以其書散漫,用法之際,官不暇偏閱,吏因得以容奸,令敕令所分門編類為一書,名曰《淳熙條法事類》,前此法令之所未有也。四年七月,頒之。淳熙末,議者猶以新書尚多遺闕,有司引用,間有便於人情者。復令刑部詳定,迄光宗之世未成。慶元四年,右丞相京鏜始上其書,為百二十卷,號《慶元敕令格式》。



 理宗寶慶初,敕令所言:「自慶元新書之行,今二十九年,前
 指揮殆非一事,或舊法該括未盡,文意未明,須用續降參酌者;或舊法元無,而後因事立為成法者;或已有舊法,而續降不必引用者;或一時權宜,而不可為常法者。條目滋繁,無所遵守,乞考定之。」淳祐二年四月,敕令所上其書,名《淳祐敕令格式》。十一年,又取慶元法與淳祐新書刪潤。其間修改者百四十條,創入者四百條,增入者五十條,刪去者十七條,為四百三十卷。度宗以後,遵而行之,無所更定矣。其餘一司、一路、一州、一縣《敕》,前後
 時有增損,不可勝紀云。



 五季衰亂,禁罔煩密。宋興,削除苛峻,累朝有所更定。法吏浸用儒臣,務存仁恕,凡用法不悖而宜於時者著之。太祖受禪,始定折杖之制。凡流刑四:加役流,脊杖二十,配役三年;流三千里,脊杖二十,二千五百里,脊杖十八,二千里,脊杖十七,並配役一年。凡徒刑五:徒三年,脊杖二十;徒二年半,脊杖十八;二年,脊杖十七;一年半,脊杖十五;一年,脊杖十三。凡杖刑五:杖一百,臀杖二十;九十,臀杖十八;八十,臀杖十七;七十,
 臀杖十五;六十,臀杖十三。凡笞刑五:笞五十,臀杖十下;四十、三十,臀杖八下;二十、十,臀杖七下。常行官杖如周顯德五年制,長三尺五寸,大頭闊不過二寸,厚及小頭徑不得過九分。徒、流、笞通用常行杖,徒罪決而不役。



 先是,藩鎮跋扈,專殺為威,朝廷姑息,率置不問,刑部按覆之職廢矣。建隆三年,令諸州奏大闢案,須刑部詳覆。尋如舊制:大理寺詳斷,而後覆於刑部。凡諸州獄,則錄事參軍與司法掾參斷之。自是,內外折獄蔽罪,皆有官以相覆
 察。又懼刑部、大理寺用法之失,別置審刑院讞之。吏一坐深,或終身不進,由是皆務持平。



 唐建中令:竊盜贓滿三匹者死。武宗時,竊盜贓滿千錢者死。宣宗立,乃罷之。漢乾祐以來,用法益峻,民盜一錢抵極法。周初,深懲其失,復遵建中之制。帝猶以其太重,嘗增為錢三千,陌以八十為限。既而詔曰:「禁民為非,乃設法令,臨下以簡,必務哀矜。竊盜之生,本非巨蠹。近朝立制,重於律文,非愛人之旨也。自今竊盜贓滿五貫足陌者死。」舊法,強盜
 持杖,雖不傷人,皆棄市。又詔但不傷人者,止計贓論。令諸州獲盜,非狀驗明白,未得掠治。其當訊者,先具白長吏,得判,乃訊之。凡有司擅掠囚者,論為私罪。時天下甫定,刑典弛廢,吏不明習律令,牧守又多武人,率意用法。金州防禦使仇超等坐故入死罪,除名流海島,自是人知奉法矣。



 開寶二年五月,帝以暑氣方盛,深念縲系之苦,乃下手詔:「兩京諸州,令長吏督獄掾,五日一檢視,灑掃獄戶,洗滌杻械。貧不能自存者給飲食,病者給醫藥,
 輕系實時決遣,毋淹滯。」自是,每仲夏申敕官吏,歲以為常。帝每親錄囚徒,專事欽恤。凡御史、大理官屬,尤嚴選擇。嘗謂侍御史知雜馮炳曰:「朕每讀《漢書》,見張釋之、於定國治獄,天下無冤民,此所望於卿也。」賜金紫以勉之。八年,廣州言:「前詔竊盜贓至死者奏裁,嶺南遐遠,覆奏稽滯,請不俟報。」帝覽奏,惻然曰:「海隅習俗,貪獷穿窬,固其常也。」因詔:「嶺南民犯竊盜,贓滿五貫至十貫者,決杖、黥面、配役,十貫以上乃死。」



 太宗在御,常躬聽斷,在京獄
 有疑者,多臨決之,每能燭見隱微。太平興國六年,下詔曰:「諸州大獄,長吏不親決,胥吏旁緣為奸,逮捕證佐,滋蔓逾年而獄未具。自今長吏每五日一慮囚,情得者即決之。」復制聽獄之限:大事四十日,中事二十日,小事十日,不他逮捕而易決者,毋過三日。後又定令:「決獄違限,準官書稽程律論,逾四十日則奏裁。事須證逮致稽緩者,所在以其事聞。」然州縣禁系,往往猶以根窮為名,追擾輒至破家。因江西轉運副使張齊賢言,令外縣罪人
 五日一具禁放數白州。州獄別置歷,長吏檢察,三五日一引問疏理,月具奏上。刑部閱其禁多者,命官即往決遣,冤滯則降黜州之官吏。會兩浙運司亦言:「部內州系囚滿獄,長吏輒隱落,妄言獄空,蓋懼朝廷詰其淹滯。」乃詔:「妄奏獄空及隱落囚數,必加深譴,募告者賞之。」



 先是,諸州流罪人皆錮送闕下,所在或寅緣細微,道路非理死者十恆六七。張齊賢又請:「凡罪人至京,擇清強官慮問。若顯負沈屈,致罷官吏。且令只遣正身、家屬俟旨,其
 乾系者免錮送。」乃詔:「諸犯徒、流罪,並配所在牢城,勿復轉送闕下。」



 雍熙元年,令諸州十日一具囚帳及所犯罪名、系禁日數以聞,俾刑部專意糾舉。帝閱諸州所奏獄狀,有系三百人者。乃令門留、寄禁、取保在外並邸店養疾者,咸準禁數,件析以聞。其鞫獄違限及可斷不斷、事小而禁系者,有司駁奏之。開封女子李嘗擊登聞鼓,自言無兒息,身且病,一旦死,家業無所付。詔本府隨所欲裁置之。李無它親,獨有父,有司因系之。李又詣登聞,訴
 父被縶。帝駭曰:「此事豈當禁系,輦轂之下,尚或如此。天下至廣,安得無枉濫乎?朕恨不能親決四方之獄,固不辭勞爾!」即日遣殿中侍御史李範等十四人,分往江南、兩浙、四川、荊湖、嶺南審決刑獄。吏之弛怠者,劾其罪以聞;其臨事明敏、刑獄無滯者,亦以名上。始令諸州十日一慮囚。帝嘗謂宰相曰:「御史臺,閣門之前,四方綱準之地。頗聞臺中鞫獄,御史多不躬親,垂簾雍容,以自尊大。鞫按之任,委在胥吏,求無冤濫,豈可得也?」乃詔御史決
 獄必躬親,毋得專任胥吏。又嘗諭宰臣曰:「每閱大理奏案,節目小未備,移文按覆,動涉數千里外,禁系淹久,甚可憐也。卿等詳酌,非人命所系,即量罪區分,勿須再鞫。」始令諸州笞、杖罪不須證逮者,長吏即決之,勿復付所司。群臣受詔鞫獄,獄既具,騎置來上,有司斷已,復騎置下之州。凡上疑獄,詳覆之而無疑狀,官吏並同違制之坐。其應奏疑案,亦騎置以聞。



 二年,令竊盜滿十貫者,奏裁;七貫,決杖、黥面、隸牢城;五貫,配役三年,三貫,二年,一
 貫,一年。它如舊制。八月,復分遣使臣按巡諸道。帝曰:「朕於獄犴之寄,夙夜焦勞,慮有冤滯耳。」十月,親錄京城系囚,遂至日旰。近臣或諫勞苦過甚,帝曰:「儻惠及無告,使獄訟平允,不致枉橈,朕意深以為適,何勞之有?」因謂宰相曰:「中外臣僚,若皆留心政務,天下安有不治者。古人宰一邑,守一郡,使飛蝗避境,猛虎渡河。況能惠養黎庶,申理冤滯,豈不感召和氣乎?朕每自勤不怠,此志必無改易。或云有司細故,帝王不當親決,朕意則異乎是。若
 以尊極自居,則下情不能上達矣。」自是祁寒盛暑或雨雪稍愆,輒親錄系囚,多所原減。諸道則遣官按決,率以為常,後世遵行不廢,見各帝紀。



 先是,太祝刁衎上疏言「「古者投奸人於四裔,今乃遠方囚人,盡歸象闕,配務役?神京天子所居,豈可使流囚於此聚役。《禮》曰:『刑人於市,與眾棄之。』則知黃屋紫宸之中,非行法用刑之所。望自今外處罪人,勿許解送上京,亦不留於諸務充役。御前不行決罰之刑,殿前引見司鉗黥法具、敕杖,皆以付御
 史、廷尉、京府。或出中使,或命法官,具禮監科,以重明刑謹法之意。」帝覽疏甚悅,降詔褒答,然不能從也。



 三年,始用儒士為司理判官,令諸州訊囚,不須眾官共視,申長吏得判乃訊囚。刑部張佖言:「官吏枉斷死罪者,請稍峻條章,以責其明慎。」始定制:應斷獄失入死刑者,不得以官減贖,檢法官、判官皆削一任,而檢法仍贖銅十斤,長吏則停任。尋置刑部詳覆官六員,專閱天下所上案牘,勿復遣鞫獄吏。置御史臺推勘官二十人,皆以京朝
 官為之。凡諸州有大獄,則乘傳就鞫。陛辭日,帝必臨遣諭之曰:「無滋蔓,無留滯。」咸賜以裝錢。還,必召問所推事狀,著為定令。自是,大理寺杖罪以下須刑部詳覆。又所駁天下案牘未具者,亦令詳覆乃奏。判刑部李昌齡言:「舊制,大理定刑送部,詳覆官入法狀,主判官下斷語,乃具奏。至開寶六年,闕法直官,致兩司共斷定覆詞。今宜令大理所斷案牘,寺官印署送詳覆。得當,則送寺共奏,否即疏駁以聞。」



 淳化初,始置諸路提點刑獄司,凡管內州
 府,十日一報囚帳。有疑獄未決,即馳傳往視之。州縣稽留不決、按讞不實,長吏則劾奏,佐史、小吏許便宜按劾從事。帝又慮大理、刑部吏舞文巧詆,置審刑院於禁中,以樞密直學士李昌齡知院事,兼置詳議官六員。凡獄上奏,先達審刑院,印訖,付大理寺、刑部斷覆以聞。乃下審刑院詳議申覆,裁決訖,以付中書省。當,即下之;其未允者,宰相覆以聞,始命論決。蓋重慎之至也。凡大理寺決天下案牘,大事限二十五日,中事二十日,小事十日。
 審刑院詳覆,大事十五日,中事十日,小事五日。三年,詔御史臺鞫徒以上罪,獄具,令尚書丞郎、兩省給舍以上一人親往慮問。尋又詔:「獄無大小,自中丞以下,皆臨鞫問,不得專責所司。」自端拱以來,諸州司理參軍,皆帝自選擇,民有詣闕稱冤者,亦遣臺使乘傳按鞫,數年之間,刑罰清省矣。既而諸路提點刑獄司未嘗有所平反,詔悉罷之,歸其事轉運司。



 至道二年,帝聞諸州所斷大闢,情可疑者,懼為有司所駁,不敢上其獄。乃詔死事有可
 疑者,具獄申轉運司,擇部內詳練格律者令決之,須奏者乃奏。



 真宗性寬慈,尤慎刑闢。嘗謂宰相曰:「執法之吏,不可輕授。有不稱職者,當責舉主,以懲其濫。」審刑院舉詳議官,就刑部試斷案三十二道,取引用詳明者。審刑院每奏案,令先具事狀,親覽之,翌日,乃候進止,裁處輕重,必當其罪。咸平四年,從黃州守王禹偁之請,諸路置病囚院,徒、流以上有疾者處之,餘責保於外。



 景德元年,詔:「諸道州軍斷獄,內有宣敕不定刑名,止言當行極斷
 者,所在即寘大闢,頗乖平允。自今凡言處斷、重斷、極斷、決配、朝典之類,未得論決,具獄以聞。」



 四年,復置諸路提點刑獄官。先是,帝出筆記六事,其一曰:「勤恤民隱,遴柬庶官,朕無日不念也。所慮四方刑獄官吏,未盡得人,一夫受冤,即召災沴。今軍民事務,雖有轉運使,且地遠無由周知。先帝嘗選朝臣為諸路提點刑獄,今可復置,仍以使臣副之,命中書、樞密院擇官。」又曰:「河北、陜西,地控邊要,尤必得人,須性度平和有執守者。」親選太常博士
 陳綱、李及,自餘擬名以聞,咸引對於長春殿遣之。內出御前印紙為歷,書其績效,代還,議功行賞。如刑獄枉濫不能擿舉,官吏曠弛不能彈奏,務從畏避者,寘以深罪。知審刑院朱巽上言:「官吏因公事受財,證左明白,望論以枉法,其罪至死者,加役流。」從之。御史臺嘗鞫殺人賊,獄具,知雜王隨請臠剮之,帝曰:「五刑自有常制,何為慘毒也。」入內供奉官楊守珍使陜西,督捕盜賊,因請「擒獲強盜至死者,望以付臣凌遲,用戒兇惡」。詔:「捕賊送所屬,
 依法論決,毋用凌遲。」凌遲者,先斷其支體,乃抉其吭,當時之極法也。蓋真宗仁恕,而慘酷之刑,祖宗亦未嘗用。



 初,殿中侍御史趙湘嘗建言:「聖王行法,必順天道。漢制大闢之科,盡冬月乃斷。此古之善政,當舉行之。且十二月為承天節,萬方祝頌之時,而大闢決斷如故。況十一月一陽始出,其氣尚微,議獄緩刑,所以助陽抑陰也。望以十一月、十二月內,天下大闢未結正者,更令詳覆;已結正者,未令決斷。所在厚加矜恤,掃除獄房,供給飲食、
 薪炭之屬,防護無致他故。情可憫者,奏聽敕裁。合依法者,盡冬月乃斷。在京大闢人,既當春孟之月,亦行慶施惠之時。伏望萬幾之暇,臨軒躬覽,情可憫者,特從末減,亦所以布聖澤於無窮。況愚民之抵罪未斷,兩月亦非淹延。若用刑順於陰陽,則四時之氣和,氣和則百穀豐實,水旱不作矣。」帝覽奏,曰:「此誠嘉事。然古今異制,沿革不同,行之慮有淹滯,或因緣為奸矣。」



 天禧四年,乃詔:「天下犯十惡、劫殺、謀殺、故殺、鬥殺、放火、強劫、正枉法贓、偽
 造符印、厭魅咒詛、造妖書妖言、傳授妖術、合造毒藥、禁軍諸軍逃亡為盜罪至死者,每遇十二月,權住區斷,過天慶節即決之。餘犯至死者,十二月及春夏未得區遣,禁錮奏裁。」



 在仁宗時,四方無事,戶口蕃息,而克自抑畏,其於用刑尤慎。即位之初,詔內外官司,聽獄決罪,須躬自閱實,毋枉濫淹滯。刑部嘗薦詳覆官,帝記其姓名,曰:「是嘗失入人罪不得遷官者,烏可任法吏?」舉者皆罰金。



 獄疑者讞,所從來久矣。漢嘗詔「讞而後不當讞者不為
 失」,所以廣聽察、防繆濫也。時奏讞之法廢。初,真宗嘗覽囚簿,見天下斷死罪八百人,憮然動容,語宰執曰:「雜犯死罪條目至多,官吏儻不盡心,豈無枉濫?故事,死罪獄具,三覆奏,蓋甚重慎,何代罷之?」遂命檢討沿革,而有司終慮淹系,不果行。至是,刑部侍郎燕肅奏曰:「唐大闢罪,令尚書、九卿讞之。凡決死刑,京師五覆奏,諸州三覆奏。貞觀四年,斷死罪三十九,開元二十五年,財五十八。今天下生齒未加於唐,而天聖三年,斷大闢二千四百三
 十六,視唐幾至百倍。京師大闢雖一覆奏,而州郡獄疑上請,法寺多所舉駁,率得不應奏之罪,往往增飾事狀,移情就法,失朝廷欽恤之意。望準唐故事,天下死罪皆得覆奏。議者必曰待報淹延。漢律皆以季秋論囚,唐自立春至秋分不決死刑,未聞淹留以害漢、唐之治也。」下其章中書,王曾謂:「天下皆一覆奏,則必死之人,徒充滿狴犴而久不得決。諸獄疑若情可矜者,聽上請。」



 天聖四年,遂下詔曰:「朕念生齒之蕃,抵冒者眾。法有高下,情有
 輕重,而有司巧避微文,一切致之重闢,豈稱朕好生之志哉?其令天下死罪情理可矜及刑名疑慮者,具案以聞。有司毋得舉駁。」其後,雖法不應奏、吏當坐罪者,審刑院貼奏,率以恩釋為例,名曰「貼放」。吏始無所牽制,請讞者多得減死矣。



 先是,天下旬奏獄狀,雖杖、笞皆申覆,而徒、流罪非系獄,乃不以聞。六年,集賢校理聶冠卿請罷覆杖、笞,而徒以上雖不系獄,皆附奏。詔從其說。自定折杖之制,杖之長短廣狹,皆有尺度,而輕重無準,官吏得
 以任情。至是,有司以為言,詔毋過十五兩。



 初,真宗時,以京師刑獄多滯冤,置糾察司,而御史臺獄亦移報之。八年,御史論以為非體,遂詔勿報。祖宗時,重盜剝桑柘之禁,枯者以尺計,積四十二尺為一功,三功以上抵死。殿中丞於大成請得以減死論,下法官議,謂當如舊。帝意欲寬之,詔死者上請。



 刑部分四按,大闢居其一,月覆大闢不下二百數,而詳覆官才一人。明道二年,令四按分覆大闢,有能駁正死罪五人以上,歲滿改官。法直官與
 詳覆官分詳天下旬奏,獄有重闢,獄官毋預燕游迎送。凡上具獄,大理寺詳斷,大事期三十日,小事第減十日。審刑院詳議又各減半。其不待期滿而斷者,謂之「急按」。凡集斷急按,法官與議者並書姓名,議刑有失,則皆坐之。至景祐二年,判大理寺司徒昌運言:「斷獄有期日,而炎□曷之時,系囚淹久,請自四月至六月減期日之半,兩川、廣南、福建、湖南如急按奏。」其後猶以斷獄淹滯,又詔月上斷獄數,列大、中、小事期日,以相參考。



 是歲,改強盜
 法:不持杖,不得財,徒二年;得財為錢萬及傷人者,死。持杖而不得財,流三千里;得財為錢五千者,死;傷人者,殊死。不持杖得財為錢六千,若持杖罪不至死者,仍刺隸二千里外牢城。能告群盜劫殺人者第賞之,及十人者予錢十萬。既而有司言:「竊盜不用威力,得財為錢五千,即刺為兵,反重於強盜,請減之。」遂詔至十千始刺為兵,而京城持杖竊盜,得財為錢四千,亦刺為兵。自是盜法惟京城加重,餘視舊益寬矣。



 慶歷五年,詔罪殊死者,若
 祖父母、父母年八十及篤疾無期親者,列所犯以聞。



 承平日久,天下生齒益蕃,犯法者多,歲斷大闢甚眾,而有司未嘗上其數。嘉祐五年,判刑部李綖言:「一歲之中,死刑無慮二千餘。夫風俗之薄,無甚於骨肉相殘;衣食之窮,莫急於盜賊。今犯法者眾,豈刑罰不足以止奸,而教化未能導其為善歟?願詔刑部類天下所斷大闢,歲上朝廷,以助觀省。」從之。



 凡在京班直諸軍請糧,斗斛不足,出戍之家尤甚。倉吏自以在官無祿,恣為侵漁。神宗謂
 非所以愛養將士之意,於是詔三司始立《諸倉丐取法》。而中書請主典役人,歲增祿至一萬八千九百餘緡。凡丐取不滿百錢,徒一年,每百錢則加一等;千錢流二千里,每千錢則加一等,罪止流三千里。其行貨及過致者,減首罪二等。徒者皆配五百里,其賞百千;流者皆配千里,賞二百千;滿十千,為首者配沙門島,賞三百千,自首則除其罪。凡更定約束十條行之。其後內則政府,外則監司,多仿此法。內外歲增吏祿至百餘萬緡,皆取諸坊
 場,河渡,市利,免行、役剩息錢。久之,議臣欲稍緩倉法,編敕所修立《告捕獲倉法給賞條》,自一百千分等至三百千,而按問者減半給之,中書請依所定,詔仍舊給全賞,雖按問,亦全給。呂嘉問嘗請行貨者宜止以不應為坐之,刑部始減其罪。及哲宗初,嘗罷重祿法,而紹聖復仍舊。



 熙寧四年,立《盜賊重法》。凡劫盜罪當死者,籍其家貲以賞告人,妻子編置千里;遇赦若災傷減等者,配遠惡地。罪當徒、流者,配嶺表;流罪會降者,配三千里,籍其家
 貲之半為賞,妻子遞降等有差。應編配者,雖會赦,不移不釋。凡囊橐之家,劫盜死罪,情重者斬,餘皆配遠惡地,籍其家貲之半為賞。盜罪當徒、流者,配五百里,籍其家貲三之一為賞。竊盜三犯,杖配五百里或鄰州。雖非重法之地,而囊橐重法之人,以重法論。其知縣、捕盜官皆用舉者,或武臣為尉。盜發十人以上,限內捕半不獲,劾罪取旨。若復殺官吏,及累殺三人,焚舍屋百間,或群行州縣之內,劫掠江海船□伐之中,非重地,亦以重論。



 凡重
 法地,嘉祐中始於開封府諸縣,後稍及諸州。以開封府東明、考城、長垣縣,京西滑州,淮南宿州,河北澶州,京東應天府、濮、齊、徐、濟、單、兗、鄆、沂州、淮陽軍,亦立重法,著為令。至元豐時,河北、京東、淮南、福建等路皆用重法,郡縣浸益廣矣。元豐敕,重法地分,劫盜五人以上,兇惡者,方論以重法。紹聖後,有犯即坐,不計人數。復立《妻孥編管法》。至元符三年,因刑部有請,詔改依舊敕。



 先是,曾布建言:「盜情有重輕,贓有多少。今以贓論罪,則劫貧家情雖
 重,而以贓少減免,劫富室情雖輕,而以贓重論死。是盜之生死,系於主之貧富也。至於傷人,情狀亦殊。以手足毆人,偶傷肌體,與夫兵刃湯火,固有間矣,而均謂之傷。朝廷雖許奏裁,而州郡或奏或否,死生之分,特幸與不幸爾。不若一變舊法,凡以贓定罪及傷人情狀不至切害者,皆從罪止之法。其用兵刃湯火,情狀酷毒,及污辱良家,或入州縣鎮砦行劫,若驅虜官吏巡防人等,不以傷與不傷。凡情不可貸者,皆處以死刑,則輕重不失其當
 矣。」及布為相,始從其議,詔有司改法。未幾,侍御史陳次升言:「祖宗仁政,加於天下者甚廣。刑法之重,改而從輕者至多。惟是強盜之法,特加重者,蓋以禁奸宄而惠良民也。近朝廷改法,詔以強盜計贓應絞者,並增一倍;贓滿不傷人,及雖傷人而情輕者奏裁。法行之後,民受其弊,被害之家,以盜無必死之理,不敢告官,而鄰里亦不為之擒捕,恐怨仇報復。故賊益逞,重法地分尤甚。恐養成大寇,以貽國家之患,請復行舊法。」布罷相,翰林學士
 徐績復言其不便,乃詔如舊法,前詔勿行。



 先是,諸路經略、鈐轄,不得便宜斬配百姓。趙抃嘗知成都,乃言當獨許成都四路。王安石執不可,而中書、樞密院同立法許之。其後,謝景初奏:「成都妄以便宜誅釋,多不當。」於是中書復刪定敕文,惟軍士犯罪及邊防機速,許特斷。及抃移成都,又請立法,御史劉孝孫亦為之請依舊便宜從事,安石寢其奏。



 武臣犯贓,經赦敘復後,更立年考升遷。帝曰:「若此,何以戒貪吏?」故命改法。熙寧六年,樞密都承
 旨曾孝寬等定議上之,大概仿文臣敘法而少增損爾。七年,詔:「品官犯罪,按察之官並奏劾聽旨。毋得擅捕系、罷其職奉。」



 元豐二年,成都府、利路鈐轄言:「往時川峽絹匹為錢二千六百,以此估贓,兩鐵錢得比銅錢之一。近絹匹不過千三百,估贓二匹乃得一匹之罪,多不至重法。」令法寺定以一錢半當銅錢之一。



 元祐二年,刑部、大理寺定制:「凡斷讞奏獄,每二十緡以上為大事,十緡以上為中事,不滿十緡為小事。大事以十二日,中事九日,小
 事四日為限。若在京、八路,大事十日,中事五日,小事三日。臺察及刑部舉劾約法狀並十日,三省、樞密院再送各減半。有故量展,不得過五日。凡公案日限,大事以三十五日,中事二十五日,小事十日為限。在京、八路,大事以三十日,中事半之,小事參之一。臺察及刑部並三十日。每十日,斷用七日,議用三日。」



 五年,詔命官犯罪,事干邊防軍政,文臣申尚書省,武臣申樞密院。中丞蘇轍言:「舊制,文臣、吏民斷罪公案歸中書,武臣、軍士歸樞密,而
 斷例輕重,悉不相知。元豐更定官制,斷獄公案並由大理、刑部申尚書省,然後上中書省取旨。自是斷獄輕重比例,始得歸一,天下稱明焉。今復分隸樞密,必有罪同斷異,失元豐本意,請並歸三省。其事干邊防軍政者,令樞密院同進取旨,則事體歸一,而兵政大臣各得其職。」六年,乃詔:「文武官有犯同按乾邊防軍政者,刑部定斷,仍三省、樞密院同取旨。」



 刑部論:「佃客犯主,加凡人一等。主犯之,杖以下勿論,徒以上減凡人一等。謀殺盜詐、有
 所規求避免而犯者不減。因毆致死者不刺面,配鄰州,情重者奏裁。凡命士死於官或去位,其送徒道亡,則部轄將校、節級與首率眾者徒一年,情輕則杖百,雖自首不免。」



 政和間,詔:「品官犯罪,三問不承,即奏請追攝;若情理重害而拒隱,方許枷訊。邇來有司廢法,不原輕重,枷訊與常人無異,將使人有輕吾爵祿之心。可申明條令,以稱欽恤之意。」又詔:「宗子犯罪,庭訓示辱。比有去衣受杖,傷膚敗體,有惻朕懷。其令大宗正司恪守條制,違者
 以違御筆論。」又曰:「其情理重害,別被處分。若罪至徒、流,方許制勘,餘止以眾證為定,仍取伏辨,無得輒加捶考。其合庭訓者,並送大宗正司,以副朕敦睦九族之意。」中書省言:「《律》,『在官犯罪,去官勿論』。蓋為命官立文。其後相因,掌典去官,亦用去官免罪,有犯則解役歸農,幸免重罪。」詔改《政和敕》掌典解役從去官法。



 左道亂法,妖言惑眾,先王之所不赦,至宋尤重其禁。凡傳習妖教,夜聚曉散,與夫殺人祭祀之類,皆著於法,訶察甚嚴。故奸軌不
 逞之民,無以動搖愚俗。間有為之,隨輒報敗,其事不足紀也。



\end{pinyinscope}