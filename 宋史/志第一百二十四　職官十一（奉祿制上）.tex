\article{志第一百二十四 職官十一(奉祿制上)}

\begin{pinyinscope}

 奉祿匹帛職錢祿粟傔人衣糧廚料薪炭諸物



 奉錄自宰臣而下至岳瀆廟令,凡四十一等。



 宰相,樞密使,月三百千。春、冬服各綾二十匹,絹三十匹,冬綿百兩。樞密使帶
 使相,侍中樞密使,春、冬衣同宰相、



 節度使同中書門下平章事已上及帶宣徽使,並前兩府除節度使及節度使移鎮,樞密使、副、知院帶節度使,四百千。



 參知政事,樞密副使,知樞密院事,同知樞密院事,及宣微使不帶節度使,或檢校太保簽書樞密院事,三司使,二百千。



 春、冬各綾十匹,春絹十匹,冬二十匹,綿五十兩。自宰相而下,春各加羅一匹。檢校太保簽書者,春、冬絹二十匹,綿五十兩。



 節度觀察留後知樞密院事及充樞密副使、同知樞密院事,並帶宣徽使簽書樞密院事,三百千,綾、絹、羅、綿同參知政事。



 觀文殿
 大學士,料錢、衣賜隨本官。



 資政殿大學士,料錢、衣賜隨本官。



 翰林學士承旨、學士,龍圖、天章閣直學士,知制誥,龍圖、天章閣學士,綾各五匹,絹十七匹,自承旨而下加羅一匹,綿五十兩。已上奉隨本官,衣賜如本官例,大即依本官例,小即依逐等。



 三師,三公,百二十千絹三十匹。東宮三師,僕射,九十千。綾各五匹,絹二十匹。



 東宮三少,御史大夫,尚書,六十千。門下、中書侍郎,太常、宗正卿,左、右丞,諸行侍郎,御史中丞,五十五千。



 春、冬各綾五匹,絹十七匹,惟中丞綾七匹,絹二十匹。權御史中丞者給本官奉。



 太子賓客,四十五千。綾、絹同中丞。



 左、右散騎常侍,六十千。給事中,中書
 舍人,大卿、監,國子祭酒,太子詹事,四十五千。諫議。四十千。



 春、冬綾各三匹,絹十五匹。舊志:太常宗正卿、左右丞、侍郎充翰林承旨及侍讀、侍講,各綾七匹,絹二十匹;中書舍人若充翰林學士,綾五匹,絹十七匹;他官充龍圖閣學士、樞密直學士,並淮此。龍圖閣學士知制誥,同諫議之數。



 權三司使,並權發遣使公事,料錢、衣賜並同本官。



 副使,五十千。



 春綾二匹,冬綾五匹,春、冬絹各十五匹。自三師以下,春各加羅一匹,冬綿五十兩,權者同。判官並權及發遣,以至子司主判,河渠勾當公事,同管勾河渠公事,料錢、衣賜並同本官數。



 左、右諭德,少卿、監,司業,郎中,三十五千。左、右庶子,起居郎、舍人,侍御史,知雜事同。如正郎知雜,即支本官奉料。



 左、右司諫,殿中侍御史,員外郎,赤
 令,三十千;丞,十五千。如京朝官願請本官衣奉者,仍支米麥。少詹事,二十九千。



 春、冬絹各十三匹,惟赤縣令衣賜隨本官。



 左、右正言,監察御史,太常博士,通事舍人,國子五經博士,太常、宗正、秘書、殿中丞,著作郎,大理正,二十千。太子率更令、中允、贊善中舍、洗馬,殿中省六尚奉御,十八千。



 太常博士以上春、冬絹各十匹,諭德以下春加羅一匹,冬綿三十兩,餘各絹七匹。太常博士、著作、洗馬舊各有增減。



 司天五官正,十三千。春、冬絹各五匹,冬綿十五兩。



 秘書郎,著作佐郎,十七千。



 春冬絹各六匹,冬綿各二十兩。五官正以下春羅各一匹。秘書郎舊無奉,兼三館職事者給八十千;至道二年,令同著作佐郎給之。



 大理奪丞,十四
 千,諸寺、監丞,十二千,春、冬絹各五匹。



 大理評事,十千。春、冬各絹三匹。自大理寺丞以下冬綿各加十五兩。諸寺、監丞,大理評事,舊有增損不同。



 太祝,奉禮,八千,司天監丞,五千,春、冬絹各五匹。主簿,五千,春、冬絹各三匹,丞,簿各綿十五兩。



 靈臺郎,三千,保章正,二千。春、冬絹各三匹,惟靈臺郎冬隨衣錢三千。



 節度使,四百千。



 管軍同。如皇子充節度使兼侍中、帶諸王,皇族節度使同中書門下平章事,並散節度使及帶王爵,奉同節度使。惟春、冬加絹各百匹,大綾各二十匹,小綾各三十匹,羅各十匹,綿各五百兩。



 節度觀察留後,官制行,改承宣使。三百千。



 管軍同。兩省都知押班、諸司使遙領者準此。如皇族充留後及帶郡王同,惟春加絹二十匹,冬三十匹,大小綾各十匹,春羅一匹,冬綿百兩。



 觀察使,二百千。



 管軍同。兩
 省都知押班、諸司使並橫行遙領者,奉準此。春、冬加絹各十匹,綿五十兩。如皇族充觀察者,即三百千,仍春、冬加絹各十五匹,綾十匹,春羅一匹,冬綿五十兩。



 防禦使,三百千。



 管軍、皇族同。其皇族及兩省都知押班、諸司使並橫行、諸衛大將軍將軍遙領者,百五十千,皇族春、冬加絹各十五匹,綾十匹,春羅一匹,綿五十兩。兩省都知押班並橫行,諸衛大將軍領者,春、冬絹各十匹,綿五十兩。



 團練使,百五十千。



 管軍及皇族並軍班除充者同。其皇族及兩省都知押班、諸司使並橫行、諸衛大將軍遙領者,百千。皇族春、冬加絹各十五匹,綾十匹,春羅一匹,綿五十兩。兩省都知押班並橫行、諸衛大將軍將軍領者,春、冬絹各十匹,冬綿五十兩。



 六軍統軍,百千,諸衛上將軍,六十千。春、冬綾各五匹,絹十匹,綿五十兩,如皇子充諸衛上將軍,二百千,春、冬綾各十匹,春絹十匹,羅一匹,冬絹二十匹,綿五十兩。左、右
 金吾衛大將軍,三十五千。諸衛大將軍,二十五千。



 春、冬綾各三匹,絹七匹,冬綿三十兩。



 將軍,二十千。春、冬綾各二匹,絹五匹,綿二十兩。



 率府率、副,中郎將,十三千。春、冬絹各五匹,冬綿十五兩。自諸衛上將軍以下,春衣羅一匹。



 內客省使,六十千。客省使,三十七千。延福宮、景福殿、宣慶、引進、四方館、宣政、昭宣、合門使,二十七千。皇城以下諸司使,二十五千。



 春絹各十匹,冬十匹,綿三十兩,惟客省使春、冬絹各一十匹。



 客省及皇城以下諸司副使,二十千。內殿承制,十七千,崇班,十四千。春絹各五匹,冬十匹,綿三十兩。帶合門祗候並同。



 供奉官,十千。合帶合門祗候者,十二千。春絹四
 匹,冬五匹,綿二十兩,



 侍禁,七千。帶合門祗候者,一十千。殿直,五千。帶合門祗候者,九千。並春、冬絹各四匹,冬綿十五兩。



 三班奉職、借職,四千。春、冬絹各三匹,錢二千。



 下茶酒班殿侍,一千。春、冬絹七匹,冬綿十五兩。



 下班殿侍,七百。春、冬絹各五匹,二項並蕃官並土人補充者。



 皇親任諸衛大將軍領刺史,八千;將軍刺史,六十千。春、冬綾十匹,春絹十二匹,冬十三匹,綿五十兩,舊志:春、冬綾十匹,絹十五匹,各加羅一匹。



 將軍,三十千。春、冬綾三匹,絹五匹,羅一匹,冬綿四十兩。



 率府率,二十千;副率,十五千。春、冬綾各二匹,絹五匹,羅一匹,綿四十兩。



 舊志:諸衛將軍有五十千、四十千、三十千三等。一等春、冬
 各綾五匹,絹十匹;一等綾二匹,絹五匹。春並加羅一匹,冬並綿二十兩。



 諸司使有四十千、三十千二等。副使以下與異姓同,並給實錢。自諸司使至殿直,春、冬各羅一匹,綾一匹,絹各五匹,冬綿各四十兩。



 入內內侍省都知、副都知、押班,不帶遙郡諸司使充者,二十五千。春絹七匹,冬十匹,綿三十兩。



 副使充者,二十千。春絹五匹,冬七匹,綿二十兩。入內內侍省供奉官,十二千。春絹五匹,冬七匹,綿三十兩。



 殿頭,七千。高品、高班,五千。春絹各五匹,冬六匹,綿二十兩。



 黃門,三千。春、冬絹各五匹,綿十五兩。



 祗候殿頭,祗候高品,祗候高班內品,祗候內品,祗
 侯小內品,貼祗候內品,入內內品,後苑內品,後苑散內品,七百。春、冬絹各五匹,綿十五兩。



 雲韶部內品,七百。春、冬絹各四匹,綿十五兩。入內內品管勾,二千。奉替祗應,一千五百。打牧祗應,一千。春、冬絹各五匹,綿各十五兩。



 內侍省內常侍,供奉官,十千,春、冬絹各五匹,內常侍春加羅一匹,冬綿十五兩。供奉官冬止加綿二十兩。



 殿頭,五千。高品、高班,三千,春、冬絹各四匹,冬綿各二十兩。



 黃門,二千。春、冬絹各四匹,冬綿十五兩。



 殿頭內侍,入內高品,二千。春、冬絹各三匹,錢二千。



 高班內品,一千五百,衣糧帶舊。



 黃門內品在京人事,一千。春、冬各碧羅、碧綾半匹,黃絹、生白絹各一匹,綿八兩。



 寄班小底,二千。春、冬絹各十匹。



 入內小黃門,前殿祗候內品,北班內品,外處揀來並城北班、後苑、把門內品,掃灑院子及西京內品依北班內品,依舊在西京收管,七百。西京內品,五百。



 春、冬絹各五匹,綿各十五兩,惟入內小黃門、前殿祗候內品,春、冬絹各四匹。



 郢、唐、復州內品,三百。春、冬絹各二匹。布半匹,錢一千。舊志載內官不詳,奉料皆減少。



 樞密都承旨,四十千。副都承旨,副承旨,樞密院諸房副承旨,逐房副承旨,已上如帶南班官同。



 中書堂後官提點五房公事,三十千,都承旨以下春、冬絹各十五匹,春羅一匹,逐房副承旨絹各十三匹。都承旨、承旨春加綾三匹,冬五匹,綿五十兩。副
 都承旨以下,綿各三十兩。



 中書堂後官,二十千;特支五千。已上如帶京朝官同。



 中書、樞密主事,二十千。錄事、令史,二千。春、冬絹各十匹,春羅一匹,主事已上,冬綿五十兩,錄事、令史三十兩。



 主書,七千。守當官,書令史,五千。春、冬絹各二匹。主書書令史春錢三千,冬綿十二兩、錢一千,守當官春錢一千。



 自中書、樞密並曾任兩府,雖不帶職,曾任兩府而致仕同。



 宣徽,三司,觀文、資政、翰林、端明、翰林侍讀侍講、龍圖、天章學士,樞密、龍圖,天章直學士,知制誥,中書舍人,待制,御史臺,開封府,節度使至刺史,三館,秘閣,審刑院,刑部,大理寺,諸王府記室、翊善以
 下至諸王宮教授,知審官院,勾當三班院,糾察刑獄,判吏部銓、南曹,+-登聞檢院、鼓院,司農寺及國子監直講、丞、簿,河北、河東、陜西轉運使,皇子親王,諸衛大將軍至率府副率,兩省都知、押班,不帶遙郡諸司使、副,兩府供奉官以下至內品,惟內品特給一分見錢。



 及樞密都承旨以下,並給見錢。餘官並防禦使以下諸衛將軍、橫行、諸司使遙領者,悉一分見錢,二分他物。



 其兩省都知、副都知遙領刺史以上者,即給一半見錢。



 三司檢法官,十千。春、冬絹各五匹,冬綿十五兩。願請前任請受者聽,若轉京朝官,隨本官料
 錢、衣賜。



 權知開封府並判官、推官,料錢、衣賜並隨本官。舊志云:判官三十千,推官二十千,並給見錢。司錄,二十千。如差員外郎已上充。隨本官料錢、衣賜。



 功曹,法曹,十二千。倉、戶、士、兵四曹,十千。差京朝官充,隨本官料錢、衣賜。



 刑部檢法官、法直官,大理寺法直官、副法直官,十千。春、冬絹各五匹,冬綿十五兩。如轉京朝官。隨本官料錢、衣賜。



 西京軍巡判官,十五千。內開封府轉至京官,支本官衣奉。



 西京、南京、北京留守判官,河南、應天、大名府判官,三十千。春、冬絹各十二匹,冬綿二十兩。



 節度、觀察判官,二十五千。春、冬絹各六匹,冬綿十二兩半。



 節度副使,三十千。行軍司馬,二十五千。如簽書
 本州公事,衣奉依節、察判官。若監當即給一半折支,衣賜、廚料不給。



 節度掌書記,觀察支使,二十千。綿、絹如推官。



 留守推官,府推官,節度、觀察推官,十五千。春、冬絹各五匹,冬綿十兩。



 防禦、團練副使。二十千。如監當即給一半折支。防禦、團練判官,十五千。《兩朝志》云:奉給依本州錄事參軍,如無,依倚郭縣令。



 防禦、團練軍事推官,軍、監判官,七千,軍事判官如本州錄事參軍之數。



 京府司錄參軍,二十千。諸曹參軍,十千。以京官知者奉從多給。景德三年,詔司錄、六曹悉給春、冬衣。



 五萬戶已上州三京同。



 錄事參軍,二十千;司理,司法,十二千;司戶,十千;三萬戶已上州
 錄事,十八千;司理,司法,十二千;司戶,九千。一萬戶已上州錄事,十五千;司理,司法,十千;司戶,八千。五千戶已上州錄事,十二千;司理,司法,十千;司戶,七千。不滿五千戶州錄事,司理,司法,十千;司戶,七千。別駕,長史,司馬,司士參軍,如授士曹,依司士。



 文學參軍,七千。



 東京畿縣七千戶已上知縣,朝官二十二千,京官二十千;五千戶已上知縣,朝官二十千,京官十八千;三千戶已上知縣,朝官十八千,京官十五千;三千戶已下知縣,止命京官,十二千。



 已上
 衣賜並隨本官。



 主簿,尉,十二千至七千,有四等。並給見錢。



 河南府河南、洛陽縣令,三十千。諸路州軍萬戶已上縣令,二十千;簿、尉,十二千。七千戶已上令,十八千;簿、尉,十千。五千戶已上令,十五千;簿、尉,八千。三千戶已上令,十二千;簿、尉,七千。不滿三千戶令,十千;簿、尉,六千。京朝官及三班知縣者,亦許給縣令奉。本官奉多者,以從多給。



 兼監兵者,止請本奉添給。



 岳瀆廟令,十千。丞,主簿,七千。全折。



 幕職、州縣料錢,諸路支一半見錢,一半折支。縣尉全給見錢。



 廣東、川峽並給見錢。



 元豐制行:宰相,三百千。



 衣賜綾、絹、綿皆如舊制。然以左、右僕射為宰相。政和中,以三公為真相。靖康依舊制。樞密使帶使相,侍中,樞密使,節度使同中書門下平章事以上及帶宣微使,並前兩府除節度使移鎮,樞密使、副知院帶節度使,四百千。自治平末至元豐四年,如文彥博、呂公弼、馮京、吳充先後為使、副,是年十一月,始詔樞密院置知院、同知院,餘並罷。至是,既罷使、副,只置知院、同知院,直至靖康不改。



 知樞密院,門下、中書侍郎,尚書左、右丞,同知樞密院事,二百千。衣賜如舊。元祐中,復置簽書樞密院事,紹聖中罷。



 太師,太傅,太保,少師,少傅,少保,四百千。



 春服羅三匹,小綾三十匹,絹四十匹,冬服小綾三十匹,絹四十匹,綿二百兩。舊制,奉錢百二十千,春服小綾十匹,絹三十匹,羅一匹,冬服小綾十匹,絹三十匹,綿五十兩。大觀間增改。



 開
 府儀同三司,百二十千。春、冬各小綾十匹,絹三十匹,春羅一匹,冬綿五十兩。大觀二年,以無特任者,遂刪去。



 特進,九十千。春、冬各小綾十匹,絹二十五匹,春羅一匹,冬綿五十兩。



 金紫光祿大夫,銀青光祿大夫,光祿大夫,六十千。春、冬各小綾七匹,絹二十匹,春羅一匹,綿五十兩。



 宣奉、正奉、正議、通奉大夫,五十五千。春、冬各小綾五匹,絹十七匹,春羅一匹,冬綿五十兩。



 通議、太中大夫,五十千。《無豐令》,太中大夫以上丁憂解官,給舊官料錢。



 中大夫,中奉、中散大夫,四十五千。春、冬各小綾三匹,絹十五匹,春羅一匹,冬綿五十兩。



 朝議、奉直、朝請、朝散、朝奉大夫,三十五千。春、冬絹各十三匹,春羅一匹,冬綿三十兩。



 朝請、朝散、朝奉郎,三十
 千。春、冬服同正郎。



 承議、奉議、通直郎,二十千。承議春、冬絹各十匹,春羅一匹,冬綿三十兩。奉議、通直,春、冬各絹七匹。



 宣教郎,十七千。春、冬絹各六匹,春羅一匹,冬綿二十兩。《無豐格》:有出身十七千,無出身十四千。六年,敕不以資考有無出身,並十五千,衣無羅。



 宣義郎,十二千。春、冬各絹五匹,冬綿十五兩。



 承事郎,十千。春、冬絹各三匹,冬綿十五兩。



 承奉郎,八千。承務郎,七千。元豐以來,厘務止支驛料。大觀二年,定支。



 承直郎,二十五千。



 春、冬絹各六匹,綿十二兩半。元豐,留守判官、府判官,奉錢三十千,春、冬絹各十二匹,綿二十兩;節度、觀察判官,奉錢二十五千,春、冬絹各六匹,綿十二兩半,凡二等,崇寧二年,改從一等。



 儒林郎,二十千。



 春、冬絹各五匹,綿十兩。元豐,節度掌書記、觀察支使,奉錢衣賜如上;防、團軍事判官考任合入令錄者,奉錢十
 五千,凡二等。崇寧改從一等。



 文林郎,十五千。春、冬服同儒林。



 從事、從政、修職郎,十五千。



 從事郎,元豐舊制,考第合入令錄者,視令錄支,未合入令錄者,視判、司、簿、尉支。從政郎,元豐,三京、州、府、軍、監司錄、錄事參軍,五萬戶以上二十千,三萬戶以上十八千,一萬戶以上十五千,五千戶以上十二千,不滿五千戶十千。縣令,一萬戶以上二十千,七千戶以上十八千,五千戶以上十五千,三千戶以上十二千,不滿二千戶十千,凡二等。崇寧改從一等。



 迪功郎,十二千。



 元豐,四京軍巡判官,十五千。三京,州、府、軍、監司法參軍,五萬、三萬戶以上十二千,二萬戶及不滿五千戶七千。三京、州、府、軍、監司戶參軍,及五萬戶以上十千,三萬戶以上九千,一萬戶以上八千,不滿五千戶七千,凡三等。崇寧改。初,熙寧四年,中書門下言:「天下選人奉薄而多少不均,不足以勸廉吏。今欲月增料錢:縣令、錄事參軍三百六十七員,舊請十千、十二
 千者,增至十五千;司理、司法、司戶參軍,主簿、縣尉二千一百五十三員,舊請七千、八千、十千者,增至十二千;防、團軍事推官,軍、監判官一百七十二員,舊請七千者,增至十二千。月通增奉錢一萬二千餘貫,米麥亦有增數。」從之。



 太尉,一百千。春、冬各小綾十匹,春羅一匹,絹十匹,冬絹二十匹,綿五十兩。帶節度使依本格。



 節度使,四百千。曾任執政以上除,及移鎮、初除,及管軍,並同舊制。承宣使,三百千。即節度觀察留後。



 觀察使,防禦使,二百千。團練使,百五十千。刺史,一百千。自節度使以下至諸衛中郎將,並如舊制。



 通侍大夫,三十七千。正侍、宣正、協忠、中侍、中亮、中衛、翊衛、親衛、拱衛、左武、右武大夫,二十七千。武功、武德、武顯、
 武節、武略、武經、武義、武翼大夫,二十五千。



 春、冬絹各十匹,綿二十兩。惟通侍大夫十二匹。



 正侍、宣正、履正、協忠、中侍、中亮、中衛、翊衛、親衛、拱衛、左武、右武、武功、武德、武顯、武節、武略、武經、武義、武翼郎,二十千。敦武郎,十七千。修武郎,十四千。



 春絹五匹,冬七匹,綿二十兩。帶合門祗候並同。



 從義、秉義郎,十千。帶合門祗候十二千。



 成忠、保義郎,五千。帶合門祗候者九千,並春、冬絹各四匹,冬綿十五兩。



 承節、承信郎,四千。春、冬絹各三匹,錢二千。



 進武校尉,三千,進義校尉,二千。春、冬絹各三匹。



 進武副尉,三千。守闕進武副尉、進義副尉、守闕進義副尉,一千。



 凡文武官料錢,並支一分
 見錢,二分折支。曾任兩府雖不帶職,料錢亦支見錢。



 職錢



 御史大夫,六曹尚書,行,六十千。守,五十五千;試,五十千。



 翰林學士承旨,翰林學士五十千。衣賜,本官例。官小,春、冬服小綾各三匹,絹各十五匹,綿五十兩。



 左、右散騎常侍,御史中丞,開封尹,行,一百千。守,九十千;試,八十千。崇寧四年復位。



 六曹侍郎,元祐中,置權六曹書,奉給依守侍朗。紹聖中罷。



 行,五十五千。守,五十千;試,四十五千。



 太子賓客、詹事,行,五十千。守,四十七千;試,四十五千。



 給事中,中書舍人,行,五十千。守,四十五千;試,四十千。



 左、右諫議大
 夫,元祐中,置權六曹侍郎,奉給依諫議大夫,紹聖中,罷。



 行,四十五千。守,四十千;試,三十七千。



 太常、宗正卿,行,三十八千。守,三十五千;試,三十二千。



 秘書監,行,四十二千。守,三十八千;試,三十五千。



 七寺卿,國子祭酒,太常、宗正少卿,秘書少監,行,三十五千。守,三十二千;試,三十千。



 太子左、右庶子,行,四十千。守,三十七千;試,三十五千。



 七寺少卿,行,三十二千。守,三十千;試;二十八千。



 中書、門下省檢正諸房公事,尚書左、右司郎中,行,四十千。守,三十七千;試三十四千。



 國子司業,少府、將作、軍器監,行,三十二千。守,三十千;試,二十八千。



 太子少詹事,行,三十五千。守,三十二千;試,三十千。



 太子左、右諭德,行,三十二千。守,三十千;試,二十九千。



 起居郎,起居舍人,侍御史,左、右司員外郎,樞密院檢詳諸房文字,尚書六曹郎中,行,三十七千。守,三十五千;試,三十二千。



 殿中侍御史,左、右司諫,行,三十五千。守,三十二千;試,三十千。左、右正言,行,三十二千。守,三十千;試,二十七千。



 諸司員外郎,行,三十五千。守,三十二千;試,三十千。



 少府、將作、軍器少監,行,三十千。守,二十八千;試,二十五千。



 太子侍讀、侍講,行,二十五千。守,二十二千;試,二十千。



 監察御史,行,三十二千。守,三十千;試,二十七千。



 太子中舍,太子舍人,行,二十二千。守,二十千;試,十
 八千。



 太常、宗正、知大宗正,秘書丞,大理正,著作郎,太醫令,行,二十五千。守,二十二千;試,二十千。



 七寺丞,行,二十二千。守,二十千;試,十八千。



 秘書郎,行,二十二千。守,二十千;試,十八千。



 太常博士,著作佐郎,行、守,二十千。試,十八千。



 國了監丞,行,二十二千。守,二十千。



 大理司直、評事,行,二十二千。守,二十千;試,十八千。



 少府、將作、軍器、都水監丞,行,二十千。守,十八千。



 秘書省校書郎,行,十八千。守,十六千;試,十四千。秘書省正字,行,十六千。守,十五千;試。十四千。



 御史檢法官,主簿,行,二十千。守,十八千。



 宗學、太學、武學博士,行,二十千。守,十八千;試,十六千。



 律學博士,行,十八千。守,十七千;試,十六千。



 太常寺奉禮郎,行,十六千,太常寺太祝、郊社令,行,十八千。守,十六千。太學正、錄,武學諭,行,十八千。守,十七千;試,十六千。



 律學正,行,十六千。守,十五千;試,十四千。



 凡職事官職錢,不言「行」、「守」、「試」者,準「行」給,衣隨寄錄官例支。及無立定例者,並隨寄祿官給料錢,米麥計實數給,應兩給者,謂職錢、米麥。



 從多給。承直郎以下充職事官,謂大理司直、評事,秘書省正字,太學博士、正、錄,武學博士、諭,律學博士、正。



 聽支階官請給。衣及廚料、米麥不支。



 唐貞元四年,定百官月俸。僖、昭亂離,國用窘闕,至天祐
 中,止給其半。梁開平三年,始令全給。後唐同光初,租庸使以軍儲不充,百官奉錢雖多,而折支非實,請減半數而支實錢。是後所支半奉,復從虛折。周顯德三年,復給實錢。



 宋初之制,大凡約後唐所定之數。乾德四年七月,詔曰:「州縣官奉皆給他物,頗聞貨鬻不充其直,責以廉隅,斯亦難矣。至有賦於廛肆,重增煩擾,且復抵冒公憲,自罹刑闢,甚無謂也。漢乾祐中,置州縣官奉戶,除二稅外,蠲其它役,周顯德始革其制。自今宜逐處置回易料
 錢戶,每本官所受物,凡一千,分納兩戶,恣其貿易,戶輸錢五百,蠲役之令,悉如漢詔;所賦官物,令諸州計度充一歲所給之數,與蠶鹽同時並給之。其萬戶縣令、五萬戶州錄事、兩京司錄,舊月奉錢二萬者,給四十戶,率是為差;簿、尉及戶、法掾,舊月奉六千者,增一千,如其所增之數,給與奉戶。」是歲,令西川官全給實錢。開寶三年,令西川州縣官常奉外別給鐵錢五千。四年十二月,詔:「節、察、防、團副使權知州事,節度掌書記自朝廷除授及判
 別廳公事者,亦給之。副使非知州、掌書記奏授而不厘務者,悉如故,給以折色。」



 太平興國元年,詔曰:「耕織之家,農桑為本,奉戶月輸緡錢,蠢茲細民,不易營置,罷天下奉戶。其本官奉錢,並給以官物,令貨鬻及七分,仍依顯德五年十二月詔,增給米麥。」二年二月,詔:「諸道所給幕職、州縣官奉,頗聞官估價高,不能充七分之數。宜令三分給一分見錢,二分折色,令通判面估定官物,不得虧損其價。」四月,令西川諸州幕職官奉外,更增給錢五千。
 雍熙三年,文武官折支奉錢,舊以二分者,自今並給以實價。端拱元年六月,詔曰:「州郡從事之職,皆參贊郡畫,助宣條教;而州縣之任,並飭躬蒞政,以綏吾民。廩祿之制,宜從優異,庶幾豐泰,責之廉隅。除川峽、嶺南已給見錢外,其諸州府幕職、州縣官料錢,舊三分之二給以他物,自今半給緡錢,半給他物。」淳化元年五月,詔:「致仕官有曾歷外職任者給半奉,以他物充。」三年十一月,令京東西、河北、河東、陜西幕職州縣官料錢,當給以他物者,
 每千給錢七百。



 初,川峽、廣南、福建幕職州縣,並許預借奉錢。大中祥符間,又詔江、浙、荊湖遠地,麟、府等州,河北、河東緣邊州軍,自今許預借兩月,近地一月奉錢。



 至道二年詔;先是,京官滿三十月罷給,自今續給之。



 真宗即位,以三司估百官奉給折支直,率增數倍,詔有司復位,率優其數。咸平元年六月,詔:「文武群臣有分奉他所而身沒,未聞訃已給者,例追索,可憫。自今川峽、廣南、福建一季,餘處兩月,悉蠲之。」



 大中祥符七年詔。」三班使臣自今父母亡,勿住奉。」



 三年九月,詔群臣月奉折支物,無收其算。五年七月,增川峽路朝官使臣等
 月給添支。景德四年九月,上以承平既久,賦斂至薄,軍國用度之外,未嘗廣費自奉,且以庶官食貧勸事,遂詔:「自今掌事文武官月奉給折支,京師每一千給實錢六百,在外四百,願給他物者聽。」大中祥符五年,詔文武官並增奉。



 三師、三公、東宮三師、僕射各增二十千。三司、御史大夫、六尚書中丞、郎、兩省侍郎、太常宗正卿、內客省使、上將軍各增十千。橫班諸司各增五千。朝官五品正、中郎將已上、諸司使、副各增三千。京官、內殿承制、崇班、合門祗候各增二千。供奉官各增一千五百。奉職、借職增一千。餘如舊。



 自乾興以後『,更革為多。至嘉祐始著《錄令》。



 元豐一新官制,職事官職錢
 以寄祿官高下分行、守、試三等。大率官以《錄令》為準,而在京官司供給之數,皆並為職錢。如大夫為郎官,既請大夫奉,又給郎官職錢,視嘉祐為優矣。至崇寧間,蔡京秉政,吳居厚、張康國輩,於奉錢、職錢外,復增供給食料等錢。如京,僕射奉外,又請司空奉,其餘傔從錢米並支本色,余執政皆然,視元豐制祿復倍增矣。



 武臣奉給



 殿前司,自宣武都指揮使三十千,差降至歸明神武、開
 封府馬步軍都指揮使十五千,凡二等。殿前左、右班虞候三十千,至天武、剩員都虞候十九千,凡四等。殿前班指揮使二十千,至揀中、剩員僚直、廣德指揮使十千,凡三等。殿前班都知十三千,至招箭班都知四千,凡七等。殿前班副都知十千,至招箭班副都知三千,凡五等。殿前押班七千,至招箭押班二千,凡五等。散指揮都頭復有押班之名者,如押班給焉。兵士內員僚直復有副指揮使、行首、副行首,招箭班亦有行,七千至三千,凡三等。
 御龍直副指揮使、都頭、副都頭、十將、虞候十千至三千,凡五等。殿前指揮使五千,至殿侍一千,凡五等。捧日、天武指揮使十千,至揀中、廣德指揮使四千,凡四等。捧日、天武副指揮使七千,至擒戎副指揮使三千,凡五等。捧日軍使、天武都頭五千,至擒戎軍使千五百,凡五等。捧日副兵馬使三千,至擒戎副兵馬使一千,凡四等。天武副都頭二千,至廣德副都頭千五百,凡二等。捧日軍將二千,至龍猛、驍騎、帶甲剩員軍頭、十將三百,凡八等。天
 武將虞候而下五百,至飛猛驍雄將虞候已下三百,凡六等。此奉錢之差也。



 其外,月給粟:自殿前班都頭、虞候十五石,至廣健副都頭、吐渾十將二石五斗,凡六等。殿前指揮使五石,鞭箭、清朔二石,凡五等。殿前班都虞候已下至軍士,歲給春、冬服三十匹至油絹六匹,而加綿布錢有差。復月給傔糧自十人以至一人。諸班、諸直至捧日、天武、拱聖、龍猛、驍騎、吐渾、歸明渤海、契丹歸明神武、契丹直、寧朔、飛猛、宣武、虎翼、神騎、驍雄、威虎、衛聖、清
 朔、擒戎軍士,皆給傔一人以至半分,餘軍不給焉。



 侍衛馬軍、步軍司,自員僚直、龍神衛都虞候月給二十千,至有馬勁勇員七千,凡五等。指揮使自員僚直、龍神衛十千,至順化三千,凡五等。副指揮使自員僚直、龍神衛七千,至順化二千,凡七等。軍使、都頭自龍、神衛五千,至看船神衛一千,凡七等。副兵馬使、副都頭自龍、神衛三千,至順化一千,凡五等。軍頭、十將自龍、神衛千三百,至順化三百,凡五等。此外員僚直有行首、副行首、押番軍頭、
 都知、副都知之名,自行首五千,至副都知一千,凡六等。而高陽關有驍捷左、右廂都指揮使,月給三十千。開封府有馬步軍都虞候,月給二十千。六軍復有都虞候,月給五千。



 員僚直、龍神衛而下,皆月給粟,自都虞候五石,至順化、忠勇軍士二石,凡五等。自都虞候以下至軍士,皆歲給春冬服,自絹三十匹至油絹五匹,又加綿布錢有差。復有給傔糧,自十人至一人。其員僚直、龍神衛、雲騎、驍捷、橫塞、及神衛上將、虎翼、清衛、振武、忠猛軍士,
 皆給傔一人至半分,他軍不給焉。



 宣徽院、軍頭司,自員僚至軍士,咸月給錢粟及春冬服有差。



 諸道州府廂軍,自馬步軍都指揮使至牢城副都頭,凡五等,月給奉錢凡十五千至五百,凡十有二等。自河南府等五十州、府,鄧州等三十四州,萊州等一百四十四州、軍,廣濟軍等三十九軍、監,所給之數,差而減焉,咸著有司之籍。外有給司馬芻秣,歲給春、冬服加紬、綿、錢、布,亦各有差。



 祿粟自宰相至入內高品十八等



 宰相,參知政事,樞密使同中書門下平章事,樞密使、副使、知院事、同知院事,及宣徽使簽書樞密院事,節度觀察留後知樞密院事及充樞密副使、同知樞密院事,並帶宣微使簽書,檢校太保簽書,及三司使,中書、門下侍郎,尚書左、右丞,太尉,月各一百石。



 樞密使帶使相,節度使同中書門下平章事已上及帶宣徽使,並前兩府除節度使,樞密使、副、知院事帶節度使,月各給二百石。



 三公、三少,一百五十石。權三司使公事,七十石。權發遣使,
 三十五石。內客省使,二十五石。



 節度使,一百五十石。管軍同。如皇族節度使同中書門下平章事已上,並散節度使及帶王爵者,並一百石。



 留後後改承宣使,觀察、防禦使,一百石。管軍並兩省都知押班、諸衛大將軍、橫行遙領者同。惟皇族遙領防禦使七十石。



 團練使,七十石。管軍並皇族及軍班除充者同。其餘正任並五十石。若皇族並兩省都知押班、諸衛大將軍、將軍、橫行遙領者同。



 刺史,五十石。



 皇族並軍班除充者同。其餘正任並管軍三十石。兩省都知押班、通侍大夫遙領者二十五石。諸衛大將軍、將軍遙領者十石。橫行遙領者全分二十五石,減定十石。捧日、天武左右廂都指揮使,龍衛、神衛右廂都指揮使帶遙郡團練使五十石。殿前諸班直、都虞候、龍衛、神衛及諸軍都指揮使帶遙郡刺史二十五石。



 凡一石給六斗,
 米麥各半。管軍支六分米,四分麥。



 赤令,七石;丞,四石,京府司錄,五石。諸曹參軍,四石至三石,有二等。畿縣知縣六石至三石,有四等。主簿、尉米麥三石至二石,有二等。諸州錄事,五石至三石,有三等。司理、司法,四石至三石,有二等。司戶,三石、二石,有二等。諸縣令,五石至三石,有三等。



 惟河南洛陽縣令隨戶口支。



 簿、尉,三石、二石,有二等。四京軍巡、判官,四石。軍、監判官,防、團推官,二石。司天監丞,四石。主簿,靈臺郎,保章正,二石。已上並給米麥。



 入內內侍省供奉官,四石。殿頭,高品,
 三石。高班,黃門,入內內品,管勾奉輦祗應,入輦祗應,二石。打牧祗應,一石五斗。已上並給粳米。



 祗候殿頭,祗候高品,祗候高班內品,祗候內品,祗候小內品,貼祗候內品,入內內品,後苑內品,後苑散內品,三石。雲韶部內品,一石。



 已上並給丹糧。惟雲韶內品給細色。



 內侍省供奉官,三石。殿頭,高品,高班,二石。黃門,一石五斗。已上並給粳米。



 黃門內品在京人事,二石五斗。北班內品,前殿祗候內品,處處揀來並城北班、後苑、把門內品,掃灑院子及西京內品與北班內品,依舊在
 西京收管,西京內品,郢、唐、復州內品,二石。入內小黃門,一石。寄班小底,四石。



 已上並給月糧。惟入內小黃門給細色。



 殿頭內侍,入內高班,一石。米麥各半。



 熙寧四年,中書門下言:「天下選人奉薄,多少不一,不足以勸廉吏。欲月增米麥、料錢:縣令、錄事參軍三百七十六員,舊請米麥三石者,並增至四石。司理、司法、司戶、主簿、縣尉二千五百一十三員,舊請米麥兩石者,並增至三石。防、團軍事推官,軍、監判官一百七十二員,舊請米麥二石者,並增至三石。每月通增米
 麥三千七十餘石。」從之。



 元隨傔人衣糧任宰相執政者有隨身,任使相至正任刺史已上者有隨身,餘止傔人。



 宰相,並文臣充樞密使同中書門下平章事,及樞密使,七十人。宰相舊五十人衣糧,二十人日食,後加。



 樞密使帶使相,侍中樞密使,節度使同中書門下平章事已上及帶宣微使,並前兩府除節度使及節度使移鎮,樞密使、副、知院事帶節度使,一百人。



 參知政事,文臣充樞密副使、知院事、同知院事,及宣徽使不帶節度使簽書樞密院事,節度觀察
 留後知樞密院事並充樞密副使、同知樞密院事,並帶宣微使簽書樞密院事,三司使,門下侍郎,中書侍郎,尚書左、右丞,五十人。檢校太保簽書樞密院事,三十五人。權三司使,三十人。權發遣公事,十五人。副使、判官、判子司,五人。



 副使、判官權並權發遣同。



 觀文殿大學士,二十人。觀文殿學士,資政殿大學士,十人。資政、端明、翰林侍讀侍講、龍圖、天章學士,樞密直學士,保和、宣和、延康殿學士,寶文、顯謨、微猷閣學士,七人。



 舊止給日食,政和月糧二石。



 玉清昭應宮、景靈
 宮、會靈觀三副使,十人;判官,五人。



 節度使,留後改承宣使,觀察使。五十人。



 管軍同。如皇族節度使同中書門下平章事已上,並散節度使帶王爵,及節度觀察留後帶郡王,並五十人。觀察使,二十人。兩省都知、押班帶諸司使領節度觀察留後,五十人。兩省都知、押班並橫行領觀察使,十五人。



 防禦使,三十人。管軍同。皇族並遙領,並二十人。兩省都知、押班帶諸司使。並諸衛大將軍,及橫行遙領,並十五人。



 團練使,三十人。管軍及軍班除充者同。其餘除授者,二十人。皇族充及帶領,十五人。兩省都知、押班帶諸司使,並橫行遙領者,十人。



 刺史,二十人。軍班除充者同。其餘除授並管軍,十人。皇族充,十五人。兩省都知、押班帶諸司使,五人。橫行遙領全分者,五人。減定者不給。



 內客省使。舊有景福殿使。



 二十人。



 樞密都承旨,十
 人。副都承旨,副承旨,諸房副承旨,中書堂後官提點五房公事,七人。逐房副承旨,五人。中書堂後官至樞密院主事已上,各二人。錄事,令史,寄班小底,各一人。



 傔人餐錢中書、樞密、宣微、三司及正刺史已上,皆有衣糧,餘止給餐錢。



 自判三館、秘書監、兩制、兩省帶修撰,五千。朗中以下帶修撰者三千。



 直館閣,校理,史館檢討,校勘,各三千。直龍圖閣,審刑院詳議官,國子監書庫官,五千。自修撰已上又有職錢五千,校勘已上三千。



 京畿諸司庫、務、倉、場監官:朝官自二十千至五千,凡七等。
 京官自十五千至三千,凡八等。諸司使、副,合門通事舍人,承制,崇班,二十千至五千,凡九等,合門祗候及三班,十五千至三千,凡十等。內侍,十七千至三千,凡九等。寄班,八千至五千,凡三等。



 舊志訛舛,今並從《兩朝志》。



 茶、酒、廚料之給



 學士、權三司使以上兼秘書監,日給酒自五升至一升,有四等,法、糯酒自一升至二升,有二等。又宮觀副使,文明殿學士,即觀文。



 次政殿大學士,龍圖、樞密直學士,並有
 給茶。節度使、副以下,各給廚料米六斗,面一石二斗。



 薪、蒿、炭、鹽諸物之給宰相舊無,後加。



 宰相,樞密使,月給薪千二百束。參知政事,樞密副使,宣徽使,簽書樞密院事,三司使,三部使,權三司使,四百束。三部副使,樞密都承旨,一百五十束。樞密副都承旨,中書提點五房,一百束。開封判官,節度判官,薪二十束,蒿四十束。開封推官,掌書記,支使,留守、節度推官,防、團軍事判官,薪十五束,蒿三十束。留守判官,薪二十束,蒿三
 十束,防、團軍事推官,薪十束,蒿二十束。



 宰相,樞密使,歲給炭自十月至正月二百秤,餘月一百秤。參知政事,樞密副使,宣徽使,簽書樞密院事,三司使,三部使,三十秤,文明殿學士,資政殿大學士,龍圖閣學士,十五秤。都承旨,二十秤。



 給鹽:宰相,樞密使,七石。參知政事,樞密副使,簽書院事,宣微使,三司使,三部使,權三司使,二石。節度使,七石。掌兵遙領,五石。留後,觀察,防禦,團練,刺史,五石。



 掌兵、遙領皆不給。



 給馬芻粟者,自二十匹至一匹,凡七等。



 其軍職,內侍,三班,伎術,中書,樞密、宣微院侍衛,殿前司,皇城司,內侍省,入內內侍省吏屬借官馬者,其本廄馬芻粟隨給焉。



 給紙者,中書,樞密,宣微,三司,宮觀副使、判官,判官,諫官,皆月給焉。



 自給茶、酒而下,《兩朝志》無,《三朝志》雖不詳備,亦足以見一代之制云。



\end{pinyinscope}