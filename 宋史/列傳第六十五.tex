\article{列傳第六十五}

\begin{pinyinscope}

 謝泌孫何弟僅朱臺符戚綸張去華子師德樂黃目柴成務



 謝泌,字宗源,歙州歙人。自言晉太保安二十七世孫。少好學,有志操。賈黃中知宣州,一見奇之。太平興國五年
 進士,解褐大理評事、知清川縣,徙彰明,遷著作佐郎。端拱初,為殿中丞,獻所著文十編、《古今類要》三十卷,召試中書,以直史館賜緋。時言事者眾,詔閣門,非涉僥望乃許受之。繇是言路稍壅。泌抗疏陳其不可,且言:「邊鄙有事,民政未乂,狂夫之言,聖人擇焉。茍詰而拒之,四聰之明,將有所蔽。願採其可者,拒其不可者,庶顒顒之情,得以上達。」復言:「國家圖書,多失次序。唐景龍中,嘗分經、史、子、集為四庫,命薛稷、沉佺期、武平一、馬懷素分掌,望遵
 復故事。」遂令直館分典四部,以泌知集庫。改左正言,使嶺南採訪。



 淳化二年,久旱,復上言時政得失。時王禹偁上言:「請自今庶官候謁宰相,並須朝罷於政事堂,樞密使預坐接見,將以杜私請。」詔從之。泌上言曰:「伏睹明詔,不許宰相、樞密使見賓客,是疑大臣以私也。《書》曰:『任賢勿貳,去邪勿疑。』張說謂姚元崇曰:『外則疏而接物,內則謹以事君。此真大臣之禮。』今天下至廣,萬機至繁,陛下以聰明寄於輔臣,自非接下,何以悉知外事?若令都堂
 候見,則庶官請見咨事,略無解衣之暇。今陛下囊括宇宙,總攬英豪,朝廷無巧言之士,方面無姑息之臣,奈何疑執政,為衰世之事乎。王禹偁昧於大體,妄有陳述。」太宗覽奏,即追還前詔,仍以泌所上表送史館。會修正殿,頗施採繪,泌復上疏。亟命代以丹堊,且嘉其忠藎,拜左司諫,賜金紫、錢三十萬。一日,得對便殿,太宗稱其任直敢言,泌奏曰:「陛下從諫如流,故臣得以竭誠。昔唐季孟昌圖者,朝疏諫而夕去位,鑒於前代,取亂宜矣。」太宗動
 色久之。時,群臣升殿言事者,既可其奏,得專達於有司,頗容巧妄。泌請自今凡政事送中書,機事送樞密,金谷送三司,覆奏而行,從之。



 俄判三司鹽鐵勾院。奉詔解送國學舉人,黜落既多,群聚喧詬,懷甓以伺泌出。泌知之,潛由他塗入史館,數宿不敢出,請對自陳。太宗問:「何官騶導嚴肅,都人畏避?」有以臺雜對者,即授泌虞部員外郎兼侍御史知雜事。上元觀燈,泌特預召,自是為例。轉金部員外郎,充鹽鐵副使。頃之,魏羽為使,即泌之外舅,
 以親嫌,改度支副使。因郊祀,條上軍士賞給之數。太宗曰:「朕惜金帛,止備賞賜爾。」泌因曰:「唐德宗朱泚之亂,後唐莊宗馬射之禍,皆賞軍不豐所致。今陛下薄於躬御,賞賜特優,實歷代之所難也。」俄與王沔同磨勘京朝官。太宗孜孜為治,每御長春殿視事罷,復即崇政殿臨決,日旰未進御膳。泌言:「請自今長春罷政,既膳後御便坐。」不報。俄知三班、通進銀臺司,出知湖州。再遷主客郎中、知虢州。



 真宗初,邊人屢寇,泌上疏曰:



 臣竊惟聖心所切
 者,欲天下朝夕太平爾。雍熙末,趙普錄唐姚崇《太平十事》以獻。未幾,普復相,時稱致治之策無出於此。尋普病,又遼騎擾邊,因循未行。今北邊謐寧,繼遷請命,則可行於今日矣。臣以為先朝未盡行者,俟陛下爾。陛下自臨大寶,邊不加兵,西北肅然,民安歲登,則太平之象,復何遠哉。至於省不急之務,削煩苛之政,抑奔競,來直言,斯皆致太平之術,又豈讓唐開元之治也。議者或謂,方今用兵異於開元,且開元邊戎孔熾,明皇卒與之和。至如
 漢高祖亦然。此皆屈己以寧天下,豈以輕大國而競小忿乎。請以近事言,往歲討交□止,王師一動,南方幾搖。先皇以為得之無用,棄之實便,及授官為蕃屏,則至今竄伏。石晉之末,恥講和契丹,遂致天下橫流,豈得為強?或者有言,敵所嗜者禽色,所貪者財利,餘無他智計。先朝平晉之後,若不舉兵臨之,但與財帛,則幽薊不日納土矣。察此,乃知其情古猶今也、漢祖、明皇所用之計,正可以餌其心矣。



 臣伏睹近詔,以不逞之徒所陳述,皆閭閻
 事。臣聞古先哲王詢於芻蕘,察於邇言者,蓋慮視聽之蔽,故採此以達物情,亦罕行其事也。先朝有侯莫陳利用、陳廷山、鄭昌嗣、趙贊之徒,喋喋利口,賴先帝聖聰,尋翦除之,然為患已深矣。臣又聞輔時佐主,建萬世之基,立不拔之策者,必倚老成之人。至如成、康刑措,由任周、召;文、景清靜,不易蕭、曹;明皇太平,亦資姚、宋。夫精練國政,斟酌王度,未聞市井之胥,走塵之吏,可當其任也。惟陛下察往古用賢致治之道,則賢者亦必盡忠竭力,以
 輔成太平之治矣。



 咸平二年,徙知同州。代還,知鼓司、登聞院。五年,與陳恕同知貢舉,復知通進、銀臺司,加刑部,出為兩浙轉運使。近制,文武官告老皆遷秩,令錄授朝官,並給半俸。泌言:「請自今七十以上求退者,許致仕;因疾及歷任犯贓者,聽從便。」詔可。徙知福州,代還,民懷其愛,刻石以紀去思。轉兵部郎中,復知審官院,直昭文館。知荊南府,改襄州,遷太常少卿、右諫議大夫、判吏部銓。大中祥符五年卒,年六十三。



 泌性端直,然好方外之學,
 疾革,服道士服,端坐死。帝聞而嘆異,遣使臨問恤賜,錄其子衍為太常寺奉禮郎,衒將作監主簿。衍為太子中舍。



 孫何,字漢公,蔡州汝陽人。祖鎰,唐末秦宗權據州,強以賓佐起之。鎰偽疾不應,還家,以講授為業。父庸,字鼎臣,顯德中,獻《贊聖策》九篇,引唐貞觀所行事,以魏元成自況。得對,言曰:「武不可黷,斂不可厚,奢不可放,欲不可極。」世宗奇其言,命中書試,補開封兵曹掾。建隆初,為河南
 簿。太平興國六年,鴻臚少卿劉章薦其材,改左贊善大夫。歷殿中丞、知龍州而卒。



 何十歲識音韻,十五能屬文,篤學嗜古,為文必本經義,在貢籍中甚有聲。與丁謂齊名友善,時輩號為「孫丁」。王禹偁尤雅重之。嘗作《兩晉名臣贊》、《宋詩》二十篇、《春秋意》、《尊儒教儀》,聞於時。淳化三年舉進士,開封府、禮部俱首薦,及第又得甲科,解褐將作監丞、通判陜州。召入直史館,賜緋,遷秘書丞、京西轉運副使。歷右正言,改右司諫。



 真宗初,何獻五議:其一,請擇
 儒臣有方略者統兵;其二,請世祿之家肄業太學,寒雋之士州郡推薦,而禁投贄自媒者;其三,請復制舉;其四,請行鄉飲酒禮;其五,請以能授官,勿以恩慶例遷。上覽而善之。咸平二年,舉入閣故事,何次當待制,獻疏曰:



 六卿分職,邦家之大柄也。有吏部辨考績而育人材,有兵部簡車徒而治戎備,有戶部正版圖而阜貨財,有刑部謹紀律而誅暴強,有禮部祀神示而選賢俊,有工部繕宮室而修堤防,六職舉而天下之事備矣。故周之會
 府,漢之尚書,立庶政之根本,提百司之綱紀。令、僕率其屬,丞、郎分其行,二十四司粲焉星拱,郎中、員外判其曹,主事、令史承其事。四海九州之大,若網在綱。



 唐之盛時,亦不聞別分利權,創使額,而軍須取足。及元宗侈心既萌,召發既廣,租調不充,於是蕭景、楊釗始以地官判度支,而宇文融為租調地稅使,始開利孔,以構禍階。至於肅、代,則有司之職盡廢,而言利之臣攘臂於其間矣。於是叛亂相仍,經費不充,迫於軍期,切於國計,用救當時之
 急,卒以權宜裁之。五代短促,曾莫是思。



 今國家三聖相承,五兵不試,太平之業,垂統立制,在此時也。所宜三部使額,還之六卿,慎擇戶部尚書一人,專掌鹽鐵使事,俾金部郎中、員外郎判之。又擇本行侍郎二人,分掌度支、戶部使事,各以本曹郎中、員外郎分判之,則三使洎判官,雖省猶不省也。仍命左右司郎中、員外總知帳目,分勾稽違。職守有常,規程既定,則進無掊克之慮,退有詳練之名,周官唐式,可以復矣。茲事非艱,在陛下行之爾。



 是冬,從幸大名,詔訪邊事。何疏曰:



 陛下嗣位以來,訓師擇將,可謂至多,以高祖之大度,兼蕭王之赤心,神武冠於百王,精兵倍於前代。分閫仗鉞者,固當以身先士卒為心,賊遺君父為恥。而列城相望,堅壁自全,手握強兵,坐違成算,遂使腥膻得計,蛇豕肆行,焚劫我郡縣,系累我黎庶。陛下攄人神之忿怒,憫河朔之生靈,爰御六師,親幸澶、魏,天聲一振,敵騎四逃,雖鎮、定道路已通,而德、棣烽塵未息,此殆將帥或未得人,邊奏或有壅閼,鄰境
 不相救援,糗糧須俟轉輸之所致也。



 將帥者何?或恃勇無謀,或忌功玩寇,但全城堡,不恤人民。邊奏者何?護塞之臣,固祿守位,城池焚劫,不以實聞,老幼殺傷,托言他盜。不救援者何?緣邊州縣,城壘參錯,如輔車唇齒之相依,若頭目手足之相衛,托稱兵少不出,或待奏可乃行。俟輦輸者何?敵騎往還,猋馳鳥逝,贏糧景從,萬兩方行,迨乎我來,寇已遁去。此四者,當今急務。擇將帥,則莫若文武之內,參用謀臣;防壅閼,則莫若凡奏邊防,陛見庭
 問;合救援,則莫若督以軍令,聽其便宜;運糗糧,則莫若輕繼疾驅,角彼趫捷。



 今大駕既駐鄴下,契丹終不敢萌心南牧,所慮薦食者,惟東北無備之城,繕完周防,不可不慎。且蜂蠆有毒,豺狼無厭。今契丹西畏大兵,北無歸路,獸窮則搏,物不可輕,餘孽尚或稽誅,奔突亦宜預備。大河津濟,處處有之,亦望量屯禁兵,扼其要害,則請和之使,不日可待。



 真宗覽而嘉之。及傅潛逗撓無功,何又請斬潛以徇。俄權戶部判官,出為京東轉運副使,又獻
 疏請擇州縣守宰,省三司冗員,遴選法官,增秩益奉。未幾,徙兩浙轉運使,加起居舍人。景德初,代還,判太常禮院。俄與晁迥、陳堯咨並命知制誥,賜金紫,掌三班院。何先已被疾,勉強親職。一日,奏事上前,墜奏牘於地,俯而取之,復墜笏。有司劾以失儀,詔釋之。何慚,上章求改少卿監,分司西京養疾,上不許,第賜告,遣醫診視。醫勉其然艾,何答曰:「死生有命。」卒不聽。是冬卒,年四十四。上在澶淵,聞之憫惜,錄其子言為大理評事。



 何樂名教,勤接
 士類,後進之有詞藝者,必為稱揚。然性褊急,不能容物。在浙右專務峻刻,州郡病焉。好學,著《駁史通》十餘篇,有集四十卷。弟僅。



 僅字鄰幾。少勤學,與何俱有名於時。咸平元年,進士甲科,兄弟連冠貢籍,時人榮之。解褐舒州團練推官,會詔舉賢良方正之士,趙安仁以僅名聞。策入第四等,擢光祿寺丞、直集賢院,俄知浚儀縣。景德初,拜太子中允、開封府推官,賜緋。北邊請盟,遣使交聘,僅首為國母生辰
 使。改本府判官,遷右正言、知制誥,賜金紫,同知審官院。是冬,永興孫全照求代,真宗思擇循良任之,御書邊肅洎僅二名示宰相。或言僅嘗倅京府,諳民政,乃命知永興軍府。僅純厚長者,為政頗寬,嘗詔戒焉。大中祥符元年,加比部員外郎。代還,知審刑院。頃之,拜右諫議大夫、集賢院學士、權知開封府。改左諫議大夫,出知河中府。歸朝,復領審刑院。久次,進給事中。天禧元年正月卒,年四十九。錄其子大理評事和為衛尉寺丞。



 僅性端愨,中
 立無競,篤於儒學,士大夫推其履尚,有集五十卷。僅弟侑亦登進士第,至殿中丞。



 朱臺符字拱正,眉州眉山人。父賦,舉拔萃,歷度支判官,卒於殿中丞。臺符少聰穎,十歲能屬辭,嘗作《黃山樓記》,士友稱之。及長,善詞賦。時太宗廷試貢士,多擢敏速者,臺符與同輩課試,以尺晷成一賦。淳化三年,進士登甲科,解褐將作監丞、通判青州。召入直史館,賜緋魚,再遷秘書丞、知浚儀縣。



 咸平元年,與楊礪、李若拙、梁顥同知
 貢舉,俄以京府舊僚,擢太常博士,出為京西轉運副使。時北邊為梗,臺符上言曰:



 臣聞蠻夷猾夏,《帝典》所載,商、周而下,數為邊害。或振旅薄伐,或和親修好,歷代經營,斯為良策。至於秦築長城而黔首叛,漢絕大漠而海內虛,逞志一時,貽笑萬代,此商鑒不遠也。頃者,晉氏失御,中原亂離,太祖深鑒往古,酌取中道,與民休息,遣使往來。二十年間,罕聞入寇,大省戍邊之卒,不興出塞之兵。關防寧謐,府庫充溢,信深得制御之道也。



 幽薊之地,實
 維我疆,尚隔混同,所宜開拓。太宗平晉之後,因其兵勢,將遂取之。人雖協謀,天未厭亂,螗斧拒轍,用稽靈誅。重興吊伐之師,又作遷延之役。自茲厥後,大肆兇鋒,殺略軍民,攻拔城砦,長驅深入,莫可禁止。當是時也,以河為塞,而趙、魏之間,幾非國家所有。既阻歡盟,乃為備御,屯士馬,益將帥,芻粟之飛挽,金帛之委輸,贍給賞賜,不可勝數。繇是國家之食貨,匱於河朔矣。



 陛下自天受命,與物更始,繼遷授節,黎桓加爵,咸命使者鎮撫其邦。惟彼
 契丹,未加渥澤,非所以柔遠能邇,昭王道之無偏也。今祥禫將終,中外引頸觀聽德音。臣愚以為宜於此時赦契丹罪,擇文武才略習知邊境辨說之士,為一介使,以嗣位服除,修好鄰國,往告諭之。彼十年以來,不復犯塞,以臣計之,力有不足,志欲歸向,而未得其間也。今若垂天覆之仁,假來王之便,必歡悅慕義,遣使朝貢。因與之盡捐前惡,復尋舊盟,利以貨財,許以關市,如太祖故事,使之懷恩畏威。則兩國既和,無北顧之憂,可以專力西
 鄙,繼遷自當革心而束手矣,是一舉而兩得也。



 臺符又自請往使,時論韙之。



 咸平二年春,旱,詔求直言。臺符上疏,請重農積穀,任將選兵,慎擇守令,考課黜陟,輕徭節用,均賦慎刑,責任大臣,與圖治道。奏入,優詔褒答。入為鹽鐵判官,改判戶部勾院,拜工部員外郎,換度支判官。景德初,鄭文寶為陜西轉運,或言其張皇生事,徙臺符代之,仍賜金紫。



 臺符俊爽好謀,然頗以刻碎為舉職。與楊覃聯事,覃頗欲因仍舊貫,臺符則更革煩擾,議事違
 戾,交相掎奏,以不協聞,命御史視其狀。九月,徙臺符知郢州,覃知隨州。三年,召還,會執政有不喜者,復出知洪州,卒於舟次,年四十二。賜其子公佐同學究出身,賵錢二十萬。



 臺符好學,敏於屬辭,喜延譽後進,有集三十卷。公佐及臺符弟昌符,大中祥符中,舉進士,廷試並得第五人。初,昌符登科,宰相言昌符即臺符弟,上因言臺符有文學及著述可採,甚嗟悼之。公佐卒,又以次子壽隆試將作監主簿。昌符為屯田員外郎。



 戚綸字仲言,應天楚丘人。父同文,字文約,自有傳。綸少與兄維以文行知名,篤於古學,喜談名教。太平興國八年舉進士,解褐沂水主簿。按版籍,得逋戶脫口漏租者甚眾。徙知太和縣。同文卒於隨州,綸徒步奔訃千里餘。俄詔起復蒞職,就加大理評事。江外民險悍多構訟,為《諭民詩》五十篇,因時俗耳目之事,以申規誨,老幼多傳誦之。每歲時必與獄囚約,遣歸祀其先,皆如期而還。遷光祿丞,坐鞫獄陳州失實,免官。著《理道評》十二篇,錢若
 水、王禹偁深所賞重。久之,復授大理評事、知永嘉縣。境有陂塘之利,浚治以備水旱。復為光祿寺丞,轉運使又上其政績,連詔褒之。



 真宗即位,轉著作佐郎、通判泰州。將行,秘書監楊徽之薦其文學純謹,宜在館閣,命為秘閣校理。受詔考校司天臺職官,定州縣職田條制。詔館閣官以舊文獻,上嘉綸所著,特改太常丞,俄判鼓司、登聞院。出內府緡帛市邊糧,詔綸乘傳往均市之。



 景德元年,判三司開拆,賜緋魚,改鹽鐵判官。上疏言邊事,甚被
 嘉獎。十月,拜右正言、龍圖閣待制,賜金紫。時初建是職,與杜鎬並命,人皆榮之。綸久次州縣,留意吏事,每便殿請對,語必移晷,或夜中召見,多所敷啟。俄上奏曰:「夫出納獻替,王臣之任;章疏奏議,諫者之職。臣屢蒙召對,皆延數刻,屈萬乘之尊,接一介之士,聖德淵深,包納荒穢,體其至愚,不罪觸犯,安敢循嘿不言。謹摭十事該治本者附於章左:一曰王畿關輔,二曰五等封建,三曰復制科,四曰崇國學,五曰闢曠土,六曰修貢舉,七曰任大臣,
 八曰置平糴,九曰益廂軍、減禁兵,十曰修《六典》令式。」詞頗深切,上為嘉獎。



 二年,與趙安仁、晁迥、陳充、朱巽同知貢舉,綸上言取士之法,多所規制,並納用焉。預修《冊府元龜》,會置官總在京諸司之務,凡百三十司,命綸與劉承珪同領其事。判鴻臚寺。先是,群臣詔葬,公私所費無定式。綸言其事,詔同晁迥、朱巽、劉承珪校品秩之差,定為制度,遂遵行之。綸以三公、尚書、九列之任,唐末以來,有司漸繁,綱目不一,謂宜採《通禮》、《六典》令式,比類沿革,
 著為大典,時論稱之。進秩右司諫、兵部員外郎。時詔禁群臣匿名上封及非次升殿奏事,綸謂「忠讜之入,當開獎言路,若疏遠之士,尤艱請對」,上頗嘉之。



 大中祥符元年,掌吏部選事。上初受靈文,綸上疏曰:「臣遐稽載籍,歷考秘文,驗靈應之垂祥,顧天人之相接。陛下紹二聖丕業,啟萬世鴻基,勤行企道,恭默思元,上天降鑒,瑞牒昭錫,聿示臨民之戒,用恢奕葉之祥。乞詔有司,速修大祀,載命侍從,摹寫祥符,勒於嘉玉,藏之太廟,別以副本秘
 於中禁,傳示萬葉,無敢怠荒。然臣恐流俗幻惑狂謀,以人鬼之妖辭,亂天書之真旨。伏望端守元符,凝神正道,以答天貺,以惠烝黎。」是冬,封泰山,命綸同計度發運事。禮成,遷戶部郎中、直昭文館,待制如故。被詔,同編《東封祥瑞封禪記》。會峻待制之秩,又兼集賢殿修撰。建議修釋奠儀,頒於天下;立常平倉,隸司農寺,以平民糴,皆從之。嘗宴餞種放於龍圖閣,詔近臣為序,上覽綸所作,稱其有史才。



 三年,擢樞密直學士,上作詩寵之。祀汾陰,復
 領發運之職。居無何,出知杭州,就加左司郎中。屬江潮為患,乃立埽岸,以易柱石之制,雖免水患,而眾頗非其變法。胡則時領發運,嘗居杭州,肆縱不檢,厚結李溥,綸素惡之。通判吳耀卿,則之黨也,伺綸動靜,密以報則。則時為當塗者所暱,因共捃摭綸過,徙知揚州。惟揚亦溥、則巡內,持之益急,求改僻郡,徙徐州。八年,與劉綜並罷學士,授左諫議大夫。代還,復知青州。歲饑,發公廩以救餓殍,全安甚眾。徙鄆州,王遵誨為勸農副使,嘗佐西邊,
 寓家永興,閨門不肅,事將發,知府寇準為平之。綸因戲謔語及準,遵誨恚怒,以為污己,遂奏綸謗訕,坐左遷岳州團練副使,易和州。天禧四年,改保靜軍副使。是冬,以疾求歸故里,改太常少卿,分司南京。五年,卒,年六十八。



 綸篤於古學,善談名理,喜言民政,頗近迂闊。事兄維友愛甚厚,維卒,訃聞,哀慟不食者數日。與交游故舊,以信義著稱。士子謁見者,必詢其所業,訪其志尚,隨才誘誨之。嘗云:「歸老後,得十年在鄉閭講習,亦可以恢道濟世。」
 大中祥符中,繼修禮文之事,綸悉參其議,與陳彭年並職,屢召對,多建條式,恩寵甚盛。樂於薦士,每一奏十數人,皆當時知名士。晚節為權幸所排,遂不復振。善訓子弟,雖至清顯,不改其純儉。既沒,家無餘貲。張知白時知府事,輟奉以助其喪。家人於幾閣間,得《遺戒》一篇,大率皆誘勸為學。有集二十卷。又前後奏議,有機務利害、備邊均田之策,別為《論思集》十卷,分上下篇。天聖中,其子舜賓獻之,詔贈左諫議大夫。舜賓,官太子中舍。



 張去華,字信臣,開封襄邑人。父誼,字希賈。好學,不事產業。既孤,諸父使督耕隴上,他日往視之,見閱書於樹下,怒其不親穡事,詬辱之。誼謂其兄曰:「若不就學於外,素志無成矣。」遂潛詣洛陽龍門書院,與宗人沆、鸞、湜結友,故名聞都下。



 長興中,和凝掌貢舉,誼舉進士,調補耀州團練推官。晉天福初,代還。會凝由內署拜端明殿學士,署門不接賓客,誼聞之,即日致書於凝,以為「切近之職,實當顧問,四方利害,所宜詢訪,若不接賓客,聾瞽耳目,
 坐虧職業,雖為自安計,其可得乎?」凝大奇之,他日,薦於宰相桑維翰曰:「凝門生中有張誼者,性介直,頗涉辭藝,可備諫職。」未幾,超拜左拾遺。誼以晉室新造,典禮未完,數上章請復有唐故事。又言契丹有援立之助,所宜敦信謹備,不可自逸,以啟釁端。改右補闕,充集賢殿修撰,歷禮部員外郎、侍御史。改倉部、知制誥,加禮部郎中。



 乾祐初,真拜中書舍人。時蘇逢吉、楊邠、王章輩攀附漢祖,驟得大用,搢紳多附之,誼不為屈,故共嫉之。遣誼為吳
 越宣諭使,與兵部郎中馬承翰同往賜官告。浙人每迓朝使,必列步騎以自誇詫,誼與承翰竊笑之。又乘酒,言詞有輕發者,錢俶甚恥之,乃奏誼擅棰防援官。又夜集,與承翰使酒,語相侵,坐貶均州司戶,改房州司馬,歲餘卒。



 去華幼勵學,敏於屬辭,以蔭補太廟齋郎。周世宗平淮南,去華時年十八,慨然嘆曰:「兵戰未息,民事不修,非馭國持久之術。」因著《南征賦》、《治民論》,獻於行在。召試,授御史臺主簿。屬三院議事,不得預坐,謂所親曰:「簿領之
 職,非壯夫所為。」即棄官歸鄭州,杜門不出者三載。



 建隆初,始攜文游京師,大為李昉所稱。明年,舉進士甲科,即拜秘書郎、直史館。以歲滿不遷,上章自訴,因言制誥張澹、盧多遜、殿中侍御史師頌文學膚淺,願得校其優劣。太祖立召澹輩與去華臨軒策試,命陶穀等考之。澹以所對不應問,降秩,即擢去華為右補闕,賜襲衣、銀帶、鞍勒馬。朝議薄其躁進,以是不遷秩者十六年。嘗得對便殿,詢及家世,遂訴父始忤權貴,因罹重貶。宰相薛居正
 亦為言之,太祖為之動容,且曰:「漢室不道,奸臣擅權,此朕所親見也。」荊湖平,命通判道州。去華上言:「桂管為五嶺沖要,令劉鋹保境固守,賴之為捍蔽,若大軍先克其城,以趣番禺,如踐無人之境。」且言桂州可取之狀,有詔嘉獎。代還,知磁、乾二州,選為益州通判,遷起居舍人、知鳳翔府。



 從太宗征太原,監隨駕左藏庫,就命為京東轉運使。歷左司員外郎、禮部郎中。太平興國七年,為江南轉運使。雍熙中,王師討幽州,去華督宋州饋運至拒馬
 河,就命掌河北轉運事。三年,知陜州,未行,著《大政要錄》三十篇以獻,上覽而嘉之,詔書褒美,賜彩五十匹,因留不遣。會許王尹京,命為開封府判官,殿中侍御史陳載為推官,並賜金紫。謂曰:「卿等皆朝之端士,特加選用,其善佐吾子。」各賜錢百萬。逾歲,就拜左諫議大夫,又令樞密使王顯傳旨,諭以輔成之意。未幾,有廬州尼道安訟弟婦不實,府不為治,械系送本州。弟婦即徐鉉妻之甥。道安伐登聞鼓,言鉉以尺牘求請,去華故不為治。上怒,
 去華坐削一任,貶安州司馬。歲餘,召授將作少監、知興元府,未行,改晉州。遷秘書少監、知許州。



 真宗嗣位,復拜左諫議大夫。未幾,遷給事中、知杭州。兩浙自錢氏賦民丁錢,有死而不免者,去華建議請除之,有司以經費所仰,固執不許。咸平二年,徙蘇州。頃之,以疾求分司西京。在洛葺園廬,作中隱亭以見志。景德元年,改工部侍郎致仕。三年,卒,年六十九。



 去華美姿貌,善談論,有蘊藉,頗尚氣節。在營道得父同門生何氏二子,教其學問。受代,
 攜之京師,慰薦館穀,並登仕籍。嘗獻《元元論》,大旨以養民務穡為急,真宗深所嘉賞,命以縑素寫其論為十八軸,列置龍圖閣之四壁。然不飾邊幅,頗為清議所貶,以是不登顯用。有集十五卷。子師古至國子博士,師錫殿中丞,師顏國子博士。



 師德,字尚賢。去華十子,最器師德。嘗欲任以官,辭不就。去華曰:「此兒必繼吾志。」真宗祀汾陰,知河南府薛映薦其學行,又獻《汾陰大禮頌》於行在。是歲,舉進士亦為第
 一,時人榮之。除將作監丞、通判耀州。遷秘書省著作郎、集賢校理、判三司都理欠憑由司。建言:「有逋負官物而被系,本非侵盜,若煢獨貧病無以自償,願特蠲之。」帝用其言。嘗奏事殿中,帝訪以時事,而條對甚備。帝喜曰:「朕藩邸知卿父名,今又知卿才。」其後每遣使,帝輒曰:「張師德可用。」契丹、高麗使來,多以師德主之。天禧初,安撫淮南,苦風眩,改判司農寺。擢右正言、知制誥,判尚書刑部。頃之,出知穎州,遷刑部員外郎、判大理寺,為群牧使、景
 靈宮判官,再遷吏部郎中。以疾,知鄧州,徙汝州,拜左諫議大夫,罷知制誥。



 師德孝謹有家法,不交權貴,時相頗不悅之。然亦多病,在西掖九年不遷,卒於官。有文集十卷。子景憲,為太中大夫。



 樂黃目,字公禮,撫州宜黃人。世仕江左李氏。父史,字子正。齊王景達鎮臨川,召掌奏箋,授秘書郎。入朝,為平原主簿。太平興國五年,與顏明遠、劉昌言、張觀並以見任官舉進士。太宗惜科第不與,但授諸道掌書記。史得佐武
 成軍,既而復賜及第。上書言事,擢為著作佐郎、知陵州,獻《金明池賦》,召為三館編修。



 雍熙三年,獻所著《貢舉事》二十卷,《登科記》三十卷,《題解》二十卷,《唐登科文選》五十卷,《孝弟錄》二十卷,《續卓異記》三卷。太宗嘉其勤,遷著作郎、直史館。轉太常博士、知舒州,遷水部員外郎。淳化四年春,與司封員外郎、直昭文館李蕤同使兩浙巡撫,加都官、知黃州。又獻《廣孝傳》五十卷,《總仙記》一百四十一卷。詔秘閣寫本進內。史好著述,然博而寡要,以五帝、三
 王,皆雲仙去,論者嗤其詭誕。



 咸平初,遷職方,復獻《廣孝新書》五十卷,《上清文苑》四十卷。出知商州。史前後臨民,頗以賄聞。俄以老疾為言,聽解職,分司西京。五年,郊祀畢,奉留守司表入賀,因得召對。上見其矍鑠不衰,又知篤學,盡取所著書藏秘府,復授舊職,與黃目同在文館,人以為榮。出掌西京磨勘司,黃目為京西轉運。改判留司御史臺。車駕幸洛,召對,賜金紫。史久在洛,因卜居,有亭榭竹樹之勝,優游自得。未幾卒,年七十八。所撰又有《
 太平寰宇記》二百卷,《總記傳》百三十卷,《坐知天下記》四十卷,《商顏雜錄》、《廣卓異記》各二十卷,《諸仙傳》二十五卷,《宋齊丘文傳》十三卷,《杏園集》、《李白別集》、《神仙宮殿窟宅記》各十卷,《掌上華夷圖》一卷。又編己所著為《仙洞集》百卷。



 黃目淳化三年舉進士,補伊闕尉。遷大理寺丞、知壽安縣。咸平中,徙知壁州,未行,上章言邊事,召對,拜殿中丞。久之,直史館、知浚儀縣。俄上言曰:「伏以從政之原,州縣為急;親民之任,牧宰居先。今朝官以數任除知州,簿
 尉以兩任入縣令,雖功過易見,而能否難明。伏見唐開元二年選群官,有宏才通識、堪致理化者,授刺史、都督。又引新授縣令於宣政殿,試理人策一道,惟鄄城令袁濟及格,擢授醴泉令,餘二百人,且令赴任,十餘人並放令習學。臣欲望自今審官院差知州,銓曹注縣令,候各及三二十人,一次引見於御前,試時務策一道。察言觀行,取其才識明於吏治、達於教化者充選;其有不分曲直、罔辨是非者,或黜之厘務,或退守舊資。如此,則官得
 其人,事無不治。」上頗嘉其好古。歷度支、鹽鐵判官,遷太常博士、京西轉運使。丁內艱,時真宗將幸洛,以供億務繁,起令蒞職。史尋卒,上復詔權奪。



 大中祥符中,使契丹還,改工部員外郎、廣南西路轉運使。就拜起居郎,改陜西轉運使,賜金紫。陳堯咨知永興,好以氣凌黃目,因表求解職,不許。堯咨多縱恣不法,有密言其事者,詔黃目察之,得實以聞,堯咨坐罷龍圖閣職,徙知鄧州。八年,黃目入判三司三勾院。天禧初,馬元方奏黃目職事不舉,
 遂分三勾院,以三人掌之。黃目罷任,奉朝請。逾月,拜兵部員外郎、知制誥,充會靈觀判官。黃目屬辭淹緩,朝議以為不稱職。時以盛度知京府,辭不拜,即遷黃目右諫議大夫、權知開封府,度為會靈觀判官,兩換其任。



 仁宗升儲,拜給事中兼左庶子。入內副都知張繼能,嘗以公事請托黃目,至是未申謝,事敗,降左諫議大夫、知荊南府。明年,復為給事中,徙潭州。長沙月給,減於荊渚,特詔增之,又諭以兵賦繁綜寄任之意。五年,代還,知審官
 院。黃目以風疾題品乖當,改知通進、銀臺司兼門下封駁事。數月,求外任,得知亳州。俄而幼子死,聞訃慟絕,所疾加甚,卒,年五十六。錄其子理國為衛尉寺丞,定國為大理評事。



 黃目面柔簡默,為吏處劇,亦無敗事。有集五十卷,又撰《學海搜奇錄》四十卷,《聖朝郡國志》二十卷。黃目兄黃裳,弟黃庭,黃裳孫滋,並進士及第。黃裳、黃庭皆至太常博士。



 柴成務,字寶臣,曹州濟陰人也。父自牧,舉進士,能詩,至
 兵部員外郎。成務乾德中京府拔解,太宗素知其名,首薦之,遂中進士甲科,解褐陜州軍事推官。改曹、單觀察推官,遷大理寺丞。太平興國五年,轉太常丞,充陜西轉運副使,賜緋,再遷殿中侍御史。八年,與供奉官葛彥恭使河南,案行遙堤。歷知果、蘇二州,就為兩浙轉運使,改戶部員外郎、直史館,賜金紫。入為戶部判官、遷本曹郎中。太宗選郎官為少卿監,以成務為光祿少卿。



 俄奉使高麗,遠俗尚拘忌,以月日未利拜恩,稽留朝使。成務貽
 書,往反開諭大體,國人信服,事具《高麗傳》。淳化二年,為京東轉運使。會宋州河決,成務上言:「河水所經地肥澱,願免其租稅,勸民種藝。」從之。召拜司封郎中、知制誥,賜錢三十萬。時呂蒙正為宰相,嘗與之聯外姻,避嫌辭職,不許。俄與魏庠同知京朝官考課。四年,又與庠同知給事中事,凡制敕有所不便者,許封駁以聞。



 蜀寇平,使峽路安撫,改左諫議大夫、知河中府。時銀、夏未寧,蒲津當饋挽之沖,事皆辦集,得脫戶八百家以附籍。府城街陌
 頗隘狹,成務曰:「國家承平已久,如車駕臨幸,何以駐千乘萬騎邪?」乃奏撤民廬以廣之。其後祀汾陰,果留蹕河中,衢路顯敞,咸以為便。



 真宗即位,遷給事中、知梓州。未幾代還,又遣知青州,表求俟永熙陵復土畢之任。旋受詔與錢若水等同修《太宗實錄》,書成,知揚州。入判尚書刑部,本司小吏倨慢,成務怒而笞之,吏擊登聞鼓訴冤,有詔問狀。成務嘆曰:「忝為長官,杖一胥而被劾,何面目據堂決事邪!」乃求解職。景德初,卒,年七十一。



 成務有詞
 學,博聞稽古,善談論,好諧笑,士人重其文雅。然為郡乏廉稱,時論惜之。文集二十卷。成務年六十六始有子,比卒,裁六歲,授奉禮郎,名貽範,後為國子博士。



 論曰:泌述唐、漢之治,臺符陳商、周之鑒,歷布腹心,奏議反復論當世事,盡言無隱。何建五議,綸摭十事,皆切於輔治。何勤接士類,綸樂於薦士,皆足以儀表當世者也。去華頗尚氣節,而能作成後進;黃目屬辭淹緩,而著述浩瀚;成務寡清白之操,而專對不辱,俱有足稱者焉。



\end{pinyinscope}