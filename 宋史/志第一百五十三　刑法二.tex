\article{志第一百五十三 刑法二}

\begin{pinyinscope}

 律令者,有司之所守也。太祖以來,其所自斷,則輕重取舍,有法外之意焉。然其末流之弊,專用己私以亂祖宗之成憲者多矣。



 乾德伐蜀之役,有軍大校割民妻乳而
 殺之,太祖召至闕,數其罪。近臣營救頗切,帝曰:「朕興師伐罪,婦人何辜,而殘忍至此!」遂斬之。



 時郡縣吏承五季之習,黷貨厲民,故尤嚴貪墨之罪。開寶四年,王元吉守英州,月餘,受贓七十餘萬,帝以嶺表初平,欲懲掊克之吏,特詔棄市。陜州民範義超,周顯德中,以私怨殺同里常古真家十二口,古真小子留留幸脫走,至是,擒義超訴有司。陜州奏引赦當原,帝曰:「豈有殺一家十二人可以赦論邪?」命正其罪。八年,有司言:「自三年至今,詔所貸
 死罪凡四千一百八人。」帝注意刑闢,哀矜無辜,嘗嘆曰:「堯、舜之時,四兇之罪止於投竄。先王用刑,蓋不獲已,何近代憲綱之密耶!」故自開寶以來,犯大闢,非情理深害者,多得貸死。



 太平興國六年,自春涉夏不雨,太宗意獄訟冤濫。會歸德節度推官李承信因市蔥笞園戶,病創死。帝聞之,坐承信棄市。



 初,太祖嘗決系囚,多得寬貸。而開封婦人殺其夫前室子,當徒二年,帝以其兇虐殘忍,特處死。至是,有涇州安定婦人,怒夫前妻之子婦,絕其
 吭而殺之。乃下詔曰:「自今繼母殺傷夫前妻子,及姑殺婦者,同凡人論。」雍熙元年,開封寡婦劉使婢詣府,訴其夫前室子王元吉毒己將死。右軍巡推不得實,移左軍巡掠治,元吉自誣伏。俄劉死。及府中慮囚,移司錄司案問,頗得其侵誣之狀,累月未決。府白於上,以其毒無顯狀,令免死,決徒。元吉妻張擊登聞鼓稱冤,帝召問張,盡得其狀。立遣中使捕元推官吏,御史鞫問,乃劉有奸狀,慚悸成疾,懼其子發覺而誣之。推官及左、右軍巡使等
 削任降秩;醫工詐稱被毒,劉母弟欺隱王氏財物及推吏受贓者,並流海島;餘決罰有差。司錄主吏賞緡錢,賜束帛。初,元吉之系,左軍巡卒系縛搒治,謂之「鼠彈箏」,極其慘毒。帝令以其法縛獄卒,宛轉號叫求速死。及解縛,兩手良久不能動。帝謂宰相曰:「京邑之內,乃復冤酷如此,況四方乎?」



 端拱間,虜犯邊郡,北面部署言:「文安、大城二縣監軍段重誨等棄城遁,請論以軍法。」帝遣中使就斬之。既行,謂曰:「此得非所管州軍召之邪?往訊之乃決。」
 使至,果訊得乾寧牒令部送民入居城,非擅離所部,遽釋之。



 咸平間,有三司軍將趙永昌者,素兇暴,督運江南,多為奸贓。知饒州韓昌齡廉得其狀,乃移轉運使馮亮,坐決杖停職。遂撾登聞鼓,訟昌齡與亮訕謗朝政,仍偽刻印,作亮等求解之狀。真宗察其詐,於便殿自臨訊,永昌屈伏,遂斬之,釋亮不問,而昌齡以他事貶郢州團練副使。曹州民蘇莊蓄兵器,匿亡命,豪奪民產,積贓計四十萬。御史臺請籍其家,帝曰:「暴橫之民,國有常法,籍之,斯
 過也。」論如律。其縱舍輕重,必當於義,多類此。



 凡歲饑,強民相率持杖劫人倉廩,法應棄市,每具獄上聞,輒貸其死。真宗時,蔡州民三百一十八人有罪,皆當死。知州張榮、推官江嗣宗議取為首者杖脊,餘悉論杖罪。帝下詔褒之。遣使巡撫諸道,因諭之曰:「平民艱食,強取餱糧以圖活命爾,不可從盜法科之。」天聖初,有司嘗奏盜劫米傷主,仁宗曰:「饑劫米可哀,盜傷主可疾。雖然,無知迫於食不足耳。」命貸之。五年,陜西旱,因詔:「民劫倉廩,非傷主
 者減死,刺隸他州,非首謀又減一等。」自是,諸路災傷即降敕,饑民為盜,多蒙矜減,賴以全活者甚眾。司馬光時知諫院,言曰:「臣聞敕下京東、西災傷州軍,如貧戶以饑偷盜觔斗因而盜財者,與減等斷放,臣竊以為非便。《周禮》荒政十有二,散利、薄征、緩刑、弛力、舍禁、去幾,率皆推寬大之恩以利於民,獨於盜賊,愈更嚴急。蓋以饑饉之歲,盜賊必多,殘害良民,不可不除。頃年嘗見州縣官吏,有不知治體,務為小仁。遇兇年,劫盜觔斗,輒寬縱之,則
 盜賊公行,更相劫奪,鄉村大擾,不免廣有收捕,重加刑闢,或死或流,然後稍定。今若朝廷明降敕文,豫言與減等斷放,是勸民為盜也。百姓乏食,當輕徭薄賦、開倉振貸以救其死,不當使之自相劫奪。今歲府界、京東、京西水災極多,嚴刑峻法以除盜賊,猶恐春冬之交饑民嘯聚,不可禁禦,又況降敕以勸之。臣恐國家始於寬仁,而終於酷暴,意在活人而殺人更多也。」事報聞。



 帝嘗御邇英閣經筵,講《周禮》「大荒大札,薄征緩刑」。楊安國曰:「緩刑
 者,乃過誤之民耳,當歲歉則赦之,憫其窮也。今眾持兵杖劫糧廩,一切寬之,恐不足以禁奸。」帝曰:「不然,天下皆吾赤子也。一遇饑饉,州縣不能振恤,饑莩所迫,遂至為盜,又捕而殺之,不亦甚乎?」



 仁宗聽斷,尤以忠厚為主。隴安縣民誣平民五人為劫盜,尉悉執之,一人掠死,四人遂引服。其家辨於州,州不為理,悉論死。未幾,秦州捕得真盜,隴州吏當坐法而會赦,帝怒,特貶知州孫濟為雷州參軍,餘皆除名流嶺南。賜錢粟五家,復其役三年。因
 下詔戒敕州縣。廣州司理參軍陳仲約誤入人死,有司當仲約公罪應贖。帝謂審刑院張揆曰:「死者不可復生,而獄吏雖廢,復得敘官。」命特治之,會赦勿敘用。尚書比部員外郎師仲說請老,自言恩得任子,帝以仲說嘗失入人死罪,不與。其重人命如此。



 時近臣有罪,多不下吏劾實,不付有司議法。諫官王贄言:「情有輕重,理分故失,而一切出於聖斷,前後差異,有傷政體,刑法之官安所用哉?請自今悉付有司正以法。」詔可。近臣間有干請,
 輒為言官所斥。諫官陳升之嘗言:「有司斷獄,或事連權幸,多以中旨釋之。請有緣中旨得釋者,劾其干請之罪,以違制論。」許之。仁宗於賞罰無所私,尤不以貴近廢法。屢戒有司:「被內降者,執奏,毋輒行。」未嘗屈法以自徇也。知虢州周日宣詭奏水災,有司論請如上書不實法。帝曰:「州郡多言符瑞,至水旱之災,或抑而不聞。今守臣自陳墊溺官私廬舍,意實在民,何可加罪?」



 英宗在位日淺,於政令未及有所更制。然以吏習平安,慢於奉法,稍欲振
 起其怠惰。三班奉職和欽貸所部綱錢,至絞,帝命貸死免杖,刺隸福建路牢城。知審刑院盧士宗請稍寬其罪,帝曰:「刑故而得寬,則死者滋眾,非『刑期無刑』之道。俟有過誤,貸無傷也。」富國倉監官受米濕惡,壞十八萬石,會恩當減,帝特命奪官停之。



 熙寧二年,內殿崇班鄭從易母、兄俱亡於嶺外,歲餘方知,請行服。神宗曰:「父母在遠,當朝夕為念。經時無安否之問,以至逾年不知存亡邪?」特除名勒停。四年,王存立言:「嘉祐中,同學究出身,為碭
 山縣尉,嘗納官贖父配隸罪,請同舉人法,得免丁徭。」帝憫之,復賜出身,仍與注官。九年,知桂州沉起欲經略交址,取其慈恩州,交人遂破欽,犯邕管。詔邊人橫遭屠戮,職其致寇,罪悉在起,特削官爵,編置遠惡州。



 復仇,後世無法。仁宗時,單州民劉玉父為王德毆死,德更赦,玉私殺德以復父仇。帝義之,決杖、編管。元豐元年,青州民王贇父為人毆死,贇幼,未能復仇。幾冠,刺仇,斷支首祭父墓,自首。論當斬。帝以殺仇祭父,又自歸罪,其情可矜,詔
 貸死,刺配鄰州。宣州民葉元有同居兄亂其妻,縊殺之,又殺兄子,強其父與嫂為約契不訟。鄰里發其事,州為上請,帝曰:「罪人以死,奸亂之事特出葉元之口,不足以定罪。且下民雖無知,固宜哀矜,然以妻子之愛,既罔其父,又殺其兄,戕其侄,逆理敗倫,宜以毆兄至死律論。」



 紹聖以來,連起黨獄,忠良屏斥,國以空虛。徽宗嗣位,外事耳目之玩,內窮聲色之欲,徵發亡度,號令靡常。於是蔡京、王黼之屬,得以誣上行私,變亂法制。崇寧五年,詔曰:「
 出令制法,重輕予奪在上。比降特旨處分,而三省引用敕令,以為妨礙,沮抑不行,是以有司之常守,格人主之威福。夫擅殺生之謂王,能利害之謂王,何格令之有?臣強之漸,不可不戒。自今應有特旨處分,間有利害,明具論奏,虛心以聽。如或以常法沮格不行,以大不恭論。」明年,詔:「凡御筆斷罪,不許詣尚書省陳訴。如違,並以違御筆論。」又定令:「凡應承受御筆官府,稽滯一時杖一百,一日徒二年,二日加一等,罪止流三千里,三日以大不恭
 論。」由是吏因緣為奸,用法巧文浸深,無復祖宗忠厚之志。窮極奢侈,以竭民力,自速禍機。靖康雖知悔悟,稍誅奸惡,而謀國匪人,終亦末如之何矣。



 高宗性仁柔,其於用法,每從寬厚,罪有過貸,而未嘗過殺。知常州周□巳擅殺人,帝曰:「朕日親聽斷,豈不能任情誅僇,顧非理耳。」即命削杞籍。大理率以儒臣用法平允者為之。獄官入對,即以慘酷為戒。臺臣、士曹有所平反,輒與之轉官。每臨軒慮囚,未嘗有送下者,曰:「吾恐有司觀望,鍛煉以為重
 輕也。」吏部員外郎劉大中奉使江南回,遷左司諫,帝尋以為秘書少監。謂宰臣朱勝非曰:「大中奉使,頗多興獄,今使為諫官,恐四方觀望耳。」其用心忠厚如此。後詔用刑慘酷責降之人,勿堂除及親民,止與遠小監當差遣。



 當建、紹間,天下盜起,往往攻城屠邑,至興師以討之,然得貸亦眾。同知樞密院事李回嘗奏強盜之數,帝曰:「皆吾赤子也,豈可一一誅之?誅其渠魁三兩人足矣。」至待貪吏則極嚴:應受贓者,不許堂除及親民;犯枉法自盜
 者,籍其名中書,罪至徒即不敘,至死者,籍其貲。諸文臣寄祿官並帶「左」、「右」字,贓罪人則去之。是年,申嚴真決贓吏法。令三省取具祖宗故事,有以舊法棄市事上者,帝曰:「何至爾耶?但斷遣之足矣。貪吏害民,雜用刑威,有不得已,然豈忍寘縉紳於死地邪?」



 在徽宗時,刑法已峻。雖嘗裁定笞、杖之制,而有司猶從重。比中興之初,詔用政和遞減法,自是迄嘉定不易。自蔡京當國,凡所請御筆以壞正法者,悉厘正之。諸獄具,令當職官依式檢校。枷
 以干木為之,輕重長短刻識其上,笞、杖不得留節目,亦不得釘飾及加筋膠之類,仍用官給火印。暑月,每五日一洗濯枷杻,刑、寺輪官一員,躬親監視。諸獄司並旬申禁狀,品官、命婦在禁,別具單狀。合奏案者,具情款招伏奏聞,法司朱書檢坐條例、推司錄問、檢法官吏姓名於後。



 各州每年開收編配羈管奴婢人及斷過編配之數,各置籍。各路提點刑獄司,歲具本路州軍斷過大闢申刑部,諸州申提刑司。其應書禁歷而不書,應申所屬而
 不申,奏案不依式,檢坐開具違令,回報不圓致妨詳覆,與提刑司詳覆大闢而稽留、失覆大闢致罪有出入者,各抵罪。知州兼統兵者,非出師臨陳,毋用重刑。州縣月具系囚存亡之數申提刑司,歲終比較,死囚最多者,當職官黜責,其最少者,褒賞之。



 舊以絹計贓者,千三百為一匹,竊盜至二貫者徒。至是,又加優減,以二千為一匹,盜至三貫者徒一年。三年,復詔以三千為一匹,竊盜及凡以錢定罪,遞增五分。四年,又詔:「特旨處死,情法不
 當者,許大理寺奏審。」



 五年,歲終比較,宣州、衢州、福州無病死囚,當職官各轉一官。舒州病死及一分,惠州二分六厘,當職官各降一官。六年,令刑部體量公事,邵州、廣州、高州勘命官淹系至久不報,詔知州降一官,當職官展二年磨勘,當行吏永不收敘。德慶府勘封川縣令事,七月不報,詔知州、勘官各抵罪。九年,大理寺朱伯文廣西催斷刑獄,還言:「雷州海賊兩獄,並系平人七人,內五人已死。」帝惻然,詔本路提刑以下重致罰。十二年,御史
 臺點檢錢塘、仁和縣獄具,錢塘大杖,一多五錢半;仁和枷,一多一斤,一輕半斤,詔縣官各降一官。十三年,詔:「禁囚無供飯者,臨安日支錢二十文,外路十五文。」十六年,詔:「諸鞫獄追到乾證人,無罪遣還者,每程給米一升半,錢十五文。」二十一年,詔官支病囚藥物錢。



 舊法,刑部郎官四人,分左、右廳,或以詳覆,或以敘審,同僚而異事,有防閑考覆之意。南渡以來,務從簡省,大理少卿止一員,刑部郎中初無分異,獄有不得其情,法有不當於理者,
 無所平反追改。二十六年,右司郎中汪應辰言之。詔刑部郎官依元豐法,分左、右廳治事。二十七年,詔四川以錢引科罪者,準銅錢。



 孝宗究心庶獄,每歲臨軒慮囚,率先數日令有司進款案披閱,然後決遣。法司更定律令,必親為訂正之。丞相趙雄上《淳熙條法事類》,帝讀至收騾馬、舟船、契書稅,曰:「恐後世有算及舟車之譏。」《戶令》:「戶絕之家,許給其家三千貫,及二萬貫者取旨。」帝曰:「其家不幸而絕,及二萬貫乃取之,是有心利其財也。」又《捕亡
 律》:「公人不獲盜者,罰金。」帝曰:「罰金而不加罪,是使之受財縱盜也。」又:「監司、知州無額上供者賞。」帝曰:「上供既無額,是白取於民也,可賞以誘之乎?」並令削去之。其明審如此。且於用刑,未嘗以私廢法。鎮江都統戚方以刻剝被罪,宰臣陳俊卿言內臣有主之者,帝曰:「朕亦聞之。」乃以內侍陳瑜、李宗回等付大理獄,究其賂狀,獄成,決配之。乾道二年,下詔曰:「獄,重事也。用法一傾,則民無所措手足。比年以來,治獄之吏,巧持多端,隨意輕重之,朕甚
 患焉。其自今革玩習之弊,明審克之公,使奸不容情,罰必當罪,用迪於刑之中,勉之哉,毋忽!」三年,詔曰:「獄,重事也。稽者有律,當者有比,疑者有讞。比年顧以獄情白於執政,探取旨意,以為輕重,甚亡謂也。自今其祗乃心,敬於刑,惟當為貴,毋習前非。不如吾詔,吾將大寘於罰,罔攸赦。」六年,詔:「以絹計贓者,更增一貫。以四千為一匹。」議者又言:「犯盜,以敕計錢定罪,以律計絹。今律以絹定罪者遞增一千,敕內以錢定罪,亦合例增一千。」從之。



 臨安
 府左右司理、府院三獄,杖直獄子以無所給,至為無籍。七年,詔:「人月給錢十貫,米六斗,每院止許置一十二人。」時州縣獄禁淹延,八年,詔:「徒以上罪入禁三月者,提刑司類申刑部,置籍立限以督之。」其後,又詔中書置禁,奏取會籍,大臣按閱,以察刑寺稽違,與夫不應問難而問難,不應會而會者。



 淳熙初,浙西提刑鄭興裔上《檢驗格目》,詔頒之諸路提刑司。凡檢覆必給三本:一申所屬,一申本司,一給被害之家。紹興法,鞫獄官推勘不得實,故
 有不當者,一案坐之。乾道法,又恐有移替事故者,即致淹延,乃令先決罪人不當,官吏案後收坐。至是,所司請更定死罪依紹興法,餘依乾道施行,從之。其後,有司以覆勘不同,則前官有失入之罪,往往雷同前勘。帝知其弊,十四年,詔特免一案推結一次。於是小大之獄,多得其情。二廣州軍獄吏,畏憲司點檢送勘之害,凡有重囚,多斃於獄。臣僚以為請,乃詔二廣提刑司詳覆公事,若小節不完,不須追逮獄吏,委本州究實保明。遇有死者,
 必根究其所以致死。



 三衙及江上諸軍,各有推獄,謂之「後司」。獄成,決於主帥,不經屬官,故軍吏多受財為奸。光宗時,乃詔通曉條制屬官兼管之。廣東路瘴癘,惟英德府為最甚,謂之「人間生地獄」。諸司公事欲速成者,多送之,自非死罪,至即誣伏,亟就刑責以出。五年,臣僚言之,詔本路諸司公事應送別州者,無送英德府。



 至寧宗時,刑獄滋濫。嘉泰初,天下上死案,一全年千八百一十一人,而斷死者才一百八十一人,餘皆貸之。乃詔諸憲臺,
 歲終檢舉州軍有獄空並禁人少者,申省取旨。嘉定四年,詔以絹計贓定罪者,江北鐵錢依四川法,二當銅錢一。江西提刑徐似道言:「檢驗官指輕作重,以有為無,差訛交互,以故吏奸出入人罪。乞以湖南正背人形隨《格目》給下,令於傷損去處,依樣朱紅書畫,唱喝傷痕,眾無異詞,然後署押。」詔從之,頒之天下。五年,詔三衙及江上、四川諸軍,以武舉人主管後司公事。



 理宗起自民間,具知刑獄之弊。初即位,即詔天下恤刑,又親制《審刑銘》以
 警有位。每歲大暑,必臨軒慮囚。自謀殺、故殺、鬥殺已殺人者,偽造符印、會子,放火,官員犯入己贓,將校軍人犯枉法外,自餘死罪,情輕者降從流,流降從徒,徒從杖,杖已下釋之。大寒慮囚,及祈晴祈雪及災祥,亦如之。有一歲凡數疏決者。後以建康亦先朝駐蹕之地,罪人亦得視臨安減降之法。帝之用刑可謂極厚矣,而天下之獄不勝其酷。每歲冬夏,詔提刑行郡決囚,提刑憚行,悉委倅貳,倅貳不行,復委幕屬。所委之人,類皆肆行威福,以
 要饋遺。監司、郡守,擅作威福,意所欲黥,則令入其當黥之由,意所欲殺,則令證其當死之罪,呼喝吏卒,嚴限日時,監勒招承,催促結款。而又擅置獄具,非法殘民,或斷薪為杖,掊擊手足,名曰:「掉柴」;或木索並施,夾兩脰,名曰「夾幫」;或纏繩於首,加以木楔,名曰「腦箍」;或反縛跪地,短豎堅木,交辮兩股,令獄卒跳躍於上,謂之「超棍」,痛深骨髓,幾於殞命。富貴之家,稍有罥偏,動籍其貲。又以趁辦月樁及添助版帳為名,不問罪之輕重,並從科罰。大率
 官取其十,吏漁其百。



 諸重刑,皆申提刑司詳覆,或具案奏裁,即無州縣專殺之理,往往殺之而待罪。法無拘鎖之條,特州縣一時彈壓盜賊奸暴,罪不至配者,故拘鎖之,俾之省愆。或一月、兩月,或一季、半年,雖永鎖者亦有期限,有口食。是時,州縣殘忍,拘鎖者竟無限日,不支口食,淹滯囚系,死而後已。又以己私摧折手足,拘鎖尉砦。亦有豪強賂吏,羅織平民而囚殺之。甚至戶婚詞訟,亦皆收禁。有飲食不充,饑餓而死者;有無力請求,吏卒凌
 虐而死者;有為兩詞賂遺,苦楚而死者。懼其發覺,先以病申,名曰「監醫」,實則已死;名曰「病死」,實則殺之。至度宗時,雖累詔切責而禁止之,終莫能勝,而國亡矣。



 詔獄,本以糾大奸慝,故其事不常見。初,群臣犯法,體大者多下御史臺獄,小則開封府、大理寺鞫治焉。神宗以來,凡一時承詔置推者,謂之「制勘院」,事出中書,則曰「推勘院」,獄已乃罷。



 熙寧二年,命尚書都官郎中沈衡鞫前知杭州祖無擇於秀州,內侍乘驛追逮。御史張戩等言:「
 無擇三朝近侍,而驟系囹圄,非朝廷以廉恥風厲臣下之意,請免其就獄,止就審問。」不從。又命崇文院校書張載鞫前知明州、光祿卿苗振於越州。獄成,無擇坐貸官錢及借公使酒,謫忠正軍節度副使;振坐故入裴士堯罪及所為不法,謫復州團練副使。獄半年乃決,辭所連逮官吏,坐勒停、沖替、編管又十餘人,皆御史王子韶啟其事。自是詔獄屢興,其悖於法及國體所系者著之,其餘不足紀也。



 八年,沂州民朱唐告前餘姚主簿李逢謀
 反。提點刑獄王庭筠言其無跡,但謗讟,語涉指斥及妄說休咎,請編配。帝疑之,遣御史臺推直官蹇周輔劾治。中書以庭筠所奏不當,並劾之。庭筠懼,自縊死。逢辭連宗室秀州團練使世居、醫官劉育等、河中府觀察推官徐革,詔捕系臺獄,命中丞鄧綰、同知諫院範百祿與御史徐禧雜治。獄具,賜世居死,李逢、劉育及徐革並凌遲處死,將作監主簿張靖、武進士郝士宣皆腰斬,司天監學生秦彪、百姓李士寧杖脊,並湖南編管。餘連逮者追
 官落職。世居子孫貸死除名,削屬籍。舊勘鞫官吏並劾罪。



 李士寧者,挾術出入貴人門,常見世居母康,以仁宗御制詩上之。百祿謂士寧熒惑世居致不軌,且疑知其逆謀,推問不服。禧乃奏:「士寧贈詩,實仁宗御制,今獄官以為反因,臣不敢同。」百祿以士寧嘗與王安石善,欲鍛煉附致妖言死罪,卒論士寧徒罪,而奏「禧故出之,以媚大臣」。詔詳劾理曲者以聞。百祿坐報上不實,落職。若凌遲、腰斬之法,熙寧以前未嘗用於元兇巨蠹,而自是以
 口語狂悖致罪者,麗於極法矣。蓋詔獄之興,始由柄國之臣藉此以威縉紳,逞其私憾,朋黨之禍遂起,流毒不已。



 紹聖間,章惇、蔡卞用事,既再追貶呂公著、司馬光,及謫呂大防等嶺外,意猶未快,仍用黃履疏高士京狀追貶王珪,皆誣以「圖危上躬」,其言浸及宣仁,上頗惑之。最後,起同文館獄,將悉誅元祐舊臣。時太府寺主簿蔡渭奏:「臣叔父碩,嘗於邢恕處見文及甫元祐中所寄恕書,具述奸臣大逆不道之謀。及甫,彥博子也,必知奸狀。」詔
 翰林承旨蔡京、吏部侍郎安惇同究問。初,及甫與恕書,自謂:「畢禫當求外,入朝之計未可必,聞已逆為機阱,以榛塞其途。」又謂:「司馬昭之心,路人所知。」又云:「濟之以粉昆,朋類錯立,欲以眇躬為甘心快意之地。」及甫嘗語蔡碩,謂司馬昭指劉摯,粉昆指韓忠彥,眇躬,及甫自謂。蓋俗稱駙馬都尉為「粉侯」,人以王師約故,呼其父克臣為「粉父」,忠彥乃嘉彥之兄也。及甫除都司,為劉摯論列。又摯嘗論彥博不可除三省長官,故止為平章重事。及彥
 博致仕,及甫自權侍郎以修撰守郡,母喪除,與恕書請補外,因為躁忿詆毀之辭。及置對,則以昭比摯如舊,眇躬乃以指上,而粉昆乃謂指王巖叟面如傅粉,故曰「粉」,梁燾字況之,以「況」為兄,故曰「昆」,斥摯將謀廢立,不利於上躬。京、惇言:「事涉不順,及甫止聞其父言,無他證佐,望別差官審問。」乃詔中書舍人蹇序辰審問,仍差內侍一員同往。蔡京、安惇等共治之,將大有所誅戮,然卒不得其要領。會星變,上怒稍息,然京、惇極力鍛煉不少置。既
 而梁燾卒於化州,劉摯卒於新州,眾皆疑二人不得其死。明年五月,詔:「摯、燾據文及甫等所供言語,偶逐人皆亡,不及考驗,明正典刑。摯、燾諸子並勒停,永不收敘。」先時,三省進呈,帝曰:「摯等已謫遐方,朕遵祖宗遺志,未嘗殺戮大臣,其釋勿治。」



 初,元祐更政,嘗置訴理所,申理冤濫。元符元年,中丞安惇言:「神宗厲精圖治,明審庶獄,而陛下未親政時,奸臣置訴理所,凡得罪熙寧、元豐之間者,咸為除雪,歸怨先朝,收恩私室。乞取公案,看詳從初
 加罪之意,復依元斷施行。」時章惇猶豫未應,蔡卞即以「相公二心」之言迫之。惇懼,即日置局,命蹇序辰同安惇看詳案內文狀陳述,及訴理所看詳於先朝言語不順者,具名以聞。自是,以伸雪復改正重得罪者八百三十家。



 及徽宗即位,改正元祐訴理之人。右正言陳瓘言:「訴理得罪,自語言不順之外,改正者七百餘人。無罪者既蒙昭雪,則看詳之官如蹇序辰、安惇者,安可以不加罪乎?序辰與惇受大臣諷諭,迎合紹述之意,因謂訴理之
 事,形跡先朝,遂使紛紛不已。考之公議,宜正典刑。」會中書省亦請治惇、序辰罪,詔蹇序辰、安惇並除名、放歸田里。



 靖康初元,既戮梁方平,太傅王黼責授崇信軍節度副使、永州安置。言者論黼欺君罔上,專權怙寵,蠹財害民,壞法敗國,朔方之釁,黼主其謀。遣吏追至雍丘殺之,取其首以獻,仍籍其家。又詔賜拱衛大夫、安德軍承宣使李彥死。彥根括民田,奪民常產,重斂租課,百姓失業,愁怨溢路,官吏稍忤意,捃摭送獄,多至憤死,故特誅之。
 暴少保梁師成朋比王黼之罪,責彰化軍節度副使,行一日,追殺之。臺諫極論朱勉肆行奸惡,起花石綱,竭百姓膏血,罄州縣帑藏,子侄承宣、觀察者數人,廝役為橫行,媵妾有封號,園第器用悉擬宮禁。三月,竄勉廣南,尋賜死。趙良嗣者,本燕人馬植。政和初,童貫使遼國,植邀於路,說以覆宗國之策,貫挾之以歸,卒用其計,以基南北之禍。至是,伏誅。七月,暴童貫十罪,遣人即所至斬之。九月,言者論蔡攸興燕山之役,禍及天下,驕奢淫佚,載
 籍所無。詔誅攸並弟翛。



 高宗承大亂之後,治王時雍等賣國之罪,洪芻、余大均、陳沖、張卿才、李彞、王及之、周懿文、胡思文並下御史臺獄。獄具,刑寺論芻納景王寵姬,大均納喬貴妃侍兒,及之苦辱寧德皇后女弟,當流;沖括金銀自盜,與宮人飲,當絞;懿文、卿才、彞與宮人飲,卿才、彞當徒,懿文當杖;思文於推擇張邦昌狀內添諂奉之詞,罰銅十斤:並該赦。上閱狀大怒,李綱等共解之,上亦新政,重於殺士大夫,乃詔芻、大均、沖各特貸命、流沙
 門島,永不放還;卿才、彞、及之、懿文、思文並以別駕安置邊郡。宋齊愈下臺獄,法寺以犯在五月一日赦前,奏裁。詔齊愈謀立異姓,以危宗社,非受偽命臣僚之比,特不赦,腰斬都市。詔東京及行在官擅離任者,並就本處根勘之。淮寧守趙子崧,靖康末,傅檄四方,語頗不遜。二年,詔御史置獄京囗鞫之。情得,帝不欲暴其罪,以棄鎮江罪貶南雄州。



 建炎三年四月,苗傅等疾閹宦恣橫,及聞王淵為樞密,愈不平,乃與王世修謀逆。詔御史捕世修
 鞫之,斬於市。七月,韓世忠執苗傅等,磔之建康。統制王德擅殺軍將陳彥章,臺鞫當死,帝以其有戰功,特貸之。慶遠軍節度使範瓊領兵入見,面對不遜。知樞密院張浚奏瓊大逆不道,付大理寺鞫之,獄具,賜死。越州守郭仲荀,寇至棄城遁,過行在不朝。付御史臺、大理寺雜治,貶廣州。神武軍統制魯玨坐賊殺不辜,掠良家子女,帝以其有戰功,貸之,貶瑞州。



 紹興元年,監察御史婁寅亮陳宗社大計,秦檜惡之。十一月,使言者論其父死匿不
 舉哀,下大理寺劾治,迄無所得,詔免所居官。十一年,樞密使張俊使人誣張憲,謂收岳飛文字謀為變。秦檜欲乘此誅飛,命萬俟卨鍛煉成之。飛賜死,誅其子云及憲於市。汾州進士智浹上書訟飛冤,決杖、編管袁州。廣西帥胡舜陟與轉運使呂源有隙,源奏舜陟贓污僭擬,又以書抵檜,言舜陟訕笑朝政。檜素惡舜陟,遣大理官往治之。十三年六月,舜陟不服,死於獄。飛與舜陟死,檜權愈熾,屢興大獄以中異己者,名曰詔獄,實非詔旨也。其
 後所謂詔獄,紛紛類此,故不備錄云



\end{pinyinscope}