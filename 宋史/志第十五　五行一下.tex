\article{志第十五 五行一下}

\begin{pinyinscope}

 水下



 建隆三年春,延、寧二州雪盈尺,溝洫復冰,草木不華。丹州雪二尺。



 太平興國七年三月,宣州霜雪害桑稼。



 雍熙二年冬,南康軍大雨雪,江水冰,勝重載。



 端拱元年閏五
 月,鄆州風雪傷麥。



 淳化三年九月,京兆府大雪害苗稼。四年二月,商州大雪,民多凍死。



 咸平四年三月丁丑,京師及近畿諸州雪,損桑。



 天禧元年十一月,京師大雪,苦寒,人多凍死,路有殭尸,遣中使埋之四郊。二年正月,永州大雪,六晝夜方止,江、溪魚皆凍死。



 慶歷三年十二月丁巳,大雨雪。



 皇祐四年十二月己丑,雪。初,帝以愆亢,責躬減膳,每見輔臣,憂形於色。龐籍等因言:「臣等不能燮理陰陽,而上煩陛下責躬引咎,願守散秩以避賢路。」
 帝曰:「是朕誠不能感天而惠不能及民,非卿等之過也。」是夕,乃得雪。



 至和元年正月,京師大雪,貧弱之民凍死者甚眾。



 嘉祐元年正月甲寅朔,御大慶殿受朝。前一夕,殿庭設仗衛既具,而大雨雪折宮架。是日,帝因感風眩,促禮行而罷。壬午,大雨雪,泥途盡冰。都民寒餓,死者甚眾。



 元祐二年冬,京師大雪連月,至春不止。久陰恆寒,罷上元節游幸,降德音諸道。八年十一月,京師大雪,多流民。



 元符二年正月甲辰朔,御大慶殿受朝賀,以雪罷。



 政
 和三年十一月,大雨雪,連十餘日不止,平地八尺餘。冰滑,人馬不能行,詔百官乘轎入朝。飛鳥多死。七年十二月,大雪。詔收養內外乞丐老幼。



 靖康元年閏十一月,大雪,盈三尺不止。天地晦冥,或雪未下時,陰雲中有雪絲長數寸墮地。二年正月丁酉,大雪,天寒甚,地冰如鏡,行者不能定立。是月乙卯,車駕在青城,大雪數尺,人多凍死。



 建炎三年六月,寒。



 紹興元年二月寒食日,雪。五年二月乙巳,雨雪。六年二月癸卯,雪。十三年三月癸丑,雨雪。
 十七年二月丙申,雪。十八年二月癸卯,雪。二十八年三月丙寅,雨雪。二十九年二月戊戌,大雪。三十一年正月戊子,大雨雪,至於己亥,禁旅壘舍有壓者,寒甚。



 乾道元年二月,大雪。三月,暴寒,損苗稼。二年春,大雨,寒,至於三月,損蠶麥。二月丙申,雪。四年二月癸丑,大雪。五年二月戊子,雪。六年五月,大風雨,寒,傷稼。七年二月丙辰,雨雪。



 淳熙十二年,淮水冰,斷流。是冬,大雪。自十二月至明年正月,或雪,或霰,或雹,或雨水,冰冱尺餘,連日不解。臺州
 雪深丈餘,凍死者甚眾。十六年四月戊子,天水縣大雨雪,傷麥。



 紹熙元年三月,留寒至立夏不退。十二月,建寧府大雪深數尺。查源洞寇張海起,民避入山者多凍死。二年正月,行都大雪積冱,河冰厚尺餘,寒甚。是春,雷雪相繼,凍雨彌月。四年二月己未,雪。



 慶元五年二月庚午,雪。六年二月乙酉,雪。五月,亡暑,氣凜如秋。



 開禧三年二月戊申,雪。



 嘉定元年二月甲寅,雪。四年二月丙子,雪。六年二月丁亥,雪。六月,亡暑,夜寒。九年二月乙酉、丙申,雪。
 十年二月庚申、壬戌,雪。十七年三月癸丑,雪。



 寶慶元年四月辛卯,雪。



 紹定四年二月己巳,雨雪。六年三月壬子,雨雪。



 端平元年二月癸酉,雨雪。二年三月乙未,雨雪。



 嘉熙二年二月乙未,雨雪。



 淳祐六年二月壬申,雨雪。



 寶祐元年二月壬子,雨雪。二年三月戊子,雨雪。六年二月,雨雪。



 開慶元年二月庚辰,雨雪。



 景定五年二月辛亥,雨雪。



 建隆三年春,厭次縣隕霜殺桑,民不蠶。



 淳化三年三月,商州霜,花皆死。



 景德四年七月,渭州瓦亭砦早霜傷稼。



 大中祥符九年十二月,大名、澶、相州並霜,害稼。



 至和二年,河東自春隕霜殺桑。



 紹興七年二月庚申,霜殺桑稼。



 淳熙十六年七月,階、成、鳳、西和州霜,殺稼幾盡。



 紹熙三年九月丁未,和州隕霜連三日,殺稼。是月,淮西郡國稼皆傷。



 嘉熙元年三月,霜。



 建隆元年十月,臨清縣雨雹傷稼。二年七月,義川、雲巖二縣大雨雹。四年七月,海州風雹。



 乾德二年四月,陽武縣雨雹。宋州寧陵縣風雨雹傷民田。六月,潞州風雹。七
 月,同州合陽縣雨雹害稼。八月,膚施縣風雹霜害民田。三年四月,尉氏、扶溝二縣風雹,害民田,桑棗十損七八。



 開寶二年,風雹害夏苗。



 太平興國二年六月,景城縣雨雹。七月,永定縣大風雹害稼。五年四月,冠氏、安豐二縣風雹。七年五月,蕪湖縣雨雹傷稼。八年五月,相州風雹害民田。



 端拱元年三月,霸州大雨雹,殺麥苗。閏五月,潤州雨雹傷麥。



 淳化元年六月,許州大風雹,壞軍營、民舍千一百五十六區。魚臺縣風雹害稼。



 至道二年十一月,
 代州風雹傷田稼。



 咸平元年九月,定州北平等縣風雹傷稼。三年四月丁巳,京師雨雹,飛禽有隕者。六年四月甲申,京師暴雨雹,如彈丸。



 大中祥符三年丙申,京師雨雹。五年八月丙辰,京師雨雹。



 天禧元年九月,鎮戎軍彭城砦風雹,害民田八百餘畝。



 天聖元年五月丙辰,大雨雹。二年七月壬午,大雨雹。六年,京師雨雹。



 嘉祐四年四月丙戌,震雷雨雹。



 熙寧元年秋,鄜州雨雹。三年七月、七年四月五月,京師雨雹。八年夏,鄜州、涇州雨雹。九年二
 月,京師雨雹。十年夏,鄜州雨雹。秦州大雨雹。



 紹聖二年十月辛未,西南方有雷聲,次大雨雹。四年閏二月癸卯,京師雨雹,自辰至申。



 建中靖國元年二月丙申,京師雨雹。五月辛酉,京師大雨雹。



 崇寧三年十月辛丑,京師雨雹。



 大觀元年十月己巳、三年五月戊申,京師大雨雹。



 政和七年六月,京師大雨雹,皆如拳,或如一升器,幾兩時而止。



 宣和四年二月癸卯,京師雨雹。四年三月朔,雨雹。



 靖康元年十二月己卯、庚辰,京師雨雹。



 建炎三年八月
 甲戌,大雨雹。



 紹興元年二月壬辰,高宗在越州,雨雹震雷。二年二月丙子,臨安府大雨雹。三年正月,雨雹震雷。四年三月己未,大雨雹傷稼。五年閏月乙巳朔,雨雹而雪。十月丁未夜,秀州華亭縣大風電,雨雹,大如荔枝實,壞舟覆屋。十二月戊辰,雨雹。七年二月癸丑,雨雹。先一夕雷,後一日雪,癸丑又雹。八年六月丙辰,大雨雹。九年二月甲戌,雨雹傷麥。十二月辛未,雨雹。十年二月辛亥,大雨雹。十二月庚辰,雨雹。十一年正月辛酉,雨雹。十三
 年二月甲子,雨雹傷麥。五月戊午夜,雹。七月庚午、壬申,雹害稼。十一月己未,雨雹。十七年正月庚辰,雨雹;五月丙寅,又雹。二十一年三月己卯,雹傷禾麥。二十八年四月辛亥,雨雹。二十九年二月戊戌,雹損麥。



 隆興元年三月丙申夜,雨雹。二年二月丁丑,雹與霰俱。四月庚午,雹。六月,雨雹。七月丁未,雨雹。十月辛卯,雨雹。十二月己亥,雨雪而雹。閏月,雨雹。



 乾道元年二月庚寅夜,雹。二年十月辛卯,雨雹。三年二月壬午,雪;癸未,雹。四年正月癸未
 夜,雹,有霰。二月丁酉、癸丑,雨雹;乙卯,雹而雪。五年二月丙午,雹損麥;六年二月壬午,亦如之。八年七月壬辰,雨雹。



 淳熙三年四月丁亥,雨雹。癸巳,天臺、臨海二縣大風雹,傷麥。四年正月,建康府雨雹。五月丙寅,雨雹。五年,建康府雨雹者再。六年正月丁丑,雹傷麥。三月壬申夜,大雨雹。八年十二月甲寅,雨雹。十二年二月辛酉夜,雨雹。十三年閏月丙午,雨雹。十五年二月丁亥,雨雪而雹。六月丁卯,雨雹。十六年二月己卯,雹而雨。



 紹熙元年二月
 丙申,雪;丁酉,雹。二年正月戊寅,大雨雹,震雷電以雨,至二月庚辰,大雪連數日。是月庚寅朔,建寧府大風雨雹,僕屋殺人。三月癸酉,大風雨雹,大如桃李實,平地盈尺,壞廬舍五千餘家,禾麻、蔬果皆損;瑞安縣亦如之,壞屋殺人尤甚。秋,祐川縣大風雹,壞粟麥。



 慶元三年二月戊辰,雪;己巳,雹。四月乙丑,雨雹,大如杯,破瓦,殺燕爵。



 嘉泰元年三月丙寅,雨雹三日。五月丁丑,雨雹。七月癸亥,大雨而雹。二年四月庚寅,雨雹傷稼。六月庚子,大風雹而
 寒。四年正月壬辰,雪而雹。



 開禧二年正月己酉,雹而雷。



 嘉定元年閏月壬申,雨雹害稼。二年三月乙未,雨雹。六年夏,江、浙郡縣多雨雹害稼。十五年九月癸丑,大震雨雹。十六年秋,雨雹。



 紹定元年五月丁酉,雨雹。五年九月壬寅,雨雹。六年三月丙辰,大雨雹。



 端平二年五月乙未,雹。三年六月庚戌,雨雹。



 嘉熙元年二月壬辰,雨雹。



 淳祐二年四月壬申,雨雹。八年二月壬辰,雨雹。三月乙丑,雨雹。九年正月,雨雹。



 寶祐三年五月,嘉定府大雨雹。



 開慶
 元年五月辛亥,雨雹。



 景定元年二月庚申,雨雹。



 建隆四年四月癸巳,宿州晝日無雨,雷霆暴作,軍校傅韜震死。是夜夜半,雷起於京師。開封縣署役夫劉延嗣、萬進震死,頃之復蘇,有煙焰自牖入室,因駭僕,遍體焦灼。



 乾德二年正月辛巳,雷起京師西南,東行有電。五月戊寅,大名府大雨,雷震焚蒿聚。四年七月,海州雷震長吏廳,傷刺史梁彥超。



 開寶七年六月,易州雷,震死耀武軍士八人。八年八月,邛州延貴鎮震死民費貴及其子
 四人。



 太平興國二年七月,景城縣震死牛商馮異。



 端拱二年八月,興化軍民劉政震死,有文在胸曰「大不孝」。



 淳化三年七月,泗州大風雨,震僧伽塔柱。



 至道元年三月甲戌,雷未發聲,召司天監寺趙昭問之,答云:「按占書,雷不發聲,寬政之應也。」七月,泗州大風雨,雷震僧伽塔及壞鐘樓。



 咸平元年正月戊寅,京師西北有雷電。十一月,瀛州、順安軍並東北有雷。三年冬,黃州西北雷震,似盛夏時。十二月,真定府東南雷。四年十月乙巳,京師西南
 雷電。閏十二月,大名府雷。六年十一月甲午,京師暴雷震,司天言:「國家發號布德,未及黎庶。」時議改元肆赦,詔宰相增廣條目,採民病悉除之。



 景德三年九月丙寅夕,京師大震雷。



 大中祥符元年正月癸未,京師西北方雷。五年十二月己巳,京師西北雷電。九年五月,殿侍張信奉南海祝版乘驛至唐州,震死。



 嘉祐四年四月丙戌,大震雷,雨雹。



 慶歷六年五月,雷雹、地震。



 紹聖三年十月十五日,西南方有雷聲,次雨雹。



 大觀三年十月戊子,大雷
 雹而雨。



 建炎四年正月己未,雷。時御舟次溫州章安鎮,高宗謂宰臣曰:「雷聲甚厲,前史以為君弱臣強,四夷兵不制。」是夕,金人破明州。壬戌,又雷。



 紹興五年九月戊寅,雷。十月丁巳,雷。六年十月丙午,雷。九年九月甲午、十月丁卯,雷。十一年十一月己酉,雷。十五年十月辛卯、十二月甲寅,雷。十六年,溫州大雷電,震死六人於龍翔寺。十八年閏月甲戌,雷。十九年十月甲寅,雷。二十一年二月辛未,南安軍大雷電,大庾縣震死四人。十一月辛未夜,
 震雷。十二月癸酉,雷。二十二年十二月戊寅、己卯,雷。二十六年十二月甲子,雷。二十七年九月癸未,雷。三十一年正月丁丑,雷。



 乾道三年十一月丙寅,雷雨,不克郊。戊辰,日南至,大震雷。八年九月乙酉,雷。九年閏月癸卯,雷。



 淳熙九年九月壬午,雷。十二年十一月戊子,雷。十二月丁丑,雷。十三年正月己丑,雷;後三十五日,雪。十四年十一月乙卯,雷。十六年七月乙丑,大雷震太室齋殿東鴟吻。



 紹熙元年九月辛酉,雷。十一月壬午,日南至。郊祀,風
 雨大至,帝震恐,因致疾。四年十一月己卯,日南至;辛巳,雷。五年十月癸巳,大雷電。



 慶元二年正月戊子,雷。十一月,雷。三年十月癸亥,雷。六年九月己未,雷。



 嘉泰二年正月己巳,雷。三年正月,雷。四年正月辛卯,雷。



 開禧二年正月,雪、雷。九月,雷。三年十月辛未、癸酉,雷。



 嘉定二年九月戊子,雷。三年正月,雷。十月壬申,雷。八月辛丑、九月辛酉,雷。四年九月,雷。五年七月戊辰,雷雨震太室之鴟吻。十月丁酉,雷。六年閏月壬辰,雷震電;乙未昧爽,洊雷。七年
 九月癸亥,雷。八年九月丙寅,雷。十一年九月辛巳,祀明堂,肆赦,震雷。十四年十月庚午,雷。十五年九月癸丑,雷。十六年九月乙卯、十二月壬辰,雷。十七年九月丁亥,雷。



 寶慶二年九月庚申、十月辛丑,雷。



 紹定二年九月庚辰,雷。五年九月壬寅,雷。



 端平二年十二月辛亥,雷。三年九月庚午,雷。是月,祀明堂,大雨震電。十月戊戌,雷。



 嘉熙元年九月丁巳,雷。二年九月己酉、十月庚戌,雷。



 淳祐元年十二月丙寅,雷。二年九月己丑,雷。三年三月丙辰,雷。十
 年十一月壬午,雷。十二年十二月丁丑,雷。



 寶祐三年九月,雷。



 開慶元年十月乙酉,雷。



 景定二年十月戊戌,雷電;己亥,雷電。



 咸淳四年閏月丁巳、九月庚申,雷。九年十月癸亥、十二月丙辰、壬戌,雷。



 建炎七年五月,汴京無雲而雷。



 紹興三十年十月壬戌,晝漏半,無雲而雷;癸亥,日過中,無雲而雷。



 淳熙十四年六月甲申昧爽,禱雨太乙宮,乘輿未駕,有大聲自內發,及和寧門,人馬闢易相踐,有失巾屨者。



 至道元年十二月,廣州大魚擊海水而出。魚死,長六丈三尺,高丈餘。



 政和七年夏中,有二魚落殿中省廳屋上。



 宣和二年三月,內出魚,純赤色,蔡京等乞付史館,拜表賀。



 紹興十八年,漳浦縣崇照鹽場海岸連有巨魚,高數丈。割其肉數百車,剜目乃覺,轉鬣而傍艦皆覆。又漁人獲魚,長二丈餘,重數千斤,剖之,腹藏人骼,膚發如生。二十四年四月,海鹽縣海洋有巨鰍,群蝦從之,聲若謳歌。抵岸偃沙上,猶揚鬣撥刺,其高齊縣門。



 乾道六年,行都
 北闕有鯰魚,色黑,腹下出人手於兩傍,各具五指。七年十一月丁亥,洞庭湖巨黿走沙擁舟,身廣長皆丈餘,升舟,以首足壓重艦沒水。



 淳熙十三年二月庚申,錢塘龍山江岸有大魚如象,隨潮汐復逝。十六年六月甲辰,錢塘旁江居民得魚,備五色,鯽首鯉身。民詭言夢得魚,覺而在手猶躍,事聞,有司令縱之。



 慶元三年二月,饒州景德鎮漁人得魚,赬尾鯉鱗而首異常魚。鎮之老人言其不祥。紹興二年嘗出,後為水災。蓋是歲五月,鎮果大水,
 皆魚孽也。



 嘉定十七年,海壞畿縣鹽官地數十里。先是,有巨魚橫海岸,民臠食之,海患共六年而平。



 建隆元年七月,澶州蝗。二年五月,範縣蝗。三年七月,深州蝻蟲生。四年六月,澶、濮、曹、絳等州有蝗。七月,懷州蝗生。



 乾德二年四月,相州蝻蟲食桑。五月,昭慶縣有蝗,東西四十里,南北二十里。是時,河北、河南、陜西諸州有蝗。三年七月,諸路有蝗。



 開寶二年八月,冀、磁二州蝗。



 太平興國二年閏七月,衛州蝻蟲生。六年七月,河南府、宋州
 蝗。七年四月,北陽縣蝻蟲生,有飛鳥食之盡。滑州蝻蟲生。是月,大名府、陜州、陳州蝗。七月,陽谷縣蝻蟲生。



 雍熙三年七月,鄄城縣有蛾、蝗自死。



 淳化元年七月,淄、澶、濮州、乾寧軍有蝗。滄州蝗蝻蟲食苗。棣州飛蝗自北來,害稼。三年六月甲申,京師有蝗起東北,趣至西南,蔽空如雲翳日。七月,真、許、滄、沂、蔡、汝、商、兗、單等州,淮陽軍、平定、彭城軍蝗、蛾抱草自死。



 至道二年六月,亳州、宿、密州蝗生,食苗。七月,長葛、陽翟二縣有蝻蟲食苗。歷城、長清等
 縣有蝗。三年七月,單州蝻蟲生。



 景德二年六月,京東諸州蝻蟲生。三年八月,德、博蝝生。四年九月,宛丘、東阿、須城三縣蝗。



 大中祥符二年五月,雄州蝻蟲食苗。三年六月,開封府尉氏縣蝻蟲生。四年六月,祥符縣蝗。七月,河南府及京東蝗生,食苗葉。八月,開封府祥符、咸平、中牟、陳留、雍丘、封丘六縣蝗。九年六月,京畿、京東西、河北路蝗蝻繼生,彌覆郊野,食民田殆盡,入公私廬舍。七月辛亥,過京師,群飛翳空,延至江、淮南,趣河東,及霜寒始。斃



 天禧元年二月,開封府、京東西、河北、河東、陜西、兩浙、荊湖百三十州軍,蝗蝻復生,多去歲蟄者。和州蝗生卵,如稻粒而細。六月,江、淮大風,多吹蝗入江海,或抱草木殭死。二年四月,江陰軍蝻蟲生。



 天聖五年七月丙午,邢、洺州蝗。甲寅,趙州蝗。十一月丁酉朔,京兆府旱蝗。六年五月乙卯,河北、京東蝗。



 景祐元年六月,開封府、淄州蝗。諸路募民掘蝗種萬餘石。



 寶元二年六月癸酉,曹、濮、單三州蝗。四年,淮南旱蝗。是歲,京師飛蝗蔽天。



 皇祐五年,建
 康府蝗。



 熙寧元年,秀州蝗。五年,河北大蝗。六年四月,河北諸路蝗。是歲,江寧府飛蝗自江北來。七年夏,開封府界及河北路蝗。七月,咸平縣鴝鵒食蝗。八年八月,淮西蝗,陳、穎州蔽野。九年夏,開封府畿、京東、河北、陜西蝗。



 元豐四年六月,河北蝗。秋,開封府界蝗。五年夏,又蝗。六年夏,又蝗。五月,沂州蝗。



 元符元年八月,高郵軍蝗抱草死。



 崇寧元年夏,開封府界、京東、河北、淮南等路蝗。二年,諸路蝗,令有司酺祭。三年、四年,連歲大蝗,其飛蔽日,來自
 山東及府界,河北尤甚。



 宣和三年,諸路蝗。五年,蝗。



 建炎二年六月,京師、淮甸大蝗。八月庚午,令長吏修酺祭。



 紹興二十九年七月,盱眙軍、楚州金界三十里,蝗為風所墮,風止,復飛還淮北。三十二年六月,江東、淮南北郡縣蝗,飛入湖州境,聲如風雨;自癸巳至於七月丙申,遍於畿縣,餘杭、仁和、錢塘皆蝗。丙午,蝗入京城。八月,山東大蝗。癸丑,頒祭酺禮式。



 隆興元年七月,大蝗。八月壬申、癸酉,飛蝗過都,蔽天日。徽、宣、湖三州及浙東郡縣,害稼。京
 東大蝗,襄、隨尤甚,民為乏食。二年夏,餘杭縣蝗。



 乾道元年六月,淮西蝗,憲臣姚岳貢死蝗為瑞,以佞坐黜。



 淳熙三年八月,淮北飛蝗入楚州、盱眙軍界,如風雷者逾時,遇大雨皆死,稼用不害。九年六月,全椒、歷陽、烏江縣蝗。乙卯,飛蝗過都,遇大雨,墮仁和縣界。七月,淮甸大蝗,真、揚、泰州窖撲蝗五千斛,餘郡或日捕數十車,群飛絕江,墮鎮江府,皆害稼。十年六月,蝗遺種於淮、浙,害稼。十四年七月,仁和縣蝗。



 紹熙二年七月,高郵縣蝗。至於泰州。
 五年八月,楚、和州蝗。



 嘉泰二年,浙西諸縣大蝗,自丹陽入武進,若煙霧蔽天,其墮亙十餘里,常之三縣捕八千餘石,湖之長興捕數百石。時浙東近郡亦蝗。



 開禧三年,夏秋久旱,大蝗群飛蔽天,浙西豆粟皆既於蝗。



 嘉定元年五月,江、浙大蝗。六月乙酉,有事於圜丘、方澤,且祭酺。七月又酺,頒酺式於郡縣。二年四月,又蝗,五月丁酉,令諸郡修酺祀。六月辛未。飛蝗入畿縣。三年,臨安府蝗。七年六月,浙郡蝗。八年四月,飛蝗越淮而南。江、淮郡蝗,食
 禾苗、山林草木皆盡。乙卯,飛蝗入畿縣。己亥,祭酺,令郡有蝗者如式以祭。自夏徂秋,諸道捕蝗者以千百石計,饑民競捕,官出粟易之。九年五月。浙東蝗。丁巳,令郡國酺祭。是歲,薦饑,官以粟易蝗者千百斛。十年四月,楚州蝗。



 紹定三年,福建蝗。



 端平元年五月,當塗縣蝗。



 嘉熙四年,建康府蝗。



 淳祐二年五月,兩淮蝗。



 景定三年八月,兩浙蝗。



 紹興十年春,有野豕入海州,市民刺殺之。時州已陷,夏,
 鎮江軍帥王勝攻取之;明年,以其郡屬金,悉空其民。



 乾道六年,南雄州民家豕生數豚,首各具他獸形,有類人者。



 慶元初,樂平縣民家豕生豚,與南雄同而更具他獸蹄。三年四月,餘乾縣民家豕生八豚,其二為鹿。古田縣豕食嬰兒。



 淳化三年六月,黑風晝晦。



 景祐四年七月,黑氣長丈餘,出畢宿下。



 康定元年,黑風晝晦。



 元豐末,嘗有物大如席,夜見寢殿上,而神宗崩。元符末,又數見,而哲宗崩。至大觀間,漸晝見。政和元年以後,大作,每得人語
 聲則出。先若列屋摧倒之聲,其形廑丈餘,徬佛如龜,金眼,行動硜硜有聲。黑氣蒙之。不大了了,氣之所及,腥血四灑,兵刃皆不能施。又或變人形,亦或為驢。自春歷夏,晝夜出無時,遇冬則罕見。多在掖庭宮人所居之地,亦嘗及內殿,後習以為常,人亦不大怖。宣和末,浸少,而亂遂作。



 政和三年夏至,宰臣何執中奉祀北郊。有黑氣長數丈,出自齋宮,行一里許,入壇壝,繞祭所,皆近人穿燈燭而過。俄又及於壇,禮將畢,不見。



 宣和中,洛陽府畿間,
 忽有物如人,或蹲踞如犬。其色正黑,不辨眉目。始,夜則掠小兒食之後,雖白晝,入人家為患,所至喧然不安,謂之「黑漢」。有力者夜執槍棒自衛,亦有托以作過者,如此二歲乃息。已而北征事起,卒成金人之禍。三年春,日有眚,忽青黑無光,其中洶洶而動,若鈹金而湧沸狀。日旁有黑正如水波,周面旋繞,將暮而稍止。



 建炎三年二月甲寅,日初出,兩黑氣如人形,夾日旁,至巳時乃散。



 乾道四年春,舒州雨黑米,堅如鐵,破之,米心通黑。



 淳熙十一
 年二月。臨安府新城縣深浦天雨黑水終夕。十六年六月,行都錢塘門啟,黑風入,揚沙石。



 慶元元年,徽州黃山民家古井,風雨夜出黑氣,波浪噴湧。



 咸平元年五月,撫州王羲之墨池水色變黑如云。



 大中祥符元年五月丁丑,泰山王母池水變紅紫色。四年二月己未,河中府寶鼎縣瀵泉有光,如燭焰四五炬,其聲如雷。三年八月,解州鹽池紫泉場水次二十里許不種自生,其味特嘉,命屯田員外郎何敏中往祭池廟。八月,
 東池水自成鹽,僅半池,潔白成塊,晶瑩異常。祀汾陰經度制置使陳堯叟繼獻,凡四千七百斤,分賜近臣及諸列校。



 紹興十四年,樂平縣河沖里田隴數十百頃,田中水類為物所吸,聚為一直行,高平地數尺,不假堤防而水自行;里南程氏家井水溢,亦高數尺,夭矯如長虹,聲如雷,穿墻毀樓。二水鬥於杉墩,且前且卻,約十刻乃解,各復故。



 天聖四年十月甲午,昏霧四塞。



 靖康元年正月丁未,霧
 氣四塞,對面不見。



 建炎二年十一月甲子,北京大霧四塞,是夕,城陷。三年三月,車駕發溫州航海,乙丑,次松門,海中白霧,晝晦。六月,久陰。四年三月乙丑,四方霧下如塵。



 紹興三年,自正月陰晦,陽光不舒者四十餘日。五年正月甲申,霧氣昏塞。七月,劉豫毀明堂,天地晦冥者累日。七年,氛氣翳日。八年三月甲寅,晝晦,日無光,陰霧四塞。乙卯,晝夜雲氣昧濁。四月,積雨方止,氛霧四塞,晝日無光。



 隆興元年五月丙午,朝霧四塞。二年六月,積陰彌
 月。



 乾道二年十一月,久陰。五年正月甲申,晝蒙。六年五月,連陰。六月,日青無光。



 淳熙六年十二月乙丑,晝蒙。十三年正月丁亥,亦如之。



 慶元二年一月己卯,晝暝,四方昏塞。三年二月丁卯,晝晦,昏霧四塞。六年十二月辛卯、嘉定三年正月丙午、十年正月乙未、十三年三月壬辰,皆晝蒙。



 建炎四年三月辛亥,白虹貫日。



 紹興八年三月辛巳,白虹亙天。二十七年二月壬寅,白虹貫日。三十年十二月
 辛酉,曲虹見日之西。



 乾道三年十月丙申,虹見。



 淳熙元年十月戊寅,曲虹見日東。二年十月庚辰,虹見。五年十月丁巳,曲虹見日東。



 慶元元年正月丙辰,白虹貫日。



 嘉泰三年七月壬午,亦如之。四年十一月,虹見。



 嘉定十一年二月丙辰,白虹貫日。



 嘉熙三年十月乙丑,虹見。四年二月辛丑,白虹貫日。



 淳祐十年十二月丁巳,虹見。



 寶祐五年十月,虹見。



 太祖從周世宗征淮南,戰於江亭,有龍自水中向太祖
 奮躍。



 乾德五年夏,京師雨,有黑龍見尾於雲際,自西北趨東南。占主大水。明年,州府二十四水壞田廬。



 開寶六年四月,單父縣民王美家龍起井中,暴雨飄廬舍,失族屬,及壞舊鎮廨舍三百五十餘區,大木皆折。七年六月,棣州有火自空墮於城北門樓,有物抱東柱,龍形金色,足三尺許,其氣甚腥。旦視之,壁上有煙痕,爪跡三十六。



 大中祥符二年八月,青蛇出無為軍廨,長數尺。



 宣和元年夏,雨,晝夜凡數日。及霽,開封縣前茶肆中有異物如
 犬大,蹲踞臥榻下。細視之,身僅六七尺,色蒼黑,其首類驢,兩頰作魚頷而色正綠,頂有角,生極長,於其際始分兩歧,聲如牛鳴,與世所繪龍無異。茶肆近軍器作坊,兵卒來觀,共殺食之。已而京城大水,訛言龍復仇云。



 紹興初,朱勝非出守江州,過梁山,龍入其舟,才長數寸,赤背綠腹,白尾黑爪甲,目有光,近龍孽也。行都柴垛橋旌忠廟三蛇出沒庭廡,大者盈尺,方鱗金色,首脊有金錢,遇霽,或變化數百於蕉卉間。廟徙而蛇孽亦絕。十一年四
 月,衡山縣凈居巖有蛇長二丈,身圍數尺,黑色而方文,震死,山水大至。先是,山氣遇夜輒昏昧,蛇斃始明。二十五年六月,湖口縣赤龍橫水中如山,寒風怒濤,覆舟數十艘,士卒溺者數十人。三十年春,宜黃縣大蛇見於丞治,長二丈。捕之縱數里外,俄復至者數四。



 乾道五年七月乙亥,武寧縣龍鬥於復塘村,大雷雨,二龍奔逃,珠墜,大如車輪,牧童得之。自是連歲有水災。



 太平興國三年,靈州獻官馬駒,足有二距。



 雍熙二年,虔
 州吏李祚家馬生駒,足有距。四年,鄜州直羅縣民高英家馬生前兩足如牛。端拱二年,夏州民程真家馬生二駒。



 大中祥符九年十二月,大名監馬生駒,赤色,肉尾無□□。



 宣和五年,馬生兩角,長三寸,四足皆生距。時北方正用兵。



 紹興八年,廣西海□需有海獸如馬,蹄鬣皆丹,夜入民舍。聚眾殺之,明日海溢,環村百餘家皆溺死,近馬禍也。五年,廣西市馬,全綱疫死。



 淳熙六年十二月,宕昌西馬、金州馬皆大疫。十二年,黎、雅州獻馬,有角長二寸。京
 房《易傳》曰:「臣易上,政不順,厥妖馬生角,茲謂賢士不足。」



 紹熙元年二月丙申,右丞相留正乘馬早朝,入禁扉,馬斃,近馬禍也。



 嘉定五年正月,史彌遠入賀於東宮,馬驚墮地,衣幘皆敗,其額微損,事與上同。



 建隆元年,雄州歸義軍民劉進妻產三男。二年,孟州民孟福、定州民孟公禮等妻各產三男。三年,齊州、晉州大旱,民家多生魃。龍岡縣民林嗣妻、京師龍捷軍卒宜超妻產三男。



 乾德三年,江陵府民劉暉妻產三男。四年,安
 州驍健軍卒趙遠妻產三男。五年,光州民高與、德州民趙嗣、乾寧軍卒王進妻產三男。



 開寶元年,沂州民王政、澶州民謝興妻產三男。二年,閬州民孫延廣、開州民董遠妻產三男。七年,青城縣王宥妻產三男。河南府民劉元妻產三男。



 太平興國二年,邢州招收軍卒李遇、汝州歸化軍卒魚霸、常州民謝祚妻產三男。晉原縣民楊萬妻產三男。七年,澶州龍衛軍卒靳興、普州民鄭彥福妻產三男。汾州民鄭訓妻產三女。雁門縣民劉習妻產四
 男。滑州歸化軍卒安旺妻產二男一女。八年,揚州順化軍卒俞釗、溫州民李遇、榮州民李祚妻產三男。九年,揚子縣民妻生男,毛被體半寸餘,面長、頂高、烏肩、眉毛粗密,近發際有毛兩道軟長眉,紫唇、紅耳、厚鼻、大類西域僧。至三歲,畫圖以獻。



 雍熙二年,奉新縣民何靖妻產三男。三年,魯山縣民張美、相州林慮縣民張欽妻產三男。四年,晉原縣民周承暉、固始縣民楊升妻產三男。



 端拱元年,祁州民馮遇妻產三男。二年,齊州民徐美、並州民
 侯遠、常州卒徐流妻產三男。



 淳化元年正月,河陽縣民王斌、新息縣民李珪妻產三男。八月,汾州悉達院僧智嚴頭生角三寸。二年,晉陵縣民黃釗、南充縣民彭公霸、龍陽縣民周信、王屋縣民李清、臨清縣民國忠、鄰水縣吏謝元升、奉化縣卒朱旺妻產三男。瀛州民胡立、邢州民高德妻產三男。四年,邯鄲縣民鄭安、河間縣民王希輦、安州民宋和妻產三男。五年,雍丘縣營卒盛泰妻產三男。



 至道元年,保州敵軍校李深、宋城縣民王洽、臨淮
 縣民賀用、永清縣民董美、鄄城縣民馬方妻產三男。二年,安豐縣民王構、伊陽縣民張壽、成都縣民彭操妻產三男。三年,汾州民趙演、沂州民李嗣、南劍州民劉相、饒安縣民睦鸞、衛州宣武軍卒李筠妻產三男。



 咸平元年,臺州永安縣王旺、澶州靜戎軍卒鄭穗妻產三男。莘縣民懷梁、獲嘉縣民王貴、永康縣民羅彥□、溫縣民楊榮、毗陵縣民魏吉妻產三男。三年,睢縣民朱進、鄆州武威軍卒徐繞、深州民彭遠妻產三男。四年,望都縣民郭瑩、
 邕州澄海軍卒梁濟妻產三男。五年,夏津縣民趙替妻產三男。六年,石城縣民劉詵、堂邑縣民戴玉妻產三男。平鄉縣民郭讓妻產四男。



 景德元年,南昌縣民李聰妻產三男。二年,奉新縣民魏勇妻產三男。四年,八作司匠趙榮、南頓縣民任登老、棗強縣民張緒妻各產三男。



 大中祥符元年,高郵軍民王言妻產四男。二年,崞縣民張留、清平軍民楊泉妻產三男。三年,獲嘉縣民馮可妻產三男。宋城縣民李悔妻產二男一女。四年,河池縣民馮
 守欽妻產三男。五年,大名府宣勇軍卒徐璘、贊皇縣民李釗妻產三男。七年,銅鞮縣民李謙、宋城縣民白德、霍丘縣民朱璘、平涼縣民焦思順妻產三男。八年,河南府民宋再興、真陽縣民周元、歷亭縣民田用侯言、霍丘縣民王忠杜戩、蒙陽縣民衛志聰、定州驍武軍卒張吉、雍丘縣懷勇軍卒黃進妻產三男。永嘉縣民張保妻產四男。九年,曹州雄勇軍卒聶德、瀛州民劉元、澧州民張貴、廣州民劉吉妻產三男。



 天禧元年,連江縣民陳霸妻產
 三男。三年,錢塘縣民謝文信、遂安縣民李承遇妻產三男。四年,孝感縣民杜明、平恩縣民劉順妻產三男。七月,耒陽縣民張中妻產三男,其額有白志方寸餘,上生白發。



 自天聖迄治平,婦人生四男者二,生三男者四十四,生二男一女者一。熙寧元年距元豐七年,郡邑民家生三男者八十四,而四男者一,三男一女者一。元豐八年至元符二年,生三男者十八,而四男者二,三男一女者一。元符三年至靖康,生三男者十九,而四男者一。前志
 以為人民蕃息之驗。



 宣和六年,都城有賣青果男子,孕而生子,蓐母不能收,易七人,始免而逃去。豐樂樓酒保朱氏子之妻,可四十餘,楚州人,忽生髭,長僅六七寸,疏秀而美,宛然一男子,特詔度為女道士。



 紹興三年,建康府桐林灣婦產子,肉角、有齒。是歲,人多產鱗毛。二十年八月,真符縣民家一產三男。



 隆興元年,建康民流寓行都而婦產子,二首具羽毛之形。



 乾道五年,衡、湘間人有化為虎者。餘杭縣婦產子,青而毛,二肉角,又有二家婦
 產子亦如之,皆連體兩面相鄉。三家才相距一二里。潮州城西婦孕過期產子,如指大、五體皆具者百餘,蠕蠕能動。



 淳熙十年,番昜南鄉婦產子,肘各有二臂,及長,鬥則六臂並運。十三年,行都有人死十有四日復生。十一月辛未,鄧家巷婦產肉塊三,其一直目而橫口。十四年六月,臨安府浦頭婦產子,生而能言,四日。暴長四尺。



 紹熙元年三月癸酉,行都市人夜以殺相驚,奔迸者良久乃定。是歲,昆山縣工採石而山壓。三年六月,它工採石
 鄰山,聞其聲呼,相應答如平生。其家鑿石出之,見其妻,喜曰:「久閉乍風,肌膚如裂。」俄頃,聲微噤不語,化為石人,貌如生。



 慶元元年,樂平縣民婦產子有尾。永州民產子首有角,腋有肉翅。二年七月,進賢縣婦產子亦如之,而面有三目。



 嘉定四年四月,鎮江府後軍妻生子,一身二首而四臂。



 淳化五年六月,京師疫,遣太醫和藥救之。



 至道二年,江南頻年多疾疫。



 大觀三年,江東疫。



 建炎元年三月,金人
 圍汴京,城中疫死者幾半。



 紹興元年六月,浙西大疫,平江府以北,流尸無算。秋冬,紹興府連年大疫,官募人能服粥藥之勞者,活及百人者度為僧。三年二月,永州疫。六年,四川疫。十六年夏,行都疫。二十六年夏,行都又疫,高宗出柴胡制藥,活者甚眾。



 隆興二年冬,淮甸流民二三十萬避亂江南,結草舍遍山谷,暴露凍餒,疫死者半,僅有還者亦死。是歲,浙之饑民疫者尤眾。



 乾道元年,行都及紹興府饑,民大疫,浙東、西亦如之。六年春,民以冬
 燠疫作。八年夏,行都民疫,及秋未息。江西饑民大疫,隆興府民疫,遭水患,多死。



 淳熙四年,真州大疫。八年,行都大疫,禁旅多死。寧國府民疫死者尤眾。十四年春,都民、禁旅大疫,浙西郡國亦疫。十六年,潭州疫。



 紹興二年春,涪州疫死數千人。三年,資、榮二州大疫。



 慶元元年,行都疫。二年五月,行都疫。三年三月,行都及淮、浙郡縣疫。



 嘉泰三年五月,行都疫。



 嘉定元年夏,淮甸大疫,官募掩骼及二百人者度為僧。是歲,浙民亦疫。二年夏,都民疫死
 甚眾。淮民流江南者饑與暑並,多疫死。三年四月,都民多疫死。四年三月,亦如之。十五年,贛州疫。十六年,永、道二州疫。



 德祐元年六月庚子,是日,四城遷徙,流民患疫而死者不可勝計,天寧寺死者尤多。二年閏三月,數月間,城中疫氣熏蒸,人之病死者不可以數計。



 熙寧元年七月戊子夜,西南雲間有聲鳴,如風水相激,寢周四方。主民勞,兵革歲動。六年七月丙寅夜,西北雲間有聲如磨物,主百姓勞。七年七月庚子夜,西北天嗚,
 主驚憂之事。



 紹興二十一年八月乙亥,天有聲如雷,水響於東南,四日乃止。



 開禧元年六月壬寅,天鳴有聲。



 天禧三年正月晦,沉丘縣民駱新田聞震,頃之,隕石入地七尺許。



 淳熙十六年三月壬寅,隕石於楚州寶應縣,散如火,甚臭腥。



 慶元二年六月辛未,黃巖縣大石自隕,雷雨甚至,山水瀵湧。



\end{pinyinscope}