\article{列傳第二百四十八外國五}

\begin{pinyinscope}

 占城
 國在中國之西南,東至海,西至雲南,南至真臘國,北至歡州界。泛海南去三佛齊五日程。陸行至賓陀羅國一月程,其國隸占城焉。東去麻逸國二日程,蒲端國七日程。北至廣州,便風半月程。東北至兩浙一月程。西北至交州兩日程,陸行半月程。其地東西七百里,南北三千里。南曰施備州,西曰上源州,北曰烏里州。所統大小州三十八,不盈三萬家。其國無城郭,有百餘村,村落戶三五百,或至七百,亦有縣鎮之名。



 土地所出:箋沉香、
 檳榔、烏樠木、蘇木、白藤、黃蠟、吉貝花布、絲絞布、白㲲布、藤簟、貝多葉簟、金銀鐵錠等物。五穀無麥,有粳米、粟、豆、麻子。官給種一斛,計租百斛。果實有蓮、甘蔗、蕉子、椰子。鳥獸多孔雀、犀牛。畜產多黃牛、水牛而無驢;亦有山牛,不任耕耨,但殺以祭鬼,將殺,令巫祝之曰「阿羅和及拔」,譯云「早教他托生」。民獲犀、象皆輸于王。國人多乘象或軟布兜,或於交州市馬,頗食山羊、水兕之肉。



 其風俗衣服與大食國相類。無絲蠶,以白ふ布纏其胸,垂至於足,
 衣衫窄袖。撮髮為髻,散垂餘髾於其後。互市無緡錢,止用金銀較量錙銖,或吉貝錦定博易之直。樂器有胡琴、笛、鼓、大鼓,樂部亦列舞人。其王腦後髽髻,散披吉貝衣,戴金花冠,七寶裝纓絡為飾,脛股皆露,躡革履,無襪。婦人亦腦後撮髻,無笄梳,其服及拜揖與男子同。王每日午坐禪椅。官屬謁見膜拜一而止,白事畢復膜拜一而退。或出遊,看象、采獵、觀漁,皆數日方還。近則乘軟布兜,遠則乘象,或乘一木杠,四人舁之,先令一人持檳榔盤前
 導,從者十餘輩,各執弓箭刀槍手牌等,其民望之膜拜一而止。日或一再出。每歲稻熟,王自刈一把,從者及群婦女競割之。



 其王或以兄為副王,或以弟為次王。設高官凡八員,東西南北各二,分治其事,無奉祿,令其所管土俗資給之。別置文吏五十餘員,有郎中、員外、秀才之稱,分掌資儲寶貨等事,亦無資奉,但給龜魚充食及免調役而已。又有司帑廩者十二員,主軍卒者二百餘員,皆無月奉。勝兵萬餘人,月給粳米二斛,冬夏衣布各三
 匹至五匹。每夕,唯王升床而臥,諸臣皆寢於地蓐。親近之臣見王即胡跪作禮,稍疏遠者但拱手而已。



 其風俗,正月一日牽象周行所居之地,然後驅逐出郭,謂之逐邪。四月有遊船之戲。定十一月十五日為冬至,人皆相賀,州縣以土產物帛獻其王。每歲十二月十五日,城外縛木為塔,王及人民以衣物香藥置塔上焚之以祭天。人有疾病,旋采生藥服食。地不產茶,亦不知醞釀之法,止飲椰子酒,兼食檳榔。



 刑禁亦設枷鎖,小過以四人拽
 伏於地,藤杖鞭之,二人左右更互捶撲,量其罪或五六十至一百。當死者以繩系於樹,用梭槍舂喉而殊其首。若故殺、劫殺,令象踏之,或以鼻卷撲於地。象皆素習,將刑人,即令豢養之人以數諭之,悉能曉焉。犯奸者,男女共入牛以贖罪。負國王物者,以繩拘於荒塘,物充而後出之。



 其國前代罕與中國通。周顯德中,其王釋利因德漫遣其臣莆訶散貢方物,有雲龍形通犀帶、菩薩石。又有薔薇水灑衣經歲香不歇,猛火油得水愈熾,皆貯以璢
 璃瓶。



 建隆二年,其王釋利因陀盤遣使莆訶散來朝。表章書于貝多葉,以香木函盛之。貢犀角、象牙、龍腦、香藥、孔雀四、大食瓶二十。使回,錫齎有差,以器幣優賜其王。三年,又貢象牙二十二株、乳香千斤。



 乾德四年,其王悉利因陀盤遣使團因陀玢李帝婆羅貢馴象、牯犀、象牙、白氎、哥縵、越諾,王妻波良僕瑁、男占謀律秀瓊等各貢香藥。五年,又遣使李𠰢、李被瑳相繼來貢獻。



 開寶三年,遣使貢方物雌象一。四年,悉利多盤、副國王李耨、王妻郭
 氏、子蒲路雞波羅等並遣使來貢。五年,其王波美稅褐印茶遣使莆訶散來貢。六年,又貢。七年,又貢孔雀傘二、西天烽鐵四十斤。九年,遣使朱陀利、陳陀野等來貢。



 太平興國二年,其王波美稅陽布印茶遣使李牌來貢。三年,其王及男達智遣使來貢。四年,遣使李木吒哆來貢。六年,交州黎桓上言,欲以占城俘九十三人獻于京師。太宗令廣州止其俘,存撫之,給衣服資糧,遣還占城,詔諭其王。七年,遣使乘象入貢,詔留象廣州畜養之。八年,
 獻馴象,能拜伏,詔畜于京畿寧陵縣。



 雍熙二年,其王施利陀盤吳日歡遣婆羅門金歌麻獻方物,且訴為交州所侵,詔答令保國睦鄰。三年,其王劉繼宗遣使李朝仙來貢。儋州上言,占城人蒲羅遏為交州所逼,率其族百口來附。四年秋,廣州上言,雷、恩州關送占城夷人斯當李娘並其族一百五十人來歸,分隸南海、清遠縣。端拱元年,廣州又言,占城夷人忽宣等族三百一人來附。



 淳化元年,新王楊陀排自稱新坐佛逝國。楊陀排遣使李
 臻貢馴犀方物,表訴為交州所攻,國中人民財寶皆為所略。上賜黎桓詔,令各守境。三年,遣使李良莆貢方物。賜其王白馬二、兵器等。本國僧淨戒獻龍腦、金鈴、銅香爐、如意等,各優賜之。



 至道元年正月,其王遣使來貢,奉表言:



 前進奉使李良莆回,伏蒙聖慈賜臣細馬二匹、旗五面、銀裝劍五口、銀纏槍五條、弓弩各五張及箭等,戴恩感懼,稽首,稽首!



 臣生長外國,夐遠天都。竊承皇帝聖明,威德廣大,臣不憚介居海裔,遣使入朝。皇帝不棄蠻
 夷山國,曲加優賜。然臣自為土長,聲勢尚卑,常時外國頗相侵撓,況以前民庶如芥,隨風星散,流離各不自保。近蒙皇帝賜臣內閑駔駿及旗幟兵器等,鄰國聞之,知臣荷大國之寵,而各懼天威,不敢謀害。今臣一國安寧,流民來復,若非皇帝天德加護,何以至此!臣之一國仰望仁聖,覆之如天,載之如地。臣自思惟,鴻恩不淺。且自天子之都至臣所居之國,涉海綿邈,不啻數萬里,而所賜之馬及器械等並安全而至,皆聖德之所及也。



 自前
 本國進奉,未嘗有旌旗弓矢之賜,臣今何幸,獨受異恩!此蓋天威廣被,壯臣土疆。臣雖殞身無以上報。兼臣貢使往復,資給備至,恩重山嶽,不可具陳。今特遣專使李波珠、副使訶散、判官李磨勿等進奉犀角十株,象牙三十株,玳瑁十斤,龍腦二斤,沉香百斤,夾箋黃熟香九十斤,檀香百六十斤,山得雞二萬四千三百雙,胡椒二百斤,簟席五。前件物固非珍奇,惟表誠懇。



 臣生居異域,幸遇明時,不貴殊珍,惟重良馬。儻皇帝念及外國,不罪懇
 求,若使介南歸,願垂頒賜,臣之幸矣。兼臣本國元有流民三百,散居南海,曾蒙聖旨許令放還,今有猶在廣州者。本國舊有進奉夷人羅常占見駐廣州,乞詔本州盡數點集,具籍以付常占,令造舶船,乘便風部領歸國,冀得安其生聚,以實舊疆。至於萬里感恩,一心事上,臣之志也。



 上覽表,遣使詣廣州詢問,願還者悉付波珠。使還,復賜白馬二,遂為常制。



 咸平二年,其王楊普俱毗茶逸施離遣使朱陳堯、副使蒲薩陀婆、判官黎姑倫以犀象、
 玳瑁、香藥來貢,賜堯等冠帶衣褥有差。景德元年,又遣使來貢。詔以良馬、介胄、戎器等賜之。四年,遣使布祿爹地加等奉表來朝,表函藉以文錦,詞曰:



 占城國王楊普俱毗茶室離頓首言:臣聞二帝封疆,南止屆于湘、楚;三王境界,北不及于幽燕。仰矚昌時,實邁往跡。伏惟皇帝陛下乾坤授氣,日月儲英,出震居尊,承基御極。慈悲敷於天下,聲教被於域中。業茂前王,功芳徂後,蒼生是念,黃屋非心。無方不是生靈,有土並為臣妾。真風遍佈,霈
 澤周行,凡沐照臨,共增聳抃。



 臣生於邊鄙,幸襲華風。蟻垤蜂房,聊為遂性;龍樓鳳閣,尚阻觀光。再念自假天威,獲全封部,鄰無侵奪,俗有舒蘇。每歲拜遣下臣,問甯上國,蒙陛下恩沾行葦,福及豚魚,特因回人,頒賜戎器。臣本土惟望闕焚香,歡呼拜受,心知多幸,曷答洪恩。聖君既念于賓王,誠懇肯忘於述職。今遣專信臣布祿爹地加、副使臣除逋麻瑕珈耶、判官臣皮霸抵一行人力等,部署土毛,遠充歲貢。雖表楚茅之禮,實懷魯酒之憂。虔
 望睿明,甫寬譴戮。專信臣等回日,軍容器仗耀武之物,伏願重加賜齎。蓋念忝為臣子,合告君親,服飾車輿,威儀斧鉞,不敢私制,惟望恩頒。幹冒冕旒,不任死罪。



 布祿爹地加言本國舊隸交州,後奔于佛遊,北去舊所七百里。使還,賜物甚厚。



 大中祥府三年,國主施離霞離鼻麻底遣使朱氵孛禮來貢。四年,遣使貢師子,詔畜于苑中。使者留二蠻人以給豢養,上憐其懷土,厚給資糧遣還。八年,遣使波輪訶羅帝來貢。訶羅帝因上言有弟陶珠頃
 自交州押馴象赴闕,今幸得見,欲攜以還。許之,仍賜陶珠衣幣裝錢。



 天禧二年,其王屍嘿排摩惵遣使羅皮帝加以象牙七十二株、犀角八十六株、玳瑁千片、乳香五十斤、丁香花八十斤、豆蔻六十五斤、沉香百斤、箋香二百斤、別箋一劑六十八斤、茴香百斤、檳榔千五百斤來貢。羅皮帝加言國人詣廣州,或風漂船至石塘,即累歲不達矣。三年,使還,詔賜屍嘿排摩惵銀四千七百兩並戎器鞍馬。



 海上又有蒲端國、三麻蘭國、勿巡國、蒲婆衆
 國,大中祥符四年祀汾陰,並遣使來貢。先是,咸平、景德中,蒲端國主其陵數遣使來貢方物及獻紅鸚鵡。其後,國主悉離琶大遐至亦以金版鐫表來上,其使已絮漢上言:「伏見詔旨給賜占城使鞍勒馬、大神旗各二,乞如恩例。」有司以蒲端在占城下,請賜雜彩小旗五,從之。



 天聖八年十月,占城王陽補孤施離皮蘭德加拔麻疊遣使李蒲薩麻瑕陀琶來貢木香、玳瑁、乳香、犀角、象牙。



 慶曆元年九月,廣東商人邵保見軍賊鄂鄰百餘人在占
 城,轉運司選使臣二人齎詔書器幣賜占城,購鄰致闕下,餘黨令就戮之。明年十一月,其王刑卜施離值星霞弗遣使獻馴象三。皇祐二年正月,又使俱舍唎波微收羅婆麻提楊卜貢象牙二百一、犀角七十九。表二通,一以本國書,一以中國書。五年四月,其使蒲思馬應來貢方物。



 嘉祐元年閏三月,其使蒲息陀琶貢方物,還至太平州,江岸崩,沉失行橐。明年正月,詔廣州賜銀千兩。六年九月,又獻馴象。七年正月,廣西安撫經略司言:「占臘
 素不習兵,與交阯鄰,常苦侵軼;而占城復近修武備,以抗交阯,將繇廣東路入貢京師,望撫以恩信。」五月,其使頓琶尼來貢方物。六月,賜其王施里律茶盤麻常楊溥白馬一,從其求也。



 熙寧元年,其王楊卜屍利律陀般摩提婆遣使貢方物,乞市驛馬。詔賜白馬一,令于廣州買騾以歸。五年,貢璢璃珊瑚酒器、龍腦、乳香、丁香、蓽澄茄、紫礦。七年,交州李乾德言其王領兵三千人並妻子來降,以正月至本道。



 九年,復遣使來言:其國自海道抵真
 臘一月程,西北抵交州四十日,皆山路。所治聚落一百五,大略如州縣。王年三十六歲,著大食錦或川法錦大衫、七條金瓔珞,戴七寶裝成金冠,躡紅皮履。出則從者五百人,十婦人執金柈合貯檳榔,導以樂。



 王師討交阯,以其素仇,詔使乘機協力除蕩。行營戰棹都監楊從先遣小校樊實諭旨。實還,言其國選兵七千扼賊要路,其王以木葉書回牒,詔使上之。然亦不能成功。後兩國同入貢,占城使者乞避交人。詔遇朔日朝文德殿,分東西
 立;望日則交人入垂拱殿,而占城趨紫宸;大宴則東西坐。



 元祐七年,又表言如天朝討交阯,願率兵掩襲。朝廷以交阯數入貢,不絕臣節,難以興師,答敕書報之,而以其使良保故倫軋丹、副使傍木知突為保順郎將。政和中,授其王楊卜麻疊金紫光祿大夫,領廉、白州刺史。楊卜麻疊言身縻化外,不沾祿食,願得薄授奉給,壯觀小國,許之。



 宣和元年,進檢校司空兼御史大夫、懷遠軍節度、琳州管內觀察處置使,封占城國王。自是,每遇恩輒
 降制加封邑。



 建炎三年,楊卜麻疊遣使入貢,遇郊恩,制授檢校太傅,加食邑。紹興二十五年,其子鄒時闌巴嗣立,遣使進方物,求封爵,錫宴於懷遠驛,以其父初封之爵授之,報賜甚厚。



 乾道三年,子鄒亞娜嗣,掠大食國方物遣人來貢,以求封爵,為其國人所訴。詔卻之,遂不議其封。七年,閩人有浮海之吉陽軍者,風泊其舟抵占城。其國方與真臘戰,皆乘大象,勝負不能決。閩人教其王當習騎射以勝之,王大說,具舟送之吉陽,市得馬數十
 匹歸,戰大捷。明年復來,瓊州拒之,憤怒大掠而歸。淳熙二年,嚴馬禁,不得售外蕃。三年,占城歸所掠生口八十三人,求通商,詔不許。四年,占城以舟師襲真臘,傳其國都。



 慶元以來,真臘大舉伐占城以復仇,殺戮殆盡,俘其主以歸,國遂亡,其地悉歸真臘。



 真臘國亦名占臘,其國在占城之南,東際海,西接蒲甘,南抵加羅希。其縣鎮風俗同占城,地方七千餘里。有銅台,列銅塔二十有四、銅象八以鎮其上,象各重四千斤。
 其國有戰象兒二十萬,馬多而小。


政和六年十二月,遣進奏使奉化郎將鳩摩僧哥、副使安化郎將摩君明稽●
 \gezhu{
  缺字:田思}
 等十四人來貢,賜以朝服。僧哥言:「萬里遠國,仰投聖化,尚拘卉服,未稱區區向慕之誠,願許服所賜。」詔從之,仍以其事付史館,書諸策。明年三月辭去。宣和二年,又遣郎將摩臘、摩禿防來,朝廷官封其王與占城等。建炎三年,以郊恩授其王金裒賓深檢校司徒,加食邑,遂定為常制。其屬邑有真里富,在西南隅,東南接波斯蘭,西
 南與登流眉為鄰。所部有六十餘聚落。慶元六年,其國主立二十年矣,遣使奉表貢方物及馴象二。詔優其報賜,以海道遠涉,後毋再入貢。



 蒲甘國,崇寧五年,遣使入貢,詔禮秩視注輦。尚書省言:「注輦役屬三佛齊,故熙寧中敕書以大背紙,緘以匣袱,今蒲甘乃大國王,不可下視附庸小國。欲如大食、交阯諸國禮,凡制詔並書以白背金花綾紙,貯以間金鍍管籥,用錦絹夾袱緘封以往。」從之。



 邈黎國,元祐四年,般次冷移、四抹粟迷等齎于闐國黑汗王並本國王表章來。有司以其國未嘗入貢,請視於闐條式。從之。



 三佛齊國,蓋南蠻之別種,與占城為鄰,居真臘、闍婆之間,所管十五州。土產紅藤、紫礦、箋沉香、檳榔、椰子。無緡錢,土俗以金銀貿易諸物。四時之氣,多熱少寒,冬無霜雪。人用香油塗身。其地無麥,有米及青白豆,雞魚鵝鴨頗類中土。有花酒、椰子酒、檳榔酒、蜜酒,皆非曲蘖所醞,
 飲之亦醉。樂有小琴、小鼓,昆侖奴踏曲為樂。國中文字用梵書,以其王指環為印,亦有中國文字,上章表即用焉。累甓為城,周數十里,用椰葉覆屋。人民散居城外,不輸租賦,有所征伐,隨時調發。立酋長率領,皆自備兵器糧糗。泛海使風二十日至廣州。其王號詹卑,其國居人多蒲姓。唐天祐元年貢物,授其使都蕃長蒲訶栗立甯遠將軍。



 建隆元年九月,其王悉利胡大霞里檀遣使李遮帝來朝貢。二年夏,又遣使蒲蔑貢方物。是冬,其王室
 利烏耶遣使茶野伽、副使嘉末吒朝貢。其國號生留,王李犀林男迷日來亦遣使同至貢方物。三年春,室利烏耶又遣使李麗林、副使李鴉末、判官吒吒璧等來貢。回,賜以白嫠牛尾、白瓷器、銀器、錦線鞍轡二副。開寶四年,遣使李何末以水晶、火油來貢。五年,又來貢。七年,又貢象牙、乳香、薔薇水、萬歲棗、褊桃、白沙糖、水晶指環、琉璃瓶、珊瑚樹。八年,又遣使蒲陀漢等貢方物,賜以冠帶、器幣。



 太平興國五年,其王夏池遣使茶龍眉來。是年,潮州
 言,三佛齊國蕃商李甫誨乘舶船載香藥、犀角、象牙至海口,會風勢不便,飄船六十日至潮州,其香藥悉送廣州。八年,其王遐至遣使蒲押陀羅來貢水晶佛、錦布、犀牙、香藥。雍熙二年,舶主金花茶以方物來獻。端拱元年,遣使蒲押陀黎貢方物。淳化三年冬,廣州上言:「蒲押陀黎前年自京回,聞本國為闍婆所侵,住南海凡一年。今春乘舶至占城,偶風信不利,復還。乞降詔諭本國。」從之。



 咸平六年,其王思離咮啰無尼佛麻調華遣使李加排、
 副使無陀李南悲來貢,且言本國建佛寺以祝聖壽,願賜名及鐘。上嘉其意,詔以「承天萬壽」為寺額,並鑄鐘以賜,授加排歸德將軍,無陀李南悲懷化將軍。大中祥符元年,其王思離麻囉皮遣使李眉地、副使蒲婆藍、判官麻河勿來貢,許赴泰山陪位於朝覲壇,遣賜甚厚。天禧元年,其王霞遲蘇勿吒蒲迷遣使蒲謀西等奉金字表,貢真珠、象牙、梵夾經、昆侖奴,詔許謁會靈觀,遊太清寺、金明池。及還,賜其國詔書、禮物以慰獎之。



 天聖六年八
 月,其王室離疊華遣使蒲押陀羅歇及副使、判官亞加盧等來貢方物。舊制還國使人貢,賜以間金塗銀帶,時特以渾金帶賜之。



 熙寧十年,使大首領地華伽囉來,以為保順慕化大將軍,賜詔寵之,曰:「吾以聲教覆露方域,不限遠邇,苟知夫忠義而來者,莫不錫之華爵,耀以美名,以寵異其國。爾悅慕皇化,浮海貢琛,吾用汝嘉,並超等秩,以昭忠義之勸。」元豐中,使至者再,率以白金、真珠、婆律薰陸香備方物。廣州受表入言,俟報,乃護至闕下。
 天子念其道里遙遠,每優賜遣歸。二年,賜錢六萬四千緡、銀一萬五百兩,官其使群陀畢羅為寧遠將軍,官陀旁亞里為保順郎將。畢羅乞買金帶、白金器物,及僧紫衣、師號、牒,皆如所請給之。五年,廣州南蕃綱首以其主管國事國王之女唐字書,寄龍腦及布與提舉市舶孫迥,迥不敢受,言於朝。詔令估直輸之官,悉市帛以報。



 五年,遣使皮襪、副使胡仙、判官地華加羅來,入見,以金蓮花貯真珠、龍腦撒殿。官皮襪為懷遠將軍、胡仙加羅為郎
 將。加羅還至雍丘病死,賻以絹五十匹。六年,又以其使薩打華滿為將軍,副使羅悉沙文、判官悉理沙文為郎將。紹聖中,再入貢。



 紹興二十六年,其王悉利麻霞囉陀遣使入貢。帝曰:「遠人向化,嘉其誠耳,非利乎方物也。」其王復以珠獻宰臣秦檜,時檜已死,詔償其直而收之。淳熙五年,復遣使貢方物,詔免赴闕,館於泉州。



 闍婆國在南海中。其國東至海一月,泛海半月至昆侖國;西至海四十五日,南至海三日,泛海五日至大食國。
 北至海四日,西北泛海十五日至勃泥國,又十五日至三佛齊國,又七日至古邏國,又七日至柴曆亭,抵交阯,達廣州。



 其地平坦,宜種植,產稻、麻、粟、豆,無麥。民輸十一之租,煮海為鹽。多魚、鱉、雞、鴨、山羊,兼椎牛以食。果實有木瓜、椰子、蕉子、蔗、芋。出金銀、犀牙、箋沉檀香、茴香、胡椒、檳榔、硫黃、紅花、蘇木。亦務蠶織,有薄絹、絲絞、吉貝布。剪銀葉為錢博易,官以粟一斛二斗博金一錢。室宇壯麗,飾以金碧。中國賈人至者,待以賓館,飲食豐潔。地不產
 茶。其酒出於椰子及蝦蝚丹樹,蝦蝚丹樹華人未嘗見;或以桄榔、檳榔釀成,亦甚香美。不設刑禁,雜犯罪者隨輕重出黃金以贖,惟寇盜者殺之。



 其王椎髻,戴金鈴,衣錦袍,躡革履,坐方床,官吏日謁,三拜而退,出入乘象或腰輿,壯士五七百人執兵器以從。國人見王皆坐,俟其過乃起。以王子三人為副王。官有落佶連四人,共治國事,如中國宰相,無月奉,隨時量給土產諸物。次有文吏三百餘員,目為秀才,掌文簿,總計財貨。又有卑官殆千
 員,分主城池、帑廩及軍卒。其領兵者每半歲給金十兩,勝兵三萬,每半歲亦給金有差。



 土俗婚聘無媒約,但納黃金於女家以娶之。五月遊船,十月遊山,有山馬可乘跨,或乘軟兜。樂有橫笛、鼓板,亦能舞。土人被發,其衣裝纏胸以下至於膝。疾病不服藥,但禱神求佛。其俗有名而無姓。方言謂真珠為「沒爹蝦羅」,謂牙為「家囉」,謂香為「昆燉盧林」,謂犀為「低密」。



 先是,宋元嘉十二年,遣使朝貢,後絕。淳化三年十二月,其王穆羅茶遣使陀湛、副使蒲
 亞里、判官李陀那假澄等來朝貢。陀湛云中國有真主,本國乃修朝貢之禮。國王貢象牙、真珠、繡花銷金及繡絲絞、雜色絲絞、吉貝織雜色絞布、檀香、玳瑁檳榔盤、犀裝劍、金銀裝劍、藤織花簟、白鸚鵡、七寶飾檀香亭子。其使別貢玳瑁、龍腦、丁香、藤織花簟。



 先是,朝貢使泛舶船六十日至明州定海縣,掌市舶監察御史張肅先驛奏其使飾服之狀與嘗來入貢波斯相類。譯者言云:今主舶大商毛旭者,建溪人,數往來本國,因假其鄉導來朝
 貢。又言其國王一號曰夏至馬囉夜,王妃曰落肩娑婆利,本國亦署置僚屬。又其方言目舶主為「葧荷」,主妻曰「葧荷比尼贖」。其船中婦人名眉珠,椎髻,無首飾,以蠻布纏身,顏色青黑,言語不能曉,拜亦如男子膜拜;一子,項戴金連鎖子,手有金鉤,以帛帶縈之,名阿嚕。其國與三佛齊有仇怨,互相攻佔。本國山多猴,不畏人,呼以霄霄之聲即出。或投以果實,則其大猴二先至,土人謂之猴王、猴夫人,食畢,群猴食其餘。使既至,上令有司優待;久
 之使還,賜金幣甚厚,仍賜良馬戎具,以從其請。其使云:鄰國名婆羅門,有善法察人情,人欲相危害者皆先知之。大觀三年六月,遣使入貢,詔禮之如交阯。



 又有摩逸國,太平興國七年,載寶貨至廣州海岸。



 建炎三年,以南郊恩制授闍婆國主懷遠軍節度、琳州管內觀察處置等使、金紫光祿大夫、檢校司空、使持節琳州諸軍事、琳州刺使、兼御史大夫、上柱國、闍婆國王、食邑二千四百戶、實封一千戶;悉里地茶蘭固野可特授檢校司徒,加
 食邑實封。紹興二年,復加食邑五百戶,實封二百戶。



 南毗國在大海之西南,由三佛齊風颿月餘可至。其國王每巡行,先期遣兵百余人持水灑地上,以防颶風揚沙塵;列鼎百以進食,日一易之,置翰林官供王飲食。俗喜戰鬥,習刀槊,善射。鑿雜白銀為錢。產真珠、番布。其國最遠,番舶罕到。時羅巴智力幹父子,其種類也,居泉之城南。自是,舶舟多至其國矣。



 勃泥國在西南大海中,去闍婆四十五日程,去三佛齊
 四十日程,去占城與摩逸各三十日程,皆計順風為則。其國以版為城,城中居者萬餘人,所統十四州。其王所居屋覆以貝多葉,民舍覆以草。在王左右者為大人。王坐繩床,若出,即大布單坐其上,衆舁之,名曰阮囊。戰鬥者則持刀被甲,甲以銅鑄,狀若大筒,穿之於身,護其腹背。



 其地無麥,有麻稻,又有羊及雞魚,無蠶絲,用吉貝花織成布。飲椰子酒。昏聘之資,先以椰子酒,檳榔次之,指環又次之,然後以吉貝布,或量出金銀成其禮。喪葬亦
 有棺斂,以竹為輦,載棄山中。二月始耕則祀之,凡七年則不復祀矣。以十二月七日為歲節。地熱,多風雨。國人宴會,鳴鼓、吹笛、擊鈸,歌舞為樂。無器皿,以竹編貝多葉為器盛食,食訖棄之。其國鄰于底門國,有藥樹,取其根煎為膏,服之及塗其體,兵刃所傷皆不死。前代未嘗朝貢,故史籍不載。



 太平興國二年,其王向打遣使施弩、副使蒲亞里、判官哥心等齎表貢大片龍腦一家底、第二等八家底、第三等十一家底、米龍腦二十家底、蒼龍腦
 二十家底,凡一家底並二十兩;龍腦版五、玳瑁殼一百、檀香三橛、象牙六株。表云:「為皇帝千萬歲壽,望不責小國微薄之禮。」其表以數重小囊緘封之,非中國紙,類木皮而薄,瑩滑,色微綠,長數尺,闊寸餘,橫卷之僅可盈握。其字細小,橫讀之。以華言譯之,云:「勃泥國王向打稽首拜,皇帝萬歲萬歲萬萬歲,願皇帝萬歲壽,今遣使進貢。向打聞有朝廷,無路得到。昨有商人蒲盧歇船泊水口,差人迎到州,言自中朝來,比詣闍婆國,遇猛風破其船,
 不得去。此時聞自中國來,國人皆大喜,即造舶船,令蒲盧歇導達入朝貢,所遣使人只願平善見皇帝。每年令人入朝貢,每年修貢,慮風吹至占城界,望皇帝詔占城,令有向打船到,不要留。臣本國別無異物,乞皇帝勿怪。」其表文如是。詔館其使於禮賓院,優賜以遣之。



 元豐五年二月,其王錫理麻喏復遣使貢方物,其使乞從泉州乘海舶歸國,從之。



 注輦國東距海五里,西至天竺千五百里,南至羅蘭二
 千五百里,北至頓田三千里,自古不通中國,水行至廣州約四十一萬一千四百里。其國有城七重,高七尺,南北十二里,東西七里。每城相去百步,凡四城用磚,二城用土,最中城以木為之,皆植花果雜木。其第一至第三皆民居,環以小河;第四城四侍郎居之;第五城主之四子居之;第六城為佛寺,百僧居之;第七城即主之所居,室四百余區。



 所統有三十一部落,其西十二,曰只都尼、施亞盧尼、羅琶離鱉琶移、布林琶布尼、古檀布林蒲登、
 故里、娑輪岑、本蹄揭蹄、閻黎池離、舟阝部尼、遮古林、亞里者林;其南八,曰無雅加黎麻藍、眉古黎苦低、舍里尼、密多羅摩、伽藍蒲登、蒙伽林伽藍、琶里琶離游、亞林池蒙伽藍;其北十二,曰撥囉耶、無沒離江、注林、加里蒙伽藍、漆結麻藍、楃折蒙伽藍、皮林伽藍、浦棱和藍、堡琶來、田注離、盧婆囉、迷蒙伽藍。



 今國主相傳三世矣。民有罪,即命侍郎一員處治之,輕者縶於木格,笞五十至一百;重者即斬,或以象踐殺之。其宴,則國主與四侍郎膜拜于階,
 遂共坐作樂歌舞,不飲酒,而食肉。俗衣布。亦有餅餌。掌饌執事用婦人。其嫁娶,先用金銀指環使媒婦至女家,後二日,會男家親族,約以土田、生畜、檳榔酒等,稱其有無為禮;女家復以金銀指環、越諾布及女所服錦衣遣婿。若男欲離女則不取聘財,女卻男則倍償之。



 其兵陣,用象居前,小牌次之,梭槍次之,長刀又次之,弓矢在後,四侍郎分領其衆。國東南約二千五百里有悉蘭池國,或相侵伐。



 地產真珠、象牙、珊瑚、頗黎、檳榔、豆蔻、吉貝布。
 獸有山羊、黃牛。禽有山雞、鸚鵡。果有餘甘、藤羅、千年棗、椰子、甘羅、昆侖梅、婆羅密等。花有白末利、散絲、蛇臍、佛桑、麗秋、青黃碧娑羅、瑤蓮、蟬紫、水蕉之類。五穀有綠豆、黑豆、麥、稻。地宜竹。



 自昔未嘗朝貢。大中祥符八年九月,其國主羅茶羅乍遣進奉使侍郎娑里三文、副使蒲恕、判官翁勿、防援官亞勒加等奉表來貢。三文等以盤奉真珠、碧玻璃升殿,布於御坐前,降殿再拜,譯者導其言曰:「願以表遠人慕化之誠。」其國主表曰:



 臣羅茶羅乍言,
 昨遇𦨴舶船商人到本國告稱:钜宋之有天下也,二帝開基,聖人繼統,登封太嶽,禮祀汾陰,至德升聞,上穹眷命。臣昌期斯遇,吉語幸聞,輒傾就日之誠,仰露朝天之款。



 臣伏聞人君之御統也,無遠不臻;臣子之推誠也,有道則服。伏惟皇帝陛下功超邃古,道建大中。衣裳垂而德合乾坤,劍戟鑄而範圍區宇。神武不殺,人文化成。廓明明之德以臨御下民,懷翼翼之心以昭事上帝。至仁不傷於行葦,大信爰及於淵魚。故得天鑒孔彰,帝文有
 赫,顯今古未聞之事,保家邦大定之基。



 竊念臣微類醯雞,賤如芻狗,世居夷落,地遠華風,虛荷燭幽,曾無執贄。今者竊聽歌頌,普及遐陬。恨年屬於桑榆,阻躬陳於玉帛。矧滄溟之曠絕,在跋涉以稍艱。是敢傾倒赤心,遙瞻丹闕。任土作貢,同螻蟻之慕膻;委質事君,比葵藿之向日。謹遣專使等五十二人,奉土物來貢,凡真珠衫帽各一、真珠二萬一千一百兩、象牙六十株、乳香六十斤。



 三文等又獻珠六千六百兩、香藥三千三百斤。



 初,羅茶羅
 乍既聞商船言,且曰十年來海無風濤,古老傳云如此則中國有聖人,故遣三文等入貢。三文離本國,舟行七十七晝夜,曆舟阝勿丹山、娑里西蘭山至占賓國。又行六十一晝夜,曆伊麻羅里山至古羅國。國有古羅山,因名焉。又行七十一晝夜,曆加八山、占不牢山、舟寶龍山至三佛齊國。又行十八晝夜,度蠻山水口,曆天竺山,至賓頭狼山,望東西王母塚,距舟所將百里。又行二十晝夜,度羊山、九星山至廣州之琵琶洲。離本國凡千一百五
 十日至廣州焉。詔閣門祗候史祐之館伴,凡宴賜恩例同龜茲使。其年承天節,三文等請於啟聖禪院會僧以祝聖壽。明年使回,降詔羅茶羅乍,賜物甚厚。



 天禧四年,又遣使琶攔得麻烈呧奉方物入貢,至廣州病死。守臣以其表聞。詔廣州宴犒從者,厚賜以遣之。



 明道二年十月,其王尸離囉茶印
 
  
   
  
 
 囉注囉遣使蒲押陀離等以泥金表進真珠衫帽及真珠一百五兩、象牙百株,西染院副使、閣門通事舍人符惟忠假鴻臚少卿押伴。蒲押陀
 離自言數朝貢,而海風破船不達,願將上等珠就龍床腳撒殿,頂戴瞻禮,以申向慕之心。乃奉銀盤升殿,跪撒珠於御榻下而退。景祐元年二月,以蒲押陀離為金紫光祿大夫、懷化將軍,還本國。



 熙寧十年,國王地華加羅遣使奇囉囉、副使南卑琶打、判官麻圖華羅等二十七人來獻踠豆珠、麻珠、琉璃大洗盤、白梅花腦、錦藥、犀牙、乳香、瓶香、薔薇水、金蓮花、木香、阿魏、鵬砂、丁香。使副以真珠、龍腦登陛,跪而散之,謂之撒殿。既降,詔遣御藥宣勞
 之,以為懷化將軍、保順郎將,各賜衣服器幣有差;答賜其王錢八萬一千八百緡、銀五萬二千兩。



 丹眉流國,東至占臘五十程,南至羅越水路十五程,西至西天三十五程,北至程良六十程,東北至羅斛二十五程,東南至闍婆四十五程,西南至程若十五程,西北至洛華二十五程,東北至廣州一百三十五程。



 其俗以版為屋;跣足,衣布,無紳帶,以白紵纏其首;貿易以金銀。其主所居,廣袤五里,無城郭;出則乘象車,亦有小駟。地
 出犀、象、鍮石、紫草、蘇木諸藥。四時炎熱,無雪霜。未嘗至中國。



 咸平四年,國主多須機遣使打吉馬、副使打臘、判官皮泥等九人來貢木香千斤、鍮镴各百斤、胡黃連三十五斤、紫草百斤、紅氈一合、花布四段、蘇木萬斤、象牙六十一株。召見崇德殿,賜以冠帶服物。及還,又賜多須機詔書以敦獎之。



\end{pinyinscope}