\article{本紀第十二}

\begin{pinyinscope}

 仁宗四



 二年春正月癸卯,以歲饑,罷上元觀燈。壬子,命近臣同三司較天下財賦出入之數。



 二月甲申,出內庫絹五十萬,下河北、陜西、河東路,以備軍賞。



 三月戊子朔,詔季秋
 有事於明堂。己丑,以大慶殿為明堂。甲午,遣官祈雨。丁酉,月犯軒轅大星。戊戌,詔明堂禮成,群臣毋上尊號。庚子,契丹遣使以伐夏師還來告。丙午,雨。己酉,詔兩浙流民聽人收養。翰林學士趙概報使契丹。夏五月丁亥朔,新作明堂禮神玉。己亥,旌定州義民李能。



 六月己未,出新制明堂樂八曲。丁卯,以自制黃鐘五音五曲,並肄於太常。庚午,定選舉縣令法。壬申,月犯填星。癸未,錄系囚。



 熒惑入輿鬼,犯積尸。癸亥,出內藏絹百萬市糴軍儲。壬
 申,深州大雨,壞廬舍。



 九月丁亥,閱雅樂。己酉,朝饗景靈宮。庚戌,饗太廟。辛亥,大饗天地於明堂,以太祖、太宗、真宗配,如圜丘。大赦,百官進秩一等。詔自今內降指揮,百司執奏毋輒行。敢因緣乾請者,諫官、御史察舉之。



 冬十月庚午,熒惑犯太微上將。乙亥,宴京畿父老於錫慶院。閏十一月己未,詔后妃之家毋得除二府職任。丙寅,秀州地震,有聲如雷。丁卯,詔中書門下省、兩制及太常官詳定大樂。河北水,詔蠲民租,出內藏錢四十萬緡、絹四
 十萬匹付本路,使措置是歲芻糧。



 十二月甲申,定三品以上家廟制。唃廝囉、西蕃瞎氈、西南蕃龍光㴰、占城、沙州來貢,涇原路生戶都首領那龍男阿日丁內附。



 三年春正月乙丑,幸魏國大長公主第視疾。



 二月丙戌,宰臣文彥博等進《皇祐大饗明堂記》。己亥,復行河北沿邊州軍入中糧草見錢法。



 三月庚申,宋庠罷,以劉沆參知政事。癸酉,儂智高表獻馴象及金銀,卻之。



 夏四月癸未,詔:「河北流民相屬,吏不加恤,而乃飾廚傳,交賂使客,
 以取名譽。自今非犒設兵校,其一切禁之。」丙申,太白晝見。



 五月庚戌,以恩、冀州旱,詔長吏決系囚。壬申,置河渠司。乙亥,頒《簡要濟眾方》,命州縣長吏按方劑以救民疾。丁丑,錄系囚。



 六月丁亥,無為軍獻芝草,帝命姑免知軍茹孝標罪,戒州郡自今勿復獻。



 秋七月癸丑,詔:「少卿、監以下,年七十不任厘務者,御史臺、審官院以聞。嘗任館閣、臺諫及提刑者,中書裁處。待制以上能自引年,則優加恩禮。」丙辰,以孔氏子孫復知仙源縣事。丁巳,兩制、禮
 官上大樂,名曰《太安》。辛酉,河決大名府郭固口。乙丑,罷、徙州縣長吏不任事者十有六人。丙子,減郴、永州、桂陽監丁身米歲十萬餘石。



 八月丙戌,遣使安撫京東、淮南、兩浙、荊湖、江南饑民。辛卯,詔諸路監司具所部長吏治狀能否以聞。是月,汴河絕流。



 冬十月庚子,文彥博罷,以龐籍同中書門下平章事、昭文館大學士,高若訥為樞密使,梁適參知政事,王堯臣為樞密副使。



 十一月辛亥,減漳州、泉州、興化軍丁米。十二月庚辰,新作渾儀。庚子,
 詔文武官七十以上未致仕者,更不考課遷官。甲辰,罷災傷州軍貢物。是歲,涇原樊家族密廝歌內附。



 四年春正月己巳,詔諸路貸民種。乙亥,塞大名府決河。



 二月庚子,蠲湖州民所貸官米。



 三月己酉,詔禮部貢舉。丙辰,蠲江南路民所貸種數十萬斛。辛酉,錄系囚。辛未,詔宮禁市物給實直,非所闕者毋市。



 夏四月庚辰,詔修河兵夫逃亡死傷,會其數,以議官吏之罰。廣源州蠻儂智高反。



 五月乙巳朔,智高陷邕州,遂陷橫、貴等八州,圍
 廣州。壬申,命知桂州陳曙率兵討智高。六月乙亥,起前衛尉卿餘靖為秘書監、湖南安撫使、知潭州,前尚書屯田員外郎、直史館楊畋體量安撫廣南、提舉經制盜賊事。庚辰,改余靖為廣西安撫使、知桂州,命同提點廣東刑獄李樞與陳曙討智高,廣東轉運鈐轄司發兵援之。丁亥,以狄青為樞密副使。



 秋七月乙巳,出內藏錢絹助河北軍儲。丙午,命餘靖經制廣南盜賊事。丁巳,大風拔木。壬戌,智高引眾去廣州,廣東兵馬鈐轄張忠、知英州
 蘇緘邀擊於白田,忠戰歿。甲子,廣東兵馬鈐轄蔣偕又敗於路田。



 八月癸未,詔開封府比大風雨,民廬摧圮壓死者,官為祭斂之。辛卯,命樞密直學士孫沔安撫湖南、江西,內侍押班石全斌副之。



 九月丁巳,命餘靖提舉廣南兵甲經制賊盜事。庚申,廣西兵馬鈐轄王正倫討智高於昭州館門驛,戰歿。智高入昭州。庚午,以狄青為宣徽南院使。宣撫荊湖路、提舉廣南經制賊盜事。是月,智高襲殺蔣偕於太平場。



 冬十月丙子,太白犯南斗。詔鄜
 延、環慶、涇原路擇蕃落廣銳軍各五千人赴廣南行營。丁丑,智高入賓州。甲申,復入邕州。丁亥,以諸路饑疫並征徭科調之煩,令轉運使、提點刑獄、親民官條陳救恤之術以聞。



 十一月壬寅朔,日有食之。戊午,詔免江西、湖南、廣南民供軍須者今年秋租十之三。十二月壬申朔,廣西兵馬鈐轄陳曙討智高兵,戰於金城驛。壬辰,觀新樂。乙未,錄唐顏真卿後。是歲,河北路及鄜州水,蠲河北民積年逋負、鄜州民稅役。



 五年春正月壬寅朔,御大慶殿受朝。庚戌,以廣南用兵,罷上元張燈。白虹貫日。丁巳,會靈觀火。戊午,狄青敗智高於邕州,斬首五千餘級,智高遁去。甲子,遣使撫問廣南將校,賜軍士緡錢。



 二月癸未,狄青復為樞密副使。甲申,赦廣南。凡戰沒者,給槥櫝護送還家,無主者葬祭之。賊所過郡縣,免其田租一年,死事家科徭二年。貢舉人免解至禮部,不預奏名者亦以名聞。丙戌,詔廣西都監蕭注等追捕智高。丁亥,下德音:減江西、湖南系囚罪一
 等,徒以下釋之。丁壯饋運廣南軍須者,減夏稅之半,仍免差徭一年。戊子,詔百官遇南郊奏薦,無子孫者聽奏期親一人。乙未,詔宗室通經者,大宗正司以聞。



 三月癸亥,遣使奉安太祖御容於滁州,太宗御容於並州,真宗御容於澶州。是月,賜禮部奏名進士、諸科及第出身千四十二人。



 夏四月甲午,命劉沆、梁適監議大樂。



 五月乙巳,詔輔臣凡有大政,許復對後殿。高若訥罷,以狄青為樞密使。丁未,孫沔為樞密副使。戊申,詔轉運使毋取羨
 餘以助三司。庚戌,詔智高所至州,無城壘,若兵力不敵而棄城者,奏裁。壬子,錄系囚。丁巳,詔轉運司振邕州貧民,戶貸米一石。甲子,詔諫官、御史毋挾私以中善良,及臣僚言機密事毋得漏洩。



 六月乙亥,御紫宸殿,按《太安樂》,觀宗廟祭器。丙戌,作集禧觀成。乙未,詔河北薦饑,轉運使察州縣長吏能招輯勞來者,上其狀;不稱職者舉劾之。



 秋七月乙巳,詔荊湖北路民因災傷所貸常平倉米免償。己酉,詔薦舉非其人者,令御史臺彈奏,見任監
 司以上弗許薦論。戊午,詔太常定謚,毋為溢美。



 閏月戊辰,詔廣南民逃未還者,限一年歸業,其復三歲。壬申,龐籍罷,以陳執中同中書門下平章事、昭文館大學士,梁適同中書門下平章事、集賢殿大學士。乙亥,詔武臣知州軍,須與僚屬參議公事,毋專決。庚辰,秦鳳路言總管劉煥等破蕃部,斬首二千餘級。



 八月丁酉朔,詔民訴災傷而監司不受者,聽州軍以狀聞。辛酉,策制舉、武舉人。壬戌,詔南郊以太祖、太宗、真宗並配。



 九月乙酉,觀新樂。



 冬十月丙申朔,日有食之。壬子,作「鎮國神寶」。丁巳,詔以蝗旱,令監司諭親民官上民間利害。



 十一月丁卯,朝饗景靈宮。戊辰,饗太廟、奉慈廟。己巳,祀天地於圜丘,大赦。丁丑,加恩百官。戊子,放天下逋負。十二月戊午,詔轉運官毋得進羨餘。壬戌,以曹、陳、許、鄭、滑州為輔郡,隸畿內,置京畿轉運使。是歲,占城國來貢。



 至和元年春正月辛未,詔京師大寒,民多凍餒死者,有司其瘞埋之。壬申,碎通天犀,和藥以療民疫。癸酉,貴妃
 張氏薨,輟視朝七日,禁京城樂一月。丁丑,追冊為皇后,賜謚溫成。辛卯,錄系囚,減三京、輔郡雜犯死罪一等,徒以下釋之。



 二月庚子,詔治河堤民有疫死者,蠲戶稅一年;無戶稅者,給其家錢三千。壬戌,孫沔罷,以田況為樞密副使。



 三月己巳,王貽永罷,以王德用為樞密使。辛未,命曾公亮等同試入內醫官。壬申,賜邊臣攻守圖。置京畿提點刑獄。乙亥,太史言日當食四月朔。庚辰,下德音:改元,減死罪一等,流以下釋之。癸未,易服,避正殿,減常
 膳。乙酉,詔京西民饑,宜令所在勸富人納粟以振之。



 夏四月甲午朔,日有食之,用牲於社。辛丑,御正殿,復常膳。祥源觀火。



 五月戊寅,以河北流民稍復,遣使安撫。壬辰,太白晝見。



 秋七月丁卯,以程戡參知政事。立溫成園。戊辰,梁適罷。己巳,出御史馬遵、呂景初、吳中復。



 八月丁酉,詔:「前代帝王後嘗仕本朝,官八品以下,其祖父母、妻子犯流以下罪,聽贖;未仕而嘗受朝廷賜者,所犯非兇惡,亦聽贖。」丙午,以劉沆同中書門下平章事、集賢殿大學
 士。命修起居注官侍經筵。



 九月乙亥,契丹遣使來告夏國平。辛巳,遣三司使王拱辰報使契丹。己丑,太白晝見。



 冬十月辛卯朔,太白晝見。壬辰,詔士庶家毋得以嘗傭顧之人為姻,違者離之。丁酉,葬溫成皇后。丙午,溫成皇后神主入廟。戊午,幸城北炮場觀發炮,宴從臣,賜衛士緡錢。



 十一月甲子,出太廟禘袷、時饗及溫成皇后樂章,肄於太常。十二月丙午,詔司天監天文算術官毋得出入臣僚家。癸丑,詔內侍傳宣,令都知司札報,被旨者覆
 奏。是歲,融州大丘洞楊光朝內附。



 二年春正月丁卯,奉安真宗御容於萬壽觀。減畿內、輔郡囚罪一等,徒以下釋之。賜諸軍緡錢。戊辰,邕州言蘇茂州蠻內寇,詔廣西發兵討之。丁亥,晏殊薨。



 二月壬辰,汾州團練推官郭固上車戰法,既試之,授衛尉丞。



 三月丁卯,詔修起居注立於講讀官之次。丙子,封孔子後為衍聖公。是月,以旱,除畿內民逋芻及去年秋逋稅,罷營繕諸役。



 夏四月己亥,契丹遣使賀乾元節,以其主之命
 持本國三世畫像來求御容。辛亥,定差衙前法。乙卯,出米京城門,下其價以濟流民。



 五月己未,錄系囚、。辛酉,詔中書公事並用祖宗故事。戊寅,詔戒百官務飭官守。



 六月戊戌,陳執中罷。以文彥博同中書門下平章事、昭文館大學士,劉沆監修國史,富弼同中書門下平章事、集賢殿大學士。乙巳,儂智高母儂氏、弟智光、子繼宗、繼封伏誅。秋八月戊子,減畿內、輔郡囚罪一等,徒以下釋之。乙未,置臺諫章奏簿。壬子,詔中書、樞密院第宗姓服屬,
 自明堂覃恩後及十年者,咸與進官。



 九月戊午,契丹使來告其國主宗真殂,帝為發哀,成服於內東門幕次,遣使祭奠、吊慰及賀其子洪基立。戊辰,詔試醫官須引《醫經》、《本草》以對,每試十道,以六通為合格。辛己,罷輔臣、宣徽、節度使乾元節任子恩。



 冬十月丙戌,錄唐長孫無忌後。己丑,詔京畿毋領輔郡,罷京畿轉運使、提點刑獄。癸丑,下溪州蠻彭仕羲內寇,詔湖北路發兵捕之。



 十一月乙卯,交址來告李德政卒,其子日尊上德政遺留物及
 馴象。己未,行並邊見錢和糴法。十二月丁亥,修六塔河。丁酉,詔武臣有贓濫者毋得轉橫行,其立戰功者許之。庚子,契丹遣使致其主宗真遺留物及謝吊祭。庚戌,太白晝見。壬子,作醴泉觀成。是歲,西界阿訛等內附,詔遣還。龍賜州彭師黨以其族來歸,大食國、西蕃、安化州蠻來貢。



 嘉祐元年春正月甲寅朔,御大慶殿受朝。是日,不豫。辛酉,輔臣禱祠於大慶殿,齋宿殿廡。近臣禱於寺觀,及遣
 諸州長吏禱於岳瀆諸祠。壬戌,御崇政殿。癸亥,賜在京諸軍緡錢。甲子,赦天下,蠲被災田租及倚閣稅。戊辰,罷上元張燈。辛未,命輔臣禱天地、宗廟、社稷。是月,大雨雪,木冰。



 二月甲辰,帝疾愈,御延和殿。



 三月丁巳,詔禮部貢舉。辛未,司天監言:自至和元年五月,客星晨出東方守天關,至是沒。壬申,遣官謝天地、宗廟、社稷、寺觀、諸祠。癸酉,契丹遣使來謝。



 閏月癸未朔,以王堯臣參知政事,程戡為樞密副使。詔前後殿間日視事。



 夏四月壬子朔,六
 塔河復決。丙辰,裁定補蔭選舉法。甲戌,錄系囚。是月,大雨,水注安上門,門關折,壞官私廬舍數萬區。諸路言江、河決溢,河北尤甚。



 六月辛亥朔,詔雙日不御殿,伏終如舊。辛未,免畿內、京東西、河北被水民賦租。乙亥,雨壞太社、太稷壇。戊寅,遣使安撫河北。己卯,詔群臣實封言時政闕失。



 秋七月乙酉,命京東西、湖北監司分行水災州軍振饑蠲租。丙戌,賜河北流民米,壓溺死者,賜其家錢有差。己丑,出內藏銀絹三十萬振貸河北。月入南斗。乙
 巳,貸被水災民麥種。是月,彗出紫微垣,長丈餘。環州小遇族叛,知州張揆破降之。



 八月庚戌朔,日有食之。癸亥,狄青罷,以韓琦為樞密使。是夕彗滅。甲子,出恭謝樂章,肄於太常。乙亥,朝謁景靈宮,減京城系囚徒罪一等,杖笞釋之。戊寅,詔湖北招安彭仕羲。



 九月庚寅,命宰臣攝事於太廟。辛卯,恭謝天地於大慶殿,大赦,改元。丁酉,加恩百官。庚子,賜致仕卿、監以上及曾任近侍之臣粟帛酒饌。癸卯,舉行御史遷次格。自京至泗州置汴河木岸。



 十一月辛巳,王德用罷,賈昌朝為樞密使。十二月壬子,劉沆罷,以曾公亮參知政事。甲子,白虹貫日。是歲,西蕃磨氈角、占城、大食國來貢。融、桂州蠻楊克端等內附。



 二年春二月己酉,梓夔路三里村夷人寇淯井監。庚戌,錄系囚,降罪一等,徒以下釋之。遣使錄三京、輔郡系囚。壬戌,杜衍薨。澧州羅城洞蠻內寇,發兵擊走之。癸酉,王德用卒。是月,雄、霸州地震。



 三月戊寅,振河北被災民。乙未,契丹使耶律防、陳覬來求御容。戊戌,淮水溢。遣張昪
 報使契丹。癸卯,狄青卒。是月,賜禮部奏名進士、諸科及第出身八百七十七人。親試舉人免黜落始此。



 夏四月丁未,以河北地數震,遣使安撫。丙寅,幽州地大震,壞城郭,覆壓死者數萬人。己巳,邕州火峒蠻儂宗旦入寇。癸酉,以彭仕羲未降,遣官安撫湖北。



 五月庚辰,管勾麟府軍馬公事郭恩為夏人所襲,歿於斷道塢。己亥,詔舉行磨勘法。



 六月戊午,夏國主諒祚遣人來謝使吊祭。戊辰,以淑妃苗氏為賢妃。



 秋七月辛巳,詔河北諸道總管分
 遣兵官教閱所部軍。辛卯,命孫抃、張昪磨勘轉運使及提點刑獄課績。丁酉,詔陜西、河北諸路經略安撫舉文武官材堪將領者各一人。



 八月己酉,詔:每歲賜諸道節鎮、諸州錢有差。命長吏選官和藥,以救民疾。壬子,命富弼等詳定《編敕》。庚申,錄系囚,降罪一等,徒以下釋之。癸亥,策制舉人。丁卯,置廣惠倉。



 九月庚子,契丹再使蕭扈、吳湛來求御容。



 冬十月乙巳,遣胡宿報使契丹。丙午,班《祿令》。



 十一月丙申,詔三司使體量判官才否以聞。十二
 月戊申,詔:「自今間歲貢舉,天下進士、諸科解舊額之半,置明經科,罷說書舉人。」辛亥,立內降關白二府法。是歲,西蕃瞎氈並諸族、西平州黔南道王石自品、西南蕃鶼州來貢。



 三年春正月戊戌,鑿永通河。



 二月癸卯,契丹使來告其祖母哀,輟視朝七日,遣使祭奠吊慰。癸丑,錄系囚,降罪一等,徒以下釋之。



 三月甲戌,詔禮部貢舉。



 夏四月甲子,吳育卒。乙丑,罷睦親宅祖宗神御殿。丙辰,詔:「守令或貪
 恣耄昏,以弛為寬,以苛為察,以增賦斂為勞,以出入刑罰為能,而部使者莫之舉劾。自今其各思率職,毋撓權幸,毋縱有罪,以稱朕意。」五月壬申,增國子監生員。甲午,契丹遣使致其祖母遺留物。



 六月丙午,文彥博、賈昌朝罷,以富弼為昭文館大學士,韓琦同中書門下平章事、集賢殿大學士,宋庠、田況為樞密使,張昪為樞密副使。甲寅,詔學士院編國朝制誥。丁卯,交阯貢異獸。秋七月丙子,詔廣濟河溢,原武縣河決,遣官行視民田,振恤被
 水害者。癸巳,以夔州路旱,遣使安撫。



 八月己亥朔,日有食之。己未,王堯臣卒。庚申,彭仕羲率眾降。



 九月癸酉,議罷榷茶法。己丑,契丹遣使來謝。



 冬十月癸亥,除河北坊郭客戶乾食鹽錢。



 十一月癸酉,議減冗費。己丑,置都水監,罷三司河渠司。十二月己巳,詔三司歲上天下稅賦之數,三歲一會虧贏以聞。



 閏月丁卯朔,詔:「吏人及伎術官職,毋得任知州軍、提點刑獄,自軍班出至正任者,方得知邊要州軍。」丁丑,詔裁定制科及進士高第人恩數。
 庚辰,詔明年正旦日食,其自丁亥避正殿,減常膳。宴契丹使,毋作樂。壬午,錄系囚,降三京囚罪一等,徒以下釋之。是歲,安化上中下州、北遐鎮蠻人來貢。



 四年春正月丙申朔,日有食之。用牲於社。辛丑,御正殿,復常膳。以自冬雨雪不止,遣官分行京城,賜孤窮老疾錢,畿縣委令佐為糜粥濟饑。壬寅,賜在京諸軍班緡錢。頒《嘉祐驛令》。



 二月己巳,罷榷茶。庚午,廣南言交址寇欽州。乙亥,以廣惠倉隸司農寺。戊子,白虹貫日。



 三月戊戌,
 命近臣同三司減定民間科率。是月,賜進士、諸科及第出身三百三十九人。



 夏四月丁卯,詔孟冬大袷於太廟。癸酉,封柴氏後為崇義公,給田千頃,奉周室祀。丙子,復銀臺司封駁制。癸未,陳執中薨。辛卯,詔中外臣庶居室、器用、冠服、妾媵,有違常制,必罰毋貸。壬辰,錄系囚,降罪一等,徒以下釋之。大震電,雨雹。



 五月戊戌,詔:「兩制臣僚舊制不許詣執政私第,所舉薦不得用為御史,今除其法。」庚子,詔內臣員多,權罷進養子入內。壬子,遣官經界
 河北牧地,餘募民種藝。



 六月己巳,群臣請加尊號曰「大仁至治」,表五上,不許。癸酉,詔諸路經略安撫、轉運使、提點刑獄各舉本部官有行實政事者三人,以備升擢。嘗任兩府者,許舉內外官。丁丑,詔轉運司,凡鄰州饑而輒閉糶者,以違制論。辛卯,放宮女二百十四人。



 秋七月丁未,放宮女二百三十六人。



 八月乙亥,策制舉人。



 冬十月壬申,朝饗景靈宮。癸酉,大袷於太廟,大赦。詔諸路監司察士有學行為鄉里所推者,同長吏以聞。民父母年八
 十以上,復其一丁。復益州為成都府,並州為太原府。戊寅,加恩百官。



 十一月庚子,汝南郡王允讓薨。十二月丁丑,白虹貫日。是歲,唃廝囉來貢。



 五年春正月辛卯朔,白虹貫日,太白犯歲星。己亥,錄劉繼元後。



 二月壬戌,錄系囚。



 三月壬辰,詔禮部貢舉。癸巳,劉沆薨。乙未,歲星晝見。壬子,詔以蝗澇相仍,敕轉運使、提點刑獄督州縣振濟,仍察不稱職者。



 夏四月癸未,程戡罷,以孫抃為樞密副使。丙戌,命近臣同三司議均稅。



 五月戊子朔,京師民疫,選醫給藥以療之。己丑,京師地震。丁酉,詔三司置寬恤民力司。己酉,王安石召入為三司度支判官。丁巳,錄系囚,降罪一等,徒以下釋之。



 六月乙丑,詔戒上封告訐人罪或言赦前事,及言事官彈劾小過不關政體者。乙亥,遣官分行天下,訪寬恤民力事。



 秋七月癸巳,邕州言交址與甲峒蠻合兵寇邊,都巡檢宋士堯拒戰,死之,詔發諸州兵討捕。丙申,詔待制、臺諫官、正刺史以上各舉諸司使至三班使臣堪將領及行
 陣戰鬥者三人。戊戌,翰林學士歐陽修上新修《唐書》。庚戌,詔中書門下採端實之士明進諸朝,辨激巧偽者放黜之。



 八月壬申,詔求逸書。庚辰,置陜西估馬司。乙酉,罷諸路同提點刑獄使臣。丙戌,置江、湖、閩、廣、四川十一路轉運判官。



 九月己丑,太白晝見。



 冬十月乙酉,深州言野蠶成繭,被於原野。



 十一月辛卯,罷內臣寄遷法。辛丑,宋庠罷。以曾公亮為樞密使,張昪、孫抃為參知政事,歐陽修、陳升之、趙概為樞密副使。十二月己卯,蘇茂州蠻寇
 邕州。辛巳,補諸州父老百歲以上者十二人為州助教。是歲,大食國來貢。



 六年春正月乙未,許兩制與臺諫相見。



 二月丁巳,詔宗室賜名授官者,須年及十五方許轉官。乙丑,詔良民子弟或為人誘隸軍籍,自今兩月內,父母訴官者還之。丙寅,錄系囚,降罪一等,徒以下釋之。



 三月己亥,富弼以母喪去位。庚子,以富弼母喪,罷大宴。戊申,給西京周廟祭享器服。是月,賜進士、諸科及第同出身二百九十五人。



 夏四月辛酉,詔嶺南官吏死於儂賊而其家流落未能自歸者,所在給食,護送還鄉。庚辰,陳升之罷,以包拯為樞密副使。出諫官唐介、趙抃、御史範師道、呂誨。



 五月丙戌,官諸路敦遣行義文學之士七人。庚戌,錄系囚,降罪一等,徒以下釋之。分命官錄三京系囚。



 六月壬子朔,日有食之。乙丑,太白晝見。壬申,歲星晝見。丙子,以司馬光知諫院,入對。戊寅,以王安石知制誥。



 秋七月乙酉,泗州淮水溢。丙戌,詔淮南、江、浙水災,差官體量蠲稅。戊子,錄
 昭憲皇太后、孝明孝惠孝章淑德皇后家子孫,進秩授官者十有九人。癸巳,詔:「臺諫為耳目之官,乃聽險陂之人興造飛語,中傷善良,非忠孝之行也。中書門下其申儆百工,務敦行實,循而弗改者絀之。」八月乙亥,策制舉人。丁丑,詔:「諸路刺舉之官,未有以考其賢否,比令有司詳定厥制,其各務祗新書,以稱朕意。仍令考校轉運、提刑,課績院以新定條目施行。」戊寅,詔州縣長吏有清白不擾而實惠及民者,令本路監司保薦再任,政跡尤異,
 當加獎擢。



 閏月乙酉,復以成都府為劍南西川節度。庚子,以韓琦為昭文館大學士,曾公亮同中書門下平章事、集賢殿大學士,張昪為樞密使。辛丑,以胡宿為樞密副使。



 冬十月壬午,定內侍磨勘法。丙戌,詔京西、淮、浙、荊湖增置都同巡檢。壬辰,起復皇侄、前右衛大將軍、岳州團練使宗實為泰州防禦使、知宗正寺。辭以喪,不拜。



 十一月己巳,許夏國用漢衣冠。癸酉,賜昭憲皇太后家信陵坊第。戊寅,許康州刺史李樞以己官封贈父母。十二
 月丙戌,復豐州。庚寅,命諸路總管集隨軍功過簿,以備遷補。是歲,冬無冰。占城國獻馴象,安化州蠻來貢。



 七年春正月辛未,復命皇侄宗實為泰州防禦使、知宗正寺。乙亥,詔南郊以太祖配為定制。改溫成皇后廟為祠殿。



 二月己卯朔,更江西鹽法。詔開封府市地於四郊,給錢瘞民之不能葬者。癸未,錄系囚,命官錄被水諸州系囚。



 三月辛亥,詔禮部貢舉。乙卯,孫抃罷,以趙概參知政事,吳奎為樞密副使。甲子,以旱,罷大宴。乙丑,祈雨於
 西太一宮。庚午,謝雨。壬申,徐州彭城、濠州鐘離地生面十餘頃,民皆取食。



 夏四月壬午,頒《嘉祐編敕》。己丑,夏國主諒祚進馬,求賜書,詔賜《九經》,還其馬。



 五月戊午,太白晝見。庚午,包拯卒。



 六月丙子朔,歲星晝見。



 秋七月戊申,太白經天。壬子,詔季秋有事於明堂。



 八月乙亥朔,出明堂樂章,肄於太常。己卯,詔以宗實為皇子。癸未,賜名曙。丁亥,奉安真宗御容於壽星觀。庚子,以立皇子告天地宗廟諸陵。



 九月乙巳朔,以皇子為齊州防禦使,進封鉅
 鹿郡公。己酉,朝饗景靈宮。庚戌,饗太廟。辛亥,大饗明堂,奉真宗配,大赦。己未,加恩百官。



 冬十月乙亥,皇子表辭所除官,賜詔不允。丙戌,白虹貫日。乙未,太白晝見。丙申,詔內藏庫、三司共出緡錢一百萬,助糴天下常平倉。



 十二月甲午,德妃沈氏為貴妃,賢妃苗氏為德妃。丙申,幸龍圖、天章閣,召群臣宗室觀祖宗御書。又幸寶文閣,為飛白書,分賜從臣。作《觀書詩》,命韓琦等屬和,遂宴群玉殿。庚子,再召從臣於天章閣觀瑞物,復宴群玉殿。是歲,
 冬無冰。占城來貢。



 八年春正月辛亥,交址貢馴象九。



 二月癸未,帝不豫。甲申,下德音:減天下囚罪一等,徒以下釋之。丙戌,中書、樞密奏事於福寧殿之西閣。



 三月戊申,龐籍薨。癸亥,御內東門幄殿,優賜諸軍緡錢。甲子,御延和殿,賜進士、諸科及第同出身三百四十一人。辛未,帝崩於福寧殿,遺制皇子即皇帝位,皇后為皇太后,喪服以日易月,山陵制度務從儉約。謚曰神文聖武明孝皇帝,廟號仁宗。十月
 甲午,葬永昭陵。



 贊曰:仁宗恭儉仁恕,出於天性,一遇水旱,或密禱禁庭,或跣立殿下。有司請以玉清舊地為御苑,帝曰:「吾奉先帝苑囿,猶以為廣,何以是為?」燕私常服浣濯,帷帟衾裯,多用繒絁。宮中夜饑,思膳燒羊,戒勿宣索,恐膳夫自此戕賊物命,以備不時之須。大闢疑者,皆令上讞,歲常活千餘。吏部選人,一坐失入死罪,皆終身不遷。每諭輔臣曰:「朕未嘗詈人以死,況敢濫用闢乎!」至於夏人犯邊,御
 之出境;契丹渝盟,增以歲幣。在位四十二年之間,吏治若偷惰,而任事蔑殘刻之人;刑法似縱弛,而決獄多平允之士。國未嘗無弊幸,而不足以累治世之體;朝未嘗無小人,而不足以勝善類之氣。君臣上下惻怛之心,忠厚之政,有以培壅宋三百餘年之基。子孫一矯其所為,馴致於亂。《傳》曰:「為人君,止於仁。」帝誠無愧焉。



\end{pinyinscope}