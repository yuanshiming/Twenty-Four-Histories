\article{列傳第五十三}

\begin{pinyinscope}

 掌
 禹錫蘇紳王洙子欽臣胥偃柳植聶冠卿馮元趙師民張錫張揆楊安國



 掌禹錫,字唐卿,許州郾城人。中進士第,為道州司理參
 軍。試身言書判第一,改大理寺丞,累遷尚書屯田員外郎、通判並州。擢知廬州,未行,丁度薦為侍御史,上疏請嚴備西羌。時議舉兵,禹錫引周宣薄伐為得,漢武遠討為失;且建畫增步卒,省騎兵。舊法,薦舉邊吏,貪贓皆同坐。禹錫奏謂:「使貪使愚,用兵之法也。若舉邊吏必兼責士節,則莫敢薦矣。材武者孰從而進哉?」後遂更其法。



 出提點河東刑獄。杜衍薦,召試,為集賢校理,改直集賢院兼崇文院檢討。歷三司度支判官、判理欠司、同管勾國
 子監。歷判司農、太常寺。數考試開封國學進士,命題皆奇奧,士子憚之,目為「難題掌公」。遷光祿卿,改直秘閣。英宗即位,自秘書監遷太子賓客。御中劾禹錫老病不任事,帝憐其博學多記,令召至中書,示以彈文。禹錫惶怖自請,遂以尚書工部侍郎致仕,卒。



 禹錫矜慎畏法,居家勤儉,至自舉幾案。嘗預修《皇祐方域圖志》、《地理新書》,奏對帝前,王洙推其稽考有勞,賜三品服。及校正《類篇》、《神農本草》,載藥石之名狀為《圖經》。喜命術,自推直生日,年
 庚寅,日乙酉,時壬午,當《易》之《歸妹》、《困》、《震》初中末三卦。以世應飛伏納五甲行軌析數推之,卦得二十五少分,三卦合七十五年約半,祿秩算數,盡於此矣。著《郡國手鑒》一卷,《周易集解》十卷。好儲書,所記極博,然迂漫不能達其要。常乘駑馬,衣冠污垢,言語舉止多可笑,僚屬或慢侮之,過閭巷,人指以為戲云。



 蘇紳,字儀甫,泉州晉江人。進士及第。歷宜、復、安三州推官,改大理寺丞。母喪,寓揚州。州將盛度以文學自負,見
 其文,大驚,自以為不及,由是知名。再遷太常博士,舉賢良方正科,擢尚書祠部員外郎、通判洪州,徙揚州。歸,上十議,進直史館,為開封府推官、三司鹽鐵判官。時眾星西流,並代地大震,方春而雷,詔求直言,紳上疏極言時事。



 安化蠻蒙光月率眾寇宜州,敗官軍,殺鈐轄張懷志等六人。紳上言曰:



 國家比以西北二邊為意,而鮮復留意南方,故有今日之患,誠不可不慮也。臣頃從事宜州,粗知本末。安化地幅員數百里,持兵之眾,不過三四千
 人。然而敢肆侵擾,非特恃其險絕,亦由往者守將失計,而國家姑息之太過也。



 向聞宜州吏民言,祥符中,蠻人騷動,朝廷興兵討伐。是時,唯安撫都監馬玉勒兵深入,多殺所獲。知桂州曹克明害其功,累移文止之,故玉志不得逞。蠻人畏伏其名,至今言者猶惜之。使當時領兵者皆如玉,則蠻當殄滅,無今日之患矣。至使乘隙蹂邊,屠殺將吏,其損國威,無甚於此。朝廷儻不以此時加兵,則無以創艾將來,而震疊荒裔。彼六臣者,雖不善為馭,
 自致喪敗,然銜冤負恥,當有以刷除。



 臣觀蠻情,所恃者地形險□厄,據高臨下,大軍難以並進。然其壤土磽確,資蓄虛乏,刀耕火種,以為餱糧。其勢可以緩圖,不可以速取;可以計覆,不可以力爭。今廣東西教閱忠敢澄海、湖南北雄武等軍,皆慣涉險阻。又所習兵器,與蠻人略同。請速發詣宜州策應,而以他兵代之。仍命轉運使備數年軍食,今秋、冬之交,嵐氣已息,進軍據其出路,轉粟補卒,為曠日持久之計。伺得便利,即圖深入,可以傾蕩巢
 穴,杜絕蹊徑。縱使奔迸林莽,亦且壞其室廬,焚其積聚,使進無鈔略之獲,退無攻守之備。然後諭以國恩,許以送款,而徙之內郡,收其土地,募民耕種,異時足以拓外夷為屏蔽也。



 仍詔旁近諸蠻,諭以朝廷討叛之意,毋得相為聲援;如獲首級,即優賞以金帛。計若出此,則不越一年,逆寇必就殄滅。況廣西溪峒、荊湖、川峽蠻落甚多,大抵好為騷動。因此一役,必皆震讋,可保數十年無俶擾之虞矣。



 朝廷施用其策,遣馮伸己守桂州經制之,蠻
 遂平。



 又陳便宜八事:



 一曰重爵賞。先王爵以褒德,祿以賞功,名以定流品,位以民才實。未有無德而據高爵,無功而食厚祿,非其人而受美名,非其才而在顯位者。不妄與人官,非惜寵也,蓋官非其人,則不肖者逞。不妄賞人,非愛財也,蓋實非其人,則徼幸者眾。非特如此而已,則又敗國傷政,納侮詒患。上干天氣,下戾人心,災異既興,妖孽乃見。故漢世五侯同日封,天氣赤黃,及丁、傅封而其變亦然。楊宣以為爵土過制,傷亂土氣之祥也。



 二
 曰慎選擇。今內外之臣,序年遷改,以為官濫,而復有論述微效,援此希進者。朝臣則有升監司,使臣則有授橫行。不問人材物望,可與不可,並甄祿之。不三數年,坐致清顯。如此不止,則異日必以將相為賞矣。



 三曰明薦舉。今有位多援親舊,或迫於權貴,甚非薦賢助國,為官擇人之道。若要官闕人,宜如祖宗故事,取班簿親擇五品以上清望官,各令舉一二人,述其才能德業,陛下與執政大臣,參驗而擢之。試而有效,則先賞舉者,否則黜責
 之。如此,則人人得以自勸。又選人條約太嚴。舊制,三人保者,得選京官,今則五人。舊轉運使、提點刑獄率當三人,今止當一人。舊大兩省官歲舉五人,今才舉三人;升朝官舉三人,今則舉一人。舊不以在任及所統屬皆得奏舉,今則須在任及統屬方許論薦。驅馳下僚,未免有賢愚同滯之歡也。



 四曰異服章。朝班中執技之人與丞郎清望同佩金魚,內侍班行與學士同服金帶,豈朝廷待賢才、加禮遇之意?宜加裁定,使採章有別,則人品定
 而朝儀正矣。



 五曰適才宜。古者自黃、散而下,及隋之六品,唐之五品,皆吏部得專去留。今審官院、流內銓,則古之吏部;三班院,古之兵部。不問官職之閑劇,才能之長短,惟以資歷深淺為先後,有司但主簿籍而已。欲賢不肖有別,不可得也。太宗皇帝始用趙普議,置考課院以分中書之權,今審官是也,其職任豈輕也哉?宜擇主判官,付之以事權,責成其選事。若以為格例之設久,不可遽更。或有異才高行,許別論奏,如寇準判銓,薦選人錢
 若水等三人,並遷朝官為直館。其非才亦許奏殿,如唐盧從願為吏部,非才實者並令罷選,十不取一是也。



 六曰擇將帥。漢制邊防有警,左右之臣,皆將帥也。唐室文臣,自員外、郎中以上,為刺史、團練、防禦、觀察、節度等使,皆是養將帥之道,豈嘗限以文武?比年設武舉,所得人不過授以三班官,使人監臨,欲圖其建功立事,何可得也?臣僚舉換右職者,必人才弓馬兼書算策略,亦責之太備。宜使有材武者居統領之任,有謀畫者任邊防之
 寄,士若素養之,不慮不為用也。



 七曰辨忠邪。夫忠賢之嫉奸邪,謂之去惡,惡不去則害政而傷國。奸邪陷忠良,謂之蔽明,明不蔽,則無以稔其慝而肆其毒矣。忠邪之端,惟人主深辨之。自古稱帝之聖者,莫如唐堯,然而四兇在朝,圮毀善類。好賢之甚者,莫如漢文,然而絳、灌在列,不容賢臣。願監此而不使譽毀之說得行,愛憎之徒逞志,則忠賢進而邪慝消矣。



 八曰修預備。國家承平,天下無事將八十載,民食宜足而不足,國用宜豐而未豐,
 甚可怪也。往者明道初,蟲螟水旱,幾遍天下。始之以饑饉,繼之以疾疫,民之轉流死亡,不可勝數。幸而比年稍稔,流亡稍復,而在位未嘗留意於備預之道,莫若安民而厚利,富國而足食。欲民之安,則不之擇守宰、明教化;欲民之利,則為之去兼並、禁游末。恤其疾苦,寬其徭役,則民安而利矣。欲國之富,則必崇節儉,敦質素,蠲浮費。欲食之足,則省官吏之冗,去兵釋之蠹,絕奢靡之弊,塞凋偽之原,則國食足矣。民足於下,國富於上,雖有災沴,不
 足憂也。



 書奏,帝嘉納之。進史館修撰,擢知制誥,入翰林為學士。再遷尚書禮部郎中。



 王素、歐陽修為諫官,數言事,紳忌之。會京師閔雨,紳請對,言:「《洪範》五事,『言之不從,是謂不乂,厥咎僭,厥罰常□昜。』蓋言國之號令,不專於上,威福之柄,或移臣下,虛嘩憤亂,故其咎僭。」又曰:「庶位逾節茲謂僭。刑賞妄加,群陰不附,則陽氣勝,故其罰常□昜。今朝廷號令,有不一者,庶位有逾節而陵上者,刑賞有妄加於下者,下人有謀而僭上者。此而不思,雖禱於上
 下神祇,殆非天意。」紳意以指諫官。諫官亦言紳舉御史馬端非其人,改龍圖閣學士、知揚州,復為翰林學士、史館修撰、權判尚書省。



 紳銳於進取,善中傷人。陰中王德用,其疏至有「宅枕乾岡,貌類藝祖」之語,帝惡之,匿其疏不下。遂出紳,以吏部郎中改侍讀學士、集賢殿修撰、知河陽,徙河中。未行感疾,為醫者藥所誤,猶力疾笞之,已而卒。



 紳博學多知,喜言事。嘗請罷連日視朝,復唐制朔望喚仗入閣,間開便殿,延對輔臣;寬制舉科格,以收才
 傑;選命諫員,勿侵御史職事。趙元昊反,請詔邊帥為入討之計,且曰:「以十年防守之費,為一歲攻取之資;不爾,則防守之備,不止於十年矣。」又曰:「今邊兵止備陜西,恐賊出不意窺河東,即麟、府不可不慮,宜稍移兵備之。鄜、延與原州、鎮戎軍,皆當賊沖,而兵屯從寡不均。或寇原州、鎮戎軍,則鄜、延能應援。陜西屯卒太多,永興為關、隴根本,而戍者不及三千。宜留西戍之兵,壯關中形勢,緩急便於調發。郡縣備盜不謹,請增尉員,益弓手藉。」其論
 利害甚多。



 紳與梁適同在兩禁,人以為險詖,故語曰:「草頭木腳,陷人倒卓。」子頌,別有傳。



 王洙,字原叔,應天宋城人。少聰悟博學,記問過人。初舉進士,與郭稹同保。人有告稹冒祖母禫,主司欲脫洙連坐之法,召謂曰:「不保,可易也。」洙曰:「保之,不願易。」遂與稹俱罷。再舉,中甲科,補舒城縣尉。坐覆縣民鐘元殺妻不實免官



 後調富川縣主簿。晏殊留守南京,厚遇之,薦為府學教授。召為國子監說書,改直講。校《史記》、《漢書》,擢史
 館檢討、同知太常禮院,為天章閣侍講。專讀寶訓、要言於邇英閣。累遷太常博士、同管勾國子監,預修《崇文總目》成,遷尚書工部員外郎。修《國朝會要》,加直龍圖閣、權同判太常寺。坐赴進奏院賽神與女妓雜坐,為御史劾奏,黜知濠州,徙襄州。



 會貝卒叛,州郡皆恟□,襄佐史請罷教閱士,不聽。又請毋給真兵,洙曰:「此正使人不安也。」命給庫兵,教閱如常日,人無敢嘩者。



 徙徐州。時京東饑,朝廷議塞商胡,賦楗薪,輸半而罷塞。洙命更其餘為穀
 粟,誘願輸者以餔流民,因募其壯者為兵,得千餘人,盜賊衰息。有司上其最,為京東第一,徙亳州。復為天章閣侍講、史館檢討。



 帝將祀明堂,宋祁言:「明堂制度久不講,洙有《禮》學,願得同具其儀。」詔還洙太常,再遷兵部員外郎,命撰《大饗明堂記》。除史館修撰,遷知制誥。詔諸儒定雅樂,久未決。洙與胡瑗更造鐘磬,而無形制容受之別。皇祐五年,有事於南郊,勸上用新藥,既而議者多非之,卒不復用。



 夏竦卒,賜謚文獻。洙當草制,封還其目曰:「臣
 下不當與僖祖同謚。」因言:「前有司謚王溥為文獻,章得像為文憲,字雖異而音同,皆當改。」於是太常更謚竦文莊,而溥、得像皆易謚。



 嘗使契丹,至靴澱。契丹令劉六符來伴宴,且言耶律防善畫,向持禮南朝,寫聖容以歸,欲持至館中。洙曰:「此非瞻拜之地也。」六符言恐未得其真,欲遣防再往傳繪,洙力拒之。



 嘗言天下田稅不均,請用郭諮、孫琳千步開方法,頒州縣以均其稅。貴妃張氏薨,治喪皇儀殿,追冊溫成皇后。洙鉤摭非禮,陰與內侍石
 全彬附會時事。陳執中、劉沆在中書,喜其助己,擢洙為翰林學士。既而溫成即園立廟,且欲用樂,詔禮院議。禮官論未一,洙令禮直官填印紙,上議請用樂,朝廷從其說。禮官吳充、鞠直卿移文開封府,治禮直官擅發印紙罪。知府蔡襄釋不問,而諫官範鎮疏禮院議園陵前後不一,請詰所以。御史繼論之不已,宰相意充等風言者,皆罷斥。



 既而洙以兄子堯臣參知政事,改侍讀學士兼侍講學士。罷一學士,換二學士且兼講讀,前此未嘗有
 也。是歲,京東、河北秋大稔。洙言:「近年邊糴,增虛價數倍,雖復稍延日月之期,而終償以實錢及山澤之物,以致三司財用之蹙。請借內藏庫禁錢,乘時和糴京東、河北之粟,以供邊食,可以坐紓便糴之急。」又言:「近時選諫官、御史,凡執政之臣嘗所薦者,皆不與選。且士之飭身勵行,稍為大臣所知,反置而不用,甚可惜也。」及得疾逾月,帝遣使問:「疾少間否,能起侍經席乎?」時不能起矣。



 洙泛覽傳記,至圖緯、方技、陰陽、五行、算數、音律、詁訓、篆隸之
 學,無所不通。及卒賜謚曰文,御史吳中復言官不應得謚,乃止。預修《集韻》、《祖宗故事》、《三朝經武聖略》、《鄉兵制度》,著《易傳》十卷、雜文千有餘篇。子欽臣。



 欽臣字仲至,清亮有志操,以文贄歐陽修,修器重之。用蔭入官,文彥博薦試學士院,賜進士及第。歷陜西轉運副使。元祐初,為工部員外郎。奉使高麗,還,進太僕少卿,遷秘書少監。開封尹錢勰入對,哲宗言:「此閱書詔,殊不滿人意,誰可為學士者?」勰以欽臣對。哲宗曰:「章惇不喜。」
 乃以勰為學士,欽臣領開封。改集賢殿修撰、知和州。徙饒州,斥提舉太平觀。徽宗立,復待制、知成德軍。卒,年六十七。



 欽臣平生為文至多,所交盡名士,惟嗜古,藏書數萬卷,手自讎正,世稱善本。



 胥偃,字安道,潭州長沙人。少力學,河東柳開見其所為文曰:「異日必得名天下。」舉進士甲科,授大理評事、通判湖、舒二州,直集賢院、同判吏部南曹、知太常禮院,再遷太常丞、知開封縣。



 與御史高升試府進士,既封彌卷首,
 輒發視,擇有名者居上。降秘書省著作佐郎、監光化軍酒。起通判鄧州,復太常丞。林特知許州,闢通判州事,徙知漢陽軍。還判三司度支勾院、修起居注。累遷商書刑部員外郎,遂知制誥,遷工部郎中,入翰林為學士,權知開封府。



 忻州地震,偃以為:「地震,陰之盛。今朝廷政令,不專上出,而後宮外戚,恩澤日蕃,此陽不勝陰之效也。宜選將練師,以防邊塞。」趙元昊朝貢不至,偃曰:「遽討之,太暴。宜遣使問其不臣狀,待其辭屈而後加兵。則其不直
 者在彼,而王師之出有名矣。」又奏:「戍兵代還,宜如祖宗制,閱其藝後殿次進之。」



 會有衛卒賂庫吏求揀冬衣,坐系者三十餘人。時八月,霜雪暴至。偃推《洪範》「急,恆寒若」之咎,請從未減,奏可。西塞用兵,士卒妻子留京師者犯法當死,帝不忍用刑,或欲以毒置飲食中,令得善死。偃極言其不可,帝亦悔而止。宦人程智誠與三班使臣馮文顯八人抵罪,帝使赦智誠三人,而文顯五人坐如法。偃曰:「恤近遺遠,非政也,況同罪異罰乎?」詔並釋之。未幾,
 卒。



 偃未仕時,家有良田數十頃,既貴,悉以予族人。初,天下職田,無日月之限,而赴官者多以前後為斷。偃請水陸田各限以月,因著為令。嘗與謝絳受詔試中書吏,而大臣有以簡屬偃者,偃不敢發視,亟焚之。歐陽修始見偃,偃愛其文,召置門下,妻以女。偃糾察刑獄,範仲淹尹京,偃數糾其立異不循法者。修方善仲淹,因與偃有隙。



 子元衡,有學行,能自立,為尚書都官員外郎,並其子茂諶咸早卒。偃妻,直史館刁約之妹。與元衡婦韓、茂諶婦
 謝皆寡居丹陽,閨門有法,江、淮人至今稱之。



 柳植,字子春,真州人。少貧,自奮為學,從祖開頗器之。舉進士甲科,為大理評事、通判滁州。遷著作郎、直集賢院、知秀州。除三司度支判官,出知宣州。擢修起居注、知制誥。求知蘇州,徙杭州,累遷尚書工部員外、郎中。召還,為翰林學士,遷諫議大夫、御史中丞。既而以疾辭,改侍讀學士、知鄧州。遷給事中、移穎州。



 先是,張海、郭邈山叛京西,攻掠縣鎮,而光化卒邵興亦率其徒作亂,逐官吏,取
 庫兵而去。時植領京西安撫使,坐賊發部中不能察,降右諫議大夫、知黃州。久之,復其官。坐薦張得一落職,未幾,復其職如故。歷知壽、亳、蔡、揚四州,分司西京,遂致仕。累遷吏部侍郎,卒。



 植平居畏慎,寡言笑,所至官舍,蔬果不輒採,家無長物,時稱其廉。



 聶冠卿,字長孺,歙州新安人。五世祖師道,楊行密版奏,號問政先生,鴻臚卿。冠卿舉進士,授連州軍事推官。楊億愛其文章,於是大臣交薦,召試學士院,校勘館閣書
 籍。遷大理寺丞,為集賢校理、通判蘄州。坐嘗校《十代興亡論》謬誤落職。



 再遷太常博士,復集賢校理。言:「天下旬奏獄,雖笞、杖並覆,而徒、流不系獄者乃不以聞,非所以矜慎刑罰之意。請自今罷覆笞、杖罪,自徒以上雖不系獄,亦奏覆。」從之。判登聞鼓院,歷開封府判官、三司鹽鐵度支判官,同修起居注。累遷尚書工部郎中。



 初,翰林侍講學士馮元修大樂,命冠卿檢新閱事跡。又預選《景祐廣樂記》,特遷刑部郎中、直集賢院。以兵部郎中、知制誥判
 太常禮院,糾察刑獄。奉使契丹,其主謂曰:「君家先世奉道,子孫固有昌者。」嘗觀所著《蘄春集》,詞極清麗,因自擊球縱飲,命冠卿賦詩,禮遇甚厚。還,同知通進銀臺司、審刑院,入翰林為學士。母亡,起復,判昭文館。未幾,兼侍讀學士。



 冠卿每進讀《左氏春秋》,必引尊王黜霸之義以諷。一日,墜笏上前,帝憫冠卿喪毀羸瘠,既退,賜禁中湯劑。未幾,告歸葬親,至揚州卒。詔以其弟太常博士世卿通判宣州。初,世卿監延豐倉,掘地得古磚,有隸書字,半漫
 滅。其可辨者云:「公先世餌霞棲雲,高尚不仕,累石於江濱。」又云:「昭王大丞相聶。」又云:「水龍夜號,夕雞駭飛。其年九月十二日卒,年五十有五。」冠卿始見而惡之,至是,校所卒歲月及其享年,無少異者。



 冠卿嗜學好古,手未嘗釋卷,尤工詩,有《蘄春集》十卷。



 論曰:學士大夫異於眾人者,以操行修爾。《詩》曰:「靡不有初,鮮克有終。」君子不可不慎也。禹錫迂陋,不知止足之戒,取譏當世。紳急進喜傾。洙阿諛附會,晚節污變,卒忘
 平生之學。偃之恬正,植之廉介,冠卿之雅尚,其列侍從,庶亡愧焉。



 馮元,字道宗。高祖禧,唐末官廣州,以術數仕劉氏。傳三世至父邴,廣南平,入朝為保章正。元幼從崔頤正、孫奭為《五經》大義,與樂安孫質、吳陸參、譙夏侯圭善,群居講學,或達旦不寢,號「四友」。進士中第,授江陰尉。



 時詔流內銓取明經者補學官,元自薦通《五經》。謝泌笑曰:「古治一經,或至皓首,子尚少,能盡通邪?」對曰:「達者一以貫之。」更
 問疑義,辨析無滯。補國子監講書,遷大理評事,擢崇文院檢討兼國子監直講。王旦聞其名,嘗令說《論語》、《老子》,群子弟侍聽,因薦之。



 真宗試進士殿中,召元講《易》。元進說曰:「地天為《泰》者,以天地之氣交也。君道至尊,臣道至卑,惟上下相與,則可以輔相天地,財成萬化。」帝悅。未幾,遷太子中允、直龍圖閣,詔預內朝,直龍圖閣預內朝自此始。



 天禧初,數與查道、李虛己、李行簡入講《易》於宣和門北閣。遷太常丞兼判禮部、吏部南曹。皇子為壽春郡
 王,王旦又薦元宜講經資善堂。帝以元少,更用崔遵度。會遵度卒,擢左正言兼太子右諭德。



 仁宗即位,遷戶部員外郎,為直學士兼侍講。與孫奭以經術並進講論,自是仁宗益響學。歷會靈觀副使、知通進銀臺司、判登聞檢院、同判國子監。故事,國子監多宿儒典領,後頗用公卿子弟,任均管庫。及奭、元並命,士議悅服。同知貢舉,進龍圖閣學士,預修《三朝正史》。為翰林學士、判都省三班院、史館修撰、判流內銓兼群牧使,四遷給事中。



 明道元
 年,當監護宸妃葬事。及帝親政,追冊宸妃為莊懿皇后,改葬永定陵。既發壙而流泉沮洳,言者以監護不職,罷翰林學士、知河陽。王曾為言元東朝舊臣,不宜以細故棄外。即召為翰林侍講學士,遷禮部侍郎、知審官院,復判禮院、國子監。上《金華五箴》,賜書褒答。修《景祐廣樂記》,書成,遷戶部侍郎。足疾氣駁,屬李淑、宋祁為銘志。卒,贈本部尚書,謚章靖。



 元性簡厚,不治聲名,非慶吊未嘗過謁二府。執親喪,自括發至祥練,皆案禮變服,不為世俗
 齋薦,遇祭日,與門生對坐,誦說《孝經》而已。多識古今臺閣品式之事,尤精《易》。



 初,七歲,方讀《易》,每夜夢異人,以紺蓮華與元吞之,且曰:「善讀此,後必貴顯。」元且老,率三日一誦《易》。無子,以兄之子譓為後。



 趙師民,字周翰,青州臨淄人。九歲能屬文,舉進士第,孫奭闢兗州說書,領諸城主簿。師民學問精博,奭自以為不及。夏竦尤所奇重,稱為「盛德君子」,論其文行,願回兩子恩,授以京秩。除齊州推官、青州教授,更天平軍節度
 推官。



 年五十來京師,近臣張觀、宋郊、王堯臣、龐籍、韓琦、明鎬列薦,為國子監直講,兼潤、冀二王宮教授。改著作佐郎、宗正寺主簿,加崇文院檢討、崇政殿說書,遷宗正丞。



 會趙元昊反,罷進講。師民上書陳十五事:一曰咨輔相,二曰命將帥,三曰柬侍從,四曰擇守宰,五曰治軍旅,六曰修邊防,七曰求諫諍,八曰延講誦,九曰革貢舉,十曰久官政,十一曰謹財用,十二曰不遺年,十三曰容誹謗,十四曰除忌諱,十五曰慎出令。因獻《勸講箴》。明年春,
 帝遂御迎陽門,召近臣觀圖畫,復命講讀經史。師民見朝廷厭兵,屈意以招元昊,內不能平。乃上言請任方面,以圖報效。遷天章章閣侍講、同知貢舉,進待制、同判宗正寺。



 嘗講《詩》「如彼泉流」,曰:「水之初出,喻王政之發。順行則通,通故清潔;逆亂則壅,壅故濁敗。賢人用,則王政通而世清平;邪人進,則王澤壅而世濁敗。幽王失道,用邪絀正,正不勝邪,雖有善人,不能為治,亦將相牽而淪於污濁也。」帝曰:「水何以喻政?」對曰:「水者,順行而潤下,利萬物,
 故以喻政,此於比興,義最大。」



 後講《論語》,問「修文德」,曰:「文者,經天緯地之總稱。君人之道,撫之以仁,制之以義,接之以禮,講之以信,皆是。」帝曰:「然其所先者,無若信也。」曰:「信者,天下之大本,仁義禮樂,皆必由之,此實至道之要。」復問「鉆燧改火」,曰:「古之聖王,舉動必順天時,所以四時變,火隨木色。近世漸務茍簡,以為非治具而遂廢之,至其萬事皆不如古。」又問:「子夏、子張所言交道孰勝?」曰:「聖哲之道,含覆廣大,與天地參。善者有以進德,惡者俾之
 改行。子張之言為優。」



 他日讀《漢記》,問長安城,眾莫能知,共推師民。因陳自古都雍年世,舊址所在,若畫諸掌。帝悅曰:「何其所記如此!」在經筵十餘年,甚見器異。嘗盛夏屬疾家居,帝飛白書團扇為「和平」字,賜以寄意。



 累請補郡,除龍圖閣直學士、知耀州。帝自寫詩寵行,目以「儒林舊德」。將行,上疏曰:



 近睹太陽食於正朔,此雖陰陽之事,亦慮是天意欲以感動聖心。臣非瞽史,不知天道,但率愚意言之。其月在亥,亥為水,水為正陰。其日在丙,丙
 為正陽。月掩日,陰侵陽,下蔽上之象也。《詩》曰:「十月之交,朔日辛卯。」又曰:「彼月而微,此日而微。」謂以陰奸陽,失其敘也。又曰:「百川沸騰,山塚崒崩。高岸為谷,深谷為陵。」謂下陵上,侵其權也。又曰:「皇父卿士,番惟司徒。家伯維宰,中允膳夫。聚子內史,蹶維趣馬,楀維師氏。」謂大小之臣,有不得其人者也。宗周之間,時王失德。今而引喻,蓋事有所譬,固當不諱。



 凡天之示象,由人君有失,不然,則下蔽其上。古人君之失,不過暴虐怠慢,奢侈縱放,不師古始。
 舍是,何失道之有?今聖心慈仁恭勤,儉約自檢,動循典禮,如此自非下蒙上、邪撓正,使主恩不下究,而誰之咎歟?望陛下朝夕咨於丞弼心膂之臣,洎左右近侍耳目之官。其忠而純者,與之慎柬內外百執事及州縣牧宰,使主恩究於下,不為群邪所蔽塞,則億兆之幸也。



 三遷刑部郎中,復領宗正,卒。



 師民淳靜剛敏,舉止凝重。幼喪父,哀感,不畜婢妾,年四十四始婚。志尚清遠,專以讀書為事。性極慈恕,勤於吏治,政有惠愛。嘗奏蠲陜西旱租。
 又欲論榷酤諸敝,會仁宗不豫而止。常患近世官失其守,作《正官名》,議多不載。有集三十卷。子彥若,試中書舍人。



 張錫,字貺之,其先京兆人。曾祖山甫,嘗從唐僖宗入蜀,蜀平,徙家漢陽。錫進士甲科,為試秘書省校書郎、知南昌縣。遷著作郎、知新州。初建學於州,自是人始知學。再遷太常博士、監染院。詔選能吏治畿縣,乃以錫知東明。始至,令其下曰:「吾所治者三:恃力、恃富、恃贖者,吾所先
 也。」歲中以治跡聞。樞密直學士李及薦為監察御史。丁謂貶崖州,議還內地。錫疏謂:「奸邪弄國,本與天下共棄之;今復還,是違天下意。」由是止徙雷州。



 王清昭應宮災,連系甚眾。錫言:「天災反以罪人,恐重天怒,願修德以應之。」會論者眾,獄遂解。遷殿中侍御史,權三司鹽鐵判官,出為荊湖北路轉運使,改尚書兵部員外郎,還判度支勾院,為京東轉運使。淄、青、齊、濮、鄆諸州人冒耕河壖地,數起爭訟。錫命籍其地,收租絹歲二十餘萬,訟者亦息。
 判鹽鐵勾院,為河北轉運使,改江、淮制置發運使,召兼侍御史知雜事、判大理寺、權知諫院,安撫利、夔路。歷度支、鹽鐵副使。喪母,起復,擢天章閣待制、知河中府,累遷右司郎中,以龍圖閣直學士知滑州,遷右諫議大夫、知審官院。進翰林侍讀學士、判太常寺、國子監。卒,贈尚書工部侍郎。



 錫淳重清約,雖貴,奉養如少賤時。讀書老而彌篤。初,舉廣文館進士,考官任隨以為第一,及隨死,無子,錫屢賙其家。



 張揆字貫之,其先範陽人,後徙齊州。擢進士第,歷北海縣尉,改大理寺丞。以疾解官,十年不出戶。讀《易》,因通揚雄《太玄經》。陳執中安撫京東,薦揆經明行淳,召為國子監直講,徙諸王府侍講。以尚書度支員外郎直史館、荊王府記室參軍。府罷,權三司戶部判官。上所著《太玄集解》數萬言。詔對邇英閣,令揲耆,得斷首,且言:「斷首準《易》之《□》,蓋以陽剛決陰柔,君子進、小人退之象。」仁宗悅。擢天章閣待制兼侍讀,累遷右諫議大夫,進龍圖閣直學
 士、給事中、判太常寺。一日,進讀漢《馬後傳》。至服大練、抑止外家,因言:「今妃族太盛,不可不裁損,使保其家。」帝嘉納之。詔改王溥謚,有議欲為文忠者,揆曰:「溥,周之宰相,國亡不能死,安得為忠?」乃謚為文康。加翰林侍讀學士、知審刑院,出知齊州。卒,贈尚書禮部侍郎。



 揆性剛狷少容,闊於世務,然好讀書,老而不倦。與弟掞相友愛,掞,為龍圖閣直學士。



 楊安國字君倚,密州安丘人。父光輔,居馬耆山,學者多
 從受經,州守王博文薦為太學助教。孫奭知兗州,又薦為太常寺奉禮郎,州學講書。既而奭與馮元薦安國為國子監直講,並召光輔至。仁宗命說《尚書》,光輔曰:「堯、舜之事,遠而未易行,願講《無逸》一篇。」時年七十餘矣,而論說明暢。帝悅,欲留為學官,固辭,以國子監丞老於家。



 安國《五經》及第,為枝江縣尉,後遷大理寺丞。光輔教授兗州,請監兗州酒稅,徙監益州糧料院,入為國子監直講,景祐初,置崇政殿說書,安國以國子博士預選。久之,進
 天章閣侍講、直龍圖閣,遂為天章閣待制、龍圖閣直學士,皆兼侍講。進翰林侍講學士,歷判尚書刑部、太常寺,糾察在京刑獄,累遷給事中。年七十餘,卒,贈尚書禮部侍郎。



 安國講說,一以注疏為主,無他發明,引喻鄙俚,世或傳以為笑。尤喜緯書及注疏所引緯書,則尊之與經等。在經筵二十七年,仁宗稱其行義淳質,以比先朝崔遵度。



 嘗講《易》至《鼎卦》,帝問:「九四象如何?」安國對:「九四上承至尊,上應初爻,行重非據,故折足覆餗。亦猶任得其
 人,則雖重可勝,非其人,必有顛覆之患。」帝稱善。又嘗講《周官》至「大荒大札,則薄征緩刑」,因進言曰:「古所謂緩刑,乃貰過誤之民爾。今眾持兵仗取民廩食,一切寬之,恐無以禁奸。」帝曰:「不然,天下皆吾赤子,迫於餓莩。至起為盜。州縣既不能振恤,乃捕而殺之。不亦甚乎。」嘗請書《無逸篇》於邇英閣之後屏,帝曰:「朕不欲背聖人之言,」,命蔡襄書《無逸》、王洙書《孝經》四章列置左右。



 論曰:馮元質直博雅,有古君子之風,歐陽修稱師民醇
 儒碩學,在仁宗時,並繇宿望,先後執經勸講,庶有所補益矣。張錫清慎斂晦,晚始見知。揆及安國父子俱侍經幄,考求其說,亡過人者。夫博習修潔之士,潛德隱行,不聞於世者多矣。由是言之,士遇不遇,豈非命哉!



\end{pinyinscope}