\article{列傳第一百六十}

\begin{pinyinscope}

 ○辛
 棄疾何異劉宰劉爚柴中行李孟傳



 辛棄疾,字幼安,齊之歷城人。少師蔡伯堅,與黨懷英同學,號「辛黨」。始筮仕,決以蓍,懷英遇《坎》,因留事金,棄疾得《
 離》,遂決意南歸。



 金主亮死,中原豪傑並起。耿京聚兵山東,稱天平節度使,節制山東、河北忠義軍馬,棄疾為掌書記,即勸京決策南向。僧義端者,喜談兵,棄疾間與之游。及在京軍中,義端亦聚眾千餘,說下之,使隸京。義端一夕竊印以逃,京大怒,欲殺棄疾。棄疾曰:「丐我三日期,不獲,就死未晚。」揣僧必以虛實奔告金帥,急追獲之。義端曰:「我識君真相,乃青兕也,力能殺人,幸勿殺我。」棄疾斬其首歸報,京益壯之。



 紹興三十二年,京令棄疾奉表
 歸宋,高宗勞師建康,召見,嘉納之,授承務郎、天平節度掌書記,並以節使印告召京。會張安國、邵進已殺京降金,棄疾還至海州,與眾謀曰:「我緣主帥來歸朝,不期事變,何以復命?」乃約統制王世隆及忠義人馬全福等徑趨金營,安國方與金將酣飲,即眾中縛之以歸,金將追之不及。獻俘行在,斬安國於市。仍授前官,改差江陰僉判。棄疾時年二十三。



 乾道四年,通判建康府。六年,孝宗召對延和殿。時虞允文當國,帝銳意恢復,棄疾因論南
 北形勢及三國、晉、漢人才,持論勁直,不為迎合。作《九議》並《應問》三篇、《美芹十論》獻於朝,言逆順之理,消長之勢,技之長短,地之要害,甚備。以講和方定,議不行。遷司農寺主簿,出知滁州。州罹兵燼,井邑凋殘,棄疾寬征薄賦,招流散,教民兵,議屯田,乃創奠枕樓、繁雄館。闢江東安撫司參議官。留守葉衡雅重之,衡入相,力薦棄疾慷慨有大略。召見,遷倉部郎官、提點江西刑獄。平劇盜賴文政有功,加秘閣修撰。調京西轉運判官,差知江陵府兼
 湖北安撫。



 遷知隆興府兼江西安撫,以大理少卿召,出為湖北轉運副使,改湖南,尋知潭州兼湖南安撫。盜連起湖湘,棄疾悉討平之。遂奏疏曰:「今朝廷清明,比年李金、賴文政、陳子明、陳峒相繼竊發,皆能一呼嘯聚千百,殺掠吏民,死且不顧,至煩大兵翦滅。良由州以趣辦財賦為急,吏有殘民害物之狀,而州不敢問,縣以並緣科斂為急,吏有殘民害物之狀,而縣不敢問。田野之民,郡以聚斂害之,縣以科率害之,吏以乞取害之,豪民以兼
 並害之,盜賊以剽奪害之,民不為盜,去將安之?夫民為國本,而貪吏迫使為盜,今年剿除,明年劃蕩,譬之木焉,日刻月削,不損則折。欲望陛下深思致盜之由,講求弭盜之術,無徒恃平盜之兵。申飭州縣,以惠養元元為意,有違法貪冒者,使諸司各揚其職,無徒按舉小吏以應故事,自為文過之地。」詔獎諭之。



 又以湖南控帶二廣,與溪峒蠻獠接連,草竊間作,豈惟風俗頑悍,抑武備空虛所致。乃復奏疏曰:「軍政之敝,統率不一,差出占破,略無
 已時。軍人則利於優閑窠坐,奔走公門,茍圖衣食,以故教閱廢弛,逃亡者不追,冒名者不舉。平居則奸民無所忌憚,緩急則卒伍不堪征行。至調大軍,千里討捕,勝負未決,傷威損重,為害非細。乞依廣東摧鋒、荊南神勁、福建左翼例,別創一軍,以湖南飛虎為名,止撥屬三牙、密院,專聽帥臣節制調度,庶使夷獠知有軍威,望風懾服。」



 詔委以規畫,乃度馬殷營壘故基,起蓋砦柵,招步軍二千人,馬軍五百人,傔人在外,戰馬鐵甲皆備。先以緡錢
 五萬於廣西買馬五百匹,詔廣西安撫司歲帶買三十匹。時樞府有不樂之者,數沮撓之,棄疾行愈力,卒不能奪。經度費鉅萬計,棄疾善斡旋,事皆立辦。議者以聚斂聞,降御前金字牌,俾日下住罷。棄疾受而藏之,出責監辦者,期一月飛虎營柵成,違坐軍制。如期落成,開陳本末,繪圖繳進,上遂釋然,時秋霖幾月,所司言造瓦不易,問:「須瓦幾何?」曰:「二十萬。」棄疾曰:「勿憂。」令廂官自官舍、神祠外,應居民家取溝敢瓦二,不二日皆具,僚屬嘆伏。軍
 成,雄鎮一方,為江上諸軍之冠。



 加右文殿修撰,差知隆興府兼江西安撫。時江右大饑,詔任責荒政。始至,榜通衢曰:「閉糴者配,強糴者斬。」次令盡出公家官錢、銀器,召官吏、儒生、商賈、市民各舉有乾實者,量借錢物,逮其責領運糴,不取子錢,期終月至城下發糶,於是連檣而至,其直自減,民賴以濟。時信守謝源明乞米救助,幕屬不從,棄疾曰:「均為赤子,皆王民也。」即以米舟十之三予信。帝嘉之,進一秩,以言者落職,久之,主管沖祐觀。



 紹熙二
 年,起福建提點刑獄。召見,遷大理少卿,加集英殿修撰、知福州兼福建安撫使。棄疾為憲時,嘗攝帥,每嘆曰:「福州前枕大海,為賊之淵,上四郡民頑獷易亂,帥臣空竭,急緩奈何!」至是務為鎮靜,未期歲,積鏹至五十萬緡,榜曰:「備安庫」。謂閩中土狹民稠,歲儉則糴於廣,今幸連稔,宗室及軍人入倉請米,出即糶之,候秋賈賤,以備安錢糴二萬石,則有備無患矣。又欲造萬鎧,招強壯補軍額,嚴訓練,則盜賊可以無虞。事未行,臺臣王藺劾其用錢
 如泥沙,殺人如草芥,旦夕望端坐「閩王殿」。遂丐祠歸。



 慶元元年落職,四年,復主管沖祐觀。久之,起知紹興府兼浙東安無使,四年,寧宗召見,言鹽法,加寶謨閣待制、提舉祐神觀,奉朝請。尋差知鎮江府,賜金帶。



 坐繆舉,降朝散大夫、提舉沖祐觀,差知紹興府、兩浙東路安撫使,辭免。進寶文閣待制,又進龍圖閣、知江陵府。令赴行在奏事,試兵部侍郎,辭免。進樞密都承旨,未受命而卒。賜對衣、金帶,守龍圖閣待制致仕,特贈四官。



 棄疾豪爽尚氣
 節,識拔英俊,所交多海內知名士。嘗跋紹興間詔書曰:「使此詔出於紹興之前,可以無事仇之大恥;使此詔行於隆興之後,可以卒不世之大功。



 今此詔與仇敵俱存也,悲夫!」人服其警切。帥長沙時,士人或訴考試官濫取第十七名《春秋》卷,棄疾察之信然,索亞榜《春秋》卷兩易之,啟名則趙鼎也。棄疾怒曰:「佐國元勛,忠簡一人,胡為又一趙鼎!」擲之地。次閱《禮記》卷,棄疾曰:「觀其議論,必豪傑士也,此不可失。」啟之,乃趙方也。嘗謂:「人生在勤,當以
 力田為先。北方之人,養生之具不求於人,是以無甚富甚貧之家。南方多末作以病農,而兼並之患興,貧富斯不侔矣。」故以「稼」名軒。為大理卿時,同僚吳交如死,無棺斂,棄疾嘆曰:「身為列卿而貧若此,是廉介之士也!」既厚賻之,復言於執政,詔賜銀絹。



 棄疾嘗同朱熹游武夷山,賦《九曲棹歌》,熹書「克己復禮」、「夙興夜寐」,題其二齋室。熹歿,偽學禁方嚴,門生故舊至無送葬者。棄疾為文往哭之曰:「所不朽者,垂萬世名。孰謂公死,凜凜猶生!」棄疾雅
 善長短句,悲壯激烈,有《稼軒集》行世。紹定六年,贈光祿大夫。咸淳閑,史館校勘謝枋得過棄疾墓旁僧舍,有疾聲大呼於堂上,若嗚其不平,自昏暮至三鼓不絕聲。枋得秉燭作文,旦且祭之,文成而聲始息。德祐初,枋得請於朝,加贈少師,謚忠敏。



 何異,字同叔,撫州崇仁人。紹興二十四年進士,調石城主簿,歷兩任,知蘋鄉縣。丞相周必大、參政留正以院轄擬異,孝宗問有無列薦,正等以萍鄉政績對,乃遷國子
 監主簿。遷丞,轉對,所言帝喜之,曰:「君臣一體,初不在事形跡,有所見聞,於銀臺司繳奏。」擢監察御史。異奏與丞相留正舊同官,不敢供職,御札不許引嫌,遂拜命。



 遷右正言。時光宗愆于定省,異入疏諫,不報。約臺官聯名,言奸人離間父子,當明正典刑,語極峻,又不報。丐外,授湖南轉運判官。偶攝帥事,辰蠻侵擾邵陽,異募山丁捕首亂者,蒲來矢以眾來降。尋為浙西提點刑獄。以太常少卿召,改秘書監兼實錄院檢討官,權禮部侍郎、太常寺。



 太廟芝草生,韓侂胄率百官觀焉,異謂其色白,慮生兵妖,侂胄不悅。又以劉光祖於異交密,言者遂以異在言路不彈丞相留正及受趙汝愚薦,劾罷之,久乃予祠。



 起知夔州兼本路安撫。異以夔民土狹食少,同轉運司糴米樁積,立循環通濟倉。七月丙戌,西北有星白芒墜地,其聲如雷,異曰:「戌日酉時,火土交會,而妖星自東南沖西北,化為天狗,蜀其將有兵乎?」丐祠,以寶謨閣待制提舉太平興國宮。



 後四年,吳曦果叛。起知潭州,乞閑予祠
 者再。



 嘉定元年,召為刑部侍郎。五月不雨,異上封事言:「近日號令或從中出,而執政不得與聞其事,臺諫不得盡行其言。陛下閔念饑民,藥病殯死,遐荒僻嶠,安得實惠?多方稱提,不如縮造楮幣;阜通商米,不如稍寬關市之徵。」明年,權工部尚書。告老,抗章言:「近臣求去,類成虛文,中外相觀,指為禮數,無以為風俗廉恥之勸。」以寶章閣直學士知泉州,從所乞予祠,進寶章閣學士,轉一官致仕。



 卒,年八十有一。異高自標致,有詩名,所著《月湖詩
 集》行世。



 劉宰,字平國,金壇人。既冠,入鄉校,卓然不茍於去就取舍。紹熙元年舉進士,調江寧尉。江寧巫風為盛,宰下令保伍互相糾察,往往改業為農。歲旱,帥守命振荒邑境,多所全活。有持妖術號「真武法」、「空雲子」、「寶華主」者,皆禁絕之。書其坐右曰:「毋輕出文引,毋輕事棰楚。」緣事出郊,與吏卒同疏食水飲。去官,惟篋藏主簿趙師秀酬倡詩而已。調真州司法。詔仕者非偽學,不讀周惇頤、程頤等
 書,才得考試,宰喟然曰:「平生所學者何?首可斷,此狀不可得。」



 卒弗與。



 授泰興令,有殺人獄具,謂:「禱於叢祠,以殺一人,刃忽三躍,乃殺三人,是神實教我也。」為請之州,毀其廟,斬首以徇。鄰邑有租牛縣境者,租戶於主有連姻,因喪會,竊券而逃。它日主之子徵其租,則曰牛鬻久矣。子累年訟於官,無券可質,官又以異縣置不問。至是訴於宰,宰曰:「牛失十載,安得一旦復之。」



 乃召二丐者勞而語之故,托以它事系獄,鞫之,丐者自詭盜牛以賣,遣詣
 其所驗視。



 租戶曰:「吾牛因某氏所租。丐者辭益力,因出券示之,相持以來,盜券者憮然,為歸牛與租。富室亡金釵,惟二僕婦在,置之有司,咸以為冤。命各持一蘆,曰:「非盜釵者,詰朝蘆當自若;果盜,則長於今二寸。」明旦視之,一自若,一去其蘆二寸矣,即訊之,果伏其罪。有姑訴婦不養者二,召二婦並姑置一室,或餉其婦而不及姑,徐伺之,一婦每以己饌饋姑,姑猶呵之,其一反之。如是累日,遂得其情。



 父喪,免,至京,韓侂胄方謀用兵,宰啟鄧友
 龍、薛叔似極言輕挑兵端,為國深害,迄如其言。為浙東倉司干官,職事修舉,亟引去,默觀時變,頓不樂仕。尋告歸,監南嶽廟。江、淮制置使黃度闢之入幕,宰辭曰:「君命召不往,今矧可出耶?」嘉定四年,堂審召命且再下,不至。時相亦屢諷執政、從官貽書挽宰,宰峻辭以絕。俄題考功歷,示決不復仕。



 理宗初即位,以為籍田令,屢辭,改添差通判建康府,又辭,乞致仕,乃以直秘閣主管仙都觀。拜改秩予祠之命,辭秘閣,不允。端平元年,升直寶謨閣,
 祠如故,且盡還磨勘歲月。未幾,遷太常丞,郡守以朝命趣行,不得已勉就道,至吳門,拜疏徑歸。一時譽望,收召略盡,所不能致者,宰與崔與之耳。帝側席以問侍御史王遂,且俾宣撫。遷將作少監,又以直敷文閣知寧國府,皆不拜。進直顯謨閣、主管玉局觀,帝猶冀宰一來也。召奏事,訖不為起。尋卒,鄉人罷市走送,袂相屬者五十里,人人如哭其私親。



 宰剛大正直,明敏仁恕,施惠鄉邦,其烈實多。置義倉,創義役,三為粥以與餓者,自冬徂夏,日
 食凡萬餘人,薪粟、衣纊、藥餌、棺衾之類,靡謁不獲。某無田可耕,某無廬可居,某之子女長矣而未昏嫁,皆汲汲經理,如己實任其責。橋有病涉,路有險阻,雖巨役必捐貲先倡而程其事。宰生理素薄,見義必為,既竭其力,藉質貸以繼之無倦。若定折麥錢額,更縣斗斛如制,毀淫祠八十四所,凡可以白於有司、利於鄉人者,無不為也。



 宰隱居三十年,平生無嗜好,惟書靡所不讀。既竭日力,猶坐以待,雖博考訓注,而自得之為貴。有《漫塘文集》、《語
 錄》行世。



 劉爚,字晦伯,建陽人。與弟韜仲受學於朱熹、呂祖謙。乾道八年舉進士,調山陰主簿。爚正版籍,吏不容奸。調饒州錄事,通判黃奕將以事污爚,而己自以贓抵罪去。都大坑冶耿某閔遺骸暴露,議用浮屠法葬之水火,爚貽書曰:「使死者有知,禍亦慘矣。」請擇高阜為叢塚以葬。



 調蓮城令,罷添給錢及綱運例錢,免上供銀錢及綱本、二稅甲葉、鈔鹽、軍期米等錢,大修學校,乞行經界。改知閩
 縣,治以清簡,庭無滯訟,興利去害,知無不為。差通判潭州,未上,丁父憂。偽學禁興,爚從熹武夷山講道讀書,怡然自適。築雲莊山房,為終老隱居之計。調贛州坑冶司主管文字,差知德慶府,大修學校,奏便民五事,又奏罷兩縣無名租錢,糾集武勇民兵。入奏言:「前者北伐之役,執事者不度事勢,貽陛下憂。今雖從和議,願益恐懼修省,必開言路以廣忠益,必張公道以進人才,必飭邊備以防敵患。」



 提舉廣東常平。令守臣歲以一半易新,春末
 支,及冬復償,存其半以備緩急。



 逋欠亭戶錢十萬,轉運司五萬,爚以公使,公用二庫贏錢補之。奏議倉之敝、客丁錢之敝、小官奉給之敝、舉留守令之敝、吏商之敝。召入奏事,首論:「公道明,則人心自一,朝廷自尊,雖危可安也;公道廢,則人心自貳,朝廷自輕,雖安易危也。」帝嘉獎。遷尚左郎官,請節內外冗費以收楮幣。轉對言:「願於經筵講讀、大臣奏對,反復問難,以求義理之當否,與政事之得失,則聖學進而治道隆矣。」



 乞收拾人才及修明軍
 政。遷浙西提點刑獄,巡按不避寒暑,多所平反。有殺人而匿權家者,吏弗敢捕,爚竟獲之。



 遷國子司業,言於丞相史彌遠,請以熹所著《論語》、《中庸》、《大學》、《孟子》之說以備勸講,正君定國,慰天下學士大夫之心。奏言:「宋興,《六經》微旨,孔、孟遺言,發明於千載之後,以事父則孝,以事君則忠,而世之所謂道學也。慶元以來,權佞當國,惡人議己,指道為偽,屏其人,禁其書,學者無所依鄉,義利不明,趨向污下,人欲橫流,廉恥日喪。追惟前日禁絕道學之
 事,不得不任其咎。望其既仕之後,職業修,名節立,不可得也。乞罷偽學之詔,息邪說,正人心,宗社之福。」又請以熹《白鹿洞規》頒示太學,取熹《四書集注》刊行之。又言:「浙西根本之地,宜詔長吏、監司禁戢強暴,撫柔善良,務儲積以備兇荒,禁科斂以紓民力。」



 兼國史院編修官、寶錄院檢討官。接伴金使於盱眙軍。還,言:「兩淮之地,藩蔽江南,干戈盜賊之後,宜加經理,必於招集流散之中,就為足食足兵之計。臣觀淮東,其地平博膏腴,有陂澤水泉
 之利,而荒蕪實多。其民勁悍勇敢,習邊鄙戰鬥之事,而安集者少。誠能經畫郊野,招集散亡,約頃畝以授田,使毋廣占拋荒之患,列溝洫以儲水,且備戎馬馳突之虞。為之具田器,貸種糧,相其險易,聚為室廬,使相保護,聯以什伍,教以擊刺,使相糾率。或鄉為一圍,里為一隊,建其長,立其副。平居則耕,有警則守,有餘力則戰。」帝嘉納之。



 進國子祭酒兼侍立修注官。論貢舉五敝。兼權兵部侍郎,改兼權刑部侍郎,封建陽縣開國男,賜食邑。權刑
 部侍郎兼國子祭酒,兼太子左諭德,升同修國史、實錄院同修撰。時廷臣爭務容默,有論事稍切者,眾輒指以為異。爚奏:「願明詔大臣,崇獎忠讜以作士氣,深戒諛佞以肅具僚。乞擇州縣獄官。」冬雷,上恐懼,爚奏:「遴選監司以考察貪吏為先,訪求民瘼,有澤未下流、令未便民者,悉以實上,變而通之,則民心悅而天意解矣。」又請擇沿邊諸將。



 兼工部侍郎。奏「乞使沿邊之民,各自什伍,教閱於鄉,有急則相救援,無事則耕稼自若,軍政隱然寓於
 田里之間,此非止一時之利也。」請城沿邊州郡、罷遣賀正使。試刑部侍郎,兼職依舊,賜對衣、金帶,辭,不允。兩請致仕,不允。奏絕金人歲幣,建制置司於歷陽以援兩淮。夏旱,應詔上封事,曰:「言語方壅而導之使言,人心方鬱而疏之使通,上既開不諱之門,下必有盡言之士,指陳政事之闕失,明言朝廷之是非。或者以為好名要譽,而陛下聽之,則苦言之藥,至言之實,陛下棄之而不恤矣,甘言之疾,華言之腴,陛下受之而不覺矣。」氣罷瑞慶聖
 節,謝絕金使。



 進封子爵。權工部尚書,賜衣帶、鞍馬。兼太子右庶子,仍兼左諭德。每講讀至經史所陳聲色嗜欲之戒,輒懇切再三敷陳之。進讀《詩》之說,詹事戴溪讀之為之吐舌。卒,贈光祿大夫,官其後,賜謚文簡。所著有《奏議》、《史稿》、《經筵故事》、《東宮詩解》、《禮記解》、《講堂故事》、《雲莊外稿》。



 柴中行,字與之,餘干人。紹熙元年進士,授撫州軍事推官。權臣韓侂胄禁道學,校文,轉運司移檄,令自言非偽
 學,中行奮筆曰:「自幼讀程頤書以收科第,如以為偽,不願考校。」



 調江州學教授,母喪,免,廣西轉運司闢為幹官,帥將薦之,使其客嘗中行,中行正色曰:「身為大帥,而稱人為恩王、恩相,心竊恥之。毋污我!」攝昭州郡事,蠲丁錢,減苗斛,賑饑羸。轉運司委中行代行部,由桂林屬邑歷柳、象、賓入邕管,問民疾苦,先行而後聞,捐鹽息以惠遠民。嘉定初,差主管尚書吏部架閣文字,遷太學正,升博士。轉對,首論主威奪而國勢輕;次論士大夫寡廉隅、乏
 骨鯁,宜養天下剛毅果敢之氣;末論權臣用事,包苴成風,今舊習猶在,宜舉行先朝痛繩贓吏之法。謂太學風化首,童子科覆試胄子舍選,有挾勢者,中行力言於長,守法無秋豪私。遷太常主簿,轉軍器監丞。



 出知光州,嚴保伍,精閱習,增闢屯田,城壕營砦、器械糗糧,百爾具備,治行為淮右最。又條畫極邊、次邊緩急事宜上之朝廷,大概謂:「邊兵宜如蛇勢,首尾相應。草寇合兵大入,則鄰道援之;分兵輕襲,則鄰郡援之。援兵既多,雖危不敗。」又
 言:「淮、襄土豪丁壯,往者用兵,傾貲效力者,朝廷吝賞失信,宜亟加收拾,亦可激昂得其死力。」



 遷西京轉運使兼提點刑獄。中行謂襄陽乃自古必爭之地,備御尤宜周密。時任邊寄者政令煩苛,日夜與民爭利,中行諷之,不聽。天方旱,盡捐酒稅,斥征官,黥務吏,甘澍隨至。官取鹽鈔贏過重,課日增,入中日寡,鈔日壅。中行揭示通衢,一錢不增,商賈大集。改直秘閣、知襄陽兼京西帥,仍領漕事。江陵戎司移屯襄州,兵政久馳。中行白於朝,考核軍
 實,舊額二萬二千人,存者才半,亟招補虛籍。自是朝廷以節制之權歸帥司。重劾李珙不法以懲貪守,時扈再興有功以厲宿將,上關朝廷,下關制閫。



 遷江東轉運司判官,旋改湖南提點刑獄。豪家習殺人,或收養亡命,橫行江湖,一繩以法。華亭令貪虐,法從交疏薦之,中行笑曰:「此欲斷吾按章也。」卒發其辜。入為吏部郎官。以立志啟迪君心,言好進、好同、好欺,士大夫風俗三敝。選曹法大壞,吏緣為奸,中行遇事持正,不為勢屈,由是銓綜平
 允。



 擢宗正少卿。上疏謂:「陛下初政則以剛德立治本,更化則以剛德除權奸,今者顧乃垂拱仰成,安於無為。夫剛德實人主之大權,不可以久出而不收,覆轍在前,良可鑒也。」又曰:「朝廷用人,外示涵洪而陰掩其跡,內用牢籠而微見其機,觀聽雖美,實無以大服天下之心。曩者更化,元氣復挽回矣。比年欲求安靜,頗厭人言,於是臣下納說,非觀望則希合,非回緩則畏避,而面折廷諍之風未之多見,此任事大臣之責也。」



 兼國史編修、實錄檢
 討。孟春,大雨震電,雷雹交作,邊烽告急,至失地喪師,淮甸震洶。中行亟奏內外二失,朝廷十憂,大要言:「今日之事,人主盡委天下以任一相,一相盡以天下謀之三數腹心,而舉朝之士相視以目,噤不敢言。甚至邊庭申請,久不即報,脫有闕誤,咎當誰執?」



 調秘書監、崇政殿說書。極論「往年以道學為偽學者,欲加遠竄,杜絕言語,使忠義士箝口結舌,天下之氣豈堪再沮壞如此耶?」又謂:「欲結人心,莫若去貪吏;欲去貪吏,莫若清朝廷。大臣法則
 小臣廉,在高位者以身率下,則州縣小吏何恃而敢為?」又論內治外患,辨君子小人,大略謂:「執政、侍從、臺諫、給舍之選,與三衙、京尹之除,皆朝廷大綱所在,故其人必出人主之親擢,則權不下移。



 今或私謁,或請見,或數月之前先定,或舉朝之人不識。附會者進,爭為妾婦之道,則天下國家之利害安危,非惟己不敢言,亦且並絕人言矣。大臣為附會之說所誤,邊境之臣實遁者掩以為誣,真怯者譽以為勇,金帛滿前,是非交亂,以欺廟堂,以
 欺陛下。願明詔大臣,絕私意,布公道。」



 進秘閣修撰、知贛州。漢盜有方,境內清肅。丐祠得請,以言罷。理宗即位,以右文殿修撰主管南京鴻慶宮,賜金帶。卒。所著有《易系集傳》、《書集傳》、《詩講義》、《論語童蒙說》。



 李孟傳,字文授,資政殿學士光季子也。光謫嶺海,孟傳才六歲,奉母居鄉,刻志於學。賀允中、徐度皆奇之,而曾幾妻以其孫。龍大淵黜為浙東總管,知孟傳為名門子,解後必就語,孟傳正色辭之。乾辦江東提刑司,易浙東
 常平司。



 母喪,免,調江山縣丞,棄去,監南嶽廟、行在編估局,未上,改楚州司戶參軍,單車赴官。公退,閉戶讀《易》。郡守、部使者不敢待以屬吏。徐積墓在境內,蕪沒既久,加葺之。修復陳公塘,有灌溉之利。知象山縣,守薦為邑最,從官多合薦之,主管官告院,與同列上封事,請詣北宮,又移書宰相。



 遷將作監主簿。丞相趙汝愚初當國,適大侵,遣孟傳按視江、池、鄂三大軍所屯積粟,道除太府丞。既復命,汝愚去國,黨論起,而孟傳奉使無失指,面對言:「
 比以使事往返四千里,所過民生困窮,衣食不贍。國之安危,以民為本,今根本既虛,形勢俱見,保邦之慮,宜勤聖念。」時韓侂胄連逐留正及汝愚,太府簿吳璹與侂胄有連姻,因言臺諫將論朱熹,孟傳奮然曰:「如此則士大夫爭之,鼎鑊且不避。」



 兼考功郎。復因對言:「國家長育人才,猶天地之於植物,滋液滲漉,待其既成而後足以供大廈之用。今士大夫皆有茍進之心,治功未優,功能尚薄,而意已馳騖於臺閣,不稍有以扶持正飭之,其敝將
 甚。」又言:「武舉及軍士比試,專取其力,臨敵難以必勝。唐世取人由步射、弓弩以至馬射,各以其中之多寡為等級,宜採取行之。」韓侂胄與孟傳故,嘗致侂胄意,孟傳謝曰:「行年六十,去意已決。」



 侂胄慚而退。請外,知江州,獄訟止息。侂胄不悅。丐歸,復知處州。



 遷廣西提點刑獄,改江東提舉常平,移福建。詔入對,首論用人宜先氣節後才能,益招徠忠讜以扶正論。故人有在政府者,折簡問勞勤甚,孟傳逆知其意,即謝曰:「孤蹤久不造朝,獲一望清
 光而去,幸矣。」對畢即出關。至閩,大饑,發廩勸分,民無流莩。侂胄誅,就遷提點刑獄,移江東,又辭。丞相史彌遠,其親故也,人謂進用其時矣,卒歸使節,角巾還第。再奉祠,以倉部郎召,又辭。



 遷浙東提點刑獄,未數月,申前請,章再上,加直秘閣,移江東,不赴,主管明道宮。進直寶謨閣,致仕,卒,年八十四。常誡其子孫曰:「安身莫若無競,修己莫若自保。守道則福至,求祿則辱來。」有《磐溪集》、《宏詞類稿》、《左氏說》、《讀史》、《雜志》、《記善》、《記異》等書行世。



 論曰:古之君子,出處不齊,同歸於是而已。辛棄疾知大義而歸宋。何異篤實君子,而切諫光宗朝重華宮。柴中行寧不校臨川之試,終不肯自言非程頤偽學。劉爚表章朱熹《四書》以備勸講,衛道之功莫大焉。李孟傳所立不愧其父。至於劉宰飄然遠引,屢征不起,所謂鴻飛冥冥者耶。



\end{pinyinscope}