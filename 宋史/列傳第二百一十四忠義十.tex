\article{列傳第二百一十四忠義十}

\begin{pinyinscope}

 ○陳東
 歐陽澈馬伸呂祖儉呂祖泰楊宏中華嶽鄧若水僧真寶莫謙之徐道明



 陳東,字少陽,鎮江丹陽人。早有雋聲,俶儻負氣,不戚戚於貧賤。蔡京、王黼方用事,人莫敢指言,獨東無所隱諱。所至宴集,坐客懼為己累,稍引去。以貢入太學。欽宗即位,率其徒伏闕上書,論:「今日之事,蔡京壞亂於前,梁師成陰謀於後。李彥結怨於西北,朱勔結怨於東南,王黼、童貫又結怨於遼、金,創開邊隙。宜誅六賊,傳首四方,以謝天下。」言極憤切。明年春,貫等挾徽宗東行,東獨上書請追貫還正典刑,別選忠信之人往侍左右。金人迫京
 師,又請誅六賊。時師成尚留禁中,東發其前後奸謀,乃謫死。



 李邦彥議與金和,李綱及種師道主戰,邦彥因小失利罷綱而割三鎮,東復率諸生伏宣德門下上書曰:



 在廷之臣,奮勇不顧、以身任天下之重者,李綱是也,所謂社稷之臣也。其庸繆不才、忌疾賢能、動為身謀、不恤國計者,李邦彥、白時中、張邦昌、趙野、王孝迪、蔡懋、李棁之徒是也,所謂社稷之賊也。



 陛下拔綱列卿之中,不一二日為執政,中外相慶,知陛下之能任賢矣。斥時中而
 不用,知陛下之能去邪矣。然綱任而未專,時中斥而未去,復相邦彥,又相邦昌,自餘又皆擢用,何陛下任賢猶未能勿貳,去邪猶未能勿疑乎?今又聞罷綱職事,臣等驚疑,莫知所以。



 綱起自庶官,獨任大事。邦彥等疾如仇讎,恐其成功,因用兵小不利,遂得乘閑投隙,歸罪於綱。夫一勝一負,兵家常勢,豈可遽以此傾動任事之臣。竊聞邦彥、時中等盡勸陛下他幸,京城騷動,若非綱為陛下建言,則乘輿播遷,宗廟社稷已為丘墟,生靈已遭魚
 肉。賴聰明不惑,特從其請,宜邦彥等讒嫉無所不至。陛下若聽其言,斥綱不用,宗社存亡,未可知也。邦彥等執議割地,蓋河北實朝廷根本,無三關四鎮,是棄河北,朝廷能復都大梁乎?則不知割太原、中山、河間以北之後,邦彥等能使金人不復敗盟乎?



 一進一退,在綱為甚輕,朝廷為甚重。幸陛下即反前命,復綱舊職,以安中外之心,付種師道以閫外之事。陛下不信臣言,請遍問諸國人,必皆曰綱可用,邦彥等可斥也。用舍之際,可不審諸!



 軍
 民從者數萬。書聞,傳旨慰諭者旁午,眾莫肯去,方舁登聞鼓撾壞之,喧呼震地。有中人出,眾臠而磔之。於是亟詔綱入,復領行營,遣撫諭,乃稍引去。



 金人既解去,學官觀望,時宰議屏伏闕之士,先自東始。京尹王時雍欲盡致諸生於獄,人人惴恐。朝廷用楊時為祭酒,復東職,遣聶山詣學撫諭,然後定。吳敏欲弭謗,議奏補東官,賜第,除太學錄。東又請誅蔡氏,且力辭官以歸,前後書五上。既歸,復預鄉薦。



 高宗即位五日,相李綱,又五日召東至。
 未得對,會綱去,乃上書乞留綱而罷黃潛善、汪伯彥。不報。請親征以還二聖,治諸將不進兵之罪,以作士氣;車駕歸京師,勿幸金陵。又不報。潛善輩方揭示綱幸金陵舊奏,東言綱在中途,不知事體,宜以後說為正,必速罷潛善輩。



 會布衣歐陽澈亦上書言事,潛善遽以語激怒高宗,言不亟誅,將復鼓眾伏闕。書獨下潛善『所。府尹孟庾召東議事,東請食而行,手書區處家事,字畫如平時,已乃授其從者曰:「我死,爾歸致此於吾親。」食已如廁,吏
 有難色,東笑曰:「我陳東也,畏死即不敢言,已言肯逃死乎?」吏曰:「吾亦知公,安敢相迫。」頃之,東具冠帶出,別同邸,乃與澈同斬於市。四明李猷贖其尸瘞之。東初未識綱,特以國故,至為之死,識與不識皆為流涕。時年四十有二。



 潛善既殺二人,明日府尹白事,獨詰其何以不先關白,微示慍色,以明非己意。越三年,高宗感悟,追贈東、澈承事郎。東無子,官有服親一人,澈一子,令州縣撫其家。及駕過鎮江,遣守臣祭東墓,賜緡錢五百。紹興四年,並
 加朝奉郎、秘閣修撰,官其後二人,賜田十頃。



 歐陽澈,字德明,撫州崇仁人。年少美須眉,善談世事,尚氣大言,慷慨不少屈,而憂國閔時,出於天性。靖康初,應制條敝政,陳安邊禦敵十策,州未許發。退而復採朝廷之闕失,政令之乖違,可以為保邦御俗之方、去蠹國殘民之賊者十事,復為書,並上聞。已而復論列十事,言:「臣所進三書實為切要,然而觸權臣者有之,迕天聽者有之,或結怨富貴之門,或遺怒臺諫之官,臣非不知,而敢
 抗言者,願以身而安天下也。」所上書為三巨軸,廄置卒辭不能舉,州將為選力士荷之以行。



 會金人大入,要盟城下而去,澈聞,輒語人曰:「我能口伐金人,強於百萬之師,願殺身以安社稷。有如上不見信,請質子女於朝,身使穹廬,御親王以歸。」鄉人每笑其狂,止之不可,乃徒步走行在。高宗即位南京,伏闕上封事,極詆用事大臣,遂見殺,見《陳東傳》。死時年三十七。



 許翰在政府,罷朝,問潛善處分何人,曰:「斬陳東、歐陽澈耳。」翰驚失色,因究其書
 何以不下政府,曰:「獨下潛善,故不得以相視。」遂力求罷。為東、澈著哀詞。澈所著《飄然集》六卷,會稽胡衍既刻之,豐城範應鈐為之祠學中。



 馬伸,字時中,東平人。紹聖四年進士。不樂馳騖,每調官,未嘗擇便利。為成都郫縣丞,守委受成都租。前受輸者率以食色玩好蠱訹而敗,伸請絕宿弊。民爭先輸,至沿途假寐以達旦,常平使者孫俟早行,怪問之,皆應曰:「今年馬縣丞受納,不病我也。」俟薦於朝。



 崇寧初,範致虛攻
 程頤為邪說,下河南府盡逐學徒。伸注西京法曹,欲依頤門以學,因張繹求見,十反愈恭,頤固辭之。伸欲休官而來,頤曰:「時論方異,恐貽子累,子能棄官,則官不必棄也。」曰:「使伸得聞道,死何憾,況未必死乎?」頤嘆其有志,進之。自是公暇雖風雨必日一造,忌娼者飛語中傷之,弗顧,卒受《中庸》以歸。



 靖康初,孫傅以卓行薦召,御史中丞秦檜迎闢之,擢監察御史。及汴京陷,金人立張邦昌,集百官,環以兵脅之,俾推戴。眾唯唯,伸獨奮曰:「吾職諫爭,
 忍坐視乎!」乃與御史吳給約秦檜共為議狀,乞存趙氏,復嗣君位。會統制官吳革起義,募兵圖復二帝,伸預其謀。



 邦昌既僭立,賊臣多從臾之,伸首具書請邦昌速迎奉元帥康王。同院無肯連名者,伸獨持以往,而銀臺司視書不稱臣,辭不受。伸投袂叱之曰:「吾今日不愛一死,正為此耳,爾欲吾稱臣邪?」即繳申尚書省,以示邦昌。其書略曰:



 相公服事累朝,為宋輔臣。比不幸迫於強敵,使當偽號,變出非常,相公此時豈以義為可犯,君為可忘,
 宗社神靈為可昧邪?所以忍須臾死而詭聽之者,其心若曰:與其虛遜於人而實亡趙氏之宗,孰若虛受於己而實存以歸之耳。忠臣義士未即就死,闔城民庶未即生變者,亦以相公必能立趙孤也。



 今金人北還,相公義當憂懼,自列於朝。康王在外,國統有屬,獄訟謳歌,人皆歸往。宜即發使通問,掃清宮室,率群臣共迎而立之。相公易服退處,省中庶事皆稟命太后,其赦書施恩惠、收人心等事,日下拘收,俟康王御極施行。然後相公北面
 引咎,以明身為人臣,昧於防患,遭寇仇脅污,當時不能即死,以待陛下,今復何面目事君,請歸死司寇,為人臣失節之戒,伏闕下俟命。如此,則明主必能察相公忠實存國,義非茍生,且棄過而錄功矣。



 今乃謀不出此,時日已多,肆然尚當非據,偃寢禁闥,若固有之。群心狐疑,道路混澒,謂相公方挾強金,使人游說康王,姑令南遁,為久假不歸之計。上天難欺,下民可畏。相公若以愚言粗知覺悟,及此改圖,猶可轉禍為福於匪朝伊夕之間。過
 此以往,則相公包藏已深,志慮轉異,外飾事端,心妻日待期,而陰結寇仇,合從為亂,九廟在天,萬無成理,伸必不能輔相公為宋朝叛臣也。請先伏死都市,以明此心。」



 邦昌得書,氣沮謀喪。明日,議迎哲宗後孟氏垂簾,追還偽赦,乃遣馮澥、李回等迎康王。



 時王及之等猶請籍龍德宮寶貨,斥賣靈沼魚藕,以資官用。伸復慨然引義檄之曰:「古者人臣去國,三年不反,然後收其田里。君之禮臣如此,臣之報君宜如何?今二聖遠狩,猶未出境,天下之
 人方且北首,欲追挽而還之。君之府藏燕游,忍一朝而毀乎?爾等逆節甚矣!」力爭乃止。



 高宗即位,伸拜章以城陷不能救,主遷不能死,請就竄削。上知其有忠力於國,擢殿中侍御史,撫諭荊湖、廣南,以誅邦昌及其黨王時雍等。所過州縣,諏察吏之賢否與民利疚,以次列上於朝。



 伸自湖、廣將入奏黃潛善、汪伯彥不法凡十有七事,草疏已具,朝廷方召孫覿、謝克家,乃先奏:「覿、克家趨操不正,在靖康間與王時雍、王及之等七人結為死黨,附
 耿南仲倡為和議,助成賊謀。有不主和議者,則欲執送金人。覿受金人女樂,草表媚之,極其筆力,乃負國之賊,宜加遠竄。」不報。伸又進疏曰:



 陛下得黃潛善、汪伯彥以為輔相,委任不復疑。然自入相以來,處事未嘗愜當物情,遂使女真日強,盜賊日熾,國本日蹙,威權日削。且三鎮未服,汴都方危,前日遽下還都之詔,至今鑾輿未能順動。其不謹詔命如此。草茅對策不如式,考官罰金可矣,一日黜三舍人,乃取沈晦、孫覿、黃哲輩諸群小以掌
 誥命。其黜陟不公如此。吳給、張訚以言事被逐,邵成章緣上言遠竄。其壅塞言路如此。祖宗舊制,諫官御史有闕,御史中丞、翰林學士具名以進,三省不敢預,厥有深旨。近擬用臺諫,多取親舊,不過欲為己助。其毀法自恣如此。張愨、宗澤、許景衡公忠有才,皆可任重,潛善、伯彥忌之,沮抑至死。其妨功害能如此。或責以救焚拯溺之事,則曰難言,蓋謂陛下制之不得施設也。或問陳東之死,則曰不知,蓋謂其事繇於陛下也。其過則稱君、善則
 稱己如此。呂源狂橫,陛下逐去,不數月由郡守升發運。其強狠自專如此。御營使雖主兵權,凡行在諸軍皆其所統,潛善、伯彥別置親兵一千人,請給居處,優於眾兵。其務收軍情如此。廣市私恩,則多復祠官之闕;同惡相濟,則力庇王安中之罪。摭其所為,豈不辜陛下倚任之重哉?



 陛下隱忍不肯斥逐,塗炭遺民固已絕望,二聖還期在何時邪?臣每念此,不如無生。歲月如流,時幾易失,望速罷潛善、伯彥政柄,別選賢者,共圖大事。



 疏入,留中。
 明日,改衛尉少卿。伸以論事不行,辭不拜,錄其疏申御史臺,且疊上章言:「臣言可採,即乞施行,若臣言非是,合坐誣罔之罪。」移疾待命。旬日,詔伸言事不實,送吏部責濮州監酒稅。時用事者恚甚,必欲殺之,以濮迫寇境,故有是命。趣使上道,伸怡然袱被而行,死道中。或曰王淵在濮,潛善密嗾其不利於伸。天下識與不識皆冤痛之。



 明年,金人陷廣陵,伸言始驗,潛善、伯彥始以誤國竄殛。於是臺臣奏伸嘗論潛善等罪,乃復以衛尉少卿召,實
 未知其存亡也。尋加直龍圖閣。



 紹興初,胡安國上《時政論》,有曰:「伸言潛善、伯彥措置乖方,條其罪狀,凡舉一事,必立一證,皆眾所共知共見,不敢以無為有,以是為非。而當時曾不從用,反以為言事不實而重責之,是罰沮忠讜,邪說何由而息,公道何由而明乎?伸既遠貶,雖有詔命,邈無來期,君子閔焉。賁以龍圖,猶未盡褒勸之典。乞重加追獎,及其子孫,以承天意。」詔贈諫議大夫。



 伸天資純確,學問有原委,勇於為義,而所韞深厚,恥以自名。
 建炎初,右正言鄧肅嘗論朝士臣邦昌者,例貶二秩,伸不辨也。凡有建明,輒削其稿,人罕知之。居官,晨興必整衣端坐,讀《中庸》一遍,然後出涖事。每曰:「吾志在行道。以富貴為心,則為富貴所累;以妻子為念,則為妻子所奪,道不可行也。」故在廣陵,行篋一擔,圖書半之。山東已擾,家尚留於鄆。常稱:「孔子言:『志士不忘在溝壑,勇士不忘喪其元。』今日何日,溝壑乃吾死所也。」



 有何兌者,昭武人,受學於伸。伸沒,兌嘗輯其事狀。紹興中,為辰州通判,都郵
 報,秦檜自陳其存趙之功,謂它人莫預。兌徑取所輯事狀達尚書省,檜大怒,下兌荊南詔獄,獄辭皆出吏手,兌坐削官竄真陽。檜死始放還,復其官。尋卒。



 呂祖儉字子約,祖謙之弟也,受業祖謙如諸生。監明州倉,將上,會祖謙卒。部法半年不上者為違年,祖儉必欲終期喪,朝廷從之,詔違年者以一年為限,自祖儉始。



 終更赴銓,丞相周必大語尚書尤袤招之,祖儉已調衢州法曹而後往見。潘時經略廣東,欲闢為屬,祖儉辭。尋以
 侍從鄭僑、張杓、羅點、諸葛庭瑞薦,召除籍田令。



 中丞何澹所生父繼室周氏死,澹欲服伯母服,下太常百官雜議。祖儉貽書宰相曰:「《禮》曰:『為伋也妻者,是為白也母。』今周氏非中丞父之妻乎?將不謂之母而謂之何?中丞為風憲首,而以不孝令,百僚何觀焉。」除司農簿,已而乞補外,通判臺州。寧宗即位,除太府丞。



 時韓侂胄浸用事,正言李沐論右相趙汝愚罷之。祖儉奏:「汝愚亦不得無過,然未至如言者所云。」侂胄怒曰:「呂寺丞乃預我事邪?」會
 祭酒李祥、博士楊簡皆上書訟汝愚,沐皆劾罷之。祖儉乃上封事曰:「陛下初政清明,登用忠良,然曾未逾時,朱熹老儒也,有所論列,則亟使之去;彭龜年舊學也,有所論列,亦亟許之去;至於李祥老成篤實,非有偏比,蓋眾聽所共孚者,今又終於斥逐。臣恐自是天下有當言之事,必將相視以為戒,鉗口結舌之風一成而未易反,是豈國家之利邪?」



 又曰:「今之能言之士,其所難非在於得罪君父,而在忤意權勢。姑以臣所知者言之,難莫難於
 論災異,然言之而不諱者,以其事不關於權勢也。若乃御筆之降,廟堂不敢重違,臺諫不敢深論,給、舍不敢固執,蓋以其事關貴幸,深慮乘間激發而重得罪也。故凡勸導人主事從中出者,蓋欲假人主之聲勢,以漸竊威權耳。比者聞之道路,左右NJ御,於黜陟廢置之際,間得聞者,車馬輻湊,其門如市,恃權怙寵,搖撼外庭。臣恐事勢浸淫,政歸幸門,不在公室。凡所薦進皆其所私,凡所傾陷皆其所惡,豈但側目憚畏,莫敢指言,而阿比順從,
 內外表里之患,必將形見。臣因李祥獲罪而深及此者,是豈矯激自取罪戾哉?實以士氣頹靡之中,稍忤權臣,則去不旋踵。私憂過計,深慮陛下之勢孤,而相與維持宗社者浸寡也。」



 疏既上,束簷待罪。有旨:呂祖儉朋比罔上,安置韶州。中書舍人鄧馹繳奏,祖儉罪不至貶。御筆:「祖儉意在無君,罪當誅。竄逐已為寬恩。」會樓鑰進讀呂公著元祐初所上十事,因進曰:「如公著社稷臣,猶將十世宥之,前日太府寺丞呂祖儉以言事得罪者,其孫也。
 今投之嶺外,萬一即死,聖朝有殺言者之名,臣竊為陛下惜之。」上問:「祖儉所言何事?」然後知前日之行不出上意。侂胄謂人曰:「復有救祖儉者,當處以新州矣。」眾莫敢出口。有謂侂胄曰:「自趙丞相去,天下已切齒,今又投祖儉瘴鄉,不幸或死,則怨益重,曷若少徙內地。」侂胄亦悟。祖儉至廬陵,將趨嶺,得旨改送吉州。遇赦,量移高安。二年卒,詔令歸葬。



 祖儉之謫也,朱熹與書曰:「熹以官則高於子約,以上之顧遇恩禮則深於子約,然坐視群小之
 為,不能一言以報效,乃令子約獨舒憤懣,觸群小而蹈禍機,其愧嘆深矣。」祖儉報書曰:「在朝行聞時事,如在水火中,不可一朝居。使處鄉閭,理亂不知,又何以多言為哉?」在謫所,讀書窮理,賣藥以自給。每出,必草履徒步,為逾嶺之備。嘗言:「因世變有所摧折,失其素履者,固不足言矣;因世變而意氣有所加者,亦私心也。」所為文有《大愚集》。祖儉從弟祖泰。



 祖泰。字泰然,夷簡六世孫,寓常之宜興。性疏達,尚氣誼,
 學問該洽。遍游江、淮,交當世知名士,得錢或分挈以去,無吝色。飲酒至數斗不醉,論世事無所忌諱,聞者或掩耳而走。



 慶元初,祖儉以言事安置韶州。既移瑞州,祖泰徒步往省之,留月餘,語其友王深厚曰:「自吾兄之貶,諸人箝口。我雖無位,義必以言報國,當少須之,今未敢以累吾兄也。」及祖儉沒貶所,嘉泰元年,周必大降少保致仕,祖泰憤之,乃詣登聞鼓院上書,論侂胄有無君之心,請誅之以防禍亂。其略曰:「道學,自古所恃以為國也。丞
 相汝愚,今之有大勛勞者也。立偽學之禁,逐汝愚之黨,是將空陛下之國,而陛下不知悟邪?陳自強,侂胄童孺之師,躐致宰輔。陛下舊學之臣,若彭龜年等,今安在邪?蘇師旦,平江之史胥,以潛邸而得節鉞;周筠,韓氏之廝役,以皇后親屬得大官。不識陛下在潛邸時果識師旦乎?椒房之親果有筠乎?凡侂胄之徒,自尊大而卑朝廷,一至於此也!願亟誅侂胄及師旦、周筠,而罷逐自強之徒。獨周必大可用,宜以代之,不然,事將不測。」書出,中外大駭。



 有旨:「呂祖泰挾私上書,語言狂妄,拘管連州。」右諫議大夫程松與祖泰狎友,懼曰:「人知我素與游,其謂預聞乎?」乃獨奏言:「祖泰有當誅之罪,且其上書必有教之者,今縱不殺,猶當杖黥竄遠方。」殿中侍御史陳讜亦以為言。乃杖之百,配欽州牢城收管。



 初,監察御史林採言偽習之成,造端自必大,故有少保之命。祖泰知必死,冀以身悟朝廷,無懼色。既至府廷,尹為好語誘之曰:「誰教汝共為章?汝試言之,吾且寬汝。」祖泰笑曰:「公何問之愚也。吾
 固知必死,而可受教於人,且與人議之乎?」尹曰:「汝病風喪心邪?」祖泰曰:「以吾觀之,若今之附韓氏得美官者,乃病風喪心耳。」



 祖泰既貶,道出潭州,錢文子為醴陵令,私贐其行。侂胄使人跡其所在,祖泰乃匿襄、郢間。侂胄誅,朝廷訪得祖泰所在,詔雪其冤,特補上州文學,改授迪功郎、監南嶽廟。喪母無以葬,至都謀於諸公,得寒疾,索紙書曰:「吾與吾兄共攻權臣,今權臣誅,吾死不憾。獨吾生還無以報國,且未能葬吾母,為可憾耳。」乃卒。尹王柟
 為具棺斂歸葬焉。



 楊宏中字充甫,福州人。弱冠補國子生。孝宗崩,光宗以疾不能執喪。時趙汝愚知樞密院,奏請太皇太后迎立寧宗於嘉邸,以成喪禮,朝野晏然。遂命汝愚為右丞相,登進耆德及一時知名之士,有意慶歷、元祐之治。韓侂胄竊弄國柄,引將作監李沐為右正言,首論罷汝愚,中丞何澹、御史胡紱章繼上,竄汝愚永州。國子祭酒李祥、博士楊簡連疏救爭,俱被斥。宏中曰:「師儒能辨大臣之
 冤,而諸生不能留師儒之去,於誼安乎?」眾莫應,獨林仲麟、徐範、張行、蔣傅、周端朝五人願預其議。遂上書曰:



 自古國家禍亂之由,初非一道,惟小人中傷君子,其禍尤慘。君子登庸,杜絕邪枉,要其處心實在於愛君憂國。小人得志,仇視正人,必欲空其朋類,然後可以肆行而無忌。於是人主孤立,而社稷危矣。黨錮敞漢,朋黨亂唐,大率由此。元祐以來,邪正交攻,卒成靖康之變,臣子所不忍言,而陛下所不忍聞也。



 臣竊見近者諫臣李沐論前
 宰相趙汝愚數談夢兆,擅權植黨,將不利於陛下。以此加誣,實不其然。汝愚乞去,中外咨憤,而言者以為父老歡呼,蒙蔽天聽,一至於此。章穎力辨其非,首遭斥逐,聞者已駭;既而祭酒李祥、博士楊簡相繼抗論,毅然求去,告假幾月,善類皇皇。一旦有外補之命,言者惡其扶植正論,極力牴排,同日報罷,六館之士為之憤惋涕泣。今李沐自知邪正之不兩立,而公議之不直己也,乃欲盡去正人以便其私,於是托朋黨以罔陛下之聽。臣謂二
 人之去若未足惜,殆恐君子小人消長之機於此一判,則靖康已然之監,豈堪復見於今日邪?陛下厲精圖政,方將正三綱以維人心,採群議以定國是,遽聽奸回,概疑善類,此臣等之所未諭也。



 臣願陛下鑒漢、唐之禍,懲靖康之變,精加宸慮,特奮睿斷。念汝愚之忠勤,察祥簡之非黨,灼李沐之回邪,明示好惡,旌別淑慝,竄李沐以謝天下,還祥、簡以收士心,臣雖身膏鼎鑊,實所不辭。



 書奏不報,則繳副封於臺諫、侍從。侂胄大怒,坐以不合上
 書之罪,六人皆編置,以宏中為首,將竄之嶺南。中書舍人鄧馹上書救之,不聽。右丞相餘端禮拜於榻前至數十,丐免遠徙。上惻然許之,乃送太平州編管。天下號為「六君子」。



 明年,移福州聽讀。嘉泰三年,寧宗幸學,持旨放還。開禧元年,宏中登進士第,教授南劍州。太守餘嶸,故相端禮子,與之相得甚歡。侂胄誅,先以言得罪者悉加褒錄。嘉定元年,特遷宏中一秩,亦不拜。六年,以嶸與汪逵、趙彥橚薦,授戶部架閣,俄遷太學正。八年夏旱,上封
 事,指切無隱。遷武學博士,改宣教郎。



 時諫官應武論一學官,宏中季試策士及其故,武聞而銜之。秋戊祀武成王,祭酒行事。故事,博士攝亞獻,至是不命宏中,宏中白於祭酒。於是武劾宏中與同列競,且謂其激矯不自愛,遂通判潭州。以親老請祠,差知武岡軍,未受卒,年五十三。



 端朝字子靜,嘉定三年試禮部第一,終刑部侍郎兼侍講。行字用叟,以父任補官,有二子,與端朝同登進士第。仲麟字景仲,傅字象夫,久居學校,忠鯁有聞,咸以不
 偶死。範自有傳。



 華嶽,字子西,為武學生,輕財好俠。韓侂胄當國,嶽上書曰:



 旬月以來,都城士民徬徨四顧,若將喪其室家;諸軍妻子隱哭含悲,若將驅之水火。闤闠籍籍,欲語復噤,駭於傳聞,莫曉所謂。臣徐考之,則侍衛之兵日夜潛發,樞機之遞星火交馳,戎作之役倍於平時,郵傳之程兼於疇昔,乃知陛下將有事於北征也。



 侂胄以後族之親,位居極品,專執權柄,公取賄賂;畜養無籍吏僕,委以腹心,
 賣名器,私爵賞,睥睨神器,窺覘宗社,日益炎炎,不敢向爾。此外患之居吾腹心者也。



 朝臣有以庸瑣之資,請姻師旦,驟入政府者;有以諛佞之資,附阿侂胄,致身顯貴者。陳自強老不知恥,貪不知止,私植黨與,陰結門第,凡見諸行事,惟知侂胄,不知君父。此外患之居吾股肱者也。



 爽、奕、汝翼諸李之貪懦無謀,倪、僎、倬、杲諸郭之膏粱無用,諸吳之恃寵專僭,諸彭之庸孱不肖;皇甫斌、魏友諒、毛致通、秦世輔之雕瘵軍心、瘡痍士氣,以致陳孝慶、
 夏興祖、商榮、田俊邁之徒,皆以一卒之材,各得把麾專制,平日剜膏刻血,包苴侂胄,以致通顯,饑寒之士咸願食其肉而不可得。萬一陛下付以大事,彼之首領自不可保,奚暇為陛下計哉?此外患之居吾爪牙者也。



 程松之納妾求知,或以售妹入府,或以獻妻入閣,魯之貢子為郎,富宮之庸駑充位。此外患之居吾耳目者也。



 蘇師旦以穢吏冒節鉞,牙儈名爵;周筠以隸卒冒戎鈐,市易將相。此外患之扼吾咽喉者也。彼之所謂外患者實
 未足憂,而此之外患蓋已周吾一身之間矣。



 「禮樂征伐,自天子出」。所貴乎中國者,皆聽命於陛下也。今也與奪之命、黜陟之權,又不出於陛下,而出於侂胄。是吾有二中國也。命又不出於侂胄,而出於蘇師旦、周筠。是吾有三中國也。女真以區區之地,猶能逼我淮、漢,曾謂外患之居吾腹心、股肱、耳目、爪牙及吾咽喉,而不馮陵吾之宗廟社稷乎?曾謂一家之中自為秦、越,一舟之中自為敵國,而能制遠人乎?比年軍皆掊克,而士卒自仇其將
 佐;民皆侵漁,而百姓自畔其守令,家自為戰。此又啟吾中國億萬之仇敵也。今不務去吾腹心、股肱、爪牙、耳目、咽喉與夫億萬之仇敵,而欲空國之師,竭國之財,而與遠人相從於血刃相塗之地,顧不外用其心歟?



 臣嘗推演兵書,自去歲上元甲子,五福太一初度吳分,四神直符對臨荊、楚,始擊蜚符旁臨甌、粵,青門直使交次於幽、冀,黑殺黃道正按於燕、趙。考之成法,主算最長,客算最短。兵以先發為客,後發為主。自太歲乙丑至庚午六年
 之間,皆不利於先舉。儻其畔盟犯義,撓我疆場,至於事不獲已,然後應之,則反主為客,猶曰庶幾。萬一國家首事倡謀,則將帥內睽,士卒外畔,肝腦萬民,血刃千里。此天數之不利於先舉也。矧將帥庸愚,軍民怨懟,馬政不講,騎士不熟,豪傑不出,英雄不收,饋糧不豐,形便不固,山砦不修,堡壘不設,吾雖帶甲百萬,餫餉千里,而師出無功,不戰自敗。此人事之不利於先舉也。



 臣願陛下除吾一身之外患。吾國中之外患既已除,然後公道開明,
 正人登用,法令自行,紀綱自正,豪傑自歸,英雄自附,侵疆自還,中原自復;天下自底於和平,四海自躋於仁壽,何俟乎兵革哉?不然,則亂臣賊子毀冕裂冠,哦九錫隆恩之詩,恃貴不可侔之相,私妾內姬,陰臣將相,魚肉軍士,塗炭生靈,墜百世之遠圖,虧十廟之遺業。陛下此時雖欲不與之偕亡,則禍迫於身,權出於人,俯首待終,何臍可噬。



 事之未然,難以取信,臣願以身屬之廷尉,待其軍行用師,勞還奏凱,則梟臣之首風遞四方,以為天下
 欺君罔上者之戒。儻或干戈相尋,敗亡相繼,強敵外攻,奸臣內畔,與臣所言盡相符契,然後令臣歸老田里,永為不齒之民。



 書奏,侂胄大怒,下大理,貶建寧圜土中。郡守傅伯成憐之,命獄卒使出入毋系。伯成去,又迕守李大異,復置獄。



 侂胄誅,放還,復入學登第,為殿前司官屬,鬱不得志。謀去丞相史彌遠,事覺,下臨安獄。獄具,坐議大臣當死。寧宗知嶽名,欲生之,彌遠曰:「是欲殺臣者。」竟杖死東市。



 鄧若水,字平仲,隆州井研人。博通經史,為文章有氣骨。吳曦叛,州縣莫敢抗,若水方為布衣,憤甚,將殺縣令,起兵討之。夜刲雞盟其僕曰:「我明日謁知縣,汝密懷刃以從,我顧汝,即殺之。」僕佯許諾,至期三顧不發。歸責其僕以背盟,僕曰:「平人尚不可殺,況知縣乎?此何等事,而使我為之。」若水乃仗劍徒步如武興,欲手刃曦,中道聞曦死,乃還。人皆笑其狂,而壯其志。



 登嘉定十三年進士第。時史彌遠柄國久,若水對策極論其奸,請罷之,更命賢
 相,否則必為宗社憂。考官置之末甲。策語播行,都士爭誦之。彌遠怒,諭府尹使逆旅主人幾其出入,將置之罪,或為之解,乃已。



 理宗即位,應詔上封事曰:



 行大義然後可以弭大謗,收大權然後可以固大位,除大奸然後可以息大難。



 寧宗皇帝晏駕,濟王當繼大位者也,廢黜不聞於先帝,過失不聞於天下。史彌遠不利濟王之立,夜矯先帝之命,棄逐濟王,並殺皇孫,而奉迎陛下。曾未半年,濟王竟不幸於湖州。揆以《春秋》之法,非弒乎?非篡乎?
 非攘奪乎?當悖逆之初,天下皆歸罪彌遠而不敢歸過於陛下者,何也?天下皆知倉卒之間,非陛下所得知,亦諒陛下必無是心也,亦料陛下必能清表妖氛,以雪先帝、濟王父子終天之憤。今逾年矣,而乾剛不決,威斷不行,無以大慰天下之望。昔之信陛下之必無者,今或疑其有。昔之信陛下不知者,今或疑其知。陛下何以忍清明天日,而以此身受此污辱也?蓋亦求明是心於天下,而俾有辭於千古乎?為陛下之計,莫若遵泰伯之至德,
 伯夷之清名,季子之高節,而後陛下之本心明於天下。此臣所謂行大義以弭大謗,策之上也。



 自古人君之失大權,鮮有不自廢立之際而盡失之。當其廢立之間,威動天下。既立則眇視人主,是故強臣挾恩以陵上,小人怙強以無上,久則內外相為一體,為上者喑默以聽其所為,日朘月削,殆有人臣之所不忍言者。威權一去,人主雖欲固其位,保其身,有不可得。宣繒、薛極,彌遠之肺腑也;王愈,其耳目也;盛章、李知孝,其鷹犬也;馮榯,其爪
 牙也。彌遠之欲行某事,害某人,則此數人者相與謀之,曷嘗有陛下之意行乎其間哉?臣以為不除此數兇,陛下非惟不足以弭謗,亦未可以必安其位,然則陛下何憚久而不為哉?此臣所以謂收大權以定大位,策之次也。



 次而不行,又有一焉,曰:除大奸然後可以弭大難。李全,一流民耳,寓食於我,兵非加多,土地非加廣,勢力非特盛也。賈涉為帥,庸人耳,全不敢妄動,何也?名正而言順也。自陛下即位,乃敢倔強,何也?彼有辭以用其眾也。
 其意必曰:「濟王,先皇帝之子也,而彌遠放弒之。皇孫,先皇帝之孫也,而彌遠戕害之。」其辭直,其勢壯,是以沿淮數十萬之師而不敢睥睨其鋒。雖曰今暫無事,未也,安知其不一日羽檄飛馳,以濟王為辭,以討君側之惡為名?彌遠之徒,死有餘罪,不可復惜,宗社生靈何辜焉?陛下今日而誅彌遠之徒,則全無辭以用其眾矣。上而不得,則思其次,次而不得,則思其下,悲夫!



 制置司不敢為附驛,卻還之。以格當改官,奏上,彌遠取筆橫抹之而罷。



 嘉熙間,召為太學博士,當對,草奏數千言,略曰:「寧宗不豫,彌遠急欲成其詐,此其心豈復願先帝之生哉?先帝不得正其終,陛下不得正其始,臣請發塚斫棺,取其尸斬之,以謝在天之靈。往年臣嘗上封事,請禪位近屬,以洗不義之污,無路自達,今其書尚在,謹昧死以聞。」



 將對前一日,假筆吏於所親潘允恭,允恭素知若水好危言,諭筆吏使竊錄之。允恭見之,懼並及禍,走告丞相喬行簡,亦大駭。翼日早朝,奏出若水通判寧國府。退朝,召閣
 門舍人問曰:「今日有輪對官乎?」舍人以若水對,行簡曰:「已得旨補外矣,可格班。」若水袖其書待廡下,舍人諭使去,若水怏怏而退。自知不為時所容,到官數月,以言罷,遂不復仕,隱太湖之洞庭山。



 賈似道在京湖,聞其名,闢參軍事。若水雅思其鄉,乃起從其招,因西歸蜀。居山中,有盜夜劫之,若水危坐不動,盜擊其首,流血被面,亦不動,乃舍去。若水為學務躬行,恥為空言。削木為主,大書曰「自古以來忠臣孝子義夫節婦之位」,歲時祀之。有一
 子,膂力絕人,築山砦,以兵捍衛鄉井。砦破,舉家遇害。



 僧真寶,代州人,為五臺山僧正。學佛,能外死生。靖康之擾,與其徒習武事於山中。欽宗召對便殿,眷齎隆縟。真寶還山,益聚兵助討。州不守,敵眾大至,晝夜拒之,力不敵,寺舍盡焚。酋下令生致真寶,至則抗詞無撓,酋異之,不忍殺也。使郡守劉騊誘勸百方,終不顧,且曰:「吾法中有口四之罪,吾既許宋皇帝以死,豈當妄言也?」怡然受戮。北人聞見者嘆異焉。



 莫謙之,常州宜興僧人也。德祐元年,糾合義士捍禦鄉閭,詔為溧陽尉。是冬,沒於戰陳,贈武功大夫。



 時萬安僧亦起兵,舉旗曰「降魔」,又曰:「時危聊作將,事定復為僧。」旋亦敗死。



 徐道明,常州天慶觀道士也。為管轄,賜紫。德祐元年,北兵圍城,道明謁郡守姚訔請曰:「事急矣,君侯計將安出?」訔曰:「內無食,外無援,死守而已。」道明亟還,慨然告其徒曰:「姚公誓與城俱亡,吾屬亦不失為義士。」乃取觀之文
 籍置石函,藏坎中。兵屠城,道明危坐焫香,讀《老子》書。兵使之拜,不顧,誦聲瑯然;以刃脅之,不為動,遂死焉。



\end{pinyinscope}