\article{列傳第三十一}

\begin{pinyinscope}

 楊業子延昭等王貴附荊罕儒從孫嗣曹光實從子克明張暉司超



 楊業,並州太原人。父信,為漢麟州刺史。業幼倜儻任俠,善騎射,好畋獵,所獲倍於人。嘗謂其徒曰:「我他日為將
 用兵,亦猶用鷹犬逐雉兔爾。」弱冠事劉崇,為保衛指揮使,以驍勇聞。累遷至建雄軍節度使,屢立戰功,所向克捷,國人號為「無敵」。



 太宗征太原,素聞其名,嘗購求之。既而孤壘甚危,業勸其主繼元降,以保生聚。繼元既降,帝遣中使召見業,大喜,以為右領軍衛大將軍。師還,授鄭州刺史。帝以業老於邊事,復遷代州兼三交駐泊兵馬都部署,帝密封橐裝,賜予甚厚。會契丹入雁門,業領麾下數千騎自西陘而出,由小陘至雁門北口,南向背擊
 之,契丹大敗。以功遷雲州觀察使,仍判鄭州、代州。自是,契丹望見業旌旗即引去。主將戍邊者多忌之,有潛上謗書斥言其短,帝覽之皆不問,封其奏以付業。



 雍熙三年,大兵北征,以忠武軍節度使潘美為雲、應路行營都部署,命業副之,以西上閣門使、蔚州刺史王侁,軍器庫使、順州團練使劉文裕護其軍。諸軍連拔雲、應、寰、朔四州,師次桑乾河,會曹彬之師不利,諸路班師,美等歸代州。



 未幾,詔遷四州之民於內地,令美等以所部之兵護
 之。時契丹國母蕭氏與其大臣耶律漢寧、南北皮室及五押惕隱領眾十餘萬,復陷寰州。業謂美等曰:「今遼兵益盛,不可與戰。朝廷止令取數州之民,但領兵出大石路,先遣人密告云、朔州守將,俟大軍離代州日,令雲州之眾先出。我師次應州,契丹必來拒,即令朔州民出城,直入石碣穀。遣強弩千人列於谷口,以騎士援於中路,則三州之眾,保萬全矣。」侁沮其議曰:「領數萬精兵而畏懦如此。但趨雁門北川中,鼓行而往。」文裕亦贊成之。業
 曰:「不可,此必敗之勢也。」侁曰:「君侯素號無敵,今見敵逗撓不戰,得非有他志乎?」業曰:「業非避死,蓋時有未利,徒令殺傷士卒而功不立。今君責業以不死,當為諸公先。」將行,泣謂美曰:「此行必不利。業,太原降將,分當死。上不殺,寵以連帥,授之兵柄。非縱敵不擊,蓋伺其便,將立尺寸功以報國恩。今諸君責業以避敵,業當先死於敵。」因指陳家谷口曰:「諸君於此張步兵強弩,為左右翼以援,俟業轉戰至此,即以步兵夾擊救之,不然,無遺類矣。」美
 即與侁領麾下兵陣於谷口。自寅至巳,侁使人登托邏臺望之,以為契丹敗走,欲爭其功,即領兵離穀口。美不能制,乃緣交河西南行二十里。俄聞業敗,即麾兵卻走。業力戰,自午至暮,果至谷口。望見無人,即拊膺大慟,再率帳下士力戰,身被數十創,士卒殆盡,業猶手刃數十百人。馬重傷不能進,遂為契丹所擒,其子延玉亦沒焉。業因太息曰:「上遇我厚,期討賊捍邊以報,而反為奸臣所迫,致王師敗績,何面目求活耶!」乃不食,三日死。



 帝聞
 之,痛惜甚,俄下詔曰:「執干戈而衛社稷,聞鼓鼙而思將帥。盡力死敵,立節邁倫,不有追崇,曷彰義烈!故雲州觀察使楊業誠堅金石,氣激風雲。挺隴上之雄才,本山西之茂族。自委戎乘,式資戰功。方提貔虎之師,以效邊陲之用。而群帥敗約,援兵不前。獨以孤軍,陷於沙漠;勁果猋厲,有死不回。求之古人,何以加此!是用特舉徽典,以旌遺忠。魂而有靈,知我深意。可贈太尉、大同軍節度,賜其家布帛千匹、粟千石。大將軍潘美降三官,監軍王侁
 除名、隸金州,劉文裕除名、隸登州。」



 業不知書,忠烈武勇,有智謀。練習攻戰,與士卒同甘苦。代北苦寒,人多服氈罽,業但挾纊露坐治軍事,傍不設火,侍者殆殭僕,而業怡然無寒色。為政簡易,御下有恩,故士卒樂為之用。朔州之敗,麾下尚百餘人,業謂曰:「汝等各有父母妻子,與我俱死,無益也,可走還,報天子。」眾皆感泣不肯去。淄州刺史王貴殺數十人,矢盡遂死,餘亦死,無一生還者。聞者皆流涕。業既沒,朝廷錄其子供奉官延朗為崇儀副
 使,次子殿直延浦、延訓並為供奉官,延瑰、延貴、延彬並為殿直。



 延昭本名延朗,後改焉。幼沉默寡言,為兒時,多戲為軍陣,業嘗曰:「此兒類我。」每征行,必以從。太平興國中,補供奉官。業攻應、朔,延昭為其軍先鋒,戰朔州城下,流矢貫臂,鬥益急。以崇儀副使出知景州。時江、淮兇歉,命為江、淮南都巡檢使。改崇儀使、知定遠軍,徙保州緣邊都巡檢使,就加如京使。



 咸平二年冬,契丹擾邊,延昭時在遂
 城。城小無備,契丹攻之甚急,長圍數日。契丹每督戰,眾心危懼,延昭悉集城中丁壯登陴,賦器甲護守。會大寒,汲水灌城上,旦悉為冰,堅滑不可上,契丹遂潰去,獲其鎧仗甚眾。以功拜莫州刺史。時真宗駐大名,傅潛握重兵頓中山。延昭與楊嗣、石普屢請益兵以戰,潛不許。及潛抵罪,召延昭赴行在,屢得對,訪以邊要。帝甚悅,指示諸王曰:「延昭父業為前朝名將,延昭治兵護塞有父風,深可嘉也。」厚賜,遣還。是冬,契丹南侵,延昭伏銳兵於羊
 山西,自北掩擊,且戰且退。及山西,伏發,契丹眾大敗,獲其將,函首以獻。進本州團練使,與保州楊嗣並命。帝謂宰相曰:「嗣及延昭,並出疏外,以忠勇自效。朝中忌嫉者眾,朕力為保庇,以及於此。」五年,契丹侵保州,延昭與嗣提兵援之,未成列,為契丹所襲,軍士多喪失。命李繼宣、王汀代還,將治其罪。帝曰:「嗣輩素以勇聞,將收其後效。」即宥之。六年夏,契丹復侵望都,繼宣逗遛不進,坐削秩,復用延昭為都巡檢使。時講防秋之策,詔嗣及延昭條
 上利害,又徙寧邊軍部署。



 景德元年,詔益延昭兵滿萬人,如契丹騎入寇,則屯靜安軍之東。令莫州部署石普屯馬村西以護屯田。斷黑盧口、萬年橋敵騎奔沖之路,仍會諸路兵掎角追襲,令魏能、張凝、田敏奇兵牽制之。時王超為都部署,聽不隸屬。延昭上言:「契丹頓澶淵,去北境千里,人馬俱乏,雖眾易敗,凡有剽掠,率在馬上。願飭諸軍,扼其要路,眾可殲焉,即幽、易數州,可襲而取。」奏入,不報,乃率兵抵遼境,破古城,俘馘甚眾。



 及請和,真宗
 選邊州守臣,御筆錄以示宰相,命延昭知保州兼緣邊都巡檢使。二年,追敘守禦之勞,進本州防禦使,俄徙高陽關副都部署。在屯所九年,延昭不達吏事,軍中牒訴,常遣小校周正治之,頗為正所罔,因緣為奸。帝知之,斥正還營而戒延昭焉。大中祥符七年,卒,年五十七。



 延昭智勇善戰,所得奉賜悉犒軍,未嘗問家事。出入騎從如小校,號令嚴明,與士卒同甘苦,遇敵必身先,行陣克捷,推功於下,故人樂為用。在邊防二十餘年,契丹憚之,目
 為楊六郎。及卒,帝嗟悼之,遣中使護櫬以歸,河朔之人多望柩而泣。錄其三子官,其常從、門客亦試藝甄敘之。子文廣。



 文廣字仲容。以班行討賊張海有功,授殿直。範仲淹宣撫陜西,與語奇之,置麾下。從狄青南征,知德順軍,為廣西鈐轄,知宜、邕二州,累遷左藏庫使、帶御器械。治平中,議宿衛將,英宗曰:「文廣,名將後,且有功。」乃擢成州團練使、龍神衛四廂都指揮使,遷興州防禦使。秦鳳副都總
 管韓琦使築篳篥城,文廣聲言城噴珠,率眾急趣篳篥,比暮至其所,部分已定。遲明,敵騎大至,知不可犯而去,遺書曰:「當白國主,以數萬精騎逐汝。」文廣遣將襲之,斬獲甚眾。或問其故,文廣曰:「先人有奪人之氣。此必爭之地,彼若知而據之,則未可圖也。詔書褒諭,賜襲衣、帶、馬。知涇州、鎮戎軍,為定州路副都總管,遷步軍都虞候。遼人爭代州地界,文廣獻陣圖並取幽燕策,未報而卒,贈同州觀察使。



 王貴者,並州太原人。廣順初,補衛士。宋初,累遷至散員都指揮使、馬步軍都軍頭,領勝州刺史。太平興國二年,出為淄州刺史。受詔從潘美北征,攻沁州,頗立戰功。及從楊業,為遼兵所圍,親射殺數十人,矢盡,張空弮又擊殺數人,遂遇害。年七十三。擢其子文晟供奉官、文昱殿直。



 荊罕儒,冀州信都人。父基,王屋令。罕儒少無賴,與趙鳳、張輦為群盜。晉天福中,相率詣範陽,委質燕王趙延壽,
 得掌親兵。開運末,延壽從契丹主德光入汴,署罕儒密州刺史。漢初,改山南東道行軍司馬。周廣順初,為率府率,奉朝請,貧不能振。顯德初,世宗戰高平,戮不用命者,因求驍勇士。通事舍人李延傑以罕儒聞,即召赴行在,命為招收都指揮使。會征太原,命罕儒率步卒三千先入敵境。罕儒令人負束芻徑趨太原城,焚其東門。擢為控鶴弩手、大劍直都指揮使。從平淮南,領光州刺史,改泰州,為下蔡守禦都指揮使兼舒、蘄二州招安巡檢使。
 四年,泰州初下,真拜刺史兼海陵、鹽城兩監屯田使。明年三月,世宗幸泰州,以罕儒為團練使,賜金帶、銀器、鞍勒馬。六年春,軍吏耆艾詣闕請留,恭帝詔褒之。



 建隆初,升鄭州防禦,以罕儒為使,改晉州兵馬鈐轄。罕儒恃勇輕敵,嘗率騎深入晉境,人多閉壁不出,虜獲甚眾。是年冬,復領千餘騎抵汾州城下,焚其草市,案兵以退。夕次京土原,劉鈞遣大將郝貴超領萬餘眾襲罕儒,黎明及之。罕儒遣都監、氈毯副使閻彥進分兵以御貴超。罕儒
 錦袍裹甲據胡床享士,方割羊臂臑以啖,聞彥進小卻,即上馬麾兵徑犯賊鋒。並人攢戈舂之,罕儒猶格鬥,手殺十數人,遂遇害。劉鈞素畏罕儒之勇,常欲生致,及聞其死,求殺罕儒者戮之。太祖痛惜不已,擢其子守勛為西京武德副使。因索京土原之不效命者,黜慈州團練使王繼勛為率府率,閻彥進為殿直,斬其部下龍捷指揮使石進德等二十九人。



 罕儒輕財好施。在泰州,有煮海之利,歲入鉅萬,詔聽十收其八,用猶不足。家財入有
 籍,出不問其數。有供奉官張奉珪使泰州,自言後唐張承業之子。罕儒曰:「我生平聞張特進名,幸而識其子。」厚加禮待,遺錢五十萬,米千斛。



 罕儒雖不知書,好禮接儒士。進士趙保雍登科覆落,客游海陵。罕儒問其所欲,保雍以將歸京師,且言緣江榷務以絲易茗有厚利。罕儒立召主藏奴,令籍藏中絲,得四千餘兩,盡以與之。然好勇善戰,不顧勝負。常欲削平太原,志未果而及於敗,人皆惜之。罕儒兄延福。延福孫嗣。



 嗣,乾德初,應募為控鶴卒,從李繼勛討河東。繼勛擇悍勇百人,間道截洛陽砦。嗣出行間請行,手斬五十餘級,賊焚砦宵遁。進薄汾河,賊將楊業扼橋路,嗣與眾轉戰,賊退逾橋。殺業所部兵千計,射中業從騎,獲旗鼓鎧甲甚眾,業退保城。進焚南門,奪羊馬城,矢集於面。賊數千夜來薄砦,繼勛選勇敢五百人接戰,而嗣為冠。及旦,戰數合,多所斬馘。



 從太祖征太原,賊來拒,焚洞子。遣殿前楊信領百人援之,嗣預焉,率先陷陣。召見,補御龍直。太
 平興國初,三遷至天武軍校。太宗再徵太原,嗣自陳願率一隊先登,命主城西洞子。車駕巡師,嗣登城,手刃數賊,足貫雙箭,中手炮,折二齒。太宗見之,亟召賜錦袍、銀帶。從征幽州,隸殿前崔翰,斬三十級,補龍猛副指揮使。



 五年,契丹侵雄州,據龍灣堤。嗣隸袁繼忠,繼忠令率千兵力戰奪路。內侍有至州閱城壘者,出郛外,敵進圍之,亟出兵接戰,十數合,斬騎卒七百餘。嗣軍夜相失,在古城莊外,三鼓突敵圍,壁於莫州城下,又領百人斧敵望
 櫓,斬五十級。敵為橋界河,將遁,嗣邀擊之,殺獲甚眾。六年,從崔彥進捍契丹於靜戎北,砦於慎興口。彥進遣嗣率所部度河,與契丹戰,敗之,追奔二十餘里。八年,李繼遷寇邊,嗣從袁繼忠、田欽祚戍三叉口,為前鋒,斬賊千餘,追之,獲牛羊、鎧甲、弓矢數千計。進至萬井口、狐路谷,餘賊復來請戰。初以雄武千人為後殿,為賊所掩。繼忠命嗣援之,凡數戰,始與雄武合隊,因列陣格鬥,復奪人馬七百餘。欽祚夜還,依山為營,賊亦砦其下。募勁卒五
 十往襲之,嗣為其帥。抵賊所,刺殺百餘人,焚其砦而還,詔賜錦袍、銀帶。



 雍熙三年,從田重進、譚延美率師入遼境,疾戰飛狐口,遼師不利。重進引全師合擊,遼騎引去。進至飛狐城北,遼將大鵬翼率眾復至。重進陣壓東偏,數戰不勝,命嗣出西偏,麾兵薄山崖,以短兵接戰。遼兵敗,投崖而下,手斬百餘級。散卒千餘在野,嗣呵止之,悉斷弦折筈來降,追至河槽,復擊退。餘眾屯土嶺,裨將黃明與戰不勝,將退,嗣謂之曰:「汝且頓兵於此,為我聲援,
 我當奪此嶺。」遂力戰,追奔五十餘里,抵倉頭而還。又領招收卒千人,克倉頭、小治二砦。黃明與戰,克直谷砦,命嗣屯焉。數日,遼人復致師,重進與戰,奔突往來,大軍頗擾。重進召嗣合戰,悉走之,奪炮具、鎧冑。賊乘夜復圍直谷、石門二砦,重進遣嗣以精兵五百濟之,嗣曰:「敵二萬餘,今援師甚寡,難以解圍。」重進頗憂之。嗣曰:「譚師屯小治,綰兵二千,願間道以往,邀其策應。」中夜,匹馬詣延美,延美曰:「敵勢若此,何可解也?」嗣曰:「請移全軍就平川,植
 旗立隊,別擇三二百人張白旗於道側。彼見旗幟綿亙遠甚,謂大軍繼至,嗣自以所部五百疾驅往斗,必克其砦。」延美許焉。一日凡五七戰,遼兵遂引去,咸如嗣所料。



 蔚州之降也,重進先命嗣率勇士數十人縋入,見守將,得其實狀。翌日,將受降,而敵反拒大軍所出之路,遂與鬥,殺傷甚眾,屢縋入城,取守將之歸服者。重進之壘,糧運頗乏,嗣遣降卒輦州廩濟之。遼援兵大至,副都指揮使江謙妄言惑眾,嗣即斬之。悉收兵斂輜重還重進砦,
 與遼人轉戰。時軍校五人,其四悉鬥死,至大嶺,嗣與戰,敗走之。師還,太宗引見便殿,重進言其有勞,補本軍都虞候。



 又從李繼隆禦敵於北平砦,將赴蒲城,道遇敵,疾戰,俘獲甚眾。又戰於鸞女祠,繼隆遣步卒二千,伏定州古城,為敵所攻,命嗣援之。至唐河橋,嗣扼橋路出戰,解敵圍數重,與伏兵合,分為三隊,背水為陣。敵將於越率騎百餘隊臨烽臺求戰,嗣整兵與戰,數合,得與繼隆會,又陣於東偏,大敗之。繼隆以聞,詔嘉獎之,遷本軍都指
 揮使,領澄州刺史。



 至道二年,加御前忠佐馬步軍頭、屯定州。遼人入侵,隸範廷召,提偏師捍遼兵於嘉山。廷召徙高陽,命嗣以兵二千為殿。過平敵城,遼眾十餘萬來,嗣屢出戰。及桑贊、秦翰來援,夜二鼓,敵再至,嗣曰:「彼不利夜戰,我當破其砦,以趣大軍。」即與贊、翰合勢,戒所部望敵炬火多處並力沖之,詰旦,至瀛州。咸平三年,加領本州團練使,出為郎山路都巡檢使,破敵砦於蒲陰,俘獲甚眾。四年,命嗣領萬人斷西山路。會敵遽至,大兵不
 及進而止。五年,真拜蔡州團練使、趙州部署。逾年,徙滄州。是冬,遼人入侵,命率所部自齊州抵淄、青警備。景德初,又命與劉漢凝、田思明率兵至冀州防邊。俄赴澶州行在所。會遼人請和,復遣還任。歷鄆州、鳳翔、永興部署。車駕幸亳州,留為舊城內同都巡檢使。大中祥符七年,改虢州防禦使、邠寧環慶副部署,卒。嗣起行間,以勞居方面,經百五十餘戰,歿。兄子信、貴,並為左侍禁,貴至內殿崇班。



 曹光實,雅州百丈人。父疇,為蜀靜南軍使,控扼邛崍,以捍蠻夷。光實少武勇,有膽氣,輕財好施,不事細行,意豁如也。疇卒,光實嗣職,遷永平軍節度管內捕盜游奕使。



 乾德中,太祖命王全斌等平蜀。俄而盜賊群起,夷人張忠樂者,嘗群行攻劫,且憾光實殺其徒黨,率眾數千,中夜奄至,環其居,鼓噪並進。光實負其母,揮戈突圍以出,賊眾闢易不敢近,賊殺其族三百餘口。又發塚墓,壞其棺槨。光實詣全斌,具以事白,誓雪冤憤。時蜀中諸郡未
 下,乃圖雅州地形要害,兼陳用兵攻取之策,請官軍先下之。全斌壯其志,令率兵先導,果克其城,獲忠樂而甘心焉。全斌乃署光實為義軍都指揮使。殘寇猶據沈黎,光實以所部盡平之。遂以光實知黎、雅二州兼都巡檢使,安集勞來,蠻族懷之。



 六年秋,全斌遣入貢京師,遂言境內安乂,乞罷義軍歸農。太祖喜,謂左右曰:「此蜀中傑俊也。」詔升殿,勞問久之,以為黎州刺史。開寶三年,改唐州刺史。及平交、廣,群盜未息,以光實為嶺南諸州都巡
 檢使。既至,捕逐群盜,海隅以寧。太平興國二年,就遷本州團練使。車駕征河東,以光實知威勝軍事,令調軍食。光實入告,願提一旅奮銳先登,帝曰:「資糧事重,亦足宣力也。」河東平,命為汾、遼、石、沁等州都巡檢使。五年,改汝州團練使。大軍北征,與潘美分道出雁門。光實為前鋒,遇敵迎擊,敗之,斬首數千級,優詔嘉獎。



 李繼捧之入朝也,以光實為銀、夏、綏、麟、府、豐、宥州都巡檢使。繼捧弟繼遷逃入蕃落,為邊患,光實乘間掩襲至地斤澤,俘斬甚
 眾,破其族帳,獲繼遷母妻及牛羊萬計。繼遷僅免,使人紿光實曰:「我數奔北,勢窘不能自存矣,公許我降乎?」因致情款,陳甥舅之禮,期某日降於葭蘆川。光實信之,且欲專其功,不與人謀。及期,繼遷先設伏兵,令十數人近城迎致光實,光實從數百騎往赴之。繼遷前導北行,將至其地,舉手麾鞭而伏兵應之,光實遂遇害,卒,年五十五。帝聞之驚悼,賵賻加等,以其子大理評事克讓為右贊善大夫,克恭為殿直。淳化二年,又錄克己為奉職,後
 至內殿承制;克廣至閣門祗候。從子克明。



 克明字堯卿。既生,會敵攻百丈縣,父光遠遇害,姆抱克明匿葦蒲中得免。既長,喜兵法,善騎射,從父光實奇之。補為衙內都虞候。光實擊敵於葭蘆州,戰歿。克明時護輜重在後,聞光實死,懼軍亂,秘不發喪。陽令人西來傳光實命還軍銀州,而潛與僕張貴入敵中,獲光實尸以還,葬京師,由是顯名。



 初,蜀人留京師者禁不得還鄉里,克明以母老,間道歸。李順反,聞克明將家子,且有名,欲
 脅以官。克明攜母遁山谷,夜止神祠中,夢有人叱之起,既覺而去,賊果至。及賊陷雅州,克明募眾數萬人以迎王師,遂復名山、火井、夾江等九縣。分兵嘉、眉、邛三州,立七砦以邀賊。復收雅州,斬六十餘人,賊將何承祿等走雲南。蜀平,擢西頭供奉官、黎州兵馬監押。以餘寇未息,權邛州駐泊巡檢。明年,峽路潰卒鄧紹等復起攻雅州,克明又平之。還軍邛州,遇賊王珂,戰於延貢鎮,擊以矛,中左踝。後又設伏山下,以數十騎與賊接戰,克明偽北,
 而所部失期,伏不發。克明挺身走,賊追急,乃倚大石引弓三發,斃三人,由是獲免。入朝,改內殿崇班,為溫、臺等七州都巡檢使。



 景德中,蠻寇邕州,改供備庫副使、知邕州。左、右江蠻洞三十六,克明召其酋長,諭以恩信,是歲承天節,相率來集。克明慰拊,出衣服遺之,感泣而去。獨如洪峒恃險不至,克明諭兩江防遏使黃眾盈引兵攻之,斬其首領陸木前,梟於市。



 宜州澄海軍校陳進反。時鬱江暴漲,州城摧圯,克明率丁夫伐木為連舫,維之水
 上,狀如郛郭。又多張旗幟,浮巨筏,陳兵其上,為守禦備。募溪峒兵三千,而黃眾盈亦濟兵千五百,將趣象州。會巡撫使曹利用約克明會兵,行次貴州,遇賊,大敗之,斬首四百餘級。賊平,利用專其功。代還,真宗問南方事,對稱旨,賜一子官,遷供備庫使,江、淮、兩浙都大提舉捉賊。克明使人捕賊,輒出私錢資之,以故人人盡力。視賊中趫勇者,釋縛,使還捕其黨,前後獲千餘人。知江寧府張詠以其事聞,賜錢四十萬,領平州刺史、知辰州。撫水蠻
 叛,徙宜、融、桂、昭、柳、象、邕、欽、廉、白十州都巡檢使兼安撫使。既至,蠻酋獻藥一器,曰「溪峒藥」,藥箭中人,以是解之可不死。克明曰:「何以驗之?」曰:「請試以雞犬。」克明曰:「當試以人。」乃取藥箭刺酋股而飲以藥,即死,群蠻慚懼而去。



 是年冬,安撫都監王文慶、馬玉出天河砦東,克明與中人楊守珍出環州樟嶺西,磴道危絕,林木深阻,蠻多伏弩以待。玉所向力戰,屢敗蠻軍。是時朝廷意在招附,數詔諭克明,而克明亦憚深入,屢移文止玉。玉至如門團,
 為蠻所扼,不得進。克明遷延顧望,月餘,乃至撫水州,與知州蒙承貴等約盟而還。



 未幾,知桂州兼管勾溪峒公事,始置溪峒司。又奏閱廣南兩路土軍為忠敢軍。州人覆茅為屋,歲多火,克明選北軍教以陶瓦,又激江水入城,以防火災。代還,知滁州,徙鼎州。會交址李公蘊寇邕州,以文思使復知邕州。既至,遣人入交址諭以利害,公蘊拜表謝罪。遷西上閣門使,歷知登、舒、邵三州,復徙鼎州,卒。



 張暉,幽州大城人。後唐清泰初,隸控鶴軍,累遷奉國、弩手都頭。晉開運末,與武行德奪契丹甲船於河陰。行德領河陽,以暉為弩手指揮使,復令引兵趣懷州。契丹將遁去,因領州軍。漢祖入汴,暉迎於滎陽,授懷州刺史。乾祐初,郢州刺史慕容業治多不法,以暉為緣漢都巡檢使,領唐州,屯兵至郢州,即代業。還京,改郢州刺史。



 周廣順初,劉崇寇晉、絳,召暉為步軍左廂排陣使。師還,改沂州刺史。三年,吏民詣闕舉留,俄改冀州。會詔築李晏口、
 束鹿、安平、博野、百八橋、武強等城,命暉護其役,逾月而就。從世宗征淮甸,充壕砦都指揮使。既拔楚、泗,即授泗州。未幾,改耀州,俄為西南面橋道使。



 宋初,從征澤、潞,為行營壕砦使,先登陷陣。事平,遷華州團練使,在郡頗有治狀。建隆二年,太原未下,詔入覲問計,暉對曰:「澤、潞經李筠之叛,瘡痍未復,軍旅一興,恐人力重困。不若戢兵育民,俟富庶而後為謀。」乃賜襲衣、金帶、鞍勒馬,令還州。朝廷方議伐蜀,遷鳳州團練使兼緣邊巡檢壕砦橋道
 使。暉盡得山川險易,因密疏陳之,太祖覽之大悅。乾德二年,大軍西下,乃以暉充西川行營先鋒都指揮使。督兵開大散關路,躬撫士卒,且役且戰,人忘其勞。十二月,至青泥嶺,卒。



 天禧五年,暉妻年百五歲,家貧,詣闕自陳。詔賜束帛,錄其孫永德為三班借職。



 司超,大名元城人。初事邢帥安叔千。漢祖在太原,超往依之,隸帳下為小校。漢祖將渡河,遣超先領勁騎,由晉、絳趨河陽。及入汴,以超為鄆州必敵指揮使。時京東諸
 州寇盜充斥,以超為宋、宿、亳三州游奕巡檢使。改宿州西固鎮守禦都指揮使,移屯穎州下蔡鎮。屢與淮人戰,有功。周世宗命宰相李谷討淮南,以超為步軍先鋒副都指揮使,又為廬、壽、光、黃等州巡檢使。大敗淮人三千餘眾於盛唐縣,獲棹船四十餘艘,禽其監軍高弼、果毅指揮使許萬以獻。時黃州未下,即命超遙領刺史兼樓櫓戰棹右廂都校。師還,改光州刺史,敗吳軍千餘於麻城北。顯德四年冬,與王審琦攻舒州,敗吳軍三千,先禽
 刺史施仁望獻於行在。即以超為舒州團練使。



 宋初,命副宋偓領舟師巡撫江徼,月餘,特詔升舒州為防禦,以超充使。太祖討李重進,以為前軍步軍都指揮使,及平,遣歸治所。建隆三年春,遷蔡州防禦使。乾德六年,改絳州防禦使,徙晉州兵馬鈐轄。是秋,又副趙贊為邠州行營都部署,進攻河東。及太祖親征,為行營前軍步軍都指揮使,改鄭州防禦使。開寶七年,朝廷將討江左,以超久在淮右,習知江山險易,徙蘄州防禦使,行至淮西,卒,
 年七十一。天禧元年,錄其孫文睿為三班奉職。



 論曰:昔許子卒於師,葬之加等。《春秋》書之,所以褒臣節而儆官守也。業、罕儒、光實咸當捍城之寄,臨戎力戰,歿於敵境。雖罕儒恃勇不戒,光實甘賊遷之言,失在輕敵,然其忘軀徇節,誠可嘉也。業本太原驍將,感太宗寵遇,思有以報。常勝之家,千慮一失。然其素得士心。部卒不忍離去,從之以歿,則忠義之風概可見矣。嗣與延昭並克紹勛伐。延昭久居邊閫,總戎訓士,威名方略,聞於敵
 人,於嗣為優。暉於危時則有陷陣之功,平日則獻息戎之諫。超頻戰以清淮海,其忠誠勇果,率有可尚者焉



\end{pinyinscope}