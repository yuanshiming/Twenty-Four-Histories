\article{列傳第八十二}

\begin{pinyinscope}

 蔚
 昭敏高化周美閻守恭孟元劉謙趙振張忠範恪馬懷德安俊向寶



 蔚昭敏,字仲明,開封祥符人。父興,事周世宗,數戰伐有
 功,又從太宗平太原,終龍尉都虞候。真宗為襄王,昭敏自東班殿侍選隸襄王府。帝即位,授西頭供奉官,累遷崇儀使、冀貝行營兵馬都監。契丹以五千騎突至冀州城南,昭敏帥部兵與戰,敗之,得其器甲,賊遁去,而師不失一人。



 咸平四年,領順州刺史、定州行營鈐轄兼押大陣,又為鎮、定、高陽關三路先鋒。契丹入寇,帝北巡至大名,契丹退趨莫州,昭敏與範廷召追至莫州東三十里,斬首萬餘級,擒生口甚眾,契丹委器甲遁去。拜唐州團
 練使,累遷至殿前副都指揮使,遷都指揮使、保靜軍節度使。以足疾,命入謁無拜。卒,贈侍中。



 高化,字仲熙,真定人。少沉勇有力,不事耕稼,學擊劍,善射。契丹犯河北,應募轉餉飛狐口。楊業留戲下,使捕賊酋大鵬翼,獲之。會契丹又犯真定,乃辭業還家,家屬盡為契丹所掠去。從州將入京師,遂隸禁軍,選為襄王牽□龍官。王尹京,命巡內外八廂,積獲奸盜甚眾。盜有遺化金帛者,化弗受。一日,王趨急召出府門,馬驚墮,化掖之
 而起。王曰:「微爾。吾幾殆。」益親信之。



 真宗即位,擢御龍弩直雙員都頭,累遷御龍骨朵直都虞候。乾興初,授天武右第二軍都指揮使、榮州刺史,遷天武右廂都指揮使、蜀州團練使。天聖六年夏,大雨,命護汴堤。夜馳至城西,堤欲壞,督守兵負土不能遏。時夏守恩方典軍,積材木城隅,化盡取以塞堤,乃得無患。仁宗嘉之,進神龍衛四廂都指揮使、龔州防禦使,為鄜延路馬步軍副都總管,徙涇原路、權知渭州,遷捧日、天武四廂都指揮使。



 發兵
 襲明珠族,不利,降滑州總管。改興州防禦使、真定路副都總管,徙高陽關路。修護章惠太后園陵,累拜殿前副都指揮使,歷建武軍節度使。以老,辭管軍。詔入朝,化又固請,改武安軍節度使、知滄州,未行,改相州。部有大獄已具,皆當論死。化疑之,遣移訊,果出無罪者三人。逾年,復告老,以右屯衛上將軍致仕。卒,年八十。贈太尉,謚曰恭莊。



 化謹質少過,馭軍有法。雖起身行伍,然頗知民事焉。



 周美,字之純,靈州回樂人。少隸朔方軍,以材武稱。趙保吉陷靈州,美棄其族,間走歸京師,天子召見,隸禁軍。契丹犯邊,真宗幸澶州,御城北門,美慷慨自陳,願假數騎縛契丹將至闕下,帝壯之,常令宿衛。



 天聖初,德明部落寇平涼方渠,美以軍候戍邊,與州將追戰,破之於九井原、烏侖河,斬首甚眾。累遷天武都虞候。元昊反,陜西用兵,經略使夏竦薦其材,擢供備庫使、延州兵馬都監。夏人既破金明諸砦,美請於經略使範仲淹曰:「夏人新得
 志,其勢必復來。金明當邊沖,我之蔽也,今不亟完,將遂失之。」仲淹因屬美復城如故。數日賊果來,其眾數萬薄金明,陣於延安城北三十里。美領眾二千力戰,抵暮,援兵不至,乃徙軍山北,多設疑兵。夏人望見,以為救至,即引去。既而復出艾蒿砦,遂至郭北平,夜鬥不解。美率眾使人持一炬從間道上山,益張旗幟,四面大噪,賊懼走。獲牛羊、橐駝、鎧甲數千計,遂募兵築萬安城而還。敵復寇金明,美引兵由虞家堡並北山而下,敵即引卻。遷文
 思使,徙知保定軍。經略使龐籍表留之,改東路都巡檢使。敗敵於金湯城,焚其族部二十一。



 元昊大入,據承平砦。諸將會兵議攻討,洛苑副使種世衡請繼三日糧直搗敵穴。美曰:「彼知吾來,必設伏待我。不如間道掩其不意。」世衡不聽。美獨以兵西出芙蓉谷,大破敵。世衡等果無功。未幾,敵復略土堆砦,美迎擊於野家店,追北至拓跋谷,大敗其眾。以功遷右騏驥使。軍還,築柵於蔥梅官道穀,以據敵路。令士卒益種營田,而收穀六千斛。復率
 眾繇廳子部西濟大理河,屠札萬多移二百帳,焚其積聚以歸。籍、仲淹交薦之,除鄜延路兵馬都監,遷賀州刺史。



 初,美自靈武來,上其所服精甲,詔藏軍器庫。至是,加飾黃金,遣使即軍中賜之。又破敵於無定河,乘勝至綏州,殺其酋豪,焚廬帳,獲牛馬、羊駝、器械三百計,因城龍口平砦。敵以精騎數千來襲,美從百餘騎馳擊破之。加本路鈐轄,遂為副總管。遷龍神衛四廂都指揮使、通州刺史;進捧日、天武四廂都指揮使,陵州團練使。



 慶歷中,
 又城清水、安定、黑水、佛堂、北橫山、乾谷、土明、柳谷、雕巢、盧兒、原安砦十一堡。安定之役,諜報敵數萬將大至,經略使遣管勾機宜楚建中分諸將兵,趣城黑水以待。諸將憚敵且至,不肯與兵。美曰:「兵常以寡擊眾,何自怯也。」卒以兵二千與建中,而敵亦引去。每邊書至,諸將各擇便利,獨美未嘗辭難,然所向輒克,諸將以此服之。歷侍衛親軍馬軍殿前都虞候、眉州防禦使、步軍副都指揮使、遂州觀察使、鄜延副都總管。召還,授耀州觀察使,又
 進馬軍副都指揮使。卒,贈忠武軍節度使,謚忠毅。



 自陜西用兵,諸將多不利,美前後十餘戰,平族帳二百,焚二十一,招種落內附者十一族,復城堡甚多。在軍中所得祿賜,多分其戲下,有餘,悉饗勞之。及死,家無餘貲。子蚤卒,以孫永清為子,官至引進副使。



 閻守恭,並州榆次人。父榮,倜儻有志略,劉繼元欲召至帳下,辭以母老不就。守恭生而體貌奇偉,榮曰:「是必當事太平天子,吾無恨矣。」後十七年,劉氏平,徙太原民於
 大名府,因家焉。往來負販於並、汾間,過西山,聞郭進為都巡檢使,太宗甚寵遇之。乃慨然曰:「進不遇主,亦行伍爾,吾自度豈不及進邪?」遂應募,隸拱聖軍,擢殿前押班。



 咸平中,從幸河北,以功為捧日副指揮使,歷拱聖、龍衛、捧日指揮使,累遷左第二軍指揮使、乾州刺史。明道中,落軍職,以德州刺史為永興軍兵馬鈐轄,徙並代路。



 守恭性沉勇,御軍嚴。雖家居如對賓客。常訪求士大夫,取郭進事而師法之。所得奉祿悉散予人。在並州,因春社
 會賓客曰:「守恭,太原一貧民爾。徒步位刺史,老復官鄉里,逾分多矣。今日與卿輩訣。」後十日卒。



 孟元,字善長,洺州人。性謹願少過,頗喜讀書。少隸禁軍,以挽強選補殿侍,累遷散都頭班指揮使,擢如京使、並代州兵馬都監,改鈐轄,徙高陽關路,又徙真定路。



 王則據貝州反,元赴城下攻戰,被數十創,又中機石,墜濠中。既出,戰愈力。更募死士由永濟渠穴地以進。賊平,改右騏驥使,徙大名府路鈐轄。河朔饑,權知滄州。民鬻鹽為
 生,歲荒鹽多不售,民無以自給。元度軍食有餘,悉用易鹽,繇是民不轉徙。



 御史中丞郭勸言其貝州功而賞未當,乃擢普州刺史,遷宮苑使,專管勾麟府軍馬事。護築永寧堡,敵不敢動。為龍神衛四廂都指揮使、忠州團練使、高陽關馬步軍總管,遷天武、捧日四廂都指揮使,又遷步軍都虞候、眉州防禦使、並代路副都總管。判北京賈昌朝奏為大名府路副都總管,徙定州路,遷馬軍都虞候,徙鄜延路,行至鄭州卒,贈遂州觀察使。



 劉謙,字漢宗,開封人。少補衛士,數遷至捧日右廂都指揮使,領嘉州團練使兼京城巡檢。元昊反,改博州團練使、環慶路馬步軍總管兼知邠州。謙不讀書,然鬥訟曲直,皆區處當理。前守者多強市民物以飾廚傳,謙獨無所撓,邠人頗愛之。夏竦奏為涇原路總管,徙知涇州,未行,會賊寇鎮戎軍,謙引兵深入賊境,破其聚落而還。以功擢龍神衛四廂都指揮使、象州防禦使。暴疾卒,贈永清軍節度觀察留後。



 趙振,字仲威,雄州歸信人。景德中,從石普於順安軍。獲契丹陣圖,授三班借職。後數年,為隰州兵馬監押,捕盜於青灰山,殺獲甚眾。



 高平蠻叛,徙湖北都巡檢使兼制置南路。以南方暑濕,弓弩不利,別創小矢,激三百步,中輒洞穿,蠻遂駭散。歲中,遷慶州沿邊都巡檢使。時,金湯李欽、白豹神木馬兒、高羅跛臧三族尤悍難制,振募降羌,啖以利,令相攻,破十餘堡。欽等詣振自歸。振為置酒,先酹,取細仗,圍財數分,植百步外共射。欽等百發不中,
 振十矢皆貫,欽等皆驚,誓不復敢犯。



 明年,涇原屬羌胡薩逋歌等叛,鈐轄王懷信以兵數千屬振游奕,屢捷。從數十騎詣懷信,遇賊十倍,射殪數十,餘悉退散。數月,賊數萬圍平遠砦,都監趙士龍戰沒。振出別道,力戰抵砦,奪取水泉,率敢死士破圍,賊走,追斬數千級,徙涇原都鹽,歷知順安、保安、廣信軍、霸州,改京東都大提舉捉賊。明年,知環州,累遷象州防禦使。



 元昊將反,為金銀冠佩隱飾甲騎遺屬羌,振潛以金帛誘取之,以破其勢,得冠
 佩銀鞍三千、甲騎數百。告鄰部俾以環為法,不聽,於是東茭、金明、萬劉諸族勝兵數萬,悉為賊所有。及劉平等皆敗,唯環慶無患。自本路馬步軍副總管擢龍神衛四廂都指揮使、鄜延路副都總管、知延州,代範雍。尋改捧日、天武四廂。振謂將吏曰:「今賊以我夷傷,必乘勝以進,勢宜固守。尚慮諸城不能皆如吾謀,茍延州弗支,則陜西未可測,此天下安危之機也。」



 未幾,賊寇塞門砦。振有兵幾八千,按甲不動。砦中兵才千人,屢告急,被圍五月,
 才遣百餘人赴之,砦遂陷。砦主高延德、鹽押王繼元皆沒於賊。振坐擁兵不救,為都轉運使龐籍所奏,貶白州團練使、知絳州。未行,會延德、繼元家復訴於朝,敕御史方偕就劾振。法當斬,再貶太子左清道率府率、潭州安置。逾年,復右武衛將軍、惠州團練使、並代路兵馬鈐轄,就遷副總管、祁州團練使。



 元昊既破豐州,將襲近砦,振率鈐轄張亢、麥允言出麟州深柏堰,擊破之。兼領嵐、憲六州軍事。河外饑,振設法通砦外商,得米數十萬斛,軍
 民以濟。進博州防禦使,改解州致仕。復起為左神武軍大將軍,卒。



 振剛強自負,有武力,便弓馬,喜謀畫,輕財尚氣,眾樂為用。子珣、瑜,皆工騎射。



 珣年十六,仁宗召試便殿,授三班借職。景祐中,有言珣藝益進,且習書史。復召見閱武伎,又試策略於中書,條對數千言。自殿直進閣門祗候,未幾,除濠州兵馬都監。



 初,珣隨父在西邊,訪得五路徼外形勝利害,作《聚米圖經》五卷。詔取其書,並召珣至,又上《五陣圖》、《兵事》十餘篇。帝給步騎使按陣,既成,
 臨觀之。陳執中招討陜西,薦為緣邊巡檢使。呂夷簡、宋庠為奏曰:「用兵以來,策士之言以萬計,無如珣者。」即擢通事舍人、招討都監。珣自以年少新進,辭都監。授兵萬人,御賜鎧仗,令自擇偏裨、參佐,居涇原,兼治籠竿城。



 麻氈、黨留百餘帳處近塞為暴,珣白府,引兵二萬,自靜邊歷揆吳抵木寧襲賊,俘獲數千計。靜邊將劉滬殿後,為賊所掩。珣登阪望見,從騎數百復入,拔滬之眾以出,土皆嘆服。瞎氈居龕谷無所屬,珣與書招之,遺以綈綿,瞎氈
 聽命。



 改本路都監,詔追入朝。將行,適元昊大入,府檄留珣,會葛懷敏於瓦亭。懷敏已屯五穀口西至馬欄城,聞夏人徙軍新壕外,議欲質明掩襲。珣謂懷敏曰:「敵遠來,眾倍鋒銳,莫若依馬欄城布柵以扼其路,守鎮戎城以便餉道,俟其衰擊之,此必勝之道也。不然,必為賊所屠。」懷敏不聽,兵遂逼鎮戎城,越界壕,抵定川。未及陣,夏人引鐵騎來犯,珣居陣西北,瑜亦在軍中,戰甚力。東壁兵輒潰,中軍大擾,珣擁刀斧手前鬥,夏眾稍卻,我軍復
 陣。懷敏詰朝退走,就食鎮戎。俄夏騎四合,珣被擒,瑜以身免。



 珣美風儀,性勁特好學,恂恂類儒者。既沒,人多惜之。贈莫州刺史,後卒賊中。瑜弟璞,亦知名。



 張忠,字聖毗,開封人。先世業農,忠慷慨不事生產。初隸禁軍,累遷龍、神衛左第二軍指揮使。仁宗即位,遷天武左第三指揮使、融州刺史,改天武右廂指揮使、潮州團練使。未幾,真拜齊州團練使,擢知滄州、本路鈐轄。楊懷敏以忠御下急,因奏對言之,徙澶州總管。會河決商胡,
 詔留戍滿卒以助堤役,輒群噪,將劫庫兵為亂。州將恐,召忠議。忠潛捕倡前者數人,斬以徇。明年,以疾求醫京師,卒。



 範恪,字許國,開封人。初名全,少隸軍籍於許州,選入捧日軍,又選為殿前指揮使,歷行門、龍旗直、散員押班。康定元年,元昊數寇邊。試武伎,擢內殿崇班、慶州北路都巡檢使,與攻白豹城,破之。既還,夏人遣騎襲其後。恪設伏崖險,敵半度,邀擊之,斬首四百級,生獲七十餘人。以
 功遷內殿承制。



 嘗會諸道兵攻十二盤暨咄當、迷子砦,中流矢,督戰愈力。視炮石中有火爨者,恪取號於眾曰:「賊矢石盡,用灶下甓矣。」於是士卒爭奮,果先得城。遷供備庫副使。



 恪有弓勝一石七斗,其箭鏃如鏵,名曰鏵弓。又於羽間識其官稱、姓氏,凡所發必中,至一箭貫二人。他日,取蕉蒿砦歸,恪獨殿後,為數千騎所襲。屬視矢箙止有二鏵,即為引滿之勢,賊遽卻。嘗與總管杜惟序、鈐轄高繼隆將兵分討漢乞、薛馬、都嵬等三砦,恪先破都
 嵬,而繼隆圍薛馬不能下,恪馳往取之,既又援惟序下漢乞砦。改左騏驥副使。



 虜犯大順城,諸將皆閉城自守。恪率兵二千餘,戰克之。改宮苑副使、環慶路兵馬都監,因特召見。仁宗謂曰:「適有邊奏,賊犯高平軍劉璠堡,可乘驛亟往。」遂遷禮賓使、榮州刺史、環慶路鈐轄,手詔令趣範仲淹麾下起兵赴援。恪晝夜兼行,比至平涼,賊已解。頃之,遷洛苑使,權秦鳳路兵馬總管。



 恪驍勇善射,臨難敢前,故數有戰功,自龍、神衛四廂都指揮使累遷至
 侍衛親軍馬步軍副都指揮使,歷坊州刺史、解州防禦、宣州觀察使、保信軍節度觀察留後,以疾出為永興軍路副都總管,數月卒,贈昭化軍節度使。



 馬懷德,字得之,開封祥符人。父玉,東頭供奉官,言懷德可試引弓、擊劍、角抵,補三班奉職,為延州南安砦主、東路巡檢。數以少擊西賊,敗其眾。範仲淹知延州,修青澗城,奏懷德為兵馬監押,以所部兵入賊境,破遮鹿、要冊二砦,親射殺其酋狗兒廂主,遷左班殿直。又率蕃漢燒
 蕩賊海溝、茶山、龍柏、安化十七砦三百餘帳,斬首數百級,虜馬駝牛羊萬數,遷右侍禁。



 以範仲淹、韓琦薦,授閣門祗候,延州龐籍入奏為東路都巡檢使。夷黑神、厥保等十八砦,賊以四萬騎犯邊,趨僕射谷。懷德以兵數千據穀旁高原待之,斬首二百級,得畜產、器械以千數。遷內殿崇班。又以兵修龍安城,虜不敢犯,遂為鄜延路都監。又城綏平,破賊青化、押班、吃當三砦,殺獲甚眾。



 元昊為夏國主,命國子博士高良夫與懷德會西人畫界。龐
 籍具論其前後功,遷供備庫副使兼閣門通事舍人。時用兵久,民多亡散,懷德招輯有方,經略使梁適奏請推其法諸路。歷知保安軍、環州、環慶益利路鈐轄,累遷至四方館使、舒州團練使,徙鄜延路副都總管。



 坐違法賂宦官閻士良,為安撫呂景初奏,降四方館使、英州刺史。大名府路總管,侍衛親軍步軍都虞候、象州防禦使、鄜延路副都總管,遷馬軍都虞候,徙環慶路。環州蕃官蘇恩以其屬叛,往降之。又遷殿前都虞候、步軍副指揮使、
 隨州觀察使。



 英宗即位,遷靜難軍節度觀察留後,召還,卒,贈安遠軍節度使。嘗因戰,流矢中其顙,鏃入於骨,以弩弦系鏃,發機而出之。



 安俊,字智周,其先太原人。祖贇,高州團練使。仁宗為皇太子,俊以將家子謹厚,選為資善堂祗候。及即位,補右班殿直,累遷東頭供奉官、閣門祗候,為環州都監。破趙元昊吃□江、井那等諸砦,安撫使韓琦上其功,遷內殿崇班、環慶路都監,徙涇原。契丹欲渝盟,與狄青、範恪同召
 至京師,將使備北邊,擢內園副使。翌日,改禮賓使。



 會葛懷敏敗,命為秦鳳路鈐轄,復徙涇原。因條上御戎十三事,改原州,徙麟州,遷六宅使、貴州刺史、知忻州,徙代州。為帥臣誣奏,降京東路鈐轄。富弼知青州,為之辨理,真除虢州刺史,徙高陽關路,又遷原州刺史,知滄、涇、冀三州。秦州築古渭城,蕃部大擾,徙秦鳳路總管。歷龍神衛、捧日、天武四廂都指揮使,果州團練使,環慶路副總管;遷侍衛步軍都虞候、陵州防禦使。卒,贈閬州觀察使。



 俊
 久在邊,羌人識之。環州得俘虜,知州種世衡問之曰:「若屬於吾將孰畏?」曰:「畏安大保。」指俊於坐曰:「此長髯將軍是也。」



 向寶,鎮戎軍人,為御前忠佐,換禮賓使,涇原、秦鳳鈐轄。積勞,自皇城使帶御器械,歷真定、鄜延副總管,遷龍神衛四廂都指揮使、嘉州團練使,卒。



 寶善騎射,年十四,與敵戰,斬首二級。及壯,以勇聞。有虎踞五原卑邪州,東西百里斷人跡,寶一矢殪之。道過潼關,巨盜郭邈山多載
 關中金帛、子女,寶射走之,盡得其所掠。嘗至太原,梁適射弩再中的,授寶矢射之,四發三中。適曰:「今之飛將也。」神宗稱其勇,以比薛仁貴。及死,厚恤其家。



 論曰:蔚昭敏、高化、周美,蓋皆有功於邊鄙者。化在蜀州,取軍中積材以塞水患,又能平反冤獄,脫人於死,蓋武人之知民事者。美敗夏人,焚族部,城堡砦,未嘗擇便利,而所向輒勝;所得祿賜,悉分與麾下,士亦樂為之用,推古良將,何以加此。閻守恭慕郭進為人,而慷慨自效,起
 徒步至刺史,其志亦豈小哉。孟元、劉謙、馬懷德、範恪皆經略西鄙,數戰有功。其初起自卒伍,而能練習民事、招輯散亡,不獨一武夫而已。趙振挽強命中,精曉兵機。塞門之敗,振擁兵不救,何獨暗於此邪?」子珣年少習書史,閱武技,用兵以來,人以為無如珣者。籠竿一戰,西人奔走不暇,從容而拔劉滬於死,英風義烈,何可少哉!葛懷敏以不用珣計而取敗,珣亦力戰而沒,惜哉!安俊、向寶無多戰功,夏人皆識其名而畏之。張忠區區,較之諸人,
 未可同日語也。



\end{pinyinscope}