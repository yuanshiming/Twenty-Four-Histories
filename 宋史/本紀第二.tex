\article{本紀第二}

\begin{pinyinscope}

 太祖二



 三
 年春正月癸酉朔,以出師,不御殿。甲戌,王全斌克劍門,斬首萬餘級,禽蜀樞密使王昭遠、澤州節度趙崇韜。乙亥,詔瘞征蜀戰死士卒,被傷者給繒帛。壬午,全斌取
 利州。乙酉,蜀主孟昶降。得州四十五、縣一百九十八、戶五十三萬四千三十有九。高麗國王遣使來朝獻。戊子,吏部郎中鄧守中坐試吏不當,責本曹員外郎。癸巳,劉光義取萬、施、開、忠四州,遂州守臣陳愈降。乙未,詔撫西川將吏百姓。丙申,赦蜀,歸俘獲,除管內逋賦,免夏稅及沿征物色之半。二月癸卯,南唐、吳越進長春節御衣、金銀器、錦綺以千計。甲辰,遣皇城使竇思儼迎勞孟昶。丁未,全州大水。庚申,王全斌殺蜀降兵二萬七千人於成
 都。三月癸酉,詔置義倉。



 是月,兩川賊群起,先鋒都指揮使高彥暉死之,詔所在攻討。夏四月乙巳,回鶻遣使獻方物。癸丑,職方員外郎李嶽坐贓棄市。南唐進賀收蜀銀絹以萬計。戊午,遣中使給蜀臣鞍馬、車乘於江陵。癸亥,募諸軍子弟導五丈河,通皇城為池。五月辛未朔,詔還諸道幕職、令錄經引對者,以涉途遠近,差減其選。壬申,幸迎春苑宴射。乙亥,遣開封尹光義勞孟昶於玉津園。丙戌,見孟昶於崇元殿,宴昶等於大明殿。丁亥,賜將
 士衣服錢帛。戊子,大赦,減死罪一等。壬辰,宴孟昶及其子弟於大明殿。六月甲辰,以孟昶為中書令、秦國公,昶子弟諸臣錫爵有差。庚戌,孟昶薨。秋七月,珍州刺史田景遷內附。壬辰,追封孟昶為楚王。丁酉,幸教船池,遂幸玉津園宴射。八月戊戌朔,詔籍郡國驍勇兵送闕下。癸卯,河決陽武縣。庚戌,詔王全斌等廩蜀亡命兵士家。乙卯,河溢河陽,壞民居。戊午,殿直成德鈞坐贓棄市。己未,鄆州河水溢,沒田。辛酉,壽星見。九月己巳,閱諸道兵,以
 騎軍為驍雄,步軍為雄武,並隸親軍。壬申,詔蜀諸郡各置克寧軍五百人。辛巳,河決澶州。戊子,幸西水磑。十月丁酉朔,大霧。己未,太子中舍王治坐受贓殺人,棄市。丙寅,濟水溢鄒平。十一月丙子,甘州回鶻可汗遣僧獻佛牙、寶器。乙未,劍州刺史張仁謙坐殺降,貶宋州教練。十二月丁酉朔,詔婦為舅姑喪者齊、斬。己亥,詔西川管內監軍、巡檢毋預州縣事。戊午,甘州回鶻可汗、于闐國王等遣使來朝,進馬千匹、橐駝五百頭、玉五百圍、琥珀五
 百斤。



 四年春正月丙子,遣使分詣江陵、鳳翔,賜蜀群臣家錢帛。丁亥,命丁德裕等率兵巡撫西川。己丑,幸迎春苑宴射。二月癸卯,視皇城役。丙辰,於闐國王遣其子德從來獻。安國軍節度使羅彥瑰等敗北漢於靜陽,擒其將鹿英。辛酉,試下第舉人。甲子,免西川今年夏稅及諸征之半,田不得耕者盡除之。岳州火。三月癸酉,罷義倉。甲戌,占城國遣使來獻。癸未,僧行勤等一百五十七人各賜
 錢三萬,游西域。夏四月丁酉,占城遣使來獻。丙午,潭州火。壬子,罷光州貢鷹鷂。丁巳,契丹天德軍節度使於延超與其子來降。進士李藹坐毀釋氏,辭不遜,黥杖,配沙門島。庚申,幸燕國長公主第視疾。五月,南唐賀文明殿成,進銀萬兩。甲戌,光祿少卿郭□巳坐贓棄市。乙亥,閱蜀法物、圖書。丁丑,詔蜀郡敢有不省父母疾者罪之。辛巳,潭州火。壬午,澶州進麥兩歧至六歧者百六十五本。辛卯,熒惑犯軒轅。六月甲午,東阿河溢。甲辰,河決觀城。月
 犯心前星。丙午,澧州刺史白全紹坐縱紀綱規財部內免官。詔人臣家不得私養宦者,內侍年三十以上方許養一子,士庶敢有閹童男者不赦。己酉,果州貢禾,一莖十三穗。秋七月丙寅,詔蜀官將吏及姻屬疾者,所在給醫藥、錢帛。戊辰,西南夷首領董暠等內附。己巳,幸造船務,又幸開封尹北園宴射。癸酉,賜西川行營將士錢帛有差。庚辰,罷劍南蜀米麥征。華州旱,免今年租。給州縣官奉戶。八月丁酉,詔除蜀倍息。庚子,水壞高苑縣城。壬
 寅,詔憲臣及吏、刑部官三周歲滿日,即轉授加恩。庚戌,樞密直學士馮瓚、綾錦副使李美、殿中侍御史李楫為宰相趙普陷,以贓論死,會赦,流沙門島,逢恩不還。辛亥,幸玉津園宴射。京兆府貢野蠶繭。壬子,衡州火。乙卯,錄囚。丙辰,河決滑州,壞靈河大堤。普州兔食稼。閏月乙丑,河溢入南華縣。己巳,衡州火。乙亥,詔:民能樹藝、開墾者不加征,令、佐能勸來者受賞。九月壬辰朔,水。虎捷指揮使孫進、龍衛指揮使吳瑰等二十七人,坐黨呂翰亂伏
 誅,夷進族。庚子,占城獻馴象。乙巳,幸教船池,遂幸玉津園觀衛士騎射。丙午,詔吳越立禹廟於會稽。冬十月辛酉朔,命太常復二舞。癸亥,詔諸郡立古帝王陵廟,置戶有差。己巳,禁吏卒以巡察擾民。十二月庚辰,妖人張龍兒等二十四人伏誅,夷龍兒、李土、楊密、聶贇族



 五年春正月戊戌,治河堤。丁未,合州漢初縣上青樛木中有文曰「大連宋」。甲寅,王全斌等坐伐蜀黷貨殺降,全斌責崇義軍節度使,崔彥進責昭化軍節度使,王仁贍
 責右衛大將軍。丙辰,詔伐蜀將校有受蜀人錢物者,並即還主。丁巳,賞伐蜀功,曹彬、劉光義等進爵有差。二月庚申朔,幸造船務,遂幸城西觀衛士騎射。甲子,薛居正、呂餘慶並為吏部侍郎、依前參知政事。己丑,幸教船池。三月甲辰,詔翰林學士、常參官於幕職、州縣及京官內各舉堪任常參官者一人,不當者連坐。乙巳,詔諸道舉部內官吏才德優異者。丙午,以普為尚書左僕射兼門下侍郎、同中書門下平章事,崇矩檢校太傅。是日,幸教
 船池,又幸玉津園宴射。丙辰,北漢石盆砦招收指揮使閻章以砦來降。五星聚奎。夏五月乙巳,賜京城貧民衣。北漢鴻唐砦招收指揮使樊暉以砦來降。甲寅,王溥為太子太傅。六月戊午朔,日有食之。辛巳,幸建隆觀,遂幸飛龍院。丁亥,牂牁順化王子等來獻方物。七月丁酉,禁毀銅佛像。己酉,免水旱災戶今年租。八月甲申,河溢入衛州城,民溺死者數百。



 九月壬辰,倉部員外郎陳郾坐贓棄市。甲午,西南蕃順化王子部才等遣使獻方物。己
 酉,畋近郊。十一月乙酉朔,工部侍郎毋守素坐居喪娶妾免。供奉武仁海坐枉殺人棄市。



 十二月丙辰,禁新小鐵鑞等錢、疏惡布帛入粉藥者。癸酉,升麟州為建寧軍節度。趙普以母憂去位,丙子,起復。



 開寶元年春正月甲午,增治京城。陜之集津、絳之垣曲、懷之武陟饑,振之。己亥,北漢偏城砦招收指揮使任恩等來降。三月庚寅,班縣令、尉捕盜令。癸巳,幸玉津園。乙巳,有馴象自至京師。夏四月乙卯,幸節度使趙彥徽第
 視疾。五月丁未,賜南唐米麥十萬斛。六月癸丑朔,詔民田為霖雨、河水壞者,免今年夏稅及沿征物。癸亥,詔荊蜀民祖父母、父母在者,子孫不得別財異居。丁丑,太白晝見。戊寅,復見。辛巳,龍出單父民家井中,大風雨,漂民舍四百區,死者數十人。秋七月丙申,幸鐵騎營,賜軍錢羊酒有差。北漢穎州砦主胡遇等來降。丙午,幸鐵騎營,遂幸玉津園。戊申,坊州刺史李懷節坐強市部民物,責左衛率府率。北漢主劉鈞卒,養子繼恩立。八月乙卯,按
 鶻於近郊,還,幸相國寺。戊午,又按鶻於北郊,還,幸飛龍院。丙寅,遣客省使盧懷忠等二十二人率禁軍會潞州。戊辰,命昭義軍節度使李繼勛等征北漢。九月辛巳朔,禁錢出塞。癸未,監察御史楊士達坐鞠獄濫殺棄市。庚子,李繼勛敗北漢於銅溫河。己酉,北漢供奉官侯霸榮弒其主繼恩,繼元立。冬十月己未,畋近郊,還,幸飛龍院。丙子,吳越王遣其子惟浚來朝貢。十一月癸卯,日南至,有事南郊,改元開寶。大赦,十惡、殺人、官吏受贓者不原。
 宰相普等奉玉冊寶,上尊號曰應天廣運大聖神武明道至德仁孝皇帝。十二月甲子,行慶,自開封興元尹、宰相、樞密使及諸道蕃侯,並加勛爵有差。乙丑,大食國遣使獻方物。



 二年春正月己卯朔,以出師,不御殿。二月乙卯,命昭義軍節度使李繼勛為河東行營前軍都部署,侍衛步軍指揮使黨進副之,宣徽南院使曹彬為都監,棣州防禦使何繼筠為石嶺關部署,建雄軍節度使趙贊為汾州
 路部署,以伐北漢。宴長春殿。命彰德軍節度使韓仲贇為北面都部署,彰義軍節度使郭延義副之,以防契丹。戊午,詔親征。己酉,以開封尹光義為上都留守,樞密副使沉義倫為大內部署、判留司三司事。甲子,發京師。乙亥,雨,駐潞州。三月壬辰,發潞州。乙未,李繼勛敗北漢軍於太原城下。戊戌,駕傅城下。庚子,觀兵城南,築長連城。辛丑,幸汾河,作新橋。發太原諸縣丁數萬集城下。癸卯,北漢史昭文以憲州來降,乙巳,臨城南,謂汾水可以灌
 其城,命築長堤壅之,決晉祠水注之。遂砦城四面,繼勛軍於南,贊軍於西,彬軍於北,進軍於東,乃北引汾水灌城。辛亥,遣海州刺史孫方進率兵圍汾州。四月戊申,幸城東觀築堤。壬子,復幸城東。己未,何繼筠敗契丹於陽曲,斬首數千級,俘武州刺史王彥符以獻,命陳示所獲首級、鎧甲於城下。壬戌,幸汾河觀造船。戊辰,幸城西上生院。丙子,復幸城西。五月癸未,韓仲贇敗契丹於定州北。自戊子至庚寅,命水軍載弩環攻,橫州團練使王廷
 義、殿前都虞候石漢卿死之。甲午,北漢趙文度以嵐州來降。甲辰,都虞候趙廷翰奏,諸軍欲登城以死攻,上愍之,不允。閏月戊申,雉圮,水注城中,上遽登堤觀。己酉,右僕射魏仁浦薨。壬子,以太常博士李光贊言,議班師。己未,命兵士遷河東民萬戶於山東。庚申,分命使臣率兵赴鎮、潞。壬戌,駕還。戊辰,駐蹕於鎮州。六月丙子朔,發鎮州。癸巳,至自太原。曲赦京城囚。秋七月丁巳,幸封禪寺。詔鎮、深、趙、邢、洺五州管內鎮、砦、縣悉城之。甲子,大宴,賜
 宰相、樞密使、翰林學士、節度、觀察使襲衣金帶。戊辰,西南夷順化王子武才等來獻方物。癸酉,幸新水磑。汴決下邑。乙亥,壽星見。八月丁亥,詔川峽諸州察民有父母在而別籍異財者,論死。九月乙巳朔,幸武成王廟。壬戌,幸玉津園宴射。冬十月戊子,畋近郊。庚寅,散指揮都知杜延進等謀反伏誅,夷其族。詔相、深、趙三州丁夫死太原城下者,復其家。庚子,以王溥為太子太師,武衡德為太子太傅。癸卯,西川兵馬都監張延通、內臣張嶼、引進
 副使王玨為丁德裕所譖,延通坐不遜誅,嶼、玨並杖配。十一月丙午,幸鎮寧軍節度使張令鐸第視疾。甲寅,畋近郊,還,幸金鳳園。庚申,回鶻、于闐遣使來獻方物。十二月癸未,幸中書視宰相趙普疾。己亥,右贊善大夫王昭坐監大盈倉,其子與倉吏為奸贓,奪兩任、配隸汝州。丁德裕誣奏西川轉運使李鉉指斥,事既直,猶坐酒失,責授右贊善大夫。



 三年春正月癸卯朔,雨雪,不御殿。癸丑,增河堤。辛酉,詔
 民五千戶舉孝弟彰聞、德行純茂者一人,奇才異行不拘此限,裏閭郡國遞審連署以聞,仍為治裝詣闕。二月庚寅,幸西茶庫,遂幸建隆觀。三月庚戌,詔閱進士十五舉以上司馬浦等百六人,並賜本科出身。辛亥,賜處士王昭素國子博士致仕。丙辰,殿中丞張顒坐先知穎州政不平,免官。己未,幸宰相趙普第視疾。夏四月辛未朔,日有食之。丁亥,幸寺觀禱雨。辛卯,雨。甲午,幸教船池。己亥,罷河北諸州鹽禁。詔郡國非其土產者勿貢。五月丁
 未,禁京城民畜兵器。癸丑,幸城北觀水磑。癸亥,賜諸班營舍為雨壞者錢有差。六月乙未,禁諸州長吏親隨人掌廂鎮局務。秋七月乙巳,立報水旱期式。壬子,詔蜀州縣官以戶口差第省員加祿,尋詔諸路亦如之。戊辰,幸教船池,又幸玉津園宴射。八月戊子,幸教船池,又幸玉津園。九月己亥朔,命潭州防禦使潘美為貴州道兵馬行營都部署,朗州團練使尹崇珂副之。遣使發十州兵會賀州,以伐南漢。甲辰,詔:西京、鳳翔、雄、耀等州,周文、成、
 康三王,秦始皇,漢高、文、景、武、元、成、哀七帝,後魏孝文,西魏文帝,後周太祖,唐高祖、太宗、中宗、肅宗、代宗、德、順、文、武、宣、懿、僖、昭諸帝凡二十七陵嘗被盜發者,有司備法服、常服各一襲,具棺槨阜重葬,所在長吏致祭。己酉,幸開寶寺觀新鐘。丙辰,女直國遣使繼定安國王烈萬華表,獻方物。丁卯,潘美等敗南漢軍萬眾於富州,下之。十月庚辰,克賀州。十一月壬寅,下昭、桂二州。乙巳,減桂陽歲貢白金額。癸丑,右領軍衛將軍石延祚坐監倉與吏為
 奸贓,棄市。癸亥,定州駐泊都監田欽祚敗契丹於遂城。丙寅,以曹州舉德行孔蟾為章丘主簿。十二月壬申,潘美等下連州。辛卯,大敗南漢軍萬餘於韶州,下之。癸巳,增河堤。



 四年春正月戊戌朔,以出師,不視朝。丙午,罷諸道州縣攝官。丁未,右千牛衛大將軍桑進興坐贓棄市。癸丑,潘美等取英州、雄州。二月丁亥,南漢劉金長遣其左僕射蕭漼等以表來上。己丑,潘美克廣州,俘劉鋹,廣南平。得州
 六十、縣二百十四、戶十七萬二百六十三。辛卯,大赦廣南,免二稅,偽署官仍舊。三月乙未,幸飛龍院,賜從臣馬。丙申,詔廣南有賣人男女為奴婢轉傭利者,並放免。偽政有害於民者具以聞,除之。增前代帝王守陵戶二。夏四月丙寅朔,前左監門衛將軍趙玭訴宰相趙普,坐誣毀大臣,汝州安置。丁卯,三佛齊國遣使獻方物。己巳,詔禁嶺南商稅、鹽、曲,如荊湖法。辛未,幸永興軍節度使吳廷祚第視疾。癸未,幸開寶寺。辛卯,南唐遣其弟從諫來
 朝貢。發廂軍千人修前代陵寢之在秦者。壬辰,監察御史閭丘舜卿坐前任盜用官錢,棄市。五月乙未朔,御明德門受劉鋹俘,釋之;斬其柄臣龔澄樞、李托、薛崇譽。大宴於大明殿,鋹預焉。丁酉,賞伐廣南功,潘美、尹崇珂等進爵有差。六月癸酉,遣使祀南海。丁丑,命翰林試南漢官,取書判稍優者,授令、錄、簿、尉。壬午,以孝子羅居通為延州主簿。封劉鋹為恩赦侯。乙酉,罷賀州銀場。賜劉鋹月奉外錢五萬、米麥五千斛。河決原武,汴決穀熟。秋七
 月戊戌,賜開封尹光義門戟十四。庚子,幸新修水磑,賜役人錢帛有差。戊午,復著內侍養子令。癸亥,幸建武軍節度使何繼筠第視疾。汴決宋城。八月壬申,文武百官上尊號,不允。辛卯,景星見。冬十月癸亥朔,日有食之。己巳,詔偽作黃金者棄市。庚午,太子洗馬王元吉坐贓棄市。辛巳,除廣南舊無名配斂。甲申,詔十月後犯強竊盜者,郊赦不原。丙戌,放廣南民驅充軍者。十一月癸巳朔,南唐遣其弟從善,吳越國王遣其子惟浚,以郊祀來朝
 貢。南唐主煜表乞去國號呼名,從之。庚戌,詔諸道所罷攝官三任無遺闕者以聞。河決澶州,通判姚恕坐不即上聞,棄市。己未,日南至,有事南郊,大赦,十惡、故劫殺、官吏受贓者不原。詔置諸州幕職官奉戶。壬戌,蜀班內殿直四十人,援御馬直例乞賞,遂撾登聞鼓,命各杖二十,翌日,悉斬於營,都指揮單斌等皆杖、降。十二月癸亥朔,賜南郊執事官器幣有差。丁卯,行慶,開封尹光義、興元尹光美、貴州防禦使德昭、宰相趙普並益食邑。己巳,內
 外文武官遞進勛爵。辛未,賜九經李符本科出身。壬午,畋近郊。



\end{pinyinscope}