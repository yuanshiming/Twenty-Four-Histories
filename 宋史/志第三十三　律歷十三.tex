\article{志第三十三 律歷十三}

\begin{pinyinscope}

 紀元歷



 步
 交
 會



 交終分:一十九萬八千三百七十七、秒八百八十。



 交終日:二十七、餘一千五百四十七、秒八百八十。



 交中日:一十三、餘四千四百一十八、秒五千四百四十。



 朔差日:二、餘二千三百二十、秒九千一百二十。



 望策:一十四、餘五千五百七十九。



 已上秒母一萬。



 交率:三百二十四。



 交數:四千一百二十七。



 交終度:三百六十三、約分七十九、秒四十四。



 交中度:一百八十一、約分八十九、秒七十二。



 交象度:九十、約分九十四、秒八十六。



 半交象度:四十五、約分四十七、秒四十三。



 日食陽歷限:三千四百,定法三百四十。



 陰歷限:四千三百,定法四百三十。



 月食限:六千八百,定法四百四十。



 已上分秒母各同一百。



 推天正十一月經朔加時入交:置天正十一月經朔加時積分,以交終分及秒去之,不盡,滿日法為日,不滿為
 餘秒,即天正十一月經朔加時入交泛日及餘秒。



 求次朔及望入交:置天正十一月經朔加時入交泛日及餘秒,求次朔,以朔差加之;求望,以望策加之:滿交終日及餘秒去之,即各得次朔及望加時入交泛日及餘秒。若以經朔、望小餘減之,各得朔、望夜半入交泛日及餘秒。



 求定朔望夜半入交:因經朔、望夜半入交泛日及餘秒,視定朔、望日辰有進退者,亦進退交日,否則因經為定,各得所求。



 求次定朔夜半入交:各因定朔夜半入交泛日及餘秒,大月加二日,小月加一日,餘皆加五千七百四十二、秒九千一百二十,即次朔夜半入交;若求次日,累加一日:滿交終日及餘秒皆去之,即每日夜半入交泛日及餘秒。



 求定朔望加時入交:置經朔、望加時入交泛日及餘秒,以入氣、入轉朏朒定數朏減朒加之,即得定朔、望加時入交泛日及餘秒。



 求定朔望加時月行入交積度:置定朔、望加時入交泛日及餘秒,以日法通日,內餘,進一位,如五千四百五十三而一為度,不滿,退除為分,即定朔、望加時月行入交積度及分。每日夜半,準此求之。



 求定朔望加時月行入交定積度:置定朔、望加時月行入交積度及分,以定朔、望加時入轉遲疾度遲減疾加之,滿與不足,進退交終度及分。



 即定朔、望加時月行入交定積度及分。每日夜半,準此求之。



 求定朔望加時月行入陰陽歷積度:置定朔、望加時月行入交定積度及分,如在交中度及分已下為入陽歷積度;已上者去之,餘為入陰歷積度。每日夜半,準此求之。



 求定朔望加時月去黃道度:視月入陰陽歷積度及分,如交像已下為在少象;已上,覆減交中度,餘為入老象。置所入老、少象度及分於上,列交象度於下,以上減下,餘以乘上,五百而一,所得,用減所入老、少象度及分,餘,列交中度於下,以上減下,餘以乘上,滿一千三百七十
 五而一,所得為度,不滿,退除為分,即為定朔、望加時月去黃道度及分。每日夜半,準此求之。



 求朔望加時入交常日:置其月經朔、望加時入交泛日及餘秒,以其月入氣朏朒定數朏減朒加之,滿與不足,進退其日,即得朔、望加時入交常日及餘秒。近交初為交初,在二十六日、二十七日為交初;近交中為交中,在十三日、十四日為交中。



 求日月食甚定數:以其朔望入氣、入轉朏朒定數,同名相從,異名相消,副置之;以定朔、望加時入轉算外損益
 率乘之,如日法而一,其定朔、望如算外在四七日者,視其餘在初數已下,初率乘之,初數而一;初數已上,以末率乘之,末數而一。



 所得,視入轉,應朒者依其損益,應朏者益減損加其副;以朏減朒加經朔望小餘,為泛餘。滿與不足,進退大餘。



 日食者視泛餘,如半法已下,為中前;列半法於下,以上減下,餘以乘上,如一萬九百三十五而一,所得,為差;以減泛餘,為食甚定餘;用減半法,為午前分。如泛餘在半法已上,減去半法,為中後;列半法於下。以上減下,餘以乘上,如日法而一,所得,為差;以加泛餘,為食
 甚定餘;乃減去半法,為午後分。月食者視泛餘,如半法已上減去半法,餘在一千八百二十二半已下自相乘,已上者,覆減半法,餘亦自相乘,如三萬而一,所得,以減泛餘,為食甚定餘;如泛餘不滿半法,在日出分三分之二已下,列於上位,已上者,用減日出分,餘倍之,亦列於上位,乃四因三約日出分,列之於下,以上減下,餘以乘上,如一萬五千而一,所得,以加泛餘,為食甚定餘。



 求日月食甚辰刻:倍食甚定餘,以辰法除之為辰數,不
 盡,五因之,滿刻法除之為刻,不滿為分。命辰數起子正,算外,即食甚辰刻及分。若加半辰,命起子初。



 求日月食甚入氣:食甚大、小餘及食定小餘,並定朔、望大餘,以此與經朔望大、小餘相減。



 置其朔望食甚大、小餘,與經朔望大、小餘相減之,餘以加減經朔望入氣日餘,經朔望少即加之,多即減之。



 為日、月食甚入氣日及餘秒。各置食甚入氣及餘秒,加其氣中積,其餘,以日法退除為分,即為日、月食甚中積及分。



 求日月食甚日行積度:置食甚入氣餘,以所入氣日盈
 縮分乘之,日法而一,加減其日先後數,至後加,分後減。



 先加後減日、月食甚中積,即為日、月食甚日行積度及分。



 求氣差:置日食甚日行積度及分,滿二至限去之,餘在象限已下為在初;已上,覆減二至限,餘為在末。皆自相乘,進二位,滿三百四十三而一,所得,用減二千四百三十,餘為氣差;以午前、後分乘之。如半晝分而一,以減氣差,為氣差定數。在冬至後末限、夏至後初限,交初以減,交中以加。



 夏至後末限、冬至後初限,交初以加,交中以減。



 如半晝分而一,所
 得,在氣差已上者,即以氣差覆減之,餘應加者為減,減者為加。



 求刻差:置日食甚日行積度及分,滿二至限去之,餘列二至限於下,以上減下,餘以乘上,進二位,滿三百四十三而一,所得為刻差。以午前、後分乘而倍之,如半法而一,為刻差定數。冬至後食甚在午前,夏至後食甚在午後,交初以加,交中以減。



 冬至後食甚在午後,夏至後食甚在午前,交初以減,交中以加。



 如半法而一,所得在刻差已上者,即倍刻差,
 以所得之數減之,餘為刻差定數,依其加減。



 求朔入交定日:置朔入交常日及餘秒,以氣、刻差定數各加減之,交初加三千一百,交中減三千,為朔入交定日及餘秒。



 求望入交定日:置望入轉朏朒定數,以交率乘之,如交數而一,所得,以朏減朒加入交常日之餘,滿與不足,進退其日,即望入交定日及餘秒。



 求月行入陰陽歷:視其朔、望入交定日及餘秒,如在中
 日及餘秒已下為月在陽歷;如中日及餘秒已上,減去中日,為月在陰歷。



 求入食限交前後分:視其朔、望月行入陰陽歷,不滿日者為交後分;在十三日上下者覆減交中日,為交前分;視交前、後分各在食限已下者為入食限。



 求日食分:以交前、後分各減陰陽歷食限,餘如定法而一,為日食之大分;不盡,退除為小分。命大分以十為限,即得日食之分。其食不及大分者,行勢稍近交道,光氣微有映蔽,其日或食或不食。



 求月食分:視其望交前、後分,如二千四百已下者,食既;已上,用減食限,餘如定法而一,為月食之大分;不盡,退除為小分。命大分以十為限,得月食之分。



 求日食泛用分:置交前、後分,自相乘,退二位,陽歷一百九十八而一,陰歷三百一十七而一,所得,用減五百八十三,餘為日食泛用分。



 求月食泛用分:置交前、後分,自相乘,退二位,如七百四而一,所得,用減六百五十六,餘為月食泛用分。



 求日月食定用分:置日、月食泛用分,副之,以食甚加時入轉算外損益率乘之,如日法而一,如算外在四、七日者,依食定餘求之。



 所得,應朒者依其損益,應朏者益減損加其副,即為日月食定用分。



 求月食既內外分:置月食交前、後分,自相乘,退二位,如二百四十九而一,所得,用減二百三十一,餘以定用分乘之,如泛用分而一,為月食既內分;用減定用分,餘為既外分。



 求日月食虧初復滿小餘:置日、月食甚小餘,各以定用分減之,為虧初;加之,為復滿;其月食既者,以既內分減之,為初既;加之,為生光:即各得所求小餘。如求時刻,依食甚術入之。



 求月食更點法:置月食甚所入日晨分,倍之,減去七百二十九,餘五約之,為更法;又五除之,為點法。



 求月食入更點:置虧初、食甚、復末小餘,在晨分已下加晨分,昏分已上減去昏分,餘以更法除之為更數,不滿,以點法除之為點數。其更數命初更,算外,即各得所入
 更、點。



 求日食所起:日在陽歷,初起西南,甚於正南,復於東南;日在陰歷,初起西北,甚於正北,復於東北。其食八分已上,皆起正西,復於正東。此據午地而論之。



 求月食所起:月在陽歷,初起東北,甚於正北,復於西北;月在陰歷,初起東南,甚於正南,復於西南。其食八分已上,皆起正東,復於正西。此亦據午地而論之。



 求日月出入帶食所見分數:各以食甚小餘與日出、入
 分相減,餘為帶食差;以乘所食之分,滿定用分而一,如月食既者,以既內分減帶食差,餘進一位,如既外分而一,所得,以減既分,即月帶食出入所見之分,不及減者,為帶食既出入。



 以減所食分,即日月出、入帶食所見之分。其食甚在晝,晨為漸進,昏為已退;其食甚在夜,晨為已退,昏為漸進。



 求日月食甚宿次:置食甚日行積度,望即更加半周天。



 以天正冬至加時黃道日度加而命之,即各得日、月食甚宿度及分。



 步五星



 木星周率:二百九十萬七千八百七十九、秒六十四。



 周差:二十四萬五千二百五十三、秒六十四。



 歷率:二百六十六萬二千六百三十六、秒二十二。



 周日:三百九十八、約分八十八、秒六十。



 歷度:三百六十五、約分二十四、秒五十。



 歷中度:一百八十五、約分六十二、秒二十五。



 歷策度:一十五、約分二十一、秒八十五。



 伏見度:一十三。



 木星盈縮歷



 火星周率:五百六十八萬五千六百八十七、秒六十四。



 周差:三十六萬四百一十四、秒四十四。



 歷率:二百六十六萬二千六百四十七、秒二十。



 周日:七百七十九、約分九十二、秒九十七。



 歷度:三百六十五、約分二
 十四、秒六十五。



 歷中度:一百八十二、約分六十二、秒三十二半。



 歷策度:二十五、約分二十一、秒八十六。



 伏見度:一十九。



 火星盈縮歷



 土星周率:二百七十五萬六千二百八十八、秒七十八。



 周差:九萬三千六百六十二、秒七十八。



 歷
 率:二百六十六萬九千九百二十五、秒九十。



 周日:三百七十八、約分九、秒一十七。



 歷度:三百六十六、約分二十四、秒四十九。



 歷中度:一百八十三、約分一十二、秒二十四半。



 歷策度:一十五、約分二十六、秒二。



 伏見度:一十七。



 土星盈縮歷



 金星周率:四百二十五萬六千六百五十一、秒四十三半。



 合日:二百九十一、約分九十五、秒一十四。



 歷率:二百六十六萬二千六百九十六、秒一十六。



 周日:五百八十三、約分九十、秒二十八。



 歷度:三百六十五、約分二十五、秒三十二。



 歷中度:一百八十二、約分六十二、秒六十六。



 歷策度:一十五、約分二十一、秒八十九。



 伏見度:一十半。



 金星盈縮歷



 水星周率:八十四萬四千七百三十八、秒五。



 合日:五十七、約分九十三、秒八十一。



 歷率:二百六十六萬二千七百九十四、秒九十五。



 周日:一百一十五、約分八十七、秒六十二。



 歷度:三百六十五、約分二十六、秒六十八。



 歷中度:一百八十二、約分六十三、秒三十四。



 歷策度:一十五、約分二十一、秒九十四半。



 晨伏夕見:一十四。



 夕伏晨見:一十九。



 水星盈縮歷



 推五星天正冬至後平合及諸段中積中星:置氣積分,各以其星周率除之,所得周數。不盡者,為前合。以減周
 率,餘滿日法為日,不滿,退除為分、秒,即其星天正冬至後平合中積;命之為平合中星,以諸段常日、常度累加之,即諸段中積、中星。其段退行者,以常度減之,即其段中星。



 求木火土三星平合諸段入歷:置其星周數,求冬至後合,皆加一數置之。



 以周差乘之,滿其星歷率去之,不盡,滿日法為度,不滿,退除為分、秒,即為其星平合入歷度及分、秒。以其段限度依次累加之,即得諸段入歷。



 求金水二星平合及諸段入歷:置氣積分,各以其星歷率去之,不盡,滿日法除之為度,不滿,退除為分、秒,以加平合中星,即為其星天正冬至後平合入歷度及分、秒;以其星其段限度依次累加之,即得諸段入歷。



 求五星平合及諸段盈縮定差:各置其星其段入歷度及分,如歷中已下為在盈;已上減去歷中,餘為在縮;以其星歷策除之為策數,不盡,為入策度及分;命策數,算外,以其策損益率乘之,如歷策而一為分,分滿百為度;
 以損益其下盈縮積,即其星其段盈縮定差。



 求五星平合及諸段定積:各置其星其段中積,以其段盈縮定差盈加縮減之,即其段定積日及分;以天正冬至大餘及約分加之,即為定日及分;盈紀法六十去之,不盡,命己卯,算外,即得日辰。



 求五星平合諸段所在月日:各置其段定積,以天正閏日及約分加之,滿朔策及約分除之為月數,不盡,為入月已來日數及分。其月數命天正十一月,算外,即其星
 其段入其月經朔日數及分,乃以日辰相距為定朔月、日。



 求五星平合及諸段加時定星:各置其段中星,以其段盈縮定差盈加縮減之,金星倍之水星三之,乃可加減。



 即五星諸段定星;以天正冬至加時黃道日度加而命之,即其星其段加時所在宿度及分秒。五星皆因前留為前段初日定星,後留為後段初日定星,餘依術算。



 求五星諸段初日晨前夜半定星:各以其段初行率乘
 其段加時分,百約之,乃以順減退加其日加時定星,即為其段初日晨前夜半定星;加命如前,即得所求。



 求諸段日率度率:各以其段日辰距至後段日辰,為其段日率;以其段夜半定星與後段夜半定星相減,為其段度率及分秒。



 求諸段平行度:各置其段度率及分秒,以其段日率除之,為其段平行度及分秒。



 求諸段總差:各以其段平行分與後段平行分相減,餘
 為泛差;並前段泛差,四因,退一位,為總差。若前段無平行分相減為泛差者,因後段初日行分與其段平行分相減,餘為半總差;倍之,為總差。若後段無平行分相減為泛差者,因前段末日行分與其段平行分相減,餘為半總差,倍之,為總差。晨遲末段,視段無平行分,因前初段末日行分與晨遲末段平行分相減,為半總差;其退行者,各置本段平行分,十四乘之,十五而一,為總差。內金星依順段術入之,即得所求。夕遲初段,視前段無平行分,因後末段初日行
 分與夕遲初段平行分相減,為半總差。



 求諸段初末日行分:各半其段總差,加減其段平行分,後段平行分多者,減之為初,加之為末;後段平行分少者,加之為初,減之為末。其在退行者,前減之為初,加之為末;後加之為初,減之為末。



 各為其星其段初、末日行度及分秒。如前後段平行分俱多、俱少者,平注之;本段總差不滿大分者,亦平注之。



 求每日晨前夜半星行宿次:置其段總差,減日率一以除之,為日差;累損益初日行分,後行分少,損之;後行分多,益之。



 為每日行度及分秒;乃順加退減其段初日晨前夜半宿次命
 之,即每日晨前夜半星行所在宿次。



 徑求其日宿次:置所求日,減一,半之,以日差乘而加減初行日分,後行分少,減之;後行分多,加之。



 以所求日乘之,為積度;乃順加退減其段初日宿次,即得所求日宿次。



 求五星平合及見伏入氣:置定積,以氣策及約分除之為氣數,不盡,為入氣已來日數及分秒。其氣數命天正冬至,算外,即五星平合及見、伏入氣日及分秒。其定積滿歲周日及分,去之,餘,在來年冬至後。



 求五星合見伏行差:木、火、土三星,以其段初日星行分減太陽行分,餘為行差。金、水二星順行者,以其段初日太陽行分減星行分,餘為行差。金、水二星退行者,以其段初日星行分並太陽行分,為行差。



 求五星定合及見伏泛積:木、火、土三星,各以平合晨疾、夕伏定積,便為定合定見、定伏泛積。金、水二星,各置其段盈縮定差,內水星倍之,以其段行差除之為日,不滿,退除為分秒,在平合夕疾、晨伏者,乃盈減縮加定積,為
 定合定見、定伏泛積;在退合夕伏、晨見者,用盈加縮減定積,為定合定見、定伏泛積。



 求五星定合定積定星:木、火、土三星,以平合行差除其日先後數,為距合差日;以先後數減之,為距合差度;以差日、差度後加先減其星定合泛積,為其星定合日定積、定星。金、水二星順合者,以平合行差除其日先後數,為距合差日;以先後數加之,為距合差度;以差日、差度先加後減其星定合泛積,為其星定合日定積、定星。金、
 水二星退合者,以退合行差除其日先後數,為距合差日;以減先後數,為距合差度;以差日先減後加,以差度先加後減再定合泛積,為其星再定合積星。各以冬至大餘及約分加定積,滿紀法去之,命己卯,算外,即得定合日辰;以冬至加時黃道日度加定星,依宿次去之,即得定合所在宿次。



 求木火土三星定見伏定積日:各置其星定見、伏泛積,晨加夕減象限日及分秒,如二至限已下自相乘,已上,
 覆減歲周,餘亦自相乘,百約為分,以其星伏見度乘之,十五除之,為差;其差如其段行差而一為日,不滿,退除為分、秒,見加伏減泛積,為定積;如前加命,即得日辰。



 求金水二星定見伏定日:夕見、晨伏,以行差除其日先後數,為日;先加後減泛用積,為常用積。晨見、夕伏,以行差除其日先後數,為日;先減後加泛用積,為常用積。如常用積在二至限已下為冬至後;已上去之,餘為夏至後。其二至後日及分在象限已下自相乘,已上,用減二
 至限,餘亦自相乘,如法而一,所得為分;冬至後晨,夏至後夕,以十八為法;冬至後夕、夏至後晨,以七十五為法。



 以伏見度乘之、十五除之,為差;滿行差而一為日,不滿,退除為分秒,加減常用積,為定用積;加命如前,即得定見、伏日辰。冬至後,晨見、夕伏加之,夕見、晨伏減之;夏至後,晨見、夕伏減之,夕見、晨伏加之。



 其水星,夕疾在大暑氣初日至立冬氣九日三十五分已下者,不見;晨留在大寒氣初日至立夏氣九日三十五分已下者,春不晨見,秋不夕見。



 熙寧六年六月,提舉司天監陳繹言:「渾儀尺度與《法要》
 不合,二極、赤道四分不均,規、環左右距度不對,游儀重澀難運,黃道映蔽橫簫,游規璺裂,黃道不合天體,天樞內極星不見。天文院渾儀尺度及二極、赤道四分各不均,黃道、天常環、月道映蔽橫簫,及月道不與天合,天常環相攻難轉,天樞內極星不見。皆當因舊修整,新定渾儀,改用古尺,均賦辰度,規、環輕利,黃赤道、天常環並側置,以北際當天度,省去月道,令不蔽橫蕭,增天樞為二度半,以納極星,規、環、二極,各設環樞,以便游運。」詔依新
 式制造,置於司天監測驗,以較疏密。七年六月,司天監呈新制渾儀、浮漏於迎陽門,帝召輔臣觀之,數問同提舉官沉括,具對所以改更之理。尋又言:「準詔,集監官較其疏密,無可比較。」詔置於翰林天文院。七月,以括為右正言,司天秋官正皇甫愈等賞有差。初,括上《渾儀》、《浮漏》、《景表》三議,見《天文志》。朝延用其說,令改造法物、歷書。至是,渾儀、浮漏成,故賞之。



 元豐五年正月,翰林學士王安禮言:「詳定渾儀官歐陽發所上渾儀、浮漏木樣,具新器
 之宜,變舊器之失,臣等竊詳司天監浮漏,疏謬不可用,請依新式改造。其至道、皇祐渾儀、景表亦各差舛,請如法條奏修正。」從之。元祐四年三月,翰林學士許將等言:「詳定元祐渾天儀象所先奉詔制造水運渾儀木樣,如試驗候天不差,即別造銅器,今校驗皆與天合。」詔以銅造,仍以元祐渾天儀象為名。將等又言:「前所謂渾天儀者,其外形圓,可遍布星度;其內有璣、有衡,可仰窺天象。今所建渾儀象,別為二器,而渾儀占測天度之真數,又
 以渾象置之密室,自為天運,與儀參合。若並為一器,即像為儀,以同正天度,則渾天儀像兩得之矣。請更作渾天儀。」從之,七年四月,詔尚書左丞蘇頌撰《渾天儀象銘》。六月,元祐渾天儀象成,詔三省、樞密院官閱之。紹聖元年十月,詔禮部、秘書省,即詳定制造渾天儀象所,以新舊渾儀集局官同測驗,擇其精密可用者以聞。



 宣和六年七月,宰臣王黼言:



 臣崇寧元年邂逅方外之士於京師,自云王其姓,面出素書一,道璣衡之制甚詳。比嘗請
 令應奉司造小樣驗之,逾二月,乃成璇璣,其圓如丸,具三百六十五度四分度之一,置南北極、昆侖山及黃、赤二道,列二十四氣、七十二候、六十四卦、十干、十二支、晝夜百刻,列二十八宿、並內外三垣、周天星。日月循黃道天行,每天左旋一周,日右旋一度,冬至南出赤道二十四度,夏至北入赤道二十四度,春秋二分黃、赤道交而出卯入酉。月行十三度有餘,生明於西,其形如鉤,下環,西見半規,及望而圓;既望,西缺下環,東見半規,及晦而
 隱。某星始見,某星已中,某星將入,或左或右,或遲或速,皆與天象吻合,無纖毫差。玉衡植於屏外,持扼樞斗,注水激輪,其下為機輪四十有三,鉤鍵交錯相持,次第運轉,不假人力,多者日行二千九百二十八齒,少者五日行一齒,疾徐相遠如此,而同發於一機,其密殆與造物者侔焉。自餘悉如唐一行之制。



 然一行舊制機關,皆用銅鐵為之,澀即不能自運,今制改以堅木若美玉之類。舊制外絡二輪,以綴日月,而二輪蔽虧星度,仰視躔次
 不審,今制日月皆附黃道,如蟻行磑上。舊制雖有合望,而月體常圓,上下弦無辨,今以機轉之,使圓缺隱見悉合天象。舊制止有候刻辰鐘鼓,晝夜短長與日出入更籌之度,皆不能辨,今制為司辰壽星,運十二時輪,所至時刻,以手指之,又為燭龍,承以銅荷,時正吐珠振荷,循環自運。其制皆出一行之外。即其器觀之,全像天體者,璇璣也;運用水鬥者,玉衡也。昔人或謂璣衡為渾天儀,或謂有璣而無衡者為渾天象,或謂渾儀望筒為衡:皆
 非也。甚者莫知璣衡為何器。唯鄭康成以運轉者為璣,持正者為衡,以今制考之,其說最近。



 又月之晦明,自昔弗燭厥理,獨揚雄云:「月未望則載魄於西,既望則終魄於東,其溯於日乎?」京房云:「月有形無光,日照之乃光。」始知月本無光,溯日以為光。本朝沉括用彈況月,粉塗其半,以象對日之光,正側視之,始盡圓缺之形。今制與三者之說若合符節。宜命有司置局如樣制,相址於明堂或合臺之內,築臺陳之,以測上象。又別制三器,一納御
 府,一置鐘鼓院,一備車駕行幸所用。仍著為成書,以詔萬世。



 詔以討論制造璣衡所為名,命黼總領,內侍梁師成副之。



\end{pinyinscope}