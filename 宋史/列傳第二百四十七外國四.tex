\article{列傳第二百四十七外國四}

\begin{pinyinscope}

 交趾,本漢初南越之地。漢武平南越,分其地為儋耳、珠崖、南海、蒼梧、郁林、合浦、交趾、九真、日南,凡九郡,置交趾
 刺史以領之。後漢置交州,晉、宋、齊、梁、陳因之,又為交趾郡。隋平陳,廢郡置州;煬帝初,廢州置郡。唐武德中,改交州總管府;至德中,改安南都護府。梁貞明中,土豪曲承美專有其地,送款於末帝,因授承美節鉞。時劉陟擅命嶺表,遣將李知順伐承美,執之,乃並有其地。後有楊廷藝、紹洪皆受廣南署,繼為交趾節度使。紹洪卒,州將吳昌岌遂居其位。昌岌死,其弟昌文襲。



 乾德初,昌文死,其參謀吳處玶、峰州刺史矯知護、武寧州刺史楊暉、牙將
 杜景碩等爭立,管內一十二州大亂。部民嘯聚,起為寇盜,攻交州。先是,楊廷藝以牙將丁公著攝歡州刺史兼御蕃都督,部領即其子也。公著死,部領繼之。至是,部領與其子璉率兵擊敗處玶等,賊黨潰散,境內安堵,交民德之,乃推部領為交州帥,號曰大勝王,署其子璉為節度使。凡三年,遜璉位。璉立七年,聞嶺表平,遂遣使貢方物,上表內附。制以權交州節度使丁璉以檢校太師充靜海軍節度使、安南都護。又詔以進奉使鄭琇、王紹祚
 並為檢校左散騎常侍兼御史大夫。開寶八年,遣使貢犀、象、香藥。朝廷議崇寵部領,降制曰:「率土來王,方推以恩信;舉宗奉國,宜洽于封崇。眷拱極之外臣,舉顯親之茂典。爾部領世為右族,克保遐方;夙慕華風,不忘內附。屬九州混一,五嶺廓清,靡限溟濤,樂輸琛贐。嘉乃令子,稱吾列藩。特被鴻私,以旌義訓。介爾眉壽,服茲寵章。可授開府儀同三司、檢校太師,封交趾郡王。」



 太宗即位,璉又遣使以方物來賀。部領及璉既死,璉弟璿尚幼,嗣立,稱節
 度行軍司馬權領軍府事。大將黎桓擅權樹黨,漸不可制,劫遷璿於別第,舉族禁錮之,代總其衆。太宗聞之,怒,乃議舉兵。太平興國五年秋,詔以蘭州團練使孫全興、八作使張璿、左監門衛將軍崔亮為陸路兵馬部署,自邕州路入;寧州刺史劉澄、軍器庫副使賈湜、供奉官閣門祗候王僎為水路兵馬部署,自廣州路入。是冬,黎桓遣牙校江巨湟齎方物來貢,仍為丁璿上表曰:「臣族本蠻酋,辟處海裔,修職貢於宰旅,假節制于方隅。臣之父
 兄,代承閫寄,謹保封略,罔敢怠遑。爰暨淪亡,將墜堂構,將吏耆耋,乃屬於臣,俾權軍旅之事,用安夷落之衆。土俗獷悍,懇請愈堅,拒而弗從,慮其生變。臣已攝節度行軍司馬權領軍府事,願賜真秩,令備列藩,幹冒宸扆,伏增震越。」上察其欲緩王師,寢而不報。王師進討,破賊萬餘衆,斬首二千餘級。六年春,又破賊于白藤江口,斬首千餘級,獲戰艦二百艘,甲胄萬計。轉運使侯仁寶率前軍先進,全興等頓兵花步七十日以候澄,仁寶累促
 之,不進。及澄至,並軍由水路至多羅村,不遇賊,復擅回花步。桓詐降以誘仁寶,遂為所害。轉運使許仲宣馳奏其事,遂班師。上遣使就劾澄、湜、僎,澄尋病死,戮湜等邕州市。全興至闕,亦下吏誅,餘抵罪有差。仁寶贈工部侍郎



 七年春,桓懼朝廷終行討滅,復以丁璿為名,遣使貢方物,上表謝罪。八年,桓自稱權交州三使留後,遣使貢方物,並以璿表來上,帝賜桓詔曰:「丁氏傳襲三世,保據一方,卿既受其倚毗,為之心膂,克徇邦人之請,無負丁氏
 之心。朕且欲令璿為統帥之名,卿居副貳之任,剸裁制置,悉系於卿。俟丁璿既冠,有所成立,卿之輔翼,令德彌光,崇獎忠勳,朕亦何吝!若丁璿將材無取,童心如故,然其奕世紹襲,載綿星紀,一旦舍去節鉞,降同士伍,理既非便,居亦靡安。詔到,卿宜遣丁璿母子及其親屬盡室來歸。俟其入朝,便當揆日降制,授卿節旄。凡茲兩途,卿宜審處其一。丁璿到京,必加優禮。今遣供奉官張宗權齎詔諭旨,當悉朕懷。」亦賜璿詔書如旨。時黎桓已專據
 其土,不聽命。是歲五月上言,占城國水陸象馬數萬來寇,率所部兵擊走之,俘斬千計。



 雍熙二年,遣牙校張紹馮、阮伯簪等貢方物,繼上表求正領節鎮。三年秋,又遣使貢方物。儋州言,占城國人蒲羅遏率其族百餘衆內附,言為交州所逼故也。是歲十月,制曰:「王者懋建皇極,寵綏列藩。設邸京師,所以盛會同之禮;胙土方面,所以表節制之雄。矧茲ㄢ鳶之隅,克修設羽之貢,式當易帥,爰利建侯,不忘請命之恭,用舉醻勞之典。權知交州三
 使留後黎桓,兼資義勇,特稟忠純,能得邦人之心,彌謹藩臣之禮。往者,丁璿方在童幼,昧於撫綏。桓乃肺腑之親,專掌軍旅之事,號令自出,威愛並行。璿盡解三使之權,以徇衆人之欲。遠輸誠款,求領節旄。土燮強明,化越俗而鹹乂;尉佗恭順,稟漢詔以無違。宜正元戎之稱,以列通侯之貴,控撫夷落,對揚天休。可檢校太保、使持節、都督交州諸軍事、安南都護,充靜海軍節度、交州管內觀察處置等使,封京兆郡侯,食邑三千戶,仍賜號推誠
 順化功臣。」遣左補闕李若拙、國子博士李覺為使以賜之。



 端拱元年,加桓檢校太尉,進邑千戶,實封五百戶。遣戶部郎中魏庠、虞部員外郎直史館李度往使焉。淳化元年夏,加桓特進,邑千戶,實封四百戶。遣左正言直史館宋鎬、右正言直史館王世則又使焉。明年六月,歸闕,上令條列山川形勢及黎桓事蹟以聞。鎬等具奏曰:



 去歲秋末抵交州境,桓遣牙內都指揮使丁承正等以船九艘、卒三百人至太平軍來迎,由海口入大海,冒涉風
 濤,頗曆危險。經半月至白藤,徑入海氵義,乘潮而行。凡宿泊之所皆有茅舍三間,營葺尚新,目為館驛。至長州漸近本國,桓張惶虛誕,務為誇詫,盡出舟師戰棹,謂之耀軍。



 自是宵征抵海岸,至交州僅十五里,有茅亭五間,題曰茅徑驛。至城一百里,驅部民畜產,妄稱官牛,數不滿千,揚言十萬。又廣率其民混於軍旅,衣以雜色之衣,乘船鼓噪。近城之山虛張白旗,以為陳兵之象。俄而擁從桓至,展郊迎之禮,桓斂馬側身,問皇帝起居畢,按轡偕行。時
 以檳榔相遺,馬上食之,此風俗待賓之厚意也。城中無居民,止有茅竹屋數十百區,以為軍營。而府署湫隘,題其門曰明德門。



 桓質陋而目眇,自言近歲與蠻寇接戰,墜馬傷足,受詔不拜。信宿之後,乃張筵飲宴。又出臨海氵義,以為娛賓之遊。桓跣足持竿,入水標魚,每中一魚,左右皆叫噪歡躍。凡有宴會,預坐之人悉令解帶,冠以帽子。桓多衣花纈及紅色之衣,帽以真珠為飾,或自歌勸酒,莫能曉其詞。嘗令數十人扛大蛇長數丈,饋於使館,
 且曰:「若能食此,當治之為饌以獻焉。」又羈送二虎,以備縱觀。皆卻之不受。士卒殆三千人,悉黥其額曰「天子軍」。糧以禾穗日給,令自舂為食。兵器止有弓弩、木牌、梭槍、竹槍,弱不可用。



 桓輕亻兌殘忍,昵比小人,腹心閹豎五七輩錯立其側。好狎飲,以手令為樂。凡官屬善其事者,擢居親近左右,有小過亦殺之,或鞭其背一百至二百。賓佐小不如意,亦捶之三十至五十,黜為閽吏;怒息,乃召復其位。有木塔,其制樸陋,桓一日請同登遊覽。地無寒
 氣,十一月猶衣夾衣揮扇云。



 四年,進封桓交趾郡王。五年,遣牙校費崇德等來修職貢。然桓性本兇狠,負阻山海,屢為寇害,漸失藩臣禮。至道元年春,廣南西路轉運使張觀、欽州如洪鎮兵馬監押衛昭美皆上言,有交州戰船百餘艘寇如洪鎮,略居民,劫廩實而去。其夏,桓所管蘇茂州,又以鄉兵五千寇邕州所管綠州,都巡檢楊文傑擊走之。太宗志在撫寧荒服,不欲問罪。觀又言,風聞黎桓為丁氏斥逐,擁餘衆山海間,失其所據,故以寇
 鈔自給,今則桓已死。觀仍上表稱賀。詔太常丞陳士隆、高品武元吉奉使嶺南,因偵其事。士隆等復命,所言與觀同。其實桓尚存,而傳聞者之誤,觀等不能審核。未幾,有大賈自交趾回,具言桓為帥如故。詔劾觀等,會觀病卒,昭美、士隆、元吉抵罪。



 先是,欽州如洪、咄步、如昔等三鎮皆瀕海,交州潮陽民卜文勇等殺人,並家亡命至如昔鎮,鎮將黃令德等匿之。桓令潮陽鎮將黃成雅移牒來捕,令德固不遣,因茲海賊連年剽掠。二年,以工部員
 外郎、直史館陳堯叟為轉運使,因賜桓詔書。堯叟始至,遣攝雷州海康縣尉李建中齎詔勞問桓。堯叟又至如昔,詰得匿文勇之由,盡擒其男女老少一百三十口,召潮陽鎮吏付之,且戒勿加酷法。成雅得其人,以狀謝堯叟。桓遂上章感恩,並捕海賊二十五人送於堯叟,且言已約勒溪洞首領,不得騷動。七月,太宗遣主客郎中、直昭文館李若拙齎詔書,充國信使,以美玉帶往賜桓。若拙既至,桓出郊迎,然其詞氣尚悖慢,謂若拙曰:「向者劫
 如洪鎮乃外境蠻賊也,皇帝知此非交州兵否?若使交州果叛命,則當首攻番禺,次擊閩、越,豈止如洪鎮而已!」若拙從容謂桓曰:「上初聞寇如洪鎮,雖未知其所自,然以足下拔自交州牙校,授之節制,固當盡忠以報,豈有他慮!及見執送海賊,事果明白。然而大臣僉議,以為朝廷比建節帥,以寧海表,今既蠻賊為寇害,乃是交州力不能獨制矣。請發勁卒數萬,會交兵以剪滅之,使交、廣無後患。上曰:'未可輕舉,慮交州不測朝旨,或致驚駭,不
 若且委黎桓討擊之,亦當漸至清謐。'今則不復會兵也。」桓愕然避席,曰:「海賊犯邊,守臣之罪也。聖君容貸,恩過父母,未加誅責。自今謹守職約,保永清於漲海。」因北望頓首謝。



 真宗即位,進封桓南平王兼侍中。桓前遺都知兵馬使阮紹恭、副使趙懷德以金銀七寶裝交椅一、銀盆十、犀角象牙五十枚、絹糸由布萬匹來貢。詔陳于萬歲殿太宗神御,許紹恭等拜奠。及回,賜桓帶甲馬,詔書慰獎。咸平四年,又遣行軍司馬黎紹、副使何慶常,以馴
 犀一、象二、象犬朋二、七寶裝金瓶一來貢。其年欽州言,交州效誠場民及頭首八州使黃慶集等數百人來投,有詔慰撫,遣還本道。廣南西路言,黎桓迎受官告使黃成雅附奏,自今國朝加恩,願遣使至本道,以寵海裔。先是,使至交州,桓即以供奉為辭,因緣賦斂。上聞之,止令疆吏召授命,不復專使。景德元年,又遣其子攝歡州刺史明提來貢,懇求加恩使至本道慰撫遐裔,許之,仍以明提為歡州刺史。二年上元節,賜明提錢,令與占城、大食
 使觀燈宴飲,因遣工部員外郎邵曄充國信使。



 三年,桓卒,立中子龍鉞。龍鉞兄龍全劫庫財而遁,其弟龍廷殺龍鉞自立。龍廷兄明護率扶闌砦兵攻戰。明提以國亂不能還,特詔廣州優加資給。知廣州淩策等言:「桓諸子爭立,衆心離叛,頭首黃慶集、黃秀蠻等千餘人以不從驅率,戮及親族,來投廉州,請發本道二千人平之,慶集等願為前鋒。」上以桓素忠順,屢修職貢,今幸亂而伐喪,不可。就改國信使邵曄為緣海安撫使,令曉譬之。慶集
 等仍計口賜田糧。曄乃貽書交州,諭以朝廷威德,如其自相魚肉,久無定位,偏師問罪,則黎氏盡滅矣。明護懼,即奉龍廷主軍事。龍廷自稱節度、開明王,遂欲修貢。曄以聞,上曰:「遐荒異俗,不曉事體,何足怪也?」令削去偽官。曄又言,頭首黃慶集先避亂歸化,其種族尚多,若復遣還,慮遭屠戮。詔以慶集隸三班,厘務於郴州,遂許入貢。



 四年,龍廷稱權安南靜海軍留後,遣弟峰州刺史明昶、副使安南掌書記殿中丞黃成雅等來貢。會含光殿大宴,
 上以成雅坐遠,欲稍升位著,訪于宰相王旦,旦曰:「昔子產朝周,周王饗以上卿之禮,子產固辭,受下卿之禮而還。國家惠綏遠方,優待客使,固無嫌也。」乃升成雅於尚書省五品之次。詔拜龍廷特進、檢校太尉,充靜海軍節度觀察處置等使、安南都護,兼御史大夫、上柱國,仍封交趾郡王,食邑三千戶,食實封一千戶。賜推誠順化功臣,仍賜名至忠,給以旌節。又追贈桓中書令、南越王。進奉使黎明昶等並進秩。大中祥符元年,天書降,加翊戴功
 臣,食邑七百戶,實封三百戶。東封畢,加至忠同平章事,食邑一千戶,食實封四百戶。二年,廣南西路言,欽州蠻人劫海口蜑戶,如洪砦主李文著以輕兵襲逐,中流矢死。詔督安南捕賊。明年,執狄獠十三人以獻。至忠又遣推官阮守疆以犀角、象齒、金銀、紋縭等來貢。並獻馴犀一。上以犀違土性,不可豢畜,卻不納。又以逆至忠意,使者既去,乃令縱之海ㄛ。三年,遣使來朝,表求甲胄具裝,詔從其請。又求互市於邕州,本道轉運使以聞,上曰:「瀕海
 之民,數患交州侵寇,仍前止許廉州及如洪砦互市,蓋為邊隅控扼之所。今或直趨內地,事頗非便。」詔令本道以舊制諭之。



 至忠才年二十六,苛虐不法,國人不附。大校李公蘊尤為至忠親任,嘗令以黎為姓。其年,遂圖至忠,逐之,殺明提、明昶等,自稱留後,遣使貢奉。上曰:「黎桓不義而得,公蘊尤而效之,甚可惡也。」然以其蠻俗不足責,遂用桓故事,制授特進、檢校太傅,充靜海軍節度觀察處置等使、安南都護,兼御史大夫、上柱國,封交趾郡
 王,食邑三千戶,實封一千戶,賜推誠順化功臣。公蘊又表求太宗御書,詔賜百軸。四年,祀汾陰後土,公蘊遣節度判官梁任文、觀察巡官黎再嚴以方物來貢,禮成,加公蘊同平章事,食邑一千戶,實封四百戶,任文等並翁優進秩。五年夏,以進奉使李仁美為誠州刺史、陶慶文為太常丞,其從隸有道病死者,所賜附還其家。是冬,聖祖降,加公蘊開府儀同三司,食邑七百戶,實封三百戶,賜翊戴功臣。七年春,又加保節守正功臣,食邑一千戶,實
 封四百戶。詔交趾諸國使入貢者,所在館餼供億,務令豐備。其年,遣知唐州刺史陶碩等來貢。詔以碩為順州刺史,充安南靜海軍行軍司馬;副使吳懷嗣為澄州刺史,充節度副使。先是,交州狄獠張婆看避罪來奔,知欽州穆重穎召之,至中路復拒焉,都巡檢臧嗣遂令如洪砦犒以牛酒。交州偵知其事,因捕狄獠,故鈔如洪砦,掠人畜甚衆。詔轉運司督公蘊追索,仍令疆吏自今不得誘召蠻獠致生事。公蘊或間歲或仍歲以方物入貢。天
 禧元年,進封公蘊南平王,加食邑一千戶,實封四百戶。三年,加檢校太尉,食邑一千戶,實封四百戶。每加恩皆遣使將命至其境上,仍賜器幣、襲衣、金帶、鞍馬焉。仁宗即位,加公蘊檢校太師。遣長州刺史李寬泰、都護副使阮守疆來貢。天聖六年,遣歡州刺史李公顯來貢,除敘州刺史。既而令其子弟及其婿申承貴率衆內寇,詔廣南西路轉運司發溪峒丁壯討捕之。未幾,卒,年四十四。


其子德政自稱權知留後事,來告哀。贈公蘊為侍中、南
 越王,命本路轉運使王惟正為祭奠使,又為賜官告使。除德政檢校太尉、靜海軍節度使、安南都護、交趾郡王。天聖九年,遣知峰州刺史李偓佺、知愛州刺史帥日新等來謝,以偓佺為歡州刺史、日新為珍州刺史。明道元年,恭謝,加同中書門下平章事。景祐中,郡人陳公永等六百餘人內附,德政遣兵千餘境上捕逐之。詔遣還,仍戒德政毋得輒誅殺。尋遣靜海軍節度判官陳應機、掌書記王惟慶來貢,以應機為太子中允、惟慶為大理
 寺丞,德政加檢校太師。三年,其甲峒及諒州、門州、蘇茂州、廣源州、大發峒、丹波縣蠻寇邕州之思陵州、西平州、石西州及諸峒,略居人馬牛,焚室廬而去。下詔責問之,且令捕酋首正其罪以聞。寶元元年,進封南平王。康定元年,遣知峰州刺史帥用和、節度副使杜猶興等來貢。慶曆三年,又遣節度副使杜慶安、三班奉職梁材來,以慶安為順州刺史、材為太子左監門率府率。六年,又遣兵部員外郎蘇仁祚、東頭供奉官陶惟●
 \gezhu{
  缺字:巾雇}
 来,以仁祚為工
 部郎中、惟●
 \gezhu{
  缺字:巾雇}
 為內殿崇班。明年,又遣秘書丞杜文府、左侍禁文昌來,以文府為屯田員外郎、昌為內殿崇班。



 初,德政發兵取占城,朝廷疑其內畜奸謀,乃訪自唐以來所通道路凡十六處,令轉運使杜杞度其要害而戍守之,然其後亦未嘗寇邊。前後累貢馴象。皇祐二年,邕州誘其蘇茂州韋紹嗣、紹欽等三千余人入居省地,德政表求所誘。詔盡還之,仍令德政約束邊戶,毋相侵犯。其後,廣源州蠻儂智高反,德政率兵二萬由水路欲入助
 王師,朝廷優其賜而卻其兵。至和二年,卒。



 其子日尊遣人告哀,命廣南西路轉運使、尚書屯田員外郎蘇安世為吊贈使,贈德政為侍中、南越王,賻齎甚厚。尋除日尊特進、檢校太尉、靜海軍節度使、安南都護,封交趾郡王。嘉祐三年,貢異獸二。四年,寇欽州思稟管。五年,與甲峒賊寇邕州,詔知桂州蕭固發部兵與轉運使宋鹹、提點刑獄李師中同議掩擊;又詔安撫使余靖等發兵捕討。靖遣諜誘占城同廣南西路兵甲趨交趾,日尊惶怖,上
 表待罪。詔未得舉兵,聽日尊貢奉至京師。八年,遣文思使梅景先、副使大理評事李繼先貢馴象九。四月戊寅,以大行皇帝詔及遺留物賜日尊,加同中書門下平章事。是日,交趾使辭,命內侍省押班李繼和喻以申紹泰入寇,本路屢乞討伐,而朝廷以紹泰一夫肆狂,又本道已遣使謝罪,故未欲興兵。治平初,知桂州陸詵言,交州來求儂宗旦男日新及欲取溫悶洞等地,帝問交趾于何年割據,輔臣對曰:「自唐至德中改安南都護府,梁
 貞明中,土豪曲承美專有此地。」韓琦曰:「向以黎桓叛命,太宗遣將討伐,不服,後遣使詔誘,始效順。交州山路險僻,多潦霧瘴毒之氣,雖得其地,恐不能守也。」神宗即位,進封日尊南平王。熙甯元年,加開府儀同三司。二年,表言:「占城國久闕貢,臣親帥兵討之,虜其王。」詔以其使郭士安為六宅副使、陶宗元為內殿崇班。日尊自帝其國,僭稱法天應運崇仁至道慶成龍祥英武睿文尊德聖神皇帝,尊公蘊為太祖神武皇帝,國號大越,改元寶象,
 又改神武。



 五年三月,日尊卒。命廣西轉運使康衛為吊贈使。予所奪州縣。詔報之曰:「卿撫有南交,世受王爵,而乃背德奸命,竊暴邊城。棄祖考忠順之圖,煩朝廷討伐之舉。師行深入,勢蹙始歸。跡其罪尤,在所絀削。今遣使修貢,上章致恭,詳觀詞情,灼見悛悔。朕撫綏萬國,不異邇遐。但以邕、欽之民,遷劫炎陬,久失鄉井,俟盡送還省界,即以廣源等賜交州。」乾德初約歸三州官吏千人,久之,才送民二百二十一口,男子年十五以上皆刺額曰「
 天子兵」,二十以上曰「投南朝」,婦人刺左手曰「官客」。以舟載之而泥其戶牖,中設燈燭,日行一二十里則止,而偽作更鼓以報,凡數月乃至,蓋以紿示海道之遠也。順州落南深,置戍鎮守,被罹瘴霧多病沒,陶弼亦終於官。朝廷知其無用,乃悉以四州一縣還之。然廣源舊隸邕管羈縻,本非交趾所有也。



 元豐五年,獻馴象二、犀角象齒百。六年,以追捕儂智會為辭,犯歸化州。又遣其臣黎文盛來廣西辦理順安、歸化境界,經略使熊本遣左江巡
 檢成卓與議,文盛稱陪臣,不敢爭執。詔以文盛能遵乾德恭順之意,賜之袍帶及絹五百匹。仍以八隘之外保樂六縣、宿桑二峒予乾德。哲宗立,加同中書門下平章事。元祐中,又數上表求勿惡、勿陽峒地,詔不許。二年,遣使入貢,進封南平王。徽宗時,累加開府儀同三司、檢校太師。大觀初,貢使至京乞市書籍,有司言法不許,詔嘉其慕義,除禁書、卜筮、陰陽、曆算、術數、兵書、敕令、時務、邊機、地理外,余書許買。政和末,又詔以交人自熙甯以來,
 全不生事,特寬和市之禁。宣和元年,加乾德守司空。建炎元年,詔廣西經略安撫司禁邊民毋受安南逋逃,從其主乾德之請也。四年,安南入貢,詔卻其方物之華靡者,賜敕書,厚其報以懷柔之。



 紹興二年,乾德卒。贈侍中,追封南越王。子陽煥嗣,授靜海軍節度使、特進、檢校太尉,封交趾郡王,賜推誠順化功臣。八年,陽煥卒,以轉運副使朱芾充弔祭使,贈陽煥開府儀同三司,追封南平王。子天祚嗣,授官如其父初封之制。九年,詔廣西帥司
 毋受趙智之入貢。初,乾德有側室子奔大理,變姓名為趙智之,自稱平王。聞陽煥死,大理遣歸,與天祚爭立,求入貢,欲假兵納之,帝不許。十七年,詔文思院制鞍韉以賜天祚。二十一年,累加天祚崇義懷忠保信鄉德安遠承和功臣。二十五年,詔館安南使者于懷遠驛,賜宴,以彰異數。進封天祚南平王,賜襲衣、金帶、鞍馬。二十六年,命右司郎中汪應辰宴安南使者于玉津園。八月,天祚遣李國等以金珠、沉水香、翠羽、良馬、馴象來貢。詔加天
 祚檢校太師,增食邑。隆興二年,天祚遣尹子思、鄧碩儼等貢金銀、象齒、香物。乾道六年,累加天祚歸仁協恭繼美遵度履正彰善功臣。帝自即位,屢卻安南貢使。九年,天祚復遣尹子思、李邦正求入貢。帝嘉其誠,許之,詔館於懷遠驛。廣南西路經略安撫使范成大言:「本司經略諸蠻,安南在撫綏之內,其陪臣豈得與中國王官亢禮?政和間,貢使入境,皆庭參,不復報謁。宜遵舊制,於禮為得。」朝廷從其請。淳熙元年二月,進封天祚安南國王,加
 號守謙功臣。二年,賜安南國印。三年,賜安南國曆日。天祚卒。



 明年,子龍榦嗣位,授靜海軍節度使觀察處置等使、特進、檢校太尉兼御史大夫、上柱國,特封安南國王,加食邑;仍賜推誠順化功臣,制曰:「即樂國以肇封,既從世襲;極真王而錫命,何待次升?」示殊禮也。五年,貢方物,上表稱謝。九年,詔卻安南所貢象,以其無用而煩民,他物亦止受什一。十六年,累加龍榦守義奉國履常懷德功臣。光宗即位,奉表入貢稱賀。甯宗朝,賜衣帶、器幣,累
 加謹度思忠濟美勤禮保節歸仁崇謙協恭功臣及食邑焉。



 嘉定五年,龍榦卒。詔以廣西運判陳孔碩充弔祭使,特贈侍中。依前安南國王制,以其子昊旵襲封其爵位,給賜如龍榦始封之制,仍賜推誠順化功臣。其後謝表不至,遂輟加恩。



 昊旵卒,無子,以女昭聖主國事,遂為其婿陳日煚所有。李氏有國,自公蘊至昊旵,凡八傳,二百二十餘年而國亡。淳祐二年,詔安南國王陳日煚,元賜效忠順化保節功臣增「守義」二字。寶祐六年,詔安南
 情狀叵測,申飭邊備。景定二年,貢象二。三年,表乞世襲。詔日煚授檢校太師、安南國大王,加食邑;男威晃,授靜海軍節度使、觀察處置使、檢校太尉兼御史大夫、上柱國、安南國王、效忠順化功臣,賜金帶、器幣、鞍馬。咸淳五年,詔安南國王父日煚、國王威晃加食邑。八年,明堂禮成,日煚、威晃各加食邑,賜鞍馬等物。



 大理國,即唐南詔也。熙寧九年,遣使貢金裝碧玕山、氈罽、刀劍、犀皮甲鞍轡。自後不常來,亦不領於鴻臚。



 政和
 五年,廣州觀察使黃璘奏,南詔大理國慕義懷徠,願為臣妾,欲聽其入貢。詔璘置局於賓州,凡有奏請,皆俟進止。六年,遣進奉使天駟爽彥賁李紫琮、副使坦綽李伯祥來,詔璘与廣東轉運副使徐惕偕詣闕,其所經行,令監司一人主之。道出荊湖南,當由邵州新化縣至鼎州,而璘家潭之湘鄉,轉運判官喬方欲媚璘,乃排比由邵至潭,由潭至鼎一路。御史劾其當農事之際,而觀望勞民,詔罷方。紫琮等過鼎,聞學校文物之盛,請於押伴,求
 詣學瞻拜宣聖像,邵守張察許之,遂往,遍謁見諸生。又乞觀御書閣,舉笏扣首。



 七年二月,至京師,貢馬三百八十匹及麝香、牛黃、細氈、碧玕山諸物。制以其王段和譽為金紫光祿大夫、檢校司空、雲南節度使、上柱國、大理國王。朝廷以為璘有功,並其子暉、昨皆遷官,少子𣆳為閣門宣贊舍人。已而知桂州周璘詐冒,璘得罪。自是大理復不通於中國,間一至黎州互市。



 紹興三年十月,廣西奏,大理國求入貢及售馬,詔卻之,不欲以虛名勞
 民也。朱勝非奏曰:「昔年大理入貢,言者深指其妄,黃璘由是獲罪。」帝曰:「遐方異域,何由得實,但仇當其馬價,則馬方至,用益騎兵,不為無補也。」六年七月,廣西經略安撫司奏,大理復遣使奉表貢象、馬,詔經略司護送行在,優禮答之。九月,翰林學士朱震上言,乞諭廣西帥臣,凡市馬當擇謹厚者任之,毋遣好功喜事之人,以啟邊釁。異時南北路通,則漸減廣西市馬之數,庶幾消患于未然。詔從之。



 淳熙二年十一月,知靜江府張┉申
 嚴保伍之禁,又以邕管戍兵不能千人,左、右江峒丁十余萬,每恃以為藩蔽,其邕州提舉、巡檢官宜精其選,以撫峒丁。欲制大理,當自邕管始云。



\end{pinyinscope}