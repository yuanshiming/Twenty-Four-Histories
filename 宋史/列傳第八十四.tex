\article{列傳第八十四}

\begin{pinyinscope}

 劉平弟兼濟郭遵附任福王珪武英桑懌耿傅王仲寶附



 劉平,字士衡,開封祥符人。父漢凝,從太宗征河東岢嵐、憲州,累遷崇儀使。平剛直任俠,善弓馬,讀書強記。進士及第,補無錫尉,擊賊殺五人,擢大理評事。知鄢陵縣,徙
 南充。夷人寇淯井監,轉運使以平權瀘州事,平率土丁三千擊走之。祠汾陰,遷本寺丞。還,路由安州,遇賊十數人,平發矢斃三賊,餘駭散。以寇準薦,為殿中丞、知瀘州,夷人懲前敗,不敢擾邊。



 召拜監察御史,數上疏論事,為丁謂所忌。久之,除三司鹽鐵判官、河北安撫,改殿中侍御史、陜西轉運使。與副使論事不合,徙知襄州。仁宗即位,遷侍御史。



 初,真宗知其才,將用之。丁謂乘間曰;「平,將家子,素知兵,若使將西北,可以制敵。」後章獻太后思謂
 言,特改衣庫使、知邠州。屬戶明珠、磨糜族數反復,平潛兵殺數千人,以功領賓州刺史、鄜延路兵馬鈐轄,徙涇原路,兼知渭州。胡則為陜西都轉運使,平奏曰:「則,丁謂黨,今隸則部,慮掎摭致罪。」徙汝州,改淮南、江、浙、荊湖制置發運副使,行數驛,召還,真拜信州刺史、知雄州。居四年,遷忻州團練使、知成德軍。



 景祐元年,拜龍神衛四廂都指揮使、永州防禦使、知定州,徙環慶路副都總管,進侍衛親步軍都虞候。奏言:「元昊勢且叛,宜嚴備之。」尋坐
 被酒破鎖入甲仗庫,為轉運使蘇耆所劾,落管軍,知同州。上疏自列,召入問狀,復為步軍都虞候、知澶州。時議塞河,而平言不知河事,乃徙滄州副都總管。



 時呂夷簡為宰相,臺諫官數言政事闕失,平奏書曰:「臣見範仲淹等毀訾大臣,此必有要人授旨仲淹輩,欲逐大臣而代其位者。臣於真宗朝為御史,顧當時同列,未聞有奸邪黨與詐忠賣直,所為若此。臣慮小臣以淺文薄伎,偶致顯用,不識朝廷典故,而論事浸淫,遂及管軍將校。且武
 人進退,與儒臣異路,若掎摭短長,妄有舉劾,則心搖而怨結矣。願明諭臺諫官,毋令越職,仍不許更相引薦。或闕員,則朝廷自擇忠純耆德用之。」論者以謂希夷簡意也。改高陽關副總管。



 寶元元年,以殿前都虞候為環慶路馬步軍副總管。會元昊反,遷邕州觀察使,為鄜延路副總管兼鄜延、環慶路同安撫使。頃之,兼管勾涇原路兵馬,進步軍副都指揮使、靜江軍節度觀察留後。獻攻守之策曰:



 五代之末,中國多事,唯制西戎為得之。中國
 未嘗遣一騎一卒,遠屯塞上,但任土豪為眾所伏者,封以州邑,征賦所入,足以贍兵養士,由是無邊鄙之虞。太祖定天下,懲唐末藩鎮之盛,削其兵柄,收其賦入,自節度以下,第坐給奉祿,或方面有警,則總師出討,事已,則兵歸宿衛,將還本鎮。彼邊方世襲,宜異於此,而誤以朔方李彞興、靈武馮繼業一切亦徙內地。自此靈、夏仰中國戍守,千里運糧,兵民並困。



 其後靈武失守,而趙德明懼王師問罪,願為藩臣。於時若止棄靈、夏、綏、銀,與之限
 山為界,則無今日之患矣。而以靈、夏兩州及山界蕃漢戶並授德明,故蓄甲治兵,漸窺邊隙,鄜延、環慶、涇原、秦隴所以不能弛備也。



 今元昊嗣國,政刑慘酷,眾叛親離,復與唃廝囉構怨,此乃天亡之時。臣聞寇不可玩,敵不可縱。或元昊不能自立,別有酋豪代之,西與唃廝囉復平,北約契丹為表裏,則何以制其侵軼?今元昊國勢未強,若乘此用鄜延、環慶、涇原、秦隴四路兵馬,分兩道,益以蕃漢弓箭手,精兵可得二十萬,三倍元昊之眾,轉糧
 二百里,不出一月,可收山界洪、宥等州。招集土豪,縻之以職,自防禦使以下、刺史以上,第封之,給以衣祿金帛;又以土人補將校,使勇者貪於祿,富者安於家,不期月而人心自定。及遣使諭唃廝囉,授以靈武節度,使撓河外族帳,以窘元昊。復出麟、府、石州蕃漢步騎,獵取河西部族,招其酋帥,離其部眾,然後以大軍繼之,元昊不過鼠竄為窮寇爾,何所為哉!



 且靈、夏、綏、銀地不產五穀,人不習險阻,每歲資糧,取足洪、宥。而洪、宥州羌戶勁勇善
 戰,夏人恃此以為肘腋。我茍得之,以山為界,憑高據險,下瞰沙漠,各列堡障,量以戎兵鎮守,此天險也。廟朝之謀,不知出此,而爭靈、夏、綏、銀,連年調發,老師費財,以致中國疲弊,小醜猖獗,此議臣之罪也。



 今朝廷或貸元昊罪,更示含容,不惟宿兵轉多,經費尤甚。萬一元昊潛結契丹,互為掎角,則我一身二疾,不可並治。必輕者為先,重者為後,如何減兵以應河北?請召邊臣,與二府定守禦長策。



 疏奏未報。



 屬元昊盛兵攻保安軍,時平屯慶州,
 範雍以書召平,平率兵與石元孫合軍趨土門。既又有告敵兵破金明、圍延州者,雍復召平與元孫救延州。平素輕敵,督騎兵晝夜倍道行,明日,至萬安鎮。平先發,步軍繼進,夜至三川口西十里止營,遣騎兵先趨延州爭門。時鄜延路駐泊都監黃德和將二千餘人,屯保安北碎金谷,巡檢萬俟政、郭遵各將所部分屯,範雍皆召之為外援,平亦使人趣其行。詰旦,步兵未至,平與元孫還逆之。行二十里,乃遇步兵,及德和、萬俟政、郭遵所將兵悉
 至,將步騎萬餘結陣東行五里,與敵遇。



 時平地雪數寸,平與敵皆為偃月陣相響。有頃,敵兵涉水為橫陣,郭遵及忠佐王信薄之,不能入。官軍並進,殺數百人,乃退。敵復蔽盾為陣,官軍復擊卻之,奪盾,殺獲及溺水死者幾千人。平左耳、右頸中流矢。日暮,戰士上首功及所獲馬,平曰:「戰方急,爾各志之,皆當重賞汝。」語未已,敵以輕兵薄戰,官軍引卻二十步。黃德和居陣後,望見軍卻,率麾下走保西南山,眾從之,皆潰。平遣其子宜孫馳追德和,
 執轡語曰:「當勒兵還,人並力抗敵,奈何先奔?」德和不從,驅馬遁赴甘泉。平遣軍校杖劍遮留士卒,得千餘人。轉鬥三日,賊退還水東。平率餘眾保西南山,立七柵自固。敵夜使人叩柵,問大將安在,士不應。復使人偽為戍卒,遞文移平,平殺之。夜四鼓,敵環營呼曰:「如許殘兵,不降何待!」平旦,敵酋舉鞭麾騎,自山四出合擊,絕官軍為二,遂與元孫皆被執。



 初,德和言平降賊,朝廷發禁兵圍其家。及命殿中侍御史文彥博即河中府置獄,遣龐籍往訊
 焉,具得其實。遂釋其家,德和坐腰斬。而延州吏民亦詣闕訴平戰沒狀,遂贈朔方軍節度使兼侍中,謚壯武,賜信陵坊第,封其妻趙氏為南陽郡太夫人,子孫及諸弟皆優遷,未官者錄之。其後降羌多言平在興州未死,生子於賊中。及石元孫歸,乃知平戰時被執,後沒於興州。弟兼濟。



 兼濟字寶臣,以父蔭補三班奉職。善騎射,讀兵書知大旨。為襄州兵馬監押。漢江暴漲,兼濟解衣涉水,率眾捍
 城,州賴以完。擢閣門祗候、雄霸州界河巡檢,徙晉、絳、澤、潞都巡檢使。歲饑,太行多盜,禽二百餘人。改左侍禁、鄜延路兵馬都監,權知保安軍,歷同提點陜西、河東刑獄,徙知籠竿城。



 夏人寇邊,眾號數萬,兼濟將兵千餘,轉戰至黑松林,敗之。屬其兄平戰沒於三川口,特授內殿崇班、知原州。入辭,仁宗慰勉之曰:「國憂未弭,家仇未報,不可不力也。」屬戶明珠族叛,諸將欲亟討。兼濟第日縱飲擊鞠,繆為不知,以疑其意。既而叛者自潰,乃追襲之,射
 殺其酋長,收餘眾以歸。徙寧州,破靳廝襪砦,徙鄜州。



 元昊既稱藩,徙梓夔路鈐轄,又徙知鎮戎軍。兼濟御下嚴急,轉運使言士心多怨,請徙諸內地。改涇原路鈐轄,復知寧州,又知原州,徙冀州、廣信軍。累遷文思使、惠州刺史、河北緣邊安撫副使,擢西上閣門使、同管勾三班院,出知雄州。



 先是,邊民避罪逃者,契丹輒納之,守將畏事不敢詰,兼濟悉移檄責還。徙冀州,逾月,改忻州,復管勾三班院,卒。



 郭遵者,開封人也,家世以武功稱。遵少隸軍籍,稍遷殿前指揮使。乾興中,改左班殿直、並代路巡檢。遷右侍禁、慶州柔遠砦兵馬監押。召試騎射優等,遷左侍禁、閣門祗候。為秦州三陽砦主,徙延州西路都巡檢使。



 元昊寇延州,遵以裨將屬劉平,遇敵,馳馬入敵陣,殺傷數十人。敵出驍將揚言當遵,遵揮鐵杵破其腦,兩軍皆大呼。復持鐵槍進,所向披靡。會黃德和引兵先潰,敵戰益急。遵奮擊,期必死,獨出入行間。軍稍卻,即復馬以殿,又持大
 槊橫突之。敵知不可敵,使人持大SZ索立高處迎遵馬,輒為遵所斷。因縱遵使深入,攢兵注射之,中馬,馬踠僕地,被殺。特贈果州團練使。以其父斌為太子右清道率府副率;母賀,封仁壽郡君;妻尹,安康郡君;弟青石侍禁,逵三班奉職。四子尚幼,仁宗悉為賜名,忠嗣西頭供奉官,忠紹左侍禁,忠裔右侍禁,忠緒左班殿直。女舊為尼,亦賜紫方袍。



 遵用鐵杵、槍、槊、共九十斤,其後耕者得其器於戰處,皇祐中,乃人並與其衣冠葬之河南。逵自有傳。



 任福,字祐之,其先河東人,後徙開封。咸平中,補衛士,由殿前諸班累遷至遙郡刺史。元昊反,除莫州刺史、嵐石隰州緣邊都巡檢使。既辭,奏曰:「河東地介大河,斥堠疏闊,願嚴守備,以戒不虞。」仁宗善之,命知隴州,擢秦鳳路馬步軍副總管。詔陜西增城壘、器械,福受命四十日,而戰守之備皆具。以忻州團練使為鄜延路副總管、管勾延州東路蕃部事。



 尋知慶州,復兼環慶路副總管。上言:「慶州去蕃族不遠,願勒兵境上,按亭堡,謹斥堠。」因經度
 所過山川道路,以為緩急攻守之備。帝益善之,聽便宜從事。



 夏人寇保安、鎮戎軍,福與子懷亮、侄婿成暠自華池鳳川鎮聲言巡邊,召諸將牽制敵勢。行至柔遠砦,犒蕃部,即席部分諸將,攻白豹城。夜漏未盡,抵城下,四面合擊。平明,破其城,縱兵大掠,焚巢穴,獲牛馬、橐駝七千有餘,委聚方四十里,平骨咩等四十一族。以功拜龍神衛四廂都指揮使、賀州防禦使,改侍衛馬軍都虞候。



 康定二年春,朝廷欲發涇原、鄜延兩路兵西討,詔福詣涇
 原計事。會安撫副使韓琦行邊趨涇原,聞元昊謀寇渭州,琦亟趨鎮戎軍,盡出其兵,又募敢勇得萬八千人,使福將之。以耿傅參軍事,涇原路駐泊都監桑懌為先鋒,鈐轄朱觀、都監武英、涇州都監王珪各以所部從福節制。琦戒福等人並兵,自懷遠城趨得勝砦,至羊牧隆城,出敵之後。諸砦相距才四十里,道近糧餉便,度勢未可戰,則據險設伏,待其歸邀擊之。福引輕騎數千,趨懷遠城捺龍川,遇鎮戎軍西路巡檢常鼎、劉肅,與敵戰於張家
 堡南,斬首數百。夏人棄馬羊橐駝佯北,懌引騎趨之,福踵其後。諜傳敵兵少,福等頗易之。薄暮,與懌合軍屯好水川,觀、英屯龍落川,相距隔山五里,約翌日會兵川口。路既遠,芻餉不繼,士馬乏食已三日。追奔至籠竿城北,遇夏軍,循川行,出六盤山下,距羊牧隆城五里結陣,諸將方知墮敵計,勢不可留,遂前格戰。懌馳犯其鋒,福陣未成列,賊縱鐵騎突之,自辰至午,陣動,眾傅山欲據勝地。俄伏發,自山背下擊,士卒多墜崖塹,相覆壓,懌、肅戰
 死。敵分兵數千,斷官軍後,福力戰,身被十餘矢。有小校劉進者,勸福自免。福曰:「吾為大將,兵敗,以死報國爾。」揮四刃鐵簡,挺身決鬥,槍中左頰,絕其喉而死。乃人並兵攻觀、英。戰既合,王珪自羊牧隆城引兵四千,陣於觀軍之西;渭州駐泊都監趙津將瓦亭騎兵二千繼至。珪屢出略陣,陣堅不可破,英重傷,不能視軍。敵兵益至,官軍遂大潰,英、津、珪、傅皆死;內殿崇班訾贇、西頭供奉官王慶、侍禁李簡、李禹亨、劉鈞亦戰沒;軍校死者數十人,士死
 者六千餘人。唯觀以兵千餘保民垣,四響縱射,會暮,敵引去,與福戰處相距五里,然其敗不相聞也。福子懷亮亦死之。



 方元昊傾國入寇,福臨敵受命,所統皆非素撫之兵,既又分出趨利,故至於甚敗。奏至,帝震悼,贈福武勝軍節度使兼侍中,賜第一區,月給其家錢三萬,粟、麥四十斛。追封母為隴西郡太夫人,妻為瑯琊郡夫人,錄其子及從子凡六人。



 王珪,開封人也。少拳勇,善騎射,能用鐵杵、鐵鞭。年十九,
 隸親從官,累遷殿前第一班押班,擢禮賓副使、涇州駐泊都監。



 康定初,元昊寇鎮戎軍,珪將三千騎為策先鋒,自瓦亭至師子堡,敵圍之數重,



 珪奮擊披靡,獲首級為多。叩鎮戎城,請益兵,不許。城中惟縋糗糧予之。師既飽,因語其下曰:「兵法,以寡擊眾必在暮,我兵少,乘其暮擊之,可得志也。」復馳入,有驍將持白幟植槍以詈曰:「誰敢與吾敵者!」槍直珪胸而傷右臂,珪左手以杵碎其腦。繼又一將復以槍進,珪挾其槍,以鞭擊殺之。一軍大驚,
 遂引去。珪亦以馬中箭而還,仁宗特遣使撫諭之;然以其下死傷亦多,止賜名馬二匹,黃金三十兩,裹創絹百匹;復下詔暴其功塞下,以厲諸將。



 是歲,改涇原路都監。明年,為本路行營都臨,勒金字處置牌賜之,使得專誅殺。尋至黑山,焚敵族帳,獲首級、馬駝甚眾。會敵大入,以兵五千從任福屯好水川,連戰三日,諸將皆敗。任福陷圍中,望見麾幟猶在,珪欲援出之,軍校有顧望不進者,斬以徇。乃東望再拜曰:「非臣負國,臣力不能也,獨有死報
 爾。」乃復入戰,殺數十百人,鞭鐵撓曲,手掌盡裂,奮擊自若。馬中鏃,凡三易,猶馳擊殺數十人。矢中目,乃還,夜中卒。



 珪少通陰陽術數之學,始出戰,謂其家人曰:「我前後大小二十餘戰,殺敵多矣,今恐不得還。我死,可速去此,無為敵所仇也。」及敵攻瓦亭,購甚急,果如所料。鎮戎之戰,以所得二槍植山上,其後邊人即其處為立祠。贈金州觀察使,追封其妻安康郡君,錄其子光祖為西頭供奉官、閣門祗候,後為東上閣門使;光世,西頭供奉官;光
 嗣,左侍禁。



 武英字漢傑,太原人。父密,隨劉繼元歸朝,仕至侍禁、鎮定同巡檢。與契丹戰,沒於望都,贈西京左坊使,錄英為三班借職,以右班殿直為忻、代州同巡檢。會州將出獵,因留帳飲,英曰;「今空郡而來,萬一敵乘間入城,奈何?」既而敵百餘騎果入寇,英領眾左右馳射,悉禽獲之。以功遷左班殿直、監雄州榷場,改右侍禁、閣門祗候,為環州都巡檢使,徙洪德砦主,又徙慶州柔遠砦。



 元昊寇延
 州,英主兵攻後橋,以分敵勢。擢內殿承制、環慶路駐泊都監。破黨平族,又從任福破白豹城,遷禮賓副使,尋兼涇原行營都監。與任福合諸將戰張家堡,斬首數十百,敵棄羊馬偽遁。諸將皆趨利爭進,英以為前必有伏,眾不聽,已而伏發。福等既敗,英猶力戰,自辰至申,矢盡遇害。贈邢州觀察使。錄其子三班奉職永符為東頭供奉官、閣門祗候;永孚,西頭供奉官;永昌,左侍禁。侄永保,右班殿直;永錫,三班奉職。



 桑懌,開封雍丘人。勇力過人,善用劍及鐵簡,有謀略。其為人不甚長大,與人接,常祗畏若不自足,語言如不出其口,卒遇之,不知其勇且健也。兄慥,舉進士,有名。懌以再舉進士,不中。



 嘗遭大水,有粟二廩,將以舟載之,見百姓走避水者,遂棄其粟而載之,得皆不死。歲饑,聚人共食其粟,盡而止。後徙居汝、穎間,耕龍城廢田數頃以自給。



 諸縣多盜,懌自請補耆長,得往來察奸,因召里中惡少年戒曰:「盜不可為,吾不汝容也。」有頃,里老父子死未
 斂,盜夜脫其衣去,父不敢告縣。懌疑少年王生者,夜入其家,得其衣,不使之知也。明日,見而問之曰:「爾許我不為盜,今里中盜尸衣者,非爾邪?」少年色動,即推僕地,縛之,詰共盜者姓名,盡送縣,皆伏辜。



 嘗之郟城,遇尉出捕盜,招懌飲酒。與俱行,至賊所藏,尉怯甚,陽為不知,將去。懌曰:「賊在此,欲何之?」乃下馬,獨格殺數人,因盡縛之。又聞襄城有盜十許人,獨提一劍以往,殺數人,盡縛其餘,汝旁縣為之無盜。京西轉運使奏其事,補郟城尉。



 天聖
 中,河南諸縣多盜,轉運使奏移澠池尉。群盜保青灰山,時出攘剽。有宿盜王伯者,尤為民害,朝廷每授巡檢使,必疏姓名使捕之。懌至官,巡檢偽為宣頭以示懌,牒招致之。懌不知其偽也,因挺身入賊中,與伯同臥起,十餘日,伯遂與懌出至山口,為巡檢伏兵所執,懌幾不免。懌曰:「巡檢懼無功爾。」即以伯與巡檢,使自為功。巡檢俘獻京師,而懌不復自言。朝廷知之,為黜巡檢,擢懌右班殿直、永安縣巡檢。



 明道末,京西旱蝗,有惡賊二十三人,樞
 密院召懌至京師,授以賊名姓,使往捕。懌曰:「盜畏吾名,必潰,潰則難得矣,宜先示之以怯。」至則閉柵,戒軍吏不得一人輒出。居數日,軍吏不知所為,數請出自效,輒不許。夜,與數卒變為盜服以出,跡盜所嘗行處。入民家,民皆走,獨一媼留,為具飲食,如事群盜。懌歸,閉柵三日,復往,自攜具就媼饌,而以餘遺媼,媼以為真盜。乃稍就媼,與語及群盜,一媼曰:「彼聞桑殿直來,皆遁去。近聞閉營不出,知其不足畏,今皆還矣,某在某處。」懌又三日往,厚
 遺之,遂以實告曰:「我桑殿直也,為我察其實而慎勿洩。」後三日復來,於是媼盡得居處之實以告。懌明日部分軍士,盡擒諸盜。其尤強梁者,懌自馳馬以往,士卒不及從,惟四騎追之,遂與賊遇,手殺三人。凡二十三人者,一日皆獲。



 還京師,樞密吏求銀,為致閣門祗候。懌曰:「用賂得官,非我欲,況貧無銀;有,固不可也。」吏怒,匿其功狀,止免其短使而已。除兵馬監押,未行,會宜州蠻叛,殺海上巡檢,官軍不能制,因命懌往,盡手殺之。還,乃授閣門祗
 候。懌曰:「是行也,非獨吾功,位有居吾上者,吾乃其佐也。今彼留而我還,我賞厚而彼輕,得不疑我蓋其功而自伐乎?受之,徒慚吾心。」將讓其賞以歸己上者。或譏以好名,懌嘆曰:「士顧其心如何爾,當自信其心以行,若欲避名,則善皆不可為也。」益辭之,不許。



 寶元初,遷西頭供奉官、廣西駐泊都監。元昊反,參知政事宋庠薦其有勇略,遷內殿崇班、鄜延路兵馬都監。逾月,徙涇原路,屯鎮戎軍,與任福遇敵於好水川,力戰而死。贈解州防禦使;子
 湜皇城使。



 耿傅字公弼,河南人。祖昭化,為蜀州司戶參軍。盜據城,欲脅以官,昭化大罵,至斷手足,不屈而死。



 傅少喜俠尚氣,初以父蔭為三班奉職,換伊陽縣尉,歷明州司理參軍,遷將作監丞、知永寧縣。河南守宋綬薦其材,遷通判儀州,徙慶州。時議進兵西討,以傅督一道糧饋。



 會元昊入寇,參任福行營軍事,遇敵姚家川,諸將失利,敵騎益至,武英勸傅避去,傅不答。英嘆曰:「英當死,君文吏,無軍
 責,奈何與英俱死?」朱觀亦白傅少避賊鋒,而傅愈前,指顧自若,被數創,乃死。



 始,傅與觀營籠落川,夜作書遺福,以其日小勝,前與敵大軍遇,深以持重戒之。自寫題觀名,以致福軍中。傅死後,韓琦得其書於隨軍孔目官彭忠,奏上之。詔贈傅右諫議大夫,官其子瑗為太常寺太祝,琚為太常寺奉禮郎,璋為將作監主簿,珪試秘書省校書郎,琬同學究出身。



 王仲寶字器之,密州高密人。初為刑部史,補齊州章邱
 尉。以捕群盜六十餘人有功,用開封府判官鞠仲謀薦,召對,改右班殿直,為鎮、定、保、深、永寧、天雄六州軍巡檢。又以捕賊功,遷左班,徙河北西路提舉捉賊,擒磁州名賊王遇仙、博州孫流油輩,凡四十人。



 夜有盜叩戶外乞降,左右欲殺之,為首級論功,仲寶不可,納舍中使寢。擢閣門祗候,命乘驛捕登州海賊百餘人,獲之。還,為河北提舉捉賊,又捕斬百餘人。知信安軍,復為河北提舉捉賊。有盜百餘依西山,官軍不能捕,仲寶悉招出,隸軍籍,
 奏以自隨。徙澤、潞、晉、絳、慈、隰、威勝軍巡檢使,至官才八日,獲太行山宿賊八十人。累賜金帛、緡錢。使契丹,積遷內殿承制。



 天聖初,知鎮戎軍,改供備庫副使。破康奴族,獲首領百五十、羊馬七千,詔獎其功。凡五年,還,巡護惠民河堤岸,遷供備庫使、麟府路兵馬鈐轄、知麟州。會鎮戎軍蕃族內寇,徙涇原路鈐轄,復知鎮戎軍,又徙原、環二州。以西京左藏庫使、惠州刺史知利州,徙並、代州鈐轄,改西上閣門使。建言:「緣邊博糴,屬羌苦之,數逃去。請寬
 其法,使得復業,以捍邊境。」久之,遷東上閣門使。



 元昊寇延州,仲寶將兵至賀蘭谷,以分兵勢,敗蕃將羅逋於長雞嶺。遷四方館使,領濮州團練使,為涇原路總管、安撫副使兼管勾秦鳳路軍馬事。與西羌戰六盤山,俘馘數百人。



 時任福大敗好水川,別將朱觀被圍於姚家堡,仲寶以兵救之,拔觀出圍,乘以從馬。時諸將皆沒,獨仲寶與觀得還。徙環慶路副都總管、知慶州。未幾,兼本路經略安撫、招討副使。破金湯城,復賜詔獎諭,徙澶州副總
 管。安撫使範仲淹以仲寶武幹未衰,奏留之。明年以磁州防禦使知代州,除左屯衛大將軍致仕,卒。



 論曰:元昊乘中國弛備,悉眾寇邊,王師大衄者三,夫豈天時不利哉?亦人謀而已。好水之敗,諸將力戰以死。噫,趨利以違節度,固失計矣;然秉義不屈,庶幾烈士者哉!



\end{pinyinscope}