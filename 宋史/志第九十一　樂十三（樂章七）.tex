\article{志第九十一 樂十三(樂章七)}

\begin{pinyinscope}

 朝會御樓肆赦恭上皇帝皇太后尊號上



 建隆乾德朝會樂章二十八首



 皇帝升坐,《隆安》



 天臨有赫,上法乾元。鏗鏘六樂,儼恪
 千官。



 皇儀允肅,玉坐居尊。文明在御,禮備誠存。



 公卿入門,《正安》



 堯天協紀,舜日揚光。涉慎爾止,率由舊章。



 佩環濟濟,金石鏘鏘。威儀炳煥,至德昭彰。



 上壽,《禧安》



 乾健為君,坤柔曰臣。惟其臣子,克奉君親。



 永御皇極,以綏兆民。稱觴獻壽,山嶽嶙峋。



 舜《韶》更奏,堯酒浮觴。皇情載懌,洪算無疆。



 基隆郟鄏,德茂陶唐。山巍日煥,地久天長。



 皇帝舉酒,第一盞用《白龜》



 聖德昭宣,神龜出焉。載白
 其色,或游於川。



 名符在沼,瑞應巢蓮。登歌丹陛,紀異靈篇。



 第二盞,《甘露》



 天德冥應,仁澤載濡。其甘如醴,其凝如珠。



 雲表潛結,顥英允敷。降於竹柏,永昭瑞圖。



 第三盞,《紫芝》



 煌煌茂英,不根而生。蒲茸奪色,銅池著名。



 晨敷表異,三秀分榮。書於瑞典,光我文明。



 第四盞,《嘉禾》



 嘉彼合穎,致貢升平。異標南畝,瑞應西成。



 德至於地,皇祗效靈。和同之象,煥發祥經。



 第五盞,《玉兔》



 盛德好生,網開三面。明視標奇,昌辰乃見。



 育質雪園,淪精月殿。著於樂章,色含江練。



 群臣舉酒,《正安》



 戶牖嚴丹扆,鵷鸞造紫庭。懇祈南嶽壽,勢拱北辰星。



 得士於茲盛,基邦固以寧。誠明一何至,金石與丹青。



 簪紱若雲屯,晨趨閶闔門。侁□羅禹會,濟濟奉堯樽。



 周禮觀明備,天儀仰晬溫。高卑陳表著,同拱帝王尊。



 待漏造王庭,威儀盛莫京。紛綸簪組列,清越佩環聲。



 禮飲終三爵,《韶》音畢九成。永固鳧藻樂,千載奉升平。



 群臣第一盞畢,作《玄德升聞》



 治定資神武,功成顯睿文。貢輪庭實旅,朝會羽儀分。



 偃革千年運,垂衣萬乘君。孰知堯、舜力,明德自升聞。



 約法皇綱正,崇文寶歷昌。遒人振木鐸,農器鑄干將。



 瑞日含王宇,卿雲藹帝鄉。萬邦成一統,鴻祚與天長。



 六變



 宸扆威容盛,聲明禮樂宣。九州臻禹會,萬國戴堯天。



 貢職輸琛贐。皇猷煥簡編。含和均暢茂,鴻慶結
 非禋。



 朝會儼威儀,司常建九旗。舞容分綴兆,文物辨威蕤。



 運格桃林牧,祥開洛水龜。帝功潛日用,化俗自登熙。



 螭階聊載筆,紀瑞軼唐、虞。丹鳳儀金奏,黃龍負寶圖。



 群材薪棫樸,仁政煦蒲盧。蕩蕩巍巍德,豚魚信自孚。



 接聖宅神都,方來五達區。國賢熙帝載,靈命握乾符。



 至化當純被,斯文益誕敷。車書今混一,聖治奉三無。



 聖王臨大寶,八表湊才賢。經緯文天賦,剛柔德日宣。



 建邦隆柱石,造物運陶甄。共致升平業,綿長保億年。



 神化妙無方,巍巍邁百王。鶴書搜隱逸,龍陛策賢良。



 拱揖朝群後,賓筵闢四方。洪圖基億載,淳曜備彌光。



 第二盞畢,《天下大定》



 皇猷敷八表,武誼肅三邊。蘭錡韜兵日,靈臺偃伯年。



 奉珍皆述職,削衽盡朝天。功德超前古,音徽播管弦。



 伐叛天威震,恢疆帝業多。削平侔肅殺,涵煦極陽和。



 蹈厲觀周舞,風雲入漢歌。功成推大定,歸馬偃琱戈。



 六變



 惕厲日幹幹,潛蟠或躍淵。伐謀參上策,受鉞總中堅。



 田訟歸周日,民謠戴舜年。風雲自冥感,嘉會翼飛天。



 壺關方逆命,投袂起親征。虎旅聊攻伐,梟巢遽蕩平。



 天威清朔漠,仁澤被黎氓。按節皇輿復,洋洋載頌聲。



 蠢茲淮海帥,保據毒黎苗,不悟龍興漢,猶同犬吠堯。



 六師方雨施,孤壘自冰消。千載逢嘉運,華夷奉聖朝。



 上游荊楚要,澤國洞庭深。自識同文世,皆回拱極心。



 一戎聊杖鉞,九土盡輸金。大定功成後,熏風入舜琴。



 席卷定巴、邛,西遐盡率從。岷、峨難負阻,江、漢自朝宗。



 述職方舟集,驅車九折通。粲然書國史,冠古耀豐功。



 銳旅慶回旋,邊防盡晏然。鍵櫜方偃武,飛將亦韜弦。



 震曜資平壘,文明協麗天。洸洸成大業,赫奕在青編。



 淳化中朝會二十三首



 上壽,《和安》



 四序伊始,三陽肇開。條風入律,玉管飛灰。



 望雲肅謁,鳴佩斯來。稱觴獻壽,瞻拱星回。



 一陽應候,萬國同文。天正紀節,太史書云。



 凝旒在御,列敘爰分。壽觴斯薦,祝慶明君。



 皇帝初舉酒,用《祥麟》



 聖皇御寓,仁獸誕彰。在郊旅貢,游畤呈祥。



 星辰是稟,草木無傷。紀異信史,登歌太常。



 再舉酒,《丹鳳》



 九苞薦瑞,戴德膺仁。藻翰爰奮,靈音載振。



 非時不見,有道則臻。降岐匪匹,儀舜為鄰。



 三舉酒,《河清》



 沔彼涇瀆,澄明鑒如。清應寶運,光涵帝居。



 洞分沉璧,徹見游魚。聖祚無極,神休偉與。



 四舉酒,《白龜》



 稽彼靈物,允昭聖皇。浮石可躡,巢蓮益光。



 金方正色,介族殊祥。信書永耀,帝德無疆。



 五舉酒,《瑞麥》



 芃芃嘉麥,擢秀分岐。甘露夕灑,惠風晨吹。



 良農告瑞,循吏稱奇。歸美英主,折而貢之。



 群臣初舉酒畢,作《化成天下》



 軒、昊方同德,成、康粗比肩。素風惟普暢,皇道本無偏。



 陰魄重輪滿,陽精五色圓。要荒咸率服,卓越聖功全。



 聖德比陶唐,千年祚運昌。茂功雖不宰,鴻業自無疆。



 極塞成清謐,齊民益阜康。文明同日月,遐邇仰輝光。



 六變



 蕩蕩無私世,巍巍至聖君。山河分國寶,日月耀人文。



 厭浥凝甘露,輪囷吐慶雲。正聲兼《大雅》,洋溢應南熏。



 鴻範合彞倫,調元四序均。歲功天吏正,御苑物華新。



 底貢陳方物,來賓列遠人。奉常呈九奏,嘉貺動穹旻。



 大君隆至化,興運契千齡。覲禮俄班瑞,夷賨盡實庭。



 成文調露樂,奉聖拱辰星。舞佾方更進,朝陽上楚萍。



 禮樂昭王業,寰區致太平。革車停北狩,雲稼屢西成。



 國有詳延詔,鄉聞講誦聲。日華融五色,遐邇仰文明。



 亭障戢干戈,人心浹太和。務農登寶穀,獵俊設雲羅。



 儀鳳書良史,祥麟載雅歌。嘉辰資宴喜,星拱弁峨峨。



 冠古耀鴻徽,深仁及隱微。《二南》、《江漢》詠,九奏鳳凰飛。



 設虡羅鐘律,盈庭列舞衣。文明資厚德,怡懌兆民歸。



 再舉酒畢,《威加海內》



 革輅征汾、晉,隳城比燎毛。桓桓勖軍旅,將將御英豪。



 神武誠無敵,天威詎可逃。王師
 宣利澤,霈若沃春膏。



 振萬方明德,疾徐咸可觀。鏗鏘動金奏,蹈厲總朱干。



 夾進昭威武,申嚴警宴安。守方推猛士,當用冒□為冠。



 六變



 宣榭始觀兵,桓桓稱鼓行。一戎期大定,載纘議徂征。



 善政從師律,神功冀《武成》。勖載勤誓眾,王業自經營。



 聲教方柔遠,甌、閩禮可招。獻圖連日際,歸國象江潮。



 撫運重熙盛,提封萬里遙。還同有虞氏,文德格三苗。



 南暨宣皇化,東吳奉乃神。舞干方耀德,執玉自來賓。



 巢伯朝丹陛,韓侯覲紫宸。古今歸一揆,懷遠道彌新。



 遺俗續陶唐,來蘇徯聖皇。布昭湯吊伐,恢復漢封疆。



 金鉞申戡剪,朱干示發揚。宜哉七德頌,千載播洋洋。



 乃眷嘗西顧,偏師暫首征。靈旗方直指,獷俗自亡精。



 禹敘終馴致,堯封漸化成。不須嚴尉候,於廓海彌清。



 干戚有司傳,威容著凱旋。像成王業盛,役輟武功全。



 兵寢西郊閱,書惟北闕縣。聖神膺景命,卜世萬斯年。



 景德中朝會一十四首



 皇帝升坐,《隆安》



 金奏在庭,群后在位。天威煌煌,響明負扆。



 高拱穆清,弁冕端委。盛德日新,禮容有煒。



 公卿入門,《正安》



 萬邦來同,九賓在位。奉璋薦紳,陟降庭止。



 文思安安,威儀棣棣。臣哉鄰哉,介爾蕃祉。



 上壽,《和安》



 天威煌煌,山龍採章。庭實旅百,上公奉觴。



 拱揖群後,端委垂裳。永錫難老,萬壽無疆。



 皇帝初舉酒,《祥麟》



 帝圖會昌,二獸效祥。雙角共抵,示
 武不傷。



 四靈為畜,玄枵耀芒。公族信厚,元元阜康。



 再舉酒,《丹鳳》



 矯矯長離,振羽來儀。和音中律,藻翰揚輝。



 珍符沓至,品物攸宜。至德玄感,受天之祺。



 三舉酒,《河清》



 德水湯湯,發源靈長。皎鑒澄徹,千年效祥。



 積厚流濕,資生阜昌。朝宗潤下,善利無疆。



 群臣舉酒,《正安》



 思皇多士,靖恭著位。鳴玉飛緌,鏘鏘濟濟。



 宴有折俎,以示慈惠。罔敢不祗,福祿來暨。



 金奏在庭,有酒斯旨。顒顒仰仰,響明負扆。



 湛湛露斯,
 式宴以喜。佩玉蕊兮,罔不由禮。



 酒以成禮,樂以侑食。露湛朝陽,星環紫極。



 涉慎爾容,既飽以德。進退周旋,威儀抑抑。



 初舉酒畢,《盛德升聞》



 八佾具呈,萬舞有奕。既以象功,又以觀德。



 進旅退旅,執鑰秉翟。至化懷柔,遠人來格。



 閶闔天開,群后在位。設業設虡,庭燎晰晰。



 斧扆當陽,虎賁夾陛。舞之蹈之,四隩來暨。



 再舉酒畢,《天下大定》



 武功既成,綴兆有翼。以節八音,
 以象七德。



 俁俁蹲蹲,朱干玉戚。發揚蹈厲,其儀不忒。



 偃伯靈臺,功成作樂。以昭德容,以清戎索。



 萬邦會同,邪匿銷鑠。盡善盡美,侔彼《韶箾》。



 降坐,《隆安》



 被袞當陽,穆穆皇皇。擊石拊石,頌聲揚揚。



 和樂優洽,終然允臧。禮成而退,荷天百祥。



 大中祥符朝會五首



 皇帝舉酒,《醴泉》



 觱沸檻泉,寒流清泚。地不愛寶,其旨如醴。



 上善至柔,靈休所啟。利澤無疆,允資岱禮。



 再舉酒,《神芝》



 彼茁者芝,茂英煌煌。敷秀喬嶽,寔繁其房,



 適符修貢,封巒允臧。永言登薦,抑惟舊章。



 三舉酒,《慶雲》



 惟帝祐德,卿雲發祥。紛紛鬱鬱,五色成章。



 奉日逾麗,回風載翔。歌薦郊廟,播厥無疆。



 四舉酒,《靈鶴》



 玄文申錫,嘉祥紹至。偉茲胎禽,羽族之異。



 翻翰來儀,徘徊嘹唳。祚聖儲休,韋昭天意。



 五舉酒,《瑞木》



 天生五材,木曰曲直。維帝順天,厚其生植。



 連理效祥,成文表德。總萃坤珍,永光秘刻。



 熙寧中朝會三首



 皇帝初舉酒,《慶雲》



 乾坤順夷,皇有嘉德。爰施慶雲,承日五色。



 輪囷下乘,萬物皆飾。惟天祚休,長彼無極。



 再舉酒,《嘉禾》



 彼美嘉禾,一莖九穗。農疇告祥,史牒書瑞。



 擊壤歡歌,如京委積。留獻春種,昭錫善類。



 三舉酒,《靈芝》



 皇仁溥博,品物蕃滋。慶祥回復,秀發神芝。



 靈華雙舉,連葉四施。披圖按牒,永享純禧。



 元符大朝會三首



 皇帝初舉酒,《靈芝》



 嘉瑞降臨,應我皇德。燁燁神芝,不根而植。



 春秋三秀,晝夜一色。物播詩歌,聲被金石。



 再舉酒,《壽星》



 倬彼星象,於昭於天。維南有極,離丙之躔。



 既明且大,應聖乘乾。誕受景福,億萬斯年。



 三舉酒,《甘露》



 泫泫零露,雲英醴溢。和氣凝津,流甘委白。



 飴泛泮林,珠聯竹柏。天不愛道,聖功允格。



 哲宗傳受國寶三首,與大朝會兼用



 《永昌》



 於穆我王,繼序不忘。明昭上帝,上帝是皇。



 長發
 其祥,惠我無疆。受命於天,既壽永昌。



 《神光》



 惟皇上德,伊嘏我王。將受厥明,載錫之光。



 於昭於天,曄曄煌煌。緝熙欽止,其永無疆。



 《翔鶴》



 彼鳴在陰,亦白其羽。聲聞於天,來集斯所。



 勉勉我王,咸遂厥宇。播於異物,受天多祜。



 紹興朝會十三首



 皇帝升坐,《乾安》



 鉤陳肅列,金奏充庭。顒仰南面,如日之升。



 垂衣拱手,治無能名。順履獻歲,大安大榮。



 公卿入門,《正安》升降同。



 天子當陽,臣工率職。流水朝宗,眾星拱極。



 環佩鏘鏘,威儀抑抑。上下交欣,同心同德。



 上公上壽,《和安》



 八音克諧,萬舞有奕。上公奉觴,率茲百闢。



 聲效呼嵩,祝聖人壽。億載萬年,天長地久。



 皇帝初舉酒,《瑞木成文》



 厚地效珍,嘉森紀瑞。匪刻匪雕,具文見意。



 三登太平,允協聖治。《詩雅》詠歌,有光既醉。



 再舉酒,《滄海澄清》



 百穀王,符聖治。不揚波,效殊祉。



 德
 淪淵,滄海清。應千秋,敘五行。



 三舉酒,《瑞粟呈祥》



 至治發聞惟馨香,播厥百穀臻穰穰。農夫之慶歲其有,禾易長畝盈倉箱。



 時和物阜粟滋茂,嘉生駢穗來呈祥。自今以始大豐美,行旅不用繼餱糧。



 群臣酒行,《正安》



 群公卿士,咸造在庭。式燕以衎,思均露零。



 穆穆明明,於斯為盛。歸美報上,一人有慶。



 明明天子,萬福來同。嘉賓式燕,曷不肅雍。



 燕以示慈。
 式禮莫愆。樂胥君子,容止可觀。



 酒一行,文舞



 帝德誕敷,銷爍群慝。近悅遠來,惟聖時克。



 玉振金聲,治功興起。《韶箾》象之,盡善盡美。



 文物以紀,藻色以明。禮備樂舉,遹觀厥成。



 睿知有臨,誕敷文德。教雨化風,洽此四國。



 酒載行,武舞



 用戒不虞,誰能去兵。師出以律,動必有名。



 拆彼遐沖,布昭聖武。和眾安民,時惟多助。



 止戈曰武,惟聖為能。禦得其道,無敢不庭。



 整我六
 師,稽諸七德。不吳不揚,有嚴有翼。



 皇帝降坐,《乾安》



 帝坐熒煌,廷紳肅穆。對揚天休,各恭爾服。



 頌聲洋洋,彌文鬱鬱。禮備樂成,永膺多福。



 建隆御樓三首



 南郊回仗,駕至樓前,《採茨》



 高煙升太一,明祀達乾坤。天仗回嶢闕,皇輿入應門。



 簪裳如霧集,車騎若雲屯。兆庶皆翹首,巍巍萬乘尊。



 升坐,《隆安》



 禋祀畢圓丘,嘉辰慶澤流。天儀臨觀魏,盛
 禮藹風猷。



 洋溢歡聲動,氛氳瑞氣浮。上穹垂眷祐,邦國擁鴻休。



 降坐,《隆安》



 華纓就列,左衽來王。帝儀炳煥,大樂鏗鏘。



 禮成嶢闕,言旋未央。一人有慶,萬壽無疆。



 咸平御樓四首



 《採茨》



 禮成於郊,迎日之至。時乘六龍,天旋象魏。



 端門九重,虎賁萬騎。四夷來王,群后輯瑞。



 索扇,《隆安》



 應門有翼,羽衛斯陳。山龍袞冕,律度聲身。



 峨峨奉璋,肅肅九賓。清明在躬,志氣如神。



 升坐,《隆安》



 圜丘類上帝,六變降天神。禋燔禮云畢,仗衛肅以陳。



 天顏瞻咫尺,王澤熙陽春。玉帛臻禹會,動植沾堯仁。



 降坐,《隆安》



 肆眚雲畢,淳熙溥將。雷雨麗澤,雲物效祥。



 禮容濟濟,天威皇皇。大賚四海,富壽無疆。



 咸平籍田回仗禦樓二首



 《採茨》



 農皇既祀,禮畢躬耕。商輅旋軫,周頌騰聲。



 觀魏
 將陟,服御爰更。輿人瞻仰,如日之明。



 升坐,《隆安》



 應門斯禦,雉扇爰開。人瞻日月,澤動雲雷。



 同風三代,均禧九垓。歡心允洽,時詠康哉。



 乾興御樓二首



 升坐,《隆安》



 夾鐘紀月,初吉在辰。眚災流慶,布德推仁。



 採章震耀,典禮具陳。茂昭丕貺,永庇斯民。



 降坐,《隆安》



 皇衢赫敞,黼坐穹崇。華纓在列,嚴令發中。



 王制鉅麗,寶瑞豐融。均禧綿寓,萬壽無窮。



 紹興登門肆赦二首



 升坐,《乾安》



 拜況於郊,皇哉唐哉!熙事休成,六騑鼎來。



 天閫以決,地垠以開。隤祉發祥,如登春臺。



 降坐,《乾安》



 鴻霈普洽,言歸端門。蕩蕩巍巍,旋乾轉坤。



 穆然宣室,儲思垂恩。於萬斯年,敷錫群元。



 寧宗登門肆赦二首



 升坐,《乾安》



 帝饗於郊,荷天之休。五福敷錫,皇明燭幽。



 雲行雨施,仁翔德游。聖人多男,歌頌九州。



 降坐,《乾安》



 天日清晏,朝野靖安。三靈答祉,萬國騰歡。



 帝命不違,王業艱難。天子萬年,永迪監觀。



 皇帝上尊號一首



 冊寶入門,《正安》



 於穆元後,天臨紫宸。飛緌星拱,建羽林芬。



 徽冊是奉,鴻名愈新。荷茲介祉,永永無垠。



 明道元年章獻明肅皇太后朝會十五首



 皇太后升坐,《聖安》



 聖母有子,重光類禋。聖皇事母,感極天人。



 百闢在庭,九儀具陳。禮容之盛,萬國咸賓。



 公卿入門,《禮安》



 帝率四海,承顏盡恭。端闈肅設,群後來同。



 玉佩鏘鳴,衣冠有容。《英》、《韶》節步,磬管雍雍。



 皇帝上壽酒,《崇安》



 天子之德,形於四方。尊親立愛,化洽風揚。



 聖母禕衣,明君黼裳。因時獻壽,克盛朝章。



 上壽,《福安》



 盛禮煌煌,六衣有光。千官在位,百福稱觴。



 坤備慈仁,邦斯淑祥。如山之壽,祐聖無疆。



 皇太后初舉酒,《玉芝》



 燁燁靈芝,生於殿闈。照映華拱,紛敷玉蕤。



 感召元和,光符聖期。祥篇協吉,百福咸宜。



 再舉酒,《壽星》



 現彼南極,昭然瑞文。騰光丙位,薦壽中宸。



 太史駢奏,升歌有聞。軒宮就養,億萬斯春。



 三舉酒,《奇木連理》



 王化無外,坤珍效靈。旁枝內附,直幹來並。



 群分非一,祺祥紹登。至誠攸感,海縣斯寧。



 群臣酒行,《禮安》



 肅肅臨下,有威有容。循循事上,惟信惟忠。



 盛禮興樂,示慈訓恭。君臣協吉,惟道之從。



 湛湛零露,晞於載陽。我有旨酒,群臣樂康。



 既飲以德,亦圖爾良。永言修輔,用協天常。



 禮均孝慈,樂合《韶》、《武》。至德光矣,



 鴻恩亦溥。上下和濟,華夷樂湑。盞斝三行,盛儀斯舉。



 酒一行畢,作《厚德無疆》之舞



 堯母之聖,放勛為子。同心協謀,柔遠能邇。



 以德康俗,以文興治。斯焉象功,罔不昭濟。



 至矣坤元,道符惟聖。就養宸極,助隆善政。



 翟鑰紛舉,笙鏞協應。翱翔有容,表德之盛。



 酒再行,《四海會同》之舞



 七德之舞,四朝用康。有如姬、
 姒,助集周邦。



 威克厥愛,居安不忘。風旋山立,濟濟皇皇。



 左秉朱干,右揮玉戚。以象武綴,以明皇德。



 天子榮養,群臣述職。四夷賓附,罔不承式。



 降坐,《聖安》



 長樂居尊,盛容有煒。文王事親,萬國歸美。



 朝會之則,邦家之紀。受福於天,克昭隆禮。



 治平皇太后、皇后冊寶三首



 皇帝升坐,《乾安》



 王化之始,治由內孚。時庸作命,玉簡
 金書。



 磬筦在庭,其縱繹如。天臨法扆,禮與誠俱。



 太尉等奉冊寶入門,《正安》



 晬儀臨拱,丕命明。鸞回寶勢。鴻貫瑤光。



 禮成樂備,德裕名芳。肇基王化,永懋天祥。



 皇帝降坐,《乾安》



 袞衣繡裳,嚴威肅莊。八音具張,簨虡龍驤。



 玉簡瑤章,金書煌煌。壽千萬年,與天比長。



 熙寧皇太后冊寶三首



 出入,《正安》



 煌煌鳳字,玉氣宛延。天門崛岉,飛驂後先。



 龍簨四合,奏鼓淵淵。母儀天下,何千萬年。



 升坐,《乾安》



 峨峨繡扆,旋佩以登。如彼杲日,凌天而升。



 玉色下照,亹亹繩繩。猗歟大孝,四海其承!



 降坐,《乾安》



 皇帝降席,流雲四開。堯趨舜步,下躡天階。



 恭授寶冊,翠旄裴回。明明純孝,鴻厘大來。



 哲宗上太皇太后冊寶五首



 皇帝升坐,《乾安》



 大矣孝熙,帥民以躬!奉承寶冊,欽明兩宮。



 萬樂具舉,一人肅雍。化由上始,四海來同。



 降坐,《乾安》



 皇帝仁孝,總臨萬方。褒顯其親,日嚴以莊。



 龍袞翼翼,玉書煌煌。傳之億世,休有烈光。



 太皇太后升坐,《乾安》



 總裁庶政,擁祐嗣皇。金書玉簡,爛其文章。



 眾歌警作,筦磬將將。保安四極,降福無疆。



 降坐,《乾安》



 塗山之德,渭涘之祥。圖徽寶冊,玉色金相。



 管弦燁煜,鐘鼓喤□。天之所啟,既壽而昌。



 太尉等奉冊寶出入門,《正安》



 玉車臨御,鳳蓋棽麗。奉承寶冊,彌文盛儀。



 抗聲極律,助我孝熙。天之所祐,萬
 壽無期。



 紹興十年發皇太后冊寶八首



 皇帝隨冊寶降殿,《聖安》



 景祚有開,符天媲昊。誕毓聖神,是崇位號。



 星拱天隨。祗嚴冊寶。還御慈寧,增光舜道。



 中書令奉冊詣皇帝褥位,《禮安》



 聲樂備陳,禮容罔忒。相維闢公,虔奉玉冊。



 皇則受之,慕形於色。即壽且康,與天無極。



 侍中奉寶詣皇帝褥位,《禮安》



 祖啟瑤光,誕生明聖。尊極母儀,帝康作命。



 寶章煌煌,導以笙磬。還燕慈寧,邦家徯慶。



 太傅奉冊寶出門,《聖安》



 肅肅東朝,帝隆孝治。猗歟丕稱,寶冊斯備!



 皇扉四闢,導迎慶瑞。德邁大任,有周卜世。



 太傅奉冊寶入門,《聖安》



 靜順坤儀,聖神是育。懿鑠昭陳,鏤文華玉。



 樂奏既備,禮儀不瀆。導迎善祥,翟車歸
 毣。



 太傅奉冊授提點官,《禮安》



 孝奉天儀,信維休德。發越徽音,禮文靡忒。



 永保嘉祥,時萬時億。歸於東朝,含飴燕息。



 太傅奉冊授提點官,《禮安》



 肅雍長樂,克篤其慶。河洲茂德,沙麓啟聖。



 是生睿哲,蚤隆丕運。欽稱鴻寶,永膺天命。



 冊寶升慈寧殿幄,《聖安》



 禮行東朝,樂奏大呂。羽衛森
 陳,簪紳式序。



 雲幄邃嚴,宏典是舉。天子萬年,母儀寰宇。



 乾道七年恭上太上皇帝、太上皇后尊號十一首



 冊寶降殿,《正安》



 元祀介福,孰綏孰將。歸於尊親,孝哉君王!



 載鏤斯牒,載琢斯章。得名得壽,如虞如唐。



 中書令、侍中奉冊寶詣殿下,《正安》



 宗郊斯成,交舉典冊。汝輔汝弼,威儀是力。



 陳於廣庭,迨此上日。巍巍煌煌,烏睹在昔。



 皇帝奉太上皇帝冊寶授太傅,用《禮安》奉太上皇后同。



 儀物陳矣,禮樂明矣。天子戾止,詒爾臣矣。



 陟降維則,恭且勤矣。茫茫四海,德教形矣。



 冊寶出門,《正安》



 天門九重,蕩蕩開徹。金支秀華,垂紳佩玦,



 或導或陪,率履不越。注民耳目,四表胥悅。



 冊寶入德壽宮門,《正安》



 禮神頌祗,福祿來下。不有榮名,孰緝伊嘏。



 千乘萬騎,魚魚雅雅。皇扉洞開,鞠躬如也。



 太上皇帝升御坐降同。



 穆穆聖顏,安安天步。有縟者儀,以莫不舉。



 天人和同,恩澤洋普。億載萬年,為眾父父。



 太傅奉太上皇帝冊寶升殿,用《聖安》



 大哉堯乎,南響垂裳!君向舜也,拜而奉觴!



 繅藉光華,鼓鐘鏗鏘。三事稽首,宋德無疆。



 太傅奉太上皇后冊寶升殿,用《聖安》



 乾元資始,坤元資生。允也聖德,同實異名。



 春王三朝,典冊並行。咨爾上公,相儀以登。



 皇帝從太上皇后冊寶詣宮中,用《正安》



 維冊伊何?鏤玉垂鴻。維寶伊何?範金鈕龍。



 翊以御,間以笙鏞。誰敢不恭,天子實從!



 太上皇后出閣升御坐,《坤安》降同。



 帝膺永福,功靡專有。既尊聖父,亦燕壽母。



 怡怡在宮,大典時受。彤管紀之,天長地久。



 內侍官舉太上皇后冊詣讀冊位,用《聖安》



 斂福於郊,逢時之泰。揭名日月,侔德覆載。



 自我作古,域中有大。
 永言保之,眉壽無害。



 淳熙二年發太上皇帝、太上皇后冊寶十一首



 冊寶降殿,《正安》



 高明者乾,博厚者坤。以清以寧,資始資生。



 壽胡可度,德胡可評!願言從欲,誕受強名。



 中書令、侍中奉冊寶詣殿下,《正安》



 受命既長,福祿即康。如日之升,如月之常。



 追琢其章,金玉其相。君子萬年,保其家邦。



 皇帝奉太上皇帝冊寶授太傅,《禮安》奉太上皇后同。



 翠華之
 旗,靈鼉之鼓。陳於廣宇,相我盛舉。



 來汝公傅,肅乃儀矩。毋愆於素,以篤多祜。



 冊寶出門,《正安》



 蚴蟉青龍,婉嬗象輿。其載伊何?煌煌金書。



 乃由端門,乃行康衢。於以榮親,振古所無。



 冊寶入德壽宮門,《正安》



 惟天為大,其德曰誠。惟堯則之,其性曰仁。



 乃文乃武,得壽得名。於萬斯年,以莫不增。



 太上皇帝升御坐,《乾安》降同。



 天行惟健,天步惟安。聖子
 中立,臣工四環。



 民無能名,威不違顏。宋德宜頌,漢儀可刪。



 太傅奉太上皇帝冊升殿,《聖安》奉寶同。



 天畀遐福,允彰父慈。維昔曠典,我能舉之。



 徐爾陟降,敬爾威儀。申錫無疆,永言保之。



 太傅奉太上皇后冊寶升殿,《聖安》



 乾健坤從,陽剛陰相。迨茲受祉,允也並況。



 虡業在下,儀物在上。咨時三公,執事無曠。



 皇帝從太上皇后冊寶詣宮中,用《正安》



 丕顯文王,之德之純。亦有太姒,式揚徽音。



 維冊維寶,乃玉乃金。伊誰從之?一人事親。



 太上皇后出閣升御坐,《坤安》降同。



 重翟出房,禕衣被躬。委委佗佗,河潤山容。



 聖皇臨軒,聖母在宮。並受鴻名,與天無窮。



 內侍官舉太上皇后冊詣讀冊位,用《聖安》舉寶同。



 鈱玉玢豳,褭蹄精良。既刻厥文,亦鑄之章。



 象德維何?至靜
 而方。輔我光堯,萬壽無疆。



 淳熙十二年加上太上皇帝、太上皇后尊號十一首



 大慶殿發冊寶降殿,《正安》



 維天蓋高,維地克承。父尊母親,天地難名。



 疆名廣大,建號安榮。衍登壽嘏,闡繹皇明。



 中書令、侍中奉太上皇帝冊寶、太上皇后冊寶詣殿下,用《正安》



 二儀同尊,兩耀齊光。巍巍煌煌,不顯亦彰。



 實茂號榮,玉振金相。於萬斯年,既壽且昌。



 皇帝奉太上皇帝冊寶授太傅太上皇后冊寶同。



 我尊我親,承天之祉。壽名兼美,家國咸喜。



 公傅秉禮,寶冊有煒。惟千萬祀,令聞不已。



 冊寶出門,《正安》



 羽衛有嚴,寶書有輝。昭衍尊名,鋪張上儀。



 出其端闈,由於康逵。比屋延瞻,歌之舞之。



 德壽宮冊寶入殿門,《正安》



 南山之鞏,皇壽無窮。太極之尊,皇名是崇。



 奉茲寶冊,於皇之宮。皇則受之,於昭盛容。



 太上皇帝出宮升御坐,《乾安》降坐同。



 聖明太上,天子有尊。玉坐高拱,慈顏睟溫。



 震禁嘉承,朝弁昈分。盛禮縟典,邃古未聞。



 太傅、中書令、侍中奉太上皇帝冊寶升殿,用《聖安》



 天錫伊嘏,地效其珍。誕作寶典,奉於尊親。



 爾公爾相,爾恭爾寅協舉令儀,遹臻厥成。



 太傅、中書令、侍中奉太上皇后冊寶升殿,用《聖安》



 坤載有元,乾行是順。施生萬匯,厥德彌盛。



 翼翼母道,贊
 我皇訓。相維群公,奉典斯敬。



 皇帝從太上皇后冊寶詣宮中,用《正安》



 大矣母慈,德備且純!思古齊敬,佐我皇文。



 明章茂典,金玉其音。帝親奉之,以翼以欽。



 太上皇后出閣升御坐,用《坤安》降坐同。



 天相慈皇,慶臻壺闈。徽柔內修,壽與天齊。



 既承皇歡,載覿母儀。懿典鴻名,永綏多祺。



 內侍舉太上皇后冊寶詣讀冊寶位,用《聖安》



 有美英
 瑤,於昭祥金。為策為章,並著徽音。



 德聖而尊,備舉彌文。億載萬年,永輔堯勛。



\end{pinyinscope}