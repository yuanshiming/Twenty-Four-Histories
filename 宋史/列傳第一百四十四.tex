\article{列傳第一百四十四}

\begin{pinyinscope}

 葛邲
 錢端禮魏杞周葵施師點蕭燧龔茂良



 葛邲,字楚輔,其先居丹陽,後徙吳興。世以儒學名家,高祖密邲五世登科第,大父勝仲至邲三世掌詞命。邲
 少警敏,葉夢得、陳與義一見稱為國器。



 以蔭授建康府上元丞。會金人犯江,上元當敵沖,調度百出,邲不擾而辦,留守張浚、王綸皆器重之。登進士第。蕭之敏為御史,薦其才,除國子博士。輪對,論州縣受納及鬻爵之弊,孝宗獎諭曰:「觀所奏,知卿材。」除著作郎兼學士院權直。



 除正言,首疏言:「盈虛之理,隱於未然;治亂之分,生於所忽。宜專以畏天愛民為先。」又論:「征榷歲增之害,如輦下都稅務,紹興間所趁茶鹽歲以一千三百萬緡為額,乾道
 六年後增至二千四百萬緡。成都府一務,初額四萬八千緡,今至四十餘萬緡,通四川酒額遂至五百餘萬緡,民力重困。至若租稅有定數,而暗耗日增,折帛益多,民安得不窮乎?願明詔有司,茶鹽酒稅比原額已增至一倍者,毋更立新額,官吏不增賞,庶少蘇疲甿。」上特召,復令條陳,邲以六事對,皆切中時病。除侍御史,論救荒三事,累遷中書舍人。



 歲旱,詔求初政得失,邲應詔,大略謂:「虞允文制國用,南庫之積日以厚,戶部之入日以削,故
 近年以來,常有不足之憂。罷兵以來,諸將皆以賂得升,其勢必至於掊刻取償,益精其選。」遷給事中。張嶷以說之子除知閣,裴良琮以顯仁之侄女夫落階官,邲皆繳奏。廣西議更鹽法,邲言:「鈔法之行,漕臣嘗紿群商,沒入其貲。楮幣行之二廣,民必疑慮,且有後悔。」除刑部尚書。



 邲為東宮僚屬八年,孝宗書「安遇」字以賜,又出《梅花詩》命邲屬和,眷遇甚渥。光宗受禪,除參知政事。邲勸上專法孝宗,正風俗,節財用,振士氣,執中道,恤民力,選將帥,
 收人才,擇監司,明法令,手疏歷言之,上嘉納。除知樞密院事。紹熙四年,拜左丞相,專守祖宗法度,薦進人物,博採公論,惟恐其不聞之。未期年,除觀文殿大學士、知建康府。改隆興,請祠。



 寧宗即位,邲上疏言:「今日之事莫先於修身齊家,結人心,定規模。」判紹興府,簡稽期會,錢穀刑獄必親。或謂大臣均佚有體,邲曰:「崇大體而簡細務,吾不為也。」嘗曰:「十二時中,莫欺自己。」其實踐如此。



 改判福州,道行感疾,除少保,致仕。薨,年六十六。贈少師,謚文
 定,配饗光宗廟庭。有文集二百卷、《詞業》五十卷。



 錢端禮,字處和,臨安府臨安人。父忱,瀘川軍節度使。端禮以恩補官。紹興間,通判明州,加直秘閣,累遷右文殿修撰,仕外服有聲。高宗材之,知臨安府。



 御史中丞汪澈論版曹闕官,當遴選,權戶部侍郎兼樞密都承旨。端禮嘗建明用楮為幣,於是專委經畫,分為六務,出納皆有法,幾月易錢數百萬。



 孝宗銳意恢復,詔張浚出師。會符離稍失利,湯思退遂倡和議,端禮奏:「有用兵之名,無用
 兵之實,賈怨生事,無益於國。」思退大喜,奏除戶部侍郎。未幾,兼吏部。端禮與戶部尚書韓仲通同對,論經費,奏:「所入有限,兵食日增,更有調發,不易支吾。」上云:「須恢復中原,財賦自足。」仲通奏:「恢復未可必,且經度目前所用。」端禮奏:「仲通言是,乞採納。」



 思退與張浚議和戰不決,浚方主戰,上意甚向之。思退詭求去,端禮請對乞留,又奏:「兵者兇器,願以符離之潰為戒,早決國是,為社稷至計。」於是思退復留,命浚行邊,還戍兵,罷招納。以端禮充淮
 東宣諭使,王之望使淮西,端禮入奏:「兩淮名曰備守,守未必備;名曰治兵,兵未必精。有用兵不勝,僥幸行險,輕躁出師,大喪師徒者,必勝之說果如此,皆誤國明甚。」端禮既以是詆浚,右正言尹穡亦劾浚,罷都督,自此議論歸一矣。



 端禮至淮還,極言守備疏略,恐召金兵,宜早定和議。遂除吏部侍郎,再往淮上,驛疏言:「遣使、發兵當並行,使以盡其禮,兵以防其變,不必待金書至而後遣使。」書中或有見脅之語,不若先遣以釋其疑,於計為得。」上
 云:「端禮所奏未是。」思退傳旨撤海、泗二州戍兵,語在《思退傳》。



 金帥僕散忠義分兵入,上意中悔,令思退都督江、淮軍馬,端禮試兵部尚書,參贊軍事。思退畏怯不行,端禮赴闕,上曰:「前後廷臣議論,獨卿不變。」兼戶部尚書,俄拜端明殿學士、簽書樞密院事兼權參知政事。上嘗問:「欲遣楊由義持金帥書,而辭行甚力,誰可遣?」端禮請以王抃行,俾與金帥議,許割商、秦地,歸被俘人,惟叛亡者不與,餘誓目略同紹興,世為叔侄之國,減銀絹五萬,易
 歲貢為歲幣。及抃還,上見書,金皆聽許。端禮贊上如其式報之:「謀國當思遠圖,如與之和,則我得休息以修內治,若為忿兵,未見其可。」抃遂行。諜報北軍已回,端禮以和議既定,乞降詔。除參知政事兼權知樞密院事。



 時久不置相,端禮以首參窺相位甚急。皇長子鄧王夫人,端禮女也,殿中侍御史唐堯封論端禮帝姻,不可任執政,不報,遷太常少卿。館閣士相與上疏排端禮,皆坐絀。刑部侍郎王茀陰附端禮,建為「國是」之說以助其勢。吏部
 侍郎陳俊卿抗疏,力詆其罪,且謂本朝無以戚屬為相,此懼不可為子孫法。逮進讀《寶訓》,適及外戚,因言:「祖宗家法,外戚不與政,最有深意,陛下所宜守。」上納其言。端禮憾之,出俊卿知建寧府。



 鄧王夫人生子,太上甚喜。先兩月,恭王夫人李氏亦生子,於是恭王府直講王淮白端禮云:「恭王夫人子是為皇長嫡孫。」端禮不懌,翌日奏:「嫡庶具載《禮經》,講官當以正論輔導,不應為此邪說。」遂指淮傾邪不正,與外任。鄧王立為太子,端禮引嫌,除資
 政殿大學士、提舉德壽宮兼侍讀,改提舉洞霄宮。起知寧國府,移紹興,進觀文殿學士。



 端禮籍人財產至六十萬緡,有詣闕陳訴者,上聞之,與舊祠。侍御史範仲芑劾端禮貪暴不悛,降職一等。淳熙四年八月,復元職。薨,贈銀青光祿大夫,後謚忠肅。孫象祖,嘉定元年為左丞相,自有傳。



 魏杞,字南夫,壽春人。祖蔭入官。紹興十二年,登進士第。知宣州涇縣。從臣錢端禮薦其才,召對,擢太府寺主簿,
 進丞。端禮宣諭淮東,杞以考功員外郎為參議官,遷宗正少卿。



 湯思退建和議,命杞為金通問使,孝宗面諭:「今遣使,一正名,二退師,三減歲幣,四不發歸附人。」杞條上十七事擬問對,上隨事畫可。陛辭,奏曰:「臣若將指出疆,其敢不勉。萬一無厭,願速加兵。」上善之。



 行次盱眙,金所遣大將僕散忠義、紇石烈志寧等方擁兵闖淮,遣權泗州趙房長問所以來意,求觀國書,杞曰:「書御封也,見主當廷授。」房長馳白僕散忠義,疑國書不如式,又求割商、
 秦地及歸正人,且欲歲幣二十萬。杞以聞,上命盡依初式,再易國書,歲幣亦如其數。忠義以未如所欲,遂與志寧分兵犯山陽。戰不利,驍將魏勝死之。



 上怒金反復,詔以禮物犒督府師,杞奏:「金若從約,而金繒不具,豈不瘠國體、格事機乎?」乃以禮物行。至燕,見金主褒,具言:「天子神聖,才傑奮起,人人有敵愾意,北朝用兵能保必勝乎?和則兩國享其福,戰則將士蒙其利,昔人論之甚悉。」金君臣環聽拱竦。館伴張恭愈以國書稱「大宋」,脅去「大」字,
 杞拒之,卒正敵國體,損歲幣五萬,不發歸正人北還。上慰藉甚渥。



 守起居舍人,遷給事中、同知樞密院事,進參知政事、右僕射兼樞密使。時方借職田助邊,降人蕭鷓巴賜淮南田,意不愜,以職田請,杞言:「圭租食功養廉,借之尚可,奪之不可。」上是其言。杞以使金不辱命,繇庶官一歲至相位。上銳意恢復,杞左右其論。會郊祀冬雷,用漢制災異策免,守左諫議大夫、提舉江州太平興國宮。



 六年,授觀文殿學士、知平江府。諫官王希呂論杞貪墨,
 奪職。後以端明殿學士奉祠,告老,復資政殿大學士。淳熙十一年十一月薨,贈特進。嘉泰中,謚文節。



 周葵,字立義,常州宜興人。少力學,自鄉校移籍京師,兩學傳誦其文。宣和六年,擢進士甲科。調徽州推官。高宗移蹕臨安,諸軍交馳境上,葵與判官攝郡事,應變敏速,千里帖然。教授臨安府,未上,吏部侍郎陳與義密薦之,召試館職。將試,復引對,高宗曰:「從班多說卿端正。」



 除監察御史,徙殿中侍御史。在職僅兩月,言事至三十章,且
 歷條所行不當事凡二十條,指宰相不任責。高宗變色曰:「趙鼎、張浚肯任事,須假之權,奈何遽以小事形跡之?」葵曰:「陛下即位,已相十許人,其初皆極意委之,卒以公議不容而去,大臣亦無固志。假如陛下有過,尚望大臣盡忠,豈大臣有過,而言者一指,乃便為形跡,使彼過而不改,罪戾日深,非所以保全之也。」高宗改容曰:「此論甚奇。」



 張浚議北伐,葵三章力言「此存亡之機,非獨安危所系。」或言葵沮大計,罷為司農少卿,以直秘閣知信州。未
 上,鼎罷,陳與義執政,改湖南提刑,以親老易江東,皆不就。



 和議已定,被召,論:「為國有道,戰則勝,守則固,和則久。不然,三者在人不在我矣。」除太常少卿。時秦檜獨相,意葵前論事去,必憾趙鼎。再降殿中侍御史。葵語人曰:「元鎮已貶,葵固不言,雖門下客亦不及之也。」內降差除四人,奏言:「願陛下以仁祖為法,大臣以杜衍為法。」檜始不樂。又論國用、軍政、士民三弊,高宗曰:「國用當藏之民,百姓足則國用非所患。」又言薦舉改官之弊,宜聽減舉員,
 詔吏部措置。



 檜所厚權戶部尚書梁汝嘉將特賜出身,除兩府,汝嘉聞葵欲劾之,謂中書舍人林待聘曰:「副端將論君矣。」待聘乘檜未趨朝,亟告之,檜即奏為起居郎。葵方待引,檜下殿諭閣門曰:「周葵已得旨除起居郎。」隔下。八月庚辰也。



 參政李光擬除呂廣問館職,檜不許。時有詔從官薦士,葵以廣問應,初不相知也。光既絀,葵以附會落職,主管玉隆觀。復直秘閣,起知湖州,移平江府。時金使絡繹於道,葵不為禮,轉運李椿年希檜旨劾之,
 落職,主管崇道觀。屏居鄉閭,憂患頻仍,人不能堪,葵獨安之。



 檜死,復直秘閣、知紹興府。過闕,權禮部侍郎,尋兼國子祭酒。奏:「科舉所以取士。比年主司迎合大臣意,取經傳語可諛者為問目,學者競逐時好。望詔國學並擇秋試考官,精選通經博古之士,置之前列,其穿鑿乖謬者黜之。」



 兼權給事中。侍御史湯鵬舉言:「葵以魏良臣薦,躐處侍從;呂廣問,葵之死黨。乞並罷之。」太學生黃作、詹淵率諸生都堂投牒留葵。翌日,博士何輔等言於朝,乞
 懲戒,詔作、淵皆送五百里外州編管,葵出知信州,隨罷。



 起知撫州,引疾,改提舉興國宮,加直龍圖閣、知太平州。水壞圩堤,悉繕完,凡百二十里。傍郡圩皆沒,惟當塗歲熟。市河久堙,雨暘交病,葵下令城中,家出一夫,官給之食,並力浚導,公私便之。進集英殿修撰、敷文閣待制、知婺州。



 孝宗即位,除兵部侍郎兼侍講,改同知貢舉兼權戶部侍郎。孝宗數手詔問錢穀出入,葵奏:「陛下勞心庶政,日有咨詢,若出人意表。今皆微文細故,此必有小人
 乘間欲售其私,不可不察。」蓋指龍大淵、曾覿也。孝宗色為動。



 金主亮為其下所斃,張浚自督府來朝,密言:「敵失泗州,其懼罪者皆欲來歸,願遣軍渡淮赴之,此恢復之機也。」葵請對,謂不可輕舉,累數百言。及遣李顯忠、邵宏淵取靈壁、虹二縣,敗績。孝宗思其言,拜參知政事。葵始終守自治之說。



 兼權知樞密院事。臺諫交章言議和太速,葵與陳康伯、湯思退乞令侍從、臺諫集議,眾益洶洶,諸公待罪乞罷,不許。葵獨留身固請,孝宗曰:「卿何請之
 力也?」曰:「自預政以來,每與宰相論事,有以為然而從者;有不得以強從者;有絕不肯從者,十常四五。洎至榻前,陛下又或不然,大率十事之中,不從者七八,安得不愧於心,此臣所以欲去也。」



 嘗乞召用侍從、臺諫,孝宗曰:「安得如卿直諒者。」遂薦李浩、龔茂良,孝宗皆以為佳士,次第用之。太常奏郊牛斃,葵言:「《春秋》鼷鼠食牛角免郊,況邊虞未靖,請展郊以符天意。」詔從之。



 虞允文、陳康伯相,葵即求退,除資政殿學士、提舉洞霄宮。起知泉州,告老,
 加大學士致仕。閑居累年,不以世故縈心。淳熙元年正月,薨,年七十有七。上聞震悼,贈正奉大夫。後以子升朝,累贈太傅。



 葵孝於事親,當任子,先孤侄。其薨也,幼子與孫尚未命。平生學問不泥傳注,作《聖傳詩》二十篇、文集三十卷、奏議五卷。晚號惟心居士。四年,有司請謚,賜謚曰惠簡。



 施師點,字聖與,上饒人。十歲通《六經》,十二能文。弱冠游太學,試每在前列,司業高宏稱其文深醇有古風。尋授
 以學職,以舍選奉廷對,調復州教授。未上,丁內艱。服除,為臨安府教授。



 乾道元年,陳康伯薦,賜對,言:「歷年屢下詔恤民,而惠未加浹。陛下軫念,惟恐一夫失所;郡邑搜求,惟恐財賦不集。毋惑乎日降絲綸,恩不沾被。細民既困於倍輸,又困於非泛,重以歲惡,室且垂磬,租不如期,積多逋負。今明堂肆赦,戶自四等以下,逋自四年以前,願悉除免。」上曰:「非卿不聞此言。」詔從之。



 八年,兼權禮部侍郎,除給事中。時太子詹事已除,上又特令增員為二,
 命兼之。賜對,言:「比年人物骫骳,士氣耗TC,當廣儲人材以待用。」上曰:「觀卿所奏,公輔器也。」



 假翰林學士、知制誥兼侍讀使金。致命金廷,立班既定,相儀者以親王將至,命師點退位,師點立。相儀者請數四,師點正色曰:「班立已定,尚欲何為。」不肯少動。在廷相顧駭愕,知其有守,不敢復以為請。九年,使還,有言其事於上者,上嘉嘆不已。及後金使賀正旦至闕,問館伴:「師點今居何官?」館伴宇文價於班列中指師點以示之,金使恍然曰:「一見正
 人,令人眼明。」



 十年,除端明殿學士、簽書樞密院事。入奏,控免,上曰:「卿靖重有守,識慮深遠,朕欲用卿久矣。」復詔兼參知政事,除參知政事兼同知樞密院事。師點嘗同宰相奏事退,復同樞密周必大進呈,上曰:「適一二事卿等各陳所見,甚關大體。前此宰相奏事,執政不措辭,今卿等如此,深副所望。」必大奏:「祖宗時,宰執奏事自相可否,或至面相切責,退不相銜。自秦檜用事,執政畏避不敢言。今陛下虛心兼聽,若只宰相奏事,何用執政為?」師
 點復奏:「臣敢不竭股肱之力。」上因諭之曰:「朕欲天下事日往來胸中,未嘗釋也。」



 先是,州郡上供或不以時進,立歲終稽考法,及是,主計臣有喜為督促者,乞不待歲終先期行之。畫命已下,師點矍然曰:「此策若行,上下逼迫,民不聊生。」或謂:「令已出矣。」師點曰:「事有為天下病,惟恨更之不速。」即追寢其議。樞密周必大舉手賀師點曰:「使天下赤子不被其毒者,公之賜也。」一日,入對後殿,上曰:「朕前飲冰水過多,忽暴下,幸即平復。」師點曰:「自古人君
 當無事時,快意所為,忽其所當戒,其後未有不悔者。」上深然之。



 十三年,辭兼同知樞密院事。權提舉國史院,權提舉《國朝會要》。十四年,除知樞密院事。師點惓惓搜訪人才,手書置夾袋中,謂蜀去朝廷遠,人才難以自見,蜀士之賢者,俾各疏其所知,差次其才行、文學,每有除授,必列陳之。十五年春,以資政殿大學士知泉州,除提舉臨安府洞霄宮。



 紹熙二年,除知隆興府、江西安撫使。師點嘗謂諸子曰:「吾平生仕宦,皆任其升沉,初未嘗容
 心其間,不枉道附麗,獨人主知之,遂至顯用。夫人窮達有命,不在巧圖,惟忠孝乃吾事也。」三年,得疾薨,年六十九。贈金紫光祿大夫。有奏議七卷、制槁八卷、《東宮講議》五卷、《易說》四卷、《史識》五卷、文集八卷。



 蕭燧,字照鄰,臨江軍人。高祖固,皇祐初為廣西轉運使,知儂智高兇狡,條上羈縻之策於樞府,不果用,智高後果叛。父增,紹興初嘗應制舉。



 燧生而穎異,幼能屬文。紹興十八年,擢進士高第。授平江府觀察推官。時秦檜當
 國,其親黨密告燧,秋試必主文漕臺,燧詰其故,曰:「丞相有子就舉,欲以屬公。」燧怒曰:「初仕敢欺心耶!」檜懷之,既而被檄秀州,至則員溢,就院易一員往漕闈,秦熹果中前列。秩滿,當為學官,避檜,調靜江府察推而歸。



 燧未第時,夢神人示以文書,記其一聯云:「如火烈烈,玉石俱焚;在冬青青,松柏不改。」已而果符前事。未幾,丁憂。三十二年,授靖州教授。孝宗初,除諸王宮大小學教授。輪對,論「官當擇人,不當為人擇官。」上喜,制《用人論》賜大臣。淳熙
 二年,累遷至國子司業兼權起居舍人,進起居郎。



 先是,察官闕,朝論多屬燧,以未歷縣,遂除左司諫。上諭執政:「昨除蕭燧若何?」龔茂良奏:「燧純實無華,正可任言責,聞除目下,外議甚允。」燧首論辨邪正然後可以治,上以外臺耳目多不稱職,時宦官甘忭之客胡與可、都承旨王抃之族叔秬皆持節於外,有所依憑,無善狀,燧皆奏罷之。



 時復議進取,上以問燧,對曰:「今賢否雜揉,風俗澆浮,兵未強,財未裕,宜臥薪嘗膽以圖內治。若恃小康,萌驕
 心,非臣所知。」上曰:「忠言也。」因勸上正紀綱;容直言;親君子,遠小人;近習有勞可賞以祿,不可假以權。上皆嘉納。擢右諫議大夫,入謝,上曰:「卿議論鯁切,不求名譽,糾正奸邪,不恤仇怨。」



 五年,同知貢舉。有旨下江東西、湖南北帥司招軍,燧言:「所募多市井年少,利犒繼,往往捕農民以應數,取細民以充軍。乞嚴戒諸郡,庶得丁壯以為用。」從之。



 夔帥李景嗣貪虐,參政趙雄庇之,臺臣謝廓然不敢論,燧獨奏罷之。雄果營救,復命還任。燧再論,並及雄。
 雄密奏燧誤聽景嗣仇人之言,遂下臨安府捕恭州士人鐘京等置之獄,坐以罪,景嗣復依舊職。燧乃自劾,詔以風聞不許,竟力求去。徙刑部侍郎,不拜,固請補外。出知嚴州,吏部尚書鄭丙、侍郎李椿上疏留之,上亦尋悔。



 嚴地狹財匱,始至,官鏹不滿三千,燧儉以足用。二年之間,積至十五萬,以其羨補積逋,諸邑皆寬。先是,宣和庚子方臘盜起,甲子一周,人人憂懼,會遂安令朘士兵廩給,群言恟恟。燧急易令,且呼卒長告戒,悉畏服。城中惡
 少群擾市,燧密籍姓名,涅補軍額,人以按堵。上方靳職名,非功不予,詔燧治郡有勞,除敷文閣待制,移知婺州。父老遮道,幾不得行,送出境者以千數。



 婺與嚴鄰,人熟知條教,不勞而治。歲旱,浙西常平司請移粟於嚴,燧謂:「東西異路,不當與,然安忍於舊治坐視?」為請諸朝,發太倉米振之。



 八年,召還,言:「江、浙再歲水旱,願下詔求言,仍令諸司通融郡縣財賦,毋但督迫。」除吏部右選侍郎,旋兼國子祭酒。九年,為樞密都承旨。近例,承旨以知閣門
 官兼,或怙寵招權,上思復用儒臣,故命燧以龍圖閣待制為之。燧言:「債帥之風未殄,群臣多迎合獻諛,強辨干譽,宜察其虛實。」上稱善。除權刑部尚書,充金使館伴。



 十年,兼權吏部尚書。上言廣西諸郡民身丁錢之弊。兼侍講,升侍讀。言:「命令不可數易,憲章不可數改。初官不許恩例免試,今或竟令注授。既卻羨餘之數,今反以出剩為名。諸路錄大闢,長吏當親詰,若死囚數多,宜如漢制殿最以聞。」事多施行。慶典霈澤,丁錢減半,亦自燧發之。



 高宗山陵,充按行使,除參知政事,尋充永思陵禮儀使,權監修國史日歷。十六年,權知樞密院。以年及自陳,上留之,不可,除資政殿學士,與郡。復請閑,提舉臨安府洞霄宮。紹熙四年卒,年七十七。謚正肅。



 孝宗每稱其全護善類,誠實不欺,手書《二十八將傳》以賜。子逵,登淳熙十四年進士第,唱名第四,孝宗曰:「逵才氣甚佳,父子高科,殊可喜。」逵累官至太常。



 龔茂良,字實之,興化軍人。紹興八年,進士第。為南安簿、
 邵武司法。父母喪,哀號擗踴,鄰不忍聞。調泉州察推,以廉勤稱。改宣教郎,以同知樞密院事黃祖舜薦,召試館職,除秘書省正字。累遷吏部郎官。



 張浚視師江、淮,茂良言:「本朝禦敵,景德之勝本於能斷,靖康之禍在於致疑,願仰法景德之斷,勿為靖康之疑。」除監察御史。



 江、浙大水,詔陳闕失,茂良疏曰:「水至陰也,其占為女寵,為嬖佞,為小人專制。崇、觀、政和,小人道長,內則憸腐竊弄,外則奸回充斥,於是京城大水,以至金人犯闕。今進退一人,
 施行一事,命由中出,人心嘩然,指為此輩。臣願先去腹心之疾,然後政事闕失可次第言矣。」內侍梁珂、曾覿、龍大淵皆用事,故茂良及之。



 遷右正言。會內侍李珂沒,贈節度,謚靖恭,茂良諫曰:「中興名相如趙鼎,勛臣如韓世忠,皆未有謚,如朝廷舉行,亦足少慰忠義之心。今施於珂為可惜。」竟寢其謚。嘗論大淵、覿奸回,至是又極言之,曰:「今積陰弗解,淫雨益甚,熒惑入斗,正當吳分,天意若有所怒而未釋。二人害政,甚珂百倍。」上諭以「皆潛邸舊,
 非他近習比,且俱有文學,敢諫爭,未嘗預外事。」



 翌日,再疏言:「唐德宗謂李泌:『人言盧杞奸邪,朕獨不知,何耶?』泌曰:『此其所以為奸邪也』。今大淵、覿所為,行道之人能言之,而陛下更頌其賢,此臣所以深憂。」疏入,不報,即家居待罪。章再上,除太常少卿,五辭不拜,除直秘閣、知建寧府。自以不為群小所容,請祠,不允。



 上後知二人之奸,既逐於外,起茂良廣東提刑,就知信州。即番山之址建學,又置番禺南海縣學,既成,釋奠,行鄉飲酒以落之。城東
 舊有廣惠庵,中原衣冠沒於南者葬之,歲久廢,茂良訪故地,更建海會浮圖,菆寄暴露者皆揜藏無遺。召對崇政殿,左丞相陳俊卿欲留之,右相虞允文不樂。會俊卿亦罷,除直顯謨閣、江西運判兼知隆興府。



 上以江西連歲大旱,知茂良精忠,以一路荒政付之。茂良戒郡縣免積稅,上戶止索逋,發廩振贍。以右文殿修撰再任,疫癘大作,命醫治療,全活數百萬。進待制敷文閣,賞其救荒之功。召對,奏:「潢池弄兵之盜,即南畝負耒之民。今諸郡
 荒田極多,願詔監司守臣條陳,募人從便請耕,民有餘粟,雖驅之為寇,亦不從矣。」除禮部侍郎。



 上亟用茂良,手詔問國朝典故有自從官徑除執政例,明日即拜參知政事。奏事,賜坐,上顧葉衡及茂良曰:「兩參政皆公議所與。」衡等起謝,上從容曰:「自今諸事毋循私,若鄉曲親戚,且未須援引。朕每存公道,設有誤,卿等宜力爭,君臣之間不可事形跡。」茂良曰:「大臣以道事君,遇有不可,自當啟沃,豈容跡見於外。」請詔有司刊定七司法。



 淮南旱,茂良
 奏取封樁米十四萬,委漕帥振濟。或謂:「救荒常平事,今遽取封樁米,毋乃不可?」茂良以為:「淮南咫尺敵境,民久未復業,饑寒所逼,萬一嘯聚,患害立見,寧能計此米乎?」他日,上獎諭曰:「淮南旱荒,民無饑色,卿之力也。」



 潮州守奏通判不法,得旨,下帥臣體訪。通判,茂良鄉人也,同列密以省吏付棘寺推鞫,欲及茂良。奏事退,同列留身,出獄案進上,茂良不知也。上厲聲曰:「參政決無此!」茂良遜謝,不復辯。



 葉衡罷,上命茂良以首參行相事。慶壽禮行,
 中外凱恩,茂良慨然嘆曰:「此當以身任怨,不敢愛身以弊天下。若自一命以上覃轉,不知月添給奉與來歲郊恩奏補幾何,將何以給?」



 宣諭獎用廉退,茂良奏:「朱熹操行耿介,屢召不起,宜蒙錄用。」除秘書郎。群小乘間讒毀,未幾,手詔付茂良,謂「虛名之士,恐壞朝廷。」熹迄不至。錢良臣侵盜大軍錢糧,累數十萬,茂良奏其事,手詔令具析。俄召良臣赴闕,駸駸柄用,其後茂良之貶,良臣與有力焉。



 茂良之以首參行相事也,逾再歲,上亦不置相,因
 諭茂良:「史官近奏三臺星不明,蓋實艱其選耳。」淳熙四年正月,召史浩於四明,茂良亦覺眷衰,因疾力求去。上曰:「朕以經筵召史浩,卿不須疑。」



 時曾覿欲以文資祿其孫,茂良以文武官各隨本色蔭補格法繳進。覿因茂良入堂道間,俾直省官賈光祖等當道不避。街司叱之,曰:「參政能幾時!」茂良奏:「臣固不足道,所惜者朝廷大體。」上諭覿往謝,茂良正色曰:「參知政事者,朝廷參知政事也。」覿慚退。上諭茂良先遣人於覿,沖替而後施行。茂良批
 旨,取賈光祖輩下臨安府撻之。手詔宣問施行太遽,茂良待罪。上使人宣諭委曲,令繳進手詔,且謂:「卿去雖得美名,置朕何地?」茂良即奉詔。



 謝廓然賜出身,除殿中侍御史,廓然附曾覿者也。中書舍人林光輔繳奏,不書黃,遂補外。茂良力求去,上諭曰:「朕極知卿,不敢忘,欲保全卿去,俟議恢復,卿當再來。」是日,除職與郡,令內殿奏事,乃手疏恢復六事,上曰:「卿五年不說恢復,何故今日及此?」退朝甚怒,曰:「福建子不可信如此!」謝廓然因劾之,乃
 落職放罷;尋又論茂良擅權不公,矯傳上旨,輒斷賈光祖等罪,遂責降,安置英州。父子卒於貶所。



 覿與廓然死後,茂良家投匭訟冤,遂復通奉大夫。周必大獨相,進呈復職,上曰:「茂良本無罪。」遂復資政殿學士,謚莊敏。



 茂良平生不喜言兵,去國之日乃言恢復事,或謂覿密令人訹之云:「若論恢復,必再留。」茂良信之。廓然論茂良,亦以此為罪。茂良沒數年,朱熹從其子得副本讀之,則事雖恢復,而其意乃極論不可輕舉,猶平生素論也,深為之
 嘆息云。



 論曰:葛邲在相位雖不久,而能守法度,進人才,其處己也,則以不欺為本。錢端禮以戚屬為相,周葵晚雖不附秦檜,而與龔茂良皆主和議。若乃魏杞奉使知尊國體,施師點之靖重有守,蕭燧忠實敢言,仕於紹興之間,可謂不幸矣。



\end{pinyinscope}