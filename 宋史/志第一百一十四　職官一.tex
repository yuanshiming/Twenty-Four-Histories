\article{志第一百一十四 職官一}

\begin{pinyinscope}

 三師
 三公宰執門下省
 中書省尚書省



 昔武王克商,史臣紀其成功,有曰:「列爵惟五,分土惟三,建官惟賢,位事惟能。」後世曰爵,曰官,曰職,分而任之,其原蓋始乎此。然周初之制,已不可考。周公作六典,自天
 官塚宰而下,小大高下,各帥其屬以任其事,未聞建官而不任以事,位事而不命以官者;至於列爵分土,此封建諸侯之制也,亦未聞以爵以土,如後世虛稱以備恩數者也。秦、漢及魏、晉、南北朝,官制沿革不常,不可殫舉。後周復《周禮》六典官稱,而參用秦、漢。隋文帝廢《周禮》之制,惟用近代之法。唐承隋制,至天授中,始有試官之格,又有員外之置,尋為檢校、試、攝、判、知之名。其初立法之意未嘗不善,蓋欲以名器事功甄別能否,又使不肖者
 絕年勞序遷之覬覦。而世戚勛舊之家,寵之以祿,而不責以猷為。其居位任事者,不限資格,使得自竭其所長,以為治效。且黜陟進退之際,權歸於上,而有司若不得預。殊不知名實混殽,品秩貿亂之弊,亦起於是矣。



 宋承唐制,抑又甚焉。三師、三公不常置,宰相不專任三省長官,尚書、門下並列於外,又別置中書禁中,是為政事堂,與樞密對掌大政。天下財賦,內庭諸司,中外筦庫,悉隸三司。中書省但掌冊文、覆奏、考帳;門下省主乘輿八寶,
 朝會板位,流外考較,諸司附奏挾名而已。臺、省、寺、監,官無定員,無專職,悉皆出入分蒞庶務。故三省、六曹、二十四司,類以他官主判,雖有正官,非別敕不治本司事,事之所寄,十亡二三。故中書令、侍中、尚書令不預朝政,侍郎、給事不領省職,諫議無言責,起居不記注;中書常闕舍人,門下罕除堂侍,司諫、正言非特旨供職亦不任諫諍。至於僕射、尚書、丞、郎、員外,居其官不知其職者,十常八九。其官人受授之別,則有官、有職、有差遣。官以寓祿
 秩、敘位著,職以待文學之選,而別為差遣以治內外之事。其次又有階、有勛、有爵。故仕人以登臺閣、升禁從為顯宦,而不以官之遲速為榮滯;以差遣要劇為貴途,而不以階、勛、爵邑有無為輕重。時人語曰:「寧登瀛,不為卿;寧抱槧,不為監。」虛名不足以砥礪天下若此。外官,則懲五代藩鎮專恣,頗用文臣知州,復設通判以貳之。階官未行之先,州縣守令,多帶中朝職事官外補;階官既行之後,或帶或否,視是為優劣。



 大凡一品以下,謂之「文武
 官」;未常參者,謂之「京官」;樞密、宣徽、三司使副、學士、諸司而下,謂之「內職」;殿前都校以下,謂之「軍職」。外官則有親民、厘務二等,而監軍、巡警亦比親民。此其概也。故自真宗、仁宗以來,議者多以正名為請。咸平中,楊億首言:「文昌會府,有名無實,宜復其舊。」既而言者相繼,乞復二十四司之制。至和中,吳育亦言:「尚書省,天下之大有司,而廢為閑所,當漸復之。」然朝論異同,未遑厘正。神宗即位,慨然欲更其制。熙寧末,始命館閣校《唐六典》。元豐三年,
 以摹本賜群臣,乃置局中書,命翰林學士張璪等詳定。八月,下詔肇新官制,省、臺、寺、監領空名者一切罷去,而易之以階。九月,詳定所上《寄祿格》。會明堂禮成,近臣遷秩即用新制,而省、臺、寺、監之官,各還所職矣。五年,省、臺、寺、監法成。六年,尚書新省成,帝親臨幸,召六曹長貳以下,詢以職事,因誡敕焉。初,新階尚少,而轉行者易以混雜。及元祐初,於朝議大夫六階以上始分左右。既又以流品無別,乃詔寄祿官悉分左右,詞人為左,餘人為右。紹聖
 中罷之。崇寧初,以議者有請,自承直至將仕郎,凡換選人七階。大觀初,又增宣奉至奉直大夫四階。政和末,自從政至迪功郎,又改選人三階,於是文階始備。而武階亦詔易以新名:正使為大夫,副使為郎,而橫班十二階使、副亦然。故有郎居大夫之上者。繼以新名未具,增置宣正履正大夫、郎凡十階,通為橫班,而文武官制益加詳矣。



 大抵自元祐以後,漸更元豐之制:二府不分班奏事,樞密加置簽書,戶部則不令右曹專典常平而總於
 其長,起居郎、舍人則通記起居而不分言動,館職則增置校勘黃本。凡此,皆與元豐稍異也。其後蔡京當國,率意自用。然動以繼志為言,首更開封守臣為尹、牧,由是府分六曹,縣分六案。又內侍省職,悉仿機廷之號。已而修六尚局,建三衛,即又更兩省之長為左輔、右弼,易端揆之稱為太宰、少宰。是時員既濫冗,名且紊雜。甚者走馬承受升擁使華;黃冠道流,亦濫朝品。元豐之制,至此大壞。及宣和末,王黼用事,方且追咎元祐紛更,乃請設
 局,以修《官制格目》為正名,亦何補矣。



 建炎中興,參酌潤色,因呂頤浩之請,左、右僕射並同中書門下平章事,兩省侍郎改為參知政事,三省之政合乎一。乾道八年,又改左、右僕射為左、右丞相,刪去三省長官虛稱,道揆之名遂定。然維時多艱,政尚權宜。御營置使,國用置使,修政局置提舉,軍馬置都督,並以宰相兼之。總制司理財,同都督、督視理兵,並以執政兼之。因事創名,殊非經久。惟樞密本兵,與中書對掌機務,號東、西二府,命宰相兼
 知院事。建炎四年,實用慶歷故典。其後,兵興則兼樞密使,兵罷則免;至開禧初,始以宰臣兼樞密為永制。



 當多事時,諸部或長貳不並置,或並郎曹使相兼之,惟吏部、戶部不省不並。兵休稍稍增置。其後,詔非曾任監司、守臣,不除郎官,著為令。又增館閣員,廣環衛官。然紹興務行元祐故事,以「左右」二字分別流品,其後,以人言省去,寧清濁相涵,無絕人遷善之路。橫班以郎居大夫之上,既厘而正之矣,而介冑之士與縉紳同稱,寧名號未正,
 毋示人以好武之機。陳傅良欲定史官遷次之序,眾論韙之,而未及行。洪邁欲改三衙軍官稱謂,當時嘉之,卒未暇講。考古之制,量今之宜,蓋自元祐以逮政和,已未嘗拘乎元豐之舊。中興若稽成憲,二者並行而不悖。故凡大而分政任事之臣,微而筦庫監局之官,沿襲不革者,皆先後所同便也。或始創而終罷,或欲革而猶因,則有各當其可者焉。類而書之,先後互見,作《職官志》。以至廩給、傔



 從,雖微必錄,並從舊述云。



 三師三公宋承唐制,以太師、太傅、太保為三師,太尉、司徒、司空為三公,為宰相、親王使相加官,其特拜者不預政事,皆赴上於尚書省。凡除授,則自司徒遷太保,自太傅遷太尉,檢校官亦如之。太尉舊在三師下,由唐至宋加重,遂以太尉居太傅之上。若宰臣官至僕射致仕者,以在位久近,或已任司空、司徒,則拜太尉、太傅等官。若太師則為異數,自趙普以開國元勛,文彥博以累朝耆德,方特拜焉。雖太傅王旦、司徒呂夷簡各任宰相二十
 年,止以太尉致仕。



 熙寧二年,富弼除守司空兼侍中、平章事,辭司空、侍中。三年,曾公亮除守司空、檢校太師兼侍中,以兩朝定策之功辭相位也。六年,文彥博除守司徒兼侍中。九年,彥博除守太保兼侍中,辭太保。元豐三年,以曹佾檢校太師、守司徒兼中書令。九月,詔檢校官除三公、三師外並罷。又以文彥博落兼侍中,除守太尉,富弼守司徒,皆錄定策之功也。六年,彥博守太師致仕。八年,王安石守司空,曹佾守太保。元祐元年,文彥博落
 致仕,太師、平章軍國重事,呂公著守司空、同平章軍國重事。崇寧三年,蔡京授司空,行尚書左僕射。大觀元年,京為太尉;二年,為太師。政和二年,京落致仕,依前太師,三日一至都堂治事。九月,詔:「以太師、太傅、太保,古三公之官,今為三師,古無此稱,合依三代為三公,為真相之任。司徒、司空,周六卿之官,太尉,秦主兵之任,皆非三公,並宜罷之。仍考周制,立三孤少師、少傅、少保,亦稱三少,為三次相之任。」至是,京始以三公任真相。



 三公自國
 初以來,未嘗備官。獨宣和末,三公至十八人,三少不計也。太師三人:蔡京、童貫、鄭紳;太傅四人:王黼、燕王俁、越王偲、鄆王楷;太保十一人:蔡攸、肅王樞至儀王。渡江後,秦檜為太師,張俊、韓世忠為太傅,劉光世為太保。乾道初,楊沂中、吳璘並為太傅。紹熙初,史浩為太師,嗣秀王為太保。自紹熙後,三公未嘗備官。其後,韓乇冑、史彌遠、賈似道專政,皆至太師焉。



 宰相之職佐天子,總百官,平庶政,事無不統。宋承唐制,以同平章事為真相之任,
 無常員;有二人,則分日知印。以丞、郎以上至三師為之。其上相為昭文館大學士、監修國史,其次為集賢殿大學士。或置三相,則昭文、集賢二學士並監修國史,各除。唐以來,三大館皆宰臣兼,故仍其制。國初,範質昭文學士,王溥監修國史,魏仁浦集賢學士,此為三相例也。神宗新官制,於三省置侍中、中書令、尚書令,以官高不除人,而以尚書令之貳左、右僕射為宰相。左僕射兼門下侍郎,以行侍中之職;右僕射兼中書侍郎,以行中書令
 之職。政和中,改左、右僕射為太宰、少宰,仍兼兩省侍郎。靖康中,復改為左、右僕射。



 建炎三年,呂頤浩請參酌三省之制,左、右僕射並加同中書門下平章事,門下、中書二侍郎並改為參知政事,廢尚書左、右丞。從之。乾道八年,詔尚書左、右僕射可依漢制改為左、右丞相。詳定敕令所言:「近承詔旨,改左、右僕射為左、右丞相,令刪去侍中、中書、尚書令,以左、右丞相充。緣舊左、右僕射非三省長官,故為從一品。今左、右丞相系充侍中、中書、尚書令
 之位,即合為正一品。」從之。丞相官以太中大夫以上充。



 平章軍國重事元祐中置,以文彥博太師、呂公著守司空相繼為之,序宰臣上。所以處老臣碩德,特命以寵之也。故或稱「平章軍國重事」,或稱「同平章軍國事」。五日或兩日一朝,非朝日不至都堂。其後,蔡京、王黼以太師總三省事,三日一朝,赴都堂治事。開禧元年,韓侂冑拜平章,討論典禮,乃以「平章軍國事」為名。蓋省「重」字則所預者廣,去「同」字則所任者專。邊事起,乃命一日一朝,省印
 亦歸其第,宰相不復知印。其後,賈似道專權,竊位日久,尊寵日隆,位皆在丞相上。



 使相親王、樞密使、留守、節度使兼侍中、中書令、同平章事者,皆謂之使相。不預政事,不書敕,惟宣敕除授者,敕尾存其銜而已。乾德二年,範質等三相皆罷,以趙普同平章事,李崇矩樞密使。命下,無宰相書敕,使問翰林陶穀。穀謂:「自昔輔相未嘗虛位。惟唐大和中甘露事,數日無宰相,時左僕射令狐楚等奉行制書。今尚書亦南省
 長官,可以書敕。」竇儀曰:「穀之所陳,非承平令典。今皇弟開封尹、同平章事,即宰相之任也,可書敕。」從之。



 參知政事掌副宰相,毗大政,參庶務。乾德二年置,以樞密直學士薛居正、兵部侍郎呂餘慶並本官參知政事。先是,已命趙普為相,欲置之副,而難其名稱。以問翰林學士陶穀曰:「下宰相一等有何官?」對曰:「唐有參知機務、參知政事。」故以命之。仍令不押班,不知印,不升政事堂,殿廷別設磚位,敕尾著銜降宰相,月奉雜給半之,未欲
 與普齊也。開寶六年,始詔居正、餘慶於都堂與宰相同議政事。至道元年,詔宰相與參政輪班知印,同升政事堂。押敕齊銜,行則並馬,自寇準始,以後不易。



 元豐新官制,廢參知政事,置門下、中書二侍郎,尚書左、右丞以代其任。建炎三年,復以門下、中書侍郎為參知政事,而省左、右丞。乾道八年,改左、右僕射為左、右丞相,其參知政事如故,以中大夫以上充,常除二員或一員。嘉泰三年,始除三員。故事,丞相謁告,參預不得進擬。惟丞相未除,
 則輪日當筆,然多不逾年,少僅旬月。淳熙初,葉衡罷相,龔茂良行相事近三年,亦創見也。



 門下省受天下之成事,審命令,駁正違失,受發通進奏狀,進請寶印。凡中書省畫黃、錄黃,樞密院錄白、畫旨,則留為底。及尚書省六部所上有法式事,皆奏覆審駁之。給事中讀,侍郎省,侍中審,進入被旨畫聞,則授之尚書省、樞密院。即有舛誤應舉駁者,大則論列,不則改正。凡文書自內降者,著之籍。章奏至,則受而通進,俟頒降,分
 送所隸官司。凡吏部擬六品以下職事官,則給事中校其仕歷、功狀,侍郎。侍中引驗審察,非其人則論奏。凡遷改爵秩、加敘勛封、四選擬注奏鈔之事,有舛誤,退送尚書省。覆刑部大理寺所斷獄,審其輕重枉直,不當罪,則以法駁正之。



 國初循舊制,以中書門下平章事為宰相之職,復用兩制官一員判門下省事。官制行,始厘正焉。凡官十有一:侍中、侍郎、左散騎常侍各一人,給事中四人,左諫議大夫、起居郎、左司諫、左正言各一人。先是,中
 書人吏分掌五房:曰孔目房、吏房、戶房、兵禮房、刑房;又有主事、勾銷二房。至是,厘中書為三省,分兵與禮為六房,各因其省之事而增益之。門下凡分房十:曰吏房,曰戶房,曰禮房,曰兵房,曰刑房,曰工房,皆視其房之名,而主行尚書省六曹二十四司所上之事;曰開拆房,曰章奏房,曰制敕庫房,亦皆視其名,而受遣文書、表狀,與供閱敕令格式、擬官爵封勛之類,惟班簿、本省雜務則歸吏房。吏四十有九:錄事、主事各三人,令史六人,書令史
 十有八人,守當官十有九人。而外省吏十有九人:令史一人,書令史二人,守當官六人,守闕守當官十人。元豐八年,以門下、中書外省為後省,門下外省復置催驅房。元祐三年,詔吏部注通判,赴門下引驗;應省、臺、寺、監諸司人吏四分減一。復置點檢房。四年,又別立吏額。紹聖二年,守闕守當官,門下、中書省各以百人,尚書省百五十人為額。四年,三省吏員並依元豐七年額。



 侍中掌佐天子議大政,審中外出納之事。大祭祀則版
 奏中嚴外辦,導輿輅,詔升降之節;皇帝齋則請就齋室。大朝會則承旨宣制、告成禮,祭祀亦如之。冊后則奉寶以授司徒。國朝以秩高罕除。知建隆至熙寧,真拜侍中才五人,雖有用他官兼領,而實不任其事。官制行,以左僕射兼門下侍郎行侍中職,別置侍郎以佐之。南渡後,置左、右丞相,省侍中不置。



 侍郎掌貳侍中之職,省中外出納之事。大祭祀則前導輿略,詔進止。大朝賀則授表以奏祥瑞。冊后則奉節及
 寶位。與知樞密院、同知樞密院、中書侍郎、尚書左右丞為執政官。南渡後,復置參知政事,省門下侍郎不置。



 左散騎常侍左諫議大夫左司諫左正言同掌規諫諷諭。凡朝政闕失、大臣至百官任非其人、三省至百司事有違失,皆得諫正。國初雖置諫院,知院官凡六人,以司諫、正言充職;而他官領者,謂之知諫院。正言、司諫亦有領他職而不預諫諍者。官制行,始皆正名。



 元豐八年,諫議大夫孫覺言:「據《官制格目》,諫官之職,凡發令
 舉事,有不便於時,不合於道,大則廷議,小則上封。若賢良之遺滯於下,忠孝之不聞於上,則以事狀論薦,乞依此以修舉職事。」八月,門下省言:「諫議大夫、司諫、正言合通為一。」詔並從之。十月,詔仿《六典》置諫官員。元祐元年二月,詔諫官雖不同省,許二人同上殿。後又從司諫虞策之請,如獨員,許與臺官同對。九月,左、右正言久闕,侍御史王巖叟言:「國家仿近古之制,諫官六員,方之先王,已自為少,望詔補足,無令久空職。」十月,司諫王覿言:「自
 今中書舍人闕,勿以諫官兼權。」從之。十一月,巖叟又言:「近降聖旨,兩省諫官各令出入異戶,勿與給事中、中書舍人通。實欲限隔諫官,不使在政事之地,恐知本末,數論列爾。」尋詔諫官直舍仍舊。八年,詔執政親戚不除諫官。建中靖國元年,言者謂諫官論事,惟憑詢訪,而百司之事,六曹所報外,皆不得其詳。遂詔諫官案許關臺察。



 給事中四人,分治六房,掌讀中外出納,及判後省之事。若政令有失當,除授非其人,則論奏而駁正之。凡章奏,
 日錄目以進,考其稽違而糾治之。故事,詔旨皆付銀臺司封駁。官制行,給事中始正其職,而封駁司歸門下。



 元豐五年五月,詔給事中許書畫黃,不書草,著為令。六月,給事中陸佃言:「三省、密院文字,已讀者尚令封駁,慮失之重復。」。詔罷封駁房。六年,詔駁正事赴執政稟議。七年,有旨,舉駁事,依中書舍人封還詞頭例。既而令稟議如初,給事中韓忠彥言:「給、舍職位頗均,一則不稟白而聽封還,一則許舉駁而先稟議,於理未允。且朝廷之事執
 政所行,職當封駁則已與執政異,自當求決於上,尚何稟議之有?」詔從之。紹聖四年,葉祖洽言:「兩省置給、舍,使之互察。今中書舍人兼權封駁,則給事中之職遂廢。」詔特旨書讀不回避,餘互書判。元符三年,翰林學士曾肇言:「門下之職,所以駁正中書違失。近日給事封駁中書錄黃,乃令舍人書讀行下,隳壞官制,有損治體。願正紀綱,為天下後世法。」重和元年,給事中張叔夜言:「凡命令之出,中書宣奉,門下審讀,然後付尚書頒行,而密院被
 旨者,亦錄付門下,此神宗官制也。今急速文字,不經三省,而諸房以空黃先次書讀,則審讀殆成虛設矣,乞立法禁。」從之。



 凡分案五:曰上案,主寶禮及朝會所行事;曰下案,主受發文書;曰封駁案,主封駁及試吏,校其功過;曰諫官案,主關報文書;曰記注案,主錄起居注。其雜務則所分案掌焉。紹興以後,止除二人或一人。



 起居郎一人,掌記天子言動。御殿則侍立,行幸則從,大朝會則與起居舍人對立於殿下螭首之側。凡朝廷命令赦
 宥、禮樂法度損益因革、賞罰勸懲、群臣進對、文武臣除授及祭祀宴享、臨幸引見之事,四時氣候、四方符瑞、戶口增減、州縣廢置,皆書以授著作官。



 國朝舊置起居院,命三館校理以上修起居注。熙寧四年,詔諫官兼修注者,因後殿侍立,許奏事。元豐二年,兼修注王存乞復起居郎、舍人之職,使得盡聞明天子德音,退而書之。神宗亦謂:「人臣奏對有頗僻讒慝者,若左右有史官書之,則無所肆其奸矣。」然未果行。故事,左、右史雖日侍立,而欲奏事,
 必稟中書俟旨。存因對及之。八月,乃詔雖不兼諫職,許直前奏事。蓋存發之也。官制行,改修注為郎、舍人。六年,詔左、右史分記言動;元祐元年,仍詔不分。七年,詔邇英閣講讀罷,有留身奏事者,許侍立。紹聖元年,中丞黃履言:「所奏或幹機密,難令旁立,仍依先朝故事。」先是,御後殿則左、右史分日侍立;崇寧三年,詔如前殿之儀,更不分日。大觀元年,詔事有足以勸善懲惡者,雖秩卑亦書之。紹興二十八年,用起居郎洪遵言,起居郎、舍人自今
 後許依講讀官奏事。隆興元年,用起居郎兼侍講胡銓言,前殿依後殿輪左、右史侍立。



 符寶郎二人,掌外廷符寶之事。禁中別有內符寶郎。官制行,未嘗除。大觀初,八寶成,詔依《唐六典》增置。靖康罷之。



 通進司隸給事中,掌受三省、樞密院、六曹、寺監百司奏牘,文武近臣表疏及章奏房所領天下章奏案牘,具事目進呈,而頒布於中外。



 進奏院隸給事中,掌受詔敕及三省、樞密院宣扎,六曹、寺監百司符牒,頒於諸路。凡章奏至,則具事目上門下省。若案牘及申稟文書,則分納諸官司。凡奏牘違戾法式者,貼說以進。



 熙寧四年,詔:「應朝廷擢用材能、賞功罰罪事可懲勸者,中書檢正、樞密院檢詳官月以事狀錄付院,謄報天下。」元祐初,罷之。紹聖元年,詔如熙寧舊條。靖康元年二月,詔:「諸道監司、帥守文字,應邊防機密急切事,許進奏院直赴通進司投進。」



 舊制,通進、銀臺
 司,知司官二人,兩制以上充。通進司,掌受銀臺司所領天下章奏案牘,及閣門在京百司奏牘、文武近臣表疏,以進御,然後頒布於外。銀臺司,掌受天下奏狀案牘,抄錄其目進御,發付勾檢,糾其違失而督其淹綬。發敕司,掌受中書、樞密院宣敕,著籍以頒下之。



 登聞檢院,隸諫議大夫;登聞鼓院,隸司諫、正言掌受文武官及士民章奏表疏。凡言朝政得失、公私利害、軍期機密、陳乞恩賞、理雪冤濫,及奇方異術、改換文資、改
 正過名,無例通進者,先經鼓院進狀;或為所抑,則詣檢院。並置局於關門之前。



 中興後,檢、鼓、糧、審計、官告、進奏,謂之六院。例以京官知縣有政績者充;亦有自郡守除者,繼即除郎。恩數略視職事官,而不入雜壓。紹興十一年,胡汝明以料院除監察御史,遂遷侍御史。乾道後,相繼入臺者數人,六院彌重,為察官之儲。淳熙初,班寺監、丞之上。紹熙二年,詔六院官復入雜壓,在九寺簿之下,六院各隨所隸。



 中書省掌進擬庶務,宣奉命令,行臺諫章疏、群臣奏請興創改革,及中外無法式事應取旨事。凡除省、臺、寺、監長貳以下,及侍從、職事官,外任監司、節鎮、知州、軍通判,武臣遙郡橫行以上除授,皆掌之。



 凡命令之體有七:曰冊書,立后妃,封親王、皇子、大長公主,拜三師、三公、三省長官,則用之。曰制書,處分軍國大事,頒赦宥德音,命尚書左右僕射、開府儀同三司、節度使,凡告廷除授,則用之。曰誥命,應文武官遷改職秩、內外命婦除授及封敘、
 贈典,應合命詞,則用之。曰詔書,賜待制、大卿監、中大夫、觀察使以上,則用之。曰敕書,賜少卿監、中散大夫、防禦使以下,則用之。曰御札,布告登封、郊祀、宗祀及大號令,則用之。曰敕榜,賜酺及戒勵百官、曉諭軍民,則用之。皆承制畫旨以授門下省,令宣之,侍郎奉之,舍人行之。留其所得旨為底:大事奏稟得旨者為「畫黃?,小事擬進得旨者為「錄黃」。凡事乾因革損益,而非法式所載者,論定而上之。諸司傳宣、特旨,承報審覆,然後行下。



 設官十有
 一:令、侍郎、右散騎常侍各一人,舍人四人,右諫議大夫、起居舍人、右司諫、右正言各一人。



 分房八、曰吏房,曰戶房,曰兵禮房,曰刑房,曰工房,曰主事房,曰班簿房,曰制敕庫房。元祐以後,析兵、禮為二,增催驅、點檢,分房十有一,後又改主事房為開拆。凡吏房,掌行除授、考察、升黜、賞罰、廢置、薦舉、假故、一時差官文書。曰戶房,掌行廢置升降郡縣、調發邊防軍須、給貸錢物。曰禮房,掌行郊祀陵廟典禮、后妃皇子公主大臣封冊、科舉考官、外夷書
 詔。曰兵房,掌行除授諸蕃國王爵、官封。曰刑房,掌行赦宥及貶降、敘復。曰工房,掌行營造計度及河防修閉。凡尚書省所上奏請、臺諫所陳章疏、內外臣僚官司申請無法式應取旨者,六房各視其名而行之。曰主事房,掌行受發文書。曰班簿房,掌百官名籍具員。曰制敕庫房,掌編錄供檢敕、令、格、式及架閣庫。曰催驅房,督趣稽違。曰點檢房,省察差失。吏四十有五:錄事三人,主事四人,令史七人,書令史十有四人,守當官十有七人。而外省
 吏十有九人:令史一人,書令史二人,守當官六人,守闕守當官十人。



 元豐八年,詔待制以上磨勘,本省進擬。元祐三年,詔應除授從中批付中書省者,並三省行。紹聖五年,詔臣僚上殿扎子,中書省進呈取旨;其承受傳宣、內降,非有司所可行者,申中書省或樞密院奏審。



 令掌佐天子議大政,授所行命令而宣之。祀大神祇則升壇,享宗廟則升阼階而相其禮。臨軒冊命則讀冊。建儲則升殿宣制,持冊及璽綬以授太子。大朝會則詣御
 坐前奏方鎮表及祥瑞。國朝未嘗真拜,以他官兼領者不預政事,然止曹佾一人,餘皆贈官。官制行,以右僕射兼中書侍郎行令之職,別置侍郎以佐之。中興後,置左、右丞相,省令不置。



 侍郎掌貳令之職,參議大政,授所宣詔旨而奉之。凡大朝會則押表及祥瑞案。臨軒冊命則押冊引案,以所奏文及冊書授令。四夷來朝則奏其表疏,以贄幣付有司。南渡後,復置參知政事,省中書侍郎不置。



 舍人四人,舊六人。掌行命令為制詞,分治六房,隨房當制,事有失當及除授非其人,則論奏封還詞頭。國初,為所遷官,實不任職,復置知制誥及直舍人院,主行詞命,與學士對掌內外制。凡有除拜,中書吏赴院納詞頭。其大除拜,亦有宰相召舍人面授詞頭者。若大誥命,中書並敕進入,從中而下,餘則發敕官受而出之。及修官制,遂以實正名,而判後省之事。分案五:曰上案,掌冊禮及朝會所行事;曰下案,掌受付文書;曰制誥案,掌書錄制
 詞及試吏,校其功過;曰諫官案,掌受諸司關報文書;曰記注案,掌錄記注。其雜務則隨所分案掌之。



 元豐六年,詔中書省置點檢房,令舍人通領。元祐元年,詔舍人各簽諸房文字,其命詞則輪日分草。九月,詔時暫闕官,依門下、尚書省例,送本省官兼權。紹聖四年,蹇序辰請自今命詞,以元行遣文書同檢送當制舍人。從之。建炎後同,他官兼攝者則稱權舍人,資淺者為直舍人院。



 起居舍人一人,掌同門下省起居郎。侍立修注官,元豐
 前,以起居郎、舍人寄祿,而更命他官領其事,謂之同修起居注。官制行,以郎、舍人為職任。淳熙十五年,羅點自戶部員外郎為起居舍人,避其祖諱,乃以為太常少卿兼侍立修注官。其後兩史或闕而用資淺者,則降旨以某人權侍立修注官。



 右散騎常侍右諫議大夫右司諫右正言與門下省同,但左屬門下,右屬中書,皆附兩省班籍,通謂之兩省官。元豐既新官制,職事官未有不經除授者,惟御
 史大夫、左右散騎常侍,始終未嘗一除人。蓋兩官為臺諫之長,無有啟之者。中興初,詔諫院不隸兩省。紹興二年,詔並依舊赴三省元置局處。淳熙十五年,用林慄言,置左右補闕、拾遺,專任諫正,不任糾劾之事。逾年減罷。法司令史、書令史、守當官各一人,守闕守當官三人,乾道六年減二人。



 檢正官五房各一人,掌糾正省務。熙寧三年置,以京朝官充,選人即為習學公事。官制行,罷之,而其職歸左右
 司。建炎三年,中書門下省言:「軍興以來,天下多事,中書別無屬官。元豐以前,有檢正官,後因置左右司,遂不差,致朝廷及應報四方行移稽留,無檢舉催促。今欲差官兩員充中書門下省檢正諸房公事。



 內一員檢正吏、禮、兵房,一員檢正戶、刑、工房。



 從之。至次年,詔並罷。紹興二年,詔中書門下省復置檢正官一員。



 建炎三年指揮,中書門下省並為一。中書省錄事、主事、令史、書令史、守當官共四十三人;門下省錄事、主事、令史、書令史、守當官共四十六人,依祖額
 以八十九人為額。守闕守當官兩省各一百人,共存留一百五十人,中書省六分,門下省四分。



 尚書省掌施行制命,舉省內綱紀程序,受付六曹文書,聽內外辭訴,奏御史失職,考百官庶府之治否,以詔廢置、賞罰。曰吏部,曰戶部,曰禮部,曰兵部,曰刑部,曰工部,皆隸焉。凡天下之務,六曹所不能與奪者,總決之;應取裁者,隨所隸送中書省、樞密院。事有成法,則六曹準式具鈔,令、僕射、丞檢察簽書,送門下省畫聞。審察吏部注
 擬文武官及封爵承襲、賜勛定賞之事。朝廷有疑事,則集百官議其可否。凡更改申明敕令格式、一司條法,則議定以奏覆,太常、考功謚議亦如之。季終,具賞罰勸懲事付進奏院,頒行於天下。大祭祀則誓戒執事官。



 設官九:尚書令、左右僕射、左右丞、左右司郎中、員外郎各一人。分房十:曰吏房,曰戶房,曰禮房,曰兵房,曰刑房,曰工房,各視其名而行六曹諸司所上之事;曰開拆郎,主受遣文書;曰都知雜房,主行進制敕目、班簿具員,考
 察都事以下功過遷補;曰催驅房,主考督文牘稽違;曰制敕庫房,主編檢敕、令、格、式,簡納架閣文書。置吏六十有四都事三人,主事六人,令史十有四人,書令史三十有五人,守當官六人。元豐四年,詔尚書都省及六曹,各輪郎官一員宿直。五年,詔得旨行下並用扎子。紹聖元年,詔在京官司所受傳宣、內降,隨事申尚書省或樞密院覆奏。二月,詔尚書都省彈奏六察御史,糾不當者。



 令掌佐天子議大政,奉所出命令而行之。其屬有六曹,
 凡庶務皆會而決之。凡官府之紀綱程序,無不總焉。大事三省通議,則同執政官合班;小事尚書省獨議,則同僕射、丞分班論奏。若事由中書、門下而有失當應奏者,亦如之。與三師、三公、侍中中書令俱以冊拜。自建降以來不除,惟親王元佐元儼以使相兼領,不與政事。政和二年,詔:「尚書令,太宗皇帝曾任,今宰相之官已多,不須置。」然是時說者以謂為令者唐太宗也,熙陵未嘗任此,蓋時相蔡京不學之過。宣和七年,詔復置令,亦虛設其
 名,無有除者。南渡後,並省不置。



 左僕射右僕射掌佐天子議大政,貳令之職,與三省長官皆為宰相之任。大祭祀則掌百官之誓戒,視滌濯告潔,贊玉幣爵玷之事。自官制行,不置侍中、中書令,以左僕射兼門下侍郎,右僕射兼中書侍郎,行侍中、中書令職事。政和中,詔曰:「昔我神考,訓迪厥官,有司不能奉承,仰惟前代以僕臣之賤,充宰相之任,可改左僕射為太宰,右僕射為少宰。」靖康元年,詔依元豐舊制,復為左、
 右僕射。南渡後,置左、右丞相,省僕射不置。



 左丞右丞掌參議大政,通治省事,以貳令、僕射之職。僕射輪日當筆,遇假故,則以丞權當筆知印。大祭祀酌獻,薦饌進熟,則受爵酒以授僕射。舊班六曹尚書下,官制行,升其秩為執政。元豐五年五月,詔左右僕射、丞合治省事。是月,御史言:「左、右丞蒲宗孟、王安禮於都堂下馬,違法犯分。」安禮爭論帝前,神宗是之。今左、右丞於都堂上下馬,自此始。南渡後,復置參知政事,省左、右丞不
 置。



 左司郎中右司郎中左司員外郎右司員外郎各一人,掌受付六曹之事,而舉正文書之稽失,分治省事:左司治吏、戶、禮、奏鈔、班簿房,右司治兵、刑、工、案鈔房,而開拆、制敕、御史、元豐六年,都司置御史房,主行彈糾御史案察失職。催驅、封椿



 印房,則通治之,有稽滯,則以期限舉催。初,於都司置吏設案,而議者謂臺郎宰掾不當自為官司。遂隨省房分治所領之事,惟置手分、書奏各四人,主行校定省
 吏都事以下功過及遷補之事。



 元豐七年,都司御史房置簿,以書御史、六曹官糾察之多寡當否為殿最,歲終取旨升黜。紹聖元年,詔都司以歲終點檢六曹稽違最多者,具郎官姓名上省取旨。二年,詔御史臺察六曹稽緩違失者,送左司籍記。宣和二年,左司員外郎王蕃奏:「都司以彌綸省闥為職,事無不預。今宰、丞入省,諸房文字填委,次第呈覆,自朝至於日中,或昏暮僅絕,其勢不暇一一檢閱細故,而省吏徑稟宰、丞
 請筆,以草檢令承從官齋赴郎官廳落日押字。」謂「宜遵守元豐及崇寧舊法,諸房各具簽帖,先都事自點檢,次郎官押訖,赴宰、丞請筆行下。」於是詔曰:「先帝肇正三省,詔給舍、都司以贊省務。今都司浸以曠官,緣省吏強悍,敢肆侵侮。自今違法事,其左右司官、尚書具事舉劾。」



 建炎三年,詔減左、右司郎官兩員,置中書門下省檢正諸房公事二員。至次年,檢正省罷,其左、右司郎官依舊四員。紹興三十二年,詔尚書省吏房、兵房,三省、樞密院機
 速房,尚書省刑房、戶房、工房,三省、樞密院看詳賞功房,尚書省禮房,令左、右司郎官四員從上分房書擬。隆興元年,詔左、右司郎官各差一員。乾道六年,詔榷貨務都茶場依建炎三年指揮,委都司官提領措置。乾道七年,復添置右司郎官二人。



 榷貨務都茶場,都司提領。



 提轄官一員,京朝官充。



 監場官二員,京選通差。



 掌鹺、茗、香、礬鈔引之政令,以通商賈、佐國用。舊制,置務以通榷易。建炎中興,又置都茶場,給賣茶引,隨行在所榷貨務置場雖分兩司,而
 提轄官、監官並通銜管幹。外置建康、鎮江務場,並冠以行在為名,以都司提領,不系戶部經費。建康、鎮江續分隸總領所。開禧初,以總領所侵用儲積錢,令徑隸提領所。乾道七年,提領所置乾辦官一員。



 右提轄官與雜買務雜賣場、文思院、左藏東西庫提轄,並稱四轄。外補則為州,內遷則為寺監丞、簿,亦有徑為雜臨司,或入三館。



 乾道間,榷務王禋除市舶,左藏王揖除坑冶鑄錢司,淳熙間,熊克自文思除校書郎。



 紹熙以後,往往更遷六院官,或出為添倅,有先後輕重之異焉。



 左
 藏封樁庫,都司提領。



 監官一員,監門官一員。淳熙九年,以都司提領。初創,非奉親與軍須不支。後或撥入內庫,或以供宮廷諸費,亦以備振恤之用。



 提舉修敕令自熙寧初,編修《三司令式》,命宰臣王安石提舉,是後,皆以宰執為之。詳定官,以侍從之通法令者充,舊制二員。宣和中,增至七員。靖康初,減為三員。刪定官,無常員。先是,嘗別修一司敕命。大觀三年,詔六曹刪定官並入詳定一司敕令所,為一
 局。



 制置三司條例司掌經畫邦計,議變舊法以通天下之利。熙寧二年置,以知樞密院陳升之參知政事王安石為之,而蘇轍、程顥等亦皆為屬官。未幾,升之相,乃言:「條例者有司事爾,非宰相之職,宜罷之。」帝欲並歸中書,安石請以樞密副使韓絳代升之焉。三年,判大名府韓琦言:「條例司雖大臣所領,然止是定奪之所。今不關中書而徑自行下,則是中書之外又有一中書也。」五月,罷歸中書。



 三司會計司熙寧七年,置於中書,以宰相韓絳提舉。先是,絳言總天下財賦,而無考較盈虛之法,乃置是司。既而事多濡滯,八年,絳坐此罷相,局亦尋廢。



 編修條例司熙寧初置,八年罷。



 經撫房專治邊事。宣和四年,宰臣王黼主伐燕之議,置於三省,不復以關樞密院。六年,罷。



 提舉講議司崇寧元年七月,詔如熙寧條例司故事,都省置講議司。以宰相蔡京提舉,侍從為詳定官,卿監為
 參詳官;又置檢討官,凡宗室、冗官、國用、商旅、鹽鐵、賦調、尹牧,每一事各三人主之。



 時又分武備一房,別為樞密院講議司。三年三月,知樞密院事蔡卞奏罷。



 三年四月結局。宣和六年,又於尚書省置講議司。十二月,命太師致仕蔡京兼領,聽就私第裁處,仍免簽書。



 議禮局大觀元年,詔於尚書省置,以執政兼領。詳議官二員,以兩制充。應凡禮制本末,皆議定取旨。政和三年,《五禮儀注》成,罷局。



 禮制局討論古今宮室、車服、器用、、冠昏、喪祭沿革制度。政和二年,置於編類御筆所,有詳議、同詳議官,宣和二年,詔與大晟府制造所協聲律官並罷。



\end{pinyinscope}