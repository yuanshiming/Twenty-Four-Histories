\article{列傳第二百四十三周三臣}

\begin{pinyinscope}

 周三臣○韓通李筠李重進



 《五代史記》有《唐六臣傳》,示譏也。《宋史》傳周三臣,其名似之,其義異焉,求所以同,則歸於正名義、扶綱常而已。韓
 通與宋太祖比肩事周,而死於宋未受禪之頃,然不傳於宋,則忠義之誌何所托而存乎?李筠、李重進舊史書叛,叛與否未易言也,洛邑所謂頑民,非殷之忠臣乎?孔子定《書》,不改其舊稱焉。或曰:三人者嚐臣唐、晉、漢矣。曰:智氏之豫讓非歟!作《周三臣傳》。



 韓通,并州太原人。弱冠應募,以勇力聞,補騎軍隊長。晉開運末,漢祖建義於太原,置通帳下。尋從漢祖至東京,累遷為軍校。漢祖典衛兵,以通為衙隊副指揮使,從討
 杜重威,得銀青階,檢校國子祭酒。漢祖開國,加檢校左仆射。隱帝即位,遷奉國指揮使。



 乾祐初,周祖為樞密使,統兵伐河中。知通謹厚,命之自隨,先登,身被六創,以功遷本軍都虞候。周祖鎮大名,奏通為天雄軍馬步軍都校,委以心腹,及入汴,通甚有力焉。授奉國左第六軍都校,領雷州刺史。



 廣順初,為虎捷右廂都校,遷左廂,充孟州巡檢,繼領永、睦二州防禦使。周祖親征兗州,以通為在京右廂都巡檢。時河溢,灌河陰城,命通率廣銳卒千
 二百浚汴口,又部築河陰城,創營壁。未幾,拜保義軍節度觀察留後,周祖親郊,正授節度。并州劉崇南侵,命通副河中王彥超出晉州道擊之,敗於高平。以通為太原北面行營部署,為地道攻其城。俄班師,移鎮曹州,檢校太保。



 世宗即位,以深、冀之間有胡蘆河,東西橫亙數百里,堤堨非峻,不能扼契丹奔突。顯德二年,命通與王彥超浚治之,功未就,契丹至,通出兵迎擊退之,遂城李晏口為靜安軍,四旬而完。又城束鹿及鼓城,並葺祁州。時
 大兵之後,遺骸布野,通悉收瘞為萬人塚。又城博野、安平,往來深、定間,夜宿古寺,晝披荊棘。在安平領百餘騎督役,會契丹騎數百奄至,通率麾下與戰,日暮大風雨,契丹解去,擒十餘騎。又城百八橋鎮及武強縣,皆旬日畢。歸朝,會攻秦、鳳,以通為西南面行營馬步軍都虞候,入大散關,圍鳳州,分兵城固鎮,以斷蜀餉道。未幾,拔鳳州,以功授侍衛馬步軍都虞候。



 世宗征淮南,命通為京城都巡檢。世宗以都城狹小,役畿甸民築新城,又廣舊
 城街道。命左龍武統軍薛可信、右衛上將軍史佺、右監門衛上將軍蓋萬、右羽林將軍康彥環分督四面,通總領其役。功未就,世宗幸淮上,留通為在京內外都巡檢、權點檢侍衛司。是役也,期以三年,才半歲而就。三年,追敘秦、鳳功,改領忠武軍節度、檢校太傅,又改侍衛馬步軍都虞候。世宗幸壽春,為京城內外都巡檢。淮南平,為歸德軍節度。



 六年春,詔通河北按行河堤,因發徐、宿、宋、單等州民浚汴渠數百里。世宗將北征,命通與高懷德、
 張鐸先赴滄州,賜襲衣、金帶、鞍馬、器帛。即領兵入契丹境乾寧軍之南。俄為陸路都部署,殿前都虞候石守信副焉。又命通巡北邊,自浮陽至淤口浦壞坊三十六,遂通瀛、莫。初克益津關,以為霸州,役濱、棣民數千城之,命通董其役。師還,以為檢校太尉、同平章事,充侍衛親軍馬步軍副都指揮使。恭帝即位,移領鄆州。



 太祖奉詔北征,至陳橋為諸軍推戴。通在殿閣,聞有變,惶遽而歸。軍校王彥升遇通於路,策馬逐之,通馳入其第,未及闔門,
 為彥升所害,妻子皆死。太祖聞通死,怒彥升專殺,以開國初,隱忍不及罪。即下詔曰:「易姓受命,王者所以應期;臨難不苟,人臣所以全節。故周天平軍節度、檢校太尉、同中書門下平章事、侍衛親軍馬步軍副指揮使韓通,振跡戎伍,委質前朝,彰灼茂功,踐更勇爵。夙定交於霸府,遂接武於和門,艱險共嚐,情好尤篤。朕以三靈眷祐,百姓樂推,言念元勳,將加殊寵,蒼黃遇害,良用憮然。可贈中書令,以禮收葬。遣高品梁令珍護喪事。」



 通性剛而
 寡謀,言多忤物,肆威虐,眾謂之「韓瞠眼」。其子頗有智略,幼病傴,人目為「橐駝兒」。見太祖有人望,常勸通早為之所,通不聽。後太祖幸開寶寺,見通及其子畫像於壁,遽命去之。



 李筠,并州太原人。善騎射。後唐秦王從榮判六軍諸衛,募勇士為爪牙,筠操弓矢求見。弓力及百斤,府中無能挽者,從榮令筠射,引滿有餘力,再發皆中,因以隸麾下。從榮難作,筠騎從至天津橋,射殺十數人,知事不濟,棄
 馬遁去。清泰初,應募為內殿直,遷控鶴指揮使。



 晉開運末,契丹犯汴京,其將趙延壽聞筠驍勇,召置帳下。及契丹主北歸,死欒城,延壽至常山,為永康王所縶。契丹眾數萬,據常山,後北去。留耶律解裏,眾才二千騎,又分別部首領楊袞以千騎掠邢、洺。來還中朝士大夫多在城中,契丹與漢相雜,解裏性貪恣自奉,削漢軍日食,眾皆菜色。筠乘其怨,密與王蕘、石公霸、何福進等謀,以閏七月二十九日伺契丹守閽者旦食,撞寺鍾為期,相率入
 據兵庫,次焚牙門,大呼市人,並力擊焉。契丹眾大驚,由北門而出,解裏趣族乘列之於野。明日集眾入郛力戰,屬晉士卒分掠,唯控鶴一軍與市民禦之,死傷相繼。午後,郛外民千餘知契丹奔敗者,持兵趣其族乘,將劫之,守者入郛馳告,解裏聞之,遂挈族而去。初,筠建謀約諸將同力,控鶴左廂都校白再榮首匿於室不敢應,筠拔佩刀破幕引臂迫之,再榮不得已而行,諸將次第赴之。及契丹去,百姓死者二千餘人。諸將互伐其功,筠詣故
 相馮道請權領節度事,道曰:「子主奏事而已,留後事當議功臣為之。」道恐諸將爭功復亂,乃以再榮前職貴加諸將,權推為留後,人心遂定。是戰,筠功居多,即送款漢祖,以其子赴朝,漢祖深賞之。以控鶴一軍力戰,優加賜與,授再榮留後,筠博州刺史。筠以賞薄不悅。



 周祖鎮大名,表為先鋒指揮使,又為北面緣邊巡檢。周祖起兵入汴,筠同郭崇從,與慕容彥超戰於留子陂,彥超東奔。廣順初,權知滑州,俄真拜義成軍節度。數月,改彰德軍節
 度。會並人侵晉州,王峻率師往拒,筠亦請西征,詔褒之。又乞免黃澤關商稅,奏可。周祖征兗,還次濮,筠因朝,獻馬,賜襲衣、金帶。從至澶,宴訖遣還。及召潞州常思入朝,命筠權知軍府,思改宋、亳,以筠為昭義軍節度。三年,加檢校太傅。時王峻兼節製,以筠及王殷、何福進皆創業功臣,故並加恩焉。顯德初,周祖親郊,加同平章事。



 世宗即位,並人入侵,其將張暉率先鋒自團柏穀入營梁侯驛,攻劫堡柵,所至焚略蕩盡,筠遣護軍穆令均率步騎
 二千拒之。令均營於太平驛,驛東南距潞八十里,失於偵邏,暉淩晨奄至,潞兵被甲介馬,暉見之佯退,潞兵追之,並伏遂發。令均且鬥且卻,步卒降並者數百人,騎不復者百人,餘眾還保潞。世宗親征沁州,降之,命筠率沁之行營兵赴太原,符彥卿戍忻口,拒契丹援兵。彥卿請益師,詔筠與張永德以三千騎益之,既至,以偏師繞契丹後,奮擊走之。師還,加兼侍中。



 二年,筠破並軍於榆社,獲其將安濬、康超等七十餘人。三年,筠遣行軍司馬範
 守圖率兵入遼州界,殺並卒三百餘,獲小校數人以獻。四年,又遣守圖入河東界,降二砦。五年,筠自將入石會關,破並人六砦。是冬,又破遼州長清砦,擒其磁州刺史李戴興以獻。俄又敗並人於境,斬三百餘級。六年,平遼州,獲刺史張丕旦等二百四十五人以獻。筠在鎮擅用征賦,頗集亡命,嚐以私忿囚監軍使,世宗心不能堪,但詔責而已。恭帝即位,加檢校太尉。是秋,令裨將劉繼忠將兵與吐渾入並境,平賈家砦,斬百餘級,獲牛羊而還。



 太祖建隆初,加兼中書令,遣使諭以受周禪。筠即欲拒命,左右為陳曆數,方僶俛下拜,貌猶不恭。及延使者升階,置酒張樂,遽索周祖畫像懸壁,涕泣不已。賓佐惶駭,告使臣曰:「令公被酒失其常性,幸勿為訝。」及太原劉鈞以蠟書結筠共舉兵,筠雖緘書上太祖,心已畜異謀,太祖手詔慰撫之。是時,筠子守節為皇城使,嚐泣諫,筠不聽。太祖又遣守節諭旨曰:「吾聞汝諫汝父,汝父不聽,吾今殺汝,何如汝歸語汝父,我未為天子時,任自為之,既
 為天子,獨不能臣我耶?」守節白筠,筠謀愈甚,遂起兵,令幕府為檄書,辭多不遜。從事閭丘仲卿獻策於筠曰:「公以孤軍舉事,其勢甚危,雖倚河東之援,亦恐不得其力。大梁兵甲精銳,難與爭鋒。不如西下太行,直抵懷、孟,塞虎牢,據洛邑,東向而爭天下,計之上也。」筠曰:「吾周朝宿將,與世宗義同昆弟,禁衛皆舊人,聞吾之來,必倒戈歸我,況有儋珪槍、撥汗馬,何憂天下哉。」儋珪,筠愛將,有勇力,善用槍;撥汗,筠駿馬,日馳七百里,故筠誇焉。執監軍
 亳州防禦使周光遜、閑廄使李廷玉,遣判官孫孚、衙校劉繼忠送於劉鈞求濟師。又遣人殺澤州刺史張福,往據其城。



 劉鈞遂率兵與契丹數千眾來援,至太平驛,筠以臣禮迎謁,見鈞兵衛寡弱,甚悔之,而業已然矣。鈞封筠西平王,賜馬三百匹,召與之語,筠自言受周祖大恩,敢愛死不寤。鈞與周祖有世仇,鈞默然,遂疑之。命其宣徽使盧讚監筠軍,筠心不能平,頗與讚不協,鈞復命平章事衛融和解之。



 筠有馬三千匹,辟鞠場閱習,日夜謀
 畫為寇。留其子守節守上黨,引眾南向。太祖遣石守信、高懷德將兵討之。敕曰:「勿縱筠下太行,急進師扼其隘,破之必矣。」又遣慕容延釗、王全斌由東路會守信,與監軍李崇矩破筠眾於長平,斬首三千級。又攻大會砦,下之。



 太祖遂親征。山路險峻多石不可行,太祖先於馬上負數石,群臣六軍皆負之,即日平為大道。與守信、懷德會,破筠眾三萬於澤南,降者三千餘,殺筠監軍使盧讚,擒筠河陽節度範守圖,筠走還保澤。太祖至,列柵圍之,
 筠龍捷使王廷魯、吐渾留後汾州團練使王全德率所部自昭義來降,筠益失援。太祖親督戰,拔其城,筠赴水死,獲鈞相衛融,鈞懼而遁歸。太祖進伐上黨,守節以城降,釋其罪,賜襲衣、金帶、銀鞍勒馬。是日宴從官,守節預焉,以為單州團練使;以昭義軍節度副使趙處願為郢州刺史;節度判官孫孚為屯田郎中;觀察判官史文通為水部郎中;前遼州衙內指揮使馬廷禹為右監門衛將軍,領壁州刺史。



 筠性雖暴,事母甚孝,每怒將殺人,母
 屏風後呼筠,筠趨至,母曰:「聞將殺人,可免乎?為吾曹增福爾。」筠遽釋之。筠稍知書,頗好調謔。初名榮,避周世宗諱,將改之,或令名「筠」,筠曰:「李筠,李筠,玉帛雲乎哉。」聞者皆笑。



 筠有愛妾劉氏,隨筠至澤,時被攻城危,劉謂筠曰:「城中健馬幾何?」筠曰:「爾安問此?」劉曰:「孤城危蹙,破在俄頃,今誠得馬數百,與腹心潰圍,出保昭義,求援河東,猶愈於坐待死也。」筠然之。召左右計馬尚不減千匹,以是夕將出,或謂筠曰:「今帳前計議,皆云一心,縣門既發,不
 可保矣,倘劫公而降,悔其可及。」筠猶豫不決。明日城陷,筠將赴火,劉欲俱死,筠以其有娠,麾令去。守節既購得之,果生子焉。



 字節字得臣,初補東頭供奉官。廣順中,嚐以心疾乘醉擊殺供禦白鶻,筠上章待罪,詔釋之。四遷至皇城使,曆單、濟二州團練使。乾德六年,出知遼州。開寶三年,改和州團練使。四年,卒,年三十三。無後,以劉氏所生之弟為嗣。



 李重進,其先滄州人。周太祖之甥,福慶長公主之子也,
 生於太原。晉天福中,仕為殿直。漢初,從周祖征河中。廣順初,遷內殿直都知,領泗州刺史,改小底都指揮使。二年,改大內都點檢、權侍衛馬步軍都軍頭,領恩州團練使,遷殿前都指揮使。三年,加領泗州防禦使。顯德初,領武信軍節度。



 重進年長於世宗,及周祖寢疾,召重進受顧命,令拜世宗,以定君臣之分。世宗嗣位,為侍衛親軍馬步軍都虞候。從世宗征劉崇,戰於高平,不利,大將樊愛能、何徽以其眾遁,唯重進與白重讚勒兵不動。既而
 太祖先以麾下犯敵,重讚繼領所部力戰,世宗躬率衛兵合勢,周師復振,崇遂大敗。以功領忠武軍節度。及進討太原,又為行營馬步軍都虞候。師還,加同中書門下平章事,改歸德軍節度兼侍衛馬步軍都指揮使。



 世宗親征淮南,命重進將兵先赴正陽。俄聞李穀攻壽春不克,退保正陽,促重進兵助之。吳人以穀退為懼,乃發兵三萬餘,旌旗輜重亙數百里;又發戰棹二百艘以張斷橋之勢,列陣鼓噪而北,橫布拒馬以萬數,皆貫以利刃,
 維以鐵索;又刻木為戰形,立陣前,號「揵馬牌」,皮囊貯鐵蒺莉以布戰地。時周師未朝食,吳師奄至,周師望其陣皆笑之。宣祖領前軍與重進、韓令坤合勢擊之,一鼓而敗,斬首萬餘級,追奔二十餘里,殺大將劉彥貞,擒裨將盛師朗數十人,降三千人,獲戈甲三十萬。世宗大悅,詔書褒諭,即以重進代穀為行營招討使,賜襲衣、金帶、玉鞍、名馬。



 三年,以重進為廬、壽等州招討使。時李繼勳主壽春,重進駐軍城北,聞城南洞屋為淮人所焚,將議退
 軍。會太祖自六合歸,道出壽州,因駐師旬餘,重進倚以為援,兵威復振。吳人大懼,以重進色黔,號「黑大王」。



 張永德屯下蔡,與重進不協。永德每宴將吏,多暴重進短,後乘醉謂重進有奸謀,將吏無不驚駭。永德密遣親信乘驛上言,世宗不之信,亦不介意。二將俱握重兵,人情益憂恐。重進遂自壽陽單騎直詣永德帳中,命酒飲,親酌謂永德曰:「吾與公皆國家肺腑,相與戮力,同獎王室,公何疑我之深也。」永德意解,二軍皆安。李景知之,密令人
 齎蠟書誘重進,啖以厚利,重進表其事。時行濠州刺史齊藏珍亦說重進,世宗知之,假他事誅藏珍。



 詔重進夾淮城正陽、下蔡,既成,上其圖。俄又敗淮兵二千餘於塌山北。時圍壽經年未下,吳遣將許文緽、邊鎬舟師數萬,溯淮來援。文緽維舟淮南,據紫金山,山距壽數里,設十餘砦,連亙相望,與城中烽火相應;又南築夾道,將抵壽為饋路。重進伺其城北展砦,出兵擊之,敗五千餘眾,奪二砦,獲器甲甚眾。世宗幸壽,宴從官,召重進賜戎服、玉
 帶、金銀器、繒彩、鞍勒馬。及克壽,錄功加檢校太傅兼侍中,又改天平軍節度,仍為招討使。



 四年,攻取濠州南關城,其團練使郭廷謂以兵萬餘降,獲糧數萬斛。從平楚州,命先還揚州。五年,世宗在迎鑾,遣重進將兵赴廬州。會李景請畫江為界,世宗遂還,留重進戍守,景遣人以牛酒來犒,俄乃還鎮。六年,世宗北征,次博州,重進來朝,賜宴行宮,即命將兵先趣北面,及世宗駐瓦橋關,重進與諸將率師而至。時關南已平,議進取幽州,會世宗不
 豫而止。即命率所部赴河東,次百井路,敗並人五千餘,斬二千餘級。恭帝嗣位,加檢校太尉,改淮南道節度。



 太祖即位,以韓令坤代為侍衛都指揮使,加重進中書令。既而移鎮青州,加開府階。重進與太祖俱事周室,分掌兵柄,常心憚太祖。太祖立,愈不自安,及聞移鎮,陰懷異志。太祖知之,遣六宅使陳思誨齎賜鐵券,以安其心。重進欲治裝隨思誨入朝,為左右所惑,猶豫不決。又自以周室近親,恐不得全,遂拘思誨,治城隍,繕兵甲,遣人求
 援李景,景懼而不納,聞之太祖。監軍安友規常為重進所忌,至是友規謀與親信數人斬關出,為眾所拒,逾城得脫。重進捕軍校不附者數十人,盡殺之。



 太祖遣石守信、王審琦、李處耘、宋偓四將率禁兵討重進。會友規至,賜襲衣、金帶、器幣、鞍馬,以為滁州刺史,監前軍。太祖謂左右曰:「朕於周室舊臣無所猜間,重進不體朕心,自懷反側,今六師在野,當暫往慰撫之爾。」遂親征,次大儀鎮。石守信遣使馳奏,揚州破在旦夕,願車駕臨視。太祖徑
 至城下,即日拔之。初,城將陷,重進左右勸殺思誨,重進曰:「吾今舉族將赴火死,殺此何益。」即縱火自焚,思誨亦為其黨所害。太祖入駐城西南,閱逆黨數百人,盡戮之。重進兄深州刺史重興,聞其叛,自殺。弟解州刺史重讚、子尚食使延福並戮於市。



 初,重進謀舉兵,遣親吏翟守珣往潞,陰結李筠。守珣素識太祖,往還京師,潛詣樞密承旨李處耘求見,太祖問曰:「我欲賜重進鐵券,彼信我乎?」守珣曰:「重進終無歸順之誌。」



 太祖厚賜守珣,許以爵
 位,且令說重進緩其謀,無令二凶並作,以分兵勢。守珣歸,勸重進養威持重,未可輕發,重進甚信之。及李筠誅,重進反書聞,並如太祖之策,其不信鐵券,亦如守珣所雲。揚州既平,購得守珣,補殿直,俄為供奉官。



 又有張崇詁者,周廣順初,為樞密承旨。二年,出為解州刺史、兩池權鹽使,多規畫鹽池利害。顯德三年,改德州,又改泗州、澤州。崇詁本名崇訓,恭帝嗣位,避諱改焉。重進赴淮南時,道出泗上,崇詁說以畜兵完城之計。重進敗,事露,詔
 捕之,棄市,籍其家。



\end{pinyinscope}