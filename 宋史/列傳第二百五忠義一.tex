\article{列傳第二百五忠義一}

\begin{pinyinscope}

 ○康保裔馬遂董元亨曹覲孔宗旦趙師旦蘇緘秦傳序詹良臣江仲明李若水劉韐傅察楊震父宗閔張克戩
 張確朱昭史抗孫益



 士大夫忠義之氣,至於五季,變化殆盡。宋之初興,範質、王溥,猶有餘憾,況其他哉!藝祖首褒韓通,次表衛融,足示意向。厥後西北疆場之臣,勇於死敵,往往無懼。真、仁之世,田錫、王禹偁、範仲淹、歐陽修、唐介諸賢,以直言讜論倡於朝,於是中外搢紳知以名節相高,廉恥相尚,盡去五季之陋矣。故靖康之變,志士投袂,起而勤王,臨難不屈,所在有之。及宋之亡,忠節相望,班班可書,匡直輔
 翼之功,蓋非一日之積也。



 奉詔修三史,集儒臣議凡例,前代忠義之士,咸得直書而無諱焉。然死節、死事,宜有別矣:若敵王所愾,勇往無前,或銜命出疆,或授職守土,或寓官閑居,感激赴義,雖所處不同,論其捐軀徇節,之死靡二,則皆為忠義之上者也;若勝負不常,陷身俘獲,或慷慨就死,或審義自裁,斯為次矣;若蒼黃遇難,霣命亂兵,雖疑傷勇,終異茍免,況於國破家亡,主辱臣死,功雖無成,志有足尚者乎!若夫世變淪胥,毀跡冥遁,能以
 貞厲保厥初心,抑又其次歟!至於布衣危言,嬰鱗觸諱,志在衛國,遑恤厥躬,及夫鄉曲之英,方外之傑,賈勇蹈義,厥死惟鈞。以類附從,定為等差,作《忠義傳》。



 康保裔,河南洛陽人。祖志忠,後唐長興中,討王都戰沒。父再遇,為龍捷指揮使,從太祖征李筠,又死於兵。保裔在周屢立戰功,為東班押班,及再遇陣沒,詔以保裔代父職,從石守信破澤州。明年,攻河東之廣陽,獲千餘人。開寶中,又從諸將破契丹於石嶺關,累遷日騎都虞候,
 轉龍衛指揮使,領登州刺史。端拱初,授淄州團練使,徙定州、天雄軍駐泊部署。尋知代州,移深州,又徙高陽關副都部署,就加侍衛馬軍都虞候,領涼州觀察使。真宗即位,召還,以其母老勤養,賜以上尊酒茶米。俄領彰國軍節度,出為並代都部署,徙知天雄軍,並代列狀請留,詔褒之,復為高陽關都部署。



 契丹兵大入,諸將與戰於河間,保裔選精銳赴之,會暮,約詰朝合戰。遲明,契丹圍之數重,左右勸易甲馳突以出,保裔曰:「臨難無茍免。」遂
 決戰。二日,殺傷甚眾,蹴踐塵深二尺,兵盡矢絕,援不至,遂沒焉。



 時車駕駐大名,聞之震悼,廢朝二日,贈侍中。以其子繼英為六宅使、順州刺史,繼彬為洛苑使,繼明為內園副使,幼子繼宗為西頭供奉官,孫惟一為將作監主簿。繼英等奉告命,謝曰:「臣父不能決勝而死,陛下不以罪其孥幸矣,臣等顧蒙非常之恩!」因悲涕伏地不能起。上惻然曰:「爾父死王事,贈賞之典,所宜加厚。」顧謂左右曰:「保裔父、祖死疆場,身復戰沒,世有忠節,深可嘉也。」
 保裔有母年八十四,遣使勞問,賜白金五十兩,封為陳國太夫人,其妻已亡,亦追封河東郡夫人。



 保裔謹厚好禮,喜賓客,善騎謝,弋飛走無不中。嘗握矢三十,引滿以射,筈鏑相連而墜,人服其妙。屢經戰陣,身被七十創。貸公錢數十萬勞軍,沒後,親吏鬻器玩以償,上知之,乃復厚賜焉。



 繼英仕至左衛大將軍、貴州團練使,嚴於馭軍,厚於撫宗族,其卒也,家無餘財。



 方保裔及契丹血戰,而援兵不至,惟張凝以高陽關路鈴轄領先鋒,李重貴以
 高陽關行營副都部署率眾策應,遇契丹兵交戰,保裔為敵所覆,重貴與凝赴援,腹背受敵,自申至寅力戰,敵乃退。當時諸將多失部分,獨重貴、凝全軍還屯,凝議上將士功狀,重貴喟然曰:「大將陷沒,而吾曹計功,何面目也。」上聞而嘉之。重貴仕至知鄭州,領播州防禦使,改左羽林軍大將軍致仕。凝加殿前都虞候,卒,贈彰德軍節度使。



 馬遂,開封人。初隸龍衛軍,補散直,改三班奉職,為北京
 指使。聞王則叛,中夜叱吒,晨起詣留守賈昌朝請擊賊。昌朝因使持榜入貝州招降,則盛服見之,遂諭以禍福,輒不答。遂將殺則,而無兵仗自隨。時張得一在側,欲其助己,目得一,得一不動。遂奮起,投杯抵則,扼其喉,驅之流血,而左右卒無助之者。賊黨攢刃聚噪至,斷一臂,猶詈則曰:「妖賊,恨不斬汝萬段!」賊縛遂廳事前,支解之。則倉猝被驅駭,傷病數日乃起。



 事聞,仁宗嘆息久之,贈宮苑使,封其妻為旌忠縣君,賜冠帔,官其子五人。後得殺
 遂者驍捷卒石慶,使其子剖心而祭之。



 董元亨,深州束鹿人。累官至國子博士,通判貝州。王則據城叛,是日冬至,元亨方與州將張得一朝謁天慶觀,夜漏未盡,變起倉猝,眾莫知所為。元亨促馬馳還,坐廳事,賊黨十餘人擐甲露刃,排闥而入,左右皆奔潰。賊脅元亨曰:「大王遣我來索軍資庫鑰。」元亨據案叱之曰:「大王誰也,妖賊乃敢弄兵乎!我有死耳,鑰不可得也。」賊將郝用繼來,索愈急,曰:「庫帑,今日大王所有也,可不上鑰
 乎!」元亨厲聲張目罵賊,用遂殺之,賊爭入,攜鑰而去。事聞,仁宗曰:「守法之臣也。」贈太常少卿,錄其子孫三人。賊平,獲郝用,斬以祭元亨。



 曹覲,字仲賓,曹修禮子也。叔修古卒,無子,天章閣待制杜杞為言於朝,授覲建州司戶參軍,為修古後。皇祐中,以太子中舍知封州。儂智高叛,攻陷邕管,趨廣州。行至封州,州人未嘗知兵,士卒才百人,不任戰鬥,又無城隍以守,或勸覲遁去,覲正色叱之曰:「吾守臣也,有死而已,
 敢言避賊者斬。」麾都監陳曄引兵迎擊賊,封川令率鄉丁、弓手繼進。賊眾數百倍,曄兵敗走,鄉丁亦潰。覲率從卒決戰不勝,被執。賊戒勿殺,捽使拜,且誘之曰:「從我,得美官,付汝兵柄,以女妻汝。」覲不肯拜,且詈曰:「人臣惟北面拜天子,我豈從爾茍生邪!速殺我,幸矣。」賊猶惜不殺,徙置舟中,覲不食者兩日,探懷中印章授其從卒曰:「我且死,若求間道以此上官。」賊知其無降意,害之。至死詬賊聲不絕,投尸江中,時年三十五。事聞,贈太常少卿,錄
 其子四人,妻劉避賊死於林峒,追封彭城郡君,加賜冠帔。又贈修古尚書工部侍郎,封修古妻陳潁川郡君。



 當智高之反,乘嶺南無備,州縣吏往往望風竄匿,故賊所向輒下,獨覲與孔宗旦、趙師旦能以死守。後田瑜安撫廣南,乃為覲立廟封州。



 孔宗旦,魯人,為邕州司戶參軍。儂智高未反時,州有白氣出庭中,江水溢,宗旦以為兵象,度智高必反,以書告知州陳珙,珙不聽。後智高破橫州,即載其親往桂州,曰:「
 吾有官守,不得去,無為俱死也。」既而州破被執,賊欲任以事,宗旦叱賊,且大罵,遂被害。始,宗旦官京東,與李師道、徐程、尚同等四人為監司耳目,號為「四瞠」,人多惡之,其後立節如此。知袁州祖無擇以其事聞,贈太子中允。



 趙師旦字潛叔,樞密副使稹之從子。美容儀,身長六尺。少年頗涉書史,尤刻意刑名之學。用稹蔭,試將作監主簿,累遷寧海軍節度推官。知江山縣,斷治出己,吏不能得民一錢,棄物道上,人無敢取。以薦者改大理寺丞、知
 彭城縣,遷太子右贊善大夫,移知康州。



 儂智高破邕州,順流東下,師旦使人覘賊,還報曰:「諸州守皆棄城走矣!」師旦叱曰:「汝亦欲吾走矣。」乃大索,得諜者三人,斬以徇。而賊已薄城下,師旦止有兵三百,開門迎戰,殺數十人。會暮,賊稍卻,師旦語其妻,取州印佩之,使負其子以匿,曰:「明日賊必大至,吾知不敵,然不可以去,爾留,死無益也。」遂與監押馬貴部士卒固守州城。召貴食,貴不能食,師旦獨飽如平時;至夜,貴臥不安席,師旦即臥內大鼾。
 遲明,賊攻城愈急,左右請少避,師旦曰:「戰死與戮死何如?」眾皆曰:「願為國家死。」至城破無一人逃者。矢盡,與貴俱還,據堂而坐。智高麾兵鼓噪爭入,脅師旦,師旦大罵曰:「餓獠,朝廷負若何事,乃敢反邪!天子發一校兵,汝無遺類矣。」智高怒,並貴害之。賊既去,州人為立廟。事平,贈光祿少卿,賜其母王長安縣太君冠帔,錄其子弟並從子三人。師旦遇害時,年四十二。柩過江山,江山之人迎師旦喪,哭祭於路,絡繹數百里不絕。



 同時有王從政者,
 以東頭供奉官、閣門祗候,與儂智高戰於太平場,被執,罵賊不已,至以沸湯沃之,終不屈而死。贈信州刺史,錄其孫二人。



 蘇緘,字宣甫,泉州晉江人。舉進士,調廣州南海主簿。州領蕃舶,每商至,則擇官閱實其貲,商皆豪家大姓,習以客禮見主者,緘以選往,商樊氏輒升階就席,緘詰而杖之。樊訴於州,州召責緘,緘曰:「主簿雖卑,邑官也,商雖富,部民也,邑官杖部民,有何不可?」州不能詰。再調陽武尉,
 劇盜李囊橐於民,賊曹莫能捕。緘訪得其處,萃眾大索,火旁舍以迫之。李從中逸出,緘馳馬逐,斬其首送府。府尹賈昌朝驚曰:「儒者乃爾輕生邪!」累遷秘書丞,知英州。



 儂智高圍廣,緘曰:「廣,吾都府也,且去州近,今城危在旦暮而不往救,非義也。」即募士數千人,委印於提點刑獄鮑軻,夜行赴難,去廣二十里止營。廣人黃師宓陷賊中,為之謀主,緘擒斬其父。群不逞並緣為盜,復捕殺六十餘人,招其詿誤者六千八百人,使復業。賊勢沮,將解去,
 緘分兵先扼其歸路,布槎木亙四十里。賊至不得前,乃繞出數舍渡江,由連、賀而西。緘與賊戰,摧傷甚眾,盡得其所掠物。時諸將皆罷,獨緘有功,仁宗喜,換為供備庫副使、廣東都監,管押兩路兵甲,遣中使賜朝衣、金帶。襲賊至邕,大將陳曙以失律誅,緘亦貶房州司馬。復著作佐郎,監越州稅十餘年,始還副使。知廉州,屋多茅竹,戍卒楊禧醉焚營,延燒民廬,因乘以為竊,緘戮之於市,又坐謫潭州都監。未幾,知鼎州。



 熙寧初,進如京使、廣東鈴
 轄。四年,交阯謀入寇,以緘為皇城使知邕州。緘伺得實,以書抵知桂州沈起,起不以為意。及劉彞代起,緘致書於彞,請罷所行事。彞不聽,反移文責緘沮議,令勿得輒言。八年,蠻遂入寇,眾號八萬,陷欽、廉,破邕四砦。緘聞其至,閱州兵得二千八百,召僚吏與郡人之材者,授以方略,勒部隊,使分地自守。民驚震四出,緘悉出官帑及私藏示之曰:「吾兵械既具,蓄聚不乏,今賊已薄城,宜固守以遲外援。若一人舉足,則群心搖矣,幸聽吾言,敢越佚
 則孥戮汝。」有大校翟績潛出,斬以徇,由是上下脅息。緘子子元為桂州司戶,因公事攜妻子來省,欲還而寇至。緘念人不可戶曉,必以郡守家出城,乃獨遣子元,留其妻子。選勇士拿舟逆戰,斬蠻酋二。



 邕既受圍,緘晝夜行勞士卒,發神臂弓射賊,所殪甚眾。緘初求救於劉彞,彞遣將張守節救之,逗遛不進。緘又以蠟書告急於提點刑獄宋球,球得書驚泣,督守節。守節皇恐,遽移屯大夾嶺,回保昆侖關,猝遇賊,不及陣,舉軍皆覆。蠻獲北軍,知
 其善攻城,啖以利,使為雲梯,又為攻濠洞子,蒙以華布,緘悉焚之。蠻計已窮,將引去,而知外援不至,或教賊囊土傅城者,頃刻高數丈,蟻附而登,城遂陷。緘猶領傷卒馳騎戰愈厲,而力不敵,乃曰:「吾義不死賊手。」亟還州治,殺其家三十六人,藏於坎,縱火自焚。蠻至,求尸皆不得,屠郡民五萬餘人,率百人為一積,凡五百八十餘積,隤三州城以填江。邕被圍四十二日,糧盡泉涸,人吸漚麻水以濟渴,多病下痢,相枕藉以死,然訖無一叛者。



 緘憤沈
 起、劉彞致寇,又不救患,欲上疏論之。屬道梗不通,乃榜其罪於市,冀朝廷得聞焉。神宗聞緘死,嗟悼,贈奉國軍節度使,謚曰忠勇,賜都城甲第五、鄉里上田十頃,聽其家自擇。以子元為西頭供奉官、閣門祗候,召對,謂曰:「邕管賴卿父守御,儻如欽、廉即破,則賊乘勝奔突,桂、象皆不得保矣。昔張巡、許遠以睢陽蔽遮江、淮,較之卿父,不能過也。」改授殿中丞,通判邕州。次子子明、子正,孫廣淵、直溫,與緘同死,皆褒贈焉。起與彞皆坐謫官。緘沒後,交
 人謀寇桂州,行數舍,其眾見大兵從北來,呼曰:「蘇皇城領兵來報怨。」懼而引歸。邕人為緘立祠,元祐中賜額懷忠。



 秦傳序,江寧人。淳化五年,充夔峽巡檢使。李順之亂,賊眾奄至,傅夔州城下,傳序督士卒晝夜拒戰,嬰城既久,危蹙日甚,長吏皆奔竄投賊。傳序謂士卒曰:「吾為監軍,盡死節以守城,吾之職也,安可茍免乎!」城中乏食,傳序出囊橐服玩,盡市酒肉以犒士卒,慰勉之,眾皆感泣力
 戰。傳序度力不能拒,乃為蠟書遣人間道上言:「臣盡死力,誓不降賊。」城壞,傳序赴火死。



 傳序家寄荊湖間,子奭溯峽求父尸,溺死。人以為父死於忠,子死於孝。奏至,太宗嗟惻久之,錄傳序次子煦為殿直,以錢十萬賜其家。煦卒,復以煦弟昉為三班奉職。



 詹良臣,字元公,睦州分水人。舉進士不第,以恩得官,調縉雲縣尉。方臘起,其黨洪再犯處州,守貳俱棄城遁。又有他盜霍成富者,用臘年號,剽掠縉雲。良臣曰:「捕盜,尉
 職也,縱不勝,敢愛死乎?」率弓兵數十人出御之,為所執。成富誘使降,良臣曰:「汝輩不知求生,顧欲降我邪!昔年李順反於蜀,王倫反於淮南,王則反於貝州,身首橫分,妻子與同惡,無少長皆誅死,旦暮官軍至,汝肉飼狗鼠矣。」賊怒,臠其肉,使自啖之。良臣吐且罵,至死不絕聲,見者掩面流涕,時年七十二。徽宗聞而傷之,贈通直郎,官其子孫二人。



 江仲明,臺州人。宣和寇亂,載老母逃山澗中,猝遇寇於
 東城之岡,逼使就降,仲明義不辱,奮起罵賊,卒死之,丞相呂賾浩誄以文。



 有蔣煜者,州之仙居人,有文學。寇欲妻以女,煜拒之,脅以拜,亦不從,寇曰:「吾戮汝矣!」煜伸頸就刃,詈聲不絕而死。



 李若水,字清卿,洺州曲周人,元名若冰。上舍登第,調元城尉、平陽府司錄。試學官第一,濟南教授,除太學博士。蔡京晚復相,子絳用事,李邦彥不平,欲謝病去。若水為言:「大臣以道事君,不可則止,胡不取決上前,使去就之
 義,暴於天下。顧可默默托疾而退,使天下有伴食之譏邪?」又言:「積蠹已久,致理惟難。建裁損而邦用未豐,省科徭而民力猶困,權貴抑而益橫,仕流濫而莫澄。正宜置驛求賢,解榻待士,採其寸長遠見,以興治功。」凡十數端,皆深中時病,邦彥不悅。



 靖康元年,為太學博士。開府儀同三司高俅死,故事,天子當掛服舉哀,若水言:「俅以幸臣躐躋顯位,敗壞軍政,金人長驅,其罪當與童貫等。得全首領以沒,尚當追削官秩,示與眾棄;而有司循常習
 故,欲加縟禮,非所以靖公議也。」章再上,乃止。



 欽宗將遣使至金國,議以賦入贖三鎮,詔舉可使者,若水在選中。召對,賜今名,遷著作佐郎。為使,見粘罕於雲中。才歸,兵已南下,復假徽猷閣學士,副馮澥以往。甫次中牟,守河兵相驚以金兵至,左右謀取間道去,澥問「何如」?若水曰:「戍卒畏敵而潰,奈何效之,今正有死耳。」令敢言退者斬,眾乃定。



 既行,疊具奏,言和議必不可諧,宜申飭守備。至懷州,遇館伴蕭慶,挾與俱還。及都門,拘之於沖虛觀,獨
 令慶、澥入。既所議多不從,粘罕急攻城,若水入見帝,道其語,帝命何𬃄行。桌還,言二人欲與上皇相見,帝曰:「朕當往。」明日幸金營,過信而歸。擢若水禮部尚書,固辭。帝曰:「學士與尚書同班,何必辭。」請不已,改吏部侍郎。



 二年,金人再邀帝出郊,帝殊有難色,若水以為無他慮,扈從以行。金人計中變,逼帝易服,若水抱持而哭,詆金人為狗輩。金人曳出,擊之敗面,氣結僕地,眾皆散,留鐵騎數十守視。粘罕令曰:「必使李侍郎無恙。」若水絕不食,或勉
 之曰:「事無可為者,公昨雖言,國相無怒心,今日順從,明日富貴矣。」若水嘆曰:「天無二日,若水寧有二主哉!」其僕亦來慰解曰:「公父母春秋高,若少屈,冀得一歸覲。」若水叱之曰:「吾不復顧家矣!忠臣事君,有死無二。然吾親老,汝歸勿遽言,令兄弟徐言之可也。」



 後旬日,粘罕召計事,且問不肯立異姓狀。若水曰:「上皇為生靈計,罪己內禪,主上仁孝慈儉,未有過行,豈宜輕議廢立?」粘罕指宋朝失信,若水曰:「若以失信為過,公其尤也。」歷數其五事曰:「
 汝為封豕長蛇,真一劇賊,滅亡無日矣。」粘罕令擁之去,反顧罵益甚。至郊壇下,謂其僕謝寧曰:「我為國死,職耳,奈並累若屬何!」又罵不絕口,監軍者撾破其唇,噀血罵愈切,至以刃裂頸斷舌而死,年三十五。



 寧得歸,具言其狀。高宗即位,下詔曰:「若水忠義之節,無與比倫,達於朕聞,為之涕泣。」特贈觀文殿學士,謚曰忠愍。死後有自北方逃歸者云:「金人相與言,『遼國之亡,死義者十數,南朝惟李侍郎一人』。臨死無怖色,為歌詩卒,曰:『矯首問天兮,
 天卒無言,忠臣效死兮,死亦何愆?』聞者悲之。」



 劉韐,字仲偃,建州崇安人。第進士,調豐城尉、隴城令。王厚鎮熙州,闢狄道令,提舉陜西平貨司。河、湟兵屯多,食不繼,韐延致酋長,出金帛從易粟,就以餉軍,公私便之。遂為轉運使,擢中大夫、集英殿修撰。



 劉法死,夏人攻震武。韐攝帥鄜延,出奇兵搗之,解其圍。夏人來言,願納款謝罪,皆以為詐。韐曰:「兵興累年,中國尚不支,況小邦乎?彼雖新勝,其眾亦疲,懼吾再舉,故款附以圖自安,此情
 實也。」密疏以聞,詔許之。夏使愆期不至,諸將言夏果詐,請會兵乘之。韐曰:「越境約會,容有他故。」會再請者至,韐戒曰:「朝廷方事討伐,吾為汝請,毋若異時邀歲幣,軼疆場,以取威怒。」夏人聽命,西邊自是遂安。



 韐求東歸,拜徽猷閣待制,提舉崇福宮。起知越州,鑒湖為民侵耕,官因收其租,歲二萬斛。政和間,涸以為田,衍至六倍,隸中宮應奉,租太重而督索嚴,多逃去。前勒鄰伍取償,民告病,韐請而蠲之。方臘陷衢、婺,越大震,官吏悉遁,或具舟請
 行。韐曰:「吾為郡守,當與城存亡。」不為動,益厲戰守備。寇至城下,擊敗之,拜述古殿直學士,召為河北、河東宣撫參謀官。



 時邊臣言,燕民思內附,童貫、蔡攸方出師,而種師道之軍潰。韐意警報不實,見師道計事。師道曰:「契丹兵勢尚盛,而燕人未有應者,恐邊臣誕謾誤國事。」韐即馳白貫、攸,請班師。又論燕薊不可得,正使得之,屯兵遣餉,經費無藝,必重困中國。還次莫州,會郭藥師以涿州降,戎車再駕,以韐議異,徙知真定府。藥師入朝,韐密奏
 乞留之,不報。徙知建州,改福州,加延康殿學士。或言其過闕時,見御史中丞有所請,遂罷。起知荊南,河北盜起,復以守真定。首賊柴宏本富室,不堪征斂,聚眾剽奪,殺巡尉,統制官亦戰死。韐單騎赴鎮,遣招之,宏至服罪。韐飲之酒,請以官,縱其黨還田里,一路遂平。藥師請馬,詔盡以河北戰馬與之,不足,又賦諸民。韐曰:「空內郡駔駿,付一降將,非計也。」奏止之。金人已謀南牧,朝廷方從之求雲中地。韐諜得實,急以聞,且陰治城守以待變。是
 冬,金兵抵城下,知有備,留兵其旁,長驅內向。及還,治梯沖設圍,示欲攻擊,韐發強駑射之,金人知不可脅,乃退。自金兵之來,諸郡皆塞門,民坐困,韐獨縱樵牧如平日,以時啟閉。欽宗善之,拜資政殿學士。



 時已許割地賂金人,而議者乘士民之憤,復議追躡,韐以亟戰為非。是時,諸將救太原,種師中、姚古敗。以韐為宣撫副使,至遼州,招集糾募,得兵四萬人,與解潛、折可求約期俱進,兩人又繼敗。初,韐遣別將賈瓊自代州出敵背,且許義軍以爵
 祿,得首領數十。既復五臺,而潛、可求敗聞,遂不果進。太原陷,召入覲,為京城四壁守禦使,宰相沮罷之。



 京城不守,始遣使金營,金人命僕射韓正館之僧舍。正曰:「國相知君,今用君矣。」韐曰:「偷生以事二姓,有死,不為也。」正曰:「軍中議立異姓,欲以君為正代,得以家屬行,與其徒死,不若北去取富貴。」韐仰天大呼曰:「有是乎!」歸書片紙曰:「金人不以予為有罪,而以予為可用。夫貞女不事二夫,忠臣不事兩君;況主憂臣辱,主辱臣死,以順為正者,妾
 婦之道,此予所以必死也。」使親信持歸報諸子。即沐浴更衣,酌卮酒而縊。燕人嘆其忠,瘞之寺西岡上,遍題窗壁,識其處。凡八十日乃就殮,顏色如生。建炎元年,贈資政殿大學士,後謚曰忠顯。



 韐莊重寬厚,與人交,若有畏者;至臨大事則毅然不可回奪。初在西州為童貫所知,故首尾預其軍事,及以忠死,論者不復短其前失雲。子子羽、孫珙,自有傳。



 傅察,字公晦,孟州濟源人,中書侍郎堯俞從孫也。年十
 八,登進士第。蔡京在相位,聞其名,遣子鯈往見,將妻以女,拒弗答。調青州司法參軍,歷永平、淄川丞,入為太常博士,遷兵部、吏部員外郎。



 宣和七年十月,接伴金國賀正旦使。是時,金將渝盟,而朝廷未之知也。察至燕,聞金人入寇,或勸毋遽行。察曰:「受使以出,聞難而止,若君命何。」遂至韓城鎮。使人不來,居數日,金數十騎馳入館,強之上馬,行次境上,察覺有變,不肯進,曰:「迓使人,故例止此。」金人輒易其馭者,擁之東北去,行百里許,遇所謂二
 太子斡離不者領兵至驛道,使拜。察曰:「吾若奉使大國,見國主當致敬,今來迎客而脅我至此!又止令見太子,太子雖貴人,臣也,當以賓禮見,何拜為?」斡離不怒曰:「吾興師南向,何使之稱?凡汝國得失,為我道之,否則死。」察曰:「主上仁聖,與大國講好,信使往來,項背相望,未有失德。太子干盟而動,意欲何為?還朝當具奏。」斡離不曰:「爾尚欲還朝邪!」左右促使拜,白刃如林,或捽之伏地,衣袂顛倒,愈植立不顧,反覆論辨。斡離不曰:「爾今不拜,後日
 雖欲拜,可得邪!」麾令去。



 察知不免,謂官屬侯彥等曰:「我死必矣,我父母素愛我,聞之必大戚。若萬一脫,幸記吾言,告吾親,使知我死國,少紓其亡窮之悲也。」眾皆泣。是夕隔絕,不復見。金兵至燕,彥等密訪存亡,曰:「使臣不拜太子,昨郭藥師戰勝有喜色,太子慮其劫取,且銜往忿,殺之矣。」將官武漢英識其尸,焚之,裹其骨,命虎翼卒沙立負以歸。立至涿州,金人得而系諸土室,凡兩月。伺守者怠,毀垣出,歸以骨付其家。副使蔣噩及彥輩歸,皆能
 道察不屈狀,贈徽猷閣待制。



 察自幼嗜學,同輩或邀與娛嬉,不肯就。為文溫麗有典裁。平居恂恂然,無喜慍色,遇事若無所可否,非其意,崒然不可犯。恬於勢利,在京師,故人鼎貴,罕至其門,間一見,寒溫談笑而已。及倉卒徇義,犖犖如此,聞者哀而壯之,時年三十七。乾道中,賜謚曰忠肅。



 楊震,字子發,代州崞人。以弓馬絕倫為安邊巡檢。河東軍征臧底河,敵據山為城,下瞰官軍,諸將合兵城下,震
 率壯士拔劍先登,斬數百級,眾乘勝平之,上功第一。



 從折可存討方臘,自浙東轉擊至三界鎮,斬首八千級。追襲至黃巖,賊帥呂師囊扼斷頭之險拒守,下石肆擊,累日不得進。可存問計,震請以輕兵緣山背上,憑高鼓噪發矢石,賊驚走,已復縱火自衛。震身被重鎧,與麾下履火突入,生得師囊,及殺首領三十人,進秩五等。還知麟州建寧砦。



 初,契丹之亡,其將小鞠䩮西奔,招合雜羌十餘萬,破豐州,攻麟府諸城郭。震父宗閔領本道兵馬屢
 摧敗之,俘其父母妻子。靖康元年十月,太原陷,鞠䩮驅幽薊叛卒與夏人奚人圍建寧,扣壁語震曰:「汝父奪我居,破我兵,掩我骨肉,我忍死到今,急舉城降,當全汝軀命。」時城中守兵不滿百,震與戰士約,斬一級賞若干,官帑竭,繼以家人服珥,吏士感激自奮。越旬,矢盡力乏,城不守,與子居中、執中力戰沒,閤門俱喪,唯長子存中從征河北獨免。明年,宗閔亦死事於長安。



 震時年四十四。建炎二年,詔贈武經郎。存中貴,請於朝,謚曰恭毅。



 張克戩,字德祥,侍中耆曾孫也。第進士,歷河間令,知吳縣。吳為浙劇邑,民喜爭,大姓怙勢持官府。為令者踵故抑首,務為不生事、幸得去而已。克戩一裁以法,奸猾屏氣,使者以狀聞,召拜衛尉丞。初,克戩從弟克公為御史,劾蔡京。京再輔政,修怨於張氏,以微事黜克戩。逾年,起知祥符縣,司開封戶曹,提舉京東常平,入辭,留為庫部員外郎。



 宣和七年八月,知汾州。十二月,金兵犯河東,圍太原。太原距汾二百里,遣將銀朱孛堇來攻,縱兵四掠,
 克戩畢力捍禦。燕人先內附在城下者數十,陰結黨欲為內應,悉收斬之。數選勁卒撓敵營,出不意焚其柵,敵懼引去,論功加直秘閣。



 靖康元年六月,金兵復逼城。朝廷命經略使張孝純之子灝、都統制張思正、轉運使李宗來援,思正誅求無藝,民不堪命。克戩引誼開曉,皆願自奮。宣撫使李綱表其守城之勞,連進直龍圖閣、右文殿修撰。太原不守,思正紿云出戰,遂率灝、宗奔慈、隰,於是人無固志。戍將麻世堅中夜斬關出,通判韓琥相繼
 亡,克戩召令兵民曰:「太原既陷,吾固知亡矣。然義不忍負國家、辱父祖,願與此城終始以明吾節,諸君其自為謀。」皆泣不能仰視,同辭而對曰:「公父母也,願盡死聽命。」乃益厲兵儆守。賊至,身帥將士擐甲登陴,雖屢卻敵而援師訖不至。



 金兵破平遙,平遙為汾大邑,久與賊抗,既先陷,又脅降介休、孝義諸縣,據州南二十村,作攻城器具,兩遣使持書諭克戩,焚不啟。具述危苦之狀,募士間道言之朝,不報。十月朔,金益萬騎來攻愈急,有十人唱
 為降語,斬以徇。諸酋列城下,克戩臨罵極口,炮中一酋,立斃。度不得免,手草遺表及與妻子遺書,縋州兵持抵京師。明日,金兵從西北隅入,殺都監賈亶,克戩猶帥眾巷戰,金人募生致之。克戩歸索朝服,焚香南向拜舞,自引決,一家死者八人。金將奉其尸禮葬於後園,羅拜設祭,為立廟。事聞,詔贈延康殿學士,贈銀三百兩、絹五百匹,表揭門閭。紹興中,謚忠確。



 張確,字子固,邠州宜祿人。元祐中,擢進士第。徽宗即位,
 應詔上書言十事,乞誅大奸,退小人,進賢能,開禁錮,起老成,擢忠鯁,息邊事,修文德,廣言路,容直諫,遂列於上籍。



 宣和二年,召至京師。青溪盜起,確言:「此皆王民,但庸人擾之耳。願下哀痛之詔,省不急之務,租賦之外,一切寢罷,敢以花石淫巧供上者死。撫綏脅附,毋以多殺為功,旬浹之間,可以殄滅。」忤王黼意,通判杭州,攝睦州事。有自賊中逃歸者,悉宥之,訪得虛實以告,諸將用其言。盜平,知坊、汾二州。



 宣和七年,徙解州,又徙隆德府。金兵
 圍太原,忻、代降,平陽兵叛。確表言:「河東天下根本,安危所系,無河東,豈特秦不可守,汴亦不可都矣。敵既得叛卒,勢必南下,潞城百年不修築,將兵又皆戍邊。臣生長西州,頗諳武事,若得秦兵十萬人,猶足以抗敵,不然,唯有一死報陛下耳。」書累上不報。明年二月,金兵至,知城中無備,諭使降。確乘城拒守,或獻謀欲自東城潰圍出,且探確意。確怒叱曰:「確守土臣,當以死報國,頭可斷,腰不可屈。」乃戰而死。



 欽宗聞之悲悼,優贈述古殿直學士,
 召見其子乂,慰撫之曰:「卿父今之巡、遠也,得其死所矣,復何恨。使為將為守者皆如卿父,朕顧有今日邪!」斂容嘆息者久之。



 朱昭,字彥明,府穀人。以效用進,累官秉義郎,浮湛班行,不自表異。宣和末,為震威城兵馬監押,攝知城事。金兵內侵,夏人乘虛盡取河外諸城鎮。震威距府州三百里,最為孤絕。昭率老幼嬰城,敵攻之力,昭募驍銳兵卒千餘人,與約曰:「賊知城中虛實,有輕我心,若出不意攻之,
 可一鼓而潰。」於是夜縋兵出,薄其營,果驚亂,城上鼓噪乘之,殺獲甚眾。



 夏人設木鵝梯沖以臨城,飛矢雨激,卒不能施,然晝夜進攻不止。其酋悟兒思齊介胄來,以氈盾自蔽,邀昭計事。昭常服登陴,披襟問曰:「彼何人,乃爾不武!欲見我,我在此,將有何事?」思齊卻盾而前,數宋朝失信,曰:「大金約我夾攻京師,為城下之盟,畫河為界;太原旦暮且下,麟府諸壘悉已歸我,公何恃而不降?」昭曰:「上皇知奸邪誤國,改過不吝,已行內禪,今天子聖政一
 新矣,汝獨未知邪?」乃取傳禪詔赦宣讀之,眾愕眙,服其勇辯。是時,諸城降者多,昭故人從旁語曰:「天下事已矣,忠安所施?」昭叱曰:「汝輩背義偷生,不異犬彘,尚敢以言誘我乎?我唯有死耳!」因大罵引弓射之,眾走。凡被圍四日,城多圮壞,昭以智補御,皆合法,然不可復支。昭退坐廳事,召諸校謂曰:「城且破,妻子不可為賊污,幸先戕我家而背城死戰,勝則東向圖大功,不勝則暴骨境內,大丈夫一生之事畢矣。」眾未應。昭幼子戲階下,遽起手刃
 之,長子驚視,又殺之,徑領數卒屠其家人,舁尸納井中。部將賈宗望母適過前,昭起呼曰:「媼,鄉人也,吾不欲刃,請自入井。」媼從之,遂並覆以土。將士將妻孥者,又皆盡殺之。昭謂眾曰:「我與汝曹俱無累矣!」



 部落子有陰與賊通者,告之曰:「朱昭與其徒各殺其家人,將出戰,人雖少,皆死士也。」賊大懼,以利啖守兵,得登城。昭勒眾於通衢接戰,自暮達旦,尸填街不可行。昭躍馬從缺城出,馬蹶墜塹,賊歡曰:「得朱將軍矣!」欲生致之。昭瞋目仗劍,無一
 敢前,旋中矢而死,年四十六。



 史抗,濟源人。宣和末,為代州沿邊安撫副使。金人圍代急,抗夜呼其二子稽古、稽哲謂曰:「吾昔語用事者,『雁門控制一道,宜擇帥增戍以謀未形之患,若使橫流,則無所措矣』。言雖切,皆不吾省。今重圍既固,外援不至,吾用六壬術占之,明日城必陷,吾將死事,汝輩亦勿以妻子為念而負國也。能聽吾言,當令家屬自裁,然後同赴義。」二子泣曰:「唯吾父命。」明日,城果破,父子三人突圍力戰,
 死於城隅。



 孫益,不知其所以進。宣和末,以福州觀察使知朔寧府,被命救太原。時敵勢張甚,或言不若引兵北搗雲中,彼之將士室家在焉,所謂攻其所必救也。益曰:「此策固善,奈違君命。」因躍馬冒圍至城下,張孝純不肯啟門,遂死之。



 益天資忠勇,每傾貲以賞戰士,能得人死力。小鞠䩮為邊患,遣將致討,益子在行間,師無功,益謂子必死。朝廷聞之,恤錄其孤甚厚。其子遣信至益所報平安,益怒
 其子不能死,以狀自列,盡上還官所賜,而斬其持書來者。



 初,益在朔寧,察郡人孫穀可用,奏為掾屬,待之異於常僚。益出師,屬以後事。益死,敵騎來攻,且別命郡守。眾議欲開關迎之,谷爭弗得,嘆曰:「吾身已許國,又不忍負孫公之托,諸人不見容,是吾死所也。」或舉刃脅之,無懾容,遂見殺。



\end{pinyinscope}