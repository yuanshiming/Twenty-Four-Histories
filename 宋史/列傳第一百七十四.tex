\article{列傳第一百七十四}

\begin{pinyinscope}

 ○傅
 伯成葛洪曾三復黃疇若袁韶危稹程公許羅必元王遂



 傅伯成,字景初,吏部員外郎察之孫。少從朱熹學。登隆
 興元年進士第,調連江尉。試中教官科,授明州教授。以年少,嫌以師自居,日與諸生論質往復,後多成才。改知閩清縣。丁父艱,服除,知連江縣。東湖溉田餘二千頃,堤壞。即下流南港為石堤三百尺,民蒙其利。



 慶元初,召為將作監,進太府寺丞。言呂祖儉不當以上書貶。又言於御史,朱熹大儒,不可以偽學目之。又言朋黨之敝,起於人主好惡之偏。坐是不合,出知漳州,以律己愛民為本。推熹遺意而遵行之,創惠民局,濟民病,以革禨鬼之俗。
 由郡南門至漳浦,為橋三十五,治道千二百丈。



 兩為部使者,遷工部侍郎。時權臣方開邊,語尚秘。伯成言:「天下之勢,譬如乘舟,中興且八十年矣,外而望之,舟若堅緻,歲月既久,罅漏浸多,茍安旦夕,猶懼覆敗,乃欲徼幸圖古人之所難,臣則未之知也。」相府災,同列相率唁丞相,或以為偶然者,伯成正色謂:「天意如此,官師相規時也,以為偶然乎?」丞相色動。遂陳三事:一曰失民心,二曰隳軍政,三曰啟邊釁。進右司郎官,權幸有私謁者,皆峻拒
 之。出為湖、廣總領。朝議欲納金人之叛降者,伯成言不宜輕棄信誓,乞戒將帥毋生事。御史中丞鄧友龍遂劾伯成,罷之。



 嘉定元年,召對,面諭:「前日失於戰,今日失之和。小使雖返,要求尚多。陛下不獲己,悉從之。使和議成,猶可以紓一時之急;否則虛帑藏以資敵人,驅降附以絕來者,非計也。今之策雖以和為主,宜惜日為戰守之備。」權戶部侍郎史彌遠初拜相,麻詞有「昆命元龜」之語,閩帥倪思以為不當用,御史劾罷思。伯成因對及其事,
 帝曰「過當」者再。對曰:「思固過當,但恐摧抑太過,遂塞言路,乞明詔臺諫侍從,竭盡底蘊,無以思為戒。」李壁謫居撫州,伯成言:「侂胄之誅,壁與有功,不酬近功,乃追前罪,他日負罪之臣,不容以功贖過矣。」



 伯成未為諫官也,嘗言:「彌遠謀誅侂胄,事不遂則其家先破,侂胄誅而史代之,勢也。諸公要相協和,共議國事;若立黨相擠,必有勝負,非國之福。」又勸丞相錢象祖:「安危大事,以死爭之;差除小者,何必乖異?」拜左諫議大夫,抗疏十有三,皆軍國
 大義。或致彌遠意,欲使有所彈劾,謂將引以共政。謝之曰:「吾豈傾人以為利哉。」疏乞詔大臣以公滅私。



 左遷權吏部侍郎。以集英殿修撰知建寧府。蔡元定謫死道州,歸葬建陽,乃雪其冤於朝。進寶謨閣待制、知鎮江府。全活饑民,瘞藏野殍,不可勝數。制置司欲移焦山防江軍於圌山石牌,伯成謂:「虛此實彼,利害等耳。包港在焦、圌之中,不若兩砦之兵迭戍焉。」圌山砦兵,素與海盜為地,伯成廉知姓名,會郡都試捕而鞫之,無一逸去。獄具,請
 貸其死,黥隸諸軍。



 嘉定八年,召赴闕,辭不獲,行至莆,拜疏曰:「臣病不能進矣。」除寶謨閣直學士、通奉大夫,致仕。理宗即位,升直學士,落致仕,予祠,錫金帶。伯成辭免,乃進「昭明天常、扶持人極」之說,詔進一官。



 寶慶元年,與楊簡同召,尋加寶文閣學士,提舉祐神觀,奉朝請。雖力以老病辭,而愛君憂國之念不少衰。聞大理評事胡夢昱坐論事貶,蹙然語所親曰:「向呂祖儉之謫,吾為小臣,猶嘗抗論。今蒙國恩,叨竊至此而不言,誰當言者。」遂抗疏
 曰:「臣恐陛下不復聞天下事矣。方今內無良吏,田里怨咨,外無名將,邊陲危急,而廉恥道喪,風俗益偷,賄賂流行,公私俱困。謂宜君臣上下,憂邊恤民,以弭禍亂。奈何今日某人言某事,未幾而斥,明日某人言某事,未幾而斥,則是上疏者以共工、驩兜之刑加之矣。昔韓愈論後世人主奉佛,運祚短促,唐憲宗大怒,將抵以死,自崔群、裴度戚里諸賢皆為愈言,止貶潮州,尋復內徙。今上疏者非可愈比,然在列之臣,無一為言者,萬一死於瘴癘,
 陛下與大臣有殺諫者之謗,史冊書之,有累聖治。臣垂盡之年,與斯人相去,風馬牛之不相及,獨以受恩優異,效其瞽言。」不報。明年,加龍圖閣學士,轉一官,提舉鴻慶宮,復辭。



 伯成純實無妄,表裏洞達,每稱人善,不啻如己出,語及奸人誤國,邪人害正,詞色俱厲,不少假借,常慕尸諫,疏草畢,亟命繕寫,朝服而逝,年八十有四。贈開府儀同三司。端平三年,賜謚忠簡。



 葛洪,字容父,婺州東陽人。從呂祖謙學,登淳熙十一年
 進士第。嘉定間,為樞密院編修官兼國史院編修官、實錄院檢討官。遷守尚書工部員外郎兼權樞密院檢詳諸房文字。上疏言:



 今之將帥,其才與否,臣不得而盡知。惟忠誠所在,凡為人臣者斯須所不可離,則不可不以是責之耳。今安居無事,非必奮不顧死,冒水火,蹈白刃,而後謂之忠也。第職思其憂謂之忠,公爾忘私謂之忠,純實不欺謂之忠。



 且拊循士卒,帥之職也,朝廷每嚴掊克之禁,蠲營運之逋,其儆之者至矣。今乃有別為名色,
 益肆貪黷,視生理之稍豐者而誣以非辜,動輒估籍,擇廩給之稍優者而強以庫務,取辦芻粟,抑配軍需,於拊循何有哉!訓齊戎旅,亦帥之職也,朝廷每嚴點試之法,申階級之令,其儆之亦切矣。今顧有教閱視為具文,坐作僅同兒戲,技勇者不與旌賞,拙懦者未嘗勸懲,士日橫驕,類難役使,於訓齊何有哉



 況乃有沉酣聲色之奉,溺意田宅之圖,而不恤國事者矣。又有營營終日,專務納交,書幣往來,道路旁午,而妄希升進者矣。自謂繕治
 器甲,修造戰艦,究其實,則飾舊為新而已爾。自謂撙節財用,聲稱羨餘,原其自,則剝下罔上而已爾。乞嚴飭將帥,上下振厲,申致軍實,常若有寇至之憂。磨礪振刷,以求更新,亦庶乎其有用矣。



 帝嘉納之。



 進直煥章閣,為國子祭酒,仍兼國史編修、實錄檢討。遷工部侍郎,仍兼祭酒兼同修國史實錄院同修撰,拜工部尚書,亦兼祭酒兼侍讀。進端明殿學士、同簽書樞密院事,拜參知政事,封東陽郡公。贊討平李全,援王素諫仁宗卻王德用進
 女事,以止備嬪御,世多稱之。以資政殿學士、提舉洞霄宮,進大學士。召赴行在,仍舊職充萬壽觀使兼侍讀,尋提舉萬壽觀兼侍讀,守本官致仕,卒。帝輟視朝一日,謚端獻。杜範稱其侃侃守正,有大臣風。有奏議、雜著文二十四卷。



 曾三復字無玷,臨江人。乾道六年進士。淳熙末,為主管官告院,遷太府寺簿,歷將作、太府丞。登朝數年,安於平進,搢紳稱之。紹熙初,出知池州,改常州。召為御史檢法,
 拜監察御史,轉太常少卿,進起居舍人,遷起居郎兼權刑部侍郎,以疾告老。詔守本官職致仕。三復性耿介,恥奔競,故位不速進。在臺餘兩年,持論正平,不隨不激。其沒也,士論惜之。



 黃疇若,字伯庸,隆興豐城人。一歲而孤,外大母杜教之。淳熙五年舉進士,授祁陽縣主簿。邑民有訴僧為盜且殺人,移鞫治,疇若疑其無證,以白提點刑獄馬大同,且爭之甚力,已而得真盜,大同薦之,調柳州教授,又調靈
 川令。會萬安軍黎蠻竊發,經略司選疇若條畫招捕事宜。疇若謂須稽原始亂,為區處之方。再任嶺外,用舉考改知廬陵縣。州常以六月督畸零稅,疇若念民方艱食,取任內縣用錢三十餘緡為民代輸兩年。諸司舉為邑最官,召赴都堂審察,差監行在都進奏院。



 開禧元年,都城火。疇若應詔上言曰:「當今之急務有三:一曰賦斂徵求之無藝,二曰都鄙軍民之無法,三曰守令牧養之無狀。」遷太府寺主簿,又遷將作監丞兼皇弟吳興郡王府
 教授。遷太府寺丞,又遷秘書丞兼權禮部郎官,兼資善堂說書。遷著作郎,拜監察御史。首章乞天子擇宰相,宰相擇監司。又言:「善為國者必以恐懼修省之訓陳於前,善為相者必以危亡災異之事告於上。」



 韓侂胄敗,畤若上章丐去,帝批其奏曰:「卿懷忠藎,朕固知之。」疇若遂疏鄧友龍、陳景俊之惡。先是,江、淮督府既罔功,罷不更置。疇若奏,以為和戰未決,不遣近臣置幕府,無以統諸將。乞檢會前奏,亟詔大臣科條人才為宣撫使。帝即日以
 丘崈為江、淮制置使。尋遷疇若殿中侍御史兼侍講。朝廷與金人約和,金人約函致侂胄首。詔令臺諫、侍從、兩省雜議。疇若與章燮等奏:「乞梟首,然後函送敵國。」人譏其有失國體。



 疇若奏:「今帑藏無餘,歲幣若必睥睨於百姓,願自宮禁以及宰執百官共為撙節,逐年樁積。」遂置安邊所。戶部侍郎沈詵條具合節省拘催者,疇若復乞:「依仁宗、孝宗兩朝成訓,凡節省事:在內諸司選內侍長一員,令自行搜訪,條具來上;在外廷三省則委宰掾、樞
 屬,六曹則委長貳,事干浮費者聞奏。」又乞:「以官司房廊及激賞庫四季所獻並侂胄萬畝莊等,一並拘樁。」既而內廷及酒所減省,議多格,獨得估籍奸賊及房廊非泛供須五項,總緡錢九百一十三萬有奇,外樁留產業,每歲又可得七十一萬五千三百餘緡,疇若乞:「令後省類聚更化以來臣下章奏,察其可行者以聞,付之中書。」



 都城穀踴貴,詔減價糶樁管米十萬石,於是淮、浙流民交集。臨安府按籍振濟,僅不滿五千人,以三月後麥熟罷
 振濟,各給糧遣歸。疇若謂:「此實驅之使去耳。」遂奏:「乞令核實,近甸之人,願歸就田者勿問,其有未能歸者,更振濟兩月;淮民見在都城者,其家既破,又無贏貲,必難遽去,仍與振恤,俟早熟乃罷。」於是詔振濟至六月乃止。



 帝以蝗災,令刺舉監司不才者,疇若同臺監考察上之。又言:「湖、廣盜賊,固迫於饑寒,然亦有激而成之者。黑風峒寇,實由官不為決訟所致。宜戒湖、廣諸司,申明法禁為賊,關防以時,平心決訟,勿令砦官巡尉侵漁。」權戶部侍
 郎,金使告主亡,差充館伴。



 自軍興費廣,朝廷給會子數多,至是折閱日甚。朝論頗嚴稱提,民愈不售,郡縣科配,民皆閉門牢避。行旅持券,終日有不獲一錢一物者。詔令侍從、臺省,條上所見。疇若奏曰:「物少則貴,多則賤,理之常也。曷若令郡縣姑以漸稱提,先收十一界者消毀,勿復支出。上下流通,則不待稱提矣。」由是峻急之令少寬。又疏奏:「乞崇忠厚,延質樸,屏絕浮薄之論。乞撥買官田充糴本,以廣常平之儲。乞令戶察一員,專監安邊所。」
 帝皆是之。



 因面求補外,退上章,降詔不允,又連疏丐去。會旱蝗復熾,御筆令在朝百執事條上封事,疇若奏「官吏苛刻、科役頻並、賦斂繁重、刑法淹延」四事。冊皇太子,差充引見禮儀使。進華文閣待制、知成都府。蜀自吳曦畔後,制置使移司興元,朝論有偏重之嫌。朝廷擇人,故輟疇若以往,三辭不允。避諱,改寶謨閣待制。詔:「凡屬軍民利病,吏治藏否,並許諮訪以聞。」當征積欠十餘萬,疇若亟命榜九邑盡蠲之。考官吏冗員,非敕命差注者悉
 罷之。為民代輸六年布估錢,計二十萬二千四百緡;又別立庫儲二十五萬三千緡,期於異日接續代輸;又糴米十五萬石有奇,足廣惠倉之儲;又減他賦之重者,民力遂寬。



 初,沈黎蠻屢犯邊,疇若至,則鏤榜曉以禍福,青、彌兩羌遂乞降。四年,董蠻合其部族入寇犍為利店。疇若亟調兵,且設方略捕之,皆遁去。先是,疇若廉知嘉定邊備廢弛,而平戎莊子弟可用,遂檄嘉定府權免平戎莊是年炭估、麻租,令莊子弟即日上邊為守備。會嘉定
 闕守,蠻窺利店無備,遂入寇。疇若復選西軍,欲且往防拓,牒轉運司折支,不報。蠻再犯龍鳩堡,轉運司始頗從所請。蠻復到龍門隘,知有備乃退。進龍圖閣待制,依舊知成都府。



 大使司之師出,東路提刑亦徵兵,三垂告警,敘南之報復急,兩路震動。疇若亟移書兩軍,俾速還師守險為後圖,西師遂退守沐川。既而疇若兼制敘州兵甲公事,既得專行,益嚴守備,蠻首昔丑竟降,朝廷賞平蠻功,進疇若一秩。



 疇若留蜀四年,弊根蠹穴,苗耨發櫛。
 如乞揀留移屯西兵義勇,以防竊發,以救偏重;更用東南賢士使蜀四路,而拔蜀守之有治功者為東南監司,庶杜州縣姻婭之私;輕取錢引貼期之費,以紓民力:皆抗疏請於朝,乞力行之。復念大玄城乃張儀所築,高駢所修,圮壞歲久,復修費重,乃以節縮餘錢四十萬貫為修城備。疇若以制置使留漢中,則護諸將為得宜。召赴行在,入對延和殿,遷權兵部尚書、太子右庶子。



 八年,四月不雨,詔求直言。疇若條具三事,首言:「比稱提楮幣,州
 縣奉行切迫,故因坐減陌被估籍者眾,乞與給還;乞蠲閣下戶畸零稅賦;乞振贍雄淮軍之乏。」尋皆行之。落權,升左庶子,仍兼修史,擢太子詹事。疇若引範鎮故事,乞歸田里。



 十年春,差知貢舉,試禮部尚書,以足疾乞歸。進煥章閣學士、知福州,力辭,乃改提舉鴻慶宮。關外軍潰,言者論及疇若,落職罷祠,後以煥章閣學士致仕。所著有《竹坡集》、奏議、講議、《經筵故事》。



 袁韶字彥淳,慶元府人。淳熙十四年進士。嘉泰中,為吳
 江丞。蘇師旦恃韓侂胄威福,撓役法,提舉常平黃榮檄韶核田以定役。師旦密諭意言:「吳江多姻黨,儻相容,當薦為京朝官。」韶不聽。是歲更定戶籍,承徭賦,皆師旦黨,師旦諷言者將論去。榮亟以是事白於朝,且薦之。未幾,師旦敗。改知桐廬縣。桐廬多宗室,持縣事無有善去者。韶始至,絕私謁,莫敢撓。錢塘岸歲為潮嚙,率取石桐廬,韶言:「廟子山有石,不必旁取鄰郡。」遂得求免。嘉定四年,召為太常寺主簿,父老旗鼓蔽江以餞,至於富陽,泣謝
 曰:「吾曹不復輸石矣。」



 後為右司郎官、接伴金使。使者索歲幣,語慢甚,韶曰:「昔兩國誓約,止令輸燕,不聞在汴。」使者語塞。十三年,為臨安府尹,幾十年,理訟精簡,道不拾遺,里巷爭呼為「佛子」,平反冤獄甚多。



 紹定元年,拜參知政事。胡夢昱論濟王事,當遠竄,韶獨以夢昱無罪,不肯署文書。李全叛,揚州告急,飛檄載道,都城爭有逃避者。乃拜韶浙西制置使,仍治臨安鎮遏之。丞相史彌遠懲韓侂胄用兵事,不欲聲討。韶與範楷言於彌遠曰:「揚失
 守則京口不可保,淮將如卞整、崔福皆可用。」適福至,韶夜與同見彌遠,言福實可用。彌遠從之,遂討全。韶卒以言罷。端平初,奉祠,卒年七十有七,贈少傅。後以郊恩,累贈太師、越國公。



 韶之父為郡小吏,給事通判廳,勤謹無失,歲滿當代,不聽去。後通判至,復留用之,因致豐饒。夫妻俱近五十,無子,其妻資遣之往臨安置妾。既得妾,察之有憂色,且以麻束發,外以彩飾之。問之,泣曰:「妾故趙知府女也,家四川,父歿家貧,故鬻妾以為歸葬計耳。」即
 送還之。其母泣曰:「計女聘財猶未足以給歸費,且用破矣,將何以酬汝?」徐曰:「賤吏不敢辱娘子,聘財盡以相奉。」且聞其家尚不給,盡以囊中貲與之,遂獨歸。妻迎問之曰:「妾安在?」告以其故,且曰:「吾思之,無子命也。我與汝周旋久,若有子,汝豈不育,必待他婦人乃育哉?」妻亦喜曰:「君設心如此,行當有子矣。」明年生韶。



 危稹,字逢吉,撫州臨川人。舊名科,淳熙十四年舉進士,孝宗更名稹。時洪邁得稹文,為之賞激。調南康軍教授。
 轉運使楊萬里按部,驟見嘆獎,偕游廬山,相與酬倡。調廣東帳司,未上,服父喪,免,調臨安府教授。倪思薦之,且語人曰:「吾得此一士,可以報國矣。」丁母憂,免,乾辦京西安撫司公事。入為武學諭,改太學錄。



 明年,遷武學博士,又遷諸王宮教授。稹謂以教名官,而實未嘗教,請改創宗子學,立課試法如兩學,從之。嘉定九年,新學成,改充博士,其教養之規,稹所論建。遷秘書郎、著作佐郎,兼吳益王府教授。升著作郎兼屯田郎官。



 稹始進對,請敘復
 軍功之賞以立大信,抆拭功臣之罪以厲忠節,置局以立武事,遣使以省邊防,厚賞以精間諜。次論和、戰、守利害,而請顓意於守。是歲春至夏不雨,稹應詔言:「安邊所徵斂之害,與無罪而籍沒之害;楮幣之改,以一奪二;鹽鈔之更,以新廢舊;至於沮格軍賞,放散死士,皆足以召怨而致旱。」



 明年又論:「謀國者欲以安靖為安靖,憂國者欲以振厲為安靖,自二議不合,是以國無成謀,人無定志。願詔大臣合二議共圖之,且欲下兩淮帥臣,講明守
 禦之備。」最後言:「事無成規者,皆不可為。意向不明,無以一眾聽;信誓不立,無以結人心;報應不亟,無以趨事機;賞罰不果,無以作士氣。」



 番易柴中行去國,稹賦詩送之,迕宰相,出知潮州。尋以通金華徐僑書論罷,提舉千秋鴻禧觀。久之,知漳州。漳俗視不葬親為常,往往棲寄僧剎,稹命營高燥地為義塚三,約期責之葬,其無主名、若有主名而力弗給者,官為葬之,凡二千三百有奇,刻石以識。郡有臨漳臺,據溪山最勝處,作龍江書院其上。既
 成,橫經自講,人用歆動。邑令有賄聞者,劾去之,籍其財以還民。郡有經、總制無名錢歲五千緡,厲民為甚,前守趙汝讜奏蠲五之二,稹疏於朝,悉罷之。會常平使有言,稹不欲辯,即自請以歸。久之,提舉崇禧觀,與鄉里耆艾七人為真率會。卒,年七十四。



 稹性至孝,父疾,願損己算益親年,疾尋愈。真德秀登從班,舉稹自代,沒,又為銘其墓。所著有《巽齋集》,諸經有講義、集解,諸魏、晉、唐詩文皆有編,輯先賢奏議曰《玉府》、曰《藥山》。



 弟和,字祥仲。開禧元
 年進士,為上元主簿,大闢祠宇祀程顥,真德秀為記之。知德興,振荒有惠政。有《蟾塘文集》。



 程公許,字季與,一字希穎,敘州宣化人。少知孝敬,大母侯疾,公許不交睫者數月,病革,嘗其痰沫,既卒,哀毀逾制。嘉定四年舉進士,調溫江尉,未上,丁母憂。服除,授華陽尉,再調綿州教授。制置使崔與之大加器賞,改秩知崇寧縣,蠲預借,免抑配,人甚德之。



 差通判簡州。改隆州,未上。會金人犯閬中,制置使桂如淵遁,三川震動,朝廷
 擢李𡌴代之,闢公許通判施州,行戶房公事。當兵將奔潰之後,公許盡力佐之,節浮費,疏利原,民不增賦而用自足。時諸將乘亂抄劫,事定自危,以重賂結幕府。大將和彥威懷金寶以獻,公許正色卻之,彥威慚而退。吳彥者,緘僧牒於書尾以進,公許卷還之而責其使,聞者畏服。有獻議招秦、鞏大姓於𡌴者,眾多從臾,獨公許謂山東覆轍未遠,反覆論難,𡌴從之。其後趙彥吶開閫,復行其策。未幾,金人搗成都,大姓者實導之,始服公許先見。



 端平初,授大理司直,遷太常博士。秋祀明堂,雷雨,應詔言事。嘉熙元年,御史杜範論執政李鳴復,不行,徙右史,竟拂衣東歸,鳴復坐政府自若。公許輪對,言:「志士仁人,嬰逆鱗,賈眾怒,不過為陛下通耳目,為朝廷立綱紀而已。今也假以職而棄其諫,幸其退而優其遷,則是自裂其綱紀,自蔽其耳目,遂使居是職者雖被親擢,言不得行,始焉固辭而弗從,終焉強留而飲愧。臣恐自此同類沮失,各起遐心,來者相戒,以為容默,陛下愈孤立無助
 矣。」



 夏,行都大火,殿中侍御史蔣峴逢君希寵,創為邪說,禁錮言者。公許應詔曰:「群臣忠告者眾,而聖意確不可回;聖意不可回,而言者不免於激。陛下宜以大舜無藏怒宿怨為心,而參酌於漢文帝之待淮南厲王、我太宗待秦邸之故事,以召和氣,弭眚災,特在一念轉移之頃耳。」遷秘書丞兼考功郎官,竟為峴劾去,差主管雲臺觀、和衢州,未上。改江東宣撫司參議官,不赴。



 李宗勉入相,以著作佐郎召,兼權尚左郎官兼直舍人院,遷著作郎。
 時諫官郭磊卿以論事不報出關,徐榮叟亦抗章引去,公許奏:「乞還言官,俾安厥位。」既而史嵩之自江上入相,臺諫謝方叔、王萬及磊卿相繼他徙,公許又奏:「外難憑陵,國勢岌若綴旒,朝廷上自為弗靖,陽為遷除,陰奪言職,此中外所以怏怏。」



 遷將作少監。大旱,應詔疏時事四條。又言:「儲極虛位,天下寒心。」時朝廷令侍從、臺諫條具易楮利害,尋降旨以新造十八界折五行使。公許繳申省,謂:「廟堂決意更革,本欲重十八界,亦當令十六界、十
 七界稍有分別,若一時皆以五折一,安保將來十七界與十八界並行而不折閱乎。曷若將十七界且以三兌一,使民間尚知寶此一界,不至一旦貿易不行,令三界各有等第,庶幾公私兩便。」嵩之格不行,徑揭黃榜。公許謂:「不經鳳閣鸞臺,不得為敕。朝廷出令而宰相擅行如此,則掖垣可廢。」累上奏牘,徑欲引去,宗勉及參知政事游似面奏留之,兼國史編修、實錄檢討。



 淳祐元年,遷秘書少監,輪對,言蜀事十條。兼直學士院,拜太常少卿,力
 請外,為右正言濮斗南之所論罷。尋以直寶謨閣知袁州,請蠲和糴之半。改命郡吏部總所綱運,而厚其貲,免募平民,民甚便之。新周敦頤祠,葺張栻書院,聘宿儒胡安之為諸生講說。杜範薦於上,召拜宗正少卿,再遷起居舍人。濮斗南繳還,疏有「臣等恥與為伍」之語,遂以舊職提舉玉局觀。範見疏曰:「程季與肯與汝為伍耶?」



 退處二年,召赴行在,屬嵩之以父憂去位,經營起復,益憚公許,密柬韓祥嗾殿中侍御史王贊奏寢召命。帝雖曲從
 而意不悅。及逐不才臺諫,擢公許起居郎兼直學士院。公許入奏不可不堅凝者七。帝語之曰:「卿一去三年,今用卿,出自朕意。」是日晚命下,嵩之罷起復,相範鐘及範,三制皆公許為之。兼權中書舍人。



 時二相尚遜,機務多壅。公許奏:「輔臣崇執謙遜,避遠形跡,相示以色而不明言,事幾無窮,日月易失。今最急莫若疆場之事,帥才不蓄,一旦欲議易置,茫然莫知所付。九江擇守,至以近所廢斥朋附為欺之臺察充其選。同時任言責者,雖心跡
 有顯晦,過惡有重輕,而獲罪於清議則同。一人抆拭之驟若是,三人者寧不引領以望玷缺之復。況近者言官方以劉晉之、鄭起潛、濮斗南三人乞明正其罪,以示警戒,而忽聞龔基先之用,議者咸謂改紀之初,所為錯繆,邪枉窺伺善類,何可高枕而臥。」帝見公許疏稱善,且言基先之用太早。



 右史徐元傑暴亡,司諫謝方叔、御史劉應起言,不報。公許亟奏曰:「正月,侍御史劉漢弼死。四月,右丞相杜範死。六月,右史徐元傑死。漢弼之死固可疑,
 範之死人言已籍籍,然漢弼類風淫末疾,範亦尪弱多病,諉曰天命,猶可也。元傑氣體魁碩,神採嚴毅,議論英發,甫聞謁告,奄至暴亡,口鼻四體變異之狀,使人為之雪涕不已。六館諸生叩閽告,陛下始命有司置獄鞫勘,謂當於朝紳中選公正明決無所顧忌者專蒞其事,盡情研究,務使得實。集議朝堂,分列首從,必誅無赦。」疏入,不報。物論沸騰,臨安尹趙與TP奏乞置獄天府,帝從之。公許繳奏:「與TP乃嵩之死黨,乞改送大理寺,命臺臣
 董之。」詔殿中侍御史鄭寀,寀回懦首鼠,事竟不白,然公論莫不偉公許。



 權禮部侍郎,差充執綏官。鄭起潛、劉晉之及陳一薦以臺臣論劾遷謫,公許疏其附下罔上之罪,乞下各州軍嚴行押發。鄭清之以少保奉祠,侍講幄中,批復其子士昌官職,與內祠,且許侍養行在所。蓋士昌嘗以詔獄追逮,或云詐以死聞,清之造闕,泣請於帝,故有是命。公許繳奏:「士昌罪重,京都浩穰,奸宄雜糅,恐其積習沉痼,重為清之累;莫若且與甄復,少慰清之,
 內祠侍養之命宜與收寢。」帝密遣中貴人以公許疏示清之。項容孫以罪遣還家,道死,時敘官復職,公許駁奏,命遂格。



 遷中書舍人,進禮部侍郎。嵩之免喪,以觀文殿大學士提舉洞霄宮,臺諫、給舍交章論奏,公許疏:「乞睿斷亟下明詔,正邦典。」殿中侍御史章琰、正言李昴英以論執政及府尹,帝怒,出二人,公許力爭之。公許自繳士昌之命,清之日夜於經筵短公許。周坦妻與清之妻善,因拜坦殿中侍御史。坦首疏劾公許,以寶章閣待制知
 建寧府;諫議大夫鄭寀又劾之,命遂寢。



 清之再相,公許屏居湖州者四年,再提舉玉隆觀、差知婺州,未上;帝欲召為文字官,清之奏已令守婺,帝曰:「朕欲其來。」乃授權刑部尚書,屢辭弗獲。入對,上疏貨財,興繕、逐諫臣、開邊釁時弊七事,薦知名士二十九人。



 時罷京學類申,散遣生徒,公許奏:「京學養士,其法本與三學不侔。往者立類申之法,重輕得宜,人情便安,近一旦忽以鄉庠教選而更張之,為士亦當自反,未可盡歸咎朝廷也。令行之始,
 臣方還朝,未敢強聒以撓既出之令。今士子擾擾道途,經營朝夕,今既未能盡復舊數,莫若權宜以五百為額,仍用類申之法,使遠方游學者,得以肄習其間。京邑四方之極,而庠序一空,弦誦寂寥,遂使逢掖皇皇,市廛昉怨而不敢議,非所以作成士氣、尊崇教化也。」清之益不樂。授稿殿中侍御史陳垓以劾公許,參知政事吳潛奏留之,帝夜半遣小黃門取垓疏入。後二日,二府奏公許不宜去,同知樞密院徐清叟上疏論垓。太學生劉黻等
 百餘人、布衣方和卿伏闕上書論垓。朝廷尋授寶章閣學士、知隆興府,而公許已死矣。遺表上,帝嗟悼,進龍圖閣學士致仕,贈宣奉大夫,官其後,賜賻如令式。



 公許沖澹寡欲,晚年惟一僮侍,食無重味,一裘至十數年不易。家無羨儲,敬愛親戚備至。蜀有兵難,族姻奔東南者多依公許以居。所著有《塵缶文集》、內外制、奏議、《奏常擬謚》、《掖垣繳奏》、《金革講義》、《進故事》行世。



 羅必元,字亨父,隆興進賢人。嘉定十年進士。調咸寧尉,
 撫州司法參軍,崇仁丞,復攝司法。郡士曾極題金陵行宮龍屏,迕丞相史彌遠,謫道州,解吏窘極甚。必元釋其縛,使之善達。真德秀入參大政,必元移書曰:「老醫嘗云,傷寒壞證,惟獨參湯可救之,然其活者十無二三。先生其今之獨參湯乎?」調福州觀察推官。有勢家李遇奪民荔支園,必元直之;遇為言官,以私憾罷之」知餘干縣。趙福王府驕橫,前後宰貳多為擠陷,至是以汝愚墓占四周民山,亦為直之,言於州曰:「區區小官,罷去何害?」人益
 壯其風力。



 淳祐中,通判贛州。賈似道總領京湖,克剝至甚。必元上疏,以為蠹國脈、傷民命,似道銜之。改知汀州,為御史丁大全按去,後起乾行在糧料院。錢塘有海鰍為患,漂民居,詔方士治之,都人鼓扇成風。必元上疏力止之。帝召見曰:「見卿《梅花詩》,足知卿志。」度宗即位,以直寶章閣兼宗學博士致仕。卒,年九十一。必元嘗從危稹、包遜學,最為有淵源,見理甚明,風節甚高,至今鄉人猶尊慕之云。



 王遂字去非,一字穎叔,樞密副使韶之玄孫,後為鎮江府金壇人。嘉泰二年進士,調富陽主簿,歷官差乾辦諸司審計司。紹定三年,福建寇擾甫定,朝廷選賢能吏,勞來安集,以遂知邵武軍兼福建招捕司參議官。遂過江山、浦城道中,遇邵武避地之人,即遺金為歸資,從者如市。至郡,撫摩創痍,翦平兇孽,民恃以安。未幾,言者以遂妄自標致,邀譽沽名,罷。



 改知安豐軍,遷國子監主簿,又遷太常寺主簿,拜監察御史。疏奏極論進君子,退小人。
 又言正風俗,息奔競。又言:「朝廷謂史嵩之小黠為大智,近功為遠略。忽臣之言,必欲僥幸嵩之於不敗,非為國至計也。欺君誤國,天下知之,而朝廷猶且惑焉,勢甚凜凜也。」入對,言帝知、仁、勇,學有未至。



 遷右正言,尋拜殿中侍御史。疏言:「三十年來兇德參會,未有如李知孝、梁成大、莫澤肆無忌憚者。三兇之罪,上通於天,乞重其刑。」又取劉光祖為殿中侍御史時奏格,擇其關於風化切於時宜者,請頒示中外。皆從之。又請於並淮置屯田,且條
 上邊事曰:「當今之急務:在朝廷者五,定規摹,明意向,一心力,謹事權,審號令;在邊閫者六,恤歸附,精間諜,節財用,練士兵,擇將才,計軍實。」又言:「君德必純乎剛。」帝皆善之。



 遷戶部侍郎兼同修國史實錄院同修撰,時暫兼權侍左侍郎。以寶章閣待制差知遂寧府。進煥章閣待制、四川安撫制置副使兼知成都府。差知平江府。進敷文閣待制、知慶元府,改知太平州,以論罷。進顯謨閣待制、知泉州。改溫州、寧國府。以寶章閣直學士知建寧府。以
 華文閣直學士差知隆興府兼江西轉運副使。改知太平州,復知隆興兼江西安撫使。召赴闕,授權工部尚書。



 遂與同里劉宰素同志,宰嘗稱遂為文雅健,無世俗浮靡之氣,足以名世。遂守平江,宰贈之言曰:「士友當親,而賢否不可不辨;財利當遠,而會計不可不明。折獄以情,毋為私意所牽;薦士以才,毋為權要所奪。當言則言,不視時而退縮;可去則去,不計利而遲回。庶幾名節之全,不愧簡冊所載。」蓋格言也。



 論曰:「傅伯成晚與楊簡為時蓍龜。葛洪守正不阿。曾三復澹然無躁競之心。黃疇若優於政治。袁韶力請討李全,蓋丞相史彌遠腹心也。危稹以通問徐僑獲罪,其人可知,矧治州之政,有循吏之風焉。羅必元受學於稹者也。程公許、王遂讜論疊見,豈不偉哉。



\end{pinyinscope}