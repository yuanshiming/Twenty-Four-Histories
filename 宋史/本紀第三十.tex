\article{本紀第三十}

\begin{pinyinscope}

 高宗七



 十二年春正月癸卯,罷樞密行府。庚申,孫近分司、漳州居住。



 二月丁丑,加建國公瑗為檢校少保,進封普安郡王。己卯,賜楊沂中名存中。丙戌,詔諸州修學宮。辛卯,蠲
 廣南東、西路駱科殘擾州縣今年租。鎮江、太平、池州、蕪湖大火。癸巳,金主許歸梓宮及皇太后,遣何鑄等還。



 三月丙申,臨安府火。壬寅,命普安郡王出就第,朝朔望。辛亥,以士人褭嘗營護岳飛為朋比,責建州居住。丙辰,胡世將卒。



 夏四月甲子朔,遣孟忠厚為迎護梓宮禮儀使,王次翁為奉迎兩宮禮儀使。丁卯,皇太后偕梓宮發五國城,金遣完顏宗賢、劉祹護送梓宮,高居安護送皇太后。庚午,賜禮部進士陳誠之以下二百五十四人及第、出
 身。戊寅,封韋淵平樂郡王。辛巳,皇后邢氏崩訃初至。甲申,增修臨安府學為太學。



 五月甲午,以鄭剛中為川、陜宣撫副使。乙未,遣沉昭遠等賀金主生辰。置淮西、京西、陜西諸路榷場。丙午,增築慈寧殿。停給度僧牒。乙卯,復試教官法。



 六月甲子,命侍從、臺諫、禮官雜議權奉攢宮。戌辰,以萬俟離為攢宮按行使。辛未,責降王庶為向德軍節度副使、道州安置。壬午,金國歸孟庾、李正民。甲申,以吳璘為檢校少師、階、成、岷、鳳四州經略使。



 秋七月壬
 辰朔,福州簽判胡銓除名、新州編管。丁酉,上皇后謚曰懿節,祔神主於別廟。己亥,以何鑄權參知政事。己酉,始制常行儀仗及造玉輅。乙卯,蠲廣南、湖北沿邊州軍免行錢。



 八月辛酉朔,兀□使來求商州及和尚、方山二原。丙寅,何鑄罷。甲戌,以萬俟離參知政事,充金國報謝使。壬午,皇太后至,入居慈寧宮。己丑,帝易緦服,奉迎徽宗及顯肅、懿節二后梓宮至,奉安於龍德別宮。是月,鄭剛中分畫陜西地界,割商、秦之半畀金國,存上津、豐陽、
 天水三縣及隴西成紀餘地,棄和尚、方山二原,以大散關為界。



 九月乙未,以孟忠厚為樞密使,充攢宮總護使。壬寅,大赦。乙巳,加秦檜太師,封魏國公。丙午,金使劉筈、完顏宗表等九人入見。戊申,以王次翁等充金國報謝使。藏金國誓書於內侍省。辛亥,加張中孚開府儀同三司,中彥靖海軍節度使。甲寅,杖殺偽福國長公主李善靜。以金州郭浩為金、房、開、達四州經略安撫使。始遣楊願使金賀正旦。



 冬十月乙丑,始聽中外用藥。丙寅,權攢
 徽宗皇帝及顯肅皇后於會稽永固陵,懿節皇后祔。乙亥,以翰林學士程克俊簽書樞密院事、權參知政事。丁丑,以皇太后回鑾,推恩進封秦檜為秦、魏兩國公,辭不拜。庚辰,以何鑄黨援岳飛,不主和議,責授秘書少監、徽州居住。甲申,皇太后生辰,上壽於慈寧宮。丁亥,置福建路提舉茶事司。



 十一月癸巳,樞密使張俊罷,進封清河郡王。以左司郎中李椿年為兩浙轉運副使,專治經界。乙未,加楊存中少保。己亥,禁貶謫人私至行在。庚子,作
 祟政、垂拱二殿。辛丑,劉光世薨。壬寅,曾祖姑秦、魯國大長公主薨。丙午,尹焞卒。庚戌,孟忠厚罷。左承事郎張戒坐黨趙鼎、岳飛停官。辛亥,遣張中孚、中彥還金國。十二月甲子,詔侍從、監察御史已上、監司、郡守各舉所知宗室。丙寅,幸秦、魯國大長公主第臨奠,又幸劉光世第臨奠。庚午,命太學弟子員以三百人為額。壬申,秦檜上《六曹寺監通用敕令格式》。癸酉,以李顯忠為保信軍節度使、御前選鋒軍統制,王進為御前諸軍都統制。是歲,斷
 大闢二十四人。



 十三年春正月戊戌,加上徽宗謚曰體神合道駿烈遜功聖文仁德憲慈顯孝皇帝。己亥,親饗太廟,奉上冊寶。癸卯,增建國子監太學。乙巳,復兼試進士經義、試賦。



 二月壬戌,初御前殿,特引四參官起居。甲子,制郊廟社稷祭器。乙丑,更永固陵曰永祐。丙寅,封韓世忠咸安郡王。乙亥,蠲雷、化等十州免行錢。丙子,造金、象、革、木四輅。庚辰,立太學及科舉試法。辛巳,秘書少監秦熹修《建炎以
 來日歷》成。乙酉,建景靈宮,奉安累朝神御。



 三月己亥,造鹵簿儀仗。乙巳,建社稷壇。丙午,築圜丘。振淮南饑民。仍禁遏糴。



 夏四月癸亥,頒鄉飲酒儀於郡國。甲戌,毀獄吏訊囚非法之具。



 閏月己丑,立貴妃吳氏為皇后。戊申,命史館編《靖康建炎忠義錄》。庚戌,楊政入見,加檢校少保,賜田五十頃。壬子,蠲諸路無名月樁錢。乙卯,王次翁罷。



 五月甲子,張九成坐黨趙鼎,南安軍居住。壬申,置國子博士、正、錄。乙亥,命諸路置放生池。丁丑,天申節,始上壽
 錫宴如故事。



 六月壬戌,禁三衙及諸軍市易,月增將官供給錢有差。壬寅,程克俊罷,以萬俟離兼權簽書樞密院事。戊申,詔諸路提刑歲舉部內廉明平恕獄官。庚戌,金遣洪皓、張邵、朱弁來歸。



 秋七月甲子,詔求遺書。罷捕賊補官格。丙寅,處州兵士楊興等謀作亂,事覺伏誅。戊辰,置諸州銅作務。壬申,雨雹。蠲浙西貧民逋負丁鹽錢。



 八月丙戌,遣吏部侍郎江邈奉迎累朝神御於溫州。丁亥,命諸路有出身監司一員提舉學事。戊戌,洪皓至自
 金國,入見。己亥,遣鄭樸等使金賀正旦,王師心等賀金主生辰。鄭剛中獻黃金萬兩。辛丑,復昌化、萬安、吉陽軍。知階州田晟將所部三千人赴行在。丁未,以晟主管侍衛馬軍司公事,其眾隸焉。己酉,加錢愐太尉。庚戌,詔監司、守臣講求恤民事宜。



 九月丁巳,宗室子偁卒於秀州。甲子,洪皓出知饒州。戊辰,命諸路置敦宗院。己巳,詔淮東、京西監司歲終上州縣所增戶口,為守令殿最。庚午,以兵部侍郎司馬樸死節,贈兵部尚書,賜其家銀絹。癸
 酉,詔諸州守、貳提舉學事,縣令、佐主管學事。戊寅,蠲淮南逋欠坊場錢及上供帛。



 冬十月己丑,秦檜上《監學敕令格式》。庚寅,制渾天儀。乙未,奉安累朝帝後神御於景靈宮。



 十一月庚申,日南至,合祀天地於圜丘,太祖、太守並配,大赦。十二月癸未朔,日食,雲陰不見。辛卯,毀私鑄毛錢。癸巳,建秘書省。丁酉,增太學弟子員二百。己亥,郭浩入見。丁未,命行在宗子入宮學。己酉,金遣完顏曄等來賀明年正旦。



 是月,始頒來歲歷於諸路監司、守臣。是
 歲,關外初行營田。



 十四年春正月丁巳,遣羅汝楫等報謝金國。甲子,臨安府火。戊寅,命普安郡王為子戴解官持服。



 二月丁亥,復置靖州新民學。癸巳,蠲江、浙諸路逋欠錢帛。戊戌,初命四川都轉運司歲撥總制司錢百七十三萬緡,市紬絹綿輸於鄂州總領所。丙午,罷萬俟離。定宗學生額為百員。己酉,以資政殿學士樓照簽書樞密院事兼權參知政事。加郭浩檢校少保。



 三月乙卯,蠲江、浙、京、湖積欠上
 供錢米。丁卯,避金太祖嫌名,改岷州為西和州,川、陜宣撫司為四川宣撫司。己巳,幸太學。蠲汀、漳、泉、建四州經賊殘蹂民戶賦役一年。壬申,解潛坐黨趙鼎,責授濠州團練副使、南安軍安置。己卯,詔舉賢良。



 夏四月甲申,詔刑部及監司決絕滯訟。丁亥,初禁野史。虔州民析其屋,朽柱中有文曰「天下太平年」。甲午,金人來求淮北人之在南者,詔原者聽還。遣馬軍司統領張守忠討海賊朱明。



 五月丙辰,詔階、成、西和、鳳四州募兵赴行在。甲子,樓
 照罷。乙丑,以御史中丞李文會簽書樞密院事兼權參知政事。丙寅,婺州大水。己巳,金始遣烏延和等來賀天申節。辛未,楚州鹽城縣海水清。是月,嚴、信、衢、建四州水。



 六月甲申,蠲江、浙州縣酒稅、坊場、綱運、倉庫積年逋負。孫近再奪三官,移南安軍居住。丁亥,加高世則少保。戊子,安南國入貢。癸巳,宣州涇縣妖賊俞一作亂,守臣捕滅之。乙未,振江、浙、福建被水之民。丙申,內侍白鄂坐誹謗,及其客張伯麟俱黥配吉陽軍。特贈子偁太子少師,
 官給葬事。庚子,奪萬俟離三官、歸州居住。乙巳,置國子監小學。



 秋七月戊午,金人殺王倫於河間府。丙寅,立明法科兼經法。丙子,幸秘書省。



 八月癸未,撫州獻瑞禾。庚寅,以李椿年權戶部侍郎,仍治經界。乙未,遣林保使金賀正旦,宋之才賀金主生辰。



 九月辛酉,分利州為東、西路,以吳璘為利州西路安撫使,楊政利州東路安撫使。甲子,命郡守終更入見,各舉所部縣令一人。壬申,趙鼎移吉陽軍安置。癸酉,命臨安府索蔡京子孫逮赴貶所,
 遇赦永不量移。



 冬十月甲午,從右正言何若言,請戒內外師儒之官,黜伊川程氏之學。乙未,加韋淵少師。己亥,以永、道、郴三州、桂陽監及茶陵縣民多不舉子,永蠲其身丁錢絹米麥。



 十一月甲子,復內教,即禁中閱試三衙將士。癸酉,李光移瓊州安置。乙亥,朱勝非薨。十二月丁丑朔,潼川府路轉運判官宋蒼舒獻嘉禾一莖九穗。己卯,命諸郡收養老疾貧乏之民,復置漏澤園,葬死而無歸者。丁酉,李文會罷,尋責筠州居住。庚子,以御史中丞
 楊願簽書樞密院事兼權參知政事。癸卯,金遣孛散溫等來賀明年正旦。是月,汀賊華齊寇漳州長泰縣,安撫司遣兵捕之,為所敗,將佐趙成等死之。是歲,四川宣撫司始取民戶稱提錢歲四十萬緡,以備軍費。



 十五年春正月丁未朔,御大慶殿,初行大朝會禮。戊申,瀘南安撫使馮楫獻嘉禾。己未,分經義、詩賦為兩科取士。辛酉,初置籍田。丁卯,減成都府路對糴米三之一、宣撫司激賞錢三十萬緡。戊辰,命戶部侍郎王鈇措置兩
 浙經界。辛未,初命僧道納免丁錢。



 二月戊寅,增太學弟子員百人。乙未,詔州縣科折之數,第三等戶毋或均配。己亥,封崇國公璩為恩平郡王,出就第。



 三月甲子,遣敷文閣待制周襟、馬觀國、史願、諸將程師回、馬欽、白常皆還金國。



 夏四月丙子朔,賜秦檜第一區。戊寅,彗星出東方。癸未,避殿減膳,命監司、郡守條上便民事宜,提刑巡行決獄。賜禮部進士劉章以下三百人及第、出身。丁亥,以彗出,大赦。癸巳,彗沒。甲午,遣後軍統制張淵討捕福
 建盜賊。庚子,罷四川都轉運司。



 五月丙辰,客星見。戊午,命貧民產子賜義倉米一斛。甲子,金遣完顏宗尹等來賀天申節。六月乙亥朔,日有食之。丁丑,幸秦檜第。乙酉,加檜妻婦子孫官封。丁亥,客星沒。秋七月戊申,復置利州鑄錢監。戊午,命監司審查縣令治狀顯著及老懦不職者,上其名以為黜陟。蠲廬、光二州上供錢米一年。丁卯,免汀、漳二州秋稅及處州三縣被水民家紬絹,鄂州舊額絹各一年。己巳,蠲四川轉運司積貸常平錢十三
 萬緡。



 八月申戌朔,禁收折帛合零錢,止輸實數。乙亥,蠲京西路請佃田租及州縣場務稅錢二年。己亥,改諸路提舉茶鹽官為提舉常平茶鹽公事,川、廣以憲臣兼領。辛丑,復增太學弟子員二百。



 九月辛酉,遣錢周材使金賀正旦,嚴抑賀金主生辰。



 冬十月乙亥,帝書「一德格天之閣」賜秦檜,仍就第賜宴。丙子,楊願罷。癸未,以樞密都承旨李若谷簽書樞密院事兼權參知政事。武岡軍徭人楊再興降。庚寅,以翰林學士承旨秦熹為資政殿學
 士、提舉萬壽觀兼侍讀,恩數視執政。辛卯夜,雷。癸巳,蠲安豐軍上供錢米二年。甲午,以汪勃言折彥質黨趙鼎,郴州安置。庚子,置四川宣撫司總領錢糧官。辛丑,命秦熹班簽書樞密之下。



 十一月甲辰,加錢忱少保,錢愐開府儀同三司。丙辰,郭浩卒。丙寅,全給秦檜歲賜公使錢萬緡。



 閏月己卯,罷明法新科。十二月戊午,置江陰軍市舶務。甲子,命右司員外郎李朝正同措置經界。丁卯,金遣蒲察說等來賀明年正旦。



 十六年春正月戊子,增太學外舍生額至千人。壬辰,親饗先農於東郊,行籍田禮,執耒耜九推,詔告郡縣。



 二月辛丑,割金州豐陽縣、洋州乾祐縣畀金人。壬寅,毀諸路淫祠。癸丑,建秦檜家廟。



 三月庚午朔,建武學,置弟子員百人。辛卯,造秦檜家廟祭器。乙未,增建太廟。己亥,立淮東、江東、兩浙、湖北州縣歲較營田賞罰格。



 夏四月壬子,禁州縣預借民稅及和買錢。戊午,定選試武士弓馬去留格。



 五月壬申,浚運河。命諸路漕臣兼提舉學事。癸未,
 初作太廟祏室。丙戌,作景鐘。丁亥,金遣烏古論海等來賀天申節。



 六月,安南獻馴象十。



 秋七月壬申,以張浚上疏論時事,落節鉞、連州居住。壬辰,立秘書省獻書賞格。丙申,復何鑄為端明殿學士兼侍讀。



 八月辛丑,築高禖壇。壬子,遣邊知白使金賀正旦,周執羔賀金主生辰。



 九月甲戌,命何鑄等為金國祈請使,請國族。甲午,賞統制張淵、韓京等討捕福建、廣東諸盜功,各進官有差。



 冬十月戊戌,帝觀新作禮器於射殿,撞景鐘,奏新樂。十一日
 丙子,合祀天地於圜丘,大赦。庚辰,罷州縣新創稅場。癸未,復置御書院。己丑,加潘正夫少保。



 十二月戊戌,彗見西南方,乙巳,滅。辛酉,金遣盧彥倫等來賀明年正旦。



 十七年春正月己巳,命諸路收試中原流寓士人。己卯,禁監司、郡守進羨餘。辛卯,以舉人多冒貫,命州縣每三歲行鄉飲酒禮以貢士。壬辰,以李若穀參知政事,御史中丞何若簽書樞密院事。癸巳,進秦熹為資政殿大學士。



 二月乙巳,親祠高禖。辛酉,李若谷罷。



 三月乙亥,何若
 罷。己卯,以翰林學士段拂參知政事。乙酉,改封秦檜為益國公。戊子,改命張俊為靜江、寧武、靖海軍節度使,韓世忠鎮南、武安、寧國軍節度使。落李若谷資政殿學士、江州居住。



 夏四月丙申,蠲諸路免行錢三之一。己亥,以御史中丞汪勃簽書樞密院事。己未,詔趙鼎遇赦永不檢舉。以前貶所潮州錄事參軍石恮待遇鼎厚,除名、潯州編管。



 五月甲子,詔舉賢良。乙丑,雨雹。乙巳,洪皓責濠州團練副使、英州安置。辛巳,金遣完顏卞等來賀天申
 節。



 六月乙卯,禁招安盜賊。戊午,改命普安郡王瑗為常德軍節度使,恩平郡王璩武康軍節度使。



 秋七月庚辰,召鄭剛中赴行在。辛巳,太白晝見。以徽猷閣待制、知成都府李璆權四川宣撫使。癸未,命李璆同總領四川財賦符行中參酌減放四川重斂。戊子,以吳璘充御前諸軍都統制兼知興州。



 八月庚子,罷建州創置賣鹽坊。癸卯,趙鼎薨於吉陽軍。戊申,遣沈該使金賀正旦,詹大方賀金主生辰。丁巳,以諸路羨餘錢充月樁之數。加邢孝
 揚太尉。



 九月己巳,減四川科率虛額錢歲二百八十五萬緡。癸酉,詔以四川宣撫司降賜庫米一百萬石,均減對糴。乙亥,蠲江南東、西道諸州月樁錢。丙子,鄭剛中罷。丙戌,減江、浙諸州折帛錢。



 冬十月辛卯朔,日有食之。癸卯,建太一宮。丁未,命太常歲以春秋二仲薦獻攢宮,季秋遣御史按視。己酉,進楊存中為少傅。己未,臨安府甘露降。



 十一月丙寅,秦檜上《重修免役敕令格式》。丁卯,復賜進士聞喜宴。十二月辛卯朔,禁諸州擅釋放流配命
 官及事乾邊防切要之人。甲寅,鄭剛中落職、桂陽監居住。丙辰,金遣完顏宗藩等來賀明年正旦。



 十八年春正月己巳,幸天竺寺,遂幸玉津園。



 二月乙未,段拂罷,尋落職、興國軍居住。以汪勃兼權參知政事。辛亥,聽趙鼎歸葬。



 三月丁丑,命楊政、吳璘招關、陜流民補殿前軍。戊寅,罷汀州諸縣上供銀,蠲茶鉛本錢之半。庚辰,幸新太一宮。壬午,以秦熹知樞密院事。乙酉,禁民私渡淮及招納叛亡。



 夏四月戊子朔,日有食之。庚子,秦熹
 乞避父子共政,以為觀文殿學士、提舉萬壽觀兼侍讀、提舉秘書省。壬寅,命熹恩禮視宰臣班次,亞右僕射。甲辰,賜禮部進士王佐以下三百三十人及第、出身。丙辰,加士□開府儀同三司。



 五月戊辰,加吳益太尉。乙亥,裁損奉使賞給。丙子,金遣蕭秉溫等來賀天申節。癸未,以李顯忠私取故妻於金,降為平海軍承宣使、臺州居住。甲申,罷四川宣撫司,以李璆為四川安撫制置使。是月,徽州慶雲見。



 六月甲辰,築九宮貴神壇於東郊。戊申,士
 民曹溥等上尊號,不許。是月,遣太府丞宋仲堪詣江州置獄,鞫鄭剛中欺隱官錢。福州候官縣有竹實如米,饑民採食之。是夏,浙東西、淮南、江東旱。



 八月丙申,汪勃罷。丁酉,以工部尚書詹大方簽書樞密院事兼權參知政事。禁州縣士民飾詞舉留官吏。



 閏月庚申,免江、浙、湖南今歲和糴。甲子,命臨安、平江二府、淮東西、湖北三總領所歲糴米百二十萬石,以廣儲蓄。壬申,遣王墨卿使金賀正旦,陳誠之賀金主生辰。甲申,辛道宗降官、房州羈
 管。乙酉,禁奉使三節人出境博易。福建諸州賊平,以所創招奇兵為殿前司左翼軍。



 九月丙午,詹大方薨。



 冬十月丙辰,以御史中丞餘堯弼簽書樞密院事兼權參知政事。



 十一月乙酉朔,升感生帝為上祀。己亥,胡銓移吉陽軍編管。壬寅,鄭剛中責濠州團練副使、復州安置。戊申,禁四川買馬官吏私市蠻馬。辛亥,振紹興府饑。十二月乙卯朔,振明、越、秀、潤、徽、婺、饒、信諸州流民。丙寅,借給被災農民春耕費。丁卯,命利路三都統措置營田,以其
 租充減免對糴之數。戊辰,蠲被災下戶積欠租稅。庚辰,金遣召守忠等來賀明年正旦。



 十九年春正月甲申朔,以皇太后年七十,帝詣慈寧殿行慶壽禮。甲午,罷國信所回易北貨。癸卯,幸天竺寺,遂幸玉津園。牰瀼露〕螅□麑睟畢□從萌思攔砑霸旃貧荊庵剛



 三月癸未朔,日有食之。甲辰,鄭剛中移封州安置,子良嗣等亦除名編管。



 夏四月丁巳,立孳生牧馬監賞罰格。丙寅,秘閣修撰張邵上秦檜在金國代
 徽宗與粘罕書稿,詔付史館,以邵為徽猷閣待制。戊寅,湖、廣、江西路、建康府並甘露降。



 五月壬午朔,汀、漳、泉三州民田被賊蹂踐,蠲其二稅。戊戌,賞平福建群盜功,以選鋒軍統制劉寶為武泰軍承宣使,餘將士遷秩有差。庚子,金遣唐括德溫等來賀天申節。丁未,減連、英、循、惠、新、恩六州免行錢。



 六月丁巳,茶陵縣丞王庭珪作詩送胡銓,坐謗訕停官、辰州編管。戊午,秦檜上《吏部續降七司通用法》。



 秋七月壬寅,頒諸農書於郡邑。



 八月辛未,刺
 浙東諸州強盜當配者充沿海諸軍。



 九月戊申,命繪秦檜像,仍作贊賜之。



 冬十月己未,湖南副總管辛永宗停官、肇慶府編管。



 十一月壬辰,合祀天地於圜丘,大赦。辛丑,李椿年以經界不均罷。丁未,立州縣墾田增虧賞罰格。是月,命復蠟祭。十二月丁巳,金岐王亮弒其主但自立。己未,詔無子女戶、得解舉人、太學生之獨居者並免役。己巳,命四川制置司歲募扈衛三百人赴行在。丁丑,金遣完顏袞等來賀明年正旦。



 二十年春正月丁亥,秦檜入朝,殿前司軍士施全道刺之,不中。壬辰,磔全於市。癸卯,趣諸路轉運司及守臣畢經界事。丙午,兩浙轉運副使曹泳言,李孟堅誦其父光所撰私史,語涉譏謗,詔送大理寺。



 二月戊申朔,立守貳、令尉營田增虧賞罰格。庚戌,禁民春月捕鳥獸。蠲靜江府、昭州上供折布錢三之一。壬子,罷經界所覆實官吏。庚申,免海外四州及瀘、敘二州、長寧軍經界。



 三月庚辰,金遣完顏思恭等來報即位。癸未,以餘堯弼參知政事,
 給事中巫伋簽書樞密院事。丙戌,遣堯弼等賀金主即位。戊子,以秦熹為觀文殿大學士、萬壽觀使。丙申,李孟堅獄具。詔李光遇赦永不檢舉,孟堅除名、峽州編管,胡寅、程瑀、潘良貴、張燾等八人緣坐,黜降有差。戊戌,詔改正經界法之厲民者。庚子,以巫伋兼權參知政事。壬寅,胡寅責果州團練副使、新州安置。



 夏四月壬子,以沒入官田悉歸常平司,禁募民佃種。癸酉,置力田科,募江、漸、福建民耕兩淮閑田。是月,信州妖賊黃曾等作亂,陷貴
 溪縣,江西兵馬鈐轄李橫等討平之。



 五月庚辰,申禁諸軍差承接文字使臣伺察朝政。癸未,秦檜上《中興聖統》。甲午,金就遣完顏思恭等來賀天申節。



 六月癸亥,加秦熹少保。詔大理寺鞫前太常主簿吳元美譏謗獄。丙寅,禁民結集經社。是月,建州民張大一作亂。



 秋七月丙子,罷招刺禁軍。庚寅,罷泉、漳、汀三州經界。



 八月申辰朔,量移張浚永州、孫近虔州、萬俟離沅州、李若谷饒州、李文會江州、段拂南康軍,並居住。雷州守臣王趯坐交通趙
 鼎、李光停官。戊申,改建大理寺。辛酉,遣陳誠之使金賀正旦,王𥍓賀金主生辰。



 九月甲申,以吳元美譏毀大臣,除名、容州編管。丙申,侍御史曹筠以附下罔上罷。



 冬十月戊辰,右迪功郎安誠坐文字謗訕,送惠州編管。秦檜有疾。庚午,命執政赴檜第議事。



 十二月甲子,檜始朝,命肩輿入宮門,二孫扶掖升殿,不拜。己巳,金遣蕭頤等來賀明年正旦。



 二十一年春正月癸未,以兩淮民復業未久,寬其租稅。
 庚子,蠲平江府折帛錢三年。



 二月甲寅夜,雨雹。乙卯,詔諸州置惠民局,官給醫書。壬戌,遣巫伋等為金國祈請使,請歸淵聖皇帝及皇族、增加帝號等事。癸亥,以餘堯弼兼簽書樞密院事。



 三月丁丑,雨雹。丁亥,蠲江、浙、荊湖等路中戶以下積年逋負。夏閏四月己卯,禁三衙掊克諸軍。丁亥,賜禮部進士趙逵以下四百四人及第、出身。



 五月辛亥,罷利州路選刺義士。戊午,金遣劉長言等來賀天申節。以吳璘、楊政、田師中並為太尉。



 六月甲戌,括
 淮南佃田所隱頃畝,以理租稅。辛巳,命歲給大理寺、三衙及州縣錢,和藥劑療病囚。



 秋七月壬寅,以集英殿修撰、知衢州曹筠為四川安撫制置使。辛亥,罷柴米稅。癸亥,詔州縣官嘗被科率害民重罪者,不得任守令親民官。



 八月辛未,秦檜上《重修諸路茶鹽法》。壬申,韓世忠薨,詔進太師致仕。癸酉,追封通義郡王。禁郡守特斷。乙亥,加岳陽軍節度使士撙開府儀同三司,充萬壽觀使。甲申,遣陳夔使金賀正旦,陳相賀金主生辰。



 九月戊戌朔,
 籍寺觀絕產以贍學。乙巳,均科處州丁鹽錢。丁巳,增築景靈宮。是月,巫伋使還,所請皆不許。



 冬十月甲戌,幸張俊第。壬午,進俊為太師,升從子子蓋為安德軍節度使。甲申,夜有赤氣。



 十一月庚戌,餘堯弼罷。乙卯,命提舉常平官修復陂湖。丁巳,進義副尉劉允中坐指斥謗訕棄市。十二月壬申,雷。癸巳,金遣兀□魯定方等來賀明年正旦。



 二十二年春正月丁未,加韋淵太保。



 三月丁酉,以王庶
 二子之奇、之荀謗毀朝政,並除名,之奇梅州、之荀容州編管。甲辰,以直龍圖閣葉三省、監都作院王遠通書趙鼎、王庶,力詆和議,言涉謗訕,三省落職、筠州居住;遠除名、高州編管。丁巳,遣司農丞鐘世明詣福建路籍寺觀絕產田宅入官,其後歲入錢三十四萬緡。



 夏四月丙子,巫伋罷。辛巳,以御史中丞章復簽書樞密院事兼權參知政事。



 五月癸丑,金遣田秀穎等來賀天申節。是月,襄陽大水,容州野蠶成繭。



 秋七月甲午朔,加封程嬰、公孫
 杵臼、韓厥為公,升中祀。丁巳,虔州軍卒齊述殺殿前司統制吳進、江西同統領馬晟,據州叛。



 八月己卯,遣鄂州都統制田師中發兵同江西安撫使張澄、殿前司游奕軍統制李耕討述。



 九月乙未,又遣左翼軍統制陳敏相繼討之。癸丑,章復罷。



 冬十月甲戌,以御史中丞宋樸簽書樞密院事兼權參知政事。就命李耕知虔州。庚辰,以黃巖縣令楊煒誹謗,除名、萬安軍編管;知臺州蕭振落職、池州居住。



 十一月戊申,合祀天地於圜丘,大赦。丁巳,
 立薦舉受財刑名。李耕入虔州,盡誅叛兵,虔州平。十二月辛酉朔,減夔州路及蒲江、淯井兩監鹽錢歲八萬二千緡有奇。戊子,金遣張利用等來賀明年正旦。



\end{pinyinscope}