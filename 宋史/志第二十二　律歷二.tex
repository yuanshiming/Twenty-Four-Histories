\article{志第二十二 律歷二}

\begin{pinyinscope}

 應天乾
 元儀天歷



 步月離入先後歷《乾元》謂之月離。《儀天》謂之步月離。



 離總:五萬五千一百二十、秒一千二百四十二。《乾元》轉分一萬六千二百、秒一千二百四。《儀天》歷終分二十七萬八千三百一、秒一百六十五。



 轉日:二十七、五千五百四十六、秒六千二百一十。《乾元》轉歷二十七、一千六百三十、秒六千二十。《儀天》歷周二十七、五千六百一、秒一百六十五。



 歷中日:一十三、七千七百七十四、秒三千一百五。《乾元》不立此法。《儀天》歷中十三日、七千八百五十、秒五千八十二半。《儀天》有象限六日、八千九百七十五、秒二千五百四十一少。



 朔差日:一、九千七百六十二、秒三千七百九十。《乾元》轉差一、三千八百六十九、秒三千九百八十。《儀天》會差日一、九千八百五十七、秒九千八百三十五。



 《儀天》又有象差日空、四千九百八十、秒四千九百五十八太;望一百八十二度六千三百四十四、秒四千
 九百五十。



 度母:一萬一百。



 秒法:一萬。二歷同



 求天正十一月朔入先後歷:《乾元》謂之求月離入歷,求弦、望入歷。《儀天》謂之推天正經朔入歷。



 以通餘減元積,餘以離總去之為總數;不盡者,半而進位,以元法收為日,不滿為分。如歷中日以下為入先歷;以上者去之,為入後歷。命日,算外,即得天正十一月朔入先後歷日分。累加七日、三千八百二十七分、
 秒六,盈歷中日及分秒去之,各得次朔、望入先後歷日分。《乾元》以朔餘減歲積分,以轉分去之,餘以五因之,滿元率收之為度;以弦策加之,即弦、望所入。以轉差加之,得後朔歷;累加之,即得弦、望入歷及分。《儀天》以閏餘減歲積分,餘以歷終分去之,不滿,以宗法除之為日;在象限以下為初限,以上去之,餘為末限,各為入遲疾歷初、末限。



 七日:初數八千八百八十八,《乾元》初二千六百
 一
 十二。



 末數一千一百一十四。末三百二十八。



 十四日:初數七千七百七十四,《乾元》初二千二百八十五。



 末數二千二百二十八。末六百五十五。《乾元》又有二十一日:初一千九百五十八,末九百八十二;二十八日:初一千六百三十二,末一千三百九。



 又《儀天》法月離先後度數:《乾元》謂之月離陰陽差。《儀天》謂之求朔弦望升平定
 數。



 以月朔、弦、望入歷先後分通減元法,餘進位,下以其日損益率展之,以元法收為分,所得,損益次日下先後積為定數。其七日、十四日,如初數以下者,返減之,以上者去之,餘,返減末數,皆進位,下以損益率展之,各滿末數為分,損益次
 日下先後積為定數。《乾元》置入歷分,以其日損益率乘之,元率收為分,損益其下陰陽差為定數。四七術,如初數已下者,以初率乘之,如初數而一,以損益陰陽差為定數;若初數以上者,以初數減之,餘乘末率,末數除之,用減初率,餘加陰陽差,各為定數。



 朔弦望定日:以日躔、月離先後定數,先加後減朔、弦、望中日,為定日。二歷法同。



 推定朔弦望日辰七直:以天正所盈之日加定積,視朔、弦、望中日,如入大、小雪氣,即加去年天正所盈之日分;若入冬至氣者,即加今年天正所盈之日分。



 日滿七十六去之,不滿者,命從金星甲子,算外,即得定朔、弦、望日辰星直也。視朔乾名與後朔同者大,不同者小,其月無中氣者為閏。又視朔所入辰分皆與二分相減,
 餘二收,用減八分之六,其朔定小餘如此;以上者進一日;朔或有交正見者,其朔不
 進。
 定望小
 餘在日出分以下者,退一日,若有虧初在辰分以下亦如之。二歷法同。



 《儀天》又有求朔弦望加時月度,置弦、望加時日度,其合朔加時月與太陽同度,其日、度便為月離所次;餘加弦、望象度及餘秒,滿黃道宿次去之,即定朔、弦、望加時日、度也。



 九道宿度:《乾元》、《儀天》皆謂之月行九道。



 凡合朔所交,冬在陰歷,夏在陽歷,月行青道;冬至、夏至後,青道半交在春分之宿,出黃道東;立夏、立冬後,青道半交在立春之宿,出黃道東南:至所沖之宿亦如之。



 冬在陽歷,夏在陰歷,月行白道;冬至、夏至後,白道半交在秋分之宿,出黃道西;立冬、立夏後,白道半交在立秋之宿,出黃道西北:至所沖之宿亦如之。



 春在陽歷,秋在陰歷,月行朱道;春分、秋分後,朱道半交在夏至之宿,出黃道南;立春、立秋後,朱道半交在立夏之宿,出黃道西南:至所沖之宿亦如之。



 春在陰歷,秋在陽歷,月行黑道。春分、秋分後,黑道半交在冬至之宿,出黃道北;立春、立秋後,黑道半交在立冬之宿,出黃道東北:至所沖之宿亦如之。



 四序月離為八節,九道斜正不同,
 所入七十二候,皆與黃道相會。各距交初黃道宿度,每五度為限。初限十二,每限減半,終九限又減盡,距二立之宿減一度少強,卻從減盡起,每限減半,九限終十二而至半交,乃去黃道六度;又自十二,每限減半,終九限又減一度少強,更從減盡起,每限增半,九限終十二,復與日軌相會。交初、交中、半交,各以限數,遇半倍使,乘限度為泛差。其交中前後各九限,以距二至之宿前後候數乘之,半交前後各九限,各至二分之宿前後候數乘
 之,皆滿百而一為黃道差。在冬至之宿後,交初前後各九限為減,交中前後各九限為加;夏至之宿後,交初前後各九限為加,交中前後各九限為減。大凡月交後為出黃道外,交中後為入黃道內。半交前後各九限,在春分之宿後出黃道外,秋分之宿後入黃道內,皆以差為加;在春分之宿後入黃道內,秋分之宿後出黃道外,皆以差為減。倍泛差,退一位,遇減,身外除三;遇加,身外除一。



 又以黃道差減,為赤道差。交初、交中前後各九限,以差加;半交前後
 各九限,皆以差減。以黃赤道差減黃道宿度為九道宿度,有餘分就近收為太、半、少之數。《乾元》初數九,每限減一,終於一,限數並同,即八十四除之。《儀天》初數一百一十七,每限減一十,終於二十七,以一百一除。二歷皆不身外為法。初中正交、春秋二分、冬夏二至前後各九限,加減並同《應天》。又《儀天》即除法是九十乘黃道泛差,一百一收為度,乃得月與黃、赤道定差。以上入交定月出入各六度相較之差,黃道隨其日行所向,斜正各異,餘皆同《應天》。《儀天》有求定朔望加時入遲疾歷初末限,置經朔、望入遲疾初末限日及餘秒,如求定朔、弦、望法入之,即各得所求。又求初中正交入歷,置其朔、望加時入遲疾歷初末限日及餘秒,視其日月行入陰陽歷日及餘秒,如近前交者即加,近後交者即返減交中日餘,乃如之,各得初、中、正交入遲疾歷初末限日及餘秒也。其加減滿或不足,即進
 退象限及餘秒,各得所求。又求朔望加時及初、中、正交入遲疾限日入歷積度,各置小餘,以其日歷定分乘之,宗法收之為分,一百一除之為度,以加其日下歷積度,各得所求。又《乾元》、《儀天》有求正交黃道月度,《乾元》元率通定交度及分,以一百二十七乘之,滿九十五而一,進一等,復收為入交度,用減其朔加時日度,即朔前月離正交黃道宿度。《儀天》置朔、望及正交歷積度,以少減多,餘為月行去交度及分;乃視其朔望在交前者加、交後者減朔望加時黃道月度,為初、中、正交黃道月度也。



 九道交初月度:《乾元》謂之月離入交九道正交月度、九道朔度。《儀天》謂之求月離正交九道宿度。



 置月離交初黃道宿度,各以所入限數乘之,遇半倍使



 如百而一,為泛差;用求黃、赤二道差,依前法加減之,即月
 離交初九道宿度。《乾元》以日躔陰陽差陽加陰減,為朔、望常分;又以所入限率乘,正交黃道宿度相從之,以求黃、赤二道差,如前加減,為月離正交九道宿度;以入交定度加而命之,即朔月離宿度。《儀天》置正交月離黃道,以距度下月九道差,宗法乘之,以距度所入限數乘度,餘從之,為總差;半而退位,一百一收之,又計冬、夏二至以求度數乘,滿九十而一為度差,依前法加減,為正交月離九道。



 求九道朔月度:百約月離先後定數,後加先減四十二,用減中盈而從朔日,乃加交初九道宿次,即得所求。《乾元》置九道正交之度及分,以入交定度加之,命以九道宿次,即其朔加時月離宿度及分也。《儀天》法見下。《乾元》又有定交度,置月離陰陽定數,以七十一乘之,滿九百一除之為分,用陰減陽加常分為度及分。



 求九道望月度:《儀天》謂之求定朔、望加時日月度。



 以象積加朔九道月度,命以其道,即得所求。《乾元》置朔、望加時日相距之度,以天中度及分加之,為加時象積;用加九道朔月度,命以其道宿次去之,即望日月度及分也。自望推朔亦如之。《儀天》求定朔望加時九道日度,以其朔、望去交度,交前者減之,交後者加之,滿九道宿度去之,即定朔、望加時九道日度也。求定朔望加時九道月度,置其日加時九道日度,其合朔者非正交,即日在黃道、月在九道各入宿度,多少不同,考其去極,若應繩準。故云月與太陽同度也。如求黃道月度法,盈九道宿次去之,各得其日加時九道宿度,自此以後,皆如求黃道月度法入之,依九道宿度行之,各得所求也。



 求晨昏月:《乾元》謂之月離晨昏度。《儀天》謂之求晨昏月度。



 置後歷七日下離分,
 與其日離分相比較,取多者乘朔、望定分,取少者乘晨昏分,皆滿元法為分,百除為度分,仍相減之,朔、望度多者為後,少者為前。



 各得晨昏前後度分;前加後減朔、望九道月度為晨昏月。《乾元》置其月離差,在三百九十三以上者,用乘朔、望定分,以下者,只用三百九十三乘,為加時分;元率除之,進一位,二百九十四收為度;又以離差乘晨昏分,亦如前收之為度,與加時度相減之,加時度多為後、少為前,即得晨昏前後度及分,加減如《應天》。《儀天》以晨昏分減定朔、弦、望小餘為後,不足者,返減之為前,以乘入歷定分,宗法除之,一百一約之為度,乃以前加後減加時月度為晨昏月度。



 晨昏象積:《儀天》謂之求晨昏程積度。



 置加時象積,以前象前後度前
 減後加,又以後象前後度前加後減,即得所求。《乾元》法同。《儀天》以所求朔、弦、望加時日度減後朔、弦、望加時日度,餘加弦、望度及餘,為加時程積;以所求前後分返其加減,又以後朔、弦、望前後度分依其加減,各為晨昏程積度及餘也。



 求每日晨昏月:《儀天》謂之求每日入歷定度。



 累計距後象離分,百除為度分,用減晨昏象積為加,不足,返減,以距後象日數除之,為日差;用加減每日離分,百除為度分,累加晨昏月,命以九道宿次,即得所求。《乾元》法同。《儀天》從所求日累計距後歷每日歷度及分,以減程積為進,不足,返減之,餘為退,以距後朔、弦、望日數均之,進加退減每日歷定度及分,各為每日歷定
 度及分也。



 步晷漏



 求每日晷景去極度晨分:《乾元》謂之晷景距中度晨分。《儀天》別立法,具後。



 各以氣數相減為分,自雨水後法十六,霜降後法十五,除分為中率,二率相減,為合差;半之,加減中率為初、末率。前多者,加為初、減為末;前少者,減為初、加為末。



 又以元法除合差,為日差;後多者累益初率,後少者累減初率。



 為每日損益率;以其數累積之,各得諸氣初數
 也。《乾元》法同。



 求昏分:以晨分減元法為昏分。《乾元》謂之元率,《儀天》謂之宗法。



 求每日距中度:《乾元》同。《儀天》謂之求每日距子度。



 以百乘晨分,如二千七百三十八為度,不盡,退除為距子度,用減半周天度,餘為距中星度分;倍距子度分,五等除,為每更度分。《乾元》百約晨分,進一位,以三千六百五十三乘,如元率收為度,餘同《應天》。《儀天》置晷漏母,五因,進一位,以一千三百八十二、小分五十五、微分三十五除為度,不盡,以一千三百六十八、小分八十六退除,皆為距子度,餘同《應天》。



 求每日昏明中星:《乾元》謂之昏曉率星。



 置其日赤道日躔宿次,以
 距南度分加而命之,即其日昏中星;以距子度分加之,為夜半中星;又加之,為曉中星。二歷法同。



 求五更中星:置昏中星為初更中星;以每更度分加之,得二更初中星;又加之,得三更初中星;累加之,各得五更初中星所臨。二歷法同。



 求日出入時刻:《乾元》謂之求晝夜出入辰刻。《儀天》謂之求日出入晨刻及分。



 以二百五十加晨減昏為出入分,以八百三十三半除為時,不滿,百除為刻分,命如前,即得所求。《乾元》以七十三半加晨減昏為出入分,各以辰
 法除之。為辰數;不盡,以五因之,滿刻法為刻,命辰數起子正,算外,即日出入辰刻也。《儀天》置其日晷漏母,以加昏明,餘以三因,滿辰法除為辰數,餘以刻法除為刻,不滿為分,辰數命子正,算外,即日出辰刻及分。乃置日出辰刻及分,以加晝刻及分,滿辰法及分除為辰數,不滿,為入時之刻及分。乃置其辰數,命子正,算外,即得日入辰刻及分。



 晝夜分:《乾元》謂之晝夜刻。《儀天》謂之求每日夜半定漏、求每日晝夜刻。



 倍日出分,為夜分;減元法,為晝分;百約,為盡夜分。《乾元》置日入分,以日出分減之為晝分,以減元率為夜分,以五因之,以刻法除為晝夜刻分。《儀天》先求夜半定漏,置其日晷漏母,以刻法除之為刻,不滿,三因為分,為夜半定漏及分。置夜半定漏刻及分,倍之,其分滿刻法為刻,不滿為分,即得夜刻及分。以夜刻減一
 百刻,餘者為晝刻及分,減晝五刻,加夜刻,為日出沒刻之數。



 更籌:《乾元》謂之更點差分。



 倍晨分,以五收,為更差;又五收,為籌差。《乾元》法同。《儀天》不立此法。



 步晷漏



 冬至後初夏至後次象:八十八日、小餘八千八百九十九半,約餘八千八百一十一分。



 夏至後初冬至後次象:九十三日、小餘七千四百八十五,約餘七千四百一十二分。



 前限:一百八十八十一日、小餘六千二百八十五,約餘六千二百二十太。



 辰法:八百四十一分三分之二。



 刻法:一百一分。



 辰:八刻三十三分三分之二。



 昏明:二百五十二分半。



 冬至後上限五十九日,下限一百二十三日、小餘六千二百八十五,約餘六千二百二十二太。



 中晷:一丈二尺七寸一分半。



 冬至後上差、夏至後下差:二千一百三十分。



 升法:一十五萬六千四百二十八分。



 冬至後下差、夏至後上差:四千八百一十二分。



 平法:一十七萬四千三分。



 夏至後上限同冬至後下限,夏至後下限同冬至後上限。



 中晷:一尺四寸七分、小分八十四。



 《
 儀天》求每日陽城晷景常數:置入冬、夏二至後求日數及分,以所入象日數下盈縮分盈減縮加之為其日定積,又以減其象小餘為夜半定積及分。又隔位除一,用若夜半定積及分在二至上限以下者,為入上限之數;以上者,以返減前限日及約餘,為入下限日及分。若冬至後上限、夏至後下限,以十四乘之,所得,以減上下限差分,為定差法;以所入上下限日數再乘之,所得,滿一百萬為尺,不滿為寸及分,以減冬至晷影,餘為其日中
 晷景常數也。若夏至後上限、冬至後下限,以三十五乘之,以上下差分為定法;以入上下限日數再乘之,退一等,滿一百萬為尺,不滿尺為寸及分,用加夏至晷景,即得其日中晷景常數。



 《儀天》求晷景每日損益差:以其日晷景與次日晷景相減,其日景長於次日晷影為損,短於次日晷景為益。



 《儀天》求陽城中晷景定數:置五千分,以其日晷景定數損益差乘之,所得,以萬約之為分,冬至後用減,夏至
 後用加;冬至一日有減無加,夏至一日有加無減。



 《儀天》求晷漏損益度入前後限數:置入冬至後來日數,在前限以下者為損;以上者,減去前限,餘為入後限日數者為益。若算立成,自冬至後一日,日加滿初象,即加象下約餘,為一象之數。



 《儀天》求每日晷漏損益數:置入前後限損益日數及分,如初象以下為在上限;以上者,返減前限,餘為下限,皆自相乘之,其分半以下乘,半以上收之;以一百通日,內
 其分,乃乘之;所得,在冬至後初象、夏至後次象,以升法除之。若冬至後次象、夏至後初象,以平法除之;皆為分,不滿,退除為小分;所得,置於上位,又別置五百五分於下,以上減下,以下乘上;用在升法者,以二千八百五十除之;用在平法者,以五千五百五十二除之;皆為分,不滿,退除為小分;所得,以加上位,為其日損益數。



 《儀天》求每日黃道去極度及赤道內外度分:若春分後置損益差,以五十乘之,以一千五十二除之為度,不滿,
 以一千四十二除之為分,以加六十七度三千八百四十五。若秋分後,置損益差,以五十乘之,以一千六十除之為度,不滿,以一千五十退除為分,以減一百一十五度二千二百二十二分,即得黃道去極度。置去極度分,與九十一度三千八百四十五相減,餘者為赤道內外度分。若黃道去極度分在九十一度三千八百四十五以下者為內,若在以上者為外度及分。



 《儀天》求每日晷漏母:各以其日損益差,自春分初日以
 後加一千七百六十八,自秋分初日以後減二千七百七十七,各得其日晷漏母,又曰晨分。



 《儀天》求每日昏分及距午分:置日元分,以其日晷漏母減之,餘者為昏分。又以其日晷漏母減五千五十分,餘者為其日距午分。



 月離九道交會《乾元》謂之交會,《儀天》謂之步交會。



 交總:七十一萬七千八百一、秒八十二。



 正交:三百六十三度、八千二百八十三、秒七。



 半交:一百八十一度、九千一百四十二、秒五十三半。



 少交:九十度、九千五百二十一、秒二十六太。



 平朔:一度、四千六百三十二。



 平望:空、七千三百一十六。



 朔差:二度、八千八百四十一。



 望差:二度、一千五百二十五。



 初準:一萬六千六百四十一。



 中準:一萬八千一百九十一。



 末準:一千五百五十。



 《乾元》交會



 交率:一萬六千、秒七千八百九十一。



 交策:二十七、餘六百二十三、秒九千四百五十五。



 朔準:二、九百三十六、秒五百四十五。



 望準:十四、二千二百五十。



 初限:三萬六千五百九十四。



 中限:四萬二。



 末限:三千四百八。



 《儀天》步交會



 交終分:二十七萬四千八百四十三、秒二千二百七十九。



 交終日:二十七、餘二千一百四十三、秒二千二百七十九。



 交中日:一十三、餘六千一百二十一、秒六千一百二十一。



 交朔日:二、餘三千二百一十五、秒七千七百二十一。



 交望日:一十四、餘七千七百二十九、秒五千。



 前限日:一十二、餘四千五百一十三、秒七千二百七十九。



 後限日:一、餘一千六百七、秒八千八百六十半。



 交差:四十五。



 交數:五百七十二。



 秒母:一萬。



 陰限:七千二百八十六。



 交日:空、小餘六千一百四十六、秒三百七十三。



 陽限:三千一百七十四。



 月食既限:二千五百八十二。



 月食分法:九百一十二半。



 中盈度:《乾元》謂之求平交朔日。《儀天》謂之求天正朔入交。



 以通餘減元積,七十五展之,以四百六十七除為分,滿交總去之,為總數;不盡,半而進位,倍總數,百收為分,用減之,餘以元法收為
 度,不滿為分,命曰中盈度及分。《乾元》置朔分,以交率去之,餘以五因之,滿元率收為日,即得平交朔日及分;次朔、望,以朔、望準加之,即得所求。《儀天》置天正朔積分,以交終分去之,滿宗法為日,即得所求。



 求次朔望中盈:《儀天》謂之求次朔入交。



 各置天正經朔中盈度分,視十一月望,十二月朔、望中日,如二十九日五千三百七以下者,即加朔、望差度分秒,餘月即加平朔、望度分秒,即得所求。《乾元》法見上。《儀天》置天正朔入交泛日餘秒,如交朔及交望餘秒皆滿交終日及餘秒即去之,各得朔、望入交泛日及餘秒。



 月離朔交初度分:《乾元》謂之求朔望交分。《儀天》謂之求入交常日。



 置其朔中盈度分,常與其朔常日度分合之,如正交以下者減半法,以上者倍而加之。



 加減訖為定,用減天正加時黃道宿度分,餘命起天正之宿初算,即得所求。《乾元》置平交朔、望日及分,以元率通之,以日躔陰陽差陽加陰減,為朔、望交分。《儀天》以其日入盈朔限升平定數,升加平減入交泛日,即為其朔、望入交常日也。《儀天》又有求朔、望入定交日,置其日入遲疾限升平定數,以交差乘之,如交數而一,升加平減入交常日,即為入定交日。



 月入陰陽歷:《乾元》謂之求朔望陰陽定分,《儀天》謂之求月行陰陽歷。



 以月離先後定數,先加後減朔、望中盈,用加朔、望常日月分,分即
 百除,度即百通。



 如中準以下者為月出黃道外;以上者去之,餘為月入黃道內。《乾元》以一百四十二乘陰陽差,一千八百二除,陽加陰減朔、望交分,為度定分;中限以上為陽,以下為陰。《儀天》視入交定日及餘秒,在交中日以下為陽,以上者去之,餘為月入陰歷。



 求食甚定餘:置朔定分,如半法以下者返減半法,餘為午前分;前以上者減去半法,餘為午後分;以乘三百,如半晝分而一,為差。午後加之,午前半而減之。



 加減定朔分,為食定餘。以差皆加午前、後分,為距中分。其望定分,便為食定餘。《乾元》以半晝刻約刻法為時差,乃視定朔小餘,在半法以下為用減半法為午前分;以上者去之,為午後分;以
 時差乘,五因之,如刻法而一,午前減,午後加,又皆加午前、後分,為距日分;刻法而一,為距午刻分。月只以定朔小餘為食定餘。《儀天》置月行去交黃赤道差,視月道差,如黃赤道交者,依其加減;不如黃赤道交者,返其加減;定朔、望小餘為食甚餘,亦返其加減去交定分。其日食,則又以其日晝刻,其三百五十四為時差,乃視食甚餘,如半法以下,返減半法,餘為初率;半法以上者,半法去之,餘為末率;滿一百一收之,為初率;以減末率,倍之,以加食甚餘,為食定餘;亦加減初、末率,為距午退分;置之,皆如求發斂加時術入之,即日、月食甚辰刻及分也。



 入食限:置黃道內、外分,如初準已上、末準已下為入食限。望入食限則月食,朔入食限則日食。月在黃道內則日食,在外則不食,望則無問內、外皆食。末準已下為交
 後分;初準以上者,返減中準,為交前分。《乾元》置陰陽定分,在初限以上、末限以下,為入食限,餘同《應天》。《儀天》置朔、望入交月行陰陽歷日及餘秒,如前限以上、後限以下者,為入食限。望入食限則月食,朔入食限、月入陰歷則日食。如後限以下為交後限,以上以減交中日及餘秒為交前限,各得所求。



 入盈縮歷:《乾元》、《儀天》不立此法。



 置朔定積,如一百八十二日、六千二百二十三以下為入盈日分;以上者去之,餘為入縮日分。



 黃道差:《乾元》謂之求晷差。《儀天》謂之求黃道食差。



 置其朔入歷盈、縮日及分,如四十五日以上、一百三十七日以下,皆以一千
 五百乘,為泛差;如四十五日以下,返減之,餘為初限日,一百三十七日以上者減去之,餘為末限日及分,以六十七乘,半之,用減泛差,以乘距午分,以元法收為黃道定分;入盈,以定分午前內減外加、午後內加外減;入縮,以定分午前內加外減,午後內減外加。《乾元》置入氣日,以距冬至之氣,以十五乘之,以所入氣日通之,以一百八十二日以下為入陽歷,以上者去之,為入陰歷。置入歷分,在四十五日以下,以三十七乘,五除,退一等,為泛差;在四十五日以上、一百三十七日以下,只用三十三、秒三十為泛差;一百三十七以上者去之,餘以三十七乘,五除,退一位,用減三十三、秒三十為泛差;皆以距午分乘為晷差。《儀天》
 二至後日益差至立春、立秋,得一百一十三、小分六十二半,立夏、立冬後每日損,以宗法乘之;冬至、立冬後三氣用四十四萬二千三百八十四,夏至、立夏後各三氣用二十七萬九千八百五十八除,為食差;以食甚距午正刻乘其日食差,為定差;冬至後,甚在午正東,陰減陽加;甚在午正西,陰加陽減;夏至後即返此;立冬初日後,每氣益差二十、秒四十四,至冬至初日加六十二、秒三十二;自後每氣損差二十、秒四十四,終於大寒,甚在午正西,即每刻累益其差,陰歷加,陽歷減。



 赤道差:《乾元》謂之求離差,《儀天》謂之求赤道食差。



 置入盈縮歷日及分,如九十一日以下,返減之,為初限日;以上者,用減一百八十二日半,餘為末限日及分;四因之,用減三百七十四,為
 泛差;以乘距中分,如半晝分而一,用減泛差,為赤道定分;盈初縮末內減外加、縮初盈末內加外減。《乾元》計春、秋二分後日加入氣日,以十五乘,在九十以下,以九十一乘,退為泛差;九十一以上去之,餘以九十一乘,退一等,以減八百一十九,為泛差;二分氣內置入氣日,以九十一乘,退為泛差;以半晝刻而一,以乘距午分,用加減泛差,為離差;食甚在出沒以前者,不用求離差,只用泛差,春分後陰加陽減,秋分後陰減陽加。《儀天》二分後益差至二至,積差皆二千八百二十六,自後累減至二分空,冬至後日損三十一、小分八十,夏至後日益三十、小分十五,又以宗法乘積差,各以盈縮初末限分除之,為日差;乃以末限累增、初限累損,各為每日食差;又以半晝刻數約其日食差,以乘食甚距午正刻,所得以減食差,餘為定數。餘同《乾元》。



 日食差:依黃、赤二差,同名相從,異名相消,為食差。二歷法同。



 距交分:《乾元》謂之去交分。《儀天》謂之去交定分。



 置交前後分,以黃、赤二差加減之,為距交分。如月在內道不足減者,返減入外道,不食;如月在外道不足減,返減食差,為返減入內道即有食。《乾元》置陰陽歷去交前後分,以食差合加減者,依其加減,所得為去交前後定分。月在陰歷,去交前後分不足減者,即返減食差,交前減之,餘者為得陽歷交後得減之,餘者為陽歷交前定分,並不入食限。月在陽歷,去交前後分不足減者,亦返減食差,交前減之,餘者為陰歷交後定分,交後減之,餘者為陰歷交前定分,並入食限。《儀天》應食差,同名相從,異名相消,餘同《乾元》法。



 日食分:置距交分,如四百二十以下者類同陽歷分;以上者去之,為陰歷分;又以食定餘減四分之三,午前倍之,午後半之。



 皆退一等,用減陰陽歷分,為食定分;如不足減,即返減之,餘進一位,加陰歷分,為食定分;陽以四十二除,為食之大分;陰九百六十以下返減之,如九十六而一,為食之大分,命十為限。《乾元》置交前後分,以食差加減之,為定交分;在九百二十以下為陽,以上去之為陰。在陽以九十四、在陰以二百一十三除為大分,餘同《應天》。《儀天》置入限去交定分,減七百二十八,陽限以上為陰歷食,以陽限去之,餘減陰限為陰歷食分,以下者為陽歷食分,亦減三百一十七,如限除之,
 皆進一位,各命十為限,餘同《應天》。


月食分:置黃道內外前後分,如食限三百四十以下者,食既;以上者,返減末準,餘以一百二十一除,為月食之大分。其食五分以下,在子正前後八□內,以□百四十二除為食之大分,命十為限。)其前後分,以九百以上入或食或不食之限。
 \gezhu{
  《乾元》交定分在七百五十二以下,食既;以上,返減末限,以二百六十四除之為大分。《儀天》陽減陰加前後定分九百一十二半,在既限以下、食既以上,以去交分減之,以月食法除之為大分。}



 日月食虧初復末:《乾元》謂之求定用刻,《儀天》謂之求日月泛用分、求虧初復末。



 百通
 日月食之大小分,以一千三百三十七乘之,各如其日離分,為定用分;加食定餘,為復末定分;減之,為虧初定分。其月食,以食限減定用分,用減食甚,為虧初定分;如不足減者,即以食限分如望定餘為食定分,餘卻依日食加減,各得月食虧初、復末定分也。《儀天》月以五百八十八,日以五百二十九、秒二十乘所食分,退一等,半之,為定用刻。《儀天》日以五百四十五、秒四十,月以六百六,皆乘所食分,其小分以本母除,從之,為泛用分;其食又視去交定分在一千七百二十六以下增半刻,八百五十六以下又增半刻,以一千三百五十乘,以辰定分除,為定用刻;皆減定朔、望小餘為虧初,加之為復末。



 日食起虧:《乾元》謂之求日食初起。



 視距交分如四百二十以上者,初起西北,甚於正北,復於東北;如以下者,初起西南,甚於正南,復於東南。凡食八分以上者,皆初起正西,復於正東。《儀天》、《乾元》日在陰歷,初起西北;在陽歷,初起西南,餘並同《應天》。



 月食起虧:《乾元》謂之月食初定,《儀天》謂之月食初起。



 月在內道,初起東南,甚於正南,復於西南;月在外道,初起東北,甚於正北,復於西北。凡食八分以上者,初起正東,復於正西。《乾元》《儀天》以內道為陰歷,外道為陽歷,餘皆同《應天》。而《儀天》又法云,此法據古經所載,以究天體,食在午中前後一辰之內,其餘方
 若要的驗,當視日月食時所在方位高下,審祥黃道斜正、月行所向,起虧、復滿皆可知也。



 帶食出入:《儀天》謂之求帶食出入見食分數。



 視其日出入分,如在虧初定分以上、復末定分以下,即帶食出入。食甚在出入分以下,以出入分減復末定分,為帶食差;食甚在出入分以上者,以虧初定分減出入分,為帶食差;以乘食定分,滿定用而一,日陽以四十二、陰以九十六、月一百二十一除之,為帶食之大分,餘為小分。《乾元》各以食甚餘與其日晨昏分相減,餘為帶食差;其帶食差在定用刻以下者,即帶食出入;以上者,即不帶食出入也。以帶食差乘所食之分,滿定用
 刻而一,所得以減所食之分,即帶食出入所見之分也。其朔日食甚在晝者,晨為已食之分,昏為所殘之分;若食甚在夜,昏為已食之分,晨為所殘之分。其月食,見此可以知之也。《儀天》以食甚餘減晨昏分,餘為出入前分,不足者,返減食甚,餘為出入後分,以乘所食之分,其食分以本母通之,從其小分,滿定用分除之,所得以本母約之,不滿者,半以上為半強,半以下為半弱,即得帶食出入之分數也。其日、月食甚在出入前者,為所殘之分,在出入後者,為已退之分。



 更點:《乾元》、《儀天》謂之月食入定點。



 各置虧初、食甚、復末定分,如晨分以下者加晨分,昏分以上者減去昏分,皆以更分除為更數,不盡,以點分除之為點數。命初更,算外,即得所求。《
 乾元》法同。《儀天》倍其日晨分,以五除之為更分,又以五除之為點分。乃視所求小餘,如晨分以下加晨分,昏分以上減去昏分,求更點並同《應天》。



 日月食宿分:《乾元》謂之日月食宿。



 以天正冬至黃道日度加朔望常日月度,命起鬥初,算外,即日月食在宿分也。《乾元》以距日沒辰至食甚辰之數,約其日離差,用加昏度。《儀天》用加時定月度也。



\end{pinyinscope}