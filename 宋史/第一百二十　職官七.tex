\article{第一百二十 職官七}

\begin{pinyinscope}

 大都督府制置使宣諭使宣撫使總領留守經略安撫使發運使都轉運使招討使招撫使撫諭使鎮撫使提點刑獄提舉常平茶馬市舶等職提舉學事提點開封府界公事提舉河北糴便司經制邊防財用提舉解鹽保甲三白渠弓箭
 手等職府州軍監諸軍通判幕職諸曹等官諸縣令丞簿尉鎮砦官廟令丞簿總管鈐轄路分都監諸軍都統制巡檢司監當官



 大都督府都督府長史左右司馬錄事參軍司戶、司法、司士、司理、文學參軍助教大都督及長史掌同牧、尹,親王為節度則大都督領之;庶姓為節度則長史領之。端拱初,越王為威武軍節度、福州大都督府長史。淳化五年,吳王為淮南節度、揚州大都督府長史,翰林學士張洎草制,再
 表援引典故,宰相言:「越工已為長史。」上曰:「業已差誤,異日有除,並
 改正之。」至道後,因移鎮,遂為大都督。闕則置知府事一人,同次府。通判一人,京朝官充。司馬不厘務。舊制,凡都督州建官如上。南渡後,以見任宰相充都督,次有同都督,有督視軍馬,多執政為之,雖名稱略同,然掌總諸路軍
 馬,督護諸將,非舊制比也。



 初,紹興
 二年,呂頤浩首以左僕射出
 都督
 江、淮、兩浙、荊湖諸軍事,置司鎮江。其後,趙鼎、張浚、湯思退皆以宰相兼之。熙浩還朝,孟庾始以參知政事為權同都督代,後「權」字。趙鼎先以知樞密院事為都督川陜、荊襄諸軍事,其後與浚並相,並帶兼都督諸路軍馬入銜,未幾,浚獨被旨江上視師,
 置都督行府,地移文安,並依三省體式,其召赴行在,以其事分隸三省、樞密院。思退初以左相出都督,時楊存中即以太傳、寧遠軍節度使同都督,思退不行,就以楊存中充都督,非宰執而為都督自存中始。



 三十一年,葉義問以知樞密院事督視江、淮、荊襄軍馬,明年,汪澈以參知政事、湖北、京西路督視軍馬,執政為督視於是見焉。王之望辭同都督,有曰:「朝廷於兩淮,前以二大將為招撫使,後以二從臣為宣諭使,憂其不相統攝,則以
 宰相為都督,欲事權歸一也,此可以見朝廷開府之意。」凡簽廳文字,並依尚書左右司、樞密院檢詳房體式。設屬:諮議軍事、參謀、參議,並以從官充;書寫機宜文字、乾辦官、準備差遣,前後員數不一。開禧用兵,或以簽樞督視,或以元樞代之,或以參知政事督視四川軍馬,然皆未有底績而罷。



 制置使不常置,掌經晝邊鄙軍旅之事。政和中,熙、秦用兵,以內侍童貫為之。仍兼經略使。靖康初,會諸路兵解
 太原之圍,姚古、解潛相繼為河東、河北制置使,皆無功而罷。中興以後置使,掌本路諸州軍馬屯防捍禦,多以安撫大使兼之,亦以統兵馬官充;地重秩高者加制置大使,位宣撫副使上,紹興元年,趙鼎始為江西置制大使,其後席益帥潭,李綱帥江西,呂頤浩帥湖南,皆領制置大使。開禧,丘崇、何澹亦然。



 或置副使以貳之。



 呂頤浩充江、浙置使,陳彥文、程千秋副使。胡舜除沿江都制置使,工義叔副使。趙鼎為江西制置大使,岳飛為制置使,每事會議,或急速則施行,許報大使照應。



 初,建炎元年,詔令安撫使、發運、監司、州軍官,並聽制置司節制,其後,議者以守臣既帶安撫,又
 兼制置,及許便宜,權之要重,擬於朝遷,於時詔止許便宜制置軍事,其它刑獄、財賦付提刑、轉運,後又詔諸路帥臣並罷制置使之名。惟統兵官如故。隆興以後,或置或省。開禧間,江、淮、四川並置大使,休兵後,獨成都守臣帶四安撫、制置使,掌節制御前軍馬、官員升改放散、類省試舉人、銓量郡守、舉闢邊州守貳,其權略視宣撫司,惟財計、茶馬不預。又有沿海制置使,以明州守臣領之』然其職止肅清海道、節制水軍,非四川比。大使置屬
 參謀、參議、主管機宜、書寫文字各一員。乾辦公事三員。準備將領、差遣、差使各五員,餘隨時勢輕重而增損焉。



 宣諭使掌宣諭德意,不預他事,歸即結罷。紹興元年,詔秘書少監傳崧卿充淮南東路宣諭使,此其始也。二年,分遣御史五人,宣諭東南諸路,戒其興獄,責其不當,督捕盜賊,皆欲專一布惠以為民。其後,右司範直方宣諭川、陜,察院方庭實宣諭三京,均此意。及新復陜西樓照以簽書樞密院事往永興宣諭,就令招撫盜賊,鄭剛中
 為川、陜宣諭使,許按察官吏,汪澈為湖北。京西宣諭使,仍節制兩路軍馬,自是使權益重,而使事始不專。三十二年,虞允文、王之望相繼充川、陜宣諭使,皆預軍政,共權任殆亞於宣撫。其後,錢端禮、吳芾皆以侍從出膺斯寄,事畢結局。官屬軍兵,視其所任事之輕重,為賞之厚薄焉。開禧間,薛叔似、鄧友龍、吳獵皆因饑荒盜賊及平逆亂後,往敷德意,亦並以從官行。



 宣撫使不常置,掌宣布威靈、撫綏邊境及統護將帥、督
 視軍旅之事,以二府大臣充。治平末,命同簽書樞密院郭逵宣撫陜西。三年,夏兵犯順,以參知政事韓絳為陜西宣撫使,繼即軍中拜相,仍舊領使。政和中,遣內侍童貫為陜西、河東宣撫使,又兼河北。宣和三年,睦寇方臘作亂,移貫宣撫淮、浙,賊平依舊。靖康初,種師道提兵入衛京城,為京畿、河東北宣撫使,凡勤王之師屬焉。及會諸道兵救太原,又以知樞密院李綱宣撫河東、北兩路。中興初,張浚以知樞密院事、孟庚以參知政事、李綱以
 前宰相,皆出宣撫,浚又加「處置」二字入銜。



 時為川、陜、京西、湖北路。



 紹興元年,詔以淮南守臣多闕,百姓未能復業,分命呂頤浩、朱勝非、劉光世皆以安撫大使兼宣撫使。武臣非執政而為宣撫使,實自光世始。二年,李光又以吏部尚書加端明殿學士,為壽春等州宣撫使。自是韓世忠、張俊、吳玠、岳飛、吳璘皆以武臣充使,王似亦以從官由副使而升正使焉。三十二年,張浚復以少傳依前觀文殿大學士充江淮東、西路宣撫使。乾道三年,虞以文依舊
 知樞密院事充四川宣撫使。五年,王炎除四川宣撫使,依舊參知政事。開禧間,以從官出宣撫江、淮、湖北、京西等處不一。其屬有參謀官,系知州資序人,與提刑敘官;參議官,系知州資序人,與轉運判官敘官;機宜乾辦公事。並依發運司主管文字敘官。凡宰執帶三省、樞密院事出使,行移文字扎六部,六部行移即具申狀。如從官任使、副,合申六部,六部行移即用公牒。



 宣撫副使不常置,掌貳使事。宣和末,王師伐燕、命少保
 蔡攸充。靖康初,會兵救太原,又次資政殿學士劉韐為之。建炎三年,周望宣撫兩浙,以太尉郭仲荀副之。其後,福建韓世忠、川陜吳玠



 皆有此授。紹興間,張浚宣撫川、陜,將召歸,命從臣王似、盧法原為之副;王似除使,盧法原仍副之。亦有不置使而置副,如胡世將之於川、陜,岳飛之於荊、襄,楊沂中之於淮北,皆止以副使為名。飛後以功始落「副」字。亦有身為正使兼領副使,如開禧三年,安丙充利州西路宣撫使兼四川宣撫副使。



 宣撫判官不常置,掌贊使務。熙寧中,命直舍人呂大防為之。實上幕也。紹興中,張浚初以便宜命劉子羽為副,其後張宗元、呂祉亦為之。十年,楊沂中以太尉為淮北宣撫副使,劉琦以節度使為判官,禮抗權均,猶轉運使、副、判官之比。詔行移文字同其擊銜,宣判之名同,而先後輕重異焉。



 總領四人。掌措置移運應辦諸軍錢糧,以朝臣充,仍帶干階、戶部等官。朝遷科撥州軍上供錢米,則以時拘催,
 歲較諸州所納之盈虧,以聞於上而賞罰之。初,建炎間,張浚出使川、陜,用趙開總領四川財賦,置所擊銜,總領名官自此始。其後大軍在江上,間遣版曹或太府、司農卿少卿調其錢糧,皆以總領為名。



 紹興十一年,收諸帥之兵改為御前軍,分屯諸處,乃置三總領,以朝臣為之,仍帶專一報發御前軍馬文字。蓋又使之預聞軍政,不獨職餉饋而已。其序位在轉運副使之上,鎮江諸軍錢糧,淮東總領掌之;鄂州、荊南、江州諸軍錢糧,湖廣總領
 掌之;建康、池州諸軍錢糧,淮西總領掌之。十五年,復置四川總領,凡興元、興州、金州諸軍錢糧,四川總領掌之。其言屬有乾辦公事、準備差遺。



 四川又有主管文字二員。



 淮東西有分差糧料院、審計司、審計以通判權。榷貨務、都茶場、御前封樁甲仗庫、大軍倉、大軍庫、贍罕酒庫、市易抵當庫、惠民藥局。湖廣有給納場、屬官兼。



 分差糧料院、審計院、通判兼。



 御前封樁甲仗庫、大軍倉庫、贍軍酒庫。四川有分差糧料院、審計院、屬官兼。



 大軍倉庫、撥發船運官、贖藥庫、糴買場。



 淳
 熙元年,詔委諸路州軍通判,專一主管拘催逐州錢米,起發赴所,本所每半年比較,以行賞罰。紹熙二年,以淮西總領所言,定知州、通判展減磨勘法:十分欠二展二年,數足減二年。吏額:淮東九人,淮西、湖廣十人,四川二十人。



 留守副留守舊制,天子巡守、親征,則命親王或大臣總留守事。建隆元年,親征澤、潞,以樞密使吳廷祚為東京留守,其西、南、北京留守各一人,以知府兼之。西京河南,
 南京應天,北京大名。



 留守管掌宮鑰及京城守衛、修葺、彈壓之事,畿內錢穀、兵民之政皆屬焉。政和三年,資政殿大學士鄧洵武言:「河南、應天、大名府號陪京,乞依開封制,正尹、少之名。」從之。宣和三年,詔河南、大名少尹依熙守舊制,分左右廳治事;應天少尹一員。及三京司錄,通管府事。南渡初,其東京、北京並置留守,以開封、大名知府兼,又以掌兵官為副留守。其後,河南復,南京、西京置留守。紹興四年,帝將親征,以參知政事孟庚為行宮留守,奏差
 主管書寫機宜文字官一員。乾辦官二員。淮血差遣、差使各三員,使臣五十員,又置留司臺官一員。五年,罷局。其後,秦檜為行宮留守,援例置官。



 經略安撫司經略安撫使一人,以直秘閣以上充,掌一路兵民之事。皆帥其屬而聽其獄訟,頒其禁令,定其賞罰,稽其錢穀、甲械出納之名籍而行以法。若事難專決,則具可否具奏。即乾機速、邊防及士卒抵罪者,聽以便宜裁斷。帥臣任河東、陜西、嶺南路,職在綏御戎夷,則為
 經略安撫使兼都總管以統制軍旅,有屬官典領要密文書,奏達機事。河北及近地,則使事止於安撫而已,其屬有乾當公事、主管機宜文字、淮備將領、淮備差使。



 元祐元年,詔陜西河東經略安撫、都總管司,自元豐四年後,應緣軍興添置官屬並罷。又詔罷經略安撫司乾當官。二年,詔沿邊臣僚奏請事,並先赴經略司詳度以聞。元符元年,詔經略司遇軍興差發軍馬,具數關報走馬承受。崇寧二年,熙河蘭會經略王厚奏:「溪哥城乃古積
 石軍,今當為州,乞以李忠為守,置河南安撫司。」從之。四年,置河東、陜西諸路招納司,並隸經略司。五年,詔河東同管幹沿邊安撫司公事,許歲赴闕奏事一次。政和四年,詔移京西路安撫于河南府,京東路安撫於應天府。宣和二年,詔瀘州守臣帶潼川府、夔州路兵馬都鈐轄、瀘南沿邊路兵馬都鈐轄、瀘南沿邊安撫合。又詔罷置輔郡內穎冒府帶京西路安撫使。三年,詔杭、越州、江寧府、洪州守臣並帶安撫使。六年,詔瀘州止帶主管瀘南
 沿邊安撫司公事。仍差守臣。七年,詔河陽、開德守臣並帶管內安撫使。



 舊制,安撫總一路兵政,以知州兼充,太中大夫以上,或曾歷侍從乃得之,品卑者止稱主管某路安撫司公事。中興以後,職名稍高者出守,皆可兼使,如系二品以上,即稱安撫大使。廣東、西、荊南、襄陽仍舊制加「經略」二字。凡帥府皆帶馬步軍都總管。建炎初,李綱請於沿河、沿淮、沿江置帥府。以文臣為安撫使帶馬步軍都總管,武臣一員為之副,許便宜行事,闢置僚屬、
 將佐,措置調發惟轉輸屬之漕使。其後,沿江三大使司闢置過多,邊報稍寧,詔加裁定。參謀、參議官、主管機宜文字、主管書寫機宜文字各一員。乾辦公事二員。文臣淮備差遣、武臣準備差使、準備將領各以五員為額,其餘諸路或隨地輕重而損益焉。餘從省罷。後以諸路申請,或置或省不一。



 淳熙二年,詔揚州、廬州、荊南、襄陽、金州、興元、興州分為七路,每路委文臣一員充安撫使以治民,武臣一人充都總管以治兵。其逐路都總管職事,
 且令帥臣依舊帶行,候正官到日交割。慶元二年,詔利州西路安撫司於興州置司,令都統制兼。五年,臣僚言:「遴選帥才,除嘗任執政外,兩制從官必曾經作郡、庶官必曾任憲漕實有治績者。」從之。惟廣南東、西兩路則帶經略、安撫使。紹興五年。令襄陽守臣、湖北帥司各帶經略、安撫使,後罷,惟二廣如故。



 走馬承受諸路各一員。隸經略安撫總管司,無事歲一入奏,有邊警則不時馳驛上聞。然居是職者惡有所隸,
 乃潛去「總管司」字,冀以擅權。熙寧五年,帝命正其名,鑄銅記給之。仍收還所用奉使印。崇寧中,始詔不隸帥司而輒預邊事,則論以違制。大觀中,詔許風聞言事。政和五年詔:「諸路走馬承受體均使華,邇來皆貪賄賂,類不舉職,是豈設官之意?其各自勵,以稱任使。或蹈前失,罰不汝赦。」明年七月,改為兼訪使者。宣和五年詔:「近者諸路廉訪官,循習違越,附下罔上,凡邊機皆先申後奏,且侵監司、凌州縣而預軍旅、刑獄之事,復強買民物,不償
 其直,招權怙勢,至與監司表裹為惡。自今猶爾,必加貶竄。」靖康初,罷之。依祖宗舊制,復為走馬承受。



 發運使副判官掌經度山澤財貨之源,漕淮、浙、江、湖六路儲廩以輸中都,而兼制茶鹽、泉寶之政,及專舉刺官吏之事。熙寧初,輔臣陳升之、王安石領制置三司條例,建言:「發運使實總六路之出入,宜假以錢貨,繼其用之不給,使周知六路之有無而移用之。凡上供之物,皆得徙貴就賤,用近易遠,令預知在京倉庫之數所當辦者,
 得以便宜蓄買以待上令,稍收輕重斂散之權歸於公上,則國用可足,民財不匱矣。」從之。既又詔六路轉運使弗協力者宜改擇,且許發運使薛向自闢其屬。又令舉真、楚、泗守臣及兼提舉九路坑冶、市舶之事。元祐中,詔發運使兼制置茶事。至崇寧三年,始別差官提舉茶鹽。



 政和二年,罷轉般倉,六路上供米徑從本路直達中都,以發運司所拘綱船均給六路。宣和初,詔:「發運司視六路豐歉和糴上供,乃祖宗舊制,曩緣奸吏侵用糴本,遂
 壞良法。自今每歲加糴一百萬石,同年額輸京。」三年,方臘初平,江、浙諸郡皆未有常賦,乃詔陳亨伯以大漕之職經制七路財賦,許得移用,監司聽其按察。於是亨伯收民間印契及鬻糟醋之類為錢凡七色,是後州縣有所謂經制錢,自亨伯始。



 六年,詔復轉般倉,命發運判官盧宗原措置,尋以靖康之難,迄不能復。渡江後,惟領給降糴本,收糴米斛,廣行儲積,以備國用。紹興二年,用臣僚言省罷。以其職事分委漕臣。八年,戶部復言廣糴儲
 積之便,再置經制發運使,並理經制司財賦,故名。



 以微猷閣待制程邁充使,專掌糴事。邁上疏,以租庸、常平、鹽鐵、鼓鑄各分於諸司而總於戶部,發運使無所用之。固辭不行。九年,遂廢發運司,以戶部侍郎梁汝嘉為經制使,檢察中外失陷錢物,與催未到綱運、措置糴買、總領常平為職。未幾,復以臣僚言,分其責於逐路監司。乾道六年復置,以戶部侍郎史正志為兩浙、京、湖、淮、廣、福建等路都大發運使。是冬,以奏課誕謾貶。並廢其職。



 都轉運使轉運使副使判官掌經度一路財賦,而察其登耗有無,以足上供及郡縣之費。歲行所部,檢察儲積,稽考帳籍,凡吏蠹民瘼,悉條以上達,及專舉刺官吏之事。熙寧初,詔河東、河北、陜西三路漕臣許乘傳赴闕,留毋過浹日。既又詔三路漕臣,令自闢屬各二員,以京朝官曾歷知縣者為之。二年,詔川、陜、閩、廣七路除堂選守臣外,委轉運司依四選例立格就注,免赴選,具為令。元豐初,詔河北、淮南、京東、京西及陜右雖各析
 為兩路,許依未析時通治兩路之事,錢穀聽其移用。元祐初,司馬光請漕臣除三路外,餘路毋得過二員。其屬官溢員亦省之。紹聖中,詔淮、浙、江、湖六路上供米,計其近遠分三限,自季冬至明年八月,以次輸足。大觀中,陜西漕臣以四員為額。政和中,又詔陜西以三員。熙、秦兩路各二員。宣和初,又詔陜西以都漕兩員總治於長安,而漕臣三員分領六路。



 中興後,置官掌一路財賦之入,按歲額錢物觔斗之多寡,而察其稽違,督其欠負,以供
 於上。間詣所部,則財用之豐欠,民情之休戚,官吏之勤惰,皆訪問而奏陳之。有軍旅之事,則供饋錢糧,或令本官隨軍移運。或別置隨軍轉運使一員。或諸路事體當合一,則置都轉運使以總之。



 江東、西路分置三帥,置都轉運使一員,張公濟為江、浙、荊湖、廣南、福建都運。趙開為四川都運。



 隨軍及都運廢置不常,而正使不廢。若副使,若判官,皆隨資之淺深稱焉,其屬有主管文字、乾辦官各一員,文臣淮備差遣、武臣淮備差使,員多寡不一。



 招討使掌收招討殺盜賊之事,不常置。建炎四年,以檢校少保、定江昭慶軍節度使張俊充江南路招討使,定位在宣撫使之下、制置使之上,著為定制。軍中急速事宜,待報不及,許以便宜行事。差隨軍轉運使一員、參議官一員、乾辦官三員、隨軍幹辦官四員、書寫機宜文字一員,並聽奏闢。紹興五年,岳飛為湖北、襄陽招討使,請州縣不法害民者,許一面對移,或放罷以聞。從之。十年,金人犯三京,以韓世忠、岳飛、張俊並兼河南、北招討使
 以御之。三十一年,陜西、河東北、京東西等路皆置招討使,蓋又特遙領其地而已。



 招撫使不常置。建炎初,李綱秉政,以張所為河北招撫使,未及出師而廢。紹興十年,劉光世為三京招撫使,逾年而罷。三十二年,孝宗即位,以成閔、張子蓋、李顯忠三大將為湖北、京西、淮東西招撫使。子蓋死,劉寶代之。未幾結局,官吏並罷。開禧二年,山東及京東西北路並置使招撫,後皆罷之。



 撫諭使掌慰安存問,採民之利病,條奏而罷行之。亦不常置。建炎元年,帝謂輔臣曰:「京城士庶,自金人退師,人情未安,可差官撫諭。」於是以路允迪、耿延禧為京城撫諭使此置使初意也。是年八月,又令學士院降詔,且命江端友等奉詔撫諭諸路。其後,李正民以中書舍人為江、浙、湖南撫諭使,且令按察官吏,伸民冤抑。傅崧卿以吏部待郎為淮東撫諭使,採訪民間利病,及措置營田等事。或不以使名,則稱撫諭官,所至以某州撫諭司為
 名。具宣恩言,俾民知德意,初無二致。乾道元年,在合門事龍大淵差充兩淮撫諭軍馬,回日結局。是又特為軍馬出雲。



 鎮撫使舊所無有,中興,假權宜以收群盜。初,建炎四年,範宗尹為參知政事,議群盜並力以拒官軍,莫若析地以處之,盜有所歸,則可漸制,乃請稍復藩鎮之制。是年五月,宗尹為右僕射,於是請以淮南、京東西、湖南北諸路並分為鎮,除鹽茶之利仍歸朝廷置官提舉外,他監
 司並罷。上供財賦權免三年,余聽帥臣移用,更不從朝廷應副,軍興聽從便宜。時劇盜李成在舒、蘄,桑仲在襄、鄧,郭仲威在揚州,薛慶在高郵,皆即以為鎮撫使。其餘或以處歸朝之人,分畫不一,許以能捍禦外寇,顯立大功,特與世襲。官屬有參議官、書寫機宜文字各一員。乾辦公事二員,並聽奏闢。久之,諸鎮或戰死,或北降,但餘荊南解潛。及趙鼎為相,召潛主管馬軍,遂罷弗置焉。



 提點刑獄公事掌察所部之獄訟而平其曲直,所至審
 問囚徒,詳核案牘,凡禁系淹延而不決,盜竊逋竄而不獲,皆劾以聞,及舉刺官吏之事。舊制,參用武臣。熙寧初,神宗以武臣不足以察所部人材,罷之。六年,置諸路提刑司檢法官。紹聖初,以提刑兼坑冶事。宣和初,詔江西、廣東增置武提刑一員,然遇關帥不許武憲兼攝。中興,以盜賊未衰,諸路無武臣提刑處,權添置一員,建炎四年罷。紹興初,兩浙路以疆封闊遠,差提刑二員,淮南東路罷提刑,令提舉茶鹽官兼領,蓋因事之煩簡而損益
 焉。乾道六年,詔諸路分置武臣提刑一員。須選差公廉曉習法令、民事之人,如無聽闕,其後稍橫,遂不復除。八年,用臣僚言,諸路經總制錢並委提點刑獄官督責。嘉定十五年,臣僚言:「廣西所部州軍最多,提刑合照元降指揮,分上下半年,就鬱林州與靜江府兩處置司,無使僻地貧民有冤莫吐。」從之。其屬有檢法官、乾辦官。



 提舉常平司掌常平、義倉、免役、市易、坊場、河渡、水利之法,視歲之豐歉而為之斂散,以惠農民。凡役錢,產有厚
 薄則輸有多寡;及給吏祿,亦視其執役之重輕難易以為之等。商有滯貨,則官為斂之,復售於民,以平物價。皆總其政令,仍專舉刺官吏之事。熙寧初,先遣官提舉河北、陜西路常平。未幾,諸路悉置提舉官。元祐初罷之,並其職於提點刑獄司。紹聖初復置,元符以後因之。



 提舉茶鹽司掌摘山煮海之利,以佐國用。皆有鈔法,視其歲額之登損。以詔賞罰。凡給之不如期,鬻之不如式,與州縣之不加恤者,皆劾以聞。政和改元,詔江、淮、荊、浙
 六路共置一員。既而諸路皆置。中興後,通置提舉常平茶鹽司,掌常平、義倉、免役之政令。凡官田產及坊場、河渡之入,按額拘納;收糴儲積,時其斂散以便民;視產高下以平其役。建炎元年,常平職事並歸提刑司,錢歸行在。二年,始復置常平官,還其糴本,未幾復罷。紹興二年,復置主管。



 系提刑司,委通判或幕職官充。



 其後,置經制司,改常平官為經制某路乾辦常平等公事。未幾,經制司罷,復為常平官。十五年,戶部恃郎王鈇言「常平之設,科條實繁,其利
 不一,豈一主管官能勝其任?」乃詔諸路提舉茶鹽官改充提舉常平茶鹽公事。如四川無茶鹽去處,仍以提刑兼充,主管官改充常平司干辦公事。是年冬,詔提舉官依舊法為監司,與轉運判官敘官,歲舉升改,官員有不職,則按以聞。其後,常平錢多取以贍軍,所掌掌特義倉、水利、役法、振濟之事。茶鹽司置官提舉,本以給賣鈔引,通商阜財,時詣所部州縣巡歷覺察,禁止私販,按劾不法。其屬有乾辦官。既與常平合一,遂並行兩司之事焉。



 都大提舉茶馬司掌榷茶之利,以佐邦用。凡市馬於四夷,率以茶易之。慶產茶及市馬之處,官屬許自闢置,視其數之登耗,以詔賞罰。舊制,於原、渭、德順三郡市馬。熙寧七年,初復熙、河,經略使王韶言:「西人頗以善馬至邊,其所嗜唯茶,而乏茶與之為市,請趣買茶司買之。」乃命三司乾當公事李



 □巳運蜀茶至熙、河,置買馬場六而原、渭、德順更不買馬,於是杞言:「買茶買馬,一事也,乞同提舉買馬。」杞遂兼馬政,然分合不常。至元豐六年,群牧判官提舉買馬郭茂恂又言:「茶司既不兼買馬,遂立法以害馬政,恐誤國事,乞並茶場買馬為一司。」從之。先是,市馬於邊,有司幸賞,率以駑充數。紹聖中,都大茶馬程之邵始精揀汰,仍以八月至四月為限,又
 以羨茶轉入熙、秦市戰騎,故馬多而茶息厚,二法著為令。元符末,程之邵召對,徽宗詢以馬政,之邵言:「戎俗食肉飲酪,故貴茶,而病於難得,原禁沿邊鬻茶,專以蜀產易上乘。」詔可。未幾,獲馬萬匹。宣和中,以茶馬兩司吏員猥眾,於是朝奉大夫何淅請遵豐、熙成憲,稱其事之繁簡而定以員數,從之。紹興四年,初命四川宣撫司支茶博馬。七年,復置茶馬官,凡買馬州縣黎、文、敘、長寧、南平、珍皆與知州、通判同措置任責。通判許茶馬司闢置,視買馬額數之盈虧而賞罰之。歲發馬綱應副屯駐諸軍及三衙之用。
 舊有主管茶馬、同提舉茶馬、都大提舉茶馬,皆考其資歷授之。乾道初,用臣僚言省罷,委各郡知州、通判、監押任責,尋復置。紹熙三年,茶馬司拖欠馬數過多,詔將本年分馬綱錢價,責茶馬司撥付湖廣總領所,勞付軍官自買土馬。嘉泰三年,以所發綱馬不及格式,詔茶馬官各差一員,遂分為兩司。



 文臣成都主茶,武臣興元主馬。



 其屬共有乾辦公事四員、準備差使二員。



 提舉坑冶司掌收山澤之所產及鑄泉化,以給邦國之
 用,歲有定數,視其登耗而賞罰之。舊制一員。元豐初,以其通領九路,始不能周歷所部,始增為二員。分置兩司:在饒者領江東、淮、浙、福建等路,在虔者領江西、湖、廣等路。至元祐,復並為一員。紹興五年,以責任不專,職任廢弛,詔將饒州司官吏除留屬官一員外,並減罷。並歸虔州司,又加「都大」二字於「提點」之上。或病其事權太重,省並歸逐路轉運司措置,仍置提領諸路鑄錢官一員於行在,以侍從官充,自此或復或罷不一。乾道六年,並歸
 發運司。發運司罷,復置提兩司如初。淳熙二年,並贛歸饒,復加「都大」二字,與提刑序官。其屬有乾辦公事二員,檢踏官六員。稱銅官、催綱官各一員。



 提舉市舶司掌蕃貨海舶征榷貿易之事,以來遠人,通遠物。元祐初,詔福建路於泉州置司。大觀元年,復置浙、廣、福建三路市舶提舉官。明年,御史中丞石公弼請以諸路提舉市舶歸之轉運司,不報。建炎初,罷閩、浙市舶司歸轉運司,未幾復置。紹興二十九年,臣僚言:「福建、廣
 南各置務於一州,兩浙市舶乃分建於五所。」乾道初,臣僚又言兩浙提舉市舶一司抽解搔擾之弊,用言福建、廣南皆有市舶,物貨浩瀚,置官提舉實宜,惟兩浙冗蠹可罷。從之。仍委逐處知州、能判、知縣、監官同檢視,而轉運司總之。



 提舉學事司掌一路州縣學政,歲巡所部以察師儒之優劣、生員之勤惰,而專舉刺之事。崇寧二年置,宣和三年罷。



 提點開封府界諸縣鎮公事掌察畿內縣鎮刑訇、盜賊、場務、河渠之事。



 提舉河北糴便司糴便芻糧以供邊儲之用。



 提舉制置解鹽司掌鹽澤之禁令,使民入粟塞下,予鈔給鹽,以足民用而實邊備。凡鹽價高下及文鈔出納多寡之數,皆掌之。



 經制邊防財用司掌經晝錢帛、芻糧以供邊費,凡榷易貨物、根括耕地及邊部弓箭手等事,皆奏而行之。熙寧
 末,以熙、河連歲用兵,仰給支度,費用不貲,始置是司。元祐初,罷。崇寧中,復置提舉兵馬、提轄兵甲,皆守臣兼之。掌按練軍旅,督捕盜賊,以清境內;凡諸營之名籍,較其壯怯而賞罰之。



 提舉保甲司掌什伍其民,教之武藝,視其優劣而進退之。元豐初,置於開封府界,遂下其法河北、河東、陜西三路,既而悉置提舉官,如府界焉。



 提舉三白渠公事掌瀦洩三白渠,以給關中灌溉之利。



 撥發司輦運司掌以時起發綱運而督其滯留,以供京師之用。



 提舉弓箭手掌沿邊郡縣射地弓箭手之籍,及團結、訓練、賞罰之事。政和五年,復以所招弓箭手之數為殿最。



 府州軍監宋初革五季之患,召諸鎮節度會於京師,賜第以留之,分命朝臣出守列郡,號權知軍州事,軍謂兵,州謂民政焉。其後,文武官參為知州軍事,二品以上及帶中書、樞密院、宣徽使職事,稱判某府、州、軍、監。諸府
 置知府事一人,州、軍、監亦如之。掌總理郡政,宣布條教,導民以善而糾其奸慝,歲時勸課農桑,旌別孝悌,其賦役、錢穀、獄訟之事,兵民之政皆總焉。凡法令條制,悉意奉行,以率所屬。有赦宥則以時宣讀,而班告於治境。舉行祀典。察郡吏德義材能而保任之,若疲軟不任事,或奸貪冒法,則按劾以聞。遇水旱,以法振濟。安集流之,無使失所。若河南、應天、大名府則兼留守司公事。太原府、延安府、慶州、渭州、熙州、秦州則兼經略安撫使、馬步軍
 都總管。定州真定府、瀛州、大名府、京兆府則兼安撫使、馬步軍都總管。瀘州、潭州、廣州、桂州、雄州則兼安撫使、兵馬鈐轄。穎冒府、青州、鄆州、許州、鄧州則兼安撫使、兵馬巡檢。其餘大藩府或沿邊州郡,或當一道沖要者,並兼兵馬鈐轄、巡檢,或帶沿邊安撫、提轄兵甲、沿邊溪洞都巡檢。餘州、軍,則別其地望之高下與職務之繁簡而置之。分曹以理之。而總其綱要。凡屬縣之事皆統焉



 建炎初,詔:「河北、京東西路除帥司外,舊差文臣知州去處,
 許通差武臣一次。」又:「要郡文臣一員帶本路兵馬鈐轄,武臣一員充副鈐轄;次要郡文臣一員帶本路兵馬都監,武臣一員充副都監。」紹興三年,詔守臣帶路分鈐轄、都監去處並罷。五年,帝以守。令皆帶勸農公事,多不奉職,自今有治效顯著者,可今中書省籍記姓名,特加擢用。凡從官出知郡者,特許不避本貫。初,除授見闕及自外罷任赴闕,並令引見上殿。九年,詔應守臣以二年為任。又以武臣作郡,往往不曉民事,且多恣橫,詔新復州
 郡只差文臣續因臣僚言,極邊控扼去處,仍差武臣;其不系極邊,文武臣通差。詔:「守臣到任半年以上,具民間利病,或邊防五條聞奏,委都司看詳,有便於民者,即與施行。」續又詔不拘五條之數。十三年,詔依舊制帶提舉或主管學事。



 從官以上稱提舉餘知、通主管,淳熙中罷。



 乾道二年,令非曾任守臣不得為郎官,諸郡合文武臣通差去處,並依舊制



 通判宋初懲五代藩鎮之弊,乾德初,下湖南,始置諸州通判,命刑部郎中賈玭等充。建隆四年,詔知府公事並
 須長吏、通判簽議連書,方許行下。時大郡置二員。餘置一員。州不及萬戶不置,武臣知州,小郡亦特置焉。其廣南小州,有試秩通判兼知州者,職掌倅貳郡政,凡兵民、錢穀、戶口、賦役、獄訟聽斷之事,可否裁決,與守臣通簽書施行。所部官有善否及職事修廢,得刺舉以聞。元祐元年,詔知州系帥臣,其將下公事不許通判同管。元符元年,詔通判、幕職官,令日赴長官廳議事及都廳簽書文檄。



 南渡後,設官如舊,入則貳政,出則按縣。有軍旅之
 事,則專任錢糧之責,經制、總制錢額,與本郡協力拘催,以入於戶部。既而諸州通判有兩員處減一員。凡軍監之小者不置。又詔更不添差。其後,或以廢事請,或以控扼去處請。紹興五年以後旋添置之。除潭廣洪州、鎮江建康成都府見系兩員外,凡帥府通判並以兩員為額,餘置一員。乾道元年,詔買馬州、軍通判,令茶馬司依舊法奏闢,餘堂除差人。淳熙十四年,利州路提刑言:「關外四州通判,乞自制置司奏闢,所有金、洋、興、利、文、龍待州通判,
 乞送轉運司擬差。」並從之。



 幕職官簽書判官廳公事兩使、防、團、加事推判官節度掌書記觀察支使掌裨贊郡政,總理諸案文移,斟酌可否,以白於其長而罷行之。凡員數多寡,視郡小大及職務之煩簡。初,政和改簽書判官廳公事為司錄,建炎初復舊。凡節度推、判官從軍額,察推及支使從州、府名。凡諸州減罷通判處,則升判官為簽判以兼之。小郡推、判官不並置,或以判官兼司法,或以推官兼
 支使,亦有並判官窠闕省罷。則令錄參兼管。凡要郡簽判及推官皆堂除,餘吏部使闕,二廣間許監司闢差。紹熙元年,臣僚言:「廣西奏擬簽判,多恩科癃老,乞行下轉運司,不許差年六十以上昏眊之人。」嘉定二年,臣僚言:「監司有幹官,州郡有職官,以供簽廳之職,或非才不勝任,則按刺易置可也。今乃差兼簽廳者動軌三兩員,或四五員。其為冗費,與添差何異?乞將諸州郡所差兼簽廳官並行住罷。」從之。



 諸曹官舊制,錄事參軍掌州院庶務,糾諸曹稽違;戶曹參軍掌戶籍賦稅、倉庫受納;司法參軍掌議法斷刑;司理參軍掌訟獄勘鞫之事。中興,詔曹掾官依舊,惟司理、司法並注經任及試中刑法人。乾道以來,間以司戶兼司法。知錄亦或兼職。六年,汪大猷言:「司戶初官,令專主倉庫,知錄依司理例以獄事為重,不兼他職。」從之仍依知縣格法銓量,如有老疾昏眊難任事者,即從本州知通於判、司、簿、尉內選經一考以上無罪犯曉法
 人對換。紹熙元年,詔不曾銓試人不許注授司法。慶元五年,臣僚言:「司理獄事煩重,宜優其舉主,照提刑司合舉主三員以上許間歲舉獄官一員。」嘉定中,申明年滿六十不許為獄官之令,仍不許恩科人注授。



 教授景祐四年,詔藩鎮始立學,他州勿聽,慶歷國年,詔諸路州、軍、監各令立學,學者二百人以上,許更置縣學。自是州郡無不有學。始置教授,以經術行義訓尋諸生,掌其課試之事,而糾正不如規者。委運司及長史於幕
 職、州縣內薦,或本處舉人有德藝者充。熙寧六年,詔諸路學官委中書門下選差,至是,始命於朝廷。元豐元年,州、府學官共五十三員,諸路惟大郡有之。軍、監未盡置。元祐元年,詔齊、盧、宿、常等州各置教授一員。自是列郡各置教官。建炎三年,教授並罷。紹興三年,復置四十二州。十二年,詔無教授官州、軍,令吏部申尚書省選差。二十六年,詔並不許兼他職,令提舉司常切遵守。若試教官,則始於元豐;添差教授,則始於政和。



 縣令建隆元年,令天下諸縣除赤、畿外,有望、緊、上、中、下。掌總治民政、勸課農、桑、平決獄訟。有德澤禁令,則宣布於治境。凡戶口、賦役、錢穀、振濟、給納之事皆之,以時造戶版及催理二稅。有水旱則有災傷之訴,以分數蠲免。民以水旱流記,則撫存安集之,無使失業。有孝悌行義聞於鄉閭者,具事實上於州,激勸以勵風谷。若京、朝、幕官則為知縣事,有戍兵則兼兵馬都監或監押。



 宣教郎以下帶監押。



 初,建炎多差武臣,紹興詔專用文臣,然沿邊溪洞
 處,仍許武臣指射。邑大事煩則堂除,仍借緋、章服,嚴差出之禁,任滿有政績,則與升擢。乾道以後,定以三年為任,仍非兩任不除監察御史。初改官人必作縣,謂之「須入」。十六年,詔知縣在任不成兩考,即不合理為實歷。嘉定十二年詔:「兩經作令滿替者,實歷九考、有政聲無過犯、舉員及格,改官人特免再作知縣,許受簽判或幹官,以當知縣履歷。」



 縣丞初不置,天聖中因蘇耆請,開封兩縣始各置丞一
 員,在簿、尉之上,仍於有出身幕職、令錄內選充。皇祐中,詔赤縣丞並除新改官人。熙寧四年,編修條例所言:「諸路州、軍繁劇縣,令戶二萬已上增置丞一員,以幕職官或縣令人充。」元祐元年詔:「應因給納常平、免役置丞,並行省罷。如委事務繁劇難以省罷處,令轉運司存留。」崇寧二年,宰相蔡京言:「熙寧之初,修水土之政,行市易之法,興山澤之利,皆王政之大,請縣並置丞一員,以掌其事。」大觀三年,詔:「昨增置縣丞內,除舊額及萬戶以上縣
 事務繁冗,及雖非萬戶實有山澤,坑冶之利可以修興去處,依舊存留外,餘皆減罷。」建炎元年,詔縣丞系嘉祐以前員闕並萬戶處存留一員。餘並罷。紹興三年,以淮東累經兵火,權罷縣丞。十八年,置海陵丞一員。嘉定後,小邑不置丞,以簿兼。



 主簿開寶三年,詔諸縣千戶以上置令、簿、尉;四百戶以上置令、尉,令知主簿事;四百戶以下置簿、尉,以主簿兼知縣事。咸平四年,王欽若言:「川峽縣五千戶以上請並
 置簿,自餘仍以尉兼。」從之。自後川蜀及江南諸縣,各增置主簿。中興後,置簿掌出納官物、銷注簿書,凡縣不置丞,則簿兼丞之事。凡批銷必親書押,不許用手記,仍不許差出,以防銷注。



 尉建隆三年,每縣置尉一員,在主簿之下,奉賜並同。至和二年,開封、祥符兩縣各增置一員,掌閱羽弓手,戢奸禁暴。凡縣不置簿,則尉兼之。中興,沿邊諸縣間以武臣為尉,並帶兼巡捉私茶、鹽、礬,亦或文武通差。隆興,詔不
 許差癃老疾病年六十以上之人。邑大事煩則置二尉。紹熙中,詔恩科人年及六十不差。嘉定十三年,詔極國縣尉,護盜酬賞班改,歲以二員為額。



 鎮砦官諸鎮置於管下人煙繁盛處,設監官,管火禁或兼酒稅之事。砦置於險扼控御去處,設砦官,招收土軍,閱習武藝,以防盜賊。凡杖罪以上並解本縣,余聽決遣。



 廟令丞主簿舊制,五岳、四瀆、東海、南海諸廟各
 置令、丞。廟之政令多統於本縣令。京朝知縣者稱管勾廟事,或以令、錄老耄不治者為廟令,判、司、簿、尉為廟簿,掌葺治修飾之事。



 凡以財施於廟者,籍其名數而掌之。



 總管鈐轄司掌總治軍旅屯戍、營防守禦之政令。凡將兵隸屬官訓練、教閱、賞罰之事,皆掌之。守臣帶提舉兵馬巡檢、都監及提轄兵甲者,掌統治軍旅、訓練教閱,以督捕盜賊而肅清治境。凡諸營名籍、賞罰之事,皆掌之。崇寧四年,蔡京奏:「京畿四輔置輔郡屏衛京師,以穎冒
 府為南輔,襄邑縣升為拱州為東輔,鄭州為西輔,澶州為北輔,以太中大夫以上知州,置副總管、鈐轄各一員,知州為都總管。餘依三路帥臣法。」從之。



 大觀三年,詔東南師府總管。依三路都總管法。靖康元年,詔四道副總管並通差文武臣其諸路將官,掌統所隸禁旅,以行陣隊伍、金鼓旗幟、弓矢擊刺之法而教習訓練之,別其武藝強者,待次遷補,以激勸士卒。凡兵仗器甲之數,廩祿犒設、賞罰約束之禁令皆掌焉,副將為之貳。若屯戍防
 邊。則受帥司節制;遇寇敵,則審其戰守應援之事。若師有功,則具馘數、籍用命而旌賞之。



 路分都監掌本路禁旅屯戍、邊防、訓練之政令,以肅清所部。州府以下都監,皆掌其本城屯駐、兵甲、訓練、差使之事,資淺者為監押。紹聖三年,詔諸路將副序位在路分都監之下。大觀三年,詔帥府無路分鈐轄、望郡無路分都監者,許置一員,其餘添置處,任滿不差人。宣和二年,虔州添置都監一員。



 建炎初,分置帥府,以諸路帥臣
 兼。要郡守臣帶兵馬鈐轄,次要郡帶兵馬都監;並以武臣為之副,稱副總管、副鈐轄、副都監,許以便宜行軍馬事,闢置僚屬,依帥臣法。屯兵皆有等差。遇朝廷起兵,則副總管為帥,副鈐轄、都監各以兵從,聽其節制。其後,益、瀘、夔、廣、桂五州牧又皆以都鈐轄為稱。四年,詔建康府、江州路又置副都總管一員,於見置帥司處駐扎。紹興三年,詔要郡、次要郡守臣罷帶兵職,其逐路副總管依舊格,改充路分都監,為一路掌兵之官,其各州鈐轄或
 省或置不一,又有逐路兵馬都監、兵馬監押,掌煙火公事、捉捕盜賊。淳熙十六年,詔諸路訓練鈐,並須年六十以下曾經從軍有才武人充,紹熙元年指揮,雜流出身之人,不得過路分州鈐;諸州軍兵馬都監,獨員處專注才武及曾任主兵官之人。慶元中,詔總管下至將副將等,年七十以上許自陳,與宮觀差遣。初,守臣罷帶兵職,惟江西贛州以多盜,仍帶江西兵馬鈐轄。其後,武臣為路鈐者,亦無尺籍伍符,每歲諸州按閱,特存故事,間有
 得旨葺治軍器或訓練禁軍,則仍帶入銜。



 諸軍都統制副都統制統制統領舊制,出師征討,諸將不相統一,則拔一人為都統制以總之,未為官稱也,建炎初,置御營司,擢王淵為都統制,名官自此始。其後,神武五軍及川陜宣撫司、都督府、樞密院皆置。紹興十一年,三大將兵罷,諸軍皆冠以:御前」二字,擢其偏裨為御前統領官,以統制御前軍馬入銜,秩高者為御前諸軍都統制,且令仍舊駐扎,以屯駐州名冠軍額
 之上。其後,興元、江、陵、建、康、鎮江府、興、金、鄂、江、池州及平江、許溥水軍,皆除都統制,恩數略視三衙,權任在帥臣右,官卑者稱副都統制。設屬有計議、機宜、乾辦公事、準備差遣,省置不一。次有副都統制。乾道三年,帝諭輔臣「欲今後江上諸軍各置副都統一員,兼領軍事,豈惟儲帥,亦使主將顧忌,不敢專擅。」因言:「都、副統制禮有隆殺,且為條約。」上曰:「如此,他日不致爭權越禮。」遂行之。然其後都、副鮮有並除者。初,渡江後,大軍又有統制、同統制、
 副統制、統領、同統領、副統領,其下有正將、淮備將、訓練官、部將、隊將等名,皆偏裨也。舊制,準備將而上,皆主帥升差,仍先申樞密院審察。乾道七年,詔訓練官、部隊將而下,許軍中徑差,申朝廷照會。紹熙間,詔諸軍升差統制至準備將者,主帥解發三人,赴總領所選一名,諸將不以為便。慶元三年,詔主帥選擇,總領所或屯軍處守臣審核保明,申樞密院。



 巡檢司有沿邊溪峒都巡檢,或蕃漢都巡檢,或數州數
 縣管界、或一州一縣巡檢,掌訓治甲兵、巡邏州邑、擒捕盜賊事;又有刀魚船戰棹巡檢,江、河、淮、海置捉賊巡檢,及巡馬遞鋪、巡河、巡捉私茶鹽等,各視其名以修舉職業,皆掌巡邏幾察之事。中興以後,分置都巡檢使、都巡檢、巡檢、州縣巡檢,掌土軍、禁軍招填教習之政令,以巡防捍禦盜賊。凡沿江沿海招集水軍,控扼要害及地分闊遠處,皆置巡檢一員,往來接連合相應援處,則置都巡檢以總之,皆以材武大小使臣充。各隨所在,聽州縣
 守令節制,本砦事並申取州縣指揮。若海南瓊管及歸、峽、荊門等處跨連數郡,控制溪峒,又置水陸都巡檢使或三州都巡檢使以增重之。



 監當官掌茶、鹽、酒稅場務征輸及冶鑄之事,諸州軍隨事置官,其徵榷場務歲有定額,歲終課其額之登耗以為舉刺。凡課利所入,日具數以申於州。建炎初,詔監當官闕,許轉運司具名奏闢一次,以二年為任,實有六考,方許關升。煩劇去處,許添差一員。凡交割必置歷以稽
 其剩欠,合選差文臣處,更不差武臣。淳熙二年,詔二萬貫以下庫分,選有才幹存留一員,指揮、諸班直、親從親事官、保義郎以下差充。建炎四年,詔每州每以五員為
 額。



\end{pinyinscope}