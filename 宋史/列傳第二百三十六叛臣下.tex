\article{列傳第二百三十六叛臣下}

\begin{pinyinscope}

 ◎
 叛
 臣下○李全下



 寶慶三年二月,楊氏使人行成於夏全曰:「將軍非山東歸附耶?狐死兔泣,李氏滅,夏氏寧獨存?願將軍垂盼。」全
 諾。楊氏盛飾出迎,與按行營壘,曰:「人傳三哥死,吾一婦人安能自立?便當事太尉為夫,子女玉帛、幹戈倉廩,皆太尉有,望即領此,誠無多言也。」夏全心動,乃置酒歡甚,飲酣,就寢如歸,轉仇為好,更與福謀逐卓矣。



 辛卯,夏全令賊黨圍州治。焚官民舍,殺守藏吏,取貨物。時卓精兵尚萬餘,窘束不能發一令,太息而已,夜半縋城,僅以身免。鎮江軍與賊戰死者太半,將校多死,器甲錢粟悉為賊有。卓步至揚州,借州兵自衛,猶劄揚州造旗幟。林拱
 繳奏於朝,聞者大笑。夏全既逐卓,幕歸,楊氏拒之,意楊氏反目圖己,明日大掠,趨盱眙欲為亂,張惠、範成進閉門,不得入,翱翔淮上。惠、成進出兵欲剿之,夏全狼狽歸金,金人納之。是舉也,張正忠不從亂,經妻女於庭,並己自焚。報至,中外大恐,劉卓自劾,未幾,死。



 初,姚翀從賈涉辟楚州推官,全喜其附己,為引重當路,得改秩,全請以通判青州。國之死,全借翀撫定以誑眾,以功入朝。三月,以翀為軍器少監、知楚州兼製置。翀辟鄭子恭、杜耒等
 為幕客,留母及其子於京,買二妾以行。至城東,艤舟以治事。間入城見楊氏,用晞稷故事而禮過之。楊許翀入城,乃入,寄治僧寺,極意娛之。



 時全在圍一年,食牛馬及人且盡,將自食其軍。初軍民數十萬,至是餘數千矣。四月辛亥,全欲歸於大元,懼眾異議,乃焚香南向再拜,欲自經,而使鄭衍德、田四救之,曰:「譬如為衣,有身,愁無袖耶?今北歸蒙古,未必非福。」全從之,乃約降大元。大元兵入青州,承製授全山東行省。



 慶福在山陽,自知己為厲
 階,懷不自安,欲圖福以自贖。福知之,亦謀去慶福。二人互相猜貳,不相見。福偽病旬餘,諸將問疾,慶福不往。張甫者,素厚慶福,懼福疑己,乃勸慶福往。後慶福約甫同往,乃寢,遙見福臥不解衣,心恐,不得已至床前,見床頭鞘刀,慶福口問疾而手按鞘,懼福先發。福疑慶福就刀見害,乃躍起拔刀傷慶福,慶福徒手不支,甫救之。左右群起殺慶福及甫。



 甫本金元帥,封高陽公,最善馭眾。金亡河北,甫據雄、霸、清、莫、河間、信安不下。信安出白溝,距
 燕二百里而阻巨濼,大元兵不能涉,甫每潛師窺伺。大元將俚砦奴屢欲滅甫以取雄、霸。驍將窩羅虎者,歸甫,甫納之。其後窩羅虎遁去,且竊甫千里馬以獻俚砦奴。俚砦奴喜,待遇益厚。嚐會飲燕京之大悲閣,窩羅虎醉俚砦奴而推使投閣,幾斃焉。窩羅虎乃佯醉下樓,復乘所獻馬以歸甫,追者莫及,人始服甫之用間焉。其後歸全。



 福以慶福頭納翀,翀大喜,耒曰:「慶福首禍,一世奸雄,今頭落措大手耶!」飛報於朝,遣子恭繼奏捷。卓之敗,儲
 積掃地,綱運不續,賊黨籍籍,謂福所致。福數見翀及僉幕促之,皆謝以朝廷撥降未下,福曰:「朝廷若不養忠義,則不必建閫開幕,今建閫開幕如故,獨不支忠義錢糧,是欲立製閫以困忠義也。」六月,福乘眾怒,與楊氏謀,召翀飲。翀至而楊氏不出,就坐賓次,左右散去。福與翀命召諸幕客,以楊氏命召翀二妾。諸幕客知有變,不得已往。耒朝服至八字橋,福兵腰戮之,耒南望再拜就斃。二妾之入,翀及見之。福兵欲害翀,鄭衍德救之得免,去須
 鬢,縋城西夜走,徒步歸明州,未幾。死。



 朝廷以淮亂相仍,遣帥必斃,莫肯往來。始欲輕淮而重江,楚州不復建閫,就以帥楊紹雲兼製置,改楚州名淮安軍,命通判張國明權守,視之若羈縻州然。賊徒黨塞南門,開北門,支邑民田皆以少價抑買之,自收賦以贍軍,錢糧不繼如故。賊將國安用、閻通歎曰:「我曹米外日受銅錢二百,楚州物賤可以樂生,而劉慶福為不善。怨仇相尋,使我曹無所衣食。」張林、邢德亦謂:「嚐受宋恩,中遭全間隙,今歸於
 此,豈可不與朝廷立事?」王義深亦嚐遭全屈辱,且謂:「我本賈帥帳前人,與彭安撫舉義不成而歸。」五人相謂曰:「朝廷不降錢糧,為有反者未除耳!」乃共議殺福及楊氏以獻,於是眾帥兵趨楊氏家。福出,德手刃之,相屠者數百人。有郭統製者,殺全次子。通殺一婦人,以為楊氏,函其首並福首馳獻於紹雲。紹雲驛送京師,傾朝甚喜。檄彭𢖲、張惠、範成進、時青並兵往楚州,便宜盡戮餘黨。未幾,傳楊氏故無恙,婦人頭乃全次妻劉氏也。



 𢖲輕儇,每
 供四總管弄戲,得檄不敢自決,力遜。惠、成進二人即提兵入楚城,與林等五人歡宴,議分北軍為五,使五人分掌之,每軍無過千人,一屯南渡門,一屯平河橋,一屯北神鎮,城中城西各一;在山東人老幼並絕錢糧,出淮陰戰艦,陳淮岸以斷全歸路,請製府及朝廷處之。廟議謂青望重,惟聽青區畫。省檄之下,不及惠、成進。青亦恐禍及,密遣人報全於青州,遷延不決。惠等歸盱眙,賊黨復振。紹雲赴樞密稟議,淮東總領岳珂攝製府事。



 惠、成進
 既歸,錢糧缺乏,密約降金,盧鼓槌許之。時鎮江軍及滁州虎兒軍在盱眙者尚眾,二人紿𢖲曰:「南北軍易致激變,宜令軍人出入無得帶刃。」又勸早發虎兒軍折洗,𢖲從之。二人每宴𢖲,必遍迨皂隸,𢖲皆不悟,方感其拒夏全之功,轉兩軍官資。二人同戲下合辭曰:「不願得官,欲得錢糧。」八月辛酉,惠、成進燕𢖲,𢖲左右知有謀,多不往,𢖲往如平時。酒半,縛𢖲,𢖲從者無寸鐵,且醉,皆就縛。即日渡淮輸款,以盱眙附盧鼓槌於泗州。金兵至,開門接
 之,諸軍不戰皆降。於是塞南門,開北門,導淮水以通泗之東西域焉。盧鼓槌與惠釋憾連姻,金官惠有加,俾專製河南,以拒大元。自是金人窺淮東益急,朝廷調京湖製置司兵萬人屯青平山以備全。



 全得青報慟哭,力告大元大將,求南歸,不許;斷一指示歸南必畔,許之。承製授山東、淮南行省,得專製山東,而歲獻金幣。十月丙辰,全與大元張宣差並通事數人至楚州,服大元衣冠,文移紀甲子而無號。義深走金,安用殺林、德自贖。丁巳,全
 邀青及張國明於淮陰,國明辭疾,青父子同至。全推殺其子者郭統製斬之,又收田成瑤、田之昂、李英等八人下獄,云:「非朝廷殺我妻子,吾惟問汝。」李英,全腹心,狡而密,與李平皆山東胥吏。全之乍逆乍順,二人所教也。平又數致全書至廟堂,以覘朝廷。青繳所授檄於全曰:「我素推尊相公,豈肯為此!」全亦惡青反覆。辛酉,與登城南樓飲,殺青,馳騎往紿青妻,言青病,見與禱禳。青妻至,盡殺之。遂並青軍,擢小校胡義為將,徙其半於漣、海。



 紹定
 元年春,全厚募人為兵,不限南北,宋軍多亡應之。天長民保聚為十六砦,比歲失業,官振之,不能繼,壯者皆就募。射陽湖浮居數萬家,家有兵仗,侵掠不可製,其豪周安民、穀汝礪、王十五長之,亦蜂結水砦,以觀成敗。翟朝宗知揚州,權製置。全厚賞捕趙邦永,邦永乃變名必勝。全知東南利舟師,謀習水戰,米商至,悉並舟糴之。留其柁工,一以教十。又遣人泛江湖市桐油煔筏,厚募南匠,大治舭達船,自淮及海相望。於是善湘禁桐油煔筏下
 江,嚴甚。朝宗市煔木往揚州,善湘亦聞於朝,請以鬆木易留之。全不得已,代以榆板,舟成多重滯。六月,試舟射陽湖,善湘恐其乘便搗通、泰,亟牒池州求通、泰入湖之路。七月壬辰,全使衍德提兵三萬如海州。乙未,全及楊氏大閱戰艦於海洋。八月,全趨青州,為嚴實及石小哥邀擊,敗走。小哥,珪子也,遂奪青厓崮,據之,九月,全歸海州,治舟益急,驅諸崮人習水。十一月,全至楚州。全山東經理未定,而歲貢於大元者不缺,故外恭順於宋以就
 錢糧,往往貿貸輸大元。宋得少寬北顧之憂,遣餉不輟。全縱遊說於朝,不若復建山陽製置司。全又與金合縱,約以盱眙與之,金亦遣靳經曆者聘全,皆不遂。



 二年四月,全以糧少為詞,遣海舟自蘇州洋入平江、嘉興告糴,實欲習海道,覘畿甸也。六月,全資淮安牛馬會趙五嘯合亡命,雜北軍分往盱眙略牛馬。九月,全往漣、海視戰艦,陽言歸東平葬方士許先生。未幾,還。嚐燕張國明等,忽曰:「我乃不忠不孝之人。」眾曰:「節使何為有是言也?」全
 曰:「縻費朝廷錢糧至多,乃殺許製置,不忠;我兄被人殺,不能報復,不孝。二月二十五日事,吾之罪也。十一月十三日事,誰之罪耶?」蓋指卓與夏全也。全密遣軍掠高郵、寶應、天長之間,知高郵軍葉秀發遣宗雄武領民兵捍禦,為賊所敗。



 三年二月壬寅,御前軍器庫火。得縱火者,楚州軍穆椿也。全欲銷宋兵備,故使椿行,且伏奸於外,謀入為亂,以不得入而止。於是先朝兵甲盡喪。椿臨刑笑曰:「事濟矣。」全欲先據揚州以渡江,分兵徇通、泰以趨
 海。諸將皆曰:「通、泰,鹽場在焉,莫若先取為家計,且使朝廷失鹽利。」全欲朝廷不為備,且雖反而難遽絕錢糧,乃挾大元李、宋二宣差恫疑虛喝,而使國明達諸朝。而大元實未嚐資全兵。有識李宣差者,曰:「此青州賣藥人也。」七月,召國明稟議,全以寶玉資其行,賓從所過,揚言:「 李相公英略絕倫,其射五百步,朝廷莫若裂地王之,與增錢糧,使當邊境。」遍饋要津,求主其說。既見廟堂,以百口保全不叛。



 八月,全將閱舟師,風不順,焚香禱曰:「使全有
 天命,當反風。」語畢風反。大閱數日。會全糴麥舟過鹽城縣,朝宗嗾尉兵奪之。全怒,以捕盜為名,庚午,水陸數萬徑搗鹽城,戍將陳益、樓強皆遁,全入城據之。知縣陳遇逾城走,公私鹽貨皆沒於全。朝宗倉皇遣幹官王節入鹽城,懇全退師;又遣吏曾玠、李易入山陽,求楊氏裏言之助,皆不答。朝宗乃遣卞整領兵扼境。全留鄭祥、董友守鹽城,提兵往楚。整與遇麾軍道左,擊柝聲諾。全言於朝,稱遣兵捕盜過鹽城,令自棄城遁去,慮軍民驚擾,未
 免入城安眾。乃加全兩鎮節,令釋兵,命製置司幹官耶律均往諭之。全曰:「朝廷待我如小兒,啼則與果。」不受。朝廷為罷朝宗,謀再用紹雲,紹雲辭以官卑不能製;命鄭損,損辭。通判揚州趙敬夫暫攝事。



 全造舟益急,至發塚取煔板,煉鐵錢為釘鞠,熬人脂搗油灰,列炬繼晷,招沿海亡命為水手。又紿敬夫以大元為詞,邀增五千人錢糧,求誓書鐵券。朝廷猶遣餉不絕。全得米,即自轉輸淮海入鹽城以贍其眾。他軍士見者曰:「朝廷惟恐賊不飽,
 我曹何力殺賊!」射陽湖人至有「養北賊戕淮民」之語,聞者太息。



 王十五附全,全又遣人以金牌誘脅周安民等,造浮梁於諭口,以便鹽城來往;又開馬攞港、壽河,引淮船入湖,為攻撓水砦計。復言於製置司云:「全復歸三年,淮甸寧息,雖荷大丞相力主安靖之說,深有覆護之恩,奈何趙製置、嶽總管、二趙兄弟人自為政,使全難處!全欲決定去就,親往鹽城存劄。若有疾全者、疑全者,如趙知府之輩,便可提兵決戰。如能滅全,高官重祿任彼取
 之,倘不能滅,方表全心。」善湘見之甚憤,範亦請調兵。



 時彌遠多在告,執政無可否,舉朝率謂:「大丞相老於經綸,豈不善處?」獨參知政事鄭清之深憂之,密與樞密袁韶、尚書範楷議,二人所見合。清之乃約韶見帝,韶曆言全狀,帝有憂色。清之即力讚討全,帝意決。清之退,以帝意告彌遠,彌遠意亦決。乙巳,金字牌進善湘煥章閣學士、江淮製置大使,範直徽猷閣、知揚州、淮東安撫副使,葵直寶章閣、淮東提點刑獄兼知滁州,俱節製軍馬,全子
 才軍器監簿、製置司參議官。下詔曰:



 君臣,天地之常經;刑賞,軍國之大枋。順斯柔撫,逆則誅夷。惟我朝廷兼愛南北,念山東之歸附,即淮甸以綏來。視爾遺黎,本吾赤子,故給資糧而脫之餓殍,賜爵秩而示以寵榮,坐而食者逾十年,惠而養之如一日,此更生之恩也,何負汝而反耶?蠢茲李全,儕於異類,蜂屯蟻聚,初無橫草之功;人麵獸心,曷勝擢發之罪!繆為恭順,公肆陸梁。因饋餉之富,以嘯集儔徒;挾品位之崇,以脅製官吏。淩蔑帥閫,殺
 逐邊臣,虔劉我民,輸掠其眾。狐假威以為畏己,犬吠主旁若無人。姑務包含,愈滋猖獗,遽奪攘於鹽邑,繼掩襲於海陵,用怨酬恩,稔惡恣暴。為封豕以洊食,貪婪無厭;怒螳螂而當車,滅亡可待。故神人之共憤,豈覆載之所容!舍是弗圖,孰不可忍!李全可削奪官爵,停給錢糧。敕江、淮製臣,整諸軍而討伐;因朝野僉議,堅一意以剿除。蔽自朕心,誕行天罰。



 顧予眾士,久銜激憤之懷;暨爾邊氓,期洗沈冤之痛。益勉思於奮厲,以共赴於功名。凡曰
 脅從,舉官效順,當察情而宥過,庸加惠以褒忠。爰飭邦條,式孚群聽:應擒斬到全者,賞節度使,錢二十萬,銀絹二萬匹;同謀人次第擢賞。能取奪見占城壁者,州,除防禦使;縣,除團練使;將佐官民以次推賞。逆全頭目兵卒皆我遺黎,豈甘從叛?諒由劫製,必非本心。所宜去逆來降,並與原罪;若能立功效者,更加異賞。鄭衍德、國安用雖與逆全管兵,然屢效忠款,乃心本朝,馮垍、於世珍雖為逆全信用,然俱通古今,宜曉逆順,如率眾來降,當
 加擢用。四方士人流落淮甸,一時陷賊,實非本心,如能相率來歸,當與赦罪。海州、漣水軍、東海縣等處有為逆全守城壁者,舉城來降,當各推恩。時青以忠守境,屢立駿功;彭義斌以忠拓境,大展皇略,亦為逆全謀害,俱加贈典,追封立廟。



 噫,以威報虐,既有辭於苗民,惟斷乃成,斯克平於淮、蔡。布告中外,鹹使聞知。詔詞,清之所代也。促荊襄、淮西諸軍赴援。



 壬子,全兵突至灣頭,敬夫恐,欲走,副都統丁勝劫閽者止之。全攻城南門,都統趙勝自
 堡砦提勁弩赴大城注射,全稍退。全遣劉全奄至堡砦西城下,欲奪之以瞰大城。先是,趙勝屯西城,見濠淺,每曰:「設有寇至,未圍大城,先襲堡砦,何可不備?」盛暑中督軍浚濠,人皆苦之,翟朝宗亦以為笑。既浚,勝決新塘水注焉。及是,劉全不能進,勝又浚市河,人尤謂不急。全至,勝開水門納賈舟千餘艘,活者數千人,糧貨不與焉。



 時朝廷雖下詔討全,而猶有內圖戰守、外用調停之說。是日,敬夫得彌遠書,許增萬五千人糧,勸全歸楚州。敬夫
 亟遣劉易即全壘授全。全笑曰:「丞相勸我歸,丁都統與我戰,非相紿耶?」擲書不受,惟留省劄。敬夫始知全紿己,亟發牌印迓範。癸丑,全塞泰州城濠。於邦傑、宗雄武通全,戒守者無得發矢,俟薄城而蹙之,全得距堙。宋濟恐,令縣尉某如全壘,全以增糧省檄示之,尉復出,獻錢二百萬以降。乙卯,邦傑、雄武開門導全,濟帥僚吏出迎。全入坐郡治,濟發帑出所獻錢,全曰:「獻者,獻汝私藏耶?若泰州府庫,則我固有,何假汝獻為!」乃舍濟僉判廳,入郡
 堂,盡收子女貨幣。



 庚申,全聞範、葵既入,鞭衍德曰:「我計先取揚州渡江,爾曹勸我先取通、泰,今二趙入揚州矣,江其可渡耶?」莫敢對。既而曰:「今惟有徑搗揚州耳。」甲子,全配兵守泰州,悉出眾宜陵。丙寅,至灣頭立砦,據運河之衝。使胡義將先鋒騎駐平山堂,伺三城機便。丁卯,全攻城東門不利,賊將張友呼城東請見葵,全隔濠立馬相勞苦,葵切責之,全彎弓抽矢向葵而去。戊辰,張璡、戴友龍、王銓、張青以天長製勇三軍至,阻全不得前,遣人
 請援。範、葵親出堡塞西門,列陳待之,全不敢動,璡等乃入城。庚午,全晨率步騎五千餘攻堡塞西門,趙勝出兵,戰不利,範、葵以兵益之。全兵亦增,葵擊卻之。辛未,賊引兵三萬沿州城東向西門,李虎、趙必勝、張璡、崔福力戰,自巳至申,全乃沿東門以歸,丁勝、王鑒、於俊擊走之。襄兵萬人至真州上壩,統製張達、監軍張大連不設備,魚貫而行。全哨馬帥田四擊之為數截,殲者五千,達、大連死之;淮西援兵至,亦遇全統領桑青力戰,城中俱不知
 也。襄兵敗,全凶焰益振,每曰:「我不要淮上州縣,渡江浮海,徑至蘇、杭,孰能當我!」甲戌,復引輕騎犯州城南門,且欲破堰泄濠水,統製陳達率勁弩射之,範、葵出軍迎擊,乃去。是日,金玠等距淮安十里,焚全砦柵,全將劉全出戰,玠軍不利,退屯寶應。



 全誌吞三城,而兵每不得傅城下,宗雄武獻全計曰:「城中素無薪,且儲蓄為總所支借殆盡,若築長圍,三城自困。」乙亥,全悉眾及驅鄉農合數十萬列砦圍三城,製司總所糧援俱絕。範、葵命三城諸
 門各出兵劫砦,舉火為期,夜半縱兵衝擊,殲賊甚眾。自是賊一意長圍,以持久困官軍,不復薄城。戊寅,全張蓋奏樂平山堂,布置築圍,指揮閑暇。範、葵令諸門以輕兵牽製,親帥將士出堡砦西,全分路鏖戰,自辰至未,殺傷相當。庚辰,範出師大戰,玠等破全將張友於都倉,獲糧船數十艘。甲申,葵出戰,賊大敗。



 四年正月辛卯,全兵浚圍城塹,範、葵遣諸將出城東門掩擊,全走土城,官軍躡之,蹂溺甚眾。是日,玠破全將鄭祥,獲糧百艘。甲午,全兵
 千餘犯州城東門,城中出兵應之,全即引去。乙未,李虎出南門,楊義出東門,王鑒出西門,崔福出北門,各徑扼賊圍,開土城數處,範、葵提兵策應,全步騎數千出戰,諸軍奮擊,俘馘甚眾。夜,賊復合所開城。丁酉,趙勝遣統製陸昌、孫舉立橋堡砦於北門,賊步騎分道來戰,勝擊退之。範陳於西門,賊閉壘不出。葵曰:「賊俟我收兵而出爾。 」乃伏騎破垣門,收步卒誘之。賊兵數千果趨濠側,虎力戰,城上矢石雨注,賊退。有頃,賊別隊自東北馳至,範、葵
 揮步騎夾浮橋、吊橋並出,為三迭陳以待之,自巳至未,賊與大戰;別遣虎、顯廣、必勝、義以馬步五百出賊背,而葵帥輕兵橫衝之,三道夾擊,用範所製長槍,果大利,賊敗走。翼日,全遣步卒三百餘向城西門,乍進乍退,以誘揚州兵,復驅壯丁增濠麵,培鹿角。範、葵遣騎將出,夾城東西牽製之,親出州城西門,分三道以進,賊望風潰,乃募勇力齎薪炮,焚其樓櫓十餘。賊自平山堂麾騎下救,道遇於俊軍而歸。



 始,全反計雖成,然多顧忌,且懼其黨
 不皆從逆。邊陲好進喜事者,欲挾賊為重,或陰讚之,謂激作愈甚,朝廷愈畏,則錢糧愈增,又許身任調停之責。故全兵將舉而張國明先召,全之托詞陳遇棄城,及歸過三趙圖己,蓋成謀也。及三趙用,宋師集,諸閫易,國明沮,削全官爵,罷支錢糧,攻城不得,欲戰不利,全始自悔,忽忽不樂。或令左右抱其臂曰:「是我手否?」人皆怪之。



 時正月望,城中放燈張樂,姑示整暇。全見之。亦往海陵載妓女,張燈平山堂,矯情自肆。是晚,燕大元宣差,宣差激
 全曰:「相公服飾器用多南方物,乃心終在南耳!」全乃取誥敕,朝服南向,曆述平生梗概,再拜褫服,焚之,歎曰:「國明誤我。」淚下如雨,抆淚就坐強歡。有朐山於道士者,老矣,全迎致之,初見全即歎曰:「我業債合在此償耶?」占事多驗,尊為軍師。及見全焚誥命,謂人曰:「相公死明日,我死今日矣!」人問之,曰:「朝廷以安撫、提刑討逆,然為逆者,節度使也。豈有安撫、提刑能擒節度使哉?誥敕既焚,則一賊爾。盜固安撫、提刑所得捕,不死何為!」入見全曰:「相
 公明日出帳門必死。」全怒以為厭己,斬之。



 範、葵夜議誥朝所向,葵曰:「東向利,不如出東門。」範曰:「西出嚐不利,賊必見易,因其所易而圖之,必勝。不如出堡塞西門。」壬寅,全置酒高會平山堂,有堡塞候卒識其槍垂雙拂為號,以報。範喜謂葵曰:「此賊勇而輕,若果出,必成擒矣。」乃悉精銳數千而西,取官軍素為賊所易者,張其旗幟以易之。全望見,喜謂宣差曰:「看我掃南軍。」官軍見賊突鬥而前,亦不知其為全也。範麾軍並進,葵親搏戰,諸軍爭奮。
 賊始疑非前日軍,欲走入土城,李虎軍已塞其甕門。全窘,從數十騎北走,葵率諸將以製勇、寧淮軍蹙之,賊趨新塘。新塘自決水後,淖深數尺,會久晴,浮戰塵如燥壤,全騎陷淖不能拔。製勇軍奮長槍三十餘亂刺之,全曰:「無殺我,我乃頭目。」先是,令諸陣上,眾獲頭目無得爭以為獻,故群卒碎其屍,而分其鞍馬器甲,並殺三十餘人,類非卒伍,俱不暇問。



 甲辰,賊軍全椒人周海請降,報全已殺,餘黨議潰去。未幾,聞安用歎恨飲泣,初議推一人
 為首,以竟其逆,莫肯相下,欲還淮安奉楊氏主之。範夜上捷書製置司,議翼日追賊。乙巳早,安用引五百騎徑南門趨灣頭,範伏弩射之,賊呼曰:「爾襄陽援兵已敗走,汝知之乎?」城中應曰:「汝李全已為戮,汝何不降?」賊不應,諸將欲追賊,範懼有伏兵,先分兵燒圍城樓櫓,夜半火光燭天,命東南諸門皆出兵,範、葵繼提精兵進。四鼓,賊大潰。丙午黎明,葵追及賊於灣頭,一戰又破之,俘斬及奪回糧畜蔽野。別將追至大儀,不及。葵使人瘞新塘骸
 骨,得左掌無一指,蓋全支解也。先是,全乞靈茅司徒廟無應,全怒,斷神像左臂。或夢神告曰:「全傷我,全死亦當如我。」至是果然。



 揚州平,善湘以露布上,帝驚喜,太后舉手加額。國明輩懼禍及己,唱論雲全未死,至有資遊士吳大理等助煽之。及泰州凱奏繼上,浮言始定。朝中皆擬隨表入賀,彌遠以小寇就平,謝止之。甲寅,善湘來犒師。二月,命胡穎部所獲賊酋二十人獻俘於朝,且定奇功二十有九人及其餘,促行賞;又遣趙楷往稟廟算。



 三
 月庚寅,祃祭,有梟鳴於牙,占之吉,別遣全子才率王旻等將萬五千人,與於玠掎角取鹽城。癸巳,步騎十萬發揚州,留勝權守。庚子,鹽城賊董友、王海以兵圍卞整砦,玠擊卻之。癸卯,遣總轄韓亮、戚永升率多槳船及民船四百入射陽湖,擊賊於諭口。丁未,亮破賊於崔溝。己酉,範、葵分兵進至平河橋,剿賊甚多。壬子,玠、整敗賊將王國興於岡門,斬首千級。四月丁巳,敗賊於十里亭,賊兵爭門,墜濠如蟻。庚申,別將範勝、趙興破賊砦於壽河,拔
 農民脅從者萬家。



 壬戌,範、葵遣諸軍薄淮安城下,賊大敗,死者萬餘,焚二千家,城中哭聲振天。甲子,子才自他道進攻,賊將董友拒之,大戰於港口,敗之。庚辰,舟師過漣水,戰勝,達淮安。五月丙戌朔,天大霧,官兵攻上城,賊守者尚臥,倉皇起鬥。官軍互踏肩為梯,前者或墜,後者繼至,自醜至未,五城俱破,斬首數千級,生擒數百人。兵士有故隸楚州左右軍者,家屬數為賊虐,至是泄憤,無老幼皆殺之,燒砦柵萬餘家,腥焰蔽天。餘寇爭橋入大
 城,重濠皆滿。淮北賊歸赴援,舟師又剿擊,焚其水柵,夷五城餘址,賊始懼。己亥,子才率趙必勝、王旻軍移砦西門,道遇賊大戰,至夜不解。子才為銳陣左右救,乃勝。



 楊氏諭鄭衍德等曰:「二十年梨花槍,天下無敵手,今事勢已去,撐拄不行。汝等未降者,以我在故爾。殺我而降,汝必不忍。若不圖我,人誰納降?今我欲歸老漣水,汝等宜告朝廷,本欲圖我來降,為我所覺,已驅之過淮矣。以此請降可乎?」眾曰:「諾。」翼日,楊氏絕淮而去。賊黨即遣偽計
 議馮垍、潘於款於軍門,範等密聞於朝,朝論不可,範曰:「若明諭朝旨,是堅賊誌,不如陽許以誤之,我自為必討之計。」乃遣範用吉入城諭賊曰:「朝廷已許納降,但令安撫交過北軍。」衍德等遣潘於隨用吉報謝,許獻玉帶、犒軍黃金四千兩。範曰:「我欲款賊,賊更來款我。」於歸,鄭衍德等自知降亦不免,始送款於金。至是,金遣其副統軍許奕、萬戶兀林答以其京東元帥牒來言曰:「此賊不降,能為兩國患,請與大國夾攻之,各勿受降。」範怪其來無
 故,而難於陰絕,遣王貴報之,不從其請。



 六月己未,大戰於河西三砦,賊大敗,楊氏歸漣水。壬戌,賊先遣妻孥過淮,軍爭欲往,斬之不能禁,反有起殺頭目者。甲子,復大戰,淮安遂平。議乘勝復淮陰,兵未行,淮陰降金。繼得探報云:宋師遲一宿攻城,淮安亦為金有矣。於是全所據州悉平。楊氏竄歸山東,又數年而後斃。



 全之寇泰州,官屬十有九人皆迎降,獨教授高夢月不汙,詔贈三官。



 全子壇。



\end{pinyinscope}