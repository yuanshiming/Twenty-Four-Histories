\article{志第六十八 禮十八(嘉禮六)}

\begin{pinyinscope}

 皇太子冠禮皇子附公主笄禮公主下降儀宗室附親王納妃品官婚禮士庶人婚禮



 皇太子冠儀,嘗行於大中祥符之八年。徽宗親制《冠禮
 沿革》十一卷,命儀禮局仿以編次。



 其儀:前期奏告天地、宗廟、神稷、諸陵、宮觀。殿中監帥尚舍張設垂拱、文德殿門之內,設香案殿下螭陛間,又為房於東朵殿。大晟展宮架樂於橫街南,太常設太子冠席東階上、東宮官位於後,設褥位,陳服於席南,東領北上。遠游冠簪導、袞冕簪導同箱,在服南。設罍洗阼階東,罍在洗東,篚在洗西,實巾一,加勺冪。光祿設醴席西階上,南面,實側尊在席南。又設饌於席,加冪。執事者並公服,立罍洗酒饌之所。
 九旒冕、遠游冠、折上巾各一匴,奉禮郎三人執以侍於東階之東、西北上。設典儀位於宮架東北,贊者二人在南,西向。



 禮直官、通事舍人、太常博士引太子詣朵殿東房。皇帝乘輦,駐垂拱殿,百官起居如月朔視朝儀。左輔版奏中嚴,內外符寶郎奉寶先出;左輔奏外辦,皇帝服通天冠、絳紗袍詣文德殿,簾卷。大樂正令撞黃鐘之鐘,右五鐘皆應。殿上鳴鞭,皇帝出西閣乘輦,協律郎俯伏,跪,舉麾,興,工鼓柷,奏《乾安》之樂,殿上扇合。禮直官、太常
 博士引禮儀使導皇帝出,降輦即坐,簾卷扇開,鞭鳴樂止,爐煙升。符寶郎奉寶陳於御坐左右,禮直官、通事舍人、太常博士引掌冠、贊冠者入門,《肅安》之樂作,至位,樂止。典儀曰「再拜」,在位者皆再拜。左輔詣御坐前,承制降東階,詣掌冠者前西向稱有制,典儀贊在位官再拜訖,宣制曰:「皇太子冠,命卿等行禮。」掌冠、贊冠者再拜訖,文臣侍從官、宗室、武臣節度使以上升殿,東西立,應行禮官詣東階下立。



 東宮官入,詣太子東房,次禮直官等引
 太子,內侍二人夾侍,東宮官後從,《欽安》之樂作,即席西向坐,樂止。引掌冠、贊冠者以次詣罍洗,樂作,搢笏,盥帨訖,出笏,升,樂止。執折上巾者升,掌冠者降一等受之,右執項,左執前,進皇太子席前,北向立,祝曰:「咨爾元子,肇冠於阼。筮日擇賓,德成禮具。於萬斯年,承天之祜。」乃跪冠,《順安》之樂作,掌冠者興,



 席南北面立,後準此。



 贊冠者進席前,北面跪正冠,興,立於掌冠者之後。太子興,內侍跪進服,服訖,樂止。



 掌冠者揖太子復坐,禮直官等引掌冠
 者降詣罍洗,如上儀。贊冠者進席前,北向跪,脫折上巾置於匴,興,內侍跪受,興,置於席。執遠游冠者升,掌冠者降二等受之,右執項,左執前,進太子席前,北向立,祝曰:「爰即令辰,申加元服。崇學以讓,三善皆得。副予一人,受天百福。」乃跪冠,《懿安》之樂作,掌冠者興。贊冠者進,跪簪結紘,興。太子興,內侍跪進服,服訖,樂止。



 掌冠者揖太子復坐,掌冠者降詣罍洗,及贊冠者跪,脫遠游冠,並如上儀。執袞冕者升,掌冠者降三等受之,右執項,左執前,進
 太子席前,北向立,祝曰:「三加彌尊,國本以正。無疆惟休,有室大競。懋昭厥德,保茲永命。」乃跪冠,《成安》之樂作。掌冠者興。贊冠者如上儀,跪簪結紘。內侍進服,服訖,樂止。禮直官等引太子降自東階,樂作,由西階升,即醴席南向坐,樂止。又引掌冠者詣罍洗,樂作,盥帨訖,升西階,樂止。贊冠者跪取爵,內侍注酒,掌冠者受爵,跪進太子席前,北向立,祝曰:「旨酒嘉薦,有飶其香。拜受祭之,以定爾祥。令德壽豈,日進無疆。」太子搢圭,跪受爵,《正安》之樂作,
 飲訖,奠爵執圭。太官令設饌席前,太子搢圭,食訖,樂止,執圭興,太官令徹饌、爵。



 禮直官等引自西階詣東房,易朝服,降立橫街,南北向,東宮官復位,西向。太子初行,樂作,至位,樂止。



 禮直官等引掌冠、贊冠者詣前,西向,掌冠者少進,字之曰:「始生而名,為實之賓。既冠而字,以益厥文。永受保之,承天之慶。奉敕字某。」太常博士請再拜,太子再拜訖,搢笏,舞蹈,再拜,奏聖躬萬福,又再拜。左輔承旨,降自東階,詣太子前,西向,宣曰「有敕」,太子再拜,宣敕曰:「事親以孝,
 接下以仁。遠佞近義,祿賢使能。古訓是式,大猷是經。」宣訖,太子再拜訖,禮直官等引太子前,俯伏,跪,奏稱:「臣雖不敏,敢不祗奉!」奏訖,興,復位,再拜訖,引出殿門,樂作,出門,樂止。侍立官並降復位,典儀曰「拜」,贊者承傳,在位者皆再拜。禮儀使奏禮畢,鳴鞭。大樂正令撞蕤賓之鐘,左五鐘皆應,《乾安》之樂作,皇帝降坐,左輔奏解嚴,放仗,在位官皆再拜,退。



 太子入內,朝見皇后,如宮中儀。乃擇日謁太廟、別廟,宿齋於本宮。質明,服遠游冠、朱明衣,乘金
 輅。至廟,改服袞冕,執圭行禮,群臣稱賀,皇帝賜酒三行。



 皇子冠,前期擇日奏告景靈宮,太常設皇子冠席文德殿東階上,稍北東向,設褥席,陳服於席南,東領北上。九旒冕服、七梁進賢冠服、折上巾公服、七梁冠簪導、九旒冕簪導同箱,在服南。設罍洗、酒饌、旒冕、冠、巾及執事者,並如皇太子儀。



 其日質明,皇帝通天冠、絳紗袍,御文德殿。皇子自東房出,內侍二人夾侍,王府官從,《恭安》之樂作,即席南向坐,樂止。掌冠者進折上巾,北向跪冠,《修安》
 之樂作;贊冠者進,北面跪正冠,皇子興,內侍跪進服訖,樂止。掌冠者揖皇子復坐,以爵跪進,祝曰:「酒醴和旨,籩豆靜嘉。授爾元服,兄弟具來。永言保之,降福孔皆。」皇子搢笏,跪受爵,《翼安》之樂作,飲訖,太官令進饌訖。再加七梁冠,《進安》之樂作。掌冠者進爵,祝曰:「賓贊既戒,肴核惟旅。申加厥服,禮儀有序。允觀爾成,承天之祜。」皇子跪受爵,《輔安》之樂作,太官奉饌。三加九旒冕,《廣安》之樂作。掌冠者進爵,祝曰:「旨酒嘉慄,甘薦令芳。三加爾服,眉壽無
 疆。永承天休,俾熾而昌。」皇子跪受爵,《賢安》之樂作,太官奉饌,饌徹。



 皇子降,易朝服,立橫階南,北向位,掌冠者字之曰:「歲日云吉,威儀孔時。昭告厥字,君子攸宜。順爾成德,永受保之。奉敕字某。」皇子再拜舞蹈,又再拜,奏聖躬萬福,又再拜。左輔宣敕,戒曰:「好禮樂善,服儒講藝。蕃我王室,友於兄弟。不溢不驕,惟以守之。」皇子再拜,進前俯伏,跪稱:「臣雖不敏,敢不祗奉!」俯伏,興,復位,再拜,出。殿上侍立官並降,復位,再拜,放仗。明日,百僚詣東上閣門賀。



 公主笄禮。年十五,雖未議下嫁,亦笄。笄之日,設香案於殿庭;設冠席於東房外,坐東向西;設醴席於西階上,坐西向東;設席位於冠席南,西向。其裙背、大袖長裙、褕翟之衣,各設於椸,陳下庭;冠笄、冠朵、九翬四鳳冠,各置於盤,蒙以帕。首飾隨之,陳於服椸之南,執事者三人掌之。櫛總置於東房。內執事宮嬪盛服旁立,俟樂作,奏請皇帝升御坐,樂止。



 提舉官奏曰:「公主行笄禮。」樂作,贊者引公主入東房。次行尊者為之總髻畢,出,即席西向坐。次
 引掌冠者東房,西向立,執事奉冠笄以進,掌冠者進前一步受之,進公主席前,北向立,樂止,祝曰:「令月吉日,始加元服。棄爾幼志,順爾成德。壽考綿鴻,以介景福。」祝畢,樂作,東向冠之,冠畢,席南北向立;贊冠者為之正冠,施首飾畢,揖公主適房,樂止。執事者奉裙背入,服畢,樂作,公主就醴席,掌冠者揖公主坐。贊冠者執酒器,執事者酌酒,授於掌冠者執酒,北向立,樂止,祝曰:「酒醴和旨,籩豆靜嘉。受爾元服,兄弟具來。與國同休,降福孔皆。」祝畢,
 樂作,進酒,公主飲畢,贊冠者受酒器,執事者奉饌,食訖,徹饌。



 復引公主至冠席坐,樂止。贊冠者至席前,贊冠者脫冠置於盤,執事者徹去,樂作。執事者奉冠以進,掌冠者進前二步受之,進公主席前,北向立,樂止,祝曰:「吉月令辰,乃申爾服,飾以威儀,淑謹爾德。眉壽永年,享受遐福。」祝畢,樂作,東向冠之,冠畢,席南北向立。贊冠者為之正冠,施首飾畢,揖公主適房,樂止。執事奉大袖長裙入,服畢,樂作。公主至醴席,掌冠者揖公主坐。贊冠者執酒
 器,執事者酌酒,授於掌冠者執酒,北向立,樂止,祝曰:「賓贊既戒,肴核惟旅。申加爾服,禮儀有序。允觀爾成,永天之祜。」祝畢,樂作,進酒,公主飲畢,贊冠者受酒器,執事者奉饌食訖,徹饌。



 復引公主至冠席坐,樂作。贊冠者至席前,贊冠者脫冠置於盤,執事者徹去,樂作。執事奉九翬四鳳冠以進,掌冠者進前三步受之,進公主席前,向北而立,樂止,祝曰:「以歲之吉,以月之令,三加爾服,保茲永命。以終厥德,受天之慶。」祝畢,樂作,東向冠之,冠畢,席南
 北向立。贊冠者為之正冠、施首飾畢,揖公主適房,樂止。執事者奉褕翟之衣入,服畢,樂作,公主至醴席,掌冠者揖公主坐。贊冠者執酒器,執事者酌酒,授於掌冠者執酒,北向立,樂止,祝曰:「旨酒嘉薦,有飶其香。咸加爾服,眉壽無疆。永承天休,俾熾而昌。」祝畢,樂作,進酒,公主飲畢,贊冠者受酒器。執事者奉饌,食訖,徹饌。



 復引公主至席位立,樂止,掌冠者詣前相對,致辭曰:「歲日具吉,威儀孔時。昭告厥字,令德攸宜。表爾淑美,永保受之。可字曰某。」
 辭訖,樂作,掌冠者退。引公主至君父之前,樂止,再拜起居,謝恩再拜。少俟,提舉進御坐前承旨訖,公主再拜。提舉乃宣訓辭曰:「事親以孝,接下以慈。和柔正順,恭儉謙儀。不溢不驕,毋詖毋欺。古訓是式,爾其守之。」宣訖,公主再拜,前奏曰:「兒雖不敏,敢不祗承!」歸位再拜,見後母之禮如之。



 禮畢,公主復坐,皇后稱賀,次妃嬪稱賀,次掌冠、贊冠者謝恩,次提舉眾內臣稱賀,其餘班次稱賀,並依例程。禮畢,樂作;駕興,樂止。



 公主下降。初被選尚者即拜駙馬都尉,賜玉帶、襲衣、銀鞍勒馬、採羅百匹,謂之系親。又賜辦財銀萬兩,進財之數,倍於親王聘禮。出降,賜甲第。餘如諸王夫人之制。掌扇加四,引障花、燭籠各加十,皆行舅姑之禮。諸親遞加賜賚。其縣主系親以金帶,賜辦財銀五千兩,納財賜賚,大率三分減其二。宗室女特封郡君者,又差降焉。



 嘉祐初,禮官言:「禮閣新儀,公主出降前一日,行五禮。古者,結婚始用行人,告以夫家採擇之意,謂之納採。問女之名,
 歸卜夫廟,吉,以告女家,謂之問名、納吉。今選尚一出朝廷,不待納採。公主封爵已行誕告,不待問名。若納成則既有進財,請期則有司擇日。宜稍依五禮之名,存其物數,俾知婚姻之事重、而夫婦之際嚴如此,亦不忘古禮之義也。」時兗國公主下嫁李瑋,詔賜出降日,令夫家主婚者具合用雁、幣、玉、馬等物,陳於內東門外,以授內謁者,進入內侍掌事者受,唯馬不入。



 神宗即位,詔以「昔侍先帝,恭聞德音,以舊制士大夫之子有尚帝女者,輒皆
 升行,以避舅姑之尊。豈可以富貴之故,屈人倫長幼之序。宜詔有司革之,以厲風俗。」於是著為令。仍命陳國長公主行舅姑之禮,駙馬都尉王師約更不升行。公主見舅姑行禮自此始。舊例,長公主凡有表章不稱妾,禮院議謂:「男子、婦人,凡於所尊稱臣若妾,義實相對。今宗室伯叔近臣悉皆稱臣,即公主理宜稱妾。況家人之禮,難施於朝廷。請自大長公主而下,凡上箋表,各據國封稱妾。」從王師約之請也。



 康國公主下降,太常寺言:「按令,公
 主出降,申中書省,請皇后帥宮闈掌事人送至第外,命婦從,今請如令。」詔:「出降日,婉儀帥宮闈掌事者送至第外,命婦免從。」



 徽宗改公主為姬,下詔曰:「在熙寧初,有詔厘改公主、郡主、縣主名稱,當時群臣不克奉承。近命有司稽考前世,周稱『王姬』,見於《詩·雅》。『姬』雖周姓,考古立制,宜莫如周。可改公主為帝姬、郡主為宗姬、縣主為族姬。其稱大長者,為大長帝姬,仍以美名二字易其國號,內兩國者以四字。」



 其出降日,婿家具五禮,修表如上儀。太
 史局擇日告廟。



 親迎。前一日,所司於內東門外量地之宜,西向設婿次。其日,婿父醮子如上儀。乃命之曰:「往迎肅雍,以昭惠宗祏。」子再拜,曰:「祗率嚴命!」又再拜,降,出乘馬,至東華門內下馬,禮直官引就次。有司陳帝姬鹵簿、儀仗於內東門外,候將升厭翟車,引婿出次於內東門外,躬身西向。掌事者執雁,內謁者奉雁以進,俟帝姬升車,婿再拜,先還第。



 同牢。其日初昏,掌事者設巾、洗各二於東階東南,一於室北。水在洗東,尊於室中,實四爵、兩巹
 於篚。婿至本第,下馬以俟。帝姬至,降車,贊者引婿揖之以入,及寢門又揖,導之升階,入室盥洗。掌事者布對位,又揖帝姬,皆即坐受盞三飲,俱興,再拜,贊者徹酒。



 見舅姑。夙興,帝姬著花釵、服褕翟以俟見。贊者設舅姑位於堂上,舅位於東,姑位於西,各服其服就位。女相者引帝姬升自西階,詣舅位前再拜,贊者以棗慄授帝姬奉置舅位前,舅即坐,贊者進徹以東,帝姬退,復位,又再拜。女相者引詣姑位前再拜,贊者以腶修授帝姬奉置姑位
 前,姑即坐,贊者亦徹以東,帝姬退,復位,又再拜。次醴婦、盥饋、饗婦如儀。



 諸王納妃。宋朝之制,諸王聘禮,賜女家白金萬兩。敲門,即古之納採。


用羊二十口、酒二十壺、彩四十匹。定禮,羊、酒、彩各加十,茗百斤,頭
 \gezhu{
  須巾}
 巾段、綾、絹三十匹,黃金釵釧四雙,條脫一副,真珠虎珀瓔珞、真珠翠毛玉釵朵各二副,銷金生色衣各一襲,金塗銀合二,錦繡綾羅三百匹,果盤、花粉、花冪、眠羊臥鹿花餅、銀勝、小色金銀錢等物。納財,
 用金器百兩、彩千匹、錢五十萬、錦綺、綾、羅、絹各三百匹,銷金繡畫衣十襲,真珠翠毛玉釵朵各三副,函書一架纏束帛,押馬函馬二十匹,羊五十口,酒五十壺,系羊酒紅絹百匹,花粉、花冪、果盤、銀勝、羅勝等物。親迎,用塗金銀裝肩輿一,行障、坐障各一,方團掌扇四,引障花十樹,生色燭籠十,高髻釵插並童子八人騎分左右導扇輿。其宗室子聘禮,賜女家白金五千兩。其敲門、定禮、納財、親迎禮皆減半,遠屬族卑者又減之。



 政和三年四月,議
 禮局上皇子納夫人儀:



 採擇。使者曰:「奉制,某王之儷,屬子懿淑。謹之重之,使某行採擇之禮。」儐者入告,主人曰:「臣某之子顓愚,不足以備採擇,恭承制命,臣某不敢辭。」



 問名。使者曰:「某王之儷,採擇既諧。將加官占,奉制問名。」儐者入告,主人曰:「制以臣某之子,可以奉侍某王,臣某不敢辭。」



 告吉。使者曰:「官占既吉,奉制以告。」儐者入告,主人曰:「臣某之子,愚弗克堪。占貺之吉,臣與有幸。臣某謹奉典制。」



 告成。使者曰:「官占云吉,嘉偶既定,制使某以儀
 物告成。」儐者入告,主人曰:「奉制賜臣以重禮,臣某謹奉典制。」



 告期。使者曰:「涓辰之良,某月某日吉,制使某告期。」儐者入告,主人曰:「臣某謹奉典制。」



 前期,太史局擇日,奏告景靈宮。



 賜告。前一日,主人設使者次,如常儀,使者以內侍為之。



 又設告箱之次於中門外,北向,隨闕所向,設香案於寢庭。其日大昕,使者公服至,主人出迎於大門外,北向再拜,使者不答拜。謁者引使者入門而左,主人入門而右,舉告箱者同入。主人立香案左,使者在右,舉告箱者
 以告置於香案。女相者引夫人出,面闕立,使者稱有制,女相者贊再拜,使者曰:「賜某國夫人告。」又贊再拜,退,使者出。



 皇帝醮戒於所御之殿,皇子乘象輅親迎。同牢、夫人朝見、盥饋、皇帝皇后饗夫人如儀。



 其諸王以下:



 納採。賓曰:「某官以伉儷之重,施於某王,某官謂主人,某王謂婿。



 某王率循彞典,以某將事,敢請納採。」某王謂婿父,某謂賓。



 儐者入告,主人曰:「某之子弗閑於姆訓,維是腶修、棗慄之饋,未知所以告虔也。某聽命於廟,敢不拜嘉。」



 問名。賓曰:「合二姓之好,
 必稽諸龜筮,敢請問名。」儐者入告,主人曰:「某王恭謹,重正昏禮,將以加諸卜,某敢不以告。」



 納吉。賓曰:「某王承嘉命,稽諸卜筮,龜筮協從,使某以告。」儐者入告,主人曰:「某王不忘寒素,欲施德於某未教之女,而卜以吉告,其曷敢辭。」



 納成。賓曰:「某官以伉儷之重,施於某王,某王,上謂婿,下謂婿父。



 率循彞典,有不腆之幣,以某將事,敢請納成。」儐者入告,主人曰:「某王順彞典,申之以備物,某敢不重拜嘉。」



 請期。賓曰:「某王謹重嘉禮,將卜諸近日,使某請期。」儐者
 入告,主人再辭。儐者出告,賓曰:「某既不獲受命於某官,某王得吉卜曰某日,敢不以告。」儐者入告,主人曰:「謹奉命以從。」



 親迎。前一日,主人設賓次,賓謂婿。



 如常儀。其日大昕,婿之父服其服,告於檷廟,無廟者設神位於廳東,不應設位者不設。



 子將行,父醮之於廳事。贊者設父位中間,南向,設子位父位之西,近南,東向。父即坐,子公服升自西階,進立位前。贊者注酒於盞,西向授子,子再拜,跪受,贊者又設饌父位前,子舉酒興,即坐飲食訖,降,再拜,進立於父位前。命之
 曰:「躬迎嘉偶,厘爾內治。」子再拜,曰:「敢不奉命。」又再拜,降出,詣女家。主人服其服,告於檷廟,如請期之儀。賓將至,主人設神位於寢戶外之西,設醴女位於戶內,南向,具酒饌。賓至,贊者引就次,女盛服於房中,就位南向立,姆位於右,從者陪其後。父公服升自東階,立於寢戶外之東,西向。內贊者設酒饌,女就位坐,飲食訖,降,再拜,內贊者徹酒饌。主人降立東階東南,西面,贊者引賓出次,立於門西,東面,儐者進受命,出請事,賓曰:「某受命於父,以
 茲嘉禮,躬聽成命。」儐者入告,主人曰:「某固願從命。」儐者出告訖,入引主人迎賓大門外之東,西面揖賓,賓報揖。主人入門而右,賓入門而左,執雁者從入,陳雁於庭,三分庭,一在南,北向。主人升立於東階上,西面;賓升西階進,當寢戶前,北面再拜,降出,主人不降送。賓初入門,母出,立於寢戶外之西,南面,賓拜訖,姆引女出於母左,父命之曰:「往之汝家,以順為正,無忘肅恭!」母戒之曰:「必恭必戒,無違舅姑之命!」庶母申之曰:「爾誠聽於訓言,毋作
 父母羞!」女出門,婿先還第。



 其同牢、廟見、見舅姑諸禮,皆如儀。



 凡宗室婚姻,治平中,宗正司言:「宗室女舅姑、夫族未立儀制,皆當創法。」詔:「婿家有二世食祿,即許娶宗室女,未仕者與判、司、簿、尉,已任者隨資序推恩。即婿別祖、女別房,舊為婚姻而於今卑尊不順者,皆許。婿之三代、鄉貫、生月、人材書札,止令婚主問驗,以告宗正寺、大宗正司,寺、司詳視,如條保明。所進財皆賜婿家,令止於本宮納財,媒妁、使令之非理求丐,許告。宗室女事舅姑
 及見夫之族親,皆如臣庶之家。」其後又令宗室女再嫁者,祖、父有二代任殿直若州縣官已上,即許為婚姻。



 熙寧十年,又詔:「應袒免以上親不得與雜類之家婚嫁,謂舅嘗為僕、姑嘗為娼者。若父母系化外及見居沿邊兩屬之人,其子孫亦不許為婚。緦麻以上親不得與諸司胥吏出職、納粟得官及進納伎術、工商、雜類、惡逆之家子孫通婚。後又禁刑徒人子孫為婚。



 應婚嫁者委主婚宗室,擇三代有任州縣官或殿直以上者,列姓名、家世、州里、歲數奏
 上,宗正司驗實召保,付內侍省宣系,聽期而行。嫁女則令其婿召保。其冒妄成婚者,以違制論。主婚宗室與媒保同坐,不以赦降,自首者減罪,告者有賞。非袒免親者依庶姓法。宗室離婚,委宗正司審察,若於律有可出之實或不相安,方聽。若無故捃拾者,劾奏。如許聽離,追完賜予物,給還嫁資。再娶者不給賜。非袒免以上親與夫聽離,再嫁者委宗正司審核。其恩澤已追奪而乞與後夫者,降一等。」尋詔:「宗女毋得與嘗娶人結婚,再適者不
 用此法。」



 品官婚禮。納採、問名、納吉、納成、請期、親迎、同牢、廟見、見舅姑、姑醴婦、盥饋、饗婦、送者,並如諸王以下婚。四品以下不用盥饋、饗婦禮。



 士庶人婚禮。並問名於納採,並請期於納成。其無雁奠者,三舍生聽用羊,庶人聽以雉及雞鶩代。其辭稱「吾子」。



 親迎。質明,掌事者設檷位廳事東間,南向。婿之父服其服,北面再拜,祝曰:「某子某,年若干,禮宜有室,聘某氏第
 幾女,以某日親迎,敢告。」子將行,父坐廳事,南向,子服其服,三舍生及品官子孫假九品服,餘並皂衫衣、折上巾。



 立父位西,少南,東向。贊者注酒於盞授子,子再拜,跪受,贊者又以饌設位前,子舉酒興,即坐飲食訖,降,再拜,進立父位前,命之曰:「厘爾內治,往求爾匹。」子再拜,曰:「敢不奉命。」又再拜,降出。



 初婚,掌事者設酒饌室中,置二盞於盤,婿服其服如前服,至女家,贊者引就次,掌事者設檷位,主人受禮,如請期之儀。主人謂女父。



 女盛服立房中,父升階立房外之東,西向。非南向者,
 各隨其所向。父立於門外之左,餘放此。



 贊者注酒於盞授女,女再拜受盞;贊者又以饌設於位前,女即坐飲食訖,降,再拜。父降立東階下,賓出次,賓謂婿。



 主人迎於門,揖賓入,賓報揖,從入。主人升東階,西面;賓升西階,進當房戶前,北面。掌事者陳雁於階,賓曰:「某受命於父,以茲嘉禮,躬聽成命。」主人曰:「某固願從命。」賓再拜,降出,主人不降送。初,女出,父戒之曰:「往之汝家,無忘肅恭!」母戒之曰:「夙夜以思,無有違命!」諸母申之曰:「無違爾父母之訓!」女出,婿先還,俟於門
 外。婦至,贊者引就北面立,婿南面,揖以入,至於室。掌事者設對位室中,婿婦皆即坐,贊者注酒於盞授婿及婦,婿及婦受盞飲訖。遂設饌,再飲、三飲,並如上儀。婿及婦皆興,再拜,贊者徹酒饌。



 見祖檷、見舅姑、醴婦、饗送者,如儀。



\end{pinyinscope}