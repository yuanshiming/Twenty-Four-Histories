\article{列傳第一百八十五循吏}

\begin{pinyinscope}

 ○陳靖張綸邵曄崔立魯有開張逸吳遵路趙尚寬高賦程師孟韓晉卿葉康直



 宋法有可以得循吏者三:太祖之世,牧守令錄,躬自召見,問以政事,然後遣行,簡擇之道精矣;監司察郡守,郡守察縣令,各以時上其殿最,又命朝臣專督治之,考課之方密矣;吏犯贓遇赦不原,防閑之令嚴矣。



 承平之世,州縣吏謹守法度以修其職業者,實多其人。其間必有絕異之績,然後別於賞令,或自州縣善最,他日遂為名臣,則撫子之長又不足以盡其平生,故始終三百餘年,循吏載諸簡策者十二人。作《循吏傳》。



 陳靖,字道卿,興化軍莆田人。好學,頗通古今。父仁壁,仕陳洪進為泉州別駕。洪進稱臣,豪猾有負險為亂者,靖徒步謁轉運使楊克巽,陳討賊策。召還,授陽翟縣主簿。契丹犯邊,王師數不利,靖遣從子上書,求入奏機略。詔就問之,上五策,曰:明賞罰;撫士眾;持重示弱,待利而舉;帥府許自闢士;而將帥得專制境外。太宗異之,改將作監丞,未幾,為御史臺推勘官。



 時御試進士,多擢文先就者為高等,士皆習浮華,尚敏速。靖請以文付考官第甲
 乙,俟唱名,或果知名士,即置上科。喪父,起復秘書丞,直史館,判三司開拆司。淳化四年,使高麗還,提點在京百司,遷太常博士。



 太宗務興農事,詔有司議均田法,靖議曰:「法未易遽行也。宜先命大臣或三司使為租庸使,或兼屯田制置,仍擇三司判官選通知民事者二人為之貳。兩京東西千里,檢責荒地及逃民產籍之,募耕作,賜耕者室廬、牛犁、種食,不足則給以庫錢。別其課為十分,責州縣勸課,給印紙書之。分殿最為三等:凡縣管墾田,
 一歲得課三分,二歲六分,三歲九分,為下最;一歲四分,二歲七分,三歲至十分,為中最;一歲五分,未及三歲盈十分者,為上最。其最者,令佐免選或超資;殿者,即增選降資。每州通以諸縣田為十分,視殿最行賞罰。候數歲,盡罷官屯田,悉用賦民,然後量人授田,度地均稅,約井田之制,為定以法,頒行四方,不過如此矣。」太宗謂呂端曰:「朕欲復井田,顧未能也,靖此策合朕意。」乃召見,賜食遣之。



 他日,帝又語端。曰:「靖說雖是,第田未必墾,課未必
 入,請下三司雜議。」於是詔鹽鐵使陳恕等各選判官二人與靖議,以靖為京西勸農使,命大理寺丞皇甫選、光祿寺丞何亮副之。選等言其功難成,帝猶謂不然。既而靖欲假緡錢二萬試行之,陳恕等言:「錢一出,後不能償,則民受害矣。」帝以群議終不同,始罷之,出靖知婺州,再遷尚書刑部員外郎。



 真宗即位,復列前所論勸農事,又言:「國家御戎西北,而仰食東南,東南食不足,則誤國大計。請自京東、西及河北諸州大行勸農之法,以殿最州縣官
 吏,歲可省江、淮漕百餘萬。」復詔靖條上之,靖請刺史行春,縣令勸耕,孝悌力田者賜爵,置五保以檢察奸盜,籍游惰之民以供役作。又下三司議,皆不果行。



 歷度支判官,為京畿均田使,出為淮南轉運副使兼發運司公事,徙江南轉運使。極論前李氏橫賦於民凡十七事,詔為罷其尤甚者。徙知譚州,歷度支、鹽鐵判官。祀汾陰,為行在三司判官。又歷京西、京東轉運使,知泉、蘇、越三州,累遷太常少卿,進太僕卿、集賢院學士,知建州,徙泉州,拜
 左諫訴大夫。初,靖與丁謂善,謂貶,黨人皆逐去,提點刑獄、侍御史王耿乃言靖老疾,不宜久為鄉里官,於是以秘書監致仕,卒。



 靖平生多建畫,而於農事尤詳,嘗取淳化、咸平以來所陳表章,目曰《勸農奏議》,錄上之,然其說泥古,多不可行。



 張綸,字公信,潁州汝陰人。少倜儻任氣。舉進士不中,補三班奉職,遷右班殿直。從雷有終討王均於蜀,有降寇數百據險叛,使綸擊之,綸馳報曰:「此窮寇,急之則生患,
 不如諭以向背。」有終用其說,賊果棄兵來降。以功遷右侍禁、慶州兵馬監押,擢閣門祗候,益、彭、簡等州都巡檢使。所部卒縱酒掠居民,綸斬首惡數人,眾乃定。徙荊湖提點刑獄,遷東頭供奉官、提點開封府界縣鎮公事。



 奉使靈夏還,會辰州溪峒彭氏蠻內寇,以知辰州。綸至,築蓬山驛路,賊不得通,乃遁去。徙知渭州。改內殿崇班、知鎮戎軍。奉使契丹,安撫使曹瑋表留之,不可。蠻復入寇,為辰州、澧鼎等州緣邊五溪十峒巡檢安撫使,諭蠻酋
 禍福,購還所掠民,遣官與盟,刻石於境上。



 久之,除江、淮制置發運副使。時鹽課大虧,乃奏除通、泰、楚三州鹽戶宿負,官助其器用,鹽入優與之直,由是歲增課數十萬石。復置鹽場於杭、秀、海三州,歲入課又百五十萬。居二歲,增上供米八十萬。疏五渠,導太湖入於海,復租米六十萬。開長蘆西河以避覆舟之患,又築漕河堤二百里於高郵北,旁錮鉅石為十䃮,以洩橫流。泰州有捍海堰,延袤百五十里,久廢不治,歲患海濤冒民田。綸方議修
 復,論者難之,以為濤患息而畜潦之患興矣。綸曰:「濤之患十九,而潦之患十一,獲多而亡少,豈不可邪?」表三請,願身自臨役。命兼權知泰州,卒成堰,復逋戶二千六百,州民利之,為立生祠。



 居淮南六年,累遷文思使、昭州刺史。契丹隆緒死,為吊慰副使。歷知秦、瀛二州,兩知滄州,再遷東上閣門使,真拜乾州刺史,徙知潁州,卒。綸有材略,所至興利除害。為人恕,喜施予,在江、淮,見漕卒凍餒道死者眾,嘆曰:「此有司之過,非所以體上仁也。」推奉錢
 市絮襦千數,衣其不能自存者。



 邵曄,字日華,其先京兆人。唐末喪亂,曾祖岳挈族之荊南謁高季興,不見禮,遂之湖南。彭玗刺全州,闢為判官。會賊魯仁恭寇連州,即署嶽國子司業、知州事,遂家桂陽。祖崇德,道州錄事參軍。父簡,連山令。



 曄幼嗜學,恥從闢署。太平興國八年,擢進士第,解褐,授邵陽主簿,改大理評事、知蓬州錄事參軍。時太子中舍楊全知州,性悍率蒙昧,部民張道豐等三人被誣為劫盜,悉置於死,獄
 已具,曄察其枉,不署牘,白全當核其實。全不聽,引道豐等抵法,號呼不服,再系獄按驗。既而捕獲正盜,盜豐等遂得釋,全坐削籍為民。曄代還引對,太宗謂曰:「爾能活吾平民,深可嘉也。」賜錢五萬,下詔以全事戒諭天下。授曄光祿寺丞,使廣南採訪刑獄。俄通判荊南,賜緋魚。遷著作佐郎、知忠州。歷太常丞、江南轉運副使,改監察御史。以母老乞就養,得知朗州。入判三司磨勘司,遷工部員外郎、淮南轉運使。



 景德中,假光祿卿,充交址安撫國
 信使。會黎桓死,其子龍鉞嗣立,兄龍全率兵劫庫財而去,其弟龍廷殺鉞自立,龍廷兄明護率扶蘭砦兵攻戰。曄駐嶺表,以事上聞,改命為緣海安撫使,許以便宜設方略。曄貽書安南,諭朝廷威德,俾速定位。明護等即時聽命,奉龍廷主軍事。初,詔曄俟其事定,即以黎桓禮物改賜新帥。曄上言:「懷撫外夷,當示誠信,不若俟龍廷貢奉,別加封爵而寵賜之。」真宗甚嘉納。使還,改兵部員外郎,賜金紫。初受使,假官錢八十萬,市私覿物,及為安撫,
 已償其半,餘皆詔除之。嘗上《邕州至交州水陸路》及《宜州山川》等四圖,頗詳控制之要。



 俄判三司三勾院,坐所舉季隨犯贓,曄當削一官,上以其遠使之勤,止令停任。大中祥符初,起知兗州,表請東封,優詔答之。及遣王欽若、趙安仁經度封禪,仍判州事,就命曄為京東轉運使。封禪禮畢,超拜刑部郎中,復判三勾院,出為淮南、江、浙、荊湖制置發運使。四年,改右諫議大夫、知廣州。州城瀕海,每蕃舶至岸,常苦颶風,曄鑿內濠通舟,颶不能害。俄
 遘疾卒,年六十三。



 崔立,字本之,開封鄢陵人。祖周度,仕周為泰寧軍節度判官。慕容彥超叛,周度以大義責之,遂見殺。立中進士第。為果州團練推官,役兵輦官物,道險,乃率眾錢,傭舟載歸。知州姜從革論如率斂法,當斬三人,立曰:「此非私己,罪杖爾。」從革初不聽,卒論奏,詔如立議。真宗記之,特改大理寺丞,知安豐縣。大水壞期斯塘,立躬督繕治,逾月而成。進殿中丞,歷通判廣州、許州。



 會滑州塞決河,調
 民出芻楗,命立提舉受納。立計其用有餘,而下戶未輸者尚二百萬,悉奏弛之。知江陰軍,屬縣有利港久廢,立教民浚治,既成,溉田數千頃,及開橫河六十里,通運漕。累遷太常少卿,歷知棣、漢、相、潞、兗、鄆、涇七州。兗州歲大饑,募富人出穀十萬餘石振餓者,所全活者甚眾。



 立性淳謹,尤喜論事。大中祥符間,帝既封禪,士大夫爭奏上符瑞,獻贊頌,立獨言:「水發徐州,旱連江、淮,無為烈風,金陵火,天所以警驕惰、戒淫泆也,區區符瑞,尚何足為治
 道言哉?」前後上四十餘事。以右諫議大夫知耀州,改知濠州,遷給事中。告老,進尚書工部侍郎致仕,卒。識韓琦於布衣,以女妻之,人嘗服其鑒云。



 魯有開,字元翰,參知政事宗道從子也。好《禮》學,通《左氏春秋》。用宗道蔭,知韋城縣。曹、濮劇盜橫行旁縣間,聞其名不敢入境。知確山縣,大姓把持官政,有開治其最甚者,遂以無事。興廢陂,溉民田數千頃。富弼守蔡,薦之,以為有古循吏風。



 知金州,有蠱獄,當死者數十人,有開曰:「
 欲殺人,衷謀之足矣,安得若是眾邪?」訊之則誣。天方旱,獄白而雨。知南康軍,代還。熙寧行新法,王安石問江南如何,曰:「法新行,未見其患,當在異日也。」以所對乖異,出通判杭州。



 知衛州,水災,人乏食,擅貸常平錢粟與之,且奏乞蠲其息。徙冀州,增堤,或謂:「郡無水患,何以役為?」有開曰:「豫備不虞,古之善計也。」卒成之。明年河決,水果至,不能冒堤而止。朝廷遣使河北,民遮誦有開功狀,召為膳部郎中,元祐中,歷知信陽軍、洺滑州,復守冀,官至中
 大夫,卒。



 張逸,字大隱,鄭州滎陽人。進士及第,為試秘書省校書郎。知襄州鄧城縣,有能名。積州謝泌將薦逸,先設幾案,置章其上,望闕再拜曰:「老臣為朝廷得一良吏。」乃奏之。他日引對,真宗問所欲何官,逸對曰:「母老在家,願得近鄉一幕職官,歸奉甘旨足矣。」授澶州觀察推官,數日,以母喪去。服除,引對,帝又固問之,對曰:「願得京官。」特改大理寺丞。帝雅賢泌,再召問逸者,用泌薦也。



 知長水縣,時
 王嗣宗留守西京,厚遇之,及徙青神縣,貧不自給,嗣宗假奉半年使辦裝。既至縣,興學校,教生徒。後邑人陳希亮、楊異相繼登科,逸改其居曰桂枝裏。縣東南有松柏灘,夏秋暴漲多覆舟,逸禱江神,不逾月,灘為徙五里,時人異之。再遷太常博士、知尉氏縣。擢監察御史,提點益州路刑獄,開封府判官。使契丹,為兩浙轉運使。徙陜西,未赴,又徙河東,居數月,復徙陜西。以龍圖閣待制知梓州。



 累遷尚書兵部郎中,知開封府。有僧求內降免田稅,
 而逸固執不許。仁宗曰:「有司能守法,朕何憂也。」又言:「頃禁命婦干禁中恩,比來稍通女謁,願令官司糾劾。」從之。



 以樞密直學士知益州。逸凡四至蜀,諳其民風。華陽騶長殺人,誣道旁行者,縣吏受財,獄既具,乃使殺人者守囚。逸曰:「囚色冤,守者氣不直,豈守者殺人乎?」囚始敢言,而守者果服,立誅之,蜀人以為神。會歲旱,逸使作堰壅江水,溉民田,自出公租減價以振民。初,民饑多殺耕牛食之,犯者皆配關中。逸奏:「民殺牛以活將死之命,與盜
 殺者異,若不禁之,又將廢穡事。今歲少稔,請一切放還,復其業。」報可。未幾,卒於官。



 吳遵路,字安道。父淑,見《文苑傳》。第進士,累官至殿中丞,為秘閣校理。章獻太后稱制,政事得失,下莫敢言。遵路條奏十餘事,語皆切直,忤太后意,出知常州。嘗預市米吳中,以備歲儉,已而果大乏食,民賴以濟,自他州流至者亦全十八九。累遷尚書司封員外郎,權開封府推官,改三司鹽鐵判官,加直史館,為淮南轉運副使。會罷江、
 淮發運使,遂兼發運司事。嘗於真楚泰州、高郵軍置斗門十九,以畜洩水利。又廣屬郡常平倉儲畜至二百萬,以待兇歲。凡所規畫,後皆便之。



 遷工部郎中,坐失按蘄州王蒙正故入部吏死罪,降知洪州。徙廣州,辭不行。是時發運司既復置使,乃以為發運使,未至,召修起居注。元昊反,建請復民兵。除天章閣待制、河東路計置糧草。受詔料揀河東鄉民可為兵者,諸路視以為法。進兵部郎中、權知開封府,馭吏嚴肅,屬縣無追逮。



 時宋庠、鄭戩、
 葉清臣皆宰相呂夷簡所不悅,遵路與三人雅相厚善,夷簡忌之,出知宣州。上《御戎要略》、《邊防雜事》二十篇。徙陜西都轉運使,遷龍圖閣直學士、知永興軍,被病猶決事不輟,手自作奏。及卒,仁宗聞而悼之,詔遣官護喪還京師。



 遵路幼聰敏,既長,博學知大體。母喪,廬墓蔬食終制。性夷雅慎重,寡言笑,善筆札。其為政簡易不為聲威,立朝敢言,無所阿倚。平居廉儉無他好,既沒,室無長物,其友範仲淹分奉賙其家。



 子瑛,為尚書比部員外郎,不
 待老而歸。



 趙尚寬,字濟之,河南人,參知政事安仁子也。知平陽縣。鄰邑有大囚十數,破械夜逸,殺居民,將犯境,尚寬趣尉出捕,曰:「盜謂我不能來,方怠惰,易取也。宜亟往,毋使得散漫,且為害。」尉既出,又遣徼巡兵躡其後,悉獲之。



 知忠州,俗畜蠱殺人,尚寬揭方書市中,教人服藥,募索為蠱者窮治,置於理,大化其俗。轉運使持鹽數十萬斤,譚民易白金,期會促,尚寬發官帑所儲副其須,徐與民為市,
 不擾而集。



 嘉祐中,以考課第一知唐州。唐素沃壤,經五代亂,田不耕,土曠民稀,賦不足以充役,議者欲廢為邑。尚寬曰:「土曠可益墾闢,民稀可益招徠,何廢郡之有?」乃按視圖記,得漢召信臣陂渠故跡,益發卒復疏三陂一渠,溉田萬餘頃。又教民自為支渠數十,轉相浸灌。而四方之民來者云布,尚寬復請以荒田計口授之,及貸民官錢買耕牛。比三年,榛莽復為膏腴,增戶積萬餘。尚寬勤於農政,治有異等之效,三司使包拯與部使者交上
 其事,仁宗聞而嘉之,下詔褒焉,仍進秩賜金。留於唐凡五年,民像以祠,而王安石、蘇軾作《新田》、《新渠》詩以美之。



 徙同、宿二州,河中府神勇卒苦大校貪虐,刊匿名書告變,尚寬命焚之,曰:「妄言耳。」眾乃安。已而奏黜校,分士卒隸他營。又徙梓州。尚寬去唐數歲,田日加闢,戶日益眾,朝廷推功,自少府監以直龍圖閣知梓州。積官至司農卿,卒,詔賜錢五十萬。



 高賦子正臣,中山人。以父任為右班殿直。復舉進士,改
 奉禮郎,四遷太常博士。歷知真定縣,通判劍刑石州、成德軍。知衢州,俗尚巫鬼,民毛氏、柴氏二十餘家世蓄蠱毒,值閏歲,害人尤多,與人忿爭輒毒之。賦悉擒治伏辜,蠱患遂絕。



 徙唐州,州田經百年曠不耕,前守趙尚寬菑墾不遺力,而榛莽者尚多。賦繼其後,益募兩河流民,計口給田使耕,作陂堰四十四。再滿再留,比其去,田增闢三萬一千三百餘頃,戶增萬一千三百八十,歲益稅二萬二千二百五十七。璽書褒諭,宣布治狀以勸天下,兩
 州為生立祠。擢提點河東刑獄,又加直龍圖閣、知滄州。程昉欲於境內開西流河,繞州城而北注三塘泊。賦曰:「滄城近河,歲增堤防,猶懼奔溢,矧妄有開鑿乎?」昉執不從,後功竟不成。



 歷蔡、潞二州,入同判太常寺,進集賢院學士。在朝多所建明,嘗言:「二府大臣或僦舍委巷,散處京城,公私非便。宜仿前代丞相府,於端門前列置大第,俾居之。」又言:「仁宗朝為兗國公主治第,用錢數十萬緡。今有五大長公主,若悉如前比,其費無藝。願講求中制,
 裁為定式。」請諸道提點刑獄司置檢法官,庶專平讞,使民不冤。乞於禁中建閣,繪功臣像,如漢雲臺、唐凌煙之制。言多施行。以通議大夫致仕,退居襄陽,卒年八十四。



 程師孟,字公闢,吳人。進士甲科。累知南康軍、楚州,提點夔路刑獄。瀘戎數犯渝州邊,使者治所在萬州,相去遠,有警,率浹日乃至。師孟奏徙于渝。夔部無常平粟,建請置倉,適兇歲,振民不足,即矯發他儲,不俟報。吏懼,白不可,師孟曰:「必俟報,饑者盡死矣。」竟發之。



 徙河東路。晉地
 多土山,旁接川谷,春夏大雨,水濁如黃河,俗謂之「天河」,可溉灌。師孟勸民出錢開渠築堰,淤良田萬八千頃,裒其事為《水利圖經》,頒之州縣。為度支判官。知洪州,積石為江堤,浚章溝,揭北閘,以節水升降,後無水患。



 判三司都磨勘司,接拌契丹使,蕭惟輔曰:「白溝之地當兩屬,今南朝植柳數里,而以北人漁界河為罪,豈理也哉?」師孟曰:「兩朝當守誓約,涿郡有案牘可覆視,君舍文書,騰口說,詎欲生事耶?」惟輔愧謝。



 出為江西轉運使。盜發袁州,
 州吏為耳目,久不獲,師孟械吏數輩送獄,盜即成擒。加直昭文館,知福州,築子城,建學舍,治行最東南。徙廣州,州城為儂寇所毀,他日有警,民駭竄,方伯相踵至,皆言土疏惡不可築。師孟在廣六年,作西城,及交址陷邕管,聞廣守備固,不敢東。時師孟已召還,朝廷念前功,以為給事中、集賢殿修撰,判都水監。



 賀契丹主生辰,至涿州,契丹命席,迎者正南向,涿州官西向,宋使價東向。師孟曰:「是卑我也。」不就列,自日昃爭至暮,從者失色,師孟辭氣
 益厲,叱儐者易之,於是更與迎者東西向。明日,涿人錢於郊,疾馳過不顧,涿人移雄州以為言,坐罷歸班。復起知越州、青州,遂致仕,以光祿大夫卒,年七十八。



 師孟累領劇鎮,為政簡而嚴,罪非死者不以屬吏。發隱擿伏如神,得豪惡不逞跌宕者必痛懲艾之,至剿絕乃已,所部肅然。洪、福、廣、越為立生祠。



 韓晉卿,字伯修,密州安丘人。為童子時,日誦書數千言。長以《五經》中第,歷肥鄉嘉興主簿、安肅軍司法參軍、平
 城令大理詳斷、審刑詳議官,通判應天府,知同州、壽州,奏課第一,擢刑部郎中。



 元祐初,知明州,兩浙轉運使差役法復行,諸道處畫多倉卒失敘,獨晉卿視民所宜而不戾法指。入為大理少卿,遷卿。



 晉卿自仁宗朝已黃訟臬,時朝廷有疑議,輒下公卿雜議。開封民爭鶉殺人,王安石以為盜拒捕斗而死,殺之無罪,晉卿曰:「是鬥殺也。」登州婦人謀殺夫,郡守許遵執為按問,安石復主之,晉卿曰:「當死。」事久不決,爭論盈庭,終持之不肯變,用是
 知名。



 元豐置大理獄,多內庭所付,晉卿持平考核,無所上下。神宗稱其才,每讞獄雖明,若事連貴要、屢鞠弗成者,必以委之,嘗被詔按治寧州獄,循故事當入對,晉卿曰:「奉使有指,三尺法具在,豈應刺候主意,輕重其心乎?」受命即行。



 諸州請讞大闢,執政惡其多,將劾不應讞者。晉卿曰:「聽斷求所以生之,仁恩之至也。茍讞而獲譴,後不來矣。」議者又欲引唐日覆奏,令天下庶戮悉奏決。晉卿言:「可疑可矜者許上請,祖宗之制也。四海萬里,必須系
 以聽朝命,恐自今庾死者多於伏辜者矣。」朝廷皆行其說,故士大夫間推其忠厚,不以法家名之。卒於官。



 葉康直,字景溫,建州人。擢進士第,知光化縣。縣多竹,民皆編為屋,康直教用陶瓦,以寧火患。凡政皆務以利民。時豐稷為穀城令,亦以治績顯,人歌之曰:「葉光化,豐穀城,清如水,平如衡。」



 曾布行新法,以為司農屬。歷永興、秦鳳轉運判官,徙陜西,進提點刑獄、轉運副使。五路兵西征,康直領涇原糧道,承受內侍梁同以餉惡妄奏,神宗
 怒,械康直,將誅之,王安禮力救,得歸故官。



 元祐初,加直龍圖閣,知秦州。中書舍人曾肇、蘇轍劾康直諂事李憲,免官,究實無狀,改知河中府,復為秦州。夏人侵甘谷,康直戒諸將設伏以待,殲其二酋,自是不敢犯境。進寶文閣待制、陜西都運使。以疾請知亳州,通浚積潦,民獲田數十萬畝。召為兵部侍郎,卒,年六十四。



\end{pinyinscope}