\article{列傳第二百一十八隱逸下}

\begin{pinyinscope}

 ○徐中行蘇云卿譙定王忠民劉勉之胡憲郭雍劉愚魏掞之安世通



 徐中行,臺州臨海人。始知學,聞安定胡瑗講明道學,其徒轉相傳授,將往從焉。至京師,首謁范純仁,純仁賢之,薦於司馬光,光謂斯人神清氣和,可與進道。會福唐劉彞赴闕,得瑗所授經,熟讀精思,攻苦食淡,夏不扇,冬不爐,夜不安枕者逾年。乃歸葺小室,竟日危坐,所造詣人莫測也。父死,跣足廬墓,躬耕養母。推其餘力,葬內外親及州里貧無後者十餘喪。晚年教授學者,自灑掃應對、格物致知達於治國平天下,不失其性,不越其序而後
 已。



 其友羅適持節本路,舉以自代,又率部使者以遺逸薦。崇寧中,郡守李諤又以八行薦。時章、蔡竊國柄,竄逐善類且盡,中行每一聞命輒淚下。一日,去之黃巖,會親友,盡毀其所為文,幅巾藜杖,往來委羽山中。客有詰以避舉要名者,中行曰:「人而無行,與禽獸等。使吾得以八行應科目,則彼之不被舉者非人類與?吾正欲避此名,非要名也。」客慚而退。陳瓘謫臺州,聞名納交,暨其沒,錄其行事,謂與山陽徐積齊名,呼為「八行先生」。



 子三人,庭
 筠其季也,童草有志行,事父兄孝友天至。居喪毀甚,既免喪,猶不忍娶者十餘年。秦檜當國,科場尚諛佞,試題問中興歌頌,庭筠嘆曰:「今日豈歌頌時耶!」疏其未足為中興者五,見者尤之,庭筠曰:「吾欲不妄語,而敢欺君乎?」



 黃巖尉鄭伯熊代去,請益,庭筠曰:「富貴易得,名節難守。願安時處順,主張世道。」伯熊受其言,迄為名臣。有詔舉人嘗五上春官者予岳祠。庭筠適應格,所親咸勸之,庭筠辭曰:「吾嘗草封事,謂岳廟冗祿無用。既心非之,可躬
 蹈耶?」



 其學以誠敬為主,夜必就榻而後脫巾,旦必巾而後起。居無惰容,喜無戲言,不事緣飾,不茍臧否。聞人片善,記其姓名。遇饑凍者,推食解衣不靳。僦屋以居,未嘗戚戚。尤袤為守,聞其名,遣書禮之。一日,巾車歷訪舊游,徜徉幾月。歸感微疾,端坐瞑目而逝,年八十有五。鄉人崇敬之,以其父子俱隱遁,稱之曰二徐先生。淳熙間,常平使者朱熹行部,拜墓下,題詩有「道學傳千古,東甌說二徐」之句,且大書以表之曰「有宋高士二徐先生之墓」。



 庭筠之兄庭槐、庭蘭,皆有父風。孫日升,苦學有守,於是徐氏詩書不絕六世矣。



 蘇云卿,廣漢人。紹興間,來豫章東湖,結廬獨居。待鄰曲有恩禮,無良賤老稚皆愛敬之,稱曰蘇翁。身長七尺,美須髯,寡言笑,布褐草履,終歲不易,未嘗疾病。披荊畚礫為圃,藝植耘芟,灌溉培壅,皆有法度。雖隆暑極寒,土焦草凍,圃不絕蔬,滋鬱暢茂,四時之品無闕者。味視他圃尤勝,又不二價,市鬻者利倍而售速,先期輸直。夜織屨,
 堅韌過革舄,人爭貿之以饋遠。以故薪米不乏,有羨則以周急應貸,假者負償,一不經意。溉園之隙,閉門高臥,或危坐終日,莫測識也。



 少與張浚為布衣交,浚為相,馳書函金幣屬豫章帥及漕曰:「余鄉人蘇云卿,管、樂流亞,遁跡湖海有年矣。近聞灌園東湖,其高風偉節,非折簡能屈,幸親造其廬,必為我致之。」帥、漕密物色,曰:「此獨有灌園蘇翁,無雲卿也。」帥、漕乃屏騎從,更服為游士,入其圃,翁運鋤不顧。進而揖之,翁曰:「二客何從來耶?」延入室,
 土銼竹幾,地無纖塵,案上有《西漢書》一冊。二客恍若自失,默計此為蘇云卿也。既而汲泉煮茗,意稍款浹,遂扣其鄉里,徐曰:「廣漢。」客曰:「張德遠廣漢人,翁當識之。」曰:「然。」客又問:「德遠何如人?」曰:「賢人也。第長於知君子,短於知小人,德有餘而才不足。」因問:「德遠今何官?」二客曰:「今朝廷起張公,欲了此事。」翁曰:「此恐怕他未便了得在。」二客起而言曰:「張公令某等致公,共濟大業。」因出書函金幣置幾上。雲卿鼻間隱隱作聲,若自咎嘆者。二客力請共
 載,辭不可,期以詰朝上謁。旦遣使迎伺,則扃戶闃然,排闥入,則書幣不啟,家具如故,而翁已遁矣,竟不知所往。



 帥、漕復命,浚拊幾嘆曰:「求之不早,實懷竊位之羞。」作箴以識之,曰:「雲卿風節,高於傅霖。予期與之,共濟當今。山潛水杳,邈不可尋。弗力弗早,予罪曷針。」



 譙定,字天授,涪陵人。少喜學佛,析其理歸於儒。後學《易》於郭曩氏,自「見乃謂之象」一語以入。郭曩氏者,世家南平,始祖在漢為嚴君平之師,世傳《易》學,蓋象數之學也。
 定一日至汴,聞伊川程頤講道於洛,潔衣往見,棄其學而學焉。遂得聞精義,造詣愈至,浩然而歸。其後頤貶涪,實定之鄉也,北山有巖,師友游泳其中,涪人名之曰讀易洞。



 靖康初,呂好問薦之,欽宗召為崇政殿說書,以論弗合,辭不就。高宗即位,定猶在汴,右丞許翰又薦之,詔宗澤津遣詣行在。至惟揚,寓邸舍,窶甚,一中貴人偶與鄰,饋之食不受,與之衣亦不受,委金而去,定袖而歸之,其自立之操類此。上將用之,會金兵至,失定所在。復歸
 蜀,愛青城大面之勝,棲遁其中,蜀人指其地曰譙巖。敬定而不敢名,稱之曰譙夫子,有繪像祀之者,久而不衰。定《易》學得之程頤,授之胡憲、劉勉之,而馮時行、張行成則得定之餘意者也。定後不知所終,樵夫牧童往往有見之者,世傳其為仙雲。



 初,程頤之父珦嘗守廣漢,頤與兄顥皆隨侍,游成都,見治篾箍桶者挾冊,就視之則《易》也,欲擬議致詰,而篾者先曰:「若嘗學此乎?」因指「《未濟》男之窮」以發問。二程遜而問之,則曰:「三陽皆失位。」兄弟渙
 然有所省,翌日再過之,則去矣。其後袁滋入洛,問《易》於頤,頤曰:「《易》學在蜀耳,盍往求之?」滋入蜀訪問,久無所遇。已而見賣醬薛翁於眉、邛間,與語,大有所得,不知所得何語也。



 憲、勉之、滋皆閩人,時行、行成蜀人,郭曩氏及篾叟、醬翁皆蜀之隱君子也。



 王忠民,潁陽人,世業醫。忠民幼通經史,自靖康以來,數言邊方利害於朝,累召弗至。高宗渡江,忠民隱居不出,諸鎮翟興等皆重之,弗能致;張浚授以迪功郎,不受。興
 徙治藥川,忠民避地南下,遇商虢鎮撫使董先於內鄉,留軍中,事以師禮。



 時劉豫僭立,忠民作《九思圖》及定亂四象達之金主,及鏤板印圖散於偽境,以明天下之義。紹興三年,翟琮薦其忠節於朝,特授宣教郎,詔董先津遣詣行在。既至,宰相呂頤浩、簽書樞密院事徐俯見之皆拜,舍於政府。忠民上疏辭官,言:「臣憤金人無道,故三上金主書,乞還二帝,本心報國,非冀名祿。」上不許。忠民以誥置櫝中,藏七寶山下,力懇求去。復依董先軍中,遂
 不出。



 時又有蘇庠者,丹陽人。紳之後,頌之族也。少能詩,蘇軾見其《清江曲》,大愛之,由是知名。徐俯薦其賢,上特召之,固辭;又命守臣以禮津遣,庠辭疾不至,以壽終。



 劉勉之,字致中,建州崇安人。自幼強學,日誦數千言。逾冠,以鄉舉詣太學。時蔡京用事,禁止毋得挾元祐書,自是伊、洛之學不行。勉之求得其書,每深夜,同舍生皆寐,乃潛抄而默誦之。譙定至京師,勉之聞其從程頤游,邃《易》學,遂師事之。已而厭科舉業,揖諸生歸,見劉安世、楊
 時,皆請業焉。及至家,即邑近郊結草為堂,讀書其中,力耕自給,澹然無求於世。與胡憲、劉子翬相往來,日以講論切磋為事。



 紹興間,中書舍人呂本中疏其行義志業以聞,特召詣闕。秦檜方主和,慮勉之見上持正論,乃不引見,但令策試後省給札而已。勉之知不與檜合,即謝病歸。杜門十餘年,學者踵至,隨其材品,為說聖賢教學之門及前言往行之懿。所居有白水,人號曰白水先生。賢士大夫自趙鼎以下皆敬慕與交。後秦檜益橫,鼎竄
 死,諸賢禁錮,勉之竟不復出。



 勉之一介不妄取。婦家富,無子,謀盡以貲歸於女,勉之不受,以畀族之賢者,命之奉祀。其友朱松卒,屬以後事,且戒其子熹受學。勉之經理其家,而誨熹如子侄。熹之得道,自勉之始。紹興十九年,卒,年五十九。



 胡憲,字原仲,居建之崇安。生而靜愨,不妄笑語,長從從父胡安國學。平居危坐植立,時然後言,雖倉卒無疾言遽色,人犯之未嘗校。紹興中以鄉貢入太學。會伊、洛學
 有禁,憲獨陰與劉勉之誦習其說。既而學《易》於譙定,久未有得,定曰:「心為物漬,故不能有見,唯學乃可明耳。」憲喟然嘆曰:「所謂學者,非克己工夫耶?」自是一意下學,不求人知。一旦,揖諸生歸故山,力田賣藥,以奉其親。安國稱其有隱君子之操。從游者日眾,號籍溪先生,賢士大夫亦高仰之。



 折彥質、範沖、朱震、劉子羽、呂祉、呂本中共以其行義聞於朝,上特召之,憲辭母老。及彥質入西府,又言於上,趣召愈急,憲力辭。乃賜進士出身,授左迪功
 郎、添差建州教授,憲猶不屈。太守魏矼遣行義諸生入里致詔,且為手書陳大義,開譬甚力,憲不得已就職。日與諸生接,訓以為己之學。聞者始而笑,中而疑,久而觀其所以修身、事親、接人者,無一不如所言,遂翕然悅服。郡人程元以篤行稱,龔何以廉節著,皆迎致俾參學政,學者自是大化。



 因七年不徙官,以母年高不樂居官舍,求監南嶽廟以歸。久之,起為福建路安撫使司屬官。時帥張宗元榷鹽急,私販者銖兩亦重坐。憲告以為政大
 體,宗元不悅,憲復請祠而去。



 秦檜方用事,諸賢零落,憲家居不出。檜死,以大理司直召,未行,改秘書正字。既至,次當奏事,而病不能朝,乃草疏言:「金人大治汴京宮室,勢必敗盟。今元臣、宿將惟張浚、劉錡在,識者皆謂金果南牧,非此兩人莫能當。願亟起之,臣死不恨。」時兩人皆為積毀所傷,未有敢顯言其當用者,憲獨首言之。疏入,即求去。上嘉其忠,詔改秩與祠歸。



 初,憲與劉勉之俱隱,後又與劉子翬、朱松交。松將沒,屬其子熹受學於憲與
 勉之、子翬。熹自謂從三君子游,而事籍溪先生為久。方憲之以館職召也,適秦檜諱言之後,憲與王十朋、馮方、查籥、李浩相繼論事,太學士為《五賢詩》以歌之。人始信憲之不茍出,而惜其在位僅半年,不究其底蘊云。紹興三十二年,卒,年七十七。



 郭雍,字子和,其先洛陽人。父忠孝,官至太中大夫,師事程頤,著《易說》,號兼山先生,自有傳。雍傳其父學,通世務,隱居峽州,放浪長楊山谷間,號白雲先生。



 乾道中,以峽
 守任清臣、湖北帥張孝祥薦於朝,旌召不起,賜號沖晦處士。孝宗稔知其賢,每對輔臣稱道之,命所在州郡歲時致禮存問。後更封頤正先生,令部使者遣官就問雍所欲言,備錄繳進。於是,雍年八十有三矣。



 淳熙初,學者裒集程顥、程頤、張載、游酢、楊時及忠孝、雍凡七家,為《大易粹言》行於世。其述雍之說曰:



 《易》貫通三才,包括萬理。伏羲氏之畫,得於天而明天。文王之畫,得於人而明人。羲畫為天,天,君道也,故五之在人為君。文重為地,地,臣
 道也,故二之在人為臣。以上下二卦別而言之如此。合六爻而言之,則三四皆人道也,故謂之中爻。



 《乾》,元亨利貞,初曰四德。後又曰乾元,始而亨者也。利牝馬貞,利君子貞。是以四德為二義亦可矣。乾,陽物也。坤,陰物也。由《乾》一卦論之,則元與亨陽之類,利與貞陰之類也。是猶春夏秋冬雖為四時,由陰陽觀之,則春夏為陽,秋冬為陰也。天之所謂元亨利貞者,如立天之道,陰與陽之類也。地之所謂元亨利貞者,如立地之道,柔與剛之類也。
 人之所謂元亨利貞者,如立人之道,仁與義之類也。



 又《坤》之六五,坤雖臣道,五實君位,雖以柔德,不害其為君;猶《乾》之九二,雖有君德,不害其為臣。故乾有兩君,德無兩君;坤有兩臣,德無兩臣。六五以柔居尊,下下之君也。江海所以能為百穀王者,以其善下下也。下下本坤德也。黃,中色也,色之至美也;裳,下服也,是以至美之德而下人也。



 其發明精到如此。淳熙十四年。卒。



 劉愚,字必明,衢州龍游人。幼警敏力學。弱冠入太學,有
 聲,受業者甚眾。侍御史柴瑾、祭酒顏師魯、博士林光朝深器重之。瑾每奏對稱上意,則曰:「臣客劉愚為臣言。」師魯嘗奏愚行藝,上記曰:「此向者柴瑾所薦也。」上舍釋褐,居第一。調江陵府教授,早晚為諸生講說,同僚相率以聽。愚益謙下,與葉適、項安世講論不倦,每以隱居學道為樂。



 歲滿,帥王藺致書剡闢,固辭,貧不能歸。外移安鄉縣令,邑逋賦萬計,愚核實數,寬限期,民不見吏而賦自足。會歲歉,出常平米賑貸,邑佐持不可,愚曰:「有罪不以
 相累。」出緡錢數千萬,召商糴他郡而收元直,米價頓平,猶積廩數千石以備饑旱。邑有範仲淹讀書地,為繪像立祠,興學,士競知勸。



 諸司交薦,改秩,愚雅不樂仕進,遂致仕。丞相餘端禮,鄉人也,與愚有舊,且召堂審,愚竟舍去不顧。結廬城南,頹坦敗壁,蓬蒿蕭然。著書自適,《書》、《禮》、《語》、《孟》皆有解。年八十三而卒。故友與其門人私謚曰謙靖先生,後更謚曰靖君,鄉郡祠之。



 妻徐氏在家時,其母將以嫁姑子之富者,徐泣曰:「為富人妻,不願也。」遂歸於
 愚,居破屋中,一事機杼。愚嘗懷白金歸,徐怒曰:「我以子為賢而若是,亟具歸。」愚出書以示,束修得也,乃已。有梁鴻之風焉。



 子克、幾、凡。克蚤以詩名,葉適嘗稱其可繼陶、韋。



 魏掞之,字子實,建州建陽人,初字元履。自幼有大志。師胡憲,與朱熹游。兩以鄉舉試禮部不第。嘗客衢守章傑所。趙鼎以謫死,其子汾將喪過衢。傑雅憾鼎,又希秦檜意,遣尉翁蒙之領卒掩取鼎平時與故舊來往簡牘。蒙
 之先遣人告汾焚之,逮至一無所得。傑怒,治蒙之,拘汾於兵家所,且以告檜。掞之以書責傑,長揖徑歸。築室讀書,榜以「艮齋」,自是人稱曰艮齋先生。



 閩帥汪應辰、建守陳正同知其賢,薦於朝,時相尼之,不果召。乾道中,詔舉遺逸,部刺史芮燁與帥、守共表其行誼,特詔召之,掞之力辭。時宰相陳俊卿,閩人也,雅知掞之,招之甚力。乃以布衣入見,極陳當時之務,大要勸上以修德業、正人心、養士氣為恢復之本。上嘉納之,賜同進士出身,守太學
 錄。



 先是,學官養望自高,不與諸生接。掞之既就職,日進諸生教誨之,又增葺其舍,人人感勵。將釋菜,掞之請廢王安石父子從祀,追爵程顥、程頤,列於祀典,不報。復言「太學之教宜以德行經術為先,其次則通習世務。今乃專以空言取人」,又不報。遂丐去。



 會福州副總管曾覿秩滿還,在道,掞之累疏以諫,移疾杜門,遺書陳俊卿責其不能救止,語甚切。遂以迎親請歸,行數曰,罷為臺州教授。方掞之之未行也,覿至國門外已久,伺掞之去,乃敢
 入。掞之在朝不能半歲,既歸,喟然嘆曰:「上恩深厚如此,而吾學不足以感悟聖意。」乃日居艮齋,條理舊聞,以求其所未至。



 其居家,謹喪祭,重禮法。從父有客於南者,千里迎養,死葬如禮,而字其孤。建俗生子多不舉,為文以戒,全活者甚眾。又白於官,請督不葬其親者,富與期,貧與財,而無主後者掩之。每遇歲饑,為粥以食饑者。後依古社倉法,請官米以貸民,至冬取之以納於倉。部使者素敬掞之,捐米千餘斛假之,歲歲斂散如常,民賴以濟。
 諸鄉社倉自掞之始。



 與人交,嘉其善而救其失。後進以禮來者,茍有寸長,必汲汲推挽成就之。至或訾其近名,則蹙然曰:「使夫人而避此嫌,為善之路絕矣。」病革,母視之,不巾不見。戒其子「毋以僧巫俗禮浼我。」以書召朱熹至,委以後事而訣。卒,年五十八。



 後上思其直諒,將召用之,大臣言已死,乃贈直秘閣。熹平日趣向與掞之同。乾道中,熹亦被召,將行,聞掞之去國,乃止。



 青城山道人安世通者,本西人。其父有謀策,為武官,數
 以言乾當路不用,遂自沈於酒而終。世通亦隱居青城山中不出。



 吳曦反,乃獻書於成都帥楊輔曰:「世通在山中,忽聞關外之變,不覺大慟。世通雖方外人,而大人先生亦嘗發以入道之門。竊以為公初得曦檄,即當還書,誦其家世,激以忠義,聚官屬軍民,素服號慟,因而散金發粟,鼓集忠義,閉劍門,檄夔、梓,興仗義之師,以順討逆,誰不願從?而士大夫皆酒缸飯囊,不明大義,尚云少屈以保生靈,何其不知輕重如此!夫君乃父也,民乃子也,
 豈有棄父而救子之理?此非曦一人之叛,乃舉蜀士大夫之叛也。聞古有叛民無叛官,今曦叛而士大夫皆縮手以聽命,是驅民而為叛也。且曦雖叛逆,猶有所忌,未敢建正朔、殺士大夫,尚以虛文見招,亦以公之與否卜民之從違也。今悠悠不決,徒為婦人女子之悲,所謂停囚長智,吾恐朝廷之失望也。凡舉大事者,成敗死生皆當付之度外。區區行年五十二矣,古人言:『可以生而生,福也;可以死而死,亦福也。』決不忍汗面戴天,同為叛民
 也。」



 輔有重名,蜀中士大夫多勸以舉義者,而世通之言尤切至。輔不能決,遂東如江陵,請吳獵舉兵以討曦。未幾,曦敗,獵使蜀,薦士以世通為首云。



 ◎卓行



 ○劉庭式巢谷徐積曾叔卿劉永一



 父子有親,夫婦有別,朋友有信,天下之所共知而共由者也,乃有卓行於斯焉。徐積於其所天,劉庭式於其室
 家,巢谷於其知己,皆行常人之難。行其所難而安焉,豈非卓乎?曾叔卿之不欺,劉永一之不茍取,皆以一事而人譽之終身,蓋有其所矣,其可忽諸!撰《卓行傳》。



 劉庭式,字得之,齊州人,舉進士。蘇軾守密州,庭式為通判。初,庭式未第時,議娶鄉人之女,既約,未納幣。庭式乃及第,女以病喪明,女家躬耕貧甚,不敢復言。或勸納其幼女,庭式笑曰:「吾心已許之矣,豈可負吾初心哉。」卒娶之。生數子,後死,庭式喪之逾年,不肯復娶。軾問之曰:「哀
 生於愛,愛生於色。今君愛何從生,哀何從出乎?」庭式曰:「吾知喪吾妻而已。吾若緣色而生愛,緣愛而生哀,色衰愛弛,吾哀亦忘,則凡揚袂倚市,目挑而心招者,皆可以為妻也耶?」軾深感其言。庭式後監太平觀,老於廬山,絕粒不食,目奕奕有紫光,步上下峻阪如飛,以高壽終。



 巢谷,初名穀,字元修,眉州眉山人。父中,谷傳其學,雖樸而博。舉進士京師。谷素多力,見舉武藝者心好之,遂棄其舊學,蓄弓箭,習騎射,久之業成而不中第。聞西邊多
 驍勇,為四方冠,去游秦鳳、涇原間。所至友其秀桀,與韓存寶尤相善,教之兵書。



 熙寧中,存寶為河州將,有功,號熙河名將。會滬州蠻乞弟擾邊,諸郡不能制,命存寶出兵討之。存寶不習蠻事,邀穀至軍中問焉。及存寶得罪,將就逮,自度必死,謂穀曰:「我涇原武夫,死非所惜。顧妻子不免寒餓,橐中有銀數百兩,非君莫可使遺之者。」谷許諾,即變姓名,懷銀步往授其子,人無知者。存寶死,穀逃避江、淮間,會赦乃出。



 蘇軾謫黃州,與穀同鄉,幼而識
 之,因與之游。乃軾與弟轍在朝,穀浮沉里中,未嘗一來相見。紹聖初,軾、轍謫嶺海,平生親舊無復相聞者,穀獨慨然自眉山誦言欲徒步訪兩蘇,聞者皆笑其狂。



 元符二年,穀竟往,至梅州遺轍書曰:「我萬里步行見公,不意自全,今至梅矣,不旬日必見,死無恨矣。」轍驚喜曰:「此非今世人,古之人也。」既見,握手相泣,已而道平生,逾月不厭。時穀年七十三,瘦瘠多病,將復見軾於海南,轍愍而止之曰:「君意則善,然循至儋數千里,當復渡海,非老人
 事也。」穀曰:「我自視未即死也,公無止我。」閱其橐中無數千錢,轍方困乏,亦強資遣之。舟行至新會,有蠻隸竊其橐裝以逃,獲於新州,穀從之至新,遂病死。轍聞,哭之失聲,恨不用己言而致死,又奇其不用己言而行其志也。



 徐積,字仲車,楚州山陽人。孝行出於天稟。三歲父死,旦旦求之甚哀,母使讀《孝經》,輒淚落不能止。事母至孝,朝夕冠帶定省。從胡翼之學。所居一室,寒一衲裘,啜菽飲水,翼之饋以食,弗受。



 應舉入都,不忍舍其親,徒載而西。
 登進士第,舉首許安國率同年生入拜,且致百金為壽,謝卻之。以父名「石」終身不用石器,行遇石則避而不踐,或問之,積曰:「吾遇之則怵然傷吾心,思吾親,故不忍加足其上爾。」母亡,水漿不入口者七日,悲慟嘔血。廬墓三年,臥苫枕塊,衰絰不去體,雪夜伏墓側,哭不絕音。翰林學士呂溱過其廬適聞之,為泣下曰:「使鬼神有知,亦垂涕也。」甘露歲降兆域,杏兩枝合為乾。既終喪,不徹筵幾,起居饋獻如平生。



 中年有聵疾,屏處窮里,而四方事無
 不知。客從南越來,積與論嶺表山川險易、鎮戍疏密,口誦手畫,若數一二。客嘆曰:「不出戶而知天下,徐公是也。」自少及老,日作一詩,為文率用腹稿,口占授其子。嘗借人書策,經宿還之,借者紿言中有金葉,積謝而不辨,賣衣償之。鄉人有爭訟,多就取決。州以行聞,詔賜粟帛。



 元祐初,近臣合言:「積養親以孝著,居鄉以廉稱,道義文學,顯於東南。今年過五十,以耳疾不能出仕。朝廷方詔舉中外學官,如積之賢,宜在所表。」乃以揚州司戶參軍為
 楚州教授。每升堂,訓諸生曰:「諸君欲為君子,而勞己之力,費己之財,如此而不為,猶之可也;不勞己之力,不費己之財,何不為君子?鄉人賤之,父母惡之,如此而不為,可也。鄉人榮之,父母欲之,何不為君子?」又曰:「言其所善,行其所善,思其所善,如此而不為君子者,未之有也。言其不善,行其不善,思其不善,如此而不為小人者,未之有也。」聞之者斂衽敬聽。



 居數歲,使者又交薦之,轉和州防禦推官,改宣德郎,監中嶽廟。卒,年七十六。政和六年,
 賜謚節孝處士,官其一子。



 曾叔卿,建昌南豐人,鞏族兄也。家苦貧,即心存不欺。嘗買西江陶器,欲貿易於北方,既而不果行。有從之轉售者,與之。既受直矣,問將何之,其人曰:「欲效君前策耳。」叔卿曰:「不可。吾聞北方新有災饉,此物必不時洩,故不以行。餘豈宜不告以誤子。」其人即取錢去。居鄉介潔,非所宜受,一介不取。妻子困於饑寒,而拊庇孤煢,唯恐失其意。起家進士,至著作佐郎。熙寧中,卒。



 劉永一,陜州夏縣人。孝友廉謹。熙寧初,巫咸水溢入縣城,民多溺死。永一持竿立門前,見他人物流入者輒擿出之。有僧寓錢數萬於其室,無何而僧死,永一詣縣自言,請以錢歸其弟子。鄉人負債不肯償,立焚其券。行事類此。兄大為,醫助教。居親喪,不飲酒食肉,終三年。司馬光傳之,以為今士大夫所難。



\end{pinyinscope}