\article{本紀第四十四}

\begin{pinyinscope}

 理宗四



 二年春正月乙亥朔,大元城利州、閬州。詔湘潭縣民陳克良孝行,表其門。



 二月甲辰朔,詔太常厘正秦檜謚,因諭輔臣曰:「謚『繆狠』可也。」熒惑犯權星。乙巳,詔利州統制
 呂達戰沒,贈官四轉,官一子承信郎,一子下班祗應。己酉,餘晦兼四川屯田使。庚申,詔饒州布衣饒魯不事科舉,一意經學,補迪功郎、饒州教授。辛酉,日暈周匝。戊辰,故直華文閣李燔,先儒朱熹門人,賜謚文定。



 三月壬午,王元善使大元,留七年來歸。戊子,雪。詔蠲江、淮今年二稅。己丑,詔錄襄城功,高達帶行環衛官、遙郡團練使,職任依舊;王登行軍器監丞、制司參議官;程大元、李和以下將士六千六百一十三人補轉官資有差。甲午,城東海,
 賈似道以圖來上。



 夏四月辛亥,詔邊兵貧困可閔,閑田甚多,擇其近便者分給耕種,制司守臣治之。乙丑,以徐清叟知樞密院兼參知政事,董槐參知政事。



 六月壬寅朔,罷臨安府臨平鎮稅場。甲辰,四川制司言:合州、廣安軍北兵入境,王堅、曹世雄等戰御有功。詔堅官兩轉,餘各補轉官資。甲寅,侍御史吳燧等論故蜀帥餘玠聚斂罔利七罪,玠死,其子如孫盡竊帑庾之積以歸。詔簿錄玠家財。以李曾伯為資政殿學士,依舊節制四川。丙辰,
 利州王佐堅守孤壘,降將南永忠以兵薄城下,佐罵之,永忠流涕而退。初,隆慶教授鄭炳孫不從南永忠降,先縊死其妻女,亦朝服自縊。詔獎諭:佐進官一秩,炳孫贈朝奉郎、直秘閣,仍訪其子,官以文資。王伯大乞致仕,詔進一秩,允所請。丁巳,以賈似道同知樞密院事,職任依舊。庚午,詔餘晦赴闕。閏六月壬申,董槐疏:蜀事孔棘,願假臣宣撫之名,置司夔門,以通荊、蜀。上優詔答曰:「士大夫以事功自勉者鮮,卿請帥蜀,足見忠壯。然經理西事,
 當在廟堂,宜竭謀猷,以副委任。」詔蒲擇之暫權四川制置司事。甲戌,錄嘉定戰功。先是,大元兵圍城五旬,帥守俞興、元用等夜開關力戰而圍解。詔俞興等十六人各官五轉,將士補轉有差。以包恢提點浙西刑獄,招捕荻浦鹽寇。乙亥,臺州海寇積年,民罹其害,路分董橚洎進士周自中等擒獲,詔橚官一轉,餘推賞有差。壬午,以李曾伯為四川宣撫使兼京湖制置大使,進司夔路,詔賜曾伯同進士出身。罷江灣浮鹽局,戊戌,大元使離揚州
 北歸。



 秋七月己酉,詔「前蜀帥餘玠鎮撫無狀,兵苦於征戌,民困於徵求,茲俾其家輸所取蜀財,犒師振民;並邊諸郡田租,其復三年。」詔思、播兩州連年捍禦,其守臣田應庚、楊文各官一轉,餘推恩。詔賈似道開閫,以樞密行府為名。庚戌,有流星大如太白。甲寅,故光祿大夫賈涉謚忠肅。壬戌,復安西堡。己巳,荻浦海寇平,包恢進直龍圖閣,劉達授橫行、帶遙郡。李性傳赴闕,以王堅為興元都統兼知合州。



 八月乙亥,詔以前知閬州兼利西安撫
 王惟忠付大理獄,尋命臺臣監鞫。辛巳,徐清叟乞罷機務,詔不允。癸巳,謝方叔等上《玉牒》、《日歷》、《會要》及《七朝經武要略》、《中興四朝志傳》,詔方叔、徐清叟、董槐等各進秩。戊戌,籍王惟忠家財。



 九月辛亥,祀明堂,大赦。辛酉,詔詣西太一宮,為國祈祥,起居郎牟子才再疏諫而止。丙寅,詔戒外戚毋乾請。詔:山陰、蕭山、諸暨、會稽四縣水,其除今年田租。丁卯,太白晝見。



 冬十月庚午朔,謝方叔等進寶祐編《吏部七司續降條令》。癸酉,皇子祺進封忠王。甲
 午,斬王惟忠於都市。丁酉,追削餘玠資政殿學士,奪餘晦刑部侍郎告身。戊戌,段元鑒上隆慶堡戰功。



 十一月壬寅,日南至。忠王冠。丁未,大元城光化舊治。十二月庚午,排保甲,行自實法。癸未,雷。四川苦竹隘捷至。甲午,隆慶部兵周榮被獲歸北,密約段元鑒入隘解圍,事覺就禽,不屈而死,馬徽、白端戰歿。詔四川宣撫司為之立廟。安西堡受攻五月,將士力戰解圍,居民以資糧助軍實。詔四川宣撫司具名推恩,在城人並賞一資,復租賦
 五年。餘玠男如孫征所認錢三千萬將足。詔如孫削三秩、勒停。



 三年春正月己未,迅雷。巴州捷至。庚申,城均州龍山。起居郎牟子才上疏言:「元夜張燈侈靡,倡優下賤,奇技獻笑,媟污清禁,上累聖德。今因震霆示威,臣願聖明覺悟,天意可回。」帝納其言。壬戌,詔宗正寺所擬宗子名,以用、宜、季、次、紹五字,續大、由、友、嗣、甫之下。



 二月乙亥,詔右千牛衛上將軍乃猷授蘄州防禦使,奉沂靖惠王祠事。兼
 給事中王野言:「國家與大元本無深仇,而兵連禍結,皆原於入洛之帥輕啟兵端。二三狂妄如趙楷、全子才、劉子澄輩,輕而無謀,遂致只輪不返。全子才誕妄慘毒,今乃援劉子澄例自陳改正,乞寢二人之命,罷其祠祿,以為喪師誤國之戒。」從之。己卯,復廣陵堡城,賈似道以圖來上。壬午,詔發緡錢二百萬給四川調度。乙酉,詔以告身、祠牒、新會、香、鹽,命臨安府守臣馬光祖收換兩界舊敝會子。



 三月己酉,詔沿邊耕屯,課入登羨,管屯田官推
 賞,荊襄、兩淮及山砦如之。庚戌,邵武寇平。癸丑,詔自實法宜寬期限,監司、守臣其嚴戢吏奸,毋煩擾民。以吳淵為觀文殿學士、京湖制置使、知江陵府。己未,雨土。



 夏四月乙酉,以江萬里知福州、福建安撫使。



 五月,久雨。丁未,以監司、州郡闢書冗濫,詔申嚴禁止。己酉,李性傳薨。辛酉,太陰入畢。嘉定大雨、雹,與敘南同日地震。浙西大水。



 六月辛未,大風。甲戌,太陰入氐。甲戌,李全子松壽葺舊海城,窺海道,賈似道調兵敗之,敕書獎諭,趣上立功等
 第、姓名推賞。戊子,洪天錫劾內官盧允叔、董宋臣,疏不報,竟去,詔遷太常少卿。辛卯,王野以御史胡大昌言罷給事中,依舊端明殿學士、提舉洞霄宮。



 秋七月辛丑,太陰入氐。癸丑,以呂文德知鄂州,節制鼎、澧、辰、沅、靖五州。丙辰,謝方叔、徐清叟以御史朱應元言罷。辛酉,有流星大如太白。詔三省樞府機政,令董槐、程元鳳輪日判事取旨。壬戌,以謝方叔為觀文殿大學士、提舉臨安府洞霄宮。



 八月乙丑朔,以董槐為右丞相兼樞密使,程元鳳
 簽書樞密院事、權參知政事,蔡抗為端明殿學士、同簽書樞密院事,徐清叟資政殿學士、提舉玉隆萬壽宮、任便居住。丁卯,歲星、熒惑在柳。己巳,太陰在氐。馬光祖兼節制和州、無為、安慶三郡屯田使。丙子,鄭性之薨。庚寅,福建安撫江萬里以臺臣李衢言罷新命、提舉武夷山沖祐觀。辛卯,應彳繇薨。



 九月甲午朔,雷。丙午,以徐清叟為資政殿學士、提舉洞霄宮。丙辰,陳顯伯兼資善堂翊善,皮龍榮兼資善堂贊讀。壬戌,權中書舍人陳大方言:「劉
 子澄端平入洛之師,賈勇贊決,北兵方入唐州界,子澄巳率先遁逃,一敗塗地,二十年來,為國家患者,皆原於此,宜投之四裔。」詔罷子澄祠祿。



 冬十月甲戌,太白晝見。丁丑,有流星出自畢。



 十一月丁巳,熒惑犯太微垣上相星。十二月乙丑,嗣濮王善奐薨。丙子,少傅、節度使與歡薨,贈少師,追封奉化郡王。



 四年春正月乙未,詔謝方叔奪職罷祠,謝修削三秩勒停。乙巳,太陰犯歲星。己酉,太陰犯熒惑。辛亥,以吳淵為
 京湖制置使兼夔路策應使,軍馬急切,便宜行事。庚申,蜀閫奏捷。辛酉,詔史嵩之觀文殿大學士,依前金紫光祿大夫、永國公致仕。



 二月戊辰,雨雹。丙子,詔襲封衍聖公孫孔洙添差通判吉州,不厘務。



 三月壬寅,以少師、嗣榮王與芮為太傅。乙卯,日暈周匝。丙辰,帝制《字民訓》賜改秩親民官。



 夏四月庚午,月暈周匝。癸未,以程元鳳參知政事;蔡抗同知樞密院事;賈似道參知政事,職任依舊;李曾伯資政殿大學士、福建安撫使;吳淵進二秩,職
 任依舊;吳潛沿海制置使、判慶元府;馬光祖煥章閣直學士,職任依舊。



 五月甲午,孫夢觀兼資善堂贊讀,章鑒兼資善堂直講。先聖五十代孫孔元龍賜迪功郎,授初品官。甲辰,羅氏鬼國遣報思、播言:大元兵屯大理國,取道西南,將大入邊。詔以銀萬兩,使思、播結約羅鬼為援。徐清叟奪資政殿大學士、罷祠祿,王野奪端明殿學士罷祠,仍褫執政恩數。丁未,太白晝見。詔申嚴老鼠隘防戍。襄、樊閫臣奏捷。甲寅,賜禮部進士文天祥以下六百
 一人及第、出身有差。



 六月甲戌,朱祀孫太府寺簿、知瀘州兼潼川路安撫,任責瀘、敘、長寧邊防。浙江堤成。癸未,董槐罷。臺臣丁大全既累疏擊之,辭極詆毀,且以臺牒役隅兵夜半迫槐出關,物論殊駭;三學生屢上書以為言,詔以槐為觀文殿大學士、提舉臨安府洞霄宮。詔程元鳳、蔡抗可輪日判事,軍國重務取旨。丁亥,太白入井。



 秋七月甲寅,知敘州史俊調舟師與大元兵戰,凡十三合,詔俊官三轉,仍帶閣門行宣贊舍人。乙卯,以程元鳳
 為右丞相兼樞密使,蔡抗參知政事,張磻端明殿學士、簽書樞密院事。



 八月甲子,程元鳳上疏言正心、待臣、進賢、愛民、備邊、守法、謹微、審令八事。



 九月壬辰,西南蕃呂告蠻目寧名天兄弟慕義與烏蘇蠻合力為國御難,詔各補承信郎。丙申,知邕州程芾以貪暴,詔削二秩罷之。甲寅,監察御史朱熠言:「境土蹙而賦斂日繁,官吏增而調度日廣。景德、慶歷時,以三百二十餘郡之財賦,供一萬餘員之奉祿;今日以一百餘郡之事力,贍二萬四千
 餘員之冗官。邊郡則有科降支移,內地則欠經常納解。欲寬民力,必汰冗員。」帝納焉。冬十月壬戌,太陰犯鬥。



 十一月戊子朔,荊、襄閫臣以功狀來上,詔推賞將士。戊戌,京湖繼上戰功。詔:「蜀罹兵革,吾民重困,所當勞來撫摩,使之樂業。比聞官吏乃肆誅求,殊失培植邦本之意。自今四川制司戒飭屬郡,違者罪無赦,御史臺其嚴覺察。」乙巳,以監察御史吳衍、翁應弼劾太學武學生劉黻等八人不率,詔拘管江西、湖南州軍,宗學生與人匈等七人
 並削籍,拘管外宗正司。癸丑,以張磻同知樞密院事,丁大全端明殿學士、簽書樞密院事,馬天驥端明殿學士、同簽書樞密院事。詔戒群臣洗心飭行,毋縱於貨賄,其或不悛,舉行淳熙成法。又開國以來勛臣之裔,有能世濟其美而不世其祿者,所在州郡以聞。參知政事蔡抗輒擅去國,勉留不返,詔授職予祠,尋以林存言,寢其命。十二月戊午朔,熒惑犯填星。庚申,大元城棗陽。乙丑,以張磻兼參知政事。甲戌,獎諭荊閫吳淵,其有功將士,趣
 上姓名、等第推賞。



 五年春正月丁亥朔,以趙葵為少保、寧遠軍節度使、京湖宣撫使、判江陵府兼夔路策應大使,進封衛國公;賈似道進知樞密院事、職任依舊;吳淵參知政事;李曾伯荊湖南路撫使兼知潭州;吳潛、趙與各官一轉。乙巳,雷。丙午,禁奸民作白衣會,監司、郡縣官等失覺察者坐罪。辛亥,吳淵薨,贈少師,謚莊敏。



 二月戊午,四川嘉定上戰功。以賈似道為兩淮安撫使。辛酉,命趙葵兼湖廣
 總領財賦,餘晦淮西總領財賦。壬戌,築思州三隘。丁丑,布衣餘一飛、高杞陳襄陽備御策,詔命趙葵行之。



 夏四月丁卯,詔襄陽安撫高達以白河戰功,轉行右武大夫、帶遙郡防禦使;……



 五月庚申,雨。丁卯,城荊山,置懷
 遠軍荊山縣。詔賈似道官兩轉。戊寅,詔京湖、沿江、海道嚴備舟師防遏。辛巳,復劍門壘,賞蒲擇之官兩轉,朱祀孫、蒲黼、楊大淵、韓勇各官四轉。壬午,夏貴正任吉州刺史、帶御器械、鎮江駐扎都統制、知懷遠軍。



 六月丙戌,太白、歲星合於翼。辛卯,太陰入氐。丁酉,祈雨。馬天驥以臺臣言罷,詔依舊端明殿學士、提舉臨安洞霄宮。



 秋七月丙辰,祈雨。戊午,雨。己未,太白晝見。丁卯,有流星大如桃。丙子,太陰入井。



 八月丙戌,光化軍奏捷。臺州火。癸巳,詔
 謝方叔仍舊職,蔡抗以資政殿學士並領祠在京。甲午,給事中邵澤等言謝方叔罪狀,詔寢祠命。丙申,京城火。庚子,以張磻參知政事,丁大全同知樞密院事兼權參知政事。己酉,史嵩之薨,贈少師,謚莊肅。



 九月壬子朔,詔今後臺臣遷他職,輒出關,以違制論,仍著為令。辛酉,祀明堂,大赦。



 冬十月庚寅,張磻薨,贈少師。癸巳,雷。甲午,虹見。丁酉,以林存簽書樞密院事。庚子,詔皇子忠王祺授鎮南、遂安軍節度使,皇女進封升國公主。



 十一月丙辰,
 李曾伯兼節制廣南,任責邊防。乙丑,獎諭安南國,賜金器幣、香茗。乙亥,詔京湖帥臣,黃平、清浪、平溪分置屯戍。庚辰,詔三邊郡縣官毋擅離職守,諸制帥臣其嚴糾察。十二月壬午,李曾伯依舊資政殿學士、湖南安撫使兼廣南制置使,移司靜江府。丁未,熒惑入氐。



 六年春正月辛亥朔,以丁大全參知政事兼同知樞密院事,林存兼權參知政事。癸亥,詔出封樁庫銀萬兩付蜀閫。辛未,詔授成穆皇后弟太師郭師禹孫善庸承務
 郎,仍免銓注差。癸酉,罷李曾伯廣西經略,以廣南制置大使兼知靜江府。其經略司官屬,改充制司官屬。甲戌,詔樞密院編修官呂逢年詣蜀閫,趣辦關隘、屯柵、糧餉,相度黃平、思、播諸處險要緩急事宜,具工役以聞。戊寅,雷。



 二月辛巳朔,以馬光祖為端明殿學士、京湖制置使、知江陵府,兼夔路策應、湖廣總領財賦屯田事。壬辰,雨土。



 三月辛亥朔,祈雨。丙辰,馬光祖請以呂文德、王鑒、王登、汪立信等充制司參議官及闢制司準備差使等
 官,詔光祖開閫之初,姑從所請。戊辰,以馬光祖兼荊湖北路安撫使。庚午,熒惑退入氐。甲戌,詔湖北提點刑獄文復之移司江陵,兼京湖制司參議官。



 夏四月庚辰朔,詔:自冬徂春,天久不雨,民失東作。自四月一日始,避殿減膳,仰答譴告。癸未,程元鳳等以久旱乞解機務,詔不允。甲申,大雨。丙申,群臣三表請御正殿,從之。丁酉,詔田應己思州駐扎御前忠勝軍副都統制,往播州共築關隘防禦。己亥,臺臣朱熠劾沿江制置副使呂好問黃州
 之役貪酷誤事,詔褫職。乙巳,程元鳳罷,以觀文殿學士判福州,尋提舉洞霄宮。丙午,趙葵三辭免福建安撫使,詔授醴泉觀使兼侍讀。丁未,以丁大全為右丞相兼樞密使,林存同知樞密院事兼權參知政事,朱熠端明殿學士、簽書樞密院事。



 五月庚戌朔,詔襄、樊解圍,高達、程大元應援,李和城守,皆有勞績,將士用命,深可嘉尚,其亟議行賞激。癸丑,詔懷遠、漣水相繼奏功,夏貴官兩轉,兼河南招撫使。毛興轉右武大夫,並依舊任。丁巳,李曾
 伯言:「廣西多荒田,民懼增賦不耕,乞許耕者復三年租,後兩年減其租之半,守令勸墾闢多者賞之。」奏可。丙寅,命嗣榮王與芮判大宗正事。丁卯,嗣秀王師彌薨。



 六月癸巳,臺臣戴慶□劾淮東總領趙與時,奪職鐫秩。



 秋七月庚戌,城凌霄山,詔朱祀孫進一秩,易士英帶行閣門宣贊,餘轉官有差。癸丑,熒惑犯房宿。戊午,趙葵四辭免醴泉觀使兼侍讀,乞外祠,從之。戊辰,蜀郡劉整上捷,詔推恩賞。癸酉,知平江府餘晦,以臺臣戴慶□言曩敗績
 於蜀,誤國欺君,詔奪寶章閣待制罷任,追冒支官錢。甲戌,詔前福建漕臣高斯得已奪職鐫官,其贓百餘萬嚴限征償,以懲貪吏。乙亥,呂文德入播州,詔京湖給銀萬兩。



 八月癸未,太陰行犯熒惑。戊戌,詔上流鎖江防禦。癸卯,詔申嚴倭船入界之禁。



 九月壬子,詔蜀、廣、海道申嚴防遏。甲寅,詔安南情狀叵測,申飭邊防。戊辰,安豐上戰功。有流星透霞。



 冬十月丙子朔,詔蜀中將帥雖未克復成都,而暴露日久,戰功亦多,宜與序升,其亟條具以聞。
 丁丑,以俞興為四川制置副使、知嘉定府兼成都安撫副使。乙酉,詔知隆慶府楊禮守安西堡有功,官兩轉。戊子,大元兵攻通、泰州。庚寅,廣南劉雄飛奏橫山之功,詔雄飛官三轉,部兵將校官兩轉。辛卯,詔常州、江陰、鎮江發米振贍淮民。



 十一月己酉,林存罷,以資政殿學士知建寧府。癸丑,穎州上戰功,詔亟推賞,以示激厲。詔追復餘玠官職。甲寅,築黃平,賜名鎮遠州,呂逢年進一秩。詔撫諭沿邊將士。丙辰,給事中張鎮言:徐敏子曩帥廣右,
 嗜殺黷貨,流毒桂府。詔仍舊羈管隆興府。丁巳,葉夢鼎依舊職知隆興府。壬戌,以朱熠同知樞密院事兼權參知政事,饒虎臣端明殿學士、同簽書樞密院事,賈似道樞密使、兩淮宣撫使。甲子,太陰犯權星。丁卯,東海失守,賈似道抗章引咎,詔令以功自贖,特與放罪。甲戌,淮東帥臣奏大元兵退。填星、熒惑在危。十二月戊寅,詔改來年為開慶元年。庚辰,大元兵渡馬湖入蜀,詔馬光祖時暫移司峽州,六郡鎮撫向士璧移司紹慶府,以便策應。
 癸未,房州上戰功。丙戌,詔置橫山屯。丁亥,向士璧不俟朝命進師歸州,捐貲百萬以供軍費;馬光祖不待奏請招兵萬人。捐奉銀萬兩以募壯士,遂有房州之功。詔士璧、光祖各進一秩。辛丑,詔李曾伯城築關隘,訓練民兵峒丁,申嚴防遏。填星、太白、熒惑合於室。



 開慶元年春正月乙巳朔,詔飭中外奉公法,圖實政。馬光祖與執政恩數。李曾伯進觀文殿學士。己酉,大元兵攻忠、涪,漸薄夔境,詔蒲擇之、馬光祖戰守調遣,便宜行
 事。辛亥,詔:「戍蜀將士,頻年戰御,暴露可閔。今申命蒲擇之從優犒師,春防畢日即與更戍,其輒逃歸者從軍令。」癸丑,詔呂文德城黃平,深入蠻地,撫輯有方,與官三轉。庚申,詔知賓州呂振龍,知象州奚必勝,兵至聞風先遁,兵退乃返,並追毀出身文字,竄遠郡。橫州守臣劉清卿設隘堅守,與官一轉。壬戌,監察御史章士元言謝方叔帥蜀誤國,詔方叔更與鐫秩,其子修竄廣南。癸亥,左司諫沈炎言餘晦壞蜀,幕屬李卓、王克己濟惡斂怨,詔晦、
 卓、克己各奪兩官。丙寅,印應飛依舊職知鄂州兼湖北轉運使。丁卯,賈似道以樞密使為京西湖南北四川宣撫大使、都大提舉兩淮兵甲、湖廣總領、知江陵府。蜀帥蒲擇之以重兵攻成都,不克。大元兵破利州、隆慶、順慶諸郡,閬、蓬、廣安守將相繼納降,又造浮梁於涪州之藺市。戊辰,以李庭芝權知揚州。



 二月乙亥朔,詔京西提刑王登提兵援蜀,功未及成,繼志以歿,贈官五轉,致仕恩外,仍官一子。」庚辰,以趙與為觀文殿學士、兩淮安撫
 制置使兼知揚州。乙酉,出內庫緡錢三千萬助邊用。丙戌,以馬光祖為資政殿學士、沿江制置使、江東安撫、知建康府、行宮留守。己丑,詔蠲建康、太平、寧國、池州、廣德等處沙田租。壬辰,詔蠲漣水軍制司所收屯田租。乙未,發平糶倉米三萬,減直振在京民。辛丑,涪州報大元兵退。



 三月庚戌,詔印應雷、黃夢桂赴都堂稟議。命有司縣重賞募將士,毀藺市浮梁。癸丑,詔:蜀死節臣、雲頂山諸處將士,咸褒錄其後。丁巳,以呂文德為保康軍節度使、
 四川制置副使兼知重慶府。庚申,馬光祖奏大元兵自烏江還北。辛酉,雨土。



 夏四月甲戌朔,以段元鑒、楊禮堅守城壁,歿於王事,詔各贈奉國軍節度使,封「二字」侯,立廟賜額。致仕恩外,更官一子成忠郎。丁丑,以向士璧為湖北安撫副使、知峽州,兼歸、峽、施、珍、南平軍、紹慶府鎮撫使。甲申,詔:守合州王堅嬰城固守,百戰彌厲,節義為蜀列城之冠,詔賞典加厚。乙酉,知施州謝昌元自備緡錢百萬,米麥千石,築郡城有功,詔官一轉。乙未,詔賜夏
 貴溧陽田三十頃。丙申,以呂文德兼四川總領財賦。



 五月甲辰朔,城金州、開州。辛亥,雨雹。乙卯,達州上呂文德等戰功,詔遷補有功將士。丁巳,詔湖北諸郡去年旱潦饑疫,令江陵、常、澧、岳、壽諸州發義倉米振糶,仍嚴戢吏弊,務令惠及細民。乙丑,行開慶通寶錢。辛未,賜禮部進士周震炎以下四百四十二人及第、出身有差。婺州大水,發義倉米振之。



 六月甲戌,呂文德兵入重慶。詔諭四川軍民共奮忠勇,效死勿去,有功行賞,靡間邇遐。有能效
 順來歸,悉當宥過加恤。仍獎呂文德斷橋信道之功,命兼領馬軍行司。辛巳,以朱熠參知政事,饒虎臣同知樞密院事。丙戌,南平來報戰功。戊戌,詔申嚴海道防禦。己亥,詔獎諭賈似道。壬寅,以李庭芝直寶謨閣、湖北安撫副使兼知峽州。太白晝見。



 秋七月辛亥,太白入井。癸亥,蔡抗薨,贈少保,謚文肅。以知播州楊文、知思州田應庚守御勤勞,詔各官一轉。



 八月甲申,以濠州統制張斌柘塘之戰,歿於王事,贈官三轉,仍與一子下班祗應。乙酉,
 降人來言:大元憲宗皇帝崩於軍中。戊子,詔吳潛開閫海道,勤勞三年,屢疏求退,仍舊觀文殿大學士、判寧國府、特進、崇國公。辛卯,命呂文德兼湖北安撫使。庚子,太白犯權星、熒惑。



 九月壬子,賈似道表言大元兵自黃州沙武口渡江、,中外震動。己未,嗣濮王善騰薨。庚申,以吳潛兼侍讀、奉朝請,戴慶□端明殿學士、簽書樞密院事。下詔責己,勉諭諸閫進兵。壬戌,詔出內府緡錢千萬、銀五萬兩、帛五萬匹給宣司,緡錢五百萬、銀三萬兩、帛三
 萬匹給沿江副司犒師。詔:已命御史陳寅趣淮東調兵五萬,應援上流。癸亥,趙葵特進、觀文殿大學士,封衛國公,判慶元府、沿海制置使。命侍御史沈炎往沿江制置副司趣兵援鄂渚。再出內庫緡錢五百萬、銀二萬兩、帛二萬匹給兩淮制司,緡錢三百萬、銀萬兩、帛萬匹給沿江制司,以備軍賞。戊辰,太白犯熒惑。己巳,詔賈似道兼節制江西、二廣人馬,通融應援上流。庚午,合州解圍,詔王堅寧遠軍節度使,依前左領軍衛上將軍、興元府駐
 扎御前諸軍都統制兼知合州、節制軍馬,進封清水縣開國伯。



 冬十月辛未朔,丁大全罷,以觀文殿大學士判鎮江府。壬申,以吳潛為左丞相兼樞密使,進封相國公;賈似道為右丞相兼樞密使,進封茂國公,宣撫大使等如舊。癸酉,命趙葵為江東宣撫使,馬光祖移司江州應援鄂州,史巖之沿江制置副使,移司壽昌軍應援鄂州。丙子,改封吳潛為慶國公。丁丑,詔給還浙西提舉常平司歲收上亭戶沙地租二百萬,永勿復徵。庚辰,詔合州圍
 解,宣閫制臣及二三大將之功,宜加優賞。呂文德授檢校少師,李遇龍進三秩、權刑部侍郎,各賜金幣;將佐以下,進秩、賜金有差。詔自今月十一日始,避殿減膳徹樂。又詔:「比者蜀道稍寧,然干戈之餘,瘡痍未復,流離蕩析,生聚何資。咨爾旬宣之寄,牧守之臣,輕徭薄賦,一意撫摩,恤軍勞民,庶底興復。其被兵百姓,遷入城郭,無以自存者,三省下各郡以財粟振之。」壬午,御史陳寅言:知江州袁玠貪贓不悛,殘賤州邑。詔削玠五秩、竄南雄州。癸
 未,丁大全落職、罷新任。乙酉,雷。丙戌,以趙葵為沿江、江東宣撫使,置司建康,任責捍禦。癸巳,向士璧權兵部侍郎、湖南安撫使兼知潭州,任責廣西邊防。十一月壬寅,以朱熠權知樞密院事,饒虎臣、戴慶□並權參知政事。癸卯,呂文福帶遙郡防禦使、河南招撫使、知淮安軍。詔追毀袁玠出身以來文字,除名不敘,移萬安軍。戊申,詔求直言。辛亥,舟師戰滸黃洲。乙卯,詔趙葵授少保、觀文殿大學士、江東西宣撫使,進封益國公,其饒、信、袁、臨、撫、
 吉、隆興官軍民兵,並聽節制調遣,諮訪、罷行、黜陟皆得便宜行事。以緡錢五百萬、銀五萬兩給其用。丙辰,詔選精銳招信、泗州千人,揚州拱衛軍千人,安豐、濠州各千五百人,赴京聽調遣。庚申,夏貴入見,帝撫勞甚至。閏十一月甲戌,詔出內帑緡錢五千萬犒內外諸軍。丁丑,以向士璧為湖南制置副使,餘職仍舊,賜金帶。己卯,熒惑入氐。癸未,諸將陶林、文通進兵有功,詔林帶行遙郡刺史,文通轉武功大夫,賜銀有差。甲申,以印應雷為軍器
 監、淮西總領財賦兼江東轉運判官,呂文德檢校少保、京西湖北安撫使兼制置使、知鄂州兼侍衛馬軍都指揮使。己丑,皮龍榮兼資善堂翊善。庚寅,陶林奏沼山寺戰功。癸巳,向士璧連以功狀來上。乙未,詔降周震炎第四甲出身。丙申,賈似道表:大戰數合,皆有功。



 十二月己亥朔,賈似道言鄂州圍解,詔論功行賞。丁未,熒惑犯房宿、鉤鈐星。辛亥,詔改來年為景定元年。壬子,改封吳潛為許國公,賈似道為肅國公。



\end{pinyinscope}