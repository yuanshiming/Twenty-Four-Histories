\article{列傳第八十一}

\begin{pinyinscope}

 何郯吳中復從孫擇仁陳薦王獵孫思恭周孟陽齊恢楊繪劉庠朱京



 何郯,字聖從,本陵州人,徙成都。第進士,由太常博士為監察御史,轉殿中侍御史,言事無所避。王拱辰罷三司使守亳,已而留經筵,郯乞正其營求之罪。石介死,樞密使夏竦讒其詐,朝廷下京東體實,郯與張忭極陳竦奸狀,事得寢。楊懷敏以衛卒之亂,猶為副都知,郯又與忭及魚周詢論之。仁宗召諭云:「懷敏實先覺變,宜有所寬假。」郯等皆言不可,卒出之。郯爭辨尤力。帝曰:「古有碎首諫者,卿能之乎?」對曰:「古者君不從諫,則臣有碎首;今陛
 下受諫如流,臣何敢掠美而歸過君父。」帝欣納之。



 夏竦倡張貴妃之功,諫官王贄遂言賊根本起於皇後閣,請究其事,冀搖動中宮,而陰為妃地。帝以語郯,郯曰:「此奸人之謀也。」乃止不究。辣負罪不去,郯等奏出知河南,竦乞留京師。郯言:「佞人在君側,為善政累,願勿革前命。」竦遂行。



 時詔群臣陳左右朋邪、中外險詐,久而無所行。郯請閱實其是否,因言曰:「誠以待物,物必應以誠。誠與疑,治亂之本也,不可以一臣詐而疑眾臣,一士詐而疑眾
 士。且擇官者宰相之職,今用一吏,則疑其從私,故細務或勞於親決。分閫者將帥之任,今專一事,則疑其異圖,故多端而加羈制。博訪者大臣之體,今見一士,則疑其請托。相先後者士之常,今進其類,則疑為朋黨。君臣交疑,而欲天下無否塞之患,不可得矣。」



 都知王守忠以修祭器勞,遷景福殿使,給兩使留後奉。郯曰:「守忠勞薄賞重。舊制,內臣遙領止於廉察。今雖不授留後,而先給其祿;既得其祿,必得其官;若又從之,則何求不可。」既又詔
 許如正班。守忠移合門,欲綴本品坐宴,郯又言:「祖宗之制,未有內臣坐殿上者。此弊一開,所損不細。」守忠聞之,不敢赴。知雜御史闕,執政欲進其黨,帝以郯不阿權勢,越次用之。郯遍歷三院,有直聲。晚節頗回畏,因地震言陰盛臣強,以譏切韓琦;又乞召還王陶以迎合上意,由是聲名損於御史時也。



 以母老求西歸,加直龍圖閣、知漢州。將行,上疏言:「張堯佐緣後宮親,叨竊非據,外庭竊議,謂將處以二府。若此命一出,言事之臣,必以死爭之。
 倘罷堯佐則傷恩,黜言者則累德,累德、傷恩,皆為不可。臣謂莫若富貴堯佐而不假之以權,如李用和可也。」其後卒罷堯佐宣徽之命。進集賢殿修撰、知梓州,擢天章閣待制,還判銀臺司。時封駁之職廢,郯乞準故事,凡詔敕並由門下,從之。唐介出荊南,敕過門下,郯封還之,介復留諫院。遷龍圖閣直學士,為河東都轉運使。故相梁適帥太原,病不能事,內臣蘇安靜鈐轄兵馬,怙寵不法,皆劾奏之。



 歷知永興、河南。治平末,再知梓州。居三年,老
 而病,猶乞進用。神宗薄之,詔提舉成都玉局觀。從臣外祠自此始。遂以尚書右丞致仕。卒,年六十九。



 吳中復,字仲庶,興國永興人。父仲舉,仕李煜為池陽令。曹彬平江南,仲舉嘗殺彬所招使者。城陷,彬執之,仲舉曰:「世祿李氏,國亡而死,職也。」彬義而不殺。



 中復進士及第,知峨眉縣。邊夷民事淫祠太盛,中復悉廢之。廉於居官,代還,不載一物。通判潭州,御史中丞孫抃薦為監察御史,初不相識也。或問之,抃曰:「昔人恥為呈身御史,今
 豈有識面臺官耶?」遷殿中侍御史。彈宰相梁適,仁宗曰:「馬遵亦言之矣。」且問中復曰:「唐自天寶後治亂分,何也?」中復歷引姚、宋、九齡、林甫、國忠用舍以對。適罷,中復亦通判虔州,未至,復還臺。



 富弼主李仲昌開六漯河,內臣劉恢密告所斷岡與國姓上名同,賈昌朝陰助之,欲以搖弼。詔中復往治,促行甚急。中復言:「獄起奸臣,非盛世所宜有。」馳至,較其名,乃趙徵村也,亦無岡勢,獄以故得止。又彈宰相劉沆,沆罷。改右司諫,同知諫院。遷御史知
 雜事、戶部副使,擢天章閣待制,知澤州、瀛州,移河東都轉運使,進龍圖閣直學士、知江寧府。郵兵苦巡轄官苛刻,縶而鞭之。獄具,法不至死,中復以便宜戮首惡,流其餘,入奏為令。歷成德軍、成都府、永興軍。



 河北行青苗法,使者至,將先下州縣。中復檄之曰:「斂散自有期,今先事擾之,何也?」拒不聽,且以報。安撫司韓琦方疏諫青苗,錄其語以上。熙寧人並省郡邑,以永康為縣,中復言:「永康控威、茂,不可廢。」其後因夷竟復之。關內大旱,民多流亡。中
 復請加賑恤,執政惡之,遣使往視,謂為不實,削一階,提舉玉隆觀。起知荊南,坐過用公使酒,免。卒,年六十八。中復樂易簡約,好周人之急,士大夫稱之。從孫擇仁。



 擇仁字智夫,以父任,為開封雍丘主簿。元祐中,金水河堤壞,十六縣皆選屬庀役,得詣朝堂白事。宰相范純仁獨異之,曰:「簿領中乃有是人邪?」



 建中靖國初,畿內饑,多盜,以擇仁知太康縣。始至,召令賊曹曰;「民窮而盜,非天性也,我以靜鎮之。若亡命椎埋故犯,我一切誅之,毋得
 貸。」群盜相戒不入境。中貴人譚稹奴犯法,按致於理。稹羞恚造譖,徽宗召戶部郎中宋喬年往鞫。喬年,伉吏也,疾驅至。候者惶遽入白,擇仁著衣冠坐廡下。喬年慮囚擿隱,剔抉帑庾出入,不能得毫毛罪,乃歸傳舍。擇仁上謁,喬年迎笑曰:「所以來,為察君罪,顧乃得一奇士,吾今薦君矣。」居數日,召詣闕。



 方有事青唐,擢熙河路轉運判官,即以直秘閣為副使,從招討使王厚領兵深入,克蘭、廓城柵十三。加龍圖,進集賢殿修撰,為京畿都轉運使。
 鄭州城惡,受命更築之。或讒於帝曰;:「新城雜以沙土,反不如故,且速圮。」帝怒,密遣取塊城上,緘以來,令衛卒三投之,堅致如削鐵,讒不能售。遂拜戶部侍郎兼知開封府。故事,尹以三日聽訟,右曹吏十輩列庭下,自占姓名,一人云:「某人送某獄,某人當杖,某人去」,而尹無所可否。有竇鑒者,以捕盜寵,官諸司使,服金帶。擇仁視事,狃舊態來前,叱而械諸獄,一府大驚。賣珠人居民貨久不返,度事急,匿宦官楊戩第,擇仁跡取之,竄於遠。



 戩中以事,
 出為顯謨閣直學士、知熙州,從永興軍。走馬承受藍從熙言其擅改茶法,奪職,免。再閱歲,以徽猷閣待制領江、淮發運,還直學士、知渭州。以病提舉崇福宮,起知青州,不克拜,卒,年六十六。



 陳薦字彥升,邢州沙河人。舉進士,為華陽尉。盜殺人,棄尸民田。薦出驗,有以移尸告者。田主又殺其母。縣欲聞致殺二人,以逭薦失盜之責。薦不可,曰:「焉有誣人以自貰者邪!」已而獲盜。



 從韓琦定州、河東幕府。性木強簡澹,
 獨琦知之最深,每語人曰:「廉於進,勇於退,嫌疑間毫發不處,與人交久而不變,如彥升者,無幾也。」琦輔政,薦為秘閣校理、判登聞檢院、知太常禮院。



 英宗諸王出閣,選為記室參軍,直集賢院。穎王為皇太子,加右諭德;王即位,拜天章閣待制,進知制誥、知諫院。薛向首謀取橫山,功不成,薦請以漢王恢之罪罪向。楊繪論曾公亮用人不當,言既行而遷侍讀,罷諫職。薦曰;「此乃宰相欲杜繪言爾,所言是,宜責宰相。」疏入不報。



 除龍圖閣直學士、河
 北都轉運使。河決棗強,水官議於恩、冀、深、瀛之間築堤三百六十里,期一月就功,役丁夫八萬。薦曰:「河未能為數州害,民力方困,願以歲月為之。」還,判流內銓、太常寺。議學校貢舉法,請會三年貢士數均之諸路,計口察孝廉如漢制。權主管御史臺,言李定匿所生母喪,不宜為御史。罷臺事。又以議典禮不合,出知蔡州。召為寶文閣學士兼侍讀,進資政殿學士。



 屢求退,以為本州,命兩省燕餞資善堂。擢其子厚御史臺主簿。未幾,提舉崇福宮。
 卒,年六十九,贈光祿大夫。



 王獵,字得之,長垣人。累應進士不第,乃治生積錢,既而嘆曰:「此敗吾志也。」悉以班諸親族。慶歷用兵,詔求遺逸,範仲淹薦之,得出身為永興藍田主薄。府使之掌學。諸生有犯法者,獵自責數,以為教之不至,屏出之府。帥意其私,捕生下獄,獵前白曰:「此特年少不率教爾。致於理,不足以益美化,恐適貽士類辱。」帥悟而喜曰:「吾慮初不及此。」即釋生而待獵加敬。徙林慮令,縣依山,俗以搜田
 為生,不知學。獵立孔子廟,擇秀民誨之。漢杜喬墓在境中,往奠謁,建祠其旁。居官無絲發擾,吏民愛信,共目為清長官。



 入為吳王潭王宮教授、睦親廣親宅講書、諸王侍講。凡在京藩十二年,宗室無高卑少長,各得其歡如一日。英宗在邸,尊禮之;入為皇子,即拜說書;及即位,拜天章閣待制兼侍講。方議濮王稱,以問獵,獵不可。帝曰:「王待侍講厚,亦持此說邪?」對曰:「臣荷皇恩厚,不敢以非禮名號加於王,所以報王也。」帝大悟,自是不復議。以疾
 請謝事,不許。疾愈入見,帝喜曰:「侍講乃欲舍朕去乎?」



 神宗立,進龍圖閣直學士。求知襄州,未行,改滑州。自工部郎中為本曹侍郎致仕,給全奉。後八年卒,年八十。詔賻絹千匹,官其二孫,賜家人冠帔,人以為寵。



 孫思恭,字彥先,登州人。擢第後,即遭父喪,不肯復從官,二十年間才三書吏考。為宛丘令,轉運使以水災時調春夫,爭弗得,乃棄官去。吳奎薦其學行,補國子直講,加秘閣校理。事神宗藩邸為說書,又為侍講、直集賢院。以
 居中都久,力請補外,王奏留之。及即位,擢天章閣待制。



 思恭性不忤物,犯而不校,篤於事上。有所見,必密疏以聞。帝亦間訪以政。歐陽修初不知思恭,修出政府,思恭盡力救解。出知江寧府、鄧州,以疾移單州,管幹南京留司御史臺。卒,年六十一。



 思恭精關氏《易》,尤妙於《大衍》。嘗修天文院渾儀,著《堯年至熙寧長歷》,近世歷數之學,未有能及之者。



 周孟陽,字春卿,其先成都人,徙海陵。醇謹夷緩。第進士,
 為潭王宮教授、諸王府記室。



 英宗居環列,以其質厚,禮重之;會除知宗正寺,力辭,凡上十八表,皆孟陽為文。又從容陳古事以諷,英宗悚然起拜;及為皇子,愈堅臥不出。孟陽入見臥內,勸之曰:「天子知太尉賢,參以天人之助,乃發德音。何為堅拒如此?」英宗曰:「非敢徼福,以避禍也。」孟陽曰:「今已有此跡,設固辭不拜,使中人別有所奉,遂得燕安無患乎?」時中使趣召十輩,又命宗諤傾一宮往請,不能動,及是,意乃決。



 帝即位,命為皇子位說書,以
 嘗侍藩邸,固辭。加直秘閣、同知太常禮院。數引對,訪以時務。最後,召至隆儒殿,在邇英苑中,群臣未嘗至。人疑且大用,帝亦諭以不次進擢意。孟陽稱他人,使代己,乃遷集賢殿修撰、同判太常寺兼侍讀。神宗初立,入奏事,方升殿,帝望見慟哭,左右皆泣下。拜天章閣待制。卒,年六十九。詔特官其婿及子孫二人,除其家負官緡錢數萬。



 齊恢,字熙業,蒲陰人。唐宰相映之裔也。第進士,歷通判
 陳州,提點成都府路刑獄三年,徙河東。凡公帑格外饋餉之物,一無所受。單車而東,入為戶部判官。神宗出閣,精簡宮僚,韓琦薦其賢,以直昭文館,為穎王府翊善,進太子左諭德。帝即位,拜天章閣待制,知通進、銀臺司。出知相州,召知審官西院,糾察在京刑獄。卒,年六十六。恢居鄉里,恂恂稱君子;臨政府,明白簡約,不苛擾,所至人愛之。帝念舊僚,自諫議大夫特贈工部侍郎。



 楊繪,字元素,綿竹人。少而奇警,讀書五行俱下,名聞西
 州。進士上第,通判荊南。以集賢校理為開封推官,遇事迎刃而解,諸吏惟日不足,繪未午率沛然。仁宗愛其才,欲超置侍從,執政見其年少,不用。以母老,請知眉州,徙興元府。吏請攝穿窬盜庫縑者,繪就視之,蹤跡不類人所出入,則曰;「我知之矣。」呼戲沐猴者詰於庭,一訊具伏,府中服其明。在郡獄無系囚。



 神宗立,召修起居注、知制誥、知諫院。詔遣內侍王中正、李舜舉等使陜西,繪言:「陛下新即位,天下拭目以觀初政。館閣、臺省之士,朝廷所
 素養者不之遣,顧獨遣中人乎?」向傳範安撫京東西路,繪請易之,以杜外戚干進之漸。執政曰:「不然,傳範久領郡,有政聲,故使守鄆,非由外戚也。」帝曰:「諫官言是,斯可窒異日妄求矣。」曾公亮請以其子判登聞鼓院,用所厚曾鞏為史官。繪爭曰:「公亮持國,名器視如己物。向者公亮官越,占民田,為郡守繩治,時鞏父易占亦官越,深庇之。用鞏,私也。」帝為寢其命。繪亦解諫職,改兼侍讀,繪固辭,滕甫言於帝。帝詔甫曰:「繪抗跡孤遠,立朝寡援,不畏
 強御,知無不為。朕一見許其忠藎,擢置言職,信之亦篤矣。今日之除,蓋難與宰相並立於輕重之間,姑令少避爾,卿其諭朕意。」繪曰:「諫官不得其言則去,經筵非姑息之地。」卒不拜。未閱月,復知諫院,擢翰林學士,為御史中丞。



 時安石用事,賢士多謝去。繪言:「老成之人,不可不惜。當今舊臣多引疾求去:範鎮年六十有三、呂誨五十有八、歐陽修六十有五而致仕;富弼六十有八而引疾;司馬光、王陶皆五十而求散地,陛下可不思其故乎?」又言:「
 方今以經術取士,獨不用《春秋》,宜令學者以《三傳》解經。」免役法行,繪陳十害。安石使曾布疏其說。詔繪分析,固執前議,遂罷為侍讀學士、知亳州,歷應天府、杭州。再為翰林學士。



 議者欲加孔子帝號,繪以為非禮,又言不宜用遼歷改置閏,悉從之。繪常薦屬吏王永年,御史蔡承禧言其私通饋賂,坐貶荊南節度副使。詳在《竇卞傳》。數月,分司南京,改提舉太平觀,起知興國軍。元祐初,復天章閣待制,再知杭州。卒,年六十二。



 繪為吏敏強,主愛利,
 而受性疏曠,訖以是見廢斥。然表裏洞達,一出於誠,為範祖禹所咨重。為文立就,有集八十卷。



 劉庠,字希道,彭城人。八歲能詩。蔡齊妻以子,用齊遺奏,補將作監主簿。復中進士第,為高密廣平院教授。



 英宗求直言,庠上書論時事。帝以示韓琦,琦對之「未識」,帝益嘉重,除監察御史裏行。日食甫數日,苑中張具待幸,庠言非所以祗天戒,詔罷之。會聖宮修仁宗神御殿,甚宏麗。庠言:「天子之孝,在繼先志,隆大業,不在宗廟之靡。宜
 損其制,以昭先帝儉德。」奉宸庫被盜,治守藏吏。庠言:「皇城幾察厲禁,實近侍主之,當並按。」仁宗外家李珣犯銷金法,庠奏言,法行當自貴近始。帝不豫,儲嗣未正,庠拜疏謂:「太子,天下本。漢文帝於初元即為無窮計。穎王長且賢,宜亟立,使日侍禁中,閱四方章奏。」帝皆行之。



 神宗立,遷殿中侍御史,為右司諫。言:「中國御戎之策,守信為上。昔元昊之叛,五來五得志,海內為之困弊。今莫若示大信、舍近功,為國家長利。」奉使契丹。故事,兩國忌日不
 相避。契丹張宴白溝,日當英宗祥祭,庠丐免,契丹義而聽之。



 除集賢殿修撰、河東轉運使。庠計一路之產,鐵利為饒,請復舊冶鼓鑄,通隰州鹽礬,博易以濟用。又請募民入粟塞下,豫為足食。進天章閣待制、河北都轉運使。契丹侵霸州土場,或言河北不可不備。庠上五策,料其必不動,已而果然。大河東流,議者欲徙而北。內侍程昉希功,請益兵濟役。庠請遲以歲月,徐觀其勢而順導之。朝廷是其議。移知真定府,又為河東都轉運使,召知開
 封府。



 庠不肯屈事王安石。安石欲見之,戒典謁者曰:「今日客至勿納,惟劉尹來,即告我。」有語庠者曰:「王公意如此,盍一往見。」庠謂:「見之,何所言?自彼執政,未嘗一事合人情。脫問青苗、免役,將何辭以對?」竟不往。奏論新法,神宗諭之曰:「奈何不與大臣協心濟治乎?」庠曰:「臣子於君父各伸其志。臣知事陛下,不敢附安石。」會與蔡確爭廷參禮,遂以為龍圖閣直學士、知太原府。請復憲州募民子弟剽銳工技擊者,籍為勇敢,仿漢謫戍法,貰流以下
 罪徙實河外。



 契丹建牙雲中,遣騎涉內地,邊吏執之。契丹檄取紛然,又遣使議疆事。眾疑其造兵端,欲大為備。庠奏言:「雲朔歲儉,軍無見糧。契丹張形示強,造端首禍,曲在彼不在我,願勿聽。宜先諭以理,然後飭兵觀釁。」帝嘉使者辭順,訖以黃嵬山分水嶺立新疆。遭母喪,服終,知成都府。乞禁西山六州與漢人婚姻,勿蹈吐蕃取維州之害。徙秦州。坐失舉,降知虢州,移江寧府、滁州,徙永興軍。時西征無功,關內騷動。庠過關,力言虛內事外,恐
 搖根本,帝感納其忠。



 元祐初,加樞密直學士、知渭州。卒,年六十四。宣仁聞之曰:「帥臣極難得,劉庠可惜也。」庠有吏能,淹通歷代史,王安石稱其博。卒後,蘇頌論庠治平建儲之功,詔褒錄其子。



 朱京,字世昌,南豐人。父軾,有隱德。京博學淹貫,登進士甲科。教授亳州、應天府,入為太學錄。神宗數召見論事,擢監察御史。時中丞及同僚多罷去,京抗疏曰:「御史假之則重,略之則輕。今耳目之官,屢進屢卻,則言者不若
 靜默為賢,直者不若柔從為智。偷安取容,雖得此百數,亦何益國邪?」他日入見,帝勞之曰:「昨覽奏疏,所補多矣。」京風神峻整,見者憚之,目為真御史。



 初,臺臣奏事,必先移合門,得班乃入。京嘗以名聞,翌旦既入,會有先之者,不及對而退。帝問京安在,左右以告,詔趣之入,辰漏且盡,為留班以須。未幾,論大臣除擬有愛憎之私。中書言其失實,謫監興國軍鹽稅。歷太常博士、湖北、京西、江東轉運判官,提點淮西刑獄、司封員外郎。元符初,遷國子
 司業。京在元祐時,嘗為《幸太學頌》,或擿其語有及先朝者,京亦固辭不拜。徽宗初立,復命之,逾月而卒。



 論曰:何郯、吳中復,皆良御史也。郯出夏竦,阻王守忠,奸人庶幾少戢矣。中復恥識面臺官,其所守可見矣。薦之論李定,思恭之右歐陽修,繪請惜老成,庠不附新法,數子所見,何其同也。獵為令而興孔子廟,孟陽以教授而參決大計,此其卓然者乎。恢臨政簡約,無可議者。京持論端確,竟以去位,君子惜之。



\end{pinyinscope}