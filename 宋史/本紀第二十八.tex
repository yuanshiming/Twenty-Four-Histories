\article{本紀第二十八}

\begin{pinyinscope}

 高宗
 五



 五年春正月乙巳朔,日有食之。帝在平江府。金人去濠州。丁未,戒諸軍戰陳毋殺中原民籍充金兵者。命鬻官田宅輸錢專充軍費。戊申,進廬、泰二州守御官屬各一
 官。己酉,詔前宰執呂頤浩等十九人及行在職事官各條上攻戰備御措置綏懷之策。免淮南官吏去職之罪,仍令還任。承州水砦統領仲諒復入楚州。庚戌,張俊遣統領楊忠閔、王進夾擊金人於淮南岸,敗之,降其將程師回、張延壽。辛亥,淮東統制崔德明襲敗金兵於盱眙。召張浚赴行在。乙卯,浚入見。賞沿江監司、帥臣供億之勞,各進官一等。戊午,趣修建康行宮。己未,詔減淮南諸州雜犯死罪,釋流以下囚。庚申,置諸州軍教場,選兵專
 習弓弩,立格按試。辛酉,贈殿中侍御史馬伸左諫議大夫。韓世忠、劉光世、張俊入見。壬戌,以世忠為少保、淮東宣撫使,駐鎮江;光世少保、淮西宣撫使,駐太平;俊開府儀同三司、江東宣撫使,駐建康。甲子,酈瓊復光州,降其守許約。乙丑,罷淮南茶鹽提刑司,置提點兩路公事官一員,兼領刑獄、茶鹽、漕運、市易事。淮西要會州軍並置市易務。戊辰,詔川、陜宣撫司招諭陷賊官民。庚午,命王進合江西、廣東諸將兵討周十隆。海賊朱聰犯廣州,又
 犯泉州。壬申,劉光世、韓世忠、張俊入辭,命升殿,以光世、世忠有隙,賜酒諭釋之,皆感激奉詔。癸酉,偽齊知亳州馬秦犯光州,權州事王萃率兵拒之。是月,金主晟殂,旻之孫但立。岳飛自池州入朝。二月丙子,以飛為鎮寧、崇信軍節度使。命常州布衣陳得一造新歷。丁丑,帝發平江。戊寅,遣權太常少卿張銖奉迎太廟神主於溫州。壬午,帝至臨安,進扈從官吏秩一等。丙戌,以趙鼎為左僕射,張浚右僕射,並同中書門下平章事兼知樞密院事、
 都督諸路軍馬。岳飛為荊湖南北、襄陽府路制置使,將兵平湖賊楊太。丁亥,吳璘、楊政攻拔秦州,執偽齊守胡宣,金帥撒離曷來援,政復擊敗之。己丑,詔建太廟。壬辰,命張浚詣江上措置邊防,詔諭諸路宣撫制置司,示以專任之旨。以右司諫趙霈論奏得體,賜三品服。丁酉,進執政官秩一等,以賞防秋之功。戊戌,詔淮南宣撫司撫司淮北來歸官吏軍民。己亥,直史館範沖上《神宗實錄考異》。庚子,詔翰林學士孫近、胡交修類編臣僚條具利
 害章疏以聞。甲辰,蠲湖南路上供三年。是月,偽齊商元寇信陽軍,守臣舒繼明被禽,死之。



 閏月乙巳朔,雨雹。丁未,胡松年罷。戊申,雪。己酉,留四川上供銀帛就充軍費。乙卯,以孟庾、沉與求並兼權樞密院事。丙辰,並諸路提舉常平入茶鹽司。罷福建鑄錢,令轉運、坑冶司辦集。丁巳,撒離曷欲犯秦州,吳玠遣部將牛皓伺之,遇於瓦五穀,戰死。癸亥,海賊陳感犯雷州,官軍屢敗。丁卯,王□燮罷。命戶部尚書章誼措置財用,孟庾提領,號總制司。命川、
 陜宣撫司幕僚攝司事,仍權節制軍馬。戊辰,置路分總管,以處閑退武臣。辛未,復置宗正丞,掌修屬籍。再蠲荊南府、歸、峽二州、荊門、公安二軍歲貢上供二年。



 三月甲戌朔,以王□燮貪縱不武,敗師誤國,責授濠州團練使。丙子,遣樞密計議官呂用中等分使兩浙、江東、西路檢察經、總制司財用。丁丑,詔侍從至監察御史、館職已上,在內館職、在外侍從官、監司、帥守,各舉所知充監司、守令,尋命館職專舉縣令。己卯,以韓世忠兼鎮江府宣撫使,
 劉光世兼太平州宣撫使。壬午,以都督府參議軍事邵溥兼權川、陜宣撫副使。罷御前軍器所提舉官,並隸工部。壬辰,命廣東、福建路招捕朱聰。乙未,初榷鉛、錫。張浚親討湖賊。丁酉,復移浙西安撫司於臨安府。庚子,罷饒州牧馬監。



 夏四月丙午,貴池縣丞黃大本坐枉法贓,杖脊、刺配南雄州。丁未,遣司農丞蓋諒持詔撫諭川、陜。召解潛赴行在,王彥知荊南府,諸鎮撫使至是盡罷。戊申,太廟神主至自溫州。己酉,以審量濫賞,追左銀青光祿
 大夫王序八官及職名,仍改正出身。庚戌,詔內侍遇特恩轉宮,止武功郎。壬子,訪得周後柴叔夏襲封崇義公。戊午,奉安太廟神主。己未,更免役保正長法。甲子,太上皇帝崩於五國城。丙寅,帝即射殿行朝獻景靈宮之禮,始以惠恭皇后祔祭。募民耕營田,官給牛、種。庚午,省四川添差官。辛未,以諸路稅賦畸零增收錢專充上供。是月,龍圖閣直學士致仕楊時卒。



 五月乙亥,初謁太廟。庚辰,命邵溥、吳玠裁省四川冗官浮費。辛巳,名行宮新作
 書院為資善堂。遣何蘚等奉使金國,通問二帝。中書舍人胡寅言,國家與金世仇,無通使之義。張浚奏:「使事兵家機權,後將闢地復土,終歸於和,未可遽絕。」乃遣行。丁亥,立殘破州縣守令勸民墾田及拋荒殿最格。己丑,以孟庾知樞密院事。壬辰,召張浚還行在。丁酉,詔浚提舉詳定一司敕令。戊戌,以貴州防禦使瑗為保慶軍節度使,封建國公。徽猷閣待制範沖兼資善堂翊善,起居郎朱震兼贊讀。以盛暑,命監司行部慮囚。己亥,岳飛軍次
 鼎州。庚子,周十隆降。辛丑,命川、陜訪求元祐黨人子孫。



 六月甲辰,封武經大夫令矼為安定郡王。湖賊楊欽、全琮、劉詵相繼率眾詣岳飛降。乙巳,名新歷曰《統元》。丁未,並饒州鑄錢司於虔州。己酉,命建國公瑗出就資善堂聽讀,拜範沖、朱震。出內帑錢賜宗室貧者。壬子,復省淮南州縣冗官。癸丑,以久旱,減膳、祈禱。禁諸路科率,自租稅、和市、軍須外皆罷。岳飛急攻湖賊水砦,賊將陳□降,楊太赴水死,餘黨劉衡等皆降。飛急擊夏誠,斬之。丁巳,
 湖賊黃誠斬楊太首,挾鐘子儀、周倫詣都督府降,湖湘悉平,得戶二萬七千,悉遣歸業。戊午,減福建貢茶歲額之半。庚申,以旱罷諸路檢察財用官。丁卯,以賊平,免沿湖民前二年逋租。己巳,罷福建諸州槍杖手。



 秋七月壬申朔,以仇悆為沿海制置使。甲戌,免蘄州上供及租稅三年。戊寅,獎諭岳飛,撫勞將士,趣張浚還朝。己卯,孟庾罷,以沉與求兼權樞密院事及措置財用。壬午,以金、均、房州隸襄陽府路。偽齊兵寇湖陽縣,執唐州守臣高青,
 復釋之。丁亥,賜宇文虛中家福建田十頃。甲午,詔殘破州縣親民官,計到、罷之日戶口考殿最。韓世忠復鎮淮軍,禽偽齊守王拱。丙申,蠲湖南路上供米三年及秋租之半。丁酉,置高峰、王口二砦都巡檢使,益兵戍之。



 八月壬寅朔,罷荊南營田司,令安撫司措置官兵耕種。甲辰,定館職額為十八員。壬子,詔淮南山水砦都巡檢各聽守令節制。癸丑,蠲福建州軍供撥常平錢米。己未,下詔示章惇、蔡卞詆誣宣仁聖烈皇后之罪,追貶惇昭化軍
 節度副使,卞單州團練副使,子孫不許在朝。命廣宮學,教內外宗子。辛酉,詔淮南、襄陽府等路團結民社。丙寅,以諸盜平,減湖、廣、江西二十二州雜犯死罪,釋徒、杖以下囚。海賊朱聰降,命補水軍統領。是月,偽齊陷光州。



 九月辛未朔,罷總制司所增收頭子等諸色錢。乙亥,賜禮部進士汪洋以下二百二十人及第、出身。唱名始遵故典,令館職侍立殿上。壬午,加岳飛檢校少保。偽齊兵寇固始縣,統領華旺拒戰,卻之,尋復光州。甲申,命沿海州
 軍籍海舶,分守要害。乙酉,趙鼎上《重修神宗實錄》。壬辰,詔元符上書邪等範柔中等二十七人各官一子。以解潛部兵三千隸馬軍司。甲午,周十隆復叛,犯汀州。戊戌,遣統領王進、李貴討之。



 冬十月庚戌,張浚入見。乙卯,以席益為四川制置大使,位宣撫副使上,州軍兵馬並隸大使司,邊防重事仍令宣撫司處置。李綱為江西制置大使,呂頤浩為湖南制置大使。戊午,詔川、陜類試合格第一人依殿試第三人例推恩,餘並同賜進士出身,特
 奏名進士命宣撫選官試時務策。澧州賊雷德進降。乙丑,偽齊兵寇漣水軍,韓世忠遣統制呼延通等逆擊,敗之。



 十一月庚午朔,初置節度使以下金字牙符,命都督府掌之,給將帥立戰功者。命州縣賣戶帖以助軍費。癸酉,詔守臣死節昭著者,毋限品秩,並賜謚。乙亥,征和靖處士尹焞於涪州,命為崇政殿說書。戊寅,郊。辛巳,復置淮南提舉鹽事官。壬午,出宮女三十人。甲申,權減宰執及行在官吏奉。乙酉,以趙開為四川都轉運使。丙戌,命
 張浚視師荊、襄、川、陜。戊子,知衡州裴廩坐調夫築城凍死二千餘人,除名、嶺南高州編管。乙未,出內帑綿絹賜宗室。丁酉,罷催稅戶長。十二月己亥朔,以岳飛為荊湖南北、襄陽府路、蘄黃州招討使。楊沂中權主管殿前司,並統神武中軍。庚子,改神武四軍及巡衛軍號行營五護軍。辛丑,以都督府兵隸三衙。命左右司、樞密院檢詳官參考中興已行條例,修為定法。乙巳,禁服用翠羽。己酉,免侍從官輪對。庚戌,汰橫江水軍三之一。癸丑,命兩
 淮、川陜、荊襄、荊南諸帥府參謀官各一員提點屯田。癸亥,禁川陜州縣官悉用川陜人。丙寅,都督府遣參議軍事劉子羽、主管機宜文字熊彥詩撫諭川陜,且察邊備虛實。戊辰,夜雨雹。



 六年春正月辛未,蠲貧民戶帖錢之半,無物產者悉除之。癸酉,命給事中、中書舍人甄別元祐黨籍。乙亥,以內重外輕,命省臺、寺監及監司、守令居職及二年者,許更迭出入除擢。丁丑,詔凡入粟補官者,毋授親民、刑法之
 職。壬午,賜宗子伯玖名琚,為和州防禦使。罷綿州宣撫副使,命吳玠專治兵事。罷御史平反刑獄賞。丙戌,張浚視師荊襄,入辭。己丑,安定郡王令矼薨。庚寅,還預借坊場錢。辛卯,詔監司、帥臣慢令失職者,令張浚黜陟以聞。甲午,振江、湖、福建、浙東饑民,命監司、帥臣分選僚屬及提舉常平官躬行檢察。戊戌,命鬻通直郎、閣門宣贊舍人以下官。



 二月庚子,以諸路宣撫制置大使並兼營田大使,宣撫副使、招討安撫使並兼營田使。壬寅,雨雪。改
 江、淮屯田為營田。甲辰,置行在交子務,印交子錢引給諸路,令公私同見錢行用。戊申,岳飛入見。復以襄陽府路為京西南路。辛亥,詔張浚暫赴行在奏事。甲寅,以兵部尚書、都督府參謀折彥質簽書樞密院事。乙卯,韓世忠引兵攻宿遷縣,統制呼延通與金兵戰,敗之,禽其將孛堇牙合。澧州賊徒伍俊殺雷德進,持其首詣鼎州降。丙辰,韓世忠圍淮陽軍。復置諸路市易務。戊午,命楊沂中以兵萬人聽都督行府調遣。己未,遣戶部侍郎劉寧
 止如鎮江府,總領三宣撫司錢糧。辛酉,兀□救淮陽,韓世忠引兵歸楚州。壬戌,以折彥質兼權參知政事。癸亥,沉與求罷。李綱入見。是月,張浚至江上會諸將議事,命張俊進屯盱眙。



 三月戊辰朔,初收官告綾紙錢。名金、均、房州民兵曰保勝,又命招刺三千人,賜名必勝軍。己巳,以韓世忠為京東、淮東路宣撫處置使,岳飛為京西、湖北路宣撫副使。辛未,蠲旱傷州縣民積欠錢帛租稅。己卯,趣岳飛如鄂州措置軍事。辛巳,以樞密副都承旨馬擴
 為沿海制置副使。壬午,金、齊兵犯漣水軍,韓世忠擊敗之。壬辰,寬四川災傷州縣戶帖錢之半。



 夏四月戊戌朔,湖南賊黃旺犯桂陽監。甲辰,偽齊兵陷唐州,團練判官扈舉臣、推官張從之等皆死。岳飛以母喪去官。丙午,詔飛起復。己酉,詔文武臣僚能決勝強敵恢復境土者,賜功臣號。庚戌,始訓諸宗子名。甲寅,賞淮陽功,呼延通等進官有差,余受賞者凡萬七千人。劉光世遣副統制王師晟、酈瓊襲偽齊兵於劉龍城,破之,禽其統制華知剛。
 己未,命福建安撫司發水軍討海賊鄭慶。辛酉,禁四川伐並邊山林。甲子,以韓世忠為橫海、武寧、安化軍節度使,號揚武翊運功臣。除商旅緡錢稅。丙寅,復行在官吏奉。蠲東京民渡淮南商販之稅。



 五月戊辰朔,禁以鹿胎為冠。癸酉,詔未經上殿臣僚,先令三省審察,然後引對。戊寅,以四川監司地遠玩法,應有違戾,令制置大使按劾。壬午,詔大理寺議獄不合,即詣刑部關決,刑部不能定,同赴都堂稟議。賜吳玠四川戶帖錢十萬緡犒軍。癸
 未,禁淮南州縣收額外雜色租。乙酉,改交子為關子,罷交子務。庚寅,以劉光世為保靜、寧武、寧國軍節度使。壬辰,以張俊進屯盱眙,改崇信、奉寧軍節度使。甲午,禁銷錢及私鑄銅器。丙申,詔監司慮囚不能遍及者,聽遣官,著為令。



 六月乙巳夜,地震。乙酉,求直言。甲寅,張浚渡江,撫淮上諸屯。命劉光世自當塗進屯廬州,岳飛自九江進屯襄陽,楊沂中屯泗州。戊午,詔兩淮沿江守臣並以三年為任。辛酉,封集英殿修撰令廣為安定郡王。



 秋七月壬申,以司農少卿樊賓提領營田公事。癸未,詔張浚暫赴行在。癸巳,罷川陜便宜差遣監司、守貳。以金州隸川陜路,均、房二州隸京西南路。郭浩為永興軍路經略安撫使兼知金州,閣門宣贊舍人邵隆知商州,聽浩節制,經理商、虢。是月,劉光世復壽春府。



 八月己亥,範宗尹薨。庚子,賜左司諫陳公輔三品服。癸卯,以徽猷閣直學士李迨為四川都轉運使。甲辰,詔諭將士將親征。岳飛遣統制牛皋破偽齊鎮汝軍,禽其守薛亨。乙巳,命
 權殿前司解潛等帥精兵扈從,主管步軍司邊順留兵守臨安,知臨安府梁汝嘉為巡幸隨軍都轉運使。丁未,以秦檜為醴泉觀使兼侍讀、行宮留守,孟庾提舉萬壽觀兼侍讀、同留守。戊申,岳飛遣將楊再興復西京長水縣。己酉,命秦檜、孟庾權參決尚書省、樞密院事。庚戌,蠲虔州殘破諸縣逋負、梅州夏秋兩稅,聽廣東經略安撫司便宜措置盜賊。辛亥,奉神主發臨安。丁巳,權罷經筵進講。己未,預借江、浙民來年夏稅絁帛,折米輸官。庚申,
 增給職事官米月三斛。是月,張俊城盱眙,進屯泗州。岳飛及偽齊李成、孔彥舟連戰至蔡州,克之,偽守劉永壽舉城降。



 九月丙寅朔,帝發臨安。岳飛遣統制王貴、郝晸、董先復虢州盧氏縣。癸酉,帝次平江。戊寅,命職事官日一員輪對。壬午,岳飛以孤軍無援,復還鄂州。癸未,權奉安神主於平江能仁寺。戊子,以戶部郎官霍蠡總領岳飛軍錢糧。庚寅,張浚入奏,復如鎮江。辛卯,立賊徒相招首罪賞格。賞鎮淮軍功,進統制王德等官。是月,劉豫聞
 親征,告急於金主但求援,但不許,豫自起兵三十萬,命子麟趣合肥,侄猊出渦口,引兵分道入寇。



 冬十月丙申,招西北流寓人補闕額禁軍。丁酉,裁定淮南路租額。劉麟寇淮西,張俊遣楊沂中、張宗顏等分兵御之。戊戌,沂中至濠州,劉光世已棄廬州而南,浚遣人督還,光世不得已駐兵應沂中,遣統制王德、酈瓊及賊將崔皋、賈澤、王遇戰,皆敗之。賊兵攻壽春府芍陂砦,守臣孫暉拒戰,又敗之。辛丑,罷四川監酒官百餘員。壬寅,以梁汝嘉兼
 浙西、淮東沿海制置使,前護副軍都統制王彥副之。癸卯,趙鼎請降敕諭張浚,令光世、沂中及張俊全軍引還,為防江之計。甲辰,又詔浚督將士僇力破賊,皆未達。劉猊犯定遠縣,沂中進戰,大敗之於藕塘,猊挺身遁,麟在順昌聞之,拔砦去。劉光世遣王德及沂中追麟,至南壽春而還。孔彥舟亦解光州圍而去。戊申,命解潛遣兵千人守青龍港口。癸丑,張俊、楊沂中引兵攻壽春府,不克而還。乙卯,詔諸軍所俘人民給錢米遣歸。丁巳,惠州軍
 賊曾袞作亂。庚申,摧鋒軍統制韓京募敢死士,夜襲破之,袞尋出降。壬戌,日中有黑子沒。



 十二月申午朔,詔降廬、光、濠等州死罪,釋流以下囚。召秦檜赴行在。張浚入見,請幸建康;趙鼎請還臨安。戊戌,韓世忠攻淮陽軍,及金人戰,敗之。辛丑,城南壽春府。壬寅,趙鼎罷。遣右司員外郎範直方宣諭川、陜,撫問吳玠將士。甲辰,命都督府參議軍事呂祉如建康,措置移蹕。丙午,折彥質罷。丁未,賞淮西功,加張俊少保,改鎮洮、崇信、奉寧軍節度使,楊
 沂中保成軍節度使、殿前都虞候。戊申,命秦檜赴講筵供職,孟庾為行宮留守。辛亥,以資政殿學士張守參知政事,兼權樞密院事。丙辰,以呂頤浩為浙撫西安制置大使、判臨安府。丁巳,以劉光世為護國、鎮安、保靜軍節度使。戊午,詔凡因民事被罪者,不許親民。己未,命辰、沅、靖、澧四州以閑田募刀弩手,三千五百人為額。右司諫陳公輔乞禁程氏學。詔:「士大夫之學宜以孔、孟為師,庶幾言行相稱,可濟時用。」庚申,以安化郡王王稟死節太
 原,賜其家田十頃。辛酉,以山陰、諸暨等四十縣為大邑,並命堂除。



 七年春正月癸亥朔,帝在平江,下詔移蹕建康。蠲無為軍稅役一年。置建康御前軍器局。丁卯,賞張浚以破敵功,遷特進。己巳,發米萬石濟京東、陜西來歸之民。張浚入見。甲戌,罷都督府諸州市易官。丁丑,解潛罷,以劉錡權主管馬軍司,並殿前步軍司公事。庚辰,築採石、宣化渡二城。癸未,以翰林學士陳與義參知政事,資政殿學
 士沈與求同知樞密院事。詔廣西帥臣訓練土丁、保丁。乙酉,復置樞密使、副,知院以下仍舊,張浚改兼樞密使。丙戌,禁諸軍互納亡卒。西蕃三十八族首領趙繼忠等來歸。丁亥,以秦檜為樞密使。何蘚、範寧之至自金國,始聞上皇及寧德皇后崩。己丑,帝成服,下詔降徒囚,釋杖以下。辛卯夜,東北有赤氣如火。



 二月癸巳朔,日有食之。百官七上表,請遵以日易月之制。徽猷閣待制、知嚴州胡寅請服喪三年,衣墨臨戎,以化天下。帝欲遂終服,而
 張浚連疏論喪服不可即戎,遂詔外朝勉從所請,宮中仍行三年之喪。丙申,太平州火。丁酉,鎮江府火。庚子,遣王倫等使金國迎奉梓宮。岳飛入見。辛丑,以日食,求直言;以久旱,命諸州慮囚。乙巳,詔凡闢舉官犯贓罪,罪及所舉官。丙午,吳玠置銀會子於河池。丁未,詔席益募陜西、河東、河北兵二千,部送行在充扈衛。癸丑,雨雹。丙辰,始御便殿。果州守臣宇文彬等進《禾登九穗圖》,俱奪一官,罷之。丁巳,以岳飛為太尉、湖北京西宣撫使。己未,帝
 發平江。



 三月癸亥朔,次丹陽,韓世忠入見,命世忠扈從,岳飛次之。甲子,次鎮江,楊沂中入見,命沂中總領彈壓巡幸事務。乙丑,蠲駐蹕及經從州縣積年逋賦。丁卯,以吏部侍郎呂祉為兵部尚書、都督府參謀軍事。辛未,帝至建康。壬申,詔尚書省常程事從參知政事分治。癸酉,減建康流罪以下囚,蠲建康府、太平、宣州逋賦及下戶今年身丁錢。岳飛乞並統淮西兵以復京畿、陜右,許之,命飛盡護王德等諸將軍。既而秦檜等以合兵為疑,事
 遂寢。戊寅,手詔撫勞將士。進沉與求知樞密院事。己卯,尊宣和皇后為皇太后。庚辰,以王彥兵隸侍衛馬軍司。呂頤浩為少保兼行宮留守。孟庾罷。甲申,以劉光世為少師、萬壽觀使,以其兵隸都督府,張浚因分為六軍,命呂祉節制。乙酉,賜光世第於建康府。丁亥,命虔、吉、南安軍諸縣各募土兵百人,責知縣訓練,防禦盜賊。是春,廣西大饑,李實變為桃。



 夏四月癸巳,築太廟於建康,以臨安府太廟為聖祖殿。戊戌,修浚建康城池。丁未,岳飛乞
 解官持餘服,遂棄軍去,詔不許。戊申,日中有黑子。庚戌,以張浚累陳岳飛積慮專在並兵,奏牘求去,意在要君,遂命兵部侍郎兼都督府參議軍事張宗元權湖北、京西宣撫判官,實監其軍。壬子,張浚如太平州、淮西視師。庚申,以信陽軍隸京西路。罷淮南提點司,東西兩路各置轉運兼提點刑獄、提舉茶鹽常平事。



 五月丁卯,詔李綱趣捕虔、吉諸盜。壬申,命禮官舉文宣王、武成王、熒惑、壽星、岳鎮、海、瀆、農、蠶、風、雷、雨師之祀。甲戌,以胡安國提
 舉萬壽觀兼侍讀,趣赴行在,未至而罷。癸未,以酈瓊為行營左護軍副都統制。甲申,初試樞密院都督府效士。乙酉,命侍從官通舉材堪知縣者二十人。丙戌,偽齊陷隨州。己丑,禁四川增印錢引。



 六月辛卯朔,改上惠恭皇后謚曰顯恭皇后。岳飛入見。壬辰,命歲辰戌月祀大火,配以閼伯。乙未,罷江、淮營田司,令諸路安撫、轉運司兼領其事。丙申,以《重修神宗實錄》去取未當,命史館復加考訂。丁酉,岳飛引過自劾,詔放罪,慰諭之。戊戌,命劉錡
 兼都督府咨議軍事,率兵戍廬州。乙巳,沉與求薨。召王德以所部兵赴行在。遣呂祉如淮西撫諭諸軍。丙辰,詔吳玠、李迨共議四川經費,贍軍恤民。岳飛復職。



 秋七月戊辰,詔侍從各舉可任監司、郡守者一二人。癸酉,以旱,禱於天地、宗廟、社稷。甲戌,嗣濮王仲湜薨。癸未,以久旱,命中外臣庶實封言事。甲申,蠲諸路民積年逋租。以建康疫盛,遣醫行視,貧民給錢,葬其死者。命疏決滯獄。乙酉,詔即建康權正社稷之位。戊子,詔戶部長貳迭出巡
 按諸路,考究財賦利病,違者劾之。己丑,詔諸路歸業民墾田,及八年始輸全稅。



 八月乙未,以張俊為淮西宣撫使,駐盱眙;楊沂中為淮西制置使,主管侍衛馬軍司劉錡副之,並駐廬州。命酈瓊率兵赴行在。戊戌,瓊叛,殺中軍統制張景等,執呂祉及趙康直、趙不群,以兵四萬人奔劉豫。辛丑,手詔赦廬州屯駐行營左護軍。壬寅,酈瓊引兵至淮,殺祉及康直,釋不群,使還。劉錡、吳錫至廬州,以兵追之不及,命張宗元往招之。張浚乞去位。甲辰,以
 趙鼎為萬壽觀使兼侍讀。甲寅,詔命官犯贓,刑部不得擅黥配,聽朝廷裁斷。乙卯,賜岳飛軍錢十萬緡。招歸正復業人耕湖北、京西閑田。



 九月甲子,上太上皇帝謚曰聖文仁德顯孝皇帝,廟號徽宗,皇后曰顯肅皇后。丁卯,韓世忠、張俊入見,乃命俊自盱眙移屯廬州。壬申,張浚罷。癸酉,命參知政事輪日當筆,權三省事,更不分治常程。罷都督府。甲戌,以臺諫累疏,落張浚觀文殿大學士,仍領宮祠。丙子,復以趙鼎為尚書左僕射、同中書門下
 平章事兼樞密使。戊寅,以廬州、壽春府民遭酈瓊虜掠,蠲租稅一年。己卯,朝獻聖祖於常朝殿。庚辰,朝饗太廟。辛巳,合祭天地於明堂,大赦。召劉光世赴行在。戊子,禁諸路進羨餘。以劉錡知廬州兼淮西制置副使。



 冬十月庚寅朔,詔仍舊開經筵。辛卯,命後省官看詳上書有可採者,條上行之。丁酉夜,敕張浚安置嶺表。戊戌,趙鼎累請浚母老,改永州居住。偽齊犯泗州,守臣劉綱擊走之。丙午,命戶部郎官薛弼、霍蠡同總領江西、湖、廣五路財
 賦。壬子,統制呼延通、王權等襲擊金人於淮陽軍,敗之,丁巳,詔六參日,輪行在百官一員轉對。



 閏月癸亥,贈趙康直徽猷閣待制。乙丑,蠲江東路月樁錢萬緡。發米二萬石振京西、湖北饑民。丙寅,尹焞入見,命為秘書郎兼崇政殿說書。甲戌,始作徽宗皇帝、顯肅皇后神主。庚辰,韓世忠引兵渡淮,逆擊金人於劉冷莊,敗之。辛巳,李綱罷。癸未,復漢陽軍。是月,張俊棄盱眙,引兵還建康。



 十一月丙申,賜吳玠犒軍錢百五十萬緡。丁酉,以知溫州李
 光為江西安撫制置大使。丁未,金帥撻懶、兀□入汴京,執偽齊劉豫,廢為蜀王。癸丑,詔來春復幸浙西。是月,偽齊知臨汝軍崔虎詣岳飛降。十二月庚辰,復置都大提舉四川茶馬監牧官。丁卯,祔徽宗皇帝、顯肅皇后神主於太廟。庚午,以解潛權主管馬步軍司,命韓世忠留屯楚州,屏蔽江、淮。己卯,詔內外大將及侍從官,舉武臣智略器局堪帥守謀議官者。癸未,王倫等使還,入見,言金國許還梓宮及皇太后,又許還河南諸州。甲申,城泗州。
 丁亥,復遣王倫等奉迎梓宮。是冬,吳玠遣裨將馬希仲攻熙州,鄭宗、李進攻鞏州,不克,宗死於城下,希仲遁還,玠斬以徇。



\end{pinyinscope}