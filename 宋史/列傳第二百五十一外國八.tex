\article{列傳第二百五十一外國八}

\begin{pinyinscope}

 唃廝囉 董氈 阿裏骨 瞎征 趙思忠



 吐蕃本漢西羌之地,其種落莫知所出。或雲南涼禿發利鹿孤之後,其子孫以禿發為國號,語訛故謂之吐蕃。
 唐貞觀後,常來朝貢。至德後,因安、史之亂,遂陷河西、隴右之地。大中三年,其國宰相論恐熱以秦、原、安樂及石門等七關來歸。四年,又克成、維、扶三州。五年,其國沙州刺史張義潮以瓜、沙、伊、肅十一州之地來獻。唐末,瓜、沙之地復為所隔。然而其國亦自衰弱,族種分散,大者數千家,小者百十家,無復統一矣。自儀、渭、涇、原、環、慶及鎮戎、秦州暨於靈、夏皆有之,各有首領,內屬者謂之熟戶,餘謂之生戶。涼州雖為所隔,然其地自置牧守,或請命
 於中朝。



 天成中,權知西涼府留後孫超遣大將拓拔承誨來貢,明宗召見,承誨云:「涼州東距靈武千里,西北至甘州五百里。舊有鄆人二千五百為戍兵,及黃巢之亂,遂為阻絕。超及城中漢戶百餘,皆戍兵之子孫也。其城今方幅數里,中有縣令、判官、都押衙、都知、兵馬使,衣服言語略如漢人。」即授超涼州刺史,充河西軍節度留後。乾祐初,超卒,州人推其土人折逋嘉施權知留後,遣使來貢,即以嘉施代超為留後。



 涼州郭外數十里,尚有漢
 民陷沒者耕作,餘皆吐蕃。其州帥稍失民情,則眾皆嘯聚。城內有七級木浮圖,其帥急登之,紿其眾曰:「爾若迫我,我即自焚於此矣。」眾惜浮圖,乃盟而舍之。周廣順三年,始以申帥厚為河西節度。帥厚初至涼州,奏請授吐蕃首領折逋支等官,並從之。顯德中,師厚為其所迫,擅還朝,坐貶。涼州亦不復命帥。



 建隆二年,靈武五部以橐駝良馬致貢,來離等八族酋長越嵬等護送入界,敕書獎諭。秦州首領尚波於傷殺采造務卒,知州高防捕係
 其黨四十七人,以狀聞。上乃以吳廷祚為雄武軍節度代防安輯之,令廷祚齎敕書賜尚波於等曰:「朝廷製置邊防,撫寧部落,務令安集,豈有侵漁?曩者秦州設置三砦,止以采取材木,供億京師,雖在蕃漢之交,不妨牧放之利。汝等占據木植,傷殺軍人。近得高防奏汝等見已拘執,聽候進止。朕以汝等久輸忠順,必悔前非,特示懷柔,各從寬宥。已令吳廷祚往伸安撫及還舊地。所宜共體恩旨,各歸本族。」仍以錦袍銀帶賜之,尚波於等感悅。
 是年秋,乃獻伏羌地。



 乾德四年,知西涼府折逋葛支上言:「有回鶻二百餘人、漢僧六十餘人自朔方路來,為部落劫略。僧雲欲往天竺取經,並送達甘州訖。」詔褒答之。五年,首領閭逋哥、督廷、督南、割野、麻裏六人來貢馬。開寶六年,涼州令步奏官僧吝氈聲、逋勝拉蠲二人求通道於涇州以申朝貢,詔涇州令牙將至涼州慰撫之。八年,秦州大石、小石族寇土門,略居民,知州張炳擊走之。



 太平興國二年,秦州安家族寇長山,巡檢使韋韜擊走
 之。三年,秦州諸族數來寇略三陽、┒穰、弓門等砦,監軍巡檢使周承晉、任德明、耿仁恩等會兵擊敗之,斬首數十級,腰斬不用命卒九人於境上。太宗乃詔曰:「秦州內屬三族等頃慕華風,聿求內附,俾之安輯,咸遂底寧。近聞乘蕃育之資,稔寇攘之志,敢忘大惠,來撓邊疆。豈朕信之未孚,而吏撫之不至?並蠲釁咎,特示威懷。今後或更剽剠,吏即捕治,置之於法,不須以聞。」是年,又寇八狼砦,巡檢劉崇讓擊敗之,梟其帥王泥豬首以徇。三月,小
 遇族寇慶州,知州慕容德豐擊走之。八年,諸種以馬來獻,太宗召其酋長對於崇政殿,厚加慰撫,賜以束帛,因謂宰相曰:「吐蕃言語不通,衣服異制,朕常以禽獸畜之。自唐室以來,頗為邊患。以國家兵力雄盛,聊舉偏師,便可驅逐數千里外。但念其種類蕃息,安土重遷,倘因攘除,必致殺戮,所以置於度外,存而勿論也。」九年秋,秦州言蕃部以羊馬來獻,各已宴犒,欲用茶絹答其直。詔從之。



 淳化元年,秦州大、小馬家族獻地內附。二年,權知西
 涼州、左廂押蕃落副使折逋阿喻丹來貢。先是,殿直丁惟清往涼州市馬,惟清至而境大豐稔,因為其所留。靈州命蕃落軍使崔仁遇往迎惟清。又吐蕃賣馬還過靈州,為党項所略,表訴其事,因請留惟清至來年同入朝。詔答之。四年,阿喻丹死,以其弟喻龍波為保順郎將代其任。五年,折平族大首領、護遠州軍鑄督延巴率六穀諸族馬千餘匹來貢。既辭,復撾登聞鼓,言儀州八族首領逋波鵄等侵奪地土。上降敕書告諭之。知秦州溫仲
 舒上言,每歲伐木,多為蕃族攘奪,今已驅其部落於渭北。太宗慮生邊患,乃以知鳳翔薛惟吉對易其任,語見《惟吉傳》。是年春,知西涼府左廂押蕃落副使折逋喻龍波、振武軍都羅族大首領並來貢馬。



 至道元年,涼州蕃部當尊以良馬來貢,引對慰撫,加賜當尊虎皮一,歡呼致謝。二年四月,折平族首領握散上言,部落為李繼遷所侵,願會兵靈州以備討擊,賜幣以答之。七月,西涼府押蕃落副使折逋喻龍波上言,蕃部頻為繼遷侵略,
 乃與吐蕃都部署沒暇拽於會六穀蕃眾來朝,且獻名馬。上厚賜之。是歲,涼州復來請帥,詔以丁惟清知州事,賜以牌印。



 咸平元年十一月,河西軍左廂副使、歸德將軍折逋遊龍缽來朝。遊龍缽四世受朝命為酋,雖貢方物,未嘗自行,今始至,獻馬二千餘匹。河西軍即古涼州,東至故原州千五百里,南至雪山、吐谷渾、蘭州界三百五十里,西至甘州同城界六百里,北至部落三百里。周回平川二千里。舊領姑臧、神烏、蕃禾、昌鬆、嘉麟五縣,戶二
 萬五千六百九十三,口十二萬八千一百九十三。今有漢民三百戶。城周回十五里,如鳳形,相傳李軌舊治也。皆龍缽自述雲。詔以龍缽為安遠大將軍。



 二年,以儀州延蒙八部都首領渴哥領化州刺史,首領透逋等為懷化郎將。四年,知鎮戎軍李繼和言,西涼府六穀都首領潘羅支願戮力討繼遷,請授以刺史,仍給廩祿。經略使張齊賢又請封六穀王兼招討使。上以問宰相,皆曰:「羅支已為酋帥,授刺史太輕;未領節制,加王爵非順;招討
 使號不可假外夷。」乃以為鹽州防禦使兼靈州西面都巡檢使。時西涼使來,且言六穀分左右廂,左廂副使折逋遊龍缽實參羅支戎事。朝廷方務綏懷,又以龍缽領宥州刺史,六族首領褚下箕等三人為懷化將軍。其年,潘羅支遣部下李萬山率兵討賊,貽書繼和請師期。先是,遣宋沆、梅詢等為安撫使、副,未行,上謂宰相曰:「 朕看《盟會圖》,頗記吐蕃反覆狼子野心之事。今已議王超等領甲馬援靈州,若難為追襲,即靈州便可製置,沆等不
 須遣,止走一使以會兵告之。」



 五年十月,羅支又言賊遷送鐵箭誘臣部族,已戮一人、縶一人,聽朝旨。詔褒諭之,聽自處置。十一月,使來,貢馬五千匹。詔厚給馬價,別賜彩百段、茶百斤。六年,又遣咩逋族蕃官成逋馳騎至鎮戎軍,請會兵討賊。邊臣疑成逋詐,護送部署司,成逋懼,逸馬墜崖死。上聞,甚歎息之,曰:「此泥埋之子,族人畏其勇,父子皆有戰功,凡再詣闕,朕皆召見,獎其向化。」詔劾鎮戎官吏,仍令渭州以禮葬之。其年,原、渭蕃部三十二
 族納質來歸。羅支又遣蕃官吳福聖臘來貢,表言感朝廷恩信,憤繼遷倔強,已集騎兵六萬,乞會王師收復靈州。乃以羅支為朔方軍節度、靈州西面都巡檢使,賜以鎧甲器幣。又以吳福聖臘為安遠將軍,次首領兀佐等七人為懷化將軍。羅支屢請王師助擊賊,議者以西涼去渭州限河路遠,不可預約師期。上曰:「繼遷常在地斤三山之東,每來寇邊,及官軍出,則已遁去。使六穀部族近塞捍禦,與官軍合勢,亦國家之利。」降詔許之。六月,知
 渭州曹瑋言隴山西延家族首領禿逋等納馬立誓,乞隨王師討賊,以漢法治蕃部,且稱其忠。詔授本族軍主。八月,者龍族首領來貢名馬,上嘉其嘗與潘羅支協力抗賊,令復優待之。其年十一月,繼遷攻西蕃,遂入西涼府,知州丁惟清陷沒。羅支偽降,未幾,集六穀諸豪及者龍族合擊繼遷。繼遷大敗,中流矢遁死。



 景德元年二月,遣其甥廝陀完來獻捷。六月,又遣其兄邦逋支入奏,且欲更率部族及回鶻精兵直抵賀蘭山討除殘孽,願發
 大軍援助。詔涇原部署陳興等候羅支已發,即率眾鼓行赴石門策應。邦逋支又言前賜羅支牌印、官告、衣服、器械為賊劫掠,有詔別給羅支;又言修洪元大雲寺,詔賜金箔物彩。先是,繼遷種落迷般囑及日逋吉羅丹二族亡歸者龍族,而欲陰圖羅支。是月,會遷黨攻者龍,羅支率百餘騎急赴,將議合擊,遂為二族戕於帳。詔贈羅支武威郡王,遣使贈恤其家。



 者龍凡十三族,而六族附迷般囑及日逋吉羅丹。西涼府既聞羅支遇害,乃率龕穀、
 蘭州、宗哥、覓諾諸族攻者龍六族,六族悉竄山谷中,詔使者安集之。六穀諸豪乃議立羅支弟廝鐸督為首領。且言鐸督剛決平恕,每會戎首,設觴豆飲食必先卑者,犯令雖至親不貸,數更戰討,威名甚著。詔授鐸督鹽州防禦使、靈州西面沿邊都大巡檢使。上以遷黨未平,藉其腹背攻製,遂加鐸督朔方軍節度、押蕃落等使、西涼府六穀大首領。



 涇原路言隴山縣王、狸、延三族歸順。又渭州言龕穀、懶家族首領尊氈磨壁餘龍及便囑等獻
 名馬,願率所部助討不附者;又言西涼市馬道出本族,自今保無他虞。詔賜馬直,以便囑等為郎將。石、隰州又言河西諸蕃四十五族內附。其年,遷黨寇永寧,為藥令族合蘇擊敗之,斬首百餘級。鎮戎軍上言,先叛去蕃官茄羅、兀贓、成王等三族及移軍主率屬歸順,請獻馬贖罪,特詔宥之。



 二年,廝鐸督遣其甥嗬昔來貢,仍上與趙德明戰鬥功狀;又言蕃帳周斯那支有智勇,久參謀議,請授以六穀都巡檢使。上嘉獎,從其請,仍賜茶彩。又
 追錄潘羅支子失吉為歸德將軍,厚賜器幣;者龍七族首領有捍寇之勞,並月給千錢。舊制,弓矢兵器不入外夷。時西涼樣丹族上表求市弓矢,上以樣丹宣力西陲,委以捍蔽,特令渭州給賜。因別賜廝鐸督,以重恩意。



 三年,又以者龍族合窮波、黨宗族業羅等為本族首領、檢校太子賓客,皆鐸督外姻也。鐸督遣安化郎將路黎奴來貢。黎奴病於館,特遣尚醫視療。及卒,上憐之,厚加賵給。五月,鐸督又言部落疾疫。詔賜白龍腦、犀角、硫黃、安
 息香、白紫石英等藥,凡七十六種。使者感悅而去。又製加鐸督檢校太傅,其族帳李波逋等四十九人為檢校太子賓客,充本族首領。鐸督遣所部波機進賣馬,因言積官奉半歲,乞就京給賜市所須物,從之。渭州言妙娥、延家、熟嵬等族率三千餘帳、萬七千餘口及羊馬數萬款塞內附。詔遣使撫勞之,賜以袍帶茶彩,仍以折平族首領撒逋渴為順州刺史,充本族都軍主。是年,宗家、當宗、章迷族來貢,移逋、攃父族歸附。九月,詔釋西面納質
 戎人。先是,諸蕃有鈔劫為惡嘗經和斷者,恐異時復叛,故收其子弟為質,乃有禁錮終身者。上憫而縱之,族帳感恩,皆稽顙自誓不為邊患。四年,邊臣言趙德明謀劫西涼,襲回鶻。上以六穀、甘州久推忠順,思撫寧之,乃遣使諭廝鐸督令援結回鶻為備,並賜鐸督茶藥、襲衣、金帶及部落物有差。鐸督奉表謝。



 大中祥符元年十一月,宗哥族大首領溫逋等來貢。三年,西涼府覓諾族瘴疫,賜首領溫逋等藥。四年,廝鐸督遣增藺氈單來貢,賜紫
 方袍。五年,又遺其子來貢。其年,者龍族都首領舍欽波遣使詣闕獻馬,求賜印。詔從其請,仍優齎之。七年,知秦州張佶置大落門新砦。先是,佶欲近渭置采木場,蕃族聞之,即徙帳去。佶不能遂撫之,戎人輒悔,因鄉導鈔劫,佶深入掩擊,悉敗走。至是求和,佶不許。



 三月,秦州曹瑋言熟戶郭廝敦、賞樣丹皆大族,樣丹輒作文法謀叛,廝敦密以告,約半月殺之,至是,果攜樣丹首來。上以廝敦陰害樣丹,不欲明加恩獎,以疑懼諸族。時方議築南使
 城,遂以廝敦獻地為名,詔授順州刺史。先是,張佶深入蕃境,邊事數擾。及瑋破魚角蟬,戮賞樣丹二酋,由是前拒王師者伏匿避罪,瑋誘召之,許納罰首過。既而至者數千人,凡納馬六十匹,給以匹彩。或以少為訴者,瑋叱之曰:「是贖罪物,汝輩敢希利耶!」戎族聞之,皆畏服。八月,曹瑋言伏羌砦廝雞波與宗哥族李磨論聚為文法,領兵趣之,悉潰散,夷其城帳。九月,瑋又言宗哥唃廝囉、羌族馬波叱臘魚角蟬等率馬銜山、蘭州、龕谷、氈
 毛山、洮河、河州羌兵至伏羌砦三都谷,即率兵擊敗之,逐北二十里,斬馘千餘級,擒七人,獲馬牛、雜畜、衣服、器仗三萬三千計。吹麻城張族都首領張小哥以功授順州刺史。瑋又言永寧砦隴波、他廝麻二族召納質不從命,率兵擊之,斬首二百級。十一月,詔給秦州七砦熟戶首領、都軍主以下百四十六人告身。



 天禧元年,詔以冶坊砦都首領郭廝敦為本族巡檢,賦以奉祿。又補大馬家族阿廝鐸為本族軍主。十月,秦州部署言鬼留家族累歲違
 命,討平之。二年,又言吹麻城及河州諸族皆破宗哥文法來附。唃廝囉少衰,數為囉瞎力骨所困,今還舊地。諸砦羌族及空俞、廝雞波等納質者凡七百五十六帳。



 唃廝囉者,緒出讚普之後,本名欺南陵溫逋。篯逋猶讚普也,羌語訛為篯逋。生高昌磨榆國,既十二歲,河州羌何郎業賢客高昌,見廝囉貌奇偉,挈以歸,置貢刂心城,而大姓聳昌廝均又以廝囉居移公城,欲於河州立文法。河州人謂佛「唃」,謂兒子「廝囉」,自此名唃廝囉。於是宗
 哥僧李立遵、邈川大酋溫逋奇略取廝囉如廓州,尊立之。部族浸強,乃徙居宗哥城,立遵為論逋佐之。



 立遵或曰李遵,或曰李立遵,又曰郢成藺逋叱。論逋者,相也。立遵念貪,且喜殺戮,國人不附,既與曹瑋戰三都穀不勝,又襲西涼為所敗。廝囉遂與立遵不協,更徙邈川,以溫逋奇為論逋,有勝兵六七萬,與趙德明抗,希望朝廷恩命。知秦州張佶奏請拒絕。涇原鈐轄曹瑋上言,宜厚唃廝囉以扼德明。而立遵屢表求讚普號,朝議以讚普戎王
 也,立遵居廝囉下,不應妄予,乃用廝鐸督恩例,授立遵保順軍節度使,賜襲衣、金帶、器幣、鞍馬、鎧甲等。



 大中祥符八年,廝囉遺使來貢。詔賜錦袍、金帶、器幣、供帳什物、茶藥有差,凡中金七千兩,他物稱是。其年,廝囉立文法,聚眾數十萬,請討平夏以自效。上以戎人多詐,或生他變,命周文質監涇原軍,曹瑋知秦州兼兩路沿邊安撫使以備之。宗哥城東南至永寧九百一十五里,東北至西涼府五百里,西北至甘州五百里,東至蘭州三百里,
 南至河州四百一十五里。又東至龕穀五百五十里,又西南至青海四百里,又東至新渭州千八百九十里。九年,廝囉、立遵等獻馬五百八十二匹。詔賜器幣總萬二千計以答之。數使人至秦州求內屬。



 明道初,即授廝囉寧遠大將軍、愛州團練使,授逋奇歸化將軍。已而逋奇為亂,囚廝囉置阱中,出收不附己者,守阱人間出之。廝囉集兵殺逋奇,徙居青唐。



 景祐中,以廝囉為保順軍節度觀察留後,歲以奉錢令秦州就賜。元昊侵略其界,兵
 臨河湟,廝囉知眾寡不敵,壁鄯州不出,陰間元昊,頗得其虛實。元昊已渡河,插幟誌其淺,廝囉潛使人移植深處以誤元昊。及大戰,元昊潰而歸,士視幟渡,溺死十八九,所鹵獲甚眾。自是,數以奇計破元昊,元昊遂不敢窺其境。及元昊取西涼府,潘羅支舊部往往歸廝囉,又得回紇種人數萬。廝囉居鄯州,西有臨穀城通青海,高昌諸國商人皆趨鄯州貿賣,以故富強。



 寶元元年,加保順軍節度使,仍兼邈川大首領。時以元昊反,遣左侍禁魯經持
 詔諭廝囉,使背擊元昊以披其勢,賜帛二萬匹。經還,以勞擢閣門祗候。廝囉奉詔出兵向西涼,西涼有備,廝囉知不可攻,捕殺遊邏數十人亟還,聲言圖再舉。元昊既屢寇邊,仁宗召對魯經,欲再遣,經固辭,貶經為左班殿直。募敢使者,屯田員外郎劉渙應詔。渙至,斯囉迎導供帳甚厚,介騎士為先驅,引渙至庭。廝囉冠紫羅氈冠,服金線花袍、黃金帶、絲履,平揖不拜,延坐勞問,稱「阿舅天子安否」。道舊事則數十二辰屬,曰兔年如此,馬年如此。
 渙傳詔,已而廝囉召酋豪大犒,約盡力無負,然終不能有大功。後累加恩兼保順河西節度使、洮涼兩州刺史,又加階勳檢校官、功臣、食邑,賜器幣鞍勒馬。



 嘉祐三年,摖羅部阿作等叛廝囉歸諒詐,諒祚乘此引兵攻掠境上,廝囉與戰敗之,獲酋豪六人,收橐駝戰馬頗眾,因降隴逋、公立、馬頗三大族。會契丹遣使送女妻其少子董氈,乃罷兵歸。



 治平二年夏,羌邈奔及阿叔溪心以隴、珠、阿諾三城叛諒祚歸廝囉,廝囉不禮,乃復歸諒祚,請兵
 還取所獻地,諒祚不之罪,為出萬餘騎隨邈奔、溪心往取,不能克,但取邈川歸丁家五百餘帳而還。廝囉其年冬死,年六十九,第三子董氈嗣。



 董氈母曰喬氏,廝囉三妻。喬氏有色,居曆精城,所部可六七萬人,號令明,人憚服之。方董氈少時,擇酋長子年與董氈相若者與之遊,衣服飲食如一,以此能附其眾。董氈自九歲廝囉為請於朝,命為會州刺史,而喬氏封太原郡君。其二妻皆李立遵女也,生瞎氈及磨氈角。立
 遵死,李氏寵衰,斥為尼,置廓州,錮其子瞎氈。磨氈角結母黨李巴全竊載其母奔宗哥,廝囉不能制,磨氈角因撫有其眾。李氏以寶元二年恩賜紫衣。磨氈角亦累奉貢,初補嚴州團練使,後以思州團練使卒。所部立其子瞎撒欺丁,李氏懼孤弱不能守,乃獻皮帛、入庫廩文籍於廝囉,廝囉因受之。嘉祐三年,命欺丁為順州刺史。瞎氈居龕穀,屢通貢,授澄州團練使,先卒。子木征居河州,母弟瞎吳叱居銀川。



 廝囉地既分,董氈最強,獨有河北
 之地,其國大抵吐蕃遺俗也。懷恩惠,重財貨,無正朔。市易用五穀、乳香、硇砂、罽毯、馬牛以代錢帛。貴虎豹皮,用緣飾衣裘。婦人衣錦,服緋紫青綠。尊釋氏。不知醫藥,疾病召巫覡視之,焚柴聲鼓,謂之「逐鬼」。信咒詛,或以決事,訟有疑,使詛之。訟者上辭牘,藉之以帛,事重則以錦。亦有鞭笞杻械諸獄具。人喜啖生物,無蔬茹醯醬,獨知用鹽為滋味,而嗜酒及茶。居板屋,富姓以氈為幕,多並水為秋遷戲。貢獻謂之「般次」,自言不敢有貳則曰「心白向
 漢」云。其後,河州、武勝軍諸族浸驕,閉于闐諸國朝貢道,擊奪般次。詔邊將問罪。已而董氈遣使奉貢入謝,上慰納焉。



 初,廝囉死,董氈嗣為保順軍節度使、檢校司空。神宗即位,加太保,進太傅。熙寧元年,封其母安康郡太君,以其子藺逋比為錦州刺史。三年,夏人寇環慶,董氈乘虛入其境,大克獲。賜璽書袍帶獎激之。王韶既定熙河,其首領青宜結鬼章寇河州踏白城,景思立死焉。帝命邊臣招來之。十年,以鬼章及阿裏骨皆為刺史。董氈貢
 真珠、乳香、象牙、玉石、馬,賜以銀、彩、茶、服、緡錢,改西平節度使,遣供奉官郭英齎詔書、器幣至其國。



 方鬼章犯境時,列帳訥兒溫及祿尊率部族叛附之。既來降,又陰與董氈通。元豐初,詔知岷州種諤集酋長斬之,以妻女田產賜降將俞龍珂。二年,遣景青宜黨令支貢方物,以令支為珍州刺史,賜董氈錢萬緡,銀彩千計。三年,邈川城主溫訥支郢成及叔溪心、弟阿令京等款塞,以郢成為會州團練使,溪心內殿崇班,令京西頭供奉官,餘族人
 皆殿直奉職。



 四年,王師討夏,會其兵。董氈遣酋長抹征等率三萬人赴黨龍耳江及隴、朱、珂諾,又集六部兵十二萬,約以八月分三路與官軍會。帝以其協濟軍威,事功可紀,由常樂郡公進封武威郡王,鬼章、阿裏骨、黨令支皆團練使,心牟欽氈、阿星、李叱臘欽為刺史。



 夏人欲與之通好,許割賂斫龍以西地,雲如歸我,即官爵恩好一如所欲。董氈拒絕之,訓整兵甲,以俟入討,且遣使來告。帝召見其使,使歸語董氈盡心守圉;每稱其上書情
 辭忠智,雖中國士大夫存心公家者不過如此。知邈川事力固不足與夏人抗,但欲解散其謀,使不與結和而已,故終不能大有功。



 哲宗立,加檢校太尉。元祐元年,卒。藺逋叱已死,養子阿裏骨嗣。



 阿裏骨,本于闐人。少從其母給事董氈,故養為子。元豐蘭州之戰最有功,自肅州團練使進防禦使。董氈病革,召諸酋領至青唐,謂曰:「吾一子已死,惟阿裏骨母嘗事我,我視之如子。今將以種落付之,何如?」諸酋聽命。既嗣
 事,遣使修貢。



 元祐元年,以起復冠軍大將軍、檢校司空為河西軍節度使,封寧塞郡公。裏骨頗峻刑殺,其下不遑寧。詔飭以推廣恩信,副朝廷所以封立、前人所以付與之意。二年,遂逼鬼章使率眾據洮州。羌結藥密者使所部怯陵來告,裏骨執怯陵,結藥密懼,攜妻子南歸。鬼章又使其子結篯咓齪入寇,心牟欽氈、溫溪心不肯從,詔以二人為團練使。八月,鬼章就擒,檻送京師;尋赦之,授陪戎校尉,遣居秦州,聽招其子以自贖。



 明年,裏骨奉表
 謝罪。詔熙河無復出兵,許貢奉如故,加金紫光祿大夫、檢校太保。其廓州主魯尊欲焚拆河橋歸漢,熙州以聞。哲宗以裏骨既通貢,不可有納叛之名,欲弗納,又封其妻溪尊勇丹為安化郡君,子邦彪篯為鄯州防禦使,弟南納支為西州刺史。鬼章死,詔焚付其骨。



 紹聖元年,以師子來獻。帝慮非其土性,厚賜而還之。三年,卒,年五十七。瞎征嗣。



 瞎征,即邦彪篯也。以紹聖四年正月為河西軍節度使、
 檢校司空、寧塞郡公。性嗜殺,部曲睽貳。大酋心牟欽氈之屬有異志,忌瞎征季父蘇南黨征雄勇多智,共誣其謀逆,瞎征不能察而殺之,盡誅其黨,獨篯羅結逃奔溪巴溫。



 溪巴溫者,董氈疏族也,自阿裏骨之立,去依隴逋部,河南諸羌多歸之。篯羅結奉溪巴溫長子杓拶據溪哥城。瞎征討殺杓拶,篯羅結奔河州,說王贍以取青唐之策。已而溫入溪哥城,自稱王子。



 元符二年七月,贍取邈川。八月,瞎征自青唐脫身來降。欽氈迎溪巴溫入青
 唐,立木征之子隴拶為主。九月,贍軍至青唐,隴拶出降。以邈川為湟州,青唐為鄯州。二酋雖降,然其種人本無歸漢意。議者謂:「今不先修邈川以東城障而遽取青唐,非計也。以今日觀之,有不可守者四:自炳靈寺渡河至青唐四百里,道險地遠,緩急聲援不相及,一也;羌若斷橋塞隘,我雖有百萬之師,倉卒不能進,二也;王贍提孤軍以入,四無援兵,必生他變,三也;設遣大軍而青唐、宗哥、邈川食皆止支一月,內地無糧可運,難以久處,四也。
 官軍自會州還者皆憔悴,衣屨穿決,器仗不全,羌視之有輕漢心,旦夕必叛。」



 閏九月,欽氈等果與青唐城中人相結,謀復奪城。山南諸羌亦叛。贍遣將破之,戮結咓齪及欽氈等九人。青唐圍解而邈川益急,夏人十萬助之。總管王湣以死戰固守,乃得免。贍棄青唐歸,巴溫與其子溪賒羅撒據之。朝論請並棄邈川,且謂董氈無後,隴拶乃木征之子、唃廝囉嫡曾孫,最為親的。於是以隴拶為河西軍節度使、知鄯州,封武威郡公,充西蕃都護,依
 府州折氏世世承襲。尋賜姓名曰趙懷德;其弟邦辟勿丁咓曰懷義,為廓州團練使、同知湟州;加瞎征檢校太傅、懷遠軍節度使。



 三年三月,懷德及所降契丹、夏國、回鶻公主入見,各賜冠服,退易之,於邇英閣前後立班謝,賜食於橫門。徽宗命輔臣呼與語,問何以招致溪巴溫,對曰:「譬如乳牛,係其子即母須來,係其母即子須來。俟至岷州,當遣人往諭,使之歸漢。」遂與瞎征俱還湟州。溪賒羅撒謀襲殺懷德,懷德奔河南。瞎征不自安,求內徙,
 詔居鄧州。崇寧元年,卒。三年,王厚復湟、鄯。懷德至京師,拜感德軍節度使,封安化郡王。



 趙思忠即瞎氈之子木征也。瞎氈死,木征不能自立,青唐族酋瞎藥雞囉及僧鹿遵迎之居洮州,欲立以服洮岷疊宕、武勝軍諸羌。秦州以其近邊,逐之,乃還河州,後徙安江城,董氈欲羈屬之,不能有也。母弟瞎吳叱,別居銀川聶家山,至和初,補本族副軍主。嘉祐中,為河州刺史。王韶經略熙河,遣僧智緣往說之,啖以厚利,因隨以
 兵;前後殺其老弱數千,焚族帳萬數,得腹心酋領十餘人,又禽其妻子,皆不殺。遂以熙寧七年四月舉洮、河二州來降,賜以姓名,拜榮州團練使。封其母郢成結遂寧郡太夫人,妻包氏咸寧郡君。弟董穀賜名繼忠,補六宅副使。結吳延征賜名濟忠,瞎吳叱曰紹忠,巴氈角曰醇忠,巴氈抹曰存忠;長子邦辟勿丁咓曰懷義,次蓋咓日秉義,皆超拜官。以思忠為秦州鈐轄,不涖事,而乞主熙河羌部,經略司以為不可。詔以二州給地五十頃。後遷
 合州防禦使,卒,贈鎮洮軍節度觀察留後。



\end{pinyinscope}