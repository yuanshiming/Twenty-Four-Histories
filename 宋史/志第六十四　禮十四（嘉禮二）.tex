\article{志第六十四 禮十四(嘉禮二)}

\begin{pinyinscope}

 冊
 立皇后儀冊命皇太子儀冊皇太子妃儀公主受封儀冊命親王大臣儀



 冊立皇后。建隆元年,立瑯邪郡夫人王氏為皇后,命所
 司擇日備禮冊命。自後,凡制書云冊命者,多不行冊禮。後妃皆寫冊命告身,以金花龍鳳羅紙、金塗褾袋,有司進入,學士院草制,宣於正殿。近臣、牧守、宗室皆修貢禮,群臣拜表稱賀,又詣內東門奉箋賀皇后。



 真宗冊德妃劉氏為皇后,不欲令藩臣貢賀,不降制於外廷,止命學士草詞付中書。



 仁宗冊皇後曹氏,其冊制如皇太子,玉用鈱玉五十簡,匣依冊之長短;寶用金,方一寸五分,高一寸,其文曰「皇后之寶」,盤螭紐,綬並緣冊寶法物約舊
 制為之,匣、盝並朱漆金塗銀裝。其禮與《通禮》異,不立仗,不設縣。



 前一日,守宮設次於朝堂,設冊寶使、副次於東門外,命婦次於受冊寶殿門外,設皇后受冊寶位於殿庭階下北向。奉禮設冊寶使位於內東門外,副使、內侍位於其南,差退,東向北上,冊寶案位於使前南向,又設內給事位於北廂南向。



 其日,百官常服早入次,禮直官、通事舍人先引中書令、侍中、門下侍郎、中書侍郎及奉冊寶官,執事人絳衣介幘,詣垂拱殿門就次,以俟冊降。
 禮直官、通事舍人分引宰臣、樞密、冊寶使副、百官詣文德殿立班,東西相向。內侍二員自內承旨降皇后冊寶出垂拱殿,奉冊寶官俱搢笏率執事人,禮直官導中書侍郎押冊,中書令後從,門下侍郎押寶,侍中後從,由東上閣門出,至文德殿庭權置。



 禮直官、通事舍人引使、副就位,次引侍中於使前,西向稱「有制」,典儀曰「再拜」,贊者承傳,使、副、在位官皆再拜,宣曰:「贈尚書令、冀王曹彬孫女冊為皇后,命公等持節展禮。」使、副再拜,侍中還位,門
 下侍郎帥主節者詣使東北,主節以節授門下侍郎,門下侍郎執節授冊使,冊使跪受,興,付主節,幡隨節立於使左。次引中書令、侍中詣冊寶東北,西向立,中書侍郎引冊案立於中書令右,中書令取冊授冊寶使,使跪受,興,置於案,中書令、中書侍郎退復班。門下侍郎引寶案於侍中之右,取寶授冊寶使如上儀,退復位,典儀贊拜訖,禮直官、通事舍人引使、副押冊寶,持節者前導,奉冊寶官奉舁,援衛如式,以次出朝堂門,詣內東門附
 內臣入進。



 內臣引內外命婦入就位,內侍詣閣請皇后服禕衣。冊寶至,使、副俱東向內給事前,北內跪稱:「冊寶使李迪、副使王隨奉制授皇后冊寶。」俯伏,興,退復位。內給事入詣受冊寶殿門皇后前跪奏訖,內侍進詣使前,西面跪受冊寶,以授內謁者監,使退復位。內謁者監、主當內臣持冊寶入內東門,內侍從之,以次入詣殿庭。內侍贊引皇后降立庭中北向位,內侍跪取冊,次內侍跪取寶,興,立皇後右少前,西向,內侍二員進立皇后左少
 前東向,內侍稱「有制」,內侍贊皇后再拜,內侍奉冊進授皇后,皇后受以授內侍,次內侍奉寶亦然。復贊再拜訖,導皇后升坐,內臣引內外命婦稱賀如常儀。禮畢,內侍導皇后降坐還閣,內外命婦班退。皇後易常服,謝皇帝、皇太后,用常禮。百官詣東上閣門表賀。



 元祐五年八月,太皇太后詔:以皇帝納後,令翰林學士、御史中丞、兩省與太常禮官檢詳古今六禮沿革,參考《通禮》典故,具為成式。群臣又議勘昏,御史中丞鄭雍等請不用陰
 陽之說,呂大防亦言不可,太后納之。



 六年八月,三省、樞密院言:「六禮,命使納採、問名、納吉、納成、告期,差執政官攝太尉充使,侍從官或判宗正官攝宗正卿充副使。以舊尚書省權為皇后行第。納採、問名同日,次日納吉、納成、告期,。納成用穀圭為贄,不用雁。『請期』依《開寶禮》改為『告期』,『親迎』為『命使奉迎』。納採前,擇日告天地、宗廟。皇帝臨軒發冊,同日,先遣冊禮使、副,次遣奉迎使,令文武百官詣行第班迎。」又言:「據《開元禮》,納採、問名合用一使,納
 吉、納成各別日遣使。今未委三禮共遣一使,或各遣使。又合依發冊例立仗。」詔:「各遣使,文德殿發制依發冊立仗。」



 七年正月,詔尚書左丞蘇頌撰冊文並書。學士院上六禮辭語,其納採制文略曰:「太皇太后曰:「咨某官封姓名,渾元資始,肇經人倫,爰及夫婦,以奉天地、宗廟、社稷。謀於公卿,咸以為宜。率由舊典,今遣使持節太尉某、宗正卿某以禮納採。」其答文曰:「太皇太后嘉命,訪婚陋族,備數採擇,臣之女未閑教訓,衣履若而人。欽承舊章,肅
 奉典制。某官封糞土臣姓某稽首再拜承制詔。」問名制曰:「兩儀合德,萬物之統,以聽內治,必咨令族。重宣舊典,今遣使持節某官以禮問名。」答曰:「使者重宣中制,問臣名族。臣女,夫婦所生,先臣故某官之遺微孫,先臣故某官之遺曾孫,先臣故某官之遺孫,先臣故某官之外孫女,年若干。欽承舊章,肅奉典制。」納吉制曰:「人謀龜筮,同符元吉,恭順典禮,今使某官以禮納吉。」答曰:「使者重宣中制,臣陋族卑鄙,憂懼不堪。欽承舊章,肅奉典制。」納成
 制曰:「咨某官某之女,孝友恭儉,實維母儀,宜奉宗廟,永承天祚。以黝纁、穀圭、六馬以章典禮,今使某官以禮納成。」答曰:「使者重宣中制,降婚卑陋,崇以上公,寵以豐禮,備物典策。欽承舊章,肅奉典制。」告期制曰:「謀於公卿,大筮元龜,罔有不臧,吉日惟某月、某甲子可迎。率遵典禮,今遣某官以禮告期。」答曰:「使者重宣中制,以某月、某甲子吉日告期。臣欽承舊章,肅奉典制。」奉迎制曰:「禮之大體,欽順重正,其期維吉,典圖是若,今遣某官以禮奉迎。」
 答曰:「使者重宣中制,今日吉辰,備禮以迎。螻蟻之族,猥承大禮,憂懼戰悸。欽率舊章,肅奉典制。」餘如式。



 三月,禮部、太常寺上納后儀注:



 發六禮制書。太皇太后御崇慶殿,內外命婦立班行禮畢,內給事出殿門,置六禮制書案上,出內東門。禮直官、通事舍人引由宣祐門至文德殿後門入,權置案於東上閣門。



 命使納採、問名。文德殿,宰臣、親王、執政官、宗室、百僚、大小使臣易朝服,樂備而不作。班定,內給事奉制書案置橫街北稍東,西向北
 上,禮直官、通事舍人引門下、中書侍郎,次引使、副就橫街南承制位,北向東上,內給事詣使者東,北面稱「太皇太后有制」,典儀曰「再拜」,在位官皆再拜。宣制曰:「皇帝納後,命公等持節行禮。」典儀曰「再拜」,使、副皆再拜。授制書訖,典儀曰「再拜」,在位官皆再拜。禮直官、通事舍人、太常博士引使、副從制案出,載於油絡網犢車,出宣德門,鼓吹備而不作。至皇后行第大門外,令史二人對奉制案立,主人立大門內,儐者立主人之左,北面,進受命,出曰:「
 敢請事。」使者曰:「某奉制納採。」儐者入告,主人曰:「臣某之女若而人,既蒙制訪,臣某不敢辭。」儐者出告,入引主人出大門外,再拜。使者先入,使者曰:「太皇太后制。」主人再拜。宣制書畢,主人再拜受訖,主人進表訖,再拜,使者出。問名同上儀。使者曰:「將加卜筮,奉制問名。」主人曰:「臣某之女若而人,既蒙制命,臣某不敢辭。」



 命使納吉、納成、告期並同命使納採、問名儀。納吉,使者曰:「加請卜筮,占曰從制,使某納吉。」主人曰:「臣某之女若而人,龜筮云吉,臣預
 有焉。臣某謹奉典制。」告期,使者曰:「某奉制告期。」主人曰:「臣某謹奉典制。」以上納吉、納成、告期。請見、授制、接表並如納採儀。



 臨軒命使冊后及奉迎於文德殿。百官朝服,皇帝常服乘輦至殿後閣,侍中奏中嚴外辦,乃服通天冠、絳紗袍,乘輦出自西房,降輦即御坐。兩省官及待制、權侍郎、觀察使以上,分東西入殿門,各就位,東西相向立。奉寶置御坐前,奉宣後冊由東上閣門出,至文德殿庭橫行,典儀曰「拜」,在位官皆再拜。使、副受冊,宣制曰:「冊
 某氏為皇后,命公等持節展禮。」典儀曰「拜」,使、副再拜受冊寶訖,典儀贊百官再拜。宣制曰:「太皇太后制:命公等持節奉迎皇后。」典儀贊使、副再拜受節,又贊百官再拜。侍中奏禮畢解嚴,百官再拜出,皇帝常服還內。冊寶至皇后行第,如納採儀。使者曰:「某奉制授皇后備物典冊。」皇后受冊寶,內外命婦序立如儀,主人以書奉使者。



 奉迎。百官常服班宣德門外行第,儐者請,使者曰:「某奉制以禮奉迎。」儐者入告,主人曰:「臣某謹奉典制。」儐者出告,
 入引主人出大門外再拜。使者先入,曰「有制」,主人再拜,使者宣制畢,主人再拜受制,答表又再拜。姆導皇后,尚宅前引,升堂出立房外,典儀贊使、副再拜。使者曰:「今月吉日,某等承制以禮奉迎。」內侍受以入,使、副退,主人以書授使者,奉於司言,受以奏聞。皇后降立堂下再拜訖,升堂,主人升自東階,西向曰:「戒之戒之,夙夜無違命!」主人退,母進西階上東向,施衿、結帨曰:「勉之戒之,夙夜無違命!」皇后升輿至中門,升車出大門,使、副及群臣前引。
 將至宣德門,百官、宗室班迎,再拜訖,分班。皇后入門,鳴鐘鼓,班迎官退,乃降車入,次升輿入端禮門、文德殿、東上閣門,出文德殿後門,入至內東門內降輿,司輿前導,詣福寧殿門大次以俟。晡後,皇后車入宣德門,侍中版奏請中嚴,內侍轉奏,皇帝服通天冠、絳紗袍,御福寧殿,尚宮引皇后出次,詣殿庭之東,西向立。尚儀跪奏外辦,請皇帝降坐禮迎,尚宮前引,詣庭中之西,東面揖皇后以入,導升西階入室,各就榻前立。尚食跪奏具,皇帝揖
 皇后皆坐,尚食進饌,食三飯,尚食進酒,受爵飲,尚食以饌從;再飲如初,三飲用巹如再飲。尚儀跪奏禮畢,俱興,尚宮請皇帝御常服,尚寢請皇后釋禮服入幄。次日,以禮朝見太皇太后、皇太后,參皇太妃,如宮中之儀。



 詔從之。



 四月,太皇太后手書曰:「皇帝年長,中宮未建,歷選諸臣之家,以故侍衛親軍馬軍都虞候、贈太尉孟元孫女為皇后。」制詔:「六禮,尚書左僕射兼門下侍郎呂大防攝太尉,充奉迎使,同知樞密院事韓忠彥攝司徒副之;尚
 書左丞蘇頌攝太尉,充發冊使,簽書樞密院事王巖叟攝司徒副之;尚書左丞蘇轍攝太尉,充告期使,皇叔祖、同知大宗正事宗景攝大宗正卿副之;皇伯祖、判大宗正事、高密郡王宗晟攝太尉,充納成使,翰林學士範百祿攝宗正卿副之;吏部尚書王存攝太尉,充納吉使,權戶部尚書劉奉世攝宗正卿副之;翰林學士梁燾攝太尉,充納採、問名使,御史中丞鄭雍攝宗正卿副之。」



 五月甲午,行納採、問名禮。丁酉,行納吉、納成、告期禮。戊戌,帝
 御文德殿發冊及命使奉迎皇后。己亥,百官表賀於東上閣門,次詣內東門賀太皇太后,又上箋賀皇后,上箋賀皇太妃。皇后擇日詣景靈宮行廟見禮。



 大觀四年,冊貴妃鄭氏為皇后,議禮局復位儀注:臨軒冊使,皇帝御文德殿,服通天冠、絳紗袍,百官朝服,陳黃麾細仗,依古用宮架。冊使出殿門,依近儀不乘輅。權以穆清殿為受冊殿。其日,皇后服禕衣,其奉冊寶授皇后,皆用內侍。受冊訖,皇后上表謝皇帝,內外命婦立班稱賀,群臣入殿
 賀皇帝,於內東門上箋賀皇后。其上禮儀注,乞依進馬條令施行;其會群臣,及皇后會外命婦儀注,並依《開元》、《開寶禮》。受冊之殿陳宮架,用女工,升降行止並以樂節,而別定樂名、樂章。



 皇后上表乞免受冊排黃麾仗及乘重翟車、陳小駕鹵簿等,而於延福宮受冊。其朝謁景靈宮,亦止依近例云。



 紹興十三年閏四月十七日,冊貴妃吳氏為皇后。前期,於文德殿內設東西房、東西閣,凡香案、宮架、冊寶幄次、舉麾位、押案位、權置冊寶褥位、受制
 承制宣制位、奉節位、贊者位、奉冊寶位、舉冊舉寶官位及文武百僚、應行事官、執事官位,皆儀鸞司、太常典儀分設之,以俟臨軒發冊。



 其日質明,皇帝服通天冠、絳紗袍出西閣,協律郎舉麾奏《乾安》之樂,皇帝降輦即御坐,樂止,冊使、副以下應在位官皆再拜。侍中宣制曰:「冊貴妃吳氏為皇后,命公等持節展禮。」冊使、副再拜,參知政事以節授冊使,冊使跪受,以授掌節者。中書令以冊授冊使,侍中以寶授副使,並權置於案,冊使、副以下應在
 位官皆再拜。冊使押冊,副使押寶,持節者前導,《正安》之樂作,出文德殿門,樂止,至穆清殿門外幄次,權置以俟。



 皇后首飾、禕衣出閣,協律郎舉麾,《坤安》之樂作,皇后至殿上中間南向立定,樂止。冊使、副就內給事前東向跪稱:「冊使副姓某奉制授皇后備禮典冊。」內給事入詣皇后前,北向奏訖,冊使舉冊授內侍,內侍轉授內謁者監;副使舉寶授內侍,內侍轉授內謁者監;掌節者以節授掌節內侍,內侍持節前導,冊寶並案進行入詣殿庭。冊
 寶初入門,《宜安》之樂作,至位,樂止。皇后降自東階,至庭中北向位,初行,《承安》之樂作,至位,樂止。皇后再拜,舉冊官搢笏跪舉冊,讀冊官搢笏跪宣冊,內謁者監奉冊進授皇后,皇后受以授司言,又奉寶進授皇后,皇后受以授司寶。司言、司寶置冊寶於案,舉冊寶官並舉案官俱搢笏舉冊寶並案興,詣東階之東,西向位置定。皇后初受冊寶,《成安》之樂作,受訖,樂止。皇后再拜,禮畢。



 冊皇太子。至道元年八月壬辰,詔立皇太子,命有司草
 其冊禮,以翰林學士宋白為冊皇太子禮儀使。有司言:「前代太子無執圭之文,請如王公之制執桓圭,餘如舊制。」



 九月丁卯,太宗御朝元殿,陳列如元會儀,帝袞冕,設黃麾仗及宮縣之樂於庭,百官就位。太子常服乘馬,就朝元門外幄次,易遠游冠、朱明衣,所司贊引三師、三少導從至殿庭位,再拜起居畢,分班立。



 太常博士引攝中書令就西階解劍、履,升殿詣御坐前,俯伏,興,奏宣制,降就劍、履位,由東階至太子位東,南向稱「有制」,太子再拜。
 中書侍郎引冊案就太子東,中書令北面跪讀冊畢,太子再拜受冊,以授右庶子;門下侍郎進寶授中書令,中書令授太子,太子以授左庶子,各置於案。由黃道出,太子隨案南行,樂奏《正安》之曲,至殿門,樂止,太尉升殿稱賀,侍中宣制,答如儀。



 皇太子易服乘馬還宮,百官賜食於朝堂。中書、門下、樞密院、師、保而下詣太子參賀,皆序立於宮門之外。庶子版奏外備,內臣褰簾,太子常服出次坐,中書、門下、文武百官、樞密、師、保、賓客而下再拜,並
 答拜;四品以下官參賀,升坐受之。越三日,具鹵簿,謁太廟,常服乘馬,出東華門升輅,儀仗內行事官乘車者,並服禮衣,餘皆褲褶乘馬導從。



 有司言:「唐禮,宮臣參賀皆舞蹈,開元始罷之。故事,百官及東宮接見只呼皇太子,上箋啟稱皇太子殿下,百官稱名,宮官稱臣;常行用左春坊印,宮中行令。又按唐制,凡東宮處分論事之書,太子並畫令,左、右庶子以下署名姓,宣奉行書按畫日;其與親友、師傅,不用此制。今請如開元之制,宮臣止稱臣,
 不行舞蹈之禮。今皇太子兼判開封府,其所上表狀即署太子之位,其當申中書、樞密院狀,祗判官等署,餘斷案及處分公事並畫諾。」詔惟改『諾』為『準』,餘並從之。其朝皇后儀,止用宮中常禮。時真宗以壽王為皇太子,兼判開封,請見僚屬,稱名而免稱臣。



 神宗未及受冊禮而即位,乃以冊寶送天章閣,遂為故事。



 紹興三十二年五月,詔曰:「朕以不德,躬履艱難三十有六年,憂勞萬機,宵旰靡怠。屬時多故,未能雍容釋負,退養壽康,今邊鄙粗寧,可
 遂如志。皇子毓德允成,神器有托,朕心庶幾可立為皇太子,仍改名,所司擇日備禮冊命。」未及行禮,六月十一日內禪。



 乾道元年八月十日,制立皇子鄧王□為皇太子。十月,詔以知樞密院洪適為禮儀使,撰冊文,簽書樞密院事葉顒書冊,工部侍郎王弗篆寶。



 十六日,皇帝御大慶殿行冊禮,皇太子服遠游冠、朱明衣,執桓圭。前期,習儀禮官及有司並先一日入宿衛,展宮架樂,設太子次、冊寶幄次、百官次,又設皇太子受冊位、典寶褥位,應
 行禮等皆有位,列黃麾半仗於殿門內外。質明,百官就次,皇太子常服詣幕次,符寶郎陳八寶於御位之左右,有司奉冊寶至幄次,百官朝服入班殿庭。



 有司自幄次奉冊寶至褥位,參知政事、中書令導從,退各就位,侍中升殿俟宣制,皇太子易服執圭俟於殿門外。樂正撞黃鐘之鐘,《乾安》之樂作,皇帝即御坐,殿上侍臣起居,樂止。行禮官贊引皇太子入就殿庭,東宮官從,初入殿門,《明安》之樂作,樂止,皇太子起居,次百官起居,各拜舞如儀。



 皇太子詣受冊位,侍中前承旨,降階宣制曰:「冊鄧王□為皇太子。」皇太子拜舞如儀,侍中升殿復位。中書令詣讀冊位,捧冊官奉冊至,中書令跪讀畢,興,皇太子再拜,有司奉冊至皇太子位,中書令跪以冊授皇太子,皇太子跪受,以授右庶子,置於案;次侍中以寶授皇太子,皇太子跪受,以授左庶子,如上儀。皇太子再拜。中書舍人押冊、中允押寶以出,次皇太子出,如來儀。初行樂作,出殿門樂止。次百官稱賀,樂正撞蕤賓之鐘,《乾安》之樂作,
 皇帝降坐,樂止,放仗,在位官再拜以出。



 禮畢,百官易常服,赴內東門司拜箋賀皇后,次赴德壽宮拜表箋賀,諸路監司、守臣等並奉表稱賀。明日,車駕詣德壽宮謝。又明日,上御紫宸殿,引皇太子稱謝,還東宮,百官赴東宮參賀。



 皇太子擇日先朝謁景靈宮,次日朝謁太廟、別廟,又擇日詣德壽宮稱謝。先是,禮官言:「皇太子朝謁景靈宮無所服典故,乞止用常服。次朝謁太廟、別廟,當袞冕,乘金輅,設仗。」從之。皇太子言:「乘輅、設仗,雖有至道、天禧
 故事,非臣子所安。」詔免。



 冊皇太子妃。政和五年三月,詔選皇太子妃。六年六月,詔選少傅、恩平郡王朱伯材女為皇太子妃,令所司備禮冊命。庚辰,帝服通天冠、絳紗袍,御文德殿發冊。先是,議禮局上《五禮新儀》:「皇太子納妃,乘金輅親迎。」皇太子三奏辭乘輅及臨軒冊命,詔免乘輅,而發冊如禮焉。



 公主受封,降制有冊命之文,多不行禮,惟以綸告進內。至嘉祐二年,封福康公主為兗國公主,始備禮冊命。



 前
 一日,百官班文德殿,內降冊印,宣制,冊案、援衛一如冊皇后儀。有司先設冊使等幕次於內東門外,命婦次於公主本位門之外,公主受冊印位於本位庭階下北向,冊使位於內東門、副使及內給事於其南差退並東向,設冊印案位於冊使前南向,內給事位於冊使北南向。



 自文德殿奉冊印將至內東門,內給事詣本位,請公主服首飾、褕翟。冊印至內東門外褥位置訖,內臣引內命婦入就位,禮直官引冊使、副等俱就東向位,內給事就南向
 位。



 通事舍人、博士引冊使就內給事前東向,躬稱「冊使某、副使某奉制授公主冊印」,退復位,內給事入詣所設受冊印位公主前,言訖退。內給事進詣冊使前西向,冊使跪以冊印授內給事,內給事跪授內謁者,內謁者及主當內臣等持入內東門,內給事從入詣本位,贊公主降詣庭中北向立,跪取冊,興,立公主右少前西向。又內給事立公主左少前東向,稱「有制」,贊者曰「拜」,公主再拜,右給事奉冊跪授之,公主受以授左給事,右給事又奉
 印授公主,如上儀。贊者曰「拜」,公主再拜畢,引公主升位。次內臣行內命婦賀畢,遂引公主謝皇帝、皇后,一如內中之儀。群臣進名賀。其冊印如貴妃,有匣,文曰「兗國公主之印」。遂為定制。



 神宗進封邠國大長公主、魯國公主皆請免冊禮,止進告入內云。



 冊命親王大臣之制,具《開寶通禮》,雖制書有備禮冊命之文,多上表辭免,而未嘗行。每命親王、宰臣、使相、樞密使、西京留守、節度使,並翰林草制,夜中進入,翼日自內
 置於箱,黃門二人舁之,立御坐東。內朝退,乃奉箱出殿門外,宣付閣門,降置於案,俟文德殿立班,閣門使引制案置於庭,宣付中書、門下,宰相跪受,復位,以授通事舍人,赴宣制位唱名訖,奉詣宰相,宰相受之,付所司。



 若立后妃,封親王、公主,即先稱有制,百官再拜,宣制訖,復再拜舞蹈稱賀。若宰相加恩制書,即宣付通事舍人,引宰相於宣制石東,北向再拜立,聽訖,拜舞復位。若百官受制,即自班中引出聽麻,文班於宣制石東,武班於西,並
 如宰相儀,聽訖,出赴朝堂。其罷相者,即引出赴朝堂金吾仗舍。



 諸王、宰相朝謝,前一日,內降官告,從內出東上閣門外宣詞以賜,授節者,仍交旌節。授者俯伏,執旌節交於頸上者三。參知政事、宣徽使、樞密使、大兩省、兩制、秘書監、上將軍、觀祭使以上授官告敕牒者,皆拜敕舞蹈,若止授敕或宣頭者止再拜,餘官悉不拜敕、不舞蹈,惟御史大夫、中丞拜授東上閣門使,又引至殿門外中籠門再拜。



 親王、節度、使相官告,並載以彩輿迎歸第。親
 王輿中,設銀師子香合,輦官十二人,並帕頭、緋繡寬衣;旌節各二,馬四,犦槊官十六人,執旌節攏馬對引,由乾元門西偏門出至門外;馬技騎士五十人,槍牌步兵六十人,教坊樂工六十五人,及百戲、蹴鞠、鬥雞、角抵次第迎引,左右軍巡使具軍容前導至本宮。使相輿中用銀香爐,輦官十二人,金鵝帽、錦絡縫紫絁寬衣;旌節各一,馬二,犦槊官八人,馬技騎士二十人,槍牌步兵二十四人,軍巡使不前導,餘如親王制。有故則罷。



 凡諫、舍、刺史
 以上在外任加恩者,悉令其親屬乘傳繼詔,就以告牒賜之。



 政和禮局上冊命親王、大臣儀,迄不果行。



\end{pinyinscope}