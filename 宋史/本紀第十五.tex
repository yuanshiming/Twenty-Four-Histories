\article{本紀第十五}

\begin{pinyinscope}

 神宗二



 三年春正月癸丑,錄唐李氏、周柴氏後。乙卯,詔諸路散青苗錢禁抑配。戊午,判尚書省張方平罷知陳州。



 二月壬申,以翰林學士司馬光為樞密副使,凡九辭,詔收還
 敕誥。甲戌,以河州刺史瞎欺丁木徵為金紫光祿大夫、檢校刑部尚書。乙酉,韓琦罷河北安撫使,為大名府路安撫使。



 三月丙申,孫覺、呂公著、張戩、程顥、李常上疏極言新法,不聽。己亥,始策進士,罷詩、賦、論三題。戊申,李常言青苗斂散不實,有旨具析,翰林學士兼知通進、銀臺司範鎮封還詔書,以為不當,坐罷職,守本官。壬子,賜禮部奏名進士、明經及第八百二十九人。乙卯,詔諸路毋有留獄。丙辰,立試刑法及詳刑官。右正言孫覺以奉詔
 反復,貶知廣德軍。



 夏四月癸亥,幸金明池觀水嬉,宴射瓊林苑。丙寅,遼遣耶律寬來賀同天節。丁卯,給兩浙轉運司度僧牒,募民入粟。戊辰,御史中丞呂公著貶知穎州。己卯,趙抃罷知杭州,以韓絳參知政事。監察御史裏行程顥罷為京西路同提點刑獄。壬午,右正言李常貶通判滑州,監察御史裏行張戩貶知公安縣,王子韶貶知上元縣。癸未,侍御史知雜事陳襄罷為同修起居注,程顥簽書鎮寧軍節度判官公事,前秀州軍事判官李
 定為太子中允、監察御史裏行。



 五月癸巳,詔並邊州郡毋給青苗錢。太白晝見。壬寅,詔令司馬光詳定轉對封事。甲辰,詔罷制置三司條例歸中書。辛亥,賜進士蘇丕號安退處士。壬子,罷入閣儀。丁巳,詔以審官院為東院,別置西院。



 六月癸酉,日有五色雲。丁丑,封宗室秦、魯、蔡、魏、燕、陳、越七王後為公。戊寅,詔修武成王廟。丙戌,知諫院胡宗愈貶通判真州。



 秋七月辛卯,歐陽修徙知蔡州。壬辰,呂公弼罷樞密使,以知太原府馮京為樞密副使。
 罷潞州交子務。戊戌,雨雹。癸丑,詳定宗室襲封制度。甲寅,置三班院主簿。



 八月戊午,罷看詳銀臺司文字所。丙寅,以旱慮囚,死罪以下遞減一等,杖、笞者釋之。以衛州旱,令轉運司振恤,仍蠲租賦。戊寅,詔川陜、福建、廣南七路官令轉運司立格就注,具為令。遣張景憲等賀遼主生辰、正旦。己卯,夏人犯大順城,知慶州李復圭以方略授環慶路鈐轄李信、慶州東路都巡檢劉甫、監押種詠出戰,兵少取敗。復圭誣信等違其節制,斬信及劉甫,種
 詠死於獄。是月,慶州巡檢姚兕敗夏人於荔原堡。鈐轄郭慶、都監高敏死之。九月戊子朔,中書置檢正官。乙未,韓絳罷為陜西宣撫使。己亥,始試法官。庚子,曾公亮罷為司空兼侍中、河陽三城節度使。辛丑,以馮京參知政事,翰林學士吳充為樞密副使。乙巳,親策賢良方正及武舉。壬子,太白晝見。癸丑,作東、西府以居執政。司馬光罷知永興軍。詔環慶陣亡義勇餘丁當刺者,悉免之。



 冬十月辛酉,詔延州毋納夏使。甲子,雨木冰。壬申,朝謁神
 御殿。丙子,知慶州李復圭擅興兵敗績,誣裨將李信、劉甫、種詠以死,御史劾之,貶保靜軍節度副使。戊寅,陳升之以母憂去位。乙酉,詔罷諸場務內侍監當。



 十一月戊子,振河北饑民徙京西者。己丑,官節行之士二十一人。壬辰,蠲陜西蕃部貸糧。癸卯,授布衣王存下班殿侍、三班差使、宣撫司指揮使。甲辰,夏人寇大順城,都監燕達等擊走之。庚戌,詔升朝官除南郊赦封贈父母外,不得以加恩轉官。乙卯,以韓絳兼河東宣撫使。梓州路轉運
 使韓璹等以能興利除害,賜帛有差。十二月己未,詔立諸路更戍法,舊以他路兵雜戍者遣還。乙丑,立保甲法。丁卯,以韓絳、王安石並同中書門下平章事,王珪參知政事。賜布衣陳知彥進士出身,知縣王輔同進士出身。庚午,夏人寇鎮戎軍三川砦,巡檢趙普伏兵邀擊,敗之。丁丑,增廣南攝官奉。戊寅,初行免役法。賜西蕃董氈詔並衣帶、鞍馬。庚辰,命王安石提舉編修三司令式。壬午,遼遣蕭遵道等來賀正旦。癸未,命宋敏求詳定命官、使
 臣過犯。是歲,振河北、陜西旱饑,除民租。交址入貢,廣源、下溪州蠻來附。



 四年春正月丁亥朔,不視朝。己丑,種諤襲夏兵於囉兀北,大敗之,遂城囉兀。自是夏人日聚兵為報復計,言者以諤為稔邊患不便。壬辰,王安石請鬻天下廣惠倉田為三路及京東常平倉本,從之。乙未,渝州夷賊李光吉叛,巡檢李宗閔等戰死,命夔州路轉運使孫構討平之。詔詳定大闢覆讞法。丁酉,朝謁太祖、太宗神御殿。庚子,
 幸集禧觀,宴從臣,又幸大相國寺,御宣德門觀燈。韓絳等言種諤領兵入西界,斬獲甚眾,詔遣使撫問。乙巳,停括牧地。丁未,立京東、河北賊盜重法。庚戌,罷永興軍買鹽鈔場。甲寅,定文德殿朔望視朝儀。



 二月丁巳朔,罷詩賦及明經諸科,以經義、論、策試進士。置京東西、陜西、河東、河北路學官,使之教導。辛酉,詔治吏沮青苗法者。戊辰,詔振河北民乏食者。賻恤西界戰死軍人。庚午,於闐國來貢。壬申,進封高密郡王頵為嘉王。癸酉,詔審官院
 所定人赴中書,察堪任者引見。甲戌,賜討渝州夷賊兵特支錢。丁丑,禱雨。詔增漳河等役兵。



 三月丁亥,夏人陷撫寧堡。戊子,慶州廣銳卒叛,尋討平之。庚寅,詔給諸路學田,增教官員。辛卯,遣使察奉行新法不職者。癸卯,減河東、陜西路囚罪一等,徒以下釋之。民緣軍事科役者,蠲其租賦。丙午,種諤坐陷撫寧堡,責授汝州團練副使、潭州安置。丁未,韓絳坐興師敗衄罷,以本官知鄧州。辛亥,錄唐李氏後。



 夏四月丙辰朔,恤刑。辛酉,遼遣蕭廣等來
 賀同天節。壬戌,遣環慶都鈐轄幵贇以兵屯邠、涇、河中,以備西夏。癸亥,罷陜西交子法。癸酉,司馬光權判西京留臺。種諤再貶賀州別駕。甲戌,詔司農寺月進諸路所上雨雪狀。丙子,遣使按視宿、亳等州災傷,仍令修飭武備。壬午,定進士考轉官。



 五月甲午,右諫議大夫呂晦卒。壬寅,詔許富弼養疾西京。丙午,高麗國來貢。辛亥,詔宗室率府副率以上,遭父母喪及嫡孫承重,並解官行服。壬子,詔恩、冀等州災傷,遣使振恤,蠲其稅。



 六月丁巳,河
 北饑民為盜者,減死刺配。庚申,群臣三上尊號曰紹天法古文武仁孝皇帝,不許。甲子,歐陽修以太子少師致仕。丙寅,慮囚。甲戌,富弼坐格青苗法,徙判汝州。



 秋七月戊子,層檀國來貢。甲午,振恤兩浙水災。乙未,錄死事將校崔達子遇為三班奉職。丁酉,監察御史裏行劉摯罷監衡州鹽倉,御史中丞楊繪貶知鄭州。庚子,詔宗室不得祀祖宗神御。丁未,詔唐、鄧給流民田。



 八月癸丑朔,高麗來貢。遣官體量陜西差役新法及民間利害。甲寅,詔
 郡縣保甲與賊鬥死傷者,給錢有差。庚申,復《春秋三傳》明經取士。癸酉,遣楚建中等賀遼主生辰、正旦。置洮河安撫司,命王韶主之。



 九月丙戌,河決鄆州。辛卯,大饗明堂,以英宗配。赦天下,內外官進秩有差。庚子,夏人入貢。癸卯,增選人奉。



 冬十月壬子朔,罷差役法,使民出錢募役。立選人及任子出官試法。丙辰,置樞密院檢詳官。戊辰,立太學生內、外、上舍法。丙子,詔罪人配流,遇冬者至中春乃遣。



 十一月壬午朔,詔凡賞功罰罪,事可懲勸者,
 月頒之天下。甲申,詔蠲逋租。丁亥,作中太一宮。壬寅,開洪澤河達於淮。十二月辛亥朔,詔增賜國子監錢四千緡。戊午,歸夏俘。己未,安定郡王從式薨。甲子,封越國公世清為會稽郡王。丙寅,省諸路廂軍。乙亥,崇義公柴詠致仕,子若訥襲封。丙子,遼遣耶律紀等來賀正旦。



 五年春正月己丑,詔聽降羌歸國。己亥,詔太廟時饗,以宗室使相已上攝事。置京城邏卒,察謗議時政者收罪之。



 二月壬子,龜茲來貢。以兩浙水,賜穀十萬石振之,仍
 募民興水利。壬戌,詔罷陜西遞運銅錫。癸亥,太白晝見。丙寅,以知鄭州呂公弼為宣徽南院使、判秦州,龍圖閣直學士蔡挺為樞密副使。



 三月甲午,李日尊卒,子乾德嗣,遣使吊贈。戊戌,富弼以司空致仕,進封韓國公。立文武換官法。丙午,以內藏庫錢置市易務。



 夏四月庚戌朔,立殿前馬步軍春秋校試殿最法。乙卯,遼遣耶律適等來賀同天節。己未,括閑田。置弓箭手。辛未,塞北京決河。



 五月辛巳,詔以古渭砦為通遠軍,命王韶兼知軍。行教
 閱法。宗室非袒免親者許應舉。庚寅,以青唐大首領俞龍珂為西頭供奉官,賜姓名包順。壬辰,以趙尚寬等前守唐州闢田疏水有功,增秩以勸天下。丙午,太白晝見。行保馬法。



 六月壬子,曾公亮以太傅致仕。癸亥,詔以四場試進士。丙寅,作京城門銅魚符。乙亥,置武學。



 秋七月壬寅,初以文臣兼樞密都承旨。



 閏月庚戌,遣中書檢正官章惇察訪荊湖北路。詔入內供奉官以下,已有養子,更養次子為內侍者斬。



 八月甲申,太子少師致仕歐陽
 修薨。秦鳳路沿邊安撫王韶復武勝軍。丁亥,詔求歐陽修所撰《五代史》。壬辰,以武勝軍為鎮洮軍。癸巳,遣崔臺符等賀遼主生辰、正旦。乙未,詔侍從及諸路監司各舉有才行者一人。甲辰,王韶破木徵於鞏令城。頒方田均稅法。



 九月癸丑,許宗室試換文資。癸亥,始御便殿,旬校諸軍武技。丙寅,少華山崩,詔壓死者賜錢,貧者官為葬祭。淮南分東、西路。



 冬十月戊戌,升鎮洮軍為熙州、鎮洮軍節度,置熙河路。減秦鳳囚罪一等。



 十一月癸丑,河州
 首領瞎藥等來降,以為內殿崇班,賜姓名包約。丁卯,貶權監察御史裏行張商英監荊南稅。壬申,分陜西為永興、秦鳳路,仍置六路經略司。章惇開梅山,置安化縣。十二月丙子,赦亡命荊湖溪洞者。丁丑,詔太原置弓箭手。戊寅,詔寺觀奉聖祖及祖宗陵寢神御者免役錢。改溫成廟為祠。壬午,陳升之為樞密使。癸未,雨土。乙未,築熙州南、北關及諸堡砦。己亥,遼遣蕭瑜等來賀正旦。



 六年春正月辛亥,復僖祖為太廟始祖,以配感生帝。祧
 順祖於夾室。



 二月辛卯,夏人寇秦州,都巡檢使劉惟吉敗之。丙申,永昌陵上宮東門火。王韶復河州,獲木徵妻子。壬寅,以韓絳知大名府。



 三月己酉,詔贈熙河死事將田瓊禮賓使,錄其子三人、孫一人。庚戌,親策進士。置經局,命王安石提舉。辛亥,試明經諸科。丙辰,以四月朔日當食,自丁巳避殿、減膳,降天下囚罪一等,流以下釋之。己未,置諸路學官。壬戌,賜奏名進士、諸科及第出身五百九十六人。甲子,交州來貢。丁卯,宰相上表請復膳,不
 許。詔進士、諸科並試明法注官。戊辰,置刑獄檢法官。庚午,封李日尊子乾德為交址郡王。



 夏四月甲戌朔,日食,不見。乙亥,御殿復膳。西南龍蕃諸夷來貢。置律學。丁丑,遼遣耶律寧等來賀同天節。甲午,定齊、徐等州保甲。戊戌,裁定在京吏祿。



 五月癸卯朔,播州楊貴遷遣子光震來貢,以光震為三班奉職。戊申,禱雨。乙丑,詔京東路察士人有行義者以聞。遣中書檢正官熊本措置瀘夷。西京左藏庫副使景思忠等攻燒遂州夷囤戰歿,錄其子
 昌符等七人,軍士死者,賜其家錢帛有差。辛未,西南龍蕃來貢。



 六月己亥,置軍器監。



 秋七月乙巳,詔京西、淮南、兩浙、江西、荊湖等六路各置鑄錢監。丙午,大食陀婆離來貢。己酉,禱雨。甲寅,錄在京囚,死罪以下降一等,杖罪釋之。丁巳,詔沿邊吏殺熟戶以邀賞者戮之。乙丑,分河北為東、西路。丙寅夜,西北有聲如磑。



 八月壬申朔,遣賈昌衡等賀遼主生辰、正旦。甲申,罷簡州歲貢綿紬。甲午,賜熙河、涇原軍士特支錢。戊戌,復比閭族黨之法。



 九月
 壬寅,置兩浙和糴倉,立斂散法。戊申,詔興水利。辛亥,策武舉。戊午,岷州首領木令征以其城降,王韶入岷州。丙寅,太白犯鬥。戊辰,詔禱雨,決獄。



 冬十月辛未,章惇平懿、洽州蠻。辛巳,以復熙、河、洮、岷、疊、宕等州,御紫宸殿受群臣賀,解所服玉帶賜安石。甲申,朝獻景靈宮。丙戌,振兩漸、江、淮饑。壬辰,行折二錢。丁酉,遣使瘞熙河戰骨。



 十一月癸丑,中太一宮成,減天下囚罪一等,流以下釋之。乙卯,親祀太一宮。丙寅,大雪,詔京畿收養老弱凍餒者。十
 二月戊子,詔決開封府囚。丙申,遼遣耶律洞等來賀正旦。



 七年春正月辛亥,賞復岷、洮等州功,西京左藏庫使桑湜等遷官有差。壬子,幸中太一宮宴從臣,又幸大相國寺,御宣德門觀燈。乙卯,封皇子俊為永國公。甲子,熊本平瀘夷。



 二月辛未,於闐來貢。發常平米振河陽饑民。癸未,詔三司歲會天下財用出入之數以聞。乙丑,禱雨。辛卯,置客省、引進、四方館、閣門使副等員。乙未,知河州景
 思立等與青宜結鬼章戰於踏白城,敗死。廢遼州。



 三月壬寅,木徵、鬼章寇岷州,高遵裕遣包順等擊走之。慮囚,減死罪一等,杖以下釋之。癸卯,以旱,避殿減膳。乙巳,白虹貫日。丙午,遣使分行諸路,募武士赴熙河。庚戌,詔熙河死事者家給錢有差。罷兩浙增額預置紬絹。令諸路監司察留獄。癸丑,群臣表請復膳,不許。丙辰,遼遣林牙蕭禧來言河東疆界,命太常少卿劉忱議之。己未,行方田法。甲子,遣使報聘於遼。乙丑,詔以災異求直言。



 夏四
 月癸酉,以旱,罷方田。是日,雨。遼遣耶律永寧等來賀同天節。乙亥,王韶破西蕃於結河川。丙子,御殿復膳。己卯,以高遵裕為岷州團練使。甲申,詔邊兵死事無子孫者,廩其親屬終身。乙酉,王韶進築珂諾城,與蕃兵連戰,破之,斬首七千餘級,焚二萬餘帳,木征率酋長八十餘人詣軍門降。雨雹。丙戌,王安石罷知江寧府。以韓絳同中書門下平章事、監修國史,翰林學士呂惠卿參知政事。置沅州。丁酉,詔王韶發木征及其家赴闕。遼遣樞密副
 使蕭素議疆界於代州境上。



 五月戊戌朔,減熙河路囚罪一等,流以下釋之。辛丑,詔河州瘞蕃部暴骸。壬寅,雨雹。癸卯,大雨雹。辛亥,罷賢良方正等科。乙丑,大雨水,壞陜、平陸二縣。



 六月戊寅,賜討洮州將士特支錢。丁亥,作渾儀、浮漏。廣州鳳凰見。以木徵為榮州團練使,賜姓名趙思忠。



 秋七月癸卯,群臣五上尊號曰紹天憲古文武仁孝皇帝,不許。癸亥,詔河北兩路捕蝗。又詔開封、淮南提點、提舉司檢覆蝗旱。以米十五萬石振河北西路災
 傷。



 八月丁丑,賜環慶安撫司度僧牒,以募粟振漢蕃饑民。遣張芻等賀遼主生辰、正旦。辛卯,詔免淮南、開封府來年春夫,除放邢、洺等州秋稅。癸巳,置場於南熏、安上門,給流民米。集賢院學士宋敏求上編修《閣門儀制》。



 九月戊戌,以時雨降,詔河北、京西、陜西、淮南等路勸民趨耕,有因事拘系者釋之。壬子,三司火。癸丑,置京畿、河北、京東西路三十七將。甲寅,詔樞密院議邊防。



 冬十月壬申,詔韓琦、富弼、文彥博、曾公亮條代北事宜以聞。戊寅,
 詔浙西路提舉司出米振常、潤州饑。庚辰,置三司會計司,以韓絳提舉。辛巳,以河北災傷,減州、軍文武官員。癸巳,以常平米於淮南西路易饑民所掘蝗種,又振河北東路流民。



 十一月己未,祀天地於圜丘,赦天下。十二月丙寅,省熙、河、岷三州官百四十一員。丁卯,文武官加恩。己丑,遼遣耶律寧等來賀正旦。是歲,高麗入貢,淯井、長寧夷十郡及武都夷內附。



 八年春正月庚子,蔡挺罷判南京留司御史臺,馮京罷
 知亳州。丙午,分京東為東、西路。輟江南東路上供米,均給災傷州軍。丁未,御宣德門觀燈。乙卯,詔出使廷臣,所至採吏治能否以聞。雨木冰。戊午,詔所在流民願歸業者,州縣繼遣之。己未,洮西安撫司以歲旱,請為粥以食羌戶饑者。



 二月甲子,增陜西錢監改鑄大錢。癸酉,以王安石同中書門下平章事。戊寅,詔樞密副都承旨張誠一等,以李靖營陣法教殿前馬步軍。乙酉,初行河北戶馬法。丙戌,停京畿土功七年。



 三月丁酉,振潤州饑。戊戌,
 知河州鮮於師中乞置蕃學,教蕃酋子弟,賜田十頃,歲給錢千緡,增解進士二人,從之。庚子,遼蕭禧再來,遣韓縝往河東會議。癸丑,知制誥沉括報聘。復振常、潤饑民。戊午,太白晝見。



 夏四月乙丑,詔減將作監冗官。丁卯,遼遣耶律景熙等來賀同天節。乙亥,正僖祖禘祫東向位。戊寅,以吳充為樞密使。壬午,湖南江水溢。



 閏月乙未,陳升之罷為鎮江軍節度使、判揚州。廣源州劉紀寇邕州,歸化州儂智會敗之。壬寅,沉括上《奉元歷》。癸卯,以宣徽
 北院使張方平判永興軍。分秦鳳路兵為四將。壬子,沂州民朱唐告前餘姚縣主簿李逢謀反,辭連右羽林大將軍世居及河中府觀察推官徐革,命御史中丞鄧綰、知諫院範百祿、御史徐禧雜治之。獄具,世居賜死,逢、革等伏誅。甲寅,錄趙普後。乙卯,詔西南蕃五姓蠻五年一入貢。



 五月辛酉朔,慮囚,降死罪一等,杖以下釋之。甲子,分環慶兵為四將。丁丑,雨土及黃毛。甲申,熙河路蕃官殿直頓埋謀叛,伏誅。己丑,遣使振鄜延、環慶饑。



 六月乙
 未,日上有五色雲。丙午,釃汴水入蔡河以通漕。己酉,頒王安石《詩》、《書》、《周禮義》於學官。辛亥,以安石為尚書左僕射兼門下侍郎。戊午,太師魏國公韓琦薨。己未,以琦配饗英宗廟庭。



 秋七月甲子,虔州江水溢。戊寅,太白晝見。戊子,分涇原兵為五將。命韓縝如河東割地。八月庚寅朔,日當食,雲陰不見。癸巳,募民捕蝗易粟,苗損者償之,仍復其賦。丙申,遣謝景溫等賀遼主生辰、正旦。減官戶役錢之半。詔發運司體實淮南、江東、兩浙米價,州縣所
 存上供米毋過百萬石,減直予民,斗錢勿過八十。庚戌,韓絳罷。發河北、京東兵及監牧卒修都城。丁巳,大閱。



 九月庚申朔,王安石兼修國史。立武舉絕倫法。



 冬十月庚寅,呂惠卿罷知陳州。乙未,彗出軫。己亥,詔以災異數見,不御前殿,減常膳,求直言。壬寅,赦天下。罷手實法。丁未,彗不見。丙辰,御殿復膳。



 十一月戊寅,交址陷欽州。壬午,立陜西蕃丁法。甲申,交址陷廉州。丙戌,渝州改南平軍。十二月丙申,浚河。壬寅,以翰林學士元絳參知政事,龍
 圖閣直學士曾孝寬簽書樞密院事。辛亥,天章閣待制趙離為安南道招討使,嘉州防禦使李憲副之,以討交址。癸丑,遼遣耶律世通等來賀正旦。甲寅,熙河路木宗城首領結彪謀叛,熟羌日腳族青廝扒斬其首來獻,補下班殿侍。



 九年春正月乙丑,雨木冰。戊辰,交址陷邕州,知州蘇緘死之。己卯,下溪州刺史彭師晏及天賜州降。庚辰,遣使祭南嶽、南海,告以南伐。辛巳,贈蘇緘奉國軍節度使,謚
 忠勇,以其子子元為西頭供奉官、閣門祗候。



 二月戊子,宣徽南院使郭逵為安南道招討使,罷李憲,以趙離副之。詔占城、占臘合擊交址。己丑,宗哥首領鬼章寇五牟谷,蕃官藺氈訥支等邀擊,大破之。己亥,以出師,罷春宴。乙卯,雨雹。



 三月內辰朔,進仁宗婉容周氏為妃。辛酉,御集英殿策進士。恤欽、廉、邕三州死事家,瘞戰亡士,賊所蹂踐,除其田征。甲戌,賜進士、諸科及第出身五百九十六人。丁丑,以廣西進士徐伯祥為右侍禁、欽廉白州巡
 檢。宗哥首領鬼章寇五牟谷,熙河鈐轄韓存寶敗之。庚辰,以種諤知岷州。



 夏四月辛卯,遼遣耶律庶幾等來賀同天節。乙未,以遼主母喪,罷同天節上壽。戊戌,復廣濟河漕。癸卯,詔廣南亡沒士卒及百姓為賊殘破者,轉運、安撫司具實,並議振恤以聞。甲辰,給空名告身付安南,以招降賞功。詔諸路募武勇赴廣西。贈廣西死事將士官有差。丙午,遣王克臣等吊慰於遼。辛亥,茂州夷寇邊,遣內侍押班王中正經制。甲寅,遼遣耶律孝淳以國母
 喪來告,帝發哀成服,輟視朝七日。



 五月丙辰朔,詔邕州沿邊州峒首領來降者,周惠之。癸亥,詔試醫學生。丙寅,分兩浙為東、西路。丁卯,城茂州。壬申,詔安南諸軍過嶺有疾者,所至護治。丙子,大理國來貢。庚辰,靜州下首領董整白等來降。



 六月丁亥,詔安南將吏,視軍士有疾者月以數聞。己丑,綿州都監王慶、崔昭用、劉珪、左侍禁張乂援戰茂州,死之。詔慶等子與借職,女出嫁,夫與奉職;白丁王禹錫等二人,賜錢其家。辛卯,詔濱海富民得養
 蜑戶,毋致為外夷所誘。己亥,慮囚,降死罪一等,杖以下釋之。癸卯,以水源等洞蠻主儂賀等七人為定遠、寧遠將軍。



 秋七月丙辰,朱崖軍黎賊黃嬰入寇,詔廣南西路嚴兵備之。庚申,關以西蝗蝻、虸蚄生。壬戌,築下溪州,改名會溪城。癸亥,靜州將楊文緒結蕃部謀叛,王中正斬之以徇。詔廣西死事官無子孫者許立後。乙丑,詔自今遇大禮推恩,官昭憲太后族一人。是月,安南行營次桂州,郭逵遣鈐轄和斌等督水軍涉海自廣東入,諸軍自
 廣南入。



 八月甲申朔,齊州監務左班殿直孫紀死賊,錄其一子為三班借職。戊子,以文彥博守太保兼待中,行太原尹。己丑,遣程師孟等賀遼主生辰、正旦。罷鬻祠廟錢。丁酉,禁北邊民闌出穀粟。庚子,占城來貢。



 九月戊午,浚汴河。丙寅,詔罷都大制置河北河防水利司。己卯,遼遣使回謝。詔恤嶺南死事家,表將士墓。



 冬十月乙酉,太白晝見。乙未,詔東南諸路教閱新軍。丙午,王安石罷判江寧府。以吳充監修國史,王珪為集賢殿大學士,並同
 中書門下平章事。資政殿學士馮京知樞密院。辛亥,除放沅州歸明人戶去年倚閣秋稅。



 十一月乙卯,賜廣南東路空名告敕,募入錢助軍。辛酉,錄唐相魏徵後同州司士參軍道嚴,流內銓特免試注官。乙亥,以安南行營將士疾疫,遣同知太常禮院王存禱南嶽,遣中使建祈福道場。己卯,洮東安撫司言包順等破鬼章兵於多移穀。壬午,鬼章寇岷州,知州種諤等敗之鐵城。十二月丙戌,安南偽觀察使劉紀降。置司農丞。庚寅,子傭生。丁酉,
 詔岷州界經鬼章兵燹者賜錢,脅從來歸者釋其罪。癸卯,郭逵敗交址於富良江,獲其偽太子洪真,李乾德遣人奉表詣軍門降,逵遂班師。丁未,遼遣耶律運等來賀正旦。庚戌,詔有得鬼章、冷雞樸首者,賞之。置威戎軍。



 十年春正月乙丑,御宣德門觀燈。戊辰,仙韶院火,不視朝。己巳,白虹貫日二月甲申,以崇信軍節度使宗旦同中書門下平章事。戊子,以鬼章敗,種諤等賞官有差。辛卯,日中有黑子。甲午,詔宗室使相雖及十年,更不取旨
 磨勘。丁酉,詔諸州歲以十一月給老疾貧乏者粟,盡三月乃止。己亥,以王韶知洪州。丙午,以復廣源、蘇茂等州,群臣表賀,赦廣州囚罪一等,徒以下釋之。賜行營諸軍錢,民緣征役者恤其家。以廣源州為順州,赦李乾德罪。以郭逵判潭州,趙離知桂州。己酉,以交址降,赦廣南東路、荊湖南路系囚,餘各降一等,徒以下釋之。



 三月辛未,慮囚,降死罪一等,杖以下釋之。壬申,詔州縣捕蝗。



 夏四月辛巳,復置憲州。乙酉,遼遣蕭儀等來賀同天節。癸巳,文
 州蕃賊寇邊,州兵擊走之。丁酉,賜熙河路兵特支錢,戰死者賜帛,免夏秋稅。



 五月戊午,詔修仁宗、英宗史。甲戌,太白晝見。



 六月壬午,注輦國朝貢。癸巳,王安石以使相為集禧觀使。丁未,置岷州鐵城堡。



 秋七月甲寅,禱雨。丁巳,令諸路歲上縣令課績。辛酉,群臣五上尊號曰奉天憲古文武仁孝皇帝,不許。乙亥,郭逵以安南失律,貶為左衛將軍。丙子,河決澶州曹村埽。



 八月壬寅,詔潭州置將及增武臣一員。遣蘇頌等賀遼主生辰、正旦。甲辰,詔
 侍從、臺諫、監司各舉文臣有才行者一人。



 九月庚戌,詔:「河決害民田,所屬州縣疏瀹,仍蠲其稅,老幼疾病者振之。」乙卯,詔:「諸傳宣、內批、面諭,事無法守,並從中書、樞密覆奏。其祈恩澤規免罪者劾之。」辛酉,詔鎮戎、德順軍各置都監一員。癸酉,立義倉。甲戌,宗樸兼侍中,封濮陽郡王。



 冬十月戊寅朔,宗樸薨。癸巳,昭化軍節度使宗誼封濮國公。詔濮王子以次襲封奉祀。戊戌,太子太師張昪卒。



 十一月庚午,以西蕃邈川首領董氈、都首領青宜結
 鬼章為廓州刺史,阿令骨為松州刺史。甲戌,祀天地於圜丘,赦天下。十二月丁丑朔,占城國獻馴象。壬午,詔改明年為元豐。甲申,以郊祀,文武官加恩。丁亥,封子傭為均國公。辛丑,遼遣耶律孝淳等來賀正旦。



 元豐元年春正月乙卯,以王安石為尚書左僕射、舒國公、集禧觀使。戊午,命詳定郊廟禮儀。詔減陳留捧日、天武等軍剩員。庚申,御宣德門,召從臣觀燈。乙丑,以太皇太后疾,驛召天下醫者。



 閏月辛巳,以翰林侍讀學士、寶
 文閣學士、提點中太一宮呂公著兼端明殿學士。己丑,詔贈尚書令韓琦依趙普故事。壬辰,樞密直學士孫固同知樞密院事。己亥,太傅兼侍中曾公亮薨。庚子,日中有黑子。癸卯,以公亮配饗英宗廟庭。



 二月庚戌,濮國公宗誼薨。甲寅,以邕州觀察使宗暉為淮康軍節度使,封濮國公。戊辰,詔赦安南戰棹都監楊從先等,仍論功行賞。



 三月辛巳,慮囚,降死罪一等,杖以下釋之。御邇英閣,沈季長進講《周禮》八法。癸未,詔內外文武官各舉堪應
 武舉一人。廣南西路經略司乞教閱峒丁,從之。乙未,御崇政殿閱諸軍。辰、沅猺賊寇邊,州兵擊走之。



 夏四月己酉,遼遣耶律永寧等來賀同天節。丙辰,詔增置兩浙路提舉官。庚申,詔除《九經》外,餘書不得出界。癸亥,太白晝見。乙丑,封虢國公宗諤為豫章郡王。戊辰,塞曹村決河,名其埽曰靈平。



 五月甲戌朔,賜塞河役死家錢。乙亥,詔試中刑法官以次推恩。



 六月癸卯朔,日有食之。乙巳,詔以靈平功遷太常博士苗師中等各一官。



 秋七月癸酉
 朔,命西上閣門使、忠州團練使韓存寶經制瀘州納溪夷。己亥,詔齊州預備水災。辛丑,夔州言甘露降。



 八月癸卯,西邊將訥兒溫、祿尊謀反,伏誅。丁未,詔河北被水者蠲其租。甲寅,遣黃履等賀遼主生辰、正旦。戊午,以韓絳為建雄軍節度使。己巳,詔濱、棣、滄三州被水民以常平糧貸之。庚午,詔青、齊、淄三州給流民食。



 九月癸酉,交址來貢。癸未,李乾德表乞還廣源等州,詔不許。乙酉,以端明殿學士呂公著、樞密直學士薛向並同知樞密院事。
 詔祀天地及配帝並用特牲。是月,武康軍嘉禾生,河中府甘露降。



 冬十月庚戌,定秋試諸軍賞格。侍禁仵全死事,錄其弟宣為三班借職。辛亥,韓存寶破瀘夷後城十有三囤。癸亥,於闐來貢。



 十一月己丑,命龍圖閣直學士宋敏求等詳定正旦御殿儀注。癸巳,辰州猺賊叛,詔沅州兵討之。乙亥,罷文武功臣號。是月,梁縣嘉禾生。十二月丙午,日中有黑子,凡十二日。辛亥,錄囚,降死罪一等,杖以下釋之。丙辰,詔青州民王贇以復父仇免死,刺配
 鄰州。戊午,置大理寺獄。己未,詔罷都大提舉在京諸司庫務司。甲子,以婉容邢氏為賢妃。詔罷三司推勘公事官,減軍器監勾當公事,審官東院、流內銓及將作監、三班院主簿,左右軍巡判官。丙寅,遼遣耶律隆等來賀正旦。



 二年春正月乙亥,罷岢嵐、火山軍市馬。丙子,詔立高麗交易法。壬午,以容州管內觀察使、上柱國、南陽郡開國公楊遂為寧遠軍節度使。癸未,詔知沅州謝麟督捕徭
 賊。甲申,御宣德門觀燈。丁亥,詔以經義、論試宗室。甲午,京兆府學教授蔣夔乞以十哲從祀孔子,從之。詔辰州敘浦縣置龍潭堡。是月,穎州、壽州甘露降。



 二月甲寅,詔瘞漢州暴骸。日中有黑子。乙卯,以瀘州夷乞弟犯邊,詔王光祖等討之。丙辰,詔定解鹽歲額。乙丑,滄州饑,發倉粟振之。



 三月庚午朔,董氈遣使來貢。辛未,詔給地葬畿內寄菆之喪,無所歸者官瘞之。庚辰,親試禮部進士。壬午,試特奏名進士及武舉。癸未,試諸科明法。賜董氈緡
 錢、銀帛、對衣、金帶等物。丙戌,詔雄州兩輸戶南徙者諭令復業。庚寅,疏汴、洛。



 夏四月辛丑,幸金明池觀水嬉,宴射瓊林苑。甲辰,遼遣蕭晟等來賀同天節。丁巳,陳升之以檢校太尉依前同中書門下平章事、鎮江軍節度使、上柱國、秀國公致仕。己未,陳升之卒。癸亥,定正旦御殿儀。甲子,詔增審刑院詳議、詳斷官,罷刑部校法官。是月,南康軍甘露降,眉州生瑞竹。



 五月丙子,順州蠻叛,峒兵討平之。庚辰,詔以濮安懿王三夫人並稱王夫人,祔濮
 園。辛巳,太子太師致仕趙概上所集《諫林》。甲申,元絳罷知亳州。乙酉,詔安南軍死事孤寡廩給之。戊子,御史中丞蔡確參知政事。



 六月甲辰,廣西捕斬儂智春,執其妻子以獻。戊申,命蔡確參定編修《傳法寶錄》。癸丑,詔五路帥臣、副總管軍臣僚各舉任將領及大使臣者二人。甲寅,清汴成。辛酉,詔鎮寧軍節度使、魏國公宗懿追封舒王。是月,南康軍甘露降,忠州雨豆。



 秋七月甲戌,張方平以太子少師致仕。戊寅,詳定朝會儀。己卯,命中書句考
 四方詔獄。庚辰,以淮康軍節度使宗暉同中書門下平章事。丁亥,詳定郊廟禮儀。是月,陳州芝草生,南賓縣雨豆,瓊州甘露降。



 八月丙申朔,夏人寇綏德城,都監李浦敗之。辛丑,分涇原路兵為十一將。壬寅,復八作司為東、西兩司,各置監官,文臣一員、武臣二員。遣李清臣等賀遼主生辰、正旦。甲寅,詔:「增太學生舍為八十齋,齋三十人。外舍生二千人,內舍生三百人。月一私試,歲一公試,補內舍生。間歲一舍試,補上舍生。」以穎州為順昌軍節
 度。是月,曹州生瑞穀,河陽生芝草。



 九月癸未,降順昌軍囚罪一等,徒以下釋之。甲申,西南龍蕃來貢。丁亥,大宴集英殿。己丑,進婕妤朱氏為昭容。壬辰,出《馬步射格鬥法》頒諸軍。甲午,西南羅蕃、方蕃來貢。



 冬十月丙申,西南石蕃來貢。癸卯,置籍田令。詔立水居船戶,五戶至十戶為一甲。戊申,交址歸所掠民,詔以順州賜之。己酉,太皇太后疾,上不視事。庚戌,罷朝謁景靈宮,命輔臣禱於天地、宗廟、社稷。減天下囚死罪一等,流以下釋之。乙卯,太
 皇太后崩。戊午,詔易太皇太后園陵曰山陵。辛酉,以群臣七上表,始聽政。命王珪為山陵使。



 十一月癸未,始御崇政殿。丁亥,雨土。十二月乙巳,御史中丞李定上《國子監敕式令》並《學令》凡百四十條。丙午,復置御史六察。庚申,遼遣蕭寧等來賀正旦。是月,全州芝草生,桂州甘露降。



\end{pinyinscope}