\article{志第三十八 地理一}

\begin{pinyinscope}

 京城京畿路京東路京西路



 唐室既衰,五季迭興,五十餘年,更易八姓,宇縣分裂,莫之能一。宋太祖受周禪,初有州百一十一,縣六百三十八,戶九十六萬七千三百五十三。建隆四年,取荊南,得
 州、府三江陵府,歸、峽。



 縣一十七,戶一十四萬二千三百。平湖南,得州一十五、監一潭、衡、邵、郴、道、永、全、岳、澧、朗、蔣、辰、錦、溪、敘,桂陽監,縣六十六,戶九萬七千三百八十八。乾德三年,平蜀,得州、府四十六益、彭、眉、嘉、邛、蜀、綿、漢、資、簡、梓、黎、雅、陵、戎、瀘、維、茂、昌、榮、果、閬、渠、合、龍、普、利、興、文、巴、劍、蓬、壁、夔、忠、萬、集、開、渝、涪、黔、施、達、洋,興元府,縣一百九十八,戶五十三萬四千三十九。開寶四年,平廣南,得州六十廣、韶、潮、循、封、端、英、連、雄、龔、惠、康、恩、春、瀧、勤、新、高、潘、雷、羅、辨、桂、賀、昭、梧、蒙、恭、象、富、融、宜、柳、嚴、思唐、邕、澄、貴、蠻、橫、賓、欽、潯、容、牢、白、廉、黨、繡、鬱林、藤、竇、義、禺、順、瓊、崖、儋、萬安、振,縣二百一十四,戶一十七萬二百六十三。八年,平江南,
 得州一十九,軍三升、宣、歙、池、洪、潤、常、鄂、筠、饒、信、虔、吉、袁、撫、江、汀、建、劍、江陰、雄遠、建昌軍,縣一百八,戶六十五萬五千六十五。計其末年,凡有州二百九十七,縣一千八十六,戶三百九萬五百四。太宗太平興國三年,陳洪進獻地,得州二,漳、泉。縣十四,戶十五萬一千九百七十八。錢俶入朝,得州十三、軍一杭、蘇、越、湖、衢、婺、臺、明、溫、秀、睦、福、處,衣錦,縣八十六,戶五十五萬六百八十。四年,平太原,得州十、軍一並、汾、嵐、憲、忻、代、遼、沁、隆、石,寶興,縣四十,戶三萬五千二百二十。七年,李繼捧來朝,得州四夏、銀、綏、宥。雍熙元年,復以四州授繼捧,自
 後不復領於職方,縣八。雍熙中,天下上閏年圖,州、府、軍、監幾於四百。至是,天下既一,疆理幾復漢、唐之舊,其未入職方氏者,唯燕、雲十六州而已。



 至道三年,分天下為十五路,天聖析為十八,元豐又析為二十三:曰京東東、西,曰京西南、北,曰河北東、西,曰永興,曰秦鳳,曰河東,曰淮南東、西,曰兩浙,曰江南東、西,曰荊湖南、北,曰成都,曰梓、利、夔、曰福建,曰廣南東、西。東南際海,西盡巴僰,北極三關,東西六千四百八十五里,南北萬一千六百二十里。崇寧四年,復置京畿路。大
 觀元年,別置黔南路。三年,並黔南入廣西,以廣西黔南為名。四年,仍舊為廣南西路。當是時,天下有戶二千八十八萬二千二百五十八,口四千六百七十三萬四千七百八十四天下主客戶:自至道末四百一十三萬一千五百七十六,天禧五年,主戶六百三萬九千三百三十一,客戶不預焉。至嘉祐八年,主戶一千二百四十六萬二千五百三十一,口二千六百四十二萬一千六百五十一。至治平三年,天下主客戶一千四百一十八萬一千四百八十六,口二千五十萬六千九百八十。熙寧十年,戶一千四百二十四萬五千二百七十,口三千八十萬七千二百十一。元祐元年,戶一千七百九十五萬七千九十二,口四千七萬二千六百六。紹聖元年,戶一千九百一十二萬九百二十一,口四千
 二百五十六萬六千二百四十三。元符三年,戶一千九百九十六萬八百一十二,口四千四百九十一萬四千九百九十一。崇寧元年,戶二千二十六萬四千三百七,口四千五百三十二萬四千一百五十四。各府、州下戶口與總數少異,姑兩存之,視西漢盛時蓋有加焉。隋、唐疆理雖廣,而戶口皆有所不及。迨宣和四年,又置燕山府及雲中府路,天下分路二十六,京府四,府三十,州二百五十四,監六十三,縣一千二百三十四,可謂極盛矣。大抵宋有天下三百餘年,繇建隆初訖治平末,一百四年,州郡沿革無大增損。熙寧始務闢土,而種諤先取綏州,韓絳繼取
 銀州,王韶取熙河,章惇取懿、洽,謝景溫取徽、誠,熊本取南平,郭逵取廣源,最後李憲取蘭州,沉括取葭蘆、米脂、浮圖、安疆等砦。雖嘗以河東邊界七百里地與遼人,當時王安石議,蓋曰:「吾將取之,寧姑與之也。」迨元祐更張,葭蘆等四砦給賜夏人,而分畫久不能定。紹聖遂罷分畫,督諸路各乘勢攻討進築。自三年秋八月訖元符二年冬,凡陜西、河東建州一安西,軍二晉寧、綏德,關三龍平、會寧、金城,城九安西,平夏、威戎、興平、定邊、威羌、金湯、白豹、會川,砦二十八平羌、平戎、殄羌、暖泉、米脂、克戎、安疆、
 橫山、綏遠、寧羌、靈平、高平、西平、新泉、蕩羌、通峽、天都、臨羌、定戎、龕谷、大和、通秦、寧河、彌川、寧遠、神泉、烏龍,堡十開光、通塞、石門、通會、大和、通秦、寧河、彌川、寧川、三交,又取青唐鄯。邈川湟,寧塞廓,龍支宗哥等城。建中靖國悉還吐蕃故壤,稍紓民力。崇寧亟變前議,專以紹述為事,蔡京始任童貫、王厚,更取湟、鄯、廓三州二十餘壘。陶節夫、鐘傳、刑恕、胡宗回、曾孝序之徒,又相與鑿空駕虛,馳騖於元符封域之表。訖於重和,既立靖夏涇原、制戎鄜延、制羌西寧三城,雖夏人浸衰,而民力亦弊。西事甫定,北釁旋起。蓋自崇寧以來,益、梓、夔、
 黔、廣西、荊湖南、北迭相視效,斥大土宇,靡有寧歲,凡所建州、軍、關、城、砦、堡,紛然莫可勝紀。厥後建燕山、雲中兩路,粗閱三歲,禍變旋作,中原板蕩,故府淪沒,職方所記,漫不可考。



 高宗蒼黃渡江,駐蹕吳會,中原、陜右盡入於金,東畫長淮,西割商、秦之半,以散關為界,其所存者,兩浙、兩淮、江東西、湖南北、西蜀、福建、廣東、廣西十五路而已,有戶一千二百六十六萬九千六百八十四此寧宗嘉定十一年數。建國江左又百五十年,迨德祐丙子,遂並歸於我皇
 元版圖,而天下始復合為一焉。



 今據元豐所定,並京畿為二十四路,首之以京師,重帝都也。終之以燕、雲,以其既得而旋失,故附見於後。而凡四京之城闕宮室,及南渡行在之所,其可考者冠乎篇首,為《地理志》云。



 東京,汴之開封也。梁為東都,後唐罷,晉復為東京,宋因周之舊為都,建隆三年,廣皇城東北隅,命有司畫洛陽宮殿,按圖修之,皇居始壯麗矣。雍熙三年,欲廣宮城,詔殿前指揮使劉延翰等經度之,以居民多不欲徙,遂罷。
 宮城周回五里。



 南三門:中曰乾元宋初依梁、晉之舊,名曰明德,太平興國三年改丹鳳,大中祥符八年改正陽,明道二年改宣德。雍熙元年改今名,東曰左掖,西曰右掖。東西面門曰東華、西華舊名寬仁、神獸,開寶三年改今名。熙寧十年,又改東華門北曰謻門,北一門曰拱宸舊名玄武,大中祥符五年改今名,熙寧十年,改門內西橫門曰臨華。乾元門內正南門曰大慶,東、西橫門曰左、右升龍。左右北門內各二門,曰左、右長慶熙寧間,改左、右長慶隔門曰左、右嘉肅,左、右銀臺。東華門內一門曰左承天祥符乾德六年賜名,大中祥符元年正月,天書降其上,詔加「祥符」二字而增茸之。西華門內一門曰右承天。左承天
 門內道北門曰宣祐舊名光天,大中祥符八年改大寧,明道元年改今名。



 正南門內正殿曰大慶,東、西門曰左、右太和宋初曰日華、月華,大中祥符八年改今名。正衙殿曰文德宋初曰文明,雍熙元年改今名。熙寧間,改南門曰端禮。兩掖門曰東、西上閣,東、西門曰左、右嘉福宋初曰左、右勤政,明道元年十月改今名。大慶殿舊名崇元,乾德四年重修,改曰乾元,太平興國九年改朝元,大中祥符八年改天安,明道三年改今名。北有紫宸殿舊名崇德,明道元年改,視朝之前殿也。西有垂拱殿舊名長春,明道元年改。



 常日視朝之所也。次西有皇儀殿開寶四年,賜名滋福,明道元年十月改。又次西有集英殿舊名廣政,開寶三年曰大明,淳化間
 曰含光,大中祥符八年名會慶,明道元年十月改今名。宴殿也。殿後有需雲殿舊名玉華,後改瓊華,熙寧初改今名。



 東有升平樓舊名紫雲,明道元年改。



 宮中觀宴之所也。宮後有崇政殿舊名簡賢講武,太平興國二年改今名。熙寧間,改北橫門曰通極。



 閱事之所也。殿後有景福殿,殿西有殿北向,曰延和,便坐殿也大中祥符七年,建後苑東門,洎北向便殿成,賜名宣和門、承明殿,明道元年改端明,二年改今名。



 凡殿有門者,皆隨殿名。



 宮中又有延慶、舊名萬歲,大中祥符七年改。



 安福、觀文、舊名集聖,明道二年改肅儀,慶歷八年改今名。



 清景、慶雲、玉京等殿,壽寧堂舊名清凈,明道元年改。



 延春閣。舊名萬春,寶元元年改。



 福寧殿即延慶,明道元
 年改。



 東西有門曰左、右昭慶。觀文殿西門曰延真,其東真君殿曰積慶,前建感真閣。又有龍圖閣,下有資政、崇和、宣德、述古殿。天章閣下有群玉、蕊珠二殿,後有寶文閣,即壽昌閣,慶歷元年改。



 閣東西有嘉德、延康二殿,前有景輝門。後苑東門曰寧陽,即宣和門,明道元年改。



 苑內有崇聖殿、太清樓,其西又有宜聖、化成、即玉宸殿,明道元年改。



 金華、西涼、清心等殿,翔鸞、儀鳳二閣,華景、翠芳、瑤津三亭。延福宮有穆清殿,延慶殿北有柔儀殿,初有殿無名,章獻太后名曰崇徽,明道元年改寶慈,景祐二年改今
 名。



 崇徽殿北有欽明殿。舊名天和,明道元年改觀文,又改清居,治平三年改今名。



 延福宮北有廣聖宮,天聖二年建,名長寧,



 景祐二年改。內有太清、玉清、沖和、集福、會祥五殿,建流杯殿於後苑。明道元年八月,修文德殿成。是夜,禁中火,延燔崇德、長春、滋福、會慶、延慶、崇徽、天和、承明八殿,命宰相呂夷簡為修茸大內使,樞密副使楊崇勛副之,發京東西、河北、淮南、江東西路工匠給役,內出乘輿物,左藏庫易緡錢二十萬助其費,以故改諸殿名。



 又有慈德殿,楊太后所居,景祐元年賜名。



 觀稼殿,在後苑,觀種稻,景祐二年創建。



 延義閣,在崇政殿西。



 邇英閣,在崇政殿西南,蓋侍臣講讀之所也,與延義同,景祐三年賜名。



 隆儒殿,邇英閣後小殿,皇祐三年始賜名。



 慈壽殿,皇太后所居,治平元年賜名。



 慶壽宮,保
 慈宮,熙寧二年建。玉華殿,在後苑。



 基春殿,熙寧七年建,在玉華殿後。



 睿思殿,八年建。



 承極殿,元豐三年建。



 崇慶、隆祐二宮,元祐元年建。



 睿成宮,神宗所居東宮,紹聖二年賜名。



 宣和殿,在睿思殿後,紹聖二年四月殿成,其東側別有小殿曰凝芳,其西曰瓊芳,前曰重熙,後曰環碧。元符三年廢,崇寧初復作,大觀三年,徽宗制記刻石,實蔡京為之。



 聖瑞宮,皇太妃所居,因以名宮。顯謨閣,元符元年建,藏神宗御集,建中靖國元年,改曰熙明,尋復舊。



 玉虛殿,元符初建。



 玉華閣,大觀初建,在宣和殿後。



 親蠶宮,政和元年建。



 燕寧殿,在延福北,奉安仁宗慈聖光獻皇后御容。



 延福宮,政和三年春,新作於大內北拱辰門外。舊宮在後苑之西南,今其地乃百司供應之所,凡內酒坊、裁造院、油醋柴炭鞍轡等庫,悉移它處,又遷兩僧寺、兩軍營,而作
 新宮焉。始南向,殿因宮名曰延福,次曰蕊珠,有亭曰碧瑯玕。其東門曰晨暉,其西門曰麗澤。宮左復列二位。其殿則有穆清、成平、會寧、睿謨、凝和、昆玉,群玉,其東閣則有蕙馥、報瓊、蟠桃、春錦、疊瓊、芬芳、麗玉、寒香、拂雲、偃蓋、翠葆、鉛英、雲錦、蘭熏、摘金,其西閣有繁英、雪香、披芳、鉛華、瓊華、文綺、絳萼、穠華、綠綺、瑤碧、清陰、秋香、叢玉、扶玉、絳雲、會寧之北,疊石為山,山上有殿曰翠微,旁為二亭:曰雲巋,曰層巘。凝和之次閣曰明春,其高逾一百一十尺。閣之側為殿二:曰玉英,曰玉澗。其背附城,築土植杏,名杏岡,覆茅為亭,修竹萬竿,引流其下。宮之右為佐二閣,曰宴春,廣十有二丈,舞臺四列,山亭三峙。鑿圓池為海,跨海為二亭,架石梁以升山,亭曰飛華,橫度之四百尺有奇,縱數之二百六十有七尺。又疏泉為湖,湖中作堤以接亭,堤中作梁以通湖,梁之上又為茅亭、鶴莊、鹿砦、孔翠諸柵,蹄尾動數千,嘉花名木,類聚區別,幽勝宛若生成,西抵麗澤,不類塵境。初,蔡京命童貫、楊戩、賈
 詳、藍從熙、何欣等分任宮役,五人者因各為制度,不務沿襲,故號「延福五位」。東西配大內,南北稍劣。其東直景龍門,西抵天波門,宮東西二橫門,皆視禁門法,所謂晨暉、麗澤者也,而晨暉門出入最多。其後又跨舊城修築,號「延福第六位」。跨城之外浚壕,深者水三尺,東景龍門橋,西天波門橋,二橋之下疊石為固,引舟相通,而橋上人物外自通行不覺也,名曰景龍江。其後又闢之,東過景龍門至封丘門。景龍江北有龍德宮。初,元符三年,以懿親宅潛邸為之,及作景龍江,江夾岸皆奇花珍木,殿宇比比對峙,中途曰壺春堂,絕岸至龍德宮。其地歲時次第展拓,後盡都城一隅焉,名曰擷芳園,山水美秀,林麓暢茂,樓觀參差,猶艮岳、延福也。宮在舊城,因附見此。



 保和殿,政和三年四月作,九月殿成,總為屋七十五間。



 玉清神霄宮,政和三年建,舊名玉清和陽,在福寧殿東,七年改今名。



 上清寶陰宮,政和五年作,在景龍門東,對景暉門。既又
 作仁濟、輔正二亭於宮前,命道士施民符藥,徽宗時登皇城下視之。又開景龍門,城上作復道,通寶陰宮,以便齋醮之路,徽宗數從復道上往來。是年十二月,始張燈於景龍門上下,名曰「預賞」。其明年,乃有期門之事。



 萬歲山艮嶽。政和七年,始於上清寶菉宮之東作萬歲山。山周十餘里,其最高一峰九十步,上有亭曰介,分東、西二嶺,直接南山。山之東有萼綠華堂,有書館、八仙館、紫石巖、棲真嶝、覽秀軒、龍吟堂。山之南則壽山兩峰並峙,有雁池、噰噰亭,北直絳霄樓。山之西有藥寮,有西莊,有巢雲亭,有白龍沜、濯龍峽,蟠秀、練光、跨雲亭,羅漢巖。又西有萬松嶺,半嶺有樓曰倚翠,上下設兩關,關下有平地,鑿大方沼,中作兩洲:東為蘆渚,亭曰浮陽。西為梅渚,亭曰雪浪。西流為鳳池,東出為雁池,中分二館,東曰流碧,西曰環山,有閣曰巢鳳,堂曰三秀,東池後有揮雪廳。復由嶝道上至介亭,亭左復有亭曰極目,曰蕭森,右復有亭曰麗雲、半山。北俯景龍江,引江之上流注
 山間。西行為漱瓊軒,又行石間為煉丹、凝觀、圜山亭,下視江際,見高陽酒肆及清澌閣。北岸有勝筠庵、躡雲臺、蕭閑館、飛岑亭。支流別為山莊,為回溪。又於南山之外為小山,橫亙二里,曰芙蓉城,窮極巧妙。而景龍江外,則諸館舍尤精。其北又因瑤華宮火,取其地作大池,名曰曲江,池中有堂曰蓬壺,東盡封丘門而止。其西則自天波門橋引水直西,殆半里,江乃折南,又折北。折南者過閶闔門,為復道,通茂德帝姬宅。折北者四五里,屬之龍德宮。宣和四年,徽宗自為《艮岳記》,以為山在國之艮,故名艮岳。蔡條謂初名鳳凰山,後神降,其詩有「艮岳排空霄」,因改名艮岳。宣和六年,詔以金芝產於艮岳之萬壽峰,又改名壽嶽。蔡絳謂南山成,又改名壽嶽。岳之正門名曰陽華,故亦號陽華宮。自政和訖靖康,積累十餘年,四方花竹奇石,悉聚於斯,樓臺亭館,雖略如前所記,而月增日益,殆不可以數計。宣和五年,朱勉於太湖取石,高廣數丈,載以大舟,挽以千夫,鑿河斷橋,毀堰拆閘,
 數月乃至,賜號「昭功敷慶神運石」,是年,初得燕地故也。勉緣此授節度使。大抵群閹興築不肯已。徽宗晚歲,患苑囿之眾,國力不能支,數有厭惡語,由是得稍止。及金人再至,圍城日久,欽宗命取山禽水鳥十餘萬。盡投之汴河,聽其所之。拆屋為新,鑿石為炮,伐竹為笓籬。又取大鹿數百千頭殺之,以啖衛士云。



 舊城周回二十里一百五十五步。東二門:北曰望春,宋初名和政。



 南曰麗景。南面三門:中曰朱雀,東曰保康,大中祥符五年創建。



 西曰崇明。西二門:南曰宜秋,北曰閶闔。北三門:中曰景龍,東曰安遠,西曰天波。以上宋初仍梁、晉舊名,至太平興國四年改今名。



 新城周回五十里百六十五步。大中祥符九年增築,元豐元年重修,政和六年,詔有司度國之南展築京城,移
 置官司軍營。舊城周四十八里二百三十三步,周顯德三年築。



 南三門:中曰南熏,東曰宣化,西曰安上。東二門:南曰朝陽,北曰含輝。太平興國四年改寅賓,後復。



 西二門:南曰順天,北曰金耀。北四門:中曰通天,天聖初改寧德,後復。



 東曰長景,次東曰永泰,西曰安肅。初號衛州門。以上皆因周舊名,至太平興國四年,改今名。



 汴河上水門,南曰大通,太平興國四年賜名,天聖初,改順濟,後復今名。



 北曰宣澤。舊南北水門皆曰大通,熙寧十年改。



 汴河下,南曰上善,北曰通津。天聖初,改廣津,熙寧十年復。



 惠民河,上曰普濟,下曰廣利。廣濟河,上曰咸豐,下曰善利,舊名咸通。



 上南門曰永順。熙寧十年
 賜名。



 其後又於金耀門南置開遠門。舊名通遠。以上皆太平興國四年賜名,天聖初改今名。



 西京。唐顯慶間為東都,開元改河南府,宋為西京,山陵在焉。宮城周回九里三百步。城南三門:中曰五鳳樓,東曰興教,西曰光政。因隋、唐舊名。



 東一門,曰蒼龍。西一門,曰金虎。北一門,曰拱宸。舊名玄武,大中祥符五年改。



 五鳳樓內,東西門曰左、右永泰,門外道北有鸞和門,太平興國三年,以車輅院門改。



 右永泰門西有永福門。興教、光政門內各三門,曰:左、右安禮,左、
 右興善,左、右銀臺。蒼龍、金虎門內第二隔門曰膺福、千秋。膺福門內道北門曰建禮。



 正殿曰太極,舊名明堂,太平興國三年改。



 殿前有日、月樓、日華、月華門,又有三門,曰太極殿門。後有殿曰天興,次北殿曰武德。西有門三重,曰:應天、乾元、敷政。內有文明殿,旁有東上閣門、西上閣門,前有左、右延福門。後又有殿曰垂拱,殿北有通天門,柱廊北有明福門,門內有天福殿,殿北有寢殿曰太清,第二殿曰思政,第三殿曰延春。東又有廣壽殿,視朝之所也。北第
 二殿曰明德,第三殿曰天和,第四殿曰崇徽。天福殿西有金鸞殿,對殿南廊有彰善門。殿北第二殿曰壽昌,第三殿曰玉華,第四殿曰長壽,第五殿曰甘露,第六殿曰乾陽,第七殿曰善興。西有射弓殿。千秋門內有含光殿。拱宸門內西偏有保寧門,門內有講武殿,北又有殿相對。內園有長春殿、淑景亭、十字亭、九江池、砌臺、娑羅亭。宮城東西有夾城,各三里餘。東二門:南曰賓曜,北曰啟明。西二門:南曰金曜,北曰乾通。宮室合九千九百九十
 餘區。夾城內及內城北,皆左右禁軍所處。



 皇城周回十八里二百五十八步。南面三門:中曰端門,東西曰左、右掖門。東一門,曰宣仁。西三門:南曰麗景,與金曜相直,中曰開化,與乾通相直;北曰應福。內皆諸司處之。



 京城周五十二里九十六步。隋大業元年築,唐長壽二年增築。



 南三門:中曰定鼎,東曰長夏,西曰厚載。東三門:中曰羅門,南曰建春,北曰上東。西一門,曰關門。北二門:東曰安喜,西曰徽安。政和元年十一月,重修大內,至六年九月畢工。朱勝非言:「政和間,議朝謁諸陵,敕有司預為西幸之備,以蔡攸妻兄宋忭為京西都漕,修治西京大內,合
 屋數千間,盡以真漆為飾,工役甚大,為費不貲。而漆飾之法,須骨灰為地,科買督迫,灰價日增,一斤至數千。於是四郊塚墓,悉被發掘,取人骨為灰矣。」



 南京。大中祥符七年,建應天府為南京。宮城周二里三百一十六步。門曰重熙、頒慶。殿曰歸德。元豐六年,賜度僧牒修外城門及西橋等。



 京城周回一十五里四十步。東二門:南曰延和,北曰昭仁。西二門:南曰順成,北曰回鑾。南一門,曰崇禮。北一門,曰靜安。中有隔城,又有門二:東曰承慶,西曰祥輝。其東又有關城,南北各一門。



 北京。慶歷二年,建大名府為北京。宮城周三里一百九十八步,即真宗駐蹕行宮。城南三門:中曰順豫,東曰省風,西曰展義。東一門,曰東安。西一門,曰西安。順豫門內東西各一門,曰左、右保成。次北班瑞殿,殿前東西門二:東曰凝祥,西曰麗澤。殿東南時巡殿門,次北時巡殿,次靖方殿,次慶寧殿。時巡殿前東西門二:東曰景清,西曰景和。京城周四十八里二百六步,門一十七。熙寧九年,改正南南河門曰景風,南磚曰亨嘉,鼓角曰阜昌。正北北河門曰安平,北磚曰耀德。正東冠氏門曰華景,冠氏第二重曰
 春祺,子城東曰泰通。正西魏縣門曰寶成,魏縣第二重曰利和,子城西曰宣澤。東南朝城門曰安流,朝城第二重曰巽齊。西南觀音門曰安正,觀音第二重曰靜方。上水關曰善利,下水關曰永濟。內城創置北門曰靖武。鵩元豐七年,廢善利、永濟關。



 行在所。建炎三年閏八月,高宗自建康如臨安,以州治為行宮。宮室制度皆從簡省,不尚華飾。垂拱、大慶、文德、紫宸、祥曦、集英六殿,隨事易名,實一殿。重華、慈福、壽慈、壽康四宮,重壽、寧福二殿,隨時異額,實德壽一宮。延和、崇政、復古、選德四殿,本射殿也。慈寧殿,紹興九年,以太后有歸期建。



 欽先孝思殿,十五年建,在崇政殿東。



 翠寒堂,孝宗作。



 損齋,紹興末建,貯經史書,為燕坐之所。



 東宮,在麗正門內,孝宗、莊文、景獻、光宗皆常居之。



 講筵所,資善堂。在行宮門內,因書院而作。



 天章、龍圖、寶文、顯猷、徽猷、敷文、煥章、華文、寶謨九閣,實天章一閣。



 京畿路。皇祐五年,以京東之曹州,京西之陳、許、鄭、滑州為輔郡,隸畿內,並開封府,合四十二縣,置京畿路轉運使及提點刑獄總之。至和二年,罷京畿路轉運使、提點刑獄。其曹、陳、許、鄭、滑各隸本路,為輔郡如故。崇寧四年,
 京畿路復置轉運使及提點刑獄。先是,改開封府界為京畿路,是年,又於京畿四面置四輔郡:穎昌府為南輔,鄭州為西輔,澶州為北輔,建拱州於開封襄邑縣為東輔,並屬京畿。大觀四年,罷四輔,許、鄭、澶州還隸京西及河北路,廢拱州,復以襄邑縣隸開封府。政和四年,襄邑縣復為拱州,後與穎昌府、鄭州、開德府復為東、南、西、北輔。宣和二年,罷四輔,穎昌府、鄭州、開德府各還舊隸,拱州隸京東西路,舊開封府界依舊為京畿。



 開封府。崇寧戶二十六萬一千一百一十七,口四十四萬二千九百四十。貢方紋綾、方紋紗、藨席、麻黃、酸棗仁。縣十六:開封,赤。



 祥符,赤。東魏浚儀縣。大中祥符三年改。



 尉氏,畿。



 陳留,畿。



 雍丘,畿。



 封丘,畿。



 中牟,畿。宣和三年,改紂王城為青陽城。



 陽武,畿。



 延津,畿。舊酸棗縣,政和七年改。



 長垣,隋匡城縣。建隆元年,改為鶴丘,後又改。東明,畿。本東明鎮,乾德元年置。



 扶溝,畿。



 □焉陵,畿。



 考城,畿。崇寧四年,與太康同隸拱州。大觀四年,廢拱州,二縣復來隸。



 太康,畿。宣和二年,復隸拱州,六年,仍隸京畿。



 咸平。畿。舊通許鎮,隸陳留,咸平五年置縣。



 京東路。至道三年,以應天、兗徐曹青鄆密齊濟沂登萊
 單濮濰淄、淮陽軍廣濟軍清平軍宣化軍、萊蕪監利國監為京東路。熙寧七年,分為東、西兩路:以青淄濰萊登密沂徐州、淮陽軍為東路;鄆兗齊濮曹濟單州、南京為西路。元豐元年,割西路齊州屬東路,割東路徐州屬西路。元祐元年,諸提點刑獄不分路,京東東路、京東西路並為京東路,京西南路、京西北路並為京西路,秦鳳等路、永興軍等路並為陜府西路,河北西路、河北東路並為河北路,淮南西路、淮南東路並為淮南路,其後仍
 分為兩路。



 東路。府一,濟南。州七:青,密,沂,登,萊,濰,淄。軍一,淮陽。縣三十八。



 青州,望。北海郡,鎮海軍節度。建隆三年以北海縣置軍。淳化五年,改軍名。慶歷二年,初置京東東路安撫使。崇寧戶九萬五千一百五十八,口一十六萬二千八百三十七。貢仙紋綾、梨、棗。縣六:益都,望。



 壽光,望。



 臨朐,緊。



 博興,上。



 千乘,上。



 臨淄。上。



 密州,上。本防禦州。建隆元年,復為防禦。開寶五年,升為安化軍節度。後降防禦。六年,復為節度。崇寧戶一十四萬四千五百六十七,口三十二萬七千三百四十。貢絹、牛黃。縣五:諸城,望。



 安丘,望。唐輔郡,梁改安丘,晉膠西縣。開寶四年,復今名。



 莒,望。



 高密,上。



 膠西。元祐三年,以板橋鎮為膠西縣,兼臨海軍使。



 濟南府,上,濟南郡,興德軍節度。本齊州。先屬京東路。咸平四年,廢臨濟縣。元豐元年,割屬京東東路。政和六年,升為府。崇寧戶一十三萬三千三百二十一,口二十一
 萬四千六十七。貢綿、絹、陽起石、防風。縣五:歷城,緊。



 禹城,緊。



 章丘,中,景德三年,以章丘縣置清平軍。熙寧二年廢軍,即縣治置軍使。



 長清、中。至道二年,徙城於刺榆



 臨邑。中。建隆元年,河決公乘渡口,壞城。三年,移治孫耿鎮。政和元年,升為望。



 沂州,上,瑯琊郡,防禦。崇寧戶八萬二千八百九十三,口一十六萬五千二百三十。貢仙靈脾、紫石英、茯苓、鐘乳石。縣五:臨沂,望。



 承,望。



 沂水,望。



 費,望。



 新泰。中。



 登州,上,東牟郡,防禦。崇寧戶八萬一千二百七十三,口一十七萬三千四百八十四。貢金、牛黃、石器。縣四:蓬萊,
 望。



 文登,中。



 黃,望。



 牟平。緊。有乳山、閻家口二砦。



 萊州,中,東萊郡,防禦。崇寧戶九萬七千四百二十七,口一十九萬八千九百八。貢牛黃、海藻、牡礪、石器。縣四:掖,望。



 萊陽,望。



 膠水,望。



 即墨。中



 濰州,上,團練。建隆三年,以青州北海縣建為北海軍,置昌邑縣隸之。乾德三年,升為州,又增昌樂縣。崇寧戶四萬四千六百七十七,口一十萬九千五百四十九。貢綜絲素絁。縣三:北海,望。



 昌邑,望。本隋都昌縣,後廢。建隆三年,復置。



 昌樂。緊。本唐營
 丘縣,後廢。乾德中,復置安仁縣,俄又改。



 淄州,上,淄川郡,軍事。崇寧戶六萬一千一百五十二,口九萬八千六百一十。貢綾、防風、長理石。縣四:淄川,望。



 長山,中。



 鄒平,中下。景德元年,移治濟陽廢縣。



 高苑。下。景德三年,以縣置宣化軍。熙寧三年,廢軍為縣,隸州,即縣治置軍使。



 淮陽軍,同下州。太平興國七年,以徐州下邳縣建為軍,並以宿遷來屬。崇寧戶七萬六千八百八十七,口一十五萬四千一百三十。貢絹。縣二:下邳,望。



 宿遷。中



 西路。府四:應天,襲慶,興仁,東平。州五:徐,濟,單,濮,拱。軍一,廣濟。縣四十三。



 應天府,河南郡,歸德軍節度。本唐宋州。至道中,為京東路。景德三年,升為應天府。大中祥符七年,建為南京。熙寧五年,分屬西路。崇寧戶七萬九千七百四十一,口一十五萬七千四百四。貢絹。縣六:寧陵,畿。與楚丘同隸拱州。大觀四年,復來隸。政和四年,又撥隸拱州。宣和六年,復來隸。



 宋城,赤。



 穀熟,畿。



 下邑,畿。



 虞城,畿。



 楚丘。畿。



 襲慶府,魯郡,泰寧軍節度。本兗州。大中祥符元年,升為大都督。政和八年,升為府。崇寧戶七萬一千七百七十七,口二十一萬七千七百三十四。貢大花綾、墨、雲母、紫石英、防風、茯苓。縣七:瑕,上。大觀四年,以瑕丘縣為瑕縣。



 奉符,上。本漢乾封縣。開寶五年,移治岱嶽鎮。大中祥符元年改。



 泗水,上。



 龔,上。大觀四年,以龔丘縣為龔縣。



 仙源,中上。魏曲阜縣。大中祥符五年改。



 萊蕪,中。



 鄒。下。熙寧五年,省為鎮,入仙源。元豐七年復。



 監一,萊蕪。主鐵冶。



 徐州,大都督,彭城郡,武寧軍節度。本屬京東路,元豐元
 年,割屬京東西路。崇寧戶六萬四千四百三十,口一十五萬二千二百三十七。貢雙絲綾、紬、絹。縣五:彭城,望。



 沛,望。



 蕭,望。



 滕,緊。



 豐。緊。



 監二:寶豐,元豐六年置,鑄銅錢,八年廢。



 利國。主鐵冶。



 興仁府,輔,濟陰郡,彰信軍節度。本曹州。建中靖國元年,改賜軍額曰興仁。崇寧元年,升曹州為興仁府,復還舊節。大觀二年,以拱州為東輔,升督府。政和元年,罷督府,復為輔郡。崇寧戶三萬五千九百八十,口六萬六千九百三十一。貢絹、葶藶子。縣四:濟陰,望宛亭,望。元祐元年,改冤句縣為
 宛亭。



 乘氏,緊。



 南華。上。



 東平府,東平郡,天平軍節度。本鄆州。慶歷二年,初置京東西路安撫使。大觀元年,升大都督府。政和四年,移安撫使於應天府。宣和元年,改為東平府。崇寧戶一十三萬三百五,口三十九萬六千六十三。貢絹、阿膠。縣六:須城,望。



 陽谷,望。景德三年,徙孟店。



 中都,緊。



 壽張,上。



 東阿,緊。



 平陰。上。



 監一,東平。宣和二年復置。政和三年罷。



 濟州,上,濟陽郡,防禦。戶五萬七百一十八,口一十五萬
 九千一百三十七。貢阿膠。縣四:鉅野,望。



 任城,望。



 金鄉,望。



 鄆城,望。



 單州,上,碭郡,建隆元年,升為團練,崇寧戶六萬一千四百九,口一十一萬六千九百六十九。貢蛇床、防風。縣四:單父,望。



 碭山,望。



 成武,緊



 魚臺。上。



 濮州,上,濮陽郡,團練。崇寧戶三萬一千七百四十七,口五萬二千六百八十一。貢絹。縣四:鄄城,望。



 雷澤,緊。



 臨濮,上。



 範。上。



 拱州,保慶軍節度。本開封府襄邑縣。崇寧四年建為州,賜軍額,為東輔。以開封之考城、太康,南京之寧陵、楚丘、柘城來隸。大觀四年,廢拱州,復為襄邑縣,還隸開封。政和四年,復為州,又復為輔郡。宣和二年,罷輔郡,仍隸京東西路,以襄邑、太康、寧陵為屬縣,餘歸舊隸。六年,又以寧陵歸南京,太康歸開封,復割柘城來隸。縣二:襄邑畿。



 柘城。畿。



 廣濟軍。乾德元年,置發運務。開寶九年,改轉運司。太平
 興國二年,建為軍。四年,割曹、澶、濮、濟四州地,復置縣以隸焉。熙寧四年廢軍,以定陶縣隸曹州。元祐元年,復為軍。縣一,定陶。上。



 開封府,京東路,分為東西兩路,得兗、豫、青、徐之域,當虛、危、房、心、奎、婁之分,西抵大梁,南極淮、泗,東北至於海,有鹽鐵絲石之饒。其俗重禮義,勤耕絲任,浚郊處四達之會,故建為都。政教所出,五方雜居。睢陽當漕舟之路,定陶乃東運之沖,其後河截清水,頗涉艱阻。兗、濟山澤險迥,
 盜或隱聚。營丘東道之雄,號稱富衍,物產尤盛。登、萊、高密負海之北,楚商兼湊,民性愎戾而好訟斗。大率東人皆樸魯純直,甚者失之滯固,然專經之士為多。下邳俗尚頗類淮楚焉。



 京西路。舊分南、北兩路,後並為一路。熙寧五年,復分南、北兩路。



 南路。府一,襄陽。州七:鄧,隨,金,房,均,郢,唐。軍一,光化。縣三十一。



 襄陽府,望,襄陽郡,山南東道節度。本襄州。宣和元年,升為府。崇寧戶八萬七千三百七,口一十九萬二千六百五。貢麝香、白穀、漆器。縣六:襄陽,緊。



 鄧城,望。



 穀城,緊。



 宜城,中下。



 中盧,中下。隋義清縣。太平興國元年改。紹興五年,省入南漳。



 南漳。中下。



 鄧州,望,南陽郡,武勝軍節度。舊為上郡。政和二年,升為望郡。建隆初,廢臨瀨縣。崇寧戶一十一萬四千一百二十七,口二十九萬七千五百五十。貢白菊花。縣五:穰,上。



 南陽,中下。慶歷四年,廢方城縣為鎮入焉;元豐元年,改為縣,隸唐州。



 內鄉,中下。



 順陽。中下。太平
 興國六年,升順陽鎮為縣。



 淅川。中下。



 隨州,上,漢東郡,崇信軍節度。乾德五年,升為崇義軍節度。太平興國元年,改今名。崇寧戶三萬八百四,口六萬七千二十一。貢絹、綾、葛、覆盆子。縣三:隨,上。熙寧元年,廢光化縣為鎮入焉。



 唐城,中下。



 棗陽。中下。



 金州,上,安康郡,乾德五年,改昭化軍節度。崇寧戶三萬九千六百三十六,口六萬五千六百七十四。貢麩金、麝香、枳殼實、杜仲、白膠香、黃檗。縣五:西城,下。



 洵陽,中。乾德四年,廢
 淯陽縣入焉。



 漢陰,中。



 ……



 房州,下,房陵郡,保康軍節度。開寶中,廢上庸、永清二縣。雍熙三年並為軍。崇寧戶三萬三千一百五十一,口四萬七千九百四十一。貢麝香、絲寧布、鐘乳石、筍。縣二:房陵,上。



 竹山。下。



 均州,上,武當郡,武當軍節度。本防禦。乾德六年,移入上州防禦。宣和元年,賜軍額。崇寧戶三萬一百七,口四萬四千七百九十六。貢麝香。縣二:武當,上。



 鄖鄉。上。



 郢州,上,富水郡,防禦。崇寧戶四萬七千二百八十一,口七萬八千七百二十七。貢白紵。縣二:長壽,上。



 京山。下。



 唐州,上,淮安郡,建隆元年,升為團練。開寶五年,廢平氏縣。崇寧戶八萬九千九百五十五,口二十萬二千一百七十二。貢絹。縣五:泌陽,中下。



 湖陽,中下。有銀場。



 比陽,中下。



 桐柏,下。開寶六年,移治淮瀆故廟。



 方城。下。後魏縣。慶歷四年,廢為鎮,入鄧州南陽縣,元豐元年,復為縣,隸州。



 光化軍,同下州。乾德二年,以襄州陰城鎮建為軍,析谷城縣三鄉,置乾德縣隸焉。熙寧五年廢軍,改乾德為光
 化縣,隸襄州。元祐初,復為軍。縣一,乾德。望。



 北路。府四:河南,穎昌,淮寧,順昌。州五:鄭,滑,孟,蔡,汝。軍一,信陽。縣六十三。



 河南府,洛陽郡,因梁、晉之舊為西京。熙寧五年,分隸京西北路。崇寧戶一十二萬七千七百六十七,口二十三萬三千二百八十。貢蜜、蠟、瓷器。縣十六:河南,赤。



 洛陽,赤。熙寧五年,省入河南,元祐二年復。永安,赤。奉陵寢。景德四年,升鎮為縣。



 偃師,畿。慶歷二年廢,四年復,熙寧五年,省入緱氏,八年,復置,省緱氏縣為鎮隸焉。



 穎陽,畿。廢歷二年,廢為鎮,四年,復。熙寧二年,省入登
 封,元祐二年,復置。



 鞏,畿。



 密,畿。崇寧四年,割隸鄭州,宣和二年,還隸府。



 新安,畿。



 福昌,畿。熙寧五年,省入壽安,元祐元年,復為縣。



 伊陽,畿。熙寧二年,割欒川冶鎮入虢州盧氏縣。五年,廢伊闕縣為鎮入河南,六年,改隸伊陽。



 澠池,畿。景祐四年,改鐵門鎮曰延禧。



 永寧,畿。



 長水。畿。



 壽安,畿。慶歷三年,廢為鎮,四年,復。



 河清,畿。開寶元年,移治白波鎮。熙寧八年閏四月,置鐵監。



 登封。畿。



 監一,阜財。熙寧七年置,鑄銅錢。



 穎昌府,次府,許昌郡,忠武軍節度。本許州。元豐三年,升為府。崇寧四年,為南輔,隸京畿。大觀四年,罷輔郡。政和四年,復為輔郡,隸京畿。宣和二年,復罷輔郡,依舊隸京
 西北路。崇寧戶六萬六千四十一,口一十六萬一百九十三。貢絹、藨席。縣七:長社,次赤。熙寧四年,省許田縣為鎮入焉。



 郾城,次畿。



 陽翟,次畿。



 長葛,次畿。



 臨穎,次畿。



 舞陽,次畿。



 郟。中。元隸汝州,崇寧四年來隸。



 鄭州,輔,滎陽郡,奉寧軍節度。熙寧五年,廢州,以管城、新鄭隸開封府;省滎陽、滎澤縣為鎮入管城,原武縣為鎮入陽武。元豐八年,復州。元祐元年,還舊節;復以滎陽、滎澤、原武為縣,與滑州並隸京西路。崇寧四年,建為西輔。大觀四年,罷輔郡。政和四年,又復。宣和二年,又罷。崇寧
 戶三萬九百七十六,口四萬一千八百四十八。貢絹、麻黃。縣五:管城,望。



 滎澤,中。



 原武,上。



 新鄭,上。



 滎陽。緊。



 滑州,輔,靈河郡,太平興國初,改武成軍節度。熙寧五年,廢州,縣並隸開封府。元豐四年,復舊,縣復來隸。元祐元年,還舊節度。崇寧戶二萬六千五百二十二,口八萬一千九百八十八。貢絹。縣三:白馬,中。熙寧三年,廢靈河縣隸焉。



 韋城,望。



 胙城。緊。



 孟州,望。河陽三城節度。政和二年,改濟源郡。崇寧戶三
 萬三千四百八十一,口七萬一百六十九。貢梁米。縣六:河陽,望。



 濟源,望。



 溫,望。



 汜水,上。熙寧五年,省入河陰。元豐二年復置。大中祥符四年,改武牢關曰行慶。



 河陰,中。



 王屋。中。熙寧五年,自河南來隸。



 蔡州,緊,汝南郡,淮康軍節度。崇寧戶九萬八千五百二,口十八萬五千一十三。貢綾。縣十:汝陽,上。



 上蔡,上。



 新蔡,中。



 褒信,中。



 遂平,中。



 新息,中。



 確山,中。隋朗山縣。大中祥符五年改。



 真陽,中。



 西平,中。



 平輿。中。



 淮寧府,輔,淮陽郡,鎮安軍節度。本陳州。政和三年,改輔
 為上。宣和元年,升為府。崇寧戶三萬二千九十四,口一十五萬九千六百一十七。貢紬、絹。縣五:宛丘,緊。



 項城,上。



 商水,中。



 西華,中。南頓。中。熙寧六年,省為鎮,入商水、項城二縣。元祐元年復。



 順昌府,上,汝陰郡,舊防禦,後為團練。開寶六年,復為防禦。元豐二年,升順昌軍節度。舊穎州,政和六年,改為府。崇寧戶七萬八千一百七十四,口一十六萬六百二十八。貢紬、絁、綿。縣四:汝陰,望。開寶六年,移治於州城東南十里。



 泰和,望。



 穎上,緊。



 沉丘。緊。



 汝州,輔,臨汝郡,陸海軍節度。本防禦州。政和四年,賜軍額。崇寧戶四萬一千五百八十七,口一十四萬一千四百九十五。貢絁、絹。縣五:梁,中。



 襄城,緊。



 葉,上。



 魯山,中



 寶豐。中。舊名龍興,熙寧五年,省為鎮,入魯山。元祐元年復。宣和二年,改為寶豐縣。



 信陽軍,同下州。開寶九年,降為義陽軍,廢鐘山縣。太平興國元年,改為信陽軍。崇寧戶九千九百五十四,口二萬五十,貢絲寧布。縣二:信陽,中下。



 羅山。中下。開寶九年廢,雍熙二年復置。



 京西南、北路,本京西路,蓋《禹貢》冀、豫、荊、兗、梁五州之域,
 而豫州之壤為多,當井、柳、星、張、角、亢、氐之分。東暨汝、穎,西被陜服,南略鄢、郢,北抵河津。絲、枲、漆、纊之所出。而洛邑為天地之中,民性安舒,而多衣冠舊族。然土地褊薄,迫於營養。盟津、滎陽、滑臺、宛丘、汝陰、穎川、臨汝在二京之交,其俗頗同。唐、鄧、汝、蔡率多曠田,蓋自唐季之亂,土著者寡。太宗遷晉、雲、朔之民於京、洛、鄭、汝之地,墾田頗廣,民多致富,亦由儉嗇而然乎!襄陽為汴南巨鎮,淮安、隨、棗陽、西城、武當、上庸、東梁、信陽,其習俗近荊楚。



\end{pinyinscope}