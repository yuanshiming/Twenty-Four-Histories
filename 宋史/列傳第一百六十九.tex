\article{列傳第一百六十九}

\begin{pinyinscope}

 ○婁機沈煥舒璘附曹彥約範應鈴徐經孫



 婁機,字彥發,嘉興人。乾道二年進士,授鹽官尉。丁母憂,服除,調含山主簿。郡委治銅城圩八十有四,役夫三千
 有奇,設廬以處之,器用材植,一出於官,民樂勸趨,兩旬告畢。七攝鄰邑,率以治績聞。調於潛縣丞,輕賦稅,正版籍,簡獄訟,興學校。遭外艱,免喪,為江東提舉司干辦公事,易淮東,已而復舊,改知西安縣。巨室買地為塋域,發地遇石,復索元價。機曰:「設得金,將誰歸?」通判饒州,平反冤獄。蜀帥袁說友闢參議幕中,不就,改乾辦諸司審計司。轉對,請裁損經費,又論刑名疑慮之敝。遷宗正寺主簿,為太常博士、秘書郎,請續編《中興館閣書目》,又請寬
 恤淮、浙被旱州縣。



 時皇太子始就外傅,遴選學官,以機兼資善堂小學教授。機日陳正言正道,又以累朝事親、修身、治國、愛民四事,手書以獻,太子置之坐右,朝夕觀省。隨事開明,多所裨益。遷太常丞,仍兼資善。旋遷右曹郎官、秘書省著作郎,改兼駕部。都城大火,機應詔上封事,力言朝臣務為奉承,不能出己見以裨國論;外臣不稱職,至苛刻以困民財;將帥偏裨務為交結,而不知訓閱以強軍律。時年七十,丐閑,不許。太子得機所著《廣干
 祿字》一編,尤喜,命戴溪跋之。擢監察御史,講未退而除命頒,太子戀戀幾不忍舍,機亦為之感涕。



 論京官必兩任、有舉主、年三十以上,方許作縣。又論郡守輕濫太甚,貽害千里。蘇師旦怙勢妄作,蒙蔽自肆,語及者皆罪去,而獨憚機。韓侂胄議開邊,機極口沮之,謂:「恢復之名非不美,今士卒驕逸,遽驅於鋒鏑之下,人才難得,財力未裕,萬一兵連禍結,久而不解,奈何?」侂胄聞之不說,其議愈密,外廷罔測。又上疏極論:「雖密謀人莫得知,而羽書
 一馳,中外皇惑。」侍御史鄧友龍初不知兵,騰書投合,妄薦大將,既召還,專主此議。機語友龍曰:「今日孰可為大將?孰可為計臣?正使以殿巖當之,能保其可用乎?」



 遷右正言兼侍講,首論廣蓄人才,乞詔侍從、臺諫、學士、待制、三牙管軍各舉將帥邊郡一二人,召問甄拔,優養以備緩急。進太常少卿兼權中書舍人,詔遣宣諭荊、襄,機昌言曰:「使往慰安人情則可,必欲開邊啟釁,有死而已,不能從也。」泗州捷聞,愈增憂危,且曰:「若自此成功,以攄列
 聖之宿憤,老臣雖死亦幸,謫官,但恐進銳退速,禍愈深耳。」友龍至不能堪曰:「不逐此人,則異議無所回。」機遂以言去。



 侂胄誅,召為吏部侍郎兼太子左庶子,還朝,言:「至公始可以服天下,權臣以私意橫生,敗國殄民,今當行以至公。若曰私恩未報,首為汲引,私仇未復,且為沮抑,一涉於私,人心將無所觀感矣。」又言:「兩淮招集敢勇,不難於招而難於處。若非繩以紀律,課其勤惰,必為後害。」仍請檢校權臣、內侍等沒入家貲,專為養兵之助。機
 裏人有故官吏部,喪未舉而子赴調者,機謂彼既冒法禁,而部胥不之問,即撻數吏,使之治葬而後來。聞者韙之。



 兼太子詹事,著《歷代帝王總要》以裨考訂。遷給事中。海巡八廂親從、都軍頭、指揮使年勞轉資,恩旨太濫,乞收寢未應年格之人,年已及者予之,帝稱善良久。飛蝗為災。機應詔言:「和議甫成,先務安靜,葺罅漏以成紀綱,節財用以固邦本,練士卒以壯國威。」



 遷禮部尚書兼給事中,擢同知樞密院事兼太子賓客,進參知政事。當干戈
 甫定,信使往來之始,瘡痍方深,敝蠹紛然,機彌縫裨贊甚多。尤惜名器,守法度;進退人物,直言可否,不市私恩,不避嫌怨。有舉員及格,當改秩作邑而必欲朝闕,機曰:「若是則有勞者何以勸?孤寒者何以伸?若至上前,自應執奏。」堂吏寄資未仕,而例以升朝官賞陳乞封贈,機曰:「進士非通籍不能及親,汝輩乃以白身得之耶?」嘉定二年八月,行皇太子冊命,機攝中書令讀冊。九月祀明堂,為禮儀使。數上章告老,帝不許,皇太子遣官屬勉留之。
 以資政殿學士知福州,力辭。提舉洞霄宮以歸,遂卒,贈金紫光祿大夫,加贈特進。



 機初登第,其父壽戒之曰:「得官誠可喜,然為官正自未易爾!」機撫其弟模、棟,卒為善士。居鄉以誠接物,是非枉直判於語下,不為後言,人憚而服之。稱獎人才,不遺寸長,訪問賢能,疏列姓名及其可用之實,以備採取,其所薦進,亦不欲人之知也。所著復有《班馬字類》。機深於書學,尺牘人多藏弆云。



 沈煥,字叔晦,定海人。試入太學,始與臨川陸九齡為友,
 從而學焉。乾道五年舉進士,授餘姚尉、揚州教授。召為太學錄,以所躬行者淑諸人,蚤暮延見學者,孜孜誨誘,長貳同僚忌其立異。會充殿試考官,唱名日序立庭下,帝偉其儀觀,遣內侍問姓名,眾滋忌之。或勸其姑營職,道未可行也,煥曰:「道與職有二乎?」適私試發策,引《孟子》:「立乎人之本朝而道不行,恥也。」言路以為訕己,請黜之,在職才八旬,調高郵軍教授而去。



 後充乾辦浙東安撫司公事。高宗山陵,百司次舍供帳酒食之需,供給不暇,
 煥亟言於安撫使鄭汝諧曰:「國有大戚,而臣子宴樂自如,安乎?」汝諧屬煥條奏。充修奉官,移書御史,請明示喪紀本意,使貴近哀戚之心重,則茇舍菲食自安,不煩彈劾而須索絕矣。於是治並緣為奸者,追償率斂者,支費頓減。



 歲旱,常平使分擇官屬振恤,得上虞、餘姚二縣,無復流殍。改知婺源,三省類薦書以聞,遂通判舒州。閑居雖病,猶不廢讀書,拳拳然以母老為念、善類凋零為憂。卒,丞相周必大聞之曰:「追思立朝不能推賢揚善,予愧
 叔晦,益者三友,叔晦不予愧也。」



 煥人品高明,而其中未安,不茍自恕,常曰晝觀諸妻子,夜卜諸夢寐,兩者無愧,始可以言學。追贈直華文閣,特謚端憲。



 煥之友舒璘字元質,一字元賓,奉化人。補入太學。張栻官中都,璘往從之,有所開警。又從陸九淵游,曰:「吾惟朝於斯,夕於斯,刻苦磨厲,改過遷善,日有新功,亦可以弗畔矣乎。」朱熹、呂祖謙講學於婺,璘徒步往謁之,以書告其家曰:「敝床疏席,總是佳趣;櫛風沐雨,反為美境。」



 舉乾道八年進
 士,兩授郡教授,不赴。繼為江西轉運司干辦公事。或忌璘所學,望風心議,及與璘處,了無疑間。為微州教授,微習頓異。《詩》、《禮》久不預貢士,學幾無傳,璘作《詩禮講解》,家傳人習,自是其學浸盛。丞相留正稱璘為當今第一教官,司業汪逵首欲薦璘,或謂璘舉員已足,逵曰:「吾職當舉教官,舍斯人將誰先?」卒剡薦之。知平陽縣,郡政頗苛,及璘以民病告,辭嚴義正,守為改容。秩滿,通判宜州,卒。



 璘樂於教人,嘗曰:「師道尊嚴,璘不如叔晦,若啟迪後進,
 則璘不敢多遜。」袁燮謂璘篤實不欺,無豪發矯偽。楊簡謂璘孝友忠實,道心融明。樓鑰謂璘之於人,如熙然之陽春。淳祐中,特謚文靖。



 曹彥約字簡甫,都昌人。淳熙八年進士。嘗從朱熹講學,歷建平尉、桂陽司錄、辰溪令,知樂平縣,主管江西安撫司機宜文字。知澧州,未上,薛叔似宣撫京湖,闢主管機宜文字。漢陽闕守,檄攝軍事。時金人大入,郡兵素寡弱,彥約搜訪土豪,得許禼俾總民兵,趙觀俾防水道,黨仲
 升將宣撫司軍屯郡城。金重兵圍安陸,游騎闖漢川,彥約授觀方略,結漁戶拒守南河,觀逆擊,斬其先鋒,且遣死士焚其戰艦,晝夜殊死戰,北渡追擊,金人大敗去。又遣仲升劫金人砦,殺千餘人,仲升中流矢死。奏觀補成忠郎、漢川簿尉,贈仲升修武郎,官其後二人。彥約以守御功進秩二等,就知漢陽。



 嘉定元年,詔求言,彥約上封事,謂「敵豈不以歲幣為利,惟其所向輒應,所求輒得,以我為易與而縱其欲。莫若遲留小使,督責邊備,假以歲
 月,當知真偽。設復大舉。則民固已怨矣,欲進而我已戒嚴,欲退而彼有叛兵,決勝可期矣。」尋提舉湖北常平,權知鄂州兼湖廣總領,改提點刑獄,遷湖南轉運判官。



 時盜羅世傳、李元礪、李新等相繼竊發,桂陽、茶陵、安仁三縣皆破,環地千里,莽為盜區。彥約至攸督運,人心始定。遷直秘閣、知潭州、湖南安撫。時江西言欲招安李元礪,朝命下湖南議招討之宜,彥約言:「今不行討捕,曲徇招安,失朝廷威重。若無礪設疑詞以款重兵,則兵不可撤
 戍,民不得安業。」元礪果不可降,彥約乃督諸將逼賊巢而屯,擊破李新於酃洣,新中創死,眾推李如松為首,如松降,遂復桂陽。世傳素與元礪有隙,至是密請圖元礪以自效,彥約錄賞格報之,且告於朝,又予萬緡錢犒其師。世傳遂禽元礪。彥約還長沙,未幾,復出督戰,餘黨悉平。



 世傳既自以為功,遲留以邀重賂,彥約諭以不宜格外邀求。時池州副都統許俊駐兵吉之龍泉,厚賂以結世傳,超格許轉官資,世傳遂以元礪解江西。胡矩為右
 司,欲以世傳盡統諸峒而為之帥,悉徹江西、湖南戍兵,彥約固爭之,矩不悅,然世傳終桀驁不肯出峒。彥約密遣羅九遷為間,誘胡友睦,許以重賞,友睦遂殺世傳。江西來爭功,不與校。擢侍右郎官,以右正言鄭昭先言,寢其命。



 久之,以為利路轉運判官兼知利州。關外乏食,彥約悉發本司所儲減價遣糶,勸分免役,通商蠲稅,民賴以濟。時沔州都統制王大才驕橫,制置使董居誼既不得其柄,反曲意奉之。彥約以蜀之邊面諸司並列,兵權
 不一,微有小警,紛然奏議,理財者歸怨於兵弱,握兵者歸咎於財寡,乃作《病夫議》,獻之廟堂,曰:



 古之臨邊,求一賢者而盡付之兵權,兵權正則事體重,兵權專則號令一。今廟堂之上,患士大夫不奉行詔令,惡士大夫不恪守忠實。故雖信而用之,又以人參之;雖以事權付之,又從中御以系維之。致使知事者不敢任事,畏事者常至失事,卒有緩急,各持己見,兵權財計,互相歸咎。



 昔秦、隴之俗,以知兵善戰聞天下。自吳氏世襲以來,握兵者志
 在於怙勢,不在於尊上;用兵者志在於誅貨,不在於息民。本原一壞,百病間出,至有世將已叛而宣威不覺,四郡已割而諸將不知。更化之後,逆黨既誅,而土俗人心其實未改。任軍官而領州事者,易成藩鎮之權;起行伍而立微效者,漸無階級之分。由皂郊以至宕昌,即隴西天水之地,其忠義民兵利在戰鬥,緩急之際固易鼓率,若其恃勇貪利,犯上作亂,則又不止於大軍而已。茍不正其本原,磨之以歲月,漸之以禮義,未見其可也。



 今日
 之領帥權者,必當近邊境,必當擁親兵;有兵權者,必當領經費,必當寬用度。至於忠義之兵,又須有德者以為統率,擇知書者以為教導,如古人所謂教民而後用之也。今議不出此,乃欲幸勝以為功,茍安以求免,誤天下者必此人也。



 時朝論未以為然。



 差知寧國府,又改知隆興府、江西安撫。居亡何,蜀邊被兵,內有張福、莫簡之變,彥約之言無一不驗。遷大理少卿,又權戶部侍郎,以寶謨閣待制知成都。彥約乞赴闕奏事,不允,又申省乞入
 對,不報。改知福州,又改知潭州,彥約力辭,提舉明道觀,尋以煥章閣待制提舉崇福宮。



 理宗即位,擢兵部侍郎兼國史院同修撰。寶慶元年入對,勸帝講學,防近習。次言:「當以慶歷、元祐聽言為法,以紹聖、崇、觀諱言為戒。比年以來,有以賣直好名之說見於奏對者,願陛下倚忠直如蓍龜,去邪佞若蟊賊,其有沮撓讜言者,必加斥逐。」



 會下詔求言,彥約上封事曰:「陛下謹定省以事長樂,開王社以篤天倫,孝友之行,宜足以取信於天下。然兄弟
 至親,猶誤於狂妄小人之手,道路異說,猶襲於尺布不縫之謠。臣以為守法者,人臣之職也,施恩者,人主之柄也。漢淮南王欲危社稷,張蒼、馮敬等請論如法,文帝既赦其罪廢徙,王不幸而死,封其二子於故地。此往事之明驗,本朝太宗皇帝之所已行也。今若徇文帝緣情之義,法太宗繼絕之意,明示好惡,無隙可指,雖不止謗而謗息矣。」又言:「陛下求言之詔,惟恐不逮,然外議致疑,以為明言文武,似或止於搢紳,泛言小大,恐不及於韋布,
 引而伸之,特在一命令之間耳。」又薦隆州布衣李心傳素精史學,乞官以初品,置之史館,從之。



 尋兼侍讀,俄遷禮部侍郎。加寶謨閣直學士,提舉祐神觀兼侍讀。授兵部尚書,力辭不拜。改寶章閣學士、知常德府,陛辭,言下情未通,橫斂未革。帝曰:「其病安在?」對曰:「臺諫專言人主,不及時政,下情安得通?包苴公行於都城,則州郡橫斂,無可疑者。」提舉崇福宮,卒,以華文閣學士轉通議大夫致仕,贈宣奉大夫。嘉熙初,賜謚文簡。



 範應鈴,字旂叟,豐城人。方娠,大父夢雙日照庭,應鈴生。稍長,厲志於學,丞相周必大見其文,嘉賞之。開禧元年,舉進士,調永新尉。縣當龍泉、茶陵溪峒之沖,寇甫平,喜亂者詐為驚擾,應鈴廉得主名,捽而治之。縣十三鄉,寇擾者不時,安撫使移司兼郡,初奏弛八鄉民租二年,詔下如章。既而復催以檢核之數,應鈴力爭,不從。即詣郡自言,反覆數四,帥聲色俱厲,慶鈴從容曰:「某非徒為八鄉貧民,乃深為州家耳!民貧迫之急,將以不肖之心應
 之,租不可得而禍未易弭也。」帥色動,令免下戶。既出令,復徵之,應鈴嘆曰:「是使我重失信於民也。」又力爭之,訖得請,民大感悅。有大姓與轉運使有連,家僮恣橫厲民,應鈴笞而系之獄。郡吏庭辱令,應鈴執吏囚之,以狀聞。



 調衡州錄事,總領聞應鈴名,闢為屬。改知崇仁縣,始至,明約束,信期會,正紀綱,曉諭吏民,使知所趨避。然後罷鄉吏之供需,校版籍之欺敝,不數月省簿成,即以其簿及苗稅則例上之總領所,自此賦役均矣。夙興,冠裳聽
 訟,發擿如神,故事無不依期結正,雖負者亦無不心服。真德秀扁其堂曰「對越」。將代,整治如始至。歲杪,與百姓休息,閣債負,蠲租稅,釋囚系,恤生瘞死,崇孝勸睦,仁民厚俗之事,悉舉以行,形之榜揭,見者嗟嘆。調提轄文思院,乾辦諸軍審計,添差通判撫州,以言者罷,與祠。丁內艱,服除,通判蘄州。



 時江右峒寇為亂,吉州八邑,七被殘毀,差知吉州,應鈴慨然曰:「此豈臣子辭難時耶?」即奉親以行。下車,首以練兵、足食為先務,然後去冗吏,核軍籍,
 汰老弱,以次罷行。應鈴洞究財計本末,每鄙榷酤興利,蘄五邑悉改為戶。吉,舟車之會,且屯大軍,六萬戶,人勸之榷,應鈴曰:「理財正辭,吾縱不能禁百姓群飲,其可誘之利其贏耶?」永新禾山群盜嘯聚,數日間應者以千數。應鈴察過客趙希邵有才略,檄之攝邑,調郡兵,結隅保,分道搗其巢穴,禽之,誅其為首者七人,一鄉以定。贛叛卒朱先賊殺主帥,應鈴曰:「此非小變也。」密遣諜以厚賞捕之。部使者劾其輕發,鐫一官。閑居六年,養親讀書,泊
 如也。起廣西提點刑獄,力辭,逾年乃拜命。既至,多所平反,丁錢蠹民,力奏免之。



 召為金部郎官,入見,首言:「今以朝行暮改之規模,欲變累年上玩下慢之積習;以悠悠內治之敝政,欲圖一旦赫赫外攘之大功。」又曰:「公論不出於君子,而參以逢君之小人;紀綱不正於朝廷,而牽於弄權之閹寺。」言皆讜直,識者韙之。遷尚左郎官,尋為浙東提點刑獄,力丐便養,改直秘閣、江西提舉常平,並詭挾三萬戶,風採凜然。



 丁外艱,服除,遷軍器監兼尚左
 郎官,召見,奏曰:「國事大且急者,儲貳為先。陛下不斷自宸衷,徒眩惑於左右近習之言,轉移於宮庭嬪御之見,失今不圖,奸臣乘夜半,片紙或從中出,忠義之士束手無策矣。」帝為之動容。屬鹽法屢變,商賈之贏,上奪於朝廷之自鬻,下奪於都郡之拘留;九江、豫章扼其襟喉,江右貧民終歲食淡,商與民俱困矣。應鈴力陳四害,願用祖宗入粟易鹽之法。



 授直寶謨閣,湖南轉運判官兼安撫司。峒獠蔣、何三族聚千餘人,執縣令,殺王官,帥憲招
 捕,逾年不至,應鈴曰:「招之適以長寇,亟捕之可也。」即調飛虎等軍會隅總討之,應鈴親臨誓師,號令明壯,士卒鼓勇以前,禽蔣時選父子及AT渠五人誅之,脅從者使之安業,未一月全師而歸。授直煥章閣,上疏謝事,不允;擢大理少卿,再請又不允。一旦籍府庫,核簿書,處決官事已,遂及家務,纖悉不遺。僚屬勸以清心省事,曰:「生死,數也,平生學力,正在今日。」帥別之傑問疾,應鈴整冠肅入,言論如平常,之傑退,悠然而逝。



 應鈴開明磊落,守正
 不阿,別白是非,見義必為,不以得失利害動其心。書饋不交上官,薦舉不徇權門,當官而行,無敢撓以非義。所至無留訟,無滯獄,繩吏不少貸,亦未嘗沒其貲,曰:「彼之貨以悖入,官又從而悖入之,可乎?」進修潔,案奸贓,振樹風聲,聞者興起。家居時,人有不平,不走官府,而走應鈴之門;為不善者,輒相戒曰:「無使範公聞之。」讀書明大義,尤喜《左氏春秋》,所著有《西堂雜著》十卷,斷訟語曰《對越集》四十九卷。徐鹿卿曰:「應鈴經術似兒寬,決獄似雋不
 疑,治民似龔遂,風採似範滂,理財似劉晏,而正大過之。」人以為名言。



 徐經孫,字中立,初名子柔。寶慶二年進士,授瀏陽主簿,潭守俾部牙契錢至州,有告者曰:「朝廷方下令頒行十七界會,令若此錢皆用會,小須,則幸而獲大利矣。」經孫曰:「此錢取諸保司,出諸公庫,吾納會而私取其錢,外欺其民,內欺其心,奚可哉!」詰旦,悉以所部錢上之,其人驚服有愧色。



 闢永興令,知臨武縣,通判潭州。帥陳韡雅相
 知,事必咨而後行。秩滿,由豐儲蒼提管進權轄,國子博士兼資善堂直講。為監察御史,劾京尹厲文翁言偽而辨,疏入,留中。宣諭至再,即日出關,上遣使追之,不及。進直寶章閣、福建提點刑獄,號稱平允。歲餘升安撫使,召為秘書監兼太子諭德。經孫為安撫時,韡家居,門人故吏有撓法者不得逞,相與搖撼。至是韡起家判本郡,懷私逞忿,無復交承之禮,即日劾奏通判,語侵經孫,謂席卷府庫而去,於是罷通判,削其秩。經孫造朝,具白於政
 府。事上聞,帝大怒,諭宰執曰:「陳韡老繆至此,宜亟罷之。」於是經孫再詣政府,言:「某,韡門生也,前日之白,公事也,茍韡以是得罪,人謂我何?」請之不置,俾自乞閑,明通判無罪,識者韙之。



 遷宗正少卿、起居舍人、起居郎,入奏:「君人者當守理欲之界限。」遷刑部侍郎兼給事中,升太子左庶子、太子詹事,輔導東宮者三年,敷陳經義,隨事啟迪。太子入侍,必以其所講聞悉奏之,帝未嘗不稱善。景定三年春雷,詔求直言,經孫對曰:「三數年來,言論者以
 靖共為主,有懷者以嘩訐為戒,忠讜之氣,鬱不得行,上帝降監,假雷以鳴。」切中時病。



 公田法行,經孫條其利害,忤丞相賈似道,拜翰林學士、知制誥,未逾月,諷御史舒有開奏免,罷歸。授湖南安撫使、知潭州,不拜。授端明殿大學士,閑居十年,卒,贈金紫光祿大夫。經孫所薦陳茂濂為公田官,分司嘉興,聞經孫去國,曰:我不可以負徐公。」遂以親老謝歸,終身不起。



 論曰:嗚呼,寧宗之為君,韓侂胄之為相,豈用兵之時乎?
 故婁機力止之。小學之廢久矣,而機獨知致力於此。沈煥、舒璘學遠識明。曹彥約可與建立事功。範應鈴赫然政事如神明。徐經孫清慎有守,卒以爭公田迕賈似道去國,君子稱之。



\end{pinyinscope}