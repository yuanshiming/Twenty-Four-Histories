\article{列傳第四十二}

\begin{pinyinscope}

 王欽若
 林特附丁謂夏竦子安期



 王欽若,字定國,臨江軍新喻人。父仲華,侍祖鬱官鄂州。會。江水暴至,徙家黃鶴樓,漢陽人望見樓上若有光景,是夕,欽若生。欽若早孤,鬱愛之。太宗伐太原時,欽若才
 十八,作《平晉賦論》獻行在。鬱為濠州判官,將死,告家人曰:「吾歷官逾五十年,慎於用刑,活人多矣,後必有興者,其在吾孫乎!」



 欽若擢進士甲科,為亳州防禦推官,遷秘書省秘書郎,監廬州稅。改太常丞、判三司理欠憑由司。時毋賓古為度支判官,嘗言曰:「天下逋負,自五代迄今,理督未已,民病幾不能勝矣。僕將啟蠲之。」欽若一夕命吏勾校成數,翌日上之。真宗大驚曰:「先帝顧不知邪?」欽若徐曰:「先帝固知之,殆留與陛下收人心爾。」即日放逋
 負一千餘萬,釋系囚三千餘人。帝益器重欽若,召試學士院,拜右正言、知制誥,召為翰林學士。蜀寇王均始平,為西川安撫使。所至問系囚,自死罪以下第降之,凡列便宜,多所施行。還,授左諫議大夫、參知政事,以郊祀恩,加給事中。



 河陰民常德方訟臨津縣尉任懿賂欽若得中第,事下御史臺劾治。初,欽若咸平中嘗知貢舉,懿舉諸科,寓僧仁雅舍。仁雅識僧惠秦者與欽若厚,懿與惠秦約,以銀三百五十兩賂欽若,書其數於紙,令惠秦持
 去。會欽若已入院,屬欽若客納所書於欽若妻李氏,惠秦減所書銀百兩,欲自取之。李氏令奴祁睿書懿名於臂,並以所約銀告欽若。懿再入試第五場,睿復持湯飲至貢院,欽若密令奴索取銀,懿未即與而登科去。仁雅馳書河陰,始歸之。德方得其書,以告御史中丞趙昌言,昌言以聞。既捕祁睿等,亦請逮欽若屬吏。



 祁睿本亳小吏,雖從欽若久,而名猶隸亳州。欽若乃言:「向未有祁睿,惠秦亦不及門。」帝方顧欽若厚,命邢昺、閻承翰等於太
 常寺別鞫之。懿更云妻兄張駕識知舉官洪湛,嘗俱造湛門。始但以銀屬二僧,不知達主司為誰。昺等遂誣湛受懿銀,湛適使陜西還,而獄已具。時駕且死,睿又悉遁去,欽若因得固執祁睿休役後始傭於家,它奴使多新募,不識惠秦,故皆無證驗。湛坐削藉、流儋州,而欽若遂免。方湛代王旦入知貢舉,懿已試第三場,及官收湛贓,家無有也,乃以湛假梁顥白金器輸官,湛遂死貶所。人知其冤,而欽若恃勢,人莫敢言者。



 景德初,契丹入寇,帝
 將幸澶淵。欽若自請北行,以工部侍郎、參知政事判天雄軍、提舉河北轉運使,真宗親宴以遣之。素與寇準不協,及還,累表願解政事,罷為刑部侍郎、資政殿學士。尋判尚書都省,修《冊府元龜》,或褒贊所及,欽若自名表首以謝,即繆誤有所譴問,戒書吏但云楊億以下,其所為多此類也。歲中,改兵部,升大學士、知通進銀臺司兼門下封駁事。初,欽若罷,為置資政殿學士以寵之,準定其班在翰林學士下。欽若訴於帝,復加「大」字,班承旨上。以
 尚書左丞知樞密院事,修國史。



 大中祥符初,為封禪經度制置使兼判兗州,為天書儀衛副使。先是,真宗嘗夢神人言「賜天書於泰山」,即密諭欽若。欽若因言,六月甲午,木工董祚於醴泉亭北見黃素曳草上,有字不能識,皇城吏王居正見其上有御名,以告。欽若既得之,具威儀奉導至社首,跪授中使,馳奉以進。真宗至含芳園奉迎,出所上《天書再降祥瑞圖》示百僚。欽若又言至嶽下兩夢神人,願增建廟庭。及至威雄將軍廟,其神像如夢
 中所見,因請構亭廟中。封禪禮成,遷禮部尚書,命作《社首頌》,遷戶部尚書。從祀汾陰,復為天書儀衛副使,遷吏部尚書。明年,為樞密使、檢校太傅、同中書門下平章事。初,學士晁迥草制,誤削去官,有詔仍帶吏部尚書。聖祖降,加檢校太尉。欽若居第在太廟後壖,自言出入訶導不自安,因易賜官第於安定坊。七年,為同天書刻玉使。



 馬知節同在樞密,素惡欽若,議論不相下。會瀘州都巡檢王懷信等上平蠻功,欽若久不決,知節因面詆其短,
 爭於帝前。及趣論賞,欽若遂擅除懷信等官,坐是,罷樞密使,奉朝請。改刻玉副使、知通進銀臺司。復拜樞密使、同平章事。上玉皇尊號,遷尚書右僕射、判禮儀院,為會靈觀使。有龜蛇見拱聖營,因其地建詳源觀,命欽若總領之。尋拜左僕射兼中書侍郎、同平章事。明年,為景靈使,閱《道藏》,得趙氏神仙事跡四十人,繪於廊廡。又明年,商州捕得道士譙文易,畜禁書,能以術使六丁六甲神,自言嘗出入欽若家,得欽若所遺詩。帝以問欽若,謝不
 省,遂以太子太保出判杭州。



 仁宗為皇太子,自以東宮師保請歸朝,復為資政大學士。詔日赴資善堂侍講皇太子。會輔臣兼領三少,欽若以品高求換秩,拜司空,尋除山南道節度使、同平章事、判河南府。與宰相丁謂不相悅,以疾請就醫京師,不報。令其子從益移文河南府,輿疾而歸。謂言欽若擅去官守,命御史中丞薛映就第按問。欽若惶恐伏罪,降司農卿、分司南京,奪從益一官。



 仁宗即位,改秘書監,起為太常卿、知濠州,以刑部尚書
 知江寧府。仁宗嘗為飛白書,適欽若有奏至,因大書「王欽若」字。是時,馮拯病,太后有再相欽若意,即取字緘置湯藥合,遣中人繼以賜,且口宣召之。至國門而人未有知者。既朝,復拜司空、門下侍郎、同平章事、玉清昭應宮使、昭文館大學士,監修國史。



 帝初臨政,欽若謂平時百官敘進,皆有常法,為《遷敘圖》以獻。《真宗實錄》成,進司徒,以郊祀恩,封冀國公。知邵武軍吳植病,求外徙,因殿中丞餘諤以黃金遺欽若,未至,而植復遣牙吏至欽若第問
 之。欽若執以送官,植、諤皆坐貶。初,欽若安撫西川,植為新繁縣尉,嘗薦舉之。至是,亦當以失舉坐罪,詔勿問。兼譯經使,始赴傳法院,感疾亟歸。帝臨問,賜白金五千兩。既卒,贈太師、中書令,謚文穆,錄親屬及所親信二十餘人。國朝以來宰相恤恩,未有欽若比者。



 欽若嘗言:「少時過圃田,夜起視天中,赤文成『紫微』字。後使蜀,至褒城道中,遇異人,告以他日位至宰相。既去,視其刺字,則唐相裴度也。」及貴,遂好神仙之事,常用道家科儀建壇場以
 禮神,朱書「紫微」二字陳於壇上。表修裴度祠於圃田,官其裔孫,自撰文以紀其事。



 真宗封泰山、祀汾陰,而天下爭言符瑞,皆欽若與丁謂倡之。嘗建議躬謁元德皇太后別廟,為莊穆皇后行期服。議者以為天子當絕傍期,欽若所言不合禮。又請置先蠶並壽星祠,升天皇北極帝坐於郊壇第一龕,增執法、孫星位,別制王公以下車輅、鼓吹,以備拜官、婚葬。所著書有《鹵簿記》、《彤管懿範》、《天書儀制》、《聖祖事跡》、《翊聖真君傳》、《五岳廣聞記》、《列宿萬靈
 朝真圖》、《羅天大醮儀》。欽若自以深達道教,多所建明,領校道書,凡增六百餘卷。



 欽若狀貌短小,項有附疣,時人目為「癭相」。然智數過人,每朝廷有所興造,委曲遷就,以中帝意。又性傾巧,敢為矯誕。馬知節嘗斥其奸狀,帝亦不之罪。其後仁宗嘗謂輔臣曰:「欽若久在政府,觀其所為,真奸邪也。」王曾對曰:「欽若與丁謂、林特、陳彭年、劉承珪,時謂之『五鬼』。奸邪險偽,誠如聖諭。」



 欽若子從益,終贊善大夫,追賜進士及第。後無子,以叔之子為後。



 林特字士奇。祖揆,仕閩為南劍州順昌令,因家順昌。特少穎悟,十歲,謁江南李景,獻所為文,景奇之,命作賦,有頃而成,授蘭臺校書郎。江南平,偽官皆入見,特袖文以進。太宗以為長葛尉,改遂州錄事參軍。代還,命中書引對,授大理寺丞、通判隴州,有治狀。田重進鎮永興,太宗以重進武人,選特與楊覃並為通判,人賜白金二百兩,給實奉。會出兵五路討李繼遷,督所部轉芻粟,先期以辦。呂蒙正闢通判西京留守事。蒙正入相,薦之,入判三
 司戶部勾院。



 梁鼎制置陜西青白鹽,前後上議異同,真宗選特與知永興軍張詠同商利害,所奏合旨。累遷尚書祠部員外郎,為戶部副使,詔赴內朝。三司副使預內朝,自特始。徙鹽鐵副使。



 真宗北征,命同知留司三司公事,遷司封員外郎。車駕謁陵,為行在三司副使,詔與劉承珪、李溥比較江淮茶法。因裁定新制,歲增課百餘萬,特遷祠部郎中。封泰山,祀汾陰,皆為行在三司副使。以右諫議大夫權三司使、修玉清昭應宮副使。將祀太清
 宮,遣特儲供具,為行在三司使。禮成,進給事中,為修景靈宮副使兼修兗州景靈宮、太極觀。昭應宮成,遷尚書工部侍郎,真拜三司使。樞密使寇準言特奸邪,又數與爭事,帝為出準,特在職如故。後罷三司,以戶部侍郎同玉清昭應宮副使。兗州宮觀成,遷吏部侍郎。天禧元年,為修上《聖祖寶冊》副使,轉尚書右丞。



 時天下完富,丁謂以符瑞、土木迎帝意,而以特有心計,使乾財利佐之。然特亦天性邪險,善附會,故謂始終善特,當時與陳彭年
 等號「五鬼」,語在《王欽若傳》。



 仁宗在東宮,以工部尚書兼太子賓客,改詹事。丁謂欲引為樞密副使,而李迪執不可。仁宗即位,進刑部尚書、翰林侍讀學士。謂貶,特亦落職知許州。還朝,以戶部尚書知通進銀臺司、判尚書都省、勾當三班院。特體素羸,然未嘗一日謁告,及得疾,才五日而卒。贈尚書左僕射。太后遣中使祀奠。



 特精敏,喜吏職,據案終日不倦。真宗數訪以朝廷大事,特因有所中傷,人以此憚焉。奉詔撰《會計錄》三十卷。又為《東封西
 祀朝謁太清宮慶賜總例》三十六卷。



 子濰、洙。濰亦有吏能,歷官至三司鹽鐵副使,以秘書監致仕,卒。洙,官至司農卿、知壽州,臨事苛急,鼓角將夜入州廨,拔堂檻鐵鉤擊殺之。



 丁謂,字謂之,後更字公言,蘇州長洲人。少與孫何友善,同袖文謁王禹偁,禹偁大驚重之,以為自唐韓愈、柳宗元後,二百年始有此作。世謂之「孫丁」。淳化三年,登進士甲科,為大理評事、通判饒州。逾年,直史館,以太子中允
 為福建路採訪。還,上茶鹽利害,遂為轉運使,除三司戶部判官。峽路蠻擾邊,命往體量。還奏稱旨,領峽路轉運使,累遷尚書工部員外郎,會分川峽為四路,改夔州路。



 初,王均叛,朝廷調施、黔、高、溪州蠻子弟以捍賊,既而反為寇。謂至,召其種酋開諭之,且言有詔赦不殺。酋感泣,願世奉貢。乃作誓刻石柱,立境上。蠻地饒粟而常乏鹽,謂聽以粟易鹽,蠻人大悅。先時,屯兵施州而饋以夔、萬州粟。至是,民無轉餉之勞,施之諸砦,積聚皆可給。特遷
 刑部員外郎,賜白金三百兩。時溪蠻別種有入寇者,謂遣高、溪酋帥其徒討擊,出兵援之,擒生蠻六百六十,得所掠漢口四百餘人。復上言:黔南蠻族多善馬,請致館,犒給緡帛,歲收市之。其後徙置夔州城砦,皆謂所經畫也。居五年,不得代,乃詔舉自代者,於是入權三司鹽鐵副使。未幾,擢知制誥,判吏部流內銓。



 景德四年,契丹犯河北,真宗幸澶淵,以謂知鄆州兼齊、濮等州安撫使,提舉轉運兵馬巡檢事。契丹深入,民驚擾,爭趣楊劉渡,而
 舟人邀利,不時濟。謂取死罪紿為舟人,斬河上,舟人懼,民得悉渡。遂立部分,使並河執旗幟,擊刁斗,呼聲聞百餘里,契丹遂引去。明年,召為右諫議大夫、權三司使。上《會計錄》,以景德四年民賦戶口之籍,較咸平六年之數,具上史館,請自今以咸平籍為額,歲較其數以聞,詔獎之。尋加樞密直學士。



 大中祥符初,議封禪,未決,帝問以經費,謂對「大計有餘」,議乃決。因詔謂為計度泰山路糧草使。初,議即宮城乾地營玉清昭應宮,左右有諫者。帝
 召問,謂對曰:「陛下有天下之富,建一宮奉上帝,且所以祈皇嗣也。群臣有沮陛下者,願以此論之。」王旦密疏諫,帝如謂所對告之,旦不復敢言。乃以謂為修玉清昭應宮使,復為天書扶侍使,遷給事中,真拜三司使。祀汾陰,為行在三司使。建會靈觀,謂復總領之。遷尚書禮部侍郎,進戶部,參知政事。建安軍鑄玉皇像,為迎奉使。朝謁太清宮,為奉祀經度制置使、判亳州。帝賜宴賦詩以寵其行,命權管勾駕前兵馬事。謂獻白鹿並靈芝九萬五
 千本。還,判禮儀院,又為修景靈宮使,摹寫天書刻玉笈,玉清昭應宮副使。大內火,為修葺使。歷工、刑、兵三部尚書,再為天書儀衛副使,拜平江軍節度使、知升州。



 天禧初,徙保信軍節度使。三年,以吏部尚書復參知政事。是歲,祀南郊,輔臣俱進官。故事,嘗為宰相而除樞密使,始得遷僕射,乃以謂檢校太尉兼本官為樞密使。時寇準為相,尤惡謂,謂媒薛其過,遂罷準相。既而拜謂同中書門下平章事、昭文館大學士、監修國史、玉清昭應宮使。
 周懷政事敗,議再貶準,帝意欲謫準江、淮間,謂退,除道州司馬。同列不敢言,獨王曾以帝語質之,謂顧曰:「居停主人勿復言。」蓋指曾以第舍假準也。



 其後詔皇太子聽政,皇后裁制於內,以二府兼東宮官,遂加謂門下侍郎兼太子少傅,而李迪先兼少傅,乃加中書侍郎兼尚書左丞。故事,左、右丞非兩省侍郎所兼,而謂意特以抑迪也。謂所善林特,自賓客改詹事,謂欲引為樞密副使兼賓客,迪執不可,因大詬之。既入對,斥謂奸邪不法事,願
 與俱付御史雜治,語在《迪傳》。帝因格前制不下,乃罷謂為戶部尚書,迪為戶部侍郎;尋以謂知河南府,迪知鄆州。明日,入謝,帝詰所爭狀,謂對曰:「非臣敢爭,乃迪忿詈臣爾,願復留。」遂賜坐。左右欲設墩,謂顧曰:「有旨復平章事。」乃更以杌進,即入中書視事如故。仍進尚書左僕射、門下侍郎、平章事兼太子少師。天章閣成,拜司空。乾興元年,封晉國公。



 仁宗即位,進司徒兼侍中,為山陵使。寇準、李迪再貶,謂取制草改曰:「當醜徒干紀之際,屬先王
 違豫之初,罹此震驚,遂至沉劇。」凡與準善者,盡逐之。是時二府定議,太后與帝五日一御便殿聽政。既得旨,而謂潛結內侍雷允恭,令密請太后降手書,軍國事進入印畫。學士草制辭,允恭先持示謂,閱訖乃進。蓋謂欲獨任允恭傳達中旨,而不欲同列與聞機政也。允恭倚謂勢,益橫無所憚。



 允恭方為山陵都監,與判司天監邢中和擅易皇堂地。夏守恩領工徒數萬穿地,土石相半,眾議日喧,懼不能成功,中作而罷,奏請待命。謂庇允恭,依
 違不決。內侍毛昌達自陵下還,以其事奏,詔問謂,謂始請遣使按視。既而咸謂復用舊地,乃詔馮拯、曹利用等就謂第議,遣王曾覆視,遂誅允恭。



 後數日,太后與帝坐承明殿,召拯、利用等諭曰:「丁謂為宰輔,乃與宦官交通。」因出謂嘗托允恭令後苑匠所造金酒器示之,又出允恭嘗干謂求管勾皇城司及三司衙司狀,因曰:「謂前附允恭奏事,皆言已與卿等議定,故皆可其奏;且營奉先帝陵寢,而擅有遷易,幾誤大事。」拯等奏曰:「自先帝登遐,
 政事皆謂與允恭同議,稱得旨禁中。臣等莫辨虛實,賴聖神察其奸,此宗社之福也。」乃降謂太子少保、分司西京。故事,黜宰相皆降制,時欲亟行,止令拯等即殿廬召舍人草詞,仍榜朝堂,布諭天下。追其子珙、珝、□、碔一官,落珙館職。



 先是,女道士劉德妙者,嘗以巫師出入謂家。謂敗,逮系德妙,內侍鞫之。德妙通款,謂嘗教言:「若所為不過巫事,不若托言老君言禍福,足以動人。」於是即謂家設神像,夜醮於園中,允恭數至請禱。及帝崩,引入禁
 中。又因穿地得龜蛇,令德妙持入內,紿言出其家山洞中。仍復教云:「上即問若,所事何知為老君,第云『相公非凡人,當知之』。」謂又作頌,題曰「混元皇帝賜德妙」,語涉妖誕。遂貶崖州司戶參軍。諸子並勒停。□又坐與德妙奸,除名,配隸復州。籍其家,得四方賂遺,不可勝紀。其弟誦、說、諫悉降黜。坐謂罷者,自參知政事任中正而下十數人。在崖州逾三年,徙雷州,又五年,徙道州。明道中,授秘書監致仕,居光州,卒。詔賜錢十萬、絹百匹。



 謂機敏有智
 謀,憸狡過人,文字累數千百言,一覽輒誦。在三司,案牘繁委,吏久難解者,一言判之,眾皆釋然。善談笑,尤喜為詩,至於圖畫、博奕、音律,無不洞曉。每休沐會賓客,盡陳之,聽人人自便,而謂從容應接於其間,莫能出其意者。



 真宗朝營造宮觀,奏祥異之事,多謂與王欽若發之。初,議營昭應宮,料功須二十五年,謂令以夜繼晝,每繪一壁給二燭,七年乃成。真宗崩,議草遺制,軍國事兼取皇太后處分,謂乃增以「權」字。及太后稱制,又議月進錢充
 宮掖之用,由是太后深惡之,因雷允恭遂並錄謂前後欺罔事竄之。



 在貶所,專事浮屠因果之說,其所著詩並文亦數萬言。家寓洛陽,嘗為書自克責,敘國厚恩,戒家人毋輒怨望,遣人致於洛守劉燁,祈付其家。戒使者伺燁會眾僚時達之,燁得書不敢私,即以聞。帝見感惻,遂徙雷州,亦出於揣摩也。謂初通判饒州,遇異人曰:「君貌類李贊皇。」既而曰:「贊皇不及也。」



 夏竦,字子喬,江州德安人。父承皓,太平興國初,上《平晉
 策》,補右侍禁,隸大名府。契丹內寇,承皓由間道發兵,夜與契丹遇,力戰死之,贈崇儀使,錄竦為潤州丹陽縣主簿。



 竦資性明敏,好學,自經史、百家、陰陽、律歷,外至佛老之書,無不通曉。為文章,典雅藻麗。舉賢良方正,擢光祿寺丞、通判臺州。召直集賢院,為國史編修官、判三司都磨勘司,累遷右正言。帝幸亳州,為東京留守推官。仁宗初封慶國公,王旦數言竦材,命教書資善堂。未幾,同修起居注,為玉清昭應宮判官兼領景靈宮、會真觀事,遷尚
 書禮部員外郎、知制誥。史成,遷戶部。景靈宮成,遷禮部郎中。



 竦娶楊氏,楊亦工筆札,有鉤距。及竦顯,多內寵,浸與楊不諧,楊悍妒,即與弟媦疏竦陰事,竊出訟之,又竦母與楊母相詬詈,偕訴開封府,府以事聞,下御史臺置劾,左遷職方員外郎、知黃州。後二年,徙鄧州,又徙襄州。屬歲饑,大發公廩,不足,竦又勸率州大姓,使出粟,得二萬斛,用全活者四十餘萬人。仁宗即位,遷戶部郎中,徙壽、安、洪三州。洪俗尚鬼,多巫覡惑民,竦索部中得千餘
 家,敕還農業,毀其淫祠以聞。詔江、浙以南悉禁絕之。



 竦材術過人,急於進取,喜交結,任數術,傾側反復,世以為奸邪。當太后臨朝,嘗上疏乞與修《真宗實錄》,不報。既而丁母憂,潛至京師,依中人張懷德為內助,宰相王欽若雅善竦,因左右之,遂起復知制誥,為景靈判官、判集賢院,以左司郎中為翰林學士、勾當三班院兼侍讀學士、龍圖閣學士,又兼譯經潤文官。遷諫議大夫,為樞密副使、修國史,遷給事中。初,武臣賞罰無法,吏得高下為奸,
 竦為集前比,著為定例,事皆按比而行。改參知政事、祥源觀使。增設賢良等六科,復百官轉對,置理檢使,皆竦所發。與宰相呂夷簡不相能,復為樞密副使,遷刑部侍郎。史成,進兵部,尋進尚書左丞。



 太后崩,罷為禮部尚書、知襄州,改穎州。京東薦饑,徙青州兼安撫使。逾年,罷安撫,遷刑部尚書、徙應天府。寶元初,以戶部尚書入為三司使。趙元昊反,拜奉寧軍節度使、知永興軍,聽便宜行事。徙忠武軍節度使、知涇州。還,判永興軍兼陜西經略
 安撫招討,進宣徽南院使。與陳執中論兵事不合,詔徙屯鄜州。



 初,竦在涇州,朝廷遣龐籍就計事。竦上奏曰:



 頃者繼遷逃背,屢寇朔方。至道初,洛苑使白守榮等率重兵護糧四十萬,遇寇浦洛河,糧卒並沒,守榮僅以身免。呂端始欲發兵,由麟府、鄜延、環慶三路趣平夏,襲其巢穴,太宗難之。後命李繼隆、丁罕、範廷召、王超、張守恩五路入討。繼隆與罕合兵,行旬日,不見賊;守恩見賊不擊;超及廷召至烏白池,以諸將失期,士卒困敝,相繼引還。
 時繼遷當繼捧入朝之後,曹光實掩襲之餘,遁逃窮蹙,而猶累歲不能剿滅。先皇帝鑒追討之敝,戒疆吏謹烽候、嚴卒乘,來即驅逐之,去無追捕也。



 然拓跋之境,自靈武陷沒之後,銀、綏割棄已來,假朝廷威靈,其所役屬者不過河外小羌爾。況德明、元昊相繼猖獗,以繼遷窮蹙,比元昊富實,勢可知也。以先朝累勝之士,較當今關東之兵,勇怯可知也。以興國習戰之帥,方沿邊未試之將,工拙可知也。繼遷竄伏平夏,元昊窟穴河外,地勢可知
 也。若分兵深入,糗糧不支,師行賊境,利於速戰。儻進則賊避其鋒,退則敵躡其後,老師費糧,深可虞也。若窮其巢穴,須涉大河,長舟巨艦,非倉卒可具也。若浮囊挽梗,聯絡而進,我師半渡,賊乘勢掩擊,未知何謀可以捍禦?臣以為不較主客之利,不計攻守之便,而議追討者,非良策也。



 因條上十事。時邊臣多議征討,朝廷鄉之,而竦言出師非便。既而詔以涇原、鄜延兩路兵進討,會元昊稍求納款,範仲淹請留鄜延兵,由是涇原兵亦不行。中
 國之師,卒不出塞。



 竦上十事:一、教習強弩以為奇兵;二、羈縻屬羌以為藩籬;三、詔唃廝囉父子並力破賊;四、度地形險易遠近、砦柵多少、軍士勇怯,而增減屯兵;五、詔諸路互相應援;六、募土人為兵,州各一二千人,以代東兵;七、增置弓手、壯丁、獵戶以備城守;八、並邊小砦,毋積芻糧,賊攻急,則棄小砦入保大砦,以完兵力;九、關中民坐累若過誤者,許人入粟贖罪,銅一斤為粟五斗,以贍邊計;十、損並邊冗兵、冗官及減騎軍,以舒饋運。當時頗
 採用之。



 其募土人為兵,令下而楊偕奏言:「西兵比繼遷時十增七八,縣官困於供億,今州復益一二千人,則歲費不貲。若訓習士卒,使之精銳,選任將帥,求之方略,自然以寡擊眾,以一當百矣。竦云「土兵訓練可代東兵」,此虛言也。自德明納款以來,東兵猶不可代,況今日乎?」朝廷下竦議,竦奏:「陜西防秋之敝,無甚東兵,不慣登陟,不耐寒暑,驕懦相習,廩給至厚。土兵便習,各護鄉土,山川道路,彼皆素知,歲省芻糧鉅萬。且收聚小民,免饑餓為
 盜,代兵東歸,以衛京師,萬世利也。偕欲以寡擊眾,殆虛言也。」



 偕復奏云:



 自古將帥深入殊庭,霍去病止將輕騎八百,直棄大將軍數百里赴利,斬捕過當;又將萬騎逾烏盭,討ST僕,涉狐奴,歷五王國,過焉支山千有餘里,合兵鏖皋蘭下,殺樓蘭王、虜侯王,執昆邪王子,收休屠祭天金人。趙充國亦以萬騎破先零。李靖以驍騎三千破突厥,又以精騎一萬至陰山,斬首千餘級,俘男女十餘萬,擒頡利以獻。自漢以來,用少擊眾,不可勝數。竦在涇
 原守城壘,據險阻,來則御之,去則釋之,不聞出師也。竦懼戰或敗衄,托以兵少為辭爾。



 竦言土兵各護鄉土,自古兵有九地,士卒近家,謂之散地,言其易離散也。第以近事言之,閣門祗候王文恩出師敗北,而土兵皆竄走,惟東兵僅二百人,殺敵兵甚眾。以此知兵之強弱,88不系東西,在將有謀與無謀爾。今邊郡參用東兵、土兵,若盡罷東兵,亦非計也。古人有言:「非隴西之民有勇怯,乃將吏之制巧拙異也。」今防邊東兵,人月受米七斗五升,土兵
 二石五斗,而竦乃言東兵廩給至厚,又不知之甚也。竦又言募土兵訓練以代東兵,且土兵數萬,須募足訓練,雖三二歲未得成效,兵精猶恐奔北,豈有驟加訓練而能取勝哉?



 竦議遂屈。



 竦雅意在朝廷,及任以西事,頗依違顧避,又數請解兵柄。改判河中府,徙蔡州。慶歷中,召為樞密使。諫官、御史交章論:「竦在陜西畏懦不肯盡力,每論邊事,但列眾人之言,至遣敕使臨督,始陳十策。嘗出巡邊,置侍婢中軍帳下,幾致軍變。元昊嘗募得竦首
 者與錢三千,為賊輕侮如此。今復用之,邊將體解矣。且竦挾詐任數,奸邪傾險,與呂夷簡不相能。夷簡畏其為人,不肯引為同列,既退,乃存之以釋宿憾。陛下孜孜政事,首用懷詐不忠之臣,何以求治?」會竦已至國門,言者論不已,請不令入見。諫官餘靖又言:「竦累表引疾,及聞召用,即兼驛而馳。若不早決,竦必堅求面對,敘恩感泣,復有左右為之地,則聖聽惑矣。」章累上,即日詔竦歸鎮,竦亦自請還節。徙知亳州,改授吏部尚書。歲中,加資政
 殿學士。



 竦之及國門也,帝封彈疏示之,既至亳州,上書萬言自辨。復拜宣徽南院使、河陽三城節度使、判並州。請復置宦者為走馬承受。明年,拜同中書門下平章事、判大名府。又明年,召入為宰相。制下,而諫官、御史復言:「大臣和則政事修,竦前在關中,與執中論議不合,不可使共事。」遂改樞密使,封英國公。



 請析河北為四路。親事官夜入禁中,欲為亂,領皇城司者皆坐逐,獨楊懷敏降官,領入內都知如故。言者以為竦結懷敏而曲庇之。會
 京師同日無雲而震者五,帝方坐便殿,趣召翰林學士張方平至,謂曰:「夏竦奸邪,以致天變如此,宜出之。」罷知河南府,未幾,赴本鎮,加兼侍中。饗明堂,徙武寧軍節度使,進鄭國公,錫SS與輔臣等。將相居外,遇大禮有賜,自竦始。尋以病歸,卒。贈太師、中書令。賜謚文正,劉敞言:「世謂竦奸邪,而謚為正,不可。」改謚文莊。



 竦以文學起家,有名一時,朝廷大典策累以屬之。多識古文,學奇字,至夜以指畫膚。文集一百卷。其為郡有治績,喜作條教,於閭
 里立保伍之法,至盜賊不敢發,然人苦煩擾。治軍尤嚴,敢誅殺,即疾病死喪,拊循甚至。嘗有龍騎卒戍邊郡,剽,州郡莫能止,或密以告竦。時竦在關中,俟其至,召詰之,誅斬殆盡,軍中大震。其威略多類此。然性貪,數商販部中。在並州,使其僕貿易,為所侵盜,至杖殺之。積家財累鉅萬,自奉尤侈,畜聲伎甚眾。所在陰間僚屬,使相猜阻,以鉤致其事,遇家人亦然。



 子安期,字清卿,以父任為將作監主簿,召試,賜進士出身。累遷太常博士,擢提點荊
 湖南道刑獄。除開封府推官,徙判官,判三司鹽鐵勾院,出為京西轉運使。盜起部中,剽劫州縣,而光化軍戍卒相繼叛,勢且相合,安期督將吏捕斬殆盡。徙河東轉運使,累遷尚書工部郎中,徙江、淮發運使,入為三司戶部副使。會元昊納款,西邊罷兵,命往陜西與諸路經略安撫司議損邊費,頗奏省吏員及汰邊兵之不任役者五萬人。擢天章閣待制,遂為陜西都轉運使。徙河北,進兵部郎中。



 時竦為樞密使,為請還所遷官,丐淮、浙一郡。復
 以為工部郎中、江淮發運使,徙知永興軍。進龍圖閣直學士、吏部郎中、知渭州。簡弓箭手,得驍勇萬人為步兵,騎又半之,教以戰陣法,由是土兵勝他路。又籍塞下閑田,募人耕種,歲得穀數萬斛,以備振發,名曰貸倉。



 遷右諫議大夫,進樞密直學士,徙延州。未至,丁父憂。服除,辭所進職,復為龍圖閣直學士兼侍讀,提舉集禧觀。以學士復知延州,州東北阻山,無城郭,虜騎嘗乘之。安期至,即大築城。時方暑,士卒有怨言,安期益令廣袤計數百
 步,令其下曰:「敢言者斬。」躬自督役,不逾月而就。元昊請畫疆界,朝廷欲遣使,以問安期。安期對曰:「此不足煩王人,衙校可辦也。」議遂決。暴得疾,卒,詔遣中使護其喪以歸。



 安期雖乘世資,頗以才自厲,朝廷數器使之,然無學術,而求入侍經筵,為世所譏。其奉養聲伎,不減其父云。



 論曰:王欽若、丁謂、夏竦,世皆指為奸邪。真宗時,海內乂安,文治洽和,群臣將順不暇,而封禪之議成于謂,天書之誣造端於欽若,所謂以道事君者,固如是耶?竦陰謀
 猜阻,鉤致成事,一居政府,排斥相踵,何其患得患失也!欽若以贓賄幹吏議,其得免者幸矣。然而黨惡醜正,幾敗國家,謂其尤者哉



\end{pinyinscope}