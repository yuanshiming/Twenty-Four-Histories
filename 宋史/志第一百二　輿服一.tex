\article{志第一百二 輿服一}

\begin{pinyinscope}

 五輅大輅大輦芳亭輦鳳輦逍遙輦平輦七寶輦小輿腰輿耕根車進賢車明遠車羊車指南車記里鼓車白鷺車鸞旗車
 崇德車皮軒車黃鉞車豹尾車屬車五車涼車相風烏輿行漏輿十二神輿鉦鼓輿鐘鼓樓輿



 昔者聖人作輿,軫之方以象地,蓋之圓以象天。《易·傳》言:「黃帝、堯、舜,垂衣裳而天下治,蓋取諸乾坤。」夫輿服之制,取法天地,則聖人創物之智,別尊卑,定上下,有大於斯二者乎!舜命禹曰:「予欲觀古人之象,日、月、星辰、山、龍、華蟲作會,宗彞、藻、火、粉米、黼、黻絺繡,以五採彰施於五色,作服,汝明。」《周官》之屬,有巾車、典路、司常,有司服、司裘、內
 司服等職。以是知輿服始於黃帝,成於唐、虞,歷夏及商,而大備於周。周衰,列國肆為侈汰。秦並之,攬上選以供服御,其次以賜百官,始有大駕、法駕之制;又自天子以至牧守,各有鹵簿焉。漢興,乃不能監古成憲,而效秦所為。自是代有變更,志有詳略。《東漢》至《舊唐書》皆稱《輿服》,《新唐書》改為《車服》,鄭樵合諸代為《通志》又為《器服》。其文雖殊,而考古制作,無以尚於三代矣。



 夫三代制器,所以為百世法者,以其華質適中也。孔子答顏淵為邦之問
 曰:「乘殷之輅,服周之冕。」且《禮》謂「周人上輿」,而孔子獨取殷輅,是殷之質勝於周也。又言禹「致美乎黻冕」。而論冕以周為貴,是周之文勝於夏也。蓋已不能無損益於其間焉。不知歷代於秦已還,何所損益乎?



 宋之君臣,於二帝、三王、周公、孔子之道,講之甚明。至其規模制度,飾為聲明,已足粲然,雖不能盡合古制,而於後代庶無愧焉。宋初,袞冕綴飾不用珠玉,蓋存簡儉之風,及為鹵簿,又熾以旗幟,華以繡衣,褻以球杖,豈非循襲唐、五季之習,
 猶未能盡去其陋邪?詒之子孫,殆有甚焉者矣。迄於徽宗,奉身之欲,奢蕩靡極,雖欲不亡得乎?靖康之末,累朝法物,淪沒於金。中興,掇拾散逸,參酌時宜,務從省約。凡服用錦繡,皆易以纈、以羅;旗仗用金銀飾者,皆易以繪、以髹。建炎初,有事郊報,仗內拂扇當用珠飾。高宗曰:「事天貴質,若尚華麗,非禋祀本意也。」是以子孫世守其訓,雖江介一隅,而華質適時,尚足為一代之法。其儒臣名物度數之學,見諸論議,又有可觀者焉。今取舊史所載,
 著於篇,作《輿服志》。



 五輅。宋自神宗以降,銳意稽古,禮文之事,招延儒士,折衷同異。元豐有詳定禮文所,徽宗大觀間有議禮局,政和又有禮制局。先是,元豐雖置局造輅,而五輅及副輅,多仍唐舊。



 玉輅,自唐顯慶中傳之,至宋曰顯慶輅,親郊則乘之。制作精巧,行止安重,後載太常輿闟戟,分左右以均輕重,世之良工,莫能為之。其制:箱上置平盤、黃屋,四柱皆油畫刻鏤。左青龍,右白虎,龜文,金鳳翅,雜花,龍
 鳳,金塗銀裝,間以玉飾。頂輪三層,外施銀耀葉,輪衣、小帶、絡帶並青羅繡雲龍,周綴絲畏帶、羅文佩、銀穗球、小鈴。平盤上布黃褥,四角勾闌設圓鑒、翟羽。虛匱內貼銀鏤香罨,軾匱銀龍二,銜香囊,銀香爐,香寶,錦帶,下有障塵。青畫輪轅,銀轂乘葉,三轅,銀龍頭,橫木上有銀鳳十二。左建青旗,十有二旒,皆繡升龍;右載闟戟,繡黻文,並青繡綢杠。又設青繡門簾,銀飾梯一,拓叉二,推竿一,銀錔頭,銀裝行馬,青繒裹挽索。駕六青馬,馬有金面,插雕羽,
 鞶纓,攀胸鈴拂,青繡屜,錦包尾。又誕馬二,在輅前,飾同駕馬。



 餘輅及副輅皆有之。



 駕士六十四人。金輅色以赤,駕六赤馬,建大旗,駕士六十四人。像輅色以淺黃,駕六赭白馬,建大赤,駕士四十人。革輅色以黃,駕六騧馬,建大白,駕士四十人。木輅色以黑,駕六黑騮馬,建大麾,駕士四十人。自金輅而下,其制皆同玉輅,惟無玉飾。五副輅並駕六馬,駕士四十人,當用銀飾者,皆以銅,餘制如正輅。



 政和三年,議禮局更上皇帝車輅之制,詔頒行。玉輅,箱上平
 盤、黃屋以下皆如舊。頂輪三層,內一層素,輪頂上施金塗銀山花葉及翟羽,青絲繡雲龍絡帶二,周綴雜色絲畏帶八、銅佩八、銀穗球二。平盤上布紅羅繡雲龍褥,曲幾、扶幾,上下設銀螭首二十四。四角勾闌設圓鑒一十六,青羅繡寶相花帶,火珠二十八。香匱設香爐,紅羅繡寶相花帶香囊,香寶,銀結綬二,紅羅繡雲龍結綬一,紅錦幟龍鳳門簾一。青畫輪轅,銀轂乘葉。軾匱、橫轅、前轅並飾以金塗銀螭首,橫轅上施銀立鳳一十二。左建太常,
 十有二旒;右載闟戟,繡黻文。杠褲一,以青繡,杠首飾以銀螭首。金塗銅鈸,青犛牛尾拂,青繒裹索。駕青馬六,馬有銅面,插雕羽,鞶纓,攀胸鈴拂,青線織屜,紅錦包尾。又踏路馬二,在輅前,飾同駕馬。凡大祭祀乘之。



 金輅以下,並以次列其後。若大朝會、冊命皇太子諸王大臣,則設五輅於大慶殿庭,為充庭之儀。金輅赤質,以金飾諸末,建大旗,餘同玉輅,駕赤馬六;凡玉輅之飾以青者,金輅以緋。像輅淺黃質,金塗銅裝,以象飾諸末,建大赤,餘同
 玉輅,駕赭白馬六;凡玉輅之飾以青者,像輅以銀褐。革輅黃質,鞔之以革,建大白,餘同玉輅,駕騧馬六;凡玉輅之飾以青者,革輅以黃。木輅黑質漆之,建大麾,餘同玉輅,駕黑騮六;凡玉輅之飾以青者,木輅以皂。凡玉輅用金塗銀裝者,像輅、革輅、木輅及五副輅,並金塗銅裝。



 又禮制局言:「玉輅馬纓十二而無採,不應古制,欲以五採罽飾樊纓十有二就。輅衡、軾並無鸞和,乞添置。蓋弓二十有二,不應古制,乞增為二十八,以象星。又《巾車》言『玉輅
 建太常』而不言色,《司常》注云:『九旗之帛皆用絳,以周尚赤故也。』《禮記·月令》中央『天子乘大輅,載黃旗』,以金、象、木、革四輅及所建之旗,與四時所乘所載皆合。今玉輅所建之旗,以青帛十二幅連屬為之,有升龍而非交龍,又無三辰,皆非古制。如依成周以所尚之色則用赤,依《月令》兼四代之制則當用黃,仍分縿、斿之制及繡畫三辰於其上。今改制,太常其斿曳地,當依《周官》以六人維之。又《左傳》言:『毚、鸞、和、鈴,昭其聲也。』注:『錫在馬額,鈴在旗首。』
 今旗首無鈴,乞增置。又車蓋周以流蘇及佩各八,無所法象,欲各增為十二,以應天數。又輅之諸末,盡飾以玉,為稱其實,而羅紋雜佩乃用塗金,乞改為玉。又車箱兩轓有金塗龜文及昆翅,左龍右虎,乃後代之制,欲改用蟉龍,加玉為飾。」又言:「既建太常當車之後,則自後登車有妨。《曲禮》言:『君車將駕,則僕執策立於馬前,已駕,僕展軨,效駕,奮衣由右上,取貳綏跪乘,執策分轡,驅之,五步而立,君出就車。』則君升車亦當自右,由前而入。今玉輅
 前有式匱,不應古制,恐當更易,以便登車及改式之制。又《禮記》言『車得其式』,《周官·輿人》:『三分其隧,一在前,二在後,以揉其式,以其廣之半為之式崇。三分軫圍,去一以為式圍。三分軹圍,去一以為轛圍。』注:『立者為轛,橫者為軹。』今玉輅無式。」



 詔:「玉輅用青質,輪輈絡帶,其色如之。四柱、平盤、虛匱則用赤,增蓋弓之數為二十八,左右建旗、常,並青。太常繡日月、五星、二十八宿,旗上則繡以雲龍。朱杠,青絳,鈴垂十有二就,流蘇及佩各增十二之數。樊
 纓飾以五採之罽,衡式之上又加鸞和。輅之諸末,耀葉、螭頭、雲龍、垂牙、錘腳、花版、結綬、羅紋雜佩、羽臺、蔥臺、麻爐、香寶、壓貼牌字,皆飾以玉。自後而升,式匱不去。既成,高二丈七寸五分,闊一丈五尺。副玉輅,亦用青色,舊駕馬四,增為六,色亦以青。」



 政和四年,詔改修正副輅,討論制造金、象、革、木四輅,並依新修玉輅制度。旗、常並建,各與輅一色。除去闟戟,改車箱兩轓龜文、昆翅、左龍、右虎之飾,並用蟉龍。增蓋弓、博山、流蘇等數,軾衡加和鸞,以
 合於古。金輅朱質,飾以金塗銀;左右建太常、大旗及輪衣、絡帶等,色皆以黃;龍旗九斿,如《周官》金輅建大旗之制;駕馬以騧,飾樊纓五採九就。像輅朱質,凡制度、裝綴、名物並同金輅,飾以象及金塗銀銅瑜石;左右建太常、大赤,輪衣、絡帶等,色皆以紅;大赤繡鳥隼七斿,如《周官》象輅建大赤之制;駕馬以赤,飾樊纓七就。革輅朱質,凡制度、裝綴、名物並同金輅,飾以金塗銅瑜石;左右建太常、大白及輪衣、絡帶等,色皆以淺黃;大白繡熊虎六斿,
 如《周官》革輅建大白之制;駕馬以赭白,飾樊纓五就。木輅朱質,凡制度、裝綴、名物皆同金輅,飾以金塗瑜石;左右建太常、大麾及輪衣、絡帶等,色皆以皂;大麾繡龜蛇四斿,如《周官》木輅建大麾之制;駕馬以烏,飾樊纓三就。四輅駕馬各六。玉輅駕士六十四人,餘皆四十人。



 又禮制局增改雅飾諸輅:舊副玉輅色青,飾以金,改用黃而飾以玉;樊纓如正輅之制;建太常,色黃,飾以組,像日月於縿、星辰於斿,其長曳地。舊金輅改用青,飾以金;樊纓
 以五採罽而九就;建大旗,色青,飾以組,像交龍於絲參、升龍於斿,其長齊軫。像輅改用赤,飾以象;樊纓以五採罽而七就;建大赤,色赤,飾以組,像鳥隼於縿、斿,其長齊較。革輅改用白,飾以革;龍勒絳纓,建大白,色白,飾以組,像熊虎於縿、斿,其長齊肩。三輅皆維以縷,削幅為之。木輅依舊色,而飾以漆,其色黑;前樊鵠纓,建大麾,色黑,飾以組,像龜蛇於縿、斿,其長齊首;維以縷,充幅為之。又詔玉輅身仍用紅,太常、旗、絡帶等用黃,餘常、旗、絡帶,亦隨其
 輅色。



 高宗渡江,鹵簿、儀仗悉毀於兵。紹興十二年,始命工部尚書莫將、戶部侍郎張澄等以天禧、宣和《鹵簿圖》考究制度,及故內侍工匠省記指說,參酌制度。是年九月,玉輅成;明年,遂作金、象、革、木四輅,副輅不設。玉輅之制,青色,飾以玉,通高十九尺,輪高六十三寸,輻徑三十九寸,軸長十五尺三寸。頂上剡為輪三層,像天圜也。外施青玉博山八十一,一名耀葉。



 鏤以金塗龍文,覆以青羅,曰輪衣。綴垂玉佩,間以五色垂犛尾,曰流蘇。一名絲畏帶。



 頂四角
 分垂青羅曰絡帶,表裏繡雲龍。遇雨,則油黃繒覆之。



 輅之中四柱,像地方也,前柱卷龍。平盤上布錦褥,前有橫軾,後垂錦軟簾。登車則自後卷簾梯級以登。四面周以闌而闕其中,以備登降。執綏官先自右升,立於右柱下,以備顧問。闌柱頭有玉蹲龍。軾前有牌,鏤曰「玉輅」,以玉篆之,上有玉龍二。中設御坐,純以黃香木為之,取其黃中之正色也。下有塗金蹲龍十六。在平盤四圍下,又有拓角雲龍,金彩飾之,前後左右各二。前有轅木三,鱗體
 昂首龍形。轅木上束兩橫竿,在前者名曰鳳轅,馬負之以行;次曰推轅,班直推之以助馬力。橫於轅後者名曰壓轅,以人壓於後,欲取其平。車輪三歲一易,心用榆,圜數尺,圈以鐵,以防折裂。橫貫大木以為軸,夾以兩輪,輪皆彩畫,此輅下飾也。每新輪成,載鐵萬斤試之。



 左建太常,右建龍旗,插於輅後兩柱之金環前。駕青馬六,馬有鏤錫,鞶纓,金鈴,紅旄繡屜,金包□□,錦包尾,青繒裹索引之。駕士二百三十二人。



 誕馬十二人,左右索百二十八人,入轅馬十二人,龍頭子二人,
 前後抱轅各六人,推竿四人,捧輪四人,拓叉四人,凈席四人,前攔人員一人,後攔人員一人,前攔馬八人,後攔馬八人,踏道人員二人,踏道二十人,小拓叉四人,小梯子二人,燭臺二人,香匙剪子二人,左右索人員二人。鵩又有呵喝人員二人,教馬官二人,捧輪將軍四人,千牛衛將軍二人,推輪軸官健八人,抱太常龍旗官六人,職掌五人,專知官一人,手分一人,庫子八人,裝掛工匠二人,諸作工匠十五人,蓋覆儀鸞司十一人,監官三員。



 金輅黃色,飾以金塗銀,制如玉輅,而高減五寸;博山、輪衣、絡帶、轅輻、軸並以黃,建大旗九斿;駕黃馬六,駕士一百五十四人。像輅朱色,飾以象及金塗銅,制如金輅;博山、輪衣、絡帶並以朱,建大赤七斿;駕赤馬六,駕士一百
 五十四人。革輅淺黃白色,飾以金塗銅,制如象輅;博山、輪衣、絡帶並以淺黃白,建大白六斿;駕黃白馬六,駕士百五十四人。木輅黑色,飾以金塗銀,制如革輅;博山、輪衣、絡帶並以黑,建大麾四斿;駕黑馬六,駕士一百五十四人。五輅駕士服色:平巾幘、青絹抹額、纈絹對花鳳袍、緋纈絹對花寬袖襖、羅襪絹褲、衣蔑、麻鞋,其色各從其輅。



 大輅。政和六年,徐秉哲言:「南北郊,皇帝乘玉輅以赴齋宮。自齋宮赴壇,正當祀天祭地,乃乘大輦,疑非禮意。」下
 禮制局討論。禮制局請:「造大輅如玉輅之制,唯不飾以玉。所駕之馬,其數如之,唯樊纓一就,以稱尚質之義。仍建大旗十有二旒,龍章日月,以協象天之義。至禮畢還齋宮,則御大輦,於禮無嫌。」從之。



 大輦。《周官》巾車氏有輦車,以人組挽之,宮中從容所乘。唐制,輦有七:一曰大鳳輦,二曰大芳輦,三曰仙游輦,四曰小輕輦,五曰芳亭輦,六曰大玉輦,七曰小玉輦。



 太祖建隆四年,翰林學士承旨陶穀為禮儀使,創意造為大
 輦:赤質,正方,油畫,金塗銀葉,龍鳳裝。其上四面行龍雲朵,火珠方鑒,銀絲囊網,珠翠結絳,雲龍鈿窠霞子。四角龍頭銜香囊,頂輪施耀葉。中有銀蓮花坐龍,紅綾里,碧牙壓帖。內設圓鑒,銀絲香囊,銀飾勾闌、臺坐,紅絲絳網,□錔。中施黃褥,上置御坐,扶幾,香爐,錦結綬。幾衣、輪衣、絡帶並緋繡壓金銀線。長竿四,銀裹鐵鋦龍頭,魚鉤,錦膊褥,銀裝畫梯,拓叉,黃羅緣席、褥、帊,梯杖褥,朱索,緋繒油帊。主輦六十四人。親祀南郊、謁太廟還及具鸞駕黃
 麾仗、省方還都,則乘之。



 真宗東封,以舊輦太重,遂命別造,凡減七百餘斤,後常用焉。神宗已後,其制:赤質,正方,油畫,金塗銀龍鳳裝,朱漆天輪一,金塗銀頂龍一。四面施行龍一十六,火珠四。四角龍頭四,穗球一十二。頂輪施耀葉,紅羅輪衣一,綴銀鈴,紅羅絡帶二。中設御坐、曲幾、錦褥等,施屏風,香爐,結綬。長竿四,飾以金塗銀龍頭。祀畢,車駕還內,若不進輅,則乘大輦。



 政和之制:黃質,冒以黃衣,紘以黃帶。車箱四圍,於桯之外,高二尺二寸。設
 軾於前楹,軾高三尺二寸。建大旗於後楹,旗十二斿,其長曳地,其色黃,繪以交龍;素帛為縿,繪以日月,以弧張幅,以韣韜弧;杠以青錦綢之,注旄於竿首,系以鈴。



 國朝之輦有七,中興後,唯存大輦、平輦、逍遙三輦而已。大輦又曰大安輦,其制:赤質,正方,高十五尺三寸,方十一尺六寸。四柱,平盤,上覆青綠錦。上有天輪三層,外施金塗銀博山八十一。內有圓鏡,金塗銀頂龍一,四面行龍十六,火珠四。輪衣以青,墜以金鈴,頂有青羅十字分垂四
 角,曰絡帶。四角出龍首,銜犛牛五色尾,曰旒綏。四面拱斗,外施方鏡,九柱圍以朱闌,中設御坐、曲幾、屏風、錦褥。下舉以長竿四,攢竹筋膠丹漆之,竿為龍首。平盤下,四圍結紅絲網。輦官服色:武弁,黃纈對鳳袍,黃絹勒帛,紫生色袒帶,紫絹行縢。



 芳亭輦,黑質,頂如幕屋,緋羅衣,裙襴、絡帶皆繡雲鳳。兩面朱綠窗花版,外施紅絲網綢,金銅□錔,前後垂簾,下設牙床、勾闌。長竿四,銀龍頭,銀飾梯,行馬。主輦一百二
 十人。政和之制,簾以紅羅繡鵝為額,內設御坐,長竿飾以金塗銅螭首,橫竿二。



 鳳輦,赤質,頂輪下有二柱,緋羅輪衣,絡帶、門簾皆繡雲鳳。頂有金鳳一,兩壁刻畫龜文、金鳳翅。前有軾匱、香爐、香寶、結帶,下有勾闌二重,內設紅錦褥。長竿二,銀飾梯,行馬。主輦八十人。法駕鹵簿,不設鳳輦。



 逍遙輦,以棕櫚為屋,赤質,金塗銀裝,朱漆扶版二,云版一,長竿二,飾以金塗銀龍頭。常行幸所御。又魚鉤,□錔,
 梅紅絳。輦官十二人,春夏服緋羅衫,秋冬服白師子錦襖。東封,別造闢塵逍遙輦,加窗隔,黃繒為里,賜名省方逍遙輦。中興之制,赤質,金塗四柱,棕屋上有走脊金龍四,中起火珠凸頂,四面不設窗障,中有御踏子,制甚簡素。祗應人員服帽子、宜男方勝纈衫。



 平輦,又名平頭輦,亦曰太平輦,飾如逍遙輦而無屋。輦官十二人,服同逍遙輦。常行幸所御。東封,別造升山天平輦,施機關,賜名曰登封輦。中興之制,赤質,正方,形如一朱龍椅而加長
 竿二,飾如逍遙輦而不施棕屋,制尤簡素,止施畫雲版而已。



 又有七寶輦,隆興二年,為德壽宮所制也。高五十一寸,闊二十七寸,深三十六寸。比附大輦、平輦制度為之。上施頂輪、耀葉、角龍、頂龍、滴子、鐸子、結穗球。下施梅紅絲裙網,加綴七寶,中設香木御坐,引手為轉身龍,告背為龍首,告枰子織以紅黃藤。舁以長竿二,竿為螭首,金塗銀飾焉。初,有司言:「東都舊制,輦飾以玉,裙網用七寶,而滴子用真珠。」帝曰:「上皇意不然,止欲簡素。」遂以塗金
 易玉,梅紅絲結裙網,間綴七寶,而角牙易真珠。既而上皇卻不受,每至大內,多乘馬,而間有行幸,則用肩輿。自是,重華、壽康兩宮並不別造。



 小輿,赤質,頂輪下施曲柄如蓋,緋繡輪衣、絡帶,制如鳳輦而小。下有勾闌,牙床,繡瀝水。中設方床,緋繡羅衣,錦褥。上有小案、坐床,皆繡衣。踏床緋衣。前後長竿二,銀飾梯,行馬。奉輿二十四人。中興後,去其輪蓋,方四十九寸,高三十一寸。輿上周以勾闌,施翟羽,玉照子,中為方床
 三級。上設御坐、曲幾、踏子,曲柄緋羅繡蓋,輿下紅絲結五色花裙網。舁以長竿二,竿為螭首。宮殿從容所乘,設鹵簿則陳之。



 腰輿,前後長竿各二,金銅螭頭,緋繡鳳裙襴,上施錦褥,別設小床,緋繡花龍衣。奉輿十六人。中興制,赤質,方形,四面曲闌,下結繡裙網。制如小輿,惟無翟尾、玉照子、三級床、曲柄蓋,而上設方御床、曲幾,舁竿無螭首,用亦同小輿。



 耕根車制,青質,蓋三層,餘如五輅之副。駕六青馬,駕士四十人。親祠具大駕、法駕鹵簿,並列於仗內;若耕籍則乘之。國朝之車,自耕根而下,凡十有五。南渡所存,惟耕根車一而已,其制度並同,惟駕士七十五人。



 進賢車,古安車也。太祖乾德元年改赤質,兩壁紗窗,擎耳,虛匱,一轅,緋幰衣,絡帶、門簾皆繡鳳,紅絲網。中設朱漆床,香案,紫綾案衣,緋繒裹挽索,朱漆行馬。



 凡車皆有挽索、行馬。駕四馬,駕士二十四人。



 明遠車,古四望車也,駕以牛。太祖乾德元年改,仍舊四馬。赤質,制如屋,重簷勾闌,上有金龍,四角垂銅鐸,上層四面垂簾,下層周以花版,三轅。駕士四十人,服繡對鳳。



 羊車,古輦車也,亦為畫輪車,駕以牛。隋駕以果下馬,今亦駕以二小馬。赤質,兩壁畫龜文、金鳳翅,緋幰衣、絡帶、門簾皆繡瑞羊。童子十八人。



 指南車,一曰司南車。赤質,兩箱畫青龍、白虎,四面畫花鳥,重臺,勾闌,鏤拱,四角垂香囊。上有仙人,車雖轉而手
 常南指。一轅。鳳首,駕四馬。駕士舊十八人,太宗雍熙四年,增為三十人。仁宗天聖五年,工部郎中燕肅始造指南車,肅上奏曰:



 黃帝與蚩尤戰於涿鹿之野,蚩尤起大霧,軍士不知所向,帝遂作指南車。周成王時,越裳氏重譯來獻,使者惑失道,周公賜軿車以指南。其後,法俱亡。漢張衡、魏馬鈞繼作之,屬世亂離,其器不存。宋武帝平長安,嘗為此車,而制不精。祖沖之亦復造之。後魏太武帝使郭善明造,彌年不就,命扶風馬嶽造,垂成而為善
 明鴆死,其法遂絕。唐元和中,典作官金公立以其車及記里鼓上之,憲宗閱於麟德殿,以備法駕,歷五代至國朝,不聞得其制者,今創意成之。



 其法:用獨轅車,車箱外籠上有重構,立木仙人於上,引臂南指。用大小輪九,合齒一百二十。足輪二,高六尺,圍一丈八尺。附足立子輪二,徑二尺四寸,圍七尺二寸,出齒各二十四,齒間相去三寸。轅端橫木下立小輪二,其徑三寸,鐵軸貫之。左小平輪一,其徑一尺二寸,出齒十二;右小平輪一,其徑一
 尺二寸,出齒十二。中心大平輪一,其徑四尺八寸,圍一丈四尺四寸,出齒四十八,齒間相去三寸。中立貫心軸一,高八尺,徑三寸。



 上刻木為仙人,其車行,木人指南。若折而東,推轅右旋,附右足子輪順轉十二齒,擊右小平輪一匝,觸中心大平輪左旋四分之一,轉十二齒,車東行,木人交而南指。若折而西,推轅左旋,附左足子輪隨輪順轉十二齒,擊左小平輪一匝,觸中心大平輪右轉四分之一,轉十二齒,車正西行,木人交而南指。若欲北
 行,或東,或西,轉亦如之。



 詔以其法下有司制之。


大觀元年,內侍省吳德仁又獻指南車、記里鼓車之制,二車成,其年宗祀大禮始用之。其指南車身一丈一尺一寸五分,闊九尺五寸,深一丈九寸,車輪直徑五尺七寸,車轅一丈五寸。車箱上下為兩層,中設屏風,上安仙人一執仗,左右龜鶴各一,童子四各執纓立四角,上設關戾。臥輪一十三,各徑一尺八寸五分,圍五尺五寸五分,出齒三十二,齒間相去一寸八分。中心輪軸隨屏風貫下,下
 有輪一十三,中至大平輪。其輪徑三尺八寸,圍一丈一尺四寸,出齒一百,齒間相去一寸二分五厘,通上左右起落。二小平輪,各有鐵墜子一,皆徑一尺一寸,圍三尺三寸,出齒一十七,齒間相去一寸九分。又左右附輪各一,徑一尺五寸五分,圍四尺六寸五分,出齒二十四,齒間相去二寸一分。左右疊輪各二,下輪各徑二尺一寸,圍六尺三寸,出齒三十二,齒間相去二寸一分;上輪各徑一尺二寸,圍三尺六寸,出齒三十二,齒間相去一寸
 一分。左右車腳上各立輪一,徑二尺二寸,圍六尺六寸,出齒三十二,齒間相去二寸二分五厘。左右後轅各小輪一,無齒,系竹
 \gezhu{
  □但}
 並索在左右軸上,遇右轉使右轅小輪觸落右輪,若左轉使左轅小輪觸落左輪。行則仙童交而指南。車駕赤馬二,銅面,插羽,鞶纓,攀胸鈴拂,緋絹屜,錦包尾。



 記里鼓車,一名大章車。赤質,四面畫花鳥,重臺,勾闌,鏤拱。行一里,則上層木人擊鼓;十里,則次層木人擊鐲。一
 轅,鳳首,駕四馬。駕士舊十八人,太宗雍熙四年,增為三十人。



 仁宗天聖五年,內侍盧道隆上記里鼓車之制:「獨轅雙輪,箱上為兩重,各刻木為人,執木槌。足輪各徑六尺,圍一丈八尺。足輪一周,而行地三步。以古法六尺為步,三百步為里,用較今法五尺為步,三百六十步為里。立輪一,附於左足,徑一尺三寸八分,圍四尺一寸四分,出齒十八,齒間相去二寸三分。下平輪一,其徑四尺一寸四分,圍一丈二尺四寸二分,出齒五十四,齒間相去
 與附立輪同。立貫心軸一,其上設銅旋風輪一,出齒三,齒間相去一寸二分。中立平輪一,其徑四尺,圍一丈二尺,出齒百,齒間相去與旋風等。次安小平輪一,其徑三寸少半寸,圍一尺,出齒十,齒間相去一寸半。上平輪一,其徑三尺少半尺,圍一丈,出齒百,齒間相去與小平輪同。其中平輪轉一周,車行一里,下一層木人擊鼓;上平輪轉一周,車行十里,上一層木人擊鐲。凡用大小輪八,合二百八十五齒,遞相鉤鎖,犬牙相制,周而復始。」詔以
 其法下有司制之。



 大觀之制,車箱上下為兩層,上安木人二,身各手執木槌。輪軸共四。內左壁車腳上立輪一,安在車箱內,徑二尺二寸五分,圍六尺七寸五分,二十齒,齒間相去三寸三分五厘。又平輪一,徑四尺六寸五分,圍一丈三尺九寸五分,出齒六十,齒間相去二寸四分。上大平輪一,通軸貫上,徑三尺八寸,圍一丈一尺,出齒一百,齒間相去一寸二分。立軸一,徑二寸二分,圍六寸六分,出齒三,齒間相去二寸二分。外大平輪軸上有
 鐵撥子二。又木橫軸上關戾、撥子各一。其車腳轉一百遭,通輪軸轉周,木人各一擊鉦、鼓。



 白鷺車,隋所制也,一名鼓吹車。赤質,周施花版,上有朱柱,貫五輪相重,輪衣以緋,皂頂及緋絡帶,並繡飛鷺。柱杪刻木為鷺,銜鵝毛筒,紅綬帶。一轅。駕四馬,駕士十八人。



 鸞旗車,漢制,為前驅。赤質,曲壁,一轅。上載赤旗,繡鸞鳥。駕四馬,駕士十八人。



 崇德車,本秦闢惡車也。上有桃弧棘矢,所以禳卻不祥。太祖乾德元年,改赤質,周施花版,四角刻闢惡獸,中載黃旗,亦繡此獸。太卜署令一人,在車中執旗。駕四馬,駕士十八人。政和之制,建黃羅繡崇德旗一,彩畫刻木獬豸四。宣和元年,禮制局言:「崇德車載太卜令一員,畫闢惡獸於旗。《記》曰『前巫而後史』,《傳》曰『桃弧棘矢,以供御王事』。請以巫易太卜,弧矢易闢惡獸。」從之。



 皮軒車,漢前驅車也。冒以虎皮為軒,取《曲禮》「前有士師,
 則載虎皮」之義,赤質,曲壁,上有柱,貫五輪相重,畫虎文。駕四馬,駕士十八人。政和之制,用漆柱,貫朱漆皮軒五。



 黃鉞車,漢制,乘輿建之,在大駕後。晉鹵簿有黃鉞車。唐初無之,貞觀後始加。赤質,曲壁,中設金鉞一,錦囊綢杠。左武衛隊正一人,在車中執鉞。駕兩馬,駕士十五人。



 豹尾車。古者軍正建豹尾。漢制,最後車一乘垂豹尾,豹尾以前即同禁中。唐貞觀後,始加此車於鹵簿內,制同黃鉞車。上戴朱漆竿,首綴豹尾,右武衛隊正一人執之。
 駕兩馬,駕士十五人。



 屬車,一曰副車,一曰貳車,一曰左車。秦制,大駕屬車八十一乘,法駕三十六乘。漢法駕用三十一乘,小駕用十二乘。隋制,大駕三十六,法駕十二,小駕不用。唐大駕唯用十二乘,宋因之。黑質,兩箱CP裝,前有曲闌,金銅飾,上施紫通幰,絡帶、門簾皆繡雲鶴,紫絲網□錔。每乘駕三牛,駕士十人。



 五車。徽宗宣和元年,禮制局言:「舊鹵簿記有白鷺、鸞旗、
 皮軒三車,其制非古。按《曲禮》曰:『前有水則載青旌,前有塵埃則載鳴鳶,前有車騎則載飛鴻,前有士師則載虎皮,前有鷙獸則載貔貅。』萬乘一出,五車必載,所以警眾也。青旌、鳴鳶、飛鴻、貔貅乃以白鷺、鸞旗雜陳其間,未為合禮。今欲改五車相次於中道,繼之以崇德車,於是為備。」青旌車,赤質,曲壁,中載青旌,以絳帛為之,書青鳥於其上。鳴鳶車,赤質,曲壁,中載鳴鳶旌,以絳帛為之,畫鳴鳶於其上。飛鴻車,赤質,曲壁,中載飛鴻旌,以絳帛為之,
 畫飛鴻於其上。虎皮車,赤質,曲壁,中載虎皮旌,以絳帛為之,緣以赤,畫虎皮於上。貔貅車,赤質,曲壁,旌以絳帛為之,緣以赤,畫貔貅於上。其轅皆一。



 涼車,赤質,金塗銀裝,龍鳳五採明金,織以紅黃藤,油壁,緋絲絳龍頭,梅紅羅褥,銀螭頭,穗球,雲朵踏頭,蓮花坐,雁鉤,火珠,門沓,金屈鉞,頻伽,大小鐶,駕以橐駝。省方在道及校獵回則乘之。



 相風烏輿,上載長竿,竿杪刻木為烏,垂鵝毛筒,紅綬帶,
 下承以小盤,周以緋裙,繡烏形。輿士四人。



 行漏輿,隋大業行漏車也。制同鐘、鼓樓而大,設刻漏如稱衡。首垂銅缽,末有銅像,漆匱貯水,渴烏注水入缽中。長竿四,輿士六十人。



 十二神輿,赤質,四門旁刻十二辰神,緋繡輪衣、絡帶。輿士十二人。



 交龍鉦、鼓輿各一,皆刻木為二青龍相交,下有木臺、長竿,一持畫鼓,一掛金鉦,上皆有緋蓋,亦繡交龍。輿士各
 二人。中興後,相風、行漏、十二神、鉦鼓四輿,悉省去。



 鐘、鼓樓輿各一,本隋大駕鐘車、鼓車也。皆刻木為屋,中置鐘、鼓,下施木臺、長竿,如鉦、鼓輿。輿士各二十四人。



 行漏輿、十二神輿、交龍鉦鼓輿、鐘鼓樓,舊禮無文,皆太祖開寶定禮所增。



\end{pinyinscope}