\article{志第六十二 禮十二(吉禮十二)}

\begin{pinyinscope}

 後廟景靈宮神御殿功臣配侑群臣家廟



 後廟之制。建隆三年,追冊會稽郡夫人賀氏曰孝惠皇后,止就陵所置祠殿奉安神主,薦常饌,不設牙盤祭器。
 乾德元年,孝明皇后王氏崩,始議置廟及二後先後之次。太常博士和峴請共殿別室,以孝明正位內朝,請居上室;孝惠緣改葬,不造虞主,與孝明同祔,宜居次室。禮院又言:「後廟祀事,一準太廟,亦當立戟。」及太祖祔廟,有司言:「合奉一後配食。按唐睿宗追謚肅明、昭成二後,至睿宗崩,獨昭成以帝母之重升配,肅明止享於儀坤廟。近周世宗正惠、宣懿二後並先崩,正惠無位號,宣懿居正位,遂以配食。今請以孝明皇后配,忌日行香廢務,其
 孝惠皇后享於別廟。」從之。



 太平興國元年,追冊越國夫人符氏為懿德皇后,尹氏為淑德皇后,並祔後廟。



 至道三年,孝章皇后宋氏祔享,有司言:「孝章正位中壺,宜居上室,懿德追崇後號,宜居其次。」詔孝章殿室居懿德下。六月,禮官議:「按太平興國中追冊定謚,皆以懿德居上。淳化初,宗正少卿趙安易言,別廟祭享,懿德在淑德之上,未測升降之由。其時敕旨依舊懿德在上。按《江都集禮》,晉景帝即位,夏侯夫人應合追尊。散騎常侍任
 茂、傅玄等議云:『夏侯夫人初歸景帝,未有王基之道,不及景帝統百揆而亡,後妃之化未著遠邇,追尊無經義可據。』今之所議,正與此同。且淑德配合之初,潛躍之符未兆;懿德輔佐之始,藩邸之位已隆,然未嘗正位中宮,母臨天下。豈可生無尊極之位,沒升配享之崇?於人情不安,於典籍無據。唐順宗祔廟後十一年,始以莊憲皇后升配,憲宗祔廟後二十五年,始以懿安皇后升配。今請虛位,允協舊儀。」再詔尚書省集議及禮官同詳定。上議曰:「
 淑德皇后生無位號,沒始追崇,況在初潛,早已薨謝,懿德皇后享封大國,作配先朝,雖不及臨御之期,且夙彰賢懿之美,若以升祔,當歸懿德。又詳周世宗正惠、宣懿配食故事,當時議以正惠追尊位號,請以宣懿為配。是時以太后在位,疑宣懿祔廟之後,立忌非便。議者引晉哀帝時何太后在上,尊所生周氏為太妃,封其子為瑯邪王。及太妃薨,帝奔喪瑯邪第,七月而葬。此則奔喪行服,尚不厭降,即忌日廢務,於理無嫌。今禮官引唐順、憲
 二宗廟,享虛位之文,夫即追冊二後,即虛室亦為非便,請奉懿德神主升配。又按議者以周世宗神主祔廟,必若宣懿同祔,即正惠神主請加『太』字。今升祔懿德,請即加淑德『太』字,仍舊別廟。」詔:「以懿德配享,至於『太』者尊極之稱,加於母後,施之宗廟禮所未安。」乃不加「太」字,仍別廟配享。十二月,追尊賢妃李氏為元德皇太后。有司言:「按《周禮·春官》大司樂之職,『奏《夷則》,歌《仲呂》,以享先妣』,謂姜嫄也。是帝嚳之妃,後稷之母,特立廟曰閟宮。晉簡文
 宣後以不配食,築室於外,歲時享祭。唐先天元年,始祔昭成、肅明二后於儀坤廟。又玄宗元獻楊後立廟於太廟之西。稽於前文,咸有明據。望令宗正寺於後廟內修奉廟室,為殿三間,設神門、齋房、神廚,以備薦享。」



 咸平元年,判太常禮院李宗訥等言:「元德皇太后別建廟室,淑德皇后亦在別廟,同是帝母而無『太』字。按唐穆宗三後,除宣懿升祔,正獻、恭僖二後並立別廟,各有『太』字。又開元初,太常議昭成皇太后,請不除『太』字,云『入廟稱後,義
 系於夫,在朝稱太后,義系於子。如謚冊入陵,神主入廟,則當去太字』。按神主入廟之說,蓋為祔享太廟,以厭降故,不加『太』字,則本朝文懿諸後是也。如別建廟室,不可但稱皇后,則唐正獻、恭僖二太后是也。淑德皇后亦請加『太』字,既加之後,望遷就元德新廟,居第一室,以元德次之,仍遷莊懷又次之。」詔下中書集議。兵部尚書張齊賢等奏:「宗廟神靈,務乎安靜。況懿德作合之始,逮事舅姑,躬執婦道,祔享之禮,宜從後先,伏請仍舊。又漢因秦
 制,帝母稱皇太后。檢詳去歲議狀,請加淑德『太』字,而詔不加之者,緣當時元德皇太后未行追冊。今冊命已畢,望依禮官所言。」三年四月乙卯,祔葬元德皇太后於永熙陵。有司言:「元德神主祔廟,準禮當行祔謁,載稽前典,有未合者。伏以追薦尊稱,奉加『太』字,崇建別廟,以備蒸嘗。況當禘祫之時,不預合食之列,廟享之制與諸後不同。俟神主還京,即祔廟室,薦獻安神,更不行祔謁之禮,每歲五享、禘祫如太廟儀。」



 景德四年,奉莊穆皇后郭氏
 神主謁太廟,祔享於昭憲皇后。享畢,祔別廟,殿室在莊懷之上。帝祀汾陰,謁廟畢,親詣元德皇太后廟躬謝,自門降輦步入,酌獻如太廟,設登歌,兩省、御史、宗室防禦使以上班廟內,餘班廟外,遣官分告孝惠諸後廟。詔:「太廟、元德皇后廟享用犢,諸後廟親享用犢,攝事用羊、豕。」



 五年,龍圖閣直學士陳彭年言:「禘祫日,孝惠、淑德二後神主自別廟赴太廟,祔簡穆皇后神主之下、太祖神主之上,此蓋用《曲臺禮》別廟皇后禘祫祔享太廟之說。
 竊慮明靈合享,神禮未安,望詔禮官再加詳定。」有司言:「按《曲臺禮》載禘祫之儀,則如皇后先祔別廟,遇禘祫祔享於太廟,如是昭後,即坐於祖姑之下,南向;如是穆後,即坐於祖姑之下,北向。又按博士殷盈孫議云:『別廟皇后禘袷於太廟,祔於祖姑之下者,此乃皇后先沒,已造神主。如昭成、肅明之沒也,睿宗在位;元獻之沒也,玄宗在位;昭德之沒也,德宗在位。四后於太廟未有本室,故創別廟,當為太廟合食之主,故禘袷乃奉以入享,此明其
 後太廟有本室,即當遷祔。帝方在位,故皇后暫立別廟,禮本合食,故禘祫乃升太廟,以未有位,故祔祖姑之下。據《開寶通禮》與《曲臺禮》同。今有司不達禮意,遇禘祫歲,尚以孝惠、孝章、淑德三後神主祔享祖姑之下,乃在太祖、太宗之上。按《禮》稱『婦祔祖姑』,謂既卒哭之明日,此正禮也;稱『祖姑有三人,則祔於親者』,注,玄謂『舅之母死,而又有繼室二人,親者謂舅所生』。然則祖姑有三人同在祖室,明婦有數人亦當同在夫之本室,不可久祔於祖
 姑也。故《開元禮》但載肅明皇后別廟時享之儀,而無禘袷之禮,即知別廟時享及禘祫皆於本廟也。孝惠、孝章、淑德禘祫既祔太廟,則自今禘祫祔享本室,次於正主,庶協典禮。」六年,升祔元德皇后太宗廟室,詔以祔廟歲時為合享次序,而位明德皇后之次。



 明道二年,判河南府錢惟演請以章獻、章懿二後並祔真宗之室。太常禮院議:「夏、商以來,父昭子穆,皆有配坐,每室一帝一後,禮之正儀。唐開元中,昭成、肅明二後始並祔於睿宗。今惟
 演引唐武宗母韋太后升祔穆宗,本朝孝明、孝章禘太祖故事。按穆宗惟韋後祔,太祖未嘗以孝章配。伏尋先帝以懿德配享太宗,及明德園陵禮畢,遂得升祔。元德太后自追尊後,凡十七年始克升祔。今章穆皇后著位長秋,祔食真宗,斯為正禮。章獻太后母儀天下,與明德例同,若從古禮,止應祀后廟,若便升祔,似非先帝慎重之意,又況前代無同日並祔之比,惟上裁之。」乃詔有司更議,皆謂:「章穆位崇中壺,與懿德有異,已祔廟室,自協
 一帝一後之文。章獻輔政十年,章懿誕育帝躬,功德莫與為比,退就後廟,未厭眾心。按《周官》大司樂職,『奏《夷則》,歌《小呂》,以享先妣』者,姜嫄也,帝嚳之妃,後稷之母,特立廟曰閟宮。宜別立新廟,奉安二太后神主,同殿異室,歲時薦享用太廟儀。別立廟名,自為樂曲,以崇世享。忌前一日,不御正殿,百官奉慰,著之令甲。」乃作新廟兩廟間,名曰奉慈。



 慶歷四年,從呂公綽言:「先帝特謚二後莊懷、莊穆,及上真宗文明武定章聖元孝之謚,郭后升祔,當
 正徽號,宜於郊禮前遣官先上寶冊,改『莊』為『章』,止告太廟,更不改題。」遂如故事。將郊,所司導五後寶冊赴三廟,各於神門外幄次以待。奏告畢,皆納於室。俄又詔中書門下令禮官考故事,升祔章懿神主。禮院言:「章獻、章懿宜序章穆之次,章惠先朝遺制嘗踐太妃,至明道中始加懿號,與章懷頗同,請序章懷之次。太者生事之禮,不當施於宗廟。章獻以顧托之重,臨御之勞,欲稱別廟,義無所嫌,屬之配室,禮或未順。」學士王堯臣等言:「章獻明
 肅盛烈丕功,非一惠可舉,謚告於廟,冊藏於陵,無容追減。章惠擁祐帝躬,並均顧復,故景祐中膺保慶之冊,義專系子,禮須別祠。章穆升附,歲月已深。奉慈三室,先後已定,若再議升降,則情有重輕,請如舊制。」中書門下復議:「成憲在前,文考之意;配食一體,二慈之宜;奉承無私,陛下之孝。請如禮官及學士議。案祥符詔系章聖特旨,位敘先後,乞聖制定數,昭示無窮。」詔依所議。十月,文德殿奉安寶冊,帝服通天冠、絳紗袍,執圭。太常奏樂,百官
 宿廟堂。次日,有司薦享諸廟。寅時,復詣正衙,宰臣、行事官贊導冊寶至大慶殿庭發冊,出宣德門,攝太尉賈昌朝、陳執中受以赴奉慈廟上寶冊,告遷二主,皆塗「太」字,祔於太廟。



 至和元年七月,有司奉詔立溫成皇后廟,享祭器數視皇后廟。後以諫官言,改為祠殿,歲時令宮臣薦以常饌。



 治平元年,同判太常寺呂公著言:「按《喪服小記》『慈母不世祭』。章惠太后,仁宗嘗以母稱,故加寶慶之號。蓋生有慈保之勤,
 故沒有廟享之報。今於陛下恩有所止,禮難承祀,其奉慈廟,乞依禮廢罷。」



 熙寧二年,命攝太常卿張掞奉章惠太后神主瘞陵園。



 元豐六年,詳定所言:「按《禮》,夫婦一體,故昏則同牢、合巹,終則同穴,祭則同幾、同祝饌,未嘗有異廟者也。惟周人以姜嫄為媒神,而帝嚳無廟,又不可下入子孫之廟,乃以別廟而祭,故《魯頌》謂之閟宮,《周禮》謂之先妣,可也。自漢以來,不祔不配者,皆援姜嫄為比,或以其微,或以其繼而已。蓋其間有天下者,起於側微,而其後不及正位中宮,或以嘗正
 位矣,有所不幸,則當立繼以奉宗廟,故有『祖姑三人則祔於親者』之說。立繼之禮,其來尚矣。始微終顯,皆嫡也,前娶後繼,皆嫡也。後世乃以始微後繼置之別廟,不得伸同幾之義,則非禮意。恭惟太祖孝惠皇后、太宗淑德皇后、真宗章懷皇後實皆元妃,而孝章則太祖繼後,乃皆祭以別廟,在禮未安,請升祔太廟,增四室,以時配享。」七月,遂自別廟升祔焉。



 政和四年,有司言:「政和元年孟冬祫享,奉惠恭神主入太廟,祔於祖姑之下。今歲當祫,
 而明達皇后神主奉安陵祠,緣在城外。三代之制,未有即陵以為廟者。今明達皇后追正典冊,歲時薦享,並同諸後,宜就惠恭別廟增建殿室,迎奉神主以祔。」又言:「明達神主祔謁日,於英宗室增設宣仁聖烈皇后、明達皇后二位,及遍祭七祀、配享功臣,並別廟祔享惠恭、明達二位。」



 紹興七年,惠恭改謚為顯恭,以上徽宗聖文仁德顯孝之謚故也。十二年五月,禮部侍郎施坰言:「懿節皇后神主,候至卒哭擇日祔廟,合依顯恭皇后禮,於太廟
 內修建殿室,以為別廟安奉。」又言:「將來祔廟,其虞主合於本室後瘞埋。緣別系行在祔廟,欲於本室冊寶殿收奉,候回京日依別廟故事。」從之。七月,有司行九虞之祭奉安。三十二年,禮部、太常言:「故妃郭氏追冊為皇后,合依懿節皇后祭於別廟。所有廟殿,見安懿節皇后神主,行禮狹隘。乞分為二室,以西為上,各置戶牖,及擗截本廟齋宮,權安懿節神主,工畢還殿。」王普又請各置祏室。並從之。



 乾道三年閏七月,安恭皇后神主祔於別廟,為
 三室。



 景靈宮。創於大中祥符五年,聖祖臨降,為宮以奉之。天聖元年,詔修宮之萬壽殿以奉真宗,署曰奉真。明道二年,又建廣孝殿,奉安章懿皇后。治平元年,又詔就宮之西園建殿,以奉仁宗,署曰孝嚴,奉安御容,親行酌獻,命大臣分詣諸神御代行禮。翼日,太后酌獻,皇后、大長公主以下內外命婦陪位於廷。詔每歲下元朝謁,如奉真殿儀,有期以上喪或災異,則命輔臣攝事。名齋殿曰迎厘,
 宮西門曰廣祐。四年,建英德殿,奉英宗神御。凡七十年間,神御在宮者四,寓寺觀者十有一。



 元豐五年,始就宮作十一殿,悉迎在京寺觀神御入內,盡合帝後,奉以時王之禮。十一月,百官班於集英殿廷,帝詣蕊珠、凝華等殿,行告遷廟禮,禮儀使奉神御升彩輿出殿。明日,復行薦享如禮,禮儀使奉神輿行,帝出幄,導至宣德門外,親王、使相、宗室正任以上前引,望參官及諸軍都虞候、宗室副率以上陪位,內侍省押班整儀衛以從,奉安神御
 於十一殿。明日,帝詣宮朝獻,先謁天興殿,以次行禮,並如四孟儀。詔自今朝獻孟春用十一日,孟夏擇日,孟秋用中元日,孟冬用下元日,天子常服行事。薦聖祖殿以素饌,神御殿以膳羞,器服儀物,悉從今制。天興殿門以奉天神不立戟,諸神御門置親事官五百人,立戟二十四。累朝文武執政官、武臣節度使以上並圖形於兩廡。凡執政官除拜,赴官恭謝。其後南郊先詣宮行薦享禮,並如太廟儀。



 元祐元年,太常寺言:「季秋有事於明堂,其
 朝享景靈宮、親享太廟,當用三年不祭之禮,遣大臣攝事。」禮部言:「景靈宮天興殿,用天地之禮,即非廟享,於典禮無違。」詔明堂前二日朝享景靈宮天興殿。明年,奉安神宗神御於景靈宮,如十一殿奉安之禮。舊制,車駕上元節以十一日詣興國寺、啟聖院,朝謁太祖、太宗、神宗神御,下元節詣景靈宮朝拜天興殿,朝謁真宗、仁宗、英宗神御。至是詔分每歲四孟月拜謁之所,自孟秋始,其不當親獻,則遣官分詣。初詣天興殿、保寧閣、天元殿、太
 始殿,次詣皇武殿、儷極殿、大定殿、輝德殿,次詣熙文殿、衍慶殿、美成殿,次詣治隆殿、宣光殿,宣光後改曰顯承,徽宗又改大明殿。



 仍自來年孟春為始。皇太后崩,三省請奉安神御於治隆殿,以遵元祐初詔。復以御史劉極之言,特建原廟,廟成,名神御殿曰徽音,山殿曰寧真。



 紹聖二年,奉安神宗神御於顯承殿。元豐中,每歲四孟月,天子遍詣諸殿朝獻。元祐初,議者請以四孟分獻,一歲而遍,至是復用舊儀。詔自今四孟朝獻分二日,先日詣天興殿、保寧閣、天
 元、太始、皇武、儷極、大定、德輝諸殿,次日詣熙文、衍慶、美成、繼仁、治隆、徽音、顯承七殿。三年十月,帝詣天興諸殿朝獻。翼日,大雨,詔差已致齋官分獻熙文七殿,自是雨雪用為例云。



 徽宗即位,宰臣請特建景靈西宮,奉安神宗於顯承殿,為館御之首,昭示萬世尊異之意。建哲宗神御殿於西,以東偏為齋殿,乃給度僧牒、紫衣牒千道為營造費,戶牖工巧之物並置於荊湖北路。已而右正言陳瓘言五不可,且論蔡京矯誣。不從。



 建中靖國元年,
 詔建欽聖憲肅皇后、欽慈皇后神御殿於大明殿北,名曰柔明。尋改欽儀,又改坤元。



 又名哲宗神御殿曰觀成。尋改重光。



 詔自今景靈宮並分三日朝獻。



 崇寧三年,奉安欽成皇后神御坤元殿欽聖憲肅皇后之次,欽慈皇后又次之。



 政和三年,奉安哲宗神御於重光殿。昭懷皇后神御殿成,詔名正殿曰柔儀,山殿曰靈娭。於是兩宮合為前殿九,後殿八,山殿十六,閣一,鐘樓一,碑樓四,經閣一,齋殿三,神廚二,道院一,及齋宮廊廡,共為二千三百二十區。



 初,東
 京以來奉先之制,太廟以奉神主,歲五享,宗室諸王行事;朔祭而月薦新,則太常卿行事。景靈宮以奉塑像,歲四孟皇帝親享,帝後大忌,則宰相率百官行香,後妃繼之。遇郊祀、明堂大禮,則先期二日,親詣景靈宮行朝享禮。



 紹興十三年二月,臣僚言:「竊見元豐五年,神宗始廣景靈宮以奉祖宗衣冠之游,即漢之原廟也。自艱難以來,庶事草創,始建宗廟,而原廟神游猶寄永嘉。乃者權時之宜,四孟薦獻,旋即便朝設位以享,未副廣孝之意,
 乞命有司擇爽塏之地,仿景靈宮舊規,隨宜建置。俟告成有日,迎還晬容,奉安新廟,庶幾四孟躬行獻禮,用副罔極之恩。」從之。初築三殿,聖祖居前,宣祖至祖宗諸帝居中殿,元天大聖後與祖宗諸後居後。掌宮內侍七人,道士十人,吏卒二百七十六人。上元結燈樓,寒食設秋千,七夕設摩□羅。簾幕歲時一易,歲用酌獻二百四十羊。凡帝後忌辰,用道、釋作法事。十八年,增建道院,初本劉光世賜第,後以韓世忠第增築之。天興殿九楹,中殿
 七楹,後殿十有七楹,齋殿、進食殿皆備焉。



 神御殿,古原廟也,以奉安先朝之御容。宣祖、昭憲皇后於資福寺慶基殿。太祖神御之殿七:太平興國寺開元殿、景靈宮、應天禪院西院、南京鴻慶宮、永安縣會聖宮、揚州建隆寺章武殿、滁州大慶寺端命殿。太宗神御之殿七:啟聖禪院、壽寧堂、景福殿、鳳翔上清太平宮、並州崇聖寺統平殿及西院、鴻慶宮、會聖宮。真宗神御之殿十有四:景靈宮奉真殿、玉清昭應宮安聖殿、洪福院、壽
 寧堂、福聖殿、崇先觀永崇殿、萬壽觀延聖殿、澶州信武殿、西京崇福宮保祥殿、華州雲臺觀集真殿及西院、鴻慶宮、會聖宮、鳳翔太平宮。仁宗、英宗、神宗、哲宗四朝神御於景靈宮、應天院,章獻明肅皇后於慈孝寺彰德殿,章懿皇后於景靈宮廣孝殿,明德、章穆二后於普安院重徽殿,章惠太后於萬壽觀廣慶殿。



 景德四年,奉安太祖御容應天禪院,以宰臣向敏中為奉安聖容禮儀使,權安於文德殿。百官班列,帝行酌獻禮,鹵簿導引,升彩
 輿進發,帝辭於正陽門外,百官辭於瓊林苑門外。遣官奏告昌陵畢,群臣稱賀。



 皇祐中,以滁州通判王靖請,滁、並、澶三州建殿奉神御,乃宣諭曰:「太祖擒皇甫暉於滁州,是受命之端也,大慶寺殿名曰端命,以奉太祖。太宗取劉繼元於並州,是太平之統也,即崇聖寺殿名曰統平,以奉太宗。真宗歸契丹於澶州,是偃武之信也,即舊寺殿名曰信武,以奉真宗。」既而統平殿災,諫官範鎮言:「並州素無火災,自建神御殿未幾而輒焚,天意若曰祖
 宗御容非郡國所宜奉安者。近聞下並州復加崇建,是徒事土木,重困民力,非所以答天意也。自並州平七十七年,故城父老不入新城,宜寬其賦輸,緩其徭役,以除其患,使河東之民不忘太宗之德,則陛下孝思,豈特建一神御殿比哉?」先是,睦親、廣親二宅並建神御殿,翰林學士歐陽修言神御非人臣私家之禮。下兩制、臺諫、禮官議,以為「漢用《春秋》之義,罷郡國廟。今睦親宅、廣親宅所建神御殿,不合典禮,宜悉罷。」詔以廣親宅置已久,唯
 罷修睦親宅。



 熙寧二年,奉安英宗御容於景靈宮,帝親行酌獻,仍詔歲以十月望朝享,有期以上喪或災異,則命輔臣攝事。知大宗正丞事李德芻言:「禮法:諸侯不得祖天子,公廟不設於私家。今宗室邸第並有帝後神御,非所以明尊卑崇正統也,望一切廢罷。」下禮官詳定,請如所奏。詔諸宗室宮院祖宗神御迎藏天章閣。自是,臣庶之家凡有御容,悉取藏禁中。



 元豐五年,作景靈宮十一殿,而在京宮觀寺院神御,皆迎入禁中,所存惟萬壽
 觀延聖、廣愛、寧華三殿而已。



 宣和元年,禮部奏:「太常寺參酌立到諸州府有祖宗御容所在朔日諸節序降至御封香表及下降香表行禮儀注:



 朔日諸節序奉香表行禮儀注。齋戒,朝拜前一日,朝拜官及讀表文官早赴齋所,俟禮備,禮生引讀表文官、繼香表官集朝拜官聽,執事者以香表呈視。禮生請讀表文官稍前習讀表,或密詞即讀封題,訖,禮生贊復位。次以御封香、禮饌等呈視訖,各復齋所。朝拜官用長吏,闕,以次官充,讀表文亦
 以次官充,執事者以有服色者充。有司設香案、時果、牙盤食神御前,又設奠醪茗之器於香案前之左,置御封香表案上;設朝拜官位於殿下,西向,讀表文官位於殿之南,北向,陪位官位於其後;設焚表文位於殿庭東,南向。朝拜日,質明前,香火官先詣殿下,北向拜訖,升殿,東向侍立。有司陳設訖,禮生先引陪位官入就位,北向,次引讀表文官入就位,次引朝拜官就位,西向立定。禮生贊有司謹具,請行事。禮生贊再拜,拜訖,引讀表文官先
 升殿,於香案之右東向立,次引朝拜官詣香案前,贊搢笏、上香、奠酒茗,拜、興,少立。禮生贊搢笏、跪、讀表文,或密詞即讀封題,執笏興,降復位。朝拜官再拜,降復位。禮生贊再拜訖,引朝拜官、讀表文官詣焚表文位南向立,焚訖,退。



 一遇旦、望諸節序下降香表薦獻行禮儀注。一如上儀。惟禮生引獻官上香訖,跪,執事者以所薦之物授薦獻官,受獻訖,復授執事者,置於神御前,興、拜、退一如上儀。」



 詔頒行之。



 東京神御殿在宮中,舊號欽先孝思殿。
 建炎二年閏四月,詔迎溫州神禦赴闕。先是,神御於溫州開元寺暫行奉安,章聖皇帝與後像皆以金鑄,置外方弗便,因愀然謂宰輔曰:「朕播遷至此,不能以時薦享,祖宗神御越在海隅,念之坐不安席。」故有是命。三年二月,上覽禁中神御薦享禮物,謂宰臣曰:「朕自省閱神御,每位各用羊胃一,須二十五羊。祖宗仁厚,豈欲多害物命?謹以別味代之,在天之靈亦必歆享。」呂頤浩曰:「陛下寅奉宗廟,罔不盡禮,而又仁愛及物,天下幸甚。」



 紹興十
 五年秋,復營建神御殿於崇政殿之東,朔望節序、帝後生辰,皇帝皆親酌獻行香,用家人禮。其殿名:徽宗曰承元,欽宗曰端慶,高宗曰皇德,孝宗曰系隆,光宗曰美明,寧宗曰垂光,理宗曰章熙,度宗曰昭光。



 功臣配享。真宗咸平二年,始詔以太師、贈尚書令、韓王趙普配享太祖廟庭。繼以翰林承旨宋白等議,又以故樞密使、贈中書令、濟陽郡王曹彬配享太祖,以司空贈太尉中書令薛居正、忠武軍節度使贈中書令潘美、尚
 書右僕射贈侍中石熙載配享太宗廟庭,仍奏告本室,禘祫皆配之。祀日,有司先事設幄次,布褥位於廟庭東門內道南,當所配室西向,設位板,方七寸,厚一寸半,籩、豆各一,簠、簋、俎各一。知廟卿奠爵,再拜。



 乾興元年,詔從翰林、禮官參議,以右僕射贈太尉中書令李沆、贈太師尚書令王旦、忠武軍節度使贈中書令李繼隆配享真宗。



 嘉祐八年,詔以尚書右僕射贈尚書令王曾、太尉贈尚書令呂夷簡、彰武軍節度使贈侍中曹瑋配享仁宗。



 熙寧八年,詔以司徒兼侍中贈尚書令韓琦配享英宗;元豐元年,又以贈太師中書令曾公亮配焉。熙寧末,嘗詔太常禮院講求親祠太廟不及功臣禮例。至是,禘祫外,親享太廟並以功臣與。又從太常禮院請,配享功臣以見贈官書板位。



 元祐初,從吏部尚書孫永等議,以故司徒、贈太尉富弼配享神宗;紹聖初,又以守司空、贈太傅王安石配。三年,罷富弼配,謂弼得罪於先帝也。



 崇寧元年,詔以觀文殿大學士、贈太師蔡確配享哲宗。



 《五
 禮新儀》,配享功臣之位,設於殿庭之次:趙普、曹彬位於橫街之南道西,東向,第一次,薛居正、石熙載、潘美位於第二次,李沆、王旦、李繼隆位於第三次,俱北上;王曾、呂夷簡、曹瑋位於橫街之南道東,西向,第一次,韓琦、曾公亮位於第二次,王安石位於第三次,蔡確位於第四次,俱北上。惟冬享、祫享遍設祭位。



 迨建炎初,詔奪蔡確所贈太師、汝南郡王,追貶武泰軍節度副使,更以左僕射、贈太師司馬光配享哲宗。既又罷王安石,復以富弼配享
 神宗。



 紹興八年,以尚書左僕射、贈太師韓忠彥配享徽宗。十八年二月,監登聞鼓院徐璉言:「國家原廟佐命配享,當時輔弼勛勞之臣繪像廟庭,以示不忘,累朝不過一十餘人。今之臣僚與其家之子孫必有存其繪像者,望詔有司尋訪,復摹於景靈宮庭之壁,非獨假寵諸臣之子孫,所以增重祖宗之德業,以為臣子勸。」遂下諸路轉運司,委所管州軍尋訪各家,韓王趙普、周王曹彬、太師薛居正、石熙載、鄭王潘美、太師李沆、王旦、李繼隆、王
 曾、呂夷簡、侍中曹瑋、司徒韓琦、太師曾公亮、富弼、司馬光、韓忠彥,各令摹寫貌像投納,繪於景靈宮之壁。



 乾道五年九月,太常少卿林慄等言:「欽宗皇帝廟庭尚虛配享,當時遭值艱難,淪胥莫救,罕可稱述,而以身徇國,名節暴著,不無其人。雖生前官品不應配享之科,事變非常,難拘定制,乞特詔集議。」吏部尚書汪應辰奏:「當時死事之臣,皆有次第褒贈。若今配享欽廟,典故所無,如創行之,又當訪究本末,差次輕重,有所取舍,尤不可輕易。
 竊謂配享功臣,若依唐制,各廟既無其人,則當缺之。」乃罷集議,欽宗一廟遂無配享。



 淳熙中,高宗祔廟,翰林學士洪邁言:「配食功臣,先期議定。臣兩蒙宣諭,欲用文武臣各兩人,文臣故宰相贈太師秦國公謚忠穆呂頤浩、特進觀文殿大學士謚忠簡趙鼎,武臣太師蘄王謚忠武韓世忠、太師魯王謚忠烈張俊。此四人皆一時名將相,合於天下公論。」議者皆以為宜,遂從之。秘書少監楊萬里獨謂丞相張浚不得配食為非,爭之不得,因去位焉。



 紹熙五年十二月,以左丞相、贈太師、魯國公陳康伯配享孝宗廟庭。



 嘉熙元年正月,以右丞相、贈太師葛邲配享光宗廟庭。



 嘉定十四年八月,追封右丞相史浩為越王,改謚忠定,配享孝宗廟庭。



 端平二年八月,以太師趙汝愚配享寧宗廟庭。



 初,仁宗天聖中郊祀,詔錄故相李昉、宋琪、呂端、張齊賢、畢士安、王旦,執政李至、王沔、溫仲舒及陳洪進等子孫以官。元豐中,詔:景靈宮繪像舊臣推恩本支下兩房以上,取不食祿者,均有無,取齒長者;
 若子孫亦繪像,本房不食祿,更不取別房。紹聖初,林希請稽考慶歷以後未經編次臣僚,其子孫應錄用者以次編定。尋詔:「趙普社稷殊勛,其諸孤有無食祿者,各官其一子,以長幼為序,毋過三人。」崇寧初,詔:「哲宗繪像文武臣僚,並與子若孫一人初品官,若子孫眾多,無過家一人。」又錄藝祖功臣呂餘慶族孫偉及司徒富弼孫直柔、直道以官,使奉其祀。靖康初,臣僚言:「司馬光之後再絕,復立族子稹,稹亦卒。今雖有子,而光遺表恩澤已五
 十年,不可復奏,請許移奏見存曾孫,使之世祿。」從之。



 群臣家廟,本於周制,適士以上祭於廟,庶士以下祭於寢。唐原周制,崇尚私廟。五季之亂,禮文大壞,士大夫無襲爵,故不建廟,而四時寓祭室屋。慶歷元年,南郊赦書,應中外文武官並許依舊式立家廟。已而宋庠又以為言,乃下兩制、禮官詳定其制度:「官正一品平章事以上立四廟;樞密使、知樞密院事、參知政事、樞密副使、同知樞密院事、簽書院事,見任、前任同,宣徽使、尚書、節
 度使、東宮少保以上,皆立三廟;餘官祭於寢。凡得立廟者,許適子襲爵以主祭。其襲爵世降一等,死即不得作主祔廟,別祭於寢。自當立廟者,即祔其主,其子孫承代,不計廟祭、寢祭,並以世數疏數遷祧;始得立廟者不祧,以比始封。有不祧者,通祭四廟、五廟。廟因眾子立而適長子在,則祭以適長子主之;嫡長子死,即不傳其子,而傳立廟者之長。凡立廟,聽於京師或所居州縣。其在京師者,不得於里城及南郊禦路之側。」仍別議襲爵之制,既以
 有廟者之子孫或官微不可以承祭,而朝遷又難盡推襲爵之恩,事竟不行。



 大觀二年,議禮局言:「所有臣庶祭禮,請參酌古今,討論條上,斷自聖衷。」於是議禮局議:「執政以上祭四廟,餘通祭三廟。」「古無祭四世者,又侍從官以至士庶,通祭三世,無等差多寡之別,豈禮意乎?古者天子七世,今太廟已增為九室,則執政視古諸侯,以事五世,不為過矣。先王制禮,以齊有萬不同之情,賤者不得僭,貴者不得逾。故事二世者,雖有孝思追遠之心,無
 得而越,事五世者,亦當跂以及焉。今恐奪人之恩,而使通祭三世,徇流俗之情,非先王制禮等差之義。可文臣執政官、武臣節度使以上祭五世,文武升朝官祭三世,餘祭二世。」「應有私第者,立廟於門內之左,如狹隘,聽於私第之側。力所不及,仍許隨宜。」又詔:「古者寢不逾廟,禮之廢失久矣。士庶堂寢,逾度僭禮,有七楹、九楹者,若一旦使就五世、三世之數,則當徹毀居宇,以應禮制,豈得為易行?可自今立廟,其間數視所祭世數,寢間數不得逾廟。事二世者,寢聽用二間。」議禮局言:「《禮記·王制》:『諸侯五廟,二昭二穆,與太祖之廟而五。』所謂『太』者,蓋始封之祖,不必五世,又非臣下所可通稱。今高祖以上一祖未有名稱,欲乞稱五世祖。其家廟祭器:正一品,每室籩、豆各十有二,簠、簋各四,壺尊、罍、鉶鼎、俎、篚各二,尊、罍加勺、冪各一,爵各一,諸室共享胙俎、罍洗一。從一品籩、豆、簠、簋降殺以兩。正二品籩、豆各八,簠、簋各二。餘皆如正一品之數。」詔禮制局制造,仍
 取旨以給賜之。



 紹興十六年二月癸丑,詔太師、左僕射、魏國公秦檜合建家廟,命臨安守臣營之。太常請建於其私第中門之左,一常五室,五世祖居中,東二昭,西二穆。堂飾以黝堊。神板長一尺,博四寸五分,厚五寸八分,大書某官某大夫之神坐,貯以帛囊,藏以漆函。歲四享用孟月柔日行之,具三獻。有司言時享用常器常饌,帝仿政和故事,命制祭器賜之。其後,太傅昭慶節度平樂郡王韋淵、太尉保慶節度吳益、少傅寧遠節度楊存中並請建家廟,賜以祭器。



 隆興二年四月庚辰,少師、四川宣撫使吳璘請用存中例,從之。



 乾道八年九月,詔有司賜少保、武安節度、四川宣撫使虞允文家廟祭器如故事。



 淳熙五年七月,戶部尚書韓彥古請以賜第進父世忠家廟如存中。十二月,少傅、保寧節度衛國公史浩請建家廟,量賜祭器。



 嘉泰元年,太傅、永興節度、平原郡王韓侂冑奏:「曾祖琦效忠先朝,奕世侑食,家廟猶闕,請下禮官考其制建之。」二年,循忠烈王張俊,開禧三年,鄜武僖王劉光世子孫相繼有請,皆從之。



 嘉定十四年八月,詔右丞相史彌遠賜第,遵淳熙故事賜家廟,命臨安守臣營之。禮官討論祭器,並如侂冑之制。彌遠請並生母齊國夫人周氏及祔妻魯國夫人潘氏於生母別廟,皆下有司賜器。



 景定三年,詔丞相賈似道賜家廟,命臨安守、漕營度,禮官討論賜祭器,並如儀。



\end{pinyinscope}