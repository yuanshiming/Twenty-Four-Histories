\article{志第十四 五行一上}

\begin{pinyinscope}

 水上



 天以陰陽五行化生萬物,盈天地之間,無非五行之妙用。人得陰陽五行之氣以為形,形生神知而五性動,五性動而萬事出,萬事出而休咎生。和氣致祥,乖氣致異,
 莫不於五行見之。《中庸》:「至誠之道,可以前知。國家將興,必有禎祥;國家將亡,必有妖孽。見乎蓍龜,動乎四體。禍福將至,善必先知之,不善必先知之。」人之一身,動作威儀,猶見休咎,人君以天地萬物為體,禎祥妖孽之致,豈無所本乎?故由漢以來,作史者皆志五行,所以示人君之戒深矣。自宋儒周惇頤《太極圖說》行世,儒者之言五行,原於理而究於誠。其於《洪範》五行五事之學,雖非所取,然班固、範曄志五行已推本之,及歐陽修《唐志》,亦採
 其說,且於庶徵惟述災眚,而休祥闕焉,亦豈無所見歟?



 舊史自太祖而嘉禾、瑞麥、甘露、醴泉、芝草之屬,不絕於書,意者諸福畢至,在治世為宜。祥符、宣和之代,人君方務以符瑞文飾一時,而丁謂、蔡京之奸,相與傅會而為欺,其應果安在哉?高宗渡南,心知其非,故《宋史》自建炎而後,郡縣絕無以符瑞聞者,而水旱、札瘥一切咎徵,前史所罕見,皆屢書而無隱。於是六主百五十年,兢兢自保,足以圖存。



 《易·震》之《彖》曰:「震來虩虩,恐致福也。」人君致
 福之道,有大於恐懼修省者乎?昔禹致群臣於會稽,黃龍負舟,而執玉帛者萬國。孔甲好鬼神,二龍降自天,而諸侯相繼畔夏。桑穀共生於朝,雉升鼎耳而雊,而大戊、武丁復修成湯之政。穆王得白狼、白鹿,而文、武之業衰焉。徐偃得朱弓矢,宋愍有雀生鸇,二國以霸,亦以之亡。大概征之休咎,猶卦之吉兇,占者有德以勝之則兇可為吉,無德以當之則吉乃為兇。故德足勝妖,則妖不足慮;匪德致瑞,則物之反常者皆足為妖。妖不自作,人實
 興之哉!今因先後史氏所紀休咎之徵,匯而輯之,作《五行志》。



 潤下,水之性也。水失其性,則為災沴。舊說以恆寒、鼓妖、魚孽、豕禍、雷電、霜雪、雨雹、黑眚、黑祥皆屬之水,今從之。醴泉、河清雖為瑞應,茍非其時,未必不為異,故雜附於編。他如甘露、嘉禾、芝草一切祥瑞之物,見於後者,因其事而考其時,則休咎自見,故亦各以類相從雲。



 建隆元年十月,棣州河決,壞厭次、商河二縣居民廬舍、
 田疇。二年,宋州汴河溢。孟州壞堤。襄州漢水漲溢數丈。四年八月,齊州河決。九月,徐州水損田。



 乾德二年四月,廣陵、揚子等縣潮水害民田。七月,泰山水,壞民廬舍數百區,牛畜死者甚眾。三年二月,全州大雨水。七月,蘄州大雨水,壞民廬舍。開封府河決,溢陽武。河中府、孟州並河水漲,孟州壞中水單軍營、民舍數百區。河壞堤岸石,又溢於鄆州,壞民田。泰州潮水損鹽城縣民田。淄州、濟州並河溢,害鄒平、高苑縣民田。四年,東阿縣河溢,損民田。
 觀城縣河決,壞居民廬舍,注大名。又靈河縣堤壞,水東注衛南縣境及南華縣城。七月,滎澤縣河南北堤壞。八月,宿州汴水溢,壞堤。淄州清河水溢,壞高苑縣城,溺數百家及鄒平縣田舍。泗州淮溢。衡州大雨水月餘。五年,衛州河溢,毀州城,沒溺者甚眾。



 開寶元年六月,州府二十三大雨水,江河泛溢,壞民田、廬舍。七月,泰州潮水害稼。八月,集州霖雨河漲,壞民廬舍及城壁、公署。二年七月,下邑縣河決。是歲,青、蔡、宿、淄、宋諸州水,真定、澶、滑、博、
 洺、齊、穎、蔡、陳、亳、宿、許州水,害秋苗。三年,鄭、澶、鄆、淄、濟、虢、蔡、解、徐、岳州水災,害民田。四年六月,汴水決宋州穀熟縣濟陽鎮。又鄆州河及汶、清河皆溢,注東阿縣及陳空鎮,壞倉庫、民舍。鄭州河決原武縣。蔡州淮及白露、舒、汝、廬、穎五水並漲,壞廬舍、民田。七月,青、齊州水傷田。五年,河決澶州濮陽,絳、和、廬、壽諸州大水。六月,河又決開封府陽武縣之小劉村。宋州、鄭州並汴水決。忠州江水漲二百尺。六年,鄆州河決楊劉口。懷州河決獲嘉縣。穎州
 淮、卑水溢,渰民舍、田疇甚眾。七月,歷亭縣御河決。單州、濮州並大雨水,壞州廨、倉庫、軍營、民舍。是秋,大名府、宋、亳、淄、青、汝、澶、滑諸州並水傷田。七年四月,衛、亳州水。泗州淮水暴漲入城,壞民舍五百家。安陽縣河漲,壞居民廬舍百區。八年五月,京師大雨水。濮州河決郭龍村。六月,澶州河決頓丘縣。沂州大雨,入城,壞居舍、田苗。九年三月,京師大雨水。淄州水害田。



 太平興國二年六月,孟州河溢,壞溫縣堤七十餘步,鄭州壞滎澤縣寧王村
 堤三十餘步,又漲於澶州,壞英公村堤三十步。開封府汴水溢,壞大寧堤,浸害民田。忠州江漲二十五丈。興州江漲,毀棧道四百餘間。管城縣焦肇水暴漲,逾京水。濮州大水,害民田凡五千七百四十三頃。穎州穎水漲,壞城門、軍營、民舍。七月,復州蜀、漢江漲,壞城及民田、廬舍。集州江漲,泛嘉川縣。三年五月,懷州河決獲嘉縣北注。又汴水決宋州寧陵縣境。六月,泗州淮漲入南城,汴水又漲一丈,塞州北門。十月,滑州靈河已塞復決。四年三
 月,河南府洛水漲七尺,壞民舍。泰州雨水害稼。宋州河決宋城縣。衛州河決汲縣,壞新場堤。八月,梓州江漲,壞閣道、營舍。九月,澶州河漲。鄆州清、汶二水漲,壞東阿縣民田。復州沔陽縣湖皛漲,壞民舍、田稼。五年五月,穎州穎水溢,壞堤及民舍。徐州白溝河溢入州城。七月,復州江水漲,毀民舍,堤塘皆壞。六年,河中府河漲,陷連堤,溢入城,壞軍營七所、民舍百餘區。鄜、延、寧州並三河水漲,溢入州城:鄜州壞軍營,建武指揮使李海及老幼六十三
 人溺死;延州壞倉庫、軍民廬舍千六百區;寧州壞州城五百餘步,諸軍營、軍民舍五百二十區。七年三月,京兆府渭水漲,壞浮梁,溺死五十四人。四月,耀、密、博、衛、常、潤諸州水害稼。六月,均州溳水、均水、漢江並漲,壞民舍,人畜死者甚眾。又河決臨邑縣,漢陽軍江水漲五丈。七月,大名府御河漲,壞範濟口。南劍州江水漲,壞居民舍一百四十餘區。京兆府咸陽渭水漲,壞浮梁,工人溺死五十四人。九月,梧州江水漲三丈,入城,壞倉庫及民舍。十月,
 河決懷州武陟縣,害民田。八年五月,河大決滑州房村,徑澶、濮、曹、濟諸州,浸民田,壞居民廬舍,東南流入淮。六月,陜州河漲,壞浮梁;又永定澗水漲,壞民舍、軍營千餘區。河南府澍雨,洛水漲五丈餘,壞鞏縣官署、軍營、民舍殆盡。穀、洛、伊、瀍四水暴漲,壞京城官署、軍營、寺觀、祠廟、民舍萬餘區,溺死者以萬計。又壞河清縣豐饒務倉庫、軍營、民舍百餘區。雄州易水漲,壞民廬舍。鄜州河水漲,溢入城,壞官寺、民舍四百餘區。荊門軍長林縣山水暴
 漲,壞民舍五十一區,溺死五十六人。八月,徐州清河漲丈七尺,溢出,塞州三面門以御之。九月,宿州睢水漲,泛民舍六十里。是夏及秋。開封、浚儀、酸棗、陽武、封丘、長垣、中牟、尉氏、襄邑、雍丘等縣河水害民田。九年七月,嘉州江水暴漲,壞官署、民舍,溺者千餘人。八月,延州南北兩河漲,溢入東西兩城,壞官寺、民舍。淄州霖雨,孝婦河漲溢,壞官寺、民田。孟州河漲,壞浮梁,損民田。雅州江水漲九丈,壞民廬舍。新州江漲,入南砦,壞軍營。



 雍熙二年七
 月,朗江溢,害稼。八月,瀛、莫州大水,損民田。三年六月,壽州大水。



 端拱元年二月,博州水害民田。五月,英州江水漲五丈,壞民田及廬舍數百區。七月,磁州漳、滏二水漲。



 淳化元年六月,吉州大雨,江漲,漂壞民田、廬舍。黃梅縣堀口湖水漲,壞民田、廬舍皆盡,江水漲二丈八尺。洪州漲壞州城三十堵、民廬舍二千餘區,漂二千餘戶。孟州河漲。二年四月,京兆府河漲,陜州河漲,壞大堤及五龍祠。六月乙酉,汴水溢於浚儀縣,壞連堤,浸民田。上親臨視,督
 衛士塞之。辛卯,又決於宋城縣。博州大霖雨,河漲,壞民廬舍八百七十區。亳州河溢,東流泛民田、廬舍。七月,齊州明水漲,壞黎濟砦城百餘堵。許州沙河溢。雄州塘水溢,害民田殆盡。嘉州江漲,溢入州城,毀民舍。復州蜀、漢二江水漲,壞民田、廬舍。泗州招信縣大雨,山河漲,漂浸民田、廬舍,死者二十一人。八月,藤州江水漲十餘丈,入州城,壞官署、民田。九月,邛州蒲江等縣山水暴漲,壞民舍七十區,死者七十九人。是秋,荊湖北路江水注溢,浸
 田畝甚眾。三年七月,河南府洛水漲,壞七里、鎮國二橋;又山水暴漲,壞豐饒務官舍、民廬,死者二百四十人。十月,上津縣大雨,河水溢,壞民舍,溺者三十七人。四年六月,隴城縣大雨,牛頭河漲二十丈,沒溺居人、廬舍。九月,澶州河漲,沖陷北城,壞居人廬舍、官署、倉庫殆盡,民溺死者甚眾。梓州玄武縣涪河漲二丈五尺,壅下流入州城,壞官私廬舍萬餘區,溺死者甚眾。十月。澶州河決,水西北流入御河,浸大名府城,知府趙昌言壅城門御之。



 至道元年四月甲辰,京師大雨、雷電,道上水數尺。五月,虔州江水漲二丈九尺,壞城流入深八尺,毀城門。二年六月,河南瀍、澗、洛三水漲,壞鎮國橋。七月,建州溪水漲,溢入州城,壞倉庫、民舍萬餘區。鄆州河漲,壞連堤四處。宋州汴河決穀熟縣。閏七月,陜州河漲。是月,廣南諸州並大雨水。



 咸平元年七月,侍禁、閣門祗候王壽永使彭州回,至鳳翔府境,山水暴漲,家屬八人溺死。齊州清、黃河泛溢,壞田廬。二年十月,漳州山水泛溢,壞民舍千餘
 區,民黃拏等十家溺死。三年三月,梓州江水漲,壞民田。五月,河決鄆州王陵埽。七月,洋州漢水溢,民有溺死者。四年七月,同州洿谷水溢夏陽縣,溺死者數十人。五年二月,雄、霸、瀛、莫、深、滄、諸州、乾寧軍水,壞民田。六月,京師大雨,漂壞廬舍,民有壓死者。積潦浸道路,自朱雀門東抵宣化門尤甚,皆注惠民河,河復漲,溢軍營。



 景德元年九月,宋州汴水決,浸民田,壞廬舍。河決澶州橫隴埽。二年六月,寧州山水泛溢,壞民舍、軍營,多溺死者。三年七
 月,應天府汴水決,南注亳州,合浪宕渠東入於淮。八月,青州山水壞石橋。四年六月,鄭州索水漲,高四丈許,漂滎陽縣居民四十二戶,有溺死者。鄧州江水暴漲。南劍州山水泛溢,漂溺居人。七月,河溢澶州,壞王八埽。八月,橫州江漲,壞營舍。



 大中祥符元年六月,開封府尉氏縣惠民河決。二年七月,徐、濟、青、淄大水。八月,鳳州大水,漂溺民居。十月,京畿惠民河決,壞民田。三年六月,吉州、臨江軍並江水泛溢,害民田。九月,河決河中府白浮梁村。
 四年七月,洪、江、筠、袁州江漲,害民田,壞州城。八月,河決通利軍,大名府御河溢,合流壞府城,害田,人多溺死。九月,河溢於孟州溫縣。蘇州吳江泛溢,壞廬舍。十一月,楚、泰州潮水害田,人多溺者。五年正月,河決棣州聶家口。七月,慶州淮安鎮山水暴漲,漂溺居民。六年六月,保安軍積雨河溢,浸城壘,壞廬舍,判官趙震溺死,又兵民溺死凡六百五十人。七年六月,泗州水害民田。河南府洛水漲。秦州定西砦有溺死者。八月,河決澶州。十月,濱州河
 溢於安定鎮。八年七月,坊州大雨河溢,民有溺死者。九年六月,秦州獨孤谷水壞長道縣鹽官鎮城橋及官廨、民舍二百九十五區,溺死六十七人。七月,延州洎定平、安遠、塞門、栲栳四砦山水泛溢,壞堤、城。九月,雄、霸州界河泛溢。利州水漂棧閣萬二千八百間。



 天禧三年六月,河決滑州城西南,漂沒公私廬舍,死者甚眾,歷澶州、濮、鄆、濟、單至徐州,與清河合,浸城壁,不沒者四板。明年既塞。六月,復決於西北隅。



 乾興元年正月,秀州水災,民多艱食。
 十月己酉夜,滄州鹽山、無棣二縣海潮溢,壞公私廬舍,溺死者甚眾。是歲,京東、淮南路水災。



 天聖初,徐州仍歲水災。三年十一月辛卯,襄州漢水壞民田。四年六月丁亥,劍州、邵武軍大水,壞官私廬舍七千九百餘區,溺死者百五十餘人。是月,河南府、鄭州大水。十月乙酉,京山縣山水暴漲,漂死者眾,縣令唐用之溺焉。是歲,汴水溢,決陳留堤,又決京城西賈陂入護龍河,以殺其勢。五年三月,襄、穎、許、汝等州水。七月辛丑,泰州鹽官鎮大水,民
 多溺死。六年七月壬子,江寧府、揚、真、潤三州江水溢,壞官私廬舍。是月,雄、霸州大水。八月甲戌,臨潼縣山水暴漲,民溺死者甚眾。是月,河決楚王埽。七年六月,河北大水,壞澶州浮梁。



 明道元年四月壬子,大名府冠氏等八縣水浸民田。



 景祐元年閏六月甲子,泗州淮、汴溢。七月,澶州河決橫隴埽。八月庚午,洪州分寧縣山水暴發,漂溺居民二百餘家,死者三百七十餘口。三年六月,虔、吉諸州久雨,江溢,壞城廬,人多溺死。四年六月乙亥,杭州
 大風雨,江潮溢岸,高六尺,壞堤千餘丈。八月甲戌,越州大水,漂溺居民。



 寶元元年,建州自正月雨,至四月不止,溪水大漲,入州城,壞民廬舍,溺死者甚眾。



 康定元年九月甲寅,滑州大河泛溢,壞民廬舍。



 慶歷元年三月,汴流不通。八年六月乙亥,河決澶州商胡埽。是月,恆雨。七月癸丑,衛州大雨水,諸軍走避,數日絕食。是歲,河北大水。



 皇祐元年二月甲戌,河北黃、御二河決,並注于乾寧軍。河朔頻年水災。二年,鎮定復大水,並邊尤被其害。三年
 七月辛酉,河決館陶縣郭固口。八月,汴河絕流。四年八月,鄜州大水,壞軍民廬舍。



 嘉祐二年六月,開封府界及京東西、河北水潦害民田。自五月大雨不止,水冒安上門,門關折,壞官私廬舍數萬區,城中系□伐渡人。七月,京東西、荊湖北路水災。淮水自夏秋暴漲,環浸泗州城。是歲,諸路江河溢決,河北尤甚,民多流亡。三年七月,京、索、廣濟河溢,浸民田。五年七月,蘇、湖二州水災。六年七月乙酉,泗州淮水溢。七年六月,代州大雨,山水暴入城。七
 月,竇州山水壞城。河決北京第五埽。



 治平元年,慶、許、蔡、穎、唐、泗、濠、楚、廬、壽、杭、宣、鄂、洪、施、渝州、光化軍水。九月,陳州水災。二年八月庚寅,京師大雨,地上湧水,壞官私廬舍,漂人民畜產不可勝數。是日,御崇政殿,宰相而下朝參者十數人而已。詔開西華門以洩宮中積水,水奔激,殿侍班屋皆摧沒,人畜多溺死,官為葬祭其無主者千五百八十人。



 熙寧元年秋,霸州山水漲溢,保定軍大水,害稼,壞官私廬舍、城壁,漂溺居民。河決恩、冀州,漂溺居民。二
 年八月,河決滄州饒安,漂溺居民,移縣治於張為村。泉州大風雨,水與潮相沖泛溢。損田稼,漂官私廬舍。四年八月,金州大水,毀城,壞官私廬舍。七年六月,熙州大雨,洮河泛溢。八年四月,潭、衡、邵、道諸州江水溢,壞官私廬舍。九年七月,太原府汾河夏秋霖雨,水大漲。十月,海陽、潮陽二縣海潮溢,壞廬舍,溺居民。十年七月,河決曹村下埽,澶淵絕流,河南徙,又東匯於梁山、張澤濼,凡壞郡縣四十五,官亭、民舍數萬,田三十萬頃。洺州漳河決,注
 城。大雨水,二丈河、陽河水湍漲,壞南倉,溺居民。滄、衛霖雨不止,河濼暴漲,敗廬舍,損田苗。



 元豐元年,章丘河水溢,壞公私廬舍、城壁,漂溺民居。舒州山水暴漲,浸官私廬舍,損田稼,溺居民。四年四月,澶州臨河縣小吳河溢北流,漂溺居民。五月,淮水泛漲。五年秋。陽武、原武二縣河決,壞田廬。七年六月,青田縣大水,損田稼。七月,河北東、西路水。北京館陶水,河溢入府城,壞官私廬舍。八月,趙、邢、洺、磁、相諸州河水泛溢,壞城郭、軍營。是年,相州漳
 河決,溺臨漳縣居民。懷州黃、沁河泛溢,大雨水,損稼,壞廬舍、城壁。磁州諸縣鎮,夏秋漳、滏河水泛溢。臨漳縣斛律口決,壞官私廬舍,傷田稼,損居民。



 元祐四年,夏秋霖雨,河流泛漲。八年,自四月,雨至八月,晝夜不息,畿內、京東西、淮南、河北諸路大水。詔開京師宮觀五日,所在州令長吏祈禱,宰臣呂大防等待罪。



 紹聖元年七月,京畿久雨,曹、濮、陳、蔡諸州水,害稼。



 元符元年,河北、京東等路大水。二年六月,久雨,陜西、京西、河北大水,河溢,漂人民,
 壞廬舍。是歲,兩浙蘇、湖、秀等州尤罹水患。



 大觀元年夏,京畿大水。詔工部都水監疏導,至於八角鎮。河北、京西河溢,漂溺民戶。十月,蘇、湖水災。二年秋,黃河決,陷沒邢州鉅鹿縣。三年七月,階州久雨,江溢。四年夏,鄧州大水,漂沒順陽縣。



 政和五年六月,江寧府、太平、宣州水災。八月,蘇、湖、常、秀諸郡水災。七年,瀛、滄州河決,滄州城不沒者三版,民死者百餘萬。



 重和元年夏,江、淮、荊、浙諸路大水,民流移、溺者眾,分遣使者振濟。發運使任諒坐不奏
 泗州壞官私廬舍等勒停。



 宣和元年五月,大雨,水驟高十餘丈,犯都城,自西北牟駝岡連萬勝門外馬監,居民盡沒。前數日,城中井皆渾,宣和殿後井水溢,蓋水信也。至是,詔都水使者決西城索河堤殺其勢,城南居民塚墓俱被浸,遂壞藉田親耕之稼。水至溢猛,直冒安上、南熏門,城守凡半月。已而入汴,汴渠將溢,於是募人決下流,由城北入五丈河,下通梁山濼,乃平。十一月,東南州縣水災。四年十二月戊戌,詔:「訪聞德州有京東、西來流
 民不少,本州振濟有方,令保奏推恩。餘路遇有流移,不即存恤,按劾以聞。」六年秋,京畿恆雨。河北、京東、兩浙水災,民多流移。



 建炎二年春,東南郡國水。



 紹興二年閏月,徽、嚴州水,害稼。三年七月丙子,泉州水三日,壞城郭、廬舍。五年秋,西川郡國水。六年冬,饒州雨水壞城四百餘丈。十四年五月丙寅,婺州水。乙丑,蘭溪縣水侵縣市,丙寅中夜,水暴至,死者萬餘人。十六年,潼川府東、南江溢,水入城,浸民廬。十八年八月,紹興府、明、婺州水。二十二
 年,淮甸水。二十三年,金堂縣大水。潼川府江溢,浸城內外民廬。宣州大水,其流泛溢至太平州。七月,光澤縣大雨,溪流暴湧,平地高十餘丈,人避不及者皆溺,半時即平。二十七年,鎮江、建康、紹興府、真、太平、池、江、洪、鄂州、漢陽軍大水。二十八年六月丙申,興、利二州及大安軍大雨水,流民廬,壞橋棧,死者甚眾。九月,江東、淮南數郡水。浙東、西沿江海郡縣大風、水,平江、紹興府、湖、常、秀、潤為甚。二十九年七月戊戌,福州水入城,閩、候官、懷安三縣
 壞田廬,官吏不以聞,憲臣樊光遠坐黜。三十年五月辛卯夜,於潛、臨安、安吉三縣山水暴出,壞民廬、田桑,溺死者甚眾。三十一年八月,建始縣大水,流民廬,死者甚眾。三十二年四月,淮溢數百里,漂民田廬,死者尤眾。六月,浙西郡縣山湧暴水,漂民舍,壞田覆舟。



 隆興元年八月,浙東、西州縣大風、水,紹興、平江府、湖州及崇德縣為甚。二年七月,平江、鎮江、建康、寧國府、湖、常、秀、池、太平、廬和光州、江陰、廣德、壽春、無為軍、淮東郡皆大水,浸城郭,壞
 廬舍、圩田、軍壘。操舟行市者累日,人溺死甚眾。越月,積陰苦雨,水患益甚,淮東有流民。



 乾道元年六月,常、湖州水壞圩田。二年八月丁亥,溫州大風,海溢,漂民廬、鹽場、龍朔寺,覆舟,溺死二萬餘人,江濱胔骼尚七千餘。三年六月,廬、舒、蘄州水,壞苗稼,漂人畜。七月己酉,臨安府天目山湧暴水,決臨安縣五鄉民廬二百八十餘家,人多溺死。八月,湖、秀州、上虞縣水,壞民田廬。時積潦至於九月,禾稼皆腐。江東山水溢,江西諸郡水,隆興府四縣為
 甚。四年七月壬戌,衢州大水,敗城三百餘丈,漂民廬,孳牧,壞禾稼。諸既縣大水害稼。江寧、建康府水。是歲,饒、信亦水。五年七月丁巳,建寧府瑞應場大漈、山棗等山暴水湧出,漂民廬,溺死甚眾。是歲夏秋,溫、臺州凡三大風,水漂民廬,壞田稼,入畜溺死者甚眾,黃巖縣為甚,郡守王之望、陳巖肖不以聞,皆黜削。六年五月,平江、建康、寧國府、溫、湖、秀、太平州、廣德軍及江西郡大水,江東城市有深丈餘者,漂民廬,湮田稼,潰圩堤,人多流徙。八年五
 月,贛州、南安軍山水暴出,及隆興府、吉、筠州、臨江軍皆大雨水,漂民廬,壞城郭,潰田害稼。六月壬寅,四川郡縣大雨水,嘉、眉、邛、蜀州、永康軍及金堂縣尤甚,漂民廬,決田畝。九年五月戊午,建康、隆興府、嚴、吉、饒、信、池、太平州、廣德軍水,漂民居,壞圩湮田,分水縣沙塞四百餘畝,採石流民多渡江。六月,湖北郡縣水。



 淳熙元年七月壬寅、癸卯,錢塘大風濤,決臨安府江堤一千六百六十餘丈,漂居民六百三十餘家,仁和縣瀕江二鄉壞田圃。三年
 八月辛巳,臺州大風雨,至於壬午,海濤、溪流合激為大水,決江岸,壞民廬,溺死者甚眾。癸未,行都大雨水,壞德勝、江漲、北新三橋及錢塘、餘杭、仁和縣田,流入湖、秀州,害稼。浙東西、江東郡縣多水,婺州、會稽嵊、廣德軍建平三縣尤甚。四年五月庚子,建寧府、福、南劍州大雨水,至於壬寅,漂民廬數千家。己亥夜,錢塘江濤大溢,敗臨安府堤八十餘丈;庚子,又敗堤百餘丈。明州瀕海大風,海濤敗定海縣堤二千五百餘丈、鄞縣堤五千一百餘丈,
 漂沒民田。九月丁酉、戊戌,大風雨駕海濤,敗錢塘縣堤三百餘丈;餘姚縣溺死四十餘人,敗堤二千五百六十餘丈;敗上虞縣堤及梁湖堰及運河岸;定海縣敗堤二千五百餘丈;鄞縣敗堤五千一百餘丈。五年六月戊辰,古田縣大水,漂民廬,圮縣治市橋。閏月己亥,階州水,壞城郭。乙巳,興化軍及福清縣及海口鎮大水,漂民廬、官舍、倉庫,溺死者甚眾。六年夏,衢州水。秋,寧國府、溫、臺、湖、秀、太平州水,壞圩田,樂清縣溺死者百餘人。七年五月
 戊戌,分宜縣大水,決田害稼。八年五月壬辰,嚴州大水,漂浸民居萬九千五百四十餘家、壘舍六百八十餘區。紹興府大水,五縣漂浸民居八萬三千餘家,田稼盡腐;漁浦敗堤五百餘丈,新林敗堤通運河。是歲,徽、江二州亦水。十年五月辛巳,信州大水入城,沉廬舍、市井。襄陽府大水,漂民廬,蓋藏為空。江東、浙東數郡亦水。八月辛酉,雷州大風激海濤,沒瀕海民舍,死者甚眾。九月乙丑,福、漳州大風雨,水暴至,長溪、寧德縣瀕海聚落、廬舍、人
 舟皆漂入海,漳城半沒,浸八百九十餘家。丁卯,吉州龍泉縣大水,漂民廬,壞田畝,溺死者眾。十一年四月,和州水,湮民廬,壞圩田。五月丙申,階州白江水溢,決堤圮城,浸民廬、壘舍、祠廟、寺觀甚多。建康府、太平州水。六月甲申,處州龍泉縣大雨,水浸民舍,壞杠梁,匯田害稼。七月壬辰,明州大風雨,山水暴出,浸民市,圮民廬,覆舟殺人。十二年六月,婺州及富陽縣皆水,浸民廬,害田稼。八月戊寅,安吉縣暴水發棗園村,漂廬舍、寺觀,壞田稼殆盡,
 溺死千餘人,郡守劉藻不以聞,坐黜。是歲,鄂州自夏徂冬,水浸民廬。九月,臺州水。十四年三月辛未,汀州水,漂百餘家、軍壘六十餘區。十五年五月,淮甸大雨水,淮水溢,廬、濠、楚州、無為、安豐、高郵、盱眙軍皆漂廬舍、田稼,廬州城圮。荊江溢,鄂州大水,漂軍民壘舍三千餘。江陵、常德、德安府、復、岳、澧州、漢陽軍水。戊午,祁門縣群山暴匯為大水,漂田禾、廬舍、塚墓、桑麻、人畜什六七,浮胔甚眾,及害及浮梁縣。六月,建寧、隆興府、袁、撫州、臨江軍水圮
 民廬。七月,黃巖縣水敗田瀦。番昜湖溢番昜縣,漂民舍、田稼,有流徙者。十六年四月甲戌,紹興府新昌縣山水暴作,害稼湮田,漂民廬。五月丙辰,沅、靖州山水暴溢至辰州,常德府城沒一丈五尺,漂民廬舍。汀州大水,浸民廬千五百餘家,溺死三千人。分宜縣水。丁巳,階州白江水溢,浸城市民廬。六月庚寅,鎮江府大雨水五日,浸軍民壘舍三千餘。辛卯,潼川府東南二江溢,決堤,毀橋,浸民廬,涪城、中江、射洪、通泉、郪縣沒田廬。



 紹熙二年三月,
 寧化縣連水漂廬舍、田畝,溺死二十餘人。五月戊申,建寧州水。己酉,福州水,浸附郭民廬,懷安、候官縣漂千三百餘家,古田、閩清縣亦壞田廬。庚午,利州東江溢。壞堤、田、廬舍。辛未,潼川府東、南江溢;六月戊寅,又溢,再壞堤橋,水入城,沒廬舍七百四十餘家,郪、涪、射洪、通泉縣匯田為江者千餘畝。七月癸亥,嘉陵江暴溢,興州圮城門、郡獄、官舍凡十七所,漂民居三千四百九十餘,潼川崇慶府、綿、果、合、金、龍、漢州、懷安、石泉、大安軍魚關皆水。時
 上流西蕃界古松州江水暴溢,龍州敗橋閣五百餘區,江油縣溺死者眾。三年五月壬辰,常德府大雨水,浸民田廬。乙未,潼川府東、南江溢,後六日又溢,浸城外民廬,人徙於山。己亥,池州大雨水連夕,青陽縣山水暴湧,漂田廬殺人,蓋藏無遺;貴池縣亦水。庚子,涇縣大雨水,敗堤,圮縣治、廬舍。六月辛丑,建平縣水,敗堤入城,漂浸民廬。甲戌,祁門縣水。七月壬申,天臺、仙居縣大水連夕,漂浸民居五百六十餘,壞田傷稼。襄陽、江陵府大雨水,漢
 江溢,敗堤防,圮民廬、沒田稼者逾旬,復州、荊門軍水亦如之。鎮江府三縣水,損下地之稼。四年四月,上高縣水,浸二百餘家。五月壬申、癸酉,奉新縣大雷雨、水,漂浸八百二十餘家。五月辛未、丙子,鎮江府大雨水,浸營壘六千餘區。戊寅,安豐軍大水,平地三丈餘,漂田廬,絲麥皆空。是月,諸暨、蕭山、宣城、寧國縣大水,壞田稼。廣德軍屬縣水害稼。筠州水浸民廬。戊寅,進賢縣水,圮百二十餘家。六月丙申,興國軍水,池口鎮及大冶縣漂民廬,有溺
 死者。戊戌,靖安縣水,漂三百二十餘家。是夏,江、贛州、江陵府亦水。七月乙酉,豐城縣水,壬午,臨江軍水,皆圮民廬。丁亥,新淦縣漂浸二千三百餘家。八月辛丑,隆興府水,圮千二百七十餘家。吉州水,漂浸民廬及泰和縣官舍。自夏及秋,江西九州三十七縣皆水。是歲,興化軍大風激海濤,漂沒田廬尤多。五年五月辛未,石埭、貴池、涇縣皆水,圮民廬,溺死者眾。是月,泰州大水。七月壬申,慈溪縣水,漂民廬,決田害稼,人多溺死。乙亥,會稽、山陰、蕭
 山、餘姚、上虞縣大風駕海濤,壞堤,傷田稼。八月辛丑,錢塘、臨安、新城、富陽、於潛縣大雨水,餘杭縣尤甚,漂沒田廬,死者無算。安吉縣水,平地丈餘。平江、鎮江、寧國府、明、臺、溫、嚴、常州、江陰軍皆水。是秋。武陵縣江溢,圮田廬甚眾。



 慶元元年六月壬申,臺州及屬縣大風雨,山洪、海濤並作,漂沒田廬無算,死者蔽川,漂沉旬日。至於七月甲寅,黃巖縣水尤甚。常平使者莫漳以緩於振恤,坐免。七月,臨安府水。二年秋,浙東郡國大水。三年九月,紹興府
 屬縣二婺州屬縣二,水害稼,五年秋,臺、溫衢、婺水,漂民廬,人多溺死,衢守張經以匿災吝振坐黜。六年五月,建寧府、嚴、衢、婺、饒、信、徽、南劍州及江西郡縣皆大水,自庚午至於甲戌,漂民廬,害稼。



 嘉泰二年七月丙午,上杭縣水,圮田廬,壞稼,民多溺死。建安縣漂軍民廬舍百二十餘,山摧,覆民廬七十七家,溺壓死者六十餘。丁未,長溪縣漂民廬二百八十餘家。古田縣漂官舍、民廬甚眾,溺死者二百七十。劍浦縣圮二百五十餘家,死者亦眾。三
 年四月,江南郡邑水害稼。



 開禧元年九月丙戌,漢、淮水溢,荊襄、淮東郡國水,楚州、盱眙軍為甚,圮民廬,害稼。二年五月庚寅,東陽縣大水,山千七百三十餘所同夕崩洪,漂聚落五百四十餘所,湮田二萬餘畝,溺死者甚眾。三年,江、浙、淮郡邑水,鄂州、漢陽軍尤甚。



 嘉定二年五月己亥,連州大水,敗城郭百餘丈,沒官舍、郡庠、民廬,壞田畝聚落甚多。六月辛酉,西和州水,沒長道縣治、倉庫。丙子,昭化縣水,沒縣治,漂民廬。成州水,入城,圮壘舍。同谷
 縣及遂寧府、閬州皆水。七月壬辰,臺州大風雨激海濤,漂圮二千二百八十餘家,溺死尤眾。三年四月甲子,新城縣大水。五月,嚴、衢、婺徽州、富陽、餘杭、鹽官、新城、諸暨、淳安大雨水,溺死者眾,圮田廬、市郭,首種皆腐。行都大水,浸廬舍五千三百,禁旅壘舍之在城外者半沒,西湖溢。四年七月辛酉,慈溪縣大水,圮田廬,人多溺者。八月,山陰縣海敗堤,漂民田數十里,斥地十萬畝。五年五月庚戌,嚴州水。六月丁丑,臺州及建德、諸暨、會稽縣水,壞
 田廬。六年六月丁丑,淳安縣山湧暴水,陷清泉寺,漂五鄉田廬百八十里,溺死者無算,巨木皆拔。丁亥,於潛縣大水。戊子,諸暨縣風雷大雨,山湧暴作,漂十鄉田廬,溺死者尤多。錢塘縣、臨安、餘杭、於潛、安吉縣皆水。九年五月,行都及紹興府、嚴、衢、婺、臺、處、信、饒、福、漳、泉州、興化軍大水,漂田廬,害稼。十年冬,浙江濤溢,圮廬舍,覆舟,溺死甚眾。蜀、漢二州江沒城郭。十一年六月戊申,武康、吉安縣大水,漂官舍、民廬,壞田稼,人畜死者甚眾。十二年,鹽
 官縣海失故道,潮汐沖平野三十餘里,至是侵縣治,廬州港瀆及上下管、黃灣岡等場皆圮。蜀山淪入海中,聚落、田疇失其半,壞四郡田。後六年始平。十四年,建康府大水。十五年七月,蕭山縣大水。時久雨,衢、婺、徽、嚴暴流與江濤合,圮田廬,害稼。十六年五月,江、浙、淮、荊、蜀郡縣水,平江府、湖、常、秀、池、鄂、楚、太平州、廣德軍為甚,漂民廬,害稼,圮城郭、堤防,溺死者眾。鄂州江湖合漲,城市沉沒,累月不洩。是秋,江溢,圮民廬。餘杭、錢塘、仁和縣大水。
 福、漳、泉州、興化軍水壞稼十五六。十七年五月,福建大水,漂水口鎮民廬皆盡,候官縣甘蔗砦漂數百家,人多溺死;建寧府沒平政橋,入城;南劍州圮郡治、城樓、郡獄、官舍,城壞,民避水樓上者皆死。乙卯,建昌軍大水,城不沒者三板,漂民廬,圮官舍、城郭、橋梁,害稼。



 紹定二年,天臺、仙居縣大水。四年,沿江水災。



 端平三年三月辛酉,蘄州大雨水,漂民居。是年,英德府、昭州及襄、漢江皆大水。



 嘉熙元年,饒、信州水。二年,浙江溢。



 淳祐二年,紹興府、處、
 婺州水。七年,福建水。十年,嚴州水。十一年八月甲辰,汀州山水暴至,漂人民。九月,江陵水。是年,江、浙多水,饒州亦水。十二年六月,建寧府、嚴、衢、婺、信、臺、處、南劍州、邵武軍大水,冒城郭,漂室廬,死者以萬數。



 寶祐元年七月,溫、臺、處、信、饒州大水。



 開慶元年五月己未,婺州水,漂民廬。是歲,滁、嚴州水。



 景定二年,浙東水。



 咸淳六年五月,大雨水。七年五月甲申,諸暨縣大水,漂廬舍。是月,重慶府江水泛溢者三,漂城壁,壞樓櫓。十年三月,廬州水。四月,紹
 興府大雨水。八月,臨安府水,安吉、武康縣水。



 太平興國四年八月,滑州黎陽縣河清。



 端拱元年二月,澶、濮二州河清二百餘里。



 大中祥符三年十一月丁酉,陜西河清。十二月乙巳,河再清,當汾水合流處清如汾水。



 元豐四年十月,環州河水變甘。



 大觀元年八月,乾寧軍河清。二年十二月,陜州河清,同州韓城縣、合陽縣至清及百里,涉春不變。自是迄政和、宣和,諸路數奏河清,輒遣郎官致祭,宰臣等率百官拜表賀,歲以為常。



 大中祥符元年二月,醴泉出蔡州汝陽鳳原鄉,有疾者飲之皆愈。八年十一月,通州軍言醴泉出汶山下,有疾者飲之皆愈。



 熙寧元年五月,京師開化坊醴泉出。



 政和五年正月,河陽臺觀醴泉出。



\end{pinyinscope}