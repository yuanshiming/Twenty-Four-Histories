\article{列傳第六十三}

\begin{pinyinscope}

 周
 渭梁鼎範正辭子諷劉師道王濟方偕曹穎叔劉元瑜楊告趙及劉湜王彬仲簡



 周渭,字得臣,昭州恭城人。幼孤,養於諸父。力學,工為詩。劉鋹據五嶺,昭州皆其地也,政繁賦重,民不聊生。渭率鄉人六百逾嶺,將避地零陵。未至,賊起,斷道絕糧,復還恭城,則廬舍煨燼,遂奔道州。為盜所襲,渭脫身北上。



 建隆初,至京師,為薛居正所禮。上書言時務,召試,賜同進士出身,解褐白馬主簿。縣大吏犯法,渭即斬之。上奇其才,擢右贊善大夫。時魏帥符彥卿專恣,朝廷選常參官強幹者蒞其屬邑,以渭知永濟縣。彥卿郊迎,渭揖於馬
 上,就館始與相見,略不降屈。縣有盜傷人而逸,渭捕獲,並暴廋匿者按誅之,不以送府。



 乾德中,通判興州。州領罝口砦多戍兵,監軍敖很,縱其下為暴,居人苦之。渭馳往諭以禍福,斬其軍校,眾皆懾服。詔書嘉獎,命兼本砦鈐轄。開寶元年,鳳州七房冶主吏盜隱官銀,擇渭往代。周歲,羨課數倍,賜緋魚,又遷知棣州。殿直傅延翰為監軍,謀作亂走契丹,為部下所告,渭擒之以聞;命械至闕下,鞫得實,斬於西市。渭在郡以簡肅稱,及還,吏民遮道
 泣留,俄詔賜錢百萬。



 太平興國二年,為廣南諸州轉運副使。初,渭之入中原,妻子留恭城。開寶三年,平廣南,詔昭州訪求,賜錢米存恤之。及是,渭始還故里,鄉人以為榮。渭奏去劉鋹時稅算之繁者,復位田賦,興學校。遷殿中丞。屬有事交址,主將逗撓無功。有二敗卒擐甲先至邕州市,奪民錢,渭捕斬之。後至者悉令解甲以入,訖無敢犯。移書交址,諭朝廷威信,將刻日再舉。黎桓懼,即遣使入貢。就加監察御史,在嶺南凡六年。徙知揚州,進殿
 中侍御史,改兩浙東、西路轉運使,入為鹽鐵判官。遷侍御史,歷判戶部、度支二勾院,出知亳州,賜金紫,俄換宋州。加職方員外郎,為益州轉運使。坐從子違詔市馬,黜為彰信軍節度副使。咸平二年,真宗聞其清節,召還,將復用,詔下而卒,年七十七。上閔其貧不克葬,賻錢十萬,以其子建中為乘氏主簿。



 渭妻莫荃,賢婦人也。渭北走時,不暇與荃訣,二子孩幼,荃尚少,父母欲嫁之。荃泣誓曰:「渭非久困者,今違難遠適,必能自奮。」於是親蠶績碓
 舂,以給朝夕,二子皆畢婚娶。凡二十六年,復見渭,時人異之。朱昂著《莫節婦傳》紀其事。



 梁鼎字凝正,益州華陽人。祖鉞,仕蜀為劍門關使。父文獻,乘氏令。鼎,太平興國八年進士甲科,解褐大理評事、知秭歸縣,再遷著作佐郎。端拱初,獻《聖德徽號頌》萬餘言,試文,遷殿中丞、通判歙州,以能聲聞,有詔嘉獎。徙知吉州,民有蕭甲者,豪猾為民患,鼎暴其兇狀,杖脊黥面徙遠郡。太宗尤賞其強幹,代還,賜緋魚,舊例當給銀寶
 瓶帶,太宗特以犀帶賜之,記其名於御屏。



 淳化中,上言曰:「《書》云:『三載考績,三考黜陟幽明。』此乃堯、舜氏所以得賢人治天下也。三代而下,典章尚存,兩漢以還,沿革可見。至於唐室,此道尤精,有考功之司,明考課之令,下自簿尉,上至宰臣,皆歲計功過,較定優劣,故人思激厲,績效著聞。五代兵革相繼,禮法陵夷,顧惟考課之文,祇拘州縣之輩,黜陟既異,名存實亡。且夫今之知州,即古之刺史,治狀顯著者,朝廷不知;方略蔑聞者,任用如故。大
 失勸懲之理,浸成茍且之風。是致水旱薦臻,獄訟填溢,欲望天下承平,豈可得也。伏惟陛下繼二聖之丕圖,為億兆之司牧,念百官之未乂,思四海之未康,特詔有司,申明考績之法,庶幾官得其人,民受其賜矣。」



 俄為開封府判官,遷太常博士、三司右計判官,又為總計判官,會復三部,換度支判官。至道初,鼎洎陳堯叟建議興三白渠,及陳、許、鄧、穎、蔡、宿、亳數州用水利墾田,事具《食貨志》。遷都官員外郎、江南轉運副使,就改起居舍人,徙陜西。
 二年,五將分道擊李繼遷,李繼隆擅出赤檉路無功,還奏軍儲失期,鼎坐削三任。復為殿中丞,領職如故。以母老求郡,歷知徐、密二州。真宗踐位,復舊官。咸平四年,遷兵部員外郎、知制誥,賜金紫。時三司督逋負嚴急,有久被留系者,命鼎與薛映按籍詳定,多所蠲免。逾月,拜右諫議大夫、度支使。



 時西鄙未寧,建議陜西禁解池鹽,所在官鬻,詔從之。以鼎為制置使,楊覃為轉運使,張賀副之,又以內殿崇班杜承睿同制置鹽事。議者多言:「邊民
 舊食青鹽,其價甚賤。洎禁青鹽以困賊,令商賈入粟,運解鹽於緣邊,價直與蕃鹽不相遠,故蕃部繼鹽至者,不能貨鬻。今若禁解池鹽,與內地同價,則民必冒禁復市青鹽,乃資盜糧也。」時劉綜為陜西轉運使,鼎奏罷之。綜歸朝,亦密陳其非便。鼎既行,即移文禁止鹽商,所在約束乖當,延州劉廷偉、慶州鄭惟吉皆不從規畫。



 又鼎奏運咸陽倉粟以實邊,粟已陳腐,鼎即與民,俟秋收易新粟,朝廷聞而止之,上封章密陳其煩擾者甚眾,鼎始謀
 多沮,遂令林特乘傳與永興張詠會鼎等同議可否,於是依舊通鹽商。鼎坐首議改作非是,詔罷度支使,守本官。未幾,丁內艱,起復。景德初,知三班院、通進銀臺司兼門下封駁事,出知鳳翔府。以居憂哭泣傷目,表求判西京留司御史臺。三年,卒,年五十二,賜二子出身。



 鼎偉姿貌,磊落尚氣,有介節,居官峻厲,名稱甚茂。好學,工篆、籀、八分。嘗著《隱書》三卷,《史論》二十篇,《學古詩》五十篇。子申甫、吉甫。



 範正辭,字直道,齊州人。父勞謙,獲嘉令。正辭治《春秋公羊》、《穀梁》,登第,調補安陽主簿。開寶中,判入等,遷國子監丞、知戎州,改著作佐郎。代還,治逋欠於淄州,轉運使稱其能,轉左贊善大夫,就知淄州。太宗征河東,諸州部糧多不及期,正辭所部長山縣吏張秀督民輸,受錢二千,即杖殺之,郡中畏服。



 太平興國中,改殿中丞,通判棣、深二州,遷國子博士。御史中丞劉保勛奏充臺直,會有言饒州多滯訟,選正辭知州事,至則宿系皆決遣之,胥吏
 坐淹獄停職者六十三人。會詔令料州兵送京師,有王興者,懷土憚行,以刃故傷其足,正辭斬之。興妻詣登聞上訴,太宗召見,正辭廷辨其事。正辭曰:「東南諸郡,饒實繁盛,人心易動。興敢扇搖,茍失控馭,則臣無待罪之地矣。」上壯其敢斷,特遷膳部員外郎,充江南轉運副使,賜錢五十萬。



 饒州民甘紹者,積財鉅萬,為群盜所掠,州捕系十四人,獄具,當死。正辭按部至,引問之,囚皆泣下,察其非實,命徙他所訊鞫。既而民有告群盜所在者,正辭
 潛召監軍王願掩捕之。願未至,盜遁去,正辭即單騎出郭二十里,追及之。賊控弦持弰來逼,正辭大呼,以鞭擊之,中賊雙目,執之。賊自刃不殊,餘賊渡江散走,追之不獲,旁得所棄贓。賊尚有餘息,正辭即載歸,令醫傅藥,創既愈,按其奸狀,伏法,而前十四人皆得釋。



 端拱二年,代歸,與洛苑副使綦仁澤、西京作坊副使尹宗諤同監折中倉。先是,令商人輸米豆而以茶鹽酬其直,謂之「折中」,復有言其弊,罷之,至是復置焉。遷倉部員外郎,同知幕
 府州縣官考課,改判刑部,歷戶部、鹽鐵二判官,遷考功員外郎,通判定、揚、杭三州。真宗即位,遷膳部郎中,召判三司勾院,俄復為鹽鐵判官。咸平二年,出為河東轉運使。三年,以本官兼侍御史知雜事。



 時李昌齡自忠武行軍起知梓州,董儼知壽州,王德裔、楊緘皆任轉運使,後失官宰畿邑。正辭上言:「昌齡輩貪墨著聞,願陛下罷其民政。」詔追還儼敕,餘悉代之。又言:「治民之官,牧宰為急。」舉吳奮等五人堪任大郡,復請令奮等各舉知縣、縣令,
 從之。坐鞫任懿獄,貶滁州團練副使。會赦,復為倉部考功員外郎、通判鄆州,知淮陽軍,復膳部郎中,以年老,求監兗州商稅。大中祥符三年四月卒,年七十五。子識、諷,並進士及第。



 諷字補之,以蔭補將作監主簿,獻《東封賦》,遷太常寺奉禮郎。又獻所為文,召試入等,出知平陰縣。會河決王陵埽,水去而土肥,失阡陌,田訟不能決,諷分別疆畔,著為券,民持去不復爭。諷辨數激昂,喜為名聲,然亦操持在
 己,吏不敢欺。為縣存視貧弱,至豪猾大家,峻法治之。



 舉進士第,遷大理評事、通判淄州。歲旱蝗,他穀皆不立,民以蝗不食菽,猶可藝,而患無種,諷行縣至鄒平,發官廩貨民。縣令爭不可,諷曰:「有責,令無預也。」即出貸三萬斛;比秋,民皆先期而輸。徙知梁山軍,以母老不行,得通判鄆州。時知州李迪貶衡州副使,宰相丁謂戒使者持詔書促上道,諷輒留迪數日,為治裝祖行。詔塞決河,州募民入芻揵,而城邑與農戶等,諷曰:「貧富不同而輕重相
 若,農民必大困。且詔書使度民力,今則均取之,此有司誤也。也。」即改符,使富人輸三之二,因請下諸州以鄆為率,朝廷從其言。



 徙知廣濟軍,民避水堤居,凡給徭於官者,諷悉縱使護其家,奏除其租賦。累遷太常博士,以疾監舒州靈仙觀。尚御藥張懷德至觀齋祠,諷頗要結之,懷德薦於章獻太后,遂召還。問所欲言,對曰:「今權臣驕悍,將不可制。」蓋指曹利用也。利用貶,拜右司諫、三司度支判官。百官轉對,敕近臣閱視其可行者,類次以聞。諷奏
 曰:「非上親覽決可否,則誰肯為陛下極言者。」玉清昭應宮災,下有司治火所起,諷曰:「此天之戒告,乃復置獄以窮治之,非所以應天也。」獄由是得解。議者疑復修,諷上書諫:「山木已盡,人力已竭,宮必不成。臣知朝廷亦不為此,其如疑天下何。宜詔示四方,使明知之。」於是下詔罷修。改尚書禮部員外郎兼侍御史知雜事。



 錢惟演自許州來朝,圖相位,諷奏:「惟演嘗為樞密使,以皇太后姻屬罷之,示天下以不私,固不可復用。」遂以惟演守河南。使
 契丹,道過幽州北,見原野平曠,慨然曰:「此為戰地,不亦信哉。」遼人相目不敢對。擢天章閣待制、知審刑院,出知青州,再遷戶部郎中。時山東饑,宰相王曾,青人,家積粟多,諷發取數千斛濟饑民,因請遣使安撫京東。入為右諫議大夫、權御史中丞。又請益漕江、淮米百萬,自河陽、河陰東下以賑貸之。錢惟演倡議獻、懿二太后宜祔真宗廟室,諷彈奏之;及言其在太后時權寵甚盛,且與後族連姻,請絀去。仁宗不聽,諷袖告身以對曰:「陛下不聽
 臣言,臣今奉使山陵,而惟演守河南,臣早暮憂刺客。願納此,不敢復為御史中丞矣。」帝不得已可之,諷乃趨出,遂貶惟演隨州。



 陳堯佐罷參知政事,有王文吉者,告堯佐謀反,仁宗遣中官訊問,復以屬諷。夜中被旨究詰,旦得其誣狀奏之。時上章懿皇后謚,宰相張士遜、樞密使楊崇勛日中不赴慰班,諷彈士遜與崇勛,俱罷。諷嘗侍對,帝語及郭後亡子。諷言亡子大義當廢,陰合帝旨,以龍圖閣直學士權三司使。時狄棐為直學士已久,諷盛
 氣凌棐,宰相李迪右之,遂特詔班棐上,論者非之。尋轉閣學士,又疾免三司使,改翰林侍讀學士、管勾祥源觀。徙會靈觀,復改閣學士、給事中、知兗州。



 既至郡,而龐籍為廣南東路轉運使,未行,上言:「向為侍御史,嘗奏彈諷以三司使曲為左藏監庫吳守則奏課遷官。尚美人同父弟娶守則女,諷以銀鞍勒遺守則相結納。既出兗州,乃紿言貧,假翰林白金器數千兩自隨,而增產於齊州,市官田虧平估。」置獄於南京劾之,諷坐方聽旨擅馳驛
 還兗州,當贖。籍所奏有不實,當免官。宰相呂夷簡嫉諷詭激,特貶諷武昌軍節度行軍司馬;貸籍,止降官知臨江軍。由是宰相李迪等坐親善諷皆斥。



 歲中徙保信軍,聽居舒州持母喪,又許歸齊州。日飲酒自縱,為時所譏。服除,改將作少監、知淮陽軍,遷光祿卿、知陜州,道改潞州。入見帝言:「元昊不可擊,獨以兵守要害,捍侵掠,久當自服。倘內修百度,躬節儉,如祖宗故事,則疆事不足憂。」復給事中,卒。



 諷嘗建議朝廷當差擇能臣,留以代大臣
 之不稱職者。大臣聞而惡之。又數短參知政事王隨於帝前,因奏:「外人謂臣逐隨將取其位,願先出臣,為陛下引奸邪去,而朝廷清矣。」又嘗與張士遜議事不合,諷曰:「世謂大事未易可議,小事不足為,所為終何事邪?」及為龐籍訟,人謂大臣陰諷籍焉。



 諷類曠達,然捭闔圖進,不守名檢,所與游者輒慕其所為,時號「東州逸黨」。山東人顏太初作《逸黨詩》刺之,而姜潛者又嘗貽書以疏其過雲。



 子寬之,終尚書刑部郎中、知濠州。



 劉師道,字損之,一字宗聖,開封東明人。父澤,右補闕。師道,雍熙二年舉進士,初命和州防禦推官,歷保寧、鎮海二鎮從事,凡十年。王化基、呂祐之、樂史薦於朝,擢著作佐郎,才一月,會考課,又遷殿中丞,出知彭州,就加監察御史。轉運使劉錫、馬襄上其治跡,召歸。會浦洛之敗,奉詔劾白守榮輩,獄成,太宗獎其勤,面賜緋魚。



 川陜豪民多旁戶,以小民役屬者為佃客,使之如奴隸,家或數十戶,凡租調庸斂,悉佃客承之。時有言李順之亂,皆旁戶
 鳩集,請擇旁戶為三耆長迭主之,疇歲勞則授以官,詔師道使兩川議其事。師道以為迭使主領則爭忿滋多,署以名級又重增擾害,廷奏非便,卒罷之。改祠部員外郎,出為京東轉運使。真宗嗣位,進秩度支。咸平初,範正辭薦其材堪長民,徙知潤州。三年,改淮南轉運副使兼淮南、江、浙、荊湖發運使。四年,以漕事入奏,特遷司封,俄為正使,改工部郎中,代查道為三司度支副使。七月,擢樞密直學士,掌三班。俄擢權三司使,從幸澶淵,判隨駕三
 司,充都轉運使。



 師道弟幾道,舉進士禮部奏名,將廷試,近制悉糊名較等,陳堯咨當為考官,教幾道於卷中密為識號。幾道既擢第,事洩,詔落其籍,永不預舉。師道固求辨理,詔曹利用、邊肅、閻承翰詣御史府推治之。坐論奏誣罔,責為忠武軍行軍司馬,堯咨免所居官,為鄆州團練副使。二年,以郊祀恩,起為工部郎中、知復州,換秀州。



 大中祥符二年,以兵部郎中知潭州,遷太常少卿。師道敏於吏事,所至有聲,吏民畏愛。長沙當湖、嶺都會,剖
 煩析滯,案無留事。歲滿,復加樞密直學士,換左司郎中,留一任。七年,李應機代還。應機未至郡,六月,師道暴病卒,年五十四,錄幾道為試秘書省校書郎。



 師道性慷慨尚氣,善談世務,與人交敦篤。工為詩,多與楊億輩酬唱,當時稱之。



 王濟,字巨川。其先真定人,祖卿,有祠辨,趙王熔召置幕府。熔政衰,卿懼禍,避地深州饒陽,遂為縣人。父恕,後唐時童子及第,開寶中,知秀州。會盜起,城陷,為盜所殺,將
 並害濟。濟伏柩號慟,謂賊曰:「吾父已死,吾安用生為,但恨力不能殺汝,以報父仇爾!」賊義之,舍去。濟攜父骨匿山谷間。既而官軍大集,濟脫身謁其帥朱乙,陳討賊之計。乙嘉之,遺以束帛,奏假驛置遣歸。



 先是,濟母終於岳陽,權窆佛舍。至是,乃並護二喪還饒陽。州將以聞,太祖召見,以其尚少,且俾就學。雍熙中,上書自陳死事之孤,得試學士院,補龍溪主簿。時調福建輸鶴翎為箭羽。鶴非常有物,有司督責急,一羽至直數百錢,民甚苦之。濟
 諭民取鵝翎代輸,仍驛奏其事,因詔旁郡悉如濟所陳。縣有陂塘數百頃,為鄉豪斡其利,會歲旱,濟悉導之,分溉民田。汀州以銀冶構訟,十年不決,逮系數百人,轉運使使濟鞫之,才七日情得,止坐數人。



 再調胙城尉,徙臨河主簿。轉運使王嗣宗被詔舉法官,以濟名聞。遷光祿寺丞、權大理丞,改刑部詳覆官、通判鎮州。牧守多勛舊武臣,倨貴陵下,濟未嘗撓屈。戍卒頗恣暴不法,夜或焚民舍為盜。一夕,報有火,濟部壯士數十潛往偵伺,果得
 數輩並所盜物,即斬之。馳奏其事,太宗大悅。都校孫進使酒無賴,毆折人齒,濟不俟奏,杖脊送闕下,繇是軍城畏肅。就遷太子中舍,詔書獎勞。召判登聞鼓院,拜監察御史。上疏陳統天下之術、節民物之道,大者有十:擇左右,別賢愚,正名器,去冗食,加奉祿,謹政教,選良將,分兵戍,修民事,開仕進。其言切於時,詞多不載。



 咸平初,濟以刑綱尚繁,建議請刪定制敕,乃命張齊賢領其事,濟預焉。《刑統》舊條:持仗行劫,不以贓有無,悉抵死。齊賢議貨
 不得財者,濟曰:「刑,期於無刑。以死懼之,尚不畏,況緩其死乎?」因與齊賢廷爭數四。濟詞氣甚厲,目齊賢為腐儒。然卒從齊賢議,人以濟為刻。改鹽鐵判官。



 車駕巡師大名,調丁夫十五萬修黃、汴河,濟以為勞民,詔濟馳往經度,還奏省十六七。齊賢時為相,以河決為憂。因對,並召濟見,齊賢請令濟署狀保河不決,濟曰:「河決亦陰陽災沴,宰相茍能和陰陽,弭災沴,為國家致太平,河之不決,臣亦可保。」齊賢曰:「若是,則今非太平邪?」濟曰:「北有契丹,
 西有繼遷,兩河、關右歲被侵擾。以陛下神武英略,茍用得其人,可以馴致,今則未也。」上動容,獨留濟問邊事。濟曰:「陛下承二聖之基,擁百萬之眾,蠢茲醜虜,敢爾憑陵,蓋謀謨當國之人未有如昔之比。臣謂國家所恃,獨一洪河耳!此誠急賢之秋;不然,臣懼敵人將飲馬於河渚矣。」又著《備邊策》十五條以獻。



 三年,選官判大理寺,上曰:「法寺宜擇當官不回者,茍非其人,或有冤濫,即感傷和氣。王濟近數言事,似有操持,可試之。」遂令濟權判大理
 寺事。福津尉劉瑩集僧舍,屠狗群飲,杖一伶官致死,濟論以大闢,遇赦從流。時王欽若知審刑,與濟素不相得,又以濟嘗忤齊賢,乃奏瑩當以德音原釋。齊賢、王欽若議濟坐故入,停官。逾年,復為監察御史、通判河南府。



 景德初,徙知河中府。契丹南侵,上幸澶淵,詔緣河斷橋梁,毀船舫,稽緩者論以軍法。濟曰:「陜西有關防隔閡,舳艫遠屬,軍儲數萬,一旦沉之,可惜;又動搖民心。」因密奏寢其事,上深嘉嘆,遣使褒諭。未幾,召拜工部員外郎兼侍
 御史知雜事。三年,判司農寺。時周伯星見,濟乘間言曰:「昔唐太宗以豐年為上瑞。臣願陛下日慎一日,居安慮危,則天下幸甚。」受詔與劉綜改定茶法,頗易舊制,由是忤丁謂、林特、劉承規輩,因與欽若迭詆訾之。



 四年,拜本曹郎中,出知杭州。上面加慰諭,仍戒以朝廷闕失許密上言。遷刑部郎中。郡城西有錢塘湖,溉田千餘頃,歲久湮塞。濟命工浚治,增置斗門,以備潰溢之患,仍以白居易舊記刻石湖側,民頗利之。睦州有狂僧突人州廨,出
 妖言,與轉運使陳堯佐按其實,斬之,上嘉其能斷。大中祥符三年,徙知洪州,兼江南西路安撫使。屬歲旱民饑,躬督官吏為糜粥,日親嘗而給之;錄饑民為州兵,全活甚眾。是歲卒,年五十九,遺奏大旨以進賢退諛佞、罷土木不急之費為言。



 濟頗涉經史,好讀《左氏春秋》,性剛直,無所畏避。少時,深州刺史念金鎖一見器之,且托後於濟。金鎖沒,濟撫其孤,授置祿仕。素與內臣裴愈有隙,愈坐事,上怒甚,命憲府鞫之,濟適知雜事,力為辨理,遂獲
 輕典。子孝傑,國子博士。



 論曰:渭有清節,臨事多從便文。鼎好規畫。師道喜論世務。正辭按貪吏,辨冤獄。濟議論挺特,無所畏避。五臣者,仕不過監司、郡守,而名稱甚茂,可尚哉。



 方偕,字齊古,興化莆田人。年二十,及進士第,為溫州軍事推官。歲饑,民欲隸軍就廩食,州不敢擅募。偕乃詣提點刑獄呂夷簡曰:「民迫流亡,不早募之,將聚而為盜矣。」夷簡從之,籍為軍者七千人。後遷汀州判官,權知建安
 縣。縣產茶,每歲先社日,調民數千鼓噪山旁,以達陽氣。偕以為害農,奏罷之。



 遷秘書省著作佐郎,歷知福清、資陽縣。累遷尚書屯田員外郎,為御史臺推直官。澧州逃卒傭民家自給,一日,誣告民事摩駝神,歲殺十二人以祭。州逮其族三百人系獄,久不決。偕被詔就劾,令卒疏所殺主名,按驗皆亡狀,事遂辨,卒以誣告論死。知雜事龐籍薦為御史裏行,再遷侍御史。南京鴻慶宮災,偕引漢罷原廟故事,請勿復修。



 元昊寇塞門,鄜延副總管趙
 振逗撓不出救,詔偕往按之,法當斬。偕奏:「兵寡不敵,茍出以餌賊,無益也。」振由是得不死。為開封府判官、江南安撫。三司歲出乳香、綿綺下州郡配民,偕奏罷之。更鹽鐵判官,遷兵部員外郎兼御史知雜事,言:「以罪謫監當者,監司勿得差權親民官。」判大理寺,改度支副使,擢天章閣待制、江淮制置發運使、知杭州,遷刑部郎中。



 偕以吏事進,治杭州有能聲。喜飲酒,至酣宴無節。數月,暴中風,以太常少卿分司西京,遷光祿卿,卒。



 曹穎叔,字秀之,亳州譙人。初名熙,嘗夢之官府,見穎叔名,遂更名穎叔。進士及第,歷威勝軍判官、渭州軍事推官。御史中丞蔡齊薦為臺主薄,改大理寺丞。韓億知亳州,闢僉書節度判官事,通判儀州。韓琦、文彥博薦其才,徙夔州路轉運判官。夔、峽尚淫祠,人有疾,不事醫而專事神,穎叔悉禁絕之,乃教以醫藥。提點陜西路刑獄,夏人納款,詔與戶部副使夏安期、轉運使柳灝減戍卒吏員之冗者。為開封府判官,時御史宋禧鞫衛士獄於內
 侍省,禧不能辨,及獄具,內侍使禧自為牒,穎叔言禧為制使辱命,請置之法。元昊死,為夏國祭奠使。除直史館、知鳳翔府,徙益州路轉運使,權度支副使。



 儂智高寇嶺南,朝議以閩中久弛兵備,擢天章閣待制、知福州。累遷右司郎中,為陜西都轉運使。自慶歷鑄大鐵錢行陜西,民盜鑄不已,三司上榷鐵之議。穎叔曰:「鐵錢輕而貨重,不可久行,況官自榷鐵乎?請罷鑄諸郡鐵錢,以三鐵錢當銅錢之一。」從之。兩川和買絹給陜西兵,而蜀人苦於
 煩斂,穎叔為歲出本路緡錢五十萬,以易軍衣之餘者,兩川之民始無擾焉。進龍圖閣直學士、知永興軍;然年老,漸昏耄,事頗壅積,人或嘲誚之,卒於官。



 劉元瑜字君玉,河南人。進士及第,補舞陽縣主簿,改秘書省著作佐郎、知雍丘縣,通判隰、並二州,知郢州。以太常博士為監察御史,上言:「考課之法,自朝廷至員外郎、郎中、少卿,須清望官五人保任始得遷,故浮薄輩日趨權門,非所以養廉恥也。」詔罷之。



 提舉河北便糴。會永寧
 雲翼軍士謀為變,吏窮捕,黨與謀劫囚以反,百姓竊知多逃避。元瑜馳至,斬為首者,其餘皆釋去不問。歷京西、河東轉運使,遷右司諫。劾奏「集賢校理陸經謫官在河南日,杖死爭田寡婦,且貸民鏹,監司列薦其才,投托權要,遂復館職,請重置於法,並坐保薦者。」詔屬吏,遂竄經袁州。



 又疏「李用和、曹琮、李昭亮不可典軍;梁適不當翰林學士;範仲淹以非罪貶,既復天章閣待制,宜在左右;尹洙、餘靖、歐陽修皆以朋黨斥逐。此小人惡直醜正
 者也。」既而與靖等相失,反言:「前除夏竦為樞密使,諫臣數人摭其舊過,召至都門而罷之。自此以進退大臣為己任,激訐陰私為忠直,薦延輕薄,列之館閣,以唱和為朋比。近除兩府,出自聖斷,獨黨人以進用不出於己,議論紛然,臣恐復被疏罷矣。前日孫甫薦葉清臣,毀丁度,效此也。」因論:「靖知制誥不宜兼領諫職,且奉使契丹,對契丹主,效六國語,辱國命,請加罪。」修、靖深惡之,繇是論者以元瑜為奸邪。



 後除三司鹽錢副使,以天章閣待制
 知潭州。猺人數為寇,元瑜使州人楊謂入梅山,說酋長四百餘人出聽命,因厚犒之,籍以為民,凡千二百戶。徙桂州,固辭,降鄧州。坐在潭州擅補畫工易元吉為畫助教,降知隨州。又失保任,改信州,徙襄州。富人子張銳少孤弱,同里車氏規取其財,乃取銳父棄妾他姓子養之。比長,使自訴,陰賕吏為助,州斷使歸張氏,銳莫敢辨。既同居逾年,車即導令求析居。元瑜察知,窮治得奸狀,黥車竄之,人伏其明,歷河中府,以左諫議大夫知青州,卒。



 元瑜性貪,至竅販禁物,親與小人爭權,時論鄙之。



 楊告字道之,其先漢州綿竹人。父允恭,西京左藏庫使,數任事有功。既死,賜告同學究出身,調廬江尉。時張景笞吏死而吏捕急,逃歸告,懼告不見納,告曰:「君勿憂也,吾死生以之。」景卒免。改豐城主簿,邑有賊殺人,投尸於江,人知主名,而畏不敢言,告聞,親往擒賊。有言賊欲報怨者,告不為動。既而果乘夜欲刺告,告又捕得,致於法,境內肅然。



 再調南劍州判官,知南安、六合、錢塘、寧國縣,
 改大理寺丞、通判江寧州。盜殺商人,鑿舟沉尸江中。有被誣告者笞服,獄具,告疑其無狀,後數日,果得真盜。徙知池州,累遷尚書司封員外郎、開封府推官、開拆司。為趙元昊旌節官告使,元昊專席自尊大,告徙坐即賓位,莫之屈也。除京西轉運副使。屬部歲饑,所至發公廩,又募富室出粟賑之。民伐桑易粟,不能售,告命高其估以給酒,官民獲濟者甚眾。以疾,權管勾西京留臺。頃之,判三司憑由、理欠司,為淮南轉運使,徙制置發運使,除三
 司戶部副使,更度支,安撫河東,改鹽鐵副使。歷祠部、度支、司封郎中,以少府監復為制置發運使。拜右諫議大夫、知鄭州,徙江寧府、壽州。



 告曉法令,頗知財利,而不務苛刻,時號能吏,然喜事權貴以要進。一子,力學有文,數為近臣薦,召試,賜同進士出身,未幾卒。告悲傷之,尋卒。



 趙及,字希之,其先幽州良鄉人。父的,事契丹為蔚州靈丘令,雍熙中,王師北征,乃歸,授偃師令,因家焉。及舉進士,為慈州軍事推官,徙廣信軍判官,改秘書省著作佐
 郎、知魏縣,徙九隴,以母老監葉縣稅,歷黃河、御河催綱,通判青州、大名府,累遷尚書屯田員外郎,被舉為殿中侍御史、權宗正丞。詔劾夏守恩獄,內侍岑守中用賄撓法,及劾正其罪。遷侍御史,夏守贇經略西鄙還,及言其無功,不可復樞府。又疏罷郭承祐團練使。



 未幾,請知懷州,徙徐州,還為三司戶部判官,遷兵部員外郎、京東轉運按察使。知萊州張周物貪暴,及劾奏,貶周物嶺外。擢兼侍御史知雜事,數論時政,權判吏部流內銓。初,銓吏
 匿員闕,與選人為市,及奏闕至即榜之,吏部榜闕自及始。遷戶部副使,以疾,改刑部郎中、直昭文館、知衛州,召為鹽鐵副使。又以疾,請知汝州,歲餘,復召為副使,不赴。徙知河中府,特拜天章閣待制、右司郎中。祀明堂,遷右諫議大夫。還判大理寺、流內銓。出知徐州,疾甚,求解近職,還州事,乃以本官管勾南京留司御史臺,未赴,卒。



 及和厚謙退,內行尤篤,所治有聲,民吏愛之。



 劉湜,字子正,徐州彭城人。舉進士,為澶州觀察推官,再
 調湖南節度推官,改秘書省著作佐郎、知益都縣,徙陰平。再遷太常博士、通判劍州。審閬州獄,活死囚七人。王堯臣安撫陜西,薦之,擢知耀州。富平有盜掠人子女者,既就擒,陽死,伺間逸去;捕得,復陽死,守者以報,湜趣焚其尸。拜監察御史,王德用自隨州詔還,近臣言其有反相,湜保右之。歷開封府推官、三司鹽鐵判官,遷殿中侍御史。上言:「轉運使掎摭郡縣,苛束官吏,人不得騁其材,宜稍寬假,不為改者繩治之。」詔詣渭州劾尹洙私用公
 使錢,頗傅致重法,以故洙坐廢。還,為尚書禮部員外郎兼侍御史知雜事,同判吏部流內銓,除鹽鐵副使。議者謂湜探宰相意,深致洙罪,故得優擢焉。



 明年,宴紫宸殿,副使當坐殿東廡,湜不即坐,趣出。閣門奏之,坐謫知沂州,徙兗州。又坐沂州誤出囚死罪,降知海州。起為河東轉運使,遷戶部員外郎,復為鹽鐵副使兼領河渠事。汴水絕,鑿河陰新渠,通漕運如故。會江南饑,擢天章閣待制、知江寧府,奏運蘇州米五十萬斛,以貸饑民。除戶部
 郎中、知廣州。儂智高初平,湜練士兵,葺械器,作鐵鎖斷江路。有盜據山,敕貸罪招之,不肯降。湜知並山民資之食,即徙民絕餉,盜困蹙乞降,民安之。居二年,母老求內徙,遂徙徐州。湜喜曰:「昔布衣隨計,今以侍從官三品復典鄉郡,過始望矣。」又以左司郎中知鄆州,遷龍圖閣直學士、知慶州。



 湜少賤,母更嫁營卒,既登第,具袍笏趨卒舍迎母,里人觀嘆。然嗜酒,持法少恕,改知密州,以病卒。



 王彬,光州固始人。祖彥英,父仁侃,從其族人潮入閩。潮
 有閩土,彥英頗用事,潮惡其逼,陰欲圖之。彥英覺之,挈家浮海奔新羅。新羅長愛其材,用之,父子相繼執國政。



 彬年十八,以賓貢入太學。淳化三年,進士及第,歷雍丘尉。皇城司陰遣人下畿縣刺史,多厲民,令佐至與為賓主。彬至,捕鞫之,得所受賂,致之法,自是詔親事官毋得出都城。易右班殿直,辭不受。後以秘書省著作佐郎通判筠州,歷知撫州。撫州民李甲、饒英恃財武斷鄉曲,縣莫能制。甲從子詈縣令,人告甲語斥乘輿。彬按治之,索
 其家得所藏兵械,又得服器有龍鳳飾,甲坐大逆棄市。並按英嘗強取人孥,配嶺南,州里肅然。



 擢提點荊湖南路刑獄,徙知潭州,入判三司戶部勾院,出為京西轉運使,徙河北。部吏馬崇正倚章獻太后姻家豪橫不法,彬發其奸贓,下吏。忤太后意,徙京東,又徙河東、陜西。復為三司鹽鐵判官,判都理欠、憑由司,累遷太常少卿,卒。



 仲簡字畏之,揚州江都人。以貧,傭書楊億門下,億教以詩賦,遂舉進士。歷通判鄭州、河南府推官。改秘書省著
 作佐郎、知蕪湖縣,通判楚州,累遷尚書都官員外郎。改侍御史、安撫京東,遷知真州,入為三司度支判官。經制陜西糧草,就遷兵部員外郎、直史館、知陜州。徙江東轉運使,除侍御史知雜事,為三司鹽鐵副使、工部郎中。奉使陜西,多任喜怒,以馬棰擊軍士流血,仁宗面詰之,不能對,出為河東轉運使。



 逾年,復為鹽鐵副使,再遷兵部,擢天章閣待制、知廣州。儂智高犯邕州,沿江而下,人告急,簡輒囚之,仍榜於道,敢妄言惑眾者斬,以是人不復
 為避賊計。比智高至,始令民入城,民爭道,競以金帛遺閽者,相蹂踐至死者甚多,其不得入者,皆附賊。賊既去,以其能守城,徙知荊南。既而言者論之,遂落職,又降刑部郎中、知筠州。復為兵部郎中,徙洪州,卒。



 論曰:士抱一藝者,思奮勵以功名自效,況其設施見於政事者乎?方偕、曹穎叔、楊告、趙及、王彬之流皆文吏,能推恩行利,鏟煩去蠹,其治不下古人。劉元瑜、劉湜輩亦不減此數人,然而元瑜譏詆餘靖,湜文致尹洙,公議所
 不與也。仲簡小才,所謂斗筲之器也,何足道哉!



\end{pinyinscope}