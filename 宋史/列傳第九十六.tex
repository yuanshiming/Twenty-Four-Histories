\article{列傳第九十六}

\begin{pinyinscope}

 範鎮
 從子百祿從孫祖禹



 範鎮,字景仁,成都華陽人。薛奎守蜀,一見愛之,館於府舍,俾與子弟講學。鎮益自謙退,每步行趨府門,逾年,人不知其為帥客也。及還朝,載以俱。有問奎入蜀何所得,
 曰:「得一偉人,當以文學名世。」宋庠兄弟見其文,自謂弗及,與為布衣交。



 舉進士,禮部奏名第一。故事,殿廷唱第過三人,則首禮部選者,必越次抗聲自陳,率得置上列。吳育、歐陽修號稱耿介,亦從眾。鎮獨不然,同列屢趣之,不為動。至第七十九人,乃從呼出應,退就列,無一言,廷中皆異之。自是舊風遂革。



 調新安主簿,西京留守宋綬延置國子監,薦為東監直講。召試學士院,當得館閣校理,主司妄以為失韻,補校勘。人為忿鬱,而鎮處之晏如。
 經四年,當遷,宰相龐籍言:「鎮有異材,不汲汲於進取。」超授直秘閣,判吏部南曹、開封府推官。擢起居舍人、知諫院。上疏論:「民力困敝,請約祖宗以來官吏兵數,酌取其中為定制,以今賦入之數什七為經費,儲其三以備水旱非常。」又言:「周以塚宰制國用,唐以宰相判鹽鐵、度支。今中書主民,樞密主兵,三司主財,各不相知。財已匱,樞密益兵無窮;民已困,三司取財不已。請使二府通知兵民大計,與三司同制國用。」



 契丹使至,虛聲示強,大臣益
 募兵以塞責,歲費百千萬。鎮言:「備契丹莫若寬三晉之民,備靈夏莫若寬秦民,備西南莫若寬越、蜀之民,備天下莫若寬天下之民。夫兵所以衛民而反殘民,臣恐異日之憂不在四夷,而在冗兵與窮民也。」



 商人輸粟河北,取償京師,而榷貨不即予鈔,久而鬻之,十才得其六。或建議出內帑錢,稍增價與市,歲可得羨息五十萬。鎮謂:「外府內帑,均為有司。今使外府滯商人,而內帑乘急以牟利,至傷國體。」仁宗遽止之。



 葬溫成後,太常議禮,前謂
 之園,後謂之陵,宰相劉沆前為監護使,後為園陵使。鎮曰:「嘗聞法吏舞法矣,未聞禮官舞禮也。請詰前後議禮異同狀。」集賢樣理刁約論壙中物侈麗,吳充、鞠真卿爭論禮,並補外,皆上章留之。石全斌護葬,轉觀察使,他吏悉優遷兩官。鎮言:「章獻、章懿、章惠三後之葬,推恩皆無此比。乞追還全斌等告敕。」副都知任守忠、鄧保吉同日除官,內臣無故改官者又五六人。時有敕,凡內降非準律令者,並許執奏。曾未一月,大臣輒廢不行。鎮乞正中
 書、樞密之罪,以示天下。



 帝天性寬仁,言事者競為激訐,至污人以帷箔不可明之事。鎮獨務引大體,非關朝廷安危,生民利疚,則闊略不言。陳執中為相,鎮論其無學術,非宰相器。及嬖妾笞殺婢,御史劾奏,欲逐去之。鎮言:「今陰陽不和,財匱民困,盜賊滋熾,獄犴充斥,執中當任其咎。御史舍大責細,暴揚燕私,若用此為進退,是因一婢逐宰相,非所以明等級,辨堂陛。」識者韙之。



 文彥博、富弼入相,詔百官郊迎。鎮曰:「隆之以虛禮,不若推之以至
 誠。陛下用兩人為相,舉朝皆謂得人。然近制,兩制不得詣宰相居第,百官不得間見,是不推之以誠也。願罷郊迎,除謁禁,則於御臣之術為兩得矣。」議減任子及每歲取士,皆自鎮發之。又乞令宗室疏屬補外官,帝曰:「卿言是也。顧恐天下謂朕不能睦族耳。」鎮曰:「陛下甄別其賢者用之,不沒其能,乃所以睦族也。」雖不行,至熙寧初,卒如其言。



 帝在位三十五年,未有繼嗣。嘉祐初,暴得疾,中外大小之臣,無不寒心,莫敢先言者。鎮獨奮曰:「天下事
 尚有大於此者乎?」即拜疏曰:「置諫官者,為宗廟社稷計。諫官而不以宗廟社稷計事陛下,是愛死嗜利之人,臣不為也。方陛下不豫,海內皇皇莫知所為,陛下獨以祖宗後裔為念,是為宗廟之慮,至深且明也。昔太祖舍其子而立太宗,天下之大公也。真宗以周王薨,養宗子於宮中,天下之大慮也。願以太祖之心,行真宗故事,拔近屬之尤賢者,優其禮秩,置之左右,與圖天下事,以系億兆人心。」疏奏,文彥博使客問何所言,以實告,客曰:「如是,
 何不與執政謀?」鎮曰:「自分必死,故敢言。若謀於執政,或以為不可,豈得中輟乎?」章累上,不報。執政諭之曰:「奈何效希名干進之人。」鎮貽以書曰:「比天象見變,當有急兵,鎮義當死職,不可死亂兵之下。此乃鎮擇死之時,尚何顧希名干進之嫌哉?」又言:「陛下得臣疏,不以留中而付中書,是欲使大臣奉行也。臣兩至中書,大臣皆設辭拒臣,是陛下欲為宗廟社稷計,而大臣不欲也。臣竊原大臣畏避之意,恐行之而陛下中變耳。中變之禍,不過一
 死。國本不立,萬一有如天象所告急兵之變,死且有罪,其為計亦已疏矣。願以臣章示大臣,使其自擇死所。」聞者股慄。



 除兼侍御史知雜事,鎮以言不從,固辭。執政諭鎮曰:「今間言已入,為之甚難。」鎮復書執政曰:「事當論其是非,不當問其難易。諸公謂今日難於前日,安知異日不難於今日乎?」凡見上面陳者三,言益懇切。鎮泣,帝亦泣,曰:「朕知卿忠,卿言是也,當更俟三二年。」章十九上,待命百餘日,須發為白。朝廷知不能奪,乃罷知諫院,改集
 賢殿修撰,糾察在京刑獄,同修起居注,遂知制誥。鎮雖解言職,無歲不申前議。見帝春秋益高,每因事及之,冀以感動帝意。至是,因入謝,首言:「陛下許臣,今復三年矣,願早定大計。」又因祫享,獻賦以諷。其後韓琦遂定策立英宗。



 遷翰林學士。中書議追尊濮王,兩制、臺諫與之異,詔禮官檢詳典禮。鎮判太常寺,率其屬言:「漢宣帝於昭帝為孫,光武於平帝為祖,其父容可稱皇考,議者猶非之,謂其以小宗合大宗之統也。今陛下既以仁宗為考,又
 加於濮王,則其失非特漢二帝比。凡稱帝若考,若寢廟,皆非是。」執政怒,召鎮責曰:「方令檢詳,何遽列上!」鎮曰:「有司得詔,不敢稽留,即以聞,乃其職也。奈何更以為罪乎?」會草制,誤遷宰相官,改侍讀學士。



 明年,還翰林,出知陳州。陳方饑,視事三日,擅發錢粟以貸。監司繩之急,即自劾,詔原之。是歲大熟,所貸悉還。神宗即位,復為翰林學士兼侍讀、知通進銀臺司。故事,門下封駁制旨,省審章奏,糾擿違滯,皆著所授敕,後乃刊去。鎮始請復之,使知
 所守。



 王安石改常平為青苗,鎮言:「常平之法,起於漢盛時,視穀貴賤發斂,以便農末,最為近古,不可改。而青苗行於唐之衰世,不足法。且陛下疾富民之多取而少取之,此正百步、五十步之間耳。今有兩人坐市貿易,一人故下其直以相傾,則人皆知惡之,可以朝廷而行市道之所惡乎?」呂惠卿在邇英言:「今預買紬絹,亦青苗之比。」鎮曰:「預買,亦敝法也。若府庫有餘,當並去之,豈應援以為比。」韓琦極論新法之害,送條例司疏駁,李常乞罷青
 苗錢,詔命分析,鎮皆封還。詔五下,鎮執如初。司馬光辭樞密副使,詔許之,鎮再封還。帝以詔直付光,不由門下。鎮奏曰:「由臣不才,使陛下廢法,有司失職,乞解銀臺司。」



 舉蘇軾諫官,御史謝景溫奏罷之;舉孔文仲制科,文仲對策,論新法不便,罷歸故官。鎮皆力爭之,不報。即上疏曰:「臣言不行,無顏復立於朝,請謝事。臣言青苗不見聽,一宜去;薦蘇軾、孔文仲不見用,二宜去。李定避持服,遂不認母,壞人倫,逆天理,而欲以為御史,御史臺為之罷
 陳薦,舍人院為之罷宋敏求、呂大臨、蘇頌,諫院為之罷胡宗愈。王韶上書肆意欺罔,以興造邊事,事敗,則置而不問,反為之罪帥臣李師中。及御史一言蘇軾,則下七路掎摭其過;孔文仲則遣之歸任。以此二人況彼二人,事理孰是孰非,孰得孰失,其能逃聖鑒乎?言青苗有見效者,不過歲得什百萬緡錢,緡錢什百萬,非出於天,非出於地,非出於建議者之家,蓋一出於民耳。民猶魚也,財猶水也,養民而盡其財,譬猶養魚而竭其水也。」



 疏五
 上,其後指安石用喜怒為賞罰,曰:「陛下有納諫之資,大臣進拒諫之計;陛下有愛民之性,大臣用殘民之術。臣知言入觸大臣之怒,罪且不測。然臣職獻替而無一言,則負陛下矣。」疏入,安石大怒,持其疏至手顫,自草制極詆之。以戶部侍郎致仕,凡所得恩典,悉不與。鎮表謝,略曰:「願陛下集群議為耳目,以除壅蔽之奸,任老成為腹心,以養和平之福。」天下聞而壯之。安石雖詆之深切,人更以為榮。既退,蘇軾往賀曰:「公雖退,而名益重矣!」鎮愀
 然曰:「君子言聽計從,消患於未萌,使天下陰受其賜,無智名,無勇功;吾獨不得為此,使天下受其害而吾享其名,吾何心哉!」日與賓客賦詩飲酒,或勸使稱疾杜門,鎮曰:「死生禍福,天也,吾其如天何!」同天節乞隨班上壽,許之,遂為令。軾得罪,下臺獄,索與鎮往來書文甚急,猶上書論救。久之,徙居許。



 哲宗立,韓維言:「鎮在仁宗時,首啟建儲之議,未嘗以語人,人亦莫為言者。」具以十九疏上之。拜端明殿學士,起提舉中太一宮兼侍讀,且欲以為
 門下侍郎。鎮雅不欲起,從孫祖禹亦勸止之,遂固辭,改提舉崇福宮。祖禹謁告歸省,詔賜以龍茶,存勞甚渥。復告老,以銀青光祿大夫再致仕,累封蜀郡公。



 鎮於樂尤注意,自謂得古法,獨主房庶以律生尺之說。司馬光謂不然,往復論難,凡數萬言。初,仁宗命李照改定大樂,下王樸樂三律。皇祐中,又詔胡瑗等考正。神宗時詔鎮與劉幾定之。鎮曰:「定樂當先正律。」神宗曰:「然,雖有師曠之聰,不以六律不能正五音。」鎮作律尺、龠合、升斗、豆區、釜
 斛,欲圖上之,又乞訪求真黍,以定黃鐘。而劉幾即用李照樂,加用四清聲而奏樂成。詔罷局,賜繼有加。鎮曰:「此劉幾樂也,臣何與焉。」至是,乃請太府銅為之,逾年而成,比李照樂下一律有奇。帝及太皇太后御延和殿,召執政同閱視,賜詔嘉獎。下之太常,詔三省、侍從、臺閣之臣,皆往觀焉。鎮時已屬疾,樂奏三日而薨,年八十一。贈金紫光祿大夫,謚曰忠文。



 鎮平生與司馬光相得甚歡,議論如出一口,且約生則互為傳,死則作銘。光生為《鎮傳》,
 服其勇決;鎮復銘光墓云:「熙寧奸朋淫縱,險詖憸猾,賴神宗洞察於中。」其辭峭峻。光子康屬蘇軾書之,軾曰:「軾不辭書,懼非三家福。」乃易他銘。



 鎮清白坦夷,遇人必以誠,恭儉慎默,口不言人過。臨大節,決大議,色和而語壯,常欲繼之以死,雖在萬乘前,無所屈。篤於行義,奏補先族人而後子孫,鄉人有不克婚葬者,輒為主之。兄鎡,卒於隴城,無子,聞其有遺腹子在外,鎮時未仕,徒步求之兩蜀間,二年乃得之,曰:「吾兄異於人,體有四乳,則兒亦
 必然。」已而果然,名曰百常。少受學於鄉先生龐直溫,直溫子昉卒於京師,鎮娶其女為孫婦,養其妻子終身。



 其學本《六經》,口不道佛、老、申、韓之說。契丹、高麗皆傳誦其文。少時賦《長嘯》,卻胡騎,晚使遼,人相目曰:此「長嘯公」也。兄子百祿亦使遼,遼人首問鎮安否。



 百祿字子功,鎮兄鍇之子也。第進士,又舉才識兼茂科。時治平水災,大臣方議濮禮,百祿對策曰:「簡宗廟、廢祭祀,則水不潤下。昔漢哀尊共皇,河南、穎川大水;孝安尊
 德皇,京師、郡國二十九大水。蓋大宗隆,小宗殺;宗廟重,私祀輕。今宜殺而隆,宜輕而重,是悖先王之禮。禮一悖,則人心失而天意睽,變異所由起也。」對入三等。



 熙寧中,鄧綰舉為御史,辭不就。提點江東、利、梓路刑獄,加直集賢院。利州武守周永懿以賄敗,百祿請復至道故事,用文吏領兵,以轄邊界,從之。熊本治瀘蠻事,有夷酋力屈請降,裨將賈昌言欲殺以為功,百祿諭之不聽,往謂本曰:「殺降不祥,活千人者封子孫。奈何容驕將橫境內乎?」本
 矍然,即檄止之。



 七年,召知諫院。屬歲旱,請講求急務,收還法令之未便者,以救將死之民。論手實法曰:「造薄手實,許令告匿。戶令雖有手實之文,而未嘗行。蓋謂使人自占,必不以實告,而明許告訐,人將為仇。然則禮、義、廉、恥之風衰矣。」五路置三十七將,專督所部兵,至許闢置布衣參軍謀。百祿察其中,或以恩澤市,或以□敗收,或未歷邊方,或起於群盜,疏列其亡狀者十四人,請仍舊制,將佐顓教閱,餘付之州縣,事多施行。



 與徐禧治李士
 寧獄,奏士寧熒惑童婦,致不軌生心,罪死不赦。禧右士寧,以為無罪。執政主禧,貶百祿監宿州酒。元豐末,入為司門吏部郎中、起居郎。



 哲宗立,遷中書舍人。司馬光復差役法,患吏受賕,欲加流配。百祿固爭曰:「民今日執事,受謝於人,明日罷役,則以財賂人。茍繩以重典,黥面赭衣必將充塞道路。」光悟曰:「微君言,吾不悉也。」遂已。



 元祐元年,為刑部侍郎。諸郡以故鬥殺情可矜者請讞,法官曰:「宜貸。」光曰:「殺人不死,法廢矣。」百祿曰:「謂之殺人,則可;若
 制刑以為無足疑,原情以為無足憫,則不可。今概之死,則二殺之科,自是遂無足疑憫者矣。」時又詔天下獄不當讞而輒讞者抵罪。有司重於請,至枉情以求合法。百祿曰:「熙寧之法,非可疑可憫而讞者免駁勘,元豐則刊之,近則有奏劾之詔,故官吏畏避,不憚論殺。」因條五年死貸之數以聞。門下省猶駁正當貸者,又例在有司者還中書,百祿又爭之,後悉從其請。



 改吏部侍郎。議者欲汰胥吏,呂大防趣廢其半,百祿曰:「不可。廢半則失職者
 眾,不若以漸消之,自今闕吏勿補,不數歲,減斯過半矣。」不聽。



 都水王孝先議回河故道,大防意向之,命百祿行視。百祿以東流高仰,而河勢順下,不可回,即馳奏所以然之狀,且取神宗詔令勿塞故道者並上之。大防猶謂:「大河東流,中國之險限。今塘濼既壞,界河淤淺,河且北注矣。」百祿言:「塘濼有限寇之名,無御寇之實。借使河徙而北,敵始有下流之憂,乃吾之利也。先帝明詔具在,奈何妄動搖之。」乃止。



 俄兼侍讀,進翰林學士。為帝言分別
 邪正之目,凡導人主以某事者為公正,某事者為奸邪,以類相反,凡二十餘條。願概斯事以觀其情,則邪正分矣。



 以龍圖閣學士知開封府。勤於民事,獄無系囚。僚吏欲以圄空聞,百祿曰:「千里之畿,無一人之獄,此至尊之仁,非尹功也。」不許。經數月,復為翰林學士,拜中書侍郎。是歲郊祀,議合祭天地,禮官以「昊天有成命」為言。百祿曰:「此三代之禮,奈何復欲合祭乎?『成命』之頌,祀天祭地,均歌此詩,亦如春夏祈穀而歌《噫嘻》,亦豈為一祭哉?」爭
 久不決,質於帝前。宰相曰:「百祿之言,禮經也;今日之用,權制也。陛下始郊見,宜以並事天地為恭。」於是合祭。



 熙河範育言:「阿裏骨酷暴且病,溫溪心八族皆思內附,可以計納。」百祿曰:「中國以信撫四夷,阿裏骨未有過,溪心虛實未可知,無釁而動,非策也。」又請進築納迷等三城,百祿曰:「是皆良田,為必爭之地,我既城之,若賊騎時出,我何以耕?後雖欲棄之,為費已甚,亦不能矣。」帝皆從之。右僕射蘇頌坐稽留除書免,百祿以同省罷為資政殿
 學士、知河中,徙河陽、河南。薨,年六十五,贈銀青光祿大夫。



 子祖述,監穎州酒稅,攝獄掾,閱具獄,活兩死囚,州人以為神。知鞏縣,鑿南山導水入洛,縣無水患,文彥博稱其能。以父墮黨籍,監中嶽廟。久之,通判涇州。知臺州,奏罷黃甘、葛蕈之貢。主管西京御史臺。靖康多難,避地至汝州。汝守趙子櫟邀與共守,於是旁郡盡陷,汝獨全。累官朝議大夫,卒。從弟祖禹。



 祖禹字淳甫,一字夢得。其生也,母夢一偉丈夫被
 金甲入寢室,曰:「吾漢將軍鄧禹。」既寤,猶見之,遂以為名。幼孤,叔祖鎮撫育如己子。祖禹自以既孤,每歲時親賓慶集,慘怛若無所容,閉門讀書,未嘗預人事。既至京師,所與交游,皆一時聞人。鎮器之曰:「此兒,天下士也。」



 進士甲科。從司馬光編修《資治通鑒》,在洛十五年,不事進取。書成,光薦為秘書省正字。時王安石當國,尤愛重之。王安國與祖禹友善,嘗諭安石意,竟不往謁。富弼致仕居洛,素嚴毅,杜門罕與人接,待祖禹獨厚;疾篤,召授以密疏,大
 抵論安石誤國及新法之害,言極憤切。弼薨,人皆以為不可奏,祖禹卒上之。



 神宗崩,祖禹上疏論喪服之制曰:「先王制禮,君服同於父,皆斬衰三年,蓋恐為人臣者不以父事其君。自漢以來,不惟人臣無服,人君遂不為三年之喪。國朝自祖宗以來,外廷雖用易月之制,宮中實行三年服。君服如古典,而臣下猶依漢制,故十二日而小祥,期而又小祥,二十四日而大祥,再期而又大祥。既以日為之,又以月為之,此禮之無據者也。古者再期而
 大祥,中月而禫。禫,祭之名,非服之色。今乃為之慘服三日然後禫,此禮之不經者也。服既除,至葬又服之,祔廟後即吉,才八月而遽純吉,無所不佩,此又禮之無漸者也。朔望,群臣朝服以造殯宮,是以吉服臨喪;人主衰服在上,是以先帝之服為人主之私喪,此二者皆禮之所不安也。」



 哲宗立,擢右正言。呂公著執政,祖禹以婿嫌辭,改祠部員外郎,又辭。除著作佐郎、修《神宗實錄》檢討,遷著作郎兼侍講。



 神宗既祥,祖禹上疏宣仁後曰:「今即吉
 方始,服御一新,奢儉之端,皆由此起。凡可以蕩心悅目者,不宜有加於舊。皇帝聖性未定,睹儉則儉,睹奢則奢,所以訓導成德者,動宜有法。今聞奉宸庫取珠,戶部用金,其數至多,恐增加無已,願止於未然。崇儉敦樸,輔養聖性,使目不視靡曼之色,耳不聽淫哇之聲,非禮勿言,非禮勿動,則學問日益,聖德日隆,此宗社無疆之福。」故事,服除當開樂置宴,祖禹以為因除服而開樂設宴,則似除服而慶賀,非君子不得已而除之之意,不可。



 夏暑
 權罷講,祖禹言:「陛下今日之學與不學,系他日治亂。如好學,則天下君子欣慕,願立於朝,以直道事陛下,輔佐德業,而致太平;不學,則小人皆動其心,務為邪諂,以竊富貴。且凡人之進學,莫不於少時,今聖質日長,數年之後,恐不得如今日之專,竊為陛下惜也。」遷起居郎,又召試中書舍人,皆不拜。呂公著薨,召拜右諫議大夫。首上疏論人主正心修身之要,乞太皇太后日以天下之勤勞、萬民之疾苦、群臣之邪正、政事之得失,開導上心,曉
 然存之於中,使異日眾說不能惑,小人不能進。



 蔡確既得罪,祖禹言:「自乾興以來,不竄逐大臣六十餘年,一旦行之,流傳四方,無不震聳。確去相已久,朝廷多非其黨,間有偏見異論者,若一切以為黨確去之,懼刑罰失中,而人情不安也。」



 蔡京鎮蜀,祖禹言:「京小有才,非端良之士。如使守成都,其還,當使執政,不宜崇長。」時大臣欲於新舊法中有所創立。祖禹以為:「朝廷既察王安石之法為非,但當復祖宗之舊,若出於新舊之間,兩用而兼存
 之,紀綱壞矣。」遷給事中。



 吳中大水,詔出米百萬斛、緡錢二十萬振救。諫官謂訴災者為妄,乞加驗考。祖禹封還其章,云:「國家根本,仰給東南。今一方赤子,呼天赴訴,開口仰哺,以脫朝夕之急。奏災雖小過實,正當略而不問。若稍施懲譴,恐後無復敢言者矣。」



 兼國史院修撰,為禮部侍郎。論擇監司守令曰:「祖宗分天下為十八路,置轉運使、提點刑獄,收鄉長、鎮將之權悉歸於縣,收縣之權歸於州,州之權歸於監司,監司之權歸於朝廷。上下相
 維,輕重相制,建置之道,最為合宜。監司付以一路,守臣付以一州,令宰付以一縣,皆與天子分土而治,其可不擇乎?祖宗嘗有考課之法,專察諸路監司,置簿於中書,以稽其要。今宜委吏部尚書,取當為州者,條別功狀以上三省,三省召而察之,茍其人可任,則以次表用之。至官,則令監司考其課績,終歲之後,可以校優劣而施黜陟焉。如此則得人必多,監司、郡守得人,縣令不才,非所患也。」



 聞禁中覓乳媼,祖禹以帝年十四,非近女色之時,
 上疏勸進德愛身,又乞宣仁後保護上躬,言甚切至。既而宣仁諭祖禹,以外議皆虛傳,祖禹復上疏曰:「臣言皇帝進德愛身,宜常以為戒。太皇太后保護上躬,亦願因而勿忘。今外議雖虛,亦足為先事之戒。臣侍經左右,有聞於道路,實懷私憂,是以不敢避妄言之罪。凡事言於未然,則誠為過;及其已然,則又無所及,言之何益?陛下寧受未然之言,勿使臣等有無及之悔。」拜翰林學士,以叔百祿在中書,改侍講學士。百祿去,復為之。範氏自鎮
 至祖禹,比三世居禁林,士論榮慕。



 宣仁太后崩,中外議論洶洶,人懷顧望,在位者畏懼,莫敢發言。祖禹慮小人乘間害政,乃奏曰:「陛下方攬庶政,延見群臣,此國家隆替之本,社稷安危之機,生民休戚之端,君子小人進退消長之際,天命人心去就離合之時也,可不畏哉?先後有大功於宗社,有大德於生靈,九年之間,始終如一。然群小怨恨,亦為不少,必將以改先帝之政、逐先帝之臣為言,以事離間,不可不察也。先後因天下人心,變而更
 化。既改其法,則作法之人有罪當退,亦順眾言而逐之。是皆上負先帝,下負萬民,天下之所仇疾而欲去之者也,豈有憎惡於其間哉?惟辨析是非,深拒邪說,有以奸言惑聽者,付之典刑,痛懲一人,以警群慝,則貼然無事矣。此等既誤先帝,又欲誤陛下,天下之事,豈堪小人再破壞邪?」初,蘇軾約俱上章論列,諫草已具,見祖禹疏,遂附名同奏,曰:「公之文,經世之文也。」竟不復出其稿。



 祖禹又言:「陛下承六世之遺烈,當思天下者祖宗之天下,人
 民者祖宗之人民,百官者祖宗之百官,府庫者祖宗之府庫。一言一動,如臨之在上,質之在傍,則可以長享天下之奉。先後以大公至正為心,罷安石、惠卿所造新法,而行祖宗舊政。故社稷危而復安,人心離而復合,乃至遼主亦戒其臣勿生事曰:『南朝專行仁宗之政矣。』外夷之情如此,中國之人心可知。先後日夜苦心勞力,為陛下立太平之基。願守之以靜,恭己以臨之,虛心以處之,則群臣邪正,萬事是非,皆了然於聖心矣。小人之情專
 為私,故不便於公;專為邪,故不便於正;專好動,故不便於靜。惟陛下痛心疾首,以為刻骨之戒。」章累上,不報。



 忽有旨召內臣十餘人,祖禹言:「陛下親政以來,四海傾耳,未聞訪一賢臣,而所召者乃先內侍,必謂陛下私於近習,望即賜追改。」因請對,曰:「熙寧之初,王安石、呂惠卿造立新法,悉變祖宗之政,多引小人以誤國,勛舊之臣屏棄不用,忠正之士相繼遠引。又用兵開邊,結怨外夷,天下愁苦,百姓流徙。賴先帝覺悟,罷逐兩人,而所引群小,
 已布滿中外,不可復去。蔡確連起大獄,王韶創取熙河,章惇開五溪,沉起擾交管,沈括、徐禧、俞充、種諤興造西事,兵民死傷皆不下二十萬。先帝臨朝悼悔,以謂朝廷不得不任其咎。以至吳居厚行鐵冶之法於京東,王子京行茶法於福建,蹇周輔行鹽法於江西,李稷、陸師閔行茶法、市易於西川,劉定教保甲於河北,民皆愁痛嗟怨,比屋思亂。賴陛下與先後起而救之,天下之民,如解倒縣。惟是向來所斥逐之人,窺伺事變,妄意陛下不以
 修改法度為是,如得至左右,必進奸言。萬一過聽而復用之,臣恐國家自此陵遲,不復振矣。」又論:「漢、唐之亡,皆由宦官。自熙寧、元豐間,李憲、王中正、宋用臣輩用事總兵,權勢震灼。中正兼乾四路,口敕募兵,州郡不敢違,師徒凍餒,死亡最多;憲陳再舉之策,致永樂摧陷;用臣興土木之工,無時休息,罔市井之微利,為國斂怨。此三人者,雖加誅戮,未足以謝百姓。憲雖已亡,而中正、用臣尚在,今召內臣十人,而憲、中正之子皆在其中。二人既入,
 則中正、用臣必將復用,願陛下念之。」



 時紹述之論已興,有相章惇意。祖禹力言惇不可用,不見從,遂請外。上且欲大用,而內外梗之者甚眾,乃以龍圖閣學士知陜州。言者論祖禹修《實錄》詆誣,又摭其諫禁中雇乳媼事,連貶武安軍節度副使、昭州別駕,安置永州、賀州,又徙賓、化而卒,年五十八



 祖禹平居恂恂,口不言人過。至遇事,則別白是非,不少借隱。在邇英守經據正,獻納尤多。嘗講《尚書》至「內作色荒,外作禽荒」六語,拱手再誦,卻立云:「
 願陛下留聽。」帝首肯再三,乃退。每當講前夕,必正衣冠,儼如在上側,命子弟侍,先按講其說。開列古義,參之時事,言簡而當,無一長語,義理明白,粲然成文。蘇軾稱為講官第一。



 祖禹嘗進《唐鑒》十二卷,《帝學》八卷,《仁宗政典》六卷。而《唐鑒》深明唐三百年治亂,學者尊之,目為「唐鑒公」云。建炎二年,追復龍圖閣學士。子沖,紹興中仕至翰林侍讀學士,《儒林》有傳。



 論曰:熙寧、元豐之際,天下賢士大夫望以為相者,鎮與
 司馬光二人,至稱之曰君實、景仁,不敢有所軒輊。光思濟斯民,卒任天下之重;鎮嶷然如山,確乎其不可拔。君子之道,或出或處,易地則皆然,未易以功名優劣論也。百祿受學於鎮,故其議論操修,粹然一出於正。祖禹長於勸講,平生論諫,不啻數十萬言。其開陳治道,區別邪正,辨釋事宜,平易明白,洞見底蘊,雖賈誼、陸贄不是過雲。



\end{pinyinscope}