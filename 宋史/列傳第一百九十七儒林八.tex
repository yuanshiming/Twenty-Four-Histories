\article{列傳第一百九十七儒林八}

\begin{pinyinscope}

 ○湯漢何基王柏徐夢莘弟得之從子天麟附李心傳葉味道王應麟黃震



 湯漢,字伯紀,饒州安仁人。與其兄幹、巾、中皆知名當時,
 柴中行見而奇之。真德秀在潭,致漢為賓客。嘗造趙汝談,汝談曰:「第一流也。」江東提刑趙汝騰薦漢於朝,詔免解差,充象山書院堂長。赴禮部別院試,正奏名,授上饒縣主簿。江東轉運使趙希塈言:「漢,今海內知名士也,豈得吏之州縣哉!」詔循兩資,差信州教授兼象山書院長。



 淳祐十二年,差充史館校勘,改國史實錄院校勘。會大水,上封事曰:「君心敬肆之分,實上天喜怒之由。一念之敬,上帝臨汝,祥風慶雲所從出也。一念之肆,上帝震怒,
 妖浸陰沴所從生也。」火災,應詔上封事曰:



 臣聞任天下之大,立心不可不公;守天下之重,持心不可不敬。陛下膺皇天之眷命,受祖宗之寶圖,則不當懷私恩;為天下共主,為億兆寄命,則不當隆私親。大臣邇臣,服休服採,皆陛下所倚仗也,則不當信私人。三省、密院者,陛下之朝廷,發號布政所從出也,則不當有私令。四海九州,土宇昄章,皆陛下之倉廩府庫也,則不當殖私財。陛下於皇天祖宗之德弗永念,而報答私恩;於群黎百姓之疾
 苦弗深恤,而富貴私親;公卿在廷,其信任不若近習之篤;中書造命,其除行不若內批之專,則陛下之立心,既未能盡合乎天下之公矣。



 往者陛下上畏天戒,下恤人言,內則拘制於權臣,外則恐怯於強敵,敬心既不敢盡弛,則私意亦未得盡行。比年以來,天戒人言既以玩熟,而貪濁柄國,黷貨無厭,彼既將恣行其私,則不得不縱陛下之所欲為。於是前日之敬畏盡忘,而一念之私始四出而不可御矣。姑以近事跡之:定策之碑,忽從中出,
 鄉未欲親其文也;貴戚子弟,參錯中外,鄉不如是之放也;土木之禍,展轉流毒,訟牒細故,胥吏賤人皆得藉群榼之勢,徹清都之邃,鄉不如是之熾也;御筆之出,上則廢朝令,下則侵有司,鄉不如是之多也;賄賂之通,書致之操,鄉不如是其章也。



 故凡陛下之所以未能任大守重,而至於召怨宿禍者,始於立心之未公,成於持心之不敬,私以為主,而肆以行之。此所以感動天地,而水火之災捷出於數月之內也。陛下得不亟為治亂持危之
 計,而可復以常日玩易之心處之乎!



 授太學博士,轉對,言:「太祖之天下壞其半者,蔡京、王黼也。高宗之天下壞其半者,鄭清之也。」又曰:「茍有志焉,則其紀綱必先正,其根本必先強,其藩籬必先固。夫然後心廣體胖,泮渙而優游,其樂無極矣。舍此不務,而徒以九重之深、一笑之適以為樂。樂極而思之,吾有朝廷而不能治也,吾有黎民而無與保之也,起視四境,而外侮又至矣。雖有鄭、衛之音,燕、趙之色,建章之麗,瓊林之積,亦獨何樂哉!」



 召試
 館職,遷秘書省校書郎。皇太子冠,差充太常博士,引賓贊,受命進《冠箴》,詔令太子拜謝。升秘書郎,轉對,極言邊事,以為:「今日扶危救亂無復他策,在乎人主清心無欲,盡用天下之財力以治兵。大臣公心無我,盡用天下之人才以強本,庶幾尚有以亡為存之理耳。」



 提舉福建常平,劾福州守史巖之、泉州守謝埴。召為禮部郎官兼太子侍讀。尋以直華文閣、福建運判,改知寧國府。遷提舉江西常平兼知吉州。移江東運判、知隆興府。召為尚左
 郎官兼太子侍讀、兼玉牒所檢討官,入奏:「願陛下端本澄源,虛己盡下,恢大公之道,開不諱之門,使朝廷之上,光明洞達,而無邪孽之根以撓其正。四海之內,歡欣交通,而無怨戾之氣以奸其和。臣之忠愛,莫切於此。」



 遷太府少卿,升兼太子諭德,改秘書少監。疏論:「比年董宋臣聲焰薰灼,其力能去臺諫,排大臣,結連兇渠,惡德參會,以致兵戈相尋之禍。陛下灼見其故,斥而遠之,臣意其影滅而跡絕矣。豈料夫陰消而再凝,冰解而驟合,既得
 自便,即圖復用,以其罪戾之餘,一旦復使之出入壺奧之中,給事宗廟之內,此其重干神人之怒,再基禍亂之源,上下皇惑,大小切齒。而陛下方為之辨明,大臣方與之和解,臣竊重傷此過計也。自古小人復出,其害必慘,將逞其憤怨,嘯其儔伍,顛倒宇宙,陛下之威神有時而不得以自行,甚可畏也。」



 乞休致,擢太常少卿,太子以書勉留。求補外,以秘閣修撰知福州、福建安撫,改知隆興府。



 度宗即位,召奏事,授太常少卿兼國史院編修官、實
 錄院檢討官。遷起居郎兼侍讀,入奏,言:「願陛下持一敬心以正百度,則追養繼孝,所以報先帝者,必益致其隆,先意承志,所以事太母者,必益致其謹。其愛身也,必不以物欲撓其和平;其正家也,必不以私暱隳其法度。政事必出於朝廷,而預防於多門,人才必由於明揚,而深杜於邪徑。



 兼權中書舍人,權兵部侍郎,升兼同修國史、實錄院同修撰兼直學士。累請致仕,授華文閣待制、知寧國府,賜金帶。久之,又召為刑部侍郎兼侍讀,以龍圖
 閣待制知福州,福建安撫使。改知太平州、權工部尚書兼侍讀。以顯文閣直學士提舉玉隆宮。進華文閣學士,以端明殿學士致仕。卒,年七十一。特贈正奉大夫,謚文清。



 漢介潔有守,恬於進取,有文集六十卷。



 何基,字子恭,婺州金華人。父伯熭為臨川縣丞,而黃幹適知其縣事,伯熭見二子而師事焉。乾告以必有真實心地、刻苦工夫而後可,基悚惕受命。於是隨事誘掖,得聞淵源之懿。微辭奧義,研精覃思,平心易氣,以俟其通,
 未嘗參以己意,立異以為高,徇人而少變也。凡所讀無不加標點,義顯意明,有不待論說而自見者。



 朱熹門人楊與立一見推服。來學者眾,嘗謂:「為學立志貴堅,規模貴大,充踐服行,死而後已。讀《詩》之法,須掃蕩胸次凈盡,然後吟哦上下,諷詠從容,使人感發,方為有功。」謂:「以《洪範》參之《大學》、《中庸》,有不約而符者。」謂:「讀《易》者,當盡去其膠固支離之見,以潔凈其心,玩精微之理,沉潛涵泳,得其根源,乃可漸觀爻象。」蓋其確守師訓,故能精義造約。



 王柏既執贄為弟子,基謙抑不以師道自尊。柏高明絕識,序正諸經,弘論英辨,質問難疑,或一事至十往返,基終不變以待其定。嘗曰:「治經當謹守精玩,不必多起疑論。有欲為後學言者,謹之又謹可也。」基淳固篤實,絕類漢儒。雖一本於熹,然就其言發明,則精義新意愈出不窮。基文集三十卷,而與柏問辨者十八卷。



 郡守趙汝騰守婺,延聘請講,辭不就。復首薦於朝,又率名從官列薦。通判鄭士懿、守蔡抗、楊棟相繼以請,皆辭。景定五年,詔
 舉賢,特薦基與建人徐幾,同被命添差婺州學教授,兼麗澤書院山長,力辭未竟,理宗崩。咸淳初,授史館校勘兼崇政殿說書,屢辭,改承務郎,主管西嶽廟,終亦不受也。卒,年八十一。國子祭酒楊文仲請於朝,謚文定。



 所著《大學發揮》、《中庸發揮》、《大傳發揮》、《易啟蒙發揮》、《通書發揮》、《近思錄發揮》。



 王柏,字會之,婺州金華人。大父崇政殿說書師愈,從楊時受《易》、《論語》,既又從朱熹、張栻、呂祖謙游。父瀚,朝奉郎、
 主管建昌軍仙都觀,兄弟皆及熹、祖謙之門。



 柏少慕諸葛亮為人,自號長嘯。年逾三十,始知家學之原,捐去俗學,勇於求道。與其友汪開之著《論語通旨》,至「居處恭,執事敬」,惕然嘆曰:「長嘯非聖門持敬之道。」亟更以魯齋。



 從熹門人游,或語以何基嘗從黃幹得熹之傳,即往從之,授以立志居敬之旨,且作《魯齋箴》勉之。質實堅苦,有疑必從基質之。於《論語》、《大學》、《中庸》、《孟子》、《通鑒綱目》標注點校,尤為精密。作《敬齋箴圖》。夙興見廟,治家嚴飭。當暑閉
 閣靜坐,子弟白事,非衣冠不見也。



 少孤,事其伯兄甚恭。季弟早喪,撫其孤,又割田予之。收合宗族,周恤扶持之。開之沒,家貧,為之斂且葬焉。



 來學者眾,其教必先之以《大學》。蔡抗、楊棟相繼守婺,趙景緯守臺,聘為麗澤、上蔡兩書院師,鄉之耆德皆執弟子禮。理宗崩,率諸生制服臨於郡。



 柏之言曰:「伏羲則《河圖》以畫八卦,文王推八卦以合《河圖》者,先天後天之宗祖也。《河圖》是逐位奇偶之交,後天是統體奇偶之交,惟四生數不動。以四成數而
 下上之,上偶下奇,莫匪自然。」又曰:「大禹得《洛書》而列九疇,箕子得九疇而傳《洪範》,範圍之數,不期而暗合。《洪範》者,經傳之宗祖乎!『初一曰五行』以下六十五字為《洪範》,『五皇極』以下六十四字為皇極經,此帝王相傳之大訓,非箕子之言也。」又曰:「今《詩》三百五篇,豈盡定於夫子之手?所刪之詩,容或有存於閭巷浮薄之口,漢儒取於補亡。」乃定《二南》各十有一篇,兩兩相配。退《何彼穠矣》、《甘棠》歸之《王風》,削去《野有死麕》,黜鄭、衛淫奔之詩。又作《春秋
 發揮》。又曰:「《大學致知格物章》未嘗亡。」還《知止》章於《聽訟》之上。謂「《中庸》古有二篇,誠明可為綱,不可為目。」定《中庸》誠明各十一章,其卓識獨見多此類也。



 其卒,整衣冠端坐,揮婦人勿近。國子祭酒楊文仲請於朝,謚曰文憲。



 所著有《讀易記》、《涵古易說》、《大象衍義》、《涵古圖書》、《讀書記》、《書疑》、《詩辨說》、《讀春秋記》、《論語衍義》、《太極衍義》、《伊洛精義》、《研幾圖》、《魯經章句》、《論語通旨》、《孟子通旨》、《書附傳》、《左氏正傳》、《續國語》、《閫學之書》、《文章復古》、《文章續古》、《濂洛文統》、《擬道
 學志》、《朱子指要》、《詩可言》、《天文考》、《地理考》、《墨林考》、《大爾雅》、《六義字原》、《正始之音》、《帝王歷數》、《江左淵源》、《伊洛精義雜志》、《周子》、《發遣三昧》、《文章指南》、《朝華集》、《紫陽詩類》、《家乘》、文集。



 徐夢莘字商老,臨江人。幼慧,耽嗜經史,下至稗官小說,寓目成誦。紹興二十四年舉進士。歷官為南安軍教授。改知湘陰縣。會湖南帥括田,號增耕稅,他邑奉令惟謹。夢莘獨謂邑無新田,租稅無從出。帥恚其私於民,欲從簿書間攈摭其過,終莫能得,由是反器重之。



 尋主管廣
 西轉運司文字。時朝廷議易二廣鹽法,遣廣西安撫司干官胡廷直與東西漕臣集議於境。夢莘從行,謂:「廣西阻山,止當仍官般法,則害不及民。廣東諸郡並江,或可容客販,未宜遽以二廣概行。」議與廷直不合。廷直竟遂其說,以客販變法得為轉運使。夢莘既知賓州,猶以前議為梗法,罷去。不三年,二廣商賈毀業,民苦無鹽,復從官般法矣。



 夢莘恬於榮進,每念生於靖康之亂,四歲而江西阻訌,母襁負亡去得免。思究見顛末,乃網羅舊聞,
 會稡同異,為《三朝北盟會編》二百五十卷,自政和七年海上之盟,訖紹興三十一年完顏亮之斃,上下四十五年,凡曰敕、曰制、誥、詔、國書、書疏、奏議、記序、碑志,登載靡遺。帝聞而嘉之,擢直秘閣。



 夢莘平生多所著,有《集補》,有《會錄》,有《讀書記志》,有《集醫錄》,有《集仙錄》,皆以「儒榮」冠之。其嗜學博文,蓋孜孜焉死而後已者。開禧元年秋八月卒,年八十二。夢莘弟得之,從子天麟。



 得之字思叔,淳熙十年舉進士。部使者以廉吏薦,以通直郎致仕。安貧樂
 分,不貪不躁,著《左氏國紀》、《史記年紀》作《具敝篋筆略》、《鼓吹詞》、《郴江志》。



 天麟字仲祥,開禧元年進士。調撫州教授,歷湖廣總領所乾辦公事、臨安府教授、浙西提舉常平司干官、主管禮兵部架閣、宗學諭、武學博士。輪對,言人主當持心以敬。奉祠仙都觀,通判惠、潭二州,權英德府,權發遣廣西轉運判官。所至興學明教,有惠政。



 著《西漢會要》七十卷、《東漢會要》四十卷、《漢兵本末》一卷、《西漢地理疏》六卷、《山經》三十卷。既謝官,作亭蕭灘之上,畫嚴子
 陵像而事之。



 李心傳,字微之,宗正寺簿舜臣之子也。慶元元年薦於鄉,既下第,絕意不復應舉,閉戶著書。晚因崔與之、許奕、魏了翁等合前後二十三人之薦,自制置司敦遣至闕下。為史館校勘,賜進士出身,專修《中興四朝帝紀》。甫成其三,因言者罷,添差通判成都府。尋遷著作佐郎,兼四川制置司參議官。詔無入議幕,許闢官置局,踵修《十三朝會要》。端平三年成書。召赴闕,為工部侍郎,言:



 臣聞「大
 兵之後,必有兇年」。蓋其殺戮之多,賦斂之重,使斯民怨怒之氣,上幹陰陽之和,至於此極也。陛下所宜與諸大臣掃除亂政,與民更始,以為消惡運、迎善祥之計。而法弊未嘗更張,民勞不加振德,既無能改於其舊,而殆有甚焉。故帝德未至於罔愆,朝綱或苦於多紊,廉平之吏,所在鮮見,而貪利無恥,敢於為惡之人,挾敵興兵,四面而起,以求逞其所欲。如此而望五福來備,百穀用成,是緣木而求魚也。



 臣考致旱之由,曰和糴增多而民怨,曰
 流散無所歸而民怨,曰檢稅不盡實而民怨,曰籍貲不以罪而民怨。凡此皆起於大兵之後,而勢未有以消之,故愈積而愈極也。成湯聖主也,而桑林之禱,猶以六事自責。陛下願治,七年於此,災祥饑饉,史不絕書,其故何哉?朝令夕改,靡有常規,則政不節矣;行齎居送,略無罷日,則使民疾矣;陪都園廟,工作甚殷,則土木營矣;潛邸女冠,聲焰茲熾,則女謁盛矣;珍玩之獻,罕聞卻絕,則包苴行矣;鯁切之言,類多厭棄,則讒夫昌矣。此六事者一
 或有焉,猶足以致旱。願亟降罪己之詔,修六事以回天心。群臣之中有獻聚斂剽竊之論以求進者,必重黜之,俾不得以上誣聖德,則旱雖烈,猶可弭也。然民怨於內,敵逼於外,事窮勢迫,何所不至!陛下雖謀臣如云,猛將如雨,亦不知所以為策矣。



 帝從之。未幾,復以言去,奉祠居潮州。淳祐元年罷祠,復予,又罷。三年,致仕,卒,年七十有八。



 心傳有史才,通故實,然其作吳獵、項安世傳,褒貶有愧秉筆之旨。蓋其志常重川蜀,而薄東南之士云。



 所
 著成書,有《高宗系年錄》二百卷、《學易編》五卷、《誦詩訓》五卷、《春秋考》十三卷、《禮辨》二十三卷、《讀史考》十二卷、《舊聞證誤》十五卷、《朝野雜記》四十卷、《道命錄》五卷、《西陲泰定錄》九十卷、《辨南遷錄》一卷、詩文一百卷。



 葉味道,初諱賀孫,以字行,更字知道,溫州人。少刻志好古學,師事朱熹。試禮部第一。時偽學禁行,味道對學制策,率本程頤無所避。知舉胡紘見而黜之,曰:「此必偽徒也。」既下第,復從熹於武夷山中。學禁開,登嘉定十三年
 進士第,調鄂州教授。



 理宗訪問熹之徒及所著書,部使者遂以味道行誼聞,差主管三省架閣文字。遷宗學諭,輪對,言:「人主之務學,天下之福也。必堅志氣以守所學,謹幾微以驗所學,正綱常以勵所學,用忠言以充所學。」至若口奏,則又述帝王傳心之要,與四代作歌作銘之旨,其終有曰:「言宣則力減,文勝則意虛。」從臣有薦味道可為講官,乃授太學博士,兼崇政殿說書。



 故事,說書之職止於《通鑒》,而不及經。味道請先說《論語》,詔從之。帝忽
 問鬼神之理,疑伯有之事涉於誕。味道對曰:「陰陽二氣之散聚,雖天地不能易。有死而猶不散者,其常也。有不得其死而鬱結不散者,其變也。故聖人設為宗祧,以別親疏遠邇,正所以教民親愛,參贊化育。今伯有得罪而死,其氣不散,為妖為厲,使國人上下為之不寧,於是為之立子洩以奉其後,則庶乎鬼有所知,而神莫不寧矣。」蓋諷皇子竑事也。



 三京用師,廷臣邊閫交進機會之說。味道進議狀,以為:「開邊浸闊,應援倍難,科配日繁,饋餉
 日迫,民一不堪命,龐勛、黃巢之禍立見,是先搖其本,無益於外也。」經筵奏事,無日不申言之,而洛師尋以敗聞。於是人謂味道見微慮遠。



 味道所奏陳,無一言不開導引翼,求切於君身;旁引折旋,推致於治道。遷秘書著作佐郎而卒。訃聞,帝震悼,出內帑銀帛賻其喪,升一官以任其後,故事所未有也。



 所著《四書說》、《大學講義》、《祭法宗廟廟享郊社外傳》、《經筵口奏》、《故事講義》。



 王應麟,字伯厚,慶元府人。九歲通《六經》,淳祐元年舉進
 士,從王野受學。調西安主簿,民以年少易視之,輸賦後時。應麟白郡守,繩以法,遂立辦。諸校欲為亂,知縣事翁甫倉皇計不知所出,應麟以禮諭服之。差監平江百萬東倉。調浙西提舉常平茶鹽主管帳司,部使者鄭霖異待之。丁父憂,服除,調揚州教授。



 初,應麟登第,言曰:「今之事舉子業者,沽名譽,得則一切委棄,制度典故漫不省,非國家所望於通儒。」於是閉門發憤,誓以博學宏辭科自見,假館閣書讀之。寶祐四年中是科。應麟與弟應鳳
 同日生,開慶元年亦中是科,詔褒諭之,添差浙西安撫司干辦公事。



 帝御集英殿策士,召應麟覆考。考第既上,帝欲易第七卷置其首。應麟讀之,乃頓首曰:「是卷古誼若龜鏡,忠肝如鐵石,臣敢為得士賀。」遂以第七卷為首選。及唱名,乃文天祥也。遷主管三省、樞密院架閣文字。



 遷國子錄,進武學博士。疏言:「陛下閱理多,願治久。當事勢之艱,輿圖蹙於外患,人才乏而民力殫,宜強為善,增修德,無自沮怠;恢弘士氣,下情畢達,操綱紀而明委任,
 謹左右而防壅蔽,求哲人以輔後嗣。」既對,帝問其父名,曰:「爾父以陳善為忠,可謂繼美。」



 丁大全欲致應麟,不可得。遷太常寺主簿,面對,言:「淮戍方警,蜀道孔艱,海表上流皆有藩籬唇齒之憂。軍功未集而吝賞,民力既困而重斂,非修攘計也。陛下勿以宴安自逸,勿以容悅之言自寬。」帝愀然曰:「邊事甚可憂。」應麟言:「無事深憂,臨事不懼。願汲汲預防,毋為壅蔽所欺。」時大全諱言邊事,於是應麟罷。



 未幾,大全敗,起應麟通判臺州。召為太常博士,
 擢秘書郎,俄兼沂靖惠王府教授。彗星見,應詔極論執政、侍從、臺諫之罪,積私財、行公田之害。又言:「應天變莫先回人心,回人心莫先直受言。箝天下之口,沮直臣之氣,如應天何?」時直言者多迕權臣意,故應麟及之。遷著作佐郎。



 度宗即位,攝禮部郎官,草百官表。舊制,請聽政,四表已上。一夕入臨,宰臣諭旨增撰三表,應麟操筆立就。丞相總護還,辭位表三道,使者立以俟,應麟從容授之。丞相驚服,即授兼禮部郎官、兼直學士院。



 馬廷鸞知
 貢舉,詔應麟兼權直,俄兼崇政殿說書。遷著作郎,守軍器少監。經筵值人日雪,帝問有何故事,應麟以唐李嶠、李乂等應制詩對。因奏:「春雪過多,民生饑寒,方寸仁愛,宜謹感召。」遷將作監。



 帝視朝,謂應麟曰:「為學要灼見古人之心。」應麟對曰「嚴恭寅畏,不敢怠皇,克勤克儉,無自縱逸,強以馭下,制事以斷,此古人之心。然操舍易忽於眇綿,兢業每忘於游衍。」帝嘉納之。既而轉對,言:「人君防未萌之欲,存不已之誠。」擢兼侍立修注官,升權直學士
 院,遷秘書少監兼侍講。上疏論市舶,不報。



 會賈似道拜平章事,葉夢鼎、江萬里各求去,似道亦求去。應麟奏,孝宗朝闕相者亦逾年,帝亟取以諭之。似道聞應麟言,大惡之,語包恢曰:「我去朝士若王伯厚者多矣,但此人素著文學名,不欲使天下謂我棄士。彼盍思少自貶!」恢以告,應麟笑曰:「迕相之患小,負君之罪大。」遷起居舍人,兼權中書舍人。冬雷,應麟言:「十月之雷,惟東漢數見。命令不專,奸UB並進,卑逾尊,外陵內之象。當清天君,謹天命,
 體天德,以回天心。守成必法祖宗,御治必總威福。」似道聞之,斥逐之意決矣。



 應麟牒閣門直前奏對,謂用人莫先察君子小人。方袖疏待班,臺臣亟疏駁之,由是二史直前之制遂廢。以秘閣修撰主管崇禧觀。



 久之,起知徽州。其父捴嘗守是郡,父老皆曰:』此清白太守子也。」摧豪右,省租賦,民大悅。



 召為秘書監,權中書舍人,力辭,不許。兼國史編修、實錄檢討兼侍講。遷起居郎兼權吏部侍郎,指陳成敗逆順之說,且曰:「國家所恃者大江,襄、樊其
 喉舌,議不容緩。朝廷方從容如常時,事幾一失,豈能自安?」朝臣無以邊事言者,帝不懌。似道復謀斥逐,適應麟以母憂去。



 及似道潰師江上,授中書舍人兼直學士院,即引疏陳十事,急征討、明政刑、厲廉恥、通下情、求將材、練軍實、備糧餉、舉實材、擇牧守、防海道,其目也。且言:「圖大患者必略細故,求實效者必去虛文。」因請集諸路勤王之師,有能率先而至者,宜厚賞以作勇敢之氣,並力進戰,惟能戰,斯可守。進兼同修國史、實錄院同修撰兼
 侍讀,遷禮部侍郎兼中書舍人。日食,應麟詔論答天戒五事,陳備御十策,皆不及用。



 尋轉尚書兼給事中。左丞相留夢炎用徐囊為御史,擢江西制置使黃萬石等,應麟繳奏曰:「囊與夢炎同鄉,有私人之嫌,萬石粗戾無學,南昌失守,誤國罪大。今方欲引以自助,善類為所搏噬者,必攜持而去。吳浚貪墨輕躁,豈宜用之?況夢炎舛令慢諫,讜言弗敢告,今之賣降者,多其任用之士。」疏再上,不報。出關俟命,再奏曰:「因危急而紊紀綱,以偏見而咈公
 議,臣封駁不行,與大臣異論,勢不當留。」疏入,又不報,遂東歸。



 詔中使譚純德以翰林學士召,識者以為奪其要路,寵以清秩,非所以待賢者。應麟亦力辭,後二十年卒。



 所著有《深寧集》一百卷、《王堂類稿》二十三卷、《掖垣類稿》二十二卷、《詩考》五卷、《詩地理考》五卷、《漢藝文藝志考證》十卷、《通鑒地理考》一百卷、《通鑒地理通釋》十六卷、《通鑒答問》四卷、《困學紀聞》二十卷、《蒙訓》七十卷、《集解踐阼篇》、《補注急就篇》六卷、《補注王會篇》四十卷、《小學紺珠》十卷、《玉海》二百
 卷、《詞學指南》四卷、《詞學題苑》四十卷、《姓氏急就篇》六卷、《漢制考》四卷、《六經天文編》六卷、《小學諷詠》四卷。



 黃震,字東發,慶元府慈溪人。寶祐四年登進士第,調吳縣尉。吳多豪勢家,告私債則以屬尉,民多饑凍窘苦,死尉卒手。震至,不受貴家告。府檄攝其縣。及攝長洲、華亭,皆有聲。



 浙東提舉常平王華甫闢主管帳司文字。時錢庚孫守常,朱熠守平江,吳君擢守嘉興,皆倚嬖幸厲民。
 華甫病革,強起劾罷三人,震贊之也。沿海制置司闢乾辦、提領浙西鹽事,不就。改闢提領鎮江轉般倉分司。公田法行,改提領官田所,言不便,不聽,復轉般倉職。



 入為點校贍軍激賞酒庫所檢察官。擢史館檢閱,與修寧宗、理宗兩朝《國史》、《實錄》。輪對,言當時之大弊,曰民窮,曰兵弱,曰財匱,曰士大夫無恥。乞罷給度僧人道士牒,使其徒老死即消弭之,收其田入,可以富軍國,紓民力。時宮中建內道場,故首及此。帝怒,批降三秩,即出國門。用諫
 官言,得寢。



 出通判廣德軍。初,孝宗頒朱熹社倉法於天下,而廣德則官置此倉。民困於納息,至以息為本,而息皆橫取,民窮至自經。人以為熹之法,不敢議。震曰:「不然。法出於堯、舜、三代聖人,猶有變通,安有先儒為法,不思救其弊耶?況熹法,社倉歸之於民,而官不得與。官雖不與,而終有納息之患。」震為別買田六百畝,以其租代社倉息,約非兇年不貸,而貸者不取息。



 郡有祠山廟,歲合江、淮之民禱祈者數十萬,其牲皆用牛。郡惡少挾兵刃
 舞牲迎神為常,鬥爭致犯法。其俗又有自嬰桎梏、自拷掠以徼福者。震見,問之,乃兵卒。責自狀其罪,卒曰:「本無罪。」震曰:「爾罪多,不敢對人言,特告神以免罪耳。」杖之示眾。又其俗有所謂埋藏會者,為坎於庭,深、廣皆五尺,以所祭牛及器皿數百納其中,覆以牛革,封鐍一夕,明發視之,失所在。震以為妖,而殺牛淫祀非法,言之諸司,禁絕之。郡守賈蕃世以權相從子驕縱不法,震數與爭論是非,蕃世積不堪,疏震撓政,坐解官。



 尋通判紹興府,獲
 海寇,僇之。撫州饑起,震知其州,單車疾馳,中道約富人耆老集城中,毋過某日。至則大書「閉糶者藉,強糴者斬」揭於市,坐驛舍署文書,不入州治,不抑米價,價日損。親煮粥食餓者。請於朝,給爵賞旌勞者,而後入視州事。轉運司下州糴米七萬石,震曰:「民生蹶矣,豈宜重困之!」以沒官田三莊所入應之。若補刻《六經》、《儀禮》,修復朱熹祠,樹晏殊里門曰「舊學坊」,制祭社稷器,復風雷祀,勸民種麥,禁競渡船,焚千三百餘艘,用其丁鐵創軍營五百間,
 皆善政也。



 詔增秩,遂升提舉常平倉司。舊有結關拒逮捕事系郡獄二十有八年,存者十無三四,以事關尚書省,無敢決其獄者,以結關為作亂也。震謂結關猶他郡之結甲也,非作亂比,況已經數赦,於是皆釋之。新城與光澤地犬牙相入,民夾溪而處,歲常忿鬥爭漁。會知縣事蹇雄為政擾民,因相結拒,起焚掠。震乃劾罷雄,諭其民散去。初,常平有慈幼局,為貧而棄子者設,久而名存實亡。震謂收哺於既棄之後,不若先其未棄保全之。乃
 損益舊法,凡當免而貧者,許里胥請於官贍之,棄者許人收養,官出粟給所收家,成活者眾。震論役法,先令縣核民產業,不使下戶受抑於上戶。大興水利,廢陂、壞堰及為豪右所占者,復之。



 改提點刑獄,決滯獄,清民訟,赫然如神明。有貴家害民,震按之,貴家怨。又強發富人粟與民,富人亦怨。御史中丞陳堅以讒者言,劾震去,讒者,乃怨震者也。遂奉雲臺祠。賈似道罷相,以宗正寺簿召,將與俞浙並為監察御史,有內戚畏震直,止之,而浙亦
 以直言去。



 移浙東提舉常平,鎮安饑民,折盜賊萌芽。時皇叔大父福王與芮判紹興府,遂兼王府長史。震奏曰:「朝廷之制,尊卑不同,而紀綱不可紊。外雖藩王,監司得言之。今為其屬,豈敢察其非,奈何自臣復壞其法?」固不拜長史。命進侍左郎官及宗正少卿,皆不拜。



 震嘗告人曰:「非聖人之書不可觀,無益之詩文不作可也。」居官恆未明視事,事至立決。自奉儉薄,人有急難,則周之,不少吝。所著《日抄》一百卷。卒,門人私謚曰文潔先生。



\end{pinyinscope}