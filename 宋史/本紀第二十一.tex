\article{本紀第二十一}

\begin{pinyinscope}

 徽宗三



 二年春正
 月甲子,制:上書邪等人並不除監司。



 二月戊子朔,蔡京復太師致仕,賜第京師。庚子,以婉容崔氏為賢妃。



 三月戊午朔,定國公□共薨。己巳,御集英殿策進士。
 己卯,賜禮部奏名進士及第、出身七百十三人。



 夏四月己丑,詔縣令以十二事勸農於境內,躬行阡陌,程督勤惰。辛卯,復行方田。日中有黑子。甲午,宴蔡京等於太清樓。乙巳,以定國軍節度使仲忽為開府儀同三司。庚戌,以何執中為司空。壬子,賜張商英自便。



 五月癸亥,慮囚。丁卯,封子椿為慶國公。己巳,蔡京落致仕,三日一至都堂議事。



 六月己丑,以資政殿學士餘深為門下侍郎。乙卯,白虹貫日。



 秋七月壬申,訪天下遺書。丙子,置禮制局。



 九月壬午,改太尉以冠武階。癸未,正三公、三孤官。改侍中為左輔,中書令為右弼,左、右僕射為太宰、少宰,罷尚書令。



 冬十月乙巳,得玉圭於民間。



 十一月己未,置知客省、引進、四方館、東西上閣門事。戊寅,日南至,受元圭於大慶殿,赦天下。辛巳,蔡京進封魯國公。以何執中為少傅、太宰兼門下侍郎,執政皆進秩。十二月甲申,行給地牧馬法。乙酉,以鄭居中為特進。丙戌,以武信軍節度使童貫為太尉。乙巳,定命婦名為九等。丙午,燕輔臣於延
 福宮。辛亥,封子屋為衛國公。是歲,成都府、蘇州火。出宮女三百八十三人。高麗入貢。成都路夷人董舜咨、董彥博內附,置祺、亨二州。



 三年春正月己未,以定王桓、嘉王楷並為太保。庚申,以廣平郡王構為檢校太保。甲子,詔以天賜元圭,遣官冊告永裕、永泰陵。丙寅,以燕王俁為太傅。癸酉,追封王安石為舒王,子雱為臨川伯,配饗文宣王廟。丁丑,吳居厚罷,以觀文殿學士鄭居中知樞密院事。己卯,以越王偲
 為太傅,封子楗為韓國公。



 二月甲申,以德妃王氏為淑妃。庚寅,罷文臣勛官。辛卯,崇恩太后暴崩。甲午,以遼、女真相持,詔河北治邊防。丁酉,詔百官奉祠祿者並以三年為任。乙巳,增定六朝勛臣一百一十六人。



 三月壬子朔,日有食之。戊辰,進神宗淑妃宋氏為貴妃。升永安縣為永安軍。癸酉,賜上舍生十九人及第。



 夏四月戊子,作保和殿。庚寅,以復溱、播,等州降德音於梓夔路。癸巳,鄧洵仁罷。乙巳,以福寧殿東建玉清和陽宮。丙午,升定州
 為中山府。己酉,以資政殿學士薛昂為尚書右丞。庚戌,班《五禮新儀》。



 閏月丙辰,改公主為帝姬。戊午,復置醫學。辛酉,上崇恩太后謚曰昭懷。庚午,慶國公椿薨。



 五月乙酉,慮囚。丙申,升蘇州為平江府。庚子,大盈倉火。壬寅,以築溱、播進執政官一等。丙午,葬昭懷皇后於永泰陵。丁未,詔尚書內省分六司,以掌外省六曹所上之事;置內宰、副宰、內史、治中等官及都事以下吏員。己酉,班新燕樂。



 六月癸亥,祔昭懷皇后神主於太廟。戊辰,降兩京、河
 陽、鄭州囚罪一等,民緣園陵役者蠲其賦。



 秋七月癸未,升趙城縣為慶祚軍。甲申,還王珪、孫固贈謚,追復韓忠彥、曾布、安燾、李清臣、黃履等官職。庚子,貴妃劉氏薨。壬寅,復置白州。



 八月甲戌,以燕樂成,進執政官一等。丙子,以何執中為少師。丁丑,升潤州為鎮江府。戊寅,封四鎮山為王。



 九月庚寅,詔大理寺、開封府不得奏獄空,其推恩支賜並罷。戊戌,追冊貴妃劉氏為皇后,謚曰明達。



 冬十月乙丑,閱新樂器於崇政殿,出古器以示百官。戊辰,
 詔冬祀大禮及朝景靈宮,並以道士百人執威儀前導。冬十一月辛巳,朝獻景靈宮。壬午,饗太廟,加上神宗謚曰體元顯道法古立憲帝德王功英文烈武欽仁聖孝皇帝,改上哲宗謚曰憲元繼道世德揚功欽文睿武齊聖昭孝皇帝。癸未,祀昊天上帝於圜丘,大赦天下。升端州為興慶府。乙酉,以天神降,詔告在位,作《天真降臨示現記》。乙丑,以賢妃崔氏為德妃。壬辰,築祥州。己亥,詔有官人許舉八行。



 十二月癸丑,詔天下訪求道教仙經。乙
 卯,詔天下貢醫士。辛酉,太白晝見。是歲,江東旱,溫、封、滋三州火。出宮女二百七十有九人。



 四年春正月戊寅朔,置道階凡二十六等。辛丑,追封濮王子宗誼為祁王,宗詠為萊王,宗師為溫王,宗輔為楚王,宗博為蕭王,宗沔為霍王,宗藎為建王,宗勝為袁王。



 二月丁巳,賜上舍生十七人及第。癸亥,改淯井監為長寧軍。癸酉,長子桓冠。



 三月丙子朔,以淑妃王氏為貴妃。



 夏四月庚戌,幸尚書省,以手詔訓誡蔡京、何執中,各官
 遷秩,吏賜帛有差。癸丑,閱太學、闢雍諸生雅樂。甲子,改戎州為敘州。



 五月丙戌,始祭地於方澤,以太祖配。降德音於天下。子機薨。



 六月戊午,慮囚。壬申,以廣西溪洞地置隆、兌二州。



 秋七月丁丑,置保壽粹和館,以養宮人有疾者。戊寅,焚苑東門所儲毒藥可以殺人者,仍禁勿得復貢。甲午,祔明達皇后神主於別廟。



 八月乙巳,改端明殿學士為延康殿學士,樞密直學士為述古殿直學士。癸亥,定武臣橫班以五十員為額。



 九月乙卯,以安靜軍
 節度使王憲為開府儀同三司。己亥,詔諸路兵應役京師者,並以十月朔遣歸。



 冬十月乙巳,復置拱州。



 十一月丁丑,封子梴為相國公。十二月己酉,以禁中神御殿成,減天下囚罪一等。癸丑,定朝議、奉直大夫以八十員為額。己未,詔廣南市舶司歲貢真珠、犀角、象齒。是歲,相州野蠶成繭。出宮女六十八人。



 五年春正月庚辰,瀘南晏州夷反,尋詔梓州路轉運使趙遹等督兵討平之。己丑,令諸州縣置醫學,立貢額。甲
 午,改龍州為政州。



 二月乙巳,立定王桓為皇太子。甲寅,冊皇太子,赦天下。庚午,以童貫領六路邊事。三月辛未朔,太白晝見。己卯,御集英殿策進士。甲申,追論至和、嘉祐定策功,封韓琦為魏郡王,覆文彥博官。丁亥,詔以立皇太子,見責降文武臣僚並與牽復甄敘,凡千五百人。壬辰,升舒州為德慶軍。癸巳,賜禮部奏名進士出身六百七十人。



 夏四月甲辰,作葆真宮。丁未,詣景靈宮,還,幸秘書省,進館職官一等。庚戌,改集英殿為右文殿。癸亥,
 置宣和殿學士。詔東宮講讀官罷讀史。



 五月壬辰,慮囚。



 六月癸丑,以修三山河橋,降德音於河北、京東、京西路。



 秋七月戊辰朔,日有食之。乙亥,升汝州為陸海軍。丁丑,詔建明堂於寢殿之南。甲申,昭慶軍節度使蔡卞為開府儀同三司。丁亥,封子□越為瀛國公。



 八月己酉,以秘書省地為明堂。辛亥,升通利軍為浚州、平川軍節度。嗣濮王仲增薨。



 九月己卯,封仲御為嗣濮王。丙戌,封子柍為惠國公。冬十月癸卯,以嵩山道人王仔昔為沖隱處士。
 戊午,夏國入貢。



 十一月癸酉,錄昭憲皇后杜氏之裔。庚寅,高麗遣子弟入學。十二月己亥,升遂州為遂寧府。庚申,以平晏夷,曲赦四川。癸亥,置緣邊安撫司於瀘州。是歲,平江府、常、湖、秀州水。出宮女五十人。



 六年春正月戊子,以瀘南獻捷,轉宰執一官。以童貫宣撫陜西、河北。



 閏月壬寅,升穎州為順昌府。丁未,置道學。



 二月丁亥,詔增廣天下學舍。庚寅,詔廣京城。



 三月癸丑,賜上舍生十一人及第。



 夏四月乙丑,會道士於上清寶
 菉宮。辛未,以何執中為太傅致仕,朝朔望。丁丑,詔天寧諸節及壬戌日,杖已下罪聽贖。丙戌,卻監司、守臣進獻。庚寅,詔蔡京三日一朝,正公相位,總治三省事。



 五月丁酉,廢錫錢。庚子,以鄭居中為少保、太宰兼門下侍郎,劉正夫為特進、少宰兼中書侍郎。壬寅,以保大軍節度使鄧洵武知樞密院事。



 六月丙寅,班中書官制格。庚午,慮囚。甲戌,詔堂吏遷官至奉直大夫止。癸未,皇太子納妃朱氏。



 秋七月壬辰朔,以震武城為震武軍。甲午,以德妃
 崔氏為貴妃。辛亥,以河陽三城節度使王薦為開府儀同三司。諸盜晏州卜漏闕一字、沅州黃安俊、定邊軍李吪□移伏誅,詔函首於甲庫。壬子,曲赦湖北。己未,解池生紅鹽。辛酉,改走馬承受公事為廉訪使者。



 八月壬戌朔,戒北邊帥臣毋生事。壬午,詔天下監司、郡守搜訪巖谷之士,雖恢詭譎怪自晦者悉以名聞。丁亥,幸蔡京第。己丑,升晉州為平陽、壽州為壽春、齊州為濟南府。



 九月辛卯朔,詣玉清和陽宮,上太上開天執符御歷含真體道昊天
 玉皇上帝徽號寶冊。丙申,赦天下。令洞天福地修建宮觀,塑造聖像。以西內成,曲赦京西。己未,以童貫為開府儀同三司。



 冬十月乙丑,太白晝見。



 十一月丁酉,朝獻景靈宮。戊戌,饗太廟。己亥,祀昊天上帝於圜丘,赦天下。庚子,以禮部尚書白時中為尚書右丞。辛丑,魏國公材薨。戊申,以侯蒙為中書侍郎,薛昂為尚書左丞。己未,徙封衛國公屋為鄆國公。增橫班為十三階。十二月己巳,以婉儀劉氏為賢妃。戊寅,以熙河進築功成,進執政一官。乙
 酉,奠九鼎於圜像徽調閣。劉正夫為開府儀同三司致仕。戊子,以宗粹為開府儀同三司。是歲,冀州三山黃河清。出宮女六百人。高麗、占城、大食、真臘、大理、夏國入貢,茂州夷郅永壽內附。



 七年春正月丁酉,於闐入貢。庚子,以殿前都指揮使高俅為太尉。



 二月癸亥,以大理國主段和譽為雲南節度使、大理國王。甲子,會道士二千餘人於上清寶菉宮,詔通真先生林靈素諭以帝君降臨事。丁卯,御集英殿策
 高麗進士。辛未,改天寧萬壽觀為神霄玉清萬壽宮。乙亥,幸上清寶菉宮,命林靈素講道經。



 三月庚寅,賜高麗祭器。高麗進士權適等四人賜上舍及第。乙未,以童貫權領樞密院。丙申,升鼎州為常德軍。



 夏四月庚申,帝諷道菉院上章,冊己為教主道君皇帝,止於教門章疏內用。辛酉,升溫州為應道軍。



 五月戊子朔,升慶州為慶陽軍、渭州為平涼軍。己丑,如玉清和陽宮,上承天效法厚德光大後土皇地祇徽號寶冊。辛卯,命蔡攸提舉秘書
 省並左右街道菉院。乙未,詔權罷宮室修造。辛丑,祭地於方澤,降德音於諸路。以監司州縣共為奸贓,令廉訪使者察奏,仍許民徑赴尚書省陳訴。癸卯,改玉清和陽宮為玉清神霄宮。



 六月戊午朔,以明堂成,進封蔡京為陳、魯國公。戊辰,以嘉王楷為太傅。改節度觀察留後為承宣使。己巳,蔡京辭兩國不拜,詔官其親屬二人。壬午,詔禁巫覡。丙戌,貴妃宋氏薨。



 秋七月壬辰,熙河、環慶、涇原地震。庚子,詔八寶增定命寶。



 八月癸亥,詔明堂並祠
 五帝。鄭居中以母憂去位。



 九月戊子,詔湖北民力未紓,胡耳西道可罷進築。辛卯,大饗明堂,赦天下。乙未,劉正夫卒。丁酉,西蕃王子益麻黨徵降,見於紫宸殿。壬寅,進宰執官一等。甲辰,以薛昂為特進。癸丑,貴妃王氏薨。



 冬十月乙卯朔,初御明堂,班朔布政。戊寅,侯蒙罷。



 十一月庚寅,命蔡京五日一赴都堂治事。辛卯,鄭居中起復。以餘深為特進、少宰兼中書侍郎,白時中為中書侍郎。壬辰,復置醴州。丙申,何執中卒。升石泉縣為軍。十二月戊
 申朔,有星如月。丁巳,以薛昂為門下侍郎。戊辰,詔天神降於坤寧殿,刻石以紀之。庚午,以童貫領樞密院。命戶部侍郎孟揆作萬歲山。是歲,三山河水清。出宮女六十八人。



 重和元年春正月甲申朔,受定命寶於大慶殿。戊子,封孫諶為崇國公。己丑,赦天下。應元符末上書邪中等人,依無過人例。乙巳,封侄有奕為和義郡王。庚戌,以翰林學士承旨王黼為尚書左丞。



 二月戊辰,增諸路酒價。庚
 午,遣武義大夫馬政由海道使女真,約夾攻遼。甲戌,升六安縣為六安軍。丁丑,詔監司輒以禁錢買物為苞苴饋獻,論以大不恭。



 三月丙戌,詔監司、郡守自今須滿三歲乃得代,仍毋得通理。癸巳,令嘉王楷赴廷對。丙申,以茂州蕃族平,曲赦四川。丁酉,知建昌陳並等改建神霄宮不虔及科決道士,詔並勒停。戊戌,御集英殿策進士。戊申,賜禮部奏名進士及第、出身七百八十三人。有司以嘉王楷第一,帝不欲楷先多士,遂以王昂為榜首。



 夏
 四月癸丑朔,築靖夏城、制戎城。錄呂餘慶後。癸亥,減捶刑。己卯,詔每歲以季秋親祠明堂,如孟月朝獻禮。以太上混元上德皇帝二月十五日生辰為貞元節。



 五月壬午朔,日有食之。乙酉,詔諸路選漕臣一員,提舉本路神霄宮。丁亥,以林靈素為通真達靈元妙先生,張虛白為通元沖妙先生。壬辰,班禦制《聖濟經》。以青華帝君八月九日生辰為元成節。庚戌,手敕兩浙漕司,以權添酒錢盡給御前工作。



 六月乙卯,以賢妃劉氏為淑妃。己巳,以
 淮西盜平,曲赦。庚子,慮囚。甲戌,以西邊獻捷,曲赦陜西、河東路。



 秋七月壬午,以西師有功,加蔡京恩,官其一子。鄭居中為少傅,餘深為少保,鄧洵武為特進,進執政官一等。己酉,遣廉訪使者六人振濟東南諸路水災。



 八月甲寅,以童貫為太保。辛酉,詔班御注《道德經》。壬申,詔執政非入謝及丐去,毋得獨留奏事。癸酉,封子椅為嘉國公。乙亥,升兗州為襲慶府。



 九月辛巳,大饗明堂。壬午,詔罷拘白地、禁榷貨、增方田稅、添酒價、取醋息、河北加折
 耗米、東南水災強糴等事。丙戌,詔太學、闢雍各置《內經》、《道德經》、《莊子》、《列子》博士二員。己丑,以歲當戌、月當壬為元命,降德音於天下。庚寅,薛昂罷。以白時中為門下侍郎,王黼為中書侍郎,翰林學士承旨馮熙載為尚書左丞,刑部尚書範致虛為尚書右丞。壬辰,禁州郡遏糴及邊將殺降以幸功賞者。癸巳,禁群臣朋黨。丁酉,用蔡京言,集古今道教事為紀志,賜名《道史》。辛丑,鄭居中罷,乞持餘服,詔從之。詔察縣令治行、諸路監司能改正州縣
 事者,較為殿最。詔視中大夫林靈素、視中奉大夫張虛白並特授本品真官。



 閏月庚申,詔江、淮、荊、浙、閩、廣監司督責州縣還集流民。丁卯,進封楷為鄆王。丙子,詔:周柴氏後已封崇義公,復立恭帝後以為宣義郎,監周陵廟,世世為國三恪。



 冬十月己卯朔,太白晝見。己亥,改興慶軍為肇慶府。甲辰,置道官二十六等,道職八等。十一月己酉朔,改元,大赦天下。辛亥,日中有黑子。丙辰,以婉容王氏為賢妃。辛酉,補上書人安堯臣官。己巳,升梓州為
 潼川府。



 十二月戊寅朔,復京西錢監。己丑,置裕民局。是歲,江、淮、荊、浙、梓州水。出宮女百七十八人。黃巖民妻一產四男子。於闐、高麗入貢。



\end{pinyinscope}