\article{列傳第一百}

\begin{pinyinscope}

 王存孫固趙瞻傅堯俞



 王存,字正仲,潤州丹陽人。幼善讀書,年十二,辭親從師於江西,五年始歸。時學者方尚雕篆,獨為古文數十篇,鄉老先生見之,自以為不及。



 慶歷六年,登進士第,調嘉
 興主簿,擢上虞令。豪姓殺人,久莫敢問,存至,按以州吏受賕,豪賂他官變其獄,存反為罷去。久之,除密州推官。修潔自重,為歐陽修、呂公著、趙概所知。治平中,入為國子監直講,遷秘書省著作佐郎,歷館閣校勘、集賢校理、史館檢討、知太常禮院。存故與王安石厚,安石執政,數引與論事,不合,即謝不往。存在三館歷年,不少貶以干進。嘗召見便殿,累上書陳時政,因及大臣,無所附麗,皆時人難言者。



 元豐元年,神宗察其忠實無黨,以為國史
 編修官、修起居注。時起居注雖日侍,而奏事必稟中書俟旨。存乞復唐貞觀左右史執筆隨宰相入殿故事,神宗韙其言,聽直前奏事,自存始也。



 明年,以右正言、知制誥、同修國史兼判太常寺。論圜丘合祭天地為非古,當親祠北郊如《周禮》。官制行,神宗切於用人,存請自熙寧以來群臣緣論事得罪,或詿誤被斥而情實納忠非大過者,隨材召擢,以備官使。語合神宗意。收拔者甚眾。又言:「赦令出上恩,而比歲議法治獄者,多乞不以赦降原
 減。官司謁禁,本防請托,而吊死問疾,一切杜絕,皆非便也。」執政不悅。



 五年,遷龍圖閣直學士、知開封府。京師並河居人,盜鑿汴堤以自廣,或請令培築復故,又按民廬侵官道者使撤之。二謀出自中人,既有詔矣。存曰:「此吾職也。」入言之。即曰馳其役,都人歡呼相慶。進樞密直學士,改兵部尚書,轉戶部。神宗崩,哲宗立,永裕陵財費,不逾時告備,宰相乘間復徙之兵部。太僕寺請內外馬事得專達,毋隸駕部。存言:「如此,官制壞矣。先帝正省、臺、寺、
 監之職,使相臨制,不可徇有司自便,而隳已成之法。」元祐初,還戶部,固辭不受。二年,拜中大夫、尚書右丞。三年,遷左丞。



 有建議罷教畿內保甲者,存言:「今京師兵籍益削,又廢保甲不教,非國家根本久長之計。且先帝不憚艱難而為之,既已就緒,無故而廢之,不可。」門下侍郎韓維罷,存言:「去一正人,天下失望,忠黨沮氣,讒邪之人爭進矣。」又論杜純不當罷侍御史,王覿不當罷諫官。



 四方奏讞大闢,刑部援比請貸,都省屢以無可矜恕卻之。存
 曰:「此祖宗制也。有司欲生之,而朝廷破例殺之,可乎?」又言:「比廢進士專經一科,參以詩賦,失先帝黜詞律、崇經術之意。」河決而北幾十年,水官議還故道,存爭之曰:「故道已高,水性趨下,徒費財力,恐無成功。」卒輟其役。蔡確以詩怨訕,存與范純仁欲薄其罪,確再貶新州,存亦罷,以端明殿學士知蔡州。始,存之徙兵部,確力也。至是,為確罷,士大夫善其能損怨。歲餘,加資政殿學士、知揚州。揚、潤相去一水,用故相例,得歲時過家上塚,出賜錢給
 鄰里,又具酒食召會父老,親與酬酢,鄉黨傳為美談。



 召為吏部尚書。時,在廷朋黨之論浸熾,存為哲宗言:「人臣朋黨,誠不可長,然或不察,則濫及善人。慶歷中,或指韓琦、富弼、範仲淹、歐陽修為黨,賴仁宗聖明,不為所惑。今日果有進此說者,願陛下察之。」由是復與任事者戾,除知大名府,改知杭州。



 紹聖初,請老,提舉崇禧觀,遷右正議大夫致仕。舊制,當得東宮保傅,議者指存嘗議還西夏侵地,故殺其恩典,既而降通議大夫。存嘗悼近世學
 士貴為公卿,而祭祀其先,但循庶人之制。及歸老築居,首營家廟。建中靖國元年,卒,年七十九。贈左銀青光祿大夫。



 存性寬厚,平居恂恂,不為詭激之行,至其所守,確不可奪。司馬光嘗曰:「並馳萬馬中能駐足者,其王存乎!」



 孫固,字和父,鄭州管城人。幼有立志。九歲讀《論語》,曰:「吾能行此。」徂徠石介一見,以公輔期之。擢進士第,調磁州司戶參軍。從平貝州,為文彥博言脅從罔治之義,與彥博意協,故但誅首惡,餘無所及。轉霍邑令,遷秘書丞,為
 審刑詳議官。宰相韓琦知其賢,諭使來見,固不肯往。琦益器重之,引為編修中書諸房文字。



 治平中,神宗為穎王,以固侍講;及為皇太子,又為侍讀。至即位,擢工部郎中、天章閣待制、知通進銀臺司。種諤取綏州,固知神宗志欲經略西夏,欲先事以戒,即上言:「待遠人宜示之信,今無名舉兵,非計之得。願以漢韓安國、魏相、唐魏徵論兵之略,參校同異,則是非炳然矣。兵,兇器也,動不可妄,妄動將有悔。」大臣惡其說,出知澶州。



 還知審刑院,復領
 銀臺、封駁兼侍讀,判少府監。神宗問:「王安石可相否?」對曰:「安石文行甚高,處侍從獻納之職,可矣。宰相自有其度,安石狷狹少容。必欲求賢相,呂公著、司馬光、韓維其人也。」凡四問,皆以此對。及安石當國,更法度,固數議事不合;青苗法出,又極陳其不便。及韓琦疏至,神宗感動,謂固曰:「朕熟計之,誠不便。」固出語執政曰:「及上有意,宜亟圖之,以福天下。」既而竟從安石。固復領銀臺司。



 孔文仲對制策忤時政,報罷。固言:「陛下以名求士,而士以實
 應,今反過之,何哉?今謂文仲之言以惑天下,臣恐天下不惑文仲之言,以文仲之黜為惑也。」胡宗愈坐言事逐,蘇頌、陳薦以論李定罷,固皆引誼爭之。



 時議尊僖祖為始祖,固議曰:「漢高以得天下與商、周異,故太上皇不得為始封;光武中興,不敢祖舂陵而祖高帝。宋有天下,傳之萬世,太祖功也,不當替其祀;請以為始祖,而為僖祖別立廟。禘祫之日,奉其祧主東向以伸其尊,合所謂祖以孫尊、孫以祖屈之意。」韓琦見而嘆曰:「孫公此議,足以
 不朽矣。」



 加龍圖閣直學士、知真定府。遼人盜耕解子平地,歲且久,吏爭弗能還。固微得其要領,折愧之,正疆地二百里。熙寧末,以樞密直學士知開封府。元豐初,同知樞密院事。時征安南,建順州,其地瘴癘不堪守,固請棄之,內徙者二萬戶。



 諜者告夏人幽其主,神宗欲西討,固數言舉兵易,解禍難。神宗曰:「夏有釁不取,則為遼人所有,不可失也。」固曰:「必不得已,請聲其罪薄伐之,分裂其地,使其酋長自守焉。」神宗笑曰:「此真酈生之說爾。」時執
 政有言便當直度河,不可留行。固曰:「然則孰為陛下任此者?」神宗曰:「朕已屬李憲。」固曰:「伐國,大事也,豈可使宦官為之!今陛下任李憲,則士大夫孰肯為用乎?」神宗不悅。他日,固又曰:「今五路進師而無大帥,就使成功,兵必為亂。」神宗曰:「大帥誠難其人。」呂公著曰:「既無其人,曷若已之。」固曰:「公著言是也。」初議五路入討,會於靈州,李憲由熙河入,輒不赴靈州,乃自開蘭、會,欲以弭責。固曰:「兵法期而後至者斬。今諸路皆進,而憲獨不行,雖得蘭、會,
 罪不可赦。」神宗不聽,其後師果無功。神宗曰:「朕始以孫固言為迂,今悔無及矣。」



 改太中大夫、樞密副使,進知院事,以疾避位,拜觀文殿學士、知河陽,尋提舉嵩山崇福宮。哲宗即位,以正議大夫知河南府,徙鄭州。元祐二年,召除侍讀、提舉中太一宮,遂拜門下侍郎。哲宗與太皇太后矜其年高,每朝會豫節拜儀,聽休於幄次。固數乞骸骨,太皇太后曰:「卿,先帝在東宮時舊臣。今帝新聽政,勉留輔導;或體中未安,取文書於家治之可也。」固感激,
 強起視事,復知樞密院事,累官右光祿大夫。五年,卒,年七十五。哲宗、太皇太后皆出聲泣。時文彥博致仕歸洛,將宴餞崇政殿,以固在殯,罷之。輟視朝二日,贈開府儀同三司,謚曰溫靖。



 固宅心誠粹,不喜矯亢,與人居久而益信,故更歷夷險,而不為人所疾害。嘗曰:「人當以聖賢為師,一節之士,不足學也。」又曰:「以愛親之心愛其君,則無不盡矣。」司馬光退處,固每勸神宗召歸;及光為陳州,過鄭,固與論天下大事至數十,曰:「公行且相,宜視先後
 緩急審處之。」傅堯俞銘其墓曰:「司馬公之清節,孫公之淳德,蓋所謂不言而信者也。」世以為確論。紹聖時奪遺澤,元符二年,奪所贈官,列元祐黨籍。政和中,徽宗以固嘗為神宗宮僚,特出籍,悉還所奪。



 趙瞻,字大觀,其先亳州永城人。父剛,太子賓客,徙鳳翔之盩厔。瞻舉進士第,調孟州司戶參軍,移萬泉令。捐圭田修學宮,士自遠而至。改知夏縣,作八監堂,書古賢令長治跡以自監。又以秘書丞知永昌縣,築六堰灌田,歲
 省科斂數十萬,水訟咸息,民以比召、杜。升太常博士,知威州。瞻以威、茂雜群獠,險而難守,不若合之而建郡於汶川,條著其詳,為《西山別錄》。後熙寧中,朝廷經理西南,就瞻取其書考焉。



 遷尚書屯田員外郎。英宗治平初,自都官員外郎除侍御史。上疏曰:「英斷獨化,人主至權也。審至權者,當主以天下之大公,揆以天下之正論,如是而後權可一也。若夫積久之敝,陛下其思焉。刑賞施設之失,可革則革;號令言動之過,可止則止。輔相賴其用,
 宜責其效;臺諫知其才,宜信其說。兵柄宜削諸宦官,邊議宜付諸宿將。蓋權不可矯而為也,以從天下之望耳。」英宗稱善。



 久之,詔遣內侍王昭明等四人為陜西諸路鈐轄,招撫諸部。瞻以唐用宦者為觀軍容、宣慰等使,後世以為至戒,宜追還內侍,責成守臣,章三上,言甚激切。會文彥博、孫沔經略西夏,別遣馮京安撫諸路,瞻又請罷京使,專委宿將。夏人入侵王官,慶帥孫長卿不能御,加長卿集賢院學士,瞻言長卿當黜不宜賞,賞罰倒置。京
 東盜賊數起,瞻請易置曹、濮守臣之不才者,未報。乃求退,力言追還昭明等,英宗改容,納其言。



 二年秋,京師大水,詔百官言事,多留中,瞻請「悉出章疏,付兩省詳擇以聞」,從之。時議追崇濮安懿王,瞻引漢師丹、董宏事,謂其屬薛溫其曰:「事將類此,吾必以死爭,固吾所也。」中書請安懿王稱親,瞻爭曰:「仁宗既下明詔子陛下,議者顧惑禮律所生所養之名,妄相訾難,彼明知禮無兩父貳斬之義,敢裂一字之詞,以亂厥真。且文有去婦出母者,去
 已非婦,出不為母,辭窮直書,豈足援以斷大議哉?臣請與之庭辨,以定邪正。」已而皇太后手書尊王為皇,瞻嘆曰:「向者太后切責大臣,議乃得罷。今邪臣與中官交締,歸過至尊而自為之地,吾與首議之臣,不並生矣!」因復力陳。會假太常少卿接契丹賀正使,入對,英宗問前事,對曰:「陛下為仁宗子,而濮王又稱皇考,則是二父,二父非禮。」英宗曰:「御史嘗見朕欲皇考濮王乎?」瞻曰:「此乃大臣之議,陛下未嘗自言。」英宗曰:「是中書過耳,朕自數歲
 時,先帝養為子,豈敢稱濮考?」瞻曰:「臣請退諭中書,作詔以曉天下。」時連日晦冥,英宗指天示瞻曰:「天道如此,安敢妄為褒尊。朕意已決,無庸宣告。」瞻曰:「陛下祗畏天戒,不以私妨公,甚盛德也。」及使還,聞呂誨等諫濮議皆罷去,乞與同貶,不報。趣入對,英宗曰:「卿欲就龍逢、比干之名,孰若效伊尹、傅說哉?」瞻皇懼,言:「臣不敢奉詔,使朝廷有同罪異罰之譏。」遂通判汾州。



 神宗即位,遷司封員外郎、知商州,又除提點陜西刑獄。熙寧三年,為開封府判
 官。神宗問:「卿知青苗法便乎?」對曰:「青苗法,唐行之於季世擾攘中,掊民財誠便。今欲為長久計,愛養百姓,誠不便。」初,王安石欲瞻助己,使其黨餌以知雜御史。瞻不應,由是不得留京師,出為陜西轉運副使,改永興軍轉運使。以親老,請知同州。七年,朝廷患錢重,議以交子權之,命瞻制置。瞻曰:「有本錢足恃,法乃可行,如多出空券,是罔民也。」議不合,移京西轉運使;又以親老不行,徙陜州,請還鄉里,除提舉鳳翔太平宮。丁外艱,服除,易朝請大
 夫、知滄州。



 哲宗立,轉朝議大夫,召為太常少卿,遷戶部侍郎。元祐三年,擢樞密直學士、簽書樞密院事。明年,以中大夫同知院事。因進對言:「機政所急,人才而已。今臣選武臣難遽盡知,請詔諸路安撫、轉運使舉使臣,科別其才,第為三等,籍之以備選注。」



 初,元豐中,河決小吳,北注界河,東入於海。神宗詔,東流故道淤高,理不可回,其勿復塞。乃開大吳以護北都。至是,都水王令圖請還河故道,下執政議。瞻曰:「自河決已八年,未有定論。今遽興
 大役,役夫三十萬,用木二千萬,臣竊憂焉。朝廷方遣使相視,若以東流未便,宜亟從之;若以為可回,宜為數歲之計,以緩民力」。議者又謂河入界河而北,則失中國之險,昔澶淵之役,非河為限,則北兵不止。瞻曰:「王者恃德不恃險。昔堯、舜都蒲、冀,周、漢都咸、鎬,皆歷年數百,不聞以河障外國。澶淵之役,蓋廟社之靈,章聖之德,將相之智勇,故敵帥授首,豈獨河之力哉?」後使者以東流非便,水官復請塞北流,瞻固爭之,卒詔罷役,如瞻所議。



 洮、河
 諸族以青唐首領浸弱可制,欲倚中國兵威以廢之,邊臣亟請興師。瞻曰:「不可。御外國以大信為本,且既爵命之,彼雖失眾心,無犯王略之罪,何辭而伐之?若其不克,則兵端自此復起矣。」乃止。瞻又奏廢渠陽軍,以紓荊湖之力;乞詔諭西夏使歸永樂遺民,夏人聽命。



 五年,卒,年七十二。太皇太后語輔臣曰:「惜哉,忠厚君子也。」車駕親臨,輟視朝二日。贈銀青光祿大夫,謚曰懿簡。紹聖中,言者以傅會元祐諸臣,追奪所贈官,列於黨籍。



 瞻著《春秋
 論》二十卷,《史記抵牾論》五卷,《唐春秋》五十卷,《奏議》十卷,《文集》二十卷,《西山別錄》一卷。四子:孝諶,瀛州錄事參軍;獻誠,唐城令;某,蚤卒;彥詒,太康主簿。



 傅堯俞,字欽之,本鄆州須城人,徙孟州濟源。十歲能為文,及登第,猶未冠。石介每過之,堯俞未嘗不在,介曰:「君少年決科,不以游戲為娛,何也?」堯俞曰:「性不喜囂雜,非有他爾。」介嘆息奇之。嘗監西京稅院事,留守晏殊、夏竦皆謂曰:「子有清識雅度,文約而理盡,卿相才也。」



 知新息
 縣,累遷太常博士。嘉祐末,為監察物史。袞國公主下嫁李瑋,為家監梁懷吉、張承照所間,與夫不相中。仁宗斥二人於外,未幾,復還主家,出瑋知衛州。堯俞言:「主恃愛薄其夫,陛下為逐瑋而還隸臣,甚悖禮,為四方笑,後何以誨諸女乎?」



 皇城邏卒吳清誣奏富民殺人,鞠治無狀,有司須清辨,內侍主者不遣。堯俞言:「陛下惜清,恐不復聞外事矣。臣以為不若使付外,暴其是非而行賞罰焉,則事之上聞者皆實,乃所以廣視聽也。縱而不問,則讒
 者肆行,民無所措手足,尚欲求治,得乎?」內侍李允恭、朱晦屈法任其子,趙繼寵越次管當天章閣,蔡世寧掌內藏,而以珠私示內人。堯俞以為嬖寵恩幸過失,當防之於漸,悉劾之。



 時乏國用,言利者爭獻富國計。堯俞奏曰:「今度支歲用不足,誠不可忽,然欲救其弊,在陛下宜自儉刻,身先天下,無奪農時,勿害商旅,如是可矣。不然,徒欲紛更,為之無益,聚斂者用,則天下殆矣。」



 仁宗春秋高,皇嗣未立,堯俞請建宗室之賢,以慰天下望。及英宗為
 皇子,有司闕供饋,仁宗未知。堯俞言:「陛下既以宗社之重建皇嗣,宜以家人禮,使皇子朝夕侍膳左右,以通慈孝之誠。今禮遇有闕,非所以隆親親、重國本也。」於是詔有司供具甚厚。



 英宗即位,轉殿中侍御史,遷起居舍人。皇太后與英宗同聽政,英宗有疾,既平,堯俞上書皇太后,請還政。久之,聞內侍任守忠有讒間語,堯俞諫皇太后曰:「外間物論紛惑,兩宮之情未通。臣謂天下之可信者,無大於以天下與人,亦無大於受天下以公,況皇帝
 以明睿之資,貫通古今,而受人之天下乎?如誅竄讒人,則慈孝之聲並隆矣。」於是皇太后還政,逐守忠。堯俞言於英宗曰:「皇太后給事左右之人,宜頗錄其勤勞,少加恩惠,上慰母後,下安反側。且守忠已去,其餘不問可也。」



 遷右司諫、同知諫院。英宗眷遇堯俞,嘗雪中賜對,堯俞自東廡升,英宗傾身東向以待,每奏事退,多目送之。嘗問曰:「多士盈庭,孰忠孰邪?」堯俞曰:「大忠大佞,固不可移;中人之性,系上所化。」英宗納其言。



 時英宗初躬庶政,猶
 謙讓任大臣,堯俞言:「大臣之言是,陛下偶以為然而行之可也;審其非矣,從而徇之,則人主之柄安在?願君臣之際,是是非非,毋相面從。總覽眾議,無所適莫,則威柄歸陛下矣。」嘗因論事,英宗曰:「卿何不言蔡襄?」對曰:「若襄有罪,何不自正典刑,安用臣言?」英宗曰:「欲使臺諫言,以公議出之。」對曰:「若付之公議,臣但見襄辦山陵事有功,不見其罪。臣身為諫官,使臣受旨言事,臣不敢。」



 陜西言,近邊熟戶頗逃失。詔以內侍李若愚等為陜西四路鈐
 轄,專使招納,歲一入奏事。堯俞言:「此安撫、經略使職也。且若愚等,陛下不信其言,則如不用;言必見從,則邊帥之權,移於四人矣。」尋罷之。



 大臣建言濮安懿王宜稱皇考,堯俞曰:「此於人情禮文,皆大謬戾。」與侍御史呂誨同上十餘疏,其言極功。主議者知恟□不可遏,遂易「考」稱「親」。堯俞又言:「『親』,非父母而何?亦不可也。夫恩義存亡一也,先帝既以陛下為子,當是時,設濮王尚無恙,陛下得以父名之乎?」又因水災言:「簡宗廟,則水不潤下。今以
 濮王為皇考,於仁宗之廟,簡孰甚焉。」



 俄命堯俞與趙瞻使契丹,比還,呂誨、呂大防、范純仁皆以諫濮議罷,復除堯俞侍御史知雜事。堯俞拜疏必求罷去,英宗面留之。堯俞言:「誨等已逐,臣義不當止。」因再拜辭,英宗愕然,曰:「是果不可留也。」遂出知和州。通判楊洙乘間問曰:「公以直言斥居此,何為未嘗言及御史時事?」堯俞曰:「前日言職也,豈得已哉?今日為郡守,當宣朝廷美意,而反呫呫追言前日之闕政,與誹謗何異?」



 神宗即位,徙知廬州。熙
 寧三年,至京師。王安石素與之善,方行新法,謂之曰:「舉朝紛紛,俟尹來久矣,將以待制、諫院處君。」堯俞曰:「新法世以為不便,誠如是,當極論之。平生未嘗好欺,敢以為告。」安石慍之,但授直昭文館、權鹽鐵副仗,俄出為河北轉運使,改知江寧府。陛辭,言:「仁廟一室,與藝祖、太宗並為百代不遷之主。」



 徙許州、河陽、徐州,再歲六移官,困於道路,知不為時所容,請提舉崇福宮。先是,徐人告有談天文休咎者,堯俞以事未白,不受辭。談者後伏誅,堯俞
 坐不即捕,削官職。稍起,監黎陽縣倉草場,郡掾行縣,堯俞從眾出迎盡禮。守為遣他吏代主出納,堯俞不可,曰:「居其官安得曠其職。」雖寒暑,必日至庾中治事,凡十年。



 哲宗立,自知明州召為秘書少監兼侍講,擢給事中、吏部侍郎、御史中丞。奏言:「人才有能有不能,如使臣補闕拾遺以輔盛德,明善正失以平庶政,舉直措枉以正大臣,臣雖不才,敢不盡力。若使窺人陰私,抉人細故,則非臣所能,亦非臣之志也。」御史張舜民以言事罷,詔堯俞
 更舉御史,堯俞封還詔書,請留舜民。不聽,即以堯俞為吏部侍郎,堯俞不可,遂以龍圖閣待制知陳州。未幾,復為吏部侍郎、御史中丞。



 前宰相蔡確坐詩誹謗,貶新州,宰執、侍從以下,罷者七八人,御史府為之一空。堯俞曰:「確之黨,其尤者固宜逐,其餘可以一切置之。」且言:「以陛下盛德,而乃於此不能平?願聽之如蚊虻之過耳,無使有纖微之忤,以奸太和之氣。事至,以無心應之,聖人所以養至誠而御遐福也。」



 水官李偉議大河可從孫村導
 之還故道。堯俞言:「河事雖不可隃度,然比遣使按之,皆言非便。而偉又繆悠不肯任責,豈可以遽興大役。」朝廷遂置偉議。進吏部尚書兼侍讀。元祐四年,拜中書侍郎。六年,卒,年六十八。哲宗與太皇太后哭臨之,太皇太后語輔臣曰:「傅侍郎清直一節,終始不變,金玉君子也。方倚以相,遽至是乎!」贈銀青光祿大夫,謚曰獻簡。紹聖中,以元祐黨人,奪贈謚,著名黨籍。後黨錮解,下詔褒贈,錄其子。



 堯俞厚重言寡,遇人不設城府,人自不忍欺。論事
 君前,略無回隱,退與人言,不復有矜異色。初,自諫官補郡,眾疑法令有未安者,必有所不從,堯俞一切遵之,曰:「君子素其位而行,諫官有言責也,為郡知守法而已。」徐前守侵用公錢,堯俞至,為償之,未足而去。後守移文堯俞使償入之,考實非堯俞所用,卒不辯。司馬光嘗謂河南邵雍曰:「清、直、勇之德,人所難兼,吾於欽之見焉。」雍曰:「欽之清而不耀,直而不激,勇而能溫,是為難爾。」從孫察,見《忠義傳》。



 論曰:存、固、瞻、堯俞,初皆善王安石;及其秉政,未嘗受所誘餌,與論新法,終不詭隨。及元祐區別正邪,其論蔡確詩謗之罪恐為已甚,將啟朋黨之禍,豈非先知之明乎?他有更張,隨事諫止,不少循默。然無矯枉過中之失,故能不亟不徐,進退有道,在元祐諸臣中,身名俱全,亦難矣哉。



\end{pinyinscope}