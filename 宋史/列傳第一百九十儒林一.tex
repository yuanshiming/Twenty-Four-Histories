\article{列傳第一百九十儒林一}

\begin{pinyinscope}
○聶崇義邢昺孫奭王昭素孔維孔宜崔頌
 \gezhu{
  子曥附}
 尹拙田敏辛文悅李覺崔頤正
 \gezhu{
  弟偓佺附} 李之才



 聶崇義,河南洛陽人。少舉《三禮》,善《禮》學,通經旨。漢乾祐中,累官至國子《禮記》博士,校定《公羊春秋》,刊板於國學。周顯德中,累遷國子司業兼太常博士。先是,世宗以郊廟祭器止由有司相承制造,年代浸久,無所規式,乃命崇義檢討摹畫以聞。四年,崇義上之,乃命有司別造焉。



 五年,將禘於太廟,言事者以宗廟無祧室,不當行禘祫之禮。崇義援引故事上言,其略曰:「魏明帝以景初三年
 正月上仙,至五年二月祫祭,明年又禘,自茲後以五年為禘。且魏以武帝為太祖,至明帝始三帝,未有毀主而行禘祫。其證一也。宋文帝元嘉六年,祠部定十月三日大祠,其太學博士議云:案禘祫之禮,三年一,五年再。宋高祖至文帝裁亦三帝,未有毀主而行禘祫。其證二也。梁武帝用謝廣議,三年一禘,五年一祫,謂之大祭,禘祭以夏,祫祭以冬。且梁武乃受命之君,裁追尊四朝而行禘祫,則知祭者是追養之道,以時移節變,孝子感而思
 親,故薦以首時,祭以仲月,間以禘祫,序以昭穆,乃禮之經也。非關宗廟廟與未備。其證三也。」終從崇義之議。



 未幾,世宗詔崇義參定郊廟祭玉,又詔翰林學士竇儼統領之。崇義因取《三禮圖》再加考正,建隆三年四月表上之,儼為序。太祖覽而嘉之,詔曰:「禮器禮圖,相承傳用,浸歷年祀,寧免差違。聶崇義典事國庠,服膺儒業,討尋故實,刊正疑訛,奉職效官,有足嘉者。崇義宜量與酬獎。所進《三禮圖》,宜令太子詹事尹拙集儒學三五人更同參
 議,所冀精詳。茍有異同,善為商確。」五月,賜崇義紫袍、犀帶、銀器、繒帛以獎之。拙多所駁正,崇義復引經以釋之,悉以下工部尚書竇儀,俾之裁定。儀上奏曰:「伏以聖人制禮,垂之無窮,儒者據經,所傳或異,年祀浸遠,圖繪缺然。踳駁彌深,丹青靡據。聶崇義研求師說,耽味禮經,較於舊圖,良有新意。尹拙爰承制旨,能罄所聞。尹拙駁議及聶崇義答義各四卷,臣再加詳閱,隨而裁置,率用增損,列於注釋,共分為十五卷以聞。」詔頒行之。



 拙、崇義復
 陳祭玉鼎釜異同之說,詔下中書省集議。吏部尚書張昭等奏議曰:



 按聶崇義稱:祭天蒼璧九寸圓好,祭地黃琮八寸無好,圭、璋、琥並長九寸。自言周顯德三年與田敏等按《周官》玉人之職及阮諶、鄭玄舊圖,載其制度。



 臣等按:《周禮》玉人之職,只有「璧琮九寸」、「瑑琮八寸」及「璧羨度尺、好三寸以為度」之文,即無蒼璧、黃琮之制。兼引注有《爾雅》「肉倍好」之說,此即是注「璧羨度」之文,又非蒼璧之制。又詳鄭玄自注《周禮》,不載尺寸,豈復別作畫圖,違
 經立異?



 《四部書目》內有《三禮圖》十二卷,是隋開皇中敕禮官修撰。其圖第一、第二題云「梁氏」,第十後題云「鄭氏」,又稱不知梁氏、鄭氏名位所出。今書府有《三禮圖》,亦題「梁氏」、「鄭氏」,不言名位。厥後有梁正者,集前代圖記更加詳議,題《三禮圖》曰:「陳留阮士信受《禮》學於潁川綦冊君,取其說,為圖三卷,多不按《禮》文而引漢事,與鄭君之文違錯。」正刪為二卷,其阮士信即諶也。如梁正之言,可知諶之紕謬。兼三卷《禮圖》刪為二卷,應在今《禮圖》之內,亦
 無改祭玉之說。



 臣等參詳自周公制禮之後,叔孫通重定以來,禮有緯書,漢代諸儒頗多著述,討尋祭玉,並無尺寸之說。魏、晉之後,鄭玄、王肅之學各有生徒,《三禮》、《六經》無不論說,檢其書,亦不言祭玉尺寸。臣等參驗畫圖本書,周公所說正經不言尺寸,設使後人謬為之說,安得便入周圖?知崇義等以諸侯入朝獻天子夫人之琮璧以為祭玉,又配合「羨度」、「肉好」之言,強為尺寸,古今大禮,順非改非,於理未通。



 又據尹拙所述禮神之六玉,稱
 取梁桂州刺史崔靈恩所撰《三禮義宗》內「昊天及五精帝圭、璧、琮、璜皆長尺二寸,以法十二時;祭地之琮長十寸,以效地之數。」又引《白虎通》云:「方中圓外曰璧,圓中方外曰琮。」崇義非之,以為靈恩非周公之才,無周公之位,一朝撰述,便補六玉闕文,尤不合禮。



 臣等竊以劉向之論《洪範》,王通之作《元經》,非必挺聖人之姿,而居上公之位,有益於教,不為斐然。臣等以靈恩所撰之書,聿稽古訓,祭玉以十二為數者,蓋天有十二次,地有十二辰,日
 有十二時,封山之玉牒十二寸,園丘之籩豆十二列,天子以鎮圭外守,宗後以大琮內守,皆長尺有二寸。又祼圭尺二寸,王者以祀宗廟。若人君親行之郊祭,登壇酌獻,服大裘,搢大圭,行稽奠,而手秉尺二之圭,神獻九寸之璧,不及禮宗廟祼圭之數,父天母地,情亦奚安?則靈恩議論,理未為失,所以自《義宗》之出,歷梁、陳、隋、唐垂四百年,言禮者引為師法,今《五禮精義》、《開元禮》、《郊祀錄》皆引《義宗》為標準。近代晉、漢兩朝,仍依舊制。周顯德中,田
 敏等妄作穿鑿,輒有更改。自唐貞觀之後,凡三次大修五禮,並因隋朝典故,或節奏繁簡之間稍有厘革,亦無改祭玉之說。伏望依《白虎通》、《義宗》、唐禮之制,以為定式。



 又尹拙依舊圖畫釜,聶崇義去釜畫鑊。臣等參詳舊圖,皆有釜無鑊。按《易。說卦》云「坤為釜」,《詩》云「惟錡及釜」,又云「溉之釜鬲」,《春秋傳》云「錡釜之器」,《禮記》云「燔黍捭豚」,解云「古未有甑釜,所以燔捭而祭。」即釜之為用,其來尚矣,故入於《禮圖》。今崇義以《周官》祭祀有省鼎鑊,供鼎鑊,又以《
 儀禮》有羊鑊、豕鑊之文,乃雲畫釜不如畫鑊。今諸經皆載釜之用,誠不可去。又《周》、《儀禮》皆有鑊之文,請兩圖之。又若觀諸家祭祀之畫,今代見行之禮,於大祀前一日,光祿卿省視鼎鑊。伏請圖鑊於鼎下。



 詔從之。未幾,崇義卒,《三禮圖》遂行於世,並畫於國子監講堂之壁。



 崇義為學官,兼掌禮,僅二十年,世推其該博。郭忠恕嘗以其姓嘲之曰:「近貴全為聵,攀龍即作聾。雖然三個耳,其奈不成聰。」崇義對曰:「僕不能為詩,聊以一聯奉答。」即云:「勿笑
 有三耳,全勝畜二心。」蓋因其名以嘲之。忠恕大慚,人許其機捷而不失正,真儒者之戲云。



 邢昺,字叔明,曹州濟陰人。太平興國初舉《五經》,廷試日,召升殿講《師》、《比》二卦,又問以群經發題。太宗嘉其精博,擢《九經》及第,授大理評事、知泰州鹽城監,賜錢二十萬。昺以是監處楚、泰間,泰僻左而楚會要,鹽食為急,請改隸楚州,從之。明年,召為國子監丞,專講學之任。遷尚書博士,出知儀州,就轉國子博士。代還,賜緋,選為諸王府
 侍講。雍熙中,遷水部員外郎,改司勛。端拱初,賜金紫,累遷金部郎中。



 真宗即位,改司勛郎中,俄知審刑院,以昺儒者不達刑章,命劉元吉同領其事。是冬,昺上表自陳夙侍講諷,遷右諫議大夫。咸平初,改國子祭酒。二年,始置翰林侍講學士,以昺為之。受詔與杜鎬、舒雅、孫奭、李慕清、崔偓牷等校定《周禮》、《儀禮》、《公羊》、《穀梁春秋傳》、《孝經》、《論語》、《爾雅義疏》,及成,並加階勛。俄為淮南、兩浙巡撫使。初置講讀之職,即於便坐令昺講《左氏春秋》,侍讀預焉。
 五年講畢,宴近臣於崇政殿,賜昺襲衣、金帶,加器幣,仍遷工部侍郎,兼國子祭酒、學士如故。知審官院陳恕丁內艱,以昺權知院事。



 景德二年,上言:「亡兄素嘗舉進士,願沾贈典。」特贈大理評事。是夏,上幸國子監閱庫書,問昺經版幾何,昺曰:「國初不及四千,今十餘萬,經、傳、正義皆具。臣少從師業儒時,經具有疏者百無一二,蓋力不能傳寫。今板本大備,士庶家皆有之,斯乃儒者逢辰之幸也。」上喜曰:「國家雖尚儒術,非四方無事,何以及此!」上
 又訪以學館故事,有未振舉者,昺不能有所建明。先是,印書所裁餘紙,鬻以供監中雜用,昺請歸之三司,以裨國用。自是監學公費不給,講官亦厭其寥落。上方興起道術,又令昺與張雍、杜鎬、孫奭舉經術該博、德行端良者,以廣學員。三年,加刑部侍郎。



 昺居近職,常多召對,一日從容與上語及宮邸舊僚,嘆其淪喪殆盡,唯昺獨存。翌日,賜白金千兩,且詔其妻至宮庭,賜以冠帔。四年,昺以羸老艱於趨步上前,自陳曹州故鄉,願給假一年歸
 視田里,俟明年郊祀還朝。上命坐,慰勞之,因謂曰:「便可權本州,何須假耶?」昺又言楊礪、夏侯嶠同為府僚,二臣沒皆贈尚書。上憫之,翌日,謂宰相曰:「此可見其志矣。」即超拜工部尚書、知曹州、職如故。



 入辭日,賜襲衣、金帶。是日,特開龍圖閣,召近臣宴崇和殿,上作五、七言詩二首賜之,預宴者皆賦。昺視壁間《尚書》、《禮記圖》,指《中庸》篇曰:凡為天下國家有九經。因陳其大義,上嘉納之。及行,又令近臣祖送,設會於宜春苑。大中祥符初,上東封泰山,
 昺表曹州民請車駕經由本州,仍令濟陰令王範部送父老詣闕,優詔答之。俄召還。車駕進發,命判留司御史臺。禮畢,進位禮部尚書。



 上勤政憫農,每雨雪不時,憂形於色,以昺素習田事,多委曲訪之。初,田家察陰晴豐兇,皆有狀候,老農之相傳者率有驗,昺多採其說為對。又言:「民之災患,大者有四:一曰,疫,二曰旱,三曰水,四曰畜。災歲必有其一,但或輕或重耳。四事之害,旱暵為甚,蓋田無畎澮,悉不可救,所損必盡。《傳》曰:『天災流行,國家代
 有。』此之謂也。」



 三年,被病請告,詔太醫診視。六月,上親臨問疾,賜名藥一奩、白金器千兩、繒彩千匹。國朝故事,非宗戚將相,無省疾臨喪之禮,特有加於昺與郭贄者,以恩舊故也。未幾,有旨命中書召其子太常博士知東明縣仲寶、國子博士知信陽軍若思還侍疾。逾月卒,年七十九,則左僕射,三子並進秩。



 初,雍熙中,昺撰《禮選》二十卷獻之,太宗探其帙,得《文王世子篇》,觀之甚悅,因問衛紹欽曰:「昺為諸王講說,曾及此乎?」紹欽曰:「諸王常時訪
 昺經義,昺每至發明君臣父子之道,必重復陳之。」太宗益喜。上嘗因內閣暴書,覽而稱善,召昺同觀,作《禮選贊》賜之。昺言:「家無遺稿,願得副本。」上許之。繕錄未畢而昺卒,亟詔寫二本,一本賜其家,一本俾置塚中。



 昺在東宮及內庭,侍上講《孝經》、《禮記》、《論語》、《書》、《易》、《詩》、《左氏傳》。據傳疏敷引之外,多引時事為喻,深被嘉獎。上嘗問:「管仲、召忽皆事公子糾,小白之入,召忽死之,管仲乃歸齊相桓公。豈非召忽以忠死,而管仲不能固其節,為臣之道當若
 是乎?又鄭注《禮記。世子篇》云:『文王以勤憂損壽,武王以安樂延年。』朕以為本經旨意必不然也。且夏禹焦勞,有玄圭之賜,而享國永年。若文王能憂人之心,不自暇逸,縱無感應,豈至虧損壽命耶?」各隨其事理以對。



 先是,咸平中,王欽若知貢舉,有告其受舉人賄賂者,下御史臺鞫得狀,欽若自訴,詔昺與邊肅、毋賓古、閻承翰就太常寺覆推。昺力辨欽若,而洪湛抵罪,欽若以是德之。昺之厚被寵顧,欽若與有功焉。



 仲寶貪猥不才,舉止率易,士
 大夫多鄙笑之。欽若在中書,用為三司判官,後至祠部郎中,坐贓黜官,卒。若思終於駕部郎中。



 孫奭,字宗古,博州博平人。幼與諸生師里中王徹,徹死,有從奭問經者,奭為解析微指,人人驚服,於是門人數百皆從奭。後徙居須城。



 《九經》及第,為莒縣主簿,上書願試講說,遷大理評事,為國子監直講。太宗幸國子監,召奭講《書》,至「事不師古,以克永世,匪說攸聞」,帝曰:「此至言也。商宗乃得賢相如此耶!」因咨嗟久之。賜五品服。真宗
 以為諸王府侍讀。會詔百官轉對,奭上十事。判太常禮院、國子監、司農寺,累遷工部郎中,擢龍圖閣待制。



 奭以經術進,守道自處,即有所言,未嘗阿附取悅。大中祥符初,得天書於左承天門,帝將奉迎,召宰相對崇政殿西廡。王旦等曰:「天貺符命,實盛德之應。」皆再拜稱萬歲。又召問奭,奭對曰:「臣愚,所聞『天何言哉』,豈有書也?」帝既奉迎天書,大赦改元,布告其事於天下,築玉清昭應宮。是歲,天書復降泰山,帝以親受符命,遂議封禪,作禮樂。王
 欽若、陳堯叟、丁謂、杜鎬、陳彭年皆以經義左右附和,由是天下爭言符瑞矣。



 四年,又將祀汾陰,是時大旱,京師近郡穀踴貴,奭上疏諫曰:「先王卜征,五年歲習其祥,祥習則行,不習則增修德而改卜。陛下始畢東封,更議西幸,殆非先王卜征五年慎重之意,其不可一也。夫汾陰后土,事不經見。昔漢武帝將封禪,故先封中嶽,祠汾陰,始巡幸郡縣,遂有事於泰山。今陛下既已登封,復欲幸汾陰,其不可二也。古者圜丘方澤,所以郊祀天地,今南
 北郊是也。漢初承秦,唯立五畤以祀天,而後土無祀,故武帝立祠於汾陰。自元、成以來,從公卿之議,遂徙汾陰后土於北郊,後之王者多不祀汾陰。今陛下已建北郊,乃舍之而遠祀汾陰,其不可三也。西漢都雍,去汾陰至近。今陛下經重關,越險阻,輕棄京師根本,而慕西漢之虛名,其不可四也。河東,唐王業之所起也。唐又都雍,故明皇間幸河東,因祠后土。聖朝之興,事與唐異,而陛下無故欲祀汾陰,其不可五也。昔者周宣王遇災而懼,故
 詩人美其中興,以為賢主。比年以來,水旱相繼,陛下宜側身修德,以答天譴,豈宜下徇奸回,遠勞民庶,盤游不已,忘社稷之大計?其不可六也。夫雷以二月啟蟄,八月收聲,育養萬物,失時則為異。今震雷在冬,為異尤甚。此天意丁寧以戒陛下,而反未悟,殆失天意,其不可七也。夫民,神之主也,是以聖王先成民而後致力於神。今國家土木之功累年未息,水旱幾沴,饑饉居多,乃欲勞民事神,神其享之乎?此其不可八也。陛下必欲為此者,不
 過效漢武帝、唐明皇,巡幸所至,刻石頌功,以崇虛名,誇示後世爾。陛下天資聖明,當慕二帝、三王,何為下襲漢、唐之虛名,其不可九也。唐明皇以嬖寵奸邪,內外交害,身播國屯,兵交關下,亡亂之跡如此,由狃於承平,肆行非義,稔致禍敗。今議者引開元故事以為盛烈,乃欲倡導陛下而為之,臣切為陛下不取,此其不可十也。臣言不逮意,陛下以臣言為可取,願少賜清問,以畢臣說。」



 帝遣內侍皇甫繼明就問,又上疏曰:



 陛下將幸汾陰,而京
 師民心弗寧,江、淮之眾困於調發,理須鎮安而矜存之。且土木之功未息,而奪攘之盜公行,外國治兵,不遠邊境,使者雖至,寧可保其心乎?昔陳勝起於徭戍,黃巢出於兇饑,隋煬帝勤遠略而唐高祖興於晉陽,晉少主惑小人而耶律德光長驅中國。陛下俯從奸佞,遠棄京師,涉仍歲薦饑之墟,修違經久廢之祠,不念民疲,不恤邊患。安知今日戍卒無陳勝,饑民無黃巢,英雄將無窺伺於肘腋,外敵將無觀釁於邊陲乎?



 先帝嘗議封禪,寅畏
 天種,尋詔停寢。今奸臣乃贊陛下力行東封,以為繼成先志。先帝嘗欲北平幽朔,西取繼遷,大勛未集,用付陛下,則群臣未嘗獻一謀、畫一策,以佐陛下繼先帝之志者,反務卑辭重幣,求和於契丹,蹙國縻爵,姑息於繼遷,曾不思主辱臣死為可戒,誣下罔上為可羞。撰造祥瑞,假托鬼神,才畢東封,便議西幸,輕勞車駕,虐害饑民,冀其無事往還,便謂成大勛績。是陛下以祖宗艱難之業,為奸邪僥幸之資,臣所以長嘆而痛哭也。夫天地神祇,
 聰明正直,作善降之百祥,作不善降之百殃,未聞專事籩豆簠簋,可邀福祥。《春秋傳》曰:「國之將興,聽於民;將亡,聽於神。」愚臣非敢妄議,惟陛下終賜裁擇。



 後天下數有災變,又言:「古者五載巡守,有國之事爾,非必有紫氣黃雲,然後登封,嘉禾異草,然後省方也。今野雕山鹿,郡國交奏,秋旱冬雷,群臣率賀,退而腹非竊笑者比比皆是。孰謂上天為可罔,下民為可愚,後世為可欺乎?人情如此,所損不細,惟陛下深鑒其妄。」



 六年,又上疏曰:「陛下封
 泰山,祀汾陰,躬謁陵寢,今又將祠於太清宮,外議籍籍,以謂陛下事事慕效唐明皇,豈以明皇為令德之主耶?甚不然也。明皇禍曹之跡有足為深戒者,非獨臣能知之,近臣不言者,此懷奸以事陛下也。明皇之無道,亦無敢言者,及奔至馬嵬,軍士已誅楊國忠,請矯詔之罪,乃始諭以識理不明,寄任失所。當時雖有罪己之言,覺寤已晚,何所及也。臣願陛下早自覺寤,抑損虛華,斥遠邪佞,罷興土木,不襲危亂之跡,無為明皇不及之悔,此天
 下之幸,社稷之福也。」帝以為「封泰山,祠汾陰,上陵,祀老子,非始於明皇。《開元禮》今世所循用,不可以天寶之亂,舉謂為非也。秦為無道甚矣,今官名、詔妙、郡縣猶襲秦舊,豈以人而廢言乎?」作《解疑論》以示群臣。然知奭樸忠,雖其言切直,容之而弗斥。



 久之,以父老請歸田里,不許,以知密州。居二年,遷左諫議大夫,罷待制。還,糾察在京刑獄。是時初置天慶、天祺、天貺、先天、降聖節,天下設齋醮張燕,費甚廣。奭又請裁省浮用,不報。復出知河陽,又
 求解官就養,遷給事中,徙坰州。



 天禧中,朱能獻《乾祐天書》。復上疏曰:



 朱能者,奸憸小人,妄言祥瑞,而陛下崇信之,屈至尊以迎拜,歸秘殿以奉膊,上自朝廷,下及閭巷,靡不痛心疾首,反唇腹非,而無敢言者。



 昔漢文成將軍以帛書飯牛,既而言牛腹中有奇書,殺視得書,天子識其手跡。又有五利將軍妄言,方多不仇,二人皆坐誅。先帝時有侯莫陳利用者,以方術暴得寵用,一旦發其奸,誅於鄭州。漢武可謂雄材,先帝可謂英斷。唐明皇得《靈
 寶符》、《上清護國經》、《寶券》等,皆王鉷、田同秀等所為,明皇不能顯戮,怵於邪說,自謂德實動天,神必福我。夫老君,聖人也。儻實降語,固宜不妄,而唐自安、史亂離,乘輿播越,兩都蕩覆,四海沸騰,豈天下太平乎?明皇雖況得歸闕,復為李輔國劫遷,卒以憂終,豈聖壽無疆、長生久視乎?以明皇之英睿,而禍患猥至曾不知者,良由在位既久,驕亢成性,謂人莫己若,謂諫不足聽#心玩居常之安,耳熟導諛之說,內惑寵嬖,外任奸回,曲奉鬼神,過崇妖
 妄。今收見老君於閣上,明日見老君於山中。大臣尸祿以將迎,端士畏威而緘默。既惑左道,既紊政經,民心用離,變起倉卒。當是之時,老君寧肯御兵,寶符安能排蒲邪?今朱能所為,或類於此,願陛下思漢武之雄材,法先帝之英斷,鑒明皇之蹤禍,庶幾災害不生,禍亂不作。



 未幾,能果敗。奭又嘗請減修寺度僧,帝雖未用其言,嘗令向敏中諭令陳時政訪失奭以納諫、恕直、輕徭、薄斂四事為言,頗施行焉。



 仁宗即位,宰相請擇名儒以經術侍
 講讀,乃召為翰林侍講學士、知審官院,判機子監,修《真宗實錄》。丁父憂,起復,兼判太常寺及禮院,三遷兵部侍郎、龍霞閣學士。每講論至前世亂君亡國,必反覆規諷。仁宗意或不在書,昺則拱默以酃,帝為竦然改聽。嘗畫《無逸圖》上之,帝施於講讀閣。時章憲明肅皇后每五收一御殿,與帝同聽政,奭言:「古帝王朝朝暮夕,未有曠日不朝。陛下宜每日御殿,以覽萬機。」奏留中不報。然帝與皇太后尤愛重之,每進見,未嘗不加禮。



 三請致仕。召對
 承明殿,敦諭之,以年逾七十固請,泣下,帝亦惻然,詔與馮知講《老子》三章,各賜帛二百匹。以不得請,求近郡,優拜工部尚書,復知兗州。詔須宴而後行,又留數月,特宴太清樓,近臣皆預,帝作飛白大字以賜二府,而小字賜諸學生,獨奭與晁迥兼賜大小字。詔群臣即席賦詩,太后又別出禁中卒器勸酒。翌日,奭入謝,又命講《老子》,賜襲衣、金帶、銀鞍勒馬。及行,賜隅瑞聖園,又賜詩,詔近臣皆賦。以恭謝恩改禮部尚書,既而累表乞歸,以太子速
 傅致仕。疾甚,徙正寢,屏婢妾,謂子瑜曰:「無令我死婦人之手。」卒。奏至,帝謂張士遜曰:「朕方欲召奭還,而奭遂死矣。」嗟惜者久之,罷朝一日,贈左僕射,謚曰宣。



 奭性方重,事親篤孝。父亡,舐其面以代颒。常掇《五經》切於治道者,為《經典徽言》五十卷。又撰《崇祀錄》、《樂記圖》、《五經節解》、《五服制度》。嘗奉詔與邢昺、杜鎬校定諸經正義,《莊子》、《爾雅》釋文,考正《尚書》、《論語》、《孝經》、《爾雅》謬誤及律音義。



 初,圜丘無外壝,五郊從祀不設席,尊不施冪;七祠時饗飲福用一尊,不設數登,升歌不以《雍》徹;
 冬至攝祀昊天上帝,外級止十七位,而不以星辰從;饗先農在祈穀之前;上丁釋奠無三獻;宗廟不備二舞;諸臣當謚者,或既葬乃請。奭吭援古奏正,遂著於禮。又請冬至罷祀五帝,大雩設五帝而罷祠昊天上帝。事下有司議,不合而止。



 瑜,官至工部侍郎致仕。



 王昭素,開封酸棗人。少篤學不仕,有至行,為鄉里所稱。常聚徒教授以自給,李穆與弟肅及李惲皆常師事焉。鄉人爭訟,不詣官府,多就昭素決之。



 昭素博通《九經》,兼
 究《莊》、《老》,尤精《詩》、《易》,以為王、韓注《易》及孔、馬疏義或未盡是,乃著《易論》二十三篇。



 開寶中,穆薦之朝,詔召赴闕,見於便殿,時年七十七,精神不衰。太祖問曰:「何以不求仕進,致相見之晚?」對曰:「臣草野蠢愚,無以裨聖化。」賜坐,令講《易。乾卦》,召宰相薛居正等觀之,至「飛龍在天」,上曰:「此書豈可令常人見?」昭素對曰:「此書非聖人出不能合其象。」因訪以民間事,昭素所言誠實無隱,上嘉之。以衰老求歸鄉里,拜國子博士致仕,賜茶藥及錢二十萬,留月
 餘,遣之。年八十九,卒於家。



 昭素頗有人倫鑒。初,李穆兄弟從昭素學《易》,常謂穆曰:「子所謂精理,往往出吾意表。」又語人曰:「穆兄弟皆令器,穆尤沈厚,他日必至廊廟。」後果參知政事。



 昭素每市物,隨所言而還直,未嘗論高下。縣人相告曰:「王先生市物,無得高取其價也。」治所居室,有椽木積門中,夜有盜者抉門將入,昭素覺之,即自門中潛擲椽於外,盜者慚而去,由是里中無盜。家有一驢,人多來假,將出,先問僮奴曰:「外無假驢者乎?」對云「無」,然
 後出。其為純質若此。



 子仁著,亦有隱德。



 孔維,字為則,開封雍丘人。乾德四年《九經》及第,解褐東明、鄢陵二主簿。開寶中,禮部再奏為考試官,調滁州軍事推官。太宗即位,擢授太子左贊善大夫、知河南縣,通判滑、梓二州。太平興國中,就拜國子《周易》博士,代還,遷《禮記》博士。七年,使高麗,王治問禮於維,維對以君父臣子之道,升降等威之序,治悅,稱之曰:「今日復見中國之夫子也。」九年,判國學事。雍熙初,遷主客員外郎。三年,擢
 為國子司業,賜金紫。



 會將有事於籍田,維起《周禮》至於《唐書》,凡沿革制度並錄之以獻,觀者稱其博。又上書請禁原蠶以利國馬。直史館樂史駁之曰:



 《管子》云:「倉稟實,知禮節;衣食足,知榮辱。」是以古先哲王厚農桑之業,以其為衣食之原耳。一夫不耕,天下有受其饑者;一婦不蠶,天下有受其寒者。故天子親耕,後妃親蠶,屈身以化下者,邦國之重務也。《吳都賦》曰:「國賦再熟之稻,鄉貢八蠶之綿。」則蠶之有原,其來舊矣。今孔維請禁原蠶以利
 國馬,徒引前經物類同氣之文,不究時事確實之理。夫所市國馬來自外方,涉遠馳驅,虧其秣飼,失於善視,遂至玄黃,致斃之由,鮮不以此。今乃欲禁其蠶事,甚無謂也。唐朝畜馬,具存監牧之制,詳觀本書,亦無禁蠶之文。況近降明詔,來年春有事於籍田,是則勸農之典方行,而禁蠶之制又下,事相違戾,恐非所長。



 臣嘗歷職州縣,粗知利病,編民之內,貧窶者多,春蠶所成,止充賦調之備,晚蠶薄利,始及卒歲之資。今若禁其後圖,必有因緣
 為弊,滋彰撓亂,民豈皇寧。渙汗絲綸,所宜重慎。



 上覽之,遂寢晚蠶之禁。維復抗疏曰:



 按《周禮。夏官。司馬》職禁原蠶者,為傷馬也。原,再也。天文,辰為馬。《蠶書》,蠶為龍精,月直大火,則浴其種。是蠶與馬同氣,物莫能兩大,故禁再蠶以益馬也。又郭璞云:「重蠶為原,今晚蠶也。」臣少親耕桑之務,長歷州縣之職,物之利害,盡知之矣。蚩蚩之氓知其利而不知其害,故有早蠶之後,重養晚蠶之繭,出絲甚少,再採之葉來歲不茂,豈止傷及於馬,而桑亦損
 矣。臣自縣歷官,路見坰野之地官馬多死,若非明援典據,助其畜牧,安敢妄有舉陳哉。



 按《本草》注:「以殭蠶塗馬齒,則不能食草。」物類相感如此。《月令》仲春祭馬祖,季春享先蠶,皆謂天駟房星也,為馬祈福,謂之馬祖,為蠶祈福,謂之先蠶,是蠶與馬同其類爾。蠶重則馬損,氣感之而然也。臣謂依《周禮》禁原蠶為當。



 上雖不用維言,而嘉其援引經據,以章付史館。籍田畢,拜國子祭酒。淳化初,兼工部侍郎。二年,卒,年六十四。



 維通經術。準舊制,舉《九
 經》,一上不中第即改科。開寶中,維論其事非便,詔禮部自今《九經》同諸科許再赴舉。



 太宗尹京日,維為屬邑吏,頗以經術受知。即位後,維始升郎署。自以通經,求為司業,即以授之。使外國者皆假服紫,維自高麗還,會東使至,維自恥衣緋,因求見上,詭言:「高麗使問臣獲何罪降服,臣無以對。」因泣下。上憐之,即賜以金紫。及為祭酒,又奏言:「朝廷久不置此官,少有知者,臣之親戚故舊有書信來者,多云祭酒郎中。田敏晉朝任祭酒,仍兼侍郎。願
 循前例,兼領是官,庶獲美稱。」上從之。然縉紳惡其儒者躁求,無退讓之風。



 嘗建議乞廣太學,上以侵壞民舍不許。受詔與學官校定《五經疏義》,刻板行用,功未及畢,被病,上遣太醫診視,使者撫問。初,維私用印書錢三十餘萬,為掌事黃門所發,維憂懼,遽以家財賞之,疾遂亟,上赦而不問。維將終,召其婿鄭革口授遺表,以《五經疏》未畢為恨。



 景德四年,錄其孫禹圭同學究出身。



 孔宜,字不疑,兗州曲阜人,孔子四十四世孫。孔子生鯉,
 字伯魚。鯉生伋,字子思。伋生白,字子上。白生求,字子家。求生箕,字子京。箕生穿,字子高。穿生謙,字子慎。謙生鮒,字子魚,以弟子騰為嗣。騰字子襄,值秦難,藏其家書於屋壁。騰生正,字季忠。忠生武。武生延年及安國。延年生霸,字次孺,漢昭帝時為博士,宣帝時為太中大夫,授皇太子經。元帝即位,賜爵關內侯,號褒成君。霸生福。福生房。房生均,字長平,好學有才,為尚書郎,平帝元始元年,封均為褒成侯,食邑二千戶,追謚夫子為褒成宣尼公。
 王莽以均為太尉,三以疾辭,得還,莽敗,失國。後漢世祖建武十四年,復封均子志為褒成侯,謚元成。志生損,襲爵,和帝永元四年,徙封損為褒亭侯。損卒,子曜嗣侯、邑千戶。子完嗣,邑百戶。完早卒無子,以弟子羨襲爵。



 羨仕魏為議郎,黃初二年,封宗聖侯、邑百戶。羨生震,晉武帝泰始三年,徙封奉聖亭侯,邑二百戶,歷太常、黃門侍郎。震生嶷。嶷生撫,舉孝廉,闢太尉掾,歷豫章太守。撫生懿。懿生鮮,有度量,好學,宋文帝元嘉十九年,襲封奉聖侯。
 鮮生乘,博學有才藝,後魏孝文延興初舉孝廉三年,封乘為崇聖大夫,復十戶,以供灑掃。乘生靈珍,襲爵,歷秘書郎,太和十九年,改封崇聖侯,邑百戶。靈珍生文泰。文泰生渠,北齊文宣帝天保元年,改封恭聖侯。後周宣帝大象二年,追封孔子為鄒國公,以渠襲爵,邑百戶。



 渠生長孫,隋文帝復封長孫為鄒國公。長孫生嗣哲,應制舉,歷涇州司兵參軍、太子通事舍人,大業四年,改封紹聖侯、邑百戶。嗣哲生德倫,唐太宗貞觀十一年,封褒聖侯,
 邑百戶,朝會位同三品,復其子孫。則天天授二年,賜德倫璽書、衣服。德化生崇基,襲侯,中宗神龍元年,授朝散大夫。崇基生璲之,玄宗開元中,歷國子四門博士、邠王府文學、蔡州長史。二十七年,詔追謚孔子為文宣王,改封褒聖侯禘之為襲文宣公,兼兗州長史。璲之生萱,襲封,歷兗州泗水令。萱生齊卿,德宗建中三年,詔以齊卿為兗州司馬,陷於東平,卒。至憲宗元和十三年,平李師道,其子惟晊歸魯,詔以惟晊為兗州參軍,奉夫子祀,復五
 十戶,以供灑掃。惟晊生策,會昌元年,歷國子監丞、尚書博士。大中元年,宰相白敏中奏歲給封戶絹百匹,充春秋奉祀。自璲之至策,五世並襲封文宣公。策生振,懿宗咸通四年,舉進士甲科,歷兗州觀察判官,至刑部員外郎。振生昭儉,歷袞州司馬、曲阜令。自策至昭儉,三世歲給封絹,以供享祀。昭儉生光嗣,哀帝天祐中,為泗水主簿,奉孔子祀。



 光嗣生仁玉,九歲通《春秋》,姿貌雄偉。後唐明宗長興元年,以為曲阜主簿,三年,遷龔丘令,襲文宣
 公,晉高祖天福五年,改曲阜令。周高祖廣順二年,平慕容彥超,幸曲阜,拜孔子廟及墓,召仁玉,賜五品服,復以為本縣令。



 仁玉四子,長曰宜,舉進士不第,乾德中詣闕上書,述其家世,詔以為曲阜主簿,歷黃州軍事推官,遷司農寺丞,掌星子鎮市征。宜上言:「星子當江湖之會,商賈所集,請建為軍。」詔以為縣,就命宜知縣事,後以為南康軍。



 宜代還,獻文賦數十篇,太宗覽而嘉之,召見,問以孔子世嗣,因下詔曰:「素王之道,百代所崇,傳祚襲封,抑
 存典制。文宣王四十四代孫、司農寺丞宜服勤素業,砥礪廉隅,亟歷官聯,洽聞政績,聖人之後,世德不衰,俾登朝倫,以光儒胄。可太子右贊善大夫,襲封文宣公,復其家。」未幾,通判密州。太平興國八年,詔修曲阜孔子廟,宜貢方物為謝,詔褒之,遷殿中丞。雍熙三年,王師北征,受詔督軍糧,涉拒馬河溺死,年四十六。



 子延世字茂先,以父死事,賜學究出身,為曲阜主簿,歷閩、長葛二令。真宗至道三年十一月,召赴闕,以為曲阜令,襲封文宣公,賜
 白金、束帛及太宗御書印《九經》。咸平三年,詔本道轉運使、本州長吏待以賓禮,仍留三年,卒官,年三十八。次曰憲,太平興國二年進士及第,至工部員外郎、知浚儀縣。次曰冕,應城主簿。次曰勖,雍熙中進士及第。



 延世子聖祐,景德初,始九歲,特賜同學究出身。大中祥符元年,東封泰山,特聽聖祐衣綠陪位,綴京官班後。及還至兗州,十一月朔,幸曲阜,謁孔子廟,行酌獻之禮,孔氏宗屬並令倍位。又幸孔林,觀其墓久之。又御北亭,召從臣觀古
 碑,加謚孔子為玄聖文宣王,追封孔子父叔梁紇齊國公,母顏氏魯國太夫人。擢聖祐為太常寺奉禮郎,又錄其近屬進士謂同《三傳》出身,習進士延祐、習學究延渥、延魯、延齡並同學究出身,共賜銀二百兩、絹三百匹,以充奉祠廟。時勖為殿中丞、通判廣州,王欽若言其有聲於鄉曲,召赴闕,改太常博士,賜緋,令知曲阜縣,專主祠廟。二年三月,又遣使賜太宗御書及《九經》書疏、《三史》藏於廟,令本州選儒生講說。聖祐後改大理評事。天禧五
 年,授光祿寺丞,襲封文宣公、知仙源縣事。後改名祐,遷太子中舍,卒,年三十。



 勖為司封郎中。延魯,大中祥符五年復舉進士及第,後改名道輔,為左司諫、龍圖閣待制,自有傳。



 崔頌,字敦美,河南偃師人。父協,後唐門下侍郎、平章事。頌幼喪母,為外祖母所鞠養。以蔭補河南府巡官,歷開封主簿、鄧州錄事參軍,以疾去官。未幾,詣闕上書言事,宰相桑維翰覽而奇之,擢為左拾遺,選右補闕。



 漢初,加
 朝散階,副右散騎常侍張煦冊錢俶為吳越王。梁末,協嘗使兩浙,至是,越人美之,贈賄甚厚。及還,值周祖入京師,為軍士剽奪悉盡。世宗鎮澶淵,擇僚佐,頌與王樸、王敏中皆中其選,以頌為觀察判官,贈金紫。世宗尹京,拜司封員外郎、充判官,以斷獄誤失罷職,守本官。即位,拜駕部郎中,遷吏部,復副尹日就使兩浙。世宗讀唐元稹《均田疏》,命寫為圖賜近臣,遣使均諸道租賦,頌使兗州,頗增舊額。恭帝嗣位,改左諫議大夫。



 宋初,判國子監。會
 重修國學及武成王廟,命頌總領其事。建隆三年夏,始會生徒講說,太祖遣中使以酒果賜之。每臨幸國學,召頌與語。因及經義,頌應答無滯。及郊祀,以頌攝太僕,升車執綏,上問以一時典禮,頌占對閑雅,上甚重之。未幾,坐請托有司為所親求便官,出為保大軍行軍司馬。乾德六年,暴得疾卒,年五十。



 頌好詼諧,善筆札,受命書世宗謚冊文,當時稱其遒麗。篤信釋氏,睹佛像必拜。性多疑,在鄜州官舍,嘗召圬墁者治堂室,以帛蒙其目,人皆
 笑之。



 子曉,至太子右贊善大夫。



 曥字文炳,雍熙二年進士,淹雅有士行,累為屯田員外郎、開封三司戶部判官。景德中,雍王元份薨,府官皆坐黜。時戚維為曹國公元儼府翊善,上謂宰相曰:「元儼年少,尤資贊導,維迂懦循默,不能規戒,聞崔曥性純謹,以之代維,庶有裨益。」因召對,遷都官員外郎,充記室參軍,賜金紫。遷兵部郎中,出知河中府,轉太常少卿、將作監,卒。



 尹拙,潁州汝陰人。梁貞明五年舉《三史》,調補下邑主簿,
 攝本鎮館驛巡官。後唐長興中,召為著作佐郎、直史館,遷左拾遺,依前直史館,加朝散大夫。應順初,出為宣武軍掌書記、檢校虞部員外郎兼殿中侍御史。清泰初,加檢校駕部員外郎兼御史大夫。二年,改檢校虞部郎中、忠武軍掌書記。



 晉天福四年,入為右補闕。明年,轉侍御史。會詔拙與張昭、呂琦等同修《唐史》,改倉部員外郎,賜金紫。八年,遷左司員外郎。契丹入寇,趙延壽鎮常山。以拙為掌書記。漢初,召為司馬郎中、弘文館直學士。



 周廣
 順初,遷庫部郎中兼太常博士,仍充直學士。奉使荊南還,改兵部郎中。顯德初,拜檢校右散騎常侍、國子祭酒、通判太常禮院事,與張昭同修唐應順、清泰及周《祖實錄》,又與昭及田敏同詳定《經典釋文》。丁憂,免。宋初,改檢校工部尚書、太子詹事、判太府寺,遷秘書監、判大理寺。乾德六年告老,以本官致事。



 拙性純謹,博通經史。周世宗北征,命翰林學士為文祭白馬祠,學士不知所出,遂訪於拙,拙歷舉郡國祠白馬者以十數,當時伏其該博。
 開寶四年卒,年八十一。



 子季通,至國子博士。



 田敏,淄州鄒平人。少通《春秋》之學。梁貞明中登科,調補淄州主簿,不令之任,留為國子四門博士。後唐天成初,改《尚書》博士,賜緋。滿歲,為國子博士。上言請四郊置齋宮,不報。秩滿,轉屯田員外郎,以詳明典禮兼太常博士。建議請依《春秋》每歲藏冰薦宗廟,頒公卿,如古禮。奉詔與太常卿劉岳、博士段顒、路航、李居浣、陳觀等刪定唐鄭餘慶《書儀》,又詔與馬鎬等同校《九經》。改戶部員外郎,
 賜金紫。清泰初,遷國子司業。



 晉天福四年授祭酒,仍檢校工部尚書,俄兼戶部侍郎。開運初,遷兵部侍郎,充弘文館學士、判館事。議者以敏止可任學官,宰相桑維翰聞之,即改授檢校右僕射,復為祭酒。漢乾祐中,拜尚書右丞,判國子監。



 周廣順初,改左丞,遣使契丹,將歲賂錢十萬貫,止其侵剽,契丹不許。周祖將親郊,命權判太常卿事。世宗即位,真拜太常卿、檢校左僕射,加司空。顯德五年,上章請老,賜詔曰:「卿詳明禮樂,博涉典墳,為儒學
 之宗師,乃薦紳之儀表。朕方資舊德,以訪話言,遽覽封章,願致官政。引年之制難著舊文,尊賢之心方深虛佇,所請宜不允。」遷工部尚書。俄再上表願歸故鄉,以遂首丘之志,改太子少保致仕,歸淄州別墅。恭帝即位,加少傅。開寶四年,卒,年九十二。



 敏解官歸鄉,有良田數十頃,多釀美酒待賓客。體強少疾,徒步往來閭巷間,不以杖。每日親授諸子經。自作父墓碑,辭甚質。敏嘗使湖南,路出荊渚,以印本經書遺高從誨,從誨謝曰:「祭酒所遺經
 書,僕但能識《孝經》耳。」敏曰:「讀書不必多,十八章足矣。如《諸侯章》云『在上不驕,高而不危,制節謹度,滿而不溢』,皆至要之言也。」時從誨兵敗於郢,故敏以此諷之,從誨大慚。



 敏雖篤於經學,亦好為穿鑿,所校《九經》,頗以獨見自任,如改《尚書。盤庚》「若網在綱」為「若綱在綱」,重言「綱」字。又《爾雅》「椴,木槿」注曰:「日及」,改為「白及」。如此之類甚眾,世頗非之。



 子章,至殿中丞。



 辛文悅者,不知何許人。以《五經》教授,太祖幼時從其肄
 業。周顯德中,太祖歷禁衛為殿前都點檢,節制方面。文悅久不獲接見,一日,夢邀車駕請見,既拜,乃太祖也。太祖亦夢其來謁,因令左右尋訪,文悅果自至,太祖異之。及登位,召見,授太子中允,判太府事。開寶三年,出知房州。時周鄭王出居是州,上以文悅長者,故命焉。文悅後累遷至員外郎。



 又有張遁、張文旦者,嘗與太宗同學校,太平興國中,詣闕自言,各起家為主簿。



 李覺,字仲明,本京兆長安人。曾祖鼎,唐國子祭酒、蘇州
 刺史,唐末避亂,徙家青州益都。鼎生瑜,本州推官。瑜生成,字咸熙,性曠蕩,嗜酒,喜吟詩,善琴奕,畫山水尤工,人多傳秘其跡。周樞密使王樸將薦其能,會樸卒,鬱鬱不得志。乾德中,司農卿衛融知陳州,聞其名,召之,成因挈族而往,日以酣飲為事,醉死於客舍。



 子覺,太平興國五年舉《九經》,起家將作監丞、通判建州,秩將滿,州人借留,有詔褒之,就遷左贊善大夫、知泗州,轉秘書丞。太宗以孔穎達《五經正義》刊板詔孔維與覺等校定。王師征燕、
 薊,命覺部京東諸州芻糧赴幽州。維薦覺有學,遷《禮記》博士,賜緋魚。



 雍熙三年,與右補闕李若拙同使交州,黎桓謂曰:「此土山川之險,中朝人乍歷之,豈不倦乎?」覺曰:「國家提封萬里,列郡四百,地有平易,亦有險固,此一方何足云哉!」桓默然色沮。使還,久之,遷國子博士。



 端拱元年春,初令學官講說,覺首預焉。太宗幸國子監謁文宣王畢,升輦將出西門,顧見講坐,左右言覺方聚徒講書,上即召覺,令對御講。覺曰:「陛下六龍在御,臣何敢輒升
 高坐。」上因降輦,令有司張帟幕,設別坐,詔覺講《周易》之《泰卦》,從臣皆列坐。覺因述天地感通、君臣相應之旨,上甚悅,特賜帛百匹。



 俄獻時務策,上頗嘉獎。是冬,以本官直史館。右正言王禹偁上言:「覺但能通經,不當輒居史職。」覺仿韓愈《毛穎傳》作《竹穎傳》以獻,太宗嘉之,故寢禹偁之奏。淳化初,上以經書板本有田敏輒刪去者數字,命覺與孔維詳定。二年,詳校《春秋正義》成,改水部員外郎、判國子監。四年,遷司門員外郎,被病。假滿,詔不絕奉,
 卒。



 覺累上書言時務,述養馬、漕運、屯田三事,太宗嘉其詳備,令送史館,語見本志。覺性強毅而聰敏,嘗與秘閣校理吳淑等同考試開封府秋賦舉人,語及算雉兔首足法,覺曰:「此頗繁,吾能易之。」及成,果精簡。淑意其宿制,即試以別法,皆能立就,坐中皆嘆伏。



 子宥,大中祥符五年進士,為祠部員外郎、集賢校理。



 崔頤正,開封封丘人。與弟偓佺並舉進士,明經術。頤正雍熙中為高密尉,秩滿,國子祭酒孔維薦之,以為國學
 直講,遷殿中丞。太宗召見,令說《莊子》一篇,賜錢五萬。判監李至上言:「本監先校定諸經音疏,其間文字訛謬尚多,深慮未副仁君好古誨人之意也。蓋前所遣官多專經之士,或通《春秋》者未習《禮記》,或習《周易》者不通《尚書》,至於旁引經史,皆非素所傳習,以是之故,未得周詳。伏見國子博士杜鎬、直講崔頤正、孫奭皆苦心強學,博貫《九經》,問義質疑,有所依據。望令重加刊正,冀除舛謬。」從之。



 咸平初,又有學究劉可名言諸經版本多舛誤,真宗
 命擇官詳正,因訪達經義者,至方參知政事,以頤正對。曰:「朕宮中無事,樂聞講誦。」翌日,召頤正於苑中,說《尚書。大禹謨》,賜以牙緋。自是日令赴御書院待對,說《尚書》至十卷。頤正年老步趨艱蹇,表求致仕,上命坐,問恤甚至,賜器幣,聽以本官致仕,仍充直講,改國子博士。三年,卒,年七十九。



 偓佺,淳化中歷福州連江尉,判國子監李至奏為直講,引對便坐,太宗顧謂曰:「李覺嘗奏朕云,『四皓』中一先生,或言姓『用』字加撇,或云加點。爾知否?」偓牷曰:「
 昔秦時程邈撰隸書,訓如僕隸之易使也。今字與古或異。臣聞刀用為角音榷,兩點為TZ。音鹿,用上一撇一點俱不成字。」



 咸平二年,真宗幸國學,召偓佺說《尚書》,即特賜緋。景德後,令講《道德經》,日於崇文院候對。終篇,賜以白金繒彩。三年,卒,年七十九。嘗撰《帝王手鑒》十卷,並注曹唐《大游仙詩》十五卷。其子世安上之,特賜出身。



 李之才字挺之,青社人也。天聖八年同進士出身,為人樸且率,自信,無少矯厲。師河南穆修,修性莊嚴寡合,雖
 之才亦頻在訶怒中,之才事之益謹,卒能受《易》。時蘇舜欽輩亦從修學《易》,其專授受者惟之才爾。脩之《易》受之種放,放受之陳摶,源流最遠,其圖書象數變通之妙,秦、漢以來鮮有知者。



 之才初為衛州獲嘉主簿、權共城令。時邵雍居母憂於蘇門山百源之上,布裘蔬食,躬爨以養父。之才叩門來謁,勞苦之曰:「好學篤志果何似?」雍曰:「簡策之外,未有跡也。」之才曰:「君非跡簡策者,其如物理之學何?」他日,則又曰:「物理之學學矣,不有性命之學乎?」
 雍再拜,願受業,於是先示之以陸淳《春秋》,意欲以《春秋》表儀《五經》,既可語《五經》大旨,則授《易》而終焉。其後雍卒以《易》名世。



 之才器大,難乎識者,棲遲久不調。或惜之,則曰:「宜少貶以圖榮進。」石延年獨曰:「時不足以容君,盍不棄之隱去。」再調孟州司法參軍。時範雍守孟,亦莫之知也。雍初自洛建節守延安,送者皆出境外,之才獨別近郊。或病之,謝曰:「故事也。」頃之,雍謫安陸,之才沿檄見之洛陽,前日遠送之人無一來者,雍始恨知之之晚。



 友人
 尹洙以書薦於中書舍人葉道卿,因石延年致之,曰:「孟州司法參軍李之才,年三十九,能為古文章,語直意遂,不肆不窘,固足以蹈及前輩,非洙所敢品目,而安於卑位,無仕進意,人罕知之。其才又達世務,使少用於世,必過人遠甚,恨其貧不能決其歸心,知之者當共成之。」延年復書曰:「今業文好古之士至鮮且不張,茍遺若人,其學益衰矣。」延年素不喜謁貴仕,凡四五至道卿門,通其書乃已。道卿薦之,遂得應銓新格,有保任五人,改大理
 寺丞,為緱氏令。未行,會延年與龍圖閣直學士吳遵路調兵河東,闢之才澤州簽署判官。澤人劉羲叟從受歷法,世稱「羲叟歷法」,遠出古今上,有楊雄、張衡所未喻者,實之才授之。



 在澤轉殿中丞,丁母憂,甫除喪,暴卒於懷州官舍,慶歷五年二月也。時尹洙兄漸守懷,哭之才過哀,感疾,不逾月亦卒。之才歸葬青社,邵雍表其墓,有曰:「求於天下,得聞道之君子李公以師焉。」



\end{pinyinscope}