\article{列傳第八十八}

\begin{pinyinscope}

 常秩鄧綰子洵武李定舒但蹇周輔子序辰徐鐸王廣淵弟臨王陶王子韶何正臣陳繹



 常秩,字夷甫,穎州汝陰人。舉進士不中,屏居里巷,以經
 術著稱。嘉祐中,賜束帛,為穎州教授,除國子直講,又以為大理評事;治平中,授忠武軍節度推官、知長葛縣,皆不受。



 神宗即位,三使往聘,辭。熙寧三年,詔郡「以禮敦遣,毋聽秩辭」。明年,始詣闕,帝曰:「先朝累命,何為不起?」對曰:「先帝亮臣之愚,故得安閭巷。今陛下嚴詔趣迫,是以不敢不來,非有所抉擇去就也。」帝悅,徐問之:「今何道免民於凍餒?」對曰:「法制不立,庶民食侯食,服侯服,此今日大患也。臣才不適用,願得辭歸。」帝曰:「既來,安得不少留?異
 日不能用卿,乃當去耳。」即拜右正言、直集賢院、管幹國子監,俄兼直舍人院,遷天章閣侍講、同修起居注,仍使供諫職,復乞歸,改判太常寺。



 七年,進寶文閣待制兼侍讀,命其子立校書崇文院。九年,病不能朝,提舉中太一宮、判西京留司御史臺。還穎。十年,卒,年五十九,贈右諫議大夫。



 秩平居為學求自得。王回,里中名士也,每見秩與語,輒TB然自以為不及。歐陽修、胡宿、呂公著、王陶、沉遘、王安石皆稱薦之。翕然名重一時。



 初,秩隱居,既不肯
 仕,世以為必退者也。後安石為相更法,天下沸騰,以為不便,秩在閭閻,見所下令,獨以為是,一召遂起。在朝廷任諫爭,為侍從,低首抑氣,無所建明,聞望日損,為時譏笑。秩長於《春秋》,至斥孫復所學為不近人情。著講解數十篇,自謂「聖人之道,皆在於是」。及安石廢《春秋》,遂盡諱其學。



 立,始命為天平軍推官,秩死,使門人趙沖狀其行,云:「自秩與安石去位,天下官吏陰變其法,民受塗炭,上下循默,敗端內萌,莫覺莫悟。秩知其必敗。」紹聖中,蔡卞
 薦立為秘書省正字、諸王府說書侍講,請用為崇政殿說書,得召對,又請以為諫官。卞方與章惇比,曾布欲傾之,乘間為哲宗言立附兩人,因暴其行狀事,以為詆毀先帝。帝亟下史院取視,言其不遜,以責惇、卞,惇、卞懼,請貶立,乃黜監永州酒稅。



 鄧綰,字文約,成都雙流人。舉進士,為禮部第一。稍遷職方員外郎。熙寧三年冬,通判寧州。時王安石得君專政,條上時政數十事,以為宋興百年,習安玩治,當事更化。
 又上書言:「陛下得伊、呂之佐,作青苗、免役等法,民莫不歌舞聖澤。以臣所見寧州觀之,知一路皆然;以一路觀之,知天下皆然。誠不世之良法,願勿移於浮議而堅行之。」其辭蓋媚王安石。又貽以書頌,極其佞諛。



 安石薦於神宗,驛召對。方慶州有夏寇,綰敷陳甚悉。帝問安石及呂惠卿,以不識對。帝曰:「安石,今之古人;惠卿,賢人也。」退見安石,欣然如素交。宰相陳升之,馮京以綰練邊事,屬安石致齋,復使知寧州。綰聞之不樂,誦言:「急召我來,乃
 使還邪?」或問:「君今當作何官?」曰:「不失為館職。」「得無為諫官乎?」曰:「正自當爾。」明日,果除集賢校理、檢正中書孔目房。鄉人在都者皆笑且罵,綰曰:「笑罵從汝,好官須我為之。」



 尋同知諫院。獻所著《洪範建極錫福論》,帝曰:「《洪範》,天人、自然之大法,朕方欲舉而措諸天下,矯革眾敝。卿當SW淫朋比德之人,規以助朕。」綰頓首曰:「敢不力行所學,以奉聖訓。」明年,遷侍御史知雜事、判司農寺。



 時常平、水利、免役、保甲之政,皆出司農,故安石藉綰以威眾。綰請
 先行免役於府界,次及諸道。利州路歲用錢九萬六千緡,而轉運使李瑜率三十萬,綰言:「均役本以裕民,今乃務聚斂,積寬餘,宜加重黜。」富弼在亳,不散青苗錢,綰請付吏究治。畿縣民訴助役,詔詢其便否兩行之,綰與曾布輒上還堂帖。中丞楊繪言未聞司農得繳奏者,不報。凡呂公著、謝景溫所置推直官、主簿,悉罷去之,而引蔡確、唐坰為御史。



 五年春,擢御史中丞。國朝故事,未有臺雜為中丞者,帝特命之。又加龍圖閣待制。建言:「頃時御
 史罷免,猶除省府職司,蓋厥初選用既審,則議論雖不合,人材亦不可遺,願籍前後諫官、御史得罪者姓名,以次甄錄,使於進退間與凡僚稍異,則思竭盡矣。」



 遼人來理邊地,屯兵境上,聲言將用師,於是兩河戒嚴,且令河北修城守之具。綰曰:「非徒無益,且大擾費。」帝從其言而止。又言:「遼妄為地訟,意在窺我。去冬聚兵累月,逡巡自罷,其情偽可見。今當御之以堅強,則不渝二國之平,平則彼不我疑,而我得以遠慮。茍先之以畏屈,彼或將
 力爭,則大為中國之恥。」帝覽疏嘉之。



 安石去位,綰頗附呂惠卿。及安石復相,綰欲彌前跡,乃發惠卿置田華亭事,出知陳州。又論三司使章惇協濟其奸,出知湖州。初,惠卿弟和卿創手實法,綰曰:「凡民養生之具,日用而家有之。今欲盡令疏實,則家有告訐之憂,人懷隱匿之慮,無所措手足矣。商賈通殖貨財,交易有無,不過服食、器用、米粟、絲麻、布帛之類,或春有之而夏以蕩析,或秋貯之而冬已散亡,公家簿書,何由拘錄,其勢安得不犯?徒
 使嚚訟者趨賞報怨以相告訐,畏怯者守死忍困而已。」詔罷其法。遷翰林學士,仍為中丞。



 綰慮安石去失勢,乃上言宜錄安石子及婿,仍賜第京師。帝以語安石,安石曰:「綰為國司直,而為宰臣乞恩澤,極傷國體,當黜。」又薦彭汝礪為御史,安石不悅,遽自劾失舉。帝謂綰操心頗僻,賦性奸回,論事薦人,不循分守,斥知虢州。逾歲,為集賢院學士、知河陽,元豐中,以待制知荊南、陳、陜,徙永興軍,改青州。奏言歲大稔,鬥粟五七錢。帝知其佞,令提舉
 官酌市價以聞。進龍圖閣直學士、知鄧州。



 無祐初,徙揚州。言者論其奸,改滁州,未去鄧而卒,年五十九。子洵仁、洵武。洵仁,大觀中為尚書右丞。



 洵武字子常,第進士,為汝陽簿。紹聖中。哲宗召對,為秘書省正字、校書郎、國史院編修官,撰《神宗史》,議論專右蔡卞,詆誣宣仁後尤切,史禍之作,其力居多。遷起居舍人。



 徽宗初,改秘書少監,既而用蔡京薦,復史職,御史陳次升、陳師錫言:「洵武父綰在熙寧時以曲媚王安石,神
 宗數其邪僻奸回,今置洵武太史,豈能公心直筆,發揚神考之盛德,而不掩其父之惡乎?且其人材凡近,學問荒繆,不足以污此選。」不聽。遷起居郎。



 時韓忠彥、曾布為相,洵武因對言:「陛下乃先帝子,今相忠彥乃琦之子。先帝行新法以利民,琦嘗論其非,今忠彥為相,更先帝之法,是忠彥能繼父志,陛下為不能也。必欲繼志述事,非用蔡京不可。」京出居外鎮,帝未有意復用也,洵武為帝言:「陛下方紹述先志,群臣無助者。」乃作《愛莫助之圖》以
 獻。其圖如《史記》年表,列旁行七重,別為左右,左曰元豐,右曰元祐,自宰相、執政、侍從、臺諫、郎官、館閣、學校各為一重。左序助紹述者,執政中唯溫益一人,餘不過三四,若趙挺之、範致虛、王能甫、錢遹之屬而已。右序舉朝輔相、公卿、百執事咸在,以百數。帝出示曾布,而揭去左方一姓名。布請之,帝曰:「蔡京也。洵武謂非相此人不可,以與卿不同,故去之。」布曰洵武既與臣所見異,臣安敢豫議?」明日改付溫益,益欣然奉行,請籍異論者,於是決意
 相京。進洵武中書舍人、給事中兼侍講,修撰《哲宗實錄》,遷吏部侍郎。



 洵武疏言:「神宗稽古建官,既正省、臺、寺、監之職,而以寄祿階易空名矣。今在選七階,自兩使判官至主簿、尉,有帶知安州雲夢縣而為河東乾當公事者,有河中司錄參軍而監楚州鹽場者,有瀛州軍事推官、知大名府元城縣充濮州教授者,殽亂紛錯,莫甚於此。謂宜造為新名,因而制錄。」詔悉更之。遷刑部尚書,又請初出官人兼用刑法試,俾知為吏之方。崇寧三年,拜尚
 書右丞,轉左丞、中書侍郎。



 妖人張懷素獄興,其黨有與洵武連昏者,坐出知隨州。提舉明道宮,復端明殿學士,知亳州、河南府,召為中太一宮使,連進觀文殿學士,為大名尹。政和中,夏祭,入侍祠。以祐神觀使兼侍讀留修國史,改保大軍節度使。未幾,知樞密院。



 五溪蠻擾邊,即仿陜西弓箭手制,募邊民習知溪洞險易者,置所司教以戰陣,勸以耕牧,得勝兵幾萬人以鎮撫之。遷特進,拜少保,封莘國公,恩典如宰相。宣和元年,薨,年六十五,贈
 太傅,謚曰文簡。



 鄧氏自綰以來,世濟其奸,而洵武阿二蔡尤力。京之敗亂天下,禍源自洵武起焉。



 李定,字資深,揚州人。少受學於王安石。登進士第,為定遠尉、秀州判官。熙寧二年,孫覺薦之,召至京師,謁諫官李常,常問曰:「君從南方來,民謂青苗法何如?」定曰:「民便之,無不喜者。」常曰:「舉朝方共爭是事,君勿為此言。」定即往白安石,且曰:「定但知據實以言,不知京師乃不許。」安石大喜,謂曰:「君且得見,盍為上道之。」立薦對。神宗問青
 苗事,其對如曩言,於是諸言新法不便者,帝皆不聽。命定知諫院,宰相言前無選人除諫官之比,遂拜太子中允、監察御史裏行。知制誥宋敏求、蘇頌、李大臨封還制書,皆罷去。



 御史陳薦疏:「定頃為涇縣主簿,聞庶母仇氏死,匿不為服。」詔下江東、淮、浙轉運使問狀,奏云:「定嘗以父年老,求歸侍養,不云持所生母服。」定自辯言,實不知為仇所生,故疑不敢服,而以侍養解官。曾公亮謂定當追行服,安石力主之,改為崇政殿說書。御史林旦、薛昌
 朝言,不宜以不孝之人居勸講之地,並論安石,章六七上,安石又白罷兩人,定亦不自安,蘄解職,以集賢校理、檢正中書吏房、直舍人院同判太常寺。八年,加集賢殿修撰、知明州。



 元豐初,召拜寶文閣待制、同知諫院,進知制誥,為御史中丞。劾蘇軾《湖州謝上表》,擿其語以為侮慢。因論軾自熙寧以來,作為文章,怨謗君父,交通戚里。逮赴臺獄窮治,當會赦,論不已,竄之黃州。方定自鞫軾獄,勢不可回。一日,於崇政殿門外語同列曰:「蘇軾乃奇
 才也。」俱不敢對。



 請復六案糾察之職,並諸路監司皆得鉤考,從之。彗出東方,求直言,太史謂有兵變,帝命宦者視衛士飲食。定言一飯不足市恩,適起小人之心,乃止。或議廢明堂祀,帝以訪定。定曰:「三歲一郊或明堂,祖宗以來,未之有改。誰為此言,願治其妄。」帝曰:「聽卿言足矣。」遷翰林學士。坐論府界養馬事失實,罷知河陽,留守南京,召為戶部侍郎。哲宗立,以龍圖閣學士知青州,移江寧府。言者爭暴其前過,又謫居滁州。元祐二年,卒。



 定於
 宗族有恩,分財振贍,家無餘貲。得任子,先及兄息。死之日,諸子皆布衣。徒以附王安石驟得美官,又陷蘇軾於罪,是以公論惡之,而不孝之名遂著。



 舒但,字信道,明州慈溪人。試禮部第一,調臨海尉。民使酒詈逐後母,至但前,命執之,不服,即自起斬之,投劾去。王安石當國,聞而異之,御史張商英亦稱其材,用為審官院主簿。使熙河括田,有績,遷奉禮郎。鄭俠既貶,復被逮,但承命往捕,遇諸陳。搜俠篋,得所錄名臣諫草,有言
 新法事及親朋書尺,悉按姓名治之,竄俠嶺南,馮京、王安國諸人皆得罪。擢但太子中允、提舉兩浙常平。



 元豐初,權監察御史裏行。太學官受賂,事聞,但奉詔驗治,凡辭語微及者,輒株連考竟,以多為功。加集賢校理。同李定劾蘇軾作為歌詩議訕時事。但又言:「王詵輩公為朋比,如盛僑、周邠固不足論,若司馬光、張方平、範鎮、陳襄、劉摯,皆略能誦說先王之言,而所懷如此,可置而不誅乎?」帝覺其言為過,但貶軾、詵,而光等罰金。



 未幾,同修起
 居注,改知諫院。張商英為中書檢正,遺但手帖,示以子婿所為文。但具以白,云商英為宰屬而干請言路,坐責監江陵稅。始,但以商英薦得用;及是,反陷之。進知雜御史、判司農寺,超拜給事中、權直學士院。逾月,為御史中丞。舉劾多私,氣焰熏灼,見者側目,獨憚王安禮。



 但在翰林,受廚錢越法,三省以聞,事下大理。初,但言尚書省凡奏鈔法當置籍,錄其事目。今違法不錄,既案奏,乃謾以發放歷為錄目之籍,但以為大臣欺罔。而尚書省取臺
 中受事籍驗之,亦無錄目,但遽雜他文書送省,於是執政復發其欺。大理鞫廚錢事,謂但為誤。法官吳外厚駁之,御史楊畏言但所受文籍具在,無不承之理。帝曰:「但自盜為贓,情輕而法重;詐為錄目,情重而法輕。身為執法,而詐妄若是,安可置也!」命追兩秩勒停。但比歲起獄,好以疑似排抵士大夫,雖坐微罪廢斥,然遠近稱快。十餘年,始復通直郎。



 崇寧初,知南康軍。辰溪蠻叛,蔡京使知荊南,以開邊功,由直龍圖閣進待制,明年,卒,贈直學
 士。



 蹇周輔,字磻翁,成都雙流人。少與範鎮、何郯為布衣交。年未冠,試大廷,不第。鎮、郯既貴達,周輔始特奏名,再舉進士,知宜賓、石門二縣,通判安肅軍,為御史臺推直官。善於訊鞫,鉤索微隱,皆用智得情。嘗有詔獄,事連掖庭掌寶侍史,它司累月不能決,乃命周輔。度不可追逮,奏請以要辭示主者詰服之,時以為知體。及治李逢獄竟,臺臣雜治無異辭,神宗稱其能,擢開封府推官,出為淮
 南轉運副使。盜廖恩聚黨閩中,多害兵吏,改使福建,護諸將以討之,恩遂降。



 元豐初,循唐制,歸百司獄於大理寺,選為少卿,遷三司度支副使。先是,湖南例食淮鹽,周輔始請運廣鹽數百萬石,分鬻郴、全、道州;又以淮鹽增配潭、衡諸郡,湘中民愁困,法既行,遂領於度支。以集賢殿修撰為河北都轉運使,進寶文閣待制,召為戶部侍郎、知開封府,事多不決。授中書舍人,不拜,改刑部侍郎。元祐初,言者暴其立江西、福建鹽法,掊克欺誕,負公擾
 民,罷知和州。徙廬州。卒,年六十六。



 周輔強學,善屬文,神宗嘗命作《答高麗書》,屢稱善。為吏深文刻核,故老而獲戾。子序辰。



 序辰字授之,登第後數年,以泗州推官主管廣西常平。周輔方使閩,上言父子並祗命遠方,家無所托,蘄改一近地。乃易京西,旋提舉江西常平,繼父行鹽法。為監察御史,遷殿中侍御史、右司諫。哲宗立,改司封員外郎。周輔得罪,以序辰成其惡,降簽書廬州判官。起知楚州,提
 點江東刑獄。



 紹聖中,遷左司員外郎,進起居郎、中書舍人、同修國史。疏言:「朝廷前日正司馬光等奸惡,明其罪罰,以告中外。惟變亂典刑,改廢法度,訕讟宗廟,睥睨兩宮,觀事考言,實狀彰著,然蹤跡深秘,包藏禍心,相去八年之間,蓋已不可究質。其章疏案牘,散在有司,若不匯緝而藏之,歲久必致淪棄。願悉討奸臣所言行,選官編類,入為一帙,置之一府,以示天下後世大戒。」遂命序辰及徐鐸編類。由是縉紳之禍,無一得脫者。遷禮部尚
 書,與安惇看詳訴理事。以奉使遼國無狀,黜知黃州。閱四月,除龍圖閣待制、知揚州。



 徽宗立,中書言序辰類元祐章牘,傅致語言,指為謗訕。詔與惇並除名勒停,放歸田里。蔡京為相,復拜刑部、禮部侍郎,為翰林學士,進承旨。有言其在先帝遏密中以音樂自娛者,黜知汝州。二年,徙蘇州。坐縱部民盜鑄錢,謫單州團練副使、江州安置。又坐守蘇時以天寧節同其父忌日,輒於前一日設宴,及節日不張樂,移永州。會赦,復官中奉大夫,遂卒。序
 辰亦有文,善傅會,深文刻核,似其父云。



 徐鐸字振文,興化莆田人。熙寧進士第一,簽書鎮東軍判官,紹聖末,以給事中直學士院。蹇序辰建議編類元祐諸臣章牘事狀,詔鐸同主之。凡一時施行文書,捃拾附著,纖悉不遺。遷禮部侍郎。鐸雖雲封駁,而是時凡給事中不肯書讀者,輒命代行之。貢院獲舉人挾書,開封尹蔣之奇將以徒定罪,鐸爭不可,之奇為從輕比。既上省,章惇怒,罰府吏,舉人竟坐刑,鐸不復敢有言,眾傳以
 為笑。後議除御史中丞,或摭此事以為無所執持,乃止。



 徽宗立,以龍圖閣待制知青州。御史中丞豐稷論鐸編類事狀,率視章惇好惡為輕重,存歿名臣,橫罹竄斥,序辰既放歸田里,鐸之罪不在其下。詔落職知湖州。崇寧中,拜禮部尚書。方議廟制,鐸請增為九室。議者疑已祧之主不可復祔、鐸言:「唐之獻祖、中宗、代宗與本朝之僖祖,皆嘗祧而復,今宜存宣祖於當祧,復翼祖於已祧,禮無不稱。」從之。進吏部尚書,卒。



 論曰:「士學不為己,而俯仰隨時,如挈皋居井上,求其立朝不撓,不可得已。常秩在嘉祐、治平時,三辭羔雁之聘,若能隱居以求其志者。及王安石用事,一召即至,容容歷年,曾無一嘉謨,而竊顯位。至定之黨附,但之兇德,宜為世所指名。綰及周輔二家,父子並同惡相濟,而序辰與鐸編類事狀,流毒元祐名臣,忠義之士,為之一空,馴致靖康之禍,可勝嘆哉。



 王廣淵,字才叔,大名成安人。慶歷中,上曾祖《明家集》,詔
 官其後,廣淵推與弟廣廉,而以進士為大理法直官、編排中書文字。裁定祖宗御書十卷,仁宗喜之,以知舒州,留不行。



 英宗居藩邸,廣淵因見暱,獻所為文,及即位,除直集賢院。諫官司馬光言:「漢衛綰不從太子飲,故景帝待之厚。周張美私以公錢給世宗,故世宗薄之。廣淵交結奔競,世無與比,當仁宗之世,私自托於陛下,豈忠臣哉?今當治其罪,而更賞之,何以厲人臣之節?」帝不聽,用為群牧、三司戶部判官,從容謂曰:「朕於《洪範》得高明沉
 潛之義,剛內以自強,柔外以應物,人君之體,無出於是。卿為朕書之於欽明殿屏,以備觀省,非特開元《無逸圖》也。」加直龍圖閣。帝有疾,中外憂疑,不能寢食,帝自為詔諭之曰:「朕疾少間矣。」廣淵宣言於眾。



 神宗立,言者劾其漏洩禁中語,出知齊州,改京東轉運使,得於內省傳達章奏。曾公亮、王安石持不可,乃止。廣淵以方春農事興而民苦乏,兼並之家得以乘急要利,乞留本道錢帛五十萬,貸之貧民,歲可獲息二十五萬,從之。其事與青苗
 錢法合,安石始以為可用,召至京師。御史中丞呂公著摭其舊惡,還故官。程顥、李常又論其抑配掊克,迎朝廷旨意以困百姓。會河北轉運使劉庠不散青苗錢奏適至,安石曰:「廣淵力主新法而遭劾,劉庠故壞新法而不問,舉事如此,安得人無向背?」故顥與常言不行。徙使河東,擢寶文閣待制、知慶州。



 宣撫使興師入夏境,檄慶會兵。方授甲,卒長吳逵以眾亂,廣淵亟召五營兵御之。逵率二千人斬關出,廣淵遣部將姚兕、林廣追擊,降其眾。
 柔遠三都戍卒欲應賊,不果,廣淵陽勞之,使還戍,潛遣兵間道邀襲,盡戮之。猶以盜發所部,削兩秩。二年,進龍圖閣直學士、知渭州。



 廣淵小有才而善附會,所闢置類非其人。帝謂執政曰:「廣淵奏闢將佐,非貴游子弟,即胥史輩,至於濮宮書吏亦預選,蓋其人與時君卿善。一路官吏不少,置而不取,乃用此輩,豈不誤朝廷事?已下詔切責,卿等宜貽書申戒之。」卒,年六十,贈右諫議大夫。元豐初,詔以其被遇先帝之故,弟臨自皇城使擢為兵部
 郎中、直昭文館,子得君賜進士出身。



 臨字大觀,亦起進士,簽書雄州判官。嘉祐初,契丹泛使至,朝論疑所應,臨言:「契丹方饑困,何能為?然《春秋》許與之義,不可以不謹。彼嘗求馴象,可拒而不拒;嘗求樂章,可與而不與,兩失之矣。今橫使之來,或謂其求聖像,聖像果可與哉?」朝廷善其議。治平中,詔求武略,用近臣薦,自屯田員外郎換崇儀使、知順安軍,改河北沿邊安撫都監。上備御數十策,大略皆自治而已。



 契丹刺兩輸
 人為義軍,來歸者數萬。或請遣還,臨曰:「彼歸我而遣之,必為亂,不如因而撫之。」詔從其請,自是來者益多,契丹悔失計。進安撫副使,歷知涇、鄜州、廣信、安肅軍。



 召對,還文階,知齊州、滄州、荊南,入為戶部副使,以寶文閣待制知廣州府、河中,卒。



 王陶,字樂道,京兆萬年人。第進士,至太常丞而丁父憂。陶以登朝在郊祀後,恩不及親,乞還所遷官,丐追贈。詔特聽之,仍俟服闋,除太子中允。



 嘉祐初,為監察御史裏
 行。衛卒入延福宮為盜,有司引疏決恩降其罪。陶曰:「禁省之嚴,不應用外間會降為比。」於是流諸海島,主者皆論罰。中貴人導煉丹者入禁廷,陶言:「漢、唐方士,名為化黃金、益年壽以惑人主者,後皆就戮。請出之。」陳升之為樞密副使,論其不當,升之去,陶亦知衛州,改蔡州。明年,復以右正言召。陶言:「臣與四人同補郡,今獨兩人召,請並還唐介、呂誨等。」



 英宗知宗正寺,逾年不就職。陶上疏曰:「自至和中聖躬違豫之後,天下顒顒,無所寄命,交章
 抗疏,請早擇宗室親賢,以建儲嗣,危言切語,動天感人。夫為是議者,豈皆懷不忠孝、為奸利附托之人哉?發於至誠,念宗廟社稷無窮大計而已。陛下順民欲而安人心,故親發德音,銳為此舉,中外搖搖之心,一旦定矣。厥後浸潤稽緩,豈免憂疑?流言或云事由嬪御、宦侍姑息之語,聖意因而惑焉。婦人近幸,詎識遠圖?臣恐海內民庶,謂陛下始者順天意民心命之,今者聽左右姑息之言而疑之,使遠近奸邪得以窺間伺隙,可不惜哉!」因請
 對,仁宗曰:「今當別與一名目。」既而韓琦決策,遂立為皇子。英宗即位,加直史館、修起居注、皇子位伴讀、淮陽穎王府詡善、知制誥,進龍圖閣學士、知永興軍,召為太子詹事。



 神宗立,遷樞密直學士,拜御史中丞。郭逵以簽書樞密宣撫陜西,詔令還都。陶言:「韓琦置逵二府,至用太祖故事,出師劫制人主,琦必有奸言惑亂聖德。願罷逵為渭州。」帝曰:「逵先帝所用,今無罪黜之,是章先帝用人之失也,不可。」陶既不得逞,遂以琦不押文德常朝班奏
 劾之。陶始受知於琦,驟加獎拔。帝初臨御,頗不悅執政之專,陶料必易置大臣,欲自規重位,故視琦如仇,力攻之,琦閉門待罪。帝徙陶為翰林學士,旋出知陳州,入權三司使。呂公著言其反復不可近,又以侍讀學士知蔡州,歷河南府、許、汝、陳三州,以東宮舊臣加觀文殿學士。帝終薄其為人,不復用。元豐三年,卒,年六十一,贈吏部尚書,謚曰文恪。



 陶微時苦貧,寓京師教小學。其友姜愚氣豪樂施,一日大雪,念陶奉母寒餒,荷一鍤鏟雪,行二
 十里訪之。陶母子凍坐,日高無炊煙。愚亟出解所衣錦裘,質錢買酒肉、薪炭,與附火飲食,又捐數百千為之娶。陶既貴,尹洛,愚老而喪明,自衛州新鄉往謁之,意陶必念舊哀己。陶對之邈然,但出尊酒而已。愚大失望,歸而病死。聞者益薄陶之為人。



 王子韶,字聖美,太原人。中進士第,以年未冠守選,復游太學,久之乃得調。王安石引入條例司,擢監察御史裏行,出按明州苗振獄。安石惡祖無擇,子韶迎其意,發無
 擇在杭州時事,自京師逮對,而以振獄付張載,無擇遂廢。中丞呂公著等論新法,一臺盡罷。子韶出知上元縣,遷湖南轉運判官。御史張商英劾其不葬父母,貶知高郵縣。由司農丞提舉兩浙常平。入對,神宗與論字學,留為資善堂修定《說文》官。官制行,為禮部員外郎,以入省後期,改庫部。



 元祐中,歷吏部郎中、衛尉少卿,遷太常諫官。劉安世言:「熙寧初,士大夫有『十鉆』之目,子韶為『衙內鉆』,指其交結要人子弟,如刀鉆之利。又陷祖無擇於深
 文,搢紳所共鄙薄,豈宜污禮樂之地!」改衛尉卿。安世復言:「七寺正卿班少常上,因彈擊而獲超遷,是啟僥幸也。」乃出知滄州。入為秘書少監,迎伴遼使,御下苛刻,軍吏因被酒刃傷子韶及其子。又出知濟州,建言乞追復先烈以貽後法,復以太常少卿召,進秘書監,拜集賢殿修撰、知明州,卒。崇寧二年,子相錄元祐中所上疏稿聞於朝,詔贈顯謨閣待制。



 何正臣,字君表,臨江新淦人。九歲舉童子,賜出身,復中
 進士第。元豐中,用蔡確薦,為御史裏行。遂與李定、舒但論蘇軾,得五品服,領三班院。會正御史專六察,正臣言:「幸得備言路,以激濁揚清為職,不宜兼治它曹。」神宗善之,為悉罷御史兼局,而正臣解三班,加直集賢院,擢侍御史知雜事。



 韓存寶討瀘夷無功,命治其獄,被以逗撓罪誅之。還,除寶文閣待制、知審官東院,尚書省建為吏部侍郎。逾年,嫚於奉職,銓擬多抵牾。事聞,以制法未善為解。王安禮曰:「法未善,有司所當請,豈得歸罪於法?」乃
 出知潭州。時詔州縣聽民以家貲易鹽,吏或推行失指。正臣條上其害,謂無益於民,亦不足以佐國用,遂寢之,民以為便。後歷刑部侍郎、知宣州,卒。



 陳繹,字和叔,開封人。中進士第,為館閣校勘、集賢校理,刊定《前漢書》,居母喪,詔即家讎校。英宗臨政淵嘿,繹獻五箴,曰主斷、明微、廣度、省變、稽古。同判刑部,獄訟有情法相忤者,讞之。或言刑曹唯知正是否,不當有所輕重。繹曰:「持法者貴審允,心知失刑,惡得坐視?」由是多所平
 反。帝稱其文學,以為實錄檢討官。



 神宗立,為陜西轉運副使,入直舍人院、修起居注、知制誥,拜翰林學士,以侍講學士知鄧州。繹不能肅閨門,子與婦一夕俱殞於卒伍之手,傲然無慚色。召知通進、銀臺司,帝語輔臣曰:「繹論事不避權貴。」命權開封府。時獄有小疑,輒從中覆;至繹,特聽便宜處決。久之。還翰林,仍領府。治司農吏盜庫錢獄未竟,中書檢正張諤判寺事,懼失察,以帖詰稽留,繹遣吏示以成牘。言者論其徇宰屬、縱有罪,出知滁州。
 郊祀恩,復知制誥,言者再論之,得秘書監、集賢院學士。



 元豐初,知廣州。庫有檀香佛像,繹以木易之。事覺,有司當為官物有剩利。帝曰:「是以事佛麗重典矣。」時繹已加龍圖閣待制、知江寧府,乃貶建昌軍,奪其職。後復太中大夫以卒,年六十八。



 繹為政務摧豪黨,而行與貌違,暮年繆為敦樸之狀,好事者目為「熱熟顏回」。



 論曰:王廣淵在仁宗時,因近暱獻文於英宗潛邸,固已有竊取功名之心,蓋為臣之不忠者,雖列侍從,烏足道
 哉!王陶始為韓琦所知,在御史時,頗能譏切時政。及為中丞,則承望風旨,攻琦如仇讎,欲自取重位。其忘姜愚布衣之義,又不足責矣。王子韶之陷祖無擇,何正臣之論蘇軾,皆小人之盜名。陳繹希合用事,固無足道,然於獄事多所平反,惜乎閨門不肅,廉恥並喪,雖明曉吏事,亦何取焉。



\end{pinyinscope}