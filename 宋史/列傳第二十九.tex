\article{列傳第二十九}

\begin{pinyinscope}

 顏衎劇可久趙逢蘇曉高防馮瓚邊珝王明許仲宣楊克讓段思恭侯陟李符魏丕董樞



 顏衎,字祖德,兗州曲阜人。自言兗國公四十五世孫。少
 苦學,治《左氏春秋》。梁龍德中擢第,解褐授北海主簿,以治行聞。再調臨濟令。臨濟多淫祠,有針姑廟者,里人奉之尤篤。衎至,即焚其廟。



 後唐天成中,為鄒平令。符習初鎮天平,習,武臣之廉慎者,以書告屬邑毋聚斂為獻賀。衎未領書,以故規行之,尋為吏所訟。習遽召衎笞之,幕客軍吏咸以為辱及正人,習甚悔焉,即表為觀察推官,且塞前事。長興初,召拜太常博士,習力奏留之。習致仕,衎東歸養親。



 未幾,房知溫鎮青州,復闢置幕下。知溫險
 愎,厚斂多不法,衎每極言之,不避其患。晉祖入洛,知溫恃兵力偃蹇,衎勸其入貢。知溫以善終,衎之力也。知溫諸子不慧,衎勸令以家財十萬餘上進。晉祖嘉之,歸功於衎。知溫子彥儒授沂州刺史,衎拜殿中侍御史。



 俄遷都官員外郎,充東都留守判官,改河陽三城節度副使、檢校左庶子,知州事。居半歲,得家問,父在青州有風痺疾,衎不奏棄官去侍疾,不復有仕宦意。歲餘,父疾不能起,衎親自掬矢,未嘗少倦。晉祖聞之,召為工部郎中、樞
 密直學士,連使促召至闕,辭曰:「臣無他才術,未知何人誤有聞達。望放臣還,遂其私養。」晉祖曰:「朕自知卿,非他人薦也。」俄廢樞密院,以本官奉朝請。逾年,上表請還侍養,授青州行營司馬。丁父憂,哀毀甚。俄召為駕部郎中、鹽鐵判官。以母老懇辭,有詔止守本官。



 未幾,復出為天平軍節度副使。開運末,授左諫議大夫,權判河南府,召拜御史中丞。喪亂之後,朝綱不振,衎執憲頗有風採。嘗上言:「才除御史者,旋授外藩賓佐,復有以私故細事求
 假外拜,州郡無參謁之儀,出入失風憲之體,漸恐四方得以輕易,百闢無所準繩。請自今藩鎮幕僚,勿得任臺官;雖親王、宰相出鎮,亦不得奏充賓佐。非奉制勘事,勿得出京,自餘不令厘雜務。」詔惟闢召入幕如故,餘從其請。復抗表求侍養,改戶部侍郎。衎又堅乞罷免,詔書褒許,即與其母東歸。



 漢乾祐末,丁憂。服除,詔鄆州高行周津遣赴闕,衎辭以足疾,不至。周廣順初,起為尚書右丞,俄充端明殿學士。太祖征兗州,駐城下,遣衎往曲阜祠
 文宣王廟。城平,以衎權知州事。歸朝,權知開封。



 時王峻持權,衎與陳觀俱為峻所引用。會峻敗,觀左遷,衎罷職,守兵部侍郎。顯德初,上表求解官,授工部尚書,致仕還鄉里,臺閣縉紳祖餞都門外,冠蓋相望,時人榮之。建隆三年春,卒於家,年七十四。



 衎守章句,無文藻,然諒直孝悌,為時所推。



 劇可久,字尚賢,涿州範陽人。沉毅方正,明律令。與馮道、趙鳳為友。後唐同光初,鳳薦於朝,補徐州司法,以干職
 聞。召為大理評事,賜緋。逾年,遷大理正,坐誤治獄責授登州司戶。遇赦,召為著作郎。仕晉,歷殿中少監、太子右諭德、大理少卿,賜金紫。晉祖崩,可久方在病告,有司糾以不赴國哀,坐免。未幾復官,遷大理卿。



 周廣順初,改太僕卿,復為大理卿。會鄭州民李思美妻詣御史臺訴夫私鬻鹽,罪不至死,判官楊瑛置以大闢。有司攝治瑛,瑛具伏。可久斷瑛失入,減三等,徒二年半。宰相王峻欲殺瑛,召可久謂之曰:「死者不可復生,瑛枉殺人,其可恕耶?」
 可久執議益堅,瑛得免死。由是忤峻,改太僕卿,分司西京。顯德三年,所舉官犯臟,可久坐停任。明年,復起為右庶子。



 世宗以刑書深古、條目繁細,難於檢討。又前後敕格重互,亦難詳審,於是中書門下奏曰:「伏以刑法者,御人之銜勒,救弊之斧斤,有國家者不可一日而廢也。雖堯、舜之世,亦不能舍此而致治。今奉制旨,刪定律令,有以見明罰敕法之意也。竊以朝廷之所用者,《律》十二卷《律疏》三十卷、《式》二十卷、《令》三十卷《開成格》一十卷《大中
 統類》一十二卷,後唐以來至漢末編敕三十二卷,及國朝制敕等。律令則文辭古質,或難以詳明,格敕則條目繁多,或有所疑誤。將救舞文之弊,宜伸畫一之規。所冀民不陷刑,吏有所守。臣等商議,望準制旨施行。仍命侍御史知雜事張湜、太子右庶子劇可久、殿中侍御史率汀、職方郎中鄧守中、倉部郎中王瑩、司封員外郎賈玭、太常博士趙礪、國子博士李光贊、大理正蘇曉、太子中允王伸等十人編集新格,勒成部秩。律令之有難解者,
 就文訓釋;格敕之有繁雜者,隨事刪削;其有矛盾相違、輕重失宜者,盡從改正,無或拘牽。候畢日,委御史臺、尚書省四品以上及兩省五品以上官參詳可否,送中書門下議定。」從之。自是湜等於都省集議刪定,仍令大官供膳。五年,書成,凡三十卷,目曰《刑統》。宰相請頒天下,與律、疏、令、式並行。可久復拜大理卿。建隆三年,告老,改光祿卿致仕。卒,年七十七。



 可久在廷尉四十年,用法平允,以仁恕稱。



 趙逢,字常夫,媯州懷戎人。性剛直,有吏乾。父崇事劉守光為牙校。後唐天祐中,莊宗遣周德威平幽州,因誅崇。逢尚幼,德威錄為部曲,令與諸子同就學。及德威戰沒胡柳陂,逢乃游學河朔間。久之西游,客鳳翔李從□門下。從□卒,侯益領節制,逢又依之。漢乾祐中,益入為開封尹,表逢為巡官,逢不樂,乃求舉進士。是歲,禮部侍郎、集賢殿學士司徒翊典貢舉,擢登甲科。解褐授秘書郎、直史館。周廣順中,歷左拾遺、右補闕,皆兼史職。世宗嗣
 位,遷禮部員外郎、史館修撰。顯德四年,改膳部員外郎、知制誥。逾年,轉水部郎中,仍掌誥命,恭帝即位,賜金紫。



 宋初,拜中書舍人。太祖征澤、潞,逢從行。次河內,聞李筠擁兵入寇,又慮太行艱險,乃妄言墜馬傷足,留於懷州。駕還京,有密旨除拜,逢當草制,又稱疾不入。太祖謂宰相曰:「此人得非規避行役者耶?」對曰:「誠如聖言。」遂貶房州司戶。會恩,量移汝州司馬。



 乾德初,召赴闕,授都官郎中、知制誥,充史館修撰、判館事。二年,改判昭文館。未幾,
 充樞密直學士,加左諫議大夫。蜀平,出知閬州。時部內盜賊攻州城,逢防禦有功。賊既平,誅滅者僅千家。妻朱氏病死京師,詔給葬事。代還,遷給事中,充職。六年,權知貢舉。



 太祖征太原,以逢為隨軍轉運使,鑄印賜之。會發諸道丁壯數十萬,築堤壅汾水灌晉陽城。逢白太祖乞效用,即命督其版築。時方盛暑,逢於烈日中親課力役,因而遘疾,輿歸京師。開寶八年,卒。



 逢揚歷清近,所至有聲,然傷慘酷,又言多詆訐,故縉紳目之為「鐵橛」。大中祥
 符三年,特詔錄其子極為三班借職。



 蘇曉,字表東,京兆武功人。父瓚,仕後唐,歷秘書少監。長興初,曉闢鄧州從事。漢祖鎮太原,表為觀察支使。周廣順初,由華州支使入為大理正。以讞獄有功,遷少卿。顯德中,歷屯田郎中。



 宋初,詔與竇儀、奚嶼、張希讓等同詳定《刑統》為三十卷及《編敕》四卷。建隆四年,權大理少卿事,遷度支郎中。乾德三年,出為淮南轉運使,建議榷蘄、黃、舒、廬、壽五州茶,置十四場,規其利,歲入百餘萬緡。開
 寶三年,遷司勛郎中,改西川轉運使,仍掌京城市征。



 先是,朝廷遣供備庫使李守信市木秦、隴間,守信盜官錢巨萬,既受代,為部下所發,守信至中牟,自剄於傳舍。太祖命曉案之,逮捕甚眾。右拾遺、通判秦州馬適妻李,即守信息女。守信嘗用木為筏以遺適,曉得守信所送書以進,太祖將舍之,曉上章固請置於法,仍籍其家。餘所連及者,多至破產,盡得所隱沒官錢。擢拜曉右諫議大夫、判大理寺,賜金紫,遷左諫議大夫。七年,監在京商稅。
 九年六月,卒,年七十三。



 曉深文少恩,當時號為酷吏。及卒,無子,有一女甚鐘愛,亦先曉卒,人以為深刻所致。



 高防,字修己,並州壽陽人。性沉厚,守禮法。累世將家。父從慶,戍天井關,與梁軍戰死。防年十六,護柩以歸。事母孝,好學,善為詩。初,張從恩為北京副留守,奏攝太原府倉曹掾。從恩移澶州防禦使,表為判官。有親校段洪進盜官木造器,市取其直。從恩聞之怒,將殺之。洪進懼,思緩其罪,紿曰:「判官使為之。」從恩召防詰之,防即引伏,洪
 進得免。從恩遺防錢十千、馬一匹遣之。防拜受而去,終不自明。既而悔之,命騎追及,防不得已而還,賓主如初。又居帳下歲餘,稍稍有言防自誣以活人,從恩益加禮重。從恩入為樞密副使,防授國子監丞。從恩留守西洛,又為推官。召拜殿中丞,充鹽鐵推官。以母憂去官,服除,隨從恩歷鄆、晉、潞三鎮判官。契丹入汴,晉主北行。從恩欲歸款契丹,召拜計議,防為陳逆順,請固守臣節。為左右所搖,從恩不用其言,遂歸契丹。既行,命副使趙行遷
 知留後,從恩所親王守恩為巡檢,與防同領郡事。防與守恩謀誅行遷,以城歸漢祖。漢祖召防赴太原,加檢校金部郎中。



 乾祐初,授屯田員外郎,改浚儀令。時楊邠用事,與防有隙,未幾,免職。居數月,夢一吏以白帕裹印,自門入授防,防寤而思曰:「白主刑,吾當為主刑官乎?」俄而周祖即位,起為刑部員外郎,吏繼印至,一如夢中所睹。改開封令,遷本府少尹,除刑部郎中。宿州民以刃殺妻,妻族受賂,偽言風狂病喑。吏引律不加考掠,具獄上請
 覆。防云:「其人風不能言,無醫驗狀,以何為證?且禁系逾旬,亦當須索飲食。願再劾,必得其情。」周祖然之,卒置於法。



 世宗尹京,判官崔頌忤旨,簡求僚佐,宰相首以防薦。周祖曰:「朕方欲用之。」乃以防代頌。世宗即位,拜左諫議大夫,賜金紫、鞍勒馬。顯德二年,遷給事中。從征淮南,初下泰州,即命防權知州事兼判海陵監事。會吳師至,乃遷州民入牙城,分兵固守,以俟外援。俄而揚帥韓令坤馳騎召防,吳軍復至廣陵,防與令坤敗之。詔書嘉獎。三
 年,改左散騎常侍。其秋,召歸闕。復歷知蔡、宋二州。再從世宗南征,判行泗州,及城降,命防知州事,復知蔡州。五年,遷戶部侍郎。世宗謀取蜀,以防為西南面水陸轉運制置使,屢發芻糧赴鳳州,為征討之備。



 太祖還自陳橋,防所居為里民所略,詔賜綾絹、衣服、衾裯、鞍馬。及徵李筠,防又為潞州東北路計度轉運使。澤、潞平,拜尚書左丞,賜銀器、彩帛、鞍勒馬。



 建隆二年,出知秦州,州與夏人雜處,罔知教養,防齊之以刑,舊俗稍革。州西北夕陽鎮,
 連山谷多大木,夏人利之。防議建採造務,闢地數百里,築堡要地。自渭而北,夏人有之;自渭而南,秦州有之。募卒三百,歲獲木萬章。夏部尚波千等率諸族千餘人,涉渭奪木筏,殺役兵。防出與戰,俘四十七人以獻。太祖慮擾邊郡,詔諭酋帥,賜所獲之俘錦袍、銀帶以遣之,遂罷採木之役,命吳廷祚為節度以代防。歸為樞密直學士,復出知鳳翔。乾德元年卒,年五十九。太祖甚悼惜,賜其子太府寺丞延緒詔曰:「爾父有幹蠱之才,懷匪躬之節,
 朕所毗倚。遽茲淪亡,聞之鋋傷,不能自己。矧素尚清白,諒無餘資,殯殮所須,特宜優恤。今遣供奉官陳彥珣部署歸葬西洛,凡所費用,並從官給。」



 馮瓚,字禮臣,齊州歷城人。性便佞,任數,務巧進。父知兆,後唐司農卿。瓚以蔭補,解褐授秘書省校書郎,遷著作佐郎,出為諸城令。歲滿,授太子右贊善大夫。漢初,改監察御史。周廣順元年,遷殿中侍御史。河陽判官宋仁範與洛陽嫠婦交訟,詔瓚劾之。獄成,大理斷以官當徒,追
 兩官告身,刑部員外郎張處素復核無異,奏行。仁範詣闕訴其事,詔還一官,瓚洎處素俱坐降一階。顯德初,遷刑部員外郎,充三司判官。歲餘,改祠部郎中,充集賢院直學士。



 宋初,轉兵部郎中,加金紫階。瓚風神俊爽,善談論,有吏材,太祖甚寵之,擢拜左諫議大夫,出知舒州。境內有菰蒲魚鱉之饒,居民採以自給,防禦使司超盡征之,瓚奏奪民利,請蠲除,從之。建隆四年春,徙知廬州。乾德三年,以本官充樞密直學士。



 時劍外初平,卒有亡命
 者散匿為盜,命瓚知梓州。無何,蜀軍校上官進率亡命三千餘人,掠民數萬,夜攻州城。瓚曰:「賊乘夜奄至,此烏合之眾,以棰梃相擊,必無固志。正可持重以鎮之,旦自潰矣。」城中止有雲騎兵三百,令分守城門。瓚坐城樓,密令促其更籌,未夜分擊五鼓,賊悉遁去。因縱兵追之,擒上官進,斬於市。誘其餘黨千餘人,並釋其罪,境內獲安。



 初,太祖欲任用瓚,常與趙普言瓚有奇材。普忌之,乃遣詣蜀平冠,潛令所親信從其行,密察其過,即亡入京師
 擊登聞鼓,訟瓚及監軍綾錦副使李美、通判殿中侍御史李楫受賕為奸事。急召歸闕,親問之,詞理屢屈,乃屬吏。既而普遣人至潼關,閱其囊裝,得金帶珍玩之物,皆封題將以賂劉蘗,蘗方在太宗幕府。瓚具伏,普言法當死,太祖欲貸之,普固執不可,乃削去名籍。瓚流登州沙門島,美配隸通州海門島,蘗免所居官。李楫者,嘗與王德裔佐王饒幕,太祖納孝明皇后,因識之。德裔輕率而楫謹厚,太祖薄德裔而厚楫,至是,楫特免配流。未幾,復
 為御史。



 瓚在海上凡十年不得召,開寶末,遇赦放還。太宗即位,授左贊善大夫。太平興國元年冬,與禮部員外郎賈黃中、左補闕程能分掌左藏三庫。先是,貨泉與金帛通。至是,以帑藏充溢,乃命分之。二年,復賜金紫。明年,判大理寺,改度支判官,遷秘書少監,充職。四年,上親征太原,以瓚為隨駕三司判官。凱旋,改大理卿兼判秘書省。以足疾求解,優詔免朝請,令於本司視事。瓚抗章請退,除給事中致仕,復舊勛階。五年,卒,年六十七。子克忠,
 至內殿崇班、閣門祗候。



 邊珝,字待價,華州鄭人也。曾祖頡,石圌令。祖操,下邳令。父蔚,太常卿。珝,晉天福六年,舉進士,解褐秘書省校書郎、直洪文館。漢乾祐初,為右拾遺,加朝散大夫。澤州饑,奉詔視民田。周廣順元年,遷右補闕。三年,轉起居舍人。顯德二年,改庫部員外郎。丁外艱,服闋,授職方員外郎,知通州。珝課鬻鹽於狼山,歲增萬餘石。



 宋初,詣衛州視秋稼及掌京倉。建隆二年,兄玕自河南令入為吏部員
 外郎,復以珝為洛陽令。兄弟迭尹赤邑,時人榮之。乾德初,召為倉部郎中。蜀平,命珝知三泉縣。開寶初,遷職方郎中,監京兆曲務,又掌永安軍榷貨,奏徙務揚州。有富民訴廣陵尉謝圖殺其父,本部收尉囚之,官吏推劾累三百日,獄未具,州以狀聞。詔珝案鞫,盡得其實。乃富民以私憾誣告尉,即反坐之。就命權知州事,仍兼榷貨務。罷郡,又兼掌酒稅鹽礬務。未幾,丁母憂,起復,知州事。會征江表,兼領淮南轉運使。金陵平,知江北諸州轉運事。



 太宗即位,遷吏部郎中。召還,賜金紫,充廣南轉運使。初至,桂州守張頌卒。頌,濰州人,蒿葬城外。舊制不許以族行,僕人乃分匿其家財,珝召官吏悉追取之,部送其柩歸濰州。又屬郡守與護軍有忿隙者,但奏令易地,不致之於罪釁。太平興國五年,代歸。拜右諫議大夫,領吏部選事。七年,移知開封府。明年夏,卒,年六十三。



 珝精力有吏材,帝方欲倚用,及聞其卒,嘆惜數四,賻其家絹四百匹,錢二十萬。珝一子早卒,以其從子俊為尉氏主簿。兄
 玕至金部郎中,弟玢右贊善大夫,從子仿至殿中丞,倚為比部員外郎。



 王明,字如晦,大名成安人。晉天福中,舉進士不第。驍騎將藥元福為原州刺史,闢為從事。馮暉節制靈武,表為觀察巡官。周廣順初,元福領陳州防禦使,奏署判官。會劉崇寇晉州,命元福將兵援之,事多咨於明。



 先是,州縣吏部送丁壯餉糧,一夕,夫盡遁去。元福怒,盡驅官吏出軍門,將就戮。明馳往止之,入白元福曰:「今軍儲無闕,丁
 夫數萬人,文吏懦不能制,斬之何益,不如寬以待之。賊敗凱旋,公無專殺之名,不亦善乎?」元福感悟,盡免其死。既而崇眾宵遁,即命元福為建雄軍節度留鎮,因奏署明為書記,賜緋魚。



 顯德初,元福移鎮陜,恃功多驕恣,明以直道規之,忤其左右,多毀明於元福,元福亦稍疏之。明以父病求歸省,元福數召明,明因謝絕之。詣闕上書,求任州縣,歷清平、郾城二縣令。



 宋初,荊南高繼沖入覲,授彭門節鉞,以明為武寧軍節度掌書記。乾德初,召公
 卿近臣各舉清白有吏乾者一人,給事中馬士元以明塞詔,召為左拾遺。蜀平,選知榮州,代歸,遷右補闕。會用兵於嶺南,選為荊湖轉運使。開寶三年,大舉南征,以明為隨軍轉運使。山路險絕,舟車不通,但以丁壯數萬人轉遞,供億不闕。每下一郡一城,必先保其簿書,守其倉庫。既而賀州未下,明入與主帥計曰:「當急取之,恐援兵至,則我師勝負未可知。」諸將頗猶豫。明乃擐甲冑,率所部護送輜重卒百人,擁丁夫數千,畚鍤皆作,堙其塹,直
 抵城門。城中懼,開門納款,遂據有之。因抵廣州,賊眾十餘萬拒戰。是夕,大風發屋折木,眾乃驚懼。明與都部署潘美等謀,命丁夫數千人,人持二炬,間道先搗賊壘,大軍蓐食,陣以待之。俄而萬炬皆發,焚其柵。賊驚,果來犯,大軍因迎擊之,賊大敗,斬首數萬,劉鋹以城降。廣州平,為本道轉運使。太祖嘉其功,擢授秘書少監,領韶州刺史,充轉運使。俄以潘美、尹崇珂為嶺南轉運使,以明為副使。明遍歷部內,視民疾苦,舊無名科斂,悉條奏除之,
 嶺表遂安。



 七年,代歸,帝召見勞問,賜襲衣、金帶、鞍勒馬。是歲,將用師南唐,以明為黃州刺史,帝密授成算。明既視事,即完葺城壘,訓練士卒,眾莫解其意。俄而王師自荊渚乘戰艦而下,即以明為池州至岳州江路巡檢戰棹都部署。擊鄂州軍於江南,斬首三百級。又破萬餘人於武昌,殺江南軍七百人,拔樊山砦。破江州軍,斬首三千級。又破江南軍三百人於江中,獲船十餘艘。又擊敗湖口軍萬餘眾,奪戰艦五百艘。



 時南唐將朱令贇自上
 江領眾十五萬,連大艦沿流而下,將焚採石浮梁,抵金陵為援。明率所部舟師屯獨樹口,遣其子馳奏,請添造戰艦三百艘以襲令贇。帝曰:「非應急策也,令贇朝夕至,金陵之圍解矣。」乃密遣人諭明,令樹長木於洲浦間,若帆檣之狀。令贇望見之,果疑大軍襲其後,逗撓不敢進。明移檄諸軍,相為掎角,因督兵棹襲之。至小孤山,與諸軍合勢,大破之,擒令贇,眾赴水死者十五六。金陵平,詔明安撫諸郡,因命知洪州。太宗即位,兼領江南諸路轉
 運使。召為右諫議大夫,充三司副使。



 太平興國七年,與侯陟同判三司事。八年,召分三司,各命使領之,改左諫議大夫,為鹽鐵使,遷給事中。雍熙四年,改光州刺史,出知並州。端拱元年,代還。表求換秩,改禮部侍郎。會契丹擾邊,詔以明知真定府。契丹遁去。淳化初,詔歸闕,知京朝官差遣事。二年,卒,年七十三。



 子挺、扶,並進士及第。歷臺省,累為轉運使,皆知名。挺至殿中侍御史,扶嘗直集賢院,至工部員外郎。景德中,錄幼子掞為光祿寺主簿。
 大中祥符八年,又錄其孫師顏為三班借職,掞至殿中丞。



 許仲宣,字希粲,青州人。漢乾祐中,登進士第,時年十八。周顯德初,解褐授濟陰主簿,考功員外郎張乂薦為淄州團練判官。宋初赴調,引對便殿。仲宣氣貌雄偉,太祖悅之。擢授太子中允,受詔知北海軍。仲宣度其山川形勢、地理廣袤可以為州郡,因畫圖上之,遂升為濰州。



 初,議建牧馬監,令仲宣行視諸州,頗得善地。從征並門,掌
 給納,四十餘州資糧悉能集事。帝益知其強幹。開寶四年,知荊南轉運事。及征江南,又兼南面隨軍轉運事,兵數十萬,供饋無闕。南唐平,以漕挽功拜刑部郎中。中謝日,召升殿獎諭,賜緋。九年,詔知永興軍府事。



 太宗嗣位,遷兵部郎中,驛召赴闕,賜金紫。授西川轉運使,屬西南夷寇鈔邊境,仲宣親至大度河,諭以逆順,示以威福,夷人率服。會言事者云,江表用兵時,仲宣乾沒官錢,召還,令御史臺盡索財計簿鉤校,凡數年而畢,無有欺隱。



 改
 廣南轉運使,會征交州,其地炎瘴,士卒死者十二三,大將孫全興等失律,仲宣因奏罷其兵。不待報,即以兵分屯諸州。開庫賞賜,草檄書以諭交州。交州即送款內附,遣使修貢。仲宣復上章待罪,帝嘉之。



 太平興國六年冬,南郊畢,遷吏部郎中。八年,與膳部郎中、知雜滕中正,兵部郎中劉保勛,刑部郎中辛仲甫皆以久次郎署,擢升諫垣,仲宣為左諫議大夫。未幾,召還,以本官權度支。雍熙四年,出知廣州,未上,移知江陵府,俄改河南府。端拱
 中,遷給事中。淳化元年,卒,年六十一。



 仲宣性寬恕,倜儻不檢,有心計。初,為濟陰主簿時,令與簿分掌縣印。令畜嬖妾,與其室爭寵,令弗能禁。嬖欲陷其主,竊取其印藏之,封識如故,以授仲宣。翌日署事,發匣,則無其印,因逮捕縣吏數輩及令、簿家人,下獄鞫問,果得之於令舍灶突中。令聞之,倉皇失措,仲宣處之晏然,人服其量。嘗從征江南,都部署曹彬令取陶器數萬,給士卒為燈具。仲宣已預料置,奉之如其數。其才幹類此。



 子待用至國子
 博士,待問再舉及第,至殿中丞,待旦至比部員外郎。待用子巨源,亦登進士第。



 楊克讓,字慶孫,同州馮翊人。高祖公略,洪州都督。晉末,舉進士不第,州將劉繼勛闢為戶曹掾。漢乾祐中,本府節度張彥成表授掌書記。



 周廣順初,彥成移鎮安陽、穰下,克讓以舊職從行。彥成入為執金吾,病篤,奏稱其材可用。克讓以彥成死未葬,不忍就祿,退居別墅,俟張氏子外除。時論稱之。歷鎮寧軍掌書記。顯德二年,調授鳳
 翔府司錄參軍,加兼監察御史,以祖母老解官歸養。未幾,改延州觀察推官,與通判宋琪並為節度使趙贊所禮。累加朝散大夫兼殿中侍御史,連以家難去職。



 太祖素知其名,會贊入覲,復稱其才,即起為左補闕,掌蘄口榷貨務。乾德六年,知果州。上言願畢襄事,特賜緡錢,許葬畢赴任。開寶三年,就命為西川轉運副使,蜀民懷其善政,璽書褒美。代歸闕下,疏民利病十事,稱旨。太祖召升殿,賜坐勞問,面賜金紫。將大用,為侯陟所沮,事見陟
 傳。



 征南唐,命克讓知升州行府。升州平,就知州事兼水陸計度轉運使事,加兵部員外郎。太平興國初,就加刑部郎中、知大名府。會錢俶、陳洪進來歸疆土,以克讓為兩浙西南路轉運使。泉州民嘯聚為盜,克讓在福州,即率其屯兵至泉州,與王明、王文寶共討平之。四年,徙知廣州,俄兼轉運市舶使。明年,卒,年六十九。



 克讓少好學,手寫經籍,盈於篋笥。多收圖畫墨跡。歷官廉謹幹局,所至有聲。每視事,自旦至暮,或通夕,斷決如流,無有凝滯,
 當時稱為能吏。



 子希閔字無間。生而失明,令諸弟讀經史,一歷耳輒不能忘。屬文善緘尺,趙普守西洛,府中箋疏,皆希閔所為。將奏署本府掾,固辭不受,普優加給贍。張齊賢、李沆、薛惟吉、張茂宗繼領府事,皆優待之。卒,年三十九,有集二十卷。自教三子:日華,日嚴,日休,皆登進士第。日華都官員外郎,日嚴職方員外郎,日休殿中丞。希閔弟希甫,淳化三年進士,至屯田員外郎。從子日宣,亦登進士第。



 段思恭,澤州晉城人。曾祖約,定州司戶。祖昶,神山令。父希堯,晉祖鎮太原,闢為從事,與桑維翰同幕府。晉有天下,希堯累歷清顯。思恭以門蔭奏署鎮國軍節度使官。天福中,希堯任棣州刺史兼權鹽礬制置使。思恭解官侍養,奉章入貢,改國子四門博士,賜緋。開運初,出為華、商等州觀察支使。劉繼勛節制同州,闢為掌書記。繼勛入朝,會契丹入汴,軍士喧噪,請立思恭為州帥,思恭諭以禍福,拒而弗從,乃止。



 漢祖建國,授左補闕。隱帝時,蝗,
 詔遍祈山川。思恭上言:「赦過宥罪,議獄緩刑,茍獄訟平允,則災害不生。望令諸州速決重刑,無致淹濫,必召和氣。」從之。歷度支、駕部。周顯德中,定濱州田賦,世宗嘉之,賜金紫。丁外艱,服闋,拜左司員外郎。



 建隆二年,除開封令,遷金部郎中。乾德初,平蜀,通判眉州。時亡命集眾,攻逼州城,刺史趙廷進懼不能敵,將奔嘉州,思恭止之,因率屯兵與賊戰彭山。軍人皆觀望無鬥志,思恭募軍士先登者厚賞,於是諸軍賈勇,大敗賊,思恭矯詔以上供
 錢帛給之。後度支請按其罪,太祖憐其果干,不許,令知州事。丁母憂,起復,俄召為考功郎中,知泗州。



 會馮繼業自靈州舉宗來朝,帝以思恭代知州事,仍語之曰:「馮繼業言靈州非衛、霍名將鎮撫之不可,汝其往哉!」思恭曰:「臣奉詔而往,必能治之。」帝壯之,賜窄衣、金帶、錢二百萬,仍以途涉諸部,令別繼金帛以遺之。思恭下車,矯繼業之失,綏撫夷落,訪求民病,悉條奏免之。俄而回鶻入貢,路出靈州,交易於市,思恭遣吏市□砂,吏爭直,與之競。
 思恭釋吏,械其使,數日貰之。使還醞其主,復遣使繼牒詣靈州問故,思恭理屈不報。自是數年,回鶻不復朝貢。



 久之,遷右諫議大夫、知揚州。朝廷方經略江表,命思恭兼沿江巡檢。每出巡,委州事於通判,以牌印、鼓角、金鉦自隨。驛書自京師來者,令繼至其所,事多稽滯。因與通判李岧相告訐,詔以屬吏。思恭辭不直,責授太常少卿、改知宿州。太宗即位,遷將作監、知秦州。坐擅借官庫銀造器,又妄以貢奉為名,賤市狨毛虎皮為馬飾,為通判
 王廷範所發,降授少府少監、知邢州。太平興國六年,遷少府監雍熙元年,南郊畢,表乞復舊官,再為右諫議大夫。二年,知壽州。端拱初,遷給事中,尋知陜州。淳化三年,卒,年七十三。



 思恭以門資歷顯官,不知書,無學術;然踐更吏事,所至亦著勤績。子惟一至太常博士、三司度支判官。從子惟幾,第進士,仕至兵部員外郎。



 侯陟,淄州長山人。漢末,舉明經。周廣順初,試校書郎,為西州回鶻國信使判官,還補雷澤主簿。司門員外郎姚
 恕凡四薦陟,為襄城令、汝州防禦判官、濮陽襄邑令。建隆初,為冤句令,以清幹聞。二年,擢為左拾遺,仍知縣事。節度袁彥頗為不法,陟抗章言之,彥上表謝,自陳無罪,太祖亦不窮治。四年,令兼領本縣屯兵,俄改淮南轉運使,賜緋衣、黑銀帶,遷右補闕。乾德三年,就改侍御史。明年,入為左司員外郎、度支判官。朝議欲以本官領省事,改度支員外郎,依前充判官。開寶五年,復為左司員外郎。六年,權判吏部銓,俄賜金紫。十二月,詔與戶部員外
 郎、知制誥王祐等同知貢舉,未鎖宿,出知揚州。會出師收金陵,陟以所部敗南唐軍千人於宣化城。俄為部下所訟,追赴闕,陟度理窮,乃求哀盧多遜,多遜素與陟善,為其畫計。時江表未拔,太祖厭兵,南土暑熾,軍卒疫死,方議休兵,以為後圖。陟適從揚州來,知金陵危甚,多遜令上急變求見。陟時被病,令掖入,即大言曰:「南唐平在朝夕,陛下奈何欲班師,願急取之。臣若誤陛下,願夷三族。」上屏左右,召升殿問狀,遂寢前議,並赦陟罪,復知吏
 部選事。



 太平興國初,遷戶部郎中。俄而選人有妄冒,事發,詞涉於陟。南曹雷德驤將奏劾之,陟造便殿自首,出為河北轉運使。徵太原,為太原東路轉運使。駕還,次鎮州,命先還上都供頓軍需。以功遷左諫議大夫,權御史中丞事。五年,同知貢舉。開寶末,趙普在中書,陟嘗上疏言其短。至是,普再入相,陟頗憂恚。六年,南郊畢,加給事中。七年,三司使王仁贍左降,以陟與王明同判三司。八年,卒,贈工部尚書。



 陟有吏乾,性狡獪,好進,善事權貴,巧
 中傷人。太祖嘗召刑部郎中楊克讓,命坐與語,且諭以將大用。陟素忌克讓,偵知之。因奏事,上問識楊克讓否,陟曰:「臣與克讓甚善,知其人才識,朝廷佳士也。近聞其自言上許以大用,多市白金作飲器以自奉,臣頗怪之。」上怒,亟令克讓出典郡。其險詖如此。



 李符,字德昌,大名內黃人。漢乾祐中,郭從義討趙思綰於京兆,闢符在幕府,表為京兆府戶曹掾。歷郿縣主簿、保義軍節度推官。丁內艱,服除,調汝州防禦判官,權知
 州事。右庶子楊恪薦為大理正。乾德中,知歸州轉運司制置。



 歸朝,以京西諸州錢帛不登,選知京西南面轉運事。奏便宜百餘條,凡四十八事,命著為令,賜緋魚。因奏對稱旨,遷起居郎。後荊湖轉運許仲宣隨軍討南唐,詔符赴荊湖調發芻糧,符領船數千艘順流而下。事畢,賜金紫。符又建議鑿橫江河以通漕運,發和州三縣丁壯給其役。太祖欲幸西京,有事於南郊。符上書陳八難曰:「京邑凋弊,一也;宮闕不備,二也;郊廟未修,三也;百司不
 具,四也;畿內民困,五也;軍食不充,六也;壁壘未設,七也;千乘萬騎盛暑扈行,八也。」不從。禮畢還京,改比部員外郎、判刑部。



 太平興國初,遷駕部,轉祠部郎中,知廣州兼轉運使。二年,符圖海外諸城及嶺外花木各一以獻。在任有善政,民為立生祠。五年,召為右諫議大夫、判吏部銓兼大理寺理。三司副使範旻得罪,以符代之。賜白金三千兩。車駕幸大名,領行在三司。未幾,坐與官屬競課最,罷職守本官。



 七年春,開封尹秦王廷美出守西京,以
 符知開封府。廷美事發,太宗令歸第省過。趙普令符上言:「廷美在西洛非便,恐有他變,宜遷遠郡,以絕人望。」遂有房陵之貶。普恐洩言,坐符用刑不當,貶寧國軍行軍司馬。盧多遜貶崖州也,符白普曰:「珠崖雖遠在海中,而水土頗善。春州稍近,瘴氣甚毒,至者必死,願徙多遜處之。」普不答。先是,太宗尹京,符因宋琪薦弭德超事藩邸。符貶,德超為樞密副使,屢稱其冤。會德超以事貶,帝惡其朋黨,徙符嶺表,普移符知春州。至郡歲餘卒,年五十
 九。



 符無文學,有吏乾,好希人主意以求進用,終以此敗。至道二年,郊祀,追復右諫議大夫。祥符五年,錄其子璜試將作監主簿。



 魏丕,字齊物,相州人,頗涉學問。周世宗鎮澶淵,闢司法參軍。有盜五人獄具,丕疑其冤,緩之。不數日,果獲真盜,世宗嘉其明。慎歷頓丘、冠氏、元城三縣令。世宗即位,改右班殿直。自陳本以儒進,願受本資官。世宗曰:「方今天下未一,用武之際,藉卿幹事,勿固辭也。」未幾,出監明靈
 砦軍。世宗征淮甸,丕獲江南諜者四人,部送行在。詔獎之,賜錢十萬,遷供奉官、供備庫副使。



 太祖即位,改作坊副使。時楊承信帥河中,或言其反側未安,命丕賜承信生辰禮物,陰察之。還,言其無狀。太祖嘗召對,語丕曰:「作坊久積弊,爾為我修整之。」丕在職盡力,以久次轉正使。開寶九年,領代州刺史。凡典工作十餘年,討澤潞、維揚,下荊廣,收川峽,征河東,平江南,太祖皆先期諭旨,令修創器械,無不精辦。舊床子弩射止七百步,令丕增造至
 千步。及改繡衣鹵簿,亦專敕丕裁制。丕撤本坊舊屋,為舍衢中,收僦直及鬻死馬骨,歲得錢七千餘緡,工匠有喪者均給之。太祖幸洛郊祀,三司使王仁贍議雇民車牛運法物,太祖以勞民,不悅,召丕議之。丕請手柬本坊匠少壯者二千餘,分為遞鋪輸之,時以為便。



 雍熙四年,代郝正為戶部使。端拱初,遷度支使。是冬,出為黃州刺史。還朝,召對便坐,賜御書《急就章》、《朱邸集》。丕退作歌以獻,因自述願授臺省之職。太宗面諭曰:「知卿本儒生,然清
 望官奉給不若刺史之優也。」淳化初,改汝州刺史。歷知鳳州,改襄州。境內久旱,丕以誠禱之,一夕,雨沾足。明年,召還,屢求退居西洛,不許。



 四年,表求致仕,授左武衛大將軍,仍領汝州刺史。俄判金吾街仗。初,六街巡警皆用禁卒,至是,詔左右街各募卒千人,優以廩給,使傳呼備盜。丕以新募卒引對,遂分四營,營設五都,一如禁兵之制。五年,改領郢州刺史。俄改領復州,遷左驍衛大將軍。咸平二年,卒,年八十一。



 丕好歌詩,頗與士大夫游接,有
 時稱。南唐主李煜妻卒,遣丕充吊祭使,且使觀其意趣。煜邀丕登升元閣賦詩,丕有「朝宗海浪拱星辰」之句,以風動之。太宗嘗賜詩,令丕與柴禹錫和焉。



 董樞,真定元氏人。後唐太清中,以獻書授校書郎。累歷賓佐。晉天福中,為左拾遺、知樞密院表奏。周廣順初,為左補闕。世宗即位,詔常參官各奏封事,樞上平吳策。淮南平,遷浚儀令。恭帝即位,遷殿中侍御史。



 太祖乾德初,遷主客員外郎。上書請伐蜀,蜀平,通判劍州。會全師雄
 叛,攻劍。刺史張仁謙足疾不能戰,欲棄城走。樞固爭,戰賊敗之,因招餘眾降。仁謙飲樞令醉,密殺降數百,誣奏樞與賊通。會中使自成都還,備言其事,太祖並召之,庭辯曲直,仁謙遂屈。下御史臺鞫之,黜宋州教練使,以樞嘗貢西伐計,遷比部郎中。三年,出兼桂陽監使,上書請伐廣南。詔益桂陽戍卒三千,令樞統之。



 開寶二年,又上方略。會劉鋹令內侍曾居實侵桂陽,樞擊退之。三年,大舉伐鋹,令樞率兵趨連口,克之。改兵部郎中,權知連州
 兼行營招撫使。嶺南平,賜錢三百萬。四年,移知襄州,又為河北轉運使,改判西京留司御史臺。



 初,樞罷桂陽監,以左贊善大夫孔璘代之。璘通《三禮》,嘗講學於河朔。擢第,歷州縣。及升朝,蒞桂陽,歲滿,以太子洗馬趙瑜代之。



 瑜,趙州人。家世豪右,自言諳練邊事。開寶中,命為易州通判,歲滿,移桂陽。瑜至,即稱疾,遂以著作郎張侃代之。侃至月餘,奏瑜在任累月,得羨銀數千斤,雖送官而不具數聞,計樞與璘隱沒可知矣。詔下御史案之,獄具。有
 司計盜臟法,俱當死。太祖曰:「趙瑜非自盜,但不能發擿耳。」樞、璘並坐死,瑜決杖流海島。擢侃為屯田員外郎。



 論曰:顏衎振舉風憲,不避強御。劇可久居廷尉之任,以平允聞。趙逢果斷之士,而獨尚嚴酷,處之要密之職,則非所宜。蘇曉銳意深刻,樂致人罪,後嗣衰謝,厥報不誣。高防陳逆順以聳臣節,體明慎而究疑獄,治跡清操,沒而彌章。若其自誣以救人之死,古人何加焉。馮瓚省關市之苛賦,設方略以擊賊,功若可稱,而巧宦任數,竟致
 傾敗,理固然矣。邊珝、王明、許仲宣、楊克讓當官效用,以清乾稱。然仲宣寬簡持重,造次不撓,蓋人之難能者。王明累參戎事,預立戰功,至若開諭元福,止其暴誅,此赴蹈之仁也。段思恭遏亂兵,擊群寇,便宜從事,以著奇績,斯亦可矣。然不能動遵規矩,速訟左降者再焉。侯陟吏才適用,患在忮刻。李符博通時務,乃事深文,以致投荒自弊,遂為口實。魏丕久典工效,以濟戎用,至於平反冤盜之獄,救楊承信之誣,善尤可稱。董樞論平吳伐蜀及
 取廣南,咸克舉之,且多戰功,而以貪墨取敗。惜哉!



\end{pinyinscope}