\article{列傳第二百一十三忠義九}

\begin{pinyinscope}

 ○趙時賞趙希洎劉子薦黃文政呂文信鐘季玉潘方耿世安丁黼米立楊壽孫趙文義侯畐
 王孝忠高應松張山翁黃申陳羍蕭雷龍宋應龍褚一正鄒洬劉子俊劉沐孫𬃄彭震龍蕭燾夫陳繼周陳龍復張鏜張云張汴呂武鞏信蕭明哲杜滸林琦蕭資徐臻金應何時陳子敬劉士昭王士敏趙孟壘趙孟松



 趙時賞字宗白,和州宗室也,居太平州。咸淳元年擢進士第,累官知宣州旌德縣。德祐元年,北軍至境,時賞擁民兵捍戰有功,升直寶章閣、軍器太監。從二王入閩中。
 益王即位,擢知邵武軍。未幾,言者以棄城論罷之。



 文天祥開都督府於南劍,奏闢參議軍事、江西招討副使。與宗室孟溁提兵趣贛州,取道石城,復寧都縣。數以偏師當一面,戰比有勝。時賞風神明俊,議論慷慨,有策謀,尤為天祥所知。及空坑之役,兵敗走吳溪,為追兵所執,不屈死之。



 時賞在軍中時,見同列盛輜重,飾姬侍,嘆曰:「軍行如春游,其能濟乎?」及被執,見系累它僚屬至者,時賞輒麾去,云:「小小簽廳官爾,執此何為?」由是得脫者眾。



 趙希洎,宗室子,居宜春。歷官至戶部尚書。咸淳中,迕丞相賈似道,出領廣東轉運使。德祐元年,制置使黃萬石檄其勤王,得潰卒數百,道經廬陵,郡守邀其軍,遂與從子必向避地贛州。亂定歸里,時袁守聶嵩孫,希洎內姻也,勉之內款,不能屈。文天祥兵敗,以失言與必向俱被囚,辭節愈厲,家人饋食,則碎器覆諸地,俱不食,據榻而死。



 劉子薦,字貢伯,吉州安福人。父夢驥,以進士歷官知澧
 州,沒於王事。子薦以父任為湘鄉尉,以獲盜功調撫州司錄。有訴王應亨毆死荷擔黃九者,獄成矣,子薦閱爰書,疑而駁之。俄烈風迅雷闢獄戶,裂吏手契,殺人者實孔目馮汝能,非應亨也。獄遂白,得免死者八人。事聞,頒諭天下之為理官者。改知贛縣,監行在左藏庫,通判常德府,知融州。陛辭,度宗尉之曰:「廣郡凋瘵,賴卿撫摩。」子薦對曰:「臣當推行德化,以安其民。」至官,以廉靜著聞。



 主管仙都觀,廣西經略司檄為參議官。德祐二年十一月,北
 兵至靜江,權經略使馬塈遣子薦提徭兵藥弩手守城東門,勢不支。時瀛國公已入燕,子薦取笏書其上云:「我頭可斷,膝不可屈。」登城北望再拜,取所衣袍瘞之,語左右曰:「事急不可為,吾有以死守。」或諷子薦遁去,子薦曰:「死事,義也,何以遁為?」竟死之。



 有黃文政者,淮人。戍蜀,軍潰,間道走靜江。馬塈邀與同守,城破,文政被執,大詬不屈。大軍斷其舌,以次劓刖之,文政含胡叱咄,比死不絕聲。



 呂文信,文德之弟也。仕至武功大夫、沿江副司諮議官。德祐初,帥舟師次南康斛林,夾白鹿磯與北兵遇,戰死。特贈寧遠軍承宣使。子師憲,特與帶行閣職,與兩子承信郎恩澤。仍立廟賜額。



 河湖砦巡檢張興宗亦死之。贈武翼郎,賜緡錢三萬,仍與一子承信郎恩澤。



 鐘季玉,饒州樂平人。淳祐七年舉進士,調為都大坑冶屬,改知萬載縣。淮東制置使李庭芝薦之,遷審計院,改宗正寺簿,又遷樞密院編修,出知建昌軍。會有旨江西
 和糴,季玉至郡才半年,屬歲旱,度其經賦不能辦,請於朝,和糴得減三之一。遷提舉常平,未幾,改轉運判官,皆不赴。後以江西轉運判官強起之。郡大胥以賄敗,前使百計護之,季玉卒窮治,投嶺表。俄以秘書丞召還,遭前使構讒而封駁之,改都大提點坑冶。北兵渡江,季玉徙寓建陽,兵至,不屈死之。



 有潘方者,溫州平陽人。寶祐四年進士,調監慶元府市舶。慶元降附,方不屈赴水死。



 耿世安,為武翼大夫、淮東副總管、兩淮都撥發官。初,諜
 報大兵至,制置使賈似道調世安提兵往漣水軍增戍。眾方猶豫,世安徑迎至漁溝,以三百騎入陳鏖擊,自午至酉,身被七創,猶能追殺潰兵。收兵還,至數里沒。事聞,贈五官,立廟淮安,賜額忠武。



 丁黼,成都制置使也。嘉熙三年,北兵自新井入,詐豎宋將李顯忠之旗,直趨成都。黼以為潰卒,以旗榜招之,既審知其非,領兵夜出城南迎戰,至石筍街,兵散,黼力戰死之。方大兵未至,黼先遣妻子南歸,自誓死守。至是,從
 黼者惟幕客楊大異及所信任數人,大異死而復蘇。黼帥蜀,為政寬大,蜀人思之。事平,賜額立廟。



 米立,淮人,三世為將。從陳奕守黃州,奕降,立潰圍出。江西制置使黃萬石署為帳前都統制。大兵略江西,立迎戰於江坊,被執不降,系獄。行省遣萬石諭之曰:「吾官階一個先牌寫不盡,今亦降矣。」立曰:「侍郎國家大臣,立一小卒爾,何足道。但三世食趙氏祿,趙亡,何以生為?立乃生擒之人,與投拜者不同。」萬石再三說之,不屈,遂遇害。



 趙文義者,郢州都統制。更戍歸,與北兵遇,力戰死之。初,開州之役,文義兄武義亦死焉。



 有楊壽孫者,為雲安軍主簿兼教參佐忠勝軍。端平中,北兵至中江縣,與將官何庚、安惟臣、田廣澤、歹坤等連戰二日,俱死之。壽孫贈通直郎,官一子下州文學。庚等各贈承節,一子進勇副尉。



 侯畐字道子,溫州樂清人。三貢於鄉,兩試轉運司,皆第一。以武舉授合浦尉,柳城令,侍衛步軍司干辦公事,侍衛
 馬軍行司計議官。寶祐五年,制置使賈似道闢通判海州兼河南府計議官。李松壽據山東,突出漣、泗,畐鏖城下,死之,懃室遇害。太學生三十一人言於朝,即海州賜廟旌忠,謚曰節毅,仍立廟其鄉。畐所著有《霜崖集》。



 王孝忠,為鎮江前軍統制兼淮東路分,戍淮陰。楊貴叛,孝忠率眾迎戰,勝氣百倍。俄水軍統制朱信降賊,孝忠孤軍力不敵,死焉。



 高應松,開慶元年進士。繇衡州教授通判廣德軍,召為
 國子監丞,權禮部員外郎、翰林權直。北兵自湧金門入,舉朝奔竄,從官留者九人,應松其一也。遷中書舍人、直學士院,尋遷權工部侍郎,進端明殿學士、簽書樞密院事。從瀛國公至燕,絕粒不語,越七日卒。



 張山翁字君壽,普州人。景定三年進士。德祐元年,為荊湖宣撫司干官。鄂守張晏然議納款,山翁以書譙讓之。晏然既降,山翁被執軍前,諭曰:「若降,不失作顯官。」山翁酬對不屈。行省官賈思貞義之,貸不殺。後居黃鵠山,聚徒教授而終。有《
 南紀》、《緇林藏》、《雲山》、《相鋤》等集。



 黃申,字酉鄉,井研人。開慶元年進士,授德安尉,攝主簿兼提點江西刑獄司簽廳,獄事多所辨明。丞相江萬里、提刑黃震交薦之,調樂安丞。



 申為政廉謹,有治聲。以恩升從事郎。大兵拔撫州,下諸縣索降狀,樂安令率其僚聯署以上。申初聞變,悉遣家人遠避,至是獨抗不往。令遣吏促之,申不動。吏白令,令怒。俄而吏民數百人集於庭,強輿致之,申顛踣於地,若中風然。眾捽蹴詬叱曰:「為
 爾不順,將累我輩。」申陽死為不聞,令無如之何。申有惠愛在民,至暮,眾舁入置中堂,翼日或食以粥,得免。遂去,隱巴山中以終。



 陳羍,字肇芳,一字偉節,饒州安仁人。父詩川,以武功補沭陽令。咸淳元年,父子同舉進士。調滁州司戶參軍。父喪免,改荊閫糧料院,又以母憂去。調朐山主簿。制置使印應雷闢入幕。德祐元年秋,羍繇海道歸杭,授南安軍教授,不就,還家。



 羍少與謝枋得游,會枋得起兵安仁,首
 拔入幕。執安仁令李景,景,羍里人也。景請得以家貲二萬贖罪,羍曰:「普天之下,莫非王土。家財獨非朝廷錢耶?」聲其罪斬之。景子率鄉民五千報怨,羍度勢不敵,引兵趨信州。會守吏遁去,羍聞於朝,就攝郡事。



 益王即位,羍入覲,遷宗正寺簿、太府寺丞、領江東安撫使。出上饒,接應郡縣,所部才千餘人,屯火燒山。越數月,戰潰,被執至豫章,元帥憐其才,羈縻館留之,遁去。後三年復起兵,尋敗入積煙山中,自剄死。所著有《鶴心集》,其詩多譏刺當
 時之士大夫。弟年同時被執,死焉。



 蕭雷龍,字顯辰,建昌新城人。景定三年進士,調臨安府學教授,通判衢州。及州守棄城遁,朝命雷龍權知府事。



 北兵薄城下,不降,脫去還建昌。建昌已降,雷龍與同里人黃巡檢起兵。時大兵四合,雷龍度不可支,與黃巡檢及麾下數人奔入閩,未出境,為同安武人徐浚沖獲送縣。權縣尹劉聖仲素與雷龍有怨,殺之。後聖仲北來,泊舟小孤山,有巨艦沖前,建大旗書曰「蕭知府兵」,繼見雷
 龍坐船上,聖仲大呼,有頃不見,以驚死。



 宋應龍者,儒生。通兵,出入行陳三十餘年,為諮議官,寓泰州。德祐二年六月甲寅,大兵至泰州,裨校孫貴、胡惟孝、尹端甫、李遇春開門迎降,應龍與其妻自縊於圃中。



 是時,提刑諮議褚一正字粹翁,廬州人,武舉進士。督戰高沙被創,竟沒於水。知興化縣胡拱辰,縣破,亦死之。



 鄒洬,字鳳叔,吉水人,後徙永豐。少慷慨有大志,以豪俠鳴。從文天祥勤王,補武資至將軍。益王立,改寺丞,領江
 西招諭副使。聚兵寧都,得數萬,改授江西安撫副使。復興國、永豐二縣,進兵部侍郎兼江東、西處置副使。及永豐敗,繼從天祥間關嶺道,未幾,復出開督府,分司永豐、興國境上。北兵驟至,大戰,洬脫身走至潮州。及天祥被執,洬自殺。



 當是時,從天祥勤王死事者,洬與劉子俊等凡十有九人,因次第其名,附見左方。



 劉子俊字民章,廬陵人。嘗中漕試。少與文天祥同里閈,相友善。天祥開督府興國,子俊詣府計事,補宣教郎、帶
 行軍器監簿兼督府機宜。空坑兵敗,子俊收兵保洞源,接應郡縣。尋入廣,與大兵遇,戰潰,復招集散亡,與鄒洬同趨潮州。天祥兵敗,子俊被執,自詭為天祥,意使大兵不窮追,天祥可間走也。未幾,別隊執天祥至,相遇於途,各爭真贗,至大將前,始得其實,乃烹子俊。



 劉沐字淵伯,廬陵人。文天祥鄰曲也,少相狎暱,天祥好奕,與沐對奕,窮思忘日夜以為常。及起兵,闢補宣教郎、督府機宜。暨天祥出使,沐領兵還。天祥歸,開府南劍,沐
 收部曲來會,改授太府寺簿,專將一軍,為督府親衛。會空坑兵敗,被執至豫章,父子同日死焉。仲子死亂兵,季子復從天祥死嶺南。當時江西忠義皆沐所號召。沐性沈實而圓機,晝夜應酬,亹亹不倦云。



 孫𬃄字實甫,吉州龍泉人,獻簡公抃之後,天祥長妹婿也。天祥起兵,檄𬃄招忠義士,補宣教郎、帶行監官告院、知吉州龍泉縣。天祥擁兵出贛,里人奉𬃄復龍泉,拒守不下,尋為叛者所陷,執至隆興殺之。



 彭震龍字雷可,永新人,天祥次妹婿也。性跌蕩喜事,嘗以罪墨。天祥起兵,補宣教郎、帶行太社令、知永新縣。會天祥出使被執,震龍遁歸,吉州已失,乃結峒獠起兵。天祥兵出嶺,震龍接應,復永新。大兵至,震龍為親黨所執,至帥府,腰斬之,屠永新。



 蕭燾夫,永新人,與兄敬夫俱天祥客。燾夫為詩有豪俊氣。天祥起兵,補從仕郎。及彭震龍謀復其縣,燾夫贊之。縣受屠,兄弟俱死之。



 陳繼周字碩卿,寧都人。淳祐三年貢於鄉。以捕盜功行,未奏名,授廉州司法,南豐縣知錄,淮東總領幹官,藤州觀察推官,知吉州永豐縣,改知高安縣、廣東經略司準備差遣、知衡陽縣,闢淮東轉般倉、江東提點刑獄乾辦公事。



 未上,會咸淳十年,詔徵勤王,文天祥方守贛州,即日舉兵,造繼周問計。繼周慨然為具言閭里豪傑子弟與凡起兵之處,其為方略甚詳。於是留繼周幕中,晝夜調度,授繼周江西安撫司準備差遣,率贛士以從。繼周
 雖弱不勝衣,而年德有以服人,士視為父兄,進止疾徐惟指呼,無敢先後。詔改繼周合入官,帶行監文思院,差充江、浙制置司主管機宜。所部夜襲大兵於南柵門,殺傷相當,質明猶戰,渴赴水死。



 張汴字朝宗,一字次山,蜀人。少客丞相吳潛兄弟門,出入荊閫歷年,明習韜略。潛兄弟既罷,廢斥者十餘年。繼文天祥起兵,闢為秘閣修撰,領廣東提舉、督府參謀,左右幕府,知無不為。空坑兵敗,為亂兵所殺。處置使鄒洬
 得其尸葬之。



 呂武,太平州步卒也。文天祥出使,武應募從行,偕脫鎮江之難,沿淮東走海道,賴武力為多。天祥開府南劍,武以武功補官,遣之結約州縣起兵相應。道阻,復崎嶇數千里即天祥於汀、梅,挺身患難,化賊為兵。以環衛官將數千人出江西,以遇士大夫無禮,死於橫逆,一軍揮涕而葬之。武忠梗出天性,不避強御,而好面折人過,多觸忌諱,故及於禍云。



 鞏信,安豐軍人。為荊湖都統,沈勇有謀。本隸蘇劉義部曲,文天祥開督府,劉義以信與王福、張必勝詣天祥。信官至團練使、同督府都統制、江西招討使。初至都府,天祥以義士千人付之,信曰:「此輩徒累人爾。」乃招淮士數千自隨,然常怏怏曰:「有將無兵,其如彼何!」天祥自興國趨永豐,大兵追其後,信戰於方石嶺,中數矢,傷重不能戰,自投崖石而死。士人葬之,顏色如生。贈清遠軍承宣使,立廟旌之。



 蕭明哲字元甫,太和人。性剛毅有膽氣,明大節。少舉進士,天祥開府汀州,闢充督乾架閣監軍。師出嶺,明哲以贛縣民義復萬安,連結諸砦拒守。兵敗,被執不屈,死於隆興。臨刑大罵不絕口,聞者壯之。



 杜滸字貴卿,丞相範從子也,少負氣游俠。德祐元年,有詔勤王,滸時宰縣,糾集民兵得四千人。文天祥開閫平江,往附焉。時陳志道等贊天祥出使,滸力爭不可,志道逐之去。已而天祥果見留,志道竊藏逃歸。天祥北行,諸
 客無敢從者,滸獨慨然請行。特改兵部架閣。從京口,以計賂守夜劉千戶者,得官鐙,脫天祥,偕走淮甸,繇海道以達永嘉。



 益王即位,授司農卿、廣東提舉、招討副使、督府參謀。尋往溫、臺招集兵財。福安陷,與天祥相失,遂趨行朝。蘇劉義疑滸自來,欲殺之,陳宜中、張世傑不可,使人監護之,乃免。久之,奉命復入天祥幕。及空坑兵敗,又與跋涉患難以出。天祥移屯潮州,滸議趨海道,天祥不聽,使護海舟至官富場。滸懼力單,徑趨崖山,兵潰被執,
 以憂憤感疾卒。



 林琦,閩人也。德祐二年,大兵既迫臨安,琦於赭山結集忠義數千人,捍御海道。以功補宣教郎、督府主管機宜文字,充檢院。文天祥開府南劍,琦佐其幕。琦外文採,內忠實,數涉患難,無怨懟辭。及潮州移屯,琦俱被執,至惠州遁,復執之北行,赴水,為吏所拔,至建康,以憂憤死。



 蕭資,天祥幕下書史也。天祥起兵,資於患難中扶持甚至。空坑兵敗,以全督府印功,升閣門、路鈐轄。資性和厚,
 臨機應變,輯穆將士,總攝細務,任腹心之寄。潮陽移屯,與大兵遇,死之。



 徐臻,溫州人。父官河南,德祐元年春,臻往省,以道阻。會天祥勤王,臻往依之,以筆札典樞密,小心精練。天祥被執,臻脫難復來,願從天祥北行,扶持患難,備殫忠款,至隆興病死。



 金應者,性少剛知義。為天祥職書司,入京補承信郎,官路分。天祥奉使被執,左右皆散,應獨無畔志。及脫走鎮
 江,至淮東,以憂憤死焉。



 何時字了翁,撫州樂安人,天祥同年進士也。調廬陵尉,尋入江西轉運司幕府,還臨江軍司理參軍。郡獄相傳,舊斬一寇,尸能行一里許。眾神之,塐為肉身皋陶。時至,取故牘閱,此寇嘗掠殺數人,曰:「如此可為神乎?」命鞭之,湛於水,人服其明。改知興國縣。



 天祥起兵,闢署帥府機宜、帶行監文思院。天祥入衛,時任留司,分司吉州。餉運平江,天祥奏時知撫州。吉州下,時脫身歸鄉里。益王立,
 天祥開府南劍,時起兵趨興國接引,以時帶行卿監、江西提刑。時聚兵復崇仁縣,未幾,大軍奄至,兵敗,削發為僧,竄跡嶺南,賣卜自給,變姓名,自號堅白道人。



 又有陳子敬者,贛州人。以貲雄鄉里,嘗從天祥游。天祥開閫汀州,子敬募集民兵屯皂口,據贛下流。及天祥攻贛,子敬與合謀,忠效甚著。空坑兵敗,復聚兵屯黃塘砦,連結山砦不降。大軍以重兵襲其砦,砦潰,子敬不知所終。



 劉士昭,太和人,嘗為針工。與鄉人同謀復太和縣,敗,血
 指書帛云:「生為宋民,死為宋鬼,赤心報國,一死而已。」因以其帛自縊死。



 其黨入獄,多乞憐茍免。有王士敏者,獨慷慨不撓,題其裾:「此生無復望生還,一死都歸談笑間,大地盡為腥血污,好收吾骨首陽山。」臨刑嘆曰:「恨吾病失聲,不能大罵耳。」



 同時有趙孟壘者,合州人。登開慶元年第,為金華尉。臨安降,與從子由鑒懷太皇太后帛書詣益王,擢宗正寺簿、監軍。復明州,戰敗見獲,不屈磔死。



 方大軍駐紹興,福王與芮從子曰孟松,謀舉兵,事洩,被
 執至臨安。範文虎詰其謀逆,孟松詬曰:「賊臣負國厚恩,共危社稷,我帝室之胄,欲一刷宗廟之恥,乃更以為逆乎?」文虎怒,驅出斬之,過宋廟,呼曰:「太祖、太宗列聖之靈在天,何以使孟松至此?」都人莫不隕淚。既死,雷電晝晦者久之。



\end{pinyinscope}