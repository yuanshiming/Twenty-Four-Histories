\article{本紀第二十九}

\begin{pinyinscope}

 高宗六



 八
 年春正月戊子朔,帝在建康。丙申,減臨安府夏稅折輸錢。戊戌,張守罷。辛丑,偽齊知壽州宋超率兵民來歸。蔡州提轄白安時殺金將兀魯,執其守劉永壽來降。詔
 以方議和好,禁沿海州郡遣人過淮招納。丁未,大閱張俊軍。戊申,以兵部侍郎胡世將為四川安撫制置使。二月戊午,劉錡入見。減建康府夏稅折輸錢,蠲民戶逋租、和市科調。庚申,日中有黑子。以呂頤浩為江東安撫制置大使兼行宮留守。壬戌,岳飛乞增兵,不許。癸亥,帝發建康。丙寅,以胡安國《春秋傳》成書,進寶文閣直學士。戊寅,帝至臨安。己卯,以戶部尚書章誼為江東安撫制置大使兼行宮留守,呂頤浩為醴泉觀使。甲申,減紹興府
 和市絹萬匹。



 三月己丑,以知南外宗正事仲儡嗣濮王。庚寅,以禮部尚書劉大中參知政事,兵部尚書王庶為樞密副使。壬辰,復以秦檜為尚書右僕射、同中書門下平章事兼樞密使。甲午,陳與義罷。戊戌,增夔州路路分都監一員,修治關隘,練義兵。己亥,蠲農器及牛稅。以李天祚為靜海軍節度使、交趾郡王。壬寅,定以故相韓忠彥配享徽宗廟廷。丁未,蠲所過州縣民積欠稅賦。戊申,蠲江西、湖南諸州月椿錢各萬緡。己酉,命考核川、陜宣
 撫司便宜所授官,冒濫尤甚者悉與裁減。



 夏四月庚申,初置戶部和糴場於臨安。壬戌,遣王庶巡視江、淮邊防。丁丑,復置六路發運司。癸未,詔三衙管軍輪宿禁中。



 五月庚戌,詔鎮江府募橫江軍千人。竄內侍羅但於海島。庚子,禁貧民不舉子,其不能育者給錢養之。壬寅,貶劉子羽為單州團練副使、漳州安置。丁未,金國使烏陵思謀、石慶充與王倫等偕來。戊申,以資政殿學士葉夢得為江東安撫制置大使。己酉,王庶至淮南,檄張宗顏將
 兵七千屯廬州,巨師古三千屯太平州,分韓世忠軍屯泗州及天長縣。



 六月壬戌,賜衍聖公孔玠衢州田五頃,奉先聖祠事。癸亥,趙鼎上《重修哲宗實錄》。壬申,賜禮部進士黃公度以下三百九十五人及第、出身。王庶自淮南還入見。乙亥,以中護軍統制張宗顏知廬州,命劉錡率兵移屯鎮江府。丁丑,烏陵思謀、石慶充入見。



 秋七月乙酉朔,復命王倫及藍公佐奉迎梓宮。錄司馬光曾孫伋補承務郎。辛亥,彗出東方。



 八月戊午,詔:「日者遣使報
 聘鄰國,期還梓宮。尚慮邊臣未諭,遂馳戎備,以疑眾心。其各嚴飭屬城,明告部曲,臨事必戒,無忘捍禦。」甲子,蠲江東路月椿錢萬三千緡有奇。丁丑,彗滅。遣監察御史李寀宣諭江西,措置盜賊。



 冬十月丁巳,劉大中罷。甲戌,趙鼎罷。乙亥,日中有黑子。丁丑,金國使張通古、蕭哲與王倫皆來。韓世忠乞奏事行在,不許。戊寅,樞密副使王庶乞免簽書和議文字,累疏求去,不許。



 十一月甲申,以翰林學士承旨孫近參知政事。丙戌,遣大理寺丞薛倞、
 朱斐詣廣南路決滯獄。戊戌,王倫入見。己亥,復以倫為國信計議使,中書舍人蘇符副之,符辭以疾。庚子,以孫近兼權同知樞密院事。辛丑,詔:「金國遣使入境,欲朕屈己就和,命侍從、臺諫詳思條奏。」從官張燾、晏敦復、魏矼、曾開、李彌遜、尹焞、梁汝嘉、樓照、蘇符、薛徽言、御史方廷實皆言不可。甲辰,王庶罷。辛亥,以樞密院編修官胡銓上書直諫,斥和議,除名、昭州編管。壬子。改差監廣州都鹽倉。十二月甲寅,以趙鼎為醴泉觀使。乙卯,以宗正少
 卿馮楫為國信計議副使。己未,以吏部尚書李光參知政事。戊辰,王倫言金使稱「詔諭江南」,其名不正。秦檜以未見國書,疑為封冊。帝曰:「朕嗣守祖宗基業,豈受金人封冊。」癸酉,館職胡珵、朱松、張擴、凌景夏、常明、範如圭上書,極論不可和。甲戌,以端明殿學士韓肖冑簽書樞密院事。乙亥,命肖冑等為金國奉表報謝使。丙子,張通古、蕭哲至行在,言先歸河南地,徐議餘事。以監察御史施廷臣為侍御史,權吏部尚書張燾、侍郎晏敦復以廷臣
 主和議而升用,執奏不行。御史中丞勾龍如淵、右諫議大夫李誼、殿中侍御史鄭剛中凡再至都堂,及宰執議取國書。丁丑,詔:「金國使來,盡割河南、陜西故地,通好於我,許還梓宮及母兄親族,餘無需索。令尚書省榜諭。」庚辰,帝不御殿。以方居諒陰,難行吉禮,命秦檜攝塚宰,受書以進。是月,虛恨蠻犯嘉州忠鎮砦。是歲,始定都於杭。



 九年春正月壬午朔,帝在臨安。丙戌,以金國通和,大赦。河南新復州軍官吏並不易置,蠲其民租稅三年,徭役
 五年。以王倫同簽書樞密院事,充奉護梓宮、迎請皇太后、交割地界使。戊子,遣判大宗正事士人褭、兵部侍郎張燾詣河南修奉陵寢。庚寅,賜劉光世號和眾輔國功臣,張俊加少傅、安民靖難功臣,韓世忠為少師,張浚復左宣奉大夫。辛卯,以尹焞為徽猷閣待制、提舉萬壽觀兼侍讀,焞力辭不拜。壬辰,加岳飛、吳玠並開府儀同三司,楊沂中太尉。癸巳,建皇太后宮。甲午,金宿州守臣趙榮來歸。丙申,金主詔諭河南諸州以割地歸我之意。改發
 運經制司為經制司,命戶部長貳一人領使,仍置副或判官。戊戌,以王倫為東京留守,郭仲荀為副,戶部侍郎梁汝嘉兼江、淮、荊、浙、閩、廣路經制使,司農卿霍蠡為判官。己亥,以吳玠為四川宣撫使。



 二月癸丑,以徽猷閣待制周聿為陜西宣諭使,監察御史方廷實宣諭三京、淮北。丁巳,以郭仲荀為太尉、東京同留守。慕洧寇環州。戊午,以知金州郭浩為陜西宣撫判官。壬戌,以李綱為湖南路安撫大使,張浚知福州,尋復資政殿大學士,為福
 建路安撫大使。命周聿、方廷實搜訪隱士。甲子,均定諸州縣月樁錢。己巳,以郭浩為陜西宣諭使。壬申,命修《徽宗實錄》。癸酉,詔盜賊已經招安而復嘯聚者,發兵加誅毋赦。是月,日中有黑子,月餘乃沒。江西統制官李貴以其軍歸楊沂中。



 三月丁亥,以和州防禦使璩為保大軍節度使,封崇國公。丙申,王倫受地於金,得東西南三京、壽春、宿、亳、曹、單州及陜西、京西之地。兀□還祁州。己亥,分河南為三路,廢拱州。辛丑,以翰林學士樓照簽書樞
 密院事。甲辰,偽齊知開封府鄭億年上表待罪,召赴行在。丁未,正偽齊所改州縣名。是春,夏人陷府州。



 夏四月庚戌朔,呂頤浩薨。辛亥,命樓照宣諭陜西諸路。壬午,金鄜延路經略使關師古上表待罪,命知延安府。癸丑,落趙鼎奉國軍節度使為特進,仍知泉州。金陜西諸路節制使張中孚上表待罪,命為檢校少保、寧國軍節度使、知永興軍、節制陜西諸路軍馬。甲子,以觀文殿學士孟庾為西京留守,資政殿學士路允迪南京留守。丙寅,金
 秦鳳經略使張中彥上表待罪,命知渭州。以孫近兼權同知樞密院事。壬申,移壽春府治淮北舊城。癸酉,詔新復諸路監司、帥臣按劾官吏之殘民者。韓世忠、張俊入見。



 五月庚寅,奉迎東京欽先、孝思殿累朝御容赴臨安。辛卯,復命江、淮守臣二年為任。乙未,復置淮東提舉茶鹽司。癸卯,復召募耆長法。丙午,鄜延副將李世輔部兵三千自鳳翔來歸,賜名顯忠。



 六月庚戌,皇后邢氏崩於五國城。辛亥,夏國主乾順卒。壬子,樓照以東京見卒四
 千四百人為忠銳三將。庚申,盜入邵武軍。壬戌,以新復州縣官吏懷不自安,降詔開諭。己巳,吳玠薨。壬申,樓照承制以李顯忠為護國軍承宣使、樞密行府前軍都統制,率部兵及夏國招撫使王樞赴行在。癸酉,澧州軍事推官韓紃坐上書論講和非計,送循州編管。乙亥,以孟庾兼東京留守。王倫自東京赴金國議事。樓照承制以楊政為熙河經略使,吳璘為秦鳳經略使,仍並聽四川宣撫司節制;郭浩為鄜延經略使、同節制陜西軍馬。丙
 子,分宣撫司兵四萬人出屯熙、秦,六千人隸郭浩,留吳玠精兵二萬人屯興元府、興、洋二州。戊寅,置錢引務於永興軍。是月,撫州鈐轄伍俊謀據桃源復叛,湖北安撫薛弼召誅之。



 秋七月甲申,以文臣為新復諸縣令。丙戌,東京耆老李茂松、寇璋等二百人奉表稱賀,皆引見,補官遣還。復置都水南、北丞各一員。丁亥,金人拘王倫於中山。丙申,命詳驗劉豫偽官,換給告身。乙巳,給還偽齊所沒民間資產。以胡世將兼權主管四川宣撫司。



 八月
 己酉,復淮南諸州學官。庚戌,賜陜西諸軍冬衣,絹十五萬匹。命前川、陜宣撫司便宜所補官,限一年自陳,換給告身。丙辰,金國以撻懶主和割地,疑其二心,殺之。壬戌,蠲成都、潼川路歲輸對糴等米五十四萬石、水運錢七十九萬緡。乙丑,給新法度牒、紫衣師號錢二百萬緡付陜西市軍儲。己巳,命陜西復行鐵錢。庚午,遣蘇符等使金賀正旦。乙亥,遣前知宿州趙榮、知壽州王威俱還金國。以關師古為行營中護軍前軍統制。



 九月己卯,命鄜
 延、秦鳳、熙河路招納蕃部熟戶及陷沒夏國軍民。丙戌,封叔士人褭為齊安郡王。庚寅,罷經制司,令提刑兼領常平事。甲午,名皇太后殿曰慈寧。丙申,以威州防禦使溫濟告韓世忠陰事勒停、南劍州編管。世忠又奏欲殺之,詔移萬安軍。己亥,郭仲荀率東京兵五千至鎮江。



 冬十月辛亥,詔侍從官各舉所知二人。王倫見金主於御林子,被拘於河間,遣其副藍公佐先歸。甲寅,王樞入見,並其俘百九十人皆縱遣還夏國。己未,蠲階、成、岷、鳳四州
 民稅之半。戊辰,慈寧宮成。甲戌,日中有黑子。丙子,賜李顯忠軍錢十萬緡。是月,岳飛入見。十一月戊寅朔,賜吳玠家錢三萬緡,以其弟璘為龍神衛四廂都指揮使。申命刑部大理官編次刑名斷例。癸未,嗣濮王仲儡薨。己丑,詔三省官屬詳覆在京通用令。追復張所為直龍圖閣。



 十二月甲寅,命續編《紹興因革禮》。甲子,李光罷。戊辰,命續修《元豐會要》。兀□留蘇符等於東京,謀復取河南。



 十年春正月丙戌,遣莫將等充迎護梓宮、奉迎兩宮使。
 辛卯,李綱薨。甲辰,以顯謨閣直學士、提舉醴泉觀鄭億年復資政殿學士,奉朝請。



 二月戊申,命陜西復募蕃漢弓箭手。詔贓吏罪抵死,情犯甚者,奏取旨。辛亥,雨雹。以劉錡為東京副留守,李顯忠南京副留守。壬子,命兩宗正官各舉所知宗室二人。癸丑,展省試期一年。壬戌,詔新復州軍搜舉隱逸,諸路經理屯田。丁卯,罷史館,以日歷歸秘書省,置監修國史官。以孟庾知開封府,為東京留守;仇悆知河南府、西京留守。癸酉,罷吏部審量宣和
 濫賞。



 三月甲申,封閼伯為商丘宣明王。戊子,增印錢引五百萬緡,付宣撫司市軍儲。川、陜宣撫副使胡世將屢言金人必渝盟,宜為備。己丑,罷諸路增置稅場。韓世忠、張俊入見。始罷內教。復營建康行宮。丙申,蘇符自東京還。丁酉,命川、陜宣撫司軍事不及待報者,聽隨宜措置。己亥,以郭浩知永興軍兼節制陜西諸路軍馬,楊政徙知興元府。是月,命胡世將與夏人議入貢,夏人不報。



 夏四月丙午,訪求亡逸歷書及精於星歷者。辛酉,以張中
 孚為醴泉觀使,中彥提舉祐聖觀,趙彬為兵部侍郎。癸亥,命部使者歲舉廉吏一人。庚午,復四川諸州學官。壬申,韓肖冑罷。五月己卯,金人叛盟,兀□等分四道來攻。甲申,名徽宗御制閣曰敷文。乙酉,兀□入東京,留守孟庾以城降,知興仁府李師雄、知淮寧府李正民及河南諸州繼降。丙戌,金人陷拱州,守臣王慥死之。撒離曷自河中趨永興軍,陜西州縣官皆降。丁亥,金人陷南京,留守路允迪降。劉錡引兵至順昌府。己丑,金人陷西京,留
 守李利用、副總管孫暉皆棄城走,鈐轄李興率兵拒戰,不克。辛卯,胡世將自河池遣涇原經略使田晟以兵三千人迎敵金人。京、湖宣撫司忠義統領李寶敗金人於興仁府境上。癸巳,知亳州王彥先叛降於金。金人陷永興軍,趨鳳翔。丁酉,命胡世將移陜西之右護軍還屯蜀口。以福建、廣東盜起,命兩路監司出境共討。己亥,命劉光世為三京招撫處置使,以援劉錡。庚子,以吳璘同節制陜西諸路軍馬,聽胡世將便宜黜陟、處置軍事。辛丑,
 金人犯鳳翔府之石壁砦,吳璘遣統制姚仲等拒卻之。金人圍耀州,郭浩遣兵救之,金兵解去。壬寅,金人圍順昌府,三路都統葛王褒以大軍繼至,劉錡力戰,敗之。



 六月甲辰朔,以韓世忠太保,張俊少師,岳飛少保,並兼河南、北諸路招討使。乙巳,劉錡遣將閻充戰敗金人於順昌之李村。丙午,命兩浙、江東、福建諸州團結弓弩手。以仇悆為沿海制置使。詔將佐士卒能立奇功者,賞以使相節鉞官告,臨軍給受。丁未,罷建康府行宮營繕。戊申,
 以劉錡為沿淮制置使。己酉,吳璘遣統制李師顏等戰敗金人於扶風,拔之。壬子,兀□及宋叛將孔彥舟、酈瓊、趙榮等帥眾十餘萬攻順昌府,劉錡率將士殊死戰,大敗之。初,秦檜奏命錡擇利班師,錡不奉詔,戰益力,遂能以寡勝眾。乙卯,順昌圍解,兀□還。以知平江府梁汝嘉兼浙西沿海制置使。丙辰,岳飛將牛皋及金人戰於京西,敗之。己未,劉光世進軍和州。郭浩遣統制鄭建充攻破金人於醴州,復其城。壬戌,詔諸司錢物量留經費外,
 悉發以贍軍。樓照以父喪去位。甲子,撒離曷攻青溪嶺,鄜延經略使王彥率兵戰敗之,撒離曷還屯鳳翔。命士人褭主奉濮王祠事。張俊遣左護軍都統制王德援劉錡,德暫至順昌,值圍已解,復還廬州。遣司農少卿李若虛詣岳飛軍諭指班師,飛不聽。丙寅,下詔撫諭順昌府官吏兵民。庚午,以劉錡為武泰軍節度使、侍衛馬軍都虞候。韓世忠遣統制王勝、背嵬將成閔率兵至淮陽軍南,與金人遇,擊敗之。是月,金人圍慶陽府,權守臣宋萬年
 固守,金人不能下。岳飛領兵援劉錡,與金人戰於蔡州,敗之,復蔡州。



 閏月癸酉朔,張俊遣統制宋超敗金人於永城縣朱家村。甲戌,追孟庾、路允迪官,徙家屬遠郡。丙子,詔三衙管軍及觀察使已上,各舉智略勇猛、材堪將帥者二人。金人犯涇州,守臣曲汲棄城去,經略使田晟率兵來救,金人敗走。甲申,晟及金人再戰於涇州,敗之,金人引歸鳳翔。乙酉,降陜西雜犯死罪,釋流以下囚。丙戌,以胡世將為端明殿學士,吳璘為鎮西節度使,楊政
 武當節度使,郭浩奉國節度使。王德攻金人於宿州,夜破之,降其守馬秦。丁亥,詔釋順昌府流以下囚,再復租稅二年,守御官吏進官一等。己丑,永興軍鈐轄傅忠信等與金人戰於華陰縣,敗之。壬辰,岳飛遣統制張憲擊金將韓常於穎昌府,敗之,復穎昌。丙申,張憲復淮寧府。丁酉,趙鼎分司、興化軍居住。岳飛遣統制郝晸等與金人戰於鄭州北,復鄭州。李興復汝州,與金人戰於河清縣,敗之,復伊陽等八縣,李成遁去。韓世忠遣統制王勝、
 王權攻海州,克之,執其守王山。戊戌,張俊率統制宋超等及王德兵會於城父縣,酈瓊及葛王褒遁去,遂復亳州。己亥,金人救海州,王權等逆戰,敗之,復懷仁縣。庚子,張俊棄亳州,引軍還壽春。再貶趙鼎漳州居住,又貶清遠軍節度副使、潮州安置。



 秋七月癸卯,岳飛遣將張應、韓清入西京,會李興復永安軍。丙午,以御史中丞王次翁參知政事。己酉,岳飛及兀□戰於郾城縣,敗之。庚戌,曲赦海州。永興軍統領辛鎮及金人戰於長安城下,敗
 之。癸丑,以楊沂中為淮北宣撫副使,劉錡為判官。甲寅,岳飛遣統制楊再興、王蘭等擊金人於小商橋,皆戰死。乙卯,金人攻穎昌,岳飛遣將王貴、姚政合兵力戰,敗之。壬戌,飛以累奉詔班師,遂自郾城還,軍皆潰,金人追之不及。穎昌、蔡、鄭諸州皆復為金有。甲子,以釋奠文宣王為大祀。乙丑,增收州縣頭子錢為激賞費。金人圍淮寧府,趙秉淵棄城南歸。辛未,金人犯盭厔縣,王俊逆戰於東洛谷,卻之。



 八月壬申朔,以張九成、喻樗、陳剛中、凌景
 夏、樊光遠、毛叔度、元盥等七人嘗不主和議,皆降黜之。乙亥,韓世忠圍淮陽軍,不克。庚辰,金人及酈瓊合兵駐於千秋湖陵,韓世忠遣統制劉寶等夜襲破之。壬午,李成犯西京,李興擊卻之。楊沂中軍於宿州。丙戌,以郭浩知夔州。丁亥,楊沂中自宿州夜襲柳子鎮,軍潰,遂自壽春府渡淮歸,金人屠宿州。甲午,川、陜宣撫司統領王喜等遇金人於汧陽縣,敗之。



 九月壬寅朔,遣起居舍人李易諭韓世忠罷兵。時秦檜專主和議,諸大帥皆還鎮。丁
 未,楊政遣統制楊從儀夜襲金人於鳳翔府,敗之。戊申,金人復入西京,李興棄城去。庚戌,合祀天地於明堂,大赦。辛酉,臨安火。戊辰,以郭浩知金州,節制陜西、河東軍馬兼措置河東忠義軍。是秋,知代州王忠植舉兵復石、代等十一州。



 冬十月癸酉,復張浚觀文殿大學士。甲戌,以王忠植為建寧軍承宣使、河東路經略安撫使。戊寅,秦檜上《重修紹興在京通用敕令格式》。庚辰,金人犯慶陽府,守臣宋萬年以城降。辛卯,金人犯陜州,吳琦率兵
 迎擊,敗之。庚子,金人襲洮州,攻鐵城堡,統制孔文清、惠逢擊敗之。是月,劉錡入見。胡世將命王忠植救慶陽,叛將趙惟清執之降於金,忠植不屈而死。



 十一月丁未,金將合喜復犯陜州,吳琦擊卻之。又犯寶溪縣,統制楊從儀敗之。壬子,以令為保寧軍節度使。是月,宜章洞民駱科叛,犯桂陽、郴、道、連、賀諸州,命發大兵討之。十二月壬午,上皇太后冊寶於慈寧殿。丁亥,贈王忠植奉國軍節度使,謚義節。辛卯,起諸路耆長役錢隸總制司,專給
 軍用。是月,楊沂中引兵還行在。



 十一年春正月癸卯,鳳翔統制楊從儀敗金人於渭南。庚戌,張浚入見。乙卯,金人犯壽春府,守臣孫暉、統制雷仲合兵拒之。丁巳,壽春陷,暉、仲棄城去。己未,劉錡自太平州率兵二萬援淮西。庚申,金人渡淮。辛酉,雨雹。乙丑,劉錡至廬州還。丙寅,兀□陷廬州。戊辰,金人陷商州,守臣邵隆棄城去。己巳,命楊沂中引兵赴淮西,岳飛進兵江州。



 二月癸酉,張俊遣王德渡江,屯和州,金人退屯昭
 關。邵隆破金人於洪門,復商州。乙亥,金人復來爭和州,張俊敗之。命韓世忠以兵援淮西。丙子,趣岳飛會兵蘄、黃。王德等敗金人於含山縣東。己卯,統制關師古、李橫擊敗金人於巢縣,復之。庚辰,岳飛發鄂州。辛巳,知泰州王□奐兼通、泰二州制置使。癸未,王德、田師中等擊破金人,復含山縣,奪昭關。劉錡自東關擊敗金人於青溪。甲申,金人復犯昭關,王德等又敗之。李顯忠遣統領崔皋擊敗金人於舒城縣。丁亥,楊沂中、劉錡等大敗兀□軍
 於柘皋。己丑,兀□親率兵逆戰於店步,沂中等又敗之,乘勝逐北,遂復廬州。是月,虔、吉州盜賊悉平。



 三月庚子朔,張俊進鬻田及賣度牒錢六十三萬緡助軍用。壬寅,韓世忠引兵趨壽春。癸卯,復張俊特進。金人圍濠州。岳飛發舒州。甲辰,張俊、楊沂中、劉錡議班師,乙巳,沂中、錡先行,俊以輕兵留後。丙午,詔釋淮西雜犯死罪以下囚。丁未,金人陷濠州,執守臣王進,夷其城,鈐轄邵青死之。戊申,張俊遣楊沂中、王德入濠州,遇金伏兵,敗還。己酉,
 韓世忠至濠州,不利而退。辛亥,岳飛次定遠縣,聞金兵退,還屯舒州。楊沂中歸行在。壬子,金人渡淮北歸。癸丑,張俊歸建康府。丁巳,劉錡歸太平州。甲子,行營統制張彥及金人遇於汧陽之劉坊砦,第八將張宏戰沒。



 夏四月丙子,復收免行錢。己卯,孫近罷。辛巳,以王次翁兼權同知樞密院事。韓世忠、張俊、岳飛相繼入覲。壬辰,以世忠、俊並為樞密使,飛樞密副使,命三省、樞密院官復分班奏事。乙未,張俊請以所部兵隸御前。罷三宣撫司,改
 統制官為御前統制官,各屯駐舊所。丙申,以廣西經略使胡舜陟節制廣東、湖南兵,趣討駱科。慕容洧破新泉砦,又攻會州,將官朱勇破之。



 五月辛丑,置兩淮、江東西、湖廣京西三道總領軍馬錢糧官,仍掌報發御前軍馬文字。癸卯,賻恤戰沒將士。丁未,遣張俊、岳飛於楚州巡視邊防。召劉光世赴行在。甲寅,命樞密行府置司鎮江,令遍行巡歷措置。庚申,加楊沂中檢校少保、開府儀同三司。



 六月乙亥,造克敵弓。加秦檜特進,進尚書左僕射、同
 中書門下平章事兼樞密使。癸未,張俊、岳飛至楚州。俊以海州城不可守,毀之,遷其民,統韓世忠軍還鎮江,惟背嵬一軍赴行在。甲申,知河南府李興部兵至鄂州,以興為左軍統制。乙丑,明州僧王法恩等謀反,伏誅。壬辰,劉光世罷為萬壽觀使。



 秋七月戊戌,秦檜上《徽宗實錄》,進修撰以下各一官。庚子,以翰林學士範同參知政事。以旱,減膳祈禱,遣官決滯獄,出系囚。丁未,加秦檜少保。甲寅,罷劉錡兵,命知荊南府。乙卯,詔優獎永興、鳳翔、秦
 隴等州縣官,到任半年減磨勘,任滿遷一官。己未,加張俊太傅。癸亥,大雨。是月,命張俊復如鎮江措置軍務,留岳飛行在。



 八月戊辰,立祚德廟於臨安,祀韓厥。甲戌,罷岳飛。乙亥,命諸王後各推年長一人權主祀事。癸巳,胡世將起復。



 九月癸卯,命軍器少監鮑琚如鄂州根括宣撫司錢穀。鄂州前軍副統制王俊告副都統制張憲謀據襄陽為變,張俊收憲屬吏以聞。丁未,坐監司不按贓吏罪。辛亥,吳璘拔秦州,州將武誼降。壬子,璘率姚仲及
 金人戰於丁劉圈,敗之。楊政克隴州,破岐下諸屯。郭浩復華州,入陜州。甲寅,建康大火。丙申,遣劉光遠等充金國通問使。吳璘及金人戰於剡家灣,大敗之,遂圍臘家城。癸亥,璘自臘家城受詔班師,楊政、郭浩皆引軍還。乙丑,邵隆復虢州,郝晸討禽駱科,斬之。



 冬十月丙寅朔,金人陷泗州,遂陷楚州。丁卯,命樞密都承旨鄭剛中宣諭川、陜。戊辰,楊政及金人戰於寶雞縣,敗之,禽通檢孛堇。乙亥,兀□遣劉光遠等還。戊寅,詔修玉牒。下岳飛、張憲
 大理獄,命御史中丞何鑄、大理卿周三畏鞫之。壬午,遣魏良臣、王公亮為金國稟議使。乙酉,虛恨蠻主歷階詣嘉州降。癸巳,韓世忠罷為醴泉觀使,封福國公。是月,金人陷濠州,邵隆復陜州。



 十一月己亥,範同罷。責降李光為建寧軍節度副使、藤州安置。辛丑,兀□遣審議使蕭毅、邢具瞻與魏良臣等偕來。丁未,範同分司、筠州居住。罷判大宗正事士人褭、同知宗正事士撙,申嚴戚里宗室謁禁。己酉,雷。壬子,蕭毅等入見,始定議和盟誓。乙卯,以
 何鑄簽書樞密院事,充金國報謝進誓表使。庚申,命宰執及議誓撰文官告祭天地、宗廟、社稷。辛酉,以張浚為檢校少傅、崇信軍節度使、萬壽觀使。是月,與金國和議成,立盟書,約以淮水中流畫疆,割唐、鄧二州界之,歲奉銀二十五萬兩、絹二十五萬匹,休兵息民,各守境土。詔川、陜宣撫司毋出兵生事,招納叛亡。駱科餘黨歐幻四等復叛桂陽藍山,犯平陽縣,遣江西兵馬都監程師回討平之。十二月丁卯,責降徽猷閣待制劉洪道為濠州
 團練副使使、柳州安置。癸酉,命尚書省置籍勾考諸路滯獄。甲戌,罷川、陜宣撫司便宜行事。乙亥,兀□遣何鑄等如會寧見金主,且趣割陜西餘地。遂命周聿、莫將、鄭剛中分畫京西唐鄧、陜西地界。壬午,命州縣三歲一置產業簿,籍民貲財田宅以定賦役,禁受賕虧隱舊額。丁亥,立譏察海舶條法。癸巳,賜岳飛死於大理寺,斬其子云及張憲於市,家屬徙廣南,官屬於鵬等論罪有差。



\end{pinyinscope}