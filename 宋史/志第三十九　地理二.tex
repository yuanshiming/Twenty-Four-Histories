\article{志第三十九 地理二}

\begin{pinyinscope}

 河北路河東路



 河北路。舊分東、西兩路,後並為一路。熙寧六年,再分為兩路。



 東路。府三:大名,開德,河間。州十一:滄,冀,博,棣,莫,雄,霸,德,濱,恩,清。軍五:德清,保順,永靜,信安,保定。縣五十七。



 大名府,魏郡。慶歷二年,建為北京。八年,始置大名府路安撫使,統北京、澶懷衛德博濱棣、通利保順軍。熙寧以來並因之,六年,分屬河北東路。崇寧戶一十五萬五千二百五十
 三,口五十六萬八千九百七十六。貢花紬、綿紬、平紬、紫草。縣十二:元城,赤。熙寧六年,省大名縣為鎮入焉。



 莘,畿。



 大名,次赤。熙寧六年,省入元城。紹聖三年復。政和六年,徙治南樂鎮。



 內黃,畿。



 成安,畿。熙寧六年,省洹水縣為鎮入焉。



 魏,次畿。



 館陶,畿。熙寧五年,省永濟縣為鎮入焉,尋復舊。



 臨清,次畿。



 夏津,畿。



 清平,畿。宋初,自博州來隸。熙寧二年,又割博平
 縣
 明靈砦隸焉,本縣移置明靈。



 冠氏,畿



 宗城。畿。熙寧五年,省臨清縣為鎮入焉。當年復舊,尋以永濟隸臨清。熙寧六年,又省經城縣為鎮入焉。



 開德府,上,澶淵郡,鎮寧軍節度。本澶州。崇寧四年,建為北輔。五年,升為府。宣和二年,罷輔郡,仍隸河北東路。崇寧戶三萬一千八百七十八,口八萬二千八百二十六。貢莨莠席、南粉。縣七:濮陽,中。



 觀城,望。皇祐元年,省入濮陽、頓丘。四年,復置。



 臨河,緊。



 清豐,中。慶歷四年,徙清豐縣治德清軍,即縣置軍使,隸州。熙寧六年,省頓丘縣入清豐。



 衛南,中。



 朝城,畿。舊隸大名府,崇寧四年,與南樂來隸。



 南樂。畿。



 德清軍,見上。



 滄州,上,景城郡,橫海軍節度。崇寧戶六萬五千八百五十一,口一十一萬八千二百一十八。貢大絹、大柳箱。縣五:清池,望。熙寧四年,省饒安縣為鎮入清池。有乾符、巷姑、三女、泥姑、小南河五砦。政和二年,改巷姑曰海清,三女曰三河,泥姑曰河平。



 無棣,望。治平中,徙無棣縣治保順軍,即縣治置軍使,隸州。



 鹽山,緊。



 樂陵,緊。熙寧二年,徙治咸平鎮。



 南皮。中。熙寧六年,省臨津縣入焉。



 保順軍。周置軍於滄州無棣縣南三十里。開寶三年,又以滄棣二州界保順、吳橋二鎮之地益焉,仍棣滄州。



 冀州,上,信都郡,舊團練。慶歷八年,升安武軍節度。崇寧
 戶六萬六千二百四十四,口一十萬一千三十。貢絹。縣六:信都,望。



 蓨上。



 南宮,上。皇祐四年,升新河鎮為縣,廢南宮。六年,省新河為鎮入焉。棗強,上。熙寧元年,省為鎮入信都。十年,復置。



 武邑,上。



 衡水。中。



 河間府,上,河間郡,瀛海軍節度。至道三年,以高陽隸順安軍。舊名關南,太平興國元年,改名高陽關。慶歷八年,始置高陽關路安撫使,統瀛莫雄貝冀滄、永靜保定乾寧信安一十州軍。本瀛州,防禦。大觀二年升為府,賜軍額。崇寧戶三萬一千九百三十,口六萬二百六。貢絹。縣
 三:河間,望。雍熙中,即縣西置平虜砦,景德二年,改為肅寧城。



 樂壽,望。至道三年,自深州來隸。熙寧六年,省景城為鎮入焉。



 束城。上。熙寧六年,省為鎮入河間,元祐元年復。



 博州,上,博平郡,防禦。淳化三年,河決,移治於孝武渡西。崇寧戶四萬六千四百九十二,口九萬一千三百三十三。貢平絹。縣四:聊城,望。



 高唐,望。



 堂邑,望。



 博平。緊。熙寧二年,割明靈砦隸北京清平。



 棣州,上,樂安郡,防禦。建隆二年,升為團練,俄為防禦。大中祥符八年,移治陽信縣界八方寺。崇寧戶三萬九千
 一百三十七,口五萬七千二百三十四。貢絹。縣三:厭次,上。



 商河,中。



 陽信。下。



 莫州,上,文安郡,防禦。熙寧六年,省長豐縣為鎮,又省鄚縣入任丘。元祐二年,復鄚縣,尋又罷為鎮。崇寧戶一萬四千五百六十,口三萬一千九百九十二。貢綿。縣一:任丘。上。有馬村王家二砦。政和三年改馬村砦曰定安,王家砦曰定平。



 雄州,中,防禦。本唐涿州瓦橋關。政和三年,賜郡名曰易陽。崇寧戶一萬三千一十三,口五萬二千九百六十七。
 貢紬。縣二:歸信,中。有張家、木場、三橋、雙柳、大渦、七姑垣、紅城、新垣八砦。



 容城。中。建隆四年復置。



 霸州,中,防禦。本唐幽州永清縣地,後置益津關。周置霸州,以莫之文安、瀛州之大城來屬。政和三年,賜郡名曰永清。崇寧戶一萬五千九百一十八,口二萬一千五百一十六。貢絹。縣二:文安,上。景祐二年,廢永清縣入焉。有劉家渦、刁魚、莫金口、阿翁、雁頭、黎陽、喜渦、鹿角八砦。元豐四年,割鹿角砦隸信安軍。政和三年,改劉家渦砦曰安平,阿翁曰仁孝,雁頭曰和寧,喜渦曰喜安。



 大城。上。



 德州,上,平原郡,軍事。宋初,省歸化縣。景祐二年,廢安陵縣入將陵,後割屬永靜軍。熙寧六年,省德平縣為鎮,入安德。崇寧戶四萬四千五百九十一,口八萬二千二十五。貢絹。縣二:安德,望。



 平原。緊。



 濱州,上,軍事。大觀二年,賜渤海郡名。大中祥符五年,廢蒲臺縣。崇寧戶四萬九千九百九十一,口一十一萬四千九百八十四。貢絹。縣二:渤海,望



 招安。上。慶歷三年,升招安鎮為縣。熙寧六年,省為鎮入渤海。元豐二年復為縣。



 恩州,下,清河郡,軍事。唐貝州,晉永清軍節度,周為防禦。宋初,復為節度。慶歷八年,改州名,罷節度。崇寧戶五萬一千三百四十二,口八萬五千九百八十六。貢絹、白毯。縣三:清河,望。端拱元年,徙治永寧鎮。淳化五年徙今治。熙寧四年,省清陽縣入焉。



 武城,望。



 歷亭。緊。至和元年,廢漳南縣入焉。



 永靜軍,同下州。唐景州。太平興國六年,以軍直屬京。淳化元年,以冀州阜城來屬。景德元年,改軍名。崇寧戶三萬四千一百九十三,口三萬九千二十二。貢簟、絹。縣三:
 東光,緊。



 將陵。望。景祐元年,移治於長河鎮。



 阜城。中。嘉祐八年,省為鎮入東光。熙寧十年復為縣。



 清州,下,本乾寧軍。幽州蘆臺軍之地,晉陷契丹。周平三關,置永安縣,屬滄州。太平興國七年置軍,改縣曰乾寧隸焉。大觀二年,升為州。政和三年,賜郡名曰乾寧。崇寧戶六千六百一十九,口一萬二千七十八。貢絹。縣一:乾寧。熙寧六年省為鎮。元符二年復。崇寧三年再省。政和五年又復。



 砦六。釣臺、獨流北、獨流東、當城、沙渦、百萬。



 信安軍,同下州。太平興國六年,以霸州淤口砦建破虜軍。景德二年,改為信安。崇寧戶七百一十五,口一千四百三十七。貢絹。砦七。周河、刁魚、田家、狼城、佛聖渦、鹿角、李詳。元豐四年,霸州鹿角砦始隸軍。



 保定軍,同下州。太平興國六年,以涿州新鎮建平戎軍。景德元年,改為保定軍。景祐元年,析霸州文安、大城二縣五百戶隸軍。宣和七年,廢保定軍為保定縣,隸莫州,知縣事仍兼軍使,尋依舊。崇寧戶一千二十九,口二千
 四百八十四。貢絁。砦二。桃花、父母。政和三年,改父母砦曰安寧。



 西路。府四:真定,中山,信德,慶源。州九:相,浚,懷,衛,洺,深,磁,祁,保。軍六:天威,北平,安肅,永寧,廣信,順安。縣六十五。



 真定府,次府,常山郡,唐成德軍節度。本鎮州,漢以趙州之元氏、欒城二縣來屬。開寶六年,廢九門、石邑二縣。端拱初,以鼓城隸祁州。淳化元年,以束鹿隸深州。慶歷八年,初置真定府路安撫使,統真定府、磁相邢趙洺六州。崇寧戶九萬二千三百五十三,口一十六萬三千一
 百九十七。貢羅。縣九:真定,次畿。



 槁城,次畿。



 欒城,次畿。



 元氏,次畿。



 井陘,次畿。熙寧六年,省入獲鹿、平山。八年復置,徙治天威軍,即縣治置軍使,隸府。有天威軍、小作口、王家穀三砦。



 獲鹿,次畿。



 平山,次畿。有甘泉、嵐州、沂州、檀明、夫婦、柏嶺、黃岡、洪山、赤箭、抱兒、石虎、中子、雕□共、東臨山、西臨山十五砦。



 行唐,次畿。靈壽。次畿。熙寧六年,省為鎮入行唐。八年復。有赤陘、飛吳二砦。



 砦一:北砦。咸平二年置。熙寧八年,析行唐縣二鄉隸砦。



 天威軍。見上。



 相州,望,鄴郡,彰德軍節度。崇寧戶三萬六千三百四十,口七萬一千六百三十五。貢暗花牡丹花紗、知母、胡粉、
 絹。縣四:安陽,緊。熙寧五年,省永和縣入焉。



 湯陰,緊。宣和二年,以湯陰縣隸浚州,尋復來隸。



 臨漳,緊。熙寧五年,省鄴縣入焉。



 林慮。中。



 中山府,次府,博陵郡。建隆元年,以易北平並來屬。太平興國初,改定武軍節度。本定州,慶歷八年,始置定州路安撫使,統定、保、深、祁、廣信、安肅、順安、永寧八州。政和三年,升為府,改賜郡名曰中山。崇寧戶六萬五千九百三十五,口一十八萬六千三百五。貢羅、大花綾。縣七:安喜,緊。



 無極,緊。曲陽,上。



 唐,上。



 望都,中。



 新樂,中。



 北平。中下。



 砦
 一,軍城。隸曲陽縣。



 北平軍。慶歷二年,以北平砦建軍。四年復隸州,即北平縣治置軍使,隸州。



 信德府,次府,鉅鹿郡。後唐安國軍節度。本邢州。宣和元年,升為府。崇寧戶五萬三千六百一十三,口九萬五千五百五十二。貢絹、白磁盞、解玉砂。縣八:邢臺,上。宣和二年,改龍岡縣為邢臺。



 沙河,上。



 任,中。



 堯山,中。



 平鄉,上。熙寧六年,省平鄉縣為鎮入鉅鹿。元祐元年復。



 內丘,上。熙寧六年,省堯山縣入焉,元祐元年復。



 南和,中。熙寧五年,省任縣為鎮入焉。元祐元年復。



 鉅鹿。上。



 浚州,平川軍節度。本通利軍。端拱元年,以滑州黎陽縣
 為軍。天聖元年,改通利為安利。四年,以衛州衛縣隸軍。熙寧三年廢為縣,隸衛州。元祐元年復為軍。政和五年升為州,號浚川軍節度,改今額。崇寧戶三千一百七十六,口三千二百二。縣二:衛,上。熙寧六年,廢為鎮入黎陽。後復。



 黎陽。中。



 懷州,雄,河內郡,防禦。建隆元年,升為團練,俄為防禦。崇寧戶三萬二千三百一十一,口八萬八千一百八十五。貢牛膝、皂角。縣三:河內,緊。熙寧六年,省武德縣為鎮入焉。



 修武,上。熙寧六年,省為鎮入武陟。元祐元年復。



 武陟。
 中。



 衛州,望,汲郡,防禦。崇寧戶三萬三千二百四,口四萬六千三百六十五。貢絹、綿。縣四:汲,中。



 新鄉,緊,熙寧六年,廢為鎮入汲。元祐二年復。獲嘉,上。天聖四年,自懷州來隸。



 共城。中。



 監一:黎陽。熙寧七年置,鑄銅錢。



 洺州,望,廣平郡,建隆元年,升為防禦。熙寧三年,省曲周縣為鎮,入雞澤。六年,省臨洺縣為鎮,入永年。元祐二年,曲周、臨洺復為縣,尋復為鎮。四年,曲周、雞澤依舊別為兩縣。崇寧戶三萬八千八百一十七,口七萬三千六百。貢紬。縣五:永年,上。



 肥鄉,望。



 平恩,緊。



 雞澤,中。



 曲周。中。



 深州,望,饒陽郡,防禦。雍熙四年,廢陸澤縣。崇寧戶三萬八千三十六,口八萬三千七百一十。貢絹。縣五:靜安,望。本漢下博縣,周置靜安軍,以縣隸,俄復焉。太平興國七年,又隸靜安軍。雍熙二年軍廢,還屬,三年廢,四年復置,改今名。



 束鹿,望。淳化中,自真定來屬。



 安平,望。



 饒陽,望。武強。望。



 磁州,上,滏陽郡,團練。舊名慈,政和三年改作磁。崇寧戶三萬六千四百九十一,口九萬六千九百二十二。貢磁石。縣三:滏陽,上。熙寧六年,省昭德縣為鎮入焉。



 邯鄲,上。



 武安。上。有固鎮、永安、黃澤、海回四砦。



 祁州,中,蒲陰郡,團練。端拱初,以鎮州鼓城來屬。景德元年,移治於定州蒲陰,以無極隸定州。熙寧六年,省深澤縣為鎮,入鼓城。元祐元年復。崇寧戶二萬四千四百八十四,口四萬九千九百七十五。貢花絁。縣三:蒲陰,望。



 鼓城,緊。



 深澤。中。



 慶源府,望,趙郡,慶源軍節度。本趙州,軍事。大觀三年,升為大藩。崇寧四年,賜軍額。宣和元年,升為府。崇寧戶三萬四千一百四十一,口六萬一百三十七。貢絹、綿。縣七:
 平棘,望。



 寧晉,望。



 臨城,上。唐縣。熙寧六年,省隆平縣為鎮入焉,元祐元年復。



 高邑,中。熙寧五年,省柏鄉、贊皇二縣為鎮入焉,元祐元年皆復。



 隆平,中。



 柏鄉,中。



 贊皇。下。



 保州,下,軍事。本莫州清苑縣。建隆初,置保塞軍。太平興國六年,建為州,政和三年,賜郡名曰清苑。崇寧戶二萬七千四百五十六,口二十三萬二百三十四。貢絹。縣一:保塞。望,太平興國六年,析易州滿城之南境入焉。



 安肅軍,同下州。本易州遂城縣。太平興國六年,建為靜戎軍,析易州遂城三鄉置靜戎縣隸焉。景德元年並縣,
 改安肅軍。宣和七年,廢軍為安肅縣。知縣事仍兼軍使,尋依舊。崇寧戶七千一百九十七,口一萬四千七百五十一。貢素絲□。縣一:安肅。中。



 永寧軍,同下州。雍熙四年,以定州博野縣建寧邊軍。景德元年,改永寧軍。宣和七年,廢為博野縣,知縣事仍兼軍使,尋依舊。縣一:博野。望。



 廣信軍,同下州。太平興國六年,改易州遂城縣為威勇軍。景德元年,改廣信軍。崇寧戶四千四百四十五,口八
 千七百三十八。貢紬、慄。縣一:遂城。中。



 順安軍,同下州。本瀛州高陽關砦。太平興國七年,置唐興砦。淳化三年,建為順安軍。至道三年,以瀛州高陽來屬。熙寧六年,省高陽縣為鎮。十年,復為縣。崇寧戶八千六百五,口一萬六千五百七十八。貢絹。縣一:高陽。中。



 河北路,蓋《禹貢》兗、冀、青三州之域,而冀、兗為多。當畢、昴、室、東壁、尾、箕之分。南濱大河,北際幽、朔,東瀕海岱,西壓上黨。繭絲、織絲任之所出。人性質厚少文,多專經術,大率
 氣勇尚義,號為強忮。土平而近邊,習尚戰鬥。有河漕以實邊用,商賈貿遷,芻粟峙積。宋初募置鄉義,大修戰備,為三關,置方田以資軍廩。契丹數來侵擾,人多去本,及薦修戎好,益開互市,而流庸復來歸矣。大名、澶淵、安陽、臨洺、汲郡之地,頗雜斥鹵,宜於畜牧。浮陽際海,多鬻鹽之利。其控帶北地,鎮、魏、中山皆為雄鎮云。



 河東路。府三:太原,隆德,平陽。州十四:絳,澤,代,忻,汾,遼,憲,嵐,石,隰,慈,麟,府,豐。軍八:慶祚,威勝,平定,岢嵐,寧化,火山,
 保德,晉寧。縣八十一。



 太原府,太原郡,河東節度。太平興國四年,平劉繼元,降為緊州,軍事,毀其城,移治於榆次縣。又廢太原縣,以平定、樂平二縣屬平定軍,交城屬大通監。七年,移治唐明監。舊領河東路經略安撫使。元豐為次府,大觀元年升大都督府。崇寧戶一十五萬五千二百六十三,口一百二十四萬一千七百六十八。貢大銅鑒、甘草、人參、礬石。縣十:陽曲,次赤。有百井、陽興二砦。



 太谷,次畿。



 榆次,次畿。



 壽陽,次畿。



 盂,次畿。



 交
 城。次畿。寶元二年,自大通監來隸。



 文水,次畿。



 祁,次畿。清源,次畿。



 平晉。中。熙寧三年,廢入陽曲。政和五年復。



 監二:大通,永利。



 隆德府,大都督府,上黨郡,昭義軍節度。太平興國初,改昭德。舊領河東路兵馬鈐轄,兼提舉澤晉絳州、威勝軍屯駐泊本城兵馬巡檢事。本潞州。建中靖國元年,改為軍。崇寧三年,升為府,仍還昭德舊節。崇寧戶五萬二千九百九十七,口一十三萬三千一百四十六。貢人參、蜜、墨。縣八:上黨,望。



 屯留,上。襄垣,上。



 潞城,上。



 壺關,中。



 長子,中。



 涉,中。



 黎城。中。天聖三年,徙治涉之東南白馬驛。熙寧五年,省入潞城縣。元祐元年復。



 平陽府,望,平陽郡,建雄軍節度。本晉州,政和六年,升為府。崇寧戶七萬五千九百八,口一十八萬五千二百五十四。貢蜜、蠟燭。縣十:臨汾,望。



 洪洞,緊。



 襄陵,緊。熙寧五年,廢慈州鄉寧縣分隸焉。有雕掌、豹尾二砦。神山,上。有韓買、安國、史壁、疊頭等堡。



 趙城,上。熙寧五年,省為鎮隸洪洞。元豐三年復為縣。



 汾西,中。有厚裔、青岸、石橋、青山、邊柏五砦。



 霍邑,中。



 冀氏,中。有府城、永興二砦,陶川、白練、當穀、橫嶺四堡。



 岳陽,中下。



 和川。中下。太平興國六年,廢沁州,以縣來屬。熙寧五年,省為鎮入冀氏。元祐元年復為縣。



 務二:煉礬、礬山。



 慶祚軍。政和三年,以趙城造父始封之地升為軍,以軍事領之。



 絳州,雄,絳郡,防禦。崇寧戶五萬九千九百三,口九萬四千二百三十七。貢防風、蠟燭、墨。縣七:正平,望。



 曲沃,望。



 太平,望。熙寧五年,廢慈州,以鄉寧縣分隸太平、稷山。



 翼城,上。



 稷山,中。



 絳,中。有中山、花崖、華山三砦。



 垣曲。下。有銅錢一監。



 澤州,上,高平郡。崇寧戶四萬四千一百三十三,口九萬一千八百五十二。貢白石英、禹餘糧、人參。縣六:晉城,緊。



 高平,上。



 陽城,上。



 端氏,中。



 陵川中。



 沁水。中下。



 關一:雄定。舊天井
 關,屬晉城縣。靖康元年改今名。



 代州,上,雁門郡,防禦。景德二年,廢唐林縣。舊置沿邊安撫司。崇寧戶三萬三千二百五十八,口一十五萬九千八百五十七。貢麝香、青、碌。縣四:雁門,中下。有西陘、胡谷,雁門三砦。



 崞,中下。有樓板、陽武、石峽、土墱四砦。



 五臺,中下。



 繁畤。下。有繁畤、茹越、大石、義興冶、寶興軍、瓶形、梅回、麻穀八砦。



 忻州,下,定襄郡,團練。崇寧戶一萬八千一百八十六,口四萬二千二百三十二。貢解玉砂、麝。縣二:秀容,緊。熙寧五年,省
 定襄入焉。元祐元年,定襄復為縣。有石嶺關、忻口、雲內、徒合四砦。



 定襄。中下。



 汾州,望,西河郡,軍事。崇寧戶五萬一千六百九十七,口一十八萬五千六百九十八。貢土絁、石膏。縣五:西河,望。有永利西監。



 平遙,望。



 介休,上。



 靈石,中。有陽涼南關、陽涼北關。



 孝義。上。太平興國元年,改為中陽,後復為孝義。熙寧五年,省為鎮入介休。元祐元年復。



 遼州,下,樂平郡。熙寧七年州廢,省平城、和順二縣為鎮,入遼山縣,隸平定軍;省榆社縣為鎮,入威勝軍武鄉縣。元豐八年,復置州,縣鎮並復來隸。元祐元年,復置和順、
 榆社、平城縣。崇寧戶七千三百一十五。貢人參。縣四:遼山,下。有黃澤砦。



 和順,下。



 榆社,中下。



 平城。中。



 憲州,中,汾源郡,軍事。初治樓煩,咸平五年,移治靜樂軍、縣,遂廢軍,又廢,樓煩改隸嵐州。熙寧三年,廢憲州,以靜樂縣隸嵐州。十年,復憲州,仍領靜樂縣。政和五年,賜郡名。崇寧戶二千七百二十二,口七千四百四十四。貢麝香。縣一:靜樂。中。咸平五年,廢天池、玄池二縣入焉。



 嵐州,下,昌煩郡,軍事。太平興國五年,以嵐谷隸岢嵐軍。
 崇寧戶一萬三千二百六十九,口六萬六千二百二十四。貢麝香。縣三:宜芳,中。有飛鳶堡。



 合河,中下。有乳浪砦。



 樓煩。下。咸平五年,自憲州來隸。



 石州,下,昌化郡,軍事。舊帶嵐、石、隰三州都巡檢使。元豐五年,置葭蘆、吳堡二砦隸州,因置二砦沿邊都巡檢使,遂令三州各帶沿邊都巡檢使。初領縣五,元符二年,升葭蘆砦為晉寧軍,以州之臨泉縣隸焉。大觀三年,復以定胡縣隸晉寧軍。崇寧戶一萬五千八百九,口七萬二
 千九百二十九。貢蜜、蠟。縣三:離石,中。



 平夷,中。有伏落津砦。



 方山。下



 隰州,下,大寧郡,團練。熙寧五年,廢慈州,以吉鄉縣隸州,即縣治置吉鄉軍使,仍省文城縣為鎮隸焉。元祐元年,復慈州。七年,以州之上平、永寧兩關俯逼西界,以州為次邊。崇寧戶三萬八千二百八十四,口一十三萬八千四百三十九。貢蜜、蠟。縣六:隰川,上



 溫泉,上。有碌礬一務,水頭、白壁、先鋒三砦。



 蒲,中。



 大寧,中。



 石樓,中。有上平、永寧二砦。



 永和。中。



 慈州,下,團練。舊領吉鄉、文城、鄉寧三縣。熙寧五年廢州,以吉鄉隸隰州,即縣治置吉鄉軍使,仍省文城為鎮,隸焉。又以鄉寧隸晉州襄陵縣。元祐元年,復吉鄉軍為慈州。戶口闕。縣一:吉鄉。中。



 麟州,下,新秦郡。乾德初,移治吳兒堡。五年,升建寧軍節度。端拱初,改鎮西軍節度。崇寧戶三千四百八十二,口八千六百八十四。貢柴胡。縣一:新秦。上。政和四年,廢銀城、連穀二縣入焉。有神堂、靜羌二砦,惠寧、鎮川二堡。銀城有屈野川,五原塞,銀城、神木、建寧三砦,肅定、神木、通津、闌乾四堡。連谷
 有屈野川、橫陽堡。



 大和砦,地名大和谷,元符二年進築,賜名。東至神木砦五十五里,南至彌川砦三十里,西至饒咩浪界堠七十里,北至清水穀二十里。



 大和堡。地名麻乜娘,元符二年進築,賜名。東至肅定堡界二十五里,南至清水穀二十里,西至松木骨堆界六十五里,北至銀城砦二十五里。



 府州,中,靖康軍節度。本永安軍,崇寧元年,改軍額。政和五年,賜郡名曰榮河。舊置麟府路軍馬司,以太原府代州路鈐轄領之。崇寧戶一千二百四十二,口三千一百八十五。貢甘草。縣一:府穀。下。有安豐、寧府、百勝三砦,河濱、斥堠、靖安、西安四堡。



 寧川堡,府州安豐砦外第九砦,元符元年賜名。東至斥堠堡三十五里,南至安豐砦界四十五里,西至豐州
 寧豐砦四十里,北至青沒怒川界堠一百五十里。



 寧邊砦,地名端正平,元符二年進築,賜今名。東至寧府砦界三十里,南至靖化堡界三十里,西至吳厖烽一十五里,北至保寧砦界三十里。



 寧疆堡,宣和六年,獨移莊嶺建堡,賜名寧疆。



 震威城,宣和六年,鐵爐骨堆建砦,賜名。



 豐州,下。慶歷元年,元昊攻陷州地。嘉祐七年,以府州蘿泊川掌地復建為州。今軍事。政和五年,賜郡名寧豐。崇寧戶一百五十三,口四百一十一。貢甘草、柴胡。砦二:永安,保寧。



 威勝軍,同下州。太平興國三年,於潞州銅鞮縣亂柳石
 圍中建為軍。崇寧戶一萬九千九百六十二,口三萬七千七百二十六。貢土絁。縣四:銅鞮,中。太平興國初,與武鄉自潞州來隸。



 武鄉,上。熙寧七年廢遼州,以榆社縣為鎮入焉。元豐八年,復置遼州,以榆社往隸。



 沁源,中下。太平興國六年廢沁州。以縣來隸。



 綿上。中下。寶元二年,自大通監來隸。慶歷六年,徙治軍西北大覺寺地。



 平定軍,同下州。太平興國二年,以鎮州廣陽砦建為軍。四年,以並州平定、樂平二縣來屬。崇寧戶九千三百六,口二萬八千六百七。貢絹。縣二:平定,中。唐廣陽縣,太平興國四年改。有故井陘關、百井砦。



 樂平。中。有靜
 陽砦。



 岢嵐軍,同下州。太平興國五年,以嵐州嵐谷縣建為軍。崇寧戶二千九百一十七,口六千七百二十。貢絹。縣一:嵐谷。下。熙寧三年廢,元豐六年復置。有永和、洪穀等六砦。



 寧化軍,同下州。崇寧戶一千七百一十八,口三千八百二十一。貢絹。縣一:寧化。熙寧三年廢,元祐元年復,崇寧三年又廢為鎮。有西陽、腦子、細腰、窟谷四砦。



 火山軍,同下州。本嵐州之地。太平興國七年,建為軍。治平四年,置火山縣,熙寧四年,廢之。崇寧戶五千四十五,口九
 千四百八十。貢柴胡。砦一:下鎮。火山軍舊領雄勇、偏頭、董家、橫谷、桔槔、護水六砦。慶歷初,置下鎮砦。嘉祐六年,廢偏頭砦。熙寧元年,廢桔槔砦。《元豐九域志》:領砦一。



 保德軍,同下州。淳化四年,析嵐州地置定羌軍。景德元年改。崇寧戶九百六十三,口四千五十。貢絹。津二:大堡、沙谷。



 晉寧軍,本西界葭蘆砦。元豐五年收復,六月,並吳堡砦並隸石州。元祐四年,以葭蘆砦給賜西人。紹聖四年收復。元符二年,以葭蘆砦為晉寧軍,割石州之臨泉隸焉。知軍領嵐石路沿邊安撫使,兼嵐、石、隰州都巡檢使。大
 觀三年,復以石州定胡縣來隸。東至克胡砦隔河五里,南至吳堡砦一百七十里,西至神泉砦二十五里,北至通秦砦二十里。領縣二:定胡,中。舊領定胡、天渾津、吳堡三砦。按吳堡砦元豐四年收復,東至黃河,南至綏德軍白草砦九十里,西至綏德軍義合砦六十里,北至晉寧軍一百七十里。



 臨泉。中下。舊領克胡、葭蘆二砦。按葭蘆砦乃元豐五年收復,後為晉寧軍。



 神泉砦,地名榆木川,在廢葭蘆砦北。元符元年賜今名。東至晉寧軍二十五里,南至烏龍砦一十五里,西至隔祚嶺界堠五十里,北至通秦砦四十里。



 三交堡,地名三交川嶺。元符元年,神泉砦築堡畢
 工,賜名。



 烏龍砦,元符二年進築,賜名。東至神泉砦二十五里,南至暖泉砦二十里,西至暖泉砦三十里,北至女萌烽一十七里。



 通秦砦,地名升囉嶺,元符二年賜今名。東至黃河二十九里,南至神泉砦四十二里,西至女萌骨堆界堠五十里,北至通秦堡一十七里。



 寧河砦,地名窟薛嶺,元符二年賜名。東至黃河三十里,南至通秦堡一十七里,西至尹遇合一十三里,北至章堡二十五里。



 彌川砦,地名彌勒川,元符二年賜名。東至黃河六十里,南至彌川堡十五里,西至砦浪骨堆界堆七十里,北至麟州大和砦三十里。



 通秦堡,地名精移堡,元符二年,同砦賜名。東至黃河一十七里一百二十步,南至通秦砦一十七里,西至龍移川界堠五十里,北至寧河砦一十一里。



 寧河堡,地名哥崖嶺,元符二年,同砦賜名。



 彌川堡,地名小紅崖,元符二年,同砦賜名。東至黃河四十里,南至寧河砦一十五里,西至祖平四十里,北至秦平堡一十里。



 靖川堡。東至黃河三十里,南至寧河砦十四里,西至界首立子穀四十五里,北至彌川堡一十三里。



 河東路,蓋《禹貢》冀、雍二州之域,而冀州為多。當觜、參之分。其地東際常山,西控黨項,南盡晉、絳,北控雲、朔,當太行之險地,有鹽、鐵之饒。其俗剛悍而樸直,勤農織之事業,寡桑柘而富麻苧。善治生,多藏蓄,其靳嗇尤甚。朔方、樓煩,馬之所出,歲增貿市以充監牧之用。太宗平太原,慮其恃險,徙州治焉。然猶為重鎮,屯精兵以控邊部云。



\end{pinyinscope}