\article{志第七十四 禮二十四(軍禮)}

\begin{pinyinscope}

 榪
 祭閱武受降獻俘田獵打球救日伐鼓



 榪,師祭也,宜居軍禮之首。講武次之,受降、獻俘又次之。田獵以下,亦各以類附焉。



 軍前大旗曰牙,師出必祭,謂
 之榪。後魏出師,又建纛頭旗上。太宗征河東,出京前一日,遣右贊善大夫潘慎修出郊,用少牢一祭蚩尤、榪牙;遣著作佐郎李巨源即北郊望氣壇用香、柳枝、燈油、乳粥、酥蜜餅、果,祭北方天王。



 咸平中,詔太常禮院定榪儀。所司除地為壇,兩壝繞以青繩,張幄帟,置軍牙、六纛位版。版方七寸,厚和三分。祭用剛日,具饌。牲用大牢,以羊豕代。其幣長一丈八尺,軍牙以白,六纛以皂。都部署初獻,副都部署亞獻,部署三獻,皆戎服,清齋一宿。將校陪
 位。禮畢焚幣,釁鼓以一牢。又擇日祭馬祖、馬社。



 閱武,仍前代制。太祖、太宗征伐四方,親講武事,故不盡用定儀,亦不常其處。鑿講武池朱明門外以習水戰。復築講武臺城西楊村,秋九月大閱,與從臣登臺觀焉。



 真宗詔有司擇地含輝門外之東武村為廣場,馮高為臺,臺上設屋,構行宮。其夜三鼓,殿前、侍衛馬步諸軍分出諸門。詰旦,帝乘馬,從官並戎服,賜以窄袍。至行宮,諸軍陣臺前,左右相向,步騎交屬亙二十里,諸班衛士翼從
 於後。有司奏成列,帝升臺東向,御戎帳,召從臣坐觀之。殿前都指揮使王超執五方旗以節進退,又於兩陣中起候臺相望,使人執旗如臺上之數以相應。初舉黃旗,諸軍旅拜。舉赤旗則騎進,舉青旗則步進。每旗動則鼓駴士噪,聲震百里外,皆三挑乃退。次舉白旗,諸軍復再拜,呼萬歲。有司奏陣堅而整,士勇而厲,欲再舉,詔止之,遂舉黑旗以振旅。軍於左者略右陣以還,由臺前出西北隅;軍於右者略左陣以還,由臺前出西南隅,並凱旋
 以退。乃召從臣宴,教坊奏樂。回禦東華門,閱諸軍還營,鈞容奏樂於樓下,復召從臣坐,賜飲。明日,又賜近臣飲於中書,諸軍將校飲於營中,內職飲於軍器庫,諸班衛士飲於殿門外。



 神宗閱左藏庫副使幵斌所教牌手於崇政殿,乃命殿前步軍司擇驍健者依法教習。自是,營屯及更戍諸軍、畿甸三路民兵皆隨伎藝召見親閱焉。凡閱試禁衛、戍軍、民兵,總率第其精觕,賜以金帛;而超等高者,至命為吏選官,其典領者優加職秩。涇原經略
 蔡挺肄習諸將軍馬,點閱周悉,隊伍有法,入為樞密副使,因言於上而引試之。舊以七軍營陣校試,而分數不齊,前後抵牾。命校試官採掇定為八軍法。及軍法成,頒行諸路。既又定九軍法,以一軍營陣,即城南好草披閱之,皆有賞賚。其按閱炮場連弩及便坐日閱召募新軍時,令習戰如故事。



 建炎三年六月,高宗諭輔臣曰:「朕欲親閱武。」宰臣呂頤浩曰:「方右武之時,理當如此。祖宗時不忘武備,如鑿金明池,益欲習水戰。」張浚曰:「祖宗每上
 巳游,必命衛士馳射,因而激賞,亦所以講武也。」帝曰:「朕非久命諸將各閱所部人馬,當召卿等共觀,足以知諸將能否。」後以巡幸,不果行。



 紹興五年正月,始御射殿,閱諸班直殿前司諸軍指教使臣、親從宿衛親兵並提轄部押親兵使臣射射。共一千二百六十人,每六十人作一撥。遂詔戶部支金千兩,付樞密院激賞庫充犒用。三月,御射殿,閱等子趙青等五十人角力,轉資,支賜錢銀有差。八月,御射殿,閱廣東路經略司解發到韶州士
 庶子弟陳裕試神臂弓,特補進武校尉,賜紫羅窄衫、銀束帶,差充本路經略司指使。十四年十一月,閱殿前馬步軍將士藝精者,賞有差。自是,歲以冬月行之,號曰冬教。三十年十月,御射殿,引三衙統制、同統制、統領、同統領入內射射,詔餘合赴內殿教人,依年例支降例物,令逐司自行按試等第給散舊例,每歲引三衙官兵教。是日,止引統制、統領,故有是詔。三十二年四月二十五日,御射殿隔門特坐,引呈新舊行門射射。



 乾道二年十一月,幸候潮門外大教場,次幸
 白石教場。應從駕臣僚,自祥曦殿並戎服起居,從駕往回。內管軍、御帶、環衛官從駕,宰執以下免從。就逐幕次賜食,俟進晚膳畢,免奏萬福,並免茶,從駕還內。二十四日,幸候潮門外大教場,進早膳,次幸白石教場閱兵。三衙率將佐等導駕詣白石,皇帝登臺,三衙統制、統領官等起居畢,舉黃旗,諸軍皆三呼萬歲拜訖,三衙管軍奏報取旨,馬軍上馬打圍教場。舉白旗,三司馬軍首尾相接;舉紅旗,向臺合圍,聽一金止。軍馬各就圍地,作圓形
 排立。射生官兵隨鼓聲出馬射獐兔,一金止。疊金,射生官兵各歸陣隊。舉黃旗,射生官兵就御臺下獻所獲。帝遂慰勞,賜賚諸將鞍馬金帶,以及士卒。諸軍歡騰,鼓舞就列。百姓觀者如山。時久陰曀,暨帝出郊,雲霧解駁,風日開霽。帝遣諭主管殿前司王琪等曰:「前日之教,師律整嚴,人無嘩囂,分合應度,朕甚悅之,皆卿等力也。」琪等曰:「此陛下神武之化,六軍恭謹所致。臣願得以此為陛下剿絕奸宄。」



 四年十月,殿前司言:「相視龍王堂北、江岸
 以東茅灘一帶平地,可作教場。已修築將壇,將來三司馬步軍並各全裝,披帶衣甲,執色器械,至日,先赴教場下方營排辦,俟駕登臺,聽金鼓起居畢,依資次變陣教閱。所有聖駕出郊,除禁衛外,欲於本司入陣馬軍內摘差護聖馬軍八百人騎、弓箭、器械,作十六隊,於儀衛前後引從,各分八隊,隊各五十人,往回沿路,各奏隨軍鼓笛大樂。及摘差本司入教陣隊內諸軍步親隨一千人,並統領將官三員,至日,先赴將臺下,各分左右,於後壁周
 圍留空地三十步,以容禁衛,外作三重環立。」十六日。車駕至灘上。諸軍人馬,前一日於教場東列幕宿營。是日,三衙管軍與各軍統領將佐導駕乘馬至護聖步軍大教場亭,更御甲冑至灘上。皇帝登臺,三衙起居畢,權主管殿前司王逵奏三司人馬齊,舉黃旗,諸軍呼拜者三。逵奏請從頭教。中軍鳴角,倒門角旗出營,馬步軍簇隊成,收鼓訖。連三鼓,馬軍上馬,步人撮起旗槍。四鼓舉白旗,中軍鼓聲旗應,變方陣為備敵之形。別高一鼓,步軍
 四向作御敵之勢,且戰且前,馬軍出陳作戰鬥之勢。別高一鼓,各分歸地分。五鼓舉黃旗,變圓陣為自環內固之形。如前節次訖。三鼓舉赤旗,變銳陣,諸軍相屬,魚貫斜列,前利後張,為沖敵之形。亦依前節次訖。王逵奏人馬教絕,取旨。舉青旗,變放教直陣,收鼓訖,一金止。重鼓三,馬軍下馬,步人齪落旗槍,皆應規矩。帝大悅,犒賞倍之。士卒歡呼謝恩如儀。鳴角聲簇隊訖,放教拽隊。步人分東西引拽,馬軍交頭於御臺下,隨隊呈試驍銳大刀
 武藝,繼而進呈車炮、火炮、煙槍。及赭山打圍射生,馬步軍統制官蕭鷓巴以所獲獐鹿等就御臺下進獻,人馬拽絕。皇帝復御常服,乘馬至車子院,宣喚殿前司撥發官馬定遠、侯彥昌各賜馬一匹,彥昌仍自準備將特升副將。進御酒,上謂王逵曰:「今日教閱,進止分合,軍律整肅,皆卿之力也。」逵奏:「陛下神武,四海共知。六師軍容,孰敢不肅!」時賜酒俱以十分,逵奏以軍馬事不敢飲,帝曰:「少飲之。」親減大半。飲畢,謝恩退。又宣問主管侍衛馬軍
 司李舜舉:「今日按閱之兵,比向時所用之師何如?」舜舉奏曰:「今日所治之兵,皆陛下平時躬親訓練,撫以深恩,賜之重賞,忠勇百倍,非昔日可比。」其儀:皇帝至祥曦殿,行門、禁衛等並戎服迎駕常起居。皇帝至,知閣門官以下並戎服常起居,訖。皇帝乘馬出,從駕官從駕至候潮門外大教場御幄殿下馬,入幄更衣訖,皇帝被金甲出幄,行門、禁衛等迎駕,奏萬福。皇帝乘馬至教場臺下馬,升臺入幄。從駕官宰執、親王、使相、正任知閣、御帶、環衛官升臺,於幄殿分東西相向立。管軍並令全裝衣甲帶御器械執骨朵升臺,於幄殿指南面西立,俟入內官喝排立。皇帝出幄,行門、禁衛等迎駕,奏萬福。皇帝出,閣門分引殿前馬步三司統制、統領官常起居訖。次三司將佐以下,聽鼓聲常起居。次殿帥執骨朵赴御坐前,奏教直陣。俟教閱畢,再赴
 御坐前奏教圓陣。俟教閱畢,再赴御坐前奏教銳陣。俟教閱畢,再赴御坐前奏教閱畢,歸侍立。內侍傳旨與殿前太尉某,諸軍謝恩承旨訖,轉與撥發官引三司統制、統領、將佐再拜謝恩訖,各歸本軍。皇帝起,入幄更衣訖,皇帝出幄。皇帝坐,舍人引宰執塾後立,俟進御茶床。舍人贊就坐,宰執躬身應喏訖,直身立,就坐。進第一盞酒,起立塾後,俟皇帝飲酒訖,舍人贊就坐,躬身應喏訖,直身立。俟宰執酒至,接盞飲酒訖,盞付殿侍。次舍人贊食,並如儀。至第四盞,傳旨宣勸訖,御藥傳旨不拜,舍人承旨贊不拜,贊就坐。第五盞宣勸如第四盞儀。酒食畢,舉御茶床。舍人分引宰執於幄殿重行立,御藥傳旨不拜,舍人承旨訖,揖宰執躬身贊不拜,各祗候直身立,降踏道歸幕次。皇帝起,乘馬至車子院下馬。皇帝出幄,至車子院門樓上,出賜親王酒,再拜謝訖;次賜使相、正任並管軍,知閣、御帶、環衛官酒訖;逐班再拜謝,訖,依舊相向立。次親王執盞進皇帝酒,皇帝飲酒訖,一班再拜謝,訖;俟
 皇帝觀畢,起,降車子院門樓歸幄。親王以下退,皇帝乘馬出車子院門,行門、禁衛等迎駕,奏萬福。皇帝乘馬至候潮門外大教場,應從駕官並戎服乘馬從駕回。皇帝乘馬入和寧門,至祥曦殿上下馬還宮。餘仿此。



 淳熙四年十二月,大閱於茅灘。十年十一月,大閱於龍山。十六年十月,大閱於城南大教場。並如上儀。慶元元年十月,以在諒闇,令宰執於大教場教閱。二年十月,大閱於茅灘。嘉泰二年十二月,幸候潮門外教場大閱。端平二年四月大閱,以時暑,不及行。



 受降獻俘。太祖平蜀,孟昶降,詔有司約前代儀制為受
 降禮。昶至前一日,設御坐仗衛於崇元殿,如元會儀。至日,大陳馬步諸軍於天街左右,設昶及其官屬素案席褥於明德門外,表案於橫街北。通事舍人引昶及其官屬素服紗帽北向序立。昶跪奉表授閣門使,復位待命。表至御前,侍臣讀訖,閣門使承旨出。昶等俯伏。通事舍人掖昶起,官屬亦起,宣制釋罪,昶等再拜呼萬歲。衣庫使導所賜襲衣、冠帶陳於前,昶等又再拜跪受,改服乘馬,至升龍門下馬,官屬至啟運門下馬,就次。帝常服升
 坐,百官先入起居,班立。閣門使引昶等入,舞蹈拜謝。召昶升殿,閣門使引自東階升,宣撫使承旨安撫之。昶至御坐前,躬承問訖,還位,與官屬舞蹈出。中書率百官稱賀,遂宴近臣及昶於大明殿。



 嶺南平,劉鋹就擒,詔有司撰獻俘禮。鋹至,上御明德門,列仗衛,諸軍、百官常服班樓前。別設獻俘位於東西街之南,北向;其將校位於獻俘位前,北上西向。有司率武士系鋹等白練,露布前引。至太廟西南門,鋹等並下馬,入南神門,北向西上立,監
 將校官次南立。俟告禮畢,於西南門出,乘馬押至太社,如上儀。乃押至樓南禦路之西,下馬立俟。獻俘將校,戎服帶刀。攝侍中版奏中嚴,百官班定;版奏外辦,帝常服御坐。百官舞蹈起居畢,通事舍人引鋹就獻俘位,將校等詣樓前舞蹈訖,次引露布案詣樓前北向,宣付中書、門下,如宣制儀。通事舍人跪受露布,轉授中書,門下轉授攝兵部尚書。次攝刑部尚書詣樓前跪奏以所獻俘付有司。上召鋹詰責,鋹伏地待罪。詔誅其臣龔澄樞等,
 特釋鋹縛與其弟保興等罪,仍賜襲衣、冠帶、靴笏、器幣、鞍馬,各服其服列謝樓下。百官稱賀畢,放仗如儀。



 南唐平,帝御明德門,露布引李煜及其子弟官屬素服待罪。初,有司請如獻劉鋹。帝以煜奉正朔,非若鋹拒命,寢露布弗宣,遣閣門使承制釋之。



 太宗征太原,劉繼元降,帝幸城北,陳兵衛,張樂,宴從臣於城臺。繼元帥官屬素服臺下。遣閣門使宣制釋罪,召繼元親勞之。從臣詣行宮稱賀。時以在軍中,故不備禮。繼元至京師,詔告獻太廟。
 前一日,所司陳設如常告廟儀。告日黎明,博士引太尉就位,通事舍人引繼元西階下東向立,其官屬重行立。贊者贊太尉再拜訖,博士引就盥爵如常儀,詣東階解劍脫舄,升第一室進奠,再拜,太祝跪讀祝文訖,又再拜。通事舍人引繼元及官屬詣室前西階下北向立,舍人贊云:「皇帝親征,收復河東,偽主劉繼元及偽命官見。」贊者曰再拜,拜訖退位。次至第二、第三、第四、第五室,皆如第一室。博士引太尉降階,佩劍納履復位,贊者曰再拜,太
 尉與繼元等皆再拜,退。焚祝版於齋坊。繼元既命以官,故不稱俘焉。



 元符二年,西蕃王攏拶、邈川首領瞎徵等降,詔具儀注。以受降日御宣德門,設諸班直、上四軍仗衛,諸軍素服陳列。降者各服蕃服以見,審問訖,有旨放罪,各等第賜首服袍帶。百官稱賀,而再御紫宸殿賜宴會。哲宗崩,樞密院留攏拶等西京聽旨。詔罷御樓立仗,但引見於後殿。攏拶一班,契丹公主一班,夏國、回鶻公主次之,瞎徵一班,邊廝波結並族屬次之。應族屬首領
 各從其長,以次起居。僧尼公主皆蕃服蕃拜。並賜冠服,謝訖,賜酒饌橫門外。



 政和初,議禮局上《受降儀》。皇帝乘輿升宣德門樓,降輿坐御幄,百官與降王、蕃官各班樓下,如大禮肆赦儀。東上閣門以紅絳袋班齊牌引升樓,樓上東上閣門官附內侍承旨索扇,扇合,帝即御坐,簾卷。內侍又贊扇開,侍衛如常儀。諸班親從並裡圍降王人等迎駕,自贊常起居。次舍人贊執儀將士常起居。次管幹降王使臣並隨行舊蕃官常起居。次禮直官、舍人
 引百官橫行北向,贊者曰拜,在位官皆再拜舞蹈,三稱萬歲,又再拜。班首奏聖躬萬福,又再拜退,百官各就東西位。舍人引降王服本國衣冠詣樓前北向,女婦少西立,僧又少西,尼立於後。入內省官詣御坐前承旨,傳樓上東上閣門官承旨錄訖,以紅條袋降制樓下,東上閣門官承旨退。降王以下俯伏,東上閣門官至,令通事舍人掖之起,首領以下皆起,鞠躬。閣門宣有敕,降王以下再拜,僧尼止躬呼萬歲。閣門錄敕旨付管幹官,降王等
 躬聽詰問。如有復奏,閣門錄訖,仍以紅絳袋引升樓。如無復奏,入內省官詣御坐承旨,傳樓上閣門官稱有敕放罪訖,舍人贊謝恩,降王以下再拜稱萬歲,復序立。入內省官詣御坐承旨,傳樓上閣門官稱有敕各賜首服袍帶。樓下閣門官承旨引所賜簷床陳於西,舍人宣曰有敕,降王以下再拜鞠躬,舍人稱各賜某物,賜物畢,又再拜稱萬歲。若賜官、即贊謝再拜,並歸次易所賜服。舍人先引降王以下至授遙郡以上當樓前北向東上立,
 贊再拜,稱萬歲,又再拜;次贊服冠帔婦女再拜。們尼別謝,引還。次贊樓上侍立官稱賀再拜,禮直官,舍人分引百官橫行北向立,贊拜訖,班首少前,俯伏跪,稱賀,其詞中書隨事撰述,賀訖,復位。在位者又再拜舞蹈,三稱萬歲,又再拜。東上閣門官進詣樓前承旨,就班首宣曰有制,贊者曰拜,在位官皆再拜,宣答,其詞學士院隨事撰述,又贊再拜,三稱萬歲,又再拜。樓上樞密院前跪奏,稱某官臣某言,禮畢,內侍索扇,扇合簾垂,帝降坐。內侍贊
 扇開,所司承旨放仗,樓下鞭鳴,百官再拜退。



 開禧三年三月,四川宣撫副使安丙函逆臣吳曦首並違制創造法物、所受金國加封蜀王詔及金印來獻。四月三日,禮部太常寺條具獻馘典故,俟逆曦首函至日,臨安府差人防守,殿前司差甲士二百人同大理寺官監引赴都堂審驗。奏獻太廟、別廟差近上宗室南班,奏獻太社、太稷差侍從官。各前一日赴祠所致齋,至日行奏獻之禮,大理寺、殿前司計會行禮時刻,監引首函設置以俟。奏
 獻禮畢,梟於市三日,付大理寺藏於庫。



 端平元年,金亡。四月,京湖制置司以完顏守緒函骨來上,差官奏告宗廟社稷如儀。



 田獵。太祖建隆二年,始校獵於近郊。先出禁軍為圍場,五坊以鷙禽細犬從。帝親射走兔三,從官貢馬稱賀。其後多以秋冬或正月田於四郊,從官或賜窄袍暖靴,親王以下射中者賜以馬。



 太宗將北征,因閱武獵近郊,以多盜獵狐兔者,命禁之。有衛士奪人獐,當死,帝曰:「若殺
 之,後世必謂我重獸而輕人。」特貰其罪。帝常以臘日校獵,諭從臣曰:「臘日出狩,以順時令,緩轡從禽,是非荒也。」回幸講武臺,張樂,賜群臣飲。其後,獵西郊,親射走兔五。詔以古者搜狩,以所獲之禽薦享宗廟,而其禮久廢,今可復之。遂為定式。帝雅不好弋獵,詔除有司行禮外,罷近甸游畋,五坊所畜鷹犬並放之,諸州不得以鷹來獻。已而定難軍節度使趙保忠獻鶻一,號「海東青」,詔還賜之。臘日,但命諸王略畋近郊,而五坊之職廢矣。



 真宗
 復詔教駿所養鷹鶻量留十餘,以備諸王從時展禮。禁圍草地,許民耕牧。



 至仁宗時,言者言校獵之制所以順時令、調戎事,請修此禮。於是詔樞密院奏定制度。獵日五鼓,帝御內東門,賜從官酒三行,奏鈞容樂,幸瓊林苑門,賜從官食。遂獵於楊村,宴於幄殿,奏教坊樂。遣使以所獲馳薦太廟。既而召父老臨問,賜以飲食茶絹,及五坊軍士銀絹有差。宰相賈昌朝等曰:「陛下暫幸近郊,順時田獵,取鮮殺而登廟俎,所以昭孝德也;即高原而閱
 軍實,所以講武事也;問耆老而秩飫,所以養老也;勞田夫而賜惠,所以勸農也。乘輿一出,而四美皆具。伏望宣付史館。」從之。明年,復獵於城南東韓村。自玉津園去輦乘馬,分騎士數千為左右翼,節以鼓旗。合圍場徑十餘里,部隊相應。帝按轡中道,親挾弓矢,屢獲禽焉。是時,道傍居人或畜狐兔鳧雉驅場中。帝謂田獵以訓武事,非專所獲也,悉縱之。免圍內民田一歲租,仍召父老勞問。其後以諫者多,罷獵近甸。自是,終靖康不復講。



 打球,本軍中戲。太宗令有司詳定其儀。三月,會鞠大明殿。有司除地,豎木東西為球門,高丈餘,首刻金龍,下施石蓮華坐,加以採繢。左右分朋主之,以承旨二人守門,衛士二人持小紅旗唱籌,御龍官錦繡衣持哥舒棒,周衛球場。殿階下,東西建日月旗。教坊設龜茲部鼓樂於兩廊,鼓各五。又於東西球門旗下各設鼓五。閣門豫定分朋狀取裁。親王、近臣、節度觀察防禦團練使、刺史、駙馬都尉、諸司使副使、供奉官、殿直悉預。其兩朋官,宗室、
 節度以下服異色繡衣,左朋黃衣蘭;右朋紫衣蘭打球供奉官左朋服紫繡,右朋服緋繡,烏皮靴,冠以華插腳折上巾。天廄院供馴習馬並鞍勒。帝乘馬出,教坊大合《涼州曲》,諸司使以下前導,從臣奉迎。既御殿,群臣謝,宣召以次上馬,馬皆結尾,分朋自兩廂入,序立於西廂。帝乘馬當庭西南駐。內侍發金合,出朱漆球擲殿前。通事舍人奏云:御朋打東門。帝擊球,教坊作樂奏鼓。球既度,颭旗、嗚鉦、止鼓。帝回馬,從臣奉觴上壽,貢物以賀。賜酒,即列
 拜,飲畢上馬。帝再擊之,始命諸王大臣馳馬爭擊。旗下擂鼓。將及門,逐廂急鼓。球度,殺鼓三通。球門兩旁置繡旗二十四,而設虛架於殿東西階下。每朋得籌,既插一旗架上以識之。帝得籌,樂少止,從官呼萬歲。群臣得籌則唱好,得籌者下馬稱謝。凡三籌畢,乃御殿召從臣飲。又有步擊者、乘驢騾擊者,時令供奉者朋戲以為樂云。



 救日伐鼓。建隆元年,司天監言日食五月朔,請掩藏戈兵鎧冑。事下有司,有司請皇帝避正殿,素服,百官各守
 本司,遣官用牲太社如故事。景德四年五月朔,日食。上避正殿不視事。



 至和元年四月朔,日食,既內降德音:改元,易服,避正殿,減膳。百官詣東上閣門拜表請御正殿,復常膳。三表乃從。至日,遣官祀太社,而陰雨以雷,至申,乃見食,九分之餘。百官稱賀。先是皇祐初,以日食三朝不受賀,百官拜表。嘉祐四年,詔正旦日食毋拜表,自十二月二十一日不御前殿,減常膳,宴遼使罷作樂。至日,仍遣官祀太社。百官三表,乃御正殿,復膳。六年六月朔
 日食,詔禮官驗詳典故。皇帝素服,不御正殿,毋視事,百官廢務守司。合朔前二日,郊社令及門僕守四門,巡門監察鼓吹令率工人如方色執麾斿,分置四門屋下。龍蛇鼓隨設於左東門者立北塾南面,南門者立東塾西面,西門者立南塾北面,北門者立西塾東面。隊正一人執刀,率衛士五人執五兵之器,立鼓外。矛處東,戟處南,斧鉞在西,槊在北。郊社令立□於壇,四隅縈朱絲繩三匝。又於北設黃麾,龍蛇鼓一次之,弓一、矢四次之。諸兵
 鼓俱靜立,俟司天監告日有變,工舉麾,乃伐鼓;祭告官行事,太祝讀文,其詞以責陰助陽之意。司天官稱止,乃罷鼓。如霧晦不見,即不伐鼓,自是,日有食之,皆如其制。



 治平四年,詔:「古者日食,百司守職,蓋所以祗天戒而備非常,今獨闕之,甚非王者小心寅畏之道。可令中書議舉行。」熙寧六年四月朔,日食,詔易服、避殿、減膳如故事。降天下死刑,釋流以下罪。



 政和上《合朔伐鼓儀》:有司陳設太社玉幣籩豆如儀。社之四門,及壇下近北,各置鼓
 一,並植麾斿,各依其方色。壇下立黃麾,麾杠十尺,斿八尺。祭告日,於時前,太官令帥其屬實饌具畢,光祿卿點視;次引監察御史、奉禮郎、太祝、太官令先入就位,次引告官就位,皆再拜;次引御史、奉禮郎、太祝升,就位。太官令就酌尊所,告官盥洗,詣太社三上香,奠幣玉,再拜復位。少頃,引告官再盥洗,執爵三祭酒,奠爵,俯伏興,少立,引太祝詣神位前跪讀祝文。告官再拜退,伐鼓。其日時前,太史官一員立壇下視日。鼓吹令率工十人,如色服
 分立鼓左右以俟。太史稱日有變,工齊伐鼓。明復,太史稱止,乃罷鼓。其日廢務,而百司各守其職如舊儀。



\end{pinyinscope}