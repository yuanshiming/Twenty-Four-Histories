\article{志第三十 律歷十}

\begin{pinyinscope}

 觀天歷



 元祐《觀天歷》



 演紀上元甲子,距元祐七年壬申,歲積五百九十四萬四千八百八算。上考往古,每年減一;下驗將來,每年加二。



 步氣朔



 統法:一萬二千三十。



 歲周:四百三十九萬三千八百八十。



 歲餘:六萬三千八十。



 氣策:一十五、餘二千六百二十八、秒一十一。



 朔實:三十五萬五千二百五十三。



 朔策:二十九、餘六千三百八十三。



 望策:一十四、餘九千二百六、秒一十八。



 弦策:七、餘四千六百三、秒九。



 歲閏:一十三萬八百四十四。



 中盈分:五千二百五十六、秒二十四。



 朔虛分:五千六百四十七。



 沒限分:九千四百二。



 閏限:三十四萬四千三百四十九、秒一十二。



 旬周:七十二萬一千八百。



 紀法:六十。



 以上秒母同三十六。



 推天正冬至:置距所求積年,以歲周乘之,為氣積分;滿旬周去之,不盡,以統法約之為大餘,不滿為小餘。其大餘命甲子,算外,即所求年天正冬至日辰及餘。



 求次氣:置天正冬至大、小餘,以氣策及餘秒累加之,秒盈秒法從小餘一,小餘盈統法從大餘一,大餘盈紀法去之。



 命甲子,算外,即各得次氣日辰及餘秒。



 推天正經朔:置天正冬至氣積分,以朔實去之,不盡為閏餘;以減天正冬至氣積分,餘為天正十一月經朔加
 時積分;滿旬周去之,不盡,以統法約之為大餘,不滿為小餘。其大餘命甲子,算外,即所求年天正十一月經朔日辰及餘。



 求弦望及次朔經日:置天正十一月經朔大、小餘,以弦策累加之,去命如前,即各得弦、望及次朔經日及餘秒。



 求沒日:置有沒之氣小餘,以三百六十乘之,其秒進一位,從之,用減歲周,餘滿歲餘除之為日,不滿為餘。其日,命其氣初日日辰,算外,即為其氣沒日日辰。凡氣小餘在沒限以
 上者,為有沒之氣。



 求滅日:置有滅之朔小餘,以三十乘之,滿朔虛分除之為日,不滿為餘。其日命其月經朔初日日辰,算外,即為其月滅日日辰。凡經朔小餘不滿朔虛分者,為有滅之朔。



 步發斂



 候策:五、餘八百七十六、秒四。



 卦策:六、餘一千五十一、秒一十二。



 土王策:三、餘五百二十五、秒二十四。



 月閏:一萬九百三、秒二十四。



 辰法:二千五。



 半辰法:一千二半。



 刻法:一千三百三。



 秒母:三十六。



 推七十二候:各因中節大、小餘命之,為初候;以候策加之,為次候;又加之,為末候。



 求六十四卦:各因中氣大、小餘命之,為初卦用事日;以
 卦策加之,為中卦用事日;又加之,得終卦用事日。以土王策加諸侯內卦,得十有二節之初外卦用事日;又加之,得大夫卦用事日;復以卦策加之,得卿卦用事日。



 推五行用事:各因四立之節大、小餘命之,即春木、夏火、秋金、冬水首用事日;以土王策減四季中氣大、小餘,命甲子,算外,為其月土始用事日。



 求中氣去經朔:置天正冬至閏餘,以月閏累加之,滿統法約之為日,不盡為餘,即各得每月中氣去經朔日及
 餘秒。其閏餘滿閏限者,為月內有閏也;仍定其朔內無中氣者為閏月。



 求卦候去經朔:以卦、候策累加減中氣,去經朔日及餘,中氣前,減;中氣後,加。



 即各得卦、候去經朔日及餘秒。



 求發斂加時:倍所求小餘,以辰法除之為辰數,不滿,五因之,滿刻法為刻,不滿為餘。其辰數命子正,算外,即各得所求加時辰、刻及分。



 步日躔



 周天分:四百三十九萬四千三十四、秒五十七。



 周天度:三百六十五、餘三千八十四、秒五十七。



 歲差:一百五十四、秒五十七。



 二至限日:一百八十二、餘七千四百八十。



 冬至後盈初夏至後縮末限日:八十八、餘一萬九百五十八。



 夏至後縮初冬至後盈末限日:九十三、餘八千五百五十二。



 求每日盈縮分:置入二至後全日,各在初限已下為初
 限;已上,用減二至限,餘為末限。列初、末限日及分於上,倍初、末限日及約分於下,相減相乘。求盈縮分者,在盈初、縮末,以三千二百九十四除之。在盈末、縮初,以三千六百五十九除之,皆為度,不滿,退除為分秒。求朏朒積者,各退二位,在盈初縮末,以三百六十六而一;在盈末縮初,以四百七而一,各得所求。以盈縮相減,餘為升降分;盈初縮末為升,縮初盈末為降。



 以朏朒積相減,餘為損益率。在初為益,在末為損。



 求經朔弦望入盈縮限:置天正閏日及餘,減縮末限日及餘,為天正十一月經朔入縮末限日及餘;以弦策累加之,滿盈縮限日去之,即各得弦望及次朔入盈縮限日及餘秒。



 求經朔弦望朏朒定數:各置所入盈縮限日小餘,以其日下損益率乘之,如統法而一,所得,損益其下朏朒積為定數。



 求定氣:冬夏二至以常氣為定氣。自後,以其氣限日下
 盈縮分盈加縮減常氣約餘,即為所求之氣定日及分秒。



 赤道宿度



 斗:二十六



 牛:八



 女:十二



 虛:十少秒六十四



 危:十七



 室:十六



 壁:九



 北方七宿九十八度少、秒六十四。



 奎:十六



 婁:十二



 胃:十四



 昴:十一



 畢:十七



 觜:一



 參:十



 西方七宿八十一度。



 井:三十三



 鬼:三



 柳:十五



 星:七



 張:十八



 翼:十八



 軫:十七



 南方七宿一百一十一度。



 角:十二



 亢:九



 氐:十五。



 房:五



 心:五



 尾:十八



 箕:十一



 東方七宿七十五度。



 前皆赤道宿度,與古不同。自《大衍歷》依渾儀測為定,
 用紘帶天中,儀極攸憑,以格黃道。



 推天正冬至加時赤道日度:以歲差乘所求積年,滿周天分去之,不盡,用減周天分,餘以統法除之為度,不滿為餘。命起赤道虛宿四度外去之,至不滿宿,即為所求年天正冬至加時赤道日度及餘秒。



 求夏至赤道日度:置天正冬至加時赤道日度,以二至限及餘加之,滿赤道宿次去之,即得夏至加時赤道日度及餘秒。因求後昏後夜半赤道日度者,以二至小餘減統法,餘以加二至赤道日度之餘,即二至
 初日昏後夜半赤道日度,以每日累加一度,去命如前,各得所求。



 求二十八宿赤道積度:置二至加時日躔赤道全度,以二至加時赤道日度及約分減之,餘為距後度。以赤道宿次累加之,即得二十八宿赤道積度及分秒。



 求二十八宿赤道積度入初末限:各置赤道積度及分秒,滿象限九十一度三十一分、秒九即去之,若在四十五度六十五分、秒五十四半已下為初限;已上,用減象限,餘為末限。



 求二十八宿黃道度:各置赤道宿入初、末限度及分,三之,為限分。用減四百,餘以限分乘之,一萬二千而一為度,命曰黃赤道差。至後以減、分後以加赤道宿積度,為黃道積度;以前宿黃道積度減之,餘為二十八宿黃道度及分。其分就近約為太、半、少,若二至之宿不足減者,即加二至限,然後減之,餘依術算。



 黃道宿度



 斗:二十三半



 牛:七半



 女:十一半



 虛:十少秒六十四。



 危:十七太



 室:十七少



 壁:九太



 北方七宿九十七度半、秒六十四。



 奎:十七太



 婁:十二太



 胃:十四半



 昴:十一太



 畢:十六



 觜:一



 參:九少



 西方七宿八十二度。



 井:三十



 鬼:二太



 柳:十四少



 星:七



 張:十八太



 翼:十九半



 軫:十八太



 南方七宿一百一十一度。



 角:十三



 亢:九半



 氐:十五半



 房:五



 心:四太



 尾:十七



 箕:十



 東方七宿七十四度太。



 前黃道宿度,乃依今歷歲差變定。若上考往古,下驗將來,當據歲差,每移一度,依歷推變,然後可步七曜,知其所在。



 求天正冬至加時黃道日度:置天正冬至加時赤道日度及約分,三之,為限分;用減四百,餘以限分乘之,一萬二千而一為度,命曰黃赤道差;用減天正冬至加時赤
 道日度及分,即為所求年天正冬至加時黃道日度及分。夏至日度,準此求之。



 求二至初日晨前夜半黃道日度:置一萬分,以其日升降分升加降減之,以乘二至小餘,如統法而一,所得,以減二至加時黃道日度,餘為二至初日晨前夜半黃道日度及分。



 求每日晨前夜半黃道日度:置二至初日晨前夜半黃道日度及分,每日加一度,百約其日下升降分,升加
 降減之,滿黃道宿次去之,即各得二至後每日晨前夜半黃道日度及分。



 求太陽過宮日時刻:置黃道過宮宿度,以其日晨前夜半黃道宿度及分減之,餘以統法乘之,如其太陽行分而一,為加時小餘;如發斂求之,即得太陽過宮日、時、刻及分。



 黃道過宮太史局吳澤等補治有此一段,開封進士吳時舉、國學進士程喜、常州百姓張文進本並無之。



 危宿十五度少,入衛之分,亥。



 奎宿三度半,入魯之分,戌。



 胃宿五度半,入趙之分,酉。



 畢宿十度半,入晉之分,申。



 井宿十二度,入秦之分,未。



 柳宿七度半,入周之分,午。



 張宿十七度少,入楚之分,巳。



 軫宿十二度,入鄭之分,辰。



 氐宿三度少,入宋之分,卯。



 尾宿八度,入燕之分,寅。



 斗宿九度,入吳之分,醜。



 女宿六度少,入齊之分,子。



 步月離



 轉周分:三十三萬一千四百八十二、秒三百八十九。



 轉周日:二十七、餘六千六百七十二、秒三百八十九。



 朔差日:一、餘一萬一千七百四十、秒九千六百一十一。



 弦策:七、餘四千六百三、秒二千五百。



 望策:一十四、餘九千二百六、秒五千。



 以上秒母同一萬。



 七日:初數一萬六百九十,初約八十九;末數一千三百四十,末約一十一。



 十四日:初數九千三百五十一,初約七十八;末數二千
 六百七十九,末約二十二。



 二十一日:初數八千一十一,初約六十七;末數四千一十九,末約三十三。



 二十八日:初數六千六百七十二,初約五十五。



 上弦:九十一度三十一分、秒四十一。



 望:一百八十二度六十二分、秒八十二。



 下弦:二百七十三度九十四分、秒二十三。



 平行:一十三度三十六分、秒八十七半。



 以上秒母同一百。



 求天正十一月經朔加時入轉:置天正十一月經朔加時積分,以轉周分秒去之,不盡,以統法約之為日,不滿為餘。命日,算外,即得所求年天正十一月經朔加時入轉日及餘秒。若以朔差日及餘秒加之,滿轉周日及餘秒去之,即次朔加時入轉日及餘秒。各以其月經朔小餘減之,餘為其月經朔夜半入轉。



 求弦望入轉:因天正十一月經朔加時入轉日及餘秒,以弦策累加之,去命如前,即得弦、望入轉日及餘秒。
 求朔弦望入轉朏朒定數:置入轉餘,乘其日算外損益率,如統法而一,所得,以損益其下朏朒積為定數。其在四七日下餘如初數已下,初率乘之,初數而一,以損益
 其下朏朒積為定數。若初數已上者,以初數減之,餘乘末率,末數而一,用減初率,餘加其日下朏朒積為定數。其十四日下餘若在初數已上者,初數減之,餘乘末率,末數而一,便為朏定數。



 求朔弦望定日、各以入限、入轉朏朒定數,朏減朒加經朔、弦、望小餘,滿若不足,進退大餘,命甲子,算外。各得定日及餘。若定朔幹名與後朔幹名同者月大,不同者月小,其月內無中氣者為閏月。凡注歷,觀定朔小餘,秋分後在統法四分之三已上者,進一日;若春分後定朔晨昏差如春分之日者,三約之,用減四分之三;定朔小餘在此數已上者,亦進一
 日;或當交虧初在日入已前者,其朔不進。弦、望定小餘不滿日出分者,退一日;望若有交,虧初在日出分已前者,其定望小餘雖滿日出分,亦退一日。又有月行九道遲疾,歷有三大二小者;依盈縮累增損之,則有四大三小,理數然也。若俯循常儀,當察加時早晚,隨其所近而進退之,使不過三大二小。



 求定朔弦望加時日度:置定朔、弦、望約分,副之,以乘其日升降分,一萬約之,所得,升加降減其副,以加其日夜半日度,命如前,各得定朔、弦、望加時日躔黃道宿度及分秒。



 求月行九道:凡合朔初交,冬入陰歷,夏入陽歷,月行青
 道。冬至、夏至後,青道半交在春分之宿,出黃道東;立冬、立夏後,青道半交在立春之宿,出黃道東南:至所沖之宿亦如之。



 冬入陽歷,夏入陰歷,月行白道。冬至、夏至後,白道半交在秋分之宿,出黃道西;立冬、立夏後,白道半交在立秋之宿,出黃道西北;至所沖之宿亦如之。



 春入陽歷,秋入陰歷,月行朱道。春分、秋分後,朱道半交在夏至之宿,出黃道南;立夏、立秋後,朱道半交在立夏之宿,出黃道西南:至所沖之宿亦如之。



 春入陰歷,秋入陽歷,月行黑道。春分、秋分後,黑道半交在冬至之宿,出黃道北;立春、立秋後,黑道半交在立冬之宿,出黃道東北:至所沖之宿亦如之。



 四序離為八節,至陰陽之所交,皆與黃道相會,故月行有九道。各視月行所入正交積度,滿交象去之,
 入交積度及交象度,並在交會術中。



 若在半交像已下為初限;已上,覆減交象,餘為末限。置初、末限度及分,三之,為限分;用減四百,餘以限分乘之,二萬四千而一為度,命曰月道與黃道差數。距正交後、半交前,以差數加;距半交後、正交前,以差數減。此加減出入黃道六度,單與黃道相校之數,若校赤道,則隨氣遷變不常。



 仍計去冬、夏二至已來度數,乘差數,如九十而一,為月道與赤道差數。凡日以赤道內為陰,外為陽;月以黃道內為陰,外為陽。故月行宿度,入春分交後行陰歷,秋分交後行陽歷,皆為同名;入春分交後行陽歷,秋分交後行陰歷,皆為異名。



 其在同名者,以差
 數加者加之,減者減之;其在異名者,以差數加者減之,減者加之。二差皆增益黃道宿積度,為九道宿積度;以前宿九道積度減之,為其宿九道度及分秒。其分就近約之為太、半、少。



 求月行九道平交入氣:各以其月閏日及餘,加經朔加時入交泛日及餘秒,盈交終日及餘秒去之,乃減交終日及餘秒。即各得平交入其月中氣日及餘秒;若滿氣策即去之,餘為平交入後月節氣日及餘秒。若求朏朒定數,如求
 朔、望朏朒術入之,即得所求。



 求平交入轉朏朒定數:置所入氣餘,加其日夜半入轉餘,乘其日算外損益率,如統法而一,所得,以損益其下朏朒積,乃以交率乘之,交數而一,為定數。



 求正交入氣:以平交入氣、入轉朏朒定數,朏減朒加平交入氣餘,滿若不足,進退其日,即正交入氣日及餘秒。



 求正交加時黃道日度:置正交入氣餘,副之,以乘其日升降分,一萬約之,升加降減其副,乃以一百乘之,如統
 法而一,以加其日夜半日度,即正交加時黃道日度及分秒。



 求正交加時月離九道宿度:置正交度加時黃道日及分,三之,為限分。用減四百,餘以限分乘之,二萬四千而一,命曰月道與黃道差數。以加黃道宿度,仍計去冬、夏二至已來度數,以乘差數,如九十而一,為月道與赤道差數。同名以加,異名以減,二差皆增損正交度,即正交加時月離九道宿度及分秒。



 求定朔弦望加時月離黃道宿度:置定朔、弦、望加時日躔黃道宿度及分,凡合朔加時,月行潛在日下,與太陽同度,是為加時月度。各以弦、望度加其所當日度,滿黃道宿次去之,即各得定朔、弦、望加時月離黃道宿度及分秒。



 求定朔弦望加時月離九道宿度:置定朔、弦、望加時月離黃道宿度及分秒,加前宿正交後黃道積度,如前求九道術入之,以前定宿正交後九道積度減之,餘為定
 朔、弦、望加時月離九道宿度及分秒。凡合朔加時,若非正交,即日在黃道、月在九道所入宿度。雖多少不同,考其去極,若應繩準,故曰加時九道。



 求定朔午中入轉:各視經朔夜半入轉日及餘秒,以半法加之,若定朔及餘有進退者,亦進退轉日,否則因經為定。因求次日,累加一日,滿轉周日及餘秒去之,即每日午中入轉。



 求晨昏月度:以晨分乘其日算外轉定分,如統法而一,為晨轉分;用減轉定分,餘為昏轉分;乃以朔、弦、望小餘乘其日算外轉定分,如統法而一,為加時分;以減晨昏
 轉分,餘為前;不足減者,覆減之,餘為後;以前加後減定朔、弦、望月度,即晨、昏月所在度。



 求朔弦望晨昏定程:各以其朔昏定月減上弦昏定月,餘為朔後昏定程;以上弦昏定月減望昏定月,餘為上弦後昏定程;以望晨定月減下弦晨定月,餘為望後晨定程;以下弦晨定月減後朔晨定月,餘為下弦後晨定程。



 求每日轉定度數:累計每程相距日轉定分,以減定程,
 餘為盈;不足減者,覆減之,餘為縮;以相距日除之,所得,盈加縮減每日轉定分,為每日轉定度及分秒。



 求每日晨昏月:置朔、弦、望晨昏月,以每日轉定度及分加之,滿宿次去之,為每日晨昏月。凡注歷,自朔日注昏月,望後一日注晨月。



 已前月度並依九道所推,以究算術之精微,如求速要,即依後術求之。



 求天正十一月經朔加時平行月:置歲周,以天正閏餘減之,餘以統法約之為度,不滿,退除為分秒,即天正十
 一月經朔加時平行月積度及分秒。



 求天正十一月定朔夜半平行月:置天正經朔小餘,以平行月度分秒乘之,如統法而一為度,不滿,退除為分秒,以減天正十一月經朔加時平行月積度,即天正十一月經朔晨前夜半平行月。其定朔大餘有進退者,亦進退平行度,否則因經為定,即天正十一月定朔晨前夜半平行月積度及分秒。



 求次定朔夜半平行月:置天正十一月定朔晨前夜半
 平行月積度及分秒,大月加三十五度八十分、秒六十一,小月加二十二度四十三分、秒七十三半,滿周天度及約分、秒去之,即得次定朔晨前夜半平行月積度及分秒。



 求弦望定日夜半平行月:各計朔、弦、望相距之日,乘平行度及分秒,以加其月定朔晨前夜半平行月積度及分秒,即其月弦望定日晨前夜半平行月積度及分秒。



 求定朔晨前夜半入轉:置其月經朔晨前夜半入轉日
 及餘秒,若定朔大餘有進退者,亦進退轉日,否則因經為定,其餘如統法退除為分秒,即得其月定朔晨前夜半入轉日及分秒。因求次日,累加一日,滿轉周二十七日五十五分、秒四十六去之,即每日晨前夜半入轉。



 求定朔弦望晨前夜半定月:置定朔、弦、望晨前夜半入轉分,乘其日算外增減差,百約為分,分滿百為度,增減其下遲疾度,為遲疾定度;遲減疾加定朔、弦、望晨前夜半平行月積度及分秒,以天正冬至加時黃道日度加
 而命之,即各得定朔、弦、望晨前夜半月離宿度及分秒。如求每日晨、昏月,依前術入之,即得所求。



 步晷漏



 二至限:一百八十二日六十二分。



 一象:九十一日三十一分。



 消息法:九千七百三。



 半法:六千一十五。



 辰法:二十五。



 半辰法:一十二半。



 刻法:一千二百二。



 辰刻:八、餘四百一。



 昏明分:三百太。



 昏明刻:二、餘六百一半。



 冬至嶽臺晷影常數:一丈二尺八寸五分。



 夏至嶽臺晷影常數:一尺五寸七分。



 冬至後初限夏至後末限:四十五日、六十二分。



 冬至後末限夏至後初限:一百三十七日、空分。



 求嶽臺晷影入二至後日數:計入二至以來日數,以二至約分減之,乃加半日之分五十,即入二至後來午中日數及分。



 求嶽臺午中晷影定數:置入二至後日及分,如初限已下者為初;已上,覆減二至限,餘為末。其在冬至後初限、夏至後末限者,以入限日入分減一千九百三十七半,為泛差。仍以入限日及分乘其日盈縮積,其盈縮積者,以入盈縮限
 日及分與二百相減相乘,為盈縮積也。



 五因百約,用減泛差,為定差;乃以入限日及分自相乘,以定差乘之,滿一百萬為尺,不滿為寸、分,以減冬至嶽臺晷影常數,餘為其日午中晷影定數。其在冬至後末限、夏至後初限者,以三約入限日及分,減四百八十五少,為泛差;仍以盈縮差度減去極度,餘者春分後、秋分前,四約,以加泛差,為定差。春分前、秋分後,以去二分日數乘之,六百而一,以減泛差,為定差。乃以入限日及分自相乘,以定差乘之,滿一百萬為
 尺,不滿為寸分,以加夏至嶽臺晷影常數,為其日午中晷影定數。



 求每日午中定積日:置其日午中入二至後來日數及分,以其日盈縮分盈加縮減之,即每日午中定積日及分。



 求每日午中消息定數:置定積日及分,在一像已下自相乘,已上,用減二至限,餘亦自相乘,七因,進二位,以消息法除之,為消息常數;副置之,用減六百一半,餘以乘
 其副,以二千六百七十除之,以加常數,為消息定數。冬至後為息,夏至後為消。



 求每日黃道去極度:置其日消息定數,十六乘之,滿四百一除之為度,不滿,退除為分,春分後加六十七度三十一分,秋分後減一百一十五度三十一分,即每日午中黃道去極度及分。



 求每日太陽去赤道內外度:置其日黃道去極度及分,與一象度相減,餘為太陽去赤道內、外度及分。去極多為日在
 赤道外,去極少為日在赤道內。



 求每日晨昏分及日出入分半晝分。置其日消息定數,春分後加二千一百少,秋分後減三千三百八少,各為其日晨分;用減統法,餘為昏分。以昏明分加晨分,為日出分;減昏分,為日入分;以日出分減半法,餘為半晝分。



 求每日距中度:置其日晨分,進位,十四因之,以四千六百一十一除之為度,不滿,退除為分,即距子度。用減半周天,餘為距中度;五而一,為每更差數。



 求每日夜半定漏:置晨分,進一位,如刻法而一為刻,不滿為刻分,即每日夜半定漏。



 求每日晝夜刻及日出入辰刻:置夜半定漏,倍之,加五刻,為夜刻。減百刻,為晝刻。以昏明刻加夜半定漏,命子正,算外,得日出辰刻。以晝刻加之,命如前,即日入辰刻。其辰數依發斂術求之。



 求更點辰刻:置其日夜半定漏,倍之,二十五而一為籌差;半之,進位,為更差。以昏明刻加日入辰刻,即甲夜辰
 刻;以更籌差累加之,滿辰刻及分去之,各得每更籌所在辰刻及分。若用司辰漏者,倍夜半定漏,減去待旦十刻,餘依術算,即得內中更籌也。



 求每日昏曉中星及五更中星:置距中度,以其日昏後夜半赤道日度加而命之,即得其日昏中星所格宿次,命之曰初更中星。以每更差度加而命之,即乙夜中星。以更差度累加之,去命如前,即五更及曉中星。若依司辰星漏倍距子度,減去待旦三十六度五十二分半,餘依術求更點差度,即內中昏曉五更及攢點中星也。



 求九服距差日:各於所在立表候之,若地在嶽臺北,測
 冬至後與嶽臺冬至晷影同者,累冬至後至其日,為距差日。若地在嶽臺南,測夏至後與嶽臺晷影同者,累夏至後至其日,為距差日。



 求九服晷影:若地在嶽臺北冬至前後者,以冬至前後日數減距差日,為餘日。以餘日減一千九百三十七半,為泛差。依前術求之,以加嶽臺冬至晷影常數,為其地其日午中晷影定數。冬至前後日多於距差日者,乃減去距差日,餘依法求之,即得其地其日午中晷影定數。
 若地在嶽臺南夏至前後者,以夏至前後日數減距差日,為餘日。乃三約之,以減四百八十五少,為泛差。依前術求之,以減嶽臺夏至晷影常數,即其地其日午中晷影定數。如夏至前後日數多於距差日,乃減去距差日,餘依法求之,即得其地其日午中晷影定數,即晷在表南也。



 求九服所在晝夜漏刻:各於所在下水漏,以定二至夜刻,乃相減,餘為二至差刻。乃置嶽臺其日消息定數,以
 其處二至差刻乘之,如嶽臺二至差刻二十除之,所得為其地其日消息定數。乃倍消息定數,進位,滿刻法約之為刻,不滿為分,以加減其處二至夜刻,春分後、秋分前,以加夏至夜刻;秋分後、春分前,以減冬至夜刻。



 為其地其日夜刻;以減百刻,餘為晝刻。求日出入差刻及五更中星,並依嶽臺法求之。



\end{pinyinscope}