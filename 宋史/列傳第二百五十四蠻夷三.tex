\article{列傳第二百五十四蠻夷三}

\begin{pinyinscope}

 撫水州在宜州南,有縣四:曰撫水,曰京水,曰多逢,曰古勞。唐隸黔南。其酋皆蒙姓同出,有上、中、下三房及北遐
 一鎮。民則有區、廖、潘、吳四姓。亦種水田、采魚,其保聚山險者,雖有纁田,收穀粟甚少,但以藥箭射生,取鳥獸盡,即徙他處,無羊馬、桑柘。地曰帚洞,五十里至前村,川原稍平,合五百餘家,夾龍江居,種稻似湖湘。中有樓屋戰棚,衛以竹柵,即其酋所居。兵器有環刀、摽牌、木弩。善為藥箭,中者大叫,信宿死,得邕州藥解之即活。



 雍熙中,數寇邊境,掠取民口、畜產。詔書招安,補其酋蒙令地殿直,蒙令劄奉職。咸平中,又數為寇盜,止令邊臣驅逐出境。其
 黨狡獪者凡三十餘人,宜州守將因擒送闕下,上召見詰責之,對曰:「臣等蠻陬小民,為饑寒所迫耳。」上顧謂左右曰:「昨不欲盡令剿絕,若縱殺戮,顧無噍類矣!」因釋罪,賜錦袍、冠帶、銀彩,戒勖遣之。逾年,酋長蒙頂等六十五人詣闕,納器甲百七十事。又蒙漢誠、蒙虔瑋、蒙填來朝,上器甲數百及毒藥箭,誓不搔邊。比歲皆遣使來貢及輸兵器,乃授漢誠官,賜物有差,既而侵軼如故。景德三年,蠻酋蒙填詣宜州自陳,願朝貢謝罪,詔守臣諭以盡還所掠民
 貲畜,乃從其請。



 大中詳符六年,首領指揮使蒙但挈族來歸,徙於桂州。九年,數寇宜、融州界,轉運使俞獻可言:「知宜州董元己不善綏撫,昨蠻人饑,來質餱糧,公縱主者克剝概量;及求入貢,復驟沮其意:遂使忿恚為亂。」詔出元己,遂遣潭州都監季守睿代元己招撫,群蠻拒命,侵掠不已。獻可請以本道澄海軍及募丁壯進討,乃詔益以潭州兵五千人,命東染院使、平州刺史曹克明為宜融等州都巡檢安撫使,內殿崇班王文慶、閣門祗候
 馬玉、內供奉官楊守珍等為都監。上猶以蠻夷異類,攻剽常理,不足以剿絕。又意其道險難進師,第令克明、獻可設方略攝其酋首,索所鈔生口,因而撫之。克明、獻可上言:「蠻人去冬寇天河,今又鈔融州廂陽諸砦,剽劫居民,害巡檢樊明,累依宣旨詔諭,曾不悛革,臣請便宜掩擊。」從之。



 克明乃與守珍領兵入樟嶺路,文慶、玉趨宜州西路,又令宜、桂都巡檢程化鵬取樟嶺古牢隘路會合。化鵬遇蠻於上房兩水口,擊破之。文慶、玉至如門團,為
 蠻所扼,不能進。克明、守珍乃過橫溪恩德砦,召山獠向導,開路進師。蠻依篁竹間,時出戰鬥,輒敗走。旬餘,上黃泥嶺杉木隘路,溪穀險邃,蠻據要害以拒官軍,自辰至午,大潰。其黨遂過霸苑抵帚洞,乃入中房前村。克明等頓兵下砦,中夕,群蠻大嘩噪,擊鉦鼓,攻砦甚急,出兵擊之,傷殺頗眾,因縱火焚其廬室積聚,自此恐懼,竄入山谷。又緣龍江南岸而東,至昏暮,過石峽隘險,士不並行。蠻復連弩北岸,克明遣猛士步涉與鬥,至即退走,砦於
 下房博賀村,克明設伏砦外。其夜,蠻眾大集,遇伏發,內外合擊,追斬殆盡。乘勝搜山,悉得馬牛享士卒。



 克明等知其窮蹙,乃曉諭恩信,許以改過,於是酋帥蒙承貴等麵縛詣軍自首,克明厚加犒宴,且數責之,皆俯伏謝罪。及聞詔旨赦令勿殺,莫不泣下,北望稱萬歲。上以夷性無厭,習知朝廷多釋其罪,故急則來歸,緩則叛去,切詔克明等諭以悉還所掠漢口、資畜,即許要盟。承貴等感悅奉詔,乃歃貓血立誓,自言奴山摧倒,龍江西流,不敢
 復叛。克明等師還,宜州蠻人納器甲凡五千數,願遷處漢地者七百餘口,詔分置廣西及荊湖州軍,給以田糧。凡立功使臣將士遷補、賜齎者千八百一十六人。承貴因請改州縣名,以固歸順之意。詔以撫水州為安化州,撫水縣為歸仁縣,京水縣為長寧縣。自是間歲朝貢,不復為邊患矣。



 獻可等又言:「殿直蒙肚知歸化州,州與撫水相接,數遣子文寶及其妻族甘堂偵軍事,又其子格與官軍鬥敵,悉部送赴闕。有蒙隻者,亦肚之子,先嘗告
 賊,署為昭州押牙。」詔補肚密州別駕,隻海州都押牙,賦以官田。文寶、格、甘堂並黥配登、萊州。寶元元年,復率眾寇融、宜州,發邵、澧、潭三州戍兵合數千人往擊。時蠻勢方熾,至殺運糧官吏。復詔趣兵進討,逾年乃平。



 慶曆中,再以方物入貢。至和二年,復至。詔以知州蒙全會為三班奉職,又以監州姚全料為借職。嘉祐六年,又來貢。是後,月赴宜州參謁及貿巨板,每歲州四管犒。及三歲,聽輸所貢兵械於思立砦,以其直償之,遞以官資遷補。熙
 寧初,知宜州錢師孟、通判曹覿擅裁損侵剝之,土人羅世念、蒙承想、蒙光仲等為亂。五年,攻德謹砦,襲將官費萬,殺之。經略司問致寇狀,而宜州但以饑為言,故朝廷賜粟二萬石以安輯之。已而守臣王奇戰死,事聞,乃詔知沅州謝麟、帶禦器械和斌經制溪洞,發在京驍騎兩營及江南、福建將兵三千五百人,以聽師期。明年,世念等遂與諸蠻峒首領族類四千五百人出降。以世念為內殿承製,承想、光仲等十人各拜官。崇寧二年,其酋蒙
 光有者復嘯聚為寇,經略司遣將官黃忱等擊卻之。大觀二年,遂以三州一鎮戶口六萬一千來上。詔以知融州程鄰往黔南路撫諭,官吏推恩有差。至和後,又有融州屬蠻大丘峒首領楊光朝請內附,又有楊克端等百三人來歸,皆納之。



 諸蠻族類不一,大抵依阻山谷,並林木為居,椎髻跣足,走險如履平地。言語侏離,衣服斒斕。畏鬼神,喜淫祀。刻木為契,不能相君長,以財力雄強。每忿怒則推刃同氣,加兵父子間,復仇怨不顧死。出入腰
 弓矢,匿草中射人,得牛酒則釋然矣。親戚比鄰,指授相賣。父子別業,父貧則質身於子,去禽獸無幾。其族鑄銅為大鼓,初成,懸庭中,置酒以召同類,爭以金銀為大釵叩鼓,去則以釵遺主人。相攻擊,鳴鼓以集眾,號有鼓者為「都老」,眾推服之。



 唐末,諸酋分據其地,自為刺史。宋興,始通中國,奉正朔,修職貢。間有桀黠貪利或疆吏失於撫禦,往往聚而為寇,抄掠邊戶。朝廷禽獸畜之,務在羈縻,不深治也。熙寧間,以章惇察訪經制蠻事,諸溪峒相
 繼納土,願為王民,始創城砦,比之內地。元祐初,諸蠻復叛,朝廷方務休息,乃詔諭湖南、北及廣西路並免追討,廢堡砦,棄五溪諸郡縣。崇寧間,復議開邊,於是安化上三州及思廣諸峒蠻夷,皆願納土輸貢賦,及令廣西招納左、右江四百五十餘峒。尋以議者言,以為招致熟蕃非便,乃詔悉廢所置州郡,復祖宗之舊焉。



 紹興初,監察御史明橐言:「湖南邊郡及二廣之地,舊置溪峒歸明官,比年浸廣其員,及諸州措置隘砦,闕人把拓,又令管押
 兵夫,素不習知法令,率貪婪無厭。況管押又皆鄉民,甚為邊患,遭困苦折辱者往往無所赴訴。議者欲俾帥臣籍其姓名,每三年一遷易,如州縣官故事。或云止循舊添差,並罷管押兵夫,宜令二廣、湖南帥臣處置適宜,無啟邊禍,以害遠人。」詔下其議。三年,安化蠻蒙全劍等八百人劫普議砦,火其屋宇,廣西帥臣遣縣砦將佐發兵討平之。



 四年,廣南東、西路宣諭明橐言:



 :平、觀二州本王口、高峰二砦,處廣右西偏,舊常無虞。崇寧、大觀間,邊臣
 啟釁,奏請置州拓境,深入不毛,如平、從、允、孚、庭、觀、溪、馴、敘、樂、隆、兌等十有二州,屬之黔南,其官吏軍兵請給費用,悉由內郡,於是騷然,莫能支吾。政和間,朝廷始悟其非,罷之。或者謂平州為西南重鎮,兼製王江、從、允等州及湖南之武岡軍、湖北之靖州、桂州之桑江峒猺,觀州則控製南丹、陸家砦、茆灘十道及白崖諸蠻,以故二州獨不廢。臣自曆邊,即乞罷平、觀者,前後非一。內攝官吳芾嘗充經略司準備幹當,頗得其詳。



 :



 :觀州初為宜州富
 仁監,大觀間,帥臣王祖道欲招納文、蘭州,都巡檢劉惟忠謂得文、蘭不若取南丹之利,因誣其州莫公佞阻文、蘭不令納土,為公佞罪,惟忠遂禽殺公佞。帥司奏其功,乃改南丹為觀州,命惟忠守之。公佞之死,人以為冤。其弟公晟結溪峒圖報復,連歲攻圍,惟忠中傷死,繼以黃璘代守。璘度不能支,辭疾告罷,以岑利疆代之。黃忱復建議,欲增築高峰砦於富仁監側,為觀聲援。會朝廷罷新邊,遂請以高峰砦為觀州,設知州一人、兵職官二人、
 曹官一人、指使砦保官七人,吏額五十人,廂禁軍、土丁、家丁又千餘人。歲費錢一萬二千九百餘貫、米八千八百一十七石有奇。州無稅租戶籍,皆仰給鄰郡。飛挽涉險阻,或遇蠻寇設伏,陰發毒矢,中人輒死。人畏賊,率委棄道路,縱然達州,縻費亦不可勝計。昔為富仁監時,不聞有警,惟是邊吏欲以刺探為功,故時時稱警急,因以為利,遂欲存而不廢也。比年戶籍日削,民多流離,或轉入溪洞,公私困弊為甚。



 :



 :平州初隸融州,亦羈縻州峒也。
 舊通湖北渠陽軍,置融江砦及文村、臨溪、潯江堡,後以地隔生蠻,遂廢。崇寧間,復隸融。王口砦地接王江,更為懷遠軍,後更為平州;更吉州為從州、王江為允州;並隸黔南。政和二年,復廢。邊吏黃忱、李坦誑其帥臣程鄰,乞存平州,設知州一人、兵職官二人、曹官一人,縣令簿二人,提舉溪峒公事;本州管界都同巡檢二人,五砦堡監官指揮十人,吏額百人,禁軍、土丁千人。歲費錢一萬四千四百一十八貫六百文、米一萬一千一百二十五
 石有奇。州無租賦戶籍,轉運司歲移桂、融、象、柳之粟以給之。及徙融州西北金溪鄉稅米四百九十餘石隸懷遠,縻費甚於觀州。況守臣到任,即奏推恩其子,州、縣、砦、堡例得遷官酬賞,而稅場互市之利又為守臣邊吏所私,獨百姓有征戍轉輸之苦,誠為可憫。臣以為宜罷平、觀二州便。



 :



 :然尚有可議者,觀州初為富仁監時,有銀冶二,官取其利有常額,熙寧元降條例具在,宜先下經略司,責公晟等依熙寧條例施行。況公晟實公佞弟,理宜
 掌州事,近雖逃歸,未為蠻族信服,察其情勢,不得不倚重中國。若乘時授之,彼知恩出朝廷,必深感悅。



 樞密院亦上言:「廣西沿邊堡砦,昨因邊臣希賞,改建州城,侵擾蠻夷,大開邊釁。地屬徼外,租賦亦無所入,而支費煩內郡,民不堪其弊,遂皆廢罷。唯平、觀二州以帥臣所請,故存。今睹明橐所奏,利害之實昭然可見。緣帥臣又稱公晟於南丹、觀州、寶監境上不時竊發,若廢二州,恐於緣邊事宜有所未盡。」詔令廣南西路帥、漕、憲司共條具利
 害以聞。既而諸司交言:「平、觀二州困弊已甚,有害無益,請復祖宗舊制為便。」詔從其言。



 乾道六年,詔補蒙澤進武副尉。初,宜州蠻莫才都為亂,廣西經略劉焞遣進勇副尉蒙明質賊巢,諭降才都。既而復肆猖獗,戕賊官兵。未幾,禽才都,械送經略司伏法,悉破其黨,而明亦遇害,備極慘酷,邊人憐之。焞乞推恩其子澤以旌死事,朝廷從之,故有是命。



 淳熙十年冬,安化蠻突入內地,焚砦柵,殺居民為亂。宜州駐紮將官田昭明與蠻力戰敗,死之。
 十一年,廣西路鈐轄沙世堅言:「官軍與瑤人兵器利鈍不同,宜敕沿邊軍州多置強弩毒矢,以懼瑤人。」從之。是年,安化蠻蒙光漸率眾抄掠,世堅討平之。初,知宜州馬寧祖不支思立砦鹽錢,執議以為前守所積逋,止給錢一月,不能遍及蠻部,而權思立砦準備將領楊良臣復鎮撫乖方,遂致激變光漸等。詔罷良臣,貶寧祖秩,敕帥、漕以時給溪峒鹽錢。



 十二年正月,廣西漕臣胡庭直上言:「邕州之左江、永年、太平等砦,在祖宗時,以其與交阯
 鄰壤,實南邊藩籬重地,故置州縣,籍其丁壯,以備一旦之用,規模宏遠矣。比年邊民率通交阯,以其地所產鹽雜官鹽貨之,及減易馬鹽以易銀,忽而不防,恐生邊釁,所宜禁戢。」既而諸司上言:「經略司初準朝旨,置馬鹽倉,貯鹽以易馬,歲給江上諸軍及御前投進,用銀鹽錦,悉與蠻互市。其永平砦所易交阯鹽,貨居民食,皆舊制也。況邊民素與蠻夷私相貿易,官不能制。今一切禁絕,非惟左江居民乏鹽,而蠻情亦叵測,恐致乖異也。」乃牒邕
 州,禁民毋私販交阯鹽,以妨鈔法。是年,詔以楊世俊襲父進通職,補承信郎。



 紹熙初,廣西帥以本路副總管沙世堅素有韜略,累立邊功,為群蠻所畏服,嘗破蒙光漸,示以威信,光漸不敢寇邊者累年。乞以世堅兼知宜州,實能制伏蠻夷,為久遠之利。帝從之。慶元四年,宜州蠻蒙峒、袁康等寇內地,奪官鹽為亂,廣西帥司調官兵招降之,朝廷推賞有差。



 嘉定三年,章戡知靜江府,建議以為廣西所部二十五郡,三方鄰溪峒,與蠻瑤、黎、蜑雜處,
 跳梁負固,無時無之,西南最為重地。邕、欽之外,羈縻七十有二,地裏綿邈,鎮戍非一,請增置雄邊軍二百人及調憲司甲軍二百隸帥司。初,安平州李密侵鄰洞,劫掠編民,並取古甑峒,以其幼子變姓名為趙懷德知峒事,戡諭邕守推古甑一人主之。十一年,臣僚復上言:「慶曆間,張方平嘗以為朝廷每備西北,孰不知瑤蠻衝突嶺外,南鄰交阯,勢須經營。唐時西備吐蕃,其後安南寇邊,旋致龐勳之禍。國朝每憂契丹、元昊,而儂智高陷邕州,
 南徼騷動,天子為之旰食,豈細故哉?臣等比見淮甸間版築薦興,更戍日益,而廣南城隍摧圮不葺,戍兵逃亡殆盡,春秋教閱,郡無百人。雖有鄉兵、義丁、土丁之名,實不足用,緩急豈能集事?宜於嶺南要地增築城堡,籍其民兵,歲時練習,定賞罰格,以示懲勸。如此則號令嚴明,守禦完固,民習戰鬥,可息瑤蠻侵掠之患,措四十州民於久安之域矣。」詔從之。



 廣源州蠻儂氏,州在邕州西南鬱江之源,地峭絕深阻,
 產黃金、丹砂,頗有邑居聚落。俗椎髻左衽,善戰鬥,輕死好亂。其先,韋氏、黃氏、周氏、儂氏為首領,互相劫掠。唐邕管經略使徐申厚撫之,黃氏納質,而十三部二十九州之蠻皆定。自交阯蠻據有安南,而廣源雖號邕管羈縻州,其實服役於交阯。



 初,有儂全福者,知儻猶州,其弟存祿知萬涯州,全福妻弟儂當道知武勒州。一日,全福殺存祿、當道,並有其地。交阯怒,舉兵執全福及其子智聰以歸。其妻阿儂本左江武勒族也,轉至儻猶州,全福納
 之。全福見執,阿儂遂嫁商人,生子名智高。智高生十三年,殺其父商人,曰:「天下豈有二父耶?」因冒儂姓,與其母奔雷火洞,其母又嫁特磨道儂夏卿。



 久之,智高復與其母出據儻猶州,建國曰大曆。交阯攻拔儻猶州,執智高,釋其罪,使知廣源州,又以雷火、頻婆四洞及思浪州附益之。居四年,內怨交阯,襲據安德州,僭稱南天國,改年景瑞。皇祐元年,寇邕州。明年,交阯發兵討之,不克。廣西轉運使蕭固遣邕州指使亓贇往刺候,而贇擅發兵攻
 智高,為所執,因問中國虛實,贇頗為陳大略,說智高內屬。乃遣贇還,奉表請歲貢方物,未聽。又以馴象、金銀來獻,朝廷以其役屬交阯,拒之。後復齎金函書以請,知邕州陳珙上聞,不報。智高既不得請,又與交阯為仇,且擅山澤之利,遂招納亡命,數出敝衣易穀食,紿言洞中饑,部落離散。邕州信其微弱,不設備也。乃與廣州進士黃瑋、黃師宓及其黨儂建侯、儂誌忠等日夜謀入寇。一夕,焚其巢穴,紿其眾曰:「平生積聚,今為天火焚,無以為生,
 計窮矣。當拔邕州,據廣州以自王,否則必死。」



 四年四月,率眾五千沿鬱江東下,攻破橫山砦,遂破邕州,執知州陳珙等,兵死千餘人。智高閱軍資庫,得所上金、函,怒謂珙曰:「我求一官統攝諸部,汝不以聞,何也?」珙對:「嘗奏,不報。」索奏草不獲,遂扶珙出,珙惶恐呼萬歲,救自效,不聽,乃並其屬及廣西都監張立害之。立臨刑大罵,不為屈。於是智高僭號仁惠皇帝,改年啟曆,赦境內。師宓以下皆稱中國官名。



 是時,天下久安,嶺南州縣無備,一旦
 兵起倉卒,不知所為,守將多棄城遁。故智高所向得志,相繼破橫、貴、龔、潯、藤、梧、封、康、端九州,害曹覲於封州、趙師旦馬貴於康州,餘殺官吏甚眾。所過焚府庫,進圍廣州。初,智高將至,守將仲簡不許民入保城中,民不得入者皆附智高,智高勢益張。先是,魏瓘築州城,鑿井畜水,作大弩為守備。至是,智高為雲梯土山,攻城甚急,又斷流水,而城堅,井飲不竭,弩發,中輒洞潰,智高力屈。會知英州蘇緘屯兵邊渡村,扼其歸路;番禺縣令蕭注募土丁
 及海上強壯二千餘人,與智高眾格鬥,焚其戰艦;轉運使王罕亦自外至,益修守備。智高知不可拔,圍五十七日,七月壬戌,解去。由清遠濟江,擁婦女作樂而行,遇張忠戰於白田,忠死之。去攻賀州,不克,夜害蔣偕於太平場。九月庚申,破昭州,害王正倫等於館門驛。州之山有數穴,大可容數百千人,民聞兵至,走匿其中,智高知之,縱火,皆焚死。十月丁丑,破賓州。甲申,復據邕州,日夜伐木治舟楫,揚言復趨廣州。十二月壬申,又敗陳曙於金
 城驛。初,智高以反聞,朝廷命曙就擊之,既而楊畋、曹修、張忠、蔣偕相繼出,又以余靖、孫沔為安撫使。畋、修聞智高至,退軍避之。忠、偕勇而無謀,皆死。智高益自恣,南土騷然。仁宗以為憂,命狄青為宣撫使,諸將皆受青節制。曙恐青至有功,亟挑戰,故敗。



 五年正月,青及沔、靖會兵賓州,官軍、土丁合三萬一千餘人,按軍法誅曙及指揮使袁用等三十二人於坐,一軍大振。於是進兵,青將前陣,沔將次陣,靖將後陣,以一晝夜絕昆侖關歸仁鋪。智
 高聞王師絕險而至,出其不意,悉眾來拒,執大盾、摽槍,衣絳衣,望之如火,青陣少卻,先鋒孫節死之。青起麾蕃落騎兵,張在左翼出其後交擊,左者右,右者左,已而左者復左,右者復右,其眾不知所為,大敗走。會日暮,智高復趨邕州,夜焚城遁,由合江口入大理國。得屍五千三百四十一,築為京觀,所掠生口萬餘人,復其業。獲偽印九,黃師宓而下偽官五十七人,梟其首城上,收馬牛、金帛以钜萬計。智高自起兵幾一年,暴踐一方,如行無人
 之境,吏民不勝其毒。朝廷為下赦令,優除復,慰拊瘡痍,百姓始得更生雲。先是,謠言「農家種,糴家收。」已而智高叛,為青破,皆如其謠。



 智高母阿儂有計謀,智高攻陷城邑,多用其策,僭號皇太后,性慘毒,嗜小兒肉,每食必殺小兒。智高敗走,阿儂入保特磨,依其夫儂夏卿,收殘眾得三千餘人,習騎戰,復欲入寇。至和初,余靖督部吏黃汾黃獻珪石鑒、進士吳舜舉發峒兵入特磨,掩襲之,獲阿儂及智高弟智光、子繼宗繼封,檻至京師。初未欲殺,
 日給食飲,欲以誘出智高,或傳智高死,乃悉棄市。既而西川復奏智高未死,謀寇黎、雅州,詔本路為備。御史中丞孫抃又請敕益州先事經制,以安蜀人。然智高卒不出,其存亡莫可知也。



 儂氏又有宗旦者,知雷火洞,稍桀黠。嘉祐二年,嘗入寇,知桂州蕭固招之內屬,以為忠武將軍,又補其子知溫悶峒日新為三班奉職。七年,宗旦父子請以所領雷火、計城諸峒屬縣官,願得歸樂州,永為王民。詔各遷一官,以宗旦知順安州,仍賜耕牛、鹽彩。
 是歲,儂夏卿、儂平、儂亮亦自特磨來歸,皆其族也。日新後嘗監邕州稅。治平中,宗旦與交阯李日尊、劉紀有隙,畏逼,知桂州陸詵因使人說之,遂棄其州內徙,命為右千牛衛將軍。



 有甲峒蠻者,亦役屬交阯,間出寇邕州。景祐三年,嘗掠思陵州憑祥峒生口,殺登龍鎮將而去。嘉祐五年,合交阯、門州等蠻五千餘人復為寇,與官兵拒戰,斬首數百。詔知桂州蕭固趨邕州發諸郡兵,與轉運使宋咸、提
 點刑獄李師中合議追討。是歲數入寇,又詔安撫使余靖擊之。蘇茂州蠻亦近邕州,至和、嘉祐中,皆嘗擾邊。



 黎洞,唐故瓊管之地,在大海南,距雷州泛海一日而至。其地有黎母山,黎人居焉。舊說五嶺之南,人雜夷獠,朱崖環海,豪富兼並,役屬貧弱;婦人服緦緶,績木皮為布,陶土為釜,器用瓠瓢;人飲石汁,又有椒酒,以安石榴花著甕中即成酒。俗呼山嶺為「黎」,居其間者號曰黎人,弓刀未嘗去手。弓以竹為弦。今儋崖、萬安皆與黎為境,其
 服屬州縣者為熟黎,其居山洞無征徭者為生黎,時出與郡人互市。



 至和初,有黎人符護者,邊吏嘗獲其奴婢十人,還之。符護亦嘗犯邊,執瓊、崖州巡檢慕容允則及軍士,至是,以軍士五十六人與允則來歸。允則道病死,詔軍士至者貸其罪。



 乾道二年,從廣西經略轉運司議,詔「海南諸郡倅守慰撫黎人,示以朝廷恩信,俾歸我省地,與之更始。其在乾道元年以前租賦之負逋者,盡赦免之。能來歸者,復其租五年。民無產者,官給田以耕,亦
 復其租五年。守倅能慰安黎人及收復省地者,視功大小為賞有差,失地及民者有重罰。六年,黎人王用休為亂,權萬安軍事、同主管本路巡檢孫滋等招降之。九年八月,樂昌縣黎賊劫省民,焚縣治為亂,黎人王日存、王存福、陳顏招降之。瓊管安撫司上其功,得借補承節郎。



 淳熙元年,詔承節郎王日存子孫許襲職。四年冬,萬安軍王利學寇省地,蓋旻進率眾拒之,兵弱戰沒。八年六月,詔三十六峒都統領王氏女襲封宜人。初,王氏居化
 外,累世立功邊陲,皆受封爵。紹興間,瓊山民許益為亂,王母黃氏撫諭諸峒,無敢從亂者,以功封宜人。至是,黃氏年老無子,請以其女襲封,朝廷從之。十二年正月,樂會縣白沙峒黎人王邦佐等率賊眾五百為寇,殺掠官軍,保義郎陳升之撫降其眾,俘獲林智福等,瓊管司上其功,詔減升之三年磨勘。十六年,詔以大寧砦黃弼補承信郎,彈壓本界黎峒。瓊管司言弼沉鷙有謀,為遠近推服,故用之。弼,宜人黃氏侄也。



 嘉定九年五月,詔宜人
 王氏女吳氏襲封,統領三十六峒。



 環州蠻區氏,州隸宜州羈縻,領思恩、都亳二縣。



 有區希範者,思恩人也。狡黠頗知書,嘗舉進士,試禮部。景祐五年,與其叔正辭應募,從官軍討安化州叛蠻。既而希範擊登聞鼓求錄用,事下宜州,而知州馮伸己言其妄,編管全州。正辭亦嘗自言功,不報。二人皆觖望。希範後輒遁歸,與正辭率其族人及白崖山酋蒙趕、荔波洞蠻謀為亂,將殺伸己,且曰:「若得廣西一方,當建為大唐國。」會
 有日者石太清至,因使之筮,太清曰:「君貴不過封侯。」乃令太清擇日殺牛,建壇場,祭天神,推蒙趕為帝,正辭為奉天開基建國桂王,希範為神武定國令公、桂州牧,皆北向再拜,以為受天命。又以區丕績為宰相,餘皆偽立名號,補置四十餘人。



 慶曆四年正月十三日,率眾五百破環州,劫州印,焚其積聚。以環州為武城軍,又破帶溪砦,下鎮寧州及普義砦,有眾一千五百。宜州捉賊李德用出韓婆嶺擊卻之,前後斬獲甚眾,俘偽將二。希範懼,
 入保荔波洞,間出拒官軍。朝廷下詔購之,獲希範、正辭及趕者,人賜袍帶、錢三十萬、鹽千斤。



 明年,轉運使杜杞大引兵至環州,使攝官區曄、進士曾子華、宜州校吳香誘趕等出降,殺馬牛具酒,紿與之盟,置曼陀羅花酒中,飲者皆昏醉,稍呼起問勞,至則推仆後廡下。比暮,眾始覺,驚走,而門有守兵不得出,悉擒之。後數日,又得希範等,凡獲二百餘人,誅七十八人,餘皆配徙。仍醢希範,賜諸溪峒,繢其五藏為圖,傳於世,餘黨悉平。



 鎮寧州亦隸
 宜州。景祐二年,蠻酋莫陵等七百餘人內寇,遣西京作坊使郭誌高、閣門祗候梁紹熙往討,未至,陵等詣桂、宜州巡檢李仲政請降。廣西轉運使不俟詔,貸其罪。詔劾之,已而釋之。



 是歲,高、竇州犾獠陳友朋等亦寇海上,本路會兵擊之,潰去。



\end{pinyinscope}