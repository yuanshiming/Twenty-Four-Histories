\article{列傳第一百六十三}

\begin{pinyinscope}

 ○汪若海張運柳約李舜臣孫逢吉章穎商飛卿劉穎徐邦憲



 汪若海,字東叟,歙人。未弱冠,游京師,入太學。靖康元年,
 金人侵擾,朝廷下詔求知兵者,若海應詔,未三刻而文成,擢高等。時已割河北地。其年冬,再犯京師,若海謂:「河北國家重地,當用河北以攬天下之權,不可怯懦以自守,閉關養敵,坐受其敝。」屬康王起兵相州,乃上書樞密曹輔,請立王為大元帥,擁兵鎮撫河北,以掎金人之後,則京城之圍自解。輔大喜,即以其書進欽宗,用為參謀,遣如康王所。宰相何㮚執異議,以道梗為辭,不果遣。



 京城失守,若海述麟為書以獻。及二帝北行,袖書抗粘罕,
 請存趙氏。縋而出,謁康王於濟州,謂神器久虛,異姓僭竊,宜蚤即位,以圖中興。一日間三被顧問,補修職郎,充帳前差使。高宗既即位,推恩改承奉郎,遷江南經制使,轉承事郎,監登聞檢院。五府交闢,改屬右府。



 朝廷以張浚宣撫川、陜,議未決。若海曰:「天下者,常山蛇勢也,秦、蜀為首,東南為尾,中原為脊。今以東南為首,安能起天下之脊哉?將圖恢復,必在川、陜。」乃往見浚,極談終日,浚大驚,闢以自隨,以親老辭。繼論軍食,迕執政,通判沅州,以
 讒奪籍,謫英州。道出臨川,時節制江夏軍馬李允文擁眾數十萬,跋扈不用朝命,朝廷命招討使張俊屯江西,參謀官湯東野與若海故,得若海道中,喜甚。謂曰:「李允文懷反側,非君莫能開其自新。」若海即馳往,諭以成敗逆順,示以朝廷威德,復談三策以動之,辭旨明暢。允文大感悟,即舉軍東下。



 若海復為書招其徒張用、曹成、李宏、馬友同歸朝廷。用一見,以其眾二十萬解甲效順,惟成疑貳有他志,若海移書責之。成怒,將殺若海,若海夜宿
 王林軍帳,以計得林軍印,遂奪其眾五千人。翼日,成遂遁。若海遺宏書,使刺成以自歸;宏得書圖成而力不勝,復走長沙刺友,群盜解散。若海遂以林五千人歸招討使張俊,俊乃班師凱旋,軍容愈盛。



 時朝廷方出師,若海以為為國家者,當化盜賊為我用,不可失英雄為國患。因獻平寇策,朝廷悉用之。其後李宏為劉忠所並,死長沙;劉忠為韓世忠所破,走劉豫;曹成走廣而復降,湖湘遂安。尋復承務郎、監潭州南嶽廟、通判辰州。



 紹興九年,
 復三京,祗謁陵寢,事還,以前功,旬月四遷至承議郎、通判順昌府。金人奄至,太尉劉錡甫至,眾不滿三萬,遣人丐援於朝,無敢往者。若海毅然請行,具述錡明方略,善用兵,以偏師濟之,必有成功,朝廷從之,金兵果敗去。闢淮北宣撫司主管機宜文字。拓皋之役,復以勞兩轉至朝散郎、通判洪州,未上,丁內艱。服除,添差通判信州。秩滿,遷湖北帥司參議。知道州,陛辭得對,上曰:「久不見卿,卿向安在?」授直秘閣、知江州,丁父憂。時方經略中原,朝
 廷議起若海,而若海死矣。



 若海豁達高亮,深沈有度,恥為世俗章句學,為文操紙筆立就,蹈厲風發。高宗嘗以片紙書若海名諭張浚曰:「似此人材,卿宜收拾。」會浚去國,不果召。



 張運,字南仲,信之貴溪人,唐宰相文瓘之後。父貫,右通直郎,累贈太中大夫。運年二十五,以太學生登宣和三年進士第,賜同上舍出身,調桂陽監藍山縣丞。縣闕令,運攝縣事。縣與諸獠接壤,因俗為治,吏民安之。臨武寇
 與諸獠合,大剽掠,運親帥兵禽之。遷潭州攸縣尉。高宗南渡,劇賊王在據岐山,潭帥徵兵戍岳,運將二千人先至嶽。賊平,改臨江新淦丞。縣新被兵,令不能支,沿江撫諭使張匯劾罷之,以運攝縣事。運撥煨燼,考版籍,正租賦,數月之間,敝除而民定。



 紹興五年,通判鼎州。賊楊麼、黃誠擁眾數萬,殘破城邑,跳梁湖北。高宗遣張浚以都督董師,岳飛以招討舉兵擊之,賊率輕銳徑趨武溪南興,以臨鼎州,城中大震。運與太守程昌宇勒兵登城,控
 扼上下,以張其勢,賊宵潰。澧賊雷德進柵險稱亂,帥檄運討之。運將都統梁吉等率兵直搗其巢,破四十二柵,降其眾。



 移貳濡須。金人犯廬、壽等州,大將駐兵淮壖以拒之,運給餉未嘗乏絕。歲餘,以親老還江東,寓居鄱。既而丁母及父憂,服除,起知桂陽監。五月而境內稱治,與部使者奏升監為軍。大修庠序之教,祠漢以來守令有功德於桂陽者衛颯、唐羌等七人於學,刻《續顏氏家訓》、《四時纂要》等書,散之民間,使之修德而務本。召入對,除
 知達州。方大旱,入境而雨。奏除病民五事。



 召為度支郎中。臨安樓店務錢歲三十餘萬緡,請以十萬歸省額。戶部所儲三佛齊國所貢乳香九萬一千五百斤,直可百二十餘萬緡,請分送江、浙、荊湖漕司賣之,以糴軍餉。及陳諸路綱運七弊,懲革十術,遠近遞輸以均勞逸。事皆施行。兼樞密院檢詳,遷軍器監。尋改大理少卿,請正兩浙鹽法,以寬私鬻之禁。紹興永裕、昭慈二陵官地與民犬牙相入,請縣重價聽民持券獻納,以免誤犯之罪。尤
 明於治獄,獄為之空。



 拜刑部侍郎,言:諸斥逐累赦未還者,宜從湔洗。諸申請條制,多重復牴牾,失於太煩。諸編置不以赦原、不以蔭論之類,失於太重。外路刑獄三經翻異,移送大理,刀鋸數施,非所以示遠。及諸不便。皆從之。又請廣儲蓄,興鼓鑄,修屯田,作鄉兵。亦皆聽納。兼權戶部侍郎。時久雨傷蠶麥,及邊報有警,詔侍從臺諫陳弭災禦侮之策。運言:「天災人事,有甚可畏而不足畏者,視吾政之修不修;有甚可憂而不足憂者,視吾自治之
 善不善。」及「宜邊淮建三大鎮以守之」。



 會金人渝盟,特遷戶部侍郎,以專饋餉。丞相陳康伯議遣李寶自四明控制海道,眾論紛紜,運直入贊決,以為上策,金人果敗走。因上疏:「乞降詔撫將士,蠲租賦,遣信使,結豪傑,堅城守,督漢中將士趨關陜以制其後。置四鎮三帥於兩淮、襄漢之間以為內固,以圖進取。」以御營隨軍都轉運使從上勞師江上。及駕還,因入對,固請補外。乃授集英殿修撰,出知太平州。當兵饑疾癘之餘,殫勞徠安輯之方,嚴
 斥堠攻守之備。理財賦,造戰艦,繕甲兵,申禁令,民賴以安。



 孝宗既受禪,運亦請老,以敷文閣待制提舉江州太平興國宮,尋授廣東經略,不赴,乃復祠祿。乾道七年,鄱大饑,運首發粟二千石以振之,自是民爭出粟以濟。連上章致政,不許,以疾卒。贈少師、左光祿大夫,官其後三人。嘉定六年,贈開府儀同三司。



 柳約字元禮,秀州華亭人。大觀三年上舍進士,試中學官,為霸州教授。徙睦州,入為闢雍正。遷博士,改宣議郎,
 充廣親宅宗子博士。約深於經學,屬辭粹微,大為學者師慕。提舉福建鹽事,召對,論內外學政,次乞罷內外官到堂日投牒求官,以厚風俗。授秘書省校書郎,進著作佐郎、徽州司錄,改通判宿州,召拜監察御史。靖康初,兼權殿中侍御史,論三鎮不可棄。改尚書工部員外郎,進左司員外郎。父憂去官,服除,以直顯謨閣充御營司參謀官,遷太常少卿。



 高宗將幸平江,約疏言「兵可進,毋退以示怯於敵。」乃以直龍圖閣知臺州,未赴,徙嚴州,兼浙西
 兵馬都監、節制管內軍馬。當是時,金人大入,杜充擁眾北去,列郡震恐,莫有奔問官守者。約於橫潰中屹保孤城,悉力捍禦。境內按堵,則慨然上書,請糾合諸郡克復吳會。上嘉其忠,進右文殿修撰,守郡如故。詔以軍興費出無藝,吏慢弗虔,柳約獨謹賦輸,率先程督,進秩一等。又詔:「約郡當兵沖,而能不辭難、不避事,益嚴列柵,保綏一方,朕甚嘉之。其以約充集英殿修撰。」召入對,獎勞再三,擢權戶部侍郎。



 約於是感激盡言,凡例外宣索,皆執
 奏不進。論「吳並等罪未正,非所以厲臣節。諸大將提兵入覲,各名其家,將有尾大不掉之患」。皆人不敢言者。又言:「軍興科需百出,望官戶名田過制者,與編戶均一科敷。請增諸路酒錢,其半令提刑司樁管,以備軍費。」皆從之。會高麗請修貢,議遣使報聘,上顧廷臣無出約右,加試戶部侍郎充其選,且將大用。當路忌之,諷言者誣以事,罷為提舉太平觀。居七年,復秘閣修撰。



 金人歸侵疆,起知蔡州,被命而往,一無顧避。既而金人渝平,傳檄河
 南,守臣皆舉城降,約獨遣使數輩於武昌,得報而後返。未幾,以敷文閣待制食祠祿。十有五年,卒。贈四官。



 約天性至孝,母病甚,泣禱於天,願損壽以益親壽。母尋愈,約竟先母兩月卒。



 李舜臣,字子思,隆州井研人。生四年知讀書,八歲能屬文,少長通古今,推跡興廢,洞見根本,慨然有志於天下。



 紹興末,張浚視師江、淮,舜臣應詔上書,言:「乘輿不出,無以定大計,宜徙幸武昌。」又謂:「江東六朝皆嘗取勝北方,
 不肯乘機爭天下,宜為今日監。」著《江東勝後之鑒》十篇上之。中乾道二年進士第。時朝廷既罷兵,而為相者益不厭天下望。舜臣對策,論金人世仇,無可和之義,宰輔大臣不當以奉行文字為職業。考官惡焉,絀下第,調邛州安仁縣主簿。歲大侵,饑民千百持鉏棘大呼,響震邑市,令懼閉門。舜臣曰:「此非盜也,何懼為?」亟出慰勞遣之。



 教授成都府。時虞允文撫師關上,闢置幕府,用舉者改宣教郎、知饒州德興縣,專尚風化。民有母子昆弟之訟
 連年不決,為陳慈孝友恭之道,遂為母子兄弟如初。間詣學講說,邑士皆稱「蜀先生」。罷百姓預貸,償前官積逋逾三萬緡。民病差役,舜臣勸糾諸鄉,以稅數低昂定役期久近為義役。期年役成,民大便利。銀坑罷雖久,小戶猶敷銀本錢,官為償之。天申大禮助賞及軍器所需,皆不以煩民。



 乾辦諸司審計司,遷宗正寺主簿,重修《裕陵玉牒》。當曾布、呂惠卿初用,必謹書,或謂非執政除免,格不應書。舜臣曰:「治忽所關,何可拘常法。」他所筆削類此。
 尤邃於《易》,嘗曰:「《易》起於畫,理事象數,皆因畫以見,舍畫而論,非《易》也。畫從中起,乾坤中畫為誠敬,坎離中畫為誠明。」著《本傳》三十三篇。朱熹晚歲,每為學者稱之。所著書《群經義》八卷、《書小傳》四卷、《文集》三十卷、《家塾編次論語》五卷、《鏤玉餘功錄》二卷。子心傳、道傳、性傳。以性傳官二府,贈太師、追封崇國公。



 孫逢吉,字從之,吉州龍泉人也。隆興元年進士第,授郴州司戶。乾道七年,太常黃鈞薦於丞相虞允文、梁克家,
 將處以學官,逢吉竟就常德教授以歸。李燾、劉珙、鄭伯熊、劉焞相繼薦之,知萍鄉縣,以治最聞。除諸軍審計司、國子博士。遷司農寺丞兼實錄院檢討官。紹熙元年,遷秘書郎兼皇子嘉王府直講。



 二年春二月,雷雪之沴交作,詔求直言,疏八事:去蔽諛,親講讀,伸論駁,崇氣節,省用度,惜名器,拔材武,飭戎備。擢為右正言,建言:「都城之民,安居憚徙。宗戚營繕浸廣,每建一第,撤民居數百,咨怨者多。」時親王方更造樓觀未已,聞之,亟令罷役。浙漕
 沈詵見逢吉,謝曰:「非正言,漕計殆不可支。」初,工部侍郎兼知臨安府潘景珪結貴幸以進,司諫鄧馹屢疏其罪,景珪反以計傾之,除馹匠監。逢吉曰:「優遷其官而罷言職,後來者且以言為戒。」兩疏乞收馹新命,不報;並劾景珪脅持臺諫,蔑視朝綱,景珪遂罷。在諫垣七十日,章二十上,詞旨剴切,皆人所難言者。改國子司業,求去,為湖南提刑。以秘書監召,兼吏部侍郎。俄為孝宗攢宮按行事。



 朱熹在經筵持論切直,小人共不便,潛激上怒,中批
 與祠。劉光祖與逢吉同在講筵,吏請曰:「今日某侍郎輪講,以疾告,孫侍郎居次,請代之。」逢吉曰:「常所講《論語》,今安得即有講義?」已而問某侍郎講義安在,取觀之,則講《詩權輿篇》刺康公與賢者有始而無終,與逐朱熹事相類,逢吉欣然代之講。因於上前爭論甚苦。上曰:「朱熹言多不可用用。」逢吉曰:「熹議祧廟與臣不合,他所言皆正,未見其不可用。」浸失上意。



 會彭龜年論韓侂胄專僭,出補郡。逢吉入疏曰:「道德崇重,陛下所敬禮者無若朱熹,志
 節端亮,陛下所委信者無若彭龜年。熹既以論侂胄去,龜年復以論侂胄絀,臣恐賢者皆無固志。陛下所用皆庸鄙憸薄之徒,何以立國?」侂胄見而惡之。丞相趙汝愚既罷,侂胄專國。一日從臣扈從重華宮,上行禮畢,駕興,扈從者出宮門上馬,忽傳呼侂胄至,扈從者卻入,斂板甚恭。逢吉曰:「既出復入揖,臣子事君父之禮當如是耶?」不揖而去。



 會部中會食,吏密報優人王喜除閣職。逢吉即言:「於上前效朱侍講進趨以儒為戲者,豈可令污閣
 職?」即抗疏力爭之。同列密以告侂胄。時王喜之命實未出,遂以誣詆,出知太平州。丐祠,提舉江州太平興國宮。起知贛州,已屬疾,卒,謚獻簡。弟逢年、逢辰,皆有文學行義,時稱「孫氏三龍。」



 章穎,字茂獻,臨江軍人。以兼經中鄉薦。孝宗嗣服,下詔求言,穎為萬言書附驛以聞,禮部奏名第一,孝宗稱其文似陸贄。調道州教授,作周敦頤祠。會宜章寇為亂,郡僚相繼引去,穎獨留。寇平,郡守以功入為郎,奏穎有協
 贊之功,可大用。乃召對,除太學錄。禮部正奏第一人,初任即召對者自穎始。時樞密都承旨王抃以言者奉外祠。穎復言其風金使過求,欲己任調護以為功。孝宗謂其言太訐,久之不遷。及奏考試官,孝宗曰:「章穎可。」乃知上猶記其讜論也。頃之,遷太學博士。丁內艱,服闋,添差通判贛州,除太常博士。



 御史中丞何澹聞繼母訃,引不逮事之文,穎定議解官,澹猶未決去,乞下侍從朝列集議。太學諸生攻之曰:「朝廷專設奉常,議禮之所由出也。
 今不從議禮所由出之地,反以議禮不公,而欲侍從朝列集議,豈將啟逢迎希合,而為茍留進身之計乎?」除左司諫,時左相留正去,右相葛邲當國,穎論邲不足任大事,凡二十餘疏。從官議欲超除穎,俾去言職,庶可兩留。光宗曰:「是好諫官,何以遷之?」邲始出。穎屢疏請上問安重華宮,悉焚其稿。



 寧宗即位,除侍御史兼侍講,尋權兵部侍郎。韓侂胄用事,穎侍經幃。上曰:「諫官有言及趙汝愚者,卿等謂何?」同列謾無可否,穎奏言:「天地變遷,人情
 危疑,加以敵人嫚侮,國勢未安,未可容易進退大臣,願降詔宣諭汝愚,無聽其去。」不報。奏請待罪,與郡;御史劾穎阿黨,罷。太學生周端朝等六人伏闕,辨汝愚被誣,且謂章穎言發於忠,首遭斥逐。端朝等皆被罪,自是黨論遂起矣。



 穎家居久之,起知衢州,侍御史林行可劾罷之。尋知贛州,御史王益祥復劾,寢其命,再祠,需次知建寧府。侂胄誅,除集英殿修撰。累遷刑部侍郎兼侍講,對延和殿,上嘆曰:「卿為權臣沮抑甚久。」穎乞修改《甲寅龍飛
 事跡》誣筆。除吏部侍郎,尋遷禮部尚書,升侍讀。詔穎以紹熙、慶元譙令憲《玉牒辨誣》,餘端禮、趙彥逾《甲寅龍飛記》及趙汝愚當時所記事,考訂削誣,從實上之。丐去,奉祠。以嘉定十一年卒,年七十八。



 穎操履端直,生平風節不為窮達所移。雖仕多偃蹇,而清議與之。方黨論之興,朱熹遣以書,略曰:「世道反覆,已足流涕;而握其事者怒猶未已,未知終安所至極耶?然宗社有靈,公論未泯,異日必有任是責者,非公吾誰望耶?」贈光祿大夫,謚文肅。



 商飛卿,字翬仲,臺州臨海人。淳熙初,由太學登進士第,任無為軍教授,累官至工部郎官。時韓侂胄柄國,氣焰薰灼,飛卿既至,未嘗輒一造請,逾月即丐去,提舉福建路常平茶鹽事。擢監察御史,以言事迕侂胄,罷為奉常。請外,以秘閣修撰為荊湖南路轉運判官。後改司農卿,總領江東、淮西軍馬錢糧。金陵故有帥、漕治所,合戎騎二帥、留鑰、內侍,號六司,宴飲饋遺,費動萬計。飛卿以身率儉,節縮浮苛,糧餉時斂散,稍稍以裕聞。開禧中,就擢
 戶部侍郎。侂胄將舉師,嘗問餉計豐約,飛卿以實告。比調遣浩繁,不克支,屬有旨俾飛卿軍前傳宣撫勞,值金兵大至,幾不免,以憂卒。



 劉穎,字公實,衢州西安人。紹興二十七年進士,調溧陽主簿。時張浚留守建康,金師初退,府索民租未入者,穎白浚言:「師旅之後,宜先撫摩,當盡蠲逋賦。」浚喜,即奏閣免,由是知之,遣其子栻與游。教授全州,改官知鉛山縣,以外艱去。再知常熟縣,簽判潭州。王佐為帥,負其能,盛
 氣以臨僚吏,穎約以中道,多屈而改為。及陳峒反,所擒賊多穎計策,帥上其功,曰:「簽判宜居臣上。」召監進奏院,進太常寺主簿,遷丞,兼兵部郎官。



 提舉浙西常平茶鹽,還澱山湖,以洩吳松江,二水禁民侵築,毋使逼塞大流,民田賴之。就遷提刑,以洗冤澤物為任,間詣獄,察不應系者縱遣之。御史以介僻劾罷。除江西運判。江州德化縣田逃徙太半,守乞蠲稅,不報。穎以見種之稅均於荒萊,民願耕者第減之,上供自若,而逃田盡復。



 除直秘閣、
 淮東轉運副使。初,水敗楚州城,修補未竟,劉超欲移築,穎因接伴金國使,入對言:「國家何苦捐百萬緡為軍帥幸賞地邪?」光宗從之。除戶部郎中、淮東總領。務場以額鈔抵賞,陰耗餉計,二十年無知此弊者,穎究核得之,以所賣數論賞而總餉增羨,遷司農少卿、淮西總領。前主計者請自為都釀,抱凈息而利贏餘,其後稍虧,反以大軍錢佐之,邀糴江、淮,回易如負販狀。穎以為失王人之體,遂罷之。內府宣限既迫,每移供軍錢以應歲輸。穎搜
 吏弊,汰冗員,分月綱解,自是不復那移。



 尋除直寶謨閣、江東運副、知平江府,皆未行。除宗正少卿,遷起居郎兼實錄院檢討官,權戶部侍郎,升同修撰。以疾丐祠,提舉興國宮。除集英殿修撰、知寧國府,改知紹興府。未幾,知平江府,徑歸,提舉興國宮。起知泉州,升華文閣待制,請興國祠以歸。興國祠滿,除敷文閣待制致仕。嘉定改元,召赴行在,落致仕,除刑部侍郎,辭,進龍圖閣待制、知婺州。請老,以寶謨閣直學士致仕。六年,卒於家,年七十八。
 贈光祿大夫。



 在孝宗朝,人臣爭承意自獻。穎奏:「今日之失在輕聽人言,昔之施為,今復棄置,大損盛德。」孝宗嘉納之。光宗時,論人主難克而易流者四:曰逸豫無節,賜予無度,儒臣易疏,近幸易暱。寧宗時,學禁初起,黨論日興。穎奏:「願陛下御之以道,容之以德,不然,元祐、崇、觀之事可鑒也。」其言皆切中於時。



 自浙西請外,凡徙麾節十餘年,有以淹速訊之,穎笑曰:「吾所欲也。」其在從班日,韓侂胄舊與周旋無間,方居中用事,而穎謝絕之。常言:「士
 以不辱身為重。」其為少宗正,而丞相趙汝愚適歸,相遇於廢寺,泥雨不能伸足,但僧床立語曰:「寄謝餘參政,某雖去而人才猶在朝遷,幸善待之。」穎曰:「相公人才即參政人才也,使果賢,參政之責,非宰相之憂也。」餘參政,端禮也。餘繼相,卒於善類多所全祐,穎之助云。



 徐邦憲,字文子,婺州義烏人。幼穎悟,從陳傅良究名物義理,以通史傳百家之書。紹熙四年,試禮部,第一人登進士第。三遷為秘書郎。



 韓侂胄開兵端,同惡附和,無敢
 先發一語議其非者,邦憲獨首言之。丐外,知處州,陛辭,力諫用兵不可太驟。再歲召還,言:「求名義以息兵,莫若因建儲而肆赦,借殊常之恩,為弭兵之名,因行赦宥,大霈德澤。東委宣諭,西委宣撫,洗弄兵之咎,省戍邊之師;發倉粟以賑餓殍,及農時而復民業。如此則建儲之義,正與息兵相為表裏也。」



 又上侂胄書,侂胄惡其言,嗾御史徐柟擊之,鐫秩罷祠。未幾復官,除江西憲,改江東漕,以戶部郎為淮西總領。侂胄已誅,尚書倪思舉邦憲自
 代。召對,上言:「今日更化,未可與紹興乙亥同論。秦檜專權,天下猶可以緝理,今侂胄專權,天下敗壞盡矣。」除尚右郎兼太子侍講,除左司,為金賀正使接伴。除宗正少卿,回權工部侍郎、知臨安府。丐祠,知江州,奏乞郡,得節制屯戍兵,至郡疾,以寶謨閣待制致仕,卒於官,年五十七,謚文肅。



 論曰:汪若海、柳約仕於南渡播遷之時,其志將以尊君父,故讀其《麟書》而悲之。張運、李舜臣職舉事修,遺愛在
 民。孫逢吉、章穎辨正人之非邪,正學之非偽,君子哉!商飛卿、劉穎、徐邦憲皆有立於權臣柄國之日,卓乎不為勢利所移,故能爾耶!



\end{pinyinscope}