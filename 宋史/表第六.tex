\article{表第六}

\begin{pinyinscope}

 宗室世系一



 昔者,帝王之有天下,莫不眾建同姓,以樹蕃屏,其不得以有國者,則亦授之土田,使帥其宗氏,輯其分族。故繼別之宗百世不遷,豈惟賴其崇獎維持以成不拔之基哉。蓋親親之仁,為國大經,理固然也。《周官》宗伯掌三族之別以辨親疏,於是敘昭穆之法隆殺行焉。此世系之所以不可不謹也。後世封建廢而宗法壞,帝王之裔,至或雜於民伍,淪為皂隸,甚可嘆也。宋太祖、太宗、魏王之子孫可謂藩衍盛大矣,支子而下,各以一字別其昭穆,而宗正所掌,有牒、有籍、有錄、有圖、有譜,以敘其系,而第其服屬之遠近,列其男女昏因及官爵敘遷,而著其功罪生死歲月,雖封國之制不可以復古而宗法之嚴,恩禮之厚,亦可概見。然靖康之變,往往淪徙死亡於兵難,南渡所存十無二三,而國之枝葉日以悴矣。今因載籍之舊,考其源委,作《宗室世系表》。



 太祖四子:長滕王德秀,次燕王德昭,次舒
 王德林,次秦王德芳。德秀、德林無後。



 表略



\end{pinyinscope}