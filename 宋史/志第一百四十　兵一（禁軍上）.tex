\article{志第一百四十 兵一(禁軍上)}

\begin{pinyinscope}

 宋
 之兵制,大概有三:天子之衛兵,以守京師,備征戍,曰禁軍;諸州之鎮兵,以分給役使,曰廂軍;選於戶籍或應募,使之團結訓練,以為在所防守,則曰鄉兵。又有蕃兵,
 其法始於國初,具籍塞下,團結以為藩籬之兵;其後分隊伍,給旗幟,繕營堡,備器械,一律以鄉兵之制。今因舊史纂修《兵志》,特置於熙寧保甲之前,而附之鄉兵焉。



 其軍政,則有召募、揀選、廩給、訓練、屯戍、遷補、器甲、馬政八者之目,條分而著之,以見歷朝因革損益之不同,而世道之盛衰亦具是矣。



 嗟乎!三代遠矣。秦、漢而下得寓兵於農之遺意者,惟唐府衛為近之。府衛變而召募,因循姑息,至於藩鎮盛而唐以亡。更歷五代,亂亡相踵,未有
 不由於兵者。太祖起戎行,有天下,收四方勁兵,列營京畿,以備宿衛,分番屯戍,以捍邊圉。於時將帥之臣入奉朝請,獷暴之民收隸尺籍,雖有桀驁恣肆,而無所施於其間。凡其制,為什長之法,階級之辨,使之內外相維,上下相制,截然而不可犯者,是雖以矯累朝藩鎮之弊,而其所懲者深矣。



 咸平以後,承平既久,武備漸寬。仁宗之世,西兵招刺太多,將驕士惰,徒耗國用,憂世之士屢以為言,竟莫之改。神宗奮然更制,於是聯比其民以為保
 甲,部分諸路以隸將兵,雖不能盡拯其弊,而亦足以作一時之氣。時其所任者,王安石也。元祐、紹聖遵守成憲。迨崇寧、大觀間,增額日廣而乏精銳,故無益於靖康之變。時其所任者,童貫也。建炎南渡,收潰卒,招群盜,以開元帥府。其初兵不滿萬,用張、韓、劉、岳為將,而軍聲以振。及秦檜主和議,士氣遂沮。孝宗有志興復而未能。光、寧以後,募兵雖眾,土宇日蹙,況上無馭將之術,而將有中制之嫌。然沿邊諸壘,尚能戮力效忠,相與維持至百五
 十年而後亡。雖其祖宗深仁厚澤有以固結人心,而制兵之有道,綜理之周密,於此亦可見矣。



 禁兵者,天子之衛兵也,殿前、侍衛二司總之。其最親近扈從者,號諸班直,其次總於御前忠佐軍頭司、皇城司、騏驥院。餘皆以守京師、備征伐。其在外者,非屯駐、屯泊,則就糧軍也。太祖鑒前代之失,萃精銳於京師,雖曰增損舊制,而規模宏遠矣。



 建隆元年,詔殿前、侍衛二司各閱所掌兵,揀其驍勇升為上軍,老弱怯懦置剩員以處之。詔
 諸州長吏選所部兵送都下,以補禁旅之闕。又選強壯卒定為兵樣,分送諸道。其後代以木梃,為高下之等,散給諸州軍。委長吏、都監等召募教習,俟其精練,即送闕下。二年,改左右雄捷、左右驍武軍並為驍捷,左右備徵為雲騎,左右平遠為廣捷,左右懷德為懷順。四年,賜河東樂平縣歸降卒元威以下二百六十六人衣服、錢絹有差,立為效順指揮。



 乾德二年,詔遼州降軍宜以效順、懷恩為名。三年四月,詔改西川感化、耀武等軍並為虎
 捷。九月,上御講武殿閱諸道兵,得萬餘人,以騎兵為驍雄,步軍為雄武,並隸侍衛司,且命王繼勛主之,給緡錢俾娶妻。繼勛縱之白日掠人妻女,街使不能禁。帝聞大怒,捕斬者百人,小黃門閻承翰見而不奏,亦杖數十。



 開寶七年,泰寧軍節度使李從善部下及江南水軍凡千三十九人,並黥面隸籍,以歸化、歸聖為額。



 太平興國二年,詔改簇御馬直曰簇御龍直,鐵騎曰日騎,龍捷曰龍衛,控鶴曰天武,虎捷曰神衛,骨金朵子直曰御龍骨朵子
 直,寬衣控鶴曰寬衣天武,雄威曰雄勇,龍騎曰雄猛。八年,改濮州平海指揮為崇武。



 雍熙四年,改殿前司日騎金屈直指揮為捧日金屈直,日騎改為捧日,驍猛改為拱辰,雄勇改為神勇,上鐵林改為殿前司虎翼,腰弩改為神射,侍衛步軍司鐵林改為侍衛司虎翼。



 至道元年,帝閱禁兵有挽強弩至一石五斗,連二十發而有餘力者,顧謂左右曰:「今宇內阜安,材武間出,弧矢之妙,亦近代罕有也。」又令騎步兵各數百,東西列陣,挽強彀弩,視其進
 退發矢如一,容止中節,因曰:「此殿庭間數百人爾,猶兵威可觀,況堂堂之陣數萬成列者乎!」



 咸平三年,詔定州等處本城廳子、無敵、忠銳、定塞指揮,已並升充禁軍馬軍雲翼指揮,依逐州軍就糧,令侍衛馬軍司管轄。定州揀中廳子第一充雲翼第一,第二充雲翼第二;相州廳子第一充雲翼第三,第二充雲翼第四;保州無敵第一充雲翼第五,第二充雲翼第六,忠銳充雲翼第七;威勇軍無敵第一充雲翼第八,第二充雲翼第九,忠銳充雲
 翼第十;靜戎軍無敵充雲翼第十一;寧邊軍無敵充雲翼第十二;北平塞無敵充雲翼第十三;深州無敵充雲翼第十四。北面諸處應管本城、定塞指揮已下鎮定州、高陽關路都總管,並充禁軍馬軍雲翼指揮,才候升立訖,分析逐指揮員兵士人數、就糧州府、本指揮見在去處以聞。



 四年,詔陜西沿邊州軍兵士先選中者,並升為禁軍,名保捷。五年正月,置廣捷兵士五指揮。五月,命使臣分往邠、寧、環、慶、涇、原、儀、渭、隴、鄜、延等州,於保安、保毅
 軍內,與逐處官吏選取有力者共二萬人,各於本州置營,升為禁軍,號曰振武指揮。既而帝曰:「邊防闕兵,朝廷須為制置,蓋不得已也。候邊鄙乂寧,即可銷弭。」六月,以河東州兵為神銳二十四指揮、神虎十指揮,又升石州廳子軍為禁軍,又以威虎十指揮隸虎翼。



 景德四年,詔河東廣銳、神銳、神虎軍以見存為定額,缺則補之。



 大中祥符元年,詔侍衛步軍司閱保寧軍士,分為四等,其第一等徙營亳州永城縣,余聽歸農;無家可還者,隸諸州
 為剩員。四年,宣示永安縣永安指揮兵八千餘人以奉諸陵,其軍額猶隸西京本城廂軍,可賜名奉先指揮,升為禁軍,在清塞之下。八年,置禁軍左右清衛二指揮,在雄武弩手之上,散卒月給鐵錢五百,以奉宮觀。



 仁宗即位,海內承平,而留神武備,始幸安肅教場觀飛山雄武發炮,命捧日、天武、神衛、虎翼四軍為戰陣法,拔其擊刺騎射之精者,稍遷補之。由天聖至寶元間,增募諸軍:陜西蕃落、廣銳,河北雲翼,京畿廣捷、虎翼、效忠,陜西、河
 東清邊弩手,京西、江、淮、荊湖歸遠,總百餘營。



 康定初,趙元昊反,西邊用兵,詔募神捷兵,易名萬勝,為營二十。所募多市井選心耎,不足以備戰守。是時禁兵多戍陜西,並邊土兵雖不及等,然驍勇善戰。京師所遣戍者,雖稱魁頭,大率不能辛苦,而摧鋒陷陣非其所長。又北兵戍及川峽、荊湘、嶺嶠間,多不便習水土,故議者欲益募土兵為就糧。於是增置陜西蕃落、保捷、定功,河北雲翼、有馬勁勇,陜西、河北振武,河北、京東武衛,陜西、京西壯勇,延州
 青澗,登州澄海弩手,京畿近郡亦增募龍騎、廣勇、廣捷、虎翼、步斗、步武,復升河北招收、無敵、廳子馬,陜西制勝,並州克戎、騎射,麟州飛騎,府州威遠,秦州建威,慶州有馬安塞,保州威邊,安肅軍忠銳,嵐、府州建安,登州平海,皆為禁兵,增內外馬步凡數百營。又京東西、河北、河東、江、淮、荊湖、兩浙、福建路各募宣毅,大州二營,小州一營,凡二百八十八。岢嵐軍別置床子弩炮手。時吏以所募多寡為賞罰格,諸軍子弟悉聽隸籍,禁軍闕額多選本
 城補填,故慶歷中外禁、廂軍總一百二十五萬,視國初為最多。西師既罷,上患兵冗,帑庾不能給,乃詔省兵數萬人。



 皇祐二年,川峽增置寧遠。五年,江、淮、荊湖置教閱忠節,州一營,大州五百人,小州三百人。於是宣毅浸廢不復補,而荊湖、廣南益募雄略。至和二年,廣、桂、邕州置有馬雄略。明年,並萬勝為十營。其後,議者謂東南雖無事,不宜馳備。嘉祐四年,乃詔荊南、江寧府、揚、廬、洪、潭、福、越州募就糧軍,號威果,各營於本州。又益遣禁軍駐泊,長
 吏兼本路兵馬鈐轄,選武臣為都監,專主訓練。於是東南稍有備矣。



 七年,宰相韓琦言:



 祖宗以兵定天下,凡有征戍則募置,事已則並,故兵日精而用不廣。今二邊雖號通好,而西北屯邊之兵,常若待敵之至,故竭天下之力而不能給。不於此時先慮而豫備之,一旦邊陲用兵,水旱相繼,卒起而圖之,不可及矣。



 又三路就糧之兵雖勇勁服習,然邊儲貴踴,常苦難贍。若其數過多,復有尾大不掉之患。京師之兵雖雜且少精,然漕於東南,廣而
 易供設,其數多,得強幹弱枝之勢。祖宗時,就糧之兵不甚多,邊陲有事,則以京師兵益之,其慮深而其費鮮。願詔樞密院同三司量河北、陜西、河東及三司榷貨務歲入金帛之數,約可贍京師及三路兵馬幾何,然後以可贍之數立為定額。額外罷募,闕即增補。額外數已盡而營畸零,則省並之。既見定額,則可以定其路馬步軍一營,以若干為額。仍請核見開寶、至道、天禧、慶歷中外兵馬之數。蓋開寶、至道之兵,太祖、太宗以之定天下、服四
 方也。天禧之兵,真宗所以守成備豫也。慶歷之兵,西師後增置之數也。以祖宗之兵,視今數之多少,則精冗易判,裁制無疑矣。



 於是詔中書、樞密院同議。樞密院奏:開寶之籍總三十七萬八千,而禁軍馬步十九萬三千;至道之籍總六十六萬六千,而禁軍馬步三十五萬八千;天禧之籍總九十一萬二千,而禁軍馬步四十三萬二千;慶歷之籍總一百二十五萬九千,而禁軍馬步八十二萬六千。視前所募後浸多,自是稍加裁制,以為定額。



 英宗即位,詔諸道選軍士能引弓二石、弩四石五斗送京師閱試,第升軍額。明年,並萬勝為神衛。三年,京師置雄武第三軍。時宣毅僅有存者,然數詔諸路選廂軍壯勇者補禁衛,而退其老弱焉。蓋治平之兵一百十六萬二千,而禁軍馬步六十六萬三千云。



 熙寧元年十二月,詔:「京東武衛四十二指揮並分隸河北都總管司六指揮,隸大名府路三十六指揮,均隸定州、高陽關兩路更戍;其休番者,選差兵官三人依河北教閱新法訓練,
 仍差使臣押教。」又詔京東路募河北流民,招置教閱廂軍二十指揮,以忠果為額。青、鄆、淄、齊州各三指揮,濟、兗、曹、濮州各兩指揮。



 三年十二月,樞密使文彥博等上在京、開封府界及京東等路禁軍數,帝亦參以治平中兵數而討論焉。遂詔:殿前虎翼除水軍一指揮外,存六十指揮,各以五百人為率,總三萬四百人;在京增廣勇五指揮,共二千人;開封府界定六萬二千人,京東五萬一千二百人,兩浙四千人,江東五千二百人,江西六千八
 百人,湖南八千三百人,湖北萬二千人,福建四千五百人,廣南東、西千二百人,川峽三路四千四百人為額。在京其餘指揮並河東、陜西、京西、淮南路既皆撥並,唯河北人數尚多,乃詔禁軍以七萬為額。初,河北兵籍比諸路為多,其緣邊者且仰給三司,至是而撥並畸零,立為定額焉。是時,京東增置武衛軍,分隸河北四路,後又以三千人戍揚、杭州、江寧府,其後又團結軍士置將分領,則謂之將兵云。



 七年正月,詔頒諸班直禁軍名額:



 殿
 前司諸班:殿前指揮使、內殿直、散員、散指揮、散都頭、散祗候、金槍、東西、招箭、散直、鈞容直。諸直:御龍、御龍骨朵、御龍弓箭、御龍弩直。諸軍:捧日金屈直、捧日左射、捧日、寬衣天武、金屈直天武、左射天武、歸明渤海、拱聖、神勇、吐渾、驍騎、驍勝、宣武、虎翼水軍、寧朔、龍猛、捧日第五軍、捧日第七軍、天武第五軍、天武第七軍、契丹直第一、契丹直第二、神騎、廣勇、步斗、龍騎、驍猛、雄勇、太原府就糧吐渾、潞州就糧吐渾、左射清朔、擒戎、廣捷、廣德、、驍雄、雄威。



 侍
 衛馬軍司龍衛金屈直、龍衛左射、龍衛、恩冀州員僚直、忠猛、定州散員、驍捷、雲騎、武騎、龍衛第十軍、揀中龍衛、新立驍捷、飛捷、驍武、廣銳、雲翼、禁軍有馬勁勇、廳子馬、無敵、克勝、飛騎、威遠、克戎、萬捷、雲捷、橫塞、慶州有馬安塞、蕃落、有馬雄略、員僚剩員直。



 侍衛步軍司神衛、虎翼水軍、神衛第十軍、步武、武衛、床子弩雄武、飛山雄武、神衛、振武、來化、雄武弩手、上威猛、招收、雄勝、澄海水軍弩手、神虎、保捷、捉生、清邊弩手、制勝、定功、青澗、平海、雄武、效
 忠、宣毅、建安、威果、川效忠、揀中雄勇、懷順、懷恩、勇捷、威武、靜戎弩手、忠遠、寧遠、忠節、教閱忠節、川忠節、神威、歸遠、雄略、下威猛、強猛、壯勇、橋道、清塞、武嚴、宣效、神衛剩員、奉先園、揀中六軍、左龍武、右龍武、左羽林、右羽林、左神武、右神武。



 御營喝探、新團立揀中剩員。



 諸班直資次相壓殿前指揮使、御龍直、御龍骨朵子直、內殿直、散員、散指揮使、散都頭、散祗候、金槍、東西班、御龍弓箭直、御龍弩直、招箭班、散直、鈞容直。



 諸軍資次相壓捧日金屈直、捧日左射、捧日、寬衣天武、
 天武金屈直、天武左射、天武、龍衛金屈直、龍衛左射、龍衛、神衛、歸明渤海、拱聖、神勇、恩冀州員僚直、忠猛、定州散員、吐渾、驍騎、驍捷、雲騎、驍勝、宣武、武騎、殿前司虎翼、殿前司虎翼水軍、寧朔、龍猛、步軍司虎翼、步軍司虎翼水軍、捧日第五軍、捧日第七軍、天武第五軍、天武第七軍、龍衛第十軍、揀中龍衛、神衛第十軍、契丹直第一、契丹直第二、神騎、廣勇、步斗、龍騎、驍猛、雄勇、太原府就糧吐渾、潞州就糧吐渾、清朔、擒戎、新立驍捷、飛捷、驍武、廣銳、雲
 翼、禁軍有馬勁勇、步武、武衛、床子弩雄武、飛山雄武、神銳、振武、來化、雄武弩手、上威猛、廳子馬、無敵、招收、雄勝、廣捷、廣德、克勝、飛騎、威遠、澄海水軍弩手、克戎、驍雄、雄威、萬捷、雲捷、橫塞、神虎、保捷、慶州有馬安塞、蕃落、捉生、清邊弩手、制勝、定功、有馬雄略、青澗、平海、雄武、效忠、宣毅、建安、威果、川效忠、揀中雄勇、懷順、懷恩、勇捷、威武、下威武、靜戎弩手、忠勇、寧遠、忠節、教閱忠順、川忠節、神威、歸遠、雄略、下威猛、強猛、壯勇、員僚剩員直、橋道、川橋道、
 步軍司清塞、武嚴、宣效、神衛剩員、奉先園、揀中六軍、御營喝探、新團立揀中剩員。



 諸禁軍名額系捧日、天武、龍衛、神衛為上軍,五百文已上料錢見錢為中軍,不滿五百文料錢見錢並捧日天武第五第七軍、龍衛神衛第十軍、驍猛、雄勇、驍雄、雄威為下軍。元豐五年十月,詔諸路教閱廂軍,於下禁軍內增入指揮名額,排連並同禁軍。蓋熙寧之籍,天下禁軍凡五十六萬八千六百八十八人;元豐之籍,六十一萬二千二百四十三人。



 哲宗即
 位,四方用兵,增戍益廣。元祐元年三月,寄招河北路保甲,充填在京禁軍闕額。龍、神衛以年二十以下,中軍以下以年二十五以下者,雖短小一指並招刺焉。二年,詔西關堡防拓禁軍和雇入役。復置河北、河東、陜西、府界馬步軍。七年,河東、陜西路諸帥府敢勇以一百人為額,專隸經略司。



 紹聖四年,陜西路增置蕃落馬軍。是年,蘭州金城置步軍保捷、馬軍蕃落。



 元符元年,利州路興元府、閬州各增置就糧武寧;又湖北、江東各增置有馬雄
 略。涇原路新築南牟會,賜名西安州,戍守共以七千人為額,仍招置馬軍蕃落、步軍保捷。天都、臨羌砦戍守各以三千人為額,仍各置馬軍蕃落、步軍保捷。永興軍等路創置蕃落;河北大名府等二十二州共創置馬軍廣威、步軍保捷,以河北大水,招刺流民故也。



 二年正月,環慶增置敢勇二百人。四月,環慶路都總管司言:「本路新展定邊城,比之橫山、興平等處城砦尤深,乞增置住營馬軍蕃落、步軍保捷。」六月,環慶路都總管司言:「展築慶
 州白豹城,合增置住營馬步軍。」又鄜延路都總管司言:「本路新築米脂等八堡砦,合增置土兵、馬步軍。」皆從之。三年,樞密院奏:「河北增置馬軍廣威、步軍保捷二萬餘人,欲令揀選升換在京闕額軍分。」從之。自紹聖以來,陜西、河東連用兵六年,進築未已,覆軍殺將,供給不可勝紀。



 徽宗崇寧元年九月,荊湖北路增置禁軍,以靖安名。十月,川陜置安遠軍。三年三月,隴右都護奏:乞於鄯州置水軍,守河浮橋。又樞密院乞增置府界、京東西等路
 步軍,荊湖南路雄略,皆從之。十月,京東西、河東北、開封府界創置馬步軍五萬人,馬軍以崇捷、崇銳名,步軍以崇武、崇威名,合用緡錢二百八十萬有奇,以常平、封樁等錢支,用蔡京之請也。京又言:今拓地廣,戍兵少,當議添置兵額,以為邊備。」從之。



 四年十一月,廣西路置刀牌手三千人,於切要州軍更戌,以寧海名。十二月,詔:「四輔屏翰京師,兵力不可偏重,可各以二萬人為額。」五年,環慶路進築徐丁臺城,賜名安邊,置馬軍蕃落、步軍保捷。



 大觀元年五月,延安置錢監兵。閏十月,靖州置宣節。十一月,兩浙東、西路各增置禁軍。宣和三年,內侍、制置使譚奏,以方臘既平,乞節鎮增添禁軍兩指揮,餘州軍一指揮;又乞除溫、處、衢、婺外,將禁軍更招置成十指揮。又乞增置嚴州威果禁軍。並從之。五年二月,尚書省言:「古者,六軍為王之爪牙,羽林則禁衛之總名也。今臣僚使令兵卒所居營分曰六軍,而復有左、右羽林之名,稱謂失當。若將揀中六軍並六軍指揮並改為廣效,內揀
 中六軍作第一指揮,左龍武第二,左羽林第三,左神武第四,右龍武第五,右羽林第六,右神武第七。」從之。



 靖康元年,詔;「廣西宜、融二州實為極邊,舊置馬軍難議減省,且依元降指揮招置。」



 自元豐而後,民兵日盛,募兵日衰,其募兵闕額,則收其廩給,以為民兵教閱之費。元祐以降,民兵亦衰。崇寧、大觀以來,蔡京用事,兵弊日滋,至於受逃亡,收配隸,猶恐不足。政和之後,久廢搜補,軍士死亡之餘,老疾者徒費廩給,少健者又多冗占,階級既壞,
 紀律遂亡。童貫握兵,勢傾內外,凡遇陣敗,恥於人言,第申逃竄。河北將兵,十無二三,往往多住招闕額,以其封樁為上供之用。陜右諸路兵亦無幾,種師道將兵入援,止得萬五千人。故靖康之變,雖畫一之詔,哀痛激切,而事已無及矣。



 高宗南渡,始建御營司,未幾,復並御營歸樞密院。建炎四年,改御前五軍為神武軍,御營五軍為神武副軍,並隸樞密院。五年,上以祖宗故事,兵皆隸三衙,乃廢神武中軍隸殿前司,於是殿司兵柄始一。乾道元年,
 詔殿前兵馬權以七萬三千人為額。



 諸屯駐大軍則皆諸將之部曲,高宗開元帥府,諸將兵悉隸焉。建炎後,諸大將兵浸盛,因時制變,屯無常所。如劉光世軍或在鎮江、池州、太平,韓世忠軍或屯江州、江陰,岳飛一軍或屯宜興、蔣山,王彥八字軍隨張浚入蜀,吳玠兵多屯鳳州、大散關、和尚原。是時合內外大軍十九萬四千餘,川、陜不與焉。及楊沂中將中軍總宿衛,江東劉光世、淮東韓世忠、湖北岳飛、湖南王□燮四軍共十九萬一千六百,亦
 未嘗有屯。



 紹興十一年,範同以諸將握兵難制,獻謀秦檜,且以柘皋之捷言於上,召張俊、韓世忠、岳飛入覲,張俊首納所部兵。分命三大帥副校各統所部,自為一軍,更銜曰統制御前軍馬。罷宣撫司,遇出師取旨,兵皆隸樞密院,屯駐仍舊。而四川大將兵曰興、成、階、鳳、文、龍、利、閬、金、洋、綿、房、西和州、大安軍、興元、隆慶、潼川府凡十七郡,亦分屯就糧焉。



 乾道之末,各州有都統司領兵:建康五萬,池州一萬二千,鎮江四萬七千,楚州武鋒軍一萬
 一千,鄂州四萬九千,荊南二萬,興元一萬七千,金州一萬一千。其後分屯列戍,增損靡常。所可考者,統制、統領、正將、副將、準備將之目也。



 至於水軍之制,則有加於前者,南渡以後,江、淮皆為邊境故也。建炎初,李綱請於沿江、淮、河帥府置水兵二軍,要郡別置水兵一軍,次要郡別置中軍,招善舟楫者充,立軍號曰凌波、樓船軍。其戰艦則有海鰍、水哨馬、雙車、得勝、十棹、大飛、旗捷、防沙、平底、水飛馬之名。隆興以後至於寶祐、景定間,江、淮沿流
 堡隘相望,守禦益繁,民勞益甚。迨咸淳末,廣東籍蜑丁,閩海拘泊船民船,公私俱弊矣。



 其禁軍將校,則有殿前司都指揮使、副都指揮使、都虞候各一人;諸班有都虞候、指揮使、都知、副都知、押班;御龍諸直有四直都虞候,本直各有都虞候、指揮使、副指揮使、都頭、副都頭、十將、將虞候;馬步軍有捧日、天武左右四廂都指揮使,捧日、天武左右各有都指揮使,每軍有都指揮使、都虞候,每指揮有指揮使、副指揮使,每都有軍使、步軍謂之都頭。



 副兵馬
 使、步軍謂之副都頭。



 十將、將虞候、承局、押官。



 所領諸班直、指揮,騎兵、步兵之額敘列如左。以其前後之異同者分為建隆以來之制、熙寧以後之制,而將兵、水兵之制可考者,因附著於後云。



 建隆以來之制



 騎軍



 殿前指揮使左右班二。宋初,以舊府親從帶甲之士及諸班軍騎中選武藝絕倫者充。



 內殿直左右班四。周制,簡軍校暨武臣子弟有材勇者立。又有川班內殿直,乾德三年平蜀得奇兵,簡閱材貌魁偉便習騎射者凡百二十人立,開寶四年廢。



 散員左右班四。周制,招置諸州豪傑立,散指揮、
 散都頭、散祗候凡十二班。又於北面驍捷員僚直及諸軍內簡閱填補。咸平五年,定州路都部署王超言;「緣邊有強梁輩常居四界,擾動邊境,請厚給金帛募充散員。」從之。



 散指揮左右班四。



 散都頭左右班二。



 金槍班左右班二,舊名內直。太平興國初,改選諸軍中善用槍槊者補之。



 東西班弩手、龍旗直、招箭班共十二,舊號東西班承旨。淳化二年,改為殿前侍,東西各第一第二弩手、龍旗直班六,並帶甲,選諸班及不帶甲班增補。其東第二茶酒及第三、西第四班不帶甲,並以諸軍員、使臣及沒王事者子弟為之。又擇善弓箭者為招箭班。



 散直左右班四。雍熙四年,以諸道募置藩鎮廳頭軍將及詣登聞院求試武藝者立。咸平元年,選諸節度使從人、騎禦馬小底增補。



 鈞容直班二。太平興國三年,選諸軍諳曉音樂、騎禦馬小底立。淳化二年,改之。



 外殿直班一。諸班衛士中年多者號看班外殿直,後削看班之號。或詣
 諸道攝軍校之職部分州兵,謂之權管。國初又有內員僚直,開寶中廢。太平興國四年,徵太原,得上軍。天禧四年,並入此班。



 捧日並左射、金屈直、弩手、左第五軍,總指揮三十五。京師三十三,雍丘、鄭各一。舊號小底,周改為鐵騎,太平興國二年改為日騎,雍熙四年改今名。分左、右廂,各四軍。雍熙三年,選善槍槊者充金屈直。淳化三年,選善左射者為左射。咸平五年,選天武、拱聖、驍騎善弩射者為弩手。



 契丹直三。咸平、許、壽各一。後唐置,旋廢。開寶三年,以遼人內附之眾復置。太平興國中,因事復置,旋廢。



 歸明渤海指揮二。京師。太平興國四年,徵幽州,以渤海降兵立。



 拱聖指揮二十一。京師。乾德中,選諸州騎兵送闕下,立為驍雄,後改驍猛。雍熙四年,又改拱辰。未幾改今名。



 吐渾小底舊指揮五,治平中並為二。京師。太平興國四年,平太原,獲吐渾子弟,又選監牧諸軍中所有者充。



 驍騎指揮二十三。京師。太平興國四年置,後又選掉
 搨索兵及左右教駿兵增置。雍熙四年,改殿前司步鬥弩手為驍騎弩手。淳化四年,選壯勇超絕者為上驍騎,在本軍之上。咸平五年,分左、右廂。舊又有殿前小底。至道二年,選驍騎馬直及善射者充,後廢。



 驍勝左右指揮各五。京師。咸平三年,選教駿、驍騎諸軍備徵子弟材勇者立。



 寧朔指揮十。京師、尉氏各三,雍丘、渭、河陽、河陰各一。咸平三年,選教駿諸軍備征及外州兵立。



 龍猛指揮八。京師。太平興國中,揀閱龍騎及諸州部送招獲群盜,取其材勇者立。淳化四年,又擇精悍者為教閱龍猛以備禽盜,在本軍之上。景德四年,又選龍騎、驍騎兵增之。



 飛猛指揮二。咸平二年,選龍猛、驍騎兵子弟之材勇者立。



 驍猛指揮四。尉氏三、太康一。舊號驍雄,太平興國中改。雍熙四年,以拱聖年多者為拱辰軍,其次等者如故。景德四年,以拱聖年多者隸之。



 神騎指揮十八。雍丘十三,咸平五。端拱二年,選驍雄新配人及教駿、借事等兵立。淳化二年,廢
 掉搨索軍隸之。咸平二年,又擇教駿、備征及外州增之。



 驍雄指揮四。咸平、陳留各二。太平興國八年,遷驍猛中次等者立。景德中,以驍騎、驍勝、寧朔軍年多者隸之。



 吐渾直指揮三。太原二,潞一。太平興國八年,太原遷雲州及河界吐渾立,屯並、代州。雍熙三年,又得雲、朔歸明吐渾增立,屯潞州。



 安慶直四。太原一,潞三。太平興國四年,遷雲、朔及河東歸明安慶民分屯並、潞等州,給以土田。雍熙四年立。



 三部落指揮一。太原。太平興國四年,親征幽州,遷雲、朔、應等州部落於並州,因立。



 清朔指揮四。西京二,許、汝各一。太平興國四年,遷雲、朔州民於內地,得自置馬以為騎兵,謂之家戶馬。雍熙四年立。



 擒戎指揮五。西京、許各二,汝一。太平興國四年,遷雲、朔州民於西京、許汝等州,給以土田,充家戶馬。端拱二年立。



 新安內員僚直五。端拱二年,成德軍節度使田重進言;「易州靜砦兵先屯鎮州,賊陷勇陷谷,盡俘其家,請以其
 軍備宿衛。」因而立此直。後廢,天聖後無。



 散祗候左右班二。天聖前無。



 步斗指揮六。尉氏、太康各一,蔡四。慶歷中增置,天聖前無。



 步軍



 御龍直左右二。舊號簇御馬直,太平興國二年改為簇御龍直,後改今名。



 御龍骨朵子直左右二。舊號骨金朵子直,太平興國二年改為御龍散手直,後改今名。



 御龍弓箭直五。選天武諸軍材貌魁傑者充。



 御龍弩直五。



 天武並寬衣、金屈直、左射,總指揮三十四。京師三十三,咸平一。



 神勇上下共二十一指揮。乾德中,揀閱諸軍壯實而大體者立為雄威。太平興國二年,改為雄勇。雍熙四年改今名。淳化四年,選武藝超絕者立為上神勇,以備擒盜。



 宣武上下共二十指揮。京師。太平興國二年,並效節、忠猛二軍立,又選諸軍及鄉兵增之。至道二年,又選軍頭司步直善用槍槊掉刀者立殿前步直,後廢。



 虎翼太平興國中,揀雄武弩手立為上鐵林,又於雄武、定
 遠、寧勝床子弩手、飛山雄武等軍選勁兵以增其數。雍熙四年,改分左右四軍。淳化四年,選本軍精銳者為上虎翼,以備禽盜。咸平二年,並廣勇軍隸之。大中祥符六年,詔在京諸軍選江、淮士卒善水者習戰於金明池,立為虎翼水軍。舊指揮六十二,景德中增六。京師。



 雄勇舊號雄威,太平興國二年改今名。雍熙四年,改神勇,復於本軍選退入次等者為之。舊指揮五,至和五年增為八。咸平三,鄆二,許、鄭、滑各一。



 廣德開寶四年,平廣南,以其兵隸殿前司,次等隸八作司,闕則選廣南諸州兵補之。雍熙三年,選八作司之強壯者為揀中。總指揮十。咸平、尉氏、陽武、河陽、滄、鞏、白波各一,西京三。



 廣勇淳化二年,選神射、鞭箭、雄武、效忠等軍強壯善射者立為廣武,大中祥符二年改今名。舊指揮二十三,慶歷中增為四十三,每指揮十為一軍。京師五,陳留二十二,咸平、東明、太原、胙城、南京各二,襄邑、陽武、鄆各一,滑三。



 廣捷舊名左右平遠,建隆二年改。咸平五
 年,又選廣德、神威等軍教以標槍旁牌補之。舊指揮五,景祐中增五,明道中增十,慶歷增三十六,總五十六。陳留八,咸平六,雍丘四,襄邑、尉氏、許各三,太康、扶溝、南京、亳、河陰、穎、寧陵各二,陳六,滑、曹、鄧、蔡、廣濟、穀熟、永城、襄城、葉各一。



 雄威雍熙四年,選神勇兵退入第二等立為神威,後改今名。指揮十。考城、襄邑、陳留各一,南京四,陳二。



 宣威雍熙四年,選神勇、宣武兵退入次等者立。上下指揮二。咸平、襄邑各一。



 龍騎建隆間以諸道招致及捕獲群寇立,號有馬步人,見陣即步斗。淳化三年,選本軍年多者為帶甲剩員。咸平以後,又以本軍及龍猛退兵增之。舊指揮八,康定中,取配隸充軍者增置為指揮二十,分三軍。京師四,尉氏、雍丘、咸平、鄭各二,南京、陳、蔡、河陽、穎、單、四波各一。



 神射兩浙州兵,舊號腰弩。雍熙四年改今名。淳化元年,部送闕下,選其強者為廣武,次等復為本軍。指揮五。陳留三,雍丘二。



 步鬥雍熙三年,選諸州廂軍之壯勇者立,
 後廢。此下二軍,天聖後無。



 鞭箭雍熙三年,選兩浙兵為鞭箭,次等者為忠節鞭箭。端拱二年並為一。至道元年,發此兵援靈州芻粟,喪車重兵器於浦洛河,詔免死,後廢。



 侍衛司侍衛親軍馬步軍都指揮使、副都指揮使、都虞候各一人。馬軍都指揮使、副都指揮使、都虞候各一人,步軍亦如之。自馬步軍都虞候已上,其員全闕,即馬、步軍都指揮使等各領其務,與殿前號為三司。馬步軍有龍衛神衛左右四廂都指揮使、都虞候。每指揮有指揮使、副指揮使。餘如殿前司之制。所領騎兵步兵之額
 敘列如左:



 騎軍



 員僚直顯德中,周平三關,召募強人及選高陽關馳捷兵為北面兩直。建隆初,選諸州騎兵及蕃鎮廳頭召募人等為左三直。太平興國四年,平太原,選其騎兵為右三直。北面兩直,營貝、冀隸高陽關都部署。大中祥符中,改為貝州左直、冀州右直,後改四直。京師二,恩、冀各一。



 龍衛舊號護聖。周廣順中,改龍捷。建隆二年,揀去衰老,以諸州所募精勁者補之。太平興國二年,改分左、右廂。四年,平太原,選其降兵為揀中龍衛。雍熙二年,又揀善槍槊者為金屈直。淳化三年,選剩員堪披甲者為帶甲剩員。五年,又揀善左射者為左射。指揮四十四。京師三十八,雍丘、尉氏、河陽各一,澶三。



 忠猛咸平一年置。指揮一。定州。



 散員咸平五年置。指揮一。定州。



 驍捷周顯德中,平三關,揀諸州士卒壯勇者為河北驍捷。宋初,隸高陽關都部署。建隆
 二年,廢左右驍武,以其兵來隸。乾德中,又選備征及嵐州歸附之兵為河南驍捷,其後止以驍捷為名。太平興國四年,平太原,揀閱降兵為揀中驍捷。淳化四年,又置新立驍捷。至道三年,分驍捷為左、右廂。咸平五年,以其年多者為帶甲剩員。指揮二十六。尉氏新立、陳揀中各一,恩十四,冀十。



 雲騎舊號左右備征,建隆二年改。開寶以後,募子弟為雲騎,以其次為武騎,又選騎兵之次等為武騎,又選本軍年多者為帶甲剩員。指揮十五。京師十一,陳留、西京各一,鞏二。



 歸明神武太平興國四年,親征幽州,以其降兵立此軍。初指揮一,後增為四。雍丘。



 克勝本潞州騎兵,端拱初升。指揮二。潞。



 驍銳舊名散員指揮,咸平四年改。指揮四。莫三,冀一。



 驍武本河北諸州忠烈、威邊、騎射等兵。淳化四年,揀閱其材,與雲騎、武騎等立,得自置馬,分左、右廂。指揮二十。北京七,真定三,定六,相、懷、洺、邢各一。



 廣銳本河州忠烈、宣勇能結社買馬者,馬死則市補,官
 助其直。至道元年立。咸平以後選振武兵增之,老疾者以親屬代。景德二年詔:非親屬願代者聽。大中祥符五年,以其退兵為帶甲剩員。舊河東指揮三十一,陜西七。景祐、康定中,增為四十二。太原、代、並各三,汾五,嵐、石、岢嵐各二,晉、熙、慈、絳、澤、隰、憲、寧化、威勝、平定、火山各一,涇、原、鄜各二,秦、渭、環、邠、寧各一。



 武清晉州騎兵。端拱二年,以其久在北鄙,有屯戍之勞,選勇悍者就升。指揮一。晉。



 有馬勁勇咸平四年,選江東諸州兵立。慶歷中,分置第六、第七。總指揮七。太原二,代、嵐各一,磁三。



 雲翼舊指揮三十三,景祐以後,增置二十三,分左、右廂,總五十六。真定、雄、瀛、深、趙、永寧各三,定、冀各六,保五,滄、北平、永靜、順安、保定各二,莫、邢、霸各一,廣信、安肅各四。



 廳子本石州城立。景德元年,改徙營相州。慶歷初,升禁軍。指揮六。定一,相五。



 萬捷開寶中,募趙、相、滄、冀州民立。大中祥符中,以驍武、雲騎退兵隸之。指揮七。相、冀、遼各二,滄一。



 雲捷太平
 興國四年,選諸軍中應募子弟及教駿、借事、備徵等有武幹者立。大中祥符五年,以寧朔退兵隸之。指揮十二。尉氏、咸平、西京、北京、澶各二,汝、懷各一。



 橫塞咸平三年,選諸軍威邊、騎射及在京借事立。指揮七。雍丘、咸平、考城、襄邑、寧陵各一,衛二。



 員僚剩員直禁軍員僚以罪責降者充。此下至騎捷凡六軍,天聖後無。



 清塞周立,指揮二。其一北蕃歸附之眾,營壽州;其一破淮南紫金山砦所得騎軍,營延州。宋初,選本軍子弟補其缺。太平興國三年,又得泉州、兩浙兵以益之。



 飛捷本威虜軍、保州、易州靜塞兵、定州廳子軍立。淳化元年,詔赴闕揀閱,以靜塞為三等,廳子為一等,改今名。指揮四。



 驍駿本壽州咸聖軍,咸平三年改。指揮一。



 揀中夏州廳子本夏州家戶。淳化五年,河西行營都部署李繼隆遣部送京師立,指揮一。



 騎捷本雍州強人指揮,咸平三年改。分營瀛、莫。指揮四。



 武騎指揮一十一。京師、雍丘各一,尉
 氏三,陳留、考城、咸平、鄭各一,西京二。此下至有馬雄略凡十二軍。《三朝志》無。



 驍騎指揮一,太原。



 無敵河北沿邊廂兵,慶歷二年升禁軍。總指揮六。定、北平各二,安肅、廣信各一。



 忠銳廣信廂兵有馬者,慶歷二年升禁軍。指揮一。



 威邊諸州廂兵,惟保州教戰射,隸巡檢司。慶歷初,升禁軍。指揮二。定、保各一。



 飛騎麟州廂兵,慶歷初,升禁軍。指揮二。



 威遠府州廂兵,本胡騎之精銳,慶歷初,升禁軍。指揮二。



 克戎並州廂軍有馬者,康定中,升禁軍。指揮一。



 有馬安塞慶州廂軍,慶歷中,升禁軍。指揮一。



 蕃落陜西沿邊廂兵有馬者,天禧後,升禁軍,極邊城砦悉置。至慶歷中,總指揮八十三。環五,延、慶各四,秦並外砦十七,原、渭並外砦各十二,德順並外砦十二,鳳翔、涇並外砦、儀、保安各二,隴外一。



 並州騎射諸道廂軍惟並州路有馬備征役,慶歷五年升禁軍。指揮一。



 有馬雄略至和二年
 置,指揮三。廣、桂、邕各一。



 步軍



 神衛晉曰奉國軍,周改虎捷。建隆二年,揀閱諸州所募禁軍增補。乾德三年,西川行營都部署王全斌偽署感化、耀武等軍平寇者功,請備禁旅,詔並為虎捷。太平興國二年改。舊水虎翼即軍中習水戰者,是歲改為神衛水軍,又於剩員中選可備征役者立為揀中神衛。大中祥符後,剩員又有帶甲、看倉草場、看船之名,凡四等,皆選本軍年多者補。宋初,指揮四十六,仁宗後,止存指揮三十一。京師。



 步武本鄉軍選充神勇、宣武,雍熙三年,揀其次等者立。慶歷中,增指揮六。陳。



 虎翼宋初,號雄武弩手。太平興國二年,選壯勇者為上鐵林,其次為下鐵林。雍熙四年,改為左、右廂,各三軍。咸平五年,以威虎軍來隸。景德三年,選效順兵補其缺。大中祥符五年,擇本軍善水戰者為上虎翼,六,年又選江、淮習水卒於金明池按試戰棹,立為虎翼軍。江、浙、淮南諸州,亦準此選置。七年,改為虎翼水軍。舊指揮七十五,慶歷中,增置二十一,
 總九十六。京師九十並水軍一,襄邑、東明、單各一,長葛一。



 奉節乾德三年平蜀,得其兵立為奉議,後改今名。景德三年,又選立上奉節。指揮五,京師。



 武衛太平興國中,募河北諸州兵立。舊指揮十六,慶歷中,河北增置為指揮六十七。南京、真定、淄各四,北京、澶、相、邢、懷、趙、棣、洺、德、祁、通利、乾寧、廣濟各一,青五,鄆、徐、兗、曹、濮、沂、濟、單、萊、濰、登、淮陽、瀛、博各二,齊、密、滄各三。



 雄武並雄武弩手、床子弩雄武、揀中雄武、飛山雄武、揀中歸明雄武,總指揮三十四。京師十三,太原、尉氏、南京、鄭、汝、寧陵各二,咸平、東明、雍丘、襄邑、許、曹、廣濟、穀熟、長葛各一。



 川效忠太平興四三年,選諸州廂兵歸京師者立。淳化四年,又選川峽威棹、克寧兵部送京師者產為川效忠。景德元年,以德清廂軍及威遠兵增之。舊指揮二十八,後減為七。南京六,寧陵一。



 效順宋初,徵潞州,以降卒立。指揮一。襄邑。



 雄勝開寶中,以剩員立。太平興國中,選入上鐵林,餘如故。又有雄勝剩員。指揮
 三。峽、冀、濟各一。



 揀中雄勇開寶中立,以常寧雄勇、效順等軍剩員中選其強者立為揀中。大中祥符二年,又選歸遠軍為新立。舊指揮四,後損為一。襄邑。



 懷勇開寶四年,揀蜀兵之在京師者立,指揮三。雍丘二,陳一。



 威寧淳化中,部送西川賊帥五小波脅從之兵歸京師立。咸平元年,又以散員直增補。指揮一。許。



 飛虎本虎翼、廣武兵屯西川無家屬者,太平興國中,歸京師。指揮三。陳留二,咸平一。



 懷順本淮南兵,舊號懷德。建隆二年改。指揮一。霸。



 歸聖開寶七年,以李從善所領兵及水軍立。八年,平江南,又以其降兵增補,指揮一。雍丘。



 順聖太平興國中,部送兩浙兵歸京師立。指揮一。鞏。



 懷恩乾德三年,平蜀,得其軍立。指揮三。荊南二,鄂一。



 揀中懷愛本蜀兵,與懷恩同立,又拔精銳者為揀中。淳化四年,又選川峽威棹、克寧兵次等者立為牽船,以給河漕之役。舊指揮三,後損為一。寧陵。



 勇捷太平興國四年,徵太
 原立,分左、右廂,以諸州庫兵補左廂,廣濟、開山兵補右廂。指揮二十六。襄邑、北京、澶、陳、壽、汝、曹、宿各二,咸平、西京、南京、亳、寧、洪、河陰、鞏、長葛、韋城各一。



 威武太平興國四年,徵太原立,分左、右廂,以江南歸化兵補左廂,兩浙順化兵補右廂。大中祥符五年,又立下威武。共指揮十三。西京、河陽、鄭、鄆、澶、滑、濮、通利、鞏、河陰、永城各一,曹二。



 靜戎弩手選江南歸化兵及諸州廂兵壯實者立。指揮四。河陽、澶、衛、通利各一。



 平塞弩手本兩浙順化軍,揀其強壯立為弩手,又以江、浙逋負官物隸窯務徒役者為揀中平塞。指揮四。咸平、亳、河陽、白波各一。



 新立弩手本勁勇兵,太平興國中,選其善弩者立。指揮一。廣濟。



 忠勇咸平五年,以易州兵能禽賊者立。指揮一。成都。



 寧遠大中祥符六年,選西川克寧、威棹兵立。舊指揮五,皇祐及至和中,增置為八。戎三,遂、梓、嘉、雅、江安各一。



 忠節太平興國三年,選諸州廂軍之強壯者立。淳化四年,
 又選川峽威棹、克寧兵立為川忠節。舊指揮二十四,後增教閱忠節總為六十。雍丘、襄邑、寧陵各三,陳留、咸平、東明、亳、河陰、永城各二,南京五,太康、陽武、許、江寧、揚、廬、宿、壽、楚、真、泗、泰、滁、岳、澧、池、歙、信、太平、饒、宣、洪、虔、吉、臨江、興國、廣濟、南康、廣德、長葛各一,合流四。



 神威咸平三年,選京師諸司庫務兵立。上下指揮十三。陳留三,許、鞏各二,雍丘、考城、咸平、河陽、廣濟、白波各一。



 歸遠雍熙三年,王師北征,拔飛狐、靈丘,得其降卒立,咸平二年,選諸州雜犯兵增之。舊指揮三,天聖中,增置為十六。陳、許、亳、壽、宿、鄧、襄、鼎各一,荊南、澧、潭、洪各二。



 雄略咸平六年,選諸州廂兵及香藥遞鋪兵立。舊指揮十五,皇祐五年,增置為二十五。荊南五,潭四,鼎、澧各二,廣、辰、桂各二,許、全、邵、容各一。



 威猛咸平三年,選諸州廂兵及召募者立。上下指揮十。襄邑四,咸平、許、長葛各二。



 神銳咸平六年,料簡河東兵立。大中祥符五年,以本軍及神虎兵年多者為帶甲剩員。指揮
 二十六。太原六,潞、晉各三,澤、汾、隰、平定各二,代、絳、忻、遼、邢、威勝各一。



 神虎咸平五年,選陜西州兵馬立。六年,又料簡河東州兵立,以西路河東兵之。指揮二十六。永興六,鳳翔、河中、忻、晉、威勝各二,太原、秦、延、鄜、華各一,潞州三。



 保捷咸平四年,詔陜西沿邊選鄉丁保毅升充。舊指揮四十五,慶歷中,揀鄉弓手增置,總一百三十五。永興十二,同九,秦八,河中、汾、涇各七,渭、寧、耀各六,鳳翔、延、儀、華、隴、解、幹各五,陜、原、鄜各四,成三,慶、鳳、坊、晉、鎮戎各二,環、丹、商、虢、階、慶成、德順各一。



 振武舊指揮四十,慶歷後,河北增置為指揮四十二,陜西增置為指揮三十九,總八十一。北京、澶、相、懷、衛、霸、莫、祁、棣、趙、濱、洺、保安、永寧、通利、安肅、儀各一,真定、定、瀛、保、恩、邢、深、博、永靜、乾、寧陵、涇各二,延六,邠、隴各七,鄜、寧各五,磁四,滄、原各三。



 橋道太平興國三年,選諸州廂兵次等者立。淳化四年,又選川峽威棹、克寧為川橋道。總指揮十八。襄邑、咸平、陽武各二,陳留、東明、尉氏、太康、西京、
 河陽、濮、鄆、鞏、河陰、白波、寧陵各一。



 清塞太平興國初立。左、右廂,舊指揮二十三,嘉祐中並為十三。曹二,鄭、鄆、滑、通利、鞏、河陰、白波、汜水、長葛各一。



 招收端拱中,獲通州大沙洲賊眾立,缺則以江、浙招致海賊補之。又收端拱中逃軍來復者,原其罪為德壽軍,後改今名,隸保州巡檢司,慶歷初,升禁軍,為指揮十七。保四,霸、信安各三,定、軍城砦各二,廣信、安肅、順安各一。



 壯勇本招獲群盜配近京徒役者揀拔立,咸平三年,選諸雜犯兵增之。至道三年,江、浙發運使楊允恭禽海賊送闕下增補,旋廢。舊指揮三,慶歷中,增置為七。耀、解、滑各二,許一。



 宣效咸平三年,選六軍、窯務、軍營務、天駟監效役、店宅務、州兵立。景德元年,又揀本軍材勇者為揀中宣效。舊指揮五,後損為二。京師。



 來化雍熙中,以飛狐、靈丘歸附之眾立,又以朔州內附牽擺兵立,後廢。舊指揮三,後損為二。寧陵。



 歸恩雍熙中,平塞陷邊之民黥面放還立,分有家屬者隸左廂,無者
 隸右廂。指揮二。亳。



 順化太平興國三年,以兩浙兵之次等者立。指揮二。河陽、鄆各一。



 左右清衛大中祥符八年立,以奉諸宮觀灑掃之役。指揮二。此下至強壯軍員凡八軍,天聖後無。



 川員僚直本西蜀賊全師雄所署將領,乾德中立。



 造船務乾德初,平荊湖,選其軍善治舟楫者立。



 歸明羽林太平興國四年,徵幽州,獲其兵立。



 新立清河緣河舊置鋪兵以備河決,後揀閱立。指揮二。



 保寧大中祥符元年,馬步軍都虞候王超請以病軍經行陣者立。



 新立歸化開寶七年,以江南李從善所領部曲水軍立,八年,平江南,又以降兵增之。指揮一。



 強壯軍員咸平六年置,指揮一。



 澄海弩手慶歷二年置,隸海州都巡檢司。指揮二。登。此下至武嚴凡十三軍。



 捉生延州廂兵,天聖五年升禁軍,指揮二。



 清邊弩手寶元初,選陜西、河東廂軍之伉健者置,以弩手名。指
 揮四十三。太原九,秦五,涇四,河中、隴各三,永興、代、潞、晉各二,慶、環、滑、同、坊、鎮戎、慈、丹、隰、汾、憲各一。



 制勝陜西廂兵,慶歷中,升禁軍。指揮九。永興、華各二,鳳翔、耀、同、解、幹各一。



 定功陜西廂軍,慶歷四年,升禁軍增置,為指揮十。永興、秦、慶、原、渭、涇、儀、鄜、延、鎮戎各一。



 清澗慶歷初,募土人精悍者充,因其地名。指揮二。



 建威秦州廂兵,慶歷八年升禁軍。指揮一。



 效勇景祐中,募川峽流民增置,為指揮二十七。陳留三,太康、尉氏、襄邑、河陽、曹、合流各二,咸平、鄭、亳、衛、許、單、澶、磁、廣濟、河陰、寧陵、白波各一。



 宣毅慶歷中,京東、京西、河北、河東、淮南、江南、兩浙、荊湖、福建九路募健勇或選廂軍為之。指揮二百八十八,至治平中,管一百七十四。京東路:南京、鄆、徐、曹、齊各二,青、兗、密、濮、沂、單、濟、淄、萊、濰、登、淮陽、廣濟各一;京西路:西京、滑、許、河陽、陳、襄、鄭、穎、蔡、汝、隨、信陽各一,鄧二。河北路:真定、德、棣、博、邢、祁、恩、磁、深、定、濱、通利、永靜、乾寧各一。河東路:太原、汾各六,晉四,
 澤、絳、石、代各三,潞、嵐、忻、遼、威勝、平定各二,慈、隰、寧化各一。淮南路:揚、亳各二,廬、宿、壽、楚、真、泗、蘄海、舒、泰、濠、和、光、黃、通、無為、高郵、漣水各一。江南路:江寧、洪、虔、吉、撫、袁、筠、建昌、南安各一。兩浙路:杭二,越、蘇、明、湖、婺、潤、溫、衢、常、秀、處各一。荊湖路:潭、全、鼎各三,荊南、邵、衡、永、郴、道、安、鄂、岳、澧、復、峽、歸、辰、荊門、漢陽、桂陽各一。福建路:建二,泉、南劍、漳、汀、邵武、興化各一。



 宣毅床子弩炮手慶歷中置。指揮一。岢嵐。



 建安府州廂兵,慶歷二年升禁軍。指揮二。府、嵐各一。



 威果嘉祐四年置,指揮二十五。荊南、江寧、杭、揚、廬、潭各三,洪、越、福各二,虔一。



 武嚴指揮一。京師。



 御前忠佐軍頭司馬步軍都軍頭、副都軍頭,馬軍都軍頭、副都軍頭,步軍都軍頭、副都軍頭。其所轄散員,有副
 指揮使、軍使、副兵馬使、十將。馬步直自指揮使而下,皆如殿前司之制。



 御前忠佐散員本許州員僚剩員,淳化中,立為軍頭司散員一班。又五代以來,軍校立功無可門署者,第令與諸校同其飲膳,名健飯都指揮使,後唯被譴者居此。大中祥符二年,改為散指揮使。班一。



 馬直雍熙四年置,指揮一。



 步直端拱元年置,指揮一。



 備軍一千九百六十人。



 皇城司親從官太平興國四年,分親事官之有材勇者為之,給諸殿灑掃及契勘巡察之事。指揮三。



 入內院子天聖元年,揀親事官年高者為之。九年,選輦官六十以上者充。治平二年,詔以五百人為額。



 騏驥院騎禦馬直太平興國二年置,分左右番。八年,分為二直。其後增置八直。



 左右教駿舊名左右備征,建隆二年改。指揮四。



\end{pinyinscope}