\article{列傳第六十九}

\begin{pinyinscope}

 李迪子柬之肅之承之及之孫孝基孝壽孝稱王曾弟子融張知白杜衍



 李迪字復古,其先趙郡人,後徙幽州。曾祖在欽,避五代亂,又徙家濮。迪深厚有器局,嘗攜其所為文見柳開,開
 奇之曰:「公輔材也。」



 舉進士第一,授將作監丞,歷通判徐、兗州。改秘書省著作郎、直史館,為三司鹽鐵判官。東封泰山,復通判兗州,坐嘗解開封府進士失當,謫監海州稅。改右司諫,起知鄆州,詔糾察在京刑獄,遷起居舍人,安撫江、淮,以尚書吏部員外郎為三司鹽鐵副使,擢知制誥。



 真宗幸亳,為留守判官,遂知亳州。亡卒群剽城邑,發兵捕之,久不得。迪至,悉罷所發兵,陰聽察知賊區處,部勒驍銳士,擒賊,斬以徇。代歸,會唃廝囉叛,帝憂關中,
 召對長春殿,進右諫議大夫、集賢院學士、知永興軍。城中多無賴子弟,喜犯法,迪奏取其甚者,部送闕下。徙陜西都轉運使,入為翰林學士。



 嘗歸沐,忽傳詔對內東門,出三司使馬元方所上歲出入材用數以示迪。時頻歲蝗旱,問何以濟,迪請發內藏庫以佐國用,則賦斂寬,民不勞矣。帝曰:「朕欲用李士衡代元方,俟其至,當出金帛數百萬借三司。」迪曰:「天子於財無內外,願下詔賜三司,以示恩德,何必曰借。」帝悅。又言:「陛下東封時,敕所過毋
 伐木除道,即驛舍或州治為行宮,裁令加塗塈而已。及幸汾、亳,土木之役,過往時幾百倍。今蝗旱之災,殆天意所以儆陛下也。」帝深然之。



 他日,又召對龍圖閣,命迪草詔,徐謂迪曰:「曹瑋在秦州,屢請益兵,未及遣,遽辭州事,第怯耳。誰可代瑋者?」迪對曰:「瑋知唃廝囉欲入寇,且窺關中,故請益兵為備,非怯也。且瑋有謀略,諸將皆非其比,何可代?陛下重發兵,豈非將上玉皇聖號,惡兵出宜秋門邪?今關右兵多,可分兵赴瑋。」帝因問關右兵幾何,
 對曰:「臣向在陜西,以方寸小冊書兵糧數備調發,今猶置佩囊中。」帝令自探取,目黃門取紙筆,具疏某處當留兵若干,餘悉赴塞下。帝顧曰:「真所謂頗、牧在禁中矣。」未久,唃廝囉果犯邊。秦州方出兵,復召迪問曰:「瑋此舉勝乎?」對曰:「必勝。」居數日,奏至,瑋與敵戰三都谷,果大勝。帝曰:「卿何以知瑋必勝?」迪曰:「唃廝囉兵遠來,使諜者聲言以某日下秦州會食,以激怒瑋。瑋勒兵不動,坐待敵至,是以逸待勞也。臣用此知其勝。」帝益重之,自是欲大用
 矣。



 初,上將立章獻後,迪屢上疏諫,以章獻起於寒微,不可母天下。章獻深銜之。天禧中,拜給事中、參知政事。周懷政之誅,帝怒甚,欲責及太子,群臣莫敢言。迪從容奏曰:「陛下有幾子,乃欲為此計。」上大寤,由是獨誅懷政等。仁宗為皇太子,除太子太傅,迪辭以太宗時未嘗立保傅,止兼太子賓客,詔皇太子禮賓客如師傅。加禮部侍郎。寇準罷,帝欲相迪,迪固辭。一日,對滋福殿,有頃,皇太子出拜曰:「陛下用賓客為宰相,敢以謝。」帝顧謂迪曰:「尚
 可辭邪!」拜吏部侍郎兼太子少傅、同中書門下平章事、景靈宮使、集賢殿大學士。



 初,真宗不豫,寇準議皇太子總軍國事,迪贊其策,丁謂以為不便,曰:「即日上體平,朝廷何以處此?」迪曰:「太子監國,非古制邪?」力爭不已。於是皇太子於資善堂聽常事,他皆聽旨。準既貶,謂浸擅權用事,至除吏不以聞。迪憤然語同列曰:「迪起布衣至宰相,有以報國,死猶不恨,安能附權幸為自安計邪!」自此不協。時議二府皆進秩兼東宮官,迪以為不可。謂又欲
 引林特為樞密副使,而遷迪中書侍郎兼尚書左丞。故事,宰相無為左丞者。既而帝禦長春殿,內出制書置榻前,謂輔臣曰:「此卿等兼東宮官制書也。」迪進曰:「東宮官屬不當增置,臣不敢受此命。宰相丁謂罔上弄權,私林特、錢惟演而嫉寇準。特子殺人,事寢不治,準無罪罷斥,惟演姻家使預政,曹利用、馮拯相為朋黨。臣願與謂俱罷,付御史臺劾正。」帝怒,留制不下,左遷迪戶部侍郎。謂再對,傳口詔入中書復視事,出迪知鄆州。



 仁宗即位,太
 后預政,貶準雷州,以迪朋黨傅會,貶衡州團練副使。謂使人迫之,或諷謂曰:「迪若貶死,公如士論何?」謂曰:「異日諸生記事,不過曰『天下惜之』而已。謂敗,起為秘書監、知舒州,歷江寧府、兗州、青州,復兵部侍郎、知河南府。來朝京師,時太后垂簾,語迪曰:「卿向不欲吾預國事,殆過矣。今日吾保養天子至此,卿以為何如?」迪對曰:「臣受先帝厚恩,今日見天子明聖,臣不知皇太后盛德,乃至於此。」太后亦喜。以尚書左丞知河陽,遷工部尚書。太后崩,召
 為資政殿學士、判尚書都省。未幾,復拜同中書門下平章事、集賢殿大學士。



 景祐中,範諷得罪,迪坐姻黨,罷為刑部尚書,知亳州,改相州。既而為資政殿大學士、翰林侍讀學士,留京師。迪素惡呂夷簡,因奏夷簡私交荊王元儼,嘗為補門下僧惠清為守闕鑒義。夷簡請辨,詔訊之,乃迪在中書所行事,夷簡以齋祠不預。降太常卿、知密州。復刑部尚書、知徐州。迪奏所部鄰兗州,欲行縣因祠岳為上祈年、禱皇子。仁宗語輔臣曰:「大臣當為百姓
 訪疾苦,祈禱非迪所宜,其毋令往。」久之,改戶部尚書、知兗州,復拜資政殿大學士。



 元昊攻延州,武事久弛,守將或為他名以避兵。迪願守邊,詔不許,然甚壯其意。除彰信軍節度使、知天雄軍,徙青州。逾年,之本鎮。請老,以太子太傅致仕,歸濮州。後其子柬之為侍御史知雜事,奉迪來京師。帝數遣使問勞,欲召見,以疾辭。薨,年七十七。贈司空、侍中,謚文定。帝篆其墓碑曰「遺直之碑」,又改所葬鄧侯鄉曰遺直鄉。子柬之、肅之、承之、及之,孫孝壽、孝
 基、孝稱。



 柬之字公明,曉國朝典故。獻文,召試,賜進士出身,為館閣校勘、宣化軍使。境上有廢河故道,官收行者稅,謂之「干渡錢」,奏除之。進直集賢院、判吏部南曹、開封府推官、鹽鐵判官,歷知邢漢廬州、鳳翔府,京東、陜西轉運使,擢侍御史知雜事。



 柬之自少受知於寇準,至是論準保護之功。仁宗惻然,即賜其碑曰「旌忠」。拜天章閣待制、河北都轉運使,加龍圖閣直學士。建言補蔭之門太廣,遂詔
 裁定,自二府而下,通三歲減入仕者一千人。知荊南、河陽、澶州,改集賢院學士,判西京留司御史臺。



 英宗即位,富弼薦其學行,復舊職,兼侍讀。帝勞之曰:「卿通議耆儒,方咨訪以輔不逮,豈止經術而已。」帝頗欲肅正宮省,柬之諫曰:「陛下,長君也,立自宗藩,眾方觀望,願曲為容覆。」賜穎王生日禮物,故事,王拜賜竟,即退。帝諭王令留柬之食,冀其從容也。王即位未幾,柬之請老,自工部尚書拜太子少保致仕。舊無閣門謝辭式,特賜對延和,命之
 坐,仍置宴資善堂,遣使諭之曰:「以先帝梓宮在殯,朕不得為詩。」令講讀官皆賦詩,勸勞甚渥,又敕王珪敘其事。柬之出都門,即幅巾白衣以見客。再遷少師。熙寧六年,卒,年七十八。



 有李受者,字益之,長沙之瀏陽人也。仕於治平中,至右諫議大夫、天章閣待制兼侍讀。屢以老乞骸骨,不聽。神宗立,進給事中、龍圖閣直學士。復言:「臣在先帝時,年已七十,不敢竊祿以自安。今又加數年,筋力憊矣,惟陛下哀之。」於是拜刑部侍郎致仕,賜宴賦詩及
 序,如柬之禮。相去數月,故時稱「二李」。卒年八十,贈工部尚書。



 肅之字公儀,迪弟子也。以迪蔭,監大名府軍資庫。大河溢,府檄修冠氏堤,工就弗擾,民悅之,請為宰。邑多盜,時出害人。肅之令比戶置鼓,有盜,輒擊鼓,遠近皆應,盜為之衰止。為御河催綱。橫隴之決,使者檄護金堤,滿歲無河患。



 通判澶州。契丹泛使將過郡,而樓堞壞圮,肅之謂郡守曰:「吾州為景德破敵之地,當示雄疆,今保障若是,
 且奈何?」遂鳩工構城屋,凡千區。已而中貴人銜命來視,規置一新,驚賞嗟異,聞之朝。擢知德州,提點開封府界內縣鎮,夔路、湖南刑獄。儂蠻暴嶺外,肅之親捍諸境,會蔣偕失利,亟率兵往躡於臨賀,賊引去。狄青、孫沔交薦之,徙湖北轉運使。辰陽彭仕羲叛,討平之,猶以過左遷,知齊州。改江東、兩浙、河北轉運使,進度支副使、江淮發運使。



 神宗初即位,諒祚寇大順城。肅之入奏,帝訪以西夏事,奏對稱旨。以為右諫議大夫、知慶州;數日,徙瀛州。
 大雨地震,官舍民廬推陷。肅之出入泥潦中,結草囷以儲庾粟之暴露者,為茇舍以居民,啟廩振給,嚴儆盜竊,一以軍法從事。天子聞而嘉之,遣使勞賜。遷天章閣待制、知開封府,出知定州。還,遷三司使,又出為永興軍、青、齊二州。元豐二年,復知開封,為樞密都承旨,加龍圖閣直學士、知鄆州。四年,提舉太極觀。卒,年八十二。



 肅之內行修飭,母喪,廬墓三年,不入城郭。季弟承之,生而孤,鞠育誨道,至於成人,遂相繼為侍從。帝稱其一門忠孝云。



 承之字奉世,性嚴重,有忠節。從兄柬之將仕以官,辭不受,而中進士第,調明州司法參軍。郡守任情骫法,人莫敢忤,承之獨毅然力爭之。守怒曰:「曹掾敢如是邪?」承之曰:「事始至,公自為之則已,既下有司,則當循三尺之法矣。」守憚其言。



 嘗建免役議,王安石見而稱之。熙寧初,以為條例司檢詳文字,得召見。神宗語執政曰:「承之言制置司事甚詳,非他人所及也。」改京官。他日,謂之曰:「朕即位以來,不輕與人改秩,今以命汝,異恩也。」



 檢正中書刑
 房,察訪淮浙常平、農田水利、差役事,還奏《役書》二十篇,加集賢校理。又察訪陜西,時郡縣昧於奉法,斂羨餘過制。承之曰:「是豈朝廷意邪?」悉裁正其數。遷集賢殿修撰,擢寶文閣待制,為同群牧使,糾察在京刑獄兼樞密都承旨,出知延州,入權三司使。



 蔡確治相州獄,多引朝士,皆望風自折服。承之為帝言其險詖之狀,帝意始悟,趣使詰竟。遷龍圖閣直學士,懇辭,乞授兄肅之,曰:「臣少鞠於兄,且兄為待制十年矣。」帝曰:「卿兄弟孝友,足厲風俗。
 肅之亦當遷也。」即並命焉。



 商人犯禁貨北珠,乃為公主售,三司久不敢決。承之曰:「朝廷法令,畏王姬乎?」亟索之。帝聞之曰:「有司當如此矣。」進樞密直學士。坐補吏不當,降待制、知汝州。未幾,為陜西都轉運使,召拜給事中、吏部侍郎、戶部尚書,復以樞密直學士知青州。歷應天府、河陽、陳、鄆、揚州而卒。



 及之字公達,亦迪弟之子。由蔭登第,通判安肅軍。康定中,夏人犯邊,契丹復發兵並塞,疆候戒嚴。及之言:「契丹
 以與夏人甥舅之故,特此慰其心,且姑張虛勢以疑我,必不失誓好,願毋過虞。」已而果然。



 徙通判河南府。亡卒張海倚山嘯聚,白晝掠城市。及之督捕,單騎與海語,諭使歸命,當奏貸其死。海感動弛備,奏方上,而眾兵集,悉獲之。知信州,靈鷲山浮屠,犯法者眾,及之治其奸,流數十人,乃自劾。朝廷嘉之,釋不問。入判刑部。嘗撰次唐史有益治體者,為《君臣龜監》八十卷。王堯臣上其書,並表其學行,韓琦亦以館職薦之。召試,除直秘閣,歷開封府
 判官、知涇、晉、陜三州。



 及之吏事精明,所居官皆稱職。以太中大夫致仕,再轉正議大夫。卒,年八十五。



 柬之子孝基,及之子孝壽、孝稱。



 孝基字伯始。進士高第,唱名至墀下,仁宗顧侍臣曰:「此李迪孫邪?能世其家,可尚也。」晏殊、富弼薦其材任館閣,欲一見之。孝基曰:「名器可私謁邪?」竟不往。



 知汝陰、雍丘縣,通判閬州、舒州,知隨州。所治雖劇,然事來亟斷,不為證左回枉,甫日中,庭已空矣。或問其術,曰:「無他,省事耳。」閬
 中江水嚙城幾沒,郡吏多引避,孝基率其下決水歸旁谷,城賴以全。舒吏受賂鬻獄,以殺人罪加平民,孝基劾治三日,得其情,乃抵吏罪。以親須養,求監崇福宮,判西京國子監。凡就閑十年,累官光祿卿,與父柬之同謝事,才年五十,士大夫美之,以比二疏。



 孝基為人沖澹,善養生,平居輕安。弟孝稱進對,帝問起居狀,歡曰:「度越常人遠矣。」後十一年,無疾卒。



 孝壽字景山,為開封府戶曹參軍。元符中,呂嘉問知府
 事,受章惇、蔡卞指,鍛煉上書人,命孝壽攝司錄事,成其獄。徽宗即位,嘉問先已得罪,孝壽亦削秩。蔡京為政,以為府推官,遷大理、太僕卿,擢顯謨閣待制,為開封尹。



 前此,閭里亡賴子,自斷截臂腕,托廢疾凌良民,無所憚畏。孝壽悉搜出之,部付旁郡,一切治理。加直學士,出知興仁、開德府。京起蘇州章綖獄,還孝壽開封,使往即訊。至蘇州,窮治鑄錢,逮系逾千數,方冬慘掠囚,墮指脫足不可計,死則投於垣外。日夜鍛煉,疑未就,京猶嫌其緩,召
 使還。其後,綖兄弟竟用此黥竄。又知虢、兗二州。坐守興仁日與巡檢戲射狂人張立死,除名。居無何,起知蘇州。



 政和初,拜刑部侍郎,復改開封尹。奉宸庫吏呂壽盜金,系獄而逃。孝壽盡執守兵,論為故縱,非任事之吏與不上直者,亦以不即追掩繩之。凡配隸四十人,陰賂杖者使加重,六七人才出關而死。帝聞之,命悉還餘人。於是諫議大夫毛注論其殘忍苛虐,乞加譴,不聽。孝壽猶以獄空上表賀。



 孝壽雖亡狀,亦時有可觀。有舉子為僕所
 凌,忿甚,具牒欲送府,同舍生勸解,久乃釋。戲取牒效孝壽花書判云:「不勘案,決杖二十。」僕明日持詣府,告其主仿尹書判私用刑。孝壽即追至,備言本末,孝壽幡然曰:「所判正合我意。」如數與僕杖,而謝舉子。時都下數千人,無一僕敢肆者,時以此稱之。明年,以疾,罷為龍圖閣學士、提舉醴泉觀。卒,贈正奉大夫。



 孝稱字彥聞,以蔭登朝。值郊恩得封父,及之已官通議大夫,有司限以格,孝稱言,恐非朝廷所以推恩優老之
 意,詔特許之,遂為著令。



 崇寧中,提舉湖北、京西常平,提點京西南路刑獄。蔡京之姻宋喬年為京畿轉運使,有囚逸,捕得之。孝稱上其功,喬年受賞,而孝稱用是得工部員外郎。不閱月,遷大理少卿。連奏獄空,進為卿,且數增秩,擢工部、戶部二侍郎,為開封尹。



 陳瓘之子正匯在杭州上書,告京不利社稷。郡守蔡薿執送京師,並逮瓘詣獄,孝稱脅使證其子,瓘不可。暨獄上,竟竄正匯海島。京愈德之,進刑部尚書,而以其兄孝壽代為尹。孝稱請
 班兄下,不許。避親嫌,徙工部。卒,贈光祿大夫。



 王曾,字孝先,青州益都人。少孤,鞠於仲父宗元,從學於里人張震,善為文辭。咸平中,由鄉貢試禮部、廷對皆第一。楊億見其賦,嘆曰:「王佐器也。」以將作監丞通判濟州。代還,當召試學士院,宰相寇準奇之,特試政事堂,授秘書省著作郎、直史館、三司戶部判官。



 景德初,始通和契丹,歲遣使致書稱南朝,以契丹為北朝。曾曰:「從其國號足矣。」業已遣使,弗果易。遷右正言、知制誥兼史館修撰。
 時瑞應沓至,曾嘗入對,帝語及之。曾奏曰:「此誠國家承平所致,然願推而弗居,異日或有災沴,則免輿議。」及帝既受符命,大建玉清昭應宮,下莫敢言者,曾陳五害以諫。舊用郎中官判大理寺,帝欲重之,特命曾。且謂曾曰:「獄,重典也,今以屈卿。」曾頓首謝。仍賜錢三十萬,因請自闢僚屬,著為令。遷翰林學士。帝嘗晚坐承明殿,召對久之,既退,使內侍諭曰:「向思卿甚,故不及朝服見卿,卿勿以我為慢也。」其見尊禮如此。



 知審刑院。舊違制無故失,
 率坐徒二年,曾請須親被旨乃坐。既而有犯者,曾乃以失論。帝曰:「如卿言,是無復有違制者。」曾曰:「天下至廣,豈人人盡曉制書,如陛下言,亦無復有失者。」帝悟,卒從曾議。再遷尚書主客郎中。知審官院、通進銀臺司,勾當三班院,遂以右諫議大夫參知政事。



 時宮觀皆以輔臣為使。王欽若方挾符瑞,傅會帝意,又陰欲排異己者,曾當使會靈,因以推欽若,帝始疑曾自異。及欽若相,會曾市賀皇后家舊第,其家未徙去,而曾令人舁土置門外,賀
 氏訴禁中。明日,帝以語欽若,乃罷曾為尚書禮部侍郎、判都省,出知應天府。天禧中,民間訛言有妖起若飛帽,夜搏人,自京師以南,人皆恐。曾令夜開里門,敢倡言者即捕之,卒無妖。徙天雄軍,復參知政事,遷吏部侍郎兼太子賓客。



 真宗不豫,皇后居中預政,太子雖聽事資善堂,然事皆決於後,中外以為憂。錢惟演,後戚也,曾密語惟演曰:「太子幼,非宮中不能立。加恩太子,則太子安;太子安,所以安劉氏也。」惟演以為然,因以白後。帝崩,曾奉
 命入殿廬草遺詔:「以明肅皇后輔立皇太子,權聽斷軍國大事。」丁謂入,去「權」字。曾曰:「皇帝沖年,太后臨朝,斯已國家否運。稱『權』,猶足示後。且增減制書有法,表則之地,先欲亂之邪?」遂不敢去。仁宗立,遷禮部尚書。群臣議太后臨朝儀,曾請如東漢故事,太后坐帝右,垂簾奏事,丁謂獨欲帝朔望見群臣,大事則太后召對輔臣決之,非大事令入內押班雷允恭傳奏禁中,畫可以下。曾曰:「兩宮異處,而柄歸宦官,禍端兆矣。」謂不聽。既而允恭坐誅,
 謂亦得罪。自是兩宮垂簾,輔臣奏事如曾議。



 謂初敗,任中正言:「謂被先帝顧托,雖有罪,請如律議功。」曾曰:「謂以不忠得罪宗廟,尚何議邪!」時真宗初崩,內外洶洶,曾正色獨立,朝廷倚以為重。拜中書侍郎兼本官、同中書門下平章事、集賢殿大學士、會靈觀使。王欽若卒,曾以門下侍郎兼戶部尚書為昭文館大學士、監修國史、玉清昭應宮使。曾以帝初即位,宜近師儒,即召孫奭、馮元勸講崇政殿。天聖四年夏,大雨。傳言汴口決,水且大至,都
 人恐,欲東奔。帝問曾,曾曰:「河決奏未至,第民間妖言爾,不足慮也。」已而果然。陜西轉過使置醋務,以榷其利,且請推其法天下,曾請罷之。



 曾方嚴持重,每進見,言利害事,審而中理;多所薦拔,尤惡僥幸。帝問曾曰:「比臣僚請對,多求進者。」曾對曰:「惟陛下抑奔競而崇恬靜,庶幾有難進易退之人矣。」曹利用惡曾班己上,嘗怏怏不悅,語在《利用傳》。及利用坐事,太后大怒,曾為之解。太后曰:「卿嘗言利用強橫,今何解也?」曾曰:「利用素恃恩,臣故嘗以
 理折之。今加以大惡,則非臣所知也。」太后意少釋,卒從輕議。



 始,太后受冊,將御大安殿,曾執以為不可,及長寧節上壽,止共張便殿。太后左右姻家稍通請謁,曾多所裁抑,太后滋不悅。會玉清昭應宮災,乃出知青州。以彰信軍節度使復知天雄軍,契丹使者往還,斂車徒而後過,無敢嘩者。人樂其政,為畫像而生祠之。改天平軍節度使、同中書門下平章事、判河南府。景祐元年,為樞密使。明年,拜右僕射兼門下侍郎、平章事、集賢殿大學士,
 封沂國公。



 曾進退士人,莫有知者。範仲淹嘗問曾曰:「明揚士類,宰相之任也。公之盛德,獨少此耳。」曾曰:「夫執政者,恩欲歸己,怨使誰歸?」仲淹服其言。初,呂夷簡參知政事,事曾謹甚,曾力薦為相。及夷簡位曾上,任事久,多所專決,曾不能堪,論議間有異同,遂求罷。仁宗疑以問曾曰:「卿亦有所不足邪?」時外傳知秦州王繼明納賂夷簡,曾因及之。帝以問夷簡,曾與夷簡交論帝前。曾言亦有過者,遂與夷簡俱罷,以左僕射、資政殿大學士判鄆州。
 寶元元年冬,大星晨墜其寢,左右驚告。曾曰:「後一月當知之。」如期而薨,年六十一。贈侍中,謚文正。



 曾資質端厚,眉目如畫。在朝廷,進止皆有常處,平居寡言笑,人莫敢干以私。少與楊億同在侍從,億喜談謔,凡僚友無不狎侮。至與曾言,則曰:「餘不敢以戲也。」平生自奉甚儉,有故人子孫京來告別,曾留之具饌,食後,合中送數軸簡紙,啟視之,皆它人書簡後裁取者也。皇祐中,仁宗為篆其碑曰「旌賢之碑」,後又改其鄉曰旌賢鄉。大臣賜碑篆自
 曾始。仁宗既祔廟,詔擇將相配享,以曾為第一。曾無子,養子曰縡。又以弟子融之子繹為後,尚書兵部郎中、秘閣校理致仕,卒。



 子融字熙仲。初以曾奏,為將作監主簿。祥符進士及第,累遷太常丞、同知禮院。獻所為文,召試,直集賢院。嘗論次國朝以來典禮因革,為《禮閣新編》上之。以其書藏太常。



 權三司度支、鹽鐵判官。任布請鑄大錢,行之京城。三司使程琳集官議,子融曰:「今軍營半在城外,獨行大錢
 城中,可乎?」事遂寢。權同糾察刑獄、知河陽。又集五代事,為《唐餘錄》六十卷以獻。進直龍圖閣,累遷太常少卿、權判大理寺。乃取讞獄輕重可為準者,類次以為斷例。



 拜天章閣待制、尚書吏部郎中、知荊南。盜張海縱掠襄、鄧,至荊門,子融閱州兵,將迎擊之,賊引去。遷右諫議大夫、知陜州,徙河中府。既而勾當三班院,遷給事中,以尚書工部侍郎、集賢院學士知兗州。不赴,改刑部侍郎致仕。英宗即位,進兵部,卒。



 本名皞,字子融。元昊反,請以字為
 名。性儉嗇,街道卒除道,侵子融邸店尺寸地,至自詣開封府訴之。然教飭子孫,嚴厲有家法。晚學佛氏,從僧懷璉游。



 張知白,字用晦,滄州清池人。幼篤學,中進士第,累遷河陽節度判官。咸平中疏,言當今要務,真宗異之,召試舍人院,權右正言。獻《鳳扆箴》,出知劍州。逾年,召試中書,加直史館,面賜五品服,判三司開拆司。



 江南旱,與李防分路安撫。及還,權管勾京東轉運使事。周伯星見,司天以
 瑞奏,群臣伏閣稱賀。知白以為人君當修德應天,而星之見伏無所系,因陳治道之要。帝謂宰臣曰:「知白可謂乃心朝廷矣。」東封,進右司諫。又言:「咸平中,河湟未平,臣嘗請罷郡國所上祥瑞。今天下無事,靈貺並至,望以《泰山諸瑞圖》寘玉清昭應宮,其副藏秘閣。」



 陜西饑,命按巡之。尋知鄧州。會關右流傭至境,知白既發倉廩,又募民出粟以濟。擢龍圖閣待制、知審官院,再遷尚書工部郎中,使契丹。知白以朝廷制官,重內輕外,為引唐李嶠議
 遷臺閣典藩郡,乃自請補外,不許,遂命糾察在京刑獄,固請,知青州。還京師,求領國子監。帝曰:「知白豈倦於處劇邪?」宰臣言:「知白更踐中外,未嘗為身謀。」乃遷右諫議大夫、權御史中丞、拜給事中、參知政事。



 郊禮成,遷尚書工部侍郎。時同列王曾遷給事中,猶班知白上,知白心不能平,累表辭之。曾亦固請列知白下,乃加知白金紫光祿大夫,復為給事中、判禮儀院。曾罷,還所辭官。時王欽若為相,知白論議多相失,因稱疾辭位,罷為刑部侍
 郎、翰林侍讀學士、知大名府。及欽若分司南京,宰相丁謂素惡欽若,徙知白南京留守,意其報怨。既至,待欽若加厚。謂怒,復徙知白亳州,遷兵部。仁宗即位,進尚書右丞,為樞密副使,以工部尚書同中書門下平章事、會靈觀使、集賢殿大學士。時進士唱第,賜《中庸篇》,中書上其本,乃命知白進讀,至修身治家之道,必反復陳之。



 知白在相位,慎名器,無毫發私。常以盛滿為戒,雖顯貴,其清約如寒士。然體素羸,憂畏日侵,在中書忽感風眩,輿歸
 第。帝親問疾,不能語,薨。為罷上巳宴,贈太傅、中書令。禮官謝絳議謚文節,御史王嘉言言:「知白守道徇公,當官不撓,可謂正矣,謚文正。」王曾曰:「文節,美謚矣。」遂不改。



 知白九歲,其父終邢州,殯於佛寺。及契丹寇河北,寺宇多頹廢,殯不可辨。知白既登第,徒行訪之,得佛寺殿基,恍然識其處。既發,其衣衾皆可驗,眾嘆其誠孝。嘗過陜州,與通判孫何遇,讀道旁古碑凡數千言,及還,知白略無所遺。天聖中,契丹大閱,聲言獵幽州,朝廷患之。帝以問二
 府,眾曰:「備粟練師,以備不虞。」知白曰:「不然,契丹修好未遠,今其舉者,以上初政,試觀朝廷耳,豈可自生釁邪!若終以為疑,莫如因今河決,發兵以防河為名,彼亦不虞也。」未幾,契丹果罷去。無子,以兄子子思為後,仕至尚書工部侍郎致仕。



 杜衍,字世昌,越州山陰人。父遂良,仕至尚書度支員外郎。衍總發苦志厲操,尤篤於學。擢進士甲科,補揚州觀察推官,改秘書省著作佐郎、知平遙縣。使者薦之,通判
 晉州。



 詔舉良吏,擢知幹州。陳堯咨安撫陜西,有詔藩府乃賜宴,堯咨至乾州,以衍賢,特賜宴,仍徙衍權知鳳翔府。及罷歸,二州民邀留境上,曰:「何奪我賢太守也?」以太常博士提點河東路刑獄,遷尚書祠部員外郎。按行潞州,折冤獄,知州王曙為作《辨獄記》。高繼升知石州,人告繼升連蕃族謀變,逮捕系治,久不決,衍辯其誣,抵告者罪。寧化軍守將鞫人死罪,不以實,衍覆正之。守將不伏,訴之,詔為置獄,果不當死。徒京西路,又徙知揚州。有司
 奏衍辨獄法當賞,遷刑部。章獻太后遣使安撫淮南,使還,未及他語,問杜衍安否,使者以治狀對。太后嘆曰:「吾知之久矣。」



 徙河東轉運副使、陜西轉運使。召為三司戶部副使,擢天章閣待制、知江陵府。未行,會河北乏軍費,選為都轉運使,遷工部郎中,不增賦於民而用足。還為樞密直學士。求補外,以右諫議大夫知天雄軍。



 始,衍為治謹密,不以威刑督吏,然吏民亦憚其清整。仁宗特召為御史中丞。奏言:「中書、樞密,古之三事大臣,所謂坐而
 論道者也。止只日對前殿,何以盡天下之事?宜迭召見,賜坐便殿,以極獻替可否,其它,不必親煩陛下也。」又議常平法曰:「歲有豐兇,穀有貴賤,官以法平之,則農有餘利矣。今豪商大賈,乘時賤收,水旱,則稽伏而不出,冀其翔踴,以圖厚利,而困吾民也。請量州郡遠近,戶口眾寡,嚴賞罰,課責官吏,出納無壅,增損有宜。公糴未充,則禁爭糴以規利者;糴畢而儲之,則察其以供軍為名而假借者。州郡闕母錢,願出官帑助之。否則勸課之官,家至
 日見,亦奚益於事哉。」



 兼判吏部流內銓。選補科格繁長,主判不能悉閱,吏多受賕,出縮為奸。衍既視事,即敕吏函銓法,問曰:「盡乎?」曰:「盡矣。」力閱視,具得本末曲折。明日,令諸吏無得升堂,各坐曹聽行文書,銓事悉自予奪,由是吏不能為奸利。數月,聲動京師。改知審官院,其裁制如判銓時。遷尚書工部侍郎、知永興軍。民有晝亡其婦者,為設方略捕,立得殺人賊,發所瘞尸,並得賊殺他婦人尸二,秦人大驚。徙並州。元昊反,以太原要沖,加龍圖
 閣學士。



 寶元二年,遷刑部侍郎、復知永興軍。時方用兵,民苦調發,吏因緣為奸。衍區處計畫,量道里遠近,寬其期會,使民得次第輸官,比他州費,省錢過半。召還,權知開封府,權近聞衍名,莫敢干以私。拜同知樞密院事,改樞密副使。夏竦上攻守策,宰相欲用出師。衍曰:「僥幸成功,非萬全計。」爭議久之,求罷不許,賜手詔敦勉。為河東宣撫使,拜吏部侍郎、樞密使。每內降恩,率寢格不行,積詔旨至十數。,輒納帝前。諫官歐陽修入對,帝曰:「外
 人知杜衍封還內降邪?凡有求於朕,每以衍不可告之而止者,多於所封還也。」



 契丹與元昊戰黃河外,參知政事範仲淹宣撫河東,欲以兵自從。衍曰:「二國方交鬥,勢必不來,我兵不可妄出。」仲淹爭議帝前,詆衍,語甚切。仲淹嘗父行事衍,衍不以為恨。契丹婿劉三嘏避罪來歸,輔臣議厚館之,以詰契丹陰事。諫官歐陽修亦請留三嘏,帝以問衍。衍曰:「中國主忠信,若自違誓約,納叛亡,則不直在我。且三嘏為契丹近親,而逋逃來歸,其謀身若此,尚
 足與謀國乎!納之何益,不如還之。」乃還三嘏。拜同平章事、集賢殿大學士兼樞密使。



 衍好薦引賢士,而沮止僥幸,小人多不悅。其婿蘇舜欽,少年能文章,論議稍侵權貴,監進奏院,循前例,祠神以伎樂娛賓,集賢校理王益柔為衍所知,或言益柔嘗戲作《傲歌》,御史皆劾奏之,欲因以危衍。諫官孫甫言:「丁度因對求大用,請屬吏。」度知甫所奏誤,力求置對。衍以甫方奉使契丹,寢甫奏,度深銜之。及衍罷,度草制指衍朋比。時範仲淹、富弼欲更
 理天下事,與用事者不合,仲淹、弼既出宣撫,言者附會,益攻二人之短。帝欲罷仲淹、弼政事,衍獨左右之,然衍平日議論,實非朋比也。以尚書左丞出知兗州。慶歷七年,衍甫七十,上表請還印綬,乃以太子少師致仕。



 衍為宰相,賈昌朝不喜,議者謂故相一上章得請,以三少致仕,皆非故事,蓋昌朝抑之也。皇祐元年,特遷太子太保,召陪祀明堂,仍詔應天府敦遣就道,都亭驛設帳具幾杖待之,稱疾固辭。進太子太傅,賜其子同進士出身,又
 進太子太師。知制誥王洙謁告歸應天府,有詔撫問,封祁國公。



 衍清介不殖私產,既退,寓南都凡十年,第室卑陋,才數十楹,居之裕如也。出入從者十許人,烏帽、皂履、綈袍、革帶。或勸衍為居士服,衍曰:「老而謝事,尚可竊高士名邪!」善為詩,正書、行、草皆有法、病革,帝遣中使賜藥,挾太醫往視,不及,卒,年八十。贈司徒兼侍中,謚正獻。戒其子努力忠孝,斂以一枕一席,小壙庳塚以葬。自作遺疏,其略曰:「無以久安而忽邊防,無以既富而輕財用,宜
 早建儲副,以安人心。」語不及私。



 論曰:李迪、王曾、張知白、杜衍,皆賢相也。四人風烈,往往相似。方仁宗初立,章獻臨朝,頗挾其才,將有專制之患。迪、曾正色危言,能使宦官近習,不敢窺覦;而仁宗君德日就,章獻亦全令名,古人所謂社稷臣,於斯見之。知白、衍勁正清約,皆能靳惜名器,裁抑僥幸,凜然有大臣之概焉。宋之賢相,莫盛於真、仁之世,漢魏相,唐宋璟、楊綰,豈得專美哉!



\end{pinyinscope}