\article{列傳第二百一十五孝義}

\begin{pinyinscope}

 ○李璘甄婆兒徐承珪劉孝忠呂升王翰羅居通黃德輿齊得一李罕澄邢神留沈正許祚李琳等胡仲堯仲容陳兢洪文撫
 易延慶董道明郭琮畢贊顧忻李瓊朱泰成象陳思道方綱龐天祐劉斌樊景溫榮恕旻祁暐何保之李玭侯義王光濟李祚等江白裘承詢孫浦等常真子晏王洤等杜誼姚宗明鄧中和毛安輿李訪朱壽昌侯可申積中郝戭支漸鄧宗古沈宣蘇慶文臺亨仰忻趙伯深
 彭瑜毛洵李籌楊芾楊慶陳宗郭義申世寧茍與齡王珠顏詡張伯威蔡定鄭綺鮑宗巖附



 冠冕百行莫大於孝,範防百為莫大於義。先王興孝以教民厚,民用不薄;興義以教民睦,民用不爭。率天下而由孝義,非履信思順之世乎。太祖、太宗以來,子有復父仇而殺人者,壯而釋之;刲股割肝,咸見褒賞;至於數世同居,輒復其家。一百餘年,孝義所感,醴泉、甘露、芝草、異
 木之瑞,史不絕書,宋之教化有足觀者矣。作《孝義傳》。



 李璘,瀛州河間人。晉開運末,契丹犯邊,有陳友者乘亂殺璘父及家屬三人。乾德初,璘隸殿前散祗候,友為軍小校,相遇於京師寶積坊北,璘手刃殺友而不遁去,自言復父仇,案鞫得實,太祖壯而釋之。



 雍熙中,又有京兆鄠縣民甄婆兒,母劉與同里人董知政忿競,知政擊殺劉氏。婆兒始十歲,妹方襁褓,托鄰人張氏乳養。婆兒避仇,徙居赦村,後數年稍長大,念母為知政所殺,又念其
 妹寄張氏,與兄課兒同詣張氏求見妹,張氏拒之,不得見。婆兒憤怒悲泣,謂兄曰:「我母為人所殺,妹流寄他姓,大仇不報,何用生為!」時方寒食,具酒肴詣母墳慟哭,歸取條桑斧置袖中,往見知政。知政方與小兒戲,婆兒出其後,以斧斫其腦殺之。有司以其事上請,太宗嘉其能復母仇,特貸焉。



 徐承珪,萊州掖人。幼失父母,與兄弟三人及其族三十口同甘藜藿,衣服相讓,歷四十年不改其操。所居崇善
 鄉緝俗裏,木連理,瓜瓠異蔓同實,州以聞。乾德元年,詔改鄉名義感,里名和順。承珪嘗為贊皇令。



 劉孝忠,並州太原人。母病經三年,孝忠割股肉、斷左乳以食母;母病心痛劇,孝忠然火掌中,代母受痛。母尋愈。後數歲母死,孝忠傭為富家奴,得錢以葬。富家知其孝行,養為己子。後養父兩目失明,孝忠為舐之,經七日復能視。以親故,事佛謹,嘗於像前割雙股肉,注油創中,然燈一晝夜。劉鈞聞而召見,給以衣服、錢帛、銀鞍勒馬,署
 宣陵副使。開寶二年,太祖親征太原,召見慰諭。



 呂升,萊州人。父權失明,剖腹探肝以救父疾,父復能視而升不死。冀州南宮人王翰,母喪明,翰自抉右目睛補之,母目明如故。淳化中,並下詔賜粟帛。



 羅居通,益州成都人。母死,廬墓三年,有甘露降墳樹,芝草生其旁。開寶四年,長吏以聞,詔以居通為延長主簿。



 大中祥符初,資州人黃德輿葬父母,負土成墳,甘泉湧其側,降詔旌表。



 齊得一,密州諸城人。幼嗜學,及長,能讀《五經》,善於教授鄉里。士大夫子弟不遠百里,皆就之肄業焉。晉末,皇甫暉為密州防禦使,得一父為客將。及暉叛歸淮南,屢率眾剽劫於故郡,民之牛羊犬豕悉取以犒士卒,得一之家被略殆盡。後王萬敢為防禦使,性貪暴,執鄉民十八家,責其嘗以牛酒饋賊,盡殺之而取其資產,得一親屬死者十餘人,唯得一與兄脫身獲免。明年詣闕上訴,朝廷遣使按鞫之得實,萬敢削官,判官胡轍輒坐死。得一乃
 歸鄉里,布衣蔬食,不樂仕進。開寶中,詔郡國舉廉退孝悌之士,本郡即以得一應詔。至闕,策試中選,授章丘主簿。



 李罕澄,冀州阜城人也,七世同居。漢乾祐三年,詔改鄉里名及旌其門閭。太平興國六年,長吏以漢所賜詔書來上,復旌表之。



 刑神留,深州陸澤人。父超,逋官租,里胥督租,與超斗,超歐里胥死。神留年十六,詣吏求代父死。州以聞,特詔減
 死,賜裏胥家萬錢為棺斂具。



 端拱初,泰州海陵人沈正父為屯田院衙官,兇暴無賴,使酒毆平人死,正中塗見,父恐懾,述其故,正即號呼褫衣,就毆其尸。巡警者捕送官,獄具,怡然就死,聞者悲之。



 許祚,江州德化人。八世同居,長幼七百八十一口。太平興國七年,旌其門閭。淳化二年,本州言祚家春夏常乏食,詔歲貸米千斛。



 又有信州李琳十五世同居,貝州田祚、京兆惠從順十世同居,廬州趙廣、順安軍鄭彥圭、信
 州俞雋八世同居,陜州張文裕六世同居,襄州張巨源、劉芳、潭州瞿景鴻、溫州陳偘、江陵褚彥逢五世同居,徐州彭程四世同居,皆賜詔旌表門閭。巨源素習法律,太平興國五年,賜明法及第。芳淳化四年來賀壽寧節,賜進士出身。偘事母至孝,賜其母粟帛。彥逢兄弟五人皆年七十餘,至道元年,轉運使表其事,詔補彥逢教練使。



 胡仲堯,洪州奉新人。累世聚居,至數百口。構學舍於華林山別墅,聚書萬卷,大設廚廩,以延四方游學之士。南
 唐李煜時嘗授寺丞。雍熙二年,詔旌其門閭。仲堯詣闕謝恩,賜白金器二百兩。淳化中,州境旱歉,仲堯發廩減市直以振饑民,又以私財造南津橋。太宗嘉之,除本州助教,許每歲以香稻時果貢於內東門。五年,遣弟仲容來賀壽寧節。召見仲容,特授試校書郎,賜袍笏犀帶,又以御書賜之。公卿多賦詩稱美。仲堯稍遷國子監主簿,致仕,卒。



 仲容字咸和,咸平三年,復至闕貢土物,改大理評事,屢被賜齎。仲容建本縣孔子廟,頗為宏敞。後遷光
 祿丞致仕,天禧中,特賜緋魚。卒,年七十九。以弟之子用訥為後,試校書郎。仲容弟克順,端拱二年進士,至都官員外郎、三司戶部判官。仲容子用之洎從子用莊、用舟,並進士及第。



 陳兢,江州德安人,陳宜都王叔明之後。叔明五世孫兼,唐右補闕。兼生京,秘書少監、集賢院學士,無子,以從子褒為嗣,褒至鹽官令。褒生灌,高安丞。灌孫伯宣,避難泉州,與馬總善注司馬遷《史記》行於世;後游廬山,因居德
 安,嘗以著作佐郎召,不起,大順初卒。伯宣子崇為江州長史,益置田園,為家法戒子孫,擇群從掌其事,建書堂教誨之。僖宗時嘗詔旌其門,南唐又為立義門,免其徭役。崇子袞,江州司戶。袞子昉,試奉禮郎。



 昉家十三世同居,長幼七百口,不畜僕妾,上下姻睦,人無間言。每食,必群坐廣堂,未成人者別為一席。有犬百餘,亦置一槽共食,一犬不至,群犬亦皆不食。建書樓於別墅,延四方之士,肄業者多依焉。鄉里率化,爭訟稀少。開寶初,平江南,
 知州張齊上請仍舊免其徭役,從之。昉弟之子鴻。太平興國七年,江南轉運使張齊賢又奏免雜科。兢即鴻之弟。淳化元年,知州康戩又上言兢家常苦食不足,詔本州每歲貸粟二千石。



 後兢死,其從父弟旭每歲止受貸粟之半,雲省嗇而食,可以及秋成。屬歲儉穀貴,或勸其全受而糶之,可邀善價,旭曰:「朝廷以旭家群從千口,軫其乏食,貸以公粟,豈可見利忘義,為罔上之事乎?」至道初,遣內侍裴愈就賜御書,還,言旭家孝友儉讓,近於淳
 古。太宗嘗對近臣言之,參知政事張洎對曰:「旭宗族千餘口,世守家法,孝謹不衰,閨門之內,肅於公府。」且言及旭受貸事。上以遠民義聚,復能固廉節,為之嘆息。大中祥符四年,以旭為江州助教。旭卒,弟蘊主家事。天聖元年,又以蘊繼為助教。蘊卒,弟泰主之。泰弟度,太子中舍致仕。從子延賞、可,並舉進士。延賞職方員外郎。



 洪文撫,南康建昌人,本姓犯宣祖偏諱,改焉。曾祖諤,唐虔州司倉參軍,子孫眾多,以孝悌著稱。六世義居,室無
 異爨。就所居雷湖北創書舍,招來學者。至道中,本軍以聞,遣內侍裴愈齎御書百軸賜其家。文撫遣弟文舉詣闕貢土物為謝,太宗飛白一軸曰「義居人」以賜之,命文舉為江州助教。三年八月,又詔表其門閭。自是每歲遣子弟入貢,必厚賜答之。文撫兄子待用,登咸平二年進士第,至都官員外郎。



 易延慶字餘慶,筠州上高人。父贇,以勇力仕南唐至雄州刺史。延慶幼聰慧,涉獵經史,尤長聲律,以父蔭為奉
 禮郎。顯德四年,周師克淮南,贇歸朝,授道州刺史;延慶亦授大名府兵曹參軍,後為大理評事,知臨淮縣。乾德末,贇卒,葬臨淮。延慶居喪摧毀,廬於墓側,手植松柏數百本,旦出守墓,夕歸侍母。紫芝生於墓之西北,數年又生玉芝十八莖。本州將表其事,延慶懇辭。或畫其芝來京師,朝士多為詩賦,稱其孝感。



 服闋,延慶以母老稱疾不就官。母卒後,槁殯數年,延慶出為大理寺丞。嘗司建安市征,及母葬有期,私歸營葬,掩壙而返。知軍扈繼升
 言其擅去職,坐免所居官,復廬墓側數年。母平生嗜慄,延慶樹二慄樹墓側,二樹連理。蘇易簡、朱臺符為贊美之。後知端州,卒。子綸,大中祥符元年,進士及第。



 董道明,蔡州褒信人。母死出葬,道明潛匿墓中,人瘞之,經三日,家人發塚取之,道明無恙,終身廬於墓側。



 郭琮,臺州黃巖人。幼喪父,事母極恭順。娶妻有子,移居母室。凡母之所欲,必親奉之。居常不過中食,絕飲酒茹葷者三十年,以祈母壽。母年百歲,耳目不衰,飲食不減,
 鄉里異之。至道三年,詔書存恤孝悌,鄉老陳贊率同里四十人狀琮事於轉運使以聞,有詔旌表門閭,除其徭役。明年,母無疾而終。琮哀號幾乎滅性,鄉閭率金帛以助葬。



 又有越州應天寺僧者,幼貧無以養母,剃發乞食以給晨夕。母年一百五歲而終。



 潭州長沙人畢贊,仕郡為引贊吏,性至孝,父母皆年八十餘。轉運使表其事,詔贊解職終養。



 顧忻,泰州泰興人。十歲喪父,以母病,葷辛不入口者十
 載。雞初鳴,具冠帶率妻子詣母之室,問其所欲,如此五十年,未嘗離母左右。母老,目不能睹物,忻日夜號泣祈天,刺血寫佛經數卷。母目忽明,燭下能縫衽,九十餘無疾而終。



 又有杭州仁和人李瓊,以鬻繒為業,事母孝,夜常十餘起省母。母喜食時新,瓊百方求市,得必十倍酬其直。



 朱泰,湖州武康人。家貧,鬻薪養母,常適數十里外易甘旨以奉母。泰服食粗糲,戒妻子常候母色。一日,雞初鳴
 入山,及明,憩於山足,遇虎搏攫負之而去。泰已瞑眩,行百餘步,忽稍醒,厲聲曰:「虎為暴食我,所恨母無托爾!」虎忽棄泰於地,走不顧,如人疾驅狀。泰匍匐而歸。母扶持以泣,泰亦強舉動,不逾月如故。鄉里聞其孝感,率金帛遺之,里人目為朱虎殘。



 成象,渠州流江人。以詩書訓授里中,事父母以孝聞。母病,割股肉食之,詔賜束帛醪酒。淳化中,李順盜據郡縣,象父母驚悸而死,燼骨寄浮圖舍,象號泣營葬。賊平,鄉
 里率錢三百萬贈之。象廬於墓側,以衰服襟袂篩土於墳上,日三斗。每慟,聞者戚愴。未嘗食肉衣帛,或贈之亦不受。虎豹環廬而臥,象無畏色。燕百餘集廬中,禾生墓側吐九穗。服終猶未還家,知禮者為書以諭之,遂歸教授,遠近目為成孝子。



 陳思道,江陰人。喪父,事母兄以孝悌聞。鬻醯市側,以給晨夕,買物不酬價,如所索與之。母病,思道衣不解帶者數月,雙目瘡爛,飲食隨母多少。洎母喪,水漿不入口七
 日。既葬,裒鬻醯之利,得錢十萬,奉其兄。結廬墓側,日夜悲慟,其妻時攜兒女詣之,拒不與見。夏日種瓜,以待過客。晝則白兔馴狎,夜則虎豹環其廬而臥。咸平元年,知軍上其事,詔賜束帛,旌其門。



 方綱,池州青陽人。八世同爨,家屬七百口,居室六百區,每旦鳴鼓會食。嘗出稻五千𥮬振貸貧民。景德二年,轉運使馮亮以聞,詔旌其門。天禧中,侍御史韓億安撫江南,使還,言綱家稅籍錢四百餘千,米二千五百斛,同居
 四百年,而本縣科率一無寬假,望蠲其戶雜科,詔從之。



 龐天祐,江陵人。以經籍教授里中。父疾,天祐割股肉食之;疾愈,又復病目喪明,天祐號泣祈天舐之。父年八十餘,大中祥符四年卒,天祐負土封墳,結廬其側,晝夜號不絕聲。知府陳堯咨親往致奠,上其事,詔旌表門閭。天祐家無儋石儲,居委巷中,堯咨為徙里門之右,築闕表之。



 劉斌,定州人。父加友,端拱中為從弟志元所殺。斌兄弟
 皆幼,隨母改適人,母嘗戒之曰:「爾等長,必復父仇。」景德中,斌兄弟挾刀伺志元於道,刺之不殊,即詣吏自陳。州具獄上請,詔志元黥面配隸汝州,釋斌等罪。



 樊景溫,陜州芮城人;榮恕旻,雄州歸信人。兄弟異居積年。大中祥符中,景溫樗樹五枝並為一,恕旻家榆樹兩本自合,兩家感其異,復義聚,鄉人稱雍睦。



 祁暐字坦之,萊州膠水人。淳化三年進士,歷度支員外郎、直集賢院。天禧中,出知濰州,母卒。葬於州城之南。暐
 既解官,就墳側構小室,號泣守護,蔬食,經六冬,墮足二指。有白烏白兔馴擾墳側,州人異之,以狀聞。有詔旌美,賜帛三十匹、粟三十石,令長吏每月存問。



 何保之,梓州通泉人。業進士,有至行。母卒,負土成墳,廬於其側。日有群烏飛集墳上,哀鳴不去,又嘗有兔馴於坐隅,人稱異焉。大中祥符降詔旌恤。



 李玭,大名宗城人。性篤孝,力耕以事母。母卒,讓田與其弟堅,遂廬於葬所,晝夜號泣,負土築墳高丈餘。又以二
 代及諸族父母槁葬者盡禮築之,凡三年成六墳,皆丈餘。不食肉衣帛,不預人事,遑遑然唯恐築之不及,墳成,復留守墳三年。常令兄之子賣藥以自給。年六十餘,足未嘗入縣門。鄉人目為李孝子。天禧中,知府張知白以狀聞,詔賜粟帛,令府縣安存之。里中有母在而析產者聞玭被旌,兄弟慚懼,復相率同居。



 侯義,應天府楚丘人。貧無產,傭田以事母。里人有葬其親而遽返者,義母過其塚,泣謂義曰:「我死,其若是乎!」義
 乃感激自誓而不欲言,但慰其母曰:「勿悲,義必不爾。」咸平中,母卒,義力自辦葬,不掩墳壙,晝則負土築墳,夜則慟哭柩側。妻子困匱不給,田主曹氏哀憐之,資以餱糧。逾年,墳間瓜異蒂、木連理,又有巨蛇繞其側不暴物,野鴿飛而不去。嘗遇盜劫其衣服,既而知是義物,悉還之。



 王光濟,廬州人。喪母,因刻像日夕奉事如平生,孝道純篤。咸平二年,本州以孝聞,有詔旌之。



 時又有徐州豐人李祚,親喪,廬墓側凡二十七年,家人百計勉諭,不聽。益
 州雙流人周善敏,喪父,廬於墓側。母病,又割股肉以啖之,遂愈。大中祥符九年,特詔旌表祚,賜善敏粟帛存慰之。



 江白,建昌人。景德二年進士。父禹錫,有節義,高年不仕,躬自教授,大中祥符初,獻《東封詩》十五篇,有詔嘉美,賜以粟帛,歲時遣使存問。五年,卒。白自鄞尉罷還,負土營葬,廬於墓側,藜羹芒屩,晝夜號泣,將終制猶然。轉運使以其狀聞,詔賜帛二十匹,粟麥二十石,醪酒十缸。



 裘承詢,越州會稽人。居雲門山前,十九世無異爨。子弟習弦誦,鄉里稱其敦睦。州以聞,詔旌其門閭。



 咸平後,又有保定軍孫浦、襄州常元紹、蔡州王美、解州董孝章並十世同居;莫州高珪、永定軍朱仁貴、潞州邢濬、相州趙祚八世同居;麟州楊榮、隰州趙友、開封李居正、潁州張可象、衛州張珪、滄州崔諒七世同居;邢州王覺、趙州曹遵六世同居;兗州童升、陳州樊可行、京兆元守全、平定軍段德五世同居;開封張仁遇、亳州王子上、建昌軍瞿
 肅四世同居。肅家百五十口,長幼孝悌,鄉人化之。又河陰王世及、大名李宗祐、陳州劉閏、宣州汪政、潭州李耕,或聚居至七百口,累數十百年。並所在請加旌表,詔從之,仍蠲其課調。



 大中祥符初,東封泰山,判兗州王欽若言曲阜東野宜、乾封竇益合居五六世,有節行。四年,祀汾陰,考制度使馬起言陜州張化基、閻用和、楊忠義聚族累世,孝悌可稱。並即行在所降詔褒美,各優賜粟帛。



 常真,陳州項城人。父母死,廬墓終喪,負土成墳,不茹葷
 血。周廣順中,詔旌其門閭。開寶七年,本州以聞,詔再加旌表。真妻病,子晏割股肉以養母,及死,次子守規徒跣,日一食,廬墓三年。太平興國八年,詔旌表之。



 又有齊州王洤、河南李繼成、滄州胡元興,並母死負土成墳,晝夜哭不絕聲。州郡繼以聞,皆降詔旌其門閭,賜以粟帛。



 杜誼字漢臣,臺州黃巖人。事父母至孝。父剛嚴,誼獨失愛,惴惴不自容,伺顏色而後進。繼喪父母,號慟晝夜不絕,勺水不入口者累日。卜葬,徒跣負土為墳,往來十餘
 里,日渡塘澗,泥水沒骭,雖大雨雪未嘗少止。手足皸裂血流,以漆塗之。每覆一畚,必三繞墳號而後去。既葬,遂茇舍墓旁,負土終喪,人往視之,輒遣去。日一飯,不葷。雖虎狼交於墓側,誼泰然無所畏。明年,吳越大水,山皆發洚,推巨石走十數里。臺州山最高而水又夜至,旁山之民,居廬、墓田、畜牧漂壞者甚眾,而獨不及誼。邑人狀其事以聞,詔書嘉獎。



 事族父衍甚謹,衍愛之均諸子。以祖垂象蔭入官,至贊善大夫。嘗知永城縣,歲捐奉錢三
 十萬,以收瘞汴渠之溺死者凡四十餘。又出奉錢率其下新文宣王廟,兩旁為學舍數十區,旦夕講學於其堂。永城父老稱誼之政為不可及。



 誼生平敦厚,尚信義,有大志,家貧,不恤有無,常推以濟親友。後通判梓州,卒。子揆才十六歲,哭誼墓旁卒。



 姚宗明,河中永樂人也。其十世祖棲雲。當唐貞元中,調卒戍邊,棲雲之父語其兄曰:「兄嗣未立,可無往。某幸有子,請代兄行。」遂戰沒塞上。時棲雲方三歲,其母再嫁,棲
 雲養於伯母。既長,事伯母如其母,伯母亡,棲雲葬之。又招魂葬其父,痛其父死於邊,乃廬於墓次,終身哀慕不衰。縣令蘇轍以俸錢買地,開阡刻石表之。河中尹渾瑊上其事,詔加優賜,表其門,名其鄉曰孝悌,社曰節義,里曰敬愛。



 棲雲生岳,岳生君儒,君儒生師正。自岳至師正,四世廬墓。五世孫曰厚,六世曰雅,七世曰文,八世曰敬真,九世曰直,十世曰宗明。當慶歷初,有司以姚氏十世同居聞於朝,仁宗詔復其家。十一世孫用和,十二世孫
 士明,十三世孫德。自宗明至德又三世,自慶歷以後又五十餘年,而其家孝睦不替。



 姚氏世為農,無為學者。家不甚富,有田數十頃,聚族百餘人。子孫躬事農桑,僅給衣食,歷三百餘年無異辭者。經唐末、五代,兵戈亂離,而子孫保守墳墓,骨肉不相離散,求之天下,未或有焉。



 鄧中和字祖德,開封長垣人。舉《三禮》。景祐、慶歷間喪親,廬墓終其喪,定省往來如事生者二十年,負土累墳高三丈。



 毛安輿,嘉州洪雅人。年九歲父死,負土為墳,廬於其側三年。知益州張方平聞之,遺以酒餼,狀其事以聞。



 李訪,韶州人,業進士。廬父母墓,有虎暴傷旁人而不近訪,又有白烏集墓上。



 朱壽昌字康叔,揚州天長人。以父巽蔭守將作監主簿,累調州縣,通判陜州、荊南,權知嶽州。州濱重湖,多水盜。壽昌籍民船,刻著名氏,使相伺察,出入必以告。盜發,驗船所向窮討之,盜為少弭,旁郡取以為法。



 富弼、韓琦為
 相,遣使四出寬恤民力,擇壽昌使湖南。或言邵州可置冶採金者,有詔興作。壽昌言州近蠻,金冶若大發,蠻必爭,自此邊境恐多事,且廢良田數百頃,非敦本抑末之道也。詔亟罷之。



 知閬州,大姓雍子良屢殺人,挾財與勢得不死。至是,又殺人而賂其里民出就吏。獄具,壽昌覺其奸,引囚詰之曰:「吾聞子良與汝錢十萬,許納汝女為婦,且婿汝子,故汝代其命,有之乎?」囚色動,則又擿之曰:「汝且死,書券抑汝女為婢,指錢為顧直,又不婿汝子,將
 奈何?」囚悟,泣涕覆面,曰:「囚幾誤死。」以實對。立取子良正諸法。郡稱為神,蜀人至今傳之。



 知廣德軍。壽昌母劉氏,巽妾也。巽守京兆,劉氏方娠而出。壽昌生數歲始歸父家,母子不相聞五十年。行四方求之不置,飲食罕御酒肉,言輒流涕。用浮屠法灼背燒頂,刺血書佛經,力所可致,無不為者。熙寧初,與家人辭訣,棄官入秦,曰:「不見母,吾不反矣。」遂得之於同州。劉時年七十餘矣,嫁黨氏有數子,悉迎以歸。京兆錢明逸以其事聞,詔還就官,由是
 以孝聞天下。自王安石、蘇頌、蘇軾以下,士大夫爭為詩美之。壽昌以養母故,求通判河中府。數歲母卒,壽昌居喪幾喪明。既葬,有白烏集墓上。拊同母弟妹益篤。



 又知鄂州,提舉崇禧觀,累官司農少卿,易朝議大夫,遷中散大夫,卒,年七十。壽昌勇於義,周人之急無所愛,嫁兄弟兩孤女,葬其不能葬者十餘喪,天性如此。



 侯可,字無可,華州華陰人。少倜儻不羈,以氣節自許。既壯,盡易前好,篤志為學。隨計入京,里中醵金贐行。比還,
 悉散其餘與同舉者,曰:「此金,鄉里所以資應詔者也,不可以為他利。」且行,聞鄉人病,念曰:「吾歸,則彼死矣!」遂留不去。病者愈,輟己馬載之,徒步而歸。



 孫沔征儂猺,請參軍事,奏功得官,知巴州化城縣。巴俗尚鬼而廢醫,唯巫言是用。娶婦必責財,貧人女至老不得嫁。可為約束,立制度,違者有罪,幾變其習。再調華原主簿。富人有不占田籍而質人田券至萬畝,歲責其租。可晨馳至富家,發櫝出券歸其主。郡吏趙至誠貪狡兇橫,持守以下短長,
 前後莫能去。可暴其罪,荷校置獄,言於大府誅之,聞者快服。



 簽書儀州判官。西夏寇邊,使者使可按視,即以數十騎涉夏境,猝與之遇,亟分其騎為三四,令之曰:「建爾旗幟,旋山徐行。」夏人循環間見,疑以為誘騎不敢擊。韓琦鎮長安,薦知涇陽縣。說渭源羌酋輸地八千頃,因城熟羊以撫之。琦上其功。又議復鄭白渠,得召對,旋以微罪罷。官至殿中丞,卒於家,年七十二。



 可輕財樂義,急人之急,憂人之憂。與田顏為友。顏病重,千里求醫,未歸而
 顏死,目不瞑。人曰:「其待侯君乎?」且斂而可至,拊之乃瞑。顏無子,不克葬,可辛勤百營,鬻衣相役,卒葬之。方天寒,單衣以居,有饋白金者,顧顏之妹處室,舉以佐其奩具。一日自遠歸,家以窶告,適友人郭行扣門曰:「吾父病,醫邀錢百千,賣吾廬而不售。」可惻然,計橐中裝略當其數,盡與之。關中稱其賢。



 申積中,成都人。襁褓中,楊繪從其父起求之為子。及長,知非楊氏而絕口不言。年十九,登進士第。事所養父母,
 盡孝終身。有二弟一妹,為畢婚娶,始歸本族,復為申氏,蜀人以純孝歸之。政和六年,以奉議郎通判德順軍。翰林學士許光凝嘗守成都,得其事薦諸朝,召赴京師,擢提舉永興軍學事,道卒。光凝復與宣和殿學士薛嗣昌、中書舍人宇文黃中表其操行,詔予一子官。



 初,光凝所同薦者三人:其一河陽故大理丞陳芳,一門十四世,同居三百年;一鄧州王襄,經術登科,年未六十,請老,事孀嫂如母,養孤甥如子,教誨後進,周恤鄉里貧民,以學行
 稱。乞加獎異。詔表芳門閭,賜襄號「處士」。



 郝戭,字伯牙,石州定胡人。家貧,竭力營養。或憐傷之,貸以錢數百萬,使取息自贍,戭重謝,留錢五六年不用,復返之。舉進士,調宛丘尉、舞陽主簿、通山令。時年未五十,以父樵老不第,上書請致仕,為父求官。執政諭使赴官而後請,曰:「如是,則可升朝籍,遇恩及親矣。」於是留妻子於家,獨奉父行,逾歲竟謝事。上官以其治縣有績,惜其去,固留之;耆老拜庭遮道,皆不能止。得太子中允以歸,
 未至鄉里而樵卒。自畚土造塚,人有助之者,使置土塚上,去則隨撤之。服除,州以狀聞,詔賜粟帛。



 治平末,以翰林學士呂公著薦,起為奉寧軍推官,涇原經略使亦奏闢幕府。ρ曰:「向所以未老致仕,欲官及親也。既不能及,尚庶幾以恩得贈,今則無及矣!」姻族語其妻聶氏,使勸ρ仕,曰:「吾不德,無以助君子,矧敢強其所不欲以累其高哉。」聶事舅姑亦以孝義著。戭忠信自將,篤行苦節,不仕而卒。司馬光為銘其墓。



 支漸,資州資陽人。年七十,持母喪,既葬,廬墓側,負土成墳,蓬首垢面,三時號泣,哀毀瘠甚。白蛇貍兔擾其旁,白雀白烏日集於壟木,五色雀至萬餘,回翔悲鳴若助哀者。鄉人句文鼎自娶婦即與父母離居,睹漸至行,深自悔責,號慟而歸,孝養盡志。鄉閭觀感而化者甚眾。



 鄧宗古,簡州陽安人。父死,自培土為墳,廬其側,晨夕號慟,甘露降於墓木。里中號為鄧孝子。



 沈宣,汝州梁人。母亡,既葬,不塞墓門三十有六月,晝負
 土,夜拊棺而臥,為墳廣百尺。妻高氏亦有孝行。



 漸以下三人,元豐中,皆褒賜粟帛。



 蘇慶文、臺亨,皆夏縣人。慶文事父母以孝聞。母少寡,慶文懼其妻不能敬事,每戒之曰:「汝事吾母,少不謹必逐汝。」妻奉教,母得安其室終身。



 亨工畫,元豐中,朝廷修景靈宮,調天下畫工詣京師,選試其優者待詔翰林,畀以官祿,亨名第一。以父老固辭歸養,閭里賢之。



 仰忻,字天貺,溫州永嘉人。力學,以篤行稱。年五十餘,執母喪盡孝禮。躬自負土,廬於墓側,有慈烏白竹之瑞。紹聖中,郡守楊蟠表其里「孝廉坊」。大觀二年,以行取士,郡以忻應詔。未幾卒,特贈將仕郎。



 趙伯深,字逢原。父子佪,宣和間為棣州兵官屬。會兵動燕云,子佪被檄往塞上。伯深時尚幼,與其母張留居棣州。既而金人渡河,伯深母子相失。子佪亦隔絕,建炎二年,始得南歸。子佪卒,伯深訪尋其母二十餘年。一旦聞
 在滬南,伯深徒步入蜀,間關累年。紹興二十一年,乃得其母,相持號泣,哀感行路。曾慥在夔州,賦詩以美其孝。



 彭瑜,字君玉,吉之安福人。熙寧間失其母,瑜朝夕焚香祈天,願知母所在,如是十餘年。俄有人言母為泰和倪氏婦,瑜竟迎以歸。



 毛洵字子仁,吉州吉水人。天聖二年進士,又中拔萃科。性至孝,凡守四官,再以親疾解任,執藥調膳,嘗而後進,三月不之寢室。父應佺通判太平州,卒官,母高繼卒於
 池陽舟次。持鍤荷土以為墳,手胝面黔,親友不能識,廬於墓凡二十一月,朝夕哭踴,食裁脫粟。諸生請問經義,對之流涕,未嘗言文。抱疾歸,數日而卒。郡以孝聞,賜其家帛五十匹、米五十斛。兄溥,字文祖,亦以哀毀卒於舟中。



 李籌者,洵同縣人,字彥良。與弟衡字平國生同乳,二歲喪母,十歲喪父,兄弟每以不逮事親為恨。政和中,改葬其母於楊山,負土成墳,廬於墓左。未幾,廬所產木一本兩干,高丈許復合於一,至其末乃分兩干五枝,鄉人
 以為瑞。



 有楊芾者,亦同縣人,字文卿,性至孝,歸必市酒肉以奉二親,未嘗及妻子。紹興五年大饑,為親負米百里外,遇盜奪之不與,盜欲兵之,芾慟哭曰:「吾為親負米,不食三日矣。幸哀我。」盜義而釋之。



 楊慶,鄞人。父病,貧不能召醫,乃刲股肉啖之,良已。其後母病不能食,慶取右乳焚之,以灰和藥進焉,入口遂差,久之乳復生。宣和三年,守樓異名其坊曰「崇孝」。紹興七年,守仇悆為之請。十二年,詔表其門,復之。悆曰:「韓退之
 作《鄠人對》,以毀傷支體為害義。而匹夫單人,身膏草莽,軌訓之理未宏,汲引之徒多闕,而乃行成於內,情發自天。使稍知詩書禮義之說,推其所存,出身事主,臨難伏節死義,豈減介之推、安金藏哉!」



 陳宗,永嘉人。年十六,母蔡病篤,刲股為餌,病愈。已而復病不救,宗一慟而絕。郡守陸德輿云:「陳宗自毀其體,哀慟傷生,雖非孝道之正,而能為人所難為之事,亦天性之至。」官為合葬,榜曰「陳孝子墓」。



 郭義,興化軍人。早游太學,以操尚稱。年四十餘,客錢塘,聞母喪,徒跣奔喪,每一慟輒嘔血。家貧甚,故人有所饋,不受。聚土為墳,手蒔松竹,而廬於其旁。甘露降於墓上,烏鵲馴集。郡上其事,詔旌表其閭,於所居前安綽楔,左右建土臺,高一丈二尺,方正,下廣上狹,飾白,間以赤,仍植所宜木。



 申世寧,信州鉛山人。紹興六年,潘達兵襲鉛山,父愈年七十,未及出戶遇賊,賊意其有藏金,欲殺之。世寧年未
 冠,亟引頸願代父死,賊感其孝,兩全之。



 茍與齡字壽隆,滁州來安人。志尚高潔,事其親,生養死葬,力竭而禮盡,鄉黨稱之。母歿,廬墓側,有芝十九莖生於墓亭。郡縣以事聞,旌其門。



 王珠字仲淵,吉州龍泉人,以孝謹聞。建炎間,居父憂,芝數本生墓側,倒植竹以為杙,復生柯葉。紹興間,再罹母喪,復有雙竹靈芝之祥。



 顏詡,唐太師真卿之後。真卿嘗謫廬陵,故詡為吉州永
 新人。詡少孤,兄弟數人,事繼母以孝聞。一門千指,家法嚴肅,男女異序,少長輯睦,匜架無主,廚饌不異。義居數十年,終日怡愉,家人不見其喜慍。年七十餘卒。



 張伯威,大安軍人。武翼大夫、御前前軍正將祥之子。紹熙元年,武舉進士。調神泉尉。大母黃,年九十八,不忍之官。黃得血痢疾瀕殆,伯威剔左臂肉食之,遂愈。繼母楊因姑病篤,驚而成疾,伯威復剔臂肉作粥以進,其疾亦愈。伯威妹嫁崔均,其姑王疾,妹亦剔左臂肉作粥以進,
 達旦即愈。知大安軍羅植即伯威所居立純孝坊,崔均所居立孝婦坊。事聞,詔伯威與升擢,倍賜其妹束帛。



 蔡定,字元應,越州會稽人。家世微且貧。父革,依郡獄吏傭書以生,資定使學,游鄉校,稍稍有稱。郡獄吏一日坐舞文法被系,革以詿誤,年七十餘矣,法當免系。鞫胥任澤削其籍年而入之,罪且與獄吏等。案具,府奏上之。方待命於朝,故俱久囚,而革不得獨決。定切痛念父當耆年,以非辜墮圄狴,誓將身贖。數詣府號訴,請代坐獄,弗
 許;請效命於戎行,弗許;請隸五符為兵,又弗許。定知父終不可贖也,仰而呼曰:「天乎!將使定坐視父纏徽纆乎!父老耄,不應連系;傭書,罪不應與獄吏等。理明矣,而無所云訴。父老而刑,定之生其何益乎?定圖死矣,庶有司哀憐而釋父,則雖死無憾矣!」於是預為志銘其墓,又為狀若詣府者結置袂間,皆敘陳致死之由,冀其父之必免也。以建炎元年十二月甲申,自赴河死。府帥聞之,驚曰「真孝」,立命出革,厚為定具棺斂事,而撫周其家。



 鄭綺,婺州浦江人。善讀書,通《春秋穀梁》學。以肅睦治家,九世不異爨。四世孫德珪、德璋,孝友天至,晝則聯幾案,夜則同衾寢。德璋素剛直,與物多迕,宋亡,仇家遂陷以死罪,當會逮揚州。德珪哀弟之見誣,乃陽謂曰:「彼欲害吾也,何預爾事?我往則奸狀白,爾去得不死乎!」即治行。德璋追至諸暨道中,兄弟相持頓足哭,爭欲就死。德珪默計沮其行,遂紿以無往,夜將半,從間道逸去。德璋復追至廣陵,德珪已斃於獄。德璋聞之,慟絕者數四,負骨
 歸葬。廬墓再期,每一悲號,烏鳥皆翔集不食。德珪之子文嗣,幼病僂,德璋鞫之如己子。



 有鮑宗巖者,字傅叔,徽州歙人。子壽孫字子壽。宋末,盜起里中。宗巖避地山谷間,為賊所得,縛宗巖樹上,將殺之。壽孫拜前願代父死,宗巖曰:「吾老矣,僅一子奉先祀,豈可殺之?吾願自死。」盜兩釋之。



\end{pinyinscope}