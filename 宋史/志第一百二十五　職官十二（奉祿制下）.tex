\article{志第一百二十五 職官十二(奉祿制下)}

\begin{pinyinscope}

 增給公用錢給券職田



 增給



 權三司使,知開封府,百千。權發遣三司使,五十千。玉清昭應宮、景靈宮、會靈觀三
 副使,觀文殿大學士,三十千。觀文殿學士,資政殿大學士,元豐添保和殿大學士。



 宮觀、三司判官,判子司,權及權發遣同。開封府判官,提舉諸司庫務,管轄三司軍大將,提點內弓箭庫,二十千。宮觀都監、勾當官,十七千。



 任都知、押班者,二十千。資政、端明、翰林侍讀、元祐復置翰林侍讀、侍講學士,紹聖中罷。



 龍圖、天章學士,元豐添保和、延康、定文、顯謨、徽猷學士。



 樞密直、後改述古殿。



 龍圖、天章直學士,元豐添定文、顯謨、微猷直學士,保和、龍圖、天章、寶文、顯謨、徽奠待制。



 十五千。



 春、冬綾各五匹,絹十七匹,羅一匹,綿五十兩。已上大學士至待制,奉隨本官,衣賜如本官例,大即依本官例;小即依逐等。
 大觀
 二年,戶部尚書左睿言:「見編修《祿格》,學士添支比正任料錢相去遼邈,如觀文殿大學士、節度使從二品,大學士添支三十千而已,節度使料錢乃四百千,傔



 從、粟帛等稱是。或謂大學士有寄祿官料錢,故添支數少。今以銀青光祿大夫任觀文殿大學士較之,則通料錢不及節度使之半,其厚薄不均明矣。自餘學士視諸正任,率皆不等。欲將職錢改作貼職錢以別之。正任料錢、公使為率,參酌立定。自學士至直閣以上貼職錢,
 不以內外,並
 給。觀文殿大學士,
 百千。觀文學士,資政大學士,八十千。端明後改延康殿學士,五十千。前執政加二十千。龍圖、天章、寶文、顯謨、徽猷學士,樞密直改述古學士,四十千。龍圖、天章、寶文、顯謨、徽猷直學士,三十千。待制,二十千。集賢改集英殿修撰,十五千。直龍圖閣至直秘閣,十千。」詔從之。宣和三年,戶部尚書沈積中、侍郎王蕃言:「元豐法,帶職人依《嘉祐祿令》,該載觀文殿大學士以下至直學士,添支錢三等,自三十千至十五千。大觀中,因敕令所啟請,改作貼職錢,觀文大學士至直秘閣,自百千至十千,凡九等。兼增添
 在京供職米麥,觀文殿大學士至待制,自五十石至二十五石四等,比舊法增多數倍。」又奏:「學士提舉在京官,除本身請給外,更請貼職,並差遣添支,比六曹尚書、翰林學士承旨幾及一倍以上,非稱事制祿之意。」詔並依元豐法,御史中丞二十千,察案御史十千,籍田令七千;並依元豐三年詔,司農寺丞十五千,主簿京朝官十二千,選人十千。熙寧三年,詔廣親、睦親宅記室、講書十五千,教授十二千,軍巡使十七千,權使及判官七千。已上並元豐制,已下惟增散官而已。



 群牧使、副使,開封推官,三司河渠勾當公事、同管勾河渠案公事,十五千。群牧都監,十三千。銀臺司,審官院,三班院,吏部銓,登聞檢院、鼓院,太常禮院主判官,糾祭在京刑獄,群牧判官,監察使,十千。判司農
 寺,七千。



 其知、判諸路州、軍、府,有六十千至七千,凡八等。有以官者:三師,三公,六十千。僕射,東宮三師,並曾任中書、樞密,特進,五十千。尚書並左、右丞,東宮三少,金紫光祿大夫至光祿大夫,學士,給事中,諫議,舍人,待制已上,並橫班使、副,三十千。



 橫班有二百千者。待制已上充益、梓、利、夔州路知州,給鐵錢二百千。橫班副使知夔州,一百五十千,知諸州、軍者,八十千。



 大卿監,諸司使、副至供奉官,中大夫至中散大夫,武功郎至秉義郎,合門祗候已上,十五千。



 十五千已上有從州、府地望給者。不系大卿,充益、梓、利、夔知州,給鐵錢一百五十千。諸司副使
 至供奉官、合門祗候已上知四州同。若知諸州、軍,八十千。惟諸司使一百千。



 朝官忠翊郎,侍禁,合門祗候,十千。朝官權知軍、州、府者同。若知四路諸州、府,給鐵錢八十千,知軍六十千。侍禁、合門祗候、知諸軍、州同。



 保義郎,殿直,合門祗候,八千。若知四路諸州、軍者,給鐵錢五十千。



 京官十千至七千,有二等。知四路州、府,給鐵錢六十千;知軍,五十千。



 試銜及州縣官,職官兼知春州,七千。有以州望者:河南、大名、荊南、永興、江寧,杭、揚,潭、並、代州,三十千。應天、真定、鳳翔、陜府,秦、青、洪州,二十千。河中,鄆、許、襄、孟、滑、鄭、滄、刑、澶、貝、相、華、晉、潞、廬、壽、宿、泗、楚、蘇、越、潤、常州,十五千。



 廣州知州,歲七
 百千,逐月均給。舊月給百千,大中祥符六年,令歲取五百千,餘充添給。益州給鐵錢三百千,梓州二百千,夔州百五十千,餘州約銅錢數而給之。



 有都總管、經略安撫等使者、河北四路,真定、瀛州、定州、大名。



 陜西逐路,永興、秦州、渭州、慶州、延州。



 河東路,太原。



 前任兩府,並五十千;諫議、舍人、待制、太中大夫已上,三十千。並特添二十千。知大名府帶河北路安撫使同。知並州帶學士即五十千,而無特給。



 三路管勾機宜文字,朝官十千,京官七千。石桂州充文南西路都鈐轄、經略安撫使,自諫議、舍人、待制及大卿監、太中大夫、中散大夫已上,三十千。



 朝臣充廣西路兵馬都鈐轄兼本路安撫管勾經略司公事,即二十。



 河北沿邊安撫副使、都監以橫行使充者,三十千。



 自橫行副使並諸司使、副至崇班、武功大夫、敦武郎以上充者,二十千。供奉官、秉義郎、合門祗候充都監,十五千。



 同管勾河東緣邊安撫司公事,以橫行副使至內殿崇班、敦武郎以上,二十千。



 通判,大藩有二十千至十五千者。餘州、軍,朝官有十千至七千者,京官七千。朝官通判益州,給鐵錢八十千,京官六十千。



 朝官通判益、梓、利、夔路州、軍、府,給鐵錢七十千,京官五十千。簽判,朝官十千,京官七千。朝官簽判益、梓州,給鐵錢七十千,京官五十千。



 三路轉運使,淮南、江浙、荊湖制置茶鹽等稅都大發運使,諫議、待制、大卿
 監以下,太中、中散以上,三十千。朝官充發運使、副,二十千。



 武功大夫至武翼郎、諸司使副充發運使副、都監同朝官;充判官,十千。



 三門、白波發運使,朝官二十千;朝官充判官,十千,京官七千。諸路轉運使、副,朝官宣德郎以下,二十千,任四路者,給鐵錢一百五十千。判官十千。



 任福建、廣南東西路,十五千。任益、梓、利、夔四路,給鐵錢八十千。



 諸路提點刑獄,勸農使、副,開封府界提點諸縣鎮公事,二十千。忠翊郎、侍禁、合門祗候以下任諸路提點刑獄、勸農使副並府界同提點,武郎、內殿崇班已上者,十五千。朝官並秉義郎、供奉官、合門祗候已上任四路提點刑獄,給鐵錢一百五十千。忠翊郎、侍禁、合門祗候以下,一百千。



 諸路副都
 總管,權總管,都鈐轄,路分鈐轄,州鈐轄,路分都監,有五十千至八千,凡六等。任四路,給鐵錢有二百千至一百千,凡三等。



 府界及諸路州、府、軍、監、縣、鎮都監、巡檢、砦主、監押,自諸司使以下至三班借職,武功大夫至承信郎已上,十五千至五千,凡六等。任四路,給鐵錢有一百千至五十千,凡四等。陜西、河東沿邊諸族蕃官巡檢,自十五千至四千,凡六等。諸路走馬承受公事,自從義郎至保義郎。供奉官至殿直,並兩省自供奉官至黃門,自十千至五千,凡四等。



 任四路,給鐵錢自六十千至四十千,



 凡三等。
 府界並諸路州、府、軍、監、縣、鎮監當,朝官七千,京官五千至四千,凡二等。武功大夫以下至進義校尉,諸司使以下至三班使臣,自十千至三千,同七等。朝官任川峽州、府、軍、監,給鐵錢五十千,京官三十千至二十五千,凡二等。三班使臣任四路者,自六十千至二十五千,凡五等。



 朝官充陜西及江、浙、荊湖、福建、廣南提舉、提點鑄錢等公事,自二十千至十五千,凡二等。朝官充都大提舉河渠司,勾當及提舉宮觀,並催遣輦運、催納,諸州監物務等,自十五千至七千,凡三等。



 任四路,給鐵錢七十千。



 京官充催促輦運、催
 裝觔斗綱船,並諸州監物務等,自七千至五千,凡二等。任四路,給鐵錢五十千。



 都大提舉修護黃河堤埽岸,諸處巡檢,並監北京大內軍器庫,並蔡河撥發催綱等。並以兩省供奉官以下至內品充,自十千至三千,凡七等。



 舊志有諸路都部署、鈐轄,有五十千至十五千,凡四等。駐泊都監,兵馬都監,有二十千至十五千,凡六等。諸州監場務,朝官供奉以上七千,京官殿直五千,奉職內品三千,內課頤大者,京朝官與京官同,使臣與兵馬監押同。



 大中祥符二年,詔外任官不得挈家屬赴任者,許分添給錢贍本家。
 添給羊,凡外任給羊有二十口至二口,凡六等。給米,有二十石至二石,凡七等。給面,有三十石至二石,凡七等。傔從,有二十人至二人,凡七等。馬,有十匹至一匹,凡六等。舊志數不同。今從《四朝志》。



 建炎南渡以後,奉祿之制,參用嘉祐、元豐、政和之舊,少所增損。惟兵興之始,宰執請受權支三分之一,或支三分之二,或支賜一半,隆興及開禧自陳損半支給,皆權
 宜也。其後,內外官有添支料錢,職事官有職錢、廚食錢,職纂修者有折食錢,在京厘務官有添支錢、添支米,選人、使臣職田不及者有茶湯錢,其餘祿粟、傔人,悉還疇昔。今合新舊制而參記之。



 元豐定制,以官寄祿。南渡重加修定、開府儀同三司,料錢一百貫。特進,九十貫。春、冬衣絹各二十五匹,小綾一十匹,春羅一匹,冬綿五十兩。



 金紫光祿大夫,銀青光祿大夫。料錢各六十貫,春、冬絹各二十匹,小綾七匹,春羅一匹,冬綿五十兩。



 宣奉大夫,正奉大夫,正議大夫,通奉大夫。料錢
 各五十貫,春、冬絹各十七匹,小綾五匹,春羅一區。冬綿五十兩。



 通議大夫,太中大夫,中大夫,中奉大夫,中散大夫。



 料錢各四十五貫,春、『冬絹各二直五匹,小綾三匹,春羅一匹,冬綿五十兩。朝議大夫,奉直大夫,朝請大夫,朝散大夫,朝奉大夫。以上料錢各三十五貫,春、冬絹各一十五匹,春羅一匹,冬綿三十兩。朝請郎,朝散郎,朝奉郎。



 以上料錢各三十貫,春、冬絹各一十三匹春羅一匹,綿三十兩議郎。料錢二十貫,春、冬絹各一十匹,冬綿三十兩。



 奉議郎。料錢二十貫,春、冬絹各十匹,春羅一匹,冬綿三十兩通直郎。料錢十八貫,春、冬絹各七匹,春羅一匹,冬綿三十兩



 宣教郎。料錢十五貫,春、冬絹五匹,冬綿十五兩。



 宣議郎。



 料錢十二貫,春、冬絹各五匹,冬綿十五兩。



 承事郎。料錢十貫,春、冬絹各
 五匹,冬綿十五兩。



 承奉郎。料錢八貫。



 承務郎。料錢七貫,元豐以來,厘務止支驛料,大觀二年定支。



 以上料錢,一分見錢,二分折支。每貫折錢,在京六百文,在外四百文。到任添給驛料。



 承直郎。



 料錢二十五貫,茶湯錢一十貫,廚料米六斗,面一石五斗,蒿四十束,柴二十束,馬一匹,春、冬絹六匹,綿一十二兩。儒林郎。料錢二十貫,茶湯錢一十貫,廚料米六斗,面一石五斗,蒿三十束,柴一十五束,春、冬絹各五匹,冬綿十兩,文林郎料錢一十五貫,茶湯錢十貫,廚料米六斗,面一石五斗,蒿三十束,柴一十五束,春、冬絹各五匹,綿十兩。從事郎,從政郎,修職郎。已上料錢各一十五貫,茶湯錢一十貫,米、麥各二石。迪功郎。料錢一十二貫,茶湯錢一十貫,米麥各一石五斗。以上錢折支中給一半見錢,一
 半折支。每貫折見錢七百文。厘務日給,滿替日住。



 武臣請奉:太尉。料錢一百貫,春服羅一匹,小綾及絹各十匹,冬服小綾十匹,絹二十匹,綿五十兩。



 正任節度使。



 在光祿大夫之下,初授及帶管軍同,料錢四百貫,祿粟一百五十石。承宣使。在通議大夫之下,料錢三百貫,祿粟一百石。觀察使。在中大夫之下,料錢各二百貫,祿粟一百石,米麥十五石。



 防禦使。在中散大夫之下,料錢二百貫,祿粟一百石,米麥各十二石五斗。



 團練使。



 在中散大夫之下,料錢一百五十貫,祿粟七十石,米麥各九石。諸州刺史。在中散大夫之下,料錢一百貫,祿粟五十石,米、麥各七石五斗。自承宣使以下,不帶階官者為正任,帶階官者為遙郡,遙郡各在正任之下,請奉與次任、正任
 一同。靖康指揮:遙郡以上奉錢、衣賜、傔人、奉馬,權支三分之二。



 殿前三衙四廂、捧日、天武左右廂都指揮使遙郡團練使。料錢一百貫文。春、冬服絹各十匹。



 殿前諸班直都虞候,諸軍都指揮使遙郡刺史。料錢五十貫,衣同前。



 龍衛、神衛右廂都指揮使遙郡團練使。同捧日、天武。



 龍、神衛諸軍都指揮使遙郡刺史。同殿前。



 左、右金吾衛上將軍,左、右衛上將軍,在光祿大夫之下。



 諸衛上將軍。在通奉大夫之下。以上料錢各六十貫,



 春、冬綾各五匹,絹各一十匹,春羅一匹,冬綿五十兩。左、右金吾衛大將軍。



 在中散大夫之下,料錢三十五貫,春、冬綾三匹,絹七匹,春
 羅一匹,綿三十兩。



 諸衛大將軍。在中散大夫之下,料錢二十五貫,春、冬綾三匹,絹各七匹,春羅一匹,冬綿二十兩。



 諸衛將軍。在朝奉郎之下,料錢二十五貫,春、冬綾各二匹,絹各七匹,春羅一匹,冬綿十五兩。



 率府率,在奉議郎之下。



 率府副率。在通直郎之下。料錢十三貫,春、冬絹各五匹,春羅一匹,冬綿一十五兩。



 通侍大夫。在中散大夫之下。料錢五十貫,祿粟二十五石。春絹七匹,冬絹十匹,綿三十兩。



 傔二十人,馬三匹。



 正侍大夫,宣正大夫,履正大夫,協忠大夫,中侍大夫。



 以上在中散大夫之下。料錢各三十七貫,祿粟二十五石,春絹七匹,冬絹十匹,綿三十兩,傔二十人,馬三匹。



 中亮大夫。在中散大夫之下。料錢三十七貫,祿粟二十五石,春絹七匹,冬絹十匹,綿三十兩,傔二十人,馬三匹。



 中衛大夫,翊衛大夫,親衛大夫,
 在中散大夫之下,防禦使之上。



 拱衛大夫,左武大夫,右武大夫。並在奉直大夫之下,諸司正使之上。以上料錢並二十七貫,春絹七匹,冬絹十匹,綿三十兩。



 武功大夫,武德大夫,武顯大夫,武節大夫,武略大夫,武經大夫,武義大夫,武翼大夫。



 並在朝奉大夫之下。以上各料錢二十五貫,廚料米一石、面二石,春絹七匹,冬絹十匹,綿三十兩。



 正侍郎,宣正郎,履正郎,協忠郎,中侍郎,中亮郎,中衛郎,翊衛郎,親衛郎,拱衛郎,左武郎,右武郎。



 以上並在朝奉郎之下。錢各二十貫,春絹五匹,冬絹七匹,綿三十兩。



 武功郎,武德郎,武顯郎,武節郎,武略郎,武翼郎,武義郎。並在承議郎之下。以上各料錢二十貫,廚料米、面各一
 石,春絹五匹,冬絹七匹,綿三十兩。



 訓武郎。料錢一十七貫,春絹五匹,冬絹七匹,綿二十兩。



 修武郎。料錢一十七貫,春絹五匹,冬絹七匹,綿二十兩。



 從義郎,秉義郎。並料錢十貫,帶職錢十二貫,春絹四匹,冬絹五匹,綿一十兩。



 忠訓郎,忠翊郎。並料錢七貫,帶職錢十貫,春、冬絹各四匹,冬綿十五兩。



 成忠郎,保義郎。並料錢五貫,帶職錢七貫,春、冬絹各四匹,綿一十五兩。



 承節郎,承信郎。並料錢四貫,春、冬絹各三匹,錢二貫文。



 進武校尉。料錢三貫,春、冬絹各三匹。



 進義校尉。料錢二貫,春、冬絹各三匹。



 下班祗應。各隨差使理年不等。自三年至十二月,料錢七百文,糧二石五斗,春、冬絹各五匹。



 進武副尉。料錢三貫。進義副尉。料錢一貫。



 守闕進義副尉。料錢二貫。



 料錢、職錢,紹興仍政和之舊:宰相,樞密使,料錢月三百貫。



 政和左輔、右弼為宰相,紹興左右僕射同中書門下平章事為宰相。舊制,春、冬服小綾各二十匹,絹各三十匹,春羅一匹,冬綿一百兩。初,建炎元年指揮,宰執請受並權支三分之二,支賜支一半。



 知樞密院事,參知政事,樞密副使,同知樞密院事,簽書樞密院事。料錢二百貫,春、冬服小綾各十匹,絹各二十匹,春羅一匹,冬綿五十兩。



 太師,太傅,太保,少師,少傅,少保。



 料錢三百貫,春服羅三匹,權支一匹;小綾三十匹,支二十匹;絹四十匹,支三十匹,冬服綾、絹同。綿二百兩,支一百兩。



 以下職事官並支職錢:開封牧,錢一百貫。春服羅一匹,小綾、絹各十匹,冬服小綾十匹,絹二十匹。綿五十兩。



 太子太師,太傅,太
 保,職錢二百貫。春服羅一匹,小綾十匹,絹二十五匹,冬服綾、絹同,綿五十兩。



 少師,少傅,少保,百五十貫。春、冬服小綾各七匹,絹各二十匹,春羅一匹,冬綿五十兩。



 御史大夫,六部尚書。行,六十貫;守,五十五貫;試,五十貫。春服羅一匹,小綾五匹,絹十七匹,冬服綾、絹同,綿五十兩。



 翰林學士承旨,翰林學士,五十貫。春服同上。



 左、右散騎常侍。行,五十五貫;守,五十貫;試,四十五貫。春服小綾三匹,絹十五匹,羅一匹,冬綾、絹同,綿五十兩。



 權六曹尚書,御史中丞,六曹侍郎並同常侍,太子賓客。



 行,五十貫;守,四十七貫;試,四十五貫。春服小綾七匹,絹二十匹,羅一匹,冬綾、絹同,綿三十兩。



 太子詹事。錢、衣同賓客,小綾各止三匹。



 給事中,中書舍人。行,五十貫;守,四十五貫;試,四十貫。服同詹
 事。



 左、右諫議大夫。行,四十五貫;守,四十貫;試,三十七貫。餘同舍人。



 權六曹侍郎。職錢四十貫,絹同上。



 太常、宗正卿。行,三十八貫;守,三十五貫;試,三十二貫。春、冬衣隨官序。



 秘書監。行,四十二貫;守,三十八貫;試,三十五貫。



 七寺卿,國子祭酒。行,三十五貫;守,三十二貫;試,三十貫。



 太常、宗正少卿,秘和少監。行,三十二貫;守,三十貫;試,二十八貫。



 中書門下省檢正諸房公事,左、右司郎中。行,四十貫;守,三十七貫;試,三十五貫。



 國子司業,少府、將作、軍器監。行,三十二貫;守,三十貫;試,二十八貫。



 太子少詹事。行,三十五貫;守,三十二貫;試,三十貫。



 太子左、右諭德。行,三十三貫;守,三十貫;試,二十九貫;



 起居郎,起居舍人,侍御史。行,三十七貫;守,三十五貫;試,三十二貫。



 左、
 右司員外郎,六曹郎中。同上。



 殿中侍御史,左、右司諫。行,三十五貫;守,三十二貫;試,三十貫。



 左、右正言。行,三十二貫;守,三十貫;試,二十七貫。諸司員外郎。同司諫。



 少府、將作、軍器少監。行,三十貫;守,二十八貫;試,二十五貫。



 太子侍讀、侍講。行,二十五貫;守,二十二貫;試,二十貫。



 監察御史。同正言。



 太子中舍人,太子舍人。行,二十貫;守,十九貫;試,十八貫。



 太常丞,太醫令,宗正丞,知大宗正丞,秘書丞,大理正,著作郎。



 行,二十五貫;守,二十二貫;試,二十貫。紹興元年指揮,宣教郎任館職,寺監丞、簿、評事,臺法、主簿,寺簿、正、司直,添給職錢一十六貫,指揮每月特支米三石。



 七寺丞。行,二十二貫;守,二十貫。



 秘書郎。行,二十二貫;守,二十貫;試,一十八貫。



 太常博士。同七寺丞。



 著作佐郎。同秘書郎。



 國子監丞。同七寺丞。



 大理司直、評事。同著作郎。



 少府、將作、都水監丞。行,二十貫;守,十八貫。



 秘書省校書郎;行,十八貫;守,十六貫;試,十四貫。



 正官。行,十六貫;守,十五貫;試,十四貫。



 御史臺檢法、主簿,九寺簿,行,二十貫;守,十八貫。



 諸王宮大小學教授,太學、武學博士。行,二十貫;守,十八貫;試,十六貫。今諸王府翊善、贊讀、直講、記室料錢,並支見錢。



 律學博士。行,十八貫;守,十七貫;試,十六貫。



 太常寺奉禮郎。十六貫。



 太常寺太祝、郊社令。行,十八貫、守,十六貫。



 太官令。十六貫。五監主簿。行,十八貫;守,十六貫。



 太學正、錄,武學諭。行,十八貫;守,十七貫;試,十六貫。



 律學正。行,十六貫;守,十五貫;試,十四貫。



 樞密院官屬:都承旨,承旨。料錢四十貫,職錢三十貫,承旨二十五貫。春服羅一匹,小綾三匹,絹十五匹,冬服小綾五匹,絹十五匹,綿五十兩。



 副都承旨。料錢三十貫,職錢二十貫,副承旨、諸房副承旨十五貫,若諸房副承旨同主管承旨司公事,加五貫。春衣羅一匹,絹十五匹,冬絹同,綿三十兩。



 檢詳諸房文字。職錢三十五貫,廚食錢每日五百。



 計議、編修官。添支錢十貫,第三等折食錢二十五貫,廚食錢每日五百。



 凡諸職事官職錢不言「行」、「守」、「試」者,準「行」給。職事官衣,如寄祿官例,及無立定則例者,隨寄祿官給。職料錢、米麥計實數給,兩應給者,從多給。



 謂職錢、米麥。



 諸承直以下充職事官,謂大理司直、評事。秘書省正字,太學博士、正、錄,武學博士、
 諭,律學博士、正。



 聽支階官請受、添給。諸稱請受者,謂衣糧、料錢,餘並為添給。



 舊制,觀文殿大學士,三十貫。米三石,面五石。



 觀文殿學士,資政、保和殿大學士,二十貫。米三石,面五石。



 資政、保和殿學士,十五貫。米三石,面五石,同上。春、冬小綾各五匹,絹各十七匹,春羅一匹,冬綿五十兩。



 龍圖、天章、寶文、顯謨、徽猷、敷文閣學士、直學士,十五貫。春、冬小綾各三匹,絹各十五匹,春羅一匹,冬綿五十兩。



 保和殿,龍圖、天章、寶文、顯謨、徽猷、敷文閣待制同。



 先是,大觀,或言添支厚薄不均,其後,自學士而下改名貼職錢:觀文殿大學士;貼職錢一百貫文,米
 麥各二十五石,添支米三石,面五石,萬字茶二斤。



 觀文殿學士,資政、保和殿大學士;貼職錢八十貫,米麥同,添支錢十貫,添支米面同。



 資政、保和殿學士;貼職錢七十貫,米麥同,添支米面同,萬字茶二斤,春、冬綾五匹,絹一十七匹,綿五十兩。羅一匹,



 端明殿學士;貼職錢五十貫,米麥二十石,添支米三石,面五石,萬字茶二斤,春、冬綾五匹,絹一十七匹。羅一匹,冬綿五十兩。



 龍圖、天章、寶文、顯謨、徽猷、敷文閣學士,樞密直學士;正三品,貼職錢四十貫,米麥各一十石,添支米二石,面五石,萬字茶二斤,春、冬綾五匹,絹一十七匹,春羅一匹,冬綿五十兩。



 龍圖、天章、寶文、徽奠、敷文閣直學士,保和殿待制;貼職錢三十貫,米麥各一十七石五斗,春、冬綾各三匹,絹一十五匹,春羅一匹,冬綿五十兩。龍圖、天
 章、寶文、顯謨、徽猷、敷文閣待制;貼職錢二十貫,米麥各一十二石五斗,春、冬綾各三匹,絹一十五匹,春羅一匹,冬綿五十兩。



 集英殿修撰,右文殿修撰,秘閣修撰;以上貼職錢各一十五貫。



 直龍圖、天章、寶文閣,直顯謨、徽猷、敷文閣,直秘閣。以上貼職錢各一十貫。



 宣和間,罷支貼職錢,仍舊制添支。紹興因之,令諸觀文殿大學士至保和殿大學士料錢、春冬服隨本官;資政殿學士至待制料錢隨本官,春、冬服從一多給。又諸學士添支錢,曾任執政官以上者,在京、外任並支;其餘在京支,外任不支。米、面、茶、炭、奉
 馬、傔人衣糧,內外任並給。酒、添支、馬草料,外任勿給。外依祖例添支,如六部尚書而下職事官,分等第支廚食錢,自十五貫至九貫,凡四等,並依宣和指揮。修書官折食錢,監修國史四十千,史館修撰、直史館、本省長貳三十七貫五百,檢討、著作三十五貫,並依自來體例。



 有特旨添支。如紹興元年指揮:館職,寺監丞、簿、評事,臺法、主簿,寺正、司直,博士,添職錢一十貫。六年指揮:五寺、三監、秘書、大宗正丞,太常博士,著作、秘書、校書郎,著作佐郎,正字,大理寺正、司直、評事,臺簿,刪定官,檢、鼓、奏告院,特支米三石計議、編修官二石。



 祿粟及隨身、兼人:宰相,一百石,紹興:三公,侍中,中書、尚書令,左、右僕射同平章事,並為宰相。隨身七十
 人。知樞密院事,參知政事,樞密副使,同知樞密院事,一百石,隨身五十人。太師、太傅,太保,少師,少傅,少保,一百石,舊制百五十石。隨身一百人。太尉,一百石,隨身五十人。節度使,祿粟已具奉祿類。隨五十人,承宣使,元隨五十人。觀察使,防禦使,元隨三十人。團練使,已上並具奉祿類。元隨三十人。諸州刺史,同上。元隨二十人,通侍大夫,正侍大夫,宣正大夫,履正、協忠、中侍、中亮大夫,祿粟、傔人並具奉祿類。捧日、天武左右廂都旨揮使遙郡團練使,五十石,傔十人。



 龍、神衛右廂都指揮使帶遙
 郡團練使同。殿前諸班直都虞候,諸軍都指揮使遙郡刺史,二十五石,傔五人。龍、神衛諸軍都指揮使帶遙郡刺史同。



 諸學士添支米已附於前,今載:觀文殿大學士,傔二十人。觀文殿學士,資政、保和殿大學士,傔十人。資政、保和殿學士,龍圖、天章、寶文、顯謨、徽猷、敷文閣學士,傔七人。樞密都承旨,傔十人;副都承旨、諸房副承旨,七人。其餘京畿守令、幕職曹官,自十石、七石、五石至於二石各有等。中書堂後官提點五房公事,逐房副承旨,自七人、五人至於一人
 各有數。因仍前制,舊史已書。凡任宰相、執政有隨身,太尉至刺史有元隨,餘止傔人。



 紹興折色:凡祿粟每石細色六斗米麥中支。管軍給米六分,麥四公。隨身、元隨、傔人糧,每斗折錢三十文,衣紬絹每匹一貫。布每匹三百五十文,綿每兩四十文。



 公用錢



 自節度使兼使相,有給二萬貫者。其次萬貫至七千貫,凡四等。節度使,萬貫至三千貫,凡四等。節度觀察留後,
 五千貫至二千貫,凡四等。觀察使,三千貫至二千五百貫,凡二等。防禦使,三千貫至千五百貫,凡四等。團練使,二千貫至千貫,凡三等。刺史,千五百貫至五百貫,凡三等。亦有不給者。



 觀察使以下在禁軍校者,皆不給。



 京守在邊要或加錢給者,罷者如故,皆隨月給受,如祿奉焉。咸平五年,令河北、河東、陜西諸州,皆逐季給。



 京師月給者:玉清昭應宮使,百千。景靈宮使,崇文院,七十千。會靈觀使,六十千。祥源觀都大管勾,五十千。御史臺,三百千。大理寺,二百五十三千。刑部,九十六千。舍
 人院,二十千。太常寺,二十五千,秘閣,二十千。宗正寺,十五千。太常禮院,起居院,十千。門下省,登聞檢院、鼓院,官誥院,三班院,各五十千。



 歲給者:尚書都省,銀臺司,審刑院,提舉諸司庫務司,每給三十千,用盡續給,不限年月;餘文武常參官內職知州者,歲給王千至百千,凡十三等,皆長吏與通判署籍連署以給用;少卿監以上,有增十千至百千者。淳化元年九月,詔諸州、軍、監、縣無公使處,遇誕降節給茶宴錢,節度州百千,防、團、刺史州五十千,監、三泉縣三十千,嶺南州、軍以幕府州縣官權知州十千。



 給券



 文武君臣奉使於外,藩郡入朝,皆往來備饔餼,又有賓幕、軍將、隨身、牙官,馬驢、橐駝之差:節、察俱有賓幕以下;中書、樞密、三司使有隨身而無牙官、軍將隨;諸司使以上有軍將、橐駝。餘皆有牙官、馬□盧,惟節、察有賓幕。諸州及四夷貢奉使,諸司職掌祗事者,亦有給焉。四夷有譯語、通事、書狀、換醫、十券頭、首領、部署、子弟之名,貢奉使有廳頭、子將、推船、防授之名,職掌有傔。



 京朝官、三班外任無添給者,止續給之。京府按事畿內,幕職、州縣出境比較錢穀,覆
 按刑獄,並給券。其赴任川峽者,給驛券,赴福建、廣南者,所過給倉券,入本路給驛券,皆至任則止。車駕巡幸,群臣扈從者,中書、樞密、三司使給館券,餘官給倉券。



 職田



 周自卿以下有圭田不稅,晉有芻田,後魏宰人之官有公田,北齊一品以下公田有差,唐制內外官各給職田,五代以來遂廢。咸平中,令館閣檢校故事。申定其制,以官莊及遠年逃亡田充。悉免租稅,佃戶以浮客充,所
 得課租均分,如鄉原例。州縣長吏給十之五。自餘差給。其兩京、大藩府四十頃,次藩鎮三十五頃,防禦、團練州三十頃,中、上刺史州二十頃,下州及軍、監十五頃,邊遠小州、上縣十頃,中縣八頃,下縣七頃,轉運使、副十頃,兵馬都監押、砦主、厘務官、錄事參軍、判司等,比通判、幕職之數而均給之。



 景德二年七月,詔諸州職田如有災傷,準例蠲課。大中祥符九年,殿中侍御史王奇上言,請天下納職田以助振貨。帝曰:「奇未曉給納之理。然朕每覽
 法寺奏款,外官占田多窬往制,不能自備牛種,水旱之際又不蠲省,致民無告。」遂罷奇奏,因下詔戒飭之。



 天聖中,上患職田有無不均,吏或多取以病民;詔罷天下職田,悉以歲入租課送官,具數上三司,計直而均給之。朝廷方議措置未下,仁宗閱具獄,見吏以賄敗者多,惻然傷之;詔復給職田,毋多占佃戶,及無田而配出所租,違者以枉法論。



 又十餘年,至慶歷中,詔限職田,有司始申定其數。凡大藩長吏二十頃,通判八頃,判官五頃,幕職
 官四頃。凡節鎮長吏十五頃,通判七頃,判官四頃,幕職官三頃五十畝。凡防、團以下州軍長吏十頃,通判六頃,判官三頃五十畝,幕職官三頃。其餘軍、監長吏七頃,判官、幕官,並同防、團以下州軍。凡縣令,萬戶以上六頃,五千戶以上五頃,不滿五千戶並四頃。凡簿、尉,萬戶以上三頃,五千戶以上二頃五十畝,不滿五千戶二頃。錄事參軍比本判官。曹官比倚郭簿、尉。發運制置、轉運使副,武臣總管,比節鎮長吏。發運制置判官,比大藩府通判。
 安撫都監,路分都監,比節鎮通判。大藩府都監,比本府判官。黃汴河、許汝石塘河都大催綱,比節鎮判官。節鎮以下至軍監,諸路走馬承受並砦主,都同巡檢,提舉捉賊,提點馬監,都大巡河,不得過節鎮判官。在州監當及催綱、撥發,巡捉私茶鹽賊盜,駐泊捉賊,不得過簿、尉。自此人有定制,士有定限,吏以職田抵罪者,視昔為庶幾焉。



 至熙寧間,復詔詳定:



 凡知大藩府三京、京兆、成都、太原、荊南、江寧府,延、秦、揚、杭、潭、廣州。二十頃,節鎮十五頃,餘州及軍淮陽、無為、臨江、廣德、興國、
 南康、南安、建昌、邵武、興化。並十頃,餘小軍、監七頃。通判、藩府八頃,節鎮七頃,餘州六頃。留守、節度、觀察判官,藩府五頃,節鎮四頃。掌書記以下幕職官三頃五十畝。防禦、團練軍事推官,軍、監判官三頃。令、丞、簿、尉。萬戶以上,縣令六頃,丞四頃;不滿萬戶,令五頃,丞三頃;不滿五千戶,令四頃,丞二頃五十畝。簿、尉減令之半。藩府、節鎮錄參,視本州判官,餘視幕職官。藩府、節鎮曹官,視萬戶縣簿、尉,餘視不滿萬戶者。



 發運、轉運使、副,視節鎮知州。開封府界提點,視
 餘州。發運、轉運判官,常平倉司提舉官,視藩府通判。同提舉,視萬戶縣令。發運司乾當公事,視節鎮通判。轉運司管幹文字,提刑司檢法官,提舉常平倉司乾當公事,視不滿萬戶縣令。蔡河、許汝石塘河都大催綱,管幹機宜文字,府界提點司乾當公事,視節鎮判官。



 總管,視節鎮知州。路分鈐轄,視餘州知州。安撫、路分都監,州鈐轄,視節鎮通判。藩府都監,視本州判官。諸路正將,視路分都監;副將,視藩府都監。走馬承受,諸州都監,都同巡,都
 大巡河,並視節鎮判官。巡檢,堡砦都監,砦主,在州監當及催綱、撥發,巡捉私茶鹽賊盜,駐泊捉賊,並視幕職官。巡轄馬遞鋪,監堰並縣、鎮、砦監當,並視本縣簿、尉。諸路州學教授,京朝視本州判官,選人視本州曹官。



 又詔:「成都府路提點刑獄司,以本路職田令逐州軍歲以子利稻麥等拘收變錢,從本司以一路所收錢數,又紐而為觔斗價直,然後等第均給。」自熙寧三年始,知成都府,一千石。轉運使,六百石。鈐轄二員,各五百石。轉運判官,視
 鈐轄。通判二員,各四百五十石。簽判,節推,察推,知錄,乾當糧料院,監軍資庫,都監,都巡檢,巡檢,系大使臣。走馬承受,京朝官知縣,各二百石。內職官系兩使支掌以上資序者同。如系初等及權入者,各一百五十石。監商稅、市賣院、交子務,系京朝官或大使臣充者。視職官。城外巡檢,監排岸,十縣巡檢,系三班使臣者。各一百五十石。司理,司戶,司法,府學教授,系敕扎正授者。



 監甲仗庫,各一百石。知眉、蜀、彭、雅、邛、嘉、簡、陵州,永康軍,視成都通判。其通判減三之一。知威、黎茂州,視眉、蜀通判。其都監,監押,駐泊,都巡
 檢,系大使臣者。



 簽判,推、判官,系兩使職官並支掌以上資序。



 知錄,京朝官並職官知縣,監棚口鎮,系京朝官。視成都職官。監押,巡檢,同巡檢,駐泊,系三班使臣。初等職官或權入職官,錄事參軍,縣令,試銜知縣,視成都城外巡檢。司理,司戶,司法,諸縣主簿、尉,應監當場務選人監稅、監鹽,巡轄馬鋪,系三班使臣。



 視成都曹官。應諸縣令佐系職員權攝者不給。歲有豐兇,則數有少剩,皆隨時等級為之增減。初,權御史中丞呂誨、御史知雜劉述奉詔同均定成都、梓、利、夔四路職田,誨等以成都路歲收子利稻麥、桑絲、麻竹等物逐處不同,遂計實直紐作稻穀
 一色,每斗中價百有二十,自知成都府以下官屬等第均定。及再詔詳定,而三路數少,均分不足,用定到成都路數目以聞,中書再行詳定,而有是詔。



 元豐中,詔熙河、涇原、蘭州路州軍官屬職田,每頃歲給錢鈔十千。以其元給田及新造之區,募弓箭手及留其地以為營田,元符三年,朝散郎杜子民奏:「職田之法,每患不均。神宗首變兩川之法,無給上下,一路便之。元祐中推廣此意,以限月之法,變而均給。士大夫貪冒者,或窮日之力以赴期會,或交書請屬以幸權攝,奔競之風長,廉恥之節喪。乞復元豐均給之法,以養士廉節。」從之。



 建中靖國
 元年,知延安府范純粹奏:「昨帥河東日,聞晉州守臣所得職田,因李君卿為州,諭意屬邑增廣租入,比舊數倍。後襄陵縣令周汲力陳共弊,郡守時彥歲減所入十七八,佃戶始脫苛斂之苦。而晉、絳、陜三州圭腴,素號優厚,多由違法所致。或改易種色,或遣子弟公皂監獲,貪污猥賤,無所不有。乞下河東、陜西監司,悉令改正。」從之。



 大觀四年,臣僚言:「圭田欲以養廉,無法制以防之,則貪者奮矣。奸吏挾肥瘠之議,以逞其私,給田有限,課入無算,
 祖宗深慮其弊,以提點弄獄官察之,而未嘗給以圭租,庶不同其利而分其心也。近歲提點刑獄所受圭租,同於他司,故積年利病,壅於上聞。元豐舊制,檢法官,其屬也,當視其長。自元祐初並提舉常平司職事入提刑司,兼領編敕,遂將提舉官合給之數撥與提刑司,參詳修立,而檢法官亦預焉。」詔依舊法。



 政和八年,臣僚言:「尚書省以縣令之選輕,措置自不滿五千戶至滿萬戶遞增給職田一頃。夫天下圭租,多寡不均久矣,縣令所得,亦
 復不齊。多至九百斛,如淄州高苑;八百斛,如常之江陰;六百斛,常之宜興。亦六百斛。自是而降,或四五百,或三二百。凡在河北、京東京西、荊湖之間,少則有至三二十斛者;二廣、福建有自來無圭租處;川峽四路自守倅至簿、尉,又以一路歲入均給,令固不得而獨有。今欲一概增給一頃,豈可得哉?」詔應縣令職田頃畝未及條格者,催促摽撥。



 宣和無年詔:「諸路職官各有職田,所以養廉也。縣召客戶、稅戶,租佃分收,災傷檢覆減放,所以防貪
 也。諸縣多窬法抑都保正長,及中上戶分佃認納。不問所收厚薄,使之必輸,甚至不知田畝所在,虛認租課。聞之惻然。應違法抑勒及詭名委保者,以違詔論;災傷檢放不盡者,計贓以枉法論;入己者以自盜論。」



 靖康元年,詔諸路職田租存田亡者,並與落租額。紹興間,懼其不均,則詔諸路提刑司依法摽撥,官多田少,即於鄰近州縣通融,須管數足。又詔將空閑之田為他司官屬所占者,撥以足之,仍先自簿、尉始。其有無職田,選人並親民
 小使臣,每員月支茶湯錢一十貫文。內雖有職田,每月不及十貫者,皆與補足,所以厚其養廉之利。懼其病民,則委通判、縣令核實,除其不可力耕之田,損其已定過多之額。凡職租不許輒令保正催納,或抑令折納見錢,或無田平白監租,或以虛數勒民代納,或額外過數多取,皆申嚴禁止之令。察以監司,坐以贓罪,所以防其不廉之害。罷廢未幾而復舊,拘借未久而給還,移充糴本,轉收馬料,旋復免行,皆所以示優恩,厲清操也。



 若其頃
 畝多寡,具有成式:知藩府,謂三京、穎昌、京兆、成都、太原、建康、江陵、延安、興仁隆德、開德、臨安府,秦、揚潭、廣州。



 二十頃。發運、圍運使副,總管,副總管,知節鎮,一十五頃。知餘州及廣濟、淮陽、無為、臨江、廣德、興國、南康、南安、建昌、邵武、興化、漢陽、永康軍,並路分鈐轄,一十頃。發運、轉運判官,提舉淮南、兩浙、江南、荊湖東西、河北路鹽事官,通判藩府,八頃。知餘軍及監,並通判節鎮州,鈐轄,安撫副使,都監,路分都監,將官,發運司干辦公事,七頃。通判餘州及軍,滿萬戶縣令,六頃。藩府判官,錄
 事參軍,州學教授,並謂承務郎以上者。



 都監,發運、轉運司主管文字,滿五千戶縣令,副將官,五頃。節鎮判官,錄事參軍,州學教授,並謂承務郎以上者。



 轉運司主管帳司,不滿五千戶縣令,滿萬戶縣丞,餘州都監,走馬承受公事,主管機宜文字,同巡檢,都大巡河,提點馬監,四頃。節度掌書記,觀察支使,藩府及節鎮推官,巡檢,縣、鎮、砦都監、砦主,巡捉私茶鹽,駐泊捉賊,在城監當,餘州判官、學教授,並謂承務郎以上者。



 軍、監都監,三頃五十畝。



 軍、監判官,餘州推官,餘州及
 軍、監錄事參軍,巡檢,縣、鎮、砦都監,砦主,巡捉私茶鹽,駐泊捉賊,在城監當,藩府及節鎮曹官,州學教授,謂承直郎以下。



 滿五千戶縣丞,滿萬戶縣簿、尉,巡轄馬遞鋪,縣、鎮、砦監當及監堰,三頃。餘州及軍、監曹官,州學教授,謂承直郎以下。



 不滿五千戶縣丞,滿五千戶縣簿、尉,巡轄馬遞鋪,縣、鎮、砦監當及監堰,二頃五十畝。不滿五千戶縣簿、尉,巡轄馬遞鋪,縣、鎮、砦監當及監堰,二頃。



\end{pinyinscope}