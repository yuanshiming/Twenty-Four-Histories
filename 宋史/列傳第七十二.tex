\article{列傳第七十二}

\begin{pinyinscope}

 富弼子紹庭文彥博



 富弼,字彥國,河南人。初,母韓有娠,夢旌旗鶴雁降其庭,云有天赦,已而生弼。少篤學,有大度,範仲淹見而奇之,曰:「王佐才也。」以其文示王曾、晏殊,殊妻以女。



 仁宗復制
 科,仲淹謂弼:「子當以是進。」舉茂材異等,授將作監丞、簽書河陽判官。仲淹坐爭廢後事貶,弼上言:「是一舉而二失也,縱未能復後,宜還仲淹。」不聽。通判絳州,遷直集賢院。趙元昊反,弼疏陳八事,乞斬其使者。召為開封府推官、知諫院。康定元年,日食正旦,弼請罷宴徹樂,就館賜北使酒食。執政不可,弼曰:「萬一契丹行之,為朝廷羞。」後聞契丹果罷宴,帝深悔之。時禁臣僚越職言事,弼因論日食,極言應天變莫若通下情,遂除其禁。



 元昊寇鄜延,
 破金明,鈐轄盧守懃不救,內侍黃德和引兵走,大將劉平戰死,德和誣其降賊。弼請按竟其獄,德和坐要斬。夏守贇為陜西都部署,又以入內都知王守忠為鈐轄。弼言:「用守贇既為天下笑,今益以守忠,殆與唐監軍無異。守勤、德和覆車之轍,可復蹈乎!」詔罷守忠。又請令宰相兼領樞密院。時西夏首領二人來降,位補借奉職。弼言當厚賞以勸來者。事下中書,宰相初不知也。弼嘆曰:「此豈小事,而宰相不知邪!」更極論之,於是從弼言。除鹽鐵
 判官、史館修撰,奉使契丹。慶歷二年,為知制誥,糾察在京刑獄。堂吏有偽為僧牒者,開封不敢治。弼白執政,請以吏付獄,呂夷簡不悅。



 會契丹屯兵境上,遣其臣蕭英、劉六符來求關南地。朝廷擇報聘者,皆以其情叵測,莫敢行,夷簡因是薦弼。歐陽修引顏真卿使李希烈事,請留之,不報。弼即入對,叩頭曰:「主憂臣辱,臣不敢愛其死。」帝為動色,先以為接伴。英等入境,中使迎勞之,英托疾不拜。弼曰:「昔使北,病臥車中,聞命輒起。今中使至而君
 不拜,何也?」英矍然起拜。弼開懷與語,英感悅,亦不復隱其情,遂密以其主所欲得者告曰:「可從,從之;不然,以一事塞之足矣。」弼具以聞。帝唯許增歲幣,仍以宗室女嫁其子。



 進弼樞密直學士,辭曰:「國家有急,義不憚勞,奈何逆以官爵賂之。」遂為使報聘。既至,六符來館客。弼見契丹主問故,契丹主曰:「南朝違約,塞雁門,增塘水,治城隍,籍民兵,將以何為?群臣請舉兵而南,吾以謂不若遣使求地,求而不獲,舉兵未晚也。」弼曰:「北朝忘章聖皇帝之
 大德乎?澶淵之役,茍從諸將言,北兵無得脫者。且北朝與中國通好,則人主專其利,而臣下無獲;若用兵,則利歸臣下,而人主任其禍。故勸用兵者,皆為身謀耳。」契丹主驚曰:「何謂也?」弼曰:「晉高祖欺天叛君,末帝昏亂,土宇狹小,上下離叛,故契丹全師獨克,然壯士健馬物故太半。今中國提封萬里,精兵百萬,法令修明,上下一心,北朝欲用兵,能保其必勝乎?就使其勝,所亡士馬,群臣當之歟,抑人主當之歟?若通好不絕,歲幣盡歸人主,群臣
 何利焉?」契丹主大悟,首肯者久之。弼又曰:「塞雁門者,以備元昊也。塘水始於何承矩,事在通好前。城隍皆修舊,民兵亦補闕,非違約也。」契丹主曰:「微卿言,吾不知其詳。然所欲得者,祖宗故地耳。弼曰:「晉以盧龍賂契丹,周世宗復取關南,皆異代事。若各求地,豈北朝之利哉?」



 既退,六符曰:「吾主恥受金帛,堅欲十縣,何如?」弼曰:「本朝皇帝言,朕為祖宗守國,豈敢妄以土地與人。北朝所欲,不過租賦爾。朕不忍多殺兩朝赤子,故屈己增幣以代之。若
 必欲得地,是志在敗盟,假此為詞耳。澶淵之盟,天地鬼神實臨之。今北朝首發兵端,過不在我。天地鬼神,其可欺乎!」明日,契丹主召弼同獵,引弼馬自近,又言得地則歡好可久。弼反復陳必不可狀,且言:「北朝既以得地為榮,南朝必以失地為辱。兄弟之國,豈可使一榮一辱哉?」獵罷,六符曰:「吾主聞公榮辱之言,意甚感悟。今惟有結昏可議耳。」弼曰:「婚姻易生嫌隙。本朝長公主出降,繼送不過十萬緡,豈若歲幣無窮之利哉?」契丹主諭弼使歸,
 曰:「俟卿再至,當擇一受之,卿其遂以誓書來。」



 弼歸復命,復持二議及受口傳之詞於政府以往。行次樂壽,謂副使張茂實曰:「吾為使者而不見國書,脫書詞與口傳異,吾事敗矣。」啟視果不同,即馳還都,以晡時入見,易書而行。及至,契丹不復求婚,專欲增幣,曰:「南朝遺我之辭當曰『獻』,否則曰『納』。」弼爭之,契丹主曰:「南朝既懼我矣,於二字何有?若我擁兵而南,得無悔乎!」弼曰:「本朝兼愛南北,故不憚更成,何名為懼?或不得已至於用兵,則當以曲
 直為勝負,非使臣之所知也。」契丹主曰:「卿勿固執,古亦有之。」弼曰:「自古唯唐高祖借兵於突厥,當時贈遺,或稱獻納。其後頡利為太宗所擒,豈復有此禮哉!」弼聲色俱厲,契丹知不可奪,乃曰:「吾當自遣人議之。」復使劉六符來。弼歸奏曰:「臣以死拒之,彼氣折矣,可勿許也。」朝廷竟以「納」字與之。始受命,聞一女卒;再命,聞一子生,皆不顧。又除樞密直學士,遷翰林學士,皆懇辭,曰:「增歲幣非臣本志,特以方討元昊,未暇與角,故不敢以死爭,其敢受
 乎!」



 三年,拜樞密副使,辭之愈力,改授資政殿學士兼侍讀學士。七月,復拜樞密副使。弼言:「契丹既結好,議者便謂無事,萬一敗盟,臣死且有罪。願陛下思其輕侮之恥,坐薪嘗膽,不忘修政。」以誥納上前而罷。逾月,復申前命,使宰相諭之曰:「此朝廷特用,非以使遼故也。」弼乃受。帝銳以太平責成宰輔,數下詔督弼與範仲淹等,又開天章閣,給筆札,使書其所欲為者;且命仲淹主西事,弼主北事。弼上當世之務十餘條及安邊十三策,大略以進
 賢退不肖、止僥幸、去宿弊為本,欲漸易監司之不才者,使澄汰所部吏,於是小人始不悅矣。



 元昊遣使以書來,稱男不稱臣。弼言:「契丹臣元昊而我不臣,則契丹為無敵於天下,不可許。」乃卻其使,卒臣之。四年,契丹受禮云中,且發兵會元昊伐呆兒族,於河東為近,帝疑二邊同謀。弼曰:「兵出無名,契丹不為也。元昊本與契丹約相左右,今契丹獨獲重幣,元昊有怨言,故城威塞以備之。呆兒屢寇威塞,契丹疑元昊使之,故為是役,安能合而寇
 我哉?」或請調發為備,弼曰:「如此正墮其計,臣請任之。」帝乃止,契丹卒不動。夏竦不得志,中弼以飛語。弼懼,求宣撫河北,還,以資政殿學士出知鄆州。歲餘,讒不驗,加給事中,移青州,兼京東路安撫使。



 河朔大水,民流就食。弼勸所部民出粟,益以官廩,得公私廬舍十餘萬區,散處其人,以便薪水。官吏自前資、待缺、寄居者,皆賦以祿,使即民所聚,選老弱病瘠者廩之,仍書其勞,約他日為奏請受賞。率五日,輒遣人持酒肉飯糗慰藉,出於至誠,人
 人為盡力。山林陂澤之利可資以生者,聽流民擅取。死者為大塚葬之,目曰「叢塚」。明年,麥大熟,民各以遠近受糧歸,凡活五十餘萬人,募為兵者萬計。帝聞之,遣使褒勞,拜禮部侍郎。弼曰:「此守臣職也。」辭不受。前此,救災者皆聚民城郭中,為粥食之,蒸為疾疫,及相蹈藉,或待哺數日不得粥而僕,名為救之,而實殺之。自弼立法簡便周盡,天下傳以為式。



 王則叛,齊州禁兵欲應之,或詣弼告。齊非弼所部,恐事洩變生,適中貴人張從訓銜命至
 青,弼度其可用,密付以事,使馳至齊,發吏卒取之,無得脫者。即自劾顓擅之罪,帝益嘉之,復以為禮部侍郎,又辭不受。遷大學士,徙知鄭、蔡、河陽,加觀文殿學士,改宣徽南院使、判並州。至和二年,召拜同中書門下平章事、集賢殿大學士,與文彥博並命。宣制之日,士大夫相慶於朝。帝微覘知之,以語學士歐陽修曰:「古之命相,或得諸夢卜,豈若今日人情如此哉?」修頓首賀。帝弗豫,大臣不得見,中外憂慄。弼、彥博入問疾,因托禳禬事止宿連
 夕,每事皆關白乃行,宮內肅然,語在《彥博傳》。嘉祐三年,進昭文館大學士、監修國史。



 弼為相,守典故,行故事,而傅以公議,無容心於其間。當是時,百官任職,天下無事。六年三月,以母憂去位,詔為罷春宴。故事,執政遭喪皆起復。帝虛位五起之,弼謂此金革變禮,不可施於平世,卒不從命。英宗立,召為樞密使。居二年,以足疾求解,拜鎮海軍節度使、同中書門下平章事、判揚州,封祁國公,進封鄭。



 熙寧元年,徙判汝州。詔入覲,許肩輿至殿門。神
 宗御內東門小殿,令其子掖以進,且命毋拜,坐語,從容訪以治道。弼知帝果於有為,對曰:「人主好惡,不可令人窺測;可測,則奸人得以傅會。當如天之監人,善惡皆所自取,然後誅賞隨之,則功罪無不得其實矣。」又問邊事,對曰:「陛下臨御未久,當布德行惠,願二十年口不言兵。」帝默然。至日昃乃退。欲以集禧觀使留之,力辭赴郡。明年二月,召拜司空兼侍中,賜甲第,悉辭之,以左僕射、門下侍郎同平章事。



 時有為帝言災異皆天數,非關人事
 得失所致者。弼聞而嘆曰:「人君所畏惟天,若不畏天,何事不可為者!此必奸人欲進邪說,以搖上心,使輔拂諫爭之臣,無所施其力。是治亂之機,不可以不速救。」即上書數千言,力論之。又言:「君子小人之進退,系王道之消長,願深加辨察,勿以同異為喜怒、喜怒為用舍。陛下好使人伺察外事,故奸險得志。又多出親批,若事事皆中,亦非為君之道;脫十中七八,積日累月,所失亦多。今中外之務漸有更張,大抵小人惟喜生事,願深燭其然,無
 使有悔。」是時久旱,群臣請上尊號及用樂,帝不許,而以同天節契丹使當上壽,故未斷其請。弼言此盛德事,正當以此示之,乞並罷上壽。帝從之,即日雨。弼又上疏,願益畏天戒,遠奸佞,近忠良。帝手詔褒答之。



 王安石用事,雅不與弼合。弼度不能爭,多稱疾求退,章數十上。神宗將許之,問曰:「卿即去,誰可代卿者?」弼薦文彥博,神宗默然,良久曰:「王安石何如?」弼亦默然。拜武寧節度使、同中書門下平章事、判河南,改亳州。青苗法出,弼以謂如是
 則財聚於上,人散於下,持不行。提舉官趙濟劾弼格詔旨,侍御史鄧綰又乞付有司鞫治,乃以僕射判汝州。安石曰:「弼雖責,猶不失富貴。昔鯀以方命殛,共工以象恭流,弼兼此二罪,止奪使相,何由沮奸?」帝不答。弼言:「新法,臣所不曉,不可以治郡。願歸洛養疾。」許之。遂請老,加拜司空,進封韓國公致仕。弼雖家居,朝廷有大利害,知無不言。郭逵討安南,乞詔逵擇利進退,以全王師;契丹爭河東地界,言其不可許;星文有變,乞開廣言路;又請速
 改新法,以解倒縣之急。帝雖不盡用,而眷禮不衰,嘗因安石有所建明,卻之曰:「富弼手疏稱『老臣無所告訴,但仰屋竊嘆』者,即當至矣。」其敬之如此。



 元豐三年,王堯臣之子同老上言:「故父參知政事時,當仁宗服藥,嘗與弼及文彥博議立儲嗣,會翌日有瘳,其事遂寢。」帝以問彥博,對與同老合,帝始知至和時事。嘉弼不自言,以為司徒。六年八月,薨,年八十。手封遺奏,使其子紹庭上之。其大略云:



 陛下即位之初,邪臣納說圖任之際,聽受失宜,
 上誤聰明,浸成禍患。今上自輔臣,下及多士,畏禍圖利,習成敝風,忠詞讜論,無復上達。臣老病將死,尚何顧求?特以不忍上負聖明,輒傾肝膽,冀哀憐愚忠,曲垂採納。



 去年永樂之役,兵民死亡者數十萬。今久戍未解,百姓困窮,豈諱過恥敗不思救禍之時乎?天地至仁,寧與羌夷校曲直勝負?願歸其侵地,休兵息民,使關、陜之間,稍遂生理。兼陜西再團保甲,又葺教場,州縣奉行,勢侔星火,人情惶駭,難以復用,不若寢罷以綏懷之。臣之所陳,
 急於濟事。若夫要道,則在聖人所存,與所用之人君子、小人之辨耳。陛下審觀天下之勢,豈以為無足慮邪?



 帝覽奏震悼,輟朝三日,內出祭文致奠,贈太尉,謚曰文忠。



 弼性至孝,恭儉好修,與人言必盡敬,雖微官及布衣謁見,皆與之亢禮,氣色穆然,不見喜慍。其好善嫉惡,出於天資。常言:「君子與小人並處,其勢必不勝。君子不勝,則奉身而退,樂道無悶。小人不勝,則交結構扇,千岐萬轍,必勝而後已。迨其得志,遂肆毒於善良,求天下不亂,不
 可得也。」其終身皆出於此云。元祐初,配享神宗廟庭。哲宗篆其碑首曰:「顯忠尚德」,命學士蘇軾撰文刻之。紹聖中,章惇執政,謂弼得罪先帝,罷配享。至靖康初,詔復舊典焉。



 紹庭字德先,性靖重,能守家法。弼薨,兩女與婿及甥皆同居,紹庭待之與父時不殊,一家之事毫發不敢變,族里稱焉。歷宗正丞、提舉三門白波輦運、通判絳州。建中靖國初,除提舉河北西路常平,辭曰:「熙寧變法之初,先
 臣以不行青苗被罪,臣不敢為此官。」徽宗嘉之,擢祠部員外郎。未幾,出知宿州。卒,年六十八。子直柔,紹興中,同知樞密院事,別有傳。



 文彥博,字寬夫,汾州介休人。其先本敬氏,以避晉高祖及宋翼祖諱改焉。少與張忭、高若訥從穎昌史照學,照母異之,曰:「貴人也。」待之甚厚。及進士第,知翼城縣,通判絳州,為監察御史,轉殿中侍御史。



 西方用兵,偏校有監陳先退、望敵不進者,大將守著令皆申覆。彥博言:「此可
 施之平居無事時爾。今擁兵數十萬,而將權不專,兵法不峻,將何以濟?」仁宗嘉納之。黃德和之誣劉平降虜也,以金帶賂平奴,使附己說以證。平家二百口皆械系。詔彥博置獄於河中,鞫治得實。德和黨援盛,謀翻其獄,至遣他御史來。彥博拒不納,曰:「朝廷慮獄不就,故遣君。今案具矣,宜亟還,事或弗成,彥博執其咎。」德和並奴卒就誅。以直史館為河東轉運副使。麟州餉道回遠,銀城河外有唐時故道,廢弗治,彥博父洎為轉運使日,將復之,
 未及而卒。彥博嗣成父志,益儲粟。元昊來寇,圍城十日,知有備,解去。遷天章閣待制、都轉運使,連進龍圖閣、樞密直學士、知秦州,改益州。嘗擊球鈐轄廨,聞外喧甚,乃卒長杖一卒,不伏。呼入問狀,令引出與杖,又不受,復呼入斬之,竟球乃歸。召拜樞密副使、參知政事。



 貝州王則反,明鎬討之,久不克。彥博請行,命為宣撫使,旬日賊潰,檻則送京師。拜同中書門下平章事、集賢殿大學士。薦張環、韓維、王安石等恬退守道,乞褒勸以厲風俗。與樞
 密使龐籍議省兵,凡汰為民及給半廩者合八萬,論者紛然,謂必聚為盜,帝亦疑焉。彥博曰:「今公私困竭,正坐兵冗。脫有難,臣請死之。」其策訖行,歸兵亦無事。進昭文館大學士。御史唐介劾其在蜀日以奇錦結宮掖,因之登用。介既貶,彥博亦罷為觀文殿大學士、知許州,改忠武軍節度使、知永興軍。至和二年,復以吏部尚書同中書門下平章事、昭文館大學士,與富弼同拜,士大夫皆以得人為慶,語見《弼傳》。



 三年正月,帝方受朝,疾暴作,扶
 入禁中。彥博呼內侍史志聰問狀,對曰:「禁密不敢漏言。」彥博叱之曰:「爾曹出入禁闥,不令宰相如天子起居,欲何為邪?自今疾勢增損必以告,不爾,當行軍法。」又與同列劉沆、富弼謀啟醮大慶殿,因留宿殿廬。志聰曰:「無故事。」彥博曰:「此豈論故事時邪?」知開封府王素夜叩宮門上變,不使入;明旦言,有禁卒告都虞候欲為亂。沆欲捕治,彥博召都指揮使許懷德,問都虞候何如人,懷德稱其願可保。彥博曰:「然則卒有怨,誣之耳。當亟誅之以靖
 眾。」乃請沆判狀尾,斬於軍門。



 先是,弼用朝士李仲昌策,自澶州商胡河穿六漯渠,入橫□故道。北京留守賈昌朝素惡弼,陰約內侍武繼隆,令司天官二人俟執政聚時,於殿庭抗言國家不當穿河於北方,致上體不安。彥博知其意有所在,然未有以制之,後數日,二人又上言,請皇后同聽政,亦繼隆所教也。史志聰以其狀白執政。彥博視而懷之,不以示同列,而有喜色,徐召二人詰之曰:「汝今日有所言乎?」曰:「然。」彥博曰:「天文變異,汝職所當
 言也。何得輒預國家大事?汝罪當族!」二人懼,色變。彥博曰:「觀汝直狂愚耳,未忍治汝罪,自今無得復然。」二人退,乃出狀示同列。同列皆憤怒曰:「奴敢爾僭言,何不斬之?」彥博曰:「斬之,則事彰灼,於中宮不安。」眾皆曰:「善。」既而議遣司天官定六漯方位,復使二人往。繼隆白請留之,彥博曰:「彼本不敢妄言,有教之者耳。」繼隆默不敢對。二人至六漯,恐治前罪,更言六漯在東北,非正北也。帝疾愈,彥博等始歸第。當是時,京師業業,賴彥博、弼持重,眾心
 以安。沆密白帝曰:「陛下違豫時,彥博擅斬告反者。」彥博聞之,以沆判呈,帝意乃解。御史吳中復乞召還唐介。彥博因言,介頃為御史,言臣事多中臣病,其間雖有風聞之誤,然當時責之太深,請如中復奏。時以彥博為厚德。久之,以河陽三城節度使同平章事、判河南府,封潞國公,改鎮保平、判大名府。又改鎮成德,遷尚書左僕射、判太原府。俄復鎮保平、判河南。丁母憂,英宗即位,起復成德軍節度使,三上表乞終喪,許之。



 初,仁宗之不豫也,彥
 博與富弼等乞立儲嗣。仁宗許焉,而後宮將有就館者,故其事緩。已而彥博去位,其後弼亦以憂去。彥博既服闋,復以故官判河南,有詔入覲。英宗曰:「朕之立,卿之力也。」彥博竦然對曰:「陛下入繼大統,乃先帝聖意,皇太后協贊之力,臣何聞力之有?兼陛下登儲纂極之時,臣方在外,皆韓琦等承聖志受顧命,臣無與焉。」帝曰:「備聞始議,卿於朕有恩。」彥博遜避不敢當。帝曰:「暫煩西行,即召還矣。」尋除侍中,徙鎮淮南、判永興軍,入為樞密使、劍南西
 川節度使。



 熙寧二年,相陳升之,詔:「彥博朝廷宗臣,其令升之位彥博下,以稱遇賢之意。」彥博曰:「國朝樞密使,無位宰相上者,獨曹利用嘗在王曾、張知白上。臣忝知禮義,不敢效利用所為,以紊朝著。」固辭乃止。夏人犯大順,慶帥李復圭以陳圖方略授鈐轄李信等,趣使出戰。及敗,乃妄奏信罪。彥博暴其非,宰相王安石曲誅信等,秦人冤之。慶州兵亂,彥博言於帝曰:「朝廷行事,務合人心,宜兼採眾論,以靜重為先。陛下厲精求治,而人心未安,
 蓋更張之過也。祖宗法未必皆不可行,但有偏而不舉之敝爾。」安石知為己發,奮然排之曰:「求去民害,何為不可?若萬事隳脞,乃西晉之風,何益於治?」御史張商英欲附安石,摭樞密使他事以搖彥博,坐不實貶。彥博在樞府九年,又以極論市易司監賣果實,損國體斂民怨,為安石所惡,力引去。拜司空、河東節度使、判河陽,徙大名府。身雖在外,而帝眷有加。



 時監司多新進少年,轉運判官汪輔之輒奏彥博不事事,帝批其奏以付彥博曰:「以
 侍中舊德,故煩臥護北門,細務不必勞心。輔之小臣,敢爾無禮,將別有處置。」未幾,罷去。初,選人有李公義者,請以鐵龍爪治河,宦者黃懷信沿其制為浚川杷,天下指笑以為兒戲,安石獨信之,遣都水丞範子淵行其法。子淵奏用杷之功,水悉歸故道,退出民田數萬頃。詔大名核實,彥博言:「河非杷可浚,雖甚愚之人,皆知無益,臣不敢雷同罔上。」疏至,帝不悅,復遣知制誥熊本等行視,如彥博言。子淵乃請覲,言本等見安石罷,意彥博復相,故
 傅會其說。御史蔡確亦論本奉使無狀。本等皆得罪,獨彥博勿問。尋加司徒。



 元豐三年,拜太尉,復判河南。於是王同老言至和中議儲嗣事,彥博適入朝,神宗問之,彥博以前對英宗者復於帝曰:「先帝天命所在,神器有歸,實仁祖知子之明,慈聖擁祐之力,臣等何功?」帝曰:「雖雲天命,亦系人謀。卿深厚不伐善,陰德如丙吉,真定策社稷臣也。」彥博曰:「如周勃、霍光,是為定策。自至和以來,中外之臣獻言甚眾,臣等雖嘗有請,弗果行。其後韓琦等
 訖就大事,蓋琦功也。」帝曰:「發端為難,是時仁祖意已定,嘉祐之末,止申前詔爾。正如丙吉、霍光,不相掩也。」遂加彥博兩鎮節度使,辭不拜。將行,賜宴瓊林苑,兩遣中謁者遺詩祖道,當世榮之。



 王中正經制邊事,所過稱受密旨募禁兵,將之而西。彥博以無詔拒之,中正亦不敢募而去。久之,請老,以太師致仕,居洛陽。元祐初,司馬光薦彥博宿德元老,宜起以自輔。宣仁後將用為三省長官,而言事者以為不可,及命平章軍國重事,六日一朝,一
 月兩赴經筵,恩禮甚渥。然彥博無歲不求退,居五年,復致仕。紹聖初,章惇秉政,言者論彥博朋附司馬光,詆毀先烈,降太子少保。卒,年九十二。崇寧中,預元祐黨籍。後特命出籍,追復太師,謚曰忠烈。



 彥博逮事四朝,任將相五十年,名聞四夷。元祐間,契丹使耶律永昌、劉霄來聘,蘇軾館客,與使入覲,望見彥博於殿門外,卻立改容曰:「此潞公也邪?」問其年,曰:「何壯也!?軾曰:「使者見其容,未聞其語。其綜理庶務,雖精練少年有不如;其貫穿古今,雖
 專門名家有不逮。」使者拱手曰:「天下異人也。」既歸洛,西羌首領溫溪心有名馬,請於邊吏,願以饋彥博,詔許之。其為外國所敬如此。



 彥博雖窮貴極富,而平居接物謙下,尊德樂善,如恐不及。其在洛也,洛人邵雍、程顥兄弟皆以道自重,賓接之如布衣交。與富弼、司馬光等十三人,用白居易九老會故事,置酒賦詩相樂,序齒不序官,為堂,繪像其中,謂之「洛陽耆英會」,好事者莫不慕之。神宗導洛通汴,而主者遏絕洛水,不使入城中,洛人頗患
 苦之。彥博因中使劉惟簡至洛,語其故,惟簡以聞。詔令通行如初,遂為洛城無窮之利。



 彥博八子,皆歷要官。第六子及甫,初以大理評事直史館,與邢恕相善。元祐初,為吏部員外郎,以直龍圖閣知同州。彥博平章軍國,及甫由右司員外郎引嫌改衛尉、光祿少卿。彥博再致仕,及甫知河陽,召為太僕卿,權工部侍郎,罷為集賢殿修撰、提舉明道宮。蔡渭、邢恕持及甫私書造梁燾、劉摯之謗,逮詣詔獄,及甫有憾於元祐,從而實之,亦坐奪職。未
 幾,復之,卒。



 論曰:國家當隆盛之時,其大臣必有耆艾之福,推其有餘,足芘當世。富弼再盟契丹,能使南北之民數十年不見兵革。仁人之言,其利博哉!文彥博立朝端重,顧盼有威,遠人來朝,仰望風採,其德望固足以折沖禦侮於千里之表矣。至於公忠直亮,臨事果斷,皆有大臣之風,又皆享高壽於承平之秋。至和以來,建是大計,功成退居,朝野倚重。熙、豐而降,弼、彥博相繼衰老,憸人無忌,善類
 淪胥,而宋業衰矣!《書》曰:「番番良士,膂力既愆,我尚有之。」豈不信然哉!



\end{pinyinscope}