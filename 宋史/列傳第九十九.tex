\article{列傳第九十九}

\begin{pinyinscope}

 呂大防兄大忠弟大鈞大臨劉摯蘇頌



 呂大防,字微仲,其先汲郡人。祖通,太常博士。父賁,比部郎中。通葬京兆藍田,遂家焉。大防進士及第,調馮翊主簿、永壽令。縣無井,遠汲於澗,大防行近境,得二泉,欲導
 而入縣,地勢高下,眾疑無成理。大防用《考工》水地置泉之法以準之,不旬日,果疏為渠,民賴之,號曰「呂公泉」。



 遷著作佐郎、知青城縣。故時,圭田粟入以大斗而出以公鬥,獲利三倍,民雖病不敢訴。大防始均出納以平其直,事轉聞,詔立法禁,命一路悉輸租於官概給之。青城外控汶川,與敵相接。大防據要置邏,密為之防,禁山之樵採,以嚴障蔽。韓絳鎮蜀,稱其有王佐才。入權鹽鐵判官。



 英宗即位,改太常博士。御史闕,內出大防與范純仁姓
 名,命為監察御史裏行。首言:「紀綱賞罰,未厭四方之望者有五:進用大臣而權不歸上;大臣疲老而不得時退;外國驕蹇而不擇將帥;議論之臣裨益闕失,而大臣沮之;疆場左右之臣,有敗事而被賞、舉職而獲罪者。」又言:「富弼病足請解機務,章十餘上而不納;張忭年幾八十,聰明已耗,哀乞骸骨而不從;吳奎有三年之喪,以其子召之者再,遣使召之者又再;程戡辭老不能守邊,恐死塞上,免以尸柩還家為請,亦不許。陛下欲盡君臣之分,
 使病者得休,喪者得終,老者得盡其餘年,則進退盡禮,亦何必過為虛飾,使四人之誠,不得自達邪?」



 是歲,京師大水,大防曰:「雨水之患,至入宮城廬舍,殺人害物,此陰陽之沴也。」即陳八事,曰:主威不立,臣權太盛,邪議干正,私恩害公,遼、夏連謀,盜賊恣行,群情失職,刑罰失平。會執政議濮王稱考,大防上言:「先帝起陛下為皇子,館於宮中,憑幾之命,緒言在耳,皇天后土,實知所托。設使先帝萬壽,陛下猶為皇子,則安懿之稱伯,於理不疑。豈可
 生以為子,沒而背之哉?夫人君臨御之始,宜有至公大義厭服天下,以結其心。今大臣首欲加王以非正之號,使陛下顧私恩而違公義,非所以結天下之心也。」章累十數上,出知休寧縣。



 神宗立,通判淄州。熙寧元年,知泗州,為河北轉運副使。召直舍人院。韓絳宣撫陜西,命為判官,又兼河東宣撫判官,除知制誥。四年,知廷州。大防、昉欲城河外荒堆砦,眾謂不可守,大防留戍兵修堡障,有不從者斬以徇。會環慶兵亂,絳坐黜,大防亦落知制
 誥,以太常博士知臨江軍。



 數月,徙知華州。華嶽摧,自山屬渭河,被害者眾。大防奏疏,援經質史,以驗時事。其略曰:「『畏天之威,於時保之。』先王所以興也;『我生不有命在天』,後王所以壞也。《書》云:『惟先格王,正厥事。』願仰承天威,俯酌時變,為社稷至計。」除龍圖閣待制、知秦州。元豐初,徙永興。神宗以彗星求言,大防陳三說九宜:曰治本,曰緩末,曰納言。養民、教士、重谷,治本之宜三也;治邊、治兵,緩末之宜二也;廣受言之路,寬侵官之罰,恕誹謗之罪,
 容異同之論,此納言之宜四也。累數千言。時用兵西夏,調度百出,有不便者輒上聞,務在寬民。及兵罷,民力比他路為饒,供億軍須亦無乏絕。進直學士。居數年,知成都府。



 哲宗即位,召為翰林學士、權開封府。有僧誑民取財,因訟至廷下。驗治得情,命抱具獄,即其所杖之,他挾奸者皆遁去。館伴契丹使,其使黠,語頗及朝廷,大防密擿其隱事,詰之曰:「北朝試進士《至心獨運賦》,不知此題於書何出?」使錯TD不能對,自是不敢復出嫚詞。



 遷吏部
 尚書。夏使來,詔訪以待遇之計,且曰:「向者所得邊地,雖建立城堡,終慮孤絕難保。棄之則弱國,守之又有後悔,為當奈何?」大防言:「夏本無能為,然屢遣使而不布誠款者,蓋料我急於議和耳。今使者到闕,宜令押伴臣僚,扣其不賀登極,以觀厥意,足以測情偽矣。新收疆土,議者多言可棄,此慮之不熟也。至於守禦之策,惟擇將帥為先。太祖用姚內斌、董遵誨守環、慶,西人不敢入侵。昔以二州之力,禦敵而有餘;今以九州之大,奉邊而不足。由
 是言之,在於得人而已。」元祐元年,拜尚書右丞,進中書侍郎,封汲郡公。西方息兵,青唐羌以為中國怯,使大將鬼章青宜結犯邊。大防命洮州諸將乘間致討,生擒之。



 三年,呂公著告老,宣仁後欲留之京師。手札密訪至於四五,超拜大防尚書左僕射兼門下侍郎,提舉修《神宗實錄》。大防見哲宗年益壯,日以進學為急,請敕講讀官取仁宗邇英御書解釋上之,置於坐右。又摭乾興以來四十一事足以為勸戒者,分上下篇,標曰《仁祖聖學》,使
 人主有欣慕不足之意。



 哲宗御邇英閣,召宰執、講讀官讀《寶訓》,至「漢武帝籍南山提封為上林苑,仁宗曰:『山澤之利當與眾共之,何用此也。』丁度曰:『臣事陛下二十年,每奉德音,未始不及於憂勤,此蓋祖宗家法爾。』」大防因推廣祖宗家法以進,曰:「自三代以後,唯本朝百二十年中外無事,蓋由祖宗所立家法最善,臣請舉其略。自古人主事母後,朝見有時,如漢武帝五日一朝長樂宮;祖宗以來事母後,皆朝夕見,此事親之法也。前代大長公
 主用臣妾之禮;本朝必先致恭,仁宗以□至事姑之禮見獻穆大長公主,此事長之法也。前代宮闈多不肅,宮人或與廷臣相見,唐入閣圖有昭容位;本朝宮禁嚴密,內外整肅,此治內之法也。前代外戚多預政事,常致敗亂;本朝母後之族皆不預,此待外戚之法也。前代宮室多尚華侈;本朝宮殿止用赤白,此尚儉之法也。前代人君雖在宮禁,出輿入輦;祖宗皆步自內庭,出御後殿,豈乏人力哉,亦欲涉歷廣庭,稍冒寒暑,此勤身之法也。前代
 人主,在禁中冠服茍簡;祖宗以來,燕居必以禮,竊聞陛下昨郊禮畢,具禮謝太皇太后,此尚禮之法也。前代多深於用刑,大者誅戮,小者遠竄;惟本朝用法最輕,臣下有罪,止于罷黜,此寬仁之法也。至於虛己納諫,不好畋獵,不尚玩好,不用玉器,不貴異味,此皆祖宗家法,所以致太平者。陛下不須遠法前代,但盡行家法,足以為天下。」哲宗甚然之。



 大防樸厚惷直,不植黨朋,與范純仁並位,同心戮力,以相王室。立朝挺挺,進退百官,不可干以
 私,不市恩嫁怨以邀聲譽,凡八年,始終如一。



 懇乞避位,宣仁後曰:「上方富於春秋,公未可即去,少須歲月,吾亦就東朝矣。」未果而後崩。為山陵使,復命以觀文殿大學士、左光祿大夫知穎昌府。尋改永興軍,使便其鄉社。入辭,哲宗勞慰甚渥,曰:「卿暫歸故鄉,行即召矣。」未幾,左正言上官均論其隳壞役法,右正言張商英、御史周秩、劉拯相繼攻之,奪學士,知隨州,貶秘書監,分司南京,居郢州。言者又以修《神宗實錄》直書其事為誣詆,徙安州。



 兄
 大忠自渭入對,哲宗詢大防安否,且曰:「執政欲遷諸嶺南,朕獨令處安陸,為朕寄聲問之。大防樸直,為人所賣,三二年可復相見也。」大忠洩其語於章惇,惇懼,繩之愈力。紹聖四年,遂貶舒州團練副使,安置循州。至虔州信豐而病,語其子景山曰:「吾不復南矣!吾死汝歸,呂氏尚有遺種。」遂薨,年七十一。大忠請歸葬,許之。



 大防身長七尺,眉目秀發,聲音如鐘。自少持重,無嗜好,過市不左右游目,燕居如對賓客。每朝會,威儀翼如,神宗常目送之。
 與大忠及弟大臨同居,相切磋論道考禮,冠昏喪祭,一本於古,關中言《禮》學者推呂氏。嘗為《鄉約》曰:「凡同約者,德業相勸,過失相規,禮俗相交,患難相恤,有善則書於籍,有過若違約者亦書之,三犯而行罰,不悛者絕之。」



 徽宗即位,復其官。高宗紹興初,又復大學士,贈太師、宣國公,謚曰正愍。



 大忠字進伯。登第,為華陰尉、晉城令。韓絳宣撫陜西,以大忠提舉永興路義勇。改秘書丞,檢詳樞密院吏、兵房
 文字。令條義勇利害。大忠言:「養兵猥眾,國用日屈,漢之屯田,唐之府兵,善法也。弓箭手近於屯田,義勇近於府兵,擇用一焉,兵屯可省矣。」為簽書定國軍判官。



 熙寧中,王安石議遣使諸道,立緣邊封溝,大忠與範育被命,俱辭行。大忠陳五不可,以為懷撫外國,恩信不洽,必致生患。罷不遣。令與劉忱使契丹,議代北地,會遭父喪。起復,知代州。契丹使蕭素、梁穎至代,設次,據主席,大忠與之爭,乃移次於長城北。換西上閣門使、知石州。大忠數與
 素、穎會,凡議,屢以理折之,素、穎稍屈。已而復使蕭禧來求代北地,神宗召執政與大忠、忱議,將從其請。大忠曰:「彼遣一使來,即與地五百里,若使魏王英弼來求關南,則何如?」神宗曰:「卿是何言也。」對曰:「陛下既以臣言為不然,恐不可啟其漸。」忱曰:「大忠之言,社稷大計,願陛下熟思之。」執政知不可奪,議卒不決,罷忱還三司,大忠亦終喪制。其後竟以分水嶺為界焉。



 元豐中,為河北轉運判官,言:「古者理財,視天下猶一家。朝廷者家,外計者兄弟,
 居雖異而財無不同。今有司惟知出納之名,有餘不足,未嘗以實告上。故有餘則取之,不足莫之與,甚大患也。」乃上生財、養民十二事。徙提點淮西刑獄。時河決,飛蝗為災,大忠入對,極論之,詔歸故官。



 元祐初,歷工部郎中、陜西轉運副使、知陜州,以直龍圖閣知秦州,進寶文閣待制。夏人自犯麟府、環慶後,遂絕歲賜,欲遣使謝罪,神宗將許之。大忠言:「夏人強則縱,困則服,今陽為恭順,實懼討伐。宜且命邊臣詰其所以來之辭,若惟請是從,彼
 將有以窺我矣。」



 時郡糴民粟,豪家因之制操縱之柄。大忠選僚寀自旦入倉,雖斗升亦受,不使有所壅閼。民喜,爭運粟於倉,負錢而去,得百餘萬斛。



 馬涓以進士舉首入幕府,自稱狀元。大忠謂曰:「狀元云者,及第未除官之稱也,既為判官則不可。今科舉之習既無用,修身為己之學,不可不勉。」又教以臨政治民之要,涓自以為得師焉。謝良佐教授州學,大忠每過之,聽講《論語》,必正襟斂容曰:「聖人言行在焉,吾不敢不肅。」



 嘗獻曰:「夏人戍守之
 外,戰士不過十萬,吾三路之眾,足以當之矣。彼屢犯王略,一不與校,臣竊羞之。」紹聖二年,加寶文閣直學士、知渭州,付以秦、渭之事,奏言:「關、陜民力未裕,士氣沮喪,非假之歲月,未易枝梧。」因請以職事對。大抵欲以計徐取橫山,自汝遮殘井迤邐進築,不求近功。



 既而鐘傅城安西,王文鬱亦用事,章惇、曾布主之,大忠議不合;又乞以所進職為大防量移,惇、布陳其所言與元祐時異,徙知同州,旋降待制致仕。卒,詔復學士官,佐其葬。



 大鈞字和叔。父蕡,六子,其五登科,大鈞第三子也。中乙科,調秦州右司理參軍,監延州折博務。改光祿寺丞、知三原縣。請代蕡入蜀,移巴西縣。蕡致仕,大鈞亦移疾不行。



 韓絳宣撫陜西、河東,闢書寫機密文字。府罷,移知候官縣,故相曾公亮鎮京兆,薦知涇陽縣,皆不赴。丁外艱,家居講道。數年,起為諸王宮教授。求監鳳翔船務,制改宣義郎。



 會伐西夏,鄜延轉運司檄為從事。既出塞,轉運使李稷饋餉不繼,欲還安定取糧,使大鈞請於種諤。
 諤曰:「吾受命將兵,安知糧道!萬一不繼,召稷來,與一劍耳。」大鈞性剛直,即曰:「朝廷出師,去塞未遠,遂斬轉運使,無君父乎?」諤意折,強謂大鈞曰:「君欲以此報稷,先稷受禍矣!」大鈞怒曰:「公將以此言見恐邪?吾委身事主,死無所辭,正恐公過耳。」諤見其直,乃好謂曰:「子乃爾邪?今聽汝矣!」始許稷還。是時,微大鈞盛氣誚諤,稷且不免。未幾,道得疾,卒,年五十二。



 大鈞從張載學,能守其師說而踐履之。居父喪,衰麻葬祭,一本於禮。後乃行於冠昏、膳飲、慶
 吊之間,節文粲然可觀,關中化之。尤喜講明井田兵制,謂治道必自此始,悉撰次為圖籍,可見於用。雖皆本於載,而能自信力行,載每嘆其勇為不可及。



 大臨字與叔。學於程頤,與謝良佐、游酢、楊時在程門,號「四先生」。通《六經》,尤邃於《禮》。每欲掇習三代遺文舊制,令可行,不為空言以拂世駭俗。



 其論選舉曰:「古之長育人才者,以士眾多為樂;今之主選舉者,以多為患。古以禮聘士,常恐士之不至;今以法待士,常恐士之競進。古今
 豈有異哉。蓋未之思爾。夫為國之要,不過得人以治其事,如為治必欲得人,惟恐人才之不足,而何患於多?如治事皆任其責,惟恐士之不至,不憂其競進也。今取人而用,不問其可任何事;任人以事,不問其才之所堪。故入流之路不勝其多,然為官擇士則常患乏才;待次之吏歷歲不調,然考其職事則常患不治。是所謂名實不稱,本末交戾。如此而欲得人而事治,未之有也。今欲立士規以養德厲行,更學制以量才進藝,定試法以區別
 能否,修闢法以興能備用,嚴舉法以核實得人,制考法以責任考功,庶幾可以漸復古矣。」



 富弼致政於家,為佛氏之學。大臨與之書曰:「古者三公無職事,惟有德者居之,內則論道於朝,外則主教於鄉。古之大人當是任者,必將以斯道覺斯民,成己以成物,豈以爵位進退、體力盛衰為之變哉?今大道未明,人趨異學,不入於莊,則入於釋。疑聖人為未盡善,輕禮義為不足學,人倫不明,萬物憔悴,此老成大人惻隱存心之時。以道自任,振起壞
 俗,在公之力,宜無難矣。若夫移精變氣,務求長年,此山谷避世之士獨善其身者之所好,豈世之所以望於公者哉?」弼謝之。



 元祐中,為太學博士,遷秘書省正字。範祖禹薦其好學修身如古人,可備勸學,未及用而卒。



 劉摯,字莘老,永靜東光人。兒時,父居正課以書,朝夕不少間。或謂:「君止一子,獨不可少寬邪?」居正曰:「正以一子,不可縱也。」十歲而孤,鞠於外氏,就學東平,因家焉。



 嘉祐中,擢甲科,歷冀州南宮令。縣比不得入,俗化凋敝,其賦
 甚重,輸絹匹折稅錢五百,綿兩折錢三十,民多破產。摯援例旁郡,條請裁以中價。轉運使怒,將劾之。摯固請曰:「獨一州六邑被此苦,決非法意,但朝廷不知耳。」遂告於朝。三司使包拯奏從其議,自是絹為錢千三百,綿七十有六。民歡呼至泣下,曰:「劉長官活我!」是時,摯與信都令李沖、清河令黃莘皆以治行聞,人稱為「河朔三令」。



 徙江陵觀察推官,用韓琦薦,得館閣校勘。王安石一見器異之,擢檢正中書禮房,默默非所好也。才月餘,為監察御
 史裏行,欣然就職,歸語家人曰:「趣裝,毋為安居計。」未及陛對,即奏論:「亳州獄起不止,小人意在傾富弼以市進,今弼已得罪,願少寬之。」又言:「程昉開漳河,調發猝迫,人不堪命。趙子幾擅升畿縣等,使納役錢,縣民日數千人遮訴宰相,京師喧然,何以示四方?張靚、王廷老擅增兩浙役錢,督賦嚴急,人情嗟怨。此皆欲以羨餘希賞,願行顯責,明朝廷本無聚斂之意。」



 及入見,神宗面賜褒諭。因問:「卿從學王安石邪?安石極稱卿器識。」對曰:「臣東北人,少
 孤獨學,不識安石也。」退而上疏曰:「君子小人之分,在義利而已。小人才非不足用,特心之所向,不在乎義。故希賞之志,每在事先;奉公之心,每在私後。陛下有勸農之意,今變而為煩擾;陛下有均役之意,今倚以為聚斂。其有愛君之心,憂國之言者,皆無以容於其間。今天下有喜於敢為,有樂於無事。彼以此為流俗,此以彼為亂常。畏義者以進取為可恥,嗜利者以守道為無能。此風浸成,漢、唐黨禍必起矣。惟君子為能通天下之志。臣願陛
 下虛心平聽,審察好惡,前日意以為是者,今更察其非;前日意以為短者,今更用其長。稍抑虛嘩輕偽、志近忘遠、幸於茍合之人,漸察忠厚慎重、難進易退、可與有為之士。收過與不及之俗,使會於大中之道,則施設變化,惟陛下號令之而已。」



 又論率錢助役、官自雇人有十害,其略曰:「天下州縣戶役,虛實重輕不同。今等以為率,則非一法所能齊;隨其所宜,各自立法,則紛擾散殊,何以統率?一也。新法謂版籍不實,故令別立等第。且舊籍既
 不可信,今何以得其無失?不獨搔擾生事患,將使富輸少,貧輸多,二也。天下上戶少,中戶多。上戶役數而重,故以助錢為幸。中戶役簡而輕,下戶役所不及。今概使輸錢,則為不幸,三也。有司欲多得雇錢,而患上戶之寡,故不用舊籍,臨時升降,使民何以堪命?四也。歲有豐兇,而役人有定數,助錢不可闕。非若稅賦有倚閣、減放之期,五也。穀、麥、布、帛,歲有所出,而助法必輸見錢,六也。二稅科買,色目已多,又概率錢以竭其所有,斯民無有悅而
 願為農者,戶口當日耗失,七也。僥幸者又將緣法生奸,如近日兩浙倍科錢數,自以為功,八也。差法近者十餘年,遠或二十年,乃一充役,民安習之久矣。今官自雇人,直重則民不堪,輕則人不願,不免以力毆之就役,九也。且役人必用鄉戶,家有常產,則必知自愛;性既愚實,則罕有盜欺。今一切雇募,但得輕猾浮偽之人,巧詐相資,何所不至?十也。」



 會御史中丞楊繪亦言其非,安石使張琥作十難以詰之,琥辭不為,司農曾布請為之。既作十
 難,且劾摯、繪欺誕懷向背。詔問狀,繪懼謝罪。摯奮曰:「為人臣豈可壓於權勢,使天子不知利害之實!」即條對所難,以伸其說。且曰:「臣待罪言責,採士民之說以聞於上,職也。今有司遽令分析,是使之較是非,爭勝負,交口相直,無乃辱陛下耳目之任哉!所謂向背,則臣所向者義,所背者利;所向者君父,所背者權臣。願以臣章並司農奏宣示百官,考定當否。如臣言有取,幸早施行,若稍涉欺罔,甘就竄逐。」不報。



 摯明日復上疏曰:「陛下起居言動,
 躬蹈德禮,夙夜厲精,以親庶政。天下未至於安且治者,誰致之耶?陛下注意以望太平,而自以太平為己任,得君專政者是也。二三年間,開闔動搖,舉天下無一物得安其所者。蓋自青苗之議起,而天下始有聚斂之疑;青苗之議未允,而均輸之法行;均輸之法方擾,而邊鄙之謀動;邊鄙之禍未艾,而助役之事興。至於求水利,行淤田,並州縣,興事起新,難以遍舉。其議財,則市井屠販之人,皆召至政事堂。其征利,則下至歷日,而官自鬻之。推
 此而往,不可究言。輕用名器,淆混賢否:忠厚老成者,擯之為無能;狹少儇辯者,取之為可用;守道憂國者,謂之流俗;敗常害民者,謂之通變。凡政府謀議經畫,除用進退,獨與一掾屬決之,然後落筆。同列預聞,反在其後。故奔走乞丐之人,其門如市。今西夏之款未入,反側之兵未安,三邊瘡痍,流潰未定。河北大旱,諸路大水,民勞財乏,縣官減耗。聖上憂勤念治之時,而政事如此,皆大臣誤陛下,而大臣所用者,誤大臣也。」疏奏,安石欲竄之嶺
 外,神宗不聽,但謫監衡州鹽倉。繪出知鄭州,琥亦落職。摯乞詣鄆遷葬,然後奔赴貶所,許之。



 先是,倉吏與綱兵奸利相市,鹽中雜以偽惡,遠人未嘗食善鹽。摯悉意核視,且儲其羨以為賞,弊減什七。父老目為「學士鹽」。久之,簽書南京判官。會司農新令,盡斥賣天下祠廟,依坊場河度法收凈利。南京閼伯廟歲錢四十六貫,微子廟十三貫。摯嘆曰:「一至於此!」往見留守張方平曰:「獨不能為朝廷言之耶?」方平瞿然,托摯為奏曰:「閼伯遷商丘,主祀
 大火,火為國家盛德所乘,歷世尊為大祀。微子,宋始封之君,開國此地,本朝受命,建號所因。又有雙廟者,唐張巡、許遠孤城死賊,能捍大患。今若令承買小人規利,冗褻瀆慢,何所不為,歲收微細,實損大體。欲望留此三廟,以慰邦人崇奉之意。」從之。又見《方平傳》。



 入同知太常禮院。元豐初,改集賢校理、知大宗正寺丞,為開封府推官。神宗開天章閣,議新官制,除至禮部郎中,曰:「此南宮舍人,非他曹比,無出劉摯者。」即命之。俄遷右司郎中。



 初,宰
 掾每於執政分廳時,請間白事,多持兩端伺意指。摯始請以公禮聚見,共決可否。或不便摯所請,坐以開封不置歷事罷歸。明年,起知滑州。哲宗即位,宣仁後同聽政,召為吏部郎中,改秘書少監,擢侍御史。上疏曰:「昔者周成王幼沖踐祚,師保之臣,周公、太公其人也。仁宗皇帝盛年嗣服,用李維、晏殊為侍讀,孫奭、馮元為侍講,聽斷之暇,召使入侍。陛下春秋鼎盛,在所資養。願選忠信孝悌、惇茂老成之人,以充勸講進讀之任,便殿燕坐,時賜
 延對,執經誦說,以廣睿智,仰副善繼求治之志。」



 他日講筵進讀,至仁宗不避庚戌臨奠張士遜,侍讀曰:「國朝故事,多避國音。國朝角音,木也,故畏庚辛。」哲宗問:「果當避否?」摯進曰:「陰陽拘忌,聖人不取,如正月祈穀必用上辛,此豈可改也?漢章帝以反支日受章奏,唐太宗以辰日哭張公謹,仁宗不避庚戌日,皆陛下所宜取法。」哲宗然之。



 摯又言:「諫官御史員缺未補,監察雖滿六員,專以察治官司公事,而不預言責。臣請增補臺諫,並許言事。」時
 蔡確、章惇在政地,與司馬光不相能。摯因久旱上言:「《洪範》:『庶徵肅,時雨若。』《五行傳》:『政緩則冬旱。』今廟堂大臣,情志乖暌,議政之際,依違排狠,語播於外,可謂不肅。政令二三,舒緩不振。比日日青無光,風霾昏曀,上天警告,皆非小變。願進忠良、通壅塞,以答天戒。」



 蔡確為山陵使,神宗靈駕發引前夕不入宿,摯劾之,不報。及使回,既朝即視事,摯又奏確不引咎自劾。無何,確上表自陳,嘗請收拔當世之耆艾,以陪輔王室,蠲省有司之煩碎,以慰安
 民心。摯謂:「使確誠有是請,不言於先朝,為不忠之罪;言於今日,為取容之計。誠無是請,則欺君莫大於此。」又疏確過惡大略有十,論章惇兇悍輕侻,無大臣體,皆罷去。



 初,神宗更新學制,養士以千數,有司立為約束,過於煩密。摯上疏曰:「學校為育材首善之地,教化所從出,非行法之所。雖群居眾聚,帥而齊之,不可無法,亦有禮義存焉。先帝體道制法,超漢軼唐,養士之盛,比隆三代。然而比以太學屢起獄訟,有司緣此造為法禁,煩苛愈於治
 獄,條目多於防盜,上下疑貳,以求茍免。甚可怪者,博士、諸生禁不相見,教諭無所施,質問無所從,月巡所隸之齋而已。齋舍既不一,隨經分隸,則又《易》博士兼巡《禮》齋,《詩》博士兼巡《書》齋,所至備禮請問,相與揖諾,亦或不交一言而退,以防私請,以杜賄賂。學校如此,豈先帝所以造士之意哉?治天下者,遇人以君子、長者之道,則下必有君子、長者之行而應乎上。若以小人、犬彘遇之,彼將以小人、犬彘自為,而況以此行於學校之間乎?願罷其
 制。」又請雜用經義、詩賦取士,復賢良方正科,罷常平、免役,引朱光庭、王巖叟為言官。執憲數月,正色彈劾,多所貶黜,百僚敬憚,時人以比包拯、呂晦。



 元祐元年,擢御史中丞。摯上疏曰:「上之所好,下必有甚。朝廷意在總核,下必有刻薄之行;朝廷務在寬大,下必有茍簡之事。習俗懷利,迎意趨和,所為近似,而非上之意本然也。今因革之政本殊,而觀望之俗故在。昨差役初行,監司已有迎合爭先,不校利害,一概定差,一路為之騷動者。朝廷察
 其如此,固已黜之矣。以是觀之,大約類此。向來黜責數人者,皆以非法掊克,市進害民,然非欲使之漫不省事。昧者不達,矯枉過正,顧可不為之禁哉?請立監司考績之制。」



 拜尚書右丞,連進左丞、中書侍郎,遷門下侍郎。胡宗愈除右丞,諫議大夫王覿疏其非是,宣仁后怒,將加深譴。摯開救甚力,簾中厲聲曰:「若有人以門下侍郎為奸邪,甘受之否?」摯曰:「陛下審察毀譽每如此,天下幸甚!然願顧大體,宗愈進用,自有公議,必致貶諫官而後進,
 恐宗愈亦所未安。」宣仁後意解,覿得補郡守。



 摯與同列奏事論人才,摯曰:「人才難得,能否不一。性忠實而才識有餘,上也;才識不逮而忠實有餘,次也;有才而難保,可藉以集事,又其次也。懷邪觀望,隨時勢改變,此小人也,終不可用。」哲宗及宣仁後曰:「卿常能如此用人,國家何憂!」六年,拜尚書右僕射。



 摯性峭直,有氣節,通達明銳,觸機輒發,不為利怵威誘。自初輔政至為相,修嚴憲法,辨白邪正,專以人物處心,孤立一意,不受謁請。子弟親戚
 入官,皆令赴銓部以格調選,未嘗以干朝廷。與呂大防同位,國家大事,多決於大防,惟進退士大夫,實執其柄。然持心少恕,勇於去惡,竟為朋讒奇中。先是,邢恕謫官永州,以書抵摯。摯故與恕善,答其書,有「永州佳處,第往以俟休復」之語。排岸官茹東濟,傾險人也,有求於摯,不得,見其書,陰錄以示御史中丞鄭雍、侍御史楊畏。二人方交章擊摯,遂箋釋其語上之,曰:「『休復』者,語出《周易》,『以俟休復』者,俟他日太皇太后復子明闢也。」又章惇諸子
 故與摯之子游,摯亦間與之接。雍、畏謂延見接納,為牢籠之計,以冀後福。宣仁後於是面喻摯曰:「言者謂卿交通匪人,為異日地,卿當一心王室。若章惇者,雖以宰相處之,未必樂也。」摯皇懼退,上章自辨,執政亦為之言。宣仁後曰:「垂簾之初,摯排斥奸邪,實為忠直。但此二事,非所當為也。」以觀文殿學士罷知鄆州。給事中朱光庭駁云:「摯忠義自奮,朝廷擢之大位,一旦以疑而罷,天下不見其過。」光庭亦罷。七年,徙大名,又為雍等所遏,徙知
 青州。



 紹聖初,來之邵、周秩論摯變法、棄地罪,奪職知黃州,再貶光祿卿,分司南京,蘄州居住。將行,語諸子曰:「上用章惇,吾且得罪。若惇顧國事,不遷怒百姓,但責吾曹,死無所恨。正慮意在報復,法令益峻,奈天下何!」憂形於色,無一言及遷謫意。四年,陷邢恕之謗,貶鼎州團練副使,新州安置。惟一子從。家人涕泣願侍,皆不聽。至數月,以疾卒,年六十八。



 初,摯與呂大防為相,文及甫居喪,在洛怨望,服除,恐不得京官,抵書邢恕曰:「改月遂除,入朝之
 計未可必。當塗猜怨於鷹揚者益深,其徒實繁。司馬昭之心,路人所知也,濟之以『粉昆』,必欲以眇躬為甘心快意之地,可為寒心。」其謂司馬昭者,指呂大防獨當國久;『粉昆』者,世以駙馬都尉為『粉侯』,韓嘉彥尚主,以兄忠彥為『粉昆』也。恕以書示蔡碩、蔡渭,渭上書訟摯及大防等十餘人陷其父確,謀危宗社,引及甫書為證。時章惇、蔡卞誣造元祐諸人事不已,因是欲殺摯及梁燾、王巖叟等。以為摯有廢立之意,遂起同文館獄,用蔡京、安惇雜
 治,逮問及甫。及甫元祐末德大防除權侍郎,又忠彥雖罷,哲宗眷之未衰,乃托其亡父嘗說司馬昭指劉摯,「粉」謂王巖叟面白如粉,「昆」謂梁燾字況之,「況」猶「兄」也。又問實狀,但云:「疑其事勢如此。」會摯卒,京奏不及考驗,遂免其子官,與家屬徙英州,凡三年,死於瘴者十人。



 徽宗立,詔反其家屬,用子跂請,得歸葬。跂又伏闕訴及甫之誣,遂貶及甫並渭於湖外,復摯中大夫。蔡京為相,降朝散大夫。後又復觀文殿大學士、太中大夫。紹興初,贈少
 師,謚曰忠肅。



 摯嗜書,自幼至老,未嘗釋卷。家藏書多自讎校,得善本或手抄錄,孜孜無倦。少好《禮》學,其究《三禮》,視諸經尤粹。晚好《春秋》,考諸儒異同,辨其得失,通聖人經意為多。其教子孫,先行實,後文藝。每曰:「士當以器識為先,一號為文人,無足觀矣。」



 跂能為文章,遭黨事,為官拓落,家居避禍,以壽終。



 蘇頌,字子容,泉州南安人。父紳,葬潤州丹陽,因徙居之。第進士,歷宿州觀察推官、知江寧縣。時建業承李氏後,
 稅賦圖籍,一皆無藝,每發斂,高下出吏手。頌因治訊他事,互問民鄰里丁產,識其詳。及定戶籍,民或自占不悉,頌警之曰:「汝有某丁某產,何不言?」民駭懼,皆不敢隱,遂鏟剔夙蠹,成賦一邑,簡而易行,諸令視以為法,至領某民拜庭下以謝。凡民有忿爭,頌喻以鄉黨宜相親善,若以小忿而失歡心,一旦緩急,將何賴焉。民往往謝去,或半途思其言而止。時監司王鼎、王綽、楊紘於部吏少許可,及觀頌施設,則曰:「非吾所及也。」



 調南京留守推官,留
 守歐陽修委以政,曰:「子容處事精審,一經閱覽,則修不復省矣。」時杜衍老居睢陽,見頌,深器之,曰:「如君,真所謂不可得而親疏者。」衍又自謂平生人罕見其用心處,遂自小官以至為侍從、宰相所以施設出處,悉以語頌,曰:「以子相知,且知子異日必為此官,老夫非以自矜也。」故頌後歷政,略似衍云。



 皇祐五年,召試館閣校勘,同知太常禮院。至和中,文彥博為相,請建家廟,事下太常。頌議以為:「禮,大夫士有田則祭,無田則薦,是有土者乃為廟
 祭也。有田則有爵,無土無爵,則子孫無以繼承宗祀,是有廟者止於其躬,子孫無爵,祭乃廢也。若參合古今之制,依約封爵之令,為之等差,錫以土田,然後廟制可議。若猶未也,即請考案唐賢寢堂祠饗儀,止用燕器常食而已。」



 嘉祐中,詔禮院議立故郭皇后神御殿於景靈宮,頌謂:「敕書云:『向因忿鬱,偶失謙恭』,此則無可廢之事。又云:『朕念其自歷長秋,僅周一紀,逮事先後,祗奉寢園』,此則有不當廢之悔。又云:『可追復皇后,其祔廟謚冊並停。』
 此則有合祔廟及謚冊之義。請祔郭皇后於後廟,以成追復之道。」眾論未定,宰相曾公亮問曰:「郭後,上元妃,若祔廟,則事體重矣。」頌曰:「國朝三聖,賀、尹、潘皆元妃,事體正相類。今止祔後廟,則豈得有同異之言。」公亮曰:「議者以謂陰逼母後,是恐萬歲後配祔之意。」頌曰:「若加一『懷』、『哀』、『愍』之謚,則不為逼矣。」公亮嘆重。



 遷集賢校理,編定書籍。頌在館下九年,奉祖母及母,養姑姊妹與外族數十人,甘旨融怡,昏嫁以時。妻子衣食常不及,而處之晏如。
 富弼嘗稱頌為古君子,及與韓琦為相,同表其廉退,以知穎州。通判趙至忠本邊徼降者,所至與守競,頌待之以禮,具盡誠意。至忠感泣曰:「身雖夷人,然見義則服,平生誠服者,唯公與韓魏公耳。」



 仁宗崩,建山陵,有司以不時難得之物厲諸郡。頌曰:「遺詔務從儉約,豈有土不產而可強賦乎?量其有無,事亦隨集。」英宗即位,召提點開封府界諸縣鎮公事。頌言:「周制六軍出於六鄉,在三畿四郊之地;唐設十二衛,亦散布畿內郡縣,又以關內諸
 府分隸之,皆所以臨制四方,為國藩衛。國朝禁兵,多屯京師及畿內東南諸縣,雖饋運為便,而西邊武備殊闕。今中牟、長垣都門要沖,二鄙驛置皆由此,而舊不屯兵,闃無防守,請置營益兵,以備非常。」明年,饑民果乘虛犯長垣,戕官吏,如頌慮。頌又請以獲盜多寡為縣令殿最法,以謂:「巡檢、縣尉,但能捕盜,而不能使人不為盜;能使其不為盜者,縣令也。且民罹剽劫之害,而長官不任其責,可乎?」



 遷度支判官。送契丹使,宿恩州,驛舍火,左右請
 出避,頌不動。州兵欲入救,閉門不納,徐使防卒撲滅之。初火時,郡人洶洶,唱使者有變,救兵亦欲因而生事,賴頌安靜而止。遂聞京師,神宗疑焉。頌使還,入奏,稱善久之。命為淮南轉運使。召修起居注,擢知制誥、知通進銀臺司、知審刑院。



 時知金州張仲宣坐枉法贓罪至死,法官援李希輔例,杖脊黥配海島。頌奏曰:「希輔、仲宣均為枉法,情有輕重。希輔知臺,受賕數百千,額外度僧。仲宣所部金坑,發檄巡檢體究,其利甚微,土人憚興作,以金八
 兩屬仲宣,不差官比校,止系違令,可比恐喝條,視希輔有間矣。」神宗曰:「免杖而黥之,可乎?」頌曰:「古者刑不上大夫,仲宣官五品,今貸死而黥之,使與徒隸為伍,雖其人無可矜,所重者,污辱衣冠耳。」遂免仗黥,流海外,遂為定法。



 又言:「提舉青苗官不能體朝廷之意,邀功爭利,務為煩擾。且與諸司不相臨統,文移同異,州縣莫知適從。乞與常平、眾役一切付之監司,改提舉為之屬,則事有統一,而於更張之政無所損也。」不從。



 大臣薦秀州判官李
 定,召見,擢太子中允,除監察御史裏行。宋敏求知制誥,封還詞頭。復下,頌當制,頌奏:「祖宗朝,天下初定,故不起孤遠而登顯要者。真宗以來,雖有幽人異行,亦不至超越資品。今定不由銓考,擢授朝列;不緣御史,薦置憲臺。雖朝廷急於用才,度越常格,然隳紊法制,所益者小,所損者大,未敢具草。」次至李大臨,亦封還。神宗曰:「去年詔,臺官有闕,委御史臺奏舉,不拘官職高下。」頌與大臨對曰:「從前臺官,於太常博士以上、中行員外郎以下舉充。
 後為難得資敘相當,故朝廷特開此制。止是不限博士、員郎,非謂選人亦許奏舉。若不拘官職高下,並選人在其間,則是秀州判官亦可為里行,不必更改中允也。今定改京官,已是優恩,更處之憲臺,先朝以來,未有此比。幸門一啟,則士塗奔競之人,希望不次之擢,朝廷名器有限,焉得人人滿其意哉!」執奏不已,於是並落知制誥,歸工部郎中班,天下謂頌及敏求、大臨為「三舍人」。



 歲餘,知婺州。方溯桐廬,江水暴迅,舟橫欲覆,母在舟中幾溺
 矣,頌哀號赴水救之,舟忽自正。母甫及岸,舟乃覆,人以為純孝所感。徙亳州,有豪婦罪當杖而病,每旬檢之,未愈,譙簿鄧元孚謂頌子曰:「尊公高明以政稱,豈可為一婦所紿。但諭醫如法檢,自不誣矣。」頌曰:「萬事付公議,何容心焉。若言語輕重,則人有觀望,或致有悔。」既而婦死,元孚慚曰:「我輩狹小,豈可測公之用心也。」加集賢院學士、知應天府。呂惠卿嘗語人曰:「子容,吾鄉里先進,茍一詣我,執政可得也。」頌聞之,笑而不應。凡更三赦,大臨還
 侍從,頌才授秘書監、知通進銀臺司。吳越饑,選知杭州。一日,出遇百餘人,哀訴曰:「某以轉運司責逋市易緡錢,夜囚晝系,雖死無以償。」頌曰:「吾釋汝,使汝營生,奉衣食之餘,悉以償官,期以歲月而足,可乎?」皆謝不敢負,果如期而足。



 頌宴客有美堂,或告將兵欲亂,頌密使捕渠領十輩,荷校付獄中,迨夕會散,坐客不知也。及修兩朝正史,轉右諫議大夫。使契丹,遇冬至,其國歷後宋歷一日。北人問孰為是,頌曰:「歷家算術小異,遲速不同,如亥時
 節氣交,猶是今夕;若逾數刻,則屬子時,為明日矣。或先或後,各從其歷可也。」北人以為然。使還以奏,神宗嘉曰:「朕嘗思之,此最難處,卿所對殊善。」因問其山川、人情向背,對曰:「彼講和日久,頗竊中國典章禮義,以維持其政,上下相安,未有離貳之意。昔漢武帝自謂:『高皇帝遺朕平城之憂,雖久勤征討,而匈奴終不服。』至宣帝,呼韓單于稽首稱藩。唐自中葉以後,河湟陷於吐蕃,憲宗每讀《貞觀政要》,慨然有收復意。至宣宗時,乃以三關、七州歸
 於有司。由此觀之,外國之叛服不常,不系中國之盛衰也。」頌意蓋有所諷,神宗然之。



 元豐初,權知開封府,頗嚴鞭樸。謂京師浩穰,須彈壓,當以柱後惠文治之,非亳、穎臥治之比。有僧犯法,事連祥符令李純,頌置不治。御史舒但糾其故縱,貶秘書監、知濠州。



 初,頌在開封,國子博士陳世儒妻李惡世儒庶母,欲其死,語群婢曰:「博士一日持喪,當厚餉汝輩。」既而母為婢所殺,開封治獄,法吏謂李不明言使殺姑,法不至死。或譖頌欲寬世儒夫婦,
 帝召頌曰:「此人倫大惡,當窮竟。」對曰:「事在有司,臣固不敢言寬,亦不敢諭之使重。」獄久不決。至是,移之大理。意頌前次請求,移御史臺逮頌對。御史曰:「公速自言,毋重困辱。」頌曰:「誣人死,不可為已,若自誣以獲罪,何傷乎?」即手書數百言伏其咎。帝覽奏牘,以為疑,反復究實,乃大理丞賈種民增減其文傅致也,由是事得白。同列猶以嘗因人語及世儒帷薄事,頌應曰:「然。」以是為洩獄情,罷郡。



 未幾,知河陽,改知滄州。入辭,帝曰:「朕知卿久,然每欲
 用,輒為事奪,命也夫!卿直道,久而自明。」頌頓首謝。召判尚書吏部兼詳定官制。唐制,吏部主文選,兵部主武選;神宗謂三代、兩漢本無文武之別,議者不知所處。頌言:「唐制吏部有三銓之法,分品秩而掌選事。今欲文武一歸吏部,則宜分左右曹掌之,每選更以品秩分治。」於是吏部始有四選法。



 因陛對,神宗謂頌曰:「欲修一書,非卿不可。契丹通好八十餘年,盟誓、聘使、禮幣、儀式,皆無所考據,但患修書者遷延不早成耳。然以卿度,此書何時
 可就?」頌曰:「須一二年。」曰:「果然,非卿不能如是之敏也。」及書成,帝讀《序引》,喜曰:「正類《序卦》之文。」賜名《魯衛信錄》。



 帝嘗問宗子主祭、承重之義,頌對曰:「古者貴賤不同禮,諸侯、大夫世有爵祿,故有大宗、小宗、主祭、承重之義,則喪服從而異制,匹士庶人亦何預焉。近代不世爵,宗廟因而不立,尊卑亦無所統,其長子孫與眾子孫無以異也。今《五服敕》,嫡孫為祖、父為長子猶斬衰三年,生而情禮則一,死而喪服獨異,恐非先王制禮之本意。世俗之論,
 乃以三年之喪為承重,不知為承大宗之重也。臣聞慶歷中,朝廷議百僚應任子者,長子與長孫差優與官,餘皆降殺,亦近古立宗之法。乞詔禮官、博士參議禮律,合承重者,酌古今收族主祭之禮,立為宗子繼祖者,以異於眾子孫之法。士庶人不當同用一律,使人知尊祖,不違禮教也。」除吏部侍郎,遷光祿大夫。遭母喪,帝遣中貴人唁勞,賜白金千兩。



 元祐初,拜刑部尚書,遷吏部兼侍讀。奏:「國朝典章,沿襲唐舊,乞詔史官採《新》、《舊唐書》中君
 臣所行,日進數事,以備聖覽。」遂詔經筵官遇非講讀日,進漢、唐故事二條。頌每進可為規戒、有補時事者,必述己意,反復言之。又謂:「人主聰明,不可有所向,有則偏,偏則為患大矣。今守成之際,應之以無心,則無不治。」每進讀至弭兵息民,必援引古今,以動人主之意。



 既又請別制渾儀,因命頌提舉。頌既邃於律歷,以吏部令史韓公廉曉算術,有巧思,奏用之。授以古法,為臺三層,上設渾儀,中設渾象,下設司辰,貫以一機,激水轉輪,不假人力。
 時至刻臨,則司辰出告。星辰纏度所次,占候則驗,不差晷刻,晝夜晦明,皆可推見,前此未有也。



 頌前後掌四選五年,每選人改官,吏求垢瑕,故為稽滯。頌敕吏曰:某官緣某事當會某處,仍引合用條格,具委無漏落狀同上。自是吏不得逞。每訴者至,必取按牘使自省閱,訴者服,乃退;其不服,頌必往復詰難,度可行行之,茍有疑,則為奏請,或建白都堂。故選官多感德,其不得所欲者,亦心服而去。



 遷翰林學士承旨。五年,擢尚書左丞。嘗行樞密
 事。邊帥遣種樸入奏:「得諜言,阿裏骨已死,國人未知所立。契丹官趙純忠者,謹信可任,願乘其未定,以勁兵數千,擁純忠入其國立之。」眾議如其請。頌曰:「事未可知,其越境立君,使彼拒而不納,得無損威重乎?徐觀其變,俟其定而撫輯之,未晚也。」已而阿裏骨果無恙。



 七年,拜右僕射兼中書門下侍郎。頌為相,務在奉行故事,使百官守法遵職。量能授任,杜絕僥幸之原,深戒疆場之臣邀功生事。論議有未安者,毅然力爭之。賈易除知蘇州,頌
 言:「易在御史名敢言,既為監司矣,今因赦令,反下遷為州,不可。」爭論未決。諫官楊畏、來之邵謂稽留詔命,頌遂上章辭位,罷為觀文殿大學士、集禧觀使,繼出知揚州。徒河南,辭不行,告老,以中太一宮使居京口。紹聖四年,拜太子少師致仕。



 方頌執政時,見哲宗年幼,諸臣太紛紜,常曰:「君長,誰任其咎耶?」每大臣奏事,但取決於宣仁後,哲宗有言,或無對者。惟頌奏宣仁後,必再稟哲宗;有宣諭,必告諸臣以聽聖語。及貶元祐故臣,御史周秩劾
 頌。哲宗曰:「頌知君臣之義,無輕議此老。」徽宗立,進太子太保,爵累趙郡公。建中靖國元年夏至,自草遺表,明日卒,年八十二。詔輟視朝二日,贈司空。



 頌器局閎遠,不與人校短長,以禮法自持。雖貴,奉養如寒士。自書契以來,經史、九流、百家之說,至於圖緯、律呂、星官、算法、山經、本草,無所不通。尤明典故,喜為人言,亹亹不絕。朝廷有所制作,必就而正焉。



 嘗議學校,欲博士分經;課試諸生,以行藝為升俊之路。議貢舉,欲先行實而後文藝,去封彌、
 謄錄之法,使有司參考其素,行之自州縣始,庶幾復鄉貢里選之遣範。論者韙之。



 論曰:大防重厚,摯骨鯁,頌有德量。三人者,皆相於母後垂簾聽政之秋,而能使元祐之治,比隆嘉祐,其功豈易致哉!大防疏宋家法八事,言非溢美,是為萬世矜式。摯正邪之辨甚嚴,終以直道慍於群小,遂與大防並死於貶,士論冤之。頌獨巋然高年,未嘗為奸邪所污,世稱其明哲保身。然觀其論知州張仲宣受金事,犯顏辨其情
 罪重輕,又陳刑不上大夫之義,卒免仲宣於黥。自是宋世命官犯贓抵死者,例不加刑,豈非所為多雅德君子之事,造物者自有以相之歟?



\end{pinyinscope}