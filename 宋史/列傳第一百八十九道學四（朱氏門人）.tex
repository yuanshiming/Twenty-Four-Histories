\article{列傳第一百八十九道學四(朱氏門人)}

\begin{pinyinscope}

 ○黃乾李燔張洽陳淳李方子黃灝



 黃乾字直卿,福州閩縣人。父瑀,在高宗時為監察御史,
 以篤行直道著聞。瑀沒,,乾往見清江劉清之。清之奇之,曰:「子乃遠器,時學非所以處子也。」因命受業朱熹。乾家法嚴重,乃以白母,即日行。時大雪,既至而熹它出,乾因留客邸,臥起一榻,不解衣者二月,而熹始歸。干自見熹,夜不設榻,不解帶,少倦則微坐,一倚或至達曙。熹語人曰:「直卿志堅思苦,與之處甚有益。」嘗詣東萊呂祖謙,以所聞於熹者相質正。及廣漢張栻亡。熹與乾書曰:「吾道益孤矣,所望於賢者不輕。」後遂以其子妻干。



 寧宗即位,
 熹命干奉表,補將仕郎,銓中,授迪功郎,監臺州酒務。丁母憂,學者從之講學於墓廬甚眾。熹作竹林精舍成,遺乾書,有「它時便可請直卿代即講席」之語。及編《禮書》,獨以《喪》、《祭》二編屬乾,稿成,熹見而喜曰:「所立規模次第,縝密有條理,它日當取所編家鄉、邦國、王朝禮,悉仿此更定之。」病革,以深衣及所著書授干,手書與訣曰:「吾道之托在此,吾無憾矣。」訃聞,幹持心喪三年畢,調監嘉興府石門酒庫。



 時韓侂胄方謀用兵,吳獵帥湖北,將赴鎮,訪
 以兵事。乾曰:「聞議者謂今天下欲為大舉深入之謀,果爾,必敗。此何時而可進取哉?」獵雅敬幹名德,闢為荊湖北路安撫司激賞酒庫兼準備差遣,事有未當,必輸忠款力爭。



 江西提舉常平趙希懌、知撫州高商老闢為臨川令,歲旱,勸糶捕蝗極其力。改知新淦縣,吏民習知臨川之政,皆喜,不令而政行。以提舉常平郡太守薦,擢監尚書六部門,未上,改差通判安豐軍。淮西帥司檄乾鞫和州獄,獄故以疑未決,乾釋囚桎梏飲食之,委曲審問
 無所得。一夜,夢井中有人,明日呼囚詰之曰:「汝殺人,投之於井,我悉知之矣,胡得欺我。」囚遂驚服,果於廢井得尸。



 尋知漢陽軍。值歲饑,糴客米、發常平以振。制置司下令,欲移本軍之粟而禁其糴,乾報以乞候幹罷然後施行,及援鄂州例,十之一告糴於制司。荒政具舉。旁郡饑民輻湊,惠撫均一,春暖願歸者給之糧,不願者結廬居之,民大感悅。所至以重庠序,先教養。其在漢陽,即郡治後鳳棲山為屋,館四方士,立周、程、游、朱四先生祠。以病
 乞祠,主管武夷沖祐觀。



 尋起知安慶府,至則金人破光山,而沿邊多警。安慶去光山不遠,民情震恐。乃請於朝,城安慶以備戰守,不俟報,即日興工。城分十二料,先自築一料,計其工費若干,然後委官吏、寓公、士人分料主之。役民兵五千人,人役九十日,而計人戶產錢起丁夫,通役二萬夫,人十日而罷。役者更番,暑月月休六日,日午休一時,至秋漸殺其半。乾日以五鼓坐於堂,濠砦官入聽命,以一日成算授之:役某鄉民兵若干,某鄉人夫
 若干;分布於某人料分,或搬運某處土木,應副某料使用;某料民兵人夫合當更代,合散幾日錢米。俱受命畢,乃治府事,理民訟,接賓客,閱士卒,會僚佐講究邊防利病,次則巡城視役,晚入書院講論經史。築城之杵,用錢監未鑄之鐵,事畢還之。城成,會上元日張燈,士民扶老攜幼,往來不絕。有老嫗百歲,二子輿之,諸孫從,至府致謝。乾禮之,命具酒炙,且勞以金帛。嫗曰:「老婦之來,為一郡生靈謝耳,太守之賜非所冀也。」不受而去。是歲大旱,
 乾祈輒雨,或未出,晨興登郡閣,望灊山再拜,雨即至。後二年,金人破黃州沙窩諸關,淮東、西皆震,獨安慶按堵如故。繼而霖潦餘月,巨浸暴至,城屹然無虞。舒人德之,相謂曰:「不殘於寇,不滔於水,生汝者黃父也。」



 制置李玨闢為參議官,再辭不受。既而朝命與徐僑兩易和州,且令先赴制府稟議,幹即日解印趨制府。和州人日望其來,曰:「是嘗檄至吾郡鞫死囚、感夢於井中者,庶能直吾屈乎。」



 先是,幹移書玨曰:「丞相誅韓之後,懲意外之變,
 專用左右親信之人,往往得罪於天下公議。世之君子遂從而歸咎於丞相,丞相不堪其咎,繼然逐去之,而左右親信者其用愈專矣。平居無事,紀綱紊亂,不過州縣之間,百姓受禍。至於軍政不修,邊備廢弛,皆此曹為之,若今大敵在境,更不改圖,大事去矣。今日之急,莫大於此。」又曰:「今日之計,莫若用兩淮之人,食兩淮之粟,守兩淮之地。然其策當先明保伍,保伍既明,則為之立堡砦,蓄馬、制軍器以資其用,不過累月,軍政可成。且淮民遭丙
 寅之厄,今聞金人遷汴,莫不狼顧脅息,有棄田廬、挈妻子渡江之意,其間勇悍者。且將伺變竊發。向日胡海、張軍之變,為害甚於金,今若不早為之圖,則兩淮日見荒墟,卒有警急,攘臂而起矣。」玨皆不能用。



 及至制府,玨往惟揚視師,與偕行,乾言:「敵既退,當思所以賞功罰罪者。崔惟揚能於清平山豫立義砦,斷金人右臂,方儀真能措置捍禦,不使軍民倉皇奔軼,此二人者當薦之。泗上之敗,劉倬可斬也。某州官吏三人攜家奔竄,追而治之,然
 後具奏可也。」其時幕府書館皆輕儇浮靡之士,僚吏士民有獻謀畫,多為毀抹疏駁。將帥偏,人心不附,所向無功。流移滿道,而諸司長吏張宴無虛日。乾知不足與共事,歸自惟揚,再辭和州之命,仍乞祠,閉閣謝客,宴樂不與。乃復告玨曰:



 浮光敵退已兩月,安豐已一月,盱眙亦將兩旬,不知吾所措置者何事,所施行者何策。邊備之弛,又甚於前,日復一日,恬不知懼,恐其禍又不止今春矣。



 向者輕信人言,為泗上之役,喪師萬人。良將勁卒、
 精兵利器,不戰而淪於泗水,黃團老幼,俘虜殺戮五六千人,盱眙東西數百里,莽為丘墟。安豐、浮光之事大率類此。切意千乘言旋,必痛自咎責,出宿於外,大戒於國,曰:「此吾之罪也,有能箴吾失者,疾入諫。」日與僚屬及四方賢士討論條畫,以為後圖。今歸已五日矣,但聞請總領、運使至玉麟堂賞牡丹,用妓樂,又聞總領、運使請宴賞亦然,又聞宴僚屬亦然。邦人諸軍聞之,豈不痛憤。且視牡丹之紅艷,豈不思邊庭之流血;視管弦之啁啾,豈
 不思老幼之哀號;視棟宇之宏麗,豈不思士卒之暴露;視飲饌之豐美,豈不思流民之凍餒。敵國深侵,宇內騷動,主上食不甘味,聽朝不怡;大臣憂懼,不知所出。尚書豈得不朝夕憂懼,而乃如是之迂緩暇逸耶!



 今浮光之報又至矣,金欲以十六縣之眾,四月攻浮光,侵五關,且以一縣五千人為率,則當有八萬人攻浮光,以萬人刈吾麥,以五萬人攻吾關。吾之守關不過五六百人,豈能當萬人之眾哉?則關之不可守決矣。五關失守,則蘄、黃
 決不可保;蘄、黃不保,則江南危。尚書聞此亦已數日,乃不聞有所施行,何耶?



 其它言皆激切,同幕忌之尤甚,共詆排之。厥後光、黃、蘄繼失,果如其言。遂力辭去,請祠不已。



 俄再命知安慶,不就,入廬山訪其友李燔、陳宓,相與盤旋玉淵、三峽間,俯仰其師舊跡,講《乾》、《坤》二卦於白鹿書院,山南北之士皆來集。未幾,召赴行在所奏事,除大理丞,不拜,為御史李楠所劾。



 初,乾入荊湖幕府,奔走諸關,與江、淮豪傑游,而豪傑往往願依幹。及倅安豐、武定,
 諸將皆歸心焉。後倅建康,守漢陽,聲聞益著。諸豪又深知干倜儻有謀,及來安慶,且兼制幕,長淮軍民之心,翕然相向。此聲既出,在位者益忌,且慮幹入見必直言邊事,以悟上意,至是群起擠之。



 乾遂歸里,弟子日盛,巴蜀、江、湖之士皆來,編禮著書,日不暇給,夜與之講論經理,亹亹不倦,借鄰寺以處之,朝夕往來,質疑請益如熹時。俄命知潮州,辭不行,差主管亳州明道宮,逾月遂乞致仕,詔許之,特授承議郎。既沒後數年,以門人請謚,又特
 贈朝奉郎,與一子下州文學,謚文肅。有《經解》、文集行於世。



 李燔字敬子,南康建昌人。少孤,依舅氏。中紹熙元年進士第,授岳州教授,未上,往建陽從朱熹學。熹告以曾子弘毅之語,且曰:「致遠固以毅,而任重貴乎弘也。」燔退,以「弘」名其齋而自儆焉。至岳州,教士以古文六藝,不因時好,且曰:「古之人皆通材,用則文武兼焉。」即武學諸生文振而識高者拔之,闢射圃,令其習射;稟老將之長於藝
 者,以率偷惰。以祖母卒,解官承重而歸。



 改襄陽府教授。復往見熹,熹嘉之,凡諸生未達者先令訪燔,俟有所發,乃從熹折衷,諸生畏服。熹謂人曰:「燔交友有益,而進學可畏,且直諒樸實,處事不茍,它日任斯道者必燔也。」熹沒,學禁嚴,燔率同門往會葬,視封窆,不少怵。及詔訪遺逸,九江守以燔薦,召赴都堂審察,辭,再召,再辭。郡守請為白鹿書院堂長,學者雲集,講學之盛,它郡無與比。



 除大理司直,辭,尋添差江西運司干辦公事,江西帥李玨、
 漕使王補之交薦之。會洞寇作亂,帥、漕議平之,而各持其說。燔徐曰:「寇非吾民耶?豈必皆惡。然其如是,誠以吾有司貪刻者激之,及將校之邀功者逼城之耳。反是而行之,則皆民矣。」帥、漕曰:「乾辦議是。誰可行者?」燔請自往,乃駐兵萬安,會近洞諸巡尉,察隅保之尤無良者易置之,分兵守險,馳辯士諭賊逆順禍福,寇皆帖服。



 洪州地下,異時贛江漲而堤壞,久雨輒澇,燔白於帥、漕修之,自是田皆沃壤。漕司以十四界會子新行,價日損,乃視民
 稅產物力,各藏會子若干,官為封識,不時點閱,人愛重之則價可增,慢令者黥籍,而民言壽張,持空券益不售。燔與國子學錄李誠之力爭不能止。燔又入札爭之曰:「錢荒楮湧,子母不足以相權,不能行楮者,由錢不能權之也。楮不行而抑民藏之,是棄物也。誠能節用,先穀粟之實務,而不取必於楮幣,則楮幣為實用矣。」札入,漕司即弛禁,詣燔謝。燔又念社倉之置,僅貸有田之家,而力田之農不得沾惠,遂倡議裒穀創社倉,以貸佃人。



 有旨改
 官,通判潭州,辭,不許。真德秀為長沙帥,一府之事咸咨燔。不數月,辭歸。當是時,史彌遠當國,廢皇子竑,燔以三綱所關,自是不復出矣。真德秀及右史魏了翁薦之,差權通判隆興府,江西帥魏大有闢充參議官,皆辭,乃以直秘閣主管慶元至道宮。燔自惟居閑無以報國,乃薦崔與之、魏了翁、真德秀、陳宓、鄭寅、楊長孺、丁黼、棄宰、龔維藩、徐僑、劉宰、洪咨夔於朝。



 紹定五年,帝論及當時高士累召不起者,史臣李心傳以燔對,且曰:「燔乃朱熹高
 弟,經術行義亞黃乾,當今海內一人而已。」帝問今安在,心傳對曰:「燔,南康人,先帝以大理司直召,不起,比乞致仕。陛下誠能強起之,以置講筵,其裨聖學豈淺淺哉。」帝然其言,終不召也。九江蔡念成稱燔心事有如秋月。燔卒,年七十,贈直華文閣,謚文定,補其子舉下州文學。



 燔嘗曰:「凡人不必待仕宦有位為職事,方為功業,但隨力到處有以及物,即功業矣。」又嘗曰:「仕宦至卿相,不可失寒素體。夫子無入不自得者,正以磨挫驕奢,不至居移
 氣、養移體。」因誦古語曰:「分之所在,一毫躋攀不上,善處者退一步耳。」故燔處貧賤患難若平素,不為動,被服布素,雖貴不易。入仕凡四十二年,而歷官不過七考。居家講道,學者宗之,與黃乾並稱曰「黃、李。」孫鑣,登進士第。



 張洽字元德,臨江之清江人。父紱,第進士。洽少穎異,從朱熹學,自《六經》傳注而下,皆究其指歸,至於諸子百家、山經地志、老子浮屠之說,無所不讀。嘗取管子所謂「思之思之,又重思之,思之不通,鬼神將通之」之語,以為窮
 理之要。熹嘉其篤志,謂黃乾曰:「所望以永斯道之傳,如二三君者不數人也。」



 時行社倉法,洽請於縣,貸常平米三百石,建倉里中,六年而歸其本於官,鄉人利之。嘉定元年中第,授松滋尉。湖右經界不正,弊日甚,洽請行推排法,令以委洽。洽於是令民自實其土地疆界產業之數投於匱,乃籌核而次第之,吏奸無所匿。其後十餘年,訟者猶援以為證云。



 改袁州司理參軍。有大囚,訊之則服,尋復變異,且力能動搖官吏,累年不決,而逮系者甚
 眾。洽以白提點刑獄,殺之。有盜黠甚,辭不能折。會獄有兄弟爭財者,洽諭之曰:「訟於官,祗為胥吏之地,且冒法以求勝,孰與各守分以全手足之愛乎?」辭氣懇切,訟者感悟。盜聞之,自伏。民有殺人,賄其子焚之,居數年,事敗,洽治其獄無狀,憂之,且白郡委官體訪。俄夢有人拜於庭,示以傷痕在脅。翌日,委官上其事,果然。



 郡守以倉稟虛,籍倉吏二十餘家,命洽鞫之,洽廉知為都吏所賣。都吏者,州之巨蠹也,嘗干於倉不獲,故以此中之。洽度守
 意銳未可嬰,姑系之,而密令計倉庾所入以白守曰:「君之籍二十餘家者,以胥吏也。今校數歲之中所入,已豐於昔,由是觀之,胥吏妄矣。君必不忍受胥吏之妄,而籍無罪之家也。若以罪胥吏,過乃可免。」守悟,為罷都吏,而免所籍之家。



 知永新縣。一日謁告,聞獄中榜笞聲,蓋獄吏受賕,乘間訊囚使誣服也。洽大怒,亟執付獄,明日以上於郡,黥之。湖南酃寇作亂,與縣接壤,民大恐。洽單車以往,邑佐、寓士交諫,弗聽。至則寇未嘗至,乃延見隅官,
 訪利害而犒之,因行安福境上,結約土豪,得其歡心。未幾,南安舒寇將犯境,聞有備,乃去。



 以江東提舉常平薦,通判池州。獄有張德修者,誤𧾷就人以死,獄吏誣以故殺,洽訊而疑之,請再鞫,守不聽。會提點常平袁甫至,時方大旱,禱不應,洽言於甫曰:「漢、晉以來,濫刑而致旱,伸冤而得雨,載於方冊可考也。今天大旱,焉知非由德修事乎?」甫為閱款狀於獄,德修遂從徒罪。復白郡請蠲征稅,寬催科,以召和氣,守為寬稅。三日果大雨,民甚悅。洽數
 以病請祠,至是主管建昌仙都觀,以慶壽恩賜緋衣、銀魚。



 時袁甫提點江東刑獄,甫以白鹿書院廢弛,招洽為長。洽曰:「嘻,是先師之跡也,其可辭!」至則選好學之士日與講說,而汰其不率教者。凡養士之田乾沒於豪右者復之。學興,即謝病去。



 端平初,大臣多薦洽,召赴都堂審察,洽以疾不赴,乃除秘書郎,尋遷著作佐郎。度正、葉味道在經幄,帝數問張洽何時可到,將以說書待洽,洽固辭,遂除直秘閣,主管建康崇禧觀。嘉熙元年,以疾乞致
 仕,十月卒,年七十七。



 洽自少用力於敬,故以「主一」名齋。平居不異常人,至義所當為,則勇不可奪。居閑不言朝廷事,或因災異變故,輒顰蹙不樂,及聞一君子進用,士大夫直言朝廷得失,則喜見顏色。所交皆名士,如呂祖儉、黃乾、趙崇憲、蔡淵、吳必大、輔廣、李道傳、李燔、葉味道、李閎祖、李方子、柴中行、真德秀、魏了翁、李𡌴、趙汝讜、陳貴誼、杜孝嚴、度正、張嗣古,皆敬慕之。卒後一日,有旨除直寶章閣。所著書有《春秋集注》、《春秋集傳》、《左氏蒙求》、《續
 通鑒長編事略》、《歷代郡縣地理沿革表》、文集。



 子㯝、檉,賜同進士出身。



 陳淳字安卿,漳州龍溪人。少習舉子業,林宗臣見而奇之,且曰:「此非聖賢事業也。」因授以《近思錄》,淳退而讀之,遂盡棄其業焉。



 及朱熹來守其鄉,淳請受教,熹曰:「凡閱義理,必窮其原,如為人父何故止於慈,為人子何故止於孝,其他可類推也。」淳聞而為學益力,日求其所未至。熹數語人以「南來,吾道喜得陳淳」,門人有疑問不合者,
 則稱淳善問。後十年,淳復往見熹,陳其所得,時熹已寢疾,語之曰:「如公所學,已見本原,所闕者下學之功爾。」自是所聞皆要切語,凡三月而熹卒。



 淳追思師訓,前自裁抑,無書不讀,無物不格,日積月累,義理貫通,洞見條緒。故其言太極曰:太極只是理,理本圓,故太極之體渾淪。以理言,則自末而本,自本而末,一聚一散,而太極無所不極其至。自萬古之前與萬古之後,無端無始,此渾淪太極之全體也。自其沖漠無朕,而天地萬物皆由是出,
 及天地萬物既由是出,又復沖漠無朕,此渾淪無極之妙用也。聖人一心渾淪太極之全體,而酬酢萬變,無非太極流行之用。學問工夫,須從萬事萬物中貫過,湊成一渾淪大本,又於渾淪大本中散為萬事萬物,使無少窒礙,然後實體得渾淪至極者在我,而大用不差矣。」



 其言仁曰:「仁只是天理生生之全體,無表裏、動靜、隱顯、精粗之間,惟此心純是天理之公,而絕無一毫人欲之私,乃可以當其名。若一處有病痛,一事有欠闕,一念有間
 斷,則私意行而生理息,即頑痺不仁矣。」



 其語學者曰:「道理初無玄妙,只在日用人事間,但循序用功,便自有見。所謂『下學上達』者,須下學工夫到,乃可從事上達,然不可以此而安於小成也。夫盈天地間千條萬緒,是多少人事;聖人大成之地,千節萬目,是多少功夫。惟當開拓心胸,大作基址。須萬理明徹於胸中,將此心放在天地間一例看,然後可以語孔、孟之樂。須明三代法度,通之於當今而無不宜,然後為全儒,而可以語王佐事業。須
 運用酬酢,如探諸囊中而不匱,然後為資之深,取之左右逢其原,而真為已物矣。至於以天理人欲分數而驗賓主進退之幾,如好好色,惡惡臭,而為天理人欲強弱之證,必使之於是是非非如辨黑白,如遇鏌鎁,不容有騎墻不決之疑,則雖艱難險阻之中,無不從容自適,夫然後為知之至而行之盡。」此語又中學者膏肓,而示以標的也。



 淳性孝,母疾亟,號泣於天,乞以身代。弟妹未有室家者,皆婚嫁之。葬宗族之喪無歸者。居鄉不沽名徇
 俗,恬然退守,若無聞焉。然名播天下,世雖不用,而憂時論事,感慨動人,郡守以下皆禮重之,時造其廬而請焉。



 嘉定九年,待試中都,歸過嚴陵郡守鄭之悌,率僚屬延講郡庠。淳嘆陸、張、王,學問無源,全用禪家宗旨,認形氣之虛靈知覺為天理之妙,不由窮理格物,而欲徑造上達之境,反托聖門以自標榜。遂發明吾道之體統,師友之淵源,用功之節目,讀書之次序,為四章以示學者。明年,以特奏恩授迪功郎、泉州安溪主簿,未上而沒,年六
 十五。其所著有《語孟大學中庸》口義、字義、詳講,《禮》、《詩》、《女學》等書,門人錄其語,號《筠穀瀨口金山所聞》。



 李方子字公晦,昭武人。少博學能文,為人端謹純篤。初見朱熹,謂曰:「觀公為人,自是寡過,但寬大中要規矩,和緩中要果決。」遂以「果」名齋。長游太學,學官李道傳折官位輩行具刺就謁。



 嘉定七年,廷對擢第三,調泉州觀察推官。適真德秀來為守,以師友禮之,郡政大小咸咨焉。暇則辨論經訓,至夜分不倦。故事,秩滿必先通書廟堂
 乃除,方子曰:「以書通,是求也。」時丞相彌遠聞之怒,逾年始除國子錄。無何,將選入宮僚,而方子不少貶以求合。或告彌遠曰:「此真德秀黨也。」使臺臣劾罷之。



 方子既歸,學者畢集,危坐竟日,未始傾側,對賓客一語不妄發,雖奴隸亦不加詬詈,然常嚴憚之。嘗語人曰:「吾於問學雖未能周盡,然幸於大本有見處,此心常覺泰然,不為物欲所漬爾。」其亡也,天子閔之,與一子恩澤。



 黃灝字商伯,南康都昌人。幼敏悟強記,肄業荊山僧舍
 三年,入太學,擢進士第。教授隆興府,知德化縣,以興學校、崇政化為本。歲饉,行振給有方。王藺、劉穎薦於朝,除登聞鼓院。光宗即位,遷太常寺簿,論今禮教廢闕,請敕有司取政和冠昏喪葬儀,及司馬光、高閌等書參訂行之。



 除太府寺丞,出知常州,提舉本路常平。秀州海鹽民伐桑柘,毀屋廬,莩殣盈野,或食其子持一臂行乞,而州縣方督促捕欠,顥見之蹙然。時有旨倚閣夏稅,遂奏乞並閣秋苗,不俟報行之。言者罪其專,移居筠州,已而寢
 謫命,止削兩秩,而從其蠲閣之請。



 灝既歸里,幅巾深衣,騎驢匡山間,若素隱者。起知信州,改廣西轉運判官,移廣東提點刑獄,告老不赴。卒。



 灝性行端飭,以孝友稱。朱熹守南康,灝執弟子禮,質疑問難。熹之沒,黨禁方厲,灝單車往赴,徘徊不忍去者久之。



\end{pinyinscope}