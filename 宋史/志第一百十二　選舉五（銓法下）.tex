\article{志第一百十二 選舉五(銓法下)}

\begin{pinyinscope}

 遠州銓補蔭流外補



 川峽、閩、廣,阻遠險惡,中州之人,多不願仕其地。初,銓格稍限以法,凡州縣、幕職,每一任近,即一任遠。川峽、廣南
 及沿邊,不許挈家者為遠,餘悉為近。既分川峽為四路,廣南東、西為二路,福建一路,後增荊湖南一路,始立八路定差之制,許中州及土著在選者隨意就差,名曰「指射」,行之不廢。



 太平興國初,選人孟巒擬賓州錄事參軍,詣匭訴冤,坐流海島。自是,得遠地者不敢辭。既而詔:「川峽、嶺南、福建注授,計程外給兩月期,違則本州不得放上,遣送闕下,除籍不齒。或被疾,則所至陳牒,長吏按驗,付以公據;廢痼末損,則條狀以聞。」雍熙四年,又詔:「選人
 年六十,勿注遠地;非土人而願者聽。凡任廣、蜀、福建州縣,並給續食。」初,嶺南闕官,往往差攝。至是,詔州長吏試可者選用之;罷秩,奏送闕下,與出身。淳化間,又詔:「嶺南攝官,各路惟許選二十員以承乏,餘悉罷歸。」



 始,令嶺南幕職,許攜族行,受代不得寄留。至道初,申詔:「劍南州縣官,不得以族行。敢有妄稱妻為女奴,攜以之官,除名。」初,榮州司理判官鄭蛟,冒禁攜妻之任。會蜀賊李順構亂,其黨田子宣攻陷城邑,而蛟捕得之,擢為推官。至是,知
 梓州張雍奏其事,上命戮蛟,而有是詔。



 咸平間,以新、恩、循、梅四州瘴地,選荊湖、福建人注之。吏部銓擬官,悉標其過犯,自是,凡注惡地,令不須書。又詔:「規避遐遠,違期受代,勘鞫責罰,就移遠地。」



 神宗更制,始詔:「川峽、福建、廣南,之官罷任,迎送勞苦,其令轉運司立格就注,免其赴選。」於是七路自常選知州而下,轉運司置員闕籍,具書應代時日,下所部郡眾示之。凡見任距受代半年及已終更者,許用本資序指射。有司受而閱之,定其應格當
 差者,上之審官東院、流內銓,審覆如令,即奏聞降敕。若占籍本路,或游注此州,皆從其便;惟不許官本貫州縣及鄰境,其參擬銓次悉如銓格。無願注者,上其闕審官,而在選者射之。武臣之屬西院、三班院者,令樞密院放此具制。後荊湖南亦許就注。或言:「土人知州非便。法應遠近迭居,而川人許連任本路,常獲家便,實太偏濫。」王安石曰:「分遠近,均勞佚也。中州士不願適遠,四路人樂就家便,用新法即兩得所欲;況可以省吏卒將迎、官府
 浮費邪?」何正臣又言:「蜀人之在仕籍者特眾,今自郡守而下皆得就差,一郡之官,土人太半,寮寀吏民皆其鄉里親信,難於徇公,易以合黨。請收守令闕歸之朝廷,而他官兼用土人,量立分限,庶經久無弊。兼聞差注未至盡公,願許提刑司索案牘究察之。」奏上,法不為改,但申嚴提刑司互察之法。



 元祐初,御史上官均言:「定差不均之弊有七:諸路赴選中試乃差,八路隨意取射,一也。諸路吏部待試,需次率及七年,方成一任;八路就注,若及
 七年,已更三任,二也。八路雖坐停罷,隨許射注;而見在吏部待次之人,至有歷任無過,尚須試法,候及一年方有注擬,三也。其待次者又許權攝,祿無虛日;而吏選無愆犯,亦大率四年方再得祿,四也。土人得射奏名者,免試就注家便,年高力憊,不復望進,往往營私廢職,五也。仕久知識既多,土人就射本路,不無親故請托,六也。八路監司地遠而專,設漫滅功過名次,人亦不敢爭校;故有力者多得優便,而孤寒滯卻,七也。請並八路差盡歸吏部為便。」既而吏部亦請用常格差除,遂悉歸之銓。



 紹聖復行舊制,且許八
 路人陰補出官,即轉運司試中注闕。重和間,臣僚又言其弊:「轉運以軍儲、吏祿、供饋、支移為己責,而視差注為末務,往往付之主案吏胥定擬,而簽廳視成書判而已。注闕之高下,視賄之厚薄。無賂,則定差之牘,脫漏言詞,隱落節目。及其上部,必致退卻,參會重上,又半歲矣。以是闕多而不調者眾。宜督典領之官,歲終取吏部退難有無、多寡,為之課而賞罰之,庶可公注擬而絕吏賕。」乃命立《考課法》。



 建炎初,詔福建、二廣闕並歸吏部,惟四川
 仍舊制。初,累朝以廣南地遠,利入不足以資正官,故使舉人兩與薦選者,試刑法於漕司,以合格者注攝兩路,謂之「待次」。攝官更兩任無過,則錫以真命。至是,雖歸之吏部,逾年無願就者,復歸漕司。自神宗朝,宗室不許調川峽官;至是宗室多避難入蜀,乃聽於四路注擬。紹興六年,詔:「川峽轉運司每季孟月上旬集注。」為定法焉。八年,直學士院勾龍如淵上疏謂:「行都去蜀萬里,而比歲窠闕歸之朝廷,寒遠之士,困抑者眾。願參酌前制,稍還漕銓
 之舊,立為定格,使與堂除不相侵紊。」遂命以小郡知州、監以下,仍付漕司差注,其選人改官詣司公參,理為「到部」。人稱便焉。



 補蔭之制。凡奏戚屬,太皇太后、皇太后、皇后本服期親,奉禮郎;大功,守監簿;小功,初等幕職官;元豐前,試大理評事。



 緦麻,知令、錄。元豐前試校書郎。



 異服親亦如之。有服女之夫,則本服大功以上女夫,知令、錄;小功,判、司、主簿或尉;緦麻,試監簿。周功女之子,知令、錄;孫及大功女之子,判、司、主簿或
 尉;曾孫及大功女之孫、小功女之子,並試監簿;其非所生子若孫,各降一等;緦麻女之子,試監簿。



 每祀南郊、誕聖節,太皇太后、皇太后並錄親屬四人,皇后二人。非遇推恩而特旨賜官,不用此法。凡諸妃期親守監簿,餘判、司、主簿或尉;異姓親試監簿。婉容以上有服親,才人以上小功親,並試監簿。凡大長公主、長公主、公主夫之期親,判、司、主簿或尉,餘試監簿;子,補殿中丞;孫,光祿寺丞;婿,太常寺太祝;外孫,試銜、知縣。凡親王婿,大理評事;外
 孫,初等職官;女之子婿,試監簿。宗室緦麻以上女之夫,試銜、知縣;袒免,判、司、主簿或尉。其願補右職,依換官法。奉禮郎即右侍禁,幕職官即左班殿直,知令、錄即右班殿直,判、司、主簿、尉即奉職,試監簿即借職。



 凡文臣:三公、宰相子,為諸寺丞;期親,校書郎;餘親,本宗大功至緦麻服者。



 以屬遠近補試銜。使相、參知政事、樞密院使、副使、宣徽使子,為太祝、奉禮郎;期親,校書、正字;餘親,補試銜。節度使、僕射、尚書、太子三少、御史大夫、文明殿學士、資政殿大學
 士子,校書郎、正字;期親,寺、監主簿;餘親,試銜。三司使,翰林、資政殿侍講、龍圖閣學士,樞密直學士,太常、宗正卿,中丞,丞、郎,留後,觀察使,內客省使子,正字;期親,寺、監主簿;餘親,試銜及齋郎。兩省五品、龍圖閣直學士、待制、三司副使、知雜御史子,寺、監主簿;期親,試銜;餘親,齋郎。諸司大卿、監子,寺監主簿;期親,試銜。小卿、監兼職者子,試銜;期親,齋郎。



 凡武臣:宰相子,為東頭供奉官,使相、知樞密院子,為西頭供奉官;期親,皆左侍禁;餘屬,自左班殿
 直以下第官之。樞密使、副使、宣徽節度使子,西頭供奉官;期親,右侍禁;餘屬,自右班殿直以下第官之。六統軍諸衛上將軍、節度觀察留後、觀察使、內客省使子,右侍禁;期親,右班殿直;餘屬,三班奉職以下第官之。客省使、引進防禦使、團練使、四方館使、樞密都承旨、閣門使子,右班殿直;期親,三班奉職;餘屬,為差使、殿侍。諸衛大將軍、內諸司使、樞密院諸房副承旨子,三班奉職;期親,借職;餘屬,為下班殿侍。諸衛將軍、內諸司副使、樞密分房
 副承旨子,為三班借職。



 凡兼職在館閣校理、檢討,王府記室、翊善、侍講,三司主判官,開封府判官、推官,江淮發運,諸路轉運,始許奏及諸親。提點刑獄,惟許奏男。其嘗以贓抵罪,得復故官。文臣至郎中及員外郎任館閣職,武臣至諸司副使、諸衛將軍者,止許蔭子若孫一人,尚在謫籍者弗預。



 太祖初定任子之法,臺省六品、諸司五品,登朝嘗歷兩任,然後得請。始減歲補千牛、齋郎員額;齋郎須年貌合格,誦書精熟,乃得奏。



 太宗踐極,諸州進
 奏者授以試銜及三班職,初推恩授散試官者,不得赴選。太平興國二年,乃詔授試銜等人特定七選集,遂為定令。凡誕聖節及三年大祀,皆聽奏一人。而淳化改元恩,文班中書舍人、武班大將軍以上,並許蔭補;如遇轉品,許更蔭一子,由是奏薦之恩始廣。每誕聖節,朝臣多請奏疏屬,不報。至道二年,始限以翰林學士、兩省五品、尚書省四品以上,賜一子出身,此聖節奏薦例也。先是,任子得攝太祝、奉禮,未幾即補正員。帝謂:「膏粱之子,不
 十年坐致閨籍。」是年,悉授同學究出身赴選集。



 真宗東封,祀汾陰,進奉人已官者進秩,未官者令翰林試藝,與試銜、齋郎、借職。公主、郡縣主以下諸親,外命婦入內者,亦有恩慶。而東封恩,則提點刑獄、朝臣、使臣,皆得奏一人。奏戚屬,舊無定制。有求補閣門祗候者,真宗以宣贊之職,非可以恩澤授,乃詔:「自今求敘遷者,至殿直止。」大中祥符二年,以門蔭授京官,年二十五以上求差使者,令於國學受業,及二年,審官院與判監官考試其業,乃
 以名聞。內諸司使、副授邊任官者,陛辭時許奏子。詔樞密院定其制,凡妄名孫及從子為子求蔭者,坐之。七年,帝幸南京,詔臣僚逮事太祖者,賜一子恩澤,令翰林學士李維等定,自給諫、觀察使以上得請。初,轉運使辭日,許奏一人。天禧後,惟川、廣、福建者聽,餘路再任始得奏。又詔:「承天節恩例所蔭子孫,不許以他親及已食錄者。」特許西京分司官,郊禋奏蔭一子。自是分務西洛者得以為例,南京則否。



 仁宗慶歷中,裁損奏補入仕之路,凡
 選人遇郊赴銓試,其不赴試亦無舉者,永不預選。罷聖節奏蔭恩,學士以下,遇效恩得奏大功以上親,再遇郊得奏小功以下親。郎中、帶職員外郎,初遇郊蔭子若孫,再郊及期親,四遇郊聽蔭大功以下親。初得奏而年過六十無子孫,蔭期親。其皇親大將軍以上妻,再遇郊亦許之。武臣蔭例仿此。凡蔭長子孫皆不限年,諸子孫須年過十五;若弟侄須過二十,必五服親乃許。已嘗蔭而物故者,無子孫祿仕,聽再蔭。自是,任子之恩殺矣。



 英宗
 即位,郡縣致貢奉人,悉命以官。知諫院司馬光建言:「監司、太守,遣親屬奉表京師,不問官職高下、親屬近遠,推恩至班行、幕職、權知州軍,或所遣非親,亦除齋郎及差使、殿侍,此蓋國初承五代姑息藩鎮之弊,因循不革。爵錄本待賢才,今此等受官,誠為大濫。縱不能盡罷其人,若五服內親,等第受以一官,其無服屬量賜金帛,庶少救濫官之失。」然詔令已行,不從其議。時方患官冗,言者皆謂:「由三歲一磨勘,其進甚亟,易至高位,故獲蔭者眾。」
 乃令待制以上,自遷官後六歲,無故則復遷之,有過益展年,至諫議大夫止。京朝官四歲磨勘,至前行郎中止,少卿、監限七十員,員有闕,以前行郎中久次者補之。少卿、監以上遷官,聽旨。



 仁宗雖罷聖節恩,而猶行之妃、主。神宗既裁損臣僚奏蔭,以宮掖外戚恩尤濫,故稍抑之。舊,諸妃遇聖節奏親屬一人,間一年許奏二人,郊禮許奏一人。嬪御每遇郊奏一人,兩遇聖節與一奏。後定,諸妃每遇聖節並郊許奏有服親一人。淑儀、充儀、婕妤、貴
 人遇郊,許奏小功以上親一人,位號別而資品同者,許比類奏薦。舊,公主每遇聖節、郊禮,奏夫之親屬一人;公主生日,許奏一人。後罷生日恩,所奏須有服親。皇親妻兩遇郊,許奏期親一人,後罷奏。舊,郡、縣主遇郊,許奏親生子右班殿直,若庶子及其夫之親,兩遇郊許奏借職一人。後親子惟注幕職,孫若庶子,兩遇郊方許奏一人,夫之親屬勿奏。舊,臣僚之妻為國夫人者,得遺表恩,後除之。妃嬪、公主以下,非有服親之婿不許奏。既而曾布
 等又言:「臣僚陳請恩澤,宜有定制。」乃許見任二府歲乞差遣一人。宰臣、樞密使兼平章事因事罷者,陳乞轉官一人,指射差遣二人。余執政官,並各一人。待制以上乞差遣遷學士者又一人。三路、廣桂安撫使、知成都府、梓州差遣一人,親孫、子循一資。廣南轉運、提點刑獄奏子孫或期親合入官一人。成都、梓、利、夔路差遣一人,子孫循一資。中書檢正官、樞密院檢詳官至員外郎,在職及二年,遇大禮許補親屬。中書堂後官、提點五房官,雖未
 至員外,聽奏補。邕、宜、欽極邊煙瘴知州,聽奏子孫一人。凡因戰陣物故及歿於王事,許官其子孫。又功臣繪像之家,如無食錄人,則許特奏子孫一人入官。既定《銓試法》,任子中選者得隨銓擬注,其入優等,往往特旨賜進士出身。



 元祐元年詔:「諸軍致仕停放人,其遺表恩該及子而過五年自陳者,慮有冒濫,毋推恩。職事官卿、監以下應任子者,須官至朝奉郎,乃許奏。」三年,定宰臣、執政初遇郊,許奏本宗異姓親各一人,次遇郊,奏數如初。願
 用其恩與有官人,則許轉官並循資,或乞差遣,惟不得轉入朝官、循入支掌。應奏承務郎、殿直以上,許換升一任;不得升入通判。餘官三遇郊,許奏有官人。舊制,應奏兩人止者,次郊,止許奏有官人。其後,遇郊更合補蔭者,並準此為間隔之次;已致仕而遇大禮應奏補者,再奏而止。宣仁太皇太后諭輔臣曰:「近已裁減入流,本家恩澤,宜減四分之一。」呂公著等曰:「陛下臨朝同聽斷,本殿恩澤,自不當限數。先來所定,止與皇太后同等,豈可更
 損?」宣仁曰:「裁減恩澤,凡自上而始,則均一矣。」乃詔曰:「官冗之患,實極於今,茍非裁入流之數,無以清取士之原。吾以眇身率先天下,今後每遇聖節、大禮、生辰,合得親屬恩澤,並四分減一,皇太后、皇太妃同之。」



 哲宗既親政,詔復舊。凡乞致仕而不願轉官者,中大夫至朝奉郎及諸司使,許奏補本宗有服親一人;自奉議郎、內殿承制以下,許與有服親一人恩例;惟中大夫、中散大夫、諸司使帶遙郡者,蔭補外仍與有服親恩例;若致仕未受敕
 而身亡者,在外以陳乞至門下省日,在京以得旨日,亦許乞有服親恩例一人。初,《任子法》以長幼為序,若應奏者有廢疾,或嘗犯私罪至徒,或不肖難任從仕,許越奏其次。至是,始刪去格令「長幼為序」四字。



 五年,定《親王女郡主蔭補法》,遇大禮,許奏親屬一人,所生子仍與右班殿直;兩遇,奏子或孫與奉職;即用奏子孫恩回授外服親之夫,及夫之有服親者,有官人轉一官,毋得升朝,選人循一資,無官者與借職,須期以下親,乃得奏。吏部言:「
 皇太妃遇大禮,以應奏恩與其親屬,而服行不應法。」詔用皇后緦麻女之子為比,補借職。舊法,母後之家,十年一奏門客,而太妃未有法。紹聖初,詔皇太妃用興龍節奏親屬恩,回授門客。自是,太后每及八年、太妃十年,奏門客一名,與假承務郎,許參選。如年數未及,凡恩皆毋回授。



 元符後,命婦生皇子許依大禮奏有服親,三品以上三人。宗室緦麻親,許視異姓蔭孫。凡蔭補異姓,惟執政得奏,如簽書樞密院事雖依執政法,而所蔭即不理
 選限。後因轉官礙止法者,許回授未仕子孫,而貪冒者又請回授異姓,有司每沮止之,然亦多御筆許特補。



 政和間,尚書省定《回授格》,謂無官可轉,或可轉而官高不欲轉,或事大而功效顯著為一格,許奏補內外白身有服親;官有止法不可轉,功績次著為一格,許奏本宗白身袒免親;官不甚高、而功績大為一格,許奏本宗白身有服親;官不甚高、功不甚大為一格,而分為三,一與內外有官有服親,一與有官有服本宗親,一與有官有服
 者之子孫。凡為六等。



 宣和二年,殿中侍御史張汝舟言:「今法所該補奏,與先朝同。昔之官至大夫,歷官不下三五十年,而今閱三五年,有已至大夫者矣;諸翼將軍至武翼郎,須出官三十年,方許奏補;今文武官奏補,未嘗限年,此太濫也。至若中大夫以下及武功、武翼大夫,已求致仕而不及受敕,乃格其恩,於是有身謝而未受敕者,其家或至匿哀須限;然不及親受而不與沾恩者多矣,此太吝也。欲自今中大夫至帶職朝奉郎以上,雖遇
 郊恩,入官不及二十年,皆未許蔭補;雖已經奏薦,再遇郊恩年仍未及者,亦寢其奏,庶抑其濫。至於文武官及大夫以上嘗求休致,而身謝在出敕前,欲並許奏蔭,以補其不及。」詔尚書省文武官致仕,雖不及受敕,若無曾受蔭人,自有遺表恩。又寺、監長貳至開封少尹,系用職事蔭補,不合限年。餘從之。



 崇寧以來,類多泛賞,如曰「應奉有勞」、「獻頌可採」、「職事修舉」特授特轉者,皆無事狀可名,而直以與之。孟昌齡、朱勉父子、童貫、梁師成、李邦彥等,
 凡所請求皆有定價,故不三五年,選人有至正郎或員外,帶職小使臣至正、副使或入遙郡橫行者。而蔡京拔用從官,不論途轍,一言合意,即日持橐。又優堂吏,往往至中奉大夫,或換防禦、觀察使。由此任子百倍。飲宗即位,赦恩覃轉,惟許宗室;其文武臣止令回授有官有服親,且詔:「非法應回授及特許者,毋錄用。」



 高宗中興,復位《補蔭法》,內外臣僚子孫期親大功以下及異姓親隨,文武各有等秩,見《職官志》。建炎元年,詔:「宰執子弟以恩澤
 任待制以上者,並罷。」紹興四年詔:「文武太中大夫以上及見帶兩制職名,依舊不限年。內無出身自授官後以及十五年,年及三十、不系宮觀責降之人,聽依條補蔭。」七年,中書舍人趙思誠言:「孤寒之士,名在選部,皆待數年之闕,大率十年不得一任。今親祠之歲,任子約四千人,是十年之後,增萬二千員,科舉取士不與焉。將見寒士有三十年不得調者矣。祖宗時,仕至卿、監者,皆實以年勞、功績得之,年必六十,身不過得恩澤五六人。厥後
 私謁行,橫恩廣,有年未三十而官至大夫者,員數比祖宗時不知其幾倍,而恩例未嘗少損。有一人而任子至十餘者,此而不革,實蠹政事,望議革其弊。」會思誠去國,議遂格。舊法,惟贓罪不許任子,新令並及私罪徒,有司以為拘礙者多,遂罷新令。又詔:「宰執、侍從致仕遺表,惟補緦麻以上親,毋及異姓。」二十二年,以武臣多出軍中,爵秩高而族姓少,凡有薦奏,同姓皆期功,異姓皆中表,閭巷之徒附會以進。命須經統轄長官結罪保明,詭冒
 者連坐之。帝於後妃補蔭,每加裁抑,詔後族不得任從官。



 孝宗即位,思革冗官。初詔百官任子遇郊恩權免奏薦,年七十人,遇郊不許奏子。俄又詔,未奏者許一名。隆興元年,以張宋卿言蔭補冗濫,立為定法。凡員外轉正郎,正郎轉侍從,卿監之至中大夫,每初遇郊,則聽任一子;再經,則不許復請。遺表之恩,各減其一。減年之類,亦去其半。至府史之屬,武功之等,亦仿此差降之。



 乾道二年詔:「非泛補官,如宗室、戚里女夫捧香,異姓上書獻頌,
 隨奉使補官,陣亡女夫,異姓給使減年之類,轉至合奏薦官,候致仕與奏一名,嘗奏者不再奏。」四年,詔:「宗室袒免親諸衛將軍、武功大夫至武翼郎以上,遇大禮奏補親屬,並依外官法,著為令。」九年,詔:「文臣帶職員外郎及武翼大夫以上,生前未嘗奏薦者,與致仕恩澤一名;即已嘗奏薦而被蔭人身亡,許再請。應朝奉郎、武翼郎以上補授及三十年者,亦與一名。」又詔:「武臣嘗任執政官,遇郊聽補文資。」於是恩數視執政者亦得之。蓋戚里、宗
 王與夫攀附之臣,皆爭以文資祿其子,不可復正矣。自隆興著酬賞實歷對用轉官之法,遷官稍緩。至是,郊恩之奏視為減半,然猶未大艾也。淳熙九年,始詔:「減任子員數。自宰相、執政、侍從、卿監、正郎、員外郎,分為五等,每等降殺,以兩酌中定為止數,武臣如之。宰相十人,執政八人,侍從六人,中散大夫至中大夫四人,帶職朝奉郎至朝議大夫三人,通減三分之一。」於是冗濫漸革。



 寧宗慶元中,立《補蔭新格》,自使相以下有差,文臣中大夫、武
 臣防禦使以下,不許遺表推恩。嘉泰初,以官冗恩濫,凡宗女夫授官者,依舊法終身止任一子,兩府使相不得以郊恩奏門客,著為令。



 凡流外補選,五省、御史臺、九寺、三監、金吾司、四方館職掌,每歲遣近臣與判銓曹,就尚書同試律三道,中者補正名,理勞考。三館、秘閣楷書,皆本司試書札,中書覆試,補受。後以就試多懷挾傳授,乃鎖院、巡搜、糊名。凡試百司吏人,問律及疏,既考合格,復令口誦所對,以妨其弊。
 其自敘勞績,臣僚為之陳請,特免口誦,謂之「優試」。得優試者,率中選。後遂考試百司人,歲以二十人為額,毋得僥幸求優試。為職掌者,皆限年,授外州司戶、勒留,有至諸衛長吏、兩省主事者。



 學士、審官、審刑院,登聞檢鼓院,糾察刑獄司,皆選取諸司吏人,或以年限,或理本司選。然中書制敕及五院員闕,多即遣官特試書札,驗視材質。制敕院須堂後官以下親屬,五院須父祖有官者,樞密院亦如之,惟本院試驗。宣徽院、三司、各省、閣門、三班
 院,皆本司召補,至其首者出職。



 凡出職者,樞密院、三司,皆補借職以上,餘或補州縣。內廷諸司主吏、三司大將,亦有補三班借職者。中書主事以下,三司勾覆官以上,各帶諸州上佐;樞密院主事以上,皆帶同正將軍;餘多帶遠地司戶、簿、尉。



 先是,勒留、出官及選限,皆無定制。其隸近司,有才三二年即堂除外官者。咸平末,命翰林學士承旨宋白,與兩制、御史中丞同詳定之。白等請令「中書沿堂五院行首、副行首,依舊制補三班;通引、堂門、直
 省、發敕驗使臣,遇闕,依名次補正名;三年授勒留官,遇恩則一年,授後,七年出官。宣徽院貼房至都勾押官,軍將至知客、押衙各六等,並以次補;至勾押官、押衙,及五年以上出官,補三班或簿、尉。學士院孔目官,補正三年授勒留官,遇恩一年,授後,五年出官;驅使官,補正四年授勒留官,遇恩二年,授後,八年出官。三館孔目官,書直庫表奏、守當官,四年授勒留官,遇恩二年,授後,守當官八年、書直庫表奏官七年、孔目官六年出職;其職遷補
 者,許通計年考,有奉錢官者,更留三年。典書、楷書五選集,準格三館入流,歲數已少,無得以諸色優勞減選。閣門、客省、承受、驅使官轉次第,並依本司舊例補正名,四年授勒留官,遇恩則二年,授後,七年出授簿、尉;其行首並如舊制。審刑院本無職掌名額,於諸司選差正名,令不以有無勒留。審官五年、審刑三年,出官以前,諸司請自今勒留,並比七選集授官例,赴選日不以州縣地望為資敘。」從之。後又定客省承受、行首歲滿補殿直、奉職;
 御書院、翰林待詔、書藝祗候,十年以上無犯者聽出職。



 太祖嘗親閱諸司流外人,勒之歸農者四百人。開寶間,詔:「流外選人經十考入令、錄者,引對,方得注擬。驅使散從官、伎術人,資考雖多,亦不注擬。」堂後官多為奸臟,欲更用士之在令、錄、簿、尉選者充之;或不屑就,而所選不及數,乃如舊制。雍熙時,以堂後官充職事官,入謝外不赴朝參,見宰相禮同胥吏。端拱初,以河南府法曹參軍梁正辭、楚丘縣主簿喬蔚等五人為將作監丞,充中書
 堂後官,拔選人授京官為堂吏,自此始。



\end{pinyinscope}