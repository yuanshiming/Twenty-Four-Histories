\article{志第七十 禮二十(賓禮二)}

\begin{pinyinscope}

 入閣儀明堂聽政儀肆赦儀附皇太后垂簾儀皇太子正至受賀儀皇太子與百官師保相見儀



 入閣儀。唐制:天子日御正衙以見群臣,必立仗。朔望薦
 食陵寢,不能臨前殿,則御便殿,乃自正衙喚仗由宣政兩門而入,是謂東、西上閣門,群臣俟於正衙者因隨以入,故謂之入閣。五代以來,正衙既廢,而入閣亦希闊不講,宋復行之。



 建隆元年八月朔,太祖常服御崇元殿,設仗衛,文武百官入閣,始置待制、候對官,乃以工部尚書竇儀待制,太常卿邊光範候對。仗退,賜食廊下。



 乾德四年四月朔,常服通天冠、絳紗袍,御崇元殿視朝,設金吾仗衛,群臣入閣。



 太宗淳化二年十一月,詔以十二月朔
 御文德殿入閣,令史館修撰楊徽之、張洎定為新儀。前一日,有司供帳於文德殿宋初曰文明。是日既明,先列文武官於殿庭之東西,百官、軍校、行軍副使等序班於正衙門外屏南階下;次御史中丞、三院御史序立,中丞獨穿金吾班過揖兩班,一揖歸本位;次監察御史兩員監閣,於正衙門外屏北階上北面立;次中書、門下、文明翰林樞密直學士、兩省官分班立;次司天奏辰刻;次閣門版奏班齊。皇帝服靴袍乘輦,至長春殿駐輦,樞密使以下
 奏謁,前導至文德殿。殿上承旨索扇,卷簾。皇帝升位,扇卻,儀鸞使焚香;次文武官等拜;次司天雞唱;次閣門勘契;次閣門使承旨呼四色官喚仗,南班有辭謝者再拜先退,中書、門下班對揖,序立正衙門外屏北階上;次翰林學士、兩省官、中丞、侍御史序立;次金吾將軍押細仗入正衙門後,橫行拜訖,分行上黃道,仗隨入,金吾將軍至龍墀分班揖訖,序立;次吏部、兵部侍郎執文武班簿入,對揖立;次中書、門下、學士、兩省、御史臺官入,北面拜訖,上
 黃道,將至午階,靴急趨赴丹墀,彈奏御史至吏部侍郎南便落黃道,急趨就位;起居郎、舍人至兵部、吏部侍郎後,急趨而進,飛至香案前,皆揖訖序立;次金吾大將軍先對揖並鞠躬,靴行至折方石位又對揖,北行至奏事石位鞠躬,一員奏軍國內外平安,倒行就位;次引文武班就位,揖訖,鞠躬,靴急趨入沙墀;次引侍從班橫行,宰相祝月起居畢,分班序立;文武兩班出,序立於衙門外。刑法、待制官赴監奏位,中書、門下夾香案侍立,
 兩省、御史臺官、學士、兵部吏部侍郎、金吾將軍、監閣御史並相次出,就衙門外立惟學士立門側北候宰相。中書、門下詣香案前奏曰:「中書公事,臣等已具奏聞。」訖,乃退,揖殿出。次刑法官、待制官各奏事,並宣徽使答訖,乃出就班。次彈奏官、左右史出。閣內失儀者,彈糾如式。彈奏官失儀,起居郎糾之;起居郎失儀,閣門使糾之;閣門使失儀,宣徽使糾之。凡出者皆靴急趨揖殿。次中書、門下、學士就位,閣門使宣放仗,再拜,賜廊下食,又再拜。次閣門使奏
 閣內無事,文武官出,殿上索扇,垂簾,輦還宮。其賜廊下食,自左右勤政門北東西兩廊,文東武西,以北為上立定;中丞至本位,面南一揖,乃就坐食;至臺吏,贊乃搢笏食,食訖復贊,食畢而罷。五月朔,命有司增黃麾仗三百五十人,令文武官隨中書、門下橫行起居,徙翰林學士位於參知政事後,與節度使分東西揖殿出。真宗凡三行之,景德以後,其禮不行。仁宗從知制誥李淑議,仍讀時令,詔禮官詳定儀注,以言者謂未合典禮而罷。



 熙寧
 三年,知制誥宋敏求等言:「奉詔重修定閣門儀制內文德殿殿入閣儀,按今文德殿,唐宣政殿也;紫宸殿,唐紫宸殿也。然祖宗視朝,皆嘗御文德入閣。唐制,常設仗衛於宣政殿,或遇上坐紫宸,即喚仗入閣。如此,則當御紫宸殿入閣,方合舊典。」翰林學士王珪等議:「按入閣者,乃唐舊日紫宸殿受常朝之儀也。唐紫宸與今同,宣政殿即今文德殿。唐制,天子坐朝,必立仗於正衙。若止禦紫宸,即喚正衙仗自宣政殿東西閣門入,故為入閣。五代
 以來遂廢正衙立仗之制。今閣門所載入閣儀者,止是唐常朝之儀,非盛禮也。」自是入閣之禮遂罷。



 敏求又言:「本朝惟入閣乃御文德殿視朝,今既不用入閣儀,即文德遂闕視朝之禮。請下兩制及太常禮院,約唐制御宣政殿,裁定朔望禦文德殿儀,以備正衙視朝之制。」學士韓維等以《入閣圖》增損裁定上儀曰:



 朔日不值假,前五日,閣門移諸司排辦,前一日,有司供帳文德殿。其日,金吾將軍常服押本衛仗,判殿中省官押細仗,先入殿庭,
 東西對列;文武官東西序立;諸軍將校分入,北向立;朝堂引贊官引彈奏御史二員入殿門踏道,當下殿北向立;次催文武班分入,並東西相向立;諸軍將校即於殿庭北向立班。皇帝服靴袍禦垂拱殿,鳴鞭,內侍、閣門、管軍依朔望常例起居;次引樞密、宣徽、三司使副、樞密直學士、內客省使以下至醫官、待詔及修起居注官二員並大起居。諸司使以下,退排立。帝輦至文德殿後,閣門奏班齊,帝出,殿上索扇,升榻,鳴鞭;扇開,卷簾,儀鸞使焚
 香,喝文武官就位,四拜起居;雞人唱時;舍人於彈奏御史班前西向喝大起居。御史由文武班後至對立位,次引左右金吾將軍合班於宣制石南大起居,班首出班躬奏軍國內外平安,歸位再拜,各歸東西押仗位。通喝舍人於宣制石南北向對立。舍人退於西階,次揖宰臣、親王以下,躬奏文武百僚、宰臣某姓名以下起居,分引宰臣以下橫行,諸軍將校仍舊立。閣門使喝大起居,舍人引宰臣至儀石北,俯伏跪致詞祝月訖,其詞云:「文武
 百僚、宰臣全銜臣某姓名等言:孟春之吉,伏惟皇帝陛下膺受時祉,與天無窮,臣等無任歡呼抃蹈之至。」歸位五拜。閣門使揖中書由東階升殿,樞密使帶平章事以上由西階升殿侍立;給事中一員歸左省位立;轉對官立於給事中之南如罷轉對官,每遇御史臺前期牒請。文官二員並依轉對官例,先於閣門投進奏狀;吏部侍郎及刑法官立於轉對官南;兵部侍郎於右省班南,與吏部侍郎東西相向立,搢笏,各出班籍置笏上吏部、兵部侍郎以知審官東、西院官充,刑法官以知審刑、大理寺官充;親王、使相以
 下分班出;引轉對官於宣制石南,宣徽使殿上承旨宣答如儀;次吏部、兵部侍郎及刑法官對揖出;次彈奏御史無彈奏對揖出如有彈奏,並如儀。引給事中至宣制石南揖,躬奏殿中無事;喝祗候,揖,西出;次引修起居注官,次引排立供奉官以下各合班於宣制石南躬;喝祗候揖,分班出;喝天武官等門外祗候,出。索扇,垂簾,皇帝降坐,鳴鞭;舍人當殿承旨放仗,四色官靴急趨至宣制石南,稱奉敕放仗。金吾將軍並判殿中省官對拜,訖,隨仗出,
 親王、使相、節度使至刺史、學士、臺省官、諸軍將校等並序班朝堂,謝賜茶酒。帝復御垂拱殿,中書、樞密及請對官奏事;不引見謝、辭班。後殿坐,臨時取旨。其日遇有德音、制書、御札,仍候退御垂拱殿坐,制箱出外。應正衙見、謝、辭文武臣僚,並依御史臺儀制喚班,依序分入於文武班後,以北為首,分東西相向,重行異位,依見、辭、謝班序位。餘押班臣僚於班稍前押班,候刑法官對揖出,分引近前揖躬。舍人當殿宣班,引轉對班見、謝、辭,並如紫
 宸儀。樞密使不帶平章事、參知政事至同簽書樞密院事、宣徽使並立於宣制石南稍北,宰臣、親王、樞密使帶平章事、使相系押班者,立於儀石南,餘官並立於宣制石南,如合通喚,閣門使引並如儀。贊喝訖,系中書、樞密並揖升殿辭謝,揖,西出,其合問聖體者,並如儀;餘官分班出彈奏御史候見、謝、辭班絕,對揖出。其朝見,如謝都城門外禦筵,及召赴闕謝茶藥撫問之類,不可合班者,各依別班中謝對。賜酒食等並門賜。其系正衙見門謝辭,亦門外唱放。



 應正衙見、謝、辭臣僚,前一日於閣門投詣正衙榜子,閣門上奏目;又投正
 衙狀於御史臺、四方館。應朔日或得旨罷文德殿視朝,止禦紫宸殿起居,其已上奏目。正衙見、謝、辭班並放免,依官品隨赴紫宸殿引,或值改,依常朝文德殿,自有百官班日,並如舊儀。應外國蕃客見、辭,候喚班先引赴殿庭東,依本國職次重行異位立,候見、辭、謝班絕,西向躬。舍人當殿通班轉於宣制石南,北向立,贊喝如儀,西出。其酒食分物並門賜,如有進奉,候彈奏御史出,進奉入唯御馬及擔床自殿西偏門入,東偏門出。其進奉出入,文武官起居,舍人通某國進奉,宣徽使喝進奉出,節次
 如紫宸儀。候進奉出,給事中奏殿中無事,出。其後殿再坐,合引出者,從別儀。



 其日,賜茶酒,宰臣、樞密於閣子,親王於本廳,使相、宣徽使、兩省官、待制、三司副使、文武百官、皇親使相以下至率府副率,及四廂都指揮使以下至副都頭,並於朝堂如朝堂位次不足,即於朝堂門外設次。管軍節度使至四廂都指揮使、節度使、兩使留後至刺史,並於客省廳。



 詔從所定。



 徽宗初建明堂,禮制局列上七議:



 一曰:古者朔必告廟,
 示不敢專。請視朝聽朔必先奏告,以見繼述之意。



 二曰:古者天子負扆南向以朝諸侯,聽朔則各隨其方。請自今御明堂正南向之位,布政則隨月而御堂,其閏月則居門焉。



 三曰:《禮記·月令》,天子居青陽、總章,每月異禮。請稽《月令》十二堂之制,修定時令,使有司奉而行之。



 四曰:《月令》以季秋之月為來歲受朔之日。請以每歲十月於明堂受新歷,退而頒之郡國。



 五曰:古者天子負扆,公、侯、伯、子、男、蠻夷戎狄四塞之國各以內外尊卑為位。請自
 今元正、冬至及大朝會並御明堂,遼使依賓禮,蕃國各隨其方,立於四門之外。



 六曰:古者以明堂為布政之宮,自今若有御札、手詔並請先於明堂宣示,然後榜之朝堂,頒之天下。



 七曰:赦書、德音,舊制宣於文德殿,自今非禦樓肆赦,並於明堂宣讀。



 政和七年九月一日,詔頒朔、布政自十月為始。是月一日,上御明堂平朔左個,頒天運、政治及八年戊戌歲運、歷數於天下。自是每月朔御明堂布是月之政。先是,群臣五上表請負扆聽朝,詔弗
 允,至是復再請,始從之。十一月一日上御明堂,南面以朝百闢,退坐於平朔頒政。其禮:百官常服立明堂下,乘輿自內殿出,負坐斧扆明堂。大晟樂作,百官朝於堂下,大臣升階進呈所頒布時令,左右丞一員跪請付外施行,宰相承制可之,左右丞乃下授頒政官,頒政官受而讀之訖,出,閣門奏禮畢。帝降坐,百官乃退。自是以為常。其歲運、歷數、天運、政治之辭,文多不載。是後則各隨歲月星歷氣運推移沿改,而易其辭焉。



 初,尚書左丞薛昂
 條具崇寧以來紹述熙、豐政事,各條其節目,系之月令,頒於明堂。尋詔:「頒月之朔,使民知寒暑燥濕之化,而萬里之遠,雖驛置日行五百里,已不及時。其千里外當前期十日進呈取旨,頒布諸州長吏封掌,俟月朔宣讀之。」



 宣和元年,蔡京言:「周觀治象於正月之始和,以十二月頒告朔於邦國,皆不在十月。後世以十月者,祖秦朔故也。秦以十月為歲首,故月令以孟冬頒來歲之朔,今不當用。請以季冬頒歲運於天下。」詔自今以正月旦進呈宣
 讀。四年二月,太常王黼編類《明堂頒朔布政詔書》、《條例》、《氣令應驗》,凡六十三冊,上之。靖康元年,詔罷頒朔布政。



 御樓肆赦。每郊祀前一日,有司設百官、親王、蕃國諸州朝貢使、僧道、耆老位宣德門外,太常設宮縣、鉦鼓。其日,刑部錄諸囚以俟。駕還至宣德門內幄次,改常服,群臣就位,帝登樓御坐,樞密使、宣徽使侍立,仗衛如儀。通事舍人引群臣橫行再拜訖,復位。侍臣宣曰「承旨」,舍人詣
 樓前,侍臣宣敕立金雞。舍人退詣班南,宣付所司訖,太常擊鼓集囚。少府監立雞竿於樓東南隅,竿末伎人四面緣繩爭上,取雞口所銜絳幡,獲者即與之。樓上以朱絲繩貫木鶴,仙人乘之,奉制書循繩而下,至地,以畫臺承鶴,有司取制書置案上。閣門使承旨引案宣付中書、門下,轉授通事舍人,北面宣云「有制」,百官再拜。宣赦訖,還授中書、門下,付刑部侍郎承旨放囚,百官稱賀。閣門使進詣前,承旨宣答訖,百官又再拜、舞蹈,退。若德音、赦
 書自內出者,並如文德殿宣制之儀。其降御札,亦閣門使跪授殿門外置箱中,百官班定,閣門授宰臣讀訖,傳告,百僚皆拜舞稱萬歲。真宗宣制,有司請用儀仗四千人,自承天殿設細仗導衛,近臣起居訖,則分左右前導之。



 皇太后臨朝聽政。乾興元年,真宗崩,遺旨以皇帝尚幼,軍國事兼權取皇太后處分。宰相率百官稱賀,復前奉慰,又慰皇太后於簾前。有司詳定儀式:內東門拜表,合
 差入內都知一員跪授傳進;皇太后所降批答,首書「覽表具之」,末云「所請宜許」或「不許」。初,丁謂定皇太后稱「予」,中書與禮院參議,每下制令稱「予」,便殿處分稱「吾」。皇太后詔:「止稱『吾』,與皇帝並御承明殿垂簾決事。」百官表賀。



 英宗即位,輔臣請與皇太后權同聽政。禮院議:自四月內東門小殿垂簾,兩府合班起居,以次奏事,非時召學士亦許至小殿。時帝以疾權居柔儀殿東閣西室,太后垂簾處分稱「吾」,唯兩府日入候問聖體,因奏政事,退詣
 小殿簾外,覆奏太后。帝疾間,御前後殿聽政,兩府退朝,猶於小殿覆奏。



 哲宗即位,太皇太后權同聽政。三省、樞密院按儀注:未釋服以前,遇只日皇帝御迎陽門,日參官並赴起居,依例奏事。每五日,遇只日於迎陽門垂簾,皇帝坐於簾內之北,宰執奏事則權屏去左右侍衛;事有機速,許非時請對,及賜宣召,亦許升殿。禮部、御史臺、閣門奏討論御殿及垂簾儀制,每朔、望、六參,皇帝御前殿,百官起居,三省、樞密院奏事,應見、謝、辭班退,各令詣
 內東門進榜子。皇帝只日御延和殿垂簾,日參官起居太皇太后,移班少西起居皇帝,並再拜。三省、樞密院奏事,三日以上四拜,不舞蹈,候祔廟畢,起居如常儀。簾前通事以內侍,殿下以閣門。吏部磨勘奏舉人,垂簾日引。應見、謝、辭臣僚遇朔、望參日不坐,並先詣殿門,次內東門,應抬賜者並門賜之。於是帝御迎陽門幄殿,同太皇太后垂簾,宰臣、親王以下合班起居。常制分班十六,至是合班,以閣門奏請故也。禮官請如有祥瑞、邊捷,宰臣
 以下紫宸殿稱賀皇帝畢,赴內東門賀太皇太后。從之。



 徽宗即位,皇太后權同聽政。三省、樞密院聚議:故事,嘉祐末,英宗請慈聖同聽政,五月同御內東門小殿垂簾,至七月十三日英宗間日御前後殿,輔臣奏事,退詣內東門簾前覆奏。又故事,唯慈聖不立生辰節名,不遣使契丹;若天聖、元豐則御殿垂簾,立生辰節名,遣使與契丹往還及避家諱等。曾布曰:「今上長君,豈可垂簾聽政?請如嘉祐故事。」蔡卞曰:「天聖、元豐與今日皆遺制處分,
 非嘉祐比。」布曰:「今日之事,雖載遺制,實出自德音,又皆長君,正與嘉祐事相似。」有旨:依嘉祐、治平故事。布語同列曰:「奏事先太后,次覆奏皇帝,如今日所得旨。」遂為定式矣。尋以哲宗靈駕發引,太后手書罷同聽斷焉。



 皇太子元正、冬至受群臣賀儀。《政和新儀》:前一日,有司於東門外量地之宜,設三公以下文武群官等次如常儀;典儀設皇太子答拜褥位於階下,南向,又設文武群官版位於門之外。其日,禮直官、舍人先引三公以下文
 武群臣以次入,就位立定。禮直官、舍人引左庶子詣皇太子前,跪請內嚴;少頃,又言外備。內侍褰簾,皇太子常服出次,左右侍衛如常儀。皇太子降階詣南向褥位,典儀曰「再拜」,贊者承傳曰「再拜」,三公以下皆再拜,皇太子答拜。班首少前稱賀云:「元正首祚冬至云「天正長至」,景福維新。伏惟皇太子殿下,與時同休。」賀訖,少退,復位。左庶子前,承命詣群臣前答云:「元正首祚冬至云「天正長至」,與公等均慶。」典儀曰「再拜」,班首以下皆再拜,皇太子答拜。訖,禮直官、
 通事舍人引三公以下文武百官以次出,內侍引皇太子升階,還次,降簾,侍衛如常儀。



 少頃,禮直官、舍人引知樞密院官以下入,就位立定,內侍引皇太子降階,詣南向褥位,樞密以下參賀如上儀。訖,退。次引師、傅、保、賓客以下入,就位,參賀如上儀。師、傅、保以下以次出。



 內侍引皇太子升坐,禮直官引文武宮官入,就位,重行北向立,典儀曰「再拜」,在位官皆再拜。左庶子少前,跪言:「具官某言:元正首祚冬至云「天正長至」



 ,伏惟皇太子殿下,與時同休。」俯
 伏,興,復位。典儀曰「再拜」,在位者皆再拜,分東西序立。左庶子少前,跪言禮畢。左右近侍降簾,皇太子降坐,宮官退,左右侍衛以次出。



 皇太子與百官相見。至道元年,有司言:「百官見皇太子,自兩省五品、尚書省御史臺四品、諸司三品以上皆答拜,餘悉受拜。宮官自左右庶子以下,悉用參見之儀。其宴會位在王公上。」



 與師、傅、保相見。《政和新儀》:前一日,所司設師、傅、保以下次與宮門外道,西南向;設軒架之樂
 於殿庭,近南,北向。其日質明,諸衛率各勒所部屯門列仗,典謁設皇太子位於殿東階下西向,設師、傅、保位,於殿西階之西,三少位於傅、保之南稍卻,俱東向北上。師、傅、保以下俱朝服至宮門,通事舍人引就次,左庶子請內嚴。通事舍人引師、傅、保立於正殿門之西,三少在其南稍卻,俱東向北上。左庶子言外備,諸侍奉之官各服其器服,俱詣閣奉迎。皇太子朝服以出,左右侍衛如常儀,軒架作《翼安》之樂,至東階下西向立,樂止。通事舍人
 引師、傅、保及三少入,就位,軒架作《正安》之樂,至位樂止。皇太子再拜,師、傅、保以下答拜若三少特見,則三少先拜。通事舍人引師、傅、保以下出,軒架《正安之樂》作,出門,樂止。左庶子前跪稱:「左庶子某言,禮畢。」皇太子入,左右侍衛及樂作如來儀。



\end{pinyinscope}