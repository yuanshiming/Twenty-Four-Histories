\article{本紀第三十一}

\begin{pinyinscope}

 高宗八



 二十三年春正月癸卯,進韋淵太傅。己酉,復以李顯忠為寧國軍節度使。



 二月癸亥,幸玉津園,遂幸延祥觀。庚午,臠虔州軍賊黃明等八人於都市。辛未,改虔州為贛
 州。壬申,申嚴冒貫請舉法。癸未,賞平贛盜功,以李耕為金州觀察使,將士進秩、給賞有差。



 三月丙午,齊安郡王士人褭薨於建州,追封循王。詔凡民認復軍莊營田者,償開耕錢。丁未,禁州縣都監、巡尉擅置刑獄。戊申,以前太府丞範彥輝謗訕,除名、荊門軍編管。是春,金主亮徙都燕京。



 夏四月辛巳,詔諸州編管、羈管人,遵舊法,長吏月一驗視,不許囚禁。乙酉,減利州歲鑄錢為九萬緡。



 五月庚寅,禁州縣以私意籍罪人貲產。乙巳,復以蕭振為四
 川制置使。辛亥,金遣紇石烈大雅等來賀天申節。乙卯,立淮南諸州舉人解額。



 六月乙卯,潼川大水。



 秋七月壬辰,寬理平江府、湖、秀二州被水民夏稅。戊戌,從秦檜所請,命臺州取綦崇禮草檜罷相制所受墨敕。庚戌,禁諸軍瀕太湖擅作壩田。



 八月乙丑,士撙薨,追封韶王。丙寅,左宣教郎王孝廉謀據成都叛,事覺,伏誅。己卯,賜秦檜建康府永豐圩田。乙酉,命敕令所編輯中興以後寬恤詔令。



 九月甲午,振潼川被水州縣,仍蠲其賦。庚子,禁採
 鹿胎。



 冬十月丁巳,詔郡守年七十者聽自陳,命主宮觀。戊午,遣吳□使金賀正旦,施鉅賀金主生辰。戊辰,宋樸罷。壬申,以右諫議大夫史才簽書樞密院事兼權參知政事。丁丑,遣戶部郎官鐘世明修築宣州、太平州圩田。是月,命大理鞫妖人孫士道獄。



 十一月壬寅,詔立張叔夜廟於信州。甲辰,班《大宗正司條令》。乙丑,以經筵終帙,賜宰執、講讀等官宴於秘書省,為故事。十二月丁巳,詔州縣稅額少者,罷其監官。癸亥,韋淵薨。癸未,禁民車服
 逾制。



 閏月丙申,命檢正都司官詳定郡守所上利病以聞。辛丑,命諸軍保任統制官在職十年無過者進秩。庚戌,金遣蔡松年等來賀明年正旦。是歲,減池州青陽縣田租萬七千石。



 二十四年春正月辛未,幸延祥觀。癸酉,初詔郡國同以八月十五日試舉人。丙子,封婉容劉氏為貴妃。戊寅,地震。



 二月丁亥,前左從政郎楊炬坐其弟煒嘗上書誹謗,送邕州編管。丙午,加吳益太尉。



 三月壬申,楊再興復寇
 邊,前軍統制李道討平之,禽再興及其子正修、正拱,檻送行在。乙亥,賜禮部進士張孝祥以下三百五十六人及第、出身。庚辰,秦檜以私憾捃摭知建康府王循友,詔大理鞫之。是春,始榷夔州路茶。



 夏四月丙戌,詔諸路招補三衙諸軍,期三年課其殿最。辛丑,西南小張蕃貢方物。己酉,羅殿國貢名馬。



 五月癸丑朔,日有食之。衢州民俞八作亂,圍州城,通判州事汪召錫拒卻之,遂掠嚴州壽昌縣,遣殿前司正將辛立討平之。辛未,金遣耶律安
 禮等來賀天申節。



 六月癸巳,史才罷。甲午,以御史中丞魏師遜簽書樞密院事兼權參知政事。辛丑,王循友貸死、藤州安置。癸卯,詔:「嘗命四川州縣減免財物,以寬民力,尚慮未周,令制置司、總領所同共措置,務在不妨軍食,可以裕民。」尋遣鐘世明如四川同議。以主管侍衛馬軍司成閔為慶遠軍節度使。



 秋七月癸丑,張俊薨。勒停人王趯坐交通李光,下大理獄。乙卯,臠徭人楊正修、正拱於市。乙未,復置邛、雅二州博易場三所。壬戌,詔捐四
 川茶馬司羨餘錢給軍費,以寬民力。甲子,復落蕭振職、池州居住。乙丑,以總領財賦符行中為四川制置使。乙亥,南丹州莫公晟及宜州界外諸蠻納土內附。戊寅,幸張俊第臨奠。



 八月壬辰,禁百官避免輪對。甲午,罷溫州市黃柑、福州貢荔枝。丙午,追封張俊為循王。以湘潭縣丞鄭杞、主簿賈子展嘲毀朝政,除名,杞容州、子展德慶府編管。



 九月辛亥朔,李道如衡州措置盜賊。丁巳,賞平衢賊功,升辛立領忠州團練使,將士遷職、給錢有差。



 冬
 十月壬午,蠲旱傷州縣租賦。戊子,遣沉虛中使金賀正旦,張士襄賀金主生辰。



 十一月乙丑,魏師遜罷。丁卯,以權吏部侍郎施鉅參知政事,鄭仲熊簽書樞密院事。戊辰,進秦熹少傅,封嘉國公。是月,以通判武岡軍方疇通書胡銓及他罪,除名、永州編管。十二月丙戌,以故龍圖閣學士程瑀有《論語講解》,秦檜疑其譏己,知饒州洪興祖嘗為序,京西轉運副使魏安行鏤版,至是命毀之。興祖昭州、安行欽州編管,瑀子孫亦論罪。丁亥,王趯除名、
 辰州編管。丁酉,知鄞縣程緯為其丞王肇所告,慢上無人臣禮,除名、貴州編管,籍其貲。壬寅,刺諸路編管人充廂軍。乙巳,金遣白彥恭等來賀明年正旦。



 二十五年春正月辛未,賞討楊再興功,保寧軍承宣使李道落階官,加龍神衛四廂都指揮使,將士進官、賜錢有差。



 二月乙酉,以鎮江都統制劉寶為安慶軍節度使,建康都統制王權為清遠軍節度使。壬寅,以通判常州沉長卿、仁和縣尉芮燁作詩譏訕,除名,長卿化州、燁武
 岡軍編管。



 三月己酉,右司郎中張士襄自金國使還,坐奉使不肅罷官。壬申,地震。



 夏四月乙酉,施鉅罷,以鄭仲熊兼權參知政事。戊子,命四川制置司許就類省試院校試刑法。己亥,減廣西路折米錢。



 五月丁未朔,日有食之。太廟仁宗室柱生芝九莖。戊申,罷諸路免行錢歲百八萬緡。癸丑,以前知泉州宗室令衿譏訕秦檜,遂坐交結罪人、汀州居住。乙丑,金遣李通等來賀天申節。壬申,賜劉錡湖南田百頃。



 六月庚辰,鄭仲熊罷。辛巳,以禮部
 侍郎湯思退簽書樞密院事兼權參知政事。癸卯,以言者追譖岳飛,改嶽州為純州,岳陽軍為華容軍。是月,安南入貢。



 秋七月丙辰,減四川絹估、稅斛、鹽酒等錢歲百六十餘萬緡,蠲州縣積欠二百九十餘萬緡。詔四川營田有占民田者,常平司按驗給還。甲戌,封李天祚為南平王。



 八月丁丑,申嚴誣告加等法。辛巳,命大理鞫趙汾及令衿交通獄。丙戌,以吏部侍郎董德元參知政事。蠲諸路身丁、免丁錢一年。壬辰,建執政府。



 九月丁巳,秦檜
 上《紹興寬恤詔令》。



 冬十月庚辰,復置鴻臚寺。壬午,遣王岷使金賀正旦,鄭柟賀金主生辰。乙酉,命大理鞫張祁附麗胡寅獄。乙未,幸秦檜第問疾。夜,檜諷右司員外郎林一飛、臺諫徐嘉張扶等請拜熹為相。丙申,進封檜建康郡王,熹為少師,並致仕。命湯思退兼權參知政事。是夕,檜薨。丁酉,檜姻黨戶部侍郎兼知臨安府曹泳停官、新州安置。朱敦儒、薛仲邕、王彥傅、杜思旦皆罷。命有司具上執政、侍從官居外任及主宮觀與在謫籍者職位、
 姓名。辛丑,徙殿中侍御史徐哲、右正言張扶皆出為他官。



 十一月乙巳朔,追封檜申王,謚忠獻,賜神道碑,額為「決策元功,精忠全德」。戊申,奪趙汾二官。壬子,以敷文閣直學士魏良臣參知政事。癸亥,合祀天地於圜丘,大赦。甲子,幸秦檜第臨奠。乙丑,復洪皓官,釋張祁獄。丁卯,罷大理寺官旬白。庚午,詔監司、郡守,事無鉅細,皆須奏聞裁決,毋得止上尚書省。臣僚薦舉人才,必三人以上同薦。封叔和州防禦使、右監門衛大將軍士俴為崇慶軍
 節度使、嗣濮王,福建路提刑令詪為利州觀察使、安定郡王。辛未,知建康府王會及列郡守臣王晌、王鑄、鄭僑年、鄭震、方滋俱以諂附貪冒罷。真臘、羅斛國貢馴象。十二月甲戌朔,詔曰:「臺諫風憲之地,比用非其人,黨於大臣,濟其喜怒,殊非耳目之寄。朕今親除公正之士,以革前弊。繼此者宜盡心乃職,毋合黨締交,敗亂成法,當謹茲戒,毋自貽咎。」詔張浚、折彥質、萬俟離、段拂聽自便。量移李光郴州安置。乙亥,復以離為資政殿學士,提舉萬
 壽觀兼侍讀。戊寅,鄭億年責建武軍節度副使、南安軍安置。壬午,詔監司、守臣禁羨餘,罷權攝,戢苞苴,節宴飲。詔前後告訐者莫汲、汪召錫、陸升之等九人除名,廣南諸州編管。甲申,召孟忠厚奉朝請。命胡寅、張九成等二十八人並令自便,仍復其官。乙酉,董德元罷。丙戌,以劉錡知潭州。辛卯,命三省、六部條具續降敕旨來上,審詳施行。甲午,以敷文閣待制沉該參知政事。乙未,以王會恃權貪橫,停官、循州編管。丙申,復以蕭振為四川制置
 使。復張浚、折彥質、趙汾、葉三省、王趯、劉岑官。移胡銓衡州。丁酉,禁閩、浙、川、廣貢真珠、文犀。戒州縣加收耗糧。己亥,金遣耶律歸一等來賀明年正旦。



 二十六年春正月壬子,省諸州稅場,以寬商賈。甲子,追復趙鼎、孫近、鄭剛中、汪藻舊職。乙丑,詔選擇監司,須七品以上清望官,或經朝擢及治郡著績者。丙寅,曹泳吉陽軍編管。封伯令衿明州觀察使、安定郡王,以其從弟令詪讓也。戊辰,除民事律。蠲諸路積負及黃河竹索錢。



 二月乙亥,命四川州縣,凡預借民賦稅分限理析。己卯,定諸州流寓士人解額。庚辰,罷進奏院定本朝報。乙酉,進士林東追諂秦檜,上書狂妄,英州編管。右朝奉郎林一飛坐指使林東,責監高州鹽稅。庚寅,三佛齊國入貢。辛卯,魏良臣罷。庚子,以左朝散大夫王曮為秦檜親黨,直徽猷閣呂願中貪虐附檜,曮建昌軍居住,願中責果州團練副使、封州安置。



 三月甲寅,以邊事已定,罷宰相兼領樞密使。丁巳,詔兩淮邊民未復業者,復其租十年。
 己未,以萬俟離參知政事。癸亥,加吳璘開府儀同三司。乙丑,以東平府進士梁勛伏闕上書言北事,送千里外州軍編管。丙寅,詔曰:「講和之策,斷自朕志,秦檜但能贊朕而已,豈以其存亡而渝定議耶?近者無知之輩,鼓倡浮言,以惑眾聽,至有偽撰詔命,召用舊臣,抗章公車,妄議邊事,朕甚駭之。自今有此,當重置典憲。」丁卯,蠲閩、浙諸州歲供軍器所物料三之一,減諸州工匠千人。己巳,募四川民佃淮南、京西閑田,並邊復租稅十年,次邊五
 年。



 夏四月戊子,增溫、臺等十六州解額。命湖北路以增戶、墾田為守令殿最。庚寅,遣陳誠之等賀金主尊號禮成。癸巳,置武學官及弟子員百人。甲午,禁州郡進祥瑞。戊戌,立六科以舉士。加韋謙太尉。詔大闢情犯無可矜憫者,禁刑、寺妄引例奏裁貸減。罷鄉飲酒舉士法。詔淮南、京西占射官田逾二年未盡墾者,募人更佃。



 五月壬寅,以沈該為尚書左僕射,萬俟離為右僕射,並同中書門下平章事。湯思退知樞密院事。丁未,詔州軍教授毋
 兼他職。丙辰,蠲楚州、盱眙軍民租十年。己未,金遣敬嗣暉等來賀天申節。



 六月辛未朔,罷諸路鬻戶絕田。丁丑,以端明殿學士程克俊參知政事。戊寅,復權要親族中第覆試法。乙酉,詔取士毋拘程頤、王安石一家之說。丁亥,流星晝隕。辛卯,以秦檜既死,命史館重修日歷。



 秋七月辛丑,詔三衙主帥舉武臣堪知州者。壬寅,蠲諸路丁絹一年為二十四萬匹。丙午,右奉議郎薛仲邕連州編管。丁未,彗出井,避殿減膳。辛亥,詔諸州守貳考各縣丁
 籍,依年格收除。民間市物,官戶、勢家與編氓均科。丙辰,彗滅。詔進士因事送諸州軍聽讀,特放逐便,仍許取應。辛酉,雨水銀。



 八月戊寅,班元豐、崇寧學制於諸路。革正前舉登第奏塤、曹冠等九人出身,以淮南提舉常平朱冠卿言,秦檜挾私廢法,塤等皆其子孫、親戚、門下憸人,於是有官應試者,所授階官易左為右,白身者駁放。占用省額,復還後科。庚辰,裁州縣吏額。己丑,蠲建康府積欠內帑錢帛。庚寅,安南國遣使入貢。辛卯,程克俊罷。甲
 子,以吏部侍郎張綱參知政事。



 九月乙巳,以翰林學士陳誠之同知樞密院事。丙午,立互易薦舉坐罪法。壬子,詔成都、潼川兩路漕臣同制置、總領、茶馬司審度四川財賦利害,其實惠得以及民、調度可以經久者,條具以聞。甲寅,以天聖、紹興真決贓吏指揮班示諸路。丙寅,增大理寺吏祿。戊辰,命吏、刑二部修條例為成法。



 冬十月己巳朔,詔許秦檜在位之日,無辜被罪者自陳厘正。罷浙東常平司平準務。乙亥,詔四川監司、帥臣、制置、總領、
 茶馬司,各舉可守郡者。甲午,蠲郴、道、永三州、桂陽軍民身丁米。乙未,王會移瓊州編管。以宋貺黨附秦檜,責梅州安置。丁酉,以張浚上書論用兵,依舊永州居住。辛丑。遣李琳使金賀正旦,葛立方賀金主生辰。



 閏月丙午,罷廉州貢珠,縱蛋丁自便。己酉,命離軍人願歸農者,人給江、淮、湖、廣荒田百畝,復其租稅十年。乙卯,初置臨安府左、右廂官,分掌訟牒。



 十一月甲戌,命吏部侍郎陳康伯、戶部侍郎王俁稽考國用歲中出納之數。丙戌,裁定六
 曹、寺監百司吏額。十二月辛丑,命三省錄臺諫所言事報樞密院。癸丑,萬俟離上《重修貢舉敕令格式》。甲寅,罷諸路鑄錢司。庚申,賞應詔論事切當者。壬戌,三佛齊國入貢。甲子,金遣梁□求等來加明年正旦。



 二十七年春正月乙酉,幸延祥觀。戊子,命侍從各薦宗室京朝官才識、治行者二人。



 二月丁酉朔,復兼習經義、詩賦法。庚子,楊政卒。壬寅,太廟仁宗、英宗兩室柱芝草生。戊午,以御史中丞湯鵬舉參知政事。庚申,更定福建
 路鹽法。癸亥,加劉錡太尉。



 三月己巳,命京局改官人先除知縣。乙酉,赤氣出紫微垣。丙戌,賜禮部進士王十朋以下四百二十六人及第、出身。丁亥,詔焚交址所貢翠羽於通衢,仍禁宮人服用銷金翠羽。己丑,減三川對糴米歲十六萬九千石,夔路激賞絹五萬匹,兩川絹估錢二十八萬緡及茶司引息虛額錢歲九十五萬緡。辛卯,萬俟離卒。壬辰,以符行中前在蜀恣橫,南雄州安置。甲午,除耕牛稅。



 五月癸未,金遣耶律守素等來賀天申節。
 辛卯,復以五帝、神州地祇等十三祭為大祀。



 六月甲辰,命臣僚轉對,盡忠開陳,毋摭細微以應故事。戊申,以湯思退為尚書右僕射、同中中書門下平章事。庚戌,復餘深、黃潛善並觀文殿大學士。乙卯,裁定離軍將士諸州添差數。戊午,初命太廟冬饗祭功臣,臘饗祭七祀,祫饗兼之。己未,進錢忱少傅。增命官捕獲私茶鹽賞典。



 秋七月己巳,復饒、贛、韶三州鑄錢監。癸酉,戒監司、郡守舉劾守令觀望徇私。乙亥,以龍圖閣學士李文會為四川安撫
 制置使。丙子,詔凡出命,令先經兩省書讀,如舊制。



 八月乙未,以湯鵬舉知樞密院事。庚申,復置提領諸路鑄錢司於行在,以戶部侍郎榮薿領之。



 九月癸酉,張綱罷。戊寅,以吏部尚書陳康伯參知政事。蠲淮南、京西、湖北積欠內藏錢帛。丁亥,校書郎葉謙亨言:「祀典散逸,隆殺不當,名稱或舛,請敕禮官、秘書酌景德故事,取祭祀之式,定為一書,名曰《紹興正祠錄》,以為恆制。」詔從之。



 冬十月壬寅,有赤氣隨日入。癸卯,築通、泰、楚三州捍海堰。辛酉、
 詔四川諸司察旱傷州縣,捐其稅,振其饑民。



 十一月癸亥朔,減福建鹽鈔錢歲八萬緡。乙丑,遣孫道夫使金賀正旦。辛巳,遣劉章賀金主生辰,丁亥,湯鵬舉罷。戊子,蠲廬州二稅及上供錢米一年。十二月甲午,詔廣南經略、市舶司察蕃商假托入貢。丙辰,初命州縣置禁歷。戊午,金遣高思廉等來賀明年正旦。



 二十八年春正月己巳,申禁三衙強刺平民為兵。己卯,幸延祥觀,遂幸玉津園。壬午,禁諸路二稅折納增價。癸
 未,遣戶部郎中莫蒙等檢視淮南、浙西、江東沙田蘆場。甲申,命臺諫、侍從三人以上公薦監司治狀。



 二月癸巳,命史館重修徽宗大觀以前實錄。丙申,以陳誠之知樞密院。戊戌,禁沿海州軍博買。乙巳,以工部侍郎王綸同知樞密院事。乙酉,命六曹長貳詳定差役舊法。癸丑,加楊存中少師,謚張俊曰忠烈。



 三月辛酉朔,日有食之。丙寅,雪。丁丑,加田師中開府儀同三司。戊寅,詔:「自今用人,選帥臣、監司曾任郎官已上者為侍從,監司、郡守有政
 績者為卿監、郎官,朝官二年乃遷,卿監、郎官未歷監司者更迭補外。」戊子,責秦檜黨宋樸徽州居住,沉虛中筠州居住。



 夏四月丙申,復詔文武官非犯贓罪,並許以致仕恩任子。辛亥,雨雹。嚴州遂安賊江大明寇衢州,官軍捕斬之。



 五月,金遣蕭恭等來賀天申節。



 六月壬辰,太白晝見。癸巳,流星晝隕。甲寅,增浙西、江東、淮東沙田蘆場租課,置提領官田所掌之。



 秋七月庚申,立江西上供米綱賞格。戊辰,詔:「監司按發官吏,不得送置司州軍推鞫。
 所犯涉重,即以奏聞,命鄰路監司選官就鞫。」己卯,命取公私銅器悉付鑄錢司,民間不輸者罪之。庚辰,親制郊廟樂章。乙酉,復鬻沒官田。



 八月戊子朔,置國史院,修神、哲、徽三朝正史。己丑,檢放風水災傷州縣苗稅,仍振貸饑民。乙未,增四川十七州舉人解額。戊戌,湯思退等上《徽宗實錄》。壬寅,命戶部侍郎令詪提領諸路鑄錢。甲寅,地震。



 九月辛未,定銅錢出界罪賞。甲戌,詔以吏部七司舊制與續降參訂異同,立為定法。丁丑,置殿前司虎翼
 水軍千人。庚辰,以中書舍人王剛中為四川安撫制置使。辛巳,封叔建州觀察使士輵為昭化軍節度使、嗣濮王。癸未,蠲平江、紹興、湖州被水民逋賦。



 冬十月丁亥朔,遣沈介使金賀正旦,黃中賀金主生辰。辛丑,禁監司、帥、守私役軍匠。



 十一月己卯,合祀天地於圜丘,大赦。壬午,復命檢舉諸人因赦移放者,告訐得罪者不預。十二月庚寅,安定郡王令衿薨。辛丑,修睦親宅,建宮學。丁未,復李光官,放自便。戊申,蠲楚州歸附民賦役五年。壬子,金
 遣蘇保衡等來賀明年正旦。是歲,興元都統制姚仲復籍興元府等五州義士,得二萬餘人。



 二十九年春正月丙辰朔,以皇太后年八十,詣慈寧殿行慶壽禮。庚申,浚平江三十六浦以洩水。庚午,振湖、秀諸州饑民。癸酉,幸延祥觀,遂幸玉津園。庚辰,禁諸州科賣倉鹽。癸未,蠲沙田蘆場為風水所侵者租之半。是月,金國罷沿邊榷場,惟泗州如舊。



 二月丙戌朔,亦罷沿邊榷場,存其在盱眙者。加吳璘少保。己丑,禁海商假托風
 潮私往北界。壬辰,除臨安府歲供修內司錢三萬六千緡。丁酉,蠲四川折估糴本積欠錢三百四十萬緡。戊戌,大雪,雨雹。己亥,禁貿易廣南羈縻州物貨。命廣西教閱峒丁。庚戌,罷諸路斥侯遞卒。甲寅,取具貶死臣僚姓名,議加恩典。



 三月丙子,除州縣積欠錢三百九十七萬緡有奇,及中下戶所欠入宮錢物。丁丑,詔侍從、臺諫、帥臣、監司歲舉可任將帥者二人。限命官子孫制田減父祖之半,並其詭名寄產者,格外田畝同編戶科役。己卯,除
 湖州、平江、紹興流民公私逋負。



 夏四月壬辰,國子司業黃中自金國使還,言金人將徙居汴京以見逼,望早飭邊備。宰相怒,不聽。己亥,修三省法。庚子,增置帶御器械四員。丙午,禁內外將佐營造、回易,掊斂軍士。辛亥,命縣令有政績者諸司同薦,不次升擢,以風厲之。



 五月甲寅朔,罷鬻福建閃生沙田。丁巳,詔殿前司選統制官部兵千人戍江州,彈壓盜賊,每歲一易。己未,樁頓江、浙四路折帛錢於三總領所及浙西提刑司,以備軍用。辛酉,禁
 權要、豪民舉錢軍中取息。丁卯,命印給三總領所見錢公據、關子,許商人入納。己巳,立監司、守臣舉劾八條。金遣王可道等來賀天申節。



 六月甲辰朔,遣王綸等為金國奉表稱謝使。丁亥,禁江、淮私渡北人。丙申,陳誠之罷。禁積錢民戶過萬緡,官戶過二萬緡,滿二年不易他物者沒入之。丁酉,申禁包苴請托。己亥,以陳康伯兼權樞密院事。辛丑,李光卒。壬寅,以主管步軍司趙密為太尉。己酉,沉該以貪冒罷。



 閏月甲寅,益荊南戍卒千人,守臣
 劉錡亦募效用三千人。丁巳,命江、湖、浙西五漕司增價糴米二百二十萬石赴沿江十郡,自荊至常州,以備振貸。戊午,罷成都府路隔槽酒務監官七十一員,令民承買。己未,罷江、浙、淮東沙田蘆場所增租課。甲子,落沈該觀文殿大學士,致仕。罷福建安撫司官賣鹽。戊辰,大省淮西冗官。辛未,復置江、淮、荊、浙、福建、廣南路提點坑冶鑄錢官。



 秋七月丁亥,以權吏部尚書賀允中參知政事。癸巳,封權戶部侍郎令詪為安定郡王。戊戌,福州大水。
 己酉,禁諸路抑買官田。庚戌,以四川經、總制及田晟錢糧錢共百三十四萬緡充增招軍校費。



 八月甲子,募商人輸米行在諸倉,願以茶、鹽、礬鈔等償直者聽。丁卯,除南雄、英、連三州經界,復丁米舊額。甲戌,並史館歸秘書省,玉牒所歸宗正寺。



 九月甲申,詔建炎以來使未還而後嗣無祿者,與一子官。乙酉,王綸使還入見,言金國和好無他。丙戌,湯思退等稱賀。甲午,以湯思退為尚書左僕射,陳康伯為右僕射,並同中書門下平章事。乙未,以
 皇太后不豫,大赦,不視朝。丙申,為太后祈福。蠲中下戶所欠稅賦及江、浙蝗潦州縣租。丁酉,減僧道免丁錢。己亥,蠲見監贓罰賞錢。庚子,皇太后韋氏崩。癸卯,遣周麟之等為金國奉表哀謝使。



 冬十月甲寅,以群臣五上表,始聽政。命保康軍節度使吳益為攢宮總護使。乙亥,立諸路和糴募民妄運米賞格。戊寅,冊謚皇太后曰顯仁。



 十一月丁亥,遣賀允中等為金國遺留國信使。丙午,權攢顯仁皇后於永祐陵。十二月甲寅,諜言北界禁民傳
 起兵,帝諭大臣常自治,為安邊息民之計。甲子,祔顯仁皇后神主於太廟。辛未,以王綸知樞密院事。壬申,減三省、樞密院激賞庫及諸書局歲用錢二十萬緡,鼎州程昌寓所增蔡州官兵衣糧錢四之一,西和州官賣鹽直之半,蔣州上供經、總制司無額錢如之。丙子,金遣施宜生等來賀明年正旦。



 三十年春正月戊子,給劉錡軍費錢六十萬緡。丙申,以吏部侍郎葉義問同知樞密院事。廢御書院。丁酉,罷鈞
 容班樂工及甲庫酒局。壬寅,募人墾淮南荒田。甲辰,定御輦院三營兵額為九百人。



 二月甲寅,罷夔州路榷茶。乙卯,金遣大懷忠等來吊祭。戊午,遣葉義問為金國報謝使。癸酉,詔立普安郡王瑗為皇子,更名瑋。丙子,進封建王。



 三月辛巳,復館職召試,然後除擢。免湖北、京西宣撫司諸庫未輸錢八十九萬緡。癸未,以淮東茶鹽司錢十萬緡充募民墾田費。乙酉,加吳益少保,趙密開府儀同三司,以賞攢宮之勞。丁酉,初置金州御前諸軍都統
 制,以知金州王彥為之。癸卯,賜禮部進士梁克家以下四百一十二人及第、出身。甲辰,置牧馬監于潮、惠二州。丙午,加恩平郡王璩開府儀同三司、判大宗正事,始稱皇侄。夏四月己酉朔,以孫□為蘄州防禦使,愷貴州團練使,惇榮州刺史。丙辰,以賀允中兼權同知樞密院事。



 五月辛巳,刺海賊罪不至死者為龍猛、龍騎軍。初置荊南府御前諸軍都統制,以劉錡兼領之。乙酉,初置江州御前諸軍都統制,以步軍司前軍都統制戚方為之。詔
 諸路刺強盜貸死少壯者為兵。丙戌,定鑄錢司歲鑄五十萬緡。辛卯,臨安、於潛、安吉三縣大水。海賊陳演添作亂,掠高、雷二州境上,南恩州民林觀禽殺之,命觀以官。丙申,金遣蕭榮等來賀天申節。壬寅,落沉該致仕,復觀文殿大學士、知明州。丙午,加吳益太尉。



 六月庚戌,復出諸軍見錢關子三百萬緡,聽商賈以錢銀請買。庚午,王倫罷。辛未,以江西廣東湖南折帛、經總制錢合六十萬緡,江西米六萬石充江州軍費。後益以四川利路經總
 制、江西茶引合二十萬緡。



 秋七月戊寅,遣明州水軍三百戍昆山黃魚垛,巡捕槽船之為盜者。甲申,詔諸路帥司,春秋教閱禁兵弓弩手。戊戌,以葉義問知樞密院,翰林學士周麟之同知院事,御史中丞朱倬參知政事。



 八月丙午朔,日有食之。壬子,賀允中使還,言金人必叛盟,宜為之備。癸丑,允中致仕。甲寅,復以四川經、總制錢五十萬緡給總領所,增招兵士。壬申,淮東總管許世安奏,金主亮至汴京,起重兵五十餘萬,屯宿、泗州,謀來攻。



 九
 月庚寅,以帶御器械李寶為浙西副總管,提督海船,駐平江。丙申,命劉寶招制勝軍千人。丁酉,罷內侍省。



 冬十月丙午,罷內侍官承受諸軍奏報文字。丁未,遣虞允文使金賀正旦,徐度賀金主生辰。庚戌,雷。辛酉,鎮江都統制劉寶以專悍貪橫罷。壬戌,以劉錡為鎮江都統制,荊南右軍統制李道為都統制。癸亥,日中無雲而雷。癸酉,蠲舒、和、蘄、黃四州民附種田租。



 十一月庚辰,禁諸路折輸職田錢。癸已夜,有白氣出入危、昴間。十二月乙巳朔,
 湯思退罷。初行會子於東南。戊申夜,白氣亙天。海南黎賊王文滿平。己酉,罷招刺三衙及江上諸軍。庚戌,禁掠賣生口入溪峒。癸丑,命戶部立經、總制錢十年中數為定額。丁卯,金遣僕散權等來賀明年正旦。



\end{pinyinscope}