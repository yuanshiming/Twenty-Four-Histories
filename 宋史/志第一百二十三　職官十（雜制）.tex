\article{志第一百二十三 職官十(雜制)}

\begin{pinyinscope}

 贊
 引導
 從賜食邑實封使職宮觀贈官敘封致仕蔭補



 贊引



 舊中書門下、翰林學士、御史中丞並緋衣雙引,仍傳呼。
 開寶中,學士止令一吏前導,亦罷傳呼,惟謝恩初上日,雙引傳呼云。



 使相、僕射、兩省五品已上,一吏前引。樞密使兼相者,二吏,不贊引。大中祥符五年,止令於本廳贊引。不帶相及副使,止令本院紫衣吏前贊引之。



 淳化四年,令東宮三少、尚書、丞、郎入朝以緋衣吏前導,並通官呵止。二品已上用朝堂驅使官,餘用本司驅使官,宰臣、親王仍令紫衣二吏引馬。



 導從



 中書、樞密、宣微院、御史臺、開封府、金吾司皆有常從。景德三年詔:「諸行尚書、文明殿學士、資政殿大學士,給從
 七人;學士、丞郎,六人;給事、諫議、舍人,五人;諸司三品,四人。於開封府、金吾司差借,每季代之。」中書先差金吾從人,自今亦令參用開封府散從官。宰臣、參知政事、僕射、御史大夫、中丞、知雜,皆通官呵止行人。淳化四年,令東宮三少、尚書丞、郎,並通官呵止。



 大中祥符五年,以群官導從不合品式,命翰林學士李宗諤、龍圖直學士陳彭年與禮官詳定。宗諤等請:自今除中書、樞密、宣徽使、御史中丞、知雜御史、金吾並攝事清道如舊制呵導外,僕射已上及三司使、知開
 封府,止四節;尚書、文明殿學士、資政殿大學士,三節;丞郎已上、三司副使,兩節;大兩省、卿、監,一節;小兩制御史、郎中、員外、諸司四品,三司、開封府判官推官,二人前行引,不得過五步。合於金吾借從人者,以諸軍剩員代之。又外任節鎮知州、都監,從軍士七十人;通判,十五人;防、團、軍事知州都監,五十人;通判,十人;河北、河東、陜西見泊兵處,第鎮知州、都監百人,防、團、軍事知州都監七十人。



 轉運使,三十人;咸平二年,詔節度、觀察、防、團、刺史,或別鎮、他鎮,其給使者,止令本使給之。景德六年,令牧守以州兵隨行者以一年為限。



 副使,二十五人;
 提點刑獄官,亦給軍士;副留守、節度行軍副使、留守兩使判官,給散從官十五人;小尹、掌書記、支使、防禦、團練副使、兩使推官,十人;兩浙推官、防團軍事判官推官、軍監判官,七人;錄事諸曹,給承符人;縣令、簿、尉、手力、弓手,其代還者,給人護送有差。



 賜六



 劍履上殿詔書不名贊拜不名入朝不趨紫金魚袋緋魚袋



 右升朝官該恩。著綠二十周年賜緋魚袋,著緋及二十周年賜紫金魚袋。特旨者,系臨時指揮。



 食邑



 一萬戶八千戶七千戶六千戶五千戶四千戶三千戶二千戶一千戶七百戶五百戶四百戶三百戶二百戶



 右宰相、親王、樞密使經恩加一千戶,兩府、使相、節
 度使七百戶。宣徽、三司使,觀文殿大學士以下至直學士,文臣侍郎、武臣觀察使、宗室正任以上、皇子上將軍、駙馬都尉加五百戶。宗室大將軍以上加四百戶。知制誥、待制並文臣少卿監、武臣諸司副使、宗室副率已上,並承制、崇班、軍員等,初該恩加三百戶;承制、崇班、軍員再該恩二百戶。二千戶以上雖有加例,緣無定法,親王、重臣特加有至萬戶者。



 食實封



 一千戶八百戶五百戶四百戶三百戶二百戶一百戶



 右宰臣、親王、樞密使經恩加四百戶。兩府、使相、節度、宣徽使、皇子上將軍,並宗室駙馬都尉任觀察使已上加三百戶。觀文殿學士並宗室正任已上,騎都尉加二百戶。武臣崇班、宗室副率已上加一百戶。五百戶已上雖有加例,緣無定法。親王、重臣
 有特加至數千戶者。



 《三朝志》云:檢校、兼、試官之制,檢校則三師、三公、僕射、尚書、散騎常侍、賓客、祭酒、卿、監、諸行郎中、員外郎之類,兼官則御史大夫、中丞,侍御、殿中、監察御史,試秩則大理司直、評事、秘書省校書郎。凡武官內職、軍職及刺史已上,皆有檢校官、兼官。內殿崇班初授檢校祭酒兼御史大夫。三班及吏職、蕃官、諸軍副都頭加恩,初授檢校太子賓客兼監察御史,自此累加焉。廂軍都指揮使止於司徒,軍都指揮使、忠
 佐馬步都頭止於司空,親軍都虞候、忠佐副都頭以上止於僕射,諸軍指揮使止於吏部尚書。其官止,若遇恩例,則或加階、爵、功臣。幕職初授則試校書郎,再任如至兩使推官,則試大理評事。掌書記、支使、防禦團練判官以上試大理司直、評事,又加則兼監察御史,亦有至檢校員外郎已上者。行軍副使皆檢校員外已上。朝官階、勛高,遇恩亦有加檢校官,郎中則卿、監、少監,員外郎則郎中,太常博士以下則員外郎,並無兼官。其解褐評事、校書郎、正字、寺監主簿、助教者,謂之試銜。有選集,同出身例。



 使職



 兼領者:親祀南郊,則有大禮、禮儀、儀仗、鹵簿、橋道頓遞五使,藉田、泰山封禪、汾陰奉祀、恭上寶冊、南郊恭謝皆如之。自餘行禮,或止有大禮、禮儀使。建隆中南郊,置儀仗都部署、副都署。



 經始大禮,則有經度制置使、副。巡幸,有行宮都部署,行宮有三司使、副使、判官、行宮使、都監。舊,南郊止有御營使,咸平中,置行宮使。又有車駕前後、行宮四面、闌前收后、郊壇巡檢巡闌儀仗勾當,編排鹵簿。其百司皆有行在之名。舊巡幸,百司皆稱隨駕。大中祥符初,並同行在某司。



 京師居留,則有大內都部署、皇城
 都點檢、巡檢及增新舊巡檢。大閱亦置。



 征行,則有招討使、招安使、或云捉賊、招安、安撫使名者。



 排陳使、都監,前軍、先鋒、大陳、行營、壕砦、頭車、洞子、招收部署、鈐轄、都監,策應之名。又有拐子馬、無地名馬,選武幹者別領之。親征,則冠以駕前之號。廉訪民瘼,則有巡撫大使、副大使,字撫使、副使、都監,採訪使、副使。或官卑者止雲巡撫、安撫,無使字。



 加禮外國,則有國信、接伴、送伴使副;吊祭,大帥若是;又有翻譯經潤文使,宰相為使,以翰林學士為潤文官。



 伸達冤濫,則有理檢使。勸課農桑,則有勸農使。講修馬政,則有
 群牧制置使。最後明堂祫



 饗,置五使,如南郊。其一時特置者,則各具志傳。或臨事更制才者,事畢即停。內外名務繁細者,猶不具載。



 敘階之法開府儀同三司至將仕郎為文散官,驃騎大將軍至陪戎副尉為武散官。太平興國元年,改正議大夫為正奉,通議大夫為朝奉,朝議郎為朝奉,承議郎為承直,奉議郎為奉直,宣義郎為通直。



 京朝官、幕職自將仕郎至朝奉郎,每加五階;至朝散大夫已上,每加一階。朝散、銀青者須已服緋紫者。入令錄、判司簿尉,每加一階;
 並幕職計考當服緋紫者,皆奏加朝散、銀青階。諸司使已上,如使額高者加金紫階。內殿崇班初授則銀青階。三班軍職、使職遇恩檢校,兼官,並除銀青階。



 丁尤者起復,使相則授雲麾將軍,使相仍加金吾上將軍,同正節度使,大將軍同正留後,以下無之。



 其胥吏掌事而至衣緋者,則授游擊將軍,千牛備身則授陪戎副尉以上。



 改賜功臣勛官,自上柱國至武騎尉。五代以來,初敘勛官,即授柱國。淳化無年詔:「自今京官、幕職州縣官始武騎尉,朝官始騎都尉,三班及軍員、吏職經恩並授武騎
 尉。」又詔:「古之勛爵,悉有職奉之蔭贖,宜以今之所授與散官等,不得用以蔭勛。」封爵之差,唐制:王,食邑五千戶;郡王、國公,三千戶;開國郡公,二千戶;縣公,千五百戶;縣侯,千戶伯,七百戶;子,五百戶;男,三百戶。又有食實封者,戶給縑帛,每賜爵,遞加一級。唐末及五代始有加邑特戶,而罷去實封之給,又去縣公之名,封侯以郡。宋初沿其制,文臣少監、少卿以上,武臣副率以上,內職崇班以上有封爵;丞、郎、學士、刺史、大將軍、諸司使以上有實封。
 但以增戶數為差,不系爵級。邑過其爵,則並進爵焉,止於郡公。每加食邑,自千戶至二百戶,實封自六百戶至百戶。親王、重臣或特加,有逾千戶者。郡公食邑有累加至萬餘,實封至數千戶者。



 皇屬特封郡公、縣公或贈侯者,無「開國」字。侯亦在開國郡公之上。



 又採秦制賜爵曰「公士」。端拱二年,賜諸州高年一百二十七人爵公士,景德中,福建民有擒獲強盜者,當授鎮將,以遠俗非所樂,並賜公士,自後率為例。



 功臣者,唐開元間賜號「開元功臣」,代宗時有「寶應功臣」,德宗時有「奉天定難元從功臣」之號,僖宗將相多加功臣美名,五代浸
 增其制。宋初因之,凡宣制而授者,多賜焉。參知政事、樞密副使、刺史以上階、勛高者亦賜之。中書、樞密則「推忠」、協謀,親王則「崇仁、「佐運」,餘官則「推誠」,「保德」、「翊戴」,掌兵則「忠果」、「雄勇」、「宣力」,外臣則「純誠」、「順化」。宰相初加即六字,餘並四安,其累加則二字,中書、樞密所賜,若罷免或出鎮,則改之。其諸班直將士禁軍,則賜「拱衛」、「翊衛」等號,遇恩累加,但改其名,不過兩字。



 宮觀



 宋制,設祠祿之官,以佚老優賢。先時員數絕少,熙寧以後乃增置焉。在京宮觀,舊制以宰相、執政充使,或丞、郎、學士以上充副使,兩省或五品以上為判官,內侍官或諸司使、副政和改武臣官制,以使為大夫,以副使為郎。



 為都監,又有提舉、提點、主管。其戚里、近屬及前宰執留京師者,多除宮觀,以示優禮。時朝廷方經理時政,患疲老不任事者廢職,欲悉罷之。乃使任宮觀,以食其祿。王安石亦欲以此處異議者,遂詔:「宮觀毋限員。並差知州資序人。以三十月為
 任。」又詔:「杭州洞霄宮、毫州明道宮、華州雲臺觀、建州武夷觀、臺州崇道觀、成都玉局觀、建昌軍仙都觀、江州太平觀、洪州玉隆觀、五嶽廟自今並依嵩山崇福宮、舒州靈仙觀置管幹或提舉、提點官。」「奉給,大兩省、卿、監及職司資序人視小郡知州,知州資序人視小郡通判,武臣仿此。」四年,詔:「宮觀、嶽廟留官一員,余聽如分司、致仕例,人便居住。」六年,詔:「卿、監、職司以上提舉,餘官管幹。」又有以京官為乾當者。又詔:「年六十以上者乃聽差,毋過兩
 任。又詔:「兼用執政恩例者,通不得過三任。」



 元豐中,王安石以左僕射、觀文殿大學士為集禧觀使,呂公著、韓維以資政殿學士兼侍讀、仍提舉中太一宮兼集禧觀公事。元祐間,馮京以觀文殿學士、梁燾以資政殿學士為中太一宮、醴泉觀使。範鎮落致仕,以端明殿學士提舉中太一宮兼集禧觀公事。三年,詔:「橫行使、副無兼領者,許兼宮觀一處。」」六年,詔:「橫行狄諮、宋球既領皇城司,罷提點醴泉觀。」元符元年,高遵固年八十一,乞再任宮觀,
 高遵禮年七十六,乞再任毫州太清宮,又從其再任之請,以待遇宣仁親屬故也。大觀元年,趙挺之以觀文殿大學士為祐神觀使。政和六年詔。「措置宮觀,如萬壽、醴泉近百員,更不立額。」靖康元年,詔內外官見帶提舉、主管神霄玉清萬壽宮並罷。大抵祠館之設,均為佚老優賢,而有內外之別,京祠以前宰相、見任使相充使,次充提舉;餘則為提點,為主管,皆隨官之高下,處以外祠。選人為監嶽廟,非自陳而朝廷特差者,如黜降之例。



 紹興
 以來,士大夫多流離,困厄之餘,未有闕以處之。於是許以承務郎以上權差宮觀一次,續又有選人在部無闕可入與破格嶽廟者,亦有以宰執恩例陳乞而與之者,月破供給。非責降官並月破供給,依資序降二等支。



 理為資任,意至厚也。然初將以撫安不調之人,末乃重僥求泛與之弊。於是臣僚交章,欲罷供給以絕干請,變理任以抑僥幸,嚴按格以去泛濫。上並從之。自是以後,稍復祖宗條法之舊。又有年及七十,耄昏不堪牧養而不肯自陳宮觀者,復申
 明舊法,著為定令以律之。舊制,六十以上知州資序人,本部長官體量精神不致昏昧堪厘務者,許差一任,兼用執政官陳乞者加一任。紹興二十二年,臣僚言:「郡守之職,其任至重,昨朝廷以年及七十,令吏部與自陳宮觀,乞將前項指揮永為著令。」從之。



 蓋不當請而請,則冗瑣者流競竊優閑廩稍;或當請而不請,則知進而不知退,識者羞焉。一祠館之與奪,不可不謹如是。故重內祠,專使職,所以崇大臣之體貌,一次以定法,再任以示恩,紹熙五年慶壽赦,應文武臣宮觀、嶽廟已滿,不應再陳者,該今來慶壽恩,年八十以上,特許更陳一次。



 京官以上二年,選人三年,凡待庶僚者,皆於優厚之中寓閑
 制之意焉。



 贈官



 建隆已來,凡有恩例,文武朝官、諸司使副、禁軍及藩方馬步都指揮使以上,父亡皆贈官。親王贈三官,可贈者贈二官,追加大國。皇屬近親如之。追加封爵。服疏及諸親之服近者贈一官。宰相、樞密使贈二官。使相、參知政事、樞密副使、尚書已上、三司使、節度使、留後、觀察使、統軍上將軍、內臣任都知副都知者,贈一官。此皇族及臣
 僚薨卒贈官之法也。其官秩未至,而因勛舊褒錄或沒王事,雖卑秩皆贈官加等者,並系臨時取旨。至於母后、後族、臣僚,錄其先世,各有等差。太皇太后、皇太后、皇后並贈三世,婕妤二世,貴人止贈其父而已。宰相、本師、三公、王、尚書令、中書令、侍中、樞密使副、知院、同知院事、參知政事、宣微使、度簽書同簽書樞密院事、觀文殿大學士、節度使,並贈三世。東宮三師、僕射、留守、節度使、三司使、觀文殿學士、資政殿大學士,並贈二世。餘官或見任,或
 致仕,並贈一世。有兄弟同贈者,贈官加一等,父在止一資,文臣有出身,贈至秘書監,無出身,至光祿卿。武臣至金吾衛上將軍止。



 凡贈官至三世者,初贈東宮三少,次陳宮三太,次三公,次中書令,次尚書令,次封小國,自小國升次國,自次國升大國,已大國者移國名而已。亦有不移者。若父、祖舊官已高者,自從舊官加贈。凡追封,不得至王爵。兩省官及待制、大卿監、諸衛上將軍、觀察使、正任防禦使、遙郡觀察使、景福殿使、客省使,若子見任或父
 曾任此官,並贈至三公止。父子官俱不至者,文臣贈至諸行尚書止,武臣贈至節度使、諸衛上將軍止,即父曾任中書、樞密使、使相、節度使並一品官者,無止限。待制已上持服經恩,服闋亦許封贈。



 尚藥奉御至醫官使曾任文資,許換南班官。司天監官贈不得過大卿、監,仍不許換南班官。凡贈至正郎,許以所贈官換朝散大夫階,大卿、監以上許換銀青階,贈至二世者即除朝散大夫階,三世則金紫階。咸平四年,詔舍人院詳定。知制誥李宗諤等請:「追贈三世如舊。其東宮
 一品以下雖曾任宰相,止從本品。文武群臣功隆位極者,特恩追封王爵亦如舊。若因子孫封贈,雖任將相,並不許封王,仍須歷品而贈,勿得超越。」從之。宰相初拜,有即贈三世者。其後簽書樞密以上皆實時贈,他官須經恩,學士及刺史以上,內侍都知、押班皆中書奉行,餘則有司奏請。



 敘封



 唐制,視本官階爵。建隆三年,詔定文武郡臣母妻封號:太皇太后皇太后皇后曾祖母、祖母、母並封國太夫人;諸妃曾祖母、祖母、母並封郡太夫人,婕妤祖母、母並封郡太君;貴人母封縣太君。宰相、使相、三師、三公、王、侍中、
 中書令,舊有尚書令。



 曾祖母、祖母、母封國太夫人;妻,國夫人。樞密使副、知院、同知、參知政事、宣徽節度使,曾祖母、祖母、母封郡太夫人;妻,郡夫人。簽書樞密院事曾祖母、祖母、母封郡太君;妻,郡君。同知樞密院以上至樞密使、參知政事再經恩及再除者,曾祖母、祖母、母加國太夫人。三司使祖母、母封郡太君妻,郡君。東宮三太、文武二品、御史大夫、六尚書、兩省侍郎、太常卿、留守、節度使、諸衛上將軍、嗣王、郡王、國公、郡公、縣公,母,郡太夫人;妻,郡夫
 人。常侍、賓客、中丞、左右丞、侍郎、翰林學士至龍圖閣直學士、給事中、諫議大夫、中書舍人、卿、監、祭酒、詹事、諸王傳、大將軍、都督、中都護、副都護、觀察留後、觀察使、防禦使、團練使,並母郡太君;妻,郡君。庶子、少卿監、司業、郎中、京府少尹、赤縣令、少詹事、諭德、將軍、刺史、下都督、下都護、家令、率更令、僕,母封縣太君;妻,縣君,其餘升朝官已上遇恩。並母封縣太君;妻,縣君,雜五品官至三任與敘封,官當敘封者不復論階爵。致仕同見任。亡母及亡祖
 母當封者並如之。父亡無嫡、繼母,聽封所生母。伎術官不得敘封。自宰相至簽書樞密院敘封與三世同,他官惟品至者實時擬封,餘皆俟恩乃封。咸平四年,從舍人院詳定群臣母、妻所封郡縣,依本姓望封。天禧元年,令文武升朝官無嫡母者聽封生母,曾任升朝而致仕,即許敘封。令給諫、舍人母並封郡太君,妻,郡君。四年,又令翰林學士至龍圖閣直學士如給、舍例。封贈之典,舊制有三代、二代、一代之等,因其官之高下而次第焉。凡初除及每遇大禮封贈三代者,太師、太傳、太保、左右丞相、少師、少傅、少保、樞密使、開
 府儀同三司、知樞密院事、參知政事、同知樞密院事、樞密副使、簽書樞密院事。凡遇大禮封贈三代者,節度使。三代初封,曾祖,朝奉郎;祖,朝散郎;父,朝請郎簽書樞密院事降一等,謂如父與朝散郎之類。凡封父、祖系武臣者,視文武臣封贈對換格。封贈一代亦如之。初贈,曾祖,太子少保;祖,太子少傅;父,太子少師。封贈曾祖母、祖母、母、妻國夫人。執政官、簽書樞密院事,郡夫人。



 凡遇大禮封贈二代者,太子太師、太子太傅、太子太保、特進、觀文殿大學士、太子少師、太子少傅、太子少保御史大
 夫、觀文殿學士、資政、保和殿大學士、金紫光祿大夫、銀青光祿大夫、光祿大夫、左右金吾衛上將軍、左右衛上將軍。二代初封,祖,通直郎,父,奉議郎。初贈,祖,朝奉郎;父,朝散郎。封贈祖母、母、妻郡夫人。觀文殿學士,資政,保和殿大學土,並淑人。



 凡遇大禮封贈一代者,文臣通直郎以上,武臣修武郎以上。一代初封贈父,文臣承事郎,武臣、內侍、伎術官、將校並忠訓郎,母、妻孺人。



 凡文臣贈官



 通直郎以上,寺、監官以上未升朝者,雜壓在通直郎之上同。



 每贈兩官,至奉直大夫一官。有出身不贈奉直大夫、中散大夫。



 太子太師、太子太傅、太子太保、特進、觀文殿大學士、太子少師、太子少傅、太子少保、御史大夫、觀文殿學士、資政保和殿大學士、六曹尚書、金紫光祿大夫、銀青光祿大夫、光祿大夫、翰林學士承旨、翰林學士、資政保和端明殿學士、龍圖天章寶文顯謨徽猷敷文閣學士、左右散騎常侍、權六曹尚書、御史中丞、開封尹、六曹侍郎、樞密直學士、龍圖天章寶文
 顯謨徽猷敷文閣直學士,每贈三官,至奉直大夫二官,至通議大夫一官。有出身人不贈奉直、中散二大夫。



 凡文武臣封贈封換諸文武臣封贈對換,以所加官準格對換,並聽從高。



 承事郎換忠訓郎,宣義郎換從義、秉義郎,宣教郎換訓武、修武郎,通直郎換武義、武翼郎,奉議郎換武節、武略、武經郎,承議郎換武功、武德、武顯郎。朝奉郎換武義、武翼大夫,朝散郎換武節、武略、武經大夫,朝請郎換武功、武德、武顯大夫。朝奉大夫換遙郡刺史,朝散大夫換遙
 郡團練使,朝請大夫換遙郡防禦使。奉直、朝議大夫換刺史,中散、中奉大夫換團練使,中大夫換防禦使,太中大夫、通議、通奉大夫換觀察使,正議、正奉、宣奉大夫換承宣使,光祿人夫、銀青、金紫光祿大夫換節度使。



 凡文武官父任承直郎以下贈官



 承直郎,留守、節察判官——留守府判官、節度判官,承議郎。儒林郎,支、掌、防、團判官——節度掌書記、觀察支使、防禦判官、團練判官,奉議郎。文林郎、從事郎、從政郎,兩使初等
 職官、令、錄——留守推官、觀察推官、軍事判官、軍事推官、司錄參軍、錄事參軍,團練推官、軍監判官、防禦判官,縣令,通直郎。修職郎,知令、錄——知司錄參軍、知錄事參軍、縣丞,宣教郎。迪功郎,判、司、簿、尉——軍巡判官、司理參軍、司法參軍、司戶參軍、主簿、縣尉,宣義郎。



 致仕



 凡文武朝官、內職引年辭疾者,多增秩從其請,或加恩其子孫。乾德元所,太子太師致仕侯益來預郊祀,太祖
 優待之,因詔曰:「群官列位,自有通規,舊德來朝,所宜加禮,且表優賢之意,用敦尚齒之風。自今一品致仕官曾帶平章事者,每遇朝會,宜綴中書門下班。」二年,令藩鎮帶平章事求休致者亦如之。



 咸平五年,詔文武官年七十以上求退者,許致仕,因疾及有贓犯者聽從便。牧伯、內職、三班皆換環衛、幕職、州縣外官。景德元年三月,詔三班使臣七十以上視聽未衰者與厘務,其老昧不任及年七十五以上者,借職授支郡上佐,奉職、殿直授
 節鎮上佐,不願者聽歸鄉里。凡升朝官遇慶恩,父在者授致仕官,其不在者,文官始大理評事,武官始副率,再經恩累加焉。祖在而求回授者亦聽。皆不給奉,亦有子居要近加賜章服者。



 天聖、明道間,員外郎已上致仕者,錄其子試秘書省校書郎。三丞已上為太廟齊郎。無子,聽降等官其嫡孫若弟侄一。景祐三年詔曰:「致仕官舊皆給半奉,而未嘗為顯官者或貧不能自給,豈所以遇高年養廉恥也。其大兩省、大卿監、正刺史、合門使以上
 致仕者,自今給奉並如分司官例,仍歲時賜羊酒、米面,令所在長吏常加存問。」其後,又許致仕官子孫免選除近官。四年,臣僚有請致仕,未及錄其子孫而遽亡者,命既出,輔臣皆謂法當追收,仁宗憫之,竟官其後。侍御史知雜事司馬池言:「文武官年七十以上不自請致仕者,許御史臺糾劾以聞。」慶歷中,權御史中丞賈昌朝又言:「臣僚年七十而筋力衰者,並優與改官致仕;雖七十而未衰及別有功狀、朝廷固留任使者,勿拘此令。在京若
 尚書工部侍郎俞獻卿、少府監畢世長、太常少卿李孝若、尚書駕部郎中李士良,在外若給事中盛京、光祿卿王盤、太常少卿張效、尚書兵部郎中張億,皆耄昏不可任事,並請除致仕。」詔:「在京者令中書體量,在外者下諸處曉諭之。」



 皇祐中,知諫院包拯、吳奎亦言:「願令御史臺監察年七十已上,移文趣其請老不即自陳者,直除致仕。」朝廷未行。奎復言:「國家謹禮法以維君子,明威罰以御小人。君子所顧者,禮法也;小人所畏者,威罰也。繇文
 武二選為士大夫,是皆君子之地也,儻不以禮法待之,則是廢名器而輕爵祿。七十致仕,學者所知,而臣下引年自陳,分之常也,人君好賢樂善而留之,仁之至也。自三代以來,用此以塞貪墨、聳廉隅,近者句希仲、陸軫等,皆以年高特與分司,初欲風動群臣,而在位殊未有引去者,是臣言未效也。請詳前奏施行。」於是詔:「少卿監以下年七十不任厘務者,外任令監司、在京委御史臺及所屬以狀聞。嘗任館閣、臺諫官及提點刑獄者,令中書
 裁處。待制已上能自引年,則優加恩禮。」



 然是時言事之人,競欲擊劾大臣,有高年者俱不自安。仁宗手詔曰:「老臣,朕之所眷禮也,進退體貌,恩意豈不有異哉!凡嘗預政事之臣,自今毋或遽求引去,臺諫官勿以為言。」其風動勸勵之方又如此。至於因事責降分司,或老病不任官職之事,或居官犯法,或以不治為所部劾奏,沖替而求致仕者,子孫更不推恩,雖或推恩,其除官例皆降等,若耆老舊臣體貌優異,賞或延於子孫,奉或全給半給。
 歲時問勞,皆有禮意。



 治平四年,神宗即位,龍圖閣直學士兼侍讀李柬之、李受相繼致仕。舊制,合門無謝辭例,帝特召柬之對延和殿,命坐賜茶;以受先朝藩府舊僚,升其子一任差遣,並錄其孫。皆宴餞資善堂,命講讀官賦詩,禦制詩序以寵其行,示異數也,是歲,又以果州團練使何誠用、惠州防禦使馮承用、嘉州團練使劉保吉、昭州刺史鄧保壽皆年七十以上至八十餘,並特令致仕,以樞密院言,致仕雖有著令,臣僚鮮能自陳故也。熙
 寧元年,以定國軍節度使李端願為太子少保致仕。故事,多除上將軍,帝令討閱唐制,優加是命。二年,以觀文殿學士、吏部尚書趙概為太子少師致仕。故事,再請則許;概三乞始從,優耆舊也。三年,編修中書條例所言::



 人臣非有罪惡,致仕而去,人君遇之如在位時,禮也。近世致仕並與轉官,蓋以昧利者多,知退者少,欲加優恩,以示勸獎。推行既久,姑從舊制。若兩省正言以上官,三班使臣、大使臣、橫行、正任等,並不除為致仕官。致仕帶職
 者,皆落職而後優遷其官。看詳別無義理,但致仕恩例不均。如諫議大夫不可改給事中,並轉工部侍郎,乃是超轉兩資;工部尚書並除太子少保,乃是超轉六資,若知制誥、待制官卑者除卿監,緣知制誥、待制待遇非與卿監比。今他官致仕皆得遷官,此獨因致仕更見退抑。供奉官、侍禁八品,除率府副率,蓋六品。諸司副使、承制、崇班七品,除將軍,乃三品。至於節度使除上將軍,防禦、團練、刺史並除大將軍,緣諸衛名額不一,至有刺史除
 官高於防禦使者。今若令文武官帶職致仕人許仍舊職,上轉一官,及文臣正言、武臣借職以上皆得除為致仕官,則無輕重不等之患。



 若選人令、錄以上並除朝官,經恩皆得封贈,蔭及四世,旁支例得贖罪、免役。又京官致仕亦止遷一官,若光錄寺丞致仕,有出身除秘書省著作佐郎,無出身除大理寺丞,而令、錄職官乃除太子中允或中舍,殊未為當。若進納出身人例除京官,至有經覃恩遷至升朝官者,類多兼並有力之家,皆免州縣
 色役及封贈父母。如京官七品,除衙前外,亦名餘色役,尤為僥幸。條例繁雜,無所適從。如錄事參軍或除衛尉寺丞,或除大理評事,或除奉禮郎恩例不同,可以因緣生弊。



 今定:凡文臣京朝官以上各轉一官,帶職仍舊不轉官,乞親屬恩澤者依舊條。選人依本資序轉合入京朝官,進納及流外人判、司、簿、尉除司馬,令、錄除別駕。在京諸司勒留官依簿、尉以上,親賢勞舊合別推恩者取旨。歷任有入已贓,不得乞親戚恩澤,仍不遷
 官,其致仕官除中書、樞密院外,並在見任官之上,致仕及三年之上,元非因過犯,年未及七十,不曾經敘封及陳乞親戚恩澤,卻願仕宦,並許進狀敘述。若有薦舉者,各依元資序授官。其才行為眾所知,朝廷特任使者,不拘此法。



 從之。自此宰相以下並帶職致仕。



 四年,以端明殿學士、尚書右丞王素為工部尚書、端明殿學士致仕,觀文殿學士、兵部尚書歐陽修為太子少師、觀文殿學士致仕。帶職致仕,自素始也。五年,守司空兼侍中曾公亮遷守太
 傅致仕,特許入謝。以公亮逮事三朝,既加優禮,仍給見任支賜。十月,詔兩省以上致仕官毋得因大禮用子升朝敘封遷官。先是,王安石言,李端願、李柬之敘封,中書失檢舊例,法當改正。帝曰:「如此,則獨不被恩。」安石曰:「敘封初元義理,今既未能遽革,庸可承誤為例?如三師、三公官,因子孫郊恩敘授,尤非宜也。」帝從之。



 元豐三年,詔:「自今致仕官遇誕節及大禮,許綴舊班。」以禮部侍郎範鎮居都城外,遇同天節,乞隨散官班上壽,帝令鎮班見
 任翰林學士上,故有是詔。又詔:「致仕官朝失儀,勿劾,並著為令。」又詔:「自今致仕官領職事者,許帶致仕,該遷轉者轉寄祿官,若止系寄祿官,即以本官致仕。其見任致仕官,除三師、三公、東宮三師三少外,餘並易之。」六年,以守太尉、開府儀同三司、知河南府文彥博為河東、永興節度使、守太師致仕。彥博辭兩鎮,止以河東舊鎮貼麻行下。彥博又言:「前辭闕下之日,嘗奏得致仕後,當親辭天陛,今既得請,欲赴闕廷。」降詔從之。七年,詔文臣中大
 夫、武臣諸司使以下致仕,更不加恩。元祐元年,樞密院奏:「諸軍年七十,若以疾假滿百日不堪醫治差使者,諸廂都指揮使除諸衛大將軍致仕,諸軍都指揮使、諸班直都虞候帶遙郡除諸衛將軍致仕,諸班直上四軍除屯衛,拱聖以下除領軍衛,並有功勞者為左,無則為右。」從之。四年,詔:」應乞致仕而不原轉官者,受敕後,所屬保明以聞,當與推恩。中大夫至朝奉郎及諸司使,本宗有服親一人蔭補恩澤。橫行、諸司副使見有身自蔭補人,
 及內殿承制、崇班、合門祗候見理親民,並承議、奉議郎,許陳乞有服親一人恩例。中大夫、中散大夫、諸司使帶遙郡者,蔭補外準此。即朝奉郎以上及諸司使,雖未授敕而身亡,在外者以乞致仕狀到門下省日,在京以得旨日,亦許陳乞有服親一人恩例。」六年,監察御史徐君平言:「文臣致仕以年七十為斷,而使臣年七十猶與近地監當,至八十乃致仕,願許其致仕之年如文臣法,而給其奉。」從之。三省言:「張方平元系宣微南院使、檢校太
 傅、太子少師致仕。元豐官制行,廢宣微使,元祐三年復置,儀品恩數如舊制,方平依舊帶宣微南院使致仕。」紹聖三年詔:「文武官該轉官致仕,依舊出告外,其餘守本官致仕者並降敕,更不給告。內因致仕合該乞恩澤人更不具鈔,令尚省通書三司入熟狀,仍不候印畫。」又詔:「應臣僚丁憂中不許陳乞致仕。」



 建中靖國元年,尚書省言:「臣僚在憂制中不得陳乞致仕,其間有官序合得致仕恩澤之人,合行立法。」詔:「臣僚丁憂中遇疾病危篤,
 其官序合該致仕恩澤者,聽以前官經所屬自陳。」大觀二年,詔致仕官年八十以上應給奉者,以緡錢充。政和六年,提舉廣東學事孫璘言:「諸州致仕官居鄉者,乞許令赴貢士宴,擇其年彌高者而惇事之,使長幼有序,獻酬有禮,人知里選之法,孝悌之義。」從之。宣和四年,詔六曹尚書致仕遺表恩澤,共與四人,其餘侍從官三人,立為定制。



 建炎間,嘗詔:「文武官陳乞致仕,朝廷不從,致有身亡之人,許依條陳乞致仕恩澤,及陳乞致仕而道路
 不通,不曾被受敕命,亦許州、軍保明推恩。」時強行父博學清修,不緣事故疾病,慨然請老,葉份言之,許令再仕。王次翁年未六十,浩然全退,呂衣止



 言之,落致仕,特令再仕。凡類此者,蓋因其材而挽留之也。直秘閣致仕鄭南掛冠已久,年德俱高,大臣言之,詔除秘閣修撰,仍舊致仕。優其恩不奪其志也。呂頤浩以少保乞除一寄祿官致仕,詔除少傅,依前鎮南軍節度使、成國公致仕;韓世忠以太傅、鎮南武安寧國軍節度使充醴泉觀使、咸
 安郡王乞身,詔除太師致仕。因將相之知止而優其歸也。楊惟忠、刑煥皆以節度致仕。臣僚言:「祖宗時,節將、臣僚得謝,不以文武,並納節除一官。」以今日不復納節換官為非。詔今後依祖宗典故,蓋不以私恩勝公法也。昭慶軍節度使、開府儀同三司、充萬壽觀使韋淵乞守本官致仕,詔免赴朝參,仍依兩府例,合破請給人從。優親之恩而異之也。



 隆興以後,因臣僚言年七十不陳乞致仕者,除合得致仕或遺表恩澤外,並不許遇郊奏薦。已
 而復詔:郊祀在近,未致仕人更許陳乞奏薦一次。可以不予而予之,示厚恩也。執政在謫籍者陳乞致仕,雖許敘復而寢罷合得恩澤,只據見存階官蔭補。淳熙十六年,寧武軍承宣使、提舉祐神觀王友直復奉國軍節度使致仕,臣僚論列,仍守本官職致仕。可以予而不予,嚴公法也。抑揚輕重間,可以見優老恤賢之意,可以識制情抑幸之術,故備錄於篇。



 文臣蔭補



 太師至開府儀同三司:子,承事郎;孫及期親,承奉郎;大功以下及異姓親,登仕郎;門客,登仕郎。不理選限。



 知樞密院事至同知樞密院事:子,承奉郎;孫及期親,承務郎;大功以下及異姓親,登仕郎門客,登仕郎,不理選限。



 太子太師至保和殿大學士:子,承奉郎;孫及期親,承務郎;大功以下,登仕郎;導姓親,將仕郎。



 太子少師至通奉大夫:子孫及期親,承務郎;大功親,登仕郎;異姓親,登仕郎;小功以下親,將仕郎。



 御史中丞至侍御史:子,承務郎;孫
 及期親,登仕郎;大功,將仕郎;小功以下及異姓親,將仕郎,



 中大夫至中散大夫:子,通仕郎;孫及期親,登仕郎;大功,將仕郎;小功以下,將仕郎。



 太常卿至奉直大夫:子,登仕郎;孫及期親,將仕郎;大功小功親,將仕郎。



 國子祭酒至開封少尹:子孫及小功以上,將仕郎。



 朝請大夫、帶職朝奉郎以上:理職司資序及不帶職致仕者同。



 子,將仕郎;小功以上親,將仕郎;緦麻,上州文學。注權官一任,回注正官,謂帶職朝奉郎以上亡歿應蔭補者。



 廣南東、西路轉運副使:子,登仕郎;孫及期親,
 將仕郎。提點刑獄:子,將仕郎;孫及期親,將仁郎。



 武臣蔭補



 樞密使、開府儀同三司;子,秉義郎;孫及期親,忠翊郎;大功以下親,承節郎;異姓親,承信郎。



 知樞密院事、同知樞密院事、樞密副使、太尉、節度使:子,忠訓郎;孫及期親,成忠郎;大功,承節郎;小功以下及異姓親,承信郎。



 諸衛上將軍,承宣使、觀察使、通侍大夫:子,成忠郎,孫及期親,保義郎;大功以下,承信郎;及異姓親,承信郎。



 樞密
 都承旨、正侍大夫至右武大夫、防禦使、團練使、延福宮使至昭宣使任入內內侍省都知以上:子,保義郎;孫及期親,承節郎;大功以下親,內各奏異姓親者同。



 承信郎。刺史:子,承節郎;孫及期親,承信郎;大功以下,進武校尉。



 諸衛大將軍、武功至武翼大夫、樞密承旨至諸房副承旨:子,承節郎;孫及期親,承信;郎大功以下,進武校尉。



 諸衛將軍、正侍至右武郎、武功至武翼郎:子,承信郎;孫,進武校尉;期親,進義校尉。



 樞密院逐房副承旨;子,承信郎。



 訓武、修武郎及合門祗候:子,進乂校尉。



 忠佐帶遙郡者,每兩遇大禮蔭補,子:刺史,進武校尉;團練使、防禦使,承信郎。



 臣僚大禮蔭補



 宰相、執政官:本宗、異姓、門客、醫人各一人。東宮三師、三少至諫議大夫:權六曹侍郎、侍御史同。



 本宗一人。



 寺長貳、監長貳、秘書少監、國子司業、起居郎舍人、中書門下省檢正、沿書省左右司郎官、樞密院檢詳、若六曹郎中、殿中侍
 御史、左右司諫、開封少尹:子或孫一人。



 致仕蔭補



 曾任宰相及見任三少、使相:三人。曾任三少、使相、執政官、見任節度使;二人,太中大夫及曾任尚書侍郎及右武大夫以上,並曾任諫議大夫以上及侍御史:一人。



 遺表蔭補



 曾任宰相及見任三少、使相:五人。曾任執政官、見任節度使:四人。太中大夫以上:一人。諸衛上將軍、承宣使:四
 人。觀察使:三人。



\end{pinyinscope}