\article{志第一百五十六 藝文二}

\begin{pinyinscope}

 史類十三:一曰正史類,二曰編年類,三曰別史類,四曰史鈔類,五曰故事類,六曰職官類,七曰傳記類,八曰儀注類,九曰刑法類,十曰目錄類,十一曰譜牒類,十二曰
 地理類,十三曰霸史類。



 司馬遷《史記》一百三十卷裴駰等集注



 又《史記》一百三十卷陳伯宣注



 班固《漢書》一百卷顏師古注



 範曄《後漢書》九十卷章懷太子李賢注



 趙抃《新校前漢書》一百卷



 餘靖《漢書刊誤》三十卷



 劉昭《補注後漢志》三十卷



 陳壽《三國志》六十五卷裴松之注



 房玄齡《晉書》一百三十卷



 楊齊宣《晉書音義》三卷



 沈約《宋書》一百卷



 蕭子顯《南齊書》五十九卷



 姚思廉《梁書》五十六卷



 又《陳書》三十六卷



 魏收《後魏書》一百三十卷



 魏
 澹《後魏書紀》一卷本七卷



 張太素《後魏書天文志》二卷本百卷,惟存此



 李百藥《北齊書》五十卷



 令狐德棻《後周書》五十卷



 顏師古《隋書》八十五卷



 柳芳《唐書》一百三十卷《唐書敘例目》一卷。



 劉煦《唐書》二百卷



 歐陽修、宋祁《新唐書》二百五十五卷《目錄》一卷



 李繪《補注唐書》二百二十五卷



 薛居正《五代史》一百五十卷



 歐陽修《新五代史》七十四卷徐無黨注



 張守節《史記正義》三十卷



 司馬貞《史記索隱》三十卷



 張泌《漢書刊誤》一卷



 《三劉漢書標注》六卷劉敞、劉分文、劉奉世



 劉分文《漢書刊誤》四卷



 呂夏卿《唐書直筆新例》一卷



 吳縝《新唐書糾繆》二十卷



 又《五代史纂誤》三卷



 《朱梁列傳》十五卷



 張昭遠《後唐列傳》三十卷



 任諒《史論》三卷



 韓子中《新唐史辨惑》六十卷



 吳仁傑《兩漢刊誤補遺》十卷



 富弼《前漢書綱目》一卷



 劉巨容《漢書纂誤》二卷



 汪應辰《唐書列傳辨證》二十卷



 《西漢刊誤》一卷不知作者



 王旦《國史》一百二十卷



 呂夷簡《宋三朝國史》一百五十五卷



 鄧洵武《神宗正史》一百二十卷



 王珪《宋兩朝國史》一百二十卷



 王
 孝迪《哲宗正史》二百一十卷



 李燾、洪邁《宋四朝國史》三百五十卷



 《宋名臣錄》八卷



 《宋勛德傳》一卷



 《宋兩朝名臣傳》三十卷



 《咸平諸臣錄》一卷



 《熙寧諸臣傳》四卷



 《兩朝諸臣傳》三十卷



 並不知作者。



 張唐英《宋名臣傳》五卷



 葛炳奎《國朝名臣敘傳》二十卷



 右正史類五十七部,四千四百七十三卷。葛炳奎《國朝名臣敘傳》不著錄一部,二十卷



 荀悅《漢紀》三十卷



 袁宏《後漢紀》三十卷



 胡旦《漢春秋》一
 百卷



 《問答》一卷



 皇甫謐《帝王世紀》九卷



 《竹書》三卷荀勖、和嶠編



 蕭方等《三十國春秋》三十卷



 孫盛《晉陽秋》三十卷



 杜延業《晉春秋略》二十卷



 裴子野《宋略》二十卷



 《王通元經薛氏傳》十五卷



 馬總《通歷》十卷



 柳芳《唐歷》四十卷



 崔龜從《續唐歷》二十二卷



 裴煜之《唐太宗建元實跡》一卷



 路惟衡《帝王歷數圖》十卷



 陳岳《唐統紀》一百卷



 丘悅《三國典略》二十卷



 封演《古今年號錄》一卷



 薛榼《大唐聖運圖略》三卷



 《帝王照錄》一卷



 王起《五位圖》三卷



 苗臺符《古今
 通要》四卷



 馬永易《元和錄》三卷



 《大唐中興新書紀年》三卷不知作者



 韋昭度《續皇王寶運錄》十卷



 程正柔《大唐補紀》三卷



 凌璠《唐錄政要》十三卷



 《唐天祐二年日歷》一卷



 杜光庭《古今類聚年號圖》一卷



 《唐創業起居注》三卷溫大雅撰



 《唐高祖實錄》二十卷許敬宗、房玄齡等撰



 《唐太宗實錄》四十卷許敬宗撰



 《唐高宗後修實錄》三十卷。



 《唐武後實錄》二十卷



 《唐中宗實錄》二十卷



 《唐睿宗實錄》十卷又五卷



 並劉知幾、吳兢撰



 《唐玄宗實錄》一百卷元載、令狐峘撰



 《唐肅宗實錄》三十卷元載撰



 《唐
 代宗實錄》四十卷令狐峘撰



 《唐德宗實錄》五十卷裴□等撰



 《唐建中實錄》十五卷沉既濟撰



 《唐順宗實錄》五卷韓愈撰



 《唐憲宗實錄》四十卷



 《唐穆宗實錄》二十卷



 並路隋等撰



 《唐敬宗實錄》十卷李讓夷等撰



 《唐文宗實錄》四十卷魏謨修撰。



 《唐武宗實錄》二十卷



 《唐宣宗實錄》三十卷



 《唐懿宗實錄》二十五卷



 《唐僖宗實錄》三十卷



 《唐昭宗實錄》三十卷



 《唐哀帝實錄》八卷



 並宋敏求撰



 《五代梁太祖實錄》三十卷張袞、卻像等撰



 《五代唐懿祖紀年錄》一卷



 《五代唐獻祖紀年錄》一卷



 《五代唐莊宗實錄》
 三十卷



 並趙鳳、張昭遠等撰



 《五代唐明宗實錄》三十卷姚顗等撰



 《五代唐愍帝實錄》三卷張昭遠等撰



 《五代唐廢帝實錄》十七卷張昭等同撰



 《五代晉高祖實錄》三十卷



 《五代晉少帝實錄》二十卷



 並竇貞固等撰



 《五代漢高祖實錄》十卷蘇逢吉等撰



 《五代漢隱帝實錄》十五卷



 《五代周太祖實錄》三十卷



 並張昭、尹拙、劉溫叟等撰



 《五代周世宗實錄》四十卷宋王溥等撰



 《南唐烈祖實錄》二十卷高遠撰



 《後蜀高祖實錄》三十卷



 《後蜀主實錄》四十卷



 並李昊撰



 《宋太祖實錄》五十卷李沆、沈倫修



 《太宗實錄》八十卷錢若水修



 《真
 宗實錄》一百五十卷晏殊等同修



 《仁宗實錄》二百卷韓琦等修



 《英宗實錄》三十卷曾公亮等修



 《神宗實錄朱墨本》三百卷舊錄本用墨書,添入者用朱書,刪去者用黃抹



 《宋高宗日歷》一千卷



 《孝宗日歷》二千卷



 《光宗日歷》三百卷



 《寧宗日歷》五百一十卷重修五百卷



 《神宗實錄》二百卷趙鼎、範沖重修



 《哲宗實錄》一百五十卷



 《徽宗實錄》二百卷



 並湯思退進



 《徽宗實錄》二百卷李燾重修



 《欽宗實錄》四十卷洪邁修



 《高宗實錄》五百卷傅伯壽撰



 《孝宗實錄》五百卷



 《光宗實錄》一百卷



 並傅伯壽、陸游
 等修



 《寧宗實錄》四百九十九冊



 《理宗實錄初稿》一百九十冊



 《理宗日歷》二百九十二冊



 又《日歷》一百八十冊



 《度宗時政記》七十八冊



 《德祐事跡日記》四十五冊



 孫光憲《續通歷》十卷



 範質《五代通錄》六十五卷



 劉蒙叟《甲子編年》二卷



 《顯德日歷》一卷周扈蒙、董淳、賈黃中撰



 龔穎《運歷圖》三卷



 陳彭年《唐紀》四十卷



 宋庠《紀年通譜》十二卷



 鄭向《五代開皇記》三十卷



 《兩朝實錄大事》二卷



 王玉《文武賢臣治蜀編年志》一卷



 武密《帝王興衰年代錄》二卷



 《五代春秋》一
 卷



 《十代編年紀》一卷



 並不知作者



 章寔《歷代統紀》一卷



 司馬光《資治通鑒》三百五十四卷



 又《資治通鑒舉要歷》八十卷



 《通鑒前例》一卷



 《稽古錄》二十卷



 《歷年圖》六卷



 《通鑒節要》六十卷



 《帝統編年紀事珠璣》十二卷



 《歷代累年》二卷



 劉恕《資治通鑒外紀》十卷



 又《疑年譜》一卷



 《通鑒問疑》一卷



 章衡《編年通載》十卷



 王巖叟《系年錄》一卷



 《元祐時政記》一卷



 諸葛深《紹運圖》一卷



 楊備《歷代紀元賦》一卷



 胡仔《孔子編年》五卷



 朱繪《歷代帝王年運銓要》十卷



 司馬
 康《通鑒釋文》六卷



 李燾《續資治通鑒長編》一百六十八卷



 又《四朝史稿》五十卷



 《江左方鎮年表》十六卷



 《混天帝王五運圖古今須知》一卷



 《宋政錄》十二卷



 《宋異錄》一卷



 《宋年表》一卷



 史照《資治通鑒釋文》三十卷



 晁公邁《歷代記年》十卷



 熊克《九朝通略》一百六十八卷



 《中興小歷》四十一卷



 呂祖謙《大事記》二十七卷



 又《宋通鑒節》五卷



 《呂氏家塾通鑒節要》二十四卷



 朱熹《通鑒綱目》五十九卷



 又《提要》五十九卷



 《宋聖政編年》十二卷不知
 作者



 汪伯彥《建炎中興日歷》一卷



 袁樞《通鑒紀事本末》四十二卷



 喻漢卿《通鑒總考》一百十二卷



 吳曾《南北征伐編年》二十三卷



 徐度《國紀》六十五卷



 胡宏《皇王大紀》八十卷



 李丙《丁未錄》二百卷



 李心傳《建炎以來系年要錄》二百卷



 《國史英華》一卷不知作者



 何許《甲子紀年圖》一卷



 曾綎《通鑒補遺》一百篇



 李孟傳《讀史》十卷



 崔敦詩《通鑒要覽》六十卷



 王應麟《通鑒答問》四卷



 胡安國《通鑒舉要補遺》一百二十卷



 沉樞《通鑒總類》二十卷



 張根《歷代指掌
 編》九十卷



 李心傳《孝宗要略初草》二十三卷



 張公明《大宋綱目》一百六十七卷



 洪邁《節資治通鑒》一百五十卷



 又《太祖太宗本紀》三十五卷



 又《四朝史紀》三十卷



 又《列傳》一百三十五卷



 黃維之《太祖政要》一十卷



 呂中《國朝治跡要略》十四卷



 右編年類一百五十一部,一萬五百七十五卷。《寧宗實錄》以下不著錄六部,無卷。曾綎《通鑒補遺》以下不著錄十五部,九百六十八卷



 王瓘《廣軒轅本紀》三卷



 《汲塚周書》十卷



 郭璞注《穆天子
 傳》六卷



 趙曄《吳越春秋》十卷



 皇甫遵注《吳越春秋》十卷



 司馬彪《九州春秋》十卷



 趙瞻《史記抵牾論》五卷



 《漢書問答》五卷



 劉珍等《東觀漢紀》八卷



 孔衍《春秋後語》十卷



 李延壽《南史》八十卷



 又《北史》一百卷



 元行沖《後魏國典》三十卷



 《金陵六朝記》一卷



 王豹《金陵樞要》一卷



 李匡文《漢後隋前瞬貫圖》一卷



 李康《唐明皇政錄》十卷



 袁皓《興元聖功錄》



 《功臣錄》三十卷



 《唐僖宗日歷》一卷



 劉肅《唐新語》十三卷



 《唐總記》三卷



 渤海填《唐廣德神異錄》四十五卷



 歐陽迥一作「炳」



 《唐錄備闕》十五卷



 裴濆《大和新修辨謗略》三卷



 程光榮一作「柔」



 《唐補注記》「注記」一作「紀」



 三卷



 曹玄圭《唐列聖統載圖》十卷



 郭修《唐年紀錄》一卷



 南卓《唐朝綱領圖》五卷



 《唐紀年記》二卷



 吳兢《開元名臣錄》三卷



 又《唐太宗勛史》一卷



 《唐書備闕記》十卷



 高峻《小史》一百十卷



 許嵩《建康實錄》二十卷



 張詢古《五代新說》二卷



 劉軻《帝王歷數歌》一卷



 又《唐年歷》一卷



 裴庭裕《東觀奏記》三卷



 《新野史》十卷題「顯德元年終南山不名子撰」



 張傳靖《唐編記》一作「紀」



 十卷



 胡
 旦《唐乘》一作「策」



 七十卷



 王沿《唐志》二十一卷



 孫甫《唐史記》七十五卷



 王皞《唐餘錄》六十卷



 李匡文《兩漢至唐年紀》一卷



 王禹偁《五代史闕文》二卷



 陶岳《五代史補》五卷



 詹玠《唐宋遺史》四卷



 劉直方《大唐機要》三十卷



 蘇轍《古史》六十卷



 孫沖《五代紀》七十七卷



 劉分文《五代春秋》一部卷亡



 劉恕《十國紀年》四十二卷



 常璩《華陽國志》十卷



 《江南志》二十卷



 李清臣《平南事覽》二十卷



 《吳書實錄》三卷記楊行密事



 《真宗聖政紀》一百五十卷



 又《政要》十卷



 《仁宗觀文覽古圖記》十卷



 丁謂《大中祥符奉祀記》五十卷《目》二卷



 又《大中祥符迎奉聖像記》二十卷《目》二卷



 李維《大中祥符降聖記》五十卷《目》三卷



 王欽若《天禧大禮記》五十卷《目》二卷



 呂夷簡《三朝寶訓》三十卷



 李淑《三朝訓覽圖》十卷



 錢惟演《咸平聖政錄》三卷



 李昭遘《永熙政範》二卷



 張商英《神宗正典》六卷



 林希《兩朝寶訓》二十一卷



 舒但《元豐聖訓》三卷



 《六朝寶訓》一部卷亡



 鄭居中《崇寧聖政》二百五十五冊



 又《聖政錄》三百二十
 三冊



 賈緯《備史》六卷



 《史系》二十卷



 楊九齡《正史雜論》十卷



 《河洛春秋》二卷



 《歷代善惡春秋》二十卷



 李筌《閫外春秋》十卷



 薛韜玉《帝照》一卷



 沉汾《元類》一卷



 楊岑《皇王寶運錄》三十卷



 瞿一作「翟」



 驤《帝王受命編年錄》三十卷



 徐暠《三朝革命錄》三卷



 錢信《皇猷錄》一卷



 《歷代鴻名錄》八卷



 韋光美《嘉號錄》一卷



 崔倜《帝王授受圖》一卷



 牛檢《帝王事跡相承圖》三卷



 《歷代君臣圖》二卷



 龔穎《年一作「運」



 歷圖》八卷



 賈欽文《古今年代歷》一卷



 張敦素《通記一作「紀」



 建元歷》
 二卷



 柳璨《補注正閏位歷》三卷



 杜光庭《帝王年代州郡長歷》二卷



 王起《五運圖》一卷



 曹玄圭《五運圖一作「錄」



 》十二卷



 張洽《五運元紀》一卷



 《古今帝王記》十卷



 衛牧《帝王真偽記》七卷



 《紀年志》一卷



 武密《帝王年代錄》三十卷



 鄭伯邕《帝王年代圖》一卷



 又《帝王年代記》三卷



 焦□路《聖朝年代記一作「紀」



 》十卷



 韋光美《帝王年號圖》一卷



 汪奇《古今帝王年號錄》一卷



 李昉《歷代年號》一卷



 蓋君平《重編史巂》三十卷



 孫昱《十二國史》十二卷



 《西京史略》二卷



 《史記掇
 英》五卷



 並不知作者



 鄭樵《通志》二百卷



 蕭常《續後漢書》四十二卷



 李□巳《改修三國志》六十七卷



 陳傅良《建隆編》一卷一名《開基事要》



 蔡幼學《宋編年政要》四十卷



 又《宋實錄列傳舉要》十二卷



 洪偃《五朝史述論》八卷洪邁孫



 趙甡之《中興遺史》二十卷



 樓昉《中興小傳》一百篇



 右別史類一百二十三部,二千二百十八卷。趙甡之《中興遺史》以下不著錄二部,一百二十卷一篇



 《馬史精略》五十六卷



 趙世逢《兩漢類要》二十卷



 周護《三
 史菁英》三十卷



 《十七史贊》三十卷



 《三代說辭》十卷不知作者



 孫玉汝《南北史練選》十八卷



 《史略》三卷



 楊侃《兩漢博聞》十二卷



 林鉞《漢巂》十卷



 宗諫《三國採要》六卷



 薛儆《晉書金穴鈔》十卷



 荀綽《晉略》九卷



 張陟《晉略》二十卷



 杜延業《晉春秋略》二十卷



 《晉史獵精》一百三十卷



 胡寅《讀史管見》三十卷



 又《三國六朝攻守要論》十卷



 趙氏《六朝採要》十卷



 杭暕《金陵六朝帝王統紀》一卷



 薛韜玉《唐要錄》二卷



 張栻《通鑒論篤》四卷



 孫甫《唐史論斷》二卷



 石介《唐鑒》
 五卷



 範祖禹《唐鑒》十二卷



 又《帝學》八卷



 陳季雅《兩漢博議》十四卷



 李舜臣《江東十鑒》一卷



 陳傅良《西漢史鈔》十七卷



 《東萊先生西漢財論》十卷呂祖謙論,門人編



 劉熙古《歷代紀要》五十卷



 喬舜《古今語要》十二卷



 賈昌朝《通紀》八十卷



 趙善譽《讀史輿地考》六十三卷一名《輿地通鑒》



 裴松之《國史要覽》二十卷



 鄭暐《史巂》十卷



 曹化《史書集類》三卷



 朱黼《紀年備遺正統論》一卷



 《唯室先生兩漢論》一卷陳長方



 張唐英《唐史發潛》六卷



 倪遇《漢論》十三卷



 陳惇修《唐史斷》
 二十卷



 王諫《唐史名賢論斷》二十卷



 程鵬《唐史屬辭》四卷



 《唐帝王號宰臣錄》十卷



 《名賢十七史確論》一百四卷不知作者



 胡旦《五代史略》四十二卷



 韓保升《文行錄》五十卷



 李□《續帝學》一卷



 姚虞賓《諸史臣謨》八卷



 鄭少微《唐史發揮》十二卷



 陳天麟《前漢六帖》十二卷



 陳應行《讀史明辨》二十四卷



 又《讀史明辨續集》五卷



 師古《三國志質疑》十四卷



 又《西漢質疑》十九卷



 《東漢質疑》九卷



 《何博士備論》四卷何去非



 陳亮《通鑒綱目》二十三卷



 《葉學士唐史鈔》
 十卷不知名



 唐仲友《唐史義》十五卷



 又《續唐史精義》十卷



 楊天惠《三國人物論》三卷



 李石《世系手記》一卷



 《兩漢著明論》二十卷



 《十二國史略》三卷



 《章華集》三卷



 《縱橫集》二十卷



 《十三代史選》五十卷



 《南史摭實韻句》三卷



 《議古》八卷



 《史譜》七卷



 《五代纂要賦》一卷



 《國朝撮要》一卷



 《約論》十卷



 並不知作者



 李燾《歷代宰相年表》三十三卷



 又《唐宰相譜》一卷



 《王謝世表》一卷



 《五代三衙將帥年表》一卷



 竇濟《皇朝名臣言行事對》十二卷



 李心傳《舊聞證誤》十五卷



 龔
 敦頤《符祐本來》一十卷



 洪邁《記紹興以來所見》二卷



 左史鈔類七十四部,一千三百二十四卷。李燾《歷代宰相年表》以下不著錄八部,七十五卷



 班固《漢武故事》五卷



 蔡邕《獨斷》二卷



 裴烜之《承祚實跡》一卷



 王琳《魏鄭公諫錄》五卷



 武平一《景龍文館記》十卷



 吳兢《貞觀政要》十卷



 又《開元升平源》一卷



 蘇瑰《中樞龜鑒》一卷



 韓琬《御史臺記》十二卷



 韋述《集賢注記》二卷



 崔光庭《德宗幸奉天錄》一卷



 沉既濟《選舉志》三卷



 馬宇《鳳
 池錄》五卷



 韋執誼《翰林故事》一卷



 李吉甫《元和國計略》一卷



 劉公鉉《鄴城舊事》六卷



 韋處厚《翰林學士記》一卷



 元稹《承旨學士院記》一卷



 李德裕《西南備邊錄》一卷



 又《兩朝獻替記》二卷



 《次柳氏舊聞》一卷



 令狐澄《貞陵遺事》一卷



 令狐綯《制表疏》一卷



 《李司空論事》七卷唐蔣偕編,李絳所論



 南卓《綱領圖》一卷



 鄭處誨《明皇雜錄》二卷



 又《天寶西幸略》一卷



 《吳湘事跡》一卷不知作者



 王仁裕《開元天寶遺事》一卷



 盧駢《御史臺三院因話錄》一卷



 柳玭《續貞陵遺事》一卷



 鄭向《起居注故事》三卷



 蘇頌《邇英要覽》一部卷亡



 樂史《貢舉故事》二十卷《目》一卷



 鄭畋《敕語堂判》五卷



 李巨川《勤王錄》二卷



 楊鉅《翰林舊規》一卷



 張著《翰林盛事》一卷



 李構《御史臺故事》三卷



 李肇《翰林內志》一卷



 又《翰林志》一卷



 蘇易簡《續翰林志》二卷



 《杜悰事跡》一卷



 《梁宣底》三卷



 《汾陰后土故事》三卷自漢至唐



 《武成王配饗事跡》二十卷



 並不知作者



 林勤《國朝典要雜編》一卷



 李大性《典故辨疑》二十卷



 呂夷簡、林希進《五朝寶訓》六十卷



 《三朝太平寶訓》二
 十卷《三朝訓鑒圖》十卷仁宗製序



 沈該進《神宗寶訓》一百卷



 《神宗寶訓》五十卷不知集者姓名



 洪邁集《哲宗寶訓》六十卷



 《欽宗寶訓》四十卷



 《高宗聖政》六十卷



 《高宗寶訓》七十卷



 《孝宗寶訓》六十卷



 並國史寶錄院進



 史彌遠《孝宗寶訓》六十卷



 《紹興求賢手詔》一卷



 《高宗孝宗聖政編要》二十卷乾道、淳熙中修



 《高宗聖政典章》十卷不知作者



 《宋朝大詔令》二百四十卷紹興中,出於宋綬家



 《永熙寶訓》二卷李昉子宗諤纂



 《仁宗觀文鑒古圖》十卷



 王洙《祖宗故事》二十卷



 李淑《耕籍類事》五卷



 林特《東封西祀
 朝謁太清宮慶賜總例》二十六卷



 韓絳《治平會計錄》六卷



 李常《元祐會計錄》三卷



 崔立《故事稽疑》十卷



 《孝宗聖政》五十卷



 彭龜年《內治聖鑒》二十卷



 《光宗聖政》三十卷



 富弼《契丹議盟別錄》五卷



 朱勝非《秀水閑居錄》二卷



 呂本中《紫微雜記》一卷



 蔡絳《北征紀實》二卷



 萬俟離《太后回鑾事實》十卷



 湯思退等《永祐陵迎奉錄》十卷



 大惟簡《塞北紀實》三卷



 宋敏求《朝貢錄》二十卷



 張養正《六朝事跡》十四卷



 吳彥夔《六朝事跡別集》十四卷



 韓元吉《金國
 生辰語錄》一卷



 劉珙《江東救荒錄》五卷



 宋介《執禮集》二卷



 陳曄《通州鬻海錄》一卷



 龔頤正《續稽古錄》一卷



 洪遵《翰苑群書》三卷



 又《會稽和買事宜錄》七卷



 程大昌《北邊備對》六卷



 《慶歷邊議》三卷



 《開禧通和錄》一卷



 《開禧持書錄》二卷



 《開禧通問本末》一卷



 《金陵叛盟記》十卷



 並不知作者



 宋庠《尊號錄》一卷



 又《掖垣叢志》三卷



 董煟《活民書》三卷



 又《活民書拾遺》三卷



 《史館故事錄》三卷



 《五國故事》二卷



 並不知作者



 尉遲偓《中朝故事》二卷



 孔武仲《金華講義》十三
 卷



 王禹偁《建隆遺事》一卷



 田錫《三朝奏議》五卷



 曾致堯《清邊前要》五十卷



 李至《皇親故事》一卷



 杜鎬《鑄錢故事》一卷



 丁謂《景德會計錄》六卷



 王曙《群牧故事》三卷



 《兩朝誓書》一卷景德中與契丹往復書



 辛怡顯《雲南錄》三卷



 沈該《翰林學士年表》一卷



 蘇耆《次續翰林志》一卷



 錢惟演《金坡遺事》三卷



 晁迥《別書金坡遺事》一卷



 李宗諤《翰林雜記》一卷



 王皞《言行錄》一卷



 王旦《名賢遺範錄》十四卷



 餘靖《國信語錄》一卷



 李淑《三朝訓鑒圖》十卷



 陳湜《三朝逸史》一
 卷



 沉立《河防通議》一卷



 富弼《救濟流民經畫事件》一卷



 田況《皇祐會計錄》六卷



 陳次公《安南議》十篇



 宋咸《朝制要覽》十五卷



 李上交《近事會元》五卷



 範鎮《國朝事始》一卷



 又《東齋記事》十二卷



 《太平盛典》三十六卷



 《國朝寶訓》二十卷



 《慶歷會計錄》二卷



 《經費節要》八卷



 並不知作者



 張唐英《君臣政要》四十卷



 陳襄《國信語錄》一卷



 趙概《日記》一卷



 司馬光《日錄》三卷



 郟但《吳門水利》四卷



 王安石《熙寧奏對》七十八卷



 程師孟《奏錄》一卷



 羅從彥《宋遵堯錄》
 八卷



 何澹《歷代備覽》二卷



 王禹《王家三世書誥》一卷



 司馬光《涑水記聞》三十二卷



 周必大《鑾坡錄》一卷



 又《淳熙玉堂雜記》一卷



 陳模《東宮備覽》一卷



 《三朝政錄》十二卷



 《廣東西城錄》一卷



 《交廣圖》一卷



 並不知作者



 曾鞏《宋朝政要策》一卷



 畢仲衍《中書備對》十卷



 李清臣、張誠一《元豐土貢錄》二卷



 龐元英《文昌雜錄》七卷



 韓絳、吳充《樞密院時政記》十五卷



 蘇安《靜邊說》一卷



 薛向《邊陲利害》三卷



 《仁宗君臣政要》二十卷不知何人編



 範祖禹《仁皇訓典》六卷



 曾
 鞏《德音寶訓》三卷



 汪浹《榮觀集》五卷



 張舜民《使遼錄》一卷



 宋匪躬《館閣錄》十一卷



 劉永壽《章獻事跡》一卷



 曾布《三朝正論》二卷



 林虙《元豐聖訓》二十卷



 家安國《平蠻錄》三卷



 羅畸《蓬山記》五卷



 《明堂詔書》一卷不知集者



 高聿《鹽池錄》一卷



 吳若虛《崇聖恢儒集》三卷



 洪榆《創業故事》十二卷



 耿延禧《建炎中興記》一卷



 程俱《麟臺故事》五卷



 洪興祖《續史館故事錄》一卷



 張戒《政要》一卷



 李源《三朝政要增釋》二十卷



 歐陽安永《祖宗英睿龜鑒》十卷



 陳騤《中興
 館閣錄》十卷



 趙勰《廣南市舶錄》三卷



 嚴守則《通商集》三卷



 《契丹禮物錄》一卷



 《金華故事》一卷



 《兩朝交聘往來國書》一卷



 並不知作者



 臧梓《呂丞相勤王記》一卷



 李攸《通今集》二十卷



 又《宋朝事實》三十五卷



 袁夢麟《漢制叢錄》二十卷



 倪思《合宮嚴父書》一卷



 詹儀之《淳熙經筵日進故事》一卷



 又《淳熙東宮日納故事》一卷



 李心傳《建炎以來朝野雜記》十一卷



 又《朝野雜記》甲集二十卷乙集二十卷



 陸游《聖政草》一卷



 彭百川《治跡統類》四十卷



 又《中興治
 跡統類》三十卷



 江少虞《皇朝事實類苑》二十六卷



 張綱《列聖孝治類編》一百卷



 黃度《藝祖憲監》三卷



 又《仁皇從諫錄》三卷



 趙善譽《宋朝開基要覽》十四卷



 右故事類一百九十八部,二千九十四卷。彭百川《治跡統類》以下不著錄七部,二百二十一卷



 《東漢百官表》一卷不知作者



 陶彥藻《職官要錄》七卷



 又《職官要錄補遺》十八卷



 李吉甫《百司舉要》一卷



 唐玄宗《六典》三十卷



 杜英師《唐職該》一卷



 梁載言《具員故事》十七卷



 《
 大唐宰相歷任記》二卷



 任戩《官品纂要》十卷



 《宰輔年表》一卷



 《官品式律》一卷



 《歷代官號》十卷



 並不知作者



 楊侃《職林》三十卷



 孔至道《百官要望》一卷



 閻承琬《君臣政要》三十卷



 蒲宗孟《省曹寺監事目格子》四十七卷



 卻殷象《梁循資格》一卷



 王涯《唐循資格》一卷



 杜儒童《中書則例》一卷



 譚世績《本朝宰執表》八卷



 張之緒《唐文昌損益》三卷



 萬當世《文武百官圖》二卷



 陳繹《宰相拜罷錄》一卷



 又《樞府拜罷錄》一卷



 《三省樞密院除目》四卷



 司馬光《百官公卿
 表》十五卷



 孫逢吉《職官分紀》五十卷



 梁勖《職官品服》三十三卷



 趙氏《唐典備對》六卷不知名



 《三省儀式》一卷



 《職事官遷除體格》一卷



 《循資格》一卷



 《循資歷》一卷



 《唐宰相後記》一卷



 《國朝撮要》一卷



 《宋朝宰輔拜罷圖》四卷



 《宋朝官制》十一卷



 《三省總括》五卷



 並不知作者



 王益之《漢官總錄》十卷



 又《職源》五十卷



 《宋朝相輔年表》一卷中興館閣書目云:「臣繹上,《續表》曰臣易記。」



 蔡元道《祖宗官制舊典》三卷



 趙鄰幾《史氏懋官志》五卷



 趙曄《宋官制正誤沿革職官記》三卷



 何異《中興百
 官題名》五十卷



 龔頤正《宋特命錄》一卷



 司馬光《官制遺稿》一卷



 徐自明《宰輔編年錄》二十卷



 蔡幼學《續百官公卿表》二十卷



 又《續百官表質疑》十卷



 曾三異《宋新舊官制通考》十卷



 又《宋新舊官制通釋》二卷



 範沖《宰輔拜罷錄》二十四卷



 徐筠《漢官考》四卷



 董正工《職官源流》五卷



 《金國明昌官制新格》一卷不知何人撰



 楊王休《諸史闕疑》三卷



 趙粹中《史評》五卷



 王應麟《通鑒地理考》一百卷



 又《通鑒地理通釋》十四卷



 又《漢藝文志考證》十卷



 又《漢制考》
 四卷



 右職官類五十六部,五百七十八卷。楊王休《諸史闕疑》以下不著錄六部,一百三十六卷



 劉向《古列女傳》九卷



 《漢武內傳》二卷不知作者



 郭憲《洞冥記》四卷



 班昭《女戒》一卷



 伶玄《趙飛燕外傳》一卷



 皇甫謐《高士傳》十卷



 袁宏《正始名士傳》二卷



 葛洪《西京雜記》六卷



 習鑿齒《襄陽耆舊記》五卷



 蕭韶《太清紀》十卷



 杜寶《大業雜記》十卷



 劉餗《國史異纂》三卷



 梁載言《梁四公記》一
 卷



 趙毅《大業略記》三卷



 顏師古《大業拾遺》一卷



 賈潤甫《李密傳》三卷



 李筌《中臺志》十卷



 杜儒童《隋季革命記》五卷



 《隋平陳記》一卷



 魏征《隋靖列傳》一卷



 徐浩《廬陵王傳》一卷



 劉仁軌《河洛行年記》十卷



 李恕《誡子拾遺》四卷



 《越國公行狀》一卷唐鐘紹京事跡



 陳翊《郭令公家傳》十卷



 又《忠武公將佐略》一卷



 殷亮《顏杲卿家傳》一卷



 又《顏真卿行狀》一卷



 李邕《狄梁公家傳》一卷



 包住《河洛春秋》二卷



 陳鴻《東城父老傳》一卷



 張鷟《朝野僉載》二十卷



 又《僉載補遺》
 三卷



 李匡文《明皇幸蜀廣記圖》二卷



 郭湜《高力士外傳》一卷



 姚汝能《安祿山事跡》三卷



 《三朝遺事》一卷載張姚說、姚崇、宋璟事,不知作者



 甘伯宗《名醫傳》七卷



 《臨川名一作「賢」



 士賢一作「名」



 傳》三卷



 李涉一作「渤」



 《六賢傳》一卷



 孫仲《遺士傳》一卷



 《賢牧傳》十五卷



 張茂樞《張氏家傳》三卷



 吳操《蔣子文傳》一卷



 王方慶《魏玄成傳》一卷



 《郭元振傳》一卷



 範質《桑維翰傳》三卷



 李翰《張中丞外傳》一卷



 溫龠一作「畬」



 《天寶亂離記》一卷



 劉諫一作「練」



 《國朝傳記》三卷



 賀楚《奉天記》一卷



 《太和摧
 兇記》一卷



 楊棲白《南行記》一卷



 王坤《僖宗幸蜀記》一卷



 牛樸《登庸記》一卷



 江文秉《都洛私記》十卷



 胡嶠《陷遼記》三卷



 元澄《秦京內外雜記》一卷



 《蜀記》一卷



 《西戎記》二卷



 顏師古《獬豸記》一卷



 《靜亂安邦記》一卷



 《睢陽得死集》一卷載張巡、許遠事,不知作者



 沈既濟《江淮記亂》一卷



 李公佐《建中河朔記》六卷



 陳岠《朝廷卓絕事記》一卷



 穀況《燕南記》三卷



 鄭澥《涼國公平蔡錄》一卷



 李涪《刊誤》一卷



 陸贄《玄宗編遺錄》二卷



 韓昱《壺關錄》三卷



 林恩《補國史》五卷



 馬總《唐
 年小錄》六卷



 杜祐《賓佐記》一卷



 陳諫等《彭城公事跡》三卷



 王昌齡《瑞應圖》一卷



 路隋《平淮西記》一卷



 又《邠志》三卷



 李肇《國史補》三卷



 李潛用《乙卯記》一卷



 房千里《投荒雜錄》一卷



 李繁《鄴侯家傳》十卷



 李石《開成承詔錄》二卷



 李德裕《異域歸忠傳》二卷



 又《大和辨謗略》三卷



 《會昌伐叛記》一卷



 高少逸《四夷朝貢錄》十卷



 李商隱《李長吉小傳》五卷



 蔡京《王貴妃傳》一卷



 李璋《太原事跡雜記》十三卷



 張云《咸通庚寅解圍錄》一卷



 鄭樵《彭門紀亂》三卷



 韓
 偓《金鑾密記》一卷



 朱樸《日歷》一卷



 李氏《大唐列聖園陵記》一卷不知名



 丘旭《賓朋宴語》一卷



 盧言《雜說》一卷



 於政立《類林》十卷



 李奕《唐登科記》一卷



 《唐顯慶登科記》五卷



 徐鍇《登科記》十五卷



 樂史《登科記》三十卷



 《登科記》一卷



 《登科記》二卷起建隆至宣和四年



 張觀《二十二國祥異記》三卷



 徐岱《奉天記》一卷



 徽宗《宣和殿記》一卷



 又《嵩山崇福記》一卷



 《太清樓特宴記》一卷



 《筠莊縱鶴宣和閣記》一卷



 《宴延福宮承平殿記》一卷



 《明堂記》一卷



 《艮岳記》一卷



 陳繹《東
 西府記》一卷



 沈立《都水記》二百卷



 又《名山記》一百卷



 章惇《導洛通汴記》一卷



 李清臣《重修都城記》一卷



 王革《天泉河記》一卷



 《上黨記叛》一卷



 宋巨一作「宗拒」



 《明皇幸蜀錄》一卷



 趙源一《奉天錄》四卷



 陸贄《遣使錄》一卷



 李繁《北荒君長錄》三卷



 陸希聲《北戶雜錄》三卷



 蘇特一作「時」



 《唐代衣冠盛事錄》一卷



 鄭言《平剡錄》一卷



 《復交址錄》二卷



 《哥舒翰幕府故吏錄》一卷



 李巨川《許國公勤王錄》三卷



 《乾明一作「寧」



 會稽錄》一卷



 《三楚新錄》一卷



 《英雄佐命錄》一卷



 《世宗
 征淮錄》一卷



 《濠州干戈錄》一卷



 樂史《孝悌錄》二十卷《贊》五卷



 曹希逵一作「逢」



 《孝感義聞錄》三卷



 張讀《建中西狩錄》一卷



 元宏《錢塘平越州錄》一卷



 《潘氏家錄》一卷潘美行狀、告辭



 胡訥《孝行錄》二卷



 又《賢惠錄》二卷



 《民表錄》三卷



 李升《登封誥成錄》一百卷



 凌準《邠志》二卷



 郭廷誨《妖亂志》三卷



 韋管《國相事狀》七卷



 《雲南事狀》一卷



 《劉中州事跡》一卷



 《魏玄成故事》三卷



 趙寅《趙君錫遺事》一卷



 楊時《開成紀事》二卷



 楊九齡《桂堂編事》二十卷



 範鎮《東齋記事》十二
 卷



 李隱一作「隨」



 《唐記奇事》十卷



 史演《咸寧王定難實序》一卷



 《登科記解題》二十卷



 樂史《廣孝悌一作「新」



 書》五十卷



 危高《孝子拾遺》十卷



 《紹興名臣正論》一卷題瀟湘樵夫序



 《呂頤浩遺事》一卷頤浩出處大概



 《呂頤浩逢辰記卷》一卷頤浩歷官次序



 《朱勝非年表》一卷勝非孫昱上



 《朱勝非行狀》一卷劉岑撰



 《奉神述》一卷真宗制



 史浩《會稽先賢祠傳贊》二卷



 張栻《諸葛武侯傳》一卷



 趙彥博《昭明事實》二卷



 《呂文靖公事狀》一卷不知作者



 王巖叟《韓忠獻公別錄》一卷



 《韓忠獻公家傳》一卷韓琦五世
 孫庚卿作



 呂祖謙《歐公本末》四卷



 《韓莊敏公遺事》一卷韓宗武記



 邵伯溫《邵氏辨誣》三卷



 薛齊誼《六一居士年譜》一卷



 《胡剛中家傳》一卷男胡興宗撰



 黃璞《閩中名士傳》一卷



 岳珂《籲天辨誣》五卷



 李綱等《張忠文節誼錄》一卷



 陳曄《種師道事跡》一卷



 張琰《種師道祠堂碑》一卷



 《談氏家傳》一卷談鑰撰



 王淹《槐庭濟美錄》十卷



 《英顯張侯平寇錄》一卷不知作者



 洪適《五代登科記》一卷



 周鑄《史越王言行錄》十二卷



 《劉氏傳忠錄》三卷劉學裘撰



 《陳瓘墓志》一卷自撰



 《了齋陳先生言行
 錄》一卷陳瓘男正同編



 《趙文定公遺事》一卷不知何人編



 《常諫議長洲政事錄》一卷常安民撰



 《朱文公行狀》一卷黃乾撰



 李□《趙鼎行狀》三卷



 岳珂《鄂國金佗粹編》二十八卷



 吳柔勝《宗澤行實》十卷



 李樸《豐清敏遺事》一卷



 《劉岳李魏傳》二卷張穎撰



 劉球《劉鄜王事實》一十卷



 尹機《宿州事實》一卷



 石茂良《避戎夜話》一卷



 又《靖康錄》一卷



 《中興禦侮錄》一卷



 《皇華錄》一卷



 《南北歡盟錄》一卷



 《裔夷謀夏錄》二卷



 並不知作者



 張師顏《金虜南遷錄》一卷



 張棣《金亮講和事跡》一卷



 洪
 遵《泉志》十五卷



 張甲《浸銅要錄》一卷



 姚康《唐登科記》十五卷



 馬宇《段公別傳》二卷



 張陟《唐年經略志》十卷



 柳玭《柳氏序訓》一卷



 柳珵《柳氏家學》一卷



 李躍《嵐齋集》一卷



 段公路《北戶雜錄》一卷



 鄭暐《蜀記》三卷



 《野史甘露新記》二卷



 《諱行錄》一卷



 《大和野史》三卷



 《逸史》一卷



 《拓跋記》一卷



 《文場盛事》一卷



 《楊妃外傳》一卷



 並不知作者



 蕭叔和《天祚永歸記》一卷



 薛圖存《河南記》二卷



 李綽《張尚書故實》一卷



 劉昶《嶺外錄異》三卷



 王振《汴水滔天錄》一卷



 王權《汴
 州記》一卷



 高若拙《後史補》三卷



 黃彬《莊宗召禍記》一卷



 《晉朝陷蕃記》一卷不知作者



 餘知古《渚宮舊事》十卷



 張昭《太康平吳錄》二卷



 王仁裕《入洛記》一卷



 又《南行記》一卷



 《崔氏登科記》一卷不知作者



 範質《魏公家傳》三卷



 趙普《飛龍記》一卷



 勾延慶《成都理亂記》八卷



 錢儼《戊申英政錄》一卷



 高自若《唐宋泛聞錄》一卷



 《曹彬別傳》一卷曹彬之孫偃撰



 陳承韞《南越記》一卷



 蔣之奇《廣州十賢贊》一卷



 安德裕《滕王廣傳》一卷



 王延德《西州使程記》一卷



 張緒《續錦里耆舊
 傳》十卷



 沉立《奉使二浙雜記》一卷



 路振《乘軺錄》一卷



 李畋《孔子弟子贊傳》六十卷



 又《乖崖語錄》一卷載張詠政績



 張齊賢《洛陽搢紳舊聞記》五卷



 張逵《蜀寇亂小錄》一卷



 曾致堯《廣中臺記》八十卷



 又《綠珠傳》一卷



 許載《吳唐拾遺錄》十卷



 樂史《唐滕王外傳》一卷



 又《李白外傳》一卷



 《洞仙集》一卷



 《許邁傳》一卷



 《楊貴妃遺事》二卷題岷山叟上



 《李昉談錄》一卷李宗諤撰



 《潘美事跡》一卷



 《平蜀錄》一卷



 《國朝名將行狀》四卷



 《議盟記》一卷



 《寇準遺事》一卷



 《丁謂談錄》一卷



 《郭
 贄傳略》一卷



 並不知作者



 任升《梁益記》十卷



 錢惟演《錢俶貢奉錄》一卷



 《王旦遺事》一卷王素撰



 寇瑊《奉使錄》一卷



 王皞《唐餘錄》六十卷



 蔡元翰《唐制舉科目圖》一卷



 劉渙《西行記》一卷



 王曾《筆錄》一卷



 富弼《奉使語錄》二卷



 又《奉使別錄》一卷



 王曙《戴斗奉使錄》一卷



 《燕北會要錄》一卷



 《虜庭雜記》十四卷



 《契丹須知》一卷



 《陰山雜錄》十五卷



 《契丹實錄》一卷



 《學士年表》一卷



 《韓琦遺事》一卷



 《孫沔遺事》一卷



 並不知作者



 歐陽修《歸田錄》八卷



 王起《甘陵誅叛錄》一卷



 趙
 勰《廣州牧守記》十卷



 又《交址事跡》八卷



 曹叔卿《儂智高》一卷



 滕甫《征南錄》一卷



 馮炳《皇祐平蠻記》二卷



 劉敞《使北語錄》一卷



 《宋景文公筆記》五卷《契丹官儀》及《碧雲騢附》



 宋敏求《三川官下記》二卷



 又《諱行後錄》五卷



 《入番錄》二卷



 《春明退朝錄》三卷



 韓正彥《韓琦家傳》十卷



 韓漳《愛棠集》二卷



 趙寅《韓琦事實》一卷



 《杜滋談錄》一卷杜師秦等撰



 李復圭《李氏家傳》三卷



 朱定國《歸田後錄》十卷



 陳昉《北庭須知》二卷



 《王通元經薛氏傳》十五卷



 宋如愚《劍南須知》十卷



 《黃
 靖國再生傳》一卷廖子孟撰



 《曾鞏行述》一卷曾肇撰



 《曾肇行述》一卷楊時撰



 《韓琦別錄》三卷王巖叟撰



 章邦傑《章氏家傳德慶編》一卷



 《胡氏家傳錄》一卷不知作者



 《河南劉氏家傳》二卷劉唐老上



 李遠《青唐錄》一卷



 李格非《永洛城記》一卷



 又《洛陽名園記》一卷



 《趙君錫遺事》一卷趙演撰



 蘇轍《儋耳手澤》一卷



 《穎濱遺老傳》二卷



 蔡京《黨人記》一卷



 吳棫《雞林記》二十卷



 王云《雞林志》三十卷



 《韓文公歷官記》一卷程俱撰



 羅誘一作「羅綺」



 《宜春傳信錄》三卷



 呂希哲《呂氏家塾廣記》一卷



 《安
 燾行狀》一卷榮輯撰



 馬永易《壽春雜志》一卷



 李季興《東北諸蕃樞要》二卷



 何述《溫陵張賢母傳》一卷



 洪興祖《韓子年譜》一卷



 孔傳《闕裏祖庭記》三卷



 又《東家雜記》二卷



 趙令畤《侯鯖錄》一卷



 王襄《南陽先民傳》二十卷



 鄭熊《番禺雜記》三卷



 《範太史遺事》一卷



 《範祖禹家傳》八卷



 並範沖編



 《韓琦定策事》一卷韓肖冑撰



 喻子材《豐公逸事》一卷



 《劉安世譚錄》一卷韓瓘撰



 《種諤傳》一卷趙起撰



 劉棐《孝行錄》二卷



 汪若海《中山麟書》一卷



 《胡瑗言行錄》一卷關注撰



 胡珵《道護
 錄》一卷



 《劉安世言行錄》二卷



 《范純仁言行錄》三卷



 《使高麗事纂》二卷



 《平燕錄》一卷



 《三蘇言行》五卷



 並不知作者



 趙世卿《安南邊說》五卷



 洪適《宋登科記》二十一卷



 董正工《續家訓》八卷



 洪邁《皇族登科題名》一卷



 俞觀能《孝悌類鑒》七卷



 馮忠嘉《海道記》一卷



 《淮西記》一卷



 朱熹《五朝名臣言行錄》十卷



 又《三朝名臣言行錄》十四卷



 《四朝名臣言行錄》十六卷



 《四朝名臣言行續錄》十卷



 並不知何人編



 呂祖謙《閫範》三卷



 費樞《廉吏傳》十卷



 徐度《卻掃編》三卷



 張景儉《嵩
 岳記》三卷



 史願《北遼遺事》二卷



 張隱《文士傳》五卷



 《柳州記》一卷



 《洪崖先生傳》一卷



 《開運陷虜事跡》一卷



 《殊俗異聞集》一卷



 《契丹機宜通要》四卷



 《契丹事跡》一卷



 《古今家誡》二卷



 《南嶽要錄》一卷



 《豪異秘錄》一卷



 《燕北雜錄》一卷



 《遼登科記》一卷



 《三國史記》五十卷



 並不知作者



 高得相《海東三國通歷》十二卷



 金富軾《奉使語錄》一卷



 董弅《誕聖錄》三卷



 王安石《舒王日錄》十二卷



 倪思《北征錄》七卷



 張舜民《郴行錄》一卷



 關耆孫《建隆垂統略》一卷



 張浚《建炎復
 闢平江實錄》一卷



 龔頤正《清江三孔先生列傳譜述》一卷



 邵伯溫《邵氏聞見錄》一卷



 陸游《老學庵筆記》一卷



 陳師道《後山居士叢談》一卷



 僧祖秀《游洛陽宮記》一卷



 李元綱《近世厚德錄》一卷



 安丙《靖蜀編》四卷



 張九成《無垢心傳錄》十二卷



 黎良能《讀書日錄》五卷



 賀成大《濂湘師友錄》三十三卷



 汪藻《裔夷謀夏錄》三卷



 又《青唐錄》三卷



 晁公武《稽古後錄》三十五卷



 又《昭德堂稿》六十卷



 《讀書志》二十卷



 《嵩高樵唱》二卷



 範成大《吳門志》五十卷



 又《攬轡
 錄》一卷



 《驂鸞錄》一卷



 《虞衡志》一卷



 《吳船志》一卷



 洪邁《贅稿》三十八卷



 又《詞科進卷》六卷



 《蘇黃押韻》三十二卷



 張綱《見聞錄》五卷



 吳芾《湖山遺老傳》一卷



 李燾《陶潛新傳》三卷



 又《趙普別傳》一卷



 右傳記類四百一部,一千九百六十四卷。張九成《無垢心傳錄》以下不著錄二十一部,三百十二卷



\end{pinyinscope}