\article{本紀第七}

\begin{pinyinscope}

 真宗二



 六年春二月戊寅,幸飛山雄武營,觀發機石、連弩,遂宴射潛龍園。己卯,以京東西、淮南水災,遣使振恤貧民,平決獄訟。幸北宅視德潤疾。庚辰,以西涼府六穀首領潘
 羅支為朔方軍節度、靈州西面都巡檢使。甲申,封賢懿長公主為鄭國長公主。蕃部葉市族囉埋等內附。己丑,德潤卒。庚寅,屯田員外郎盛梁坐受賕枉法,流崖州。



 三月辛卯朔,欽州言交州八州使黃慶集等來歸。石、隰都巡檢使言綏州東山蕃部軍使拽臼等內屬。己酉,錢種放還山。乙卯,幸惟吉第視疾。戊午,幸元份宮視疾。



 四月,李繼遷寇洪德砦,蕃官慶香、慶等擊走之。以慶香等領刺史。契丹來侵,戰望都縣,副都部署王繼忠陷於
 敵,發河東廣銳兵赴援。辛巳,信國公玄祐薨。



 五月甲午,太白晝見。辛亥,錄望都戰沒將士子孫。癸丑,鎮州副都部署李福坐望都之戰臨陣退衄,削籍流封州。京城疫,分遣內臣賜藥。



 六月丁卯,詔命官流竄沒嶺南者,給緡錢歸葬。豐州瓦窯沒劑、如羅、昧克等族以兵濟河擊李繼遷,敗之。丁丑,隴山西首領禿逋等貢馬,願附大兵擊賊。丁亥,寇準為三司使。復鹽鐵、度支、戶部副使。



 秋七月癸丑,兗王元傑薨。



 八月庚午,太白晝見。辛未,原、渭等州
 言西蕃八部二十五族納質來歸。丙子,詔環、慶秋田經寇踐傷者,頃賜粟十五斛,民被掠者,口賜米一斛。蠲棣州民租十之三。



 九月己丑,蒲端國獻紅鸚鵡。丙申,出內府繒帛,市穀實邊。甲辰,以呂蒙正為太子太師、萊國公。



 十月丁丑,狐出皇城角樓,獲之。戊寅,給軍中傳信牌。



 十一月癸巳,慮囚,雜犯死罪以下遞減一等,杖釋之。苦寒,令諸路休役兵。己亥,閱捧日軍士教三陣於崇政殿。壬寅,幸大相國寺。庚戌,雨木冰。甲寅,有星孛於井、鬼。十二
 月庚申,遣使西北,勞賜將士。甲子,詔求直言。西面部署言李繼遷攻西涼,知府丁維清沒焉。庚午,以李繼隆為山南東道節度使。甲戌,萬安太后不豫,詔求良醫。戊寅,赦天下,死罪減一等,流以下釋之。是歲,西涼府暨龍野馬族、三佛齊、大食國來貢。河北、興元府、遂、郢州大熟。



 景德元年春正月丙戌朔,大赦,改元。丁亥,麟府路言契丹言泥族拔黃三百餘帳內屬。癸巳,幸天駟監,賜從官馬。丙申,京師地震。辛丑,詔民間天象器物、讖候禁書,並
 納所司焚之,匿不言者死。石、隰州言河西蕃部四十五族首領率屬內附。京師地再震。乙巳,廢高州。丁未,京師地復震。壬子,開定州河通漕。



 二月,環、慶部署言西涼府潘羅支集六穀蕃部合擊李繼遷,敗之,繼遷中流矢死。羅支使來獻捷。戊寅,太常卿張齊賢為兵部尚書。冀、益、黎、雅州地震。



 三月,威虜軍守將破契丹於長城口,追北過陽山,斬獲甚眾。柳谷川蕃部入寇、麟,府擊敗之,擒千餘人。己亥,皇太后崩。辛丑,群臣三上表請聽政,不允。乙
 巳,李沆等詣宮門,見帝毀瘠過甚,退上五表求見,言西北軍事方殷,力請聽政,從之。麟府路言敗西人於神堆,破其砦柵。己酉,帝始於崇政殿西廡衰服慟哭見群臣。



 夏四月甲寅,上大行皇太后謚曰明德。群臣三請御正殿,從之。丙辰,邢州地震不止。以溪蠻寧息,民多復業,蠲澧州石門縣租二年。丁卯,以隆暑,休北邊役兵。瀛州地震。



 五月甲申,邢州地連震不止,賜民租之半。蒲端國遣使來貢。丁巳,詔諸路轉運使代還日,在任興除利害、升
 黜能否,凡所經畫事,悉條上以聞。



 六月己未,幸北宅視德欽疾。洪德砦言繼遷部將都尾等率屬歸附。甲子,詔罷川峽、閩、廣州軍貢承天節,自今三千里外者罷之。鎮戎軍言敗戎人於石門川。庚午,德欽卒。洪德砦言蕃部羅泥天王本族諸首領各率屬歸附。趙保忠卒。壬午,暑甚,罷京城工役,遣使賜暍者藥。



 秋七月癸未,班用兵誅賞格。丙戌,李沆薨。庚寅,以翰林侍讀學士畢士安為吏部侍郎、參知政事。庚子,益都民李仁美、國凝母皆百餘歲,詔
 賜粟帛。



 八月,涇原部署言擊萬子軍主族帳,斬首二百餘級。己未,以畢士安、寇準並平章事,宣徽南院使王繼英為樞密使,同知樞密院事馮拯、陳堯叟並簽署樞密院事。壬申,詔常參官二人共舉州縣官可任幕職者一人。丙子,以保平軍節度石保吉為武寧軍節度、同平章事。庚辰,遣使廣南東、西路疏決系囚,犒勞軍校父老,訪民間便宜。



 九月癸未,罷北面繼御劍內臣,以劍屬主將。丙戌,令諸路轉運使考察官吏能否,己丑,詔翰林學
 士承旨宋白等舉文武官可任藩郡者各一人。丁酉,召宰相議親征。契丹耶律吳欲來降。宋州汴水決。乙巳,置祁州。河決澶州,遣使具舟濟民,給以糧餉。



 閏月乙卯,詔河北吏民殺契丹者,所至援之,仍頒賞格。壬申,江南旱,遣使決獄,訪民疾苦,祠境內山川。癸酉,明德皇太后殯沙臺。北平砦、威虜軍合兵大破契丹。乙亥,參知政事王欽若判天雄軍府兼都部署。契丹統軍撻覽率眾攻威虜、順安軍,三路都部署擊敗之,斬偏將,獲其輜重。又攻北
 平砦及保州,復為州、砦兵所敗。撻覽與契丹主及其母並眾攻定州,宋兵拒於唐河,擊其游騎。契丹駐陽城澱,因王繼忠致書於莫州石普以講和。丙子,以天雄軍都部署周瑩為駕前貝、冀路都部署,侍衛馬軍都指揮使葛霸為駕前邢、洺路都部署。己卯,高繼勛率兵擊敗契丹數萬騎於岢嵐軍。



 冬十月壬午,詔修葺歷代聖賢陵墓。癸未,麟府路率部兵入朔州,破大狼水砦。乙酉,令漕運所經州軍長吏兼輦運事。戊子,祔明德皇后於太廟。
 庚寅,命張齊賢兼青、淄、濰安撫使,丁謂兼鄆、齊、濮安撫使。癸巳,幸故鄭國長公主第。乙未,詔王超等率兵赴行在。丁酉,詔魏能、張凝、田敏屯定州。癸卯,以廝鐸督為朔方軍節度、靈州西面巡檢、西涼府六穀大首領。保、莫州、威虜、岢嵐軍及北平砦皆擊敗契丹。既而王繼忠上言契丹請和,命閣門祗候曹利用往答之。丁未,以雍王元份為東京留守。己酉,置龍圖閣待制。



 十一月辛亥,太白晝見。乙卯,遣使撫河北。契丹攻瀛州,知州李延渥率兵
 敗之,殺傷十餘萬眾,遁去。官吏進秩、賜物有差。己未,遣使安撫河東諸州。契丹逼冀州,知州王嶼擊走之。甲子,校獵近郊。丙寅,遣使安集河北流民。戊辰,以山南東道節度、同平章事李繼隆為駕前東面排陣使,武寧軍節度、同平章事石保吉為駕前西面排陣使。石州地震。庚午,車駕北巡。司天言:日抱珥,黃氣充塞,宜不戰而卻。癸酉,駐蹕韋城縣。甲戌,寒甚,左右進貂帽毳裘,卻之曰:「臣下皆苦寒,朕安用此?」王繼忠數馳奏請和,帝謂宰相曰:「
 繼忠言契丹請和,雖許之,然河冰已合,且其情多詐,不可不為之備。」契丹兵至澶州北,直犯前軍西陣,其大帥撻覽耀兵出陣,俄中伏弩死。丙子,帝次澶州。渡河,幸北砦,御城北樓,召諸將撫慰。鄆州得契丹諜者,斬之。戊寅,曹利用使契丹還。十二月庚辰朔,日有食之。契丹使韓□巳來講和。辛巳,遣使安撫河北、京東。壬午,幸城南臨河亭,賜鑿凌軍綿襦。癸未,幸北砦,又幸李繼隆營,命從官將校飲,犒賜諸軍有差。詔諭兩京以將班師。甲申,契丹
 使姚東之來獻御衣、食物。乙酉,御行營南樓觀河,遂宴從官及契丹使。丙戌,遣使撫諭懷、孟、澤、潞、鄭、滑等州,放強壯歸農。遣監西京左藏庫李繼昌使契丹定和,戒諸將勿出兵邀其歸路。丁亥,遣使安集河北流民,瘞暴骸。以閣門祗候曹利用為東上閣門使、忠州刺史。戊子,幸北砦勞軍,召李繼隆、石保吉宴射行宮西亭。壬辰,赦河北諸州死罪以下,民經蹂踐者給復二年,死事官吏追錄子孫。癸巳,雍王元份疾,命參知政事王旦權東京留
 守。甲午,車駕發澶州,大寒,賜道傍貧民襦褲。乙未,契丹使丁振以誓書來。丁酉,契丹兵出塞。戊戌,至自澶州。己亥,幸雍王元份宮視疾。辛丑,錄契丹誓書頒河北、河東諸州。癸卯,遣使撫問河北東、西路官吏將卒,訪察功狀。甲辰,改威虜諸軍名。戊申,詔恤河北傷殘。是歲,交州、西涼府、西、高、豐、甘、沙州、占城、大食、蒲端、龜茲國來貢。江南東、西路饑,陜、濱、棣州蝗害稼,命使振之。



 二年春正月庚戌朔,以契丹講和,大赦天下,非故鬥殺、
 放火、強盜、偽造符印、犯贓官典、十惡至死者悉除之。壬子,放河北諸州強壯歸農,令有司市耕牛給之。癸丑,罷諸路行營,合鎮、定兩路都部署為一。乙卯,罷北面部署、鈐轄、都監、使臣二百九十餘員。振河北饑。遣監察御史朱摶赴德清軍收瘞戰沒遺骸,致祭。罷江、淮、荊、浙增榷酤錢。丙辰,幸雍王元份宮視疾。甲子,詔淮南以上供軍儲振饑民。戊辰,以天平軍節度使王超為崇信軍節度。省河北戍兵十之五,緣邊三之一。所在量軍儲饋給,勿
 調民飛挽。癸酉,幸李繼隆第視疾。京西民轉送軍儲者賜租十二。丁丑,詔河北轉運使察官屬不任職者以名聞。戊寅,取淮、楚間踏犁式頒之河朔。



 二月,嘉、邛州鑄大鐵錢。置霸州、安肅軍榷場。癸未,李繼隆卒。甲申,定入粟實邊授官等級。乙酉,遣使安撫交州。甲午,詔緣邊得契丹馬牛,悉縱還之,沒蕃漢口歸業者,給資糧。弛邊民鐵禁。環州言戎人入寇,擊走之,俘其軍主。癸卯,遣太子中允孫僅等使契丹。丁未,呂蒙正對便殿。



 三月甲寅,御試
 禮部貢舉人。戊午,鄭州防禦使魏能坐歸師不整,責授右羽林將軍。庚申,禁邊民入外境掠奪。



 夏四月,賜進士李迪等瓊林宴。丁酉,樞密直學士劉師道責授忠武軍行軍司馬,右正言、知制誥陳堯咨單州團練使,俱坐考試不公。己亥,葺河北城池。癸卯,置資政殿學士,以王欽若為之。馮拯為參知政事。甲辰,以寧國軍留後、駙馬都尉吳元扆為武勝軍節度。戎人寇環州,擊敗之,執其酋慶,請戮之,詔釋其罪,配淮南。



 五月戊申,幸國子監。丁
 巳,司天少監史序上《乾元寶典》。己未,幸元份宮視疾。庚申,御試河北舉人。丁卯,宴近臣於資政殿。餞種放游嵩山。癸酉,詔天下榷利勿增羨為額。



 六月丁丑,詔勸學。幸諸王宮。己卯,命法直官用士人。己丑,曹州民趙諫、趙諤以恐喝贓鉅萬伏誅。辛卯,以趙德明歸款,諭河西諸蕃各守疆界。高瓊求板本經史,詔給之。



 秋七月庚戌,劉質進《兵要論》,召試中書。甲子,詔復賢良方正能直言極諫等六科。



 八月戊寅,雍王元份薨。丙戊,有司上新定權衡
 法。遣內臣奉安太祖聖容於揚州建隆寺。丁亥,翰林學士晁迥先為鄆王元份留守官屬,坐輔導無狀,責授右司郎中。辛丑,幸南宮及恭孝太子宮。有星孛於紫微。



 九月丁未,以向敏中為鄜延路都部署。庚戌,淮南旱,詔轉運使疏理系囚。癸亥,三司上《新編敕》。群臣三表上尊號,不允。庚午,幸興國寺傳法院觀新譯經。辛未,命近臣慮開封府系囚。壬申,詔荊湖溪峒民為蠻人所掠而歸者,勿限年月,給還舊產。



 冬十月庚辰,丁謂上《景德農田編
 敕》。乙酉,畢士安薨。丙戌,遣職方郎中韓國華等使契丹。



 十一月戊申,詔翰林侍講學士邢昺等舉堪為學官者十人。丙辰,享太廟。丁巳,祀天地於圜丘,大赦。庚申,大宴含光殿。癸亥,寇準加中書侍郎兼工部尚書,楚王元佐為右衛上將軍,彭城郡王元偓進封寧王,安定郡王元稱進封舒王,曹國公元儼進封廣陵郡王,安定郡公惟吉加同平章事。癸酉,契丹使來賀承天節。十二月辛巳,置資政殿大學士,以王欽若為之。癸未,以高瓊為忠武
 軍節度,葛霸為昭德軍節度。對京畿父老於長春殿,賜帛有差。契丹遣使賀明年正旦。是歲,夏州、西涼府、邛部川蠻來貢。淮南、兩浙、荊湖北路饑,京東蝻生,閩颶風不害稼,遣使分振。



 三年春正月丁巳,親釋逋負系囚。振畿縣貧民,收瘞遺骸。丁卯,詔緣邊歸業民給復三年。辛未,置常平倉。



 二月甲戌,幸北宅省德恭疾。乙亥,詔京東西、淮南、河北振乏食客戶。己卯,謁明德皇后攢宮,賜守奉人緡帛。甲申,禁
 民開近陵域地。以宋州為應天府。丁亥,王繼英卒。戊戌,以中書侍郎兼工部尚書、平章事寇準為刑部尚書,左丞、參知政事王旦為工部尚書、平章事。己亥,王欽若、陳堯叟並知樞密院事。翰林學士趙安仁參知政事。樞密都承旨韓崇訓、馬知節並簽署樞密院事。



 三月乙巳,客星出東南。辛亥,免隨州光化民貸糧。己未,詔儆諫臣悉心獻替。



 夏四月癸酉,幸秦國長公主第。丙子,幸開寶寺,遂幸御龍直班院,觀教閱弓刀。又幸左騏驥院,賜從官
 馬、群牧使等器幣。還,幸崇文院觀圖籍,賜編修官金帛有差。己卯,置清平、宣化二軍。乙酉,置河北緣邊安撫使、副、都監於雄州。壬辰,命使巡撫益、利、梓、夔、福建諸路,決獄及犒設將吏、父老。乙未,種放賜告歸終南山。己亥,遣使巡撫江、浙路。



 五月壬寅,日當食不虧。周伯星見。辛亥,置京東五路巡檢。丁巳,幸北宅視德恭疾。己未,德恭卒。西涼府廝鐸督部落多疾,賜以藥物。渭川妙娥族三千餘帳內附。復置高州。



 六月丙子,群臣固請聽樂,從之。詔
 三班考較使臣以七年為限。知廣州凌策請發兵定交□止亂,帝以黎桓素修職貢,不欲伐喪,命遵前詔安撫。戊寅,罷兩川稅課金二分。乙未,汴水暴漲,賜役兵錢。丙申,遣使振應天府水災及瘞溺死者。



 秋七月壬寅,減鄜延戍兵。乙巳,太白晝見。庚戌,詔渭州、鎮戎軍收獲蕃部牛送給內地耕民。壬子,賜廣南《聖惠方》,歲給錢五萬,市藥療病者。邵曄上邕州至交址水陸路及控制宜州山川等圖,帝曰:「祖宗闢土廣大,唯當慎守,不必貪無用地,苦
 勞兵力。」甲子,大宴含光殿,始用樂。丙寅,大風,遣中使視稼。



 八月甲戌,閱太常新集雅樂。丁丑,幸寶相院。戊寅,詔川峽戍兵二年者代之。庚辰,工部侍郎董儼坐躁競傾狡,責授山南東道行軍司馬。



 九月甲寅,宴射含芳園。丙辰,御試賢良方正直言極諫科。壬戌,幸元偓宮視疾。甲子,置諸陵齋宮。乙丑,放西州納質人。夏州趙德明奉表歸款。



 冬十月庚午,以趙德明為定難軍節度兼侍中,封西平王。甲午,兩浙轉運使姚鉉坐不法除名,為連州
 文學。丁酉,葬明德皇后。



 十一月壬寅,周伯星再見。十二月癸酉,太白晝見。戊寅,高瓊卒。乙酉,狩近郊,以親獲兔付有司薦廟。戊子,詔牛羊司畜有孳乳者放牧勿殺。辛卯,朝陵,緣路禁樂。壬辰,幸秦國長公主第,又幸北宅視德鈞疾。是歲,西涼府龕穀十族、高溪州、風琶溪洞諸蠻酋來貢。京東西、河北、陜西饑,振之。博州蝝,不為災。



 四年春正月己亥朔,御朝元殿受朝。詔京畿系囚流以下減一等。甲辰,以陳堯叟為東京留守。德鈞卒。乙巳,契
 丹使辭歸國。以丁謂為隨駕三司使。己未,車駕發京師。庚申,次中牟縣,除逋負,釋系囚,賜父老衣幣,所過如之。王顯卒。丙寅,次永安鎮。丁卯,帝素服詣諸陵。減西京及諸路系囚罪,如己亥詔。置永安縣及三陵副使、都監。



 二月己巳,幸西京,經漢將軍紀信塚、司徒魯恭廟,贈信太尉、恭太師。命吏部尚書張齊賢祭周六廟。詔從官先塋在洛者賜告祭拜。癸酉,詔西京建太祖神御殿。置國子監、武成王廟。甲戌,幸上清宮。詔賜酺三日。辛巳,錄唐白居
 易孫利用為河南府助教。壬午,幸呂蒙正第。甲申,御五鳳樓觀酺,召父老五百人,賜飲樓下。丁亥,幸元偓宮。戊子,葺周六廟。加號列子。增封唐孝子潘良瑗及其子季通墓,仍禁樵採。庚寅,詔河南府置五代漢高祖廟。辛卯,車駕發西京。甲午,次鄭州,遣使祀中岳及周嵩、懿二陵。丁酉,賜隱士楊璞繒帛。



 三月己亥,至自西京。甲辰,謁啟聖院太宗神御殿。癸丑,趙德明遣使來謝廩給,因貢駝馬,優賜答之。丁巳,詔天下收瘞遺骸,致祭。庚申,蠲河南
 府倉庫吏逋負芻糧緡帛四十五萬。



 夏四月癸酉,詔嶺南官除赴以時,以避炎瘴。辛巳,皇后郭氏崩。甲午,詔榷酤不得增課。



 五月丙申朔,日有食之。辛亥,有司上大行皇后謚曰莊穆。減並、代戍兵屯河東,以省饋運。戊午,幸元偁宮視疾。兗州增二千戶守孔子墳。



 閏月戊辰,減劍、隴等三十九州軍歲貢物,夔、賀等二十七州軍悉罷之。己巳,幸秦國長公主第省疾。壬申,御試制科舉人。丙戌,詔張齊賢等各舉供奉官、侍禁、殿直有謀略武乾知邊
 事者二人。癸巳,詔開封府斷獄,雖被旨,仍覆奏。



 六月,盛暑,減京城役工日課之半。丁未,令翰林講讀、樞密直學士各舉常參官一人充御史。司天監言五星聚而伏於鶉火。乙卯,葬莊穆皇后。



 秋七月丁卯,莊穆皇后祔別廟。庚午,置靈臺令。壬申,增置開封府判官、推官各一員。甲戌,宜州兵亂,軍校陳進殺知州劉永規等,劫判官盧成均為首。詔閣門使曹利用等討之。乙亥,交州來貢,賜黎龍廷《九經》及佛氏書。辛巳,以龍廷為靜海軍節度、交址
 郡王,賜名至忠。癸巳,復置諸路提點刑獄。



 八月壬寅,幸大相國寺,遂幸崇文院觀書,賜修書官器幣。又幸內藏庫。丁未,中書門下言莊穆皇后祥除已久,秋宴請舉樂,不允。己酉,頒宜州立功將士賞格。益州地震。辛亥,賜文宣王四十六世孫聖祐同學究出身。壬子,邢昺加工部尚書。中書門下再表請秋宴聽樂,又不允。丙辰,涇原路言瓦亭砦地震。丁巳,詔王旦、楊億等修太祖、太宗史。置龍圖閣直學士,以右諫議大夫杜鎬為之。丁謂上《景德
 會計錄》。



 九月己巳,賜交址郡王印及安南旌節。壬申,賜畿縣《聖惠方》。丁亥,幸舒王宮視疾。辛卯,賜監修國史王旦宴。壬辰,日上有五色雲。



 冬十月甲午朔,日當食,雲陰不見。曹利用破賊於象州,擒盧成均,斬陳進。優賜將士,利用等進秩、賜物有差。乙巳,頒考試進士新格。祠祭置監祭使二員,以御史充。詔翰林學士晁迥等舉常參官可知大藩者二人。丁未,升象州為防禦。甲寅,詔:宜、柳、象州、懷遠軍死罪以下,非十惡、謀故鬥殺、官吏犯枉法贓
 者,並原之。廣南東、西路雜犯死罪以下遞減一等,脅從受署者勿理。蠲宜、柳、象州、懷遠軍丁錢及夏秋租,桂、昭州秋租。乙卯,毀諸道官司非法訊囚之具。



 十一月戊辰,日南至,御朝元殿受朝。曹利用等言招安賊黨,其饋賊食物者,請追捕減死論,詔釋不問。十二月己亥,賜近臣、契丹錦綺綾縠等物。癸卯,廢兗州鐵冶。己未,甘州僧翟大秦等獻馬,給其直。是歲,河西六谷、夏州、沙州、大食、占城、蒲端國、西南蕃溪峒蠻來貢。雄州、安肅、廣信饑。宛丘、
 東阿、須城縣蝗,不為災。諸路豐稔,淮、蔡間麥斗十錢,粳米斛二百。



 大中祥符元年春正月乙丑,有黃帛曳左承天門南鴟尾上,守門卒塗榮告,有司以聞。上召群臣拜迎於朝元殿啟封,號稱天書。丁卯,紫雲見,如龍鳳覆宮殿。戊辰,大赦,改元,群臣加恩,賜京師酺。幽州旱,求市麥種;夏州饑,請易粟,並許之。己巳,詔黎、雅、維、茂四州官以瘴地二年一代。甲戌,大雪,停汴口、蔡河夫役。戊寅,蠲畿內貸糧。己
 卯,詔以天書之應,申儆在位。乙酉,制加交址郡王黎至忠功臣食邑。



 二月壬辰,禦乾元門觀酺,賜父老千五百人衣服、茶彩。丁酉,分遣中使六人錫邊臣宴。丙午,申明非命服勿服銷金及不許以金銀為箔之制。



 三月甲戌,兗州父老千二百人詣闕請封禪;丁卯,兗州並諸路進士等八百四十人詣闕請封禪;壬午,文武官、將校、蠻夷、耆壽、僧道二萬四千三百七十餘人詣闕請封禪,不允。自是表凡五上。



 夏四月甲午,詔以十月有事於泰山,遣
 官告天地、宗廟、岳瀆諸祠。乙未,以知樞密院事王欽若、參知政事趙安仁為泰山封禪經度制置使。丙申,以王旦為封禪大禮使,馮拯、陳堯叟分掌禮儀使。庚子,幸元偁宮視疾。壬寅,御試禮部貢舉人。丙午,作昭應宮。戊申,幸秦國長公主第省疾。又幸晉國、魯國長公主第,並賜白金千兩、彩二千匹。曹、濟州、廣濟軍耆老二千二百人詣闕請臨幸。



 五月壬戌,王欽若言泰山醴泉出,錫山蒼龍見。丙子,詔瘞汴、蔡、廣濟河流尸暴骸,仍致祭。丁丑,幸
 南宮視惟能疾。壬午,詔緣路行宮舊屋止加塗塈,毋別創。癸未,置天書儀衛使副、扶侍使都監、夾侍,凡有大禮即命之。詔離京至封禪以前不舉樂,所經州縣勿以聲伎來迓。甲申,放後宮一百二十人。戊子,詔:除乘輿供帳,存於禮文者如舊,自今宮禁中外進奉物,勿以銷金文繡為飾。



 六月乙未,天書再降於泰山醴泉北。丁酉,詔宮苑皇親臣庶第宅飾以五彩,及用羅制幡勝、繒帛為假花者,並禁之。壬寅,迎泰山天書於含芳園,云五色見,俄
 黃氣如鳳駐殿上。庚戌,曲赦兗州系囚流罪以下。辛亥,群臣表上尊號曰崇文廣武儀天尊道寶應章感聖明仁孝皇帝。



 秋七月庚申,太白晝見。丙寅,詔諸州市上供物,非土地所宜者罷之。



 八月己丑,上太祖尊謚曰啟運立極英武聖文神德玄功大孝皇帝,太宗曰至仁應道神功聖德文武大明廣孝皇帝。庚寅,詔東封道路軍馬毋犯民稼,開封府毋治道役民。庚子,置河東緣邊安撫司。乙巳,黔州言磨嵯、洛浦蠻首領龔行滿等率族二千
 三百人內附。己酉,王欽若獻芝草八千餘本。



 九月戊午,令有司勿奏大闢案。岳州進三脊茅。庚申,以向敏中權東京留守。甲子,奉天書告太廟,悉陳諸州所上芝草、嘉禾、瑞木於仗內。戊辰,幸元偓宮視疾。壬申,知晉州齊化基坐貪暴削籍,流崖州。乙亥,幸潛龍園宴射。丁丑,幸惟吉宮視疾。戊寅,西京諸州民以車駕東巡貢獻召對,勞賜之。己卯,以馬知節為行宮都部署。庚辰,趙安仁獻五色金玉丹、紫芝八千七百餘本。乙酉,親習封禪儀於崇
 德殿。



 冬十月戊子,上御蔬食。庚寅,以巡幸,置考制度使、副,凡巡幸則命之。是夕,五星順行同色。辛卯,車駕發京師,扶侍使奉天書先道。丙申,次澶州,宴周瑩於行宮。戊戌,許、鄆、齊等州長吏赴泰山陪位。辛丑,駐蹕鄆州,神光起昊天玉冊上。甲辰,詔扈從人毋壞民舍、什器、樹木。丁未,法駕入乾封縣奉高宮。戊申,王欽若等獻泰山芝草三萬八千餘本。己酉,五色雲起嶽頂。庚戌,法駕臨山門,黃雲覆輦,道經險峻,降輦步進。先夕大風,至是頓息。辛
 亥,享昊天上帝於圜臺,陳天書於左,以太祖、太宗配。帝袞冕奠獻,慶雲繞壇,月有黃光。命群臣享五方帝諸神於山下封祀壇,上下傳呼萬歲,振動山谷。降谷口,日有冠戴,黃氣紛鬱。壬子,禪社首,如封祀儀。紫氣下覆,黃光如星繞天書匣。縱四方所獻珍禽奇獸。還奉高宮,日重輪,五色雲見。作會真宮。癸丑,御朝覲壇之壽昌殿,受群臣朝賀。大赦天下,常赦所不原者咸赦除之。文武並進秩。賜致仕官本品全奉一季,京朝官衣緋綠十五年者
 改賜服色。令開封府及所過州軍考送服勤詞學、經明行修舉人,其懷材抱器淪於下位,及高年不仕德行可稱者,所在以聞。三班使臣經五年者與考課。兩浙錢氏、泉州陳氏近親,蜀孟氏、湖南馬氏、荊南高氏、廣南河東劉氏子孫未食祿者,聽敘用。賜天下酺三日。改乾封縣為奉符縣。泰山七里內禁樵採。大宴穆清殿。又宴近臣、泰山父老於殿門,賜父老時服、茶帛。甲寅,復常膳。次太平驛,賜從官闢寒丸、花茸袍。丙辰,次兗州,以州為大都
 督府。



 十一月戊午,幸曲阜縣,謁文宣王廟,靴袍再拜。幸叔梁紇堂。近臣分奠七十二弟子。遂幸孔林,加謚孔子曰玄聖文宣王,遣官祭以太牢,給近便十戶奉塋廟,賜其家錢三十萬,帛三百匹。以四十六世孫聖祐為奉禮郎,近屬授官、賜出身者六人。追謚齊太公曰昭烈武成王,令青州立廟;周文公曰文憲王,曲阜縣立廟。辛酉,賜諸蕃使袍笏。壬戌,次中都縣,幸廣相寺。癸亥,次鄆州,幸開元寺。丁卯,賜曲阜孔子廟經史。辛未,幸河瀆廟,加封。
 癸酉,曲宴永清軍節度使周瑩,賜兵士緡錢。丁丑,帝至自泰山,奉天書還宮。壬午,詔以正月三日為天慶節。甲申,命王旦奉上太祖、太宗謚冊,親享太廟。乙酉,大宴含光殿。十二月辛卯,禦乾元殿受尊號。庚子,葛霸卒。辛丑,王旦加中書侍郎兼刑部尚書,楚王元佐加太傅,寧王元偓為護國軍節度,舒王元偁為平江、鎮江軍節度,並兼侍中;廣陵郡王元儼進封榮王,安定郡公惟吉為威德軍節度,餘進秩有差。癸卯,幸上清宮、景德開寶寺。王
 欽若加禮部尚書。甲辰,張齊賢為右僕射,溫仲舒、寇準並為戶部尚書,王化基、邢昺、郭贄並為禮部尚書。詔天下宮觀陵廟,名在地志,功及生民者,並加崇飾。戊申,以德雍、德文、德存、惟正、惟忠、惟敘、惟和、惟憲並領諸州刺史,允升、允言、允成、允寧、允中並為各衛將軍。庚戌,幸元偁宮視疾。又幸元偓宮。辛亥,交址郡王黎至忠加同平章事。壬子,幸元偁宮。契丹使上將軍蕭智可等來賀。是歲,西涼府、甘州、三佛齊、大食國、西南蕃等來賀封禪。諸
 路言歲稔,米斗七八錢。



 二年春正月癸亥,以封禪慶成,賜宗室、輔臣襲衣、金帶、器幣。乙丑,置內殿承制。戊辰,詔:「誘人子弟析家產,或潛舉息錢,輒壞墳域者,令所在擒捕流配。」庚午,詔:「讀非聖之書及屬辭浮靡者,皆嚴譴之。已鏤板文集,令轉運司擇官看詳,可者錄奏。」乙酉,以陜西民饑,遣使巡撫。



 二月己丑,改定入內內侍省內侍名職。壬辰,詔立曲阜縣孔子廟學舍。乙未,賜撫州高年黃泰粟帛。甲辰,蠲同、華民
 租。乙巳,幸大相國等寺、上清宮祈雨。戊申,遣使祠太乙,祀玄冥。己酉,雨。癸丑,禁毀金寶塑浮屠像。甲寅,以丁謂為三司使。



 三月丙辰,日當食,陰晦不見。辛未,賜京城酺。己卯,左屯衛將軍允言坐稱疾不朝,降太子左衛率。



 夏四月戊子,升州火,遣御史訪民疾苦,蠲被火屋稅。己丑,餞種放還山。乙未,河北旱,遣使祠北嶽。己亥,以丁謂為修昭應宮使。壬寅,詔禁中外群臣非休暇無得群飲廢職。詔醫官院處方並藥賜河北避疫邊民。丙午,試服勤
 詞學、經明行修國監生。丁未,振陜西民饑。五月乙卯,追封孔子弟子七十二人。罷韶州獻頻婆果。丁卯,遣使陜西決獄,流罪以下減一等,死罪情可憫者上請。庚辰,陜西旱,遣使禱太平宮、后土、西嶽、河瀆諸祠。代州地震。



 六月乙酉,頒幕職、州縣官招集戶口賞條。甲午,幸昭應宮,賜修宮使器幣。辛卯,保州增屯田務兵三百人。戊戌,麟府言社慶族依唐龍鎮為援,數擾別部,請出兵襲之。帝曰:「均吾民也。」不許。壬寅,詔量留五坊鷹鶻,備諸王從時
 展禮,餘悉縱之。罷邕、宜州歲貢藥箭。庚戌,御試東封路服勤詞學、經明行修貢舉梁固等九十二人。



 秋七月甲寅,詔張齊賢等各舉才堪御史者一人。丁巳,置糾察在京刑獄司。辛酉,復以萬安宮為滋福殿。己巳,幸惟吉宮視疾。辛未,以昭應宮為玉清昭應宮。乙亥,蠲京東徐、濟七州水災田租。戊寅,詔孔子廟配享魯史左丘明等十九人加封爵。庚辰,蠲天下封禪赦前逋負千二百六十六萬緡。



 八月丙戌,京東惠民河溢,居民避水所過津渡,
 戒有司勿算。甲辰,西南蕃龍漢□堯來貢,賀東封,加漢□堯寧德大將軍。



 九月戊午,賜秦州被水民粟,人一斛。壬戌,合鎮、定部署為一。甲子,浚汴口。命工部侍郎馮起為契丹國信使。乙丑,幸潛龍園宴射。甲戌,遣使賜戎、瀘軍民闢瘴藥。乙亥,無為軍言大風拔木,壞城門、營壘、民舍,壓溺者千餘人。詔內臣恤視,蠲來年租,收瘞死者,家賜米一斛。丁丑,發官廩振鳳州水災。



 冬十月癸未,優賞寧朔軍士。戊子,詔江、浙運糧兵卒經冬停役兩月。甲午,詔天
 下置天慶觀。甲辰,兗州霖雨害稼,振恤其民。



 十一月丙辰,作《文武七條》戒官吏。甲子,詔諸路官吏蠹政害民,轉運使、提點刑獄官不舉察者坐之。癸酉,蕃部阿黎等來朝貢,授阿黎懷化司戈。十二月辛巳,詔:晉國大長公主喪,罷承天節上壽及明年元旦朝會。交州黎至忠貢馴犀。乙未,幸惟吉宮視疾。辛丑,丁謂上《封禪朝覲祥瑞圖》,劉承珪上《天書儀仗圖》。甲辰,幸惟吉宮視疾。契丹國母蕭氏卒,輟視朝。是歲,於闐、西涼府、西南蕃羅巖州蠻來
 貢。雄州蟲食苗即死,遣使振恤。



 三年春正月丁巳,賜建安軍父老江禹錫粟帛。



 二月乙酉,丁謂請承天節禁屠宰刑罰,從之。癸巳,交州黎至忠卒,大校李公蘊自稱留後。已亥,禁方春射獵,每歲春夏,所在長吏申明之。辛丑,以張齊賢判河陽。



 閏月辛亥,帝御文德殿,群臣入閣。甲寅,冬官正韓顯符上新造銅候儀。乙卯,詔轉運司貸恤黎州夷人。丁卯,幸開封府射堂宴射,賜開封府將吏器幣。戊辰,詔東京、畿內死罪以下
 遞減一等。將吏逮事太宗藩府者並賜予。赤縣父老本府宴犒,年九十者授攝官,賜粟帛終身;八十者爵一級。甲戌,以射堂為繼照堂。丁丑,召宰臣於宜聖殿,謁太宗聖容、玉皇像。戊寅,幸韓國長公主第視疾。



 三月壬辰,以權靜海軍留後李公蘊為靜海軍節度,封交址郡王,賜衣帶、器幣。丙申,幸石保吉第視疾。辛丑,詔戎、瀘州給復一年,艱食者振之。



 夏四月辛亥,左屯衛將軍允言坐狂率,責授太子左衛副率。壬子,石保吉卒。乙卯,陜西民疫,
 遣使繼藥賜之。丁巳,詔中書以五月一日進中外文武升朝官及奉使歲舉官名籍。辛酉,賜泰山隱士秦辨號貞素先生,放還山。甲子,契丹國母葬,廢朝,禁邊城樂。甲戌,加王旦兵部尚書,知樞密院事王欽若戶部尚書,陳堯叟工部尚書。



 五月己卯,幸惟吉宮視疾。壬午,以西涼府覓諾族瘴疫,賜藥。丙戌,惟吉卒。辛丑,京師大雨,平地數尺,壞廬舍,民有壓死者,賜布帛。



 六月庚戌,邊臣言契丹饑,來市糴,詔雄州糴粟二萬石振之。河中府父老千
 餘人請祀后土,不許。丙辰,頒天下《釋奠先聖廟儀》並《祭器圖》。詔前歲陜西民饑,有鬻子者,官為購贖還其家。壬戌,幸邢昺第視疾,賜金帛。乙丑,幸元偁宮視疾。



 秋七月丙申,溫仲舒卒。己亥,以右丞向敏中為工部尚書、資政殿大學士。置龍圖閣學士,以直學士杜鎬為之。詔南宮北宅大將軍以下,各勤講肄,諸子十歲以上並受經學書,勿令廢惰。辛丑,文武官、將校等三上表請祠汾陰后土。



 八月丁未朔,詔明年春有事於汾陰,州府長吏勿以
 修貢助祭煩民。戊申,陳堯叟為祀汾陰經度制置使。己酉,王旦為祀汾陰大禮使,王欽若為禮儀使。庚戌,詔汾陰路禁弋獵,不得侵占民田,如東封之制。辛亥,以江南旱,詔轉運使決獄。壬子,幸元偁宮視疾。升、洪、潤州屢火,遣使存撫,祠境內山川。戊午,賜占城國主馬及器甲。庚申,幸天駟監,賜從官馬。解州池鹽不種自生。辛酉,給鄆州牧馬草地還民。甲子,罷江、淮和糴,所在系囚遞減一等,盜穀食者量行論決。丁卯,群臣五表上尊號,不許。戊
 辰,詔升、洪、揚、廬州長吏兼安撫使。甲戌,以澄州團練使朱能為左龍武軍大將軍。乙亥,河中府父老千七百人來迎,上勞問之,賜以緡帛。



 九月癸未,賜錢三十萬給故盧多遜子葬其父母。丁亥,作《宗室座右銘》賜諸王。華州言父老二千餘人請幸西嶽。癸巳,杖殺入內高品江守恩於鄭州,知州俞獻卿坐論救削一任。乙未,幸崇真資聖院視吳國長公主疾。甲辰,內出《綏撫十六條》,頒江、淮南安撫使。



 冬十月辛亥,契丹使耶律寧告征高麗。河中
 民獲《靈寶真文》。庚申,丁謂等上《大中祥符封禪記》。



 十一月庚寅,遣內臣奉安宣祖、太祖聖容於二陵。乙未,甘州回鶻來貢。己亥,幸太一宮。陜州黃河清。十二月,陜州黃河再清。庚戌,集賢校理晏殊獻《河清頌》。癸丑,詔天下貧民及漁採者過津渡勿算。乙卯,告太廟。詔自今謁廟入東偏門。以資政殿大學士向敏中權東京留守。丁巳,翰林學士李宗諤等上《諸道圖經》。辛酉,謁玉清昭應宮。丙寅,詔沙門島流人特給口糧。己巳,作《奉天庇民述》示宰
 相。禁扈從人燔道路草木。辛未,以太宗御書賜交州李公蘊。是歲,龜茲、占城、交州來貢。陜西饑。江、淮南旱。



\end{pinyinscope}