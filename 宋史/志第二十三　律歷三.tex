\article{志第二十三 律歷三}

\begin{pinyinscope}

 應天乾元儀天歷



 步五星



 歲星總:七十九
 萬七千九百三十一、秒五。《乾元》率二十三萬四千五百三十五、秒五千七百二十五,《儀天》木星周率四百二萬八千五百八十七、秒七千五百六十。



 平合:三百九十八日、八千八百五十七、秒二十八。《乾元》餘二千五百五十五、秒八千六百二十五,約分八十七。《儀天》餘八千七百八十七、秒七千五百六十。二歷平合皆謂之周日,數同《應天》。



 變差:空、秒一十六。《乾元》差二十
 八、秒九千四百二十二半,秒母一萬。《儀天》歲差九
 十八、秒九千五百。上限二百五度,下限一百六十度、二十五分、秒六十三。



 熒惑總:一百五十六萬一百五十二、秒三。《
 乾元》率四十五萬八
 千五百九十二、秒九千一百八十三、十四。《儀天》火星周率七百八十七萬七千一百九十一、秒一千一百。



 平合:七百七十九日、九千二百二、秒一十八。《乾元》餘二千七百
 四、秒
 五千九百一十七,約分九十二。《儀天》餘九千二百
 九十一、秒一千一百。二歷平合皆謂之周日,數同《應天》。



 變差:三、秒空。《乾元》差二十九、秒一千一百三十五。《儀天》歲差九十八、餘三千八百。上限一百九十六度八十,下限一百六十八度四十五、秒六十三。



 鎮星總:七十五萬六千三百一十一、秒八十五。《乾元》率二
 十二萬二千三百一十一、秒二千一百六十四、二十。《儀天》土星周率三百八十一萬八千六百八、秒三千五百。



 平合:三百七十八日、八百六、秒五十一。《乾元》餘二百三十六、秒八百三十一,約分八。《儀天》餘八百八、秒三千五百。二歷平合皆謂之周日,數同《應天》。



 變差:五、秒七十九。《乾元》差二十八、秒九千五百三。《儀天》歲差一百、秒一千一百,上限一百八
 十二度、六十三分、秒八十一,下限同上限。



 太白總:一百一十六萬八千二十二、秒四十二。《乾元》率三十四萬三千三百三十九、秒一千五百四十七。《儀天》金星周率五百八十九萬七千四百八十九、秒五千四百。



 平合:五百八十三日、八千九百九十六、秒一十。《乾元》餘二千六百七十六、秒一千七百三十五,約分九十一。《儀天》餘九千一百八十九、秒五千四百。二歷平合皆謂之周日,數同《應天》。



 再合:二百九十一日、九千四百九十九、秒五。《乾元》、《儀天》不立此法。



 變差:二、秒三十六。《乾元》差二十九、秒一千七百九十八。《儀天》歲差一百二十、餘八千三百九,
 上限一百九十七度一十六,下限一百六十八度、秒六十三。



 辰星總:二十三萬一千八百六、秒四十二、八十。《乾元》率八萬八千一百三十七、秒四千四百一十,八十。《儀天》水星周率一百一十七萬三百八十七、秒二千八百。



 平合:一百一十五日,八千八百二、秒三十。《乾元》餘二千五百八十七、秒二千九十四,約分八十八。《儀天》餘八千八百八十七、秒二千八百。二歷平合皆謂之周日,數同《應天》。



 再合:五十七日、九千四百二、秒一十五。《乾元》、《儀天》不立此法。



 變差:三、秒七十八。《乾元》差二十九、秒一千一百三十八。《儀天》歲差九十八、秒三十,上限一百八十三度、六十二分,下限一百八十二度、六十二分、秒六十三。



 求五星天正冬至後加時平合日度分秒:《乾元》謂之五星平合變日,《儀天》謂之常合中日中度。



 各以星總除元積為總數,不盡者,返減星總,餘,半而進位;又置總數,木、火三之,土如其數,皆百而從之,以元法收之,為天正冬至後平合日度及分。《乾元》置歲積分,各以星率去之,不盡,用減星率,餘以五因之,滿元率收為日,不滿,退除為分。《儀天》各以其星周率去歲積分,不滿者,返減其周率,餘以宗法收為日,不盡,退除為分。



 求平合入歷分:《乾元》謂之入歷,《儀天》謂之推五星常合入歷度分。



 各以其星變差展所求積年,滿三百六十五萬三千二百九十三、秒
 一十九,去之不盡,以元法收為度,不滿為分,以減平合日,為入歷度分。《乾元》以積年乘星差,以周天策去之,不盡,以元率收為度,不滿,退除為分,用減平合變日,為入歷分。《儀天》各置其星歲差,以積年乘之,滿三百六十八萬九千八百八、秒九千九百去之,不盡,以宗法收為度,不滿,退收為分。



 求入陰陽變分:在陽末變分以下為入陽歷;以上去之,餘為入陰歷。置入陰、陽歷分,以陰、陽變量去之,不盡,為入陰、陽數及變分。



 《乾元》歲星前限二萬五百五,中限一萬二百四十八,後限一萬六千二十;熒惑前限一萬九千六百八十二,
 中限六千五百六十四,後限一萬六千八百四十四;鎮星前限一萬八千二百六十二,中限九千一百二十六,後限同前限,前、後、中皆半周天;太白前限一萬九千七百一十六,中限九千八百五十八,後限一萬六千八百九;辰星前、中、後與鎮星同。又歲星前法一千七百八,後法一千三百三十四,熒惑前法一千六百四十一,後法一千四百三;鎮星、辰星前後法皆一千五百二十二;太白前法一千六百四十三,後法一千四百二。《儀天》各置常合入歷度分,如在上限末數已下者為增數;以上者減去上限末數下度分,餘為入下限減數。又各置所入上、下限度分,以上、下限度分相近者減之,餘為入次限、下限度及分。



 《乾元》五星辰星陰、陽差分並陰、陽差度並同初、末。入陰陽定分:《乾元》謂之入諸歷變分,《儀天》謂之求五星常合入增減定數。



 以入變分各減初變分,餘卻以其變下損益率展之,百而一為分;損益次變下陰、陽積為定分。《乾元》置平
 合入歷分,以其星入段前、後限分加減之,如不足,加周天以減之,餘卻依入歷分入初末限;各
 置其段入歷分,前限以下為在前,以上者去之,為後限分;在中限以下為初限,以上去之,為末限分;置初、末,以前、後限星分除之為限數,不滿,為初末限日;各以其限差分約之,為差;初限以加、末限以減,用加減前、後限度為定度。《儀天》各置常合所入限下度數及分,以其限下損益率乘之,退一等,以百約之為度,不滿為分,以損益其限下增、減積度及分。若求諸變增、減定度者,置其變入上下限,準此求之。



 定合積日:《乾元》謂之求定日,《儀天》謂之求五星定合積日。



 日除陰、陽定分,為二;
 陽加陰減平合日,為定積日及分。《乾元》置變日,以前、後限度前加後減,為定日。《儀天》各置其星常合中日及餘,以入歷增減度增者增之、減者減之,金、水返而加減之,以日躔定差先減後加之,金、水則先加後減,即得定合積日及分。又《儀天》求入盈縮初末限,皆以半周天為準。



 入氣盈縮度分:《乾元》謂之入氣,《儀天》謂之求入盈縮初末限。



 置定積,以常數去之,不盡者,為入氣日分;置入氣日分,如求朔望盈縮術入之,即得入氣盈縮度分。《乾元》置定日,以氣策去之為氣數,不盡,為入氣日;命以冬至,算外,即得入氣日及分。《儀天》各置定合積日,在半周天以下者去之,餘為在縮,乃視在盈縮初限日及約餘以下者,便為在盈縮初限;以上者,減去盈縮初限日約餘,為在盈縮末限日及餘。



 定合日辰:《乾元》謂之日辰,《儀天》同《應天》。



 以其大、小餘加入氣日,命從甲子,算外,即得所求。《乾元》、《儀天》以冬至大、小餘加定日,各滿紀法去之,餘並同《應天》。《乾元》冬至小餘以元率退收,百為母;又有日躔陰陽度,置其氣陰
 陽分,如求朔日度分術入之,即得所求。



 求入月日數:《儀天》謂之求定合在何月日。



 置定合日辰大餘,以定朔大餘減之,餘命算外,即得所求。二歷法同。



 定合定星:《乾元》同。《儀天》謂之求日躔先後定數、求五星定合定度及分。



 各以其星入氣盈縮度分盈加縮減之,又以
 百除陰、陽定分,為度分;陽加陰減,皆加減平合,為定星;用加天正黃道日度,滿宿去之,不滿宿,即得所
 求。《乾元》各置其星平合中星,以日躔
 陰陽度陰減陽加之;又以其星入歷限度前加後減之,即為其星定合定星。餘同《應天》。《儀天》置所入限日下小餘,以其日盈
 縮率乘,以宗法除為分,以盈縮其日下先後定分,為日躔先後定度及分;又各置其星常合中度及分,以入限增定度及分增減之。金、水二星增者減,減者增;又以日躔先後定度及分,木、火、土即先減後加,金、水先加後減其日躔差,木星二因,退位,火星除二,土星退位,從下加三,金、水倍,用即得定度及分。餘同《應天》。



 歲星入段亦名入變



 熒惑入段鎮星入段太白入段辰星入段



 諸段平日平度:《乾元》謂之諸星變定積,《儀天》謂之五星諸變中日中度。



 置平合日度,以諸段下平日平度加之,即得所求。《乾元》各置其星變日,以所求入歷前加後度前加後減之。其太白辰星夕見變及晨疾變,皆以返用加減。熒惑晨見變定,置定差,以進一位滿十一除之為定差,各依加減,即得所求;在留變者,置其變定積,以前變前後度前後減之。其火星
 三因之,後退者倍之。《儀天》各置其星常合中日中度及分,以其星諸變段下常加合中日變度加減中星,即得諸變中日中度及分。



 諸段入歷:《儀天》謂之求五星諸變入限及增減定度。



 置平合入陰陽歷分,各以逐段陰陽歷分加之,為諸段入歷分。《乾元》以在諸變歷分中入歷名曰限變度。《儀天》各置其星常合入歷度分,以其星諸變段下上下限度分累加之,滿周天去之,餘依常合術入之,各得增減定度。其金星在晨疾、晨合、夕見變者,置增減定度及分,以四乘三除,為金星變定差。其火星在晨見變者,以九乘,增減定度及分,退一位,為晨星變定差。



 諸段入變分:置入歷分,各以變分去之,餘為入變分。求陰陽定分,依平合術入之。《乾元》諸段變分在入變前述。《儀天》即同《應天》。



 五星諸段定積曰:《乾元》謂之求五星諸變定日。



 置其入陰陽定分,百除,為日分;陽減陰減諸段平日。其金水夕見、晨疾返為之定積。其金星晨次、晨遲,更用盈縮度縮加盈減定積為定。求其入氣月日,如平合術入之。又熒惑前遲定積,置平合入陰陽歷分,加二萬一千六百七十五,盈三萬六千五百二十五半去之。餘與見求入陰陽歷同者,更不求之,
 如不同歷者,即依平合術入,所得,用加前遲留退、後退留平日為定積,入氣月日如前。又五星定用盈縮差及陰陽定分:歲熒惑鎮星晨見、夕疾、定合,太白定合、夕見、夕退、再合、晨見及後、晨疾,皆用盈縮定差,太白定合晨、夕見及後疾,皆用盈縮定差。內歲星後疾不用盈縮定差,辰星諸段總用盈縮定差盈加縮減。熒惑晨見陰陽定分身外加一,前疾陽定分再析,各為定分。《乾元》諸變定日在入變前。《儀天》各置其星入變中日,以其星所入變限增減定度及分,增者增之,減者減之。其金星定合、夕見、夕順
 疾、夕次疾、晨次疾,水星定合、夕見、晨疾變,皆以增減定度及分,增者減之,減者增之,各得定日。合用日躔差者,乃以日躔先後定差先減後加,乃為定日及分。其日躔差,金水定合、夕見、晨疾,以日躔差先加後減,乃為定日及分天之度數。



 定星:《乾元》謂之求五星諸變定星,《儀天》謂之求五星諸變定度。



 以合用盈縮定差加減平度分,又以陰陽定分陽加陰減。其金水夕見、晨疾返用為定星,求宿度,加平合入之。熒惑前遲、後退差度以二百三十六度加前遲定星,二百五十七度加後退定星,如半周天以下為陽度;以上者去之,餘為陰度;前
 遲陰陽度在一百一十度以上者,返減半周天,餘以五因之,後退入陰陽度在七十四度以下者,亦五因之,皆滿百為度分,陽減陰加定星,為前遲、後退定星;求宿度,加平合入之。《乾元》置其星其變中星,以入歷前後度前加後減之,又合用陰陽度者,陰減陽加之,為定星;以冬至黃道日度加之,命從斗宿,算外,即其變所入宿次也。若在留變者,更不求定星也,只用前變定星為留變定星。又熒惑留差,以一百一十九度減前遲定星,以一百三十四度減後退定星,在一百八十二度半以下為前,以上者去之為後,置前後度,在七十三度以下為在前,以上者返減一百八十三度半,餘為後度,皆倍之,百除為度,命曰留差度及分也。又前退定星度,以一百二十三度減前退定星,又以一百三十一度
 減後退定星,在一百八十二度半以下者為前,以上者去之為後,視前後度在七十三度以下為前,以上者返減一百八十二度半為後:皆以倍之,百除為度,即得前後退差度及分也;用前減後加其段定星為定星。又五星用陰陽度:歲星熒惑鎮星晨見,後疾,夕合;太白夕見、退,夕合,晨見,後疾,平合皆用日躔、陰陽度,其辰星諸段皆用之。《儀天》各置其星其變中度及分,以其變入限增減定度及分,增者增之,減者減之。其金星定合、夕見、夕定度及分,增者減之,減者增之,各得定日、次定日,各加減訖後,合用日躔先後定差者,以日躔先後定差及分先減後加之,即各得定度及分。其日躔差,木星定合,五因,半而退位,晨見先二因,退位,後五因,半而退位;後定疾先差五因,半而退位,定差二因退位;火星定合,身外除二,晨見先差七因,退位,後差身外除二,後差七因退位;土星定合,退位從下加二,晨見先差退位,後差從下加三,退位,後差退位;金星定合,二因之,夕見先差伏倍用,
 後差從下加三,晨疾伏先差從下加二,後差二因,夕退伏、晨退見六因,先後退位;水星夕見後差從下加三,先差二因,晨疾先差從下加三,後差倍用,定合乃用加減次定度為定度,置定度及分,以加天正冬至加時黃道日度及分,命從斗宿初度起算,至不滿宿,算外,即得其變加時宿度,其火星前、後退及前遲變皆為次定星,又置之,以留退定差度及分,增者增之,減者減之,得為前、後退定度,前遲,置前留定差,以三除之,乃用增減前遲定度也。又火星留差,以一百二十四半減前遲次定度,又以二百四十六度少加後退定度,若在一百八十二度六十二分以下為入在增;以上者,以減去一百八十二度六十二分為入在減。置入在增、減度及分,如在七十二度以下者為上限;以上者,返減一百八十二度六十二分,餘為下限。各置所入上、下限增減度及分,在上限四因之,在下限倍,身外加三,皆以一百約之為度及分,若在後留者,三因之為定差度及分。又,《儀天》有火星
 退定差度及分,以二百四十一度少加前退後次定度,又以一百一十九度減退次定度及分,餘,在一百八十二度六十二分以下者為入在增;以上者,減去一百八十二度六十二分,餘為入在減。又置入上、下限度分,若在七十二度以下者為上限,如在七十二度以上者為減一百八十二度六十二分,餘為下限。又置上下限增減度分,在上為度,不滿為分,即各得退定差度及分,其定差,如在後退者,倍之為定差。又有火星留定日,各置前、後留常中日,前留以前遲變入限增減定度及分,增者增之,減者減之,各以前、後留定差度及分,增者加之,減者損之,即得前、後留定日,其增減差通入歷用之。又有火星前、後退定度,各置前、後變次定度及分,以前、後退定差度及分,如在增者加之,在減者損之,即得定度及分;置定度及分,以加天正冬至黃道日度及分,命從斗宿初度去之,至不滿宿,算外,即得退行所在宿度及分也,其增減定度,三除乃用之。



 日率度率:以本段定積減後段定積,為泛日率;以本段定星減後段定星,為定度率。又置後段甲子,以前段甲子減之,餘為距後實日率。《乾元》以前段定積減後段定積為日率,以其段定星減後段定星為度率。《儀天》各置其段定日定度,以前段定日定度減之,餘者為其段日率、度率。其退行段,置前段定度減之,餘為退行度率。



 平行分:《儀天》謂之求每日平行度及分。



 以距後日率除度率,為平行分。《乾元》以日率除度率為行分。《儀天》各置其段度率及分,以其段日率除之,即得其星平行分。



 初末行分:《儀天》謂之求每段初末日度及分。



 置其段平行分,與後段平
 行分相減,為合差;半之,加減平行分,為初、末行分;後多者減平行分為初,加平行分為末;後少者加平行分為初,減平行分為末。《乾元》法同。《儀天》各以其段平行分與後段平行分相減,餘為會差,半會差,以加減其段平行分,餘同《應天》。又五星前留一段及後退段,皆加為初、減為末;後留一段及前退段,皆以半總差減為初、加為末。其總差消息前後段初、末分,令衰殺等以用總差,即得前後段初、末行分相應也。



 求日差:以距後日除合差為日差。《乾元》以日率除合差為日差。《儀天》置其段總差,以減其日率,一百除之,即為每日差行之分。



 求每日行分:以日差後多者益、後少者損初日行分,為
 每日行分。《乾元》、《儀天》法同。



 求每日星所在:以每日行分順加逆減其星,命如前,即得所求。其木火土水前、後遲段平行分倍之,前為初,後為末分,各以距後日除,為日差;前遲日損、後遲日益,為每日行分。《乾元》以日差累損益初日行分,累加其段宿次,即得每日星行宿次及分。《儀天》求每日差行度及分,各置其段總差,以減其日率一日以餘之,即為每日差行之分。以每日差分累損益初日行分,為每日行度及分。初日行分多於末日行分,累損初日行分;少於末日行分,累益初日行分。將其每日行度及分累加其星初日所在宿次,各得每日所在宿次及分。如是退行段,將每日行分累減其初日宿次及分,即得退行
 所在宿度及分。又《儀天》有直求其日星所在宿次,置其所求日,減一,以乘每日差分,所得為積差,以積差加減初日行分,初日多於末日減之;末日多於初日加之,即得其日行分;以初日行分並之,乃半之,為平行分;置平行分,以求日數乘之,為積度及分;以其積度及分加其星初日宿度,命去之,即其星其日所在宿次及分、如是退行段,以其積度及分減其星初日宿度,餘,為其星所在宿度及分。



 漏刻,《周禮》,挈壺氏主挈壺水以為漏,以水火守之,分以日夜,所以視漏刻之盈縮,辨昏旦之短長。自秦、漢至五代,典其事者,雖立法不同,而皆本於《周禮》。惟後漢、隋、五代著於史志,其法甚詳,而歷載既久,傳用漸差。國朝復
 挈壺之職,專司辰刻,署置於文德殿門內之東偏,設鼓樓、鐘樓於殿庭之左右。其制有銅壺、水稱、渴烏、漏箭、時牌、契之屬:壺以貯水,烏以引注,稱以平其漏,箭以識其刻,牌以告時於晝,牌有七,自卯至酉用之,制以牙,刻字填金。



 契以發鼓於夜,契有二:一曰放鼓。二曰止鼓制以木,刻字於上。



 常以卯正後一刻為禁門開鑰之節,盈八刻後以為辰時,每時皆然,以至於酉。每一時,直官進牌奏時正,雞人引唱,擊鼓一十五聲,惟午正擊鼓一百五十聲。



 至昏夜雞唱,放鼓契出,發鼓、擊鐘一百聲,然後下漏。
 每夜分為五更,更分為五點,更以擊鼓為節,點以擊鐘為節。每更初皆雞唱,轉點即移水稱,以至五更二點,止鼓契出,凡放鼓契出,禁門外擊鼓,然後衙鼓作,止鼓契出亦然,而更鼓止焉。



 五點擊鐘一百聲。雞唱、擊鼓,是謂攢點,至八刻後為卯時正,四時皆用此法。禁鐘又別有更點在長春殿門之外,玉清昭應宮、景靈宮、會靈觀、祥源觀及宗廟陵寢亦皆置焉,而更以鼓為節,點以鉦為節。大中祥符三年,春官正韓顯符上《銅渾儀法要》,其中有二十四氣晝夜進退、日出沒刻
 數立成之法,合於宋朝歷象,今取其氣節之
 初,載
 之
 於左:



 殿前報時雞唱,唐朝舊有詞,朱梁以來,因而廢棄,止唱和音。景德四年,司天監請復用舊詞,遂詔兩制詳定,付之習唱。每大禮、御殿、登樓、入閣、內宴、晝改時、夜改更則用之,常時改刻、改點則不用。



 五更五點後發鼓曰:



 朝光發,萬戶開,群臣謁。平旦寅,朝辨色,泰時昕。日出卯,瑞露晞,祥光繞。食時辰,登六樂,薦八珍。禺中巳,少陽時,大繩紀。日南午,天下明,萬物睹。日昳未,飛夕陽,清晚氣。晡時申,聽朝暇,湛凝神。日入酉,群動息,嚴扃守。



 初夜發鼓曰:



 日欲暮,魚鑰下,龍韜布。甲夜己,設鉤陳,備蘭錡。乙夜庚,杓位易,太階平。丙夜辛,清鶴唳,夢良臣。丁夜壬,丹禁靜,漏更深。戊夜癸,曉奏聞,求衣始。



 端拱中,翰林天文鄭昭晏上言:「唐貞觀二年三月朔,日有食之,前志不書分數、宿度、分野、虧初復末時刻。臣以《乾元歷》法推之,得其歲戊子,其朔戊申,日所食五分,一分在未出時前,四分出後,其時出在寅六刻,虧在三刻,食甚在八刻,復在卯四刻,當降婁九度。」又言:「按歷書云,凡欲取驗將來,必在考之既往。謹按《春秋》交食及漢氏以來五星守犯,以新歷及唐《麟德》、《開元》二歷覆驗三十事,以究其疏密。」



 日食:



 《春秋》,魯僖公十二年春三月庚午朔,日有食之。其年五月庚午朔,去交入食限誤為三也。文公元年春二月癸亥朔,日有食之。其年三月癸巳朔,去交入食限誤為二也。文公十五年夏六月辛丑朔,日有食之。是月泛交分入食限前。漢元光元年七月癸未晦,日有食之。今按歷法,當以癸未為八月朔,蓋日食朔、月食望,自為常理,今雲晦日食者,蓋司歷之失也。征和四年八月辛酉晦,日
 有食之。辛酉亦當為九月朔,又失之。



 五星守犯:



 後漢永元五年七月壬午,歲星犯軒轅大星。《麟德》星五度。《開元》張五度。《乾元》張八度。



 元初三年七月甲寅,歲星入輿鬼。《麟德》井二十九度。《開元》鬼一度。《乾元》柳五度。



 後魏大延二年八月丁亥,歲星入鬼。《麟德》井二十八度。《開元》鬼二度。《乾元》柳三度。



 正始二年六月己未,歲星犯昴。《麟德》昴二度。《開元》昴三度。《乾元》昴四度。



 宋大明三年五月戊辰,歲星犯東井鉞。《麟德》參四度。《開元》參六度。《乾元》井初度。



 後漢永和四年七月壬午,熒惑入南斗,犯第三星。《麟德》箕七度。《開元》斗一度。《乾元》斗十二度。



 魏嘉平三年十月癸未,熒惑犯亢南星。《麟德》角六度。《開元》亢五度。《乾元》亢三度。



 晉永和七年五月乙未,熒惑犯軒轅大星。《麟德》星七度。《開元》張二度。《
 乾元》張二度。



 後魏太常二年五月癸巳,熒惑犯右執法。《麟德》翼六度。《開元》翼十二度。《乾元》翼十三度。



 陳天嘉四年八月甲午,熒惑犯軒轅大星。《麟德》張二度。《開元》張五度。《乾元》張四度。



 後漢延光三年九月壬寅,鎮星犯左執法。《麟德》翼十九度。《開元》軫二度。《乾元》翼五度。



 晉永和十年正月癸酉,鎮星掩鉞星。《麟德》參六度。《開元》參七度。《乾元》井三
 度。



 後魏神瑞二年三月己卯,鎮星再犯輿鬼積尸。《麟德》井二十八度。《開元》井三十度。《乾元》柳初度。



 齊永明九年七月庚戌,鎮星逆在泣星東北。《麟德》危二度。《開元》虛九度。《乾元》危四度。



 陳永定三年六月庚子,鎮星入參。《麟德》參七度。《開元》參八度。《乾元》井二度。



 後漢永初四年六月癸酉,太白入鬼。《麟德》參五度。《開元》井三十度。《乾元》鬼初度。



 延光三年二月辛未,太白入昴。《麟德》晨伏。《開元》昴六度。《乾元》昴一度。



 魏黃初三年閏六月丁丑,太白晨伏。《麟德》丁亥晨伏,後十日。《開元》同,丁丑晨伏。《乾元》十月置閏,七月丁丑晨伏。



 晉咸康七年四月己丑,太白入輿鬼。《麟德》柳三度。《開元》鬼一度。《乾元》柳一度。



 晉永和十一年九月己未,太白犯天江。《麟德》尾四度。《開元》尾九度。《乾元》尾十二度。



 漢太始二年七月辛亥,辰星夕見。《麟德》伏末見。《開元》夕見軫九度。《乾元》夕見
 軫九度。



 後漢元初五年五月庚午,辰星犯輿鬼。《麟德》井二十七度。《開元》井二十八度。《乾元》井二十九度。



 漢安二年五月丁亥,辰星犯輿鬼。《麟德》夕見井二十二度。《開元》夕見鬼二度。《乾元》夕見鬼一度。



 晉隆安三年五月辛未,辰星犯軒轅大星。《麟德》夕見星五度。《開元》夕見星三度。《乾元》夕見星五度。



 後魏太和十五年六月丙子,辰星隨太白於西方。《麟德》張
 二度。《開元》星五度。《乾元》張初度。



 端拱二年四月己未,翰林祗候張玭夜直禁中,太宗手詔曰:「覽《乾元歷》細行,此夕熒惑當退軫宿乃順行,今止到角宿即順行,得非歷差否?」奏曰:「今夕一鼓,占熒惑在軫末、角初,順行也。據歷法,今月甲寅至軫十六度,乙卯順行,驗天差二度。臣占熒惑明潤軌道,兼前歲逆出太微垣,按歷法差疾者八日,此皆上天祐德之應,非歷法之可測也。」至道元年,昭晏又上言:「承詔考驗司天監丞
 王睿雍熙四年所上歷,以十八事按驗,所得者六,所失者十二。」太宗嘉之,謂宰相曰:「昭晏歷術用功,考驗否臧,昭然無隱。」由是賜昭晏金紫,令兼知歷算。二年,屯田員外郎呂奉天上言:



 「按經史年歷,自漢、魏以降,雖有編聯,周、秦以前,多無甲子。太史公司馬遷雖言歲次,詳求朔閏,則與經傳都不符合,乃言周武王元年歲在乙酉。唐兵部尚書王起撰《五位圖》,言周桓王十年,歲在甲子,四月八日佛生,常星不見;又言孔子生於周靈王庚戌之
 歲,卒於周悼王四十一年壬戌之歲,皆非是也。馬遷乃古之良史,王起又近世名儒,後人因循,莫敢改易。臣竊以史氏凡編一年,則有一十二月,月有晦朔、氣閏,則須與歲次合同,茍不合同,何名歲次?本朝文教聿興,禮樂咸備,惟此一事,久未刊詳。臣探索百家,用心十載,乃知唐堯即位之年,歲在丙子,迄太平興國元年,亦在丙子,凡三千三百一年矣。虞、夏之間,未有甲子可證,成湯既沒,太甲元年始有二月乙丑朔旦冬至,伊尹祀於先王,
 至武王伐商之年正月辛卯朔,二十有八日戊午,二月五日甲子昧爽。又康王十二年六月戊辰朔,三日庚午朏,王命作冊畢。自堯即位年,距春秋魯隱公元年,凡一千六百七年;從隱公元年,距今至道二年,凡一千七百一十五年;從太甲元年,距今至道二年,凡二千七百三十二年;從魯莊公七年四月辛卯夜常星不見,距今至道二年,凡一千六百八十一年,從周靈王二十年孔子生,其年九月庚戌、十月庚辰兩朔頻食,距今至道二年,
 凡一千五百四十五年;從魯哀公十六年四月乙丑孔子卒,距今至道二年,凡一千四百七十二年。以上並據經傳正文,用古歷推校,無不符合,乃知《史記》及《五位圖》所編之年,殊為闊略。諸如此事,觸類甚多,若盡披陳,恐煩聖覽。臣耽研既久,引證尤明,起商王小甲七年二月甲申朔旦冬至,自此之後,每七十六年一得朔旦冬至,此乃古歷一蔀;每蔀積月九百四十、積日二萬七千七百五十九,率以為常,直至《春秋》魯僖公五年正月辛亥
 朔旦冬至,了無差爽。用此為法,以推經傳,縱小有增減,抑又經傳之誤,皆可以發明也。古歷到齊、梁以來,或差一日,更有近歷校課,亦得符合。伏望聖慈,許臣撰集,不出百日,其書必成。儻有可觀,願藏秘府。」



 詔許之。書終不就。



 又司天冬官正楊文鎰上言:「新歷甲子,請以百二十年。」事下有司,以其無所依據,議寢不行。太宗曰:「支乾相承,雖止於六十,儻再周甲子,成上壽之數,使期頤之人得見所生之年,不亦善乎?」遂詔新歷甲子所紀百二
 十歲。



 國初,有司上言:「國家受周禪,周木德,木生火,則本朝運膺火德,色當尚赤。臘以戌日。」詔從之。



 雍熙元年四月,布衣趙垂慶上書言:「本朝當越五代而上承唐統為金德,若梁繼唐,傳後唐,至本朝亦合為金德。矧自國初符瑞色白者不可勝紀,皆金德之應也。望改正朔,易車旗服色,以承天統。」事下尚書省集議,常侍徐鉉與百官奏議曰:「五運相承,國家大事,著於前載,具有明文。頃以唐末喪亂,朱梁篡弒,莊宗早編屬籍,親雪國仇,中興唐
 祚,重新土運,以梁室比羿、浞、王莽,不為正統。自後數姓相傳,晉以金,漢以水,周以木,天造有宋,運膺火德。況國初祀赤帝為感生帝,於今二十五年,豈可輕議改易?」又云:「梁至周不合迭居五運,欲國家繼唐統為金德,且五運迭遷,親承歷數,質文相次,間不容發,豈可越數姓之上,繼百年之運?此不可之甚也。按《唐書》天寶九載,崔昌獻議自魏、晉至周、隋,皆不得為正統,欲唐遠繼漢統,立周、漢子孫為王者後,備三恪之禮。是時,朝議是非相半,
 集賢院學士衛包上言符同,李林甫遂行其事。至十二載,林甫卒,復以魏、周、隋之後為三恪,崔昌、衛包由是遠貶,此又前載之甚明也。伏請祗守舊章,以承天祐。」從之。



 大中祥符三年,開封府功曹參軍張君房上言:「自唐室下衰,土德隤圮,朱氏強稱金統,而莊宗旋復舊邦,則朱梁氏不入正統明矣。晉氏又復稱金,蓋謂乘於唐氏,殊不知李忭建國於江南耳。漢家二主,共止三年,紹晉而興,是為水德。洎廣順革命,二主九年,終於顯德。以上
 三朝七主,共止二十四年,行運之間,陰隱而難賾。伏自太祖承周木德而王,當於火行,上系於商,開國在宋,自是三朝迄今以為然矣。愚臣詳而辨之,若可疑者。太祖禪周之歲,歲在庚申。夫庚者,金也,申亦金位,納音是木,蓋周氏稱木,為二金所勝之象也。太宗登極之後,詔開金明池於金方之上,此誰啟之?乃天之靈符也。陛下履極當強圉之歲,握符在作噩之春,適宋道之隆興,得金天之正氣。臣試以瑞應言之,則當年丹徒貢白鹿,姑蘇
 進白龜,條支之雀來,穎川之雉至。臣又聞當封禪之時,魯郊貢白兔,鄆上得金龜,皆金符之至驗也。願以臣章下三事大臣,參定其事。」疏奏,不報。



 天禧四年,光祿寺丞謝絳上書曰:



 臣按古志,凡帝王之興,必推五行之盛德,所以配天地而符陰陽也。故神農氏以火德,聖祖以土德,夏以木德,商以金德,周以火德。自漢之興,王火德者,以謂承堯之後。且漢,堯之裔也。五帝之大,莫大於堯,漢能因之,是不墜其緒而善繼其盛德也。國家膺開光
 之慶,執敦厚之德,宜以土瑞而王天下,然其推終始傳,承周之木德而火當其次。且朱梁不預正統者,謂莊宗復興於後。自石晉、漢氏以及於周,則李忭



 建國於江左而唐祚未絕,是三代者亦不得正其統矣。昔者,秦祚促而德暴,不入正統,考諸五代之際,亦是類矣。國家誠能下黜五代,紹唐之土德,以繼聖祖,亦猶漢之黜秦,興周之火德以繼堯者也。



 夫五行定位,土德居中,國家飛運於宋,作京於汴,誠萬國之中區矣。《傳》曰:「土為群物主,故曰
 后土。」《洪範》曰「土爰稼穡,稼穡作甘。」方今四海給足,嘉生蕃衍,邇年京師甘露下,泰山醴泉湧,作甘之兆,斯亦見矣。矧靈木異卉,資生於土,千品萬類,不可勝道,非土德之驗乎?



 臣又聞之,太祖生於洛邑,而胞絡惟黃;鴻圖既建,五緯聚於奎躔,而鎮星是主。及陛下升中之次,日抱黃珥;朝祀於太清宮,有星曰含輿,其色黃而潤澤。斯皆凝命有表,微德攸屬,天意人事響效之大者,則土德之符在矣。是故天心之在茲,陛下拒而罔受;民意之若是,
 陛下謙而弗答。氣壅未宣,河決遂潰,豈不神哉!然則天淵之勃流,水德之浸患,考六府之厭鎮,驗五行之勝克,亦宜興土之運,御時之災。伏望順考符應,詳習法度,惟陛下時而行之。



 大理寺丞董行父又上言曰:「在昔泰皇以萬物生於東,至仁體乎木,故德始於木。木以生火,神農受之為火德;火以生土,黃帝受之為土德;土以生金,少昊受之為金德;金以生水,顓頊受之為水德;水以生木,高辛受之為木德;木以生火,唐堯受之為火德;火以
 生土,虞舜傳之為土德。土以生金,夏為金德;金以生水,商為水德;水以生木,周為木德;木以生火,漢應圖讖為火德;火以生土,唐受歷運為土德。陛下紹天之統,受天之命,固當上繼唐祚,以金為德,顯黃帝之嫡緒,彰聖祖之丕烈。臣又按聖祖先降於癸酉,太祖受禪於庚申,陛下即位於丁酉,天書下降於戊申。庚,金也,申、酉皆金也,天之體也。陛下紹唐、漢之運,繼黃帝之後,三世變道,應天之統,正金之德,斯又順也。」詔兩制詳議。既而獻議曰:「
 竊詳謝絳所述,以聖祖得瑞,宜承土德,且引漢承堯緒為火德之比,雖班彪敘漢祖之興有五,其一曰帝堯之苗裔。及序承正統,乃越秦而繼周,非用堯之行。今國家或用土德,即當越唐上,承於隋,彌以非順,失其五德傳襲之序。又據董行父請越五代紹唐為金德,若其度越累世,上承百代之統,則晉、漢洎周,咸帝中夏,太祖實受終於周室而陟於元後,豈可弗遵傳繼之序,續於遐邈之統?三聖臨御六十餘載,登封告成,昭姓紀號,率循火
 行之運,以輝炎靈之曜。茲事體大,非容輕議,矧雍熙中徐鉉等議之詳矣。其謝絳、董行父等所請,難以施行。」詔可。



\end{pinyinscope}