\article{志第一百六十二 藝文八}

\begin{pinyinscope}

 孔逭《文苑》十
 九卷



 蕭統《文選》六十卷李善注。



 庚自直《類文》三百六十二卷



 竇嚴《東漢文類》三十卷



 《五臣注文選》三十卷



 周明辨《文選匯聚》十卷



 《文選類聚》十卷



 常寶鼎《文
 選名氏類目》十卷



 卜鄰《續文選》二十三卷



 樂史《唐登科文選》五十卷



 宋白《文苑英華》一千卷《目》五十卷



 朱遵度《群書麗藻》一千卷《目》五十卷



 王逸《楚辭章句》二卷



 《楚辭釋文》一卷



 《離騷約》二卷



 徐鍇《賦苑》二百卷《目》一卷



 《廣類賦》二十五卷



 《靈仙賦集》二卷



 《甲賦》五卷



 《賦選》五卷



 江文蔚《唐吳英秀賦》七十二卷



 《桂香賦集》三十卷



 楊翱《典麗賦》六十四卷



 《類文賦集》一卷



 謝壁《七賦》一卷



 杜鎬《君臣賡載集》三十卷



 李虛己《明良集》五百卷



 劉元濟《正聲集》
 五卷



 王正範《續正聲集》五卷



 又《洞天集》五卷



 韋莊《採玄集》一卷



 陳正圖《備遺綴英集》二十卷



 劉明素《麗文集》五卷



 劉松《宜陽集》十卷



 《叢玉集》七十卷



 李商隱《桂管集》二十卷



 樂瞻《文囿集》十卷



 《雜文集》二十卷



 劉贊《蜀國文英》八卷



 《分門文集》十卷



 劉從義《遺風集》二十一卷



 游恭《短兵集》三卷



 《鮑溶集》六卷



 皮日休《文藪》一卷



 徐陵《玉臺新詠》十卷



 《廣玉臺集》三十卷



 《文選後名人詩》九卷



 《高仲武詩甲集》五卷



 《詩乙集》五卷



 《唐省試詩集》三卷



 雇陶《唐詩
 類選》二十卷



 鐘安禮《資吟》五卷



 張為《前賢詠題詩》三卷



 僧玄鑒《續古今詩集》三卷



 《詩纘集》三卷



 元稹、白居易、李諒《杭越寄和詩集》一卷



 《唐集賢院詩集》二卷



 《蘇州名賢雜詠》一卷



 《新安名士詩》三卷



 《應制賞花詩》十卷



 許恭宗《交館詞林詩》一卷



 喬舜《桂香詩》一卷



 雍子方、沉括編《集賢院詩》二卷



 《趙仲庠詩》十卷



 朱壽昌《樂府集》十卷



 蔣文彧《廣樂府集》三卷



 許南容《五子策林》十卷



 周仁瞻《古今類聚策苑》十四卷



 《禮部策》十卷



 楊協《論苑》十卷



 《唐凌煙
 閣功臣贊》一卷



 《國子監武成王廟贊》二卷



 《大中祥符封禪祥瑞贊》五卷



 丁謂《大中祥符祀汾陰祥瑞贊》五卷



 馬文敏《王言曾最抄》五卷



 《唐制誥集》十卷



 《元和制誥集》十卷



 《元和制策》三卷



 滕宗諒《大唐統制》三十卷



 《擬狀注制集》十卷



 費乙《舊制編錄》六卷



 《貞元制敕書奏》一卷



 毛文晏《咸通麻制》一卷



 《雜制詔集》二十一卷



 《朱梁宣底》八卷



 《制誥》一作『詔』二卷



 《後唐麻蒿集》三卷



 《長興制集》四卷



 《江南制集》七卷



 《吳越石壁集》二卷



 李慎儀《集制》二十卷



 《五代
 國初內制雜編》十卷



 《建隆景德雜麻制》十五卷



 《神哲徽三朝制誥》三卷



 李琪《玉堂遺範》三十卷



 蔡省風《瑤池集》二卷



 《唐哀冊文》四卷



 孫洙《褒恤雜錄》卷



 《晉宋齊梁彈文》四卷



 馬總《奏議集》二十卷



 張元璥《歷代忠諫事對》十卷



 《唐名臣奏》七卷



 張易《唐直臣諫奏》七卷



 《御集諫書》八十卷



 《唐奏議駁論》一卷



 趙元拱《諫爭集》十卷



 《唐初表章》一卷



 《毛漸表奏》十卷



 任諒《建中治本書》一卷



 沈常《總戎集》十卷



 雇臨、梁燾《總戎集》十卷



 《
 續羽書》六卷



 王紹顏《軍書》十卷



 李緯《縱橫集》二十卷



 趙化基《止戈書》五十卷



 張鉶《管記苑》十卷



 李大華《掌記略》十五卷



 《新掌記略》九卷



 林逢《續掌記略》十五卷



 唐格《群經雜記》十卷



 周明辨《五經手判》六卷



 徐德言《分史衡鑒》十卷



 劉分文《經史新義》一部卷亡。



 南康筆《代耕心鑒》十卷



 《干祿寶典》二十七卷



 薛廷珪《克家志》九卷



 趙世繁《忠孝錄》五卷



 趙世逢《幽居錄》五卷



 臧嘉猷《羽書集》三卷



 呂延祚注《文選》三十卷



 劉允濟《金門待詔集》五卷



 僧惠凈《續古
 今詩苑英華》十卷



 孫翌《正聲集》三卷



 崔融《珠英學士集》五卷



 竇常《南熏集》三卷



 《搜玉集》一卷唐崔湜至融,凡三十七人,集者不知名。



 《太平內制》三卷睿宗、玄宗時制詔。



 賀鑒《歸鄉集》一卷



 《唐德音》三十卷李林甫至崔湜百餘家詩。



 《張曲江雜編》一卷



 集者並不知名。



 李康《玉臺後集》十卷



 殷璠《河獄英靈集》二卷



 又《丹陽集》一卷



 蕭昕《送邢桂州詩》一卷



 曹恩《起予集》五卷



 又《類表》五十卷



 許孟容《謝亭詩集》一卷



 《竇氏聊珠集》一卷



 馬總《唐名臣
 奏議集》二十卷



 《送毛仙翁詩集》一卷牛僧孺、韓愈等贈。



 高仲武《中興間氣集》二卷錢起、張眾甫等詩。



 《集賢院諸廳壁記》二卷李吉甫、武元衡、常袞題詠集



 《大歷浙東酬唱集》一卷



 《臨平詩集》一卷



 《送白監歸東都詩》一卷



 《洛中集》一卷



 《名公唱和集》四卷



 《垂風集》一卷



 《咸通初表奏集》一卷



 《唐十九家詩》十卷



 《雲門寺詩》一卷



 《章奏集類》二十卷



 《唐百家詩選》二十卷



 《陸海》六卷



 集者並不知名。



 令狐楚《斷金集》一卷



 又《纂雜詩》一卷



 劉禹錫《彭陽唱和集》二卷



 又《彭陽唱和後
 集》一卷



 《汝洛唱和集》三卷



 《吳蜀集》一卷



 《劉白唱和集》三卷



 段成式《漢上題襟》十卷



 檀溪子道民《連璧詩集》三十二卷



 孟啟《本事詩》一卷



 盧環《抒情集》二卷



 《僧□光上人詩》一卷



 姚合《極玄集》一卷



 韋莊《又玄集》三卷



 皮日休《松陵集》十卷



 柳宗直《西漢文類》四十卷



 黃挺章《國秀集》三卷



 宋太祖、真宗《御制國子監兩廟贊》二卷



 《賜陳摶詩》八卷



 《漢魏文章》二卷



 《漢名臣奏》二卷



 《漢賢遺集》一卷



 《三國志文類》
 六十卷



 《晉代名臣集》十五卷



 《謝氏蘭玉集》十卷



 《古詩選集》十卷



 《宋二百家詩》二十三卷



 《長樂三王雜事》十四卷



 集者並不知名。



 陳彭年《宸章集》二十五卷



 宋綬《本朝大詔令》二百四十卷



 又《唐大詔令》一百三十卷《目錄》三卷



 洪遵《中興以來玉堂制草》三十四卷



 周必大《續中興玉堂制草》三十卷



 韓忠彥《追榮集》十卷



 朱翌《五制集》一卷



 熊克《京口詩集》十卷



 李仁剛《浯溪古今石刻集錄》一卷



 侍其光祖《浯溪石刻後集再集》一卷



 李燾《謝家詩集》一卷



 曾慥《宋百
 家詩選》五十卷又《續選》二十卷



 吳說編《古今絕句》三卷



 廖敏得《浯溪石刻續集》一卷



 呂祖謙《東萊集詩》二卷



 孔文仲《三孔清江集》四十卷



 《壯觀類編》一卷劉燾、楊萬里、米芾等作。



 邵浩《坡門酬唱》二十三卷



 倪恕《安陸酬唱集》六卷



 管銳《橫浦集》二卷



 方松卿《續橫浦集》二卷



 趙不敵《清漳集》三十卷



 廖遲《樵川集》十卷



 洪適《荊門惠泉詩集》二卷



 詹淵《括蒼集》三卷



 陳百朋《續括蒼集》五卷



 柳大雅《括蒼別集》五卷



 胡舜舉《劍津集》十卷



 許份《漢南酬唱集》一卷



 楊
 恕《臨江集》三十四卷



 汪浹《元祐榮觀集》五卷



 衛博《定庵類蒿》十二卷



 於霆《南紀集》五卷



 湯邦傑《南紀別集》一卷



 家求仁《名賢雜詠》五十卷



 又《草木蟲魚詩》六十八卷



 程九萬《三老奏議》七卷



 畢仲游《元祐館職詔策詞記》一卷



 謝逸《溪堂師友尺牘》六卷



 《羅唐二茂才重校唐宋類詩》二十卷



 《三洪制蒿》六十二卷洪適、遵、邁撰。



 李壁《中興諸臣奏議》四百五十卷



 洪邁《唐一千家詩》一百卷



 《三蘇文集》一百卷郎曄進。



 《臨賀郡志》二卷



 《相江集》十卷



 《豫章類集》十卷



 《
 千家名賢翰墨大全》五百一十八卷



 《三蘇文類》六十八卷



 《續本事詩》二卷



 《集選》一百卷



 《唐賢長書》一卷



 《唐三十二僧詩》一卷



 《四僧詩》八卷



 《唐雜一卷》



 《五代制詞》一卷



 《重編類啟》十卷



 《潤州金山寺詩》一卷



 集者並不知名。



 蔡省風《瑤池集》一卷



 陳匡圖《擬玄類集》十卷



 韋《唐名賢才調詩集》十卷



 李昉、扈蒙《文苑英華》一千卷



 劉吉《江南續又玄集》二卷



 田錫《唐明皇制誥後集》一百卷



 蘇易簡《禁林宴會集》一卷



 子起《家宴集》五卷不知姓。



 楊征《論苑》十卷



 馮翊
 嚴《滁州瑯琊山古今名賢文章》一卷



 朱博《叢玄集》二十卷



 《二李唱和詩》一卷李昉、李至作。



 楊億《西昆酬唱集》二卷



 陳充《九僧詩集》一卷



 《四釋聊唱詩集》一卷丁謂序。



 楊偉《虢郡文齊集》五卷



 姚鉉《唐文粹》一百卷



 《謫仙集》十卷勾龍震集古今人詞,以李白為首。



 僧仁贊《唐宋類詩》二十卷



 許洞《徐鉉雜古文賦》一卷



 郭希樸《養閑亭詩》一卷



 幼暐《金華瀛洲集》三十卷



 王咸《典麗賦》九十三卷



 《華林義門書堂詩集》一卷王欽若、錢惟演等作。



 張逸、楊諤《潼川唱和集》一卷



 李祺《天聖賦苑》一
 十八卷



 又《珍題集》三十卷



 滕宗諒《岳陽樓詩》二卷



 陶叔獻《西漢文類》四十卷



 徐徽《滁陽慶歷集》十卷



 韓琦《閱古堂詩》一卷



 《送僧符游南昌集》一卷範鎮序。



 《石聲編》一卷趙師旦家編集。



 《南犍唱和詩集》一卷吳中復、吳秘、張穀等作。



 鄭雍《古今名賢詩》二卷



 歐陽修《禮部唱和詩集》三卷



 《送元絳詩集》一卷《送文同詩》一卷鮮于侁序。



 晏殊、張士遜《笑臺詩》一卷



 慧明大師《靈應天竺集》一卷



 宋璋《錦裏玉堂編》五卷



 孫洙《褒題集》三十卷又《張氏詩傳》一卷



 宋敏求《寶刻叢章》三十卷



 《寶刻叢章拾遺》三十卷


孫氏《吳興詩》三卷
 \gezhu{
  不知名}
 。



 姚闢《荊溪唱和》一卷



 林少穎《觀瀾文集》六十三卷



 呂祖謙《皇朝文鑒》一百五十卷又《國趄名臣奏議》十卷



 曾紘《江西續宗派詩集》二卷



 石處道《松江集》一卷



 江文叔《桂林文集》二十卷



 劉褒《續集》十二卷



 黃岦《續乙集》八卷



 張修《桂林集》十二卷



 徐大觀《又續集》四卷



 丁逢《郴江前集》十卷



 又《後集》五卷



 《郴江續集》九卷



 楊倓《南州集》十卷



 王仁《澧陽集》四卷



 道士田居
 寶《司空山卷》一卷



 姜之茂《臨川三隱詩集》三卷



 熊克《館學喜雪唱和詩》二卷



 陳天麟《游山唱和》一卷



 史正心《清暉閣詩》一卷



 葛郛《載德集》四卷



 王十朋《楚東唱酬集》一卷



 莫琮《椿桂堂詩》一卷



 何紘《籍桂堂唱和集》一卷



 莫若沖《清湘泮水酬和》一卷



 陳讜《西江酬唱》一卷



 廖伯憲《岳陽唱和》三卷



 黃學行《又乙集》一卷



 劉璇《政和縣齊酬唱》一卷



 林安宅《南海集》三十卷



 曾肇《滁陽慶歷前集》十卷



 吳玨《滁陽慶歷後集》十卷



 《干越題詠》三卷李並序。



 郝篪《都梁
 集》十卷



 西湖寓隱《回文類聚》一卷



 《郢州白雪樓詩》一卷蕭德藻序。



 《三蘇翰墨》一卷蘇軾等書。



 《桂香集》六卷



 《留題落星寺詩》一卷《翰苑名賢集》一卷



 《宋賢文集》三卷



 《宋賢文藪》四十卷



 《先容集》一卷



 《制誥章表》二卷



 又《制誥章表》十五卷



 《儒林精選時文》十六卷



 《玉堂詩》三十六卷



 《辭林類蒿》三卷



 《海南集》十八卷



 《鄞江集》九卷



 《嘉禾詩文》一卷



 《潯陽琵琶亭紀詠》三卷



 《潯陽庚樓題詠》一



 《翰苑名賢集》一卷



 《滕王閣詩》一卷



 《膾炙集》一卷



 《玉枝集》三十二卷



 《永康題紀詩詠》十三卷



 《聖
 宋文粹》三十卷



 《布袋集》一卷



 《元祐密疏》一卷



 《唐宋文章》二卷



 《聖宋文選》十六卷



 《唐宋詩後集》十四卷



 《君山寺留題詩集》一卷



 《制誥》三卷



 《春貼子詞》一卷



 《高麗表章》一卷



 《登瀛集》五十二卷



 《羅浮寓公集》三卷



 《羅浮》一卷集者不知名。



 陳材夫《仕途必用集》十卷



 翁忱《岳陽別集》二卷



 鐘興《秭歸集》八卷



 卜無咎《廬山記拾遺》一卷



 商侑《盛山集》一卷



 劉充《唐詩續選》十卷



 王安石《建康酬唱詩》一卷



 又《唐百家詩選》二十卷



 《四家詩選》十卷



 《送朱壽昌詩》三卷



 韓忠
 彥《考德集》三卷



 元積中《江湖堂詩集》一卷



 孔延之《會稽掇英集》二十卷



 程師孟《續會稽掇英集》二十卷



 曾公亮《元日唱和詩》一卷



 孫覺《荔枝唱和詩》一卷



 蒲宗孟《曾公亮動德集》三卷



 馬希孟《揚州集》三卷



 曾旼《潤州類集》十卷



 魏泰《襄陽題詠》二卷



 蘇夢齡《摛華集》三卷



 王得臣《江夏古今紀詠集》五卷



 楊傑《高僧詩》一卷



 孫頎《抄齊唱和集》一卷



 薛傳正《錢塘詩前後集》三十卷



 唐愈《江陵集古題詠》十卷



 章粢《成都古今詩集》六卷



 孫永《永簡公崇終
 集》一卷



 道士龔元正《桃花源集》二卷



 《紹聖三公詩》三卷司馬光、歐陽修、馮京所著。



 陸經《靜照堂詩》一卷



 劉珵《宣城集》三卷



 唐庚《三謝集》一卷



 上官彞《麻姑山集》三卷



 翁公輔《下邳小集》九卷



 彈粹《鵝城豐湖亭詩》一卷



 蔡驛《惠泉詩》一卷



 林虙《西漢詔令》十二卷



 俞向《長樂集》十四卷



 《四學士文集》五卷黃庭堅、晁補之、張耒、秦觀所著。



 《內制》六卷晏殊以下所撰。



 沈晦《三沉集》六十一卷



 《輶軒唱和集》三卷洪皓、張邵、朱弁所集。



 程邁《止戈堂詩》一卷



 樊汝霖《唐書文藝補》六十三卷



 何琥《蘇黃遺編》
 一卷



 楊上行《宋賢良分門論》六十二卷



 戴覺、李丁《單題詩》十二卷



 廖剛《世絲採集》三卷



 《送王周歸江陵詩》二卷杜衍等所撰。



 許端夫《齊安集》十二卷



 黃仁榮《永嘉集》十二卷



 李知己《永嘉集》三卷



 《晁新詞》一卷晁端禮、晁沖之所撰。



 陸時雍《宏詞總類前後集》七十六卷



 《梅江三孫集》三十一卷孫立節及子勴、孫何所著。



 鮑喬《豫章類集》十卷



 鄧植《小有天後集》一卷



 蕭一致《濂溪大成集》一卷《館閣詞章》一卷《館閣詩》八卷並中興館閣諸臣所撰。右總集類四百三十五部,一萬六百五十七卷



 劉勰《文心雕龍》十卷



 鐘嶸《詩評》一卷



 任昉《文章緣起》一卷



 李允一作『元』或作『克』《翰林論》三卷



 王昌齡《詩格》一卷又《詩中密旨》一卷



 杜嗣先《兔園策府》三十卷



 柳璨《史通析微》十卷



 劉餗《史便》三卷



 劉知幾《史通》二十卷



 白居易《白氏金針詩格》三卷又《白氏制樸》一卷



 僧皎然《詩式》五卷又《詩評》一卷



 辛處信注《文心雕龍》十卷



 王正範《文章龜鑒》五卷



 範攄《詞林》一卷



 孫合《文格》二卷



 倪宥《文章龜鑒》一卷



 劉蒢《應求類》二卷



 竇蘋《載籍討源》一
 卷



 《舉要》二卷



 吳武陵《十三代史駁議》十二卷



 林《史論》二十卷



 王鵬《唐史名賢論斷》二十卷



 王損之《絲綸點化》二卷



 方仲舒《究判玄微》一卷



 樂史《登科記解題》二十卷



 蔣之奇《廣州十賢贊》一卷



 白行簡《賦要》一卷



 範傳正《賦訣》一卷



 浩虛舟《賦門》一卷



 紇于俞《賦格》一卷



 和凝《賦格》一卷



 毛友《左傳類對賦》六卷



 王維《詩格》一卷



 王□巳一作『超』《詩格》一卷



 賈島《詩格密旨》一卷



 元兢《詩格》一卷又《古今詩人秀句》二卷



 僧辭遠《詩式》十卷



 許文貴一作『貢』《詩鑒》一卷



 僧元鑒《續古今詩人秀句》二卷



 司馬光《續詩話》一卷



 姚合《詩例》一卷



 鄭谷《國風正訣》一卷



 王叡《炙子詩格》一卷



 張仲素《賦樞》一卷



 倪宥《詩體》一卷



 張為《唐詩主客圖》二卷



 僧齊己《玄機分明要覽》一卷又《詩格》一卷



 李洞《賈島詩句圖》一卷



 僧神彧《詩格》一卷



 徐銳《詩格》一卷



 馮鑒《修文要訣》二卷



 林逋《句圖》三卷



 李淑《詩苑類格》三卷



 僧定雅《寡和圖》三卷



 劉分文《詩話》一卷



 邵必《史例總論》十卷



 司馬光《詩話》一卷



 馬備《賦門魚
 金龠》十五卷



 蔡寬夫《詩史》二卷



 吳處厚《賦評》一卷



 蔡希蒢《古今名賢警句圖》一卷



 魏泰《隱居詩話》一卷



 楊九齡《正史雜編》十卷



 郭思《瑤溪集》十卷



 蔡條《西清詩話》三卷



 李頎《古今詩話錄》七十卷



 李錞《詩話》一卷



 僧惠洪《天廚標SJ》三卷



 周紫芝《竹坡詩話》一卷



 強行父《唐杜荀鶴警句圖》一卷



 黃《□溪詩話》十卷



 鄭樵《通志敘論》二卷



 曾發《磅注摘遺》三卷



 胡源《聲律發微》一卷



 費哀《文章正派》十卷



 《李善五臣同異》一卷



 嚴有翼《藝苑雌黃》二十卷



 方深
 道《集諸家老杜詩評》五卷



 方絟《續老杜詩評》五卷



 彭鬱《韓文外抄》八卷



 趙師懿《柳文筆記》一卷



 葛立方《韻語陽秋》二十卷



 呂祖謙《古文關鍵》二十卷



 《新集詩話》十五卷集者不知名。



 《元祐詩話》一卷



 《歷代吟譜》二十卷



 《唐宋名賢詩話》二十卷



 《金馬統例》三卷



 《詩談》十五卷



 《韓文會覽》四十卷



 並不知作者。右文史類九十八部,六百卷。



 凡集類二知三百六十九部,三萬四千九百六十卷



\end{pinyinscope}