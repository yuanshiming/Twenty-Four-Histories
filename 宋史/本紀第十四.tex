\article{本紀第十四}

\begin{pinyinscope}

 神宗一



 神宗紹天法古運德建功英文烈武欽仁聖孝皇帝,諱頊,英宗長子,母曰宣仁聖烈皇后高氏。慶歷八年四月戊寅生於濮王宮,祥光照室,群鼠吐五色氣成云。八月,
 賜名仲金咸。授率府副率,三遷至右千牛衛將軍。嘉祐八年,侍英宗入居慶寧宮,嘗夢神人捧之登天。英宗即位,授安州觀察使,封光國公。是年五月壬戌,受經於東宮。帝隆準龍顏,動止皆有常度。而天性好學,請問至日晏忘食,英宗常遣內侍止之。帝正衣冠拱手,雖大暑,未嘗用扇。侍講王陶入侍,帝率弟顥拜之。九月,加忠武軍節度使、同中書門下平章事,封淮陽郡王,改今諱。治平元年六月,進封穎王。三年三月,納故相向敏中孫女為夫
 人。十月,英宗不豫,帝引仁宗故事,請兩日一御邇英閣講讀,以安人心。十二月壬寅,立為皇太子。



 四年正月丁巳,英廟崩,帝即皇帝位。戊午,赦天下常赦所不原者。遣馮行己告哀於遼。己未,尊皇太后曰太皇太后,皇后曰皇太后。命宰相韓琦為山陵使。辛酉,遣孫坦等告即位於遼,以大行皇帝詔賜夏國主及西蕃唃廝囉。丙寅,群臣表三上,始御迎陽門幄殿聽政。內醫侍先帝疾者,皆坐不謹貶之。詔東平郡王允弼、襄陽郡王允良朝朔望。
 以吳奎終喪,復授樞密副使。戊辰,以韓琦守司空兼侍中,曾公亮行門下侍郎兼吏部尚書、進封英國公,文彥博行尚書左僕射、檢校司徒兼中書令,富弼改武寧軍節度使、進封鄭國公,曹佾改昭慶軍節度使、檢校太傅,張昪改河陽三城節度使,宗諤同中書門下平章事,改集慶軍節度使、檢校尚書左僕射,歐陽修、趙概並加尚書左丞,仍參知政事,陳升之為戶部侍郎,呂公弼為刑部侍郎,允弼、允良並加守太保,弟東陽郡王顥進封昌
 王,鄠國公頵進封樂安郡王。群臣進秩有差。



 二月乙酉,初御紫宸殿。立向氏為皇后。丁亥,詔入內內侍省、皇城司合覆奏事並執條覆奏。戊子,進封交址郡王李日尊為南平王。加邈川首領董氈檢校太保。詔山陵所須,應委三司、轉運司計置,毋輒擾民。詔提舉醫官院試堪診御脈者六人。庚寅,以四月十日為同天節。辛卯,白虹貫日。壬辰,詔公主下嫁者行見舅姑禮。甲辰,西蕃首領拽羅缽、鳩令結二人誘蕃部三百餘帳投夏國,捕獲,斬之
 以徇。



 三月壬子,曹佾加檢校太尉兼侍中。賜禮部進士及第、出身四百六十一人。甲寅,陜西宣撫使郭逵討蕃部黨令徵等,平之。賜昌王顥公使錢歲萬緡,半給之。丙辰,昌王顥、樂安郡王頵乞解官行服,不許。癸亥,詔入內內侍省官已經壽聖節任子者,同天節權罷奏薦。壬申,歐陽修罷知亳州。癸酉,吳奎參知政事。乙亥,允良薨。



 閏月癸未,太白晝見。甲申,夏國主諒祚遣使謝罪。辛卯,詔齊、密、登、華、邠、耀、鄜、絳、潤、婺、海、宿、饒、歙、吉、建、汀、潮等十八
 州知州,慶、渭、秦、延四州通判,其選並從中書,毋以恩例奏授。乙未,張昪以太子太師致仕。庚子,詔求直言。御史中丞王陶乞許舉知縣資序人為御史裏行,從之。癸卯,王安石出知江寧府。甲辰,詔諸路帥臣及副總管或有移易,依慶歷故事。乙巳,詔以孟夏農勞之時,令監司戒飭州縣省事,勸民力田,民有艱食者振之。



 夏四月庚戌,請大行皇帝謚於南郊。辛酉,詔內外所上封事,令張方平、司馬光詳定以聞。丙寅,錄囚。御史中丞王陶、侍御史
 吳申、呂景以過毀大臣,陶出知陳州,申、景各罰銅二十斤。吳奎罷知青州。遣使循行陜西、河北、京東、京西路,體量安撫。壬申,奎復位。罷州郡歲貢飲食果藥。癸酉,詔陜西、河東經略、轉運司察主兵臣僚怯懦老病者以聞。



 五月辛巳,以久旱,命宰臣禱雨。乙巳,寶文閣成,置學士、直學士、待制官。



 六月己酉,遼遣蕭餘慶等來吊祭。己未,振河北流民。辛未,詔天下官吏有能知徭役利病可議寬減者以聞。乙亥,詔中書、樞密細務歸之有司。



 秋七月庚辰,詔
 察富民與妃嬪家婚姻夤緣得官者。甲申,石蕃來貢。己丑,命尚書戶部郎中趙抃、刑部郎中陳薦同詳定中外封事。辛卯,告英宗憲文肅武宣孝皇帝謚於天地、宗廟、社稷。壬辰,上寶冊於福寧殿。丙午,文州曲水縣令宇文之邵上書指陳得失。



 八月丁未朔,太白晝見。戊午,復西夏和市。己巳,京師地震。癸酉,葬英宗於永厚陵。



 九月丁丑,詔減諸路逃田稅額。壬午,祧僖祖及文懿皇后。乙酉,祔英宗神主於太廟,樂曰《大英之舞》。戊子,減兩京、畿內、
 鄭、孟州囚罪一等,民役山陵者蠲其賦。辛卯,徙封顥為岐王,頵為高密郡王。富弼為尚書左僕射。遣孫思恭等報謝於遼,且賀生辰、正旦。壬辰,錄周世宗從曾孫貽廓為三班奉職。甲午,遼遣耶律好謀等來賀即位。戊戌,以王安石為翰林學士。辛丑,韓琦罷為司徒、鎮安武勝軍節度使、判相州。吳奎、陳升之罷。樞密副使呂公弼為樞密使,張方平、趙抃並參知政事,邵亢為樞密副使。壬寅,以曾公亮為尚書左僕射,文彥博為司空。潮州地震。癸
 卯,以權御史中丞司馬光為翰林學士。



 冬十月丙午,漳、泉諸州地震。丁未,富弼罷判河陽。戊申,建州、邵武、興化軍地震。己酉,初御邇英閣,召侍臣講讀經史。以右諫議大夫、權御史中丞滕甫考諸路監司課績。張方平以父憂去位。庚戌,給陜西轉運司度僧牒。令糴穀振霜旱州縣。癸丑,詔翰林學士、御史中丞、侍御史知雜事舉材堪御史者各二人。詔將作監主簿常秩赴闕。甲寅,制《資治通鑒序》賜司馬光。癸酉,知青澗城種諤復綏州。



 十一月
 丁丑,詔近臣各舉才行可任使者一人。戊寅,詔求直言。丙戌,詔二府各舉所知。丁亥,令考課院詳定諸州所上縣令治狀。戊子,分命宰臣祈雪。置馬監於河東交城縣。庚寅,詔近臣以舉官不當,經三劾者,中書別奏取旨。乙未,詔令內外文武官各舉有材德行能者。十二月丙辰,西南龍蕃來貢。辛酉,以來歲日食正旦,自乙丑避殿、減膳、罷朝賀。壬戌,詔起居日增轉對官二人。丙寅,詔州縣吏並緣為奸,致獄多瘐死,歲終會死者多寡,以制其罪。
 著為令。己巳,遼遣蕭傑等來賀正旦。



 熙寧元年春正月甲戌朔,日有食之。詔改元。丁丑,以旱,減天下囚罪一等,杖以下釋之。壬午,令州縣掩暴骸。丁亥,命宰臣曾公亮等極言闕失。庚寅,御殿復膳。壬辰,幸寺觀祈雨。丙申,趙概罷知徐州,三司使唐介參知政事。丁酉,詔修《英宗實錄》。壬寅,增太學生百人。



 二月辛亥,令諸路每季上雨雪。乙卯,孔若蒙襲封衍聖公。壬戌,貸河東饑民粟。



 三月庚辰,夏主諒詐卒,遣使來告哀。丙戌,詔
 恤刑。戊子,作太皇太后慶壽宮、皇太后寶慈宮。丁酉,簡州木連理,潭州雨毛。



 夏四月乙巳,詔翰林學士王安石越次入對。戊申,命宰臣禱雨。以樞密直學士李參為尚書右丞、判西京留守司御史臺。辛亥,同天節,群臣及遼使初上壽於紫宸殿。



 五月甲戌,募饑民補廂軍。庚辰,詔兩制及國子監舉諸王宮學官。戊戌,廢慶成軍。



 六月癸卯,錄唐魏徵、狄仁傑後。丁未,占城來貢。辛亥,詔諸路興水利。乙亥,河決棗強縣。丙寅,命司馬光、滕甫裁定國用。



 秋七月癸酉,詔謀殺已傷,案問欲舉自首者,從謀殺減二等。乙亥,名秦州新築大甘谷口砦曰甘谷城。丁丑,詔諸路帥臣、監司及兩制、知雜御史已上,各舉武勇謀略三班使臣二名。賜布衣王安國進士及第。己卯,群臣三表請上奉元憲道文武仁孝之號,不許。陳升之知樞密院事。給濮州雷澤縣堯陵守戶。壬午,以恩、冀州河決,賜水死家緡錢及下戶粟。甲申,京師地震。乙酉,又震,大雨。辛卯,以河朔地大震,命沿邊安撫司及雄州刺史候遼
 人動息以聞。賜壓死者緡錢。京師地再震。壬辰,遣御史中丞滕甫、知制誥吳充安撫河北。癸巳,疏深州溢水。甲午,減河北路囚罪一等。丁酉,賜河北安撫司空名誥敕,募民入粟。己亥,回鶻來貢。



 八月壬寅,詔京東、西路存恤河北流民。京師地震。甲辰,又震。乙卯,賜河東及鄜延路轉運司空名誥敕,募民入粟實邊。甲子,詔中書門下,考屬近行尊者一人,王之。丙寅,罷宗諤平章事。丁卯,遣張宗益等賀遼主生辰、正旦。



 九月辛未,太祖曾孫舒國公
 從式進封安定郡王。丁亥,減后妃臣僚薦奏推恩。戊子,莫州地震,有聲如雷。丁酉,詔三司裁定宗室月料,嫁娶、生日、郊禮給賜。



 冬十月辛丑,給天下系囚衣食薪炭。乙卯,出奉宸庫珠,付河北買馬。戊辰,禁銷金服飾。十一月癸酉,太白晝見。癸未,命宰臣禱雪。丙戌,朝饗太廟,遂齋於郊宮。廢青城後苑。丁亥,祀天地於圜丘,大赦,群臣進秩有差。乙未,京師及莫州地震。



 十二月己亥朔,命宰臣禱雪。癸卯,瀛州地大震。庚戌,賜夏國主秉常詔,許納塞
 門、安遠二砦歸其綏州。辛亥,錄唐段秀實後。癸丑,禱雪於郊廟、社稷。庚申,以判汝州富弼為集禧觀使,詔乘驛赴闕。壬戌,雪。甲子,遼遣耶律公質等來賀正旦。



 二年春正月甲午,奉安英宗神御於景靈宮英德殿。



 二月己亥,以富弼同中書門下平章事。庚子,以王安石參知政事。命翰林學士呂公著修《英宗實錄》。乙巳,帝以災變避正殿,減膳徹樂。甲子,陳升之、王安石創置三司條例,議行新法。



 三月乙酉,詔漕運、鹽鐵等官各具財用利
 害以聞。丙戌,命宰臣禱雨。戊子,秉常上誓表,納塞門、安遠二砦,乞綏州,詔許之。乙未,以旱慮囚。



 四月丁酉朔,群臣再上尊號,不許。戊戌,省內外土木工。壬寅,遼遣耶律昌等來賀同天節。丁未,唐介薨,臨其喪。戊申,宰臣富弼、曾公亮以旱上表待罪,詔不允。癸丑,命曾公亮為西京奏安仁宗、英宗御容禮儀使。丁巳,遣使諸路,察農田水利賦役。戊午,外任大使臣年七十以上,令監司體量,直除致仕者,更不與子孫推恩。甲子,御殿復膳。免河北歸
 業流民夏稅。



 五月辛未,宴紫宸殿,初用樂。己卯,賜河北役兵特支錢。癸未,翰林學士鄭獬罷知杭州,宣徽北院使王拱辰罷判應天府,知制誥錢公輔罷知江寧府。丁亥,奉安仁宗、英宗御容於會聖宮及應天院。甲午,減西京囚罪一等。臺州民延贊等九人,年各百歲以上,並授本州助教。



 六月丁巳,右諫議大夫、御史中丞呂誨以論王安石,罷知鄧州。以翰林學士呂公著為御史中丞。命龍圖閣直學士張掞兼編排錄用勛臣子孫。壬戌,太白
 晝見。



 秋七月乙丑朔,日當食,雲陰不見。庚午,詔御史中丞舉推直官及可兼權御史者。甲戌,東平郡王允弼薨。辛巳,立淮、浙、江、湖六路均輸法。壬午,振恤被水州軍,仍蠲竹木稅及酒課。癸未,詔自今文臣換右職者,須實有謀勇,曾著績效,即得取旨。甲申,日下有五色雲。己丑,韓琦上《仁宗實錄》,曾公亮上《英宗實錄》。



 八月癸卯,侍御史劉琦貶監處州鹽酒務,御史裏行錢顗貶監衢州鹽稅,亦以論安石故。乙巳,殿中侍御史孫昌齡以論新法,貶
 通判蘄州。丙午,同修起居注范純仁以言事多忤安石,罷同知諫院。戊申,河徙東行。夏國請從舊蕃儀,詔許之。己酉,范純仁知河中府。甲寅,朝神御殿。辛酉,以秘書省著作佐郎程顥、王子韶並為太子中允、權監察御史裏行。壬戌,待御史知雜事劉述、同判刑部丁諷坐受刑名敕不即下,述貶知江州,諷貶通判復州。審刑院詳議官王師元坐言許遵所議刑名不當,貶監安州稅。



 九月甲子朔,交州來貢。乙丑,以古勿峒效順首領儂智會為右
 千牛衛大將軍。丁卯,立常平給斂法。戊辰,出內庫緡錢百萬糴河北常平粟。丁丑,遣孫固等賀遼主生辰、正旦。辛卯,廢奉慈廟。壬辰,以秘書省著作佐郎呂惠卿為太子中允、崇政殿說書。



 冬十月丙申,富弼罷為武寧軍節度使、判亳州。曾公亮、陳升之並同中書門下平章事。城綏州,命郭逵選將置守具。逵遣趙離交夏人所納安遠、塞門二砦,就定地界。夏人渝初盟,離請城綏州,不以易二砦,因改名綏德城。戊戌,以蕃官禮賓使折繼世為忠
 州刺史,左監門衛將軍嵬名山為供備庫使,仍賜姓名趙懷順。丙辰,詔御史請對,並許直由閣門上殿。戊午,宗諤復平章事。己未,夏人來謝封冊。辛酉,錄楊承信曾孫立、田重進曾孫章為三班借職。十一月乙丑,命韓絳制置三司條例。甲戌,詔祖宗之後世襲補外官,非袒免親罷賜名授官。丙子,罷諸路提刑武臣。頒《農田水利約束》。壬午,御邇英閣聽講。賜汴口役兵錢。己丑,減天下囚罪一等,徒以下釋之。



 閏月庚子,浚御河。壬子,置交子務。是
 月,差官提舉諸路常平廣惠倉,兼管勾農田水利差役事。



 十二月癸亥朔,復減后妃公主及臣僚推恩。癸酉,增失入死罪法。丙戌,增三京留司御史臺、國子監及宮觀官,以處卿監、監司、知州之老者。戊子,遼遣蕭惟禧來賀正旦。是歲,交州來貢。



\end{pinyinscope}