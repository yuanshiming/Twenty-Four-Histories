\article{列傳第五十一}

\begin{pinyinscope}

 李諮程戡
 夏侯嶠盛度丁度張觀鄭戩明鎬王堯臣孫抃田況



 李諮,字仲詢,唐趙國公峘之後。峘貶死袁州,因家新喻,
 遂為新喻人。諮幼有至性,父文捷出其母,諮日夜號泣,食飲不入口,父憐之而還其母,遂以孝聞。舉進士,真宗顧左右曰:「是能安其親者。」擢第三人,除大理評事、通判舒州,召試中書,為太子中允、直集賢院。歷三司、開封府判官,再遷左正言,出為淮南轉運副使。帝幸亳,以勞,遷尚書禮部員外郎。會江南饑,徙江東轉運副使,為度支判官。擢知制誥,寇準數改諮所擬制辭,諮不樂,以父留鄉里請外,遂出知荊南。會翰林學士闕,宰相擬他官,帝
 曰:「不如李諮。」遂為學士。



 仁宗即位,超遷本曹郎中、權知開封府,數月,權三司使,拜右諫議大夫。嘗奏事兩宮曰:「天下賦調有定,今西北寢兵且二十年,而邊饋如故。戍兵雖未可滅,其末作浮費非本務者,宜一切裁損以厚下。」即詔諮與御史中丞劉筠等同議冗費,以景德較天禧,計所減得十三之上。



 時陜西緣邊數言軍食不給,度支都內錢不足支月奉,章獻太后憂之,命呂夷簡、魯宗道、張士遜與諮等經度其事。諮曰:「舊法商人入粟邊郡,
 算茶與犀象、緡錢,為虛實三估,出錢十四文,坐得三司錢百文。」諮請變法以實錢入粟,實錢售茶,三者不得相為輕重。既行而商人果失厚利,怨謗蜂起。諮以疾累請郡,改樞密直學士、知洪州。行數月,而御史臺鞫吏王舉、句獻私商人,多請慈州礬,會計茶法不折虛費錢,妄稱增課百萬緡,以覬恩賞。諮坐不察奪職。



 久之,進給事中、知杭州,復樞密直學士、知永興軍。衣冠子弟恃蔭無賴者,諮悉杖之,境內肅然。還,勾當三班院,坐舉吏降左諫
 議大夫。權三司使事,是歲,禁中火,倉卒營造,應辦舉集。



 進尚書禮部侍郎,拜樞密副使。數月,遭父喪,起復,遷戶部侍郎、知諫院事。是時榷茶法浸壞,乃詔諮、蔡齊等更議之。諮以前坐變法得罪,固辭,不許。於是復用諮所變法,語具《食貨志》。卒,贈右僕射,謚憲成。



 諮性明辨,周知世務,其處煩猝,常若閑暇,吏不敢欺。在樞府,專務革濫賞,抑僥幸,人以為稱職。無子,以族子為後。



 程戡,字勝之,許州陽翟人。少力學,舉進士甲科,補涇州
 觀察推官,再遷秘書丞、通判許州。曹利用貶,戡以利用婿降通判蘄州。徙虔州,州人有殺母,暮夜置尸仇人之門,以誣仇者。獄已具,戡獨辨之,正其罪。以尚書屯田員外郎知歸州,召為侍御史、三司度支判官。



 寶元初,忻、代地震,壞城郭、廬舍,死傷甚眾,命戡安撫,頗以便宜從事。改起居舍人、知諫院,遷兵部員外郎兼侍御史知雜事、三司戶部副使。擢天章閣待制、陜西都轉運使。



 未幾,知渭州。陜西有保毅軍,人苦其役。戡奏曰:「保毅在鄉兵外,
 不黥而有籍,所以佐邊備也。已隸保捷兵,而保毅籍如故,州縣以供力役,率困憊,至破析財產售田者,猶數戶出一夫,民不勝苦。」因詔:私役保毅者以計傭律坐之。



 進樞密直學士、知成都府。坐嘗保任貝州張得一,得一伏誅,奪職出知鳳翔府,尋徙河中。御史中丞張觀辨之,復為樞密直學士、知永興軍,徙瀛州,四遷給事中。契丹使過,稱疾,求著帽見,戡使謂曰:「有疾,可毋相見,見當如禮。」使者語屈,冠而見。



 人言歲在甲午蜀且有變,孟知祥之割
 據,李順之起而為盜,皆此時也。仁宗自擇戡再知益州,遷端明殿學士,召見慰遣。至彭州,民妄言有兵變,捕斬之。守益州者以嫌,多不治城堞,戡獨完城浚池自固,不以為嫌也。



 召拜參知政事,奏禁蜀人妖言誣民者。避宰相文彥博親,改尚書戶部侍郎、樞密副使。數與宋庠爭議,諫官、御史皆論之,戡亦自請罷。除吏部侍郎、觀文殿學士兼翰林侍讀學士、同群牧制置使,尋拜宣徽南院使、鄜延路經略安撫使、判延州。



 英宗即位,以安武軍節
 度使留再任。初,覃恩,蕃官例不序遷。至是,用戡奏始皆得遷。又請首領有戰功材武,皆得召見,選補為蕃官。延州夾河為兩城,雉堞頗卑小。敵登九州臺,則下瞰城中。戡調兵夫大增築之。橫山酋豪怨諒祚,欲率其屬叛,取靈、夏,來求兵為援。戡言:「豺虎非自相搏,則未易取也;癰疽非其自潰,則未易攻也。諒祚久悖慢,宜乘此許之,所謂以蠻夷攻蠻夷,中國之利也。」會英宗不豫,大臣重生事,不報。



 言者請選大臣帥永興,屯重兵以制五路,敕戡
 具利害以聞。戡以為「四路距永興皆十數驛,設有警,使聽節制,則不及事矣。且關中財賦不贍,宿軍多,何以給之?」



 治平初,命宦官王昭明等領四路蕃部事。戡曰:「蕃部所以亡去,苦邊吏苛暴,為西人誘略爾。今昭明等徒能呼召首領,犒以牛酒,恐未足以結其心也。而甚動邊聽,宜更置路分鈐轄、都監,各部一將兵,兼沿邊巡檢使,無復專蕃部事。」從其奏。夏人遣使入貢,僭漢官移文於州,稱其國中官曰樞密。戡止令稱使副不以官,稱樞密曰「
 領廬」,方許之。



 戡告老章累上,終弗聽,遣使以手詔問勞,賜茶藥、黃金,乃再上章曰:「臣老疾劇矣,高奴屯勁兵為要地,豈養病所耶?」召還,道卒。贈太尉,謚康穆。



 戡久在邊,安重習事,治不近名。然不為言者所與,或傳戡交通宦官閻士良,至令妻出見之。



 夏侯嶠,字峻極,其先幽州人。高祖秀,為濟州鉅野鎮游奕使,因家焉。父浦,梁開平中,以明經至棣州錄事參軍。嶠幼好學,弱冠,以辭賦稱,周相李谷延置門下。又依西
 京留守向拱,攝伊陽令;拱移安州,又令攝錄事參軍。



 太平興國初,舉進士甲科,解褐大理評事、通判興州,累遷右贊善大夫。從征太原,督芻糧於河朔。遷殿中丞、通判邠州。歲滿,拜監察御史、通判興元府,進秩殿中。



 雍熙二年代還,對便殿。太宗語有司曰:「此人朕自知其才行,勿須奏擬。」即日改左補闕、直史館,賜緋魚。會王師護邊,乘傳督河間餫道,就命知莫州。逾月,徙洪州,改起居郎。真宗在襄邸,太宗擇朝士謹厚者為官屬,即召入為翊
 善,賜金紫,加直昭文館。真宗尹京府,命兼推官,加司封員外郎。東宮建,復兼中舍,遷工部郎中。及嗣位,拜給事中、知審刑院。數月,擢樞密院副使。



 咸平元年,以戶部郎中罷。二年,始建講讀之職,命嶠為翰林侍讀學士。及楊徽之卒,又命兼秘書監。是秋,江、浙饑,命為江南巡撫使,所過疏理刑訟,存問耆老,務從寬簡,人以為便。使還,採病民二十餘事上之,亟詔厘革。又判吏部選事。



 嶠善鼓琴,好讀莊、老書,淳厚謹慎,居官無過失。真宗尤愛重之,
 多所詢訪,每以善人目之。素好道,留意養生,少疾。景德元年五月,以選人俟對崇政殿,暴中風眩,亟詔取金丹,上尊酒餌之,肩輿還第,遣內侍召外內名醫診視。其夕卒,年七十二。詔贈兵部尚書,賵賜外,增賜白金三百兩給葬。錄其子大理寺丞晟為太子中舍,孫恭為奉禮郎,侄孫蔚賜同學究出身。嶠在近侍,恩遇甚渥。卒後數月,畢士安為相,撫坐嘆曰:「使夏侯君在,吾豈先據此位!」有集十五卷。



 大中祥符初,晟上《漢武封禪圖》,繢金匱、玉匱、
 石□感、石距之狀,咸有注釋,上覽而善之。至駕部員外郎。恭至太子中舍。



 盛度,字公量,世居應天府,後徙杭州餘杭縣。曾祖璫,仕錢氏為餘杭縣令。父豫,從錢俶入朝,終尚書度支郎中。度舉進士第,補濟陰尉。選為封丘主簿,改府倉曹參軍,為光祿寺丞、御史臺推勘官,改秘書省秘書郎。試學士院,為直史館、三司戶部判官,累遷尚書屯田員外郎。



 契丹寇邊,從幸大名,數上疏論邊事。奉使陜西,因覽疆域,
 參質漢、唐故地,繪為《西域圖》以獻。改開封府判官,坐決獄失實,降監洪州稅。起知建昌軍、三司鹽鐵判官,改起居舍人、知制誥。度嘗奏事便殿,真宗問其所上《西域圖》,度因言:「酒泉、張掖、武威、敦煌、金城五郡之東南,自秦築長城,西起臨洮,東至遼碣,延袤萬里。有郡、有軍、有守捉,襟帶相屬,烽火相望,其為形勢備御之道至矣。唐始置節度,後以宰相兼領,用非其人,故有河山之險而不能固,有甲兵之利而不能御。今復繪山川、道路、壁壘、區聚,
 為《河西隴右圖》,願備上覽。」真宗稱其博學。



 後遷右諫議大夫、權知開封府。以疾不拜,改會靈觀判官,入翰林為學士,加史館修撰。歷兵部郎中、景靈宮副使。寇準罷相,度以交通周懷政,出知光州。乾興初,再謫和州團練副使。丁謂貶,起為祠部郎中,復兵部郎中,遷太常少卿、知筠州,更虔、滁、蘇三州。還知審刑院,以右諫議大夫知揚州,加集賢院學士。



 初,度謫洪州,建請復賢良方正科,又請建四科以取士,曰:博通墳典達於教化科,才識兼茂
 明於體用科,軍謀宏遠堪任將帥科,明曉法律能按章覆問科。既而用夏竦議,置六科,其議亦自度始。



 復為翰林學士、史館修撰,遷給事中。嘗受詔與御史中丞王隨議通解鹽,聽商旅入錢算鹽,語在《食貨志》。尋進承旨,以禮部侍郎兼端明殿學士,召問邊計,退而條十事上之。又兼侍讀學士。



 景祐二年,拜參知政事。時王曾、呂夷簡為相,度與宋綬、蔡齊並參知政事,曾與齊善,而夷簡與綬善,惟度不得志於二人。及二人俱辭相,仁宗問度曰:「
 王曾、呂夷簡力求退,何也?」度對曰:「二人腹心之事,臣不得而知,陛下詢二人以孰可代者,則其情可察矣。」仁宗果以問曾,曾薦齊,又問夷簡,夷簡薦綬,於是四人俱罷,而度獨留。遷知樞密院事。



 章得像既相,以度嘗位其上,即拜武寧軍節度使。坐令開封府吏馮士元強取其鄰所賃官舍,以尚書右丞罷。復知揚州,加資政殿學士、知應天府。暴感風眩,以太子少傅致仕,卒。贈太子太保,謚文肅。



 度好學,家居列圖書,每歸,未嘗釋手。敏於為文,而
 泛濫不精。嘗奉詔同編《續通典》、《文苑英華》,注釋御集。真宗祀汾陰,仁宗在藩邸,詔掌起居箋奏及留司章奏。有《愚穀》、《銀臺》、《中書》、《樞中》四集,又有《中書》、《翰林》二制集。



 天禧三年,詔許中書舍人、給事中、諫議大夫母封郡太君,而學士不預。時度官兵部郎中,因請追封其母,自是學士官未至諫議者,其母皆得封郡君。



 度體肥大,艱於拜起,賓客有拜之者,則俯伏不能興,往往瞪視而詬詈之。性極猜險,雖平居,僚友不敢易語言。所至,下貧無賴,多所
 縱舍;稍有貲者,一切繩之以法。



 子申甫,終尚書兵部郎中、集賢校理,嘗為福建轉運使,頗以修潔稱。



 從兄京,有吏能,以尚書工部侍郎致仕,卒。



 丁度,字公雅,其先恩州清河人。祖顗,後唐清泰初陷契丹,逃歸,徙居祥符。父逢吉,以醫術事真宗藩邸,然好聚書,與儒者游。度強力學問,好讀《尚書》,尚擬為《書命》十餘篇。大中祥符中,登服勤詞學科,為大理評事、通判通州,改太子中允、直集賢院。坐解送國子監進士失實,監齊
 州稅。還知太常禮院,判吏部南曹。上書論六事:一、增講讀官;二、增諫員;三、補蔭用大功以上親;四、選河北、河東役兵補禁軍;五、籍令佐墾田為殿最;六、凡緣公事坐私罪仗者,聽保任遷官。章獻太后善之。



 舊制,監司及藩鎮辭謁皆賜對。仁宗初即位,止令附中書、樞密奏之,度言,附奏非所以防壅蔽也。又嘗獻《王鳳論》於章獻太后,以戒外戚。歷三司磨勘司、京西轉運使。司天言永昌陵有白氣,請增築以厭之,有詔按視。度奏神道貴靜,不可輕
 繕治,乃止。入知制誥,遷翰林學士,糾察在京刑獄,判太常禮院兼群牧使。



 劉平、石元孫敗,帝遣使問所以御邊。度奏曰:「今士氣傷沮,若復追窮巢穴,饋糧千里,輕用人命以快一朝之意,非計之得也。唐都長安,天寶後,河、湟覆沒,涇州西門不開,京師距寇境不及五百里,屯重兵,嚴烽火,雖常有侵軼,然卒無事。太祖時,疆場之任,不用節將。但審擢材器,豐其廩賜,信其賞罰,方陲輯寧幾二十年。為今之策,莫若謹亭障,遠斥堠,控扼要害,為制御
 之全計。」因條上十策,名曰《備邊要覽》。



 時西疆未寧,二府三司,雖旬休不廢務。度言:「苻堅以百萬師寇晉,謝安命駕出游以安人心。請給假如故,無使外夷窺朝廷淺深。」從之。累遷中書舍人,為承旨。



 時葉清臣請商州置監鑄大錢,以一當十。度奏曰:「漢之五銖,唐之開元及國朝錢法,輕重大小,最為折中。歷代改更,法雖精密,不能期年,即復改鑄。議者欲繩以峻法,革其盜鑄。昔漢變錢幣,盜鑄死者數十萬。唐鑄乾元及重輪乾元錢,錢輕幣重,嚴
 刑不能禁止。今禁旅戍邊,月給百錢,得大錢裁十,不可畸用,舊錢不出,新錢愈輕,則芻糧增價。臣嘗知湖州,民有抵茶禁者,受千錢立契代鞭背。在京西,有強盜殺人,取其弊衣,直不過數百錢。盜鑄之利,不啻數倍。復有湖山絕處,兇魁嘯聚,爐冶日滋,居則鑄錢,急則為盜。民間銅鉛之器,悉為大錢,何以禁止。」



 度又言:「祥符、天聖間,牧馬至十餘萬,其後言者以天下無事,不可虛費,遂廢八監。然猶秦渭環階麟府文州、火山保德岢嵐軍,歲市馬
 二萬二百匹,補京畿、塞下之闕。自西鄙用兵,四年所牧,三萬而已。馬少地閑,坊監誠可罷;若賊平馬歸,則不可闕。今河北、河東、京東西、淮南皆籍丁壯為兵,請令民畜一戰馬者,得免二丁,仍不許貲產以升戶等,則緩急有備,而國馬蕃矣。」



 慶歷中,副杜衍宣撫河東。久之,遷端明殿學士、知審刑院。時江西轉運使移屬州,凡市米鹽鈔,每百緡貼納錢三之一。通判吉州李虞卿受財免貼納,事覺,大理將以枉法論。度曰:「枉法,謂於典憲有所阿曲。
 虞卿所違者,轉運使移文爾。」遂貸虞卿死。



 帝嘗問,用人以資與才孰先?度對曰:「承平時用資,邊事未平宜用才。」時度在翰林已七年,而朝廷方用兵,故對以此。諫官孫甫論度所言,蓋自求柄用,帝諭輔臣曰:「度在侍從十五年,數論天下事,顧未嘗及私,甫安從得是語。」



 未幾,擢工部侍郎、樞密副使。因言:「周世宗募驍健,有朝出群盜、夕備宿衛者;太祖閱猛士實騎軍。請擇河北、河東、陜西就糧馬軍,以補禁旅之闕。」又言:「契丹嘗渝盟,預備不可忽。」
 因上《慶歷兵錄》五卷、《贍邊錄》一卷。明年,參知政事。會春旱,降秩中書舍人,逾月,復官。



 後二年,衛士為變,事連宦官楊懷敏,樞密使夏竦請御史與宦官同於禁中鞫之,不可滋蔓,令反側者不自安。度曰:「宿衛有變,事關社稷,此而可忍孰不可忍!請付外臺窮治黨與。」爭於帝前。仁宗從竦言,度遂求解政事,罷為紫宸殿學士兼侍讀學士。御史何郯言,紫宸非官稱所宜。改觀文殿學士、知通進銀臺司、判尚書都省,再遷尚書左丞,卒。贈吏部尚書,
 謚文簡。



 度性淳質,不為威儀,居一室十餘年,左右無姬侍。然喜論事,在經筵歲久,帝每以學士呼之而不名。嘗問蓍龜占應之事,乃對:「卜筮雖聖人所為,要之一技而已,不若以古之治亂為監。」又嘗示以欹器曰:「朕欲臨天下以中正之道。」度對曰:「臣等亦願無傾滿以事陛下。」因奏太宗嘗作此器,真宗亦嘗著論,於是帝制《後述》以賜之。



 度著《邇英聖覽》十卷、《龜鑒精義》三卷、《編年總錄》八卷,奉詔領諸儒集《武經總要》四十卷。子諷,集賢校理。



 張觀,字思正,絳州絳縣人。少謹願好學,有鄉曲名。中服勤辭學科,擢為第一,授將作監丞、通判解州。會鹽池吏以贓敗,坐失舉劾,降監河中府稅。復通判果州,改秘書省秘書郎。



 仁宗即位,遷太常丞,擢右正言、直史館,為三司度支判官,同修起居注,改右司諫、知制誥、判登聞檢院,出知杭州。還判國子監,權發遣開封府事,進為翰林學士、知審官院,累遷左司郎中,以給事中權御史中丞。



 時星流、地震、雷發正月,詔求直言。觀謂:「承平日久,政寬
 法慢,用度漸侈,風俗漸薄,以致災異。」因上四事:一曰知人,二曰嚴禁,三曰尚質,四曰節用。河北大雨水,又條七事,曰:「導積水以廣播種,緩催欠以省禁錮,寬刑罰以振淹獄,收逃田以募歸復,罷工役以先急務,止配率以阜民財,通商旅以濟艱食。復知審官院,遂拜同知樞密院事。



 康定中,西兵失利,因議點鄉兵,久之不決,遂與王鬷、陳執中俱罷,以資政殿學士、尚書禮部侍郎知相州。徙澶州。河壞孫陳埽及浮梁,州人大恐,或請趨北原以避
 水患。觀曰:「太守獨去,如州民何。」乃躬率卒徒增築之,堤完,水亦退。



 徙鄆州。舊法,亦東通安邑鹽,而瀕海之地禁私煮。觀上言:「利之所在,百姓趨之,雖日殺於市,恐不能止,請弛禁以便民。」歲免黥配者不可勝計。歷知應天府、孟州、河南府,以吏部侍郎兼御史中丞。以父居業高年多病,請便郡,以觀文殿學士知許州。月餘,拜左丞。丁父憂,哀毀過人,既練而卒。贈吏部尚書,謚文孝。



 觀性至孝,初為秘書郎,其父方為州從事,因上書願以官授父。真
 宗嘉之,以居業為京官。及觀貴,居業繇恩至太府卿。居業嘗過洛,嘉其山川風物,曰:「吾得老於此足矣。」觀於是買田宅、營林榭,以適其意。早起奉藥、膳,然後出視事,未嘗一日廢也。趣尚恬曠,持廉少欲,平生書必為楷字,無一行草,類其為人。仁宗飛白書「清」字賜觀,以賞其節。然於吏事非所長,知開封府,民犯夜禁,觀詰之曰:「有人見否?」眾傳以為笑。



 鄭戩,字天休,蘇州吳縣人。早孤力學。客京師,事楊億,以
 屬辭知名,後復還吳。及億卒,賓客弟子散去,戩乃倍道會葬。舉進士,擢甲科,授太常寺奉禮郎、簽書寧國軍節度判官事,召試學士院,為光祿寺丞、集賢校理、通判越州。還,改太子中允、同知太常禮院,注釋禦制《發願文》、《三寶贊》,升直史館、三司戶部判官,同修起居注,以右正言知制誥。判國子監;選明經生講解經義。徙知審刑院,遷起居舍人、龍圖閣直學士、權知開封府。



 吏馮士為奸利,有告士元受賕藏禁書者,戩窮治之。辭連宰相呂夷
 簡、知樞密院盛度、參知政事程琳,遂逮捕夷簡子公綽、公弼參劾其狀。既而士元流海島,度、琳坐嘗交關士元罷去,其餘絀罰者自御史中丞孔道輔、天章閣待制龐籍又十餘人,朝議畏其皦核。戩敏強善聽決,喜出不意,獨假貸細民,即豪宗大姓,繩治益急,政有能跡。徙權三司使,復轉運使考課格,分別殿最。又勾較三司出入,得羨錢四百萬緡,以右諫議大夫、同知樞密院改樞密副使。



 戩與參知政事宋庠,為宰相呂夷簡所忌,與庠皆罷,
 以資政殿學士知杭州。錢塘湖溉民田數十頃,錢氏置撩清軍,以疏淤填之患。既納國後不復治,葑土堙塞,為豪族僧坊所占冒,湖水益狹。戩發屬縣丁夫數萬闢之,民賴其利。事聞,詔本郡歲治如戩法。



 遷給事中,徙並州,道改鄆州,又徙永興軍。建言:「凡軍行所須,願下有司相緩急,析為三等,非急罷去。」先是,衙吏輸木京師,浮渭泛河,多漂沒,既至,則斥不中程,往往破家不能償,戩奏歲減二十餘萬;又奏罷括糴,以勸民積粟。長安故都多豪
 惡,戩治之尚嚴,甚者至黥竄,人皆惕息。



 未幾,為陜西四路都總管兼經略、安撫、招討使,駐涇州,聽便宜從事。遷尚書禮部侍郎。時知慶州滕宗諒、知渭州張亢過用公使錢,戩致於法。行邊至鎮戎軍,趣蓮花堡,天寒,與將佐置酒,元昊擁兵近塞。會暮塵起,有報敵騎至者,戩曰:「此必三川將按邊回,非敵騎也。」已而果然。及疆事少寧,詔還,知永興軍。



 初,靜邊砦主劉滬謀築水洛、結公二城,以通秦、渭援兵,招生羌大王族為邊衛。戩使滬與著作佐
 郎董士廉督其役。會罷戩四路,宣撫使韓琦、知渭州尹洙皆以為不便,召滬、士廉罷役歸,不聽。乃使裨將狄青將兵以往,械送德順軍獄。戩力爭於朝,卒城之。



 進戶部侍郎、資政殿大學士、知並州。契丹與元昊方交兵,邊奏互上,獨戩不以聞。詔遣使問其故,戩對曰:「敵自相攻,中國不足憂也。」麟、府間有棄地曰草城川,戩募土人為弓箭手,計口給田。初,兵興,用不足。河東行鐵錢,山多炭、鐵,鼓鑄利厚,重闢不能止。戩乃請三當一。令既下,兵民相
 扇動,數千人邀走馬承受訴。承受,中貴人,不能遏。又群噪州門,守門者拒不得入。戩聞,悉召至庭下,推首謀者數十人,黥隸他州,事乃定。



 遷吏部侍郎,改宣徽北院使,拜奉國軍節度使,卒。贈太尉,謚文肅。戩遇事,果敢必行。然憑氣近俠,用刑峻深,士民多怨之。



 明鎬字化基,密州安丘人。中進士第,補蘄州防禦推官。真宗崩,上《真頌》四十六篇,改大理寺丞。薛奎領秦州,闢為節度判官。奎徙益州,闢知錄事參軍。程琳代奎,奏為
 簽書節度判官,就通判州事,遷太常博士。還朝,仁宗問鎬所能,奎稱其沉鷙有謀,能斷大事,除開封推官。獻《六冗書》,進尚書祠部員外郎,為三司戶部判官,改刑部員外郎、京東轉運使,遷兵部員外郎、直史館、益州路轉運使。會歲饑,民無積聚,盜賊間發,鎬為平物價,募民為兵,人賴以安。



 知陵州,楚應幾贓敗,或告以先期奏之,鎬曰:「獲罪則已,安可欺朝廷耶?」卒坐失察,降知同州。未逾月,會元昊寇延州,起為陜西轉運使。虜破金明砦,既去,議
 修復其城,帥臣擁兵不即進,而鎬止以百餘騎,自督將士,一月而成。又嘗閱同州廂軍,得材武者三百餘人,教以強弩,奏為清邊軍,號最驍悍。其後,陜西、河東頗仿置之。



 遷戶部郎中、直昭文館、知陜州,徙江、淮制置發運使。未行,會賊破豐州,擢天章閣待制、河東都轉運使。修建寧中候百勝砦、鎮川清塞堡,凡五城,以勞遷左司郎中。



 明年,擢龍圖閣直學士、知並州。鎬大巡邊以備賊。時邊任多紈褲子弟,鎬乃取尤不職者杖之,疲軟者皆自解
 去,遂奏擇習事者守堡砦。軍行,娼婦多從之,鎬欲驅逐,惡傷士卒心,會有忿爭殺娼婦者,吏執以白,鎬曰:「彼來軍中何耶?」縱去不治,娼婦聞皆散走。以樞密直學士、左諫議大夫知成德軍,入知開封府。



 王則叛,命鎬為體量安撫使;則未下,又命參知政事文彥博為宣撫使,以鎬副之。貝州平,遷端明殿學士、給事中、權三司使,諸將悉超遷,都虞候、士卒八千四百人,第其功為五等,每等遷一資。彥博數推鎬功,拜參知政事。



 已而疽發背,帝謂輔
 臣曰:「鎬忠亮有勞,及其未亂,思一見之。」臨問,惻然曰:「方賴卿謀國事,何遽被疾!」鎬氣憊,猶能頓首謝。翌日,卒,謚文烈。鎬端挺寡言,所至安靜有體,而遇事不茍,為世所推重。



 王則者,本涿州人。歲饑,流至恩州,自賣為人牧羊,後隸宣毅軍為小校。恩、冀俗妖幻,相與習《五龍》、《滴淚》等經及圖讖諸書,言釋迦佛衰謝,彌勒佛當持世。初,則去涿,母與之訣別,刺「福」字於其背以為記。妖人因妄傳字隱起,爭信事之,而州吏張巒、卜吉主其謀,黨連德、齊諸
 州,約以慶歷八年正旦,斷澶州浮梁,亂河北。會其黨潘方凈以書謁北京留守賈昌朝,事覺被執,故不待期,亟以七年冬至叛。



 時知州張得一方與官謁天慶觀,則率其徒劫庫兵,得一走保驍捷營。賊焚門,執得一囚之。兵馬都監、內殿承制田斌以從卒巷斗,不勝而出。城扉闔,提點刑獄田京、任黃裳持印,棄其家縋城出,保南關。賊從通判董元亨取軍資庫鑰,元亨拒之,殺元亨。又出獄囚,囚有憾司理參軍王獎者,遂殺獎。既而節度判官李
 浩、清河令齊開、主簿王浟皆被害。



 則僭號東平郡王,以張巒為宰相,卜吉為樞密使,建國曰安陽。榜所居門曰中京,居室廄庫皆立名號,改年曰得聖,以十二月為正月。百姓年十二以上、七十以下,皆涅其面曰「義軍破趙得勝」。旗幟號令,率以「佛」為稱。城以一樓為一州,書州名,補其徒為知州,每面置一總管。然縋城下者日眾。於是令守者伍伍為保,一人縋,餘悉斬。



 有州民汪文慶、郭斌、趙宗本、汪順者,自城上系書射鎬帳,約為內應,夜垂緪
 以引官軍。既內數百人,焚樓櫓,賊覺,率眾拒戰。初,官軍既登,欲專其功,斷緪以絕後來者。及與賊戰,兵寡不敵,與文慶等復縋而下。是夜,城幾克。則期正月十四日出要劫契丹使,諜者以告。鎬遣殿侍安素伏兵西門,賊果以數百人夜出,伏發,皆就獲。



 城峻不可攻,乃為距闉,將成,為賊所焚。遂即南城為地道,日攻其北牽制之。及文彥博至,穴通城中,選壯士中夜由地道入,眾登城。賊縱火牛,官軍以槍中牛鼻,牛還攻之,賊大潰,開東門遁。閣
 門祗候張姻緣壕與戰,死之。總管王信捕得則,其餘眾保村舍,皆焚死。檻送則京師,支解以徇。則叛凡六十六日。



 王堯臣,字伯庸,應天府虞城人。舉進士第一,授將作監丞、通判湖州。召試,改秘書省著作郎、直集賢院。會從父沖坐事,出堯臣知光州。父喪,服除,為三司度支判官,再遷右司諫。



 郭皇后薨,議者歸罪內侍都知閻文應,堯臣請窮治左右侍醫者,不報。時上元節,有司張燈,堯臣俟
 乘輿出,即上言:「後已復位號,今方在殯,不當游幸。」帝為罷張燈。擢知制誥、同知通進銀臺司、提舉諸司庫務,知審刑院,入翰林為學士、知審官院。



 陜西用兵,為體量安撫使。將行,請曰:「故事,使者所至,稱詔存問官吏將校,而不及於民。自元昊反,三年於今,關中之民凋弊為甚,請以詔勞來,仍諭以賊平蠲租賦二年。」仁宗從之。



 使還,上言:



 陜西兵二十萬,分屯四路,然可使戰者止十萬。賊眾入寇,常數倍官軍。彼以十戰一,我以一戰十,故三至而
 三勝,由眾寡不侔也。涇原近賊巢穴,最當要害,宜先備之。今防秋甚邇,請益團士兵,以二萬屯渭州,為鎮戎山外之援;萬人屯涇州,為原、渭聲勢;二萬屯環慶,萬人屯秦州,以制其沖突。



 且賊之犯邊,不患不能入,患不能出也。並塞地形,雖險易不同,而兵行須由大川,大川率有砦柵為控扼。賊來利在虜掠,人自為戰,故所向無前。若延州之金明、塞門砦,鎮戎之劉璠、定川堡,渭州山外之羊牧隆城、靜邊砦,皆不能扼其來。故賊不患不能入也。
 既入漢地,分行鈔略,驅虜人畜,劫掠財貨,士馬疲困,奔趨歸路,無復鬥志。若以精兵扼險,強弩注射,旁設奇伏,斷其首尾,且追且擊,不敗何待。故賊之患在不能出也。



 賊屢乘戰勝,重掠而歸,諸將不能追擊者,由兵寡而勢分也。若尚循故轍,必無可勝之理。



 又論:「延州、鎮戎軍、渭州山外三敗之由,皆為賊先據勝地,誘致我師,將帥不能據險擊歸,而多倍道趨利。兵方疲頓,乃與生羌合戰;賊始縱鐵騎沖我軍,繼以步卒挽強注射,鋒不可當,遂
 致掩覆,此主帥不思應變以懲前失之咎也。願敕邊吏,常遠斥候,遇賊至,度遠近立營砦,然後量敵奮擊,毋得輕出。」詔以其言戎邊吏。



 時韓琦坐好水川兵敗徙秦州,範仲淹亦以擅復元昊書降耀州。堯臣言:二人者,皆忠義智勇,不當置之散地。又薦種世衡、狄青有將帥才。明年,賊果自鎮戎軍、原州入寇,敗葛懷敏,乘勝掠平涼、潘原,關中震恐,自邠、涇以東,皆閉壘自守。仲淹自將慶州兵捍賊,賊引去。仁宗思其言,乃復以琦、仲淹為招討使,
 置府涇州,益屯兵三萬人,而使堯臣再安撫涇原。



 初,曹瑋開山外地,置籠竿等四砦,募弓箭手,給田使耕戰自守。其後將帥失撫御,稍侵奪之,眾怨怒,遂劫德勝砦將姚貴,閉城畔。堯臣適過境上,作書射城中,諭以禍福,眾遂出降。乃為申明約束如舊而去。



 既還,上言:「自陜西用兵,夏竦、陳執中並以兩府舊臣,為陜西經略、安撫、招討使,韓琦、範仲淹止為經略、安撫副使。既而張存知延州,王沿知渭州,張奎知慶州,俱是學士、待制之職,亦止管
 勾本路總管司事。及竦、執中罷,四路置帥,遂各帶都總管及經略、安撫、招討等使,因而武臣副總管亦為副使。今琦、仲淹、龐籍既為陜西四路都總管、緣邊經略安撫招討等使,四路當稟節制,而尚帶經略使名者九人,各置司行事。名號不異,而所稟非一。今請逐路都總管、副總管並罷經略,只充緣邊安撫使。」既而滕宗諒亦以為請,遂罷之。



 又言:「鄜延、環慶路,其地皆險固而易守;惟涇原自漢、唐以來,為沖要之地。自鎮戎軍至渭州,沿涇河大川直抵
 涇、邠,略無險阻。雖有城砦據平地,賊徑交屬,難以捍防,如郭子儀、渾瑊,常宿重兵守之。自元昊叛命數年,由此三入寇。朝廷置帥府於涇州,為控扼關、陜之會,誠合事機。然頻經敗覆,邊地空虛,士氣不振。願深監近弊,精擇將佐;其新集之兵,未經訓練,宜易以舊人。儻一路兵力完實,則賊不敢長驅入寇矣。」因論沿邊城砦、控扼要害、賊徑通屬及備御輕重之策為五事上之。又請涇、原五州營田,益置弓箭手,及請徹潼關樓櫓,皆報可。



 以戶部
 郎中權三司使,闢張溫之、杜杞等十餘人為副使、判官。時入內都知張永和建議,收民僦舍錢十之三以助軍費。堯臣入對曰:「此衰世之事,召怨而攜民,唐德宗所以致朱泚之亂也。」度支副使林濰畏永和,附會其說,堯臣奏黜濰,議乃定。



 夔州轉運使請增鹽井歲課十餘萬緡,堯臣以為上恩未嘗及遠人,而反牟取厚利,適足以斂怨,罷之。遷翰林學士承旨兼端明殿學士,為群牧使。丁母喪,服除,轉右諫議大夫。



 初,學士蘇易簡、丁度皆自郎中
 進中書舍人充承旨,及堯臣為承旨,不遷官,意宰相賈昌朝所抑。及是,文彥博為相,因其歲滿,遂優遷之。大享明堂,加給事中。與三司更議茶法,較天下每歲財賦出入,上其數,遂拜樞密副使。



 會儂智高反,請析廣西宜、容、邕州為三路,以融、柳、象隸宜州,白、高、竇、雷、化、鬱林、儀、藤、梧、龔、瓊隸容州,欽、賓、廉、橫、潯、貴隸邕州;遇蠻入寇,三路會支郡兵掩擊,令經略、安撫使守桂州以統制焉;益募澄海、忠敢土軍分屯,運全、永、道三州米以餉之,罷遣
 北兵遠戍。時狄青經制嶺南,詔青審議,以為便。



 居樞密三年,務裁抑徼幸,於是有鏤匿名書以布京城,然仁宗不以為疑也。以戶部侍郎參知政事。久之,帝欲以為樞密使,而當制學士胡宿固抑之,乃進吏部侍郎。卒,贈尚書左僕射,謚文安。



 堯臣以文學進,典內外制十餘年,其為文辭溫麗。執政時,嘗與宰相文彥博、富弼、劉沆勸帝早立嗣,且言英宗嘗養宮中,宜為後,為詔草挾以進,未果立。



 元豐三年,子同老進遺稿論父功,帝以訪文彥博,
 具奏本末,遂加贈太師、中書令,改謚文忠。



 孫抃,字夢得,眉州眉山人。六世祖長孺,喜藏書,號「書樓孫氏」,子孫以田為業。至抃始讀書屬文。中進士甲科,以大理評事通判絳州。召試學士院,除太常丞、直集賢院,為開封府推官,判三司開拆司,同修起居注,以右正言知制誥,遷起居舍人、翰林學士兼侍讀學士、史館修撰,累遷尚書吏部郎中。抃雖久處顯要,罕所建明。



 皇祐中,以右諫議大夫權御史中丞。制下,諫官韓絳論奏抃非糾
 繩才,不可任風憲。抃即手疏曰:「臣觀方今士人,趨進者多,廉退者少。以善求事為精神,以能訐人為風採;捷給若嗇夫者謂之有議論,刻深若酷吏者謂之有政事。諫官所謂才者,無乃謂是乎?若然,臣誠不能也。」仁宗察其言,趣視事,且命知審官院。抃辭以任言責不當兼事局,乃止。



 在臺,數言事,不為矯激,尤喜稱薦人才。帝欲除入內都知王守忠領武寧軍節度使,抃奏罷之。溫成皇后葬,以劉沆為監護使,抃奏沆為宰相,不當為后妃護葬喪
 事。時又議為後建陵立廟,抃率官屬言非禮。因相與請對,固爭不能得,伏地不起,帝為改容遣之。御史請罷宰相梁適,未聽,抃奏曰:「適在相位,上不能持平權衡,下不能篤訓子弟。言事官數論奏,未聞報可,非罷適無以慰物論。」宰相陳執中婢為嬖妾張氏榜殺,置獄取證左,執中弗遣,有詔勿推。抃復與官屬請對論列,疏十上,適、執中卒皆罷。



 改翰林學士承旨,復兼侍讀學士。帝讀《史記龜策傳》,問:「古人動作必由此乎?」對曰:「古有大疑,既決於
 己,又詢於眾,猶謂不有天命乎,於是命龜以斷吉兇。所謂『謀及乃心,謀及卿士,謀及庶人,謀及卜筮』。蓋聖人貴誠,不專人謀,默與神契,然後為得也。」帝善其對。



 諫官陳升之上選用、責任、考課轉運使三法,命抃與御史中丞張升典之,卒亦無所進退焉。再遷禮部侍郎。抃久居侍從,泊如也,人以為長者。既而樞密副使程戡罷,帝欲用舊人,即以命抃。歲中,參知政事。



 抃性篤厚寡言,質略無威儀。居兩府,年益耄,無所可否。又善忘,語言舉止多可
 笑,好事者至傳以為口實。御史韓縝彈奏之,罷為觀文殿學士、同群牧制置使,復兼侍讀學士。英宗即位,進戶部侍郎。告老,以太子少傅就第,卒。贈太子太保,謚文懿。



 田況,字符均,其先冀州信都人。晉亂,祖行周沒於契丹。父延昭,景德中脫身南歸,性沉鷙,教子甚嚴,累官至太子率府率。況少卓犖有大志,好讀書。舉進士甲科,補江陵府推官,再調楚州判官,遷秘書省著作佐郎。舉賢良方正,改太常丞、通判江寧府。



 趙元昊反,夏竦經略陜西,
 闢為判官。時竦與韓琦、尹洙等畫上攻守二策,朝廷將用攻策,範仲淹議未可出師。況上疏曰:



 昔繼遷擾邊,太宗部分諸將五路進討,或遇賊不擊,或戰衄而還。又嘗令白守素、馬紹忠護送糧餉於靈州,諸將多違詔自奮,浦洛河之敗,死者數萬人。今將帥士卒,素已懦怯,未甚更練。又知韓琦、尹洙同建此策,恐未甚稟服,臨事進退,有誤大舉。其不可一也。



 計者以為賊常並力而來,我常分兵以御,眾寡不敵,多貽敗衄,今若全師大舉,必有成功,
 此思之未熟爾。夫三軍之命,系於將帥。人之才有大小,智有遠近,以漢祖之善將,不若淮陰之益辦,況庸人乎?今徙知大眾可以威敵,而不思將帥之才否,此禍之大者也。兩路之人,眾十餘萬,庸將驅之,若為舒卷;賊若據險設伏,邀截沖擊,首尾前後,勢不相援,一有不利,則邊防莫守,別貽後患。安危之計,決於一舉。其不可二也。



 自西賊叛命以來,雖屢乘機會,然終不敢深寇郡縣,以厭其欲者,非算之少也。直以中國之大,賢俊之盛,甲兵之
 眾,未易可測。今師深入,若無成功,挫國威靈,為賊輕侮,或別墮奸計,以致他虞。其不可三也。



 計者又云,將帥雖未足倚,下流勇進,或有其人。自劉平、石元孫陷沒,士氣挫怯,未能振起。今兵數雖多,疲懦者眾,以庸將驅怯兵,入不測之地,獨其下使臣數輩,干賞蹈利,欲邀奇功,未見其利。其不可四也。



 計者又云,非欲深絕沙磧,以窮妖巢,但淺入山界,以挫賊氣,如襲白豹城之比。臣謂乘虛襲掠,既不能破戎首、拉兇黨,但殘戮孥弱,以厚怨毒,非
 王師吊伐招徠之體。然事出無策,為彼之所為,亦當霆發雷逝,往來輕速,以掩其不備。今興師十萬,鼓行而西,賊已清野據險以待,我師何襲挫之有?其不可五也。



 自元昊寇邊,人皆知其誅賞明、計數黠。今未有間隙可窺,而暴為興舉,計事者但欲決勝負於一戰。幸其或有所成,否則願自比王恢以待罪,勇則勇矣,如國事何。其不可六也。



 昨仲淹奏乞朝廷,敦包荒之量,存鄜延一路。今諸將勒兵嚴備,未行討伐,容示以恩意,歲時之間,或可
 招納。若使涇原一路獨入,則孤軍進退,憂患不淺。傳聞賊謀,俟我師諸路入界,並兵以敵,此正陷賊計中。其不可七也。



 以臣所見,夏竦、韓琦、尹洙同獻此策,今若奏乞中罷,則是自相違異;欲果決進討,則又仲淹執議不同。乞召兩府大臣定議,但令嚴設邊備,若有侵掠,即出兵邀擊;或賊界謹自守備,不必先用輕舉。如此則全威制勝,有功而無患也。



 於是罷出師議。



 況又言治邊十四事。遷右正言,管勾國子監、判三司理欠憑由司,專供諫職,
 權修起居注,遂知制誥。嘗面奏事,論及政體,帝頗以好名為非,意在遵守故常,況退而著論上之。其略曰:



 名者由實而生,非徒好而自至也。堯、舜三代之君,非好名者。而鴻烈休德,倬若日月,不能纖晦者,有實美而然也。設或謙弱自守,不為恢閎睿明之事,則名從而晦矣,雖欲好之,豈可得耶。



 方今政令寬弛,百職不修,二虜熾結,凌慢中國,朝廷恫矜下民橫罹殺掠,竭瀝膏血,以資繕備,而未免侵軼之憂。故屈就講和,為翕張予奪之術。自非
 君臣朝夕恥憤,大有為以遏後虞,則勢可憂矣。陛下若恐好名而不為,則非臣之所敢知也。陛下倘奮乾剛,明聽斷,則有英睿之名;行威令,懾奸宄,則有神武之名;斥奢汰,革風俗,則有崇儉之名;澄冗濫,輕會斂,則有廣愛之名;悅亮直,惡巧媚,則有納諫之名;務咨詢,達壅蔽,則有勤政之名;責功實,抑偷幸,則有求治之名。今皆非之而不為,則天下何所望乎?抑又聖賢之道曰名教,忠誼之訓曰名節,群臣諸儒所以尊輔朝廷,紀綱人倫之
 大本也。陛下從而非之,則教化微,節義廢,無恥之徒爭進,而勸沮之方不行矣,豈聖人率下之意耶。



 時邊奏契丹修天德城及多葺堡砦。況意其蓄奸謀,乃上疏曰:



 朝廷予契丹金帛歲五十萬,朘削生民,輸將道路,疲弊之勢,漸不可久。而近西羌通款,歲又予二十萬,設或復肆貪瀆,再有規求,朝廷尚可從乎?臣至愚,不當大責,每念至此,則惋嘆不已。矧兩府大臣,皆宗廟社稷、天下生民所望而系安危者,豈不為陛下思之哉?每旦垂拱之對,
 不過目前政事數條而已,非陛下所以待輔臣,非輔臣所以憂朝廷之意也。



 有唐故事,肅宗以天下未乂,除正衙奏事外,別開延英以詢訪宰相,蓋旁無侍衛,獻可替否,曲盡討論。今北敵桀慢,而河朔將佐之良愚,中兵之善窳,道路之夷險,城壘之堅弊,軍政之是否,財糧之多少,在兩府輔臣,實未有知之者。萬一變發所忽,制由中出,少有差跌,則事不測矣。如前歲蕭英、劉六符始來,和議未決,中外惶擾,不知為計,此臣所目睹也。和議既定,又
 復恬然若無事者,是豈得為安哉。



 願因燕閑,召執政大臣於便殿,從容賜坐,訪逮時政,專以慮患為急。則人人惟恐不知以誤應對,事事惟恐不集以孤聖懷,旦夕憂思,不敢少懈,同心協力,必有所為。今不以此為務,而日以委瑣之事,更相辯對,議者羞之。臣叨備近列,實系朝廷休戚,惟陛下不以人廢言。



 尋為陜西宣撫副使,還領三班院。保州雲翼軍殺州吏據城叛,詔況處置
 之。既而除龍圖閣直學士、知成德軍。況督諸將攻,以敕榜招降叛卒二千餘人,坑其構逆者四百二十九人,以功遷起居舍人。徙秦州。丁父憂,詔起復,固辭。又遣內侍持手敕起之,不得已,乞歸葬陽翟。既葬,托邊事求見,泣請終制,仁宗惻然許之。帥臣得終喪自況始。服除,以樞密直學士、尚書禮部郎中知渭州。



 遷右諫議大夫、知成都府。蜀自李順、王均再亂,人心易搖,守得便宜決事,多擅殺以為威,雖小罪,猶並妻子徙出蜀,至有流離死道路者。況至,拊循教誨,非有甚惡不使遷,蜀人尤愛之。



 遷給事中,召為御史中丞。既至,權三司使,加龍圖閣學士、翰林學士。況鉤考財賦,盡知其出入,乃約《景德會計錄》,以今財賦所入,多於景德,而歲之所出,又多於所入。因著《皇祐會計錄》上之。以禮部侍郎為三司使。至和元年,擢樞密副使,遂為樞密使。以疾,罷為尚書右丞、觀文殿學士兼翰林侍讀學士,提舉景靈宮,遂以太子少傅致仕,卒。贈太子太保,謚宣簡。



 況寬厚明敏,有文武材。與人若無不可,至其所守,人亦不能移也。其論天下事甚多,至並樞密院於中書以一政本,日輪兩制館閣官一員於便殿備訪問,以錫慶院廣太學,興鎮戎軍、原渭等州營田,汰諸路宣毅、廣捷等冗軍,策元昊勢屈納款,必令盡還延州侵地,毋過許歲幣,並入中青鹽,請戮陜西陷歿主將隨行親兵。其論甚偉,然不盡行也。有奏議二十卷。



 始,契丹寇澶州,略得數百人,以屬其父延昭。延昭哀之,悉縱去,因自脫歸中國。延昭生八男,子多
 知名,況長子也。保州之役,況坑殺降卒數百人,朝廷壯其決,後大用之。然卒無子,以兄子為後。



 論曰:時治平而文德用,則士之負藝者致位政府,宜矣。李諮、程戡曉暢吏事。諮變茶法,雖浮議動搖,乍行乍止,卒無能易其說;戡任邊寄,守以安靜,非必智謀,抑所遇之時耳。嶠尚莊、老,以善著稱。張觀、丁度、孫抃,世推其德性淳易,而盛度每為寮友猜憚,心跡固何如也。戩明偉宏放,亦一時之俊。堯臣論議鏗鏗,正誼而不謀利,其最優乎。鎬堅正寡合,馭軍嚴,臨事果,其安撫河東邊塞,後來父老道其舉動措置,輒嗟嘆追思。況有文武才略,言事精暢,然欲懲兵驕,乃坑降卒,弗忌陰禍,惜哉!



\end{pinyinscope}