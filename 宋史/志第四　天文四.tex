\article{志第四 天文四}

\begin{pinyinscope}

 二
 十
 八舍下



 西方



 奎宿十六星,天之武庫,一曰天豕,一曰封豕,主以兵禁暴,又主溝瀆。西南大星曰天豕目,亦曰大將。明動,則兵、水大出。日食,魯國兇,邊兵起及水旱。日暈,為兵,為
 火。月食,聚斂之臣有憂。月暈,兵敗,糴貴,將戮,人疾疫。月犯之,其分亂。歲星犯之,近臣為逆;守之,蟲為災,人民饑,盜起,多獄訟;久守,北兵降;色潤澤,大熟;守二十日以上,兵起魯地;逆行守之,君好兵,民流亡。熒惑犯之,環繞三十日以上,將相兇,大水,民流;守二十日以上,魯地有兵;動搖、進退,有赦;舍,歲大熟;留,臣下專權,多獄訟;守百日以上,多盜。填星入犯,吳、越有兵,一曰齊、魯,一曰兵、喪;守之,有貴女執政;出入,水泉溢。太白犯之,大水,有兵,霜殺
 物;入,則外兵入國;晝見,將相死。辰星犯之,江河決,有兵,為旱,為火。守之,王者憂,兵、旱。客星犯之,有溝瀆事;守,則王者有憂,軍敗,賊臣在側;入之,破軍殺將;舍留不去,人饑;出,則為謀臣惑天子。彗犯,為饑,為兵、喪;出,則有水災。星孛之,其下兵出,民饑,國無繼嗣;出,則西北有兵起。流星入犯之,有溝瀆事,破軍殺將。《乙巳占》:流星出入,色黃白光潤,文昌武偃;赤如火光作聲,為弓弩用;一曰入則有聚眾事。赤雲氣入犯,為兵;黃,為天子喜;黑,則大人
 有憂。



 按漢永元銅儀,以奎為十七度,唐開元游儀十六度。舊去極七十六度,景祐測驗同。



 天溷七星,在外屏南,主天廁養豬之所,一曰天之廁溷也。暗,則人不安;移徙,則憂。



 土司空一星,在角南,一曰天倉,主土事。凡營城邑、浚溝洫、修堤防、則議其利,建其功,四方小大功課,歲盡則奏其殿最而行賞罰。星大、色黃,則天下安。五星犯之,男女
 不得耕織。彗、客犯之,水旱,民流,兵大起,土功興。客星守之,有土功、哭泣事。黃雲氣入,土功興,移京邑。



 策一星,在王良北,天子僕也,主執策御。流星、彗、孛、客星犯之,皆為大兵起,天子自將於野;近之,下有謀亂者。



 附路一星附一作傅,在閣道南旁,別道也。一曰在王良東,主太僕,主御風雨。芒角,則車騎在野;星亡,有道路之變;不具,則兵起。太白、熒惑入,兵起。彗、孛犯之,道路不通。客星入,馬賤。蒼白雲氣入,太僕有憂;赤,為太僕誅;黃白,太僕
 受賜;黑,為太僕死。



 閣道六星,在王良前,飛道也,從紫宮至河神所乘也。一曰主輦閣之道,天子游別宮之道也。星不見,則輦閣不通;動搖,則宮掖有兵。彗、孛、客星犯之,主不安國,有喪。白雲氣入,有急事;黑,主有疾;黃,則天子有喜。



 王良五星,在奎北,居河中,天子奉車御官也。其四星曰天駟,旁一星曰王良,亦曰天馬星,動則車騎滿野。一曰為天橋,主御風雨、水道。星不具,或客星守之,津梁不通。
 與閣道近,有江河之變。星明,馬賤;暗,則馬災。太白、熒惑入守,為兵。彗、客犯之,為兵、喪,天下橋梁不通。流星犯,大兵將出。青雲氣入犯之,王良奉車憂墜車。雲氣赤,王良有斧鍎憂。



 外屏七星,在奎南,主障蔽臭穢。



 軍南門,在天大將軍南,天大將軍之南門也。主誰何出入。星不明,外國叛;動搖,則兵起;明,則遠方來貢。



 按《步天歌》,以上諸星俱屬奎宿。以《晉志》考之,王良、附
 路、閣道、軍南門、策星,俱在天市垣,別無外屏、天溷、土司空等星,《隋志》有之。而武密以王良、外屏、天溷皆屬於壁,或以外屏又屬奎。《乾象新書》以王良西一星屬壁,東四星屬奎,外屏西一星屬壁,東六星屬奎,與《步天歌》各有不合。



 婁三星,為天獄,主苑牧犧牲,供給郊祀,亦為興兵聚眾。明大,則賦斂以時。星直,則有執主命者;就聚,國不安。日食於婁,宰相、大人當之,郊祀神不享。日暈,有兵,大人多
 死。月食,其分後妃憂,民饑。月暈,在春百八十日有赦,又為糴貴,三日內雨解之。月犯,多畋獵,其分憂,將死,民流。一曰多冤獄。歲星犯之,牛多死,米賤,有赦;守之,國安,一曰:民多疫,六畜貴,有兵自罷。熒惑犯守,為旱,為火,穀貴;又曰:守二十日以上,大臣死。星動,人多死;若逆行入成勾巳者,國廩災。填星犯之,天子戒邊境,不可遠行,將兵兇;守之,穀豐,民樂;若逆行,女謁行;留舍於婁,外國兵來。太白犯之,有聚眾事;守之,期三十日有兵,民饑。辰星犯
 之,刑罰急,多水旱,大臣憂,王者以赦除之;守而芒角、動搖、色赤黑者,臣下起兵。客星犯,大兵;守之,五穀不成,又曰:臣惑主,專政,歲多獄訟;環繞三日,大赦。彗星犯之,民饑死;出,則先旱後水,穀大貴,六畜疾,倉庫空,又曰國有大兵。星孛,其分為兵,為饑。流星出犯之,有法令清獄。青赤雲氣入,為兵、喪;黑,為大水。



 按漢永元銅儀,以婁為十二度,唐開元游儀十三度。舊去極八十度。景祐測驗,婁宿十二度,距中央大星
 去極八十度,在赤道內十一度。



 天倉六星,在婁宿南,倉穀所藏也,待邦之用。星近而數,則歲熟粟聚;遠而疏,則反是。



 月犯之,主發粟。五星犯,兵起,歲饑,倉粟出。熒惑、太白合守,軍破將死。熒惑入,軍轉粟千里;近之,天下旱。太白犯之,外國人相食,兵起西北。辰星守之,大水、客、彗犯之,五穀不成。客星入,歲饑糴貴。流星入,色赤,為兵;犯之,粟以兵出;色黃白,歲大稔。蒼白雲氣入,歲饑;赤,為兵、旱,倉廩災;黃白,歲大
 熟。



 右更五星,在婁西,秦爵名,主牧師官,亦主禮義。星不具,天下道不通。太白、熒惑犯守,山澤兵起。



 左更五星,在婁東,亦秦爵名,山虞之官,主山澤林藪竹木蔬菜之屬,亦主仁智。占同右更。



 天大將軍十一星,在婁北,主武兵。中央大星,天之大將也;外小星,吏士也。動搖,則兵起,大將出;小星動搖,或不具,亦為兵;旗直揚者,隨所擊勝。五星犯守,大將憂。客星守之,大將不安,軍吏以饑敗。流星入,大將憂。蒼白雲氣
 犯之,兵多疾;赤,為兵出。



 天庾四星,在天倉東南,主露積。占與天倉同。



 按《晉志》,天倉、天庾在二十八宿之外,天大將軍屬天市垣,左更、右更惟《隋志》有之。《乾象新書》以天倉屬奎。武密亦以屬奎,又屬婁。《步天歌》皆屬婁宿。



 胃宿三星,天之廚藏,主倉廩,五穀府也。明,則天下和平,倉廩實,民安;動,則輸運。暗,則倉空;就聚,則穀貴、民流;中星眾,穀聚;星小,穀散;芒,則有兵。日食,大臣誅,一曰乏食,
 其分多疾,穀不實,又曰有委輸事。日暈,穀不熟。月食,王後有憂,將亡,亦為饑,郊祀有咎。月暈,兵先動者敗,妊婦多死,又曰國主死,大多雨,或山崩,有破軍。歲星在暈內,天子有德令。月暈在四孟之月,有赦。熒惑在暈中,為兵。月犯之,鄰國有暴兵,天下饑,外國憂,穀不實,民多疾;變色,將軍兇。歲星犯之,大人憂,兵起;守,則國昌;入,則國令變更,天下獄空;若逆行,五穀不成,國無積蓄。熒惑犯之,兵亂,倉粟出,貴人憂;守之,旱饑,民疫,客軍大敗;入,則改
 法令,牢獄空;進退環繞勾巳、凌犯及百日以上,天下倉庫並空,兵起。填星犯之,大臣為亂;守之,無蓄積,有德令,歲穀大貴;若逆行守勾巳者,有兵;色赤,兵起流血;青,則有德令。辰星犯,其分不寧;守之,有兵,國有立侯,巫咸曰:「為旱,穀不成,有急兵。」又逆行守之,倉空,水災。客星犯之,王者憂,倉廩用;退行入,則有赦;守之,強臣凌國,穀不熟;乘之,為火;舍而不去,人饑;出,其分君有憂。彗星犯之,兵動,臣叛,有水災,穀不登。星孛,其分兵起,王者惡之。流星
 犯之,倉庫空;色赤,為火災。蒼白雲氣出入犯之,以喪糴粟事;黑,為倉穀敗腐;青黑,為兵;黃白,倉實。



 按漢永元銅儀,胃宿十五度。景祐測驗,十四度。



 天囷十三星,如乙形,在胃南,倉廩之屬,主給御廩粢盛。星明,則豐稔;暗,則饑。月犯之,有移粟事。五星犯之,倉庫空虛。客、彗入,倉庫憂,水火焚溺。青白雲氣入,歲饑,民流亡。



 大陵八星,在胃北,亦曰積京,主大喪也。中星繁,諸侯喪,
 民疫,兵起。月犯之,為兵,為水、旱,天下有喪。月暈前足,大赦。五星入,為水、旱、兵、喪。熒惑守之,天下有喪。客、彗入,民疫。流星出犯之,其下有積尸。蒼白雲氣犯之,天下兵、喪;赤,則人多戰死。



 積尸一星,在大陵中。明,則有大喪,死人如山。月犯之,有叛臣。五星犯之,天下大疾。客、彗犯,有大喪。蒼色雲氣入犯之,人多死;黑,為疫。



 天船九星,在大陵北,河之中,天之船也,主通濟利涉。石
 申曰:「不在漢中,津河不通。」明,則天下安;不明及移徙,天下兵、喪。月犯之,百川流溢,津梁不通。五星犯之,水溢,民移居。彗星犯之,為大水。客星犯,為水,為兵。青雲氣入,天子憂,不可御船;赤,為兵,船用;黃白,天子喜。



 天廩四星,在昴宿南,一曰天BW,主蓄黍稷,以供享祀。《春秋》所謂御廩,此之象也。又主賞功,掌九穀之要。明,則國實歲豐;移,則國虛;黑而稀,則粟腐敗。月犯之,穀貴。五星犯之,歲饑。客星犯,倉庫空虛。流星入,色青為憂;赤,為旱,
 為火;黃白,天下熟。青雲氣入,蝗,饑,民流;赤,為旱;黑,為水;黃,則歲稔。



 積水一星,在天船中,候水災也。明動上行,舟船用。熒惑犯,有水。



 按《晉志》大陵、積尸、天船、積水俱屬天市垣,天囷、天廩在二十八宿之外。武密以天囷、大陵屬婁,又屬胃;天船屬胃,又屬昴。《乾象新書》,天囷五星屬婁,餘星屬胃,大陵西三星屬婁,東五星屬胃,與《步天歌》互有不同。


昴宿七星,天之耳目也,主西方及獄事。又為旄頭,北星也,又主喪。昴、畢間為天街,天子出,旄頭、
 \gezhu{
  □干}
 畢以前驅,此其義也。黃道所經。明,則天下牢獄平;六星皆明與大星等,為大水。七星皆黃,兵大起。一星亡,為兵、喪。搖動,有大臣下獄及有白衣之會。大而數盡動,若跳躍者,北兵大起。一星獨跳躍而動,北兵欲犯邊。日食,王者疾,宗姓自立,又占邊兵起。日暈,陰國失地,北主憂,趙地兇,又云大饑。月食,大臣誅,女主憂,為饑,邊兵起,將死,北地叛。月歲三
 暈,弓弩貴,民饑。暈在正月上旬,有赦;犯之,為饑,北主憂,天子破北兵;變色,民流,國亡,下有暴兵,有赦;出昴北,天下有福;乘之,法令峻,大水,穀不登。歲星犯之,獄空;乘之,陰國有兵,北主憂;守之,主急刑罰,獄空,一曰臣下獄有解者;守其北,有德令,又曰水物不成;久守,大臣坐法,民饑;留守,破軍殺將。熒惑犯守,為兵,為旱、饑;守東,齊、楚、越地有兵;守南,荊、楚有兵;西,則兵起秦、鄭;北,則兵起燕、趙,又為貴人多死,北地不寧;入則有喜,有赦,天下無兵;守
 而環繞勾巳,為赦;久守,糴貴。填星犯,或出入守之,北地為亂,有土功,五穀不成,水火為災,民疫,又為女主失勢;入,則地動水溢,宗廟壞;留,則大將出征。太白入犯之,大赦;在東,六畜傷;在西,六月有兵;又曰守之,北兵動,將下獄;晝見,邊兵起;出、入、留、舍,在南為男喪,北為女喪。辰星犯,北主憂,守之,穀不成,民饑;久守,為水,為兵。客星犯,貴人有急,北兵大敗,讒人在內;守之,臣叛主,兵起;入,則其分有喪。彗星犯之,大臣為亂;出,則邊兵起,有赦。星孛,其
 分臣下亂,有邊兵,大臣誅。流星出入犯之,夷兵起。《乙巳占》:「流星入,北方來朝;出,則天子有赦令恤民。」蒼赤雲氣犯之,民疫;黑,則北主憂;青,為水,為兵;青白,人多喪;黃,則有喜。



 按漢永元銅儀,昴宿十二度,唐開元游儀十一度。舊去極七十四度。景祐測驗,昴宿十一度,距西南星去極七十一度。



 芻稿六星,在天苑西,一曰在天囷南,主積稿之屬。一
 曰天積,天子之藏府。星明,則芻稿貴;星盛,則百庫之藏存;無星,則百庫之藏散。月犯之,財寶出。辰星、熒惑犯之,芻稿有焚溺之患。赤雲氣犯之,為火;黃,為喜。



 天陰五星,主從天子弋獵之臣。不明,則為吉;明,則禁言洩。



 天河一星一作天阿,在天廩星北。《晉志》在天高星西,主察山林妖變。五星、客、彗犯之,主妖言滿路。



 卷舌六星,在昴北,主樞機智謀,一曰主口舌語,以知讒
 佞。曲而靜,則賢人升;直而動,多讒人,兵起,天下有口舌之害。徙出漢外,則天下多妄說。星繁,人多死。月犯之,天下多喪。五星犯,佞人在側。彗、客犯之,侍臣憂。



 天苑十六星,在昴、畢南,如環狀,天子養禽獸之苑。明,則禽獸牛馬盈;不明,則多瘠死;不具,有斬刈事。五星犯之,兵起。客、彗犯,為兵,獸多死。流星入,色黑,禽獸多死;黃,則蕃息。《雲氣占》同。



 天讒一星,在卷舌中,主巫醫。暗,則為吉;明盛,人君納佞
 言。



 月一星,在昴宿東南,蟾蜍也,主日月之應,女主臣下之象,又主死喪之事。明大,則女主大專。太白、熒惑守之,臣下起兵為亂。彗、客犯之,大臣黜,女主憂。



 礪石四星,在五車星西,主百工磨礪鋒刃,亦主候伺。明,則兵起;常,則吉。熒惑入,邊兵起;守之,諸侯發兵。客星守之,為兵。



 按《晉志》,天河、卷舌、天讒俱屬天市垣,天苑在二十八
 宿之外,芻稿、天陰、月、礪石,《晉志》不載,《隋史》有之。武密又以芻稿屬胃,卷舌屬胃,又屬昴。《乾象新書》以芻稿屬婁,卷舌西三星屬胃,東三星屬昴,天苑西八星屬胃,南八星屬昴。《步天歌》以上諸星皆屬昴宿,互有不合。


畢宿八星,主邊兵弋獵。其大星曰天高,一曰邊將,主四夷之尉也。《天官書》曰:「畢為
 \gezhu{
  □干}
 車。」明大,則遠人來朝,天下安;失色,邊兵亂;一星亡,為兵、喪;動搖,則邊兵起;移徙,天
 下獄亂;就聚,則法令酷。日食,邊王死,軍自殺其主,遠國有謀亂。日暈,有邊兵;不則北主憂,又占有風雨。月食,有赦,趙分有兵,或趙君憂。月暈,兵亂,饑,喪;暈三重,邊有叛者,七日內風雨解之,又為陰國有憂,天下赦。犯畢大星,下犯上,大將死,陰國憂;入畢口,多雨;穿畢,歲饑,盜起;失行,離於畢,則雨;居中,女主憂;又曰犯北,則陰國憂;南,則陽國憂。歲星犯之,冬多風雨,又曰為水;入畢口,邊兵起,民饑,有赦;守三十日,客兵起;出陽,為旱;陰,為水。熒惑犯
 右角,大戰;左角,小戰;入,則邊兵憂;守之,為饑,有赦;成勾巳環繞,大赦;一曰入畢中,有兵兵罷;又曰守之,有畋獵事,北主憂,天下道路不通;入畢口,有赦;逆行至昴,為死喪;已去還守,貴臣憂;舍畢口,趙國憂。填星犯之,兵起西北,不戰;守之,兵有降軍,有赦,一曰土功徭役煩,兵起;入,則地震水溢;守畢口,大人當之;出、入、留、舍,其野兵起,客軍死。太白犯右角,戰敗,將死;入畢口,將相為亂,大赦,國易政令,諸侯起兵,為水,五穀不成;貫畢,倉廩空,四國兵
 起。辰星犯之,邊地災;入畢口,國易政;守之,水溢,民病,物不成,邊兵起;守畢口,人為亂。客星犯之,大人憂,無兵兵起,有兵兵罷;入,則多獄事;守之,為饑,邊兵起;出,為車馬急行。彗星犯之,北地為亂,人民憂。星孛,其分土功興,多徭役。色蒼,為饑,破軍;黃,則女為亂;白,為兵、喪;黑,為水。流星犯之,邊兵大戰;色赤貫之,戎兵大至;入而復出,為赦;入而黃白有光,外人入貢。蒼白雲氣入,歲不收;赤,為兵、旱,為火;黃白,天子有喜。



 按漢永元銅儀,畢十六度。舊去極七十八度。景祐測驗,畢宿十七度,距畢口北星去極七十七度。



 天節八星,在畢、附耳南,主使臣持節宣威四方。明大,則使忠;不明,則奉使無狀。熒惑守之,臣有謀逆,或使臣死。太白守之,大將出。客、彗犯之,法令不行。客星守,持節臣有憂。



 九州殊口九星,在天節南下,曉方俗之官,通重譯者也。常以十一月候之。亡一星,一國憂;二星以上,天下亂,兵
 起。太白、熒惑守之,亦為兵。客星入,民憂,水負海,國不安,有兵。



 附耳一星,在畢下,主聽得失,伺愆邪,察不祥也。星盛,則中國微,有盜賊,邊候警,外國反。動搖,則讒臣在君側。歲星犯之,有兵,將相喪。太白犯之,佞臣在側。



 九斿九星,在玉井西南,一曰在九州殊口東,南北列,主天下兵旗,又曰天子之旗也。太白、熒惑犯之,兵騎滿野。客星犯,諸侯兵起,禽獸多疾。



 天街二星,在昴、畢間,一曰在畢宿北,為陰陽之所分。《大象占》:近月星西,街南為華夏,街北為外邦。又曰三光之道,主伺候關梁中外之境。明,則王道正。月犯天街中,為中平,天下安寧;街外,為漏洩,讒夫當事,民不得志;不由天街,主政令不行。月暈其宿,關梁不通。熒惑守之,道路絕;久守,國絕禮。歲星居之,色赤,為殃,或大旱。太白守之,兵塞道路,六夷旄頭滅,一曰民饑。



 天高四星,在坐旗西,《乾象新書》:在畢口東北。臺榭之高,
 主望八方雲霧氛氣,今仰觀臺也。不見,為主失禮;守常,則吉;微暗,陰陽不和。月、五星犯之,則水旱不時;乘之,外臣誅。月暈,不出六月有喪。熒惑入十日,為小赦;留三十日,大赦。客、彗守之,大旱。蒼白雲氣犯,亦然。



 諸王六星,在五車南,主察諸侯存亡。明,則下附上;不明,則下叛;不見,宗廟危,四方兵起。熒惑入之,諸王妃恣,為下所謀;守之,下不信上。太白、熒惑犯,諸王當之,一曰宗臣憂。客、彗守,諸侯黜。



 五車五星、三柱九星,在畢宿北,五帝坐也,又五帝之車舍也。主天子五兵,又主五穀豐耗。一車主蕡麻,一車主麥,一車主豆,一車主黍,一車主稻米。西北大星曰天庫,主太白,秦分及雍州,主豆。東北一星曰天獄,主辰星,燕、趙分及幽、冀,主稻。東南一星曰天倉,主歲星,魯分徐州,衛分並州,主麻。次東南一星曰司空,主填星,楚分荊州,主黍粟。次西南一星曰卿,主熒惑,魏分益州,主麥。《天文錄》曰:「太白,其神令尉,辰星,其神風伯;歲星,其神雨師;熒
 惑,其神豐隆;填星,其神雷公。此五車有變,各以所主占之。」三柱,一曰天淵,一曰天休,一曰天旗,欲其均明闊狹有常,星繁,則兵大起。石申曰:「天庫星中河而見,天下多死人,河津絕。」又曰:「天子得靈臺之禮,則五車、三柱均明有常。」天旗星不見,則大風折木;天休動,則四國叛。一柱出,或不見,兵半出;三柱盡出,及不見,兵亦盡出。柱外出一月,穀貴三倍;出二月、三月,以次倍貴;外出不盡兩間,主大水。月犯天庫,兵起,道不通;犯天淵,貴人死,臣逾主。
 月暈,女主惡之;在正月,為赦;暈一車,赦小罪;五車俱暈,赦殊罪;四、七、十月暈之,為水;暈十一、十二月,穀貴。五星犯,為旱,喪;犯庫星,為兵起。歲星入之,糴貴。熒惑入之,為火,或與歲星占同。填星入天庫,為兵,為喪;舍中央,為大旱,燕、代之地當之;舍東北,畜蕃,帛賤;舍西北,天下安。太白入之,兵大起;守五車,中國兵所向懾伏;舍西北,為疾疫,牛馬死,應酒泉分。辰星入舍為水;犯之,兵以水潦起。客星犯,則人勞;庚寅日候近之,為金車,主兵;甲寅日
 候近之,為木車,主槥增價;戊寅日候近之,為土車,主土功;丙寅日候近之,為火車,主旱;壬寅日候近之,為水車,主水溢;入之,色青為憂,赤為兵;守天淵,有大水;守天休,左為兵,右為喪;黃為吉。彗、孛犯之,兵起,民流。流星入,甲子日,主粟;丙午日,主麥;戊寅日,主豆;庚申日,主蕡,壬戌日,主黍:各以其日占之,而粟麥等價增。白雲氣入,民不安;赤,為兵起。



 天潢五星,在五車中。主河梁津渡。星不見,則津渡不通。
 月入天潢,兵起。五星失度,留守之,皆為兵。熒惑、填星入之,為大旱,為火。熒惑舍之,牛馬疫,為兵。辰星出天潢,有赦。客星入,為兵;留守,則有水害。蒼白或黑雲氣入,為喪;赤,為兵;黃白,則天子有喜。



 咸池三星,在天潢南,主陂澤池沼魚鱉鳧雁。明大,則龍見,虎狼為害;星不具,河道不通。



 月入,為暴兵。五星入,為兵,為旱,失忠臣,君易政;守之,為饑,為兵。客星入,天下大水。流星入,為喪;出,則兵起。雲氣入,色蒼白,魚多死;赤,為
 旱;白,為神魚見;黑,為大水。



 參旗九星,一曰天旗,一曰天弓,司弓弩,候變禦難。星如弓張,則兵起;明,則邊寇動;暗,為吉。又曰天弓不具,天下有兵。五星犯之,兵起。熒惑守之,下謀上,諸侯起兵;一曰有邊兵。太白守之,兵亂。客星守,天下憂。流星入,北地兵起。雲氣犯之,色青,入自西北,兵來,期三年。



 天關一星,在五車南,亦曰天門,日月之所行,主邊方,主關閉。星芒角,為兵;不與五車合,大將出。月歲三暈,有
 赦;犯之,有亂臣更法。五星守之,貴人多死。歲星、熒惑守之,臣謀主,為水,為饑。太白、熒惑守之,大赦,關梁有兵。太白入,則大亂。填星守,王者壅蔽;犯之,臣謀主。太白失行,兵起。客星犯之,民多疾,關市不通;又曰諸侯不通,民相攻。客星入,多盜。流星犯之,天下有急,關梁不通,民憂,多盜。黃雲氣犯,四方入貢。



 天園十三星,在天苑南,植菜果之處。曲而鉤,菜果熟。白雲氣犯之,兵起。



 按《步天歌》,以上諸星皆屬畢宿。武密書以天節屬昴,參旗、天關、五車、三柱皆屬觜,與《步天歌》不同。《乾象新書》以天節、參旗皆屬畢;天園西八星屬昴,東五星亦屬畢;五車北西南三大星屬畢,東二星及三柱屬參。說皆不同,今皆存之。



 觜觿三星,為三軍之候,行軍之藏府,葆旅收,斂萬物。明,則軍糧足,將得勢;動,則盜賊行,葆旅起;暗,則不可用兵。日食,臣犯主,戒在將臣。暈及三重,其下穀不登,民疫;
 五重,大赦,期六十日。月食,為旱,大將憂,有叛主者。正月月暈,有赦,外軍不勝,大將憂,偏裨有死者。歲星犯之,其分兵起;守,則農夫失業,後有憂,丁壯多暴死,下有叛者,民多疾疫;入,則多盜,天時不和;國君誅伐不當,則逆行。熒惑犯之,其分有叛者,為旱,為火,為兵起,為糴貴;與觜觿合,趙分相憂;入,則其下有兵。填星入犯,為兵,為土功,其分失地;女主恣,則填星逆行而色黃。太白犯之,兵起;守之,其分易令,大臣叛,物不成,民疫。辰星犯之,不可舉
 兵;一曰趙地水,有叛者;守之,趙分饑。客星出入其宿,青為憂,赤為兵,黑為水,白為喪,黃白為吉。彗星犯之,兵起;出入其分,失地,民流。星孛之,為兵亂,軍破,其色與客星同占。流星入犯之,有叛者,有破軍。雲氣犯之,赤,為兵;蒼白,為兵、憂;黑,趙地大人有憂;色黃,有神寶入。



 按漢永元銅儀、唐開元游儀,皆以觜觿為三度。舊去極八十四度。景祐測驗,觜宿三星一度,距西南星去極八十四度,在赤道內七度。



 坐旗九星,在司怪西北,君臣設位之表也。星明,則國有禮。



 司怪四星,在井鉞星前,主候天地、日月、星辰變異,鳥獸、草木之妖,明主聞災,修德保福。星不成行列,宮中及天下多怪。



 按《步天歌》,坐旗、司怪俱屬觜宿,武密書及《乾象新書》皆屬於參。



 參宿十星,一曰參伐,一曰天市,一曰大辰,一曰鈇鉞,主
 斬刈萬物,以助陰氣;又為天獄,主殺,秉威行罰也;又主權衡,所以平理也;又主邊城,為九譯,故不欲其動。參為白虎之體,其中三星橫列者,三將也;東北曰左肩,主左將;西北曰右肩,主右將;東南曰左足,主後將軍;西南曰右足,主偏將軍。參應七將,中央三小星曰伐,天之都尉,主鮮卑外國,不欲其明。七將皆明大,天下兵精;王道缺,則芒角張;伐星明與參等,大臣有謀,兵起;失色,軍散敗;芒角動,邊有急,兵起,有斬伐之事;星移,客伐主;肩細微,
 天下兵弱;左足入玉井中,兵起,秦有大水,有喪,山石為怪;星差戾,王臣貳;左股星亡,東南不可舉兵;右股,則主西北。又曰參足移北為進,將出有功;徙南為退,將軍失勢。三星疏,法令急。日食,大臣憂,臣下相殘,陰國強。日暈,有來和親者,一曰大饑。月食其度,為兵,臣下有謀,貴臣誅,其分大饑,外兵大將死,天下更令。月暈,將死,人殃亂,戰不利。月犯,貴臣憂,兵起,民饑;犯參伐,偏將死。歲星犯之,水旱不時,大疫,為饑;守之,兵起,民疫;入,則天下更政。
 熒惑犯之,為兵,為內亂,秦、燕地兇;守之,為旱,為兵,四方不寧;逆行入,則大饑。填星犯之,有叛臣;守之,其下國亡,奸臣謀逆,一云有喪,後、夫人當之;逆行留守,兵起。太白犯之,天下發兵;守之,大人為亂,國易政,邊民大戰。辰星犯之,為水,為兵,貴臣黜。辰星與參出西方,為旱,大臣誅。逆守之,兵起。客星入犯之,國內有斬刈事;守之,邊州失地;環繞者,邊將有斬刈事。彗星犯之,邊兵敗,君亡,遠期三年;貫之,色白,為兵、喪。星孛於參,君臣俱憂,國兵敗。流
 星入犯之,先起兵者亡。《乙巳占》曰:「流星出而光潤,邊安,有赦,獄空。」青雲氣入犯之,天子起邊城;蒼白,為臣亂;赤,為內兵;黃色潤澤,大將受賜;黑,為水災,大臣憂。白雲氣出貫之,將死,天子疾。



 按漢永元銅儀,參八度。舊去極九十四度。景祐測驗,參宿十星十度,右足入畢十三度。



 玉井四星,在參左足下,主水泉,以給庖廚。動搖,為憂。客星入,為水,為喪國失地;出,則國得地,一云將出。流星入,
 為大水。雲氣入而色青,井水不可食。



 屏二星,一作天屏,在玉井南,一云在參右足。星不具,人多疾。不明,大人寢疾。星亡,主多病。月、五星犯之,為水。客星出於屏,亦為大人有疾。彗星犯之,水旱不時。



 軍井四星,在玉井東南,軍營之井,主給師,濟疲乏。月犯,芻稿財寶出。熒惑入,為水,兵多死。太白入,兵動,民不安。客星入,憂水害。



 廁四星,在屏星東,一曰在參右腳南,主溷。色黃,為吉,歲
 豐;青黑,人主腰下有疾。星不具,則貴人多病。客星入,為穀貴。彗、孛入,歲饑。青雲氣入,為兵;黑,為憂;黃,則天子有喜。



 天屎一星,在天廁南。色黃,則年豐。凡變色,為蝗,為水旱,為霜殺物。常以秋分候之。星亡不見,天下荒;星微,民多流。



 按《步天歌》,玉井、軍井、廁各四星,屏二星,天屎一星,俱屬參宿。《晉志》玉井在參左足,武密書屬觜,《乾象新書》
 屬畢;軍井,《晉志》在玉井南,武密亦屬觜,《乾象新書》亦屬畢,唐開元游儀在玉井東南;屏、廁、天屎,《晉志》皆不載,《隋志》屏在玉井南,開元游儀在觜,《隋志》廁在屏東,屎在廁南,《乾象新書》皆屬參,與《步天歌》互有不合。



 南方



 東井八星,天之南門,黃道所經,七曜常行其中,為天之亭候,主水衡事,法令所取平也。武密占曰:井中為三光正道,五緯留守若經之,皆為天下無道。不欲明,明則大水。又占曰:用法平,井宿明。鉞一星,附井宿前,主伺
 奢淫而斬之;明大與井宿齊,則用鉞於大臣。



 月宿,其分有風雨。日食,秦地旱,民流,有不臣者;暈,則多風雨;有青赤氣在日,為冠,天子立侯王。月食,有內亂,大臣黜,後不安,五穀不登,分有兵、喪。月暈,為旱,為兵,為民流,國有憂,一曰有赦。陰陽不和則暈,暈及三重,在三月為大水,在十二月日壬癸為大赦。月犯之,將死於兵,水官黜,刑不平;犯井鉞,大臣誅,有水事。歲星犯之,主急法,多獄訟,水溢,將軍惡之;犯井鉞,近臣為亂,兵起;逆行入井,川流壅
 塞。熒惑犯之,兵先起者殃,又曰天子以水敗;入守經旬,下有兵,貴人不安;守三十日,成勾巳,角動,色赤黑,貴人當之,百川溢,兵起。填星入犯之,兵起東北,大臣憂;入井鉞,王者惡之;在觜而去東井,其下亡地。太白犯之,咎在將;久守,其分君失政,臣為亂。辰星犯之,星進則兵進,退則兵退,刑法平,又曰北兵起,歲惡。芒角、動搖,色赤黑,為水,為兵起。客星犯之,穀不登,大臣誅,有土功,小兒妖言。彗星犯之,民讒言,國失政,一曰大臣誅,其分兵災。流星
 犯之,在春夏則秦地謀叛,在秋冬則宮中有憂。《乙巳占》:流星色黃潤,國安;赤黑,秦分民流,水災。蒼黑雲氣入犯之,民有疾疫;黃白潤澤,有客來言水澤事。黑氣入,為大水。常以正月朔日入時候之,井宿上有云,歲多水潦。



 按漢永元銅儀,井宿三十度,唐開元游儀三十三度,去極七十度。景祐測驗,亦三十三度,距西北星去極六十九度。



 五諸侯五星,在東井北,主斷疑、刺舉、戒不虞、理陰陽、察
 得失,亦曰主帝心。一曰帝師,二曰帝友,三曰三公,四曰博士,五曰太史,五者常為帝定疑議。星明大、潤澤,則天下治。五禮備,則光明,不相侵陵;暗,則貴人謀上;芒角,禍在中。歲星犯之,兵起三年。熒惑犯之,大臣叛不成。太白犯之,諸侯興兵亡國;經天晝見,則諸侯受誅。客星犯,王室亂,諸侯亡地,秦國殃;守之,諸侯親屬失位。彗、孛犯之,執法臣誅,又曰貴臣當之,期一年。雲氣犯之,色蒼白,諸侯有喪;不,則臣有誅戮。天下表大水。



 積水一星,在北河西北,所以供酒食之正也。不見,為災。歲星犯之,水物不成,魚鹽貴,民饑。熒惑犯之,為兵,為水。辰星犯之,為水、旱。客星犯之,兵起,大水,大臣憂,期一年。蒼白雲氣入犯之,天下有水。



 積薪一星,在積水東北,供庖廚之正也。星不明,五穀不登。熒惑犯之,為旱,為兵,為火災。客星守之,薪貴。赤雲氣入犯之,為水災。



 南河三星,與北河夾東井,一曰天之關門也,主關梁。南
 河曰南戍,一曰南宮,一曰陽門,一曰越門,一曰權星,主火。兩河戍間,日、月、五星之常道也。河戍動搖,中國兵起。河星不具,則路不通。水泛溢。月出入兩河間中道,民安,歲美,無兵;出中道之南,君惡之,大臣不附。星明,為吉;昏昧動搖,則邊兵起,遠人叛,主憂。月犯之,為中邦憂,一曰為兵,為喪,為旱,為疫;行西南,為兵、旱;入南戍,則民疫;暈,則為土功;乘之,四方兵起;經南戍南,則為刑罰失。歲星犯之,北主憂。熒惑犯兩河,為兵;守三十日以上,川溢;守
 南河,穀不登,女主憂;守南戍西,果不成;在東,則有攻戰。填星乘南河,為旱,民憂;守之,為兵,道不通。太白舍三十日,川溢;一曰有奸謀;守兩河,為兵起。客星守之,為旱,為疫。彗、孛出,為兵;守,為旱。流星出,為兵、喪,邊戍有憂。蒼白雲氣入之,河道不通;出而色赤,天子兵向諸侯。黃氣入之,有德令;出,為災。



 北河亦三星,北河曰北戍,一曰北宮,一曰陰門,一曰胡門,一曰衡星,主水。五星出、入、留、守之,為兵起;犯之,為女
 喪;乘之,為北主憂。歲星入北戍,大臣誅。熒惑從西入北戍,六十日有喪;從東入,九十日有兵;一曰出北戍北守之,邊將有不請於上而用兵外國者勝。填星守之,兵起,六十日內有赦,一曰有土功;若守戍西,五穀不實。太白舍北戍,三十日為女喪,有內謀;守陰門,不出百日天下兵悉起。辰星守之,外兵起,邊臣有謀;留止,則兵起四方。客星入犯之,有喪於外,奸人在中;入自東,兵起,期九十日;入自西,有喪,期六十日;守之,為大水。流星經兩河間,
 天下有難;入,為北兵入中國,關梁不通。雲氣蒼白入犯之,邊有兵,疾疫,又為北主憂。



 四瀆四星,在東井南垣之東,江、河、淮、濟之精也。明大,則百川決。



 水位四星,在積薪東,一曰在東井東北,主水衡。歲星犯之,為大水;一曰出南,為旱。熒惑守之,田不治。客星犯之,水道不通,伏兵在水中;一曰客星若水、火,守犯之,百川流溢。彗、孛出,為大水,為兵,穀不成。流星入之,天下有水,
 穀敗民饑。赤雲氣入,為旱、饑。



 天樽三星,在五諸侯南,一曰在東井北,樽,器也,主盛饘粥,以給貧餒。明,為豐;暗,則歲惡。



 闕丘二星,在南河南,天子雙闕,諸侯兩觀也。太白、熒惑守之,兵戰闕下。



 軍市十三星,狀如天錢,天軍貿易之市,有無相通也。中星眾,則軍餘糧;小,則軍饑。月入,為兵起,主不安。五星守之,軍糧絕。客星入,則有刺客起,將離卒亡。流星出,為大將
 出。



 野雞一星,在軍市中,主變怪。出市外,天下有兵。守靜,為吉;芒角,為兇。



 狼一星,在東井東南,為野將,主侵掠。色有常,不欲動也。芒角、動搖,則兵起;明盛,兵器貴;移位,人相食;色黃白,為兇;赤,為兵。月犯之,有兵不戰,一曰有水事。月食在狼,外國有謀。五星犯之,兵大起,多盜。彗、孛犯之,盜起。客星守之,色黃潤,為喜;黑,則有憂。赤雲氣入,有兵。



 弧矢九星,在狼星東南,天弓也,主行陰謀以備盜,常屬矢以向狼。武密曰:「天弓張,則北兵起。」又曰:「天下盡兵。」動搖明大,則多盜;矢不直狼,為多盜;引滿,則天下盡為盜。月入弧矢,臣逾主。月暈其宿,兵大起。客星入,南夷來降;若舍,其分秋雨雪,穀不成;守之,外夷饑;出入之,為兵出入。流星入,北兵起,屠城殺將。赤雲氣入之,民驚,一曰北兵入中國。



 老人一星,在孤矢南,一名南極。常以秋分之旦見於丙,
 候之南郊,春分之夕沒於丁。見,則治平,天子壽昌;不見,則兵起,歲荒,君憂。客星入,為民疫,一曰兵起,老者憂。流星犯之,老人多疾,一曰兵起。白雲氣入之,國當絕。



 丈人二星,在軍市西南,主壽考,悼耄矜寡,以哀窮人。星亡,人臣不得自通。



 子二星,在丈人東,主侍丈人則。不見,為災。



 孫二星,在子星東,以天孫侍丈人側,相扶而居以孝慈。不見,為災;居常,為無咎。



 水府四星,在東井西南,水官也,主堤塘、道路、梁溝,以設堤防之備。熒惑入之,有謀臣。辰星入,為水。客星入,天下大水。流星入,色青,所主之邑大水;赤,為旱。



 按《步天歌》自五諸侯至水府常星一十八坐,俱屬東井。武密書以丈人二星、子、孫各一星屬牛宿。《乾象新書》以丈人與子屬參、孫屬井;又以水府四星亦屬參。武密以水府屬井。餘皆與《步天歌》合。



 輿鬼五星,主觀察奸謀,天目也。東北星主積馬,東南星
 主積兵,西南星主積布帛,西北星主積金玉,隨變占之。中央星為積尸,主死喪祠祀;一曰鈇鍎,主誅斬。星明大,穀不成;不明,民散。鍎欲其忽忽不明,明則兵起,大臣誅;動而光,賦重役煩,民懷嗟怨。日食,國不安,有大喪,貴人憂。暈,則其分有兵,大臣有誅廢者。月食,貴臣、皇后憂,期一年。暈,為旱,為赦。月犯之,秦分君憂,一曰軍將死,貴臣、女主憂,民疫。歲星犯之,穀傷民饑,君不聽事;犯鬼鍎,執法臣誅。熒惑犯之,忠臣誅,一曰兵起,後失勢;入,則後及
 相憂,一曰賊在君側,有兵、喪;勾巳,國有赦;留守十日,諸侯當之;二十日,太子當之;勾巳環繞,天子失廟。填星犯之,大臣、女主憂;守之,憂在後宮,為旱,為土功;入鍎,王者惡之;犯積尸,在陽為君,在陰為後,在為太子,右為貴臣,隨所守惡之。太白入犯之,為兵,亂臣在內,一曰將有誅;貫之而怒,下有叛臣;久守之,下有兵,為旱,為火,萬物不成。辰星犯之,五穀不登;守,為有喪,憂在貴人。客星犯之,國有自立者敗,一曰多土功;入之,有詛盟祠鬼事。彗星
 犯之,兵起,國不安。星孛,其下有喪,兵起,宜修德禳之。流星犯鬼鍎,有戮死者;入,則四國來貢。白雲氣入,有疾疫;黑,後有憂;赤,為旱;黃,為土功;入犯積尸,貴臣有憂;青,為病。



 按漢永元銅儀,輿鬼四度。舊去極六十八度。景祐測驗,輿鬼三度,距西南星去極六十八度。



 爟四星,在鬼宿西北,一曰在軒轅西,主烽火,備邊亭之警急。以不明為安,明大則邊有警。赤雲氣入,天下烽火
 皆動。



 天狗七星,在狼星北,主守財。動移,為兵,為饑,多寇盜,有亂兵。填星守之,人相食。客、彗守之,則群盜起。



 外廚六星,為天子之外廚,主烹宰,以供宗廟。占與天廚同。



 積尸氣一星,在鬼宿中,孛孛然入鬼一度半,去極六十九度,在赤道內二十二度,主死喪祠祀。



 天紀一星,在外廚南,主禽獸之齒。太白、熒惑守犯之,禽
 獸死,民不安。客星守之,則政隳。



 天社六星,在弧矢南。昔共工氏之勾龍能平水土,故祀之以配社,其精上為星。明,則社稷安;不明、動搖,則下謀上。太白、熒惑犯之,社稷不安。客星入,有祀事於國內;出,則有祀事於國外。



 按《晉志》,爟四星屬天市垣,天狗七星在七星北。武密以天狗屬牛宿,又屬輿鬼,《乾象新書》屬井。外廚六星,《晉志》在柳宿南,武密書亦屬柳,《乾象新書》與《步天歌》
 皆屬輿鬼。天紀一星,武密書及《乾象新書》皆屬柳,惟《步天歌》屬鬼宿。天社六星,武密書屬井,又屬鬼。《乾象新書》以西一星屬井,中一星屬鬼,末一星屬柳。今從《步天歌》,以諸星俱屬輿鬼,而備存眾說。



 柳宿八星,天之廚宰也,主尚食,和滋味,又主雷雨。《爾雅》曰:「咮謂之柳。柳,鶉火也。」又主木功。一曰天庫,又為鳥喙,主草木。明,則大臣謹重,國家廚餐具;開張,則人饑死;亡,則都邑振動;直,則為兵。日食,宮室不安,王者惡之,廚官、
 橋道、堤防有憂。日暈,飛鳥多死,五穀不成;三抱而戴者,君有喜。月食,宮室不安,大臣憂。月暈,林苑有兵,天下有土功,廚獄官憂,又為兵,為饑,為旱、疫。歲星犯之,國多義兵。熒惑犯之,色赤而芒角,其下君死,一曰宮中憂火災;守之,有兵,逆臣在側;逆行守之,王不寧。填星犯守,君臣和,天下喜;石申曰:「天子戒飲食之官。」出、入、留、舍,有急令。太白犯之,有急兵。逆行勾巳,臣謀主;晝見,為兵。辰星犯之,民相仇,歲旱,君戒在酒食。客星犯之,咎在周國;守,
 則布帛、魚鹽貴。色蒼白,殺邊地諸侯。彗星犯之,大臣誅,為兵,為喪。星孛於柳,南夷叛,甘德曰:「為兵,為喪。」流星出犯之,周分憂;色黃,為喜;入,則王者內有火災;《乙巳占》:「出,則宗廟有喜,賢人用;入,為天廚官有憂,木功廢。」赤雲氣入,為火;黃,為赦;黃白,為天子有喜,起宮室。



 按漢永元銅儀,以柳為十四度,唐開元游儀十五度。舊去極七十七度。景祐測驗,柳八星一十五度,距西頭第三星去極八十三度。



 酒旗三星,在軒轅右角南,酒官之旗也,主宴享飲食。星不具,則天下有大喪,帝王宴飲,沉昏非禮,以酒亡國;明,則宴樂謹。五星守之,天下大酺,有酒肉賜宗室。熒惑犯之,飲食失度。太白犯之,三公九卿有謀。客、彗犯,主以酒過為相所害。赤雲氣入,君以酒失。



 按《晉志》,酒旗在天市垣。《步天歌》以酒旗屬柳宿。以《通占鏡》考之,亦屬柳,又屬七星。《乾象新書》亦屬七星,與《步天歌》不同,今並存之。



 七星七星,一名天都,主衣裳文繡,又主急兵。故星明,王道昌;暗,則賢良去,天下空;動,則兵起;離,則易政。蓋天曰:七星為朱雀頸。頸者,文明之粹,羽儀所承。日食其宿,主不安,刑在門戶之神。又曰:文章士受誅,其分兵起,臣為亂。日暈,周邦君憂;青色抱而順,在兵為東軍吉。月食,後及大臣有憂,又為歲饑,民流,其國更政。暈,其地旱,獄官兇。歲星犯之,主憂兵,五穀多傷。熒惑犯之,橋梁不通;逆行,則地動為火災;出、入、留、舍,其國失地,水決。填星犯守,
 世治平,王道興,後、夫人喜。太白犯之,兵暴起,大臣為亂;經天,防詐偽。辰星犯之,賊臣在側;守,則其分有憂,萬物不成,兵從中起,貴臣有罪,民疫流亡。客星犯之,為兵,《荊州占》云:「河水決,民流。」彗犯,有亂兵起,貴臣戮;武密曰:「彗星出七星,狀如杵,為兵。」星孛於星,有亂兵起宮殿,貴臣戮,大臣相譖。流星犯之,為兵、憂;又曰:入,則有急使來。《乙巳占》:「流星入,庫官有喜,錦繡進,女工用。」蒼白雲氣入,貴人憂;出,則天子用急使。赤入,為兵;黑,為賢士死;黃,則遠
 人來貢;白,為天子遣使賜諸侯帛。



 按景祐測驗,七星七度,距大星去極九十七度。



 軒轅十七星,在七星北,後妃之主,士職也。一曰東陵,一曰權星,主雷雨之神。南大星,女主也;次北一星,夫人也,屏也,上將也;次北一星,妃也,次將也;其次諸星,皆次妃之屬也。女主南小星,女御也;左一星少民,後宗也;右一星太民,太后宗也。欲其色黃小而明。武密曰:「后妃後宮之象,陰陽交合,感為雷,激為電,和為雨,怒為風,亂為霧,
 凝為霜,散為露,聚為雲氣,立為虹蜺,離為背璚,分為抱珥,此二十四變皆權主之。」微細,則皇后不安;黑,則憂在大人;移徙,則民流;東西角大張而振,後族敗。月入之,女主失勢,或火災;犯左右角,大臣以罪免;中犯乘守太民,為饑,太后宗有罪;守少民,小有饑,女主失勢;守禦女,有憂。月暈,女主有喪。月、五星凌犯、環繞、乘守,皆為女主有禍。月食,女主憂。歲星犯之,女主失勢,一曰大臣當之;乘守大民,為大饑,太后宗黜;中犯乘守少民,為小饑,後宮
 有黜者。熒惑犯守勾巳,後妃離德;犯御女,天子僕妾憂;犯太民、少民,憂在後宗;守之,宮中有戮者。填星行其中,女主失勢,有喪。太白犯之,皇后失勢。客星犯之,近臣謀滅宗族。彗、孛犯,女主為寇,一曰兵起。流星入之,後宮多讒亂《乙巳占》:「流星出之,後有中使出。」一曰天子有子孫喜。



 天稷五星,在七星南,農正也,取百穀之長以為號。明,則歲豐;暗,或不具為饑;移徙,天下荒歉。客星入之,有祠事
 於內;出,有祠事於國外。



 天相三星,在七星北,一曰在酒旗南,丞相大臣之象。武密曰:「占與相星同。」五星犯守之,後妃、將相憂。彗、客犯之,大臣誅。雲氣入,黃,為大臣喜;黑,為將憂。



 內平四星,在三臺南,一曰在中臺南,執法平罪之官。明,則刑罰平。



 按軒轅十七星,《晉志》在七星北,而列於天市垣;武密以軒轅屬七星,又屬柳;《乾象新書》以西八星屬柳,中
 屬七星,末屬張。天稷五星,《晉志》在七星南;武密亦以天稷屬七星,又屬柳;《乾象新書》以西二星屬柳,餘屬七星。天相三星,《晉志》在天市垣,武密書屬七星,《乾象新書》屬軫宿。內平四星,《晉志》在天市垣,武密書屬柳,《乾象新書》屬張,《步天歌》屬七星。諸說皆不同,今並存之。



 張宿六星,主珍寶、宗廟所用及衣服,又主天廚飲食、賞賚之事。明,則王行五禮,得天下之中;動,則賞賚不明,王
 者子孫多疾;移徙,則天下有逆;就聚,則有兵。日食,為王者失禮,掌御饌者憂。甘德曰:「後失勢,貴臣憂,期七十日。」暈及有黃氣抱日,主功臣效忠。又曰:「財寶大臣黜,將相憂。」月食,其分饑,臣失勢,皇后有憂。暈,為水災。陳卓曰:「五穀、魚鹽貴。」巫咸曰:「后妃惡之,宮中疫。」月犯之,將相死,其國憂。歲星入犯之,天子有慶賀事;守之,國大豐,君臣同心;三十日不出,天下安寧,其國升平。熒惑犯之,功臣當封;入,則為兵起;又曰色如四時休王,其分貴人安,社稷
 無虞;又曰熒惑春守,諸侯叛;逆行守之,為地動,為火災,又曰將軍驚,土功作,又曰會則不可用兵。填星犯之,為女主飲宴過度,或宮女失禮;入,為兵;出,則其分失地;守之,有土功。太白犯之,國憂;守之,其國兵謀不成,石申曰:「國易政。」舍留,其國兵起。辰星犯守,五穀不成,兵起,大水,貴臣負國,民疫,多訟,芒角,臣傷其君;入,為火災;出;則有叛臣。客星犯之,天子以酒為憂;守之,周、楚之國有隱士出;入於張,兵起,國饑;舍留不去,前將軍有謀。又曰利先
 起兵。彗星犯之,國用兵,民亡;守,為兵;出,為旱;又曰犯守,君欲移徙宮殿。星孛於張,為民流,為兵大起。《乙巳占》:「流星出入,宗社昌,有赦令,下臣入賀。」蒼白雲氣入之,庭中觴客有憂;黃白,天子因喜賜客;黑,為其分水災;色赤,天子將用兵。



 按漢永元銅儀,張宿十七度,唐開元游儀十八度。舊去極九十七度。景祐測驗,張十八度,距西第二星去極一百三度。



 天廟十四星,在張宿南,天子祖廟也。明,則吉;微細,其所有兵,軍食不通。客星中犯之,有白衣會,兵起。又曰祠官有憂。武密曰:「與虛梁同占。」



 按天廟十四星,《晉志》雖列於二十八宿之外,而亦曰在張宿南,與《隋志》所載同,兼與《步天歌》合。



 翼宿二十二星,天之樂府,主俳倡戲樂,又主外夷遠客、負海之賓。星明大,禮樂興,四國賓;動搖,則蠻夷使來;離徙,天子將舉兵。日食,王者失禮,忠臣見譖,為旱災。暈,為
 樂官黜;上有抱氣三,敵心欲和。月食,亦為忠臣見譖,飛蟲多死,北方有兵,女主惡之,石申曰:「大臣有謀。」月犯之,國憂,其分有兵,大將亡,女主惡之。歲星犯,五穀為風所傷;守之,王道具,將相忠,文術用;逆行入之,君好畋獵。熒惑犯之,其分民饑,臣下不從命,邊兵起;出、入、留、舍,為兵;守之,佞臣為亂。填星犯之,大臣憂;守之,主聖臣賢,歲豐,後有喜;出、入、留、舍,兵起;逆行,則女主失政。太白入或犯之,皆為兵起;出、入、留、舍,大風水災,其分君不安;舍左,為
 旱;守犯、勾巳、凌突,則大臣專君令。辰星凌抵,下臣為亂伏誅;守之,旱,饑,民流,龍蛇見;守其中,兵大起;同見西方,大臣憂。客星入犯之,國有兵,大臣憂,一曰負海國有使來;守之,為兵起。彗星犯之,大臣憂,國有兵、喪。星孛于翼,亦為大臣憂,其分失禮樂;出,則其地有謀,下有兵、喪;芒所指,有降人。流星犯之,亦為憂在大臣;出,則其下有兵;入,為貴臣囚系,《乙巳占》曰:「流星入,天下賢士入見,南夷來貢,國有賢臣。」赤雲氣出入,有暴兵;黃而潤澤,諸侯來
 貢;黑,為國憂。



 按漢永元銅儀,翼宿十九度,唐開元游儀十八度。舊去極九十七度。景祐測驗,翼宿一十八度,距中行西第二星去極百四度。



 東甌五星,在翼南,蠻夷星也。《天文錄》曰:「東甌,東越也,今永嘉郡永寧縣是也。」芒角、動搖,則蠻夷叛。太白、熒惑守之,其地有兵。



 按東甌五星,《晉志》在二十八宿之外,《乾象新書》屬張
 宿;武密書屬翼宿,與《步天歌》合。



 軫宿四星,主塚宰、輔臣,主車騎,主載任。有軍出入,皆占於軫。又主風,占死喪。明大,則車駕備;移徙,天子有憂;就聚,則兵起。轄二星,傅軫兩旁,主王侯,左轄為王者同姓,右轄為異姓。星明,兵大起;遠軫,兇;轄舉,南蠻侵;車無轄,國有憂。日食,憂在將相,戒車駕之官,一曰後不安。暈而生背氣,其下兵起,城拔,視背所向擊之勝,又曰王者惡之。月食,後及大臣憂。月暈,有兵,歲旱,多大風。歲星犯之,
 為火災,為民疫,大臣憂,主庫者有罪;入,則其國將死;守之,國有喪;七日不移,有赦,又曰君有憂。熒惑犯之,有亂兵;入軫,將軍為亂,水傷稼,民多妖言;逆行,為火,為兵。填星犯之,為兵,為土功;入,則兵敗;逆行,女主憂;出、入、舍、留,六十日兵起,大旱。太白犯之,為兵起,得地;入,為兵;守之,亡地,將憂;起左角,逆行至軫,失地;經天,則兵滿野。辰星犯之,民疫,大臣憂,中國有貴喪;守之,大水;入,則天下以火為憂,一曰國有喪。客星犯之,為兵,為喪;入,則有土功,
 糴貴,諸侯使來;出,則君使諸侯;守之,邊兵起,民饑;守轄,軍吏憂。彗星犯之,為兵,為喪;色赤,為君失道,又曰天子起兵,王公廢黜。星孛於軫,亦為兵、喪,又曰下謀上,主憂。流星犯之,有兵起,亦有喪,不出一年,庫藏空;春夏犯之,為皮革用;秋冬,為水旱不調。



 按漢永元銅儀,以軫宿為十八度。舊去極九十八度。景祐測驗,亦十八度,去極一百度。



 長沙一星,在軫宿中,入軫二度,去極百五度,主壽命。明,
 則君壽長,子孫昌。



 青丘七星,在軫東南,蠻夷之國號。星明,則夷兵盛;動搖,夷兵為亂;守常,則吉。



 軍門二星,在青丘西,一曰在土司空北,天子六宮之門。主營候,設豹尾旗,與南門同占。星非其故,及客星犯之,皆為道不通。



 器府三十二星,在軫宿南,樂器之府也。明,則八音和,君臣平;不明,則反是。客、彗犯之,樂官誅。赤雲氣掩之,天下
 音樂廢。



 土司空四星,在青丘西,主界域,亦曰司徒。均明,則天下豐;微暗,則稼穡不登。太白、熒惑犯之,男女廢耕桑。客、彗犯之,為兵起,民流。



 按《步天歌》,以左轄右轄二星、長沙一星、軍門二星、土司空四星、青丘七星、器府三十二星俱屬軫宿;《晉志》惟轄星、長沙附於軫,餘在二十八宿之外;《乾象新書》以軍門、器府、土司空屬翼,青丘屬軫;武密書以軍門
 屬翼,餘皆屬軫。今從《步天歌》,而附見諸家之說。



\end{pinyinscope}