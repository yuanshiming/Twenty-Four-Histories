\article{列傳第二百四十二世家六}

\begin{pinyinscope}

 世家六○湖南周氏荊南高氏漳泉留氏陳氏



 湖南周行逢,朗州武陵人。少無賴,不事產業。嘗犯法配
 隸鎮兵,以驍勇累遷裨校。自唐乾寧二年,馬氏專有湖南二十州之地,雖稟朝廷正朔,其郡守官屬皆自署。至周廣順初,兄弟爭國,求援於江南李景,景遣大將邊鎬率兵赴之,因下長沙,遷馬氏之族於建康,封希萼為楚王,居洪州,希崇鎮舒,居揚州。宋興,希崇率兄弟十七人歸朝,皆為美官。景以鎬為潭帥。會朗州眾亂,推衙將劉言為留後,言以行逢為都指揮使。行逢以眾情表於景,請授言節鉞,景不從。召言入金陵,言懼,遣副使王進逵、
 行軍何景真與行逢帥舟師襲破潭州,鎬遁去,行逢等據其城。言遣使上言長沙兵亂,焚燒公府,請移治朗州。周祖即以言為朗帥,王進逵為潭帥,行逢為潭州行軍司馬、領集州刺史。未幾,進逵寇朗州,害劉言,周祖即以進逵為朗州節度,以行逢領鄂州節度、知潭州軍府事。初,朗州人謂劉言為「劉咬牙」,馬氏將亂,湘中童謠云:「馬去不用鞭咬牙過今年。」及邊鎬俘馬氏,鎬為劉言所逐,而言亦被害。



 顯德中,世宗將用師淮甸,詔朗州王進逵
 出師入鄂州界,進逵遣裨將潘叔嗣領兵五千為先鋒。行及鄂州界,叔嗣乃回戈襲進逵,進逵聞之,倍道先入武陵。叔嗣攻其城,進逵敗走,為叔嗣所殺,迎行逢為節度。行逢至,即斬叔嗣以徇。世宗乃授行逢郎州大都督、武平軍節度、製置武安靜江等州軍事兼侍中,盡有湖南之地。宋初,加兼中書令。



 行逢在鎮,盡心為治,辟署官屬,必取廉介之士。有女婿求補吏,不許,返給以耒耜,語之曰:「吏所以治民也,汝才不能任職,豈敢私汝以祿邪?
 姑歸墾田以自活。」其公正多此類。條教簡約,民皆悅之。然性多猜忌,左右少有忤意者必置於法,麾下之人重足累息。有何景山者,為王進逵記室,常狎侮行逢。及行逢為帥,署景山益陽令,數月,縛投於江。又館驛巡官鄧洵美與翰林學士李昉同年進士,會昉使行逢,召至傳舍,與話終日。行逢疑其泄己陰事,黜為易俗場官,潛遣殺之。由是士流不附。



 馬氏舊僚有天策府學士徐仲雅,性滑稽,頗恃才倨傲,行逢以為節度判官。行逢多署溪
 洞蠻酋為司空、太保,一日謂仲雅曰:「吾奄有湖湘,兵強俗阜,四鄰其懼我乎?」仲雅曰:「公部內司空滿川,太保遍地,孰敢不懼?」行逢不悅,擯斥仲雅。行逢妻潘氏貌醜,性剛狠。行逢為帥,妻不為屈,不入府署,躬率奴仆耕織以自給,賦調必先期輸送。行逢止之,不從,曰:「稅,官物也,若主帥自免其家,何以率下?」



 建隆三年十月,行逢卒,追封汝南郡王。



 子保權,年十一。初為武平軍節度副使,太祖授以起復檢校太尉、朗州大都督、武平軍節度。初,行逢
 疾且亟,召將校托保權曰:「吾部內凶狠者誅之略盡,唯張文表在焉,吾死,文表必亂。諸公善佐吾兒,無失土宇,必不得已,當舉族歸朝,無令陷於虎口。」行逢卒,明年春,文表果自衡州舉兵據潭州,將取朗陵,盡滅周氏。保權乞師於朝廷,江陵高繼衝亦以其事聞。上遣中使趙遂齎詔諭文表,而保權之奏繼至。乃遣山南東道節度慕容延釗為湖南道行營都部署,宣徽南院使李處耘為都監,率淄州刺史尹崇珂、申州刺史聶章、郢州刺史趙
 重進、判四方館事武懷節、氈毯使張繼勳、染院副使康延澤、內酒坊副使盧懷忠等將步騎往平之,又發安、復等十州兵會於襄陽。師及江陵,趙遂至潭州,文表已為保權之眾所殺。



 保權牙校張從富輩,以為文表已平而王師繼進不已,懼為襲取,相與拒守。延釗令閣門使丁德裕先路安撫。及至城下,從富輩拒而不納,盡撤部內橋梁,沉舫伐樹塞路。德裕以不奉詔不敢與戰,退軍以須朝旨。延釗以聞,太祖遣中使諭保權及將校曰:「爾本
 請師救援,故發大軍以拯爾難。今妖孽既殄,是有大造於爾輩,反拒王師何也?無自取塗炭,重擾生聚。」何權出軍於澧州南,未及交鋒,望風而潰。復還朗州,焚廬舍廩庫皆盡,驅略居人奔竄山谷,城郭為之一空。王師長驅而南,獲從富於西山下,梟首朗市。其大將汪端劫保權並家屬,棄城亡匿山洞,王師至數月,獲保權。武懷節分兵克岳州,端擁保權眾寇略,未幾亦就擒,礫於市,湖湘悉平。



 保權至,上章待罪,優詔釋之。賜。襲衣、金帶、鞍勒馬、
 茵褥、銀器千兩、帛二千匹、錢千貫,授右千牛衛上將軍,葺京城舊邸院為第,令居焉。仍下詔朗州,增築行逢之墓。保權乾德五年累遷右羽林統軍。太平興國元年,知并州,賜錢三百萬。雍熙二年,卒,年三十四。



 李觀象,桂州臨桂人。行逢署為掌書記。行逢性殘忍,多誅殺。觀象懼及禍,清苦自勵,以求知遇,帳幃、寢衣悉以紙為之。行逢頗加信任,軍府之政一皆取決。



 觀象涉經史,有文辭,忌才怙寵,湖南士人多為所排擯。行逢臨終
 托以後事,令其子保權善待之。及張文表難作,王師壓境,觀象謂保權曰:「我所恃者北有荊渚,以為唇齒,今高氏拱手聽命,朗州勢不獨全,莫若幅巾歸朝,則不失富貴。」保權幼懦,不能用其言。及湖湘平,太祖聞觀象嘗為保權畫謀,以為左補闕。



 張文表,朗州武陵人。從王進逵、周行逢舉兵逐邊鎬,行逢署文表衡州刺史,頗心忌之,常欲誅文表,未有以發。及行逢卒,保權遣兵代永州戍卒,路出衡陽,文表遂驅
 之以襲潭州。時行軍司馬廖簡知留後,素輕文表,不為之備。方宴飲,外報文表兵至,簡殊不以介意,謂四坐曰:「此黃口小兒,至則成擒,何足患也?」飲啖如故。俄文表率眾徑入府中,簡醉不能彀弓弩,但按膝叱之,文表遂害簡及坐客十餘人。保權遣其將楊師璠悉眾以禦文表,保權泣謂眾曰:「先君可謂知人矣。今墳土未幹,文表構逆,軍府安危,在此一舉,諸公勉之!」眾皆感憤,遂破其眾於平津亭,擒文表臠而食之。



 初,文表將攻長沙,猶豫未
 決,有小校夢文表龍出領下,明日以告,文表喜曰:「天命也。」及敗,梟首於朗陵市。



 荊南高保融字德長,其先陝州峽石人。祖季興,唐末為荊南節度,曆梁、後唐封南平王,卒。子從誨嗣,至太傅、中書令,《五代史》有傳。



 從誨生保融,以長興初蔭補太子舍人,賜緋。晉天福中,製授檢校司空、判內外諸軍,俄遷節度副使。開運末,領峽州刺史,累加至檢校太傅。漢初,從誨卒,權知軍府事,製授起復檢校太尉、同平章事、江陵
 尹、荊南節度、荊歸峽觀察使,遣翰林使郭允明賜衣幣。乾祐二年,加檢校太師兼侍中。周廣順初,加兼中書令,封勃海郡王,正衙命使禮部尚書王易、副使刑部郎中景範發冊命,仍賜禮服冠劍。顯德初,進封南平王。世宗即位,加守中書令。



 世宗征淮南,詔保融出水軍數千人抵夏口為犄角。淮甸平,璽書褒美,以絹數萬匹賞其軍。世宗將議伐蜀,保融上言請率舟師趣三峽。六年,恭帝即位,加守太保。宋初,守太傅,連遣使貢獻,恩顧甚厚。是
 歲八月,卒,年四十一。廢朝三日,遣儀鸞使李繼超賜賻物,兵部尚書李濤、兵部郎中率汀持節冊贈太尉,諡正懿。



 保融性迂闊淹緩,禦兵治民,一時術略政事,悉委於母弟保勖焉。子繼沖、繼充,繼充至歸州刺史。



 保勖字省躬,從誨第十子,保融同母弟也。晉天福初,起家領漢州刺史。保融嗣政,令判內外諸軍事。周廣順元年,加檢校太傅,充荊南節度副使。顯德初,從保融之請,加檢校太尉,充行軍司馬,領寧江軍節度。融卒,保勖權
 知軍府,奉章以聞,太祖即授以節度使。建隆二年,遣其弟保寅入貢。初,保融於紀南城北決江水瀦之七里餘,謂之北海,以閡行者。至是太祖因保寅歸,諭旨令決去,使道路無阻。



 保勖幼多病,體貌臞瘠,淫泆無度。日召娼妓集府署,擇士卒壯健者令恣調謔,保勖與姬妾垂簾共觀,以為娛樂。又好營造台榭,窮極土木之工,軍民咸怨。政事不治,從事孫光憲切諫不聽。三年十一月,卒,年三十九。廢朝二日,贈侍中,遣禦廚使李光睿賻祭。



 初,保
 勖在保抱,從誨獨鍾愛,故或盛怒,見之必釋然而笑,荊人目為「萬事休」。及保勖之立,藩政離弱,卒裁數月遂失國,亦預兆也。



 繼沖字讚平,保融長子也。周顯德六年,以蔭檢校司空,為荊州節度副使。建隆三年,保勖寢疾,以繼沖為節度副使,權知軍府。保勖卒,四年正月,製授繼沖為檢校太保、江陵尹、荊南節度。



 時湖南張文表叛,周保權求救於朝廷,詔江陵發水軍三千人赴潭州,繼沖即遣親校李
 景威將之而往。二月,慕容延釗、李處耘等率眾至,繼沖以牛酒犒師,開門納延釗等。即遣客將王昭濟、蕭仁楷奉表納土。太祖令禦廚使郜嶽持詔安撫,樞密承旨王仁贍為荊南都巡檢使,仍令齎衣服、玉帶、器幣、鞍勒馬以賜繼沖。授繼沖馬步都指揮使,梁延嗣為復州防禦使,節度判官孫光憲為黃州刺史,右都押衙孫仲文為武勝軍節度副使,知進奏鄭景玫為右驍衛將軍,王昭濟左領軍衛將軍,蕭仁楷供奉官。繼沖籍管內芻糧錢
 帛之數來上,又獻錢五萬貫、絹五千匹、布五萬匹,復遣支使王崇範詣闕貢金器五百兩、銀器五千兩、錦綺二百段、龍腦香十斤、錦繡帷幕二百事。三月,詔鞍轡庫使翟光裔齎官告、旌節賜繼沖,並存問參佐官吏等;又以保融兄弟、諸父江陵少尹保紳為衛尉卿,節院使保寅為將作監、充內作坊使,左衙都將保緒為鴻臚少卿,右衙都將保節為司農少卿,合州刺史從翊為右衛將軍,衙將保遜為左監門衛將軍,巴州刺史保衡為歸州刺
 史,知峽州事保膺為本州刺史,衙將從詵為右衙率府率,從讓為左清道率府率,從謙為左司禦率府率;又以王崇範為節度判官,高若拙觀察判官,梁守彬江陵少尹,韋仲宣掌書記,胡允修節度推官,州縣官悉仍舊,別賜管內符印。五月,保紳等來朝,各賜京城第一區。六月,命王仁贍兼知軍府事。



 會是歲將郊祀,表求入覲,可之。十月,至闕下,獻金銀器、錦帛、寶裝弓劍、繡旗幟、象牙、玉鞍勒等,賜賚甚厚。郊禋畢,授繼沖徐州大都督府長史、
 武寧軍節度使、徐宿觀察使。繼沖鎮彭門幾十年,委政僚佐,部內亦治。開寶六年,卒,年三十一。廢朝二日,贈侍中,遣中使護喪,葬事官給。



 自高季興據有荊南、歸峽之地,傳襲三世五帥,凡四十餘年。



 保寅字齊巽。晉天福七年,以蔭授太子舍人,賜緋,累加檢校司空。兄保融襲封,奏署節院使,賜金紫。宋興,保勖既襲封,遣保寅入覲,太祖召對便殿,授掌書記遣還。保寅語保勖曰:「真主出世,天將混一區宇,兄宜首率諸國
 奉土歸朝,無為他人取富貴資。」保勖不聽。



 王師討武陵,道出荊渚,保寅奉牛酒迎犒軍鋒。太祖嘉之,驛召赴闕,授將作監,充內作坊使,賜第一區。俄知宿州。乾德四年,丁外艱,起復,轉少府監。開寶五年,知懷州,曆司農、衛尉二卿。是州本隸河陽,時趙普為帥,與保寅素有隙,事多抑製,保寅心不能平,手疏請罷支郡之制,詔從之。又為西川諸州都巡檢使,改光祿卿,曆知同、汝二州,改光化軍。卒,年六十八。廢朝,賻錢十萬。



 初,保寅在懷州,蘇易簡、
 王欽若並妙年始趨學;在同州,錢若水為從事;在光化軍,張士遜其邑人也。保寅一見皆獎拔,許以遠大,議者多其知人。



 子輔政、輔之、輔堯、輔國,並進士及第。輔政至秘書丞,輔之至太常丞。



 孫光憲字孟文,陵州貴平人。世業農畝,惟光憲少好學。遊荊渚,高從誨見而重之,署為從事。曆保融及繼沖三世皆在幕府,累官至檢校秘書監兼御史大夫,賜金紫。慕容延釗等救朗州之亂,假道荊南,繼沖開門納延釗,
 光憲乃勸繼沖獻三州之地。太祖聞之甚悅,授光憲黃州刺史,賜賚加等。在郡亦有治聲。乾德六年,卒。時宰相有薦光憲為學士者,未及召,會卒。



 光憲博通經史,尤勤學,聚書數千卷,或自抄寫,孜孜讎校,老而不廢。好著撰,自號葆光子,所著《荊台集》三十卷,《鞏湖編玩》三卷,《筆傭集》三卷,《橘齋集》二卷,《北夢瑣言》三十卷,《蠶書》二卷。又撰《續通曆》,紀事頗失實,太平興國初,詔毀之。子謂、讜,並進士及第。



 梁延嗣,京兆長安人。少事高季興,頗見委任。表授檢校司空、領綿州刺史,充衙內馬步軍都指揮使。曆事四帥,人稱其忠藎。繼沖之納士也,延嗣亦嘗勸之。復率荊之水軍從慕容延釗越戰,太祖嘉之,授復州防禦使,充湖南前軍步軍都指揮使兼排陣使。事因郊禮,自復州入朝,太祖慰撫之曰:「使高氏不失富貴,爾之力也。」改濠州防禦使,有善政,詔書褒美。



 延嗣頗知書,好接士。嘗暴疾,禳於城隍神,是夕,夢神人告以九九之數,俄疾愈。開寶
 九年,卒,年八十一。



 漳泉留從效,泉州永春人。幼孤,事母兄以孝悌聞。頗知書,好兵法。



 唐末,王審知據有福建之地,子延鈞,後唐長興中僭稱帝,國號閩,都福州,為其下所殺,立審知次子延羲。晉天福末,部將朱文進殺延羲據其位,署其黨黃紹頗為泉州刺史,程贇為漳州刺史,許文稹為汀州刺史。時審知子延政為建州刺史,亦僭稱帝。



 泉人念王氏失國,群逆分據,時從效為泉州散指揮使,與其黨王忠
 順、董思安及所親蘇光誨相與圖議,興復王氏。從效倡言:「吾等皆受王氏恩遇,今王氏子孫未復位而不思報,可謂忠義乎?聞建州士卒謀盡力擊福州以復王氏,苟一旦功先成,王氏復位,我輩何麵見之邪?」於是忠順、思安置酒從效家,募敢死士,得陳洪進等五十二人,夜持白梃逾城而入,劫庫兵,擒紹頗斬之。立延政從子繼勳為刺史,從效等三人自署為統帥,洪進等皆為指揮使。繼勳令送紹頗首於建州,奉延政為主。



 延政遂送款於
 江南李景。文進率眾攻泉州,為從效所敗。會景遣將討王氏之亂,圍福州,兩浙錢氏發兵來援。景將但克汀、建而歸,福州入於錢氏。從效以兵劫繼勳送江南,自領漳、泉二州留後,李景即建泉州為清源軍,授從效節度、泉漳等州觀察使。閩中五州自此分矣。景累授從效同平章事兼侍中、中書令,封鄂國公、晉江王。



 從效出自寒微,知人疾苦,在郡專以勤儉養民為務,常衣布素,置公服於中門之側,出則衣之。每言我素貧賤,不可忘本。民甚
 愛之,部內安治。王氏有二女嫁為郡人妻,從效奉之甚謹,資給豐厚。每歲取進士、明經,謂之「秋堂」。



 世宗征淮南,李景以兵十萬保紫金山。從效累表於景,言其頓兵老師,形勢非便。既而果敗,江北之地盡入於中朝。從效遣衙將蔡仲贇等為商人,以帛書表置革帶中,自鄂路送款內附。又遣別駕黃禹錫間道奉表,以獬豸通犀帶、龍腦香數十斤為貢。世宗錫詔書嘉納之。從效又乞置邸京師,世宗以其素附江南,慮其非便,不許。



 宋初,從效遂
 上表稱藩,貢奉不絕。會李景遷洪州,從效疑景討己,頗懼,遣其從子紹錤齎厚幣獻景,又遣使假道吳越入貢。太祖特命使厚賜以撫之,使未至,從效疽發背卒,年五十七。偽贈太尉、靈州大都督。



 從效無嗣,以兄從願之子紹錤、紹糸茲為子。從效寢疾時,從願守漳州,紹錤在金陵,紹鎡尚幼。衙校張漢思、陳洪進等率兵劫從效遷東亭,漢思自稱留後,洪進為副使,時建隆三年也。明年,洪進又廢漢思而自立。



 從效再從弟仁譓,淳化中為泗州長
 史,有清節,官散奉薄,雖藜藿不充,未嘗妄幹人。太宗聞之,召赴闕,特遷揚州觀察支使。大中祥符七年,從效孫丕式詣闕上從效所受太祖朝制書,授三班借職。



 陳洪進,泉州仙遊人。幼有壯節,頗讀書,習兵法。及長,以材勇聞。隸兵籍,從攻汀州,先登,補副兵馬使。



 從留從效殺黃紹頗,將以紹頗首送建州,請出兵為援,群下以道阻賊盛,憚其行。洪進慮事久生變,獨請往,至尤溪,賊數千人遮道不得前,洪進紿賊曰:「福州、泉州已為義師所
 襲,爾輩復為何人戍守?」即持紹頗首示之曰:「我送此於建州迎嗣君以歸國,爾輩將安歸乎?」賊遂潰,渠帥數人皆聽命。洪進至建州,延政大悅,以為本州馬步行軍都校。是歲,晉開運元年也。自是漳州殺程贇,迎延政從子繼成為刺史。許文稹以汀州降,連重遇殺朱文進,傳首建州,福人又殺重遇,延政遂遣洪進歸泉州。三年,李景陷建州,延政入江南。明年,泉州留從效劫王繼勳降江南,景以從效為清源軍節度,洪進為統軍使,與副使張
 漢思同領兵柄,累立戰功。



 從效卒,少子紹鎡典留務。月餘,洪進誣紹鎡將召越人以叛,執送江南。推副使張漢思為留後,自為副使。漢思年老醇謹,不能治軍務,事皆決於洪進。漢思諸子並為衙將,頗不平洪進,圖欲害之,漢思亦患其專。明年夏四月,漢思大享將吏,伏甲於內,將害洪進。酒數行,地忽大震,棟宇將傾,坐立者不自持。同謀者以告洪進,洪進亟去,眾驚悸而散。漢思事不成,慮洪進先發,常嚴兵為備。洪進子文顯、文顥皆為指揮
 使,勒所部欲擊漢思,洪進不許。一日,洪進袖置大鎖,從二子常服安步入府中,直兵數百人,皆叱去之。漢思方處內齋,洪進即鎖其門,使人叩門謂漢思曰:「郡中軍吏請洪進知留務,眾情不可違,當以印見授。」漢思惶懼不知所為,即自門間出印與之。洪進遽召將校吏士告之曰:「漢思昏耄不能為政,授吾印,請吾蒞郡事。」將吏皆賀。即日遷漢思別墅,以兵衛送。遣使請命於李煜,煜以洪進為清源軍節度、泉南等州觀察使。



 時太祖平澤、潞,下
 揚州,取荊湖,威振四海。洪進大懼,遣衙將魏仁濟間道奉表,自稱清源軍節度副使、權知泉南等州軍府事,且言張漢思老耄不能禦眾;請臣領州事,恭聽朝旨。太祖遣通事舍人王班齎詔撫諭,又與李煜詔曰:「泉州陳洪進遣使奉表言,為眾所推,因而總領州事,以誠控告,聽命於朝。觀其傾輸,尤足嘉尚。但聞泉州昔嘗附麗,尤荷撫綏。然變詐多端,屢移主帥,恐其地裏遼遠,制禦有所未遑。朕以書軌大同,恩威遠被,嘉其款附,己降詔書。蓋
 矜其遠俗便安,不必以彼此為意,想惟明哲,當體朕懷。」煜上言:「洪進多詐,首鼠兩端,誠不足聽。」 太祖又詔諭之,煜乃聽命。



 建隆四年,遣使朝貢。是冬,又貢白金萬兩,乳香茶藥萬斤。煜復上言,請寢洪進恩命。太祖又以諭煜。乾德二年,製改清源軍為平海軍,授洪進節度、泉漳等州觀察使、檢校太傅,賜號推誠順化功臣,鑄印賜之。以文顯為節度副使,文顥為漳州刺史。是年夏,丁家艱,起復。



 洪進每歲以修貢朝廷,多厚斂於民,第民貲百萬以
 上者令差入錢,以為試協律、奉禮郎,蠲其丁役。及江南平,吳越王來朝,洪進不自安。遣其子文顥入貢乳香萬斤、象牙三千斤、龍腦香五斤。太祖因下詔召之,遂入覲。至南劍州,聞太祖崩,歸鎮發哀。



 太宗即位,加檢校太師。明年四月,來朝,朝廷遣翰林使程德玄至宿州迎勞。既至,賜錢千萬、白金萬兩、絹萬匹,禮遇優渥。又增其食邑,以其子文顥為團練使,文顗、文頊並為刺史。洪進遂上言曰:「臣聞峻極者山也,在汙壤而不辭;無私者日也,雖
 覆盆而必照。顧惟遐僻,尚隔聲明,願歸益地之圖,輒露由衷之請。臣所領兩郡,僻在一隅,自浙右未歸,金陵偏霸,臣以崎嶇千里之地,疲散萬餘之兵,望雲就日以雖勤,畏首畏尾之不暇。遂從間道,遠貢赤誠,願傾事大之心,庶齒附庸之末。太祖皇帝賜之軍額,授以節旄,俾專達於一方,復延賞於三世。祖父荷漏泉之澤,子弟享列土之榮。棨戟在門,龜緺盈室,雖冠列藩之寵,未修肆覲之儀。暨江表底平,先皇厭世,會嬰犬馬之病,尚阻雲龍
 之庭。皇帝陛下欽嗣丕基,誕敷景命,臣遠辭海嶠,入覲天墀,獲親咫尺之顏,疊被便蕃之澤。六飛遊幸,每奉屬車之塵;三殿宴嬉,屢挹大樽之味。旬浹之內,雨露駢臻,至於童男,亦荷殊獎。恩榮若此,報效何階?誌益戀於君軒,心遂忘於坎井。臣不勝大願,願以所管漳、泉兩郡獻於有司,使區區負海之邦,遂為內地;蚩蚩生齒之類,得見太平。伏望聖慈,授臣近地別鎮。臣男文顯等早膺朝獎,皆忝郡符,牙校賓僚,久經驅策,各希玄造,稍霈鴻私。」
 太宗優詔嘉納之。以洪進為武寧軍節度、同平章事,留京師奉朝請。諸子皆授以近郡,賜白金萬兩,各令市宅。



 明年,從平太原。六年,封杞國公。雍熙元年,進封岐國公。洪進年老,富貴且極,上言求致仕,優詔免其朝請。二年,以疾卒,年七十二。廢朝二日,贈中書令,諡曰忠順,中使護喪,葬事官給。



 洪進在泉州,日方晝,有蒼鶴翔集內齋前,引吭向洪進。洪進視之,有魚鯁其喉,即以手探取之,魚猶活,鶴馴擾齋中數日而後去,人皆異之。



 洪進弟銛,
 初為泉州都指揮使。開寶四年,授漳州刺史,入貢至宿州,卒。銛子文璉,供奉官、閣門祗候。



 文顯字仲達。洪進領漳、泉節制,署左神機指揮使,遷泉州馬步軍都軍使、右軍押衙。乾德初,朝命平海軍節度副使,累加檢校太保。洪進歸朝,授文顯通州團練使、知泉州。未幾代還。時太宗征太原,朝於行在。久之,出為青齊廬壽、西京水南北、陝州四州都巡檢使。



 文顯與諸弟不睦,咸平初,御史中丞李惟清抗疏曰:「文顯等並分符
 竹,委以方面,一門榮盛,當世罕儔。先人之墳土未幹,私室之風規大壞;弟兄列訟,骨肉為仇,官奉私藏,同居異爨,屢經赦宥,而久積人言。文顯首起訟湍,當律文尊長之坐,乞置散秩,以警浮俗。」詔曰:「文顯等頗傷名教,合置邦刑,以其父有忠勳,未忍捐棄,宜賜誡諭,許其改過。倘無悛革,當正簡書,令御史臺告諭之。」以疾改通許鎮都監。六年,卒,年六十五。子宗憲,曆虞部員外郎,為西京作坊使;宗元,殿中丞。



 文顥,初為泉州右軍散兵馬使、衙內都指揮使。俄權知漳州,朝命漳州刺史,凡七年,求還泉州,署行軍司馬。



 開寶末,江南平,洪進遣第三子文顗入貢,文顗不欲行,乃遣文顥。至京師,自陳願留以俟父入覲,太祖嘉之。及洪進歸朝,授文顥房州刺史,會升房州為節鎮,換康州刺史。端拱初,出知同州,錢若水為從事,文顥深禮之,委以郡政。咸平初,知耀州,又徙徐州,坐用刑失入,責授左武衛大將軍、知漣水軍。上念其父納土效順,復以為康州
 刺史,留京師。



 大中祥符初,議東封,以濮州馳道所出,命知州事,頓置供擬頗勤至,詔褒之。駕至,召見勞問。禮畢,改衡州刺史,特給內地刺史奉料,未幾代還。以老疾累表求致仕,詔免朝謁,歲給公費及月廩並如故。六年,卒,年七十二。



 文顗,始為泉州衙內都指揮使、知漳州。洪進歸朝,授滁州刺史,仍舊知州。俄召歸,奉朝請。景德中,換光州,以久次,領和州團練使,曆知海濮濰沂黃五州、信陽軍,所至
 無能稱。卒年七十一。錄其子宗綬為大理評事,孫永弼、永升為三班借職,次子宗纘太子中舍。



 文頊,本文顯子。初,洪進在泉州,有相者言一門受祿,當至萬石。時洪進與三子皆領州郡,而文頊始生,乃以文頊為子,欲應其言。初補泉州衙內都校,又為衙內都監使,朝命領順州刺史,歸朝為登州刺史。滄、棣有寇盜,命為巡檢使。會以禁軍大校趙延溥為登州團練使,文頊改舒州刺史。淳化三年,卒,年三十五。文頊頗知書,亦工
 畫。子宗絳,為殿中丞。



\end{pinyinscope}