\article{志第二十五 律歷五}

\begin{pinyinscope}

 步日躔



 周天分:三百八十六萬八千六十五、秒二。



 周天度:三百六十五度。虛分二千七百一十五、秒二,約分二十五、秒六十四。



 歲差:一百二十五、秒二。



 乘法:三十二。



 除法:四百八十七。



 秒法:一百。



 求每日盈縮定數:以乘法乘所入氣升降分,如除法而一,為其氣中平率;與後氣中平率相減,為差率;半差率,加減其氣中平率,為其氣初、末泛率。至後加為初,減為末;分後減為初,加
 為末。



 又以乘法乘差率,除法而一,為日差;半之,加減初、末泛率,為初、末定率。至後減初加末,分後加初減末。



 以日差累加減氣之定率,為每日升降定率;至後減,分後加。



 以每日升降定率,冬至後升加降減,夏至後升減降加,其氣初日盈縮分,為每日盈縮定數;其分、至前一氣先後率相減,以前末泛率為其氣初泛率,以半日差,至前加之,分前減之。



 為其氣初日定率。餘依本術。求朏朒準此。



 求經朔弦望入氣:置天正閏日及餘,如氣策及餘秒以下者,以減氣策及餘秒,為入大雪氣;已上者去之,餘以
 減氣策及餘秒,為入小雪氣:即得天正十一月經朔入大、小雪氣日及餘秒。求弦、望及後朔入氣,以弦策累加之,滿氣策及餘秒去之,即得。



 求定氣日:冬、夏二至以常氣為定。餘即以其氣下盈縮分縮加盈減常氣約餘為定氣,滿若不足,進退大餘,命甲子,算外,即定氣日及分。



 求經朔弦望入氣朏朒定數:各以所入氣小餘乘其日損益率,如樞法而一,即得。



 求赤道宿度



 斗:二十六度牛:八度女:十二度虛:十度及分



 危:十七度室:十六度壁:九度



 北方七宿九十八度虛分二千七百一十五、秒二,約分二十五、秒六十四。



 奎:十六度婁:十二度胃:十四度昴:十一度



 畢:十七度觜:一度參:十度



 西方七宿八十一度。



 井:三十三度鬼:三度柳:十五度星:七度



 張:十八度翼:十八度軫:十七度



 南方七宿一百一十一度。



 角:十二度亢:九度氐:十七度房:五度



 心:五度尾:十八度箕:十一度



 東方七宿七十五度。



 前皆赤道度,其畢、觜、參及輿鬼四宿度數與古度不同,自《大衍歷》依渾天儀以測定,為用紘帶天中,儀極是憑,以格黃道。



 推天正冬至赤道日度:以歲差乘距所求積年,滿周天
 分去之,不盡,用減周天分,餘以樞法除之為度,不盡為餘秒。其度,命以赤道虛宿七度外起算,依宿次去之,不滿者,即得天正冬至加時赤道日躔所距宿度及餘秒。其餘以樞法退除為分及秒,各以一百為度。



 求二十四氣赤道日度:置天正冬至加時赤道日度及餘秒,以氣策及餘秒累加之。先以三十六乘赤道秒,以一百乘氣策秒,然後加之,即秒母皆同三千六百。



 滿赤道宿次去之,即各得二十四氣加時赤道日躔宿度及餘秒。



 求二十四氣昏後夜半赤道日度:各以其氣小餘減樞法,其秒亦以一百乘,然乃減之。



 餘加其氣加時赤道日躔宿度及餘秒,即其氣初日昏後夜半赤道日度及餘秒。求次日累加一度,滿宿次去之,各得所求。



 求赤道宿積度:置冬至加時日躔赤道宿全度,以冬至加時日躔赤道宿度及約分秒減之,餘為距後度及分秒;以赤道宿度累加距後度,即得各赤道宿積度及分秒。



 求赤赤道宿積度入初末限:各置赤道宿積度及分秒,滿九十一度三十一分、秒一十一去之,餘四十五度六十六分以下為入初之限;已上者,用減九十一度三十一分,餘為入末限度及分秒。



 求二十八宿黃道度:各置赤道宿入初、末限度及分,用減一百二十五,餘以初、末限度及分乘之,十二除為分,分滿百為度,命為黃、赤道差度及分;至後分前以減、分後至前以加赤道宿積度,為其宿黃道積度;以前宿黃
 道積度減其宿黃道積度,為其宿黃道度及分。其分就近約為太、半、少。



 黃道宿度



 斗:二十三太



 牛:七半



 女:十一半



 虛:十秒六十四



 危:十七太



 室:十七



 壁:九少



 北方七宿九十七度。半、秒六十四



 奎:十七半



 妻:十二太



 胃:十四太



 昴:十一



 畢:十六



 觜:一



 參:九少



 西方七宿八十二度。



 井:三十



 鬼:二



 柳:十四



 星:七



 張:十八太



 翼:十九少



 軫:十八



 南方七宿一百一十度。



 角:十三



 亢:九半



 氐:十五半



 房:五



 心:四



 尾:十七



 箕:十



 東方七宿七十四度。



 求冬至加時黃道日躔宿次:以冬至加時赤道日躔宿
 度,用減一百二十五,餘以冬至加時赤道度及分乘之,十二除為分,分滿百為度,用減九十一度赤道日度及分,即冬至加時黃道日躔宿度及分。



 求二十四氣初日加時黃道日躔宿次:置所求年冬至日躔黃道赤道差,以次年黃赤道差減之,餘以所氣數乘之,二十四而一,所得,以加其氣下中積及約分,又以其氣初日盈縮分盈加縮減之,用加冬時黃道日度,依宿次命之,即各得其氣初日加時黃道日躔所在宿度
 及分。若其年冬至加時赤道日躔度空,分、秒在歲差已下者,即如前宿全度,乃求黃赤道差,以次年冬至加時黃赤道差減之,餘依本術,各得所求。此術以究算理之微,亟求其當,止以盈縮分加減中積,以天正冬至加時黃道日度加而命之。



 求二十四氣初日晨前夜半黃道日躔宿次:置一百分,分以一百約其氣初日升降分,升加降減之,一日所行之分乘其初日約分,所得滿百為分,分滿百為度,不滿百分為秒,以減其初日黃道加時日躔宿次,即其日晨前夜半黃道日躔宿次。



 求每日晨前夜半黃道日躔宿次:各因二十四氣初日晨前夜半黃道日躔宿次,日加一度,以一百約每日升降為分秒,升加降減之,以黃道宿次命之,即每日晨前夜半黃道日躔所距宿度及分。



 步月離



 轉周分:二十九萬一千八百三、秒五百九十四。



 轉周日:二十七、餘五千八百七十三、秒五百九十四。



 朔差日:一、餘一萬三百三十五、秒九千四百六。



 望差:一十四、餘八千一百四、秒五千。



 弦策:七、餘四千五十二、秒二千五百。



 七日:初數九千四百四十一,初約分八十九;末數一千一百七十九,末約分一十一。



 十四日:初數八千二百三十二,初約分七十八;末數二千三百五十八,末約分二十二。



 二十一日:初數七千五十二,初約分六十九;末數三千五百三十八,末約分二十三。



 二十八日:初數五千八百七十三,初約分五十六。



 已上秒法一萬。



 上弦:九十一度三十一分、秒四十
 一。



 望:一百八十二度六十二分、秒八十二。



 下弦:二百七十三度九十四分、秒二十三。



 平行:一十三度三十六分、秒八十七半。



 已上秒母一百。



 推天正十一月經朔入轉:置天正十一月經朔積分,以轉周分秒去之,不盡,以樞法除之為日,不滿為餘秒,命日,算外,即所求天正十一月經朔加時入轉日及餘秒。若以朔差日及餘秒加之,滿轉周日及餘秒去之,即次日加時入轉。



 求弦望入轉:因天正十一月經朔加時入轉日及餘秒,以弦策累加之,去命如前,即上弦、望及下弦加時入轉日及餘秒。若以經朔、弦、望小餘減之,各得其日夜半入轉日及餘秒。



 求朔弦望入轉朏朒



 定數:置所入轉餘,乘其日損益率,樞法而一,所得,以損益其下朏朒積為定數。其四
 七日下餘如初數下,以初率乘之,初數而一,以損益朏朒為定數。若初數已上者,以初數減之,餘乘末率,末數而一,用減初率,餘加朏朒,各為定數。其十四日下餘若在初數已上者,初數減之,餘乘末率,末數而一,為朏定數。



 求朔望定日:各以入氣、入轉朏朒定數朏減朒加經朔、弦、望小餘,滿若不足,進退大餘,命甲子,算外,各得定日及餘。若定朔乾名與後朔同名者大,不同者小,其月無中氣者為閏月。凡注歷,觀朔小餘,如日入分已上者,進一
 日,朔或當定,有食應見者,其朔不進。弦、望定小餘不滿日出分,退一日,其望定小餘雖滿此數,若有交食虧初起在日出已前者,亦如之。有月行九道遲疾,歷有三大二小;若行盈縮累
 增損
 之,則
 有四大三小,理數然也,若俯循常儀,當察加時早晚,隨其所近而進退之,不過三大二小。若正朔有加交,時虧在晦、二正見者,消息前後一兩月,以定大小。



 求定朔弦望加時日所在度:置定朔、弦望約分,副之,以乘其日升降分,一萬約之,所得,升加降減其副,以加其
 日夜半日度,命如前,各得其日加時日躔黃道宿次。



 推月行九道:凡合朔所交,冬在陰歷,夏在陽歷,月行青道;冬、夏至後,青道半交在春分之宿,當黃道東;立冬、立夏後,青道半交在立春之宿,當黃道東南:至所沖之宿亦如之。



 冬在陽歷,夏在陰歷,月行白道;冬、夏至後,白道半交在秋分之宿,當黃道西;立冬、立夏後,白道半交在立秋之宿,當黃道西北:至所沖之宿亦如之。



 春在陽歷,秋在陰歷,月行朱道;春、秋分後,朱道半交在夏至之宿,當黃道南;立春、立秋後,朱道半交在立夏之宿,當黃道西南;至所沖之宿亦如之。



 春在陰歷,秋在陽歷,月行黑道。春、秋分後,黑道半交在冬至之宿,當黃道北;立春、立秋後,黑道半交在立冬之宿,當黃道東北:至所沖之宿亦如之。



 四
 序月離雖為八節,至陰陽之所交,皆與黃道相會,故月行有九道。各視月所入正交積度,滿象度及分去之,入交積度及象度並在交會術中。



 若在半象以下者為入初限;已上者,復減象度,餘為入末限;用減一百二十五,餘以所入初、末限度及分乘之,滿二十四而一為分,分滿百為度,所得,為月行與黃道差數。距半交後、正交前,以差數為減;距正交後、半交前,以差數為加。此加減出入六度,單與黃道相較之數,若較赤道,則隨氣遷變不常。



 計去冬、夏至以來度數,乘黃道所差,九十而一,
 為月行與赤道差數。凡日以赤道內為陰,外為陽;月以黃道內為陰,外為陽。故月行宿度,入春分交後行陰歷,秋分交後行陽歷,皆為同名;春分交後行陽歷,秋分交後行陰歷,皆為異名。其在同名,以差數加者加之,減者減之;其在異名,以差數加者減之,減者加之。皆以增損黃道宿積度,為九道宿積度;以前宿九道積度減之,為其九道宿度及分。其分就近約為少、半、太之數。



 推月行九道平交入氣:各以其月閏日及餘,加經朔加
 時入交泛日及餘秒,盈交終日去之,乃減交終日及餘秒,即各平交入其月中氣日及餘秒。滿氣策及餘秒去之,餘即平交入後月節氣日及餘秒。因求次交者,以交終日及餘秒加之,滿氣策及餘秒去之,餘為平交入其氣日及餘秒,若求其氣朏朒定數,如求朔、弦、望經日術入之,各得所求也。



 求平交入轉朏朒定數:置所入氣餘,加其日夜半入轉餘,以乘其日損益率,樞法而一,所得,以損益其下朏朒積,乃以交率乘之,交數而一,為定數。



 求正交入氣:以平交入氣、入轉朏朒定數,朏減朒加平
 交入氣餘,滿若不足,進退其日,即正交入氣日及餘秒。



 求正交加時黃道宿度:置正交入氣餘,副之,以乘其日升降分,一百約之,升加降減其副,乃一百乘之,樞法而一,以加其日夜半日度,即正交加時黃道日度及分秒。



 求正交加時月離九道宿度:以正交度及分減一百二十五,餘以正交度及分乘之,滿二十四,餘為定差。以差加黃道宿度,仍計去冬、夏至以來度數乘差,九十而一,所得,依名同異而加減之,滿若不足,進退其度,命如前,
 即正交加時月離九道宿度及分。



 推定朔、弦、望加時月離所在度:各置其日加時日躔所在,變從九道,循次相當。凡合朔加時,月行潛在日下,與太陽同度,是為加時月離宿次;先置朔、弦、望加時黃道宿度,以正交加時黃道宿度減之,餘以加其正交加時九道宿度,命起正交宿度,算外,即朔、弦、望加時所當九道宿度。其合朔加時若非正交,則日在黃道、月在九道各入宿度,雖多少不同,考其去極,若應繩準,故云月行潛在日下,與太陽同度。



 各以弦、望度及分秒加其所當九道宿度,滿宿次去之,命如前,即各得加時九道月離宿次。



 求定朔夜半入轉:各視經朔夜半入轉,若定朔大餘有進退者,亦加減轉日,不則因經為定。



 求次定朔夜半入轉:因定朔夜半入轉,大月加二,小月加一,餘皆四千七百一十六、秒九千四百六,滿轉周日及餘秒去之,即次定朔夜半入轉;累加一日,去命如前,各得次日夜半轉日及餘秒。



 求月晨昏度:以晨昏乘其日轉定分,樞法而一,為晨轉分;減轉定分,餘為昏轉分;乃以朔、弦、望定小餘乘轉定
 分,樞法而一,為加時分;以減晨昏轉分,餘為前;不足覆減,餘為後;仍前加後減加時月,即晨、昏月所在度。



 求朔、弦、望晨昏定程:各以其朔昏定月減上弦昏定月,為朔後定程;以上弦昏定月減望日昏定月,為上弦後定程;以望日晨定月減下弦晨定月,為望後定程;以下弦晨定月減後朔晨定月,為下弦後定程。



 求每日轉定度:累計每程相距日轉定分,以減定程為盈;不足,覆減為縮;以相距日均其盈縮,盈加縮減每日
 轉定分,為每日轉定度及分。



 求每日晨昏月:因朔、弦、望晨昏月,加每日轉定度及分,盈縮次去之,為每日晨昏月。凡注歷,自朔日注昏,望後次日注晨。



 已前月度並依九道所推,以究算理之精微。如求其速要,即依後術求之。



 推天正經朔加時平行月:置歲周,以天正閏餘減之,餘以樞法除之為度,不盡,退除為分秒,即天正經朔加時平行月積度。



 求天正十一月定朔夜半平行月:置天正經朔小餘,以平行分乘之,樞法而一為度,不盡,退除為分秒,所得,為加時度;用減天正經朔加時平行月,即經朔晨前夜半平行月,其定朔有進退者,即以平行度分加減之。



 即天正十一月定朔晨前夜半平行月積度。



 求次定朔夜半平行月:置天正定朔夜半平行月,大月加三十五度八十分、秒六十一,小月加二十二度四十三分、秒七十三半,滿周天度分去之,即每月定朔晨前
 夜半平行月積度及分。



 求定望夜半平行月:計定朔距定望日數,以乘平行度及分秒,所得,加其定朔夜半平行月積度及分,即定望夜半平行月積度及分。



 求天正定朔夜半入轉:因天正經朔夜半入轉,若定朔大餘有進退者,亦進退之,不則因經而定,即所求年天正定朔晨前夜半入轉及其餘;以樞法退除為約分及秒,皆一百為母。



 求定望及次定朔夜半入轉:因天正定朔夜半入轉及分秒,以朔望相距日累加之,滿轉周日二十七及分五十五、秒四十六去之,即各得定望及次定朔晨前夜半入轉日及分秒。



 求定朔望夜半定月:置定朔、望夜半入轉分,乘其日增減差,一百約之為分,分滿百為度,增減其下遲疾度,為遲疾定度,遲減疾加夜半平行月,為朔望夜半定月;以冬至加時黃道日度加而命之,即朔望夜半月離宿次。
 其入轉若在四七日下,如求朏朒術入之,即得所求。



 求朔望定程:以朔定月減望定月,為朔後定程;以望定月減次朔定月,即望後定程。



 求朔望轉積:計朔至望轉定分,為朔後轉積;自望至次朔亦如之,為望後轉積。



 求每日夜半月離宿次:各以其朔、望定程與轉積相減,餘為程差;以距後程日數除之,為日差;加歲轉定分,為每日行度及分;定程多,加之;定程少,減之。



 以每日行度及分累加朔、
 望夜半宿次,命之,即每日晨前夜半月離宿次。若求晨昏月,以其日晨昏分乘其日轉定度及分,樞法而一,以加夜半月,即晨昏月所在度及分。若以四象為程,兼求弦日平行積餘,各依次入之。若以九終轉定分累加之,依宿次命之,亦得所求。



 步晷漏



 二至限:一百八十二、六十二分。



 一象:九十一、三十二分。



 消息法:七千八百七十三。



 辰法:八百八十二半,八刻三百五十三。



 昏明刻:一百二十九半。



 昏明餘數:二百六十四太。



 冬至陽城晷景:一丈二尺七寸一分半;初限六十二,末限一百二十六、十二分。



 夏至陽城晷景:一尺四寸七分,小分八十;初限一百二十六、十二分,末限六十二。



 求陽城晷景入二至後日數:各計入二至後日數,乃如半日之分五十,又以二至約分減之,即入二至後來午
 中日數及分。



 求陽城晷景入初末限定日及分:置其日中入二至後求日數及分,以其日午中入氣盈縮分盈加縮減之,各如初限已下為在初限;已上,覆減二至限,餘為入末限定日及分。求盈縮分,置入二至後來午中日數及分,以氣策及約分除之為氣數,不盡,為入氣以來日數及分;加其氣數,命以冬、夏至,算外,即其日午中所入氣日及分。置所入氣日約分,如出朏朒術入之,即得所求。



 求陽城每日中晷定數:置入二至初、末限定日及分,如
 冬至後初限、夏至後末限者,以初、末限日及分減一百四十六,餘退一等,為定差;又以初、末限日及分自相乘,以乘定差,滿六千六百四十五為尺,不滿,退除為寸分,命曰晷差;以晷差減冬至晷數,即其日陽城午中晷景定數。如冬至後末限、夏至後初限者,以初、末限日及分減一千二百一十七,餘再退,為定差;亦以初末限日及分自相乘,以乘定差,滿二萬四千九百三十,餘為尺,不滿,退除為寸分,命曰晷差;以晷差加夏至晷數,即其日
 陽城中晷定數。若以中積求之,即得每日晷影常數。



 求每日消息定數:以所入氣日及加其氣下中積,一像已下,自相乘;已上者,用減二至限,餘亦自相乘,皆五因之,進二位,以消息法除之,為消息常數;副置常數,用減五百二十九半,餘乘其副,以二千三百五十除之,加於常數,為消息定數。冬至後為消,夏至後為息。



 求每日黃道去極度及赤道內外度:置其日消息數,十六乘之,以三百五十三除為度,不滿,退除為分,所得,在
 春分後加六十七度三十一分,秋分後減一百一十五度三十一分,即每日黃道去極度分度。又以每日黃道去極度及分,與一象度相減,餘為赤道內、外度。若去極度少,為日在赤道內;去極度多,為日在赤道外,即各得所求。其赤道內外度,為黃、赤道相去度分。



 求每日晨昏分日出入分及半晝分:以每日消息定數,春分後加一千八百五十三少,秋分後減二千九百一十二少,各為每日晨分;用減樞法,為昏分。以昏明餘數
 加晨分,為日出分;減昏分,為日入分;以日出分減半法,為晝分。



 求每日距中度:置每日晨分,三因,進二位,以八千六百九十八除為度,不滿,退除為分,即距子度;用減半周天,餘為距中度;又倍距子度,五除,為每更差度及分。



 求夜半定漏:置晨分,進一位,以刻法除為刻,不滿為分,即每日夜半定漏。



 求晝夜刻及日出入辰刻:倍夜半定漏,加五刻,為夜刻;
 減一百刻,餘為晝刻。以昏明刻加夜半定漏,命子正,算外,即日出辰刻;以晝刻加之,命如前,即日入辰刻。



 求更籌辰刻:倍夜半定漏,二十五而一,為籌差刻;五乘之,為更差刻。以昏明刻加日入辰刻,即甲夜辰刻;以更籌差刻累加之,滿辰刻及分去之,各得每更籌所入辰刻及分。



 求每日昏明度:置距中度,以其日昏後夜半赤道日度加而命之,即昏中星所格宿次;又倍距子度,加昏中星
 命之,即曉中星所格宿次。



 求五更中星:皆以昏中星為初更中星,以每更差加而命之,即乙夜所格宿次;累加之,各得五更中星所格宿次。



 求九服距差日:各於所在立表候之,若地在陽城北,測冬至後與陽城冬至晷景同者,累冬至後至其日,為距差日;若地在陽城南,測夏至後與陽城夏至晷景同者,累夏至後至其日,為距差日。



 求九服晷景;若地在陽城北冬至前後者,置冬至前後日數,用減距差日,為餘日;以餘日減一百四十六,餘退一等,為定差;以餘日自相乘而乘之,滿六千六百四十五除之為尺,不滿,退除為寸分,加陽城冬至晷景,為其地其日中晷常數。若冬至前後日多於距差日,即減去距差日,餘依陽城法求之,各其地其日中晷常數。若地在陽城南夏至前後者,以夏至前後日數減距差日,為餘日,以減一千二百一十七,餘再退,為定差;以餘日自
 相乘而乘之,滿二萬四千九百三十為尺,不滿,退除為寸分,以減陽城夏至晷數,即其地其日中晷常數;如不及減,乃減去陽城夏至日晷景,餘即晷在表南也。若夏至前後日多於距差日,即減去距差日,餘依陽城法求之,各其地其日中晷常數。若求中晷定數,先以盈縮分加減之,乃用法求之,即各得其地其日中晷定數。



 求九服所在晝夜漏刻:冬、夏至各於所在下水漏,以定其處二至夜刻數,相減為冬、夏至差刻。乃置陽城其日
 消息定數,以其處二至差刻乘之,如陽城二至差刻二十而一,所得,為其地其日消息定數。乃倍消息定數,進一位,滿刻法約之為刻,不滿為分,乃加減其處二至夜刻,秋分後、春分前,減冬至夜刻;春分後、秋分前,加夏至夜刻。



 為其地其日夜刻;用減一百刻,餘為晝刻。求日出入辰刻及距中度五更中星,皆依陽城法。



\end{pinyinscope}