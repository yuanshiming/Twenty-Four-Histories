\article{列傳第一百八十三}

\begin{pinyinscope}

 ○陸持之徐鹿卿趙逢龍趙汝騰孫夢觀洪天錫黃師雍徐元傑孫子秀李伯玉



 陸持之,字伯微,知荊門軍九淵之子也。七歲能為文。九
 淵授徒象山之上,學者數百人,有未達,持之為敷繹之。荊門郡治火,持之倉卒指授中程,九淵器之。



 韓侂胄將用兵,持之憂時之懌,乃歷聘時賢,將有以告,見徐誼於九江,時議防江,持之請擇僚吏察地形,孰險而守,孰易而戰,孰隘而伏,毋專為江守。具言:「自古興事造業,非有學以輔之,往往皆以血氣盛衰為銳惰。故三國、兩晉諸賢,多以盛年成功名。公更天下事變多矣,未舉一事,而朝思夕惟,利害先入於中,愚恐其為之難也。」誼憮然。又
 之鄂謁薛叔似、項安世,之荊謁吳獵,爭欲留之,尋皆謝歸。著書十篇,名《戇說》。



 嘉定三年,試江西轉運司預選,常平使袁燮薦於朝,謂持之議論不為空言,緩急有可倚仗。不報。豫章建東湖書院,連帥以書幣強起持之長之。嘉定十六年,寧宗特詔持之秘書省讀書,固辭,不獲。既至,又詔以迪功郎入省,乞歸,不許。理宗即位,轉修職郎,差乾辦浙西安撫司,以疾請致仕,特命改通直郎。所著有《易提綱》、《諸經雜說》。



 徐鹿卿,字德夫,隆興豐城人。博通經史,以文學名於鄉,後進爭師宗之。嘉定十六年,廷試進士,有司第其對居二,詳定官以其直抑之,猶置第十。



 調南安軍學教授。張九成嘗以直道謫居,鹿卿摭其言行,刻諸學以訓。先是周惇頤、程顥與其弟頤皆講學是邦,鹿卿申其教,由是理義之學復明。立養士綱條,學田多在溪峒,異時征之無藝,農病之,鹿卿撫恤,無逋租者。其後盜作,環城屋皆毀,惟學宮免,曰:「是無撓我者。」



 闢福建安撫司干辦公事。會
 汀、邵寇作,鹿卿贊畫備御,動中機會。避寇者入城,多方振濟,全活甚眾。郡多火災,救護有方。會都城火,鹿卿應詔上封事,言積陰之極,其徵為火,指言惑嬖寵、溺燕私、用小人三事尤切。真德秀稱其氣平論正,有憂愛之誠心。改知尤溪縣。德秀守泉,闢宰南安,鹿卿以不便養辭。德秀曰:「道同志合,可以拯民,何憚不來?」鹿卿入白其母,欣然許之。既至,首罷科斂之無名者,明版籍,革預借,決壅滯,達冤抑,邑以大治。德秀尋帥閩,疏其政以勸列邑。
 歲饑,處之有法,富者樂分,民無死徙。最聞,令赴都堂審察。以母喪去。



 詔服闋赴樞密稟議,首言邊事、楮幣。主管官告院,乾辦諸司審計司。故相子以集英殿修撰食祠祿,又幫司農少卿米麥,鹿卿曰:「奈何為一人壞成法。」持不可。遷國子監主簿。入對,陳六事,曰:「洗凡陋以起事功,昭勸懲以收主柄,清班著以儲實才,重藩輔以蔽都邑,用閩、越舟師以防海,合東南全力以守江。」上皆嘉納。改樞密院編修官,權右司,贊畫二府,通而守法。會右史方
 大琮、編修劉克莊、正字王邁以言事黜,鹿卿贈以詩,言者並劾之,太學諸生作《四賢詩》。知建昌軍,未上,而崇教、龍會兩保與建黎原、鐵城之民修怨交兵,鹿卿馳書諭之,斂手聽命。既至,則寬賦斂,禁掊克。汰贓濫,抑強御,恤寡弱,黥黠吏,訓戍兵,創百丈砦,擇兵官,城屬縣,治行大孚,田里歌誦。



 督府橫取秋苗斛面,建昌為米五千斛。鹿卿爭之曰:「守可去,米不可得。」民恐失鹿卿,請輸之以共命。鹿卿曰:「民為守計則善矣。守獨不為民計乎?」卒爭以
 免。召赴行在,將行,盜發南豐,捕斬渠首二十人,餘不問。擢度支郎官兼右司。入對,極陳時敝。改侍右郎官兼敕令刪修官,兼右司。鹿卿又言當時並相之敝。宰相以甘言誘鹿卿,退語人曰:「是牢籠也,吾不能為宰相私人。」言者以他事詆鹿卿,主管雲臺觀。越月,起為江東轉運判官。歲大饑,人相食,留守別之傑諱不詰,鹿卿命掩捕食人者,尸諸市。又奏援真德秀為漕時撥錢以助振給,不報。遂出本司積米三千餘石減半賈以糶,及減抵當庫
 息,出緡錢萬有七千以予貧民,勸居民收字遺孩,日給錢米,所活數百人。宴集不用樂。



 會岳珂守當塗,制置茶鹽,自詭興利,橫斂百出,商旅不行,國計反屈於初。命鹿卿核之,吏爭竄匿。鹿卿寬其期限,躬自鉤考,盡得其實。珂闢置貪刻吏,開告訐以罔民,沒其財,民李士賢有稻二千石,囚之半歲。鹿卿悉縱舍而勸以其餘分,皆感泣奉命。珂罷,以鹿卿兼領太平,仍暫提舉茶鹽事。弛苛征,蠲米石、蕪湖兩務蘆稅。江東諸郡飛蝗蔽天,入當塗境,
 鹿卿露香默禱,忽飄風大起,蝗悉度淮。之傑密請移鹿卿浙東提點刑獄,加直秘閣兼提舉常平。鹿卿言罷浮鹽經界鹼地,先撤相家所築,就捕者自言:「我相府人。」鹿卿曰:「行法必自貴近始。」卒論如法。丞相史彌遠之弟通判溫州。利韓世忠家寶玩,籍之,鹿卿奏削其官。



 初,鹿卿檄衢州推官馮惟說決婺獄,惟說素廉平,至則辨曲直,出淹禁。大家不快其為,會鄉人居言路,乃屬劾惟說。州索印紙,惟說笑曰:「是猶可以仕乎?」自題詩印紙而去。衢
 州鄭逢辰以繆舉,鹿卿以委使不當,相繼自劾,且共和其詩。御史兼二人劾罷之。及知泉州,改贛州,皆辭。遷浙西提點刑獄、江淮都大坑冶,皆以病固辭,遂主管玉局觀。及召還,又辭,改直寶章閣知寧國府,提舉江東常平,又辭。



 淳祐三年,以右司召,猶辭。丞相杜範遺書曰:「直道不容,使人擊節。君不出,豈以馮惟說故耶?惟說行將有命矣。」鹿卿乃出。擢太府少卿兼右司。入對,請定國本、正紀綱、立規模,「時事多艱,人心易搖,無獨力任重之臣,無
 守節伏義之士,願蚤決大計」。上嘉納之。兼中書門下省檢正諸房公事,兼崇政殿說書。逾年,兼權吏部侍郎。時議使執政分治兵財,鹿卿執議不可。以疾丐祠,遷右文殿修撰、知平江府兼發運副使。力丐祠,上諭丞相挽留之。召權兵部侍郎,固辭,上令丞相以書招之,鹿卿至,又極言君子小人,切於當世之務。兼國子祭酒,權禮部侍郎,兼同修國史,兼實錄院同修撰,兼侍講,兼權給事中。鹿卿言「瑣闥之職無所不當問,比年命下而給舍不得
 知,請復舊制」。從之。



 上眷遇益篤而忌者浸多,有撰偽疏托鹿卿以傳播,歷詆宰相至百執事,鹿卿初不知也,遂力辨上前,因乞去,上曰:「去,則中奸人之計矣。」令臨安府根捕,事連勢要,獄不及竟。遷禮部侍郎。累疏告老,授寶章閣待制、知寧國府,而引年之疏五上,不允,提舉鴻禧觀,遂致仕,進華文閣待制。卒,遺表聞,贈四官。



 鹿卿居家孝友,喜怒不形,恩怨俱泯,宗族鄉黨,各得歡心。居官廉約清峻,豪發不妄取,一廬僅庇風雨。所著有《泉穀文集》、
 奏議、講義、《鹽楮議政稿》、《歷官對越集》,手編《漢唐文類》、《文苑菁華》,謚清正。



 趙逢龍,字應甫,慶元之鄞人。刻苦自修,為學淹博純實。登嘉定十六年進士第。授國子正、太學博士,歷知興國、信、衢、衡、袁五州,提舉廣東、湖南、福建常平。每至官,有司例設供張,悉命撤去,日具蔬飯,坐公署,事至即面問決遣。為政務寬恕,撫諭惻怛,一以天理民彞為言,民是以不忍欺。居官自常奉外,一介不取。民賦有逋負,悉為代
 輸。尤究心荒政,以羨餘為平糴本。遷將作監,拜宗正少卿兼侍講。凡道德性命之蘊,禮樂刑政之事,縷縷為上開陳。疏奏甚眾,稿悉焚棄。年八十有八終於家。



 逢龍家居講道,四方從游者皆為鉅公名士。丞相葉夢鼎出判慶元,修弟子禮,常謂師門庳陋,欲市其鄰居充拓之。逢龍曰:「鄰里粗安,一旦驚擾,彼雖勉從,我能無愧於心!」逢龍寡嗜欲,不好名,揚歷日久,泊然不知富貴之味。或問何以裕後,逢龍笑曰:「吾憂子孫學行不進,不患其饑寒
 也。」



 趙汝騰,字茂實,宗室子也。居福州。寶慶二年進士。歷官差主管禮、兵部架閣,遷籍田令,召試館職,授秘書省正字,升校書郎,尋升秘書郎兼史館校勘。輪對,言節用先自乘輿宮掖始。兼玉牒所檢討官,以直煥章閣知溫州,進直徽猷閣、江東提點刑獄,又進直寶文閣,差知婺州。召赴闕,遷起居舍人,兼權中書舍人,升起居郎,時暫兼權吏部侍郎,兼國史編修、實錄檢討,兼同修國史、實錄
 院同修撰,兼侍講,遷吏部侍郎兼侍講,權工部尚書兼權中書舍人,皆兼同修撰,以左司諫陳垓論罷。召為禮部尚書兼給事中,兼修國史、實錄院修撰。入奏,言:「前後奸諛之臣,傷善害賢,自取穹官要職,何益於陛下,而深損於聖德。興利之臣,移東就西,順適宮禁,自遂溪壑無厭之欲,何益於陛下,而深戕於國脈。則陛下私惠群小之心,可以息矣。」又言:「陛下有用君子之名,無用君子之實。」



 兼直學士院,拜翰林學士兼知制誥,兼侍讀。辭歸故
 里,累召,力辭,以龍圖閣學士知紹興府、浙東安撫使。召至闕,以端明殿學士提舉祐神觀,兼翰林學士承旨,知泉州、知州南外宗正事,復提舉祐神觀兼侍讀。兼翰林學士承旨。景定二年,卒,遺表上,特贈四官。



 孫夢觀,字守叔,慶元府慈溪人。寶慶二年進士。調桂陽軍教授、浙西提舉司干辦公事,差主管吏部架閣文字,為武學諭。輪對,言:「人主不容有所憚,尤不容有所玩,憚則有言而不能容,玩則雖容其言而不能用。」力請外,添
 差通判嚴州,主管崇道觀,召為武學博士、太常寺丞兼諸王宮大小學教授,大宗正丞兼屯田郎官、將作少監。知嘉興府,仍舊班兼右司郎官、將作監。轉對,極言:「風憲之地,未聞有十八疏攻一竦者。封駁之司,未聞有三舍人不肯草制者。道揆不明,法守滋亂,天下之權將有所寄,而倒持之患作。」當路者滋不悅。出知泉州兼提舉市舶,改知寧國府。蠲逋減賦,無算泛入者盡籍於公帑。戶部遣官督賦,急若星火,闔郡皇駭,莫知為計。夢觀曰:「吾
 寧委官以去,毋寧病民以留。」力丐祠,且將以府印牒所遣官,所遣官聞之夜遁。他日夢觀去寧國,人言之為之流涕。



 丞相董槐召還,帝問江東廉吏,槐首以夢觀對,帝說,乃遷司農少卿兼資善堂贊讀。輪對,謂:「今內外之臣,恃陛下以各遂其私,而陛下獨一無可恃,可為寒心!」次論:「郡國當為斯民計,朝廷當為郡國計。乞命大臣應自前主計之臣奪州縣之利而歸版曹者,復歸所屬,庶幾郡國蒙一分之寬,則斯民亦受一分之賜。」帝善其言。遷
 太府卿、宗正少卿,兼給事中、起居舍人、起居郎。八上章辭免,以監察御史吳燧論罷,直龍圖閣與祠,授秘閣修撰、江淮等路提點鑄錢司公事。甫至官,即復召為起居郎兼侍右侍郎、給事中兼贊讀,兼國子祭酒,權吏部侍郎。奏事抗論益切,以寵賂彰、仁賢逝、貨財偏聚為言,且謂「未易相之前,敝政固不少;既易相之後,敝政亦自若。」在廷之士皆危之。夢觀曰:「吾以一布衣蒙上恩至此,雖捐軀無以報,利鈍非所計也。」



 力求補外,以集英殿修撰
 知建寧府。蠲租稅,省刑罰,郡人徐清叟、蔡抗以為有古循吏風。民有夢從者甚郡,迎祠山神,出視之則夢觀也。俄而夢觀得疾,口授遺表,不忘規諫,遂卒。帝悼惜久之,賻銀帛三百。夢觀退然若不勝衣,然義所當為,奮往直前;其居敗屋數間,布衣蔬食,而重名節云。



 洪天錫,字君疇,泉州晉江人。寶慶二年進士。授廣州司法。長吏盛氣待僚屬,天錫糾正為多。丁內艱,免喪,調潮州司理。勢家奪民田,天賜言於守,還之。



 帥方大琮闢真
 州判官,留置幕府。改秩知古田縣。行鄉飲酒禮。邑劇,牒訴猥多,天錫剖決無留難。有倚王邸勢殺人者,誅之不少貸。調通判建寧府。大水,擅發常平倉振之。擢諸司糧料院,拜監察御史兼說書。累疏言:「天下之患三:宦官也,外戚也,小人也。」劾董宋臣、謝堂、厲文翁,理宗力護文翁,天錫又言:「不斥文翁,必為王府累。」上令吳燧宣諭再三,天錫力爭,謂:「貴幸作奸犯科,根柢蟠固,乃遲回護惜,不欲繩以法,勢焰愈張,紀綱愈壞,異時禍成,雖欲治之不
 可得矣。」上又出御札,俾天錫易疏,欲自戒飭之。天錫又言:「自古奸人雖憑怙,其心未嘗不畏人主之知,茍知之而止於戒飭,則憑怙愈張,反不若未知之為愈也。」章五上,出關待罪。詔二人已改命,宋臣續處之。天錫言:「臣留則宋臣去,宋臣留則臣當斥,願早賜裁斷。」越月,天雨土,天錫以其異為蒙,力言陰陽君子小人之所以辨,又言修內司之為民害者。



 蜀中地震,浙、閩大水,又言:「上下窮空,遠近怨疾,獨貴戚巨閹享富貴耳。舉天下窮且怨,陛
 下能獨與數十人者共天下乎?」會吳民仲大論等列訴宋臣奪其田,天錫下其事有司,而御前提舉所移文謂田屬御莊,不當白臺,儀鸞司亦牒常平。天錫謂:「御史所以雪冤,常平所以均役,若中貴人得以控之,則內外臺可廢,猶為國有紀綱乎?」乃申劾宋臣並盧允升而枚數其惡,上猶力護之。天錫又言:「修內司供繕修而已,比年動曰『御前』,奸贓之老吏,跡捕之兇渠,一竄名其間,則有司不得舉手,狡者獻謀,暴者助虐,其展轉受害者皆良
 民也。願毋使史臣書之曰:『內司之橫自今始。』」疏上至六七,最後請還御史印,謂:「明君當為後人除害,不當留患以遺後人。今朝廷輕給舍臺諫,輕百司庶府,而北司獨重,倉卒之際,臣實懼焉。」言雖不果行,然終宋世閹人不能竊弄主威者,皆天錫之力,而天錫亦自是去朝廷矣。改大理少卿,再遷太常,皆不拜。



 改廣東提點刑獄,五辭。明年,起知潭州,久之始至官。戢盜賊,尊先賢,逾年大治。直寶謨閣,遷廣東轉運判官,決疑獄,劾貪吏,治財賦,皆
 有法。召為秘書監兼侍講,以聵辭,升秘閣修撰、福建轉運副使,又辭。度宗即位,以侍御史兼侍讀召,累辭,不許,在道間,監察御史張桂劾罷之。乃疏所欲對病民五事:曰公田,曰關子,曰銀綱,曰鹽鈔,曰賦役。又言:「在廷無嚴憚之士,何以寢奸謀?遇事無敢諍之臣,何以臨大節?人物稀疏,精採銷耎,隱惰惜已者多,忘身徇國者少。」進工部侍郎兼直學士院,加顯文閣待制、湖南安撫使、知潭州,改潭州,皆力辭。



 又明年,改福建安撫使,力辭,不許。亭
 戶買鹽至破家隕身者,天錫首罷之,民作佛事以報。罷荔枝貢。召為刑部尚書,詔憲守之臣趣行無虛日,不起。久之,進顯文閣直學士,提舉太平興國宮,三降御札趣之,又力辭。逾年,進華文閣直學士,仍舊宮觀,尋致仕,加端明殿學士,轉一官。疾革,草遺表以規君相。上震悼,特贈正議大夫,謚文毅。



 天錫言動有準繩,居官清介,臨事是非不可回折。所著奏議、《經筵講義》,《進故事》、《通祀輯略》、《味言發墨》、《陽巖
 文集》。



 黃師雍,字子敬,福州人。少從黃斡學。入太學。寶慶二年,舉進士。詔為楚州官屬。出盜賊白刃之沖,不畏不懾。李全反狀已露,師雍密結忠義軍別部都統時青圖之,謀洩,全殺青,師雍不為動,全亦不加害。秩滿,朝議褒異,師雍恥出史彌遠門,不往見之。調婺州教授,學政一以呂祖謙為法。李完勉、趙必願、趙汝談皆薦之。



 師雍慕徐僑有清望,欲謁之,會其有召命,師雍曰:「今不可往也。」僑聞而賢之,至闕,以其學最聞,宗勉在政府,力言於丞相喬
 行簡,行簡已許以朝除。師雍以書見行簡,勸其歸老,行簡不悅,宗勉之請遂格。



 知遂之龍溪,轉運使王伯大上其邑最。行簡罷,宗勉與史嵩之入相,召師雍審察,將至而宗勉卒。嵩之延師雍,密示相親意,師雍不領;遷糧料院,又曰:「料院與相府密邇,所以相處。」師雍亦不領。嵩之獨相,權勢浸盛,上下懼禍,未有發其奸者。博士劉應起首疏論嵩之,帝感悟,思逐嵩之。師雍與應起相善,故嵩之疑師雍左右之,諷御史梅杞擊師雍,差知興化軍,旋
 奪之,改知邵武軍。及應起為監察御史,師雍遷宗正寺簿,尋亦拜監察御史。首疏削金淵秩,送外居住。再疏斥趙綸、項容孫、史肯之。嵩之終喪,正言李昴英、殿中侍御史章琰共疏乞竄斥之,師雍亦上疏論列,帝感悟,即其日詔勒令致仕。權直舍人院劉克莊封還詞頭,乞畀嵩之以貼職如宰臣去國故事,遂得守金紫光祿大夫、觀文殿學士致仕。議者曰:「大夫,官也。觀文,職也。元降御筆但云『守官』,無『本官職』之辭。觀文之命,自克莊啟之。朋邪
 顧望,不可赦。」師雍遂劾克莊臨事失身犯義,免所居官,琰亦繼劾克莊,師雍又乞籍嵩之家隸張叔儀,皆從之。



 未幾,昴英劾臨安尹趙與TP及執政,琰亦劾執政,帝怒昴英並及琰。鄭採乘間劾琰、昴英,又嗾同列再疏,以昴英屬某人,琰屬師雍。師雍毅然不從,獨擊葉閶乃與TP腹心。琰、昴英去國,採於是薦周坦、葉大有入臺,首劾程公許、江萬里,善類日危矣。未逾月,坦攻參政吳潛去,陳垓為監察御史,時採、與TP、坦、垓、大有合為一,師雍獨立。
 採惡之尤甚,思所以去師雍,未得,招四人共謀之。會大旱求言,應招者多指採、坦等為起災之由,牟子才、李伯玉、盧鉞語尤峻。坦等偽撰匿名書,誣三士,師雍榻前辨,謂:「匿名書條令所禁,非公論也,不知何為至前。」因發其偽撰之跡。適鉞疏譽師雍,採乃以鉞附師雍,帝不聽,擢師雍左司諫。



 未幾,採入政府,謝方叔、趙汝騰疏其奸,採遂罷去。師雍與丞相鄭清之故同舍,然以劾劉用行、魏峴皆清之親故,清之不樂。坦喜曰:「吾得所以去之矣。」遣
 其婦日造清之妻,譖曰:「彼去用行、峴,乃去丞相之漸也。」帝將以師雍為侍御史,清之曰:「如此,則臣不可留。」遷起居舍人兼侍講,即力丐去。清之猶冀師雍少貶,師雍曰:「吾欲為全人。」終不屈。數月,坦卒劾師雍及高斯得俱罷。久之,以直寶文閣奉祠,陳垓又嗾同列寢之。清之卒,起師雍為左史,既而改江西轉運使,遷禮部侍郎,命下而卒於江西官舍。



 師雍簡淡寡欲,靖厚有守,言若不出口,而於邪正之辨甚明,視外物輕甚,故博採公論,當官而
 行,愛護名節,無愧師友云。



 徐元傑,字仁伯,信州上饒人。幼穎悟,誦書日數千言,每冥思精索。聞陳文蔚講書鉛山,實朱熹門人,往師之。後師事真德秀。紹定五年,進士及第。簽書鎮東軍節判官廳公事。



 嘉熙二年,召為秘書省正字,遷校書郎。奏否泰、剝復之理,因及右轄久虛,非骨鯁耆艾,身足負荷斯世者,不可輕畀。又言皇子竑當置後及蚤立太子,乞蚤定大計。時諫官蔣峴方力排竑置後之說,遂力請外,不
 許,即謁告歸,丐祠,章十二上。三年,遷著作佐郎兼兵部郎官,以疾辭。差知安吉州,辭。召赴行在奏事,辭益堅。



 淳祐元年,差知南劍州。會峽陽寇作,擒渠魁八人斬之。餘釋不問。父老或相語曰:「侯不來,我輩魚肉矣。」郡有延平書院,率郡博士會諸生親為講說。民訟,率呼至以理化誨,多感悅而去。輸苗聽其自概,闔郡德之。丁母憂去官,眾遮道跪留。既免喪,授侍左郎官。言敵國外患,乞以宗社為心。言錢塘駐蹕,驕奢莫尚,宜抑文尚質。兼崇政殿
 說書,每入講,必先期齋戒。嘗進仁宗詔內降指揮許執奏及臺諫察舉故事為戒,語多切宮壺。拜將作監,進楊雄《大匠箴》,陳古節儉。時天久不雨,轉對,極論《洪範》天人感應之理及古今遇災修省之實,辭益忠懇。



 丞相史嵩之丁父憂,有詔起復,中外莫敢言,惟學校叩閽力爭。元傑時適輪對,言:「臣前日晉侍經筵,親承聖問以大臣史嵩之起復,臣奏陛下出命太輕,人言不可沮抑。陛下自盡陛下之禮,大臣自盡大臣之禮,玉音賜俞,臣又何所
 容喙。今觀學校之書,使人感嘆。且大臣讀聖賢之書,畏天命,畏人言。家庭之變,哀戚終事,禮制有常。臣竊料其何至於忽送死之大事,輕出以犯清議哉!前日昕庭出命之易,士論所以凜凜者,實以陛下為四海綱常之主,大臣身任道揆,扶翊綱常者也。自聞大臣有起復之命,雖未知其避就若何,凡有父母之心者莫不失聲涕零,是果何為而然?人心天理,誰實無之,興言及此,非可使聞於鄰國也。陛下烏得而不悔悟,大臣烏得而不堅忍?
 臣懇懇納忠,何敢詆訐,特為陛下愛惜民彞,為大臣愛惜名節而已。」疏出,朝野傳誦,帝亦察其忠亮,每從容訪天下事,以筵益申前議。未幾,夜降御筆黜四不才臺諫,起復之命遂寢。



 元老舊德次第收召,元傑亦兼右司郎官,拜太常少卿,兼給事中、國子祭酒,權中書舍人。杜範入相,復延議軍國事。為書無慮數十,所言皆朝廷大政,邊鄙遠慮。每裁書至宗社隱憂處,輒閣筆揮涕,書就隨削稿,雖子弟無有知者。六月朔,輪當侍立,以暴疾謁告。
 特拜工部侍郎,隨乞納,詔轉一官致仕。夜四鼓。遂卒。



 先,元傑未死之一日,方謁左丞相範鐘歸,又折簡察院劉應起,將以冀日奏事。是夕,俄熱大作,詰朝不能造朝,夜煩愈甚,指爪忽裂,以死。朝紳及三學諸生往吊,相顧駭泣。訃聞,帝震悼曰:「徐元傑前日方侍立,不聞有疾,何死之遽耶?」亟遣中使問狀,賻贈銀絹二百計。已而太學諸生伏闕訴其為中毒,且曰:「昔小人有傾君子者,不過使之自死於蠻煙瘴雨之鄉,今蠻煙瘴雨不在領海,而
 在陛下之朝廷。望奮發睿斷,大明典刑。」於是三學諸生相繼叩閽訟冤,臺諫交疏論奏,監學官亦合辭聞於朝。二子直諒、直方乞以恤典充賞格。有旨付臨安府逮醫者孫志寧及常所給使鞫治。既又改理寺,詔殿中侍御史鄭採董之,且募告者賞緡錢十萬、官初品。大理寺正黃濤謂伏暑證,二子乞斬濤謝先臣。然獄迄無成,海內人士傷之,帝悼念不已,賜官田五百畝、緡錢五千給其家。賜謚忠愍。



 孫子秀,字元實,越州餘姚人。紹定五年進士。調吳縣主簿。有妖人稱「水仙太保」,郡守王遂將使治之,莫敢行,子秀奮然請往,焚其廬,碎其像,沈其人於太湖,曰:「實汝水仙之名矣。」妖遂絕。日詣學宮與諸生討論義理。闢淮東總領所中酒庫,檄督宜興縣圍田租。既還,白水災,總領恚曰:「軍餉所關,而敢若此,獨不為身計乎?」子秀曰:「何敢為身計,寧罪去爾。」力爭之,遂免。



 調滁州教授,至官,改知金壇縣。嚴保伍,厘經界,結義役,一切與民休息。訟者使
 齎牒自詣里正,並鄰證來然後行,不實者往往自匿其牒,惟豪黠者有犯,則痛繩不少貨。淮民流入以萬計,振給撫恤,樹廬舍,括田使耕,拔其能者分治之。崇學校,明教化,行鄉飲酒禮。訪國初茅山書院故址,新之,以待遠方游學之士。



 通判慶元府,主管浙東鹽事。先是,諸場鹽百袋附五袋,名「五厘鹽」,未幾,提舉官以為正數,民困甚,子秀奏蠲之。闢乾辦行在諸司糧料院。衢州冠作,水冒城郭,朝廷擇守,屬子秀行。子秀謂捕賊之責,雖在有司,
 亦必習土俗之人,乃能翦其憑依,裁其奔突。乃立保伍,選用土豪,首旌常山縣令陳謙亨、寓士周還淳等捍禦之勞,且表於朝,乞加優賞,人心由是競勸。未幾,盜復起江山、玉山間,甫七日,而眾禽四十八人以來。終子秀之任,賊不復動,水潦所及,則為治橋梁,修堰閘,補城壁,浚水原,助葺民廬,振以錢米,招通鄰糴。奏蠲秋苗萬五千石有奇,盡代納其夏稅,並除公私一切之負;坍溪沙壅之田,請於朝,永蠲其稅,民用復蘇。



 南渡後,孔子裔孫寓
 衢州,詔權以衢學奉祀,因循逾年,無專饗之廟。子秀撤廢佛寺,奏立家廟如闕里。既成,行釋菜禮。以政最遷太常丞,以言罷。未幾,遷大宗正丞,遷金部郎官。金部舊責州郡以必不可辨之泛數,吏顛倒為奸欺。子秀日夜討論,給冊轉遞以均其輸,人人如債切身,不遣一字而輸足。遷將作監、淮東總領,辭。改知寧國府,辭。為左司兼右司,再兼金部。與丞相丁大全議不合,去國。差知吉州,尋鐫罷。



 時嬖幸朱熠凡三劾子秀。開慶元年,為浙西提
 舉常平。先是,大全以私人為之,盡奪亭民鹽本錢,充獻羨之數;不足,則估籍虛攤。一路騷動,亭民多流亡。子秀還前政鹽本錢五十餘萬貫,奏省華亭茶鹽分司官,定衡量之非法多取者,於是流徙復業。徙浙西提點刑獄兼知常州。淮兵數百人浮寓貢院,給餉不時,死者相繼,子秀請於朝,創名忠衛軍,置砦以居,截撥上供贍之。盜劫吳大椿,前使者諱其事,誣大椿與兄子焴爭財,自劫其家,追毀大椿官,編置千里外,徙黥其臧獲。子秀廉得實,
 乃悉平反之。尋以兼郡則行部非便,得請專臬事。擊貪舉廉,風採凜然,犴獄為清。



 進大理少卿,直華文閣、浙東提點刑獄兼知婺州。婺多勢家,有田連阡陌而無賦稅者,子秀悉核其田,書諸牘,勢家以為厲己,嗾言者罷之。尋遷湖南轉運副使,以迎養非便辭,移浙西提點刑獄。子秀冒暑周行八郡三十九縣,獄為之清。安吉州有婦人訴人殺其夫與二僕,郡守捐賞萬緡,逮系考掠十餘人,終莫得其實。子秀密訪之,乃婦人賂宗室子殺其夫,
 僕救之,並殺以滅口。一問即伏誅,又釋偽會之連逮者,遠近稱為神明。



 初,獄訟之滯,皆由期限之不應。使者下車,或親書戒州縣勿違,而違如故,則怒之。怒之,改匣,又違則又重怒之,至再三。而專卒四出,巡尉等司繳限抱匣費不貲,則其勢必違。子秀與州縣約,到限者徑詣庭下,吏不得要索,亦無違者。其後創循環總匣屬各州主管官,凡管內諸司報應皆並入匣,一日一遣,公移則又總實於匣以往。於是事無小大,纖悉畢具,而風聞者反
 謂專卒凌州縣,劾罷之,子秀笑而已。移江東提點刑獄。度宗即位,進太常少卿兼右司,尋兼知臨安府,以言罷。起知婺州,卒。



 子秀少從上虞劉漢弼游,磊落英發,抵掌極談,神採飛動。與人交久而益親,死生患難,營救不遺力。聞一善則手錄之。



 李伯玉,字純甫,饒州餘干人。端平二年,進士第二。初名誠,以犯理宗潛諱更今名。授觀察推官、太學正兼莊文府教授、太學博士。召試館職,歷詆貴戚大臣,直聲暴起。
 改校書郎,奏言:「臺評迎合上意,論罷尤焴、楊棟、盧鉞三人,忠邪不辨,乞同罷。」帝不允。監察御史陳垓連劾罷之。



 奉雲臺祠,差知南康軍,遷著作佐郎兼沂靖惠王府教授,兼考功郎官,兼尚書右司員外郎。引故事彈臺臣蕭隸來,遷著作郎。帝怒,降兩官罷敘。復知邵武軍,改湖北提點刑獄,移福建,遷尚右郎官。侍御史何夢然論伯玉乃吳潛之死黨,奉祀,遷福建提舉常平、淮西轉運判官。召赴經筵,遷考功郎兼太子侍讀,拜太府少卿、秘書少
 監、起居郎、工部侍郎。



 度宗即位,兼侍講,權禮部侍郎,升兼同修國史、實錄院同修撰。賈似道嘗集百官議事,忽厲聲曰:「諸君非似道拔擢,安得至此!」眾默然莫敢應者,伯玉答曰:「伯玉殿試第二名,平章不拔擢,伯玉地步亦可以至此。」似道雖改容而有怒色。既退,即治歸。以顯文閣待制知隆興府,右正言黃萬石論罷。召入覲,擢權禮部尚書兼侍讀。似道益專國柄,帝以伯玉舊學,進之臥內,相對泣下,欲用以參大政,似道益忌之,而伯玉尋病
 卒。



 伯玉嘗請罷童子科,以為非所以成人材,厚風俗。趙汝騰嘗薦八士,各有品目,於伯玉曰「銅山鐵壁」。立朝風節,大較似之。所著有《斛峰集》。



 論曰:陸持之學足以承其家,而不幸蚤喪,徐鹿卿論議明達,克施有政,趙逢龍之清操,汝騰之不撓,孫夢觀之平直,洪天錫、黃師雍、徐元傑、李伯玉皆悉心直言,不避權勢,孫子秀政績著見,皆當時之傑出雲。



\end{pinyinscope}