\article{列傳第一百三十七}

\begin{pinyinscope}

 衛膚敏劉玨胡舜陟沉晦劉一止弟寧止胡交修綦崇禮



 衛膚敏,字商彥,華亭人。以上舍生登宣和元年進士第,授文林郎、南京宗子博士,尋改教授。六年,召對,改宣教
 郎、秘書省校書郎,命假給事中賀金主生辰。膚敏奏曰:「彼生辰後天寧節五日,金人未聞入賀,而反先之以失國體,萬一金使不來,為朝廷羞。請至燕山候之,彼若不來,則以幣置境上而已。」帝可其奏。既至燕,金賀使果不至,遂置幣而返。七年,復假給事中以行,及慶源府,逢許亢宗還,語金國事,曰:「彼且大入,其勢不可往。」膚敏至燕,報愈急,眾懼不敢進,膚敏叱曰:「吾將君命以行,其可止乎?」即至金國,知其兵已舉,殊不為屈。及將還,金人所答
 國書,欲以押字代璽,膚敏力爭曰:「押字豈所以交鄰國。」論難往復,卒易以璽。及受書,欲令雙跪,膚敏曰:「雙跪乃北朝禮,安可令南朝人行之哉!」爭辨逾時,卒單跪以受。金人積不說,中道羈留且半年。



 至涿州新城,與斡離不遇,遣人約相見,拒之不可,遂語之曰:「必欲相見,其禮當如何?」曰:「有例。」膚敏笑曰:「例謂趨伏羅拜,此禮焉可用?北朝止一君耳,皇子郎君雖貴,人臣也,一介之使雖賤,亦人臣也。兩國之臣相見,而用君臣之禮,是北朝一國有
 二君也。」金人氣折,始曰:「唯所欲。」膚敏長揖而入。既坐,金人出誓書示之,膚敏卻不視,曰:「遠使久不聞朝廷事,此書真偽不可知。」因論用兵事,又以語折之,幾復為所留。



 靖康初,始還,進三官,遷吏部員外郎。會高麗遣使來賀,命假太常少卿往接之。朝論欲改稱宣問使,膚敏曰:「國家厚遇高麗久矣,今邊事方作,不可遽削其禮,失遠人心,願姑仍舊。」乃復稱接伴使。既至明州,會京師多難,乃便宜稱詔厚賜使者,遣還。



 建炎元年,復命,自劾矯制之
 罪,高宗嘉賞。遷衛尉少卿。建議「兩河諸郡宜降蠟書,許以世襲,使各堅守。陜西、山東、淮南諸路,並令增陴浚隍,徙民入城為清野計。命大臣留守汴京,車駕早幸江寧。」帝頗納之。



 遷起居舍人,言:「前日金人憑陵,都邑失守,朝臣欲存趙氏者不過一二人而已,其它皆屈節受辱,不以為恥,甚者為敵人斂金帛,索妃嬪,無所不至,求其能詐楚如紀信者無有也。及金人偽立叛臣,僭竊位號,在廷之臣逃避不從及約寇退歸位趙氏者,不過一二
 人而已。其它皆委質求榮,不以為愧,甚者為叛臣稱功德,說符命,主推戴之議,草勸進之文,無所不為,求其擊朱泚如段秀實者無有也。今陛下踐祚之初,茍無典刑,何以立國?凡前日屈節敵人,委質偽命者,宜差第其罪,大則族,次則誅,又其次竄殛,下則斥之遠方,終身不齒,豈可猶畀祠祿,使塵班列哉?」又言:「今二帝北遷,寰宇痛心,願陛下愈自貶損,不忘服雪,卑宮室,菲飲食,惡衣服,減嬪御,斥聲樂,以至歲時上壽,春秋錫宴,一切罷之,雖饗
 郊廟亦不用樂。必俟兩宮還闕,然後復常,庶幾精誠昭格天地,感動人心。」拜右諫議大夫兼侍讀,言:「行在頗興土木之役,非所以示四方,乞罷築承慶院、升暘宮。」又奏:「凡黜陟自中出者,皆由三省乃得奉行,或戾祖宗成憲者,皆許執奏。」時內侍李志道以赦恩復保慶軍承宣使,添差入內都知,膚敏極論罷之。初,欽宗內侍昭慶軍承宣使容機,圍城中時乞致仕,高宗即位,命起之。膚敏言:「自古帝王未有求閹寺於閑退而用者。」遂寢。後父邢煥
 除徽猷閣待制,太后兄子孟忠厚顯謨閣直學士。膚敏言:「非祖宗法。」煥尋換武職,忠厚自若。



 俄遷膚敏中書舍人,膚敏懇奏曰:「昔司馬光論張方平不當參知政事,自御史中丞遷翰林學士。光言:『以臣為是,則方平當罷;以臣為非,則臣當貶。今兩無所問而遷臣,臣所未諭。』臣雖不肖,願附於司馬光。」又言:「事母後莫若孝,待戚屬莫若恩,勸臣下莫若賞,今陛下順太母以非法非所謂孝,處忠厚以非分非所謂恩,不用臣言而遷其官非所謂賞,
 一舉而三失矣。」帝命宰相諭膚敏曰:「朝廷以次遷官,非因論事也。」膚敏猶不拜,居家逾月,及忠厚改承宣使,詔後族勿除從官,膚敏始拜命。又言:「中書根本之地,舍人所掌,不特演綸而已。」凡命令不合公議者,率封還之。



 會膚敏知貢舉,有進士何烈對省試策,謬稱「臣」,諫官李處遁乞正考官鹵莽之罪,以集英殿修撰提舉洞霄宮。或謂膚敏在後省論事,為黃潛善、汪伯彥所惡,故因事斥之。



 三年春,召赴行在。時帝次平江。膚敏入見,言及時事
 泣下,帝亦泣曰:「卿今宜知無不言,有請不以時對。」膚敏謝曰:「臣頃嘗三為陛下言,揚州非駐蹕之地,乞早幸江寧。今錢塘亦非帝王之都,宜須事定亟還金陵。」因陳所以守長江之策,帝善其言。翌日,再對,歸得疾,然猶力疾扈蹕至臨安。俄除刑部侍郎,未拜,謁告歸華亭就醫,許之,遷禮部侍郎。



 初,膚敏久疾臥舟中,不能朝,時苗、劉之變,帝未反正,宰相朱勝非言於隆祐太后,以「膚敏稱疾坐觀成敗,無人臣節」。及卒,始明其非偽云。年四十九,特
 贈大中大夫。子仲英、仲傑、仲循。



 劉玨,字希範,湖州長興人。登崇寧五年進士第。初游太學,以書遺中書舍人鄒浩曰:「公始為博士論取士之失,免所居官,在諫省斥宮掖之非,遠遷嶺表,豈逆計禍福,邀後日報哉,固欲蹈古人行也。今庶政豈盡修明,百官豈盡忠實,從臣繼去,豈盡非才,言官屢逐,豈盡有罪!信任逾曩昔而拱默不言,天下之士竊有疑焉,願有以慰塞群望。」浩得書愧謝之。宣和四年,擢監察御史,坐言事
 知舒州,留為尚書主客員外郎。



 靖康初,議皇帝朝謁上皇儀,欲以家人禮見於內庭,玨請皇帝設大小次,俟上皇御坐,宰臣導皇帝升自東階,拜於殿上,則有君之尊,有父之敬。又謂:「君於大臣或賜劍履上殿,或許子孫扶掖。皇帝朝謁,宜令環衛士卒侍立於殿西,宰執、三衙、侍從等官扶侍於殿上。如請帝坐,即宰執等退立西隅。」遷太常少卿。討論皇帝受冊寶故事,玨言:「唐太宗、明皇皆親受父命,未嘗再行冊禮,肅宗即位於靈武,故明皇遣
 韋見素就冊之,宣政授傳國璽,群臣上尊號,至德宗踵行之,後世以為非。」議遂寢。



 除中書舍人。陳十開端之戒曰:「陛下即位罷御筆,止營繕,登俊乂,詘虛誕,戢內侍之權,開言者之路,命令既當,未嘗數改,任用既公,率皆稱職,賞必視功,政必核實,此天下所以指日而徯太平也。比者內降數出,三省罕有可否,此御筆之開端也。教子弟既有其所,又徹而新之,長入祗候之班,勢若可緩,亟而成之,此營繕之開端也。河陽付之庸才,涇原委之貪
 吏,此任用失當之開端也。花石等濫賞,既治復止,馬忠統兵,累行累召,此命令數易之開端也。三省、密院議論各有所見,啟擬各舉所知,持不同不比之說,忘同寅協恭之議,此大臣不和之開端也。內路之帥擅作聖旨指揮,行郡之守稱為外任監當,此臣下誕謾之開端也。董局務者廣闢官屬,侍帷幄者分爭殿廬,此內侍恣橫之開端也。兩省繳奏多命以次行下,或戒以不得再繳,臺諫言事失當,率責為遠小監當,此言路壅塞之開端也。
 恤民之詔累下,未可行者多,是為空文無實德,此政事失信之開端也。隨龍第賞,冠帶之工亦推恩,金兵扣闕,禮房之吏亦進秩,此爵賞僭濫之開端也。是十者雖未若前日之甚,其端已見,杜而止之,可以馴致治平,因而循之,雖有智者不能善其後矣。」



 詹度都堂稟議,中書舍人安扶持不可,改命玨書行,玨言:「伐燕之役,度以書贊童貫大舉,去秋蔡靖屢以金人點集為言,度獨謂不應有此,遂不設備,請竄度嶺表。」詔予宮祠。李綱以觀文殿
 學士知揚州,安扶又持不可,玨言:「韓琦好水之敗,韓絳西州之敗,皆不免黜責。綱勇於報國,銳於用兵,聽用不審,數有敗衄,宜降黜以示懲戒。」綱改宮祠。吏部侍郎馮澥言玨持兩端,為綱游說,提舉亳州明道宮。



 建炎元年,復召為中書舍人,至泗州,上書言:「金人尚有屯河北者,萬一猖獗而南,六飛豈能無警,乞早賜行幸。西兵驍勇,宜留以為衛。西京舟船。恐金人藉以為用,並令東下。」時李綱已議營南陽,玨未知也。既至,極言南陽兵弱財單,
 乘輿無所取給,乞駐蹕金陵以待敵。汪伯彥、黃潛善皆主幸東南,帝遂如揚州。潛善兄潛厚除戶部尚書,玨言兄弟不可同居一省,帝遣張愨諭旨,玨論如初。詔潛厚提舉醴泉觀。



 遷給事中,論內降、營繕二事曰:「陛下以前朝房院而建承慶院,議者以為營造浸廣,以隆祐太后時有御筆,議者以為內降數出。蓋除授不歸中書,工役領之內侍,此人言所以籍籍也。營繕悉歸有司,中旨皆許執奏,則眾論息矣。」孟忠厚除顯謨閣直學士,邢煥徽
 猷閣待制,玨封還,言舊制外戚未有為兩禁官者,詔煥換武階。帝曰:「忠厚乃隆祐太后族,宜體朕優奉太后之意。」玨持益堅,忠厚尋亦換武階。



 遷吏部侍郎,同修國史,言:「淮甸備敵,兵食為先,今以降卒為見兵,以糴本為見糧,無一可恃,維揚城池未修,軍旅多闕,卒有不虞,何以待之?」已而金人果乘虛大入,帝亟如臨安,以玨為龍圖閣直學士、知宣州。俄復為吏部侍郎。



 以久雨詔求言,玨疏論消天變、收人心數事,詞極激切,並陳荊、陜、江、淮守
 禦之略:「願申詔大臣,悉屏細務,唯謀守御。自京及荊、淮之郡,置大帥,屯勁兵。命沿江之守,各上措畫之方,明斥堠,設險阻,節大府之出,廣大農之入,檢察戰艦而習之,則守禦詳盡,人心安,天意回,大業昌矣。」遷吏部尚書。



 隆祐太后奉神主如江西,詔玨為端明殿學士、權同知三省樞密院事從行。時詔元祐黨籍及上書廢錮人,追復故官,錄用子孫,施行未盡者,玨悉奏行之。又言常安民、張克公嘗論蔡京罪,乞厚加恩。至洪州,疏言修治巡幸
 道路之役,略曰:「陛下遭時艱難,躬履儉約,前冬幸淮甸,供帳弊舊,道路險狹,未嘗介意。今聞衢、信以來,除治道路,科率民丁,急如星火,廣市羊豕,備造服用,使農夫不得獲,齊民不得休,非陛下儉以避難之意也。乞降詔悉罷。」金人攻吉州,分兵追太后,舟至太和縣,衛兵皆潰,玨奉太后退保虔州。監察御史張延壽論玨罪,玨亦上書自劾,逾嶺俟命,落職,提舉江州太平觀。延壽論不已,責授秘書少監,貶衡州。紹興元年,許自便。明年,以朝散大
 夫分司西京。卒於梧州,年五十五。官其二子。八年,追復龍圖閣學士。有《吳興集》二十卷、《集議》五卷、《兩漢蒙求》十卷。



 胡舜陟,字汝明,徽州績溪人。登大觀三年進士第,歷州縣官,為監察御史。奏:「御史以言為職,故自唐至本朝皆論時事,擊官邪,與殿中侍御史同。崇寧間,大臣欲便己,遂變祖宗成憲,南臺御史始有不言事者。多事之時,以開言路為急。乞下本臺,增入監察御史言事之文,以復
 祖宗之制。」以內艱去。



 服闋,再為監察御史。奏:「河北金兵已遁,備御尤不可不講。」欽宗即位,又言:「今結成邊患,幾傾社稷,自歸明官趙良嗣始,請戮之以快天下。」遂誅良嗣。又奏:「今邊境備御之計,兵可練,粟可積,獨將為難得,請詔內外之臣,並舉文武官才堪將帥者。」又奏:「上殿班先臺後諫,祖宗法也,今臺臣在諫臣下,乞今後臺諫同日上殿,以臺諫雜壓為先後。」



 遷侍御中。奏:「向者晁說之乞皇太子講《孝經》,讀《論語》,間日讀《爾雅》而廢《孟子》。夫孔
 子之後深知聖人之道者,孟子而已。願詔東宮官遵舊制,先讀《論語》,次讀《孟子》。」又奏:「涪陵譙定受《易》於郭雍,究極像數,逆知人事,洞曉諸葛亮八陣法,宜厚禮招之。」



 高宗即位,舜陟論宰相李綱之罪,帝不聽。言者論其嘗事偽廷,除集英殿修撰、知廬州。時淮西盜賊充斥,廬人震恐,日具舟楫為南渡計。舜陟至,修城治戰具,人心始安。



 冀州雲騎卒孫琪聚兵為盜,號「一海蝦」,至廬,舜陟乘城拒守。琪邀資糧,舜陟不與,其眾請以粟遺之,舜陟曰:「吾非
 有所愛,顧賊心無厭,與之則示弱,彼無能為也。」乃時出兵擊其抄掠者,琪宵遁,舜陟伏兵邀擊,得其輜重而歸。



 濟南僧劉文舜聚黨萬餘,保舒州投子山縱剽,舜陟遣介使招降之。時丁進、李勝合兵為盜蘄、壽間,舜陟遣文舜破之。



 張遇自濠州奄至梁縣,舜陟使毀竹里橋,伏兵河西,伺其半渡擊敗之。又請以身守江北,以護行宮。帝壯其言,擢徽猷閣待制,充淮西制置使。範瓊自壽春渡淮,貽書責贍軍錢帛,舜陟諭以逆順,瓊乃去。



 自軍興後,
 淮西八郡,群盜攻蹂無全城,舜陟守廬二年,按堵如故,以徽猷閣待制知建康府,充沿江都制置使。逾年,改知臨安府,復為徽猷閣待制,充京畿數路宣撫使。尋罷,遷廬、壽鎮撫使,改淮西安撫使。至廬州,潰兵王全與其徒來降,舜陟散財發粟,流民漸歸。改知靜江府,詔措置市戰馬。御史中丞常同奏舜陟兇暴傾險,罷之。



 後十八年,復為廣西經略。以知邕州俞儋有臟,為運副呂源所按,事連舜陟,提舉太平觀。先是,舜陟與源有隙,舜陟因討
 郴賊,劾源沮軍事,源以書抵秦檜,訟舜陟受金盜馬,非訕朝政。檜素惡舜陟,入其說,奏遣大理寺官袁柟、燕仰之往推劾,居兩旬,辭不服,死獄中。



 舜陟有惠愛,邦人聞其死,為之哭。妻江氏訴於朝,詔通判德慶府洪元英究實。元英言:「舜陟受金盜馬,事涉曖昧,其得人心,雖古循吏無以過。」帝謂檜曰:「舜陟從官,又罪不至死,勘官不可不懲。」遂送柟、仰之吏部。



 沉晦,字符用,錢塘人,翰林學士沈遘孫。宣和間進士廷
 對第一,除校書郎,遷著作佐郎。金人攻汴京,借給事中從肅王樞出質斡離不軍。金人再攻也,與之俱南。京城陷,邦昌偽立,請金人歸馮澥等,晦因得還,真為給事中。



 高宗即位,言者論晦雖使金艱苦,而封駁之職不可以賞勞,除集英殿修撰、知信州。帝如揚州,將召為中書舍人,侍御史張守論晦為布衣時事,帝曰:「頃在金營見其慷慨,士人細行,豈足為終身累邪?」不果召。知明州,移處州。



 帝如會稽,移守婺州。賊成皋入寇,晦用教授孫邦策,
 率民兵數百出城與戰,大敗,晦欲斬邦,已而釋之。時浙東防遏使傅崧卿在城中,單騎往說皋,皋遂降。進徽猷閣待制。以言者論晦妄用便宜指揮行事,降集英殿修撰、提舉臨安府洞霄宮。尋復徽猷閣待制、知宣州,移知建康府。甫逾月,以御史常同論罷。



 紹興四年,起知鎮江府、兩浙西路安撫使,過行在面對,言:「藩帥之兵可用。今沿江千餘里,若令鎮江、建康、太平、池、鄂五郡各有兵一二萬,以本郡財賦易官田給之,敵至,五郡以舟師守
 江,步兵守隘,彼難自渡。假使參渡,五郡合擊,敵雖善戰,不能一日破諸城也。若圍五郡,則兵分勢弱,或以偏師綴我大軍南侵,則五郡尾而邀之,敵安敢遠去。此制稍定,三年後移江北,糧餉、器械悉自隨。」又自乞「分兵二千及召募敢戰士三千,參用昭義步兵法,期年後,京口便成強藩」。時方以韓世忠屯軍鎮江,不果用。



 劉麟入寇,世忠拒於揚州,晦乞促張俊兵為世忠援。趙鼎稱晦議論激昂,帝曰:「晦誠可嘉,然朕知其人言甚壯,膽志頗怯,更
 觀臨事,能副所言與否?」然晦不為世忠所樂,尋提舉臨安府洞霄宮,起為廣西經略兼知靜江府。



 先是,南州蠻酋莫公晟歸朝,歲久,用為本路鈐轄羈縻之,後遁去,旁結諸峒蠻,歲出為邊患。晦選老將羅統戍邊,招誘諸酋,喻以威信,皆詣府請降,晦犒遺之,結誓而去。自是公晟孤立,不復犯邊。晦在郡,歲買馬三千匹,繼者皆不能及。進徽猷閣直學士,召赴行在,除知衢州,改潭州,提舉太平興國宮,卒。



 晦膽氣過人,不能盡循法度,貧時尤甚,故
 累致人言。然其當官才具,亦不可掩雲。



 劉一止,字行簡,湖州歸安人。七歲能屬文,試太學,有司欲舉八行,一止曰:「行者士之常。」不就。登進士第,為越州教授。參知政事李邴薦為詳定一司敕令所刪定官。



 紹興初,召試館職,其略曰:「事不克濟者,患在不為,不患其難,聖人不畏多難,以因難而圖事耳。如其不為,俟天命自回,人事自正,敵國自屈,盜賊自平,有是哉?」高宗稱善,且諭近臣以所言剴切知治道,欲驟用,執政不樂,除秘
 書省校書郎。考兩浙類試,以科舉方變,欲得通時務者,同列皆患無其人,一止出一卷曰:「是宜為首。」啟號乃張九成也,眾皆厭服。



 遷監察御史。上疏謂:「天下之治,眾君子成之而不足,一小人敗之而有餘,君子雖眾道則孤,小人雖寡勢易蔓,不加察,則小人伺隙而入以敗政矣。」又言:「陛下憫宿蠹未除,頹綱未振,民困財竭,故置司講究,然未聞有所施行,得無有以疑似之說欺陛下,曰『如此將失人心』。夫所謂失人心者,必刑政之苛,賦役之多,
 好惡之不公,賞罰之不明;若皆無是,則所失者小人之心耳,何病焉。」



 時庶事草創,有司以吏所省記為法,吏並緣為奸,一止曰:「法令具在,吏猶得舞文,矧一切聽其省記,所欲與則陳與例,欲奪則陳奪例,與奪在其牙頰,患可勝言哉!請以省記之文刊定頒行,庶幾絕奸吏弄法受賕之弊。」從之。逾年而書成。



 秦檜請置修政局,一止言:「宣王內修政事,修其外攘之政而已。今之所修,特簿書獄訟,官吏遷降,土木營建之務,未見所當急也。」又謂:「人
 才進用太遽,仕者或不由銓選,朝士入而不出,外官雖有異能,不見召用,非軍事而起復,皆幸門不塞之故。請選近臣曉財利者,仿劉晏法,瀕江置司以制國用,鄉村置義倉以備水旱,增重監司之選。」後多採用其言。



 遷起居郎。奏事,帝迎語曰:「朕親擢也,繇六察遷二史,祖宗時有幾?」一止謝:「先朝惟張澄、李梲耳。」因極陳堂吏宦官之蠹,執政植私黨,無憂國心。翌日罷,主管臺州崇道觀。



 召為祠部郎、知袁州,改浙東路提點刑獄,為秘書少監,復
 除起居郎,擢中書舍人兼侍講。莫將賜出身除起居郎,一止奏:「將以上書助和議,驟自太府丞綴從班,前此未有,臣乃與將同命,願並臣罷之。」不報。



 遷給事中。徐偉達者,嘗事張邦昌為郎,得知池州,一止言:「偉達既仕偽廷,今付以郡,無以示天下。」孟忠厚乞試郡,一止言:「後族業文如忠厚雖可為郡,他日有援例者,何以卻之?」汪伯彥知宣州入覲,詔以元帥府舊人,特依見任執政給奉,一止言:「伯彥誤國之罪,天下共知,以郡守而例執政,殆與
 異時非待制而視待制,非兩府而視兩府者類矣。」帝皆為罷之,於凡貴近之請,雖小事亦論執不置。御史中丞廖剛謂其僚曰:「臺當有言者,皆為劉君先矣。」



 居瑣闥百餘日,繳奏不已,用事者始忌,奏:「一止同周葵薦呂廣問,迎合李光。」罷,提舉江州太平觀。進敷文閣待制。御史中丞何若奏:「一止朋附光,偃蹇慢上。」落職,罷祠。後八年,請老,復職,致仕。秦檜死,召至國門,以病不能拜,力辭,進直學士,致仕。卒年八十三。



 一止沖澹寡欲,嘗誨其子曰:「吾
 平生通塞,聽於自然,唯機械不生,故方寸自有樂地。」博學無不通,為文不事纖刻,制誥坦明有體,書詔一日數十輒辦,嘗言:「訓誥者,賞善罰惡詞也,豈過情溢美、怒鄰罵坐之為哉。」其草顏魯公孫特命官制甚偉,帝嘆賞,為手書之。詩自成家,呂本中、陳與義讀之曰:「語不自人間來也。」有類稿五十卷。子巒、□,從弟寧止。



 寧止字無虞,登宣和進士甲科,除太學錄、校書郎。建炎初,為浙西安撫大使司參議,改兩浙轉運判官。苗傅、劉
 正彥之變,寧止自毗陵馳詣京口、金陵,見呂頤浩、劉光世,勉以忠義,退而具軍須以佐勤王。除左司郎官,辭。帝復位,除右司郎官、給事中。梁揚祖為發運使,寧止再疏論駁。



 以添差江、淮、荊湖制置發運副使扈從隆祐太后幸江西,尋為兩浙轉運副使。錄勤王功,直龍圖閣,進秘閣修撰,主管崇道觀,提點江、淮等路坑冶鑄錢,知鎮江府兼沿江安撫,進右文殿修撰。寧止言:「京口控扼大江,為浙西門戶,請分常州、江陰軍及昆山、常熟二縣隸本
 司,庶防秋時沿江號令歸一,可以固守。」權戶部侍郎,總領三宣撫司錢糧。張浚都督諸軍,以為行府屬。除史部侍郎,進徽猷閣直學士、知秀州,升顯謨閣,提舉太平觀,卒。



 寧止有文名,慷慨喜論事。當艱難時,上疏言闕失,指切隱微,多人所難言。乞禁王安石《日錄》,復賢良方正科,用司馬光十科薦士法,仿唐制宰執論事以諫官侍立,皆其顯顯者。勤王之舉,呂頤浩紀其有輸忠贊謀之勞。寧止與一止、岑皆群從昆弟,帝嘗稱寧止忠、一止清、岑
 敏云。有《教忠堂類稿》十卷。



 胡交修字已楙,常州晉陵人。登崇寧二年進士第,授泰州推官,試詞學兼茂科。給事中翟汝文同知貢舉,得其文曰:「非吾所能及也。」置之首選,除編類國朝會要所檢閱文字。政和六年,遷太常博士、都官郎,徙祠部,遷左司官,拜起居舍人、起居郎。昭慈太后垂簾聽政,除右文殿修撰、知湖州。



 建炎初,以中書舍人召,辭不至,改徽猷閣待制、提舉杭州洞霄宮。三年,復以舍人召,詔守臣津發,
 尋進給事中、直學士院兼侍講。入對,首論天下大勢曰:「淮南當吾膺,將士遇敵先奔,無藩籬之衛。湖、廣帶吾脅,群盜乘間竊發,有腹心之憂。江、浙肇吾基,根本久未立。秦、蜀張吾援,指臂不相救。宜詔二三大臣修政事,選將帥,搜補卒乘,以張國勢,撫綏疲瘵,以固國本。」



 帝又出手詔,訪以弭盜保民、豐財裕國、強兵御戎之要,交修疏言:「昔人謂甑有麥飯,床有故絮,雖儀、秦說之不能使為盜,惟其凍餓無聊,日與死迫,然後忍以其身棄之於盜賊。
 陛下下寬大之詔,開其自新之路,禁苛慝之暴,豐其衣食之源,則悔悟者更相告語歡呼而歸。其不變者,黨與攜落,亦為吏士所系獲,而盜可弭,盜弭則可以保民矣。沃野千里,殘為盜區,皆吾粳稻之地。操弓矢,帶刀劍,椎牛發塚,白晝為盜,皆吾南畝之民。陛下撫而納之,反其田里,無急征暴斂,啟其不肖之心,耕桑以時,各安其業,穀帛不可勝用,而財可豐,財豐則可以裕國矣。日者翟興連西路,董平據南楚,什伍其人,為農為兵,不數年,積
 粟充牣,雄視一方。盜賊猶能爾,況以中興二百郡地,欲強兵以禦寇,不能為翟興輩之所為乎?」世以為名言。



 李成盜江、淮,廷議欲親征,交修謂:群盜猖狂,天子自將,勝之則不武,不勝則貽天下笑。此將帥之責,何足以辱王師?」議遂格,盜尋遁。



 周杞守常州,坐殘虐免。會大旱,帝問交修致旱之由,對以殆杞佚罰之故,乃以杞屬吏。杞疑為交修所讒,上書告其罪,遣大理寺丞胡蒙詣常按驗。交修無所絓,然群從多抵罪。尋以徽猷閣待制提舉太
 平觀。



 六年,召為給事中、刑部侍郎、翰林學士、知制誥兼侍讀。久之,遷刑部尚書。汀州寧化縣論大闢十人,獄已上,知州事鄭強驗問,無一人當死,交修乞治縣令冒賞殺無辜罪。江東留獄追逮者尚六百人,交修言:「若待六百人俱至,則瘐死者眾矣,請以罪狀明白者論如律,疑則從輕。」詔皆如其言。



 朝論欲以四川交子行之諸路,交修力陳其害,謂:「崇寧大錢覆轍可鑒,當時大臣建議,人皆附和,未幾錢分兩等,市有二價,奸民盜鑄,死徙相屬。
 以今交子校之大錢,無銅炭之費,無鼓鑄之勞,一夫挾紙日作十數萬,真贗莫辨,售之不疑,一觸憲網,破家壞產,以賞告捕,禍及無辜。歲月之後,公私之錢盡歸藏鏹之家,商賈不行,市井蕭條,比及悔悟,恐無及矣。」時議大舉,交修曰:「今妄言無行之徒,為迎合可喜之論,吾無以考驗其實,遽信之以舉事,豈不誤國哉?」帝覽之矍然。翌日,出其奏示大臣曰:「交修真一士之諤諤也。」



 蜀帥席益既去,帝問交修孰可守蜀者,對以臣從子世將可用,遂
 以世將為樞密直學士、四川安撫制置使。世將在蜀五年,號為名帥。



 自重兵聚關外以守蜀,餉道險遠,漕舟自嘉陵江而上,春夏漲而多覆,秋冬涸而多膠。紹興初,宣撫副使吳玠始行陸運,調成都、潼川、利州三路夫十萬,縣官部送,徼賞爭先,十斃三四。至是交修言:「養兵所以保蜀也,民不堪命則腹心先潰,何以保蜀?臣愚欲三月以後、九月以前,第存守關正兵,餘悉就糧他州,如此則守關者水運可給,分戍者陸運可免。」帝命學士院述交
 修意,詔玠行之。



 議徽宗配享功臣,交修奏:「韓忠彥建中靖國初為相,賢譽翕然,時號『小元祐』。」從之,人大允服。



 八年夏,以親老,除寶文閣學士、知信州。入辭,上欲留侍經筵,力言母老,願奉祠裏中以便養。帝曰:「卿去,行復召矣。」改提舉江州太平興國宮。九年六月召還,除兵部尚書、翰林學士兼侍講。時河南新復,交修奏;「京西、陜右取士之法,乞如祖宗時設諸科之目,以待西北之士;別為號於南宮,以收五路之才。」詔令禮部討論。逾年,復請補外,
 除端明殿學士、知合州。卻私請,免上供以萬計,領州數月卒。



 交修簡重寡言,進止有度,為文不事琢雕,坦然明白,在詞苑號為稱職。自其從祖宿、從父宗愈至交修、世將,皆在禁林。中興以後,學士三入者自交修始。交修裒次為書,號《四世絲綸集》,以侈一門之遇。至於事繼母以孝聞,撫二弟極其友愛,遇恩以次補官,若交修者,其文行之兼副者歟!



 綦崇禮,字叔厚,高密人,後徙濰之北海。祖及父皆中明
 經進士科。崇禮幼穎邁,十歲能作邑人墓銘,父見大驚曰:「吾家積善之報,其在茲乎!」



 初入太學,諸生溺於王氏新說,少能詞藝者。徽宗幸太學,崇禮出二表,祭酒與同列大稱其工。登重和元年上舍第,調淄縣主簿,為太學正,遷博士,改宣教郎、秘書省正字,除工部員外郎,尋為起居郎、攝給事中。召試政事堂,為制誥三篇,不淹晷而就,辭翰奇偉。拜中書舍人,賜三品服,進用之速,近世所未有,高宗猶以為得之晚。



 車駕如平江,有旨鄒浩追復
 龍圖閣待制,崇禮當行詞,推帝所以褒恤遺直之意,有曰:「處心不欺,養氣至大。言期寤意,引裾嘗犯於雷霆;計不顧身,去國再遷於嶺徼。群臣動色,志士傾心。」又曰:「英爽不忘,想生氣之猶在;奸諛已死,知朽骨之尚寒。」同列推重,除試尚書吏部侍郎,時從官惟崇禮與汪藻,尋兼直學士院。以徽猷閣直學士知漳州,其俗悍強,號難治,屬有巨寇起建州,聲撼鄰境,人心動搖,崇禮牧民御眾,一如常日,訖盜息,環城內外按堵如故。



 徙知明州,召為
 吏部侍郎兼權直學士院。時有詔侍從官日輪一員,具前代及本朝事關治體者一二事進入,崇禮言:「祖宗以來選用儒臣,以奉講讀。若令從官一例獻其所聞,既非舊典,且又越職,望令講讀官三五日一進。」乃命學士與兩省官如前詔。又言:「駐蹕臨安,以浙西為根本,宜固江、淮之守,然後可以圖興復。蜀在萬里外,當召用其士夫,慰安遠人之心。」時兵革後,省曹簿書殘毀幾盡,崇禮再執銓法,熟於典故,討論沿革,援據該審,吏不得容其私。
 後有詔重刊七司條敕,崇禮所建明,悉著為令。



 移兵部侍郎,仍進直學士院。御筆處分召至都堂,令條具進討固守利害。崇禮奏:「諜傳金人並兵趣川、陜,蓋以向來江左用兵非敵之便,故二三歲來悉力窺蜀。其意以謂蜀若不守,江、浙自搖,故必圖之,非特報前日吳玠一敗而已。今日利害,在蜀兵之勝負。」又奏:「君之有臣,所以濟治。臣效實用,則君享其功;臣竊虛名,則君受其弊。實用之利在國,虛名之美在身。忠於國者,不計一己之毀譽,惟
 天下之治亂是憂;潔其身者,不顧天下之治亂,惟一己之毀譽是恤。然效力於國,其實甚難,世未必貴;竊名於己,其為則易,且以得譽。二者有關於風俗甚大,是不可不察也。」



 九月,御筆除翰林學士,自靖康後,從官以御筆除拜自此始。楊惟忠、邢煥以節度使致仕,告由舍人院出,崇禮言:「祖宗時,凡節鉞臣僚得謝,不以文武,並納節別除一官致仕。熙寧間,富弼以元勛始令特帶節鉞致仕,其後繼者曾公亮、文彥博,他人豈可援以為例。」詔自
 今如祖宗故典。



 進兼侍讀兼史館修撰。時有旨重修神宗、哲宗《正史》。兵火之後,典籍散亡,崇禮奏:「《神宗實錄》墨本,元祐所修已是成書,朱本出蔡卞手,多所附會,乞將朱墨本參照修定。《哲宗實錄》,崇寧間蔡京提舉編修,增飾語言,變亂是非,難以便據舊錄修定,欲乞訪求故臣之家文獻事跡參照。」又奏:「知湖州汪藻編類元符庚辰至建炎己酉三十年事跡,乞下藻以已成文字赴本所。」並從之。先是,藻奉詔訪求甚備,未及修纂,崇禮取而專
 之。



 嘗進唐太宗錄刺史姓名於屏風故事,曰:「連千里之封得一良守,則千里之民安;環百里之境得一良令,則百里之民說。牧民之吏咸得其良,則治功成矣。茍能效當時之事,以守令姓名詳列於屏,簡在帝心,則人知盡心職業。」再入翰林凡五年,所撰詔命數百篇,文簡意明,不私美,不寄怨,深得代言之體。



 以寶文閣直學士知紹興府。劉豫導金人入侵,揚、楚震擾,高宗躬御戎衣次吳會。崇禮以近臣承寧方面,謂:「浙東一道為行都肘腋之
 地,備御不可不謹。」密疏於朝,得便宜從事。於是繕城郭,厲甲兵,輸錢帛以犒王師,簡舟艦以扼海道,疚心夙夜,殆廢食寢。及春,帝還,七州晏然不知羽檄之遽。斯年,上印綬,退居臺州。卒年六十,贈左朝議大夫。



 崇禮妙齡秀發,聰敏絕人,不為崖岸斬絕之行。廉儉寡欲,獨覃心辭章,洞曉音律,酒酣氣振,長歌慷慨,議論風生,亦一時之英也。中年頓剉場屋,晚方登第,以縣主簿驟升華要,極潤色論思之選。端方亮直,不憚強御,秦檜罷政,崇禮草
 詞顯著其惡無所隱,檜深憾之。及再相,矯詔下臺州就崇禮家索其稿,自於帝前納之,且將修怨。會崇禮已沒,故身後所得恩澤,其家畏懼不敢陳,士大夫亦無敢為其任保。樓鑰嘗敘其文,以為氣格渾然天成,一旦當書命之任,明白洞達,雖武夫遠人曉然知上意所在雲。



 論曰:建炎、紹興之際,網羅俊彥,布於庶職,如衛膚敏以下七人者,其論議時政,指陳闕失,雖或好惡多不同,亦皆一時之表表者,矧一止、寧止兄弟之忠清,交修、崇禮
 之祠翰,又有助於治化者焉。



\end{pinyinscope}