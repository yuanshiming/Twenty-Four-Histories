\article{列傳第一百三十六}

\begin{pinyinscope}

 向
 子諲陳規季陵盧知原弟法原陳桷李璆李樸王庠王衣



 向子諲,字伯恭,臨江人,敏中玄孫,欽聖憲肅皇后再從侄也。元符三年,以後復闢恩,補假承奉郎,三遷知開封
 府咸平縣。豪民席勢犯法,獄具上,尹盛章方以獄空覬賞,卻不受,子諲以聞,詔許自論決,章大怒,劾以他事勒停。



 宣和初,復官,除江、淮發運司主管文字。淮南仍歲旱,漕不通,有欲浚河與江、淮平者,內侍主其議,無敢可否,發運司檄子諲行。子諲言:「自江至淮數百里,河高江、淮數丈,而欲浚之使平,決不可。曩有司三日一啟閘,復作澳儲水,故水不乏。比年行直達之法,加以應奉往來,啟閉無節,堰閘率不存。今復故制,嚴禁約,則無患。」使者用
 其言,漕復通,進秩一等。召對,除淮南轉運判官。以戶部奏諸路起發上供不及數,降一官。



 七年,入為右司員外郎,不就,以直秘閣為京畿轉運副使,尋兼發運副使。建炎元年,金人犯亳州,子諲自勤王所以書遺金人,言兵勢逆順,令退保河外。金人遽以亳、宋等州守御所牒報之,約日索戰,語極不遜,諸道兵畏縮不進。時康王次濟州,子諲遣進士李植獻金帛及本司錢穀之在濟州者,以助軍費。張邦昌僭位,遣人持敕書往廬州問其家安
 否,子諲檄郡守馮詢、提舉範仲使拘之以俟王命。邦昌又使其甥劉達繼手書來,子諲不啟封焚之,械系達於獄。遣子澹請康王率諸將渡河,出其不意以救二帝;遣將王儀統勤王兵至城下。



 遷直龍圖閣、江淮發運副使。子諲言:「去歲劉順奉淵聖蠟詔,命監司帥守募兵勤王,臣即鏤板遍檄所部,而六路之間漠無應者;間有團結起發者,類如兒戲,姑以避責而已。惟淮東一路,臣親率諸司,粗成紀律。然諸司猶有占吝錢物,莫肯供億,殊不
 念君父幽處圍城之中,臣當時恨無利刃以加其頸。今京城失守,二帝播遷,儻賞罰不行,恐金人再為邊患,陛下復欲起天下之兵,而諸路玩習故常,恬不知畏,將何恃以濟艱難哉?願明詔大臣按劾諸路監司向承蠟詔廢格不勤王,及名為勤王而稽緩者,悉加顯黜。」命諸路提刑司究實以聞。九月,子諲罷,以素為李綱所善,故黃潛善斥之。



 明年,知襲慶府,道梗不能赴。初,邦昌為平章軍國事,子諲乞致仕避之,坐言者降三官,起復知潭州。
 禁卒為亂,縱火掠市,出瀏陽縣,子諲遣通判孟彥卿等追及攸縣平之。



 金人破江西,移兵湖南,子諲聞警報,率軍民以死守。宗室成忠郎聿之隸東壁,子諲巡城,顧謂曰:「君宗室,不可效此曹茍簡。」聿之感激流涕。金人圍八日,登城縱火,子諲率官吏奪南楚門遁,城陷。坐敵至失守落職罷。轉運副使賈收言子諲督兵巷戰,又收潰卒復入治事,帝亦以子諲與他守臣望風遁者殊科,詔復職。



 紹興元年,移鄂州,主管荊湖東路安撫司。劇盜曹成
 據攸縣,子諲軍於安仁,遣使招之,成聽命。子諲又遣將西扼衡陽,南守宜章,成逡巡不敢南向者百餘日,諸郡遂得割獲。既而援兵不至,成忿子諲扼己,擁眾而南,子諲率親兵拒之。會官軍潰,度不可遏,單騎入賊中,諭以國家威靈。成不服,執子諲歸。會宣撫司都統制馬擴遣人持吳敏檄諭成,成許受招,始釋子諲。



 詔提舉江州太平觀。胡安國方避地湖南,以書抵秦檜,言:「子諲忠節,可以扶持三綱,願憐其無救而陷於賊,復加收用。」起知廣
 州。時恐賊度嶺,故就用子諲守之。又以言者罷,遂致仕。尋起知江州,收江東轉運使,進秘閣修撰。江東當餉劉光世軍,適劉豫入寇,光世軍合淝,以乏餉告,亟退師。子諲馳至合淝,具見糧以聞,光世由是得罪。進徽猷閣待制。徙兩浙路為都轉運使,除戶部侍郎。



 入見,論京都舊事,頗及珍玩。起居郎潘良貴故善子諲,聞其言甚怒。既而子諲奏金國報聘及奠朱震事,反復良久。良貴徑至榻前厲聲叱之曰:「子諲不宜以無益之談久煩聖聽。」子
 諲欲退,上謂良貴曰:「是朕問之也。」又諭子諲款語。子諲復語,久不止,良貴叱之退者再。上色變,欲抵良貴罪。中丞常同言:「良貴無罪,願許子諲補外。」上並怒同。張九成言:「士大夫所以嘉子諲者,以其能眷眷於善類。今以子諲故逐柱史,又逐中司,非所以愛子諲也。」上意稍解,批諭同,同言不已,於是三人俱罷。子諲以徽猷閣直學士知平江府。金使議和將入境,子諲不肯拜金詔,乃上章言:「自古人主屈己和戎,未聞甚於此時,宜卻勿受。」忤秦
 檜意,乃致仕。



 子諲相家子,能修飭自見於時。友愛諸弟,置義莊,贍宗族貧者。初,漕淮南時,張邦昌偽詔至,虹縣令已下迎拜宣讀如例程,獨武尉徐端益不拜而走。事定,子諲言於朝,易端益文資。退閑十五年,號所居曰「薌林」。卒,年六十八。



 陳規,字符則,密州安丘人。中明法科。靖康末,金人入侵,殺鎮海軍節度使劉延慶,其徒祝進、王在去為盜,犯隨、郢、復等州。規為安陸令,以勤王兵赴汴,至蔡州,道梗而
 還。會祝進攻德安府,守棄城遁,父老請規攝守事。規遣射士張立率兵討進,卻之。既而在復與進合,以炮石鵝車攻城東,規連戰敗之,二人懼,引眾去。



 建炎元年,除直龍圖閣、知德安府。李孝義、張世以步騎數萬薄城,陽稱受詔招,規登城視其營壘,曰:「此詐也。」亟為備。夜半,孝義兵圍城,遂大敗之。與群盜楊進相持十八日,進技窮,以百人自衛,抵濠上求和。規出城與交臂語,進感之,折箭為誓而去。董平引眾窺城,遣其黨李居正、黃進入城求
 犒,規斬進,授居正兵為前鋒,大破之。升秘閣修撰。尋除德安府、復州、漢陽軍鎮撫使,賜三品服,俄升徽猷閣待制。



 時桑仲剽略襄、漢間,其副霍明屯兵郢上,規請於朝,就以明守郢。張浚都督行蜀道,仲引兵窺之,為王彥所敗。仲怒,從數百騎來譙明,明殺之,奔劉豫,以書招規,規械其使以聞。李橫圍城,造天橋,填濠,鼓噪臨城。規帥軍民御之,炮傷足,神色不變,圍急糧盡,出家財勞軍,士氣益振。橫遣人來,願得妓女罷軍,規不許。諸將曰:「圍城七
 十日矣,以一婦活一城,不亦可乎。」規竟不予。會濠橋陷,規以六十人持火槍自西門出,焚天橋,以火牛助之,須臾皆盡,橫拔砦去。



 升徽猷閣直學士,詔赴行在,改顯謨閣直學士,徙知池州、沿江安撫使。入對,首言:「鎮撫使當罷,諸將跋扈,請用偏裨以分其勢。」上皆納之。遷龍圖閣直學士,改知廬州,尋又召赴行在,以疾辭,提舉江州太平觀。復起知德安府,坐失察吏職,鐫兩官。



 金人歸河南地,改知順昌府,葺城壁,招流亡,立保伍。會劉錡領兵赴
 京留守過郡境,規出迎,坐未定,傳金人已入京城,即告錡城中有粟數萬斛,勉同為死守計。相與登城區畫,分命諸將守四門,且明斥候,募土人鄉導間諜。布設粗畢,金游騎已薄城矣。既至,金龍虎大王者提重兵踵至,規躬擐甲冑,與錡巡城督戰,用神臂弓射之,稍引退,復以步兵邀擊,溺於河者甚眾。規曰:「敵志屢挫,必思出奇困我,不若潛兵斫營,使彼晝夜不得休,可養吾銳也。」錡然之,果劫中其砦,殲其兵甚眾。金人告急於兀朮。規大饗
 將士,酒半問曰:「兀朮擁精兵且至,策將安出?」諸將或謂今已累捷,宜乘勢全師而歸。規曰:「朝廷養兵十五年,正欲為緩急用,況屢挫其鋒,軍聲稍振。規已分一死,進亦死,退亦死,不如進為忠也。」錡叱諸將曰:「府公文人猶誓死守,況汝曹耶!兼金營近三十里,兀朮來援,我軍一動,金人追及,老幼先亂,必至狼狽,不獨廢前功,致兩淮侵擾,江、浙震驚。平生報君,反成誤國,不如背城一戰,死中求生可也。」



 已而兀朮至,親循城,責諸酋用兵之失,眾跪
 曰:「南兵非昔比。」兀朮下令晨飯府庭,且折箭為誓,並兵十餘萬攻城,自將鐵浮屠軍三千游擊。規與錡行城,勉激諸將,流矢及衣無懼色,軍殊死鬥。時方劇暑,規謂錡毋多出軍,第更隊易器,以逸制勞,蔑不勝矣。每清晨輒堅壁不出,伺金兵暴烈日中,至未申,氣力疲,則城中兵爭奮,斬獲無算,兀朮宵遁。錡奏功,詔褒諭之,遷樞密直學士。規至順昌,即廣糴粟麥實倉廩。會計議司移粟赴河上,規請以金帛代輸,至是得其用,成錡功者,食足故
 也。



 移知廬州兼淮西安撫,既至,疾作。有旨修郡城,規在告,吏抱文書入臥內,規力疾起曰:「帥事,機宜董之;郡城,通判董之。」語畢而卒,年七十。贈右正議大夫。有《攻守方略》傳於世。



 初,規守德安時,嘗條上營屯田事宜,欲仿古屯田之制,合射士民兵,分地耕墾。軍士所屯之田,皆相險隘立堡砦,寇至則堡聚捍禦,無事則乘時田作,射士皆分半以耕屯田。民戶所營之田,水田畝賦粳米一斗,陸田賦麥豆各五升。滿三年無逋輸,給為永業。流民自
 歸者以田還之。凡屯田事,營田司兼行,營田事,府縣官兼行,皆不更置官吏,條列以聞,詔嘉獎之,仍下其法於諸鎮。自紹興以來,文臣鎮撫使有威聲者,惟規而已。



 規端毅寡言笑,然待人和易。以忠義自許,尤好振施,家無贏財。嘗為女求從婢,得一婦甚閑雅,怪而詢之,乃雲夢張貢士女也,亂離夫死無所托,鬻身求活,規即輟女奩嫁之,聞者感泣。規功名與諸將等,而位不酬勞,時共惜之。乾道八年,詔刻《規德安守城錄》頒天下為諸守將法。
 立廟德安,賜額「賢守」,追封忠利侯,後加封智敏。



 季陵,字延仲,處之龍泉人。登政和二年上舍第,三遷太學博士。論學術邪正異同,長官怒,譖之執政,謫知舒城縣。未幾,除太常寺簿,遷比部員外郎。高宗即位,從至揚州。建炎二年,守尚書右司員外郎、太常少卿。金人南侵,帝幸杭州,朝廷儀物皆委棄之,陵奉九廟神主負之以行,拜起居郎,遷中書舍人。



 三年六月,淫雨,詔求直言。陵言:「金人累歲侵軼,生靈塗炭,怨氣所積,災異之來,固不
 足怪。惟先格王,正厥事,則在我者其可忽邪?臣觀廟堂無擅命之臣,惟將帥之權太盛;宮閫無女謁之私,惟宦寺之習未革。今將帥擁兵自衛,浸成跋扈,苗、劉竊發。勤王之師一至,凌轢官吏,莫敢誰何?此將帥之權太盛有以乾陽也。宦寺縱橫,上下共憤,卒碎賊手,可為戒矣。比聞復召藍珪,黨與相賀,聞者切齒,此宦寺之習未革有以乾陽也。《洪範》休徵曰,肅時雨若,謀時寒若;咎徵曰『,狂恆雨若,急恆寒若。自古天子之出,必載廟主行,示有尊
 也。前日倉卒迎奉,不能如禮。既至錢塘,置太廟於道宮,薦享有闕;留神御於河滸,安奉後時。不肅之咎,臣意宗廟當之。比年盜賊例許招安,未幾再叛,反墮其計。忠臣之憤不雪,赤子之冤莫報,不謀之咎,臣意盜賊當之。道路之言謂鑾輿不久居此,自臣臆度,決無是事,假或有之,不幾於狂乎?軍興以來,既結保甲,又改巡社,既招弓手,又募民兵,民力竭矣,而猶誅求焉,不幾於急乎?此皆陰道太盛所致。」帝嘉納之。



 時除梁揚祖為發運使,給事
 中劉寧止言其不可,乃以起居郎綦崇禮權給事中,書讀,陵封還錄黃。又言:「防秋已迫,願陛下先定兵衛及扈從之臣,萬一敵勢猖獗,便當整駕親按營壘,召諸道兵以為援,留將相大臣,相率死守,勿效前日百官跣足奔竄,以扈蹕為名,棄城池以予敵,使生靈墮塗炭,財用填溝壑。」



 時張浚為川、陵等路宣撫處置使,陵論其太專,忤旨,罷為徽猷閣待制、知太平州,未行,落職與祠。數月,復職,除知溫州,又改中書舍人,皆力辭。



 範宗尹薦其才,命
 知臨安府,復為中書舍人。入對,言:「事有可深慮者四,尚可恃者一:大駕未有駐蹕之地,賢人皆無經世之心,兵柄分而將不和,政權去而主益弱;所恃以僅存者,人心未厭而已。前年議渡江,人以為可,朝廷以為不可,故諱言南渡而降詔回鑾。去年議幸蜀,人以為不可,朝廷以為可,故弛備江、淮,經營關、陜。以今觀之,孰得孰失?惟揚之變,朝廷不及知而功歸宦寺;錢塘之變,朝廷不能救而功歸將帥,是致此曹有輕朝士之心。黃潛善好自用
 不能用人,呂頤浩知使能不知任賢。自張愨、許景衡飲恨而死,凡知幾自重者,往往卷懷退縮。今天下不可謂無兵,劉光世、韓世忠、張俊各招亡命以張軍勢,各效小勞以報主恩。然勝不相遜,敗不相救,大敵一至,人自為謀耳。周望在浙西,人能言之;張浚在陜右,無敢言者。夫軍事恐失機會,便宜可也,乃若自降詔書,得無竊命之嫌邪?官吏責以辦事,便宜可也,乃若安置從臣,得無忌器之嫌邪?以至賜姓氏,改寺額,此皆傷於太專,臣恐自
 陜以西不知有陛下矣。惟祖宗德澤在人心未忘,所望以中興者此耳,陛下宜有以結之。今欲薄斂以裕民財,而用度方闕;輕徭以紓民力,而師旅方興。罪己之詔屢降,憂民之言屢聞,丁寧切至,終莫之信。臣謂動民以行不以言,陛下爵當賢,祿當功,刑當罪,施設注措無不當理,天下不心服者未之有也。」



 朱勝非除江西帥,未行。陵言:「金人往年休士馬於燕山,次年移河北,又次年移京東,今寓淮甸,無復去意,患在朝夕,可謂急矣。若頤浩既
 去,勝非未至,金人南向,兵不素練,糧不素積,又不設險,何以御之?臣願陛下更擇賢副,預為經畫以待。今日非論安危,實論存亡,朝謀夕行,當如拯溺,豈可不惜分陰。」詔劉洪道趣往池州,措置防江。除戶部侍郎。



 範宗尹嘗仕偽楚,故凡受偽命者皆錄用。陵因上疏曰:「前日士大夫名節不立,論事者皆喜攻之,瑕疵既彰,不復可用,縱加抆拭,攻者踵來,雖君相制命,亦不能為之地。臣試舉其罪大者言之,崇寧、大觀以來,黨助巨奸,由詭道以饕
 寵榮者不知幾何人?邦昌亂朝,不能死節者不知幾何人?苗、劉專殺,拱手受制不知幾何人?以義責之固不容誅,以情恕之亦不幸耳。弄筆墨者,文致其罪,既得惡名,誰敢引薦。臣願明詔宰執,於罪戾中選擇實能,量付以事,勿因一眚廢其終身,仍詔臺諫為國愛人,勿復言。」詔榜其疏於朝堂。侍御史沈與求劾陵承望宰執風旨,罷官,提舉杭州洞霄宮。



 紹興元年,復右文殿修撰。二年,詔內外官言事。陵言:「軍興以來,朝廷誥牒,非強以予民則
 莫售;師旅糧草,非強取於民則莫給。舊例和買,無本可支者久矣,新行和糴,能償其直幾何?」一遇軍興,事事責辦,有不足者,預借後年之賦。雖名曰『和」,實強取之;雖名曰『借』,其實奪之。兵將衣食不取其飽暖,取其豐美;器械不取其堅利,取其華好。務末勝本,初無鬥心,賊至則偽言退保,賊去則盛言收復,遇敗以千為一,遇勝以一為千。今乘輿服御之費十去七八,百官有司之費十去五六,猶無益於國者,軍太冗也。張浚一軍以川、陜贍之,劉
 光世一軍以淮、浙贍之,李綱一軍以湖廣贍之,上供之物得至司農、太府者無幾。夫強兵不在冗食,今統領家口隨行,一聞賊至,擇精銳者護送老小,其自隨者祗辦走耳,當議者一。虜掠婦女,軍中多有,養既不足,寧免作過,當議者二。所至州軍,邀求犒賞,守令憚生事,竭取民以奉之,當議者三。詭名虛券,隨在批請,枉費官物,當議者四。或假關節,或行賄賂,寄名軍籍,規冒功賞,當議者五。願詔有司專意講求,革因循以作士氣,則軍政立。」復
 徽猷閣待制,帥廣。



 先是,惠州有狂男子聚眾數千,僭號作亂。陵入境,誘其徒曾袞,令以功贖罪,不旬日擒之。在官三年卒,年五十五,贈中大夫。有文集十卷。



 陵善言事,奏疏可觀。然附範宗尹,則謂凡受偽命者皆當進用,臺諫不當復以為言;攻張浚,則謂在蜀失於太專,自陜以西將不知有陛下。君子皆不謂然也。幸醫王繼先授榮州防禦使,陵草其制,時論亦以此少之。



 盧知原,字行之,湖州德清人。以父任知歙縣,因近臣薦,
 赴都堂審察,累遷梓州路轉運副使。時承平既久,戎備皆弛,知原招補兵籍,築城亙二十餘里。王黼當國,費出無藝,知原因疏言之,黼怒,罷去。久之,起提點京東刑獄,改江西轉運副使,過闕入奏,徽宗勉之曰:「卿在蜀道,功效甚休。」遂賜三品服。



 先是,綱運阻於重江,吏卒並緣為奸。知原悉意經理,故先諸道上京師,進一官,尋除直秘閣,為江、淮、荊、浙等路發運使。升秘閣修撰,提舉河北。以言者劾,褫職歸吏部。



 高宗即位,復龍圖閣、知溫州。時葉
 濃陷建州,揚勍陷處州,知原繕甲兵,增城浚隍,聲勢隱然。帝東幸,知原繇海道轉粟及金繒士餘萬至臺州。召見,稱獎,擢右文殿修撰、管內安撫使。在郡四年,民繪像祠之。



 王師討範汝為,召為添差兩浙轉運使。罷,提舉太平觀。都督孟庾闢為參謀,改徽猷閣待制、知臨安府。諫官唐輝言:「知原為政乖謬。」詔復為都督府參謀官。章再上,遂以舊職奉祠。紹興十一年十月卒。弟法原。



 法原字立之。自知雍丘縣積官太府少卿,賜同上舍出
 身。使遼還,遷司農卿,賜三品服。為吏部尚書,以官秩次第履歷總為一書,功過殿最,開卷了然,吏不能欺。坐王黼累,罷為顯謨閣待制。



 紹興元年,提舉臨安洞霄宮。張浚承制起知夔州,尋為龍圖閣學士、川陜等路宣撫處置副使,進端明殿學士、川陜宣撫副使。



 金人攻關輔,叛將史斌陷興州,諸郡多應者。法原命諸將堅壁,言戰者斬,眾以為怯。未幾,河東經制使王□燮以乏食班師,法原開關納之,與□燮同破斌,復興州。方巨盜充斥,秦、隴叛兵
 欲窺蜀,法原極意拊循,嚴為備御,傳檄諸路,人心稍安。視山川險阻分地置將:自洮、岷至階、成,關師古主之,屯通川;文、龍至威、茂,劉錡主之,屯巴西。前後屢捷,上所倚重。



 會兀朮攻關為吳玠所敗。法原素與玠不睦,玠因奏功訟法原不濟師,不饋糧,不銓錄立功將士。帝手詔詰問,法原自辯甚力,上頗不直之,憂恚,卒於軍。



 始,法原為川、陜宣撫使,上從容謂知原曰:「朕方以川、陜付法原。」蓋兄弟皆以材見稱於世,故並用之也。



 陳桷,字季壬,溫州平陽人。以上舍貢闢雍。政和二年,廷對第三,授文林郎、冀州兵曹參軍,累遷尚書虞部員外郎。



 宣和七年,提點福建路刑獄。福州調發防秋兵,資糧不滿望,殺帥臣,變生倉卒,吏民奔潰,闔城震駭。桷入亂兵中,諭以禍福,賊氣沮,邀桷奏帥臣自斃,桷詭從其請,間道馳奏,以前奏不實待罪,朝廷以桷知變,釋之。叛兵既調行,乃道追殺首惡二十餘人,一方以安。建炎四年五月,復除福建路提刑,尋以疾乞祠,主管江州太平觀。



 紹興三年,召為金部員外郎,升郎中。時言事者率毛舉細務,略大利害。桷抗言:「今當專講治道之本,修政事以攘敵國,不當以細故勤聖慮如平時也。」又言:「刺史縣令滿天下,不能皆得人,乞選監司,重其權,久其任。」除太常少卿。又陳攻守二策,在於得人心,修軍政。



 五年,除直龍圖閣、知泉州。明年,改兩浙西路提刑。乞置鄉縣三老以厚風俗,凡宮室、車馬、衣服、器械定為差等,重侈靡之禁。八年,遷福建路轉運副使。



 十年,復召為太常少卿。適編
 類徽宗御書成,詔藏敷文閣,桷以為:「舊制自龍圖至徽猷皆設學士、待制,雜壓著令,龍圖在朝請大夫之上,至徽猷在承議郎之上,每閣相去稍遠,議者疑其不倫。直敷文閣者綴徽猷則與諸閣小異,除之則班列太卑,欲參酌取中,並為一列,不必相遠,庶幾名位有倫,仰稱陛下嚴奉祖宗謨訓之意。」又言:「祫祭用太牢,此祀典之常。駐蹕之初,未能備禮,止用一羊,乞檢會紹興六年詔旨,復用太牢。」



 十一年,除權禮部侍郎,賜三品服。普安郡王
 出閣,奉詔與吏部、太常寺討論典故。桷等議以國本未立,宜厚其禮以系天下望,乃以《皇子出閣禮例》上之,或以為太重。詔以不詳具典故,專任己意,懷奸附麗,與吏部尚書吳表臣、禮部尚書蘇符、郎官方雲翼丁仲寧、太常屬王普蘇籍並罷。尋以桷提舉江州太平觀。



 十五年,知襄陽府,充京西南路安撫使。襄、漢兵火之餘,民物凋瘵,桷請於朝,以今之戶數視承平時才二十之一,而賦須尚多,乞重行蠲減。明年,金、戶兵叛,桷遣將平之而後
 以聞。漢水決溢,漂蕩廬舍,躬率兵民捍築堤岸,賴以無虞。以疾乞祠,除秘閣修撰、提舉江州太平興國宮。二十四年,改知廣州,充廣南東路經略安撫使,未至而卒,年六十四。



 桷寬洪醞籍,以誠接物,而恬於榮利。當秦檜用事,以永嘉為寓里,士之夤緣攀附者,無不躐登顯要。桷以立螭之舊,為人主所知,出入頓挫,晚由奉常少卿擢權小宗伯,復以議禮不阿忤意,遽罷,其節有足稱。自號「無相居士」。有文集十六卷。子汝楫、汝賢、汝諧。孫峴,以詞
 學擢第,官中書舍人、直學士院。



 李璆,字西美,汴人。登政和進士第,調陳州教授,入為國子博士,出知房州。時既榷官茶,復強民輸舊額,貧無所出,被系者數百人,璆至,即日盡釋之。



 宣和三年,廷議將取燕,璆聞之,曰:「百闢卿士,一倡共和,國家安危,其幾在是。」上疏切諫,大略謂:「太祖以聖武得天下,將士皆百戰之餘,以是而取燕云,宜易為力。然趙普輩無敢贊其決者,蓋識天下大勢,且重民命故也。今承太平之業,父老
 幸不識兵,雖不得燕云地,何闕於漢。」疏奏不省。及燕既平,責監英州清溪鎮。



 明年,赦還為郎,尋試中書舍人。建言元祐名臣子孫,久被廢錮,宜少寬之。宦官譚稹出師河北,以無功廢,將復進用,璆不肯書行。會山東盜起,州縣不能制,至河北無見糧,軍士洶洶。璆條奏十事,忤大臣意,罷。紹興四年,以集英殿修撰知吉州。江西兵素剽悍,璆始視事,有相挺為亂者,亟捕誅首謀者,撫循其餘,大布恩信,境內遂安。



 累遷徽猷閣直學士、四川安撫制
 置使。成都舊城多毀圮,璆至,首命修築。俄水大至,民賴以安。三江有堰,可以下灌眉田百萬頃,久廢弗修,田萊以荒。璆率部刺史合力修復,竟受其利,眉人感之,繪像祠於堰所。間遭歲饑,民徙,發倉振活,無慮百萬家,治蜀之政多可紀。有《清溪集》二十卷。



 李樸,字先之,虔之興國人。登紹聖元年進士第,調臨江軍司法參軍,移西京國子監教授,程頤獨器許之。移虔州教授。以嘗言隆祐太后不當廢處瑤華宮事,有詔推
 鞫。忌者欲擠之死,使人危言動之,樸泰然無懼色。旋追官勒停,會赦,注汀州司戶。



 徽宗即位,翰林承旨范純禮自言待罪四十六日,不聞玉音,謂樸曰:「某事豈便於國乎?某事豈便於民乎?」樸曰:「承旨知而不言,無父風也。」純禮泣下。



 右司諫陳瓘薦樸,有旨召對,樸首言:「熙寧、元豐以來,政體屢變,始出一二大臣所學不同,後乃更執圓方,互相排擊,失今不治,必至不可勝救。」又言:「今士大夫之學不求諸己,而惟王氏之聽,敗壞心術,莫大於此。願
 詔勿以王氏為拘,則英材輩出矣。」蔡京惡樸鯁直,他執政三擬官,皆持之不下,復以為虔州教授。又嗾言者論樸為元祐學術。,不當領師儒,罷為肇慶府四會令。



 有奸民言邑東地產金寶,立額買撲,破田疇,發墟墓,厚賂乃已,樸至,請罷之。改承事郎,知臨江軍清江縣、廣東路安撫司主管機宜文字。欽宗在東宮聞其名,及即位,除著作郎,半歲凡五遷至國子祭酒,以疾不能至。高宗即位,除秘書監,趣召,未至而卒,年六十五。贈寶文閣待制,官
 其子孫二人。



 樸自為小官,天下高其名。蔡京將強致之,俾所厚道意,許以禁從,樸力拒不見,京怒形於色,然終不害也。中書侍郎馮熙載欲邂逅見樸,樸笑曰:「不能見蔡京,焉能邂逅馮熙載邪?」居官所至有聲。在廣南,止其帥孫俟以文具勤王,不若發常賦助邊。破漕使鄭良引真臘取安南之計,以息邊患,人稱其智。樸嘗自志其墓曰:「以天為心,以道為體,以時為用,其可已矣。」蓋敘其平生云。有《章貢集》二十卷行於世。



 王庠,字周彥,榮州人。累世同居,號「義門王氏」。祖伯琪,以義聲著於鄉州,。有鹽井籍民煎輸,多至破產,惟有祿之家得免。伯琪請於州,均之官戶,而仕者誣訴之,繼恨以歿。父夢易,登皇祐第,力成父志,言於州縣不聽,言於刺史,言於三司,三司以聞,還籍沒者三百五十五家,蠲歲額三十萬斤。嘗攝興州,改川茶運,置茶鋪免役民,歲課亦辦。部刺史恨其議不出己,以他事中之,鐫三秩,罷歸而卒。母向氏,欽聖憲肅後之姑也。



 庠幼穎悟,七歲能屬
 文,儼如成人。年十三,居父喪,哀憤深切,謂弟序曰:「父以直道見擠,母撫柩誓言,期我兄弟成立贈復父官,乃許歸葬,相與勉之。且制科先君之遺意也,吾有志焉。」遂閉戶,窮經史百家書傳注之學,尋師千里,究其旨歸。蚤歲上范純仁、蘇轍、張商英書,皆持中立不倚之論,呂陶、蘇轍皆器重之。嘗以《經說》寄蘇軾,謂:「二帝三王之臣皆志於道,惟其自得之難,故守之至堅。自孔、孟作《六經》,斯道有一定之論,士之所養,反不逮古,乃知後世見《六經》之
 易,忽之不行也。」軾復曰:「《經說》一篇,誠哉是言。」



 元祐中,呂陶以賢良方正直言極諫科薦之,庠以宋邦傑學成未有薦者,推使先就,陶聞而益加敬。未幾,當紹聖諸臣用事,遂罷制科,庠嘆曰:「命也,無愧先訓,以之行己足矣。」



 崇寧壬午歲,應能書,為首選。京師蝗,庠上書論時政得失,謂:「中外壅蔽,將生寇戎之患。」張舜民見之,嘆其危言。下第徑歸,奉親養志,不應舉者八年。



 大觀庚寅,行舍法於天下,州復以庠應詔。庠曰:「昔以母年五十二求侍養,不
 復願仕,今母年六十,乃奉詔,豈本心乎?」時嚴元祐黨禁,庠自陳:「蘇軾、蘇轍、范純仁為知己,呂陶、王吉嘗薦舉,黃庭堅、張舜民、王鞏、任伯雨為交游,不可入舉求仕,願屏居田里。」以弟序升朝,贈父官,始克葬,葬而母卒。



 終喪復舉八行,事下太學,大司成考定為天下第一,詔旌其門。朝廷知其不可屈,賜號「處士」。尋改潼川府教授,賜出身及章服,一日四命俱至,竟力辭不受。雖處山林,唱酬賦詠,皆愛君憂國之言。太后念其姑,嘗欲官,庠以遜其弟、
 侄及甥,且以田均給庶兄及前母之姊。庠卒,孝宗謚曰賢節。



 序,宣和間以恩幸至徽猷閣直學士。庠浮沉其間,各建大第,或者謂其晚節隱操少衰云。



 王衣,字子裳,濟南歷城人。以門蔭仕,中明法科,歷深、冀二州法曹掾,入為大理評事,升寺正。林靈素得幸,將毀釋氏以逞其私。襄州僧杜德寶毀體然香,有司觀望靈素意,捕以聞。衣閱之曰:「律自傷者杖而已。」靈素求內批,坐以害風教竄流之,停衣官,尋予祠。為陜西都轉運司
 主管文字、詳定一司敕令所刪定官、通判襲慶府、知濠州,未行,召為刑部員外郎。



 建炎初,為司勛郎中,遷大理少卿。三年,韓世忠執苗傅、劉正彥,獻俘,檻車幾百兩,先付大理獄,將盡尸諸市。衣奏曰:「此曹在律當誅,顧其中婦女有顧買及鹵掠以從者。」高宗矍然曰:「卿言極是,朕慮不及此也。」即詔自傳?正彥妻子外皆釋之。範瓊有罪下大理寺,衣奉詔鞫之。瓊不伏,衣責以靖康圍城中逼遷上皇,擅殺吳革,迎立張邦昌事,瓊稱死罪。衣顧吏曰:「
 囚詞服矣。」遂賜死,釋其親屬將佐。



 四年,升大理卿。初,帶御器械王球為龍德宮都監,盡盜本宮寶玉器玩,事覺,帝大怒,欲誅之。衣曰:「球固可殺,然非其所隱匿,則盡為敵有,何從復歸國家乎?」乃寬之。



 先是,百司愆戾,付寺劾之,至三問取伏狀,被劾者懼對,莫敢辨。衣奏曰:「伏與辨二事也,若一切取伏,是以威迫之,不使自直,非法意也,乞三問未承者,聽辨。」從之。同詳定一司敕令,刪雜犯死罪四十七條,書成,帝嘉其議法詳明。



 紹興元年,權刑部
 侍郎。二年,除集英殿修撰,奉祠。既而趙令畤應詔薦之,復召為刑部侍郎,為言者所格。四年,卒於家。衣質直和易,持法不阿,議者賢之。



 論曰:向子諲以相家之子克飭臣節,陳規以文儒之臣有聲鎮守,可謂拔乎流俗者焉。季陵言事不諱,二盧兄弟並用,以材見稱,陳桷守禮知變,李璆為政有惠,咸足紀焉。李樸不訹權威,王庠志高而晚節頗衰,王衣明恕而用刑不刻,雖或器識不齊,亦皆不曠其職也歟!



\end{pinyinscope}