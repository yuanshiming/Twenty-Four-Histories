\article{志第四十 地理三}

\begin{pinyinscope}

 陜
 西



 陜西路。慶歷元年,分陜西沿邊為秦鳳、涇原、環慶、鄜延四路。熙寧五年,以熙、河、洮岷州、通遠軍為一路,置馬步軍都總管、經略安撫使。又以熙、河等五州軍為一路,通
 舊鄜延等五路,共三十四州軍,後分永興、保安軍、河中、陜府、商、解、同、華、耀、虢、鄜、延、丹、坊、環、慶、邠寧州為永興軍等路,轉運使於永興軍、提點刑獄於河中府置司;鳳翔府、秦、階、隴、鳳、成、涇、原、渭、熙、河、洮、岷、州、鎮戎、德順、通遠軍為秦鳳等路,轉運使於秦州、提點刑獄於鳳翔府置司;仍以永興、鄜延、環慶、秦鳳、涇原、熙河分六路,各置經略、安撫司。



 永興軍路。府二:京兆,河中。州十五:陜,延,同,華,耀,邠、鄜,解,
 慶,虢,商,寧,坊,丹,環。軍一:保安。縣八十三。其後延州、慶州改為府,又增銀州、醴州及定邊、綏德、清平、慶成四軍。凡府四,州十五,軍五,縣九十。



 京兆府,京兆郡,永興軍節度。本次府,大觀元年升大都督府。舊領永興軍路安撫使。宣和二年,詔永興軍守臣等銜不用軍額,稱京兆府。崇寧戶二十三萬四千六百九十九,口五十三萬七千二百八十八。貢靴氈、蠟、席、酸棗仁、地骨皮。縣十三:長安,次赤。



 樊川,次赤。舊萬年縣,宣和七年改。



 鄠,次畿。
 藍田,次畿。



 咸陽,次畿。



 涇陽,次畿。



 櫟陽,次畿。



 高陽,次畿。



 興平,次畿。



 臨潼,次畿。唐昭應縣,大中祥符改。



 醴泉,次畿。



 武功,次畿。政和八年,同醴泉撥入醴州。



 乾祐。次畿。



 監二。熙寧四年置,鑄銅錢;八年置,鑄鐵錢。



 河中府,次府,河東郡,護國軍節度。舊兼提舉解州、慶成軍兵馬巡檢事。大中祥符中,以榮河為慶成軍。崇寧戶七萬九千九百六十四,口二十二萬七千三十。貢五味子、龍骨。縣七:河東,次赤。隋縣。熙寧三年,省河西縣,六年,省永樂縣為鎮入焉。



 臨晉,次畿。



 猗氏,次畿。



 虞鄉,次畿。



 萬泉,次畿。



 龍門,次畿。元祐二年,置鑄錢監二。



 榮河。次畿。舊隸
 慶成軍,熙寧元年廢,以榮河隸府,即縣治置軍使。



 慶成軍。見上。



 解州,中,防禦。崇寧戶三萬二千三百五十六,口一十一萬三千三百二十一。貢鹽花。縣三:解,中。



 聞喜,望。



 安邑。緊。



 陜州,大都督府,陜郡。太平興國初,改保平軍,舊兼提舉商、虢州兵馬巡檢事。崇寧戶四萬七千八百六,口一十三萬五千七百一。貢紬」、絁、括蔞根、柏子仁。縣七:陜,中。熙寧六年,省硤石縣為石壕鎮入焉。



 平陸,上。



 夏,上。



 靈寶,上。熙寧四年,省湖城縣入焉。



 芮城,
 中下。



 湖城,中下。元豐元年,復置縣。



 閿鄉。中下。太平興國三年,自虢州與湖城二縣來隸。監二。熙寧三年置,鑄銅錢;八年置,鑄鐵錢。



 商州,望,上洛郡,軍事。崇寧戶七萬三千一百二十九,口一十六萬二千五百三十四。貢麝香、枳殼實。縣五:上洛,中。



 商洛,中下。



 洛南,中下。



 豐陽,中。



 上津。中下



 虢州,雄,虢郡,軍事。崇寧戶二萬二千四百九十,口四萬七千五百六十三。貢麝香、地骨皮、硯。縣四:盧氏,中。熙寧二年,以西京伊陽縣欒川冶鎮隸焉。



 虢略,中。唐弘農縣。建隆初,改常農。至道三年,改今名。熙寧四年,省玉城縣為鎮入焉。



 朱陽,中。乾德六年,廢入常農,太平興國七年,復置。



 欒川。元祐二年,以欒川冶為鎮,崇寧
 三年,改為縣。



 同州,望,馮翊郡,定國軍節度。崇寧戶八萬一千一十一,口二十三萬三千九百六十五。貢白蒺藜、生熟乾地黃。縣六:馮翊,緊。



 澄城,緊。



 朝邑,緊。



 合陽,上。熙寧四年,省夏陽縣為鎮入焉。



 白水,中。



 韓城。中。元祐二年,置鑄錢監。



 監一:沙苑。



 華州,望,華陰郡。建隆初,為鎮國軍節度。皇祐五年,改鎮潼軍節度。崇寧戶九萬四千七百五十,口二十六萬九千三百八十。貢茯苓、細辛、茯神。縣五:鄭,上。



 下邽,望。



 蒲城,
 望。唐奉先縣。開寶四年改。建隆中,自京兆隸同州。天禧四年,自同州來隸。



 華陰,緊。



 渭南。上。熙寧六年,省為鎮入鄭。元豐元年,復為縣。舊自京兆府來隸。



 監二。熙寧四年置,鑄銅錢;八年置,鑄鐵錢。



 耀州,緊,華原郡。開寶五年,為感義軍節度。太平興國初,改感德軍。崇寧戶一十萬二千六百六十七,口三十四萬七千五百三十五。貢瓷器。縣六:華原,上。



 富平,望。



 三原,望



 雲陽,上。



 同官,上。



 美原。中。



 清平軍。本鳳翔府盭厔縣清平鎮。大觀元年,升為軍,復置終南縣,隸京兆府。清平軍使兼知終南縣,專管勾上
 清太平宮。縣一:終南。



 延安府,中,都督府,延安郡,彰武軍節度。本延州。元祐四年,升為府。舊置鄜延路經略、安撫使,統延州、鄜州、丹州、坊州、保安軍、四州一軍;其後增置綏德軍,又置銀州,凡五州二軍。銀州尋廢。崇寧戶五萬九百二十六,口一十六萬九千二百一十六。貢黃蠟、麝香。縣七:膚施,中。熙寧五年,省豐林縣為鎮、金明縣為砦並入焉。有金明龍安二砦、安塞一堡。元豐四年,又收復塞門砦。宣和二年,改龍安曰德安砦。



 延川,中。熙寧八年,省延水縣為鎮入焉。有丹頭、綏平、懷寧、順安、白草、永平六砦,安定、黑水二堡
 及永寧關。元豐四年收復,置細浮圖、義合、米脂三砦。七年,以米脂、義合、浮圖、懷寧、順安、綏平六城砦隸綏德城。元符二年,廢順安、白草、丹頭三砦。



 延長,中。



 門山,中。



 臨真,中。



 敷政,中。有招安、萬安二砦。元符二年,廢招安砦為驛。甘泉。中下。



 城二:治平四年,收復綏州。熙寧中,改為綏德城。四年,置囉兀城、撫寧賓草二堡,尋廢。元豐五年,置永樂城,賜名銀川砦,尋廢。



 青澗城,元符二年,隸綏德城。



 綏德城。元符二年,改為軍。



 監一。熙寧八年置,鑄鐵錢。



 塞門砦,延州北蕃部舊砦,至道後與蘆關、石堡、安遠砦俱廢。元豐四年收復,仍隸延州膚施縣。東至殄羌砦五十里,西至平戎砦六十里,南至安塞堡四十里,北至烏延口九十里。



 平羌砦,地本克胡山砦,紹聖四年賜名。東至安定堡六十里,西至安塞堡三十五里,南至龍安砦五十四里,北至殄羌砦六十里。



 威戎城,地本升平塔,紹聖四年賜名。東至臨
 夏城四十里,西至威羌砦七十里,南至黑水堡六十里,北至界臺七十里。



 平戎砦,地本杏子河東山,紹聖四年賜名。東至塞門砦六十里,西至順寧砦七十里,南至園林堡五十一里,北至杏子堡四十里。



 開光堡,紹聖四年修築。元符元年賜名,二年,隸綏德城。



 殄羌砦,地名那娘山,元符元年進築,賜名。東至威羌砦四十里,西至塞門砦五十里,南至平羌砦六十里,北至御謀城三十五里。



 威羌砦,地名白洛觜,元符元年進築,賜名。東至威戎城七十里,西至殄羌砦四十里,南至安定堡七十里,北至蘆移堡七十里。



 御謀城,崇寧三年進築,賜名。東至蘆移堡三十五里,西至界臺三十五里,南至殄羌砦三十五里,北至界臺二十里。



 石堡砦,崇寧三年進築,賜名威德軍,五年復為砦。國初嘗置城,至道後廢之,地在延州北。



 制戎城,政和八年,賜鄜延路天降山新城改今名。



 新砦,蘆移堡,東至屈丁堡五十里,西至
 御謀城三十五里,南至威羌砦七十里,北至界臺一十三里。



 屈丁堡,萬安堡,東至威戎城六十里,西至蘆移堡四十里,南至威羌砦四十里,北至屈丁堡五十一里。



 丹頭堡,青石崖堡,窟囉堡。



 鄜州,上,洛交郡,保大軍節度。崇寧戶三萬五千四百一,口九萬二千四百一十五。貢麝香,今改貢蠟燭。縣一:宜川。上。後魏義川縣。太平興國中改名,以鄜州廢咸寧縣入焉。熙寧三年省汾川縣、七年省雲巖縣為鎮、八年析同州韓城縣新封鄉並入焉。



 坊州,上,中部郡,軍事。崇寧戶一萬三千四百八,口四萬
 一百九十一。貢弓弦麻、席。縣二:中部,緊。



 宜君。中。熙寧元年,省升平縣為鎮入焉。有礬場。



 保安軍,同下州。崇寧戶二千四十二,口六千九百三十一。貢毛段、蓯蓉。砦二:德靖,東至保安軍八十里,西至慶州荔原堡六十里,南至慶州平戎鎮五十里,北至金湯城六十里。



 順寧。東至平戎砦七十里,西至金湯城九十里,南至保安軍四十里,北至萬全砦四十里。



 堡一:園林。東至安塞堡七十里,西至保安軍四十里,南至招安驛七里,北至平戎堡五十一里。



 金湯城,舊金湯砦,在德靖砦西南,元符二年進築。東至順寧砦九十里,西至慶州白豹城四十里,南至德靖砦六十里,北至通慶城六十里。



 威德軍。保安軍之北,兩界上有洑流名藏底河,夏
 人近是築城,為要害必爭之地。政和三年,賈炎乞進築,不果。七年,知慶州姚古克之,即威德軍。



 綏德軍。唐綏州。熙寧二年,收復廢為城,隸延州,在州東北三十里。元豐七年,以延州米脂、義合、浮圖、懷寧、順安、綏平六城砦隸綏德城。元符二年,改為軍,並將暖泉、米脂、開光、義合、懷寧、克戎、臨夏、綏平砦、青澗城、永寧關、白草、順安砦並隸軍。暖泉砦,元符二年進築,賜名。東至河東烏龍砦二十里,西至米脂砦四十五里,南至義合砦八十里,北至清邊砦七十里。



 米脂砦,本西夏砦,元豐四年收復,為米脂城,後復為砦,隸延州延川縣。七年,改隸綏德城。元祐四年,給賜夏人。元符元年收復,仍賜舊名。東至暖泉砦四十
 五里,西至克戎城六十里,南至開光堡三十里,北至嗣武城二十里。



 開光堡,紹聖四年修築。元符元年賜名。二年,自延安府來屬。東至暖泉砦六十里,西至克戎砦五十里,南至綏德軍三十里,北至米脂砦三十里。



 義合砦,本夏人砦,元豐四年收復,隸延州延川縣。七年,改隸綏德城。東至晉寧軍六十里,西至綏德軍四十里,南至順安驛六十里,北至暖泉砦八十里。



 懷寧砦,延州延川縣舊砦。東至綏德軍四十里,西至綏平砦四十里,南至青澗城七十里,北至克戎砦六十里。



 克戎砦,本西人細浮圖砦,元豐四年收復,隸延州延川縣。七年,改隸綏德城。元祐四年,給賜夏人。紹聖四年收復,賜名。東至綏德軍六十里,西至臨夏砦三十里,南至懷寧砦六十里,北至鎮邊砦六十五里。



 臨夏砦,地名囉巖谷嶺,元符元年築城,賜今名。東至克戎砦三十里,西至威戎城四十里,南至綏平砦六十里,北至界堠八十二里。



 綏平砦,延州
 延川縣舊砦,元符二年,割隸綏德軍。東至懷寧砦四十里,南至黑水堡四十里,西至丹頭驛四十里,北至臨夏砦六十里。



 青澗城,延州舊城。東至永寧縣七十里,西至來平砦七十里,南至延川縣四十里,北至懷寧軍七十里。



 永寧關,延州延川縣舊關。



 白草砦,延州延川縣舊砦,元符二年廢,後復置。



 順安砦,延州延川縣舊砦,元符二年廢,後復置。



 嗣武砦,舊囉兀城,屬延州,元豐四年置,尋廢。崇寧三年修復,賜名。東至清邊砦二十里,西至鎮邊砦二十里,南至米脂砦三十里,北至龍泉砦二十里。



 龍泉砦,宣和二年,改名通泉,尋復故。東至清邊砦二十里,西至鎮邊砦四十里,南至嗣武城二十里,北至中山堡八里。



 清邊砦,東至河東界五十里,西至龍泉砦二十里,南至暖泉砦七十里,北至生界堠一十三里。以下砦堡,凡不書年月者,皆未詳建置本末。



 鎮邊砦,東至龍泉砦四十里,西至大蟲坑二十五里,南至克
 戎城六十五里,北至生界堠二十五里。



 龍安砦,本屬延安府膚施縣,不詳何年來屬。東至安定堡八十里,西至招安驛四十里,南至金明驛三十五里,北至御安堡四十里。



 海末堡,海末至柏林十六堡。黑水、安定、安塞本延安舊堡。



 窟兒堡,大厥堡,花佛嶺堡,臨川堡,定遠堡,馬欄堡,中山堡,黑水堡,安定堡,佛堂堡,唐推堡,雙林堡,安塞堡,浮圖堡,柏林堡。



 銀州,銀川郡。領儒林、撫寧、真鄉、開光四縣。五代以來為西夏所有,熙寧三年收復,尋棄不守。元豐四年收復。五
 年,即永樂小川築新城,距故銀州二十五里,前據銀州大川,賜名銀川砦,旋被西人陷沒。崇寧四年收復,仍為銀州。五年,廢為銀川城。



 慶陽府,中,安化郡,慶陽軍節度。本慶州。建隆元年,升團練。乾德元年,復為軍事。政和七年,升為節度,軍額曰慶陽。宣和七年,改慶州為府。舊置環慶路經略、安撫使,統慶州、環州、邠州、寧州、乾州,凡五州。其後廢乾州,置定邊軍,已而復置醴州,凡統三州一軍。崇寧戶二萬七千八
 百五十三,口九萬六千四百三十三。貢紫茸白花氈、麝香、黃蠟。縣三:安化,中。有大順一城,府城、東谷、柔遠、人順,四砦。元豐四年,廢府城砦、金村堡、平戎鎮。五年,收復礓詐砦,賜名安疆砦。元祐元年,復平戎鎮。



 合水,望。熙寧四年始置,省華池、樂蟠二縣為鎮。七年,改華池鎮為華池砦。有東華池、有東華池、西華池二砦,荔原一堡。



 彭原。熙寧三年,自寧州來隸。



 安疆砦,本西人礓詐砦,元豐五年收復,賜名。元祐四年,給賜夏人。紹聖四年修復。東至德靖砦九十里,西至東谷砦六十里,南至大順城四十里,北至白豹城四十里。又隸定邊軍。



 橫山砦,地名西攃□移,元符元年進築,賜名。東至東谷砦界攃□移四十五里,西至寧羌砦七十里,南至通塞堡三十里,北至定邊軍三十里。



 通塞堡,元符元年進築。東至東谷砦二十里,西至西谷口砦二十里,南至懷安鎮四十里,北至橫山砦三十里。



 定邊城,元符二年修築,後別為定邊軍。



 白豹城,舊屬西界,元符二年修復,賜舊名。東至安疆砦四十里,西至東谷砦二十里,南至柔遠砦五十里,北至勝羌堡五十里。別見定邊軍。



 綏遠砦,地本駱駝巷,元符二年進築,賜名。東至定邊軍二十里,西至寧羌砦六十里,南至橫山砦五十里,北至神堂砦約五十里。



 寧羌砦,地本萌門三岔,元符元年進築,賜名。東至緩遠砦六十里,西至安塞砦五十里,南至西谷砦三十里,北至王尚原界堠五十里。



 鎮安城,政和六年進築。東至鄜延路通慶城三十里,西至九陽堡二十里,南至威邊砦三十里,北至西界地名蒼雞二十里。



 麥川堡,本名麥經嶺,政和六年賜名。系環慶路,未詳屬何州軍,姑附於此,東至懷威砦二十里,西至矜戎堡二十里,南至威邊砦一十五里,北至鎮安城一十里。



 威寧堡,本名衡家堡,政和六年賜名。系環慶路,未詳屬何州軍,姑附於此。東至九陽堡一十五里,西至
 定邊軍一十五里,南至矜戎堡一十里,北至七逋哆移塔五里。



 矜戎堡,東至懷威堡四十里,西至定邊軍約二十里,南至胡博川二十里,北至通祖盧門城砦五十里。



 府城砦,元豐二年已廢,不知何年修復。



 金村堡,同上。



 勝羌堡,東至洛河川二十里,西至通塞堡約五十里,南至白豹城五十里,北至威邊砦二十里。



 定戎堡,東至啟祖峰二十里,西至那丁原五里,南至興平城二十里,北至清平關一十里。



 威邊砦,東至洛河川二十里,西至橫山砦三十五里,南至勝羌堡二十里,北至鎮安城三十里。



 懷威堡。東至鄜延路通慶城十五里,西至矜戎堡約四十里,南至威寧砦約二十里,北至西界羅輕觜約五十里。



 環州,下,軍事。舊降為通遠軍,淳化五年復為州。崇寧
 戶七千一百八十三,口一萬五千五百三十二。貢甘草。縣一:通遠。上。有烏侖、肅遠、洪德、永和、平遠、定邊、團堡、安塞八砦。



 興平城,地名灰家觜,元符元年築,賜名。東至賀子兒一十里,西至流井堡四十里,南至洪德砦二十里,北至清平關三十里。



 清平關,地名之字平,元符二年進築,賜名。東至鬼通砦二十五里,西至安邊城四十里,南至興平城三十里,北至陷道口餔二十七里。



 安邊城,地名徐丁臺,崇寧五年築,賜名。東至清平關四十里,西至折姜和市賊砦八十里,南至廢肅遠砦一百餘里,北至牛圈界堠二十里。



 羅溝堡,朱灰臺至綏遠砦中路,地名火羅溝及阿原烽,政和三年建築,賜名。東南至綏遠砦約二十里,西南至寧羌砦約六十里,南至阿原堡約四十里,西至朱臺堡約一十五里。



 阿原堡地名見「羅溝堡」,政和三年賜名。東至綏遠砦三十里,西至寧羌砦三十里,南至
 西谷砦四十里,北至羅溝堡約四十里。



 朱臺堡,本朱灰臺,政和三年建築,賜名。東至雞觜堡約一十八里,西至木瓜堡約五十里,南至阿原堡約四十里,北至蕤毛觜約二百步。



 安邊砦,大拔砦,元豐二年已廢,不知何年復修。



 方渠砦,流井堡,東至興平城四十里,西至安邊城三十里,南至黨羅原五里,北至蘭善約五十餘里。



 歸德堡,東至木瓜堡五十里,西至定戎堡約三十里,南至洪德砦四十里,北至蝦



 □麻和市賊砦約四十里。



 木瓜堡,東至寧羌砦二十五里,西至歸德堡五十里,南至惠丁堡四十里,北至界堠里羅節硯五里。



 麝香堡,東至龍札穀五里,西至打米穀八里,南至木瓜原一十五里,北至烏丁原二十里。



 通歸堡,東至歸德堡二十里,西至興平城約三十餘里,南至洪德砦二十里,北至堡子穀約一十里。



 惠丁堡。東至寧羌砦約四十里,西至麝香堡約三十里,南至
 安塞砦約三十五里,北至木瓜堡四十里。



 邠州,緊,新平郡,靜難軍節度。崇寧戶五萬八千二百五十五,口一十六萬二千一百六十一。貢火筋、蓽豆、剪刀。縣五:新平,望。



 宜祿,望



 三水,上。



 定平,緊。熙寧五年,隸寧州。政和七年,自寧州來隸。



 淳化。中。淳化四年,升耀州雲陽黎國鎮為縣。熙寧八年,置鑄錢監,元豐三年廢。宣和元年,自耀州來隸。



 寧州,望,彭原郡,興寧軍節度。本軍事州,宣和元年賜軍額。崇寧戶三萬七千五百五十八,口一十二萬二千四
 十一。貢庵閭、荊芥、硯、席。縣三:定安,緊。



 襄樂,上。



 真寧,下。



 醴州,本京兆府奉天縣。舊置乾州,熙寧五年廢,以奉天還隸府。政和七年,復以縣為州,更名醴。八年,割屬環慶路。縣五:奉天,次畿。



 永壽,下。乾德三年,自邠州來隸。熙寧五年,廢乾州,永壽及麻亭、常寧二砦,俱隸邠州。政和八年復來隸。



 武功,醴泉,二縣本屬京兆府,政和八年三月,割屬醴州。好畤。本屬鳳翔府,政和八年三月,割屬醴州。



 定邊軍。元符二年,環慶路建築定邊城,後改為軍。東至九陽堡三十五里,西至綏遠砦二十里,南至橫山砦三十里,北至通化堡二十里。



 縣一:定邊。政和六年,
 陜西、河東路宣撫使童貫奏:「環慶路已建築勒皈臺等處新城,正據控扼,包占邊面,乞依姚古所請,於定邊軍置倚郭一縣。」詔賜今名。



 白豹城,元符二年建築,賜舊名。已見「慶陽府」。



 東谷砦,舊砦,已見「慶陽府安化縣」。



 綏遠砦,地名駱駝巷,元符二年進築,賜名。東至定邊軍二十里,西至寧羌砦六十里,南至橫山砦五十里,北至神堂堡約五十里。



 神堂堡大觀二年進築,賜名。東至觀化堡三十里,西至綏遠砦多移嶺界堠十三里,南至綏遠砦三十里,北至勤崖原卓望處三里。



 觀化堡,東至逋祖嶺界堠約一十五里,西至雞觜堡約三十里,南至通化堡二十里,北至甜井觜約一十里。



 通化堡,東至逋祖嶺平界堠約三十里,西至綏遠砦二十餘里,南至定邊軍二十里,北至觀化堡二十里。



 九陽堡,東至鎮安城二十
 里,西至定邊軍三十五里,南至東谷砦九十里,北至界堠裏乾穀三里。



 雞觜堡。東至通化堡約二十里,西至多移嶺界堠約一十里,南至綏遠砦一十六里,北至神堂堡約一十四里。



 秦鳳路。府一:鳳翔。州十二:秦,涇,熙,隴,成,鳳,岷,渭,原,階,河,蘭。軍三:鎮戎,德順,通遠。縣三十八。其後增積石、震武、懷德三軍,西寧、樂、廓、西安、洮、會六州,又改通遠軍為鞏州。凡府一,州十九,軍五,縣四十八。



 秦州,下府,天水郡,雄武軍節度。舊置秦鳳路經略安撫使,統秦州、隴州、階州、成州、鳳州、通遠軍,凡五州一軍,其後割通遠軍屬熙河,凡統州五。崇寧戶四萬八千六百
 四十八,口一十二萬三千二十二。貢席、芎窮。縣四:成紀,上。有渭水、三陽、上蝸牛、下蝸牛、堡子、伏歸、小三陽、照川、土門、四顧、平戎、赤崖湫、西青、遠近湫、定西、小定西、下硤、注鹿原、上硤、圓川、伏羌、得勝、榆林、大像、菜園、探長、新水穀、舊水穀、檉林、丙龍、石人鋪、駝項、永寧、鹽泉、小永寧、冷水泉、雙泉、新土、舊土三十九堡。



 隴城,中。有靜戎、永固、定平、長山、白榆林、郭馬、安塞七堡。



 清水,中有弓門、鐵窟、斫鞍、堡子、小弓門、坐交、得鐵、冶坊、橋子、李子、古道、永安、四顧、威塞、床穰、鎮邊、和戎、安遠、挾河、定川、中城、東城、西城、靜邊、臨川、德威、廣武、寧遠、長樵二十九堡。



 天水。上。



 監一:太平。城二:伏羌,熙寧三年,廢丹山、納述、乾川三堡、增伏羌砦為城,有得勝、榆林、大像、菜園、探長、新水、檉林、丙龍、石人、駝項、舊水一十一堡。



 甘谷。熙寧元年置,有吹藏、大甘、隴諾三堡。四年,置尖竿、隴陽二堡。



 砦七:治平四年,
 置雞川。熙寧元年,改攃珠堡為通渭堡。五年,改古渭砦為通遠軍,廢者達、本當、七麻三堡,改通渭堡為砦,割永寧、寧遠、威遠、熟羊、來遠並隸軍。尋改綏遠、定邊二砦為鎮,隸隴州。



 定西,領寧西、牛鞍、上硤、下硤、注鹿原、圓川六堡。



 三陽,領渭濱、武安、上下蝸牛、聞喜、伏歸、硤口、照川、土門、四顧、平戎、赤崖湫、西青、遠近湫十四堡。



 弓門,領東鞍、安人、斫鞍、上下鐵窟、坐交、得鐵、治坊七堡。



 靜戎,領白榆林、長山、郭馬、靜塞、定平、永固、邦蹉、寧塞、長燋九堡。



 安遠,隴城,雞川。堡三:熙寧三年,改┐穰為鎮。五年,改冶坊砦為冶坊堡。八年,改┐穰鎮為堡。



 ┐穰,領白石、古道、中城、東城、西城、定戎、定安、雄邊、臨川、德威、廣武、定川、挾河、鎮邊一十四堡。



 冶坊,領橋子、古道、永安、博望、威塞、李子六堡。



 達隆。堡川城,政和六年,於秦鳳東西川口進築,賜名。東至甘泉堡一十八里,西至熙河路安西城管下龜兒鎮一十
 二里,南至甘谷城一百一十里,北至會川城一百二十里。



 甘泉堡,東至涇原路第十七堡五十里,西至堡川城一十八里,南至涇原路治平砦一百五十里,北至涇原路通安砦一百五十里。別見「渭州」。



 安遠砦。《吏部通用酬賞格》:秦州又有安遠等五砦,定邊、綏遠二砦。熙寧八年,廢為鎮,屬隴州,其後,復為砦。



 定邊砦,綏遠砦,小落門砦,保安砦,弓鐘砦,董哥平砦。



 鳳翔府,次府,扶風郡,鳳翔軍節度。乾德初,置崇信縣。淳化中,割崇信屬儀州。熙寧五年,廢乾州,以好畤縣來隸。政和八年,又以好畤隸醴州。崇寧戶一十四萬三
 千三百七十四,口三十二萬二千三百七十八。貢蠟燭、榛實、席。縣九:天興,次赤。



 岐山,次畿。



 扶風,次畿,盩厔次畿。大觀元年,以縣清平鎮置軍。



 郿,次畿。有鐵冶務。



 寶雞,次畿。



 虢次畿。



 麟游,次畿。



 普潤。次畿。



 監一:司竹。



 隴州,上,汧陽郡,防禦。崇寧戶二萬八千三百七十,口八萬九千七百五十。貢席。縣四:汧源,望。有古道銀場。熙寧八年,改秦州定邊砦為隴西鎮,隸縣。



 汧陽,緊



 吳山,中。



 隴安。中。開寶二年,析汧陽縣四鄉置縣。



 成州,中下,同谷郡。開寶六年,升為團練。崇寧戶一萬二千九百六十四,口三萬三千九百九十五。貢蠟燭、鹿茸。
 縣二:同谷,上。有骨鹿、馬邑、赤土、平原、滔山、胡桃六砦。



 慄亭。中。



 鳳州,下,河池郡,團練。本防禦,乾德元年,降為團練。崇寧戶三萬七千七百九十六,口六萬一千一百四十五。貢蜜、蠟燭。縣三:梁泉,上。



 河池,緊。開寶五年,移治固鎮。有水銀務。



 兩當。上。至道元年,移治廣鄉鎮。



 監一:開寶。建隆三年,於兩當縣置銀冶。開寶五年,升為監。治平元年罷置官,以監隸兩當縣,元豐六年廢。



 階州,中下,武都郡,軍事。本唐武州。陷西戎,後復其地改置焉。崇寧戶二萬六百七十四,口四萬九千五百二十。
 貢羚羊角、蠟燭。縣二:福津,中下。領峰貼硤武平沙灘三砦、團城堡、平定關。



 將利。中下



 砦一:故城。本故城鎮,不知何年建為砦。



 渭州,下,隴西郡,平涼軍節度。本軍事,政和七年,升為節度。舊置涇原路經略、安撫使,涇州、原州、渭州、儀州、德順軍、鎮戎軍皆屬。熙寧五年,廢儀州。元符二年,增置西安州。崇寧三年,又以熙河路會州來屬。大觀二年,又增置懷德軍。凡統五州三軍。崇寧戶二萬六千五百八十四,口六萬三千五百一十二。貢絹、蓯蓉。縣五:平涼,中。有瓦亭砦。



 潘原,中



 安化,中。熙寧七年,廢制勝關,移縣於關地,以舊地為鎮。



 崇信,中。



 華亭。中下。熙寧五年,廢儀州,與安化、崇信同來隸。



 靖夏城,政和六年,賜涇原路席葦平新城名曰靖夏。未詳屬何軍州,姑附此。



 甘泉堡。崇寧五年,涇原路經略司於甜井子修築守御,賜名。未詳屬何州軍,姑附此。別見「秦州」。



 涇州,上,安定郡。太平興國元年,改彰化軍節度。崇寧戶二萬八千四百一十一,口八萬八千六百九十九。貢紫茸、毛毼段。縣四:保定,望。有長武砦。



 靈臺,上。



 良原,上。



 長武。望。咸平四年,升長武鎮為縣。五年,省為砦,屬保定縣。大觀二年,復以砦為縣。



 原州,望,平涼郡,軍事。崇寧戶二萬三千三十六,口六萬三千四百九十九。貢甘草。縣二:臨涇,中。



 彭陽。中。唐豐義縣,太平興國初改。至道三年,自寧州來隸。



 鎮二:新城,熙寧三年,廢截原砦入焉。



 柳泉。領耳朵城一砦。



 砦五:開邊,熙寧三年,廢新門砦入焉。



 西壕,平安,綏寧,領羌城、南山、顛倒三堡。



 靖安。領中普、吃囉岔、中嶺、張巖、常理、新勒、雞川、立馬城、殺獐川九堡。



 安羌堡,新城堡。



 德順軍,同下州。慶歷三年,即渭州隴乾城建為軍。崇寧戶二萬九千二百六十九,口一十二萬六千二百四十一。貢甘草。縣一:隴乾。元祐八年,以外底堡置。



 城
 一:水洛。領王家城、石門堡。



 砦五:靜邊,別見「鎮戎軍」。



 得勝,領開邊堡。



 隆德,通邊,治平。治平四年置,領牧龍堡。



 懷遠砦,東至鎮戎砦六十里,西至得勝砦三十里,南至張義堡四十里,北至鎮羌砦二十七里。



 中安堡,威戎堡。東至章川堡三十里,西至同家堡二十五里,南至治平砦四十里,北至靜邊砦二十里。



 鎮戎軍,同下州。本原州平高縣之地。至道三年,建為軍。崇寧戶一千九百六十一,口八千五十七。貢白氈。城一:彭陽。砦七:治平四年,置信岔堡、涼棚堡。熙寧元年,置熙寧砦、硝坑堡、東西水口堡。元豐四年,廢東水口堡。六年,置故砦堡。



 東山,乾興,天聖,有信岔、涼棚二堡。



 三川,高平,
 有故砦堡。



 定川,熙寧。有硝坑堡。



 堡二:開遠,張義。熙寧四年,廢安邊堡入開遠。五年,置張義。



 平夏城,舊石門城,紹聖四年賜名。大觀二年,升為懷德軍。



 靈平砦,舊好水砦,紹聖四年賜名。大觀二年,割屬懷德軍。



 鎮羌砦,紹聖四年賜名。東至三川堡二十一里,西至寺子岔堡二十五里,南至懷遠砦二十七里,北至九羊砦二十五里。



 高平堡,元符元年修復,賜舊砦名。



 威川砦,政和七年賜名,本密多臺。



 飛泉砦,政和七年賜名。本飛井塢。



 飛井堡,乾興砦管下。



 狼井堡,熙寧砦管下狼井、安遠、竇信、梅谷、開疆,凡五堡。



 安遠堡,竇信岔堡,梅穀堡,開疆堡,李家堡,肅遠堡,堎地平堡,鎮西堡,水口堡,懷遠城,別見「德順軍」。



 德靖砦,保安軍舊有德靖砦,
 自屬鄜延路。



 靜邊砦。天禧舊砦,屬德順軍。東至德順軍。七十里,西至第十七堡三十五里,南至威戎堡三十里,北至隆德砦五十里。



 會州。元豐五年,熙河路加「蘭會」二字,時未得會州。元符二年,始建築,割安西城以北六砦隸州。崇寧三年,置倚郭縣曰敷文,又以會州隸涇原路。縣一:敷文。安西城,舊名汝遮,紹聖三年建築,賜名,屬熙河路。東至秦鳳路界六十二里,西至原川子一百里,南至定西砦二十七里,北至平西砦三十三里。



 平西砦,紹聖四年賜名,地本青石硤,屬熙河路。東至秦鳳路界三十餘里,西至勝如堡一百一十里,南至安西城三十三里,北至會寧關四十四里。



 會寧關,舊名顛耳關,元符元
 年建築,賜名通會,未幾改今名,屬秦鳳路。東至涇原路元和市七里,西至熙河路定遠城分界五十里,南至熙河路平西砦四十里,北至黃河南岸古烽臺一百餘里。



 會川城,舊名青南訥心,元符二年建築,賜名,屬秦鳳路。東至涇原路通安砦六十里,西至熙河定路遠城一百五十里,南至會寧關六十里,北至新泉砦四十里。



 新泉砦,舊名東北冷牟,元符元年賜名,屬秦鳳路。東至懷戎堡界白草原三十里,西至會川城界粗兒原三十五里,南至會川城三十里,北至會州四十里。



 懷戎堡,崇寧二年築,屬秦鳳路。東至涇原路分界定戎砦地分二十二里半,西至本堡管下水泉堡二十里,由香谷至會州共六十里,南至會川城分界三十五里,北至柔狼山界堠四十里,系與夏國西壽監軍地對境,經由枯柴穀至柔狼山,有險隘去處。



 德威城,政和六年,築清水河新城,賜名,屬秦鳳路。東至麻累山二十五里,西至黃河四里,河北倚卓囉監軍地分,
 水賊作過去處,南至囉迷穀口新移正川堡二十五里,北至北浪口至馬練賊城約二十餘里。



 靜勝堡,政和六年,賜清水河新城接應堡名靜勝,會川城管下。新修築靜勝堡,不系守禦處,在黃河南石觜上,至本城一百二十里,河北岸與夏國卓囉監軍地分相對。



 通泉堡,屬秦鳳路新泉砦管下,不系守禦處,在黃河南嶺上,至本砦四十里,與河北岸夏國卓囉監軍地分相對。



 水泉堡,系懷戎堡管下,距本堡二十里,不系守禦處。



 正川堡。系德威城管下,囉迷穀口新移正川堡距本處二十五里,不系守禦處。



 懷德軍。平夏城。紹聖四年建築。大觀二年,展城作軍,名曰懷德,以蕩羌、靈平、通峽、鎮羌、九羊、通遠、勝羌、蕭關隸之,增置將兵,與西安、鎮戎互為聲援應接。蕭關初名
 威德,又改今名。東至結溝堡一十五里,西至石門堡一十八里,南至靈平砦一十二里,北至通峽砦一十八里。蕩羌砦,故沒煙後峽,元符元年建築,賜名。東至通峽砦一十八里,西至正原堡四十里,南至石門堡三十里,北至蕭關一百三十五里。



 通峽砦,故沒煙前峽,元符元年建築,賜名。東至東彎堡七里,西至蕩羌砦一十八里,南至懷德軍一十八里,北至勝羌砦八十里。



 靈平砦,故好水砦,紹聖四年賜名。大觀二年,自鎮戎軍來屬。東至古高平堡一十五里,西至九羊砦三十二里,南至熙寧砦二十八里,北至懷德軍一十二里。



 硤口堡,東河灣堡,古高平堡,惠民堡,結溝堡,系通峽砦管下五堡。



 鎮羌堡,東至三川堡二十八里,西至寺子岔堡二十五里,南至懷遠砦二十
 七里,北至九羊砦二十五里。



 九羊砦,故九羊穀,元符元年建築,賜名。東至靈平砦三十里,西至寧安砦六十六里,南至三川砦五十里,北至Z臨羌砦八十里。



 石門堡,故石門峽東塔子觜,元符元年建築,賜名。



 通遠砦,東至龍泉穀三十五里,西至臨羌砦六十五里,南至通峽砦五十里,北至勝羌砦三十三里。



 龍泉堡,通遠砦管下。



 勝羌砦,東至漫□移口七里,西至寧韋堡四十里,南至通峽砦八十里,北至蕭關六十里。



 蕭關,崇寧四年建築。東至葫蘆河一十五里,西至綏寧堡三十里,南至勝羌砦六十里,北至臨川堡一十八里。



 臨川堡,通關堡,山西堡。系蕭關管下。



 西安州。元符二年,以南牟會新城建為西安州。東至天都砦二十六里,西至通會堡五十五里,南至寧安砦一
 百里,北至囉沒寧堡三十五里。蕩羌砦,地名沒煙峽,元符元年建築,賜名。後屬懷德軍。



 通會堡,元符元年賜名,系熙河蘭會路修築,地名祭廝堅穀口,不知何年撥屬涇原路西安州。



 天都砦,元符二年,灑水平新砦賜名天都。東至臨羌砦二十里,西至西安州二十六里,南至天都山一十里,北至綏戎堡六十五里。



 臨羌砦,元符二年,秋葦平新砦賜名臨羌。東至通遠砦六十五里,西至天都砦二十里,南至定戎砦八十里,北至綏戎砦七十里。



 橫嶺堡,系天都砦管下。



 寧韋堡,定戎堡,元符二年賜名,地本堿隈川。東至山前堡三十里,西至秦鳳路分界堠一十二里,南至通安砦一百里,北至劈通流界堠五十里。



 劈通川堡,囉沒寧堡,北嶺上堡,山前堡,高峰堡,寧安砦,崇寧五年,武延川峗朱龍山下新砦
 賜名寧安。東至九羊砦六十六里,西至通安砦六十一里,南至得勝砦九十里,北至西安州一百里。



 那羅牟堡,寺子岔堡,石棚泉堡,通安砦,崇寧五年,烏雞三岔新砦賜名通安。東至寧安砦六十一里,西至同安堡三十五里,南至甘泉堡一百五里北至定戎砦一百里。



 同安堡,系通安砦管下。



 綏戎堡,管下秋葦川口堡、鍬钁川中路堡、征通谷中路東水泉堡,皆不詳建置始末。東至蕭關三十里,西至山前堡三十五里,南至臨羌砦七十里,北至枅柂嶺界堠五十里。



 秋葦川堡,鍬钁川中路堡,徵通谷中路東水泉堡。



 熙州,上,臨洮郡,鎮洮軍節度。本武勝軍。熙寧五年收復,始改焉。尋為州。初置熙河路經略、安撫使,熙州、河州、洮
 州、岷州、通遠軍五州屬焉。後得蘭州,因加「蘭會」字。元祐改熙河蘭會路為熙河蘭岷路,元符復故。會州既割屬涇原,又改為熙河蘭廓路,宣和又改為熙河湟廓路,又改湟州為樂州,又改為熙河蘭樂路,尋復改為熙河蘭廓路。舊統五州軍,蘭、廓、西寧、震武、積石六州軍相繼來屬,又改通遠軍為鞏州,凡統九州、三軍。崇寧戶一千八百九十三,口五千二百五十四。貢毛毼段、麝香。縣一:狄道。中下。熙寧六年置,九年省。元豐二年復置。



 砦一:康樂。熙寧六年,置康樂城為砦,省馬鬃砦。馬鬃
 砦舊屬秦州長道縣。



 堡九:熙寧五年,置慶平、通谷、渭源、北關。六年,改劉家川為當川,置南關、南川。七年,置結河。元豐七年,置臨洮。



 通谷,慶平,渭源,結河,南川,當川,南關,北關,臨洮。東至定遠城四十里,西至定羌城界三十五里,南至熙州六十五里,北至阿千堡七十里。



 安羌城,宣和六年,賜熙河蘭廓路新建溢機堡名為安羌城,不知屬何州軍,姑附於此。



 廣平堡。



 河州,上,安鄉郡,軍事。熙寧六年收復。崇寧戶一千六十一,口三千八百九十五。貢麝香。縣一:寧河。熙寧六年,置枹罕縣,九年省。崇寧四年,升寧河砦為縣。舊香子城。



 城一:定羌。熙寧七年,改河諾城為定羌城。



 砦一:
 南川。熙寧七年,置南山堡,尋改為南川砦。



 堡四:熙寧七年,置東谷。八年,置閻精。元豐三年,置西原、北河二堡。



 東谷,閻精,西原,北河。關一:通會。熙寧七年置。



 循化城,舊一公城,崇寧二年收復,改今名。別見「樂州」。東至懷羌城四十五里,西至積石軍界一百餘里,南至下橋家族地分一百餘里,北至來同堡六十五里。



 大通城,舊達南城,崇寧二年收復,改今名。別見「樂州」。東至通津堡界十五里,西至菊花河六十里,南至撲水原二十一里,北至寧塞堡界十五里。



 安疆砦,舊名當標城,崇寧二年收復,改今名。別見「樂州」。東至來同堡三十三里,西至通津堡五十里,南至循化城一百一十里,北至黃河二十里。



 懷羌城,崇寧三年,王厚收復。東至南川砦六十里,西至循化城六十五里,南至洮州界一百七十餘里。北至安疆砦一百七十里。



 來羌城,崇寧三年,王厚收復。東至安鄉關七
 十里,西至大通城界三十八里,南至南川界四十八里,北至黃河二十里。



 講朱城,元符二年,洮西安撫司收復河南講朱、一公、錯鑿、當標、彤撒、東迎六城,尋棄之。崇寧二年,再收復。除一公改循化城,當標改安疆砦,餘四城皆未詳。按:講朱、錯鑿、一公、當標皆在河州之南,元符二年,邊廝波結先以此四城來降,未幾,王贍乃進據之。



 錯鑿城,彤撒城,東迎城,寧河砦,崇寧四年,已升寧河砦為縣,別有寧河砦。東至定羌城三十里,西至河州四十五里,南至通會關三十里,北至河州界二十里。



 來同堡,舊名甘撲堡,崇寧三年築,賜今名。東至南川砦九十里,西至安疆砦三十五里,南至懷羌城三十五里,北至來羌城三十里。



 通津堡,舊名南達堡,崇寧三年賜今名。東至安疆砦四十五里,西至大通城界二十五里,南至循化城一百三十里,北至大通城界二十里。



 南山堡,《元豐九域志》屬原州綏寧縣。



 安鄉
 關,舊城橋關,元符三年賜名。東至京玉關界四十里,西至臨灘堡四十里,南至河州界三十五里,北至安川堡界一十五里。



 臨灘堡。東至安鄉關四十里,西至古雞山二十里,南至南川砦界二十里,北至黃河四十里。



 鞏州,下。本通遠軍。熙寧五年,以秦州古渭砦為軍。崇寧三年,升為州。崇寧戶四千八百七十八,口一萬一千八百五十七。貢麝香。縣三:隴西,元祐五年增置。



 永寧,寧遠。崇寧三年,升永寧砦為縣,又升寧遠砦為縣。城一:定西。元豐四年,以蘭州西使城為定西城。五年,改定西城為通遠軍,以汝遮堡為定西城,屬通遠軍。崇寧二年,廢定西城管下熨斗平堡,通西砦管下榆木岔堡,並安西
 城。別見「蘭州」。東至龜兒觜鎮六十五里,西至龕穀堡一百一十五里,南至通西砦四十六里,北至安西城二十七里。砦六:熙寧五年,割秦州永寧、寧遠、威遠、通渭、熟羊、來遠六砦隸軍。六年,置鹽川砦。八年,廢威遠砦為鎮。元豐五年,收通西砦。七年,廢來遠砦為鎮,屬永寧。崇寧五年,通渭縣復為砦,未詳何年以砦為縣。



 永寧,寧遠,崇寧三年,與永寧同升為縣。



 通渭,東至甘泉城五十五里,西至鞏州六十四里,南至來遠鎮一百里,北至甘泉城界六十里。



 熟羊,鹽川,熙寧六年九月置砦,後改為鎮。



 通西。東至甘泉城一百二十里,西至熟羊砦七十里,南至三岔堡四十八里,北至定西砦四十八里。



 堡七:熙寧五年,割秦州三岔、乜羊、廣吳、渭川、啞兒五堡隸軍。七年,以岷州遮羊堡來隸。元豐元年,遮羊復隸岷州。五年,置榆木岔、熨斗平二砦堡。七年,廢乜羊、廣吳、渭川、啞兒四堡。



 三岔,舊堡,熙寧四年置。



 榆木岔,崇寧二年廢。



 熨
 斗平,崇寧二年廢。



 者達堡,秦州,熙寧五年改古渭砦為通遠軍,廢者達、本當、七麻堡。今通渭乃領七麻堡,不知何年復置者達、本當堡。



 七麻堡,本當堡,撲麻龍堡。



 岷州,下,和政郡,團練。熙寧六年收復。崇寧戶四萬五百七十,口六萬七千七百三十一。貢甘草。縣三:祐川,唐縣。崇寧三年復。



 大潭,中。建隆三年,合良恭、大潭兩鎮置縣,隸秦州。熙寧七年,自秦州來隸。長道。緊,熙寧七年,自秦州來隸。



 砦五:秦州臨江砦,熙寧六年,割隸州。七年,置荔川、┐川、閭川,又置宕昌。



 臨江,荔川,┐川,閭川,宕昌。堡三:熙寧六年,以秦州馬務堡隸州。七年,置遮羊堡,尋改為鎮。十年,置鐵城堡。元豐元年,遮羊堡復隸於州。



 遮羊,穀藏,並熙寧七年置。



 鐵城。熙寧十年置。



 監一:滔山。熙寧九年置,鑄鐵錢。



 蘭州,下,金城郡,軍事。元豐四年收復。崇寧戶三百九十五,口九百八十一。貢甘草。縣一:蘭泉。崇寧三年置,倚郭。



 砦一:元豐四年,置龕谷、吹龍二砦。七年,割吹龍屬阿千堡。



 龕穀。元祐七年廢。紹聖三年,復修為堡。東至定遠砦一百里,西至阿千堡七十里,南至通谷堡一百二十里,北至定遠城三十里。



 堡二:元豐四年,置皋蘭堡、鞏哥關。五年,置西關、勝如、質孤堡。六年,改鞏哥關作東關堡,廢西關、勝如、質孤堡,置阿千堡。七年,廢皋蘭堡。元祐五年,復修勝如、質孤二堡,尋廢。



 東關,東至質孤堡三十六里,西至蘭州一十八里,南至屈金支山三十里,北至黃河不及里。



 阿千。有阿千水。東至屈金支山二十五里,西至西關堡界二十里,南至臨洮堡七十
 里,北至蘭州界三十七里。



 定西城,元豐四年,以蘭州西使城為定西城。五年,改定西城為通遠軍,以汝遮堡為定西城,屬通遠軍。別見「鞏州」。



 定遠城,元祐七年築,舊名李諾平,本龕穀砦,因地窄及無水,故廢之,改築為定遠軍城。東至安西城八十里,西至東關堡五十里,南至龕穀堡三十里,北至黃河一百七里。



 金城關,紹聖四年建築,南距蘭州約二里。崇寧三年,王厚乞移置斫龍谷口,不行。



 京玉關,元符三年賜名,本號把拶橋。東至西關堡四十里,西至通川堡四十里,南至臨洮堡一百三十九里,北至□□六嶺分界三十里。



 通川堡。元符三年,自京玉關至囉□兀抹通城中路鏹廝狐川新築堡,賜名,尋棄之。崇寧二年,再收復。東至京玉關四十里,西至通湟砦四十里,南至圓子堡約九里,北至□□六嶺分界八十里。別見「樂州」。



 洮州。唐末陷於吐蕃,號臨洮城。熙寧五年,詔以熙、河、洮、
 岷、通遠軍為一路,時未得洮州。元符二年得之,尋棄不守。大觀二年收復,改臨洮城仍舊為洮州。三年,升團練。東至岷州界一百一十三里,西至喬家族生界二百里,南至魯黎族生界一百五里,北至河州界一百二十里。通岷砦。東至鐸龍橋六十七里,西至洮州四十里,南至洮河二十里,北至熙州界五十五里。



 廓州,元符二年,以廓州為寧塞城。崇寧三年棄之,是年收復,仍為廓州。城下置一縣,五年罷。大觀三年,為防禦。東至寧塞砦一十七里,西至同波北堡不及里,南至黃
 河不及里,北至膚公城界十五里。膚公城,舊名結囉城,崇寧三年收復,後改今名。王厚云:結囉城至廓州約三十餘里。東至來賓城界一百三里,西至懷和砦界五十七里,南至同波北堡界一十三里,北至綏平堡界二十五里。



 綏平堡,舊名保敦谷,崇寧三年興築,賜名。東至保塞砦界二十里,西至清平砦界二十里,南至膚公城界二十里,北至保塞砦界一十七里。



 米川城,舊米川縣,崇寧三年修築。王厚云:米川沿河西至廓州約六十里,過河取正路至結囉城約三程,本城至廓州約三十餘里。



 寧塞砦,東至河北堡界四十五里,西至廓州巡檢界一十三里,南至黃河一十五里,北至龍支城界五十里。



 同波堡。東至廓州巡檢界一十里,西至膚公城界一十五里,南至黃河不及里,北至膚公城一十五里。



 樂州。舊邈川城,元符二年收復,建為湟州,建中靖國元年棄之。崇寧二年又復。三年,置倚郭縣,五年罷。大觀三年,加向德軍節度。宣和元年,改為樂州。東至把拶宗六十里,西至龍支城界六十里,南至來羌城界一百四十里,北至界首賒□兀嶺一百一十里。通湟砦,故囉□兀抹通城,元符二年收復,三年賜名。東至通川堡四十里,西至湟州三十五里,南至安隴砦二十五里,北至臨宗砦界六十里。別見「蘭州」。



 寧洮砦,故瓦吹砦,元符二年收復,三年賜名。東至通湟砦四十五里,西至來賓城一十七里,南至來賓城界二十里,北至安隴砦界一十七里。



 安隴砦,故隴朱黑城,元符二年收復,三年賜名。東至赤
 沙嶺三十里,西至麻宗山腳二十五里,南至鞏藏嶺三十五里,北至湟州四十五里。



 安川堡,故□哥堡,在巴金嶺上,元符二年收復,三年賜名。東至湟州界七十里,西至來賓城界四十里,南至安鄉關三十里,北至寧川堡四十里。



 寧川堡,元符二年收復,三年賜名,尋棄之。崇寧二年,再收復。



 綏遠關,舊名灑金平,崇寧二年建築,賜今名。東至湟州二十里,西至勝宗谷口三十里,南至麻宗山腳五十五里,北至丁星原四十里。



 來賓城,舊名□□當川,崇寧三年賜名。東至安川堡分界七十里,西至青丹穀三十里,南至黃河一十里,北至安隴砦七十里。



 大通城,舊名達南城,東至通津堡界十五里,西至菊花河六十里,南至撲水原二十一里,北至寧川堡界一十五里。



 循化城,舊名一公城,別見「河州」。東至懷羌城四十五里,西至積石軍界一百餘里,南至下喬家族地分一百餘里,北至來同堡六十五里。



 安疆砦,舊名當標砦,與
 大通、循化皆崇寧二年改。別見「河州」。東至來同堡三十三里,西至通津堡五十里,南至循化城一百一十里,北至黃河二十里。



 德固砦,舊名勝鐸谷,崇寧三年築五百步城,後賜名德固砦。東至綏遠關界一十里,西至龍支城界二十里,南至渴驢嶺一十里,北至清江山腳二十里。



 臨宗砦,崇寧三年賜名。南宗堡稍南一十五里乳駱河之西。東至三諾鞏哥嶺五十餘里,西至丁星原約三十餘里,南至湟州分界二十一里,北至界首抹牟嶺七十里。



 通川堡,崇寧二年,王厚收復,系湟州管下。別見「蘭州」。東至京玉關四十里,西至通湟砦四十里,南至圓子堡約九里,北至□□六嶺分界八十里。



 南宗堡,元符二年,與囉□兀抹通城同收復,尋棄之。後再收復。



 峽口堡。與通川、南宗堡皆崇寧二年王厚收復。



 西寧州。舊青唐城。元符二年,隴拶降,建為鄯州,仍為隴
 右節度,三年棄之。崇寧三年收復,建隴右都護府,改鄯州為西寧州,又置倚郭縣。賜郡名曰西平,升中都督府。三年,加賓德軍節度。五年,罷倚郭縣。東至保塞砦五十七里,西至寧西城四十里,南至清平砦五十里,北至宣威城五十里。龍支城,舊宗哥城,元符二年改今名,尋棄之。崇寧三年收復。東至德固砦界一十八里,西至保塞砦藥邦硤二十二里,南至廓州界分水嶺四十里,北至習令波族分界八十五里。



 寧西城,舊名林金城,改今名。東至湯廝甘二十里,西至廝哥羅川一百里,南至京雕嶺二十里,北至金谷峗四十里。



 清平砦,舊名溪蘭宗堡,後改賜砦名。東至廓州綏平堡界三十五里,西至赤嶺鐵堠子一百
 二十里,南至懷和砦界二十五里,北至西寧州界二十五里。



 保塞砦,舊名安兒城。以上城砦皆崇寧三年收復,賜名。東至龍支城界二十二里,西至西寧州界三十里,南至廓州界二十里,北至青歸族一十五里。


宣威城,舊名
 \gezhu{
  牧瓦}
 牛城,崇寧三年,改今名。東至綏邊砦四十里,西至寧西城界三十五里,南至西寧州界二十五里,北至南宗嶺九十里。



 綏邊砦,舊名宗谷,崇寧三年建築,後改今名。東至龍支城界六十里,西至宣威城界三十里,南至西寧州界三十二里,北至乳駱河界南一里。



 懷和砦,舊名丁令谷,崇寧三年置砦,賜名,又隸積石軍。東至廓州界八十三里,西至青海一百三十餘里,南至順通堡界一十三里,北至清平砦界二十五里。



 制羌砦。政和八年賜名。地名□□氈嶺,屬西寧州。



 震武軍。政和六年,建築古骨龍城,賜名震武城。未幾,改
 為震武軍。不見四至,據童貫奏,古骨龍元屬湟州。通濟橋,震武城浮橋,政和六年賜名。善治堡,政和六年,震武城通濟橋堡賜名。



 大同堡,本名古骨龍城應接堡,政和六年賜名。



 德通城,本瞎令古城,政和七年,劉法既解震武軍圍,建築,賜名。



 石門堡。瞎令古城北,地名石門子,政和七年賜名。



 積石軍。本溪哥城。元符間,為吐蕃溪巴溫所據。大觀二年,臧征撲哥以城降,即其地建軍。東至廓州界八十里,西至青海一百餘里,南至蓋龍峗八十里,北至西寧州界八十里。懷和砦,已見「西寧州」。東至廓州界八十五里,西至青海一百三十餘里,南至順通
 堡界一十三里,北至清平砦界二十五里。



 順通堡,東至臨松堡一十二里,西至本軍一十八里,南至臨松堡二十五里,北至懷和砦界一十二里。



 臨松堡。東至廓州界五十里,西至順通堡界一十二里,南至把拶公原界約六十里,北至黃河一十五里。



 陜西路,蓋《禹貢》雍、梁、冀、豫四州之域,而雍州全得焉。當東井、輿鬼之分,西接羌戎,東界潼、陜,南抵蜀、漢,北際朔方。有銅、鹽、金、鐵之產,絲、枲、林、木之饒,其民慕農桑,好稼穡。鄮杜、南山,土地膏沃,二渠灌溉,兼有其利。大抵誇尚氣勢,多游俠輕薄之風,甚者好鬥輕死。蒲、解本隸河東,
 故其俗頗純厚。被邊之地,以鞍馬、射獵為事,其人勁悍而質木。梁泉少桑麻之利,布泉、鹽酪資於他郡。上洛多淫祀,申以科禁,故其俗稍變。秦、隴、儀、渭、涇、原、邠、寧、鄜、延、環、慶等皆分兵屯守,以備不虞云。



\end{pinyinscope}