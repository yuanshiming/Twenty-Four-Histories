\article{志第五十二 禮二(吉禮二)}

\begin{pinyinscope}

 南郊



 南郊壇制。梁及後唐郊壇皆在洛陽。宋初始作壇於東都南熏門外,四成、十二陛、三壝。設燎壇於內壇之外丙
 地,高一丈二尺。設皇帝更衣大次於東壝東門之內道北,南向。仁宗天聖六年,始築外壝,周以短垣,置靈星門。親郊則立表於青城,表三壝。神宗熙寧七年,詔中書、門下參定青城殿宇門名。先是,每郊撰進,至是始定名,前門曰泰禋,東偏門曰迎禧,正東門曰祥曦,正西門曰景曜,後三門曰拱極,內東側門曰夤明,西側門曰肅成,殿曰端誠,殿前東、西門曰左右嘉德,便殿曰熙成,後園門曰寶華,著為定式。元豐元年二月,詔內壝之外,眾星位
 周環,每二步植一杙,繚以青繩,以為限域。既而詳定奉祀禮文所言:「《周官》外祀皆有兆域,後世因之,稍增其制。國朝郊壇率循唐舊,雖儀注具載圜丘三壝,每壝二十五步,而有司乃以青繩代內壝,誠不足以等神位、序祀事、嚴內外之限也。伏請除去青繩,為三壝之制。」從之。



 徽宗政和三年,詔有司討論壇壝之制。十月,禮制局言:「壇舊制四成,一成二十丈,再成十五丈,三成十丈,四成五丈,成高八尺一寸;十有二陛,陛十有二級;三壝,二十五
 步。古所謂地上圜丘、澤中方丘,皆因地形之自然。王者建國,或無自然之丘,則於郊澤吉土以兆壇位。為壇之制,當用陽數,今定為壇三成,一成用九九之數,廣八十一丈,再成用六九之數,廣五十四丈,三成用三九之數,廣二十七丈。每成高二十七尺,三成總二百七十有六,《乾》之策也。為三壝,壝三十六步,亦《乾》之策也。成與壝地之數也。」詔行之。



 建炎二年,高宗至揚州,庶事草創,築壇於州南門內江都縣之東南,詔東京所屬
 官吏奉祭器、大樂、儀仗、法物赴行在所。紹興十三年,太常寺言:「國朝圓壇在國之東南,壇側建青城齋宮,以備郊宿。今宜於臨安府行宮東南修建。」於是,遂詔臨安府及殿前司修建圓壇,第一成縱廣七丈,第二成縱廣一十二丈,第三成縱廣一十七丈,第四成縱廣二十二丈。一十二陛,每陛七十二級,每成一十二綴。三壝,第一壝去壇二十五步,中壝去內壝、外壝去中壝各半之。燎壇方一丈,高一丈二尺,開上南出戶,方六尺,三出陛,在壇
 南二十步丙地。其青城及望祭殿與行事陪祠官宿齋幕次,並令絞縛,更不修蓋。先是,張杓為京尹,議築齋宮,可一勞永逸,宇文價曰:「陛下方經略河南,今築青城,是無中原也。」遂罷役。



 神位。元豐元年十一月,詳定郊廟奉祀禮文所言:「按東漢壇位,天神從祀者至千五百一十四,故外設重營,以為等限。日月在中營內南道,而北斗在北道之西,至於五星中宮宿之屬,則其位皆中營,二十八宿外宮星之
 屬,則其位皆外營。然則為重營者,所以等神位也。唐因隋制,設為三壝,天神列位不出內壝,而御位特設於壇下之東南。若夫公卿分獻、文武從祀,與夫樂架饌幔,則皆在中壝之內,而大次之設乃在外壝。然則為三壝者,所以序祀事也。」



 景德三年,鹵簿使王欽若言:「漢以五帝為天神之佐,今在第一龕;天皇大帝在第二龕,與六甲、岳瀆之類接席;帝座,天市之尊,今與二十八宿、積薪、騰蛇、杵臼之類同在第三龕。卑主尊臣,甚未便也。若以北
 極、帝坐本非天帝,蓋是天帝所居,則北極在第二,帝坐在第三,亦高下未等。又太微之次少左右執法,子星之次少孫星,望令司天監參驗。」乃詔禮儀使、太常禮院、司天監檢定之。



 禮儀使趙安仁言:「按《開寶通禮》,元氣廣大則稱昊天,據遠視之蒼然,則稱蒼天。人之所尊,莫過於帝,托之於天,故稱上帝。天皇大帝即北辰耀魄寶也,自是星中之尊。《易》曰:『日月麗乎天,百穀草木麗乎土。』又曰:『在天成象,在地成形。』蓋明辰象非天,草木非地,是則天
 以蒼昊為體,不入星辰之列。又《郊祀錄》:『壇第二等祀天皇大帝、北斗、天一、太一、紫微、五帝坐,差在行位前,餘內官諸位及五星、十二辰、河漢,都四十九坐齊列,俱在十二陛之間。』唐建中間,司天冬官正郭獻之奏:『天皇、北極、天一、太一,準《天寶敕》並合升第一等。』貞元二年親郊,以太常議,詔復從《開元禮》,仍為定制。《郊祀錄》又云:『壇第三等有中宮、天市垣、帝坐等十七坐,並在前。』《開元禮義羅》云:『帝有五坐,一在紫微宮,一在大角,一在太微宮,一
 在心,一在天市垣。』即帝坐者非直指天帝也。又得判司天監史序狀:『天皇大帝一星在紫微勾陳中,其神曰耀魄寶,即天皇是星,五帝乃天帝也。北極五星在紫微垣內,居中一星曰北辰,第一主月為太子,第二主日為帝王,第三為庶子,第四為嫡子,第五為天子之樞,蓋北辰所主非一,又非帝坐之比。太微垣十星有左右執法、上將、次將之名,不可備陳,故總名太微垣。《星經》舊載孫星,而《壇圖》止有子星,辨其尊卑,不可同位。』竊惟《壇圖》舊制,悉
 有明據,天神定位,難以躋升,望依《星經》,悉以舊禮為定。」



 欽若復言:「舊史《天文志》並云:北極,北辰最尊者。又勾陳口中一星曰天皇大帝,鄭玄注《周禮》謂:『禮天者,冬至祭天皇於北極也。』後魏孝文禋六宗,亦升天皇五帝上。按晉《天文志》:『帝坐光而潤,則天子吉,威令行。』既名帝坐,則為天子所占,列於下位,未見其可。又安仁議,以子、孫二星不可同位。陛下方洽高禖之慶,以廣維城之基,茍因前代闕文,便為得禮,實恐聖朝茂典,尤未適中。」詔天皇、
 北極特升第一龕,又設孫星於子星位次,帝坐如故。



 欽若又言:「帝坐止三,紫微、太微者已列第二等,唯天市一坐在第三等。按《晉志》,大角及心中星但云天王坐,實與帝坐不類。」詔特升第二龕。



 舊郊丘,神位板皆有司題署,命欽若改造之。至是,欽若奉板便殿,壇上四位,塗以朱漆金字,餘皆黑漆,第一等金字,第二等黃字,第三等以降朱字,悉貯漆匣,覆以黃縑帊。帝降階觀之,即付有司。又以新定《壇圖》,五帝、五岳、中鎮、河漢合在第三等。



 四年,
 判太常禮院孫奭言:「準禮,冬至祀圜丘,有司攝事,以天神六百九十位從祀。今惟有五方上帝及五人神十七位,天皇大帝以下並不設位。且太昊、勾芒,惟孟夏雩祀、季秋大享及之,今乃祀於冬至,恐未協宜。」翰林學士晁迥等言:「按《開寶通禮》:圜丘,有司攝事,祀昊天、配帝、五方帝、日月、五星、中官、外官、眾星總六百八十七位;雩祀、大享,昊天、配帝、五天帝、五人帝、五官總十七位;方丘,祭皇地祇、配帝、神州、岳鎮、海瀆七十一位。今司天監所設圜
 丘、雩祀、明堂、方丘並七十位,即是方丘有嶽、瀆從祀,圜丘無星辰,而反以人帝從祀。望如奭請,以《通禮》及神位為定,其有增益者如後敕。」從之。



 政和三年,議禮局上《五禮新儀》:皇帝祀昊天上帝,太史設神位版,昊天上帝位於壇上北方南向,席以稿秸;太祖位於壇上東方西向,席以蒲越;天皇大帝、五帝、大明、夜明、北極九位於第一龕;北斗、太一、帝坐、五帝內坐、五星、十二辰、河漢等內官神位五十有四於第二龕;二十八宿等中官神位百五
 十有九於第三龕;外官神位一百有六於內壝之內;眾星三百有六十於內壝之外。第一龕席以稿秸,餘以莞席,皆內向配位。



 太祖乾德元年,始有事於南郊。自五代以來,喪亂相繼,典章制度,多所散逸。至是,詔有司講求遺逸,遵行典故,以副寅恭之意。是歲十一月十六日,合祭天地於圜丘。初,有司議配享,請以僖祖升配,張昭獻議曰:「隋、唐以前,雖追立四廟或六七廟,而無遍加帝號之文。梁、陳南郊,祀天皇,配以皇考;北齊圜丘,祀昊天,以
 神武升配;隋祀昊天於圜丘,以皇考配;唐貞觀初,以高祖配圜丘;梁太祖郊天,以皇考烈祖配。恭惟宣祖皇帝,積累勛伐,肇基王業,伏請奉以配享。」從之。



 九年正月,詔以四月幸西京,有事於南郊。自國初以來,南郊四祭及感生帝、皇地祇、神州凡七祭,並以四祖迭配。太祖親郊者四,並以宣祖配。太宗即位,其七祭但以宣祖、太祖更配。是歲親享天地,始奉太祖升侑。雍熙元年冬至親郊,從禮儀使扈蒙之議,復以宣祖配。四年正月,禮儀使蘇
 易簡言:「親祀圜丘,以宣祖配,此則符聖人大孝之道,成嚴父配天之儀。太祖皇帝光啟丕圖,恭臨大寶,以聖授聖,傳於無窮。按唐永徽中,以高祖、太宗同配上帝。欲望將來親祀郊丘,奉宣祖、太祖同配;其常祀祈穀、神州、明堂,以宣祖崇配;圜丘、北郊、雩祀,以太祖崇配。」奏可。



 真宗至道三年十一月,有司言:「冬至圜丘、孟夏雩祀、夏至方丘,請奉太宗配;上辛祈穀、季秋明堂,奉太祖配;上辛祀感生帝、孟冬祭神州地祇,奉宣祖配;其親郊,奉太祖、太
 宗並配。」詔可。乾興元年,真宗崩,詔禮官定遷郊祀配帝,乃請:「祈穀及祭神州地祇,以太祖配;雩祀及昊天上帝及皇地祇,以太宗配;感生帝,以宣祖配;明堂,以真宗配;親祀郊丘,以太祖、太宗配。」奏可。



 景祐二年郊,詔以太祖、太宗、真宗三廟萬世不遷。南郊以太祖定配,二宗迭配,親祀皆侑。常祀圜丘、皇地祇配以太祖,祈穀、雩祀、神州配以太宗,感生帝、明堂以宣祖、真宗配如舊。慶歷元年,判太常寺呂公綽言:「歷代郊祀,配位無側向,真宗示輔
 臣《封禪圖》曰:『嘗見郊祀昊天上帝,不以正坐,蓋皇地祇次之。今修登封,上帝宜當子位,太祖、太宗配位,宜比郊祀而斜置之。』其後,有司不諭先帝以告成報功、酌宜從變之意,每郊儀範,既引祥符側置之文,又載西向北上之禮,臨時擇一,未嘗考定。」乃詔南郊祖宗之配,並以東方西向為定。皇祐五年郊,詔自今圜丘,三聖並侑。嘉祐六年,諫官楊畋論水災繇郊廟未順。禮院亦言:「對越天地,神無二主。唐始用三祖同配,後遂罷之。皇祐初,詔三
 聖並侑,後復迭配,未幾復並侑,以為定制。雖出孝思,然頗違經典,當時有司失於講求。」下兩制議,翰林學士王珪等曰:「推尊以享帝,義之至也。然尊尊不可以瀆,故郊無二主。今三後並侑,欲以致孝也,而適所以瀆乎享帝,非無以寧神也,請如禮官議。」七年正月,詔南郊以太祖定配。



 高宗建炎二年,車駕至揚州,築壇於江都縣之東南。是歲冬至,祀昊天上帝,以太祖配。度宗咸淳二年,將舉郊祀,時復議以高宗參配。吏部侍郎兼中書門下省
 檢正洪燾等議,以為:「物無二本,事無二初,舜之郊嚳,商之郊契,周郊後稷,皆所以推原其始也。禮者,所以別等差,視儀則,遠而尊者配於郊,近而親者配於明堂,明有等也。臣等謂宜如紹興故事,奉太宗配,將來明堂遵用先皇帝彞典,以高宗參侑,庶於報本之禮、奉先之孝,為兩盡其至。」詔恭依。



 儀注。乾德元年八月,禮儀使陶穀言:「饗廟、郊天,兩日行禮,從祀官前七日皆合於尚書省受誓戒,自來一日之
 內受兩處誓戒,有虧虔潔。今擬十一月十六日行郊禮,望依禮文於八日先受從享太廟誓戒,九日別受郊天誓戒,其日請放朝參。」從之。自後百官受誓戒於朝堂,宗室受於太廟。



 祭之日均用醜時,秋夏以一刻,春冬以七刻,前二日遣官奏告。配帝之室,儀鸞司設大次、小次及文武侍臣、蕃客之次,太常設樂位、神位、版位等事。前一日司尊彞帥其屬以法陳祭器於堂東,僕射、禮部尚書視滌濯告潔,禮部尚書、侍郎省牲,光祿卿奉牲,告充、告
 備,禮部尚書視鼎鑊,禮部侍郎視腥熟之節。祭之旦,光祿卿率其屬取籩、豆、簠、簋實之。及薦腥,禮部尚書帥其屬薦籩、豆、簠、簋,戶部、兵部、工部尚書薦三牲之腥熟俎。禮畢,各徹,而有司受之以出。晡後,郊社令帥其屬掃除,御史按視之。奏中嚴外辦以禮部侍郎,請解嚴以禮部郎中。贊者設亞、終獻位於小次之南,宗室位於其後;設公卿位於亞、終獻之南,分獻官位於公卿之後,執事者又在其後,俱重行,西向北上。其致福也,太牢以牛左肩、
 臂、臑折九個,少牢以羊左肩七個、犆豕以左肩五個。有司攝事、進胙皆如禮。太尉展視以授使者,再拜稽首。既享,大宴,號曰飲福,自宰臣而下至應執事及樂工、馭車馬人等,並均給有差,以為定式。是歲十一月日至,皇帝服袞冕,執圭,合祭天地於圜丘,還御明德門樓,肆赦。



 仁宗天聖二年,詔加真宗謚,上謂輔臣曰:「郊祀重事,朕欲就禁中習儀,其令禮官草具以聞。」先郊三日,奉謚冊寶於太廟。次日,薦享玉清昭應、景靈宮,宿太廟。既享,赴青
 城,至大次,就更衣壇改服袞冕行事。五年,郊後擇日恭謝,大禮使王曾請節廟樂,帝曰:「三年一享,不敢憚勞也。」三獻終,增禮生七人,各引本室太祝升殿,徹豆。三日,又齋長春殿,謝玉清昭應宮。禮畢,賀皇太后,比籍田,勞酒儀,略如元會。其恭謝云:「臣某虔遵舊典,郊祀禮成,中外協心,不勝歡抃。」宣答曰:「皇帝德備孝恭,禮成嚴配,萬國稱頌,歡豫增深。」帝再拜還內。樞密使以下稱賀,閣門使宣答,樞密副使升殿侍立,百官稱賀。酒三行,還內殿,受
 命婦賀,司賓自殿側幕次引內命婦於殿庭,北向立,尚儀奏:「請皇太后即御坐。」司賓贊:「再拜。」引班首升自西階,稱封號妾某氏等言:「郊祀再舉,福祚咸均,凡在照臨,不勝忻抃。」降,再拜。尚宮承旨,降自東階,稱「皇太后聖旨」,又再拜。司賓宣答曰「已成鉅禮,歡豫良深。」皆再拜。次外命婦賀,如內命婦儀,退,皆赴別殿賀皇帝,惟不致詞,不宣答。



 神宗元豐六年十一月二日,帝將親郊,奉仁宗、英宗徽號冊寶於太廟。是日晚,齋於大慶殿。三日,薦享於景
 靈宮,齋於太廟。四日,朝享七室,齋於南郊之青城。五日冬至,祀昊天上帝於圜丘,以太祖配。是日,帝服靴袍,乘輦至大次。有司請行禮。服大裘,被袞冕以出,至壇中壝門外,殿中監進大圭,帝執以入,宮架樂作,至午階下版位,西向立,樂止。禮儀使贊曰:「有司謹具,請行事。」宮架奏《景安》之樂,文舞作六成,止,帝再拜,詣罍洗,宮架樂作,至洗南北向,樂止。帝搢圭,盥帨訖,樂作,至壇下,樂止。升午階,登歌樂作,至壇上,樂止。殿中監進鎮圭,《嘉安》樂作,詣
 上帝神坐前,北向跪,奠鎮圭於繅藉,執大圭,俯伏,興,搢圭跪,三上香,奠玉幣,執圭,俯伏,興,再拜。內侍舉鎮圭授殿中監,樂止。《廣安》樂作,詣太祖神坐前,東向,奠圭幣如上帝儀。登歌樂作,帝降壇,樂止。宮架樂作,還位,西向立,樂止。禮部尚書、戶部尚書以下奉饌俎,宮架《豐安》樂作,奉奠訖,樂止。再詣罍洗,帝搢大圭,盥帨,洗爵拭爵訖,執大圭,宮架樂作,至壇下,樂止。升自午階,登歌樂作,至壇上,樂止。登歌《禧安》樂作,詣上帝神坐前,搢圭跪,執爵祭
 酒,三奠訖,執圭,俯伏,興,樂止。太祝讀冊,帝再拜訖,樂作。次詣太祖神坐前,如前儀。登歌樂作,帝降自午階,樂止。宮架樂作,還位,西向立,樂止。文舞退,武舞進,宮架《正安》之樂作,樂止。亞獻盥帨訖,《正安》樂作,禮畢,樂止。終獻行禮並如上儀,獻畢,宮架樂作,帝升自午階,樂止。登歌樂作,至飲福位,樂止。《禧安》樂作,帝再拜,搢圭跪,受爵,祭酒三,啐酒,奠爵,受俎,奠俎,受摶黍豆,再受爵,飲福訖,奠爵,執圭,俯伏,興,再拜,樂作。帝降,還位如前儀。禮部、戶部尚
 書徹俎豆,禮直官曰:「賜胙行事。」陪祀官再拜,宮架《宴安》樂作,一成止。宮架樂作,帝詣望燎位,南向立,樂止。禮直官曰:「可燎。」俟火燎半柴,禮儀使跪奏:「禮畢。」宮架樂作,帝出中壝門,殿中監受大圭,歸大次,樂止。有司奏解嚴。



 帝乘輿還青城,百官稱賀於端誠殿。有司轉仗衛,奏中嚴外辦。帝服通天冠、絳紗袍,乘輿以出。至玉輅所,侍中跪請降輿升輅。帝升輅,門下侍郎奏請進行,又奏請少駐,侍臣乘馬,將至宣德門,奏《採薺》一曲,入門,樂止。侍中請
 降輅赴幄次,有司奏解嚴。帝常服,乘輿御宣德門,肆赦,群臣稱賀如常儀。



 初,淳化三年,將以冬至郊,前十日,皇子許王薨,有司言:「王薨在未受誓戒之前,準禮,天地、社稷之祀不廢。」詔下尚書省議。吏部尚書宋琪等奏:「以許王薨謝,去郊禮裁十日,又詔輟十一日以後五日朝參,且至尊成服,百僚皆當入慰。有司又以十二、十三日受誓戒,按令式,受誓戒後不得吊喪問疾。今若皇帝既輟朝而未成服,則全爽禮文;百僚既受誓而入奉慰,又違
 令式。況許王地居藩戚,望著親賢,於昆仲為大宗,於朝廷為塚嗣,遽茲薨逝,朝野同哀,伏想聖情,豈勝追念。當愁慘之際,行對越之儀,臣等實慮上帝之弗歆,下民之斯惑。況祭天之禮,歲有四焉,載於《禮經》,非有差降。請以來年正月上辛合祭天地。」從之。



 神宗之嗣位也,英宗之喪未除。是歲當郊,帝以為疑,以問講讀官王珪、司馬光、王安石,皆對以不當廢。珪又謂:「『喪三年不祭,惟祭天地、社稷,為越紼而行事。』《傳》謂:『不敢以卑廢尊也。』景德二年,真
 宗居明德太后之喪,即易月而服除。明年遂享太廟,而合祀天地於圜丘。請冬至行郊廟之禮,其服冕、車輅、儀物、音樂緣神事者皆不可廢。」詔用景德故事,惟郊廟及景靈宮禮神用樂,鹵簿鼓吹及樓前宮架、諸軍音樂,皆備而不作,警場止鳴金鉦、鼓角,仍罷諸軍呈閱騎隊。故事,齋宿必御樓警嚴,幸後苑觀花,作水戲,至是悉罷之。有司言:「故事,當謁謝於祖宗神御殿,獻享月吉禮,以禮官攝。」詔遣輔臣仍罷詣佛寺。是後國有故,皆遣輔臣。



 高
 宗紹興十二年,臣僚言:「自南巡以來,三歲之祀,獨於明堂,而郊天之禮未舉,來歲乞行大禮。」詔建圜壇於臨安府行宮東城之外,自是凡六郊焉。



 孝宗隆興二年,詔曰:「聯恭覽國史,太祖乾德詔書有云:『務從省約,無至勞煩。』仰見事天之誠,愛民之仁,所以垂萬世之統者在是。今歲郊見,可令有司,除禮物、軍賞,其餘並從省約。。初降詔以十一月行事,以冬至適在晦日,以至道典故,改用獻歲上辛,遂改來年元為乾道。乃以正月一日有事南郊,
 禮成,進胙於德壽宮,以牛腥體肩三、臂上臑二。導駕官自端誠殿簪花從駕至德壽宮上壽,飲福稱賀,陳設儀注,並同上壽禮。皇帝致詞曰:「皇帝臣某言:享帝合宮,受天純嘏,臣某與百僚不勝大慶,謹上千萬歲壽。」自後郊祀、明堂進胙飲福,並如上儀。



 光宗紹熙二年十一月郊,以值雨,行禮於望祭殿。帝遂感疾。理宗四十一年,一郊而已。度宗咸淳二年,權工部尚書趙汝暨等奏:「今歲大禮,正在先帝大祥之後,臣等竊惟帝王受命,郊見天地,
 不可緩也。古者有改元即郊,不用前郊三年為計。況今適在當郊之歲,既逾大祥之期,圜丘之祀,豈容不舉?」於是降禮,以十一月十七日款謁南郊,適太史院言:「十六日太陰交蝕。」遂改來年正月一日南郊行禮,太常寺言:「皇帝既已從吉,請依儀用樂。其十二月二十九日朝獻景靈宮,三十日朝享太廟,尚在禫制之內,所有迎神、奠幣、酌獻、送神作樂外,其盥洗升降行步等樂,備而不作。」



\end{pinyinscope}