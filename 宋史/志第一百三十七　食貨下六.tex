\article{志第一百三十七 食貨下六}

\begin{pinyinscope}

 茶下



 茶天聖三年八月,詔翰林侍講學士孫奭等同究利害,奭等言:「十三場茶積而未售者六百一十三萬餘斤,蓋許商人貼射,則善者皆入商人,其入官者皆粗惡不時,故人莫肯售。又園戶輸歲課不足者,使如商人入息,而園戶皆細民,貧弱力不能給,煩擾益甚。又奸人倚貼射為名,強市盜販,侵奪官利,其弊不可不革。」十月,遂罷貼射法,官復給本錢市茶。商人入錢以售茶者,奭等又欲優之,請凡入錢京師售海州、荊南茶者,損為七萬七千,售真州等四務十三場茶者,又第損之,給茶皆直十萬。自是,河北入中復用三說法,舊給東南緡錢者,以京師榷貨務錢償之。



 奭等議既用,益以李諮等變法為非。明年,摭計置司所上天聖二年比視增虧數差謬,詔令嘗典議官張士遜等條析。夷簡言:「天聖初,環慶等路數奏芻糧不給,京師府藏常闕緡錢,吏兵月奉僅能取足。自變法以來,京師積錢多,邊計不聞告乏,中間蕃部作亂,調發兵馬,仰給有司,無不足之患。以此推之,頗有成效。三
 司比視數目差互不同,非執政所能親自較計。」然士遜等猶被罰,諮罷三司使。初,園戶負歲課者如商人入息,後不能償。至四年,太湖等九場凡逋息錢十三萬緡,
 詔悉蠲之。然自奭等改制,而茶法浸壞。



 景祐中,三司使孫居中等言:「自天聖三年變法,而河北入中虛估之敝,復類乾興以前,蠹耗縣官,請復行見錢法。」時諮已執政矣。三年,河北轉運使楊偕亦陳三說法十二害,見錢法十二利,
 以謂止用三說所支一分緡錢,足以贍一歲邊計。遂命諮與參知政事蔡齊等合議,且令詔商人訪其利害。是歲三月,諮等請罷河北入中虛估,以實錢償芻粟,實錢售茶,皆如天聖元年之制。又以北商持券至京師,舊必得交引鋪為之保任,並得三司符驗,然後給錢,以是京師坐賈率多邀求,三司吏稽留為奸,乃悉罷之,命商持券徑趣榷貨務驗實,立償之錢。初,奭等雖增商人入錢之數,而猶以為利薄,故競市虛估之券,以射厚利,而入錢者寡,縣官日以侵削,京師少蓄藏。至是,諮等請視天聖三年入錢數第損一
 千有奇,入中增直亦視天聖元年數第加三百。詔皆可之。前已用虛估給券者,給茶如舊,仍給景祐二年已前茶。



 既而諮等又言:「天聖
 四年,嘗許陜西入中願得茶者,每錢十萬,所在給券,徑趣東南受茶十一萬一千。茶商獲利,爭欲售陜西券,故不復入錢京師,請禁止之。」並言商人所不便者,其事甚悉,請為更約束,重私販之禁,聽商人輸錢五分,餘為置籍召保,期半年悉償,失期者倍其數。事皆施行。諮等復言:「自奭等變法,歲損財利不可勝計,且以天聖九年至景祐二年較之,五年之間,河北入中虛費緡錢五百六十八萬;今一旦復用舊法,恐豪商不便,依托權貴,以動
 朝廷,請先期申諭。」於是帝為下詔戒敕,而縣官濫費自此少矣。



 久之,上書者復言:「自變法以來,歲輦京師金帛,易芻粟於河北,配擾居民,內虛府庫,外困商旅,非便。」寶元元年,命御史中丞張觀等與三司議之。觀等復請入錢京師以售真州等四務十三場茶,直十萬者,又視景祐三年數損之,為錢六萬七千,入中河北願售茶者,又損一千。既而詔又第損二千,於是入錢京師止為錢六萬五千,入中河北為錢六萬四千而已。



 康定元年,葉清
 臣為三司使,是歲河北穀賤,因請內地諸州行三說法,募人入中,且以東南鹽代京師實錢。詔糴止二百萬石。慶歷二年,又請募人入芻粟如康定元年法,數足而止,自是三說稍復用矣。八年,三司鹽鐵判官董沔亦請復三說法,三司以為然,因言:「自見錢法行,京師錢入少出多,慶歷七年,榷貨務緡錢入百十九萬,出二百七十六萬。以此較之,恐無以贍給,請如沔議,以茶、鹽、香藥、緡錢四物如之。」於是有四說之法。初,詔止行於並邊諸州,而
 內地諸州有司蓋未嘗請,即以康定元年詔書從事。自是三說、四說二法並行於河北,不數年間,茶法復壞。芻粟之直,大約虛估居十之八,米斗七百,甚者千錢。券至京師,為南商所抑,茶每直十萬,止售錢三千,富人乘時收蓄,轉取厚利。三司患之,請行貼買之法,每券直十萬,比市估三千,倍為六千,復入錢四萬四千,貼為五萬,給茶直十萬。詔又損錢一萬,然亦不足以平其直。久之,券比售錢三千者,才得二千,往往不售,北商無利,入中者
 寡,公私大弊。



 皇祐二年,知定州韓琦及河北轉運司皆以為言,下三司議。三司奏:「自改法至今,凡得穀二百二十八萬餘石,芻五十六萬餘圍,而費緡錢一百九十五萬有奇,茶、鹽、香藥又為緡錢一千二百九十五萬有奇。茶、鹽、香藥,民用有限,榷貨務歲課不過五百萬緡,今散於民間者既多,所在積而不售,故券直亦從而賤。茶直十萬,舊錢六萬五千,今止二千;以至香一斤,舊售錢三千八百,今止五六百;公私兩失其利。請復行見錢法,一
 用景祐三年約束。」乃下詔曰:「比食貨法壞,芻粟價益倍,縣官之費日長,商賈不行,豪富之家,乘時牟利,吏緣為奸。自今有議者,須究厥理,審可施用,若事已上而驗問無狀者,置之重罰。」



 是時雖改見錢法,而京師積錢少,恐不足以支入中之費,帝又出內藏庫錢帛百萬以賜三司。久之,入中者浸多,京師帑藏益乏,商人持券以俟,動彌歲月,至損其直以售於蓄賈之家。言利者請出內藏庫錢稍增價售之,歲可得遺利五十萬緡。既行,而諫官
 範鎮謂內藏庫、榷貨務皆領縣官,豈有榷貨務故稽商人,而令內藏乘時射利?傷體壞法,莫斯為甚。詔即罷之,然自此並邊虛估之弊復起。



 至和三年,河北提舉糴便糧草薛向建議:「並邊十七州軍,歲計粟百八十萬石,為錢百六十萬緡,豆六十五萬石,芻三百七十萬圍,並邊租賦歲可得粟、豆、芻五十萬,其餘皆商人入中。請罷並邊入粟,自京輦錢帛至河北,專以見錢和糴。」時楊察為三司使,請用其說。因輦絹四十萬匹當緡錢七十萬,又
 蓄見錢及擇上等茶場八,總為緡錢百五十萬,儲之京師。而募商人入錢並邊,計其道里遠近,優增其直,以是償之,且省輦運之費,唯入中芻豆計直償以茶如舊。行未數年,論者謂輦運科折,煩擾居民,且商人入錢者少,芻豆虛估益高,茶益賤。詔翰林學士韓絳等即三司經度。絳等言:「自改法以來,邊儲有備,商旅頗通,未宜輕變。唯輦運之費,悉從官給,而本路舊輸稅絹者,毋得折為見錢,入中芻豆罷勿給茶,所在平其市估,至京償以銀、
 綢、絹。」自是茶法不復為邊糴所須,而通商之議起矣。



 初,官既榷茶,民私蓄盜販皆有禁,臘茶之禁又嚴於他茶,犯者其罪尤重,凡告捕私茶皆有賞。然約束愈密而冒禁愈繁,歲報刑闢,不可勝數。園戶困於征取,官司並緣侵擾,因陷罪戾至破產逃匿者,歲比有之。又茶法屢變,歲課日削。至和中,歲市茶淮南才四百二十二萬餘斤,江南三百七十五萬餘斤,兩浙二十三萬餘斤,荊湖二百六萬餘斤,唯福建天聖末增至五十萬斤,詔特損五
 萬,至是增至七十九萬餘斤,歲售錢並本息計之,才百六十七萬二千餘緡。官茶所在陳積,縣官獲利無幾,論者皆謂宜弛禁便。



 先是,天聖中,有上書者言茶、鹽課虧。帝謂執政曰:「茶、鹽,民所食,而強設法以禁之,致犯者眾。顧經費尚廣,未能弛禁爾!」景祐中,葉清臣上疏曰:



 「山澤有產,天資惠民。兵食不充,財臣兼利,草芽木葉,私不得專,對園置吏,隨處立筦。一切官禁,人犯則刑,既奪其資,又加之罪,黥流日報,逾冒不悛。誠有厚利重貨,能濟國
 用,聖仁恤隱,矜赦非辜,猶將弛禁緩刑,為民除害。度支費用甚大,榷易所收甚薄,刳剝園戶,資奉商人,使朝廷有聚斂之名,官曹滋虐濫之罰,虛張名數,刻蠹黎元。



 建國以來,法敝輒改,載詳改法之由,非有為國之實,皆商吏協計,倒持利權,幸在更張,倍求奇羨。富人豪族,坐以賈贏,薄販下估,日皆朘削,官私之際,俱非遠策。臣竊嘗校計茶利所入,以景祐元年為率,除本錢外,實收息錢五十九萬餘緡,又天下所售食茶,並本息歲課亦只及
 三十四萬緡,而茶商見通行六十五州軍,所收稅錢已及五十七萬緡。若令天下通商,只收稅錢,自及數倍,即榷務、山場及食茶之利,盡可籠取。又況不費度支之本,不置榷易之官,不興輦運之勞,不濫徒黥之闢。



 臣意生民之弊,有時而窮,盛德之事,俟聖不惑。議者謂榷賣有定率,征稅無彞準,通商之後,必虧歲計。臣按管氏鹽鐵法,計口受賦,茶為人用,與鹽鐵均,必令天下通行,以口定賦,民獲善利,又去嚴刑,口數出錢,人不厭取。景祐元
 年,天下戶千二十九萬六千五百六十五,丁二千六百二十萬五千四百四十一,三分其一為產茶州軍,內外郭鄉又居三分之一,丁賦三十,村鄉丁賦二十,不產茶州軍郭鄉村鄉如前計之,又第損十錢,歲計已及緡錢四十萬。榷茶之利,凡止九十餘萬緡,通商收稅,且以三倍舊稅為率,可得一百七十餘萬緡,更加口賦之入,乃有二百一十餘萬緡,或更於收稅則例,微加增益,即所增至寡,所聚愈厚,比於官自榷易,驅民就刑,利病相須,炳
 然可察。」時下三司議,皆以為不可行。



 至嘉祐中,著作佐郎何鬲、三班奉職王嘉麟又皆上書請罷給茶本錢,縱園戶貿易,而官收租錢與所在征算,歸榷貨務以償邊糴之費,可以疏利源而寬民力。嘉麟為《登平致頌書》十卷、《隆衍視成策》二卷上之,淮南轉運副使沉立亦集《茶法利害》為十卷,陳通商之利。時富弼、韓琦、曾公亮執政,決意向之,力言於帝。三年九月,命韓絳、陳升之、呂景初即三司置局議之。十月,三司言:「茶課緡錢歲當入二百
 二十四萬八千,嘉祐二年才及一百二十八萬,又募人入錢,皆有虛數,實為八十六萬,而三十九萬有奇是為本錢,才得子錢四十六萬九千,而輦運麋耗喪失,與官吏、兵夫廩給雜費,又不與焉。至於園戶輸納,侵擾日甚,小民趨利犯法,刑闢益繁,獲利至少,為弊甚大。宜約至和以後一歲之數,以所得息錢均賦茶民,恣其買賣,所在收算,請遣官詢察利害以聞。」詔遣官分行六路,還言如三司使議便。



 四年二月,詔曰:「古者山澤之利,與民共
 之,故民足於下,而君裕於上,國家無事,刑罰以清。自唐建中時,始有茶禁,上下規利,垂二百年。如聞比來為患益甚,民被誅求之困,日惟咨嗟,官受濫惡之入,歲以陳積,私藏盜販,犯者實繁,嚴刑重誅,情所不忍,是於江湖之間幅員數千里,為陷阱以害吾民也。朕心惻然,念此久矣,間遣使者往就問之,而皆歡然願弛其禁,歲入之課以時上官。一二近臣,條析其狀,朕猶若慊然,又於歲輸裁減其數,使得饒阜,以相為生,俾通商利。歷世之敝,
 一旦以除,著為經常,弗復更制,損上益下,以休吾民。尚慮喜於立異之人、緣而為奸之黨,妄陳奏議,以惑官司,必置明刑,無或有貸。」



 初,所遣官既議弛禁,因以三司歲課均賦茶戶,凡為緡錢六十八萬有奇,使歲輸縣官。比輸茶時,其出幾倍,朝廷難之,為損其半,歲輸緡錢三十三萬八千有奇,謂之租錢,與諸路本錢悉儲以待邊糴。自是唯臘茶禁如舊,餘茶肆行天下矣。論者猶謂朝廷志於恤人,欲省刑罰,其意良善;然茶戶困於輸錢,而商
 賈利薄,販鬻者少,州縣征稅日蹙,經費不充,學士劉敞、歐陽修頗論其事。敞疏大要以謂先時百姓之摘山者,受錢於官,而今也顧使之納錢於官,受納之間,利害百倍;先時百姓冒法販茶者被罰耳,今悉均賦於民,賦不時入,刑亦及之,是良民代冒法者受罪;先時大商富賈為國懋遷,而州郡收其稅,今大商富賈不行,則稅額不登,且乏國用。修言新法之行,一利而有五害,大略與敞意同。時朝廷方排眾論而行之,敞等雖言,不聽也。



 治平
 中,歲入臘茶四十八萬九千餘斤,散茶二十五萬五千餘斤,茶戶租錢三十二萬九千八百五十五緡,又儲本錢四十七萬四千三百二十一緡,而內外總入茶稅錢四十九萬八千六百緡,推是可見茶法得失矣。自天聖以來,茶法屢易,嘉祐始行通商,雖議者或以為不便,而更法之意則主於優民。



 熙寧四年,神宗與大臣論昔茶法之弊,文彥博、吳充、王安石各論其故,然於茶法未有所變。及王韶建開湟之策,委以經略。七年,始遣三司乾
 當公事李□巳入蜀經畫買茶,於秦鳳、熙河博馬。而韶言西人頗以善馬至邊,所嗜唯茶,乏茶與市。即詔趨趣□巳據見茶計水陸運致,又以銀十萬兩、帛二萬五千、度僧牒五百付之,假常平及坊場餘錢,以著作佐郎蒲宗閔同領其事。初,蜀之茶園,皆民兩稅地,不殖五穀,唯宜種茶。賦稅一例折輸,蓋為錢三百,折輸綢絹皆一匹;若為錢十,則折輸綿一兩;為錢二,則折輸草一圍。役錢亦視其賦。民賣茶資衣食,與農夫業田無異,而稅額總三十萬。
 □巳被命經度,又詔得調舉官屬,乃即屬諸州創設官場,歲增息為四十萬,而重禁榷之令。其輸受之際,往往厭其斤重,侵其價直,法既加急矣。八年,□巳以疾去。



 先是,□巳等歲增十萬之息,既而運茶積滯,歲課不給,即建畫於彭、漢二州歲買布各十萬匹,以折腳費,實以布息助茶利,然茶亦未免積滯。都官郎中劉佐復議歲易解鹽十萬席,顧運回車船加載蜀,而禁商販,蓋恐布亦難敷也。詔既以佐代□巳,未幾,鹽法復難行,遂罷佐。而宗閔乃議
 川峽路民茶息收什之三,盡賣於官場,更嚴私交易之令,稍重至徒刑,仍沒緣身所有物,以待賞給。於是蜀茶盡榷,民始病焉。



 十年,知彭州呂陶言:「川峽四路所出茶,比東南十不及一,諸路既許通商,兩川卻為禁地,虧損治體。如解州有鹽池,民間煎者乃是私鹽,晉州有礬山,民間煉者乃是私礬,今川蜀茶園,皆百姓己物,與解鹽、晉礬不同。又市易司籠制百貨,歲出息錢不過十之二,然必以一年為率;今茶場司務重立法,盡榷民茶,隨
 買隨賣,取息十之三,或今日買十千之茶,明日即作十三千賣之,變轉不休,比至歲終,豈止三分?」因奏劉佐、李□巳、蒲宗閔等茍希進用,必欲出息三分,致茶戶被害。始詔息止收十之一,佐坐措置乖方罷,以國子博士李稷代之,而陶亦得罪。稷依李□巳例兼三司判官,仍委權不限員舉劾。



 侍御史周尹論蜀中榷茶為民害,罷為提點湖北刑獄。利州路漕臣張宗諤、張升卿議廢茶場司,依舊通商,詔付稷,稷方以茶利要功,言宗諤等所陳皆疏謬,
 罪當無赦。雖會赦,猶皆坐貶秩二等。於是稷建議賣茶官非材,許對易,如闕員,於前資待闕官差;茶場司事,州郡毋得越職聽治。又以茶價增減或不一,裁立中價,定歲入課額,及設酬賞以待官吏,而三路三十六場大小使臣並不限員。重園戶採造黃花秋葉茶之禁,犯者沒官。蒲宗閔亦援稷比,許舉劾官吏,以重其權,二人皆務浚利刻急。茶場監官買茶精良及滿五千馱以及萬馱,第賞有差,而所買粗惡偽濫者,計虧坐贓論。凡茶場州
 軍知州、通判並兼提舉,經略使所在,即委通判。又禁南入熙河、秦鳳、涇原路,如私販臘茶法。



 自熙寧十年冬推行茶法,元豐元年秋,凡一年,通課利及舊界息稅七十六萬七千六十餘緡。帝謂稷能推原法意,日就事功,宜速遷擢,以勸在位,遂落權發遣,以為都大提舉茶場,而用永興軍等路提舉常平范純粹同提舉。久之,用稷言徙司秦州,而錄李□巳前勞,以子玨試將作監主簿。蒲宗閔更請巴州等處產茶並用榷法。



 五年,李稷死永樂城,
 詔以陸師閔代之。師閔言稷治茶五年,百費外獲凈息四百二十八萬餘緡,詔賜田十頃。而師閔榷利,尤刻於前,建言:「文、階州接連,而茶法不同,階為禁地,有博馬、賣茶場,文獨為通商地。乞文、龍二州並禁榷;仍許川路餘羨茶貨入陜西變賣,於成都府置博賣都茶場。」事皆施行。初,群牧判官郭茂恂言,賣茶買馬,事實相須,詔茂恂同提舉茶場。至是,師閔以買馬司兼領茶場,茶法不能自立,詔罷買馬司兼領;令茶場都大提舉視轉運使,同
 管幹視轉運判官,以重其任。賈種民更立茶法,師閔論奏茶場與他場務不同,詔並用舊條。初,李□巳增諸州茶場,自熙寧七年至元豐八年,蜀道茶場四十一,京西路金州為場六,陜西賣茶為場三百三十二,稅息至稷加為五十萬,及師閔為百萬。



 元祐元年,侍御史劉摯奏疏曰:「蜀茶之出,不過數十州,人賴以為生,茶司盡榷而市之。園戶有茶一本,而官市之,額至數十斤。官所給錢,靡耗於公者,名色不一,給借保任,輸入視驗,皆牙儈主之,
 故費於牙儈者又不知幾何。是官於園戶名為平市,而實奪之。園戶有逃而免者,有投死以免者,而其害猶及鄰伍。欲伐茶則有禁,欲增植則加市,故其俗論謂地非生茶也,實生禍也。願遣使者,考茶法之敝,以蘇蜀民。」右司諫蘇轍繼言:「呂陶嘗奏改茶法,止行長引,令民自販,每緡長引錢百,詔從其請,民方有息肩之望。孫迥、李稷入蜀商度,盡力掊取,息錢、長引並行,民間始不易矣。且盜賊贓及二貫,止徒一年,出賞五千,今民有以錢八百
 私買茶四十斤者,輒徒一年,賞三十千,立法茍以自便,不顧輕重之宜。蓋造立茶法,皆傾險小人,不識事體。」且備陳五害。呂陶亦條上利害,詔付黃廉體量;未至,摯又言陸師閔恣為不法,不宜仍任事。詔即罷之。先是,師閔提舉榷茶,所行職務,他司皆不得預聞,事權震灼,為患深密。及黃廉就領茶事,乃請凡緣茶事有侵損戾法,或措置未當及有訴訟,依元豐令,聽他司關送。十一月,蒲宗孟亦以附會李稷賣茶罷。



 明年,熙河、秦鳳、涇原三路
 茶仍官為計置,永興、鄜延、環慶許通商,凡以茶易穀者聽仍舊,毋得逾轉運司和糴價,其所博觔斗勿取息。七年,詔成都等路茶事司,以三百萬緡為額本。



 紹聖元年,復以陸師閔都大提舉成都等路茶事,而陜西復行禁榷。師閔乃奏龍州仍為禁茶地,凡茶法並用元豐舊條。師閔自復用,以訖哲宗之世,其掊克之跡,不若前日之著,故建明亦罕見焉。



 茶之在諸路者,神宗、哲宗朝無大更革。熙寧八年,嘗詔都提舉市易司歲賈商茶,以三百
 萬斤為額。元祐五年,立六路茶稅租錢諸州通判轉運司月暨歲終比較都數之法。七年,以茶隸提刑司,稅務毋得更易為雜稅收受。紹聖四年,戶部言:「商旅茶稅五分,治平條立輸送之限既寬,復慮課入無準,故定以限約,毋得更展。元祐中,輒展以季,課入漏失。且茶稅歲計七十萬緡,積十年未嘗檢察,請內外委官,期一年驅算以聞。」詔聽其議,展限令出一時,毋承用。



 崇寧元年,右僕射蔡京言:「祖宗立禁榷法,歲收凈利凡三百二十餘萬
 貫,而諸州商稅七十五萬貫有奇,食茶之算不在焉,其盛時幾五百餘萬緡。慶歷之後,法制浸壞,私販公行,遂罷禁榷,行通商之法。自後商旅所至,與官為市,四十餘年,利源浸失。謂宜荊湖、江、淮、兩浙、福建七路所產茶,仍舊禁榷官買,勿復科民,即產茶州郡隨所置場,申商人園戶私易之禁,凡置場地園戶租折稅仍舊。產茶州軍許其民赴場輸息,量限斤數,給短引,於旁近郡縣便鬻;餘悉聽商人於榷貨務入納金銀、緡錢或並邊糧草,即
 本務給鈔,取便算請於場,別給長引,從所指州軍鬻之。商稅自場給長引,沿道登時批發,至所指地,然後計稅盡輸,則在道無苛留。買茶本錢以度牒、末鹽鈔、諸色封樁、坊場常平剩錢通三百萬緡為率,給諸路,諸路措置,各分命官。」詔悉聽焉。



 俄定諸路措置茶事官置司:湖南於潭州,湖北于荊南,淮南於揚州,兩浙於蘇州,江東於江寧府,江西於洪州。其置場所在:蘄州即其州及蘄水縣,壽州以霍山、開順,光州以光山、固始,舒州即其州及
 羅源、太湖,黃州以麻城,廬州以舒城,常州以宜興,湖州即其州及長興、德清、安吉、武康,睦州即其州及青溪、分水、桐廬、遂安,婺州即其州及東陽、永康、浦江,處州即其州及遂昌、青田,蘇、杭、越各即其州,而越之上虞、餘姚、諸暨、新昌、剡縣皆置焉,衢、臺各即其州,而溫州以平陽。大法既定,其制置節目,不可毛舉。四年,京復議更革,遂罷官置場,商旅並即所在州縣或京師給長短引,自買於園戶。茶貯以籠篰,官為抽盤,循第敘輸息訖,批引販賣,
 茶事益加密矣。



 大觀元年,議提舉茶事司須保驗一路所產茶色高下、價直低昂,而請茶短引以地遠近程以三等之期。復慮商旅影挾舊引,冒詐規利,官吏因得擾動,以御筆申飭之。又以諸路再定茶息,多寡或不等,令後各增錢十。三年,計七路一歲之息一百二十五萬一千九百餘緡,榷貨務再歲一百十有八萬五千餘緡。京專用是以舞智固權,自是歲以百萬緡輸京師所供私奉,掊息益厚,盜販公行,民滋病矣。



 政和二年,大增損茶
 法。凡請長引再行者,輸錢百緡,即往陜西,加二十,茶以百二十斤;短引輸緡錢二十,茶以二十五斤。私造引者如川錢引法。歲春茶出,集民戶約三歲實直及今價上戶部。茶籠篰並皆官制,聽客買,定大小式,嚴封印之法。長短引輒竄改增減及新舊對帶、繳納申展、住賣轉鬻科條悉具。初,客販茶用舊引者,未嚴斤重之限,影帶者眾。於是又詔凡販長引斤重及三千斤者,須更買新引對賣,不及三千斤者,即用新引以一斤帶二斤鬻之,
 而合同場之法出矣。場置於產茶州軍,而簿給於都茶場。凡不限斤重茶委官司秤制,毋得止憑批引為定,有贏數即沒官,別定新引限程及重商旅規避秤制之禁,凡十八條,若避匿抄札及擅賣,皆坐以徒。復慮茶法猶輕,課入不羨,定園戶私賣及有引而所賣逾數,保內有犯不告,並如煎鹽亭戶法。短引及食茶關子輒出本路,坐以二千里流,賞錢百萬。



 重和元年,詔:「客販輸稅,檢括抵保,吏因擾民,其蠲之。」未幾,復輸稅如舊。大抵茶、鹽之法,
 主於蔡京,務巧掊利,變改法度,前後相逾,民聽眩惑。初,令茶戶投狀籍於官,非在籍者,禁與商旅貿易,未幾即罷。初,限計斤重,令買新引,茶有贏者,即及一千五百斤,須用新引貼販,或止願販新茶帶賣者聽;未幾,以帶賣者多,又罷其令。



 陜西舊通蜀茶,崇寧二年,始通東南茶。政和中,陜西沒官茶令估賣,繼以妨商旅,下令焚棄。俄令正茶沒官者聽與販,引外剩茶及私茶數以給告者。長引限以一年,短引限以半歲繳納。久之,令已買引而
 未得於園戶者,期七年,許民間同見緡流轉,長引聽即本路住賣,以二浙鹽香司有言而止。其科條纖悉紛更,不可勝記,慮商旅疑豫,茶貨不通,乃重扇搖之令。於時掊克之吏,爭以贏羨為功,朝廷亦嚴立比較之法。州郡樂賞畏刑,惟恐負課,優假商人,陵轢州郡,蓋莫有言者。獨邠州通判張益謙奏:「陜西非產茶地,奉行十年,未經立額,歲歲比較,第務增益,稍或虧少,程督如星。州縣懼殿,多前路招誘豪商,增價以幸其來,故陜西茶價,斤有
 至五六緡者,或稍裁之,則批改文引,轉之他郡。及配之鋪戶,安能盡售?均及稅農,民實受害,徒令豪商坐享大利。」言竟不行。



 然自茶法更張,至政和六年,收息一千萬緡,茶增一千二百八十一萬五十一百餘斤。及方臘竊發,乃詔權罷比較。臘誅,有司議招集園戶,借貸優恤,止於文具,奸臣仍用事,蠹國害民,又慮人言,扇搖之令復出矣。靖康元年,詔川茶侵客茶地者,以多寡差定其罪。



 初,熙寧五年,以福建茶陳積,乃詔福建茶在京、京東西、
 淮南、陜西、河東仍禁榷,餘路通商。元豐七年,王子京為福建轉運副使,言「建州臘茶,舊立榷法,自熙寧權聽通商,自此茶戶售客人茶甚良,官中所得惟常茶,稅錢極微,南方遺利,無過於此,乞仍舊行榷法。建州歲出茶不下三百萬斤,南劍州亦不下二十餘萬斤,欲盡買入官,度逐州軍民戶多少及約鄰路民用之數計置,即官場賣,嚴立告賞禁。建州賣私末茶,借豐國監錢十萬緡為本。」並從之;所請均入諸路榷賣,委轉運司官提舉:福建
 王子京,兩浙許懋,江東杜偉,江西朱彥博,廣東高鎛,然子京蓋未免抑配於民。



 時遠方若桂州修仁諸縣、夔州路達州有司皆議榷茶,言利者踵相躡,然神宗聞鄂州失催茶稅,輒蠲之。建州園戶等以茶粗濫當剝納,為錢三萬六千餘緡,慮其不能償,令準輸茶。初,成都帥司蔡延慶言邛部川蠻主苴克等願賣馬,即詔延慶以茶招來,後聞邊計蠻情非便,即罷之。哲宗嗣位,御史安惇首劾王子京買臘茶抑民,詔罷子京事任,令福建禁榷州
 軍視其舊,餘並通商。桂州修仁等縣禁榷及陜西碎賣芽茶皆罷。



 崇寧二年,尚書有言:「建、劍二州茶額七十餘萬斤,近歲增盛,而本錢多不繼。」詔更給度牒四百,仍給以諸色封樁。繼詔商旅販臘茶蠲其稅,私販者治元售之家,如元豐之制。臘茶舊法免稅,大觀三年,措置茶事,始收焉。四年,私販勿治元售之家,如元符令。政和初,復增損為新法。三年,詔免輸短引,許依長引於諸路住賣,後末骨茶每長引增五百斤,短引仿此;諸路監司、州郡
 公使食茶禁私買,聽依商旅買引。六年,詔福建茶園如鹽田,量土地產茶多寡,依等第均稅。重和元年,以改給免稅新引,復位福建末茶斤重,長引以六百斤為率。



 元豐中,宋用臣都提舉汴河堤岸,創奏修置水磨。凡在京茶戶擅磨末茶者有禁,並許赴官請買。而茶鋪入米豆雜物揉和者募人告,一兩賞三千,及一斤十千,至五十千止。商賈販茶應往府界及在京師,須令產茶山場州軍給引,並赴京場中賣,犯者依私販臘茶法。諸路末茶
 入府界者,復嚴為之禁。訖元豐末,歲獲息不過二十萬,商旅病焉。



 元祐初,寬茶法,議者欲罷水磨。戶部侍郎李定以失歲課,持不可廢;侍御史劉摯、右司諫蘇轍等相繼論奏,遂罷。紹聖初,章惇等用事,首議修復水磨。乃詔即京、索、大源等河為之,以孫迥提舉,復命兼提舉汴河堤岸。四年,場官錢景逢獲息十六萬餘緡,呂安中二十一萬餘緡,以差議賞。元符元年,戶部上凡獲私末茶並雜和者,即犯者未獲,估價給賞,並如私臘茶獲犯人法。
 雜和茶宜棄者,斤特給二十錢,至十緡止。



 初,元豐中修置水磨,止於在京及開封府界諸縣,未始行於外路。及紹聖復置,其後遂於京西鄭、滑、穎昌府,河北澶州皆行之,又將即濟州山口營置。崇寧二年,提舉京城茶場所奏:「紹聖初,興復水磨,歲收二十六萬餘緡。四年,於長葛等處京、索、潩水河增修磨二百六十餘所,自輔郡榷法罷,遂失其利,請復舉行。」從之。尋詔商販臘茶入京城者,本場盡買之,其翻引出外者,收堆垛錢。裁元豐制更立
 新額,歲買山場草茶以五百萬斤為率。客茶至京者,許官場買十之三,即索價故高,驗元引買價量增。三年,詔罷之。



 明年,改令磨戶承歲課視酒戶納曲錢法。五年,復罷民戶磨茶,官用水磨仍依元豐法,應緣茶事並隸都提舉汴河堤岸司。大觀元年,改以提舉茶事司為名,尋命茶場、茶事通為一司。三年,復撥隸京城所,一用舊法。政和元年,京城所請商旅販茶起引定入京住賣者,即許借江入汴,如元豐舊制;其借江入汴卻指他路住賣
 者禁,已請引者並令赴京。二年,以課入不登,商賈留滯,詔以其事歸尚書省。於是尚書省言:「水磨茶自元豐創立,止行於近畿,昨乃分配諸路,以故至弊,欲止行於京城,仍行通客販,餘路水磨並罷。」從之。四年,收息四百萬貫有奇,比舊三倍,遂創月進。



 高宗建炎初,於真州印鈔,給賣東南茶鹽。當是時,茶之產於東南者,浙東西、江東西、湖南北、福建、淮南、廣東西,路十,州六十有六,縣二百四十有二。霅川顧渚生石上者謂之紫筍,毗陵之陽羨,
 紹興之日鑄,婺源之謝源,隆興之黃龍、雙井,皆絕品也。建炎三年,置行在都茶場,罷合同場十有八,惟洪、江、興國、潭、建各置場一,監官一。罷食茶小引,捕私茶法視捕私鹽。二十一年,秦檜等始進《茶鹽法》。先是,臣僚或因事建明,朝廷亦因時損益,至是審訂成書,上之。



 孝宗隆興二年,淮東宣諭錢端禮言:「商販長引茶,水路不許過高郵,陸路不許過天長。如願往楚州及盱眙界,引貼輸翻引錢十貫五百文;如又過淮北,貼輸亦如之。」當是時,商
 販自榷場轉入虜中,其利至博,幾禁雖嚴,而民之犯法者自若也。乾道二年,戶部言:「商販至淮北榷場折博,除輸翻引錢,更輸通貨儈息錢十一緡五百文。」八年,減輸翻引錢止七緡,通貨儈見錢止八緡。淳熙二年,以長短茶引權以半依原引斤重錢數,分作四緡小引印給,而翻引貼輸錢隨小引輸送。光宗紹熙初,漳州守臣朱熹奏除屬邑科茶七千餘緡。臣僚申明長短小引相兼,從人之便。戶部言給賣小引,除金銀、會子分數入輸,餘願
 專以會子算請者聽。



 寧宗嘉泰四年,知隆興府韓邈奏請:「隆興府惟分寧縣產茶,他縣無茶,而豪民武斷者乃請引,窮索一鄉,使認茶租,非便。」於是禁非產茶縣不許民擅認茶租。



 建寧臘茶,北苑為第一,其最佳者曰社前,次曰火前,又曰雨前,所以供玉食,備賜予。太平興國始置,大觀以後制愈精,數愈多,胯式屢變,而品不一,歲貢片茶二十一萬六千斤。建炎以來,葉濃、楊勍等相因為亂,園丁亡散,遂罷之。紹興二年,蠲未起大龍鳳茶一
 千七百二十八斤。五年,復減大龍鳳及京鋌之半。十二年,興榷場,遂取臘茶為榷場本,凡胯、截、片、鋌,不以高下多少,官盡榷之,申嚴私販入海之禁。議者請鬻建茶於臨安,移茶司事於建州買發。明年,以失陷引錢,復令通商。自是上供龍鳳、京鋌茶料,凡制作之費、篚笥之式,令漕司專之。



 蜀茶之細者,其品視南方已下,惟廣漢之趙坡,合州之水南,峨眉之白牙,雅安之蒙頂,土人亦珍之,但所產甚微,非江、建比也。舊無榷禁,熙寧間,始置提舉司,
 收歲課三十萬;至元豐中,累增至百萬。建炎元年,成都轉運判官趙開言榷茶、買馬五害,請「用嘉祐故事盡罷榷茶,而令漕司買馬。或未能然,亦當減額以蘇園戶,輕價以惠行商,如此則私販衰而盜賊息。」遂以開同主管川、秦茶馬。二年,開至成都,大更茶法,仿蔡京都茶場法,以引給茶商,即園戶市茶,百斤為一大引,除其十勿算。置合同場以譏其出入,重私商之禁,為茶市以通交易。每斤引錢春七十、夏五十,市利頭子錢不預焉。所過徵
 一錢,所止一錢五分。自後引息錢至一百五萬緡。至十七年,都大茶馬韓球盡取園戶加饒之茶為額,茶司歲收二百萬,而買馬之數不加多。



 乾道末年,青羌作亂,茶司增長細馬名色等錢歲三十萬。淳熙六年以後,累減園戶重額錢十六萬,又減引息錢十六萬。至紹熙初,楊輔為使,遂定為法。成都府、利州路二十三場,歲產茶二千一百二萬斤,通博馬物帛歲收錢二百四十九萬三千餘緡。朝廷歲以一百一十三萬緡隸總領所贍軍,然
 茶馬司率多難之;乾道以後,歲撥止一二十萬緡,至淳熙十年,遂以五十萬緡為準。



 自熙、豐以來,茶司官權出諸司之上。初,元豐開川、秦茶場,園戶既輸二稅,又輸土產;隆安縣園戶二稅、土產兼輸外,又催理茶課估錢;建炎元年立為額,至寧宗慶元初,始除之。六年,詔四川產茶處歲輸經總制頭子錢五千四十一道有奇,又科租錢三千一百四十道有奇。



 宋初,經理蜀茶,置互市於原、渭、德順三郡,以市蕃夷之馬;熙寧間,又置場於熙河。南
 渡以來,文、黎、珍、敘、南平、長寧、階、和凡八場,其間盧甘蕃馬歲一至焉,洮州蕃馬或一月或兩月一至焉,疊州蕃馬或半年或三月一至焉,皆良馬也。其它諸蕃馬多駑,大率皆以互市為利,宋朝曲示懷遠之恩,亦以是羈縻之。紹興二十四年,復黎州及雅州碉門靈犀砦易馬場。乾道初,川、秦八場馬額九千餘匹,淳熙以來,為額萬二千九百九十四匹,自後所市未嘗及焉。



\end{pinyinscope}