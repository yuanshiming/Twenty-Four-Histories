\article{列傳第一百六十二}

\begin{pinyinscope}

 ○趙
 方賈涉扈再興孟宗政張威



 趙方,字彥直,衡山人。父棠,少從胡宏學,慷慨有大志。嘗見張浚於督府,浚雅敬其才,欲以右選官之,棠不為屈。累以策言兵事,浚奇之,命子栻與棠交,方遂從栻學。



 淳
 熙八年舉進士,調蒲圻尉,疑獄多所委決。授大寧監教授,俗陋甚,方擇可教者親訓誘之,人皆感勵,自是始有進士。知青陽縣,告其守史彌遠曰:「催科不擾,是催科中撫字;刑罰無差,是刑罰中教化。」人以為名言。



 主管江西安撫司機宜文字,京湖帥李大性闢知隨州。南北初講和,旱蝗相仍,方親走四郊以禱,一夕大雨,蝗盡死,歲大熟。適和議成,諸郡浸弛備,方獨招兵擇將,拔土豪孟宗政等補以官。提舉京西常平兼轉運判官、提點刑獄。時
 劉光祖以耆德為帥,方事以師禮,自言:「吾性太剛,每見劉公,使人更和緩。」嘗請光祖書「勸謹和緩」四字,揭坐隅以為戒。以金部員外郎召,尋加直秘閣,改湖北轉運判官兼知鄂州。升直煥章閣兼權江陵府,增修三海八匱,以壯形勢。進秘閣修撰、知江陵府、主管湖北要撫司事兼權荊湖置司。



 時金逼於兵,計其必南徙,日夜為備。荊門有東西兩山險要,方築堡其上,增戍兵以遏其沖。進右文殿修撰。金樊快明謀歸宋,追兵至襄陽,方遣孟宗
 政、扈再興以百騎邀之,殺千餘人,金人遁去。權工部侍郎、寶謨閣待制、京湖制置使兼知襄陽府。諜知金人決意犯境,乃下防夏之令。金相高琪及其樞密烏古倫慶壽犯陳、光化、隨、棗理、信陽、均州、方夜半呼其子範、葵曰:「朝廷和戰之說未定,觀此益亂人意,吾策決矣,惟有提兵臨邊決戰以報國耳。」遂抗疏主戰,親往襄陽。



 金人圍棗陽急,方遣宗政、再興等援棗陽,仍增戍光化、信陽、均州,以聯聲勢。已而棗陽守趙觀敗金人於城外,再興、宗
 政至,與觀夾擊,又敗之,棗陽圍解。



 方申飭諸將,當遏於境上,不可使之入而後拒之於城下。時麥正熟,方遣兵護民刈之,令清野以俟。再疏力陳不可和者七,戰議遂定。



 金將完顏賽不入境,兵號十萬。方部分諸將,金人犯棗陽者,宗政敗之於尚家川;犯隨州者,劉世興敗之於磨子平。相持逾年,方調世興移師,與許國、再興援棗陽;張興、李雄韜援隨州。隨州圍解,再興等轉戰入棗陽。時宗政守城,伏兵城東,金人遇伏敗走。未幾再至,再興又
 敗之,自是無日不戰。金人三面來攻,宗政出東門,再興出南門,世興出北門,大合戰敗之。金人朝進莫退,力不能捍;諸將表裏合謀,國自南山進,張威自瀼河進,世興、李琪出城與國會,再興出城與威會,掎角追擊,金人遂潰。光化守潘景伯亦設伏敗金人於趙家橋,孟宗德又破之於隨州鴨兒山,擒賽不妻弟王丑漢,金人遂誅賽不。方以功遷龍圖閣待制,封長沙縣男,賜食邑。



 金人復大舉,命訛可圍棗陽,塹其外,繞以土城。方計其空巢穴
 而來,若搗其虛,則棗陽之圍自解。乃命國東向唐州,再興西向鄧州,又命子範監軍,葵後殿。



 時宗政在城中,日夜鏖戰,焚其攻具,金人不敢近城。西師由光化境出,砦於三尖山,拔順陽縣,金人率眾仰攻,大敗。再興與國兩道並進,掠唐、鄧境,焚其城柵糧儲。棗陽城堅,金頓兵八十餘日,方知其氣已竭,乃召國、再興還,並東師隸於再興,克期合戰。再興敗金人於瀼河,又敗之城南,宗政自城中出夾擊,殺其眾三萬,金人大潰,訛可單騎遁,獲其
 貲糧、器甲不可勝計。進方煥章閣直學士。奏乞均官軍民兵廩給,自備馬者倍之。又奏:「使民兵夏歸,以省月給,秋復詣屯守御。」



 從之。



 方料金人數不得志於棗陽,必將同時並攻諸城,當先發以制之。命國、宗政出師向唐,再興向鄧,戒之曰:「毋深入,毋攻城,第潰其保甲,毀其城砦,空其貲糧而已。」宗政進破湖陽縣,擒其千戶趙興兒;國遣部將耶律均與金人戰於比陽,戮其將李提控;再興破高頭城,大敗金兵,遂薄鄧州。唐州兵來援,迎敗之,降
 者踵至。已而金兵至樊城,方命再興陣以待之,方視其師;金人三日不敢動,遂遁。



 金將駙馬阿海犯淮西,樞密完顏小驢屯唐州為後繼。方先攻唐伐其謀,及使再興發棗陽兵擊其西,國發桐柏兵擊其東。再興敗金人於唐城,斬小驢,圍其城五匝,垂下。會蘄、黃繼陷,詔趣方遣救,方亟命國保鄂,再興援淮西。國還鄂州保江;再興軍至蘄之靈山,伺金人歸而擊之,土豪祝文蔚橫突入陣,金人大敗,國遣張寶將兵來會,李全等兵亦至,金人遂
 潰,再興追逐六十里,擒其監軍合答。進方顯謨閣直學士、太中大夫、權刑部尚書。



 俄得疾,進徽猷閣學士、京湖制置大使。歸還,力疾犒師,第其功上之。病革,曰:「未死一日,當立一日紀綱。」引再興臥內,勉以協心報國。貽書宰相,論疆場大計。尋卒。是夕有大星隕於襄陽。以端明殿學士、正議大夫致仕,贈銀青光祿大夫,累贈太師,謚忠肅。



 方起自儒生,帥邊十年,以戰為守,合官民兵為一體,通制總司為一家。持軍嚴,每令諸將飲酒勿醉,當使日
 日可戰。淮、蜀沿邊屢遭金人之禍,而京西一境獨全。嘗問相業於劉清之,清之以留意人才對。故知名士如陳晐、游九功輩皆拔為大吏,諸名將多在其麾下。若扈再興、孟宗政皆起自土豪,推誠擢任,致其死力,藩屏一方,使朝廷無北顧之憂。故其沒也,人皆惜之。子董、薿、範、葵。範、葵有傳。



 賈涉,字濟川,天臺人。幼好讀古書,慷慨有大志。以父任高郵尉,改萬安丞。



 寶應擇令,堂差涉至邑,請城之。役興,
 以憂去。金人犯光州,起涉竟前役。通判真州,改大理司直、知盱眙軍。



 淮人季先、沈鐸說楚州守應純之以招山東人,純之令鐸遣周用和說楊友、劉全、李全等以其眾至,先招石珪、葛平、楊德廣,通號「忠義軍」。珪等反,斃鐸於漣水,純之罷,通判梁丙行守事,欲省其糧使自潰。珪、德廣等以漣水諸軍度淮屯南渡門,焚掠幾盡。謂:「朝廷欲和殘金,置我軍何地?」丙遣李全、季先拒之,不止,事甚危。涉時在寶應,上書曰:「降附踵至,而金乃請和,此正用高
 澄間侯景遣策,恐山東之禍必移於兩淮。況金人所乏惟財與糧,若舉數年歲幣還之,是以肉啖餒虎,啖盡將反噬。至若忠義之人源源而來,不立定額,自為一軍,處之北岸,則安能以有限之財應無窮之須?饑則噬人,飽則用命,其勢然也。」授淮東提點刑獄兼楚州節制本路京東忠義人兵。涉亟遣傅翼諭珪等逆順禍福,自以輕車抵山陽,德廣等郊迎,伏地請死,誓以自新。



 金太子及僕散萬忠、盧國瑞等數十萬大入,且以計誘珪等。涉慮
 珪等為金用,亟遣陳孝忠向滁州,珪與夏全、時青向濠州,先、平、德廣趨滁、濠,李全、李福要其歸路,以傅翼監軍。數日,孝忠捷至,珪屢破金人,遂與先及李全趨安豐。時金人環百餘砦,攻具甫畢,珪等解其圍,李全挾僕散萬忠以歸,見《李全傳》。金人不救窺淮東者六七年。



 南渡門之變,平、德廣等實預,涉既受降,置弗問。平等尚懷異志,涉密使先以計殺之,而先之勢亦孤。忠義諸軍在漣水、山陽者既眾,涉慮其思亂,因滁、濠之役,分珪、孝忠、夏全
 為兩屯,李全軍為五砦,又用陜西義勇法涅其手,合諸軍汰者三萬有奇,涅者不滿六萬人,正軍常屯七萬餘,使主勝客,朝廷歲省費十三四。



 涉又遣李全以萬人取海州,復取密、濰。王琳以寧海州歸,遂收登、萊二州。



 青州守張林以濱、棣、淄州降,又取濟、沂等州。自是恩、博、景、德至邢、洺十餘州相繼請降。涉傳檄中原:「以地來歸及反戈自效者,朝廷裂地封爵無所吝。」



 仍厲諸將,圖未下州郡。擢太府少卿、制置副使兼京東、河北節制。



 金十餘萬
 眾犯黃州,淮西帥趙善湘請援於朝,涉遣李全等赴之,翟朝宗等為後繼。丞相史彌遠擬升全留後,涉曰:「始全貧窶無聊,能輕財與眾同甘苦,故下樂為之用。逮為主帥,所為反是,積怨既多,眾皆不平。近棄西城,免死為幸,若無故升遷以驕其志,非全之福,亦豈國家之福。曷若待事定,與諸將同升可也。」金人破黃陷蘄,安慶甚危,全馳至,遂定。全至久長鎮,與京湖制置使趙方二子範、葵遇,掎角連戰俱勝,遣彭義斌等進至下灣渡,盡掩金人
 於淮。遷權吏部侍郎。金人再犯淮西。先是,蘄州受圍,徐暉往援,乃鼓眾宵遁,金乘間登城,一郡為血,前帥不敢問。涉斬暉以徇,諸將畏懼,無不用命,淮西之勢大振。



 初,翟朝宗得玉璽獻諸朝,至是趙拱還,又得玉印,文與璽同而加大。朝廷喜璧之歸,行慶賞。涉遺書彌遠謂:「天意隱而難明,人事切而易見,當思今日人事尚未有可答天意者。昔之患不過亡金,今之患又有山東忠義與北邊,宜亟圖之。」



 彌遠不懌,李全卒以璽賞為節度使。涉又
 言:「盜賊血氣正盛,官職過分,將有後憂。」彌遠不以為然。涉曰:「朝廷但知官爵可以得其心,寧知驕則將至於不可勸邪?」



 涉時已疾,力辭事任。值金人大入,強起視事。金將時全、合連、孛術魯答哥率細軍及眾軍三道渡淮,涉以合連善戰,乃命張惠當之。惠,金驍將,所謂「賽張飛」者,既歸宋,金人殺其妻,所部花帽軍,有紀律,它軍不及也。惠率諸軍出戰,自辰至酉,金人大敗,答哥溺死,陷失太半,細軍喪者幾二千。涉既病,乃以所獲京、河版籍及金
 銀牌銅印之屬上於朝。卒,超贈龍圖閣學士、光祿大夫。



 涉父偉嘗守開江,貽書丞相趙雄,極論武興守吳挺之橫,它日陛對,又乞裁抑郭棣、郭杲兵權,孝宗嘉納,後反為所擠以沒。涉弱冠直父冤,不避寒暑,泣訴十年,至伏書闕下。子似道有傳。



 扈再興,字叔起,淮人也。有膂力,善機變。每戰,被發肉袒徒跣,揮雙刃奮呼入陣,人馬闢易。金人犯襄陽、棗陽,京西制置使趙方檄再興等御之。金人來自團山,勢如風
 雨。再興同孟宗政、陳祥分三陳,設伏以待。既至,再興中出一陳,復卻,金人逐之,宗政與祥合左右兩翼掩擊之,金人三面受敵,大敗,血肉枕藉山谷間。授神勁統制。又犯棗陽,再興率師赴援,金人聞風夜潰。既而益兵數萬復圍城,相持九十日。再興夜以鐵蒺藜密布地,黎明佯遁,金人馳中蒺藜者十踣七八。



 敵卻走,追至十五里岡。已而金兵攻城東隅,薄南門北角,再興與宗政、劉世興各當一面,大戰數十合,大敗金兵。金帥完顏訛可擁步
 騎數萬傅城,再興與宗政縱之涉濠,半渡擊之;又令守壩者佯走,金人爭壩,急擊之,多墮水中。金人創對樓、鵝車、革洞,決濠水,運土石填城下。再興募死士著鐵面具,披氈,列陳以待之。



 金人計無所施而去,棄旗甲輜重滿野。大戰於範家莊,金人敗,追之至泊湖,禽其巡檢亢師禮酒、都監納蘭福昌,降其壯丁,獲牛馬甚眾。



 自是與宗政、世興無日不戰。再興又破順昌縣,奪甲馬三千,破淅川鎮,殺金人三百,追至馬磴砦,焚其城柵。又敗其護駕
 騎軍於瀼河。入鄧州,破高頭,敗其步軍五千、騎軍五百,焚其積聚。遂營於高頭,進攻唐州,至三家河,金騎軍二千、步軍七千出城迎戰,又敗之,死者十七八,追及城下。金將從義者收殘騎三百奔城,再興據門拒戰,斬從義。遂圍唐州,分兵焚蕩州境,截其歸路,砦於久長,嚴陳以待之。搜剿殘兵,獲其副統軍廣威將軍衲撻達。金兵殲,乃斂髑髏立人頭堠。



 尋以病卒。子世達,亦以名將稱,官至都統制。



 孟宗政,字德夫,絳州人。父林,從岳飛至隨州,因家焉。宗政自幼豪偉,有膽略,常出沒疆場間。開禧二年,金將完顏董犯襄、郢,宗政率義士據險游擊,奪其輜重。宣撫使吳獵奇之,補承節郎、棗陽令。京西路分趙方、吳柔勝皆薦其才,轉秉義郎、京西鈐轄,駐札襄陽。



 嘉定十年,金人犯襄陽、棗陽,方檄宗政節制神勁、報捷、忠義三軍。宗政與統制扈再興、陳祥分為三軍,設覆三所,蹀血以戰,金兵敗走。尋報棗陽圍急,宗政午發峴首,遲明抵棗陽,馳
 突如神。金人大駭,宵遁。方時移帥京西,聞捷大喜,差權棗陽軍。初視事,一愛僕犯新令,立斬之,軍民股慄。於是築堤積水,修治城堞,簡閱軍士。



 十一年,金帥完顏賽不擁步騎圍城,宗政與再興合兵角敵,歷三月,大小七十餘戰,宗政身先士卒。金人戰輒敗,忿甚,周城開濠,四面控兵列濠外,飛鋒鏑,以綯鈴自警,鈴響則犬吠。宗政厚募壯士,乘間突擊,金人不能支,盛兵薄城,宗政隨方力拒。隨守許國援師至白水,鼓聲相聞。宗政率諸將出戰,
 金人奔潰。賜金帶,轉武德郎。



 十二年,金帥完顏訛可擁步騎傅城,宗政囊糠盛沙以覆樓棚,列甕瀦水以堤火,募炮手擊之,一炮輒殺數人。金人選精騎二千,號弩子手,擁雲梯、天橋先登,又募鑿銀礦石工晝夜塪城,運茅葦直抵圜樓下,欲焚樓。宗政先毀樓,掘深坑,防地道;創戰棚,防城損;穿阱才透,即施毒煙烈火,鼓鞁以熏之。金人窒以濕氈,析路以刳士,城頹樓陷。宗政撤樓益薪,架火山以絕其路,列勇士,以長槍勁弩備其沖。距樓陷所
 數丈築偃月城,袤百餘尺,翼傅正城,深坑倍仞,躬督役,五日成。



 金人摘強兵披厚鎧、氈衫、鐵面而前,又濕氈濡革蒙火山,覆以冰雪,擁雲梯徑抵西北圜樓登城。城中軍以長戈舂其喉,殺之;敢勇軍自下夾擊金兵,兵墜死燎焰。



 金將於後截其軍,拒馬揮刀迫前,自昕至昃,死傷踵接,梯橋盡毀。金人連不得志,俄乘順風渡濠,飛脂革燒戰棚,宗政激將士血戰,凡十五陣,矢石交,金兵死者千餘,弩子手十七八,射其都統殪。天反風,金人愈忿,炮
 愈急。會王大任領銳卒一千冒重圍轉鬥入城,內外合勢,士氣大振,賈勇入金營,自晡至三更,金人橫尸遍地,奪其銅印十有六,訛可棄帳走,獲輜重牛馬萬計。捷至,朝廷方錄前戰守功,升武功大夫兼閣門宣贊舍人,重賜金帶。



 制置司以湖陽縣迫境金兵,檄宗政圖之。宗政一鼓而拔,燔燒積聚,夷蕩營砦,俘掠以歸。金人自是不敢窺襄、漢、棗陽。許國移金陵,宗政代為荊鄂都統制,仍知棗陽。宗政以迫濠而陳,乃於西北濠外瀦水為濘以
 限騎。中原遺民來歸者以萬數。



 宗政發廩贍之,為給田、創屋與居,籍其勇壯號「忠順軍」,俾出沒唐、鄧間,威振境外。金人呼為「孟爺爺」。俄病疽卒。轉右武大夫、團練使、防禦使。



 宗政於有功者怨必賞,有罪者親必罰。好賢樂善,出於天性。未嘗學兵法,而暗與之合。死之日,邊城為罷市慟哭。子珙,有傳。



 張威,字德遠,成州人。策選鋒軍騎兵也。軍中馬料多,匹馬給米五石,騎軍利其餘以自給。總領核實裁抑,威逃
 去。帥郭杲使其父招之歸,送隆慶府後軍效用。



 威貧甚,賣藥自給。或言其才勇,乃令戍邊。開禧用兵,威與金人戰輒捷,屢以功補本軍將領。



 吳曦既誅,遣將收復。李貴復西和州,威率眾先登,敗金人,戰於板橋,遂取西和,升統制。由是威名大振。天水縣當金人西入路,乃升縣為軍,命威為守,屢立奇功,擢充利州副都統制。丁父憂,服除,帶御器械。久之,調荊鄂都統制、襄陽府駐札,改沔州都統制。



 嘉定十二年,金人分道入蜀,犯湫池堡,又犯白
 環堡。威部將石宣、董炤連卻之。既而金人犯成州,威自西和退保仙人原。時興元都統制吳政戰死黃牛堡,李貴代政,亟走武休,金人已破武休,遂陷興元,又陷大安軍。



 先是,利州路安撫使丁焴聞金人深入,亟遣書招威東入救蜀,又檄忠義總管李好古北上捍禦。好古出魚關與統領張彪遇,以彪棄迷竹關故,斬之。彪,威弟也。



 威聞彪死,按兵不進。焴聞之,謂僚佐曰:「吳政身死,李貴復以兵敗,金人所憚惟威。今好古擅殺其弟,失威心,奈何?
 且金人在東,非威地分,今可無好古,不可無威。」遂因好古入見,數其擅殺彪罪,斬之。遣書速威進救蜀,且使進士田遂往說之。威感激,夜半調發,鼓行而前,破金人於金斗鎮。金人雖敗未退,威頓兵不動,潛遣石宣等襲於大安軍,大破之。金人之來也,擇兩齒馬及精兵凡三千人,至是殲焉,俘其將巴土魯,大將包長壽聞之宵遁。



 興元叛兵張福、莫簡作亂,以紅帕蒙首,號「紅巾隊」,焚利州,殺總領楊九鼎,破閬、果,入遂寧,游騎在潼、漢界,將窺成
 都。制置司謂賊勢欲西,非威不可御。乃遣威提精兵六千人,自劍、綿至廣漢,盛夏暑劇,休士三日。俄安丙檄威東進,時賊自遂寧入普州茗山,威進兵重圍,絕其糧道,晝夜迫之。未幾禽福等十七人戮之,簡自殺,賊遂平。



 西夏來約夾攻金人,丙許之。遣王仕信會夏人於鞏,又命威與利帥程信、興帥陳立等分道並進。威向秦州。議初起,威謂:「金人尚強,夏人反覆,未可輕動。」



 丙不聽,卒遣威,威黽勉而行,令所部毋得輕發,諸將至城下,無功而還。
 丙怒,奏罷其兵柄。是歲,卒於利州,終揚州觀察使。



 威初在行伍,以勇見稱,進充偏裨,每戰輒克,金人聞其名畏憚之。臨陳戰酣,則精採愈奮,兩眼皆赤,時號「張紅眼」,又號「張鶻眼」,威立「凈天鶻旗」以自表。每戰不操它兵,有木棓號「紫大蟲」,圜而不刃,長不六尺,揮之掠陣,敵皆靡。荊、鄂多平川廣野,威曰:「是彼騎兵之利也,鐵騎一沖,吾步技窮矣,蜀中戰法不可用。」乃意創法,名「撒星陳」,分合不常,聞鼓則聚,聞金則散。騎兵至則聲金,一軍分為數十
 簇;金人隨而分兵,則又鼓而聚之。倏忽之間,分合數變,金人失措,然後縱擊之,以此輒勝。威御軍紀律嚴整,兵行常若銜枚,罕聞其聲。每與百姓避路,買食物則賈倍於市,迄無敢喧。晚以嗜欲多疾,故不壽云。



 論曰:宋之南渡,邊將之才何其鮮哉!或曰「江南非用武之地」,然古之善兵者,若孫武子,亦吳人也。抑先王之世,文武無二道,文武既分,宜其才之各有所偏勝也。趙方少從張栻學、許國之忠,應變之略,隱然有尊俎折沖之
 風。其部曲如扈再興、孟宗政後皆為名將,亦方之能獎率也。方之子範、葵,宗政子珙,後皆以功名自見,不愧其父,有足稱者。賈涉居方面,亦號有才,及其庶孽,竟至亡國,為可嘆也。張威者善於御眾,故所至立功云。



\end{pinyinscope}