\article{列傳第二百四十五外國二}

\begin{pinyinscope}

 秉常,毅宗之長子,母曰恭肅章憲皇后梁氏。治平四年冬即位,時年七歲,梁太后攝政。



 熙甯元年三月,遣新河
 北轉運使、刑部郎中薛宗道等來告哀,神宗問殺楊定事,宗道言殺人者先已執送之矣,乃賜詔慰之。並諭令上大首領數人姓名,當爵祿之,俟崇貴至,即行冊禮。及崇貴至,雲定奉使諒祚,常拜稱臣,且許以歸沿邊熟戶,諒祚遺之寶劍、寶鑒及金銀物。初,定之歸,上其劍、鑒而匿其金銀,言諒祚可刺,帝喜,遂擢知保安。既而夏人失綏州,以為定賣己,故殺之。至是事露,帝薄崇貴等罪而削定官,沒其田宅萬計。



 二年二月,遣河南監牧使劉航
 等冊秉常為夏國主。三月,夏人入秦州,陷劉溝堡,殺範願。既而進誓表,乞班誓詔,及請以安遠、塞門二砦易綏州。初,朝議欲官爵夏之首領,計分其勢,郭逵以為彼必不受詔,且彼既恭順,宜布以大信,不當誘之以利。秉常果不奉詔,遣都羅重進來言曰:「上方以孝治天下,奈何反教小國之臣叛其君哉!」于是前議遂罷。乃賜誓詔,而綏州待得二砦乃還。夏主受冊而二砦不歸,且欲先得綏州,遣罔萌訛以誓詔來言。及趙高往交地,萌訛
 對以朝廷本欲得二砦,地界非所約。高曰:「若然,安遠、塞門二牆墟耳,安用之!」遂罷,詔城綏州。八月,表請去漢儀,復用蕃禮,從之。十月,遣使來謝封冊。



 三年五月,夏人號十萬,築鬧訛堡。知慶州李復圭合蕃、漢兵才三千,逼遣偏將李信、劉甫、種詠等出戰,信等訴以衆寡不敵,復圭威以節制,親畫陣圖方略授之,兵進,遂大敗。復圭懼,欲自解,即執信等而取其圖略,命州官李昭用劾以故違節制,詠庾死獄中,斬信、甫,配流郭貴。復出兵邛州堡,夜入欄浪、和市,
 掠老幼數百;又襲金湯,而夏人已去,惟殺其老幼一二百人,以功告捷,而邊怨大起矣。八月,夏人遂大舉入環慶,攻大順城、柔遠砦、荔原堡、淮安鎮、東谷西穀二砦、業樂鎮。兵多者號二十萬,少者不下一二萬,屯榆林,距慶州四十里,遊騎至城下,九日乃退。鈐轄郭慶、高敏、魏慶宗、秦勃等死之。



 四年正月,種諤謀取橫山,領兵先城囉兀,進築永樂川、賞逋嶺二砦。分遣都監趙璞、燕達築撫寧故城,及分荒堆三泉、吐渾川、開光嶺、葭蘆川四砦與
 河東路修築,各相去四十餘里。二月,夏人來攻順寧砦,復圍撫寧,折繼世、高永能等擁兵駐細浮圖,去撫寧咫尺,囉兀兵勢尚完。種諤在綏德節制諸軍,聞夏人至,茫然失措,欲作書召燕達,戰怖不能下筆,顧轉運判官李南公涕泗不已。于是新築諸堡悉陷,將士千餘人皆沒。初,朝議以諤新築囉兀城,去綏德百餘里,偏梁險狹,難于饋餉,且城中無井泉,遣李評、張景憲往視之,未至而撫寧陷,遂詔棄囉兀城。五月,燕達以戍卒輜重歸自囉
 兀,為夏人邀擊,達多失亡。九月,夏遣使入貢,且以二砦易綏州,乞如舊約,詔不允。



 五年正月,夏鈐轄結勝為麟州步將王文郁戰降,授供奉官。久之,謀竄歸,事覺,詔聽其去。六月,夏人還荔原堡逃背熟戶嵬通等七十八人。閏七月,遣部將景思立、王存以涇原兵出南路,王韶由東谷徑趨武勝,未至十餘里,逢夏人戰,遂至其城,瞎藥棄城夜遁,大首領曲撒四王阿南珂出奔,乃城武勝。十二月,遣使進馬贖《大藏經》,詔賜之而還其馬。



 八年三月,
 夏人以索蕃、漢部盜人畜投南界者,牒熙河經略司請高太尉赴三岔堡會議,牒稱大安二年。乃詔鄜延經略司,令牒宥州問妄稱年號,且牒非其地分邊臣會議,皆違越生事,是必夏主不知,請問之。夏人進奉山陵後期,詔令先至永厚陵設祭後至闕奉慰。帝謂輔臣曰:「元昊昔僭號,遣使上表稱臣,其辭猶遜。朝廷不先詰其所以然而遽絕之,縱邊民蕃部討虜,故元昊嘗自謂為諸羌所立不得辭,朝廷不得命,不得已而變。西師亟戰輒敗,
 天下騷然,仁宗悔之。當元昊僭書來,獨諫官吳育謂難以中國叛臣處之,或可稍易以名號,議者皆以為不然,卒困中原,而後歲賜,封冊為夏國主,良可惜哉!」



 元豐二年六月,夏人自滿堂川入大會平,殺防田人馬,兵官李浦等逼逐出塞。九月,綏德把截楊永慶聲徼循邊而掩取蕃部首級,詐言斬犯邊人,詔毀永慶出身文字,送西京編管。



 四年四月,有李將軍清者,本秦人,說秉常以河南地歸宋,國母知之,遂誅清而奪秉常政。鄜延總管種
 諤乃疏秉常遇弑,國內亂,宜興師問罪,此千載一時之會。帝然之,遂遣王中正往鄜延、環慶,稱詔募禁兵,從者將之。詔熙河李憲等,以秉常見囚,大舉征夏;及詔諭夏國嵬名諸部首領,能拔身自歸及相率共誅國仇,當崇其爵賞,敢有違拒者誅九族。八月,中正及諤言涇原、環慶會兵取靈州,復討興州,麟府、鄜延先會夏州,取懷州渡會興州。憲總七軍及董氈兵三萬,至新市城,遇夏人,戰敗之。王中正出麟州,禡辭自言代皇帝親征,提兵六萬,
 才行數里,即奏已入夏境,屯白草平九日不進。環慶經略使高遵裕將步騎八萬七千、涇原總管劉昌祚將卒五萬出慶州,諤將鄜延及畿內兵九萬三千出綏德城。九月,諤圍米脂,夏人來救,戰于無定川,大破之,斬首五千級。十月,遂克米脂,降守將令分訛遇,進攻石州。中正以河東軍渡無定河,循水北行,地皆沙濕,士馬多陷沒,遂繼諤趨夏州,而民皆潰,軍無所得。遵裕至清遠軍,攻靈州,夏人決黃河灌營,復抄絕餉道,士卒凍溺死,余兵
 才萬三千人,遂歸。夏人追戰,將官俞平死之。中正至宥州奈王井,糧盡,士卒死亡者已二萬,乃引軍還。諤兵無食,會大雪死,遂潰,入塞者才三萬人。昌祚遇夏人于磨臍隘,夏之拒者二三萬人,昌祚乃分兵渡葫蘆河,奪其隘,與統軍國母弟梁大王戰,遂大破之。憲營于天都山下,焚夏之南牟內殿並其館庫,追襲其統軍仁多㖫丁,敗之,擒百人,遂班師。涇原總兵侍禁魯福、彭孫護饋餉至鳴沙川,與夏人三戰,敗績。初,夏人聞宋大舉,梁太后
 問策于廷,諸將少者盡請戰,一老將獨曰:「不須拒之,但堅壁清野,縱其深入,聚勁兵于靈、夏而遣輕騎抄絕其饋運,大兵無食,可不戰而困也。」梁後從之,宋師卒無功。



 五年正月,遼使涿州遺書云:「夏國來稱,宋兵起無名,不測事端。」神宗報以「夏國主受宋封爵,昨邊臣言,秉常見為母黨囚辱,比令移問事端,其同惡不報。繼又引兵數萬侵犯我邊界,義當有征。今彼以屢遭敗衄,故遣使詭情陳露,意在間貳,想彼必以悉察。」夏人聞此,遂不至。五
 月,沈括請城古烏延城以包橫山,使夏人不得絕沙漠。遂遣給事中徐禧、內侍押班李舜舉往議。禧復請于銀、夏、宥之界築永樂城。永樂依山無水泉,獨種諤極言不可,禧率諸將竟城之,賜名銀川砦;禧等還米脂,以兵萬人屬曲珍守之。永樂接宥州,附橫山,夏人必爭之地。禧等既城去,九日,夏人來攻,珍使報禧,乃挾李舜舉來援,而夏兵至者號三十萬,禧登城西望,不見其際,宋軍始懼。翌日,夏兵漸逼,禧乃以七萬陣城下,坐譙門,執黃旗令
 衆曰:「視吾旗進止!」夏人縱鐵騎渡河,或曰:「此號'鐵鷂子',當其半濟擊之,乃可有逞,得地則其鋒不可當也。」禧不聽。鐵騎既濟,震盪衝突,大兵從之,禧師敗績。將校寇偉、李思古、高世才、夏儼、程博古及使臣十餘輩、士卒八百餘人盡沒。詔李憲、張世矩往援,及令括遣人與約退軍,當還永樂地。夏人進侵,及縣門,潰歸城者,決水砦為道以登,夏人因之,奔歸于城者三萬人皆沒。夏兵圍之者厚數里,游騎掠米脂。將士晝夜血戰,城中乏水已數日,
 鑿井不得泉,渴死者大半,括等援兵及饋運皆為夏大兵所隔。夏人呼珍來講和,呂整、景思義相繼而行,夏人髡思義囚之,而城圍者已浹旬矣。夜半,夏兵環城急攻,城遂陷。高永能戰沒,禧、舜舉、運使李稷皆死于亂兵,惟曲珍、王湛、李浦、呂整裸跣走免,蕃部指揮馬貴獨誓死持刀殺數十人而沒。是役也,死者將校數百人,士卒、役夫二十余萬,夏人乃耀兵米脂城下而還。宋自熙寧用兵以來,凡得葭蘆、吳保、義合、米脂、浮圖、塞門六堡,而靈
 州、永樂之役,官軍、熟羌、義保死者六十萬人,錢、粟、銀、絹以萬數者不可勝計。帝臨朝痛悼,而夏人亦困弊。夏西南都統、昴星嵬名濟乃移書劉昌祚曰:



 中國者,禮樂之所存,恩信之所出,動止猷為,必適于正。若乃聽誣受間,肆詐窮兵,侵人之土疆,殘人之黎庶,是乖中國之體,為外邦之羞。昨者朝廷暴興甲兵,大窮侵討,蓋天子與邊臣之議,為夏國方守先誓,宜出不虞,五路進兵,一舉可定。故去年有靈州之役,今秋有永樂之戰,然較其勝負,
 與前日之議,為何如哉!



 朝廷于夏國,非不經營之,五路進討之策,諸邊肆撓之謀,皆嘗用之矣。知徼幸之無成,故終于樂天事小之道。況夏國提封一萬里,帶甲數十萬,南有于闐作我歡鄰,北有大燕為我強援,若乘間伺便,角力競鬥,雖十年豈得休哉!即念天民無辜,受此塗炭之苦,國主自見伐之後,夙夜思念,為自祖宗之世,事中國之禮無或虧,貢聘不敢怠,而邊吏幸功,上聰致惑,祖宗之盟既阻,君臣之分不交,存亡之機,發不旋踵,朝
 廷豈不恤哉!



 至于魯國之憂,不在顓臾,隋室之變,生于楊感。此皆明公得于胸中,不待言而後喻。今天下倒垂之望,正在英才,何不進讜言,辟邪議,使朝廷與夏國歡好如初,生民重見太平,豈獨夏國之幸,乃天下之幸也。



 昌祚上其書,帝喻答之。



 六年二月,夏人大舉圍蘭州,已奪西關門,鈐轄王文鬱集死士七百,夜縋城而下,持短兵突營,遂拔去。五月,復來,圍九日,大戰,侍禁韋禁死之,乃解去。閏六月,遣使謨個、咩迷乞遇來貢,表曰:「夏國累
 得西蕃木征王子書,稱南朝與夏國交戰歲久,生靈荼毒,欲擬通和。緣夏國先曾請所侵疆土,不從;以此未便輕許。西蕃再遣使散入昌郡、丹星等到國,稱南朝語言計會,但當遣使齎表,自令引赴南朝。切念臣自曆世以來,貢奉朝廷,無所虧怠,至于近歲尤甚歡和。不意憸人誣間,朝廷特起大兵,侵奪疆土城砦,因茲構怨,歲致交兵。今乞朝廷示以大義,特還所侵,倘垂開納,別效忠勤。」乃賜詔曰:「頃以權強,敢行廢辱,朕用震驚,令邊臣往問,匿
 而不報,王師徂征,蓋討有罪。今遣使造庭,辭禮恭順,仍聞國政悉復故常,益用嘉納。已戒邊吏毋輒出兵,爾亦其守先盟。」遂詔陝西、河東經略司,其新復城砦,徼循毋出三二里,夏之歲賜如舊。



 七年正月,圍蘭州,李憲戰卻之。六月,攻德順軍,巡檢王友戰死。九月,圍定西城,燒龕穀族帳,遂以十月至靜邊,鈐轄彭孫敗之,殺其首領仁多㖫丁。十二月攻清遠,隊將白玉、李貴死之。



 八年三月,神宗崩,賜以遺留物。夏人攻葭蘆,供奉王英戰死。七
 月,遣使丁拿嵬名謨鐸、副使呂則陳聿精等來奠慰。十月,遣芭良、嵬名濟、賴升聶、張聿正進助山陵禮物。夏國主母梁氏薨,訃至,以朝散郎、邢部郎中杜紘充祭奠使,東頭供奉官、閣門祗候王有言充吊慰使。夏以主母遺留物來進。



 元祐元年二月,始遣使入貢。五月,遣鼎利、罔豫章來賀哲宗即位。六月,復遣訛囉聿來求所侵蘭州、米脂等五砦。使未至,蘇轍兩疏請因其請地而與之。司馬光言:「此邊鄙安危之機,不可不察。靈夏之役,本由我
 起,新開數砦,皆是彼田,今既許其內附,豈宜靳而不與?彼必曰:'新天子即位,我卑辭厚禮以事中國,庶幾歸我侵疆,今猶不許,則是恭順無益,不若以武力取之。'小則上書悖慢,大則攻陷新城。當此之時,不得已而與之,其為國家之恥,無乃甚于今日乎?群臣猶有見小忘大,守近遺遠,惜此無用之地,使兵連不解,為國家之憂。願決聖心,為兆民計。」時異議者衆,唯文彥博與光合,遂從之。秋七月乙丑,秉常殂,時年二十六。在位二十年,改元乾
 道二年,天賜禮盛國慶五年,大安十一年,天安禮定一年。諡曰康靖皇帝,廟號惠宗,墓號獻陵。子乾順立。



 乾順,惠宗之長子也。母曰昭簡文穆皇后梁氏,生三歲即位。元祐元年十月,以父殂,遣使呂則罔聿謨等來告哀。詔自元豐四年用兵所得城砦,待歸我陷執民,當畫以給還。乃遣金部員外郎穆衍充祭奠使,供備庫使張楙充吊慰使。夏遣使進馬、駝來賀興龍節。



 二年正月,遣權樞密院都承旨公事劉奉世為冊禮使,崇儀副使崔
 象先副之,冊乾順為夏國主,仍節度、西平王。三月,夏遣大使映吳嵬名諭密、副使廣樂毛示聿等詣太皇太后進駝、馬以謝奠慰。七月,夏人攻鎮戎軍諸堡,劉昌祚等禦之而退。



 三年三月,攻德靖砦,諸將米贇、郝普戰死。詔劉昌祚以涇原萬人駐德順軍,熙河五千人駐通遠軍,據秦鳳要害,以為犄角。夏人遂攻龕穀砦,砦兵及東關堡巡檢等戰不利,死者幾百人。



 四年二月,始遣使謝封冊。六月,稍歸永樂所獲人,遂以葭蘆、米脂、浮圖、安疆四
 砦與之,而畫界未定。遣崇儀使董正叟、如京使李玩押賜夏國生日禮物及冬服。七月坤成節、十二月興龍節皆遣使來賀。



 五年六月,夏人來言,畫疆界者不依綏州內十里築堡鋪供耕牧、外十里立封堠作空地例,以辨兩國界。詔曰:「已諭邊臣如約,夏之封界當亦體此。」冬,攻蘭州之質孤、勝如堡,既而遣使來賀正旦。六年七月,遣使來賀坤成節。九月,圍麟、府三日,殺掠不計,鄜延都監李儀等盡沒。



 七年,屢攻綏德城,以重兵壓涇原境。留五
 旬,大掠,築壘于沒煙峽口以自固。游師雄請自蘭州李諾平東抵通遠定西、通渭之間,建汝遮、納迷、結珠龍三砦及置護耕七堡,以固藩籬;穆衍請于質孤、勝如二堡之間,城李諾平以控要害。議未決,秦鳳都監康謂以為:「夏之所以未臣附而屢肆兵者,以我勢分于堤備,兵未練而賞罰失當耳。若擇銳結伍,伺彼之動,聚則先擊,散則復襲,則彼分而我聚,以衆擊寡,可得志也。」詔謂詣闕,而下其事于諸道。



 八年四月,復遣使以蘭州一境易塞
 門二砦,詔數其違順不常而卻其請。



 紹聖元年二月,夏進馬助太皇太后山陵。復遣使再議易地,詔不允。



 三年九月,大入鄜延,西自順寧、招安砦,東自黑水、安定,中自塞門、龍安、金明以南,二百里間相繼不絕,至延州北五里。十月,忽自長城一日馳至金明,列營環城,國主子母親督桴鼓,縱騎四掠。知麟州有備,復還金明,而後騎之精銳者留龍安。邊將悉兵掩擊不退,金明乃破。守兵二千八百人惟五人得脫,城中糧五萬石、草千萬束皆盡,
 將官皇城使張俞死之。既還,留一書置漢人頸上,曰:「貸汝命,為我投于經略使處。」其言曰:「夏國昨與朝廷議疆場,惟有小不同,方行理究,不意朝廷改悔,卻于坐團鋪處立界。本國以恭順之故,亦黽勉聽從,遂于境內立數堡以護耕,而鄜延出兵,悉行平蕩,又數數入界殺掠。國人共憤,欲取延州,終以恭順,止取金明一砦,以示兵鋒,亦不失臣子之節也。」延帥呂惠卿上于樞密院而不以聞。初,哲宗聞夏人來寇,泰然笑曰:「五十萬衆深入吾境,
 不過十日,勝不過一二砦須去。」已而果破金明引退。



 四年正月,涇原都鈐轄王文振率諸將破沒煙峽新砦,斬獲三千餘級。二月,夏復以七萬衆攻綏德,鄜延將兵戰退之。



 元符元年十二月,涇原折可適掩夏西壽統軍嵬名阿埋、監軍妹勒都逋,獲之。彗星見,乾順赦國中。



 二年正月,國母梁氏薨,遼遣使蕭德崇來為夏人議和。乃復書謂:若果出至誠,深悔謝罪,當徐度所宜,開以自新之路。五月,夏蘭會正鈐轄革瓦孃以部落來降,授內殿崇
 班,賜銀、絹、緡錢各三百。七月,環州種樸徼赤羊川,獲賞囉訛乞家屬百五十余口,孳畜五千。夏人千餘騎來追,戰卻之,擒監軍訛勃囉及首領淚丁訛遇。詔令赴闕,存恤訛乞家屬,又遣人持其家信號往招之。九月,夏人來告國母哀,因上表謝過。詔夏主:「省所上表,能抗章引慝,已諭邊臣,我疆彼界,毋相侵犯。」已而夏以二千騎出浮圖岔來戰,供奉官陳告、差使李戭死之。閏九月,古邈川部族叛,熙河將王湣率兵掩擊。翌日,夏人馬數萬圍湣
 等,力戰敗之,擒其鈐轄嵬名乞遇;統制苗履又戰于青唐峗,夏人敗績。十二月,遂遣令能、嵬名濟等進誓表曰:「臣國久不幸,時多遇凶,兩經母党之擅權,累為奸臣之竊命。頻生邊患,增怒上心,釁端既深,理訴難達。幸凶黨伏誅,稚躬反正。遐馳懇奏,陳前咎之所歸;乞紹先盟,果淵衷之俯納。故班詔而申諭,獲貢誓以輸誠,謹當飭疆吏而永絕爭端,戒國人而常遵聖化,違約則凶咎再降,背盟則基緒非延。約束事條,恭依處分。」詔報曰:「爾以凶
 黨造謀,數幹邊吏,而能悔過請命,祈紹先盟。念彼種人,均吾赤子,措之安靜,乃副朕心。嘉爾自新,俯從厥志,爾無爽約,朕不食言。自今已往,歲賜仍舊。」



 三年正月,哲宗崩,徽宗即位。九月,夏遣使來奠慰及賀即位。十月,復遣使來賀天寧節。



 建中靖國元年,乾順始建國學,設弟子員三百,立養賢務以廩食之。



 崇甯三年,蔡京秉政,使熙河王厚招夏國卓羅右廂監軍仁多保忠,厚云:「保忠雖有歸意,而下無附者。」章數上,不聽。京愈責厚急,乃遣弟
 詣保忠許,還為夏之邏者所獲,遂追保忠赴牙帳。厚以保忠縱不為所殺,亦不能復領軍政,使得之,一匹夫耳,何益于事。京怒,必令金帛招致之。夏乃點兵,延、渭、慶三路各數千騎出沒,聲言假兵于遼矣。三年,遼以成安公主嫁乾順。



 四年,詔西邊能招致者,毋問首從,賞同斬級令,用京計也。陶節夫在延州,大加招誘,乾順遣使巽請,皆拒之,又令殺其牧放者。夏人遂入鎮戎,略數萬口,執知鄯州高永年而去,又攻湟州,自是兵連者三年。大觀
 元年,始遣人修貢。



 政和四年冬,環州定遠大首領夏人李訛𠼪以書遺其國統軍梁哆㖫曰:「我居漢二十年,每見春廩既虛,秋庾未積,糧草轉輸,例給空券,方春未秋,士有饑色。若卷甲而趨,徑搗定遠,唾手可取,定遠既得,則旁十餘城不攻而下矣。我儲谷累歲,闕地而藏之,所在如是,大兵之來,鬥糧無齎,可坐而飽也。」哆㖫遂以萬人來迎。轉運使任諒先知其謀,募民盡發窖穀,哆㖫圍定邊,失所藏。越七日,訛𠼪遂以其部萬余歸夏。乾順築
 臧底河城,遂詔河東節度使童貫為陝西經略以討之。



 五年春,遣熙河經略劉法將步騎十五萬出湟州,秦鳳經略劉仲武將兵五萬出會州,貫以中軍駐蘭州,為兩路聲援。仲武至清水河,築城屯守而還。法與夏人右廂軍戰于古骨龍,大敗之,斬首三千級。貫奏凱,皆遷秩。秋,仲武、王厚復合涇原、鄜延、環慶、秦鳳之師攻夏臧底河城,敗績,死者十四五,秦鳳第三將全軍萬人皆沒。厚懼,厚賂貫而匿之。冬,夏人以數萬騎略蕭關而去。



 六年春,
 劉法、劉仲武合熙、秦之師十萬攻夏仁多泉城,三日不克,援後期不至,城中請降,法受其降而屠之,獲首三千級。種師道以十萬衆復攻臧底河城,克之。十一月,夏人大舉攻涇原靖夏城。時久無雪,夏先使數萬騎繞城,踐塵漲天,兵對不睹,乃潛穿壕為地道入城中,城遂陷,復屠之而去。



 宣和元年,童貫復逼劉法使取朔方。法不得已,引兵二萬出,至統安城,遇夏國主弟察哥郎君率步騎為三陣,以當法前軍,而別遣精騎登山出其後。大戰
 移七時,前軍楊惟忠敗入中軍,後軍焦安節敗入左軍,朱定國力戰,自朝及暮,兵不食而馬亦渴死多。法乘夜遁,比明,走七十里,至盍朱峗,守兵見,追之,墜崖折足,為一別瞻軍斬首而去。是役死者十萬,貫隱其敗而以捷聞。察哥見法首,惻然語其下曰:「劉將軍前敗我于古骨龍、仁多泉,吾常避其鋒,謂天生神將,豈料今為一小卒梟首哉!其失在恃勝輕出,不可不戒。」遂乘勝圍震武,劉仲武、何灌等赴之,乃解去。震武在山峽中,熙、秦兩路不
 能餉,自築三歲間,知軍李明、孟清皆為夏人所殺。初,夏人陷法軍,圍震武,欲拔之。察哥曰:「勿破此城,留作南朝病塊。」乃自引去。而宣撫司受解圍之賞者數百人,實自去之也。諸路所築城砦皆不毛,夏所不爭之地,而關輔為之蕭條,果如察哥之言。十月,夏遣使來賀天寧節,投以誓詔,不取。貫不能屈,但迫館伴強之,使持還,及邊,遂棄之而去。賈炎得而上之,貫始大沮。



 欽宗即位,遣使來賀正旦。先是,金人滅遼,粘罕遣撒拇使
 夏國,許割天德、雲內、金肅、河清四軍及武州等八館之地,約攻麟州,以牽河東之勢。靖康元年三月,夏人遂由金肅、河清渡河取天德、雲內、武州、河東八館之地。四月,陷震威城,兵馬監押朱昭死之。繼而金貴人兀室以數萬騎陽為出獵,掩至天德,逼逐夏人,悉奪有其地。夏人請和,金人執其使。



 歲丁未,乾順改元正德,時建炎元年也。是歲九月,金帥兀術回雲中,遣保靜軍節度使楊天吉約侵宋,乾順許之。十月,通問使傅雱見金左監軍希尹于雲中,希尹
 以國書授雱,為夏國請熙寧以來侵地。蓋彼既奪其地,乃責償于宋以報之。



 二年正月,以主客員外郎謝亮為陝西撫諭使兼宣諭使,從事郎何洋為太學博士,持詔書賜乾順。亮西入關,鄜延經略使王庶遺亮書曰:「大夫出疆,有可以安社稷、利國家者,專之可也。夏國為患小而緩,金人為患大而急。方其挫銳熙河,奔北鄜延,秋稼未登,兵士困餓。閣下苟能仗節督諸路協同義舉,雖未足盡雪舊恥,亦可驅逐渡河,全秦奠枕,徐圖恢復矣。」亮
 不能用,遂由環慶入西夏。慶曆後,夏國主嘗以賓禮見使者,亮至,乾順乃倨然見之,留居幾月,始與約和罷兵。亮歸,而夏之兵已躡其後,襲取定邊軍。



 明年,亮還行在。二月,金帥婁宿連陷長安、鳳翔,隴右大震。夏人諜知關陝無備,遂檄延安府言:「大金割鄜延以隸本國,須當理索,敢違拒者,發兵誅討之。」帥臣王庶檄報曰:「金人初犯本朝,嘗以金肅、河清畀爾,今誰與守?國家以奸臣貪得,不恤鄰好,遂至于此。貪利之臣,何國無之,豈意夏國
 躬蹈覆轍!比聞金人欲自涇原徑搗興、靈,方切寒心,不圖尚欲乘人之急。幕府雖士卒單寡,然類皆節制之師,左支右吾,尚堪一戰。果能辦此,何用多言。」因遣諜間其用事臣李遇,夏人竟不出。是歲,開封尹宗澤奏疏請北伐,且言乞遣辯士西說夏國,東說高麗,俾出助兵。



 三年,知樞密院事張浚使川、陝,謀北伐,欲通夏國為援,奏請國書,詔從之。七月,浚西行,復以主客員外郎謝亮假太常卿,權宣撫處置司參議官,再使夏國。



 四年正月,浚遣亮
 往,迄不得其要領而還。十月,環慶路統制慕洧叛,降于夏國。



 紹興元年二月,同州觀察副使劉惟輔棄德順軍輸款于夏,夏人拒不受。八月,詔以夏本敵國,毋復班曆日。十一月,川、陝宣撫副使吳玠始遣人通夏國書。



 二年九月,呂頤浩言:「聞金、夏交惡,夏國屢遣人來吳玠、關師古軍中,宜令張浚通問,以撢其情。」是歲,餘睹謀結燕雲之人圖女直,粘罕覺,欲誅之,余睹父子遁入夏國,夏人以其兵少不納。四年十二月,吳玠奏夏國數通書,有不
 忘本朝意。五年,乾順改元大德。



 七年正月,吳璘奏西蕃三十八族首領趙繼忠來歸,用可扼西夏右臂。十月,偽齊知同州李世輔謀執金帥撒里曷歸宋,不克,遂奔夏。世輔父母親族在延安者,金人殺之無遺類。



 九年,夏人陷府州。靈芝生于後堂高守忠家,乾順作《靈芝歌》,俾中書相王仁宗和之。乾順以世輔為靜難軍承宣使、鄜延岐雍等路經略安撫使。世輔請兵,將報延安之役,夏主俾先討別種酋豪號「青面夜叉」者,世輔擒之以報。乾順
 乃為出兵,遣文臣王樞、武臣𠼪訛等隨之。世輔軍至延安,撒里曷走耀州,世輔購得害其父母者,殺之東城。聞金人降赦,歸宋河南地,乃說王樞等降宋。𠼪訛不從,世輔抽刀斫之,不中;遂縛樞,命王晞韓護送行在。五月丙午,世輔以其衆三千人歸宋,授世輔護國承宣使、樞密行府前軍都統制,賜名顯忠。



 六月四日,乾順殂,年五十七。在位五十四年,改元天儀治平四年,天祐民安八年,永安三年,貞觀十三年,雍寧五年,元德八年,正德八年,
 大德五年。諡曰聖文皇帝,廟號崇宗,墓號顯陵。子仁孝嗣。



 仁孝,崇宗長子也。紹興九年六月,崇宗殂,即位,時年十六。十月,詔還王樞及夏國之俘百九十人。十一月,仁孝尊其母曹氏為國母。十二月,納後罔氏。



 十年,夏改元大慶。三月,詔胡世將與夏人議入貢,夏人不報。



 十一年六月,夏樞密使慕洧弟慕濬謀反,伏誅。仁孝上尊號曰制義去邪。十一年九月,夏國饑。



 十三年三月,地震,逾月不止;地裂,泉湧出黑沙。歲大饑,乃立井裏以分振
 之。十三年,夏改元人慶。始建學校于國中,立小學于禁中,親為訓導。



 十四年,彗星見坤宮,五十餘日而滅,占其分在夏國。



 十五年八月,夏重大漢太學,親釋奠,弟子員賜予有差。



 十六年,尊孔子為文宣帝。



 十七年,改元天盛。策舉人,始立唱名法。



 十八年,復建內學,選名儒主之。增修律成,賜名《鼎新》。



 二十八年,始立通濟監鑄錢。



 二十九年,歸宋官李宗閏上書言:「夏國副使屈移,嘗兩使南朝,以為衣冠禮樂非他國比。怨金人叛盟,奪其所與地。此
 其情可見。壬子歲,粘罕嘗聚兵雲中以窺蜀,夏人謂將圖己,舉國屯境上以待其至。今誠遣辯士往說之,夏國必不難出兵,庶足為吾聲援,以圖恢復。」書奏,不報。



 三十年,夏封其相任得敬為楚王。



 三十一年,立翰林學士院,以焦景顏、王僉等為學士,俾修實錄。金主亮犯四川,宣撫使吳璘檄西夏,俾合兵討之。



 三十二年,夏國移置中書、樞密于內門外。大禁奢侈。始封制蕃字師野利仁榮為廣惠王。夏人聞金人南侵,以騎兵二千至蔡園川及
 馬家才、禿頭嶺,將分道入攻,宣撫使吳璘命鎮戎軍守將秦弼說諭之。金兵敗,夏人乃還。



 乾道三年五月,夏國相任得敬遣間使至四川宣撫司,約共攻西蕃,虞允文報以蠟書。七月,得敬間使再至宣撫司,夏人獲其帛書,傳至金人。



 四年,夏改元乾祐。得敬以謀篡伏誅。淳熙十二年二月,諜報故遼國大石牙林假道于夏以伐金,密詔利西都統制吳挺與制置使留正議之。



 十三年四月,復詔挺結夏國。當時論議可否及夏人從違,史皆失書。



 紹熙四年九月二十日,仁孝殂,年七十。在位五十五年,改元大慶四年,人慶五年,天盛二十一年,乾祐二十四年。諡曰聖德皇帝,廟號仁宗,陵號壽陵。子純祐嗣。



 純祐,仁宗長子也,母曰章獻欽慈皇后羅氏。仁宗殂,即位,時年十七。明年改元天慶。



 開禧二年正月二十日廢,遂殂,年三十。在位十四年,諡曰昭簡皇帝,廟號桓宗,陵號莊陵。鎮夷郡王安全立。



 安全,崇宗之孫,越王仁友之子。開禧二年正月,廢其主純祐自立,明年改元應天。



 嘉
 定四年八月五日,安全殂,年四十二。在位六年,改元應天四年,皇建二年。諡曰敬穆皇帝,廟號襄宗,陵號康陵。有子曰承禎。齊國忠武王彥宗之子大都督府主遵頊立。



 遵頊,始以宗室策試進士及第,為大都督府主。嘉定四年七月三日立,時年四十九,改元光定。金衛紹王崇慶元年三月遣使冊為夏國王。



 七年夏,左樞密使萬慶義勇遣二僧齎蠟書來西邊,欲與共圖金人,復侵地,制置使黃誼不報。



 其後金人南遷,議徙都長安,遣元
 帥赤盞以重兵宿鞏州。夏主畏其侵迫,乃遣樞密使都招討甯子寧、忠翼赴蜀閫議夾攻秦、鞏;聶子述俾利西安撫丁焴答書,飭將吏嚴兵以待。時嘉定十二年三月也。子述尋罷去,焴持議不可輕動,師不可出。十二月,甯子寧遣使復申前說,且責我以失期,時安丙再開宣閫,許之,命利州副都統制程信任其責。



 十三年八月,甯子甯以師期來告,丙遂決意出師,以奏劄聞諸朝,不待報可,命將大舉,卒無功。夏人甯子寧、嵬名公輔亦率其衆
 歸國。



 十四年正月,丙回利州。



 十六年,遵頊自號上皇,傳位于其子德旺。



 寶慶二年春,遵頊殂,年六十四。改元光定十三年。諡曰英文皇帝,廟號神宗。



 丙戌七月,德旺殂,年四十六。改元乾定四年。廟號獻宗。



 清平郡王之子南平王睍立,二年丁亥秋,為大元所取,國遂亡。



 夏之境土,方二萬餘里,其設官之制,多與宋同。朝賀之儀,雜用唐、宋,而樂之器與曲則唐也。



 河之內外,州郡凡二十有二。河南之州九:曰靈、曰洪、曰宥、曰銀、曰夏、曰石、
 曰鹽、曰南威、曰會。河西之州九:曰興、曰定、曰懷、曰永、曰涼、曰甘、曰肅、曰瓜、曰沙。熙、秦河外之州四:曰西寧、曰樂、曰廓、曰積石。其地饒五穀,尤宜稻麥。甘、涼之間,則以諸河為溉,興、靈則有古渠曰唐來,曰漢源,皆支引黃河。故灌溉之利,歲無旱澇之虞。



 其民一家號一帳,男年登十五為丁,率二丁取正軍一人。每負贍一人為一抄。負贍者,隨軍雜役也。四丁為兩抄,余號空丁。願隸正軍者,得射他丁為負贍,無則許射正軍之疲弱者為之。故壯者
 皆習戰鬥,而得正軍為多。凡正軍給長生馬、駝各一。團練使以上,帳一、弓一、箭五百、馬一、橐駝五,旗、鼓、槍、劍、棍棓、炒袋、披氈、渾脫、背索、鍬䦆、斤斧、箭牌、鐵爪籬各一。刺史以下,無帳無旗鼓,人各橐駝一、箭三百、幕梁一。兵三人同一幕梁。幕梁,織毛為幕,而以木架。有炮手二百人號「潑喜」,陟立旋風炮于橐駝鞍,縱石如拳。得漢人勇者為前軍,號「撞令郎」。若脆怯無他伎者,遷河外耕作,或以守肅州。



 有左右廂十二監軍司:曰左廂神勇、曰石州祥
 祐、曰宥州嘉甯、曰韋州靜塞、曰西壽保泰、曰卓囉和南、曰右廂朝順、曰甘州甘肅、曰瓜州西平、曰黑水鎮燕、曰白馬強鎮、曰黑山威福。諸軍兵總計五十余萬。別有擒生十萬。興、靈之兵,精練者又二萬五千。別副以兵七萬為資贍,號禦圍內六班,分三番以宿衛。每有事于西,則自東點集而西;于東,則自西點集而東;中路則東西皆集。用兵多立虛砦,設伏兵包敵,以鐵騎為前軍,乘善馬,重甲,刺斫不入,用鉤索絞聯,雖死馬上不墜。遇戰則先
 出鐵騎突陣,陣亂則衝擊之,步兵挾騎以進。戰則大將居後,或據高險。其人能寒暑饑渴。出戰率用只日,避晦日,齎糧不過一旬。弓,皮弦;矢,沙柳竿。惡雨雪。晝舉煙揚塵,夜篝火以為候。不恥奔遁,敗三日,輒復至其處,捉人馬射之,號曰「殺鬼招魂」,或縛草人埋于地,衆射而還。



 篤信機鬼,尚詛祝,每出兵則先卜。卜有四:一、以艾灼羊脾骨以求兆,名「炙勃焦」;二、擗竹于地,若揲蓍以求數,謂之「擗算」;三、夜以羊焚香祝之,又焚穀
 火布靜處,晨屠羊,視其腸胃通則兵無阻,心有血則不利;四、以矢擊弓弦,審其聲,知敵至之期與兵交之勝負,及六畜之災祥、五穀之凶稔。俗皆土屋,惟有命者得以瓦覆之。



 論曰:拓跋氏考諸前史可見也。自赤辭納款于貞觀,立功于天寶,思恭以宥州著節于咸通,夏雖未稱國,而王其土久矣。子孫曆王五代。宋興,太祖即西平王加彝興太尉,德明在祥符間已追帝其父于國中。逮元昊始顯稱帝,厥後因之,與金同亡。



 概其曆世二百五十八年,雖
 嘗受封冊于宋,宋亦稱有歲幣之賜、誓詔之答,要皆出于一時之言,其心未嘗有臣順之實也。元昊結髮用兵,凡二十年,無能折其強者。乾順建國學,設弟子員三百,立養賢務;仁孝增至三千,尊孔子為帝,設科取士,又置宮學,自為訓導。觀其陳經立紀,《傳》曰:「不有君子,其能國乎?」今史所載追尊諡號、廟號、陵名,兼采《夏國樞要》等書,其與舊史有所抵牾,則闕疑以俟知者焉。



\end{pinyinscope}