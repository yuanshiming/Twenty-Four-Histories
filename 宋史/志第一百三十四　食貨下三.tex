\article{志第一百三十四 食貨下三}

\begin{pinyinscope}

 會子鹽上



 會子、交子之法,蓋有取於唐之飛錢。真宗時,張詠鎮蜀,患蜀人鐵錢重,不便貿易,設質劑之法,一交一緡,以三年為一界而換之。六十五年為二十二界,謂之交子,富
 民十六戶主之。後富民貲稍衰,不能償所負,爭訟不息。轉運使薛田、張若穀請置益州交子務,以榷其出入,私造者禁之。仁宗從其議。界以百二十五萬六千三百四十緡為額。



 神宗熙寧初,立偽造罪賞如官印文書法。河東運鐵錢勞費,公私苦之。二年,乃詔置交子務於潞州。轉運司以其法行則鹽、礬不售,有害入中糧草,遂奏罷之。四年,復行於陜西,而罷永興軍鹽鈔場,文彥博言其不便;會張景憲出使延州還,亦謂可行於蜀不可行於
 陜西,未幾竟罷。五年,交子二十二界將易,而後界給用已多,詔更造二十五界者百二十五萬,以償二十三界之數,交子有兩界自此始。時交子給多而錢不足,致價太賤,既而竟無實錢,法不可行。而措置熙河財利孫迥言:「商人買販,牟利於官,且損鈔價。」於是罷陜西交子法。



 紹聖以後,界率增造,以給陜西沿邊糴買及募兵之用,少者數十萬緡,多者或至數百萬緡;而成都之用,又請印造,故每歲書放亦無定數。



 崇寧三年,置京西北路專
 切管幹通行交子所,效川峽路立偽造法。通情轉用並鄰人不告者,皆罪之;私造交子紙者,罪以徒配。四年,令諸路更用錢引,準新樣印制,四川如舊法。罷在京並永興軍交子務,在京官吏,並歸買鈔所。時錢引通行諸路,惟閩、浙、湖、廣不行,趙挺之以為閩乃蔡京鄉里,故得免焉。明年,尚書省言:「錢引本以代鹽鈔,而諸路行之不通,欲權罷印制。在官者,如舊法更印解鹽鈔;民間者,許貿易,漸赴買鈔所如鈔法分數計給。」從之。



 大觀元年,詔
 改四川交子務為錢引務。自用兵取湟、廓、西寧,藉其法以助邊費,較天聖一界逾二十倍,而價愈損。及更界年,新交子一當舊者四,故更張之。以四十三界引準書放數,仍用舊印行之,使人不疑擾,自後並更為錢引。二年,而陜西、河東皆以舊錢引入成都換易,故四川有壅遏之弊,河、陜有道途之艱,豪家因得以損直斂取。乃詔永興軍更置務納換陜西、河東引,仍遣文臣二人監之。八月,知威州張持奏:「本路引一千者今僅直十之一,若出
 入無弊,可直八百,流通用之,官吏奉舊並用引,請稍給錢便用。」擢持為成都路轉運判官,提舉川引。後引價益賤,不可用,持復別用印押以給官吏,他無印押者皆棄無用。言者論其非法,持坐遠謫。三年,詔錢引四十一界至四十二界毋收易,自後止如天聖額書放,銅錢地內勿用。四年,假四川提舉諸司封樁錢五十萬緡為成都務本,侵移者準常平法。



 政和元年,戶部言成都漕司奏:「昨令輸官之引,以十分為率,三分用民戶所有,而七分
 赴官場買納,由是人以七分為疑。請自今無計以三七分之數,並許通用,願買納者聽。民間舊以本錢未至,引價大損,故州官官錢亦減數收市;今本錢已足,請勿減數以祛民惑。又請四十三界引俟界滿勿換給,自四十四界為改法之首。」而戶部詳度欲止行四十四界,其四十五界勿印。若通行及乏用,聽於界內續增其新引給換之,餘如舊鬻之,或於給錢之所易錢儲以為本,移用者如擅支封樁錢法。詔可。靖康元年,令川引並如舊即
 成都府務納換。以置務成都,便利歲久,至諸州則有料次交雜之弊,故有是詔。



 大凡舊歲造一界,備本錢三十六萬緡,新舊相因。大觀中,不蓄本錢而增造無藝,至引一緡當錢十數。及張商英秉政,奉詔復循舊法。宣和中,商英錄奏當時所行,以為自舊法之用,至今引價復平。



 高宗紹興元年,有司因婺州屯兵,請樁辦合用錢,而路不通舟,錢重難致。乃造關子付婺州,召商人入中,執關於榷貨務請錢,願得茶、鹽、香貨鈔引者聽。於是州縣以
 關子充糴本,未免抑配,而榷貨務又止以日輸三分之一償之,人皆嗟怨。六年,詔置行在交子務。臣僚言:「朝廷措置見錢關子,有司浸失本意,改為交子。官無本錢,民何以信?」於是罷交子務,令榷貨務儲見錢印造關子。二十九年,印公據、關子,付三路總領所:淮西、湖廣關子各八十萬緡,淮東公據四十萬緡,皆自十千至百千,凡五等。內關子作三年行使,公據二年,許錢銀中半入納。



 三十年,戶部侍郎錢端禮被旨造會子,儲見錢,於城內外
 流轉;其合發官錢,並許兌會子輸左藏庫。明年,詔會子務隸都茶場。三十二年,定偽造會子法。



 犯人處斬。賞錢十貫,不願受者,補進義校尉。若徒中及庇匿者能告首,免罪受賞,願補官者聽。



 當時會紙取於徽、池,續造於成都,又造於臨安。會子初行,止於兩浙,後通行於淮、浙、湖北、京西。除亭戶鹽本用錢,其路不通舟處上供等錢,許盡輸會子;其沿流州軍,錢、會中半;民間典賣田宅、馬牛、舟車等如之,全用會子者聽。



 孝宗隆興元年,詔會子以「隆興尚書戶部官印會子之印」為文,更造五百
 文會,又造二百、三百文會。置江州會子務。乾道二年,以會子之弊,出內庫及南庫銀一百萬收之。二年,以民間會子破損,別造五百萬換給。又詔損會貫百錢數可驗者,並作上供錢入輸,巨室以低價收者坐之。四年,以取到舊會毀抹付會子局重造,三年立為一界,界以一千萬貫為額,隨界造新換舊。以戶部尚書曾懷同共措置,鑄「提領措置會子庫」印。每道收靡費錢二十足,零百半之。凡舊會破損,貫百字存、印文可驗者,即與兌換。五年,
 令行在榷貨務、都茶場將請算茶、鹽、香、礬鈔引,權許收換第一界,自後每界收換如之。其州縣諸色綱錢,以七分收錢,三分收會。九年,定捕造偽會之賞。



 淳熙元年,詔左藏南上庫給會子二十五萬,收買臨安、平江、紹興、明秀州額外浮鹽,其繼到鈔錢,令榷貨務月終輸封樁庫,以備循環換易會子。三年,詔第三界、四界各展限三年,令都茶場會子庫以第四界續印會子二百萬貯南庫。當時戶部歲入一千二百萬,其半為會子,而南庫以金
 銀換收者四百萬,流行於界外者才二百萬耳。光宗紹熙元年,詔第七、第八界會子各展三年。臣僚言:「會子界以三年為限,今展至再,則為九年,何以示信?」於是詔造第十界立定年限。



 慶元元年,詔會子界以三千萬為額。嘉定二年,以三界會子數多,稱提無策,會十一界除已收換,尚有一千三百六十萬餘貫,十二界、十三界除燒毀尚有一萬二百餘萬貫。



 十二界四千七百萬餘貫,十三界五千七百萬餘貫。詔封樁庫撥金一百五萬兩,兩為錢四十貫。



 度牒七千道,每道為錢一千
 貫。



 官告綾紙、乳香,乳香每套一貫六百文。



 湊成三千餘,添貼臨安府官局,收易舊會,品搭入輸。十一界會子二分,十二、十三界會子各四分。



 以舊會之二,易新會之一。泉州守臣宋均、南劍州守臣趙崇亢、陳宓,皆以稱提失職,責降有差。



 紹定五年,兩界會子已及三億二千九百餘萬。端平二年,臣僚言:「兩界會子,遠者曾未數載,近者甫及期年,非有破壞塗污之弊,今當以所收之會付封樁庫貯之,脫有緩急,或可濟事。」有旨從之。淳熙二年,宗正丞韓祥奏:「壞楮幣者只緣變
 更,救楮幣者無如收減。自去年至今,楮價粗定,不至折閱者,不變更之力也。今已罷諸造紙局及諸州科買楮皮,更多方收減,則楮價有可增之理。」上曰:「善。」三年,臣僚言:「今官印之數雖損,而偽造之券愈增;且以十五、十六界會子言之,其所入之數,宜減於所出之數。今收換之際,元額既溢,舉者未已。若非偽造,其何能致多如是?大抵前之二界,盡用川紙,物料既精,工制不茍,民欲為偽,尚或難之。迨十七界之更印,已雜用川、杜之紙,至十八
 界則全用杜紙矣。紙既可以自造,價且五倍於前,故昔之為偽者難,今之為偽者易。人心循利,甚於畏法,況利可立致,而刑未即加者乎?臣愚以為抄撩之際,增添紙料,寬假工程,務極精致,使人不能為偽者,上也;禁捕之法,厚為之勸,厲為之防,使人不敢為偽者,次也。」七年,以十八界與十七界會子更不立限,永遠行使。十一年,以會價增減課其官吏。景定四年,以收買逾限之田,復日增印會子一十五萬貫。



 咸淳四年,以近頒見錢關子,貫
 作七百七十文足,十八界每道作二百五十七文足,三道準關子一貫,同見錢轉使,公私擅減者,官以贓論,吏則配籍。五年,復申嚴關子減落之禁。七年,以行在紙局所造關子紙不精,命四川制使抄造輸送,每歲以二千萬作四綱。



 川引自張浚開宣府,趙開為總餉,以供糴本,以給軍需,增印日多,莫能禁止。七年,川、陜副帥吳玠請置銀會於河池,不許。蓋前宋時,蜀交出放兩界,每界一百二十餘萬。今三界通行,為三千七百八十餘萬,至紹
 興末,積至四千一百四十七萬餘貫;所貯鐵錢,僅及七十萬貫,以鹽酒等陰為稱提。是以餉臣王之望亦謂添印錢引以救目前,不得不為朝廷遠慮。詔添印三百萬,之望止添印一百萬。孝宗隆興二年,餉臣趙沂添印二百萬。淳熙五年,以蜀引增至四千五百餘萬,立額不令再增。光宗紹熙二年,詔川引展界行使。寧宗嘉泰末,兩界出放凡五千三百餘萬緡,通三界出放益多矣。



 開禧末,餉臣陳咸以歲用不足,嘗為小會,卒不能行。嘉定初,
 每緡止直鐵錢四百以下,咸乃出金銀、度牒一千三百萬,收回半界,期以歲終不用。然四川諸州,去總所遠者千數百里,期限已逼,受給之際,吏復為奸。於是商賈不行,民皆嗟怨,一引之直,僅售百錢。制司乃諭人除易一千三百萬引,三界依舊通行,又檄總所取金銀就成都置場收兌,民心稍定。自後引直鐵錢五百有奇,若關外用銅錢,引直五百七十錢而已。



 喜定三年春,制、總司收換九十一界二千九百餘萬緡;其千二百萬緡,以茶馬
 司羨餘錢及制司空名官告,總所樁金銀、度牒對鑿,餘以九十三界錢引收兌;又造九十四界錢引五百萬緡,以收前宣撫程松所增之數;凡民間輸者,每引百貼八千。其金銀品搭,率用新引七分,金銀三分,其金銀品色官稱,不無少虧,每舊引百,貼納二十引。蓋自元年、三年兩收舊引,而引直遂復如故。昔高宗因論四川交子,最善沉該稱提之說,謂官中常有錢百萬緡,如交子價減,官用餞買之,方得無弊。



 九年,四川安撫制置大使司言:「
 川引每界舊例三年一易。自開禧軍興以後,用度不給,展年收兌,遂至兩界、三界通使;然率以三年界滿,方出令展界,以致民聽惶惑。今欲以十年為一界,著為定令,則民旅不復懷疑。」從之。



 寶祐四年臺臣奏:「川引、銀會之弊,皆因自印自用,有出無收。今當拘其印造之權,歸之朝廷,仿十八界會子造四川會子,視淳祐之令,作七百七十陌,於四川州縣公私行使。兩料川引並毀,見在銀會姑存。舊引既清,新會有限,則楮價不損。物價自平,公
 私俱便矣。」有旨從之。咸淳五年,復以會板發下成都運司掌之,從制司抄紙發往運司印造畢功,發回制司,用總所印行使,歲以五百萬為額。



 紹興末,會子未有兩淮、湖廣之分,其後會子太多而本錢不足,遂致有弊。乾道二年,詔別印二百、三百、五百、一貫交子三百萬,止行使於兩淮,其舊會聽對易。凡入輸買賣,並以交子及錢中半。如往來不使,詔給交子、會子各二十萬,付鎮江、建康府榷貨務,使淮人之過江、江南人之渡淮者,皆得對易
 循環以用。然自紹興末年,銅錢禁用於淮而易以鐵錢,會子既用於淮而易以交子,於是商賈不行,淮民以困。右司諫陳良祐言交子不便,詔兩淮郡守、漕臣調其利害,皆謂所降交子數多,而銅錢並會子不過江,是致民旅未便。於是詔銅錢並會子依舊過江行用,民間交子許作見錢輸官,凡官交,盡數輸行在左藏庫。



 三年,詔造新交子一百三十萬,付淮南漕司分給州軍對換行使,不限以年;其運司見儲交子,先付南庫交收。紹熙三年,
 詔新造交子三百萬貫,以二百萬付淮東,一百萬付淮西,每貫準鐵錢七百七十文足,以三年為界。慶元四年,詔兩淮第二界會子限滿,明年六月,更展一界。嘉定十一年,造兩淮交子二百萬,增印三百萬。十三年,印二百萬,增印一百五十萬。十四年、十五年,皆及三百萬。自是其數日增,價亦日損,稱提無術,但屢與展界而已。



 初,襄、郢等處大軍支請,以錢銀品搭。孝宗隆興元年,始措置於大軍庫儲見錢,印造五百並一貫直便會子,發赴軍
 前,並當見錢流轉。印造之權既專,印造之數日益;且總所所給止行於本路,而荊南水陸要沖,商賈必由之地,流通不便。乾道三年,收其會子印板。四年,以淮西總所關子二十萬,都茶場鈔引八十萬,付湖北漕司收換,輸左藏庫,又命降銀錢收之。五年,詔戶部給行在所會子五十萬,付荊南府兌換。淳熙七年,詔會子庫先造會子一百萬,降付湖廣總所收換破會。十一年,臣僚言:「湖北會子創於隆興初,迄今二十二年,不曾兌易,稱提不行。」
 詔湖廣總領同帥、漕議經久利便。帥、漕、總領言:「乞印給一貫、五百例湖北會子二百萬貫,收換舊會,庶幾流轉通快,經久可行。」從之。



 十三年,詔湖廣會子仍以三年為界。紹熙元年,詔湖廣總所將見行及樁貯新舊會取數,仿行在例立界收換。餉臣梁總奏:「自來不曾立界,但破損者即行換易,除累易外,尚有五百四十餘萬,見在民間行用。乞別樣制作兩界,印造收換。」從之。



 嘉定五年,湖廣餉臣王釜,請以度牒、茶引兌第五界舊會,每度牒一
 道,價千五百緡,又貼搭茶引一千五百緡,方許收買,期以一月。然京湖二十一州止置三場,不便。制臣劉光祖乃會總所以第六界新會五萬緡,令軍民以舊楮二而易其一;繼又令軍民以一楮半而易其一;又請於朝添給新楮十萬,軍民賴之。十四年,造湖廣會子三十萬易破會。十七年,造湖廣第六界會子二百萬。嘉熙二年,撥第七界湖廣會九百萬付督視參政行府。寶祐二年,撥第八界湖廣會三百萬貫付湖廣總所,易兩界破會,自
 後因仍行之。



 鹽之類有二:引池而成者,曰顆鹽,《周官》所謂盬鹽也;鬻海、鬻井、鬻堿而成者,曰末鹽,《周官》所謂散鹽也。宋自削平諸國,天下鹽利皆歸縣官。官鬻、通商,隨州郡所宜,然亦變革不常,而尤重私販之禁。



 引池為鹽,曰解州解縣、安邑兩池。墾地為畦,引池水沃之,謂之種鹽,水耗則鹽成。籍民戶為畦夫,官廩給之,復其家。募巡邏之兵百人,目為護寶都。歲二月一日墾畦,四月始種,八月乃止。安
 邑池每歲歲種鹽千席,解池減二十席,以給本州及三京,京東之濟、兗、曹、濮、單、鄆州、廣濟軍,京西之滑、鄭、陳、穎、汝、許、孟州,陜西之河中府、陜虢州、慶成軍,河東之晉、絳、慈、隰州,淮南之宿、亳州,河北之懷州及澶州諸縣之在河南者。凡禁榷之地,官立標識、候望以曉民。其通商之地,京西則蔡、襄、鄧、隨、唐、金、房、均、郢州、光化信陽軍,陜西則京兆鳳翔府、同華、耀、乾、商、涇、原、邠、寧、儀、渭、鄜阜、坊、丹、延、環、慶、秦、隴、鳳、階、成州、保安鎮戎軍,及澶州諸縣之在河
 北者。顆、末鹽皆以五斤為斗,顆鹽之直每斤自四十四至三十四錢,有三等。至道二年,兩池得鹽三十七萬三千五百四十五席,席一百一十六斤半。三年,鬻錢七十二萬八千餘貫。



 咸平中,度支使梁鼎言:「陜西沿邊解鹽請勿通商,官自鬻之。」詔以鼎為陜西制置使,又以內殿崇班杜承睿同制置陜西青白鹽事。承睿言:「鄜、延、環、慶、儀、渭等州洎禁青鹽之後,令商人入芻粟,運解鹽於邊貨鬻,其直與青鹽不至相懸,是以民食賤鹽,須至畏法,
 而蕃部青鹽難售。今聞運解鹽於邊,俗與內地同價,邊民必冒法圖利,卻入蕃界私販青鹽,是助寇資而結民怨矣。」繼又有上疏言其不便者,鼎請候至邊部斡運,及乘傳至解池即禁止商販。旋運鹽赴邊,公私大有煩費,而邊民頓無入市,物論紛擾。於是命判鹽鐵勾院林特、知永興軍張永詳議,以為公私非便,請復舊商販。詔切責鼎,罷度支使。大中祥符九年,陜西轉運使張象中言:「兩池所貯鹽計直二千一百七十六萬一千八十貫,慮
 尚有遺利,望行條約。」真宗曰:「地利之阜,此亦至矣。過求增羨,慮有時而闕。」不許。



 先是,五代時鹽法太峻。建隆二年,始定官鹽闌入法,禁地貿易至十斤、鬻堿鹽至三斤者乃坐死,民所受蠶鹽以入城市三十斤以上者,上請。三年,增闌入至三十斤、鬻堿至十五斤坐死,蠶鹽入城市百斤以上,奏裁。自乾德四年後,每詔優寬。太平興國二年,乃詔闌入至二百斤以上,鬻堿及主吏盜販至百斤以上,蠶鹽入城市五百斤以上,並黥面送闕下。至淳
 化五年,改前所犯者正配本州牢城。代州寶興軍之民私市契丹骨堆渡及桃山鹽,雍熙四年,詔犯者自一斤論罪有差,五十斤加徒流,百斤以上部送闕下。



 天聖以來,兩池畦戶總三百八十,以本州及旁州之民為之,戶歲出夫二人,人給米日二升,歲給戶錢四萬。為鹽歲百五十二萬六千四百二十九石,石五十斤,以席計,為六十五萬五千一百二十席,席百一十六斤。禁榷之地,皆官役鄉戶衙前及民夫,謂之帖頭,水陸漕運。而通商州
 軍並邊秦、延、環、慶、渭、原、保安、鎮戎、德順,又募人入中芻粟,以鹽償之。



 凡通商州軍,在京西者為南鹽,在陜西者為西鹽,若禁鹽地則為東鹽,各有經界,以防侵越。天聖初,計置司議茶鹽利害,因言:「兩池舊募商人售南鹽者,入錢京師榷貨務。乾興元年,歲入才二十三萬緡,視天禧三年數損十四萬。請一切罷之,專令入中並邊芻粟,及為之增約束、申防禁,以絕私販之弊。」久之,復詔入錢京師,從商人所便。



 三京、二十八州軍,官自輦鹽,百姓困
 於轉輸。天聖八年,上書者言:「縣官禁鹽,得利微而為害博,兩池積鹽為阜,其上生木合抱,數莫可較。宜聽通商,平估以售,可以寬民力。」詔翰林學士盛度、御史中丞王隨議更其制度。因畫通商五利上之曰:「方禁商時,伐木造船輦運,兵民不勝疲勞,今去其弊,一利也;陸運既差帖頭,又役車戶,貧人懼役,連歲逋逃,今悉罷之,二利也;船運有沉溺之患,綱吏侵盜,雜以泥沙硝石,其味苦惡,疾生重膇,今皆得食真鹽,三利也;錢幣,國之貨泉。欲使
 通流,富家多藏鏹不出,民用益蹙,今歲得商人出緡錢六十餘萬助經費,四利也;歲減鹽官、兵卒、畦夫傭作之給,五利也。」十月,詔罷三京、二十八州軍榷法,聽商人入錢若金銀京師榷貨務,受鹽兩池。行之一年,視天聖七年,增緡錢十五萬。其後歲課減耗,命翰林學士宋庠等以天聖九年至寶元二年新法較之,視乾興至天聖八年舊法,歲課損二百三十六萬緡。康定元年,詔京師、南京及京東州軍,淮南宿、亳州,皆禁如舊。未幾,復弛京師
 榷法,並詔三司議通淮南鹽給京東等八州,於是兗、鄆、宿、亳皆食淮南鹽矣。



 自元昊反,聚兵西鄙,並邊入中芻粟者寡。縣官急於兵食,調發不足,因聽入中芻粟,予券趨京師榷貨務受錢若金銀;入中他貨,予券償以池鹽。繇是羽毛、筋角、膠膝、鐵炭、瓦木之類,一切以鹽易之。猾商貪吏,表裏為奸,至入椽木二,估錢千,給鹽一大席,為鹽二百二十斤。虛費池鹽,不可勝計,鹽直益賤,販者不行,公私無費。慶歷二年,復京師榷法,凡商人虛估受券
 及已受鹽未鬻者,皆計直輸虧官錢。內地州軍民間鹽,悉收市入官,官為置場增價出之。復禁永興、同、華、耀、河中、陜、虢、解、晉、絳、慶成十一州軍商鹽,官自輦運,以衙前主之。又禁商鹽私入蜀,置折博務於永興、鳳翔,聽人入錢若蜀貨,易鹽趨蜀中以售。久之,東、南鹽地悉復禁榷,兵民輦運,不勝其苦,州郡騷然。所得鹽利,不足以佐縣官之急。並邊務誘人入中芻粟,皆為虛估,騰踴至數倍,大耗京師錢幣,帑藏益虛。



 太常博士範祥,關中人也,熟
 其利害,常謂兩池之利甚博,而不能少助邊計者,公私侵漁之害也;儻一變法,歲可省度支緡錢數十百萬。乃畫策以獻。是時韓琦為樞密副使,與知制誥田況皆請用祥策。四年,詔祥馳傳與陜西都轉運使程戡議之,而戡議與祥不合,祥尋亦遭喪去。八年,祥復申其說,乃以為陜西提點刑獄兼制置解鹽事,使推行之。其法:舊禁鹽地一切通商,聽鹽入蜀;罷九州軍入中芻粟,令入實錢,償以鹽,視入錢州軍遠近及所指東、西、南鹽,第優其
 直;東、南鹽又聽入錢永興、鳳翔、河中;歲課入錢總為鹽三十七萬五千大席,授以要券,即池驗券,按數而出,盡馳兵民輦運之役。又以延、慶、環、渭、原、保安鎮戎、德順地近烏、白池,奸人私以青白鹽入塞,侵利亂法。乃募人入中池鹽,予券優其估,還,以池鹽償之;以所入鹽官自出鬻,禁人私售,峻青白鹽之禁。並邊舊令入中鐵,炭、瓦、木之類,皆重為法以絕之。其先以虛估受券及已受鹽未鬻者,悉計直使輸虧官錢。又令三京及河中、河陽、陜、虢、
 解、晉、絳、濮、慶成、廣濟官仍鬻鹽,須商賈流通乃止。以所入緡錢市並邊九州軍芻粟,悉貿榷貨務錢幣以實中都。行之數年,黠商貪賈,無所僥幸,關中之民,得安其業,公私便之。



 皇祐元年,侍御史知雜何郯復言改法非是。明年,遣三司戶部副使包拯馳視,還言行之便,第請商人入錢及延、環等八州軍鬻鹽,皆重損其直,即入鹽八州軍者,增直以售,三京及河中等處禁官鬻鹽。而三司謂京師商賈罕至則鹽貴,請得公私並貿,餘禁止。皆聽
 之。田況為三司使,請久任祥,俾專其事。擢祥為陜西轉運使,賜金紫服。祥初言歲入緡錢可得二百三十萬,皇祐初年,入緡錢二百二十一萬;四年,二百一十五萬。以四年數視慶歷六年,增六十八萬;視七年,增二十萬。又舊歲出榷貨務緡錢,慶歷二年,六百四十七萬;六年,四百八十萬。至是,榷貨務錢不復出。其後,歲入雖贏縮不常,至五年,猶及百七十八萬;至和元年,百六十九萬。時祥已坐他罪貶,命轉運使李恭代之。三年,遂以元年入
 錢為歲課定率,量入計出,可助邊費十分之八。



 久之,並邊復聽入芻粟以當實錢,而虛估之弊滋長,券直亦從而賤,歲損官課,無慮百萬。嘉祐三年,三司使張方平及包拯請復用祥,於是復以祥總鹽事。祥請重禁入芻粟者,其券在嘉祐三年已前,每券別請輸錢一千,然後予鹽。又言商人持券若鹽鬻京師,皆虧失本錢。請置官京師,蓄錢二十萬緡,以待商人至者,券若鹽估賤,則官為售之。券紙六千,鹽席十千,毋輒增損,所以平其市估,使
 不得為輕重。詔以都鹽院監官兼領,自是稍復舊。未幾祥卒,以轉運副使薛向繼之。治平二年,歲入百六十七萬。



 初,祥以法既通商,恐失州縣徵算,乃計所歷所至合輸算錢,並率以為入中之數。自後州縣猶算如舊。嘉祐六年,向悉罷之,並奏減八州軍鬻鹽價。兩池畦戶,歲役解、河中、陜、虢、慶成之民,官司旁緣侵剝,民以為苦,乃詔三歲一代。嘗積逋課鹽至三百三十七萬餘席,遂蠲其半。中間以積鹽多,特罷種鹽一歲或二歲三歲,以寬其
 力。後又減畦戶之半,稍以傭夫代之,五州之民始安。



 青白鹽出烏、白兩池,西羌擅其利。自李繼遷叛,禁毋入塞,未幾罷,已而復禁。乾興初,嘗詔河東邊人犯青白鹽禁者如陜西法。慶歷中,元昊納款,請歲入十萬石售縣官。仁宗以其亂法,不許。自範祥議禁八州軍商鹽,重青白鹽禁,而官鹽估貴,土人及蕃部販青白鹽者益眾,往往犯法抵死而莫肯止。至和中,詔蕃部犯青白鹽抵死者,止投海島,群黨為民害者,上請。嘉祐赦書,稍遷配徒者
 於近地,自是禁法稍寬。熙寧初,詔淮南轉運使張靖究陜西鹽、馬得失。靖指向欺隱狀,王安石右向,靖竟得罪,擢向為江、淮等路發運使。諫官范純仁言賞罰失當,因子向五罪,向任如初。乃請即永興軍置賣鹽場,又以邊費錢十萬緡,儲永興軍為鹽鈔本,繼又增二十萬。



 四年,詔陜西行蜀交子法,罷市鈔;或論其不便,復舊。七年,中書議陜西鹽鈔,出多虛鈔,而鹽益輕,以鈔折兌糧草,有虛抬逼糴之患。請用交子法,使其數與見錢相當,可濟
 緩急。詔以皮公弼、熊本、宋迪分領其事,趙瞻制置。又以內藏錢二百萬緡假三司,遣市易吏行四路請買鹽引,仍令秦鳳、永興鹽鈔,歲以百八十萬為額。八年,中書奏陜西鹽鈔利害及立法八事,大抵謂買鈔本錢有限,而出鈔過多,買不盡則鈔賤而糴貴,故出鈔不可無限。然商人欲變易見錢,而官不為買,即為兼並所抑,則鈔價益賤;而邊境有急,鈔未免多出,故當置場以市價平之。今當定買兩路實賣鹽二百二十萬緡,以當用鈔數立
 額,永興路八十一萬五千,秦鳳路一百三十八萬五千,熙河路五十三萬七千;永興軍遣官買鈔,歲支轉運司錢十萬緡買西鹽鈔,又用市易務賒請法募人賒鈔變易,即民間鈔多而滯,則送解池毀之。詔從其請,然有司給鈔溢額,猶視其故。九年,乃詔御史劾陜西官吏,止三司額外出鈔。



 十年,三司言:「鹽法之弊,由熙河鈔溢額,故價賤而芻糧貴。又東、西、南三路通商郡邑榷賣官鹽,故商旅不行。今鹽法當改,官賣當罷。請先收舊鈔,印識之
 舊鹽,行加納之法。官盡買舊鈔,其已出鹽,約期聽商人自言,準新價增之,印鹽席,給符驗。東、南舊法鹽鈔,席才三千五百;西鹽鈔席減一千,官盡買。先令解州場院驗商人鈔書之,乃許賣。已請鹽,立限告賞,聽商人自陳,東、南鹽席加錢二千五百,西鹽席加三千,為易舊符,立期令賣。罷兩處禁榷官賣,提舉司賣鹽並用新價,錢承買舊鈔,商人願對行算請者聽,官為印識如法。應通商地各舉官一員,其鹽席限十日自言,乃令加納錢,為印識,
 給新引,聽以舊鈔當加納錢。」皆行之。而別定官賣鹽地,市易司以買鹽,亦加納錢。



 舊制,河南北曹、濮以西,秦、鳳以東,皆食解鹽。自仁宗時,解鹽通商,官不復榷;熙寧中,市易司始榷開封、曹濮等州。八年,大理寺丞張景溫提舉出賣解鹽,於是開封府界陽武、酸棗、封丘、考城、東明、白馬、中牟、陳留、長垣、胙城、韋城,曹、濮、澶、懷、濟、單、解州、河中府等州縣,皆官自賣。未幾,復用商人議,以唐、鄧、襄、均、房、商、蔡、郢、隨、金、晉、絳、虢、陳、許、汝、穎、隰州、西京、信陽軍通
 商,畿縣及澶、曹、濮、懷、衛、濟、單、解、同、華、陜、河中府、南京、河陽,令提舉解鹽司運鹽貨鬻,仍詔三司講求利害。



 鹽價既增,民不肯買,乃課民買官鹽,隨貧富作業為多少之差。買賣私鹽,聽人告,重給賞,以犯人家財給之。買官鹽食不盡,留經宿者,同私鹽法。於是民間騷怨。鹽鈔舊法每席六緡,至是二緡有餘,商不入粟,邊儲失備。召陜西轉運使皮公弼入議,公弼極言官賣不便。沈括為三司使,不能奪。王安石主景溫,括希安石意,言通商歲失官
 賣緡錢二十餘萬。安石去位,括在三司,乃言官賣當罷。於是河陽、同、華、解州、河中、陜府、陳留、雍丘、襄邑、中牟、管城、尉氏、鄢陵、扶溝、太康、咸平、新鄭聽通商,其入不及官賣者,官復自賣;澶、濮、濟、單、曹、懷州,南京,陽武、酸棗、封丘、考城、東明、白馬、長垣、胙城、韋城九縣,官賣如故。詔商鹽入京,悉賣之市易務,每席毋得減十;民鹽皆買之市易務,私與商人為市,許告,沒其鹽。



 皮公弼鹽法,酌前後兩池所支鹽數,歲以三百三十萬緡為額。又令京師置七
 場,買東、南鹽鈔,市易務計為錢五十九萬三千餘緡;三司闕錢,請頗還其鈔,令賣之於西;買者其三給錢,其七準沿邊鹽價給新引;庶得民間舊鈔,而新引易於變易。詔用其議。公弼請復範祥舊法平市價,詔假三司錢三十萬緡,市鈔於京師。先是,解鹽分東西,西鹽賣有分域;又並邊州軍市芻糧,給鈔過多,故鈔及鹽甚賤,官價自分為二。於是增西鹽價比東鹽,以平鈔法,歲約增十二萬緡,毋復分東西,悉廢西鹽約束。解池鹽鈔舊以二百
 二十萬緡為額,轉運使皮公弼請增十萬,以助邊糴,至是,又為二百四十二萬。商人已請西鹽,令加納錢,使與新法價平。元豐三年,三司舉張景溫賣解鹽息羨,進官賜帛。



 明年,權陜西轉運使李稷言:「自新法未行,鈔之貴賤,視有司出之多寡。新法已後,鈔有定數,起熙寧十年冬,盡元豐三年,通印給一百七十七萬餘席,而鹽池所出才一百一十七萬五千餘席,餘鈔五十九萬有餘,流布官司,其勢不得不賤。」遂下三司住給。五年,戶部猶以
 鈔多難售,歲給陜西軍儲鈔二百萬,裁其半,然鈔多,卒不能平價。



 元祐元年,戶部及制置解鹽司議:「延、慶、渭、原、環、鎮戎、保安、德順等八州軍,皆官自鬻,以萬五千五百席為額,聽商旅入納於八州軍折博務,算給交引,如範祥舊法。鹽價錢應償者,以轉運司年額鹽鈔給之,所鬻鹽錢,以待轉運司糴買。仍舉承務郎以上一員,於在京置場,以鹽鈔鬻見錢而輸之都鹽院庫,遇給解鹽額鈔盡歸之本司,毋更給轉運司。他司皆毋得販易,雖有專
 旨,聽執奏。其已買鈔,自本司拘之,若民間鈔少或給本路緡錢,即上戶部議鬻其鈔。」詔皆從之。既而又以商人入納解鹽減年額買鹽費錢二萬七千餘緡,增在京買鈔之本。入中解鹽,並效熙河鈔,而價隨事增損以折,澶懷滑州、陽武鹽價,定為錢八千二百。時,陜西民多以樸硝私煉成顆,謂之倒硝,頗與解鹽相亂。紹聖三年,制置使孫路以聞,詔犯者減私鹽法一等坐之。



 初,神宗時,官賣解鹽,京西則通商。有沉希顏者為轉運使,更為榷法,
 請假常平錢二十萬緡,自買解鹽,賣之本路,民已買解鹽盡買入官,掊克牟利,商旅苦之。哲宗即位,殿中侍御史黃降劾希顏罪。元祐元年,京西始復舊制通商,然猶官賣,元符元年乃罷之。永興軍渭州河北高陽、櫟陽、涇陽等縣,如同、華等六州軍,官仍自賣鹽,而禁官司於折博務買解鹽販易規利。俄以水壞解池,聽河中府解州小池鹽、同華等州私土鹽、階州石鹽、通遠軍岷州官井鹽鬻於本路,而京東、河北鹽亦通行焉。三年,詔陜西轉運
 副使兼制置解鹽使馬城,提舉措置催促陜西、河東木□伐薛嗣昌,提舉開修解州鹽池。



 崇寧元年,解州賈瓦南北圓池修沼畦眼,拍摩布種,通得鹽百七十八萬二千七百餘斤。初,解梁東有大鹽澤,綿亙百餘里,歲得億萬計。自元符初,霖潦池壞。至是,乃議修復;四年,池成。凡開二千四百餘畦,百官皆賀。內侍王仲千者董其役,以課額敷溢為功。然議者謂解池灌水盈尺,暴以烈日,鼓以南風,須臾成鹽,其利固博;茍欲溢額,不俟風日之便,厚
 灌以水,積水而成,味苦不適口。



 崇寧初,言事者以鈔法屢變,民聽疑惑,公家失輕重之權,商旅困往來之費,乞復範祥舊法,謹守而力行之,無庸輕改。雖可其請,未幾,蔡京建言:「河北、京東末鹽,客運至京及京西,袋輸官錢六千,而鹽本不及一千,施行未久,收息及二百萬緡。如通至陜西,其利必倍。」議遣韓敦立等分路提舉。及鹽池已復,京仍欲舊解鹽地客算東北末鹽,令榷貨務人納見緡無窮,以收己功,乃令解鹽新鈔止行陜西。五年,詔:「
 鈔法用之,民信已久,飛錢裕國,其利甚大,比考前後法度,頗究利害,其別為號驗,給解鹽換請新鈔。先以五百萬緡赴陜西。河東,止給糴買,聽商旅赴榷貨務換請東南鹽鈔。貼輸見緡四分者在舊三分之上,五分者在四分之上。且帶行舊鈔,輸四分者帶五分,輸五分者帶六分;若不願貼輸錢者,依舊鈔價減二分。」先是,患豪商擅利源輕重之柄,率減鈔直,使並邊糴價增高,乃裁限之。崇寧四年,以鈔價雖裁,其入中州郡,復增糴價,客持鈔
 算請,坐牟大利。乃詔陜西舊鈔易東南末鹽,每百緡用見錢三分,舊鈔七分。後又詔減落鈔價逾五十者,論以法。



 及大觀四年,張商英為相,議復通行解鹽如舊法,而東北鹽毋得與解鹽地相亂。繼而有司議解池已復,依舊法印鈔請。商旅已買東北鹽,隨處官司期三日盡籍,輸官償其價,隱匿者如私鹽法。解鹽未到,官鬻所得東北鹽,解鹽到即止。已請鈔已支者悉毀,已支未請者聽別議。在京仍通行,其經由州縣鄭州、中牟、開封府祥符、
 陽武縣境內,亦許通放。而王仲千所請通入京西北路陳、穎、蔡州、信陽軍,權止之。商旅已算請東北鹽,元指定東京,未至者,止今所至州軍批引;其已入京未貨者,都鹽院全袋拘買鬻之,許坐賈請買碎賣。



 政和元年,詔陜西鈔依鈔面實價,輒增減者,以違制論。未幾,復以陜西通行鹽鈔,舊雖約以銅錢六千為鈔面,然鈔貴則入粟增多,鈔平則入穀減少。若限以六千,陜西唯行鐵錢,是鹽鈔一席得六千鐵錢觔斗矣,深損公家,其隨時增減
 聽之。二年,蔡京復用事,法仍變改,鈔不可用者悉同敗楮。六年,兩池漫生鹽,募人倍力採取,且議加賞;繼生紅鹽,百官皆賀,制置解鹽使李百祿等第賞有差。七年,議復行解鹽,時童貫宣撫關、河,實主之。詔解鹽地見行東北鹽,復盡收入官,官給其直,在京於平貨、在外於市易務樁管,如解鹽法鬻之;不自陳,如私鹽法。重和元年,詔復行解鹽舊法。逾年,榷貨歲虧數百萬貫,又鈔價減落,糴買不行,三省趣講畫以聞,貫遂請罷領解鹽。俄
 而三省條奏:舊東北鹽地客販解鹽,立限盡鬻,限竟鬻未盡者,運往解鹽地,逾者論如私鹽法。京畿、京西復置官提舉。初,崇寧中,以鹽各利一方,故解鹽止行本路,東南鬻海利博,行於數路。既復行解鹽,商旅苦於折閱;即改如舊,慮商旅疑惑。遂詔輸諸路,鈔法更不改易,扇搖者論如法,仍倍之。



 靖康元年,解鹽鈔入納算請,並參照熙寧、元豐以前舊法,又增改解鹽及東北鹽地,即商旅不願鹽,則用鈔面請錢如舊法。繼定每席鈔為八貫者,盡收
 入鈔面;其入納糧草者,許直赴池請鹽,省復入京批鈔之擾。



 鬻海為鹽,曰京東、河北、兩浙、淮南、福建、廣南,凡六路。其鬻鹽之地曰亭場,民曰亭戶,或謂之灶戶。戶有鹽丁,歲課入官,受錢或折租賦,皆無常數,兩浙又役軍士定課鬻焉。諸路鹽場廢置,皆視其利之厚薄,價之贏縮,亦未嘗有一定之制。末鹽之直,斤至自四十七至八錢,有二十一等。至道三年,鬻錢總一百六十三萬三千餘貫。



 其在京東曰密州濤洛場,一歲鬻三萬二千餘石,以
 給本州及沂、濰州,唯登、萊州則通商,後增登州四場。舊南京及曹、濮、濟、兗、單、鄆、廣濟七州軍食池鹽,餘皆食二州鹽,官自鬻之。慶歷元年冬,以淄、濰、青、齊、沂、密、徐、淮陽八州軍仍歲兇菑,乃詔弛禁,聽人貿易,官收其算,而罷密、登歲課,第令戶輸租錢。其後兗、鄆皆以壤地相接,罷食池鹽,得通海鹽,收算如淄、濰等州。自是諸州官不貯鹽,而百姓蠶鹽歲皆罷給,然使輸錢如故。至和中,始詔百姓輸錢以十分為率,聽減三分。



 元豐三年,京東轉運
 副使李察言:「南京、濟、濮、曹、單行解鹽;餘十有二州行海鹽,請用今稅法置買賣鹽場。」其法,盡灶戶所鬻鹽而官自賣,重禁私為市者,歲收錢二十七萬三千餘緡,而息幾半之。吳居厚為轉運判官,承察後治鹽法,利入益多。六年,較本路及河北買賣鹽場,自改法抵今一年有半,得息錢三十六萬緡。察、居厚皆進官,加賜居厚三品服。詔運賣鹽錢儲之北京,令河北都轉運使蹇周輔、判官李南公受法於居厚,行之河北。



 其在河北曰濱州場,一
 歲鬻二萬一千餘石,以給本州及棣、祈州雜支,並京東之青、淄、齊州,若大名、真定府,貝、冀、相、衛、邢、洺、深、趙、滄、磁、德、博、濱、棣、祈、定、保、瀛、莫、雄、霸州,德河、通利、永靜、乾寧、定遠、保定、廣信、永定、安肅軍則通商。後濱州分四務,又增滄州三務,歲課九千一百四十五石,以給一路,而京東之淄、青、齊既通商,乃不復給。



 自開寶以來,河北鹽聽人貿易,官收其算,歲額為錢十五萬緡。上封者嘗請禁榷以收遺利,餘靖時為諫官,亟言:「前歲軍興,河北點義勇
 強壯及諸科率,數年之間,未得休息。臣嘗痛燕薊之地,陷入契丹幾百年,而民忘南顧心者,大率契丹之法簡易,鹽曲俱賤,科役不煩故也。昔太祖推恩河朔,故許通商。今若榷之,價必騰踴,民茍懷怨,悔將何及。河朔土多鹽鹵,小民稅地不生五穀,惟刮堿煎鹽以納二稅,禁之必至逃亡。鹽價若高,犯法亦眾,邊民怨望,非國之福,乞且仍舊通商。」其議遂寢。



 慶歷六年,三司使王拱辰復建議悉榷二州鹽入官,以專其利。都轉運使魚周詢以為
 不可,且言:「商人取鹽,與所過州縣吏交通為弊,所算十無二三。請敷州縣以十分算之,聽商人至所鬻州軍並輸算錢,歲可得緡錢之十餘萬。」三司奏用其策。仁宗曰:「使人頓食貴鹽,豈朕意哉?」於是三司更立榷法而未下,張方平見上問曰:「河北再榷鹽何也?」上曰:「始議立法,非再榷。」方平曰:「周世宗榷河北鹽,犯輒處死。世宗北伐,父老遮道泣訴,願以鹽課均之兩稅,而弛其禁,許之,今兩稅鹽錢是也。豈非再榷乎?且今未榷,而契丹盜販不已,
 若榷則鹽貴,契丹之鹽益售,是為我斂怨而使契丹獲福也。契丹鹽入益多,非用兵莫能禁,邊隙一開,所得鹽利能補用兵之費乎?」上大悟曰:「其語宰相立罷之。」方平曰:「法雖未下,民已戶知之,當直以手詔罷不可自下出也。」上喜,命方平密撰手詔下之。河朔父老相率拜迎,於澶州為佛老會七日,以報上恩,且刻詔北京。後父老過其下,必稽首流涕。



 久之,緡錢所入益耗,皇祐中,視舊額幾亡其半。陜州錄事參軍王伯瑜監滄州鹽山務,獻議
 商人受鹽滄、濱二州,以囊貯之,囊毋過三石三斗,斗為鹽六斤,除三斗為耗勿算,餘算其半。予券為驗,州縣驗券縱之,聽至所鬻州軍並輸算錢;即所貯過數,予及受者皆罰,商人私挾他鹽,並沒其貲。時知滄州田京,與伯瑜合議上聞,召試行之。逾年,歲課增三萬餘緡,遂以為定制。熙寧八年,三司使章惇又請榷河北鹽,詔提舉河北、京東鹽稅周革入議,將施行焉。文彥博論其不便,乃詔仍舊。



\end{pinyinscope}