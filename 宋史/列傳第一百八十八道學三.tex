\article{列傳第一百八十八道學三}

\begin{pinyinscope}

 ○朱熹
 張栻



 朱熹,字元晦,一字仲晦,徽州婺源人。父松字喬年,中進士第。胡世將、謝克家薦之,除秘書省正字。趙鼎都督川
 陜、荊、襄軍馬,招松為屬,辭。鼎再相,除校書郎,遷著作郎。以御史中丞常同薦,除度支員外郎,兼史館校勘,歷司勛、吏部郎。秦檜決策議和,松與同列上章,極言其不可。檜怒,風御史論松懷異自賢,出知饒州,未上,卒。



 熹幼穎悟,甫能言,父指天示之曰:「天也。」熹問曰:「天之上何物?」松異之。就傅,授以《孝經》,一閱,題其上曰:「不若是,非人也。」嘗從群兒戲沙上,獨端坐以指畫沙,視之,八卦也。年十八貢於鄉,中紹興十八年進士第。主泉州同安簿,選邑秀民
 充弟子員,日與講說聖賢修己治人之道,禁女婦之為僧道者。罷歸請祠,監潭州南嶽廟。明年,以輔臣薦,與徐度、呂廣問、韓元吉同召,以疾辭。



 孝宗即位,詔求直言,熹上封事言:「聖躬雖未有過失,而帝王之學不可以不熟講。朝政雖未有闕遺,而修攘之計不可以不早定。利害休戚雖不可遍舉,而本原之地不可以不加意。陛下毓德之初,親御簡策,不過風誦文辭,吟詠情性,又頗留意於老子、釋氏之書。夫記誦詞藻,非所以探淵源而出治道;
 虛無寂滅,非所以貫本末而立大中。帝王之學,必先格物致知,以極夫事物之變,使義理所存,纖悉畢照,則自然意誠心正,而可以應天下之務。」次言:「修攘之計不時定者,講和之說誤之也。夫金人於我有不共戴天之仇,則不可和也明矣。願斷以義理之公,閉關絕約,任賢使能,立紀綱,厲風俗。數年之後,國富兵強,視吾力之強弱,觀彼釁之淺深,徐起而圖之。」次言:「四海利病,系欺民之休戚,斯民休戚,系守令之賢否。監司者守令之綱,朝廷
 者監司之本也。欲斯民之得其所,本原之地亦在朝廷而已。今之監司,奸贓狼籍、肆虐以病民者,莫非宰執、臺諫之親舊賓客。其已失勢者,既按見其交私之狀而斥去之;尚在勢者,豈無其人,顧陛下無自而知之耳。」



 隆興元年,復召。入對,其一言:「大學之道在乎格物以致其知。陛下雖有生知之性,高世之行,而未嘗隨事以觀理,即理以應事。是以舉措之間動涉疑貳,聽納之際未免蔽欺,平治之效所以未著。」其二言:「君父之仇不與共戴天。
 今日所當為者,非戰無以復仇,非守無以制勝。」且陳古先聖王所以強本折沖、威制遠人之道。時相湯思退方倡和議,除熹武學博士,待次。乾道元年,促就職,既至而洪適為相,復主和,論不合,歸。



 三年,陳俊卿、劉珙薦為樞密院編修官,待次。五年,丁內艱。六年,工部侍郎胡銓以詩人薦,與王庭珪同召,以未終喪辭。七年,既免喪,復召,以祿不及養辭。九年,梁克家相,申前命,又辭。克家奏熹屢召不起,宜蒙褒錄,執政俱稱之,上曰:「熹安貧守道,廉
 退可嘉。」特改合入官,主管臺州崇道觀。熹以求退得進,於義未安,再辭。淳熙元年,始拜命。二年,上欲獎用廉退,以勵風俗,龔茂良行丞相事以熹名進,除秘書郎,力辭,且以手書遺茂良,言一時權幸。群小乘間讒毀,乃因熹再辭,即從其請,主管武夷山沖祐觀。



 五年,史浩再相,除知南康軍,降旨便道之官,熹再辭,不許。至郡,興利除害,值歲不雨,講求荒政,多所全活。訖事,奏乞依格推賞納粟人。間詣郡學,引進士子與之講論。訪白鹿洞書院遺址,
 奏復其舊,為《學規》俾守之。明年夏,大旱,詔監司、郡守條其民間利病,遂上疏言:



 天下之務莫大於恤民,而恤民之本,在人君正心術以立紀綱。蓋天下之紀綱不能以自立,必人主之心術公平正大,無偏黨反側之私,然後有所系而立。君心不能以自正,必親賢臣,遠小人,講明義理之歸,閉塞私邪之路,然後乃可得而正。



 今宰相、臺省、師傅、賓友、諫諍之臣皆失其職,而陛下所與親密謀議者,不過一二近習之臣。上以蠱惑陛下之心志,使陛
 下不信先王之大道,而說於功利之卑說,不樂莊士之讜言,而安於私TX之鄙態。下則招集天下士大夫之嗜利無恥者,文武匯分,各入其門。所喜則陰為引援,擢置清顯。所惡則密行訾毀,公肆擠排。交通貨賂,所盜者皆陛下之財。命卿置將,所竊者皆陛下之柄。陛下所謂宰相、師傅、賓友、諫諍之臣,或反出入其門墻,承望其風旨;其幸能自立者,亦不過齪齪自守,而未嘗敢一言以斥之;其甚畏公論者,乃能略警逐其徒黨之一二,既不能深
 有所傷,而終亦不敢正言以搗其囊橐窟穴之所在。勢成威立,中外靡然向之,使陛下之號令黜陟不復出於朝廷,而出於一二人之門,名為陛下獨斷,而實此一二人者陰執其柄。



 且云:「莫大之禍,必至之憂,近在朝夕,而陛下獨未之知。」上讀之,大怒曰:「是以我為亡也。」熹以疾請祠,不報。



 陳俊卿以舊相守金陵,過闕入見,薦熹甚力。宰相趙雄言於上曰:「士之好名,陛下疾之愈甚,則人之譽之愈眾,無乃適所以高之。不若因其長而用之,彼漸
 當事任,能否自見矣。」上以為然,乃除熹提舉江西常平茶鹽公事。旋錄救荒之勞,除直秘閣,以前所奏納粟人未推賞,辭。



 會浙東大饑,宰相王淮奏改熹提舉浙東常平茶鹽公事,即日單車就道,復以納粟人未推賞,辭職名。納粟賞行,遂受職名。入對,首陳災異之由與修德任人之說,次言:「陛下即政之初,蓋嘗選建英豪,任以政事,不幸其間不能盡得其人,是以不復廣求賢哲,而姑取軟熟易制之人以充其位。於是左右私褻使令之賤,始
 得以奉燕間,備驅使,而宰相之權日輕。又慮其勢有所偏,而因重以壅己也,則時聽外廷之論,將以陰察此輩之負犯而操切之。陛下既未能循天理、公聖心,以正朝廷之大體,則固已失其本矣,而又欲兼聽士大夫之言,以為駕馭之術,則士大夫之進見有時,而近習之從容無間。士大夫之禮貌既莊而難親,其議論又苦而難入,近習便闢側媚之態既足以蠱心志,其胥史狡獪之術又足以眩聰明。是以雖欲微抑此輩,而此輩之勢日重,
 雖欲兼採公論,而士大夫之勢日輕。重者既挾其重,以竊陛下之權,輕者又借力於所重,以為竊位固寵之計。日往月來,浸淫耗蝕,使陛下之德業日隳,綱紀日壞,邪佞充塞,貨賂公行,兵愁民怨,盜賊間作,災異數見,饑饉薦臻。群小相挺,人人皆得滿其所欲,惟有陛下了無所得,而顧乃獨受其弊。」上為動容。所奏凡七事,其一二事手書以防宣洩。



 熹始拜命,即移書他郡,募米商,蠲其征,及至,則客舟之米已輻湊。熹日鉤訪民隱,按行境內,單
 車屏徒從,所至人不及知。郡縣官吏憚其風採,至自引去,所部肅然。凡丁錢、和買、役法、榷酤之政,有不便於民者,悉厘而革之。從救荒之餘,隨事處畫,必為經久之計。有短熹者,謂其疏於為政,上謂王淮曰:「朱熹政事卻有可觀。」



 熹以前後奏請多所見抑,幸而從者,率稽緩後時,蝗旱相仍,不勝憂憤,復奏言:「為今之計,獨有斷自聖心,沛然發號,責躬求言,然後君臣相戒,痛自省改。其次惟有盡出內庫之錢,以供大禮之費為收糴之本,詔戶部
 免征舊負,詔漕臣依條檢放租稅,詔宰臣沙汰被災路分州軍監司、守臣之無狀者,遴選賢能,責以荒政,庶幾猶足下結人心,消其乘時作亂之意。不然,臣恐所憂者不止於饑殍,而將在於盜賊;蒙其害者不止於官吏,而上及於國家也。」



 知臺州唐仲友與王淮同里為姻家,吏部尚書鄭丙、侍御史張大經交薦之,遷江西提刑,未行。熹行部至臺,訟仲友者紛然,按得其實,章三上,淮匿不以聞。熹論愈力,仲友亦自辯,淮乃以熹章進呈,上令宰
 屬看詳,都司陳庸等乞令浙西提刑委清強官究實,仍令熹速往旱傷州郡相視。熹時留臺未行,既奉詔,益上章論,前後六上,淮不得已,奪仲友江西新命以授熹,辭不拜,遂歸,且乞奉祠。



 時鄭丙上疏詆程氏之學以沮熹,淮又擢太府寺丞陳賈為監察御史。賈面對,首論近日搢紳有所謂「道學」者,大率假名以濟偽,願考察其人,擯棄勿用。蓋指熹也。十年,詔以熹累乞奉祠,可差主管臺州崇道觀,既而連奉雲臺、鴻慶之祠者五年。十四
 年,周必大相,除熹提點江西刑獄公事,以疾辭,不許,遂行。



 十五年,淮罷相,遂入奏,首言近年刑獄失當,獄官當擇其人。次言經總制錢之病民,及江西諸州科罰之弊。而其末言:「陛下即位二十七年,因循荏苒,無尺寸之效可以仰酬聖志。嘗反覆思之,無乃燕閑蠖濩之中,虛明應物之地,天理有所未純,人欲有所未盡,是以為善不能充其量,除惡不能去其根,一念之頃,公私邪正、是非得失之機,交戰於其中。故體貌大臣非不厚,而便嬖側媚
 得以深被腹心之寄;寤寐英豪非不切,而柔邪庸繆得以久竊廊廟之權。非不樂聞公議正論,而有時不容;非不SW讒說殄行,而未免誤聽;非不欲報復陵廟仇恥,而未免畏怯茍安;非不愛養生靈財力,而未免嘆息愁怨。願陛下自今以往,一念之頃必謹而察之:此為天理耶,人欲耶?果天理也,則敬以充之,而不使其少有壅閼;果人欲也,則敬以克之,而不使其少有凝滯。推而至於言語動作之間,用人處事之際,無不以是裁之,則聖心洞
 然,中外融澈,無一毫之私欲得以介乎其間,而天下之事將惟陛下所欲為,無不如志矣。」是行也,有要之於路,以為「正心誠意」之論上所厭聞,戒勿以為言。熹曰:「吾平生所學,惟此四字,豈可隱默以欺吾君乎?」及奏,上曰:「久不見卿,浙東之事,朕自知之,今當處卿清要,不復以州縣為煩也。」



 時曾覿已死,王抃亦逐,獨內侍甘昪尚在,熹力以為言。上曰:「昪乃德壽所薦,謂其有才耳。」熹曰:「小人無才,安能動人主。」翌日,除兵部郎官,以足疾丐祠。本部
 侍郎林慄嘗與熹論《易》、《西銘》不合,劾熹:「本無學術,徒竊張載、程頤緒餘,謂之『道學』。所至輒攜門生數十人,妄希孔、孟歷聘之風,邀索高價,不肯供職,其偽不可掩。」上曰:「林慄言似過。」周必大言熹上殿之日,足疾未廖,勉強登對。上曰:「朕亦見其跛曳。」左補闕薛叔似亦奏援熹,乃令依舊職江西提刑。太常博士葉適上疏與慄辨,謂其言無一實者,「謂之道學」一語,無實尤甚,往日王淮表裏臺諫,陰廢正人,蓋用此術。詔:「熹昨入對,所論皆新任職事,
 朕諒其誠,復從所請,可疾速之任。」會胡晉臣除侍御史,首論慄執拗不通,喜同惡異,無事而指學者為黨,乃黜慄知泉州。熹再辭免,除直寶文閣,主管西京嵩山崇福宮。未逾月再召,熹又辭。



 始,熹嘗以為口陳之說有所未盡,乞具封事以聞,至是投匭進封事曰:



 今天下大勢,如人有重病,內自心腹,外達四支,無一毛一發不受病者。且以天下之大本與今日之急務,為陛下言之:大本者,陛下之心;急務則輔翼太子,選任大臣,振舉綱紀,變
 化風俗,愛養民力,修明軍政,六者是也。



 古先聖王兢兢業業,持守此心,是以建師保之官,列諫諍之職,凡飲食、酒漿、衣服、次舍、器用、財賄與夫宦官、宮妾之政,無一不領於塚宰。使其左右前後,一動一靜,無不制以有司之法,而無纖芥之隙、瞬息之頃,得以隱其毫發之私。陛下所以精一克復而持守其心,果有如此之功乎?所以修身齊家而正其左右,果有如此之效乎?宮省事禁,臣固不得而知,然爵賞之濫,貨賂之流,閭巷竊言,久已不勝
 其籍籍,則陛下所以修之家者,恐其未有以及古之聖王也。



 至於左右便嬖之私,恩遇過當,往者淵、覿、說、抃之徒勢焰熏灼,傾動一時,今已無可言矣。獨有前日臣所面陳者,雖蒙聖慈委曲開譬,然臣之愚,竊以為此輩但當使之守門傳命,供掃除之役,不當假借崇長,使得逞邪媚、作淫巧於內,以蕩上心,立門庭、招權勢於外,以累聖政。臣聞之道路,自王抃既逐之後,諸將差除,多出此人之手。陛下竭生靈膏血以奉軍旅,顧乃未嘗得一溫
 飽,是皆將帥巧為名色,奪取其糧,肆行貨賂於近習,以圖進用,出入禁闥腹心之臣,外交將帥,共為欺蔽,以至於此。而陛下不悟,反寵暱之,以是為我之私人,至使宰相不得議其制置之得失,給諫不得論其除授之是非,則陛下所以正其左右者,未能及古之聖王又明矣。



 至於輔翼太子,則自王十朋、陳良翰之後,宮僚之選號為得人,而能稱其職者,蓋已鮮矣。而又時使邪佞儇薄、闒冗庸妄之輩,或得參錯於其間,所謂講讀,亦姑以應文
 備數,而未聞其有箴規之效。至於從容朝夕、陪侍游燕者,又不過使臣宦者數輩而已。師傅、賓客既不復置,而詹事、庶子有名無實,其左右春坊遂直以使臣掌之,既無以發其隆師親友、尊德樂義之心,又無以防其戲慢媟狎、奇邪雜進之害。宜討論前典,置師傅、賓客之官,罷去春坊使臣,而使詹事、庶子各復其職。



 至於選任大臣,則以陛下之聰明,豈不知天下之事,必得剛明公正之人而後可任哉?其所以常不得如此之人,而反容鄙
 夫之竊位者,直以一念之間,未能徹其私邪之蔽,而燕私之好,便嬖之流,不能盡由於法度,若用剛明公正之人以為輔相,則恐其有以妨吾之事,害吾之人,而不得肆。是以選擇之際,常先排擯此等,而後取凡疲懦軟熟、平日不敢直言正色之人而揣摩之,又於其中得其至庸極陋、決可保其不至於有所妨者,然後舉而加之於位。是以除書未出,而物色先定,姓名未顯,而中外已逆知其決非天下第一流矣。



 至於振肅紀綱,變化風俗,則今
 日宮省之間,禁密之地,而天下不公之道,不正之人,顧乃得以窟穴盤據於其間。而陛下目見耳聞,無非不公不正之事,則其所以熏烝銷鑠,使陛下好善之心不著,疾惡之意不深,其害已有不可勝言者矣。及其作奸犯法,則陛下又未能深割私愛,而付諸外廷之議,論以有司之法,是以紀綱不正於上,風俗頹弊於下,其為患之日久矣。而浙中為尤甚。大率習為軟美之態、依阿之言,以不分是非不辨曲直為得計,甚者以金珠為脯醢,
 以契券為詩文,宰相可啖則啖宰相,近習可通則通近習,惟得之求,無復廉恥。一有剛毅正直、守道循理之士出乎其間,則群譏眾排,指為「道學」,而加以矯激之罪。十數年來,以此二字禁錮天下之賢人君子,復如昔時所謂元祐學術者,排擯詆辱,必使無所容其身而後已,此豈治世之事哉?



 至於愛養民力,修明軍政,則自虞允文之為相也,盡取版曹歲入窠名之必可指擬者,號為歲終羨餘之數,而輸之內帑。顧以其有名無實、積累掛欠、
 空載簿籍、不可催理者,撥還版曹,以為內帑之積,將以備他日用兵進取不時之須。然自是以來二十餘年,內帑歲入不知幾何,而認為私貯,典以私人,宰相不得以式貢均節其出入,版曹不得以簿書勾考其在亡,日銷月耗,以奉燕私之費者,蓋不知其幾何矣,而曷嘗聞其能用此錢以易敵人之首,如太祖之言哉。徒使版曹經費闕乏日甚,督促日峻,以至廢去祖宗以來破分良法,而必以十分登足為限;以為未足,則又造為比較監司、
 郡守殿最之法,以誘脅之。於是中外承風,競為苛急,此民力之所以重困也。



 諸將之求進也,必先掊克士卒,以殖私利,然後以此自結於陛下之私人,而蘄以姓名達於陛下之貴將。貴將得其姓名,即以付之軍中,使自什伍以上節次保明,稱其材武堪任將帥,然後具奏牘而言之陛下之前。陛下但見等級推先,案牘具備,則誠以為公薦而可以得人矣,而豈知其諧價輸錢,已若晚唐之債帥哉?夫將者,三軍之司命,而其選置之方乖刺如
 此,則彼智勇材略之人,孰肯抑心下首於宦官、宮妾之門,而陛下之所得以為將帥者,皆庸夫走卒,而猶望其修明軍政,激勸士卒,以強國勢,豈不誤哉!



 凡此六事,皆不可緩,而本在於陛下之一心。一心正則六事無不正,一有人心私欲以介乎其間,則雖欲憊精勞力,以求正夫六事者,亦將徒為文具,而天下之事愈至於不可為矣。



 疏入,夜漏下七刻,上已就寢,亟起秉燭,讀之終篇。明日,除主管太一宮,兼崇政殿說書。熹力辭,除秘閣修
 撰,奉外祠。



 光宗即位,再辭職名,仍舊直寶文閣,降詔獎諭。居數月,除江東轉運副使,以疾辭,改知漳州。奏除屬縣無名之賦七百萬,減經總制錢四百萬。以習俗未知禮,採古喪葬嫁娶之儀,揭以示之,命父老解說,以教子弟。土俗崇信釋氏,男女聚僧廬為傅經會,女不嫁者為庵舍以居,熹悉禁之。常病經界不行之害,會朝論欲行泉、汀、漳三州經界,熹乃訪事宜,擇人物及方量之法上之。而土居豪右侵漁貧弱者以為不便,沮之。宰相留正,
 泉人也,其里黨亦多以為不可行。布衣吳禹圭上書訟其擾人,詔且需後,有旨先行漳州經界。明年,以子喪請祠。



 時史浩入見,請收天下人望,乃除熹秘閣修撰,主管南京鴻慶宮。熹再辭,詔:「論撰之職,以寵名儒。」乃拜命。除荊湖南路轉運副使,辭。漳州經界竟報罷,以言不用自劾。除知靜江府,辭,主管南京鴻慶宮。未幾,差知潭州,力辭。黃裳為嘉王府詡善,自以學不及熹,乞召為宮僚,王府直講彭龜年亦為大臣言之。留正曰:「正非不知熹,但
 其性剛,恐到此不合,反為累耳。」熹方再辭,有旨:「長沙巨屏,得賢為重。」遂拜命。會洞獠擾屬郡,熹遣人諭以禍福,皆降之。申敕令,嚴武備,戢奸吏,抑豪民。所至興學校,明教化,四方學者畢至。



 寧宗即位,趙汝愚首薦熹及陳傅良,有旨赴行在奏事。熹行且辭,除煥章閣待制、侍講,辭,不許。入對,首言:「乃者,太皇太后躬定大策,陛下寅紹丕圖,可謂處之以權,而庶幾不失其正。自頃至今三月矣,或反不能無疑於逆順名實之際,竊為陛下憂之。猶有
 可諉者,亦曰陛下之心,前日未嘗有求位之計,今日未嘗忘思親之懷,此則所以行權而不失其正之根本也。充未嘗求位之心,以盡負罪引慝之誠,充未嘗忘親之心,以致溫凊定省之禮,而大倫正,大本立矣。」復面辭待制、侍講,上手札:「卿經術淵源,正資勸講,次對之職,勿復勞辭,以副朕崇儒重道之意。」遂拜命。



 會趙彥逾按視孝宗山陵,以為土肉淺薄,下有水石。孫逢吉覆按,乞別求吉兆。有旨集議,臺史憚之,議中輟。熹竟上議狀言:「壽皇
 聖德,衣冠之藏,當博訪名山,不宜偏信臺史,委之水泉沙礫之中。」不報。時論者以為上未還大內,則名體不正而疑議生;金使且來,或有窺伺。有旨修葺舊東宮,為屋三數百間,欲徙居之。熹奏疏言:



 此必左右近習倡為此說以誤陛下,而欲因以遂其奸心。臣恐不惟上帝震怒,災異數出,正當恐懼修省之時,不當興此大役,以咈譴告警動之意;亦恐畿甸百姓饑餓流離、阽於死亡之際,或能怨望忿切,以生他變。不惟無以感格太上皇帝之
 心,以致未有進見之期,亦恐壽皇在殯,因山未卜,幾筵之奉不容少弛,太皇太后、皇太后皆以尊老之年,煢然在憂苦之中,晨昏之養尤不可闕。而四方之人,但見陛下亟欲大治宮室,速得成就,一旦翩然委而去之,以就安便,六軍萬民之心將有扼腕不平者矣。前鑒未遠,甚可懼也。



 又聞太上皇后懼忤太上皇帝聖意,不欲其聞太上之稱,又不欲其聞內禪之說,此又慮之過者。殊不知若但如此,而不為宛轉方便,則父子之間,上怨怒而
 下憂恐,將何時而已。父子大倫,三綱所系,久而不圖,亦將有借其名以造謗生事者,此又臣之所大懼也。願陛下明詔大臣,首罷修葺東宮之役,而以其工料回就慈福、重華之間,草創寢殿一二十間,使粗可居。若夫過宮之計,則臣又願陛下下詔自責,減省輿衛,入宮之後,暫變服色,如唐肅宗之改服紫袍、執控馬前者,以伸負罪引慝之誠,則太上皇帝雖有忿怒之情,亦且霍然消散,而歡意浹洽矣。



 至若朝廷之紀綱,則臣又願陛下深詔
 左右,勿預朝政。其實有勛庸而所得褒賞未愜眾論者,亦詔大臣公議其事,稽考令典,厚報其勞。而凡號令之弛張,人才之進退,則一委之二三大臣,使之反覆較量,勿循己見,酌取公論,奏而行之。有不當者,繳駁論難,擇其善者稱制臨決,則不惟近習不得干預朝權,大臣不得專任己私,而陛下亦得以益明習天下之事,而無所疑於得失之算矣。



 若夫山陵之卜,則願黜臺史之說,別求草澤,以營新宮,使壽皇之遺體得安於內,而宗社生
 靈皆蒙福於外矣。



 疏入不報,然上亦未有怒熹意也。每以所講編次成帙以進,上亦開懷容納。



 熹又奏勉上進德云:「願陛下日用之間,以求放心為之本,而於玩經觀史,親近儒學,益用力焉。數召大臣,切劘治道,群臣進對,亦賜溫顏,反覆詢訪,以求政事之得失,民情之休戚,而又因以察其人才之邪正短長,庶於天下之事各得其理。」熹奏:「禮經敕令,子為父,嫡孫承重為祖父,皆斬衰三年;嫡子當為其父後,不能襲位執喪,則嫡孫繼統而代
 之執喪。自漢文短喪,歷代因之,天子遂無三年之喪。為父且然,則嫡孫承重可知。人紀廢壞,三綱不明,千有餘年,莫能厘正。壽皇聖帝至性自天,易月之外,猶執通喪,朝衣朝冠皆用大布,所宜著在方冊,為萬世法程。間者,遺誥初頒,太上皇帝偶違康豫,不能躬就喪次。陛下以世嫡承大統,則承重之服著在禮律,所宜遵壽皇已行之法。一時倉卒,不及詳議,遂用漆紗淺黃之服,不惟上違禮律,且使壽皇已行之禮舉而復墜,臣竊痛之。然既
 往之失不及追改,唯有將來啟殯發引,禮當復用初喪之服。」



 會孝宗祔廟,議宗廟迭毀之制,孫逢吉、曾三復首請並祧僖、宣二祖,奉太祖居第一室,祫祭則正東向之位。有旨集議:僖、順、翼、宣四祖祧主,宜有所歸。自太祖皇帝首尊四祖之廟,治平間,議者以世數浸遠,請遷僖祖於夾室。後王安石等奏,僖祖有廟,與稷、契無異,請復其舊。時相趙汝愚雅不以復祀僖祖為然,侍從多從其說。吏部尚書鄭僑欲且祧宣祖而祔孝宗。熹以為藏之夾室,
 則是以祖宗之主下藏於子孫之夾室,神宗復奉以為始祖,已為得禮之正,而合於人心,所謂有舉之而莫敢廢者乎。又擬為《廟制》以辨,以為物豈有無本而生者。廟堂不以聞,即毀撤僖、宣廟室,更創別廟以奉四祖。



 始,寧宗之立,韓侂胄自謂有定策功,居中用事。熹憂其害政,數以為言,且約吏部侍郎彭龜年共論之。會龜年出護使客,熹乃上疏斥言左右竊柄之失,在講筵復申言之。御批云:「憫卿耆艾,恐難立講,已除卿宮觀。」汝愚袖御筆
 還上,且諫且拜。內侍王德謙徑以御筆付熹,臺諫爭留,不可。樓鑰、陳傅良旋封還錄黃,修注官劉光祖、鄧馹封章交上。熹行,被命除寶文閣待制,與州郡差遣,辭。尋除知江陵府,辭,仍乞追還新舊職名,詔依舊煥章閣待制,提舉南京鴻慶宮。慶元元年初,趙汝愚既相,收召四方知名之士,中外引領望治,熹獨惕然以侂胄用事為慮。既屢為上言,以數以手書啟汝愚,當用厚賞酬其勞,勿使得預朝政,有「防微杜漸,謹不可忽」之語。汝愚方謂其
 易制,不以為意。及是,汝愚亦以誣逐,而朝廷大權悉歸侂胄矣。



 熹始以廟議自劾,不許,以疾再乞休致,詔:「辭職謝事,非朕優賢之意,依舊秘閣修撰。」二年,沈繼祖為監察御史,誣熹十罪,詔落職罷祠,門人蔡元定亦送道州編管。四年,熹以年近七十,申乞致仕,五年,依所請。明年卒,年七十一。疾且革,手書屬其子在及門人範念德、黃乾,拳拳以勉學及修正遺書為言。翌日,正坐整衣冠,就枕而逝。



 熹登第五十年,仕於外者僅九考,立朝才四十
 日。家故貧,少依父友劉子羽,寓建之崇安,後徙建陽之考亭,簞瓢屢空,晏如也。諸生之自遠而至者,豆飯藜羹,率與之共。往往稱貸於人以給用,而非其道義則一介不取也。



 自熹去國,侂胄勢益張。何澹為中司,首論專門之學,文詐沽名,乞辨真偽。劉德秀仕長沙,不為張栻之徒所禮,及為諫官,首論留正引偽學之罪。「偽學」之稱,蓋自此始。太常少卿胡紘言:「比年偽學猖獗,圖為不軌,望宣諭大臣,權住進擬。」遂召陳賈為兵部侍郎。未幾,熹有
 奪職之命。劉三傑以前御史論熹、汝愚、劉光祖、徐誼之徒,前日之偽黨,至此又變而為逆黨。即日除三傑右正言。右諫議大夫姚愈論道學權臣結為死黨,窺伺神器。乃命直學士院高文虎草詔諭天下,於是攻偽日急,選人余嘉至上書乞斬熹。



 方是時,士之繩趨尺步、稍以儒名者,無所容其身。從游之士,特立不顧者,屏伏丘壑;依阿巽懦者,更名他師,過門不入,甚至變易衣冠,狎游市肆,以自別其非黨。而熹日與諸生講學不休,或勸以
 謝遣生徒者,笑而不答。有籍田令陳景思者,故相康伯之孫也,與侂胄有姻連,勸侂胄勿為已甚,侂胄意亦漸悔。熹既沒,將葬,言者謂:四方偽徒期會,送偽師之葬,會聚之間,非妄談時人短長,則繆議時政得失,望令守臣約束。從之。



 嘉泰初,學禁稍弛。二年,詔:「朱熹已致仕,除華文閣待制,與致仕恩澤。」後侂胄死,詔賜熹遺表恩澤,謚曰文。尋贈中大夫,特贈寶謨閣直學士。理宗寶慶三年,贈太師,追封信國公,改徽國。



 始,熹少時,慨然有求道之志。
 父松病亟,嘗屬熹曰:「籍溪胡原仲、白水劉致中、屏山劉彥沖三人,學有淵源,吾所敬畏,吾即死,汝往事之,而惟其言之聽。」三人,謂胡憲、劉勉之、劉子翬也。故熹之學既博求之經傳,復遍交當世有識之士。延平李侗老矣,嘗學於羅從彥,熹歸自同安,不遠數百里,徒步往從之。



 其為學,大抵窮理以致其知,反躬以踐其實,而以居敬為主。嘗謂聖賢道統之傳散在方冊,聖經之旨不明,而道統之傳始晦。於是竭其精力,以研窮聖賢之經訓。所著
 書有:《易》本義、啟蒙、《蓍卦考誤》,《詩集傳》,《大學中庸》章句、或問、《論語》、《孟子》集注、《太極圖》、《通書》、《西銘解》、《楚辭》集注、辨證,《韓文考異》;所編次有:《論孟集議》,《孟子指要》,《中庸輯略》,《孝經刊誤》,《小學書》,《通鑒綱目》,《宋名臣言行錄》,《家禮》,《近思錄》,《河南程氏遺書》,《伊洛淵源錄》,皆行於世。熹沒,朝廷以其《大學》、《語》、《孟》、《中庸》訓說立於學官。又有《儀禮經傳通解》未脫稿,亦在學官。平生為文凡一百卷,生徒問答凡八十卷,別錄十卷。



 理宗紹定末,秘書郎李心傳乞以司馬光、
 周敦頤、邵雍、張載、程顥、程頤、朱熹七人列於從祀,不報。淳祐元年正月,上視學,手詔以周、張、二程及熹從祀孔子廟。



 黃乾曰:「道之正統待人而後傳,自周以來,任傳道之責者不過數人,而能使斯道章章較著者,一二人而止耳。由孔子而後,曾子、子思繼其微,至孟子而始著。由孟子而後,周、程、張子繼其絕,至熹而始著。」識者以為知言。



 熹子在,紹定中為吏部侍郎。



 張栻字敬夫,丞相浚子也。穎悟夙成,浚愛之,自幼學,所
 教莫非仁義忠孝之實。長師胡宏,宏一見,即以孔門論仁親切之旨告之。栻退而思,若有得焉,宏稱之曰:「聖門有人矣。」栻益自奮厲,以古聖賢自期,作《希顏錄》。



 以蔭補官,闢宣撫司都督府書寫機宜文字,除直密閣,時孝宗新即位,浚起謫籍,開府治戎,參佐皆極一時之選。栻時以少年,內贊密謀,外參庶務,其所綜畫,幕府諸人皆自以為不及也。間以軍事入奏,因進言曰:「陛下上念宗社之仇恥,下閔中原之塗炭,惕然於中,而思有以振之。臣
 謂此心之發,即天理之所存也。願益加省察,而稽古親賢以自輔,無使其或少息,則今日之功可以必成,而因循之弊可革矣。」孝宗異其言,於是遂定君臣之契。



 浚去位,湯思退用事,遂罷兵講和。金人乘間縱兵入淮甸,中外大震,廟堂猶主和議,至敕諸將無得輒稱兵。時浚已沒,栻營葬甫畢,即拜疏言:「吾與金人有不共戴天之仇,異時朝廷雖嘗興縞素之師,然旋遣玉帛之使,是以講和之念未忘於胸中,而至忱惻怛之心無以感格於天
 人之際,此所以事屢敗而功不成也。今雖重為群邪所誤,以蹙國而召寇,然亦安知非天欲以是開聖心哉。謂宜深察此理,使吾胸中了然無纖芥之惑,然後明詔中外,公行賞罰,以快軍民之憤,則人心悅,士氣充,而敵不難卻矣。繼今以往,益堅此志,誓不言和,專務自強,雖折不撓,使此心純一,貫徹上下,則遲以歲月,亦何功之不濟哉?」疏入,不報。



 久之,劉珙薦於上,除知撫州,未上,改嚴州。時宰相虞允文以恢復自任,然所以求者類非其道,
 意栻素論當與己合,數遣人致殷勤,栻不答。入奏,首言:「先王所以建事立功無不如志者,以其胸中之誠有以感格天人之心,而與之無間也。今規畫雖勞,而事功不立,陛下誠深察之日用之間,念慮云為之際,亦有私意之發以害吾之誠者乎?有則克而去之,使吾中局洞然無所間雜,則見義必精,守義必固,而天人之應將不待求而得矣。夫欲復中原之地,先有以得中原之心,欲得中原之心,先有以得吾民之心。求所以得吾民之心者,
 豈有他哉?不盡其力,不傷其財而已矣。今日之事,固當以明大義、正人心為本。然其所施有先後,則其緩急不可以不詳;所務有名實,則其取舍不可以不審,此又明主所宜深察也。」



 明年,召為吏部侍郎,兼權起居郎侍立官。時宰方謂敵勢衰弱可圖,建議遣泛使往責陵寢之故,士大夫有憂其無備而召兵者,輒斥去之。栻見上,上曰:「卿知敵國事乎?」栻對曰:「不知也。」上曰:「金國饑饉連年,盜賊四起。」栻曰:「金人之事,臣雖不知,境中之事,則知之
 矣。」上曰:「何也?」栻曰:「臣切見比年諸道多水旱,民貧日甚,而國家兵弱財匱,官吏誕謾,不足倚賴。正使彼實可圖,臣懼我之未足以圖彼也。」上為默然久之。栻因出所奏疏讀之曰:「臣竊謂陵寢隔絕,誠臣子不忍言之至痛,然今未能奉辭以討之,又不能正名以絕之,乃欲卑祠厚禮以求於彼,則於大義已為未盡。而異論者猶以為憂,則其淺陋畏怯,固益甚矣。然臣竊揆其心意,或者亦有以見我未有必勝之形,而不能不憂也歟。蓋必勝之形,
 當在於早正素定之時,而不在於兩陣決機之日。」上為竦聽改容。栻復讀曰:「今日但當下哀痛之詔,明復仇之義,顯絕金人,不與通使。然後修德立政,用賢養民,選將帥,練甲兵,通內修外攘、進戰退守以為一事,且必治其實而不為虛文,則必勝之形隱然可見,雖有淺陋畏怯之人,亦且奮躍而爭先矣。」上為嘆息褒諭,以為前始未聞此論也。其後因賜對反復前說,上益嘉嘆,面諭:「當以卿為講官,冀時得晤語也。」



 會史正志為發運使,名為均
 輸,實盡奪州縣財賦,遠近騷然,士大夫爭言其害,栻亦以為言。上曰:「正志謂但取之諸郡,非取之於民也。」栻曰:「今日州郡財賦大抵無餘,若取之不已,而經用有闕,不過巧為名色以取之於民耳。」上矍然曰:「如卿之言,是朕假手於發運使以病吾民也。」旋閱其實,果如栻言,即詔罷之。



 兼侍講,除左司員外郎。講《詩葛覃》,進說:「治生於敬畏,亂起於驕淫。使為國者每念稼穡之勞,而其後妃不忘織糸任之事,則心不存者寡矣。」因上陳祖宗自家刑國
 之懿,下斥今日興利擾民之害。上嘆曰:「此王安石所謂『人言不足恤』者,所以為誤國也。」



 知閣門事張說除簽書樞密院事,栻夜草疏極諫其不可,旦詣朝堂,質責宰相虞允文曰:「宦官執政,自京、黼始,近習執政,自相公始。」允文慚憤不堪。栻復奏:「文武誠不可偏,然今欲右武以均二柄,而所用乃得如此之人,非惟不足以服文吏之心,正恐反激武臣之怒。」孝宗感悟,命得中寢。然宰相實陰附說,明年出栻知袁州,申說前命,中外喧嘩,說竟以
 謫死。



 栻在朝未期歲,而召對至六七,所言大抵皆修身務學,畏天恤民,抑僥幸,屏讒諛,於是宰相益憚之,而近習尤不悅。退而家居累年,孝宗念之,詔除舊職,知靜江府,經略安撫廣南西路。所部荒殘多盜,栻至,簡州兵,汰冗補闕,籍諸州黥卒伉健者為效用,日習月按,申嚴保伍法。諭溪峒酋豪弭怨睦鄰,毋相殺掠,於是群蠻帖服。朝廷買馬橫山,歲久弊滋,邊氓告病,而馬不時至。栻究其利病六十餘條,奏革之,諸蠻感悅,爭以善馬至。



 孝宗聞
 栻治行,詔特進秩,直寶文閣,因任。尋除秘閣修撰、荊湖北路轉運副使。改知江陵府,安撫本路。一日去貪吏十四人。湖北多盜,府縣往往縱釋以病良民,栻首劾大吏之縱賊者,捕斬奸民之舍賊者,令其黨得相捕告以除罪,群盜皆遁去。郡瀕邊屯,主將與帥守每不相下,栻以禮遇諸將,得其歡心,又加恤士伍,勉以忠義,隊長有功輒補官,士咸感奮。並淮奸民出塞為盜者,捕得數人,有北方亡奴亦在盜中。



 栻曰:「朝廷未能正名討敵,無使
 疆場之事其曲在我。」命斬之以徇於境,而縛其亡奴歸之。北人嘆曰:「南朝有人。」



 信陽守劉大辯怙勢希賞,廣招流民,而奪見戶熟田以與之。栻劾大辨詐諼,所招流民不滿百,而虛增其數十倍,請論其罪,不報。章累上,大辯易他郡,栻自以不得其職求去,詔以右文殿修撰提舉武夷山沖祐觀。病且死,猶手疏勸上親君子遠小人,信任防一己之偏,好惡公天下之理。天下傳誦之。栻有公輔之望,卒時年四十有八。
 孝宗聞之,深為嗟悼,四方賢士大夫往往出涕相吊,而江陵、靜江之民尤哭之哀。嘉定間,賜謚曰宣。淳祐初,詔從祀孔子廟。



 栻為人表裏洞然,勇於從義,無毫發滯吝。每進對,必自盟於心,不可以人主意悅輒有所隨順。孝宗嘗言伏節死義之臣難得,栻對:「當於犯顏敢諫中求之。若平時不能犯顏敢諫,他日何望其伏節死義?」孝宗又言難得辦事之臣,栻對:「陛下當求曉事之臣,不當求辦事之臣。若但求辦事之臣,則他日敗陛下事者,未必非此人也。」栻自言:前後奏對忤上旨雖多,而上每念之,未嘗加怒者,所謂可以理奪云爾。



 其遠小人尤嚴。為都司日,肩輿出,遇曾覿,覿舉手欲揖,栻急掩其窗欞,覿慚,手不得下。所至郡,暇日召諸生告語。民以事至庭,必隨事開曉。具為條教,大抵以正禮俗、明倫紀為先。斥異端,毀淫祠,而崇社稷山川古先聖賢之祀,舊典所遺,亦以義起也。



 栻聞道甚早,朱熹嘗言:「己之學乃銖積寸累而成,如敬夫,則於大本卓然先有見者也。」所著《論語孟子說》、《太極圖說》、《洙泗言仁》、《諸葛忠武侯傳》、《經世紀年》,皆行於世。栻之言曰:「學莫先於義利之辨。義者,本心之當為,非有為而為也。有為而為,則皆人欲,非天理。」此栻講學之要也。子焯。



\end{pinyinscope}