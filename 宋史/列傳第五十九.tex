\article{列傳第五十九}

\begin{pinyinscope}

 楊偕王沿子鼎杜杞楊畋周湛徐的姚仲孫陳太素馬尋杜曾附李虛己張傅俞獻卿陳從易楊大雅



 楊偕,字次公,坊州中部人。唐左僕射於陵六世孫。父守
 慶,仕廣南劉氏,歸朝,為坊州司馬,因家焉。偕少從種放學於終南山,舉進士,釋褐坊州軍事推官、知汧源縣,再調漢州軍事判官。道遇術士曰:「君知世有化瓦石為黃金者乎?」就偕試之,既驗,欲授以方。偕曰:「吾從吏祿,安事化金哉?」術士曰:「子志若此,非吾所及也。」出戶,失所之。



 在官,數上書論時政,又上所著文論。召試學士院,不中,改永興軍節度推官。又上書論陜西邊事,復召試,不赴,即遷秘書省著作佐郎,為審刑院詳議官,再遷太常博士。
 宋綬薦為監察御史,改殿中侍御史。與曹修古連疏,言劉從德遺奏恩太濫,貶太常博士、監舒州稅。以尚書祠部員外郎知光州,改侍御史,為三司度支判官。



 時郭皇后廢,偕與孔道輔、範仲淹力爭。道輔、仲淹既出,偕止罰金。乃言願得與道輔等皆貶,不報。富民陳氏女選入宮,將以為後,偕復上疏諫上。以尚書戶部員外郎兼侍御史知雜事。馬季良以罪斥置滁州,自言得致仕。偕以謂致仕用優賢者,不當以寵罪人,又數論升降之弊,仁宗
 嘉納之。判吏部流內銓,徙三司度支副使,擢天章閣待制、河北轉運使。按知定州夏守恩贓數萬,守恩流嶺南。明年,丁母憂,願終制,不許,進龍圖閣直學士、知河中府。



 元昊反,劉平、石元孫戰沒。偕聞,乃偽為書馳告延州曰:「朝廷遣救兵十萬至矣。」命旁郡縣大具芻糧、什器以俟。比書至,賊已解去。夏竦為陜西經略使,請增置土兵,易戍兵歸衛京師。偕言:「方關中財用乏,復增土兵,徒耗國用。今賊勢方盛,雖大增土兵,亦未能減戍兵東歸,第竦
 懼敗事,欲以兵少為解爾。」竦復奏偕不忠,沮邊計,偕爭愈力。時陜西議立五保,偕又以為擾民,疏請罷之。徙陜州,又徙河東都轉運使。詔大選三路之民,募為兵。偕復言:「方今兵不為少,茍多而不練,則其勢易以敗,又困國而難供。」時論者惟務多兵,而偕論常如此。



 進樞密直學士、知並州。及元昊入寇,密詔偕選強壯萬人,策應麟、府。偕奏:「出師臨陣,無紀律則士不用命。今發農卒赴邊,慮在路逃逸及臨陣退縮、不稟號令,請以軍法從事。」詔如
 所請。並人大驚畏,都轉運使文彥博奏罷之。有中官預軍事素橫,前帥優遇之。偕至,一繩以法,命率所部兵從副總管赴河外,戒曰:「遇賊將戰,一稟副總管節度。」中人不服,捧檄訴。偕叱曰:「汝知違主帥命即斬首乎?」監軍怖汗,不覺墮笏,翌日告疾,未幾遂卒。於是軍政肅然。



 元昊大掠河北,詔修寧遠砦。偕言:寧遠砦在河外,介麟、豐二州之間,無水泉可守。請建新麟州於嵐州,有白塔地可建砦屯兵。謂「遷有五利,不遷有三害。省國用,惜民力,利
 一也。內御岢嵐、石府州沿河一帶賊所出路,利二也。我據其要,則河冰雖合,賊不敢逾河而東,利三也。商旅往來以通貨財,利四也。方河凍時,得所屯兵馬五七千人以張軍勢,利五也。今麟州轉輸束芻斗粟,費直千錢,若因循不遷,則河東之民,困於調發無已時,害一也。以孤壘餌敵,害二也。道路艱阻,援兵難繼,害三也。且州之四面,屬羌遭賊驅脅,蕩然一空,止存孤壘,猶四支盡廢,首面心腹獨存也。今契丹又與西賊共謀,待冰合來攻河
 東,若朝廷不思御捍之計而修寧遠砦,是求虛名而忽大患也。況靈、夏二州皆漢、唐郡,一旦棄之,一麟州何足惜哉!」書奏,帝謂輔臣曰:「麟州,古郡也。咸平中,嘗經寇兵攻圍,非不可守,今遽欲棄之,是將退而以河為界也。宜諭偕速修復寧遠,以援麟州。」



 明年,改左司郎中、本路經略安撫招討使,賜錢五十萬。偕列六事於朝:一、罷中人預軍事;二、徙麟州;三、以便宜從事;四、出冗師;五、募武士;六、專捕援。且曰:「能用臣言則受命,不然則已。」朝廷不從,
 偕累奏不已,乃罷知邢州,徙滄州。求面論兵事,召還,今間日入對。



 偕在並州日,嘗論《八陣圖》及進神楯、劈陣刀,其法外環以車,內比以楯。至是,帝命以步卒五百,如其法布陣於庭,善之,乃下其法於諸路。其後王吉果用偕刀楯法敗元昊于兔毛川。久之,遷翰林侍讀學士、知審官院,復以為左司郎中。元昊乞和而不稱臣,偕以謂連年出師,國力日蹙,宜權許之,徐圖誅滅之計。諫官王素、歐陽修、蔡襄累章劾奏:「偕職為從官,不思為國討賊,而
 助元昊不臣之請,罪當誅。陛下未忍加戮,請出之,不宜留處京師。」帝以其章示偕,偕不自安,乃求知越州,道改杭州。時襄謁告過杭而輕游里市,或謂偕合言於朝。對曰:「襄嘗緣公事抵我,我豈可以私報耶?」又上《太平可致十象圖》。



 還,判太常、司農寺,改右諫議大夫。請老,以尚書工部侍郎致仕。於其歸,特賜宴。嘗召問,賜不拜。卒,遺奏《兵論》一篇,帝憐之,特贈兵部侍郎。偕性剛而忠樸,敢為大言,數上書論天下事,議者以為迂闊難用。與人少合,
 尤喜古今兵法,有《兵書》十五卷,集十卷。子忱、慥,皆有雋才,早卒。



 王沿,字聖源,大名館陶人。少治《春秋》。中進士第,試秘書省校書郎,歷知彭城、新昌二縣,改相州觀察推官,知宗城縣。張知白薦其才,擢著作佐郎,入為審刑院詳議官,再遷太常博士。上書論:



 漢、唐之初,兵革才定,未暇治邊圉,則屈意以講和。承平之後,武力有餘,而外侮不已,則以兵治之,孝武之於匈奴,太宗之於突厥頡利是也。宋
 興七十年,而契丹數侵深、趙、貝、魏之間,先朝患徵調之不已也,故屈己與之盟。然彼以戈矛為耒耜,以剽虜為商賈;而我壘不堅,兵不練,而規規於盟歃之間,豈久安之策哉?



 夫善禦敵者,必思所以務農實邊之計。河北為天下根本,其民儉嗇勤苦,地方數千里,古號豐實。今其地,十三為契丹所有,餘出征賦者,七分而已。魏史起鑿十二渠,引漳水溉斥鹵之田,而河內饒足。唐至德後,渠廢,而相、魏、磁、洺之地並漳水者,累遭決溢,今皆斥鹵不
 可耕。故沿邊郡縣,數蠲租稅,而又牧監芻地,占民田數百千頃,是河北之地,雖十有其七,而得賦之實者,四分而已。以四分之力,給十萬防秋之師,生民不得不困也。且牧監養馬數萬,徒耗芻豢,未嘗獲其用。請擇壯者配軍,衰者徙之河南,孳息者養之民間。罷諸坰牧,以其地為屯田,發役卒、刑徒田之,歲可用獲穀數十萬斛。夫漳水一石,其泥數斗,古人以為利,今人以為害,系乎用與不用爾。願募民復十二渠,渠復則水分,水分則無奔
 決之患。以之灌溉,可使數郡瘠鹵之田,變為膏腴,如是,則民富十倍,而帑廩有餘矣。以此馭敵,何求而不可。



 詔河北轉運使規度,而通判洺州王軫言:「漳河岸高水下,未易疏導;又其流濁,不可溉田。」沿方遷監察御史,即上書駁軫說,帝雖嘉之而不即行,語在《河渠志》。時樞密副使晏殊以笏擊從者折齒,知開封府陳堯咨、判官張宗誨日嗜酒惰事,沿皆彈奏之。天聖五年,安撫關陜,減諸縣秋稅十二三。還,為開封府推官。又體量河朔饑民,所至
 不俟詔,發官廩濟之。就除轉運副使。上言:



 本朝制兵刑,未幾於古。自契丹通好三十年,二邊常屯重兵,坐耗國用,而未知所以處之。請教河北強壯,以代就糧禁卒之闕;罷招廂軍,以其冗者隸作屯田。行之數年,禁卒當漸銷減,而強壯悉為精兵矣。



 古者「刑國平,用中典」,而比者以敕處罪,多重於律。以絹估罪者,敕以緡直代之,律坐髡鈦而役者,敕黥竄以為卒。比諸州上言,謫卒太多,衣食不足,願勿復謫者七十餘州。以律言之,皆不至是,是
 以繁文罔之而置於理也。誠願削深文而用正律,以錢定罪者,悉從絹估;黥竄為卒者,止從髡鈦。此所謂勝殘去殺,無待百年者也。



 被詔鞫曹汭獄於真定府,遷殿中侍御史。母喪服除,改尚書工部員外郎、知邢州,復起為河北轉運使。奏罷二牧監,以地賦民。導相、衛、邢、趙水下天平、景祐諸渠,溉田數萬頃。因詣闕奏事,上所著《春秋集傳》十六卷,復上書以《春秋》論時事。授直昭文館,為三司戶部副使,徙鹽鐵,遷兵部員外郎、天章閣待制、陜西
 都轉運使。時朝廷將減卒戍,就食內地,詔與知州、總管、鈐轄等議。沿即奏減卒數萬,知樞密院李諮以為不可,復下沿邊都監議。沿上疏曰:「兵機當在廊廟之上,豈可取責小人哉!」諮惡其言,奏罷之,降知滑州,徙成德軍。建學校,行鄉飲酒禮。



 遷刑部郎中、河東都轉運使,加龍圖閣直學士、知並州。時元昊數寇河東,建議徙豐州,不報,已而州果陷。進樞密直學士、右司郎中,為涇原路經略、安撫、招討使兼知渭州。增屯兵,城中隘甚,乃築西關城
 五里。改涇州觀察使。元昊入寇,副都總管葛懷敏率兵出捍,沿教懷敏率兵據瓦亭待之。懷敏進兵鎮戎,沿以書戒勿入,第背城為砦,以羸師誘賊,賊至,發伏擊之可有功。懷敏不聽,進至定川,果為所敗。賊乘勝犯渭州,沿率州人乘城,多張旗幟為疑兵,賊遂引去。坐懷敏敗,復為龍圖閣直學士、刑部郎中、知虢州,尋降天章閣待制,而為權御史中丞賈昌朝所奏,落待制。未幾,徙知成德軍,復待制,又徙河中府,卒。



 沿好建明當世事,而其論多
 齟齬。初興河北水利,導諸渠溉民田,論者以為無益。已而邢州民有爭渠水至殺人者,然後人知沿所建為利。嘗論以《春秋》法斷事,然真定之獄,人以為沿傅致之。有文集二十卷,《唐志》二十一卷。子鼎。



 鼎字鼎臣,以進士第,累遷太常博士。王堯臣領三司,舉勾當公事,數上書論時政得失。時天子患吏治多弛,監司不舉職,而範仲淹等方執政,擇諸路使者令按舉不法,以鼎提點江東刑獄。與轉運使楊紘、判官王綽競擿
 發吏,至微隱罪無所貸。於是所部官吏怨之,目為「三虎」。仁宗聞之,不說,後傅惟幾奉使江東,戒以毋效「三虎」為也。仲淹等罷,鼎與紘、綽皆為人所言,時鼎提點兩浙刑獄,降知深州。



 王則以貝州反,深卒龐旦與其徒,謀以元日殺軍校、劫庫兵應之。前一日,有告者。鼎夜出檄,遣軍校攝事外邑,而陰為之備。翌日,會僚吏置酒如常,叛黨愕不敢動。鼎刺得實,徐捕首謀十八人送獄。獄具,俟轉運使至審決。未至,軍中恟恟謀劫囚。鼎因謂僚吏曰:「吾
 不以累諸君。」獨命取囚桀驁者數人,斬於市,眾皆失色,一郡帖然。轉運使至,囚未決者半,訊之,皆伏誅。



 明年,河北大饑,人相食,鼎經營賑救,頗盡力。徙建州,其俗生子多不舉,鼎為條教禁止。時盜販茶鹽者眾,一切杖遣之,監司數以為言,鼎弗為變。徙提點河北刑獄,治奸贓益急,所劾舉,不避貴勢。召為開封府判官,改鹽鐵判官,累遷司封員外郎、淮南兩浙荊湖制置發運副使。內侍楊永德奏請沿汴置鋪挽漕舟,歲可省卒六萬,鼎議以為
 不可。永德橫猾,執政重違其奏,乃令三司判官一員將永德就鼎議,發八難,永德不能復。鼎因疏言:「陛下幸察用臣,不宜過聽小人,妄有所改,以誤國計。」於是永德言不用。



 居二年,遂以為使。前使者多漁市南物,因奏計京師,持遺權貴。鼎一無所市,獨悉意精吏事,事無大小,必出於己。凡調發綱吏,度漕路遠近,定先後為成法,於是勞逸均,吏不能為重輕。官舟禁私載,舟兵無以自給,則盡盜官米為奸。有能居販自贍者,市人持以法,不肯償
 所逋。鼎為移州縣督償之,舟人有以自給,不為奸,而所運米未嘗不足也。入為三司鹽鐵副使。數與包拯爭議,不少屈。拯素強,然無如之何。遷刑部郎中、天章閣待制、河北都轉運使,徙使河東,卒。



 鼎性廉不欺,嘗任其子,族人欲增年以圖速仕,鼎不可。父死,分諸子以財,鼎悉推與其弟。嘗知臨邛縣,轉運使選攝新繁,新繁多職田,鬥粟不以自入。奉使契丹,得千縑,散之族人,一日盡。所至不擾,唯市飲食日用物,增直以償。事繼母孝,教育孤侄
 甚至,自奉養儉約。當官明敏,強直不可撓。所薦士多知名,有終身不識者。然性猜忌,其行部,至於藥餌,皆手自扃鐍。至潞州八義館,疾作,不知人事,左右遑遽,發藥奩,悉無題識,莫敢進,以迄於卒。初,鼎與弟豫皆有才氣,好上書言事,仁宗稱之,以為豫孟浪,鼎所言多可用。豫為人不事羈檢,以大理寺丞知伊闕縣,有異政。棄官浮游江、湖間,殖貨自給以卒。



 杜杞,字偉長。父鎬,蔭補將作監主簿,知建陽縣。強敏有
 才。閩俗,老而生子輒不舉。杞使五保相察,犯者得重罪。累遷尚書虞部員外郎、知橫州。時安化蠻寇邊,殺知宜州王世寧,出兵討之。杞言:「嶺南諸郡,無城郭甲兵之備,牧守非才。橫為邕、欽、廉三郡咽喉,地勢險阻,可屯兵為援。邕管內制廣源,外控交址,願擇文臣識權變練達嶺外事者,以為牧守,使經制邊事。」改通判真州,徙知解州,權發遣度支判官。盜起京西,掠商、鄧、均、房,焚光化軍,授京西轉運、按察使。居數月,賊平。



 會廣西區希範誘白崖
 山蠻蒙趕反,有眾數千,襲破環州、帶溪普義鎮寧砦,嶺外騷然。擢刑部員外郎、直集賢院、廣南西路轉運按察安撫使。行次真州,先遣急遞以書諭蠻,聽其自新。次宜州,蠻無至者。杞得州校,出獄囚,脫其械,使入洞說賊,不聽。乃勒兵攻破白崖、黃坭、九居山砦及五峒,焚毀積聚,斬首百餘級,復環州。賊散走,希範走荔波洞,杞遣使誘之,趕來降。杞謂將佐曰:「賊以窮蹙降我,威不足制則恩不能懷,所以數叛,不如盡殺之。」乃擊牛馬,為曼陀羅酒,
 大會環州,伏兵發,誅七十餘人。後三日,又得希範,醢之以遺諸蠻,因老病而釋者,才百餘人。御史梅摯劾杞殺降失信,詔戒諭之,為兩浙轉運使。明年,徙河北,拜天章閣待制、環慶路經略安撫使、知慶州。杞上言:「殺降者臣也,得罪不敢辭。將吏勞未錄,臣未敢受命。」因為行賞。蕃酋率眾千餘內附,夏人以兵索酋而劫邊戶,掠馬牛,有詔責杞。杞言:「彼違誓舉兵,酋不可與。」因移檄夏人,不償所掠,則酋不可得,既而兵亦罷去。



 杞性強記,博覽書傳,
 通陰陽數術之學,自言吾年四十六死矣。一日據廁,見希範與趕在前訴冤,叱曰:「爾狂僭叛命,法當誅,尚敢訴邪!」未幾卒。有奏議十二卷。



 兄植,以文雅知名,累任監司,終少府監。弟樞,亦強敏,為比部員外郎。有張彥方者,溫成皇后母越國夫人客也。坐奸利論死,語連越國夫人。開封不敢窮治,執政以後故,亦不復詰。獄上,中書遣樞慮問,樞揚言將駁正;亟改用諫官陳升之,權幸切齒於樞。前此,御史中丞王舉正留百官班論張堯佐除宣徽
 使,樞嘗出班問其故。至是,蓋累月矣,坐是罪樞,絀監衡州稅,卒。



 楊畋〈字樂道,保靜軍節度使重勛之曾孫。進士及第,授秘書省校書郎、並州錄事參軍,再遷大理寺丞、知嶽州。慶歷三年,湖南徭人唐和等劫掠州縣,擢殿中丞、提點本路刑獄,專治盜賊事。乃募才勇,深入峒討擊。然南方久不識兵,士卒多畏懾。及戰孤漿峒,前軍衄,大兵悉潰,畋踣巖下,藉淺草得不死。卒厲眾平六峒,以功,遷太常
 博士。未幾,坐部將胡元戰死,降知太平州。歲餘,賊益肆。帝遣御史按視,還言:「畋嘗戰山下,人樂為用,今欲殄賊,非畋不可。」乃授東染院使、荊湖南路兵馬鈐轄。賊聞畋至,皆恐畏,逾嶺南遁。又詔往韶、連等州招安之。乃約賊使出峒,授田為民,而轉運使欲授以官與貲,納質使還。畋曰:「賊剽攻湖、廣七年,所殺不可勝計,今使飽貲糧、據峒穴,其勢不久必復亂。」明年春,賊果復出陽山。畋即領眾出嶺外,涉夏、秋,凡十五戰,賊潰,畋感瘴疾歸。蠻平,願
 還舊官,改尚書屯田員外郎、直史館、知隨州。



 召還,為三司戶部判官,奉使河東。丁父憂,會儂智高陷邕州,召至都門外,辭以喪服不敢見。仁宗賜以服飾禦巾,入對便殿。即日,除起居舍人、知諫院、廣南東西路體量安撫、經制賊盜。畋至韶州,會張忠戰死,智高自廣州回軍沙頭,將濟。畋令蘇緘棄英州,蔣偕焚糧儲,及召開贇、岑宗閔、王從政退保韶州。賊勢愈熾,畋不能抗,遂殺蔣偕、王正倫,敗陳曙,復據邕州。畋坐是落知諫院、知鄂州,再降為
 屯田員外郎、知光化軍。明年,又降為太常博士,歲終,徙邠州。



 復起居舍人,為河東轉運使。入為三司戶部副使,遷吏部員外郎。奉使契丹,以曾伯祖業嘗陷虜,辭不行。河北舊以土絹給軍裝,三司使張方平易以他州絹。畋既同書奏聞,外議籍籍,又密陳其不可。久之,擢天章閣待制兼侍讀、判吏部流內銓。上言:「願擇宗室之賢者,使侍膳禁中,為宗廟計。」



 嘉祐三年冬,河北地震。明年,日食正旦。復上疏曰:「漢成帝時,日食地震,哀、平之世,嫡嗣屢
 絕,此天所以示戒也。陛下宜早立皇嗣,以答天意。」改知制誥。李珣自防禦使遷觀察,劉永年自團練使遷防禦,畋當草制,封還詞頭。因言:「祖宗故事,郭進戍西山,董遵誨、姚內斌守環、慶,與強寇對壘,各十餘年,未嘗轉官移鎮,重名器也。今珣等無尺寸功,特以外戚故除之,恐非祖宗意。」不報,詔他舍人草制。而範鎮言:「朝廷如以畋言為是,當罷珣等所遷官;倘以為非,乞復令畋命詞。」不允。進龍圖閣直學士,復知諫院。



 嘉祐六年,京師大水,畋上
 言:「《洪範五行傳》:『簡宗廟則水不潤下。』又曰:『聽之不聰,厥罰常水。』去年夏秋之交,久雨傷稼,澶州河決,東南數路,大水為沴。陛下臨御以來,容受直諫,非聽之不聰也。以孝事親,非簡於宗廟也。然而災異數見,臣愚殆以為萬機之聽,必有失於審者;七廟之享,必有失於順者,惟陛下積思而矯正之。」乃下其章禮官並兩制考議,咸言南郊三聖並侑,溫成皇后立廟,皆違經禮。於是詔:「自今南郊以太祖皇帝定配,改溫成廟為祠殿。」



 舊制,內侍十年
 一遷官。樞密院以為僥幸,乃更定歲數倍之。畋言:「文臣七遷,而內侍始得一磨勘,為不均。宜如文武官僚例,增其歲考。」遂詔南班以上仍舊制,無勞而嘗坐罪徒者,即倍其年。議者謂畋以士人比閹寺為失。卒,贈右諫議大夫。



 畋出於將家,折節喜學問,為士大夫所稱。大山下討蠻,家問至,即焚之,與士卒同甘苦,破諸峒。及用之嶺南,以無功斥,名稱遂衰。性情介謹畏,每奏事,必發封數四而後上之。自奉甚約,為郡待客,雖監司,菜果數器而已。
 及卒,家無餘貲,特賜黃金二百兩。其後端午贈講讀官,御飛白書扇,遣使特賜置其柩。



 周湛,字文淵,鄧州穰人。進士甲科,為開州推官。中身言書判,改秘書省著作佐郎、通判戎州。俗不知醫,病者以祈禳巫祝為事,湛取古方書刻石教之,禁為巫者,自是人始用醫藥。累遷尚書都官員外郎、知虔州,提點廣南東路刑獄。



 初,江、湖民略良人,鬻嶺外為奴婢。湛至,設方略搜捕,又聽其自陳,得男女二千六百人,給飲食還其
 家。徙京西路,鄧州美陽堰歲役工數十萬,溉州縣職田,而利不及民,湛奏罷之。為鹽鐵判官,三司帳籍浩煩,吏胥離析為弊欺。湛為立勘同法,歲減天下計帳七千。為江南西路轉運使,州縣簿領案牘,淆混無紀次,且多亡失,民訴訟無所質,至久不能決。湛為立號,以月日比次之,詔下其法諸路。又以徭賦不均,百姓巧於避匿,因條其詭名挾佃之類十二事,且許民自言,凡括隱戶三十萬。



 還為戶部判官,又為夔州路轉運使。雲安鹽井歲賦
 民薪茅,至破產責不已,湛為蠲鹽課而省輸薪茅。判鹽鐵勾院,以太常少卿直昭文館,為江、淮制置發運使。陛辭,仁宗誡以毋納包苴於京師。湛惶恐對曰:「臣蒙聖訓,不敢茍附權要,以謀進身。」湛治煩劇,能得其要,所至喜條上利害,前後至數十百事。天資強記,吏胥滿前,一見輒識其姓名。大江歷舒州長風沙,其地最險,謂之石牌灣,湛役三十萬工,鑿河十里以避之,人以為利。



 除度支副使。舊制,發運司保任軍將至三司,不得考覆而皆遷
 之。至是,以名上者三十五人,湛盡覆其濫者。拜右諫議大夫。使契丹,辭不行。



 知襄州,襄人不善陶瓦,率為竹屋,歲久侵據官道,簷廡相逼,火數為害。湛至,度其所侵,悉毀徹之,自是無火患。然豪姓不便,提點刑獄李穆奏湛擾人,徙知相州。右司諫吳及疏曰:「湛裁損居民第,為官也;百姓侵官而主司禁之,其職然也。況聞湛明著律令,約民以信,乃奉法行事,百姓自知罪不敢訴。郡從事高直溫,夏竦子婿也。竦邸店最廣,故加譖於穆,且謂湛伐
 木若干株。昔之民居侵越官道,木在道側,既正其侵地,則木在中衢,固宜翦去。又湛種楸桐千餘本,課戶貯水,以嚴火禁。又於民居得眾汲舊井四,廢而復興,人得其利。道傍之井,反在民居之下,其侵越豈不白乎?望詔執政大臣辨正湛、穆是非,明垂獎黜。若謂湛已行之命,憚於追改,是傷風敗俗,貽患於後,不若追改之愈也。湛守大郡,于湛不為重輕,但國家舉錯有所未安,奉職者將何以勸邪?」未幾卒。湛為人脫易,少威儀,然善射弩,雖隔
 屋亦中的雲。



 徐的,字公準,建州建安人。擢進士第,補欽州軍卅推官。欽土煩鬱,人多死瘴癘。的見轉運使鄭天監,請曰:「徙州瀕水可無患,請轉而上聞。」從之,天監因奏留的使辦役。的短衣持梃,與役夫同勞苦,築城郭,立樓櫓,以備戰守。畫地居軍民,為府舍、倉庫、溝渠、厘肆之類,民皆便之。



 遷大理寺丞、知吳縣,移梁山軍,通判常州。屬歲饑,出米為糜粥以食餓者。累遷尚書屯田員外郎、知臨江軍,擢廣
 南西路提點刑獄。安化州蠻攻殺將吏,所部卒畏誅,謀欲叛。的馳至宜州,慰曉之曰:「爾曹亡懼,能出力討賊,猶可立功以自贖。若朝叛則夕死。非計也。」眾皆斂手聽命。奏復澄海、忠敢軍,後皆獲其用。改知舒州,徙荊湖北路轉運使。辰州蠻彭士義為寇,的開示恩信,蠻黨悔過自歸。



 攝江陵府事,城中多惡少年,欲為盜,輒夜縱火,火一夜十數發。的籍其惡少年姓名,使相保任,曰:「爾輩遞相察,不然,皆爾罪也。」火遂息。太子洗馬歐陽景猾橫不
 法,為里人害,的發其奸,竄之嶺外。以兵部員外郎為淮南、江、浙、荊湖制置發運副使。奏通泰州海安、如皋縣漕河,詔未下,的以便宜調兵夫浚治之,出滯鹽三百萬,計得錢八百萬緡。遂為制置發運使。



 軍賊王倫起山東,轉掠淮南,的團兵待之。會青州改遣裨將傅永吉追殺人歷陽,的與賞,遷工部郎中。復治泰州西溪河,發積鹽,加直昭文館。區希範、蒙趕寇衡湘,命的招撫之。既至,再宿,會蠻酋相繼出降。三司以郊祠近,宜召還計事,既還,蠻
 復叛。除度支副使、荊湖南路安撫使,至桂陽,降者復眾。其欽景、石礙、華陰、水頭諸洞不降者,的皆討平之,斬其酋熊可清等千餘級。卒於桂陽。



 論曰:宋承平時,書生知兵者蓋寡,偕、沿數上書言邊事,策畫論議,有得有失,固皆一時之俊。畋由將家子力學第進士,再討徭賊,前勝後敗,兵家之常也。杞、的俱以征宜州蠻立功,杞則殺降失信,的則招徠以恩,其優劣概可見矣。湛強敏,所至有治績,史稱善射,抑亦文臣之習
 武事者歟。鼎性孝友,自奉甚約,而疏於財,居官清辨,土俗有生子不舉者輒禁之,獨發摘吏奸貽眾怒,或以「虎」目之,豈其然乎?



 姚仲孫,字茂宗,本曹南著姓,曾祖仁嗣,陳州商水令,因家焉。父曄,舉進士第一,官至著作佐郎。仲孫早孤,事母孝。擢進士第,補許州司理參軍。民婦馬氏夫被殺,指裏胥嘗有求而其夫不應,以為里胥殺之,官捕系辭服。仲孫疑其枉,知州王嗣宗怒曰:「若敢以身任之耶?」仲孫曰:「
 幸毋遽決,冀得徐辨。」後兩月,果得殺人者。



 調邢州推官,徙資州。轉運使檄仲孫詣富順監按疑獄,全活數十人。資州更二守,皆惛老,事多決於仲孫。改大理寺丞、知建昌縣。初,建昌運茶抵南康,或露積於道,間為霖潦所敗,主吏至破產不能償。仲孫為券,吏民輸山木,即高阜為倉,邑人利之。徙通判彭州。嘗以天下久無事,不可以弛兵備,因上前世御戎料敵之策,名《防邊龜鑒》。通判睦州,徙滁州。歲旱饑,有詔發官粟以賑民,而主吏不時給。仲
 孫既至州,立劾主吏,夜索丁籍盡給之。累遷尚書屯田員外郎。



 王鬷守益州,闢通判州事。召為右司諫。入內都知閻文應求為都知,仲孫數其罪,白上曰:「方帝齋宿太廟,而文應叱醫官,聲聞行在。郭皇后暴薨,中外莫不疑文應置毒者。」出文應為泰州兵馬鈐轄,又稱疾留,復論奏,乃亟去。



 以起居舍人知諫院,管勾國子監,以尚書戶部員外郎兼侍御史知雜事。時諫議大夫十二員,仲孫曰:「諫議大夫蓋朝廷之選,不宜以歲月序進。今諸寺卿
 至前行郎中三十五員,貼近職者猶不在數,若以年勞授,則數年之外,諫議大夫員益多。請艱其選,以處材望之臣,餘悉次補卿監。」乃詔當選者奏聽旨。先是,諸路復提點刑獄,還朝多擢為省府官。仲孫請第其課為三等升黜之,即詔仲孫司考課之法。



 歷三司戶部、度支、鹽鐵副使,進天章閣待制、河北都轉運使。大修城壘兵備,仁宗賜詔褒之。權知澶州,河壞明公埽,絕浮橋,仲孫親總役堤上,埽一夕復完。權知大名府,夜領禁兵塞金堤決
 河。是歲,澶、魏雖大水,民不及患。進禮部郎中、龍圖閣學士,徙陜西都轉運使,未行,權三司使事。屬西北備邊,募兵益屯及賞賜、聘問之費,不可勝計。仲孫悉心經度,雖病,未嘗輒廢事。坐小吏詐為文符,出知蔡州。因母憂喪一目,卒。



 陳太素,字仲華,河南緱氏人。中進士第。嘗為大理詳斷官,入審刑為詳議官,權大理少卿,又判大理事。任刑法二十餘年,朝廷有大獄疑,必召與議。太素為推原人情,
 以傅法意,眾皆釋然,自以為不及。雖號明習法令,然所論建,亦或有不中。每臨案牘,至忘寢食,大寒暑不變。子弟或止之,答曰:「囹圄之苦,豈不甚於我也。」歷知江陰軍、兗州、明州,有治跡。在大理,耳疾,數求罷,執政以為任職,弗許。累官至尚書兵部郎中,卒。



 太素家行修治,尤喜論刑名。常以為有司議法,當據文直斷,不可求曲當法;求典當法,所以亂也。



 同時有馬尋者,須城人。舉《毛詩》學究,累判大理寺,以明習法律稱。歷提點兩浙陜西刑獄、廣
 東淮南兩浙轉運使,知湖、撫、汝、襄、洪、宣、鄧、滑八州。襄州饑,人或群入富家掠囷粟,獄吏鞫以強盜,尋曰:「此脫死爾,其情與強盜異。」奏得減死,論著為例。終司農卿。



 又有杜曾者,濮州人。為吏號知法,嘗言:「國朝因唐大中制,故殺,人雖已傷未死、已死更生,皆論如已殺。夫殺人者死,傷人者刑,先王不易之典。律雖謀殺已傷則絞,蓋甚其處心積慮,陰致賊害爾。至於故殺,初無殺意,須其已死,乃有殺名;茍無殺名而用殺法,則與謀殺孰辨?自大中
 之制行,不知殺幾何人矣。請格勿用。」又言:「近世赦令,殺人已傷未死者,皆得原減,非律意。請傷者從律保辜法,死限內者論如已殺,勿赦。」皆著為令。



 李虛己,字公受,五世祖盈,自光州從王潮徙閩,遂家建安。父寅,有清節,仕江南李氏,至諸司使。江南國除,授殿前承旨,辭不拜。時偽官皆入留京師,而寅母獨在江南,乃遣其長子歸養。舉進士,起家為衢州司理參軍。母老,棄官以歸。虛己亦中進士第,歷沈丘縣尉,知城固縣,改
 大理評事,累遷殿中丞,提舉淮南茶場。召知榮州,未行,改遂州。



 時太宗勵精政事,嘗手書累二十餘紙,曰:「公勤潔己、奉法除奸、惠愛臨民者,乃可書為勞績,月給奉以實錢。」命有司擇群臣以治最聞者賜之,仍諭曰:「除奸之要,在乎奉法,不可因以生事。」時虛己被賜,因獻詩自陳父子遭遇,榮及祖母。帝悅,為批其紙尾曰:「虛己學古入官,榮親事生,奉書為郡,欲布親規,朕得良二千石矣。」遂賜五品服,又賜其祖母錢五十萬,命翰林學士張洎會
 兩制、三館儒臣遍閱所批詔。其後以南郊恩封群臣母妻,虛己又請罷其妻封以授祖母,詔悉封之,世以為榮。



 會遣使察川峽吏能否,而州多不治,唯虛己與薛顏、邵曄、查道數人,以能任職稱。再遷尚書屯田員外郎。以便親,請通判洪州。是時寅已謝歸,春秋高,寅母尚無恙,虛己雙舉迎侍。寅至豫章,樂其山水,曰:「此可以終吾身也。」遂臨州之東湖,築第宇以居。虛己為侍御史,出提點荊湖南路刑獄,徙淮南轉運副使,累遷兵部郎中,為龍圖
 閣待制,歷判大理寺。久之,求補外,真宗稱其儒雅循謹,特遷右諫議大夫。數月,出知河中府。召權御史中丞。未幾,以疾辭,進給事中、知洪州。遷尚書工部侍郎,徙池州。求分司南京,卒。初,寅之請老,年未六十。虛己分司而歸,年六十九。其季虛舟仕至餘干縣令,坐法免官,不復言仕。



 初,太宗既賜虛己錢,翌日,以語宰相曰:「虛己詩思可嘉,予錢五十緡矣。」宰相對以所予乃五十萬,帝知其誤,由是詔群臣以章獻者閣門勿受,皆由中書門下閱而
 上之。然論者謂虛己父子篤行,家甚貧,雖人主一時之誤,殆天賜也。寅事親孝,治家有法,閨門之內肅如也。虛己、虛舟又以孝友清慎世其家。虛舟之子寬,為尚書金部郎中;定,為司農少卿,為吏頗有能名。



 虛己喜為詩,數與同年進士曾致堯及其婿晏殊唱和。初,致堯謂曰:「子之詞詩雖工,而音韻猶啞。」虛己未悟。後得沈休文所謂「前有浮聲,則後須切響」,遂精於格律。有《雅正集》十卷。



 張傅,字巖卿,唐初功臣公謹之裔。祖播,為亳州團練副
 使,子孫因為譙人。傅進士及第,稍遷秘書省著作佐郎、知奉符縣。時方修會真宮、天書觀及增治岳祠,以辦事稱,賜錢二十萬。宰相向敏中冊東岳帝號還,薦之,知楚州。會歲饑,貽書發運使求貸糧,不報。因嘆曰:「民轉死溝壑矣,報可待邪?」乃發上供倉粟賑貸,所活以萬計,因拜章待罪,詔獎之。



 提點江西刑獄,徙江東,就除轉運使,入權三司鹽鐵判官。會河決濟北,民多被害,命安撫京東。累遷工部郎中,出為兩浙轉運使,改荊湖北路,復為鹽
 鐵判官,再遷兵部,為陜西轉運使,徙江、淮發運使,未至,召還。屬西京奏兵食乏,因言馮翊、華陰積粟多,可運二十萬石,繇三門下濟之。遂留為侍御史知雜事,判吏部流內銓,進三司度支副使。以疾請外,遷太常少卿、知應天府。逾月,為右諫議大夫,徙青州,遷給事中、知鄆州,復知應天府,遂以工部侍郎致仕,卒。



 傅強力治事,七為監司,所至審核簿書,勾擿奸隱,州縣憚之。傅曰:「奚為我憚哉。吾所以事事致察者,正所以愛州縣也。吏不敢慢,則
 州縣不復犯法矣。」人亦以為然。天禧中,有術士自言數百歲,少時嘗游秦悼王家,歷見唐肅宗、代宗朝,由是出入禁中,見尊重,人無敢詰其偽。傅見之,訊以唐事,術士語屈。



 俞獻卿,字諫臣,歙人。少與兄獻可以文學知名,皆中進士第。獻可有吏稱,歷吏部郎中、龍圖閣待制。獻卿起家補安豐縣尉。有僧貴寧,積財甚厚,其徒殺之,詣縣紿言師出游矣。獻卿曰:「吾與寧善,不告而去,豈有異乎?」其徒
 色動,因執之,得其所瘞尸,一縣大驚。再調昭州軍事推官,會宜州陳進亂,像州守不任事,轉運使檄獻卿往佐之。及至,守謀棄城,獻卿曰:「臨難茍免,可乎?賊至,尚當力擊;不勝,有死而已,奈何棄去。」初,昭州積緡錢鉅萬,獻卿盡用平糴,至積穀數萬,及是大兵至,賴以饋軍。改大理寺寺丞,為本寺詳斷官。歷知慎、仁和二縣,再遷太常博士、知南雄州,徙潮州。



 除殿中侍御史,為三司鹽鐵判官。上言:「天下穀帛日益耗,物價日益高,欲民力之不屈,不
 可得也。今天下穀帛之直,比祥符初增數倍矣。人皆謂稻苗未立而和糴,桑葉未吐而和買。自荊湖、江、淮間,民愁無聊,轉運使務刻剝以增其數,歲益一歲。又非時調率營造一切費用,皆出於民,是以物價積高,而民力積困也。陛下誠以景德中西、北二邊通好最盛之時一歲之用較之,天禧五年,凡官吏之要冗,財用之盈縮,力役之多寡,賊盜之增減,較然可知其利害也。況自天禧以來,日侈一日,又甚於前。夫卮不盈者漏在下,木不茂者
 蠹在內。陛下宜知其有損於彼,無益於此,與公卿大臣,朝夕圖議而救正之。」帝納其言,為罷諸宮觀兵衛,又命官除無名之費以鉅萬計。



 淮、浙鹽利不登,命獻卿往經度之,更立新法,歲增鹽課緡錢甚眾。會其兄為鹽鐵副使,徙開封府判官。朝廷擇陜西轉運使,宰相連進數人,不稱旨。他日,獻卿在所擬中。帝曰:「此可以除陜西轉運使。」時邊吏多因事邀功,涇原路鈐轄擅於武延川鑿邊壕、置堡砦,獻卿度必招寇患,亟檄罷之。未幾,賊果至,殺
 將士,塞所鑿壕而去。徙京西。因入對,甚言趙振堪將帥,範仲淹、明鎬可大用,及條上邊策甚備。



 除福建轉運使,還判三司鹽鐵勾院,累遷尚書刑部郎中、直史館、知荊南,歷戶部、度支、鹽鐵副使,以右諫議大夫、集賢院學士知杭州。暴風,江潮溢決堤,獻卿大發卒鑿西山,作堤數十里,民以為便。還,勾當三班院,知通進、銀臺司,最後知應天府,以刑部侍郎致仕,卒。



 陳從易,字簡夫,泉州晉江人。進士及第,為嵐州團練推
 官,再調彭州軍事推官。王均盜據成都,連陷綿、漢諸郡,彭人謀殺兵馬都監以應之。時從易攝州事,斬其首謀者,召餘黨曉以禍福,貰之,眾皆呼悅。乃率厲將吏,修嚴守械,戒其家僮積薪舍後,曰:「吾力不足以守,當死於此。」賊聞其有備,不敢入境。賊平,安撫使王欽若以狀聞,召為秘書省著作佐郎、大理寺詳斷官。遷太常博士,出知邵武軍。預修《冊府元龜》,改監察御史。真宗宴近臣崇和殿,召從易預,賦詩稱旨。遷侍御史,改刑部員外郎、直史
 館、知虔州。會歲大饑,有持杖盜取民穀者,請一切減死論,凡生者千餘人。



 天禧中,坐薦送別頭進士失實,降工部員外郎。以父老,求鄉郡。宰相寇準惡其疏己,除吉州,從易因對自言改福州。未行,遭父喪,服除,糾察在京刑獄,出為湖南轉運使,徙知荊南,擢太常少卿、直昭文館、知廣州。又坐嘗課校太清樓書字非偽誤而從易妄判竄之,降直史館。明年復職。在廣三年,以清德聞。入為左司郎中、知制誥。



 初,景德後,文士以雕靡相尚,一時學者
 鄉之,而從易獨守不變。與楊大雅相厚善,皆好古篤行,時朝廷矯文章之弊,故並進二人,以風天下。兼史館修撰,遷左諫議大夫。命使契丹,以年老,辭不行。又辭職請補郡,進龍圖閣直學士、知杭州,卒。



 從易好學強記,為人激直少容,喜別白是非,多面折人,或尤其過,從易終不變。王欽若最善之,嘗謂人曰:「數日不見簡夫,輒忽忽不懌。」及廢居南京,時丁謂方用事,人畏謂,無敢往見欽若者。從易將使湖南,欲過之,遇汴水旱涸,遂告謂曰:「從易
 願使湖外者,非獨為貧也,亦以王公在宋,故就省之爾。今汴涸,義不可從他道進,幸公許少留。」謂即大喜曰:「王公之門,獨君為知我者。」留權糾察刑獄,從易不敢當,乃聽歸館,須汴通乃行。時寇準貶道州,謂又謂從易曰:「廬陵之事,可以釋憾矣。」從易對曰:「當以故相事之爾。」謂有愧色。其行志多類此。所著《泉山集》二十卷,《中書制稿》五卷,《西清奏議》三卷。



 楊大雅,字子正,唐靖恭諸楊虞卿之後。虞卿孫承休,唐
 天祐初,以尚書刑部員外郎為吳越國冊禮副使,楊行密據江、淮,道阻不克歸,遂家錢塘。大雅,承休四世孫也。錢俶歸朝,挈其族寓宋州。大雅素好學,日誦數萬言,雖飲食不釋卷。進士及第,歷新息、鄢陵縣主簿,改光祿寺丞、知新昌縣,徙知潯州,監在京商稅,再遷秘書丞。



 咸平中,交趾獻犀,因奏賦,召試,遷太常博士。久之,又上書自薦,獻所為文,復召試。直集賢院,出知筠、袁二州,提舉開封府界諸縣鎮事,為三司監鐵判官,知越州,提點淮南
 路刑獄。還,考試國子監生,坐失薦,迭降監陳州酒。徙知常州,判三司都磨勘司、戶部勾院。遷集賢殿修撰、知應天府。還,糾察在京刑獄,以兵部郎中知制誥。大雅初名侃,至是,避真宗藩邸諱,詔改之。居二歲,拜右諫議大夫、集賢院學士、知亳州,卒。



 大雅樸學自信,無所阿附,直集賢院二十五年不遷,有出其後者,往往致榮顯。或笑其違世自守,大雅嘆曰:「吾不學乎世,而學乎聖人,由是以至此。吾之所有,不敢以薦於人,而嘗自獻乎天子矣。」天
 禧中,使淮南,循江按部,過金陵境上,遇風覆舟,得傍卒拯之,及岸,冠服盡喪。時丁謂鎮金陵,遣人遺衣一襲,大雅辭不受,謂以為歉。宰相王欽若亦不悅之。晚與陳從易並命知制誥。大雅嘗因轉對,上《原治》十七篇。所著《大隱集》三十卷,《西垣集》五卷,《職林》二十卷,《兩漢博聞》十二卷。



 論曰:仲孫以才力自奮於時,論事著效,號為能吏。太素、尋、曾能知法意,理官之良也。虛己、獻卿立朝雖徽,卓犖
 大節,及為他官,所至有吏稱。若從易拒釋憾之言,大雅辭襲衣之遺,卒使權奸愧歉,抑又可尚哉



\end{pinyinscope}