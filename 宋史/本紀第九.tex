\article{本紀第九}

\begin{pinyinscope}

 仁
 宗一



 仁宗體天法道極功全德神文聖武睿哲明孝皇帝,諱禎,初名受益,真宗第六子,母李宸妃也。大中祥符三年四月十四日生。章獻皇后無子,取為己子養之。天性仁
 孝寬裕,喜慍不形於色。七年,封慶國公。八年,封壽春郡王,講學於資善堂。天禧元年,兼中書令。明年,進封升王。九月丁卯,冊為皇太子,以參知政事李迪兼太子賓客。癸酉,謁太廟。四年,詔五日一開資善堂,太子秉笏南鄉立,聽輔臣參決諸司事。乾興元年二月戊午,真宗崩,遺詔太子即皇帝位,尊皇后為皇太后,權處分軍國事。遣使告哀契丹。己未,大赦,除常赦所不原者。百官進官一等,優賞諸軍。山陵諸費,毋賦於民。庚申,命丁謂為山陵
 使。出遺留物賜近臣、宗室、主兵官。甲子,聽政於崇政殿西廡。乙丑,以生日為乾元節。丙寅,遣使以先帝遺留物遺契丹。進封涇王元儼為定王,賜贊拜不名。以丁謂為司徒兼侍中、尚書左僕射,馮拯為司空兼侍中、樞密使、尚書右僕射,曹利用為尚書左僕射兼侍中。戊辰,貶道州司馬寇準為雷州司戶參軍,尚書戶部侍郎李迪為衡州團練副使,宣徽南院使曹瑋為左衛大將軍。



 三月乙酉,作受命寶。庚寅,初御崇德殿,太後設幄次於承明
 殿,垂簾以見輔臣。



 夏四月壬子,遣使以即位告契丹。丙寅,交州來貢。



 五月乙亥,錄系囚,雜犯死罪遞降一等,杖以下釋之。



 六月己酉,命參知政事王曾按視山陵皇堂。丁巳,契丹使來祭奠吊慰。庚申,入內內侍省押班雷允恭坐擅移皇堂伏誅。丁謂罷為太子少保、分司西京。甲子,改命馮拯為山陵使。丙寅,降參知政事任中正為太子賓客。



 秋七月辛未,馮拯加昭文館大學士,王曾為中書侍郎、同中書門下平章事、集賢殿大學士,呂夷簡、魯
 宗道參知政事。乙亥,遣使報謝契丹。丙子,樞密副使錢惟演為樞密使。戊寅,改翼祖定陵為靖陵。辛卯,貶丁謂為崖州司戶參軍。



 八月壬寅,遣使賀契丹主及其妻生日、正旦。乙巳,皇太后同御承明殿垂簾決事。



 九月壬申,告大行皇帝謚於天地、宗廟、社稷。癸酉,上謚冊於延慶殿。己卯,命以天書從葬。



 冬十月壬寅,契丹使來賀即位。己酉,葬真宗皇帝於永定陵。詔中外避皇太后父諱。己未,祔真宗神主於太廟,廟樂曰《大明之舞》,以莊穆皇后
 配。辛酉,降東、西京囚罪一等,杖以下釋之。蠲山陵役戶及靈駕所過民田租。



 十一月丁卯朔,錢惟演罷。甲戌,唃廝囉、立遵求內附。乙亥,以皇太后生日為長寧節。辛巳,初御崇政殿西閣講筵,命侍講孫奭、馮元講《論語》。壬午,以張知白為樞密副使。十二月壬戌,契丹使來賀明年正旦。是歲,蘇州水,滄州海潮溢,詔振恤被水及溺死之家。南平王李公蘊遣使進貢。



 天聖元年春正月丙寅朔,改元。庚午,契丹使初來賀長
 寧節。癸未,命三司節浮費,遂立計置司。戊子,以京東、淮南水災,遣使安撫。辛卯,發卒增築京城。



 二月戊戌,許唃廝囉歲一入貢。丁巳,奉安太祖、太宗御容於南京鴻慶宮。壬戌,減諸節齋醮道場。



 三月甲戌,奉安真宗御容於西京應天院。丙子,詔減西京囚罪一等,徒以下釋之。賜城中民八十以上者茶帛,仍復其家。甲申,詔自今營造,三司度實給用。辛卯,司天監上《崇天歷》。行淮南十三山場貼射茶法。



 夏四月辛丑,罷禮儀院。丁未,乾元節,百官
 及契丹使初上壽於崇德殿。癸丑,詔文武官奏蔭親屬從本資。丁巳,詔近臣舉諫官、御史各一人。



 五月甲子,行陜西、河北入中芻糧見錢法。庚午,詔禮部貢舉。辛未,錄系囚。甲戌,命魯宗道按視滑州決河。庚寅,議皇太后儀衛制同乘輿。



 六月甲辰,罷江寧府溧水縣採丹砂。乙卯,禁毀錢鑄鐘。



 秋七月壬申,除戎、瀘州虛估稅錢。詔職田遇水旱蠲租如例。辛巳,蠲天下逋負。



 八月乙未,募民輸芟塞滑州決河。丙申,下德音,減天下囚罪一等,杖以下
 釋之。廢鄆州東平馬監,以牧地賦民。甲寅,芝生天安殿柱。



 九月丙寅,馮拯罷,以王欽若為門下侍郎、同中書門下平章事、昭文館大學士。辛巳,詔凡舉官未改轉而坐贓者,舉主免劾。庚寅,宴崇德殿。



 閏月甲午,詔裁造院女工及營婦配南北作坊者,並釋之。戊戌,寇準卒於雷州。己亥,馮拯卒。丁未,禁彭州九隴縣採金。丁巳,禁伎術官求輔臣、宗室薦舉。



 冬十月辛酉朔,徙陜西緣邊軍馬屯內地。



 十一月丁酉,詔諸州配囚,錄具獄與地里,上尚書
 刑部詳覆。禁兩浙、江南、荊湖、福建、廣南路巫覡挾邪術害人者。戊午,置益州交子務。是歲,甘、沙州來貢,涇原咩迷卞杏家族納質內附。



 二年春二月庚午,遣內臣收瘞汴口流尸,仍祭奠之。



 三月丁酉,奉安真宗御容於景靈宮奉真殿。癸卯,王欽若上《真宗實錄》。是月,賜禮部奏名進士、諸秋及第出身四百八十五人。



 夏四月辛酉,詔三司歲市紬、絹非土產者罷之。乙酉,錄晉石氏後。



 五月乙未,錄系囚。



 六月壬申。罷
 天慶、天祺、天貺、先天、降聖節宮觀然燈。



 秋七月癸丑,奉安真宗御容於玉清昭應宮安聖殿。



 八月丙辰朔,宴崇德殿,初用樂之半。詔舉官己遷改而貪污者,舉主以狀聞,聞而不以實者坐之。己卯,幸國子監,謁孔子,遂幸武成王廟。甲申,太白入太微垣。



 九月辛卯,祠太一宮,賜道左耕者茶帛。



 冬十月丙辰,奉安真宗御容於洪福院。



 十一月甲午,加上真宗謚。乙未,朝饗玉清昭應、景靈宮。丙申,饗太廟。丁酉,祀天地於圜丘,大赦。百官上尊號曰聖
 文睿武仁明孝德皇帝,上皇太后尊號曰應元崇德仁壽慈聖皇太后。賜百官諸軍加等。乙巳,立皇后郭氏。辛亥,加恩百官。十二月庚午,詔開封府每歲正旦、冬至禁刑三日。是歲,龜茲、甘肅來貢。



 三年春正月辛卯,長寧節,近臣及契丹使初上皇太后壽於崇政殿。



 二月戊寅,詔陜西災傷州軍,盜廩穀非傷主者,刺配鄰州牢城,徒減一等。



 夏四月丁丑,詔三館繕書藏太清樓。



 五月庚寅,錄系囚。癸巳,幸御莊觀刈麥,聞民
 舍機杼聲,賜織婦茶帛。己亥,賜隱士林逋粟帛。己酉,禁臣僚奏薦無服子弟。



 六月壬戌,太白晝見。癸酉,環、原州屬羌叛寇邊,環慶都監趙士隆等死之,遣使者安撫陜西。



 秋七月戊子,詔諸路轉運使察舉知州、通判不任事者。丙午,詔邊戶為羌所擾者蠲租,復役二年。



 八月戊午,以忠州鹽井歲增課、夔州奉節巫山縣舊藉民為營田、萬州戶有稅者歲糴其穀,皆為民害,詔悉除之。辛未,蠲陜西州軍旱災租賦。



 九月乙巳,詔司天監奏災異據占
 書以聞。



 冬十月乙卯,太白犯南斗。辛酉,晏殊為樞密副使。



 十一月己卯朔,罷貼射茶法。辛卯,以襄州水蠲民租。晉、絳、陜、解州饑,發粟振之。戊申,王欽若卒。十二月癸丑,王曾為門下侍郎、昭文館大學士,張知白同中書門下平章事、集賢殿大學士。乙丑,張旻為樞密使。戊寅,太白晝見。是歲,龜茲、甘州、于闐來貢。環、慶蕃部嵬逋等內附。補涇原降羌首領潘徵為本族軍主。



 四年春正月己亥,命章得像與流內銓同試百司人。庚
 子,涇原兵破康奴族。



 二月甲寅,詔吏犯贓至流,按察官失舉者,並劾之。庚午,置西界和市場。



 三月甲申,詔轉運使、提點刑獄罷勸農司。己亥,鄜延蕃部首領曹守貴等內附。



 夏四月壬子,詔京東西、河北、淮南平谷價。



 五月己卯,詔禮部貢舉。壬午,詔大闢疑者奏讞,有司毋輒舉駁。戊子,錄系囚。己亥,詔士有文而行不副者,州郡毋得薦送。



 閏月戊申,減江、淮歲漕米五十萬石。除舒州太湖等九茶場民逋錢十三萬緡。己酉,詔補太廟室長、齋郎。辛
 亥,復陜西永豐渠以通解鹽。



 六月丁亥,建、劍、邵武等州軍大水,詔賜被災家米二石,溺死者官瘞之。庚寅,大雨震電,京師平地水數尺。辛卯,避正殿,減常膳。丁酉,降天下囚罪一等,徒以下釋之。畿內、京東西、淮南、河北被水民田蠲其租。癸卯,詔官物漂失,主典免償。流徙者,所在撫存之。秋七月戊申,禦長春殿,復常膳。辛未,減兩川歲輸錦綺,易綾紗為絹,以給邊費。壬申,詔諸路轉運使舉所部官通經術者。



 八月丁亥,築泰州捍海堰。己丑,詔施
 州溪峒首領三年一至京師。



 九月乙卯,詔孫奭、馮元舉京朝官通經術者。庚申,詔禮部貢院:諸科通三經者薦擢之。錄周世宗從孫柴元亨為三班奉職。辛未,廢襄、唐州營田務,以田賦民。



 冬十月甲戌朔,日有食之。壬辰,詔郎中以上致仕,賜一子官。甲午,昏霧四塞。丙申,奉安真宗御容於鴻慶宮。



 十二月丁丑,發米六十萬斛貸畿內饑。丁亥,帝白太后,欲元日先上太后壽乃受朝,太后不可。王曾奏曰:「陛下以孝奉母儀,太后以謙全國體,請如
 太后令。」



 五年春正月壬寅朔,初率百官上皇太后壽於會慶殿,遂御天安殿受朝。己未,晏殊罷。戊辰,以夏竦為樞密副使。



 二月癸酉,命呂夷簡、夏竦修先朝國史,王曾提舉。丙子,詔振京東流民。丁丑,西域僧法吉祥等來獻梵書。



 三月戊申,賜禮部奏名進士、諸科及第出身一千七十六人。秦州地震。罷瓊州歲貢玳瑁、龜皮、紫貝。



 夏四月壬辰,壽寧觀火。



 五月庚子朔,詔武臣子孫習文藝者,聽奏文
 資。壬寅,太白晝見。丙午,閱諸班騎射。辛亥,錄系囚。辛酉,命呂夷簡等詳定編敕。癸亥,楚王元佐薨。是月,京畿旱,磁州蟲食桑。六月甲戌,祈雨於玉清昭應宮、開寶寺。丙子,詔決畿內系囚。丁丑,雨。癸未,罷諸營造之不急者。



 秋七月己亥朔,振秦州水災,賜被溺家錢米。丙辰,發丁夫三萬八千、卒二萬一千、緡錢五十萬塞滑州決河。詔察京東被災縣吏不職者以聞。



 九月庚戌,閱龍衛神勇軍習戰。



 冬十月辛未,罷陜西青苗錢。癸酉,奉安真宗御容
 於慈孝寺崇真殿。己丑,頒新定《五服敕》。甲午,同皇太后幸御書院,觀太宗、真宗御書。乙未,詔西川、廣南在官物故者,遣人護送其家屬還鄉,官為給食。丙申,滑州言河平。



 十一月丁酉朔,以陜西旱蝗,減其民租賦。庚子,遣使河北體量安撫。壬寅,復作指南車。辛亥,朝饗景靈宮。壬子,饗太廟。癸丑,祀天地於圜丘,大赦。賀皇太后於會慶殿。丁巳,恭謝玉清昭應宮。十二月辛未,加恩百官。甲戌,詔輔臣南郊恩例外,更改一子官。丁亥,詔百官宗室受
 賂、冒為親屬奏官者毋赦。是歲,甘州及南平國王李公蘊遣人來貢。京兆府、邢,洺州蝗。華州旱,孑□方蟲食苗。



 六年春正月己酉,罷兩川乾元節歲貢織佛。戊午,罷提點刑獄。



 二月庚辰,大風,晝晦。壬午,張知白薨。



 三月丙申朔,日有食之。壬子,以張士遜同中書門下平章事、集賢殿大學士。癸丑,以姜遵為樞密副使。己未,以範雍為樞密副使。壬戌,作西太一宮。



 夏四月戊辰,詔審官、三班院、吏部流內銓、軍頭司各引對所理公事。自帝為皇太子,
 輔臣參決諸司事於資善堂,至是始還有司。丁丑,貸河北流民復業者種食,復是年租賦。癸未,命官減三司歲調上供物。甲申旦,有星大如斗,自北流至西南,光燭地,有聲如雷。庚寅,下德音,以星變齋居,不視事五日。降畿內囚死罪,流以下釋之。罷諸土木工。振河北流民過京師者。



 五月乙未朔,交址寇邊。



 六月丙寅,罷戎、瀘諸州穀稅錢。



 秋七月壬子,江寧府、揚、真、潤州江水溢,壞官民廬舍,遣使安撫振恤。



 八月乙丑,詔免河北水災州軍秋稅。
 乙亥,河決澶州王楚埽。丙戌,錄唐張九齡後。



 九月己亥,詔京朝官任內,五人同罪奏舉者,減一任。癸卯,祠西太一宮。甲辰,詔河北災傷,民質桑土與人者悉歸之,候歲豐償所貸。乙巳,遣使修諸路兵械。



 冬十月甲申,除福州民逋官莊錢十二萬八千緡。



 十一月戊午,京西言穀斗十錢。十二月癸亥,祠西太一宮。是歲,甘州、三佛齊來貢。



 七年春正月癸卯,曹利用罷。丙辰,降利用為左千牛衛
 上將軍。



 二月庚申朔,魯宗道卒。甲子,詔文臣歷邊有材勇、武臣之子有節義者,與換官,三路任使。丙寅,張士遜罷,以呂夷簡同中書門下平章事、集賢殿大學士。丁卯,以夏竦、薛奎參知政事,陳堯佐為樞密副使。癸酉,貶曹利用為崇信軍節度副使、房州安置,未至,自殺。乙酉,以河北水災,委轉運使察官吏,不任職者易之。



 閏月癸巳,募民入粟以振河北。戊申,禁京城創造寺觀。壬子,復制舉六科,增高蹈丘園、沉淪草澤、茂才異等科,置書判拔
 萃科及試武舉。癸酉,置理檢使,以御史中丞為之。



 三月乙丑,詔吏胥受賕毋用蔭。辛巳,詔契丹饑民所過給米,分送唐、鄧等州,以閑田處之。癸未,詔百官轉對,極言時政闕失,在外者實封以聞。



 夏四月庚寅,赦天下,免河北被水民租賦。辛卯,南平王李公蘊卒,其子德政遣人來告,以為交址郡王。



 五月乙未朔,詔禮部貢舉。庚申,詔戒文弊。己巳,頒新令。庚午,詔先朝文武官自刺史、少卿、監以上,並錄其後。癸酉,錄系囚。庚辰,御承明殿,臣僚請對
 者十九人,日昃乃罷。



 六月壬辰,置益、梓、廣南路轉運判官。丁未,大雷雨,玉清昭應宮災。甲寅,王曾罷。



 秋七月癸亥,以玉清昭應宮災,遣官告諸陵,詔天下不復繕修。乙亥,詔殿直以上毋得換文資。乙酉,罷諸宮觀使。



 八月丁亥朔,日有食之。詔罷天下職田,官收其入,以所直均給之。己丑,以呂夷簡為昭文館大學士。辛卯,夏竦復為樞密副使,陳堯佐、王曙並參知政事。己亥,詔命官犯正入贓,毋使親民。



 冬十月壬寅,閱虎翼武騎卒習戰。丙午,京
 師地震。詔知州軍歲舉判、司、簿、尉可縣令者一人。



 十一月癸亥,冬至,率百官上皇太后壽於會慶殿,遂御天安殿受朝。庚午,詔天下孤獨疾病者,致醫藥存視。詔周世宗後,凡經郊祀,錄其子孫一人。是歲,河北水。遣使決囚,振貧,瘞溺死者,給其家緡錢,察官吏貪暴不恤民者。龜茲、下溪州黔州蠻來貢。



 八年春正月甲戌,曹瑋卒。辛巳,作會聖宮於西京永安縣。



 二月戊子,詔五代時官三品以上告身存者,子孫聽
 用蔭。



 三月壬申,幸後苑,遂宴太清樓。乙亥,禁以財冒士族娶宗室女者。詔河北被水州縣毋稅牛。是月,賜禮部奏名進士、諸科及第出身八百二十二人。



 五月甲寅,賜信州龍虎山張乾曜號澄素先生。丙辰,大雨雹。辛酉,錄系囚。



 六月癸巳,呂夷簡上新修國史。己亥,詔御史臺獄勿關糾察司。乙巳,親試書判拔萃科及武舉人。



 秋七月丙子,策制舉人。



 八月丙戌,詔詳定鹽法。丁亥,詔近臣宗室觀祖宗御書於龍圖、天章閣,又觀瑞穀於元真殿,遂
 宴蕊珠殿。戊子,詔流配人道死者,其妻子給食送還鄉里。



 九月癸丑,復置諸路提點刑獄官。丙辰,罷轉對。乙丑,姜遵卒。己巳,以趙稹為樞密副使。



 冬十月壬辰,奉安太祖御容於太平興國寺開先殿。丙申,弛三京、河中府、穎、許、汝、鄭、鄆、濟、衛、晉、絳、虢、亳、宿等二十八州軍鹽禁。壬寅,置天章閣待制。



 十一月丙寅,朝饗景靈宮。丁卯,饗太廟。戊辰,祀天地於圜丘,大赦。賀皇太后於會慶殿。十二月癸未,加恩百官。辛丑,西平王趙德明、交址王李德政並
 加賜功臣。是歲,高麗、占城、邛部川都蠻來貢。



 九年春正月辛亥,詔諸路轉運判官員外郎以上,遇郊聽任子弟。丙辰,長寧節,百官初上皇太后壽於會慶殿。辛未,減畿內民租。



 二月癸巳,詔復郡縣職田。



 三月甲寅,奉安太祖、太宗、真宗御容於會聖宮。



 夏四月戊寅,詔以隴州論平民五人為劫盜抵死,主者雖更赦,並從重罰。乙巳,閱大樂。



 五月乙丑,錄系囚。



 六月庚辰,宋綬上《皇太后儀制》。



 秋七月丙午朔,契丹使來告其主隆緒殂,遣使
 祭奠吊慰,及賀宗真立。



 九月癸亥,祠西太一宮,賜道左耕者茶帛。



 冬十月丙戌,詔公卿大夫勵名節。乙未,詔常參官已授外任,毋得奏舉選人。辛丑,罷益、梓、廣南路轉運判官。



 閏月戊辰,翰林侍讀學士孫奭請老,命知兗州,曲宴太清樓送之。



 十一月丁亥,馳兩川礬禁。己丑,祈雪於會靈觀。丁酉,出知雜御史曹修古,御史郭勸、楊偕,推直官段少連。十二月甲寅,詔吏部銓選人父母年八十以上者,權注近官。辛酉,大風三日。是歲,契丹主及其國
 母遣使來致遺留物及謝吊祭。南平王李德政遣人謝加恩。龜茲、沙州來貢。女真晏端等百八十四人內附。



\end{pinyinscope}