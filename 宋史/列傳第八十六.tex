\article{列傳第八十六}

\begin{pinyinscope}

 王安石子雱唐坰附王安禮王安國



 王安石,字介甫,撫州臨川人。父益,都官員外郎。安石少好讀書,一過目終身不忘。其屬文動筆如飛,初若不經意,既成,見者皆服其精妙。友生曾鞏攜以示歐陽修,修
 為之延譽。擢進士上第,簽書淮南判官。舊制,秩滿許獻文求試館職,安石獨否。再調知鄞縣,起堤堰,決陂塘,為水陸之利;貸穀與民,出息以償,俾新陳相易,邑人便之。通判舒州。文彥博為相,薦安石恬退,乞不次進用,以激奔競之風。尋召試館職,不就。修薦為諫官,以祖母年高辭。修以其須祿養言於朝,用為群牧判管,請知常州。移提點江東刑獄,入為度支判官,時嘉祐三年也。



 安石議論高奇,能以辨博濟其說,果於自用,慨然有矯世變俗
 之志。於是上萬言書,以為:「今天下之財力日以困窮,風俗日以衰壞,患在不知法度,不法先王之政故也。法先王之政者,法其意而已。法其意,則吾所改易更革,不至乎傾駭天下之耳目,囂天下之口,而固已合先王之政矣。因天下之力以生天下之財,取天下之財以供天下之費,自古治世,未嘗以財不足為公患也,患在治財無其道爾。在位之人才既不足,而閭巷草野之間亦少可用之才,社稷之托,封疆之守,陛下其能久以天幸為常,
 而無一旦之憂乎?願監茍且因循之弊,明詔大臣,為之以漸,期合於當世之變。臣之所稱,流俗之所不講,而議者以為迂闊而熟爛者也。」後安石當國,其所注措,大抵皆祖此書。



 俄直集賢院。先是,館閣之命屢下,安石屢辭;士大夫謂其無意於世,恨不識其面,朝廷每欲俾以美官,惟患其不就也。明年,同修起居注,辭之累日。閣門吏繼敕就付之,拒不受;吏隨而拜之,則避於廁;吏置敕於案而去,又追還之;上章至八九,乃受。遂知制誥,糾察在
 京刑獄,自是不復辭官矣。



 有少年得鬥鶉,其儕求之不與,恃與之暱輒持去,少年追殺之。開封當此人死,安石駁曰:「按律,公取、竊取皆為盜。此不與而彼攜以去,是盜也;追而殺之,是捕盜也,雖死當勿論。」遂劾府司失入。府官不伏,事下審刑、大理,皆以府斷為是。詔放安石罪,當詣閣門謝。安石言:「我無罪。」不肯謝。御史舉奏之,置不問。



 時有詔舍人院無得申請除改文字,安石爭之曰:「審如是,則舍人不得復行其職,而一聽大臣所為,自非大臣
 欲傾側而為私,則立法不當如此。今大臣之弱者不敢為陛下守法;而強者則挾上旨以造令,諫官、御史無敢逆其意者,臣實懼焉。」語皆侵執政,由是益與之忤。以母憂去,終英宗世,召不起。



 安石本楚士,未知名於中朝,以韓、呂二族為巨室,欲藉以取重。乃深與韓絳、絳弟維及呂公著交,三人更稱揚之,名始盛。神宗在藩邸,維為記室,每講說見稱,維曰:「此非維之說,維之友王安石之說也。」及為太子庶子,又薦自代。帝由是想見其人,甫即位,
 命知江寧府。數月,召為翰林學士兼侍講。熙寧元年四月,始造朝。入對,帝問為治所先,對曰:「擇術為先。」帝曰:「唐太宗何如?」曰:「陛下當法堯、舜,何以太宗為哉?堯、舜之道,至簡而不煩,至要而不迂,至易而不難。但末世學者不能通知,以為高不可及爾。」帝曰:「卿可謂責難於君,朕自視眇躬,恐無以副卿此意。可悉意輔朕,庶同濟此道。」



 一日講席,群臣退,帝留安石坐,曰:「有欲與卿從容論議者。」因言:「唐太宗必得魏徵,劉備必得諸葛亮,然後可以有
 為,二子誠不世出之人也。」安石曰:「陛下誠能為堯,舜,則必有皋、夔、稷、離;誠能為高宗,則必有傅說。彼二子皆有道者所羞,何足道哉?以天下之大,人民之眾,百年承平,學者不為不多。然常患無人可以助治者,以陛下擇術未明,推誠未至,雖有皋、夔、稷、離、傅說之賢,亦將為小人所蔽,卷懷而去爾。」帝曰:「何世無小人,雖堯、舜之時,不能無四兇。」安石曰:「惟能辨四兇而誅之,此其所以為堯、舜也。若使四兇得肆其讒慝,則皋、夔、稷、離亦安肯茍食其
 祿以終身乎?」



 登州婦人惡其夫寢陋,夜以刃斮之,傷而不死。獄上,朝議皆當之死,安石獨援律辨證之,為合從謀殺傷,減二等論。帝從安石說,且著為令。



 二年二月,拜參知政事。上謂曰:「人皆不能知卿,以為卿但知經術,不曉世務。」安石對曰:「經術正所以經世務,但後世所謂儒者,大抵皆庸人,故世俗皆以為經術不可施於世務爾。」上問:「然則卿所施設以何先?」安石曰:「變風俗,立法度,正方今之所急也。」上以為然。於是設制置三司條例司,
 令判知樞密院事陳升之同領之。安石令其黨呂惠卿預其事。而農田水利、青苗、均輸、保甲、免役、市易、保馬、方田諸役相繼並興,號為新法,遣提舉官四十餘輩,頒行天下。



 青苗法者,以常平糴本作青苗錢,散與人戶,令出息二分,春散秋斂。均輸法者,以發運之職改為均輸,假以錢貨,凡上供之物,皆得徙貴就賤,用近易遠,預知在京倉庫所當辦者,得以便宜蓄買。保甲之法,籍鄉村之民,二丁取一,十家為保,保丁皆授以弓弩,教之戰陣。免
 役之法,據家貲高下,各令出錢雇人充役,下至單丁、女戶,本來無役者,亦一概輸錢,謂之助役錢。市易之法,聽人賒貸縣官財貨,以田宅或金帛為抵當,出息十分之二,過期不輸,息外每月更加罰錢百分之二。保馬之法,凡五路義保願養馬者,戶一匹,以監牧見馬給之,或官與其直,使自市,歲一閱其肥瘠,死病者補償。方田之法,以東、西、南、北各千步,當四十一頃六十六畝一百六十步為一方,歲以九月,令、佐分地計量,驗地土肥瘠,定其色
 號,分為五等,以地之等,均定稅數。又有免行錢者,約京師百物諸行利入厚薄,皆令納錢,與免行戶祗應。自是四方爭言農田水利,古陂廢堰,悉務興復。又令民封狀增價以買坊場,又增茶監之額,又設措置河北糴便司,廣積糧穀於臨流州縣,以備饋運。由是賦斂愈重,而天下騷然矣。



 御史中丞呂誨論安石過失十事,帝為出誨,安石薦呂公著代之。韓琦諫疏至,帝感悟,欲從之,安石求去。司馬光答詔,有「士夫沸騰,黎民騷動」之語,安石怒,
 抗章自辨,帝為巽辭謝,令呂惠卿諭旨,韓絳又勸帝留之。安石入謝,因為上言中外大臣、從官、臺諫、朝士朋比之情,且曰:「陛下欲以先王之正道勝天下流俗,故與天下流俗相為重輕。流俗權重,則天下之人歸流俗;陛下權重,則天下之人歸陛下。權者與物相為重輕,雖千鈞之物,所加損不過銖兩而移。今奸人欲敗先王之正道,以沮陛下之所為。於是陛下與流俗之權適爭輕重之時,加銖兩之力,則用力至微,而天下之權,已歸於流俗
 矣,此所以紛紛也。」上以為然。安石乃視事,琦說不得行。



 安石與光素厚,光援朋友責善之義,三詒書反復勸之,安石不樂。帝用光副樞密,光辭未拜而安石出,命遂寢。公著雖為所引,亦以請罷新法出穎州。御史劉述、劉琦、錢顗、孫昌齡、王子韶、程顥、張戩、陳襄、陳薦、謝景溫、楊繪、劉摯,諫官范純仁、李常、孫覺、胡宗愈皆不得其言,相繼去。驟用秀州推官李定為御史,知制誥宋敏求、李大臨、蘇頌封還詞頭,御史林旦、薛昌朝、范育論定不孝,皆罷
 逐。翰林學士范鎮三疏言青苗,奪職致仕。惠卿遭喪去,安石未知所托,得曾布,信任之,亞於惠卿。



 三年十二月,拜同中書門下平章事。明年春,京東、河北有烈風之異,民大恐。帝批付中書,令省事安靜以應天變,放遣兩路募夫,責有司、郡守不以上聞者。安石執不下。



 開封民避保甲,有截指斷腕者,知府韓維言之,帝問安石,安石曰:「此固未可知,就令有之,亦不足怪。今士大夫睹新政,尚或紛然驚異;況於二十萬戶百姓,固有蠢愚為人所惑
 動者,豈應為此遂不敢一有所為邪?」帝曰:「民言合而聽之則勝,亦不可不畏也。」



 東明民或遮宰相馬訴助役錢,安石白帝曰:「知縣賈蕃乃范仲淹之婿,好附流俗,致民如是。」又曰:「治民當知其情偽利病,不可示姑息。若縱之使妄經省臺,鳴鼓邀駕,恃眾僥幸,則非所以為政。」其強辯背理率類此。



 帝用韓維為中丞,安石憾曩言,指為善附流俗以非上所建立,因維辭而止。歐陽修乞致仕,馮京請留之,安石曰:「修附麗韓琦,以琦為社稷臣。如此人,
 在一郡則壞一郡,在朝廷則壞朝廷,留之安用?」乃聽之。富弼以格青苗解使相,安石謂不足以阻奸,至比之共、鯀。靈臺郎尤瑛言天久陰,星失度,宜退安石,即黥隸英州。唐坰本以安石引薦為諫官,因請對極論其罪,謫死。文彥博言市易與下爭利,致華岳山崩。安石曰:「華山之變,殆天意為小人發。市易之起,自為細民久困,以抑兼並爾,於官何利焉。」閼其奏,出彥博守魏。於是呂公著、韓維,安石藉以立聲譽者也;歐陽修、文彥博,薦己者也;富
 弼、韓琦,用為侍從者也;司馬光、范鎮,交友之善者也:悉排斥不遺力。



 禮官議正太廟太祖東向之位,安石獨定議還僖祖於祧廟,議者合爭之,弗得。上元夕,從駕乘馬入宣德門,衛士訶止之,策其馬。安石怒,上章請逮治。御史蔡確言:「宿衛之士,拱扈至尊而已,宰相下馬非其處,所應訶止。」帝卒為杖衛士,斥內侍,安石猶不平。王韶開熙河奏功,帝以安石主議,解所服玉帶賜之。



 七年春,天下久旱,饑民流離,帝憂形於色,對朝嗟嘆,欲盡罷法度
 之不善者。安石曰:「「水旱常數,堯、湯所不免,此不足招聖慮,但當修人事以應之。」帝曰:「此豈細事,朕所以恐懼者,正為人事之未修爾。今取免行錢太重,人情咨怨,至出不遜語。自近臣以至後族,無不言其害。兩宮泣下,憂京師亂起,以為天旱,更失人心。」安石曰:「近臣不知為誰,若兩宮有言,乃向經、曹佾所為爾。」馮京曰:「臣亦聞之。」安石曰:「士大夫不逞者以京為歸,故京獨聞其言,臣未之聞也。」監安上門鄭俠上疏,繪所見流民扶老攜幼困苦之
 狀,為圖以獻,曰:「旱由安石所致。去安石,天必雨。」俠又坐竄嶺南。慈聖、宣仁二太后流涕謂帝曰:「安石亂天下。」帝亦疑之,遂罷為觀文殿大學士、知江寧府,自禮部侍郎超九轉為吏部尚書。



 呂惠卿服闋,安石朝夕汲引之,至是,白為參知政事,又乞召韓絳代己。二人守其成謨,不少失,時號絳為「傳法沙門」,惠卿為「護法善神」。而惠卿實欲自得政,忌安石復來,因鄭俠獄陷其弟安國,又起李士寧獄以傾安石。絳覺其意,密白帝請召之。八年二月,
 復拜相,安石承命,即倍道來。《三經義》成,加尚書左僕射兼門下侍郎,以子雱為龍圖閣直學士。雱辭,惠卿勸帝允其請,由是嫌隙愈著。惠卿為蔡承禧所擊,居家俟命。雱風御史中丞鄧綰,復彈惠卿與知華亭縣張若濟為奸利事,置獄鞫之,惠卿出守陳。



 十月,彗出東方,詔求直言,及詢政事之未協於民者。安石率同列疏言:「晉武帝五年,彗出軫;十年,又有孛。而其在位二十八年,與《乙巳占》所期不合。蓋天道遠,先王雖有官占,而所信者人事
 而已。天文之變無窮,上下傅會,豈無偶合。周公、召公,豈欺成王哉。其言中宗享國日久,則曰『嚴恭寅畏,天命自度,治民不敢荒寧』。其言夏、商多歷年所,亦曰『德』而已。裨灶言火而驗,欲禳之,國僑不聽,則曰『不用吾言,鄭又將火』。僑終不聽,鄭亦不火。有如裨灶,未免妄誕,況今星工哉?所傳占書,又世所禁,謄寫偽誤,尤不可知。陛下盛德至善,非特賢於中宗,周、召所言,則既閱而盡之矣,豈須愚瞽復有所陳。竊聞兩宮以此為憂,望以臣等所言,力
 行開慰。」帝曰:「聞民間殊苦新法。」安石曰:「祁寒暑雨,民猶怨咨,此無庸恤。」帝曰:「豈若並祁寒暑雨之怨亦無邪?」安石不悅,退而屬疾臥,帝慰勉起之。其黨謀曰:「今不取上素所不喜者暴進用之,則權輕,將有窺人間隙者。」安石是其策。帝喜其出,悉從之。時出師安南,諜得其露布,言:「中國作青苗、助役之法,窮困生民。我今出兵,欲相拯濟。」安石怒,自草敕榜詆之。



 華亭獄久不成,雱以屬門下客呂嘉問、練亨甫共議,取鄧綰所列惠卿事,雜他書下制
 獄,安石不知也。省吏告惠卿於陳,惠卿以狀聞,且訟安石曰:「安石盡棄所學,隆尚縱橫之末數,方命矯令,罔上要君。此數惡力行於年歲之間,雖古之失志倒行而逆施者,殆不如此。」又發安石私書曰:「無使上知」者。帝以示安石,安石謝無有,歸以問雱,雱言其情,安石咎之。雱憤恚,疽發背死。安石暴綰罪,去「為臣子弟求官及薦臣婿蔡卞」,遂與亨甫皆得罪。綰始以附安石居言職,及安石與呂惠卿相傾,綰極力助攻惠卿。上頗厭安石所為,綰
 懼失勢,屢留之於上,其言無所顧忌;亨甫險薄,諂事雱以進,至是皆斥。



 安石之再相也,屢謝病求去,及子雱死,尤悲傷不堪,力請解幾務。上益厭之,罷為鎮南軍節度使、同平章事、判江寧府。明年,改集禧觀使,封舒國公。屢乞還將相印。元豐二年,復拜左僕射、觀文殿大學士。換特進,改封荊。哲宗立,加司空。



 元祐元年,卒,年六十六,贈太傅。紹聖中,謚曰文,配享神宗廟庭。崇寧三年,又配食文宣王廟,列於顏、孟之次,追封舒王。欽宗時,楊時以為
 言,詔停之。高宗用趙鼎、呂聰問言,停宗廟配享,削其王封。



 初,安石訓釋《詩》、《書》、《周禮》,既成,頒之學官,天下號曰「新義」。晚居金陵,又作《字說》,多穿鑿傅會。其流入於佛、老。一時學者,無敢不傳習,主司純用以取士,士莫得自名一說,先儒傳注,一切廢不用。黜《春秋》之薯,不使列於學官,至戲目為「斷爛朝報」。



 安石未貴時,名震京師,性不好華腴,自奉至儉,或衣垢不浣,面垢不洗,世多稱其賢。蜀人蘇洵獨曰:「是不近人情者,鮮不為大奸慝。」作《辯奸論》以
 刺之,謂王衍、盧杞合為一人。



 安石性強忮,遇事無可否,自信所見,執意不回。至議變法,而在廷交執不可,安石傅經義,出己意,辯論輒數百言,眾不能詘。甚者謂「天變不足畏,祖宗不足法,人言不足恤。」罷黜中外老成人幾盡,多用門下儇慧少年。久之,以旱引去,洎復相,歲餘罷,終神宗世不復召,凡八年。子雱。



 雱字符澤。為人慓悍陰刻,無所顧忌。性敏甚,未冠,已著書數萬言。年十三,得秦卒言洮、河事,嘆曰:「此可撫而有
 也。使西夏得之,則吾敵強而邊患博矣。」其後王韶開熙河,安石力主其議,蓋兆於此。舉進士,調旌德尉。



 雱氣豪,睥睨一世,不能作小官。作策二十餘篇,極論天下事,又作《老子訓傳》及《佛書義解》,亦數萬言。時安石執政,所用多少年,雱亦欲預選,乃與父謀曰:「執政子雖不可預事,而經筵可處。」安石欲上知而自用,乃以雱所作策及注《道德經》鏤板鬻於市,遂傳達於上。鄧綰、曾布又力薦之,召見,除太子中允、崇政殿說書。神宗數留與語,受詔注《
 詩》、《書》義,擢天章閣待制兼侍講。書成,遷龍圖閣直學士,以病辭不拜。



 安石更張政事,雱實導之。常稱商鞅為豪傑之士,言不誅異議者法不行。安石與程顥語,雱囚首跣足,攜婦人冠以出,問父所言何事。曰:「以新法數為人所阻,故與程君議。」雱大言曰:「梟韓琦、富弼之頭於市,則法行矣。」安石遽曰:「兒誤矣。」卒時才三十三,特贈左諫議大夫。



 唐坰者,以父任得官。熙寧初,上書云:「秦二世制於趙高,
 乃失之弱,非失之強。」神宗悅其言。又云:「青苗法不行,宜斬大臣異議如韓琦者數人。」安石尤喜之,薦使對,賜進士出身,為崇文校書。上薄其人,除知錢塘縣。安石欲留之,乃令鄧綰薦為御史,遂除太子中允。數月,將用為諫官,安石疑其輕脫,將背己立名,不除職,以本官同知諫院,非故事也。



 坰果怒安石易己,凡奏二十疏,論時事,皆留中不出。乃因百官起居日,扣陛請對,上令諭以他日,坰伏地不起,遂召升殿。坰至御坐前,進曰:「臣所言,皆大
 臣不法,請對陛下一一陳之。」乃措笏展疏,目安石曰:「王安石近御坐,聽札子。」安石遲遲,坰訶曰:「陛下前猶敢如此,在外可知!」安石悚然而進。坰大聲宣讀,凡六十條,大略以「安石專作威福,曾布等表裏擅權,天下但知憚安石威權,不復知有陛下。文彥博、馮京知而不敢言。王珪曲事安石,無異廝僕。」且讀且目珪,珪慚懼俯首。「元絳、薛向、陳繹,安石頤指氣使,無異家奴。張琥、李定為安石爪牙,臺官張商英乃安石鷹犬。逆意者雖賢為不肖,附己
 者雖不肖為賢。」至詆為李林甫、盧杞。上屢止之,坰慷慨自若,略不退懾。讀已,下殿再拜而退。侍臣衛士,相顧失色,安石為之請去。閣門糾其瀆亂朝儀,貶潮州別駕。鄧綰申救之,且自劾繆舉。安石曰:「此素狂,不足責。」改監廣州軍資庫,後徙吉州酒稅,卒官。



 論曰:朱熹嘗論安石「以文章節行高一世,而尤以道德經濟為己任。被遇神宗,致位宰相,世方仰其有為,庶幾復見二帝三王之盛。而安石乃汲汲以財利兵革為先
 務,引用兇邪,排擯忠直,躁迫強戾,使天下之人,囂然喪其樂生之心。卒之群奸嗣虐,流毒四海,至於崇寧、宣和之際,而禍亂極矣」。此天下之公言也。昔神宗欲命相,問韓琦曰:「安石何如?」對曰:「安石為翰林學士則有餘,處輔弼之地則不可。」神宗不聽,遂相安石。嗚呼!此雖宋氏之不幸,亦安石之不幸也。



 王安禮,字和甫,安石之弟也。早登科,從河東唐介闢。熙寧中,鄜延路城囉兀,河東發民四萬負餉,宣撫使韓絳
 檄使佐役,後帥呂公弼將從之。安禮爭曰:「民兵不習武事,今驅之深入,此不為寇所乘,則凍餓而死爾,宜亟罷遣。」公弼用其言,民得歸,而他路遇敵者,全軍皆覆。公弼執安禮手言曰:「四萬之眾,豈偶然哉。果有陰德,相與共之。」



 初,絳專爵賞,既上最,多失實,公弼以狀聞。詔即河東議功,公弼將受之。安禮曰:「宣撫使以宰相節制諸道,且許便宜,封授一有不韙,人猶得非之。公藩臣,乃欲隃進功狀於非其任邪?」公弼遽辭。遂薦安禮於朝,神宗召對,
 欲驟用之。安石當國,辭,以為著作佐郎、崇文院校書。他日得見,命之坐,有司言八品官無賜坐者,特命之。遷直集賢院,出知潤州、湖州,召為開封府判官。嘗偕尹奏事,既退,獨留訪以天下事,帝甚鄉納。直舍人院、同修起居注。



 蘇軾下御史獄,勢危甚,無敢救者。安禮從容言:「自古大度之主,不以言語罪人。軾以才自奮,謂爵位可立取,顧錄錄如此,其心不能無觖望。今一旦致於理,恐後世謂陛下不能容才。帝曰:「朕固不深譴也,行為卿貰之。
 卿第去,勿漏言,軾方賈怨於眾,恐言者緣以害卿也。」李定、張璪皆擿使勿救,安禮不答,軾以故得輕比。



 進知制誥。彗星見,詔求直言。安禮上疏曰:「人事失於下,變象見於上。陛下有仁民愛物之心,而澤不下究,意者左右大臣不均不直,謂忠者為不忠,不賢者為賢,乘權射利者,用力殫於溝瘠,取利究於園夫,足以幹陰陽而召星變。願察親近之行,杜邪枉之門。至於祈禳小數,貶損舊章,恐非所以應天者。」帝覽數嘉嘆,諭之曰:「王珪欲使卿條具,
 朕嘗謂不應沮格人言,以自壅障。今以一指蔽目,雖泰、華在前弗之見,近習蔽其君,何以異此,卿當益自信。」



 以翰林學士知開封府,事至立斷。前滯訟不得其情,及且按而未論者幾萬人,安禮剖決,未三月,三獄院及畿、赤十九邑,囚系皆空。書揭於府前,遼使過而見之,嘆息誇異。帝聞之,喜曰:「昔秦內史廖從容俎豆,以奪由余之謀,今安禮能勤吏事,駭動殊鄰,於古無愧矣。」特升一階。



 帝數失皇子,太史言民墓多迫京城,故不利國嗣,詔悉改
 卜,無虜數十萬計,眾洶懼。安禮諫曰:「文王卜世三十,其政先於掩骼埋胔,未聞遷人之塚以利其嗣者。」帝惻然而罷。



 邏者連得匿名書告人不軌,所涉百餘家。帝付安禮曰:「亟治之。」安禮驗所指,皆略同,最後一書加三人,有姓薛者,安禮喜曰:「吾得之矣。」呼問薛曰:「若豈有素不快者耶?」曰有持筆來售者,拒之,鞅鞅去,其意似見銜。即命捕訊,果其所為也。即梟其首於市,不逮一人,京師謂為神明。



 宗室令騑以數十萬錢買妾,久而斥歸之,訴府督
 元直。安禮視妾,既火敗其面矣,即奏言:「妾之所以直數十萬者,以姿首也,今炙敗之,則不復可鬻,此與炮烙之刑何異。請勿理其直而加厚譴,以為戒。」詔從之,仍奪令騑俸。



 後宮造油箔,約三年損者反其價,才一年有損者,中官持詣府,請如約,詞氣甚厲。安禮曰:「庸詎非置之不得其地,為風雨燥濕所壞耶。茍如是,民將無復得直,約不可用也。」卒不追。以是宗室、中貴人皆憚之。



 元豐四年,初分三省,置執政,拜中大夫、尚書右丞。轉左丞。王師問
 罪夏國,涇原承受梁同奏:「轉運使葉康直餉米,惡不可食。」帝大怒曰:「貴糴遠餉,反不可用。徒弊民力於道路,康直可斬也。」安禮曰:「此一梁同之言,疑未必實,當按之。」乃遣判官張大寧與同參核,且械系康直以俟。既而米可用者什八九,帝意解,赦康直。



 是時,伐夏不得志,李憲又欲再舉。帝以訪輔臣,王珪曰:「向所患者用不足,朝廷今捐錢鈔五百萬緡,以供軍食有餘矣。」安禮曰:「鈔不可啖,必變而為錢,錢又變為芻粟。今距出征之期才兩月,安
 能集事。」帝曰:「李憲以為已有備,彼宦者能如是,卿等獨無意乎?唐平淮蔡,唯裴度謀議與主同。今乃不出公卿而出於閹寺,朕甚恥之。」安禮曰:「淮西,三州爾,有裴度之謀,李光顏、李醞之將,然猶引天下之兵力,歷歲而後定。今夏氏之強非淮蔡比,憲材非度匹,諸將非有光顏、醞輩,臣懼無以副聖意也。」帝悟而止。後欲除憲節度使,安禮又以為不可。



 御史中丞舒但上章詆執政,且言:「尚書不置錄目,有旨按吏罪。」安禮請取臺錄以為式,乃與省
 中同,遂並列但他事,但坐廢。徐禧計議邊事,安禮曰:「禧志大才疏,必誤國。」及永樂敗書聞,帝曰:「安禮每勸朕勿用兵,少置獄,蓋為是也。」



 久之,御史張汝賢論其過,以端明殿學士知江寧府,汝賢亦罷。元祐中,加資政殿學士,歷揚、青、蔡三州。又為御史言,失學士,移舒州。紹聖初,還職,知永興軍。二年,知太原府。苦風痺,臥帳中決事,下不敢欺。卒,年六十二,贈右銀青光祿大夫。



 安禮偉風儀,論議明辨,常以經綸自任,而闊略細謹,以故數詒口語云。



 王安國,字平甫,安禮之弟也。幼敏悟,未嘗從學,而文詞天成。年十二,出所為詩、銘、論、賦數十篇示人,語皆警拔,遂以文章聞於世,士大夫交口譽之。於書無所不通,數舉進士,又舉茂材異等,有司考其所獻序言為第一,以母喪不試,廬於墓三年。



 熙寧初,韓絳薦其材行,召試,賜及第,除西京國子教授。官滿,至京師,上以安石故,賜對。帝曰:「卿學問通古今,以漢文帝為何如主?」對曰:「三代以後未有也。」帝曰:「但恨其才不能立法更制爾。」對曰:「文帝
 自代來,入未央宮,定變故俄頃呼吸間,恐無才者不能。至用賈誼言,待群臣有節,專務以德化民,海內興於禮義,幾致刑措,則文帝加有才一等矣。」帝曰:「王猛佐苻堅,以蕞爾國而令必行,今朕以天下之大,不能使人,何也?」曰:「猛教堅以峻刑法殺人,致秦祚不傳世,今刻薄小人,必有以是誤陛下者。願顓以堯、舜、三代為法,則下豈有不從者乎。」又問:「卿兄秉政,外論謂何?」曰:「恨知人不明,聚斂太急爾。」帝默然不悅,由是別無恩命,止授崇文院校
 書,後改秘閣校理。屢以新法力諫安石,又質責曾布誤其兄,深惡呂惠卿之奸。



 先是,安國教授西京,頗溺於聲色,安石在相位,以書戒之曰:「宜放鄭聲。」安國復書曰「亦願兄遠佞人。」惠卿銜之。及安石罷相,惠卿遂因鄭俠事陷安國,坐奪官,放歸田里。詔以諭安石,安石對使者泣下。既而復其官,命下而安國卒,年四十七。



 論曰:安石惡蘇軾而安禮救之,暱惠卿而安國折之,議者不以咎二弟也,惟其當而已矣。安禮為政,有足稱者。
 安國早卒,故不見於用云。



\end{pinyinscope}