\article{列傳第二百二十四外戚下}

\begin{pinyinscope}

 ○孟忠厚韋淵錢忱邢煥潘永思吳益弟蓋李道鄭興裔楊次山



 孟忠厚,字仁仲,隆祐太后兄,追封咸寧郡王彥弼子也。後退居瑤華宮,哲宗恩眷不衰,故忠厚得以仕進。宣和中,官至將作少監。靖康元年,知海州,召權衛尉卿。金人圍城,後宮火,出居忠厚家,由是免北遷。金兵退,張邦昌迎後聽政,後遣忠厚持書遺康王。王即位,將迎後,授忠厚徽猷閣待制,提舉一行事務,尋兼乾辦奉迎太廟神主事。



 帝幸揚州,除顯謨閣直學士,臺諫交章論列,帝以太后故,難之。後聞,即命易武秩,遂授常德軍承宣使,乾
 辦皇城司。未幾,奉太后幸杭州。苗傅亂平,趙鼎謂張浚曰:「太后復闢,其功甚大,當推恩外家。」浚乃奏忠厚寧遠軍節度使。尋奉太后幸南昌,歸至越,以母憂解職。



 頃之,後崩,以祔廟恩,起復鎮潼軍節度使,開府儀同三司。及後大祥,封信安郡王,充禮儀使,奉太后神御幸溫州。紹興九年,判鎮江府,改判明州兼安撫使,改判婺州。既而帝以太后攢會稽,乃命忠厚判紹興府兼修奉攢宮事,加少保。三梓宮歸,充迎護使。及營祐陵,秦檜當為總護
 使,憚往,乃除忠厚樞密使以代其行。檜與忠厚僚婿也,然心實忌之。山陵事畢,忠厚欲歸樞密府,檜諷言路引故事論列,遂判福州。



 時海寇猖獗,帝憂忠厚不能弭其患,改判建康府,又改判紹興府。會郊赦加恩,謝表有「本無時才,出為世用」語。中丞詹大方希檜意,論忠厚表辭輕侮,謂今日不足與有為,遂罷為醴泉觀使。檜死,召還行在,授保寧軍節度使、判平江府,再改判紹興府,過闕入見,復詔充萬壽觀使,提舉秘書省。二十七年,卒,贈太
 保。



 忠厚奉昭聖太后訓,避遠權勢,不敢以私干朝廷。明受之變,太后垂簾,忠厚乞裁節本家恩澤,如有夤緣,令三省執奏。御史劾秦檜當國,親姻扳援以進,忠厚獨與之忤。自越入見,語所善王銍曰:「忠厚與檜雖有親好,每懷疑心,今欲求一不傷時忌對札。」銍教之,但言乞免提舉學事而已,然亦見廢。帝以太后擁祐功,故眷忠厚特優。後在瑤華三十年,恩澤未嘗陳請,詔賜忠厚田三十頃以賞之。既奉內祠,金使至,特命押班,且令月過局,如
 宰執例。及卒,三子皆除直秘閣,親屬六人各進以一官。



 韋淵,顯仁太后季弟也。靖康末,官至拱衛大夫、忠州防禦使、勾當軍頭引見司。金人退,張邦昌遣淵持書遺康王於濟南。王即位,遷親衛大夫、寧州觀察使、知東上閣門事,言:「橫行五司尚未遵元豐舊制,乞並引進司歸客省,東、西上閣門合而為一,以省冗費。」從之。遂命同管客省、四方館、閣門事。



 淵性暴橫,不循法度,帝慮其有過,難於行法,遂遷福建路副總管。淵引疾丐祠,許之。淵乃言,
 自宣和及今,十二年未嘗磨勘,乞遷秩。吏部言,在法,橫行無以年勞磨勘者,帝遂不許。久之,落階官,除德慶軍節度使。召赴行在,除開府儀同三司。會建康軍帥邊順疾篤,留守呂頤浩奏以淵代,帝不欲以戚裏管軍,不許。淵陳乞恩數,帝詢太后家故例,賜田五十頃,房緡錢日二十千。帝久不予淵官,聞太后將入境,乃封平樂郡王,令逆於境上。既從後歸,即令致仕。又詔奉朝請,遷少師。淵在內不得逞,乞致仕,任便居住。從之。



 未幾,帝恐其肆
 橫於外,復詔落致仕,還居賜第。太后朝景靈宮,淵見後,出言詆毀,詔侍御史餘堯弼即其家鞫治,淵具伏誣罔,責授寧遠軍節度副使、袁州安置。數年復故職,累遷太保、太傅。卒,贈太師。子三人:訊、謙、讜。



 訊,紹興中,官至達州刺史,坐過,用太后旨降武德郎,與嶺外監當。謙,好學能詩,官至建康軍節度使。



 謙子璞,淳熙末,仕至太府少卿。高崇崩,擢司農少卿,為金國告哀使。金主錫宴,其館使欲用樂,璞不可,自朝至夜漏下三十刻,金人不能奪。及
 入見,其閣門令璞吉服入,璞又不可。日將中,乃以兇服見。紹熙初,除煥章閣,論者以為非祖宗舊制,遂換授明州觀察使,十年不遷。寧宗嘉其恬退,授清遠軍節度使,致仕,卒,贈太尉。



 錢忱,字伯誠,吳越王俶五世孫。父景臻,尚仁宗第十女秦魯國大長公主,生忱,神宗命賜名,除莊宅副使、騎都尉。



 帝嘗諭景臻曰:「主賢,宜有子,為擇嘉配。」娶唐介孫女,又晁迥外孫。忱從二家游,伯父勰在翰苑,因得識一時
 名卿。



 哲宗愛之,常使侍左右。徽宗覃八寶恩,為邕州觀察使,遷武寧軍觀察留後。喜其靖共,除瀘州節度使。欽宗加檢校少保,尋納節。高宗立,復拜檢校少保、瀘川節度使、中太一宮使,御書「忠孝之家」四字賜之,進開府儀同三司。紹興十五年,以秦魯主終喪,除少保,封榮國公。三十年,遷少師,仍舊節,致仕,給真奉。明年卒,年八十餘,贈太師。子端禮,自有傳。



 邢煥,字文仲,開封人。以父任調孟州汜水縣主簿,監在
 京藥局、平準務、茶場,以勞改宣德郎、莫州司錄。移知開封府陽武縣,都大提舉開德、大名府堤埽。歷開封府士、工、儀曹。



 詔納其女為康王妃。靖康初,主管亳州明道宮。王即位,升右文殿修撰,進徽猷閣待制。諫議大夫衛膚敏言,後父不當班從臣,遂改光州觀察使,除樞密都承旨。煥屢奏馬伸言事切當,宗澤忠勞可倚,黃潛善、汪伯彥誤國,其言多所裨益。



 遷保靜軍承宣使。苗、劉之變,煥自度不能爭,乃病免。兼提舉萬壽觀,求去不已,改江州
 太平觀,遂徙居忠州。



 紹興二年,入對,首陳川、陜形勢利害,請幸荊南,分兵以圖恢復,凡數百言,帝甚嘉之。復以為都承旨,引疾不拜。擢慶遠軍節度使、提舉洞霄宮。



 煥涉學有文,節儉自持,未嘗恃恩私請,識者取焉。是年,卒,贈開府儀同三司,謚恭簡,加贈少師,追封嘉國公。



 潘永思,賢妃叔父也。妃初進封,詔以梁師成第賜永思。建炎初,為閣門宣贊舍人、帶御器械。



 元祐太后在虔,帝遣永思迎歸,權三省、樞密事。盧益頗與之交結,為諫官
 吳表臣所論,範宗尹請出永思,帝曰:「未可,姑罷祿以困之,庶知悔過。」遂奪職。既而辛企宗言永思嘗捕魔賊有功,復為帶御器械。



 未幾,大理推治偽告,事連永思,帝曰:「永思雖戚里,既有過,安可廢法!」乃罷職就逮。獄成,追一官。尋復為閣門宣贊舍人,遷同知閣門事。永思乞增給飱錢,戶部言其不應格法,乃止。紹興八年,自右武郎擢右武大夫、知閣門事,尋卒。



 吳益,字叔謙,蓋字叔平,俱憲聖皇后弟也。益,建炎末,以
 恩補官,累遷乾辦御輦院、帶御器械。蓋,紹興五年,以恩補官,累遷宣贊舍人。帝與後皆喜翰墨,故益、蓋兄弟師法,亦有書名。後受冊推恩,益加成州團練使,蓋加文州刺史。帝為置皇后宅大小學教授,以王糸茲為之。糸茲明經,善訓導,益、蓋折節事之。



 益娶秦檜長孫女,又與王繼先交相薦引,故三家姻族皆躐美官。益歷官至保康軍節度使,加太尉、開府儀同三司。初,既建節,以檜故,授文資,直秘閣。檜進徽宗御制,辭免加恩,帝乃特命賜益三品服,累
 加秘閣修撰,直徽猷閣。以檜提舉編修寬恤詔令,又加益直寶文閣。檜死,其子熺復請於帝,又升敷文閣待制。中丞湯鵬舉言,益以庸瑣之才,恃親暱之勢,乞褫職名,以示至公,帝謂:「鵬舉所論甚切當,然朕於奠檜日,諭檜妻子,許以保全其家,今若遽出其婿則傷恩,臣僚無得更有論列。」自是不復遷。顯仁太后葬,為攢宮總護使,始進少保。孝宗嗣位,進少傅,又進太師,封太寧郡王。乾道七年,卒,年四十八,謚莊簡,追封衛王。



 蓋官至寧武軍節
 度使,亦累升太尉、開府儀同三司、少保,封新興郡王。乾道二年,卒,年四十二。贈太傅,追封鄭王。



 益子琚,習吏事,乾道九年,特授添差臨安府通判,其後歷尚書郎、部使者,換資至鎮安軍節度使,復以才選,除知明州兼沿海制置使。寧宗初,乃得祠,奉朝請。尋知鄂州,再知慶元府,位至少師,判建康府兼留守,卒。方孝宗崩,光宗以疾不能執喪,大臣請太后垂簾,冊立寧宗。琚言於後曰:「垂簾可暫不可久。」後遂以翌日徹簾。琚嘗使金,金人嘉其信
 義。琚死後,宋遣使至金議和,屢不合,金人言南使中惟吳琚言為可信。



 琚弟璹,仕至保靜軍節度使。蓋子環,亦至昭化軍節度使。



 李道,字行之,相州人。其中女為光宗後。初,道與兄旺聚眾歸宗澤,澤因事斬旺,命道掌其軍。澤薨,道引軍依襄陽鎮撫使桑仲,仲以為副都統制兼知隨州,奏於朝,授武義郎、閣門宣贊舍人。仲為霍明所殺,道與統制李橫率兵縞素圍明於郢,明亡去。



 劉豫遣人持書招道,道不
 從,執其使以聞,詔嘉獎之。豫怒,遣將穆楷攻道,道拒破之。除鄧、隨州鎮撫使兼知鄧州。時李橫已命別將守鄧,道憚橫,不敢受,遂命仍知隨州。樞密院以道能察軍情,不受鎮撫之命,理宜褒賞。詔領榮州團練使,進武義大夫。



 胡安中守唐州,勢孤不能自立,遂附豫。道招之,安中復來歸。會李成入寇,鎮撫使李橫棄襄陽去,道亦棄隨南歸,至江州。詔道屬岳飛為選鋒軍統制,入唐州,擒偽將,除唐、鄧、郢州、襄陽都統制。從飛收復襄陽等郡,授行
 營護軍。累至復州防禦使、果州觀察使。戌鄂州,加中侍大夫、武勝軍承宣使,又升御前諸軍統制。



 武興蠻楊再興連歲寇掠,道破其眾,擒再興及其二子,遷保寧軍承宣使。群盜朱持等聚桂陽,詔道移軍衡州經理,道遣高仲等擊平之。落階官,加龍神衛四廂都指揮使,遷鎮南軍承宣使。



 金將渝盟,命道以所部戍荊南府。帥臣劉錡奏改為御前前軍、右軍,就命道統之。錡召奏事,道代為御前諸軍都統制。金將劉士萼屯光化境,道掩擊,焚其
 舟,萼遂遁去。尋因大將言道與鄂帥不協,罷。逾年,起授捧日、天武四廂都指揮使、知荊南府。



 隆興初,湖北諸司劾其過,帝曰:「道恃戚里妄作,可罷。」久之,再為湖北副總管。及卒,乃拜慶遠軍節度使,贈太尉,謚忠毅。後既貴,進封楚王。孫孝友、孝純,皆至節度使。



 鄭興裔,字光錫,初名興宗,顯肅皇后外家三世孫也。曾祖紳,封樂平郡王。祖翼之,陸海軍節度使。父蕃,和州防禦使。興裔早孤,叔父藻以子字之,分以餘貲,興裔不受,
 請立義莊贍宗族。及藻沒,遂解官致追報之義。初以後恩授成忠郎,充乾辦祗候庫。聖獻後葬,充攢宮內外巡檢,累至江東路鈐轄。



 乾道初,建康留司請治行宮備巡幸,興裔奏勞人費財,乞罷其役,且言都統及馬軍帥皆非其人。徒福建路兵馬鈐轄,過闕入見,詢以守令臧否,興裔條析以對。帝曰:「卿識時務,習吏事,行當用卿。」會復置武臣提刑,就命為之,加遙領高州刺史。郡縣積玩,檢驗法廢,興裔創為格目,分畀屬縣,吏不得行其奸,因著
 為令。



 建、劍、汀、邵鹽策屢更,漕臣請易綱運為鈔法,興裔極言其不可。海寇倏去忽來,調兵常無及,興裔請置澳長,寇至徑率民兵御之。又言禁兵事藝不精,多充私役,乞行禁止,尉以捕盜改秩,多偽,當加審實。帝善其數論事,詔加成州團練使。



 時傳聞金欲敗盟,召興裔為賀生辰副使以覘之,使還,言無他,卒如所料。累差浙東、浙西、江東提刑,請祠以歸。尋詔知閣門事兼乾辦皇城司,又兼樞密副都承旨。軍婦楊殺鄰舍兒,取其臂釧而棄其尸,
 獄成,刑部以無證左,出之。命興裔覆治得實,帝喜,賜居第。丁母憂去官,服闋,復故職,除均州防禦使。



 再使金,還,遷潭州觀察使。復請祠,起知廬州,移知揚州。揚與廬為鄰。初,興裔在廬嘗卻鄰道互送禮,至是按郡籍,見前所卻者有出無歸,遂奏嚴其禁,揚有重屯,糧乏,例糴他境,興裔搜括滲漏以補之,食遂足。民舊皆茅舍,易焚,興裔貸之錢,命易以瓦,自是火患乃息。又奏免其償,民甚德之。修學宮,立義塚,定部轄民兵升差法,郡以大治。楚州
 議改築城,有謂韓世忠遺基不可易者,命興裔往視,既至,闕地丈餘增築之。帝閱奏,喜曰:「興裔不吾欺也。」



 紹熙元年,遷保靜軍承宣使,召領內祠,充明堂大禮都大主管大內公事。寧宗即位,除知明州兼沿海制置使。告老,授武泰軍節度使。卒,年七十四,贈太尉,謚忠肅。



 興裔歷事四朝,以材名結主知,中興外族之賢,未有其比。子三人:挺,以橫行團練使歷淮、襄兩道帥。損,登進士甲科,與抗皆有位於朝。



 楊次山,字仲甫,恭聖仁烈皇后兄也,其先開封人。曾祖全,以材武奮,靖康末,捍京城死事。祖漸,以遺澤補官,仕東南,家於越之上虞。



 次山儀狀魁偉,少好學能文,補右學生。後受職宮中,次山遂沾恩得官,積階至武德郎。後為貴妃,累遷帶御器械、知閣門事。丐祠,除吉州刺史,提舉祐神觀。後受冊,除福州觀察使,尋拜岳陽軍節度使。後謁家廟,加太尉。韓侂胄誅,加開府儀同三司。尋進少保,封永陽郡王。南郊恩加少傅,充萬壽觀使。致仕,加太
 保,授安德軍、昭慶軍節度使,改封會稽郡王。



 次山能避權勢,不預國事,時論賢之。嘉定十二年,卒,年八十一,贈太師,追封冀王。子二人。



 穀,至太傅、保寧軍節度使,充萬壽觀使、永寧郡王。



 石,字介之,乾道間入武學,以恭聖仁烈後貴,賜第。慶元中,補承信郎,差充閣門看班祗候,尋帶御器械。嘉泰四年,充賀正旦接伴使。時金使頗驕倨,自矜其善射,石從容起,挽弦三發三中的,金使氣沮。嘉定改元,除揚州觀察使、知閣門事,進保寧承宣使。久之,
 授保寧節度使,提舉萬壽觀,奉朝請,進封信安郡侯。十五年,以檢校少保進封開國公。



 寧宗崩,宰相史彌遠謀廢皇子竑而立成國公昀,命石與穀白後,後不可,曰:「皇子,先帝所立,豈敢擅變。」穀、石凡一夜七往反以告,後終不聽。穀等拜泣曰:「內外軍民皆已歸心,茍不從,禍變必生,則楊氏且無噍類矣!」後默然良久,曰:「其人安在?」彌遠等召昀入,遂矯詔廢竑為濟王,立昀,是為理宗。授開府儀同三司,充萬壽觀使。



 時寶慶垂簾,人多言本朝世有
 母後之聖。石獨曰:「事豈容概言?昔仁宗、英宗、哲宗嗣位,或尚在幼沖,或素由撫育,軍國重事有所未諳,則母后臨朝,宜也。今主上熟知民事,天下悅服,雖聖孝天通,然不早復政,得無基小人離間之嫌乎?」乃密疏章獻、慈聖、宣仁所以臨朝之由,遠及漢、唐母后臨朝稱制得失上之,後覽奏,即命擇日徹簾。進石少保,封永寧郡王。以壽明慈睿仁福三冊太后寶,進至太傅。



 石性恬澹,每拜爵命必力辭。恭聖祔廟,除太師。兄谷疑於辭受,石力言曰:「
 吾家非有元勛盛德,徒以恭聖故致貴顯,曩吾父不居是官,吾兄弟今偃然受之,是將自速顛覆耳。矧恭聖抑遠族屬,意慮深遠,言猶在耳,何可遽忘?」乃合疏懇辭,至再三,不受。及屬疾,除彰德、集慶節度使,進封魏郡王。卒,年七十一,贈太師。



\end{pinyinscope}