\article{本紀第三}

\begin{pinyinscope}

 太祖三



 五年春正月壬辰朔,雨雪,不御殿。禁鐵鑄浮屠及佛像。庚子,前盧氏縣尉鄢陵許永年七十有五,自言父瓊年九十九,兩兄皆八十餘,乞一官以便養。因召瓊厚賜之,授
 永鄢陵令。壬寅,省州縣小吏及直力人。乙巳,罷襄州歲貢魚。二月丙子,詔沿河十七州各置河堤判官一員。庚辰,以鳳州七房銀冶為開寶監。庚寅,以兵部侍郎劉熙古參知政事。閏月壬辰,禮部試進士安守亮等諸科共三十八人,召對講武殿,始放榜。庚戌,升密州為安化軍節度。三月庚午,賜穎州龍騎指揮使仇興及兵士錢。辛未,占城國王波美稅遣使來獻方物。壬申,幸教船池習戰。乙酉,殿中侍御史張穆坐贓棄市。夏四月庚寅朔,三
 佛齊國主釋利烏耶遣使來獻方物。丙午,遣使檢視水災田。丙寅,遣使諸州捕虎。五月庚申,賜恩赦侯劉鋹錢一百五十萬。乙丑,命近臣祈晴。並廣南州十三、縣三十九。丙寅,罷嶺南採珠媚川都卒為靜江軍。辛未,河決濮陽,命穎州團練使曹翰往塞之。甲戌,以霖雨,出後宮五十餘人,賜予以遣之。丁亥,河南、北淫雨,澶、滑、濟、鄆、曹、濮六州大水。六月己丑,河決陽武,汴決穀熟。丁酉,詔:淫雨河決,沿河民田有為水害者,有司具聞除租。戊申,修陽
 武堤。秋七月己未,右拾遺張恂坐贓棄市。癸未,邕、容等州獠人作亂。



 八月庚寅,高麗國王王昭遣使獻方物。己亥,廣州行營都監朱憲大破獠賊於容州。癸卯,升宿州為保靜軍節度,罷密州仍為防禦。九月丁巳朔,日有食之。癸酉,李崇矩以鎮國軍節度使罷。冬十月庚子,幸河陽節度使張仁超第視疾。甲辰,試道流,不才者勒歸俗。十一月己未,李繼明、藥繼清大破獠賊於英州。癸亥,禁僧道習天文地理。己巳,禁舉人寄應。庚辰,命參知
 政事薛居正、呂餘慶兼淮、湖、嶺、蜀轉運使。十二月乙酉朔,祈雪。己亥,畋近郊。開封尹光義暴疾,遂如其第視之。甲寅,內班董延諤坐監務盜芻粟,杖殺之。詔合入令、錄者引見後方注。乙卯,大雨雪。是歲,大饑。



 六年春正月丙辰朔,不御殿。置蜀水陸轉運計度使。癸酉,修魏縣河。二月丙戌朔,棣州兵馬監押、殿直傅延翰謀反,伏誅。丙申,曹州饑,漕太倉米二萬石振之。己亥,吳越國進銀裝花段、金香師子。三月乙卯朔,周鄭王殂於房州,
 上素服發哀,輟朝十日,謚曰恭帝,命還葬慶陵之側,陵曰順陵。己未,復密州為安化軍節度。庚申,覆試進士於講武殿,賜宋準及下第徐士廉等諸科百二十七人及第。乙亥,賜宋準等宴錢二十萬。大食國遣使來獻。翰林學士、知貢舉李昉坐試人失當,責授太常少卿。試朝臣死王事者子陸坦等,賜進士出身。丙子,幸相國寺觀新修塔。夏四月丁亥,召開封尹光義、天平軍節度使石守信等賞花、習射於苑中。辛丑,遣盧多遜為江南國信使。
 甲辰,占城國王悉利陀盤印茶遣使來獻方物。丙午,黎州保塞蠻來歸。戊申,詔修《五代史》。五月庚申,劉熙古以戶部尚書致仕。詔:中書吏擅權多奸贓,兼用流內州縣官。己巳,交州丁璉遣使貢方物。幸玉津園,觀刈麥。辛巳,殺右拾遺馬適。六月辛卯,閱在京百司吏,黜為農者四百人。癸巳,占城國遣使獻方物。隰州巡檢使李謙溥拔北漢七砦。癸卯,雷有鄰告宰相趙普黨堂吏胡贊等不法,贊及李可度並批籍沒。庚戌,詔參知政事與宰相趙
 普分知印押班奏事。秋七月壬子朔,詔諸州府置司寇參軍,以進士、明經者為之。丙辰,減廣南無名率錢。八月乙酉,罷成都府偽蜀嫁裝稅。辛卯,賜布衣王澤方同學究出身。丁酉,泗州推官侯濟坐試判假手,杖、除名。甲辰,趙普罷為河陽三城節度使、同平章事。辛酉,幸都亭驛。九月丁卯,餘慶以尚書左丞罷。己巳,封光義為晉王、兼侍中,德昭同中書門下平章事,薛居正為門下侍郎、同平章政事,戶部侍郎、樞密副使沉義倫為中書侍郎、同
 平章事,石守信兼侍中,盧多遜中書舍人、參知政事。壬申,詔晉王光義班宰相上。冬十月甲申,葬周恭帝,不視朝。丁亥,幸玉津園觀稼。戊子,流星出文昌、北斗。甲辰,特赦諸官吏奸贓。十一月癸丑,詔常參官進士及第者各舉文學一人。十二月壬午,命近臣祈雪。丙午,前中書舍人、參知政事多遜起復視事。行《開寶通禮》。限度僧法,諸州僧帳及百人,歲許度一人。



 七年春正月庚戌,不御殿。庚申,占城國王波美稅遣使
 獻方物。齊州野蠶成繭。癸亥,左拾遺秦但、太子中允呂鵠並坐贓,宥死,杖、除名。二月庚辰朔,日有食之。丙戌,日有二黑子。癸卯,命近臣祈雨。詔:《詩》、《書》、《易》三經學究,依三經、三傳資敘入官。乙巳,太子中舍胡德沖坐隱官錢,棄市。



 三月乙丑,三佛齊國王遣使獻方物。夏四月丙午,遣使檢嶺南民田。五月戊申朔,殿中侍御史李瑩坐受南唐饋遺,責授右贊善大夫。甲寅,以布衣齊得一為章丘主簿。乙丑,詔市二價者以枉法論。丙寅,幸講武池,觀習
 水戰。丙子,又幸講武池,遂幸玉津園。六月丙申,河中府饑,發粟三萬石振之。己亥,淮溢入泗州城。壬寅,安陽河溢,皆壞民居。秋七月壬子,幸講武池,觀習水戰,遂幸玉津園。丙辰,南丹州溪洞酋帥莫洪燕內附。詔減成都府鹽錢。庚午,太子中允李仁友坐不法,棄市。八月戊寅,吳越國王遣使來朝貢。丁亥,諭吳越伐江南。戊子,陳州貢芝草,一本四十九莖。己丑,幸講武池,賜習水戰軍士錢。戊戌,殿中丞趙象坐擅稅,除名。甲辰,幸講武池,觀習水
 戰,遂幸玉津園。九月癸亥,命宣徽南院使、義成軍節度使曹彬為西南路行營馬步軍戰□翟都部署,山南東道節度使潘美為都監,穎州團練使曹翰為先鋒都指揮使,將兵十萬出荊南,以伐江南。將行,召曹彬、潘美,戒之曰:「城陷之日,慎無殺戮。設若困鬥,則李煜一門,不可加害。」丁卯,以知制誥李穆為江南國信使。冬十月甲申,幸迎春苑,登汴堤觀戰艦東下。丙戌,又幸迎春苑,登汴堤觀諸軍習戰,遂幸東水門,發戰□翟東下。江南進絹數萬,
 御衣、金帶、器用數百事。壬辰,曹彬等將舟師、步騎發江陵,水陸並進。丁酉,命吳越王錢俶為升州東南行營招撫制置使。己亥,曹彬收下峽口,獲指揮使王仁震、王宴、錢興。閏月己酉,克池州。丁巳,敗江南軍於銅陵。庚申,命宰相、參知政事更知日歷。壬戌,彬等拔蕪湖、當塗兩縣,駐軍採石。癸亥,詔減湖南新制茶。甲子,薛居正等上新編《五代史》,賜器幣有差。丁卯,彬敗江南軍於採石,擒兵馬部署楊收、都監蔡震等千人,為浮梁以濟。十一月癸
 未,黥李從善部下及江南水軍一千三百九十人為歸化軍。甲申,詔省劍南、山南等道屬縣主簿。丁亥,秦、晉旱,免蒲、陜、晉、絳、同、解六州逋賦,關西諸州免其半。己丑,知漢陽軍李恕敗江南水軍於鄂。甲午,曹彬敗江南軍於新林砦。辛丑,命知雄州孫全興答涿州修好書。壬寅,大食國遣使獻方物。十二月己酉,彬敗江南軍於白鷺洲。辛亥,命近臣祈雪。甲子,吳越王帥兵圍常州,獲其人馬,尋拔利城砦。丙寅,彬敗江南軍於新林港。己巳,左拾遺
 劉祺坐受賂,黥面、杖配沙門島。庚午,北漢寇晉州,守臣武守琦敗之於洪洞。壬申,吳越王敗江南軍於常州北界。



 八年春正月甲戌朔,以出師,不御殿。丙子,知池州樊若水敗江南軍於州界,田欽祚敗江南軍於溧水,斬其都統使李雄。乙酉,禦長春殿,謂宰相曰:「朕觀為臣者比多不能有終,豈忠孝薄而無以享厚福耶?」宰相居正等頓首謝。庚寅,曹彬拔升州城南水砦。二月癸丑,彬敗江南
 軍於白鷺洲。乙卯,拔升州關城。丁巳,太子中允徐昭文坐抑人售物,除籍。甲子,知揚州侯陟敗江南軍於宣化鎮。戊辰,覆試進士於講武殿,賜王嗣宗等三十一人、諸科紀自成等三十四人及第。三月乙酉,賜王嗣宗等宴錢二十萬。己丑,命祈雨。庚寅,彬敗江南軍於江北。己亥,契丹遣使克沙骨慎思以書來講和。知潞州藥繼能拔北漢鷹澗堡。辛丑,召契丹使於講武殿觀習射。壬寅,遣內侍王繼恩領兵赴升州。大食國遣使來朝獻。夏四月乙
 巳,幸東水磑。癸丑,幸都亭驛,閱新戰船。丁巳,吳越王拔常州。壬戌,彬等敗江南軍於秦淮北。戊辰,幸玉津園,觀種稻,遂幸講武池,觀習水戰。庚午,詔嶺南盜贓滿十貫以上者死。幸西水磑。五月壬申朔,以吳越國王錢俶守太師、尚書令,益食邑。知桂陽監張侃發前官隱沒羨銀,追罪兵部郎中董樞、右贊善大夫孔璘,殺之,太子洗馬趙瑜杖配海島;侃受賞,遷屯田員外郎。辛巳,祈晴。甲申,江南寧遠軍及沿江砦並降。乙酉,詔武岡、長沙等十縣
 民為賊鹵掠者,蠲其逋租,仍給復一年。甲午,安南都護丁璉遣使來貢。辛丑,河決濮州。六月壬寅,曹彬等遣使言,敗江南軍於其城下。丁未,宋州觀察判官崔絢、錄事參軍馬德休並坐贓棄市。辛亥,河決澶州頓丘。甲子,彗出柳,長四丈,辰見東方。秋七月辛未朔,日有食之。庚辰,遣閣門使郝崇信、太常丞呂端使契丹。癸未,西天東印土王子穰結說囉來朝獻。甲申,詔吳越王班師。己亥,山後兩林鬼主、懷化將軍勿尼等來朝獻。八月乙卯,幸東
 水磑觀魚,遂幸北園。辛酉,詔權停今年貢舉。壬戌,契丹遣左衛大將軍耶律霸德等致御衣、玉帶、名馬。西南蕃順化王子若廢等來獻名馬。癸亥,丁德裕敗潤州兵於城下。九月壬申,狩近郊,逐兔,馬蹶墜地,因引佩刀刺馬殺之。既而悔之,曰:「吾為天下主,輕事畋獵,又何罪馬哉!」自是遂不復獵。戊寅,潤州降。冬十月己亥朔,江南主遣徐鉉、周惟簡來乞緩師。辛亥,詔郡國令佐察民有孝悌力田、奇材異行或文武可用者遣,詣闕。丁巳,修西京宮
 闕。江南主貢銀五萬兩、絹五萬匹,乞緩師。戊午,改潤州鎮海軍節度為鎮江軍節度。幸晉王北園。己未,曹彬遣都虞候劉遇破江南軍於皖口,擒其將朱令贇、王暉。十一月辛未,江南主遣徐鉉等再奉表乞緩師,不報。甲申,曹彬夜敗江南軍於城下。丙戌,以校書郎宋準、殿直邢文慶充賀契丹正旦使。乙未,曹彬克升州,俘其國主煜,江南平,凡得州十九、軍三、縣一百八十、戶六十五萬五千六十。臨視新龍興寺。十二月庚子,幸惠民河,觀築堰。
 辛丑,赦江南,復一歲;兵戈所經,二歲。戊申,三佛齊遣使來獻方物。己酉,幸龍興寺。辛亥,免開封府諸縣今年秋租十之三。己未,以恩赦侯劉鋹為彭城郡公。甲子,契丹遣使耶律烏正來賀正旦。丁卯,吳越國王乞以長春節朝覲,從之。



 九年春正月辛未,御明德門,見李煜於樓下,不用獻俘儀。壬申,大赦,減死罪一等。乙亥,封李煜為違命侯,子弟臣僚班爵有差。己卯,江南昭武軍節度使留後盧絳焚
 掠州縣。庚辰,詔郊西京。癸巳,晉王率文武上尊號,不允。二月癸卯,三上表,不允。庚戌,以曹彬為樞密使。辛亥,命德昭迎勞吳越國王錢俶於宋州。契丹遣使耶律延以御衣、玉帶、名馬、散馬、白鶻來賀長春節。乙卯,吳越王奏內客省使丁德裕貪狠,貶房州刺史。丁巳,觀禮賢宅。戊午,以盧多遜為吏部侍郎,仍參知政事。己未,吳越國王錢俶偕子惟浚等朝於崇德殿,進銀絹以萬計。賜俶衣帶鞍馬,遂以禮賢宅居之,宴於長安殿。壬戌,錢俶進
 賀平升州銀絹、乳香、吳綾、紬綿、錢茶、犀象、香藥,皆億萬計。甲子,召晉王、吳越國王並其子等射於苑中,俶進御衣、壽星通犀帶及金器。丁卯,幸禮賢宅,賜俶金器及銀絹倍萬。三月己巳,俶進助南郊銀絹、乳香以萬計。庚午,賜俶劍履上殿,詔書不名。癸酉,以皇子德芳為檢校太保、貴州防禦使,中書侍郎、同平章事沉義倫為大內都部署,右衛大將軍王仁贍權判留司、三司兼知開封府事。丙子,幸西京。己卯,次鞏縣,拜安陵,號慟隕絕者久之。
 庚辰,賜河南府民今年田租之半,奉陵戶復一年。辛巳,至洛陽。庚寅,大雨,分命近臣詣諸祠廟祈晴。辛卯,幸廣化寺,開無畏三藏塔。夏四月己亥,雨霽。庚子,有事圓丘,回禦五鳳樓,大赦,十惡、故殺者不原,貶降責免者量移敘用,諸流配及逋欠悉放,諸官未贈恩者悉覃賞。壬寅,大宴,賜親王、近臣、列校襲衣、金帶、鞍馬、器幣有差。丙午,駕還。辛亥,上至自洛。丁巳,曹翰拔江州,屠之,擒牙校宋德明、胡則等。詔益晉王食邑,光美、德昭並加開府儀同
 三司,德芳益食邑,薛居正、沉義倫加光祿大夫,樞密使曹彬、宣徽北院使潘美加特進,吳越國王錢俶益食邑,內外文武臣僚咸進階封。己未,著令旬假為休沐。丙寅,大食國王珂黎拂遣使蒲希密來獻方物。五月己巳,幸東水磑,遂幸飛龍院,觀漁金水河。甲戌,遣司勛員外郎和峴往江南路採訪。殺盧絳。庚辰,幸講武池,遂幸玉津園觀稼。宋州大風,壞城樓、官民舍幾五千間。甲申,以閣門副使田守奇等充賀契丹生辰使。晉州以北漢嵐、石、
 憲三州巡檢使王洪武等來獻。六月庚子,步至晉王邸,命作機輪,挽金水河注邸中為池。癸卯,吳越王進銀、絹、綿以倍萬計。乙卯,熒惑入南斗。秋七月戊辰,幸晉王第觀新池。丙子,幸京兆尹光美第視疾。戊寅,再幸光美第。泉州節度使陳洪進乞朝覲。丙戌,命近臣祈晴。丁亥,命修先代帝王及五岳、四瀆祠廟。庚寅,幸光美第。八月乙未朔,吳越國王進射火箭軍士。己亥,幸新龍興寺。辛丑,太子中允郭思齊坐贓棄市。乙巳,幸等覺院,遂幸東
 染院,賜工人錢。又幸控鶴營觀習射,賜帛有差。又幸開寶寺觀藏經。丁未,遣侍衛馬軍都指揮使黨進、宣徽北院使潘美伐北漢。丙辰,遣使率兵分五道入太原。九月甲子,幸綾錦院。庚午,權高麗國事王胄遣使來朝獻。黨進敗北漢軍於太原城北。辛巳,命忻、代行營都監郭進遷山後諸州民。庚寅,幸城南池亭,遂幸禮賢宅,又幸晉王第。冬十月甲午朔旦,賜文武百官衣有差。丁酉,兵馬監押馬繼恩率兵入河東界,焚蕩四十餘砦。己亥,幸西教
 場。庚子,鎮州巡檢郭進焚壽陽縣,俘九千人。辛丑,晉、隰巡檢穆彥璋入河東,俘二千餘人。黨進敗北漢軍於太原城北。己酉,吳越王獻馴象。癸丑夕,帝崩於萬歲殿,年五十。殯於殿西階,謚曰英武聖文神德皇帝,廟號太祖。太平興國二年四月乙卯,葬永昌陵。大中祥符元年,加上尊謚曰啟運立極英武睿文神德聖功至明大孝皇帝。



 帝性孝友節儉,質任自然,不事矯飾。受禪之初,頗好微行,或諫其輕出。曰:「帝王之興,自有天命,周世宗見諸
 將方面大耳者皆殺之,我終日侍側,不能害也。」既而微行愈數,有諫,輒語之曰:「有天命者任自為之,不汝禁也。」一日,罷朝,坐便殿,不樂者久之。左右請其故。曰:「爾謂為天子容易耶?早作乘快誤決一事,故不樂耳。」汴京新宮成,御正殿坐,令洞開諸門,謂左右曰:「此如我心,少有邪曲,人皆見之。」吳越錢俶來朝,自宰相以下咸請留俶而取其地,帝不聽,遣俶歸國。及辭,取群臣留俶章疏數十軸,封識遺俶,戒以途中密觀,俶屆途啟視,皆留己不遣
 之章也。俶自是感懼,江南平,遂乞納土。南漢劉鋹在其國,好置酖以毒臣下。既歸朝,從幸講武池,帝酌卮酒賜鋹。鋹疑有毒,捧杯泣曰:「臣罪在不赦,陛下既待臣以不死,願為大梁布衣,觀太平之盛,未敢飲此酒。」帝笑而謂之曰:「朕推赤心於人腹中,寧肯爾耶?」即取鋹酒自飲,別酌以賜鋹。王彥升擅殺韓通,雖預佐命,終身不與節鉞。王全斌入蜀,貪恣殺降,雖有大功,即加貶絀。宮中葦簾,緣用青布;常服之衣,浣濯至再。魏國長公主襦飾翠羽,
 戒勿復用,又教之曰:「汝生長富貴,當念惜福。」見孟昶寶裝溺器,樁而碎之,曰:「汝以七寶飾此,當以何器貯食?所為如是,不亡何待!」晚好讀書,嘗讀二典,嘆曰:「堯、舜之罪四兇,止從投竄,何近代法網之密乎!」謂宰相曰:「五代諸侯跋扈,有枉法殺人者,朝廷置而不問。人命至重,姑息藩鎮,當若是耶?自今諸州決大闢,錄案聞奏,付刑部覆視之。」遂著為令。乾德改元,先諭宰相曰:「年號須擇前代所未有者。」三年,蜀平,蜀宮人入內,帝見其鏡背有志「乾
 德四年鑄」者,召竇儀等詰之。儀對曰:「此必蜀物,蜀主嘗有此號。」乃大喜曰:「作相須讀書人。」由是大重儒者。受命杜太后,傳位太宗。太宗嘗病亟,帝往視之,親為灼艾,太宗覺痛,帝亦取艾自灸。每對近臣言:「太宗龍行虎步,生時有異,他日必為太平天子,福德吾所不及雲。」



 贊曰:昔者堯、舜以禪代,湯、武以征伐,皆南面而有天下。四聖人者往,世道升降,否泰推移。當斯民塗炭之秋,皇天眷求民主,亦惟責其濟斯世而已。使其必得四聖人
 之才,而後以其行事畀之,則生民平治之期,殆無日也。五季亂極,宋太祖起介冑之中,踐九五之位,原其得國,視晉、漢、周亦豈甚相絕哉?及其發號施令,名藩大將,俯首聽命,四方列國,次第削平,此非人力所易致也。建隆以來,釋藩鎮兵權,繩贓吏重法,以塞濁亂之源。州郡司牧,下至令錄、幕職,躬自引對。務農興學,慎罰薄斂,與世休息,迄於丕平。治定功成,制禮作樂。在位十有七年之間,而三百餘載之基,傳之子孫,世有典則。遂使三代而
 降,考論聲明文物之治,道德仁義之風,宋於漢、唐,蓋無讓焉。烏呼,創業垂統之君,規模若是,亦可謂遠也已矣!



\end{pinyinscope}