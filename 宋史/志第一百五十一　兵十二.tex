\article{志第一百五十一 兵十二}

\begin{pinyinscope}

 馬政



 國馬之牧,歷五代浸廢,至宋而規制備具。自建隆而後,其官司之規,廄牧之政,與夫收市之利,牧地之數,支配之等,曰券馬,曰省馬,曰馬社,曰括買,沿革盛衰,皆可得
 而考焉。



 凡御馬之等三,入殿祗候十五匹,引駕十四匹,從駕二十匹。



 給用之等十有五,曰揀中,曰不得支使,曰添價,曰明信,曰臣僚,曰諸班,曰御龍直,曰捧日、龍衛,曰拱聖,曰驍騎,曰雲、武騎,曰天武、龍猛,曰配軍,曰雜使,曰馬鋪。



 群號之字十有七,曰「左」,曰「右」,曰「千」,曰「立」,曰「水」,曰「官」,曰「吉」,曰「天」,曰「主」,曰「王」,曰「方」,曰「與」,曰「來」,曰「萬」,曰「小」,曰「囗官」,曰「退」。



 毛物之種九十有二,叱撥之別八,青之別二,白之別一,烏之別五,赤之別五,紫之別六,駿之別十一,赭白之別六,騮之別八,騧之別六,駱之別五,騅之別五,□俞之別八,駁胯之別六,駁之別三,驃之別七。



 其官司之規,則太祖承前代之制,初置左、右飛龍二院,以左、右飛龍二使領之。太平興國五年,改飛龍為天廄坊。雍熙四年,改天廄
 為左、右騏驥院,左右天駟監四、左右天廄坊二皆隸焉。



 真宗咸平元年,創置估馬司。凡市馬,掌辨其良駑,平其直,以分給諸監。



 三年,置群牧使,以內臣勾當制置群牧司,京朝官為判官。



 景德二年,改諸州牧龍坊悉為監,賜名,鑄印以給之。在外之監十有四:大名曰大名,洺州曰廣平,衛州曰淇水,並分第一、第二。河南曰洛陽,鄭州曰原武,同州曰沙苑,相州曰安陽,澶州曰鎮寧,邢州曰安國,中牟曰淳澤,許州曰單鎮。



 四年,以知樞密院陳堯叟
 為群牧制置使,又別置群牧使副、都監,增判官為二員。凡廄牧之政,皆出於群牧司,自騏驥院而下,皆聽命焉。諸州有牧監,知州、通判兼領之,諸監各置勾當官二員。又置左右廂提點。又置牧養上下監,以養療京城諸坊、監病馬。又詔左右騏驥院諸坊、監官,並以三年為滿;如習知馬事願留者,群牧司以聞,而徙蒞他監焉。



 其廄牧之政,則自太祖置養馬務一,葺舊務四,以為牧放之地始。



 太平興國四年,太宗觀兵於幽,得汾、晉、燕、薊之馬四
 萬二千餘匹,內皂充牣,始分置諸州牧養之。時殿直李諤坐贓,監牧許州,盜官菽,馬多死,並主吏斬於市。又詔擇豐曠地置牧龍坊八,以便牧養。



 淳化二年十二月,詔圉人取善馬數十匹,於便殿設皂棧,教以秣飼,且以其法諭宰執,仍頒於諸軍。復以醫馬良方賜近臣。嘗從趙守倫之請,於諸州牧龍坊畜牝馬萬五千匹,逐水草牧放,不費芻秣,生駒蕃息,足資軍用。至是,守倫復言:「諸坊牧馬萬匹,歲當生駒四千,今歲止二千五百,典司失職,
 當嚴責罰。若馬百匹歲得駒七十,則加遷擢。諸坊產駒,即籍以聞。牧放軍人,當募少壯充役。」並從之。



 真宗大中祥符元年,立牧監賞罰之令,外監息馬,一歲終以十分為率,死一分以上勾當官罰一月奉,餘等第決杖。牧倍多而死少者,給賞緡有差。凡生駒一匹,兵校而下賞絹一匹。當是時,凡內外坊、監及諸軍馬凡二十餘萬匹,飼馬兵校一萬六千三十八人。每歲京城草六十六萬六千圍,麩料六萬二千二百四石,鹽、油、藥、糖九萬五千餘
 斤、石,諸州軍不預焉。左右騏驥六坊、監止留馬二千餘匹,皆春季出就牧,孟冬則別其羸病,就棧皂養飼。其尚乘之馬,唯備用者在焉。



 凡牧監之在河南、北,天禧後,靈昌監為河決所沖。至乾興、天聖間,兵久不試,言者多以為牧馬費廣而亡補,乃廢東平監,以其地賦民。五年,廢單鎮監。六年,廢洛陽監。於是河南諸監皆廢,悉以馬送河北。既而詔取原武監馬赴京師,移河北孳生馬牧於原武。



 八年,群牧司上言:原武地廣而馬少,請增牧數。詔
 以淇水第二監四歲馬屬原武,歲取河北孳生四歲馬分屬淇水第二並原武監,移原武下等馬牧於靈昌鎮廢監,仍隸原武。



 九年,詔諸監孳生駁馬,四時游牧,勿復登廄。



 明道元年,議者謂:「自河南六監廢,京師須馬,取之河北,道遠非便。」詔遣左廂提點王舜臣往度利害。舜臣言:「鎮寧、靈昌、東平、淳澤四監雖廢,然其地猶牧本監並騏驥院馬,洛陽、單鎮去京師近,罷之非便。」乃詔復二監,以牧河北孳生馬。



 景祐二年,揀河北諸監馬一千九百
 牧於趙州界,隸安陽監。既而詔廣平廢監留其一,以趙州界牧馬復隸焉,所餘一監,毋毀廄舍。



 四年,復以原武第二監為單鎮,移於長葛縣,以縣令、都監兼領之。三年,詔院坊、監馬歲留備用外,餘為兩群,牧於咸豐門外牟駝岡。



 凡收養病馬,估馬司、騏驥院取病淺者送上監,深者送下監,分十槽醫療之。天聖六年,詔月以都監、判官一人提舉。八年,言者謂上監去京城遠,送病馬非便。詔廢之,以病淺馬分屬左右騏驥院六坊、監,季較拋死數,
 歲終第賞罰。更以騏驥院官迭往提舉。



 明道二年,復置上監,易名天坰,養無病馬,病馬並屬下監。



 景祐二年,詔以牧養監馬團群牧於陳、許州界鳳凰陂,免耗芻菽,歲以為常。



 治平二年,詔院坊、監馬之病不堪估賣者,送淇水第一監,別為一群以牧養之。



 凡馬之孳生,則大名府、洺、衛、相州七監多擇善種,合牝牡為群,判官歲以十二月巡行坊、監,閱二歲駒點印,第賞牧兵。諸軍收駒及二歲,即送官。



 天聖七年,群牧司言:「舊制,知州軍、通判領同
 群牧事,歲終較馬死數及分已上,並生駒不及四分,並罰奉。死數少,生駒多,即奏第賞。三歲都比,以該賞者聞。今請申明舊制,通判始到官,書所轄馬數,歲一考之,官滿,較總數為賞罰。」詔從之。



 嘉祐八年,群牧司言:「孳生七監,每監歲定牝馬二千,牡馬四百,歲約生駒四百,以為定數。」



 治平二年,詔:「諸監生駒滿三十月已上,每歲點印,選牡之良者送淇水第二監,餘雜大馬悉送河南三監,其淇水第二監馬,候滿六十月,給配諸監。諸監牝馬滿
 三十月,本監別立群牧放,候滿五十月,乃撥配他監。」



 凡收市馬,戎人驅馬至邊,總數十、百為一券,一馬預給錢千,官給芻粟,續食至京師,有司售之,分隸諸監,曰券馬。邊州置場,市蕃漢馬團綱,遣殿侍部送赴闕,或就配諸軍,曰省馬。陜西廣銳、勁勇等軍,相與為社,每市馬,官給直外,社眾復裒金益之,曰馬社。軍興,籍民馬而市之以給軍,曰括買。



 宋初,市馬唯河東、陜西、川峽三路,招馬唯吐蕃、回紇、黨項、藏牙族,白馬、鼻家、保家、名市族諸蕃。至
 雍熙、端拱間,河東則麟、府、豐、嵐州、岢嵐、火山軍、唐龍鎮、濁輪砦,陜西則秦、渭、涇、原、儀、延、環、慶、階州、鎮戎、保安軍、制勝關、浩亹府,河西則靈、綏、銀、夏州,川峽則益、文、黎、雅、戎、茂、夔州、永康軍,京東則登州。自趙德明據有河南,其收市唯麟、府、涇、原、儀、渭、秦、階、環州、岢嵐、火山、保安、保德軍。其後置場,則又止環、慶、延、渭、原、秦、階、文州、鎮戎軍而已。



 太祖時,歲遣中使詣邊州市馬。先是,兩河之民入蕃界盜馬入中國。官給其直。時方留意撫綏,詔禁之。



 太平
 興國四年,詔市吏民馬十七萬匹。六年,詔內屬戎人驅馬詣闕下者,首領縣次續食,且禁富民無得私市。十二月,詔:「蕃部鬻馬,官取良而棄弩,又禁其私市,歲入數既不充,且無以懷遠人。自今委長吏謹視馬之良駑,駑即印識之,許民私市焉。」先是,以銅錢給諸蕃馬直。八年,有司言戎人得錢,銷鑄為器,乃以布帛茶及他物易之。



 天禧中,宰相向敏中言國馬倍於先朝,廣費芻粟。乃詔以十三歲以上配軍馬估直出賣。先是市馬以三歲已上、十
 三歲已下為率。天聖中,詔市四歲已上、十歲已下。既而所市不足,群牧司以為言,乃詔入券並省馬市三歲已上、十二歲已下。明年,詔府州、岢嵐軍自今省馬三歲、四歲者不以等第,五歲已上十二歲已下、骨格良善行者,悉許綱送估馬司,餘非上京省馬並送並州揀馬司。



 景祐元年,御史中丞韓億言:「蕃部以馬抵永康軍中賣,所得至少,徒使羌人知蜀山川道路,非計之得。」乃詔罷之。



 四年,群牧司奏河北諸軍闕馬,請制等杖六,付天雄軍、
 真定府、定、瀛、貝、滄州,市上生馬十二歲以下,視等第給直。馬自四尺七寸至四尺二寸,凡六等。其直自二萬五千四百五十至萬六千五百五十,課自萬三千四百五十至八千九百五十九,六等,取備邊兵戶絕錢充直。以第一等送京師,餘就配諸軍。



 康定初,陜西用兵,馬不足。詔京畿、京東西、淮南、陜西路括市戰馬,馬四尺六寸至四尺二寸,其直自五十千至二十千,凡五等。宰臣、樞密使聽畜馬七,參知政事、樞密副使五,尚書、學士至知雜、閣
 門使已上三,升朝官閣門祗候已上一,餘命官至諸司職員、寺觀主首皆一。節度使至刺史,殿前馬步軍都指揮至軍頭司散員、副兵馬使皆勿括。並邊七州軍免。出內庫珠償民馬直。又禁邊臣私市,闕者官給。二年,詔:「河北州軍置場市馬,雖除等樣,如聞所得不廣,宜加增直。第一等二萬八千,第二等二萬六千,第三等二萬四千,第四等以下及牝馬即依舊直。仍自第二等以下遞減一寸。」



 慶歷四年,詔:「河北點印民間馬,凡收市外,見餘二
 萬七百,除坊郭戶三等、鄉村三等已上養飼如舊,餘點印者悉集揀市。」五年,出內藏庫絹二十萬,市馬於府州、岢嵐軍。六年,詔陜西、河東社馬死者,本營鬻錢以助馬直。



 至和元年,詔;「蜀馬送京師,道遠多病瘠。自今以春、秋、冬部送陜西四路總管司。」二年,修陜西蕃馬驛,群牧司每季檄沿路郡縣察視之。邊州巡檢兵校,聽自市馬,官償其直。又詔陜西轉運使司以銀十萬兩市馬於秦州,歲以為常。



 嘉祐元年,詔三司出絹三萬,市馬於秦州以
 給河東軍。五年,薛向言:「秦州券馬至京師,給直並路費,一馬計錢數萬。請於原、渭州、德順軍置場收市,給以解鹽交引,即不耗度支縑錢。其券馬姑存,以來遠人。歲可別得良馬八千,以三千給沿邊軍騎,五千入群牧司。」七年,陜西提舉買馬監牧司奏:「舊制,秦州蕃漢人月募得良馬二百至京師,給彩絹、銀碗、腰帶、錦襖子,蕃官、回紇隱藏不引至者,並以漢法論罪。歲募及二千,給賞物外,蕃部補蕃官,蕃官轉資,回紇百姓加等給賞。今原、渭、德
 順軍置場市馬,請如秦州例施行。」詔從之。先是,詔議買馬利害。吳奎等議於秦州古渭、永寧砦及原州、德順軍各令置場,京師歲支銀四萬兩、紬絹七萬五千匹充馬直,不足,以解鹽鈔並雜支錢給之。詔行之。八年,宰臣韓琦言:「秦州永寧砦舊以鈔市馬,自修古渭砦,在永寧之西,而蕃漢多互市其間,因置買馬場,凡歲用緡錢十餘萬,蕩然流入虜中,實耗國用。」詔復置場永寧,罷古渭砦中場。蕃部馬至,徑鬻於秦州。



 治平元年,薛向請原、渭州、
 德順軍買馬官,永興軍養馬務,如原州、德順軍並渭州同判,三年為任,悉以所市馬多少為殿最。又言:「秦州山外蕃部至原、渭州、德順軍、鎮戎軍鬻馬,充豪商錢,至秦州,所償止得六百。今請於原、渭州、德順軍,官以鹽鈔博易,使得輕繼至秦州,易蜀貨以歸。蜀商以所博鹽引至岐、雍,換監銀入蜀,兩獲其便。」群牧司請如向言施行。是歲,詔河東陜西廣銳、蕃落闕馬,復置社買,一馬官給錢三十千。久之,馬不至,乃增直如慶歷詔書,第三等三十
 五千,第四等二十八千。四年,以成都府路歲輸紬絹三萬給陜西監牧司。自是蕃部馬至者眾,官軍仰給焉。先是,以陜西轉運使兼本路監牧買馬事,後又以制置陜西解鹽官同主之。



 大抵國初市馬,歲僅得五千餘匹。天聖中,蕃部省馬至三萬四千九百餘匹。嘉祐以前,原、渭、德順凡三歲市馬至萬七千一百匹,秦州券馬歲至萬五千匹。



 凡牧地,自畿甸及近郡,使擇水草善地而標占之。淳化、景德間,內外坊、監總六萬八千頃,諸軍班又三
 萬九百頃不預焉。歲久官失其籍,界堠不明,廢置不常,而淪於侵冒者多矣。



 淳化二年十二月,通利軍上《十牧草地圖》,上慮侵民田,遣中使檢視疆理。



 嘉祐中,韓琦請括諸監牧地,留牧外,聽下戶耕佃。遣都官員外郎高訪等括河北,得閑田三千三百五十頃募佃,歲約得穀十一萬七千八百石,絹三千二百五十匹,草十六萬一千二百束。群牧司言:「諸監牧地間有水旱,每監牧放外,歲刈白草數萬束,以備冬飼。今悉賦民,異時監馬增多,及
 有水旱,無以轉徙牧放。」詔遣左右廂提點官相度,除先被侵冒已根括出地權給租佃,餘委群牧司審度存留,有閑土即募耕佃。五年,群牧司言:「凡牧一馬,往來踐食,占地五十畝。諸監既無餘地,難以募耕,請存留如故。廣平廢監先賦民者,亦乞取還。」乃詔:「河北、京東牧監帳管草地,自今毋得縱人請射,犯者論以違制。」



 群牧使歐陽修言:「唐之牧地,西起隴右金城、平涼、天水,外暨河曲之野,內則岐、幽、涇、寧,東接銀、夏,又東至於樓煩。今則沒入蕃界,
 淪於侵佃,不可復得。惟河東嵐、石之間,山荒甚多,汾河之側,草地亦廣,其間水草最宜牧養,此唐樓煩監地。跡此推之,則樓煩、元池、天池三監舊地,尚冀可得。臣往年出使,嘗行威勝以東及遼州、平定軍,其地率多閑曠。河東一路,水草甚佳,地勢高寒,必宜馬性。又京西唐、汝之間,荒地亦廣。請下河東、京西轉運司遣官審度,若可興置監牧,則河北諸監,尋可廢罷。」



 治平末,牧地總五萬五千,河南六監三萬二千,而河北六監則二萬三千。



 凡支
 配,騏驥院、估馬司以當配軍及新收馬閱於便殿,數毋過一百。凡配軍,視其奉錢之數,馬自四尺六寸至四尺三寸,奉錢自一千至三百,為四等,差次給之,至五月權止。外州軍士闕馬,先奏稟乃給。荊湖路歸遠、雄武軍士,配以在所土產馬。凡闕馬軍士,以分數配填。



 慶歷四年,詔陜西、河北、河東填五分,餘路填四分。他州軍、府界巡檢兵校聽自市,官償其直,毋過三十千。是歲,詔諸路以馬給軍士,比試武技,優者先給,比試兩給;闕馬十匹以
 下全給,十匹以上如舊數支。



 至和元年,詔軍士戍陜西、河東、河北填七分,餘路填六分。凡主兵官當借馬者,至罷兵權。殿前馬步軍都指揮使賜所借馬三,都虞候、捧日、天武、龍、神衛四廂都指揮使二,軍都指揮使一。外州在官當借馬者,經略使三,總管、鈐轄二,路分都監、承受、極邊砦至監押、都巡檢、把截、保丁指揮一,毋得乘之他州並以假人,犯者論以違制。



 寶元元年,詔群臣例賜馬者,宰相至樞密直學士,使相至正任刺史,並皇族緣姻
 事當賜者,如舊制;餘給以馬直,少卿監已上三十五千,內殿承制已下二十三千。凡群臣假官馬進奉者,置籍報左藏庫,償直四十千,其後多負不償。乃詔借馬者先輸直,久逋不償者克其奉料。



 熙寧以來,有保馬、戶馬,其後又變為給地牧馬。



 神宗嘗患馬政不善,謂樞密使文彥博曰:「群牧官非人,無以責成效。其令中書擇使,卿舉判官,冀國馬蕃息,以給戰騎。」於是以比部員外郎崔臺符權群牧判官,又命群牧判官劉航及臺符刪定《群牧
 敕令》,以唐制參本朝故事而奏決焉。



 熙寧元年,又手詔彥博等曰:「今諸州守貳雖同領群牧,而未嘗親蒞職事,其議更制,應監牧、郡守貳並朝廷選授,與坊、監使臣皆第其能否,制賞罰而升黜之,宜立法以聞。」又手詔曰:「方今馬政不修,官吏無著效,豈任不久而才不盡歟?是何監牧之多,官吏之眾,而乏才之甚也!昔唐用張萬歲三世典群牧,恩信行乎下,故馬政修舉,後世稱為能吏。今上自提總官屬,下至坊、監使臣,既非銓擇,而遷徙迅速,
 謂之『假道』,欲使官宿其業而盡其能,不可得也。為今之計者,當簡其勞能,進之以序。自坊、監而上至於群牧都監,皆課其功而第進之,以為任事者勸焉。」於是,樞密副使邵亢請以牧馬餘田修稼政,以資牧養之利。而群牧司言:「馬監草地四萬八千餘頃,今以五萬馬為率,一馬占地五十畝,大名、廣平四監餘田無幾,宜且仍舊。而原武、單鎮、洛陽、沙苑、淇水、安陽、東平等監,餘良田萬七千頃,可賦民以收芻粟。」從之。



 已而樞密院又言:「舊制,以左、
 右騏驥院總司國馬。景德中,始增置群牧使副、都監、判官,以領廄牧之政。使領雖重,未嘗躬自巡察,不能周知牧畜利病,以故馬不蕃息。今宜分置官局,專任責成。」乃詔河南、北分置監牧使,以劉航、崔臺符為之,又置都監各一員。其在河陽者,為孳生監。凡外諸監並分屬兩使,各條上所當行者。諸官吏若牧田縣令佐,並委監牧使舉劾,專隸樞密院,不領於群牧制置。先是,群牧司請於河北、河東、陜西都總管治所各置一監,以便給軍,乃遣
 官下諸路詳度。既又以知太原唐介之請,發沙苑馬五百,置監於交城。又分置河南、河北兩使。時上方留意牧監地,然諸監牧田皆寬衍,為人所冒占,故議者爭請收其餘資以佐芻粟。言利者乘之,始以增賦入為務。



 二年,詔括河南北監牧司總牧地。舊籍六萬八千頃,而今籍五萬五千,餘數皆隱於民。自是,請以牧地賦民者紛然,而諸監尋廢。是歲,天下應在馬凡十五萬三千六百有奇。



 初,內外班直、諸軍馬以四月下槽出牧,迄八月上槽,
 風雨勞逸之不齊,故多病斃。圉人歲被榜罰,吏緣牧事害民,棚井科率無寧歲。四年十月,乃命同修起居注曾孝寬較度其利害。孝寬請罷諸班直、諸軍馬出牧,以田募民出租。詔自來年如所請,仍令三司備當牧五月芻粟。



 五年,廢太原監。七年,廢東平、原武監,而合淇水兩監為一。八年,遂廢河南北八監,惟存沙苑一監,而兩監司牧亦罷矣。沙苑先以隸陜西提舉監牧,至是,復屬之群牧司。



 始議廢監時,群牧制置使文彥博言:「議者欲賦牧
 地與民而收租課,散國馬於編戶而責孳息,非便。」詔元絳、蔡確較其利害上之。於是中書、樞密院言:「河南北十二監,起熙寧二年至五年,歲出馬一千六百四十匹,可給騎兵者二百六十四,餘僅足配郵傳。而兩監牧吏卒雜費及所占地租,為緡錢五十三萬九千有奇,計所出馬為錢三萬六千四百餘緡而已。今九監見馬三萬,若不更制,則日就損耗。」於是卒廢之,以其善馬分隸諸監,余馬皆斥賣,收其地租,給市易茶本錢,分寄籍常平、出子
 錢,以為市馬之直。監兵五千,以為廣固指揮,修治京城焉。後遂廢高陽、真定、太原、大名、定州五監。凡廢監錢歸市易之外,又以給熙河歲計。



 諸監既廢,淤田司請廣行淤溉,增課以募耕者,而河北制置牧田所繼言,牧田沒於民者五千七百餘頃。乃嚴侵冒之法,而加告獲之賞,自是利入增多。元豐三年,廢監租錢遂至百一十六萬,自群牧使而下,賜賚有差。乃命太常博士路昌衡、秘書丞王得臣與逐路轉運司、開封府界提點司按租地,約
 三年中價以定歲額。若催督違滯,以擅支封樁法論。



 初,經制熙河邊防財用司奏於岷州床川荔川閭川砦、通遠軍熟羊砦置牧養十監,議者繼言蕃馬法,帝欲試之近甸。六年,手詔樞密院:「牧馬重事,經始之際,宜得左右近臣以總其政。今自霧澤陂牧馬所造法,始於畿內置十監,以次推之諸路。宜令樞密院都承旨張誠一、副都承旨張山甫經度制置,權不隸尚書駕部及太僕寺。有當自朝廷處分者,樞密院主之。」已而其說皆不效。八年,
 同提舉經度制置曹誦言:「自崇儀副使溫從吉建議創孳生監,迨今二年,駒不蕃而死者益眾。」乃命御史臺校核,自置監以來,得駒不及一分四厘,馬死已十分之六。於是責議者及提舉官,而罷畿內十監。



 元祐初,議興廢監,以復舊制。於是詔庫部郎中郭茂恂往陜西、河東所當置監,尋又下河北陜西轉運、提點刑獄司按行河、渭、並、晉之間牧田以聞。時已罷保甲,教騎兵,而還戶馬於民。於是右司諫王巖叟言:「兵之所恃在馬,而能蕃息之
 者,牧監也。昔廢監之初,識者皆知十年之後天下當乏馬。已而不待十年,其弊已見,此甚非國之利也。乞收還戶馬三萬,復置監如故,監牧事委之轉運官,而不專置使。今鄆州之東平,北京之大名、元城,衛州之淇水,相州之安陽,洺州之廣平監,以及瀛、定之間棚塞草地疆畫具存,使臣牧卒大半猶在,稍加招集,則指顧之間措置可定,而人免納錢之害,國收牧馬之利,豈非計之得哉?又況廢監以來,牧地之賦民者,為害多端,若復置監牧
 而收地入官,則百姓戴恩,如釋重負矣。」自是,洛陽、單鎮、原武、淇水、東平、安陽等監皆復。



 初,熙寧中,並天駟四監為二,而左、右天廄坊亦罷。至是,復左、右天廄坊。時又有旨,內外馬事並隸太僕寺,不由駕部而達尚書省。兵部尚書王存、右司諫王覿言:「先帝講求歷代之法,正省、臺、寺、監之職,上下相繼,各有統制。其間或濡滯不通,宜量加裁正,不可因而隳紊。」言不果行。又詔舊屬群牧司者專隸太僕寺,直達樞密院,不由尚書省及駕部。至崇寧
 中,始詔如元豐舊制。



 紹聖初,用事者更以其意為廢置,而時議復變。太僕寺言,府界牧田,占佃之外,尚存三千餘頃,議復畿內孳生十監。詔以莊宅副使麥文昺、內殿崇班王景儉充提舉。後二年而給地牧馬之政行矣。



 先是,知任縣韓筠等建議,凡授民牧田一頃,為官牧一馬而蠲其租。縣籍其高下、老壯、毛色,歲一閱,亡失者責償,已佃牧田者依上養馬。知邢州張赴上其說,且謂授田一頃為官牧一馬,較陜西沿邊弓箭脾既養馬又戍邊
 者為優,試之一監一縣,當有利而無害。樞密院是其請,且言:「熙寧中,罷諸監以賦民,歲收緡錢至百餘萬。元祐初,未嘗講明利害,惟務罷元豐、熙寧之政,奪已佃之田而復舊監。桑棗井廬多所毀伐,監牧官吏為費不貲,牧卒擾民,棚井抑配,為害非一。蓋自復監以來,臣僚屢陳公私之害。若循元祐倉卒更張之法,久當益弊。且左右廂今歲籍馬萬三千有奇,堪配軍者無幾,惟沙苑六千疋愈於他監。今赴等所陳授田養馬,既蠲其租不責以
 孳息,而不願者無所抑勒,又限以尺寸,則緩急皆可用之馬矣。」乃具為條畫,下太僕寺,應監牧州縣悉行之。



 時殿中侍御史陳次升言:「給地牧馬,其初始於邢州守令之請,未嘗下監司詳度。諸路各有利害,既不可知。民居與田相遠者,難就耕牧。一頃之地所直不多,而亡失責償,為錢四五十千,必非人情所願。」言竟不行。時同知樞密院者,曾布也。



 四年,遂廢淇水、單鎮、安陽、洛陽、原武監,罷提點所及左右廂,惟存東平、沙苑二監。曾布自敘其
 事曰:「元祐中,復置監牧,兩廂所養馬止萬三千匹,而不堪者過半。今既以租錢置蕃落十指揮於陜西,養馬三千五百。又人戶願養者亦數千,而所存兩監各可牧萬馬。馬數多於舊監,而所省官吏之費非一,近世良法,未之能及。」時三省皆稱善。其後,沙苑復隸陜西買馬監牧司,而東平監仍廢。



 崇寧元年,有司較諸路田養馬之數,凡一千八百疋有奇,而河北西路占一千四百,他路自二百匹以下,至河東路僅九匹,而開封府界、京西南路、
 京東東路皆無應募者。蓋法雖已具,而猶未及行也。



 大觀元年,尚書省言:「元祐置監,馬不蕃息,而費用不貲。今沙苑最號多馬,然占牧田九千餘頃,芻粟、官曹歲費緡錢四十餘萬,而牧馬止及六千。自元符元年至二年,亡失者三千九百。且素不調習,不中於用。以九千頃之田、四十萬緡之費,養馬而不適於用,又亡失如此,利害灼然可見。今以九千頃之田,計其磽瘠,三分去一,猶得良田六千頃。以直計之,頃為錢五百餘緡,以一頃募一馬,
 則人得地利,馬得所養,可以紹述先帝隱兵於農之意。請下永興軍路提點刑獄司及同州詳度以聞。俟見實利,則六路新邊閑田,當以次推行。時熙河蘭湟路牧馬司又請兼募願養牝馬者,每收三駒,以其二歸官,一充賞,詔行之。是歲,臣僚言岷州應募養馬者至萬餘匹,於是自守貳而下,遞賞有差。明年,詔熙河路應縣、鎮、城、砦、關、堡官並兼管幹給地牧事。四年,復罷京東西路給地牧馬,復東平監。



 政和二年,詔諸路復行給地牧馬,復罷
 東平監。五年,提舉河東給地牧馬尚中行以奏報稽違,且欲擅更法,詔授遠小監當官。於是人皆趣令,牧守、提舉以率先就緒遷官第賞者甚眾。七年,有司言給地增牧,法成令具,諸路告功。乃下諸路春秋集教,以備選用。令下,奉行之者益力。



 蔡京既罷政,新用事者更言其不便。宣和二年,詔罷政和二年以來給地牧馬條令,收見馬以給軍,應牧田及置監處並如舊制。又復東平監。凡諸監興罷不一,而沙苑監獨不廢。自給地牧馬之法罷,
 三年而復行。時牧田已多所給占,乃詔見管及已拘收,如官司輒復請占者,以違制論。



 六年,又詔立賞格,應牧馬通一路及三千匹,州通縣及一千,縣及三百,其提點刑獄、守令各遷一官,倍者更減磨勘年。於是諸路應募牧馬者為戶八萬七千六百有奇,為馬二萬三千五百。既推賞如上詔,而兵部長貳亦以兼總八路馬政遷官。然北方有事,而馬政亦急矣。



 靖康元年,左丞李綱言:「祖宗以來,擇陜西、河東、河北美水草高涼之地,置監凡三
 十六所,比年廢罷殆盡。民間雜養以充役,官吏便文以塞責,而馬無復善者。今諸軍闕馬者太半,宜復舊制,權時之宜,括天下馬,量給其直,不旬日間,則數萬之馬,猶可具也。」然時已不能盡行其說矣。



 保甲養馬者,自熙寧五年始。先是,中書省、樞密院議其事於上前,文彥博、吳充言:「國馬宜不可闕。今法,馬死者責償,恐非民願。」安石以為令下而京畿投牒者已千五百戶,決非出於驅迫,持論益堅。五月,詔開封府界諸縣保甲願牧馬者聽,
 仍以陜西所市馬選給之。



 六年,曾布等承詔上其條約:凡五路義勇保甲願養馬者,戶一匹,物力高願養二匹者聽,皆以監牧見馬給之,或官與其直令自市,毋或強與。府界毋過三千匹,五路毋過五千匹。襲逐盜賊外,乘越三百里者有禁。在府界者,免體量草二百五十束,加給以錢布;在五路者,歲免折變緣納錢。三等以上,十戶為一保;四等以下,十戶為一社,以待病斃逋償者。保戶馬斃,保戶獨償之;社戶馬斃,社戶半償之。歲一閱其肥
 瘠,禁苛留者。凡十四條,先從府界頒焉。五路委監司、經略司、州縣更度之。於是保甲養馬行於諸路矣。



 時河東騎軍馬萬一千餘匹,番戍率十年一周。議欲省費,乃行《五路義勇保甲養馬法》。兵部言:「河東正軍馬九千五百匹,請權罷官給,以義勇保甲馬五千補之以合額。俟正軍馬不及五千,始行給配。」下中書、樞密院。樞密院以為:「官養一馬,歲為錢二十七千。民養一馬,才免折變緣納錢六千五百,折米而輸其直,為錢十四千四百,餘皆出
 於民,決非所願。況減軍馬五千匹,邊防事宜何所取備?若存官軍馬如故,漸令民間從便牧馬,不以五千為限,於理為可。」中書謂:「官養一馬,以中價率之,為錢二十七千。募民牧養,可省雜費八萬餘緡。計前二年官馬死,倍於保甲馬。而保甲有馬,可以習戰御盜,公私兩便。」帝卒從樞密院議。九年,京畿保甲養馬者罷給錢布,止免輸草而增馬數。



 元豐六年,取河東路保甲十分之二以教騎戰,且以本路鹽息錢給之。每二十五千令市一馬,仍
 以五年為限。



 七年,詔京東、西路保甲免教閱,每都保養馬五十匹,匹給錢十千,限京東以十年,京西十五年而數足。置提舉保甲馬官,京西以呂公雅,京東以霍翔領之。罷鄉村物力養馬之令,養戶馬者免保甲馬,皆翔所陳也。



 翔及公雅既領提舉事,多所建白。請借常平錢,每路五萬緡,付州縣出息,以賞馬之充肥及孳息者。願以私馬印為保馬者聽。養馬至三匹,蠲役外,每疋許次丁一人贖杖罪之非侵損於人者。詔悉從之。公雅又令每都歲
 市二十匹,限十五年者促為二年半。京西不產馬,民貧乏益不堪,上慮有司責數過多,百姓未喻上意,詔如元令,稍增其數。公雅乃請每都歲市八匹,限以八年,山縣限以十年。翔又奏本路馬已及萬匹,請令諸縣弓手各養一匹,以贖失捕之罪。



 哲宗嗣位,言新法之不便者,以保馬為急。乃詔曰:「京東、西保馬,期限極寬。有司不務循守,遂致煩擾。先帝已嘗手詔詰責,今猶未能遵守。其兩路市馬年限並如元詔。」尋又詔以兩路保馬分配諸軍,
 餘數付太僕寺,不堪支配者斥還民戶而責官直。翔、公雅皆以罪去,而保馬遂罷。



 戶馬者,慶歷中,嘗詔河北民戶以物力養馬,以備官買。熙寧二年,河北察訪使曾孝寬以為言,始參考行之。是時,諸監既廢,仰給市馬,而義勇保甲馬復從官給,朝廷以乏馬為憂。



 元豐三年春,以王拱辰之請,詔開封府界、京東西、河北、陜西、河東路州縣戶各計資產市馬,坊郭家產及三千緡、鄉村五千緡、若坊郭鄉村通及三千緡以上者,各養一馬,增倍者
 馬亦如之,至三疋止。馬以四尺三寸以上,齒限八歲以下,及十五歲則更市如初,籍於提舉司。於是諸道各上其數,開封府界四千六百九十四,河北東路六百一十五,西路八百五十四,秦鳳等路六百四十二,永興路一千五百四十六,河東路三百六十六,京東東路七百一十七,西路九百二十二,京西南路五百九十,北路七百一十六。



 時初立法,上慮商賈乘時高直以病民,命以群牧司驍騎以上千疋出市,以平其直。熙寧中,嘗令德順軍
 蕃部養馬,帝問其利害。王安石謂:「今坊、監以五百緡得一馬,若委之熙河蕃部,當不至重費。蕃部地宜馬,且以畜牧為生,誠為便利。」已而得駒庳劣,亡失者責償,蕃部苦之,其法尋廢。至是,環慶路經略司復言已檄諸蕃部養馬,詔閱實及格者一匹支五縑,鄜延、秦鳳、涇原路準此。



 時西方用兵,頗調戶馬以給戰騎,借者給還,死則償直。七年,遂詔河東、鄜延、環慶路各發戶馬二千以給正兵,河東就給本路,鄜延益以永興軍等路及京西坊郭
 馬,環慶益以秦鳳等路及開封府界馬。



 戶馬既配兵,後遂不復補。京東、西既更為保馬,諸路養馬指揮至八年亦罷。其後給地牧馬,則亦本於戶馬之意云。



 至於收市,則仍嘉祐之制,置買馬司於原渭州、德順軍,而增為招市之令。後開熙河,則更於熙河置買馬司,而以秦州買馬司隸焉。八年,遂置熙河路買馬場六,而原、渭、德順諸場皆廢。繼又置熙河岷州、通遠軍、永寧砦等場,而德順軍置馬場亦復。先是,麟府路上所市馬三百,以其直增
 於熙河而又多羸憊,乃罷本路博易,令軍馬司自市。時又以邊臣之議,市岢嵐、火山軍土產馬以增戰騎。既又以邊人盜馬越疆以趣利,尋皆罷之。自是,國馬專仰市於熙河、秦鳳矣。



 熙寧七年,熙河用兵,馬道梗絕。乃詔知成都府蔡延慶兼提舉戎、黎州買馬,以經度其事。明年,延慶言:「威、雅、嘉、瀘、文、龍州,地接烏蠻、西羌,皆產善馬。請委知州、砦主,以錦採、茶、絹招市。」未及施行,會威、茂州夷人盜邊,及西邊馬已至,八月,遂詔罷提舉戎、黎買馬。



 元
 豐中,軍興乏馬。六年,復命知成都呂大防同成都府、利州路轉運司,經制邊郡之可市馬者,遂制嘉州中鎮砦、雅州靈關等買馬場,而馬皆不至。元祐初,乃罷之。



 元祐中,嘗詔以蜀馬給陜西軍,以陜西馬赴京師。崇寧五年,增黎州市馬至四千疋。然凡云蜀馬者,惟沈黎所市為多,其它如戎、瀘等州,歲與蠻人為市,第存優恤,數馬以給其直。大觀初,又詔播州夷界巡檢楊榮,許歲市馬五十疋於南平軍,其給賜視戎州之數。



 熙寧中,罷券馬而
 專於招市,歲省三司錢二十萬緡。自馬不下槽出牧,三司得復給芻秣之費更相補除,而三司歲償群牧者,為緡錢十萬,以增市馬。券馬之罷已久,紹聖初,提舉買馬陸師閔奏復行之,令蕃漢商人願以馬結券進賣者,先從諸場驗印,各具其直給券,送太僕寺償之。其說以為券馬既盛行,則綱馬可罷。行之三年,樞密院言券馬死不及厘,而綱馬之死十倍。乃賜師閔金帛,加集賢修撰,以賞其功。時議既不以券馬為是,主管買馬閻令亦言其
 枉費。然曾布力行之。崇寧中,乃詔買馬一遵元豐法。



 市馬之官,自嘉祐中,始以陜西轉運使兼本路監牧買馬事,後又以制置陜西解鹽官同主之。熙寧中,始置提舉熙河路買馬,命知熙州王韶為之,而以提點刑獄為同提舉。



 八年,提舉茶場李杞言:「賣茶買馬,固為一事。乞同提舉買馬。」詔如其請。十年,又置群牧行司,以往來督市馬者。



 元豐三年,復罷為提舉買馬監牧司。四年,群牧判官郭茂恂言:「承詔議專以茶市馬,以物帛市穀,而並茶
 馬為一司。臣聞頃時以茶易馬,兼用金帛,亦聽其便。近歲事局既分,專用銀絹、錢鈔,非蕃部所欲。且茶馬二者,事實相須。請如詔便。」奏可。仍詔專以雅州名山茶為易馬用。自是蕃馬至者稍眾。六年,買馬司復罷兼茶事。七年,更詔以買馬隸經制熙河財用司。經制司罷,乃復故。



 自李杞建議,始於提舉茶事兼買馬,其後二職分合不一。崇寧四年,詔曰:「神宗皇帝厲精庶政,經營熙河路茶馬司以致國馬,法制大備。其後監司欲侵奪其利以助
 糴買,故茶利不專,而馬不敷額。近雖更立條約,令茶馬司總運茶博馬之職,猶慮有司茍於目前近利,不顧悠久深害。三省其謹守已行,毋輒變亂元豐成法。」自是職任始一。



 市馬之數,以時增損。初,原、渭、德順凡三歲共市馬萬七千一百匹,而群牧判官王誨言:「嘉祐六年以前,秦州券馬歲至者萬五千匹。今券馬法壞,請令增市,而優使臣之賞。」熙寧三年,乃詔涇、原、渭、德順歲買萬匹,三年而會之,以十分為率,及六分七厘者進一官,餘分又
 析為三等,每增一等者更減磨勘年。自是,市馬之賞始優矣。時誨上《馬政條約》,詔頒行之。其後,熙河市馬歲增至萬五千。紹聖中,又增至二萬匹,歲費五十萬緡。後遂以為定額,特詔增市者不在此數。



 崇寧四年,提舉程之邵、孫鰲抃以額外市戰馬及二萬匹,各遷一官。鰲抃仍賜三品服。大觀元年,龐寅孫等又以買御前良馬及三萬疋,推恩如之邵例。宣和中,宇文常、何漸等更以遵用元豐成法,省費不貲,各加職遷官。時如此類頗眾。賞典
 優濫,官屬利於多市馬,取充數而已。



 支配。舊制,自御馬而下,次給賜臣僚,次諸軍,而驛馬為下。



 熙寧初,樞密院言:「祖宗時,臣僚任邊職者,或賜帶甲馬,示不忘疆埸之事。承平日久,僥幸滋長。請應使臣合門祗候以上,充三路路分州軍總管、鈐轄之類,賜馬價如故,餘皆罷給。」奏可。十年,群牧司又言:「去歲給安南行營及兩省、宗室、諸班直及諸軍、諸司馬總三千餘匹,未支者猶二千。請裁宗室以下所給馬,諸司停給。」從之。自罷監至此,始闕馬
 矣。



 熙寧初,詔河北騎軍如陜西、河東社馬例立社,更相助錢以市馬,而遞增官直。尋出奉宸庫珠十餘萬以充其費。其後,陜西馬社苦於斂率。元豐中,乃詔本路罷其法,更從買馬司給之。時又諸路置將,馬不能盡給,則給其直,而委諸將自市。其在熙河蘭會路者,即以為買馬之數。



 初,內外諸軍給馬,例不及其元額,視其闕之多寡,以分數填配。元豐更立為定制,凡諸軍闕馬應給者,在京、府界、京東西、河東、陜西路無過十之七,河北路十之
 六。然其後諸軍闕馬者多,紹聖三年,乃詔提舉陸師閔於歲額外市馬三萬匹,給鄜延、環慶路正兵,余支弓箭手,仍權不限分數。



 宣和初,真定、中山、高陽等路乏馬,復給度僧牒,令帥臣就市,以補諸軍之闕。



 高宗紹興二年,置馬監於饒州,守卒領之,擇官田為牧地,復置提舉。俄廢。四年,置監臨安之餘杭及南蕩。



 十九年,詔:「馬五百匹為一監,牡一而牝四。監為四群。歲產駒三分及斃二分以上,有賞罰。」帝謂輔臣曰:「議者言南地不宜牧馬。昨自
 牧養,今二三年,已得馬數百。」先是,川路所置馬,歲牧於鎮江。是年春,上以未見蕃息,遂分送江上諸軍。後又置監郢、鄂間,牝牡千,十餘年僅生二十駒,且不可用,乃已。故凡戰馬,悉仰秦、川、廣三邊焉。



 秦馬舊二萬,乾道間,秦、川買馬額歲萬一千九百有奇,川司六千,秦司五千九百。益、梓、利三路漕司,歲出易馬紬絹十萬四千疋。成都、利州路十一州,產茶二千一百二萬斤。茶馬司所收,大較若此。慶元初,合川、秦兩司為萬一千十有六。嘉泰末,合兩司
 為萬二千九十四。



 然累歲市易,多不及額。蓋南渡前,市馬分而為二:其一曰戰馬,生於西郵,良健可備行陣,今宕昌、峰貼峽、文州所產是也;其二曰羈縻馬,產西南諸蠻,短小不及格,今黎、敘等五州所產是也。羈縻馬每綱五十,其間良者不過三五,中等十數,餘皆下等,不可服乘。守貳貪賞格,以多為貴。經涉險遠,且綱卒盜其芻粟,道斃者相望。



 成都府馬務,歲發江上諸軍馬凡五十八綱,月券錢米二百緡,歲計萬一千六百緡。興元府馬務,歲發三衙馬
 百二十綱,其費稱是。率未嘗如數,蓋茶馬司靳錢帛,馬至,價不即償致然也。



 舊蕃蠻中馬,良駑有定價。紹興中,張松為黎卒,欲馬溢額覬賞,乃高直市之。夷人無厭,邀求滋甚。後邛部川蠻恃功。趙彥博始以細茶、錦與之。而夷人每貿馬,以茶、錦不堪借口」



 慶元中,金人既失冀北地,馬至秦司亦罕。舊川、秦市馬赴密院,多道斃者。紹興二十七年,詔川馬不赴行在,分隸江上諸軍,鎮江、建康、荊、鄂軍各七百五十,江、池軍各五百,殿前司二千五百,馬
 司、步司各千,川馬良者二百進御。此十九年所定格也。



 廣馬者,建炎末,廣西提舉峒丁李棫請市馬赴行在。紹興初,隸經略司。三年,即邕州置司提舉,市於羅殿、自杞、大理諸蠻。未幾,廢買馬司,帥臣領之。七年,胡舜陟為帥,歲中市馬二千四百,詔賞之。其後馬益精,歲費黃金五鎰,中金二百五十鎰,錦四百,絁四千,廉州鹽二百萬斤,得馬千五百。須四尺二寸已上乃市之。其直為銀四十兩,每高一寸增銀十兩,有至六七十兩者。土人云,尤駔
 駿者,在其產處,或博黃金二十兩,日行四百里,第官價已定,不能致此。



 自杞諸蕃本自無馬,蓋轉市之南詔。南詔,大理國也。乾道九年,大理人李觀音得等二十二人至橫山砦求市馬,知邕州姚恪盛陳金帛誇示之。其人大喜,出一文書,稱「利貞二年十二月」,約來年以馬來。所求《文選》、《五經》、《國語》、《三史》、《初學記》及醫、釋等書,恪厚遺遣之,而不敢上聞也。嶺南自產小駟,匹直十餘千,與淮、湖所出無異。大理連西戎,故多馬,雖互市於廣南,其實猶
 西馬也。每擇其良赴三衙,餘以付江上諸軍。



 寶慶四年,兩淮制府貿易北馬五千餘,而他郡亦往往市馬不輟。咸淳末,有紀智立者獻謀,以為兩淮軍將、武官、巨室皆畜馬,率三借二,二借一,一全起,團結隊伍,借助防江,各令飼馬役夫自乘之官,優給月錢一年,以半年為約,江面寧即放歸。又云,陳巖守招信,團馬至七千,出沒張耀,此其驗也。臣僚言:宜仿祖宗遺意,亟謀和市馬,如出一馬,則免其某色力役。惟是川、秦之馬,遵陸則崇岡復嶺,
 盤回斗絕;舟行則峽江湍急,灘磧險惡。每綱運,公私經費十倍,而人馬俱疲。上則耗國用,下則困州縣。綱兵所經,甚於寇賊。雖臣僚條奏更迭,終莫得其要領。豈馬政各因風士之宜,而非東南之利歟?



\end{pinyinscope}