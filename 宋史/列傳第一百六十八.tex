\article{列傳第一百六十八}

\begin{pinyinscope}

 ○高
 定
 子高斯得張忠恕唐璘



 高定子,字瞻叔,利州路提點刑獄兼知沔州稼之弟也。嘉泰二年舉進士,授郪縣主簿。吳曦畔,乞解官養母,曦誅,攝府事宇文公紹以忠孝兩全薦之,調中江縣丞。父
 就養得疾,定子衣不解帶者六旬。居喪,哀毀骨立。服除,成都府路諸司闢丹棱令,尋以同產弟魏了翁守眉,改監資州酒務。丁母憂,服除,差知夾江縣。



 前是,酒酤貸秫於商人,定子給錢以糴,且寬榷酤,民以為便。麻菽舊有征,定子悉弛之。會水潦洊饑,貧民競訴無所於糴,定子曰:「女毋憂,女第持錢往常所糴家以俟。」乃發縣廩給諸富家,俾以時價糶,至秋而償,須臾米溢於市。鄰邑有爭田十餘年不決,部使者以屬定子,定子察知偽為質劑,
 其人不伏。定子曰:「嘉定改元詔三月始至縣,安得有嘉定元年正月文書邪?」兩造遂決。四川總領所闢主管文字,同幕有以趣辦為能迫促諸郡者,定子白使者斥去之。總領所治利州,倚酒榷以佐軍用,吏奸盤錯,定子躬自究詰,酒政遂平。後來者復欲增課,定子曰:「前以吏蠹,亦既革之,今又求益,是再榷也。」乃止。



 制置使鄭損強愎自用,誤謂總領所擅十一州小會子之利,奏請廢之,令下,民疑而罷市。定子力爭,謂:「小會子實以代錢,百姓貿
 易,賴是以權川引,罷則關、隴之民交病,況又隆興間得旨為之,非擅也。」乃得存其半。損又欲增總領所鹽課,取舊貸軍費,定子辨其顛末,損乃釋然曰:「二司相關處,公每明白洞達言之,使人爽然自失。」尋差知長寧軍。長寧地接夷獠,公家百需皆仰淯井鹽利,來者往往因以自封殖,制置司又榷入其半。定子至,爭於制置使,得蠲重賦。



 差知綿州。大元兵穿鳳州塞,破武休,下興元,小校張鉞以其徒潰入文州,殺守臣楊必復,將自龍趨綿,以闖
 成都。安撫使黃伯固聞之,亟奏定子兼參議官,措置文、龍備御。定子乃部分諸軍扼青塘嶺,鉞就擒。已而劍南大震,定子語僚吏曰:「諸君去留不敢拘,若某則守城郭封疆之臣,有死而已。」戒群胥曰:「潰軍流民不過欲得錢糧爾,吾將盡發吾州之藏與截諸司之綱,為朝廷捍蔽全蜀。我去,聽汝等殺我;汝等逃,吾斫汝頭矣。」乃下令招潰卒,人給緡錢五十、米一石,命都監陳訓專任接納。訓忽奔告曰:「諸軍雖受招,不肯釋甲,奈何?」定子乃令帳下
 卒衷甲於兩廡以俟,戒毋輕動。俄而諸軍盛陳兵以至,吏士皆股慄,定子坐堂上,傳令勞苦之,諸軍皆拜。定子開諭以理,使還本部,以俟給犒。諸將聞之,亦來上謁,定子復慰安之。因問:「汝等何為至此?」皆曰:「制置使未知存亡,諸軍無主。」定子曰:「大帥不過暫移治爾,已遣人訪所在,茍終不獲,我當為汝曹主張。且諸軍至此以無糧故,吾州當任供億。」又曰:「敵將復會於此,盍避之?」定子曰:「我文官也,不畏死,汝將軍也,世世衣食縣官,乃欲避敵乎?
 我是守臣,死則死於此爾。有欲殺太守者,一槍足矣,軍器安用多為?今諸軍大集,萬一敵至,能戮力出戰,是汝曹立功報國之機也,不猶愈於深入內郡為罪滋大乎?」眾悅而去。乃遣吏給犒如令,闢寺觀祠宇以舍之。



 亡幾何,敗將和彥威、陳邦佐、曹篪、張涓、姚承祖等皆集於彰明,剽掠尤甚。彥威遣邦佐入州,大言駭眾,謂定子曰:「知府何不去?和太尉兼兩戎司,威權甚重,麾下兵且二萬餘,欲來駐此,今至矣。」定子謂曰:「本州素非備御之地,大
 將以兵入,欲何為者?第來,吾固有以相待。」邦佐色沮,乃曰:「已遣幕府來議。」至則一游士爾,繆為恭敬,要索甚大。定子答曰:「軍將入吾境,當受吾節制,惟各守紀律,則給以錢糧。若敵至,為國一死,作忠臣孝子,愈於病五日不汗死也。」幕府莫能對,出彥威符移,有云:「大府招戢散軍,人給錢米若干,今所部不下二萬人,願如數得之。」定子報曰:「本州已下此令,何敢食言;但所給者乃潰軍就招免罪之人,都統所部非潰也,若以此例相給,其肯受乎?」
 彥威得檄甚慚,乃乞別給錢糧以餉軍,定子即捐四十萬緡與之,仍趣其還戍。蓋定子身任兩司之責,極其勞勩,以收捕張鉞功,進三官,以防遏招收潰兵功,又進一官,進直寶章閣,再任。



 頃之,召入奏事,吏民追送,莫不流涕;鄰郡聞定子至,焚香夾道,舉手加額曰:「微公,吾屬塗炭久矣。」定子之未去郡也,伯兄稼以權利路提刑上印而歸,了翁亦至自靖州,過定子於綿,定子為築棣鄂堂,飲酒賦詩為樂,一時以為美談。入對,極言時敝。時史彌
 遠執國柄久,故有曰:「陛下優禮元勛,俾得以弛繁機而養靜壽,朝廷得新百度而革因循,不亦善乎?」既對,人為定子危之,定子曰:「乖逢得喪,是有命焉,吾得盡言,乃報君職分也。」越兩月,乃遷刑部郎中。彌遠沒,言之者紛然,識者謂定子先事有言,視諸人為難。



 尋以直寶謨閣、江南東路轉運判官。陛辭,帝曰:「淮師巡邊,卿知之乎?輔車之勢,漕運為急,卿是行宜斟酌緩急,以相通融。」定子因上疏論邊事甚周悉,帝嘉納焉。逾年,召入奏事。會稼
 死事於沔州,上疏引疾,乞歸田里,不許。尋遷軍器監,又遷太府少卿,升計度轉運副使。有事於明堂,天大雷雨,詔求言,定子反覆論敬懼災異之意。復召入,遷司農卿兼玉牒所檢討官。



 入對,言:「內治不修,外懼不謹,近親有預政之漸,近習有弄權之漸,小人有復用之漸,國柄有陵夷之漸,士氣有委靡之漸,主勢有孤立之漸,宗社有阽危之漸。天變日多,地形日蹙。昔有危脈,今有危形;昔有亡理,今有亡證。」又請明詔沿流帥守將吏,思出奇乘
 險,求為水陸可進之策。



 升兼樞密都承旨,又遷太常少卿兼國史院編修官。累言邊事,遷起居舍人,尋兼中書舍人,參贊同京湖、江西督視府事,定子親往周視新城,大犒諸軍,激厲守將。遷禮部侍郎,仍兼中書舍人,即軍中賜金帶。詔以督府事入奏,既至,帝勞問甚渥,特進一官,尋兼崇政殿說書兼直學士院。未幾,改侍講、權禮部尚書,升兼侍讀。入奏,言:「國無仁賢,無禮義,無政事,有類叔世。」帝竦然。尋兼直學士,修孝宗、寧宗《日歷》,書成上進,擢
 拜翰林學士、知制誥兼吏部尚書,升兼修國史、實錄院修撰,賜衣帶、鞍馬。乞召收李心傳卒成四朝志、傳。



 時禮部尚書杜範、吏部侍郎李韶皆以伉直稱,或乞身求去,或臥家不出。定子言:「人主寄耳目者,臺諫也,補耳目之所不逮者,法從之論思,百官之輪對,則上必論君德之粹駁,次必言朝政之得失。舍是而使之但言常程,姑應故事,畏縮乎雷霆之威,阿徇乎宰執之好,遜避乎耳目之官,則凡論思等事,皆不必講矣。宜速返李韶以開不
 諱之門,勉起杜範以伸敢言之氣。」因乞歸田甚力。



 進端明殿學士、簽書樞密院事,尋兼權參知政事。仍舊職,知福州、福建安撫,固辭,提舉洞霄宮。因請致仕,不許,改知潭州、湖南安撫大使,力辭,退居吳中,深衣大帶,日以著述自娛。以資政殿學士轉一官致仕,卒,贈少保。



 字子作同人書院於夾江,修長興學,創六先生祠,蓋以教化為先務。所著《存著齋文集》、《北門類稿》、《薇垣類稿》、《經說》、《紹熙講義》、《奏議》、《歷官表奏》行世。



 高斯得字不妄,利州路提點刑獄、知沔州稼之子也。少
 從李坤臣學,坤臣瞽,斯得左右扶持之。中成都路轉運司試,補入太學。紹定二年舉進士,授利州路觀察推官。越二年,闢差四川茶馬乾辦公事。李心傳以著作佐郎領史事,即成都修《國朝會要》,闢為檢閱文字。端平二年九月,稼死事於沔,時大元兵屯沔,斯得日夜西向號泣。會其僮至自沔,知稼戰沒處,與斯得潛行至其地,遂得稼遺體,奉以歸,見者感泣。服除而哀傷不已,無意仕進。心傳方修四朝史,闢為史館檢閱,秩同秘閣校勘,蓋創員
 也。斯得分修光、寧二帝《紀》。尋遷史館校勘,又遷軍器監主簿兼史館校勘。



 時丞相史嵩之柄國,斯得遇對,空臆盡言。冬雷,斯得應詔上封事,乞擇才並相,由是迕嵩之意。遷太常寺主簿,仍兼史館校勘。時斯得叔父定子以禮部尚書領史事,時人以為美談。會太學博士劉應起入對,拄嵩之,嵩之恚,使其黨言叔父兄子不可同朝,以斯得添差通判紹興府。淳祐二年,四朝《帝紀》書成,上之。嵩之妄加毀譽於理宗、濟王,改斯得所草《寧宗紀》末卷,
 斯得與史官杜範、王遂辨之。範報書亦有「奸人剿入邪說」之語,然書已登進矣。心傳藏斯得所草,題其末曰「前史官高某撰」而已。



 逾年,添差通判臺州。範既入相,召為太常博士,遷秘書郎。六年正月朔,日有食之,斯得應詔上封事,言:「大奸嗜權,巧營奪服,陛下奮獨斷而罷退之,是矣。諫憲之臣,交疏其惡,或請投之荒裔,或請勒之休致。陛下茍行其言,亦足昭示意向,渙釋群疑。乃一切寢而不宣,歷時既久,人言不置,然後黽勉傳諭,委曲誨奸,
 俾於襲絰之時,妄致掛冠之請,因降祠命,茍塞人言,又有奸人陰為之地。是以訛言並興,善類解體,謂聖意之難測,而大奸之必還,莽、卓、操、懿之禍,將有不忍言者。」時監察御史江萬里及它臺諫累疏論嵩之罪惡,竟不施行,第因嵩之致仕,予祠而已,故斯得封事首及之。



 又言:「大臣貴乎以道事君,今乃獻替之義少而容悅之意多,知恥之念輕而患失之心重。內降當執奏,則不待下殿而已行;濫恩當裁抑,則不從中覆而遽命。嫉正而庇邪,
 喜同而惡異,任術而詭道,樂媮而憚勞。陛下虛心委寄,所責者何事,而其應乃爾。」時範鐘獨當國,過失日章,故斯得及之。又言:「便嬖側媚之人,尤足為清明之累,腐夫巧讒而使傳幾搖,妖㜮外通而魁邪密主,陰奸伏蠱,互煽交攻,陛下之心至是其存者幾希矣。陛下之心,大化之本也。洗濯磨淬,思所以更之,乃徒立為虛言無實之名,而謂之更化,此天心之所以未當,大異之所以示儆也。」言尤切直,帝嘉納焉。



 又言:「群臣厖雜,宮禁奇邪,黷貨
 外交,豈可坐視而不之問!顧乃並包兼容之意多,別邪辨正之慮淺,憂讒避謗之心重,直前邁往之志微,遂使眾臣爭衡,大權旁落,養成積輕之勢,以開窺覬之漸。設有不幸,變故乘之,上心一移,兇渠立至,使宗社有淪亡之憂,衣冠遭魚肉之禍,生靈罹塗炭之厄。當是時也,能潔身以去,其能逃萬世之清議乎?」於是群憸悚懼,或泣訴上前,或上章求去,合力排擯,斯得遂求補外。在告幾百餘日,於是差知嚴州,斯得三請乞祠,不許。嚴環山為
 郡,雖豐歲猶仰它州。夏旱,斯得蠲租發廩,招糴勸分,請於朝,得米萬石以振濟。



 遷浙東提點刑獄,遂劾知處州趙善瀚、知臺州沈暨等七人倚勢厲民,疏上,不報。改江西轉運判官,斯得具辭免,上奏曰:「臣劾奏趙善瀚等七人,未聞報可,固疑必有黨與營救,惑誤聖聽,今奉恩除,乃知中臣所料。善瀚者,侍御史周坦之婦翁也,贓吏之魁,錮於聖世,鄭清之與之有舊,復與州符。沈塈者,同簽書樞密院事史宅之妻黨也。祖宗以來,未有監司按吏一
 不施行者,壞法亂紀,未有甚此。臣身為使者,劾吏不行,反叨易節,若貪榮冒拜,則與世之頑頓無恥者何異?乞並臣鐫罷,以戒奉使無狀者。」章既上,坦自謂己任臺諫而反見攻,遍懇同列論斯得,同列難之,計急,自上章劾罷斯得新任,未幾,坦亦罷,七人竟罷去。



 移湖南提點刑獄,薦通判潭州徐經孫等六人。攸懸富民陳衡老,以家丁糧食資強賊,劫殺平民。斯得至,有訴其事者,首吏受賕而左右之,衡老造庭,首吏拱立。斯得發其奸,械首吏
 下獄,群胥失色股慄。於是研鞫具得其狀,乃黥配首吏,具白朝省,追毀衡老官資,簿錄其家。會諸邑水災,衡老願出米五萬石振濟以贖罪。衡老婿吳自性,與衡老館客太學生馮煒等謀中傷斯得盜拆官櫝。斯得白於朝,復正其罪,出一篋書,具得自性等交通省部吏胥情狀。斯得並言於朝,下其事天府,索出賕銀六萬餘兩,黥配自性及省寺高鑄等二十餘人。初,自性厚賂宦者言於理宗曰:「斯得以緡錢百萬進,願易近地一節。」理宗曰:「高
 某硬漢,安得有是。」而斯得力求去,清之以書留之。又薦李晞顏等五人。



 加直秘閣、湖南轉運判官,改尚右郎官,未至,改禮部郎中。上疏極論時事,改權左司,力辭,內批兼侍立修注官。言水災曰:「願陛下立罷新寺土木,速反迕旨諸臣,遏絕邪說,主張善良,謹重刑闢,愛惜士類,抑遠佞臣,絕其干撓,則天意可回,和氣可召矣。」會斥左司徐霖,帝慮給事中趙汝騰爭逐霖事,乃徙汝騰翰林學士,汝騰聞命即去國。斯得言:「汝騰一世之望,宗老之重,
 飄然引去,陛下遂亦棄之有如弁髦,中外驚怪,將見賢者力爭不勝而去,小人踴躍增氣而來。陛下改紀僅數月,初意遽變,臣深惜之。」



 時上封事言得失者眾,或者惡其言雚詉,遂謂「空言徒亂人聽,無補國事。」斯得因轉對,言:「諸臣之言,上則切劘聖主,下則砥礪大臣,內則摧壓奸邪,外則銷遏寇虐,顧以為無補於實政乎?空言之譏,好名之說,欲一網君子而盡去之,其言易入,其禍難言,此君子去留之機,國家安危之候,不可不深留聖慮者也。」
 監察御史蕭泰來論罷。



 逾年,以直寶文閣知泉州,力辭,遷福建路計度轉運副使。朝廷行自實田,斯得言:「按《史記》,秦始皇三十一年,令民自實田。主上臨御適三十一年,而異日書之史冊,自實之名正與秦同。」丞相謝方叔大愧,即為之罷。董槐入相,召為司農卿。程元鳳入相,改秘書監。丁大全入相,監察御史沈炎論斯得以閩漕交承錢物,下郡吏天府,榜死數人。先是,吳自性之獄,高鑄為首惡黥配廣州,捐資免行,至是為相府監奴,嗾炎發
 其端。京尹顧巖傅會其獄,安吉守何夢然奉行其事,陵鑠甚至,斯得不少挫,竟無所得。大全既謫,朝廷罪其委任非人,遂斬鑄。斯得既拜浙西提點刑獄之命,炎,浙西人,泣於上前,乞更之,移浙東提舉常平。命下,給事中章鑒繳還。斯得杜門不出,著《孝宗系年要錄》。



 彗星見,應詔上封事,曰:「陛下專任一相,虛心委之,果得其人,宜天心克享,災害不生。而庚申、己未之歲,大水為災,浙西之民死者數百千萬。連年旱暵,田野蕭條,物價翔躍,民命如
 線。今妖星突出,其變不小。若非大失人心,何以致天怒如此之烈。」封事之上也,似道匿不以聞。



 度宗即位,召為秘書監,又論罷。復遷秘書監,屢辭不許,擢起居舍人兼國史院編修官、實錄院檢討官兼侍講。進讀之時,每於天命去留之際,人心得失之因,前代治亂之故,祖宗基業之難,必反復陳之。兼權工部侍郎,遂兼同修國史、實錄院同修撰,仍兼侍講。進《高宗系年要錄綱目》,帝善之。大元軍下襄陽,斯得疏論言事,最為切要,帝嘉納,遷工
 部侍郎。屢求補外,以顯文閣待制、知建寧府。



 度宗崩,陳宜中入相,以權兵部尚書召。斯得痛國事之阽危,疏言誅奸臣以謝天下,開言路以回天心,聚人才以濟國事,旌節義以厲懦夫,竭財力以收散亡。忠憤激烈,指陳當時之事無所遺。擢翰林學士、知制誥兼侍讀,進端明殿學士、簽書樞密院事兼參知政事,同提舉編修《敕令》及《經武要略》。大元年下饒州,江萬里赴水死,事聞,贈太傅。斯得言贈恤之典,所當度越故常,以風厲天下,遂加贈
 太師。又言賞通判池州趙卯發死節太薄,乃加贈待制。



 臺諫徐直方等四人論似道誤國之罪,乞安置嶺表,簿錄其家。丞相留夢炎庇護似道,止令散官居住,且謂簿錄擾及無辜。斯得謂「散官則安置,追降官分司則居住,祖宗制也。」夢炎語塞。夢炎乘間直罷去平章事王鑰、監察御史俞浙,並罷斯得,於是宋亡矣。所著有《詩膚說》、《儀禮合抄》、《增損刊正杜祐通典》、《徽宗長編》、《孝宗系年要錄》、《恥堂文集》行世。



 張忠恕,字行父,右僕射浚之孫。以祖任,監樓店務。入府幕,時韓侂胄權勢熏灼,嘗奪民間已許嫁女,夫家以告,忠恕白尹歸其父母,尹不能難。再調廣西轉運司主管文字,改通判沅州,主管京湖宣撫司機宜文字,知澧州。開禧末,入為籍田令。屬太廟鴟吻為雷雨壞,神主遷御,忠恕因輪對,請廣言路,通下情,寧宗嘉納。



 嘉定五年,遷軍器丞,進太府丞。出知湖州。遷司農丞、知寧國府。夏旱,請於朝,得賜僧牒五十,米十萬七千餘石。常平使者欲
 均濟而勿勸糶,忠恕慮後無以濟,遂核戶口、計歲月,嚴戒諸邑諭大家發蓋藏。所見浸異,以言去,主管沖祐觀。起知鄂州,改湖北轉運判官兼知鄂州。召為屯田郎官,丁內艱。免喪,入為戶部郎官。入對,極言邊事,其慮至遠。



 理宗即位,忠恕移書史彌遠請取法孝宗,行三年喪,且曰:「孝宗始自踐祚,服勤子職凡二十有七年,今上自外邸入繼大統,未嘗躬一日定省之勞,欲報之德,視孝宗宜有加。」既而宰輔率百僚請太母
 同聽政,忠恕復貽書史彌遠,謂:「英宗以疾,仁、哲以幼,母後垂簾,有不容已,惟欽聖出於勉強,務從抑損。今吾君長矣,若姑援以請,此亦中策爾。」詔群臣集議廟制,忠恕謂:「九廟非古。若升先帝,則十世之廟昉於今日,於禮無稽。」



 寶慶初,詔求直言,忠恕上封事,陳八事:



 一曰天人之應,捷於影響。自冬徂春,雷雪非時,西霅、東淮,狂悖洊興。客星為妖,太白見晝,正統所系,不宜諉之分野。



 二曰人道莫先乎孝,送死尤為大事。孝宗朝衣朝冠,皆以大布,
 迨寧考以適孫承重,光宗雖有疾,未嘗不服喪宮中也。洎光宗上賓,權焰方張,莫有言者。去秋禮寺受成胥吏,未嘗以義折衷。慶元間,再期而祥,百僚始純服吉。今若甫經練祭,雖朝臣一帶之微,不復有兇吉之別,則是三年之喪降而為期,害理滋甚。況人主執喪於內,而群工之服無異常日,是有父子而無君臣也。



 三曰太母方卻垂簾之請,而慶壽前期,陛下吉服稱觴,播為詩什,此世俗之見,非所以表儀於天下也。



 四曰陛下斬然在疚,大
 昏之期,固未暇問,然非豫講夙定,恐俚說乘間而入。臣所望於今日者,亦曰嚴取舍而正法度,廣詢謀而協公議爾。



 五曰陛下於濟王之恩,自謂彌縫曲盡矣。然不留京師,徙之外郡,不擇牧守,混之民居,一夫奮呼,闔城風靡,尋雖弭患,莫副初心。謂當此時,亟下哀詔,痛自引咎,優崇恤典,選立嗣子,則陛下所以身處者,庶幾無憾,而造訛騰謗者,靡所致力。自始至今,率誤於含糊,而猶不此之思,臣所不解也。



 六曰近世憸佞之徒,凡直言正論,
 率指為好名歸過;夫好名歸過,其自為者非也,若首萌逆億厭惡之心,則自今言者望風見疑,此危國之鴆毒。



 七曰當今名流雖已褒顯,而搜羅未廣,遺才尚多。經明行修如柴中行、陳孔碩、楊簡,識高氣直如陳宓、徐僑、傅伯成,僉論所推:史筆如李心傅,何惜一官,不俾與聞。況邇來取人,以名節為矯激,以忠讜為迂疏,以介潔為不通,以寬厚為無用,以趣辦為強敏,以拱默為靖共,以迎合為適時,以操切為任事。是以正士不遇,小人見親。



 八
 曰士習日異,民生益艱。第宅之麗,聲伎之美,服用之侈,饋遺之珍,向來宗戚、閹官猶或間見,今縉紳士大夫殆過之。公家之財,視為己物。薦舉、獄訟,軍伎、吏役、僧道、富民,凡可以得賄者,無不為也。至其避譏媒進,往往分獻厥餘。欲基本之不搖,殆卻行而求前也。



 疏入,朝紳傳誦。始魏了翁嘗勉忠恕以「植立名節,無隤家聲」。及是嘆曰:「忠獻有後矣!」真德秀聞之,更納交焉。



 忠恕又因輪對,引以伯父栻告孝宗之語曰:「當求曉事之臣,不求辦事之
 臣;欲求伏節死義之臣,必求犯顏敢諫之臣。」語益剴切。忠恕自知不為時所容,力請外補,遂以直秘閣知贛州。抵郡才兩月,言者指為朋比,落職,降兩官,罷。紹定三年,復元官,進秩一等,提舉沖祐觀。卒,遷一官致仕。魏了翁嘗許忠恕「拳拳體國似浚,撥繁剸劇似其父枃,斂華就實則有志義理之學,嘗有聞乎栻之教矣」。



 唐璘,字伯玉,古田人。游太學。嘉定十年舉進士,時臺臣李安行奏次對官不許論邊事,璘對策極詆之,曰:「吾始
 進,可壞於天子之庭乎?」調吳縣尉,有殺人於貨挾其舟亡者,有司求賊急,屠者自告吾兒實殺之,兒亦自誣伏。璘問:「舟安在?錢何用?」其辭差,為緩之,果得賊太湖,與舟俱至,舉縣感服。縣有勢家治圃,將鑿渠通舟,繆言古有渠,常平使者主之。璘視乾道故籍,則誠民田也,力爭,迕使者意,移監縣稅。璘遂以直聞。調瑞州學教授,用白鹿洞教法,崇禮讓,後文藝,士翕然知向。監行在榷貨務門。



 闢淮東運司催轄綱運官。屬出師楚州,盡瘁焉。捷聞,以
 金人據淮陰,欲乘勢取之。璘言:「捷奏多誇,詎得信乎?須聚兵二十萬,日費米斛餘五千,緡錢餘二萬,調夫幾萬人,僅能使賊全師北去。今出沒漣、海,謀結北邊,政欲迭出撓我,憂方大爾。淮陰緊壘與楚城等,濠之廣又過之,我士疲丁困,可一拔得乎?恢復,美名也,而賈實禍,僕竊危之。」不聽,制司恥楚城之捷自趙範與葵出,議贖淮陰二城為功。洎聞金變,即轉攻之,我師死傷者六萬,璘在兵間憤之,著《讜論》,直書其事上之。知晉陵縣,鄰州田訟,
 至有泣訴諸使願送晉陵可否者。制置使陳韡留守建康,闢為通判,舉府事以聽。



 監六部門,擢監察御史,臺吏且至,璘皇駭趨避不敢詣闕。母曰:「人言此官好,汝何得憂乎?」璘曰:「此官須為朝廷爭是非,一咈上意或迕權貴,恐重為大人累,何得不憂?」母曰:「而第盡言,吾有而兄在,忽憂。」璘拜謝,入就職。



 故事,御史惟常服拜下,有論奏繳進,至是獨召對緝熙殿,令服窄衫面讀。首疏奏:「天變而至於怒,民怨而幾於離,海宇將傾,天下有不可勝諱之
 慮。陛下謂此何時,縱欲累德,文過飾非,疏遠正人,狎暱戚宦,濁亂朝政,自取覆亡。宰相用時文之才為經世之具,不顧民命,輕挑兵端,不度事宜,頓空國帑。委政厥子,內交商人,賄塗大開,小雅盡廢。瑣瑣姻婭,敢預邪謀,視國事如俳優,以神器為奇貨,都人側目,朝士痛心。盍正無將之誅,以著不忠之戒。崔與之操行類楊綰,雖修途莫景,力不逮心,而命下之日,聞者興起。喬行簡頗識大體,朝望稍孚,而除授偏私,事多遺忘。宜擇家相,贊宗子,
 輔民物,以慰父母之望,毋使天變浸極,人心愈離也。」上為改容。又請號召土豪,經理荊、襄,亟擇帥臣,安集淮西,帝嘉納;至問邊事甚悉。



 璘感激知遇,自是彈擊無所避,再疏:「鄭清之妄庸誤國,乞褫職罷祠。其子士昌,招權納賄,拔庸將為統帥,起贓吏為守臣,乞削籍廢棄。鄭性之懦而多私,黨庇奸庸,臣受其改官舉狀,嘗蒙薦之陛下,國事至此,不敢顧私。李鳴復甘心諂鄭損,得薦入朝,適清之議張天綱之獄,迎合從輕,遂擢臺端。會趙桄夫遣
 史寅午囑清之父子,鳴復又結寅午得登政府。」會杜範亦論鳴復,不行,而範去,璘遂力丐外,疏七上,授廣西運判,改知嘉興府,尋改江東運判。



 時邊事急,置四察訪使,就詔璘分建康、太平、池州、江西。璘揭榜馬前,咨所部以利害,又戒土豪團結漁業水手、茶鹽舟夫、蘆丁,悉備燎舟之具,人人思奮。即選將總二州兵舟以耀敵,檄當塗宿設戰具,防採石,撥和糴續生券,且奏損總領所錢二十萬緡助江防,軍聲大振。



 尋升直華文閣、知廣州、廣東
 經略安撫使。梅州寇作,璘示以威信,寇尋息。江淮旱,議下廣右和糴,璘言:「公家赤立,糴本無所辦,終恐日取於民,非臣不敢撥本,召釁重朝廷多事之憂。」明年上章乞致仕,帝思見之,亟命入奏,擢太常少卿。尋丁內艱,璘居喪哀毀不食,久之疾革,卒。



 璘立臺僅百日,世謂再見唐介,至切劘上躬,盡言無隱,帝益嚴憚之。居官大節,則母教之助為多。



 論曰:觀高定子在西陲,政業著聞矣。斯得屢起而屢僕
 於權臣之手,及其再起,宋事已非。張忠恕論濟邸事,有父祖風焉。唐璘者,亦可謂古之遺直。



\end{pinyinscope}