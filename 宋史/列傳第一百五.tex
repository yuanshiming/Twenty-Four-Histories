\article{列傳第一百五}

\begin{pinyinscope}

 陳次升陳師錫彭汝礪弟汝霖汝方呂陶張庭堅龔夬孫諤陳軒江公望陳祐常安民



 陳次升,字當時,興化仙游人。入太學,時學官始得王安
 石《字說》,招諸生訓之,次升作而曰:「丞相豈秦學邪?美商鞅之能行仁政,而為李斯解事,非秦學而何?」坐屏斥。既而第進士,知安丘縣。轉運使吳居厚以聚斂進,檄尉罔徵稅於遠郊,得農家敗絮,捕送縣,次升縱遣之。居厚怒,將被以文法,會御史中丞黃履薦,為監察御史。



 哲宗立,使察訪江、湖。先是,蹇周輔父子經畫江右鹽法,為民害,次升舉劾之。還言:「額外上供之數未除,異日必有非法之斂,願從熙寧以來創行封椿名錢悉賜豁免。又役法
 未定,人情熒惑,乞速定差雇及均數之等,先為之節而審行之。」提點淮南、河東刑獄。



 紹聖中,復為御史,轉殿中。論章惇、蔡卞植黨為奸,乞收還威福之柄。禁中火,彗出西方,次升請修德求言,以弭天變。掖庭鞫厭魅獄,次升言:「事關中宮,宜付外參治。今屬於閹寺之手,萬一有冤濫,貽後世譏。」濟陽郡王宗景請以妾為妻,論其以宗藩廢禮,為聖朝累。



 初,惇、卞以次升在元祐間外遷,意其不能無怨望,卞又與同鄉里,故延置憲府,欲使出力為助,
 擠排眾賢;而一無所附。時方編元祐章疏,毒流搢紳。次升言:「陛下初即位,首下詔令,導人使諫;親政以來,又揭敕榜,許其自新。今若考一言之失,致於譴累,則前之詔令適所以誤天下,後之敕榜適所以誑天下,非所以示大信也。」又論卞客周穜貪鄙,鄭居中憸佞。由是惇、卞交惡之,使所善太府少卿林顏致己意,嘗以美官。次升曰:「吾知守官而已,君為天子卿士,而為宰相傳風旨邪?」惇、卞益不樂,乘間白為河北轉運使,帝曰:「漕臣易
 得耳,次升敢言,不當去。」更進左司諫。



 宣仁有追廢之議,次升密言:「先太后保祐聖躬,始終無間,願勿聽小人銷骨之謗。」帝曰:「卿安所聞?」對曰:「臣職許風聞,陛下毋詰其所從來可也。」呂升卿察訪廣南,次升言:「陛下無殺流人之意,而遣升卿出使。升卿資性慘刻,喜求人過,今使逞志釋憾,則亦何所不至哉?」乃止不遣。



 次升累章劾章惇,皆留中。帝嘗謂曰:「章惇文字勿令絕。」次升退告王鞏,鞏曰:「君胡不云:諫臣,耳目也;帝王,心也。心所不知,則耳目為之傅
 達;既知之,何以耳目為?」居數日,復入見,帝申前旨,乃以鞏語對。帝曰:「然。顧未有代之者爾。」訖不克去。京師富家乳婢怨其主,坐兒於上而嵩呼者三。邏系獄。次升乞戒有司無得觀望。帝問大臣何謂,蔡卞曰:「正謂觀望陛下爾。」誣其毀先烈,擬謫監全州酒稅,帝以為遠,改南安軍。



 徽宗立,召為侍御史。極論惇、卞、曾布、蔡京之惡,竄惇於雷,居卞於池,出京於江寧。遷右諫議大夫。獻體道、稽古、修身、仁民、崇儉、節用六事,言多規切。崇寧初,以寶文閣待
 制知穎昌府,降集賢殿修撰,繼又落修撰,除名徙建昌,編管循州,皆以論京、卞故。政和中,用赦恩復舊職。卒,年七十六。



 次升三居言責,建議不茍合,劉安世稱其有功於元祐人,謂能遏呂升卿之行也。它所言曾肇、王覿、張庭堅、賈易、李昭玘、呂希哲、范純禮、蘇軾等,公議或不謂然。



 陳師錫字伯修,建州建陽人。熙寧中,游太學,有俊聲。神宗知其材,及廷試,奏名在甲乙間,帝偶閱其文,屢讀屢
 嘆賞,顧侍臣曰:「此必陳師錫也。」啟封果然,擢為第三。調昭慶軍掌書記,郡守蘇軾器之,倚以為政。軾得罪,捕詣臺獄,親朋多畏避不相見,師錫獨出餞之,又安輯其家。



 知臨安縣,為監察御史。上言:「宋興,享國長久號稱太平者,莫如仁宗,切考致治之本,不過延直言,御群下,進善退邪而已。明道中,親覽萬幾,見政事之多闢,輔佐之失職,自呂夷簡、張耆、夏竦、陳堯佐、範雍、晏殊等,一日罷去。寶元初,冬雷地震,用諫官韓琦之言,王隨、陳堯佐、韓億、
 石中立同時見黜。其後,不次擢用杜衍、範仲淹、富弼、韓琦,以成慶歷、嘉祐之治。願稽皇祖納諫、御臣之意,以興治功。」帝善其言。



 時詔進士習律,師錫言:「陛下方大闡學校,用經術訓迪士類,不應以刑名之學亂之。夫道德,本也;刑名,末也。教之以本,人猶趨末,況教之以末乎?望追寢其制,使得悉意本業。」用事者謂倡為詖說,出知宿遷縣。



 元祐初,蘇軾三上章,薦其學術淵源,行己潔素,議論剛正,器識靖深,德行追蹤於古人,文章冠絕於當世。乃
 入為秘書省校書郎,遷工部員外郎,加秘閣校理,提點開封縣鎮。建言:「銓法,選人用舉者遷升,而歲有定額。今請托者溢數,而寒畯有不足之患,請為之限約。」畿內將官苛慘失士心,方大閱,群卒嘩噪,將吏莫知所為。師錫馳至軍,推首惡者致諸法,按閱如初,而劾斥其將,縣人嘆服。樞密院猶以事不先白為罪,罷知解州。歷考功員外郎,知宣州、蘇州。



 徽宗立,召拜殿中侍御史。疏言:「元豐之末,中外洶洶矣。宣仁聖後再安天下,委國而治者,司
 馬光、呂公著爾。章惇誣其包藏禍心,至於追貶。天相陛下,發潛繼統,而惇猶據高位,光等贈謚未還,墓碑未復。願早攄宸略,以慰中外之望。」



 蔡京為翰林學士,師錫言:「京與弟卞同惡,迷國誤朝。而京好大喜功,銳於改作,日夜交結內侍、戚里,以覬大用。若果用之,天下治亂自是而分,祖宗基業自是而隳矣。京援引死黨至數百人,鄧洵武內行污惡,搢紳不齒,豈可滓穢史筆?向宗回、宗良亦陰為京助。是皆國之深患,為陛下憂,為宗廟憂,為賢
 人君子憂。若出之於外,社稷之福也。」帝曰:「此於東朝有礙,卿為我處之。」對曰:「審爾,臣當具白太后。」遂上封事言:「自昔母后臨朝,危亂天下,載在史冊,可考而知。至於手書還政,未有如聖母,退抑謙遜,真可為萬世法。而蔡京陰通二向,妄言宮禁預政,以誣聖德,不可不察也。」



 詔索秘閣圖畫,師錫言:「《六經》載道,諸子言理,歷代史籍,祖宗圖畫,天人之蘊,性命之妙,治亂安危之機,善惡邪正之跡在焉。望留意於此,以唐山水圖代《無逸》為監。」



 俄改考
 功郎中,師錫抗章言曰:「臣在職數月,所言皆當今急務。若以為非,陛下方開納褒獎;若以為是,則不應遽解言職。如蔡京典刑未正,願受竄貶。」於是出知穎、廬、滑三州。坐黨論,監衡州酒;又削官置郴州。卒,年六十九。師錫始與陳瓘同論京、卞,時號「二陳」。紹興中,贈直龍圖閣。



 彭汝礪,字器資,饒州鄱陽人。治平二年,舉進士第一。歷保信軍推官、武安軍掌書記、潭州軍事推官。王安石見其《詩義》,補國子直講,改大理寺丞,擢太子中允,既而惡
 之。



 御史中丞鄧綰將舉為御史,召之不往;既上章,復以失舉自列。神宗怒,逐綰,用汝礪為監察御史裏行。首陳十事:一正己,二任人,三守令,四理財,五養民,六振救,七興事,八變法,九青苗,十鹽事。指擿利害,多人所難言者。又論呂嘉問市易聚斂非法,當罷;俞充諂中人王中正,至使妻拜之,不當檢正中書五房事。神宗為罷充,詰其語所從,汝礪曰:「如此,非所以廣聰明也。」卒不奉詔。及中正與李憲主西師,汝礪言不當以兵付中人,因及漢、唐
 禍亂之事。神宗不懌,語折之。汝礪拱立不動,伺間復言,神宗為改容,在廷者皆嘆服。宗室以女賣婚民間,有司奏罷之。汝礪言:「此雖疏屬,皆天家子孫,不可使閭閻之賤得以貨取,願更著婚法。」



 元豐初,以館閣校勘為江西轉運判官,陛辭,復言:「今不患無將順之臣,患無諫諍之臣;不患無敢為之臣,患無敢言之臣。」神宗嘉其忠藎。代還,提點京西刑獄。



 元祐二年,召為起居舍人。時相問新舊之政,對曰:「政無彼此,一於是而已。今所更大者,取士
 及差役法,行之而士民皆病,未見其可。」逾年,遷中書舍人,賜金紫。詞命雅正,有古人風。其論詩體四韻事尤力,大臣有持平者,頗相左右,一時進取者疾之,欲排去其類,未有以發。



 會知漢陽軍吳處厚得蔡確安州詩上之,傅會解釋,以為怨謗。諫官交章請治之,又造為危言,以激怒宣仁後,欲置之法。汝礪謂此羅織之漸也,數以白執政,不能救,遂上疏論列,不聽。方居家待罪,得確謫命除目草詞,曰:「我不出,誰任其責者。」即入省,封還除目,辨
 論愈切。諫官指汝礪為朋黨,宣仁後曰:「汝礪豈黨確者,亦為朝廷論事爾。」及確貶新州,又須汝礪草詞,遂落職知徐州。初,汝礪在臺時,論呂嘉問事,與確異趣,徙外十年,確為有力。後治嘉問它獄,以不阿執政,坐奪二官。至是,又為確得罪,人以此益賢之。



 加集賢殿修撰,入權兵、刑二部侍郎。有獄當貸,執政以特旨殺之,汝礪持不下。執政怒,罰其屬。汝礪言:「制書有不便,許奏論。汝礪屬又何罪?」遂自劾請去,章四上。詔免屬罰,徙汝礪禮部,真拜
 吏部侍郎。



 哲宗躬聽斷,修熙寧、元豐政事,人皆爭獻所聞,汝礪獨無建白。或問之,答曰:「在前日則無敢言,於今則人人能言之矣。」進權吏部尚書。言者謂嘗附會劉摯,以寶文閣直學士知成都府。未行,章數上,又降待制、知江州。將行,哲宗問所欲言,對曰:「陛下今所復者,其政不能無是非,其人不能無賢否。政惟其是,則無不善;人惟其賢,則無不得矣。」



 至郡數月而病去。其遺表略云:「土地已有餘,願撫以仁;財用非不饒,願節以禮。佞人初若可
 悅,而其患在後;忠言初若可惡,而其利甚博。」至於恤河北流移,察江南水旱,凡數百言。朝廷方以樞密都承旨命之而已卒,乃以告賜其家。年五十四。



 汝礪讀書為文,志於大者,言動取舍,必合於義,與人交,必盡誠敬。兄無子,為立後,官之。少時師事桐廬倪天隱,既死,並其母妻葬之,且衣食其女。同年生宋渙死,經理其後,不啻如子。所著《易義》、《詩義》、《詩文》凡五十卷。弟汝霖、汝方。



 汝霖字巖老。第進士,以曾布薦,為秘書丞,擢殿中侍御
 史,由是附布。時紹述之論復興,都水丞李夷行乞復詩賦,汝霖劾之。韓忠彥議權合祭,汝霖言其非禮。遷侍御史。門下侍郎李清臣與布異,布先諷江公望使擊之,將處以諫議大夫,公望弗聽。汝霖竟逐清臣,果得諫議。



 鞫趙諗反獄,窮其黨與。元祐禍再興,吳材、王能甫排斥不已,汝霖言:「諸人罪狀,已經紹聖出削,案籍具在,但可據以行,不必候指名彈擊。」於是司馬光以下復貶。布失位,汝霖罷知泰州,又謫濮州團練副使。後以顯謨閣待制
 卒。



 汝方字宜老。以汝礪蔭為滎陽尉、臨城主簿。汝礪卒,棄官歸葬。豐稷留守南京,闢司錄。宣和初,通判衢州,使者疏其治狀,擢知州事。



 方臘起睦之青溪,與衢接境。寇至,無兵可御,眾望風奔潰。汝方獨與其僚段約介守孤城,三日而陷,罵賊以死,年六十六。徽宗褒嘆之,超贈龍圖閣直學士、通議大夫,謚曰忠毅,官其家七人。



 呂陶,字符鈞,成都人。蔣堂守蜀,延多士入學,親程其文,
 嘗得陶論,集諸生誦之,曰:「此賈誼之文也。」陶時年十三,一坐皆驚。由是禮諸賓筵。一日,同游僧舍,共讀寺碑,酒闌,堂索筆書碑十紙,行斷句闕,以示陶曰:「老夫不能盡憶,子為我足之。」陶書以獻,不繆一字。



 中進士第,調銅梁令。民龐氏姊妹三人冒隱幼弟田,弟壯,訴官不得直,貧至庸奴於人。及是又訴。陶一問,三人服罪,弟泣拜,願以田半作佛事以報。陶曉之曰:「三姊皆汝同氣,方汝幼時,適為汝主之爾;不然,亦為他人所欺。與其捐半供佛,曷
 若遺姊,復為兄弟,顧不美乎?」弟又拜聽命。



 知太原壽陽縣。府帥唐介闢簽書判官,暇日促膝晤語,告以立朝事君大節,曰:「君廊廟人也。」以介薦,應熙寧制科。時王安石從政,改新法,陶對策枚數其過,大略謂:「賢良之旨,貴犯不貴隱。臣愚,敢忘斯義?陛下初即位,願不惑理財之說,不間老成之謀,不興疆埸之事。陛下措意立法,自謂庶幾堯、舜,然陛下之心如此,天下之論如彼,獨不反而思之乎?」及奏第,神宗顧安石取卷讀,讀未半,神色頗沮。神宗
 覺之,使馮京竟讀,謂其言有理。司馬光、範鎮見陶,皆曰:「自安石用事,吾輩言不復效,不意君及此,平生聞望,在茲一舉矣。」



 安石既怒孔文仲,科亦隨罷,陶雖入等,才通判蜀州。張商英為御史,請廢永康軍,下旁郡議,陶以為不可。及知彭州,威、茂夷入寇,陶召大姓潛具守備,城門啟閉如平時,因以永康前議上於朝,軍遂不廢。



 王中正為將,蜀道畏,事之甚謹,而其所施悉謬戾,陶奏召還之。李杞、蒲宗閔來榷茶,西州騷動。陶言:「川蜀產茶,視東
 南十不及一,諸路既皆通商,兩川獨蒙禁榷。茶園本是稅地,均出賦租,自來敷賣以供衣食,蓋與解鹽、晉礬不同。今立法太嚴,取息太重,遂使良民枉陷刑闢,非陛下仁民愛物之意也。」宗閔怒,劾其沮敗新法,責監懷安商稅。或往吊之,陶曰:「吾欲假外郡之虛名,救蜀民百萬之實禍。幸而言行,所濟多矣。敢有榮辱進退之念哉。」起知廣安軍,召為司門郎中。



 元祐初,擢殿中待御史,首獻邪正之辨曰:「君子小人之分辨,則王道可成,雜處於朝,則
 玫體不純。今蔡確、韓縝、張璪、章惇,在先朝,則與小人表裏,為賊民害物之政,使人主德澤不能下流;在今日,則觀望反復,為異時子孫之計。安燾、李清臣又依阿其間,以伺勢之所在而歸之。昔者負先帝,今日負陛下。願亟加斥逐,以清朝廷。」於是數人相繼罷去。



 時議行差役,陶言:「郡縣風俗異制,民之貧富不均,當此更法之際,若不預設防禁,則民間雖無納錢之勞,反有偏頗之害。莫若以新舊二法,裁量厥中。」會陶謁告歸,詔於本道定議。陶
 考究精密,民以為便。還朝,遂正兩路轉運使李琮、蒲宗閔之罪;又奏十事,皆利害切於蜀者。



 蘇軾策館職,為朱光庭所論,軾亦乞補郡,爭辨不已。陶言:「臺諫當徇至公,不可假借事權以報私隙。議者皆謂軾嘗戲薄程頤,光庭乃其門人,故為報怨。夫欲加軾罪,何所不可,必指其策問以為譏謗,恐朋黨之敝,自此起矣。」由是兩置之。



 陶與同列論張舜民事不合,傅堯俞、王巖叟攻之,太皇太后不納,遷陶左諫議,繼出為梓州、淮西、成都路轉運副
 使。入拜右司郎中、起居舍人。大臣上殿,有乞屏左右及史官者,陶曰:「屏左右已不可,況史官乎?大臣奏事而史官不得聞,是所言私也。」詔定為令。遷中書舍人。奏使契丹歸,乞修邊備。哲宗喜曰:「臣僚言邊事,惟及陜西,不及河北。殊不知河北有警,則十倍陜西矣!卿言甚善。」進給事中。



 哲宗始親政,陶言:「太皇保祐九年,陛下所深知,尊而報之,惟恐不盡。然臣猶以無可疑為疑,不必言而言,萬一有奸邪不正之謀,上惑淵聽,謂某人宜復用,某事
 宜復行,此乃治亂安危之機,不可不察也。」俄以集賢院學士知陳州,徙河陽、潞州,例奪職,再貶庫部員外郎,分司。徽宗立,復集賢殿修撰、知梓州,致仕。卒年七十七。



 張庭堅,字才叔,廣安軍人。進士高第,調成都觀察推官,為太學《春秋》博士。紹聖經廢,通判漢州。入為樞密院編修文學,坐折簡別鄒浩免。徽宗召對,除著作佐郎,擢右正言。帝方銳意圖治,進延忠鯁,庭堅與鄒浩、龔□、江公望、常安民、任伯雨皆在諫列,一時翕然稱得人。



 庭堅在
 職逾月,數上封事,其大要言:「世之論孝,必曰紹復神考,然後謂孝。夫前後異宜,法亦隨變,而欲纖悉必復,然則將敝於一偏,久必有不便於民而招怨者,如此而謂之孝,可乎?司馬光因時變革,以便百姓,人心所歸,不為無補於國家;陳瓘執義論諍,將以去小人,士論所推,不為無益於宮禁。乞盡復光贈典以悅人心,召還瓘言職以慰士論。又士大夫多以繼志述事勸陛下者,臣恐必有營私之人,欲主其言以自售,謂復紹先烈非其徒不可,
 將假名繼述,而實自肆焉。今遠略之耗於內者,棄不以為守,則兵可息;特旨之重於法者,刪不以為例,則刑可省。近以青唐反叛,棄鄯守湟。既以鄯為可棄,則區區之湟,亦安足守?臣謂並棄湟州便。」庭堅言論深切,退輒焚稿。



 是時,議者往往指元祐舊臣在廷者太多。庭堅為帝言司馬光、呂公著之賢,且曰:「陛下踐阼以來,合人心事甚眾,惟夫邪正殊未差別。如光、公著甄敘,但用赦恩,初未嘗別其無罪也。」又薦蘇軾、蘇轍可用,頗忤旨。曾布因
 稱其所論不常,帝命徙為郎,俄出為京東轉運判官。任伯雨言庭堅立身有本末,不應罷言職。庭堅亦辭新命,改知汝州,又送吏部。伯雨復爭之,乞以庭堅章付外,考其所言,毋使言者為三省所脅。李清臣從而擠之,改通判陳州。



 初,蔡京守蜀,庭堅在幕府與相好。及京還朝,欲引以為己用,先令鄉人諭意,庭堅不肯往。京大恨,後遂列諸黨籍。又坐嘗談瑤華非辜事,編管虢州,再徙鼎州、象州。久之,復故官。卒,年五十七。紹興初,詔贈直徽猷
 閣。



 龔夬,字彥和,瀛州人。清介自守,有重名。進士第三,簽書河陽判官。從曾布於瀛。紹聖初,擢監察御史,以親老,求通判相州,知洺州。



 徽宗立,召拜殿中侍御史。始上殿,即抗疏請辨忠邪,曰:「好惡未明,則人迷所向;忠邪未判,則眾必疑。今聖政日新,遠近忻悅,進退人材,皆出睿斷,此甚盛之舉也。然奸黨既破,必將早夜熟計,廣為身謀。或遽革面以求自文,或申邪說以拒正論,或詭稱禍福以
 動朝廷,或托言祖宗以脅人主。巧事貴戚,陰結左右,變亂是非,奸計百出,幸其既敗復用,已去復留。君子直道而行,則必墮其術中。然則天下治忽,未可知也。故宜洞察忠邪,行之以決。若小不忍,則害大政。臣願陛下明好惡以示之,使遠近知進賢退奸之意,太平之治,不難致也。」又言:「朝廷累下赦令,洗滌元祐愆負被坐之人,至於官職資蔭,多未給還。願申詔有司,亟為施行,以伸先帝寬仁之意。」



 時章惇、蔡卞用事,□首論其惡,大略以為:「昔
 日丁謂當國,號為恣睢,然不過陷一寇準而已。及至於惇,而故老、元輔、侍從、臺省之臣,凡天下之所謂賢者,一日之間,布滿嶺海,自有宋以來,未之聞也。當是時,惇之威勢震於海內,此陛下所親見。蓋其立造不根之語,文致悖逆之罪,是以人人危懼,莫能自保,俾其朽骨銜冤於地下,子孫禁錮於炎荒,忠臣義士,憤悶而不敢言,海內之人,得以歸怨先帝。其罪如此,尚何俟而不正典刑哉?卞事上不忠,懷奸深理,凡惇所為,皆卞發之,為力居
 多。望採之至公,昭示譴黜。」又論:「蔡京治文及甫獄,本以償報私仇,始則上誣宣仁,終則歸咎先帝,必將族滅無辜,以逞其欲。臣料當時必有案牘章疏,可以見其鍛煉附會。如方天若之兇邪,而京收置門下,賴其傾險,以為腹心,立起犴獄,多斥善士,天下冤之,皆京與天若為之也。願考證其實,以正奸臣之罪。」於是三人者皆去。



 又上疏乞正元祐後冊位號,及元符後不當並立,書報聞。已而元祐後冊再廢,言者論夬首尾建言,詔削籍,編管房
 州。繼徙象,又徙化。徒步適貶所,持扇乞錢以自給。逢赦令得歸,政和元年卒,年五十五。紹興元年,贈直龍圖閣。六年,再贈右諫議大夫,官其後二人。



 弟大壯,少有重名,清介自立。從兄官河陽,曾布欲見之,不可得,乃往謁夬,邀之出,從容竟日,題詩壁間,有「得見兩龔」之語。夬為御史,大壯勸使早去,夬以為畏友。不幸早卒。



 孫諤,字符忠,睢陽人。父文用,以信厚稱鄉里,死謚慈靜居士。諤少挺特不群,為張方平所器。登進士第,調哲信
 主簿,選為國子直講。陷虞蕃獄,免。



 元祐初,起為太常博士,遷丞。哲宗卜後,太史惑陰陽拘忌之說,諤上疏太皇太后言:「家人委巷之語,不足以定大計,願斷自聖慮。」出為利、梓路轉運判官,召拜禮部員外郎、左正言。



 紹聖治元祐黨,諤言:「漢、唐朋黨之禍,其監不遠。」蹇序辰編類章疏,諤又言:「朝廷當示信,以靜安天下,請如前詔書,一切勿問。」嘗侍對,論星文變咎,願修省消復,罷幸西池及寢內降除授。帝每患臺諫乏人,諤曰:「士豈乏於世,顧陛下
 不知爾。」立疏可用者二十二人。章惇惡其拂己,出知廣德軍,徙唐州,提點湖南刑獄。



 徽宗立,復為右司諫,首論大臣邪正、政事可廢置因革者,帝稱其鯁直。議者欲以群臣封事付外詳定,諤言:「君不密則失臣,是將速忠臣之禍矣,不宜宣洩。」乃止。遷左司諫,俄以疾卒。



 諤與彭汝礪以氣節相尚,汝礪亡,諤語所知曰:「吾居言責,不愧器資於地下矣。」及再入諫省,不能旬月,時論惜之。



 陳軒,字符輿,建州建陽人。進士第二,授平江軍節度推
 官。元祐中,為禮部郎中、徐王翊善,再遷中書舍人。上疏言:「祖宗舊制,諸道帥守、使者辭見之日,並召對便殿,非特可以周知利害,亦可觀閱人才。今視朝數刻而退,惟執政大臣得在帝所,或經旬閱月,臺諫官乃得覲,餘皆無因而前,殆非所謂廣覽兼聽之道。願詔有司,使如故事。」又言:「所在巡檢,招惰游惡少以隸土軍,習暴橫,為田野患,請使以廂卒代。」皆從之。高麗入貢,軒館客,其使求市歷代史、《策府元龜》,抄鄭、衛曲譜,皆為上聞。禮部尚書
 蘇軾劾其失體,以龍圖閣待制知廬州,徙杭州、江寧穎昌府。



 徽宗立,為兵部侍郎兼侍讀。論監司、守臣數易之弊,如江、淮發運使,十五年間至更三十二人,願稍久其任。又言:「比更定役法,欲以寬民力,而有司生事,包切茍營贏羨。散青苗以抑兼並,拯難困,不當以散多予賞。」入侍經闈,每勸帝以治貴清凈,願法文、景之恭儉,帝頗聽行之。加龍圖閣直學士、知成都府,不行,改杭州、福州。卒,年八十四。



 江公望,字民表,睦州人。舉進士。建中靖國元年,由太常博士拜左司諫。時御史中丞趙挺之與戶部尚書王古用赦恩理逋欠,古多所蠲釋,挺之劾古傾天下之財以為私惠。公望以為天子登極大赦,將與天下更始,故一切與民,豈容古行私惠於其間,乃上疏曰:「人君所以知時政之利病、人臣之忠邪,無若諫官、御史之為可信。若飾情肆誣,快私忿以罔上聽,不可不察也。臣聞挺之與古論事每不相合,屢見於辭氣,懷不平之心,有待而發。
 俚語有之,『私事官仇』,比小人之所不為,而挺之安為之,豈忠臣乎?」



 又上疏曰:「自哲宗有紹述之意,輔政非其人,以媚於己為同,忠於君為異。一語不合時學,必目為流俗;一談不俟時事,必指為橫議。借威柄以快私隙,必以亂君臣父子之名分感動人主,使天下騷然,泰陵不得盡繼述之美。元祐人才,皆出於熙寧、元豐培養之餘,遭紹聖竄逐之後,存者無幾矣。神考與元祐之臣,其先非有射鉤斬祛之隙也,先帝信仇人而黜之。陛下若立元
 祐為名,必有元豐、紹聖為之對,有對則爭興,爭興,則黨復立矣。陛下改元詔旨,亦稱思建皇極,蓋嘗端好惡以示人,本中和而立政,皇天后土,實聞斯言。今若欲渝之,奈皇天后土何?」



 內苑稍蓄珍禽奇獸,公望力言非初政所宜。它日入對,帝曰:「已縱遣之矣,唯一白鷴畜之久,終不肯去。」先是,帝以柱杖逐鷴,鷴不去,乃刻公望姓名於杖頭,以識其諫。蔡王似府史以語言疑似成獄,公望極言論救,出知淮陽軍。未幾,召為左司員外郎,以直龍圖
 閣知壽州。蔡京為政,編管南安軍。遇赦還家,卒。建炎中,與陳瓘同贈右諫議大夫。



 陳祐,字純益,仙井人。第進士。元符末,以吏部員外郎拜右正言。上疏徽宗曰:「有旨令臣與任伯雨論韓忠彥援引元祐臣僚事。按賈易、岑象求、豐稷、張來、黃庭堅、龔原、晁補之、劉唐老、李昭玘人才均可用,特跡近嫌疑而已。今若分別黨類,天下之人,必且妄意陛下逐去元祐之臣,復興紹聖政事。今紹聖人才比肩於朝,一切不問;元
 祐之人數十,輒攻擊不已,是朝廷之上,公然立黨也。」



 遷右司諫。言:「林希紹聖初掌書命,草呂大防、劉摯、蘇轍、梁燾等制,皆務求合章惇之意。陛下頃用臣言褫其職,自大名移揚州,而希謝表具言皆出於先朝。大抵奸人詆毀善類,事成則攄己所憤,事敗則歸過於君。至如過失未形而訓辭先具,安得為責人之實?歷辨詆誣而上侵聖烈,安得為臣子之誼?不一二年,致位樞近,而希尚敢忿躁不平,謝章慢上不敬。此而可忍,孰可不忍!」希再降
 知舒州。又論章惇、蔡京、蔡卞、郝隨、鄧洵武,忤旨,通判滁州。卞乞貶伯雨等,祐在數中,編管澧州,徙歸州。復承議郎,卒。



 常安民,字希古,邛州人。年十四,入太學,有俊名。熙寧以經取士,學者翕然宗王氏,安民獨不為變。春試,考第一,主司啟封,見其年少,欲下之。判監常秩不可,曰:「糊名較藝,豈容輒易?」具以白王安石。安石稱其文,命學者視以為準,由是名益盛。安石欲見之,不肯往。登六年進士舉,
 神宗愛其策,將使魁多士。執政謂其不熟經學,列之第十。



 授應天府軍巡判官,選成都府教授。與安惇為同僚,惇深刻奸詐,嘗偕謁府帥,輒毀素所厚善者。安民退謂惇曰:「若人不厚於君乎?何詆之深也。」惇曰:「吾心實惡之,姑以為面交爾。」安民曰:「君所謂匿怨而友其人,乃李林甫也。」惇笑曰:「直道還君,富貴輸我。」安民應之曰:「處厚貴,天下事可知,我當歸山林,豈復與君校是非邪!第恐累陰德爾。」後惇貴,遂陷安民,而惇子坐法誅死,如安民言。
 秩滿寓京師。妻孫氏與蔡確之妻,兄弟也。確時為相,安民惡其人,絕不相聞。確夫人使招其妻,亦不往。調知長洲縣,以主信為治,人不忍欺。縣故多盜,安民籍嘗有犯者,書其衣,揭其門,約能得它盜乃除,盜為之息。追科不下吏,使民自輸,先它邑以辦。轉運使許懋、孫昌齡入境,邑民頌其政,皆稱為古良吏。元祐初,李常、孫覺、範百祿、蘇軾、鮮于侁連章論薦,擢大理、鴻臚丞。



 是時,元豐用事之臣,雖去朝廷,然其黨分布中外,起私說以搖時政。安
 民竊憂之,貽書呂公著曰:「善觀天下之勢,猶良醫之視疾,方安寧無事之時,語人曰:『其後必將有大憂』,則眾必駭笑。惟識微見幾之士,然後能逆知其漸。故不憂於可憂,而憂之於無足憂者,至憂也。今日天下之勢,可為大憂。雖登進忠良,而不能搜致海內之英才,使皆萃於朝,以勝小人,恐端人正士,未得安枕而臥也。故去小人不為難,而勝小人為難。陳蕃、竇武協心同力,選用名賢,天下想望太平,然卒死曹節之手,遂成黨錮之禍。張柬之
 五王中興唐室,以謂慶流萬世,及武三思一得志,至於竄移淪沒。凡此者皆前世已然之禍也。今用賢如倚孤棟,拔士如轉巨石,雖有奇特瑰卓之才,不得一行其志,甚可嘆也。猛虎負嵎,莫之敢攖,而卒為人所勝者,人眾而虎寡也。故以十人而制一虎則人勝,以一人而制十虎則虎勝,奈何以數十人而制千虎乎?今怨忿已積,一發其害必大,可不謂大憂乎。」及章惇作相,其言遂驗。



 歷太常博士,轉為丞。與少卿朱光庭論不合,出為江西轉
 運判官,不行,改宗正丞。蘇轍薦為御史,宰相不樂,除開封府推官。紹聖初,召對,為哲宗言:「今日之患,莫大於士不知恥。願陛下獎進廉潔有守之士,以厲風俗。元祐進言者,以熙、豐為非,今之進言者反是,皆為偏論。願公聽並觀,擇其中而歸於當。」拜監察御史。論章惇顓國植黨,乞收主柄而抑其權,反復曲折,言之不置。惇遣所親信語之曰:「君本以文學聞於時,奈何以言語自任,與人為怨?少安靜,當以左右相處。」安民正色斥之曰:「爾乃為時
 相游說邪?」惇益怒。



 中官裴彥臣建慈云院,戶部尚書蔡京深結之,強毀人居室。訴於朝,詔御史劾治。安民言:「事有情重而法輕者,中官豪橫。與侍從官相交結,同為欺罔,此之奸狀,恐非法之所能盡。願重為降責,以肅百官。」獄具,惇主之甚力,止罰金。安民因論京:「奸足以惑眾,辨足以飾非,巧足以移奪人主之視聽,力足以顛倒天下之是否。內結中官,外連朝士,一不附己,則誣以黨於元祐;非先帝法,必擠之而後已。今在朝之臣,京黨過半,陛
 下不可不早覺悟而逐去之。他日羽翼成就,悔無及矣。」是時,京之奸始萌芽,人多未測,獨安民首發之。



 又言:「今大臣為紹述之說,皆借此名以報復私怨,朋附之流,遂從而和之。張商英在元祐時上呂公著詩求進,諛佞無恥,近乞毀司馬光及公著神道碑。周秩為博士,親定光謚為文正,近乃乞斫棺鞭尸。陛下察此輩之言,果出於公論乎?」章疏前後至數十百上,度終不能回,遂丐外,帝慰勉而已。



 大饗明堂,劉賢妃從侍齋宮。安民以為萬眾
 觀瞻,虧損聖德,語頗切直,帝微怒。曾布始以安民數憾章惇,意其附已,屢稱之於朝。其後並論,曾布亦恨,於是與惇比而排之,乃取其所貽呂公著書白於帝。它日,帝謂安民曰:「卿所上宰相書,比朕為漢靈帝,何也?」安民曰:「奸臣指擿臣言,推其世以文致臣爾,雖辨之,何益?」



 董敦逸再為御史,欲劾蘇軾兄弟,安民謂二蘇負天下文章重望,恐不當爾。至是,敦逸奏之,詔與知軍,惇徑擬監滁州酒稅。至滁,日親細務。郡守曾肇約為山林之游,曰:「謫
 官例不治事。」安民謝曰:「食焉而怠其事,不可。」滿三歲,通判溫州。



 徽宗立,朝論欲起為諫官,曾布沮之,以提點永興軍路刑獄。蔡京用事,入黨籍,流落二十年。政和末,卒,年七十。建炎四年,贈右諫議大夫。子同,為御史中丞,自有傳。



 論曰:次升從容一言,止呂升卿之使嶺南,大有功於元祐諸臣。師錫謂蔡京若用,天下治亂自是而分,惜其言不行於當時,而徒有驗於其後。汝礪辨救蔡確,以直報
 怨。陶言榷茶為西南害,毅然觸蒲、李之鋒。庭堅論紹復未足以盡孝道。諤言世非乏士,患上不知,乃薦可用者二十有二人,號稱鯁直,裨益尤多。軒力陳青苗貽害,願以清凈為治。祐擊林希,且論惇、京、卞輩,斥死弗悔。公望謂神宗於元祐諸臣非有射鉤。



\end{pinyinscope}