\article{列傳第五十六}

\begin{pinyinscope}

 孔道輔子宗翰鞠詠劉隨曹修古郭勸段少連



 孔道輔,字原魯,初名延魯,孔子四十五代孫也。父勖,進士及第,為太平州推官,以殿中丞通判廣州。會真宗東
 封,躬詣孔子祠。帝問宰相:「孔氏今孰為名者?」或言勖有治行,即召對,以為太常博士、知曲阜縣。初,勖在廣州,以清潔聞,及被召,蕃酋爭持寶貨以獻,皆慰遣之。後為御臺推直官,累遷秘書監、分司南京,管勾祖廟,以尚書工部侍郎致仕。後道輔卒,年八十九。



 道輔幼端重,舉進士第,為寧州軍事推官,數與州將爭事。有蛇出天慶觀真武殿中,一郡以為神,州將帥官屬往奠拜之,欲上其事。道輔徑前以笏擊蛇,碎其首,觀者初驚,後莫不嘆服。
 遷大理寺丞、知仙源縣,主孔子祠事。孔氏故多放縱者,道輔一繩以法。上言廟制庳陋,請加修崇,詔可。再遷太常博士。章獻太后臨朝,召為左正言。受命日,論奏樞密使曹利用,尚御藥羅崇勛竊弄威柄,宜早斥去,以清朝廷。立對稱刻,太后可其言,乃退。未幾,為直史館、判三司理欠憑由司。



 奉使契丹,道除右司諫、龍圖閣待制。契丹晏使者,優人以文宣王為戲,道輔艴然徑出。契丹使主客者邀道輔還坐,且令謝之。道輔正色曰:「中國與北朝
 通好,以禮文相接。今俳優之徒,慢侮先聖而不之禁,北朝之過也。道輔何謝!」契丹君臣默然,又酌大卮謂曰:「方天寒,飲此,可以致和氣。」道輔曰:「不和,固無害。」既還,言者以為生事,且開爭端。仁宗問其故,對曰:「契丹比為黑水所破,勢甚蹙。平時漢使至契丹,輒為所侮,若不較,恐益慢中國。」帝然之。歷判吏部流內銓、糾察在京刑獄。坐糾事不當,出知鄆州,徙青州。還判流內銓,遷尚書兵部員外郎,復出知徐、許二州,徙應天府。



 明道二年,召為右諫
 議大夫、權御史中丞。會郭皇后廢,道輔率諫官孫祖德、範仲淹、宋郊、劉渙,御史蔣堂、郭勸、楊偕、馬絳、段少連十人,詣垂拱殿伏奏:「皇后天下之母,不當輕議絀廢。願賜對,盡所言。」帝使內侍諭道輔等至中書,令宰相呂夷簡以皇后當廢狀告之。道輔語夷簡曰:「大臣之於帝後,猶子事父母也;父母不和,可以諫止,奈何順父出母乎?」夷簡曰:「廢後有漢、唐故事。」道輔復曰:「人臣當道君以堯、舜,豈得引漢、唐失德為法邪?」夷簡不答,即奏言:「伏閣請對,
 非太平美事。」於是出道輔知泰州。明日晨,入至待漏,聞有詔,亟馳出城。頃之,徙徐州,又徙兗州,進龍圖閣直學士,遷給事中。在兗三年,復入為御史中丞。



 道輔性鯁挺特達,遇事彈劾無所避,出入風採肅然,及再執憲,權貴益忌之。初,道輔與其父里中僦郭贄舊宅居之,有言於帝者曰:「道輔家近太廟,出入傳呼,非所以尊神明。」即詔道輔他徙。集賢校理張宗古上言,漢內史府在太廟□耎垣中,國朝以來,廟垣下皆有官私第舍,謂不須避。帝出
 宗古通判萊州。道輔嘆曰:「憸人之言入矣!」



 會受詔鞠馮士元獄,事連參知政事程琳。宰相張士遜素惡琳,而疾道輔不附己,將逐之,察帝有不悅琳意,即謂道輔:「上顧程公厚,今為小人所誣,見上,為辨之。」道輔入對,言琳罪薄不足深治。帝果怒,以道輔朋黨大臣,出知鄆州。已而道輔知為士遜所賣,頗憤惋。時大寒上道,行至韋城,發病卒,天下莫不以直道許之。皇祐三年,王素因對語及道輔,仁宗思其忠,特贈尚書工部侍郎。子宗翰。



 宗翰字周翰。登進士第,知仙源縣,而為治有條理,遇族人有恩,不以私故骫法。王珪、司馬光皆上章論薦,由通判陵州為夔峽轉運判官,提點京東刑獄、知虔州。城濱章、貢、兩江,歲為水嚙。宗翰伐石為址,冶鐵錮之,由是屹然,詔書褒美。歷陜、揚、洪、兗州,皆以治聞。哲宗初立求言,吏民上書以千數,詔司馬光採閱其可用者十五人,獨稱獎其二,乃宗翰與王鞏也。



 元祐初,召為司農少卿,遷鴻臚卿。言:「孔子之後,自漢以來有褒成、奉聖、宗聖之號,
 皆賜實封或縑帛,以奉先祀。至於國朝,益加崇禮。真宗東封臨幸,賜子孫世襲公爵,然兼領他官,不在故郡,於名為不正。講自今襲封之人,使終身在鄉里。」詔改衍聖公為奉聖公,不領他職,給廟學田萬畝,賜國子監書,立學官以誨其子弟。進刑部侍郎,屬疾求去,以寶文閣待制知徐州,未拜而卒。



 鞠詠字詠之,開封人。父勵,尚書膳部員外郎、廣南轉運使。詠十歲而孤,好學自立。舉進士,試秘書省校書郎、知
 錢塘縣,改著作郎、知山陰縣。



 仁宗即位,以太常博士召為監察御史。錢惟演自亳州來朝,圖入相。詠言:「惟演憸險,嘗與丁謂為婚姻,緣此大用。後揣知謂奸狀已萌,懼牽連得禍,因此力攻謂。今若遂以為相,必大失天下望。」太后遣內侍持奏示之,惟演猶顧望不行。詠語諫官劉隨曰:「若相惟演,當取白麻廷毀之。」惟演聞,乃亟去。



 大安殿柱生芝草,召群臣就觀。詠言:「陛下新即位,河決未寒,霖雨害稼,宜思所以應災變。臣願陛下以援進忠良、退
 斥邪佞為國寶,以訓勸兵農、豐積倉廩為天瑞。草木之怪,何足尚哉!」



 時王欽若復相,詠嫉欽若阿倚,數睥睨其短,欽若心忌之。會詠兼左巡使,率府率崇俊入朝失儀,詠言崇俊少在邊,今老矣,此不足罪。欽若奏詠廢朝廷儀,出通判信州。又坐鞠陳絳獄失實,徙邵州。欽若卒,御史中丞王臻奏還詠殿中侍御史,為三司鹽鐵判官。曹利用貶死,利用嘗所薦擢者多領兵守邊,朝廷俗罷去之,詠請一切毋治。



 天聖六年夏,大星晝隕,有聲如雷,詠
 條五事上之。因言:「太子少保致仕晁迥,雖老而有器識,宜蒙訪對,其心有補。」又言:「三司使胡則,丁謂黨也,性貪巧,不可任利權。」河北、京師旱饑,奏請出太倉米十萬石振饑民。江、淮制置使鐘離瑾因奏計,多致東南物以賂權貴。詠請御史臺劾狀,帝面諭瑾亟還所部。以尚書禮部員外郎兼侍御史知雜事、權同判吏部流內銓,為三司鹽鐵副使。



 八年,特置天章閣待制,以詠及範諷為之。判登聞檢院。定國軍節度使張士遜入覲,冀得再用。詠
 奏曰:「曹利用擅威福,士遜與之共事,相親厚,援薦以至相位。陛下以東宮僚屬用之,臣願割舊恩,伸公義,趣使之藩。」士遜乃赴鎮。明年詠卒。嘗著《道釋雜言》數十篇,別構凈室以居,自號深寧子。



 劉隨,字仲豫,開封考城人。以進士及第,為永康軍判官。軍無城堞,每伐巨木為柵,壞輒以他木易之,頗用民力。隨因令環植楊柳數十萬株,使相連屬,以為限界,民遂得不擾。屬縣令受賕鬻獄,轉運使李士衡托令於隨,不
 從。士衡憤怒,乃奏隨苛刻,不堪從政,罷歸,不得調。初,西南夷市馬入官,苦吏誅索,隨為繩按之。既罷,夷人數百訴於轉運使曰:「吾父何在?」事聞,乃得調。



 後改大理寺丞,為詳斷官。李溥以贓敗,事連權貴,有司希旨不窮治,隨請再劾之,卒抵溥罪。晁迥薦通判益州,呂夷簡安撫川峽,又言其材,以太常博士改右正言。數月,坐嘗為開封府發解巡捕官,而不察舉人,私以策辭相授,降監濟州稅,稍徒通判晉州。



 還朝,遷右司諫,為三司戶部判官。隨
 在諫職數言事,嘗言:「今之所切,在於納諫,其餘守常安靖而已。」又奏:「頻年水旱,咎在執事大臣忿爭不和。請察王欽若等所爭,為辨曲直。」又因星變言:「國家本支蕃衍,而定王之外,封策未行。望擇賢者,用唐故事,增廣嗣王、郡王之封,以慰祖宗意。」時下詔蜀中,選優人補教坊,隨以為賤工不足辱詔書。又劾奏江、淮發運使鐘離瑾載奇花怪石數十艘,納禁中及賂權貴。累疏論丁謂奸邪,不宜還之內地;胡則,謂之黨,既以罪出陳州,不當復
 進職。王欽若既死,詔塑其像茅山,列於仙官。隨言:「欽若贓污無忌憚,考其行,豈神仙耶?宜察其妄。」又言:「李維以詞臣求換武職,非所以勵廉節。」前後所論甚眾。



 帝既益習天下事,而太后猶未歸政,隨請軍國常務,專稟帝旨,又諫太后不宜數幸外家,太后不悅。會隨請外,出知濟州,改起居郎。久之,遷尚書刑部員外郎,入兼侍御史知雜事。上言:「比年庶官僥幸請托,或對見之際,涕泗祈恩,或績效甚微,衒鬻要賞。亦有藩翰之臣,位尊職重,表章
 不遜,請求靡厭。按察之司,燕安顧望,以容奸為大體,以舉職為近名,以巧詐為賢,以恬退為拙。以至貪殘者瀆於貨財,老疾者不知止足。請行申儆之法。」朝廷為下詔戒中外。



 未幾,權同判吏部流內銓,以長定格從事,吏不得為奸。改三司鹽鐵副使。使契丹,以病足痺,辭不能拜。及還,為有司劾奏,奪一官,出知信州,徙宜州,再遷工部郎中、知應天府。召為戶部副使,改天章閣待制,不旬日卒。



 隨與孔道輔、曹修古同時為言事官,皆以清直聞。隨
 臨事明銳敢行,在蜀,人號為「水晶燈籠。」初,使契丹還,會貶,而官收所得馬十五乘。既卒,帝憐其家貧,賜錢六十萬。



 曹修古,字述之,建州建安人。進士起家,累遷秘書丞、同判饒州。宋綬薦其材,召還,以太常博士為監察御史。上四事,曰行法令、審故事、惜材力、辨忠邪,辭甚切至。又奏:「唐貞觀中,嘗下詔令致仕官班本品見任上,欲其知恥而勇退也。比有年餘八十,尚任班行,心力既衰,官事何
 補。請下有司,敕文武官年及七十,上書自言,特與遷官致仕,仍從貞觀舊制,即宿德勛賢,自如故事。」因著為令。



 修古嘗偕三院御史十二人晨朝,將至朝堂,黃門二人行馬不避,呵者止之,反為所詈。修古奏:「前史稱,御史臺尊則天子奠。故事,三院同行與知雜事同,今黃門侮慢若此,請付所司劾治。」帝聞,立命笞之。晏殊以笏擊人折齒。修古奏:「殊身任輔弼,百僚所法,而忿躁亡大臣體。古者,三公不按吏,先朝陳恕於中書榜人,實時罷黜。請正
 典刑,以允公議。」



 司天監主簿苗舜臣等嘗言,土宿留參,太白晝見,詔日官同考定。及奏,以謂土宿留參。順不相犯;太白晝見,日未過午。舜臣等坐妄言災變被罰。修古奏言:「日官所定,希旨悅上,未足為信。今罰舜臣等,其事甚小,然恐人人自此畏避,佞媚取容,以災為福,天變不告,所損至大。」禁中以翡翠羽為服玩,詔市於南越。修古以謂重傷物命,且真宗時嘗禁採狨毛,故事未遠。命罷之。時頗崇建塔廟,議營金閣,費不可勝計,修古極陳其
 不可。



 久之,出知歙州,徙南劍州,復為開封府判官。歷殿中侍御史,擢尚書刑部員外郎、知雜司事、權同判吏部流內銓。未逾月,會太后兄子劉從德死,錄其姻戚至於廝役幾八十人,龍圖閣直學士馬季良、集賢校理錢暖皆緣遺奏超授官秩,修古與楊偕、郭勸、段少連交章論列。太后怒,下其章中書。大臣請黜修古知衢州,餘以次貶。太后以為責輕,命皆削一官,以修古為工部員外郎、同判杭州,未行,改知興化軍。會赦復官,卒。



 修古立朝,慷
 慨有風節。當太后臨朝,權幸用事,人人顧望畏忌,而修古遇事輒言,無所回撓。既沒,人多惜之。家貧,不能歸葬,賓佐賻錢五十萬。委女泣白其母曰:「奈何以是累吾先人也。」卒拒不納。太后崩,帝思修古忠,特贈右諫議大夫,賜其家錢二十萬,錄其婿劉勛為試將作監主簿。修古無子,以兄子覲為後。



 覲知封州,儂智高亂,死之,見《忠義傳》。弟修睦,性廉介自立,與修古同時舉進士,有聲鄉里,累官尚書都官員外郎、知邵武軍。御史中丞杜衍薦以
 為侍御史。歲餘,改司封員外郎,出知壽州,徙泉州。坐失舉,奪一官罷去。後以知吉州,不行,上書請老,不聽,分司南京,未幾致仕,年五十一。章得像表其高,詔還所奪官,卒。



 曹氏自修古以直諒聞,其女子亦能不累於利,至覲,又能死其官,而修睦亦恬於仕進,不待老而歸,世以是賢之。



 郭勸,字仲褒,鄆州須城人。舉進士,授寧化軍判官,累遷太常博士、通判密州。特遷尚書屯田員外郎、梓州路轉
 運判官。以母老固辭,復為博士、通判萊州。州民霍亮為仇人誣罪死,吏受賕傅致之,勸為辨理得免。擢殿中侍御史。



 時宋綬出知應天府,杜衍在荊南,勸言:「綬有辭學,衍清直,不宜處外。」又言:「武勝軍節度使錢惟演遷延不赴陳州,覬望相位;弟惟濟任觀察使、定州總管,自請就遷留後;胡則以罪罷三司使,乃遷工部侍郎、集賢院學士。請趣惟演上道,罷惟濟兵權,追則除命。」又論劉從德遺奏恩濫,貶太常博士、監濰州稅。



 改祠部員外郎、知萊
 州。月餘,復為侍御史、判三司鹽鐵勾院。郭皇后廢,議選納陳氏,勸進諫曰:「正家以正天下,自後妃始。郭氏非有大故,不當廢。陳氏非世閥,不可以儷宸極。」疏入,後已廢,而陳氏議遂寢。



 遷兵部員外郎兼起居舍人、同知諫院。馬季良自貶所求致仕,朝廷從之。勸言:「致仕所以待賢者,豈負罪貶黜之人可得,請追還敕誥。」又言:「發運使劉承德獻輪扇浴器,大率以媚上也。請付外毀,以戒邪佞。」



 趙元昊襲父位,以勸為官告使,所遺百萬,悉拒不受。還,
 兼侍御史知雜事、權判流內銓,遷工部郎中、度支副使,拜天章閣待制、知延州。元昊將山遇率其族來歸,且言元昊將反。勸與兵馬鈐轄李渭議,自德明納貢四十年,有內附者未嘗留,乃奏卻之。是冬,元昊果反,遣其使稱偽官來。勸視其表函猶稱臣,因上奏曰:「元昊雖僭中國名號,然尚稱臣,可漸以禮屈之,願與大臣熟議。」遂落職知齊州,改淄州,數月,移磁州。元昊益侵邊,關陜擾攘,言者猶指勸不當絕山遇事,又降兵部員外郎。丁母憂,起
 復,知鳳翔府,尋復待制。



 召權戶部副使,以龍圖閣直學士知滑州,再遷兵部郎中,徙滄州,又徙成德軍。盜起甘陵,徙鄆州。既而知成德軍韓琦言,勸所遣將張忠、劉遵,平賊功皆第一,特詔獎諭。未幾,召為翰林侍讀學士,復判流內銓,改左諫議大夫、權御史中丞。遷給事中,辭不受,而請贈其祖萊陽令寧,遂以為尚書祠部員外郎。



 衛士有相惡者,陰置刃衣篋中,從勾當皇城司楊景宗入禁門,既而為閽者所得,景宗輒隱不以聞。勸請先治景
 宗罪,章再上,不聽,又廷爭累日,卒貶景宗。祀明堂,將加恩中外官,勸就齋次,帥群御史求對,不許,又極論之。是年,復為侍讀學士、同知通進銀臺司。



 勸性廉儉,居無長物。嘗謂諸子曰:「顏魯公云,『生得五品服章紱,任子為齋郎,足矣。』」及再為侍讀,曰:「吾起諸生,志不過郡守,今年七十,列侍從,可以歸矣。」遂用元日拜章,三上不得謝,賜銀使市田宅。後二年卒。



 子源明,治平中,為太常博士。會御史知雜事呂誨等奏彈中書議追崇濮安懿王典禮非
 是,被黜,以源明補監察御史裏行。源明乞免除命,請追誨等,遂聽免。後以職方員外郎知單州,卒。



 段少連,字希逸,開封人。其母嘗夢鳳集家庭,寤而生少連。及長,美姿表,倜儻有識度。舉服勤詞學,為試秘書省校書郎、知崇陽縣。崇陽劇邑,自張詠為令有治狀,其後惟少連能繼其風跡。權杭州觀察判官。預校《道經》,改秘書省著作佐郎,歷知蒙城、名山、金華三縣,以本省丞為審刑院詳議官。張士遜守江寧,闢通判府事,還為御史
 臺推直官,遷太常博士。論劉從德遺奏恩濫,降秘書丞、監漣水軍酒稅。復為博士、通判天雄軍。



 太后崩,召為殿中侍御史,與孔道輔等伏閣言郭皇后不當廢,少連坐贖。復上疏曰:「陛下親政以來,進用直臣,開闢言路,天下無不歡欣。一旦以諫官、御史伏閣,遽行黜責,中外皆以為非陛下意。蓋執政大臣,假天威以出道輔、仲淹,而斷來者之說也。竊睹戒諭:『自今有章,宜如故事密上,毋得群詣殿門請對。』且伏閣上疏,豈非故事,今遽絕之,則國
 家復有大事,誰敢旅進而言者。昔唐城王仲舒伏閣雪陸贄,崔元亮叩殿陛理宋申錫,前史以為美事。今陛下未忍廢黜皇后,而兩府列狀議降為妃,諫官、御史,安敢緘默。陛下深惟道輔等所言為阿黨乎?為忠亮乎?」疏入不報。



 又上疏曰:



 高明粹清,凝德無累者,天之道也。氛祲蔽翳,晦明偶差,乃陰陽之沴爾。像天德者,君之體也。治陰陽者,臣之職也。陛下秉一德、臨萬方,有生之類,莫不浸涵德澤。而氛祲蔽翳,偶差晦明,以累聖德者,由大
 臣懷錄而不諫,小臣畏罪而不言。臣獨何人,敢貢狂瞽。竊痛陛下履仁聖之具美,乏骨鯁之良輔,因成不忍之忿,又稽不遠之復。臣是以瀝肝膽,披情愫,為陛下廓清氛祲蔽翳之累。



 《易》曰:「夫夫婦婦而家道正,正家而天下定。」《詩》云:「刑於寡妻,以御於家邦。」若然,則君天下修化本者,莫不自內而刑外也。況聞入道降妃之議,出自臣下。且後妃有罪,黜出告宗廟,廢則為庶人,安有不示之於天下,不告之於祖宗,而陰行臣下之議乎?且皇后以小
 過降為妃,則臣下之婦有小過者,亦當降為妾矣。比抗章請對,不蒙賜召,豈非奸邪之臣,離間陛下耶?臣等赴中書,時執政之臣,謂後有妒忌之行,始議入道,終降為妃。兼云有上封者,慮後不利於聖躬,故築高垣,置在別館。臣等備言中外之議,以為未可。願速降明詔,復中宮位號,以安民心。翌日詔出,乃云「中宮有過,掖庭具知,特示涵容,未行遽黜,置之別館,俾自省修,供給之間,一切如故。」臣未寧黜置別館,為後為妃?詔書不言,安所取信。
 況皇后事陛下一紀有餘,而輔臣倉卒以降黜之議,惑於宸聽,搢紳循默,無敢為陛下言者。臣所謂氛祲蔽翳,以累聖德者,蓋臣職有曠爾。



 臣竊恐奸邪之人,引漢武幽陳皇后故事,以諂惑陛下。且漢武驕奢淫縱之主,固不足踵其行事。而為人臣者,思致君如堯、舜,豈致君如漢武哉!今皇后置於別館,必恐懼修省,陛下仁恕之德,施於天下,而獨不加於中宮乎?願詔復中宮位號,杜絕非間,待之如初。天地以正,陰陽以和,人神共歡,豈不美
 哉。陛下茍為邪臣所蔽,不加省察,臣恐高宗王後之枉,必見於他日,宮闈不正之亂,未測於將來,惟聖神慮焉。



 未幾,除開封府判官,改尚書刑部員外郎、直集賢院,為三司度支判官,出為兩浙轉運副使。舊使者所至郡縣,索簿書,不暇殫閱,往往委之吏胥,吏胥持以為貨。少連命郡縣上簿書悉緘識,遇事間指取一二自閱,摘其非是者按之,餘不及閱者,全緘識以還。由是吏不能為奸,而州縣簿書莫敢不治矣。部吏有過,召詰曰:「聞子所為
 若此,有之乎?有當告我,我容汝自新;茍以為無,吾不使善人被謗,即為汝辨明矣。」吏不敢欺,皆以實對。少連每得其情,諄諄戒飭使去,後有能自改過者。猶保任之。秀州獄死無罪人,時少連在杭,吏畏恐聚謀,偽為死者服罪款,未及綴,屬少連已拏舟入城,訊獄吏,具服請罪,以為神明。是時,鄭向守杭,無治才。訟者不服,往往自州出,徑趨少連;少連一言處決,莫不盡其理。



 徙使淮南,兼發運司事,加兵部員外郎。又徙陜西。附馬都尉柴宗慶知
 陜州,縱其下撓民,少連入境,劾奏之。入兼侍御史知雜事,逾月,為三司度支副使。河東地震,奉使安撫。還,擢工部郎中、天章閣待制、知廣州。時元昊反,範仲淹薦少連才堪將帥,遷龍圖閣直學士、知涇州,改渭州,命未至而卒。少連通敏有才,遇事無大小,決遣如流,不為權勢所屈。既卒,仁宗嘆惜之。



 論曰:古人有言:「山有猛獸,藜藿為之不採。」當天聖、明道間,天子富於春秋,母後稱制,而內外肅然,紀綱具舉,朝
 政亡大闕失,奸人不得以自肆者,繇言路得人故也。是時,孔道輔、鞠詠、劉隨、曹修古迭為諫官、御史,郭勸、段少連繼之,皆侃侃正色,遇事輒言,被斥逐,不更其守。及帝既親政,道輔、勸、少連復任言責,郭后之廢,引議慷慨,犯人主,責大臣,其氣益壯,遺風餘烈,天下至今稱之。《詩》所謂「邦之司直」,其庶幾歟!



\end{pinyinscope}