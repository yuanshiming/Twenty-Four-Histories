\article{列傳第一百四十}

\begin{pinyinscope}

 範如圭吳表臣王居正晏敦復黃龜年程瑀張闡洪擬趙逵



 範如圭,字伯達,建州建陽人。少從舅氏胡安國受《春秋》。
 登進士第,授左從事郎、武安軍節度推官。始至,帥將斬人,如圭白其誤,帥為已署不易也。如圭正色曰:「節下奈何重易一字而輕數人之命?」帥矍然從之。自是府中事無大小悉以咨焉。居數月,以憂去。闢江東安撫司書寫機宜文字。近臣交薦,召試秘書省正字,遷校書郎兼史館校勘。



 秦檜力建和議,金使來,無所於館,將虛秘書省以處之。如圭亟見宰相趙鼎曰:「秘府,謨訓所藏,可使仇敵居之乎?」鼎竦然為改館。既而金使至悖傲,議多不可
 從,中外憤鬱。如圭與同省十餘人合議,並疏爭之,既具草,駭遽引卻者眾。如圭獨以書責檜以曲學倍師、忘仇辱國之罪,且曰:「公不喪心病狂,奈何為此,必遺臭萬世矣!」檜怒。草奏與史官六人上之。



 金歸河南地,檜方自以為功。如圭輪對,言:「兩京之版圖既入,則九廟、八陵瞻望咫尺,今朝修之使未遣,何以慰神靈、萃民志乎?」帝泫然曰:「非卿不聞此言。」即日命宗室士人褭及張燾以行。檜以不先白己,益怒。



 如圭謁告去,奉柩歸葬故鄉,既窆,差主
 管臺州崇道觀。杜門十餘歲,起通判邵州,又通判荊南府。荊南舊戶口數十萬,寇亂後無復人跡,時蠲口錢以安集之,百未還一二也。議者希檜意,遽謂流庸浸復而增之,積逋二十餘萬緡,他負亦數十萬,版曹日下書責償甚急。召圭白帥,悉奏蠲之。



 檜死,被旨入對,言:「為治以知人為先,知人以清心寡欲為本。」語甚切。又論:「東南不舉子之俗,傷絕人理,請舉漢《胎養令》以全活之,抑亦勾踐生聚報吳之意也。」帝善其言。又奏:「今屯田之法,歲之
 所獲,官盡征之。而田卒賜衣廩食如故,使力穡者絕贏餘之望,惰農者無饑餓之憂,貪小利,失大計,謀近效,妨遠圖,故久無成功。宜籍荊、淮曠土,畫為丘井,仿古助法,別為科條,令政役法,則農利修而武備飭矣。」



 以直秘閣提舉江西常平茶監移利州路提點刑獄,以病請祠。時宗藩並建,儲位未定,道路竊有異言。如圭在遠外,獨深憂之,掇至和、嘉祐間名臣奏章凡三十六篇,合為一書,囊封以獻,請深考群言,仰師成憲,斷以至公勿疑。或以
 越職危之,如圭曰:「以此獲罪,奚憾!」帝感悟,謂輔臣曰:「如圭可謂忠矣。」即日下詔以普安郡王為皇子,進封建王。復起如圭知泉州。



 南外宗官寄治郡中,挾勢為暴,占役禁兵以百數,如圭以法義正之,宗官大沮恨,密為浸潤以去如圭,遂以中旨罷,領祠如故。僦舍邵武以居,士大夫高之,學者多從之質疑。卒年五十九。



 如圭忠孝誠實,得之於天。其學根於經術,不為無用之文。所草具屯田之目數千言,未及上,張浚視師日,奏下其家取之,浚罷,
 亦不果行。有集十卷,皆書疏議論之語,藏於家。子念祖、念德、念茲。



 吳表臣,字正仲,永嘉人。登大觀三年進士第,擢通州司理。陳瓘謫居郡中,一見而器之。盛章者,朱勉黨也,嘗市婢,有武臣強取之,章誣以罪,系獄。表臣方鞫之,郡將曰:「知有盛待制乎?」表臣佯若不知者,卒直其事。累官監察御史,遷右正言。



 高宗詔臺諫條陳大利害,表臣請措置上流以張形勢,安輯淮甸以立藩蔽,擇民兵以守險阻,
 集海舶以備不虞。其策多見用。帝方鄉儒術,表臣乞選講官以裨聖德,且於古今成敗、民物情偽,邊防利害,詳熟講究。由是詔開經筵。邇臣有請用蔡京、王黼之黨者,侍御史沈與求乞明指其人,顯行黜責,執政不悅,奪其言職。表臣爭曰:「臺諫為天子耳目,所以防壅蔽、杜奸邪,若咎其切直而黜之,後誰敢言,非國家福也。請還與求以開言路。」



 時防秋,議選守邊者,患乏才。表臣曰:「唐蕭復言於德宗,陳少游任兼將相,首敗臣節,韋皋幕府下僚,
 獨建忠義,以皋代少游鎮淮南。善惡著明,則天下知逆順之理,初不以皋名賤官卑為疑。今取忠義不屈有已試之驗者,不次而用,豈特可以勸,捍禦方略,亦堪倚仗。」於是陳敏等十數人浸以錄用。久之,以病請補外,以直秘閣知信州。



 紹興元年,召為司勛郎中,遷左司。詔百官陳裕國強兵之策,表臣條十事以獻,曰:蠲稅役以墾閑田,汰懦卒以省兵費,罷添差以澄冗員,停度牒以蕃生齒,拘佃租以防乾沒,委計臣以制邦用,獎有功以厲將
 帥,招弓手以存舊籍,嚴和買以絕弊幸,簡法令以息瘡痍。



 宰相擬表臣為檢正,帝曰:「朕將自用之。」遂除左司諫。給事中胡安國以論事不合罷,表臣上疏留之。前宰相朱勝非同都督江、淮軍馬,表臣力言都督不可罷。除侍讀,又累疏爭之,不聽,遂罷。表臣送吏部。授臺州黃巖丞,尋除提點浙西刑獄,召為秘書少監,同修《哲宗實錄》。



 帝如建康,詔表臣兼留司參議官,除中書舍人、給事中、兵部侍郎。建、崇二國公就外傅,兼翊善。帝曰:「二國公誦習
 甚進,卿力也。」徙禮部侍郎,遷吏部尚書兼翰林學士。時秦檜欲使使金議地界,指政事堂曰:「歸來可坐此。」表臣不答。又以議大禮忤意,罷去。



 俄起知婺州。會大水,發常平米振貸之,然後以聞,郡人德之。課最,除敷文閣待制。三歲請祠,進直學士,提舉江州太平興國宮。家居數年,卒,年六十七。



 表臣晚號湛然居士,自奉無異布衣時,鄉論推其清約。



 王居正,字剛中,揚州人。少嗜學,工文辭。入太學,時習《新
 經》、《字說》者,主司輒置高選,居正語人曰:「窮達自有時,心之是非,可改邪?」流落十餘年,司業黃齊得其文,曰:「王佐才也。」及同知貢舉,欲擢為首,以風多士,他考官持之,置次選。調饒州安仁丞、荊州教授,皆不赴。大名、鎮江兩帥交闢教授府學,亦不就。



 範宗尹薦於朝,召至,謂宗尹曰:「時危如此,公不極所學,拔元元塗炭中,尚誰待?居正避寇陽羨山間,勉出見公,一道此意爾。」宗尹愧謝。入對,奏:「昔人有云:『君以為難,易將至矣。』今日之事,朝廷皆曰難,
 則當有易為之理。然國勢日弱,敵氣日驕,何邪?蓋昔人於難者勉強為之,今以為難,不復有所為,以俟天意自回,強敵自斃也。宣和末,以為難者十五六,至靖康與宣和孰難?靖康末,以為難者十八九,至建炎與靖康孰難?由此而言,今日雖難於前日,安知他日不難於今日?蓋宣和以為難,故有靖康之禍;靖康以為難,故有今日之憂。今而亦云,臣有所不忍聞。」高宗嘉之,諭宗尹曰:「如王居正人才,歲月間得一人亦幸矣。」



 除太常博士,遷禮部
 員外郎。建議合祭天地於明堂,請奉太祖、太宗配,宗尹是之,議遂定,天地復合祭。侍御史沈與求劾宗尹,因及居正,宗尹去,居正乞補外,不許。撫州守高衛言甘露降於州之祥符觀,為圖以獻。居正論今日恐非天降祥瑞之時,卻其圖。



 試太常少卿兼修政局參議,遷起居郎。帝方鄉規諫,居正次前世聽納事為《集諫》十五卷,以廣帝意。詔以時務訪群臣,居正獻疏數千言,論省費尤切,曰:「宋興百七十三年矣,所行多彌文之事。今陛下所至曰
 行在,於一日二日少駐蹕之頃,欲盡為向者百七十三年之事,非所謂知變也。夫不知隨時以省事,而乃隨事以省費,故今日例有減半之說,究其實未始不重費。願詔大臣計百事之實而論定之,茍非禦寇備敵,任賢使能,振恤百姓,一切姑置,則費省而國裕。」



 居正素與秦檜善,檜為執政,與居正論天下事甚銳,既相,所言皆不酬。居正疾其詭,見帝言曰:「秦檜嘗語臣:『中國人惟當著衣啖飯,共圖中興。』臣心服其言。又自謂『使檜為相數月,必
 聳動天下。』今為相施設止是,願陛下以臣所聞問檜。」檜銜之,出居正知婺州。州貢羅,舊制歲萬匹,崇寧後增五倍,建炎中減為二萬。至是,主計者請復崇寧之數,居正力言於朝,戶部督趣愈峻,居正置檄不行,語其屬曰:「吾願身坐,不以累諸君。」呼吏為文書付之曰:「即有譴,以此自解。」復手疏「五不可」以聞。詔如建炎中數。漕司市御炭,須胡桃文、鵓鴿色者,居正曰:「民以炭自業者,率居山谷,安知所謂胡桃文、鵓鴿色耶?」入朝以聞,詔止之。



 召為太
 常少卿,遷起居舍人兼權中書舍人、史館修撰。帝欲遷趙令TM大中大夫,居正奏:「官非侍從不可轉,此祖宗法,若令TM以庶官得遷,則宗室為承宣者,不旋踵求為節度,何以卻之?」遂寢其命。上書人陳東、歐陽澈已贈官,居正乞重貶黃潛善、汪伯彥,以彰二子殺身成仁之美。大將張俊遣卒至彭澤,卒故縣吏,怙俊勢侵辱令,令郭彥恭械之,俊訴於朝,帝為罷彥恭。居正言:「彥恭不畏強御,無可罪。」俊又乞免徭役,居正言:「兵興以來,士大夫及勛
 戚家賦役與編戶均,蓋欲貴賤上下,共濟國事,以寬民力,俊反不能體此乎?」和州請蠲進奉大禮絹,居正言:「大禮進奉,乃臣子享上之誠,初非朝廷取於百姓之物,若察民力無所從出,不能預降旨蠲之,至使州縣自陳,已為非是,乞速如所請。」除目有自中出者,居正奏:「近習請托,進擬不自朝廷,所系非輕。」因錄皇祐詔書以進。帝皆嘉納。



 兼權直學士院,又除兵部侍郎。入對,以所論王安石父子之言不合於道者,裒得四十二篇,名曰《辨學》,上
 之。又曰:「陛下惡安石之學,嘗於聖心灼見,其弊安在?」帝曰:「安石之學,雜以伯道,欲效商鞅富國強兵,今日之禍,人徒知蔡京、王黼之罪,而不知生於安石。」居正曰:「安石得罪萬世者不止此。」因陳安石釋經無父無君者。帝作色曰:「是豈不害名教邪?孟子所謂邪說,正謂是矣。」居正退,序帝語系於《辨學》首。



 出知饒州,尋改吉州。侍御史謝祖信劾居正兇暴詭詐,傾陷大臣,罷官,屏居括蒼三載。其弟駕部郎居修入對,帝曰:「卿兄今安在?行大用矣。」中
 書舍人劉大中侍帝,論制誥,帝曰:「王居正極得詞臣體。」侍御史蕭振論守令賢否,帝舉居正守婺免貢羅、御炭事,曰:「守臣愛百姓皆如此,朕復何憂。」



 起知溫州。是時檜專國,居正自知不為所容,以目疾請祠,杜門,言不及時事,客至談論經、史而已。檜終忌之,風中丞何鑄劾居正為趙鼎汲引,欺世盜名,奪職奉祠,凡十年。檜死,復故職。紹興二十一年卒,年六十五。



 居正儀觀豐偉,聲音洪暢。奉祿班兄弟宗族,無留者。郊祀恩以任其弟居厚,及卒,
 季子猶布衣。其學根據《六經》,楊時器之,出所著《三經義辨》示居正曰:「吾舉其端,子成吾志。」居正感厲,首尾十載為《書辨學》十三卷,《詩辨學》二十卷,《周禮辨學》五卷,《辨學外集》一卷。居正既進其書七卷,而楊時《三經義辨》亦列秘府,二書既行,天下遂不復言王氏學。



 晏敦復字景初,丞相殊之曾孫。少學於程頤,頤奇之。第進士,為御史臺檢法官。紹興初,大臣薦,召試館職,不就。特命祠部郎官,遷吏部,以守法忤呂頤浩,出知貴溪縣。
 會有為敦復直其事者,改通判臨江軍,召為吏部郎官、左司諫、權給事中,為中書門下省檢正諸房公事。



 淮西宣撫使劉光世請以淮東私田易淮西田,帝許之。敦復言:「光世帥一道,未聞為朝廷措置毫發,乃先易私畝。比者岳飛屬官以私事干朝廷,飛請加罪,中外稱美,謂有古賢將風。光世自處必不在飛下,乞以臣言示光世,且令經理淮南,收撫百姓,以為定都建康計,中興有期,何患私計之未便。」權吏部侍郎兼詳定一司敕令。



 渡江後,
 庶事草創,凡四選格法多所裁定。敦復素剛嚴,居吏部,請謁不行,銓綜平允,除給事中。冬至節,旨下禮部,取度牒四百充賜予。敦復奏:「兵興費廣,凡可助用度者尤當惜,矧兩宮在遠,陛下當此令節,欲奉一觴為萬歲壽不可得,有司乃欲舉平時例行慶賜乎?」遂寢。有卒失宣帖,得中旨給據,太醫吳球得旨免試,敦復奏:「一卒之微,乃至上瀆聖聰,醫官免試,皆壞成法。自崇寧、大觀以來,奸人欺罔,臨事取旨,謂之『暗嬴指揮』,紀綱敗壞,馴致危亂,正
 蹈前弊,不可長也。」汪伯彥子召嗣除江西監司,敦復論:「伯彥奸庸誤國,其子素無才望,難任澄清。」改知袁州。又奏:「召嗣既不可為監司,亦不可為守臣。」居右省兩月,論駁凡二十四事,議者憚之。復為吏部侍郎。



 彗星見,詔求直言。敦復奏:「昔康澄以『賢士藏匿,四民遷業,上下相徇,廉恥道消,毀譽亂真,直言不聞』為深可畏。臣嘗即其言考已然之事,多本於左右近習及奸邪以巧佞轉移人主之意。其惡直醜正,則能使賢士藏匿;其造為事端,則
 能使四民遷業;其委曲彌縫,則能使上下相徇;其假寵竊權,簧鼓流俗,則能使廉恥道消;其誣人功罪,則能使毀譽亂真;其壅蔽聰明,則能使直言不聞。臣願防微杜漸,以助應天之實。」又論:「比來百司不肯任責,瑣屑皆取決朝省,事有不當,上煩天聽者,例多取旨。由是宰執所治煩雜,不減有司,天子聽覽,每及細務,非所以為政。願詳其大,略其細。」



 八年,金遣使來要以難行之禮,詔侍從,臺諫條奏所宜。敦復言:「金兩遣使,直許講和,非畏我而
 然,安知其非誘我也。且謂之屈己,則一事既屈,必以他事來屈我。今所遣使以詔諭為名,儻欲陛下易服拜受,又欲分廷抗禮,還可從乎?茍從其一二,則此後可以號令我,小有違異,即成釁端,社稷存亡,皆在其掌握矣。」時秦檜方力贊屈己之說,外議群起,計雖定而未敢行。勾龍如淵說檜,宜擇人為臺官,使擊去異論,則事遂矣。於是如淵、施廷臣、莫將皆據要地,人皆駭愕。敦復同尚書張燾上疏言:「前日如淵以附會和議得中丞,今施廷臣
 又以此躋橫榻,眾論沸騰,方且切齒,莫將又以此擢右史。夫如淵、廷臣庸人,但知觀望,將則奸人也,陛下奈何與此輩斷國論乎?乞加斥逐,杜群枉門,力為自治自強之策。」既又與燾等同班入對,爭之。檜使所親諭敦復曰:「公能曲從,兩地旦夕可至。」敦復曰:「吾終不為身計誤國家,況吾姜桂之性,到老愈辣,請勿言。」檜卒不能屈。



 胡銓謫昭州,臨安遣人械送貶所。敦復往見守臣張澄曰:「銓論宰相,天下共知,祖宗時以言事被謫,為開封者必不
 如是。」澄愧謝,為追還。始檜拜相,制下,朝士相賀,敦復獨有憂色曰:「奸人相矣。」張致遠、魏矼聞之,皆以其言為過。至是竄銓,敦復謂人曰:「頃言秦之奸,諸君不以為然,今方專國便敢爾,他日何所不至耶?」



 權吏部尚書兼江、淮等路經制使。故事,侍從過宰相閣,既退,宰相必送數步。敦復見檜未嘗送,每曰:「人必自侮而後人侮之。」尋請外,以寶文閣直學士知衢州,提舉亳州明道宮。閑居數年卒,年七十一。



 敦復靜默如不能言,立朝論事無所避。帝
 嘗謂之曰:「卿鯁峭敢言,可謂無忝爾祖矣。」



 黃龜年,字德邵,福州永福人。登崇寧五年進士第,調洺州司理參軍,累官河北西路提舉學士。呂頤浩見而奇之,入為太常博士。



 靖康元年,除吏部員外郎,拜監察御史,尋除尚書左司員外郎、中書門下檢正房公事,充修政局檢討官。乞令檢正官察通進司,帝從其請。時頤浩再相,植黨傾秦檜,引朱勝非奉京祠兼侍讀,恐中書舍人胡安國持錄黃不下,特命龜年書行,議者譏其侵
 官。



 遷殿中侍御史。會邊報王倫來歸,龜年劾檜專主和議,沮止恢復,植黨專權,漸不可長。乃上書曰:「臣聞一言而盡事君之道曰忠,罪莫大於欺君;一言而盡輔政之道曰公,罪莫大於私己。臣人者背公而徇私,則刑賞僭濫。慮人主之照其奸,則合黨締交,相與比周,熒惑主聽。故附下罔上之黨盛,而威福之柄下移,禍有不可勝言者。伏見秦檜還自金國,陛下驟任,不一年而超至宰輔,乃不顧國家,盜威福在己,欲永塞言路。」書上,檜罷,並劾
 檜黨王□奐、王昺、王守道,皆罷之。檜乃授觀文殿大學士、提舉江州太平觀,官如故。龜年又奏:「比論檜徇私欺君,合正典刑,投諸裔土,以禦魑魅。今乃任便居住,雖陛下曲全大臣之禮,秦檜奸狀暴露,復寵以儒學最上職名,俾優游琳館,聽其自如。律斷群盜,必分首從,為之從者皆已伏誅,獨置渠魁可乎?」又曰:「臣聞恩莫隆於父子,義莫重於君臣。不義則後其君,不仁則遺其親。君親既然,則何忌憚而不為。檜厚貌深情,矯言偽行,進迫君臣之
 勢,陽為面從;退恃朋比之奸,陰謀沮格。上不畏陛下,中不畏大臣,下不畏天下之議,無忌憚如此。欺君私己,有一即可黜,況檜之欺與私顯著者為多乎?」章凡三上,遂褫檜職。復上章曰:「檜行詭而言譎,外縮而中邪,以巧詐取相位,奸回竊國柄,收召險佞,蟠結黨與。陛下以智臨而辨之早,以剛決而去之速,故端人正士,舉手相慶,蓋以公天下之同惡耳。臣願陛下發明詔,以檜潛慝隱惡暴白於天下,使知陛下數易相位真不得已也;又所以
 破為臣奸膽,庶朋比之風不復作矣。」除太常少卿,累遷起居舍人、中書舍人兼給事中。



 侍御史常同言龜年陰結大臣,致身要地,又交結諸將,趣操不正,罷歸。司諫詹大方希檜意劾龜年附麗匪人,搢紳不齒,落職,本貫居住。卒,六十三。



 龜年微時,永福簿李朝旌奇之,許妻以女。龜年既登第,而朝旌已死,家貧甚。或勸龜年別娶,龜年正色曰:「吾許以諾,死而負之,何以自立。」遂娶之。任子恩,先奏其弟之子,人皆義之。子衡,仕至湖南提舉。



 程瑀,字伯宇,饒州浮梁人。其姑臧氏婦,養瑀為子,姑沒,始復本姓。少有聲太學,試為第一,累官至校書郎。為臧氏父母服,服闋,除兵部員外郎。適高麗使回,充送伴使。先是,使者往返江、浙間,調挽舟夫甚擾,有詔禁止。提舉人舡王珣畫別敕,遇風逆水澀許調夫。瑀渡淮,見民丁挽舟如故,遂劾珣,珣反奏瑀違御筆。詔命淮南提舉潘良貴核實,良貴奏珣言非是。



 金人入侵,求可使者,瑀請往。未行,會欽宗即位,議割三鎮,命瑀往河東,秦檜往河
 中。瑀奏:「臣願奉使,不願割地。」不報。至中山,諸將已得密諭,城守不下。瑀與金使王汭俱至燕山。還,除左正言,即言股肱大臣莫肯以身任天下事,且論:「欲慕祖宗而遹追無術,欲斥奄宦而寵任益堅,欲鋤奸惡而薄示典刑,欲汰濫繆而茍容僥幸,兼聽而不能行其言,委任而不能責其效,茍且之習復成,黨與之私浸廣,最時病之大者。」帝曰:「朕非不知此,慮有未盡,決意行之有失耳。」瑀曰:「事固當熟慮,然優柔不斷,實隳事功。」帝問:「李綱宣撫兩
 路,外議謂何?」瑀曰:「僉論固以為宜。然綱前與大臣議論不合,須賴聖明照察其心,任之無疑可也。」



 金酋斡離不、粘罕爭功,故斡離不欲和,粘罕欲戰,朝廷遣人繼蠟書約余睹,皆為粘罕所得。瑀因言:「金兵圍我重鎮,數月不能解,豈能出塞共謀人之國。莫若遣使議和,然謹飭邊備,徐觀其變。」使未行。瑀復言:「徐處仁庸俗,吳敏昏懦,唐恪傾險,政事所以不振。請盡黜免,別選英賢,共圖大計。」帝嘉納之。



 時御史李光言星變,帝疑以問瑀,對言:「陛下
 毋問有無,第正事修德,則變異可消。」瑀嘗論蔡京罪,帝因言吳敏庇京,又疑光黨京,謂瑀曰:「須卿作文字來。」瑀辭。改屯田郎官,謫添監漳州監稅。



 高宗即位,召為司封員外郎,遷光祿少卿、國子司業。請祠,主管亳州明道宮。尋召赴行在,疏十事以獻。除直秘閣、提點江東刑獄,召為太常少卿,遷給事中兼侍講。



 建修政局,其目曰省費裕國、強兵息民。瑀條上十四事,皆切時務。時三衙單弱,五軍多出於盜,瑀言:「李捧、崔增輩各將其徒,張俊、王□
 燮本無兵機,今呂頤浩出征,即捧、增輩便可使隸戎行。」帝因言:「頤浩熟於軍事,在外總諸將,檜在朝廷,庶幾內外相應,然檜誠實,但太執耳。」瑀曰:「如求機警能順旨者,極不難得,但不誠實,則終不可倚。」帝然之。



 權邦彥除簽書樞密院,瑀言邦彥五罪,疏三上,不報。求罷,除兵部侍郎,不拜,以敷文閣待制知信州。待御史江公躋、左司諫方公孟卿言瑀不可去,復以為給事中。久之,復命知信州。胡安國、劉一止言:「瑀忠信可以備獻納,正直可以司風憲,
 不宜去。」遂復留。頤浩薦席益,既得旨,以御批示後省官。瑀曰:「益為人公豈不知,何必用?」頤浩曰:「給事不見御批耶?」瑀曰:「已見矣。公不能執奏,乃先示瑀輩,欲使不敢論駁耶?然益之來,非公福也。」頤浩赧然,即劾益。未幾,以言者罷,提舉亳州明道宮,尋復徽猷閣待制、知撫州,無何,提舉江州太平興國宮。



 居父母喪,服除,知嚴州,徙宣州,復奉祠。俄召赴行在,除兵部侍郎兼侍讀。因論:「鄧禹嘗言『興衰在德厚薄,初不論大小』。光武不數年定大業,禹
 言如合符契。今英俊滿朝,豈無為陛下畫至計者,願厲志而已。」尋遷翊善。論:「金人入侵,未嘗一大衄,有輕我心,豈可保其不背盟。宜省費抑末,常賦外一毫不取於民,民日益厚,兵日益強,使金人不敢窺為長計。」帝曰:「且作十年。」瑀再拜曰:「十年之說,願陛下早夜毋忘。」除兵部尚書。



 檜既主和,瑀議論不專以和為是,檜忌之,改龍圖閣學士、知信州。會大水,檜見瑀奏牘,謂同列曰:「堯之洪水,不至如是。」瑀遂稱疾,提舉江州太平興國宮。坐通書李
 光,降朝議大夫,卒,年六十六。



 瑀在朝無詭隨,嘗為《論語說》,至「弋不射宿」,言孔子不欲陰中人。至「周公謂魯公」,則曰可為流涕。洪興祖序述其意,檜以為譏己,逐興祖。魏安行鋟版京西漕司,亦奪安行官,籍其家,毀版。檜死,瑀子孫乃免錮云。有奏議六卷。



 張闡字大猷,永嘉人。幼力學,博涉經史,善屬文。將命名,夢神人大書「闡」字曰:「以是名爾。」父異之,力勉其為學。未冠,由舍選貢京師。



 登宣和六年進士第,調嚴州兵曹
 掾兼治右獄。時方臘作亂,闡倡守禦計。有義士請身督戰,既戰,稍卻,州將怒,付闡治,將殺之,闡力爭曰:「是士以義請戰,官軍卻,勢不得獨前,非首奔者,殺之何罪?」州將意解,士得免。



 李回帥江西,席益帥湖南,皆闢置幕下。群盜據洞庭,官軍多西北人,不閑水戰。闡建策造戰艦,以大艦為營,小艦出戰,乘水涸直搗賊巢,賊勢以衰。諸司交薦,改秩,吏部以微文沮之,闡弗辯,求岳祠歸。歷鄂、臺二州教授。



 紹興十年,詔侍從各舉所知,給事中林待聘以
 闡聞,召對。時金人議和,歸關中地。闡首言:「關中必爭之地,古號天府,願固守以蔽巴蜀,圖中原。」次言監司、郡守薦舉之弊。又乞嚴禁遏糴,以濟江、浙水患。召試館職,除秘書省正字,遷校書郎兼吳、益王府教授。時諸將恃功邀爵賞,有過則姑息,又兵布於外,禁衛單寡,闡上疏極論之。後稍進退諸將必當其實,且召諸道兵以益禁旅,皆如闡言。



 十三年,遷秘書郎兼國史院檢討官。秦檜每薦臺諫,必先諭以己意,嘗謂闡曰:「秘書久次,欲以臺中
 相處何如?」闡謝曰:「丞相見知,得老死秘書幸矣!」檜默然,竟罷,主管臺州崇道觀,歷泉、衢二州通判。



 二十五年冬,帝躬攬萬機,起闡提舉兩浙路市舶,入為御史臺檢法官,升吏部員外郎。孝宗在王邸,帝妙選宮僚,謂「莊重老成無逾闡者」,改命祠部兼建王府贊讀。



 三十一年春,大雨,無麥苗,荊、浙盜起,詔侍從、臺諫條陳弭災、御盜之術。闡上疏曰:「和議以來,歲有聘幣,民不堪命,臣願陛下毋以金人困中國可乎?歸正人時有遣還之命,怨聲聞道
 路,臣願陛下毋使金人得以甘心可乎?州縣吏職卑地遠,漁奪之禍被於編籍,臣願陛下嚴臟吏之誅可乎?蠲租之令,已赦復徵,寬大之澤例為虛文,臣願陛下申詔令之禁可乎?是數者能次第行之,則足以動天地,召和氣,災異、盜賊不足慮也。」又言:「金主亮將入侵,宜守要害,防海道,三邊不可無良將,督視不可無大帥。」疏奏,帝嘉納,面諭曰:「卿所言深中時病,但遣人北歸,已載約書,朕不忍渝也。」遷將作監,進宗正少卿。



 三十二年,孝宗即位,
 闡權工部侍郎兼侍講,入謝,言:「諸將以敗為捷,冒受爵秩,州廂禁軍因覃霈鼓噪,希厚賞,不可不正其罪。」時悉為施行。


金主亮死,葛王褒復求和,再議遣使。闡言:「宜嚴遣使之命,正敵國之禮,彼或不從,則有戰爾。如是,則中國之威可以復振。」帝曰:「使者報聘,故事也,舊約不從,朕志定矣。」是冬,給札侍從、臺諫條具時務,闡上十事皆
 \gezhu{
  髟方}
 切。當時應詔數十人,惟闡與國子司業王十朋指陳時事,斥權幸,無所回隱。明日,召兩人對內殿,帝大加稱賞,
 賜酒及御書。時進太上皇帝、太上皇后冊寶,工部例進官,闡辭。或曰:「公轉一階,則澤可以及子孫,奈保辭?」闡笑曰:「寶冊非吾功也,吾能為子孫冒無功賞乎?」



 隆興元年,真拜工部侍郎。闡奏:「臣去冬乞守御兩淮,陛下謂春首行之,夏秋當畢,今其時矣。」帝曰:「江、淮事盡付張浚,朕倚浚為長城。」會督府請受蕭琦降,詔問闡,闡請受其降。俄報王師收復靈壁縣,闡慮大將李顯忠、邵宏淵深入無援,奏請益兵殿後。已而王師果失利,眾論歸罪於戰。闡
 曰:「陛下出師受降是也。諸將違節度且無援而敗,當矯前失,安可遽沮銳氣。」帝壯其言,益出御前器甲付諸軍,手詔勞浚,軍聲復振。



 時數易臺諫,闡力言之,請增廣諫員。帝曰:「臺諫好名,如某人但欲得直聲而去。」闡曰:「唐德宗疑姜公輔為賣直,陸贄切諫,願陛下深以為鑒。」帝再三嘉獎。



 金人求和,帝與闡議,闡曰:「彼欲和,畏我耶?愛我耶?直款我耳。」力陳六害不可許。帝曰:「朕意亦然,姑隨宜應之。」帝記「賣直」之語,謂:「胡銓亦及此。朕非拒諫者,辨是
 非耳。」闡曰:「聖度當如天,奈何與臣下爭名。」帝曰:「卿言是也。」頃之,除工部尚書兼侍讀。



 金副元帥紇石烈志寧以書諭通好,所請三事,國書、歲幣之議已定,惟割唐、鄧、海、泗未決,將遣王之望、龍大淵通問,而眾言紛紛不已。闡謂:「不與四州乃可通和,議論先定乃可遣使,今彼為客,我為主,我以仁義撫天下,彼以殘酷虐吾民,觀金勢已衰,何必先示以弱。」朝論韙之。



 帝用真宗故事,命經筵官二員遞宿學士院,以備顧問,闡入對尤數。屢引疾乞骸
 骨,帝不忍其去。二年,闡請益力,乃除顯謨直學士、提舉太平興國宮。陛辭,帝問所欲言,闡奏:「許和則忘祖宗之仇,棄四州則失中原之心,遣歸正人則傷忠義之氣。惟陛下毋忘老臣平昔之言。」其指時事尤諄切,帝眷益篤。諭以秋涼復召,加賜金犀帶,特許佩魚。居家逾月卒,年七十四。特贈端明殿學士。



 朱熹嘗言:「秦檜挾敵要君,力主和議,群言勃勃不平。檜既摧折忠臣義士之氣,遂使士大夫懷安成習。至癸未和議,則知其非者鮮矣。朝論
 間有建白,率雜言利害,其言金人世仇不可和者,惟胡右史銓、張尚書闡耳。」子叔椿。



 洪擬,字成季,一字逸叟,鎮江丹陽人。本弘姓,其先有名璆者,嘗為中書令,避南唐諱,改今姓。後復避宣祖廟諱。遂因之。



 擬登進士甲科。崇寧中為國子博士,出提舉利州路學事,尋改福建路。坐譴,通判鄆州,復提舉京西北路學事,歷湖南、河北東路。宣和中,為監察御史,遷殿中,進侍御史。時王黼、蔡京更用事,擬中立無所附會。殿中
 侍御史許景衡罷,擬亦坐送吏部,知桂陽軍,改海州。時山東盜起,屢攻城,擬率兵民堅守。



 建炎間,居母憂,以秘書少監召,不起。終喪,為起居郎、中書舍人,言:「兵興累年,饋餉悉出於民,無屋而責屋稅,無丁而責丁稅,不時之須,無名之斂,殆無虛日,所以去而為盜。今關中之盜不可急,宜求所以弭之,江西之盜不可緩,宜求所以滅之。夫豐財者政事之本,而節用者又豐財之本也。」高宗如越,執政議移蹕饒、信間,擬上疏力爭,謂「舍四通五達而
 趨偏方下邑,不足以示形勢、固守御。」



 遷給事中、吏部尚書,言者以擬未嘗歷州縣,以龍圖閣待制知溫州。宣撫使孟庾總師討閩寇,過郡,擬趣使赴援。庾怒,命擬犒師。擬借封椿錢用之,已乃自劾。賊平,加秩一等,召為禮部尚書,遷吏部。



 渡江後,法無見籍,吏隨事立文,號為「省記」,出入自如。至是修《七司敕令》,命擬總之,以舊法及續降指揮詳定成書,上之。



 金人再攻淮,詔日輪侍從赴都堂,給札問以攻守之策。擬言:「國勢強則戰,將士勇則戰,財
 用足則戰,我為主、彼為客則戰。陛下移蹕東南,前年幸會稽,今年幸臨安,興王之居,未有定議非如高祖在關中、光武在河內也。以國勢論之,可言守,未可言戰。」擬謂時相姑議戰以示武,實不能戰也。



 紹興三年,以天旱地震詔群臣言事,擬奏曰:「法行公,則人樂而氣和;行之偏,則人怨而氣乖。試以小事論之:比者監司、守臣獻羨餘則黜之,宣撫司獻則受之,是行法止及疏遠也。有自庶僚為侍從者,臥家視職,未嘗入謝,遂得美職而去,若鼓
 院官移疾廢朝謁,則斥罷之,是行法止及冗賤也。榷酤立法甚嚴,犯者籍家財充賞,大官勢臣連營列障,公行酤賣則不敢問,是行法止及孤弱也。小事如此,推而極之,則怨多而和氣傷矣。」尋以言者罷為徽猷閣直學士、提舉江州太平觀。始,擬兄子駕部郎官興祖與擬上封事侵在位者,故父子俱罷。起知溫州,提舉亳州明道宮。卒,年七十五,謚文憲。



 初,擬自海州還居鎮江。趙萬叛兵逼郡,守臣趙子崧戰敗,遁去。擬挾母出避,遇賊至,欲兵
 之,擬曰:「死無所避,願勿驚老母。」賊舍之。他賊又至,臨以刃,擬指其母曰:「此吾母也,幸勿怖之。」賊又舍去。有《凈智先生集》及《注杜甫詩》二十卷。



 趙逵,字莊叔,其先秦人,八世祖處榮徙蜀,家於資州。逵讀書數行俱下,尤好聚古書,考歷代興衰治亂之跡,與當代名人鉅公出處大節,根窮底究,尚友其人。紹興二十年,類省奏名,明年對策,論君臣父子之情甚切,擢第一。時秦檜意有所屬,而逵對獨當帝意,檜不悅。即罷知
 舉王□嚴,授逵左承事郎、簽書劍南東川。帝嘗問檜,趙逵安在?檜以實對。久之,帝又問,除校書郎。逵單車赴闕,征稅者希檜意,搜行橐皆書籍,才數金而已。既就職,未嘗私謁,檜意愈恨。



 逵賡禦制《芝草詩》,有「皇心未敢宴安圖」之句,檜見之怒曰:「逵猶以為未太平耶?」又謂逵曰:「館中祿薄,能以家來乎?」逵曰:「親老不能涉險遠。」檜徐曰:「當以百金為助。」逵唯唯而已。又遣所親申前言,諷逵往謝,逵不答,檜滋怒,欲擠之,未及而死。



 帝臨哭檜還,即遷逵著
 作佐郎兼權禮部員外郎。帝如景靈宮,秘省起居惟逵一人。帝屢目逵,即日命引見上殿,帝迎謂曰:「卿知之乎?始終皆朕自擢。自卿登第後,為大臣沮格,久不見卿。秦檜日薦士,未嘗一語及卿,以此知卿不附權貴,真天子門生也。」詔充普安郡王府教授。逵奏:「言路久不通,乞廣賜開納,勿以微賤為間,庶幾養成敢言之氣。」帝嘉納之。普安府勸講至戾太子事,王曰:「於斯時也,斬江充自歸於武帝,何如?」逵曰:「此非臣子所能。」王意蓋有所在也。



 二
 十六年,遷著作郎,尋除起居郎。入謝,帝又曰:「秦檜炎炎,不附者惟卿一人。」逵曰:「臣不能效古人抗折權奸,但不與之同爾,然所以事宰相禮亦不敢闕。」又曰:「受陛下爵祿而奔走權門,臣不惟不敢,亦且不忍。」明年同知貢舉,盡公考閱,以革舊弊,遂得王十朋、閻安中。



 始,逵未出貢闈,蔣璨除戶部侍郎,給事中辛次膺以璨交結希進,還之。帝怒,罷次膺,付逵書讀,逵不可,璨以此出知蘇州,次膺仍得次對,逵兼給事中。未幾,除中書舍人,登第六年
 而當外制,南渡後所未有也。帝語王綸曰:「趙逵純正可用,朕於蜀士未見其比。朕所以甫二歲令至此,報其不附權貴也。」



 先是,逵嘗薦杜莘老、唐文若、孫道夫皆蜀名士,至是奉詔舉士,又以馮方、劉儀鳳、李石、郯次雲應詔,宰執以聞。帝曰:「蜀人道遠,其間文學行義有用者,不因論薦無由得知。前此蜀中宦游者多隔絕,不得一至朝廷,甚可惜也。」自檜顓權,深抑蜀士,故帝語及之。



 逵以疾求外,帝命國醫王繼先視疾,不可為矣。卒年四十一。帝
 為之抆淚嘆息。逵嘗自謂:「司馬溫公不近非色,不取非財,吾雖不肖,庶幾慕之。」



 方檜權盛時,忤檜者固非止逵一人,而帝亟稱逵不附麗,又謂逵文章似蘇軾,故稱為「小東坡」,未及用而逵死,惜其論建不傳於世。有《棲雲集》三十卷。



 論曰:如圭師於安國,居正師於楊時,敦復師於程頤,表臣交於陳瓘,其師友淵源有自來矣。故其議論讜直,剛嚴鯁峭,不惑異說,不畏強御,大略相似。若夫居正辨王
 氏《三經》之繆,龜年首劾秦檜主和之非,程瑀力排蔡京之黨,尤為有功於名教。張闡論事無避,洪擬樸實端亮,趙逵純正善文,皆一時之良,為檜所忌而不撓者。語曰:「歲寒然後知松柏之後凋。」信哉!



\end{pinyinscope}