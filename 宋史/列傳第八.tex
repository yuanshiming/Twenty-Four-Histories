\article{列傳第八}

\begin{pinyinscope}

 範質子旻兄子杲王溥父祚魏仁浦子咸人孫昭亮



 範質字文文素,大名宗城人。父守遇,鄭州防禦判官。質生之夕,母夢神人授以五色筆。九歲能屬文,十三治《尚書》,
 教授生徒。



 後唐長興四年舉進士,為忠武軍節度推官,遷封丘令。晉天福中,以文章幹宰相桑維翰,深器之,即奏為監察御史。及維翰出鎮相州,歷泰寧、晉昌二節度,皆請質為從事。維翰再相,質遷主客員外郎、直史館。歲餘,召入為林學士,加比部郎中、知制誥。契丹侵邊,少帝命漢祖等十五將出征。是夜,質入直,少帝令召諸學士分草制,質曰:「宮城已閉,恐洩機事。」獨具草以進,辭理優贍,當時稱之。漢初,加中書舍人、戶部侍郎。周祖征叛,
 每朝廷遣使繼詔處分軍事,皆合機宜。周祖問誰為此辭,使者以質對。嘆曰:「宰相器也。」



 周祖自鄴起兵向闕,京城擾亂,質匿民間,物色得之,喜甚,時大雪,解袍衣之。且令草太后誥及議迎湘陰會儀注,質蒼黃論撰,稱旨。乃白太后,以質為兵部侍郎、樞密副使。周廣順初,加拜中書侍郎、平章事、集賢殿大學士。翌日,兼參知樞密院事。郊祀畢,進位左僕射兼門下侍郎、平章事、監修國史。從征高平還,加司徒、弘文館大學士。顯德四年夏,從征壽
 州還,加爵邑。質建議以律條繁冗,輕重無據,吏得因緣為奸。世宗特命詳定,是為《刑統》。六年夏,世宗北征,質病留京師,賜錢百萬,俾市醫藥。及平關南,至瀛州,質見於路左。師還,以樞密使魏仁浦為相,命質與王溥並參知樞密院事。世宗不豫,入受顧命。恭帝嗣位,加開府儀同三司,封蕭國公。



 及太祖北征,為六師推載,自陳橋還府署。時質方就食閣中,太祖入,率王溥、魏仁浦就府謁見。太祖對之嗚咽流涕,具言擁逼之狀。質等未及對,軍校
 羅彥環舉刃擬質曰:「我輩無主,今日須得天子。」太祖叱彥環不退,質不知所措,乃與浦等降階受命。



 宋初,加兼侍中,罷參知樞密。俄被疾,太祖征澤、潞,幸其第,賜黃金器二百兩、銀器千兩、絹二千匹、錢二百萬。太祖初即位,庶事謙抑,至於藩戚尚水崇建幕府賓佐未列於位。質因上奏曰:「自古帝王開基創業,封建子弟,樹立盤維,宗戚既隆,社稷永固。伏見皇弟泰寧軍節度使光義,自居戎職,特負將材,及領藩維,尤積時望;嘉州防禦使光美,
 雄俊老成,修身樂善,嘉譽日聞。乞並行封冊,申錫命書。皇子皇女雖在襁褓者,亦乞下有司許行恩制,此臣炎願也。臣又聞為宰相者,當舉賢能,以輔佐天子。竊以端日用殿學士呂餘慶、樞密副使趙普精通治道,經事霸府,歷歲滋深,睹其公忠,誠堪毗倚。乞授以臺司,俾申才用。」帝嘉納之。



 先是,宰相見天子議大政事,必命坐面議之,從容賜茶而退,唐及五代猶遵此制。及質等憚帝英睿,每事輒具札子進呈,具言曰:「如此庶盡稟承之方,免妄
 庸之失。」帝從之。由是奏御浸多,始廢坐論之禮。



 乾德初,帝將有事圜丘,以質為大禮使。質與鹵簿使張昭、儀仗使劉溫叟討論舊典,定《南郊行禮圖》上之。帝尤嘉獎。由是禮文始備,質自為序。禮畢,進封魯國公,質奉表固辭,不允。二年正月,罷為太子太傅。九月,卒,年五十四。將終,戒其子旻勿請謚,勿刻墓碑。太祖聞之,為悲惋罷朝。贈中書令,賵絹五百匹、粟麥各百石。



 質力學強記,性明悟。舉進士時,和凝以翰林學士典貢部,鑒質所試文字,重
 之,自以登第名在十三,亦以其數處之。貢闈中謂之「傳衣缽」。其後質登相位,為太子太傅,封魯國公,皆與凝同雲。初,質既登朝,猶手不釋卷,人或勞之,質曰:「有善相者,謂我異日位宰輔。誠如其言,不學何術以處之。」後從世宗征淮南,詔令多出其手,吳中文士莫不驚伏。質每下制敕,未嘗破律,命刺史縣令,必以戶口版籍為急。朝廷遣使視民田,按獄訟,皆延見,為述天子憂勤之意,然後遣之。



 世宗初征淮南,駐壽、濠,銳意攻取,且議行幸揚州。
 質以師老,與王溥泣諫乃止。及再駕揚州,因事怒竇儀,罪在不測。質入謁請見,世宗意其救儀,起避之。質趨前曰:「儀近臣也,過小不當誅。」因免冠叩頭泣下,曰:「臣備位宰相,豈可使人主暴怒,致近臣於死地耶?願寬儀罪。」世宗意遂解,復坐,即遣赦儀。



 質性卞急,好面折人。以廉介自持,未嘗受四方饋遺,前後所得祿賜多給孤遺。閨門之中,食不異品。身沒,家無餘貲。太祖因諭輔相,謂侍臣曰:「朕聞範質止有居第,不事生產,真宰相也。」太宗亦嘗
 稱之曰:「宰輔中能循規矩、慎名器、持廉節,無出質右者,但欠世宗一死,為可惜爾。」從子校書郎杲求奏遷秩,質作詩曉之,時人傳誦以為勸戒。有集三十卷,又述朱梁至周五代為《通錄》六十五卷,行於世。子旻。



 旻字貴參,十歲能屬文。以父任右千牛備身、太子司議郎,累遷著作佐郎。



 宋初,為度支員外郎、判大理正事,俄知開封縣。太宗時領京尹,數召與語,頗器重之。



 嶺南平,遷知邕州兼水陸轉運使。俗好淫祀,輕醫藥,重鬼旻下
 令禁之。且割己奉市藥以給病者,愈者千計,復以方書刻石置廳壁,民感化之。會南漢知廣州官鄧存忠劫土人二萬眾,攻州城七十餘日。旻屢出親戰,矢集於胸,猶激勵將卒殊死戰,賊遂少卻。病創日,堅壁固守,遣使十五輩求援。廣州救兵至,圍解,賜璽書獎之。旻病甚,詔令有司以肩輿載歸闕下。疾愈,通判鎮州,有能聲,賜錢二百萬,遷庫部員外郎。



 開寶九年,知淮南轉運事。太祖謂旻曰:「朕今委卿以方面之重,凡除民隱、急軍須之
 務,悉以便宜從事,無庸一一中履也。」歲運米百餘萬石給京師,當時稱有心計。



 太平興國初,召為水部郎中。錢俶獻地,以旻為考功郎中,權知兩浙諸州軍事。旻上言:「俶在國日,徭賦繁苛,凡薪粒、蔬果、箕帚之屬悉收算。欲盡釋不取,以蠲其弊。」從之。車駕征晉陽,上書求從,召為右諫議大夫、三司副使,判行在三司,又兼吏部選事。師還,加給事中。坐受人請求擅市竹木入官,為王仁贍所發,貶房州司戶。語在《仁贍傳》。量移唐州。六年,卒,年四十
 六。有集二十卷、《邕管記》三卷。其後子貽孫上言,詔復舊官。貽孫官至主客員外郎。



 杲字師回,父正,青州從事。杲少孤,質視如己子。刻志於學,與姑臧李均、汾陽郭昱齊名,為文深僻難曉,後生多慕效之。以蔭補太廟齋郎,再遷國子四門博士。



 嘗攜文謁陶谷、竇儀,咸大稱賞,謂杲曰:「若舉進士,當待汝以甲科。」及秋試,有上書言代閱之家不當與寒士爭科第,杲遂不應舉。稍遷著作佐郎,出為許、鄧二州從事,坐事免。
 太平興國初,遷著作郎、直史館,歷右拾遺、左補闕。雍熙二年,同知貢舉。俄上書自言其才比東方朔,求顯用,以觀其效。太宗壯之,擢知制誥。



 杲家貧,貸人錢數百萬。母兄晞性嗇,嘗為興元少尹,居主洋兆,殖貨鉅萬。親故有自長安來者,紿杲曰:「少尹不復靳財物,已揮金無算矣。」杲聞之喜,因上言兄老,求典京兆以便養。太宗從其請。改工部郎中,罷知制誥。杲既至,而晞吝如故,且常以不法事幹公府。杲大悔。杲視事逾年,境內不治。會賊帥劉渥
 剽掠屬縣,吏卒解散,遂驚悸成疾。



 移知壽州,上言:「家世史官,願秉直筆,成國朝大典。」召為史館修撰,固求掌誥詞,帝從之。時翰林學士宋白左遷鄜州,賈黃中、李沆參知政事,蘇易螽轉承旨,杲連致書相府,求為學士,且言於宰相李昉曰:「先公嘗授以制誥一編,謂杲才堪此職。」因出示昉,昉屢開解之。未幾,太宗飛白書「玉堂」額以賜翰林,杲又上《玉堂記》,因請備職。太宗惡其躁競,改右諫議大夫、知濠州,復召為史館修撰。



 初,太宗以太祖朝典
 策未備,乃議召杲。杲聞命喜甚,以為將加優擢,晨夜趨進。至宋州,遇朗州通判錢熙,杲問以「朝議將任僕何官」,熙言:「重修《太祖寶錄》爾。」杲默然久之。感疾,至京師,旬月卒,年五十六。太宗閔之,錄其二子。



 杲性虛誕,與人交,好面譽背非,惟與柳開善,更相引重,始終無間。不善治生,家益貧,杲端坐終日,不知計所出,人皆笑之。子坦亦登進士第。



 王溥字齊物,並州祁人。



 父祚,為郡小吏,有心計,從晉祖
 入洛,掌鹽鐵案,以母老解職歸。漢祖鎮並門,統行營兵拒存丹,委祚經度芻粟;即位,擢為三司副使。歷周為隨州刺史。漢法禁牛革,輦送京師,遇暑雨多腐壞,祚請班鎧甲之式於諸,令裁之以輸,民甚便之。移刺商州,以奉錢募人開大秦山巖梯路,行旅感其惠。顯德初,置華州節度,以祚為刺史。未幾,改鎮穎州均部內租稅,補實流徙,以出舊籍。州境舊有通商渠,距淮三百,歲久湮塞,祚疏導之,遂通舟楫,郡無水患。歷鄭州團練使。宋初,
 升宿州為防禦,以祚為使。課民鑿井修火備,築城北堤以御水災。因求致政,至闕下,拜左領軍衛上將軍,致仕。



 溥,漢乾祐中舉進士甲科,為秘書郎。時李宗貞據河中,趙思綰反京兆,王景崇反鳳翔,周祖將兵討之,闢溥為從事。河中平,得賊中文書,多朝貴及藩相交結語。周祖籍其名,將按之,溥諫曰:「魑魅之形,伺夜而出,日月既照,氛沴自消。願一切焚之,以安反側。」周祖從之。師還,遷太常丞。從周祖鎮鄴。廣順順初,授左諫議大夫、樞密直學
 士。二年,遷中書舍人、翰林學士。三年,加戶部侍郎,改端明殿學士。周祖疾革,召學士草制,以溥為中書侍郎、平章事。宣制畢,周祖曰:「吾無憂矣。」即日崩。



 世宗將親征澤、潞,馮道力諫止,溥獨贊成之。凱還,加兼禮部尚書,監修國史。世宗嘗從容問溥曰:「漢禁止李崧以蠟書與契丹,猶有記其詞者,信有之耶?」溥曰:「崧為大臣,設有此謀,肯輕示外人?蓋蘇逢吉誣之耳。」世宗始悟,詔贈其官。世宗將討秦、鳳,求帥於溥,溥薦向拱。事平,世宗因宴酌酒賜溥
 曰:「為吾擇帥成邊功者,卿也。」從平壽春,制加階爵。顯德四年,丁外艱。起復,表四上,乞終喪。世宗大怒,宰相範質奏解之,溥懼入謝。六年夏,命參知樞密院事。



 恭帝嗣位,加右僕射。是冬,表請修《世宗實錄》,遂奏吏館修撰、都官郎中、知制誥扈蒙,右司員外郎、知制誥張淡,左拾遺王格,直史館、左拾遺董淳,同加修纂,從之。



 宋初,進位司空,罷參知樞密院。乾德二年,罷為太子太保。舊制,一品班於臺省之後,太祖因見溥,謂左右曰:「溥舊相,當寵異之。」
 即令分臺省班東西,遂為定制。五年,丁內艱。服闋,加太子太傅。開寶二年,遷太子太師。中謝曰,太祖顧左右曰:「溥十年作相,三遷一品,福履之盛,近世未見其比。」太平興國初,封祁國公。七年八月,卒,年六十一。輟朝二日,贈侍中,謚文獻。



 溥性寬厚,美風度,好汲引後進,其所薦至顯位者甚眾。頗吝嗇。祚頻領牧守,能殖貨,所至有田宅,家累萬金。



 溥在相位,祚以宿州防禦使家居,每公卿至,必首謁。祚置酒上壽,溥朝服趨侍左右,坐客不安席,
 輒引避。祚曰:「此豚犬爾,勿煩諸君起。」溥諷祚求致政,祚意朝廷未之許也,既得請,祚大罵溥曰:「我筋力未衰,汝欲自固名位,而幽囚我。」舉大梃將擊之,親戚勸諭乃止。



 溥好學,手不釋卷,嘗集蘇冕《會要》及崔弦《續會要》,補其闕漏,為百卷,曰《唐會要》。又採朱梁至周為三十卷,曰《五十會要》。有集二十卷。



 子貽孫、貽正、貽慶、貽序。貽正至國子博士。貽慶比部郎中。貽序,景德二年進士,後改名貽矩,至司封員外郎。貽正子克明,尚太宗女鄭國長公主,
 改名貽永,令與其父同行。見《外戚傳》。



 貽孫字象賢,少隨周祖典商、穎二州,署衙內都指揮使。德中,以父在中書,改朝散大夫、著作佐郎。宋初,遷金部員外郎,賜紫,累遷右司郎中。淳化中,卒。太祖平吳、蜀,所獲文史副本分賜大臣。溥好聚書,至萬餘卷,貽孫遍覽之;又多藏法書名畫。太祖嘗問趙普,拜禮何以男子跪而婦人否,普問禮官,不能對。貽孫曰:「古詩云『長跪問故夫』,是婦人亦跪也。唐太后朝婦人始拜而不跪。」普問所出,對云:「大和中,
 有幽州從事張建章著《渤海國記》,備言其事。」普大稱賞之。端拱中,右僕射李倓求郡省百官集議舊儀,貽孫具以對,事見《禮志》,時論許其諳練云。



 魏仁浦字道濟,衛州汲人。幼孤貧,母為假黃縑制暑服,仁浦年十三,嘆曰:「為人子不克供養,乃使慈母求貸以衣我,我能安乎!」因慷慨泣下。辭母詣洛陽,濟河沉衣中流,誓曰:「不貴達,不復渡此!」晉末,隸樞密院為小史,任職端謹,儕輩不能及。契丹入中原,仁浦隨眾北遷。會契丹
 主殂於真定,仁浦得脫歸。魏帥杜重威素知仁浦謹厚,善書計,欲留補牙職。仁浦以重威降將,不願事之,遂遁去。重威遣騎追之,不及。漢祖起太原,次鞏縣,仁浦迎謁道左,即補舊職。



 時周祖掌樞密,召仁浦問闕下兵數,仁浦悉能記之,手疏六萬人。周祖喜曰:「天下事不足憂也。」遷兵房主事,從周祖鎮鄴。



 乾祐末,隱帝用武德使李鄴等謀,誅大臣楊邠、史弘肇等,密詔澶帥李洪義殺騎將王殷,令郭崇害周祖。洪義知事不濟,與殷謀,遣副使陳
 光穗繼詔示周祖。周祖懼,召仁浦入計,且示以詔曰:「朝廷將殺我,我死不懼,獨不念麾下將士乎?」仁浦曰:「侍中握強兵臨重鎮,有功朝廷,君上信讒,圖害忠良,雖欲割心自明,奚可得也,事將奈何。今詔始下,外無知者,莫若易詔以盡誅將士為名,激其怒心,非徒自免,亦可為楊、史雪冤。」周祖納其言,倒用留守印,易詔書以示諸將。眾懼且怒,遂長驅渡河。及即位,以仁浦為樞密副承旨,俄遷右羽林將軍,充承旨。



 周祖嘗問仁浦諸州屯兵之數
 及將校名氏,令檢簿視之。仁浦曰:「臣能記之。」遂手疏於紙,校簿無差,周祖尤倚重焉。廣順末,太原劉崇寇晉州,仁浦居母喪,而宅邇宮城,周祖步登寬仁門,密遣小黃門召仁浦計事。明日,起復卓職。周祖大漸,謂世宗曰:「李洪義長興節鎮,魏仁浦無遣違禁密。」



 世宗即位,授右監門衛大將軍、樞密副使。從征高平,周師不利,東偏已潰,仁浦勸世宗出陣西殊死戰,遂克之。師還,拜檢校太保、樞密使。故事,惟宰相生辰賜器幣鞍馬,世宗特以賜仁
 浦。從平壽春,加檢校太傅,進爵邑,遷中書侍郎、平章事、集賢殿大學士兼樞密使。世宗欲命仁浦為禁止,議者以其不由科第,世宗曰:「古人為宰禁止者,盡由科第耶?」遂決意用之。恭帝嗣位,加刑部尚書。



 宋初,進位右僕射,以疾在告。太祖幸其第,賜黃金器二百兩、錢二百萬。再上表乞骸骨,不許。乾德初,罷寧本官。開寶二年春宴,太祖笑謂仁浦曰:「何不勸我一杯酒?」仁浦奉觴上壽,帝密謂之曰:「朕欲親征太原,如何?」仁浦曰:「欲速不達,惟陛下慎
 之。」宴罷,就第,復賜上尊酒十石、御膳羊百口。從征太原,中途遇疾。還,至梁侯驛卒,年五十九,贈侍中。



 仁浦性寬厚,接士大夫有禮,務以德報怨。漢乾祐中,有鄭元昭者,開封浚儀人,為安邑、解縣兩池榷鹽使,遷解州刺史。會詔以仁浦婦翁李溫玉為榷鹽使管兩池,元昭不得專其利。仁浦方為樞密院主事,元昭意仁浦必庇溫玉,會李守貞以河中叛,溫玉子在城中,元昭即系溫玉以變聞。時周祖總樞務,知其有間,置而不問。顯德中,仁浦為
 樞密院,元昭水自安。及代歸闕,道洛都,以情告仁浦弟仁滌,仁滌曰:「公第去,可無憂。我兄素寬仁有度,雖公事不欲傷於人,豈念私隙乎?」元昭至京師,仁浦果不介意,白周祖授元昭慶州刺史。漢陷帝寵作坊使買延徽,延徽與仁浦並居,欲並其第,屢譖仁浦,幾至不測。及周祖入汴,有擒延徽授仁浦者,仁浦謝曰:「因兵戈以報怨,不忍為也。」力保全之。當時稱其長者。世宗朝近侍有改忤上至死者,仁浦力救之,全活者眾。淮南之役,獲賊數千
 人,仁浦從容上言,俾隸諸軍,軍中無濫殺者。



 景德四年,其子咸信請謚曰宣懿。



 子咸美、咸熙、咸信。咸美以左司禦率府率致仕。咸熙性仁孝,嘗會賓客,家童數輩覆案碎器,客皆驚愕,咸熙色不變,止令更設饌具。其寬厚若此。以父任,累遷屯田郎中,後至太僕少卿。卒年四十九。子昭慶部員外郎,昭文西染院使,昭素供奉官、閣門祗侯。



 咸信字國寶,建隆初,授朝散大夫、太子右坊通事舍人,
 改供奉官。



 初,太祖在潛邸,昭憲太后嘗至仁浦第,咸信方幼,侍母側,儼如成人。太后奇之,欲結姻好。開寶中,太宗尹京,成昭憲之意,延見咸信於便殿,命與御帶黨進等較射,稱善。遂選尚永慶公主,授右衛將軍、附馬都尉。逾年,出領吉州刺史」



 太平興國初,真拜本州防禦使。四年,詔用奉外賜錢十萬。五年,坐遣親吏市木西邊,矯制免所過稅算,罰一季奉。俄遷慎州觀察使。雍熙三年冬,契丹擾邊,王師出討,悉命諸主婿鎮要地:王承衍知大
 名,石保吉知河,咸信知澶州。四年,本郡黃河清,咸信以聞,詔褒答之。籍田畢,就拜彰德軍節度。八月,遣歸治所。



 淳化四年,河決澶淵,陷北城,再命知州事。太宗親諭方略,傳置而往。時遣閻承翰修河橋,咸信請及流水未下造舟為便,承翰入奏:「方冬難成,請權罷其役。」咸信因其去,乃集工成之。奏至,上大悅。河平,遣還役兵。俄詔留築堤,咸信以為天寒地涸,無決溢之患,復奏罷之。



 真宗即位,改定國軍節度。咸平中,大閱東郊,以為舊城內都
 巡檢。車駕北征,為貝冀路行營都統署,詔督師。至貝州,敵人退,召還行在所。景德初,從幸澶州,石保吉與李繼隆為排陣使。契丹請和,帝置酒行宮,面賞繼隆、保吉,咸信避席,自愧無功,上笑而撫尉之。二年,改武成軍節度,知曹州。秋霖積潦,咸信決廣濟河堤以導之,民田無害。扈駕朝陵還,上言先墳在洛,欲立碑,求泣盟津,以便其事,即改知河陽。大中祥符初,從東封,加檢校太尉。將祀汾陰,命知澶州,令入內副都知張繼能諭旨。移領忠武
 軍節度。



 未幾召還,年已昏眊,見上,希旨求寵渥。七年,表乞任用,上出示中書向敏中曰:「咸信聯榮戚里,位居節制,復何望耶?」是冬,以新建南京,獎太祖舊臣,加同平章事。俄判天雄軍。天禧初,改陜州大都督府長史、保平軍節度。有感風疾苦,歸。真宗嘗謂宰相曰:「咸信老病,諸電子出版系統不克承順,身後復能保守其家業耶?」未幾卒,年六十九,贈中書令。錄其諸子孫侄,遷官者七人。



 咸信頗知書,善待士,然性吝喜利,仁浦所營邸舍悉擅有之。既卒,為諸
 侄所訟,時人恥之。



 子昭易、昭侃。昭易西京作坊使,知隰州。昭侃改名昺,為崇儀使。



 昭亮字克明,公主所生。幼未名,太宗召入禁中,命賦賞花詩,詩成上之,太宗大悅,酌以上尊酒,命筆題「從訓」、「昭亮」二名,令自擇之。拜如京副使,遷如京、洛苑使,掌翰林司。丁公主憂,起復,授六宅使,領富州刺史,遷內藏庫副蛤未幾,拜西上閣門使,進秩東上。上言閣門舊儀制未當,乃詔龍圖閣學士陳彭年、待制張知白、引進使白文
 肇與昭亮同加詳定,既成,賜白金千兩。又建議設儀石於內殿,加領恩州團練使。時咸信在大名,屬生日,命昭帝就賜禮物。是日,告命至,軍府榮之。父卒,遷四方館使,仍兼掌客省,多糾群官之失儀者。昭亮多病在告,詔給其奉。天禧二年,卒。



 昭亮未死日,數遣人入謁,求進用,加兼端州防禦使。未及拜命,死,仍以制書賜其家,贈貝州觀察使。以弟昭侃為供備庫使,子餘慶為內殿崇班。



 昭亮與陳彭年款暱,彭年嘗稱其才。昭亮居官務皦察,多
 遣人偵伺僚樞密承旨尹德潤嘗少之。會閣門副使焦守節、內殿崇班郭盛以役卒與德潤治第,昭亮廉知發其事,皆坐黜削。李維即王曾妻子叔父,同在翰林,曾受詔試舉人,以家事屬維。昭亮意曾受祈請,奏其竊語。遣中使參問無他狀,曾始得釋。昭亮陰險多此類,時人惡之。餘慶改名成德,為供備庫副蛤



 贊曰:五季至周之世宗,天下將定之時也。範質、王溥、魏仁浦,世宗之所拔擢,而皆有宰相之器焉。宋祖受命,遂
 為佐命元臣,天之所置,果非人之所能測歟。質以儒者曉暢軍事,及其為相,廉慎守法。溥刀筆家子,而好學終始不倦。仁浦嘗為小史,而與溥皆以寬厚長者著稱,豈非絕人之資乎。質臨終,戒其後勿請謚立碑,自悔深矣。太宗評質惜其欠世宗一死。嗚呼,《春秋》之法責備賢者,質可得免乎!



\end{pinyinscope}