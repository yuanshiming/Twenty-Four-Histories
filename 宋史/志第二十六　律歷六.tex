\article{志第二十六 律歷六}

\begin{pinyinscope}

 崇天歷



 步交會



 交終分:二十八萬八千一百七十七、秒四千二
 百七十七。



 交終日:二十七、餘二千二百四十七、秒四千二百七十七。



 交中日:一十三、餘六千四百一十八、秒七百三十八半。



 朔差日:二、餘三千三百七十一、秒五千七百二十三。



 後限日:一、餘一千六百八十五、秒七千八百六十一半。



 望策:十四、餘八千一百四、秒五十。



 前限日:十二、餘四千七百三十二、秒九千二百七十七。交率:一百四十一。



 交數:一千七百九十六。



 交終度:三百六十三度七十六分。



 交象:九十度九十四。



 半交:一百八十一度八十八。



 陽歷食限:四千二百。



 陽歷定法:四百二十。



 陰歷食限:七千。



 陰歷定法:七百。



 推天正十一月經朔加時入交:置天正十一月朔積分,以交終分秒去之,不盡,滿樞法為日,不滿為餘秒,即天
 正經朔加時入交泛日及餘秒。



 求次朔及望入交:因天正經朔加時入交泛日及餘秒,求次朔,以朔差日及餘秒加之;求望,以望策及餘秒加之:滿交終日及餘秒皆去之,即次朔及望加時所入。若以經朔、望小餘減之,即各得朔、望夜半入交泛日及餘秒。



 求定朔夜半入交:因經朔、望夜半入交,若定朔、望大餘有進退者,亦進退交日,不則因經為定,各得所求。



 求次
 定朔夜半入交:各因前定朔夜半入交,大月加日二,小月加日一,餘皆加八千三百四十二、秒五千七百二十三;若求次日,累加一日:滿交終日及餘秒皆去之,即得次定朔及每日夜半入交泛日及餘秒。



 求朔望加時入交常日:置經朔、望入交泛日及餘秒,以其朔、望入氣朏朒定數,朏減朒加之,即朔、望入交常日及餘秒。



 求朔望加時入交定日:置其朔、望入轉朏朒定數,以交
 率乘之,如交數而一,所得,以朏減朒加入交常日餘,滿若不足,進退其日,即朔、望加時入交定日及餘秒。



 求月行入陰陽歷:視其朔、望入交定日及餘秒,在中日及餘秒以下者為月在陽歷;如中日及餘秒已上者,減去之,為月在陰歷。凡入交定日,陽初陰末為交初,陰初陽末為交中。



 求朔望加時月入陰陽歷積度:置其月入陰陽歷日及餘,其餘,先以一百乘之,樞法除為約分。



 以九百九乘之,六十八除為度,不盡,退除為分,即朔、望加時月入陰陽歷積度及分。其月在陽
 歷,即為入陽歷積度;月在陰歷,即為入陰歷積度。



 求朔望加時月去黃道度:置入陰陽歷積度及分,如交象以下為在少象;已上,覆減半交,餘為入老象。置所入老少象度及分,以五因之,用減一千一十,餘,以老少象度及分乘之,八十四而一,列於上位;又置所入老少象度及分,如半象以下為在初限;已上,減去半象,餘為入末限。置初、末限底及分於上,列半象度及分於下,以上減下,餘以乘上,四十而一,所得,初限以減,末限以加,上
 位滿百為度,不滿為分,即朔、望加時月去黃道度數及分。



 求食定餘:置定朔小餘,如半法以下覆加半法,餘為午前分;已上,減去半法,餘為午後分。置午前、後分於上,列半法於下,以上減下,以下乘上,午前以三萬一千七百七十除,午後以一萬三千八百八十五除之,各為時差。午前以減、午後以加定朔小餘,各為食定小餘。以時差加午前、後分,為午前、後定分。其月食,直以定望小餘便為食定小餘。



 求日月食甚辰刻:置食定小餘,以辰法除之為辰數,不滿,進一位,刻法除之為刻,不滿為刻分。其辰數命子正,算外,即食甚辰、刻及分。



 求氣差:置其朔中積,滿二至限去之,餘在一象以下為在初;已上,覆減二至限,餘為在末。皆自相乘,進二位,滿二百三十六除之,用減三千五百三十三,為氣差。以乘距午定分,半晝分而一,所得以減氣差,為定數。春分後,交初以減,交中以加;秋分後,交初以加,交中以減。



 求刻差:置其朔中積,滿二至限去之,餘,列二至限於下,以上減下,餘以乘上,進二位,滿二百三十六除之,為刻差以乘距午定分,四因之,樞法而一,為定數。冬至後食甚在午前,夏至後食甚在午後。交初以加,交中以減。冬至後食甚在午後,夏至後食甚在午前。交初以加,交中以減。



 求日入食限:置入交定日及餘秒,以氣、刻、時三差定數各加減之,如中日及餘秒以下為不食;已上者,減去中日及餘秒,如後限以下、前限已上為入食限;後限以下
 為交後分;前限以上覆減中日,餘為交前分。



 求日食分:置入交前後分,如陽歷食限以下者為陽歷食定分;已上者,覆減一萬一千二百,餘為陰歷食定分;不足減者,不食。



 各如限陽歷定法而一,為食之大分,不盡,退除為小分,半已上為半強,半以下為半弱。命大分以十為限,得日食之分。



 求日食泛用分:置朔入陰陽歷食定分,一百約之,在陽歷者列入十四於下,在陰歷者列一百四十於下,各以
 上減下,餘以乘上,進二位,陽歷以一百八十五除,陰歷以五百一十四除,各為日食泛用分。



 求月入食限:視月入陰陽歷日及餘,如後限以下為交後分;前限已上覆減中日,為交前分。



 求月食分:置交前後分,如三千二百以下者,食既;已上,用減一萬二百,不足減者不食;餘以七百除之為大分,不盡,退除為小分,小分半已上為半強,半已下為半弱。命大分以十為限,得月食之分。



 求月食泛用分:置望入交前後分,退一等,自相乘,交初以九百三十五除,交中以一千一百五十六除之,得數用減刻率,交初以一千一百一十一為刻率,交中以九百為刻率。



 各得所求。



 求日月食定用分:置日月食泛用分,以一千三百三十七乘之,以所食日轉定分除之,即得所求。



 求日月食虧初復滿小餘:各以定用分減食甚小餘,為虧初;加食甚小餘,為復滿:即各得虧初復滿小餘。若求時刻者,依食甚術入之。



 求月食更籌定法:置其望晨分,四因之,退一等,為更法;倍之,退一等,為籌法。



 求月食入更籌:置虧初、食甚、復滿小餘,在晨分以下加晨分,昏分已上減去昏分,餘以更法除之為更數,不滿,以籌法除之為籌數。其更數命初更,算外,即各得所入更、籌。



 求朔、望食甚宿次:置其經朔、望入氣小餘,以入氣、入轉朏朒定數朏減朒加之,乘其日升降分,樞法而一,加減
 其日盈縮分,至後、分前以加,分後、至前以減。



 一百約之為分,分滿百為度,以盈加縮減其定朔、望加時中積,以天正冬至加時黃道日度及分加而命之,即定朔、望加時日躔宿次。其望加半周天,命如前,即朔、望食甚宿次。



 求月食既內外刻分:置月食交前、後分,覆減三千二百,不及減者,為食下既。



 一百約之,列六十四於下,以上減下,餘以乘上,進二位,交初以二百九十三除,交中以三百六十五除,所得,以定用分乘之,如泛用分而一,為月食既內刻
 分;覆減定用分,即既外刻分。



 求日月帶食出入分數:各以食定小餘與日出、入分相減,餘為帶食差;其帶食差滿定用分已上者,不帶食出入也。



 以帶食差乘所食分,滿定用分而一,若月食既者,以既內刻分減帶食差,餘所食分,以既外刻分而一,不及減者,為帶食既出入也。



 各以減所食分,即帶出、入所見之分。其朔日食甚在晝者,晨為漸進之分,昏為已退之分;若食甚在夜者,晨為已退之分,昏為漸進之分。其月食者,見此可知也。



 求日食所起:日在陰歷,初起西北,甚於正北,復於東北;
 日在陽歷,初起西南,甚於正南,復於東南。其食八分已上者,皆起正西,復於正東。此據午地而論之,其餘方位,審黃道斜正、月行所向,可知方向。



 求月食所起:月在陰歷,初起東南,甚於正南,復於西南;月在陽歷,初起東北,甚於正北,復於西北。其食八分已上,皆起正東,復於正西。此亦據午地而論之,其餘方位,依日食所向,即知既虧、復滿。



 步五星



 五星會策:十五度二十一分、秒九十。



 木星周率:四百二十二萬四千五十八、秒三十二。



 周日:三百九十八、餘九千二百三十八、秒三十二。



 歲差:一百三、秒六。



 伏見度:一十三。



 木星盈縮歷火星周率:八百二十五萬九千三百六十六、秒五十九。



 周日:七百七十九、餘九千七百五十六、秒五十九。



 歲差:一百三、秒五十三。



 伏見度:二十。



 火星盈縮歷



 土星周率:四百萬三千八百七十二、秒三十九。



 周日:三百七十八、餘八百五十二、秒三十九。



 歲差:一百三、秒七十八。



 伏見度:一十六。



 土星盈縮歷金星周率:六百一十八萬三千五百九十九、秒一十六。



 周日:五百八十三、餘九千六百二十九、秒一十六。



 歲差:一百三十、秒八十。



 夕見晨伏度:一十一。



 晨見夕伏度:九。



 金星盈縮歷



 水星周率:一百二十二萬七千一百七十、秒二十八。



 周日:一百一十五、餘九千三百二十、秒二十八。



 歲差:一百三、秒九十四。



 夕見晨伏度:一十四。



 晨見夕伏度:二十一。



 水星盈縮歷



 推五星天正冬至後諸變中積中星:置氣積分,各以其星周率去之,不盡,覆減周率,餘滿樞法除之為日,不滿,退除為分,即天正冬至後平合中積;命之,積平合中星,以諸段變日、變度累加之,即諸變中積中星。其經退行者,即其變度;累減之,即其星其變中星。



 求五星諸變入歷:以其星歲差乘積年,滿周天分去之,不盡,以樞法除之為度,不滿,退除為分,以減其星平合中星,即平合入歷;以其星其變限度依次加之,各得其星諸變入歷度分。



 求五星諸變盈縮定差:各置其星其變入歷度分,半周天以下為在盈;以上,減去半周天,餘為在縮。置盈縮限度及分,以五星會策除之為會數,不盡,為入會度及分;以其會下損益率乘之,會策除之為分,分滿百為度,以
 損益其下盈縮積度,即其星其變盈縮定差。若用立成者,以其所入會度下差而用之。



 其木火土三星後退、後留者,置盈縮差,各列其星盈縮極度於下,皆以上減下,餘以乘上,八十七除之,所得,木、土三因,火直用之;在盈益減損加、在縮益加損減其段盈縮差,為後退、後留定差,因為後遲初段定差。各須類會前留定差,觀其盈縮初末,審察降殺,皆裒多益少而用之。



 求五星諸變定積:各置其星其變中積,以其變盈縮定差盈加縮減之,即其星其變定積及分;以天正冬至大
 餘及分加之,即其星其變定日及分;以紀法去定日,不盡,命甲子,算外,即得日辰。



 求五星諸變在何月日:各置諸變定日,以其年天正經朔大餘及分減之,若冬至大餘少,加經朔大餘者,加紀法乃減之。



 餘以朔策及分除之為月數,不滿,為入月日數及分。其月數命以天正十一月,算外,即其星其變入其月經朔日數及分。若置定積,以天正閏月及分加之,朔策除為月數,亦得所求。



 求五星諸變入何氣日:置定積,以氣策及約分除之為
 氣數,不盡,為入氣已來日數及分。其氣數命起天正冬至,算外,即五星諸變入其氣日及分。其定積滿歲周日及分即去之,餘在來年天正冬至後。



 求五星諸變定星:各置其變中星,以其變盈縮定差盈加縮減之,其金、水二星,金以倍之,水以三之,乃可加減。



 即五星諸變定星;以天正冬至加時黃道日度加而命之,即其星其變加時定星宿次及分。五星皆以前留為前退初日定星,後留為後遲初日定星。



 求五星諸變初日晨前夜半定星:以其星其變盈縮所
 入會度下盈縮積度與次度下盈縮積度相減,餘為其度損益分;乘其變初行率,一百約之,所得,以加減其日初行率,在盈,益加損減;在縮,益減損加。



 為初行積率;又置一百分,亦依其數加減之,以除初行積率,為初日定行率;以乘其率初日約分,一百約之,順減退加其日加時定星,為其變晨前夜半定星;加冬至時日度命之,即所在宿次。



 求諸變日度率:置後變定日,以其變定日減之,餘為其變日率;又置後變夜半定星,以其變夜半定星及分減
 之,餘為其變度率及分。



 求諸變平行分:各置其變度率及分,以其變日率除之為平行分,不滿,退除為秒,即各得平行度及分秒。



 求諸變總差:各以其段平行分與後段平行分相減,餘為泛差;並前段泛差,四因之,九而一,為總差。若前段無平行分相減為泛差者,各因後段初日行分與其段平行分相減,為半總差;倍之,為總差。



 若後段無平行分相減為泛差者,各因前段末日行分與其段平行分相減,為半總差。



 其前後退行者,各置本段平行分,十四乘,十五
 除,為總差。其金星夕退、夕伏、再合、晨退,各依順段術入之,即得所求。



 求諸段初末日行分:各半其段總差,加減其段平行分,後段行分多者,減之為初,加之為末;後段行分少者,加之為初,減之為末。



 即各得其星其段初、末日行度及分秒。凡前後段平行分俱多或俱少,乃平注之;及本段總差不滿大分者,亦平注之。其退行段,各以半總差前變減之為初,加之為末;後變加之為初,減之為末。



 求每日晨前夜半星行宿次:置其段總差,減其段日率,以除之,為日差;以日差累損益初日行分,後段行分少,日損之;後段行分多,日益之。



 為每日行度及分;以每日行度及分累加其
 星其段初日晨前夜半宿次,命之,即每日星行宿次。遇退行者,以每日行分累減之,即得所求。



 徑求其日宿次:置所求日,減一,日差乘之,加減初日行分,後行分少,即減之;後行分多,即加之。



 為所求日行分;加日行分而半之;以所求日乘之,為徑求積度;加減其星初日宿次;命之,即其日星行宿次。



 求五星定合日定星:以其星平合初日行分減一百分,餘以約其日太陽盈縮分為分,分滿百為日,不滿為分,
 命為距合差日;以盈縮分減之,為距合差度;以差日、差度縮加盈減平合定積、定星,為其星定合日定積、定星。其金、水二星,以一百分減初日行分,餘以除其日太陽盈縮分,為距合差日;以盈縮分加之,為距合差度;以差日、差度盈加縮減之。



 金、水二星退合者,以初日行分加一百分,以除太陽盈縮分,為距合差日;以距合差日減盈縮分,為距合差度;以差日、差度盈減縮加再合定積定星為其星再合定日定積定星。



 其金、水二星定積,各依見伏術,先以盈縮差求其加減訖,然後以距合差日、差度加減之。



 求木火土三星晨見夕伏定日:各置其星其段定積,乃加減一象度,晨見加之,夕伏減之。



 半周天已下自相乘,半周天已
 上,覆減周天度及分,餘亦自相乘,一百約為分,以其星伏見度乘之,十五除之,為差;乃以其段初日行分覆減一百分,餘以除其差為日,不滿,退除為分,所得,以加減定積,晨見加之,夕伏減之。



 各得晨見、夕伏定積;加天正冬至大餘及分,命甲子,算外,即得日辰。



 求金水二星夕見晨伏定日:各置其星其段定積,其定積先倍其段盈縮差,縮加盈減之,乃加減一象度,夕見減之,晨伏加之。



 半周天已下自相乘,已上,覆減周天度,餘亦自相
 乘,一百約為分,以其星伏見度乘之,十五除為差;乃置其段初日行分,減去一百分,餘以除其差為日,不滿,退除為分,所得,以加減定積,夕見加之,晨伏減之。



 各得夕見、晨伏定積。



 求金水二星晨見夕伏定日:置其星其段定積,其定積先以一百乘其段盈縮差,乃以一百分加其日行分,以除其差,所得,盈加縮減之,加減一象度,晨見加之,夕伏減之。



 半周天已下自相乘,已上,覆減周天度,餘亦自相乘,一百約
 為分,以其星伏見度乘之,十五除,為差;乃置其段初日行分,如一百,以除其差為日,不滿,退除為分,所得,以加減定積,晨見加之,夕伏減之。



 各為其星晨見、夕伏定積。



 歷既成,以來年甲子歲用之,是年五月丁亥朔,日食不效,算食二分半,候之不食。



 詔候驗。至七年,命入內都知江德明集歷官用渾儀較測。時周琮言:「古之造歷,必使千百年間星度交食,若應繩準,今歷成而不驗,則歷法為未密。」又有楊皞、於淵者,與琮求較驗,而皞術於木為得,淵於金
 為得,琮於月、土為得,詔增入《崇天歷》,其改用率數如後:



 周天分:三百八十六萬八千六十六、秒一十七。



 周天:三百六十五度。虛分二千七百一十六、秒十七,約分二十五、秒六十一。



 歲差:一百二十
 六、秒一十七。



 木星



 求諸變總差:各以其段平行分與後段平行分相減,餘為泛差;並前段泛差,四因之,退一等,為總差。若前段無平行分相減為泛差,各因後段初日行分與其段平行分相減,為半總差;倍之,為總差。



 若後段無平行分相減為泛差者,各因前段末日行分與其段平行分相減,為半總差;倍之,為總差。



 其前後退行者,各置本段平行分,十四乘,十五除,為總差。其金星夕退、夕伏、再合、晨退,各依順段術入之,即得所求。



 求五星定合及見伏泛用積:其木、火、土三星,各以平合及前疾、後伏定積為泛用積,金、水二星平合及夕見、晨伏者,置其星其段盈縮差,金以倍之,水以三之,列於上位;又置盈縮差,以其段初行率乘之,退二等,以減上位;又置初行率,減去一百分,餘以除之為日,不滿,退除為分,乃盈減縮加中積,為其星其變泛用積。



 金、水二星再合及夕伏、晨見者,其星其段盈縮差,金星直用,水以倍之,進二位,以其段初行率加一百分以除之,所得,並盈縮差,以盈加縮減中積,為其星其段泛用積。



 求五星定合定積定星:其木、火、土三星平合者,以平合初日行分減一百分,餘以約其日太陽盈縮分為分,滿百為日,不滿為分,命為距合差日;以盈縮分減之,為距合差度;
 以差日、差度縮加盈減其星平合泛用積,為其星定合日定積定星。



 金、水二星平合者,以一百分減初日行分,餘以除其日太陽盈縮分,為距合差日;以盈縮分加之,為距合差度;以差日、差度盈加縮減平合泛用積,為其星定合日定積定星也。



 金、水二星退合者,以初日行分一百分,以除太陽盈縮分,為距合差日;以距合差日減盈縮分,為距合差度;以差日盈減縮加再合泛用積,為其星再合定日定積差度;盈加縮減再合泛用積,為其星再合日定星;各加冬至大、小餘及黃道加時日躔宿次命之,即得其日日辰及宿次。



 求木火土星晨見夕伏定用積:各置其星其段泛用積,乃加減一象度,晨見加之,夕伏減之。



 半周天已下自相乘,已上,覆
 減周天度,餘亦自相乘,各二因百約之,在一百六十七已上,以一百約其日太陽盈縮分減之,不滿一百六十七者即加之,以其星本伏見度乘之,十五除,為差;乃置其段初日行分,覆減一百分,餘以除其差為日,不滿,退除為分所得,以加減泛用積,晨見加之,夕伏減之。



 各得其星晨見、夕伏定用積;加天正冬至大餘,命甲子,算外,即得日辰。



 求金水二星夕見晨伏定用積:各置其星其段泛用積,乃加減一象度,夕見減之,晨伏加之。



 半周天已下自相乘,已上,覆
 減周天度,餘亦自相乘,二因百約之,滿一百六十七已上,以一百約太陽盈縮分減之,不滿一百六十七者即加之,以其星本伏見度乘之,十五除,為差;乃置其段初日行分,減去一百分,餘?



\end{pinyinscope}