\article{志第一百三 輿服二}

\begin{pinyinscope}

 後妃車輿皇太子王公以下車輿傘扇鞍勒門戟旌節。



 皇后之車,唐制六等:一曰重翟,二曰厭翟,三曰翟車,四曰安車,五曰四望車,六曰金根車。宋因之,初用厭翟車。
 其制:箱上有平盤,四角曲闌,兩壁紗窗,龜文,金鳳翅,前有虛匱、香爐、香寶,緋繡幰衣、絡帶、門簾,三轅鳳首,畫梯,推竿,行馬,緋繒裹索。駕六馬,金銅面,纓轡,鈴攀,緋屜。駕士三十人,武弁、緋繡衫。常出止用正、副金塗銀裝白藤輿各一,上覆棕櫚屋,飾以鳳,輦官服同乘輿平頭輦之制。



 徽宗政和三年,議禮局上皇后車輿之制:重翟車,青質,金飾諸末,間以五採。輪金根朱牙。其箱飾以重翟羽,四面施雲鳳、孔雀,刻鏤龜文。頂輪上施金立鳳、耀葉。
 青羅幰衣一,紫羅畫雲龍絡帶二,青絲絡網二,紫羅畫帷一,青羅畫雲龍夾幔二。車內設紅褥及坐,橫轅上施立鳳八。香匱設香爐、香寶,香匱飾以螭首。前後施簾,長轅三,飾以鳳頭,青繒裹索。駕青馬六,馬有銅面,插翟羽,鞶纓,攀胸鈴拂,青屜,青包尾。若受冊、謁景靈宮,則乘之。



 厭翟車,赤質,其箱飾以次翟羽;紫幰衣,紅絲絡網,紅羅畫絡帶,夾幔錦帷,餘如重翟車。駕赤騮四。若親蠶則乘之。翟車,黃質,其車側飾以翟羽;黃幰衣,黃絲絡網,錦帷
 絡帶,餘如重翟車。駕黃騮四。安車,赤質,金飾,間以五採,刻鏤龜文;紫幰衣,錦帷絡帶,紅絲絡網,前後施簾;車內設褥及坐,長轅三,飾以鳳頭,駕赤騮四。凡駕馬鞶纓之飾,並從車質。四望車,朱質,青幰衣,餘同安車。駕牛三。金根車,朱質,紫幰衣,餘同安車。駕牛三。自重翟車以下,備鹵簿則皆以次陳設。藤輿,金塗銀裝。上覆棕櫚屋,以龍飾,常行之儀則用之。



 龍肩輿。一名棕簷子,一名龍簷子,舁以二竿,故名簷子,
 南渡後所制也。東都,皇后備厭翟車,常乘則白藤輿。中興,以太后用龍輿,後惟用簷子,示有所尊也。其制:方質,棕頂,施走脊龍四,走脊云子六,朱漆紅黃藤織百花龍為障;緋門簾、看窗簾,朱漆藤坐椅,踏子,紅羅裀褥,軟屏,夾幔。



 隆興二年正月,皇后受冊畢,擇日朝謁,有司具儀物,乞乘肩輿龍簷。制造所受給使臣尹肇發,納中宮金塗銀葉棕櫚、朱漆紅黃藤織百花龍枰子、碌牙壓貼、鏤金雕木腰花泥版龍簷子一乘。金塗銀頂子,龍頭六,走
 脊龍四,走脊云子六,貼絡龍四十,貼絡云子三十,鐸子八,插拴坐龍四,環索全,鈸遮那一副,檀香龜背紅紗窗四扇,紅羅緣紅篸門簾一,瀝水全,看窗簾二,朱漆藤面明金雕木龍頭椅一,腳踏一,紅絲泉絳結一,朱漆小幾二,紅羅褥全,紅羅緣肩膊席褥一十六,系帶全,金塗銀鐵胎桿鞫四,魚鉤四,火踏一,朱漆梯盤全,朱漆衣匣二,金塗銅手把葉段拓叉二,金塗銅叉頭拖泥行馬二,金塗銀葉杠子二,紅茸匾絳四,紅羅夾軟屏風、夾幔各一,襯
 腳席褥、靠背坐褥及踏床各一,紅絹十字帕一,竿袋四,魚鉤帕二,紅油十字帕、竿袋、魚鉤帕數同上,兜地帕一,圍裙一。



 大安輦。真宗咸平中,為萬安太后制輿,上設行龍六。乾興元年,詔皇太后御坐簷子,名大安輦。神宗嗣位,尊皇太后為太皇太后,其行幸依治平元年之制。而皇太后、皇后常出,止用副金塗銀裝白藤輿,覆以棕櫚屋,飾以鳳。輦官服同乘輿平頭輦之制。於是詔太皇太后出入所乘,如萬安太后輿,上設行龍六,制飾率有加。
 金銅車,禮典不載,則如舊制。



 哲宗紹聖元年,議造皇太后大安輦,中書具治平、元豐中皇太后輿服儀衛以呈,曰:「元豐中,先帝手詔,皇太后行幸儀衛,並依慈聖光獻太皇太后日例,而宣仁謙恭,不乘大安輦。」哲宗曰:「今皇太后獨尊,非宣仁比。」遂詔行幸進大安輦,已而皇太后嫌避,竟不制造。



 龍興。皇太后所乘也。東都,皇太后多垂簾,皆抑損遠嫌,不肯乘輦,止用輿而已。哲宗既嗣位,尊朱貴妃為皇
 太妃,出入許乘簷子。有司請用牙魚鳳為飾,傘用青。元祐三年,太皇太后詔有司尋繹典故,於是簷子飾以龍鳳,傘用紅。九年,君臣議改簷子為輿,上設行龍五,出入由宣德東偏門。哲宗以皇太后諭旨,令太妃坐六龍輿出入,進黃傘,由宣德正門。於是三省議,皇太妃坐龍鳳輿,傘紅黃兼用,從皇太后出入,止用紅。紹聖元年,禮部太常寺言:「近奉旨:『皇太后欲令皇太妃坐六龍輿,朕常思皇太妃尊奉之禮,既不敢擬隆於皇太后,又不可不逮
 於中宮。』今參以人情,再加詳定,伏請供進龍鳳輿。」從之。



 及徽宗即位,尊太妃為聖瑞皇太妃,詔儀物除六龍輿不用,仍進龍鳳輿外,餘悉增崇焉。紹興奉迎皇太后,詔造龍輿,其制:朱質,正方,金塗銀飾,四竿,竿頭螭首,赭窗紅簾,上覆以棕,加走龍六,內設黃花羅帳、裀褥、朱椅、踏子、紅羅黃羅繡巾二。



 皇太子車輅之制。唐制三等:一曰金輅,二曰軺車,三曰四望車。太宗至道初,真宗為皇太子,謁太廟,乘金輅,常
 朝則乘馬。真宗天禧中,仁宗為皇太子,亦同此制。徽宗政和三年,議禮局上皇太子車輅之制:金輅,赤質,金飾諸末。重較,箱畫苣文鳥獸;黃屋,伏鹿軾,龍輈,金鳳一在軾前。設障塵。朱蓋黃里。輪畫朱牙。左建旗,九旒,右載闟戟。旗首金龍頭,銜結綬及鈴綏。八鸞在衡,二鈴在軾。駕赤騮四,金鍐方釳,插翟尾,鏤錫,鞶纓九就。從祀、謁太廟、納妃則供之。軺車,金飾諸末,紫油通幰,紫油纁朱裏,駕馬一。四望車,金飾諸末,青油通幰,青油纁朱裏,朱絲絡
 網,駕馬一。軺車、四望車以次列於鹵簿仗內。皇太子妃,則有厭翟車,駕以三馬。郵入亦乘簷子,中興簡儉,惟用藤簷子,頂梁、舁杠皆飾以玄漆,四角刻獸形,素藤織花為面,如政和之制。



 親王群臣車輅之制。唐制有四:一曰象輅,親五及一品乘之;二曰革輅,二品、三品乘之;三曰木輅,四品乘之;四曰軺車,五品乘之。宋親王、一品、二品奉使及葬,並給革輅,制同乘輿之副,惟改龍飾為螭。六引內三品以上乘革車,赤質,制如進賢車,無案,駕四赤
 馬,駕士二十五人。其緋幰衣、絡帶、旗戟、綢杠繡文:司徒以瑞馬,京牧以隼,御史大夫以獬豸,兵部尚書以虎,太常卿以鳳,駕士衣亦同。縣令乘軺車,黑質,兩壁紗窗,一轅,金銅飾,紫幰衣、絡帶並繡雉銜瑞草,駕二馬,駕士十八人。百官常朝皆乘馬。



 真宗大中祥符四年,知樞密院事王欽若言:「王公車輅上並用龍裝,乞下有司檢定制度。」詔下太常禮院詳定。本院言:「按《鹵簿令》,王公已下,像輅以象飾諸末,朱班輪,八鸞在衡,左建旗畫龍,一升
 一降,右載闟戟。革略以革飾諸末,左建旃,餘同象輅。木輅以漆飾之,餘同革輅。軺車,曲壁,青幰碧裏。諸輅皆朱質,朱蓋,朱旗旃,一品九旒,二品八旒,三品七旒,四品六旒,其鞶纓如之。」



 神宗元豐三年,詳定禮文所言:「《鹵簿記》公卿奉引:第一開封令,乘軺車;次開封牧,隼旗;次太常卿,鳳旗;次司徒,瑞馬旗;次御史大夫,獬豸旗;次兵部尚書,虎旗,而乘革車。考之非是。謹按《周禮》巾車職曰:『孤乘夏篆,卿乘夏縵,大夫乘墨車。』司常職曰:『孤、卿建旃,大夫建
 物。』請公卿已下奉引,先開封令,乘墨車建物;次開封牧,乘墨車建旗;太常卿、御史大夫、兵部尚書乘夏縵,司徒乘夏篆,並建旃。所以參備九旗之制。」詔從之。



 政和議禮局上王公以下車制:象輅以象飾諸末,朱班輪,八鸞在衡,左建旗,右載闟戟,駕馬四,親王昏則用之。革車,赤質,載闟戟,緋羅繡輪衣、簾、旗、韜杠、絡帶,駕赤馬四。大駕鹵簿六引,法駕鹵簿三引,開封牧第乘之。王公、一品、二品、三品備鹵簿,皆供革車一乘。其輪衣、簾、旗、韜杠、絡帶繡
 文:開封牧以隼,大司樂以鳳,少傅以瑞馬,御史大夫以獬豸,兵部尚書以虎。軺車,黑質,紫幰衣、絡帶並繡雉,施紅錦簾,香爐、香寶結帶,駕赤馬二。鹵簿內第一引官縣令乘之,駕馬皆有銅面,插羽,鞶纓,攀胸鈴拂,緋絹屜,紅錦包尾。



 六年,禮制局言:



 大觀中,用大司樂代太常卿為第三引,蓋以大司樂掌鼓吹之事。夫禮樂之官,宗伯為長,宜改用禮部尚書。又第四引司徒,即用地官之長,自漢以來為三公。朝廷近改司徒為少傅,然六引司徒乃
 地官之事,宜改用戶部尚書。其府佐依六引諸卿例,改為僚佐,其鹵簿儀仗,依兵部尚書例給。



 古之諸侯出封於外,同姓錫以金輅,異姓錫以象輅。蓋出而制節,則遠君而其道伸;入而謹度,則近君而其勢屈。故其入覲,則不敢乘金輅、象輅,以同於王,當自降而乘墨車也。若公侯採地在天子縣內者,則為都鄙之長,《大司馬》所謂「師都建旃「是矣。今開封牧列職於朝,與御史大夫同謂之卿可也,其在《周官》,則卿大夫之職是矣;又無金輅、象輅
 之錫,而乃比於古之諸侯入覲而乘墨車,可乎?



 成周上公九命,車旗以九為節,故建常九斿;侯、伯七命,車旗以七為節,故建常七斿;子、男五命,車旗以五為節,故建常五斿;其卿六命,其大夫四命,車旗亦各視其命之數。則卿之建旃當用六斿,大夫建物當用四斿,至於三斿則上士所建也。其開封令,宜乘墨車而建物四斿;開封牧、御史大夫、戶部兵部禮部尚書皆卿也,宜乘夏縵而建旃六斿。



 其年,詳定官蔡攸又言:



 六引,開封令乘軺車居
 前,開封牧、大司樂、司徒、御史大夫、兵部尚書乘革車次之。開封牧建繡隼旗,太常卿建繡鳳旗,司徒繡瑞馬旗,御史大夫繡以獬豸,兵部尚書繡以虎,皆副之以闟戟。其先後之序,所乘之車,所建之旗,揆古則不合,驗今則有戾。且大駕之出,自漢光武時始有三引:先河南尹,次執金吾,次洛陽令,先尊而後卑也。後魏亦三引:先平城令,次司隸校尉,次丞相,先卑而後尊也。唐兼用六引,五代減為三,後周復增為六。本朝因之,以開封令居前,終
 以兵部尚書。然以前為尊,則大司樂不當次令、牧;以後為尊,則兵部尚書不當繼御史大夫,此先後之序未正也。



 軺車非縣令宜駕,革車非公卿宜用,是所乘之車未稱也。鳳馬之繡,無所經見,闟戟之設,尤為訛謬,是所建之旗未宜也。司徒,三公論道之官,車徒非其所任,戶部主之可也。奉常掌禮,司樂典樂,皆專於一事,禮樂之容,非其所兼,禮部總之宜也。請改司徒用戶部尚書,改大司樂用禮部尚書,其僚佐儀制視兵部尚書。御史大夫,
 位亞三少,秩從二品,又尊於六尚書。其行,宜以兵部次令、牧,禮部、戶部又次之,終以御史大夫,則先後之序正矣。



 夏篆者,篆其車而五採畫之也,夏縵則五採畫之而不篆,墨車則漆之而不畫。孤宜乘夏篆,像其文質之備;卿宜乘夏縵,像其文採而不足於篆。開封令秩比大夫,開封牧古之諸侯,其乘皆宜墨車。其駕之馬,令以三,牧以四,御史大夫以六,尚書,卿之任也,其駕亦四,則所乘之車稱矣。《司常》曰:「孤、卿建旃,大夫、士建物,師都建旗。」蓋
 通帛為旃,其色純赤;雜帛為物,其色赤白;物為三斿,旃亦如之。開封令秩視大夫,故宜建以物;開封牧率王畿之眾而衛上,師都之任也,故宜建以旗;尚書、御史大夫,古之卿也,故宜建以旃。從之。



 七年,禮制局言:「昨討論大駕六引,開封牧乘墨車,兵部尚書、禮部尚書、戶部尚書、御史大夫乘夏縵。已經冬祀陳設訖,所有駕士衣服,尚循舊六引之制,宜行改正,況天子五輅,駕士之服,各隨其輅之色,則六引駕士之服,當亦如之。請墨車駕士衣
 皂,夏縵駕士皂質繡五色團花,於禮為稱。」從之。



 肩輿。神宗優待宗室老疾不能騎者,出入聽肩輿。熙寧五年,太宗正司請宗室以病肩輿者,踏引、籠燭不得過兩對。中興後,人臣無乘車之制,從祀則以馬,常朝則以轎。舊制,輿簷有禁。中興東征西伐,以道路阻險,詔許百官乘轎,王公以下通乘之。其制:正方,飾有黃、黑二等,凸蓋無梁,以篾席為障,左右設牖,前施簾,舁以長竿二,名曰竹轎子,亦曰竹輿。



 內外命婦之車。唐制有厭翟車、翟車、安車、白銅飾犢車,而幰網有降差。宋制,銀裝白藤輿簷,內命婦皇親所乘;白藤輿簷、金銅犢車、漆犢車,或覆以氈,或覆以棕,內外命婦通乘。



 傘。人臣通用,以青絹為之。宋初,京城內獨親王得用。太宗太平興國中,宰相、樞密使始用之。其後,近臣及內命婦出入皆用。真宗大中祥符五年,詔除宗室外,其餘悉禁。明年,復許中書、樞密院用焉。京城外,則庶官通用。神
 宗熙寧之制,非品官禁用青蓋,京城惟執政官及宗室許用。哲宗紹聖二年,詔在京官不得用涼扇。徽宗政和三年,以燕、越二王出入,百官不避,乃賜三接青羅傘一,紫羅大掌扇二,塗金花鞍韉,茶燎等物皆用塗金,遂為故事。八年,詔民庶享神,不得造紅黃傘、扇及彩繪,以為祀神之物。宣和初,又詔諸路奉天神,許用紅黃傘、扇,餘祠廟並禁。其畫壁、塑像儀仗用龍飾者易之。建炎中,初駐蹕杭州,執政張澄言:「群臣扈從兵間,權免張蓋,俟回
 鑾仍舊。」詔前宰相到闕,許張蓋。



 鞍勒之制。宋以賜群臣,其非賜者皆有令式,而不敢逾越焉。金塗銀鬧裝牡丹花校具八十兩,紫羅繡寶相花雉子方韉,油畫鞍,白銀銜鐙,以賜宰相,親王,樞密使帶使相,曾任宰相觀文殿大學士宮觀使,殿前馬軍步軍都指揮使。金塗銀鬧裝太平花校具七十兩,紫羅繡瑞草方韉,油畫鞍,陷銀銜鐙,以賜使相,樞密副使,參和政事,宣徽使,節度使,宮觀使,殿前馬軍步軍副都指揮使、都虞候。



 四廂都指揮使,韉以紫羅剜花。



 若出使,則加紅犛牛纓,金塗銀鈸。使相在外,加紅織成鞍衣復。步軍都虞候以上賜帶甲馬者,加紅皮秋轡校具七十兩,青氈圓韉,陷銀銜鐙。



 金塗銀鬧裝麻葉校具五十兩,紫羅剜花方韉,油畫鞍,陷銀銜鐙,以賜三司使,觀文殿學士,資政殿大學士,翰林學士承旨,翰林學士,資政殿、端明殿、翰林侍讀侍講,龍圖、天章、寶文閣、樞密直學士,御史中丞,兩使留後,觀察、防禦使,軍廂都指揮使。



 軍廂都指揮使初出授團練使、刺史者,賜亦同。曾任中書、樞密院後為學士、中丞者,七十兩,韉以繡瑞草。



 見任中書、樞密院、宣徽使、使相、節度使出
 使,曾任中書、樞密院充諸路都總管、安撫使,朝辭日,賜亦如之。金塗銀三環寶相花校具二十五兩,紫羅圓韉,烏漆鞍,銜鐙,以賜團練使、刺史。金塗銀促結洛州花校具三十兩,紫羅圓韉,以賜諸路承受。白成十五兩,以賜諸王宮僚、翰林侍讀侍書;金塗銀寶相花校具四十兩,蠻雲校具十五兩,以賜諸班押班、殿前指揮使以上;白成窪面校具十二兩,以賜諸班,皆藍黃絁圓韉。



 其皇親婚嫁,皆給藍黃羅繡方韉,金塗銀花鞍,金塗銀校具自
 八十兩至十二兩,有六等。宗室女婿系親,皆賜紫羅繡瑞草方韉,校具自七十兩至五十兩,有二等。其賜契丹使,則金塗銀太平花校具七十兩,紫羅繡寶相花雉子方韉;副使則槲葉校具五十兩,紫羅繡合子地圓韉,皆油畫鞍。



 射弓則使銀裝,副使銀棱。



 賜諸蕃進奉大使,則如刺史而用青絳韉;副使則如宮僚。凡京官三品以上外任者,皆許馬以纓飾。



 太宗太平興國七年,翰林學士承旨李昉言:「準詔詳定車服制度,請升朝官許乘銀裝絳子鞍勒,六
 品以下不得鬧裝,其韉皆不得刺繡、金皮飾。餘官及工商庶人,許並乘烏漆素鞍,不得用狨毛暖坐。其藍黃絳子,非宮禁不得乘。士庶、軍校乘白皮韉勒者,悉禁斷。」從之。八年,詔京朝知錄事參軍及知縣者,所乘馬並不得飾纓,後復許帶纓。端拱二年,詔內職諸班押班、禁軍指揮使、廂軍都虞候,並許乘銀裝絳子鞍勒。京官任知州、通判,許依六品朝官。真宗咸平二年,西京留臺上言:「留府群官、使臣乘馬,不得帶纓。」從之。大中祥符五年,詔繡
 韉及鬧裝校具,除宗室及恩賜外,悉禁。天禧元年,令兩省諫舍、宗室將軍以上,許乘狨毛暖坐,餘悉禁。凡京官,三班已上外任者,皆許馬以纓飾。



 仁宗景祐三年,詔官非五品以上,毋得乘鬧裝銀鞍,其乘金塗銀裝絳子促結鞍轡者,自文武升朝官及內職、禁軍指揮使、諸班押班、廂軍都虞候、防團副使以上,聽之;仍毋得以藍黃為絳、白皮為韉轡。民庶止許以氈皮絁紬為韉。京官為通判以上職任者,許權依升朝例。神宗熙寧間,文武升
 朝官、禁軍都指揮使以上,塗金銀裝盤絳促結;五品以上,復許銀鞍鬧裝。若開花繡韉,惟恩賜乃得乘。餘官及民庶,仍禁銀飾。舊制,諸王視宰相,用繡鞍韉。政和三年,始賜金花鞍韉,諸王不施狨坐。宣和末始賜,中興因之。乾道九年,重修儀制。權侍郎、太中大夫以上及學士、待制,經恩賜,許乘狨坐。三衙、節度使曾任執政官,亦如之。先是,建炎初,駐蹕杭州,詔扈從臣僚合設狨坐者,權宜撤去。故事,宰執、侍從自八月朔搭坐。紹興元年,以江、浙地
 燠,改為九月朔,著為例。乾道元年,乃詔三衙乘馬,賜狨坐。



 門戟。木為之而無刃,門設架而列之,謂之棨戟。天子宮殿門左右各十二,應天數也。宗廟門亦如之。國學、文宣王廟、武成王廟亦賜焉,惟武成王廟左右各八。臣下則諸州公門設焉,私門則府第恩賜者許之。太宗淳化二年,詔諸道州、府、軍、監奏乞鼓角戟槊,如令文合賜,即下三司指揮。仁宗天聖四年,太常禮院言:「準批狀,詳定知
 廣安軍範宗古奏,本軍乞降槊。檢會令文,京兆河南太原府、大都督府、都護門十四戟,若中都督、上都護門十二戟,下都督、諸州門各十戟,並官給。所有軍、監門不載,伏請不行。」神宗元豐之制,凡門列戟者,官司則開封、河南、應天、大名、大都督府皆十四,中都督皆十二,下都督皆十。品官恩賜者,正一品十六,二品以上十四。中興仍舊制。



 旌節。唐天寶中置,節度使受命日賜之,得以專制軍事,
 行即建節,府樹六纛。宋凡命節度使,有司給門旗二,龍、虎各一,旌一,節一,麾槍二,豹尾二。旗以紅繒九幅,上設耀篦、鐵鉆、髹杠、緋纛。旌用塗金銅螭頭,髹杠,綢以紅繒,畫白虎,頂設髹木盤,周用塗金飾。節亦用髹杠,飾以金塗銅葉,上設髹圓盤三層,以紅綠裝釘為旄,並綢以紫綾衣復囊,又加碧油絹袋。麾槍設髹木盤,綢以紫繒衣復囊,又加碧油絹袋。豹尾,制以赤黃布,畫豹文,並髹杠。



 神宗熙寧五年,詔新建節並移鎮,並降敕太常寺排比旌節,
 下左右金吾街仗司、騏驥院,給執擎人員、鞍馬。中興因之。建炎三年,表韓世忠之旗曰「忠勇」。紹興三年,表岳飛之旗曰「精忠」。孝宗詔以其藩邸旌節,迎置天章閣。淳熙中,光宗亦詔奉東宮旌節。其後,寧宗踐祚,有司言安奉皇帝藩邸旌節,宜有推飾。今用朱漆青地金字牌二:其一題曰「太上皇帝藩邸旌節」,其一曰「今上皇帝藩邸旌節。」蓋襲用元豐延安故事云。



\end{pinyinscope}