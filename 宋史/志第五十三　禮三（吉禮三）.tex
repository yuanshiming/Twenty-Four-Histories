\article{志第五十三 禮三(吉禮三)}

\begin{pinyinscope}

 北
 郊祈穀五方帝感生帝



 北郊。宋初,方丘在宮城之北十四里,以夏至祭皇地祇。別為壇於北郊,以孟冬祭神州地祇。建隆以來,迭奉四
 祖崇配二壇。太平興國以後,但以宣祖、太祖更配。真宗乃以太宗配方丘,宣祖配神州地祇。皇祐初,禮官言:「皇地祇壇四角再成,面廣四丈九尺,東西四丈六尺。上成高四尺五寸,下成高五尺,方五丈三尺,陛廣三尺五寸,卑陋不應典禮。請如唐制增廣之。」五年,諸壇皆改。嘉祐配位七十一,加羊、豕各五。慶歷用犢、羊、豕各一。既而諫官司馬光奏:「大行請謚於南郊,而皇地祇止於望告,失尊卑之序。」下禮院,定非次祭告皇地祇,請差官詣北郊行
 事。其神州之壇,方三丈一尺,皇祐增高三尺,廣四十八步,內壝四面以青繩代之。仍遣內臣降香,有司攝事如儀。



 神宗元豐元年二月,郊廟奉祀禮文所言:「古者祀天於地上之圜丘,在國之南,祭地於澤中之方丘,在國之北,其牲幣禮樂亦皆不同,所以順陰陽、因高下而事之以其類也。由漢以來,乃有夫婦共牢,合祭天地之說,殆非所謂求神以類之意。本朝親祀上帝,即設皇地祇位,稽之典禮,有所未合。」遂詔詳定更改以聞。於是陳襄、王
 存、李清臣、張璪、黃履、陸佃、何洵直、楊完等議,或以當郊之歲,冬夏至日分祭南北郊,各一日而祀遍;或於圜丘之旁,別營方丘而望祭;或以夏至盛暑,天子不可親祭,改用十月;或欲親郊圜丘之歲,夏至日遣上公攝事於方丘,議久未決。



 三年,翰林學士張璪言:「先王順陰陽之義,以冬至祀天,夏至祀地,此萬世不可易之理。議者乃欲改用他月,無所據依。必不得已,宜即郊祀之歲,於夏至之日,盛禮容,具樂舞,遣塚宰攝事。雖未能皆當於禮,
 庶幾先王之遺意猶存焉。」於是禮官請如璪議,設宮架樂、文武二舞,改制樂章,用竹冊匏爵,增配帝犢及捧俎分獻官,廣壇壝齋宮,修儀注上之。既而曾肇言:「今冬至若罷合祭,則夏至又以有司攝事,則不復有親祭地祇之時,於父天母地之義若有隆殺。請遇親祀南郊之歲,以夏至日備禮躬款北郊,以存事地之義。」四年四月,乃詔:「親祀北郊,並依南郊之儀,有故不行,即以上公攝事。」六年,禮部、太常寺上親祀儀並如南郊;其攝事唯改
 舞名及不備官,其籩豆、樂架、玉幣之數,盡如親祠。是歲十一月甲辰冬至,祀昊天上帝,以太祖配,始罷合祭,不設皇地祇位。



 哲宗初立,未遑親祀,有司攝事如元豐儀。元祐五年夏至,祭皇地祇,命尚書右丞許將攝事。將言:「王者父天母地,三歲冬至,天子親祠,遍享宗廟,祀天圜丘,而夏至方澤之祭,乃止遣上公,則皇地祇遂永不在親祠之典,此大闕禮也。望博詔儒臣,講求典故,明正祀典,為萬世法。」禮部尚書趙彥若請依元豐所定,郊祀之
 歲,親祀方丘及攝事,已合禮之正,更不須聚議。禮部郎中崔公度請用陳薦議,仍合祭天地,從祀百神。復詔尚書、侍郎、兩省及侍從、臺諫、禮官集議。於是翰林學士顧臨等八人,請合祭如故事,俟將來親祠北郊,則合祭可罷。宋興,一祖六宗,皆合祭天地,其不合祭者,惟元豐六年一郊爾。去所易而就所難,虛地祇之大祭,失今不定,後必悔之。吏部侍郎范純禮等二十二人,皆主北郊之議。中書舍人孔武仲又請以孟冬純陰之月,詣北郊親
 祠,如神州地祇之祭。彭汝礪、曾肇復上疏論合祭之非。文多不載。



 九月,三省上顧臨等議。太皇太后曰:「宜依仁宗皇帝故事。」呂大防言:「諸儒獻議,欲南郊不設皇地祇位,於祖宗之制未睹其可。」範百祿以「圜丘無祭地之禮,《記》曰:『有其廢之,莫可舉也。』先帝所廢,稽古據經,未可輕改。」大防又言:「先帝因禮文所建議,遂令諸儒定北郊祀地之禮,然未經親行。今皇帝臨御之始,當親見天地,而獨不設地祇位,恐亦未安。況祖宗以恩霈四方,慶賚將
 士,非三歲一行,則國力有限。今日宜為勉行權制,俟北郊議定及太廟享禮,行之未晚。」太皇太后以大防之言為是。而蘇頌、鄭雍皆以「古者人君嗣位之初,必郊見天地。今皇帝初郊而不祀地,恐未合古。」乃下詔曰:「國家郊廟特祀,祖宗以來命官攝事,惟三歲一親郊,則先享清廟,冬至合祭天地於圜丘。元豐間,有司援周制,以合祭不應古義,先帝乃詔定親祀北郊之儀,未之及行。是歲,郊祀不設皇地祇位,而宗廟之享率如權制。朕方修郊
 見天地之始,其冬至日南郊,宜依熙寧十年故事,設皇地祇位以嚴並況之儀。厥後躬行方澤之祀,則修元豐六年五月之制。俟郊禮畢,集官詳議典禮以聞。」十一月冬至,親祠南郊,遂合祭天地,而詔罷飲福宴。



 八年,禮部尚書蘇軾復陳合祭六議,令禮官集議以聞。已而下詔依元祐七年故事,合祭天地於南郊,仍罷集議。紹聖元年,以右正言張商英言:「先帝制詳定禮文所,謂合祭非古,據經而正之。元祐之臣,乃復行合祭,請再下禮官議。」
 御史中丞黃履謂:「南郊合祭,因王莽諂事元後,遂躋地位,同席共牢。迨先帝親郊,大臣以宣仁同政,復用莽意合祀,瀆亂典禮。」帝以詢輔臣,章惇曰:「北郊止可謂之社。」黃履曰:「郊者,交於神明之義,所以天地皆稱郊。社者,土之神爾,豈有祭大祇亦可謂之壯乎?」乃以履奏送禮部、太常寺。權禮部侍郎盛陶、太常丞王誼等言:「宜用先帝北郊儀注,以時躬行,罷合祭禮。」已而三省言:「合祭既非禮典,但盛夏祭地祇,必難親行。」詔令兩省、臺諫、禮官同
 議,可親祀北郊,然後罷合祭之禮。曾布、錢勰、范純禮、韓宗師、王古、井亮採、常安民、李琮、虞策、劉定、傅楫、黃裳、豐稷、葉祖洽等言,互有是否。蔡京、林希、蔡六、黃履、吳安持、晁端彥、翟思、郭知章、劉拯、黃慶基、董敦逸等請罷合祭。詔從之。然北郊親祀,終帝之世未克舉云。



 建中靖國元年,命禮部、太常寺詳定北郊儀制。殿中侍御史彭汝霖又請改合祭之禮,韓忠彥以為不可。曾布力主北郊之說,帝亦然之,遂罷合祭。



 政和三年,詔禮制局議方壇制
 度。是歲,新壇成。初,元豐三年七月,詔改北郊圜壇為方丘。六年,命禮部、太常定北郊壇制。哲宗紹聖三年,權尚書侍郎黃裳等言:「南郊青城至壇所五百一十八步,自瑞聖園至皇地祇壇之東壇五百五十六步,相去不遠。其壇系國初所建,神靈顧享已久。元豐間,有司請地祇、神州並為方壇,壇之外為坎,詔止改圜壇為方。請下有司,比類南郊增飾制度,除治四面,稍令低下,以應澤中之制。」詔禮部再為詳定,指畫興築。至是,禮制局言:「方壇
 舊制三成,第一成高三尺,第二成、第三成皆高二尺五寸,上廣八丈,下廣十有六丈。夫圜壇既則象於乾,則方壇當效法於坤。今議方壇定為再成,一成廣三十六丈,再成廣二十四丈,每成崇十有八尺,積三十六尺,其廣與崇皆得六六之數,以坤用六故也。為四陛,陛為級一百四十有四,所謂坤之策百四十有四者也。為再壝,壝二十有四步,取坤之策二十有四也。成與壝俱再,則兩地之義也。」齋宮大內門曰廣禋,東偏門曰東秩,西偏門
 曰西平,正東門曰含光,正西門曰咸亨,正北門曰至順,南內大殿門曰厚德,東曰左景華,西曰右景華,正殿曰厚德,便殿曰受福、曰坤珍、曰道光,亭曰承休,後又增四角樓為定式。



 其神位,崇寧初,禮部員外郎陳暘言:「五行於四時,有帝以為之主,必有神以為之佐。今五行之帝既從享於南郊第一成,則五行之神亦當列於北郊第一成。天莫尊於上帝,而五帝次之;地莫尊於大祇,而岳帝次之,今尚與四鎮、海瀆並列,請升之於第一成。」至是,
 議禮局上《新儀》:皇地祇位於壇上北方南向,席以稿秸;太祖皇帝位於壇上東方西向,席以蒲越。木神勾芒、東嶽於壇第一龕,東鎮、海瀆於第二龕,東山林、川澤於壇下,東丘陵、墳衍、原隰於內壝之內,皆在卯階之北,以南為上。神州地祇、火神祝融、南嶽於壇第一龕,南鎮、海瀆於第二龕,南山林、川澤於壇下,南丘陵、墳衍、原隰於內壝之內,皆在午階之東,以西為上。土神后土、中嶽於壇第一龕,中鎮於第二龕,中山林、川澤於壇下,中丘陵、墳
 衍、原隰於內壝之內,皆在午階之西,以西為上。金神蓐收、西嶽於壇第一龕,西鎮、海瀆於第二龕,昆侖西山林、川澤於壇下,西丘陵、墳衍、原隰於內壝之內,皆在酉階之南,以北為上。水神玄冥、北嶽於壇第一龕,北鎮、海瀆於第二龕,北山林、川澤於壇下,北丘陵、墳衍、原隰於內壝之內,皆在子階之西,以東為上。神州地祇席以稿秸,餘以莞席,皆內向。其餘並如元豐儀壇壝之制。其位板之制,上帝位板長三尺,取參天之數;厚九寸,取乾元用
 九之數;廣尺二寸,取天之備數;書徽號以蒼色,取蒼璧之義。皇地祇位板長二尺,取兩地之數;厚六寸,取坤元用六之數;廣一尺,取地之成數;書徽號以黃色,取黃琮之義。皆以金飾。配位板各如天地之制。



 又言:「《大禮格》,皇地祇玉用黃琮,神州地祇、五岳以兩圭有邸。今請二者並施於皇地祇,求神以黃琮,薦獻以兩圭有邸。神州惟用圭邸,餘不用。玉琮之制,當用坤數,宜廣六寸,為八方而不剡;兩圭之長宜共五寸,並宿一邸,色與琮同。牲幣
 如之。」又言:「常祭,地祗、配位各用冰鑒一;今親祀,盛暑,請增正配及從祀位冰鑒四十一。」並從之。



 四年五月夏至,親祭地於方澤,以皇弟燕王俁為亞獻,趙王人思為終獻。皇帝散齋七日於別殿,致齋七日於內殿,一日於齋宮。前一日告配太祖室,其有司陳設及皇帝行事,並如郊祀之儀。是後七年,至宣和二年、五年,親祀者凡四。



 高宗紹興初,惟用酒脯鹿MZ,行一獻禮。二年,太常少卿程瑀言:「皇地祇,當一依祀天儀式。」詔從之。又言:「國朝祀皇地
 祇,設位於壇之北方南向。政和四年,設於南方北向。今北面望祭,北向為難,且於經典無據。請仍南向。」



 淳熙中,朱熹為先朝南北郊之辯曰:「《禮》『郊特牲而社稷太牢』,《書》『用牲於郊,牛二』及『社於新邑』,此明驗也。本朝初分南北郊,後復合而為一。《周禮》亦只說祀昊天上帝,不說祀后土,故先儒言無北郊,祭社即是祭地。古者天地未必合祭,日月、山川、百神亦無一時合祭共享之禮。古之時,禮數簡而儀從省,必是天子躬親行事,豈有祭天卻將上
 下百神重沓累積並作一祭耶?且郊壇陛級兩邊上下,皆是神位,中間恐不可行。或問:郊祀后稷以配天,宗祀文王以配上帝,帝即是天,天即是帝,卻分祭,何也?曰:為壇而祭,故謂之天,祭於屋下而以神祇祭之,故謂之帝。」



 祈穀、雩祀。宋之祀天者凡四:孟春祈穀,孟夏大雩,皆於圜丘或別立壇。季秋大饗明堂。惟冬至之郊,則三歲一舉,合祭天地焉。開寶中,太祖幸西京,以四月有事南郊,躬行大雩之禮。淳化、至道,太宗亦以正月躬行祈穀之
 祀,悉如圜丘之禮。



 景德三年,龍圖閣待制陳彭年言:「伏睹畫日,來年正月三日上辛祈穀,至十日始立春。按《月令》,正月元日注為祈穀,郊祀昊天上帝。《春秋傳》曰:『啟蟄而郊,郊而後耕。』蓋春氣初至,農事方興,郊祀昊天,以祈嘉穀,當在建寅之月,迎春之後。自晉泰始二年,始用上辛,不擇立春之先後。齊永明元年,立春前郊,議欲遷日,王儉曰:『宋景平元年、元嘉六年並立春前郊。』遂不遷日。吳操之云:『應在立春前。』然則左氏所記,乃三代彞章;王
 儉所言,乃後世變禮。來年正月十日立春,三日祈穀,斯則襲王儉之末議,違左氏之明文。望以立春後上辛行祈穀禮。」因詔有司詳定諸祠祭祀。有司言:「今年四月五日,雩祀上帝,十三日立夏祀赤帝。按《月令》:『立夏之日,天子迎夏於南郊。』《注》云:『為祀赤帝於南郊。』又云:『是月也,大雩。』《注》云:『《春秋傳》曰:龍見而雩。』龍星謂角、亢也,立夏後,昏見於東方。按《五禮精義》云:『自周以來,歲星差度,今之龍見或在五月,以祈甘雨,於時已晚,但四月上旬卜日。』今
 則惟用改朔,不待得節,祭於立夏之前,殊違舊禮之意。茍或龍見於仲夏,雩祀於季春,相去遼闊,於禮未周。欲請並於立夏后卜日,如立夏在三月,則待改朔。」



 天禧元年十二月,禮儀院言:「準畫日,來年正月十七日祈穀,前二日奏告太祖室,緣歲以正月十五日朝拜玉清昭應宮,景德四年以前,祈谷止用上辛,其後用立春後辛日,蓋當時未有朝拜宮觀禮。王儉啟云:『近代明例,不以先郊後春為嫌。』又宋孝武朝有司奏『魏代郊天值雨,更用後
 辛』,或正月上辛,事有相妨,並許互用,在於禮典,固亦無嫌。」



 初,祈穀、大雩,皆親祀上帝。由熙寧迄靖康,惟有司攝事而已。元豐中,禮官言:「慶歷大雩宗祀之儀,皆用犢、羊、豕各一,唯祈穀均祀昊天上帝止用犢一。請依雩祀、大享明堂牲牢儀,用犢、羊、豕各一。」



 四年十月,詳定郊廟奉祀禮文所言:「近詔宗祀明堂以配上帝,其餘從祀群神悉罷。今祈穀、大雩猶循舊制,皆群神從祀,恐與詔旨相戾。請孟春祈穀、孟夏大雩,惟祀上帝,以太宗皇帝配,餘
 從祀群神悉罷。」又請改築雩壇於國南門,以嚴祀事。並從之。



 五年七月,禮部言:「雩壇當立於圜丘之左巳地,其高一丈,廣輪四丈,周十二丈,四出陛,為三壝,各二十五步,周垣四門,一如郊壇之制。」從之。大觀四年二月,禮局議以立春後上辛祈穀,詔:「以今歲孟春上辛在丑,次辛在亥,遇醜不祈而祈於亥,非禮也。」乃不果行。



 政和《祈穀儀》:前期降御札,以來年正月上辛祈穀,祀上帝。前祀十日,太宰讀誓於朝堂,刑部尚書蒞之;少宰讀誓於太
 廟齋房,刑部侍郎蒞之。皇帝散齋七日,致齋三日。前祀一日,服通天冠、絳紗袍,乘玉輅,詣青城。祀日,自齋殿服通天冠、絳紗袍,乘輿至大次,服袞冕,執圭,入正門,宮架《儀安》之樂作。禮儀使奏請行事,宮架作《景安》之樂,《帝臨降康》之舞六成,止。太常升煙,禮儀使奉請再拜。盥洗,升壇上,登歌《嘉安》之樂作。皇帝搢大圭,執鎮圭,詣上帝神位前,北向,奠鎮圭於繅藉,執大圭,俯伏,興。又奏請搢大圭,跪,受玉幣。尊訖,詣太宗神位前,東向,尊幣如上儀,登歌
 作《仁安》之樂。皇帝降階,有司進熟,禮儀使奏請執大圭,升壇,登歌《歆安》之樂作。皇帝詣上帝神位前酌獻,執爵祭酒,讀冊文訖,奏請皇帝再拜。詣太宗神位前酌獻,並如上儀,登歌作《紹安》之樂。皇帝降階,入小次,文舞退,武舞進,宮架《容安》之樂作。亞獻酌獻,宮架作《隆安》之樂,《神保錫羨》之舞。終獻如之。禮儀使奏請皇帝詣飲福位,宮架《禧安》之樂作。皇帝受爵。又請再拜。有司徹俎,登歌《成安》之樂作。送神,宮架《景安》之樂作。皇帝詣望燎位。禮畢,
 還大次。雩祀上帝儀亦如之。惟太宗神位奠幣作《獻安》之樂,酌獻作《感安之樂》。



 南渡後,以四祀二在南郊圜壇,二在城西惠照院齋宮。紹興十四年始具樂舞,用政和儀,增籩豆之數。乾道五年,太常少卿林慄乞四祭並即圜壇,禮部侍郎鄭聞謂:「明堂當從屋祭,不當在壇。有司攝事,當於望祭殿行禮。」從之。淳熙十六年,光宗受禪,始奉高宗配焉。



 五方帝。宋因前代之制,冬至祀昊天上帝於圜丘,以五
 方帝、日、月、五星以下諸神從祀。又以四郊迎氣及土王日專祀五方帝,以五人帝配,五官、三辰、七宿從祀。各建壇於國門之外:青帝之壇,其崇七尺,方六步四尺;赤帝之壇,其崇六尺,東西六步三尺,南北六步二尺;黃帝之壇,其崇四尺,方七步;白帝之壇,其崇七尺,方七步;黑帝之壇,其崇五尺,方三步七尺。天聖中,詔太常葺四郊宮,少府監遣吏繼祭服就給祠官,光祿進胙,監祭封題。慶歷用羊、豕各一,正位大尊、著尊各二,不用犧尊,增山罍
 為二,壇上簠、簋、俎各增為二。皇祐定壇如唐《郊祀錄》,各廣四丈,其崇用五行八七五九六為尺數。嘉祐加羊、豕各二。



 元祐六年,知開封府範百祿言:「每歲迎氣於四郊,祀五帝,配以五神,國之大祠也。古者天子皆親帥三公、九卿、諸侯、大夫以虔恭重事,而導四時之和氣焉。今吏部所差三獻皆常參官,其餘執事贊相之人皆班品卑下,不得視中祠行事者之例。請下禮部與太常議,宜以公卿攝事。」從之。



 景德中,南郊鹵簿使王欽若言:「五方帝位
 板如靈威仰、赤熛怒、含樞紐、白招拒、葉光紀,恐是五帝之名,理當恭避。」禮官言:「《開寶通禮義纂》,五者皆是帝號。《漢書注》自有名,即蒼帝靈符,赤帝文祖,白帝顯紀,黑帝玄矩,黃帝神鬥是也。既為美稱,不煩回避。」嘉祐元年,以集賢校理丁諷言,按《春秋文耀勾》為五帝之名,始下太常去之。



 其祀儀:皇帝服袞冕,祀黑帝則服裘被袞。配位,登歌作《承安》之樂,餘並如祈穀禮。立春祀青帝,以帝太昊氏配,勾芒氏、歲星、三辰、七宿從祀。勾芒位壇下卯階之南,歲星、析木、大
 火、壽星位壇下子階之東,西上。角、亢、氐、房、心、尾、箕宿,位於壇下子階之西,東上。



 立夏祀赤帝,以帝神農氏配,祝融氏、熒惑、三辰、七宿從祀。祝融位壇下卯階之南,熒惑、鶉首、鶉火、鶉尾位子階之東,西上。井、鬼、柳、星、張、翼、軫宿,位於壇下子階之西,東上。



 季夏祀黃帝,以黃帝氏配,後土、鎮星從祀。後土位壇下卯階之南,鎮星位壇下子階之東。



 立秋祀白帝,以帝少昊氏配,蓐收、太白、三辰、七宿從祀。蓐收位壇下卯階之南,太白、大梁、降婁、實沉位壇下子階之東,西上。奎、婁、胃、昴、畢、觜、參宿,位於子階之西,東上。



 立冬祀黑帝,以帝高陽氏配,玄冥、辰星、三辰、七宿從祀。玄冥位壇下卯階之南,辰星、諏訾、玄枵、星紀位子階之東,西上。斗、牛、女、虛、危、室、壁宿,位子階之西,東上。



 紹興
 仍舊制,祀五帝於郊。



 感生帝,即五帝之一也。帝王之興,必感其一。北齊、隋、唐皆祀之,而隋、唐以祖考升配,宋因其制。乾德元年,太常博士聶崇義言:「皇帝以火德上承正統,請奉赤帝為感生帝。每歲正月,別壇而祭,以符火德。」事下尚書省集議,請如崇義奏。乃酌隋制,為壇於南郊,高七尺,廣四丈,日用上辛,配以宣祖。牲用騂犢二,玉用四圭,有邸,幣如方色。明年正月,有司言:「上辛祀昊天上帝,五方帝從祀。今
 既奉赤帝為感生帝,一日之內,兩處俱祀,似為煩數。況同時並祀,大禮非宜。昊天從祀,請不設赤帝坐。」從之。



 乾興元年九月,太常丞同判禮院謝絳言:「伏睹本院與崇文院檢討官詳定,以宣祖配感生帝。竊尋宣祖非受命開統,義或未安。唐武德初,圜丘、方丘、雩祀並以景帝配,祈穀、大享並以元帝配。太宗初,奉高祖配圜丘、明堂、北郊,元帝配感生帝。高宗永徽二年,祀高祖於圜丘,祀太宗於明堂,兼感生帝作主。又以景帝、元帝稱祖,萬代不
 遷,停配以符古義。臣以為景帝厥初受封為唐始祖,蓋與宣祖不侔。宣祖於唐,是為元帝之比。唐有天下,裁越三世,而景、元二祖已停配典。有宋受命,既自太祖,於今四聖,而宣祖侑祀未停,恐非往典之意。請依永徽故事,停宣祖配,仍用太宗故事,宗祀真宗於明堂,兼感生帝作主。若據鄭氏說,則曰五帝迭王,王者因所感別祭,尊於南郊,以祖配之。今若不用武德、永徽故事,請以太祖兼配,正符鄭說。詳鄭之意,非受命始封之祖不得配,故
 引周后稷配靈威仰之義為證。惟太祖始造基業,躬受符命,配侑感帝,據理甚明。如恐祠日相妨,當以太宗配祈穀,太祖配雩祀,亦不失尊嚴之旨。臣以為宣廟非惟不遷,而迭用配帝,於古為疑。《禮》:『祖有功,宗有德。』但非受命之祖,親盡必毀,況配享乎?」



 翰林承旨李維等議:「按《禮·祭法》正義曰:『郊,謂夏正建寅之月,祭感生帝於南郊。』此則崇配之文也。竊惟感帝比祈穀,禮秩差輕;宣祖比太祖,功業有異。今以太祖配祈穀,宣祖配感帝,稱情立文,
 於禮斯協。」詔從所定。



 其祀儀:皇帝散齋七日,致齋三日。太史設帝位於壇上,北方南向,席以稿秸。配帝位於壇上,東方西向,席以蒲越。配位,奠幣,作《皇安》之樂,酌獻,作《肅安》之樂,餘如祈穀祀上帝儀。



 紹興十八年,臣僚言:「我朝祀赤帝為感生帝,世以僖祖配之。祖宗以來,奉事尤謹,故子孫眾多,與天無極。中興浸久,祀秩咸修。惟感生帝,有司因循,尚淹小祀,寓於招提,酒脯而已。宜詔有司升為大祀,庶幾天意潛孚,永錫蕃衍。」詔禮官議之,遂躋大
 祀。禮行三獻用籩豆十二,設登歌樂舞,望祭於齋宮。



\end{pinyinscope}