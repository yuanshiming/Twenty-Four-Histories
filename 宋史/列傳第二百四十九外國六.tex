\article{列傳第二百四十九外國六}

\begin{pinyinscope}

 ◎外國六○天竺于闐高昌回鶻大食層檀龜茲沙州拂菻



 天竺國舊名身毒,亦曰摩伽陀,復曰婆羅門。俗宗浮圖
 道,不飲酒食肉。漢武帝遣使十餘輩間出西南,指求身毒,為昆明所閉,莫能通。至漢明帝夢金人,於是遣使天竺問佛道法,由是其教傳於中國。梁武帝、後魏宣武時,皆來貢獻。隋煬帝誌通西域,諸國多有至者,唯天竺不通。唐貞觀以後,朝貢相繼。則天天授中,五天竺王並來朝獻。乾元末,河隴陷沒,遂不復至。周廣順三年,西天竺僧薩滿多等十六族來貢名馬。



 乾德三年,滄州僧道圓自西域還,得佛舍利一水晶器、貝葉梵經四十夾來獻。道
 圓晉天福中詣西域,在塗十二年,住五印度凡六年,五印度即天竺也;還經于闐,與其使偕至。太祖召問所曆風俗山川道里,一一能記。四年,僧行勤等一百五十七人詣闕上言,願至西域求佛書,許之。以其所曆甘、沙、伊、肅等州,焉耆、龜茲、于闐、割祿等國,又曆布路沙、加濕彌羅等國,並詔諭其國令人引導之。開寶後,天竺僧持梵夾來獻者不絕。八年冬,東印度王子穰結說囉來朝貢。



 天竺之法,國王死,太子襲位,餘子皆出家為僧,不復居
 本國。有曼殊室利者,乃其王子也,隨中國僧至焉,太祖令館於相國寺,善持律,為都人之所傾向,財施盈室。眾僧頗嫉之,以其不解唐言,即偽為奏求還本國,許之。詔既下,曼殊室利始大驚恨,眾僧諭以詔旨,不得已遲留數月而後去。自言詣南海附賈人船而歸,終不知所適。



 太平興國七年,益州僧光遠至自天竺,以其王沒徙曩表來上。上令天竺僧施護譯云:「近聞支那國內有大明王,至聖至明,威力自在。每慚薄幸,朝謁無由,遙望支那
 起居聖躬萬福。光遠來,蒙賜金剛吉祥無畏坐釋迦聖像袈裟一事,已披掛供養。伏願支那皇帝福慧圓滿,壽命延長,常為引導一切有情生死海中,渡諸沉溺。今以釋迦舍利附光遠上進。」又譯其國僧統表,詞意亦與沒徙曩同。



 施護者,烏塤曩國人。其國屬北印度,西行十二日至乾陀羅國,又西行二十日至曩誐囉賀囉國,又西行十日至嵐婆國,又西行十二日至誐惹曩國,又西行至波斯國,得西海。自北印度行百二十日至中印度。中
 印度西行三程至嗬囉尾國,又西行十二日至未曩囉國,又西行十二日至缽賴野迦國,又西行六十日至迦囉拿俱惹國,又西行二十日至摩囉尾國,又西行二十日至烏然泥國,又西行二十五日至囉囉國,又西行四十日至蘇囉茶國,又西行十一日至西海。自中印度行六月程至南印度,又西行九十日至供迦拿國,又西行一月至海。自南印度南行六月程得南海。皆施護之所述雲。



 八年,僧法遇自天竺取經回,至三佛齊,遇天竺僧
 彌摩羅失黎語不多命,附表願至中國譯經,上優詔召之。法遇後募緣製龍寶蓋袈裟,將復往天竺,表乞給所經諸國敕書,遂賜三佛齊國王遐至葛、古羅國主司馬佶芒、柯蘭國主讚怛羅、西天王子謨馱仙書以遣之。



 雍熙中,衛州僧辭瀚自西域還,與胡僧密坦羅奉北印度王及金剛坐王那爛陀書來。又有婆羅門僧永世與波斯外道阿裏煙同至京師。永世自云:本國名利得,國王姓牙羅五得,名阿喏你縛,衣黃衣,戴金冠,以七寶為飾。
 出乘象或肩輿,以音樂螺鈸前導,多遊佛寺,博施貧乏。其妃曰摩訶你,衣大縷金紅衣,歲一出,多所振施。人有冤抑,候王及妃出遊,即迎隨伸訴。署國相四人,庶務並委裁制。五穀、六畜、果實與中國無異。市易用銅錢,有文漫圓徑,如中國之制,但實其中心,不穿貫耳。其國東行經六月至大食國,又二月至西州,又三月至夏州。阿裏煙自云:本國王號黑衣,姓張,名哩沒,用錦彩為衣,每遊獵,三二日一還國。署大臣九人治國事。無錢貨,以雜
 物貿易。其國東行經六月至婆羅門。



 至道二年八月,有天竺僧隨舶至海岸,持帝鍾、鈴杵、銅鈴各一,佛像一軀,貝葉梵書一夾,與之語,不能曉。



 天聖二年九月,西印度僧愛賢、智信護等來獻梵經,各賜紫方袍、束帛。五年二月,僧法吉祥等五人以梵書來獻,賜紫方袍。景祐三年正月,僧善稱等九人貢梵經、佛骨及銅牙菩薩像,賜以束帛。



 于闐國,自漢至唐,皆入貢中國。安、史之亂,絕不復至。晉
 天福中,其王李聖天自稱唐之宗屬,遣使來貢。高祖命供奉官張匡鄴持節冊聖天為大寶于闐國王。



 建隆二年十二月,聖天遣使貢圭一,以玉為柙;玉枕一。本國摩尼師貢琉璃瓶二、胡錦一段。其使言:本國去京師九千九百里,西南抵蔥嶺與婆羅門接,相去三千餘里。南接吐蕃,西北至疏勒二千餘里。國城東有白玉河,西有綠玉河,次西有烏玉河,源出昆岡山,去國城西千三百里。每歲秋,國人取玉於河,謂之撈玉。土宜蒲萄,人多醞以為
 酒,甚美。俗事妖神。



 乾德三年五月,于闐僧善名、善法來朝,賜紫衣。其國宰相因善名等來,致書樞密使李崇矩,求通中國。太祖令崇矩以書及器幣報之。至是冬,沙門道圓自西域還,經于闐,與其朝貢使至。四年,又遣其子德從來貢方物。



 開寶二年,遣使直末山來貢,且言本國有玉一塊,凡二百三十七斤,願以上進,乞遣使取之。善名復至,貢阿魏子,賜號昭化大師,因令還取玉。又國王男總嘗貢玉欛刀,亦厚賜報之。四年,其國僧吉祥以其
 國王書來上,自言破疏勒國得舞象一,欲以為貢,詔許之。



 大中祥符二年,其國黑韓王遣回鶻羅廝溫等以方物來貢。廝溫跪奏曰:「臣萬里來朝,獲見天日,願聖人萬歲,與遠人作主。」上詢以在路幾時,去此幾裏。對曰:「涉道一年,晝行暮息,不知里數。昔時道路嘗有剽掠,今自瓜、沙抵于闐,道路清謐,行旅如流。願遣使安撫遠俗。」上曰:「路遠命使,益以勞費爾國。今降詔書,汝即齎往,亦與命使無異也。」



 初,太平興國中有澶州卒王貴者,晝忽見使
 者至營,急召貴偕行,南至河橋,驛馬已具,即命乘之,俄覺騰虛而去。頃之駐馬,但見屋室宏麗,使者引貴入,見其主者容衛制度悉如王者。謂貴曰:「俟汝年五十八,當往于闐國北通聖山取一異寶以奉皇帝,宜深誌之。」遂復乘馬淩虛而旋。軍中失貴已數日矣,驗所乘,即營卒之馬也。知州宋煦劾貴以聞,太宗釋之。天禧初,貴自陳年已五十八,願遵前戒,西至于闐,尋許其行。貴至秦州,以道遠悔懼,俄於市中遇一道士引貴出城,登高原,問
 貴所欲,具以實對。即命貴閉目,少頃令開,視山川頓異,道士曰:「此于闐國北境通聖山也。」復引貴觀一池,池中有仙童,出一物授之,謂曰:「持此奉皇帝。」又令瞑目,俄頃復至秦州,向之道士已失所在,發其物乃玉印也,文曰「國王趙萬永寶」,州以獻。



 天聖三年十二月,遣使羅麵於多、副使金三、監使安多、都監趙多來朝,貢玉鞍轡、白玉帶、胡錦、獨峰橐駝、乳香、硇砂。詔給還其直,館於都亭西驛,別賜襲衣、金帶、銀器百兩、衣著二百,羅麵於多金帶。



 嘉祐八年八月,遣使羅撒溫獻方物。十一月,以其國王為特進、歸忠保順後鱗黑韓王。羅撒溫言其王乞賜此號也,于闐謂金翅烏為「後鱗」,「黑韓」蓋可汗之訛也。羅撒溫等以獻物賜直少不受,及請所獻獨峰橐駝。詔以遠人特別賜錢五千貫,以橐駝還之,而與其已賜之直。其後數以方物來獻。



 熙寧以來,遠不逾一二歲,近則歲再至。所貢珠玉、珊瑚、翡翠、象牙、乳香、木香、琥珀、花蕊布、硇砂、龍鹽、西錦、玉秋轡馬、膃肭臍、金星石、水銀、安息雞舌
 香,有所持無表章,每賜以暈錦旋襴衣、金帶、器幣,宰相則盤球雲錦夾襴。



 地產乳香,來輒群負,私與商賈牟利;不售,則歸諸外府得善價,故其來益多。元豐初,始詔惟齎表及方物馬驢乃聽以詣闕,乳香無用不許貢。



 四年,遣部領阿辛上表稱「于闐國僂羅有福力量知文法黑汗王,書與東方日出處大世界田地主漢家阿舅大官家」,大略雲路遠傾心相向,前三遣使入貢未回,重復數百言。董氈使導至熙州,譯其辭以聞。詔前三輩使人皆
 已朝見,錫賚遣發,賜敕書諭之。神宗嘗問其使去國歲月,所經何國及有無鈔略。對曰:「去國四年,道塗居其半,歷回鶻
 
 黃頭回紇、青唐,惟懼契丹鈔略耳。」因使之圖上諸國距漢境遠近,為書以授李憲。八年九月,遣使入貢,使者為神宗飯僧追福。賜錢百萬,還其所貢師子。



 元祐中,以其使至無時,令熙河間歲一聽至闕。八年,請討夏國,不許。



 紹聖中,其王阿忽都董娥密竭篤又言,緬藥家作過,別無報效,已遣兵攻甘、沙、肅三州。詔厚答其意。知秦州
 遊師雄言:「于闐、大食、拂菻等國貢奉,般次踵至,有司憚於供賚,抑留邊方,限二歲一進。外夷慕義,萬里而至,此非所以來遠人也。」從之。自是訖於宣和,朝享不絕。



 高昌國,漢車師前王之地。有高昌城,取其地勢高敞、人民昌盛以為名焉。後魏初,沮渠無諱自署高昌太守。無諱死,茹茹以闞伯周為高昌王,高昌有王始於此。後魏至隋皆來貢獻。唐貞觀中,侯君集平其國,以其地為西州。安、史之亂,其地陷沒,乃復為國。語訛亦云「高敞」,然其
 地頗有回鶻,故亦謂之回鶻。



 建隆三年四月,西州回鶻阿都督等四十二人以方物來貢。乾德三年十一月,西州回鶻可汗遣僧法淵獻佛牙、琉璃器、琥珀盞。太平興國六年,其王始稱西州外生師子王阿廝蘭漢,遣都督麥索溫來獻。五月,太宗遣供奉官王延德、殿前承旨白勳使高昌。八年,其使安鶻盧來貢。



 雍熙元年四月,王延德等還,敘其行程來獻,云:



 初自夏州曆玉亭鎮,次曆黃羊平,其地平而產黃羊。渡沙磧,無水,行人皆載水。凡二
 日至都囉囉族,漢使過者,遺以財貨,謂之「打當」。次曆茅女子族,族臨黃河,以羊皮為囊,吹氣實之浮於水,或以橐駝牽木栰而渡。次曆茅女王子開道族,行入六窠沙,沙深三尺,馬不能行,行者皆乘橐駝。不育五穀,沙中生草名登相,收之以食。次曆樓子山,無居人。行沙磧中,以日為占,旦則背日,暮則向日,日中則止。夕行望月亦如之。次曆臥梁劾特族地,有都督山,唐回鶻之地。次曆大蟲太子族,族接契丹界,人衣尚錦繡,器用金銀,馬乳
 釀酒,飲之亦醉。次曆屋地因族,蓋達於於越王子之子。次至達於於越王子族。次曆拽利王子族,有合羅川,唐回鶻公主所居之地,城基尚在,有湯泉池。次曆阿墩族,經馬鬃山望鄉嶺,嶺上石龕有李陵題字處。次曆格囉美源,西方百川所會,極望無際,鷗鷺鳧雁之類甚眾。次至托邊城,亦名李僕射城,城中首領號「通天王」。次曆小石州。次曆伊州,州將陳氏,其先自唐開元二年領州,凡數十世,唐時詔敕尚在。地有野蠶生苦參上,可為綿帛。
 有羊,尾大而不能走,尾重者三斤,小者一斤,肉如熊白而甚美。又有礪石,剖之得賓鐵,謂之吃鐵石。又生胡桐樹,經雨即生胡桐律。次曆益都。次曆納職城,城在大患鬼魅磧之東南,望玉門關甚近。地無水草,載糧以行。凡三日,至鬼穀口避風驛,用本國法設祭,出詔神御風,風乃息。凡八日,至澤田寺。高昌聞使至,遣人來迎。次曆地名寶莊,又曆六種,乃至高昌。



 高昌即西州也。其地南距于闐,西南距大食、波斯,西距西天步路涉、雪山、蔥嶺,皆
 數千里。地無雨雪而極熱,每盛暑,居人皆穿地為穴以處。飛鳥群萃河濱,或起飛,即為日氣所爍,墜而傷翼。屋室覆以白堊,雨及五寸,即廬舍多壞。有水,源出金嶺,導之周圍國城,以溉田園,作水磑。地產五穀,惟無蕎麥。貴人食馬,餘食羊及鳧雁。樂多琵琶、箜篌。出貂鼠、白氎、繡文花蕊布。俗好騎射。婦人戴油帽,謂之蘇幕遮。用開元七年曆,以三月九日為寒食,餘二社、冬至亦然。以銀或鍮石為筒,貯水激以相射,或以水交潑為戲,謂之壓陽氣
 去病。好遊賞,行者必抱樂器。佛寺五十餘區,皆唐朝所賜額,寺中有《大藏經》、《唐韻》、《玉篇》、《經音》等,居民春月多群聚遨樂於其間。遊者馬上持弓矢射諸物,謂之禳災。有敕書樓,藏唐太宗、明皇御劄詔敕,緘鎖甚謹。復有摩尼寺,波斯僧各持其法,佛經所謂外道者也。所統有南突厥、北突厥、大眾熨、小眾熨、樣磨、割祿、黠戛司、末蠻、格哆族、預龍族之名甚眾。國中無貧民,絕食者共賑之。人多壽考,率百餘歲,絕地夭死。



 時四月,師子王避暑於北廷,
 以其舅阿多於越守國,先遣人致意於延德曰:「我王舅也,使者拜我乎?」延德曰:「持朝命而來,禮不當拜。」復問曰:「見王拜乎?」延德曰:「禮亦不當拜。」阿多於越復數日始相見,然其禮頗恭。師子王邀延德至其北廷。曆交河州,凡六日,至金嶺口,寶貨所出。又兩日,至漢家砦。又五日,上金嶺。過嶺即多雨雪,嶺上有龍堂,刻石記雲,小雪山也。嶺上有積雪,行人皆服毛罽。度嶺一日至北廷,憩高台寺。其王烹羊馬以具膳,尤豐潔。



 地多馬,王及王後、太子
 各養馬,放牧平川中,彌亙百餘里,以毛色分別為群,莫知其數。北廷川長廣數千里,鷹鷂雕鶻之所生,多美草,不生花,砂鼠大如,鷙禽捕食之。



 其王遣人來言,擇日以見使者,願無訝其淹久。至七日,見其王及王子侍者,皆東向拜受賜。旁有持磬者擊以節拜,王聞磬聲乃拜,既而王之兒女親屬皆出,羅拜以受賜,遂張樂飲宴,為優戲,至暮。明日泛舟於池中,池四面作鼓樂。又明日遊佛寺,曰應運太寧之寺,貞觀十四年造。



 北廷北山中出
 硇砂,山中嘗有煙氣湧起,無雲霧,至夕光焰若炬火,照見禽鼠皆赤。采者著木底鞋取之,皮者即焦。下有穴生青泥,出穴外即變為砂石,土人取以治皮。城中多樓台卉木。人白皙端正,性工巧,善治金銀銅鐵為器及攻玉。善馬直絹一匹,其駑馬充食,才直一丈。貧者皆食肉。西抵安西,即唐之西境。



 七月,令延德先還其國,其王九月始至。亦聞有契丹使來,謂其王云:「高敞本漢土,漢使來覘視封域,將有異圖,王當察之。」延德偵知其語,因謂王
 曰:「契丹素不順中國,今乃反間,我欲殺之。」王固勸乃止。



 自六年五月離京師,七年四月至高昌,所曆以詔賜諸國君長襲衣、金帶、繒帛。八年春,與其謝恩使凡百餘人復循舊路而還,雍熙元年四月至京師。



 景德元年,又遣使金延福來貢。



 回鶻本匈奴之別裔,在天德西北娑陵水上。後魏號鐵勒,唐初號特勒,後稱回紇。其君長曰可汗,自貞觀以後朝貢不絕。至德初,出兵助國討平安、史之亂,故累朝恩
 禮最重。然而恃功橫恣,朝廷雖患其邀求無厭,然頗姑息聽從之。元和中,改為回鶻。會昌中,其國衰亂,其相馺職者擁外甥將龐勒西奔安西。既而回鶻為幽州張仲武所破,龐勒乃自稱可汗,居甘、沙、西州,無復昔時之盛矣。



 歷梁、後唐、晉、漢、周,皆遣使朝貢。後唐同光中,冊其國王仁美為英義可汗。仁美卒,其弟仁裕立,冊為順化可汗。晉天福中,又改為奉化可汗。仁裕卒,子景瓊立。先是,唐朝繼以公主下嫁,故回鶻世稱中朝為舅,中朝每賜
 答詔亦曰外甥。五代之後皆因之。



 建隆二年,景瓊遣使朝獻。三年,阿都督等四十二人以方物來貢。乾德二年,遣使貢玉百團、琥珀四十斤,犛牛尾、貂鼠等。三年,遣使趙黨誓等四十七人以團玉、琥珀、紅白犛牛尾為貢。開寶中累遣使貢方物,其宰相鞠仙越亦貢馬。



 太平興國二年冬,遣殿直張璨齎詔諭甘、沙州回鶻可汗外甥,賜以器幣,招致名馬美玉,以備車騎琮璜之用。五年,甘、沙州回鶻可汗夜落紇密禮遏遣使裴溢的等四人,以橐
 駝、名馬、珊瑚、琥珀來獻。



 雍熙元年四月,西州回鶻與婆羅門僧永世、波斯外道阿里煙同入貢。四年,合羅川回鶻第四族首領遣使朝貢。端拱二年九月,回鶻都督石仁政、麽囉王子、邈拿王子、越黜黃水州巡檢四族並居賀蘭山下,無所統屬,諸部入貢多由其地。麽囉王子自雲,向為靈州馮暉阻絕,由是不通貢奉,今有內附意。各以錦袍銀帶賜之。



 咸平四年,可汗王祿勝遣使曹萬通以玉勒名馬、獨峰無峰橐駝、賓鐵劍甲、琉璃器來貢。萬
 通自言任本國樞密使,本國東至黃河,西至雪山,有小郡數百,甲馬甚精習,願朝廷命使統領,使得縛繼遷以獻。因降詔祿勝曰:「賊遷凶悖,人神所棄。卿世濟忠烈,義篤舅甥,繼上奏封,備陳方略,且欲大舉精甲,就覆殘妖,拓土西陲,獻俘北闕。可汗功業,其可勝言!嘉歎所深,不忘朕意。今更不遣使臣,一切委卿統製。」特授萬通左神武軍大將軍,優賜祿勝器服。



 景德元年,夜落紇遣使來貢。四年,又遣尼法仙等來朝,獻馬。仍許法仙遊五台山。
 又遣僧翟入奏,來獻馬,欲於京城建佛寺祝聖壽,求賜名額,不許。



 大中詳符元年,夏州萬子等軍主領族兵趨回鶻,回鶻設伏要路,示弱不與鬥,俟其過,奮起擊之,剿戮殆盡。其生擒者,回鶻驅坐於野,悉以所獲資糧示之,曰:「爾輩狐鼠,規求小利,我則不然。」遂盡焚而殺之,唯萬子軍主挺身走。鎮戎軍以聞,上曰:「回鶻嘗殺繼遷,世為仇敵。甘州使至,亦言德明侵軼之狀,意頗輕視之。量其兵勢,德明未易敵也。」其年,夜落紇、寶物公主及沒孤公
 主、娑溫宰相各遣使來貢。東封禮成,以可汗王進奉使姚進為寧遠將軍,寶物公主進奉曹進為安化郎將,賜以袍笏。又賜夜落紇介胄。



 三年,又遣左溫宰相、何居錄越樞密使、翟符守榮等來貢。是年,龜茲國王可汗遣使李延福、副使安福、監使翟進來進香藥、花蕊布、名馬、獨峰駝、大尾羊、玉鞍勒、琥珀、俞石等。四年,翟符守榮等三十人請從祀汾陰。其年,夜落紇遣使貢方物,秦州回鶻安密獻玉帶於道左。禮成,以翟符守榮為左神武軍大
 將軍,安殿民為保順郎將,餘皆賜冠帶器幣。其年,夜落紇遣使言,敗趙德明立功首領請加恩賞。詔給司戈、司階、郎將告敕十道,使得承製補署。



 六年,龜茲進奉使李延慶等三十六人對於長春殿,獻名馬、弓箭、鞍勒、團玉、香藥等,優詔答之。



 先是,甘州數與夏州接戰,夜落紇貢奉多為夏州鈔奪。及宗哥族感悅朝廷恩化,乃遣人援送其使,故頻年得至京師。既而唃廝羅欲娶可汗女而無聘財,可汗不許,因為仇敵。五年,秦州遣指揮使楊知
 進、譯者郭敏送進奉使至甘州,會宗哥怨隙阻歸路,遂留知進等不敢遣。八年,敏方得還。可汗王夜落隔上表言寶物公主疾死,以西涼人蘇守信劫亂,不時奏聞;又謝恩賜寶鈿、銀匣、曆日及安撫詔書,仍乞慰諭宗哥,使開朝貢之路。九年,楊知進亦至,遂遣郭敏賜宗哥詔書並甘州可汗器幣。其年,使來朝貢,言夜落隔卒,九宰相諸部落奉夜落隔歸化為可汗王領國事。



 天禧二年,夜落隔歸化遣都督安信等來朝。四年,又遣使同龜茲國
 可汗王智海使來獻大尾羊。初,回鶻西奔,族種散處。故甘州有可汗王,西州有克韓王,新復州有黑韓王,皆其後焉。



 天聖元年五月,甘州夜落隔通順遣使阿葛之、王文貴來貢方物。六月,詔甘州回紇外甥可汗王夜落隔通順特封歸忠保順可汗王。二年五月,遣使都督習信等十四人來貢馬及黃湖綿、細白氎。三年四月,可汗王、公主及宰相撒溫訛進馬、乳香。賜銀器、金帶、衣著、暈錦旋襴有差。五年八月,遣使安萬東等一十四人來貢方
 物。六年二月,遣人貢方物。



 熙寧元年入貢,求買金字《大般若經》,以墨本賜之。六年復來,補其首領五人為軍主,歲給彩二十匹。神宗問其國種落生齒幾何,曰三十餘萬;壯可用者幾何,曰二十萬。明年,敕李憲擇使聘阿裏骨,使諭回鶻令發兵深入夏境。憲以命殿直皇甫旦。旦往,不得前而妄奏功狀,詔逮旦赴御史獄抵罪。



 然回鶻使不常來,宣和中,間因入貢散而之陝西諸州,公為貿易,至留久不歸。朝廷慮其習知邊事,且往來皆經夏國,
 於播傳非便,乃立法禁之。



 大食國本波斯之別種。隋大業中,波斯有桀黠者探穴得文石,以為瑞,乃糾合其眾,剽略資貨,聚徒浸盛,遂自立為王,據有波斯國之西境。唐永徽以後,屢來朝貢。其王盆泥未換之前謂之白衣大食,阿蒲羅拔之後謂之黑衣大食。



 乾德四年,僧行勤遊西域,因賜其王書以招懷之。開寶元年,遣使來朝貢。四年,又貢方物,以其使李訶末為懷化將軍,特以金花五色綾紙寫官告以賜。是
 年,本國及占城、闍婆又致禮物於李煜。煜不敢受,遣使來上,因詔自今勿以為獻。六年,遣使來貢方物。七年,國王訶黎佛又遣使不囉海,九年又遣使蒲希密,皆以方物來貢。



 太平興國二年,遣使蒲思那、副使摩訶末、判官蒲囉等貢方物。其從者目深體黑,謂之昆侖奴。詔賜其使襲衣、器幣,從者縑帛有差。四年,復有朝貢使至。雍熙元年,國人花茶來獻花錦、越諾、揀香、白龍腦、白沙糖、薔薇水、琉璃器。



 淳化四年,又遣其副酋長李亞勿來貢。其
 國舶主蒲希密至南海,以老病不能詣闕,乃以方物附亞勿來獻。其表曰:



 大食舶主臣蒲希密上言,眾星垂象,回拱於北辰;百穀疏源,委輸於東海。屬有道之柔遠,罄無外以宅心。伏惟皇帝陛下德合二儀,明齊七政,仁宥萬國,光被四夷。賡歌洽《擊壤》之民,重譯走奉珍之貢。臣顧惟殊俗,景慕中區,早傾向日之心,頗鬱朝天之願。



 昨在本國,曾得廣州蕃長寄書招諭,令入京貢奉,盛稱皇帝聖德,布寬大之澤,詔下廣南,寵綏蕃商,阜通遠物。臣
 遂乘海舶,爰率土毛,涉曆龍王之宮,瞻望天帝之境,庶遵玄化,以慰宿心。今則雖屆五羊之城,猶賒雙鳳之闕。自念衰老,病不能興,遐想金門,心目俱斷。今遇李亞勿來貢,謹備蕃錦藥物附以上獻。臣希密凡進象牙五十株,乳香千八百斤,賓鐵七百斤,紅絲吉貝一段,五色雜花蕃錦四段,白越諾二段,都爹一琉璃瓶,無名異一塊,薔薇水百瓶。



 詔賜希密敕書、錦袍、銀器、束帛等以答之。



 至道元年,其國舶主蒲押陀黎齎蒲希密表來獻白龍
 腦一百兩,膃肭臍五十對,龍鹽一銀合,眼藥二十小琉璃瓶,白沙糖三琉璃甕,千年棗、舶上五味子各六琉璃瓶,舶上褊桃一琉璃瓶,薔薇水二十琉璃瓶,乳香山子一坐,蕃錦二段,駝毛褥麵三段,白越諾三段。引對於崇政殿,譯者代奏云:「父蒲希密因緣射利,泛舶至廣州,迨今五稔未歸。母令臣遠來尋訪、昉至廣州見之。具言前歲蒙皇帝聖恩降敕書,賜以法錦袍、紫綾纏頭、間塗金銀鳳瓶一對、綾絹二十匹。今令臣奉章來謝,以方物致
 貢。」



 太宗因問其國,對云:「與大秦國相鄰,為其統屬。今本國所管之民才及數千,有都城介山海間。」又問其山澤所出,對云:「惟犀象香藥。」問犀象以何法可取,對云:「象用象媒誘至,漸以大繩羈縻之耳;犀則使人升大樹操弓矢,伺其至射而殺之,其小者不用弓矢可以捕獲。」上賜以襲衣、冠帶、被褥等物,令閣門宴犒訖,就館,延留數月遣回;降詔答賜蒲希密黃金,準其所貢之直。三年二月,又與賓同隴國使來朝。



 咸平二年,又遣判官文戊至。三
 年,舶主陀婆離遣使穆吉鼻來貢。吉鼻還,賜陀婆離詔書並器服鞍馬。六年,又遣使婆羅欽三摩尼等來貢方物。摩尼等對於崇政殿,持真珠以進,自雲離國日誠願得瞻威顏即獻此,乞不給回賜。真宗不欲違其意,俟其還,優加恩齎。



 景德元年,又遣使來。時與三佛齊、蒲端國使並在京師,會上元觀燈,皆賜錢縱其宴飲。其秋,蕃客蒲加心至。四年,又遣使同占城使來,優加館餼之禮,許遍至苑囿寺觀遊覽。



 大中祥符元年十月,車駕東封,舶
 主陀婆離上言願執方物赴泰山,從之。又舶主李亞勿遣使麻勿來獻玉圭。並優賜器幣、袍帶,並賜國主銀飾繩床、水罐、器械、旗幟、鞍勒馬等。四年祀汾陰,又遣歸德將軍陀羅離進瓶香、象牙、琥珀、無名異、繡絲、紅絲、碧黃綿、細越諾、紅駝毛、間金線璧衣、碧白琉璃酒器、薔薇水、千年棗等。詔令陪位,禮成,並賜冠帶服物。五年,廣州言大食國人無西忽盧華百三十歲,耳有重輪,貌甚偉異。自言遠慕皇化,附古邏國舶船而來。詔就賜錦袍、銀帶
 加束帛。



 天禧三年,遣使蒲麻勿陀婆離、副使蒲加心等來貢。先是,其入貢路繇沙州,涉夏國,抵秦州。乾興初,趙德明請道其國中,不許。至天聖元年來貢,恐為西人鈔略,乃詔自今取海路繇廣州至京師。至和、嘉祐間,四貢方物。最後以其首領蒲沙乙為武寧司階。



 熙寧中,其使辛押陀羅乞統察蕃長司公事,詔廣州裁度。又進錢銀助修廣州城,不許。六年,都蕃首保順郎將蒲陀婆離慈表令男麻勿奉貢物,乞以自代,而求為將軍,詔但授麻
 勿郎將。其國部屬各異名,故有勿巡,有陀婆離,有俞盧和地,有麻囉跋等國,然皆冠以大食。勿巡所貢,又有龍腦、兜羅錦、球錦袂、蕃花簟,陀婆有金飾壽帶、連環臂鉤、數珠之屬。



 政和中,橫州士曹蔡蒙休押伴其使入都,沿道故滯留,強市其香藥不償直。事聞,詔提點刑獄置獄推治,因詔自今蕃夷入貢,並選承務郎以上清幹官押伴,按程而行,無故不得過一日,乞取賈市者論以自盜雲。



 其國在泉州西北,舟行四十餘日至藍里。次年乘風
 颿,又六十餘日始達其國。地雄壯廣袤,民俗侈麗,甲於諸蕃,天氣多寒。其王錦衣玉帶,躡金履,朔望冠百寶純金冠。其居以碼磠為柱,綠甘為壁,水晶為瓦,碌石為磚,活石為灰,帷幕用百花錦。官有丞相、太尉,各領兵馬二萬餘人。馬高七尺,士卒驍勇。民居屋宇略與中國同。市肆多金銀綾錦。工匠技術,咸精其能。



 建炎三年,遣使奉寶玉珠貝入貢。帝謂侍臣曰:「大觀、宣和間,茶馬之政廢,故武備不修,致金人亂華,危亡不絕如線。今復捐數十
 萬緡以易無用之珠玉,曷若惜財以養戰士?」詔張浚卻之,優賜以答遠人之意。紹興元年,復遣使貢文犀、象齒,朝廷亦厚加賜與,而不貪其利。故遠人懷之,而貢賦不絕。



 層檀國在南海傍,城距海二十里。熙寧四年始入貢。海道便風行百六十日,經勿巡、古林、三佛齊國乃至廣州。其王名亞美羅亞眉蘭,傳國五百年,十世矣。人語音如大食。地春冬暖。貴人以越布纏頭,服花錦白氎布,出入
 乘象、馬。有奉祿。其法輕罪杖,重罪死。穀有稻、粟、麥。食有魚。畜有綿羊、山羊、沙牛、水牛、橐駝、馬、犀、象。藥有木香、血竭、沒藥、鵬砂、阿魏、薰陸。產真珠、玻璃、密沙華三酒。交易用錢,官自鑄,三分其齊,金銅相半,而銀居一分,禁民私鑄。元豐六年,使保順郎將層伽尼再至,神宗念其絕遠,詔頒齎如故事,仍加賜白金二千兩。



 龜茲本回鶻別種。其國主自稱師子王,衣黃衣,寶冠,與宰相九人同治國事。國城有市井而無錢貨,以花蕊布
 博易。有米麥瓜果。西至大食國行六十日,東至夏州九十日。或稱西州回鶻,或稱西州龜茲,又稱龜茲回鶻。



 自天聖至景祐四年,入貢者五,最後賜以佛經一藏。熙寧四年,使李延慶、曹福入貢。五年,又使盧大明、篤都入貢。紹聖三年,使大首領阿連撒羅等三人以表章及玉佛至洮西。熙河經略使以其罕通使,請令於熙、秦州博買,而估所齎物價答賜遣還,從之。



 沙州本漢敦煌故地,唐天寶末陷於西戎。大中五年,張
 義潮以州歸順,詔建沙州為歸義軍,以義潮為節度使,領河沙甘肅伊西等州觀察、營田處置使。義潮入朝,以從子淮深領州事。至朱梁時,張氏之後絕,州人推長史曹義金為帥。義金卒,子元忠嗣。周顯德二年來貢,授本軍節度、檢校太尉、同中書門下平章事,鑄印賜之。



 建隆三年加兼中書令,子延恭為瓜州防禦使。興國五年元忠卒,子延祿遣人來貢。贈元忠敦煌郡王,授延祿本軍節度,弟延晟為瓜州刺史,延瑞為衙內都虞候。咸
 平四年,封延祿為譙郡王。五年,延祿、延瑞為從子宗壽所害,宗壽權知留後,而以其弟宗允權知瓜州。表求旌節,乃授宗壽節度使,宗允檢校尚書左僕射、知瓜州,宗壽子賢順為衙內都指揮使。大中祥符末宗壽卒,授賢順本軍節度,弟延惠為檢校刑部尚書、知瓜州。賢順表乞金字藏經洎茶藥金箔,詔賜之。至天聖初,遣使來謝,貢乳香、硇砂、玉團。自景祐至皇祐中,凡七貢方物。



 拂菻國東南至滅力沙,北至海,皆四十程。西至海三十
 程。東自西大食及于闐、回紇、青唐,乃抵中國。歷代未嘗朝貢。



 元豐四年十月,其王滅力伊靈改撒始遣大首領你廝都令廝孟判來獻鞍馬、刀劍、真珠,言其國地甚寒,土屋無瓦。產金、銀、珠、西錦、牛、羊、馬、獨峰駝、梨、杏、千年棗、巴欖、粟、麥,以蒲萄釀酒。樂有箜篌、壺琴、小篳篥、偏鼓。王服紅黃衣,以金線織絲布纏頭,歲三月則詣佛寺,坐紅床,使人舁之。貴臣如王之服,或青綠、緋白、粉紅、褐紫,並纏頭跨馬。城市田野,皆有首領主之,每歲惟夏秋兩得
 奉,給金、錢、錦、穀、帛,以治事大小為差。刑罰罪輕者杖數十,重者至二百,大罪則盛以毛囊投諸海。不尚鬥戰,鄰國小有爭,但以文字來往相詰問,事大亦出兵。鑄金銀為錢,無穿孔,麵鑿彌勒佛,背為王名,禁民私造。



 元祐六年,其使兩至。詔別賜其王帛二百匹、白金瓶、襲衣、金束帶。



 Category:印度



 Category:歐洲



\end{pinyinscope}