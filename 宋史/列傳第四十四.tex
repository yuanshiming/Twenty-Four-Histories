\article{列傳第四十四}

\begin{pinyinscope}

 陳執中劉沆馮拯子行己伸己賈昌朝弟昌衡從子炎伯祖父琰梁適孫子美



 陳執中,字昭譽,以父恕任,為秘書省正字,累遷衛尉寺丞、知梧州。上《復古要道》三篇,真宗異而召之。帝屬疾,春
 秋高,大臣莫敢言建儲者,執中進《演要》三篇,以蚤定天下根本為說。翌日,帝以他疏示輔臣,皆贊曰「善」。帝指其袖中曰:「又有善於此者。」出之,乃《演要》也。因召對便殿,勞問久之,擢右正言。逾月,遂立皇太子。明年,坐考御試進士卷差謬,貶衛尉寺丞、監岳州酒務。稍復殿中丞、通判撫州,復右正言。



 曹利用婿盧士倫除福建運使,憚遠不行,利用為請,乃改京東。執中嘗劾奏之,利用挾私忿,出執中知漢陽軍。及利用得罪,乃召為群牧判官、權三司
 鹽鐵判官、知諫院、提舉諸司庫務,以尚書工部員外郎兼御史知雜、同判流內銓,遷三司戶部副使。



 明道中,安撫京東,進天章閣待制。使還,知應天府,徙江寧府、揚州,再遷工部郎中,改龍圖閣直學士、知永興軍,拜右諫議大夫、同知樞密院事。



 元昊寇延州,手詔咨訪輔臣攻守方略,執中既上對,退,復奏疏曰:「元昊乘中國久不用兵,竊發西垂,以游兵困勁卒、甘言悅守臣,一旦連犯亭障,延安幾至不保。此蓋範雍納詭說,失於戒嚴;劉平輕躁,
 喪其所部。上下紛攘,遠近震駭。自金明李士彬族破,而並邊籬落皆大壞。塞門、金明相距二百里,宜列修三城,城屯兵千人,益募弓箭手。寇大至則退保,小至則出鬥。選閣門祗候以上為寨主、都監,以諸司使為盧關一路都巡檢,以兵二千屬之,使為三砦之援。熟羌居漢地久者,委邊臣拊存之;反復者,破逐之。至於新拊黠羌,如涇原康奴、滅臧、大蟲族,久居內地,常有叛心,不肆剪除,恐終為患。今軍須之出,民已愁嘆,復欲遍修城池如河北
 之制,及夏須成,使神運之猶恐不能,民力其堪此乎?」陜西地險,非如河北,惟涇州、鎮戎軍勢稍平易,若不責外守而勞內營,非策之上也。宜修並邊城池,其次如延州之鄜、同,環慶之邠、寧,不過五七處,量為營葺,則科率減、民力蘇矣。今賊勢方張,宜靜守以驕其志,蓄銳以挫其鋒,增土兵以備守御,省騎卒以減轉餉。然後徐議蕩平,改張節度,更須主張,將臣橫議不入,則忠臣盡節而捐軀矣。」



 既而議刺土兵,久不決,罷知青州。又以資政殿學
 士知河南府,改尚書工部侍郎、陜西同經略安撫招討使。與夏竦同知永興軍,議邊事多異同,詔令互出巡邊,乃屯涇州,令諸部曰:「寇籍吾水草,鈔邊圖利,不除,且復至。」命悉焚之。表解兵柄,以為兵尚神密,千里稟命,非所以制勝,宜屬四路各保疆圉。朝議善之,就知陜州,復徙青州。於是請城傅海諸州,朝廷重興役,有詔不許。執中不奉詔,卒城之。



 明年,沂卒王倫叛,趣淮南,執中遣巡檢傅永吉追至採石磯,捕殺之。召拜參知政事。諫官孫甫、
 蔡襄極論不可,帝遣使馳賜敕告。逾年,拜同中書門下平章事、集賢殿大學士兼樞密使。西夏納款,與宰相賈昌朝請解樞密。七年春,旱,昌朝罷,執中降給事中。已而加昭文館大學士、監修國史,逾月復官。



 皇祐初,以足疾辭位,自陳不願為使相、大學士,學士孫抃當制,遂以尚書左丞知陳州。宰相文彥博、宋庠以為禮薄,帖麻改兵部尚書。遷吏部、觀文殿大學士。久之,拜集慶軍節度使、同平章事、判大名府。河決商胡,走大名,程琳欲為堤,不
 果成而去。執中乘年豐調丁夫增築二十里,以障橫潰。以吏部尚書復拜同平章事、昭文館大學士。每朝退,閉中書東便門,以防漏洩。三司勾當公事及監場務官,權勢所引者,皆奏罷之,內外為之肅然。



 會張貴妃薨,治喪皇儀殿,追冊為後。王洙、石全彬務以非禮導帝意,執中隨輒奉行,至以洙為員外翰林學士,全彬領觀察使,給留後奉。久之,嬖妾笞小婢出外舍死,御史趙抃列八事奏劾執中,歐陽修亦言之。至和三年春,旱,諫官範鎮言:「
 執中為相,不病而家居。陛下欲弭災變,宜速退執中,以快中外之望。」既而御史中丞孫抃與其屬郭申錫、毋湜、範師道、趙抃請合班論奏,詔令輪日入對,卒罷執中為鎮海軍節度使、同平章事、判亳州。逾年辭節,改尚書左僕射、觀文殿大學士,封英國公,徙河南府,又徙曹州,皆不赴。過都,以疾賜告,就第拜司徒、岐國公致仕,卒,贈太師兼侍中。



 執中在中書八年,人莫敢干以私,四方問遺不及門,惟殿前都指揮使郭承祐數至其家,為御史所
 言,遂詔中書、樞密自今非聚廳無見賓客。及議謚,禮官韓維曰:「執中以公卿子,遭世承平,因緣一言,遂至貴顯。天子以後宮之喪,問所以葬祭之禮,執中位上相,不能總率群司考正儀典,知治喪皇儀非嬪御之禮,追冊位號於宮闈有嫌,建廟用樂逾祖宗舊制,皆白而行之,此不忠之大者。閨門之內,禮分不明,夫人正室疏薄自絀,庶妾賤人悍逸不制,其治家無足言者。宰相不能秉道率禮,正身齊家,方杜門深居,謝絕賓客,曰:『我無私也,我
 不黨也。』豈不陋哉?謚法:『寵祿光大曰榮』,『不勤成名曰靈』。執中出入將相,以一品就第,寵祿光大矣;得位行政,賢士大夫無述焉,不勤成名矣;請謚曰榮靈。」後改謚恭襄,詔謚曰恭。帝篆其墓碑曰「褒忠之碑」。



 子世儒,官至國子博士,妻李與群婢殺世儒所生母,世儒與謀,皆棄市。



 劉沆,字沖之,吉州永新人。祖景洪,始,楊行密得江西,衙將彭玕據州自稱太守,屬景洪以兵,欲脅眾附湖南,景洪偽許之。復以州歸行密,退居不仕。及徐溫建國,以禮
 聘之,不起,官其子煦為殿直都虞候。父素,不仕,以財雄里中,喜賓客。景洪嘗告人曰:「我不從彭玕,幾活萬人,後世當有隆者。」因名所居北山曰後隆山。山有牛僧孺讀書堂,即故基築臺曰聰明臺。沆母夢衣冠丈夫曰牛相公來,已來有娠,乃生沆。



 及長,倜儻任氣。舉進士不中,自稱「退士」,不復出,父力勉之。天聖八年,始擢進士第二,為大理評事、通判舒州,有大獄歷歲不決,沆數日決之。章獻太后建資聖浮圖,內侍張懷信挾詔命,督役嚴峻,州
 將至移疾不敢出,沆奏罷懷信。再遷太常丞、直集賢院,出知衡州。大姓尹氏欺鄰翁老子幼,欲竊取其田,乃偽作賣券,及鄰翁死,遂奪而有之。其子訴於州縣,二十年不得直,沆至,復訴之。尹氏持積歲稅鈔為驗,沆曰:「若田千頃,歲輸豈特此耶?爾始為券時,嘗如敕問鄰乎?其人固多在,可訊也。」尹氏遂伏罪。遷太常博士,歷三司度支、戶部判官、同修起居注,擢右正言、知制誥、判吏部流內銓。奉使契丹,館伴杜防強沆以酒,沆沾醉,拂袖起,因罵
 之,坐是出知潭州。又降知和州,改右諫議大夫、知江州。



 時湖南蠻數出寇,至殺官吏。以沆為龍圖閣直學士、知潭州兼安撫使,許便宜從事。沆大發兵至桂陽,招降二千餘人,使散居所部,而蠻酋降者皆奏命以官。又募土兵分捕餘黨,破桃油平、能家源,斬馘甚眾。已而賊復出,殺裨將胡元,坐降知鄂州,徙京南,遷給事中,徙洪州。還,知審刑院,除知永興軍。頃之,以龍圖閣學士權知開封府,數發隱伏。祀明堂,遷尚書工部侍郎。逾年,拜參知
 政事。



 初,沆在府,有張彥方者,客越國夫人曹氏家,受富民金,為偽告敕。既敗系獄,沆抵彥方死,辭不及曹氏。曹氏,張貴妃母也。沆既用,諫官、御史皆謂沆於彥方獨不盡,疑以此進,爭論之,帝不聽。貴妃薨,追冊皇后,沆為監護使。數月,拜同中書門下平章事、集賢殿大學士,改園陵使。御史中丞孫抃、御史範師道、毋湜言,宰相不當為贈後典葬,不報。既葬,賜後閣中金器數百兩,力辭,而請其子瑾試學士院,遂帖職。



 時中書可否多用例,人或援
 例以訟,而法有不行。沆進言三弊曰:「近臣保薦闢請,動逾數十,皆浮薄權豪之流交相薦舉。有司以之貿易,而遂使省、府、臺、閣華資要職,路分、監司邊防寄任,授非公選,多出私門。又職掌吏人遷補有常,而或減選出官、超資換職、堂除便家、先次差遣之類。此近臣保薦之弊一也。審官、吏部銓、三班當入川、廣,乃求近地,當入近地,又求在京,及堂除升陟省府、館職、檢討之類。此近臣陳丐親屬之弊二也。其敘錢穀管庫之勞、捕賊昭雪之賞,常
 格雖存,僥幸猶甚。以法則輕,以例則厚,執政者不能持法,多以例與之。此敘勞幹進之弊三也。願詔中書、樞密,凡三事毋用例,余聽如舊。」事既施行,而眾頗不悅,尋如舊。



 文彥博、富弼復入為相。彥博為昭文館大學士,弼監修國史,沆遷兵部侍郎,位在弼下。論者以為非故事,由學士楊察之誤,乃帖麻改沆監修國史,弼為集賢殿大學士。沆既疾言事官,因言:「自慶歷後,臺諫官用事,朝廷命令之出,事無當否悉論之,必勝而後已,專務抉人陰
 私莫辨之事,以中傷士大夫。執政畏其言,進擢尤速。」沆遂舉行御史遷次之格,滿二歲者與知州。御史範師道、趙抃歲滿求補郡,沆引格出之,中丞張忭等言沆挾私出御史。時樞密使狄青亦因御史言,罷知陳州,沆奏曰:「御史去陛下將相,削陛下爪牙,此曹所謀,臣莫測也。」升等益論辨不已,罷沆為觀文殿大學士、工部尚書、知應天府。遷刑部尚書,徙陳州。



 沆長於吏事,性豪率,少儀矩。然任數,善刺探權近過失,陰持之以軒輊取事,論者以
 此少之。卒,贈左僕射兼侍中。知制誥張瑰草詞詆沆,其家不敢請謚。帝為篆墓碑曰「思賢之碑」。子瑾,嘗為天章閣待制,坐法免,後以功復職。



 馮拯,字道濟。父俊,事漢湘陰公劉贇。贇死,俊與從行千餘人系侍衛獄,周太祖赦出之,授檢校太子賓客,戍安遠軍馭馬鎮,辭不行,因徙居河陽。



 拯以書生謁趙普,普奇其狀,曰:「子富貴壽考,宜不下我。」舉進士,補大理評事、通判峽州,權知澤州,徙坊州,遷太常丞。江南旱,命馳傳
 振貸貧乏,察官吏能否,還奏稱旨,權知石州,擢右正言,歲餘代歸。出使河北,與轉運使樊知古計邊儲,還,判三司戶部理欠憑由司,為度支判官。



 淳化中,有上封請立皇太子者,拯與尹黃裳、王世則、洪湛伏閣請立許王元僖,太宗怒,悉貶嶺外。拯知端州,既至,上言請遣使括諸路隱丁、更制版籍及議鹽法通商,凡十餘事。太宗欲召還參知政事,寇準素不悅拯,乃徙知鼎州。改通判廣州。郊祀畢,覃恩,拯與通判彭惟節皆遷尚書員外郎,惟節
 以太常博士為屯田員外,而拯以左正言為虞部員外。拯書名舊在惟節上,及奏事如故,準切責之。拯上書言準阿意不平,準坐此罷。



 拯以母喪請內徙,命知江州。真宗即位,進比部員外郎。御史中丞李惟清表為推直官,判三司度支勾院,遷駕部。咸平初,坐試開封進士賦涉譏訕,下拯御史臺,未幾,釋之。



 明年,兼侍御史知雜事。時西北用兵,王超、傅潛將兵出定、瀛間,觀望玩寇,拯極論之,不報。超等果逗撓覆軍。命拯按傅獄,抵潛罪,竄流之。
 擢祠部郎中、樞密直學士,權判吏部流內銓。以審官及銓法未備,建請凡蔭補京官,試讀一經,書家狀通習為中格,始得仕。同勾當三班院。向敏中宣撫河北、河東,拯及陳堯叟為副,宴餞長春殿。



 明年,以右諫議大夫同知樞密院事。帝欲修綏州,謀諸輔臣,拯與宰相向敏中等皆曰便。宰相呂蒙正、參知政事王旦、王欽若皆曰宜棄勿修。帝遣洪湛馳驛往視,還,上七利二害,卒修完之。時上封者言:「三司多滯務,州郡稟疑事,吏民訴理冤獄,依
 違不決者輒數歲,水旱或由於此。」詔拯選干強吏同三司使裁冗事、督舉稽留,遂與判度支勾院孫冕省帳牘二十一萬五千本,並廢冗官十五員。



 遷尚書工部侍郎、簽書樞密院事。賜手札訪邊事,拯謂:「備邊之要,不扼險以制敵之沖,未易勝也。若於保州、威虜間,依徐、鮑河為陣,其表勢可取勝矣。前歲王顯違詔不趨要地,契丹初壓境,王師未行,而契丹騎已入鈔,賴霖雨乃遁去。比王超奏敵已去,而東路奏敵方來,既聚軍中山以救望都,
 而兵困糧匱,將臣陷歿幾盡,超等僅以身免。今防秋,宜於唐河增屯兵至六萬,控定武之北為大陣,邢州置都總管為中陣,天雄軍置鈐轄為後陣,罷莫州狼山兩路兵。」從之。景德中,為參知政事,再遷兵部侍郎。攝事享太廟,有司供帳幔,守奉人宿廟室前,喧囂不肅,拯以聞。詔專為廟享制帟幕什器,藏宗正寺,禁吏卒登廟階。



 王濟上編敕,帝以其煩簡不一,語輔臣曰:「顯德敕尤煩,蓋世宗嚴急,出於一時之意,臣下不敢言其失也。」王旦進曰:「
 詔敕宜簡,近亦傷於煩。」拯對曰:「開寶間,除諸州通判敕,刑獄、錢穀悉條列約束,今則略矣。」時契丹始盟,拯言邊方騷動,武臣幸之以為利。帝曰:「朝廷以信為守,然戒備不可廢也,此外,當靜治以安吾民爾。爾其奉承之。」



 大中祥符初,嚴貢舉糊名法。拯與王旦論選舉帝前,拯請兼考策論,不專以詩賦為進退。帝曰:「可以觀才識者,文論也。」拯論事多合帝意如此。封泰山,為儀仗使。禮成,進尚書左丞。以疾在告,數請罷,帝以手詔諭旨,又命宰相王
 旦就第勸拯起視事。



 從祀汾陰,為儀仗使,遷工部尚書。復以疾求罷,拜刑部尚書、知河南府,聽以府事委官屬。七年,除御史中丞,又以疾辭,除戶部尚書、知陳州。真宗嘗謂王旦曰:「拯固求閑郡,何邪?」旦對曰「馬知節嘗譏拯好富貴,所欲節度使爾。拯恐為知節所量,不敢請大藩,殆為此也。」再知河南府,遷兵部尚書,入判尚書都省,以吏部尚書、檢校太傅、同中書門下平章事充樞密使。其冬,拜右僕射兼中書侍郎、太子少傅、同平章事、集賢殿
 大學士,進左僕射。



 乾興元年,進封魏國公,遷司空兼侍中。輔臣會食資善堂,召議事,丁謂獨不預。謂知得罪,頗哀請。錢惟演遽曰:「當致力,無大憂也。」拯熟視惟演,惟演踧茳。及對承明殿,太后怒甚,語欲誅謂。拯進曰:「謂固有罪,然帝新即位,亟誅大臣,駭天下耳目。謂豈有逆謀哉?第失奏山陵事耳。」太后怒少解。謂既貶,拯代謂為司徒、玉清昭應宮使、昭文館大學士、監修國史,又為山陵使,奉安真宗御容於西京。尋在病告,帝賜白金五千兩,拯
 叩頭稱謝。五上表願罷相,拜武勝軍節度使、檢校太尉兼侍中、判河南府。即臥內賜告及旌纛,遣內司賓撫問。還,奏其家儉陋,被服甚質。太后賜以衾裯錦綺屏,然拯平居自奉侈靡,顧禁中不知也。既卒,贈太師、中書令,謚文懿。



 拯氣貌嚴重,宦者傳詔至中書,不延坐。工部尚書林特嘗詣第,累日不得通,白以咨事,使詣中書。既至,又遣堂吏謂之曰:「公事何不自達朝廷?」卒不見,特大愧而去。錢惟演營入相,拯以太后姻家力言之,遂出惟演河
 陽。子行己、伸己。



 行己字肅之,以父任為右侍禁、涇原路駐泊都監、知憲州,因治狀增秩。歷石、保、霸、冀、莫五州,所至有能稱。



 夏人既納款,疆候播言契丹治兵幽燕,大為戰具,議者欲解西備北,行己言:「遼、夏為與國,元昊入貢,容懷詭計,幽燕治兵,或為虛聲,邊鄙之虞,恐不在河朔也。」



 皇祐中,知定州,韓琦薦為路鈐轄。徙知代州,管幹河東緣邊安撫事。夏人掠麟州,蕃部且盜耕屈野河西田,遇官軍逴邏者,
 輒聚射。詔行己計之。行己言:「此奸民無忌憚,非君長過,不宜以細故啟大釁,但加戒戢足矣。」



 五臺山寺調廂兵義勇繕葺,為除和糴穀三萬,行己謂不可損歲入之儲,以事不急之務。進西上閣門使,四遷客省使,更高陽關、秦鳳、定州、大名府路馬步總管,以衛州防禦使致仕,預洛陽耆英之集。元祐中,終金州觀察使,年八十四。



 伸己字齊賢,以蔭補右侍禁。累遷西頭供奉官,授閣門祗候、桂州兵馬都監。轉運使俞獻可闢知廉州。久之,安
 化蠻擾邊,獻可又薦知宜州。



 天聖中,改桂、宜、融、柳、象沿邊兵馬都監,遂專溪峒事。以禮賓使復知宜州。代還,道改供備庫使、知邕州。治舍有井,相傳不敢飲,飲輒死。伸己日汲自供,終更無恙。旁城數里,有金花木,土俗言花開即瘴起,人不敢近。伸己故以花盛時酣燕其下,亦復無害。明道恭謝,改東染院使、領榮州刺史、梓夔路兵馬鈐轄,遷洛苑使、知桂州兼廣西鈐轄。道江陵,會安化蠻犯邊,官軍不利,仁宗遣中人趣伸己討之。伸己日夜疾
 馳至宜州,繕器甲,募丁壯,轉糧餉,由三路以進。伸己臨軍,單騎出陣,語酋豪曰:「朝廷撫汝甚厚,汝乃自取滅亡耶!今我奉天子命來,汝聽吾言則生,不則無□類矣。」眾仰泣羅拜曰:「不圖今日再見馮公也。」明日,蠻渠棄兵械率眾降軍門。



 初,部卒以覆將畏匿,伸己曰:「紀律不明,主將也,戰士何罪?」請於朝,貸其死。以勞遷西上閣門使、知宜州。樂善蠻寇武陽,伸己遣諭禍福,蠻大悅,悉還所掠。又莫世堪負險強黠,抄劫邊戶,為疆場患。伸己設伏擒
 捕,皆置於法。遷果州團練使。在宜二年,徙桂州,改右武衛大將軍,守本官分司西京,卒。



 始,安化蠻叛,區希範應募擊賊。賊平,希範詣闕,自言其功。朝廷下宜州,伸己謂希範無功妄要賞,遂編管全州。其後希範遁歸,謀為亂,欲殺伸己,嶺外騷然,議者皆罪伸己焉。



 賈昌朝,字子明,真定獲鹿人。晉史官緯之曾孫也。天禧初,真宗嘗祈谷南郊,昌朝獻頌道左,召試,賜同進士出身,主晉陵簿。賜對便殿,除國子監說書。孫奭判監,獨
 稱昌朝講說有師法。他日書路隨、韋處厚傳示昌朝曰:「君當以經術進,如二公。」為穎川郡王院伴讀。再遷殿中丞,歷知宜興、東明縣。奭侍讀禁中,以老辭,薦昌朝自代,召試中書,尋復國子監說書。上言:「禮,母之諱不出於宮。今章獻太後易月制除,猶諱父名,非尊宗廟也。」詔從之。景祐中,置崇政殿說書,以授昌朝。誦說明白,帝多所質問,昌朝請記錄以進,賜名《邇英延義記注》,加直集賢院。



 太平興國寺災,是夕,大雨震雷。朝廷議修復,昌朝上言:「《
 易·震》之象曰:『洊雷震,君子以恐懼修省。』近年寺觀屢災,此殆天示警告,可勿繕治,以示畏天愛人之意。」西域僧獻佛骨、銅像,昌朝請加賜遣還,毋以所獻示中外。悉行其言。天章閣置侍講,亦首命昌朝。累遷尚書禮部郎中、史館修撰。



 劉平為元昊所執,邊吏誣平降賊,議收其家。昌朝曰:「漢族殺李陵,陵不得歸,而漢悔之。先帝厚撫王繼忠家,終得繼忠用。平事未可知,使收其族,雖平在,亦不得還矣。」乃得不收。擢知制誥、權判吏部流內銓兼侍
 講。初,銓法,縣令奉錢滿萬二千,乃舉令。昌朝曰:「法如此,則小縣終不得善令。」請概舉令,而與之奉如大縣。」



 進龍圖閣直學士、權知開封府,遷右諫議大夫、權御史中丞兼判國子監。議者欲以金繒啖契丹使攻元昊,昌朝曰:「契丹許我有功,則責報無窮矣。」力止之。乃上言曰:「太祖初有天下,監唐末五代方鎮武臣、土兵牙校之盛,盡收其威權,當時以為萬世之利。及太宗時,將帥率多舊人,猶能仗威靈,稟成算,出師禦寇,所向有功。近歲恩幸子
 弟,飾廚傳,釣名譽,多非勛勞,坐取武爵,折沖攻守,彼何自而知哉?然邊鄙無事,尚得自容。自西羌之叛,士不練習,將不得人,以屢易之將馭不練之士,故戰則必敗。此削方鎮太過之弊也。況親舊、恩幸,出即為將,素不知兵,一旦付以千萬人之命,是驅之死地矣。此用親舊、恩幸之弊也。今楊崇勛、李昭亮尚任邊鄙,望速選士代之。方鎮守臣無數更易,刺史以上,宜慎所授,以待有功。此救弊之一端也。」又上備邊六事:



 其一曰馭將帥。自古帝王,
 以恩威馭將帥,賞罰馭士卒,用命則軍政行而戰功集。太祖脫裘帽賜王全斌曰:「今日居此幄,尚寒不可御,況伐蜀將士乎?」此馭之以恩也。曹彬、李漢瓊討江南,太祖召彬至前,立漢瓊等於後,授以劍曰:「副將以下,不用命者得專戮之。」漢瓊等股慄而退,此馭之以威也。太祖雖削武臣之權,然一時賞罰及用財集事,皆聽其專,有功則賞,有敗則誅。今每命將帥,必先疑貳,非近幸不信,非姻舊不委。今陜西四路,總管而下,鈐轄、都監、巡檢之屬,
 悉參軍政,謀之未成,事已先漏,甲可乙否,上行下戾,主將不專號令,故動則必敗。請自今命將,去疑貳,推恩惠,務責以大效,得一切便宜從事。偏裨有不聽令者,以軍法論,此馭將之道也。



 其二曰復土兵。今河北河東強壯、陜西弓箭手之類,土兵遺法也。河北鄉兵,其廢已久,陜西土兵,數為賊破,存者無幾。臣以謂河北、河東強壯,已召近臣詳定法制,每鄉為軍。其材能絕類者,籍其姓名遞補之。陜西蕃落弓箭手,貪召募錢物,利月入糧奉,多
 就黥涅為營兵。宜優復田疇,使力耕死戰,世為邊用,可以減屯戍、省供饋矣。內地州縣,增置弓手,如鄉軍之法而閱試之。



 其三曰訓營卒。太祖朝,令諸軍毋得食肉衣帛,營舍有粥酒肴則逐去,士卒有服繒彩者笞責之。異時被鎧甲、冒霜露,戰勝攻取,皆此曹也。今營卒驕惰,臨敵無勇。舊例三年轉員,謂之落權正授,雖未能易此制,即不必一例使為總管、鈐轄,擇有才勇可任將帥者授之。況今之兵仗制造,殊不適用。宜按八陣、五兵之法,以
 時教習。使啟殿有次序、左右有形勢,前卻相附,上下相援,令之曰:「失一隊長,則斬一隊。」何慮眾不為用乎?



 其四曰制遠人。今四夷蕩然與中國通,在北則臣契丹,其西則臣元昊,二國合從,有掎角中國之勢。借使以歲幣羈縻之,臣恐不可勝算。古之備邊,西有金城、上郡,北則雲中、雁門。今自滄之秦,綿亙數千里,無山河之阻,獨恃州縣鎮戍爾。歲所供贍,又不下數千萬,一穀不熟,或至狼狽。契丹近歲兼用燕人治國,建官一同中夏。元昊據河
 南列郡而行賞罰,此中國患也。宜度西方諸國如沙州、唃廝、明珠、滅臧之族,近北如黑水女真、高麗、新羅之屬,舊通中國,募人往使,誘之使歸我,則勢分而釁生,體解而瓦裂矣。



 其五曰綏蕃部。屬戶者,邊垂之屏翰也。延有金明,府有豐州,皆戎人內附之地。朝廷恩威不立,強敵迫之,塞上諸州,藐焉孤壘,蕃部既壞,土兵亦衰,破敵之日,未可期也。臣請陜西緣邊諸路,守臣皆帶「安撫蕃部」之名,擇其族大有勞者為酋帥,如河東折氏之比,庶可
 為吾藩籬之固也。



 其六曰謹覘候。古者守封疆,出師旅,居則有行人之覘國,戰則有前茅之慮無,其謹如此。太祖命李漢超鎮關南,馬仁瑀守瀛州,韓令坤鎮常山,賀惟忠守易州,何繼筠領棣州,郭進控山西,武守琪戍晉陽,李謙溥守隰州,董遵誨屯環州,王彥升守原州,馮繼業鎮靈武。筦榷之利,悉輸之軍中,聽其貿易,而免其征稅。邊臣富於財,得以為間諜,羌夷情狀,無不預知。二十年間,無外顧之憂。今日西鄙任邊事者,敵之情狀與山
 川、道路險易之勢,絕不通曉。使蹈不測之淵,入萬死之地,肝腦塗地,狼狽相藉,何以破敵制勝耶?願監藝祖任將帥之制,邊城財用悉以委之。募敢勇之士為爪牙,臨陣自衛,無殺將之辱;募死力為覘候,而望敵知來,無陷兵之恥。



 書奏,多施行之。



 昌朝請度經費,罷不急。詔與三司合議,歲所省緡錢百萬。又言:「朝臣七十,筋力衰者,宜依典故致仕,有功狀可留者勿拘。」因疏耄昏不任事者八人,令致仕。慶歷三年,拜參知政事。上言:「用兵以來,天
 下民力頗困。請詔諸路轉運使,毋得承例折變科率,須科折者,悉聽奏裁。雖奉旨及三司文移,於民不便者,亦以上聞。」



 以工部侍郎充樞密使,尋拜同中書門下平章事、集賢殿大學士,仍兼樞密使。居兩月,拜昭文館大學士,監修國史。元昊歸石元孫,議賜死。昌朝獨曰:「自古將帥被執,歸者多不死。」元孫由是得免。詔有司議升祔奉慈廟三後,有司論不一。昌朝曰:「章獻母儀天下,章懿誕育聖躬,宜如詳符升祔元德皇后故事。章惠於陛下有
 慈保之恩,當別享奉慈廟如故。」乃奉二後神主,升祔真宗廟。密詔遷中外官一等,優賜諸軍,昌朝與同列力疏,乃止。又詔遷二府官,益固辭。元昊既款附,請宰相罷兼樞密使。



 六年,日食。帝謂昌朝等曰:「謫見於天,願歸罪朕躬。卿宜究民疾苦,思所以利安之。」昌朝對曰:「陛下此言,足以弭天變,臣敢不夙夜孜孜以奉陛下。」帝又曰:「人主懼天而修德,猶人臣畏法而自新也。」昌朝因頓首謝。明年春,旱,帝避正寢,減膳。昌朝引漢災異冊免三公故事,
 上表乞罷。



 參知政事吳育數與昌朝爭議上前,論者多不直昌朝。有向綬者知永靜軍,疑通判譖己,誣以事,迫令自殺。高若訥知審刑院,附昌朝議,欲從輕坐。吳育力爭,綬卒減死一等。未幾,若訥為御史中丞,言大臣廷爭不肅,故雨不時若,遂罷育,而除昌朝武勝軍節度使、檢校太傅、同中書門下平章事、判大名府兼北京留守司、河北安撫使。帝賜銀飾肩輿。尋以討貝州賊有功,移山南東道節度使。楊偕言賊發昌朝部中,不當賞。弗從。



 契丹聚亡卒勇伉者,號「投來南軍」。邊法,卒亡自歸者死。昌朝除其法,歸者輒遷補,於是來者稍眾,因廉知契丹事。契丹遂拒亡卒,黜南軍不用。邊人以地外質,契丹故稍侵邊界。昌朝為立法,質地而主不時贖,人得贖而有之,歲餘,地悉復。



 三司使葉清臣移用河北庫錢,昌朝格詔不與,清臣論列不已,遂出清臣河陽,徙昌朝判鄭州。過闕入覲,留為祥源觀使,拜尚書右僕射、觀文殿大學士、判尚書都省,朝會班中書門下,視其儀物。歲中求外,復
 除山南東道節度使、右僕射、檢校太師兼侍中、判鄭州。固辭僕射、侍中,改同中書門下平章事。賜中謝,自昌朝始也。



 母喪去位,服除,判許州。召對邇英閣,帝問《乾卦》,昌朝上奏曰:「《乾》之上九稱:『亢龍有悔。』悔者,兇災之萌,爻在亢極,必有兇災。不言兇而言悔者,以悔有可兇可吉之義,修德則免悔而獲吉矣。『用九,見群龍無首,吉』。聖人用剛健之德,乃可決萬機。天下久盛,柔不可以濟,然亢而過剛又不能久。獨聖人外以剛健決事,內以謙恭應物,
 不敢自矜為天下首,乃吉也。」手詔優答。又言:「漢、唐都雍,置三輔內翼京師,朝廷都汴,而近京諸郡皆屬他道,制度不稱王畿。請析京東之曹州,京西之陳、許、滑、鄭,皆隸開封府,以四十二縣為京畿。」帝納之。將行,命講讀官餞於資善堂。復判大名府兼河北安撫使。時河決商胡,昌朝請復故道,不從。語在《河渠志》。六塔功敗,濱、棣、德、博民多水死,昌朝振救之甚力。內侍劉恢往視,還,言河決趙徵村,與帝名嫌為不祥,時皆謂昌朝使之以搖當國者。
 嘉祐元年,進封許國公,又兼侍中,尋以同中書門下平章事為樞密使。



 三年,宰相文彥博請罷,諫官、御史恐昌朝代彥博,乃相與言昌朝建大第,別創客位以待宦官,宦官有矯制者,樞密院釋不治。遂以鎮安軍節度使、右僕射、檢校太師、侍中兼充景靈宮使,出判許州。又以保平軍節度、陜州大都督府長史移大名府兼安撫使。英宗即位,徙鳳翔節度使,加左僕射、鳳翔尹,進封魏國公。治平元年,以侍中守許州,力辭弗許。明年,以疾留京師,
 乃以左僕射、觀文殿大學士判尚書都省,卒,年六十八,謚曰文元。御書墓碑曰「大儒元老之碑」。所著《群經音辨》、《通紀》、《時令》、《奏議》、《文集》百二十二卷。



 昌朝在侍從,多得名譽。及執政,乃不為正人所與,而數有攻其結宦官、宮人者。初,昌朝侍講時,同王宗道編修資善堂書籍,其實教授內侍,諫官吳育奏罷之。及張方平留唐詢,而詢譖育,世以為昌朝指也。然言者謂昌朝釋宦官矯制,後驗問無事實云。



 子章,館閣校勘,蚤世。青,朝請大夫。弟昌衡。



 昌衡字子平。舉進士,為梓州路轉運判官。賈人請富順井鹽,吏視賄多寡為先後,昌衡一隨月日給之。瀘州邊夷蠻,故時守以武吏,昌衡請由東銓調選。蠻驅馬來市,官第其良駑為二等,上者送秦州,下者輒輕估直而抑買,昌衡請嚴禁之。徙提點淮南刑獄、廣東轉運使,徙兩浙路。



 熙寧更法度,核吏治,昌衡數以利害聞,神宗獎其論奏忠益。召為戶部副使、提舉市易司,課羨,增秩右諫議大夫,加集賢殿修撰、知河南府,歷陳、鄆、應天府、鄧州。
 以正議大夫致仕,卒。從子炎。



 炎字長卿,以昌朝蔭,更歷筦庫,積遷至工部侍郎。政和中,以顯謨閣待制知應天府,徙鄆州、永興。初,陜西行鐵錢久,幣益輕。蔡京設法盡斂之,更鑄夾錫錢,幣稍重。京去相,轉運使李譓、陳敦復見所斂已多,遽請罷鑄。鐵錢既復行,其輕加初,自關以西皆罷市,民不聊生。炎獨一切弛禁,聽從其便。其後,宣徽使童貫又以兩者重輕相形,遂盡廢夾錫不得用,民益以為苦。炎徙知延安,因表
 言:「錢法屢變,人心愈惑。今人以為利者,臣見其害;以為是者,臣見其非。中產之家,不過畜夾錫錢一二萬,既棄不用,則惟有守錢而死耳。邊氓生理蕭條,官又一再變法,鄜延去敵迫近,民殊不安。民不安則邊不可守,願得內郡以養母。」乃命為穎州,未行,復留。又與貫制疆事不合,貫沮之,改河陽,又改鄧州。加直學士、知永興。入對,留為工部侍郎。貫簽書樞密院河西、北兩房,侍從邀炎俱往賀,炎曰:「故事無簽書兩房者,彼非執政,何賀為?」會以
 疾卒,年五十八。贈銀青光祿大夫。



 昌朝伯祖父琰。琰字季華,晉中書舍人、給事中緯之子也。以蔭授臨淄、雍丘主簿,歷通判澧州。太宗尹京,奏以為開封府推官,加左贊善大夫。及即位,超拜左正議大夫、樞密直學士。未幾,擢三司副使。太平興國二年,卒。



 琰風神峻整,有吏乾,佐太宗居幕府凡五年,勤於所職。昆弟五人,琰最幼,及琰歷官而諸兄相繼死。琰拊循孤幼,聚族凡百口,分給衣食,庭無間言,士大夫以此稱之。



 琰子湜、汾。湜至軍
 器庫使。交址黎桓之篡丁璇也,朝廷以孫全興將兵討焉。湜與王僎同掌軍事,黎桓偽降,全興信之,軍遂北,湜、僎並坐失律誅。汾至殿中丞。湜子昌符,賜同學究出身。汾子昌齡,第進士,為屯田員外郎。



 梁適,字仲賢,東平人,翰林學士顥之子也。少孤,嘗輯父遺文及所自著以進,真宗曰:「梁顥有子矣。」授秘書省正字。為開封工曹,知昆山縣。徙梧州,奏罷南漢時民間折稅。更舉進士,知淮陽軍,又奏減京東預買紬百三十萬。
 論景祐赦書不當錄朱梁後,仁宗記其名,尋召為審刑詳議官。



 梓州妖人白彥歡依鬼神以詛殺人,獄具,以無傷讞。適駁曰:「殺人以刃或可拒,而詛可拒乎?是甚於刃也。」卒論死。有鳥似鶴集端門,稍下及庭中,大臣或倡以為瑞,適曰「此野鳥入宮庭耳,何瑞之云?」



 嘗與同院燕肅奏何次公案,帝顧曰:「次公似是漢時人字。」肅不能對,適進曰:「蓋寬饒、黃霸皆字次公。」帝悅,因詢適家世,益器之。他日宰相擬適提點刑獄,帝曰:「姑留之,俟諫官有闕,可
 用也。」遂拜右正言。



 林瑀由中旨侍講天章閣,適疏其過。又言:「夏守贇為將無功,不宜復典宥密。」會婦黨任中師執政,以嫌改直史館,修起居注。奉使陜西,與範仲淹條邊機十餘事。進知制誥、權發遣開封府。歲餘,出知兗州。萊蕪冶鐵為民病,當役者率破產以償,適募人為之,自是民不憂冶戶,而鐵歲溢。再遷樞密直學士、知延州。告歸治葬,過京師,得入見,自言前為朋黨擠逐,留為翰林學士。御史交劾之,以侍讀學士知澶州,徙秦州。入知審
 刑院,擢樞密副使。



 張堯佐一日除四使,言者爭之力,帝頗怒。適曰:「臺諫論事,職耳。堯佐恩實過,恐非所以全之。」遂奪二使。儂智高入寇,移嫚書求邕、桂節度,帝將受其降。適曰:「若爾,嶺外非朝廷有矣。」乃遣狄青討之。賊平,帝曰:「向非適言,南方安危,未可知也。」遷參知政事。契丹欲易國書稱南北朝,適曰:「宋之為宋,受之於天,不可改也。契丹亦其國名,自古豈有無名之國哉?」遂止。進同中書門下平章事、集賢殿大學士。大璫王守忠求為節度使,
 適持不可;張貴妃治喪皇儀殿,又以為不可。將以適為園陵使,適言國朝以來無此制,由是浸與陳執中不合。



 適曉暢法令,臨事有膽力,而多挾智數,不為清議所許。御史馬遵、吳中復極論其貪黷怙權,罷知鄭州。京師茶賈負公錢四十萬緡,鹽鐵判官李虞卿案之急,賈懼,與吏為市,內交於適子弟,適出虞卿提點陜西刑獄。及罷,帝即還虞卿三司。復加觀文殿大學士、知秦州。古渭初建砦,間為屬羌所鈔,益兵拒守,羌復驚疑。適具牛酒,召
 諭其種人,且罷所益兵,羌不為患。徙永興軍。夏人盜耕屈野河西田累年,朝廷欲正封,以適為定國軍節度使、知並州,至則悉復侵地六百里。還,知河陽,領忠武、昭德二鎮、檢校太師,復為觀文殿大學士,以太子太保致仕,進太傅。熙寧三年,卒,年七十。贈司空兼侍中,謚曰莊肅。



 孫子美,紹聖中,提舉湖南常平。時新復役法,子美先諸路成役書,就遷提點刑獄。建中靖國初,除尚書郎中,中書舍人鄒浩封還之,改京西轉運副使。諫議大夫陳次
 升又言:「子美緣章惇姻家,連使湖外,承迎其旨意,一時逐臣在封部者,多被其虐,不宜使在近畿。」及徙成都路,累遷直龍圖閣、河北都轉運使,傾漕計以奉上,至捐緡錢三百萬市北珠以進。崇寧間,諸路漕臣進羨餘,自子美始。北珠出女真,子美市於契丹,契丹嗜其利,虐女真捕海東青以求珠。兩國之禍蓋基於此,子美用是致位光顯。



 宣和四年,以疾罷為開府儀同三司、提舉嵩山崇福宮,卒,贈少保。子美為郡,縱侈殘虐,然有幹才,所至辦
 治雲。



 論曰:此五人者,皆以文吏為宰相。執中建儲一言,適契上意,不然,何超遷之驟也。然與劉沆皆寡學少文,希世用事。馮拯議論多迎合王意,昌朝明經術而尚阿私,梁適曉法令而挾智術,斯君子所不與也。若執中不受私謁,沆臨事強果,拯從容一言免謂於誅死,此又足稱者焉



\end{pinyinscope}