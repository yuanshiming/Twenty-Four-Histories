\article{列傳第八十三}

\begin{pinyinscope}

 石普張孜許懷德李允則張亢兄奎劉文質子渙滬趙滋



 石普,其先幽州人,自言唐河中節度雄之後,徙居太原。祖全,事周為鐵騎軍使。父通,事太宗於晉邸。普十歲給
 事邸中,以謹信見親,補寄班祗候,再遷東頭供奉官。賊邢橐駝、賈禿指數百人寇掠永興諸縣,命普督兵往捕,悉獲之。遷內殿崇班、帶御器械。李順叛,普為西川行營先鋒,與韓守英、馬知節誅斬之。遷西京作坊使、欽州刺史。順餘黨復寇邛蜀,偽稱邛南王。又為西川都提舉捉賊使。時蜀民疑不自安,多欲為盜者,普因馳入對,面陳:「蜀亂由賦斂苛急,農民失業,宜稍蠲減之,使自為生,則不討而自平矣。」帝許之。普即日還蜀,揭榜諭之,莫不悅
 服。賊平,賜白金三千兩、襲衣、金帶、鞍勒馬。累遷洛苑使、富州團練使、延州緣邊都巡檢使。羌酋乜羽內寇,普追殺之。



 從真宗幸大名,會王均叛,以為川峽路招安巡檢使,佐雷有終率諸將進討。至天回鎮,賊出拒戰,普領前陣力擊破之。賊退保益州,王師圍城數月不下,普繕車炮,又為地道攻城。城破,均夜半突圍,由南門遁,普引兵追擊於富順監,均自殺,餘黨皆平。遷冀州團練使,賜黃金三百兩、白金三千兩。故事,正任不兼帶御器械,帝特
 以命普。



 契丹犯邊,為保州兵馬鈐轄、北面行營押策先鋒,與契丹戰廉良城,又戰長城口,獲俘馘器甲甚眾。徙定州路副都總管。靈州失守,益兵備關中,徙永興軍副都總管。時軍制疏略,凡號令進退,及呼召將佐、會合別屯,皆遣人馳告。普上請曰:「臣嘗將兵,輒破一錢,與別將各持半,用相合為信。」帝為置傳信牌,漆木長六寸,闊三寸,腹背刻字而中分之,置鑿枘令可合。又穿二竅,容筆墨,上施紙札,每臨陣則分持,或傳令則書其言,系軍吏
 之頸,至彼為合契。又獻《御戎圖》,請設塹以陷敵馬,並上所置戰械甚眾。徙為莫州總管。



 初,契丹南侵,敗我兵於望都。既而諜者言復欲大入寇,帝自畫軍事,以手詔示輔臣曰:



 鎮、定、高陽三路兵宜會定州,夾唐河為大陣,立柵以守。量寇遠近出軍。俟敵疲則先鋒出致師,用騎卒居中,環以步卒,接短兵而已,無遠離隊伍。



 又分兵出三路:以六千騎屯威虜軍,魏能、白守素、張銳領之;五千騎屯保州,楊延昭、張禧、李懷岊領之;五千騎屯北平塞,田
 敏、楊凝、石延福領之,以當賊鋒。始至勿輕鬥,待其氣衰,背城以戰。若南越保州,與大軍遇,則令威虜之師與延昭會,使腹背受敵。若不攻定州,縱軼南侵,則復會北平田敏,合勢入契丹界,邀其輜重,令雄、霸、破虜已來,互為聲援。



 又命孫全照、五德鈞、裴自榮將兵八千屯寧邊軍,李重貴、趙守倫、張繼旻將兵五千屯邢州,扼東西路。契丹將遁,則令定州大軍與三路騎兵會擊之,令普統軍一萬於莫州,盧文壽、王守俊監之,敵騎北去,則西趨順
 安軍襲擊,斷西山之路。如河冰已合,敵由東路,則劉用、劉漢凝、田思明以兵五千會普、全照為掎角,仍命石保吉將萬兵鎮大名,以張軍勢。



 繢圖以授諸將。



 後數月,敕輔臣曰:「北邊已屯大兵,而邊奏至,敵未有釁,且聚軍虛費,民力何以給之?」宜有制畫,以為控遏。且靜戎、順安軍界,先開營田、河道,可以扼黑盧口、三臺、小李路,亦可通漕運至邊。宜乘此用眾浚治,使及軍城,彼或撓吾役,即合兵擊之。」李沆等曰:「設險以制敵,守邊之利也。」遂詔內侍
 閻文慶與靜戎、順安知軍事王能、馬濟督其事,而徙普屯順安之西,與威虜魏能、保州楊延昭、北平田敏為掎角。



 內侍馮仁俊掌御劍於莫州,與普不葉。帝曰:「勿窮治以驕將帥。」第召仁俊還。又令普率所部屯乾寧軍,復遷普冀州團練使,徙本州總管。車駕幸澶淵,時王繼忠已陷契丹,契丹欲請和,因繼忠遣人持信箭為書遺普,且通密表。事平,遷容州觀察使。向敏中為鄜延路都總管,以普副之。趙德明納款,詔降制命,普言:「不宜授以押蕃
 落使,使之總制屬羌,則強橫不可制矣。」乃止兼管內蕃落使。



 未幾,徙並代路,給公使錢二千五百緡,普援例歲給錢三千緡,樞密院言無此例。又言李漢超守河朔時,歲給以萬計,今並代屯軍多,不足以犒軍,帝不納。改桂州觀察使、鎮州路總管,遷保平軍節度觀察留後,赴本鎮。帝祀汾陰,還至陜西,普請駐蹕城中。因賜詩,令扈從至西京。拜河西軍節度使、知河陽,徙許州。築大流堰,引河通漕京師。上《軍儀條目》二卷、《用將機宜要訣》二圖。時
 方崇尚符瑞,而普請罷天下醮設,歲可省緡錢七十餘萬,以贍國用,繇是忤帝意。



 大中祥符九年,上言九月下旬日食者三;又言:「商賈自秦州來,言唃廝囉欲陰報曹瑋,請以臣所獻陣圖付瑋,可使瑋必勝。」帝以普言逾分,而樞密使王欽若言普欲以邊事動朝廷,帝怒,命知雜御史呂夷簡劾之。獄具,集百官參驗,九月下旬日不食。坐普私藏天文,下百官雜議,罪當死。議以官當。詔除名,貶賀州,遣使縶送流所。帝謂輔臣曰:「普出微賤,性輕躁,
 干求不已。既懵文藝,而假手撰述,以揣摩時事。聞在系所思其幼子,時時泣下,可聽挈家以行。」甫至賀州,授太子左清道率府副率、房州安置,增房州屯兵百人護守。



 稍復為左千牛衛將軍,其妻表求普領小郡,遷左領軍衛大將軍。仁宗即位徙安州,遷左屯衛大將軍,徙蔡州。坐失保任,降本衛將軍。歷遷左千牛、左領軍衛大將軍,起知信陽軍,徙光州。以私用孔子廟錢,貶太子左監門率府副率,滁州安置。以左衛將軍分司西京,給官第居
 蔡州,遷大將軍,卒。



 普倜儻有膽略,凡預討伐,聞敵所在,即馳赴之。兩平蜀盜,大小數十戰,摧鋒與賊角,眾推其勇。頗通兵書、陰陽、六甲、星歷、推步之術。太宗嘗曰:「普性剛驁,與諸將少合。」然藉其善戰,每厚遇之。後以罪廢,每太宗忌日,必盡室詣佛寺齋薦,率以為常。



 張孜,開封人。母微時生孜,後入宮乳悼獻太子。孜方在襁褓,真宗以付內侍張景宗曰:「此兒貌厚,汝謹視之。」景宗遂養以為子。蔭補三班奉職、給事春坊司,轉殿直。



 皇
 太子即位,遷供奉官、閣門祗候。為陳州兵馬都監,築堤袁家曲捍水,陳以無患。五遷至供備庫使,領恩州團練使、真定路兵馬鈐轄,歷知莫、貝、瀛三州。轉運使名張溫之奏罷冀、貝驍捷軍士上關銀、□奚錢,事下孜議,孜言:「此界河策先鋒兵,有戰必先登,故平時賜予異諸軍,不可罷。」溫之猶執不已,遂奏罷保州雲翼別給錢糧,軍怨果叛。



 契丹欲背盟,富弼往使,命孜為副,議論雖出弼,然孜亦安重習事。以勞遷西上閣門使、知瀛州,拜單州團練使、龍
 神衛四廂都指揮使、並代副總管。河東更鐵錢法,人情疑貳,兵相率扣府欲訴,閉門不納。是日幾亂,孜策馬從數卒往諭之,皆散還營。遷濟州防禦使、侍衛馬軍都虞候,又遷殿前都虞候,加桂州管內觀察使,遷侍衛步軍副都指揮使。虎翼兵教不中程,指揮使問狀,屈強不肯對,乘夜,十餘人大噪,趣往將害人,孜禽首惡斬之然後聞。遷昭信軍節度觀察留後、馬軍副都指揮使。



 孜長於宮禁中,內外頗涉疑似,言者請罷孜兵柄,乃出為寧遠
 軍節度使、知潞州,徙陳州。仁宗以其無他,復召為馬軍副都指揮使。御史中丞韓絳又言:「孜不當典兵,而宰相富弼薦引之,請黜弼。」弼引咎求罷政事。諫官御史皆言進擬不自弼。絳家居待罪,曰:「不敢復稱御史矣。」坐此謫知蔡州。而孜尋以罪罷,知曹州。卒,贈太尉,謚勤惠。孜初名茂實,避英宗舊名,改「孜」云。



 許懷德,字師古,開封祥符人。父均,磁州團練使。懷德長六尺餘,善騎射擊刺。少以父任為東西班殿侍,累擢至
 殿前指揮使、左班都虞候。



 元昊寇邊,選為儀州刺史、鄜延路兵馬鈐轄,遷副總管。夏人三萬騎圍承平砦,懷德時在城中,率勁兵千餘人突圍,破之。夏人復陣,有出陣前據鞍嫚罵者,懷德引弓一發而踣,敵乃去。屠金明縣,復進圍延州。懷德遽還,夜遣裨將以步騎千餘人,出不意擊之,斬首二百級,遂解延州。遷鳳州團練使,專領延州東路茭村一帶公事。



 徙秦鳳路,未行,坐夏人破塞門砦不赴援,降寧州刺史。頃之,擢龍神衛四廂都指揮使、
 陵州團練使、本路副都總管。遷康州防禦使,又坐當出討賊逗留不進,所部兵夫棄隨軍芻糧,更赦,徙秦鳳路副都總管,改捧日、天武四廂。又以賊侵掠屬羌,亡十餘帳,徙永興軍,又徙高陽關、並代路,歷殿前都虞候、遂州觀察使、侍衛親軍馬軍副都指揮使、武信軍節度觀察留後、殿前副都指揮使、寧遠軍節度使。會從妹亡,無子,懷德欲冒有其田,事覺,罷管軍,知亳州,徙徐州。歲餘,復為殿前副都指揮使。祀明堂,進都指揮使,更保寧、建雄
 二節度。



 年八十猶生子,筋力過人。在宿衛十四年,數乞身,帝不許。懷德曰:「臣年過矣,倘為御史所彈,且不得善罷。」即詔為減數歲。卒,贈侍中,謚榮毅。



 懷德自初擢守邊,連以畏懦被謫,已而與功臣並進典軍,及坐請托得罪,去而復還。時遭承平,保寵終祿。故事,節度使移鎮加恩,皆別上表再辭,每降批答,遣內侍繼賜,必有所遺。懷德以祫享加恩,既又移鎮,乃共為一表以辭。翰林學士歐陽修劾其慢朝命,詔以修章示之,懷德謝罪而已,不復
 別進表。其鄙吝如此。



 李允則,字垂範,濟州團練使謙溥子也。少以材略聞,蔭補衙內指揮使,改左班殿直。太平興國七年,幽薊還師,始置榷場於靜戎軍,允則典其事。還,使河東路決系囚,原治逋欠。又使荊湖察官吏,與轉運使檢視錢帛、器甲、刑獄,遂擢閣門祗候。浚治京師諸河,創水門,鄭州水磑。西川賊劉旰平,上官正議修城未決,命允則與王承衎、閻承翰往視。還,言西川以無城難守,宜如正議。又言兵
 分則緩急不為用,請人並屯要害,以便饋餉。高溪州蠻田彥伊入寇,遣詣辰州,與轉運使張素、荊南劉昌言計事。允則以蠻徼不足加兵,悉招輯之。



 累遷供備庫副使、知潭州。將行,真宗謂曰:「朕在南衙,畢士安嘗道卿家世,今以湖南屬卿。」初,馬氏暴斂,州人出絹,謂之地稅。潘美定湖南,計屋輸絹,謂之屋稅。營田戶給牛,歲輸米四斛,牛死猶輸,謂之枯骨稅。民輸茶,初以九斤為一大斤,後益至三十五斤。允則請除三稅,茶以十三斤半為定制,民
 皆便之。湖湘多山田,可以藝粟,而民惰不耕。乃下令月所給馬芻,皆輸本色,繇是山田悉墾。湖南饑,欲發官廩先賑而後奏,轉運使執不可,允則曰:「須報逾月,則饑者無及矣。」明年薦饑,復欲先賑,轉運使又執不可,允則請以家資為質,乃得發廩賤糶。因募饑民堪役者隸軍籍,得萬人。轉運使請發所募兵禦邵州蠻,允則曰:「今蠻不攪,無名益戍,是長邊患也。且兵皆新募,饑瘠未任出戍。」乃奏罷之。陳堯叟安撫湖南,民列允則治狀請留,堯叟
 以聞。召還,連對三日,帝曰:「畢士安不謬知人者。」



 遷洛苑副使、知滄州。允則巡視州境,浚浮陽湖,葺營壘,官舍間穿井。未幾,契丹來攻,老幼皆入保而水不乏,斫冰代炮,契丹遂解去。真宗復召謂曰:「頃有言卿浚井葺屋為勞民者,及契丹至,始見善為備也。」轉西上閣門副使、鎮定高陽三路行營兵馬都監,押大陣東面。請對,自陳武藝非所長,不可以當邊劇。帝曰:「卿為我運籌策,不必當矢石也。」賜白金二千兩,副以幃幄、什器,凡下諸路宣敕,必
 先屬允則省而後行。及王超敗,人心震搖,允則勸超衰絰向師哭,以解眾忿。真宗知允則始屢趣超進兵,手詔褒厲。



 契丹通好,徙知瀛州,上言:「朝廷已許契丹和議,但擇邊將,謹誓約,有言和好非利者,請一切斥去。」真宗曰:「茲朕意也。」遷西上閣門副使。何承矩為河北緣邊安撫、提點榷場,及承矩疾,詔自擇代,乃請允則知雄州。初,禁榷場通異物,而邏者得所易鈱玉帶。允則曰:「此以我無用易彼有用也,縱不治。」遷東上閣門使、獎州刺史。河北
 既罷兵,允則治城壘不輟,契丹主曰;「南朝尚修城備,得無違誓約乎?」其相張儉曰;「李雄州為安撫使,其人長者,不足疑。」既而有詔詰之,允則奏曰:「初通好不即完治,恐他日頹圮因此廢守,邊患不可測也。」帝以為然。



 城北舊有徹城,允則欲合大城為一。先建東嶽祠,出黃金百兩為供器,道以鼓吹,居人爭獻金銀。久之,密自徹去,聲言盜自北至,遂下令捕盜,三移文北界,乃興版築,揚言以護祠。而卒就關城浚壕,起月堤,自此徹城之人,悉內城
 中。始,州民多以草覆屋,允則取材木西山,大為倉廩營舍。始教民陶瓦甓,標里閉,置廊市、邸舍、水磑。城上悉累甓,下環以溝塹,蒔麻植榆柳。廣閻承翰所修屯田,架石橋,構亭榭,列堤道,以通安肅、廣信、順安軍。



 歲修禊事,召界河戰棹為競渡,縱北人游觀,潛寓水戰。州北舊多設陷馬坑,城上起樓為斥堠,望十里;自罷兵,人莫敢登。允則曰:「南北既講和矣,安用此為?」命徹樓夷坑,為諸軍蔬圃,浚井疏洫,列畦隴,築短垣,縱橫其中,植以荊棘,而其
 地益阻隘。因治坊巷,徙浮圖北原上,州民旦夕登望三十里,下令安撫司,所治境有隙地悉種榆,久之榆滿塞下。顧謂僚佐曰:「此步兵之地,不利騎戰,豈獨資屋材耶?」



 上元舊不燃燈,允則結彩山,聚優樂,使民夜縱游。明日,偵知北酋欲間入城中觀,允則與同僚伺郊外。果有紫衣人至,遂與俱入傳舍,不交一言,出奴女羅侍左右,劇飲而罷。且置其所乘騾廡下,使遁去,即幽州統軍也。後數日,為契丹所誅。嘗宴軍中,而甲仗庫火。允則作樂行酒
 不輟,副使請救,不答。少頃火熄,命悉瘞所焚物,密遣吏持檄瀛州,以茗籠運器甲。不浹旬,兵數已完,人無知者。樞密院請劾不救火狀,真宗曰:「允則必有謂,姑詰之。」對曰:「兵械所藏,儆火甚嚴,方宴而焚,必奸人所為。舍宴而救,事或不測。」



 又得諜,釋縛厚遇之,諜言燕京大王遣來,因出所刺緣邊金谷、兵馬之數。允則曰:「若所得謬矣。」呼主吏按籍書實數與之。諜請加緘印,因厚賜以金,縱還。未幾,諜遽至,還所與數,緘印如故,反出彼中兵馬、財力、
 地裡委曲以為報。一日,民有訴為契丹民毆傷而遁者。允則不治,與傷者錢二千,眾以為怯。逾月,幽州以其事來詰,答以無有。蓋他諜欲以毆人為質驗,比得報,以為妄,乃殺諜。雲翼卒亡入契丹,允則移文督還,契丹報以不知所在。允則曰:「在某所。」契丹駭,不敢隱,既歸卒,乃斬以徇。歷四方館引進使、高州團練使。天禧二年,以客省使知鎮州,徙潞州。仁宗即位,領康州防禦使。天聖六年,卒。



 允則不事威儀,間或步出,遇民可語者,延坐與語,
 以是洞知人情。訟至,無大小面訊立斷。善撫士卒,皆得其用。盜發輒獲,人亦莫知所由。身無兼衣,食無重羞,不畜資財。在河北二十餘年,事功最多,其方略設施,雖寓於游觀、亭傳間,後人亦莫敢隳。至於國信往來,費用儀式,多所裁定。晚年居京師,有自契丹亡歸者,皆命舍允則家。允則死,始寓樞密院大程官營。



 張亢,字公壽,自言後唐河南尹全義七世孫。家於臨濮。少豪邁有奇節,事兄奎甚謹。進士及第,為廣安軍判官、
 應天府推官。治白沙、石梁二渠,民無水患。改大理寺丞、簽書西京判官事。



 通判鎮戎軍,上言:「趙德明死,其子元昊喜誅殺,勢必難制,宜亟防邊。」因論西北攻守之計,章數十上,仁宗欲用之,會丁母憂。既而契丹聚兵幽、涿間,河北增備,遂起為如京使、知安肅軍。因入對曰:「契丹歲享金帛甚厚,今其主孱而歲歉,懼中國見伐,特張言耳,非其實也。萬一倍約,臣請擐甲為諸軍先。」



 元昊反,為涇原路兵馬鈐轄、知渭州,累遷右騏驥使、忠州刺史,徙鄜
 延路、知鄜州。上疏曰:



 舊制,諸路總管、鈐轄、都監各不過三兩員,餘官雖高,止不過一路。總管、鈐轄不預本路事。今每路多至十四五員,少亦不減十員,皆兼本路分事,不相統制,凡有論議,互報不同。按唐總管,統軍,都統,處置、制置使,各有副貳,國朝亦有經略、排陣使,請約故事,別置使名,每路軍馬事,止以三兩員領之。



 又涇原一路,自總管、鈐轄、都監、巡檢及城砦所部六十餘所,兵多者數千人,少者才千人,兵勢既分,不足以當大敵。若敵以
 萬人為二十隊,多張聲勢以綴我軍,後以三五萬人大入奔突,則何以支?



 又比來主將與軍伍移易不定,人馬強弱,配屬未均。今涇原正兵五萬,弓箭手二萬,鄜延正兵不減六七萬,若能預為團結,明定節制,迭為應援,以逸待勞,則烏合饑餒之眾,豈能窺我淺深乎?請下韓琦、範仲淹分按,逐路以馬步軍八千已上至萬人,擇才位兼高者為總領。其下分為三將:一為前鋒,一為策前鋒,一為後陣。每將以使臣、忠佐三兩人,分屯要害之地,敵
 小入則一將出,大入則大將出。



 又量敵數多少,使鄰路出兵應接,此所謂常山蛇勢也。今萬人已上為一大將,一路又有主帥,延州領三大將,鄜州一大將,保安軍及西路巡檢、德靖砦共為一大將,則鄜延路兵五萬人矣。原渭州、鎮戎軍各一大將,渭州山外及瓦亭各一大將,則涇原路五萬人矣。弓箭手、熟戶不在焉。昨延州之敗,蓋由諸將自守不相應援。請令邊臣預定其法,敵寇某所,則某將為先鋒,某將出某所為奇兵,某將出某所為
 聲援,某城砦相近出敢戰死士某所設覆,都、同巡檢則各扼要害。



 又令鄰路取某路出應,仍潛用旗幟為號。昨劉平救延州,前鋒陷賊者已二千騎,平猶不知。趙瑜部馬軍間道先進,而趙振與王逵趨塞門,至高頭平路,白馬報敵張青蓋駐山東,振麾兵掩襲,乃瑜也。臣在山外策應,未嘗用本指揮旗號,自以五行支干別為引旗。若甲子日本軍相遇,則先見者張青旗,後見者以緋旗應之,此是乾相生,其幹相克及支相生克亦如之。蓋兵馬
 出入,晝則百步之外不能相知,若不預為之號,必誤軍事。國家承平日久,失於訓練,今每指揮藝精者不過百餘人,餘皆瘦弱不可用。且官軍所恃者,步軍與強弩爾。臣知渭州日,見廣勇軍擴弩者三百五十人,引一石二斗者僅百人,餘僅及七八斗,正欲閱習時易為力爾。臣以跳鐙弩試,皆不能張,閱習十餘日,裁得百餘人。又教以小坐法,亦十餘日,又教以帶甲小坐法,五十餘日始能服熟。若安前弊以應新敵,其有必勝之理乎?



 又兵官
 務張邊事,以媒進邀賞,劉平之敗,正繇貪功輕進,鎮戎軍最近賊境,每報賊騎至,不問多寡,凡主兵者皆出,至邊壕則賊已去矣。蓋權均勢埒,各不相下,若不出,則恐得怯懦之罪。且諸路騎兵不能馳險,計其芻粟,一馬之費,可養步軍五人。馬高不及格,宜悉還坊監,止留十之三,餘以步兵代之。又比來禁衛隊長,繇年勞換前班者,或為諸司使副,白丁試武技,亦命以官,而諸路弓箭手生長邊陲,父祖效命,累世捍賊,乃無進擢之路,何以激
 勸邊民?



 竊聞大帥議五路進師,且有用兵以來,屢出無功,若一旦深入,臣切以為未可也。山界諸州城砦,距邊止二三百里,夏兵器甲雖精利,其鬥戰不及山界部族,而財糧又盡出山界。若十月後令諸將分番出界,使夏人不得耕牧。然後出步兵,負十日糧,人日給米一升,馬日給粟四升、草五分,賊界有草地,以半資放牧,亦可減挽運之半。王師既行,使唃廝囉及九姓回紇分制其後,必蕩覆巢穴。



 又言:「陜西民調發之苦,數倍常歲,宜一切權
 罷,令安撫司與逐州長吏減省他役,顓應邊須。及選殿侍軍將各三十人,以駝、騾各二百,留其半河中,以運鄜、延、保安軍軍須,其半留乾州或永興軍,以運環、慶、原、渭、鎮戎軍軍須,分一轉運使專董其事。又鄜州四路半當沖要,嘗以閑慢路遞鋪兵卒之半,貼沖要二路。驛百人,每三人挽小車,載二百五十斤至三百斤,若團人並輦運,邊計亦未至失備,而民力可以寬矣。」



 初,亢請乘驛入對,詔令手疏上之,後多施用。進西上閣門使,改都鈐轄,屯
 延州。又奏邊機軍政措置失宜者十事,言:



 王師每出不利,豈非節制不立,號令不明,訓練不至,器械不精?或中敵詭計,或自我貪功;或左右前後自不相救,或進退出入未知其便;或兵多而不能用,或兵少而不能避;或為持權者所逼,或因懦將所牽;或人馬困饑而不能奮,或山川險阻而不能通:此皆將不知兵之弊也。未聞深究致敗之由而為之措置,徒益兵馬,未見勝術。一也。



 去春敵至延州,諸路發援兵,而河東、秦鳳各逾千里,涇原、環
 慶不減十程。去秋賊出鎮戎,遠自鄜延發兵,千里遠鬥,銳氣已衰,如賊已退,乃是空勞師徒,異時更寇別路,必又如此,是謂不戰而自弊。二也。



 今鄜延副都總管許懷德兼管勾環慶軍馬,環慶副總管王仲寶復兼鄜延,其涇原、秦鳳總管等亦兼鄰路,雖令互相策應,然環州至延州十四五驛,徑赴亦不下十驛;涇原至秦鳳千里,若發兵互援,而山路險惡,人馬之力已竭。三也。



 四路軍馬各不下五六萬,朝廷罄力供億,而邊臣但言兵少,每路
 欲更增十萬人,亦未見功效。且兵無節制一弊,無奇正二弊,無應援三弊,主將不一四弊,兵分勢弱五弊。有此五弊,如驅市人而戰,雖有百萬,亦無益於事。四也。



 古人教習,須三年而後成,今之用兵已三年矣,將帥之材孰賢孰愚,攻守之術孰得孰失,累年敗衄,而居邊要者未知何謀。使更數年未罷兵,國用民力,何以克堪。若因之以饑饉,加之以他寇,則安危之策,未知如何。五也。



 今言邊事者甚眾,朝廷或即奏可,或再詳究以聞,或付有司。
 前條方行,後令即變,胥史有鈔錄之勞,官吏無商略之暇,邊防軍政,一無定制。六也。



 夏竦、陳執中皆朝廷大臣,凡有邊事,當付之不疑。今但主文書、守詔令,每有宣命,則翻錄行下;如諸處申稟,則令候朝旨。如是,則何必以大臣主事?七也。



 前河北用兵,減冗官以省費,今陜西日以增員,如制置青白鹽使副、招撫蕃部使臣十餘員,所占兵士千餘人,請給歲約萬緡。復有都大提舉馬鋪器甲之類,諸州並募克敵、致勝、保捷、廣銳、宣毅等兵,久未
 曾團結訓練,但費軍廩,無益邊備。八也。



 今軍有手藝者,管兵之官,每一指揮,抽占三之一。如延州諸將不出,即有兵二萬,除五千守城之外,其餘止一萬五千。若有警急,三日內不能團集,況四十里外便是敵境,一有奔突,何以備之?九也。



 陜西教集鄉兵,共十餘萬人。市井無賴,名掛尺籍,心薄田夫,豈無奸盜雜於其中?茍無措置,他日為患不細。十也。



 既而復請面陳利害,不報。



 會元昊益熾,以兵圍河外。康德輿無守禦才,屬戶豪乜囉叛去,導
 夏人自後河川襲府州,兵至近道才覺,而蕃漢民被殺掠已眾。攻城不能下,引兵屯琉璃堡,縱游騎鈔麟、府間,二州閉壁不出。民乏飲,黃金一兩易水一杯。時豐州已為夏人所破,麟、府勢孤,朝廷議棄河外守保德軍未果,徙亢為並代都鈐轄、管勾麟府軍馬事。單騎叩城,出所授敕示城上,門啟,既入,即縱民出採薪芻汲澗穀。然夏人猶時出鈔掠,亢以州東焦山有石炭穴,為築東勝堡;下城旁有蔬畦,為築金城堡;州北沙坑有水泉,為築安
 定堡,置兵守之。募人獲於外,腰鐮與衛送者均得。其時禁兵皆敗北,無鬥志,乃募役兵敢戰者,夜伏隘道,邀擊夏人游騎。比明,有持首級來獻者,亢以錦袍賜之,禁兵始慚奮曰:「我顧不若彼乎?」又縱使飲博,方窘乏幸利,咸願一戰。亢知可用,始謀擊琉璃堡,使諜伏敵砦旁草中,見老羌方炙羊髀占吉兇,驚曰:「明當有急兵,且趣避之。」皆笑曰:「漢兒皆藏頭膝間,何敢!」亢知無備,夜引兵襲擊,大破之。夏人棄堡去,乃築宣威砦於步駝溝捍寇路。



 時麟
 州饋路猶未通,敕亢自護賞物送麟州。敵既不得鈔,遂以兵數萬趨柏子砦來邀。亢所將才三千人,亢激怒之曰:「若等已陷死地,前鬥則生,不然,為賊所屠無餘也。」士皆感厲。會天大風,順風擊之,斬首六百餘級,相蹂踐赴崖谷死者不可勝計,奪馬千餘匹。乃修建寧砦。夏人數出爭,遂戰於兔毛川。亢自抗以大陣,而使驍將張岊伏短兵強弩數千於山後。亢以萬勝軍皆京師新募市井無賴子弟,罷軟不能戰,敵目曰「東軍」,素易之,而怯虎翼
 軍勇悍。亢陰易其旗以誤敵,敵果趣「東軍」,而值虎翼卒,搏戰良久,伏發,敵大潰,斬首二千級。不逾月,築清塞、百勝、中候、建寧、鎮川五堡,麟、府之路始通。



 亢復奏:「今所通特一徑爾,請更增並邊諸柵以相維持,則可以廣田牧,壯河外之勢。」議未下,會契丹欲渝盟,領果州團練使、知瀛州。葛懷敏敗,遷四方館使、涇原路經略安撫招討使、知渭州,亢聞詔即行,及至,敵已去。鄭戩統四路,亢與議不合,遷引進使,徙並代副都總管。御史梁堅劾亢出庫
 銀給牙吏往成都市易,以利自入,奪引進使,為本路鈐轄。及夏人與契丹戰河外,復引進使、副都總管,知代州兼河東沿邊安撫事。範仲淹宣撫河東,復奏亢前所增廣堡砦,宜使就總其事。詔既下,明鎬以為不可,屢牒止之。亢曰:「受詔置堡砦,豈可得經略牒而止耶?坐違節度,死所甘心,堡砦必為也。」每得牒,置案上,督役愈急。及堡成,乃發封自劾,朝廷置不問。蕃漢歸者數千戶,歲減戍兵萬人,河外遂為並、汾屏蔽。



 復知瀛州,因言:「州小而人
 眾,緩急無所容,若廣東南關,則民居皆在城中。」夏竦前在陜西,惡亢不附己,特沮其役,然卒城之。加領眉州防禦使,復為涇原路總管、知渭州。會給郊賞,州庫物良而估賤,三司所給物下而估高,亢命均其直,以便軍人。轉運使奏亢擅減三司所估。會竦為樞密使,奪防禦使,降知磁州。御史宋禧繼言亢嘗以庫銀市易,復奪引進使,為右領衛大將軍、知壽州。



 後陜西轉運使言亢所易庫銀非自入者,改將作監、知和州。坐失舉,徙筠州。久之,復
 為引進使、果州團練使,又復眉州防禦使、真定府路副都總管。遷客省使,以足疾知衛州,徙懷州。坐與鄰郡守議河事,會境上經夕而還,降曹州鈐轄。改河陽總管,以疾辭,為秘書監。未幾,復客省使、眉州防禦使、徐州總管,卒。



 亢好施輕財,凡燕犒饋遺,類皆過厚,至遣人貿易助其費,猶不足。以此人樂為之用。同學生為吏部,亢憐其老,薦為縣令。後既為所累,出筠州,還,所薦者復求濟,亢又贈金帛,終不以屑意。馭軍嚴明,所至有風跡,民圖像
 祠之。



 奎字仲野,先亢中進士。歷並、秀州推官,監衢州酒。徐生者毆人至死,系婺州獄,再問輒言冤。轉運使命奎復治。奎視囚籍印窾偽,深探之,乃獄吏竄易,卒釋徐生,抵吏罪,眾驚伏。同時薦者三十九人,改大理寺丞,知合淝縣,徙南充縣。



 以殿中丞通判瀘州,罷歸。會秦州鹽課虧緡錢數十萬,事連十一州。詔奎往按,還奏三司發鈔稽緩,非諸州罪。因言:「鹽法所以足軍費,非仁政所宜行。若不
 得已,令商人轉貿流通,獨關市收其征,上下皆利。孰與設重禁壅閼之為民病?」於是悉除所負。未幾,知江州,徙楚州,遷太常博士,召為殿中侍御史、知滑州,徙邢州。母病,輒割股肉和藥以進,母遂愈。其後母卒,廬於墓,自負土植松柏。



 服終,授度支判官,出為京東轉運使,以侍御史為河東轉運使,進刑部員外郎、知御史雜事。安撫京東,募民充軍凡十二萬,奏州縣吏能否數十人。還為戶部副使。及分陜西為四路,擢天章閣待制、環慶路經略
 安撫招討使、知慶州,以父名餘慶辭,不許。歷陜西都轉運使、知永興軍、河東都轉運使,加龍圖閣直學士,知澶、青、徐、揚等州,再遷吏部郎中。



 時李宥知江寧府,府廨盡焚。諫官言金陵始封之地,守臣視火不謹,宜擇才臣繕治之。遷右諫議大夫、知江寧府。奎簡材料工,一循舊制,不逾時復完。還,判吏部流內銓,徙審官院、知河南府。河南宮闕歲久頗摧圮,奎大加興葺。又按唐街陌,分榜諸坊。初,全義守洛四十年,洛人德之,有生祠。及見奎偉儀
 觀,曰:「真齊王孫也。」因復興齊王祠。歲餘,以能政聞,遷給事中,歸朝。京東盜起,加樞密直學士、知鄆州,數月,捕諸盜,悉平。



 奎治身有法度,風力精強,所至有治跡,吏不敢欺,第傷苛細。亢豪放喜功名,不事小謹。兄弟所為不同如此,然皆知名一時。子燾,龍圖閣直學士。



 劉文質,字士彬,保州保塞人,簡穆皇后從孫也。父審琦,虎牢關使,從討李重進戰死。文質幼從母入禁中,太宗授以左班殿直,遷西頭供奉官、寄班祗侯。帝頗親信之,
 數訪以外事。嘗謂內侍竇神興曰:「文質,朕之近親,又忠謹,其賜白金百斤。」出為兩浙走馬承受公事,擢西京左藏庫副使、岢嵐軍使,賜金帶、名馬。徙知麟州,改麟府濁輪砦兵馬鈐轄。擊蕃酋萬保移,走之。越河破契丹,拔黃太尉砦,殺獲萬計,賜錦袍、金帶。徙知慶州。



 李繼遷入寇,文質將出兵,而官吏不敢發庫錢。乃以私錢二百萬給軍,士皆感奮,遂大破賊。徙涇州,充麟州、清遠軍都監,又破敵於枝子平。咸平中,清遠軍陷,坐逗撓奪官,雷州安
 置。久之,起為太子率府率、杭州駐泊都監。封泰山,以內殿崇班為青、齊、淄、濰州巡檢。進禮賓副使、石隰緣邊同都巡檢使,徙秦州鈐轄。建小落門砦,親率士版築。會李浚知秦州,因就賜白金五百兩。



 天禧中,知代州。先是,蕃部獲逃卒,給絹二匹、茶五斤,卒皆論死。時捕得百三十九人,文質取二十九人,以赦後論如法,餘悉配隸他州。再遷內園使、知邠州,數從曹瑋出戰,築堡障。復徙秦州鈐轄,領連州刺史,再知代州,卒。厚賻其家,官子三人。



 文
 質以簡穆親,又父死事,故前後賜予異諸將。真宗嘗問保塞之舊,文質上宣祖、太祖賜書五函。仁宗亦以書賜之。然性剛,喜評刺短長,於貴近無所避,故不大顯。子十六人,渙、滬皆知名。



 渙字仲章,以父任為將作監主簿,監並州倉。天聖中,章獻太后臨朝久,渙謂天子年加長,上書請還政。後震怒,將黥隸白州,呂夷簡、薛奎力諫得免。仁宗親政,擢為右正言。郭后廢,渙與孔道輔、範仲淹等伏闕爭之,皆罰金。
 會河東走馬承受奏,渙頃官並州,與營妓游,黜通判磁州,尋知遼州。



 夏人叛,朝廷議遣使通河西唃氏,渙請行。間道走青唐,諭以恩信。唃氏大集族帳,誓死捍邊,遣騎護出境,得其誓書與西州地圖以獻。加直昭文館,遷陜西轉運使、由工部郎中知滄州,改吉州刺史,知保州。州自戍卒叛後,兵益驕。渙至,虎翼軍謀舉城叛,民大恐。渙單騎徐叩營,械首惡者歸,斬之,一軍帖服。徙登州,益治刀魚船備海寇,寇不敢犯,詔嘉獎之。



 歷知邢、恩、冀、涇、澶
 五州。恩承賊蹂踐後,渙經理繕葺有敘,兵民犯法,一切用重典,威令大振。治平中,河北地震,民乏粟,率賤賣耕牛以茍朝夕。渙在澶,盡發公錢買之。明年,民無牛耕,價增十倍,渙復出所市牛,以元直與民,澶民賴不失業。歷秦鳳、涇原、真定、定州路總管,四遷至鎮寧軍節度觀察留後。熙寧中,還,為工部尚書致仕。



 渙有才略,尚氣不羈,臨事無所避,然銳於進取。方升拓洮、岷,討安南,渙既老,猶露章請自效,不報。卒,年八十一。



 滬字子浚,頗知書傳,深沉寡言,有知略。以蔭補三班奉職,累遷右侍禁。康定中,為渭州瓦亭砦監押,權靜邊砦,擊破黨留等族,斬一驍將,獲馬牛橐駝萬計。時任福敗,邊城晝閉,居民畜產多為賊所掠,滬獨開門納之。



 遷左侍禁,韓琦、範仲淹薦授閣門祗候。又破穆寧生氐。西南去略陽二百里,中有城曰水洛,川平土沃,又有水輪、銀、銅之利,環城數萬帳,漢民之逋逃者歸之,教其百工商賈,自成完國。曹瑋在秦州,嘗經營不能得。滬進城章川,
 收善田數百頃,以益屯兵,密使人說城主鐸廝那令內附。會鄭戩行邊,滬遂召鐸廝那及其酋屬來獻結公、水洛、路羅甘地,願為屬戶。戩即令滬將兵往受地。既至而氐情中變,聚兵數萬合圍,夜縱火呼嘯,期盡殺官軍。滬兵才千人,前後數百里無援,滬堅臥,因令晨炊緩食,坐胡床指揮進退,一戰氐潰,追奔至石門,酋皆稽顙請服。因盡驅其眾隸麾下,以通秦、渭之路。又敗臨洮氐於城下。遷內殿崇班。



 戩以三將兵遣董士廉助築城,功未半,
 會戩罷四路招討使,而涇原路尹洙以為不便,令罷築,且召滬,不聽,日增版趣役。洙怒,使狄青械滬、士廉下獄。氐眾驚,收積聚、殺吏民為亂,朝廷遣魚周詢、程戡往視,氐眾詣周詢,請以牛羊及丁壯助工役,復以滬權水洛城砦主。城成,終以違本路安撫使節制,黜一官,為鎮戎軍西路都巡檢。復內殿崇班,瘍發首,卒。弟淵將以其柩東歸,居人遮道號泣請留,葬水洛,立祠城隅,歲時祀之。



 經略司言,得熟戶蕃官牛裝等狀,願得滬子弟主其城。
 乃命其弟淳為水洛城兵馬監押,城中有碑記滬事。



 趙滋,字子深,開封人。父士隆,天聖中,以閣門祗候為邠寧環慶路都監,戰沒。錄滋三班奉職。滋少果敢任氣,有智略。康定初,以右侍禁選捕京西叛卒有功,遷左侍禁,後為涇原儀渭、鎮戎軍都巡檢。會渭州得勝砦主姚貴殺監押崔絢,劫宣武神騎卒千餘人叛,攻羊牧隆城。滋馳至,諭降八百餘人,貴窮,走出砦。招討使令滋給賜降卒及遷補將吏,滋以為如是是誘其為亂,藏其牒不用,
 還,為招討使所怒,故賞弗行。



 範仲淹、韓琦經略陜西,舉滋可將領,得閣門祗候,為鎮戎軍西路都巡檢。時京西軍賊張海久未伏誅,命滋都大提舉陜西、京西路捉賊,數月賊平。後為京東東路都巡檢。富弼為安撫使,舉再任登州。乳山砦兵叛,殺巡檢,州將誅首惡數人,不窮按。滋承檄驗治,馳入其壘,次第推問,得黨與百餘人付獄,眾莫敢動。



 在京東五年,數獲盜,不自言,弼為言,乃自東頭供奉官超授供備庫副使、定州路駐泊都監。嘗因給
 軍食,同列言粟不善,滋叱之曰:「爾欲以是怒眾耶?使眾有一言,當先斬爾以徇。」韓琦聞而壯之,以為真將帥材。及琦在河東,又奏滋權並代路鈐轄,改管勾河東經略司公事。建言:「代州、寧化軍有地萬頃,皆肥美,可募人田作,教戰射,為堡砦。」人以為利。



 累遷西上閣門副使,歷知安肅軍、保州。滋強力精悍,有吏能,所至稱治。會契丹民數違約,乘小舟漁界河中,吏憚生事,累歲莫敢禁。後又遣大舟十餘,自海口運鹽入界河。朝廷患之,以滋可任,
 徙知雄州。滋戒巡兵,舟至,輒捕其人殺之,輦其舟,移文還涿州,漁者遂絕。契丹因使人以為言,而知瀛州彭思永、河北轉運使唐介燕度,皆以滋生事,請罷之。朝廷更以為能,擢龍神衛四廂都指揮使、嘉州團練使,遷天武、捧日四廂都指揮使。



 英宗即位,領端州防禦使、步軍都虞候,賜白金五百兩,留再任。未幾,卒,贈遂州觀察使。



 滋在雄州六年,契丹憚之。契丹嘗大饑,舊,米出塞不得過三斗,滋曰:「彼亦吾民也。」令出米無所禁,邊人德之。馭軍
 嚴,戰卒舊不服役,滋役之如廂兵,莫敢有言。繕治城壁、樓櫓,至於簿書、米鹽,皆有條法。性尤廉謹,月得公使酒,不以入家。然傲慢自譽,此其短也。



 論曰:石普曉暢軍事,習知民庸,然揣麾時政,終以罪廢。張孜雖稱持重,跡其所長,無足取者。許懷德以懦不任事,數遭貶斥,其不及普遠矣。劉文質以私錢給軍,且脫人於死,仕雖偃蹇,聲名俱章章矣。渙以小官,能抗疏母後,輯暴弭奸,則其餘事也。滬,水洛之戰,從容退師,滬之
 才略,其最優者歟?趙滋有吏能,出米塞下以振契丹,亦仁人之用心。李允則在河北二十年,設施方略,不動聲氣,契丹至以長者稱之。張亢起儒生,曉韜略,琉璃堡、兔毛川之捷,良快人意,區區書生,功名如此,何其壯哉!奎以治跡著稱,其視亢蓋所謂難為兄難為弟者歟?



\end{pinyinscope}