\article{列傳第二百三十五叛臣中}

\begin{pinyinscope}

 ◎叛臣中○李全上



 李全者,濰州北海農家子,同產兄弟三人。全銳頭蜂目,權譎善下人,以弓馬趫捷,能運鐵槍,時號「李鐵槍」。



 初,大
 元兵破中都,金主竄汴,賦斂益橫,遺民保岩阻思亂。於是劉二祖起泰安,掠淄、沂。二祖死,霍儀繼之。彭義斌、石珪、夏全、時青、裴淵、葛平、楊德廣、王顯忠等附之。楊安兒起,掠莒、密,展徽、王敏為謀主,母舅劉全為帥,汲君立、王琳、閻通、董友、張正忠、孫武正等附之,餘寇蜂起。大元兵至山東,全母及其兄死焉。全與仲兄福聚眾數千,劉慶福、國安用、鄭衍德、田四、于洋、洋弟潭等咸附之。



 大元兵退,金乃遣完顏霆為山東行省,黃摑為經曆官,將花帽
 軍三千討之,敗安兒於闌頭滴水,斷其南路。安兒輕舸走即墨,金人募其頭千金,舟人斬以獻。安兒無子,從子友偽稱「九大王」,不閑軍務。安兒妹四娘子狡悍善騎射,劉全收潰卒奉而統之,稱曰「姑姑」,眾尚萬餘,掠食至磨旗山,全以其眾附,楊氏通焉,遂嫁之。全合軍與霆戰,又敗。霆驍將張惠望見全,躍馬赴之,槍及全,若有縶其馬足而止者。全得收餘眾保東海,劉全分軍駐固上。霍儀攻沂州不下,霆自清河出徐州,斬儀,潰其眾。彭義斌歸
 李全。黃摑者,即阿魯達。霆即李二措,賜姓完顏。惠號「賽張飛」,燕俠士也。此數人者,出沒島固,寶貨山委而不得食,相率食人。



 有沈鐸者,鎮江武鋒卒也,亡命盜販山陽,誘致米商,斗米輒售數十倍,知楚州應純之償以玉貨,北人至者輒舍之。又說純之以歸銅錢為名,弛度淮之禁,來者莫可遏。安兒之未敗也,有意歸宋,招禮宋人。定遠民季先者,嘗為大俠劉祐家廝養,隨祐部綱客山陽,安兒見而說之,處以軍職。安兒死,先至山陽,寅緣鐸得
 見純之,道豪傑願附之意。時江、淮制置李玨、淮東安撫崔與之皆令純之沿江增戍,恐不能禦,乃命先為機察,諭意群豪;敘復鐸為武鋒軍副將,辟楚州都監,與高忠皎各集忠義民兵,分二道攻金。先遂以李全五千人附忠皎,合兵攻克海州,糧援不繼,退屯東海。全分兵襲破莒州,禽金守蒲察李家,別將于洋克密州,兄福克青州,始授全武翼大夫、京東副總管。純之見北軍屢捷,密聞於朝,謂中原可復。時頻歲小稔,朝野無事,丞相史彌遠
 鑒開禧之事,不明招納,密敕玨及純之慰接之,號「忠義軍」,就聽節制。於是有旨依武定軍生券例,放錢糧萬五千人,名「忠義糧」。於是東海馬良、高林、宋德珍等萬人輻湊漣水,鐸納之,全與劉全俱起羨心焉。



 嘉定十一年五月己丑,全軍至漣水,邀先白事楚城,取器甲金穀,議再攻海州,純之厚勞全金玉器用及其下有差。六月,全圍海城,金經略阿不罕、納不刺等固守不下。七月,合鄆、單、邳、徐兵來援,全與戰於高橋,不勝,退守石秋,分兵襲密
 州,禽黃摑,械至楚城。是冬,徙屯淮陰之龜山。



 十二年,山東來歸者不止,權楚州梁丙無以贍。先懇丙請預借兩月,然後帥所部五千並良等萬人往密州就食,不許;請速遣全代領其眾,又不許。丙以石珪權軍務,珪乃奪運糧之舟,二月庚辰,率軍二萬度淮大掠。丙調王顯臣、高友、趙邦永以兵逆之,至南度門,顯臣敗,友、邦永遇珪,下馬與作山東語,皆不復戰。丙窘,乃遣全出諭之。時金人圍淮西急,馬司都統李慶宗戍濠,出戰,喪騎三千,珪及
 張春皆有亡失。帥司調全與先、珪軍援盱眙。全亦欲自試,親往東海點軍赴之。癸亥,遇金人於嘉山,戰小捷。三月,先軍進駐天長,全進駐盱眙,鼎立以待金人。乙酉,全至渦口,值金將乞石烈牙吾答名「盧鼓槌」者將濟,全與其將鹿仙掩之,金兵溺淮者數千,俘獲甚眾。壬辰,與阿海戰於化陂湖,大捷,殺金數將,得其金牌,追至曹家莊而還。三圍俱解,全喪失亦眾。阿海者,金所謂四駙馬也。全進達州刺史,妻楊氏封令人。



 六月,金元帥張林以青、
 莒、密、登、萊、濰、淄、濱、棣、寧海、濟南十二州來歸。始,林心存宋,及摑敗,意決而未能達。會全還濰州上塚,揣知林意,乃薄兵青州城下,陳說國家威德,勸林早附。林恐全誘己,猶豫未納。全約挺身入城,惟數人從,林乃開門納之,相見甚歡,謂得所托,置酒結為兄弟。全既得林要領,附表奉十二州版籍以歸。表辭有云:「舉諸七十城之全齊,歸我三百年之舊主。」表,馮垍所作也。秋,授林武翼大夫、京東安撫兼總管,其餘授官有差。進全廣州觀察使、京
 東總管,劉慶福、彭義斌皆為統制,增放二萬人錢糧,徙屯楚州。先是,制置使賈涉以朝命督戰,許殺金太子者,賞節度使;殺親王,承宣使;殺駙馬,觀察使。全致所得金牌於涉,云殺四駙馬所獲者。涉上於朝,乞如約賞之,故全有是受,而四駙馬實不死也。



 十一月,大雨雪,淮冰合。全請於制府曰:「每恨泗州阻水,今如平地矣,請取東西城自效。」制府遣就盱眙劉卓議,卓集諸將燕全,時青、夏全咸願以長槍三千人從。夜半度淮,潛向泗之東城,將
 踏濠冰傅城下,掩金人不備。俄城上荻炬數百齊舉,遙謂曰:「賊李三!汝欲偷城耶?」天黑,故以火燭之。全知有備,引去。



 十三年,趙拱以朝命諭京東,過青厓固,嚴實求內附。拱與定約,奉實款至山陽,舉魏、博、恩、德、懷、衛、開、相九州來歸。涉再遣拱往諭,配兵二千,全亦請往,涉不能止,乃帥楚州及盱眙忠義萬餘人以行。拱說全曰:「將軍提兵度河,不用而歸,非示武也,今乘勢取東平,可乎?」於是全合林軍得數萬,襲東平之城南。金參政蒙古剛帥眾
 守東平,全以三千人金銀甲、赤幟,繞濠躍馬索戰。時大暑,全見城阻水,矢石不能及,乃與林夾汶水而砦,中通浮梁來往。一夕,汶水溢,漂大木,斷浮梁,全首尾幾絕,蓋金人堰汶水而決之也。詰旦,金騎兵三百奄至,全欣然上馬,帥帳前所有騎赴之,殺數人,奪其馬,逐北抵山谷。上有龍虎上將軍者,貫銀甲,揮長槊,盛兵以出,旁有繡旗女將馳槍突鬥。會諸將至,拔全以出,乃退保長清縣,精銳喪失太半,統制陳孝忠死焉。林兵還青州。全所攜
 鎮江軍五百人多怨憤,全乃分隸拱,使先歸,而以餘眾道滄州,假鹽利以慰贍之。龍虎上將軍者,東平副帥幹不搭;女將者,劉節使女也。



 全至楚州,屬召先赴行在。全自渦口之捷,有輕諸將心,獨先嘗策戰勳,威望不下己,患之。乃陰結制帥所任吏莫凱,使譖先,先卒,全喜而心益貳。涉乘先死,欲收其軍,輟統制陳選往漣水以總之。先黨裴淵、宋德珍、孫武正及王義深、張山、張友拒而不受,潛迎石珪於盱眙,奉為統帥。珪道楚城,涉不知覺,及
 選還,涉恥之,乃謀分珪軍為六,請於朝,出修武、京東路鈐轄印告各六授淵等,使之分統,謂可散其縱。淵等陽受命,涉即聞於朝,謂六人已順從,珪無能為矣。其後有教令皆不納,然後知淵等猶主珪,涉恐甚。全結府吏伺知之,乃見涉,請討珪,涉未有處。議者請以全軍布南度門,移淮陰戰艦陳於淮岸,以示珪有備,然後命一將招珪軍,來者增錢糧,不至罷支,眾心一散,珪黨自離。涉用其策,珪技果窮。珪素通好於大元,至是殺淵而挾武正、
 德珍與其謀主孟導歸大元。漣水軍未有所屬,全求並將之。客有請以附淮將者,曰:「使南將主北軍,則淮、楚為一。」涉然之,且曰:「先在時有三千虛籍,今當遣明亮核實,因可省費。」全聞之即獻計曰:「全若朝將此軍,夕與核除虛籍。」因卑辭獻珍具以自結,涉不能卻,遂以付全。翼日,復命曰:「初謂有虛額,昨夕細點,萬五千人之外尚溢十數名。」涉始悟全見紿,他日議更遣幕屬點之。吏亟報全,全忽狀白涉:「昨夕三鼓,漣水告警,云金人萬餘在邳州。
 全思漣水去邳咫尺,既無險阻,城壁復弊,一被攻劫,則直臨淮面,罪在全矣。深夜不敢驚制使,已調七千人迎敵矣。」涉知全詐,因寢點軍之議。全又白制府請於朝,以劉全為總管駐揚州,分數千兵從之,而將其眾。十一月丁未,全遊金山,作佛事,以薦國殤。知鎮江府喬行簡方舟逆之,大合樂以饗之。總領程覃迭為主禮,務訁誇北人以繁盛。全請所狎娼,覃不與,全歸,語其徒曰:「江南佳麗無比,須與若等一到。」始造舭達舟,謀爭舟楫之利焉。



 十四年正月,金人將南來,全請於涉,欲與劉卓共圖泗州,以伐其謀,涉許之。全兵至盱眙度淮,攻克泗州之西城,入城布守。卓徙盱眙芻粟以實之,防城之具俱撤以往,為必守之計。未幾,盧鼓槌來取西城,全盛兵出戰,大敗,統制賴興死,全閉城自守。明日復戰,不勝,全遁歸,資糧器械悉以委敵。金人既陷蘄州,扈再興、趙範及其弟葵邀擊於天長。全隨行襲金人後,謁而賀曰:「二監軍已立大功,乞以餘寇付全追之。」然全追之不甚力,亦以是
 進承宣使。



 十五年二月,卓再取西城,盧鼓槌背城力戰,戒惠必獲全,不獲則斬。惠數嘗敗全於山東,而不能獲,每歎曰:「天假此賊,事未可量。」及聞盧鼓槌言,自度進未必獲,退復受戮,即陳躍馬奔全壁,棄所執兵請降。全掖而起之,相與歡甚。不數日,惠戲下數千人皆潛至,全與惠歸,請於制置司官之,令自總一軍。



 膠西當登、寧海之衝,百貨輻湊,全使其兄福守之,為窟宅計。時互市始通,北人尤重南貨,價增十倍。全誘商人至山陽,以舟浮其
 貨而中分之,自淮轉海,達於膠西。福又具車輦之,而稅其半,然後從聽往諸郡貿易,車、夫皆督辦於林,林不能堪。林財計仰六鹽場,福恃其弟有大造於林,又欲分其半,林許福恣取鹽,而不分場。福怒曰:「若背恩耶?待與都統提兵取若頭爾!」林懼,訴於制置司。涉密召林戲下問之,福伏兵於途以伺,林覺不追。於是李馬兒說林歸大元,福狼狽走楚州。冬,加全招信軍節度。林猶遺涉書詆全,明己非叛。涉以咎全,全請為朝廷取之,乃提師駐海
 州以迫林。涉間道遣黥胥王翊、閻瓊勞林,林泣涕道其故。翊歸,全使人殺諸塗。全攻林急,林走,全遂入青州。



 十六年二月,涉勸農出郊,暮歸入門,忠義軍遮道,涉使人語楊氏,楊氏馳出門,佯怒忠義而揮之,道開,涉乃入城。自是以疾求去甚力。五月被召。卒。秋,全新置忠義軍籍。初,涉屯鎮江副司八千人於城中,翟朝宗統之;分帳前忠義萬人,屯五千城西,趙邦永、高友統之;屯五千淮陰,王暉及於潭統之,所以制北軍也。全輕鎮江兵,且以利
 啖其統制陳選及趙興,使不為己患;唯忌帳前忠義,乃數稱高友等勇,遇出軍必請以自隨,涉不許。全每燕戲下,並召涉帳前將校,帳前亦願隸焉,然未能合也。及丘壽邁攝帥事,全忽請曰:「忠義烏合,尺籍鹵莽。莫若別置新籍,一納諸朝,一申制閫,一留全所,庶功過有考,請給無弊。」壽邁善而諾之。全乃合帳前忠義悉籍之,盡統其軍,時人莫悟。



 十一月,許國自武階換朝議大夫、淮東安撫制置使,命下,聞者驚異。先是,國奉祠家食,數言全必
 反,欲傾涉而代之。會召國奏事,國疏全奸謀甚深,反狀已著,非有豪傑不能消弭,蓋自鬻也。至是,喬行簡為吏部侍郎,上疏論國望輕,不宜帥淮,不報。山陽參幕徐晞稷雅意開閫,及聞國用,晞稷闕望,乃譽國奏注釋以寄全,全得報,不樂。是冬,金將李二措及邳州守致書海州,欲附宋,全戲下周岊得之,即以報全。全喜,遣王喜兒以兵二千應接,而己繼之。二措納喜兒而囚之。全兵欲攻邳,四面阻水,二措積勁弩備之,全不得進,合兵索戰。
 全敗,欲還楚州,會濱、棣有亂,乃引兵趨山東。



 十七年正月,國之鎮,楊氏郊迓,國辭不見,楊氏慚以歸。國既視事,痛抑北軍,有與南軍競者,無曲直偏坐之,犒賚十裁七八。全自山東致書於國,國誇於眾曰:「全仰我養育,我略示威,即奔走不暇矣。」全固留青州,國不能致。四月,全遣小吏致再書,國喜,曲加勞接,即日真補承信郎,冀結其心。小吏曰:「小吏奉書而遽得命,諸將校謂何?」不受,歸語其徒以為笑。國見全無來朝,數致厚饋,邀全議事。會劉
 慶福亦使人覘國意向,國左右知之,語覘者曰:「制置無害汝等意。」慶福以報全,全集將校曰:「我不參制閫,則曲在我。今不計生死必往見。」八月,全上謁,賓讚戒全曰:「節使當庭趨,制使必免禮。」及庭趨,國端坐納全拜,不為止。全退,怒曰:「庭參亦常禮,全歸本朝,拜人多矣,但恨汝非文臣,本與我等。汝向以淮西都統謁賈制帥,亦免汝拜。汝有何勳業,一旦位我上,便不相假借耶?全赤心報朝廷,不反也。」國繼設盛會宴全,遺勞加厚,全終不樂。國之
 客章夢先主幕議,慶福謁見,夢先責客將,令隔簾貌喏,慶福不能堪。國以名馬十餘噭遺全,不受。國固遣,全俟其充斥階庭,伺候移時,而復卻之。如是者半月,卒不受。



 全欲往青州,懼國苛留,自計曰:「彼所爭者拜也,拜而得志,吾何愛焉!」更折節為禮。因會,席間出劄白事,國見其細故,判從之,全即席再拜謝。自是動息必請,得請必拜,國大喜,語家人曰:「吾折伏此虜矣。」義斌求趙邦永來山東,全為白之,國諾。邦永乘間告國曰:「邦永若去,制使誰
 與處?」國曰:「我自能兵,爾毋過慮。」邦永泣而辭之。全遂往青州。十一月,國集兩淮馬步軍十三萬,大閱楚城之外,以挫北人之心。楊氏及軍校留者恐其圖己,內自為備。



 寶慶元年,湖州人潘甫與其從弟丙、壬起兵,密告全黨於山陽,全黨欲坐致成敗,然其謀而不助之力。甫歸,陰勒部曲及聚販鹽盜至千餘,結束如北軍,率眾揚言自山陽來擁立濟王,事見《竑傳》。時全圖國之意已決,遣慶福還楚城,使為亂。或教楊氏畜一妄男子,間指謂人曰:「
 此宗室也。」 至語郡僚曰:「會令汝為朝士。」潛約盱眙四軍相應。忠義統領王文信有眾八百,涉徙刺揚州強勇軍。國之聚兵大閱,文信在焉,慶福與謀,令歸襲揚州,別遣將劫寶應,事濟即揮眾度江。盱眙四將不從,於是慶福等謀中輟,止欲快意於許國焉。計議官苟夢玉知之,以告國,國曰:「但使反,反即殺,我豈文儒不知兵耶?」夢玉懼禍及己,求檄往盱眙,復告慶福曰:「制帥欲圖汝。」兩為自結之計。乙卯,國晨起蒞事,忽露刃充庭,客駭走,國厲聲
 曰:「不得無禮!」矢已及顙,流血蔽面,國走。亂兵悉害其家,大縱火,焚官寺,兩司積蓄盡入賊。親兵數十人翼國登城樓,縋城走,伏道堂中宿焉。時四明人姚翀通判青州,全豫令還山陽,及漣水而復止之。至是,擁翀入城,與通判宋恭喝犒南北軍,使歸營。是日,慶福首殺夢先以報貌喏之辱,戒諸軍毋害苟夢玉家,護以五十兵。初,國倚揚州強勇軍統制彭興及淮西親兵將趙社、朱虎等為腹心,至是首降賊,且助為亂。惟丁勝、張世雄、沈興、杜靖
 毗、富道不屈,或與賊巷戰,興手殺賊將馬良。賊黨得志,更相賀,獨張正忠歎曰:「若曹不識事體,朝廷豈置汝耶?」王文信復獻計慶福曰:「我偽作重傷,提本部軍歸揚州,揚守必不疑,我生縛守,以其城獻。」慶福喜,夜飲而遣之。丙辰,許國縊於途。



 丁巳,文信將至揚州,其徒有亡入城告變者。時揚之兵皆在楚,知州兼提點刑獄汪統會同官議,鈐轄趙拱曰:「若不納,則文信必曰:『我歸營,何故見拒?』將借是以魚肉城外之民。拱素善文信,請說止其兵,
 而以單騎入,俟入城而殺之,然後撫其兵,領往盱眙,分隸張、范戲下。」統喜,遣之。遇文信於十里頭,置酒相勞苦,文信偽為裹創狀。拱曰:「忠義反楚州,揚州人見忠義暮歸,豈不相疑?不若暫駐兵城外,然後同見提刑,提刑急欲知楚州事也。」文信不疑,聯騎入城,坐客次。拱先入,勸統收戮之,統躊躇不敢發。劉全知其謀,帥甲士突入郡堂,厲聲曰:「王統領好人,提刑不必疑,請出受參。 」統不得已,出而犒之。劉全以兵翼之出,館其家。詰旦,統未有處。
 拱又請引文信出城,與議回屯楚州。文信知事泄,拱就出,劉全亦請從。至平山堂,文信責拱賣己,欲殺之,拱曰:「爾謀如此,三城人命何辜!我已存三城人,身死無憾。然我死,汝八百家老幼在城,豈得生耶?」文信及其眾動色,文信、劉全遂還楚州。



 時盱眙總管夏全聞山陽得志,亦懷異圖,劉卓厚賂之,乃止。及文信亂,卓懼夏全復動,乃使卞整將兵三千視之,使不敢動。整以邀文信為辭,引兵還揚州,因偽言盱眙失守,卞整為亂,於是揚州復震,
 城門晝閉。



 彌遠懼激他變,欲姑事涵忍而後圖之。謀帥莫可,以徐晞稷嘗倅楚州、守海州,得全歡心,晞稷亦勇往,乃授淮東制置使,令出屈撫全。時慶福以事濟報全,全又牒義斌等曰:「許國謀反,已伏誅矣,爾軍並聽我節制。」義斌得牒大罵曰:「逆賊背國厚恩,擅殺制使。此事皆因我起,我必報此仇。」呼趙邦永曰:「趙二,汝南人,正須爾明此事。」乃斬齎牒人,南向告天誓眾,見者憤激。全自青州至楚城,佯責慶福不能彈壓,致忠義之哄,斬數人,請
 待罪,朝廷未之詰。趙範時知揚州兼提點刑獄,得制置印於潰卒中,以授晞稷。全遣騎逆晞稷。己卯,晞稷入楚城。劉全躍馬登郡廳,晞稷迎之,全及門下馬,拜庭下,晞稷降等止之,賊眾乃悅。



 四月,潘壬變姓名至楚州,將度淮而北,小校明亮獲之,械送行在伏誅。



 甲午,時青使人偽為金兵,道邳州,出漣水,奪全田租而伏騎八百。翼旦,全引二百騎度淮與鬥。伏發,全敗,圍之,慶福以兵往拔全出。全與慶福俱重傷,歸楚州。丁勝、張世雄欲乘全敗
 舉兵追北軍,晞稷止之。全後知其謀,對晞稷詰之,二人不為屈。然懼禍及己,晞稷乃潛授世雄雄勝軍統制,教使逃而陽索之。北軍追世雄,世雄且戰且走,得達揚州。晞稷初至楚,緩急相濟,如囚趙社,逐朱虎,賊尚知畏。屢令全還戰馬、軍器於制司,全唯唯。退招姚翀及將校飲,酒酣,全曰:「 制司追我戰馬、軍器,若何?」忽有將校曰:「當時忠義隻百十人,其他軍皆南軍乘勢將帶,若潰將何以還?」一人曰:「制司必欲追之,不若有官者棄官,無官者歸山
 東為百姓。」一人抵掌憤然,使全反,全陽罵之。翀以告晞稷。翼日,全見晞稷求納官,晞稷撫之而去。自是不復誰何,其後至以「恩府」稱全、「恩堂」稱楊氏,而手足倒置矣。軍器庫止餘槍幹數千,全復取去。全欲戰艦,晞稷使擇二艘。全移出淮河,使軍習之。



 初,楚城之將亂也,有吏竊許國書篋二以獻慶福,皆機事。慶福賞盜篋者五百千,未之閱。全始發緘,使家僮讀之,有廟堂遺國書令圖全者,全大怒;又有苟夢玉書,即以慶福謀告國者,全始惡夢
 玉反覆。夢玉知之,時已被堂召,亟辭全如京。己卯,全饋餞夢玉如平時,潛殪諸十里之郊,復出榜捕害夢玉者。全往青州。



 五月丁卯,全取東平,不克。戊寅,劉全以券易制司錢,不如欲,復謀亂,楊氏出二千緡解之,乃止。全引兵攻恩州。明日,義斌出兵與全鬥,全敗。義斌以千五百騎追之,獲馬二千匹,皆揚州強勇軍馬也。慶福往救,又敗。全退保山崮,抽山陽忠義以北。楊氏及劉全皆欲親赴之,會全遣人求晞稷書與義斌連和,乃止。義斌納全
 降兵,兵勢大振,進攻真定,降金將武仙,眾至數十萬,致書沿江制置使趙善湘曰:「不誅逆全,恢復不成。但能遣兵扼淮,進據漣、海以蹙之,斷其南路,如此賊者,或生禽,或斬首,惟朝廷所命。賊平之後,收復一京三府,然後義斌戰河北,盱眙諸將、襄陽騎士戰河南,神州可復也。」時四總管亦各遣計議官致書,乞助討賊,範亦以為言,不報。全貽書制置司,誣義斌叛,晞稷繳達之。時朝廷知義斌之功,憚全,未欲行賞。未幾,義斌俟命不至,拓地而北,
 與大元兵戰於內黃之五馬山。大元兵說之降,義斌厲聲曰:「我大宋臣,且河北、山東皆宋民,義豈為他臣屬耶!」遂死之。戲下王義深等復歸全。



 全使人說時青附己,饋金五百兩。青見義斌死,乃附全,自移屯淮陰。全招青入城飲,折俎銅券二千,他饋稱是,恩遍麾下,人人喜悅。晞稷宴青,全饋折俎如前。全將往山東,以南軍九百從,官犒鐵錢券人五千,全犒銅錢三倍,許攜南貨免稅。於是請行者不已,得千人以俱,晞稷又以千八百人繼之。



 二
 年春,趙範奉祠,林珙知揚州、權提點刑獄。全北剽山東,南假宋以疑大元,且仰食。會金與大元爭大名,全得往來經理。三月丙辰朔,大元兵攻青州,全大小百戰,終不利,嬰城自守。大元築長圍,夜布狗砦,糧援路絕。全遣小校周興祖縋城,雜樵采者走楚州發援兵,終不能支。全與福謀,福曰:「二人俱死無益也,汝身係南北輕重,我當死守孤城,汝間道南歸,提兵赴援,可尋生路。」 全曰:「數十萬敵,未易支也。全朝出則城夕陷,不如兄歸。」於是全
 止而福行。



 朝廷初以力未能討,故用晞稷調護,及傳全被圍,稍欲圖賊。晞稷畏懦,幸全未歸以苟歲月。朝廷方謀易帥,劉卓久在盱眙,雅意建閫;又見賊勢稍孤,意功名可立,使鎮江副都統彭忄乇延譽京師,自謂:「素撫鎮江,三萬人足用,且得四總管歡心,討賊有餘力。」朝廷信之,忄乇亦垂涎代卓,從臾尤力。九月,以卓知楚州兼淮東制置使,



 忄乇代知盱眙,晞稷不知也。己亥,晞稷以戶部侍郎召,未幾,出知袁州。



 十一月壬子朔,卓至楚州,心知不能
 制馭四總管,惟以鎮江兵自隨。時青在淮陰,卓怨其移屯叛己,不召也。夏全請從,卓素畏全狡,亦俾留盱眙。忄乇自揣資望視卓更淺,曰:「卓之止夏全,是欲遺患盱眙也。卓猶憚夏全,我何能用?」乃激夏全曰:「楚城賊黨不滿三千,健將又在山東,劉制使圖之,收功在旦夕。太尉曷不往赴事會,何端坐為?」夏全欣然領兵徑入楚城,青亦自淮陰復移屯城內。卓且駭且恐,勢不容卻,復就二人謀焉。時傳全已死,福欲分兵赴援,兵少,卒不往。甲子,卓令
 夏全盛陳兵楚城,賊黨震恐,楊氏遣人賂夏全求緩師,乃



\end{pinyinscope}