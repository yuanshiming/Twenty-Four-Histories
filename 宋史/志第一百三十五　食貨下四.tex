\article{志第一百三十五 食貨下四}

\begin{pinyinscope}

 鹽中



 元豐七
 年,知滄州趙瞻請自大名府、澶、恩、信安、雄、霸、瀛、莫、冀等州盡榷賣以增其利,才半歲,獲息錢十
 有六萬七千緡。哲宗即位,監察御史王巖叟言:「河北二年以來新行鹽法,所在價增一倍,既奪商賈之利,又增居民之價以為息,聞貧家至以鹽比藥。伏惟河朔天下根本,祖宗推此為惠,願陛下不以損民為利,而以益民為利,復鹽法
 如故,以為河北數百萬生靈無窮之賜。」會河北轉運使範子奇奏,鹽稅欲收以
 十分,遣範鍔商度。巖叟復言:「臣在河北,亦知商賈有自請於官,乞罷榷買,願輸倍稅。主計者但知於商賈倍得稅緡以為利,不知商賈將於民間復增賣價以為害也。慶歷六年,既不行三司榷買之法,又不從轉運司增稅之請,仁宗直謂朕慮河北軍民驟食貴鹽,可令依舊。是時計歲增
 幾六十萬緡,仁宗豈不知為公家之利?意謂藏之官不若藏之民。今陛下即位之始,宜法仁宗之意,不宜以小利失人心也。」明年,遂罷河北榷法,仍舊通商。六年,提舉河北鹽稅司請令商賈販鹽,於場務輸稅,以及等戶保任,給小引,量道里為限,即非官監鎮店,聽以使鬻之,鹽稅舊額五分者,增為七分。則鹽稅
 蓋已行焉。



 紹聖中,河北官復賣鹽,繼詔如京東法。元符三年,崇儀使林豫言:「河北榷鹽,未必敷前日稅額,且契丹鹽益售,慮啟邊隙。」明年,給事中上官均亦以為言,皆不果行。宣和元年,京畿、四輔及滑州、河陽所產堿地,悉墾
 為田,革盜刮煎鹽之弊,知河陽王序以勸誘推賞。三年,大改鹽法,舊稅鹽並易為鈔鹽。凡未賣稅鹽鈔引及已請算或到倉已投暨未投者,並赴榷貨務改給新法鈔引,許通販;已請舊法稅鹽貨賣者,自陳,更買新鈔帶賣,已請鈔引,毋得帶支。初,茶鹽用換鈔對帶之法,民旅皆病,然河北猶未及也;至是,並河北、京東行之。



 其在兩浙曰杭州場,歲鬻七萬七千餘石,明州昌國東、西兩監二十萬一千餘石,秀州場二十萬八千餘石,溫州天富南北監、密鸚永嘉二場,七萬四千餘石,臺州黃巖監一萬
 五千餘石,以給本州及越、處、衢、婺州。天聖中,杭、秀、溫、臺、明各監一,溫州又領場三,而一路歲課視舊減六萬八千石,以給本路及江東之歙州。



 慶歷初,制置司言:比年河流淺涸,漕運艱阻,靡費益甚,請
 量增江、淮、兩浙、荊湖六路糶鹽錢。下三司議,三司奏荊湖已嘗增錢,餘四路三十八州軍,請斤增二錢或四錢。詔俟河流通運復故。既而江州置轉運般倉,益置漕船及傭客舟以運,制置司因請六路五十一州軍斤增五錢。民苦官鹽估高,無以為食,諸路皆言其不便。久之,韓絳安撫江南還,亦極言之。其後兩浙轉運使沈立、李肅之奏:「本路鹽課緡錢歲七十九萬,嘉祐三年,才及五十三萬;而一歲之內,私販坐罪者三千九十九人;弊在於
 官鹽估高,故私販不止,而官課益虧。請裁官估,罷鹽綱,令鋪戶衙前自趨山場取鹽,如此則鹽善而估平,人不肯冒禁私售,官課必溢。」發運司難之。立、肅之固請試用其法二三年,可見利害,詔可。



 立嘗論東鹽利害,條亭戶、倉場、漕運之弊,謂:「愛恤亭戶使不至困窮,休息漕卒使有以為生,防制倉場使不為掊克率斂,絕私販,減官估,果能行此五者,歲可增緡錢一二百萬。」集《鹽策》二十卷以進,其言亭戶困乏尤甚。然自皇祐以來,屢下詔書輒
 及之,命給亭戶官本,皆以實錢;其售額外鹽者,給粟帛衣糧;亭戶逋歲課久不能輸者,悉蠲之。所以存恤之意甚厚,而有司罕有承順焉。



 熙寧以來,杭、秀、溫、臺、明五州共領監六、場十有四,然鹽價苦高,私販者眾,轉為盜賊,課額大失。二年,有萬奇者獻言欲撲兩浙鹽而與民,乃遣奇從發運使薛向詢度利害。神宗以問王安石,對曰:「趙抃言衢州撲鹽,所收課敵兩浙路,抃但見衢、湖可撲,不知衢鹽侵饒、信,湖鹽侵廣德、升州,故課可增,如蘇、常
 則難比衢、湖。今宜制置煎鹽亭戶及差鹽地令督捕私販,般運以時,嚴察拌和,則鹽法自舉,毋事改制。」



 五年,以盧秉權發遣兩浙提點刑獄,仍專提舉鹽事。秉前與著作佐郎曾默行淮南、兩浙,詢究利害。異時灶戶鬻鹽,與官為市,鹽場不時償其直,灶戶益困。秉先請儲發運司錢及雜錢百萬緡以待償,而諸場皆定分數:錢塘縣楊村場上接睦、歙等州,與越州錢清場等,水勢稍淺,以六分為額;楊村下接仁和之湯村為七分;鹽官場為八分;
 並海而東為越州餘姚縣石堰場、明州慈溪縣鳴鶴場皆九分;至岱山、昌國,又東南為溫州雙穗、南天富、北天富場為十分;蓋其分數約得鹽多寡而為之節。自岱山以及二天富煉以海水,所得為最多。由鳴鶴西南及湯村則刮堿淋鹵,十得六七。鹽官、湯村用鐵盤,故鹽色青白;楊村及錢清場織竹為盤,塗以石灰,故色少黃;石堰以東近海水堿,故雖用竹盤,而鹽色尤白。秉因定伏火盤數以絕私鬻,自三灶至十灶為一甲,而鬻鹽地什伍
 其民,以相幾察;及募酒坊戶願占課額,取鹽於官賣之,月以錢輸官,毋得越所酤地;而又嚴捕盜販者,罪不至配,雖杖者皆同妻子遷五百里。仍益開封府界、京東兵各五百人防捕。



 時惟杭、越、湖三州格新法不行,發運司劾奏虧課,皆獄治。王安石為神宗言捕鹽法急,可以止刑。久之,乃詔兩浙提舉鹽事司,諸州虧課者未得遽劾,以增虧及違法輕重分三等以聞。七年,以盧秉鹽課雖增,刑獄實繁,慮無辜即罪者眾,徙其職淮南,以江東漕
 臣張靚代之,且休量其事。靚言秉在事,越州監催鹽償至有母殺子者,詔劾其罪,然竟免,仍以增課擢太常博士,升一資。歲餘,三司言兩浙漕司寬弛,鹽息大虧,命著作佐郎翁仲通更議措置。元祐初,言者論秉推行浙西鹽法,務誅剝以增課,所配流者至一萬二千餘人,秉坐降職。兩浙鹽亭戶計丁輸鹽,逋負滋廣,二年,詔蠲之。後更積負無以償,元符初,察訪使以狀聞,有司乃以朝旨不行,右正言鄒浩嘗極疏其害。



 明州鳴鶴場鹽課弗登,
 撥隸越州。宣和元年,樓異為明州,請仍舊,且於接近臺州給舊鹽五七萬囊。詔曰:「明州鹽場三,昨以施置不善,以鳴鶴一場隸越,客始輻湊。猶有二場積鹽以百萬計,未見功緒,此而不圖,東欲取於越,西欲取於臺,改令害法,動搖眾情。」令狀析以聞。



 其在淮南曰楚州鹽城監,歲鬻四十一萬七千餘石,通州豐利監四十八萬九千餘石,泰州海陵監如皋倉小海場六十五萬六千餘石,各給本州及淮南之廬和舒蘄黃州、無為軍,江南之江寧
 府、宣、洪、袁、吉、筠、江、池、太、平、饒、信、歙、撫州、廣德臨江軍,兩浙之常、潤、湖、睦州,荊湖之江陵府、安、復、潭、鼎、岳、鄂、衡、永州、漢陽軍。海州板浦、惠澤、洛要三場歲鬻四十七萬七千餘石,漣水軍海口場十一萬五千餘石,各給本州軍及京東之徐州,淮南之光、泗、濠、壽州,兩浙之杭、蘇、湖、常、潤州、江陰軍。天聖中,通、楚州場各七,泰州場八,海州場二,漣水軍場一,歲鬻視舊減六十九萬七千五百四十餘石,以給本路及江南東西、荊湖南北四路,舊並給兩
 浙路,天聖七年始罷。



 凡鹽之入,置倉以受之,通、楚州各一,泰州三,以受三州鹽。又置轉般倉二,一於真州,以受通、泰、楚五倉鹽;一於漣水軍,以受海州漣水鹽。江南、荊湖歲漕米至淮南,受鹽以歸。東南鹽利,視天下為最厚。鹽之入官,淮南、福建、兩浙之溫、臺、明斤為錢四,杭、秀為錢六,廣南為錢五。其出,視去鹽道里遠近而上下其估,利有至十倍者。



 咸平四年,秘書丞直史館孫冕請:「令江南、荊湖通商賣鹽,緣邊折中糧草,在京入納金銀
 錢帛,則公私皆便,為利實多。設慮淮南因江南、荊湖通商,或至年額稍虧,則國家折中糧草,足贍邊兵;中納金銀,實之官庫;且免和雇車乘,差擾民戶,冒寒涉遠。借如荊湖運錢萬貫,淮南運米千石,以地里腳力送至窮邊,則官費民勞,何啻數倍。」詔吏部侍郎陳恕等議。恕等謂:「江、湖官賣鹽,蓋近鬻海之地,欲息犯禁之人,今若通商,住賣官鹽,立乏一年課額。」冕議遂寢。至天禧初,始募人入緡錢粟帛京師及淮、浙、江南、荊湖州軍易鹽。乾興元年,入
 錢貨京師總為緡錢一百十四萬。會通、泰鬻鹽歲損,所在貯積無幾,因罷入粟帛,第令入錢。久之,積鹽復多。



 明道二年,參知政事王隨建言:「淮南鹽初甚善。自通、泰、楚運至真州,自真州運至江、浙、荊湖,綱吏舟卒,侵盜販鬻,從而雜以沙土。涉道愈遠,雜惡殆不可食,吏卒坐鞭笞,徒配相繼而莫能止。比歲運河淺涸,漕挽不行,遠州村民,頓乏鹽食;而淮南所積一千五百萬石,至無屋以貯,則露積苫覆,歲以損耗。又亭戶輸鹽,應得本錢或無以
 給,故亭戶貧困,往往起為盜賊,其害如此。願權聽通商三五年,使商人入錢京師,又置折博務於揚州,使輸錢及粟帛,計直予鹽。鹽一石約售錢二千,則一千五百萬石可得緡錢三千萬以資國用,一利也;江、湖遠近皆食白鹽,二利也;歲罷漕運糜費,風水覆溺,舟人不陷刑闢,三利也;昔時漕鹽舟可移以漕米,四利也;商人入錢,可取以償亭戶,五利也。」



 時範仲淹安撫江、淮,亦以疏通鹽利為言,即詔知制誥丁度等與三司使、江淮制置使同
 議。皆謂聽通商恐私販肆行,侵蠹縣官,請敕制置司益漕船運至諸路,使皆有二三年之蓄;復天禧元年制,聽商人入錢粟京師及淮、浙、江南、荊湖州軍易鹽;在通、楚、泰、海、真、揚、漣水、高郵貿易者毋得出城,餘州聽詣縣鎮,毋至鄉村;其入錢京師者增鹽予之,並敕轉運司經畫本錢以償亭戶。詔皆施行。景祐二年,諸路博易無利,遂罷,而入錢京師如故。



 康定元年,詔商人入芻粟陜西並邊,願受東南鹽者加數與之。會河北穀賤,三司因請內
 地諸州行三說法,亦以鹽代京師所給緡錢,糴二十萬石止。慶歷二年,又詔:「入中陜西、河東者持券至京師,償以錢及金帛各半之;不願受金帛者予茶鹽、香藥,惟其所欲。」而東南鹽利厚,商旅皆願得鹽。八年,河北行四說法,鹽居其一,而並邊芻粟,皆有虛估,騰踴至數倍。券至京師,反為蓄賈所抑,鹽百八斤舊售錢十萬,至是六萬,商人以賤估售券取鹽,不復入錢京師,帑藏益乏。皇祐二年,復入錢京師法,視舊錢數稍增予鹽,而並邊入中
 先得券受鹽者,河東、陜西入芻粟直錢十萬,止給鹽直七萬河北又損為六萬五千,且令入錢十萬於京師,乃聽兼給,謂之對貼,自是入錢京師稍復故。



 初,天聖九年,三司請榷貨務入錢售東南鹽,以百八十萬三千緡為額,後增至四百萬緡。嘉祐中,諸路漕運不足,榷貨務課益不登,於是即發運司置官專領運鹽公事。治平中,京師入緡錢二百二十七萬,而淮南、兩浙、福建、江南、荊湖、廣南六路歲售緡錢,皇祐中二百七十三萬,治平中三
 百二十九萬。



 江、湖運鹽既雜惡,官估復高,故百姓利食私鹽,而並海民以魚鹽為業,用工省而得利厚。繇是不逞無賴盜販者眾,捕之急則起為盜賊。江、淮間雖衣冠士人,狃於厚利,或以販鹽為事。江西則虔州地連廣南,而福建之汀州亦與虔接,虔鹽弗善,汀故不產鹽,二州民多盜販廣南鹽以射利。每歲秋冬,田事才畢,恆數十百為群,持甲兵旗鼓,往來虔、汀、漳、潮、循、梅、惠、廣八州之地。所至劫人穀帛,掠人婦女,與巡捕吏卒斗格,至殺傷
 吏卒,則起為盜,依阻險要,捕不能得,或赦其罪招之。歲月浸淫滋多,而州官糶鹽歲才及百萬斤。



 慶歷中,廣東轉運使李敷、王繇請運廣州鹽於南雄州,以給虔、吉,未報,即運四百餘萬斤於南雄;而江西轉運司不以為便,不往取。後三司戶部判官周湛等八人復請運廣鹽入虔州,江西亦請自具本錢取之。詔尚書屯田員外郎施元長等會議,皆請如湛等議。而發運使許元以為不可,遂止。



 嘉祐以來,或請商販廣南鹽入虔、汀,所過州縣收
 算;或請放虔、汀、漳、循、梅、潮、惠七州鹽通商;或謂第歲運淮南鹽七百萬斤至虔,二百萬斤至汀,民間足鹽,寇盜自息;或請官自置鋪役兵卒,運廣南、福建鹽至虔、汀州,論者不一。先嘗遣職方員外郎黃炳乘傳會所屬監司及知州、通判議,謂虔州食淮南鹽已久,不可改,第損近歲所增官估,斤為錢四十,以十縣五等戶夏秋稅率百錢令糴鹽二斤,隨夏稅入錢償官。繼命提點鑄錢沉扶覆視可否,扶等請選江西漕船團為十綱,以三班使臣
 部之,直取通、泰、楚都倉鹽。詔用炳等策,然歲增糶六十餘萬斤。



 江西提點刑獄蔡挺制置鹽事,乃令民首納私藏兵械給巡捕吏卒,而販黃魚籠挾鹽不及二十斤、徒不及五人、不以甲兵自隨者,止輸算勿捕。淮南既團新綱漕鹽,挺增為十二綱,綱二十五艘,金巢梁至州乃發。輸官有餘,以畀漕舟吏卒,官復以半價取之,繇是減侵盜之弊,鹽遂差善。又損糶價,歲課視舊增至三百餘萬斤,乃罷炳等議所率糴鹽錢。異時,汀州人欲販鹽,輒先伐
 鼓山谷中,召願從者與期日,率常得數十百人已上,與俱行。至是,州縣督責耆保,有伐鼓者輒捕送,盜販者稍稍畏縮。朝廷以挺為能,留之江西,積數年乃徙。久之,江西鹽皆團綱運致如虔州焉。



 初,荊湖亦病鹽惡,且歲漕常不足,治平二年,才及二十五萬餘石。三年,撥淮西二十四綱及傭客舟載鹽以往,是歲運及四十萬石。四年,至五十三萬餘石。



 慶歷初,判戶部勾院王琪言:「天禧初,嘗以荊湖鹽估高,詔斤減三錢或二錢,自後利入寢損。
 請復舊估,可歲增緡錢四萬。」許之。治平中,淮南轉運使李復圭、張芻、蘇頌,三司度支判官韓縝,相繼請減淮南鹽價,然卒不果行。



 熙寧初,江西鹽課不登,三年,提點刑獄張頡言:「虔州官鹽鹵濕雜惡,輕不及斤,而價至四十七錢。嶺南盜販入虔,以斤半當一斤,純白不雜,賣錢二十,以故虔人盡食嶺南鹽。乃議稍減虔鹽價,更擇壯舟,團為十綱,以使臣部押。後蔡挺以贛江道險,議令鹽船三歲一易,仍以鹽純雜增虧為綱官、舟人殿最,鹽課遂
 敷,盜販衰止。自挺去,法十廢五六,請復之便。」詔從之。仍定歲運淮鹽十二綱至虔州。及章惇察訪湖南,符本路提點刑獄朱初平措置般運廣鹽,添額出賣,然未及行。元豐三年,惇既參政,有郟但者,邪險銳進,素為惇所喜,迎合惇意,推仿湖南之法,乞運廣鹽於江西。即遣蹇周輔往江西相度。周輔承望惇意,奏言:「虔州運路險遠,淮鹽至者不能多,人苦淡食,廣東鹽不得輒通,盜販公行。淮鹽官以九錢致一斤,若運廣鹽盡會其費,減淮鹽一
 錢,而其鹽更善,運路無阻。請罷運淮鹽,通般廣鹽一千萬斤於江西虔州、南安軍,復均淮鹽六百一十六萬斤於洪、吉、筠、袁、撫、臨江、建昌、興國軍,以補舊額。」詔周輔立法以聞。周輔具鹽法並總目條上,大率峻剝於民,民被其害。舊,江西鹽場許民買撲,周輔悉籍於官賣之。遂以周輔遙領提舉江西、廣東鹽事,即司農寺置局。



 四年,周輔改漕河北。明年,提舉常平劉誼言道途洶洶,以賣鹽為患。詔江東提點刑獄範峋體量,未報,誼坐言役法等
 事罷。及峋奏至,但以州縣違法塞詔,竟無更張。未幾,周輔奏:「虔州、南安軍推行鹽法方半年,已收息十四萬緡。」自以為功。詔命發運副使李琮體訪利害,琮知周輔方被獎用,止謂鹽法宜變通而已,不敢斥言其害。六年,周輔為戶部侍郎,復奏湖南郴、道州鄰接韶、連,可以通運廣鹽數百萬,卻均舊賣淮鹽於潭、衡、永、全、邵等州,並準江西、廣東見法,仍舉郟但初議,郴、全、道三州亦賣廣鹽。詔委提舉常平張士澄、轉運判官陳偲措置。明年,士澄
 等具條約來上,詔施行之,額利增加,一方騷然。於時淮西亦推行周輔鹽法,發運使蔣之奇奏立知州、通判、鹽事官賞罰,下戶部著為令。



 紹聖三年,發運司言淮南亭戶貧瘠,官賦本錢六十四萬緡,皆倚辦諸路,以故不時至,民無所得錢,必舉倍稱之息。欲以糴本錢十萬緡給之,不足,畀以憑由,即欲質於官,與憑之七,而蠲其息,鹽本集,復給其三分,憑由毀棄。



 崇寧元年,蔡京議更鹽法,乃言東南鹽本或闕,滯於客販,請增給度牒及給封樁
 坊場錢通三十萬緡。並列七條:一、許客用私船運致,仍嚴立輒逾疆至夾帶私鹽之禁;二、鹽場官吏概量不平或支鹽失倫次者,論以徒;三、鹽商所繇官司、場務、堰閘、津渡等輒加苛留者,如上法;四、禁命吏、蔭家、貢士、胥史為賈區請鹽;五、議貸亭戶;六、鹽價大低者議增之;七、令措置官博盡利害以聞。明年,詔鹽舟力勝錢勿輸,用絕阻遏,且許舟行越次取疾,官綱等舟輒攔阻者坐之。遂變鈔法,置買鈔所於榷貨務。凡以鈔至者,並以末鹽、乳
 香、茶鈔並東北一分及官告、度牒、雜物等換給。末鹽鈔換易五分,餘以雜物,而舊鈔止許易末鹽、官告。仍以十分率之,止聽算三分,其七分兼新鈔。定民間買鈔之價,以抑豪強,以平邊糴。在河北買者,率百緡毋得下五千,東南末鹽鈔毋得下十千,陜西鹽鈔毋得下五千五百,私減者坐徒徙之罪,官吏留難、文鈔展限等條皆備。



 四年,又以算請鹽價輕重不等,載定六路鹽價,舊價二十錢以上皆遞增以十錢,四十五者如舊;算請東南末鹽,
 願折以金銀、物帛者聽其便。而亭戶貸錢,舊輸息二分者蠲之。五年,詔算請不貼納見錢,以十分率之,毋過二分。大觀元年,乃令算請東南末鹽貼輸及帶舊鈔如見條外,更許帶日前貼輸三分鹽鈔,輸四分者帶二分,五分者帶三分。後又貼輸四分者帶三分,五分者帶四分,而東南鹽並收見緡換請新鈔者,如四分五分法貼輸。其換請新鈔及見錢算東南末鹽,如不帶六等舊鈔者,聽先給;如止帶五等舊鈔,其給鹽之敘,在崇寧四年十
 月前所帶不貼輸舊鈔之上。六等者,謂貼三、貼四、貼五、當十鈔、並河北公據、免貼納錢是也。



 時鈔法紛易,公私交弊。四年,侍御史毛注言:「崇寧以來,鹽法頓易元豐舊制,不許諸路以官船回載為轉運司之利,許人任便用鈔請鹽,般載於所指州縣販易,而出賣州縣用為課額。提舉鹽事司苛責郡縣,以賣鹽多寡為官吏殿最,一有循職養民不忍侵克,則指為沮法,必重奏劾譴黜,州縣熟不望風畏威,競為刻虐?由是東南諸州每縣三等以
 上戶,俱以物產高下,勒認鹽數之多寡。上戶歲限有至千緡,第三等末戶不下三五十貫,籍為定數,使依數販易,以足歲額;稍或愆期,鞭撻隨之。一縣歲額有三五萬緡,今用為常額,實為害之大者。」



 又言:



 「朝廷自昔謹三路之備,糧儲豐溢,其術非他,惟鈔法流通,上下交信。東南末鹽錢為河北之備,東北鹽為河東之備,解地鹽為陜西之備,其錢並積於京師,隨所積多寡給鈔於三路。如河北糧草鈔至京,並支見錢,號飛鈔法;河東三路至京,
 半支見錢,半支銀、綢、絹;陜西解鹽鈔則支請解鹽,或有泛給鈔,亦以京師錢支給。為錢積於京師,鈔行於三路,至則給錢,不復滯留。當時商旅皆悅,爭運糧草,入於邊郡。商賈既通,物價亦平;官司上下,無有二價,斗米止百餘錢,束草不過三十;邊境倉廩,所在盈滿。



 自崇寧來鈔法屢更,人不敢信,京師無見錢之積,而給鈔數倍於昔年。鈔至京師,無錢可給,遂至鈔直十不得一。邊郡無人入中,糴買不敷,乃以銀絹、見錢品搭文鈔,為糴買之直。
 民間中糴,不復會算鈔直,惟計銀絹、見錢,須至高抬糧草之價,以就虛數。致使官價幾倍於民間,斗米有至四百,束草不下百三十餘錢,軍儲不得不闕,財用不得不匱。如解鹽鈔每紙六千,今可直三千,商旅凡入東南末鹽鈔,乃以見錢四分、鹽引六分,榷貨務惟得七十千之入,而東南支鹽,官直百千,則鹽本已暗有所損矣。



 臣謂鈔法不循復熙、豐,則物價無由可平,邊儲無由可積,方今大計,無急於此。薛向昔講究於嘉祐中,行之未幾,穀
 價遽損,邊備有餘,逮及熙、豐,其法始備。比年榷貨務不顧鈔法屢變,有誤邊計,惟冀貼納見錢,專買東南鹽鈔,圖增錢數,以僥冒榮賞。前鈔方行,而後鈔又復變易,特令先次支鹽,則前鈔遂為廢紙,罔人攘利,商旅怨嗟。臣願明詔執政大臣,精擇能吏,推明鈔法,無以見行為有妨,無以既往為不可復,如薛向之法己效於昔者,可舉而行之。」



 今之練政事、通鈔法,不患無人;在京三庫之積,皆四方郡縣所入,不患無備。如以三四百萬緡樁留京
 師,隨數以給鈔引,使鈔至給錢,不復邀阻,上下交信,則人以鈔引為輕繼,轉相貿易。或支請多,惟轉廊就給東南末鹽鈔或度牒之類,如東南末鹽鈔或度牒敕牒唯許以鈔引就給外,餘並令在京以見錢入易,樁留以為鈔引之資,亦計之得者。若舊出文鈔,亦當體究立法,量為分數,支鹽償之。自昔立法之難,非特造始,修復既廢,亦為非易。欲興經久之利,則目前微害,宜亦可略,惟詳酌可否施行之。



 未幾,張商英為相,乃議變通損益,復熙、
 豐之舊,令內府錢別樁一千五百萬緡,餘悉移用,以革錢、鈔、物三等偏重之弊。陜西給鈔五百萬緡,江、淮發運司給見錢文據或截兌上供錢三百萬緡。以左司員外郎張察措置東南鹽事,提舉江西常平張根管幹運淮鹽於江西,罷提舉鹽香,諸路鹽事各歸提刑司。議定五等舊鈔,商旅已換請新鈔及見錢鈔不對帶,聽先給東南末鹽諸路貨易。仍下淮、浙鹽場,以鹽十分率之,樁留五分,以待支發官綱,備三路商旅轉廊算請,餘五分以
 待算請新鈔及見錢鈔與不帶舊鈔當先給者。於是推行舊法,以商旅五色舊鈔,若用換請新鈔對帶,方許支鹽,慮伺候歲月,欲給無由,乃立增納之法。貼三鈔許於榷貨務更貼見緡七分,貼四鈔更貼六分,貼五、當十鈔貼七分,河北見錢文據貼五分算請。



 有司議,三路鈔法如熙、豐舊法,全仰東南末鹽為本,若許將舊鈔貼納算請,正與推行三路熙、豐鈔法相戾;即不令貼納算還,又鈔無所歸。議將河北見錢文據減增納二分,餘各減二
 分,以告敕、減度牒、香藥、雜物、東南鹽算請給償。帝詔:「東南六路元豐年額賣鹽錢,以緡計之,諸路各不下數十萬。自行鈔鹽,漕計窘匱,以江西言之,和、豫買欠民價不少,何以副仁民愛物之意?」令東南諸路轉運司協力措置般運。



 政和元年,詔商旅願依熙、豐法轉廊者,許先次用三路新鈔算請,往他所定價給賣。優存兩浙亭戶額外中鹽,斤增價三分。已而張察均定鹽價,視紹聖斤增二錢,詔從其說,仍斤增一錢。議者謂:「異時鹽商於榷貨
 務入納轉廊,惟視東南諸郡積鹽多寡,鹽多則請鈔者眾,所入亦倍,其闕鹽地,客不肯住。在元豐時遠地須豫備二年或三年,次遠一年至二年,最近亦半年及一年,謂之準備鹽,而後鈔法乃通。紹聖間遵用舊制,廣有準備,故均價之後,課利增倍。謂宜嚴責轉運司般運準備鹽外,更及元豐準備之數,則鈔法始通,課利且羨。亭戶煎鹽官為買納,比舊既增矣,止用元豐舊價自可,況用新價,而有本錢,復加借貸,何慮不增?若斤更增一錢,虛
 費亦大。」詔施行之。六路通置提舉鹽事官,置司於揚州,未幾罷。



 議者復謂:「客人在京榷貨務買東南末鹽者,其法有二:一曰見錢入納,二曰鈔面轉廊。今既許三路文鈔得以轉廊,若更循舊制,許以見錢入納,則客旅之錢,當入於榷貨,而不入於兼並,見錢留於京師,客旅走於東南。」詔採用焉。又有謂:「舊法聽以物貨及官錢鈔引抵當,所以扶持鈔價,不大減損,昨禁之非是。其舊轉廊鹽鈔,販至東南,轉運司乃專以見錢為務,致多壅閼。」於是
 復鈔引抵當,一如其舊。末鹽以十分率之,限以八分給末鈔,二分許鬻見緡,後又增見緡為三分。



 二年,江寧府、廣德軍、太平州斤更增錢二,宣、歙、饒、信州斤增錢三,池江州、南康軍斤增錢四,各以去產鹽地遠近為差。是歲,蔡京復用事,大變鹽法。五月,罷官般賣,令商旅赴場請販,已般鹽並封樁。商旅赴榷貨務算請,先至者增支鹽以示勸。前轉廊已算鈔未支者,率百緡別輸見緡三分,仍用新鈔帶給舊鈔三分;已算支者,所在抄數別輸帶
 賣如上法。其算請悉用見緡,而給鹽倫次,以全用見緡不帶舊鹽者為上,帶舊鹽者次之,帶舊鈔者又次之。三路糴買文鈔,算給七分東南末鹽者,聽對見緡支算二分,東北鹽亦如之。自餘文鈔,毋得一例對算。復置諸路提舉官。於是詔書褒美京功,然商旅終以法令不信為疑,算請者少,乃申扇搖之令,增賞錢五百緡。



 三年,以商人承前先即諸州投勾,乃請鹽於場,留滯,罷之。若請鹽大帶斤重者,官為秤驗,乃輸錢給鈔。時法既屢變,蔡京
 更欲巧籠商賈之利,乃議措置十六條,裁定買官鹽價,囊以三百斤,價以十千,其鬻者聽增損隨時,舊加饒腳耗並罷。客鹽舊止船貯,改依東北鹽用囊,官袋鬻之,書印及私造貼補,並如茶籠篰法,仍禁再用。受鹽、支鹽官司,析而二之,受於場者管秤盤囊封,納於倉者管察視引據、合同號簿。囊二十,則以一折驗合同遞牒給商人外,東南末鹽諸場,仍給鈔引號簿;有欲改指別場者,並批銷號簿及鈔引,仍用合同遞牒報所指處給隨鹽引;
 即已支鹽,關所指處籍記。中路改指者仿此。其引繳納,限以一年,有故展毋得逾半年;限竟,鹽未全售者毀引,以見鹽籍於官,止聽鬻其處,毋得翻改。大抵皆視茶法而多為節目,欺奪民利,故以免究盜販、私煎、大帶斤重為名,而專用對帶之法。客負鈔請鹽,往往厄不即畀,必對元數再買新鈔,方聽帶給舊鈔之半。慮令之不行也,嚴避免之禁,申沮壞之制,重扇搖之法,季輒比較,務峻督責以取辦。



 四年,以遠地商販者稀,鹽倉以地遠近為
 敘,先給遠者。繼令搭帶正鹽,期一月不買新鈔,沒官,而剩鹽即沒納。五年,偽造引者並依川錢引定罪。六年,以產鹽州軍大商弗肯止留,其用小袋住賣者聽輸錢二十給鈔,毋得輒出州界。



 宣和二年,詔六路封樁舊鹽數輸億萬,其聽商旅般販,與淮、浙鹽倉即今鹽鈔對算。四年,榷貨務建議:「古有斗米斤鹽之說,熙、豐以前,米石不過六七百,時鹽價斤為錢六七十;今米價石兩千五百至三千,而鹽仍舊六十。崇寧會定鹽價,買鹽折算,酌以
 中價,斤為錢四十,今一斤三十七錢,虧公稍多。欲囊增為十三千入納,而亭戶所輸並增價,庶克自贍,盜販衰止。」於是舊鹽盡禁住賣,而籍記、貼輸、帶賣之令復用焉。



 初,鹽鈔法之行,積鹽於解池,積錢於京師榷貨務,積鈔於陜西沿邊諸郡。商賈以物斛至邊入中,請鈔以歸。物斛至邊有數倍之息,惟患無回貨,故極利於得鈔,徑請鹽於解池,而解鹽通行地甚寬;或請錢於京師,每鈔六千二百,登時給與,但輸頭子等錢數十而已。以此所由
 州縣,貿易者甚眾。崇寧間,蔡京始變法,俾商人先輸錢請鈔,赴產鹽郡授鹽,欲囊括四方之錢,盡入中都,以進羨要寵,鈔法遂廢,商賈不通,邊儲失備;東南鹽禁加密,犯法被罪者多。民間食鹽,雜以灰土。解池天產美利,乃與糞壤俱積矣。大概常使見行之法售給才通,輒復變易,名對帶法。季年又變對帶為循環。循環者,已賣鈔,未授鹽,復更鈔;已更鈔,鹽未給,復貼輸錢,凡三輸錢,始獲一直之貨。民無貲更鈔,已輸錢悉乾沒,數十萬券一
 夕廢棄,朝為豪商,夕儕流丐,有赴水投繯而死者。



 時有魏伯芻者,本省大胥,蔡京委信之,專主榷貨務。政和六年,鹽課通及四千萬緡,官吏皆進秩。七年,又以課羨第賞。伯芻年除歲遷,積官通議大夫、徽猷閣待制,既而黨附王黼,京惡而黜之。伯芻非有心計,但與交引戶關通,凡商旅算請,率克留十分之四以充入納之數,務入納數多,以昧人主而張虛最。初,政和再更鹽法,伯芻方為蔡京所倚信,建言:「朝廷所以開闔利柄,馳走商賈,不煩號
 令,億萬之錢輻湊而至。御府頒索,百司支費,歲用之外沛然有餘,則榷鹽之入可謂厚矣。頃年,鹽法未有一定之制,隨時變革以便公私,防閑未定,奸弊百出。自政和立法之後,頓絕弊源,公私兼利。異時一日所收不過二萬緡,則已詫其太多,今日之納乃常及四五萬貫。以歲計之,有一郡而客鈔錢及五十餘萬貫者,處州是也;有一州倉而客人請鹽及四十萬袋者,泰州是也。新法於今才二年,而所收已及四千萬貫,雖傳記所載貫朽錢
 流者,實未足為今日道也。伏乞以通收四千萬貫之數,宣付史館,以示富國裕民之政。」小人得時騁志,無所顧忌,遂至於此。



 於時御府用度日廣,課入欲豐,再申歲較季比之令,在職而暫取告,其月日皆毋得計折,害法者不以官蔭並處極坐,微至於鹽袋鯗鹽,莫不有禁,州縣惟務歲增課以避罪法,上下程督加厲。七年,乃詔:「昨改鹽法,立賞至重,抑配者多,計口敷及嬰孩,廣數下逮駝畜,使良民受弊,比屋愁嘆。悉從初令,以利百姓。三省其
 申嚴近制,改奉新鈔。」然有司不能承守,故比較已罷而復用,抄札既免而復行,鹽囊既增而復止,一囊之價裁為十一千,既又復為十三千,民力因以擾匱,而盜賊滋焉。



 靖康元年,詔未降新鈔前已給見錢公據文鈔,並給還商賈,以示大信。時鹽盡給新鈔,亦用帶賣舊鹽立限之法。言者論:「王黼當國,循用蔡京弊法,改行新鈔,舊鹽貼錢對帶,方許出賣,初限兩月,再限一月。是時黼方用事,專務害民,剝下益上,改易鈔法,甚於盜賊。然今不改
 覆車之轍,又促限止半月,反不及王黼之時,商賈豈得不怨?」詔申限焉。



 南渡,淮、浙亭戶,官給本錢。諸州置倉,令商人買鈔,五十斤為石,六石為袋,輸鈔錢十八千。紹興元年,詔臨安府,秀州亭戶二稅,依皇祐法輸鹽,立監官不察亭戶私煎及巡捕漏洩之法。二年九月,詔淮、浙鹽令商人袋貼輸通貨錢三千,已算請而未售者亦如之,十日不自陳,如私鹽律。時呂頤浩用提轄張純儀,峻更鹽法。十有一月,詔淮、浙鹽以十分為率,四分支今降旨
 符以後文鈔,四分支建炎渡江以後文鈔。先是呂頤浩以對帶法不可用,令商人貼輸錢,至是復以分數如對帶法,於是始加嚴酷矣。三年,減民間蠶鹽錢。四年正月,詔淮、浙鹽鈔錢每袋增貼輸錢三貫,並計綱輸行在,尋命廣鹽亦如之。九月,以入輸遲細,減所添錢。然自建炎三年改鈔法,及今所改,凡五變,而建炎舊鈔支尚未絕,乃命以先後並支焉。



 孝宗乾道六年,戶部侍郎葉衡奏:「今日財賦,鬻海之利居其半,年來課入不增,商賈不行,
 皆私販害之也。且以淮東、二浙鹽出入之數言之,淮東鹽灶四百一十二所,歲額鹽二百六十八萬三千餘石,去年兩務場賣淮鹽六十七萬二千三百餘袋,收錢二千一百九十六萬三千餘貫;二浙課額一百九十七萬餘石,去年兩務場賣浙鹽二十萬二千餘袋,收錢五百一萬二千餘貫,而鹽灶乃計二千四百餘所。以鹽額論之,淮東之數多於二浙五之一,以去歲賣鹽錢數論之,淮東多於二浙三之二,及以灶之多寡論之,兩浙反多
 淮東四之三,蓋二浙無非私販故也。欲望遣官分路措置。」



 淳熙八年,詔住賣帶賣積鹽,以朝廷徒有帶賣之名,總所未免有借撥之弊故也。十年,先是湖北鹽商吳傳言:「國家鬻海之利,以三分為率,淮東居其二。通、泰、楚隸買鹽場十六,催煎場十二,灶四百十二。紹興初,灶煎鹽多止十一籌,籌為鹽一百斤。淳熙初,亭戶得嘗試鹵水之法,灶煎至二十五籌至三十籌,增舊額之半。緣此,鹽場買亭戶鹽,籌增稱鹽二十斤至三十斤為浮
 鹽。日買鹽一萬餘籌,其浮鹽止以二十斤為則,有二十萬斤,為二千籌,籌為錢一貫八百三十文,內除船腳錢二百文,有一貫六百三十文。其鹽並再中入官,為鈔錢四百五十一萬七千五百餘緡。又綱取鹽一代並諸窠名等,及賣又多稱斤兩,亭戶饑寒,不免私賣。若朝廷嚴究,還其本錢,而後可以盡革私賣之弊。」至是,詔還通、泰等州諸鹽場欠亭戶鹽本錢一百一十萬貫。



 寧宗慶元初,詔罷循環鹽鈔,改增剩鈔名為正支文鈔給算,與已投倉者
 通理先後支散。以淮東提舉陳損之言循環鈔多弊,故有是命。於是富商巨賈有願為貧民者矣。開禧二年,詔自今新鈔一袋,搭支舊鈔一袋;如新鈔多於舊鈔,或願全以新鈔支鹽,及無舊鈔而願全買新鈔者聽,以新鈔理資次。嘉定二年,詔淮東貼輸鹽錢免二分交子,止用錢會中半。三年詔:「停鈔引之家,增長舊鈔價直,袋賣官會百貫以上。自今令到日,鹽鈔官錢袋增收會子二十貫,三務場朱印於鈔面,作「某年某月新鈔」,俟通賣及一
 百萬袋,即免增收。其日前已未支鹽鈔並為舊鈔,期以一年持赴倉場支鹽,袋貼輸官會一十貫,出限更不行用。」此淮、浙鹽之大略也。



 唐乾元初,第五琦為鹽鐵使,變鹽法,劉晏代之;當時舉天下鹽利,歲才四十萬緡。至大歷,增至六百餘萬緡。天下之賦,鹽利居半。元祐間,淮鹽與解池等歲四百萬緡。比唐舉天下之賦已三分之二。紹興末年以來,泰州海寧一監,支監三十餘萬席,為錢六七百萬緡,則是一州之數,過唐舉天下之數矣。



 寶
 慶二年,監察御史趙至道言:「夫產鹽固藉於鹽戶,鬻鹽實賴於鹽商,故鹽戶所當存恤,鹽商所當優潤。慶元之初,歲為錢九百九十萬八千有奇,寶慶元年,止七百四十九萬九千有奇,乃知鹽課之虧,實鹽商之無所贏利。為今之計,莫若寬商旅,減征稅,庶幾慶元鹽課之盛,復見於今日矣。」從之。紹定元年,以侍御史李知孝言,罷上虞、餘姚海塗地創立鹽灶。端平二年,都省言:「淮、浙歲額鹽九十七萬四千餘袋,近二三年積虧一百餘萬袋,民食
 貴鹽,公私俱病。」有旨,三路提舉茶鹽司各置主管文字一員,專以興復鹽額、收買散鹽為務,歲終尚書省課其殿最。淳祐元年,臣僚奏:「南渡立國,專仰鹽鈔,紹興、淳熙,率享其利。嘉定以來,二三十年之間,鈔法或行或罷,而浮鹽之說牢不可破,其害有不可勝言者。望付有司集議,孰為可行,孰為可罷,天地之藏與官民共之,豈不甚盛?」從之。五年,申嚴私販苛征之禁。



 寶祐元年,都省言:「行在榷貨務都茶場上本務場淳祐十二年收趁到茶鹽
 等錢一十一千八百一十五萬六千八百三十三貫有奇,比今新額四千萬貫增一倍以上,合視淳祐九年、十年、十一年例倍償之,以勵其後。」有旨依所上推賞。四年五月,以行在務場比新額增九千一百七十三萬五千九百一十二貫有奇,本務場並三省、戶部、大府寺、交引庫,凡通管三務場職事之人,視例推賞,後以為常。十有二月,殿中侍御史朱熠言:「鹽近者課額頓虧,日甚一日。姑以真州分司言之,見虧二千餘萬,皆由臺閫及諸軍
 帥興販規利之由。」於是復申嚴私販之禁。



 五年,朱熠復言:「鹽之為利博矣。以蜀、廣、浙數路言之,皆不及淮鹽額之半。蓋以斥鹵彌望,可以供煎烹,蘆葦阜繁,可以備燔燎。故環海之湄,有亭戶,有鍋戶,有正鹽,有浮鹽。正鹽出於亭戶,歸之公上者也。浮鹽出於鍋戶,鬻之商販者也,正鹽居其四,浮鹽居其一。端平之初,朝廷不欲使浮鹽之利散而歸之於下,於是分置十局,以收買浮鹽,以歲額計之,二千七百九十三萬斤。十數年來,鈔法屢更,公
 私俱困,真、揚、通、泰四州六十五萬袋之正鹽,視昔猶不及額,尚何暇為浮鹽計邪?是以貪墨無恥之士大夫,知朝廷住買浮鹽,龍斷而籠其利;累累灶戶,列處沙洲,日藉銖兩之鹽,以延旦夕之命;今商賈既不得私販,朝廷又不與收買,則是絕其衣食之源矣。為今之計,莫若遵端平之舊式,收鍋戶之浮鹽。所給鹽本,當過於正鹽之價,則人皆與官為市。卻以此鹽售於上江,所得鹽息,徑輸朝廷,一則可以絕戎閫爭利之風,二則可以續鍋戶
 烹煎之利。」有旨從之。



\end{pinyinscope}