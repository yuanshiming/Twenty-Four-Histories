\article{志第一百五十九 藝文五}

\begin{pinyinscope}

 《燕丹子》三卷



 東方朔《神異經》二卷晉張華傳



 師曠《禽經》一卷張華注



 王子年《拾遺記》十卷晉王嘉撰



 乾寶《搜神總記》十卷



 《寶櫝記》十卷



 並不知作者



 殷蕓《小說》十卷



 劉義慶《世說新語》三卷



 任昉《述異記》二卷



 吳均《續齊諧記》一卷



 沉約《俗說》一卷



 陶弘景《古今刀劍錄》一卷



 江淹《銅劍贊》一卷



 顧烜《錢譜》一卷



 顏之推《還冤志》三卷



 陽松玠《八代談藪》二卷



 張說《五代新說》二卷



 又《鑒龍圖記》一卷



 陸藏用《神告錄》一卷



 劉餗《傳記》三卷



 又《隋唐佳話》一卷



 《小說》三卷



 段成式《酉陽雜俎》二十卷



 又《續酉陽雜俎》十卷



 《廬陵官下記》二卷



 封演《聞見記》五卷



 張讀《宣室志》十卷



 唐臨《冥報記》二卷



 陸長源《辨疑志》三卷



 柳宗元《龍城錄》一卷



 《柳氏小說舊
 聞》六卷柳公權撰



 柳珵《常侍言旨》一卷



 盧弘正《昭義軍別錄》一卷



 溫造《瞿童述》一卷



 韋絢《戎幕閑談》一卷



 又《劉公嘉話》一卷



 《賓客佳話》一卷



 房千里《南方異物志》一卷



 鐘輅《前定錄》一卷



 劉軻《牛羊日歷》一卷



 李翱《卓異記》一卷



 李德裕《志支機寶》一卷



 又《幽怪錄》十四卷



 李商隱《雜纂》一卷



 範攄《雲溪友議》十一卷



 陸勛《集異志》二卷



 李復言《續玄怪錄》五卷



 李亢《獨異志》十卷



 袁郊《甘澤謠》一卷



 裴紫芝《續卓異記》一卷



 鄭遂《洽聞記》二卷



 康駢《劇談錄》二卷



 馮贄《雲仙散錄》一卷



 尉遲樞《南楚新聞》三卷



 皇甫枚《三水小牘》二卷



 王叡《炙轂子雜錄》五卷



 胡□蒙《談賓錄》五卷



 劉崇遠《金華子雜編》三卷



 趙璘《因話錄》六卷



 郭良輔《武孝經》一卷



 《女孝經》一卷侯莫陳邈妻鄭氏撰



 皇甫松《酒孝經》一卷



 羅邵《會稽新錄》一卷



 李隱《大唐奇事》十卷



 又《瀟湘錄》十卷



 陳輸《異聞集》十卷



 焦潞《稽神異苑》十卷



 李匡文《資暇錄》三卷



 顏師古《隋遺錄》一卷



 鄭棨《開天傳信記》一卷



 俞子《螢雪叢說》一卷



 李義山《雜蒿》一卷



 劉存《事始》三卷



 劉
 睿《續事始》三卷



 馮鑒《續事始》五卷



 李浚《松窗小錄》一卷



 劉願《知命錄》一卷



 張固《幽閑鼓吹》一卷



 《會昌解頤錄》五卷



 《樹萱錄》三卷



 《桂苑叢談》一卷



 《聞奇錄》三卷



 《溟洪錄》二卷



 《靈怪集》一卷



 《燈下閑談》二卷



 《續野人閑話》三卷



 《吳越會粹》一卷



 並不知作者



 《闕史》一卷參寥子述



 《佛孝經》一卷舊題名鶚,不知姓



 陳善《捫虱新話》八卷



 吳曾《能改齋漫錄》十三卷



 盧氏《逸史》一卷



 劉氏《耳目記》二卷



 調露子《角力記》一卷



 沉氏《驚聽錄》一卷



 並不知名



 《漢武帝洞冥記》四卷東漢郭憲編



 史虛白《釣
 磯立談記》一卷



 陳致雍《晉安海物異名記》三卷



 綦師系《元道孝經》一卷



 文谷《備忘小鈔》二卷



 杜光庭《虯須客傳》一卷



 僧庭藻《續北齊還冤志》一卷



 高澤《群居解頤》三卷



 王仁裕《玉堂閑話》三卷



 石文德《唐新纂》三卷



 劉曦度《鑒誡錄》三卷



 潘遺《紀聞談》一卷



 皮光業《妖怪錄》五卷



 逢行珪《鬻子注》一卷



 李諷《譔林》五卷



 鄭餘慶《談綺》一卷



 《續同歸說》三卷



 王定保《摭言》十五卷



 李綽一作「緯」



 《尚書故實》一作「事」一卷



 柳祥《瀟湘錄》十卷



 陸希聲《頤山錄》一卷



 柳珵《家
 學要錄》二卷



 《賂子解一作「錄」》一卷



 何光遠《鑒誡錄》三卷



 又《廣政雜錄》三卷



 蒲仁裕《蜀廣政雜記一作「紀」》十五卷



 楊士逵《儆戒錄》五卷



 王仁裕《見聞錄》三卷



 又《唐末見聞錄》八卷



 韋絢《佐談》十卷



 周文□《開顏集》二卷



 皮光業《皮氏見聞錄》十三卷



 《啟顏錄》六卷



 《三餘外志》三卷



 楊九齡《三感志》三卷



 段成式《錦裏新聞》三卷



 牛肅《紀聞》十卷崔造注



 周隨《南溪子》三卷



 盧光啟《初舉子》三卷



 《玉泉筆論》五卷



 李遇之《淺疑論》三卷



 金利用《玉溪編事》三卷



 玉川子《嘯旨》
 一卷



 《章程》四卷



 孫棨《北里志》一卷



 《同歸小說》三卷



 胡節還《醉鄉小略》一卷



 楊魯龜《令圃芝蘭集》一卷



 《唐說纂》四卷



 司馬光《游山行記》十二卷



 趙瞻《西山別錄》一卷



 唐恪《古今廣說》一百二十卷



 張舜民《南遷錄》一卷



 高彥休《闕史》三卷



 林思一作「黃仁望」



 《史遺》一卷



 黃仁望《續遺》五卷



 《興國拾遺》二十卷



 姚崇《六誡》一卷



 李大夫《誡女書》一卷



 海鵬《忠經》一卷



 《正順孝經》一卷



 曹希達《孝感義聞錄》三卷



 東方朔《感應經》三卷



 王轂《報應錄》三卷



 夏大玨一作「侯大
 玨」



 《奇應錄》五卷



 狐剛子《靈圖感應歌》一卷



 周子良《冥通記》四卷



 牛僧孺《玄怪錄》十卷



 李復言《搜古異錄》十卷



 焦璐《搜神錄》三卷



 麻安石《祥異集驗》二卷



 陳邵一作「召」



 《通幽記》三卷



 吳淑《異僧記》一卷



 杜光庭《錄異記》十卷



 李玫一作「政」



 《纂異記》一卷



 元真子《神異書》三卷



 裴鉶《傳奇》三卷



 《傳載》一卷



 曹大雅《靈異圖》一卷



 裴約言《靈異志》五卷



 曾寓《鬼神傳》二卷



 曹衍《湖湘神仙顯異》三卷



 《靈怪實錄》三卷



 秦再思《洛中紀異》十卷



 《秉一作「乘」異》三卷



 《貫怪圖》二卷



 鐘
 輅《感定錄》一卷



 馮鑒《廣前定錄》七卷



 趙自勤《定命錄》二卷



 溫奢《續定命錄》一卷



 陳翰一作「翱」



 《卓異記》一卷



 樂史《續廣卓異記》三卷



 《小名錄》三卷



 陸龜蒙《古今小名錄》五卷



 《名賢姓字相同錄》一卷



 《三教論》一卷



 周明辨《五經評判》六卷



 虞荔《古今鼎錄》一卷



 《欹器圖》一卷



 史道碩畫《八駿圖》一卷



 《異魚圖》五卷



 沉如筠《異物志》二卷



 通微子《十物志》一卷



 釋贊寧《物類相感志》五卷



 丘光庭《海潮論》一卷



 《海潮記》一卷



 張宗誨《花木錄》七卷



 僧仲休《花品》一卷



 蔡襄《荔枝譜》一卷



 同塵先
 生《庭萱譜》一卷



 竇常《正元飲略》三卷



 皇甫松《醉鄉日月》三卷



 尹建峰《令海珠璣》三卷



 何自然《笑林》三卷



 路氏《笑林》三卷



 南陽德長《戲語集說》一卷



 《集補江總白猿傳》一卷



 蘇鶚《杜陽雜編》二卷



 薛用弱《集異記》一卷



 《國老閑談》二卷題君玉撰,不知姓



 《大隱居士詩話》一卷不知姓名



 《釋常談》三卷



 《王洙談錄》一卷



 並不知作者



 曾季貍《艇齋詩話》一卷



 譚世卿《廣說》二卷



 《嘯旨》、《集異記》、《博異志》一卷谷神子纂,不知姓。



 費袞《梁溪漫志》一卷



 何溪汶《竹莊書話》二十七卷



 晁氏《談助》一卷
 不知名



 《幽明雜警》三卷題退夫興仲之所纂,不著姓



 張氏《儆誡會最》一卷



 唯室先生《步里客談》一卷



 沉括《筆談》二十五卷



 又《清夜錄》一卷



 王銍《續清夜錄》一卷



 郭彖《暌車志》一卷



 洪邁《隨筆五集》七十四卷



 又《夷堅志》六十卷甲、乙、丙志



 《夷堅志》八十卷丁、戊、己、庚志



 胡仔《漁隱叢話》前後集四十卷



 姚迥《隨因紀述》一卷



 王煥《北山紀事》十二卷



 何晦《摭言》十五卷



 又《廣摭言》十五卷



 僧贊寧《傳載》八卷



 徐鉉《稽神錄》十卷



 蘇轍《龍川志》六卷



 蘇軾《東坡詩話》一卷



 楊囦道《四六餘話》
 二卷



 謝伋《四六談麈》二卷



 葉凱《南宮詩話》一卷



 葉夢得《石林避暑錄》二卷



 馬永卿《懶真子》五卷



 趙概《見聞錄》二卷



 王同《敘事》一卷



 劉斧《翰府名談》二十五卷



 又《摭遺》二十卷



 《青瑣高議》十八卷



 僧文瑩《湘山野錄》三卷



 又《玉壺清話》十卷



 李端彥《賢已集》三十二卷



 王陶《談淵》一卷



 錢明逸《衣冠盛事》一卷



 句穎《坐右書》一卷



 曾鞏《雜職》一卷



 張師正《怪集》五卷



 又《倦游雜錄》十二卷



 《括異志》十卷



 畢仲詢《幕府燕閑錄》十卷



 劉分文《三異記》一卷



 岑象求《吉兇
 影響錄》八卷



 龐元英《南齊雜錄》一卷



 孔平仲《釋裨》一卷



 又《續世說》十二卷



 《孔氏雜說》一卷



 魏泰《訂誤集》二卷



 又《東軒筆錄》十五卷



 陳正敏《劍溪野話》三卷



 又《遁齋閑覽》十四卷



 李薦《師友談記》十卷



 王山《筆奩錄》七卷



 董逌《錢譜》十卷



 王闢之《澠水燕談》十卷



 宋肇《筆錄》三卷次其祖詳遺語



 李孝友《歷代錢譜》十卷



 劉延世《談圃》三卷



 成材《朝野雜編》一卷



 張舜民《畫墁錄》一卷



 陳師道《談叢究理》一卷



 《後山詩話》一卷



 李獻民《雲齋新說》十卷



 《和平談選士》一
 卷



 章炳文《搜神秘覽》三卷



 王得臣《麈史》三卷



 令狐皞如《歷代神異感應錄》二卷



 王讜《唐語林》十一卷



 黃朝英《青箱雜記》十卷



 李注《李冰治水記》一卷



 王鞏《甲申雜記》一卷



 又《聞見近錄》一卷



 朱無惑《萍洲可談》三卷



 僧惠洪《冷齋夜話》十三卷



 汪藻《世說敘錄》三卷



 洪皓《松漠紀聞》二卷



 方勺《泊宅編》十卷



 婁伯高《好還集》十卷



 何侑《嘆息》一卷



 周輝《清波別志》二卷



 孫宗鑒《東皋雜記》十卷



 洪炎《侍兒小名錄》一卷



 陸游《山陰詩話》一卷



 秦再思《洛中記異》十
 卷



 姚寬《西溪叢話》二卷



 耿煥《牧豎閑談》三卷



 又《野人閑話》五卷



 陳纂《葆光錄》三卷



 孫光憲《北夢瑣言》十二卷



 潘若沖《郡閣雅言》二卷



 王舉《雅言系述》十卷



 吳淑《秘閣閑談》五卷



 又《江淮異人錄》三卷



 李昉《太平廣記》五百卷



 陶岳《貨泉錄》一卷



 張齊賢《太平雜編》二卷



 《賈黃中談錄》一卷張洎撰



 錢易《洞微志》三卷



 又《滑稽集》一卷



 《南部新書》十卷



 陳彭年《志異》十卷



 祖士衡《西齋話記》一卷



 樂史《廣卓異記》二十卷



 張君房《潮說》三卷



 又《乘異記》三卷



 《科名分
 定錄》七卷



 《搢紳脞說》二十卷



 王績《補妒記》八卷



 李畋《該聞錄》十卷



 蘇耆《閑談錄》二卷



 黃休復《茅亭客話》十卷



 歐靖《宴閑談柄》一卷



 上官融《友會談叢》三卷



 王子融《百一紀》一卷



 梁嗣真《荊山雜編》四卷



 邵思《野說》三卷



 勾臺符《岷山異事》一卷



 聶田《俱異志》十卷



 盧臧《範陽家志》一卷



 丘浚《洛陽貴尚錄》十卷



 宋庠《楊億談苑》十五卷



 湯巖起《詩海遺珠》一卷



 趙闢公《雜說》一卷



 江休復《嘉祐雜志》三卷



 《窮神記》十卷



 《延賓佳話》四卷



 《林下笑談》一卷



 《世說新
 語》一卷



 《翰苑名談》三十卷



 《說異集》二卷



 《墨客揮犀》二十卷



 《北窗記異》一卷



 《道山新聞》一卷



 《紺珠集》十三卷



 《儆告》一卷



 《垂虹詩話》一卷



 並不知作者



 右小說類三百五十九部,一千八百六十六卷。



 甘、石、巫咸氏《星經》一卷



 石氏《星簿贊歷》一卷



 張衡《大象賦》一卷



 苗為注張華《小象賦》一卷



 《乾象錄》一卷



 抱真子《上象握鑒歌》三卷



 呂晚成《上象鑒》三卷



 《大象玄文》二卷



 《垂象志》二卷



 閭丘業《大象玄機歌》一卷本三卷,殘闕



 《天象
 圖》一卷



 《大象歷》一卷



 《入象度》一卷



 《乾象秘訣》一卷



 祖□恆《天文錄》三十卷



 《天文總論》十二卷



 《天文廣要》三十五卷



 《立成天文》三卷



 《符天經》一卷



 曹士為《符天經疏》一卷



 《符天通真立成法》二卷



 《天文秘訣》二卷



 《天文經》三卷



 《天文錄經要訣》一卷鈔祖□恆書



 《後魏天文志》四卷



 王安禮《天文書》十六卷



 《二儀賦》一卷



 李淳風《乾坤秘奧》七卷



 《太陽太陰賦》二卷



 《日月氣象圖》五卷



 《上像二十八宿纂要訣》一卷



 《太白會運逆兆通代記圖》一卷



 《日行黃道圖》一卷



 《月行
 九道圖》一卷



 《雲氣圖》一卷



 《渾天方志圖》一卷



 《九州格子圖》一卷



 張商英《三才定位圖》一卷



 《大象列星圖》三卷



 《大象星經》一卷



 《乾文星經》二卷



 劉表《星經》一卷



 又《星經》三卷



 《上象占要略》一卷



 《天文占》三卷



 《天象占》一卷



 《乾象秘占》一卷



 《占北斗》一卷



 張華《三家星歌》一卷



 又《玉函寶鑒星辰圖》一卷



 《渾天列宿應見經》十二卷



 《眾星配位天隔圖》一卷



 《文殊星歷》二卷



 《上象星文幽棲賦》一卷



 唐昧《秤星經》三卷



 《星說系記一作「紀」》一卷



 《混天星圖》一卷



 陶隱居《天
 文星經》五卷



 徐承嗣《星書要略》六卷



 《星經手集》二卷



 《天文星經》五卷



 《皇祐星經》一卷



 《五星交會圖》一卷



 徐升《長慶算五星所在宿度圖》一卷



 《七曜雌雄圖》一卷



 《文殊七曜經》一卷



 《七曜會聚一作「歷」》一卷



 《符天九星算法》一卷



 李世績《二十八宿纂要訣》一卷



 又《日月運行要訣》一卷



 僧一行《二十八宿秘經要訣》一卷



 宋均《妖瑞星圖》一卷



 《妖瑞星雜氣象》一卷



 桑道茂《大方廣一作「大廣方」經神圖歷》一卷



 《仰覆玄黃圖十二分野躔次》一卷



 《仰觀十二次圖》一
 卷



 《宿曜度分域名錄》一卷



 《華夏諸國城名歷》一卷



 《渾儀》一卷



 《渾儀法要》十一卷



 《渾天中影表圖》一卷



 歐陽發《渾儀》十二卷



 又《刻漏》五卷》一卷



 《晷影法要》一卷



 豐稷《渾儀浮漏景表銘詞》四卷



 蘇頌《渾天儀象銘》一卷



 韓顯符《天文明鑒占》十卷



 瞿曇悉達《開元占經》四卷



 《二十八宿分野五星巡應占》一卷



 《推占龍母探珠詩》一卷



 《古今通占》三十卷



 《握掌占》十卷



 《荊州占》三卷



 《蕃占新書要略》五卷



 《占風九天玄女經》一卷



 《雲氣測賦候》一卷



 《占候雲雨賦》一卷



 《
 驗天大明歷》一卷



 《符天五德定分歷》三卷



 王洪暉《日月五星彗孛凌犯應驗圖》三十卷



 《上象應驗錄》一十卷



 郭穎夫一作「士」



 《符天大術休咎訣》一卷



 《五星休咎賦》一卷



 張渭《符天災福新術》五卷



 《天文日月星辰變現災祥圖》一卷



 仁宗《寶元天人祥異書》十卷



 徐彥卿《征應集》三卷



 《玄象應驗錄》二十卷



 《祥瑞圖》一卷



 《都利聿期斯》一卷



 《聿斯四門經》一卷



 《聿斯歌》一卷



 《樞要經》一卷



 《青霄玉鑒》二卷



 《碧霄金鑒》三卷



 《碧落經》十卷



 蔣權卿《應輪心鑒》五卷



 崔
 寓《神像氣運圖》十卷



 《紫庭秘訣》一卷



 《玄緯經》二卷



 《辨負一作「真」經》二卷



 《大霄論璧第五》一卷



 《氣象圖》一卷



 《乙巳略例》十五卷



 《唐書距子經》一卷



 陶弘景《象歷》一卷



 《括星詩》一卷



 《玄象隔子圖》一卷



 《鏡圖》三卷



 《天文圖》一卷



 《三元經傳》一卷



 《大衍明疑論》十五卷



 《交食論》一卷



 並不知作者



 王希明《丹元子步天歌》一卷



 楊惟德《乾象新書》三十卷



 《新儀象法要》一卷



 張宋臣《列宿圖》一卷



 張宏圖《天文志訛辨》一卷



 阮泰發《水運渾天機要》一卷



 鄒淮《考異天文書》一卷



 右
 天文類一百三十九部,五百三十一卷。



 郭璞《三命通照神白經》三卷



 陶弘景《五行運氣》一卷



 《青一錄班氏經》一卷不知名



 李淳風《五行元統》一卷



 王希明《太一金鏡式經》十卷



 僧一行《遁甲通明無惑十八鈐局》一卷



 元兢《祿命厄會經》一卷



 楊龍光《祿命厄運歌》一卷



 李吉甫《三命行年韜鈐秘密》二卷



 李虛中《命書格局》二卷



 《珞琭子賦》一卷不知姓名,宋李企注



 許季山《易訣》一卷



 《周易八帖》四卷



 《周易髓要雜訣》一卷



 《周易天門子訣》二卷



 《周易
 三略經》三卷



 《易林》三卷



 《諸家易林》一卷



 《易新林》一卷



 《易傍通手鑒》八卷



 《易玄圖》一卷



 《周易菥蓂訣》一卷



 《易頌卦》一卷



 《大清易經訣》一卷



 《周易通貞》三卷



 《周易子夏占》一卷



 《周易口訣開題》一卷



 《周易飛燕轉關林》一卷



 《周易括世應頌》一卷



 《周易鬼靈經》一卷



 《周易三空訣》一卷



 《周易三十六占》六卷



 《周易爻詠》八卷



 《周易鬼鎮林》一卷



 《周易金鑒歌》一卷



 《周易聯珠論》一卷



 《周卦轆轤關》一卷



 《易轆轤圖頌》一卷



 《易大象歌》一卷



 《周易卜卦》一卷



 又《玄理歌》
 一卷



 《地理觀風水歌》二卷



 《陰陽相山要略》二卷



 郭璞《周易玄義經》一卷



 《周易察微經》一卷



 《周易鬼御算》一卷



 《周易逆刺》一卷



 《易鑒》三卷



 黃子一作「景」



 玄《易頌》一卷



 王守一《周易探玄》九卷本十卷



 《易訣雜頌》一卷



 《易杜秘林一作「林秘」》一卷



 《易大象林》一卷



 李鼎祚《易髓》三卷《目



 《瓶子記》三卷



 成玄英《易流演》五卷



 虞翻注《京房周易律歷》一卷



 陶隱居《易髓》三卷



 王鄯《周易通神歌》一卷



 張胥《周易繚繞詞》一卷



 靈隱子《周易河圖術》一卷



 焦氏《周易玉鑒領》一
 卷



 《周易三備雜機要》一卷



 《周易經類》一卷



 《法易一作「易法」》一卷



 《周易竅書》一卷



 《周易靈真述》一卷



 《周易靈真訣》一卷



 《易卦林》一卷



 《周易飛伏例》一卷



 《周易火竅》一卷



 《周易備要》一卷



 《周易六神頌》一卷



 天門子《易髓》一卷



 管公明《隔山照》一卷



 《文王版詞》一卷



 王嚴《金箱要錄》一卷



 朱異《稽疑》二卷



 《罔象玄珠》五卷



 《六證括天地經》一卷



 《黃帝天輔經》一卷



 孫臏《卜法》一卷



 劉表《荊州占》二卷



 《海中占》十卷



 武密《古今通占鑒》三十卷



 李淳風《乙巳占》十卷



 又《雜占》
 一卷



 《帝王氣象占》一卷



 《氣象占》一卷



 《西天占書》一卷



 《白澤圖》一卷



 《周遁三元纂例》一卷



 《陰陽遁八一作「入」局立成法》一卷



 《陰陽二遁萬一訣》四卷



 《遁甲要用歌式》二卷



 《陽遁天元局法》一卷



 《陰陽遁甲經》三卷



 《陰陽遁甲立成》一卷



 《天一遁甲兵機要訣》二卷



 《三元遁甲經》一卷



 《遁甲符應經》三卷



 《太一玄鑒》十卷



 《太一新鑒》三卷



 《樞會賦》一卷



 《九宮口訣》三卷



 《玉帳經》一卷



 《乾坤秘一作「要」》七卷



 《蓬瀛經》三卷



 《濟家備急廣要錄》一卷



 《三元經》一卷



 《二宅賦》一
 卷



 《行年起造九星圖》一卷



 《宅心鑒式》一卷



 《相宅經》一卷



 《宅體一作「髓」經》一卷



 《九星修造吉兇歌》一卷



 《陰陽二宅歌》一卷



 《二宅相占》一卷



 《太白會運記》一卷



 《九天秘記一作「訣》一卷



 《詳思記》一卷



 《玄女金石玄悟術》三卷



 《西王母玉訣》一卷



 《通玄玉鑒頌一作「領」》二卷



 封演《元正一作「正元」



 占書》一卷



 周輔《占經要訣》二卷



 《蕃占要略》五卷



 《天機立馬占》一卷



 《統占》二卷



 《六甲五行雜占機要》二卷



 《乙巳指占圖經》三卷



 《人倫寶鑒卜法》一卷



 杜靈賁《卜法》一卷



 《占候應驗》二
 卷



 《晷菥算經法》三卷



 《易咎限算》一卷



 《晷限立成》一卷



 費直《焦貢晷限歷》一卷



 韋偉《人元晷限經》三卷



 《銘》五卷



 《軌革秘寶》一卷



 《軌革指迷照膽訣》一卷



 《軌革照膽訣》一卷



 史蘇《五兆龜經》一卷



 又《龜眼玉鈐論》三卷



 《五兆金車口訣》一卷



 《五兆秘訣》三卷



 《五行日見五兆法》三卷



 《五兆穴門術》三卷



 《靈棋經》一卷



 《龜繚繞訣》一卷



 聶承休《龜經雜例要訣》一卷



 《玄女玉函龜經》三卷



 《古龜經》二卷



 《神龜卜經》二卷



 劉玄《龜髓經論》一卷



 毛寶定《龜竅》一卷



 《龜甲歷》
 一卷



 《龜兆口訣》五卷



 《龜經要略》二卷



 《龜髓訣》二卷



 《春秋龜策經》一卷



 黃石公《備氣三元經》二卷



 《玄女五兆筮經》五卷



 李進注《靈棋經》一卷



 《金石經》三卷



 《靈骨經》一卷



 《螺卜法》一卷



 《大道通靈肉臑論》一卷



 《鼓角證應傳》一卷



 郯子《占鳥經》二卷



 《占鳥法圖》一卷



 袁天綱一作「孫思邈」



 《九天玄女墜金法》一卷



 《怪書》一卷



 《響應經》一卷



 《玄女三廉射覆經》一卷



 《通明玉帳法》一卷



 《遁甲步小游太一諸將立成圖》二卷



 《相書》七卷



 《相氣色詩》一卷



 《要訣》三卷



 《玄明經》一
 卷



 閭丘純《射覆經》一卷



 東方朔《射覆經》三卷



 又《占神光耳目法》一卷



 《相枕經》一卷



 《馬經》三卷



 《相馬經》三卷



 盧重玄《夢書》四卷



 柳璨《夢雋》一卷



 《周公解夢書》三卷



 王升《縮或無「縮」字



 占夢書》十卷



 陳襄《校定夢書》四卷



 又《校定相笏經》一卷



 《校定京房婚書》三卷



 李靖《候氣秘法》三卷



 又《六十甲子占風雨》一卷



 《五音法》一卷



 《陰陽律體一作「髓」》一卷



 《靈關訣益智》二卷



 《袖中金》五卷



 《玄女常手經》二卷



 《神訣》一卷



 《游都璧玉經》一卷



 麻安石《災祥圖》一
 卷



 《風角鳥情》三卷



 《日月風雲氣候》一卷



 《日月暈貫氣》一卷



 《日月暈蝕》一卷



 《氣色經》一卷



 諸葛亮《十二時風雲氣候》一卷



 《五行雲霧歌》一卷



 《占風雨雷電》一卷



 《年代風雲一作「雨」占》一卷



 竇維鋈《廣古今五行記》三十卷



 周麟《竹倫經》三卷



 馮思古《遁甲六經》卷亡



 丘延翰《金鏡圖》一卷



 通真子《玉霄寶鑒經》一卷



 《三命指掌訣》一卷



 文靖《通玄五命新括》三卷



 董子平《太陰三命秘訣》一卷



 楊繪《元運元氣本論》一卷



 何朝《命術》一卷



 李蒸《三命九中歌》一卷



 徐鑒《三命機要說》一
 卷



 林開《五命秘訣》五卷



 僧善嵩《訣金書一十四字要訣》一卷



 《凝神子》一卷不知姓名



 凝神子《八殺經》一卷



 凝神子《解悟經》一卷



 西城野人《三五志》二卷



 《八九變通》一卷



 白雲愚叟《五行圖》一卷



 知玄子秦浼《太一占玄歌》一卷



 劉烜《元中祛惑經》一卷



 《占雨晴法》一卷



 《金鑒占風訣》一卷



 《三元飛化九宮法》一卷



 《行年五鬼運轉九宮法》一卷



 《山岡機要賦》一卷



 《山岡氣象雜占賦》一卷



 《五音地理詩》三卷



 《五音地理經訣》十卷



 《陰陽葬經》三卷



 《掘機口訣》一卷



 《掘
 鑒經一作握鑒經」》五卷



 《洞幽識秘要圖》三卷



 《靈寶六丁通神訣》三卷



 《通天靈應寶勝法》二卷



 《黃石記》五卷



 劉啟明《雲氣測候賦》一卷



 《定風占詩》三卷



 《風角五音占》一卷



 《日月暈圖經》二卷



 《占候雲雨賦》一卷



 《風雲關金巢秘訣》一卷



 《雲氣形象玄占》三卷



 《天地照耀占》一卷



 李經表《虹蜺災祥》一卷



 《宿曜錄鬼鑒》一卷



 《日月城砦氣象災祥圖》一卷



 《中樞秘頌太一明鑒》五卷



 《太一五元新歷》一卷



 《太一七術》一卷



 《太一陰陽定計主客決勝法》一卷



 《太一循環歷》一
 卷



 《太一會運逆順通代記陣圖》一卷



 《六壬軍帳賦》一卷



 《六壬詩》一卷



 《六壬六十四卦名》一卷



 《六壬戰勝歌》一卷



 《六壬出軍立就歷》三卷



 《六壬玉帳經》一卷



 王承昭一作「紹」



 《占風雲歌》一卷



 《占風雲氣候日月星辰圖》七卷



 《望江南風角集》二卷



 張良《陰陽二遁》一卷



 胡萬頃《太一遁甲萬勝時定主客立成訣》一卷



 一行《遁甲十八局》一卷



 司馬驤《遁甲符寶萬歲經圖歷》一卷



 馮繼明《遁甲元樞》二卷



 《玄女遁甲秘訣》一卷



 《天一遁甲圖》一卷



 《天一遁甲鈐歷》
 一卷



 《天一遁甲陰局鈐圖》一卷



 《遁甲搜元經》一卷



 《遁甲陽局鈐》一卷



 《遁甲陰局鈐》一卷



 杜惟翰一作「乾」



 《太一集》八卷



 《太一年表》一卷



 《十三神太一》一卷



 《御序景祐三式太一福應集要》十卷



 王處訥《太一青龍甲寅經》一卷



 康洙序《時游太一立成》一卷



 廣夷《太一秘歌》一卷



 《太一細行草》一卷



 《太一雜集筆草》一卷



 《太一時計鈐》一卷



 《太一陽九百六經》一卷



 《太一神樞長歷》一卷



 《太一陽局鈐》一卷



 《太一陰局鈐》一卷



 《九宮太一》一卷



 樂產《王佐秘珠》五卷



 《
 神樞靈轄經》十卷



 馬先《天寶靈應式經一作「記」》五卷



 《日游太一五子元出軍勝負七十二局》一卷



 《黃帝龍首經》一卷



 《九宮經》三卷



 《九宮圖》一卷



 《九宮占事經》一卷



 桑道茂《九宮》一卷



 又《三命吉兇》二卷



 《撮要日鑒》一卷



 《六十四卦歌》一卷



 郗良玉《三元九宮》一卷



 《九宮應瑞太一圖》一卷



 楊龍光《九宮要訣》一卷



 又《九宮詩》一卷



 《九宮推事式經》一卷



 《祿命歌》一卷



 《祿命經》一卷



 《風後三命》三卷



 朱琬《六壬寸珠集》一卷



 《六壬錄》六卷



 《五真降符六壬神武經》
 一卷



 《六壬關例集》三卷



 《六壬維乾照幽歷》六卷



 張氏《六壬用三十六禽秘訣》三卷



 《大六壬式局雜占》一卷



 《六壬玄機歌》三卷



 《六壬七曜氣神星禽經一作「紀」》一卷



 馬雄《絳囊經》一卷



 《金匱經》三卷



 《髓經心經鑒》三卷



 徐琬《啟蒙纂要》一卷



 李筌《玉帳歌》十卷



 《秘寶翠羽歌》三卷



 《明鑒連珠歌》一卷



 《清華經》三卷



 《推人鉤元法》一卷



 由吾裕《式心經略》三卷



 《式合書成》一卷



 《用式法》一卷



 《式經纂要》三卷



 《玄女式鑒》一卷



 《三式訣》三卷



 《天關五符式》一卷



 《三式
 參合立就歷》三卷



 《金照式經》十卷



 《雷公式局》一卷



 《靈應式》五卷



 《小游宿歷》一卷



 《三元六紀歷》一卷



 《玉鈐歷》一卷



 《明鑒起例歷》三卷



 《枝元長歷》一卷



 《日輪歷》一卷



 《五音百忌歷》二卷



 《葬疏》三卷



 孫洪禮《萬歲循環歷》一卷



 僧德濟《勝金歷》一卷



 《畢天水歷》一卷



 《畢天六甲歷》六卷



 《選日樞要歷》四卷



 《妍神歷》一卷



 《擇十二月鉗歷》二卷



 《七門行歷》一卷



 《大要歷》三卷



 《三皇秘要歷》一卷



 《選課歲歷》一卷



 《大明歷》二卷



 杜崇龜《明時總要歷》一卷



 陳恭釗《天寶歷注例》二
 卷



 《唐七聖歷》一卷



 《橫推歷》一卷



 《兵鈐月鏡纂要立成歷》一卷



 李淳風《十二宮入式歌》一卷



 僧居白《五行指掌訣》二卷



 逍遙子《鮮鶚經》三卷不知姓名



 《三命總要》三卷



 《太一中天密旨》三卷



 《西天都例經》一卷



 《三元經》三卷



 《淘金歌》一卷



 《三元龜鑒》一卷



 《五命》一卷



 《五音鳳髓經》一卷



 《大衍五行數法》一卷



 《三局天關論》一卷



 《六十甲子釋名》一卷



 《金掌圖竅》一卷



 《三局九格六陽三命大數法》



 《奇門萬一訣》



 《遁甲萬一訣》



 《太一遁甲萬一訣》



 已上四部無卷



 《陰陽萬一訣》
 一卷



 《金樞八象統元經》三卷



 《太一陰陽二遁》一卷



 《陰陽二遁局圖》一卷



 《陰陽二遁立成歷》一卷



 《遁甲玉女反閉局》一卷



 《太一金鏡備式錄》十卷



 《太一立成圖》一卷



 《太一飛鳥十精歷》一卷



 僧重輝一作「耀」



 《正德通神歷》三卷



 《大會殺歷》卷



 史序《乾坤寶典》四百五十五卷



 《乾坤總錄》五卷



 黃淳《通乾論》五卷



 《黃帝朔書》一卷托太公、師曠、東方朔撰



 《年鑒》一卷



 劉玄一作「先」



 之《月令圖》一卷



 《陰陽寶錄》一卷



 《西天陰符紫微七政經論》一卷



 《五符圖》一卷



 《選日陰陽月鑒》一卷



 李遂《通玄三命論》三卷



 李燕《三命》一卷



 又《陰陽詩》一卷



 《三命九中歌》一卷



 珞琭子《三命消息賦》一卷



 凝神子《五行三命手鑒》一卷



 《三命大行年入局韜鈐》三卷



 《大行年推祿命法》一卷



 《三命殺歷》一卷



 孟遇《三命訣》三卷



 《祿命人元經》三卷



 《祿訣經》三卷



 《五行貴盛生月法》



 《五行消息訣》



 蕭古一作「吉」



 《五行大義》五卷



 《金書四字五行》



 《四季觀五行論》



 珞琭子《五行家國通用圖錄》



 《訓字五行歌》二卷



 珞琭《五行疏》十卷



 《樵子五行志》
 五卷



 羅賓老《五行定分經》三卷



 濮陽夏《樵子五行志》五卷



 竇塗《廣古今陽復五行記》三十卷



 《五行通用歷》



 《金河流水訣》



 王叔政《推太歲行年吉兇厄》



 李燕《穆護詞》一卷一作馬融《消息機口訣》



 《洪範碎金訓字》



 《七曜氣神星禽經》三卷



 《納禽宿經》



 廖惟馨《星禽歷》



 杜百一作「相伯」



 子《禽法》



 司馬先生《三十六禽歌》一卷



 《占課禽宿情性訣》一卷



 蘇登《神光經》一卷



 許負《形神心鑒圖》一卷



 始一作「姑」



 布子卿《相法一作「書」》一卷



 朱述《相氣色面
 圖》一卷



 玄靈子《秘術骨法圖》一卷



 《相祿歌》二卷



 《察色相書》一卷



 《人鑒書》七卷



 《龜照口訣》五卷



 《人倫真訣》十卷



 《女仙相書》三卷



 《相氣色圖》五卷



 雲蘿《通真神相一作「明」訣》十卷



 柳清風《相歌》二卷



 郭峴述《顯光師相法》一卷



 《十七家集眾相書》一卷



 《占氣色要訣圖》一卷



 柳陰一作「隨」



 風《占氣色歌》一卷



 《形神論氣色經》一卷



 《元解訣》一卷



 《相書》二卷



 《月波洞中龜鑒》一卷



 《應玄玉鑒》一卷



 《六神相字法》一卷



 《相笏經》三卷



 陳混掌《相笏經》一卷管輅、李淳風法



 蕭繹《
 相馬經》一卷



 常知非《馬經》三卷



 谷神一作「鬼谷」



 子《辨養馬一作「養良馬」



 論》一卷



 《相馬病經》三卷



 《相犬經》一卷



 王立豹《鷹鷂候訣》一卷



 《鷹鷂五藏病源方論》一卷



 《堪輿經》一卷



 《太史堪輿》一卷



 商紹《太史堪輿歷》一卷



 《黃帝四序堪輿經》一卷



 《占婚書》一卷



 《周公壇經》三卷



 王佐明《集壇經》一卷



 李遠《龍紀聖異歷》一卷



 《五音三元宅經》三卷



 《陰陽宅經》一卷



 《陰陽宅經圖》一卷



 王澄《二宅心鑒》三卷



 又《二宅歌》一卷



 《陰陽二宅圖經》一卷



 《黃帝八宅經》一卷



 《淮南王見機
 八宅經》一卷



 一行《庫樓經》一卷



 《上象陰陽星圖》一卷



 《金圖地鑒》一卷



 《地鑒書》三卷



 孫季邕《葬範》五卷



 《地理六壬六甲八山經》八卷



 《地理三寶經》九卷



 《五音山岡訣》一卷



 《地論經》五卷



 《地理正經》十卷



 朱仙桃《地理贊》一卷



 又《玄堂範》一卷



 《地理口訣》一卷



 僧一行《地理經》十二卷



 又《靈轄歌》三卷



 《玉關歌》一卷



 《含意歌》七卷



 《通玄靈應頌》三卷



 《天一通玄機微翼圖》一卷



 《天一玄成局》一卷



 《玄樞經》一卷



 《玄樞纂要》一卷



 《知人秘訣》二卷



 《玄中祛惑經》三卷



 《遁
 甲鈐》一卷



 《八門遁甲入式歌》一卷



 《三元陰局》一卷



 《難逃論》一卷



 《靈臺篇》一卷



 《藻鑒了義經》一卷



 《蔀首經》三卷



 《玄象秘錄》一卷



 《真像論》一卷



 《清霄玉鏡要訣》一卷



 《二十八宿行度口訣》一卷



 《星禽課》一卷



 《群書古鑒錄》無卷



 並不知作者



 仁宗《洪範政鑒》十二卷



 楊惟德王立《太一福應集要》一卷



 楊惟德《景祐遁甲符應經》三卷



 《七曜神氣經》二卷楊惟德、王立、李自正、何湛等撰



 丘浚《霸國環周立成歷》一卷



 張中《太一金照辨誤歸正論》一卷



 魏申《太一總鑒》一百卷



 上官
 經邦《大始元靈洞微數》一卷



 張宏國《五行志訛辨》一卷



 黃石公《地鏡訣》一卷一名《照寶歷》,題東方朔進



 庾肩吾《金英玉髓經》一卷



 陶弘景《握鏡圖》一卷



 陳樂產《神樞靈轄經》十卷



 李靖《九天玄機八神課》一卷



 《六壬透天關法》一卷



 李鼎祚《明鏡連珠》十卷



 呂才《廣濟百忌歷》二卷



 李淳風《乾坤秘奧》一卷



 《九天觀燈法》一卷



 《六壬精髓經》卷一名《竅中經》



 《質龜論》一卷李淳風得於石室



 僧一行《肘後術》一卷



 《選日聽龍經》一卷



 僧令岑《六壬翠羽歌》三卷



 漢道士姚可久《山陰道
 士經》三卷



 碧眼先生《壬髓經》三卷茅山野叟湯渭注



 《發蒙陵西集》一卷



 《發蒙入式真草》一卷



 《陰陽集正歷》三卷



 《選日纂聖精要》一卷



 《玄女關格經》一卷



 皆六壬占驗之訣



 《式法》一卷起甲子,終癸亥,皆六壬推驗之法



 《雜占覆射》一卷



 《六壬金經玉鑒》一卷載六壬生旺克殺之數



 《萬年秘訣》一卷載檢擇日辰吉兇之法



 《玉女肘後術》一卷以六壬三傳之法為歌



 《玉關歌》一卷載六壬三傳之驗



 《黃河瓶子記》一卷



 《神樞萬一秘要經》一卷



 《越覆經》一卷



 《事神歌》一卷



 《會靈經》一卷載六壬雜占之法



 《纈翠經》一卷



 《灰火經》一卷



 《蛇髓經》
 一卷以日辰衰旺為占



 《九門經》一卷



 《小廣濟立成雜歷》一卷



 《文武百官赴任上官壇經》一卷



 《玄通玉鏡占》一卷



 《六壬課秘訣》一卷



 《六壬課鈐》一卷



 玉樞真人《玄女截壬課訣》一卷



 《占燈法》一卷



 《三鏡篇》一卷



 《周易神煞旁通歷》一卷



 《雜占秘要》一卷



 《乾坤變異錄》一卷



 《玄女簡要清華經》三卷



 《太一占鳥法》一卷



 《參玄通正歷》一卷



 《擇日要法》一卷



 《選時圖》二卷



 《黃帝龍首經》一卷



 《易鑒》一卷



 《月纂》一卷



 《萬勝候天集》一卷



 並不知作者



 《雲雨賦》一卷《崇文總目》有劉啟明《占候雲賦式》,即此書



 鄭德源《飛電歌》一卷



 僧紹端《神釋應夢書》三卷



 詹省遠《夢應錄》一卷



 楊惟德《六壬神定經》十卷



 王升《六壬補闕新書》五卷



 《上官撮要》一卷



 陳從吉《類編圖注萬歷會同》三十卷



 劉氏《三歷會同》一卷



 周渭《彈冠必用》一卷



 胡舜申《陰陽備用》十三卷



 趙希道《涓吉撮要》一卷



 顧晃《壇經簪飾》一卷



 蔣文舉《陰陽備要》一卷



 趙景先《拜命歷》一卷



 徐道符《六壬歌》三卷



 陸漸《六壬了了歌》一卷



 餘琇《六壬玄鑒》一卷



 王齊《醫門玉髓課》一卷



 張玄達《相押字法》一
 卷



 苗公達《六壬密旨》二卷



 楊稠《六壬旁通歷》一卷



 劉玄之《月令節候圖》一卷



 姜岳《六壬賦》三卷



 楊可《五行用式事神》一卷



 郭璞《山海經》十八卷



 趙浮丘公《相鶴經》一卷



 左慈《助相規誡》一卷



 郭璞《葬書》一卷



 《山海圖經》十卷郭璞序,不著姓名



 袁天剛《玄成子》一卷



 孫思邈《坐照論並五行法》一卷



 柳清風、周世明等《玉冊寶文》八卷



 李淳風《立觀經》一卷



 僧一行《地理經》十五卷



 《呼龍經》一卷



 《金歌四季氣色訣》一行撰論



 孫知古《人倫龜鑒》三卷



 王澄《陰陽二宅集要》
 二卷



 僧正固《骨法明鏡》三卷



 丘延翰《銅函記》一卷



 《天定盤古局》一卷



 漢赤松子《海角經》一卷



 《明鏡碎金》七卷



 唐舉《肉眼通神論》三卷



 《金鎖歌》一卷



 鬼谷子《觀氣色出圖相》一卷



 黃石公《八宅》二卷



 許負《相訣》三卷



 李淳風《一行禪師葬律秘密經》十卷



 呂才《楊烏子改墳枯骨經》一卷



 曾楊一《青囊經歌》二卷



 楊救貧《正龍子經》一卷



 曾文展《八分歌》一卷



 李筌《金華經》三卷



 宋齊丘《玉管照神局》二卷



 《天花經》三卷序云:「黃巢得於長安。」



 晏氏《辨氣色上面詩》一卷不知
 名



 劉虛白《三輔學堂正訣》一卷



 危延真《相法》一卷



 《五星六曜面部訣》一卷



 裴仲卿《玄珠囊骨法》一卷



 劉度具《氣色真相法》一卷



 王希逸《地理秘妙歌訣》一卷



 《地理名山異形歌》一卷



 孫臏《葬白骨歷》卷亡



 隱逸人《玉環經》一卷不知姓名



 《天涯海角經》一卷不知作者,九江李麟批注



 徽宗《太平睿覽圖》一卷



 陳摶《人倫風鑒》一卷



 司空班、範越鳳《尋龍入式歌》一卷



 王洙《地理新書》三十卷



 蘇粹明《地理指南》三卷



 蔡望《五家通天局》一卷



 《報應九星妙術文局》一卷



 劉次莊《青
 囊本旨論二十八篇》一卷



 胡翊《地理詠要》三卷



 魏文卿《撥沙經》一卷



 李誡《營造法式》三十四卷



 《月波洞中記》一卷



 《月師歌》一首言葬地二十四位星辰休咎



 《麻子經》一卷



 《玄靈子》三卷



 《通心經》三卷



 《藻鑒淵微》一卷



 《雜相骨聽聲》卷亡



 《氣色微應》三卷



 《通微妙訣》卷亡



 《中定聲氣骨法》卷亡



 《金歌氣色秘錄》一卷



 《學堂氣骨心鏡訣》卷亡



 《玉葉歌》一卷



 《洞靈經要訣》一卷



 《雜相法》一卷



 《天寶星經》一卷



 《青囊經》卷亡



 《陰陽七元升降論》卷亡



 《玄女墓龍塚山年月》一卷



 《玄女星羅寶圖訣》一
 卷



 《紫微經歌》卷亡



 《白鶴望山經》一卷



 《八山二十四龍經》一卷



 《天仙八卦真妙訣》一卷



 《黃泉敗水吉兇法》三卷



 《踏地賦》一卷



 《分龍真殺五音吉兇進退法》一卷



 《地理澄心秘訣》一卷



 《八山穿珠歌》一卷



 《山頭步水經》一卷



 《山頭放水經》一卷



 《大卦煞人男女法》一卷



 《地理搜破穴訣》一卷



 《臨山寶鏡斷風訣》一卷



 《叢金訣》一卷



 《錦囊經》一卷



 《玉囊經》一卷



 《黃囊大卦訣》一卷



 《地理秘要集》一卷



 《通玄論》一卷



 《地理八卦圖》一卷



 《駐馬經》一卷



 《活曜修造吉兇法》一卷



 《
 天中寶經知吉兇星位法》一卷



 《修造九星法歷代史相》一卷



 《相具經》一卷



 並不知作者



 《李仙師五音地理訣》一卷



 赤松子《碎金地理經》二卷



 《地理珠玉經》一卷



 《地理妙訣》三卷



 《石函經》十卷



 《銅函經》三卷



 《周易八龍山水論地理》一卷



 《老子地鑒訣秘術》一卷



 《五姓合諸家風水地理》一卷



 《昭幽記》一卷



 《鬼靈經並枯骨經》二卷



 《唐刪定陰陽葬經》二卷



 《唐書地理經》十卷



 《青烏子歌訣》二卷



 《金雞歷》一卷



 《五音二十八將圖》一卷



 《赤松子》三卷



 《易括地林》一卷



 丘延
 翰《五家通天局》一卷



 《夫子掘斗記》一卷



 《孔子金金巢記》一卷



 《推背圖》一卷



 鬼谷子《白虎經》一卷



 又《白虎五通經訣》一卷



 《洞幽秘要圖》一卷



 《孝經雌雄圖》四卷



 河上公《金藏秘訣要略》一卷



 《玄珠握鑒》三卷



 《玉函寶鑒》三卷



 《真人水鑒》十三卷



 張華《三鑒靈書》三卷



 陶弘景《握鑒方》三卷



 《證應集》三卷



 《金婁先生秘訣》三卷



 《真圖秘訣》一卷



 《銘軌》五卷



 胡濟川《小游七十二局立成》一卷



 《大小游三奇五福立成》一卷



 《十一神旁通太歲甲子圖》一卷



 曹植《黃帝寶
 藏經》三卷



 《括明經》一卷



 《悟迷經》一卷



 余考一作「秀」



 《旦暮經》一卷



 《神樞萬一秘經》一卷



 《紀重政秘要》一卷



 《雷公印法》三卷



 《雷公撮殺律》一卷



 徐遂《發蒙》一卷



 《玄女十課》一卷



 呂佐周《地論七曜》一卷



 《陰術氣神》一卷



 《七曜氣神歷代帝紀》五卷



 《玉堂秘訣》一卷



 《大運秘要心髓訣》一卷



 呂才《陰陽書》一卷



 《五姓鳳髓寶鑒論》一卷



 《陰陽雜要》一卷



 《玄珠錄要》三卷



 張良《玄悟歌》一卷



 《斗書》一卷



 《陰陽》二卷



 《論》一卷



 《黃帝四序經》一卷



 《寶臺七賢論》一卷



 《五姓玉訣旁
 通》一卷



 《選日時向背》五卷



 《陰陽立成選日圖》一卷



 《七曜選日》一卷



 《周公要訣圖》一卷



 《師曠擇日法》一卷假黃帝問答



 《淮南子術》一卷



 《推貴甲子太極尊神經》一卷



 《秘訣歌》一卷



 《福應集》十卷



 《連珠經》十卷



 《玄女斷卦訣》一卷



 《明體經》一卷



 《心注瓶子記》一卷



 《錦繡囊》一卷



 《心鏡歌》三卷



 《指要》三卷



 《萬一訣》一卷



 《符應》三卷



 《隨軍樞要》三卷



 《禳厭秘術詩》三卷



 《廣知集》二卷



 《圓象玄珠經》五卷



 《脈六十四卦歌訣》一卷



 《人元秘樞經》三卷



 《陶隱居》一卷



 《風後》一
 卷



 《李寬》一卷



 《通元論》三卷



 《凝神子》三卷



 《黃帝四序經》一卷



 《聿斯四門經》一卷



 《氣神經》三卷



 《氣神帝紀》五卷



 《符天人元經》一卷



 《聿斯經訣》一卷



 《大定露膽訣》一卷



 《聿斯都利經》一卷



 《應輪心鏡》三卷



 《秤經》三卷



 《聿斯隱經》三卷



 《碧落經》十卷



 《新書》三十卷



 《三鏡》三卷



 《九天玄女訣》一卷



 《龍母探珠頌》一卷



 《通玄玉鑒頌》一卷



 《征應集》三卷



 王與之《鼎書》十七卷



 右五行類八百五十三部,二千四百二十卷。



 《
 三墳易典》三卷題箕子注



 《周易三備》三卷題孔子師徒所述,蓋依托也



 嚴遵《卦法》一卷



 焦贛《易林傳》十六卷



 京房《易傳算法》一卷



 《易傳》三卷



 管輅《遇仙訣五音歌》六卷



 《周易八仙歌》三卷



 《易傳》一卷



 郭璞《周易洞林》一卷



 呂才《軌限周易通神寶照》十五卷



 李淳風《周易玄悟》三卷



 易通子《周易菥蓂璇璣軌革口訣》三卷



 蒲乾貫《周易指迷照膽訣》三卷



 黃法《五兆曉明龜經》一卷



 祿隱居士《易英揲蓍圖》一卷不知名



 中條山道士王鄯《易鏡》三卷



 無惑先生《易鏡正經》二
 卷



 耿格《大演天心照》一卷



 牛思純《太極寶局》一卷



 任奉古《明用蓍求卦》一卷



 林壝《天道大備》五卷



 《軌革金庭玉鑒》七卷



 《周易神鏡鬼谷林》一卷



 《通玄海底眼》一卷



 《六十四卦頌諭》一卷



 《爻象雜占》一卷



 《六十四卦火珠林》一卷



 《周易靈秘諸關歌》一卷



 《齕骨林》一卷



 《靈龜經》一卷



 《軌革傳道錄》一卷



 《證六十甲子納音五行》一卷



 《龜圖》一卷



 《周易贊頌》六卷



 並不知作者



 右蓍龜類三十五部,一百卷



\end{pinyinscope}