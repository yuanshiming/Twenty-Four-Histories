\article{列傳第一百六十四}

\begin{pinyinscope}

 ○李
 宗勉袁甫劉黻王居安



 李宗勉,字強父,富陽人。開禧元年進士。歷黃州教授、浙西茶鹽司、江西轉運司干官。嘉定十四年,主管吏部架閣,尋改太學正。明年為博士,又明年遷國子博士。寶慶
 初,添差通判嘉興府。三年,召為秘書郎。



 紹定元年,遷著作郎。入對,言邊事宜夙夜震懼,以消咎殃。明年,兼權兵部郎官。時李全叛謀已露,人莫敢言,宗勉獨累疏及之。又言:「欲人謀之合,莫若通下情。人多好諂,揣所悅意則侈其言,度所惡聞則小其事。上既壅塞,下亦欺誣,則成敗得失之機、理亂安危之故,將孰從而上聞哉?不聞則不戒,待其事至乃駭而圖之,抑已晚矣。欲財計之豐,莫若節國用。善為國者常使財勝事,不使事勝財。今山東
 之旅,坐糜我金谷,湖南、江右、閩中之寇,蹂踐我州縣,茍浮費泛用,又從而侵耗之,則漏卮難盈,蠹木易壞。設有緩急,必將窘於調度,而事機失矣。欲邦本之固,莫若寬民力。州縣之間,聚斂者多,椎剝之風,浸以成習。民生窮踧,怨憤莫伸,嘯聚山林,勢所必至。救焚拯溺,可不亟為之謀哉?」尋改兼侍右郎官。明年入對,言天災甚切。



 四年,差知臺州。明年,直秘閣、知婺州。六年冬,召赴行在,未行。端平元年,進直寶章閣,依舊任。越月,以宗正丞兼權右
 司召,改尚左郎官,兼職仍舊。尋兼左司。五月,面對,言四事:「守公道以悅人心,行實政以興治功,謹命令以一觀聽,明賞罰以示勸懲。」次言楮幣:「願詔有司,始自乘輿宮掖,下至百司庶府,核其冗蠹者節之,歲省十萬,則十萬之楮可捐,歲省百萬,則百萬之楮可捐也。行之既久,捐之益多,錢楮相當,所至流轉,則操吾贏縮之柄不在楮矣。」



 拜監察御史。時方謀出師汴、洛,宗勉言:「今朝廷安恬,無異於常時。士卒未精銳,資糧未充衍,器械未犀利,城
 壁未繕修。於斯時也,守御猶不可,而欲進取可乎?借曰今日得蔡,明日得海,又明日得宿、毫,然得之者未必可守。萬一含怒蓄忿,變生倉猝,將何以濟?臣之所陳,豈曰外患之終不可平、土宇終不可復哉?亦欲量力以有為、相時而後動耳。願詔大臣,愛日力以修內治,合人謀以嚴邊防,節冗費以裕邦財,招強勇以壯國勢。仍飭沿邊將帥,毋好虛名而受實害,左控右扼,毋失機先。則以逸待勞,以主御客,庶可保其無虞。若使本根壯固,士馬精
 強,觀釁而動,用兵未晚。」已而洛師潰,又言:「昔之所慮者在當守而冒進,今之所慮者在欲守而不能。何地可控扼,何兵可調遣,何將可捍禦,何糧可給餉,皆當預作措畫。」又言內降之敝,大略謂:「王府後宅之宮僚,戚里奄寺之恩賞,綸綍直下,不經都省,竿牘陳請,時出禁廷,此皆大臣所當執奏。夫先事而言,見幾而諫,不可謂之專。善則行之,否則止之,不可謂之專。命出君上,政歸中書,不可謂之專。茍以專權為嫌,不以救過為急,每事希旨迎
 合,迨其命令已下,闕失已彰,然後言事之人從而論列之,其累聖德亦多矣。況言之未必聽,聽之未必行乎?」



 進左司諫。明年春,兼侍講。首言:「均、房、安、蘄、光、化等處兵禍甚烈,然江面可藉以無憂者,猶有襄州,今又告變矣。襄州失則江陵危,江陵危則長江之險不足恃。昔之所慮猶在秋,今之所慮者祗在旦夕。江陵或不守,則事迫勢蹙,必有存亡之憂,悔將何及?」拜殿中侍御史。時淮西制置使兼沿江制置副使史嵩之兼知鄂州,就鄂建牙。宗
 勉言:「荊、襄殘破,淮西正當南北之交,嵩之當置司淮西,則脈絡相連,可以應援,邈在鄂渚,豈無鞭不及腹之慮。若雲防江為急,欲藉嵩之於鄂渚經理,然齊安正與武昌對,如就彼措置防扼,則藩籬壯而江面安矣。所謂欲保江南先守江北也。當別擇鄂守,徑令嵩之移司齊安。」



 詔侍從、兩省、臺諫條陳邊事,宗勉率合臺奏:「蜀之四路,已失其二,成都隔絕,莫知存亡。諸司退保夔門,未必能守。襄漢昨失九郡,今郢破,荊門又破,江陵孤城,何以能
 立?兩淮之地,人民奔迸,井邑丘墟,嗚呼危哉!陛下誠能亟下哀痛之詔,以身率先,深自貶損,服御飲宴,一從簡儉,放後宮浮食之女,罷掖庭不急之費,止錫齎,絕工役,出內帑儲蓄以風動四方。然後勸諭戚畹、世臣,隨力輸財,以佐公家之調度。分上流淮西、淮東為三帥,而以江淮大帥總之。或因今任,或擇長才,分地而守,聽令而行。以公私之財分給四處,俾之招潰卒,募流民之強壯者,以充游兵,以補軍籍。仍選沿流諸郡將士為捍禦之圖,
 猶可支吾。不然將水陸俱下,大合荊楚之眾,擾我上流,江以南震蕩矣。或謂其勢強盛,宜於講和,欲出金繒以奉之,是抱薪救火,空國與敵矣。」



 進工部侍郎兼給事中,仍侍講。復上疏言:「陛下憂勤於路朝之頃,而入為宴安所移,切劘於廣廈之間,而退為便佞所惑。不聞減退宮女,而嬪嬙已溢於昔時;不聞褒錄功臣,而節鉞先加於外戚;不聞出內貯以犒戰士,而金帛多靡於浮費。陛下之舉動,人心所視以為卷舒者也。陛下既不以為憂,則
 誰復為陛下憂。」擢諫議大夫兼侍讀。首言邊事當增兵防托上流。又言:「求諫非難而受諫為難,受諫非難而從諫為難。茍聞之不以為戒,玩之不以為信,卒使危言鯁論,無益於世用,無救於時危,其與拒諫者相去一間耳。」



 進端明殿學士、同簽書樞密院事。未幾,進簽書。時王楫復求歲幣銀絹各二十萬,宗勉言:「輕諾者多後患,當守元約可也。然比之開禧時,物價騰踴奚啻倍蓰矣。」史嵩之開督府,力主和議,宗勉言:「使者可疑者三。嵩之職在
 督戰,如收復襄、光,控扼施、澧,招集山砦,保固江流,皆今所當為。若所主在和,則凡有機會可乘,不無退縮之意,必至虛捐歲月,坐失事功。」



 進參知政事。及拜左丞相兼樞密使,守法度,抑僥幸,不私親黨,召用老成,尤樂聞讜言。趙汝騰嘗以宗勉為公清之相。以光祿大夫、觀文殿大學士致仕,卒,贈少師,謚文清。



 袁甫,字廣微,寶文閣直學士燮之子。嘉定七年進士第一。簽書建康軍節度判官廳公事,授秘書省正字。入對,
 論「君天下不可一日無懼心。今之可懼者,大端有五:端良者斥,諂諛者用,杜忠臣敢諫之門,可懼也;兵戈既興,饋餉不繼,根本一虛,則有蕭墻之憂,可懼也;陛下深居高拱,群臣奉行簿書,獨運密謀之意勝,而虛心咨訪之意微,天下迫切之情無由上聞,可懼也;外患未弭,內患方深,而熙熙然無異平時,自謂雅量足以鎮浮,不知宴安實為鴆毒,可懼也;陛下恭儉有餘,剛斷不足,庸夫憸人,茍求富貴,而未聞大明黜陟,軍帥交結,州郡賄賂,皆
 自貴近化之,可懼也。其它禍幾亂萌,不可悉數,將何以答天譴、召和氣哉?」次乞嚴守帥之選,並大軍之權,興屯田之利。



 遷校書郎,轉對,言「邊事之病,不在外而在內。偷安之根不去,規摹終不立;壅蔽之根不去,血脈終不通;忌嫉之根不去,將帥終不可擇;欺誕之根不去,兵財終不可治。祖宗之御天下,政事雖委中書,然必擇風採著聞者為臺諫,敢於論駁者為給、舍,所以戢官邪、肅朝綱也。今日誠體是意以行之,豈復有偷安壅蔽者哉?」出通
 判湖州,考常平敝原以增積貯,核隱產,增附嬰兒局。



 遷秘書郎,尋遷著作佐郎、知徽州。治先教化,崇學校,訪便民事上之:請蠲減婺源綢絹萬七千餘匹,茶租折帛錢萬五千餘貫,月樁錢六千餘貫;請照咸平、紹興、乾道寬恤指揮,受納徽絹定每匹十兩;請下轉運、常平兩司,豫蓄常平義倉備荒,興修陂塘,創築百梁。丁父憂,服除,知衢州。立旬講,務以理義淑士心,歲撥助養士千緡。西安、龍游、常山三邑積窘預借,為代輸三萬五千緡,蠲放四
 萬七千緡。郡有義莊,買良田二百畝益之。



 移提舉江東常平。適歲旱,亟發庫庾之積,凡州縣窠名隸倉司者,無新舊皆住催,為錢六萬一千緡,米十有三萬七千、麥五千八百石,遣官分行振濟,饑者予粟,病者予藥,尺籍之單弱者,市民之失業者,皆曲軫之。又告於朝曰:「江東或水而旱,或旱而水,重以雨雪連月,道殣相望,至有舉家枕藉而死者。此去麥熟尚賒,事勢益急。」詔給度牒百道助費。時江、閩寇迫饒、信,慮民情易動,分榜諭安之。檄諸
 郡,關制司,聞於朝,為保境捍患之圖,寇迄不犯。遂提點本路刑獄兼提舉,移司番陽。霜殺桑,春夏雨久湖溢,諸郡被水,連請於朝,給度牒二百道賑恤之。盜起常山,調他州兵千人屯廣信以為備。



 都城大火,上封事言:「上下不交,以言為諱,天意人心,實同一機,災變之作,端由於此。願下哀痛之詔,以回天意。」詔求直言,復上疏言:「災起都邑,天意蓋欲陛下因其所可見,察其所不可見,行至公無私之心,全保護大臣之體,率屬群工,大明黜陟,與
 天下更始。」行部問民疾苦,薦循良,劾奸貪,決滯獄。所至詣學宮講說,創書院貴溪之南,祠先儒陸九淵。歲大旱,請於朝,得度牒、緡錢、綾紙以助賑恤。疫癘大作,創藥院療之。前後持節江東五年,所活殆不可數計。轉將作監,領事如故。繼力辭常平事。彗星見,詔求直言,上疏言:「皇天所以震怒者,由愁苦之民眾;人民所以愁苦者,由貪冒之風熾。願一變上下交征之習,為大公至正之歸。」



 帝親政,以直微猷閣知建寧府,明年,兼福建轉運判官。閩
 鹽隸漕司,例運兩綱供費,後增至十有二,吏卒並緣為奸,且抑州縣變賣,公私苦之,甫奏復舊例。丁米錢久為泉、漳、興化民患,會知漳州趙以夫請以廢寺租為民代輸,甫並捐三郡歲解本司錢二萬七千貫助之。郡屯左翼軍,本備峒寇,招捕司移之江西,甫檄使還營。俄寇作唐石,即調之以行,而賊悉平。遷秘書少監。入見,帝曰:「卿久勞於外,篤意愛民,每覽所陳,備見懇惻。」甫奏《無逸》之義,言知農夫稼穡艱難,自然逸欲之念不起。乞力守更
 化以來求賢如不及之初意。



 遷起居舍人兼崇政殿說書。於經筵奏:「剛之一字,最切於陛下。陛下徒有慕漢宜厲精為治之名,而乃墮元帝、文宗柔弱不振之失。元帝、文宗果斷,不用於斥邪佞,反用於逐賢人,此二君不識剛德之真。所謂真剛者,當為之事必行,不當為者則斷在勿行。」又乞「專意經訓,養育精神,務令充實,上與天一,下合人心。」帝意欲全功臣之世,詔自今中外臣僚奏事,毋得捃摭,以奏:「是消天下讜言之氣,其謂陛下何?」兼中
 書舍人,繳奏不擿苛小,謂:「監司、郡守非其人,則一道一州之蠹也。」



 時相鄭清之以國用不足,履畝使輸券。甫奏:「避是虐賤,有力者頑未應令,而追呼迫促,破家蕩產,悲痛無聊者,大抵皆中下之戶。」嘗講罷,帝問近事,甫奏:「惟履畝事,人心最不悅。」又嘗讀《資治通鑒》,至漢高祖入關辭秦民牛酒,因奏:「今日無以予人,反橫科之,其心喜乎,怒乎?本朝立國以仁,陛下以為此舉仁乎,否乎?」帝為惻然。



 時朝廷以邊事為憂,史嵩之帥江西,力主和議。甫奏
 曰:「臣與嵩之居同里,未嘗相知,而嵩之父彌忠,則與臣有故。嵩之易於主和,彌忠每戒其輕易。今朝廷甘心用父子異心之人,臣謂不特嵩之之易於主和,抑朝廷亦未免易於用人也。」疏入,不報。遂乞歸,不允。授起居郎兼中書舍人。未幾,擢嵩之刑部尚書,復奏疏云:「臣於嵩之本無仇怨,但國事所系,誼難緘默。」嵩之誥命,終不與書行,乃出甫知江州。王遂抗疏力爭,帝曰:「本以授其兄袁肅,報行誤耳。」令遂勉甫無它志。翼日,乃與肅江州。而殿
 中侍御史徐清叟復論甫守富沙日贓六十萬,湯巾等又爭之,清叟亦悔。未幾,改知婺州,不拜。



 喜熙元年,遷中書舍人。入見,陳心源之說,帝問邊事,甫奏:「當以上流為急,議和恐誤事。」時清叟與甫並召,而清叟未至。甫奏:「臺諫風聞言事,初亦何心。今人物眇然,有如清叟宜在朝廷,辭避實惟臣故,乞趣其赴闕。」又奏備邊四事,曰:固江陵,堰瓦梁,與流民復業。嵩之移京湖沿江制置使、知鄂州,甫奏曰:「嵩之輕脫難信。去年嵩之在淮西,楫由淮
 西而來,北軍踵之。今又並湖南付之,臣恐其復以誤淮西者誤湖南。」疏留中不行。翼日,權吏部侍郎。引疾至八疏,賜告一月,遂歸。從臣復合奏留之,尋命兼修玉牒官兼國子祭酒,皆辭不拜。改知嘉興府,知婺州,皆辭不拜。



 遷兵部侍郎,入見,奏:「江潮暴湧,旱魃為虐,楮幣蝕其心腹,大敵剝其四支,危亡之禍,近在旦夕,乞秉一德,塞邪徑。」兼給事中。岳珂以知兵財召,甫奏珂總餉二十年,焚林竭澤,珂竟從外補。遷吏部侍郎兼國子祭酒,日召諸
 生叩其問學理義講習之益。時邊遽日至,甫條十事,至為詳明。權兵部尚書,暫兼吏部尚書,卒,贈通奉大夫,謚正肅。有《孝說》、《孟子解》、《後省封駁》、《信安志》、《江東荒政錄》、《防拓錄》、《樂事錄》及文集行世。



 甫少服父馴,謂學者當師聖人,以自得為貴。又從楊簡問學,自謂「吾觀草木之發生,聽禽鳥之和鳴,與我心契,其樂無涯」云。



 劉黻,字聲伯,樂清人。早有令聞,讀書雁蕩山中僧寺。年三十四,以淳祐十年試入太學,儕輩已翕然稱之。時丁
 大全方為臺屬,劾奏丞相董槐,迫逐去國,將奪其位。黻率同舍生伏闕上書,大概言朝廷進退大臣,須當以禮。書上,忤執政,送南安軍安置,歸別其母解氏。解氏曰:「為臣死忠,以直被貶,分也。速行!」黻至南安,盡取濂、洛諸子之書,摘其精切之語,輯成書十卷,名曰《濂洛論語》。及大全貶,黻還太學。未幾,侍御史陳垓誣劾程公許,右正言蔡滎誣劾黃之純,二公罷出,六館相顧失色,黻又率諸生上書言:



 黻等蒙被教養,視國家休戚利害若己痛養。
 朝廷進一君子,臺諫發一公論,則彈冠相慶,喜溢肺膺。至若君子鬱而不獲用,公論沮而不克伸,則憂憤忡結,寢食俱廢。臣聞扶植宗社在君子,扶植君子在公論。陛下在位幾三十年,端平間公正萃朝,忠讜接武,天下翕然曰:「此小元祐也。」淳祐初,大奸屏跡,善類在位,天下又翕然曰:「此又一端平也。」奈何年來培養保護之初心,不能不為之轉移。



 祖宗建置臺諫,本以伸君子而折小人,昌公論而杜私說。乃今老饕自肆,奸種相仍,以諂諛承
 風旨,以傾險設機阱,以淟涊盜官爵。陛下非不識拔群賢,彼則忍於空君子之黨;陛下非不容受直言,彼則勇於倒公議之戈。不知陛下何負此輩,而彼乃負陛下至此耶?



 當陛下詔起匯髦之秋,而公許起自家食,正君子覘之,以為進退之機。乃今坐席未溫,彈章已上,一公許去,若未害也,臣恐草野諸賢,見幾深遁,而君子之脈自此絕矣。比年朋邪記焰,緘默成風,奏事者不過襲陳言、應故事而已。幸而之純兩疏,差強人意。乃今軟媚者全
 身,鯁直者去國,一之純去,若未害也,臣恐道路以目,欲言輒沮,而公論之脈自此絕矣。



 況今天下可言之事不為少,可攻之惡不為不多。術窮桑、孔,浸有逼上之嫌;勢挾金、張,濫處牧民之職。以乳臭騃子而躐登從橐,以光範私人而累典輔藩。錢神通靈於旁蹊,公器反類于互市。天下皆知之,豈陛下獨不知之。正惟為陛下紀綱者知為身謀,不為陛下謀。陛下明燭事幾,詎可墮此輩蒙蔽術中,何忍以祖宗三百年風憲之司,而壞於一二
 小人之手耶?臣汝騰,陛下之劉向也,則以忠鯁斥;臣子才、臣棟、臣伯玉,陛下之汲黯也,則以切直罷。遂使淳祐諸君子日消月磨,至今幾為之一空。彼誠何心哉?



 高宗紹興二十年之詔,有謂「臺諫風憲之地,年來用人非據,與大臣為友黨,濟其喜怒,甚非耳目之寄。」臣竊觀近事,不獨臺諫為大臣友黨,內簡相傳,風旨相諭,且甘為鷹犬而聽其指嗾焉。宰相所不樂者,外若示以優容,而陰實頤指臺諫以去之;臺諫所彈擊者,外若不相為謀,而陰
 實奉承宰相以行之。方公許之召也,天下皆知獨斷於宸衷,及公許之來也,天下亦知嘗得罪於時宰,豈料陛下之恩終不足恃,宰相之嗔竟不可逃耶?



 陛下萬機之暇,試以公許、之純與垓、滎等熟思而靜評之,其言論孰正孰邪,孰忠孰佞,雖中智以下之主,猶知判別是非,況以陛下明聖而顧不察此?近見公許奏疏,嘗告陛下揭至公以示天下;垓則以秘密之說惑上聽,公許嘗告陛下以寵賂日章,官邪無警,欲塞幸門,絕曲徑;垓則縱俠
 客以兜攬關節,持闊扁以脅取舉狀,開賂門以簸弄按章。至若之純之告陛下,力伸邪正之辯,明斥媚相之非,謇謇諤諤,流出肺肝;滎身居言責,聞其風聲,自當愧死,尚敢妄肆萋菲,略無人心乎?



 且陛下擢用臺諫,若臣磊卿、臣咨夔、臣應起、臣漢弼、臣凱、臣燧,光明俊偉,卓為天下稱首,然甫入而遽遷,或一鳴而輒斥,獨垓、滎輩貪饕頑忍,久污要津,根據而不拔,劉向所謂「用賢轉石,去佞拔山」者,乃今見之,可不畏哉?矧今國嗣未正,事會方殷,
 民生膏血,朘削殆盡,所賴以祈天命,系人心,惟君子與公論一脈耳。小人以不恤之心,為無忌憚之事,其意不過欲爵位日穹,權勢日盛,以富貴遣子孫耳,豈暇為國家計哉。



 自昔天下之患,莫大於舉朝無公論,空國無君子。我朝本無大失德於天下,而乃有宣、靖之禍,夫豈無其故哉?始則邪正交攻,更出迭入,中則朋邪翼偽,陰陷潛詆,終則倒置是非,變亂黑白,不至於黨禍不止。向使劉安世、陳瓘諸賢尚無恙,楊畏、張商英、周秩輩不久據
 臺綱,其禍豈至此烈。古語云:「前車覆,後車戒。」今朝廷善類無幾,心懷奸險者,則以文藻飾佞舌;志在依違者,則以首鼠持圓機。宗社大計,孰肯明目張膽為陛下伸一喙者,則其勢必終於空國無君子,舉朝無公論。無君子,無公論,脫有緩急,彼一二憸人者,陛下獨可倚仗之乎?



 若垓之罪,又浮於滎,雖兩觀之誅,四裔之投,猶為輕典,陛下留之一日,則長一日之禍,異時雖借尚方劍以礪其首,尚何救於國事之萬一哉?



 又曰:「自昔大奸巨孽,投
 閑散地,惟覘朝廷意向,以圖進用之機。元祐間,章惇、呂惠卿皆在貶所。自呂大防用楊畏為御史,初意不過信用私人,牢護局面,不知小人得志,搖唇鼓吻,一時正人旋被斥逐,繼而章惇復柄用,雖大防亦不能安其身於朝廷之上。今右轄久虛,奸臣垂涎有日矣。聞之道路,饋遺不止於鞭靴,脈絡潛通於禁近,正陛下明察事機之時。若公論不明,正人引去,則遲回展轉,鈞衡重寄,必歸於章惇等乃止。今日之天下,乃祖宗艱難積累之天下,
 豈堪此輩再壞耶?」



 又諫游幸疏曰:



 天下有道,人主以憂勤而忘逸樂;天下無道,人主以逸樂而忘憂勤。自昔國家乂安,四夷賓服,享國日久,侈心漸生,若漢武帝之單于震懾,而有千門萬戶之觀,唐明皇之北邊無事,而有驪山溫泉之幸。至於隋之煬帝,陳之後主,危亡日迫,游觀無度,不足效也。堯、舜、禹、湯、文、武之競業祗懼,終始憂勤,《無逸》言:「游畋則不敢,日昃則不暇食。曷嘗借祈禳之說,以事游觀之逸。比年以來,以幸為利,以玩為常,未免
 有輕視世故、眇忽天下之心。單于未嘗震懾,而有武帝多欲之費耗;北邊未嘗無事,而有明皇宴安之鴆毒。



 陛下春秋尚少,貽謀垂憲之機,悉在陛下,作而不法,後嗣何觀?自十數年間,創龍翔,創集慶,創西太一,而又示之以游幸,導之以禱祠,蠱之以虛誕不經之說。孔子曰:「少成若天性,習慣如自然。」積久慣熟,牢不可破,誰得而正之?且西太一之役,佞者進曰:「太一所臨分野則為福,近歲自吳移蜀。」信如祈禳之說,西北坤維按堵可也。今五
 六十州,安全者不能十數,敗降者相繼,福何在邪?武帝祠太一於長安,至晚年以虛耗受禍,而後悔方士之繆。雖其悔之弗早,猶愈於終不知悔者也。



 大凡人主不能無過,脫有過言過行,宰執、侍從當言之,給舍、臺諫當言之,縉紳士大夫當言之,皆所以納君於當道者也。今陛下未為不知道,未為不受人言,宰執以下希寵而不言,與夫言之而不力,皆非所以愛陛下也。其心豈以此為當而不必言哉?直以陛下為不足以望堯、舜、禹、湯、文、武
 之主,而以漢武、明皇待陛下也。



 以材署昭慶軍節度掌書記,由學官試館職。咸淳三年,拜監察御史,論內降恩澤曰:



 治天下之要,莫先於謹命令,謹命令之要,莫先於窒內批。命令,帝王之樞機,必經中書參試,門下封駁,然後付尚書省施行,凡不由三省施行者,名曰「斜封墨敕」,不足效也。臣睹陛下自郊祀慶成以來,恩數綢繆,指揮煩數,今日內批,明日內批,邸報之間,以內批行者居其半,竊為陛下惜之。



 出納朕命載於《書》,出納王命詠於《詩》,
 不專言出而必言納者,蓋以命令系朝廷之大,不能皆中乎理,於是有出而復有納焉。祖宗時,禁中處分軍國事付外者謂之內批,如取太原、下江南,韓琦袖以進呈,英宗悚然避坐,此豈非謹內批之原哉?臣日夜念此,以為官爵陛下之官爵,三省陛下之三省,所謂同奉聖旨,則是三省之出命,即出陛下之命也,豈必內批而後為恩?緣情起事,以義制欲,某事當行,某事當息,具有條貫,何不自三省行之,其有未穆於公論者,許令執奏,顧不
 韙歟。



 元祐間,三省言李用和等改官移鎮恩例,今高氏、朱氏,皆舉故事,皇太后曰:「外家恩澤,方欲除損,又可增長乎?」治平初,欲加曹佾使相,皇太后再三不許;又有聖旨,令皇后本家分析親的骨肉聞奏,亦與推恩,司馬光力諫,以為皇太后既損抑外親,則後族亦恐未宜褒進。乃今前之恩數未竟,後之恩數已乘。宰執懼有所專而不敢奏,給舍、臺諫懼有所忤而不敢言,更如此者數年,將何以為國?故政事由中書則治,不由中書則亂,天下
 事當與天下共之,非人主所可得私也。



 四年,改正字,言:「正學不明則義理日微,異端不息則鼓惑轉熾。臣非不知犯顏逆耳,臣子所難,實以君德世道,重有關系,不容不懇惻開陳。疏上逾日,未蒙付外。孟軻有云:『有言責者,不得其言則去』。臣忝職諫省,義當盡言,今既不得其言,若更貪慕恩榮,不思引去,不惟有負朝廷設官之意,其於孟軻明訓,實亦有慊。」



 會丁父憂去位,服除,授集英殿修撰,沿海制置、知慶元府事。建濟民莊,以濟士民之急,
 資貢士春官之費,備郡庠耆老緩急之需。又請建慈湖書院。八年,召還,拜刑部侍郎。九年,改朝奉郎,試吏部尚書,兼工部尚書,兼中書舍人,兼修玉牒,兼侍讀。上疏請給王十朋祠堂田土。十年,丁母憂。明年,江上潰師,丞相陳宜中起復黻為端明殿學士,不起,及賈似道,韓震死,宜中謀擁二王由溫州入海,以兵逆黻共政,將遜相位,於是黻托宗祀於母弟成伯,遂起,及羅浮,以疾卒。



 初,陳宜中夢人告之曰:「今年天災流行,人死且半,服大黃者
 生。」繼而疫癘大作,服者果得不死,及黻病,宜中令服之,終莫能救。其配林氏舉家蹈海。未幾,海上事亦瓦解矣。黻有《蒙川集》十卷行於世。



 王居安,字資道,黃巖人。始名居敬,字簡卿,避祧廟嫌易之。始能言,讀《孝經》,有從旁指曰:「曉此乎?」即答曰:「夫子教人孝耳。」劉孝韙七月八日過其家塾,見居安異凡兒,使賦八夕詩,援筆成之,有思致。孝韙驚拊其背曰:「子異日名位必過我。」入太學,淳熙十四年舉進士,授徽州推官,
 連遭內外艱,柄國者以居安十年不調,將徑授職事官,居安自請試民事,乃授江東提刑司干官。使者王厚之厲鋒氣,人莫敢嬰,居安遇事有不可,平面力爭不少屈。



 入為國子正、太學博士。入對,首言:「人主當以知人安民為要,人未易知,必擇宰輔侍從之賢,使引其類;民未易安,必求愷悌循良之吏,以布其澤。」次言:「火政不修,罪在京尹,軍律不明,罪在殿、步兩司,罪鈞異罰固不可,安有薄罰一步帥而二人置弗問乎?」遷校書郎。居安乞召試,
 言:「祖宗時惟進士第一不試,蘇軾以高科負重名,英宗欲授館職,韓琦猶執不從。」執政謂居安曰:「朝廷於節度尚不較,況館職乎?」居安因言:「節鉞之重,文非位極,武非勛高,胡可妄得。丞相言不較,過矣。」時蘇師旦命且下,故居安言及之。改司農丞。御史迎意論劾,主管仙都觀。



 逾年,起知興化軍。既至,條奏便民事,乞行經界。且言:「蕃舶多得香犀象翠,崇侈俗,洩銅鏹,有損無益,宜遏絕禁止。」皆要務也。通商賈以損米價,誅劇盜以去民害。召為秘
 書丞。轉對,言:「置宣司,不聞進取之良規;遣小使,寂無確許之實報。但當嚴飭守備,益兵據險以待之,此廟算之上也。」李壁嘗語人曰:「比年論疆事無若王秘丞之明白者。」



 遷著作郎兼國史實錄院檢討編修官,兼權考功郎官。誅韓侂胄,居安實贊其決。翼日,擢右司諫。首論:



 侂胄以預聞內禪之功,竊取大權,童奴濫授以節鉞,嬖妾竄籍於官庭。創造亭館,震驚太廟之山;燕樂語笑,徹聞神御之所,忽慢宗廟,罪宜萬死。托以大臣之薦,盡取軍國
 之權。臺諫、侍從,惟意是用,不恤公議;親黨姻婭,躐取美官,不問流品;名器僭濫,動違成法。竊弄威柄,妄開邊隙。自兵端一啟,南北生靈,壯者死鋒刃,弱者填溝壑。荊襄、兩淮之地,暴尸盈野,號哭震天。軍需百費,科擾州縣,海內騷然。跡其罪狀,人怨神怒,眾情洶洶,物議沸騰,而侂胄箝制中外,罔使陛下聞知,宦官宮妾,皆其私人,莫肯為陛下言者。西蜀吳氏,世掌重兵,頃緣吳挺之死,朝廷取其兵柄,改畀它將,其策至善。侂胄與曦結為死黨,假
 之節鉞,復授以全蜀兵權。曦之叛逆,罪將誰歸?使曦不死,侂胄未可知也。



 侂胄數年之間,位極三公,列爵為王,外則專制東西二府之權,內則窺伺宮禁之嚴,奸心逆節,具有顯狀。縱使侂胄身膏斧鉞,猶有餘罪,況兵釁未解,朝廷儻不明正典刑,何以昭國法,何以示敵人,何以謝天下?今誠取侂胄肆諸市朝,是戮一人而千萬人獲安其生也。侂胄既有非常之罪,當伏非常之誅,詎可以常典論哉?



 右丞相陳自強素行污濁,老益貪鄙,徒以貧
 賤私交,自一縣丞超遷,徑至宰輔,奸憸附麗,黷亂國經。較其罪惡,與侂胄相去無幾。乞追責遠竄,以為為臣不忠、朋邪誤國者之戒。



 又劾曦外姻郭倪、郭僎,竄嶺表,天下快之。



 繼兼侍講。方侂胄用事,箝天下之口,使不得議己,太府寺丞呂祖儉以謫死,布衣呂祖泰上書直言,中以危法,流之遠郡。居安奏請明其冤,以伸忠鯁之氣。又疏言:「古今之治本亂階,更為倚伏。以治易亂則反掌而可治,以亂治亂則亂去而復生。人主公聽則治,偏信則
 亂;政事歸外朝則治,歸內廷則亂;問百闢士大夫則治,問左右近習則亂;大臣公心無黨則治,植黨行私則亂;大臣正、小臣廉則治,大臣污、小臣貪則亂。如用人稍誤,是一侂胄死,一侂胄生也。」



 趙彥逾與樓鑰、林大中、章燮並召,居安言:「鑰與大中用,宗廟社稷之靈,天下蒼生之福,彥逾不可與之同日而語。彥逾始以趙汝愚不與同列政地,遂啟侂胄專政之謀,汝愚之斥死,彥逾之力居多,而彥逾者,汝愚之罪人也。陛下乃使與二人者同升,
 不幾於薰蕕同器、邪正並用乎?非所以示趨向於天下也。」疏已具,有微聞者,除目夜下,遷起居郎兼崇政殿說書。於是為諫官才十有八日。既供職,即直前奏日:「陛下特遷臣柱下史者,豈非欲使臣不得言耶?二史得直前奏事,祖宗法也。」遂極論之,又言:「臣為陛下耳目官,諫紙未乾,乃以迕權要徙他職,不得其言則去,臣不復留矣。」帝為改容。御史中丞雷孝友論其越職,奪一官,罷。太學諸生有舉幡乞留者。四明楊簡邂逅山陰道中,謂「此舉
 吾道增重」。江陵項安世致書曰:「左史,人中龍也。」



 逾年,復官,知太平州。當邊遽甫定,歲儉,汰去軍群聚寇攘,居安威惠流行,晏然若無事時。將副劉祐為怨家詣闕告密,置獄金陵,居安以書抵當路辯其冤,或謂「祐自誣服,得無嫌於黨逆乎?」居安曰:「郡有無辜死,奚以守為?」事果白。以直龍圖閣提點浙西刑獄。葛懌者,用戚屬恩補官,豪於貲,嘗憾父之嬖,既去而誣以盜,株連瘐死者數人,懌乃未嘗一造庭。居安一閱得實,立捕系論罪,械送他州。
 入對,帝曰:「卿有用之才也。」權工部侍朗,以集英殿修撰知隆興府。



 初,盜起郴黑風峒,羅世傳為之倡,勢張甚。湖南所在發兵扼要沖,義丁表裏應援,賊乏食,少懈,主兵者稍堅持之,則就禽矣。會江西帥欲以買降為功,遣人間道說賊,饋鹽與糧,賊喜,謀益逞。帥以病卒,繼者蹈其敝。賊陰治械,外送款,身受官峒中,不至公府。義丁皆恚曰:「作賊者得官,我輩捐軀壞產業,何所得!」於是五合六聚,各以峒名其鄉,李元勵、陳延佐之徒,並起為賊矣。放
 兵四劫,掀永新,撇龍泉,江西列城皆震。朝廷調江、鄂之兵屯衡、贛,而他兵駐龍泉者命吉守節制焉。吉守率師往,幾為賊困,池兵來援失利。朝廷憂之,遂以居安為帥。



 居安以書曉都統制許俊曰:「賊勝則民皆為賊,官軍勝則賊皆為民,勢之翕張,決於此舉。將軍素以勇名,挫於山賊可乎?」俊得書皇恐,不敢以他帥事居安,居安督戰於黃山,勝之,賊始懼,走韶州,為摧鋒軍所敗,勢日蹙。吉守前以戰不利,用招降之策,遣吏持受降圖來,書賊銜「
 江湖兩路大都統。」居安笑曰:「賊玩侮如此,猶為國有人乎?」白諸朝,吉守以祠去。遂命居安節制江、池大軍,駐廬陵督捕,領郡事。召土豪問便宜,皆言賊恃險陟降如猿猱,若鈔吾糧,吾事危矣。居安曰:「吾自有以破賊。」會元勵執練木橋賊首李才全至,居安厚待才全而賞元勵,眾皆感。羅世傳果疑元勵之貳已,遂交惡。元勵率眾攻世傳,居安語俊曰:「兩虎鬥於穴,吾可成卞莊子之功。」世傳嗾練木橋賊黨襲元勵,俘其孥,禽元勵以獻。時青草峒
 賊亦就禽,並磔於吉之南門。元勵既誅,世傳以功負恃益驕蹇,名效順而實自保。俊請班師,居安不許,俾因賊堡壁固守。居亡何,世傳果與兄世祿俱叛。居安奏乞朝廷毋憂,今落其角距,可一戰禽也。乃密為方略,遣官民兵合圍之,世傳自經死,斬其首以徇,群盜次第平。居安之在軍中也,賞厚罰明,將吏盡力,始終用以賊擊賊之策,故兵民無傷者。江西人祠而祝之,刻石紀功。徙鎮襄陽,以言者罷,閑居十有一年。



 嘉定十五年與魏了翁同
 召,遷工部侍郎。時方受寶,中朝皆動色相賀。入對,首言:「人主畏無難而不畏多難,輿地寶玉之歸,盍思當時之所以失。」言極切至。甫兩月,以集英殿修撰提舉玉隆宮。未幾,以寶謨閣待制知溫州,郡政大舉。



 理宗即位,以敷文閣待制知福州,升龍圖閣直學士,轉大中大夫,提舉崇福宮。將行,鹽寇起寧化,居安以書諭汀守曰:「土瘠民貧,業於鹽可盡禁耶?且彼執三首惡以自贖,宜治此三人,他可勿治。」部使者遣左翼軍將鄧起提兵往,起貪夜
 冒險與寇角以死,軍潰,民相驚逃去。事聞,命居安專任招捕。居安既留,募軍校劉華、丘銳者授以計畫,至汀而賊已至郡矣,州人大懼。賊知帥有撫納意,即引退。華、銳出入賊中,指期約降。有以右班攝汀守者,倔強好大言,以知兵自任,欲出不意為己功。賊知其謀,敗降約,而建、劍諸郡並江西嘯聚蜂起矣。居安議不合,嘆曰:「吾可復求焦頭爛額之功耶?」即拜疏歸。



 居安以書生,於兵事不學而能,必誅峒寇而降汀寇,皆非茍然者。卒,累贈少保。
 居安宅心公明,待物不貳。有《方巖集》行世。



 論曰:李宗勉在庶僚,論事平直,及入相,負公清之稱。袁甫學有本原,善達其用,持節所過,其民至今思之。劉黻分別邪正,侃侃敢言,亦難能者。王居安掃除群邪,以匡王國,其志壯哉!



\end{pinyinscope}