\article{志第九十五 樂十七}

\begin{pinyinscope}

 詩樂琴律燕樂教坊雲韶部鈞容直四夷樂



 詩樂虞庭言樂,以詩為本。孔門禮樂之教,自興於《詩》始。《記》曰:「十有三年學樂、誦詩。」詠歌以養其性情,舞蹈以養
 其血脈,此古之成材所以為易也。宋朝湖學之興,老師宿儒痛正音之寂寥,嘗擇取《二南》、《小雅》數十篇,寓之塤鑰,使學者朝夕詠歌。自爾聲詩之學,為儒者稍知所尚。張載嘗慨然思欲講明,作之朝廷,被諸郊廟矣。朱熹述為詩篇,匯於學禮,將使後之學者學焉。



 《小雅》歌凡六篇:



 朱熹曰:「《傳》曰:『大學始教,宵雅肄三。』謂習《小雅·鹿鳴》、《四牡》、《皇皇者華》之三詩也。此皆君臣宴勞之詩,始學者
 習之,所以取其上下相和厚也。古鄉飲酒及燕禮皆歌此三詩。及笙入,六笙間歌《魚麗》、《南有嘉魚》、《南山有臺》。六笙詩本無辭,其遺聲亦不復傳矣。《小雅》為諸侯之樂,《大雅》、《頌》為天子之樂。」



 二南《國風》歌凡六篇:



 朱熹曰:「『《周南》、《召南》,正始之道,王化之基。』『故用之鄉人焉,用之邦國焉。』《鄉飲酒》及《鄉射禮》:『合樂,《周南》:《關雎》、《葛覃》、《卷耳》;《召南》:《鵲巢》、《採蘩》、《採蘋》。』《燕禮》云:『遂歌鄉樂。』即此
 六篇也。合樂,謂歌舞與眾聲皆作。《周南》、《召南》,古房中之樂歌也。《關雎》言后妃之志,《鵲巢》言國君夫人之德,《採蘩》言夫人之不失職,《採蘋》言卿大夫妻能循法度。夫婦之道,生民之本,王化之端,此六篇者,其教之原也。故國君與其臣下及四方之賓燕,用之合樂也。」



 《小雅》詩譜:《鹿鳴》、《四牡》、《皇皇者華》、《魚麗》、《南有嘉魚》、《南山有臺》皆用黃鐘清宮。俗呼為正宮調。



 二南《國風》詩譜:《關雎》、《葛覃》、《卷耳》、《鵲巢》、《採蘩》、《採蘋》皆用無
 射清商。俗呼為越調。



 朱熹曰:「《大戴禮》言:《雅》二十六篇,其八可歌,其八廢不可歌,本文頗有闕誤。漢末杜夔傳舊雅樂四曲:一曰《鹿鳴》,二曰《騶虞》,三曰《伐檀》,又加《文王》詩,皆古聲辭。其後,新辭作而舊曲遂廢。唐開元鄉飲酒禮,乃有此十二篇之目,而其聲亦莫得聞。此譜,相傳即開元遺聲也。古聲亡滅已久,不知當時工師何所考而為此。竊疑古樂有唱、有嘆。唱者,發歌句也;和者,繼其聲也。詩
 詞之外,應更有疊字、散聲,以嘆發其趣。故漢、晉間舊曲既失其傳,則其詞雖存,而世莫能補。如此譜直以一聲協一字,則古詩篇篇可歌。又其以清聲為調,似亦非古法,然古聲既不可考,姑存此以見聲歌之徬佛,俟知樂者考焉。」



 琴律賾天地之和者莫如樂,暢樂之趣者莫如琴。八音以絲為君,絲以琴為君。眾器之中,琴德最優。《白虎通》曰:「琴者,禁止於邪,以正人心也。」宜眾樂皆為琴之臣妾。然
 八音之中,金、石、竹、匏、土、木六者,皆有一定之聲:革為燥濕所薄,絲有弦柱緩急不齊,故二者其聲難定。鼓無當於五聲,此不復論。惟絲聲備五聲,而其變無窮。五弦作於虞舜,七弦作於周文、武,此琴制之古者也。厥後增損不一。至宋始制二弦之琴,以象天地,謂之兩儀琴,每弦各六柱。又為十二弦以象十二律,其倍應之聲靡不畢備。太宗因大樂雅琴加為九弦,按曲轉入大樂十二律,清濁互相合應。大晟樂府嘗罷一、三、七、九。惟存五弦,謂
 其得五音之正,最優於諸琴也。今復俱用。太常琴制,其長三尺六寸,三百六十分,像周天之度也。



 姜夔《樂議》分琴為三準:自一暉至四暉謂之上準,四寸半,以象黃鐘之半律;自四暉至七暉謂之中準,中準九寸,以象黃鐘之正律;自七暉至龍齦謂之下準,下準一尺八寸,以象黃鐘之倍律。三準各具十二律聲,按弦附木而取。然須轉弦合本律所用之字,若不轉弦,則誤觸散聲,落別律矣。每一弦各具三十六聲,皆自然也。分五、七、九弦琴,各
 述轉弦合調圖:



 《五弦琴圖說》曰:「琴為古樂,所用者皆宮、商、角、征、羽正音,故以五弦散聲配之。其二變之聲,惟用古清商,謂之側弄,不入雅樂。」



 《七弦琴圖說》曰:七弦散而扣之,則間一弦於第十暉取應聲。假如宮調,五弦十暉應七弦散聲,四弦十暉應六弦散聲,大弦十暉應三弦散聲,惟三弦獨退一暉,於十一暉應五弦散聲,古今無知之者。竊謂黃鐘、大呂並用慢角調,故於大弦十一暉應三弦散聲;太簇、夾鐘並用清
 商調,故於二弦方十二暉應四弦散聲;姑洗、仲呂、蕤賓並用宮調,故於三弦十一暉應五弦散聲;林鐘、夷則並用慢宮調,故於四弦十一暉應六弦散聲;南呂、無射、應鐘並用蕤賓調,故於五弦十一暉應七弦散聲。以律長短配弦大小,各有其序。」



 《九弦琴圖說》曰:「弦有七、有九,實即五弦。七弦倍其二,九弦倍其四,所用者五音,亦不以二變為散聲也。或欲以七弦配五音二變,以餘兩弦為倍,若七弦分配七音,則是今之十四弦也。《聲律訣》云:『琴瑟
 齪四者,律法上下相生也。』若加二變,則於律法不諧矣。或曰:『如此則琴無二變之聲乎?』曰:『附木取之,二變之聲固在也。』合五、七、九弦琴,總述取應聲法,分十二律十二均,每聲取弦暉之應,皆以次列按。



 古者大琴則有大瑟,中琴則有中瑟,有雅琴、頌琴,則雅瑟、頌瑟,實為之合。夔乃定瑟之制:桐為背,梓為腹,長九尺九寸,首尾各九寸,隱間八尺一寸,廣尺有八寸,岳崇寸有八分。中施九梁,皆象黃鐘之數。梁下相連,使其聲沖融;首尾之下為兩
 穴,使其聲條達,是《傳》所謂「大瑟達越」也。四隅刻云以緣其武,像其出於雲和。漆其壁與首、尾、腹,取椅、桐、梓漆之。全設二十五弦,弦一柱,崇二寸七分。別以五色,五五相次,蒼為上,朱次之,黃次之,素與黔又次之,使肄習者便於擇弦。弦八十一絲而朱之,是謂朱弦。其尺則用漢尺。凡瑟弦具五聲,五聲為均,凡五均,其二變之聲,則柱後抑角、羽而取之,五均凡三十五聲。十二律、六十均、四百二十聲,瑟之能事畢矣。夔於琴、瑟之議,其詳如此。



 朱熹
 嘗與學者共講琴法,其定律之法:十二律並用太史公九分寸法為準,損益相生,分十二律及五聲,位置各定。按古人以吹管聲傅於琴上,如吹管起黃鐘,則以琴之黃鐘聲合之;聲合無差,然後以次遍合諸聲,則五聲皆正。唐人紀琴,先以管色合字定宮弦,乃以宮弦下生徵,徵上生商,上下相生,終於少商。下生者隔二弦、上生者隔一弦取之。凡絲聲皆當如此。今人茍簡,不復以管定聲,其高下出於臨時,非古法也。



 調弦之法:散聲隔四而
 得二聲;中暉亦如之而得四聲;八暉隔三而得六聲;九暉按上者隔二而得四聲,按下者隔一而得五聲;十暉按上者隔一而得五聲,按下者隔二而得四聲。每疑七弦隔一調之,六弦皆應於第十暉,而第三弦獨於第十一暉調之乃應。及思而得之,七弦散聲為五聲之正,而大弦十二律之位,又眾弦散聲之所取正也。故逐弦之五聲皆自東而西,相為次第。其六弦會於十暉,則一與三者,角與散角應也;二與四者,徵與散徵應也;四與六
 者,宮與散少宮應也;五與七者,商與散少商應也;其第三、第五弦會於十一暉,則羽與散羽應也。義各有當,初不相須,故不同會於一暉也。



 旋宮諸調之法:旋宮古有「隨月用律」之說,今乃謂不必轉軫促弦,但依旋宮之法而抑按之,恐難如此泛論。當每宮指定,各以何聲取何弦為唱,各以何弦取何律為均,乃見詳實。又以《禮運正義》推之,則每律各為一宮,每宮各有五調,而其每調用律取聲,亦各有法。此為琴之綱領,而說者罕及,乃闕典
 也。當為一圖,以宮統調,以調統聲,令其次第、賓主各有條理。仍先作三圖:一、各具琴之形體、暉弦、尺寸、散聲之位;二、附按聲聲律之位;三、附泛聲聲律之位,列於宮調圖前,則覽者曉然,可為萬世法矣。



 觀熹之言,其於琴法本融末粲,至疏達而至縝密,蓋所謂識其大者歟!



 燕樂古者,燕樂自周以來用之。唐貞觀增隋九部為十部,以張文收所制歌名燕樂,而被之管弦。厥後至坐部伎琵琶曲,盛流於時,匪直漢氏上林樂府、縵樂不應經
 法而已。宋初置教坊,得江南樂,已汰其坐部不用。自後因舊曲創新聲,轉加流麗。政和間,詔以大晟雅樂施於燕饗,御殿按試,補徵、角二調,播之教坊,頒之天下。然當時樂府奏言:樂之諸宮調多不正,皆俚俗所傳。及命劉昺輯《燕樂新書》,亦惟以八十四調為宗,非復雅音,而曲燕暱狎,至有援「君臣相說之樂」以借口者。末俗漸靡之弊,愈不容言矣。紹興中,始蠲省教坊樂,凡燕禮,屏坐伎。乾道繼志述事,間用雜攢以充教坊之號,取具臨時,而
 廷紳祝頌,務在嚴恭,亦明以更不用女樂,頒示子孫守之,以為家法。於是中興燕樂,比前代猶簡,而有關乎君德者良多。



 蔡元定嘗為《燕樂》一書,證俗失以存古義,今採其略附於下:



 黃鐘用「合」字,大呂、太簇用「四」字,夾鐘、姑洗用「一」字,夷則、南呂用「工」字,無射、應鐘用「凡」字,各以上、下分為清濁。其中呂、蕤賓、林鐘不可以上、下分,中呂用「上」字,蕤賓用「勾」字,林鐘用「尺」字。其黃鐘清用「六」字,大呂、太簇、夾鐘清各用「五」字,而以下、上、緊別之。緊「五」者,夾鐘
 清聲,俗樂以為宮。此其取律寸、律數、用字紀聲之略也。



 一宮、二商、三角、四變為宮,五徵、六羽、七閏為角。五聲之號與雅樂同,惟變徵以於十二律中陰陽易位,故謂之變。變宮以七聲所不及,取閏餘之義,故謂之閏。四變居宮聲之對,故為宮。俗樂以閏為正聲,以閏加變,故閏為角而實非正角。此其七聲高下之略也。



 聲由陽來,陽生於子、終於午。燕樂以夾鐘收四聲:曰宮、曰商、曰羽、曰閏。閏為角,其正角聲、變聲、徵聲皆不收,而獨用夾鐘為律
 本。此其夾鐘收四聲之略也。



 宮聲七調:曰正宮、曰高宮、曰中呂宮、曰道宮、曰南呂宮、曰仙呂宮、曰黃鐘宮,皆生於黃鐘。商聲七調:曰大食調、曰高大食調、曰雙調、曰小食調、曰歇指調、曰商調、曰越調,皆生於太簇。羽聲七調:曰般涉調、曰高般涉調、曰中呂調、曰正平調、曰南呂調、曰仙呂調、曰黃鐘調,皆生於南呂。角聲七調:曰大食角、曰高大食角、曰雙角、曰小食角、曰歇指角、曰商角、曰越角、皆生於應鐘。此其四聲二十八調之略也。



 竊考元定
 言燕樂大要,其律本出夾鐘,以十二律兼四清為十六聲,而夾鐘為最清,此所謂靡靡之聲也。觀其律本,則其樂可知。變宮、變徵既非正聲,而以變徵為宮,以變宮為角,反紊亂正聲。若此夾鐘宮謂之中呂宮、林鐘宮謂之南呂宮者,燕樂聲高,實以夾鐘為黃鐘也。所收二十八調,本萬寶常所謂非治世之音,俗又於七角調各加一聲,流蕩忘反,而祖調亦不復存矣。聲之感人,如風偃草,宜風俗之日衰也!夫奸聲亂色,不留聰明;淫樂慝禮,不
 接心術。使心知百體,皆由順正以行其義,此正古君子所以為治天下之本也。紹興、乾道教坊迄弛不復置雲。



 教坊自唐武德以來,置署在禁門內。開元後,其人浸多,凡祭祀、大朝會則用太常雅樂,歲時宴享則用教坊諸部樂。前代有宴樂、清樂、散樂,本隸太常,後稍歸教坊,有立、坐二部。宋初循舊制,置教坊,凡四部。其後平荊南,得樂工三十二人;平西川,得一百三十九人;平江南,得十六人;平太原,得十九人;餘藩臣所貢者八十三人;又太
 宗藩邸有七十一人。由是,四方執藝之精者皆在籍中。



 每春秋聖節三大宴:其第一、皇帝升坐,宰相進酒,庭中吹觱慄,以眾樂和之;賜群臣酒,皆就坐,宰相飲,作《傾杯樂》;百官飲,作《三臺》。第二、皇帝再舉酒,群臣立於席後,樂以歌起。第三、皇帝舉酒,如第二之制,以次進食。第四、百戲皆作。第五、皇帝舉酒,如第二之制。第六、樂工致辭,繼以詩一章,謂之「口號」,皆述德美及中外蹈詠之情。初致辭,群臣皆起,聽辭畢,再拜。第七、合奏大曲。第八、皇帝舉
 酒,殿上獨彈琵琶。第九、小兒隊舞,亦致辭以述德美。第十、雜劇。罷,皇帝起更衣。第十一、皇帝再坐,舉酒,殿上獨吹笙。第十二、蹴□匊。第十三、皇帝舉酒,殿上獨彈箏。第十四、女弟子隊舞,亦致辭如小兒隊。第十五、雜劇。第十六、皇帝舉酒,如第二之制。第十七、奏鼓吹曲,或用法曲,或用《龜茲》。第十八、皇帝舉酒,如第二之制,食罷。第十九、用角抵,宴畢。



 其御樓賜酺同大宴。崇德殿宴契丹使,惟無後場雜劇及女弟子舞隊。每上元觀燈,樓前設露臺,
 臺上奏教坊樂、舞小兒隊。臺南設燈山,燈山前陳百戲,山棚上用散樂、女弟子舞。餘曲宴會、賞花、習射、觀稼,凡游幸但奏樂行酒,惟慶節上壽及將相入辭賜酒,則止奏樂。



 都知、色長二人攝太官令,升殿對立,逡巡周,大宴則酒、唱遍,曲宴宰相雖各舉酒,通用慢曲而舞《三臺》。



 所奏凡十八調、四十大曲:一曰正宮調,其曲三,曰《梁州》、《瀛府》、《齊天樂》;二曰中呂宮,其曲二,曰《萬年歡》、《劍器》;三曰道調宮,其曲三,曰《梁州》、《薄媚》、《大聖樂》;四曰南呂宮,其曲二,曰《瀛府》、《薄媚》;五曰仙呂宮,其曲三,曰《梁州》、《保金枝》、《延
 壽樂》;六曰黃鐘宮,其曲三,曰《梁州》、《中和樂》、《劍器》;七曰越調,其曲二,曰《伊州》、《石州》;八曰大石調,其曲二,曰《清平樂》、《大明樂》;九曰雙調,其曲三,曰《降聖樂》、《新水調》、《採蓮》;十曰小石調,其曲二,曰《胡渭州》、《嘉慶樂》;十一曰歇指調,其曲三,曰《伊州》、《君臣相遇樂》、《慶雲樂》;十二曰林鐘商,其曲三,曰《賀皇恩》、《泛清波》、《胡渭州》;十三曰中呂調,其曲二,曰《綠腰》、《道人歡》;十四曰南呂調,其曲二,曰《綠腰》《罷金鉦》;十五曰仙呂調,其曲二,曰《綠腰》、《採雲歸》;十六曰黃鐘羽,其曲
 一,曰《千春樂》;十七曰般涉調,其曲二,曰《長壽仙》、《滿宮春》;十八曰正平調,無大曲,小曲無定數。不用者有十調:一曰高宮,二曰高大石,三曰高般涉,四曰越角,五曰大石角,六曰高大石角,七曰雙角,八曰小石角,九曰歇指角,十曰林鐘角。樂用琵琶、箜篌、五弦琴、箏、笙、觱慄、笛、方響、羯鼓、杖鼓、拍板。



 法曲部,其曲二,一曰道調宮《望瀛》,二曰小石調《獻仙音》。樂用琵琶、箜篌、五弦、箏、笙、觱慄、方響、拍板。龜茲部,其曲二,皆雙調,一曰《宇宙清》,二曰《感皇恩》。樂用
 觱慄、笛、羯鼓、腰鼓、揩鼓、雞樓鼓、□鼓、拍板。鼓笛部,樂用三色笛、杖鼓、拍板。



 隊舞之制,其名各十。小兒隊凡七十二人:一曰柘枝隊,衣五色繡羅寬袍,戴胡帽,系銀帶;二曰劍器隊,衣五色繡羅襦,裹交腳帕頭,紅羅繡抹額,帶器仗;三曰婆羅門隊,紫羅僧衣,緋掛子,執錫鐶拄杖;四曰醉胡騰隊,衣紅錦襦,系銀□舌鞢,戴氈帽;五曰諢臣萬歲樂隊,衣紫緋綠羅寬衫,諢裹簇花帕頭;六曰兒童感聖樂隊,衣青羅生
 色衫,系勒帛,總兩角;七曰玉兔渾脫隊,四色繡羅襦,系銀帶,冠玉兔冠;八曰異域朝天隊,衣錦襖,系銀束帶,冠夷冠,執寶盤;九曰兒童解紅隊,衣紫緋繡襦,系銀帶,冠花砌鳳冠,綬帶;十曰射鵰回鶻隊,衣盤雕錦襦,系銀□舌鞢,射鵰盤。



 女弟子隊凡一百五十三人:一曰菩薩蠻隊,衣緋生色窄砌衣,冠卷雲冠;二曰感化樂隊,衣青羅生色通衣,背梳髻,系綬帶;三曰拋球樂隊,衣四色繡羅寬衫,系銀帶,奉繡球;四曰佳人剪牲丹隊,衣紅生色砌衣,
 戴金冠,剪牲丹花;五曰拂霓裳隊,衣紅仙砌衣,碧霞帔,戴仙冠,紅繡抹額;六曰採蓮隊,衣紅羅生色綽子,系暈裙,戴雲鬟髻,乘彩船,執蓮花;七曰鳳迎樂隊,衣紅仙砌衣,戴雲鬟鳳髻;八曰菩薩獻香花隊,衣生色窄砌衣,戴寶冠,執香花盤;九曰彩雲仙隊,衣黃生色道衣,紫霞帔,冠仙冠,執旌節、鶴扇;十曰打球樂隊,衣四色窄繡羅襦,系銀帶,裹順風腳簇花帕頭,執球杖。大抵若此,而復從宜變易。



 百戲有蹴球、踏蹻、藏擫、雜旋、獅子、弄槍、鈴瓶、茶
 碗、氈齪、碎劍、踏索、上竿、觔斗、擎戴、拗腰、透劍門、打彈丸之類。



 錫慶院宴會,諸王賜食及宰相筵設時賜樂者,第四部充。



 建隆中,教坊都知李德升作《長春樂曲》;乾德元年,又作《萬歲升平樂曲》。明年,教坊高班都知郭延美又作《紫雲長壽樂》鼓吹曲,以奏御焉。太宗洞曉音律,前後親制大小曲及因舊曲創新聲者,總三百九十。凡制大曲十八:



 正宮《平戎破陣樂》,南呂宮《平晉普天樂》,中呂宮《大宋朝歡樂》,黃鐘宮《宇宙荷皇恩》,道調宮《垂衣定八方》,仙呂宮《甘露降龍庭》,小石調《
 金枝玉葉春》,林鐘商《大惠帝恩寬》,歇指調《大定寰中樂》,雙調《惠化樂堯風》,越調《萬國朝天樂》,大石調《嘉禾生九穗》,南呂調《文興禮樂歡》,仙呂調《齊天長壽樂》,般涉調《君臣宴會樂》,中呂調《一斛夜明珠》,黃鐘羽《降聖萬年春》,平調《金觴祝壽春》。



 曲破二十九:



 正宮《宴鈞臺》,南呂宮《七盤樂》,仙呂宮《王母桃》,高宮《靜三邊》,黃鐘宮《採蓮回》,中呂宮《杏園春》、《獻玉杯》,道調宮《折枝花》,林鐘商《宴朝簪》,歇指調《九穗禾》,高大石調《轉春鶯》,小石調《舞霓裳》,越調《九霞
 觴》,雙調《朝八蠻》,大石調《清夜游》,林鐘角《慶雲見》,越角《露如珠》,小石角《龍池柳》,高角《陽臺雲》,歇指角《金步搖》,大石角《念邊功》,雙角《宴新春》,南呂調《鳳城春》,仙呂調《夢鈞天》,中呂調《採明珠》,平調《萬年枝》,黃鐘羽《賀回鸞》,般涉調《鬱金香》,高般涉調《會天仙》。



 琵琶獨彈曲破十五:



 鳳鸞商《慶成功》,應鐘調《九曲清》,金石角《鳳來儀》,芙蓉調《蕊宮春》,蕤賓調《連理枝》,正仙呂調《朝天樂》,蘭陵角《奉宸歡》,孤雁調《賀昌時》,大石調《寰海清》,玉仙商《玉芙蓉》,林鐘角《泛仙槎》,
 無射宮調《帝臺春》,龍仙羽《宴蓬萊》,聖德商《美時清》,仙呂調《壽星見》。



 小曲二百七十:



 正宮十:《一陽生》、《玉窗寒》、《念邊戍》、《玉如意》、《瓊樹枝》、《鷫□裘》、《塞鴻飛》、《漏丁丁》、《息鼙鼓》、《勸流霞》。



 南呂宮十一:《仙盤露》、《冰盤果》、《芙蓉園》、《林下風》、《風雨調》、《開月幌》、《鳳來賓》、《落梁塵》、《望陽臺》、《慶年豐》、《青驄馬》。



 中呂宮十三:《上林春》、《春波綠》、《百樹花》、《壽無疆》、《萬年春》、《擊珊瑚》、《柳垂絲》、《醉紅樓》、《折紅杏》、《一園花》、《花下醉》、《游春歸》、《千樹柳》。



 仙呂宮九:《折紅蕖》、《鵲度河》、《紫蘭香》、《喜堯時》、《猗蘭殿》、《步瑤階》、《
 千秋樂》、《百和香》、《佩珊珊》。



 黃鐘宮十二:《菊花杯》、《翠幕新》、《四塞清》、《滿簾霜》、《畫屏風》、《折茱萸》、《望春云》、《苑中鶴》、《賜征袍》、《望回戈》、《稻稼成》、《泛金英》。



 高宮九:《嘉順成》、《安邊塞》、《獵騎還》、《游兔園》、《錦步帳》、《博山爐》、《暖寒杯》、《雲紛紜》、《待春來》。



 道調宮九:《會夔龍》、《泛仙杯》、《披風襟》、《孔雀扇》、《百尺樓》、《金尊滿》、《奏明庭》、《拾落花》、《聲聲好》。



 越調八:《翡翠帷》、《玉照臺》、《香旖旎》、《紅樓夜》、《珠頂鶴》、《得賢臣》、《蘭堂燭》、《金鏑流》。



 雙調十六:《宴瓊林》、《泛龍舟》、《汀洲綠》、《登高樓》、《麥隴雉》、《柳如煙》、《楊花飛》、《玉澤新》、《玳瑁
 簪》、《玉階曉》、《喜清和》、《人歡樂》、《征戍回》、《一院香》、《一片雲》、《千萬年》。



 小石調七:《滿庭香》、《七寶冠》、《玉唾盂》、《闢塵犀》、《喜新晴》、《慶雲飛》、《太平時》。



 林鐘商十:《手採秋蘭》、《紫絲囊》、《留征騎》、《塞鴻度》、《回鶻朝》、《汀洲雁》、《風入松》、《蓼花紅》、《曳珠佩》、《遵渚鴻》。



 歇指調九:《榆塞清》、《聽秋風》、《紫玉簫》、《碧池魚》、《鶴盤旋》、《湛恩新》、《聽秋蟬》、《月中歸》、《千家月》。



 高大石調九:《花下宴》、《甘雨足》、《畫秋千》、《夾竹桃》、《攀露桃》、《燕初來》、《踏青回》、《拋繡球》、《潑火雨》。



 大石調八:《賀元正》、《待花開》、《手採紅蓮》、《出穀鶯》、《游月宮》、《望回車》、《塞雲
 平》、《秉燭游》。



 小石角九:《月宮春》、《折仙枝》、《春日遲》、《綺筵春》、《登春臺》、《紫桃花》、《一林紅》、《喜春雨》、《泛春池》。



 雙角九:《鳳樓燈》、《九門開》、《落梅香》、《春冰拆》、《萬年安》、《催花發》、《降真香》、《迎新春》、《望蓬島》。



 高角九:《日南至》、《帝道昌》、《文風盛》、《琥珀杯》、《雪花飛》、《皂貂裘》、《征馬嘶》、《射飛雁》、《雪飄颻》。



 大石角九:《紅爐火》、《翠雲裘》、《慶成功》、《冬夜長》、《金鸚鵡》、《玉樓寒》、《鳳戲雛》、《一爐香》、《雲中雁》。



 歇指角九:《玉壺冰》、《卷珠箔》、《隨風簾》、《樹青蔥》、《紫桂叢》、《五色云》、《玉樓宴》、《蘭堂宴》、《千秋歲》。



 越角九:《望明堂》、《華池露》、《貯香
 囊》、《秋氣清》、《照秋池》、《曉風度》、《靖邊塵》、《聞新雁》、《吟風蟬》。



 林鐘角九:《慶時康》、《上林果》、《畫簾垂》、《水精簟》、《夏木繁》、《暑氣清》、《風中琴》、《轉輕車》、《清風來》。



 仙呂調十五:《喜清和》、《芰荷新》、《清世歡》、《玉鉤欄》、《金步搖》、《金錯落》、《燕引雛》、《草芊芊》、《步玉砌》、《整華裾》、《海山青》、《旋絮綿》、《風中帆》、《青絲騎》、《喜聞聲》。



 南呂調七:《春景麗》、《牡丹開》、《展芳茵》、《紅桃露》、《囀林鶯》、《滿林花》、《風飛花》。



 中呂調九:《宴嘉賓》、《會群仙》、《集百祥》、《憑朱欄》、《香煙細》、《仙洞開》、《上馬杯》、《拂長袂》、《羽觴飛》。



 高般涉調九:《喜秋成》、《戲馬臺》、《泛秋
 菊》、《芝殿樂》、《鸂鶒杯》、《玉芙蓉》、《偃干戈》、《聽秋砧》、《秋雲飛》。



 般涉調十:《玉樹花》、《望星斗》、《金錢花》、《玉窗深》、《萬民康》、《瑤林風》、《隨陽雁》、《倒金罍》、《雁來賓》、《看秋月》。



 黃鐘羽七:《宴鄒枚》、《雲中樹》、《燎金爐》、《澗底松》、《嶺頭梅》、《玉爐香》、《瑞雪飛》。



 平調十:《萬國朝》、《獻春盤》、《魚上冰》、《紅梅花》、《洞中春》、《春雪飛》、《翻羅袖》、《落梅花》、《夜游樂》、《鬥春雞》。



 因舊曲造新聲者五十八:



 正宮、南呂宮、道調宮、越調、南呂調,並《傾杯樂》、《三臺》;仙呂宮、高宮、小石調、大石調、高大石調、小石角、雙角、高角、大石角、歇指角、
 林鐘角、越角、高般涉調、黃鐘羽、平調,並《傾杯樂》、《劍器》、《感皇化》、《三臺》;黃鐘宮《傾杯樂》、《朝中措》、《三臺》;雙調《傾杯樂》、《攤破拋球樂》、《醉花間》、《小重山》、《三臺》;林鐘商《傾杯樂》、《洞中仙》、《望行宮》、《三臺》;歇指調《傾杯樂》、《洞仙歌》、《三臺》;仙呂調《傾杯樂》、《月宮仙》、《戴仙花》、《三臺》;中呂調《傾杯樂》、《菩薩蠻》、《瑞鷓鴣》、《三臺》;般涉調《傾杯樂》、《望征人》、《嘉宴樂》、《引駕回》、《拜新月》、《三臺》。



 若《宇宙賀皇恩》、《降聖萬年春》之類,皆藩邸作,以述太祖美德,諸曲多秘。而《平晉普天樂》者,平河東回
 所制,《萬國朝天樂》者,又明年所制,每宴享常用之。然帝勤求治道,未嘗自逸,故舉樂有度。雍熙初,教坊使郭守中求外任,止賜束帛。



 真宗不喜鄭聲,而或為雜詞,未嘗宣布於外。太平興國中,伶官蔚茂多侍大宴,聞雞唱,殿前都虞候崔翰問之曰:「此可被管弦乎?」茂多即法其聲,制曲曰《雞叫子》。又民間作新聲者甚眾,而教坊不用也。太宗所制曲,乾興以來通用之,凡新奏十七調,總四十八曲:黃鐘、道調、仙呂、中呂、南呂、正宮、小石、歇指、高平、
 般涉、大石、中呂、仙呂、雙越調,黃鐘羽。其急慢諸曲幾千數。又法曲、《龜茲》、鼓笛三部,凡二十有四曲。



 仁宗洞曉音律,每禁中度曲,以賜教坊,或命教坊使撰進,凡五十四曲,朝廷多用之,天聖中,帝嘗問輔臣以古今樂之異同,王曾對曰:「古樂祀天地、宗廟、社稷、山川、鬼神,而聽者莫不和悅。今樂則不然,徒虞人耳目而蕩人心志。自昔人君流連荒亡者,莫不由此。」帝曰:「朕於聲技固未嘗留意,內外宴游皆勉強耳。」張知白曰:「陛下盛德,外人豈知之,
 願備書時政記。」



 世號太常為雅樂,而未嘗施於宴享,豈以正聲為不美聽哉!夫樂者,樂也,其道雖微妙難知,至於奏之而使人悅豫和平,則不待知音而後能也。今太常樂縣鐘、磬、塤、篪、搏拊之器,與夫舞綴羽、鑰、乾、戚之制,類皆仿古,逮振作之,則聽者不知為樂而觀者厭焉,古樂豈真若此哉!孔子曰「惡鄭聲」,恐其亂雅。亂之云者,似是而非也。孟子亦曰「今樂猶古樂」,而太常乃與教坊殊絕,何哉?昔李照、胡瑗、阮逸改鑄鐘磬,處士徐復笑之曰:「
 聖人寓器以聲,不先求其聲而更其器,其可用乎!」照、瑗、逸制作久之,卒無所成。蜀人房庶亦深訂其非是,因著書論古樂與今樂本末不遠,其大略以謂:「上古世質,器與聲樸,後世稍變焉。金石,鐘磬也,後世易之為方響;絲竹,琴簫也,後世變之為箏笛。匏,笙也,攢之以斗;塤,土也,變而為甌;革,麻料也,擊而為鼓;木,柷吾文也,貫之為板。此八音者,於世甚便,而不達者指廟樂鎛鐘、鎛磬、宮軒為正聲,而概謂夷部、鹵部為淫聲。殊不知大輅起於椎輪,
 龍艘生於落葉,其變則然也。古者食以俎豆,後世易以杯盂;簟席以為安,後世更以榻桉。使聖人復生,不能舍杯盂、榻桉,而復俎豆、簟席之質也。八音之器,豈異此哉!孔子曰『鄭聲淫』者,豈以其器不若古哉!亦疾其聲之變爾。試在樂者,由今之器,寄古之聲,去惉懘靡曼而歸之中和雅正,則感人心、導和氣,不曰治世之音乎!然則世所謂雅樂者,未必如古,而教坊所奏,豈盡為淫聲哉!」當數子紛紛銳意改制之後,庶之論指意獨如此,故存
 其語,以俟知者。



 教坊本隸宣徽院,有使、副使、判官、都色長、色長、高班、大小都知。天聖五年,以內侍二人為鈐轄。嘉祐中,詔樂工每色額止二人,教頭止三人,有闕即填。異時或傳詔增置,許有司論奏。使、副歲閱雜劇,把色人分三等,遇三殿應奉人闕,即以次補。諸部應奉及二十年、年五十已上,許補廟令或鎮將,官制行,以隸太常寺。同天節,寶慈、慶壽宮生辰,皇子、公主生,凡國之慶事,皆進歌樂詞。



 熙寧九年,教坊副使花日新言:「樂聲高,歌者
 難繼。方響部器不中度,絲竹從之。宜去廝殺之急,歸嘽緩之易,請下一律,改造方向,以為樂準。絲竹悉從其聲,則音律諧協,以導中和之氣。」詔從之。十一月,奏新樂於化成殿,帝諭近臣曰:「樂聲第降一律,已得寬和之節矣。」增賜方響為架三十,命太常下法駕、鹵部樂一律,如教坊雲。初,熙寧二年五月,罷宗室正任以上借教坊樂人,至八年,復之,許教樂。



 政和三年五月,詔:「比以《大晟樂》播之教坊,嘉與天下共之,可以所進樂頒之天下。」八月,尚
 書省言:「大晟府宴樂已撥歸教坊,所有諸府從來習學之人,元降指揮令就大晟府教習,今當並就教坊習學。」從之。四年正月,禮部奏:「教坊樂,春或用商聲,孟或用季律,甚失四時之序。乞以大晟府十二月所定聲律,令教坊閱習,仍令秘書省撰詞。」



 高宗建炎初,省教坊。紹興十四年復置,凡樂工四百六十人,以內侍充鈐轄。紹興末復省。孝宗隆興二年天申節,將用樂上壽,上曰:「一歲之間,只兩宮誕日外,餘無所用,不知作可名色。」大臣皆言:「
 臨時點集,不必置教坊。」上曰:「善。」乾道後,北使每歲兩至,亦用樂,但呼市人使之,不置教坊,止令修內司先兩旬教習。舊例用樂人三百人,百戲軍百人,百禽鳴二人,小兒隊七十一人,女童隊百三十七人,築球軍三十二人,起立門行人三十二人,旗鼓四十人,以上並臨安府差。



 相撲等子二十一人。御前忠佐司差。



 命罷小兒及女童隊,餘用之。



 雲韶部者,黃門樂也。開寶中平嶺表,擇廣州內臣之聰警者,得八十人,令於教坊習樂藝,賜名簫韶部。雍熙初,
 改曰雲韶。每上元觀燈,上巳、端午觀水嬉,皆命作樂於宮中。遇南至、元正、清明、春秋分社之節,親王內中宴射,則亦用之。奏大曲十三:一曰中呂宮《萬年歡》;二曰黃鐘宮《中和樂》;三曰南呂宮《普天獻壽》,此曲亦太宗所制;四曰正宮《梁州》;五曰林鐘商《泛清波》;六曰雙調《大定樂》;七曰小石調《喜新春》;八曰越調《胡渭州》;九曰大石調《清平樂》;十曰般涉調《長壽仙》;十一曰高平調《罷金鉦》;十二曰中呂調《綠腰》;十三曰仙呂調《採雲歸》。樂用琵琶、箏、笙、觱
 慄、笛、方響、杖鼓、羯鼓、大鼓、拍板。雜劇用傀儡,後不復補。



 鈞容直,亦軍樂也。太平興國三年,詔籍軍中之善樂者,命曰引龍直。每巡省游幸,則騎導車駕而奏樂;若御樓觀燈、賜酺,則載第一山車。端拱二年,又選捧日、天武、拱聖軍曉暢音律者,增多其數,以中使監視,藩臣以樂工上貢者亦隸之。淳化四年,改名鈞容直,取鈞天之義。初用樂工,同雲韶部。大中祥符五年,因鼓工溫用之請,增《龜茲》部,如教坊。其奉天書及四宮觀皆用之。有指揮使
 一人、都知二人、副都知二人、押班三人、應奉文字一人、監領內侍二人。嘉祐元年,系籍三百八十三人。六年,增置四百三十四人,詔以為額,闕即補之。七年,詔隸班及二十四年、年五十以上者,聽補軍職,隸軍頭司。其樂舊奏十六調,凡三十六大曲,鼓笛二十一曲,並他曲甚眾。嘉祐二年,監領內侍言,鈞容直與教坊樂並奏,聲不諧。詔罷鈞容舊十六調,取教坊十七調肄習之,雖間有損益,然其大曲、曲破並急、慢諸曲,與教坊頗同矣。



 紹興中,
 鈞容直舊管四百人,楊存中請復收補,權以舊管之半為額,尋聞其召募騷擾,降詔止之。及其以應奉有勞,進呈推賞,又申諭止於支賜一次,庶杜其日後希望。紹興三十年,復詔鈞容班可蠲省,令殿司比擬一等班直收頓,內老弱癃疾者放停。教坊所嘗援祖宗舊典,點選入教,雖暫從其請,紹興三十一年有詔,教坊即日蠲罷,各令自便。



 東西班樂,亦太平興國中選東西班習樂者,樂器獨用
 銀字觱慄、小笛、小笙。每騎從車駕而奏樂,或巡方則夜奏於行宮殿庭。



 諸軍皆有善樂者,每車駕親祀回,則衣緋綠衣,自青城至朱雀門,列於御道之左右,奏樂迎奉,其聲相屬,聞十數里。或軍宴設亦奏之。



 棹刀槍牌翻歌等,不常置。



 清衙軍習樂者,令鈞容直教之,內侍主其事,園苑賜會及館待契丹使人。



 又有親從親事樂及開封府衙前樂,園苑又分用諸軍樂,諸州皆有衙前樂。



 四夷樂者,元豐六年五月,召見米脂砦所降戎樂四十二人,奏樂於崇政殿,以三班借職王恩等六人差監在京閑慢庫務門及舊城門敢勇三十六,與茶酒新任殿侍。《大晟樂書》曰:「前此宮架之外,列熊羆案,所奏皆夷樂也,豈容淆雜大樂!乃奏罷之。然古鞮鞻氏掌四夷樂,靺師、旄人各有所掌,以承祭祀,以供宴享。蓋中天下而
 立,得四海之歡心,使鼓舞焉,先王之所不廢也。《漢律》曰:『每大朝會宜設於殿門之外。』天子御樓,則宮架之外列於道側,豈可旋於廣庭,與大樂並奏哉!」



\end{pinyinscope}