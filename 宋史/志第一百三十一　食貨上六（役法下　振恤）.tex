\article{志第一百三十一 食貨上六(役法下 振恤)}

\begin{pinyinscope}

 役
 法



 中書舍人蘇軾在詳定役法所,極言役法可雇不可差,第不當於雇役實費之外,多取民錢,若量入為出,不至多取,則自足以利民。司馬光不然之,光言:「差役已
 行,續聞有命:雇募不足,方許定差。屢有更張,號令不一。又轉運使欲合一路共為一法,不令州縣各從其宜,或已受差卻釋役使去,或已辭雇卻復拘之入役,或仍舊用錢招雇,或不用錢白招,紛紜不定,浸違本意。」遂條舉始奏之文,嘗許州縣、監司陳列宜否。「自今外官茍見利否,縣許直上轉運司,州許直奏,使下情無壅。詳定所第當稽閱監司、州縣所陳,詳定可否;非其任職而務出奇論、不切事情者勿用,亦不可以一路、一州、一縣土風利
 害概行天下。」從之。



 未幾,詔:「諸路坊郭五等以上,及單丁、女戶、官戶、寺觀第三等以上,舊輸免役錢者並減五分,餘戶等下此者悉免輸,仍自元祐二年始。凡支酬衙前重難及綱運公皂迓送飧錢,用坊場、河渡錢給賦。不足,方得於此六色錢助用;而有餘,封樁以備不時之須。」



 臣僚上言:「朝廷雖立差法,而明許民戶雇代,州縣多已施行。近命弓手須正身,恐公私未便。」詔:「不願身自任役,許募嘗為弓手而有勞效者,雇直雖多,毋逾元募之數。」御
 史中丞劉摯言:「弓手不可不用差法者,蓋鄉人在役,則不獨有家丁子弟之助,至於族姻鄉黨,莫不與為耳目,有捕輒獲;又土著自重,無逃亡之患。自行雇募,盜寇充斥,蓋浮惰不能任責故也。如五路弓手,熙寧未變法前,身自執役,最號強勁,其材藝捕緝勝於他路。近日復差,不聞有不樂而願出錢雇人。惟是川蜀、江、浙等路,昨升差上一等戶,皆習於驕脆,不肯任察捕之責。欲乞五路必差正身,餘路即用新敕,厘為三色:舊有戶等已嘗受
 差者,曾有戰鬥勞效應留者,願雇人代己者。立此三色,所冀新舊相兼,漸習御捕。」侍御史王巖叟亦言雇代恐不能任事,略與摯同。



 監察御史上官均言:「役之最重,莫如衙前,其次弓手。今東南長名衙前招募既足,所差不及上戶,上戶必差弓手,則是以上戶就中戶之役,實為優幸。上戶產厚而役輕,下戶產薄而無役,然則所當補恤,正在中戶。今若增上戶役年,使中戶番休稍久,則補除相均矣。」又言:「近許當差弓手戶役得差人為代,此法
 最便。議者謂『身任其役,則自愛而重犯法』,熙寧募法久行,何嘗聞盜賊充斥?彼自愛之民,承符帖追逮則可,俾之與賊角死,豈其能哉?兩浙諸路以法案差弓手,必責正身,至有涕泣辭免者。此豈可恃以為用哉?今既立法許雇嘗為弓手而有勞效之人,比之泛募,宜有間矣。」



 殿中侍御史呂陶謁告歸成都,因令與轉運司議定役法。後議立增減役年之法曰:「戶多之鄉以十二年,戶少以九年,而應差之戶通輪一周。以一周月日而參之戶等,
 戶稅多者占役之日多,少者以率減下,則均適無頗矣。雖以等周差,皆許募人為代,如此則四等往往少差,而五等差所不及矣。衙前悉令招募,以坊場錢支酬重難,此法為允。」



 當是時,議役法者皆下之詳定所,久不能決。於是文彥博言:「差役之法,置局眾議,命令雜下,致久不決。」於是詔罷詳定局,役法專隸戶部。



 諫議大夫鮮于侁言:「開封府多官戶,祥符縣至闔鄉止有一戶應差,請裁其濫。凡保甲之授班行者,如進納人例,須至升朝,方免
 色役。」舊法,戶賦免役錢及三百緡者,令仍輸錢免役。侍御史王巖叟謂:「此法不見其利。借如兩戶,其一輸錢及三百千,其一及二百八九十千,相去幾何,而應差者三年五年即得休息,其應輸助者畢世入錢,無有已時,非至破家,終不得免。此其勢必巧為免計,有弟兄則析居,不則咸賣其業,但少降三百千之數,則遂可免。不出二三年,高強戶皆成中戶。」其後又詔:舊輸免役錢戶及百千以上,令如六色戶輸錢助役。蓋欲以其錢廣雇,使番
 休優久。凡戶少之鄉,應差不及三番者,許以六色錢募州役;尚不及兩番,則申戶部,移用他州錢,以紓差期。鄉戶衙前受役,當休無代,即如募法給雇食之直;若願就投募者,仍免本戶身役,不願者,速募人代之。



 元祐二年,翰林學士兼侍讀蘇軾言:「差役之法,天下皆云未便。昔日雇役,中戶歲出幾何;今者差役,中戶歲費幾何。更以幾年一役較之,約見其數,則利害灼然。而況農民在官,官吏百端蠶食,比之雇人,苦樂十倍。五路百姓樸拙,間
 遇差為胥吏,又轉雇慣習之人,尤為患苦。」尋詔郡縣各具差役法利害,條析以聞。



 四年,右正言劉安世言,御史中丞李常請復雇募,懷奸害政。先是,常言:「差法詔下,民知更不輸錢,嘗歡呼相慶。行之既久,始覺不輸錢為害。何也?差法廢久,版籍不明,重輕無準,鄉寬戶多者僅得更休,鄉狹戶空者頻年在役。上戶極等昔有歲輸百千至三百千者,今止差為弓手,雇人代役,歲不過用錢三四十千。中下戶舊輸錢不過三二千,而今所雇承符、散
 從之類,不下三十千。然則今法徒能優便上戶,而三等、四等戶困苦日甚。望詔一二練事臣僚,使與賦臣取差雇二法便於百姓者行之。無牽新書,無執舊說,民以為善,斯善矣。」而安世則以責民出錢為非,乞固守差役初議,故以常為罪。



 知杭州蘇軾亦言:



 「改行差法,則上戶之害皆去。獨有三等人戶,方雇役時,戶歲出錢極不過三四千,而令一役二年,當費七十餘千。休閑不過六年,則是八年之中,昔者徐出三十餘千,而今者並出七十餘
 千,苦樂可知。



 朝廷既取六色錢,許用雇役以代中戶,頗除一害,以全二利。今惟狹鄉戶少,役者替閑不及三番,方得用六色錢募人以代州役,此法未允。何者?百姓出錢本為免役,今乃限以番次,不用盡用。留錢在官,其名不正,又所雇者少,未足以紓中戶之勞。



 又投名衙前不足元額,而鄉差衙前又當更代,即又別差,更不支錢;若願就長名,則支酬重難盡以給之,仍計日月除其戶役及免助役錢二十千;及州役惟吏人、衙前得皆雇募,此
 外悉用差法,如休役未及三年,即以助役錢支募,此法尤為未通。自元豐前,不聞天下有闕額衙前者,豈嘗抑勒,直以重難月給可以足用故也。當時奉使如李承之之徒,所至已輒減刻,元祐改法,又行減削,既多不支月給,如何肯就招募?今不循其本,乃欲重困鄉差,全不支錢,而應募之人盡數支給,又放免役錢二千貫,欲以誘脅盡令應募,何如直添重難月給,令招募得行。乞促招闕額長名衙前刻期須足,如合增錢雇募,上之監司,議
 定即行。



 役率以二年為一番,向來尚許一戶歇役不及三番,則令雇募,是欲百姓空閑六年。今忽減作二年。幸六色錢足用有餘,正可加添番數,而乃減番添役,農民皆紛然妄謂朝廷移此錢他之。雖云量留一分備用,若有餘剩數,卻量減下無丁戶及女戶所敷役錢,此乃空言無實。丁口、產稅開收增減,年年不同,如何前知來年應役而預為樁科?若亟行減下,臨期不足,又須增取,吏緣為奸,不可勝防矣。大抵六色錢以免役取,當於雇役
 乎盡之,然後名正而人服。惟有一事不得不慮:州縣六色錢多少不同,若各隨多少以為之用,則敷錢多處,役戶優閑太久,六色人戶反覺敷錢數多。欲乞今後六色錢常存一年備用之數,而會計歲所當用,以贏餘而通一路,酌人戶貧富、色役多少預行品配,以一路六色錢能融分給,令州縣盡用雇人,以本處色役輕重為先後。如此則錢均而無弊,雇人稍廣,中戶漸蘇,則差役良法可以久行而不變矣。」



 是時,論役法未便者甚眾。五年,再
 詔中書舍人王巖叟、樞密都承旨韓川、諫議大夫點檢戶曹文字劉安世同看詳利害。戶部請:「河北、河東、陜西鄉差衙前,以投募人所得雇直為則,而減半給之。投名衙前惟差耆長,他投皆免。」



 六年,三省援三路投募衙前役例,概行他路。詔:「凡投募人免其戶二等已下色役,鄉差人戶悉用投名人代之,願長投募者聽。」又詔:「諸州衙前已許量支雇直、餐錢,慮費廣難支,轉運、提刑司其隨土俗參酌立定優重分數及月給餐錢,用支酬額錢給
 之,不得過舊法元數。」州役之應鄉差者,若一鄉人戶終役皆未及四年,許以助役錢募人為之。總計一州雇直,其助役錢不足用,即於戶狹役煩鄉分先與雇代一役,役竟按籍復差如初。諸州歲計助役錢常留一分外,以雇直對計,或闕或剩,提刑司通一路移用。應差諸縣手力,合一鄉休役皆不及三年者,亦許用助役錢雇募;既終一役,別有閑及三年者,復行差法。諸州縣置差役都鼠尾簿,取民戶稅產、物力高下差取,分五等排定,而疏
 其色役年月及其更代人姓名於逐戶之下。每遇差役,即按籍自上而下,吏毋得移竄先後。坊場、河渡錢以雇衙前而有寬剩,亦令補助其餘役人。



 三省言:



 「朝廷審定民役,差募兼行,斟酌補除,極為詳備;而州縣不盡用助役錢募人,以補頻役之地。今括具綱目,下之州縣,使恪承之。



 其一曰:應差之戶,三等以上許休役四年,四等以下許休役六年。若戶少無與更代,卸役不及應閑年數,即用助役錢募人代役以足之。其二曰:狹鄉之縣役人,
 除衙前州胥許雇、壯丁直差不雇外,凡州縣役人皆許招募,以就募月日補除應差而閑不及四年、六年之人,使及年數。每縣通計應差、應募役數若干,立定二額:差者訖役,以應差人承之;雇者有闕,別募人充數。二額悉已立定,如戶力應升應降,須俟三年造簿日按籍別定;未應造簿,止憑定額為準。若本等戶少,不充州縣合役之數,即用次等戶之物力及本等七分者為之。其三曰:寬鄉之縣,除已雇衙前、州胥外,餘役皆以序按差。其四
 曰:官雇弓手,先雇嘗充弓手之人,如不足,以武勇有雇籍者充。他役人願就雇,其選受亦如之。其五曰:壯丁皆按戶版簿名次實輪充役,半年而更。其六曰:一州一路有狹鄉役頻縣分,募錢不足,提刑司以一路助役寬剩錢通融移用;又不足,以坊場、河渡寬剩錢給之。仍通紐一歲應用支酬衙前之類費錢若干,而十分率之,每年於寬剩數內更留二分,以備支酬衙前之類,樁留至五年,通迭一全年寬剩總額,即止不樁;又不足,戶部以別
 路逐色寬剩錢移用以補足之。其七曰:助錢歲歲樁留一分,每及五分止,或時支用,即隨撥補,使常足五分之數。其八曰:「軍人應差迓送者,本以代有雇錢役人,其沿迓送軍人有費,提刑司計數歸之轉運司。其九曰:重役人應替而願仍就募者,許給雇錢受役。其十曰:役人須有稅產乃得就募。其有蔭應贖及曾犯徒刑,雖願募不雇。若工藝人,須有貲產人二戶任之。雇直雖多,皆不得加於舊法已募之數。其十一曰:陜西鎮戎德順軍、熙州
 衙前,皆受田於官以當募直,內地戶願如其法應田募者聽之,仍以坊場、河渡補還轉運司合輸租課。」



 凡縣,歲具色役輕重、鄉分寬狹、凡役雇直有無餘欠,各以其實枚別而上之州。州上監司,監司聚議,連書上戶部。仍別具一路移用及寬剩縣分錢數,致之戶部。



 先是,收到官田,嘗令:田已籍於官及見佃人逃亡,悉拘入之,留充雇募衙前。至是,遂參行田募之法。



 八年,詔:「耆長、壯丁役期已足,不許連續為之。」蓋知其利於賕請,不願更罷故也。
 民有執父母喪而應在役者,三等以下戶除之,三等以上戶令量納役錢,在戶錢十分止責輸三分,服除日仍舊。



 哲宗始親政,三省言役法尚未就緒,帝曰:「第行元豐舊法,而減去寬剩錢,百姓何有不便?」范純仁曰:「四方異宜,須因民立法,乃可久也。」遂令戶部議之。右司諫朱紱言:「輸錢免役,有過數多敷者;用錢雇役,有立直太重者;役色之內,又有優便而願自役募,不必給雇者。請詳為裁省。」中書言:「自行差法十年,民間苦於差擾,前後議者
 紛紜,更變不一,未有底止。」



 於是詔:「復免役法,凡條約悉用元豐八年見制。鄉差役人,有應募者可以更代,即罷遣之。許借坊場、河渡及封樁錢以為雇直,須有役錢日補足其數。所輸免役錢,自今年七月始。耆戶長、壯丁召雇,不得已保正、保長、保丁充代,其它役色應雇者放此。所敷寬剩錢,不得過一分,昔常過數、今應減下者,先自下五等人戶始。路置提舉官一員,視提刑置司之州為治。如方俗利害不同,事有未盡未便而應更改增損舊
 法者,畫一條疏,與轉運、提刑司連奏。」



 又詔:用舊法取量添酒錢贏數,給惟法司吏餐錢;不足,則抵當息錢亦許貼用。先嘗以七月起輸,其後又自來年始。土俗差雇不一,姑仍其舊,俟起輸,至五月盡行雇法,凡因差在役者悉罷遣之。舊免役法行,壯丁間有差而不募者,其毋敷役錢如故。凡錢額所敷,取三年雇直實支,而酌一年中數,立為歲額,以均敷取。此外所取寬餘,不得過通額十分之一。免役錢方復未輸,且以助役錢給雇直,不足,雖
 免役寬剩錢亦許給用。



 七月,戶部看詳役法所言:「幕職監當官之官、罷官,依元豐制,悉用雇役人迓送而差定其數,凡元祐溢額所添廂軍皆罷減。其有抑鄉差之人仍舊在役,或改易名字就便應募,悉計其在役月日應得更代者,以次蠲遣之。諸路舊立出等高強戶,力轉高,敷取難勝,應出免役錢百千以上,每累及百千,悉與減免三分。凡人戶匿寄財產、假借戶貫、冒名官戶茍可避免等第科配者,各以違制論;許人陳告,以其半給之。元
 豐令:在籍宗子及太皇太后、皇后緦麻親得免役。皇太妃宜亦如之。」詔皆如請。



 舊戶等簿,如可略憑即用之,若漫滅等第,即雖未及應造之年,亦令改造。戶部舉行元豐條制,以保正長代耆長,甲頭代戶長,承帖人代壯丁。二年,申詔諸路:「役人額數、雇直,並依元豐舊制,仍依已命,寬剩錢不得過一分。常平免役,元豐上用提舉官專領,轉運、提刑司自今毋預其事。」



 舊置重修編敕所看詳中外文字本,以去年所差鄉役未盡善,遂入議曰:「都、副
 保正比耆長事責已輕,又有承帖人受行文書,即大保長苦無公事。元豐本制,一都之內,役者十人,副正之外,八保各差一大長。今若常輪二大長分催十保稅租、常平錢物,一稅一替,則自不必更輪保丁充甲頭矣。凡都保所雇承帖人,必選家於本保者,而雇直皆從官給,一年一替,則自無浮浪稽留符移之弊。承帖雇直固有舊數,其今所雇保正之直視耆長,保長之直則視戶長;若應此三役不願替代者,自從其願。壯丁元不敷雇直處,
 聽如其舊。承帖雇錢許以舊寬剩錢通融支募,如土俗有不願就保正長雇役者,許募本土有產稅戶,使為耆長、壯丁以代之。其所雇耆、戶長,已立法不得抑勒矣,若保正、長不願就雇而輒差雇者,從徒二年坐罪。」詔皆從之。



 三年,左正言孫諤言:「役法之行,在官之數,元豐多,元祐省,雖省未嘗廢事,則多不若省;雇役之直,元豐重,元祐輕,雖輕未嘗不應募,則重不若輕。今役法優下戶使弗輸,而盡取諸上戶,意則美矣,而法未善也。夫先帝建
 免役之法,而熙寧、元豐有異論,元祐有更變,正惟不能無弊爾。願無以元豐、元祐為間,期至於均平便民而止。則善矣。」翰林學士蔡京言:「諤之論多省、輕重,明有抑揚,謂元豐不若元祐明矣。諤於陛下追紹之日,敢為此言,臣竊駭之。免役法復行將及一年,天下吏習而民安之,而諤指以為弊,則所詆者熙寧、元豐也。且元豐,雇法也;元祐,差法也,雇與差不可並行。元祐固嘗兼雇,已紛然無紀矣,而諤欲不間熙、祐,是欲伸元祐之奸,惑天下之
 聽。」詔罷諤正言,黜知廣德軍。



 後又詔:「諸縣無得以催稅比磨追甲頭、保長,無得以雜事追保正、副。在任官以承帖為名、占破當直者,坐贓論。所管催督租賦,州縣官輒令陪備輸物者,以違制論。」



 是歲,以常平、免役、農田、水利、保甲,類著其法,總為一書,名《常平免役敕令》,頒之天下。詔翰林學士承旨兼詳定役法蔡京依舊詳定重修敕令。侍御史董敦逸言:「京在元祐初知開封府,附司馬光行差法,祥符一縣,數日間差至一千一百人。乞以役法
 專委戶部。」詔令疏析。京奏上,復令敦逸自辨,京無責焉。



 元符二年,以蕭世京、張行為郎。二人在元祐中,皆嘗言免役法為是,帝出其疏擢之。既而詔河北東西、淮南運司,府界提點司,如人戶已嘗差充正夫,其免夫錢皆罷催。後又詔:「雖因邊事起差夫丁,須以應差雇實數上之朝廷,未得輒差。其河防並溝河歲合用一十六萬八千餘夫,聽人戶納錢以免。」



 建中靖國元年,戶部奏:「京西北路鄉書手、雜職、斗子、所由、庫秤、揀、掏之類,土人願就募,
 不須給之雇直,他路亦須詳度施行。」詔從之。知延安府范純粹言:「比年衙前公盜官錢,事發即逃。乞許輪差上等鄉戶使供衙役。」殿中侍御史彭汝霖劾純粹所言有害良法,宜加黜責。詔純粹所乞不行。其後,知襄州俞□以襄州總受他州布綱而轉致他州,是衙前重役並在一州,事理不均。臣僚謂□輒毀紹聖成法,請重黜。□坐責授散官,安置太平州。



 崇寧元年,尚書省言:「前令大保長催稅而不給雇直,是為差役,非免役也。」詔提舉司以
 元輸雇錢如舊法均給。永興軍路州縣官乞復行差役;湖南、江西提舉司以物賤乞減吏胥雇直,罷給役人雇錢,皆害法意,應改從其舊。詔戶部並遵奉《紹聖常平免役敕令格式》及先降《紹聖簽貼役法》,行之天下。



 二年,臣僚言:「常平之息,歲取二分,則五年有一倍之數;免役剩錢,歲收一分,則十年有一年之備。故紹聖立法,常平息及一倍,免役寬剩及三料,取旨蠲免,以明朝廷取於民者,非以為利也。而集賢殿修撰、知鄧州呂仲甫前為戶
 部侍郎,輒以狀申都省,乞刪去上條。」詔黜仲甫,落職知海州。後又詔:常平司候豐衍有餘日,具此制奏蠲之。



 大觀元年,詔:「諸州縣召募吏人,如有非四等以上戶及在州縣五犯杖罪,悉從罷遣,不得再占諸處名役,別募三等以上人充。」於是舊胥既盡罷,而弊根未革,老奸巨猾,匿身州縣,舞法擾民,蓋甚前日。其後,又不許上三等人戶投充弓手,所募皆浮浪,無所顧藉,盜賊公行,為害四方。至是,復詔州縣募役依元豐舊法。



 政和元年,臣僚
 言:「元豐中,鞏州歲敷役錢止四百千,今累敷至緡錢近三萬。又元豐八年,命存留寬剩錢毋得過二分,紹聖再加裁定,止許存留一分。此時考詳法意,非取寬剩,遂改名準備錢,而嚴立禁約,若擅增敷歲額及樁留準備過數者,並以違制論。今乞飭提舉常平官檢察,及核究鞏州取贏之因以聞。」從之。



 宣和元年,言者謂:「役錢一事,神宗首防官戶免多,特責半輸。今比戶稱官,州縣募役之類既不可減,雇令官戶所減之數均入下戶,下戶於常
 賦之外,又代官戶減半之輸,豈不重困?」詔:「自今二等以上戶,因直降指揮非泛補官者,輸賦、差科、免役並不得視官戶法減免,已免者改之。進納人自如本法。」保長月給雇錢,督催稅賦。比年諸縣或每稅戶一二十家,又差一人充甲頭及催稅人,十日一進,赴官比磨,求取決責,有害良民,詔禁之。七年,詔:「州縣昨因儆察私鑄,令五家為保。城郭亦差坊正、副領受文書,由此追呼陪費,或析居、逃移以避差使。其所置坊正、副可罷。」



 自紹聖復雇役,
 而建炎初罷之。已而討論其法之不可廢也,參政李固言於高宗曰:「常平法本於漢耿壽昌,豈可以王安石而廢之?」且當時招射士無以供庸直,詔官戶役錢勿減半,民戶役錢概增二分。後復減之。兼官舊給庸錢以募戶長,及立保甲,則儲庸錢以助經費。未幾,廢保甲,復戶長,而庸錢不復給,遂為總制窠名焉。



 然役起於物力,物力升降不肴,則役法公。是以紹興以來,講究推割、推排之制:凡百姓典賣典業,稅賦與物力一並推割。至於推排,
 則因其貲產之進退為之升降,三歲而下行之。然當時之弊,或以小民粗有米粟,僅存室廬,凡耕耨刀斧之器,雞豚犬彘之畜,纖微細瑣皆得而籍之。吏視賂之多寡,為物力之低昂。上之人憂之,於是又為之限制,除質庫房廊、停塌店鋪、租牛、賃船等外,不得以豬羊雜色估計,其後並耕牛租牛以免之。若江之東西,以畝頭計稅,亦有不待推排者。



 保正、長之立也,五家相比,五五為保,十大保為都保,有保長,有都、副保正;餘及三保亦置長,五
 大保亦置都保正,其不及三保、五大保者,或為之附庸,或為之均並,不一也。戶則以物力之高下為役次之久近。



 若夫品官之田,則有限制,死亡,子孫減半;蔭盡,差役同編戶。



 一品五十頃,二品四十五頃,三品四十頃,四品三十五頃,五品三十頃,六品二十五頃,七品二十頃,八品十頃,九品五頃。



 封贈官子孫差役,亦同編戶。謂父母生前無官,因伯叔或兄弟封贈者。



 凡非泛及七色補官,不在限田免役之數;其奏薦弟侄子孫,原自非泛、七色而來者,仍同差役。進納、軍功、捕盜、宰執給使、減年補授,轉至升朝官,即為官戶;身
 亡,子孫並同編戶。太學生及得解經省試者,雖無限田,許募人充役。



 單丁、女戶及孤幼戶,並免差役。凡無夫無子,則為女戶。女適人,以奩錢置產,仍以夫為戶。其合差保正、長,以家業錢數多寡為限,以限外之數與官、編戶輪差。總首、部將免保正、長差役。文州義士已免之田,不許典賣,老疾身亡,許承襲。



 凡募人充役,並募土著之人,其放停兵及嘗為公人者,並不許募。既有募人,官不得復追正身。募人憑借官勢,奸害善人,斷罪外,坐募之者。
 高宗在河朔,親見閭閻之苦,嘗嘆知縣不得人,一充役次,即便破家,是以講究役法甚便。



 乾道五年,處州松陽縣倡為義役,眾出田谷,助役戶輪充,自是所在推行。十一年,御史謝諤言:「義役之行,當從民便,其不願者,乃行差役。」上然之。朱熹謂義役有未盡善者四事。蓋始倡義役者,惟恐議之未詳,慮之未周,而踵之者不能皆善人,於是其弊日開,其流日甚。或以材知把握,而專義役之利;或以氣力凌駕,而私差役之權。是以虐貧擾富,凌寡
 暴孤。義役之名立,而役戶不得以安其業;雇役之法行,而役戶不得以安其居,信乎所謂未盡善之弊也。淳熙五年,臣僚奏令提舉官歲考屬邑差役當否,以詞訟多寡為殿最;令役戶輪管以提其役,置募人以奉官之行移,則公私便而義役立矣。



 慶元二年,吏部尚書許及之因淳熙陳居仁所奏,取祖宗免役舊法及紹興十七年以後續降旨符,修為一書,名曰《役法撮要》。五年,書成,左丞相京鏜上之。其法可以悠久,其或未久而輒弊者,人
 也。



 振恤水旱、蝗螟、饑疫之災,治世所不能免,然必有以待之,《周官》「以荒政十有二聚萬民」是也。宋之為治,一本於仁厚,凡振貧恤患之意,視前代尤為切至。諸州歲歉,必發常平、惠民諸倉粟,或平價以糶,或貸以種食,或直以振給之,無分於主客戶。不足,則遣使馳傳發省倉,或轉漕粟於他路;或募富民出錢粟,酬以官爵,勸諭官吏,許書歷為課;若舉放以濟貧乏者,秋成,官為理償。又不足,
 則出內藏或奉宸庫金帛,鬻祠部度僧牒;東南則留發運司歲漕米,或數十萬石,或百萬石濟之。賦租之未入、入未備者,或縱不取,或寡取之,或倚閣以須豐年。寬逋負,休力役,賦入之有支移、折變者省之,應給蠶鹽若和糴及科率追呼不急、妨農者罷之。薄關市之徵,鬻牛者免算,運米舟車除沿路力勝錢。利有可與民共者不禁,水鄉則蠲蒲、魚、果、蓏之稅。選官分路巡撫,緩囚系,省刑罰。饑民劫囷窖者,薄其罪;民之流亡者,關津毋責渡錢;
 道京師者,諸城門振以米,所至舍以官第或寺觀,為淖糜食之,或人日給糧。可歸業者,計日並給遣歸;無可歸者,或賦以閑田,或聽隸軍籍,或募少壯興修工役。老疾幼弱不能存者,聽官司收養。水災州縣具船□伐拯民,置之水不到之地,運薪糧給之。因饑疫若厭溺死者,官為埋祭,厭溺死者加賜其家錢粟。京師苦寒,或物價翔踴,置場出米及薪炭,裁其價予民。前後率以為常。蝗為害,又募民撲捕,易以錢粟,蝗子一升至易菽粟三升或五
 升。詔州郡長吏優恤其民,間遣內侍存問,戒監司俾察官吏之老疾、罷懦不任職者。



 初,建隆三年,戶部郎中沈義倫使吳越還,言:「揚、泗饑民多死,郡中軍儲尚餘萬斛,宜以貸民。」有司沮之曰:「若歲不稔,誰任其咎?」義倫曰:「國家以廩粟濟民,自當召和氣,致豐年,寧憂水旱耶?」太祖悅而從之。四年,詔州縣興復義倉,歲收二稅,石別收一斗,貯以備兇歉。平廣南、江南,輒詔振其饑,其勤恤遠人,德意深厚。



 太宗恭儉仁愛,諄諄勸民務農重谷,毋或妄
 費。是時惠民所積,不為無備,又置常平倉,乘時增糴,唯恐其不足。真宗繼之,益務行養民之政,於是推廣淳化之制,而常平、惠民倉殆遍天下矣。



 仁宗、英宗一遇災變,則避朝變服,損膳徹樂。恐懼修省,見於顏色;惻怛哀矜,形於詔旨。慶歷初,詔天下復立義倉。嘉祐二年,又詔天下置廣惠倉,使老幼疾貧者皆有所養。累朝相承,其慮於民也既周,其施於民也益厚。而又一時牧守,亦多得人,如張詠之治蜀,歲糶米六萬石,著之皇祐甲令。富弼
 之移青州,擇公私廬舍十餘萬區,散處流民以廩之,凡活五十餘萬人,募而為兵者又萬餘人,天下傳以為法。知鄆州劉夔發廩振饑,民賴全活者甚眾,盜賊衰止,賜詔褒美。知越州趙抃揭榜於通衢,令民有米增價以糶,於是米商輻湊,越之米價頓減,民無饑死。若是之政,不可悉書,故於先王救荒之法為略具焉。



 神宗即位以來,河北諸路水旱薦臻,兼發糴便司、廣惠倉粟以振民。熙寧二年,賜判北京韓琦詔曰:「河北歲比不登,水溢地震。
 方春東作,民攜老幼,棄田廬,日流徙於道。中夜以興,慘怛不安。其經制之方,聽便宜從事,有可以左右吾民者,宜為朕撫輯而振全之,毋使後時,以重民困。」。而王安石秉政,改貸糧法而為借助,移常平、廣惠倉錢斛而為青苗,皆令民出息,言不便者輒得罪,而民遂不聊生。又詔賣天下廣惠倉田。自是先朝良法美意,所存無幾。哲宗雖詔復廣惠倉,既而章惇用事,又罷之,賣其田如熙寧法。常平量留錢斛,不足以供振給,義倉不足,又令通一
 路兌撥。於是詔聖、大觀之間,直給空名告敕、補牒賜諸路,政日以隳,民日以困,而宋業遂衰。



 先是,仁宗在位,哀病者乏方藥,為頒《慶歷善救方》。知雲安軍王端請官為給錢和藥予民,遂行於天下。嘗因京師大疫,命太醫和藥,內出犀角二本,析而視之。其一通天犀,內侍李舜舉請留供帝服御。帝曰:「吾豈貴異物而賤百姓?」竟碎之。又蠲公私僦舍錢十日。令太醫擇善察脈者,即縣官授藥,審處其疾狀予之,無使貧民為庸醫所誤,夭閼其生。天
 禧中,於京畿近郊佛寺買地,以瘞死之無主者。瘞尸,一棺給錢六百,幼者半之;後不復給,死者暴露於道。嘉祐末,復詔給焉。



 京師舊置東、西福田院,以廩老疾孤窮丐者,其後給錢粟者才二十四人。英宗命增置南、北福田院,並東、西各廣官舍,日廩三百人。歲出內藏錢五百萬給其費,後易以泗州施利錢,增為八百萬。又詔:「州縣長吏遇大雨雪,蠲僦舍錢三日,歲毋過九日,著為令。」熙寧二年,京師雪寒,詔:「老幼貧疾無依丐者,聽於四福田院
 額外給錢收養,至春稍暖則止。」九年,知太原韓絳言:「在法,諸老疾自十一月一日州給米豆,至次年三月終。河東地寒,乞自十月一日起支,至次年二月終止;如有餘,即至三月終。」從之。凡鰥、寡、孤、獨、癃老、疾廢、貧乏不能自存應居養者,以戶絕屋居之;無,則居以官屋,以戶絕財產充其費,不限月。依乞丐法給米豆;不足,則給以常平息錢。崇寧初,蔡京當國,置居養院、安濟坊。給常平米,厚至數倍。差官卒充使令,置火頭,具飲膳,給以衲衣絮被。
 州縣奉行過當,或具帷帳,雇乳母、女使,糜費無藝,不免率斂,貧者樂而富者擾矣。



 三年,又置漏澤園。初,神宗詔:「開封府界僧寺旅寄棺柩,貧不能葬,令畿縣各度官不毛地三五頃,聽人安厝,命僧主之。葬及三千人以上,度僧一人,三年與紫衣;有紫衣,與師號,更使領事三年,願復領者聽之。」至是,蔡京推廣為園,置籍,瘞人並深三尺,毋令暴露,監司巡歷檢察。安濟坊亦募僧主之,三年醫愈千人,賜紫衣、祠部牒各一道。醫者人給手歷,以書所
 治瘞人,歲終考其數為殿最。諸城、砦、鎮、市戶及千以上有知監者,依各縣增置居養院、安濟坊、漏澤園。道路遇寒殭僕之人及無衣丐者,許送近便居養院,給錢米救濟。孤貧小兒可教者,令入小學聽讀,其衣襴於常平頭子錢內給造,仍免入齋之用。遺棄小兒,雇人乳養,仍聽宮觀、寺院養為童行。宣和二年,詔:「居養、安濟、漏澤可參考元豐舊法,裁立中制。應居養人日給粳米或粟米一升,錢十文省,十一月至正月加柴炭,五文省,小兒減半。
 安濟坊錢米依居養法,醫藥如舊制。漏澤園除葬埋依見行條法外,應資給若齋醮等事悉罷。」



 高宗南渡,民之從者如歸市。既為之衣食以振其饑寒,又為之醫藥以救其疾病;其有隕於戈甲、斃於道路者,則給度牒瘞埋之。



 若丐者育之於居養院;其病也,療之於安濟坊;其死也,葬之於漏澤園,歲以為常。



 紹興以來,歲有水旱,發常平義倉,或濟或糶或貸,如恐不及。然當艱難之際,兵食方急,儲蓄有限,而振給無窮,復以爵賞誘富人相與補助,亦權宜不得已之策也。



 元年,詔出粟
 濟糶者賞各有差。



 糶及三千石以上,與守闕進義校尉;一萬五千石以上,與進義校尉;二萬石以上,取旨優賞;已有官蔭不願補授者,比類施行。



 六年,湖、廣、江西旱,詔撥上供米振之。婺民有遏糶致盜者,詔閉糶者斷遣。殿中侍御史周秘言:「發廩勸分,古之道也,許以斷遣,恐貪吏懷私,善良被害。望戒守令多方勸諭,務令樂從,或有擾害,提舉司劾奏。」從之。是歲,潼川守臣景興宗、廣安軍守臣李瞻、果州守臣王騭、漢州守臣王梅活饑民甚眾,前吏部郎中馮楫亦出米以助振給,興宗升一職,瞻、騭、梅、楫各
 轉一官。十年,通判婺州陳正同振濟有方,窮谷深山之民,無不沾惠,以其法下諸路。



 二十八年夏,浙東、西田損於風水。在法,水旱及七分以上者振濟,詔自今及五分處亦振之。二十九年,詔諸處守臣撥常平義倉米二分振糶,臨安府撥樁積之米。三十一年正月,雪寒,民多艱食。詔臨安府並屬縣以常平米減時之半,振糶十日;臨安府城內外貧乏之家,人給錢二百、米一斗及柴炭錢,並於內藏給之;凡遇寒、遇暑、遇雨、遇火、遇赦及祈禱、即位、生辰、上尊號、生皇太子、晏駕、大祥之
 類,臨安之民暨三衙諸軍時有振恤,及放商稅、公私房賃。



 輔郡之民,令諸州以常平錢依臨安府振之。



 孝宗隆興二年秋,霖雨害稼,出內帑銀四十萬兩,變糴以濟民。乾道六年夏,振浙西被水貧民。七年八月,湖南、江西旱,立賞格以勸積粟之家。



 無官人:一千五百石補進義校尉,願補不理選將仕郎者聽;二千石補進武校尉,進士與免文解一次,四千石補承信郎,進士與補上州文學;五千石補承節郎,進士補迪功郎。文臣:一千石減二年磨勘,選人轉一官;二千石減三年磨勘,選人循一資,各與占射差遣一次;三千石轉一官,選人循兩資,各與占射差遣一次。武臣:一千石減二年磨勘,選人轉一資;二千石減三年磨勘,選人循一資,各與占射差遣一次;三千石轉一官,選人循兩資,各與占
 射差遣一次。五千石以上,文武臣並取旨優與推恩。



 九月,臣僚言:「諸路旱傷,請以檢放展閣責之運司,糶給借貸責之常平,覺察妄濫責之提刑,體量措置責之安撫。」上諭宰執曰:「轉運司止今檢放,恐他日振濟不肯任責。」虞允文奏曰:「轉運司主一路財賦,謂之省計。凡州郡有餘、不足,通融相補,正其責也。」淳熙八年,詔:「去歲江、浙、湖北、淮西旱傷處已行振糴,其鰥寡孤獨貧不自存、無錢收糴者,濟以義米。」寧宗慶元元年,以兩浙轉運副使沉詵言米價翔踴,凡商販之
 家盡令出糶,而告藏之令設矣。嘉定十六年,詔於楚州所儲米撥二萬石濟山東、西。



 淳熙八年,浙東提舉朱熹言:「乾道四年民艱食,熹請於府,得常平米六百石振貸,夏受粟於倉,冬則加息計米以償。自後隨年斂散,歉,蠲其息之半;大饑,即盡蠲之。凡十有四年,得息米造倉三間,及以元數六百石還府。見儲米三千一百石,以為社倉,不復收息,每石只收耗米三升。以故一鄉四五十里間,雖遇兇年,人不闕食。請以是行於倉司。」時陸九淵在
 敕令局,見之嘆曰:「社倉幾年矣,有司不復舉行,所以遠方無知者。」遂編入《振恤》」



 凡借貸者,十家為甲,甲推其人為之首;五十家則擇一通曉者為社首。每年正月,告示社首,下都結甲。其有逃軍及無行之人,與有稅錢衣食不闕者,並不得入甲。其應入甲者,又問其願與不願。願者,開具一家大小口若干,大口一石,小口減半,五歲以下不預請。甲首加請一倍。社首審訂虛實,取人人手書持赴本倉,再審無弊,然後排定。甲首附都簿載某人借若干石,依正簿分兩時給:初當下田時,次當耘耨時。秋成還穀不過八月三十日足,濕惡不實者罰。



 嘉定末,真德秀帥長沙行之,兇年饑歲,人多賴之。然事久而弊,或移用而無可給,或拘催無異正賦,良法美意,胥此焉失。



 寶慶三年,
 監察御史汪剛中言:「豐穰之地,穀賤傷農;兇歉之地,濟糴無策。惟以其所有餘濟其所不足,則饑者不至於貴糴,而農民亦可以得利。乞申嚴遏糴之禁,凡兩浙、江東西、湖南北州縣有米處,並聽販鬻流通;違,許被害者越訴,官按劾,吏決配,庶幾令出惟行,不致文具。」從之。端平元年六月,臣僚奏:「建陽、邵武群盜嘯聚,變起於上戶閉糴。若專倚兵威以圖殄滅,固無不可;然振救之政一切不講,餞饉所迫,恐人懷等死之心。附之者日眾。欲望朝
 廷厲兵選士,湯定已竊發之寇;發粟振饑,懷來未從賊者之心,庶人知避害,賊勢自孤,可一舉而滅矣。此成周荒政散利除害之說也。」八月,以河南州軍新復,令江、淮制置大使司科降米麥一百萬石振濟。淳熙十一年,福建諸郡旱,錫米二十五萬石振糴,一萬石振貧乏細民。



 景定元年,臨安府平糴倉舊貯米數十萬石,糶補循環,其後用而不補,所存無幾。有旨令臨安府收糴米四十萬石,用平糴倉錢三百四萬七千八百五十九貫,封樁
 庫十七界會子一千九十五萬二千一百餘貫,共輳十七界一千四百萬貫,充糴本錢。二年,以都城全仰浙西米斛,誘人入京販糶,賞格比乾道七年加優。



 咸淳元年,有旨豐儲倉撥公田米五十萬石付平糴倉,遇米貴平價出糶。二年,監察御史趙順孫言:「今日急務,莫過於平糴。乾道間,郡有米斗直五六百錢者,孝宗聞之,即罷其守,更用賢守,此今日所當法者。今粒食翔踴,未知所由,市井之間見楮而不見米。推原其由,實富家大姓所至
 閉廩,所以糴價愈高而楮價陰減。陛下念小民之艱食,為之發常平義倉,然為數有限,安得人人而濟之?願陛下課官吏,使之任牛羊芻牧之責;勸富民,使之無秦、越肥瘠之視。糴價一平,則楮價不因之而輕,物價不因之而重矣。」七年,以咸淳三年以前諸路義米一百一十二萬九千餘石減價發糶,薄收郡縣聽民不拘關、會、見錢收糶。



\end{pinyinscope}