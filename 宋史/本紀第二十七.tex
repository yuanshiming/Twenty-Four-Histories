\article{本紀第二十七}

\begin{pinyinscope}

 高宗四



 二年春正月癸巳朔,帝在紹興府,率百官遙拜二帝,不受朝賀。甲午,詔復置賢良方正直言極諫科。丙申,賜楊邦乂謚曰忠襄。韓世忠圍建州。丁酉,蠲諸路元年逋稅。
 庚子,陜西叛將白常圍岷州,關師古率兵破之。辛丑,韓世忠拔建州,範汝為自焚死,斬其二弟,餘黨悉平。壬寅,帝發紹興。曹成釋向子諲。丙午,帝至臨安府。壬子,遣韓世清捕石陂賊。癸丑,以張浚檢校少保、定國軍節度使。劉豫遣兵犯伊陽縣,翟興及其將李恭合擊敗之。曹成犯郴州永興縣。己未,修臨安城。辛酉,遣內侍任源撫問張浚。江西副總管楊惟忠以楊勍雖就招安,復謀作亂,誘誅之。



 二月甲子,楊華復叛,擾鼎、澧、潭三州。詔立賞禽
 捕首領,赦貸脅從。丙寅,命劉光世將銳卒萬人屯揚州,經理淮東。庚午,以李綱為觀文殿學士、湖廣宣撫使。仍命岳飛率馬友、李宏、韓京、吳錫等共討曹成諸盜。甲戌,以吏部尚書李光為淮西招撫使,王□燮副之。乙亥,雨雹。丙子,以施逵、謝向、陸棠黨範汝為,逵除名、婺州編管,向、棠械赴行在,俱道死。丁丑,分崔增、李捧、邵青、趙延壽、李振、單德忠、徐文所部兵為七將,名御前忠銳軍,隸步軍司,非樞密奉旨,不許調遣。減淮南營田歲租三之二,俟
 三年復舊。己卯,劉光世入見,同執政對內殿,諭以進屯揚州,光世迄不行。庚辰,詔監司避本貫。壬午,程昌寓遣杜湛募兵攻賊周倫,破之。甲申,以工部員外郎滕茂實死節於代州,贈龍圖閣直學士。丙戌,初置著作官二員修《日歷》。己丑,復荊湖東、西為荊湖南、北路,南路治潭,北路仍治鄂。申禁福建路私有、私造兵器。是月,知商州董先叛入劉豫。金人陷慶陽府,執楊可升,降之。



 三月壬辰朔,命襄、鄧鎮撫使桑仲收復陷沒諸郡,仍命諸鎮撫使
 互相應援。再貶徐秉哲惠州,吳開南雄州,莫儔韶州,並居住。水賊翟進襲漢陽軍,殺守臣趙令戣。李光執韓世清於宣州以歸。虔化縣賊李敦仁及其徒皆授官,隸諸軍。乙未,復置江陰軍。罷福建路武尉。戊戌,葉夢得罷。以李光為江東安撫大使,兼滁、濠等六州宣撫使。罷江、淮發運司。桑仲如郢州調兵,守將霍明以仲將謀逆,殺之,以其事聞。庚子,金人攻方山原,陜西統制楊政援之,金兵引去。辛丑,又犯隴安縣,吳璘等擊走之。淮南營田副
 使王寔括閑田三萬頃給六軍耕種。丙午,復置中書門下省檢正官,省樞密院檢詳官。己酉,以神武右軍中部統制楊沂中為神武中軍統制。癸丑,河南鎮撫使翟興為部將楊偉所殺。甲寅,金人復自水洛城來攻,楊政等又敗之。庚申,曹成寇賀州清水砦,守臣劉全棄城去。是月,知壽春府陳卞及鈐轄陳寶等舉兵復順昌府,尋引兵歸,為偽齊所逐,並壽春失之。



 夏四月甲子,曹成陷賀州。陳顒圍循州,焚龍川縣,命江西安撫司遣將捕之。丙
 寅,賜禮部進士張九成以下二百五十九人及第、出身。庚午,以翰林學士承旨翟汝文參知政事。壬申,釋福建諸州雜犯死罪以下囚。江西軍賊趙進寇瑞昌縣,楊惟忠討降之。戊寅,偽齊統領王資率兵來歸。富順監男子李勃偽稱徐王,召赴行在。壬午,詔內外侍從、監司、守臣各舉中原流寓士大夫三二人,以備任使。癸未,詔曰:「朕登庸二相,倚遇惟均。其所薦用之人,不得偏私離間,朋比害政。」謚孫傅曰忠定。乙酉,李綱始拜命,置司福州。是
 夜,太平州軍士陸德據城叛,囚守臣張錞,殺當塗縣令鐘大猷。戊子,命呂頤浩都督江、淮、荊、浙諸軍事。庚寅,劉豫徙居汴京。是月,王彥大破董先於馬嶺關,復商州。



 閏月癸巳,高麗遣使入貢。乙未,知池州王進討陸德,誅之。丙申,岳飛擊破曹成於賀州。置都督府隨軍轉運司。丁酉,左朝奉郎孫覿坐前知臨安府贓污,貸死除名、象州羈管。罷後苑工作。辛丑,韓世清以狂悖伏誅。丙午,岳飛敗曹成於桂嶺縣,成走連州,遣統制張憲追擊,破之,又
 走郴州,入邵州。丁未,賜福建宣撫司賞軍錢十萬緡。聽朱勝非自便。乙卯,詔諸鎮撫使非奉朝旨,毋擅出兵。劉光世聞父喪去官,特命起復。己未,詔自今明堂專祀昊天上帝,以太宗配。是月,張浚命利、夔制置使王庶與知成都府王似兩易其職。襄、鄧副都統制李橫、同副都統制李道合兵圍郢州,霍明遁去。



 五月辛酉,以兵部尚書權邦彥簽書樞密院事,以樞密將領趙琦所部兵為忠銳第八將。癸亥,呂頤浩出師,以神武後軍及忠銳兩將
 從行,百官班送。甲子,以霍明權襄、鄧、隨、郢州鎮撫使。詔觀察使已上各薦可備將帥者二人。丁卯,罷兩浙轉運司回易庫。己巳,廢紹興府餘姚、上虞縣湖田為湖,溉民田。庚午,詔修建康行宮。辛未,選宗室子偁之子伯琮育於禁中。丙子,呂頤浩總師至常州,前軍將趙延壽兵叛於呂城鎮。丁丑,延壽犯金壇縣,殺知縣胡思忠。頤浩稱疾不進。戊寅,海州賊王山犯漣水軍,總領蘇復、副統制劉靖會兵擊敗之。庚辰,臨安府火。癸未,置御前軍器所。
 甲申,親慮囚,自是歲如之。罷行在權官。乙酉,劉光世遣王德追趙延壽叛兵至建平縣,悉誅之。丙戌,置修政局,命秦檜提舉。詔侍從、臺省寺監官、監司、守令條具省費裕國強兵息民之策。丁亥,以中書門下省檢正官仇悆為沿海制置使。戊子,手詔用建隆故事,命百官日輪一人轉對。兩浙轉運副使徐康國獻銷金屏障,詔有司毀之,奪康國二官。蠲太平州被賊之家夏稅。是月,張浚以參贊軍事劉子羽知興元府,黜王庶,復以王似知成都
 府。韓世忠至洪州,遣董旼招曹成,成聽命赴行在。



 六月庚寅朔,李宏引兵入潭州,執馬友,殺之。甲午,李綱領兵三千發福州。戊戌,詔孟庾、韓世忠班師。岳飛屯駐江州。庚子,以劉光世為寧武、寧國軍節度使,韓世忠為太尉,移屯建康府。辛丑,以李橫為襄、郢鎮撫使,李道鄧、隨鎮撫使。壬寅,翟汝文罷。孔彥舟叛降偽齊。乙巳,以權邦彥兼權參知政事。戊申,仇悆兼制置福建路。辛亥,免臺諫官輪對。甲寅,召呂頤浩赴行在,令參謀官傅崧卿權主
 管都督府事。詔兩浙、江、淮守臣,令存撫東北流寓人。乙卯,韓世忠遣統制解元、巨振入潭州,執李宏以歸。



 秋七月辛酉,悉蠲福建諸州被兵之家田稅。壬戌,復置湖北提舉茶鹽司。甲子,罷福建提舉市舶司。己巳,起復翟琮為河南府、孟汝唐州鎮撫使。甲戌,罷淮東路提點刑獄司。丙子,馬友黨郝通率兵五萬歸宣撫司。戊寅,知廬州王亨復安豐、壽春縣。己卯,呂頤浩入見。庚辰,韓世忠討劉忠,駐兵於岳州之長樂渡,大破之,忠走淮西。丁亥,詔
 編次建炎以來譜牒。



 八月壬辰,以孟庾兼權同都督江、淮、荊、浙諸軍事。癸巳,順昌縣賊餘勝等作亂,通判南劍州王元鼎捕殺之。甲午,安定郡王令話薨。丙申,詔郡守除罷赴闕,皆得引對。臨安府火。以知江州劉紹先為沿淮防遏使。戊戌,命朱勝非提舉醴泉觀兼侍讀,日赴朝堂議事。沿海州縣籍民海舶,每歲一更,守海道險要。振福建饑民。己亥,停傅雱官、英州羈管。庚子,詔孟庾、韓世忠總大兵至建康,進赴行在。戊申,給事中胡安國以論
 朱勝非罷,宰執、臺諫上疏留之,皆不報。江西統制傅樞討平南雄賊張忠、鄧慶、劉軍一等。己酉,賜吳玠田。甲寅,秦檜罷。給事中程瑀等坐論駁朱勝非,疑其黨檜,並落職主宮觀。彗出胃。乙卯,減膳,戒輔臣修闕政,罷修建康行宮。



 九月戊午朔,落秦檜職。己未,罷修政局。辛酉,以彗出,大赦,許中外臣民直言時政,陜西諸叛將許令自新。壬戌,王倫自金國使還入見。遣潘致堯等為金國軍前通問使,附茶藥金幣進兩宮。甲子,以直徽猷閣郭偉為
 淮西巡撫使。乙丑,復以朱勝非為尚書右僕射、同中書門下平章事兼知樞密院事。戊辰,司空山賊李通出降,以為都督府親軍統領。癸酉,以右朝請大夫呂源為浙東、福建沿海制置使,治定海縣。知建昌軍朱芾擊石陂賊餘照,禽斬之。甲戌,彗沒。丙子,復以郭仲荀為武泰軍節度使。詔墨敕有不當者,許三省、樞密院奏稟,給事中、中書舍人繳駁,臺諫論列,有司申審。庚辰,命福建提舉茶鹽官兼領市舶司。辛巳,以韓世忠為江南東、西路宣
 撫使,他帥臣稱宣撫使者並罷。壬午,遣監察御史明橐等五人宣諭江、浙、湖、廣、福建諸路,仍降詔諭官吏以遣使按察、勸懲、誅賞之意。癸未,新作行宮南門成。甲申,提轄榷貨務張純峻立淮、浙鹽法,增其算。總領四川財賦趙開初變四川鹽法,盡榷之。乙酉,太白晝見。丙戌,以知興元府王似為川、陜宣撫處置副使。丁亥,封右監門衛大將軍、榮州防禦使令畤為安定郡王。是月,韓世忠遣統制解元襲擊劉忠於蘄陽,大破之。忠奔劉豫。



 冬十月
 戊子朔,置牧馬監於饒州。庚寅,李勃伏誅。丙申,初置江、浙、荊湖、廣南、福建路都轉運使。甲辰,潘致堯至楚州,通判州事劉晏劫其禮幣奔劉豫,守臣柴春戰死。戊申,以知平江府趙鼎為江東安撫大使。丙辰,禁溫、臺二州民結集社會。班度量權衡於諸路,禁私造者。是月,顏孝恭招降石陂餘賊李寶等。



 十一月辛酉,陳顒陷汀州武平縣,犯梅、循二州。乙丑,初榷明州鹵田鹽。辛未,議將撫師江上,召侍從官條具利害。甲戌,命李綱、劉洪道、程昌寓、
 解潛會兵捕討湖寇楊太。戊寅,範汝為餘黨範忠掠龍泉縣。庚辰,詔宣諭五使,焚所至州縣建炎以前已蠲稅籍。癸未,臨安大火。是月,關師古敗偽齊兵於抹邦山。馬友黨步諒詣李綱降,綱入潭州,其黨郝晸降王進,吳錫禽王浚。湖南盜賊悉平。



 十二月丁亥朔,命神武前軍領申世景等討捕範忠。己丑,偽稱榮德帝姬易氏伏誅。範忠犯處州。巨師古引兵入廬州,執王亨送行在。甲午,李綱罷。臨安府火。丙申,振被火家。罷浙東沿海制置司。
 丁酉,岳飛遣統領徐慶、王貴討禽萍鄉賊高聚。己亥,以胡舜陟為廬、壽等州鎮撫使。金人侵熙、秦,關師古擊敗之。庚子,遣駕部員外郎李願撫諭川、陜。江西兵馬副鈐轄張忠彥坐縱暴不法,斬於潭州。辛丑,程昌寓遣杜湛討楊欽等,敗之,殺三千餘人。癸卯,川、陜宣撫司類試陜西發解進士,得周謨等十三人,以便宜賜進士出身。甲辰,罷張浚宣撫處置使,仍知樞密院。以知夔州盧法原為川、陜宣撫處置副使,及王似同治司事。己酉,遣司封
 員外郎周隨亨同撫諭川、陜。庚戌,孟庾自建康來朝。辛亥,金人犯商州,守將邵隆退屯上津。李橫敗偽齊兵,復汝州。甲寅,命孟庾同都督江、淮、荊、浙諸軍事。詔都督府總治江東西、湖北、浙西帥臣經畫屯田。張浚承制以歸州隸夔州路。是冬,金人犯和尚原,將士乏食自潰,吳璘拔砦棄去。虔賊謝達犯惠州。



 三年春正月丁巳朔,帝在臨安,率百官遙拜二帝,不受朝賀。江西將李宗諒誘戍兵叛,寇筠州,統領趙進擊卻
 之。翟琮入西京,禽偽齊留守孟邦雄。命諸路憲臣兼提舉常平司。庚申,金人犯上津。李橫破穎順軍,偽齊知軍蘭和降。壬戌,金人犯金州洵陽縣。以仇悆為福建、兩浙、淮東路沿海制置使。癸亥,陳顒圍潮州不下,引兵趨江西。甲子,李橫復穎昌府。乙丑,詔中外刑官各務仁平,臺憲檢察,月具所平反以聞,歲終考察殿最。金人陷金州,鎮撫使王彥焚積聚,退保西鄉。庚午,罷行在宗正司,命嗣濮王仲湜兼判大宗正事。辛未,震電雨雹。造渾天儀。
 李通為其徒王全所殺。壬申,命西外宗正移司福州。癸酉,復祭大火。以湯東野為淮東安撫使。乙亥,以李橫為襄陽府、鄧隨郢州鎮撫使。丁丑,登、萊山砦統制範溫率部兵泛海來歸。庚辰,詔春秋望祭諸陵。張浚論奏王似不可為副,因引罪求罷,不報。癸未,詔民復業者,視墾田多寡定租額賦役。乙酉,減淮、浙蠶鹽錢。



 二月丁亥朔,升桂州為靜江府。乙丑,權邦彥薨。浙東賊彭友犯龍泉縣。辛卯,李通餘黨劉德圍舒州。吳玠遇金人於饒風關,王
 彥自西鄉來會,金人分兵攻關,統制郭仲敗走。丁酉,饒風關破,玠趣西縣,彥奔達州,四川大震。張浚被罷職之命,以諸軍方潰,因秘不行,復具奏審。己亥,金帥撒離曷入興元府,經略使劉子羽焚其城走三泉縣,吳玠退屯仙人關。庚子,以宗子伯琮為和州防禦使,賜名瑗,尋改貴州。辛丑,蠲廣東諸州被賊民家稅。壬寅,鄭州兵馬鈐轄牛皋、彭□率兵與李橫會,橫以便宜命皋為蔡、唐州鎮撫使,□知汝州。乙巳,翟琮遣統制李吉敗偽齊兵於
 伊陽,又殲其將梁進之眾。丁未,王似始受宣撫副使之命。戊申,虔賊周十隆犯循、梅、汀州,詔統制趙祥等合兵捕之。庚戌,以李橫為神武左軍副統制、京西招撫使。改胡舜陟為淮西安撫使。辛亥,以工部尚書席益參知政事,翰林學士徐俯簽書樞密院事。壬子,王全犯廬州。甲寅,詔守臣至官半年,具上民間利害或邊防五事。李橫遣人奏穎昌之捷,詔許橫便宜行事。乙卯,劉光世遣酈瓊等屯兵泗州為李橫聲援。是月,張浚復以王庶為參
 謀官,往巴州措置。時金兵深入至金牛鎮,疑有伏,由褒斜穀引兵還興元,吳玠、劉子羽追擊其後,殺獲甚眾。



 三月己未,詔岳飛捕虔賊。壬戌,申命統制巨師古部兵萬人屯揚州。胡舜陟至廬州,王全降。甲子,以趙鼎為江西安撫大使。李橫傳檄諸軍收復東京。己巳,金人遣兵援劉豫,李橫敗走,穎昌復陷。壬午,以韓世忠為淮南東路宣撫使。李綱遣兵擊降李宗諒,詔戮於市。



 夏四月丁亥,朱勝非以母喪去位。偽齊知虢州董震及其統制董先
 來歸,以震權商、虢、陜州鎮撫使。己丑,詔江東西、湖北、浙西募民佃荒田,蠲三年租。辛卯,以劉光世為檢校太傅、江南東路宣撫使。金人去興元。壬辰,徙都督府於鎮江。岳飛軍次虔州。甲午,偽齊知唐州胡安中來歸。丙申,偽齊李成攻陷虢州,董先、牛皋奔襄陽。己亥,改謚昭慈獻烈皇后為昭慈聖獻。復舉五帝日月之祀。庚子,增文武小官奉。辛丑,荊南統制羅廣率兵至鼎州。楊太眾益盛,自號大聖天王,立鐘相少子子義為太子,廣等不克討
 而還。丁未,岳飛遣統領張憲、王貴擊彭友,禽斬之。劉忠為部下王林所殺,傳首行在。戊申,以浙西兵馬鈐轄史康民所部兵為忠銳第九將。己酉,張浚奏王庶、王似、盧法原威望素輕,乞命劉子羽、吳玠並為判官,不報。辛亥,徐文叛奔偽齊。



 五月丙辰,以翟琮為河南府、孟汝鄭州鎮撫使,董先為副使。丁巳,遣樞密計議官任直清撫諭襄陽、商、虢、河南諸鎮。己未,命楊沂中招捕嚴州盜賊。辛酉,建睦親宅。以董先為商、虢、陜州鎮撫使。征河南布衣
 王忠民為宣教郎,至行在,辭不受。壬戌,潘致堯還,言金人欲重臣通使以取信,遂寢出師之議。乙丑,罷諸州在任守臣所闢通判。丁卯,以韓肖冑等充金國軍前通問使。安化蠻犯邊,廣西經略使許中發兵擊之。戊辰,楊沂中招降嚴州賊繆羅等,捕斬其徒百人,魔賊平。庚午,以岳州數被兵,免今年稅役。壬申,詔守、令、尉、佐,境內妖民聚集不能覺察致亂者,並坐罪。知建昌軍朱芾討南豐縣賊,禽誅其魁黃琛。乙亥,以方與金國議和,禁邊兵犯
 齊境。丙子,王彥復金州,金兵棄均、房去。韓世忠請以大軍還鎮江。己卯,詔淮南統制解元戍泗州,餘屯江北。周隨亨、李願宣押王似、盧法原至閬州,張浚始解使事。時已論金牛之功,以吳玠為利州路、階成鳳州制置使,劉子羽為寶文閣直學士,王彥為保大軍承宣使,僚屬將帥第賞有差。庚辰,浚及子羽、王庶、劉錫等赴行在。詔李橫等收軍還鎮。辛巳,罷宣撫司便宜黜陟。



 六月甲申朔,統制巨師古坐違韓世忠節制,除名、廣州編管。丙戌,復
 置六部架閣庫。丁亥,禁諸路招納淮北人及中原軍來歸者。戊子,復元祐宰相呂大防官職,贈謚。庚寅,詔降川、陜死罪囚,釋流以下。賞吳玠、關師古將士。壬辰,張浚至綿州,復奏王似不可任。甲午,命王□燮率諸軍討楊太。己亥,罷沿海制置司。丁未,置國子監及博士弟子員。戊申,以王林所部兵為忠銳第十將。己酉,岳飛自虔州班師。辛亥,發兵屯駐虔、廣二州,彈壓盜賊,州各三千人。是月,金人圍方山原,王似命吳玠發兵救之。



 秋七月己未,復
 置博學宏詞科。初許任子就試。甲子,以久旱,償州縣和市民物之直。丁卯,詔訪求累朝勛臣曹彬等三百人子孫,以備錄用。戊辰,王□燮以舟師發行在。己巳,詔減膳,禁屠,弛工役,罷苛嬈,命兩浙及諸路憲臣親按部錄囚。辛未,蠲紹興二年和市絁帛。癸酉,呂頤浩等以旱乞罷政,帝賜詔曰:「與其去位,曷若同寅協恭,交修不逮,思所以克厭天心者。」頤浩等乃復視事。乙亥,朱勝非起復。丙子,泉州水溢,壞城。丁丑,遣中使逆趣張浚於道。是月,四川
 霖雨、地震。



 八月己丑,詔岳飛赴行在,留精兵萬人戍江州。翟琮率兵突圍奔襄陽,詔屯駐其地。癸卯,罷諸路輸禁軍闕額錢。甲辰,以雨暘不時,蘇、湖地震,求直言。乙巳,復置史館修撰、直館檢討官,命郎官兼領著作郎及佐郎。戊申,罷都轉運司。己酉,詔湖南丁米三分之二均取於民田,其一取之丁口。辛亥,孟庾自軍中來朝。



 九月戊午,呂頤浩罷。詔凡遇水旱災異,監司、郡守即具奏毋隱。庚申,岳飛自江州來朝。川、陜統領官吳勝敗偽齊兵於
 黃堆砦。丙寅,以趙鼎為江西安撫制置大使。壬申,詔中書舍人、給事中,凡制敕非軍期機速,必先書押而後報行。甲戌,偽齊王彥先寇徐、宿二州。乙亥,以劉光世為江東、淮西宣撫使,置司池州;韓世忠為鎮江建康府、淮南東路宣撫使,置司鎮江府;王□燮為荊南府、岳鄂潭鼎澧黃州漢陽軍制置使,置司鄂州;岳飛為江南西路、舒蘄州制置使,置司江州;主管殿前司郭仲荀知明州,兼沿海制置使,神武中軍統制楊沂中兼權殿前司。己卯,吳
 勝克蓮花城。



 冬十月癸未,朱勝非上《重修吏部七司敕令格式》。庚寅,加吳玠檢校少保。壬辰,趣王□燮進兵。詔寬私鹽重法。甲午,卻大理國入貢。丁酉,殘破州縣視戶口增損立守令考課法。己亥,禁州縣擅增置稅場。偽齊李成陷鄧州。辛丑,南丹蠻莫公晟圍觀州,焚寶積監,殺知監陳烈。壬寅,偽齊兵逼襄陽,李橫以糧盡,棄城奔荊南,知隨州李道亦棄城去。甲辰,王□燮討湖賊,戰於鼎口,不利。偽齊陷郢州,守臣李簡棄城去。申禁私役戰士。丁未,
 命三省除銓曹奸弊。戊辰,罷諸路類省試。統制石世達及杜湛合兵大破湖賊黃誠於龍陽洲。庚戌,復置宗正少卿及寺監諸丞。是月,王彥先引兵至北壽春,將渡淮。劉光世駐軍建康,扼馬家渡;又遣酈瓊駐無為軍,為廬、濠聲援。賊乃還。



 十一月己未,以右文殿修撰王倫為都督府參議官。癸亥,詔監司、帥守察內外宗子病民害政者以聞。崔增、吳全遇湖賊於陽武口,死之。甲子,韓肖冑等使還。乙丑,禁沿淮諸砦兵擅侵齊境。庚午,臨安府火。
 甲戌,禁掠賣生口入蠻夷溪峒及以銅錢出中國。乙亥,復元祐十科舉士法。丁丑,命賓、橫、宜、觀四州市戰馬。戊寅,王□燮自鼎州引兵還鄂,留統領王渥等四軍聽程昌寓節制。己卯,蠲南劍州所負民間獻納錢十六萬緡。省淮南州縣文武官。十二月辛巳朔,降敕書撫諭吳玠及川、陜將士。乙酉,臨安府火。戊子,又火。朱勝非以屢火求罷,不允。丙申,王似承制廢通遠軍。己酉,金國元帥府遣李永壽、王翊來見。是歲,海寇黎盛犯潮州,焚民居毀城
 去。



 四年春正月辛亥朔,帝在臨安,率百官遙拜二帝。乙卯,增淮、浙路鹽鈔貼納錢。遣章誼等為金國通問使。己未,程昌寓遣杜湛、王渥攻楊太皮真砦,破之。己巳,詔諭王似、盧法原、吳玠,使之協和。金人犯宕昌、臨江砦及花石峽,關師古遣統領劉戩分兵拒卻之。庚午,詔諸路將帥毋以兩國通使輒弛邊備,淮南州郡津渡尤慎譏察。甲戌,罷州縣新置弓手。乙亥,蠲循、梅、潮、惠四州被兵家租
 賦。丙子,申敕三省、樞密院,除官並遵舊制,毋相侵紊,除拜、罷免皆明示黜陟之由。戊寅,金人犯神坌砦,沿北嶺至大散關。臨安府火。己卯,韓肖冑罷。



 二月壬午,詔贓罪至死者仍籍其貲。癸未,作建康府行宮。席益罷。乙酉,以徐俯兼參知政事。丙戌,禁川、陜諸將招納北軍。湖北軍賊檀成犯長陽縣,解潛遣統領胡勉捕斬之。群盜田政自襄陽犯峽州。己丑,解潛遣統制王恪擊政,斬之。庚寅,金人犯兩當縣。乙未,詔孟庾赴行在。己亥,詔三衙管軍
 及將帥觀察使以上,舉忠勇智略可自代者一人。辛丑,金人犯仙人關。癸卯,詔權以射殿為景靈宮,四時設位朝獻。丙午,張浚入見。



 三月辛亥朔,吳玠率楊政、吳璘、田晟、王喜諸將與兀□戰於仙人關,大敗之。兀□遁去。戊午,雨雹。以趙鼎參知政事。壬戌,孟庾至行在,罷都督府,以其兵屬張俊。乙丑,張浚以資政殿大學士罷,尋落職奉祠、福州居住。己巳,蠲淮南州縣民租一年。辛未,日有青赤黃氣。編次建炎以來詔旨,頒諸路。癸酉,蠲興元府、
 洋州被兵家稅役二年。丙子,以王似為資政殿學士、川陜宣撫使,盧法原為端明殿學士,與吳玠並充副使,關師古為熙河蘭廓路安撫制置使。



 夏四月庚辰朔,命趙開再任總領四川財賦。詔諭川、陜官吏兵民,以張浚失措當示遠竄,猶嘉其所用吳玠等能御大敵,許國一心,止從薄責。仍令宣撫司講求諮訪,凡擾民咈眾之事,速厘革之。癸未,劉子羽白州安置。乙酉,詔明堂用皇祐典禮,兼祀天皇大帝、神州地祇以下諸神。丙戌,吳玠敗金
 兵,復鳳、秦、隴州。詔特旨處死情法不相當者,許大理奏審。蠲淮南州軍上供錢一年。庚寅,置孳生牧馬監於臨安府。甲午,罷廣西提舉茶鹽司。關師古叛,以洮、岷二州降偽齊,吳玠並將師古軍。乙未,詔諸路歲上戶口。丁酉,罷諸州回易庫。庚子,命劉光世遣兵巡邊。辛丑,保靜州夷人入貢。丙午,徐俯罷。是月,王似承制廢符陽軍。知壽春府羅興叛降偽齊。



 五月庚戌朔,以岳飛兼黃復二州、漢陽軍、德安府制置使。癸丑,以範沖為宗正少卿兼直
 史館,重修神宗哲宗《正史》、《實錄》。甲寅,詔淮南帥臣兼營田使,守令以下兼管營田。岳飛復郢州,斬偽齊守荊超。甲子,以孟庾兼權樞密院事。乙丑,賜李橫軍絹萬匹。丙寅,李成棄襄陽去,岳飛復取之。金人攻金州,鎮撫使王彥遣統制許青等與戰於漢陰,敗之。罷諸縣武尉。壬申,裁省三省、樞密細務,責六曹長貳專決。癸酉,以國史日歷所為史館。偽齊收李成餘眾,益兵駐新野,岳飛與別將王萬夾擊,復大敗之。乙亥,王彥數敗金兵於洵陽縣。
 丙子,復選宗室子彥之子伯玖育於禁中。



 六月壬申,復命川、陜類試。乙未,太白晝見經天。戊戌,詔神武軍、神武副軍統制、統領官並隸樞密院。庚子,以霖雨,罷不急之役。壬寅,詔三省、樞密院,凡奉乾請墨敕,許執奏不行。置史館校勘官。作明堂行禮殿於教場。甲辰,禁諸軍強刺平人為兵,已刺者皆釋之。吳玠乞宮觀,不允。是月,熒惑犯南斗。岳飛將牛皋復隨州,執偽齊守王嵩,磔之。



 秋七月戊申朔,曲敕虔州。以吏部尚書胡松年簽書樞密院
 事。庚戌,以湖南安撫席益為安撫制置大使。建昌軍軍卒修達等作亂,殺守臣劉滂,江西制置使胡世將遣參謀侯愨、統制丘贇討之。壬子,命吳玠通信夏國。癸丑,湖賊楊欽等破社木砦,官軍敗卻,小將許筌戰歿。丙辰,賞仙人關之功,以吳玠為檢校少師、奉寧保靜軍節度使,吳璘、楊政以下論賞有差。丁巳,命左右司歲考郎官功過治狀以為賞罰。庚申,復曲端、趙哲官。壬戌,岳飛遣統制王貴、張憲擊敗李成及金兵於鄧州之西,復鄧州,禽
 其將高仲。丙寅,侯愨引兵入建昌軍,執修達等十三人,斬之。罷建州臘茶綱。詔江東安撫司招水軍千五百人。己巳,湖賊萬餘人詣鼎、澧二州降。劉光世來朝。庚午,王貴、張憲破金、齊兵,復唐州及信陽軍,襄漢悉平。辛未,章誼、孫近使還入見,粘罕致書約淮南毋得屯兵。



 八月庚辰,以趙鼎知樞密院事,充川、陜宣撫處置使。湖賊夏誠等犯枝江縣,解潛遣將蔣定舟與戰,敗之。辛巳,吳玠遣統領姚仲攻隴城縣,克之。壬午,王□燮以討賊無功,降光
 州觀察使。戊子,改命趙鼎都督川、陜、荊、襄諸軍事。乙未,遣魏良臣等充金國通問使。丙申,毀王安石舒王告。己亥,周十隆出降,為官軍所掠,復遁去,犯汀、循州。壬寅,王似罷。以岳飛為清遠軍節度使、湖北荊襄潭州制置使,代王□燮討湖賊。癸卯,以襄陽府、隨、郢、唐、鄧州、信陽軍六郡為襄陽府路。



 九月戊申,減淮、浙路鹽鈔所增貼納錢。壬子,夏誠遣將李全功犯公安軍,解潛遣統制林閏等擊斬之。安定郡王令畤薨。辛酉,合祭天地於明堂,大赦,
 蠲襄陽等六郡三年租稅。庚午,朱勝非罷。金、齊合兵自淮陽分道來犯。壬申,渡淮,楚州守臣樊敘棄城去。韓世忠自承州退保鎮江府。癸酉,以趙鼎為尚書右僕射、同中書門下平章事兼知樞密院事,吏部尚書沈與求參知政事。



 冬十月丙子朔,與趙鼎定策親征,命張俊以軍援淮東,劉光世移軍建康,車駕擇日進發。丁丑,以孟庾為行宮留守,留統制王進一軍及神武中軍五百人隸之。百司不預軍旅之務者,聽從便避兵。己卯,韓世忠自
 鎮江率兵復如揚州。金人犯滁州。以張俊為浙西、江東宣撫使。金人圍亳州。席益遣統制吳錫率兵討徭賊楊再興,大破之。壬午,偽齊兵犯安豐縣。癸未,復以張浚為資政殿學士、提舉萬壽觀兼侍讀。甲申,復以王□燮為建武軍承宣使、江西沿江制置使。丙戌,命胡松年詣江上,會諸將議進兵。戊子,韓世忠邀擊金人於大儀鎮,敗之,又遣將董旼敗之於天長縣鴉口橋。己丑,金人攻承州,韓世忠遣將成閔、解元合兵擊於北門,敗之。金人圍濠
 州。甲午,遣秘書正字楊晨持詔撫諭四川。遣侍御史魏□工、監察御史田如鰲詣劉光世、張俊軍中計事,光世始移軍太平州。丙申,命後宮自溫州泛海如泉州。金人陷濠州,守臣寇宏棄城走。丁酉,詔州縣團教弓手、土兵。戊戌,帝御舟發臨安,劉錫、楊沂中以禁兵扈從。己亥,韓世忠捷奏至,命收瘞戰死將士,仍令胡松年致祭。庚子,張俊率兵發鎮江,如建康。壬寅,帝次平江。加贈陳東、歐陽澈秘閣修撰,官其子孫二人,各賜田一頃,且追咎汪伯
 彥落觀文殿學士,黃潛善更不追復。命韓世忠、楊沂中分兵控扼沿海要地。癸卯,焚決淮東閘堰。賜扈從諸軍錢。乙巳,仇悆遣將孫暉擊金人於壽春,敗之,復霍丘、安豐二縣。是月,借江、浙坊場錢一界,以備軍費。



 十一月戊申,太白晝見。庚戌,賞承州水砦首領徐康等要擊金兵之功,轉官有差,仍蠲承、楚、泰州水砦民兵賦役十年。置沿江烽火,放浙東諸郡防城丁夫。壬子,始下詔聲劉豫逆罪,論親討之旨,以厲六師。吳玠遣統制楊從儀等率
 兵敗金人於臘家城。癸丑,玠乞納節贖劉子羽罪,遂聽子羽自便。金人入光州。甲寅,偽齊知光州許約破石額山砦,遂據之。乙卯,韓世忠遣兵夜劫金人營於承州,破之。金人犯六合縣,丙辰,掠全椒縣三城湖。丁巳,戒諸路大小臣僚借貸催科縱吏奸擾民,及務絕盜賊之伺隙者。命董旼、趙康直總領淮東水砦。戊午,以胡松年兼權參知政事。金人陷滁州。劉光世移軍建康,韓世忠移軍鎮江,張俊移軍常州。己未,復命張浚知樞密院事,以其
 盡忠竭節詔諭中外。庚申,宴犒守江將士。癸亥,劉光世遣統制王德擊金人於滁州之桑根,敗之。揭黃榜招諭湖賊。甲子,命滁、和諸州移治保聚。乙丑,金人犯滁口。己巳,劉光世遣統制王師晟等率兵夜入南壽春府襲金人,敗之,執偽齊知府王靖。廣賊區稠圍韶州樂昌縣,鈐轄韓京遣兵擊斬之。詔張浚視師江上。十二月乙亥朔,魏良臣、王繪還自泗州軍前入見。戊寅,命都督府右軍統制李貴部兵屯扼福山鎮。辛巳,命中軍統制王進屯
 兵泰州,防拓通、泰。壬午,以樞密都承旨馬擴為江西沿江制置副使。丙戌,吳倫遣兵攻臘家城,破之。丁亥,聽兩淮避兵民耕種所在閑田。壬辰,金、齊兵逼廬州,仇悆嬰城固守,岳飛所遣統制徐慶、牛皋援兵適至,敗走之。劉光世亦遣統制靳賽戰於慎縣。張俊遣統制張宗顏擊敗金人於六合。詔江、浙、荊湖十四郡各募水軍五百人,名橫江軍。兩浙十郡沿江海州縣招捕巡檢土軍。甲午,程昌寓遣杜湛、彭筠合擊楊欽,破之。己亥,以來年正旦
 日食,下詔修闕政,求直言。庚子,金人退師。辛丑,詔葬祭浙西、江東二軍之死事者。壬寅,省淮南轉運司。遣胡松年往常熟縣、江陰軍沿江計議軍事。癸卯,金人去滁州。



\end{pinyinscope}