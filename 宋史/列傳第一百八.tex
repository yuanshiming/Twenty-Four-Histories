\article{列傳第一百八}

\begin{pinyinscope}

 郝
 質賈逵竇舜卿劉昌祚盧政燕達姚兕弟麟子雄古楊燧劉舜卿宋守約子球



 郝質,字景純,汾州介休人。少從軍,挽強為第一。充殿前
 行門,換供奉官,為府州駐泊都監,主管麟府軍馬,與田朏將兵護軍須饋麟州,道遇西夏數千騎寇鈔,質先驅力戰,斬首、獲馬數百。又與朏行邊,至柏谷,敵塹道以阻官軍,質御之於寒嶺下,轉鬥逐北,遂修復寧遠諸柵,以扼賊沖。宣撫使杜衍、安撫使明鎬連薦之,且條上前後功狀,超遷內殿承制、並代路都監。大名賈昌朝又薦為路鈐轄。



 使討貝州,文彥博至,命部城西。回河上有亭甚壯,彥博慮為賊焚,遣小校藺千守,而質使千往他營度
 戰具,千辭,質曰:「亭焚,吾任其責。」千去而亭焚。彥博將斬千,質趨至帳下曰:「千之去,質實使之,罪乃在質,願代千死。」彥博壯其義,兩釋之。質自此益知名。



 賊平,遷六宅使,歷高陽關、定州、並代鈐轄,駐泊副都部署,龍神衛、捧日天武都指揮使,馬軍殿前都虞候,加領賀州刺史、英州團練、眉州防禦使。奉詔城豐州,進步軍副都指揮使、宿州觀察使。召還宿衛,改馬軍。英宗立,遷武昌軍節度觀察留後,加安德軍節度使,為殿前副指揮使。神宗立,易
 節安武軍,為都指揮使。元豐元年,卒,帝親臨其喪,贈侍中,謚曰武莊。



 質御軍有紀律,犯者不貸,而享犒豐渥,公錢不足,出己奉助之。平居自奉簡儉,食不重肉,篤於信義。田朏不振而死,為表揭前功,官其一孫。在並州,與朝士董熙善,約為婚姻。熙死,家貧無依,質已為節度使,竟以女歸董氏。自為官,不上伐閱,從微至貴,皆以功次遷云。



 賈逵,真定蒿城人。隸拱聖為卒,至殿前班副都知,換西
 染院副使。從狄青征儂智高,戰於歸仁驛。既陳,青誓眾曰:「不待令而舉者斬!」時左將孫節戰死,逵為右將軍先鋒將,私念所部兵數困易衄,兵法先據高者勝,茍復待命而賊乘勝先登,吾事去矣。即日引軍趨山。既定,賊至,逵麾眾馳下,仗劍大呼,斷賊為二。賊首尾不相救,遂潰。逵詣青請罪,青拊其背勞謝之。邕州城空,青使逵入括公私遺墜,固辭。是時,將校多以搜城故匿竊金寶,獨逵無所犯。遷西染院使、嘉州刺史、秦鳳路鈐轄。



 初,逵少孤,
 厚賂繼父,得其母奉以歸。至是,以母老辭,不許,而賜母冠帔。秦山多巨木,與夏人錯壤,逵引輕兵往採伐。羌酋馳至,畫地立表約決勝負。逵引弓連三中的,酋下馬拜伏,從逵取盈而歸。徙並代路,專主管麟府軍馬。熟戶散處邊關,苦於寇略,逵差度遠近,聚為二十七堡,次第相望,自是害乃息。畫鐵為的,激種豪使射,久皆成勁兵。一夕,烽火屢發,左右白當起,逵臥不應。旦而謂人曰:「此必妄也。脫有警,可夜出乎?」徐問之,果邊人燭遺物也。復徙
 秦鳳,去之十日,而代者郭恩敗。朝廷以逵為能,連擢捧日天武四廂都指揮使、馬步殿前都虞候,歷涇原、高陽關、鄜延路副都總管,以利州觀察使入為步軍副都指揮使。



 都城西南水暴溢,注安上門,都水監以急變聞。英宗遣逵督護,亟囊土塞門,水乃止。議者欲穴堤以洩其勢,逵請觀水所行,諭居民徙高避水,然後決之。軍校營城外者,每常朝,即未曉啟門鑰,或輟朝失報,啟鑰如平時。逵言:「禁城當謹啟閉,不宜憑報者。」乃冶鐵鑄「常朝」字,
 俾持以示信。



 遷馬軍副都指揮使,復總鄜延兵。延州舊有夾河兩城,始,元吳入寇據險,城幾不能守。逵相伏龍山、九州臺之間可容窺覘,請於其地築保障,與城相望,延人以為便。轉昭信軍節度觀察留後。逵言:「種諤處綏州降人於東偏,初云萬三千戶,今乃千一百戶耳,逋逃之餘,所存才八百。蕃漢兩下殺傷,皆不啻萬計。自延州運粟至懷寧,率以四百錢致一石。而緣邊居人,壯者但日給一升,罔冒何至大半。諤徒欲妄興邊事以自為功,
 不可不察也。」元豐初,拜建武軍節度使、殿前都指揮使。請不俟郊赦賜三世官,神宗曰:「逵武人,能有念親之志,其特許之。」數月而卒,年六十九。贈侍中,謚曰武恪。



 竇舜卿,字希元,相州安陽人。以蔭為三班奉職,監平鄉縣酒稅。有僧欲授以化汞為白金之術,謝曰:「吾祿足養親,不願學也。」闢府州兵馬監押。夏人犯塞,舜卿欲襲擊,舉烽求援於大將王凱,凱弗應。舜卿度事急,提州兵出戰,勝之。明日,經略使問狀,凱懼,要以同出為報。舜卿歡
 然相許,不自以為功。為青淄路都監。海盜行劫,執博昌鎮官吏,肆剽掠,舜卿募士三百,悉擒之。使契丹,主客馬祐言:「昔先公客省善射,君當傳家法。」置酒請射,舜卿發輒中。祐使奴持二弓示之,一挽皆折。



 湖北蠻徭彭仕羲叛,徙為鈐轄,兼知辰州。建請築州城,不擾而辦。帥師取富州,蠻將萬年州據石狗崖。舜卿選壯卒奮擊,蠻矢石交下,卒蒙盾直前,發強弩射,萬年州斃於崖下,遂拔之。左右欲盡剿其眾,舜卿不許,曰:「仕羲願內附,特為此輩
 所脅,今死矣,何以多殺為?」引兵入北江,仕羲降。擢康州刺史,加龍神衛、捧日天武四廂指揮使、馬軍殿前都虞候,三遷邕州觀察使,歷邠寧環慶路副都總管。熙寧中,十上章求退,且丐易文階。改刑部侍郎,提舉嵩山崇福宮。以光祿大夫致仕,再轉金紫光祿大夫,卒,年八十八。謚曰康敏。



 劉昌祚,字子京,真定人。父賀,戰沒於定川。錄為右班殿直,主秦州威遠砦。青唐聚兵井鹽,經年不散。昌祚奉帥
 命往詰之,諸酋曰:「聞漢家欲取吾鹽井。」昌祚曰:「國家富有四海,何至與汝爭此邪?」與酋俱來,犒賚之,歡然帥眾去。遷西路都巡檢。使遼還,神宗臨試馳射,授通事舍人。夏人寇劉溝堡,昌祚領騎二千出援。虜伏萬騎於黑山而偽遁,卒遇之,戰不解。薄暮,大酋突而前,昌祚抽矢,一發殪之,餘眾悉遁。帥李師中上其功曰:「西事以來,以寡抗眾,未有如昌祚者。」知階州,討平毋家等族,又平疊州。轉作坊使,為熙河路都監。



 從王中正入蜀,破篳篥羌。加
 皇城使、榮州刺史、秦鳳路鈐轄,又加西上閣門使、果州團練使,知河州。元豐四年,為涇原副都總管。王師西征,詔與總管姚麟率蕃漢兵五萬,受環慶高遵裕節制。今兩路合軍以出,既入境,而慶兵不至。昌祚出胡盧川,次磨齊隘,夏眾十萬扼險不可前。昌祚挾兩盾先登,夏人小卻,師乘之,斬首千七百級。進次鳴沙川,取其窖粟,遂薄靈州。城未及闔,先鋒奪門幾入,遵裕馳遣使止之,昌祚曰:「城不足下,脫朝廷謂我爭功,奈何?」命按甲勿攻。是
 夕,慶兵始距城三十里而軍,遇敵接戰,昌祚遣數千騎赴之。遲明,賊已退,遂謁遵裕,遵裕訝應援之緩,有誅昌祚意。既見,問下城如何,昌祚曰:「比欲攻城,以幕府在後未敢。前日磨齊之戰,夏眾退保東關,若乘銳破之,城必自下。」遵裕弗內,曰:「吾夜以萬人負土囊傅壘,至旦入矣。」怒未解,欲奪其兵付姚麟,麟不敢受,乃已。明日,遣昌祚巡營,凡所得馬糧,悉為慶兵所取,涇師忿噪。遵裕圍城十八日,不能下,夏人決七級渠以灌遵裕師,軍遂潰。即南還,
 復命涇師為殿。昌祚手劍水上,待眾濟然後行,為虜所及,戰退之。至渭州,糧盡,士爭入,無復行伍,坐貶永興軍鈐轄。



 明年,復徙涇原,加龍、神衛四廂都指揮使,知延州。時永樂方陷,士氣不振,昌祚先修馬政,令軍中校技擊,優者乃給焉。自義合至德靖砦,綿互七百里,堡壘疏密不齊,烽燧不相應。昌祚度屯戍險易、地望遠近、事力強弱,立為定式,上諸朝。夏人寇塞門、安遠砦,拒破之,殺其統軍葉悖麻、咩吪埋二人,蓋始謀攻永樂者。圖其形以
 獻。帝喜,遣近侍勞軍。



 哲宗立,進步軍都虞候、雄州團練使、知渭州,歷馬軍殿前都虞候。渭地宜牧養,故時弓箭手人授田二頃,有馬者復增給之,謂之「馬口分地」。其後馬死不補,而據地自若。昌祚按舉其法,不二年,耗為復初。又括隴山間田得萬頃,募士卒五千,別置將統之,勁悍出諸軍右。朝廷歸夏人四砦,昌祚以為不可。再遷殿前副都指揮使、冀州觀察使、武康軍節度使。卒,年六十八。贈開府儀同三司,謚曰毅肅。



 昌祚氣貌雄偉,最善騎
 射,箭出百步之外。夏人得箭以為神,持歸事之。所著《射法》行於世。



 盧政,太原文水人。以神衛都頭從劉平與夏人戰延州。虜薄西南隅,兵不得成列,政自變量騎挑戰,發伏弩二百射卻之。日且暮,政說平曰:「今處山間,又逼污澤,宜速退保後山,須明決鬥;不然,彼夜出,乘高蹙我,何以御之?」平不聽,遂敗。政脫身歸,黃德和誣平降賊,仁宗引政問狀,政言:「平被執,非降也。」因自陳失主將當死。帝義其言,
 赦之,以為供奉官、德州兵馬監押。預討貝州,率勇敢數百人,飛繯絓堞而登,守者莫能亢,大軍乘之以入。遷內殿承制。南征儂智高,亦有功。



 歷秦鳳、高陽關都鈐轄。治平、熙寧中,為捧日、天武四廂都指揮使三衛都虞候、副都指揮使,涇原、定州、並代、真定四路副都總管,累轉祁州團練、昌州防禦、黔州觀察使。拜武泰軍節度使,政時年七十三,氣貌不衰,侍立殿下,雖久無惰容,能上馬踴躍,觀者壯之。早朝暴卒,贈開府儀同三司。



 燕達,字逢辰,開封人。為兒時,與儕輩戲,輒為軍陳行列狀,長老異之。既長,容體魁梧,善騎射。以材武隸禁籍,授內殿崇班,為延州巡檢,戍懷寧砦。夏人三萬騎薄城,戰竟日不決,達所部止五百人,躍馬奮擊,所向披靡。擢鄜延都監,數帥兵深入敵境,九戰皆以勝歸。囉兀之棄走,遣達援取戍卒輜重,為賊所邀,且戰且南,失亡頗多。神宗以達孤軍遇敵,所全亦不為少,累遷西上閣門使、領英州刺史,為秦鳳副總管。討破河州羌,遂降木征。遷東
 上閣門使、副都總管,真拜忠州刺史、龍神衛四廂都指揮使。



 郭逵招討安南,為行營馬步軍副都總管。入辭,神宗諭之曰:「卿名位已重,不必親矢石,第激勉將土可也。」達頓首謝曰:「臣得憑威靈滅賊,雖死何憚!」初度嶺,聞前鋒遇敵苦戰,欲往援,偏校有言當先為家計然後進者,達曰:「彼戰已危,詎忍為自全計。」下令敢言安營者斬。乃卷甲趨之,士皆自奮,傳呼太尉來,蠻驚潰,即定廣源。師次富良江,蠻艤斗船於南岸,欲戰不得,達默計曰:「兵法
 致人而不致於人,吾示之以虛,彼必來戰。」已而蠻果來,擊之,大敗,乃請降。師還,拜榮州防禦使。以主帥得罪而獨蒙賞,乞同責,不聽。



 元豐中,遷金州觀察使,加步軍都虞候,改馬軍,超授副都指揮使。以訓閱精整,除一子閣門祗候。數被詔獎,進殿前副都指揮使、武康軍節度使。哲宗立,遷為使,徙節武信。卒,贈開府儀同三司,謚曰毅敏。



 達起行伍,喜讀書,神宗以其忠實可任,每燕見,未嘗不從容。嘗問:「用兵當何先?」對曰:「莫如愛。」帝曰:「威克厥愛
 可乎?」達曰:「威非不用,要以愛為先耳。」帝善之。



 姚兕,字武之,五原人。父寶,戰死定川,兕補右班殿直,為環慶巡檢。與夏人戰,一矢斃其酋,眾潰,因乘之,遂破蘭浪。敵大舉寇邊,諸砦皆受圍。兕時駐荔原堡,先羌未至,據險張疑兵,伺便輒出。有悍酋臨陣甚武,兕前射中其目,斬首還,一軍歡呼。明日,來攻益急,兕手射數百人,裂指流血。又遣子雄引壯騎馳掩其後,所向必克。敵度不可破,乃退攻大順城。兕復往救,轉鬥三日,凡斬級數千,
 卒全二城。慶軍叛,兕以親兵守西關,盜眾不得入而奔。兕追及,下馬與語,皆感泣羅拜,誓無復為亂。



 神宗聞其名,召入覲,試以騎射,屢中的,賜銀槍、袍帶。遷為路都監,徙鄜延、涇原。從攻河州,飛矢貫耳,戰益力。河州既得,又為鬼章所圍,兕曰:「解圍之法,當攻其所必救。」乃往擊隴宗,圍遂解。累遷皇城使,進鈐轄。從攻交址有功,領雅州刺史。破乞弟,領忠州團練使,進副總管,遷東上閣門使,徙熙河。與種誼合兵討鬼章於洮州,破六逋宗城,夜斷
 浮橋,援兵不得度,遂擒鬼章。真拜通州團練使。卒於鄜延總管,贈忠州防禦使。



 兕幼失父,事母孝,凡圖畫器用,皆刻「仇讎未報」字。力學兵法,老不廢書,尤喜顏真卿翰墨,曰:「吾慕其人耳。」弟麟,亦有威名,關中號「二姚」。子雄、古。



 麟字君瑞,兄兕攻河州時,俱在兵間。中矢透骨,鏃留不去,以強弩出之,笑語自若。積功至皇城使,為秦鳳副總管。從李憲討生羌,擒泠雞樸。再轉東上閣門使、英州刺史。元豐西討,以涇原副總管從劉昌祚出戰,勝於磨□
 移隘。轉戰向鳴沙,趨靈州,而高遵裕敗還,降為皇城使、永興軍路鈐轄,復為涇原副總管。夏人修貢,且乞蘭會壤土,麟言:「夏人囚其主,王師是征。今秉常不廢,即為順命,可因以息兵矣。獨蘭會不可與。願戒將帥飭邊備,示進討之形,以絕其望。」從之。督諸將討堪哥平,經略使盧秉上其功狀,賜金帛六百。



 元祐初,擢成州團練使、龍神衛四廂都指揮使,歷步軍殿前都虞候、步軍馬軍副都指揮使。紹聖三年,以建武軍節度觀察留後出知渭州。安
 燾請留之,曾布曰:「臣嘗訪麟御邊之策及熙河疆域,俱不能知。願加敕儆,使之盡力。」韓忠彥曰:「奏對語言,非所以責此輩。」哲宗乃留麟不遣。尋拜武康軍節度使、殿前副都指揮使。王贍取青唐,麟以為朝廷討伐方息肩,奈何復生此大患。已而贍果敗。徽宗立,進都指揮使,節度建雄、定武軍,檢校司徒。卒,帝詣其第臨奠,贈開府儀同三司。



 麟為將沉毅持重,不少縱舍。宿衛士嘗犯法,詔釋之,麟杖之於庭而後請拒詔之罪,故所至肅然。



 雄字毅夫,少勇鷙有謀,年十八即佐父征伐。從討金湯,以百騎先登奪隘,又成荔原之功。韓絳薦其材,閱試延和殿。安南、瀘川之役,皆在軍行。歷涇原、秦鳳將,駐甘谷城,知通遠鎮戎軍、岷州,官累左騏驥使。紹聖中,渭帥章楶城平夏,雄部熙河兵策援,夏人傾國來,與之鏖斗,流矢注肩,戰TE厲,賊引卻,追躡大破之,斬首三千級,俘虜數萬。先五日,折可適敗於沒煙,士氣方沮,雄賈勇得雋,諸道始得並力。城成,擢東上閣門使、秦州刺史。



 明年,虜
 攻平夏,勢銳甚,城幾不守。雄與弟古合兵卻之。徙知會州,領熙河鈐轄。王贍略地青唐,羌人攻湟、鄯,詔雄與苗履援之。邈川方急,雄適至,羌望見塵起,驚而潰。圍既解,遂趨鄯州,履後期乃至,贍言蘭溪宗有遺寇,宜悉蕩平之。履即往,雄諫不聽,戒所部嚴備以待。俄而履師退,賊追及,雄整眾迎擊,破之,獻馘二千。哲宗遣中使持詔勞問,徙河州。種樸戰沒,王贍軍陷敵中,雄自鄯至湟,四戰皆捷,拔出之。遂築安鄉關,夾河立堡,以護浮梁,通湟水
 漕運,商旅負販入湟者,始絡繹於道。加復州防禦使。



 建中靖國初,議棄湟州,詔訪雄利害。雄以為可棄,遂以賜趙懷德,徙雄知熙州,進華州觀察使。蔡京用王厚復河湟,治棄地罪,停雄官,光州居住。三年,得自便。後論為責輕,復竄金州。明年,乃聽歸。高永年死,西寧諸戍阻絕,起雄權經略熙河、安輯復新邊使。知滄州,加捧日、天武四廂都指揮使,復為熙州,遷安德軍節度觀察留後、步軍副都指揮使,拜武康軍節度使。召詣闕,為中太一宮
 使。引疾納節鉞,改左金吾衛上將軍,又以武康節知熙州。熙河十八年間更十六帥,唯雄三至,凡六年。未幾,以檢校司空、奉寧軍節度使致仕。卒,贈開府儀同三司,謚武憲。



 古亦以邊功,官累熙河經略。靖康元年,金兵逼京城,古與秦鳳經略種師中及折彥質、折可求等俱勒兵勤王。時朝命種師道為京畿、河北路制置使,趣召之,師道與古子平仲先已率兵入衛。欽宗拜師道同知樞密院、宣
 撫京畿、河北、河東,平仲為都統制。上方倚師道等卻敵,而種氏、姚氏素為山西巨室,兩家子弟各不相下。平仲恐功獨歸種氏,忌之,乃以士不得速戰為言,欲夜劫斡離不營。謀洩,反為所敗。



 既而議和,金兵退,詔古與種師中、折彥質、範瓊等領兵十餘萬護送之。粘罕陷隆德府,以古為河東制置,種師中副之。古總兵援太原,師中援中山、河間諸郡。粘罕圍太原,內外不相通。古進兵復隆德府、威勝軍,厄南北關,與金人戰,互有勝負。太原圍不
 解,詔古與師中掎角,師中進次平定軍,乘勝復壽陽、榆次等縣。朝廷數遣使趣戰,師中約古及張灝兩軍齊進,而二人失期不至。師中回趨榆次,兵敗而死。金人進兵迎古,遇於盤陀,古兵潰,退保隆德。詔以解潛代之。古之屯威勝軍也,帳下統制官焦安節妄傳寇至以動軍情,既又勸古遁去,故兩郡皆潰。李綱召安節,斬於瓊林苑。中丞陳過庭奏古罪不可恕,詔安置廣州。



 楊燧,開封人。善騎射,應募隸軍籍,從征貝州,穴城以入。
 賊平,功第一,補神衛指揮使。又從征儂智高,接戰,手殺數十人,眾乘之而捷。擢萬勝都指揮使,遷榮州團練使、京城左廂巡檢。救濮宮火,英宗識其面,及即位,以為鄧州防禦使、步軍都虞候。歷環慶、涇原、鄜延三路副都總管,至馬軍副都指揮使,由容州觀察使拜寧遠軍節度、殿前副都指揮使。卒,贈侍中,謚曰莊敏。



 燧初穴貝州城時,為叛兵所傷,同行卒劉順救之得免。及貴,順已死,訪恤其家甚至。故人妻子貧不能活者,一切收養之。人推
 其義。



 劉舜卿。字希元,開封人。父鈞,監鎮戎兵馬,慶歷中,與子堯卿戰死於好水。舜卿年十歲,錄為供奉官,歷昌州駐泊都監。諭降瀘水蠻八百人,誅其桀鰲驁者。知水洛城。



 神宗經略西邊,近臣薦其能,召問狀,對曰:「自元昊稱臣,秦中不復戒嚴。今宜先自治。」帝善之,命訓京東將兵。一年,入閱於內殿,帝嘆曰:「坐作有度,其可用也。爾無忘世讎,勉思忠孝,期以盡敵。」舜卿泣謝,即日加通事舍人。



 環慶
 有警,詔帥長安兵赴之,乃單騎馳往慶州,至則難已解。知原州,改秦鳳鈐轄。襲擊西市城,先登有功,遷皇城副使。久之,知代州,加客省副使。遼遣諜盜西關鎖,舜卿密易舊鑰鐍而大之。數日,虜以鎖來歸,舜卿曰:「吾未嘗亡鎖也。」引視,納之不能受,遂慚去,誅諜者。



 轉西上閣門使、知雄州。始視事,或告契丹游騎大集,請甲以俟,舜卿不為動,乃妄也。契丹系州民,檄索之,不聽。會有使者至,因捕取其一以相當,必得釋乃遣。在雄六年,恩信周浹。



 元祐
 初,進龍神衛四廂都指揮使、高州刺史、知熙州。夏人聚兵天都,連西羌鬼章青宜結,先城洮州,將大舉入寇,舜卿欲乘其未集擊之,會諸將議方略。使姚兕部洮西,領武勝兵合河州熟羌搗講珠城,遣人間道焚河橋以絕西援;種誼部洮東,由哥龍穀宵濟邦金川,黎明,至臨洮城下,一鼓克之,俘鬼章並首領九人,斬馘數千計。遷馬軍都虞候,再遷徐州觀察使、步軍副都指揮使、知渭州。召還宿衛,未上道,卒,贈奉國軍節度使,謚曰毅敏。



 舜卿
 知書,曉吏事,謹文法,善料敵,著名北州。



 宋守約,開封酸棗人。以父任為左班殿直,至河北緣邊安撫副使,選知恩州。仁宗諭以亂後撫御之意,對曰:「恩與他郡等耳,而為守者猶以反側待之,故人心不自安。臣願盡力。」徙益州路鈐轄,累遷文州刺史、康州團練使、知雄州,歷龍神衛、捧日天武都指揮使,馬步殿前都虞候。



 入宿衛,遷洋州觀察使。衛兵以給粟陳嘩噪,執政將付有司治,守約曰:「御軍安用文法!」遣一牙校語之曰:「天
 子太倉粟,不請何為?我不貸汝。」眾懼而聽命。進步軍副都指揮使、威武軍留後。神宗以禁旅驕惰,為簡練之法,屯營可並者並之。守約率先推行,約束嚴峻,士始怨終服。或言其持軍太急,帝密戒之,對曰:「臣為陛下明紀律,不忍使恩出於臣,而怨歸陛下。」帝善之,欲擢置樞府,宰相難之,乃止。故事,當郊之歲,先期籍士卒之兇悍者,配下軍以警眾,當受糧而倩人代負者罰,久而浸弛,守約悉舉行之。所居肅然無人聲,至蟬噪於庭亦擊去,人以為
 過。蒞職十年卒,年七十一。贈安武軍節度使,謚曰勤毅。



 子球,以蔭乾當禮賓院。條秦、川券馬四弊,群牧使用其議,馬商便之。再使高麗,密訪山川形勢、風俗好尚,使還,圖紀上之,神宗稱善,進通事舍人。帝崩,告哀契丹,至,則使易吉服,球曰:「通和歲久,憂患是同,大國安則為之。」契丹不能奪。積遷西上閣門使、樞密副都承旨。為人謹密,朝日所聞上語,雖家人不以告。卒於官。



 論曰:自郝質至宋守約,皆恂直忠篤,為一時名將。遭世
 承平,邊疆少警,擁節旄,立殿陛,高爵重祿,以壽考終,宜也。姚氏世用武奮,兕與弟麟並有威名,關中號「二姚」。兕之子雄,亦以戰功至節度使,而古竟以敗貶,其才否可見已。



\end{pinyinscope}