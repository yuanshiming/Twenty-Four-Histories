\article{列傳第九十一}

\begin{pinyinscope}

 滕元發李師中陸詵子師閔趙離孫路游師雄穆衍



 滕元發,初名甫,字符發。以避高魯王諱,改字為名,而字達道,東陽人。將生之夕,母夢虎行月中,墮其室。性豪雋
 慷慨,不拘小節。九歲能賦詩,範仲淹見而奇之。舉進士,廷試第三,用聲韻不中程,罷,再舉,復第三。授大理評事、通判湖州。孫沔守杭,見而異之,曰:「奇才也,後當為賢將。」授以治劇守邊之略。



 召試,為集賢校理、開封府推官、鹽鐵戶部判官、同修起居注。英宗書其姓名藏禁中,未及用。神宗即位,召問治亂之道,對曰:「治亂之道如黑白、東西,所以變色易位者,朋黨汩之也。」神宗曰:「卿知君子小人之黨乎?」曰:「君子無黨,闢之草木,綢繆相附者必蔓草,
 非松柏也。朝廷無朋黨,雖中主可以濟;不然,雖上聖亦殆。」神宗以為名言,太息久之。進知制誥、知諫院。御史中丞王陶論宰相不押班為跋扈,神宗以問元發,元發曰:「宰相固有罪,然以為跋扈,則臣以為欺天陷人矣。」



 拜御史中丞。種諤擅築綏州,且與薛向發諸路兵,環、慶、保安皆出剽掠,夏人誘殺將官楊定。元發上疏極言諒祚已納款,不當失信,邊隙一開,兵連民疲,必為內憂。又中書、樞密制邊事多不合,中書賞戰功而樞密降約束,樞密
 詰修堡而中書降褒詔。元發言:「戰守,大事也,而異同如是,願敕二府必同而後下。」宰相以其子判鼓院,諫官謂不可。神宗曰:「鼓院傳達而已,何與於事。」元發曰:「人有訴宰相,使其子達之,可乎?」神宗悟,為罷之。



 京師郡國地震,元發上疏指陳致災之由,大臣不悅,出知秦州。神宗曰:「秦州,非朕意也。」留不遣。館伴契丹使楊興公,開懷與之語,興公感動,將去,泣之而別。河北地大震,命元發為安撫使。時城舍多圮,吏民懼壓,皆幄寢茇舍,元發獨處屋
 下,曰:「屋摧民死,吾當以身同之。」瘞死食饑,除田租,修堤障,察貪殘,督盜賊,北道遂安。除翰林學士、知開封府。民王穎有金為鄰婦所隱,閱數尹不獲直。穎憤而致傴,扶杖訴於庭。元發一問得實,反其金,穎投杖仰謝,失傴所在。



 夏國主秉常被篡,元發言:「繼遷死時,李氏幾不立矣。當時大臣不能分建諸豪,乃以全地王之,至今為患。今秉常失位,諸將爭權,天以此遺陛下,若再失此時,悔將無及。請擇立一賢將,假以重權,使經營分裂之,可不勞
 而定,百年之計也。」神宗奇其策,然不果用。



 元發在神宗前論事,如家人父子,言無文飾,洞見肝鬲。神宗知其誠藎,事無鉅細,人無親疏,輒皆問之。元發隨事解答,不少嫌隱。王安石方立新法,天下哅哅,恐元發有言,神宗信之也,因事,以翰林侍讀學士出知鄆州。徙定州。初入郡,言新法之害,且曰:「臣始以意度其不可耳,既為郡,乃親見之。」歲旱求言,又疏奏:「新法害民者,陛下既知之矣,但下一手詔,應熙寧三年以來所行有不便者,悉罷之,則
 民心悅而天意解矣。」皆不聽。



 歷青州、應天府、齊、鄧二州。會婦黨李逢為逆,或因以擠之,黜為池州,未行,改安州。流落且十歲,猶以前過貶居筠州。或以為復有後命,元發談笑自若,曰:「天知吾直,上知吾忠,吾何憂哉。」遂上章自訟,有曰:「樂羊無功,謗書滿篋;即墨何罪,毀言日聞。」神宗覽之惻然,即以為湖州。



 哲宗登位,徙蘇、揚二州,除龍圖閣直學士,復知鄆州。學生食不給,民有爭公田二十年不決者,元發曰:「學無食而以良田飽頑民乎?」乃請以
 為學田,遂絕其訟。時淮南、京東饑,元發慮流民且至,將蒸為癘疫。先度城外廢營地,召諭富室,使出力為席屋,一夕成二千五百間,井灶器用皆具。民至如歸,所全活五萬。徙真定,又徙太原。



 元發治邊凜然,威行西北,號稱名帥。河東十二將,其八以備西邊,分半番休。元發至之八月,邊遽來告,請八將皆防秋。元發曰:「夏若並兵犯我,雖八將不敵;若其不來,四將足矣。」卒遣更休。防秋將懼,扣閣爭之。元發指其頸曰:「吾已舍此矣,頭可斬,兵不可
 出。」是歲,塞上無風塵警,詔以四砦賜夏人,葭蘆在河東,元發請先畫境而後棄,且曰:「取城易,棄城難。」命部將訾虎領兵護邊,夏不敢近。夏既得砦,又欲以綏德城為說,畫境出二十里外。元發曰:「是一舉而失百里,必不可。」九上章爭之。



 以老力求淮南,乃為龍圖閣學士,復知揚州,未至而卒,年七十一,贈左銀青光祿大夫,謚曰章敏。



 李師中,字誠之,楚丘人。年十五,上封事言時政。父緯為涇原都監,夏人十餘萬犯鎮戎,緯帥兵出戰,而帥司所
 遣別將郭志高逗遛不進,諸將以眾寡不敵,不敢復出,緯坐責降。師中詣宰相辯父無罪,時呂夷簡為相,詰問不屈,夷簡怒,以為非布衣所宜言。對曰:「師中所言,父事也。」由是知名。



 舉進士,鄜延龐籍闢知洛川縣。民有罪,妨其農時者必遣歸,令農隙自詣吏。令當下者榜於民,或召父老諭之。租稅皆先期而集。民負官茶直十萬緡,追系甚眾,師中為脫桎梏,語之曰:「公錢無不償之理,寬與汝期,可乎?」皆感泣聽命。乃令鄉置一匱,籍其名,許日輸
 所負,一錢以上輒投之。書簿而去。比終歲,逋者盡足。官移諸郡粟於邊,已而反之,盛冬大雪,勞且費,至賤售予兼並家。師中令過縣願輸者聽,躬坐庾門,執契以須,數日,得萬斛。使下其法於他縣。嘗出鄉亭,見戎人雜耕,皆兵興時入中國,人藉其力,往往結為婚姻,久而不歸。師中言若輩不可雜處,言之經略使,並索旁郡者,徙諸絕塞。



 龐籍為樞密副使,薦其才。召對,轉太子中允、知敷政縣,權主管經略司文字。夏人以歲賜緩,移邊曰:「願勿逾
 歲暮。」詔吏報許,師中更牒曰:「如故事。」樞密院劾為擅改制書,師中曰:「所改者郡牒耳,非制也。」朝廷是之,薄其過。



 提點廣西刑獄。桂州靈渠故通漕,歲久石窒舟滯,師中即焚石,鑿而通之。邕管有馬軍五百,馬不能夏,多死。師中謂地皆險阻,無所事騎,奏罷之。士人補攝官,銓授無法,權在吏。悉記其名,使待除於家。



 初,邕州蕭注、宜州張師正謀啟邊釁,注欲以所管蠻峒酋豪往討交址,云不用朝廷兵食。詔下經略使蕭固、轉運使宋咸,二人為注
 所餌,合詞稱便,而師中至,詔以注奏付之。師中邀注來,難之曰:「君以酋豪伐交址,能保必勝乎?」曰:「不能。」師中曰:「既不能保必勝,脫有敗衄,奈何?」注知不可,遂罷議。會蠻猺申紹泰入追亡者,害巡檢宋士堯,注又張皇為駭奏,仁宗為之旰食。師中言無足憂,因劾注邀功生事,掊斂失眾心,卒致將率敗覆,按法當斬。於是注責泰州安置,並按固、咸,皆坐貶。師中攝帥事。交址耀兵於邊,聲言將入寇。師中方宴客,飲酒自若,草六榜揭境上,披以其情
 得,不敢動,即日貢方物。紹泰懼,委巢穴遁去。儂智高子宗旦保火峒,眾無所屬,前將規討以幸賞,遂固守。師中檄諭禍福,立率其族以地降。邊人化其德,多畫像立祠以事,稱為桂州李大夫,不敢名。



 還,知濟、兗二州。濟水堙塞久,師中訪故道,自兗城西南啟鑿之,功未半而去。遷直史館、知鳳翔府。種諤取綏州,師中言:「西夏方入貢,叛狀未明,恐彼得以借口,徒啟其釁端也。」鄜延路覘知西夏駐兵綏、銀州,檄諸路當牽制,師中疏論牽制之害。時
 諸將皆請行,師中曰:「不出兵,罪獨在帥,非諸將憂也。」既而此舉卒罷。



 熙寧初,拜天章閣待制、河東都轉運使。西人入寇,以師中知秦州。詔賜以《班超傳》,師中亦以持重總大體自處。前此多屯重兵於境,寇至則戰,嬰其銳鋒,而內無以遏其入。師中簡善守者列塞上,而使善戰者中居,令諸城曰:「即寇至,堅壁固守;須其去,出戰士尾襲之。」約束既熟,常以取勝。



 王韶築渭、涇上下兩城,屯兵以脅武勝軍,撫納洮、河諸部。下師中議,遂言:「今修築必廣
 發兵,大張聲勢,及令蕃部納土,招弓箭手,恐西蕃及洮、河、武勝軍部族生疑。今不若先招撫青唐、武勝及洮、河諸族,則西蕃族必乞修城砦,因其所欲,量發兵築城堡,以示斷絕夏人鈔略之患,部人必歸心。唐於西域,每得地則建為州,其後皆陷失,以清水為界。大抵根本之計未實,腹心之患未除,而勤遠略、食土地者,未有不如此者。」詔師中罷帥事。韶又請置市易,募人耕緣邊曠土,師中奏阻其謀。王安石方主韶,坐以奏報反復罪,削職知
 舒州。徙洪、登、齊,復待制、知瀛州。又乞召司馬光、蘇軾等置左右。師中言時政得失,又自稱薦曰:「天生微臣,蓋為聖世,有臣如此,陛下其舍諸。」呂惠卿易文其語,以為罔上,遂貶和州團練副使安置。還右司郎中,卒,年六十六。



 師中始仕州縣,邸狀報包拯參知政事,或云朝廷自此多事矣。師中曰:「包公何能為,今鄞縣王安石者,眼多白,甚似王敦,他日亂天下,必斯人也。」後二十年,言乃信。



 其志尚甚高,每進見,多陳天人之際、君臣大節,請以進賢退
 不肖為宰相考課法。在官不貴威罰,務以信服人,至明而恕。去之日,民擁道遮泣,馬不得行。杜衍、範仲淹、富弼皆薦其有王佐才。然好為大言,以故不容於時而屢黜,氣未嘗少衰。



 陸詵,字介夫,餘杭人。進士起家,簽書北京判官。貝州亂,給事不乏興;賊平,又條治其獄,無濫者。加集賢校理、通判秦州。範祥城古渭,詵主饋餉,具言:「非中國所恃,而勞師屯戍,且生事。」既而諸羌果怒爭,塞下大擾,經二歲乃
 定。



 判太常禮院、吏部南曹,提黠開封縣鎮。咸平龍騎軍皆故群盜,牢廩不時得,毆蒞給官,還營不自安,大校柴元煽使亂。詔詵往視,許元以不死,命取始禍者自贖,眾皆帖然。



 提點陜西刑獄。時鑄錢法壞,議者欲變大錢當一,詵言:「民間素重小銅錢而賤大鐵錢,他日以一當三猶輕之,今減令均直,大錢必廢。請以一當二,則公私所損亡幾,而商賈可以通行;兼盜鑄者計其直無贏,將必自止。」從之。



 徙湖南、北轉運使,直集英院,進集賢殿修撰、
 知桂州。奏言:「邕去桂十八驛,異時經略使未嘗行飭武備,臣願得一往,使群蠻知省大將號令,因以聲震南交。」詔可。自儂徭定後,交人浸驕,守帥常姑息。詵至部,其使者黎順宗來,偃蹇如故態。詵絀其禮,召問折諭,導以所當為,懾伏而去。詵遂至邕州,集左、右江四十五峒首詣麾下,閱簡工丁五萬,補置將吏,更鑄印給之,軍聲益張。交人滋益恭,遣使入貢。召為天章閣待制、知諫院,命張田代之,英宗戒以毋得改詵法。



 道除知延州,趣入覲,帝
 勞之曰:「卿在嶺外,施設無不當者。鄜延最當敵要,今將何先?」對曰:「邊事難以隃度,未審陛下欲安靜邪,將威之也?」帝曰:「大抵邊陲當安靜。昨王素為朕言,惟朝廷與帥臣意如此;至如諸將,無不貪功生事者。卿謂何如?」詵曰:「素言是也。」諒祚寇慶州,以敗還,聲言益發人騎,且出嫚辭,復攻圍大順城。詵謂由積習致然,不稍加折誚,則國威不立。乃留止請時服使者及歲賜,而移宥州問故。帝喜曰:「固知詵能辦此。」諒祚聞之大沮,盤旋不敢入,乃報
 言:「邊吏擅興兵,今誅之矣。」朝廷遣何次公持詔書諭告,詵以為未可。明年,又乞留賜冬服及大行遺留二使,而自以帥牒告之故。諒祚始因詵謝罪,共貢職。



 銀州監軍嵬名山與其國隙,扣青澗城主種諤求內附,諤以狀聞,遂欲因取河南地。詵曰:「數萬之眾納土容可受,若但以眾來,情偽未可知,且安所置之。」戒諤毋妄動。諤持之力,詔詵召諤問狀,與轉運使薛向議撫納。詵、向言:「名山誠能據橫山以捍敵,我以刺史世封之,使自為守,故為中
 國之利。今無益我而輕啟西□,非計也。」乃共畫三策,令幕府張穆之入奏,而穆之陰受向指,詭言必可成。神宗意詵不協力,徙知秦鳳。諤遂發兵取綏州,詵欲理諤不稟節制之狀,未及而徙。詵馳見帝,請棄綏州而上諤罪,帝愈不懌,罷知晉州。既諤抵罪,向、穆之皆坐貶,以詵知真定,改龍圖閣學士、知成都。



 青苗法出,詵言:「蜀峽刀耕火種,民常不足。今省稅科折已重,其民輕侈不為儲積,脫歲儉不能償逋,適陷之死地,願罷四路使者。」詔獨置
 成都府一路。熙寧三年,卒,年五十九。子師閔。



 師閔以父任為官。熙寧末,李稷提舉成都路茶場,闢乾當公事;不三年,提舉本路常平,遂居稷職。在蜀茶額三十萬,稷既增而五之,師閔又衍為百萬。稷死,師閔訟其前功,乞賜之土田。詔賜稷十頃,進師閔都大提舉成都、永興路榷茶,位視轉運使。又兼買馬、監牧,事權震川,建請無不遂志,所行職事,他司莫預聞。茶禍既被於秦、蜀,又欲延荊、楚、兩河,神宗不許。元祐初,用御史中丞劉摯
 言,遣黃廉入蜀訪察。右司諫蘇轍論其六害,謂:「李稷引師閔共事,增額置場,以金銀貨拘民間物折博,賤取而貴出之,其害過於市易。自法始行,至今四變,利益深,民益困。立法之虐,未有甚於此者。」廉奏至,如轍所陳。乃貶師閔主管東嶽廟。



 久之,起知蘄州。會復置常平官,李清臣在中書,即以師閔使河北。尋加直秘閣,復領秦、蜀茶事,於是一切如初。又使掾屬詣闕奏券馬事,安壽、韓忠彥議頗異,獨曾布以為然,曰:「但行之一年,而以較綱馬,
 利害即可見矣。」師閔遂詳令蕃漢商人願持馬受券者,於熙、秦兩路印驗價給之,而請直於太僕,若此券盛行,則買馬場可罷。既用其策,明年,太僕會綱馬之籍,死者至什二,而券馬所損才百分一。詔獎之,賜以金帛。改陜西轉運使,加集賢殿修撰、知秦州。



 諸道方進築被爵賞,師閔在秦無所事,怏怏不釋。曾布議使督本部兵赴熙河共攻,師閔承命踴躍,集兵四萬以待。而章惇陰諷熙帥鐘傳先出塞,敕師閔聽傳節制,築淺井,又築□□羅,皆
 不成而還。傳更檄會兵於耳關,未至復卻。秦鳳之師再出再返,勞且弊,言者乞加責,不聽。



 旋進寶文閣待制,召為戶部侍郎。未及拜,坐秦州詐增首虜事,落職知鄞。未幾,還之。歷河南、永興軍、延安府,卒。



 趙離,字公才,邛州依政人。第進士,為汾州司法參軍。郭逵宣撫陜西,闢掌機宜文字。種諤擅納綏州降人數萬,朝廷以其生事,議誅諤,反故地歸降人,以解仇釋兵。離上疏曰:「諤無名興舉,死有餘責。若將改而還之,彼能聽
 順而亡絕約之心乎?不若諭以彼眾餓莩,投死中國,邊臣雖擅納,實無所利,特以往年俘我蘇立、景詢輩爾。可遣詢等來,與降人交歸,各遵紀律,而疆場寧矣。如其蔽而不遣,則我留橫山之眾,未為失也。



 又徙逵帥鄜延,為逵移書執政,請存綏州以張兵勢,先規度大理河川,建堡砦,畫稼穡之地三十里,以處降者。若棄綏不守,則無以安新附之眾。援種世衡招蕃兵部敵屯青澗城故事。朝廷從之,活降人數萬,為東路捍蔽。熙寧初,夏人誘殺
 知保安軍楊定等,既而以李崇貴、韓道喜來獻,且請和。朝廷欲官其任事之酋,鐫歲賜以為俸給,因使納塞門、安遠二砦而還綏州。離言:「綏實形勢之地,宜增廣邊障,乃無窮之利。若存綏以觀其變,計之得也。」神宗召問狀,對曰:「綏之存亡,皆不免用兵。降二萬人入吾肝脾,□隙已深,不可亡備。」神宗然之。除集賢校理。



 夏人犯環慶,後復來賀正。離請邊吏離其心腹,因以招橫山之眾,此不戰而屈人兵也。遷提點陜西刑獄。韓絳宣撫陜西,河東
 兵西討,離為絳言:「大兵過山界,皆砂磧,乏善水草,又亡險隘可以控扼,今切危之。若乘兵威招誘山界人戶,處之生地,當先經畫山界控扼之地,然後招降;不爾,勞師遠攻,未見其利。」絳欲取橫山,納種諤之策,遂城囉兀,以離權宣撫判官。諤趣河東兵會銀川,規以後期斬將。離白絳,令諤自往中路迎東兵。諤懼違節制,乃不敢逞。加直龍圖閣、知延州。



 夏人屢欲款塞,每以虛聲搖邊。詔問方略,離審計形勢,為破敵之策以獻。遣裨將曲珍、呂真
 以兵千人分巡東西路。夏人方以四萬眾自間道欲取綏,道遇珍,皇駭亟戰,真繼至,夏眾敗走。夏自失綏,意未能已。離揣知其情,奏言:「夏使請和,必欲畫綏界,願聽本路經略司分畫;歲賜,則俟通和之日復焉。」明年,遂用離策,以綏為綏德城。



 初,鄜延地皆荒瘠,占田者不出租賦,倚為藩蔽。寶元用兵後,凋耗殆盡,其曠土為諸酋所有。離因招問曰:「往時汝族戶若干,今皆安在?」對:「大兵之後,死亡流散,其所存止此。」離曰:「其地存乎?」酋無以對。離曰:「
 聽汝自募丁,家使占田充兵,若何?吾所得者人爾,田則吾不問也。」諸酋皆感服歸募,悉補亡籍。又檢括境內公私閑田,得七千五百餘頃,募騎兵萬七千。離以異時蕃兵提空簿,漫不可考,因議涅其手。屬歲饑,令蕃兵願刺手者,貸常平谷一斛,於是人人願刺,因訓練以時,精銳過於正兵。神宗聞而嘉之,擢天章閣待制。



 交址叛,詔為安南行營經略、招討使,總九將軍討之,以中官李憲為貳。離與議不合,請罷憲。神宗問可代者,離以郭逵老
 邊事,願為裨贊,於是以逵為宣撫使,離副之。逵至,輒與離異:離欲乘兵形未動,先撫輯兩江峒丁,擇壯勇啖以利,使招徠攜貳,隳其腹心,然後以大兵繼之,逵不聽;離又欲使人繼敕榜入賊中招納,又不聽。遂令燕達先被廣源,復還永平。離以為廣源間道距交州十二驛,趣利掩擊,出其不意,川途並進,三路致討,勢必分潰,固爭不能得。賊乘緩遂據江列戰艦數百艘,官軍不能濟。離分遣將吏伐木治攻具,機石如雨,其艦被擊皆廢。徐以罷
 卒致賊,設伏擊之,斬首數千級,馘其渠酋,遂皆降。逵怍於玩寇,乃移疾先還。逵既坐貶,離亦以不即平賊,降為直龍圖閣、知桂州。後復天章閣待制、權三司使。



 時西師大舉,五路並進,以離為河東轉運使,領降卒赴鄜延餉種諤軍。諤抵罪,離又坐饋挽不給,黜知相州。既而鐫職知淮陽軍,居數月,盡復故職。



 知慶州。羌□移名昌詭稱送幣,將入寇,離知蕃主白信可使,信適以罪系獄。破械出之,告以其故,約期日使往,果縛取以歸。明年,夏人欲襲取
 新壘,大治攻械。離具上撓夏計。及夏侵蘭州,離遣曲珍將兵直抵鹽韋,俘馘千,驅孳畜五千。其酋枻厥嵬名宿兵於賀蘭原,時出攻邊,離遣將李照甫、蕃官歸仁各將兵三千左右分擊,耿端彥兵四千趍賀蘭原,戒端彥曰:「賀蘭險要,過嶺,則砂磧也。使敵入平夏,無繇破之。」又選三蕃官各輕兵五百,取間道出敵砦後,邀其歸路。端彥與戰賀羅平,敵敗,果趍平夏。千兵伏發,敵駭潰,斬馘甚眾,生擒嵬名,斬首領六,獲戰馬七百,牛羊、老幼三萬餘。
 遷龍圖閣直學士,復帥延安。



 元祐初,梁乙埋數擾邊,離知夏將入侵,檄西路將劉安、李儀曰:「夏即犯塞門,汝徑以輕兵搗其腹心。」後果來犯,安等襲洪州,俘斬甚眾,夏遂入貢。既而以重兵壓境,諸將亟請益戍兵為備,離徐諭之曰:「第謹斥堠、整戈甲,無為寇先,戍兵不可益也。」因遣人詰夏,夏兵遂去。遷樞密直學士。



 乙埋終不悛。使間以善意問乙埋:「何苦與漢為仇。必欲寇,第數來,恐汝所得不能償所亡,洪州是也。能改之,吾善遇汝。」遺之戰袍、
 錦彩,自是乙埋不復窺塞。離乃縱間,國中疑而殺之。



 五年,拜端明殿學士,遷太中大夫。夏遣使以地界為請,朝廷許還葭蘆、米脂、浮屠、安疆四砦,以離領分畫之議。夏既得四砦,猶未有恭順意,未幾復犯涇原。會離卒、年六十五,贈右光祿大夫。紹聖四年,以離與元祐棄地議,系其名於黨籍。



 孫路字正甫,開封人。進士及第。元豐中,為司農丞。鄧潤甫薦為御史,召對,其言不合新政,神宗語輔臣以為不
 可用,下遷主簿。路鞅鞅不釋,求通判河州,徙蘭州。夏人入寇,論捍禦功,進五階,除陜西轉運判官。



 元祐初,為吏部、禮部員外郎,侍講徐王府。司馬光將棄河、湟,邢恕謂光曰:「此非細事,當訪之邊人,孫路在彼四年,其行止足信,可問也。」光亟召問,路挾輿地圖標光曰:「自通遠至熙州才通一徑,熙之北已接夏境,今自北關闢土百八十里,瀕大河,城蘭州,然後可以捍蔽。若捐以予敵,一道危矣。」光幡然曰:「賴以訪君,不然幾誤國事。」議遂止。



 遷右司
 郎中,以直龍圖閣知慶州。章惇柄國,復議取棄地。時諸道相視未進,路聲言修舊壘,載器甲樓鹵,頓大順城下,夜半趍安疆,遲明據之,六日而城完。加寶文閣待制,遂築興平、橫山。進龍圖閣直學士,徙知熙州。



 涇原城西安,詔出師牽制其勢。路即將眾臨會州,遂建取青唐之策。大將王愍、王贍搗邈川,贍先至,下之。愍與爭功,路右愍,顓屬以兵;贍有請,輒弗應。贍訴諸朝,召拜路兵部尚書,以龍圖閣學士知成都。未行,坐他事削職,知興國軍。徽
 宗立,歷太原、河南、永興軍、河中府,卒。



 游師雄,字景叔,京兆武功人。學於張載,第進士。為儀州司戶參軍,遷德順軍判官。鄜延將劉管與主帥議戰守策,欲自延安入安定、黑水,師雄以地薄賊境,懼有伏,請由他道。既而諜者言夏伏精騎於黑水傍,管謝曰:「微君言,吾不返矣。」



 趙離帥延安,闢為屬。時夏人擾邊,戍兵在別堡,龍安以北諸城兵力咸弱,離患之。師雄請發義勇以守,多聚石城上,待其至。夏人知有備,不敢入,但襲荒
 堆、三泉而還。歲饑,行諸壘振貸,計口賦糧,人無殍亡。運石瑩甲,深溝繕城,邊備益固。



 元祐初,為宗正寺主簿。執政將棄四砦,訪於師雄。師雄曰:「此先帝所立,以控制夏人者也,若何棄之,不惟示中國之怯,將起敵人無厭之求。儻瀘、戎、荊、奧視以為請,亦將與之乎?萬一燕人遣一乘之使,來求關南十縣,為之奈何?」不聽。因著《分疆錄》。遷軍器監丞。



 吐蕃寇邊,其酋鬼章青宜結乘間脅屬羌構夏人為亂,謀分據熙河。朝廷擇可使者與邊臣措置,詔
 師雄行,聽便宜從事。既至,諜知夏人聚兵天都山,前鋒屯通遠境。吐蕃將攻河州,師雄欲先發以制之,請於帥劉舜卿。舜卿曰:「彼眾我寡,奈何?」師雄曰:「在謀不在眾。脫事不濟,甘受首戮。」議三日乃定,遂分兵為二,姚兕將而左,種誼將而右。兕破六逋宗城,斬首千五百級,攻講朱城,斷黃河飛梁,青唐十萬眾不得度。誼破洮州,擒鬼章及大首領九人,斬首千七百級。捷書聞,百僚表賀,遣使告永裕陵。將厚嘗師雄,言者猶以為邀功生事,止遷一
 官,為陜西轉運判官、提點秦鳳路刑獄。



 夏人侵涇原,復入熙河,師雄言:「蘭州距賊一舍,通遠不百里,非有重山復嶺之阻。宜於定西、通渭之間建安遮、納迷、結珠三柵,及護耕七堡,以固藩籬,此無窮之利也。」詔付範育,皆如初議。



 入拜祠部員外郎,加集賢校理,為陜西轉運使。內地移粟於邊,民以輦僦為病。師雄言:「往者邊土不耕,仰給於內,今積粟已多,軍食自足,宜令內地量轉輸致之直,以免大費。」報可。召詣闕,哲宗勞之曰:「洮州之役,可謂
 雋功,但恨賞太薄耳。」對曰:「皆上稟廟算,臣何力之有焉。唯當時將士勛勞未錄,此為欠也。」因陳其本末。拜衛尉少卿。哲宗數訪邊防利病,師雄具慶歷以來邊臣施置之臧否,朝廷謀議之得失,及方今御敵之要,凡六十事,名曰《紹聖安邊策》,上之。



 出知邠州,改河中府,進直龍圖閣、知秦州,未至,詔攝熙州。以夏人擾邊,詔使者與熙帥、秦帥共謀之。使者銳於討擊,師雄謂:「進築城壘以自蔽,席卷之師未應深入也。」上章爭之,不報。既而使者知攻
 取之難,卒用師雄策。



 自復洮州之後,於闐、大食、物林、邈黎諸國皆懼,悉遣使入貢。朝廷令熙河限其二歲一進。師雄曰:「如此,非所以來遠人也。」未幾還秦,徙知陜州。卒,年六十。師雄慷慨豪邁,有志事功,議者以用不盡其材為恨。



 穆衍,字昌叔,河內人,徙河中。第進士,調華池令。民牛為仇家斷舌而不知何人,訟於縣,衍命殺之。明日,仇以私殺告,衍曰:「斷牛舌者乃汝耶?」訊之具服。



 後知淳化,耀之
 屬縣。衍從韓絳宣撫陜西,遇慶卒潰亂,衍念母在耀,亟謁歸,信宿走七驛。比至,慶卒嘗戍華池,知衍名,不敢近。時諸郡捕賊兵糧□無以給,遂擅發常平倉,且懼得罪。衍曰:「饑之不恤,則吾丘將為慶卒矣。」衍考課為一路最。元豐中,種諤西征,參其軍事。諤第賞,以死事為下。衍曰:「此非所以勸忠也。」力爭之。諤還入塞,詔往靈武援渭、慶兩軍。將行,衍曰:「吾兵惰,歸未及解甲,安能犯不測於千里外哉?」諤乃止。同幕畏罪,陽謝衍曰:「師不再舉,君之力
 也。」衍識其意,曰:「全萬眾之命,以一身塞責,衍無憾焉。」



 元祐初,大臣議棄熙、蘭,衍與孫路論疆事,以為「蘭棄則熙危,熙棄則關中震。唐自失河、湟,西邊一有不順,則警及京都。今二百餘年,非先帝英武,孰能克復。若一旦委之,恐後患益前,悔將無及矣」。議遂止。改陜西轉運判官,金部、戶部員外郎。熙河分畫未決,詔衍視之。還言:「質孤、勝如據兩川美田,實彼我必爭之地,自西關失利,遂廢不守。請界二壘之間,城李諾平以控要害,及他城堡皆起
 亭障,以通涇原。」明年,遂城李諾,名曰定遠。三遷左司郎中。



 紹聖初,以直秘閣為陜西轉運使,加直龍圖閣、知慶州,徙延安,又徙秦州,未行而卒。年六十三。敕河中官庀其葬,後追錄不棄蘭州議,官其一子。



 論曰:自熙寧至於紹聖,四方之事多矣。夏人乍服乍叛,其地或予或奪,廟堂之上,論靡有定,相為短長,元發、師中輩七人,一時謀謨,蓋可考也。元發論君子小人,言簡而盡,足動人主,而神宗惑安石之言,竟弗之悟。師中豫
 識安石于鄞令,以為目肖王敦,將亂天下,蓋又先於呂誨矣。詵能鎮撫西夏,又能靖交址之難,誠有御邊之才;其子師閔為時籠利,無足取者。趙離狃於西陲之勝,取敗南裔,後獲嵬名,庶足自贖。朝臣議棄河、湟,孫路以一言止之,使司馬光自悔幾於誤國;及取青唐,下邈川,可驗其能,然右王愍而困王贍,非大將之器也。游師雄之禽鬼章,復洮州,以致諸國入貢,校之諸將,其功獨為雋偉。衍為政得民心,既去而亂兵不忍驚其母,德之足以
 感人,有如是夫。



\end{pinyinscope}