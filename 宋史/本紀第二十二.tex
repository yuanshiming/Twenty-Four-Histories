\article{本紀第二十二}

\begin{pinyinscope}

 徽宗四



 宣和元年春正月戊申朔,日下有五色雲。壬子,進建安郡王樞為肅王,文安郡王杞為景王,並為太保。乙卯,詔:「佛改號大覺金仙,餘為仙人、大士。僧為德士,易服飾,稱
 姓氏。寺為宮,院為觀。」改女冠為女道,尼為女德。丁巳,金人使李善慶來,遣趙有開報聘,至登州而還。戊午,以餘深為太宰兼門下侍郎,王黼為特進、少宰兼中書侍郎。乙丑,改湟州為樂州。癸酉,封子棟為溫國公,侄有恭為永寧郡王。乙亥,躬耕籍田。罷裕民局。



 二月庚辰,改元。易宣和殿為保和殿。戊戌,以鄧洵武為少保。



 三月庚戌,蔡京等進安州所得商六鼎。己未,以馮熙載為中書侍郎,範致虛為尚書左丞,翰林學士張邦昌為尚書右丞。詔
 天下知宮觀道士與監司、郡縣官以客禮相見。童貫遣知熙州劉法出師攻統安城,夏人伏兵擊之,法敗歿,震武軍受圍。甲子,知登州宗澤坐建神霄宮不虔,除名編管。辛未,賜上舍生五十四人及第。甲戌,皇后親蠶。



 夏四月丙子朔,日有食之。庚寅,童貫以鄜延、環慶兵大破夏人,平其三城。己亥,曲赦陜西、河東路。辛丑,進輔臣官一等。



 五月丙午朔,有物如龍形,見京師民家。丁未,詔德士並許入道學,依道士法。丙辰,敗夏人於震武。壬申,班禦
 制《九星二十八宿朝元冠服圖》。甲戌,慮囚。是月,大水犯都城,西北有赤氣亙天。



 六月壬午,詔西邊武臣為經略使者改用文臣。甲申,詔封莊周為微妙元通真君,列禦寇為致虛觀妙真君,仍行冊命,配享混元皇帝。己亥,夏國遣使納款,詔六路罷兵。



 秋七月甲寅,以童貫為太傅。



 八月戊寅,詔諸路未方田處並令方量,均定租課。丁酉,以神霄宮成,降德音於天下。範致虛以母憂去位。



 九月甲辰朔,燕蔡京於保和新殿。辛酉,大饗明堂。癸亥,幸道
 德院觀金芝,遂幸蔡京第。丁卯,以淮康軍節度使蔡攸為開府儀同三司。



 冬十月甲戌朔,以《紹述熙豐政事書》布告天下。



 十一月癸丑,朝獻景靈宮。甲寅,饗太廟。乙卯,祀昊天上帝於圜丘,赦天下。甲子,詔東南諸路水災,令監司、郡守悉心振救。戊辰,以淮甸旱,饑民失業,遣監察御史察訪。張邦昌為尚書左丞,翰林學士王安中為尚書右丞。時朱勉以花石綱媚上,東南騷動,太學生鄧肅進詩諷諫,詔放歸田里。十二月甲戌,詔京東東路盜賊竊
 發,令東、西路提刑督捕之。辛卯,大雨雹。丙申,帝數微行,正字曹輔上書極論之,編管郴州。是歲,京西饑,淮東大旱,遣官振濟。嵐州黃河清。升邢州為信德,陳州為淮寧,襄州為襄陽,慶州為慶陽,安州為德安,鄆州為東平,趙州為慶源府;瀘州為瀘川,睦州為建德,岳州為岳陽,寧州為興寧,宜州為慶遠,光州為光山,均州為武當軍。



 二年春正月癸亥,追封蔡確為汝南郡王。甲子,罷道學。



 二月乙亥,遣趙良嗣使金國。唐恪罷。庚辰,以寧遠軍節
 度使梁子美為開府儀同三司。戊子,令所在贍給淮南流民,諭還之。甲午,詔別修《哲宗史》。



 三月壬寅,賜上舍生二十一人及第。乙卯,改熙河蘭湟路為熙河蘭廓路。



 夏四月丙子,詔江西、廣東兩界群盜嘯聚,添置武臣提刑,路分都監各一員。



 五月庚子朔,以淑妃劉氏為貴妃。己酉,日中有黑子。丁巳,祭地於方澤,降德音於諸路。布衣朱夢說上書論宦寺權太重,編管池州。戊辰,詔宗室有文行才術者,令大宗正司以聞。六月癸酉,詔開封府振
 濟饑民。丁丑,太白晝見。戊寅,蔡京致仕,仍朝朔望。辛巳,詔自今動改元豐法制,論以大不恭。丙戌,詔三省、樞密院額外吏職,並從裁汰。及有妄言惑眾、稽違詔令者,重論之。詔諸司總轄、提點之類,非元豐法並罷。丁亥,復寺院額。甲午,罷禮制局並修書五十八所。



 秋七月壬子,罷文臣起復。己未,罷醫、算學。丙寅,封子楒為英國公。



 八月庚辰,詔減定醫官額。乙未,詔監司所舉守令非其人,或廢法不舉,令廉訪使者劾之。



 九月壬寅,金人遣勃堇等
 來。乙巳,復德士為僧。辛亥,大饗明堂。丙辰,遣馬政使金國。癸亥,餘深加少傅。宴童貫第。



 冬十月戊辰朔,日有食之。以河東節度使梁師成為太尉。建德軍青溪妖賊方臘反,命譚稹討之。



 十一月己亥,餘深罷,仍少傅,授鎮西軍節度使、知福州。庚戌,以王黼為少保、太宰兼門下侍郎。己未,兩浙都監蔡遵、顏坦擊方臘,死之。十二月丁亥,改譚稹為兩浙制置使,以童貫為江、淮、荊、浙宣撫使,討方臘。己丑,以少傅鄭居中權領樞密院。庚寅,詔訪兩浙
 民疾苦。是月,方臘陷建德,又陷歙州,東南將郭師中戰死。陷杭州,知州趙霆遁,廉訪使者趙約詬賊死。是歲,淮南旱。夏國、真臘入貢。



 三年春正月壬寅,鄧洵武卒。戊午,以安康郡王栩為太保,進封濟王;鎮國公模為開府儀同三司,進封樂安郡王。己未,詔淮南、江東、福建各權添置武臣提刑一員。辛酉,罷蘇、杭州造作局及御前綱運。乙丑,罷西北兵更戌。罷木石彩色等場務。是月,方臘陷婺州,又陷衢州,守臣
 彭汝方死之。



 二月庚午,趙霆坐棄杭州,貶吉陽軍。罷方田。甲戌,降詔招撫方臘。乙酉,罷天下三舍及宗學、闢雍、諸路提舉學事官。癸巳,赦天下。是月,方臘陷處州。淮南盜宋江等犯淮陽軍,遣將討捕,又犯京東、河北,入楚、海州界,命知州張叔夜招降之。



 三月丁未,御集英殿策進士。庚申,賜禮部奏名進士及第、出身六百三十人。



 夏四月丙寅,貴妃劉氏薨。甲戌,青溪令陳光以盜發縣內棄城,伏誅。庚寅,忠州防禦使辛興宗擒方臘於青溪。詔二
 浙、江東被賊州縣給復三年。癸巳,汝州牛生麒麟。



 五月戊戌,以鄭居中領樞密院。己亥,詔杭、越、江寧守臣並帶安撫使。甲辰,追冊貴妃劉氏為皇后,謚曰明節。改睦州、建德軍為嚴州、遂安軍,歙州為徽州。丙午,金人再遣曷魯等來。戊申,以興寧軍節度使劉宗元為開府儀同三司。癸亥,詔三省覺察臺諫罔上背公者,取旨譴責。陳過庭、張汝霖以乞罷御前使喚及歲進花果,為王黼所劾,並竄貶。



 閏月丙寅,減諸州曹掾官。辛未,立醫官額。甲戌,
 復應奉司,命王黼及內侍梁師成領之。戊寅,慮囚。



 六月,河決恩州清河埽。



 秋七月丁卯,振溫、處等八州。丁亥,廢純、滋等十二州。戊子,童貫等俘方臘以獻。是月,洛陽、京畿訛言有黑眚如人,或如犬,夜出掠小兒食之,二歲乃息。



 八月甲辰,曲赦兩浙、江東、福建、淮南路。乙巳,以童貫為太師,譚稹加節度。丁未,祔明節皇后神主於別廟。丙辰,方臘伏誅。



 九月丙寅,以王黼為少傅,鄭居中為少師。庚午,進執政官一等。辛未,大饗明堂。



 冬十月甲寅,詔自
 今贓吏獄具,論決勿貨。童貫復領陜西、兩河宣撫。



 十一月丁丑,馮熙載罷。以張邦昌為中書侍郎,王安中為尚書左丞,翰林學士承旨李邦彥為尚書右丞。辛巳,封子桐為儀國公。壬午,張商英卒。十二月辛卯朔,日中有黑子。壬子,進封廣平郡王構為康王,樂安郡王模為祁王,並為太保。是歲,諸路蝗。



 四年春正月丁卯,以蔡攸為少保,梁師成為開府儀同三司。癸酉,金人破遼中京,遼主北走。



 二月丙申,以旱
 禱於廣聖宮,即日雨。癸卯,雨雹。丙午,以吳國公植為開府儀同三司,進封信都郡王。



 三月辛酉,幸秘書省,遂幸太學,賜秘書少監翁彥深、王時雍、國子祭酒韋壽隆、司業權邦彥章服,館職、學官、諸生恩錫有差。丙子,遼人立燕王淳為帝。金人來約夾攻,命童貫為河北、河東路宣撫使,屯兵於邊以應之,且招諭幽、燕。



 夏四月丙午,詔置補完校正文籍局,錄三館書置宣和樓及太清樓、秘閣。又令郡縣訪遺書。五月壬戌,以高俅為開府儀同三司。丁
 卯,封子柄為昌國公。甲戌,嗣濮王仲御薨。乙亥,以蔡攸為河北、河東宣撫副使。庚辰,以常德軍節度使譚稹為太尉。童貫至雄州,令都統制種師道等分道進兵。癸未,遼人擊敗前軍統制楊可世於蘭溝甸。乙酉,封開府儀同三司、江夏郡王仲爰為嗣濮王。丙戌,慮囚。楊可世與遼將蕭乾戰於白溝,敗績。丁亥,辛興宗敗於範村。



 六月己丑,種師道退保雄州,遼人追擊至城下。帝聞兵敗,懼甚,遂詔班師。壬寅,以王黼為少師。是月,遼燕王淳死,蕭
 乾等立其妻蕭氏。



 秋七月己未,廢貴妃崔氏為庶人。壬午,王黼以耶律淳死,復命童貫、蔡攸治兵,以河陽三城節度使劉延慶為都統制。甲申,種師道責授右衛將軍致仕,和詵散官安置。



 九月戊午,朝散郎宋昭上書諫北伐,王黼大惡之,詔除名勒停、廣南編管。己未,金人遣徒孤且烏歇等來議師期。辛酉,大饗明堂。己巳,高麗國王王俁薨,遣路允迪吊祭。甲戌,遣趙良嗣報聘於金國。己卯,遼將郭藥師以涿、易二州來降。



 冬十月庚寅,改燕
 京為燕山府,涿、易八州並賜名。癸巳,劉延慶與郭藥師等統兵出雄州。戊戌,曲赦所復州縣。己亥,耶律淳妻蕭氏上表稱臣納款。甲辰,師次涿州。己酉,郭藥師與高世宣、楊可世等襲燕,蕭乾以兵入援,戰於城中,藥師等屢敗,皆棄馬縋城而出,死傷過半。癸丑,以蔡攸為少傅、判燕山府。甲寅,劉延慶自盧溝河燒營夜遁,眾軍遂潰,蕭乾追至涿水上乃還。



 十一月丙辰朔,行新璽。戊辰,朝獻景靈宮。己巳,饗太廟。庚午,祀昊天上帝於園丘,赦天下。
 東南官吏昨緣寇盜貶責者,並次第移放,上書邪上等人特與磨勘。戊寅,金人遣李靖等來許山前六州。以彰德軍節度使鄭詳為太尉。十二月丁亥,郭藥師敗蕭乾於永清縣。戊子,遣趙良嗣報聘於金國。庚寅,以郭藥師為武泰軍節度使。辛卯,金人入燕,蕭氏出奔。壬辰,使來獻捷。乙未,詔監司未經陛對,毋得之任。丙申,貶劉延慶為率府率、安置筠州。壬寅,進封植為莘王。



 五年春正月戊午,金人遣李靖來議所許六州代租錢。
 己未,遣趙良嗣報聘,求西京等州。辛酉,以王安中為慶遠軍節度使、河北河東燕山府路宣撫使、知燕山府。甲申,錄富弼後。



 二月乙酉朔,以李邦彥為尚書左丞,翰林學士趙野為尚書右丞。丙戌,金人以議未合,斷橋梁,焚次舍。丁酉,進封雍國公樸為華原郡王,徐國公棣為高平郡王,並為開府儀同三司。三月乙卯,金人再遣寧術割等來。己未,遣盧益報聘,皆如其約。



 夏四月癸巳,金人遣楊璞以誓書及燕京、涿、易、檀、順、景、薊州來歸。庚子,童
 貫、蔡攸入燕,時燕之職官、富民、金帛、子女先為金人盡掠而去。乙巳,童貫表奏撫定燕城。庚戌,曲赦河北、河東、燕雲路。是日班師。



 五月己未,以收復燕、雲,賜王黼玉帶。庚申,以王黼為太傅,鄭居中為太保,進宰執官二等。辛酉,王黼總治三省事。癸亥,童貫落節鉞,進封徐、豫國公。蔡攸為少師。乙丑,詔正位三公立本班,帶節鉞若領他職者仍舊班,著為令。癸酉,祭地於方澤。是月,金人許朔、武、蔚三州。金主阿骨打殂,弟吳乞買立。



 六月乙酉,郭藥
 師加檢校少傅。丙戌,遼人張覺以平州來附。己丑,仲爰薨。乙未,詔今後內外宗室並不稱姓。丁酉,以安國軍節度使仲理為開府儀同三司,進封嗣濮王。己亥,慮囚。戊申,鄭居中卒。辛亥,以蔡攸領樞密院。



 秋七月戊午,以梁師成為少保。己未,童貫致仕。起復譚稹為河北、河東、燕山府路宣撫使。庚午,太傅、楚國公王黼等上尊號曰繼天興道敷文成武睿明皇帝,不允。禁元祐學術。



 八月辛巳朔,日當食不見。辛丑,命王安中作《復燕雲碑》。壬寅,太
 白晝見。是月,蕭乾破景州、薊州,寇掠燕山,郭藥師敗之。乾尋為其下所殺,傳首京師。



 九月辛酉,大饗明堂。



 冬十月乙酉,雨木冰。壬寅,罷諸路提舉常平之不職者。



 十一月乙卯,以鄭紳為太師。丙寅,幸王黼第觀芝。諸路漕臣坐上供錢物不足,貶秩者二十二人。丁卯,王安中、譚稹並加檢校少傅,郭藥師為太尉。華原郡王樸薨。壬申,王黼子弟親屬推恩有差。是月,金人取平州,張覺走燕山,金人索之甚急,命王安中縊殺,函其首送之。十二月乙
 巳,金人遣高居慶等來賀正旦。戊申,以高平郡王棣為太保,進封徐王。是歲,秦鳳旱,河北、京東、淮南饑,遣官振濟。



 六年春正月乙卯,為金主輟朝。戊午,置書藝所。癸亥,藏蕭乾首於太社。戊寅,遣連南夫吊祭金國。



 二月丁亥,以冀國公□咢為開府儀同三司,進封河間郡王;韶州防禦使令蕩為婺州觀察使,封安定郡王。己亥,躬耕藉田。丙午,詔自今非歷臺閣、寺監、監司、郡守、開封府曹官者,不
 得為郎官、卿、監,著為令。李邦彥以父憂去位。



 三月己酉朔,以錢景臻為少師。金人來丐糧,不與。



 閏月辛巳,皇后親蠶。庚子,御集英殿策進士。



 夏四月癸丑,賜禮部奏名進士及第、出身八百五人。丁巳,李邦彥起復。



 五月壬寅,慮囚。癸卯,金人遣使來告嗣位。



 六月壬子,詔以收復燕、雲以來,京東、兩河之民困於調度,令京西、淮、浙、江、湖、四川、閩、廣並納免夫錢,期以兩月納足,違者從軍法。



 秋七月戊子,遣許亢宗賀金國嗣位。丁酉,詔:應系御筆斷罪,
 不許詣尚書省陳訴改正。壬寅,詔宗室、后妃戚里、宰執之家概敷免夫錢。甲辰,置璣衡所。



 八月乙卯,譚稹落太尉、罷宣撫使,童貫落致仕,領樞密院代之。丁巳,以溢機堡為安羌城。壬戌,以復燕、雲,赦天下。



 九月乙亥,以白時中為特進、太宰兼門下侍郎,李邦彥為少宰兼中書侍郎。蔡攸落節鉞。辛巳,大饗明堂。丁亥,以趙野為尚書左丞,翰林學士承旨宇文粹中為尚書右丞,開封尹蔡懋同知樞密院。庚寅,以金芝產於艮嶽萬壽峰,改名壽嶽。
 庚子,金人遣富謨弼等以遺留物來獻。



 冬十月庚午,詔有收藏習用蘇、黃之文者,並令焚毀,犯者以大不恭論。癸酉,詔內外官並以三年為任,治績著聞者再任,永為式。



 十一月丙子,王黼致仕。太白晝見。乙酉,罷應奉司。丙戌,令尚書省置講議局。壬辰,詔監司擇縣令有治績者保奏,召赴都堂審察錄用,毋過三人。十二月甲辰朔,蔡京領講議司。詔百官遵行元豐法制。丁未,詔內外侍從以上各舉所知二人。癸亥,蔡京落致仕,領三省事。是歲,
 河北、山東盜起,命內侍梁方平討之。京師、河東、陜西地大震,兩河、京東西、浙西水,環慶、邠寧、涇原流徙,令所在振恤。夏國、高麗、于闐、羅殿入貢。



 七年春正月癸酉朔,詔赦兩河、京西流民為盜者,仍給復一年。癸巳,詔罷諸路提舉常平官屬,有罪當黜者以名聞,仍令三省修已廢之法。



 二月甲辰,復置鑄錢監。詔御史察贓吏。己酉,雨木冰。庚戌,詔京師運米五十萬斛至燕山,令工部侍郎孟揆親往措置。己巳,進封廣國公
 栻為南康郡王、福國公榛為平陽郡王,並開府儀同三司。壬申,京東轉運副使李孝昌言招安群盜張萬仙等五萬餘人,詔補官犒賜有差。



 三月癸酉朔,雨雹。甲申,知海州錢伯言奏招降山東寇賈進等十萬人,詔補官有差。丙戌,以惠國公柍為開府儀同三司,進封建安郡王。



 夏四月丙辰,降德音於京東、河北路。庚申,蔡京復致仕。復州縣免行錢。戊辰,詔行元豐官制。復尚書令之名,虛而勿授;三公但為階官,毋領三省事。



 五月壬午,封子樅
 為潤國公。丁亥,詔諸路帥臣舉將校有才略者、監司舉守令有政績者歲各三人。



 六月辛丑朔,詔宗室復著姓。丙午,封童貫為廣陽郡王。戊申,詔臣僚輒與內侍來往者論罪。辛亥,慮囚。己未,以蔡攸為太保。癸亥,詔吏職雜流出身人,毋得陳請改換。乙丑,罷減六尚歲貢物。



 秋七月庚午朔,詔士庶毋以「天」、「王」、「君」、「聖」為名字,及以壬戌日輔臣焚香。甲戌,以河間郡王□咢為太保,進封沂王。是月,河東義勝軍叛。熙河、河東路地震。



 九月辛巳,大饗明堂。
 壬辰,金人以擒遼主,遣李孝和等來告慶。是月,河東言粘罕至雲中,詔童貫復宣撫。有狐升御榻而坐。



 冬十月辛亥,賜曾布謚曰文肅。戊午,罷京畿和糴。



 十一月庚午,詔:無出身待制以上、年及三十通歷任滿十歲,乃許任子。乙亥,遣使回慶金國。甲申,朝獻景靈宮。乙酉,饗太廟。丙戌,祀昊天上帝於圜丘,赦天下。庚寅,以保靜軍節度使種師道為河東、河北路制置使。十二月乙巳,童貫自太原遁歸京師。己酉,中山奏金人斡離不、粘罕分兩道
 入攻。郭藥師以燕山叛,北邊諸郡皆陷。又陷忻、代等州,圍太原府。太常少卿傅察奉使不屈,死之。丙辰,罷浙江諸路花石綱、延福宮、西城租課及內外制造局。金兵犯中山府,詹度御之。戊午,皇太子桓為開封牧。罷修蕃衍北宅,令諸皇子分居十位。己未,下詔罪己。令中外直言極諫,郡邑率師勤王,募草澤異才有能出奇計及使疆外者。罷道官,罷大晟府、行幸局。西城及諸局所管緡錢,盡付有司。以保和殿大學士宇文虛中為河北、河東路宣
 諭使。庚申,詔內禪,皇太子即皇帝位。尊帝為教主道君太上皇帝,居於龍德宮。尊皇后為太上皇后。



 靖康元年正月己巳,詣亳州太清宮,行恭謝禮,遂幸鎮江府。四月己亥,還京師。明年二月丁卯,金人脅帝北行。紹興五年四月甲子,崩於五國城,年五十有四。七年九月甲子,兇問至江南,遙上尊謚曰聖文仁德顯孝皇帝,廟號徽宗。十二年八月乙酉,梓宮還臨安。十月丙寅,權攢於永祐陵。十二月丁卯,祔太廟第十一室。十三年正月己亥,加
 上尊謚曰體神合道駿烈遜功聖文仁德憲慈顯孝皇帝。



 贊曰:宋中葉之禍,章、蔡首惡,趙良嗣厲階。然哲宗之崩,徽宗未立,惇謂其輕佻不可以君於下。遼天祚之亡,張覺舉平州來歸,良嗣以為納之失信於金,必啟外侮。使二人之計行,宋不立徽宗,不納張覺,金雖強,何釁以伐宋哉?以是知事變之來,雖小人亦能知之,而君子有所不能制也。跡徽宗失國之由,非若晉惠之愚、孫皓之暴,
 亦非有曹、馬之篡奪,特恃其私智小慧,用心一偏,疏斥正士,狎近奸諛。於是蔡京以獧薄巧佞之資,濟其驕奢淫佚之志。溺信虛無,崇飾游觀,困竭民力。君臣逸豫,相為誕謾,怠棄國政,日行無稽。及童貫用事,又佳兵勤遠,稔禍速亂。他日國破身辱,遂與石晉重貴同科,豈得諉諸數哉?昔西周新造之邦,召公猶告武王以不作無益害有益,不貴異物賤用物,況宣、政之為宋,承熙、豐、紹聖椓喪之餘,而徽宗又躬蹈二事之弊乎?自古人君玩物
 而喪志,縱欲而敗度,鮮不亡者,徽宗甚焉,故特著以為戒。



\end{pinyinscope}