\article{列傳第一百三十一}

\begin{pinyinscope}

 朱倬王綸尹穡王之望徐俯沉與求翟汝文王庶辛炳



 朱倬,字漢章,唐宰相敬則之後,七世祖避地閩中,為閩
 縣人。世學《易》,入太學。宣和五年,登進士第,調常州宜興簿。金將犯邊,居民求避地,倬為具舟給食,眾賴以濟。未幾,民告澇於郡,郡檄倬考實,乃除田租什九,守怒,不能奪。張浚薦倬,召對,除福建、廣東西財用所屬官。宣諭使明橐再薦於朝,時方以劉豫為憂,倬因賜對,策其必敗。高宗大喜,詔改合入官。與丞相秦檜忤,出教授越州。用張守薦,除諸王府教授。檜惡言兵,倬論掩骼事,又忤之。



 梁汝嘉制置浙東,表攝參謀。有群寇就擒,屬倬鞫問,獨
 竄二人,餘釋不問。曰:「吾大父尉崇安日,獲寇二百,坐死者七十餘人。大父謂此饑民剽食爾,烏可盡繩以法?悉除其罪,不以徼賞。吾其可愧大父乎?」通判南劍。建寇阿魏眾數千,劍鄰於建,兵心耎不可用,倬重賞募卒擒獲,境內迄平。



 除知惠州。陛辭,因言嘗策劉豫必敗,高宗記其言,問:「卿久淹何所?」倬曰:「厄於檜。」上愀然慰諭,目送之。旬日間,除國子監丞,尋除浙西提舉,且命自今在內除提舉官,今朝辭上殿,蓋為倬設也。既對,上曰:「卿以朕親擢
 出為部使者,使咸知內外任均。」又曰:「人不知卿,朕獨知卿。」除右正言,累遷中丞。嘗言:「人主任以耳目,非報怨任氣之地,必上合天心。」每上疏輒夙興露告,若上帝鑒臨。奏疏凡數十,如發倉廩,蠲米價,減私鹽,核軍食,率焚稿不傳。知貢舉,遷參知政事。



 紹興三十一年,拜尚書右僕射。金兵犯江,倬陳戰、備、應三策,且謂兵應者勝,上深然之。又策敵三事:上焉者為耕築計,中焉者守備,下則妄意絕江,金必出下策。果如所料。史浩、虞允文、王淮、陳俊
 卿、劉珙之進用,皆倬所薦也。



 高宗自建康回鑾,有內禪意。倬密奏曰:「靖康之事正以傳位太遽,盍姑徐之。」心不自安,屢求去。詔以觀文殿學士提舉江州太平興國宮。孝宗即位,諫臣以為言,降資政殿學士。明年致仕,卒。復元職,恤典如宰相,贈特進。孫著,淳熙十四年登第,仕至吏部尚書。



 王綸,字德言,建康人。幼穎悟,十歲能屬文。登紹興五年進士第,授平江府昆山縣主簿,歷鎮江府、婺州、臨安府
 教授,權國子正。



 時初建太學,亡舊規,憑吏省記,吏緣為奸。綸厘正之,其弊稍革。遷敕令所刪定官、諸王宮大小學教授兼權兵部郎官。言:「孔門弟子與後世諸儒有功斯文者,皆得從祀先聖,今闢庠序,修禮樂,宜以其式頒諸郡縣。」



 二十四年,以御史中丞魏師遜薦,為監察御史,與秦檜論事,忤其意,師遜遂劾綸,且言:「智識淺昧,不能知綸。」由此罷去。逾年,知興國軍。檜死,召為起居舍人兼崇政殿說書,尋兼權禮部侍郎。



 二十六年,試中書舍人。
 高宗躬親政事,收攬威柄,召諸賢於散地,詔命填委,多綸所草。綸奏守臣裕民事,乞毋拘五條,從之。兼侍講。上喜讀《春秋左氏傳》,綸進講,與上意合。嘗同講讀官薦興化軍鄭樵學行,召對命官,且給筆札,錄其所著史。兼直學士院,遷工部侍郎,仍兼直院。撰《吳玠神道碑》,稱上旨,賜宸翰褒寵。



 二十八年,除同知樞密院事。金將渝盟,邊報沓至,宰相沈該未敢以聞。綸率參知政事陳康伯、同知樞密院事陳誠之共白其事,乞備御。已而綸病肺暍,
 告請祠,上遣御醫診視,且賜白金五百兩。



 二十九年六月,朝論欲遣大臣為泛使覘敵,且堅盟好。綸請行,乃以為稱謝使,曹勛副之。至金,館禮甚隆。一日,急召使入,金主御便殿,惟一執政在焉,連發數問,綸條對,金主不能屈。九月,還朝入見,言:「鄰國恭順和好,皆陛下威德所致。」宰臣湯思退等皆賀。然當時金已謀犯江,特以善意紿綸爾。



 綸舊疾作,力丐外,除資政殿大學士知福州,上解所御犀帶賜之。明年,知建康府兼行宮留守。敵犯江,綸
 每以守禦利害驛聞,上多從之。三十一年八月,卒。贈左光祿大夫,謚章敏。無子,以兄綽之子為後。



 尹穡,字少稷。建炎中興,自北歸南。紹興三十二年,與陸游同為樞密院編修官。權知院史浩、同知王祖舜薦其博學有文,召對稱旨,二人並賜進士出身。孝宗獎用西北之士,隆興元年,除穡監察御史,尋除右正言。二年五月,除殿中侍御史。歷遷諫議大夫,未幾而罷。



 初,符離師潰,湯思退復相,金帥移書索地,詔侍從臺諫集議。穡時
 為監察御史,以為國家事力未備,宜與敵和,惟增歲幣,勿棄四州,勿請陵寢,則和議可成。既而盧仲賢出使,為金所脅,又將遣王之望,張浚極言其不可。穡為右正言,懼和議弗就,因劾浚跋扈,未幾罷政。後將割四郡,再易國書,歲幣如所索之數,而敵分兵入寇。上意中悔。穡為侍御史,乞置獄,取不肯撤備及棄地者劾其罪,牽引凡二十餘人。



 時方以和為急,擢穡為諫議大夫。敵勢浸張,遠近震動,都督、同都督相繼辭行。上書者攻和議之失,
 且言:「穡專附大臣為鷹犬,如張浚忠誠為國,天下共知,穡不顧公議,妄肆詆誹;凡大臣不悅者皆逐之,相與表裏,以成奸謀,皆可斬。」上雖怒言者,而一時主議之臣與穡,皆相繼廢黜。先是,胡銓力言主和非是,大臣不悅,命銓與穡分往浙東西措置海道。二人挈家以行,為言者所劾,遂皆罷,語在《陳康伯傳》。



 王之望,字瞻叔,襄陽穀城人,後寓居臺州。父綱,登元符進士第,至通判徽州而卒。之望初以蔭補,紹興八年,登
 進士第。教授處州,入為太學錄,遷博士。久之,出知荊門軍,提舉湖南茶鹽,改潼川府路轉運判官,尋改成都府路計度轉運副使、提舉四川茶馬。



 朝臣薦其才,召赴行在,除太府少卿,總領四川財賦。金人渝盟,軍書旁午,調度百出,之望區畫無遺事。第括民質劑未稅者,搜抉隱匿,得錢為緡四百六十八萬,眾咸怨之。後升太府卿。



 孝宗即位,除戶部侍郎,充川、陜宣諭使。先是,敵帥合喜寇鳳州之黃牛堡,吳璘擊走之,遂取秦州,連復商、陜、原、環
 等十七郡。敵以璘精兵皆在德順,力攻之。時陳康伯秉政,方議罷德順戍,虞允文為宣諭使,力爭不從,上以手札命璘退師。之望既代允文宣諭使,贊璘命諸將棄德順,倉卒引退。敵乘其後,正兵三萬,還者僅七千人,將校所存無幾,連營慟哭,聲震原野。上聞而悔之。



 隆興初,右諫議大夫王大寶疏之望罪,除集英殿修撰、提舉江州太平興國宮。未幾,權戶部侍郎、江淮都督府參贊軍事。之望雅不欲戰,請朝,因奏:「人主論兵與臣下不同,惟奉
 承天意而已。竊觀天意,南北之形已成,未易相兼,我之不可絕淮而北,猶敵之不可越江而南也。移攻戰之力以自守,自守既固,然後隨機制變,擇利而應之。」有旨留中。俄兼直學士院。



 湯思退力主息兵,奏除之望吏部侍郎、通問使。尋議先遣小使覘敵,召之望還。之望首以守備不足恃為告,上亟罷都督府,以之望為淮西宣諭使,甫拜命,又擢右諫議大夫。之望因上章極言廷臣執偏見為身謀,乞明詔在庭,平其心於議論之際。時思退主
 和議,浚主恢復,之望言似善,實陰為思退地也。



 既而視師江上。金復犯邊,遂上和、戰二策,且言措置守禦之備,疏奏未達,拜參知政事。既入,俄兼同知樞密院事。敵兵交至,濠、楚守將或棄城遁,上命湯思退督江、淮師;未行,復令之望督視,改同都督。力辭不行。會太學諸生上書,上怒,欲加罪,之望救解之。遂以參知政事勞師江、淮。



 之望先嘗貽書敵帥。至是,王抃使敵軍,並割商、秦地;許歸被俘人,惟叛亡不預;世為叔侄之國。敵皆聽許,講解而
 罷。上聞敵師退,令督府擇利擊之,之望下令諸將不得妄進。朝廷趣行,之望言:「王抃既還,不可冒小利,害大計。」言者論罷為端明殿學士、提舉江州太平興國宮,居天臺。乾道元年,起知福州、福建路安撫使。捕海賊王大老,捷聞,加資政殿大學士,移知溫州,尋復罷。六年冬,卒。



 之望有文藝幹略,當秦檜時,落落不合或謂其有守。紹興末年,力附和議,與思退相表裏,專以割地啖敵為得計,地割而敵勢益張,之望迄以此廢焉。



 徐俯,字師川,洪州分寧人。以父禧死國事,授通直郎,累官至司門郎。靖康中,張邦昌僭位,俯遂致仕。時工部侍郎何昌言與其弟昌辰避邦昌,皆改名。俯買婢名昌奴,遇客至,即呼前驅使之。建炎初,落致仕,奉祠。



 內侍鄭諶識俯於江西,重其詩,薦於高宗。胡直孺在經筵,汪藻在翰苑,迭薦之,遂以俯為右諫議大夫。中書舍人程俱言:「俯以前任省郎遽除諫議,自元豐更制以來未之有。考之古今,非陽城、種放,則未嘗不循序而進,願姑以所應
 者命之。昔元稹在長慶間,擢知制誥,真不忝矣。緣其為荊南判司,命從中出,召為省郎,便知制誥,遂喧朝論,時謂荊南監軍崔潭峻實引之。近亦傳俯與宦寺倡酬,稱其警策,恐或者不知陛下得俯之由。」不報,俱遂罷。



 紹興二年,賜進士出身,兼侍讀。三年,遷翰林學士,俄擢端明殿學士、簽書樞密院事。四年,兼權參知政事。宰相朱勝非言:「襄陽上流,所當先取。」帝曰:「盍就委岳飛?」參政趙鼎曰:「知上流利害,無如飛者。」俯獨持不可,帝不聽。會劉光
 世乞入奏,鼎言:「方議出師,大將不宜離軍。」俯欲許之,鼎固爭,俯乃求去,提舉洞霄宮。



 九年,知信州。中丞王次翁論其不理郡事,予祠。明年,卒。俯才俊,與曾幾、呂本中游,有詩集六卷。



 沉與求,字必先,湖州德清人。登政和五年進士第,累遷至明州通判。以御史張守薦,召對,除監察御史。上疏論執政,遷兵部員外郎,自劾以為言茍不當,不應得遷。上乃行其言,除殿中侍御史。



 上在會稽,或勸幸饒、信,有急
 則入閩。與求以為今日根本正在江、浙,宜進都建康,以圖恢復。論範宗尹年少為相,恐誤國事。上不悅,以直龍圖閣知臺州。宗尹罷,召還,再除侍御史。



 時軍儲窘乏,措置諸鎮屯田,與求取古今屯田利害,為《集議》二卷上之,詔付戶部看詳。江西安撫、知江州朱勝非未至,而馬進寇江州陷之,與求論九江之陷,由勝非赴鎮太緩,勝非罷去。時方多事,百司稽違,與求援元豐舊制,請許臺諫官彈奏,上從之。與求再居言路,或疑凡範宗尹所引用
 者,將悉論出之。與求曰:「近世朋黨成風,人才不問賢否,皆視宰相出處為進退。今當別人才邪正而言之,豈可謂一時所用皆不賢哉?」人服其言。



 呂頤浩再相,御營統制辛永宗、樞密富直柔、右司諫韓璜屢言其短。與求劾直柔附會永宗兄弟,為致身之資。上遂出永宗,而璜、直柔亦相繼罷黜。



 遷御史中丞。時禁衛寡弱,諸將各擁重兵,與求言:「漢有南北軍,唐用府兵,彼此相維,使無偏重之勢。今兵權不在朝廷,雖有樞密院及三省兵房、尚書
 兵部,但行文字而已。願詔大臣益修兵政,助成中興之勢。」浙西安撫劉光世來朝,以繒帛、方物為獻,上已分乞六宮,與求奏:「今為何時而有此。」時已暮,疏入,上命追取斥還。內侍馮益請別置御馬院,自領其事,又擅穿皇城便門。與求劾益專恣,請治其罪。



 諜報劉豫在淮陽造舟,議者多欲於明州向頭設備。與求言:「使賊舟至此,則入吾腹心之地。臣聞海舟自京東入浙,必由泰州石港、通州料角崇明鎮等處,次至平江南北洋,次至秀州金山,
 次至向頭。又聞料角水勢湍險,必得沙上水手方能轉運。宜於石港、料角等處拘收水手,優給錢糧而存養之,以備緩急。」



 兩浙轉運副使徐康國自溫州進發宣和間所制間金、銷金屏障什物,與求奏曰:「陛下儉侔大禹,今康國欲以微物累盛德,乞斥而焚之,仍顯黜康國。」從之。與求歷御史三院,知無不言,前後幾四百奏,其言切直,自敵己已下有不能堪者。上時有所訓敕,每曰:「汝不識沉中丞邪?」移吏部尚書兼權翰林學士兼侍讀,遂出
 為荊湖南路安撫使、知潭州。引疾丐祠許之。



 四年,出知鎮江府兼兩浙西路安撫使。復以吏部尚書召,除參知政事。金人將入寇,上諭輔臣曰:「朕當親總六軍。」與求贊之曰:「今日親征,皆由聖斷。」上意決親征,書《車攻詩》以賜。上曰:「朕以二聖在遠,屈己通和。今豫逆亂如此,安可復忍?」與求曰:「和親乃金人屢試之策,不足信也。」因奏:「諸將分屯江岸,而敵人往來淮甸,當遣岳飛自上流取間道乘虛擊之,彼必有反顧之憂。」上曰:「當如此措置。」



 五年,兼權
 知樞密院事。時張浚視師江上,以行府為名,言知泰州邵彪及具營田利害事,乞送尚書省。有旨從之。與求不能平,曰:「三省、樞密院乃奉行行府文書邪?」六年,張浚復欲出視師,不告之同列。及得旨,乃退而嘆曰:「此大事也,吾不與聞,何以居位?」遂丐祠,罷,出知明州。



 七年,上在平江,召見,除同知樞密院事;從至建康,遷知樞密院事。薨,贈左銀青光祿大夫,謚忠敏。



 翟汝文,字公巽,潤州丹陽人。登進士第,以親老不調者
 十年。擢議禮局編修官,召對,徽宗嘉之,除秘書郎。三館士建議東封,汝文曰:「治道貴清凈。今不啟上述三代禮樂,而師秦、漢之侈心,非所願也。」責監宿州稅。久之,召除著作郎,遷起居郎。



 皇太子就傅,命汝文勸講,除中書舍人。言者謂汝文從蘇軾、黃庭堅游,不可當贊書之任,出知襄州,移知濟州,復知唐州,以謝章自辨罷。未幾,起知陳州。召拜中書舍人,外制典雅,一時稱之。命同修《哲宗國史》,遷給事中。高麗使入貢,詔班侍從之上,汝文言:「《春
 秋》之法,王人雖微,序諸侯上。不可卑近列而尊陪臣。」上遂命如舊制。內侍梁師成強市百姓墓田,廣其園輔。汝文言於上,師成諷宰相黜汝文,出守宣州。



 召為吏部侍郎,出知廬州,徙密州。密負海產鹽,蔡京屢變鹽法,盜販者眾,有司窮治黨與。汝文曰:「祖宗法度,獲私商不詰所由,欲靖民也。今系而虐之,將為厲矣。」悉縱之。密歲貢牛黃,汝文曰:「牛失黃輒死,非所以惠農,宜輸財市之,則其害不私於密。」上從之。欽宗即位,召為翰林學士,改顯謨
 閣學士、知越州兼浙東安撫使。



 建炎改元,上疏言:「陛下即位赦書,上供常數,後為獻利之臣所增者,當議裁損。如浙東和預買絹歲九十七萬六千匹,而越州乃二十萬五百匹,以一路計之,當十之三。如杭州歲起之額蓋與越州等,杭州去年已減十二萬匹,獨越州尚如舊,今乞視戶等第減罷。」楊應誠請使高麗,圖迎二帝,汝文奏:「應誠欺罔君父,若高麗辭以大圖假道以至燕云,金人卻請問津以窺吳越,將何辭以對?」後高麗果如汝文言。
 上將幸武昌,汝文疏請幸荊南,不從。



 紹興元年,召為翰林學士兼侍講,除參知政事、同提舉修政局。時秦檜相,四方奏請填委未決,吏緣為奸。汝文語檜,宜責都司程考吏牘,稽違者懲之。汝文嘗受辭牒,書字用印,直送省部;入對,乞治堂吏受賂者。檜怒,面劾汝文專擅。右司諫方孟卿因奏汝文與長官立異,豈能共濟國事?罷去以卒。



 先是,汝文在密,檜為郡文學,汝文薦其才,故檜引用之。然汝文性剛不為檜屈,對案相詬,至目檜為「濁氣」。汝
 文風度翹楚,好古博雅,精於篆籀,有文集行於世。



 王庶,字子尚,慶陽人。崇寧五年,舉進士第,改秩,知涇州保定縣。以種師道薦,通判懷德軍。契丹為金人所破,舉燕云地求援,詔師道受降。庶謂師道曰:「國家與遼人百年之好,今坐視其敗亡不能救,乃利其土地,無乃基女直之禍乎?」不聽。宣和七年,金果入寇。太宰李邦彥夜召庶問計,庶曰:「宿將無如種師道,且夷虜畏服,宜付以西兵,使之入援。」邦彥以語蔡攸,攸不然。以庶為陜西運判
 兼制置解鹽事。疆事益棘,欽宗欲幸襄、鄧,先命席益為京西安撫使,益求庶自副。高宗即位,除直龍圖閣、鄜延經略使兼知延安府。累立戰功,進集英殿修撰,升龍圖閣待制,節制陜西六路軍馬。



 先是,河東經制使王□燮既遁歸,東京留守宗澤承制以庶權陜西制置使。會宣諭使謝亮入關,庶移書曰:「夏人之患小而緩,金人之患大而迫,秋高必大舉,盍杖節率兵舉義,驅逐渡河,徐圖恢復。」亮不能從。金人大入,庶調兵自沿河至馮翊,據險以
 守。金人先已乘冰渡河犯晉寧,侵丹州,又渡清水河,破潼關,秦、隴皆震。庶傳檄諸路,會期討賊。涇原統制曲端雅不欲屬庶,以未受命辭;居數日,告身至,又辭。金人知端與庶不協,並兵寇鄜延。庶在坊州聞之,夜趨鄜延以遏其沖。金人詭道陷丹州,州界鄜、延之間,庶乃自當延安路。時端盡統涇原勁兵,庶屢督其進,端訖不行,遂陷延安。語在端傳。



 初,庶聞圍急,自收散亡往援。觀察使王□燮亦將所部發興元。庶至甘泉而延安已不守,既無所
 歸,遂以軍付□燮,而自將百騎馳至襄樂勞軍,尚倚端為助。庶至,端令每門減從騎之半,比至帳下,僅數騎。端厲聲問庶延安失守狀,且曰:「節制固知愛身,不知為天子愛城乎?」庶曰:「吾數令不從,誰其愛身者!」端怒,謀即軍中誅庶而奪其兵,乃夜走寧州,見謝亮曰:「延安,五路襟喉,今既失矣。《春秋》大夫出疆之義得以專之,請誅庶。」亮曰:「使事有指,今以人臣而擅誅於外,是跋扈也,公則自為之。」端沮而歸,乃奪庶節制使印,又拘縻其官屬。會詔庶
 守京兆,庶先以失律自劾得罷。丁內艱。



 時張浚自富平敗歸,始思庶及端之言可用,乃並召之。庶地近先至,力陳撫秦保蜀之策,勸浚收熙河、秦鳳之兵,扼關、隴以為後圖。浚不納。求終制,不許,乃版授參議官。浚念端與庶必不兼容,端未至,但復其官,移恭州。庶因謂浚曰:「端有反心。」浚亦畏端得士,始有殺端意矣。語在《端傳》。



 紹興五年,起復知興元府、利夔路制置使。庶以士卒單寡,籍興、洋諸邑及三泉縣強壯,兩丁取一,三丁取二,號「義士」,日
 閱於縣,月閱於州,厚犒之,不半年,有兵數萬。浚言於朝,升徽猷閣直學士。有讒於浚者,徙庶知成都,改嘉州。明年,浚劾庶輕率傾險,落職奉祠。尋起知遂寧,固避得請。



 六年,除湖北安撫使、知鄂州。趨闕,上因燕見,庶言:「陛下欲保江南,無所事;如曰紹復大業,都荊為可。荊州左吳右蜀,利盡南海,前臨江、漢,出三川,涉大河,以圖中原,曹操所以畏關羽者也。」上大異之。復顯謨閣待制、知荊南府、湖北經略安撫使,又復直學士。



 七年十月,以兵部侍
 郎召。明年春,入對,上曰:「召卿之日,張浚已去,趙鼎未來,此朕親擢,非有左右之助。」庶頓首謝,因奏:「恢復之功十年未立,其失在偏聽,在欲速,在輕爵賞,是非邪正混淆。誠能賞功罰罪,其誰不服?昔漢光武以兵取天下,不以不急奪其費,不知兵者不可使言兵。」又口陳手畫秦、蜀利害。上大喜,即日遷本部尚書。閱月,拜樞密副使。



 議者乞遣重臣行邊,遂命庶措置江、淮邊防。京、湖宣撫使岳飛聞庶行邊,遺書曰:「今歲若不出師,當納節請閑。」庶壯
 之。庶還朝,論金人變詐,自渝海上之盟,因及飛納節之語。當是時,秦檜再相,以和戎為事。金使烏陵思謀至,詔趣庶還。庶力詆和議,乞誅金使,其言甚切。金又遣張通古來許割地,還梓宮,歸太后。庶曰:「和議之事,臣所不知。」凡七疏乞免官,乃以資政殿學士知潭州。



 御史中丞勾龍如淵劾庶本趙鼎所薦,欺君罔上。庶罷歸,至九江,被命奪職,徙家居焉。十三年,御史胡汝明論庶譏訕朝政,責響德軍節度副使,道州安置。至貶所卒。孝宗思庶言,
 追復其官,謚敏節。子六人,之奇,乾道中,知樞密院事。



 辛炳,字如晦,福州候官縣人。登元符三年進士第,累官至監察御史兼權殿中侍御史。先是,蔡京廢發運司轉般倉為直達綱,舟入,率侵盜,沉舟而遁,戶部受虛數,人畏京莫敢言。炳極疏其弊,且以變法後兩歲所得之數,較常歲虧欠一百三十有二萬,支益廣而入寢微,乞下有司計度。徽宗以問京,京怒,以炳為沮撓,責監南劍州新豐場,尋提舉洞霄宮,起知袁州,移無為軍。靖康初,
 召為兵部員外郎。



 高宗即位,除左司員外郎,辭;未幾,起直龍圖閣、知潭州。明年,張浚調兵潭州,以炳懦怯不能,罷之,尋以起居舍人召,辭。紹興二年,復以侍御史召。首言今日公道壅塞,風俗頹薄,連疏三省所行乖失數十事,請諭大臣勿廢都堂公見之禮。時福建八州添差至百八十餘員,炳言:「艱危多事之時,冗食之官無益,當罷。」從之。



 蘇、湖地震,下詔求言。炳言:「大臣無畏天之心,何事不可為?」其言甚峻,由是宰執呂頤浩居家待罪,炳劾罷頤
 浩。知樞密院事張浚召赴行在,炳論其敗事誤國,浚坐落職。



 除御史中丞。時方遣使議和,炳方言:「金人無信,和議不可恃,宜講求守禦攻戰之策。」以疾請外,除顯謨閣直學士、知漳州,未赴而卒。詔:炳任中執法,操行清修,今其云亡,貧無以葬,賜銀帛賻其家,贈通議大夫。



 論曰:秦檜晚薦士以收人望,然一時知名之士,亦豈盡可籠絡者哉!朱倬論事輒不合,王綸代言辭合體要,若尹穡、王之望人品雖不同,其附和議則一爾。徐俯末與
 赴鼎爭辨,沮抑岳飛,異哉。沉與求止和親之議,翟汝文善料事,而檜以為異己。王庶論都荊州,當時諸臣之慮皆不及此。考夫祈寬之事,庶蓋忠義人也。辛炳雅志清修,又豈多見也歟。



\end{pinyinscope}