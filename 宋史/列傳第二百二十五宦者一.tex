\article{列傳第二百二十五宦者一}

\begin{pinyinscope}

 ○竇神寶王仁睿王繼恩李神福弟神祐劉承規閻承翰秦翰周懷政張崇貴張繼能衛紹欽石知顒孫全彬
 鄧守恩



 宋世待宦者甚嚴。太祖初定天下,掖庭給事不過五十人,宦寺中年方許養子為後。又詔臣僚家毋私蓄閹人,民間有閹童孺為貨鬻者論死。去唐未遠,有所懲也。



 厥後,太宗卻宰相之請,不授王繼恩宣徽;真宗欲以劉承規為節度使,宰相持不可而止。中更主幼母後聽政者凡三朝,在於前代,豈非宦者用事之秋乎!祖宗之法嚴,宰相之權重,貂璫有懷奸慝,旋踵屏除,君臣相與防微
 杜漸之慮深矣。



 然而宣政間童貫、梁師成之禍,亦豈細哉!南渡苗、劉之逆,亦宦者所激也。《坊記》曰:「君子之道,闢則坊與!大為之坊,民猶逾之。」可不戒哉!作《宦者傳》。



 竇神寶,父思儼,五代時為內侍,宋初皇城使。兄神興,左領軍衛大將軍致仕。神寶初為黃門,太平興國中,從征太原,擐甲登城,中流矢,稍遷入內高品,監並州戍兵。屢出襲賊,前後破砦三十六,斬千餘級,大獲鎧甲、牛馬、橐駝,因築三砦。詔褒之。九年,命與尹憲屯夏州,時岌伽羅
 膩等十四族久叛,神寶率兵大破之,焚其廬帳,斬千餘級,虜獲甚眾。



 雍熙中,朝廷遣使綏、宥、麟、府州,募邊部願攻契丹者,賜以金帛。神寶上言:「狼子野心,由此或生邊隙。」乃止。俄轉殿頭高品。淳化中,使河東,閱視堡柵兵騎。慕容德豐自邢臺徙延州,未至郡,詔神寶乘傳權州事。環州近邊內擾,與陳德玄討之,破牛家族二十八部,且規度通遠入靈武路,就命環慶同駐泊。牛家族復結眾叛,又破之,殲餘黨於極泉鎮,獲其渠帥九人。西戎寇鄜,
 以援之之勞,遷供奉官,與田紹斌部送靈州芻糧,即命駐泊。



 李繼遷入寇,與慕容德豐襲破其堡砦,焚帳幕,獲人畜數萬計。連詔嘉獎,遷內殿崇班。至道初,繼遷再寇靈武,神寶遣人間道告急闕下。賊圍之歲餘,地震二百餘日,城中糧糗皆竭,潛遣人市糴河外,宵運以入。間出兵擊賊,賊引去,以功拜西京作坊副使。又命於浦洛河、清遠軍援芻糧,與楊允恭議造小車三千,運糧至環州。三年,遷西京左藏庫副使。出使靈武,還,奏對稱旨,面授
 供備庫使。



 咸平中出為高陽關鈐轄,徙貝、冀巡檢。會原州野貍族三千餘眾徙帳於順成穀,大蟲堪與熟魏族接戰,詔神寶和洽之,至則定其經界,遣悉還舊地。入為內侍右班副都知。真宗朝陵,留與劉承珪同掌大內事。大中祥符初,勾當三班院,又掌諸王宮事。遷西京左藏庫使、領密州刺史兼掌往來國信。



 神寶蒞職精恪,性吝嗇,畜貨鉅萬。天禧初,以皇城使罷內職。三年,卒,年七十一。錄其子守志為入內供奉官。



 王仁睿,不知何許人。年十餘歲,事太宗於晉邸,服勤左右,甚淳謹。及即位,宣傳指揮頗稱旨。歷入內小底都知、洛苑副使。命典宮闈出納之命,最居親近。嘗與柴禹錫等發秦邸陰事。雍熙四年被疾,遣太醫診視。卒,年四十一,特贈內侍省內侍。



 國朝以來,內侍都知、押班不領他職。淳化、至道後,皆內殿崇班以上兼充,多至諸司使,有領觀察使者,沒皆有贈官,官給葬事。



 舊制,內侍人許養一子,以充繼嗣。開寶四年,以其爭財起訟,詔自今滿三
 十無養父者,始聽養子,仍以其名上宣徽院,違者準前詔抵死。咸平中,徐志通為溫、臺等州巡檢,坐取李歡男四人為假子,又縱卒略民家小兒,致其母抱兒投海死,決杖配掃灑班,復申前詔以戒厲之。



 王繼恩,陜州陜人。周顯德中為內班高品。初養於張氏,名德鈞。開寶中求復本宗,太祖召見,許之,因賜名焉。累為內侍行首。



 會討江南,與竇神興等部禁兵及戰船抵採石。九年春,改裏面內班小底都知,賜金紫。十月,加武
 德使。太祖崩,副杜彥圭案行陵地,尋充永昌陵使。太平興國三年,遷宮苑使。久之,領河州刺史,掌軍器弓槍庫。



 雍熙中,王師克雲、朔,命繼恩率師屯易州,又為天雄軍駐泊都監。自岐溝關、君子館敗績之後,河朔諸路為契丹所擾,城壘多圮。四年,詔繼恩與翟守素、田仁朗、郭延濬分路按行增築之。及遣將北伐,又為排陣都監,屯中山,改皇城使。端拱初,領本州團練使,又為鎮、定、高陽關三路排陣鈐轄。淳化初,賜甲第一區。五年,加昭宣使,勾
 當皇城司。李順亂成都,命為劍南兩川招安使,率兵討之。軍事委其制置,不從中覆。管內諸州系囚,非十惡正贓,悉得以便宜決遣。二月,命馬步軍都軍頭王杲趣劍門、崇儀使尹元由峽路分遣討賊,並受繼恩節度。詔前軍所至,其賊黨敢抗王師者,即須殺戮;如本非同惡,受制兇徒,先被脅從今能歸順者,悉釋其罪。四月,繼恩由小劍門路入研石砦破賊,斬首五百級,逐北過青強嶺,平劍州,進破賊五千於柳池驛,斬千六百級,賊眾望風
 奔走,殺戮溺死者不可勝計。又克閬、綿二州。五月,至成都,破賊十萬餘,斬首三萬級,獲順及鎧甲、僭偽服用甚眾。



 朝議賞功,中書欲除宣徽使。太宗曰:「朕讀前代史書,不欲令宦官預政事。宣徽使,執政之漸也,止可授以他官。」宰相力言繼恩有大功,非此任無足以為賞典。上怒,深責相臣,命學士張洎、錢若水議別立宣政使,序位昭宣使上以授之。進領順州防禦使。



 繼恩握重兵,久留成都,轉餉不給,專以宴飲為務。每出入,前後奏音樂。又
 令騎兵執博局棋枰自隨,威振郡縣。僕使輩用事恣橫,縱所部剽掠子女金帛,軍士亦無鬥志。餘賊迸伏山谷間,州縣有復陷者。太宗知之,乃命入內押班衛紹欽同領其事。又遣樞密直學士張鑒、西京作坊副使馮守規乘傳督其捕賊。議分減師徒出蜀境,以便糧運。



 高品王文壽者,隸繼恩麾下,繼恩遣領虎翼卒二千,分遂州路追討。文壽御下嚴急,士卒皆怨。一夕臥帳中,指揮使張嶙遣卒排闥入,斬文壽首以出。會夜昏黑,嶙猶疑其非,然
 炬照之,曰:「是也。」時嘉州賊帥張餘有眾萬餘,嶙即以所部與之合,賊勢甚盛。初奏至,太宗欲盡誅軍人妻子,近臣或請勿殺,悉索營中書,遣帥招撫,諭以釋罪,親屬皆全,必自引來歸,因可破賊。上然之,令巡檢程道符諭旨。亡卒斬嶙,函首送繼恩,皆自拔來歸。因使為鄉導擊賊,悉平之。



 至道二年春,布衣韓拱辰詣闕上言:「繼恩有平賊大功,當秉機務,今止得防禦使,賞甚薄,無以慰中外之望。」上大怒,以拱辰惑眾,杖脊黥面配崖州。俄召繼恩。
 太宗崩,命與李神福按行山陵,加領桂州觀察使。



 繼恩初事太祖,特承恩顧。及崩夕,太宗在南府,繼恩中夜馳詣府邸,請太宗入,太宗忠之,自是寵遇莫比。喜結黨邀名譽,乘間或敢言薦外朝臣,由是士大夫之輕薄好進者從之交往,每以多寶院僧舍為期。有潘閬者能詩詠,賣藥京師,繼恩薦之,召見,賜進士第。尋察其狂妄,追還詔書。



 及真宗初,繼恩益豪橫,頗欺罔,漏洩機事,與參知政事李昌齡緘題往來,多請托,至有連宮禁者。素與胡
 旦善,時將加恩,密諉其為褒辭。又士人詩頌盈門。上惡其朋結,黜為右監門衛將軍、均州安置,籍沒貲產,多得蜀土僭擬之物。昌齡責忠武軍節度行軍司馬,旦削籍,長流尋州。詔中外臣僚曾與繼恩交識及通書尺者,一切不問。



 咸平二年,卒於貶所,遣使將其家屬還京師,假官舍處之。四年,聽歸葬。大中祥符三年,特詔追復官爵,以白金千兩賜其家。子懷珪,轉入內高班。



 李神福,開封人。父繼美,仕後唐為內侍,顯德初為御廚
 都監。時內臣止以服色為貴,太祖特賜紫,後至右領軍衛將軍。神福少給事晉王府,謹恪解上意,未嘗少怠。太宗即位,授入內高品。從征太原,攻城之際,往來梯沖間宣傳詔命,即行在所遷殿頭。太平興國六年,擢入內高品押班,遷副都知、勾當翰林司,轉入內內班都知。兼勾當祗候內品班。淳化四年,遷崇儀副使、勾當皇城司。屬初易黃門之號,轉入內黃門都知,俄加宮苑使。太宗好筆札,神福每侍側,多獲別本之賜。及不豫,神福朝夕左
 右,躬侍藥膳。



 真宗即位,遷皇城使、內侍省入內內侍都知,領恩州團練使、勾當永熙陵行宮事。時模寫太宗聖容,以神福立侍。未幾,求罷都知,加昭宣使、勾當皇城司,賜第宮城側,遣修內工為葺之。咸平二年秋,閱兵東郊,以神福為大內都部署。是冬,幸大名,與王繼英並為行宮使。四年,勾當三班,部修含光殿,賜賚甚優。景德初,兼領親王諸宮使。三年,改宣政使。從謁諸陵,復為行宮使。進幸西京,賜酺,命神福主其事。



 大中祥符初,天書降夕,
 神福與劉承珪、鄧永遷、李神祐、石知顒、張景宗、藍繼宗同直禁中,賜以器幣、緡錢。京師酺會,又令神福與白文肇、閻承翰同典之。是歲封泰山,與曹利用同經度行宮道路。及車駕進發,又為行宮使。禮畢,授宣慶使,領昭州防禦使,整肅禁衛。先是,諸司使止於宣政,故特置使額以寵之。三年,卒,年六十四。贈潤州觀察使。



 神福性恭願和易,每為衛紹欽所詬罵,皆引避不校。在禁闥五十年,稱為長者。然久掌三班,無規制,遠近失敘,有請托者不
 能拒之,人譏其所守。子懷斌、懷贇。弟神祐。



 神祐,初以父任授殿頭高品。太祖將納孝章皇后,命神祐奉聘禮於華州。乾德五年,徵太原,負御寶從行。開寶二年,又從征太原,時有詔緣邊和市軍儲,車駕在潞州聞之,且慮擾民,令神祐馳驛止之。時詔下已五日,神祐一夕而及晉陽。一日,甲士既陣,賊潛縱火焚梯沖,亟命神祐部衛兵為援,斬賊甚眾,餘悉潰去。王師伐廣州,隨軍賞給。劉鋹平,先部帑藏之物赴京師。及土寇周瓊等
 叛,又副尹崇珂討平之。六年,隨曹彬南征。克關城,擒偽將朱令贇,命神祐馳入獻捷書,賜錦袍、金帶。



 太宗即位,遷南作坊副使。錢俶歸朝,命神祐往按府藏之積。再徵太原,領工徒千人隨駕,以備繕完甲兵。劉繼元表納降款,太宗陳儀衛城北臺以受之,繼元移時未至,神祐馳單騎入城,俄頃,引繼元至。及北伐燕薊,命與劉廷翰統精騎為大陣之援。車駕還,又令率兵屯定州以備契丹。太平興國六年,滑州治河防,材葦未具,命神祐馳往垣
 曲,伐薪蒸四百萬以濟其用。七年,契丹寇邊,命領兵屯瀛州,俄改崇儀使,提點左右藏庫,遷洛苑使。至道初,西鄙不寧,命為靈、環排陣都監,率眾至烏白池而還。俄駐永興,復護糧運抵朔方。



 真宗嗣位,轉內園使、邠州都監。車駕北巡,改天雄軍都監、子城內巡檢。時北兵充斥,道途阻塞,命神祐單騎諭密旨於諸將。敵騎數百忽至,神祐乃周麾而呼,若召伏兵,敵懼而逃,遂達其命。俄充邢州排陣都監,勾當西八作司。景德初,上幸澶州,領隨駕
 壕砦。



 三年,遷入內都知。從東封還,遷南作坊使。時內侍將遷秩,有扈從升山、不升山或不預從祀者,令神祐第其勤狀,上親閱而敘遷之。有範守遜、皇甫文、史崇貴、張延訓等,皆嘗有譴累而互陳勞效,且言神祐等品第非當,泣訴於上,止而復來者數四。守遜等先改內常侍,上怒,悉停其官。神祐洎石知顒、副都知張景宗、藍繼宗並坐削職。尋掌御廚七年,卒,年六十六。大中祥符六年,錄其孫永和為三班奉職。神祐性謹願,曉音律,頗好篇
 詠。



 子懷岊,太宗時嘗請為道士,後復內侍。多屯邊郡,常持大鐵鞭以斗賊,屢中流矢,至供奉官。懷儼為內殿崇班。



 劉承規,字大方,楚州山陽人。父延韜,內班都知。承規,建隆中補高班,太宗即位,超拜北作坊副使。時泉帥陳洪進歸朝,遣承規疾置封其府庫。會土民嘯聚為寇,承規與知州喬維岳率兵討定之。太平興國四年,命與內衣庫使張紹勍等六人率師屯定州,以備契丹,又護滑州決河。雍熙中,勾當內藏庫兼皇城司,出為鄜延路排陣
 都監,改崇儀使,遷洛苑使。至道中,與周瑩同簽書提點樞密、宣徽諸房公事,仍加六宅使。承規懇辭,帝雖不許而嘉其退讓。



 真宗立,瑩為宣徽使,以承規領勝州刺史、簽書宣徽院公事。尋讓宣徽之務,加莊宅使。咸平三年,遷北作坊使。時邊境未寧,議修天雄軍城壘,命承規乘傳經畫,又命提舉內東、崇政殿等諸門,遷宮苑使。上詢承規西事,請益環州木波鎮戍兵,以為諸路之援,從之。俄兼勾當群牧司。



 景德二年,與李允則使河間,按視嘗
 經戰陣等處將卒之勞。是歲,置官提舉京師諸司庫務,以承規領之。所創局署,多所規制。改皇城使。與林特、李溥議更茶法。四年,三司上言新課增羨,承規以勞加領昭州團練使。



 大中祥符初,議封泰山,以掌發運使遷昭宣使、長州防禦使。會修玉清昭應宮,以承規為副使。祀汾陰,復命督運。議者以自京至河中,由陸則山險,具舟則湍悍,承規決議水運,凡百供應,悉安流而達。自朝陵、東封及是皆留掌大內。禮成,當進秩,表求休致,手詔敦
 勉,仍作七言詩賜之。拜宣政使、應州觀察使。



 五年,以疾求致仕。修宮使丁謂言承規領宮職,藉其督轄,望勿許所請,第優賜告詔,特置景福殿使名以寵之,班在客省使上。仍改新州觀察使,上作歌以賜。承規以廉使月稟歸於有司,手詔褒美,復定殿使奉以給之。本名承珪,以久疾羸瘵,上為取道家易名度厄之義,改珪為規。疾甚,請解務還私第,聽之。仍許皇城常務上印日,內藏庫有創制,就取商度。又再表求罷,官檢校太傅、左騎衛上將
 軍、安遠軍節度觀察留後致仕。七月卒,年六十四。廢朝,贈左衛上將軍、鎮江軍節度,謚曰忠肅。



 承規事三朝,以精力聞,樂較簿領,孜孜無倦。自掌內藏僅三十年,檢察精密,動著條式。又制定權衡法,語在《律歷志》。性沈毅徇公,深所倚信,尤好伺察,人多畏之。上崇瑞命,修祠祀,飾宮觀,承規悉預聞。作玉清昭應宮,尤為精麗。屋室有少不中程,雖金碧已具,必毀而更造,有司不敢計所費。二聖殿塑配饗功臣,特詔塑其像太宗之側。承規遇事亦
 或寬恕,鑄錢工常訴本監前後盜銅瘞地數千斤,承規佯為不納,因密遣人發取還官,不問其罪。咸平中,朱昂、杜鎬編次館閱書籍,錢若水修祖宗實錄,其後修《冊府元龜》、國史及編著讎校之事,承規悉典領之。頗好儒學,喜聚書,間接文士質訪故實,其有名於朝者多見禮待,或密為延薦。



 自寢疾惟以公家之務為念,遺奏求免贈賻詔葬,上甚嗟惜之,遣內臣與鴻臚典喪,親為祭文。玉清昭應宮成,加贈侍中,遣內侍鄧守恩就墓告祭。子從
 願,為西染院使。



 閻承翰,真定人。周顯德中為內侍。入宋事太祖,以謹願稱。太宗時擢為殿頭高品,稍遷內侍供奉官、內殿崇班。先是,八作司材木頗有隱弊,承翰建議於都城西置事材場,治材以給之。雍熙中,知廣州徐休復奏轉運使王延範不軌狀,遣承翰馳往同逮捕下獄,就鞫之,考掠過苦,延範遂坐誅。李順亂蜀,命為川峽招安都監。賊平,授西京作坊副使。會增募金吾兵,以承翰及劉承蘊分充
 左右金吾都監兼街仗司事。俄罷之。



 真宗即位,改西京作坊使、內侍左班副都知。咸平三年,河決鄆州王陵埽,遣承翰護塞。時議徙鄆州以避河患,又詔承翰與工部郎中陳若拙乘傳規度,徙於舊治之東南。五年,入內都知韓守英為鎮、定、高陽關三路排陣都鈐轄,上以其素無執守,議別擇人,因謂宰相曰:「承翰雖無武勇,然涖事勤恪。」乃令代守英。時中山屯兵甚眾,艱於飛挽,承翰請鑿渠,計引唐河水自嘉山至定州三十二里,又至蒲陰
 東六十二里,合沙河經邊吳泊入界河以濟饋運,亦可旁為方田,上嘉而從之。渠成,人以為便,優詔褒之。景德初,契丹謀寇順安軍,承翰奉詔發雄、霸精兵,與荊嗣、張延同築壘御之,俄又遣詣德清軍規度重修城壘。車駕北征,承翰先在澶州北城,奏契丹兵在近,請不度河,上不聽,促駕度浮橋。二年,加領廉州刺史,勾當群牧司,多條上馬政,遂兼群牧副使。時契丹結好,始置國信司主交聘之事,以承翰領之,多所規置。



 大中祥符初,改西京
 左藏庫使,充夏州趙德明加恩官告使。還,請於浦洛河置館,以待夏臺進奉使,上以荒夐勞役,不許。四年,遷內園使、左班都知,領獎州團練使。



 有西京左藏庫副使趙守倫久典廄牧,至是又掌估馬,與承翰聯職任,雖素為姻家,然不相得,遂各訟訴,並付御史臺。承翰坐擅用群牧司錢,當贖金三十斤;守倫坐違制移估馬司,當免所居官;典吏當杖脊。詔寬其罰:承翰贖金十斤,守倫贖金二十斤,典吏亦降從杖。群牧都監張繼能、判官陳越田
 瑴、勾當騏驥院楊保用、估馬楊繼凝皆釋之,制置使陳堯叟特免按問。



 六年,上制《內侍箴》賜之,承翰表請刻石省中。明年,建應天府為南京,作鴻慶宮,設太祖、太宗像,遣承翰自京奉往。授南作坊使、入內都知。未幾,卒,年六十八。贈懷州防禦使。



 承翰性剛強,所至過於檢察,乏和懿之譽。子文應,西京左藏庫使。



 秦翰,字仲文,真定獲鹿人。十三為黃門,開寶中遷高品。太平興國四年,崔彥進領眾數萬擊契丹,翰為都監,以
 善戰聞。太宗因加賞異,謂可屬任。雍熙中,出為瀛州駐泊,仍管先鋒事,遷入內殿頭高品、鎮、定、高陽關三路排陣都監。淳化四年,補入內押班。



 趙保忠叛,命李繼隆率師問罪,翰監護其軍。次延州,翰慮保忠遁逸,即乘驛先往,矯詔安撫以緩其陰計。王師至,翰又諷保忠以地主之禮郊迎,因並驅而出,保忠遂就擒,以功加崇儀副使。至道初,為靈、環、慶州、清遠軍四路都監。真宗即位,加洛苑使、入內副都知。咸平中,河朔用兵,以為鎮、定、高陽
 關排陣都監,敗契丹於莫州東,追斬數萬,盡奪所掠老幼。詔褒之,徙定州行營鈐轄。



 王均之亂,為川峽招安巡檢使。時上官正與石普不協,翰恐生事,為曉譬和解之。親督眾擊賊,中流矢不卻,五戰五捷,遂克益州,上手札勞問。翼日,進至廣都,斬首千餘級,獲馬數千匹。歸朝,遷內園使,領恩州刺史。



 出為鎮、定、高陽關前陣鈐轄,又徙後陣。破契丹二萬眾於威虜軍西,俘其鐵林大將等十五人。又為邠寧、涇原路鈐轄兼安撫都監,率所部按行山
 外,召戎落酋帥,諭以恩信,凡三千餘帳相率內附。未幾,康奴族拒命,翰與陳興、許均深入擊之,斬級數千,焚其廬帳,獲牛馬甚眾。復與陳興、曹瑋襲殺童埋軍主於武延咸泊川。詔書加獎,賜錦袍、金帶、白金五百兩、帛五百匹。



 景德初,車駕將北巡,先遣翰乘傳往澶、魏裁制兵要,許便宜從事。俄充邢洺路鈐轄,與大軍會德清軍,張掎角之勢。又召為駕前西面排陣鈐轄,管勾大陣。翰即督眾環城浚溝洫以拒契丹。功畢,契丹兵果暴至,翰不脫
 甲胄七十餘日,契丹乞和,凱旋,留泊澶州。月餘,令率所部兵還京師,加宮苑使、入內都知。出為涇、原、儀、渭鈐轄。先是,西鄙無藩籬之蔽,翰規度要害,鑿巨塹,計工三十萬,役卒數年而成,不煩於民。就遷皇城使、入內都知。以翰在邊久,宣力勤盡,特置是名以寵異焉。翰表讓,不聽。



 大中祥符初,求從東封,手詔諭以西垂委任之異。改昭宣使,又為群牧副使,祀汾陰。是歲,夏州屬戶有擾境上者,即日遣翰往脽上按視,遍巡邊部。及翰至,事寧,復還
 扈從,凡行在諸司細務,悉令裁決,不須中覆。禮畢,加領平州團練使,奉祀毫州,掌如汾陰。八年,營葺大內,詔翰參領其事。閏六月,暴卒於內庭之廨,年六十四。上甚悼惜,為之泣下。贈貝州觀察使,賻襚加等。修內畢,詔遣使以襲衣、金帶賜其家。



 翰倜儻有武力,以方略自任。前後戰鬥,身被四十九創。李繼遷之未賓也,翰因使常出入其帳中,無疑間,嘗白太宗言:「臣一內官不足惜,願手刺此賊,死無所恨。」太宗深嘉其忠。



 翰性溫良謙謹,接人以
 誠信,群帥有剛狠不和者,翰皆得其歡心。輕財好施,與將士同休戚,能得眾心,皆樂為用。其歿也,禁旅有泣下者。



 九年,重贈彰國軍節度,詔楊億撰碑文,億以其不蓄財,表辭所贄物,雖朝廷不許,而時論美之。子懷志,內殿崇班。



 周懷政,並州人。父紹忠,以黃門事太宗,從征河東,得懷政於亂尸間,養為子。給事禁中,累至入內高品。大中祥
 符初,真宗東封,命修行宮頓遞。及奉泰山天書馳驛赴闕,轉殿頭。天書每出宮,與皇甫繼明並為夾侍。東封禮成,與內殿崇班康宗元留泰山,修圜臺,轉入內西頭供奉官。祀汾陰,轉東頭。六年,劉承規卒,擢內殿崇班、入內押班、勾當皇城司。會朝謁太清宮,與閻承翰等同管勾大內事。七年,奉天書摹刻於乾元殿,為刻玉都監,又為修兗州景靈宮、太極觀都監,俄遷內殿承制。是冬,命起居舍人、知制誥盛度為會真宮醮告使,懷政為都監。還,為玉清昭應宮都監兼掌景靈宮、會靈觀使。刻玉成,遷
 如京副使。九年,建資善堂,以懷政為都監。壽丘宮觀成,優賜襲衣、金帶,遷崇儀使。天禧大禮,又為修奉寶冊都監,加領長州刺史,是冬,遷洛苑使。二年春,遷左藏庫使。仁宗為皇太子,命為入內副都知、管勾左右春坊,轉左騏驥使。三年,領英州團練使,加昭宣使。



 懷政日侍內廷,權任尤盛,於是附會者頗眾,,往往言事獲從,同列位望居右者,必排抑之。中外帑庫皆得專取,因多入其家。性識凡近,酷信妖妄。有朱能者,本單州團練使田敏廝養,
 為人兇狡,遂賂懷政親信,得見,因與侍卒姚斌妄談神懌以訹之。懷政大惑,援能至御藥使、領階州刺史。俄於終南山修道觀,與劉益輩造符命,托神言國家休咎,否臧大臣。時寇準鎮永興,能為巡檢,倚準舊望,欲實其事。準好勝,喜其附己,多依違之。



 朝臣屢言懷政之妄,真宗含忍不斥,然漸疏遠之。懷政憂懼,時使小黃門自禁中出,詐稱宣召,入內東門,坐別室,久之而還,以欺同類。會準為相,逾年而罷,懷政愈畏獲譴,不自安。



 四年七月,與
 弟禮賓副使懷信謀潛召客省使楊崇勛、內殿承制楊懷吉、閣門祗候楊懷玉會皇城司,期以二十五日竊發,殺丁謂等,復相寇準,奉真宗為太上皇,傳位太子。前夕,崇勛、懷吉詣丁謂第密告之,謂即夜偕崇勛、懷吉至曹利用第計議,翌日,利用入奏,真宗怒,命收懷政。令宣徽北院使曹瑋與崇勛於御藥院鞫訊,具伏。帝坐承明殿臨問,懷政但祈哀而已,命斬於城西普安寺。父內殿承制紹忠及懷信並杖配復岳州,子侄勒停,貲產沒官。朱
 能父左武衛將軍致仕諤、母周氏,罰銅百斤,子守昱、守吉分配邵、蔡、道州。懷政僕使、親從並杖配海島、遠州,部下使臣貶秩有差。懷政之未敗也,紹忠嘗詬之曰:「斫頭豎子終累我!」懷信謂之曰:「兄前事必敗,宜早詣上首實,庶獲輕典。」及其謀亂,又泣拜止之,不聽,故皆得免死。



 右街僧錄澄遠以預聞妖詐,決杖黥配郴州。內供奉官譚元吉、高品王德信、高班胡允則、黃門楊允文與懷政協同妖妄,皆杖配遠州。入內押班鄭志誠與能書問往還,
 削兩任,配房州。入內供奉官石承慶嘗為懷政所召,夜二鼓不下皇城門鑰以待,黃門黃守忠見之,戒門卒勿納,至是言其事,承慶坐削兩任,配宿州。楊懷玉次日始詣樞密院自陳,責授侍禁、杭州都監。擢崇勛內客省使、桂州觀察使,懷吉如京使,賜以金帶、金銀。



 懷政既誅,亟遣入內供奉官盧守明、鄧文慶馳驛永興,捕朱能。劉益、李貴、康玉、唐信,道士王先、張用和悉免死,配遠州。能偵知使者至,衷甲出,殺守明以叛。詔遣內殿承制江德明、
 入內供奉官於德潤發兵捕之,能入桑林自縊死。永興、乾耀都巡檢供奉官李興、本軍十將張順斷能及其子首以獻,補興閣門祗候,順牢城都頭。以劉益等十一人黨能害中使,磔於市。王先、李貴、唐信、張用和八人皆處斬。能母妻子弟皆決杖配隸,閣門祗候穆介、知永興軍府朱巽、轉運使梅詢劉楚、知鳳翔府臧奎等坐與懷政、能交結相稱薦,皆論罪。降寇準太常卿,再貶道州。凡朝士及永興、鳳翔官吏與準厚善者,悉降黜焉。



 張崇貴,真定人。太祖時為內中高品,稍遷殿頭。太平興國中,以善射遷為御帶。錢俶納土,命馳往閱城防儲偫之數。親征太原,從崔彥進、李漢瓊先路視水草。端拱初,補內供奉官。



 淳化四年,命乘傳之延州招羌戎之內附者,發庫錢犒給,以金幣賜酋領。將行,轉內班右班押班,就命管勾鄜延屯兵,李繼隆討李繼遷,詔崇貴以延安兵掎角進討。及擒趙保忠,留崇貴與石霸守綏州,徙平夏民以實之。繼遷扼橐駝路,驅脅內屬戎人,崇貴與田
 敏率熟倉族癿遇戰於雙塠,殺二千餘級,掠牛羊、橐駝、鎧甲甚眾,連詔褒諭。繼遷走漠中,遣其將佐趙光祚、張浦求納款,會於石堡砦,崇貴椎牛釃酒犒諭之,給以錦袍帶。會改內班為黃門,命為黃門右班押班,仍加內殿崇班,又改黃門為內侍,職隨易焉。既而繼遷貢橐駝、名馬待罪,遣崇貴往賜器幣、茶藥、衣物。



 至道元年,進崇儀副使、內侍右班副都知。時繼遷復叛,劫芻饋於浦洛河。二年,詔李繼隆大發師進討。賊圍靈州急,太宗將棄之,
 廷議未決,命崇貴與馮納乘傳往議其事,乃益兵固守,就命為靈、環、慶州、清遠軍路監軍,又為排陣都監。



 真宗立,拜洛苑使、右班都知、管勾並州軍馬。自至道後,五路討賊,兵戰相繼,卒無成功。及是,保吉復修貢,詔以定難節度授之,命崇貴持詔命、衣帶、器幣以賜。使還,加六宅使。



 咸平元年,又命管勾鄜延屯兵,泊延安,改駐泊都監,又為鈐轄。其後繼遷復與熟戶李繼福為隙,因緣內擾,崇貴與張守恩擊之,焚廬舍,獲貲畜、器甲、生口甚眾。又
 與王榮禦賊,獲具裝馬數十匹,再詔褒飭。四年,詔歸。俄領獎州刺史,復涖鄜延,仍制置沿邊青白鹽事。與衛超領軍入敵境,焚廬舍帳幕,獲廩糗、牛羊,復被詔獎。崇貴屢詗契丹事傳遞以聞,願身當一隊為前鋒,詔不允。



 景德元年,保吉死,其子德明尚幼,崇貴移書諭朝廷恩信,德明請俟釋服稟命。詔書慰撫,以向敏中為緣邊安撫使。自是邊防事宜,經制小大,皆崇貴專主之。築臺保安北十里許,召戎人會議,與之盟約。二年春,召赴闕面授
 方略,許德明以定難節度、西平王,賜金帛緡錢各四萬、茶二萬斤,給內地節度奉,聽回圖往來,放青鹽禁,凡五事。而令德明納靈州土疆,止居平夏,遣子弟入宿衛,送略去官吏,盡散蕃漢兵及質口,封境之上有侵擾者稟朝旨,凡七事。德明悉如約,惟以子弟入質及納靈州為難,故亦禁鹽如舊,不許回圖。



 三年九月,以德明誓表來上,崇貴因請入朝,許之。以功拜皇城使、內侍左右班都知,領誠州團練使。又持旌節誥命授德明,太常博士趙
 湘為之副。四年,使還,會車駕上陵,次瓊林苑,崇貴對於苑中,即命為行宮使。是秋,復還延安。供奉官曹信時監邊軍,信善琴,崇貴與石普軍中宴集,令信奏之,信以久廢為辭。崇貴與普因摭其他過以聞,真宗知其誣奏,不問。大中祥符元年,加昭宣使。



 崇貴久在邊,善識羌戎情偽,西人畏服。每德明有所論述及境上交侵,皆先付裁制。夏州趣邊有二路,其文移至環慶者,皆付延州議焉。嘗請置緣邊安撫使,如北面之制。上曰:「西鄙別無經營,
 茍德明能守富貴,無慮朝廷失恩信也。增置署局,徒為張皇,不若委卿靜制之。」二年,上言久去鄉里,願得告歸葬父母。許之,錫與甚厚。復命為都鈐轄,提舉榷場。崇貴乞留京師,面諭委屬之意,聽歲入奏事。四年八月,卒,年五十七。帝悼惜之,贈豐州觀察使,內侍護喪還京師。子承素,東染院副使。



 張繼能,字守拙,並州太原人。父贊,晉末為內班。繼能,建隆初以黃門事禁中,太平興國初為內品。從征河東,命
 主城南洞屋,以勞遷高品。契丹入寇,命為高陽、鎮、定路先鋒都監,從崔彥進戰長城口,多所俘馘。明年,又與彥進敗契丹於唐興口,轉殿頭高品。



 雍熙中,夏州叛命李繼隆為銀、夏都部署,以繼能監軍。俄徙護定州屯兵,領驍捷卒三千,屯五回嶺。端拱初,遷入內殿頭,從趙保忠討李繼遷。保忠薦其有材,命與保忠同經略其事。代還,掌內弓箭庫。淳化三年,與白承睿護芻粟入靈武。會繼遷復寇邊,命繼能、承睿與知靈州侯延廣領驍卒五千,
 同主軍務,俄留為本州都監。及鄭文寶議城威州、清遠軍,繼能護其役。工畢,命與西京作坊副使張延洲同知軍事,又與田紹斌同掌積石砦。就遷內供奉官、靈環慶、清遠軍後陣都監,與西人轉鬥,敗走之。復還清遠。詣闕奏事,遷內殿崇班。未幾,拜供備庫副使,復遣護環州屯兵,徙涇、原、儀、渭都巡檢使。



 真宗即位,遷崇儀使、靈、環十州軍兵馬都監兼巡檢安撫使。咸平三年王均之亂,命為川峽兩路招安巡檢使。成都平,留為利州招安巡檢,
 尋召歸。會銀、夏寇警,復為邠寧駐泊都監。夏人寇清遠軍,營於積石河,繼能與楊瓊、馮守規在慶州逗留,不時赴援,致陷城堡,又焚棄青岡砦,特詔下御史府,免死,長流儋州。景德二年,會赦,還為內侍省內常侍,又為陜西捕賊巡檢,獲千餘人,改內殿崇班。從朝陵,為行宮四面巡檢。



 四年,宜州卒陳進為亂。初,知州劉永規馭下嚴酷,課澄海卒伐木葺州廨,數不中程即杖之,至有率妻孥趣山林以採者,雖甚風雨,不停其役。故進因眾怨,殺永
 規及監軍國鈞,擁判官盧成均為帥,據其城。



 七月,奏至,詔東上閣門使忠州刺史曹利用、供備庫使賀州刺史張煦為廣南東、西路安撫使,如京副使張從古及繼能副之,虞部員外郎薛顏同勾當轉運事,發荊湖蘄黃州兵討之。上語近臣曰:「番禺寶貨雄富,賊若募驍果,立謀主,沿流東下趣廣州,則為患深矣。」遣內侍高品周文質使廣州,監屯兵,會鄰路巡檢使控要路,集東西海戰棹,扼端州峽口。賊悉眾來攻柳城縣,殿直韓明、許貴、郝惟
 和以所部兵千餘御敵,明、貴死之,惟和僅以身免,成均奉宜州印遣使詣舒賁求赦罪。是夕,進復陷柳城,官軍退保象州。賊又寇懷遠軍,知軍殿直任吉與邕桂巡檢、殿直張崇寶、侍禁張守榮擊走之。賊退而復集者累日,吉輩固守,屢與鬥,大獲其器甲。又攻天河砦,砦兵甚少,監軍奉職錢吉部分嚴整,一戰敗之。賊眾屢衄,頗潰去,眾心的攜貳,將棄宜州,以家屬之悼耄者五百人隕江中,率其眾裁三千趣柳、象,將入容管。初至柳州,限江不能
 渡。知州王昱望賊遁走,城遂陷。



 朝廷以詔書四十分揭要路,諭賊歸順者悉釋其罪。賊挈族居思順州,分兵攻象州。利用命入內高班於德潤以千兵倍道襲逐,利用等繼至,遇賊武仙縣之李練鋪。賊初不知覺,惟進率眾來拒,直犯前軍,前軍寄班侍班郭志言麾騎士左右縱擊。賊衣順水甲、執標牌以進,飛矢攢鋒不能卻,前軍即持棹刀巨斧破其牌,史崇貴登山大呼曰:「賊走矣,急殺之!」賊心動,眾遂潰。逐北至象州城下,賊砦猶有據長
 竿瞰城中者,成均始挈其族以詔書來降,乃斬進並其黨,生擒賊帥六十餘人,斬首級、獲器甲戰馬甚眾。



 利用分兵捕餘寇,遣於德潤馳奏其事。授利用引進使,煦如京使,從古莊宅副使,繼能供備庫使,志言供備庫使。又以御前忠佐馬步軍副都軍頭郭全豐為都軍頭,領勤州刺史。歸遠軍士手殺進者李昊、劉宗、趙敏並補本軍都頭,張守榮為供奉官、閣門祗候,張崇寶、任吉並為供奉官,錢吉為右侍禁。又以知象州大理寺丞何邴最有勞,優
 拜祠部員外郎,賜緋。又賜邴三子知道、知古、知常出身,邴之親屬同捍寇者悉甄敘之。升象州為防禦使。



 初,賊攻象州,城在高丘上,素無井,閉壘之日,皆以乏水為慮。賴天雨,停水將竭而雨復下,如是者兩月,汲之以濟。山中無烽候,每欲破賊,即禱於城西神祠,或見巨蟒吞龜,是日果有克獲,眾以為神靈助順之應。張守榮俄病瘴,遣尚醫馳往視之,未至而卒,贈如京使,錄其子官。十二月,餘寇悉平。



 東封,留繼能為京舊城內巡檢鈐轄,俄加
 東染院使。



 大中祥符二年,入內都知李神祐等坐事悉罷,擢繼能入內內侍省副都知。時宗室多召侍講說書,上嘉其勤學,令講誦日別給公膳,專遣繼能主之。俄又與內殿承制岑保正提點郡縣主諸院事。三年,兼群牧都監。祀汾陰,留掌大內兼舊城內巡檢鈐轄,俄領會州刺史。謁太清宮,為天書扶侍都監。七年,以疾求解職,不許。命為涇原儀渭、鎮戎軍兩路鈐轄。未幾,徙鄜延都鈐轄。先是,內屬戶殺漢口者止罰孳畜,繼能則麗於常法,
 由是西人畏而不敢犯。德明雖受朝命,而羌部不絕寇境。繼能日課卒截竹為簽,署字其上,且言以備將士記殺獲功狀,賊聞之甚懼。歸朝,復涖群牧。仁宗在儲宮,嘗親書一幅賜之。繼能以聞,真宗亦為標題其末,人以為榮。九年,坐前護修莊穆皇后陵摧陷,左授西染院使,掌往來國信。



 天禧初,復西京左藏庫使。國信司吏陳誠者,頗巧黠,繼能欲援置群牧司,而誠先隸群牧,坐事停職。至是,群牧吏左宗抉其宿負,白制置使曹利用,故誠不
 遂所求。繼能怒宗之沮己,密遣親事卒偵宗。會宗弟元喪妻,宗嘗為假敦駿軍校馬送葬,及還,元抵飲肆與酒保相毆,系府中,而假馬之事未發。誠即白繼能,請屬府中並劾其事。知府樂黃目受屬,獄未就,為群牧副使楊崇勛所發,繼能坐罷內職,降授西京作坊使,出為邠寧鈐轄。繼能自陳不願外任,得掌瑞聖園,尋領往來國信所。三年,復為西京左藏庫使、內侍右班副都知。未幾,遷崇儀使,以衰老求解職,轉內園使,掌瓊林苑。五年,卒,年
 六十五。特贈汀州團練使,錄其子懷忠為大理寺丞,孫逖為三班奉職,遜為借職、春坊祗候。



 繼能性沉密知兵,頗勇敢,喜讀書,然好治生。晚年急於聚蓄,眾以此少之。



 何邴後歸朝,知磁州而卒。一子知崇裁十餘歲,特補太廟齋郎。又徙其侄平夷尉知古為滏陽尉。省郎無賞延之例,猶以城守勞,故甄錄焉。



 衛紹欽,開封人。父漢超,內侍高品。紹欽始以中黃門給事晉邸,太宗即位,補入內高品,甚被親倚。從征太原,命
 督諸將攻城。劉繼元降,命領驍卒先入城,燒其營柵,遷殿頭高品。雍熙二年,擢入內西頭供奉官。淳化中,部修皇城,功畢,授入內押班。五年,加崇儀副使。



 李順之亂,王師致討,與王繼恩同領招安捉賊事,遇賊,鬥學射山南。又攻清水埧,破雙流砦,招降數萬眾,斬千餘級。順死,餘黨保險為寇,又與楊瓊先扼要路以邀之,擒斬萬餘人,獲器甲槍槊千餘。遣別將曹習領兵捕餘賊於安國鎮,斬三百級。時嘉、眉二州賊尚擾城郭,又遣內殿崇班宿
 翰討之。兩川平,召還,深被褒勞。



 真宗嗣位,拜宮苑使,領愛州刺史,充入內副都知、修奉永熙陵都監,即復土,遂為陵使。景德二年,改皇城使。從幸河朔,命為車駕前後行宮四面都巡檢。次澶淵,命領扈駕兵守河橋。三年,加昭宣使。朝諸陵,復為行宮巡檢。駐洛陽,命為皇城內外都巡檢。歷掌三班院、皇城儀鸞翰林司。卒,年五十六。



 紹欽苛愎少恩,不為眾所附。太平興國中,江東有僧詣闕請修天臺壽昌寺,且言寺成願焚身以報。太宗允其請,命
 紹欽往督營繕。既訖役,遽積薪於廷,請僧如願,僧言欲見至尊面謝,紹欽曰:「昨朝辭日,親奉德音,不煩致謝。」僧惴怖偃蹇,顧道俗望有救之者,紹欽即促令躋薪上,火既盛,僧欲投下,紹欽遣左右以叉抑按而焚之。子承慶,至內殿承制。



 石知顒,真定人。曾祖承渥,梁尚食使。祖守忠,晉內供奉官。父希鐸,高品。



 知顒形貌甚偉,建隆中授內中高品。太宗即位,改供奉官。雍熙中,諸將征幽薊,以知顒隨軍。歸,
 掌儀鸞司。



 淳化中,明州初置市舶司,與蕃商貿易,命知顒往經制之。轉內殿崇班、親王諸宮都監。從王繼恩平蜀寇,就遷西京作坊副使。



 咸平初,遷正使、帶御器械。契丹犯邊,上北巡,命為天雄軍、澶州巡檢使,俄改德、博等州緣河巡檢使兼安撫,加領長州刺史。三年,戍鎮、定、高陽關三路,押大陣。是冬,改高陽關駐泊行營鈐轄。歸朝,復掌親王諸宮事。景德中,自京抵泗,遣徒治河堤,命總其役。初計工累月,及是,浹日而畢。上面加褒諭,賜白金
 千兩,授入內都知。



 大中祥符初,遷內園使。俄以定內侍遷秩品第不當,為其列所誣,坐罷都知。三年,為並、代州鈐轄,遷莊宅使,徙鎮、定、高陽關鈐轄。四年,命與內殿崇班張繼能、供奉官侍其旭同修太祖神御殿。上封求覲闕下,復掌群牧司、三班院、親王諸宮事。



 天禧二年,為並、代州鈐轄兼管勾麟府路軍馬事。三年,卒,年六十九。孫全彬。



 全彬字長卿,以知顒奏補入內小黃門,累遷西頭供奉
 官。仁宗使致香幣於南海,密詔察所過州縣吏治民俗,還,具以對,帝以為忠謹。陜右群盜殺鳳州巡檢,遣往擒滅之。



 元昊叛,全彬監鄜州兵救延州,解圍去。經略使明鎬言其勇略善將,得邊人情,除並、代州都監,加內侍押班。進鈐轄,徙鄜延,還,為押班。



 儂智高寇廣南,以為湖南、江西路安撫副使。出桂林,請於宣撫使狄青,願獨當一隊以自效。於是使將左方兵,力戰於邕州。南方平,領綿州防禦使。



 張貴妃居寧華殿閣,命全彬提舉。妃薨,治喪
 過制,皆劉沆、王洙與全彬共為之。數月,進宮苑使、利州觀察使,給兩使留後奉。俄為入內副都知,知制誥劉敞封還詞命,居三月,復授之。轉領信武軍留後,為永昭陵鈐轄。時去永定復土四十二年,有司多亡其籍,全彬以心計辦治。遷福延宮使,提點奉先院。



 熙寧中,卒,年七十六。贈太尉、定武軍節度使,謚曰恭僖。



 鄧守恩,並州人。十歲以黃門事太宗。淳化中,盜起成都,從王繼恩往討之。至道初,就護西蜀屯兵。咸平初,為入
 內高班。契丹入寇,命石保吉為鎮、定都部署,以守恩為都監。逾年,入掌騏驥院。會龍騎叛卒剽劫環、慶,遣守恩擒翦之。景德初,為澶、濮都巡檢。又使環、慶及戎、瀘等州巡察邊事。



 大中祥符初,按獄於濮州,雪冤人十餘。預監修玉清昭應宮、會靈觀。七年,又兼修真游殿、景靈宮。累遷入內高品、供奉官。宮成,遷內殿承制。八年,預修大內,改西京作坊副使。九年,營造皆畢,授東染院使,充會靈觀都監。



 天禧二年,掌軍頭引見司,又修祥源觀成,遷崇
 儀使。三年,授入內押班。河決滑州,命為修河鈐轄。郊祀,召為行宮使,改如京使,復還本任。四年春,河復故道,遷文思院使。歸朝,加領昭州刺史。是秋,掌皇城、國信二司,整肅禁衛,遷入內副都知。會建天章閣,命領其事。又勾當資善堂兼太子左右春坊司。



 守恩長七尺餘,狀貌甚偉,涖事幹敏,以強果稱於時。五年,卒,年四十八。贈淄州防禦使。錄其子官。



\end{pinyinscope}