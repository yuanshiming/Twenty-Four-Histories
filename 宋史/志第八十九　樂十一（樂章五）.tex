\article{志第八十九 樂十一(樂章五)}

\begin{pinyinscope}

 祀岳鎮海瀆祀大火祀大辰



 大中祥符五岳加帝號祭告八首



 迎神,《靜安》



 鐘石既作,俎豆在前。雲旗飛揚,神光肅然。



 當駕飆欻,來乎青圓。言備縟禮,享茲吉蠲。



 冊入門,《正安》



 節彼喬嶽,神明之府。秩秩威儀,肅肅靈宇。



 懿號克崇,庶物咸睹。帝籍升名,式綏九土。



 酌獻東嶽,《嘉安》



 節彼岱宗,有嚴廟貌。惟闢奉天,依神設教。



 帝典焜煌,嘉薦普淖。至靈格思,殊祥是效。



 南嶽



 作鎮炎夏,畜茲靈光。敷與萬物,既阜既昌。



 爰刻溫玉,式薦徽章。昭嘏神意,福熙穰穰。



 西嶽



 瞻言太華,奠芳作鎮。典冊是膺,等威以峻。



 上公
 奉儀,祀宗薦信。介祉萬邦,永配坤順。



 北嶽



 仰止靈岳,鎮於朔方。增崇懿號,度越彞章。



 祗薦嘉樂,式陳令芳。永資純祐,國祚蕃昌。



 中嶽



 巖巖神岳,作鎮中央。肅奉徽冊,尊名孔章。



 聿降飆駕,載獻蘭觴。熙事允洽,寶祚彌昌。



 送神,《靜安》



 祗薦鴻名,寅威明祀。有楚之儀,如在之祭。



 奠獻既終,禮容克備。神鑒孔昭,福禧來暨。



 天安殿冊封五岳帝一
 首



 冊出入,《正安》



 名岳奠方,帝儀克舉。吉日惟良,九賓咸旅。



 溫玉鏤文,纁裳正寧。禮備樂成,篤神之祜。



 熙寧望祭嶽鎮海瀆十七首



 東望迎神,《凝安》



 盛德惟木,勾芒御神。沂岱淮海,厥功在民。



 爰熙壇坎,裒對庶神。於以歆格,靈貽具臻。



 升降,《同安》



 紳□衣詹兮,玉佩蕊兮。於我將事,神燕喜兮。



 帝命望祀,敢有不共。往返於位,肅肅雍雍。



 奠玉幣,《明安》



 祀以崇德,幣則有儀。肅我將事,登降孔
 時。



 精明純潔,罔有弗祗。史辭無愧,神用來娭。



 酌獻,《成安》



 肇茲東土,含潤無疆。維時發春,喜薦令芳。



 祭用蒯薶沉,順性含藏。不涸不童,誕降祺祥。



 送神,《凝安》



 神之至止,熙壇為春。神之將歸,旗服振振。



 欻兮回飆,窅兮旋雲。祐於東方,永施厥仁。



 南望迎神,《凝安》



 嵩嵇衡霍,暨厥海江。時維長養,惠我南邦。



 肆嚴牲幣,神式來降。以侑以妥,百福是龐。



 酌獻,《成安》



 景風應律,朱鳥開辰。肅肅明祀,嘉籩列陳。



 牲用牷物,樂奏蕤賓。克綏永福,祐此下民。



 送神,《凝安》



 鼓鐘雲云,龠□管伊伊。神既醉飽,曰送言歸。



 山有厚藏,水有靈德。物其永依,往奠炎宅。



 中望迎神,《凝安》



 維土作德,維帝御行。含養載育,萬物以成。



 有嚴祀典,薦我德馨。神其歆止,永用億寧。



 酌獻,《成安》



 高廣融結,實維中央。宣氣報功,利彼一方。



 坎壇以祀,六樂鏘鏘。靈其有喜,酌以大璋。



 送神,《凝安》



 言旋其處,以奠中域。無替厥靈,四方是則。



 神永不息,祀永不愆。以享以報,於萬斯年。



 西望迎神,《凝安》



 品物順說,時司金行。於郊迎氣,以望庶靈。



 雅歌維樂,圭薦惟牲。作民之祉,永相厥成。



 酌獻,《成安》



 西顥沆碭,執矩司秋。諏言協靈,時祀孔修。



 禮有薦獻,爰視公侯。秩而祭之,百福是遒。



 送神,《凝安》



 我樂我神,簋俎腥饔。曰神之還,西土是宮。



 於蕃禽魚,於衍草木。富我藪隰,滋我高陸。



 北望迎神,《凝安》



 帝德乘坎,時御閉藏,爰潔牷醴,兆茲
 北方。



 海山攸宅,神施無疆。具享蠲吉,降福孔穰。



 酌獻,《成安》



 淒寒凝陰,隕籜滌場。百物順成,黍稷馨香。



 款於北郊。爰因其方。何以侑神?薦此嘉觴。



 送神,《凝安》



 維山及川,奠宅幽方。我度其靈,降止靡常。



 肅肅坎壇,既迎既將。促樂徹俎,是送是望。



 紹興祀岳鎮海瀆四十三首



 東方迎神,《凝安》



 帝奠九□廛,孰匪我疆。系我東土,山川相望。



 祀事孔時,肅雍不忘。嶪峨蒙鴻,鬱哉洋洋!



 初獻盥洗,《同安》



 青陽肇開,祀事孔飭。鬱人贊溉,其馨苾苾。



 敬爾威儀,亦孔之則。神之格思,無我有斁。



 奠玉幣,《明安》



 司歷告時,惟孟之春。爰舉時祀,旅於有神。



 鼓鐘既設,珪帛具陳。阜蕃庶物,以福我民。



 東岳位酌獻,《成安》



 巖巖天齊,自古在昔。膚寸之云,四方其澤。



 惟時東作,祀事乃飭。惠我無疆,恩沾動植。



 東鎮位



 惟山有鎮,雄於其方。東孰為雄?於沂之疆。



 祀事有時,爰舉舊章。我望匪遙,庶幾燕饗。



 東海位



 澒洞鴻蒙,天與無極。導納江、漢,節宣南北。



 順助其功,善下惟德。我祀孔時,以介景福。



 東瀆位



 我祀伊何?於彼長淮。導源桐柏,委注蓬萊。



 捍齊護楚,宣威示懷。豆籩列陳,亦孔之偕。



 亞、終獻,酌獻四位並同。



 我祀孔肅,神其安留。容與裴回,若止若浮。



 洽此重觴,申以百羞。無我斁遺,萬邦之休。



 送神,《凝安》



 蹇兮紛紛,神實戾止。以飲以食,以享以祀。



 □幼兮冥冥,神亦歸止。以醉以飽,以錫爾祉。



 南方迎神,《凝安》



 朱明盛長,我祀用飭。厥祀伊何?山川咸秩。



 如將見之,繩繩齊慄。神哉沛兮,消搖來格!



 初獻盥洗、升降,《同安》



 爰熙嘉壇,揭虔毖祀。鬱人沃盥,贊我稞事。



 於降於登,以作以止。莫不肅雍,告靈饗矣。



 奠玉幣,《明安》



 我祀我享,儀物孔周。一純斯舉,二精聿修。



 璞兮其溫,絲兮其紑。是薦潔蠲,神兮安留。



 南岳位酌獻,《成安》



 神曰司天,居南之衡。位焉則帝,於以奠方。



 南訛秩事,望禮有常。庶幾嘉虞,介福無疆。



 南鎮位



 維南有山,於彼會稽。作鎮在昔,神則司之。



 厥有舊典,以祀以時。百味維旨,靈其燕娭。



 南海位



 維水善下,利物曰功。逶迤百川,誰歟朝宗?



 蕩蕩大受,於焉會同。膋蕭列陳,以答鴻蒙。



 南瀆位



 四瀆之利,經營中國。南曰大江,險兮天設。



 維爾有神,隃其廟食。望秩孔時,我心翼翼。



 亞、終獻,酌獻



 神之游兮,洋洋對越。澹乎容與,肸蠁斯答。



 乃奏既備,八音攸節。重觴申陳,百禮以洽。



 送神曲同迎神



 薦徹豆籩,熙事備成。靈兮將歸,羽旄紛紜。



 飄其逝矣,浮空雲。悵然顧瞻,有撫懷心。



 中央迎神,《凝安》



 天作高山,屹然中峙。經營厥宇,萬億咸遂。



 火熙土王,爰舉時祀。繩繩宣延,徬佛來止。



 初獻盥洗、升降,《同安》



 思來感格,肅雍不忘。禮儀既備,濟濟蹌蹌。



 潔蠲致敬,往薦其芳。交若有承,神兮孔饗。



 奠玉幣,《明安》



 練日有望,高靈來下。何以告誠?心惟物假。



 有篚斯實。有寶斯籍。於以奠之,神光燭夜。



 中岳位酌獻,《成安》



 與天齊極,伊嵩之高。顯靈效異,神休孔昭。



 飭我祀事,實俎鸞膋。以侑旨酒,其馨有椒。



 中鎮位



 禹畫九州,河內曰冀。霍山崇崇,作鎮積勢。



 我祀如何?百末旨味。承神燕娭,諸神畢至。



 亞、終獻,酌獻



 禮樂既成,肅容有常。奄留消搖。申畢重觴。



 仰臚所求,降福滂洋。師象山則,以水兄皇章。



 送神曲同迎神



 虞至旦兮,靈亦有喜。蹇欲驤兮,像輿已轙。



 粥音送兮,靈聿歸矣。長無極兮,錫我以祉。



 西方迎神,《凝安》



 有岌斯安,有涵斯洽。聿相厥成,允祀是答。



 爰飭乃奏,乃奏既協。於昭降止,是遵是接。



 初獻盥洗、升降,《同安》



 靡實不新,靡陳不濯。人之弗蠲,矧敢將酌。



 載晞之帨,載濡之勺。洗儀告備,陟降時若。



 奠玉幣,《明安》



 彼林有,彼澤有沉。猗與西望,弗菲弗淫。



 乃追斯邸,乃□斯尋。仰禮既卒,是用是歆。



 西岳位酌獻,《成安》



 屹削厥方,風雲斯所。陰邑有宮,侐□俁俁。



 清酤在尊,靈慎在下。於俎獻兮,則莫我吐。



 西鎮位



 維吳崇崇,於水幵之西。瞻彼有隴,赫赫不迷。



 克裨於嶽,我酌俶齊。於凡有旅,視公維躋。



 西海位



 奄浸坤軸,滋殖其濊。而典斯稽,有陛有壝。



 弗替時舉,元斝斯酹。胡先於河?實委之會。



 西瀆位



 自彼昆虛,於以潛流。念茲誕潤,豈侯不猶。



 在昔中府,暨海聿修。迄既望止,神保先卣。



 亞、終獻



 肅肅其乂,既旨既溢。迨其畢酌,偏茲博碩。



 祀事既遂,不敢誶射。神或醉止,我心斯懌。



 送神曲同迎神



 乃羞既徹,乃奏及闋。無餕斯俎,式聽致謁。



 不蹇不蹶,不沸不決。厲魃其祛,永庇有截。



 北方迎神,《凝安》



 我土綿綿,孰匪疆理。惟時幽都,匪曰隃只。



 滌哉艮月,朔風其同!曷阻曷深,其亦來降。



 初獻盥洗、升降,《同安》



 壽宮輝煌,聿修時祀。繽其臨矣,吉蠲以俟。



 居乎昂昂,行乎遂遂。敬爾攸司,展採錯事。



 奠玉幣,《明安》



 相予陰威,厥功浩浩。一歲之功,何以為報?



 府有珪幣,我其敢私!肅肅孔懷,於以將之。



 北岳位酌獻,《成安》



 瞻彼芒芒,曰北之常。既高既厚,乃紀乃綱。



 薦鬯伊始,靈示孔將。玄服鐵駕,覽此下方。



 北鎮位



 赫赫作鎮,幽、朔之垂。兼福我民,食哉具宜。



 克配彼岳,有嚴等衰。蠲我灌禮,其敢不祗!



 北海位



 八裔皆水,此一會同。澐□天墟,洞蕩洪蒙。



 至哉維坎,不有斯功!所秩伊何?黃流在中。



 北瀆位



 水星之精,播液發靈。不脅於河,既介以清。



 翼翼盥薦,椒糈芬馨。載止載留,爰弭翠旌。



 亞、終獻



 俎豆紛披,金石繁會。侑以貳尊,匪瀆匪怠。



 我儀既周,我心孔戒。憺兮容與,徬佛如在。



 送神曲同迎神



 靈既醉飽,禮斯徹兮。靈亦樂康,樂斯闋兮。



 雲征飆舉,不可尼兮。薦福錫祉,曷有極兮!



 淳祐祭海神十六首



 迎神,《延安》



 宮一曲



 堪輿之間,最鉅惟瀛。包乾括坤,吐日滔星。



 祀典載新,禮樂孔明。鑒吾嘉賴,來燕來寧。



 角一曲



 四溟廣矣,八紘是紀。我宅東南,回復萬里。



 洪
 濤飆風,安危所倚。祀事特隆,神其戾止!



 征一曲



 若稽有唐,克致崇極。祝號既升,爰增祭式。



 從享於郊,神斯受職。我祀肇新,式祈陰騭。



 羽一曲



 猗與祀禮,四海會同!靈之來沛,鞭霆馭風。



 肸蠁徬佛,在位肅雍。祐我烝民,式徼神功。



 升降,《欽安》



 靈之來至,垂慶陰陰。靈之已坐,飭茲五音。



 壇殿聿嚴,陟降孔欽。靈宜安留,鑒我德心。



 東海位奠玉幣,《德安》



 百川所歸,天地之左。水項洞鴻蒙,
 功高善下。



 行都攸依,百祿是荷。制幣嘉玉,以侑以妥。



 南海位奠玉幣,《瀛安》



 祝融之位,貴乎三神。吞納江、漢,廣大無垠。



 長為委輸,祐我黎民。敬陳明享,允鑒恭勤。



 西海位奠玉幣,《潤安》



 蒲菖之澤,派引天潢。羲娥出入,浩渺微茫。



 蓋高斯覆,猶隔封疆。我思六合,肇正吉昌。



 北海位奠玉幣,《瀚安》



 瀚海重潤,地紀亦歸。吞受百瀆,限制北陲。



 一視同仁,我心則怡。嘉薦玉幣,神其格思。



 捧俎,《豐安》



 昭格靈貺,祀典肇升。牲牷告充,雕俎是承。



 薦虔效物,省德惟馨。靈其有喜,萬宇肅澄。



 東海位奠酌獻,《熙安》



 滄溟之德,東南具依。熬波出素,國計攸資。



 石臼卻敵,濟我王師。神其享錫,益畀燕綏。



 南海位酌獻,《貴安》



 南溟浮天,旁通百蠻。風檣迅疾,琛舶來還。



 民商永賴,坐消寇奸。薦茲嘉觴,弭矣驚瀾。



 西海位酌獻,《類安》



 積流疏派,被於流沙。布潤施澤,功均邇遐。



 我秩祀典,四海一家。祗薦令芳,靈其享嘉!



 北海位酌獻,《溥安》



 壝忽會同,裴回安留。牲肥酒香,晨
 事聿修。



 惟德之涼,曷奄九州?帝命是祗,多福自求。



 亞、終獻,《饗安》



 籩豆有楚,貳觴斯旅。神其醉飽,式燕以序。



 百靈秘怪,蜿蜒飛舞。錫我祺祥,有永終古。



 送神,《成安》



 告靈饗矣,錫我嘉祚。乾端坤倪,開豁呈露。



 玄云聿收,群龍咸騖。減除兇災,六幕清豫。



 紹興祀大火十二首



 降神,《高安》



 圜鐘為宮



 五緯相天,各率其職。司禮與視,則維熒惑。



 至陽之精,屆我長嬴。於以求之,祀事孔明。



 黃鐘為角



 有出有藏,伏見靡常。相我國家,鑒觀四方。



 視罔不正,終然允臧。神其來格,明德馨香。



 太簇為征



 小大率禮,不愆於儀。展採錯事,秩祀孔時。



 維今之故,閱我數度。修厥典常,神其來顧!



 姑洗為羽



 於赫我宋,以火德王。永永丕圖,繄神之相。



 神之來矣,維其時矣。禮備樂奏,神其知矣。



 升殿,《正安》



 有儼其容。有潔其衷。屹屹崇壇,伊神與通。



 神肯降格,嘉神之休。虔恭降登,神乎安留。



 熒惑位奠玉幣,《嘉安》



 馨香接神,肸蠁恍惚。求神以誠,薦誠以物。



 有藉斯玉,有篚斯幣。是用薦陳,昭茲精意。



 商丘宣明王位奠幣,《嘉安》



 熒惑在天,惟火與合。繄神主火,純一不雜。



 作配熒惑,祀功則然。不腆之幣,於以告虔。



 捧俎,《豐安》



 火遵其令,無物不長。視此牲牢,務得其養。



 豢以祀神,有腯其肥。非神之宜,其將曷歸?



 熒惑位酌獻,《祐安》



 皇念有神,介我戩穀。登時休明,有
 此美祿。



 酌言獻之,有飶其香。神兮燕娭,醉此嘉觴。



 宣明王位酌獻,《祐安》



 誰其祀神?知神嗜好。閼伯祀火,為神所勞。



 睠言配食,既與火俱。於樂旨酒,承神嘉虞。



 亞、終獻,《文安》



 神既貺施,嗜我飲食。申以累獻,以承靈億。



 神方常羊,咸畢我觴,於再於三,於誠之將。



 送神用《理安》



 登降上下,奠璧獻斝。音送粥粥,禮無違者。



 已虞至旦,神其將歸。顧我國家,遺以繁厘。



 出火祀大辰十二首



 降神,《高安》



 圜鐘為宮



 燁燁我宋,火德所畀。用火紀時,允惟象類。



 神以類歆,誠繇類至。有感斯通,孚我陽燧。



 黃鐘為角



 樂音上達,粵惟出虛。火性炎上,亦生於無。



 我鏞我磬,我笙我竽。氣同聲應,昭哉合符!



 太簇為征



 火在六氣,獨處其兩。感生維君,繄辰克相。



 何以驗之?占茲垂象。騰駕蒼虯,欻其來饗。



 姑洗為羽



 星入於戌,與火俱詘。火出於辰,與星俱伸。



 一伸一詘,孰操縱之?利用出入,民咸用之。



 升殿,《正安》



 屹彼嘉壇,赤伏始屆。掞光耀明,洋乎如在。



 俯仰重《離》,默與精會。隨我降升,肅聽環佩。



 大辰位奠玉幣,《嘉安》



 維莫之春,五陽發舒。日之夕矣,三星在隅。



 莫量匪幣,莫嘉匪玉。明薦孔時,神光下矚。



 商丘宣明王奠幣,《嘉安》



 二七儲神,與天地並。孰儷厥德?聿惟南正。



 功楙陶唐,澤流億姓。作配嚴禋,贄列惟稱。



 捧俎,《豐安》



 有嚴在滌,陳彼牲牢。孔碩其俎,薦此血毛。



 厥初生民,飲茹則然。以燔以炙,伊誰雲先?



 大辰位酌獻,《祐安》



 孰為大辰?維北有斗。曾是彗星,斯名孔有。



 幽榮報功,潔齊敢後。容與嘉觴,式歆旨酒。



 宣明王位酌獻,《祐安》



 周設司爟,雖列夏官。仍襲孔易,闓端實難。



 相彼商丘,永懷初造。不腆桂椒,匪以為報。



 亞、終獻,《文安》



 潛之伏矣,柞□□既休。有俶其來,榆柳是求。



 靈駕紛羽,尚其安留。飲我三爵,言言油油。



 送神,《理安》



 五運惟火,寔宗眾陽。宿壯用明,千載愈光。



 神保聿歸,安處火房。鬱攸不作,炎圖永昌。



 納火祀大辰十二首



 降神,《高安》



 圜鐘為宮



 赫赫皇圖,炎炎火德。侈神之賜,奄有方國。



 粢盛既豐,俎豆有飶。於萬斯年,報祀無斁。



 黃鐘為角



 火星之躔,有燁其光。表於辰位,伏於戌方。



 時和歲稔,仁顯用藏。告爾萬民,出納有常。



 太簇為征



 季秋之月,律中無射。農事備收,火功告畢。



 克禋克祀,有嚴有翼。風馬雲車,尚其來格!



 姑洗為羽



 明明我後,重祭欽祠。有司肅事,式薦晨儀。



 禮惟其稱,物惟其時。神之聽之,福祿來為。



 升殿,《正安》



 猗與明壇,右平左戚!冕服斯皇,玉佩有節。



 陟降惟寅,匪徐匪疾。式崇大祀,禮文咸秩。



 大辰位奠玉幣,《嘉安》



 金行序晚,玉露晨清。齊戒豐潔。肅恭神明。



 嘉幣惟量,嘉玉惟精。於以奠之,庶幾來聽。



 商丘宣明王位奠幣,《嘉安》



 恭惟火正,自陶唐氏。邑於商丘,配食辰祀。



 有功在民,有德在位。敢替典常。惟恭
 奉幣。



 捧俎,《豐安》



 萬匯攸成,四方寧謐。工祝致告,普存民力。



 乃薦斯牲。為俎孔碩。介以繁祉,式和民則。



 大辰位酌獻,《祐安》



 庶功備矣,休德昭明。天地釀和,鬱鬯斯清。



 玉瓚以酌,瑤觴載盈。周流常羊,來燕來寧。



 宣明王位酌獻,《祐安》



 廣大建祀,式崇其配。馨香在茲,清酒既載。



 穆穆在暉,洋洋如在。聿懷嘉慶,繄神之繼。



 亞、終獻,《文安》



 幣玉肅陳,笙簧具舉。桂醑浮觴。瓊羞溢
 俎。



 禮有三獻,式和且序。神具醉止,慶流寰宇。



 送神,《理安》



 神靈降鑒,天地回旋。惟馨薦矣,既醉歆焉。



 諸宰斯徹,式禮莫愆。隤祉降祥,天子萬年。



\end{pinyinscope}