\article{列傳第一百四十三}

\begin{pinyinscope}

 陳康伯梁克家汪澈葉義問蔣芾葉顒葉衡



 陳康伯,字長卿,信之弋陽人。父亨仲,提舉江東常平。康伯幼有學行。宣和三年,中上舍丙科。累遷太學正。丁內
 艱。貴溪盜將及其鄉,康伯起義丁逆擊,俘其渠魁,邑得全。



 建炎末,為敕令刪定官,預修《紹興敕令》。尋通判衢州,攝郡事。盜發白馬原,康伯督州兵濟王師進討,克之。除太常博士,改提舉江東常平茶鹽。高宗進蹕建康,康伯以職事過闕,得對,因請擇將,上開納。



 紹興八年,除樞密院大計議官。累遷戶部司勛郎中。康伯與秦檜太學有舊,檜當國,康伯在郎省五年,泊然無求,不偷合。十三年,始遷軍器監。借吏部尚書使金,至汴將晡,不供餉,閉戶
 臥勿問;入夜,館人扣戶謝不敏,亦不對。後因金使至,詔康伯館伴,端午賜扇帕,與論拜受禮,言者以生事論,罷知泉州。



 海盜間作,朝廷遣劉寶、成閔逐捕,康伯以上意招懷,盜多出降,籍為兵。久之,不逞者陰倡亂,康伯訊得實,論殺之,州以無事。秩滿,三奉祠,垂十年。



 檜死,起知漢州,將出峽,召對,除吏部侍郎。康伯首請節用寬民,凡州縣取民無藝,許監司互察,臺諫彈劾。尋兼禮、戶部。乞約歲用,會所入,儲什之一二備水旱。奏上,議竟不決。兼刑
 部。前此有司希檜意興大獄,康伯平讞直冤,士大夫存歿多賴之。除吏部尚書。宰臣擬用「權尚書」出命,高宗顧曰:「朕且大用,何『權』為?」尋拜參知政事。



 自孫道夫使北還,已聞金以買馬非約為言,朝廷特恃和,康伯與同知樞密院事王綸白發其端綸使還,乃言和好無他,康伯持初論不變。九月,以通奉大夫守尚書右僕射、同中書門下平章事,例賜銀絹,康伯固辭,減半,又辭。兼史院。上嘗謂其「靜重明敏,一語不妄發,真宰相也。」又命與湯思退
 輔政,事勿憚商論,惟其當而已。康伯言:「大臣事當盡公,若依阿植黨,此鄙夫患失者,臣非惟不敢,亦素不能。」高宗嘆其長者。普安郡王居潛藩,高宗一日謂康伯,當以使相封真王,今宜寇以屬籍,於是詔以為皇子,封建王,實三十年二月也。



 明年三月,拜光祿大夫、尚書左僕射。五月,金遣使賀天申節,出嫚言,求淮、漢地,指取將相大臣,且以淵聖兇問至。康伯主禮部侍郎黃中之論,持斬衰三年。先是,葉義問、賀允中使還,言金必敗盟,康伯請
 早為之備,建四策:一,增劉錡荊南軍,以重上流;二,分畫兩淮地,命諸將結民社,各保其境;三,劉寶獨當淮東,將驕卒少,不可倚;四,沿江諸郡修城積糧,以固內地。至是,召三衙帥及楊存中至都堂議舉兵,又請侍從、臺諫集議,康伯傳上旨曰:「今日更不問和與守,直問戰當如何。」時上意雅欲視師,內侍省都知張去為陰沮用兵,且陳退避策,中外妄傳幸閩、蜀,人情洶洶。右相朱倬無一語,同知樞密院事周麟之受命聘金,憚不欲行,康伯獨以
 為己任,奏曰:「金敵敗盟,天人共憤,今日之事有進無退,聖意堅決,則將士之意自倍。願分三衙禁旅助襄、漢,待其先發應之。」康伯勉周麟之以國事,麟之語侵康伯,康伯曰「使某不為宰相,當自行,大臣與國存亡,雖死安避。」麟之竟以辭行罷,尋貶責。殿中侍御史陳俊卿言當用張浚,且乞斬張去為以作士氣。康伯以俊卿振職,奏權兵部侍郎。



 九月,金犯廬州,王權敗歸,中外震駭,朝臣有遣家豫避者。康伯獨具舟迎家入浙,且下令臨安諸城
 門扃鐍率遲常時,人恃以安。敵迫江上,召楊存中至內殿議之,因命就康伯議。康伯延之入,解衣置酒,上聞之已自寬。翌日,入奏曰:「聞有勸陛下幸越趨閩者,審爾,大事去矣,盍靜以待之。」



 一日,忽降手詔:「如敵未退,散百官。」康伯焚之而後奏曰:「百官散,主勢孤矣。」上意既堅,請下詔親征,以葉義問督江、淮軍,虞允文參謀軍事。上初命朱倬為都督,倬辭,乃命義問。允文尋敗敵於採石,金主亮為其臣下所斃而還。



 方亮之犯江,國人即立葛王褒。
 三十二年,始遣高忠建來告登位,議授書禮,康伯以誼折之,於是報書始用敵國禮。



 高宗倦勤,有與子意,康伯密贊大議,乞先正名,俾天下咸知聖意,遂草立太子詔以進。及行內禪禮,以康伯奉冊。孝宗即位,命兼樞密使,進封信國公,禮遇殊渥,但呼丞相而不名。



 康伯自建康扈從回,即以病祈去位,不允。明年,改元隆興,請益堅,遂以太保、觀文殿大學士、福國公判信州。上慰勞甚勤,且曰:「有宣召,慎勿辭。」宰執即府餞別,百官班送都門外。己
 又辭郡,丐外祠,除醴泉觀使。



 二年八月,起判紹興府,且令赴闕奏事,復辭。未幾,召陪郊祀。時北兵再犯淮甸,人情驚駭,皆望康伯復相。上出手札,遣使即家居召之。未出里門,拜尚書左僕射、同中書平章事兼樞密使,進封魯國公。親故謂康伯實病,宜辭,康伯曰:「不然。吾大臣也,今國家危,當輿疾就道,幸上哀而歸之爾。」道聞邊遽,兼程以進,至闕下,詔子安節、婿文好謙掖以見,減拜賜坐。間日一會朝,許肩輿至殿門,仍給扶,非大事不署。敵師
 退,尋以目疾免朝謁,臥家,旬餘一奏事。



 乾道元年正月上辛,有事南郊,康伯起陪祠,已即丐歸,章屢上,不許。一日出殿門,喘劇,輿至第薨,年六十有九。贈太師,謚文恭,擇日臨奠,子偉節固辭,乃止。命工部侍郎何輔護喪歸。



 二子:偉節,除直秘閣;安節,賜同進士出身,五辭不受,上手札批諭,寄留省中以成其美,康伯薨,給還之。慶元初,配享孝宗廟庭,改謚文正。



 梁克家,字叔子,泉州晉江人。幼聰敏絕人,書過目成誦。
 紹興三十年,廷試第一,授平江簽判。時金主亮死,眾皆言可乘機進取,克家移書陳俊卿,謂:「敵雖退,吾兵力未振,不量力而動,將有後悔。」俊卿歸以白丞相陳康伯,嘆其遠慮。召為秘書省正字,遷著作佐郎。



 時災異數見,克家奏宜下詔求言,從之,令侍從、臺諫、卿監、郎官、館職疏闕失。克家條六事:一正心術,二立紀綱,三救風俗,四謹威柄,五定廟算,六結人心。其論定廟算,謂今邊議不過三說,曰將、兵、財,語甚切直。累遷中書舍人。



 使金,金以中
 朝進士第一,敬待之,即館宴射,連數十發中的。金人來賀慶會節,克家請令金使入朝由南門,百官由北門,從者毋輒至殿門外,以肅朝儀,詔定為令。



 郊祀有雷震之變,克家復條六事。遷給事中,凡三年,遇事不可,必執奏無隱。嘗奏:「陛下欲用實才,不喜空言,空言固無益,然以空言為懲,則諫爭之路遂塞,願有以開導之。」上欣納,因命條具風俗之弊,克家列四條,曰欺罔、茍且、循默、奔競,上手筆將諭。



 乾道五年二月,拜端明殿學士、簽書樞密
 院事。明年,參知政事。又明年,兼知院事。初修金好,金索所獲俘,啟釁未已。克家請築楚州城,環舟師於外,邊賴以安。在政府,與虞允文可否相濟,不茍同。皇太子初立,克家請選置官屬,增講讀員,遂以王十朋、陳良翰為詹事,中外稱得人。允文主恢復,朝臣多迎合,克家密諫,數不合,力丐去。上曰:「兵終不可用乎?」克家奏:「用兵以財用為先,今用度不足,何以集事?」上改容曰:「朕將思之。」詰朝,上面諭曰:「朕終夜思卿言,至當,毋庸去。」



 八年,詔更定僕
 射為左右丞相,拜克家為右丞相兼樞密使。一日,上謂宰執曰:「近過德壽宮,太上頤養愈勝,天顏悅懌,朕退不勝喜。」克家奏:「堯未得舜以為己憂,既得舜,固宜甚樂。」允文奏:「堯獨高五帝之壽以此。」上曰:「然。」允文既罷相,克家獨秉政,雖近戚權幸不少假借,而外濟以和。張說入樞府,公議不與,寢命,俄復用。說怒士夫不附己,謀中傷之,克家悉力調護,善類賴之。



 議金使朝見授書儀,時欲移文對境以正其禮,克家議不合,遂求去,以觀文殿大學
 士知建寧府。陛辭,上以治效為問,克家勸上無求奇功。既而三省、密院卒移牒泗州,敵不從,遣泛使來,舉朝震駭。後二年,湯邦彥坐使事貶,天下益服克家謀國之忠。



 淳熙八年,起知福州,在鎮有治績。趙雄奏欲令再任,降旨仍知福州。召除醴泉觀使。九年九月,拜右丞相,封儀國公。逾月而疾。十三年,命以內祠兼侍讀,賜第,在所存問不絕。十四年六月,薨,年六十。手書遺奏,上為之垂涕,贈少師,謚文靖。



 初,唱第時,孝宗由建邸入侍,愛其風度
 峻整,及登政府,眷寵尤渥。為文渾厚明白,自成一家,辭命尤溫雅,多行於世。



 汪澈,字明遠,自新安徙居饒州浮梁。第進士,教授衡州、沅州。用萬俟契薦,為秘書正字、校書郎。輪對,乞令帥臣、監司、侍從、臺諫各舉將帥,高宗善之,行其言。除監察御史,進殿中侍御史,特賜鞍馬。時和戎歲久,邊防浸弛,澈陳養民養兵、自治豫備之說,累數千言。



 顯仁皇后攢宮訖役,議者欲廣四隅,士庶墳在二十里內皆當遷,命澈
 按視。還奏:「昭慈、徽宗、顯肅、懿節四陵舊占百步,已數十年,今日何為是紛紛?漢長樂、未央宮夾樗里疾墓,未嘗遷。國朝宮陵儀制,在封堠界內,不許開故合祔,願遷出者聽,其意深矣。」高宗大悟,悉如舊。



 葉義問使金還,頗知犯邊謀,澈言:「不素備,事至倉卒,靖康之變可鑒。今將驕卒惰,宜加搜閱,使有鬥心。文武職事務選實才,不限資格。」除侍御史。左相湯思退不協人望,澈同殿中侍御史陳俊卿劾罷,又論鎮江大將劉寶十罪,詔奪節予祠。



 三十一年,上元前一夕,風雷雨雪交作,澈言《春秋》魯隱公時大雷震電,繼以雨雪,孔子以八日之間再有大變,謹而書之。今一夕間二異交至,此陰盛之證,殆為金人。今荊、襄無統督,江海乏備御,因陳修攘十二事。殿帥楊存中久握兵權,內結閹寺,王十朋、陳俊卿等繼論其罪,高宗欲存護使去,澈與俊卿同具奏,存中始罷。



 會金使高景山來求釁端,澈言:「天下之勢,強弱無定形,在吾所以用之。陛下屈己和戎,厚遺金繒,彼輒出惡言,以撼吾
 國。願陛下赫然睿斷,益兵嚴備,布告中外,將見上下一心,其氣百倍矣。」除御史中丞。



 尋遣馬帥成閔以所部三萬人屯荊、襄,以澈為湖北、京西宣諭使,詔凡吏能否、民利病悉以聞。過九江,王炎見澈論邊事,闢為屬,偕至襄陽撫諸軍。鄂帥田師中老而怯,立奏易之。時欲置襄守荊南,澈奏:「襄陽地重,為荊楚門戶,不可棄。」敵將劉萼擁眾十萬,揚聲欲取荊南,又欲分軍自光、黃搗武昌。朝廷以敵昔由此入江南,令吳拱嚴護武昌津渡。拱將引兵
 加鄂,澈聞之,馳書止拱,而自發鄂之餘兵戍黃州,俾拱留襄。敵騎奄至樊城,拱大戰漢水上,敵眾敗走。時唐、鄧、陳、蔡、汝、穎相次歸職方。未幾,金主亮死,澈乞出兵淮甸,與荊、襄軍夾擊其歸師。未報,而金新主罷兵請和,召澈入為參知政事,與宰相陳康伯同贊內禪。



 孝宗即位,銳意恢復,首用張浚使江、淮,澈以參豫督軍荊、襄,將分道進討。趙撙守唐,王宣守鄧,招皇甫倜於蔡。襄、漢沃壤,荊棘彌望,澈請因古長渠築堰,募閑民、汰冗卒雜耕,為度
 三十八屯,給種與牛,授廬舍,歲可登穀七十餘萬斛,民償種,私其餘,官以錢市之,功緒略就。



 隆興元年,入奏,還武昌,而張浚克期大舉,詔澈出師應之。澈以議不合,乞令浚並領荊、襄。諫議大夫王大寶論澈「無制勝策,皇甫倜以忠義結山砦,扼敵要沖,澈不能節制,坐視孤軍墮敵計。趙撙以千五百人救方城,敗散五百餘人,澈漫不加省。乞罷黜。」澈亦請祠,除資政殿學士、提舉洞霄宮。大寶疏再上,落職,仍祠祿。



 明年,知建康府,尋除樞密使。在
 位二年,以觀文殿學士奉洞霄祠,尋知鄂州兼安撫使。孝宗訪邊事,澈奏:「向者我有唐、鄧為藩籬,又皇甫倜控扼陳、蔡,敵不敢窺襄。既失兩郡,倜復內徙,敵屯新野,相距百里爾。臣令趙撙、王宣築城儲糧,分備要害,有以待敵。至於機會之來,難以豫料。」孝宗善之。時議廢江州軍,澈言不可。知寧國府,改福州、福建安撫使,復請祠。尋致仕。卒,年六十三。贈金紫光祿大夫,謚莊敏。



 澈為殿中日,薦陳俊卿、王十朋、陳之茂為臺官,高宗曰:「名士也,次第
 用之矣。」在樞府,孝宗密訪人材,薦百有十八人。嘗奏言:「臣起寒遠,所以報國惟無私不欺爾。」其自奉清約,雖貴猶布衣時。有文集二十卷、奏議十二卷。



 葉議問,字審言,嚴州壽昌人。建炎初,登進士第。調臨安府司理參軍。範宗尹為相,義問與沉長卿等疏其奸。為饒州教授,攝郡。歲旱,以便宜發常平米振民,提刑黃敦書劾之,詔勿問。前樞密徐俯門僧犯罪,義問繩以法,俯嘗舉義問,怒甚,乃袖薦書還之。



 知江寧縣。召秦檜所親
 役,同僚不可,義問曰:「釋是則何以服他人。」卒役之。通判江州。豫章守張宗元忤檜,或中以飛語,事下漕臣張常先。宗元道九江,常先檄義問拘其舟,義問投檄曰:「吾寧得罪,不為不祥。」常先白檜,罷去。



 檜死,湯思退薦之,上記其嘗言範宗尹,召至,言臺諫廢置在人主,檜親黨宜盡罷逐,以言得罪者宜敘復。擢殿中侍御史。樞密湯鵬舉效檜所為,植其黨周方崇、李庚,置籍臺諫,鉏異己者。義問累章劾鵬舉,有「一檜死一檜生」之語,並方崇等皆罷
 之。又言:「凡擇將遇一闕,令樞密院具三名取上旨,則軍政盡出掌握。」遷侍御史。朱樸、沉虛中奉祠裏居,義問劾其附秦檜,皆移居。郊祀赦,義問言:「頃歲附會告訐者,不應例移放。」從之。遷吏部侍郎兼史館修撰,尋兼侍讀,拜同知樞密院事。



 上聞金有犯邊意,遣義問奉使覘之,還奏:「彼造舟船,備器械,其用心必有所在,宜屯駐沿海要害備之。」金主亮果南侵。命視師,義問素不習軍旅,會劉錡捷書至,讀之至「金賊又添生兵」,顧吏曰:「『生兵』何物耶?」
 聞者掩口。至鎮江,聞瓜洲官軍與敵相持,大失措,乃役民掘沙溝,植木枝為鹿角禦敵,一夕潮生,沙溝平,木枝盡去。會建康留守張燾遣人告急,義問乃遵陸,雲往建康催發軍,市人皆媟罵之。又聞敵據瓜洲,採石兵甚眾,復欲還鎮江,諸軍喧沸曰:「不可回矣,回則有不測。」遂趨建康。已而金主亮被弒,師退,義問還朝,力請退,遂罷。



 隆興元年,中丞辛次膺論義問「頃護諸將幾敗事,且以官私其親」。謫饒州。乾道元年,詔自便。六年卒,年七十三。



 蔣芾字子禮,常州宜興人,之奇曾孫。紹興二十一年,進士第二人。孝宗即位,累遷起居郎兼直學士院。時宦者梁珂事上潛邸,撓權,尹穡論珂,與祠,芾繳奏罷之。



 簽書樞密院事,首奏加意邊防,又奏:「拔將才行伍間,識其姓名,一旦披籍可立取具。又料簡歸正人,仍以北人將之,或令深入山東,或令自荊、襄深入。」



 除權參知政事、同知國用事。芾奏:「方今財最費於養兵,藝祖取天下,不過十五萬人。紹興初,外有大敵,內有巨寇,然兵數亦不若今
 日之多。近見陳敏勇汰三千人,戚方汰四千人,然多是有官人,與以外任,請券錢、添借給如故,是減於內而添於外,何益?又招兵耗蠹愈甚,臣考核在內諸軍,每月逃亡事故,常不下四百人。若權停招兵一年有半,俟財用稍足,招丁壯,不惟省費,又得兵精。」上悟。



 一日,因進呈邊報,上顧芾曰:「將來都督非卿不可。」芾奏:「臣未嘗經歷兵間。」又奏:「方今錢穀不足,兵士不練,將帥與臣不相識,願陛下更審思其人。」南郊禮畢,宰相葉顒、魏杞罷。芾採眾
 論,參己見,為《籌邊志》上之。



 明年,拜右僕射、同中書門下平章事兼樞密使。會母疾卒,詔起復,拜左僕射,芾力辭。有密旨欲今歲大舉,手詔廷臣議,或主和,或主恢復,使芾決之。芾奏:「天時人事未至。」拂上意。服闋,除觀文殿大學士、知紹興府、提舉洞霄宮。尋以言者論,落職,建昌軍居住。期年,有旨自便。再提舉洞霄宮,卒。



 芾始以言邊事結上知,不十年間致相位,終以不能任兵事受責,豈優於論議而劣於事功歟?



 葉顒,字子昂,興化軍仙游人。登紹興元年進士第,為廣州南海縣主簿,攝尉。盜發,州檄巡、尉同捕,巡檢獲盜十餘人,歸其勞於顒,顒曰:「掠美、欺君、幸賞,三者皆罪,不忍為也。」帥曾開大善之。



 知信州貴溪縣。時詔行經界,郡議以上中下三等定田稅,顒請分為九等,守從之,令信之六邑以貴溪為式。



 知紹興府上虞縣。凡繇役,令民自推貨力甲乙,不以付吏,民欣然皆以實應。摧租各書其數與民,約使自持戶租至庭,親視其入,咸便之。帥曹泳令
 今歲夏租先期送什之八,顒請少紓其期,泳怒。及麥大熟,民輸租反為諸邑最,泳大喜,許薦於朝,顒固辭。



 賀允中薦顒靜退,遂召見,顒論國仇未復,中原之民日企鑾輿之返,其語剴切,高宗嘉納。除將作監簿。知處州,青田令陳光獻羨餘百萬,顒以所獻充所賦。湯思退之兄居處州,家奴屠酤犯禁,一繩以法,思退不悅。屬常州逋緡錢四十萬,守坐免,移顒知常州。



 金犯邊,高宗視師建康,道毗陵,顒賜對舟次,因言:「恢復莫先於將相,故相張浚
 久謫無恙,是天留以相陛下也。」顒初至郡,無旬月儲,未一年餘緡錢二十萬。或勸獻羨,顒曰:「名羨餘,非重征則橫斂,是民之膏血也,以利易賞,心實恥之。」



 召為尚書郎,除右司。詔求直言,顒上疏謂:「陛下以手足之至親,付州郡之重寄,是利一人害一方也。」人稱其直。除吏部侍郎,復權尚書。時七司弊事未去,上疏言選部所以為弊,乃與郎官編七司條例為一書,上嘉之,令刻板頒示。



 除端明殿學士,拜參知政事兼同知樞密院事。武臣梁俊彥
 請稅沙田、蘆場,帝以問顒,對曰:「沙田乃江濱地,田隨沙漲而出沒不常,蘆場則臣未之詳也。且辛巳軍興蘆場田租並復,今沙田不勝其擾。」上曰:「誠如卿言。」顒至中書,召俊彥切責之曰:「汝言利求進,萬一為國生事,斬汝不足以塞責。」俊彥皇恐汗下。是日,詔沙田、蘆場並罷。



 御史林安宅請兩淮行鐵錢,顒力言不可,安宅不能平,既入樞府,乃上章攻顒云:「顒之子受宣州富人周良臣錢百萬,得監鎮江大軍倉。」御史王伯庠亦論之。顒乞下吏辯
 明,乃以資政殿學士提舉洞霄宮。上下其事臨安府,時王炎知臨安,上令炎親鞫置對,無秋毫跡。獄奏,上以安宅、伯庠風聞失實,並免所居官,仍貶安宅筠州,召顒赴闕。入見,上勞之曰:「卿之清德自是愈光矣。」



 除知樞密院事,未拜,進尚書左僕射兼樞密使。顒首薦汪應辰、王十朋、陳良翰、周操、陳之茂、芮曄、林光朝等,可備執政、侍從、臺諫,上嘉納。又言:「自古明君用人,使賢使愚,使奸使盜,惟去泰甚。」上曰:「固然。虞有禹、皋,亦有共、驩;周有旦、奭,亦
 有管、蔡,在用不用。」顒曰:「誠如聖訓,但今日在朝雖未見有共、驩、管、蔡,然有竊弄威福者,臣不敢隱。」上問為誰,顒以龍大淵對,語在《陳俊卿傳》。



 上以國用未裕,詔宰相兼國用使,參政同知國用事,顒乃言:「今日費財養兵為甚,兵多則有冗卒虛籍,無事則費財,有事則不可用。雖曰汰之,旋即招之,欲足國用,當嚴於汰、緩於招可也。孔子曰:『節用而愛人』。蓋節用,則愛人之政自行於其間,若欲生財,祗費民財爾。」上曰「:此至言也。」上曰:「建康劉源嘗賂
 近習,朕欲遣王抃廉其奸。」顒曰:「臣恐廉者甚於奸者。」乃止。



 乾道三年冬至,上親郊而雷,顒引漢故事上印綬,提舉太平興國宮。歸至家,不疾而薨,年六十八。以觀文殿學士致仕,贈特進,謚正簡。



 顒為人簡易清介,與物若無忤,至處大事毅然不可奪。友人高登嘗上書譏切時相,名捕甚急。顒與同邸,擿令逸去,登曰:「不為君累乎?」顒曰:「以獲罪,固所願也。」即為具舟,舟移乃去。自初仕至宰相,服食、僮妾、田宅不改其舊。



 葉衡,字夢錫,婺州金華人。紹興十八年進士第,調福州寧德簿,攝尉。以獲鹽寇改秩,知臨安府於潛縣。戶版積弊,富民多隱漏,貧弱困於陪輸,衡定為九等,自五以下除其籍,而均其額於上之四等,貧者頓蘇。徵科為期限榜縣門,俾里正諭民,不遣一吏而賦自足。歲災,蝗不入境。治為諸邑最。郡以政績聞,即召對,上曰:「聞卿作縣有法。」遣還任。



 擢知常州。時水潦為災,衡發倉為糜以食饑者。或言常平不可輕發,衡曰:「儲蓄正備緩急,可視民饑
 而不救耶?」疫大作,衡單騎命醫藥自隨,偏問疾苦,活者甚眾。檄晉陵丞李孟堅攝無錫縣,有政聲,衡薦於上,即除知秀州。上之信其言如此。



 除太府少卿。合肥瀕湖有圩田四十里,衡奏:「募民以耕,歲可得穀數十萬,蠲租稅,二三年後阡陌成,仿營田,官私各收其半。」從之。



 除戶部侍郎。時鹽課大虧,衡奏:「年來課入不增,私販害之也,宜自煮鹽之地為之制,司火之起伏,稽灶之多寡,亭戶本錢以時給之,鹽之委積以時收之,擇廉能吏察之,私販
 自絕矣。」仍命措置官三人:淮南於通州,浙東於明州,浙西於秀州。



 丁母憂。起復,知盧州,未行,除樞密都承旨。奏馬政之弊,宜命統制一員各領馬若干匹,歲終計其數為殿最。李垕應賢良方正對策,近訐直,入第四等,衡奏:「陛下赦其狂而取其忠,足以顯容諫之盛。」乃賜垕制科出身。有言江、淮兵籍偽濫,詔衡按視,賜以袍帶、鞍馬、弓矢,且命衡措置民兵,咸稱得治兵之要。訖事赴闕,上御便殿閱武士,召衡預觀,賜酒,灑宸翰賜之。



 知荊南、成都、
 建康府,除戶部尚書,除簽書樞密院事,拜參知政事。衡奏二事:一,牧守將帥必擇材以稱其職,必久任以盡其材;二,令戶部取湖廣會子實數,盡以京會立限易之。從之。



 拜右丞相兼樞密使。上銳意恢復,凡將帥、器械、山川、防守悉經思慮,奏對畢,從容賜坐,講論機密,或不時召對。時會子浸患折閱,手詔賜衡曰:「會子雖曰流通,終未盡愜人意,目即流使有二千二百餘萬。今用上下庫黃金、白金、銅錢九百萬,內藏庫五百萬,並蜀中錢物七百
 萬,盡易會子之數,專命卿措置,日近而辦,卿真宰相才也。」



 一日,上曲宴宰執於凝碧,上曰:「自三代而下,至於漢、唐,治日常少,亂日常多,何也?」衡奏:「聖君不常有,周八百年,稱極治成、康而已。」上曰:「朕觀《無逸篇》,見周公為成王歷言商、周之君享國長遠,真萬世龜鑒。」衡奏:「願陛下常以《無逸》為龜鑒,社稷之福。」上又言:「朝廷所用,正論其人如何,不可有黨。如唐牛、李之黨,相攻四十年,緣主聽不明至此。文宗曰:『去河北賊易,去朝中朋黨難』。朕嘗笑之。」
 衡奏:「文宗優游不斷,故有此語。陛下英明聖武,誠非難事。」



 御寶實封令與臨安府竇思永改合入官,衡奏:「選人改官,非奏對稱旨,則用考舉磨勘,一旦特旨與之,非陛下愛惜人才之意。」上亟收前命。



 上諭執政,選使求河南,衡奏:「司諫湯邦彥有口辨,宜使金。」邦彥請對,問所以遣,既知薦出於衡,恨衡擠己,聞衡對客有訕上語,奏之,上大怒。即日罷相,責授安德軍節度副使,郴州安置。邦彥使還,果辱命,上震怒,竄之嶺南,詔衡自便,復官與祠。年
 六十有二薨,贈資政殿學士。



 衡負才足智,理兵事甚悉,由小官不十年至宰相,進用之驟,人謂出於曾覿云。



 論曰:陳康伯以經濟自任,臨事明斷。梁克家才優識遠,謀國盡忠。至若汪澈之論事忠愨,薦達人才,葉義問直言正色,掃除秦檜餘黨,然不長於兵,臨敵失措,豈優議論而劣事功者歟?葉顒清儉正直,而衡才智有餘,蓋亦一時之選云。



\end{pinyinscope}