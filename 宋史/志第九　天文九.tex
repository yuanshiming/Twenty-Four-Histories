\article{志第九 天文九}

\begin{pinyinscope}

 歲星晝見太白晝見經天五緯相犯老人星景星彗星客星



 歲星晝見



 嘉祐五年三月乙未,歲星晝見。六年六月壬申,晝見。七年六月丙子,晝見。八年七月癸亥,晝見。



 治平元年六月壬戌,晝見。



 元符二年八月癸未,晝見。



 太白晝見經天



 開寶元年六月丁丑,太白晝見。戊寅,復見。



 淳化元年六月庚午,七月丁丑,十一月戊戌,皆晝見。



 咸平三年六月己未,晝見。四年十二月丙寅,晝見在南斗。六年五月甲午、八月庚午,皆晝見。



 景德元年十一月辛亥,晝見。二年四月甲辰,晝見。三年七月乙巳,晝見。庚申,又見。十二月癸酉,又見。



 大中祥符元年七月庚申,晝見。四年六月丙午、八月乙巳,皆晝見。六年四月
 壬午,晝見。七年七月癸卯,晝見。九年五月庚午,晝見。



 天禧三年六月辛卯,復見。四年七月丁巳,晝見。五年六月丙午,晝見。



 乾興元年十一月壬辰,又見。



 天聖三年六月壬戌、十二月戊寅,皆晝見。五
 年五月壬寅,晝見。



 明道元年七月,晝見三十日。



 慶歷三年八月甲寅,晝見。



 皇祐三年四月丙午,晝見。



 至和元年五月壬辰、九月己丑、十月辛卯,皆
 晝見。三年四月己丑,晝見。



 嘉祐二年六月己未,晝見。四年正月庚寅,晝見。七月辛丑,晝見。五年九月庚寅,晝見。六年六月乙丑,晝見。七年五月戊午,晝見。七月己酉,經天,復見。十月乙未,晝見。



 治平元年正月戊戌,晝見。六月辛酉,晝見。二年七月丁丑,晝見。十二月辛亥,又見。四年二月丁酉,晝見。閏三月癸未,晝見。五月辛巳,晝見。七月癸卯,八月丁未,晝見。



 熙寧元年十一月癸酉,晝見。二年六月壬戌,晝見。三年
 五月癸巳、九月壬子、五年二月癸亥、五月丙午、八年三月戊午、七月戊寅,皆晝見。九年十月乙酉,晝見。十年五月甲戌,晝見。



 元豐元年四月癸亥,晝見。三年七月戊子,晝見。四年七月己丑,晝見。六年八月己卯,晝見。七年十月乙卯,晝見。



 元祐元年六月
 庚戌,晝見。十月庚寅,晝見。三年二月辛丑,晝見。七月辛未,又見。六年四月壬寅,晝見。閏八月乙丑,又見。七年十一月辛巳,晝見。八年四月己未,晝見。



 紹聖元年五月己酉,晝見。九月庚申,又見。二年十一月丙申,晝見。三年五月壬子,晝見。四年六月己酉,晝見。



 元符二年五月甲辰,晝見。八月癸巳,又見。



 崇寧元年六月己酉,晝見。三年正月癸卯,晝見。



 大觀二年十一月丁未,晝見。四年十月戊
 戌,又見。



 政和三年十二月辛酉,晝見。六年十月乙丑,晝見。七年三月辛未,晝見。



 重和元年十月己卯,晝見。



 宣和二年六月丁丑,晝見。六年十一月丙子,晝見。



 建炎元年十月甲戌、紹興元年四月壬申,晝見。四年六月庚子、十一月戊申,晝見經天。六年正月壬辰,晝見經天。十七年七月辛巳,晝見。二十八年六月壬辰,晝見。



 隆興元年七月丙申,經天晝見。二年六月戊辰,晝見。七月庚子,經天晝見。



 乾道元年三月甲寅,晝見。乙亥,晝見經天。二年四月甲申,晝見。五月甲寅,經天晝
 見。庚午,晝見。三年九月戊子,四年五月乙丑,晝見,與日爭明。六月辛卯,經天。五年六月庚寅,晝見。十一月甲子,晝見。庚午,晝見。



 淳熙三年五月癸酉,經天晝見。四年十一月壬戌,又見。六年七月乙丑,晝見。癸未,經天。九年六月庚申,晝見。甲子,經天。九月癸巳、十一年五月乙卯、十二年六月
 戊寅,晝見。七月丁酉,經天晝見,至八月壬申始滅。十四年六月辛卯,晝見。七月辛丑,經天。



 紹熙元年五月丙子,晝見。與日爭明。四年七月乙丑,十一月甲戌,晝見。



 慶元元年三月庚寅,經天晝見。七月己亥,晝見。四年九月壬寅,晝見。癸卯,經天。



 嘉泰元年六月丙午,經天晝見。十一月己巳,晝見。十二月己卯,經天晝見。三年六月癸亥,經天晝見。



 開禧元年三月庚申,二年五月壬寅,三年十二月乙巳,晝見,與日爭明。



 嘉定元年五月甲子,四年七月壬戌,五年九月丙午,六年二月丁丑,晝見。七年五月丁丑、八月乙巳、九
 月壬戌,晝見。九年五月癸酉、十年五月乙丑,晝見。癸酉,經天。十一月庚辰,晝見。戊戌,經天。十二年二月庚子,晝見。三月丁亥,經天晝見。六月辛未,晝見。辛亥,經天,晝見。十三年九月甲午、十四年三月甲午、十五年五月庚戌、九月辛未,晝見。十七年六月丁卯,晝見經天。



 寶慶元年六月辛卯,晝見。



 紹
 定五年四月丁丑,晝見。五月癸巳,經天。



 端平元年十一月壬戌,經天。二年四月丁亥、七月戊戌,晝見經天。



 嘉熙元年二月己酉,二年五月辛巳,八月辛酉,晝見經天。三年十二月辛酉、四年二月丁未、淳祐元年六月庚寅,晝見。十月戊戌,晝見。乙巳,經天。二年
 十二月壬戌,晝見。三年七月己亥、四年八月壬辰、五年二月辛卯,晝見經天。六年四月辛酉、八月壬子,晝見。



 九月戊辰,晝見經天。七年十月辛巳、九年十二月戊申、十一年二月乙卯、七月癸亥、寶祐二年九月丁卯、三年十月甲戌、四年五月丁未、五年七月己未、開慶元年六月壬寅、景定三年四月庚寅、閏九月甲申、五年四月
 戊午,晝見。五月乙亥、咸淳元年七月丁酉、四年九月癸酉德祐元年七月丙子,晝見。



 五緯相犯



 建隆元年正月甲子,太白犯熒惑於婁。十月壬申,又相犯於軫。三年十一月甲戌,熒惑犯歲星於房。



 乾德四年六月甲辰,太白犯熒惑於張。



 開寶四年十月甲辰,太白犯熒惑於牽牛。



 太平興國八年三月乙巳,熒惑犯歲星。



 端拱二年正月丁亥,辰星犯歲星於須女。十一日壬辰,熒惑犯歲星。



 淳化二年三月癸丑,太白犯歲星於婁。五年六月丙午,太白、歲星相犯於柳。十一月丙子,太白犯辰星於虛。



 至道元年五月戊午,熒惑犯填星於奎。



 咸平元年二月甲寅,太白犯填星。三年四月癸亥,辰星掩太白。六年正月庚戌,太白犯填星。



 景
 德二年六月己亥,太白犯歲星。三年七月戊辰,辰星犯歲星。己酉,太白犯歲星。四年七月癸巳,熒惑犯歲星。八月乙未,熒惑又犯歲星。



 大中祥符元年九月壬申,太白犯填星。二年十一月癸亥,熒惑犯歲星。四年十一月庚午,太白犯填星。辛未,辰星犯填星。五年正月壬午,熒惑犯歲星。七年三月乙巳,熒惑犯歲星。九年六月甲戌,熒惑犯歲星。



 天禧元年四月壬辰,太白犯歲星。二年六
 月戊午,太白犯歲星。七月癸酉,辰星犯太白。五年九月庚子,太白犯歲星。十月己巳,熒惑犯填星。



 天聖元年三月丁丑,熒惑犯歲星。二年九月戊申,太白犯熒惑,
 十一月壬子,辰星犯太白。三年五月癸未,太白、辰星相犯於井。五年六月辛卯,熒惑犯填星。壬辰,掩填星。七年五月辛未,太白犯填星,在畢宿一度半。八年六月乙酉,太白犯熒惑。



 景祐元年閏六月庚辰,太白犯填星。十一月甲寅,又犯熒惑。二年五月丁亥,又犯填星。九月辛巳,熒惑犯填星,在張六度。四年七月己未,太白犯熒惑。九月辛亥,熒惑犯填
 星,在翼十五度。



 康定元年九月壬申,辰星犯填星。



 慶歷三年九月甲申,太白犯歲星。



 皇祐三年十一月丁丑,熒惑犯填星。



 嘉祐元年九月乙巳,太白犯歲星。三年閏十二月甲戌,熒惑犯歲星,躔斗四度。五年正月壬辰,太白犯歲星。六年三月癸巳,熒惑犯歲星,在營室。七月己丑,太白犯填星,躔井十二度。閏八月己亥,太白犯辰星,在軫四度。七年正月庚申,太白犯歲星,在營室。六月丁丑,太白犯熒惑,在翼一度半。八年四月己丑,太白犯歲星,在胃。是日,熒惑晨見東方。五月庚辰,熒惑犯歲星,在昴四度。



 治平元年十一月庚午,辰星犯太白,在尾十六度。二年四月丁巳,太白犯歲星。
 五月癸亥,辰星犯太白。戊子,太白犯填星,在張五度。八月己亥,熒惑犯歲星,躔柳七度半。十月丙申,又犯填星,在翼二度。三年十二月癸卯,太白犯熒惑,躔危四度。四年九月癸巳,太白犯填星。丙申,犯歲星。十月甲子,熒惑犯填星。十一月己卯,又犯歲星。十二月丁卯,太白犯熒惑。



 熙寧元年十一月己丑,太白犯熒惑。三年正月己未,熒惑犯歲星。十月乙酉,太白犯填星。八年三月庚寅,太白犯填星。十年七月癸酉,太白犯歲星。



 元豐二年五月庚寅,熒惑犯歲星。
 四年十月乙
 亥,熒惑犯太白。五年三月
 丙戌,太白犯填星。十二月丙寅,辰星犯歲星。七年十一月甲寅,太白犯歲星。



 元祐元年閏二月戊申,太白犯熒惑。八年四月乙卯,太白犯熒惑。



 紹聖元年閏四月庚午,熒惑犯填星。三年九月丙午,太白犯填星。



 元符元年十二月乙未,太白犯熒惑。二年閏九月癸未,辰星犯填星。十月乙巳,太白犯填星。十二月辛亥,熒惑犯填星。三年四月丙辰,熒惑犯填星。



 崇寧元年
 十一月壬寅,太白犯填星。三年十一月庚寅,太白犯辰星。



 大觀元年十二月乙酉,太白犯熒惑。二年正月甲寅,太白犯歲星。二月壬午,熒惑犯歲星。



 十月丁酉,太白犯填星。十一月壬申,太白犯歲星。三年三月辛未,太白犯歲星。四年二月辛未,太白犯歲星。五月甲辰,熒惑犯歲星。



 政和元年二月辛丑,太白犯填星。十二月乙未,又犯。三年七月乙丑,熒惑犯太白。四年十月甲子,熒惑犯歲星。七年正月癸卯,熒惑犯歲星。



 宣和二年十月己卯,太
 白犯熒惑。三年閏五月壬午,熒惑犯歲星。六年二月己卯,熒惑犯歲星。七年七月乙未,太白犯歲星。



 靖康元年六月辛丑,太白犯歲星。



 紹興十九年六月壬戌,太白犯填星。二十年九月戊子,熒惑犯歲星。二十一年閏四月甲午,辰星犯填星。二十六年七月癸亥,太白犯熒惑。二十七年四月壬寅,太白犯歲星。二十八年十月乙未,辰星犯填星。三十年七月己亥,太白犯歲星。



 隆興元年九月丁酉,太白犯熒惑。十二月甲子,太白犯填星。二年正
 月丁亥至己丑,熒惑犯守歲星。十一月甲午,辰星犯歲星。



 乾道三年十一月乙亥,太白犯歲星。四年三月丁卯,熒惑犯填星。六年七月乙巳,熒惑犯歲星於畢。八年五月癸巳,太白犯歲星。九年二月庚申,熒惑犯歲星。七月丁巳,太白犯歲星。



 淳熙二年閏九月丁巳,太白犯熒惑。八年七月丁丑,太白犯填星。十一年七月庚戌,太白犯歲星。十四年十月庚辰,填星犯太白。十六年五月乙未,太白犯熒惑。



 紹熙二年十二月戊子,太白犯歲星。



 慶元
 元年九月戊子,太白犯熒惑。四年十月壬午,太白犯歲星。五年十一月辛丑,熒惑犯歲星。十二月辛未,太白犯填星。六年四月癸巳,熒惑犯填星。



 嘉泰二年五月庚戌,熒惑犯填星。



 開禧二年六月甲寅,熒惑犯歲星。三年十月丁未,太白犯熒惑。



 嘉定十年七月戊子,熒惑觸歲星。



 寶慶二年十月辛亥,熒惑犯填星。十一月辛酉,熒惑犯歲星。



 紹定元年十月甲子、五年六月乙丑、端平元年六月辛巳、三年六月丁未、嘉熙四年八月癸丑、寶祐四年
 十二月戊午,熒惑俱犯填星。



 開慶元年九月戊辰,太白犯熒惑。



 咸淳十年十月丙寅,熒惑犯填星。



 德祐二年正月癸酉,熒惑犯歲星。



 五緯相合



 歲星



 建隆三年十一月壬申,與熒惑合於房。



 開寶元年正月壬寅,與填星、太白合於婁。



 淳化五年六月丙午,與太白合於柳。



 至道元年五月庚戌,與太白、太陰同度,不相犯。



 景德四年九月戊子,與填星合於翼。



 天禧二年八
 月癸丑,與熒惑合於張。



 紹興十六年三月乙丑,與填星、太白合於昴。十月戊戌,與填星合於畢。十七年七月壬戌,與太白合。二十二年十二月乙丑,與熒惑合於尾。三十一年六月甲寅,與太白合於張。



 隆興元年十一月庚寅,與太白合。



 乾道元年十二月庚子,與填星合於南斗。二年十一月丁巳,與填星合於牛。六年五月戊寅,與太白合於畢。七年六月庚戌,與太白合於井。



 淳熙十四年四月癸未,與填星合於軫。十月己丑,與太白合於氐。



 慶
 元元年四月辛酉,與太白合於井。



 開禧元年七月癸未,與填星合。二年二月甲子,與填星合於昴。



 端平二年十月己未,與太白合於心。



 嘉熙四年五月甲子,與太白合於婁。



 寶祐三年八月丁卯,歲星、熒惑在柳。



 景定元年正月庚辰,與熒惑行入尾。



 熒惑



 雍熙二年七月丙戌,與歲星合於軫。



 建炎四年六月戊子,與填星合於亢。九月壬戌,與歲星合於斗。



 紹興二年六月丙午,與填星合於房。十一月乙亥,與歲星合
 於室。三年八月戊子,與太白合於張。四年二月戊子,與填星合於箕。五年閏二月丙午,與歲星合於昴。六年正月丁亥,與填星合於斗。七年五月甲申,與歲星、太白合於柳。閏十一月丁卯,與辰星行入氐。八年二月己未,與填星合於女。十三年九月辛未,與太白合於尾。十五年八月庚寅,與太白合於氐。二十年三月甲午,與太白合於畢。九月戊子,又合於軫。十一月戊子,與太白合於氐。二十二年十月己卯,與太白合於氐。十一月壬子,與歲
 星合於心。二十六年七月庚申,與填星合於軫。二十九年閏六月己未,與歲星合於井。三十年七月庚子,與填星合於氐。三十一年十一月丁未,與歲星合於翼。三十二年八月辛未,與填星合於尾。十一月壬戌,與太白合於羽林軍。



 隆興元年七月壬寅,與辰星合於柳。十二月壬申,與歲星合於氐。二年四月癸未,與歲星合於氐。八月癸酉,與填星合於箕。



 乾道元年八月辛巳,與太白合於翼。二年二月乙酉,與歲星合於斗。三月癸酉,與填星
 合於牛。四年二月庚申,與填星合。五月壬戌,與歲星合。五年十一月甲子,與太白合於房。戊辰,與辰星合於心。辛巳,又合於尾。六年二月甲申,與太白合。辛卯,合於女。三月戊午,合於危。乙丑,與歲星合於室。七月辛巳,與填星合於土。九月癸卯,合於畢。八年四月辛丑,與填星合於奎。九年三月辛丑,與歲星合於柳。四月乙丑,又合於星。



 淳熙二年六月丙寅,合於軫。四年九月己亥,合於尾。六年十一月甲子,合於危。九年二月壬寅,合於胃。十一年三
 月甲寅,合於井。



 紹熙三年九月乙亥,與填星合於尾。



 慶元四年五月庚子,又合。八月甲戌,合於虛。六年四月癸巳,合於室。



 嘉泰四年五月乙亥,合於胃。



 開禧三年十月丙辰,與太白合於箕。



 嘉定元年五月戊辰,與填星合於井。八月庚寅,與歲星合於張。六年三月癸卯,合於斗。七年三月辛巳,與太白合於參。八年四月戊午,與歲星合於室。九年十月庚午,與辰星合於房。十年七月戊寅,與歲星合於昴。十五年五月丁丑,合於軫。



 寶慶二年十月
 辛亥,與歲星、填星合於女。



 紹定元年十月丁巳,與填星合於危。二年正月丁亥,與歲星合於婁。三年十月己巳,與填星合於室。五年六月乙丑,與填星合於婁。



 端平元年六月庚午,與填星合於胃。三年六月癸卯,合於畢。



 嘉熙三年八月癸亥,與太白合於斗。四年七月己丑,與太白合於鬼。八月己酉,與填星合於柳。



 淳祐四年九月癸丑,合於軫。



 寶祐元年五月丁酉,與歲星合於昴。



 景定三年四月庚子,合於危。十一月丁未,與填星合於婁。五年
 六月戊辰,與歲星合。八月壬寅,與填星合。



 咸淳十年十月丙寅,與填星行在軫。



 填星



 端拱二年九月乙巳,與熒惑合於危。



 淳化二年正月癸丑,與太白合於須女。



 至道元年五月乙卯,與熒惑合於東壁。



 紹興十年十二月戊子、十一年三月庚子,與太白合於室。



 隆興二年十月辛巳,合於斗。



 乾道二年五月己未,與歲星合於南斗。



 淳熙五年閏六月己酉,與熒惑合於井。



 淳祐六年十月乙未,與歲星、熒惑合於亢。



 寶
 祐六年十一月甲戌,與熒惑順行在危。十二月辛丑,與太白、熒惑合於室。



 太白



 乾德四年六月己亥,與熒惑合於張。



 開寶三年五月庚戌,與填星合於畢。六月乙未,與歲星合於東井。五年十月甲辰,與熒惑合於牽牛。



 雍熙四年十二月丁巳,與填星、歲星合於南斗魁。



 淳化二年三月癸丑,與歲星合於婁,太白在南。三年正月丙辰,與熒惑合於婁,歲星在胃。



 至道元年五月丙辰,與歲星合於七星,不相犯。



 太
 中祥符元年九月乙酉,與歲星合於角、亢。



 建炎四年十一月辛丑,與歲星合於南斗。十二月壬午,與熒惑合於危。



 紹興元年九月丁酉,與熒惑合於張。十一月乙卯,與填星合於心。二年十一月甲子,與熒惑合於危。癸未,與歲星、熒惑合於室。三年四月戊子,與歲星合於奎。四年二月丁酉,合於婁。五年正月己卯,十月戊申,與填星合於斗。六年七月癸酉,與歲星合於井。七年四月丁巳,與熒惑合於東井。五月乙亥,與熒惑、辰星合於井。十一月
 癸巳,與熒惑合於尾。八年正月乙巳,與填星合於女。十一月丙午,合於虛。九年三月癸卯,與熒惑合於井。十一月壬申,與歲星合於角。十年十一月丁未,與填星合於危。十三年十二月乙巳,合於奎。十四年六月癸卯,與熒惑合於井。十七年二月庚戌,與填星合。庚申,與歲星合。十二月庚戌,與辰星合於南斗。十九年六月戊午,與填星合於井。七月丁未,與歲星、辰星合於張。二十年三月戊寅,與熒惑合於昴。四月庚戌,與填星合於東井。六月
 甲寅,與歲星合於翼。十月丙午,與歲星、熒惑合於軫。己巳,與熒惑合於角。二十二年九月庚申,與熒惑、辰星合於角。十月庚午,與熒惑合於亢。二十三年六月甲子,與填星合於張。九月癸卯,與歲星合於尾。閏十二月癸卯,合於南斗。二十五年九月壬申,與填星合於軫。十一月壬申,與辰星合於尾。二十六年七月丙辰,與熒惑合,壬戌,與熒惑、填星合於軫。二十七年三月辛卯,與熒惑、歲星合於奎。二十八年二月丁未,與歲星合於胃。六月乙
 未,與熒惑合。十一月己未,與填星合於亢。三十年七月丙申,與歲星合於柳。三十一年六月壬寅,合於星。九月庚午,與填星合於房。十二月甲辰,合於尾。



 隆興元年八月庚辰,與熒惑合於張。十月丁丑,與歲星合於亢。十二月辛酉,與填星合於箕。二年八月乙卯,與歲星合於氐。十月丙辰,與填星合於箕。



 乾道元年七月乙亥,與熒惑合於張。三年正月癸亥,與填星、歲星合。十一月壬申,與歲星合。五年四月乙巳,與熒惑合於井。十一月甲子,合
 於房。十二月癸巳,合於尾。六年正月甲子,合於斗。三月壬戌,與填星合。五月乙丑,與歲星合於昴。七年二月丙寅,與歲星合於畢。三月甲午,與熒惑合於井。八年五月癸未,與歲星合於井。九年三月辛酉,與填星合於奎。七月甲寅,與歲星合於張。



 淳熙元年正月丁未,與填星合於奎。十月乙丑,與歲星合於軫。二年閏九月甲寅,與熒惑合於尾。三年二月庚辰,與填星合於胃。五月乙丑,合於畢。六月癸巳,與熒惑合於井。四年九月壬子,與熒惑、
 歲星合於尾。五年正月庚戌,與歲星合於斗。十一月壬戌,合於牛。六年三月丁丑,六月丁酉,與填星皆合於井。八年六月壬申,合於柳。九年二月丙寅,與熒惑合於昴。五月乙亥,與填星合於柳。十一月乙亥,又與熒惑合於氐。十一年七月壬寅,與歲星合於柳。八月己卯,與填星合於翼。九月乙卯,與辰星、熒惑合於亢。十二年六月癸酉,與填星合於翼。十五年六月丙子,與填星合於亢。甲申,與歲星合於氐。



 紹熙元年十一月丁丑,與填星合。五
 年十一月庚戌,與熒惑合於危。



 慶元元年三月庚寅,與歲星合於參。六月庚午,合於井。八月癸酉,與熒惑合於張。二年十一月丙子,與填星合於牛。三年八月甲戌,與熒惑、歲星合於翼。四年十月戊寅,與歲星合於角。五年十二月辛未,與填星合於危。



 嘉泰元年五月戊午,與熒惑合於柳。二年正月丁巳,與熒惑、歲星合於南斗。十二月癸酉,與歲星合於女。



 開禧二年二月壬申,與填星、歲星合於昴。



 嘉定元年六月戊寅,與填星、熒惑合於井。二
 年四月丁丑,與填星合於井。四年八月乙酉,與填星合於室。五年九月丁未,與歲星合於心。七年六月庚子,與填星合於翼。十一月丁卯,與熒惑合於氐。九年九月庚寅,與填星合於角。十二年閏三月甲寅,七月壬寅,與歲星合於井。十三年八月丙戌,與填星合於房。



 寶慶二年正月壬午,與歲星、填星合於女。三年八月甲申,與熒惑合於星、翼。



 紹定三年閏二月乙酉,與歲星合於畢。五年八月壬申,合於張。六年五月庚戌,與熒惑合在柳。



 端平
 元年正月丁未,合於斗。二年二月壬午,與填星合於胃。三年九月庚申,與歲星合在尾。



 嘉熙元年六月乙未,與填星合於井。四年七月甲戌,與熒惑合於井。



 淳祐三年閏八月壬寅,與填星合於翼。六年三月戊午,與熒惑合於畢。十年十二月戊戌,與歲星合於危。十二年七月庚寅,與熒惑合於軫。九月戊戌,與填星合於箕。



 寶祐五年六月丙戌,與歲星合於翼。



 景定五年四月庚午,與歲星合於婁。



 咸熙三年七月己亥,與填星合於井。



 德祐元年
 十月丁巳,與填星合。



 辰星



 景德三年七月己酉,與歲星、太白合於柳。



 紹興四年三月乙亥,與太白合於畢。七年五月戊子,與熒惑、太白合於柳。九年九月乙巳,與歲星合於角。十七年三月乙卯,與填星合。二十一年閏四月壬辰,與填星合於東井。二十三年四月丙寅,與太白合於畢。二十八年十月丙申,與填星合於亢。



 隆興二年十一月庚寅,與歲星合。十二月丁亥,與太白合。



 乾道元年三月甲戌,與熒惑合
 於畢。四年二月壬子,與太白合於胃。五年六月庚寅,與歲星合。七年四月丙寅,淳熙四年五月乙巳,與太白合於井。十五年六月庚寅,與太白合於張。十二月壬戌,與歲星合於尾。



 紹熙四年三月辛己,與太白會於昴。



 五緯俱見



 乾德五年三月,五星如連珠,聚於奎、婁之次。



 景德四年七月,五星當聚鶉火而近太陽,同時伏。



 慶歷三年十一月壬辰,五星皆見東方。



 靖康元年六月丙辰,填星、熒惑、
 太白,歲星聚。



 乾道四年二月壬子、六月辛丑、八月己亥、六年五月乙亥、十月庚申、八年十月癸卯,五星俱見。



 淳熙十三年閏七月戊午,五星皆伏。八月乙亥,七曜俱聚於軫。



 老人星



 乾德三年八月辛酉、四年八月乙卯、六年正月戊申、開寶二年七月丁亥、本平興國四年八月乙亥、五年八月己卯、六年八月己卯、八年八月辛卯、雍熙三年八月己
 酉、四年八月辛亥、端拱元年八月乙卯、二年八月己亥、淳化元年八月丁卯、二年八月辛未、三年八月戊寅、四年九月己亥、五年八月己丑、至道元年八月己亥、二年閏七月己亥、三年八月辛丑、咸平元年八月癸丑、二年八月癸亥、三年八月丁卯、四年八月甲子、五年八月己丑、六年八月丙子、景德元年八月癸酉、二年八月庚辰、三年八月庚寅、四年二月己卯、八月甲午、大中祥符元年正月丁亥、八月丙申、二年二月壬辰、八月乙巳、三年
 二月辛巳、八月己酉、四年正月戊寅、八月丙寅、七年正月癸丑、八月己巳、八年七月癸酉、九年正月甲寅、八月壬午、天禧元年八月癸巳、二年正月丁巳、八月辛卯、三年八月己亥、四年八月己亥、五年二月丙午、八月己巳、老人星皆出丙。



 治平四年二月癸巳、八月戊申、熙寧元年正月乙未、八月己卯、二年二月乙卯、八月壬戌、三年正月甲寅、八月癸酉、四年二月己未、八月丁丑、五年二月己未、閏七月己亥、六年正月庚午、八月丁酉、七年二
 月甲申、八月庚寅、八年二月己丑、八月庚戌、九年二月丁酉、八月庚子、十年正月己卯、九月戊申、元豐元年二月乙酉、八月丙午、二年二月壬戌、八月乙卯、三年二月甲寅、八月己未、四年八月丁卯、五年二月甲戌、八月己巳、六年二月己未、八月丁丑、七年二月辛巳、八月己卯、八年二月庚辰、八月辛巳、元祐元年二月戊寅、八月庚子、二年二月庚寅、九月辛亥、三年二月癸巳、八月己亥、四年二月壬子、八月丁未、五年正月甲午、八月辛亥、六
 年二月己亥、閏八月壬戌、七年正月壬子、八月壬戌、八年二月丙寅、八月己巳、九年二月己丑、紹聖元年八月丙子、二年二月壬午、八月丁丑、三年二月庚午、八月癸未、四年二月甲申、八月甲申、五年二月庚辰、元符元年八月辛卯、二年二月乙未、九月壬辰、崇寧元年二月壬寅、八月癸未、二年二月甲寅、八月庚戌、三年二月戊午、八月辛酉、四年二月庚申、八月丙寅、五年二月戊辰、八月甲戌、大觀元年二月乙亥、八月丁丑、二年二月甲午、
 八月壬午、三年二月戊子、八月癸巳、四年二月乙未、閏八月丁酉、政和元年二月癸卯、八月己亥、二年二月乙巳、八月己酉、三年二月甲午、八月己未、四年二月己酉、八月辛未、五年二月庚申、八月甲子、六年閏正月壬戌、、八月丁卯、七年正月戊午、八月丙子、重和元年二月壬申、八月乙亥、宣和元年二月癸未、八月癸未、二年二月辛巳、八月己丑、三年二月丙戌、八月癸巳、四年二月己亥、八月辛丑、五年二月庚子、八月丙午、六年二月戊申、
 八月辛亥、七年二月癸丑、八月庚申、建炎四年七月戊辰、皆見於丙。



 景星



 開寶四年八月癸卯,景星見。



 景德三年四月戊寅,周伯星見,出氐南騎官西一度,狀如半月,有芒角,煌煌然可以鑒物,歷庫樓東,八月,隨天輪入濁,十一月,復見在氐。自是常以十一月辰見東方,八月西南入濁。



 大中祥符七年正月己酉,含譽星見。其年九月丙戌,又見,似彗有
 尾而不長。



 天聖元年二月己亥,奇星見。二年八月丙子,四年七月壬申,又見。



 明道二年二月戊戌,含譽星見東北方,其色黃白,光芒長二尺許。



 景祐二年正月己丑,奇星又見。



 至和三年二月辛卯、八月己未、嘉祐二年八月庚午、三年八月丙辰、四年正月庚戌、八月癸未、五年八月庚午、六年正月癸丑、八月壬辰、七年正月辛亥、八年正月辛酉、治平元年二月己丑、七月癸巳、二年二月癸巳、八月己亥、三年正月庚辰、八月庚戌、奇星皆見。



 彗孛



 彗星



 開寶八年六月甲子,出柳,長四丈,辰見東方,西南指,歷輿鬼至東壁,凡十一舍,八十三日而滅。



 端拱二年七月戊子,又出東井積水西,青白色,光芒漸長,辰見東北,旬日夕見西北,歷右攝提,凡三十日至亢沒。



 咸平元年正月甲申,又出營室北,光芒尺餘,至丁酉,凡十四日滅。六年十一月辛亥,旄頭犯輿鬼。甲寅,有彗孛於井、鬼,大如杯,色青白,光芒四尺餘,歷五諸侯及五車入參,凡
 三十餘日沒。



 天禧二年六月辛亥,彗出北斗魁第二星東北,長三尺許,與北斗第一星齊,北行經天牢,拂文昌,長三尺餘,歷紫微、三臺、軒轅速行而西,至七星,凡三十七日沒。



 景祐元年八月壬戌夜,有星孛於張、翼,長七尺,闊五寸,十二日而沒。十二月己未夜,有星出外屏,有芒氣。



 皇祐元年二月丁卯,彗出虛,晨見東方,西南指,歷紫微至婁,凡一百一十四日而沒。



 嘉祐元年七月,彗出紫微,歷七星,其色白,長丈餘,至八月癸亥滅。



 治平三年三
 月己未,彗出營室,晨見東方,長七尺許,西南指危洎墳墓,漸東速行,近日而伏。至辛巳,夕見西南,北有星,無芒彗,益東方,別有白氣一,闊三尺許,貫紫微極星並房宿,首尾入濁,益東行,歷文昌、北斗貫尾。至壬午,星復有芒彗,長丈餘,闊三尺餘,東北指,歷五車,白氣為岐橫天,貫北河、五諸侯、軒轅、太微五帝坐內五諸侯及角、亢、氐、房宿。癸未,彗長丈五尺,星有彗氣如一升器,歷營宿至張,凡一十四舍,積六十七日,星氣孛皆滅。



 熙寧八年十月
 乙未,星出軫度中,如填,青白。丙申,西北生光芒,長三尺,斜指軫,若彗。丁酉,光芒長五尺。戊戌,長七尺,斜指左轄,至丁未,入濁不見。



 元豐三年七月癸未,彗出西北太微垣郎位南,白氣長一丈,斜指東南,在軫度中。丙戌,向西北行,在翼度中。戊子,長三尺,斜穿郎位。癸卯,犯軒轅,至丁酉入濁不見。庚子晨,復出於張度中。至戊子,凡三十有六日,沒不見。



 紹聖四年八月己酉,彗出氐度中,如填,有光,色白,氣長三丈,斜指天市左星,九月壬子,光芒長
 五尺,入天市垣。己未,犯天市垣宦者。庚申,犯天市垣帝坐。戊辰,沒不見。



 崇寧五年正月戊戌,彗出西方,如杯口大,光芒散出如碎星,長六丈,闊三尺,斜指東北,自奎宿貫婁、胃、昴、畢,後入濁不見。



 大觀四年五月丁未,彗出奎、婁,光芒長六尺,北行入紫微垣,至西北入濁不見。



 靖康元年六月壬戌,彗出紫微垣。



 紹興元年九月,彗星見。十二月戊寅、二年八月甲寅,見於胃。丙辰,行犯土司空,至九月甲戌始滅。十五年四月戊寅,彗星見東方。丙申,復
 見於參度。五月丁巳,化為客星,其色青白。壬戌,留守張,至六月丁亥乃消。十六年十一月庚寅,彗星見西南危宿。二十六年七月丙午,彗星見東井,約長一丈,光芒二尺。癸丑,又犯五諸侯。三十一年六月己巳,彗星見北斗天權星東北,太史妄稱為含譽。



 淳熙二年七月辛丑,有星孛於西北方,當紫微垣外七公之上,小如熒惑,森然蓬孛,至丙午始消。



 嘉定十五年八月甲午,彗星見右攝提,光芒三尺餘,體類歲星,凡兩月,歷氐、房、心乃沒。



 紹定
 三年十一月丁酉,有星孛於天市垣屠肆星之下,明年二月壬午乃消。五年閏九月,彗星見東方,十月己未始消。



 嘉熙四年正月辛未,彗星見於室,至三月辛未乃消。



 景定五年七月甲戌,彗星見於柳,芒角燭天,長十餘丈,日高方斂,凡月餘,己卯,退行,見於輿鬼。辛巳,在井。丙申,見於參。戊戌,在參宿度內。八月末,光芒稍減,凡四月乃滅。



 客星



 建隆二年十二月己酉,出天市垣宗人星東,微有
 芒彗,三年正月辛未,西南行入氐宿,二月癸丑,至七星沒。



 太平興國八年二月甲辰,出太微垣端門東,近屏星,北行。



 端拱二年七月丁亥,出北河星西北,稍暗,微有芒彗,指西南。



 淳化元年正月辛巳,出軫宿,逆至張,七十日,經四十度乃不見。



 景德二年八月甲辰,出紫微天棓側,孛孛然如粉絮,稍入垣內,歷御女、華蓋,凡十一日沒。三年三月乙巳,出東南方。



 大中祥符四年正月丁丑,見南斗魁前。



 天禧五年四月丙辰,出軒轅前星西北,大如桃,
 速行,經軒轅大星入太微垣,掩右執法,犯次將,歷屏星西北,凡七十五日,入濁沒。



 明道元年六月乙巳,出東北方,近濁,有芒彗。至丁巳,凡十三日沒。



 至和元年五月己丑,出天關東南,可數寸,歲餘稍沒。



 熙寧二年六月丙辰,出箕度中,至七月丁卯,犯箕乃散。三年十一月丁未,出天囷。



 元祐六年十一月辛亥,出參度中,犯、掩廁星,壬子,犯九游星,十二月癸酉,入奎,至七年三月辛亥乃散。



 紹興八年五月,守婁,魯分也。九年二月壬申,守亢,陳分也。



 乾道二年三月癸酉,出太微垣內五帝坐大星西,微小,色青白。



 淳熙八年六月己巳,出奎宿,犯傳舍星,至明年正月癸酉,凡一百八十五日始滅。



 嘉泰三年六月乙卯,出東南尾宿間,色青白,大如填星。甲子,守尾。



 嘉定十七年六月己丑,犯尾宿。



 嘉熙四年七月庚寅,出尾宿。



\end{pinyinscope}