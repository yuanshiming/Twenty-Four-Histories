\article{列傳第一百五十五}

\begin{pinyinscope}

 ○史浩王淮趙雄權邦彥程松陳謙張巖



 史浩,字直翁,明州鄞縣人。紹興十四年登進士第,調紹興餘姚縣尉,歷溫州教授,郡守張栻九成器之。秩滿,除太
 學正,升國子博士。因轉對,言:「普安、恩平二王宜擇其一以系天下望。」高宗頷之。翌日,語大臣曰:「浩有用才也。」除秘書省校書郎兼二王府教授。三十年,普安郡王為皇子,進封建王,除浩權建王府教授。詔建王府置直講、贊讀各一員,浩守司封郎官兼直講。一日講《周禮》,言:「膳夫掌膳羞之事,歲終則會,惟王及後、世子之膳羞不會。至酒正掌飲酒之事,歲終則會,惟主及後之飲酒不會,世子不與焉。以是知世子膳羞可以不會,世子飲酒不可
 以無節也。」王作而謝曰:「敢不佩斯訓。」



 三十一年,遷宗正少卿。會金主亮犯邊,下詔親征。時兩淮失守,廷臣爭陳退避計,建王抗疏請率師為前驅。浩為王力言:「太子不可將兵,以晉申生、唐肅宗靈武之事為戒。」王大感悟,立俾浩草奏,請扈蹕以供子職,辭意懇到。高宗方怒,覽奏意頓釋,知奏出於浩,語大臣曰:「真王府官也。」既而殿中侍御史吳芾乞以皇子為元帥,先視師。浩復遺大臣書,言:「建王生深宮中,未嘗與諸將接,安能辦此。」或謂使王
 居守,浩復以為不可。上亦欲令王遍識諸將,遂扈蹕如建康。



 三十二年,上還臨安,立建王為皇太子,浩除起居郎兼太子右庶子。孝宗受禪,遂以中書舍人遷翰林學士、知制誥。張浚宣撫江、淮,將圖恢復,浩與之異議,欲城瓜洲、採石。浚奏:「不守兩淮而守江,不若城泗州。」除參知政事。有詔議應敵定論,洪遵、金安節、唐文若等相繼論列,宰執獨無奏。上以問浩,浩奏:「先為備御,是謂良規。儻聽淺謀之士,興不教之師,寇去則論賞以邀功,寇至則
 斂兵而遁跡,謂之恢復得乎?」薦樞密院編修官陸游、尹穡,召對,並賜出身。隆興元年,拜尚書右僕射,首言趙鼎、李光之無罪,岳飛之久冤,宜復其官爵,祿其子孫。悉從之。



 李顯忠、邵宏淵奏乞引兵進取,浩奏:「二將輒乞戰,豈督府命令有不行耶?」浚請入覲,乞即日降詔幸建康,上以問浩,浩陳三說不可,退,又以詰浚曰:「帝王之兵,當出萬全,豈可嘗試以圖僥倖。」復辨論於殿上,浚曰:「中原久陷,今不取,豪傑必起而收之。」浩曰:「中原決無豪傑,若有
 之,何不起而亡金?」浚曰:「彼民間無寸鐵,不能自起,待我兵至為內應。」浩曰:「勝、廣以鉏櫌棘矜亡秦,必待我兵,非豪傑矣。」浚因內引奏:「浩意不可回,恐失幾會,乞出英斷。」省中忽得宏淵出兵狀,始知不由三省,徑檄諸將。浩語陳康伯曰:「吾屬俱兼右府,而出兵不與聞,焉用相哉!不去尚何待乎?」因又言:「康伯欲納歸正人,臣恐他日必為陛下子孫憂。浚銳意用兵,若一失之後,恐陛下終不得復望中原。」御史王十朋論之,出知紹興。



 先是,浩因城瓜
 洲,白遣太府丞史正志往視之,正志與浚論辯。十朋亦疏史正志朋比,並及浩,遂與祠,自是不召者十三年。起知紹興府、浙東安撫使。持母喪歸,服闋,知福州。



 淳熙初,上問執政:「久不見史浩,無他否?」遂除少保、觀文殿大學士、醴泉觀使兼侍讀。五年,復為右丞相。上曰:「自葉衡罷,虛席以待卿久矣。」浩奏:「蒙恩再相,唯盡公道,庶無朋黨之弊。」上曰:「宰相豈當有黨,人主亦不當以朋黨名臣下。朕但取賢者用之,否則去之。」



 樞密都承旨王抃建議以
 殿、步二司軍多虛額,請各募三千人充之。已而殿前司輒捕市人,京城騷動,被掠者多斷指,示不可用。軍人怙眾,因奪民財。浩奏:「盡釋所捕,而禽軍民首喧呶者送獄。」獄成議罪,欲取兵民各一人梟首以徇。浩曰:「諸軍掠人奪貨至於哄,則始釁者軍人也,軍法從事固當。若市人陸慶童特與抗斗爾,可同罰乎?陛下恐軍人有語,故一其罪以安之。夫民不得其平,言亦可畏,『等死,死國可乎?』是豈軍人語。」上怒曰:「是比朕為秦二世也。」浩徐進曰:「自古
 民怨其上者多矣,『時日曷喪,予及汝偕亡』,豈二世事。」尋求去,拜少傅、保寧軍節度使,充醴泉觀使兼侍讀。後有言慶童子冤者,上曰:「史浩嘗力爭,坐此求去,至今悔之。」



 趙雄嘗薦劉光祖試館職,光祖答策,論科場取士之道,進入,上親批其後,略曰:「用人之弊,人君乏知人之哲,宰相不能擇人。國朝以來,過於忠厚,宰相而誤國,大將而敗軍,未嘗誅戮。要在人君必審擇相,相必當為官擇人,懋賞立乎前,誅戮設乎後,人才不出,吾不信也。」手詔既
 出,中外大聳。議者謂曾覿視草,為光祖甲科發也。上遣覿持示浩,浩奏:「唐、虞之世,四兇極惡,止於流竄,三考之法,不過黜陟,未嘗有誅戮之科。誅戮大臣,秦、漢法也。太祖制治以仁,待臣下以禮,列聖傳心,迨仁宗而德化隆洽,本朝之治,與三代同風,此祖宗家法也。聖訓則曰『過於忠厚』。夫為國而底於忠厚,豈有所謂過哉?臣恐議者以陛下自欲行刻薄之政,歸過祖宗,不可不審也。」



 及自經筵將告歸,乃於小官中薦江、浙之士十五人,有旨令
 升擢,皆一時選也。如薛叔似、楊簡、陸九淵、石宗昭、陳謙、葉適、袁燮、趙靜之、張子智,後皆擢用,不至通顯者六人而已。



 十年,請老,除太保致仕,封魏國公。晚治第鄞之西湖上,建閣奉兩朝賜書,又作堂,上為書「明良慶會」名其閣、「舊學」名其堂。光宗御極,進太師。紹熙五年薨,年八十九,封會稽郡王。寧宗登極,賜謚文惠,御書「純誠厚德元老之碑」賜焉。嘉定十四年,追封越王,改謚忠定,配享孝宗廟庭。



 浩喜薦人才,嘗擬陳之茂進職與郡,上知之茂
 嘗毀浩,曰:「卿豈以德報怨耶?」浩曰:「臣不知有怨,若以為怨而以德報之,是有心也。」莫濟狀王十朋行事,詆浩尤甚,浩薦濟掌內制,上曰:「濟非議卿者乎?」浩曰:「臣不敢以私害公。」遂除中書舍人兼直學士院,待之如初。蓋其寬厚類此。子彌大、彌正、彌遠、彌堅。彌遠嘉定初為右丞相,有傳。



 王淮,字季海,婺州金華人。幼穎悟,力學屬文。登紹興十五年進士第,為臺州臨海尉。郡守蕭振一見奇之,許以
 公輔器。振帥蜀,闢置幕府。振出,眾欲留,淮曰:「萬里將母,豈為利祿計。」皆服其器識,遷校書郎。



 高宗命中丞舉可為御史者,朱倬舉淮,除監察御史,尋遷右正言。首論:「大臣養尊,小臣持祿,以括囊為智,以引去為高。願陛下正心以正朝廷,正朝廷以正百官。」宰相湯思退無物望,淮條其罪數十,於是策免。至於吏部侍郎沈介之欺世盜名,都司方師尹之狡險,大將劉寶掊克結權倖,皆劾罷之。又奏:「自治之策,治內有三:正心術,寶慈儉,去壅蔽。治
 外有四:固封守,選將帥,明賞罰,儲財用。」上深嘉嘆。



 除秘書少監兼恭王府直講。時恭王生子挺,淮白於丞相,曰:「恭王夫人李氏生皇嫡長孫,乞討論典禮。」錢端禮怒其名稱,奏:「淮有年鈞以長之說。」上曰:「是何言也,豈不啟邪心?」出淮知建寧府,改浙西提刑。入見,陳閩中利病甚悉。帝褒嘉之,且令一至東宮,皇太子待以師儒,特施拜禮。尋召,除太常少卿,除中書舍人兼直學士院。龍大淵贈太師,仍畀儀同三司恩數,張栻說除太尉、在京宮觀,皆封
 還詔書。除翰林學士、知制誥,訓詞深厚,得王言體。上命擇文學行誼之士,淮薦鄭伯熊、李燾、程叔達,皆擢用。



 淳熙二年,除端明殿學士、簽書樞密院事。辛棄疾平茶寇,上功太濫。淮謂:「不核真偽,何以勸有功。」文州蕃部擾邊,吳挺奏:「庫彥威失利,靖州夷人擾邊。」楊倓奏:「田淇失利。」淮謂:「二將戰歿,若罪之,何以勸來者。」上嘗諭曰:「樞密臨事盡公,人無間言,差除能守法甚善。」薦軍帥吳拱、郭田、張宣。除同知樞密院事、參知政事。



 時宰相久虛,淮與
 李彥穎同行相事。淮謂:「授官當論賢否,不事形跡。誠賢,不敢以鄉里故舊廢之;非才,不敢以己私庇之。」上稱善。擢知院事、樞密使。上言武臣岳祠之員宜省,淮曰:「有戰功者,壯用其力,老而棄之,可乎?」趙雄言:「北人歸附者,畀以員外置,宜令詣吏部。」上曰:「姑仍舊。」淮曰:「上意即天意也。」雄又奏言:「宗室岳祠八百員,宜罷。」淮曰:「堯親睦九族,在平章百姓之先;骨肉之恩疏,可乎?」時辛棄疾平江西寇,王佐平湖南寇,劉焞平廣西寇,淮皆處置得宜,論功惟
 允。上深嘉之,謂:「陳康伯雖有人望,處事則不及卿。」



 八年,拜右丞相兼樞密事。先是,自夏不雨至秋,是日甘雨如注,士大夫相賀,上亦喜命相而雨,乃命口算諸郡絹錢盡蠲一年,為緡八十萬。



 趙雄罷相,蜀士之在朝者皆有去意。淮謂:「此唐季黨禍之胎也,豈聖世所宜有。」皆以次進遷,蜀士乃安。樞密都承旨王抃怙寵為奸,淮極陳其罪,謂:「人主受謗,鮮不由此。」上即斥之,且曰:「丞相直諒無隱,君臣之間正宜如此。」章穎論事狂直,上將
 黜之,淮曰:「陛下樂聞直言,士大夫以言相高,此風可賀也。黜之適成其名。」上說,穎復留。



 時以荒政為急,淮言:「李椿老成練達,擬除長沙帥,朱熹學行篤實,擬除浙東提舉,以倡郡國。」其後推賞,上曰:「朱熹職事留意。」淮言:「修舉荒政,是行其所學,民被實惠,欲與進職。」上曰:「與升直徽猷閣。」成都闕帥,上加訪問,淮以留正對。上曰:「非閩人乎?」淮曰:「立賢無方,湯之執中也。必曰閩有章子厚、呂惠卿,不有曾公亮、蘇頌、蔡襄乎?必曰江、浙多名臣,不有丁謂、王欽
 若乎?」上稱善。拜左丞相。



 天長水害七十餘家,或謂不必以聞,淮曰:「昔人謂人主不可一日不聞水旱盜賊,《記》曰:『四方有敗,必先知之。』豈可不以聞?」鎮江饑民強借菽粟,執政請痛懲之,淮曰:「令甲,饑民罪不至死。」進士八人求以免舉恩為升等,淮曰:「八人得之,則百人援之。」龔頤以執政之客補官,求詣銓曹,淮以此門不可啟,絕其請。嘗言跅弛之士,緩急能出死力,乃以周極知安豐軍,辛棄疾與祠。



 上章力求去,以觀文殿大學士判衢州。淮力辭,
 改提舉洞霄宮。光宗嗣位,詔詢初政,淮以盡孝進德,奉天敬民,用人立政,罔不在初。母亡,居喪如禮。得疾,忽語家人曰:「《易》卦六十四,吾年亦然。」淳熙十六年薨。訃聞,上哀悼,輟視朝,贈少師,謚文定。



 初,朱熹為浙東提舉,劾知臺州唐仲友。淮素善仲友,不喜熹,乃擢陳賈為監察御史,俾上疏言:「近日道學假名濟偽之弊,請詔痛革之。」鄭丙為吏部尚書,相與葉力攻道學,熹由此得祠。其後慶元偽學之禁始於此。



 趙雄字溫叔,資州人。為隆興元年類省試第一。虞允文宣撫四蜀,闢乾辦公事,入相,薦於朝。乾道五年,召見便殿,孝宗大奇之,即日手詔除正字。



 範成大使金,將行,雄當登對,允文招與之語。既進見,雄極論恢復。孝宗大喜曰:「功名與卿共之。」即除右史,兩月除舍人。金使耶律子敬賀會慶節,雄館伴。子敬披露事情不敢隱,邏者以聞。上夜召雄,雄具以子敬所言對,上喜。金使入辭,故事當用樂,雄奏:「卜郊有日,天子方齋,樂不可用。」上難之,遣中
 使諭雄,雄奏:「金使必不敢不順,即有他,臣得引與就館。」上大喜。雄請復置恢復局,日夜講磨,條具合上意,除中書舍人。自選人入館至此,未滿歲也。



 時金將起河南之役,議盡以諸陵梓宮歸於我。上命雄出使賀生辰,仍止奉遷陵寢及正受書儀。雄既見金主,爭辨數四。其臣屢喝起,雄辭益力,卒得請乃已,金人謂之「龍鬥」。嘗上疏論恢復計,大略謂:「莫若由蜀以取陜西,得陜西以臨中原,是秦制六國之勢也。」八年,以母憂去。



 淳熙二年,召為禮
 部侍郎,除端明殿學士,簽書樞密院事。一日奏事,上曰:「今夏蠶麥甚熟、絲米價平可喜。」雄奏:「孟子論王道始於不饑不寒。」上曰「近世士大夫好高論,恥言農事,微有西晉風。豈知《周禮》與《易》言理財,周公、孔子曷嘗不以理財為務?且不獨此,士夫諱言恢復,不知其家有田百畝,內五十畝為人所據,亦投牒理索否?」雄曰:「陛下志在大有為,敢不布堯言,書之《時政記》。」十一月,同知樞密院事。五年三月,參知政事。十一月,拜右丞相。每進見,必曰「二帝
 在沙漠」,未嘗離諸口也。



 朱熹累召不出,雄請處以外郡,命知南康軍。熹極論時事,上怒,諭雄令分析。雄奏:「熹狂生,詞窮理短,罪之適成其名。若天涵地育,置而不問可也。」會周必大亦力言之,乃止。紹興帥張津獻羨餘四十萬緡,雄乞降旨下紹興,以其錢為民代輸和買身丁折帛錢之半,使取諸民者,民復得之,足以見聖主之德。



 自雄獨相,蜀人在朝者僅十數。及眷衰,有言其私里黨者,上疑之。已而陳峴為四川制置,王渥為茶馬,命從中出。
 雄求去,詔勉留,曰:「丞相任事不避怨,選才無鄉舊。」蓋有所激也。祖宗時蜀人未嘗除蜀帥,雄請外,除觀文殿大學士、四川制置使。王藺為御史,以故事不可,上疏論之。雄乞免,改知瀘南安撫使。上思雄不忘,改知江陵府。江陵無險可恃,雄請城江陵,城成,民不告擾。



 張栻再被召,論恢復固當,第其計非是,即奏疏。孝宗大喜,翌日以疏宣示,且手詔云:「恢復當如栻所陳方是。」即除侍講,云:「且得直宿時與卿論事。」虞允文與雄之徒不樂,遂沮抑之。
 廣西橫山買馬,諸蠻感悅,爭以善馬至。上知栻治行,甚向栻,眾皆忌嫉。洎栻復出荊南,雄事事沮之。時司天奏相星在楚地,上曰:「張栻當之。」人愈忌之。



 光宗將受禪,召雄,雄上萬言書,陳修身齊家以正朝廷之道,言甚剴切。詔授寧武軍節度使、開府儀同三司,進衛國公,改帥湖北。疾甚,改判資州,又除潼川府,改隆興府。紹熙四年薨,年六十五,贈少師。嘉定二年,謚文定。



 權邦彥,字朝美,河間人。登崇寧四年太學上舍第,調滄
 州教授,入為太學博士,改宣教郎,除國子司業。宣和二年,使遼。明年,抗表請帝臨雍。為學官積十餘年,改都官郎中、直秘閣、知易州,移相州,復召為都官郎中。與王黼議不合,鐫職,知冀州。



 金人再入,高宗開大元帥府,起兩河兵衛汴京,邦彥提所部兵二千五百人,與宗澤自澶淵趨韋城,據刀馬河,諸道兵莫有進者。會敵兵大至,移屯南華。二帝北遷,邦彥與澤五表勸進。



 建炎元年五月,召還,命知荊南府,改東平府。時東州半已入金,至是圍
 益急,邦顏誓以死守,居數月城破,猶力戰不已。民義而從之,突圍以出,遂奔行在。有司議失守罪,將重坐之,帝以其父母妻子皆沒於敵,才貶二秩。俄除寶文閣直學士兼知江州、本路制置使。既抵鎮,三年冬,聞父死,乃解官。



 四年,起復,知建康府,辭,不許。劇盜張琪殘徽州,邦彥遣裨將平之。改江、淮等路制置發運使,以治辦稱。言者論:「三年天下之通喪,後世有從權奪服者,所以徇國家之急。比年如權邦彥、姜仲謙,至幕職亦起復,幾習宣、
 政之風,望革其弊,以明人倫、厚風俗。」詔邦彥任軍賦,宜如舊,餘悉罷之。



 紹興元年,召為兵部尚書兼侍讀。二年,除端明殿學士、簽書樞密院事。初,邦彥獻十議以圖中興,大略謂:「宜以天下為度,進圖洪業,恢復士宇,勿茍安於東南。駕御諸將,當威之以法,而限之以爵。命讀講之臣,取累朝訓典及三代、漢、唐中興故事,日陳於前,以裨聖學。又監觀傷善妨賢之讒,偷安茍容之佞,市恩立威之奸,懷諼罔上之欺,聽其言,察其事,則忠邪判。愛
 民先愛其力,寬民先節其用。朘己奉以佐國,當自執政始。分閫而屬大事,類非偏裨之所能為,必得賢臣大將然後可。制置一官可省,宜令沿江州縣各備境內,總以漕帥,上自荊、鄂、江、池,下至採石、京口,委任得人,乃防秋上策。宗室中豈無傑然有人望,可以濟艱難、贊密勿、留宿衛者,願求其人置諸左右。人事盡則天悔禍,不可獨歸之數。」



 呂頤浩素善邦彥,薦用之。給事中程瑀劾邦彥五罪,三疏不報。邦彥在樞密,又言:「宜乘機者三,譬奕之爭先,安
 可隨應隨解,不制人而制於人哉?」尋兼權參知政事。帝嘗對輔臣言湖南事,頤浩言:「李綱縱暴,恐治潭無善狀。」帝曰:「綱在宣和間論水災,以得時望。」邦彥曰:「綱元無章疏,第略虛名耳。」蓋助頤浩以排綱也。三年,卒。



 邦彥與政幾一年,碌碌無所建明,充位而已。無子,以侄嗣衍為後。有遺稿十卷,號《瀛海殘編》,藏於家。



 程松,字冬老,池州青陽人。登進士第,調湖州長興尉。章森、吳曦使北,松為傔從。慶元中,韓侂胄用事,曦為殿帥。
 時松知錢塘縣,諂事曦以結侂胄。侂胄以小故出愛姬,松聞,以百千市之,至則盛供帳,舍諸中堂,夫婦奉之謹。居無何,侂胄意解,復召姬,姬具言松謹待之意,侂胄大喜,除松幹辦行在諸軍審計司、守太府寺丞。未閱旬,遷監察御史,擢右正言、諫議大夫。



 呂祖泰上書,乞誅侂胄、蘇師旦,松與陳讜劾祖泰當誅,祖泰坐真決,流嶺南。松滿歲未遷,意殊怏怏,乃獻一妾於侂胄,曰「松壽」。侂胄訝其名,問之,答曰:「欲使疵賤姓常蒙記憶爾。」除同知樞密
 院事,自宰邑至執政財四年。



 開禧元年,以資政殿大學士知成都府、四川制置使。侂胄決議開邊,期以二年四月分道進兵,命松為宣撫使,興元都統制吳曦副之,尋加曦為陜西招撫使,許便宜從事。松將東軍三萬駐興元,曦將西軍六萬駐河池。松至益昌,欲以執政禮責曦庭參,曦聞之,及境而返。松用東西軍一千八百人自衛,曦多抽摘以去,松殊不悟。曦遣其客納款於金,獻關外四州地,求為蜀王。有告曦叛者,松哂其狂。及金人取
 成州,守將棄關遁,吳曦焚河池還興州。松以書從曦求援兵,曦答以「鳳州非用騎之地,漢中平衍,可騎以驅馳,當發三千騎往。」蓋紿之也。



 未幾,金人封曦為蜀王。曦遺松書諷使去,松不知所為。興元帥劉甲、茶馬範仲任見松,謀起兵誅曦,松恐事洩取禍,即揖二人起去。會報金人且至,百姓奔走相蹂躪,一城如沸。松亟望米倉山遁去,由閬州順流至重慶,以書抵曦,丐贐禮買舟,稱曦為蜀王。曦遣使以匣封致饋,松望見大恐,疑其劍也,亟逃
 奔。使者追及,松不得已啟視之,則金寶也。松乃兼程出峽,西向掩淚曰:「吾今獲保頭顱矣。」曦誅,詔落職,降三官,筠州居住,再降順昌軍節度副使,澧州安置。又責果州團練副使、賓州安置。死賓州。



 陳謙,字益之,溫州永嘉人。乾道八年進士,授福州戶曹、主管刑工部架閣文字,遷國子錄、敕令所刪修官、樞密院編修官。陳中興五事,至李綱議建鎮事,上曰:「綱何足道。」謙曰:「陛下用大臣,審出綱上,宜如聖訓。今顧出綱下
 遠甚,奈何?」上蹙然,遂極論逾數刻。



 孝宗內禪,通判江州,知常州,提舉湖北常平。平辰州峒徭,加直煥章閣,除戶部郎中,總領湖、廣財賦。謙乃丞相趙汝愚客,會黨論起坐斥。後數年,起為提點成都府路刑獄,移京西運判,復直煥章閣。



 韓侂胄謀擾金人,令獻馬者補官,七州民相扇為盜。謙移書侂胄曰:「今若倚群盜行剽掠之策,豈得以敗亡為戲乎?」既而屢論襄帥皇甫斌、李奕罪,且求罷。上諭旨薛叔似協和之。遷司農少卿、湖廣總領,除宣撫
 司參謀官。



 金兵深入,陷應城,焚漢川,漢陽空城走,武昌震懼。謙以寶謨閣待制副宣撫,即日置司北岸,命土豪趙觀覆之中流,士馬溺死甚眾,餘兵皆返走。未幾,奪職,罷。後復知江州。侂胄死,和議已決,謙復罷,奉祠。卒,年七十三。



 謙有雋聲,早為善類所予。晚坐偽禁中廢,首稱侂胄為「我王」,士論由是薄之。



 張巖,字肖翁,大梁人,徙家揚州,紹興末渡江,居湖州。為人機警,柔回善諧。登乾道五年進士第,歷官為監察御
 史,與張釜、陳自強、劉三傑、程松等阿附時相韓侂胄,誣逐當時賢者,嚴道學之禁。



 進殿中侍御史,累遷給事中,除參知政事。以言者罷為資政殿學士、知平江府,旋升大學士、知揚州。時邊釁方開,詔巖與程松分帥兩淮,已而召還,為參知政事兼同知國用事。開禧二年,遷知樞密院事。明年,除督視江、淮軍馬。



 時方信孺使金議和,值吳曦以蜀叛,議未決,曦伏誅。金人尋前議,信孺再行。侂胄趣巖遣畢再遇、田琳合兵剿敵,且募生擒偽帥。未幾,
 川、陜戰屢衄,大散關陷,敵情復變。巖開督府九閱月,費耗縣官錢三百七十餘萬緡,見和議反復,乃言不知兵,固求去。



 侂胄誅,御史章燮論巖與蘇師旦朋奸誤國,奪兩官。寧宗謂兵釁方開,巖嘗言其不可,許自便,復元官,奉祠。以銀青光祿大夫致仕,薨,贈特進。



 論曰:史浩宅心平恕,而不能相其君恢復之謀。王淮為偽學之禁,毒痡善類。趙雄與虞允文協謀用兵,而舊史謂二人沮抑張栻,何哉?邦彥守城力戰,惜乎助呂頤浩
 攻李綱,君子少之。程松、陳謙、張巖誣諛之徒,何足算哉!



\end{pinyinscope}