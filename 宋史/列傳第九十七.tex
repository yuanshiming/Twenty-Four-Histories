\article{列傳第九十七}

\begin{pinyinscope}
蘇軾
 \gezhu{
  子過、邁、迨}



 蘇軾,字子瞻,眉州眉山人。生十年,父洵游學四方,母程氏親授以書,聞古今成敗,輒能語其要。程氏讀東漢《範滂傳》,慨然太息,軾請曰:「軾若為滂,母許之否乎?」程氏曰:「
 汝能為滂,吾顧不能為滂母邪?」



 比冠,博通經史,屬文日數千言,好賈誼、陸贄書。既而讀《莊子》,嘆曰:「吾昔有見,口未能言,今見是書,得吾心矣。」嘉祐二年,試禮部。方時文磔裂詭異之弊勝,主司歐陽修思有以救之,得軾《刑賞忠厚論》,驚喜,欲擢冠多士,猶疑其客曾鞏所為,但置第二;復以《春秋》對義居第一,殿試中乙科。後以書見修,修語梅聖俞曰:「吾當避此人出一頭地。」聞者始嘩不厭,久乃信服。



 丁母憂。五年,調福昌主簿。歐陽修以才識兼茂,
 薦之秘閣。試六論,舊不起草,以故文多不工。軾始具草,文義粲然。復對制策,入三等。自宋初以來,制策入三等,惟吳育與軾而已。



 除大理評事、簽書鳳翔府叛官。關中自元昊叛,民貧役重,岐下歲輸南山木筏,自渭入河,經砥柱之險,衙吏踵破家。軾訪其利害,為修衙規,使自擇水工以時進止,自是害減半。



 治平二年,入判登聞鼓院。英宗自藩邸聞其名,欲以唐故事召入翰林,知制誥。宰相韓琦曰:「軾之才,遠大器也,他日自當為天下用。要在
 朝廷培養之,使天下之士莫不畏慕降伏,皆欲朝廷進用,然後取而用之,則人人無復異辭矣。今驟用之,則天下之士未必以為然,適足以累之也。」英宗曰:「且與修注如何?」琦曰:「記注與制誥為鄰,未可遽授。不若於館閣中近上貼職與之,且請召試。」英宗曰:「試之未知其能否,如軾有不能邪?」琦猶不可,及試二論,復入三等,得直史館。軾聞琦語,曰:「公可謂愛人以德矣。」會洵卒,賻以金帛,辭之,求贈一官,於是贈光祿丞。洵將終,以兄太白早亡,子
 孫未立,妹嫁杜氏,卒未葬,屬軾。軾既除喪,即葬姑。後官可蔭,推與太白曾孫彭。



 熙寧二年,還朝。王安石執政,素惡其議論異己,以判官告院。四年,安石欲變科舉、興學校,詔兩制、三館議。軾上議曰:



 得人之道,在於知人;知人之法,在於責實。使君相有知人之明,朝廷有責實之政,則胥史皂隸未嘗無人,而況於學校貢舉乎?雖因今之法,臣以為有餘。使君相不知人,朝廷不責實,則公卿侍從常患無人,而況學校貢舉乎?雖復古之制,臣以為不
 足。夫時有可否,物有廢興,方其所安,雖暴君不能廢,及其既厭,雖聖人不能復。故風俗之變,法制隨之,譬如江河之徙移,強而復之,則難為力。



 慶歷固嘗立學矣,至於今日,惟有空名僅存。今將變今之禮,易今之俗,又當發民力以治宮室,斂民財以食游士。百里之內,置官立師,獄訟聽於是,軍旅謀於是,又簡不率教者屏之遠方,則無乃徒為紛亂,以患苦天下邪?若乃無大更革,而望有益於時,則與慶歷之際何異?故臣謂今之學校,特可因
 仍舊制,使先王之舊物,不廢於吾世足矣。至於貢舉之法,行之百年,治亂盛衰,初不由此。陛下視祖宗之世,貢舉之法,與今為孰精?言語文章,與今為孰優?所得人才,與今為孰多?天下之事,與今為孰辦?較此四者之長短,其議決矣。



 今所欲變改不過數端:或曰鄉舉德行而略文詞,或曰專取策論而罷詩賦,或欲兼採譽望而罷封彌,或欲經生不帖墨而考大義,此皆知其一,不知其二者也。願陛下留意於遠者、大者,區區之法何預焉。臣又
 切有私憂過計者。夫性命之說,自子貢不得聞,而今之學者,恥不言性命,讀其文,浩然無當而不可窮;觀其貌,超然無著而不可挹,此豈真能然哉!蓋中人之性,安於放而樂於誕耳。陛下亦安用之?



 議上,神宗悟曰:「吾固疑此,得軾議,意釋然矣。」即日召見,問:「方今政令得失安在?雖朕過失,指陳可也。」對曰:「陛下生知之性,天縱文武,不患不明,不患不勤,不患不斷,但患求治太急,聽言太廣,進人太銳。願鎮以安靜,待物之來,然後應之。」神宗悚然
 曰:「卿三言,朕當熟思之。凡在館閣,皆當為朕深思治亂,無有所隱。」軾退,言於同列。安石不悅,命權開封府推官,將困之以事。軾決斷精敏,聲聞益遠。會上元敕府市浙燈,且令損價。軾疏言:「陛下豈以燈為悅?此不過以奉二宮之歡耳。然百姓不可戶曉,皆謂以耳目不急之玩,奪其口體必用之資。此事至小,體則甚大,願追還前命。」即詔罷之。



 時安石創行新法,軾上書論其不便,曰:



 臣之所欲言者,三言而已。願陛下結人心,厚風俗,存紀綱。人主
 之所恃者人心而已,如木之有根,燈之有膏,魚之有水,農夫之有田,商賈之有財。失之則亡,此理之必然也。自古及今,未有和易同眾而不安,剛果自用而不危者。陛下亦知人心之不悅矣。



 祖宗以來,治財用者不過三司。今陛下不以財用付三司,無故又創制置三司條例一司,使六七少年,日夜講求於內,使者四十餘輩,分行營干於外。夫制置三司條例司,求利之名也;六七少年與使者四十餘輩,求利之器也。造端宏大,民實驚疑;創法
 新奇,吏皆惶惑。以萬乘之主而言利,以天子之宰而治財,論說百端,喧傳萬口,然而莫之顧者,徒曰:「我無其事,何恤於人言。」操網罟而入江湖,語人曰「我非漁也」,不如捐網罟而人自信。驅鷹犬而赴林藪,語人曰「我非獵也」,不如放鷹犬而獸自馴。故臣以為欲消讒慝而召和氣,則莫若罷條例司。



 今君臣宵旰,幾一年矣,而富國之功,茫如捕風,徒聞內帑出數百萬緡,祠部度五千餘人耳。以此為術,其誰不能?而所行之事,道路皆知其難。汴水
 濁流,自生民以來,不以種稻。今欲陂而清之,萬頃之稻,必用千頃之陂,一歲一淤,三歲而滿矣。陛下遂信其說,即使相視地形,所在鑿空,訪尋水利,妄庸輕剽,率意爭言。官司雖知其疏,不敢便行抑退,追集老少,相視可否。若非灼然難行,必須且為興役。官吏茍且順從,真謂陛下有意興作,上糜帑廩,下奪農時。堤防一開,水失故道,雖食議者之肉,何補於民!臣不知朝廷何苦而為此哉?



 自古役人,必用鄉戶。今者徒聞江、浙之間,數郡顧役,而
 欲措之天下。單丁、女戶,蓋天民之窮者也,而陛下首欲役之,富有四海,忍不加恤!自楊炎為兩稅,租調與庸既兼之矣,奈何復欲取庸?萬一後世不幸有聚斂之臣,庸錢不除,差役仍舊,推所從來,則必有任其咎者矣。青苗放錢,自昔有禁。今陛下始立成法,每歲常行。雖云不許抑配,而數世之後,暴君污吏,陛下能保之與?計願請之戶,必皆孤貧不濟之人,鞭撻已急,則繼之逃亡,不還,則均及鄰保,勢有必至,異日天下恨之,國史記之,曰「青苗
 錢自陛下始」,豈不惜哉!且常平之法,可謂至矣。今欲變為青苗,壞彼成此,所喪逾多,虧官害民,雖悔何及!



 昔漢武帝以財力匱竭,用賈人桑羊之說,買賤賣貴,謂之均輸。於時商賈不行,盜賊滋熾,幾至於亂。孝昭既立,霍光順民所欲而予之,天下歸心,遂以無事。不意今日此論復興。立法之初,其費已厚,縱使薄有所獲,而征商之額,所損必多。譬之有人為其主畜牧,以一牛易五羊。一牛之失,則隱而不言;五羊之獲,則指為勞績。今壞常平而
 言青苗之功,虧商稅而取均輸之利,何以異此?臣竊以為過矣。議者必謂:「民可與樂成,難與慮始。」故陛下堅執不顧,期於必行。此乃戰國貪功之人,行險僥幸之說,未及樂成,而怨已起矣。臣之所願陛下結人心者,此也。



 國家之所以存亡者,在道德之淺深,不在乎強與弱;歷數之所以長短者,在風俗之薄厚,不在乎富與貧。人主知此,則知所輕重矣。故臣願陛下務崇道德而厚風俗,不願陛下急於有功而貪富強。愛惜風俗,如護元氣。聖人
 非不知深刻之法可以齊眾,勇悍之夫可以集事,忠厚近於迂闊,老成初若遲鈍。然終不肯以彼易此者,知其所得小,而所喪大也。仁祖持法至寬,用人有敘,專務掩覆過失,未嘗輕改舊章。考其成功,則曰未至。以言乎用兵,則十出而九敗;以言乎府庫,則僅足而無餘。徒以德澤在人,風俗知義,故升遐之日,天下歸仁焉。議者見其末年吏多因循,事不振舉,乃欲矯之以苛察,齊之以智能,招來新進勇銳之人,以圖一切速成之效。未享其利,
 澆風已成。多開驟進之門,使有意外之得,公卿侍從跬步可圖,俾常調之人舉生非望,欲望風俗之厚,豈可得哉?近歲樸拙之人愈少,巧進之士益多。惟陛下哀之救之,以簡易為法,以清凈為心,而民德歸厚。臣之所願陛下厚風俗者,此也。



 祖宗委任臺諫,未嘗罪一言者。縱有薄責,旋即超升,許以風聞,而無官長。言及乘輿,則天子改容;事關廊廟,則宰相待罪。臺諫固未必皆賢,所言亦未必皆是。然須養其銳氣,而借之重權者,豈徒然哉?將
 以折奸臣之萌也。今法令嚴密,朝廷清明,所謂奸臣,萬無此理。然養貓以去鼠,不可以無鼠而養不捕之貓;畜狗以防盜,不可以無盜而畜不吠之狗。陛下得不上念祖宗設此官之意,下為子孫萬世之防?臣聞長老之談,皆謂臺諫所言,常隨天下公議。公議所與,臺諫亦與之;公議所擊,臺諫亦擊之。今者物論沸騰,怨讟交至,公議所在,亦知之矣。臣恐自茲以往,習慣成風,盡為執政私人,以致人主孤立,紀綱一廢,何事不生!臣之所願陛下
 存紀綱者,此也。



 軾見安石贊神宗以獨斷專任,因試進士發策,以「晉武平吳以獨斷而克,苻堅伐晉以獨斷而亡,齊恆專任管仲而霸,燕噲專任子之而敗,事同而功異」為問,安石滋怒,使御史謝景溫論奏其過,窮治無所得,軾遂請外,通判杭州。高麗入貢,使者發幣於官吏,書稱甲子。軾卻之曰:「高麗於本朝稱臣,而不稟正朔,吾安敢受!」使者易書稱熙寧,然後受之。



 時新政日下,軾於其間,每因法以便民,民賴以安。徙知密州。司農行手實法,
 不時施行者以違制論。軾謂提舉官曰:「違制之坐,若自朝廷,誰敢不從?今出於司農,是擅造律也。」提舉官驚曰:「公姑徐之。」未幾,朝廷知法害民,罷之。



 有盜竊發,安撫司遣三班使臣領悍卒來捕,卒兇暴恣行,至以禁物誣民,入其家爭鬥殺人,且畏罪驚潰,將為亂。民奔訴軾,軾投其書不視,曰:「必不至此。」散卒聞之,少安,徐使人招出戮之。徙知徐州。河決曹村,泛於梁山泊,溢於南清河,匯於城下,漲不時洩,城將敗,富民爭出避水。軾曰:「富民出,民
 皆動搖,吾誰與守?吾在是,水決不能敗城。」驅使復入。軾詣武衛營,呼卒長曰:「河將害城,事急矣,雖禁軍且為我盡力。」卒長曰:「太守猶不避塗潦,吾儕小人,當效命。」率其徒持畚鍤以出,築東南長堤,首起戲馬臺,尾屬於城。雨日夜不止,城不沉者三版。軾廬於其上,過家不入,使官吏分堵以守,卒全其城。復請調來歲夫增築故城,為木岸,以虞水之再至。朝廷從之。



 徙知湖州,上表以謝。又以事不便民者不敢言,以詩託諷,庶有補於國。御史李定、
 舒但、何正臣摭其表語,並媒薛所為詩以為訕謗,逮赴臺獄,欲置之死,鍛煉久之不決。神宗獨憐之,以黃州團練副使安置。軾與田父野老,相從溪山間,築室於東坡,自號「東坡居士。」



 三年,神宗數有意復用,輒為當路者沮之。神宗嘗語宰相王珪、蔡確曰:「國史至重,可命蘇軾成之。」珪有難色。神宗曰:「軾不可,姑用曾鞏。」鞏進《太祖總論》,神宗意不允,遂手扎移軾汝州,有曰:「蘇軾黜居思咎,閱歲滋深,人材實難,不忍終棄。」軾未至汝,上書自言饑寒,
 有田在常,願得居之。朝奏入,夕報可。



 道過金陵,見王安石,曰:「大兵大獄,漢、唐滅亡之兆。祖宗以仁厚治天下,正欲革此。今西方用兵,連年不解,東南數起大獄,公獨無一言以救之乎?」安石曰:「二事皆惠卿啟之,安石在外,安敢言?」軾曰:「在朝則言,在外則不言,事君之常禮耳。上所以待公者,非常禮,公所以待上者,豈可以常禮乎?」安石厲聲曰:「安石須說。」又曰:「出在安石口,入在子瞻耳。」又曰:「人須是知行一不義,殺一不辜,得天下弗為,乃可。」軾戲
 曰:「今之君子,爭減半年磨勘,雖殺人亦為之。」安石笑而不言。



 至常,神宗崩,哲宗立,復朝奉郎、知登州,召為禮部郎中。軾舊善司馬光、章惇。時光為門下侍郎,惇知樞密院,二人不相合,惇每以謔侮困光,光苦之。軾謂惇曰:「司馬君實時望甚重。昔許靖以虛名無實,見鄙於蜀先主,法正曰:『靖之浮譽,播流四海,若不加禮,必以賤賢為累』。先主納之,乃以靖為司徒。許靖且不可慢,況君實乎?」惇以為然,光賴以少安。



 遷起居舍人。軾起於憂患,不欲驟
 履要地,辭於宰相蔡確。確曰:「公徊翔久矣,朝中無出公右者。」軾曰:「昔林希同在館中,年且長。」確曰:「希固當先公耶?」卒不許。元祐元年,軾以七品服入侍延和,即賜銀緋,遷中書舍人。



 初,祖宗時,差役行久生弊,編戶充役者不習其役,又虐使之,多致破產,狹鄉民至有終歲不得息者。王安石相神宗,改為免役,使戶差高下出錢雇役,行法者過取,以為民病。司馬光為相,知免役之害,不知其利,欲復差役,差官置局,軾與其選。軾曰:「差役、免役,各有
 利害。免役之害,掊斂民財,十室九空,斂聚於上而下有錢荒之患。差役之害,民常在官,不得專力於農,而貪吏猾胥得緣為奸。此二害輕重,蓋略等矣。」光曰:「於君何如?」軾曰:「法相因則事易成,事有漸則民不驚。三代之法,兵農為一,至秦始分為二,及唐中葉,盡變府兵為長征之卒。自爾以來,民不知兵,兵不知農,農出穀帛以養兵,兵出性命以衛農,天下便之。雖聖人復起,不能易也。今免役之法,實大類此。公欲驟罷免役而行差役,正如罷長
 征而復民兵,蓋未易也。」光不以為然。軾又陳於政事堂,光忿然。軾曰:「昔韓魏公刺陜西義勇,公為諫官,爭之甚力,韓公不樂,公亦不顧。軾昔聞公道其詳,豈今日作相,不許軾盡言耶?」光笑之。尋除翰林學士。



 二年,兼侍讀。每進讀至治亂興衰、邪正得失之際,未嘗不反復開導,覬有所啟悟。哲宗雖恭默不言,輒首肯之。嘗讀祖宗《寶訓》,因及時事,軾歷言:「今賞罰不明,善惡無所勸沮;又黃河勢方北流,而強之使東;夏人入鎮戎,殺掠數萬人,帥臣
 不以聞。每事如此,恐浸成衰亂之漸。」



 軾嘗鎖宿禁中,召入對便殿,宣仁後問曰:「卿前年為何官?」曰:「臣為常州團練副使。」。曰:「今為何官?」曰:「臣今待罪翰林學士。」曰:「何以遽至此?」曰:「遭遇太皇太后、皇帝陛下。」曰:「非也。」曰:「豈大臣論薦乎?」曰:「亦非也。」軾驚曰:「臣雖無狀,不敢自他途以進。」曰:「此先帝意也。先帝每誦卿文章,必嘆曰:『奇才,奇才!』但未及進用卿耳。」軾不覺哭失聲,宣仁後與哲宗亦泣,左右皆感涕。已而命坐賜茶,徹御前金蓮燭送歸院。



 三年,權
 知禮部貢舉。會大雪苦寒,士坐庭中,噤未能言。軾寬其禁約,使得盡技。巡鋪內侍每摧辱舉子,且持暖昧單詞,誣以為罪,軾盡奏逐之。



 四年,積以論事,為當軸者所恨。軾恐不見容,請外,拜龍圖閣學士、知杭州。未行,諫官言前相蔡確知安州,作詩借郝處俊事以譏太皇太后。大臣議遷之嶺南。軾密疏:「朝廷若薄確之罪,則於皇帝孝治為不足;若深罪確,則於太皇太后仁政為小累。謂宜皇帝敕置獄逮治,太皇太后出手詔赦之,則於仁孝兩
 得矣。」宣仁後心善軾言而不能用。軾出郊,用前執政恩例,遣內侍賜龍茶、銀合,慰勞甚厚。



 既至杭,大旱,饑疫並作。軾請於朝,免本路上供米三之一,復得賜度僧牒,易米以救饑者。明年春,又減價糶常平米,多作饘粥藥劑,遣使挾醫分坊治病,活者甚眾。軾曰:「杭,水陸之會,疫死比他處常多。」乃裒羨緡得二千,復發橐中黃金五十兩,以作病坊,稍畜錢糧待之。



 杭本近海,地泉咸苦,居民稀少。唐刺史李泌始引西湖水作六井,民足於水。白居易
 又浚西湖水入漕河,自河入田,所溉至千頃,民以殷富。湖水多葑,自唐及錢氏,歲輒浚治,宋興,廢之,葑積為田,水無幾矣。漕河失利,取給江潮,舟行市中,潮又多淤,三年一淘,為民大患,六井亦幾於廢。軾見茅山一河專受江潮,鹽橋一河專受湖水,遂浚二河以通漕。復造堰閘,以為湖水畜洩之限,江潮不復入市。以餘力復完六井,又取葑田積湖中,南北徑三十里,為長堤以通行者。吳人種菱,春輒芟除,不遣寸草。且募人種菱湖中,葑不復
 生。收其利以備修湖,取救荒餘錢萬緡、糧萬石,及請得百僧度牒以募役者。堤成,植芙蓉、楊柳其上,望之如畫圖,杭人名為蘇公堤。



 杭僧凈源,舊居海濱,與舶客交通,舶至高麗,交譽之。元豐末,其王子義天來朝,因往拜焉。至是,凈源死,其徒竊持其像,附舶往告。義天亦使其徒來祭,因持其國母二金塔,云祝兩宮壽。軾不納,奏之曰:「高麗久不入貢,失賜予厚利,意欲求朝,未測吾所以待之厚薄,故因祭亡僧而行祝壽之禮。若受而不答,將生
 怨心;受而厚賜之,正墮其計。今宜勿與知,從州郡自以理卻之。彼庸僧猾商,為國生事,漸不可長,宜痛加懲創。」朝廷皆從之。未幾,貢使果至,舊例,使所至吳越七州,費二萬四千餘緡。軾乃令諸州量事裁損,民獲交易之利,無復侵撓之害矣。



 浙江潮自海門東來,勢如雷霆,而浮山峙於江中,與漁浦諸山犬牙相錯,洄洑激射,歲敗公私船不可勝計。軾議自浙江上流地名石門,並山而東,鑿為漕河,引浙江及溪谷諸水二十餘里以達於江。又
 並山為岸,不能十里以達龍山大慈浦,自浦北折抵小嶺,鑿嶺六十五丈以達嶺東古河,浚古河數里達於龍山漕河,以避浮山之險,人以為便。奏聞,有惡軾者,力沮之,功以故不成。



 軾復言:「三吳之水,瀦為太湖,太湖之水,溢為松江以入海。海日兩潮,潮濁而江清,潮水常欲淤塞江路,而江水清駛,隨輒滌去,海口常通,則吳中少水患。昔蘇州以東,公私船皆以篙行,無陸挽者。自慶歷以來,松江大築挽路,建長橋以阨塞江路,故今三吳多水,
 欲鑿挽路、為十橋,以迅江勢」。亦不果用,人皆以為恨。軾二十年間再蒞杭,有德於民,家有畫像,飲食必祝。又作生祠以報。



 六年,召為吏部尚書,未至。以弟轍除右丞,改翰林承旨。轍辭右丞,欲與兄同備從官,不聽。軾在翰林數月,復以讒請外,乃以龍圖閣學士出知穎州。先是,開封諸縣多水患,吏不究本末,決其陂澤,注之惠民河,河不能勝,致陳亦多水。又將鑿鄧艾溝與穎河並,且鑿黃堆欲注之於淮。軾始至穎,遣吏以水平準之,淮之漲水
 高於新溝幾一丈,若鑿黃堆,淮水顧流穎地為患。軾言於朝,從之。



 郡有宿賊尹遇等,數劫殺人,又殺捕盜吏兵。朝廷以名捕不獲,被殺家復懼其害,匿不敢言。軾召汝陰尉李直方曰:「君能禽此,當力言於朝,乞行優賞;不獲,亦以不職奏免君矣。」直方有母且老,與母訣而後行。乃緝知盜所,分捕其黨與,手戟刺遇,獲之。朝廷以小不應格,推賞不及。軾請以己之年勞,當改朝散郎階,為直方賞,不從。其後吏部為軾當遷,以符會其考,軾謂已許
 直方,又不報。



 七年,徙揚州。舊發運司主東南漕法,聽操舟者私載物貨,征商不得留難。故操舟者輒富厚,以官舟為家,補其敝漏,且周船夫之乏,故所載率皆速達無虞。近歲一切禁而不許,故舟弊人困,多盜所載以濟饑寒,公私皆病。軾請復舊,從之。未閱歲,以兵部尚書召兼侍讀。



 是歲,哲宗親祀南郊,軾為鹵簿使,導駕入太廟。有赭傘犢車並青蓋犢車十餘爭道,不避儀仗。軾使御營巡檢使問之,乃皇后及大長公主。時御史中丞李之純為
 儀仗使,軾曰:「中丞職當肅政,不可不以聞之。」純不敢言,軾於車中奏之。哲宗遣使繼疏馳白太皇太后,明日,詔整肅儀衛,自皇后而下皆毋得迎謁。尋遷禮部兼端明殿、翰林侍讀兩學士,為禮部尚書。高麗遣使請書,朝廷以故事盡許之。軾曰:「漢東平王請諸子及《太史公書》,猶不肯予。今高麗所請,有甚於此,其可予乎?」不聽。



 八年,宣仁后崩,哲宗親政。軾乞補外,以兩學士出知定州。時國事將變,軾不得入辭。既行,上書言:「天下治亂,出於下情
 之通塞。至治之極,小民皆能自通;迨於大亂,雖近臣不能自達。陛下臨御九年,除執政、臺諫外,未嘗與群臣接。今聽政之初,當以通下情、除壅蔽為急務。臣日侍帷幄,方當戍邊,顧不得一見而行,況疏遠小臣欲求自通,難矣。然臣不敢以不得對之故,不效愚忠。古之聖人將有為也,必先處晦而觀明,處靜而觀動,則萬物之情,畢陳於前。陛下聖智絕人,春秋鼎盛。臣願虛心循理,一切未有所為,默觀庶事之利害,與群臣之邪正。以三年為期,
 俟得其實,然後應物而作。使既作之後,天下無恨,陛下亦無悔。由此觀之,陛下之有為,惟憂太蚤,不患稍遲,亦已明矣。臣恐急進好利之臣,輒勸陛下輕有改變,故進此說,敢望陛下留神,社稷宗廟之福,天下幸甚。」



 定州軍政壞馳,諸衛卒驕惰不教,軍校蠶食其廩賜,前守不敢誰何。軾取貪污者配隸遠惡,繕修營房,禁止飲博,軍中衣食稍足,乃部勒戰法,眾皆畏伏。然諸校業業不安,有卒史以贓訴其長,軾曰:「此事吾自治則可,聽汝告,軍中
 亂矣。」立決配之,眾乃定。會春大閱,將吏久廢上下之分,軾命舉舊典,帥常服出帳中,將吏戎服執事。副總管王光祖自謂老將,恥之,稱疾不至。軾召書吏使為奏,光祖懼而出,訖事,無一慢者。定人言:「自韓琦去後,不見此禮至今矣。」契丹久和,邊兵不可用,惟沿邊弓箭社與寇為鄰,以戰射自衛,猶號精銳。故相龐籍守邊,因俗立法。歲久法弛,又為保甲所撓。軾奏免保甲及兩稅折變科配,不報。



 紹聖初,御史論軾掌內外制日,所作詞命,以為譏
 斥先朝。遂以本官知英州,尋降一官,未至,貶寧遠軍節度副使,惠州安置。居三年,泊然無所蒂芥,人無賢愚,皆得其歡心。又貶瓊州別駕,居昌化。昌化,故儋耳地,非人所居,藥餌皆無有。初僦官屋以居,有司猶謂不可,軾遂買地築室,儋人運甓畚土以助之。獨與幼子過處,著書以為樂,時時從其父老游,若將終身。



 微宗立,移廉州,改舒州團練副使,徒永州。更三大赦,遂提舉玉局觀,復朝奉郎。軾自元祐以來,未嘗以歲課乞遷,故官止於此。建
 中靖國元年,卒於常州,年六十六。



 軾與弟轍,師父洵為文,既而得之於天。嘗自謂:「作文如行雲流水,初無定質,但常行於所當行,止於所不可不止。」雖嬉笑怒罵之辭,皆可書而誦之。其體渾涵光芒,雄視百代,有文章以來,蓋亦鮮矣。洵晚讀《易》,作《易傳》未究,命軾述其志。軾成《易傳》,復作《論語說》;後居海南,作《書傳》;又有《東坡集》四十卷、《後集》二十卷、《奏議》十五卷、《內制》十卷、《外制》三卷、《和陶詩》四卷。一時文人如黃庭堅、晁補之、秦觀、張耒、陳師道,舉
 世未之識,軾待之如朋儔,未嘗以師資自予也。



 自為舉子至出入侍從,必以愛君為本,忠規讜論,挺挺大節,群臣無出其右。但為小人忌惡擠排,不使安於朝廷之上。



 高宗即位,贈資政殿學士,以其孫符為禮部尚書。又以其文置左右,讀之終日忘倦,謂為文章之宗,親制集贊,賜其曾孫嶠。遂崇贈太師,謚文忠。軾三子:邁、迨、過,俱善為文。邁,駕部員外郎。迨,承務郎。



 過字叔黨。軾知杭州,過年十九,以詩賦解兩浙路,禮部
 試下。及軾為兵部尚書,任右承務郎。軾帥定武,謫知英州,貶惠州,遷儋耳,漸徙廉、永,獨過侍之。凡生理晝夜寒暑所須者,一身百為,不知其難。初至海上,為文曰《志隱》,軾覽之曰:「吾可以安於島夷矣。」因命作《孔子弟子別傳》,軾卒於常州,過葬軾汝州郟城小峨眉山,遂家穎昌,營湖陰水竹數畝,名曰小斜川,自號斜川居士。卒,年五十二。



 初監太原府稅,次知穎昌府郾城縣,皆以法令罷。晚權通判中山府。有《斜川集》二十卷。其《思子臺賦》、《颶風賦》
 早行於世。時稱為「小坡」,蓋以軾為「大坡」也。其叔轍每稱過孝,以訓宗族。且言:「吾兄遠居海上,惟成就此兒能文也。」七子:鑰、籍、節、笈、篳、笛、箾。



 論曰:蘇軾自為童子時,士有傳石介《慶歷聖德詩》至蜀中者,軾歷舉詩中所言韓、富、杜、範諸賢以問其師。師怪而語之,則曰:「正欲識是諸人耳。」蓋已有頡頏當世賢哲之意。弱冠,父子兄弟至京師,一日而聲名赫然,動於四方。既而登上第,擢詞科,入掌書命,出典方州。器識之閎
 偉,議論之卓犖,文章之雄雋,政事之精明,四者皆能以特立之志為之主,而以邁往之氣輔之。故意之所向,言足以達其有猷,行足以遂其有為。至於禍患之來,節義足以固其有守,皆志與氣所為也。仁宗初讀軾、轍制策,退而喜曰:「朕今日為子孫得兩宰相矣。」神宗尤愛其文,宮中讀之,膳進忘食,稱為天下奇才。二君皆有以知軾,而軾卒不得大用。一歐陽修先識之,其名遂與之齊,豈非軾之所長不可掩抑者,天下之至公也,相不相有命
 焉,嗚呼!軾不得相,又豈非幸歟?或謂:「軾稍自韜戢,雖不獲柄用,亦當免禍。」雖然,假令軾以是而易其所為,尚得
 為軾哉?



\end{pinyinscope}