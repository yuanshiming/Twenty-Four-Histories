\article{志第一百二十二 職官九(敘遷之制)}

\begin{pinyinscope}

 群
 臣敘遷流內銓流外出官文散官武散官爵勛功臣檢校官兼官試秩紹興以後階官



 文臣京官至三師敘遷之制



 諸寺、監主簿,秘書省校書郎,秘書省正字有出身轉大理評事,無出身轉太常寺奉禮郎。內帶館職同有出身,後族、兩府之家轉太祝。



 太常寺太祝,奉禮郎有出身轉諸寺、監丞,無出身轉大理評事,內帶館職同有出身。



 大理評事有出身轉大理寺丞,第一人及第轉著作佐郎;無出身轉諸寺、監丞。內帶館職同有出身。後族、兩府之家,審刑院詳議,刑部詳覆、詳斷、檢法、法直官,轉光錄寺丞。



 諸寺、監丞有出身轉著作佐郎,無出身轉大理寺丞。內帶館職同有出身。



 大理款丞有出身轉殿中丞,元出身轉太子中舍。內帶館職同有出身,或轉太子中允。後族、兩府之家,審刑院詳議,刑部詳覆、詳斷,中書堂後官,轉太子右贊善大夫。



 著作佐郎有出身轉秘書丞,內第一人及第太常丞;無出身轉太子左贊善大夫。內帶館職同有出身。特旨轉秘書郎、著作郎、宗正丞。



 太子左右贊善大夫、中舍、洗馬轉殿中丞。內帶館職轉太常丞。太子中允轉太常丞,特旨轉秘書郎、著作郎、宗正丞。



 太常、宗正、秘書丞,著作郎,秘書郎轉太常博士,特旨轉左、右正言,監察御史。宗正丞,無出身轉國子博士。



 殿中丞有出身轉太常博士,無出身轉國子監博士。內帶館職同有出身。



 太常、國子博士轉後行員外郎,特旨轉左、右司諫,殿中侍御史。



 左、右正言轉左、右司諫,帶待制已上職轉起居舍人。



 監察御史轉殿中侍御史。



 後行員外郎轉中行員外郎,特旨轉起居舍人、侍御史。



 左、右司諫轉起居郎、起居舍人,帶待制已上職轉吏部員外郎。



 殿中侍御史轉侍御史。



 中行員外郎轉前行員外郎。



 起居郎,起居舍人轉兵部員外郎,帶待制已上職轉禮部郎中。



 侍御史轉同封員外郎。



 前行員外郎轉後行郎中。



 後行郎中轉中行郎中。



 中行郎中轉前行郎中。



 右常調轉員外郎者轉右曹。內有出身自屯田,無出身自虞部,贓罪敘復人自水部轉。水部司門庫部虞部比部駕部屯田都官職方



 任發運、轉運使副,三司、天封府判官,侍讀,侍講,天章閣侍講,崇政殿說書、開封府推官、府界提點,三
 司子司主判官,大理少卿,提點刑獄,提點鑄錢監,諸王府翊善、侍讀、記室,中書提點五房公事堂後官轉左曹。



 內有出身自祠部,無出身自主客,堂後官自膳部轉。



 膳部倉部考功主客金部司勛祠部度支司封



 任發運、轉運使副,三司、開封府判官,左曹轉左名曹。內無出身只轉祠部、度支、司封,有出身合轉右名曹,準此。任三司副使,知雜,修撰,修起居注,直舍人院,轉左名曹。
 工部刑部兵部



 帶待制已上職,左右曹、右名曹轉左名曹,仍隔一資超轉。中行郎中轉左、右司郎中。



 戶部轉左司,刑部、度支、金部、倉部、都官、比部、司門轉右司。禮部戶部吏部



 前行郎中有出身轉太常少卿,無出身轉司農少卿,內見任左曹衛尉少卿,帶待制已上職轉右諫議大夫。



 左、右司郎中帶待制已上職轉諫議大夫。左司轉左諫議,右司轉左諫議,帶翰林學十者,轉
 中書舍人。



 衛尉、司農少卿轉光祿少卿,帶館職轉光祿卿。



 光祿少卿



 轉司家卿,帶館職轉光祿卿。



 太常少卿轉光祿卿,任三司副使、修撰,取旨。



 司家卿轉少府監,帶館職轉光祿卿。



 少府監轉衛尉卿,帶館職轉光祿卿。



 衛尉轉光祿卿。



 光祿卿轉秘書監。



 秘書監轉太子賓客。



 中書舍人轉禮部侍郎。



 諫議大夫轉給事中。



 給事中轉工部侍郎,帶翰林學士己上職轉禮部侍郎。



 太子賓客轉工部侍郎。



 工部侍郎轉刑部侍郎,兩府轉戶部侍郎,宰相轉兵部侍郎。



 禮部侍郎轉戶部侍郎,宰相轉吏部侍郎。



 刑部侍郎轉兵部侍郎,兩府轉吏部侍郎,宰相轉禮部尚書。



 戶部侍郎轉吏部侍郎,宰相轉禮部尚書。



 兵部侍郎轉右丞,兩府轉左丞,宰相轉禮部尚書。



 吏部侍郎轉左丞,宰相轉禮部尚書。



 左、右丞轉工部尚書,兩府轉禮部尚書。



 工部尚尚書轉禮部尚書,兩府轉刑部尚書。



 禮部尚書轉刑部尚書,兩府轉戶部尚書。



 刑部尚書轉戶部尚書,兩府轉兵部尚書。



 戶部尚書轉兵部尚書,兩府轉吏部尚書。



 兵部尚書轉吏部尚書,兩府轉太子少保,宰相轉右僕射。



 吏部尚書轉太子少保,宰相轉左僕射。



 太子少保轉太子少傅。



 右僕射轉左僕射。



 太子少傅轉太子少師。



 左僕射轉司空。



 司空轉司徒。



 太子少師轉太子太保



 司徒轉太保。



 太子太保轉太子太傅。



 太子太傅轉太子太師。



 太子太師轉傅太保。



 太保轉太傅。



 太傅轉太尉。



 太尉轉太師。



 太師太師、太傅、太保謂之三師,太尉、司徒、司空謂之三公。凡除授,則自司徒遷太保,自太傅遷太尉,
 檢校亦如之。



 治平三年,翰林學士賈黯奏:「近者皇子封拜,並除檢校太傅。臣按官儀,自後魏以來,以太師、太傅、太保為三師,太尉、司徒、司空為三公,國朝因之。《六典》曰:『三師,訓導之官也。』蓋天子之所師法。今皇太子以師傅名官,於義弗安,莫甚於此,蓋前世因循,失於厘正。臣愚以謂自今皇子及宗室卑者除官,並不可帶師傅之名,隨其敘遷改授三公之官。」詔候將來因加改正。自此,皇子及宗室卑行,遂不除三師官。



 宋初,臺、省、寺、監官猶多蒞本司,亦各有員額資考之制,各以曹署閑劇著為月限,考滿則遷,慶恩止轉階、勛、爵、邑。建隆二年,始以右監門衛將軍魏仁
 滌為右神武將軍,水部員外郎朱洞為都官員外郎,監察御史李鑄為殿中侍御史,以仁滌等掌曲薛、領關征外有羨也。自是,廢歲滿敘遷之典。是後,多掌事於外,諸司互以他官領之,雖有正官,非別受詔亦不領本司之務。又官有其名而不除者甚眾,皆無定員、無月限,不計資品,任官者但常食其奉而已。時議以近職為貴,中外又以差遣別輕重焉。



 武臣三班借職至節度使敘遷之制三班借職以下,亦有磨勘轉官法,緣未受真命,今不具錄。



 三班借職轉三班奉職。



 三班奉職轉右班殿直。



 右班殿直轉左班殿直。



 左班殿直轉右侍禁



 右侍禁轉右侍禁。



 左侍禁轉西頭供奉官。



 西頭供奉官轉東頭供奉官。



 東頭供奉官轉內殿崇班。



 內殿崇班轉內殿承制。



 內殿承制制轉供備庫使,有戰功轉禮賓副使,特旨東西染院、西京作坊



 副使,有戰功,並謂曾經轉官酬獎。



 供備庫使轉西京左藏庫副使,有戰功轉如京副使。



 禮賓副使轉崇儀副使,有戰功轉洛苑副使。



 西染院副使轉如京副使,有戰功轉內園副使。



 東染院副使轉洛苑副使,有戰功轉六宅副使。



 西染院使轉如京使,有戰功轉內園使。



 東染院使轉洛苑使,有戰功轉文思使。



 西京作坊使轉文思使,有戰功轉莊宅使。



 西京左藏庫使轉六宅使,有戰功轉西作坊使。



 崇儀使轉六宅使,有戰功轉西作坊使。



 如京使轉莊宅使,有戰功轉東作坊使。



 洛苑使轉西作坊使,有戰功轉左藏庫使。



 內園使轉東作坊使,有戰功轉內藏庫使。



 文思轉轉左藏庫使,有戰功轉右騏驥使。



 六宅使轉內藏庫使,有戰功轉左騏驥使。



 莊宅使轉右騏驥使,有戰功轉宮苑使。



 西作坊使轉左騏驥使,有戰功轉宮苑使。



 東作坊使轉宮苑使。



 左藏、內藏、左右騏驥、宮苑使並轉皇城使。



 皇城使轉遙郡刺史。凡已上使、副,除皇城系東班,餘並西班。其東班翰林以下十九司使、副,雖有
 見在官及遷轉法,並授伎術官。



 遙郡刺史轉遙郡團練使,特旨轉正刺史。



 遙郡團練使轉遙郡防禦使,特旨轉正團練使。



 刺史轉團練使。



 團練使,遙郡防禦使轉防禦使。



 防禦使轉觀察使。



 觀察使轉節度觀察留後。



 節度觀察留後轉節度
 使。



 節度使



 武臣自通事舍人



 轉橫班例



 通事舍人轉西上合門副使。其東上合門副使,非特恩不遷。



 東、西上合門副使轉引進副使。



 引進副使轉客省副使。



 客省副使轉西上合門使。



 西上合門使轉東上合門使。



 東上合門使轉四方
 館使。



 四方館使轉引進使。



 引進使轉客省使。



 客省使



 右內客省使至合門使謂之橫班,皇城使以下二十名謂之東班,洛苑使以下二十名謂之西班,初猶有正官充者,其後但以檢校官為之,或領觀察使、防禦使、團練使、刺史。



 景祐元年詔:「副使自今改正使,於本額下五資遷之。」舊無定員,慶歷四年詔:「客省、引進、四方館使各一人,東、西上合門使共四人,合門、引進、客省副使共
 六人,合門通事舍人八人。」治平二年,樞密院奏:「嘉祐三年詔:『非軍職當罷、橫行歲滿當遷及有戰功殊績,皆不得除正任。當遷,則改州名,或加檢校官、勛、封,食邑。』自降詔以來,正任刺史以上絕升進之望。今欲因知藩要州郡,或路分總管,如再經改州名或加檢校官、勛、封食邑已及十年者,與遷官,至節度觀察留後止。又客省、引進、四方館舊置使三員,東、西上合門舊置使四員,今並增為六員。合門、引進、客省,舊制副使六員,今並增為八員。合門舊通事舍人八員,今增為十員。凡所增置,須見任官當遷及有闕乃補。其皇城使改官及七年,如曾歷邊任、有本路監司總管五人已上共薦者,欲除遙郡刺史至遙郡防禦使止。」詔:「自今皇城、宮苑副使當磨勘者,各於本班使額自下升五資改諸司使。其自左藏庫副使已上因酬獎及非次改官者,聽如舊。餘皆從樞密院之請。」初,英宗謂執政曰:「諸司
 副使改轉使,當從供備庫使始,今對行升五資,太優。」於是合議條奏而為此例。



 宗室自率府副率至侍中敘遷之制



 太子右內率府副率轉太子右監門率府率。



 太子右監門率府率轉右千牛衛將軍。



 右千牛衛將軍轉右監門衛大將軍。



 右監門衛大將軍轉遙郡刺史。



 遙郡刺史轉遙郡團練使,繼諸王后、見封國公及特旨,即轉正刺史。



 遙郡團練使轉遙郡防禦使。繼諸王后、見封國公及特旨,即轉正團練使。



 刺史轉團練使



 團練使轉防禦使



 防禦使轉觀察使。



 觀察使轉節度觀察留後。



 節度觀察留後轉節度使,特旨轉左、右衛上將軍。



 左、右衛上將軍節度使轉節度使同中書門下平章事。



 節度使同中書門下平章事轉節度使兼侍中。



 節度使兼侍中。



 內臣自皇城使特恩遷轉例合該磨勘,並臨時用例,取旨改轉。



 皇城使轉昭宣使。國朝亦有外官為昭宣使者。



 昭宣使轉宣政使。



 宣政使轉宣慶使。



 宣慶使轉景福殿使。



 景福殿使轉延福宮使。



 延福宮使凡不轉昭宣已上五使者,並轉遙郡。



 入內內侍省內臣敘遷之制。



 祗候班雖有轉官法,近年無遷轉之人,惟敘官者一級當一官,內侍省同。



 北班內品轉後苑散內品。



 後苑散內品轉後苑勾當事內品



 後苑勾當事內品轉後苑內品。



 後苑內品轉把門內品。



 把門內品轉入內內品。



 入內內品轉貼祗候內品。



 貼祗候內品轉祗候小內品。



 祗候小內品轉祗候內品。



 祗候內品轉祗候高班內品。



 祗候高班內品轉祗候高品。



 祗候高品轉祗候殿頭。



 祗候殿頭



 右系責降及責降人保引。



 內待班轉黃門。



 黃門轉高
 班。



 高班轉高品。



 高品轉殿頭。



 內侍殿頭轉內西頭供奉官。



 內西頭供奉官轉內東頭供奉官。



 內東頭供奉官東頭供奉官已上轉官,依外官。



 內侍省內臣敘遷之制。



 祗候班



 後苑散內品轉散內品。



 散內品轉北班內品。



 北班內品轉後苑勾當事內品。



 後苑勾當事內品轉後苑內品。



 把門內品、後苑內品轉內品。



 內品轉貼祗候內品。



 貼祗候內品轉祗候內品。



 祗候內品轉祗候高班內品。



 祗候高班內品轉祗候高品。



 祗候高品



 右系責降及責降人保引亦有非賣降由奏薦而除者。入內內侍省同。



 內侍班



 黃門轉高班。



 高班轉高品。



 高品轉殿頭。



 殿頭轉內西頭供奉官。



 內西頭供奉官轉內東頭供奉官。



 內東頭供奉官東頭供奉官已上轉官,依外官例。



 右宋初以來,內侍未嘗磨勘轉官,唯有功乃遷。至景祐中,詔:「內臣入仕三十年,累有勤勞,經十年未嘗遷者,奏聽旨。」猶無磨勘定格也。慶歷以後,其制漸隳。黃門有勞至減十五年,而入仕才五七年有勞至高品已上者,兩省因著十年磨勘之例,而減年復在其中。嘉祐六年,樞密院始議厘革。乃詔:「內臣入仕並三十年磨勘,已磨勘者,其以勞得減年
 者毋得過五年。」



 選人選京官之制



 有出身:



 判、司、簿、尉,七考除大理寺丞。不及七考,光祿寺丞。不及五考,大理評事。不及三考,奉禮郎。



 初等職官,知令、錄,六考除大理寺丞。不及六考,光祿寺丞。不及三考,大理評事。



 兩使職官,知令、錄,六考除著作佐郎。不及六考,大理寺丞。不及三考,
 光祿寺丞。



 支、掌、防、團判官,六考除太子中允。不及六考,著作佐郎。



 節、察判官,六考除太常丞,不及六考,太子中允。



 無出身:



 判、司、簿、尉,七考除衛尉寺丞。不及七考,大理評事。不及五考,奉禮郎。不及三考,守將作監主簿。



 初等職官,知令、錄,六考除衛尉寺丞。不及六考,大理評事。不及三考,奉禮郎。



 兩使職官,
 知令、錄,六考除大理寺丞。不及六考,衛尉寺丞,不及三考,大理評事。



 支、掌、防、團判官,六考除著作佐郎。不及六考,大理寺丞。



 節、察判官,六考除太子中允。不及六考,著作佐郎。



 吏部流內銓諸色入流及循資磨勘選格入流



 有出身:



 進士、明經入望州判、司,次畿簿、尉。



 《九經》入緊州判、司,望縣簿、尉。



 諸科、《五經》、《三禮》、《三史》、《三傳》,今雖無此科,緣見有逐色人。



 明法入上州判、司,緊縣簿、尉。



 學究、武舉得班行人換授,入中州判、司,上縣簿、尉。



 無出身:



 太廟齊郎舊室長同。



 入中下州判、司,中縣簿、尉。



 郊社齊郎、舊掌坐同。試銜白衣送銓注官,司士、文學、參軍、長史、司馬、助教得正官,並班行試換文資,入下州判、司,中下縣簿、尉。



 三色人:



 攝官入小縣簿、尉。



 進納授試銜,入下州判、司,中下縣簿、尉。」



 授太廟齊郎,入中州判、司,中縣簿、尉;流外入下縣簿、尉。



 已上並許超折地望注授。



 循資



 常調:



 判、司、簿、尉有出身兩任四考,無出身兩任五考,攝官出判、司三任七考,並入錄事參軍。但有舉主四人或有合使舉主二人,並許通注縣令,流外出身四任十考,入錄事參軍。內系驅使官、沿堂五院人,只注大郡判、司,大縣簿、尉。進納出身三任七考,曾省試下第二任五考,入下州令、錄,仍差監當。



 酬獎:



 判、司、簿、尉初任循一資入知令、錄,次任二考已上入
 正令、錄。



 知令、錄循一資入初等職官,正令錄入兩使職官。



 初等職官循一資入兩使職官,兩資入支、掌、防、團判官,三資入節、察判官。



 恩例:



 判、司、簿、尉用祖父五路及廣、桂知州帶安撫。並知成都府、梓州及川、廣轉運提刑等恩例陳乞,循入試銜知縣,仍差監當。



 奏薦:



 判、司、簿、尉。



 舉職官,有出身四考、有舉主三人,移初等職官,仍差知縣。有出身四考、無出身六考注初等職官。有出身六考、無出身七考注兩使職官。



 舉縣令,有出身三考、無出身四考,攝官出身六考、有舉主三人,進納出身六考、有舉主四人,流外出身三任七考、有舉主六人,並移縣令。內流外人入錄
 事參軍。



 令、錄系舉人入,任內有京官舉主二人,循兩使職官、知縣。



 初等職官、知縣系舉人入,任內有京官職舉主二人,循兩使職官,如願知縣者聽。



 磨勘:



 判、司、簿、尉七考,知令、錄、職官六考,有京官舉主五人,內一員轉運使、副或提刑,並磨勘引見,轉合入京
 朝官。



 兩使職官、知縣系舉人入,並因舉循入,任內有京官舉主二人,磨勘引見,轉合入京官。



 令、錄流外出身,系舉人入,任內有班行舉主三人,磨勘引見,改換班行。



 差攝:



 長史、文學兩舉進士三舉諸科特恩與攝官



 已上,廣南東路長史、文學與舉人,中半差攝;西路長史、文學七分,舉人二分,特恩攝官一分。



 試補:



 正額及額外攝官並試公案,以合格名次高下差攝。內試不中及不能就試者,並在試中人之下。



 解發:



 入額人一任實滿四年與解發。如差監當、監稅,即以二年為一任,理兩攝,並解發赴銓。海北攝官差往海南,減一年。犯
 公罪展攝二年,監當虧少課利罰半月奉者,添攝一任,罰一月奉者添攝兩任。



 流外出官法



 尚書省書令史、都省二十四司、禮部貢院、吏部流內銓、官誥院七選,都省敕庫、兵部甲庫八選,諸司驅使官、都省散官十九選,貢院散官十八選:並補正名後理,或酬獎,減一等出簿、尉。



 門下省白院令史七選,畫頭、書院、甲庫令史贊者八選,並補正名後理;驅使官九選,授勒留官後理:並出簿、尉。



 中書省白院令史七選,甲庫令史八選,並補正名後理;驅使官九選,授勒留後理,並出簿、尉。學士院錄事補正名後理,三年出奉職。孔目官遇大禮,從上出一名,不遇大禮七選;驅使官遇大
 禮,從上三人並出簿、尉,不遇恩十選,並授勒留官後理。



 御史臺令史七選補正名,驅使官九選授勒留官,並出簿、尉。引贊官補正名後,遇大禮出錄事參軍。試中刑罰人充主推,五年出奉職。書史五年,出借職。系諸處取到人充主推,八年出借職。書史出三班差使。



 三司三部都孔目官三年出西頭供奉官;前、後行入仕三十年已上,遇大禮,從上各出二人,前行出奉職,後行出借職;子司勾覆、開拆官五年出左、右班殿直,前、後行出二人。同三部衙司都押衙三年出奉職,衙佐三年出借職;通引官行首司五年出奉職:並補正名後理。



 開封府孔目官補正名後理,五年出右班殿直。左知客押衙六年、通引官左番行首七年的出奉職,
 並補正名後理。支計官、勾覆官、開拆官、接押官出奉職,諸司行首前行出借職,並遇大禮,以入仕及三十年已上者三人出職。



 殿前司孔目官五年出右侍禁,通引官行首三年出奉職,並補正名後理。



 馬步軍司孔目官五年出右班殿直,通引官行首三年出借職,並補正名後理。



 入內、內侍兩省前、後行補正名後理,三年出奉職。



 大宗正司勾押官補正名後理,三年出借職。



 三班院勾押官補正名後理,五年出奉職。



 審官院令史授勒留官後理,七年出簿、尉。



 九寺府史,太常、大理寺七選;宗正、光祿、太府、太僕、衛尉、鴻臚、司農寺十選;驅使官十九選;宗正司楷書八選:並補正名後理,出簿、尉。



 諸監都水監勾押官補正名後理,三年出奉職。少府、將作監府史十選,國子監八選,司天監禮生、歷生選,少府,將作監驅使官十九選:並補正名後理,出簿、尉。



 群牧司都勾押官補正名後理,三年出奉職。



 客省行首補正名後理三年,勾押官五年,並出奉職。承受並驅使官授勒留官後理,七選出簿、尉。



 四方館書令史補正名後理,八選;表奏官、驅使官授勒留官後理,九選,並出簿、尉。



 合門行首補正名後理,三年出右侍禁。承受授勒留官後理,七選出簿、尉。



 太
 常禮院禮直官自補副禮直官後,六經大禮,出西頭供奉官。禮生補正名後理,六選出簿、尉。



 審刑院充本院書令史後理,六選出簿、尉。



 秘書殿中省令史、楷書並補正名後理,八選出簿、尉。



 起居院楷書八選、驅使官十九選,並補正名後理,出薄、尉。



 崇文院孔目官補正名後理,遇大禮,出奉職。



 三館孔目官、四庫書直官八選,楷書七選,書直、書庫、表奏官九選,守當官十選,並授勒留官後理;楷書補正名後理:並出簿、尉。



 秘閣典書、楷書並補正名後理,七選出簿、尉。



 軍頭引見司勾押官補正名後理,五年出右班殿直。



 皇城司勾押官補正名後理,三年出奉職。



 內東門司押司官補正名後理,三年出借職。



 管勾往來國信所勾押官補正名後理,三年出奉職。



 翰林司專知官三年界滿,大將,出奉職。



 內藏庫專知官三年界滿,出借職。



 御藥院押司官補正名後理,三年出借職。



 御書院待詔五年出左班殿直,書藝十年出右班殿直,御書祗候十五年出借職,並補正名後理。



 進奏院進奏官補正名後理,十五年遇大禮,無過犯,從上五人出職。有過犯經洗雪,曾經決責,出借職。人數無定限。



 進廚勾押官補正名後理,三年出職。



 金吾街司、仗司孔目官,表奏、勾押、驅使官,並補正名後理,十九選出簿、尉。



 文臣換右職之制



 秘書監換防禦使。



 大卿、監換團練使。



 秘書少監,太常、光祿少卿換刺史。



 少卿、監換皇城命名、遙郡刺吏。



 帶職郎中換合門使。



 前行郎中換宮苑使。



 中行郎中換內藏庫使。



 後行郎中換莊宅使。



 帶職前行員外郎



 前行員外郎並並換洛苑使。



 帶職中行員外郎,起居舍人,侍御史,
 中行員外郎並換西京作坊使。



 帶職後行員外郎,左、右司諫,殿中侍御史,後行員外郎並換供備為使。已上並帶遙郡刺史。



 帶職博士,左、右正言,監察御史換合門副使。



 太常博士換內藏庫副使。



 國子博士換左藏庫副使。



 太常丞換莊宅副合,



 秘書丞換六宅
 副使。



 殿中丞,著作郎換文思副使。



 太子中允換禮賓副使。



 太子左右贊善大夫、中舍、洗馬換供備庫副使。



 秘書郎,著作佐郎換內殿承旨。



 大理寺丞換內殿崇班。



 諸司監丞,節度、觀察判官換東頭供奉官。



 大理評事,節度掌書記,觀察支使換西頭供奉官。



 太常寺太祝,奉禮郎換左侍禁。



 初等職官,知令、錄並兩使職官,防禦、團練判官,令、錄未及三考換左班殿直。



 初等職官,知令、錄未及三考換右班殿直。



 判、司、簿、尉換三班奉職。



 試銜齊郎並判、司、簿、尉未及三考換三班借職,已上京官至太常丞帶職,加一資換。



 右文官換右職者,除流外、進納及犯私罪情重並贓罪外,年四十以下並許試換右職。三班使臣補
 換及三年、差使及五年,方許試換。已上並召京朝官或使臣二人委保。其文臣待制、武臣觀察使已上原換官,取旨。



 紹興復修試換之令,淳熙增廣尚左、尚右、待左、侍右換官之格,列而書之,以見新式。若中大夫而下文臣換官。仍政和舊制,則不書。



 諸訓武郎至進武校尉,不曾犯贓私罪及笞刑經決而願換文資者,聽召保官二員,具家狀連保狀二本,詣登
 聞鼓院投進乞試。



 外任人候替罷就試。文資換武者聽。準此,即授小使臣後未及三年,授進武校尉後未及五年,三省、樞密院書令史以下授使臣、進武校尉;若保甲及試武藝並進納、流外出身,不用此令。諸武臣試換文資,於《易》、《詩》、《周禮》、《禮記》各專一經,仍兼《論》、《孟》;原試詩賦及依法官條試斷案、《刑統》大義者,聽。



 換官:尚右,訓武、修武郎換宣教郎。侍左,承直郎換從義郎。文林、從政郎奏舉職官、知縣同。



 換忠翊郎,未滿三考成忠郎。
 從事、修職換成忠郎,未滿三考保義郎。迪功郎換成節郎,未滿三考承信郎。將仕郎換承信郎,侍右,從義郎換宣義郎。秉義郎換承事郎。忠訓郎換承奉郎。忠翊郎換承務郎。成忠郎換從事郎。保義郎換修職郎。承節、承信郎換迪功郎。進武校尉、進義校尉換將仕郎。蔭補換使臣。承奉郎換忠翊郎。承務郎換成忠郎。文林郎換保義郎。從事、從政、迪功、通事郎換成節郎。登仕、將仁郎換承信郎。



 文散官二十九



 開府儀同三司從一



 特進正二



 光祿大夫從二



 金紫光祿大夫正三



 銀青光錄大夫從三



 正奉大夫正四上階



 中奉大夫正四



 太中大夫從四上階



 中大夫從四



 中散大夫正五上



 朝奉大夫正五



 朝散大夫從五上



 朝請大夫從五



 朝奉郎正六上



 承直郎正六



 奉直郎從六上



 通直郎從六



 朝請郎正七上



 宣德郎正七



 朝散郎從七上



 宣奉郎從七



 給事郎正八上



 承事郎正八



 承奉郎從八上



 承務郎從八



 儒林郎正九上



 登仕郎正九



 文林郎從九上



 將仕郎從九



 右朝官階、勛高,遇恩加八大夫。



 武散官三十一



 驃騎大將軍從一



 輔國大將軍正二上



 鎮國大將軍正二



 冠軍大將軍正三上



 懷化大將軍正三



 雲麾將軍從三上



 歸德將軍從三



 忠武將軍正四上



 壯武將軍正四



 宣威將軍從四上



 明威將軍從四



 定遠將軍正五上



 寧遠將軍正五



 游騎將軍從五上



 游擊將從五



 昭武校尉正六上



 昭武副尉正六



 振威校尉從六上



 振威副尉從六



 致果校尉正七上



 致果副尉正七



 翊麾校尉從七上



 翊麾副尉從七



 宣節校尉正八上



 宣節副尉正八



 御武校尉從八上



 御武副尉從八



 仁勇校尉正九上



 仁勇副尉正九



 陪戎校尉從九上



 陪戎副衛從九



 右文散官階上經恩加一階,郎階上京朝官加五階,選人加一階,武散官冠軍大將軍、使相、節度使起復,改授游擊將軍,雖中書主事、諸司吏人加授,亦無累加法,餘不常授。已上文武三品已上服紫,五品已上服緋,九品已上服綠。



 《元豐寄祿格》以階易官,雜取唐及國朝舊制,自開府儀
 同三司至將仕郎,定為二十四階,崇寧初,因刑部尚書鄧洵武請,又換選人七階。大觀初又增宣奉、正奉、中奉、奉直等階。政和末,又改從政、修職、迪功,而寄祿之格始備。自開府至迪功凡三十七階。



 新官



 舊官



 開府儀同三司



 使相謂節度使兼侍中、中書令、或同平章事



 特進



 左、右僕射



 金紫光祿大夫



 吏部尚書



 銀青光祿大夫



 五曹尚書



 光祿大夫



 左、右丞



 宣奉大夫大觀新置。



 正奉大夫大觀新置。



 正議大夫



 六曹侍郎



 通奉大夫大觀新置。



 通議大夫



 給事中



 太中大夫



 右、右諫議大夫



 中大夫



 秘書監



 中奉大夫大觀新置。



 中散大夫



 光祿卿至少府監



 朝議大夫



 太常卿、少卿,左、右司郎中



 奉直大夫大觀新置。



 朝請大夫



 前行郎中



 朝散大夫



 中行郎中



 朝奉大夫



 後行郎中



 朝請郎



 前行員外郎,侍御史



 朝散郎



 中行員外郎,起居舍人



 朝奉郎



 後行員外郎,左、右司諫



 承議郎



 左、右正言,太常、國子博士



 奉議郎



 太常、秘書、殿中丞,著作郎



 通直郎



 太子中允、贊善大夫、洗馬



 宣教郎



 著作佐郎,大理寺丞



 元豐本「宣德」,政和避宣德門改。



 宣義郎



 光祿衛尉寺、將作監丞



 承事郎



 大理評事



 承奉郎



 太祝,奉禮郎



 承務郎



 校書郎,正字,將作監主簿



 承直郎



 留守、節察判官



 儒林郎



 節察掌書記、支使,防、團判官



 文林郎



 留守、節察推官,軍、監判官



 從事郎承直至此四階,並崇寧初換。



 防、團推官,監判官



 從政郎崇寧通仕,政和再換。



 錄事參軍,縣令



 修職郎崇寧登仕,政和再換。



 知錄事參軍,知縣令



 迪功郎崇寧將仕,政和再換。



 軍巡判官,司理,司法,司戶,主簿、尉



 國朝武選,自內客省至合門使、副為橫班,自皇城至供備庫使為諸司正使,副為諸司副使,自內殿承制至三班借職為使臣,元豐未及更,政和二年,乃詔易以新名,正使為大夫,副使為郎,橫班十二階使、副亦然。六年,及增置宣正、履正、協忠、翊衛、親衛大夫郎,凡十階,通為橫
 班。自太尉至下班祗應,凡五十二階。



 新官



 舊官



 太尉政和新置,以太尉本秦之主兵官、遂定為武階之首。



 通侍大夫



 內客省使



 正侍大夫



 延福宮使



 宣正大夫



 履正大夫



 協忠大夫並政和新置。



 中侍大夫



 景福殿使



 中亮大夫



 客省使



 中衛大夫



 引進使



 翊衛大夫



 親衛大夫



 拱衛大夫並政和增置。



 左武大夫



 東上合門使



 右武大夫



 西上合門使



 正侍郎



 宣正郎



 履正郎



 協忠郎



 中侍郎並政和增置。



 中亮郎



 客省副使



 中衛郎



 引進副使



 翊衛郎



 拱衛郎並政和增置。



 左武郎



 東上合門副使



 右武郎



 西上合門副使



 武功大夫



 皇城使



 武德大夫



 宮苑、左右騏驥、內藏庫使



 武顯大夫



 左藏庫、東西作坊使



 武節大夫



 莊宅、六宅、文思使



 武略大夫



 內園、洛苑、如京、崇儀使



 武經大夫



 西京左藏庫使



 武義大夫



 西京作坊、東西染院、禮賓使



 武翼大夫



 供備庫使



 武功郎



 皇城副使



 武德郎



 宮苑、左右騏驥、內藏庫副使



 武德郎



 左藏庫、東西作坊副使



 武節郎



 莊宅、六宅、文思副使



 武略郎



 內園、洛苑、如京、崇儀副使



 武經郎



 西京左藏庫副使



 武義郎



 西京作坊、東西染院、禮賓副使。



 武郎翼



 供備庫副使



 敦武郎



 內殿承制



 修武郎



 內殿崇班



 從義郎



 東頭供奉官



 秉義郎



 西頭供奉官



 忠訓郎



 左侍禁



 忠翊郎



 右侍禁



 成忠郎



 左班殿直



 保義郎



 右班殿直



 承節郎



 三班奉職



 承信郎



 三班借職



 下班祗應



 殿侍



 元豐官制定,有請並易內侍官名者,神宗曰:「祖宗為此名,有深意,豈可輕議?」政和二年,始遂改焉。凡十有二階。



 新官



 舊官



 供奉官



 內東頭供奉官



 左侍禁



 內西頭供奉官



 右侍禁



 殿頭



 左班殿直



 高品



 右班殿直



 高班



 黃門



 黃門



 祗候侍禁



 祗候殿頭



 祗候殿直



 祗候高品



 祗候黃門



 祗候高班內品



 內品



 祗候內品



 貼祗候內品已上三名仍舊不改。



 政和初,既易武階,遂改醫官之名,凡十有四階。



 新官



 舊官



 和安、成和、成安、成全大夫軍器庫使



 保和大夫



 西綾錦使



 保安大夫



 榷易使



 翰林良醫



 翰林醫官使



 和安、成和、成安、成全郎



 軍器庫副使



 保和郎



 西綾錦副使



 保安郎



 榷易副使



 翰林醫正



 翰林醫官副使



 凡除職事官,以寄祿官品之高下為準:高一品已
 上為行,下一品為守,下二品已下為試,品同者否。紹聖三年,戶部侍郎吳居厚言:「神宗官制,凡臺、省、寺、監之制,有行、守、試三等之別。元祐中,裁減冗費,而職事官帶行者第存虛名而已,請付有司講復舊制。」從之。四年,翰林學士蔣之奇言:「所謂試,則非正官也。今尚書、侍郎皆正官,而謂之試,失之矣。如以其階卑,則謂之守可也。臣請凡為正官者皆改試為守。」崇寧中,吏部授選人差遣,亦用資序高下
 分行、守、試三等。政和三年,詔選人在京職事官,依品序帶行、守、試,其外任則否。宣和以後,官高而仍舊職者謂之領,官卑而職高者謂之視,故有庶官視從官,從官視執政,執政視宰相。凡道官亦視文階雲。



 爵一十二



 王嗣王郡王國公郡公開國公開國郡公開國縣公開國侯
 開國伯開國子開國男



 右封爵,皇子、兄弟封國,謂之親王。親王之子承嫡者為嗣王,宗室近親承襲,特旨者封郡王,遇恩及宗室祖宗後承襲及特旨者封國公。餘宗室近親並封郡公。其開國公、侯、伯、子、男皆隨食邑:二千戶已上封公,一千戶已上封侯,七百戶已上封伯,五百戶已上封子,三百戶已上封男。見任、前任宰相食邑、實封共萬戶。



 嗣王、開國郡公、縣公後不封。



 勛一十二



 上柱國柱國上護軍護軍上輕車都尉輕車都尉上騎都尉



 騎都尉驍騎尉飛騎尉雲騎尉武騎尉



 右騎都尉已上,兩府並武臣正任已上經恩加兩轉,文武朝官加一轉。武騎尉已上,京官加一轉,朝官雖未至驍騎尉,經恩亦便加騎都尉。



 功臣



 推忠佐理協謀同德守正亮節



 翊戴贊治崇仁保連經邦



 右賜中書、樞密臣僚。宰相初加六字,餘官初加四字,其次並加兩字,舊有功臣者改賜。



 推忠保德翊戴守正亮節同德



 佐運崇仁
 協恭贊治宣德純誠



 保節保順忠亮竭誠奉化效順



 順化



 右賜皇子、皇親、文武臣僚、外臣初加四字,次加兩字。



 拱衛翊衛衛聖保順忠勇拱極護聖奉慶果毅肅衛



 右賜諸班直將士禁軍初加二字,再加亦
 如之。



 檢校官一十九



 太師太尉太傳太保司徒



 司空左僕射右僕射吏部尚書兵部尚書戶部尚書刑部尚書禮部尚書工部尚書左散騎常侍



 右散騎常侍太子賓客國子祭酒水部員外郎



 右皇子初授官加太尉,初授樞密使、使相及曾任宰相、樞密使除節度使加太傳,初除宣微、節度加太保。宗室初除使相加尚書左僕射,特除並換授諸司使已上加工部尚書,諸司副使加右散騎常侍。除通事舍人、內殿崇班已上,初授加太子賓客;副率已上並三班及吏職、蕃官軍員,該恩加國子祭酒。四廂都指揮使止於司徒,諸軍都指揮使、忠佐馬步都軍頭止於司空,軍班都虞候、忠佐副都
 軍頭已上止於左、右僕射,諸軍指揮使止於吏部尚書。其官止,遇恩則或加階、爵、功臣。



 兼官四



 御史大夫侍御史殿中侍御史監察御史



 右通事舍人、內殿崇班已上,初除加兼御史大夫。宗室副率已上,初授軍頭等,經恩加兼監察御史,餘經恩以次遷入。



 試秩



 大理司直大理評事秘書省校書郎正字寺、監主簿助教



 右幕職,初授則試秘書省校書郎,再任至兩使推官,則試大理評事。掌書記、支使、防禦、團練判官則試大理司直、評事,又加則兼監察御史。亦有解褐試大理評事、校書郎、正字、寺監主簿、助教者,謂之試銜。有選集,同出身例。



 紹興以後階官



 元豐新制以階易官,定為二十四階。崇寧、大觀、政和相繼潤色之。紹興舉行元祐之法,分置左右:文臣為左,餘人為右。浮熙初,因宗室善俊建言,階官並去「左」「右」字,今任子、雜流,惟紐轉通直郎、奉直、中散二大夫如故,若帶貼職,則超資。自開府至迪功,序次於後。



 文階



 開府儀同三司



 特進



 金紫光祿大夫



 銀青光錄大夫



 光祿大夫



 宣奉大夫大觀新置



 正奉大夫



 正議大夫



 通奉大夫大觀新置。



 通議大夫



 太中大夫以上舊為侍從官



 中大夫



 中奉大夫大觀新置



 中散大夫



 朝議大夫以上系卿、監。



 奉直大夫大觀新置。



 朝請大夫



 朝散大夫



 朝奉大夫以上系正郎。



 朝請郎



 朝散郎



 朝奉郎以上系員外郎。



 承議郎



 奉議郎



 通直郎



 宣教郎



 宣義郎



 承事郎



 承奉郎



 承務郎以上系京官。



 右四年一轉,無出身人逐資轉,有出身人超資轉,
 至奉議並逐資轉,至朝議大夫有止法,仍七年一轉。內奉直、中散二大夫有出身人不轉。



 承直郎



 儒林郎



 文林郎



 從事郎以上崇寧新置。



 從政郎



 修職郎



 迪功郎以上政和更定,並系選人用舉狀及功賞改官。



 通仕郎



 登仕郎



 將仕郎以上系奏補未出身官人。



 武階



 武階舊有橫行正使、橫行副使,有諸司正使、諸司副使,有使臣。政和易以新名,正使為大夫,副使為郎,橫行正、副亦然,於是有郎居大夫之上。至紹興,始厘正其序。



 太尉



 通侍大夫



 正侍大夫



 宣正大夫政和新置。



 履正大夫政和新置。



 協忠大夫政和新置。



 中侍大夫



 中亮大夫



 中衛大夫



 翊衛大夫



 親衛大夫



 拱衛大夫自翊衛至此,並政和新置。



 左武大夫



 右武大夫以上為橫行十三階。



 右並政和新置。內通侍大夫舊為內客省使,國朝未嘗除人,自易武階,不遷通侍沿初意也。轉至中侍,無磨勘,特紼除。



 武功大夫



 武德大夫



 武顯大夫



 武節大夫



 武略大夫



 武經大夫



 武義大夫



 武翼大夫以上系舊諸司正使,八階。



 正侍郎



 宣正郎



 履正郎



 協忠郎



 中侍郎自正侍至此,並政和新置。



 中亮郎



 中衛郎



 翊衛郎



 親衛郎



 拱衛郎自翊衛至此,並政和新置。



 左武郎



 右武郎以上,舊為橫行副使,



 政和更新,增益共十二階。



 右自正侍至右武,舊在右武大夫之下,武功大夫之上,今從紹興厘正書。



 武功郎



 武德郎



 武顯郎



 武節郎



 武略郎



 武經郎



 武義郎



 武翼郎以上舊諸司副使,八階。



 訓武郎



 修武郎以上為大使臣。



 從義郎



 秉節郎



 忠訓郎



 忠翊郎



 成忠郎



 保義郎



 承節郎



 承信郎以上為小使臣。



 右並五年一轉,至武功大夫,有止法。



 進武校尉



 進義校尉



 下班祗應



 進武副尉



 進義副尉



 守闕進義副尉



 進勇副尉



 守闕進勇副尉以上無品,二校尉參



 吏部,下班參兵部,以下並參刑部。



 內侍官十二階,並政和舊制。



 醫官政和既易武階,而醫官亦更定焉,紹興因之,特損其額。舊額和安大夫至良醫二十員,紹興置五員;和安郎至醫官三十員,置四員;醫效十員,置二員;醫痊十員,置一員;醫愈至祗候、大方脈一百五十員,置十五員。



 和安、成和、成安、成全大夫



 保和大夫



 保安大夫



 翰林良醫



 和安、成和、成安、成全郎



 保和郎



 保安郎



 翰林醫正



 翰林醫官



 翰林醫效



 翰林醫痊



 翰林醫愈



 翰林醫證



 翰林醫診



 翰林醫候



 翰林醫學



 右醫正而止,十四階,並政和制,餘續增焉。



\end{pinyinscope}