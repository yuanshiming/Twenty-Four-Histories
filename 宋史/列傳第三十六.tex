\article{列傳第三十六}

\begin{pinyinscope}

 張鑒姚坦索湘宋太初盧之翰鄭文寶王子輿劉綜卞袞許驤裴莊牛冕張適附欒崇吉袁逢吉韓國華何蒙
 慎知禮子從吉



 張鑒,字德明,瀛州團練使藏英之孫。父裔,以蔭補供奉官。鑒本將家,幼能嗜學,入衛州霖落山肄業,凡十餘年。太平興國三年,擢進士第,釋褐大理評事、監泰州紫墟榷務。升朝,為太子右贊善大夫、知婺州,就遷著作郎。還,拜監察御史。奉詔決獄江左,頗雪冤滯。歷殿中侍御史。



 會命曹彬等進討幽州,問群臣以方略,鑒上疏極言不可。論者以鑒燕人,沮議非忠也,太宗置不問。與趙延進
 同掌左藏,延進恃恩逾規,鑒廷奏之。有旨罷延進,以鑒判三司度支、憑由催欠司。時三部各置憑由催欠,鑒請並為一,從之。王明、李惟清薦其能,用為江南轉運使。本部有大姓為民患者,鑒以名聞。太宗盡令部送魁首及妻子赴闕,以三班職名羈縻之,江左震肅。又建議割瑞州清江、吉州新淦、袁州新喻三縣置臨江軍,時以為便。召還,特被慰獎。梓州符昭願驕僭不法,即以鑒代之。遷刑部員外郎、判大理寺,遷屯田郎中、判三司都催欠司,
 改都勾院,擢拜樞密直學士、知通進、銀臺、封駁司,又掌三班。上言供奉官以下不考校殿最,恐無沮勸,即詔鑒兼磨勘職。改三司為左右計,分天下為十道,鑒奏其非便。未幾,果復舊。



 淳化中,盜起西蜀,王繼恩討平之,而御軍無政,其下恃功暴橫。益州張詠密奏,請命近臣分屯師旅,即遣鑒與西京作坊使馮守規偕往。召對後苑門,面授方略。鑒曰:「益部新復,軍旅不和,若聞使命驟至,易其戎伍,慮或猜懼,變生不測。請假臣安撫之名。」太宗稱
 善。鑒至蜀,繼恩猶偃蹇,不意朝廷聞其縱肆。鑒之行,付以空名宣頭及廷臣數人,鑒與詠即遣部戍卒出境,繼恩麾下使臣亦多遣東還,督繼恩輩分路討捕殘寇,而鑒等招輯反側。事平歸朝,未至,拜左諫議大夫、戶部使。



 會五路進兵討西夏,令鑒乘傳往環州,與李繼隆議護送芻糧入靈州。及還,上疏曰:



 關輔之民,數年以來,並有科役,畜產蕩盡,室廬頓空。加以浦洛之行,曾經剽劫。原州之役,又致遷延。非獨令之弗從,實緣力所不逮。況復
 先棄糧草,見今逐處追科,本戶稅租,互遣他州送納,往返千里,費耗十倍,愁苦怨嘆,充塞路岐,自春徂冬,曾無暫息,餱糧乏絕,力用殫窮。顧此疲羸,尤堪軫恤。今若復有差率,益流致亡,縱令驅迫,必恐撓潰。願陛下特垂詔旨,無使重勞,因茲首春,俾務東作。



 況靈州一方,僻居絕塞,雖西陲之舊地,實中夏之蠹區。竭物力以供須,困甲兵而援送,蕭然空壘,祗益外虞。不若以賜繼遷,使懷恩奉籍,稍息飛挽之役。事當深慮,理要預防。若待川決而
 後防,火熾而方戢,則焚溺之患深矣,雖欲拯救,其可得乎?



 尋詔鑒專督軍糧,以軍興法從事,饋運頗集。



 真宗即位,遷給事中、使如故。咸平初,改工部侍郎、出知廣州。居二年,民條其政績上請刻石。三年,移知朗州。溪洞群蠻數寇擾,鑒召酋豪,諭以威信,皆俯伏聽命。



 初,鑒在南海,李夷庚為通判,謝德權為巡檢,皆與之不協。二人密言鑒以貲付海賈,往來貿市,故徙小郡。至是,鑒自陳有親故謫瓊州,每以奉米附商舶寄贍之,又言夷庚、德權憸
 人貪兇之狀,上意稍釋。召還,以疾徙知相州。有芝草生於監牧之室,鑒表其祥異,以為河朔弭兵款附之兆。優詔答之。景德初,卒,年五十八。子士廉為殿中丞,士宗太子洗馬,士程屯田員外郎。



 姚坦,字明白,曹州濟陰人。開寶中,以《尚書》擢第,調補將陵尉。歷隰州推官、將作監丞、知潯州。太平興國三年召還,為著作佐郎、通判唐州。



 八年,諸王出閣,詔給、諫以上,於朝班中舉年五十以上、通經有文行者,以備宮僚,乃
 以戶部員外郎王適、監察御史趙齊為衛王府諮議,左贊善大夫戴玄為本府翊善。水部員外郎趙令圖為廣平郡王府諮議,國子博士閻象為本府翊善。又以起居舍人楊可法、國子博士楊幼英、左贊善大夫杜新及坦並為皇子翊善,國子博士邢昺為諸王府侍講,坦仍賜緋魚。太宗召適等謂曰:「諸子生長深宮,未知世務,必資良士贊導,使日聞忠孝之道。汝等皆朕所慎簡,各宜勉之。」坦歷殿中丞、倉部員外郎,賜金紫。遷本曹郎中,轉考
 功,仍為益王府翊善。



 坦性木強固滯。王嘗於邸中為假山,費數百萬,既成,召賓僚樂飲,置酒共觀之。坦獨俯首,王強使視之,曰:「但見血山耳,安得假山!」王驚問故,坦曰:「在田舍時,見州縣催科,捕人父子兄弟,送縣鞭笞,流血被體。此假山皆民租稅所為,非血山而何?」是時太宗亦為假山,聞而毀之。



 王少佚豫,坦即醜詆,王頗鄙其為人。自是坦每暴揚其事,上嘗誡之曰:「元傑知書好學,亦足為賢王矣。少不中節,亦須婉辭規諷,況無大故而詆訐
 之,豈裨贊之道邪?」頃之,左右乃教王詐稱疾不朝。太宗日使視疾,逾月不瘳,甚憂之,召王乳母問狀,乳母曰:「王本無疾,徒以姚坦檢束,居常不得自便,王不樂,故成疾。」上怒曰:「吾選端士,輔王為善。王不能用其諫,而又詐疾,欲使朕去正人以自便,何可得也。且王年少,必爾輩為之謀耳。」因命捽至後苑,杖之數十。召坦慰諭曰:「卿居王宮,能以正為群小所疾,大為不易。卿但如是,勿慮讒間,朕必不聽。」王薨,改衛尉少卿,判吏部南曹。他日因事得
 對,上以其舊人,召升殿與語。坦言及故府,意短諸王而稱己之敢言。坦退,上謂侍臣曰:「坦在宮邸,不能以正理誨諭,事有微失,即從而揚之,此賣直取名耳。」



 景德初,求補郡,俾知鄧州。轉運使表其治狀,詔嘉獎之。大中祥符初,復知光州。二年,卒,年七十五。



 索湘,字巨川,滄州鹽山人。開寶六年進士,釋褐鄆州司理參軍。齊州有大獄,連逮者千五百人,有司不能決。湘受詔推鞫,事隨以白。太平興國四年,轉運使和峴薦其
 能,遷太僕寺丞,充度支巡官。改太子右贊善大夫,轉殿中丞,充推官,拜監察御史。九年,河決,壞民田,命與戶部推官元□同按行。會詔下東封,與劉蟠同知泰山路轉運事,又為河北轉運副使。湘經度供饋,以能幹聞。事集,加屯田員外郎。



 明年,契丹入寇,王師衄於君子館,敵兵乘勝據中渡橋,塞土門,將趨鎮州。諸將計議未定,湘為田重進畫謀,結大陣東行,聲言會高陽關兵,敵以為然,即擁眾邀我於平虜城。夜二鼓,率兵而南,徑入鎮陽,據
 唐河,乘其無備破砦柵。及敵兵覺,悉遁走。雍熙中,召為鹽鐵判官,改駕部員外郎。端拱二年,河北治方田,命副樊知古為招置營田使。會議罷,復為河北轉運使。轉虞部郎中,選為將作少監。居無何,有訟其擅易庫縑以自用者,坐授膳部員外郎、知相州。時有群盜聚西山下,謀斷澶州河橋,入攻磁、相州,援旗伐鼓,白晝抄劫。鄰郡發兵千人捕逐,無敢近。湘擇州軍得精銳三百人,偵其入境,即掩擊而盡擒之。轉運使王嗣宗以狀聞,詔復舊官,
 命為河東轉運使。湘以忻州推官石宗道、憲州錄事胡則為干職,命以自隨,所至州郡,勾檢其簿領焉。二人後皆歷清要。明年,王超等率師趨烏白池,抵無定河。水源涸絕,軍士渴乏。時湘已輦大鍬千枚至,令鑿井,眾賴以濟。



 真宗即位,入為右諫議大夫。復充河北轉運使,屬郡民有幹釀,歲輸課甚微,而不逞輩因之為奸盜。湘奏廢之。德州舊賦民馬以給驛,又役民為步遞,湘代以官馬兵卒,人皆便之。會內殿崇班閻日新建議,請於靜戎、威
 虜兩軍置場鬻茶,收其利以資軍用。湘言非便,遂止。又言事者請許榷場商旅以茶藥等物販易於北界,北界商旅許於雄、霸州市易,資其懋遷,庶息邊患。詔湘詳議以聞,乃上言曰:「北邊自興置榷場,商旅輻湊,制置深得其宜。今若許其交相販易,則沿邊商人深入戎界,竊為非便。又北界商人若至雄、霸,其中或雜奸偽,何由辨明?況邊民易動難安,蕃戎之情宜為羈制。望且仍舊為便。」會有詔規度復修定州新樂、蒲陰兩縣,湘以其地迫窄,
 非屯兵之所,遂奏罷之。



 湘少文而長於吏事,歷邊部,所至必廣儲畜為備豫計,出入軍旅間,頗著能名。先是,邊州置榷場,與蕃夷互市,而自京輦物貨以充之,其中茶茗最為煩擾,復道遠多損敗。湘建議請許商賈緣江載茶詣邊郡入中,既免道途之耗,復有徵算之益。又威虜、靜戎軍歲燒緣邊草地以虞南牧,言事者又請於北砦山麓中興置銀冶,湘以為召寇,亦奏罷之。



 咸平二年,入為戶部使。受詔詳定三司編敕,坐與王扶交相請托,擅
 易板籍,責授將作少監。三年,出知許州,徙荊南,復為右諫議大夫、知廣州。四年,卒,詔遣其子希顏護喪傳置歸鄉里。



 宋太初,字永初,澤州晉城人。太平興國三年舉進士,解褐大理評事、通判戎州,以善政聞。有詔褒美,遷將作監丞、贊善大夫、通判晉州,轉太常丞。雍熙三年,通判成都府,賜緋魚。會詔求直言,著《守成箴》以獻。淳化初,遷監察御史。時北面用兵,選為雄州通判。入判度支勾院。二年,
 為京西轉運副使。未幾,移河東。四年,遷正使。改殿中侍御史。



 至道初,遷兵部員外郎,充鹽鐵副使,賜金紫。時陳恕為使,太初有所規畫必咨恕,未嘗自用為功,恕甚德之。會西鄙有警,轉饋艱急,改刑部郎中、充陜西轉運使。二年,命白守榮、馬紹忠護芻糧,分三番抵靈州。轉運副使盧之翰違旨並往,為戎人所剽。上怒,捕太初及副使秘書丞竇玭系獄。太初責懷州團練副使,之翰、玭悉除名,之翰貶許州司馬,玭商州司戶掾。明年,起太初為祠部
 郎中,知梓州。俄復舊秩。



 真宗嗣位,召還,復命經度陜西饋運事。咸平初,拜右諫議大夫、知江陵府。蠻寇擾動,太初以便宜制遏,詔獎之。三年,再知梓州。明年,益州雷有終以母老求還,詔太初就代。時分川峽為四路,各置轉運使。上以事有緩急,難於均濟,命太初為四路都轉運使,要切之務,俾同規畫。太初與鈐轄楊懷忠頗不協,時蜀土始安,上慮其臨事矛盾,亟召太初還。會御史中丞趙昌言等坐事被劾,命權御史中丞。先是,按劾有罪必
 豫請朝旨,太初以為失風憲體,獄成然後聞上,時論韙之。俄出知杭州。太初有宿疾,以浙右卑濕不便,求近地,得廬州。疾久,頗昏忘,不能治大郡,連徙汝、光二州。景德四年卒,年六十二。錄其弟繼讓試校書郎。



 太初性周慎,所至有干職譽。嘗著《簡譚》三十八篇,自序略曰:「廣平生纂文史老釋之學,嘗謂《禮》之中庸,伯陽之自然,釋氏之無為,其歸一也。喜以古聖道契當世之事,而患未博也,忽外物觸於耳目,內機發於性情,因筆而簡之,以備闕
 忘耳。」子傳慶,後為太子中舍。



 盧之翰字維周,祁州人。曾祖玄暉,鴻臚卿。祖知誨,天雄軍掌書記。父宏,蔡州防禦判官。之翰少篤學,家貧,客游單州,防禦使劉乙館於門下。乙徙錢塘,之翰隨寓其郡。太平興國四年舉進士,不得解,詣登聞自陳,詔聽附京兆府解試。明年登第,解褐大理評事、知臨安縣,三遷殿中丞,通判洺州。



 會契丹入寇,之翰募城中丁壯,決漳、御河以固城壁,虜不能攻。吏民詣闕求借留。召還,遷太常
 博士,為河東轉運副使,徙京西轉運副使,改工部員外郎。建議導潩河合於淮,達許州,以便漕運。以勞加戶部員外郎。又改陜西轉運使,遷吏部員外郎。至道初,李順亂蜀,命兼西川安撫轉運使。賊平,還任。



 之翰嘗薦李憲為大理丞,憲坐贓抵死,之翰當削三任。時副使鄭文寶議城清遠軍,又禁蕃商貨鹽,之翰心知其非便,以文寶方任事,不敢異其議。及文寶得罪,之翰並前愆,左授國子博士,領使如故。尋復舊職。會調發芻糧輸靈州,詔分
 三道護送,命洛苑使白守榮、馬紹忠領其事。之翰違旨擅並為一,為李繼遷邀擊於浦洛河,大失輜重。詔國子博士王用和乘傳逮捕,系獄鞫問。之翰坐除名,貶許州司馬。明年,起為工部員外郎、同勾當陜西轉運使。真宗即位,復吏部員外郎,充轉運使。以久次,召拜禮部郎中,賜金紫,復遣之任。



 咸平元年,以疾命國子博士張志言代還。未幾,復出為京西轉運使。先是,朝廷議城故原州,以張守備,之翰沮罷之,其後西鄙不寧,修葺為鎮戎軍。
 之翰坐橫議非便,黜知歸州,便道之官,限五日即發。三年,授廣南西路轉運使。會廣州索湘卒,就改太常少卿、知州事。之翰無廉稱,又與轉運使凌策不協,陰發其事。五年,徙知永州,未行,卒,年五十七。



 鄭文寶,字仲賢,右千牛衛大將軍彥華之子。彥華初事李煜,文寶以蔭授奉禮郎,掌煜子清源公仲寓書籍,遷校書郎。入宋,煜以環衛奉朝請,文寶欲一見,慮衛者難之,乃被蓑荷笠,以漁者見,陳聖主寬宥之意,宜謹節奉
 上,勿為他慮。煜忠之。後補廣文館生,深為李昉所知。



 太平興國八年登進士第,除修武主簿。遷大理評事、知梓州錄事參軍事。州將表薦,轉光祿寺丞。留一歲,代歸。獻所著文,召試翰林,改著作佐郎、通判穎州。丁外艱,起知州事。召拜殿中丞,使川、陜均稅。次渝、涪,聞夔州廣武卒謀亂,乃乘舸泛江,一夕數百里,以計平之。授陜西轉運副使,許便宜從事。會歲歉,誘豪民出粟三萬斛,活饑民八萬六千口。既而李順亂西蜀,秦隴賊趙包聚徒數千,
 將趨劍閣以附之。文寶移書蜀郡,分兵討襲,獲其渠魁,餘黨殲焉。



 文寶前後自環慶部糧越旱海入靈武者十二次,曉達蕃情,習其語。經由部落,每宿酋長帳中,其人或呼為父。遷太常博士。內侍方保吉出使陜右,頗恣橫,且言文寶與陳堯叟交游,為薦其弟堯佐。驛召令辨對,途中上書自明。太宗察其事,坐保吉罪,厚賜文寶而遣之,俄又召至闕下,文寶奏對辯捷,上深眷遇。俄加工部員外郎。時龍猛卒戍環慶,七年不得代,思歸,謀亂。文寶
 矯詔以庫金給將士,且自劾,請代償。詔蠲其所費。



 先是,諸羌部落樹藝殊少,但用池鹽與邊民交易穀麥,會饋挽趨靈州,為繼遷所鈔。文寶建議以為「銀、夏之北,千里不毛,但以販青白鹽為命爾。請禁之,許商人販安邑、解縣兩池鹽於陜西以濟民食。官獲其利,而戎益困,繼遷可不戰而屈」。乃詔自陜以西有敢私市者,皆抵死,募告者差定其罪。行之數月,犯者益眾。戎人乏食,相率寇邊,屠小康堡。內屬萬餘帳亦叛。商人販兩池鹽少利,多取
 他徑出唐、鄧、襄、汝間邀善價,吏不能禁。關、隴民無鹽以食,境上騷擾。上知其事,遣知制誥錢若水馳傳視之,悉除其禁,召諸族撫諭之,乃定。



 朝廷議城古威州,遣內侍馮從順訪於文寶,文寶言:



 威州在清遠軍西北八十里,樂山之西。唐大中時,靈武朱叔明收長樂州,邠寧張君緒收六關,即其地也。故壘未圯,水甘土沃,有良木薪秸之利。約葫蘆、臨洮二河,壓明沙、蕭關兩戍,東控五原,北固峽口,足以襟帶西涼,咽喉靈武,城之便。



 然環州至伯
 魚,伯魚抵青岡,青岡拒清遠皆兩舍,而清遠當群山之口,阨塞門之要,芻車野宿,行旅頓絕。威州隔城東隅,豎石盤互,不可浚池。城中舊乏井脈,又飛烏泉去城尚千餘步,一旦緣邊警急,賊引平夏勝兵三千,據清遠之沖,乘高守險,數百人守環州甜水穀、獨家原,傳箭野貍十族,脅從山中熟戶,黨項孰敢不從,又分千騎守磧北清遠軍之口,即自環至靈七百里之地,非國家所有,豈威州可御哉?請先建伯魚、青岡、清遠三城,為頓師歸重之
 地。



 古人有言:「金城湯池,非粟不能守。」俟二年間,秦民息肩,臣請建營田積粟實邊之策,修五原故城,專三池鹽利,以金帛啖黨項酋豪子弟,使為朝廷用。不唯安朔方,制豎子,至於經營安西,綏復河湟,此其漸也。



 詔從其議。



 文寶至賀蘭山下,見唐室營田舊制,建議興復,可得粳稻萬餘斛,減歲運之費。清遠據積石嶺,在旱海中,去靈、環皆三四百里,素無水泉。文寶發民負水數百里外,留屯數千人,又募民以榆槐雜樹及貓狗鴉烏至者,厚給
 其直。地舄鹵,樹皆立枯。西民甚苦其役,而城之不能守,卒為山水所壞。又令寧、慶州為水磑,亦為山水漂去。



 繼遷酋長有嵬囉嵬悉俄者,文寶以金帛誘之,與手書要約,留其長子為質,令陰圖繼遷,即遣去。謂之曰:「事成,朝廷授汝以刺史。」文寶又預漆木為函,以備馳獻繼遷之首。又發民曳石碑石詣清遠軍,將圖紀功。而嵬囉等盡以事告繼遷,繼遷上表請罪。上怒文寶,猶含容之。既而文寶復請禁鹽,邊民冒法抵罪者甚眾。太常博士席羲
 叟決獄陜西,廉知其事,以語中丞李昌齡,昌齡以聞。文寶又奏減解州鹽價,未滿歲,虧課二十萬貫,復為三司所發。乃命鹽鐵副使宋太初為都轉運使,代文寶還,下御史臺鞫問,具伏。下詔切責,貶藍山令。未幾,移枝江令。



 真宗即位,徙京山。咸平中召還,授殿中丞,掌京南榷貨。時慶州發兵護芻糧詣靈州,文寶素知山川險易,上言必為繼遷所敗。未幾,果如其奏。轉運使陳緯沒於賊,繼遷進陷清遠軍。時文寶丁內艱,服未闋,即命相府召詢
 其策略。文寶因獻《河西隴右圖》,敘其地利本末,且言靈州不可棄。時方遣大將王超援靈武,即復文寶工部員外郎,為隨軍轉運使。至環州,或言靈州已陷,文寶乃易其服,引單騎,冒大雪,間道抵清遠故城,盡得其實,遂奏班師,就除本路轉運使,上疏請再葺清遠軍。都部署王漢忠言其好生事,遂徒河東轉運使。嘗上言管內廣銳兵萬餘,難得資糧,請徙置近南諸州,又欲令強壯戶市馬,備征役。宰相李沆等以為廣銳州兵,皆本州守城,置營
 必慮安土重遷,徙之即致紛擾。又強壯散處鄉落,無所拘轄,勒其市馬,亦恐非便。上復令文寶條對,文寶固執前議,且言土人久留,恐或生事。上曰:「前令團並軍伍,改置營壁,欲其互移本貫,行之已久。」而文寶確陳其利,因命錢若水詳度以聞。若水所對與沆等同,遂罷之。



 先是,麟、府屯重兵,皆河東輸饋,雖地裡甚邇,而限河津之阻。土人利於河東民罕至,則芻粟增價。上嘗訪使邊者,言河裁闊數十步,乃詔文寶於府州、定羌軍經度置浮橋,
 人以為便。會繼遷圍麟州,令乘傳晨夜赴之,圍解,遷刑部員外郎,賜金紫。頃之,寇準薦其熟西事,可備驅策,因復任陜西轉運使。嘗出手札,密戒令邊事與僚屬計議,勿得過有須索,重擾於下。後有言其張皇者,詔徙京西,以朱臺符代之。



 景德元年冬,契丹犯邊,又徙河東。文寶安輯所部,募鄉兵,張邊備,又領蕃漢兵赴河北,手詔褒諭。未幾,復蒞京西。契丹請和,文寶陳經久之策,上嘉之。三年,召還,未至,遇疾,表求藩郡散秩。詔聽不除其籍,續
 奉養疾,以其子鄆州推官於陵為大理寺丞、知襄城縣,以便其養。大中祥符初,改兵部員外郎。車駕祀汾陰還,文寶至鄭州請見。上以其久疾,除忠武軍行軍司馬。文寶不就,以前官歸襄城別墅。六年,卒,年六十一。



 文寶好談方略,以功名為己任。久在西邊,參預兵計,心有餘而識不足,又不護細行,所延薦屬吏至多,而未嘗擇也。晚年病廢,從子為邑,多撓縣政。能為詩,善篆書,工鼓琴。有集二十卷,又撰《談苑》二十卷、《江表志》三卷。



 王子輿字希孟,密州莒人。曾祖甲,以義勇為鄉人所推。唐末,淄、青、徐、兗皆南結吳人以拒梁,梁得三鎮,吳人北侵益急,沂、密尤被其害。州民聚為八砦以捍寇,遂署甲為八砦都指揮使。祖徽,襲父職,晉末,賊帥趙重進掠高密,徽戰沒。父璉,復嗣其事。周世宗平淮南,始去兵即農,厚自封殖。



 子輿少業文詞,太平興國八年舉進士,解褐北海主簿。歷大理評事,知臨海縣,改光祿寺丞。使西蜀決獄還,知興國軍。淳化中,雷有終為江、浙、荊湖茶鹽制
 置使,奏子輿為判官。轉太子中允,改著作郎,江、淮、兩浙制置茶鹽,就轉太常博士。真宗即位,遷殿中侍御史。因入對,與三司論列利害,以子輿為長。轉度支員外郎。子輿以每事上計司,移報稽滯,求兼省職,乃命為鹽鐵判官,仍領制置,增歲課五十餘萬貫。咸平三年,就命兼充淮南轉運使。



 子輿精於吏事,久掌茶鹽漕運,周知利害,裁量經制,公私便之。所至郡縣,以公事申請者,文牒紛委,頃刻待報,子輿皆即決遣,曾無凝滯。明年,表求代,詔
 許自擇。子輿以卞袞、劉師道名聞,即命袞與師道為轉運使。召子輿,拜右諫議大夫、戶部使。五年二月,方奏事便殿,俄疾作僕地,命中使掖之以出,至第卒。以子道宗方幼,命三司判官朱臺符檢校其家。子輿止一子,而三女皆幼。道宗尋卒,家寓楚州。子輿妻劉還父母家,子輿旅櫬在京師,景德中,官借船移柩,還葬其里,鬻京師居第,以錢寄楚州官庫,以備三女資送。從其從弟之請也。



 劉綜,字居正,河中虞鄉人。少依外兄通遠軍使董遵誨,
 遵誨嘗遣貢馬。太祖嘉其敏辯,將授三班之職。綜自陳素習詞業,願應科舉。及還,上解真珠盤龍衣以賜遵誨,綜辭曰:「遵誨人臣,安敢當此賜!」上曰:「吾委遵誨以方面,不以此為疑也。」



 雍熙二年舉進士第,解褐邛州軍事推官。就改永康軍判官,遷大理評事、通判眉州,轉太僕寺丞。代還,對便殿,因言:「蜀地富庶,安寧已久,益州長吏,望慎擇其人。」上嘉之,改太子中允。未幾,李順果為亂,復召見,面賜緋魚。尋為三門發運司水陸轉運使,通判大名
 府。連丁家難,起知建安軍。



 先是,天長軍及揚州六合縣民輸賦非便,綜奏請降天長軍為縣,隸揚州,以六合縣隸建安軍,自是民力均濟。時淮南轉運使王嗣宗兼發運事,規畫多迂滯。綜因上言請復置都大發運司,專干其職。至道二年,遷太常丞,職事修舉,多稱薦之者。



 咸平初,命代王欽若判三司都理欠憑由司,出為河北轉運副使。嘗言:「州縣幕職官,以昏耄放罷者,其間有實廉謹之士,或幼累無托,或居止無定,全藉祿廩以濟朝夕,一
 旦停罷,則饑寒無依,以傷和氣。望自今並除致仕官。」又言:「法官斷獄,皆引律令之文,以定輕重之罪,及其奏御,復云慮未得中,別取進止,殊非一成不變之道,且復煩於聖斷。望自今降旨約束,不得復然。」時河北承兵寇之後,民戶凋弊,吏部所銓幕職州縣官皆四方之人,不習風俗,且有懷土之思,以是政事多因循不舉。綜建議請自今並以河朔人充之,冀其安居,勤於職事。



 夏人擾西邊,環慶大屯士馬,詔徙綜為陜西轉運副使,轉太常博
 士。時梁鼎議禁解鹽,官自貨鬻,乃命綜與杜承睿制置青白鹽事。綜條上利害,力言非便,卒罷其事。時靈州孤危,獻言者或請棄之,綜上言曰:「國家財力雄富,士卒精銳,而未能剪除兇孽者,誠以賞罰未行,而所任非其材故也。今或輕從群議,欲棄靈川,是中賊之奸計矣。且靈州民淳土沃,為西陲巨屏,所宜固守,以為捍蔽。然後於浦洛河建軍城,屯兵積糧為之應援,此暫勞永逸之勢也。況鎮戎軍與靈州相接,今若棄之,則原、渭等州益須
 設備,較其勞費十倍而多,則利害之理昭然可驗矣。」俄充轉運使。



 四年,又獻議於鎮戎軍置屯田務,又錄唐《安國鎮制置城壕鎮戎古記》石本以進,詔從其請。俄詣闕,奏事稱旨,賜金紫、緡錢五十萬,復遣蒞職。又嘗言:「天下州郡長吏,審官皆據資例而授,未為得人。自今西川、荊湖、江、浙、福建、廣南知州,或地居津要,或戶口繁庶之處,望親加選任。其執政舊臣及給、舍以上知州處,亦擇官通判。又京朝官當任遠官者,率以父母未葬為辭,意求
 規免。請自今父母委未葬者,許請告營辦。審官投狀,並明言父母已葬,方許依例考課,違者並罷其官。」從之。



 五年,拜工部員外郎兼侍御史知雜事。六年,遷起居舍人,再為河北轉運使。時兩河用兵,邊事煩急,轉漕之任,尤所倚辦。綜繼領其職,號為詳練。至是眷矚甚厚,警急之際,輒資其奏處。契丹請和,乃遣近臣諭以擢用之意。景德三年,召拜戶部員外郎、樞密直學士、勾當三班院。綜言:「御史員數至少,每奉朝請,劾制獄,多以他官承之,甚
 紊彞制。望詔兩制以上各舉材堪御史者充,三院共置十員。若出使按獄,所經州郡,官吏能否,生民利病,刑獄枉濫,悉得察舉。」四年,西幸,道出河陽境上,時節度王顯被疾還京,以綜權知孟州事。未幾召還,復出知並州,以政績聞。州民乞留,優詔嘉獎。歸朝,知審官院,改吏、禮二部郎中,充職,兼知通進、銀臺、封駁司。



 大中祥符四年,館伴契丹使,因作《大雪歌》以獻。即命同知貢舉,以李宗諤代為館伴使。俄權知開封府。綜以貴要交結富民,為之
 請求,或托為親屬,奏授試秩,緣此謁見官司,頗紊公政,因建議請加抑止。又文武官居遠任,而家屬寓京師,其子孫弟侄無賴者,望嚴行約束,並其交游輩劾罪,從之。七年,以老疾求典河中,真宗以太寧宮廟長吏奉祠,綜艱於拜起,慮不克恭事,命知廬州。明年,罷學士,授右諫議大夫。八年卒,年六十一。



 綜強敏有吏材,所至抑挫豪右,振舉文法,時稱乾治。然尚氣好勝,不為物論所許。子建中、正中,並贊善大夫。弟綽,淳化三年進士,官刑部郎
 中。



 卞袞,字垂象,益州成都人。父震,工為詩。舉蜀進士,渝州刺史南光海闢為判官。蜀平,仍舊職。會賊杜承褒率眾圍城,援兵不至,震躬率士卒,且戰且拒,為流矢所中,創甚,不能臨軍。而州兵重傷,卷甲宵遁,刺史陳文襲不能遏賊,遂入據郡城,以偽官厚賄誘震,震皆斬其使。賊有東章者,本州兵校也。因遣人述朝廷威德,諭以禍福,章懼且信,因伏兵擊其黨類。承褒之眾素不為備,實時大
 潰,震與文襲分部餘卒夾攻之,賊眾遂平。文襲坐陷失州城,削籍為民。震以前功得贖,以虢州錄事參軍卒。



 太平興國八年,袞登進士第,累遷大理評事、知將樂縣,改光祿寺丞、通判泗州。遷著作佐郎、廣南轉運司承受公事,俄通判宣州。淳化中,上命採庶僚中廉幹者,給御書印紙,俾書課最,仍賜實奉以旌異之,袞預焉。改太常丞。咸平初,遷監察御史,為淮南轉運副使、同荊湖發運使,以干職聞,就加殿中侍御史。入判三司開拆司,再為淮
 南轉運使兼發運使。咸平六年,並三司使之職而分置副貳,以袞為刑部員外郎,充鹽鐵副使。景德初,疽發於背卒,年四十五。錄其弟扆為臨穎主簿,子咸為將作監主簿。



 袞明敏有吏乾,累掌財賦,清心治局,號為稱職。然性慘毒,掊克嚴峻,專事捶楚,至有「大蟲」之號。真宗嘗謂近臣曰:「袞公忠盡瘁,無所畏避,人罕能及,然頃在外任,頗傷殘酷,所至州縣,纖微之過,無所容貸。大凡督察部下,糾逖愆違,非有大故,所宜矜恕,官吏自當畏威懷恩,
 不敢貳過,公家之事亦無不濟。乃知為吏之方,適中為善也。」



 許驤,字允升,世家薊州。祖信,父唐,世以財雄邊郡。後唐之季,唐知契丹將擾邊,白其父曰:「今國政廢弛,狄人必乘釁而動,則朔、易之地,民罹其災。茍不即去,且為所虜矣。」信以資產富殖,不樂他徙,唐遂潛繼百金而南。未幾,晉祖革命,果以燕、薊賂契丹,唐歸路遂絕。嘗擁商貲於汴、洛間,見進士綴行而出,竊嘆曰:「生子當令如此!」因不
 復行商,卜居睢陽,娶李氏女,生驤,風骨秀異。唐曰:「成吾志矣!」



 郡人戚同文以經術聚徒,唐攜驤詣之,且曰:「唐頃者不辭父母,死有餘恨,今拜先生,即吾父矣。又自念不學,思教子以興宗緒,此子雖幼,願先生成之。」驤十三,能屬文,善詞賦。唐不識字,而罄家產為驤交當時秀彥。



 驤太平興國初詣貢部,與呂蒙正齊名,太宗尹京,頗知之。及廷試,擢甲科,解褐將作監丞、通判益州,賜錢二十萬。遷右贊善大夫。五年,轉右拾遺、直史館,改右補闕。六年,
 出為陜府西北路轉運副使。會罷副使,徙知鄜州。召還,為比部員外郎。歷知宣、升二州。雍熙二年,改江南轉運副使。洪、吉上供運船水損物,主吏懼罪,故覆舟,鞫獄者按以欺盜,當流死者數百人。驤馳往訊問,得其情實以聞,多獲輕典,優詔褒之。又上言:「劫盜配流,遇赦得原,還本鄉,讎告捕者,多所殺害,自今請以隸軍。」詔可。遷正使。端拱初,拜主客郎中,俄徙知福州。累表求還,不俟報,入朝,召對便殿,延問良久。改兵部郎中,領西川轉運使,以
 久處外任為辭,擢授右諫議大夫,就命知益州。召歸,上言:「蜀民浮窳易搖,宜擇忠厚者撫之,為預備。」既而李順叛,眾頗服其先見。命知審官院,遷御史中丞,以疾固讓,不許。占謝日,命坐勞問,出良藥賜之曰:「此朕所服得驗者。」後驤以久病不能振職。真宗即位,改兵部侍郎。屢求小郡養疾,因入朝失儀,為御史所糾,特詔不問,命知單州。咸平二年卒,年五十七。贈工部尚書。賜其子宗壽出身。



 驤雖無他才略,而人以儒厚長者稱之。宗壽後為殿
 中丞。



 裴莊字端己,閬州閬中人。曾祖琛,後唐昭州刺史。祖遠,河東觀察支使。父全福,鄮縣令。莊在蜀,以明經登第。歸宋,歷虹縣尉、高陵主簿,本府召權司理掾。轉運使雷德驤以威望自任,嘗巡按至境,官屬皆出迎候。莊獨視事本局,徐謁道周,德驤稱其有守。徙權忻州錄事參軍。先是,並州彳侍積軍儲,條制甚峻,掌出納者常十餘人,莊代之,獨任其事。擢授絳州防禦推官,提點並、嵐二州緡帛
 芻糧,改遼州判官,仍蒞舊局



 雍熙三年,命將巡邊,以莊掌隨軍糧料。內客省使楊守一稱薦之,授大理寺丞。時遷雲、朔降戶於汝、洛,遣莊安輯之。俄通判忻州,未上道,會魏咸信出鎮澶州,改命通判。未逾年,咸信表其能,遷太子中允。端拱初,潘美鎮真定,又闢為通判。時契丹掠趙、深,邊將無功,莊上書以為「周世宗誅樊愛能、何徽二將,遂取淮南,克巴蜀。願陛下申明紀律,無使玩寇。」又言:「緣邊砦柵戍兵既寡,戎人易以襲取,咸請廢罷,以益州
 兵。」會詔建方田,莊復上言:「大役兵師,慮生事於邊鄙。」上善之。



 淳化三年,召訪以邊事,稱旨,面賜緋魚,令授清資官。翌日,拜監察御史、荊湖南路轉運使。未行,改三司鹽鐵判官。上疏請給兩省官諫紙,又引故事,禁屠月勿報重刑。會劉式建議請廢緣江榷務,莊力言其非便。出為荊湖北路轉運使。五年,李順亂蜀,命與雷有終並兼峽路隨軍轉運、同知兵馬事。或言莊本蜀人,不宜此任,上益倚信之,許以便宜。事平,轉殿中侍御史,歷工部、司封二
 員外郎,特召問討賊方略。



 至道二年,遣將五路出討李繼遷,莊陰料師出無功,因請加恩繼遷,俟其倔強拒命,則按甲塞外,俘擒未晚。既而諸將果敗績。俄遷祠部郎中。真宗即位,遷度支,充河東轉運使。上章言:「慶、邠、延州、通遠軍,咸處邊要,請武幹如姚內斌、董遵誨者任之。」又言:「田紹斌嘗被疑,韓崇業本秦王婿,程德玄始事晉邸,初甚親近,後疏遠外遷,皆懷怨望,不宜委以戎寄。」未幾,移知蘇州。



 咸平二年,命巡撫江南。使還,言池州、興國軍
 得良吏,餘無足稱者。且言:「朝廷所命知州、通判,率以資考而授,至有因循偷安,無政術而繼得親民者。其素蘊公器有政績者,偶緣公坐,則黜司冗務,真偽莫辨,僥幸滋甚。自今望慎選其人,勿以資格補授,有政績者加以恩禮。」



 是年秋,契丹犯塞,命為河北轉運使。時傅潛統大軍駐定州北,莊屢條奏其無謀略,慮或失幾。會王顯掌樞密,顯與潛俱起攀附,頗庇之。莊奏至,多不報。徙知越州。俄傅潛得罪,莊因上言:「顯、潛皆非材,致誤邊事,請行
 嚴誅,以肅群議。」未幾,徙知宣州。會詔百闢上封直言,莊條列四事:一曰去暴征,二曰省煩刑,三曰擇吏職,四曰敦稼政。疏奏,詔令開陳其所宜行先後,莊對甚悉。改司封郎中。景德初,命安撫兩浙,奏能吏二十人,慢官者五人,多所升黜。又知潞、邢二州。



 大中祥符初東封,改鴻臚少卿,入判登聞鼓院。祀汾陰,遷太僕少卿,為北岳加號冊禮副使,撰《北行記》三卷以獻。六年,出知襄州。明年,車駕幸南京,莊以逮事太宗恩例,授太府卿,權判西京留司
 御史臺。天禧二年,入判刑部,以疾分司西京。郊祀,改光祿卿,求歸上都,以便醫藥。卒,年八十一。錄其孫慶孫,試將作監主簿。



 莊有吏乾,頗無清操,慷慨敢言,太宗獎其忠讜,多所聽納。好為規畫,然寡學術。嘗建議請置廣聽院西垣學士,聞者嗤之。晚年退居,制棺櫝以自隨。喜接賓客,終日無倦。子奐,咸平三年進士,屯田郎中;稷,左班殿直、閣門祗候。



 牛冕,字君儀,徐州彭城人。太平興國三年進士,解褐將
 作監丞、通判郴州,徙和州。加左贊善大夫,遷太常丞、知滁州,以勤政聞。召歸,轉監察御史。



 端拱元年,召試文章,遷左正言、直史館。出知潤州,徙泉州,未至,就命為福建轉運使,加左司諫。建議廢邵武軍歸化金坑,土人便之。至道初,召入,進秩兵部員外郎,知潭州。至郡才數日,復召拜兼侍御史知雜事。



 真宗在東宮,冕嘗奉使賜生辰禮幣,即位尚記其名,改工部郎中。永熙陵復土,會闕中丞,命為儀仗使。時三司各設官局,多不均濟。冕請合為
 一使,分設其貳,則事務不煩而辦,其後卒用冕議。



 咸平元年,選知益州,仍拜右諫議大夫。兩川自李順平後,民罹困苦,未安其業,朝廷緩於矜恤,故戍卒乘符昭壽之虐,嘯集為亂。冕與轉運使張適委城奔漢州,詔遣赴闕,至京兆,劾其罪,並削籍,冕流儋州,適為連州參軍。冕遇赦,移欽、英二州,歷鄂、海二州別駕、淮南節度副使。



 大中祥符初,真宗語宰相曰:「冕素純善,黜棄久矣,量宜甄敘。」即起知漣水軍,俄復為祠部員外郎,卒,年六十四。子昭
 儉,至殿中丞。



 張適者,太平興國五年進士。任藩郡,有治績,以廉敏稱。為水部員外郎、知鄜州。獲對,太宗喜其詞氣俊邁,賜緋魚。旋改京東轉運副使,加直集賢院,一日三被寵渥,時人榮之。徙西川轉運使,坐貶,後起為彰信軍節度副使、知淮陽軍,卒。



 欒崇吉,字世昌,開封封丘人。少為吏部令史,上書言事,調補臨淄主簿。會令坐贓敗,即命崇吉代之。復以書判
 優等,改舒州團練判官,未行,留為中書刑房堂後官,改太子右贊善大夫,出掌揚州榷務。未幾,遷殿中丞,復為堂後官兼提點五房公事。



 崇吉明習文法,清白勤事。至道初,擢度支員外郎、度支副使。時以堂後官著作佐郎楊文質為秘書丞、提點五房事,上召見,謂曰:「汝見擢用欒崇吉否?當自勖勵。」崇吉俄加祠部郎中。真宗時,累擢為江南轉運使。代還,判刑部兼鼓司、登聞院。後遷司農少卿、知洪州。有司歲斂民財造舟,崇吉至,奏罷之。以疾
 徙濠州,遷衛尉少卿,以將作監致仕,卒。子二人:源,虞部員外郎。沂,殿中丞。



 袁逢吉,字延之,開封鄢陵人。曾祖儀,仕唐,以軍功至黃州刺史。祖光甫,尉氏令。父蟾,大理評事。逢吉四歲,能誦《爾雅》、《孝經》,七歲兼通《論語》、《尚書》。周太祖召見,發篇試之,賜束帛以賞其精習。開寶八年,擢《三傳》第,釋褐清江尉。知州王明薦其能,就除豐城令。明年,又與轉運使張去華條上治狀,以《春秋》博士召。端拱初,遷國子博士、度支
 推官。又判戶部勾院、度支、憑由理欠司。淳化中,改戶部判官。歷水部、司門員外郎。出為西京轉運使,轉水部郎中。宰相呂蒙正稱其有經術,宜任學官。會蜀叛,方籍其吏資授西川轉運使。至道初,徙荊湖北路。時賊方平,夔、峽猶聚官軍,供饋出於荊楚,逢吉憚涉遠,不赴軍前計度,坐乏糧餉,罷職知夔州。會遣使川、陜採訪,因條上知州、通判有治跡者七人,逢吉與朱協、李虛己、薛顏、邵曄、查道、劉檢預焉,皆賜詔褒諭。歷司門、庫部二郎中。



 咸平
 中,復為京東轉運使,連知福、江、陳、襄四州。大中祥符中,權西京留司御史臺,徙知汝州,以逮事太祖,拜鴻臚少卿。七年,卒,年六十九。



 逢吉性修謹,練達時務。初,鄆州牧馬草地侵民田數百頃,牒訴連上,凡五遣使按視,不決。逢吉受命往,則悉還所侵田,民咸德之。兄及甫,歷京東、峽路轉運副使,至駕部郎中。逢吉子成務,至比部員外郎、京東轉運副使。從子楚材,至虞部員外郎。



 韓國華,字光弼,相州安陽人。太平興國二年舉進士,解
 褐大理評事、通判瀘州,就遷右贊善大夫。代還,除彰德軍節度判官。遷著作佐郎、監察御史。



 雍熙中,假太常少卿使高麗。時太宗將北征,以高麗接遼境,屢為其所侵,命繼詔諭之,且令發兵西會。既至,其俗頗獷驁,恃險遷延,未即奉詔。國華移檄,諭以朝廷威德,宜亟守臣節,否則天兵東下,無以逃責。於是俯伏聽命。使還,賜緋魚。雍熙三年,改右拾遺、直史館,判鼓司、登聞院,俄充三司開拆推官。四年,判本司,遷左司諫,充鹽鐵判官。



 淳化二年,
 契丹請和,朝議疑其非實,遣國華使河朔以察之。既至,盡得其詐以聞。每歲後苑賞花,三館學士皆得預。三年春,國華與潘太初因對,自言任兩省清官兼計司職,不得侍曲宴,願兼館職,即日命並直昭文館。後二日,陪預苑宴。三司屬官兼直館,自國華等始。未幾,授刑部員外郎,歷判三司勾院,復為鹽鐵判官,又為左計判官,尋都判三勾,賜金紫,改兵部員外郎、屯田郎中、京東轉運使,徙陜西路。舊制,川、陜官奉緡悉支鐵錢,資用多乏,國華奏
 增其數。加都官郎中,入判大理寺,改職方郎中。以詳定失中,命梁顥代之。知河陽、潞州,轉運使言其善綏輯,供億乾辦,詔獎之。



 景德中,假秘書監使契丹,又為江南巡撫,入權開封府判官。真宗朝陵,魏咸信自曹州召入扈從,命國華權州事。俄改太常少卿、出知泉州。大中祥符初,遷右諫議大夫。四年,代還,至建州,卒於傳舍,年五十五。賜其子珫出身。



 國華偉儀觀,性純直,有時譽。子琚、璩、琦,並進士及第。琦相英宗、神宗,自有傳。



 何蒙,字叔昭,洪州人。少精《春秋左氏傳》。李煜時,舉進士不第,因獻書言事,署錄事參軍。入宋,授洺州推官。太平興國五年,調遂寧令。時太宗親征契丹還,作詩以獻。召見賞嘆,授右贊善大夫,三遷至水部員外郎、通判廬州。時郡中火燔廨舍,榷務俱盡。蒙假民器,貸鄰郡曲米為酒,既而課增倍。戶部使上其狀,詔SS緡錢獎之。稍遷司門。巡撫使潘慎修薦其材敏,驛召至京,因面對,訪以江、淮茶法,蒙條奏利害稱旨,賜緋魚及錢十萬。後二日復
 對,又上淮南酒榷便宜,特改庫部,復賜錢二十萬,因命至淮右提總其事,自是歲有羨利。使還,知溫州,未行,留提舉在京諸司庫務。求外任,復命知溫州。坐舉人不當,削一官。



 真宗即位,復前資,因上言請開淮南鹽禁。時卞袞、楊允恭輩方以禁鹽為便,共排抑之,出知梧州。頃之,改水部郎中,上所著《兵機要類》十卷。時審官擬知漢陽軍,及引對,改知鄂州。大中祥符初,轉庫部。四年,加太府少卿。未幾,知太平州,又知袁州。州民多採金,蒙建議請
 以代租稅。上曰:「若此則農廢業矣。」不許。俄徙濠州。六年,上表謝事,授光祿少卿致仕,命未下,卒,年七十七。



 慎知禮,衢州信安人。父溫其,有詞學,仕錢俶,終元帥府判官。知禮幼好學,年十八,獻書乾俶,署校書郎。未幾,命為掌書記。



 宋初,介俶子惟濟入覲,歸,署營田副使。太平興國三年,從俶歸朝,授鴻臚卿。歷知陳州、興元府。知禮母年八十餘,居宛丘,懇求歸養。退處十年,縉紳稱其孝。及母服除,表請納祿。至道三年,以工部侍郎致仕。知禮
 自幼至白首,歲讀《五經》,周而後止。每開卷,必正衣冠危坐,未嘗少懈焉。咸平初卒,年七十一。子從吉。



 從吉字慶之,錢俶之婿也。為元帥府長史。歸宋,歷將作少監。會擇朝士有望者補少列,改太子右庶子。真宗升儲,換衛尉少卿。真宗即位,復為右庶子,遷詹事。從吉自歸朝,居散秩幾三十年,頗以文酒自娛,士大夫多與之游。景德初,上言求領事務,判刑部。頗留意法律,條上便宜,天下所奏成案率多糾駁,取本司所積負犯人告身
 鬻之,以市什器。



 大中祥符初,改授衛尉卿,糾察在京刑獄,拜右諫議大夫,判吏部銓。初,選人試判多藉地而坐,從吉以公錢市莞席給之。臨事敏速,勤心公家,所至務皦察,多請對陳事,上謂其無隱。



 八年,改給事中、權知開封府。既受命,召戒之曰:「京府浩穰,凡事太速則誤,太緩則滯,惟須酌中耳。請屬一無所受。」才數月,有咸平縣民張斌妻盧氏,訴侄質被酒詬悖。張素豪族,質本養子,而證左明白,質賄於吏。從吉子大理寺丞銳時督運石塘
 河,往來咸平,為請於縣宰,斷復質劉姓,第令與盧同居。質洎盧迭為訟,縣聞於府。從吉命戶曹參軍呂楷就縣推問。盧之從叔虢略尉昭一賂白金三百兩於楷,楷久不決。盧兄文質又納錢七十萬於從吉長子大理寺丞鈞,鈞以其事白從吉,而隱其所受。盧又詣府列訴,即下其事右軍巡院。昭一兄澄嘗以手書達錢惟演,雲寄語從吉,事逮鈞、銳,請緩之。從吉頗疑懼,密請付御史臺。即詔御史王奇、直史館梁固鞫之。獄成,從吉坐削給事勒
 停,惟演罷翰林學士,楷、鈞免官,配隸衡、郢州,銳、文質皆削一官,澄、昭一並決杖配隸。



 又高清者,庫部郎中士宏之子,景德中舉進士,宰相寇準以弟之女妻之。寇氏卒,故相李沆家復婿之。歷官以賄聞,頗恃姻援驕縱,被服如公侯家,以是欺蠹小民。知太康縣,民有詣府訴家產者,清納其賄,時已罷任,即逃居他所。銳嘗就清貸白金七十兩,清以多納賄賂,事將敗,求以為助。時方鞫盧氏獄,從吉請對,發其事,欲以自解。逮清等系獄,命比部員
 外郎劉宗吉、御史江仲甫劾之。清枉法當死,特杖脊黥面,配沙門島;銳又削衛尉寺丞。從吉坐首露在已發,當贖銅,特削諫議大夫。天禧三年,起為衛尉卿。明年,判登聞鼓院。坐與寇準善,以光祿卿致仕。未幾卒,年七十。



 從吉喜為詩,時有警語。兼工醫術。子孫登仕者甚眾,第進士升朝曳朱紱者數人。家富於財,尤能治生,多作負販器僦賃,以至鬻棺櫝於市。又善為饌具,分遺權要。晚年進趨彌篤,以至於敗,物論鄙之。子鏞,金部度支員外郎、秘
 閣校理。鍇,太常博士。



 論曰:八政之首食貨,以國家之經費不可一日而無也。然生之有道而用之有節,則存乎其人焉爾。張鑒將命西蜀,處制得宜,庶乎可與行權者也。子輿裁損經制,索湘議罷鬻茶,許驤謹守儒行,知禮篤信經學,國華不辱君命,皆有足稱者焉。太初自謂達性命之蘊,而卒流於釋、老之歸,文寶久任邊郡,而不免以生事蒙黜,劉綜著勞朔、易而短於經術,從吉勤於公務而疏於訓子,固未
 得為盡善也。自餘諸子,之翰虧潔白之操,卞袞乏仁恕之道,冕之棄其城守,坦之疏於輔導,則君子所不取也



\end{pinyinscope}