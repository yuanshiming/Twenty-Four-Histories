\article{傳第十二}

\begin{pinyinscope}

 折德扆子御勛御卿曾孫克行馮繼業王承美李繼周孫行友子全照



 折德扆世居雲中,為大族。父從阮,自晉、漢以來,獨據府州,控扼西北,中國賴之。仕周至靜難軍節度使。其鎮府
 州時,署德扆為馬步軍都校。廣順間,周世宗建府州為永安軍,以德扆為節度使,時從阮鎮邠寧,父子俱領節鎮,時人榮之。



 顯德中,德扆率師攻下河市鎮,斬並軍五百餘級。入朝,以其弟德願權總州事。時世宗南征,還次通許橋,德扆迎謁,且請遷內地。世宗以其素得蕃情,不許,厚加賜賚而遣之。德扆未至,德願又破並軍五百餘於沙谷砦,斬其將郝章、張劍。



 宋初,德扆又破河東沙谷砦,斬首五百級。建隆二年來朝,待遇有加,遣歸鎮。乾德
 元年,敗太原軍於城下,擒其將楊璘。二年,卒,年四十八,贈侍中。子御勛、御卿。



 御勛字世隆,德扆鎮府州日,表為右職。德扆卒,以御勛領汾州團練使、權知府州事。開寶二年,太祖征太原,御勛詣行在謁見,以為永安軍留後。四年,以郊祀來朝,禮畢歸鎮。九年,郊祀西洛,復來朝,道病後期,改泰寧軍節度使,留京師。太平興國二年,卒,年四十,贈侍中。



 御卿,幼補節院使,御勛知州事,署為兵馬都校。御勛徙
 鎮,召為閑廄副使、知府州。太宗征河東,命御卿與尹憲領屯兵同攻嵐州,又破岢嵐軍,擒其軍使折令圖以獻,遂下嵐州,又殺其憲州刺史霍翊,又擒其將馬延忠等七人。遷崇儀使。



 淳化三年,凡四遷而為府州觀察使。五年,拜永安軍節度使。既而契丹眾萬餘入寇,御卿大敗之於子河水義,斬首五千級,獲馬千匹,契丹將號突厥太尉、司徒、舍利死者二十餘人,擒其吐渾一人,自是契丹知所畏。太宗因遣使問御卿曰:「西北要害皆屯勁兵,戎
 人何自而至?」御卿對曰:「敵緣山峽小徑入,謀剽略。臣諜知之,遣人邀其歸路,因縱兵大擊,敗走之,人馬墜崖谷死者相忱,其大將韓德威僅以身免。皆聖靈所及,非臣之功也。」上嘉之。



 歲餘,御卿被病,德威諜知之,且為李繼遷所誘,率眾來侵,以報子河水義之役。御卿力疾出戰,德威聞其至,不敢進。會疾甚,其母密遣人召歸,御卿曰:「世受國恩,邊寇未滅,御卿罪也。今臨敵棄士卒自便,不可,死於軍中乃其分也。為白太夫人,無念我,忠孝豈兩全!」
 言訖泣下。翌日卒,年三十八。上聞悼惜久之,贈侍中,以其子惟正為洛苑使、知州事。惟正歸朝,以其弟惟昌繼之。



 咸平二年,河西黃女族長蒙異保及惟昌所部啜訛引趙保吉之眾入冠麟州萬戶谷,進至松花砦,惟昌與從叔同巡檢使海超、弟供奉官惟信率兵赴戰。會保吉兵眾,官軍不敵,惟昌臂中流矢墜馬,攝弓起,得裨將馬突圍出,海超、惟信沒焉。九月,保吉黨萬私保移埋復來寇,惟昌與宋思恭、劉文質合戰於埋井峰,敗走之。又破
 言泥族拔黃砦,焚其器甲、車帳,俘斬甚眾。以功領富州刺史,改文思使。景德元年,與王萬海等破賊砦,護芻糧抵麟州。秋,入朔州界,破狼水砦,時契丹方圍岢嵐軍,聞敗遁去。明年,拜興州刺史。



 大中祥符二年,表求赴闕。真宗命近臣與射於苑中,宴賜甚厚。上言:「先臣御卿蒙賜旗三十竿以壯軍容,請別給賜。」許之。七年,命河東民運糧赴麟州,當出兵為援,惟昌力疾領步騎屯寧遠砦,冒風沙而行。時疾已亟,猶與賓佐宴飲,談笑自若焉。明日
 卒,年三十七。以其弟惟忠繼之。



 惟忠字藎臣,初以兄惟信戰沒,補西頭供奉官,擢閣門祗候。及惟昌卒,以惟忠為六宅使、知府州兼麟府路都巡檢使,領普州刺史;再遷左藏庫使,真拜嘉州刺史,改資州,進簡州團練使。喪母,起復雲麾將軍卒。



 惟忠知兵事。天聖中,契丹與夏國會兵境上,聲言嫁娶,惟忠覘得其實,率麾下往備之,戒士卒毋輕動。一夕風霾,有騎走營中,以為寇至,惟忠堅臥不動,徐命擒之,得數誕馬,蓋虜所縱也。既卒,錄其弟
 侄子孫七人,以其子繼宣嗣州事。久之,特贈惟忠耀州觀察使。



 寶元中,繼宣坐苛虐掊刻,種落嗟怨,絀為左監門衛將軍、楚州都監,擢其弟右侍禁繼閔為西京作坊使,嗣州事。



 繼閔字廣孝。慶歷中,元昊兵攻麟州不克,進圍州城。城險且堅,東南有水門,崖壁峭絕,阻河。賊緣崖腹微徑魚貫而前,城中矢石亂下,賊轉攻城北,士卒復力戰,賊死傷甚眾,遂引去,圍豐州,豐州遂陷。繼閔以城守勞,特遷宮苑使、普州刺史。未幾,護送麟州戍卒冬服,
 賊伏兵邀擊之,盡掠所繼,繼閔脫身繇間道歸。會赦,止奪宮苑使,從復官,領果州團練使。自元昊反,繼閔招輯歸業者三千餘戶。皇祐二年,卒,以其弟繼祖嗣州事。



 繼祖字應之,由右侍禁遷西染院使,累轉皇城使、成州團練使。臨政二十餘年。奏乞書籍,仁宗賜以《九經》。韓絳發河東兵城囉兀,繼祖為先鋒,深入敵帳,降部落戶八百。加解州防禦使卒。繼祖有子當襲州事,請以授兄之子克柔,詔從之,而進其三子官,錄二孫為借職。



 弟繼世,少
 從軍,為延州東路巡檢。搜名山之內附,繼世先知之,遣其子克勤報種諤,諤用是取綏州。繼世以騎步萬軍於懷寧砦,入晉祠谷,往銀川,分名山之眾萬五千戶居於大理河。夏人來攻,再戰皆捷。諤抵罪逮系獄,以兵付之而行,遂同名山守綏州,綠功領忠州刺史。說韓絳城囉兀以撫橫山,因畫取河南之策,絳以為然。以左騏驥使、果州團練使卒。諸司使無賻禮,詔以繼世蕃官,捍邊有績,特給之。從子克行。



 克行字遵道,繼閔子也。初仕軍府,無所知名。夏人寇環慶,種諤拒之,詔河東出師為援,克行請往。諤使以兵三千護餉道,戰於葭蘆川,先登,斬級四百,降戶千,馬畜萬計。諸老將矍然曰「真折太尉子也。」擢知府州。



 秦兵討夏國,張世矩將河外軍民,克行與俱。廷議謂守臣難自行,詔克行選兵隸世矩。克行抗章願率部落先驅,未報,即委管鑰而西。大酋咩保吳良以萬騎來躡,克行為後拒,度賊半度隘,縱擊大破之,殺咩保吳良。師還自劾,釋不
 問。王中正出塞,克行先拔宥州,每出必勝,夏人畏之,益左廂兵,專以當折氏。



 太原孫覽議城葭蘆,諸將論多不合,召克行問策,即頓兵吐渾河,約勒部伍,為深入窮討之狀,敵疑不敢動。既訖役,又入津慶、龍橫川,斬級三千。



 詔河東進築八砦,信道鄜延。延帥遣秦希甫來共議,克行請兩路並力,以遠者為先,希甫曰:「由近及遠,法也。」克行曰:「不然,事有奇正。今乘士氣之銳,所利在速,故先遠役,以出其不意,若徐圖之,士心且怠矣。」希甫持不可,並
 上二義,卒用克行策。城成,諜言寇至,軍中皆戒嚴,克行止之曰:「彼自擾耳。」已而果然。



 克行在邊三十年,善拊士卒,戰功最多,羌人呼為「折家之」。官至秦州觀察使,卒,贈武安軍節度使。子可大為榮州團練使、知府州。從子可適。



 可適未冠有勇,馳射不習而能。鄜延郭逵見之,嘆曰:「真將種也。」薦試廷中,補殿侍,隸延州。從種諤出塞,遇敵馬以少年易之,可適索與鬥,斬其首,取馬而還,益知名,米脂之役,與夏人戰三角嶺,得級多,又敗之於蒲桃谷
 東。兵久不得食,千人成聚,籍籍於軍門,或欲掩殺以為功,可適曰:「此以饑而逃耳,非叛也。」單馬出詰之曰:「爾輩何至是,不為父母妻子念而甘心為異域鬼耶?皆回面聲喏,流涕謝再生,各遣歸。



 羌、夏人十萬人寇,可適先得其守烽卒姓名,詐為首領行視,呼出塞斬之,烽不傳,因卷甲疾趨,大破之於尾丁磑。回次檉楊溝,正午駐營,公騎據西山,曰:「彼若躡吾後,腹背受敵,必敗。」果舉軍來,可適所部才八千,轉戰至高嶺,乃從間道趣洪德,設伏邀
 其歸路。敵至,伏發沖之,其國母窬山而遁,焚棄輜重,雖帷賬首飾之屬亦不返,眾相蹈藉,赴崖澗死者如積。論前後功,至皇城使、成州團練使、知岷蘭州鎮戎軍。



 渭帥章楶合熙、秦、慶三道兵築好水川,命總管王文振統之,而可適將軍為副。熙州兵千人失道盡死,文振歸罪於可適,楶即下之吏,宰相章惇欲按軍法,哲宗不許,猶削十三官而罷。楶請留以責效,乃以權第十二將。嵬名阿埋、昧勒都逋,皆夏人桀黠用事者,詔可適密圖之,會二
 酋以畜牧為名會境上,可適諜知之,遣兵夜往襲,並俘其族屬三千人,遂取天都山。帝為御文德殿受賀,以其地為西安州,遷可適東上合門使、洛州防禦使、涇原鈐轄、知州事,真拜和州防禦使,進明州觀察使,為副都總管。



 帥鐘傳行邊,為敵所隔,以輕騎拔之,得歸。傳議取靈武,環慶亦請出師,命可適將萬騎往,即薄靈州川。夏人扶老挾稚,中夜入州城,明日俘獲甚伙,而慶兵不至,乃引還。詔使人覲,帝以傳策訪焉,對曰:「得之易,守之難,當
 先侵弱其地,待吾藩籬既固,然後可圖。」帝曰:「卿言是也。」進武安軍節度觀察留後、步軍都虞候。



 大城蕭關,與傳議齟齬,會覆師數百於踏口,傳劾之,貶鄭州觀察使。俄知衛州,拜淮康軍節度使。轉運使請於平夏、通峽、鎮戎、西安四砦分築場圃,置芻粟五百萬,可適以費大難之,又欲借車牛以運,及致十萬斛於熙河,皆戾其意,乃中以疑謗,召為祐神觀使。明年,復以為渭州,命其子彥質直秘閣參軍事,數月而卒,年六十一。彥質,紹興中簽書
 樞密院,別有傳。



 馮繼業字嗣宗,大名人,父暉,朔方節度,封衛王。繼業幼敏慧,有度量,以父任補朔方軍節院使,隨父歷邠、孟,及再領朔方,皆補牙職。周廣順初,暉疾,繼業圖殺其兄繼勛。暉卒,遂代其父為朔方軍留後。以郊祀恩,加靈州大都督府長史,遷朔方節度、靈環觀察、處置、度支、溫池榷稅等使。



 恭帝時,繼業既殺兄代父領鎮,頗驕恣,時出兵劫略羌夷,羌夷不附,又撫士卒少恩,繼業慮其為變,以
 太祖居鎮日常得給事,乃豫徙其孥闕下。



 建隆初,來朝,連以駝馬、寶器為獻。開寶二年,賜詔獎諭,拜靜難軍節度使。三年,改鎮定國軍,吏民立碑頌其遺愛。太平興國初來朝,封梁國公,留京師。明年,卒,年五十一,贈侍中。



 王承美,豐州人,本河西藏才族都首領。其父事契丹,為左千牛衛將軍,開寶二年率眾來歸。承美授豐州牙內指揮使,父卒,改天德軍蕃漢都指揮使、知州事,移豐州刺史。遣軍校詣闕言,願誘退渾、突厥內附,上嘉其意。



 太
 平興國七年,與契丹戰,斬獲以萬計,禽其天德軍節度使韋太以獻。明年,契丹來寇,又擊敗其眾萬餘,追北至青塚百餘里,斬獲益眾。以功授本州團練使。以乞黨族次首領弗香克浪買為歸德郎將,沒細大首領越移為懷化大將軍,瓦窯為歸德大將軍。淳化二年冬來朝,令歸所部,控子河議。自是諸蕃歲修貢禮,頗效忠順。



 景德初來朝,以其守邊歲久,遷本州防禦使以還。自承美內屬,給奉同蕃官例,至是,特詔月增五萬。尋請於州城置
 孔子廟,詔可之。未幾被疾,遣中使挾醫視之。大中祥符五年,卒,贈恩州觀察使。六年,錄其子文寶、孫懷筠以官。



 初,承美養其長孫文玉為子,奏署殿直,及卒,其本族首領上言文玉曉達軍政,請令襲承美任。下蕃漢議,議同,以為侍禁、知州事。文玉父文恭時為侍禁,在沂州,表訴其事,詔改文恭為供奉宮。九年,承美葬,詔以帛緡、米曲、羊酒賜其家。



 李繼周,延州金明人。祖計都,父孝順,皆為金明鎮使,繼
 周嗣掌本族。



 太平興國三年,東山蕃落集眾寇清化砦,繼周率眾敗之,殺三千餘人,補殿前承旨。雍熙中,又與侯延廣敗未藏、未腋等族於渾州西山。淳化四年,遷殿直,賜介胃、戎器、茶彩。明年,討李繼遷,命開治塞門、鴉兒兩路,又招降族帳首領二十餘人,率所部人夏州,敗蕃兵數千於石堡砦。以功轉供奉官,復加恩賞,仍賜官第。



 繼周以阿都關、塞門、盧關等砦最居邊要,遂規修築砦城。在磨盧家、媚咩、拽藏等族居近盧關,未嘗內順。繼周
 夜率所部往襲,焚之,斬首俘獲甚眾。至道二年,授西京作坊副使,賜袍帶、銀彩、雕戈以寵之。大軍討西夏,命為延州路踏白先鋒。會繼遷邀戰於路,繼周戰卻之。咸平初,改西京左藏庫副使。三年,復為先鋒,人賊境,焚積聚,殺人畜,獲器甲凡六十餘萬。授供備庫使,領金明縣兵馬都監、新砦解家河盧關路都巡檢。五年,授西京作坊使。蕃騎人鈔,繼周逐之出境。景德元年,夏人圍麟州,繼周受詔率兵會李繼福掩擊之。加領誠州刺史。



 大中祥
 符二年,卒,年六十七,詔邊臣擇其子可襲職者以名聞,邊臣言其子殿直士彬遜,從子士用樸忠練邊事,且為部落所伏。乃詔士彬管勾部族事,士用為巡檢都監以左右之。



 士彬後至供備庫副使、金明縣都監、新砦解家河盧關路巡檢。康定元年,元昊反,攻保安軍,而潛兵襲金明,士彬父子俱被禽。士彬兄士紹至內殿崇班,士用至供奉官、閣門祗候。



 李繼福者,亦與繼周同時歸順,授永平砦茇村軍主,以戰功歷歸德將軍,領順州刺史,
 至內殿崇班、新歸明諸族都巡檢。



 孫行友,莫州清苑人,世業農。初,定州西二百里有狼山者,當易州中路,舊有城堡,邊人賴之以避寇。山中蘭若有尼,姓孫氏,名深意,有術惑眾。行友兄方諫名之為姑師,事之甚謹。及尼坐亡,行友益神其事,因以其術然香燈,聚民漸眾。自晉少帝與契丹絕好,邊州困於轉輸,逋民往往依方諫,推以為帥。方諫懼主帥捕逐,乃表歸朝,因署為東北面招收指揮使,且賜院額曰「勝福」。每契
 丹軍來,必率其徒襲擊之,鎧仗、畜產所得漸多,人益依以避難焉。易、定帥聞於朝,因以方諫為邊界游奕使,行友副之。自是捍禦侵鐵,多所殺獲。乘勝入祁溝關、平庸城,破飛狐砦,契丹頗畏之,邊民千餘家賴以無患。然亦陰持兩端,以圖自固。



 已而晉師失律,薊人道契丹陷中原,方諫之密構也。契丹授方諫定州節度,行友易州刺史。尋以蕃將耶律忠代方諫於雲州〔10〕,方諫不受命,歸保狼山。契丹北歸,焚劫中山,方諫自狼山率眾復保定州,
 歸命於漢,授行友易州刺史,行義泰州刺史。弟兄掎角以居,寇每人,諸軍鎮閉壘坐視,一無所得。



 行友嘗遣都校王友遇巡警於石河,與契丹遇,殺百餘騎,又嘗獲其刺史蔡幅順、清苑令王璉。乾祐中,契丹復犯塞,行友御之,俘殺數百人。周太祖北征,行友道獻俘馘人馬以求見,且請自效,乃厚加賜予,留之軍門。及周祖受命,行友屢上言偵得契丹離合,願得勁兵三千乘間平定幽州,乃移言諫鎮華州,以行友為定州留後。顯德初,正授節
 鉞。世宗自河東還,加檢校太傅。六年,世宗北征,行友攻下契丹之易州,擒其刺史李在欽以獻。



 宋初,加同平章事。狼山佛舍妖妄愈甚,眾趨之不可禁,行友不自安,累表乞解官歸出,詔不允。建隆二年,乃徙其帑廩,召集丁壯,繕治兵甲,欲還狼山以自固。兵馬都監藥繼能密表其事,太祖遣閣門副使武懷節馳騎會鎮、趙之兵。稱巡邊直入其城,行友不之覺。既而出詔示之,令舉族赴關,行友蒼黃聽命。既至,命侍御史李維岳就第鞫之,得實,下
 詔切責,削奪從前官爵。勒歸私第。仍戮其部下數人,遣使馳詣狼山,輦其尼師之尸焚之。行友弟易州刺史方進、兄子保塞軍使全暉皆詣關待罪,詔釋之。



 四年秋,詔免行友禁錮。未幾,以郊祀恩,起為右能武軍將軍。乾德二年,遷右監門衛大將軍,又改左能武軍大將軍。太平興國六年,卒,年八十,贈左衛上將軍。方進至德州刺史。子全照。



 全照字繼明,以蔭補殿直,雍熙中授京南巡檢,俄隸幽
 州部署曹彬麾下,遷供奉官、閣門祗候歷靜戎、威虜二軍監軍。從田重進擊賊有功,就加西京作坊使,兼知威虜軍,運為廣韶、延二路都巡檢使。淳化五年,率兵與李繼隆克綏州,因與張崇貴等同戍守之。俄護屯兵於夏州,兼和州事。召還,為登萊路都巡檢使,遷左藏庫使、延州監軍兼阿都關盧關路都巡檢事。



 咸平初,人掌軍頭引見司。二年,加如京使,為涇原路鈐轄兼安撫都監,是冬徙並、汾等州都巡檢使。三年,改知順安軍,代還,復
 為環慶路鈐轄,與李繼和規度靈州道路。四年,加西上閣門使,復為環慶路鈐轄。五年,將城綏州,以慕興為綏州路部署,全照為鈐轄。既又慮全照素剛執,與興不協,乃以曹璨代之。既調兵夫二萬餘,全照言其非便,乃罷。又嘗命度地河北,全照言沿河高阜可分置城堡屯戍者,寧邊軍南、武強縣側凡二處,上重於興役,止命營安平南,徙置祁州。俄知天雄軍府。六年夏,上裁定防秋御戎之要,命為寧邊軍部署,領兵八千扼要害之路。以全
 照好陵人,取其嘗所保薦者王德鈞、裴自榮共事焉。



 景德元年,上幸澶淵,命為駕前西面邢洺路馬步軍鈐轄兼天雄軍駐泊,兼管勾東南貝、冀等州鈐轄。全照言:「若敵騎南逼魏城,但得騎兵千百,必能設奇取勝。」上賞其忠果,乃傳詔都部署周瑩,若全照欲擊賊,即分兵給之。既而邊騎果逼府城,全照拒退之,真宗遣使勞慰。時契丹請和,朝廷遣曹利用就其行帳議事,全照疑非誠懇,勸判府王欽若留不遣,故德清軍不能守,吏民多為賊
 所害。及契丹出境,北面將帥還師並至府城,全照令以次雙行入門,魏能不從其約,率兵馬坌入,全照坐城樓引弓射之。欽若入朝就命,全照知軍府事,以城守勞,加檢校工部尚書,增食邑三百戶。徙鎮州。召還,進東上合門使,領英州刺史。



 全照形短精悍,知兵,以嚴毅整眾,然性剛使氣,專任刑罰。中書初進擬嚴州刺史,上曰:「全照深刻,常慮人以嚴察議己,今授此州,似涉譏誚。」乃改焉。三年,為邠寧環慶都部署。趙德明納款,朝議減西鄙戍
 兵,令屯近地,全照以邊防不可無備,未即奉詔。上曰:「全照是好勇多言者,德明使已至闕,復何慮焉。」因徙全照知永興軍府,仍拜四方館使。西師移屯者至府,命全照兼駐泊鈐轄。全照許州有別墅,求典是州,可之。大中祥符中,遷引進使。逾歲表求歸朝,命掌閣門、客省、四方館事。四年,車駕西幸,留為新城都巡檢。未幾卒,年六十。



 論曰:五代之季,邊圉之不靖也久矣。太祖之興,雖不勤遠略,而向之陸梁跋扈而不可制者,莫不竭忠效節,雖
 奔走殭僕而不避,豈人心之有異哉?良由威德之並用,控御之有道也。折氏據有府谷,與李彞興之居夏州初無以異。太祖嘉其響化,許以世襲,雖不無世卿之嫌,自從阮而下,繼生名將,世篤忠貞,足為西北之捍,可謂無負於宋者矣。承美、繼周,分蒞種落,亦能世其職者也。繼業雖出賊叛之族,而有循良之風。方諫、行友介遼、晉間,持雨端以取將相,終以首鼠獲咎,其諸異端之害歟。全照職親禁衛,素稱嚴果,而昧於弭兵之利,君子所不予
 也



\end{pinyinscope}