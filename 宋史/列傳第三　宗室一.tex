\article{列傳第三 宗室一}

\begin{pinyinscope}

 魏王廷美燕王德昭秦王德芳秀王子戴附



 昔周之初興,大封建宗室,及其東遷,晉、鄭有同獎之功。
 然其衰也,幹弱而枝強。後世於是有矯其失者,而封建不復古矣。宋承唐制,宗王襁褓即裂土而爵之。然名存實亡,無補於事。降至疏屬,宗正有籍,玉牒有名,宗學有教,郊祀、明堂,遇國慶典,皆有祿秩。所寓州縣,月有廩餼。至於宗女適人,亦有恩數。然國祚既長,世代浸遠,恆產豐約,去士庶之家無甚相遠者。靖康之亂,諸王駢首以弊於金人之虐,論者咎其無封建之實,故不獲維城之助焉。



 雖然,東都之仁宗,南渡之高、寧,元良虛位,立繼小
 宗,大策一定,卒無動搖,盤石之固,亦可知矣。且宋於宗室,稍有過差,君臣之間,不吝於改,尤不憚於言。涪陵、武功,真宗即位,尋議追復改葬,封其子孫。濮邸尊稱,言者惟務格非,不少避忌。宋末濟邸,國事將亡,諫疏不息,必褒恤而後止。是蓋歷代之所難得者歟!表而出之,作《宗室傳》,



 魏悼王廷美字文化,本名光美,太平興國初,改今名。太祖兄弟五人:兄光濟,早亡,宋興,追封邕王,改曹王;弟光
 義,即太宗;次廷美;次光贊,幼亡,追封夔王,改岐王。



 建隆元年,授廷美嘉州防禦使。二年,遷興元尹、山南西道節度使。乾德二年,加同中書門下平章事。開寶六年,加檢校太保、待中、京兆尹、永興軍節度使。太宗即位,加中書令、開封尹,封齊王,又加檢校太師。從征太原,進封秦王。



 七年三月,或告秦王廷美驕恣,將有陰謀竊發。上不忍暴其事,遂罷廷美開封尹,授西京留守,賜襲衣、通犀帶,錢千萬緡,絹、彩各萬匹,銀萬兩,西京甲第一區。詔樞密
 使曹彬餞廷美於瓊林苑。以太常博士王遹判河南府事,開封府判官閻矩判留守事。以如京使柴禹錫為宣徽北院使兼樞密副使,楊守一為東上閣門使充馮樞密都承旨,賞其苦廷美陰謀功也。左衛將軍、樞密承旨陳從龍為左衛將軍,皇城使劉知信為右衛將軍,弓箭庫使惠延真為商州長史,禁軍列校皇甫繼明責為汝州馬步軍都指揮使,定人王榮為濮州教練使,皆坐交通廷美及受其燕犒也。榮未行,或又告榮嘗與廷美親吏
 狂言:「我不久當得節帥。」坐削籍,流海島。



 會趙普再相,廉得盧多遜與廷美交通事上聞。上怒,責授多遜兵部尚書,下御史獄。搏系中書守當官趙白、秦府孔目官閻密、小吏王繼勛、樊德明、趙懷祿、閻懷忠等,命翰林學士承旨李昉、學士扈蒙、衛尉卿崔仁冀、膳部郎中兼御史知雜滕中正雜治之。多遜自言:累遣趙白以中書機事密告廷美。去年九月中,又令趙白言於廷美云:「願宮車晏駕,盡力事大王。」廷美遣樊德明報多遜云:「承旨言正會
 我意,我亦願宮車早晏駕。」私遺多遜馬箭等,多遜受之。



 閻密初給事廷美,上即位,補殿直,仍隸秦王府,恣橫不法,言多指斥。王繼勛尤為廷美親信,嘗使求訪聲妓,怙勢取貨,贓污狼藉。樊德明素與趙白游處,多遜因之以結廷美。廷美又遣趙懷祿私召同母弟軍器庫副使趙廷俊與語。閻懷忠嘗為廷美詣淮海王錢俶求犀玉帶、金酒器,懷忠受俶私遺白金百兩、金器、絹扇等。廷美又嘗遣懷忠繼銀碗、錦彩、羊酒,詣其妻父御前忠佐馬軍
 都軍頭開封潘潾營燕軍校。至是,皆伏罪。



 詔文武常參官集議朝堂。太子太師王溥等七十四人奏:「多遜及廷美顧望□兄詛,大逆不道,宜行誅滅,以正刑章。趙白等處斬。」詔削奪多遜官爵,並家屬流崖州;廷美勒歸私第;趙白、閻密、王繼勛、樊德明、趙懷祿、閻懷忠皆斬於都門外,籍其家財。詔:「秦王廷美男女等宜正名稱,貴州防禦使德恭等仍為皇侄;皇侄女適韓氏去雲陽公主之號;右監門將軍韓崇業降為右千牛衛率府率,仍去附馬都
 尉之號:並發遣西京,就廷美居止。」五月,貶西京留守判官閻矩為涪州司戶參軍,前開封推官孫嶼為融州司戶參軍,皆秦王廷美官屬,坐輔導無狀也。



 趙普以廷美謫居西洛非便,復教知開封府李符上言:「廷美不悔過,怨望,乞徙遠郡,以防他變。」詔降廷美為涪陵縣公,房州安置。妻楚國夫人張氏,削國封。命崇儀使閻彥進知房州,監察御史袁廓通判州事,各賜白金三百兩。八年正月,涪陵縣公廷美母陳國夫人耿氏卒。雍熙元年,廷美
 至房州,因憂悸成疾而卒,年三十八。上聞之,嗚咽流涕,謂宰相曰:「廷美自少剛愎,長益兇惡。朕以同氣至親,不忍置之於法,俾居房陵,冀其思過。方欲推恩復舊,遽茲殞逝,痛傷奈何!」因悲泣,感動左右,遂下詔追封廷美為涪王,謚曰悼,為發哀成服。



 其後,太宗從容謂宰相曰:「廷美母陳國夫人耿氏,朕乳母也,後出嫁趙氏,生廷俊。朕以廷美故,令廷俊屬鞬左右,而廷俊洩禁中事於廷美。邇者,鑿西池,水心殿成,橋梁未備,朕將泛舟往焉。廷美
 與左右謀,欲以此時竊發,不果,即詐稱疾於邸,俟朕臨省,因而為變。有告其事者,若命有司窮究,則廷美罪不容誅。朕不欲暴揚其醜,及盧多遜交通事發,止令居守西洛。而廷美不悔過,益怨望,出不遜語,始命遷房陵以全宥之。至於廷俊,亦不加深罪,但從貶宥。朕於廷美,蓋無負矣!」言未訖,為之惻然。李昉對曰:「涪陵悖逆,天下共聞。西池,禁中事,若非陛下季曲宣示,臣等何由知之。」



 初,昭憲太后不豫,命太祖傳位太宗,因顧謂趙普曰:「爾同
 記吾言,不可違也。」命普於榻前為約誓書,普於紙尾書云「臣普書」,藏之金匱,命謹密宮人掌之。或謂昭憲及太祖本意,蓋欲太宗傳之廷美,而廷美復傳之德昭。故太宗既立,即令廷美尹開封,德昭實稱皇子。德昭不得其死,德芳相繼夭絕,廷美始不自安。已而柴禹錫等告廷美陰謀,上召問普,普對曰:「臣願備樞軸以察奸變。」退復密奏:「臣忝舊臣,為權幸所沮。」因言昭憲太后顧命及先朝自朔之事。上於宮中訪得普前所上章,並發金匱得
 誓書,遂大感悟。召普謂曰:「人誰無過,朕不待五十,已盡知四十九年非矣。」辛亥,以普為司徒兼侍中。他日,太宗嘗以傳國之意訪之趙普,普曰:「太祖已誤,陛下豈容再誤邪?」於是廷美遂得罪。凡廷美所以遂得罪,普之為也。



 至道初,命司門員外郎孫蠙為皇侄、諸孫教授,廷美諸子之在京者肄業焉。真宗即位,追復皇叔涪王廷美西京留守、檢校太師兼中書令、河南尹、秦王;張氏,楚國夫人。咸平二年閏三月,詔擇汝、鄧地,改葬汝州梁縣之新
 豐鄉。仁宗即位,贈太師、尚書令。徽宗即位,改封魏王。



 子十人:德恭、德隆、德彞、德雍、德鈞、德欽、德潤、德文、德願、德存。故事,皇族封王者物故,則本宮之長封國公,其後以次受封。於是,德鈞子承簡屬最長,襲封徐國公,官至保康軍留後;贈彰化軍節度使、安定郡王,謚和懿。承簡既薨,德雍子承亮襲封昌國公;神宗即位,拜感德軍節度使,改封榮。



 熙寧二年,詔宣祖、太祖、太宗之子,皆擇其後一人為宗,世世封公,以奉其祀,不以服屬盡故殺其恩
 禮。三年,太常禮院言:「本朝近制,諸王之後,皆用本宮最長一人封公繼襲。去年詔祖宗之子皆擇其後一人為宗,世世封公,即與舊制有異。按禮文,諸王、公、侯、伯、子、男,皆子孫承嫡者傳襲。若無嫡子及有罪疾,立嫡孫;無嫡孫,以次立嫡子同母弟;無母弟,立庶子;無庶子,立嫡孫同母弟;無同母弟,立庶孫。曾孫以下準此。合依禮令,傳嫡承襲。」詔可。乃以承亮為秦國公,奉秦王廷美祀。明年薨,贈樂平郡王,謚曰恭靜。子克愉嗣。克愉卒,子叔牙嗣。
 元符三年,改今封。



 德恭字復禮,太平興國四年,以皇子出閣,拜貴州防禦使。廷美徙房陵,諸子悉從行,因免官。廷美卒,復以德恭為峰州刺史,弟德隆為瀼州刺史韓崇業,為靜難行軍司馬。雍熙元年十二月,詔以德恭為左武衛大將軍,封安定郡侯,判濟州;德隆為右武衛大將軍,封長寧郡侯,判沂州。諸弟皆隨赴治所。令高品衛紹欽送至州,常奉外歲給錢三百萬。命起居舍人韓檢、右補闕劉蒙叟分任二州通判。上臨遣之,曰:「德恭等始
 歷郡,善裨贊之。茍有闕失而不力正,止罪爾等。」



 端拱元年,進封德恭安定郡公。淳化四年,改左驍衛大將軍。至道二年,加左神武大將軍。真宗嗣位,就轉左武衛大將軍。咸平二年召赴闕,改封樂平郡公,判虢州。乞奉朝請,從之。遷勝州團練使。景德初,改衡州防禦使。三年,被疾,子承慶刲股肉食之。五月,卒,年四十五。上臨哭之慟,廢朝三日。贈保信軍節度使,追封申國公。天禧二年,從承慶請,加贈護國軍節度兼侍中。明道二年,追封高密郡
 王,謚慈惠。子承慶、承壽。



 承慶,官至和州團練使卒贈武信軍節度使、循國公。子六人,克繼,善楷書,尤工篆隸,宗正薦之,仁宗親臨試,主令監蔡邕古文法寫《論語》、《詩》、《書》;復詔與朝士分隸《石經》。帝曰:「李陽冰,唐室之秀。今克繼,朕之陽冰也。」訓子弟力學,一門登儒科者十有二人。嘗進所集《廣韻字源》,帝稱善,藏之秘閣。元祐五年,以定武軍節度觀察留後卒,贈開府儀同三司、建國公,謚章靖。



 承壽,終南作坊使,贈德州刺史、武當侯。子四人,克己,曉
 音律,嘗作《雅樂圖》樂曲以獻。侍宴大清樓,進所學虞世南書,賜器加等。終右千牛衛大將軍,贈深州防禦使、饒陽侯。子叔韶字君和,慶歷六年,與諸宗子帝前臨真宗御書,選第一。皇祐初,進所為文,召試學士院中等,賜進士及第。自太子右監門率府副率遷右領軍衛將軍,入謝,命坐賜茶。仁宗曰:「宗子好學者頗多,獨爾以文章第進士,前此蓋未有也。朕欲天下知屬籍有賢者,宜勿忘所學。」叔韶頓首謝,既退,又出《九經》賜之。遷右屯衛大將
 軍。至和中,上書求試煩劇,加領賀州刺史,終和州防禦使,贈鎮東節度觀察留後、會稽郡公。克修字子莊,仁宗為皇子時,得出入禁中侍學,故仁宗待遇殊厚。帝嘗御大清樓召宗室試書,以克修為善。終右神武軍大將軍、成州團練使,贈同州觀察使、馮翊侯。子叔充,父早世,異母弟叔瑁甚幼,叔充拊視誨敕成人。先是,繼母無敘封法,叔充請於朝,詔從之,遂為定制。藏書至萬卷。子九人,登科者三。卒官唐州防禦使,贈崇信軍節度使、尹國公,
 謚孝齊。遺表祈任子,有司格不下,子撫之抗章自列,乞如外官法。朝廷從其請。宗室正任有遺恩自此始。



 德隆字日新。雍熙三年,卒官沂州守,年二十三,贈寧遠軍節度,追封臨沂郡公。天禧二年,從其子承訓之請,加贈崇信軍節度、同平章事。承訓官至順州刺史,卒贈深州團練使。



 德彞字可久,太祖召鞠於宮中。德降卒,授右千牛衛大將軍,封長寧郡侯,代兄德隆判沂州,時年十九。飛蝗入境,吏民請坎瘞火焚之,德彞曰:「上天降災,守臣之
 罪也。」乃責躬引咎,齋戒致禱,既而蝗自殪。儒生乙恕者,郊居肄業,一日,有尸橫舍下,所司捕恕抵獄,將置於法。德彞疑其冤,命他司按之無異,因令緩刑以俟。未幾,果獲殺人者,恕遂得釋。進封郡公。淳化四年,為右監門衛大將軍,遷左武衛大將軍,改封廣平。部民詣闕乞留,有詔嘉獎。真守初,召還。咸平二年,命判滁州,與德恭並留不遣。三年,授徐州刺史,累遷保信軍節度觀察留後。大中祥符八年卒,年四十九。上臨奠,廢朝三日。贈昭信軍
 節度使,追封信都郡王,謚安簡。明道二年,改封穎川。



 子承謨,前卒;承矩,終莊宅使,贈博州刺使;承勖至供奉官,贈六宅副使;承節、承拱,並西京作坊使;承街,內殿崇班;承錫,供奉官。



 德雍字仲達,淳化初,授右驍衛將軍,歷右羽林、龍武二將軍,累廷蔡州觀察使、咸寧郡公,終天平軍節度觀察留後,贈宣德軍節度、同中書門下平章事,謚康簡。明道中,追封廣陵郡王。



 子承睦、承亮。承睦,終左領軍衛大將軍、彭州團練、虔州觀察使、南康侯;承亮,封
 秦國公,事見上。



 德鈞字子正,性和雅,善書翰,好為篇什。淳化初,拜右武衛將軍,四遷至右衛將軍。景德二年,加右監門衛大將軍。四年,卒,贈河州觀察使,追封安鄉侯。時妻亦卒,男女十四人皆幼,上甚嗟悼之。



 子承震,早卒;承緒,供奉官;承偉、承雅、承裔、承鑒、承則,並西京作坊使;承裕,禮賓副使;承翊,內殿崇班;承簡,襲徐國公;承乾,終懷州防禦使,贈保靜軍節度使、蕭國公,子克敦,嗜經術,以宗正薦,召試中選,賜錢三十萬。元豐間,集父承乾遺
 文以進,神宗嘉之,詔:「承乾父子以藝文儒學名於宗藩,宜有褒勸。」於是追封承乾為東平王,而賜克敦敕書獎諭。以宣州觀察使卒,贈開府儀同三司、和國公。



 德欽字丕從。淳化元年,授右屯衛將軍,四遷右羽林將軍。景德元年六月卒,年三十一。贈雲州觀察使,追封雲中侯。子承遵,西京作坊使。



 德潤字溫玉,頗好學,善為詩。淳化元年,始授右領軍衛將軍,四遷右羽林將軍。咸平六年二月卒,年三十九。贈應州觀察使,追封金城侯。



 德文字子
 矼,淳化初,授右監門衛將軍,累遷滑州觀察使、馮翊郡公。少好學,凡經史百家,手自抄撮,工為辭章。真宗以其刻勵如諸生,嘗因進見,戲呼之日「五秀才」,宮中由是悉稱之。德文本廷美第八子,其兄三人早卒,故德文於次為第五也。帝封泰山、祀汾陰、幸亳,德文必奏賦頌。帝每賜詩,輒令屬和。數言願得名士為師友,特命翰林學士楊億與之游。億卒,為詩十章悼之。天聖中,遷橫海軍節度觀察留後,拜昭武軍節度使,易感德、武勝二軍,加同
 中書門下平章事。仁宗嘗稱為「五相公」而不名。慶歷四年,宗室王者四人,以德文屬尊且賢,方漢東平王蒼,進封東平郡王,加兼侍中。德文雖老,嗜學不倦。晚被足疾,不能朝。六年,薨,年七十二。初得疾,仁宗臨視,親調藥飲之。及訃聞,復臨哭,贈太尉、中收令、申王,謚恭裕。子六人,承顯,以王後襲封康國公,官至昭化軍節度使。薨,年七十四,贈太尉、樂平郡王。



 德願字公謹,淳化元年,授右千牛衛大將軍,三進秩為左武衛大將軍。咸平二年閏三
 月卒,年二十四。贈涼州觀察使,追封姑臧侯。



 德存字安世,九歲授右千牛衛將軍,歷監門,至驍衛。從祠泰山,領獎州刺史。祀汾陰,以恩遷右羽林將軍。大中祥符四年六月卒,年三十。贈洮州觀察使,追封洮陽侯。子承衍,禮賓副使。



 太祖四子:長滕王德秀,次燕懿王德昭,次舒王德林,次秦康惠王德芳。德秀、德林皆早亡,徽宗時,追賜名及王封。



 燕懿王德昭字日新,母賀皇后。乾德二年出閣。故事,皇子出閣即封王。太祖以德昭沖年,欲其由漸而進,授貴州防御御。開寶六年,授興元尹、山南西道節度使、檢校太傅、同中書門下平章事,終太祖之世,竟不封以王爵。太宗太平興國元年,改京兆尹,移鎮永興,兼侍中,始封武功郡王。詔與齊王廷美自今朝會宜班宰相之上。三年二月,娶太子太傅王溥女,封韓國夫人。是冬郊祀,加檢校太尉。



 四年,從征幽州。軍中嘗夜驚,不知上所在,有
 謀立德昭者,上聞不悅。及歸,以北征不利,久不行太原之賞。德昭以為言,上大怒曰:「待汝自為之,賞未晚也!」德昭退而自刎。上聞驚悔,往抱其尸,大哭曰:「癡兒何至此邪!」贈中書令,追封魏王,賜謚,後改吳王,又改越王。德昭喜慍不形於色。真宗即位,贈太傅。乾興初,加贈太師。子五人:惟正,惟吉,惟固,惟忠,惟和。



 慶歷四年,詔封十王之後,以惟忠子從藹襲封穎國公,而惟吉子守巽以冀王後最長,與從藹同封。守巽官至和州防禦使,贈武成軍
 節度使、楚國公。從藹至齊州防禦使,贈武勝軍節度觀察留後、韓國公。守巽、從藹卒,以惟忠子從信襲封榮國公,官至雄州防禦使,贈保寧軍節度使、楚國公。從信卒,以惟忠之孫、從恪子世規襲封崇國公,官至右龍武大將軍、沂州防禦使以卒。守巽子世清,累官茂州防禦使。以本宮之長,得封申國公。熙寧中,坐上書請襲曾祖越懿王封不當,奪一官。既而議者是其說,乃遷越州觀察使,襲封越國公,進會稽郡王,至保信軍留後。愛諸弟,作
 棣萼會於邸中。會元豐升祔四後,受命告廟,方屬疾,自力就事,未幾薨。贈安化軍節度使、開府儀同三司、虢王,謚恭安。子令廓嗣,元符三年,改今封。



 先是,熙寧中,詔封楚康惠王之孫從式為安定郡王,奉太祖祀。及從式薨,乃以懿王曾孫世準襲封安定郡王。世準,從藹子也。為人內恕外嚴,無綺羅金玉之好,凡天子郊廟,必從祀。由金州觀察使拜保靜軍節度使。薨年六十八,贈開府儀同三司,追封成王。世開襲封。



 世開,從誨子、惟和孫也。七
 八歲,日誦萬言,既長,學問該治。事後母孝,撫孤侄如己子。宮官吳申為御史,薦其學行,命試學士院,累召不赴。神宗褒異之,召對便殿,論事甚眾。時宮僚有缺,不即請,而以他官攝,故私謁公行。宗女當嫁,皆富家大姓以貨取,不復事銓擇。世開悉言之,帝嘉納,欲以為宗正,固辭,乃進一官。以其所列著為令。官至奉國軍留後。薨,贈開府儀同三司,追封信王,謚獻敏。世雄嗣。



 世雄亦從藹子,少力學知名。熙寧中,詔宗子以材能自表見者,官長及
 學官以名上。世雄子令鑠在選中。嘗請營都宅以處疏屬,立三舍以訓學者。詔用其議,置兩京敦宗院,六宮各建學。徽宗即位,以世雄於太祖之宗最為行尊,拜崇信軍節度使,襲安定郡王,知大宗正事。崇寧四年薨,年七十五。贈太尉,追封淄王,謚恭憲。世福襲封。



 世福,從信子。官至集慶軍節度使。薨,贈儀王。令蕩襲爵。令蕩,秦康惠王曾孫也。



 惟正,天聖七年,以久病,帝欲尉安之,由保信軍節度觀察留後、樂安郡公特拜建寧軍節度使。卒,贈
 侍中,追封同安郡王,謚僖靖。無子,以弟惟忠子從讜為嗣,官至左龍武大將軍、溫州團練使。坐射殺親事官削官爵,幽之別宅。從讜少好學,以剛褊廢,遂自剄死。帝甚悼之。贈濟州防禦使、濟南侯。



 惟吉字國祥,母鄭國夫人陳氏。惟吉生甫彌月,太祖命輦至內廷,擇二女媼養視之,或中夜號啼,必自起撫抱。三歲,作弱弓輕矢,植金錢為的,俾之戲射,十發八中,帝甚奇之。五歲,日讀書誦詩。帝嘗射飛鳶,一發而中,惟吉從旁雀躍,喜甚,帝亦喜,鑄
 黃金為奇獸、瑞禽賜之。常乘小乘輿及小鞍鞁馬,命黃門擁抱,出入常從。太祖崩,惟吉裁六歲,晝夜哀號,孝章皇后慰諭再三,始進饘粥。太宗即位,猶在禁中,日侍中食。太平興國八年,始出居東宮,授左監門衛將軍,封平陽郡侯,加左驍衛大將軍,進封安定郡公。淳化四年,遷左羽林軍大將軍。至道二年,授閬州觀察使。凡邸第供億,車服賜與,皆與諸王埒,自餘王子不得偕也。真宗即位,授武信軍節度,加同平章事。時石保吉先為使相,詔
 惟吉班其上。大中祥符初,封泰山,以疾不從行,詔許疾愈馳詣行在。還頓鄆州,惟吉迎謁,上勞問再三,改感德軍節度。明年,疾復作,上屢臨省之,親視灼艾,日給御膳,為營佛事。三年五月薨,時年四十五。廢朝五日,贈中書令,追封南陽郡王,謚康孝。



 惟吉好學,善屬文,性至孝。孝章皇后撫養備至,親為櫛沐。咸平初,以太祖孝章畫像、服玩、器用賜惟吉,歲時奠享,哀慕甚至。每誦《詩》至《蓼莪篇》,涕泗交下,宗室推其賢孝。雅善草隸飛白,真宗次為
 七卷,禦制序,命藏秘閣。其子守節,以父所書《真草千文》以獻,詔書褒答,仍付史館。追贈太尉,明道二年封冀王。子守節、守約、守巽、守度、守廉、守康。



 守節,累遷彰化軍節度觀察留後、同知大宗正事。卒贈鎮江軍節度使,追封丹陽郡王,謚僖穆。子世永、世延。世永,襲邢國公,官至鎮南軍留後,熙寧元年薨,贈昭信軍節度使、南康郡王,謚修孝。世延,終右武衛大將軍、絳州防禦使,贈武寧軍節度觀察留後、彭城郡公。



 寧約,終內園使、康州刺史,贈沂
 州團練使。子世靜、世長。世靜,至左武衛大將軍、均州防禦使,卒贈鎮海軍節度觀察留後、北海郡公。世長,終左武衛大將軍、解州防禦使,贈張信軍節度觀察留後、濟陽郡公。守巽及其子世清,事見上。守度,終左領軍衛大將軍、英州團練使,贈廣州觀察使、盧江侯。守廉,終供備庫副使,贈內藏庫使。守康,至供奉官。



 惟固字宗乾,本名元扆,太平興國八年,改賜名授左千牛衛將軍。是冬卒。



 惟忠安令德,初名文起,太平興國八年賜今名。授右千
 牛衛將軍,四遷右龍武軍。直宗即位,改右千牛衛大將軍。大中祥符二年,進左監門衛大將軍、敘州刺史。五年,進昌州團練使。八年卒,贈鄂州觀察使,追封江夏侯。明道二年,加贈彰化軍節度使,追封舒國公。子從恪、從藹、從秉、從穎、從謹、從質、從信、從讜。



 從恪,累官西染院使,卒,贈磁州刺史、東萊侯。子世規,襲封崇國公。從藹,終左衛大將軍、齊州防禦使,贈武勝軍節度觀察留後,追封韓國公。子世豐,終太子右衛率,追贈進士及第。世準、世雄,
 並安定郡王。從信,封榮國公,官至雄州防禦使,贈保寧軍節度使、楚國公,謚安僖。子世福,襲安定郡王。從秉、從穎、從謹,並禮賓使。從質,內殿崇班。從讜,出繼惟正。



 惟和字子禮,端拱元年,授右武衛將軍,歷右驍衛、神武龍武軍、右衛將軍。大中祥符元年,領澄州刺史。四年,遷右千牛衛大將軍。六年,卒,年三十六。贈汝州防禦使、臨汝侯。明道二年,加贈永清軍節度觀察留後,追封清源郡公。



 惟和雅好學,為詩頗清麗,工筆札優游典籍,以禮法自
 居,宗室推重。嘗和禦制詩,上稱其有理致。及卒,上謂宰相王旦等曰「惟和好文力學,加之謹願,皇族之秀也,不幸短命!」嗟悼久之,至於泣下。錄其稿二十二軸,上親制序,藏於秘閣。子從審、從誨。



 從審,終復州防禦使,贈寧國軍節度觀察留後、宣城郡公。嘗坐與人奸除名,已而復官。從誨,終左金吾衛大將軍、臺州團練使,贈襄州觀察使、襄陽侯。子世開,安定郡王,事見上。



 紹興元年,詔曰:「太祖皇帝創業垂統,德被萬世。神祖詔封子孫一人為安
 定郡王,世世勿絕。今其封不舉,朕甚憫之。有司其上合封人名,遵故事施行。」時燕、秦二王後爭襲封,禮部員外郎王居正上言:「燕王親,太祖長子,其後漢襲封。」議遂定。自紹興至嘉定,襲封者十五人,惟令畤、令懬、令詪、令衿跡頗著,餘皆繼嗣,娖娖無足稱。



 令畤字德麟,燕懿王玄孫也,蚤以才敏聞。無祐六年,簽書穎州公事。時蘇軾為守,愛其才,因薦於朝。宣仁太后曰:「宗室聰明者豈少哉?顧德行何如耳。」竟不許。軾被竄,令畤坐交通軾罰金。已
 而附內侍譚稹以進。紹興初,官至右朝請大夫。呂頤浩請以令畤主行在大宗正司,帝命易環衛官。頤浩言:「令畤讀書能文,恐不須易。」帝曰:「令畤昔事譚稹,頗違清議。」改右監門衛大將軍、榮州防禦使,權知行在大宗正事。遷洪州觀察使,襲封安定郡王。尋遷寧遠軍承宣使,同知行在大宗正事。四年薨,貧無以為殮,帝命戶部賜銀絹,贈開府儀同三司。



 令矼,紹興五年,由邵武軍兵馬都監襲封,授華州觀察使,尋除同知大宗正事。逾年薨。



 令
 懬字深之。初,懿王生昌州團練使惟忠,惟忠生楚安僖王從信,從信生益公世逢,世逢生令懬,授右班殿直,遷東頭供奉官,累監州縣場庫。監司薛昂薦其才,易資承事郎,調穎州簽判,歷綿州通判,累知蜀州、閬州、慶源府,召除衛尉少卿,擢秘閣修撰,再知慶源府。建炎二年,分西外宗子於泰州,命令懬知西外宗正事,除御營使司參贊軍事,挈宗子避地福州,因置司焉。元懿太子薨,帝命令懬選藝祖後得三四人,寺擢集英殿修撰,知南外
 宗正。再選宗子,得伯琮、伯浩養宮中,後選得伯玖,性亦聰惠。高宗喜,轉令懬知泉州,尋與祠以歸。令矼薨,令懬改閬州觀察使,襲封,除同知大宗正事。逾年,授鎮東軍承宣使,再遷保平軍節度使。紹興十三年薨,年七十五。贈少師,後追封惠王,謚襄靖。子子游,官至湖北提刑,用戶部侍郎王俁薦,加直秘閣。會建寧節度使士雪刀知南外宗正司,以事去官,言者請擇宗室文臣之廉正者代之,遂以命子游。西、南外宗官用文臣,自子游始。



 令詪,字
 君序,以父任補右班殿直。政和中,遷成忠郎,召試,授從事郎。宣和二年,以貢士試舍選合格,授宣教郎,調信州永豐縣丞。中興初,累遷福州運判,兼提點刑獄公事。秦檜方柄用,安定郡王絕封者十餘年;檜死,次令衿當封,適以事被拘,遂命令詪襲封。已而令詪以爵遜令衿,乃升令詪秘閣修撰,知臺州,移知紹興府,召權戶部侍郎,令嚴、饒二州鑄錢局。先是,諸州錢監兵匠多缺不補,積其衣糧,號三分缺額錢,令詪請以其錢付諸鹽,省朝廷
 降銅本錢。又建議州縣賣官田計所入高下,守令進秩減磨勘有差;州縣義倉多紅腐,請歲出三之一以易新粟;水旱為災,檢放不及七分處所,即許振恤:皆從之。令衿薨,令詪由崇慶軍承宣使再襲封。隆興初,除同知大宗正事,奏減生日支賜並郊祀賞給,以助軍興。詔褒之。遷敷文閣直學士,特授左中大夫、知紹興府,引疾乞祠以歸,尋薨,年六十八。令詪蒞事明敏有風採,然在廣東日,嘗與副使章茇不協,陰中以法,陷茇於死,世以此少
 之。



 今衿,嘉孝穆公世□失子也。博學有能文聲,中大觀二年舍選。靖康初,為軍器少監。言事忤旨,奪官。紹興七年,以都官員外郎召。張浚罷,令衿請對留浚,言官石公以揆論令衿阿大臣,復罷。久之,以事抵臨安,中丞李文會劾令衿「昔為大臣緩頰,今復奔走請托。」詔送吏部。吏部直令衿,奏除德安府通判,遷知泉州。泉屬邑有隱士秦系故廬,唐相姜公輔葬邑旁,令衿建堂合祠之,郡人感其化。歸寓三衢。嘗會賓客觀秦檜家廟記,口誦「君子之澤,
 五世而斬」之句。通守汪召錫,檜兄婿也,頗疑令衿,諷教官莫汲訴令衿論日月無光,謗訕朝政。侍御史董德元承風旨劾之,誣以贓私。詔下令衿獄,案驗無狀,乃論令衿謗訕不遜,追一官勒停,令南外宗正司拘之。檜除召錫湖南提舉以報之,銜令衿,必欲置死地。初,趙鼎之子汾歸過衢,令衿贐之,侍御史徐哲希檜旨,誣令衿與汾有密謀,伺朝廷機事。捕汾下大理寺,俾汾自誣與張浚、李光等謀逆,而令衿預焉。獄上,檜病不能省,乃獲免。檜
 死,復爵。二十六年,授明州觀察使襲封。引疾乞奉燕王祠,許之。尋加慶遠軍承宣使。二十八年薨,贈開府儀同三司。



 令話,建炎末,為右武衛大將軍、信州防禦使。熙寧初,首封秦王孫從式,已而更封燕王曾孫世清。宣和中,又封秦王元孫令蕩。令蕩卒,令庇年最長,禮官以為小宗不當封。紹興元年六月,令話得襲封,授寧州觀察使。二年七月薨,贈開府儀同三司。



 令德,乾道元年為武德郎。時安定郡王令詪換文階,大宗正司奏令德授定武
 軍承宣使,襲封。令德貧,幾不能出蜀。七年,令德薨,令憉當封,以沉湎聲色,不任襲。詔武德郎令抬襲封,除金州觀察使。令抬薨,時秦王後無當襲者,武翼郎子手東屬燕王後,年又最長,得襲封。子手東薨,九年九月,忠訓郎子肜襲,授容州觀察使。紹熙二年薨,年八十餘。慶元元年十月,忠翊郎子恭襲,授利州觀察使。子恭薨,嘉定二年七月,子覿襲,授金州觀察使。四年十一月,伯栩襲,授宣州觀察使。嘉定元年十月,伯柷襲,授福州觀察使。八年十
 一月,伯澤襲,授潭州觀察使。



 秦康惠王德芳,開寶九年出閣,授貴州防禦使。太平興國元年,授興元尹、山南西道節度使、同平章事。三年冬,加檢校太尉。六年三月,寢疾薨,年二十三。車駕臨哭,廢朝五日。贈中書令、岐王及謚。後加贈太師,改楚王。子三人:惟敘、惟憲、惟能。



 慶歷四年,詔封十王之後,以惟敘子從照封安國公,終左金吾衛大將軍、歸州團練使。贈同州觀察使、齊國公。從照卒,以惟能子從古封安國公,終
 延州觀察使,贈保靜軍節度使、同中書門下平章事、楚國公,謚惠恪。從古卒,惟憲子從式襲封舒國公。



 神宗即位,謂創業垂統,實自太祖,顧無以稱。乃下詔令中書門下考太祖之籍,以屬近而行尊者一人,裂土地而王之。使常從獻於郊廟,世世勿復絕。於是有司推擇,以從式應詔,封安定郡王,終保康軍節度使,贈同中書門下平章事,追封榮王,謚安僖。從式既薨,詔以越王曾孫世準襲封安定郡王,而以從式子世恩襲爵為楚國公,主楚
 王德芳之祀。遷楚州防禦使,卒贈奉國軍節度使,謚良僖。徽宗即位,改封楚王為秦王。



 惟敘字懋功,性純謹,頗好學。端拱初,授左武衛將軍,四遷左衛將軍,領勤州刺史。大中祥符四年,從祀汾陰,拜左千牛衛大將軍。八月,卒,年三十五。贈懷州防禦使,追封河內侯。明道二年,加贈保靜軍節度觀察留後、高平郡公。子從照,封安國公。從溥,至右侍禁內殿崇班。



 惟憲字有則,美豐儀,少頗縱肆,長修謹,善射,好吟詠,多讀道書。端拱初,授左屯衛將
 軍,累遷左羽林將軍、領演州刺史,加左衛大將軍、領賀州團練使,真拜資州團練使。大中祥符九年五月卒,年三十八。贈安德軍節度使兼侍中、英國公。子從式,始封安定郡王,事見上。從演,禮賓副使。從戎、從戒、從湜,並內殿崇班。從賁,供奉官。



 惟能字若拙。端拱初,授右屯衛將軍,累遷右神武軍將軍。大中祥符元年五月卒,年三十。贈蔡州防禦使、張掖侯。明道二年,加贈集慶軍節度觀察留後、南康郡公。子從古,襲安國公。從善,內殿承制。從
 贄,崇班。



 安僖秀王子戴,秦康惠王之後,高宗族兄也。康惠生英國公惟憲,惟憲生新興侯從鬱,從鬱生華陰侯世將,世將生東頭供奉官令儈,令儈生子戴。宣和元年,舍試合格,調嘉興丞。是年,子伯琮生,後被選入宮,是為孝宗。



 子戴召赴都堂審察,改宣教郎,通判湖州,尋除直秘閣,賜五品服。孝宗既封建國公,就傅,子戴召對言:「宗室之寓於外者,當聚居官舍,選尊長鈐束之。年未十五附入州
 小學,十五入大學,許依進士就舉,未出官者亦許入學聽讀,及一年,聽參選。」高宗納其說。遷朝奉郎、秘閣修撰,知處州。已而乞祠,許之。累官左朝奉大夫。紹興十三年秋致仕,明年春,卒於秀州。時孝宗為普安郡王,疑所服,詔侍從、臺諫議。秦熹等請解官如南班故事,普安亦自請持服,許之。及普安建節,子戴以恩贈太子少師。既為太子,加贈太師、中書令,封秀王,謚安僖。配張氏,封王夫人。



 孝宗受禪,稱皇伯,園廟之制未備。紹熙元年,始即湖
 州秀園立廟,奉神主,建祠臨安府,以藏神貌,如濮王故事。仍班偉



 嗣秀王伯圭字禹錫,孝宗同母兄也。初,以恩補將仕郎,調秀州華亭尉,累官至浙西提刑司干辦公事,除明州添差通判。孝宗受禪,上皇詔除集英殿修撰、知臺州。



 伯圭在郡,頗著政績,除敷文閣待制,改知明州,充沿海制置使。蕃商死境內,遺貲巨萬,吏請沒入,伯圭不可,戒其徒護喪及貲以歸。升敷文閣直學士,以憂去,服闋,再知明州。新學宮,命宗子入學,閑以規矩。詔徙戍
 定海兵於許浦。伯圭奏:「定海當控扼之沖,不可撤備,請摘制司軍以實其地。」從之。



 海寇猖獗,伯圭遣人諭降其豪葛明,又遣明禽其黨倪德。二人素號桀黠,伯圭悉撫而用之,賊黨遂散。以功進一官,累升顯謨閣、龍圖閣學士。在郡十年,政寬和,浚湖陂,均水利,辨冤獄。嘗獲鑄銅者,不忍置諸法,諭令易業,民由是無再犯。



 淳熙三年,授安德軍節度使,尋加開府儀同三司,充萬壽觀使。朝德壽宮,上皇賜玉帶,加少保,封滎陽郡王。高宗崩,入臨,充
 攢宮總護使,除少傅。光宗即位,升少師。逾年召見,遷太保,封嗣秀王,賜甲第於安僖祠側。



 臣僚上言:「治平中追崇濮邸,王子孫幾二十人,皆自環衛序遷其官。今居南班者止師夔一人,非所以強本支而固盤石也。前未建秀邸時,欲賦以祿,則不免責以吏事;今已建邸,而猶責吏事,他日或不免於議。治則傷恩,不則廢法,曷歸之南班,俾無吏責而享富貴。」遂詔伯圭諸子得換班。



 紹熙二年,除判大宗正事,建請別立宗學,以教宗子。超拜太師,
 免奉朝請。尋兼崇信軍節度使,賜第還湖州,尋薨於家。訃聞,帝為輟朝三日,追封崇王,謚憲靖。



 伯圭性謙謹,不以近屬自居。每日見,行家人禮,雖宴私隆洽,執臣節愈恭。一日,孝宗問潛龍時事,伯圭辭曰:「臣老矣,不復能記。」問至再三,終不言。帝笑曰:「何太謹也。」益愛重之。嘗欲廣其居,並湖為復閣,有司既度材矣,伯圭固辭而止。阜陵成,遷中書令,凡五讓。寧宗嘉其志,詔別議褒崇之禮,贈贊拜不名,肩輿至殿門。子九人:師夔、師揆、師垂、師離、師
 禹、師皋、師巖、師彌、師貢。



 師夔字汝一,初以祖恩補官,調太平州蕪湖簿。隆興元年,改右承務郎,歷臺州、秀州通判,直秘閣。尋知徽州,新學舍,進直徽猷閣,知湖州。時歸附從軍而廩於湖者眾,不能給,師夔請增廩,仍別給僦屋錢,以安其心。帝稱善,詔諸郡行之。除直龍圖閣,遷浙西提刑,改江東運判。



 建康務場往往奪民利,為害滋甚,師夔首罷之。守臣以郡計所資,詣師夔請復舊,不從。池州軍帥霍政與守臣交上書相攻,詔師夔究曲直。政密
 遣人求庇,師夔斥之,具言狀,政坐罷去。



 改秘閣修撰、知明州兼沿海制置使,加敷文閣待制,轉永慶軍承宣使。紹熙元年,侍父入覲,除興寧軍節度使。寧宗即位,加檢校少保,充阜陵橋道頓遞使。阜陵成,遷開府儀同三司。侍父歸,父薨未逾月,師夔亦卒,年六十一。贈少師,追封新安郡王。



 師揆字符輔,初補右承務郎奉祠。除添差湖州簽判,改婺州通判,加直秘閣。守臣韓元吉薦其材,上以問史浩,浩言其聰爽可任。召對,除江東提舉。奏免失
 陷常平人毋責償。改淮南漕,尋遷淮西提刑兼提舉,領屯田事。奏以荒圩給軍士,其屯田為民世業者勿奪,從之。及代去,吏請獻羨錢二十萬,師揆曰:「後將病民矣。」除直秘閣,改江東轉運副使,加秘閣修撰,知明州。



 紹熙元年,授觀察使。寧宗即位,除奉國軍承宣使,尋升節度使。召見,賜肩輿,超檢校太保、開府儀同三司,充萬壽觀使,襲封。開禧元年奉朝請,嘉定七年薨,贈太傅,追封澧王,謚恭惠。



 弟師禹,由保康軍節度使除開府儀同三司,襲
 封。十六年,薨,贈太傅,追封和王,謚端肅



\end{pinyinscope}