\article{列傳第一百九}

\begin{pinyinscope}

 苗授子履王君萬子贍張守約王文鬱周永清劉紹能王光祖李浩和斌子詵劉仲武曲珍劉闃
 郭成賈巖張整張蘊王恩楊應詢趙隆



 苗授,字授之,潞州人。父京,慶歷中,以死守麟州抗元昊者也。少從胡翼之學,補國子生,以蔭至供備庫副使。



 王韶取鎮洮,授為先鋒,破香子城,拔河府。羌雖敗,氣尚銳,輒圍香子以迎歸師。韶遣將田瓊救之,瓊死,乃簡騎五百屬授,授奮擊敗之。休士二日,羌復要於架麻平,注矢
 如雨,眾懼,授令曰:「第進毋恐!氈牌數百且至。」行前者傳呼,羌驚亂。力戰數十,斬首四千級。又破之於牛精谷,取珂諾城,盡得河湟地。



 知德順軍,三遷西上閣門使。鬼章寇河州,詔授往,一戰克撒宗,論功第一,遂知州事。加四方館使、榮州刺史。從燕達取銀川,降木征,獻之京師,加引進使、果州團練使、涇原都鈐轄。



 召使契丹,神宗勞之曰:「曩香子之役,非汝以寡擊眾,幾敗吾事。」以為秦鳳副總管,徙熙河,復知河州。副李憲討生羌於露骨山,斬首
 萬級,獲其大酋泠雞樸,羌族十萬七千帳內附,威震洮西。拜昌州團練使、龍神衛四廂都指揮使,徙知雄州、熙州。



 元豐西討,授出古渭取定西,蕩禹臧花麻諸族,降戶五萬。城蘭州,遇賊數萬於女遮谷,登山逆戰,敗退伏壘中,半夜遁去。授逾天都山,焚南牟,屯沒煙,凡師行百日,轉鬥千里,始入塞。



 授遇事持議不茍合。初在德順,或議城籛南,授曰:「地阻大河,糧道不濟,非萬全計也。」役即止。師征靈武,詔令援高遵裕,即條上進退利害甚切。歷進
 步軍副都指揮使、威武軍節度觀察留後。元祐三年,遷武泰軍節度使、殿前副都指揮使。逾歲,以保康節度知潞州,提舉上清太平宮,復使殿前,薨,年六十七,贈開府儀同三司,謚曰莊敏。子履。



 履束發從戎。授之降木徵也,履護送至京,得閣門祗候。歷熙、延、渭、秦四路鈐轄,知鎮戎軍。及其父時,已官四方館使、吉州防禦使矣,以事竄房州,起為西上閣門副使、熙河都監。又責右清道率府率,監峽州酒稅。元符初,悉
 還其官,以熙河蘭會都鈐轄知蘭州。



 詔同王贍取青唐,與姚雄合兵討峗羌籛羅結。贍將李忠戰敗,羅結大集眾,宣言欲圍青唐。履、雄將至,羌列陣以待,勢甚盛。履叱軍士納弓於鞬,拔刀而入。羌怙巢穴殊死鬥,梟將陳迪、王亨輩皆反走,履獨駐馬不動。有酋青袍白馬突而前,手劍擊履,帳下王拱以弓格之,僅免。復繞出履背,欲斷軍為二,別將高永年率所部力戰數十合,羌退,乘勝圍蘭宗堡,弗能拔。日暮,收兵入營,羌宵潰。明日,縱兵四掠,
 焚其族帳而還。



 既而阿章叛,詔履與種樸過河討蕩,辭以兵少,樸遂陷。錄履前功,擢龍神衛四廂都指揮使、成州團練使,知慶州,徙渭州,進捧日、天武都指揮使。是後史失其傳。子傅,在《叛臣傳》。



 王君萬,秦州寧遠人。以殿侍為秦鳳指揮使。王韶開邊,青唐大酋俞龍珂歸國,獨別羌新羅結不從。經略使韓縝期諸將一月取之。君萬詐為獵者,逐禽至其居,稍相親狎,與同獵,乘間撾之,墜馬,斬首馳歸以獻。甫及一月,
 積功得閣門祗候。



 王師定武勝,首領藥廝逋邀劫於闐貢物,帥師討焉。君萬出南山,履險略地。羌潛伏山谷間,忽一騎躍出,橫矛將及,君萬亟側身避之,回首奮擊,斬以徇。其眾驚號,相率聽命,所斬乃藥廝逋也。復破北關、南市,功最多,擢熙河路鈐轄,進領英州刺史、達州團練使,賜絹五百。



 洮西羌叛,圍河州,君萬請於王韶,以為南撒宗城小而堅,強勇所聚,若並兵破之,圍當自解。韶用其計,圍果解。累官客省使,為副總管。坐貸結糴錢數萬
 緡,為轉運使孫迥所糾,貶秩一等。討西山、鐵城有功,復故官職。君萬怨孫迥,使番官木丹訟之,鞫於秦、隴,又貶為鳳翔鈐轄,籍家貲償逋,遂以憤卒。子贍。



 贍始因李憲以進。立戰功,積官至皇城使,領開州團練使。元符中,知河州。熙帥鐘傅以冒白草原賞,獄治於秦,詔轉運使張詢諭諸將得自首。贍具伏詐增首級,因說詢云:「青唐人有叛瞎征意,可取也。」詢信之,即具奏言已令贍結約起兵。誓宗與輔臣罪其狂妄專輒,亟罷詢,而
 命孫迥究實。獄上,奪贍十一官,猶令領州。



 贍欲以功贖過,乃密畫取青唐之策,遣客詣章惇言狀。惇下其事於孫路,路以為可取。贍遂引兵趣邈川。路知贍狡獪難制,使總管王愍統軍,而以贍副。贍為前鋒度河,先下隴朱黑城。忌愍分其功,紿之曰:「晨食畢乃發。」愍信之。夜半,贍忽傳發。平明,入邈川,據府庫,徑上捷書,不以白軍府。愍過午始至,以事訴於路,路亦怒,顓以兵柄付愍,而留贍屯邈川。



 宗哥酋舍欽腳求內附,贍遣裨將王詠享五千
 騎赴之。既入,而諸羌變,詠馳書告急,王厚使高永年救之,乃免。贍與愍交訟,又訴路指畫相違。惇主贍而不直路,曰:「首謀者贍也,路欲掩其功,故抑贍。」乃徙路河南,罷愍統制,以胡宗回為帥。



 時瞎徵已來降,青唐戍將惟心牟欽氈父子百餘人在。贍不即取,二羌遂迎溪巴溫之子隴拶入守。始,孫路乞先全邈川及河南北諸城,然後進師。贍怨路,因言青唐不煩大兵可下,而路逗遛失機會。暨宗回至,乃云夏人謀攻邈川,當為守備,青唐未可
 取。宗回責其反復,日夜督出師,遣使威以軍法,且聲言欲使王愍代將。贍懼,急進攻隴拶及心牟等,皆出降。贍入據其城。詔建為鄯州,進贍四方館使、榮州防禦使、知州事。黃履謂賞薄,乃拜維州團練使,為路鈐轄。



 贍縱所部剽奪,羌眾攜貳,心牟等結諸族帳謀復青唐,其在山南者先發。贍遣將李賓領二千騎掩襲心牟以下,自守西城與羌斗。賓逾南山入保敦谷討蕩,羌戰敗奔北,四山皆空。贍戮心牟等九人,悉捕斬城中羌,積級如山。



 初,
 贍諷諸酋籍勝兵者涅其臂,無應者。籛羅結請歸帥本路為唱,贍聽之去,遂嘯集外叛,以數千人圍邈川,夏眾十萬助之,城中危甚。苗履、姚雄來援,圍始解。已而王吉、魏釗、種樸相繼敗沒,將士奪氣。書聞,帝震駭,於是轉運使李譓、秦希甫劾贍盜取二城財物,因此致變;又殺心牟欽氈以滅口。曾布言贍創造事端以生邊害,萬死不塞責。詔貶右千牛將軍,房州安置。言者論之不已,熙河又奏青唐諸族怨贍入骨髓,日圖報復,樞密院乞斬贍
 以謝一方。詔配昌化軍,行至穰縣而縊。



 崇寧初,蔡京入相,錢遹訟贍功;及王厚平鄯、廓,於是追贈保平軍節度觀察留後,除其子玨通事舍人。



 張守約,字希參,濮州人。以蔭主原州截原砦,招羌酋水令逋等十七族萬一千帳。為廣南走馬承受公事,當儂寇之後,二年四詣闕,陳南方利害,皆見納用,歐陽修薦其有智略、知邊事,擢知融州。峒將吳儂恃險為邊患,捕誅之。修復薦守約可任將帥,為定州路駐泊都監,徙秦
 鳳。居職六年,括生羌隱土千頃以募射手,築硤石堡、甘谷城,第功最多。



 夏人萬騎來寇,守約適巡邊,與之遇,不解鞍,簡兵五百逆戰,眾寡不侔,勢小卻。夏人張兩翼來,守約挺身立陣前,自節金鼓,發強弩殪其酋,敵遂退。



 神宗開拓熙河,召問曰:「王韶能辦事否?」對曰:「以天威臨之,當無不濟;但董氈忠勤效順,恐不宜侵逼。」因請名古渭為軍,以根本隴右。帝從之,建為通遠軍。加通事舍人、熙河鈐轄,仍統秦鳳羌兵駐通遠。



 河州羌率眾三萬屯於
 敦波,欲復舊地,守約度洮水擊破之,取窖粟食軍。羌老弱畜產走南山,左右欲邀之,云可獲萬萬。守約曰:「彼非敢迎戰,逃死耳,輒出者斬!」鬼章圍岷州,守約提敢死士鳴鼓張幟高山上,賊驚顧而遁,遂知岷州,降其首領千七百人。遷西上閣門使、知鎮戎軍,徙環州。



 慕家族頡佷難制,搖動種落,勒兵討擒之,餘遁入夏國。守約駐師境上,檄取不置,居數日,械以來,斬於市。



 從征靈武,至清遠軍,言於高遵裕曰:「此去靈州不三百里,用以前軍先出,
 直搗其城。今夏人以一方之力,應五路之師,橫山無人,靈州城中惟僧道數百。若裹十日糧,疾馳三日可至,軍無事矣。」又勸高遵裕令士眾護糧餉,以防抄掠,不聽,果以敗還。守約有捍海南咸平之功,亦不錄。



 進為環慶都鈐轄、知邠州,徙涇原、鄜延、秦鳳副總管,領康州刺史。夏人十萬屯南牟,畏其名,引去。知涇州,涇水善暴城,每春必增治堤堰,費不貲。適歲饑,罷其役。或曰:「如水害何?」守約曰:「歉歲勞民,甚於河患,吾且徐圖之。」河神祠故在南
 壖,禱而遷諸北,以殺河怒。一夕雷雨,明日,河徙而南,其北遂為沙磧。以龍、神衛四廂都指揮使召還,道卒,年七十五。



 守約典七州,皆有惠愛可紀。神宗嘗謂武臣可任者,以燕達、劉昌祚、姚麟、王崇極、劉舜卿等對,其後皆為名將,時稱知人。



 王文鬱,字周卿,麟州新秦人。以供奉官為府州巡檢。韓琦薦其材,加閣門祗候、麟府駐泊都監。



 熙寧討夏國,文鬱敗之吐渾河。其將香崖夜遣使以劍為信,欲舉眾降,
 許之。旦而至,與偕行,眾情忽變,噪以出。文鬱擊之,追奔二十里。據險大戰,矢下如雨,文鬱徐引度河,謂吏士曰:「前追強敵,後背天險,韓信驅市人且破趙,況爾曹皆百戰驍勇邪?」士感奮進擊,夏人大潰,降其眾二千。遷通事舍人。夏人逾屈野河,掠塞上,文鬱追至長城板,盡奪所掠而還。



 神宗召見,問曰:「向者招納香崖,群議不一,其為朕言之。」對曰:「此乃致敵上策,恨未能,多爾。並邊生羌善馳突,識鄉導,儻能撫柔之,所謂以外夷而攻外夷也。」帝
 於是決意招納,多獲其用。知文鬱善左射,並招其子弟閱肄殿庭,文鬱九發八中,詔官其二子。



 知鎮戎、德順軍,預定洮、河,遷左騏驥副使、知麟州。夏眾踐稼,襲敗之,部使者劾為生事,奪郡印。



 未幾,為熙河將。李憲討靈武,文鬱得羌戶萬餘,遷路鈐轄。夏人圍蘭州,已奪兩關門,文鬱募死士夜縋而下,持短兵突賊,即掃營去。擢東上閣門使、知蘭州。諜知夏人將大入,清野以俟,果舉國趨皋蘭,文鬱乘城御之,殺傷如積,圍九日而解,收其尸為京
 觀,加榮州團練使,以捧日、天武都指揮使為副都總管,以殿前都虞候知河州。築安西城、金城關,進秦州防禦、冀州觀察使。卒,年六十六。



 周永清字肅之。世家靈州,州陷,祖美歸京師。永清以蔭從仕,宰相寵籍言其忠勇,加閣門祗候。押時服賜夏國,至宥州,夏人受賜不跪,詰之,恐而跪。遷通事舍人、渭州鈐轄。渭兵勁而陳伍不講,永清訓以李靖法。帥蔡挺嘉其整,圖上之,詔推於諸道。



 知德順軍,夏眾入寇,擊擒其
 酋呂效忠。又募勇士夜馳百里,搗賊巢穴,斬首三百級,俘數千人,獲橐駝、甲馬萬計。城中無知者。並砦禁地三百里,盜耕不可禁,永清拓籍數千頃,置射士二千,聲聞敵廷。降者引入帳下,待之不疑。多得其死力。



 徙秦鳳鈐轄、河北沿邊安撫副使、知代州。契丹無名求地,朝廷命韓縝分畫,永清貳焉,入對言:「疆境不可輕與人,臣職守土,不願行。」固遣之,復上章陳利害,竟以母病辭。歷高陽關、定州、涇原路鈐轄,知涇州、保州,又為定州路副總管,
 終東上閣門使。



 劉紹能字及之,保安軍人。世為諸族巡檢,父懷忠,官內殿崇班、閣門祗候。元昊叛,厚以金幣及王爵招之,懷忠毀印斬使,洎入寇,力戰以死。錄紹能右班殿直,賜以名,為軍北巡檢。擊破夏右樞密院黨移賞糧數萬眾於順寧。夏人圍大順城,紹能為軍鋒,毀其柵,至奈王川,邀擊於長城嶺,熙寧中,又敗夏人於破囉川,皆策功最。累遷洛苑使、英州刺史、鄜延兵馬都監。舊制,內屬者不與漢
 官齒,至是,悉如之,仍以其子襲故職。



 元豐西討,召詣闕,神宗訪以計,對曰:「師旅遠征,儲偫不繼為大患。若俟西成後,因糧深入,乃可以得志。」帝以為然,命統兩軍進討。紹能世世邊將,為敵所忌,每設疑以間之。帝獨明其不然,手詔云:「紹能戰功最多,忠勇第一,此必夏人畏忌,為間害之計耳。」紹能捧詔感泣。嘗坐讒逮對,按驗卒無實。守邊圉四十七年,大小五十戰,以皇城使、簡州團練使卒。



 王光祖,字君俞,開封人。父珪,為涇原勇將,號「王鐵鞭」,戰死好水川。錄光祖為供奉官、閣門祗候。



 熙寧中,同提點河北刑獄,改沿邊安撫都監,進副使。界河巡檢趙用擾北邊,契丹以兵數萬壓境,造浮橋,如欲度者。光祖在舟中,對其眾盡徹戶牖。或謂:「契丹方陣,而以單舟臨之,如不測何?」光祖曰:「彼所顧者,信誓也;其來,欲得趙用耳。避之則勢張,吾死不足塞責。」已而契丹欲相與言,光祖即命子襄往。兵刃四合,然語唯在用,襄隨機折塞之。其將
 蕭禧遽揮兵去,且邀襄食,付所戴青羅泥金笠以為信,即上之。時已有詔罷光祖矣。吳充曰:「向非光祖以身對壘,又使子冒白刃取從約,則事未可知。宜賞而黜,何以示懲勸?」乃除真定鈐轄。



 徙梓夔。渝獠叛,詔熊本安撫,而命內藏庫使楊萬、成都鈐轄賈昌言、梓夔都監王宣與光祖同致討,皆受本節度。本疑光祖不為用,分三道進師,使光祖將後軍,出黃沙坎。比發,日已暮,士以杖索塗,相挽而前,夜半,抵絕頂。質明,獠望見,大駭,一
 鼓而潰。萬等困於松溪,又亟往援。出石門,敓其險,促黔兵先登襲賊,賊舍去。光祖夜泊松嶺上,旦始遇萬等,與俱還。本愧謝,上其功第一。



 吐蕃圍茂州,光祖領兵三千,會王中正破雞宗關,賊據石鼓村,扼其半道。中正召諸將問計,光祖獨請行。既抵石鼓,擇銳兵分襲吐蕃背,出其不意,皆驚遁,遂會中正於茂。



 瀘夷乞弟殺王宣,詔從韓存寶討之,軍於梅嶺。夷數萬眾出駐落個棧,欲老我師。霖雨不止,光祖勸存寶早決戰。不聽。林廣至,復從征,蕩其巢窟。
 積功至四方館使、知瀘州。置瀘南安撫使,俾兼領,邊事聽顓決。遷客省使、嘉州刺史。歷涇原、河東、定州路副總管,卒。



 李浩,字直夫,家本綏州,徙西河。浩務學,通兵法,以父定蔭,從軍破儂智高。韓絳城囉兀,領兵戰賞堡嶺川,殺大首領訛革多移,斬首千三百餘級。積官供備庫副使、廣西都監。



 裒西北疆事著《安邊策》,謁王安石。安石言之神宗,召對,改管幹麟府兵馬。未行,又從章惇於南江,引兵
 由三路屯鎮江,入遂州,討舒光貴,破盈口柵,下天府,會於洽州,入懿州。蠻酋田元猛、元哲合狤狑拒官軍,浩分兵擊之,殺狤狑,降元猛、元哲,遂城懿州。進討黔江蠻,復城黔江。惇上其功,謂不當與他將比,擢引進副使、熙河鈐轄。



 李憲討山後羌,浩將右軍至合龍嶺會戰,遣降羌乞啀輕騎突敵帳,俘其酋冷雞樸、李密撒,馘三千。遷東上閣門使,為副總管、知河州、安撫洮西。五路大舉,浩將前軍,復蘭州。遷引進使、隴州防禦使、知蘭州兼熙河、涇
 原安撫副使。坐西關失守及報上不實,再貶秩。旋以戰吃囉、瓦井連立功,復之。



 哲宗即位,拜忠州防禦使、捧日天武都指揮使、馬軍都虞候,進黔州觀察使,歷鄜延、太原、永興、環慶路副都總管,再知蘭州。卒,贈安化軍留後。



 和斌,字勝之,濮州鄄城人。選隸散直,為德順軍指揮使,凡五年,數捍敵,被重創十餘。知軍事劉兼濟以兄平敗沒,執送京師,並逮其家。斌慰安調護,為寓金帛他所,密告兼濟勿以家為恤。平冤既伸,兼濟獲免,家賴以全。定
 川之役,將曹偀喪所乘馬,斌輟騎與之,且戰且行,與俱免。



 狄青南征,使部騎兵為前鋒。青駐賓州十日以怠寇,既乃倍道兼行。斌以兵疲於險,利在速戰,即日度關。鏖賊歸仁驛,孫節死,斌引騎血戰,繞出賊後,遂敗之。師還,張破賊陳形於殿廷,仁宗拊勞,擢文思副使、權廣西鈐轄。改秦鳳,廣西以蠻事乞留,秦州亦請之,詔留廣西。



 累歲,徙涇原。召對,議者謂交州可取,斌盛言有害無益,願戒邊臣無妄動。神宗嘆曰:「卿質直如此,乃知兩路爭卿,
 為不誣矣。」進帶御器械。渭部饑,帥王廣淵命吏賑給,斌曰:「救之無術,是殺之耳。」廣淵以委斌,斌擇地營居,養視有法,所活以萬數。



 安南入寇,復徙廣西。累遷皇城使、昭州刺史。撫水蠻羅世念犯宜州,守將戰死。斌提步騎三千進討,方暑,晝夜趣兵,至懷遠寨,曰:「此要害之地,得之則生。」或曰:「奈何背龍江邪?」笑曰:「是所以生也。」因示弱驕之,蠻果大至,斌選將迎敵,戒以遇之則走,誘至平板,列八陣以待之。張疑兵左右山上,蠻登嶺望見,始大驚。斌
 分騎翼其旁,自被甲步出,為眾士先,殊死戰。蠻大敗,世念率酋黨四千八百內附。遂以榮州團練使知宜州,遷西上閣門使、知邕州,以老請還,除高陽關副總管,歷永興軍路。召拜龍、神衛四廂都指揮,至步軍都虞候,卒,年八十。贈寧州防禦使。



 斌老於為將,以恩信得邊人心,嶺南珍貨,一無所蓄。邊吏欲希功造事,皆憚不敢發;或巧為諜報啟釁,亦必折其奸謀。故所至無事,士大夫稱之。



 子詵,以蔭為河北副將,累官至右武大夫、威州刺史、知
 雄州。上制勝強遠弓式,能破堅於三百步外,邊人號為「鳳凰弓。」進相州觀察使。在雄十年,頗能偵敵。童貫攻燕,召詵計事,悅之。分麾下兵俾以副統制,從種師道軍於白溝,旬有二日而退。追兵至,北風,大雨雹,師不能視。契丹以背盟譙責,薄暮,始得還。於是貫以契丹尚盛未可圖,劾詵覘候不實,貶濠州團練副使,筠州安置。



 詵始興取燕之謀,見事勢浸異,則又以為不宜取,故平燕肆赦,獨不得還。後復官,卒。



 劉仲武,字子文,秦州成紀人。熙寧中,試射殿庭異等,補官。數從軍,累轉禮賓使,為涇原將。夏人謀犯天聖砦,渭帥檄諸將會兵,約曰:「過某日賊不至,即去。」仲武諜得的期,乞緩分屯。帥不樂,但留一將及仲武軍,如期而敵至,力戰卻之。遷皇城使、熙河都監。復湟州,進東上閣門使、知河州。



 吐蕃趙懷德、狼阿章眾數萬叛命,仲武相持數日,潛遣二將領千騎扣其營,戒曰:「彼出,勿與戰,亟還,伏兵道左。」二將還,羌果追之,遇伏大敗,斬首三千級,復西
 寧州。未幾,懷德、阿章降。累進客省使、榮州防禦使。



 副高永年西征。仲武欲持重固壘,永年易賊輕戰,遂大敗。仲武引咎自劾,坐流嶺南。命未下,與夏人戰,傷足。朝廷閔之,貸其罰,以為西寧都護。



 童貫招誘羌王子臧徵僕哥,收積石軍,邀仲武計事。仲武曰:「王師入,羌必降;或退伏巢穴,可乘其便。但河橋功力大,非倉卒可成,緩急要預辦耳。若稟命待報,慮失事機。」貫許以便宜。僕哥果約降,而索一子為質。仲武即遣子錫往,河橋亦成。仲武帥師
 渡河,挈與歸。貫掩其功,仲武亦不自言。徽宗遣使持錢至邊,賜獲王者。訪得仲武,召對,帝勞之曰:「高永年以不用卿言失律,僕哥之降,河南綏定,卿力也。」問幾子,曰:「九人。」悉命以官,錫閣門祗候。



 仲武知西寧州,徙渭州,召為龍、神衛都指揮使,復出熙州、秦州,遷步軍副都指揮使。熙帥劉法死,又以熙、渭都統制攝之。歷拜徐州觀察使、保靜軍承宣使、瀘川軍節度使。以老,提舉明道宮,再起為熙州。卒於官,年七三。贈檢校少保,謚曰威肅。子錡,
 別有傳。



 曲珍,字君玉,隴干人,世為著姓。寶元、康定間,夏人數入寇,珍諸父糾集族黨御之,敵不敢犯。於是曲氏以材武長雄邊關。



 珍好馳馬試劍,嘗與叔父出塞游獵,猝遇夏人,陷其圍中。馳擊大呼,眾披靡,得出,顧叔不至,復持短兵還決鬥,遂俱脫。秦鳳都鈐轄劉溫潤奇其材,一日,出寶劍令曰:「能射一錢於百步外者,與之。」諸少年百發不能中,珍後至,一矢破之。從溫潤城古渭,與羌戰,先登陷
 陳。為綏德城監押,提孤軍拒寇,斬其大酋,加閣門祗候。有功洮西,遷內殿崇班。



 郭逵、趙離南征,為第一將。進自右江,撫接廣源三州十二縣,降偽守已下百六十人,老稚三萬六千口。是行也,功最諸將,遷西染院使。得疾,輿還京師,神宗遣使臨問,少間,令入對。珍念二帥不和睦,上問必及之,言之必形曲直,將何以對,乃以餘疾未平為解。帝復使獎勞,賜之弓劍、鞍勒,命有司蠲其鄉徭斌,擢鄜延鈐轄,進副總管。



 從種諤攻金湯、永平川,斬二千
 級。累遷客省使,拜懷州防禦使、龍神衛四廂都指揮使。徐禧城永樂,珍以兵從。版築方興,羌數十騎濟無定河覘役,珍將追殺之,禧不許。諜言夏人聚兵甚急,珍請禧還米脂而自居守。明日果至,禧復來,珍曰:「敵兵眾甚,公宜退處內柵,檄諸將促戰。」禧笑曰:「曲侯老將,何怯邪?」夏兵且濟,珍欲乘其未集擊之,又不許。及攻城急,又勸禧曰:「城中井深泉嗇,士卒渴甚,恐不能支。宜乘兵氣未衰,潰圍而出,使人自求生。」禧曰:「此城據要地,奈何棄之?且
 為將而奔,眾心搖矣。」珍曰:「非敢自愛,但敕使、謀臣同沒於此,懼辱國耳。」數日城陷,珍縋而免,子弟死者六人。亦坐貶皇城使。帝察其無罪,諭使自安養,以圖後效。



 元祐初,為環慶副總管。夏人寇涇原,號四十萬,珍搗虛馳三百里,破之曲律山,俘斬千八百人,解其圍。進東上合門使、忠州防禦使。卒,年五十九。珍善撫士卒,得其死力。雖不知書,而忠樸好義,本於天性。



 劉闃,字靜叔,青州北海人。以拳力為軍校,從延州軍出
 塞遇敵,矢貫左耳,戰不顧,眾服其勇。從文彥博討貝州,次城下,攀壘欲登,賊以曲戟鉤其甲,闃裂之而墜。議者欲穿地道入,闃曰:「穴地積土,賊且知之。城瀕河,若晝囊土而夜投諸河,宜無知者。」彥博以為然。穴成,闃持短兵先入,眾始從,遂登陴,引繩而上,遲明,師畢入。貝州平,功第一,擢虎翼指揮使。累遷宣武神衛都指揮使、昭州刺史、辰州團練使。



 韓絳宣撫陜西,詔闃自河東為犄角。至鐵冶溝,夏人大集。眾懼,闃自殿後,率銳驍搏戰,飛矢蔽
 體不為卻,敵解去。



 為冀州駐泊總管。河水漲,堤防墊急,闃請郡守開青楊道口以殺水怒,莫敢任其責。闃躬往浚決,水退,冀人賴之。以左金吾大將軍致仕。卒,年八十五。



 郭成,字信之,德順中安堡人也。從軍,得供奉官。王師趨靈武,成將涇原兵擊破夏人於漫□移隘。至城下,有羌乘白馬馳突陣前,大將劉昌祚曰:「誰能取此者?」成躍馬梟其首以獻,進秩四等。



 朝廷築平夏城,置將戍之,又環以
 五砦。渭帥章楶問可守者於諸將,皆曰:「非郭成不可。」遂使往守。夏人恚失地,空國入爭,謀曰:「平夏視諸壘最大,郭成最知兵。」遂自沒煙峽連營百里,飛石激火,晝夜不息。成與折可適議乘勝深入,以萬騎異道並進,遂俘阿埋、都逋二大酋。捷聞,進雄州防禦使、涇原鈐轄。徽宗詔諸軍並力築綏戎、懷戎二堡,成獨當合流之役,暴露雪中,感疾卒。帝悼之甚,賻以金帛,官其子婿。



 成輕財好施,名震西鄙。既沒,廉訪使者王孝謁白於朝,帝手書報曰:「
 郭成盡忠報國,有功於民,宜載祀典。」榜其廟曰「仁勇」云。子浩,紹興中為西邊大將,至節度使。



 賈巖,字民瞻,開封人。少時,善騎射,喟然嘆曰:「大丈夫生世,要當自奮,揚名顯親可也。」遂起家從戎。神宗選材武,以為內殿承制、慶州荔原堡都監。



 林廣討瀘夷,闢將前鋒。又為河東將,敗西夏兵於明堂川。累功轉莊宅副使。遷路監。紹聖中,夏兵數萬圍麟州神堂砦甚急,巖以數百騎往援,令其下曰:「國家無事時,不惜厚祿養汝輩,正
 以待一旦之用耳。今力雖不敵,吾誓以死報!」眾感厲,即循屈野河行,且五里,據北攔坡嶺上,一矢殪其酋,眾駭潰。哲宗嘉嘆,賜以袍帶。知皇城使、威州刺史,遷路鈐轄。



 巖在兵間二十年,在智略,能拊御士卒,所鄉輒勝。時以良將入對,留擢龍、神衛四廂都指揮使,遷步軍都虞候、濠州團練使。卒,年五十二,贈雄州防禦使。



 張整,字成伯,亳州贊阜陽人。初隸皇城司御龍籍,補供奉官,為利、文州都巡檢使。邊夷歲鈔省地,吏習不與校,至
 反遺之物,留久乃去。整惡其貪暴無已,密募死士,時其來,掩擊幾盡。有司劾生事,神宗壯之,不問。



 調荊湖將領,拓溪蠻地,築九城,董兵鎮守。又破蠻於大田,歲中三遷。犬吉狑萬眾乘舟屯托口,迫黔江城,時守兵才五百,人情大恐。整伏其半於托口旁,戒曰:「須吾旦度金斗崖,舉幟,則噪而前。」及旦,率其半,縛艨艟,建旗鼓,溯流急趨。賊望見大笑。幟舉伏發,前後合擊,人人殊死鬥,蠻騰踐投江中,殺獲不可計。為廣西鈐轄,坐殺降徭,責監江州酒稅。
 復為涇原、真定、京東、環慶鈐轄。



 整蒞軍嚴明,哲宗嘗訪於輔臣,召之對,擢為龍神衛四廂都指揮使、管幹馬軍司。卒,官至威州刺史。



 張蘊,字積之,開封將家子也。從軍為小校,隸劉昌祚。至靈州,遇敵中矢,拔鏃復戰,以功賜金帶。從征安南,次富良江,諸將猶豫未進,蘊褰裳先濟,眾隨之。蠻遁走,使巫被發登崖為厭勝,蘊射之,應弦而斃,一軍歡噪。



 歷京西、涇原將,知綏德、懷寧、順寧軍等六城,儲粟至三十萬斛。
 將兵取宥州,破夏人於大吳神流堆。宥州監軍引鐵騎數千趨松林堡,蘊諜知之,頓兵長城嶺以待,戒諸部曰:「賊遠來氣盛,少休必困,困而擊之,必捷。」果以勝歸。夏人寇順寧,蘊置伏狹中,約聞呼則起,俘斬數百十人,獲馬、械甚眾。累遷皇城使、榮州刺史、成州團練使、通州防禦使,開德、河陽馬步軍副總管。



 顯肅皇后母自鄭氏再適蘊,徽宗屢欲以恩進其官,輒力辭不敢受,人以為賢。卒,年七十三,贈感德軍節度使,謚
 曰榮毅。



 王恩,字澤之,開封人。以善射入羽林,神宗閱衛士,挽強中的,且偉其貌,補供備庫副使。為河州巡檢,夏羌寇蘭州,恩搏戰城下,中兩矢,拔去復鬥,意氣彌厲。遷涇原將。嘗整軍出萬惠嶺。士饑欲食,恩倍道兼行,眾洶洶。已而遇敵數萬,引兵先入壁,井灶皆具,諸將始服。羌扣壁願見,恩單騎徑出,遙與語,一夕,羌引去。



 哲宗召見,語左右曰:「先帝時宿衛人,皆傑異如此。」留為龍、神衛都指揮使,遷馬軍都虞候。契丹使來,詔陪射,使者問:「聞涇原有王
 騎將,得無是乎?」應曰:「然。」射三發皆中,使以下相視皆嘆息。



 出為涇原副都總管,並護秦、渭、延、熙四路兵,城西安,築臨羌、天都十餘壘。羌圍平夏,諸校欲出戰,恩曰:「賊傾國遠寇,難以爭鋒,宜以全制其敝。彼野無所掠,必攜,攜而遇伏,必敗。」乃先行萬人設伏,羌既退師,果大獲。



 徽宗立,以衛州防禦使徙熙河,改知渭州。括隱地二萬三千頃,分弓箭士耕屯,為三十一部,以省饋餉。邊臣獻車戰議,帝以訪恩,恩曰:「古有之,偏箱、鹿角,今相去益遠,人非
 所習,恐緩急難用。夫操不習之器,與敵周旋,先自敗耳。」帝善其對。遷馬步軍都指揮使、殿前都指揮使、武信軍節度使。



 嘗汰禁卒數十人,樞密請命都承旨覆視,恩言:「朝廷選三帥,付以軍政,今去數十冗卒而不足信,即其它無可為者。」帝立為罷之。眷顧甚寵,賜居宅,又賜城西地為園囿。屬疾,以檢校司徒致仕。薨,年六十二,贈開府儀同三司。



 楊應詢。字仲謀,章惠皇后族孫也。歷知信安保定軍、霸
 州。塘濼之間地沮洳,水潦易集,居人浮板以濟。應詢增堤防為長衢,浚其旁以洩流,民利賴之。為河北沿邊安撫使。徽宗以歸信、容城兩縣弓手為契丹所憚,欲增為千人,或恐生事,應詢曰:「吾欲備他盜,彼安能禁我?」卒增之。



 知雄州,朝廷多取西夏地,契丹以姻婭為言,遣使乞還之,不得,擁兵並塞,中外恫疑。應詢曰:「是特為虛聲嚇我耳。願治兵積粟示有備,彼將聞風自戢。」明年,果還兵。復遣其相臣蕭保先、牛溫舒來請,詔應詢逆於境。既至,
 帝遣問所以來,應詢對:「願固守前議。」尋兼高陽關路鈐轄。



 邊人捕得北盜呂懺兒,契丹謂略執平民,有詔使縱釋。應詢言:「吾知執盜耳,因其求而遂與之,是示以怯也。」不與。遂質我民,固索之。應詢以違詔貶秩,再遷洋州觀察使。入提舉萬壽觀。館契丹使,當賜柑而貢未至,有司代以他物,使不受,慶詢以言折之,乃下拜。復為定州、真定、大名副都總管。卒,年六十三,贈昭化軍節度使,謚曰康理。



 趙隆,字子漸,秦州成紀人。以勇敢應募,從王韶取熙河。大將姚麟出戰,被重創,謂曰:「吾渴欲死,得水尚可活。」時已暮,有泉近賊營,隆獨身潛往,漬衣泉中。賊覺,隆且鬥且行,得歸,持衣裂水以飲麟,麟乃蘇。又從李憲破西市。師討鬼章,外河諸羌皆以兵應之。隆率眾先至,斧其橋,鬼章失援,乃成擒。



 為涇原將,戰平夏川,功最多。崇寧中,鈐轄熙河兵,將前軍出邈川,預復鄯、廓。夏人寇涇原,詔熙河深入分其兵,無令專鄉東方。師至鐵山,隆先登,士
 皆殊死戰,夏人解去。召詣闕,徽宗慰勞之曰:「鐵山之戰,卿力也。」



 童貫與論燕云事,隆極言不可。貫曰:「君能共此,當有殊拜。」隆曰:「隆武夫,豈敢干賞以敗祖宗二百年之好?異時起釁,萬死不足謝責。」貫知不可奪,白以知西寧州,充隴右都護。羌豪信服,十二種戶三萬六千,願屯內地。



 帥劉法西討,隆以奇兵襲羌,羌潰,城震武。遷溫州防禦使,龍神衛、捧日天武都指揮使,仍為本道馬步副都總管。卒,贈鎮潼軍節度使,命詞臣制碑,帝篆額曰「旌忠」。



 論曰:有國家者不可忘武備,故高祖以馬上得天下,而猶有「安得猛士守四方」之嘆。然所貴為將領者,非取其武勇而已也,必忠以為主,智以為本,勇以為用,及其成功,雖有小大之殊,俱足以尊主庇民也。苗授策籛南之不可城,履不肯討阿章,永清不以地與敵,文鬱撫納香崖,紹能之忠勇,珍之忠樸好義,光祖、應詢明於料敵,守約及整御眾嚴明,斌、浩之善戰,巖、恩之善射,闃之出則先登,入則殿後,其材雖殊,其可以任奔走禦侮之責於
 四境則一也。成以捍衛邊陲,服勤致死,明詔褒飭,廟食一方,宜哉。君萬挾誣報怨,贍狡譎喜功,國有常罰,父子謫死,亦宜也。詵首取燕,終變其說,既黜旋復,為失刑矣。至若仲武敗則引咎責己,勝則不自言功,隆不敢啟釁干賞,蘊甘分而辭榮,有士君子之行焉,尤武士之所難能也。



\end{pinyinscope}