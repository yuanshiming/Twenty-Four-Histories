\article{志第十三 天文十三}

\begin{pinyinscope}

 流隕四



 建中靖國元年正月癸亥,星出西南,如盂,東北急流,入尾距星沒,青黑,無尾跡,明燭地。



 崇寧元年三月庚辰,星出張,如金星,西南急流,至濁沒,赤黃,有尾跡,明燭地。五月丁卯,星出尾,如杯,西南慢流,入濁沒,青白,有尾跡,明燭地。閏六月癸酉,星出鬥,向西南慢流,至建沒,青白,有尾跡,數小星從之。八月己未,星出羽林軍,如杯,急流至濁沒,青白,有尾跡,明燭地。十月
 壬子,星出天船,如盂,急流至五車沒,青黑,有尾跡,聲隆隆然。十二月己卯,星出婁,如金星,西南慢流,至外屏沒,赤黃,有尾跡,明燭地。二年正月戊申,星出未位,如金星,急流至北河沒,青白,有尾跡,明燭地。六月戊午,星出亢,如金星,西南急流,入濁沒,赤黃,有尾跡,明燭地。九月辛巳,星出牛,如杯,西南慢流,至狗國沒,青白,有尾跡,明燭地。十一月甲辰,星出參,如金星,西南急流,至濁沒,青白,無尾跡,明燭
 地。十
 二月丁未,星出大陵,如金星,至騰蛇沒,赤黃,有尾跡,明燭地。三年
 四
 月戊申,星出軫,如杯,西北慢流,入太微垣內屏星沒,赤黃,有尾跡,明燭地;又
 入太微;又入屏星。六月丙午,星出氐,如金星,東北慢流,入天市垣,赤黃,有尾跡,明燭地。八月己酉,星出建,如杯,西南急流,至鱉沒,青白,有尾跡,明燭地。十二月甲子,星出天大將軍,如盂,西北急流,入王良沒,赤黃,無尾跡,明燭地。四年正月甲申,星出角,如盂,西南慢流,入濁沒,青白,無尾跡。閏二月壬申,星出井,如金星,西北急流,入五車沒,青白,有尾跡,明燭地。三月庚子,星出紫微垣華蓋,如杯,至鉤陳大星
 沒,赤黃,有尾跡,明燭地。五月庚申,星出河鼓,如盂,西北急流,入濁沒,青白,有尾跡,十二月甲午,星出參,如杯,東南慢流,入軍市沒,赤黃,有尾跡,明燭地。五年六月庚午,星出西咸,如金星,東北急流,入天市垣內沒,青白,有尾跡,明燭地。六月乙酉,星出庫樓,如杯,向西急流,入濁沒,赤黃,有尾跡,明燭地。九月癸卯,星出天船,如杯,慢流至諸王沒,青白,有尾跡,明燭地。十二月壬戌,星出奎,向南急流,入天倉沒,青白,有
 尾跡
 及三丈,明燭地,聲散如裂帛。



 大觀元年二月丁卯,星出參,如杯,西南急流,入濁沒,赤黃,無尾跡,明燭地。四月辛未,星出軫,如盂,向南慢流,入濁沒,青白,有尾跡,明燭地。六月乙亥,星出尾西南,如杯,西南
 慢流,入濁沒,青白,有尾跡,明燭地。七月庚戌,星出箕,如杯,西南急流,入濁沒,赤黃,無尾跡,照地明。二年十二月癸卯,星出奎,如盂,西北急流,入造父沒,赤黃,有尾跡,照地明,有聲。



 政和元年四月丙辰,星出亢,如盂,西北急流,至右攝提沒,赤黃,有尾跡,照地明。五月辛巳,日未中,星隕東南。二年九月乙卯,星出鬥,如杯,西南急流,入濁沒,赤黃,有尾跡,照地明。三年四月丙申,星出心,如盂,西南急流,至積卒沒,青白,有尾跡,照地明。四年九月庚子,星
 出墳墓,如盂,東南急流,入羽林軍沒,青白,有尾跡,照地明。七年十二月甲子,星出胃東南,如盂,西北急流,至天大將軍沒,赤黃,有尾跡,照地明。



 重和元年九月庚辰,星出鬥魁南,如盂,東南急流,至天淵沒,赤黃,有尾跡,照地明。



 宣和元年三月丁卯,星出柳,如盂,東北急流,入太微垣,赤黃,有尾跡,照地明。十月戊子,星出雲雨,如盂,西南慢流,入羽林軍內沒,青白,照地明。二年六月庚寅,星出氐南,如太白。東北急流,入天市垣,無尾跡。十二月辛巳,
 星出奎西南,如杯,西南慢流,至北沒,赤黃,有尾跡,照地明。三年七月癸未,星出鬥,如太白,東南急流,入濁沒,青白,有尾跡,照地明。四年十一月丙寅,星出王良北,如杯,急流至紫微垣內上輔北沒,赤黃,有尾跡,照地明。五年二月丙午,星出北河東北,如杯,東南慢流,至軫沒,赤黃,有尾跡,照地明。六年七月丁酉,星出太陽守,如盂,東北急流,入濁沒,赤黃,有尾跡,照地明。七年十一月戊子,星出王良北,如杯,急流入紫微垣上輔北,赤黃,有尾跡,照
 地明。



 靖康元年二月丙辰,星出張,如太白,東南急流,至濁沒,青白,有尾跡,照地明。又星出北河,如太白,東南慢流,至軫東沒,赤黃,有尾跡,照地。三月壬辰,星出紫微垣內鉤陳東南,如金星,東北慢流,至濁沒,赤黃,有尾跡,照地。五月乙未,星出權東北,如桃,西北急流,至濁沒,青白,有尾跡,照地。六月癸丑,星流大如五斗器,眾光隨之,明照地,起東南,墜西北,有聲如雷。庚申,星出紫微垣內華蓋東南,如金星,向北急流,至左樞沒。二年正月乙未,大
 星出建,向西南急流,至濁沒,赤黃,有尾跡,照地。



 建炎四年六月乙酉,星出紫微垣鉤陳。十月辛未,星出壁。



 紹興元年四月甲戌,星出東方,晝隕。七月乙未朔,星出河鼓。八月辛未,星出羽林軍。十一月庚戌,星出婁宿西南。丁巳,星出天槍北。十二月甲子朔,星出大陵西北。二年三月甲午,星出紫微垣華蓋西南。乙卯,星出角。丁巳,星出紫微垣右樞星。戊午,星出軒轅大星西南。閏四月乙巳,星出太微垣西右執法北。五月癸未,星出河鼓。五年十
 月壬戌,星出室東南,赤黃而大。六年十月壬子,星出壁西北。七年八月壬寅,星隕於汴。八年十一月乙巳,星出天囷東北。九年五月癸未,星出房宿東南。十七年八月己未,星出危宿,慢流至貫索沒,青白色,有尾跡,照地明,大如太白,二十六年六月丁亥,星出東北方,光明照地。二十八年六月戊戌,星晝隕,有尾長三丈,至西北沒。二十九年八月戊寅,星出紫微垣西南,約長三尺,赤黃色,西南急流,至鉤陳大星東北沒。三十一年六月乙卯,星
 出右攝提,赤白色,急流向東南沒,有尾跡,大如歲星。丁巳,星出,青白色,自東北急流向東南沒,有尾跡,大如盞口。甲子,星出氐,赤黃色,慢流至角宿天田沒,初小後大,如太白,後有小星隨之。九月壬午,星晝隕,約長三丈。



 隆興元年六月丁丑,星出尾宿,青白色,向東南慢流沒。七月壬寅,星出天市垣內,赤色,向西北慢流,至右攝提西南沒,炸散小星二十餘顆,有聲,尾跡大如太白,丙午,又出天市垣,慢流至氐宿沒,青白色,微有尾跡,小如填星。
 癸丑,星出織女,急流向貫索西北沒,青白色,明大如土星,照地。丙辰,星出輦道,急流入天棓西南沒,赤黃色,有尾跡,小如土星。八月庚申,星出羽林軍,赤黃色,向東南急流,至濁沒。戊辰,星出虛宿,赤黃色,急流至牛宿西南沒。壬申,星出天市垣,赤青色,慢流至西咸西北沒。癸酉,星出壁宿,赤黃色,急流犯王良星沒,如太白。丙子,星出羽林軍門,青白色,慢流委曲行,至東南濁沒。辛巳,星出南斗,赤黃色,慢流入羽林軍沒,有尾跡,大如金星;次有
 星一,赤黃色,有尾跡,亦如金星,出雲雨星,慢流向西南,至女宿之下沒。戊子,星出羽林軍門東南,慢流至濁沒,青白色,有尾跡,大如土星。又星一,青白色,出天倉,向東南急流,有尾跡,小如木星,至濁沒。九月庚戌,星出紫微垣外坐,赤黃色,向西北急流,抵紫微垣內坐尚書星沒。十一月庚寅,星出軫宿,急流向東南騎官星沒,赤黃色,有尾跡,大如木星。丁未,飛星出天船,急流向紫微垣外坐內廚西北沒,炸出二小星,青白色,有尾跡,照地,大如
 木星。二年二月辛酉,飛星出權星,慢流至太微垣內五帝坐大星西南沒,青白色,微有尾跡,大如歲星。六月丁丑,星出王良,青白色,急流犯天津西南沒。己卯,飛星出造父,急流入紫微垣內鉤陳大星東南沒,青白色,大如填星。辛亥,星出天關,急流貫入畢口西北沒,有尾跡,照地明,大如太白,赤黃色。十月丙辰,星出趙國,向西南慢流,犯趙東星沒,有尾跡,大如填星,赤黃色。十一月壬午朔,星出卯位,慢流至西南沒,有尾跡,照地明,大如太白,
 青白色。癸未,星出,犯弧矢,急流至天廟東南沒,有尾跡,大如太白,青白色。丁亥,星出天苑,向西南慢流,至濁沒,微有尾跡,大如太白,色赤黃。癸卯,星出羽林軍,慢流向西南濁沒,大如太白,色赤黃。辛亥,星出南河,向東南慢流,至翼宿沒,微有尾跡,大如太白,色赤黃。十二月壬午,星出弧矢,向東南至濁沒,有尾跡,照地明,大如太白,色青白。



 乾道元年三月丙辰,星出周國,急流至天雞沒,微有尾跡,大如歲星,色黃白。甲子,星出張宿,慢流向西南,
 至濁沒,有尾跡,照地明,大如太白,色赤黃。五月丁丑,星出河鼓,白色,向東北慢流,至濁沒,有尾跡,照地明,大如太白,六月甲辰,星出東北,慢流向西南沒,有尾跡、音聲,大如太白,色赤黃。七月壬戌,星出西南,慢流至東南沒,大如歲星,色赤黃。庚午,星出代國,慢流至趙國沒,大如歲星,色青白。九月戊申,星出王良,慢流至尾宿沒。十月癸未,星出權星東南,急流至太微垣沒,有尾跡,照地明,如太白,色青白。二年二月庚子,星出西北方,急流至濁
 沒,明大如歲星,色青白。六月丙子,星出角宿,急流至軫宿沒,有尾跡,大如太白,色赤黃。七月己巳,星出織女,急流至天市垣內宗星沒,有尾跡,大如歲星,青白色。十一月己未,星出,急流東南蒼黑雲間沒,大如歲星,色青白。十二月,星出天關,急流至外屏星沒,有二小星隨之,赤黃色,微有尾跡,大如歲星。三年九月甲午,星出卷舌,急流至婁宿沒,有尾跡,大如歲星,黃白色。又有星青白色,出北斗,急流至少宰西北沒,大如歲星。五年七月甲子,
 星出宗正,赤色,慢流至女宿沒,有尾跡,照地明,大如歲星。九月丙辰,星出,赤黃色,如蛇,入天棓沒。六年九月辛巳,星出狼星,入弧矢,至濁沒,微有尾跡,大如填星,赤黃色。十月庚戌,星出天囷,急流至濁沒,有尾跡,大如歲星,赤黃色。七年七月戊戌,星大如拳,急流向西北方,至濁沒,有尾跡,照地如電。九月甲午,透雲星出,急流向西南方,至濁沒,高丈餘,有尾跡,照地明,大如太白,色青白。



 淳熙三年正月辛未,星出狼星,急流至濁沒,尾跡照地明,
 大如太白,四月戊戌,星出角宿,青白色。五年八月乙巳,星出狼星,急流向東南沒,微有尾跡,大如太白,青白色。六年八月壬辰,星出紫微垣鉤陳大星,慢流至濁沒,有尾跡,大如盞口,青白色。七年五月乙亥,星出天市垣內東海星,慢流,炸作三小星,有尾跡,照地,大如盞口,青白色。八月丁未,星出貫索大星西北,急流至濁沒,有尾跡,照地明,大如太白,色青白。十一年四月乙丑,星出自中天,慢流向東北方沒,微有尾跡,炸作小星相從,有聲,明
 大如太白,色青白。十五年二月辛未,星出太尊,大如盞口,急流至濁沒,色青白。



 慶元二年九月甲午、四年六月甲午,星皆晝隕。七月壬寅,星出羽林軍下,青白色,大如碗。九月丁巳,星出奎宿,向壁壘陣沒,赤白色,大如太白,五年六月丁丑,星出東北,慢流至西南方沒,大如歲星,青白色。九月壬子,星出西南,慢流向東北沒,大如太白,青白色,



 嘉泰二年四月辛巳,星出西北,急流東北至濁沒,色赤。十月乙酉,星出五車,大如歲星。四年十一月庚
 午,星出天津,急流入天市垣沒。



 開禧元年正月庚子,星出中天,赤色,大如太白。向濁沒。七月癸亥,星出天津,入斗宿東南沒,色赤,大如太白,二年六月癸丑,星出招搖,入庫樓,色赤,大如太白。



 嘉定元年六月辛未,星出天津東北,慢流向天市垣沒。二年六月壬午,星出織女東南,慢流入天市垣沒,色赤,有尾跡,照地明,大如太白。庚寅,星出中天,急流向東北,至濁沒。三年九月己酉,星夕隕。五年七月乙巳,星出中天,慢流向西南方,至濁沒。六年
 五月癸亥,星晝隕。九月癸卯,星夕隕。丁巳,星晝隕。十月戊戌,星出昴宿西南,慢流向天廩東南沒。壬戌,星出西南,慢流至濁沒,青白色。十二月壬寅,星晝隕。七年三月壬午,星出軫宿距星東南,慢流至濁沒。五月辛卯,星出天津西南,慢流向心宿西北沒。八年七月癸未,星出室宿距星東北,急流向天倉星西北沒。乙酉,星出織女東南,慢流向牛宿西北沒,有尾跡,照地明,大如太白,青白色。八月甲辰,星出天津西南,慢流向河鼓東北沒。十二
 月丙申,星出五諸侯東北,慢流向天關西南沒,有聲及尾跡,明照地,赤黃色。九年六月乙巳,星出牛宿距星東北,慢流至濁沒。十年五月壬申,星出尾宿距星西北,慢流向牛宿距星東南沒。十一年六月乙卯,星出河鼓距星西南,急流向正西,至濁沒。十二年十一月乙亥,星出昴宿東南,急流至濁沒。十三年十二月丁巳,星出軫旗東北,慢流至濁沒,赤黃色。十四年二月壬午,星出南河距星東南,慢流向西南,至濁沒,赤黃色。八月戊午,星出
 房宿距星,急流至濁沒,有尾跡,照地明,大如太白,赤黃色。十一月甲申,星出天倉距星四北,慢流向東南方,至濁沒,赤黃色。十六年十一月壬戌,星出五諸侯東北,急流向西北,至濁沒,色赤黃,隆隆有聲,及尾跡照地,大如盞。



 寶慶二年四月辛亥,星出,大如太白,



 紹定元年六月己酉,星晝隕。二年正月庚辰、九月壬辰,星出,大如太白,三年十一月丁未,星晝隕。四年七月庚戌,星出,大如太白,九月甲辰,星晝隕。五年八月甲寅,星夕隕。閏九月己
 酉,星出,大如太白,



 端平元年六月丙戌,星西南行,大如太白,有尾跡,照地明。二年四月戊子,星出,大如太白,六月庚辰,星晝隕。七月丁酉,星出,大如太白,辛丑,星晝隕。十月辛卯,星出,大如太白,三年五月庚辰,星出心宿,大如太白,六月癸巳,星夕隕。



 嘉熙元年正月壬午,星出,大如太白,二月己丑,星夕隕。九月癸丑,星出七公西,至濁沒。十月戊戌,星出,大如桃。二年四月甲子,七月辛卯,九月乙未,星出,大如太白,六月甲辰、八月癸亥,星晝隕。三
 年三月甲戌,星晝隕。八月辛丑,星出,大如太白,四年正月辛巳,六月戊午,星出,大如太白,二月辛丑、三月癸未,星晝隕。



 淳祐元年六月癸酉,星出,大如太白,己卯,星晝隕。三年六月甲戌,星出氐宿距星,大如太白,八月乙卯,星晝隕。四年四月丙子,星出尾宿距星下,大如太白,六月乙未,星出畢宿,大如太白,六年七月癸酉,星出室宿,大如太白,九月甲子,星出鬥宿,尾跡青白照地,大如太白,七年九月丙辰,星出室宿。八年六月甲辰,星出河鼓,
 大如太白,十月丙辰,星出角宿距星。九年六月壬戌,其日,星自南方急流,至濁沒,赤黃色,大如太白,十月壬申,星出織女。十年四月丁酉朔,星夕隕。十一年七月丁丑,星出畢宿距星,赤黃色,大如太白,八月己丑朔,星夕隕。十二年四月庚申,星出角宿、亢星,大如太白,八月癸丑,星出角,色赤照地。



 寶祐元年四月丁巳,星出,大如太白,二年七月庚戌,星出,大如太白,三年七月辛酉,星出,大如太白,十月丁丑,星出畢宿距星。五年七月丁卯,星出,
 大如桃。六年九月戊辰,透霞星出。



 開慶元年六月己亥,星出鬥宿河鼓,急流向東南,至濁沒,赤黃色,有音聲,尾跡照地明,大如太白,



 景定元年七月丙子,星出東南,大如太白,十月乙卯,星出東北,急流向太陰,有音聲,尾跡照地明,大如桃。三年四月甲辰,星出,大如盞。六月己酉,星出,大如熒惑。九月丙子,星出,大如太白,閏九月丙戌,透霞星出,大如太白,庚子,星出,大如太白,四年五月戊戌,星出角宿距星。六月丁卯,星出河鼓。八月乙卯,星出
 天倉。五年二月壬戌,星出畢宿。五月甲午,星出河鼓大星東南,急流向西北,至濁沒,赤黃,有尾跡,照地明,大如太白,七月己卯,星出右攝提。



 咸淳二年六月甲戌,星出左攝提。三年七月庚寅,星出昴宿東南,急流至濁沒,赤黃,有尾跡,照地明,大如太白,四年七月戊午,星出氐宿距星西北,急流入騎官星沒,赤黃,有尾跡,照地明,大如桃。五年五月庚申,星出鬥宿距星東北,急流向牛,至濁沒。六月庚寅,星出鬥宿。七月壬戌,星出東南河鼓距星
 西北,急流至濁沒。



 德祐元年四月癸亥,有大星自心東北流入濁沒。



 妖星



 建隆二年五月己丑,天狗墮西南。



 紹興十七年正月乙亥,妖星出東北方女宿內,小如歲星,光芒長五丈,二月丙寅始消。



 淳熙十三年九月辛亥,星出,大如太白,色先赤後黃白,尾跡約二尺,委曲如蛇行,類枉矢。十四年五月,有星出濁際,大如日,與日相摩蕩而入。



 嘉定十一年
 五月癸未,蚩尤旗竟天。



 端平二年春,天狗墜懷安金堂縣,聲如雷,三州之人皆聞之,化為碎石,其色紅。



 咸淳十年九月壬寅,有星見西方,曲如蚓。



 德祐元年二月丁亥,有星二斗於中天,頃之,一星墜。



 星變



 紹興三十一年六月戊午,大角星東北生角。



 隆興二年九月戊戌,大角光體搖動。十月丙子,弧矢九星內矢一星偏西不向狼星。



 乾道元年八月乙巳,大角光體搖動。



 淳熙元年七月辛亥,奎宿生芒。



 雲氣



 乾德三年七月己卯夜,西方起蒼白氣,長五十丈,貫天船、五車,亙井宿。



 開寶元年十月乙未旦,西北起蒼白氣三道,長二十丈,趨東散。



 太平興國四年四月己巳夜,西北有白氣壓北斗。



 雍熙三年正月己未夜,赤氣如城。四年正月癸酉夜,白氣起角、亢,經太微垣,歷軒轅大星,至月傍散。



 端拱元年十月壬申遲明,巽上有雲過中天,連
 地,濃潤,前赤黃,後黑蒼色,先廣後大,行勢如截。十一月戊午夜,西北方有赤氣如日腳,高二丈。



 至道二年二月丙子夜,西方蒼白色氣長短八道,如彗掃,稍經天漢,參錯如交蛇。



 咸平三年十月辛亥,黑氣貫北斗。十二月庚午,黑氣長三丈餘,貫心宿,入天市垣抵帝坐,久方散。四年三月丙申,白氣二,亙天。十月辛亥,黑氣貫北斗。五年正月,白氣如虹貫日,久而散。七月戊戌,白氣如陣貫東井。六年四月己巳,白氣東西亙天。丁丑,白氣貫日。



 五月辛
 亥,白氣出昴,至東壁沒。六月辛未,赤氣出貫天。丙子,白氣出河鼓左右旗,分為數道沒。七月癸卯,白氣如彗,起西南。



 景德元年三月,白氣貫軒轅,蒼白氣十餘如布亙天。五月乙巳,白氣數道如芒帚,長七尺許。七月辛亥,黃氣出壁,長五丈餘。十一月癸丑,黑氣十餘道沖日。二年正月丙寅,黃白氣貫月,黑氣環之。二月丁丑,白氣五道貫北斗。十月丙子,白氣出閣道東西,孛孛有光。三年三月丙辰,北方赤氣亙天,白氣貫月。四月癸卯,黃氣如柱貫
 月。十月甲午,黑氣貫北斗魁。四年三月己未,白氣東西亙天。庚申,白氣出南方,長一丈許,久而不散。四月庚午,白氣貫北斗,長十丈。庚寅,白氣如布襲月,三丈許。甲午,南方有黑氣貫心宿,長五丈許。十一月己巳,中天有赤氣如掃,長七尺,在輿鬼南。



 大中祥符元年正月癸亥朔,黃氣出於艮。丁丑,白氣二,東西亙天。七月,西北方白雲氣如彗帚三十餘條。二年九月戊午,黃氣如柱起東南方,長五丈許。三年四月丁巳,中天黑氣東西亙天。十二
 月癸亥,青赤氣貫太微。五年二月壬寅,白氣長五丈,出東井,貫北斗魁及軒轅。七年五月,有氣出紫微為宮闕狀,光燭地。



 天禧二年四月,黃氣如柱貫月。



 天聖七年二月己卯夜,蒼黑雲長三十丈,貫弧矢、翼、軫。



 明道元年十月庚子夜,黃白氣五,貫紫微垣。十二月壬戌,西北有蒼白氣亙天。



 景祐元年八月壬戌,青黃白氣如彗,長七尺餘,出張、翼之上,凡三十三日不見。四年七月戊申夜,黑氣長丈餘,出畢宿下。



 寶元二年正月壬子夜,蒼黑雲起
 西北方,長三十尺,漸東南行,歷婁、胃、昴、畢及火、木,相次中天而散。三月甲寅夜,細黑雲起西北方,長三十丈,貫王良及營室。



 康定元年三月丙子夜,東南方近濁,黑色橫亙數丈,闊尺許,良久散。六月壬子,黑氣起心宿西,長五十丈,首尾侵濁,久之散。



 慶歷元年八月庚辰夜,東方有白氣,長十尺許,在星宿度中,至十日,長丈餘,沖天相,居星宿大星南九十餘日沒。壬午夜,黑氣起西南,長七丈,貫危宿、羽林,入濁,至天津,良久散。癸卯夜,蒼白雲起
 西北,闊二尺許,首尾至濁,良久沒。二年十一月壬申,黑氣貫北斗柄。八月甲申,白雲貫北斗。三年正月戊戌,中天有白氣,長二十丈,向西南行,貫日。



 四月癸卯,白氣二,生西北隅,上中天,首尾至濁,東南行,良久散。七月戊辰,西南生黑氣,長三丈許,經天而散。八月壬子夜,白氣貫北斗魁。四年五月甲子夜,黑氣起東北方,近濁,長五丈許,良久散。九月辛巳夜,中天有氣長二丈許,貫卷舌、南河東北,少頃散。十一月甲子夜,蒼白雲起,南近濁,久方
 散。八年正月丁酉夜,黑氣生,首尾至濁,漸東行,久之乃散。二月辛卯夜,西方近濁生黑氣,長三丈,良久散。



 皇祐四年十一月壬寅夜,黑氣生東方,南北至濁,貫參宿、軒轅。辛酉夜,白氣起北方,近濁,長五丈許,歷北斗,久之散。



 治平元年六月戊午夜,蒼白雲起東北方,長一丈許,貫畢。二年二月乙未夜,蒼黑雲起西北方,長五丈許,貫東井及北斗,良久散。四月癸巳夜,蒼黑雲起西北方,長三十尺,西至軒轅大民,北抵鉤陳。丙午夜,西北方有白氣,
 漸東南行,首尾至濁,貫角宿,移西北,久方散。九月庚申夜,西北蒼黑雲長三丈許,貫營室壁壘陣及天河。三年六月丁未夜,東方有蒼白雲,長一丈許,貫畢。四年二月癸巳夜,蒼白雲起南方,長三丈,闊尺,貫南門星。三月甲寅夜,西南方起蒼白雲二,長三丈,闊尺,相距二尺,貫東井南河,久之乃散。閏三月辛巳夜,蒼黑雲起南方,兩首至濁,闊尺,貫尾、箕、斗、牛、庫樓、騎官。五月戊寅夜,蒼黑雲起北方,長三丈,闊尺,貫紫微垣、王良。壬寅夜,蒼黑雲起
 北方,長三丈,闊尺,貫紫微垣。甲辰夜,蒼黑雲起東方,長丈,闊尺,貫天苑、五車、參旗。六月癸亥夜,白雲起東北方,長五丈,上闊下狹,貫天船、閣道、傳舍、紫微垣、天棓。戊辰夜,黑雲起北方,長三丈,闊尺,貫北斗、紫微垣、王良。八月乙亥夜,黑氣起西北方,長丈,闊尺,貫北斗。十月庚申夜,黃氣一,上下貫月中。十一月丙子夜,蒼黑氣起南方,長五丈,闊二尺,東至庫樓,北至南河,橫貫翼。十二月庚戌夜,蒼黑雲起南方,長三丈,闊二尺,貫五車、東井、五諸侯。



 熙寧元年正月乙酉夜,蒼白雲起西南方,長四丈,闊尺,貫月及南河、輿鬼、軒轅。六月己酉夜,蒼黑雲起北方,長二丈,闊尺,貫北斗魁,東貫文昌。十月庚申夜,蒼黑雲起北方,東西兩首至濁,貫織女、天棓、紫微垣、北斗魁。二年四月甲辰夜,蒼白雲起東南方,長三丈,闊尺,貫天市垣。六月辛酉夜,蒼黑雲起西南方,長四丈,闊二尺,貫大角、左右攝提、天市垣、斗、女、牛。七月甲申,日下有五色雲。十一月,每夕有赤氣見西北隅,如火,至人定乃滅。三年二
 月庚申夜,蒼黑雲起西北方,長三丈,闊二尺,貫王良、扶箱、天廚。六月己未夜,蒼黑雲起西北方,長丈,闊尺,貫五車。又起西北,長丈餘,貫北斗魁、文昌。五年七月丁亥夜,白雲起南方,長丈,貫氐、房、心。六年五月庚申夜,蒼黑雲起東北方,長五丈,闊二尺,貫雲雨、閣道。七年三月壬子,蒼白雲起西南方,長二丈,闊尺,貫日,經中天過,白氣如帶。四月壬申夜,蒼白雲起北方,長五丈,闊二尺,貫北斗魁、鉤陳、王良、閣道,東至奎。丙戌夜,蒼白雲起西北方,長
 三丈,闊尺,貫東井、紫微垣鉤陳。六月辛未夜,蒼黑雲起天河中,長五丈,南北兩首至濁,貫尾、箕;又蒼黑雲起東方,長五丈,貫羽林、外屏。甲戌,蒼白雲起西方,長三丈,貫軫、角、太微。丙戌夜,蒼白雲起南方,長二丈,貫危、室、壁及八魁。丁亥夜,蒼白雲起東方,長二丈,貫月及畢、奎、婁、外屏;又起南方,長二丈,貫危、室、壁及八魁。壬辰夜,蒼白雲起西南方,長二丈,貫天棓、紫微垣。癸巳夜,蒼黑雲起東方,長五丈,貫牛、天倉、歲、太白、卷舌。七月庚戌夜,蒼白雲
 起東方,長丈餘,貫參旗及參。八年二月己巳夜,蒼黑雲起西方,長丈,貫軫、軒轅。乙酉夜,蒼黑雲起東方,長三丈,貫心、天市垣列肆、宗人。五月壬戌夜,蒼黑雲起西南方,長二丈,貫氐、房、心。癸亥,蒼黑雲起西方,長三丈,貫軒轅、太微垣五帝坐。十月庚子夜,黑雲起西北方,長三丈,貫畢、大陵、鉤星。九年四月庚寅夜,白氣起東北方天棓,入天市垣。辛亥夜,蒼黑雲起南方,長二丈,貫庫樓、騎官、積卒、心、尾。六月乙未夜,蒼白雲起東北方,長四丈,貫室、壁、
 閣道。七月己亥夜,蒼黑雲起南方,長四丈,貫軍市、天園。十月乙酉夜,蒼黑雲起西北方,長四丈,貫北斗、鉤、車府。十年六月癸未夜,蒼黑雲起南方,長三丈,闊尺,貫龜、鱉、天淵。乙巳夜,蒼白雲起東北方,長三丈,闊尺,貫五車及畢。七月丙子夜,蒼黑雲起北方,長丈,貫北斗魁。八月庚辰,蒼黑雲起東北方,長二丈,貫參、井、北河、五諸侯。九月庚申夜,蒼黑雲起北方,由北斗魁杓貫紫微垣,至天棓。十月辛丑夜,蒼黑雲起南方,長二丈,貫斧鉞、鈇鍎。



 元豐
 二年四月戊申夜,白雲起南方,長三丈,貫庫樓、積卒、龍尾。辛亥夜,蒼白雲起南方,長三丈,貫房。王年四月壬申夜,蒼白雲起北方,長二丈,出太微垣,貫五帝坐、常陳。八年十月庚申夜,蒼黑雲生北方,長三丈,闊尺,貫北斗、文昌、天槍。



 元祐三年七月戊辰夜,東北方近濁,天明照地,如月將出,偏西北有白氣經天。九月己酉夜,赤氣起北方,漸生白氣數道。



 紹聖二年十一月,桂陽監慶雲見。



 元符二年九月戊辰夜,赤氣起北方紫微垣北斗星東南,
 次有白氣十道,各長五尺。



 崇寧元年十一月己酉,赤氣隨日沒。二年五月戊子夜,蒼白雲起東南方,長三丈,貫尾、箕、斗。



 政和元年十一月甲戌夜,蒼白氣起紫微垣,貫四輔。五年四月庚子,有白雲自北直徹中天,漸成五色,如華蓋。七年五月乙卯夜,赤雲、白氣起東北方。



 宣和元年六月辛巳夜,赤氣起北方,半天如火。七月戊午夜,赤雲起東北方,貫白氣三十餘道。二年二月戊戌夜,赤雲起東北,漸向西北,入紫微垣。三年九月壬午夜,蒼白氣
 長三丈,貫月。四年九月丁丑,西方日下有赤氣。七年四月壬子夜,有赤雲入紫微垣。



 靖康元年正月丁丑夜,赤白氣起西方。九月戊寅,有赤氣隨日出。九月乙未,西方日下有赤氣。十一月乙丑,日下有赤氣。閏十一月丁酉,赤氣亙天。二年正月己亥夜,四北陰雲中有火光,長二丈餘,闊數尺,時時見。二月壬午夜,白氣如虹,自南亙北,漸移西南至東北。



 三月戊子夜,白氣貫斗。



 建炎元年八月壬申,東北有赤氣。四年五月壬子,赤雲亙天中,有白
 氣十餘道,貫之如練,起於紫微,犯北斗及文昌,由東南而散。



 紹興元年二月己巳,白氣亙天。七年正月辛未夜,東北赤氣如火,出紫微宮;二月癸卯,又如之。十一月癸卯,有赤雲如火,隨日入。八年九月甲申朔夜,有赤氣如火,出紫微垣內。十八年八月丁亥,西北方赤氣如火。二十七年二月乙酉,赤氣出紫微垣。十月壬寅,赤氣隨日出。三十年正月壬申,東北方赤氣一帶五處如火影。十一月甲午,西南方白氣自尾歷壁、婁、昴宿。十二月戊申,
 其夜白氣出尾宿,歷心、房、氐、亢、角,入天市,貫太微,至郎位止,有類天漢。三十一年十二月辛丑,其夜,白氣出鬥宿,歷牛、女、危,至婁止,約廣六丈,類天漢,東西亙天。



 隆興元年十二月壬午,其夜,白氣出危宿,歷室、壁、奎、胃、婁至昴止。二年十一月庚寅,其日,赤雲氣遍天,隨日入。



 乾道元年正月庚午,其夜,白氣出奎宿,漸上,經婁、胃、昴,貫畢,入參宿內止。三月戊辰,其夜,白氣自參宿至角宿止,與天漢相接,約廣七丈。四月丁酉,其夜,蒼白氣自西北漸
 上,東北入天市垣;辛丑,入北斗魁中及入文昌星;乙巳,入紫微垣內至北極、天樞中。



 十月己丑,蒼白雲氣長二丈,穿入翼宿。十一月丙寅,白氣出女宿,歷虛、危、室、壁、奎、婁、胃宿,入昴宿止。二年十二月庚子,白氣亙天。六年十月庚午,赤氣隨日出。十一月丁丑,赤氣隨日入。七年七月壬寅,赤氣隨日入。十月己未,赤氣隨日出。八年十月乙巳,赤氣隨日入;丙午,隨日出。九年十月壬申,其日,矞雲見。



 淳熙元年十月戊寅,東北方生曲虹。三年八月丁
 酉,赤氣隨日入;戊戌,隨日出。五年十月丁巳,生曲虹。十年正月戊子,西南有白氣,如天漢而明,南北廣六丈,東西亙天。十四年十一月甲寅,赤氣隨日入。



 紹熙四年十一月甲戌夜,赤雲、白氣見。五年六月壬寅,白敢如帶亙天;己酉,又如之。



 慶元四年八月庚辰,白氣如帶亙天。五年二月癸酉夜,白氣如帶亙天,八月癸亥,又如之。



 嘉泰四年二月庚申,赤氣亙天。十一月壬申,其日,白氣如帶亙天。癸酉,虹見。



 嘉定六年十月乙卯,赤氣隨日出;十一
 月辛卯,隨日入。



 嘉熙四年二月丙辰,白氣亙天。



 淳祐二年二月癸丑朔,白氣亙天。十年十一月丁丑,虹見。



 景定三年七月甲申夜,白氣亙天,如匹布。



\end{pinyinscope}