\article{本紀第十八}

\begin{pinyinscope}

 哲宗二



 紹聖元年春正月癸酉朔,群臣詣西上閣門進名奉慰。丙申,夏人來貢。辛丑,遣中書舍人呂希純等行河。罷河東大銅錢。



 二月丁未,以戶部尚書李清臣為中書侍郎,
 兵部尚書鄧潤甫為尚書右丞。己酉,葬宣仁聖烈皇后於永厚陵。己未,祔神主於太廟。癸亥,減兩京、河陽、鄭州囚罪一等,民緣山陵役者蠲其賦。甲子,詔依章獻明肅皇后故事,罷避高遵惠諱。



 三月壬申朔,日有食之。乙亥,呂大防罷。庚辰,詔大學合格上舍生推恩免省試,附科場春榜。乙酉,御集英殿策進士。丁亥,策武舉。戊子,以徐王顥為太師,徙封冀王。癸巳,詔振京東、河北流民,貸以穀麥種,諭使還業,蠲是年租稅。丁酉,賜禮部奏名進士、
 諸科及第出身九百七十五人。蘇轍罷。



 夏四月乙巳朔,阿裏骨進獅子。丙午,以旱,詔恤刑。己酉,詔中外決獄。庚戌,詔有司具醫藥治京師民疾。壬子,蘇軾坐前掌制命語涉譏訕,落職知英州。癸丑,改元。白虹貫日。甲寅,以王安石配饗神宗廟庭。蔡確追復右正議大夫。戊午,復新城兩廂。庚申,減四京囚罪一等,杖以下釋之。壬戌,以資政殿學士章惇為尚書左僕射兼門下侍郎。范純仁罷。丙寅,罷五路經、律、通禮科。丁卯,詔諸路復元豐免役法。
 戊辰,同修國史蔡卞請重修《神宗實錄》。



 閏月壬申,復提舉常平官。癸酉,罷十科舉士法。甲申,以觀文殿學士安燾為門下侍郎。丙戌,復義倉。丁亥,詔神宗隨龍人趙世長等遷秩、賜繼有差。戊子,詔在京諸司,所受傳宣中批,並候朝廷覆奏以行。乙未,西南張蕃遣人入貢。丙申,命左僕射章惇提舉修《神宗國史》。丁酉,詔添差徐州兵馬都監。



 五月壬寅,罷修官制局。甲辰,罷進士習試詩賦,令專二經,立宏詞科。己酉,修國史曾布請以王安石《日錄》
 載之《神宗實錄》。太白晝見。辛亥,劉奉世罷。癸丑,詔中外學官,非制科、進士、上舍生入官者並罷。編類元祐群臣章疏及更改事條。甲寅,右正言張商英言先帝謂天地合祭非古,詔禮部、太常詳議以聞。乙丑,鄧潤甫卒。丁卯,嗣濮王宗暉薨。



 六月甲戌,來之邵等疏蘇軾詆斥先朝,詔謫惠州。丙子,罷制置解鹽使。壬午,封高密郡王宗晟為嗣濮王。癸未,以翰林學士承旨曾布同知樞密院事。甲申,除進士引用王安石《字說》之禁。



 秋七月丁巳,以御
 史黃履、周秩、諫官張商英言,奪司馬光、呂公著贈謚,王巖叟贈官;貶呂大防為秘書監,劉摯為光祿卿,蘇轍為少府監,並分司南京;梁燾提舉舒州靈仙觀。戊午,詔:「大臣朋黨,司馬光以下各輕重議罰,布告天下。餘悉不問,議者亦勿復言。」八月丙戌,召輔臣觀稼後苑。日有五色雲。壬辰,應制科趙天啟以累上書狂妄黜。



 九月癸卯,遣御史劉拯按河北水災,振饑民。丙午,御集英殿,策賢良方正能直言極諫科。庚戌,罷制科。罷廣惠倉。癸丑,令監
 司歲察守臣課績優者以聞。甲寅,知廣州唐義問坐棄渠陽砦,責授舒州團練副使。庚申,太白晝見。丁卯,詔京東西、河北振恤流民。戊辰,流星出紫微垣。



 冬十月丙申,三佛齊遣使入貢。丁酉,河北流斷絕。



 十一月己亥朔,復八路差官法。壬子,以冬無雪,決系囚。蔡確特追復觀文殿大學士。甲寅,開封男子呂安斥乘輿,當斬,貸之。丁己,詔河北振饑,諸路恤流亡,官吏有善狀、才能顯著者以聞。十二月辛未,申嚴銅錢出外界法。庚辰,命諸路祈雪。丙
 戌,滑州浮橋火。己丑,漳河決溢,浸洺、磁等州,令計置堙塞。甲午,範祖禹、趙彥若、黃庭堅坐史事責授散官,永、澧、黔州安置。是歲,京師疫,洛水溢,太原地震,河北水,發京東粟振之。



 二年春正月甲辰,詔國史院增補先帝御集。丙午,立宏詞科。己未,遷奉太平興國寺三朝御容於天章閣。乙丑,殿前司奏獄空,詔賜緡錢。



 二月乙亥,呂大防以監修史事貶秩,分司南京、安州居住。辛巳,出內庫錢帛二十萬
 助河北振饑。甲午,罷廣文館解額。



 三月己亥,宗晟薨。己未,試宏詞黃符等五人各循一資。



 夏四月戊辰,詔職事官罷帶職,朝請大夫以下勿分左右,易集賢院學士為集賢殿修撰,直集賢院為直秘閣,集賢校理為秘閣校理。壬申,封華容郡王宗愈為嗣濮王。詔許將等七人,不限資格,各舉才行堪備任使者二人。丁亥,詔依元豐條置律學博士二員。



 五月乙巳,命蔡卞詳定國子監三學及外州州學制。乙卯,六月壬辰,禁京城士人輿轎。上皇太妃宮名曰聖瑞。



 秋七月丙辰,詔大理寺復置右治獄,仍依元豐例添置官屬。



 八月壬申,命彰信軍節度使宗景為開府儀同三司,封濟陰郡王。甲申,宗愈薨。乙酉,錄趙普后希莊為閣門祗候。



 九月甲午,以安定郡王宗綽為嗣濮王。壬寅,告遷神宗神御於景靈宮顯承殿。癸卯,詣景靈宮,行奉安禮。戊申,加上神宗謚曰紹天法古運德建功英文烈武欽仁聖孝皇帝。己酉,朝獻景靈宮。庚戌,朝饗太廟。辛亥,大饗明堂,赦天下。



 冬十月甲子,鄭雍
 罷。癸酉,告遷宣仁聖烈皇后神御於景靈宮徽音殿。甲戌,詣宮行奉安禮。以吏部尚書許將為尚書左丞,翰林學士蔡卞為尚書右丞。辛巳,進封冀王顥為楚王。辛卯。河南府地震。



 十一月乙未,安燾罷知河南府。丙申,太白晝見。戊戌,範鍔自轉運使入對,言有捕盜功,乞賜章服。帝曰:「捕盜,常職也,何足言功?」黜知壽州。甲寅,梁惟簡除名、全州安置。丙辰,贈蔡確為太師,賜謚忠懷。十二月乙丑,復置監察御史三人,分領六察,不言事。令翰林學士
 蔡京、御史中丞黃履各舉御史二人。壬申,白虹貫日。戊子,詔如元豐例,孟月朝獻景靈宮。是歲,蘇州夏、秋地震。桂陽監慶雲見。出宮女九十一人。交址、三佛齊、韋蕃、阿裏骨入貢。



 三年春正月庚子,韓忠彥罷知真定府。甲辰,朝獻景靈宮,遍詣諸殿,如元豐禮。庚戌,引見蕃官包順、包誠等,賜繼有差。詔鞫獄非本章所指而蔓求他罪者,論如律。乙卯,詔戶部尚書勿領右曹。戊午,詔罷合祭,間因大禮之
 歲,夏至日躬祭地祇於北郊。



 二月癸亥,出元豐庫緡錢四百萬於陜西、河東糴邊儲。辛未,復元豐《恤孤幼令》。癸酉,罷富弼配饗神宗廟庭。癸未,詔封濮王子未王者三人:宗楚為南陽郡王,宗祐為景城郡王,並開府儀同三司;宗漢為東陽郡王。乙酉,宗綽薨。丙戌,詔三歲一取旨,遣郎官、御史按察監司職事。丁亥,夏人寇義合砦。



 三月壬辰,以禁中屢火,罷春宴及幸池苑,不御垂拱殿三日。癸巳,夏人圍塞門砦。丁酉,尚書省火。戊午,劍南東川地
 震。己亥,封宗楚為嗣濮王。辛亥,封大寧郡王佖為申王,遂寧郡王佶為端王。丁巳,幸申王、端王府。



 夏四月辛酉,罷宣徽使。丙子,詔自今景靈宮四孟朝獻,分為二日。



 五月壬子,太白晝見。丙辰,錄囚。



 六月癸亥,令真定立趙普廟。乙酉,立北郊齋宮於瑞聖園。



 秋七月庚戌,依元豐職事官以行、守、試三等定祿秩。罷元祐所增聚義錢。甲寅,令熙河立王韶廟。



 八月辛酉,夏人寇寧順砦。壬戌,日上有五色暈,下有五色氣。己卯,復置檢法官。庚辰,以範祖
 禹、劉安世在元祐中構造誣謗,祖禹責授昭州別駕、賀州安置,安世新州別駕、英州安置。



 九月己亥,邈川首領阿裏骨卒。己酉,滁、沂二州地震。壬子,楚王顥薨。乙卯,廢皇後孟氏為華陽教主、玉清妙靜仙師,賜名沖真。



 冬十月丁巳朔,以楚王薨,罷文德殿視朝。壬戌,夏人寇鄜、延,陷金明砦。戊辰,詔被邊諸路相度城砦要害,增嚴守備。辛未,西南方雷聲,雨雹。癸酉,鐘傳言築汝遮,詔以為安西城。



 十一月丁未,章惇上《神宗實錄》。庚戌,宴修實錄官。
 十二月辛酉,宗景坐以立妾罔上,罷開府儀同三司、判大宗正司事。癸酉,置施州鑄錢廣積監。甲戌,蔡京上《新修大學敕令式》、《詳定重修敕令》。遺棄饑貧小兒三歲以下,聽收養為真子孫。是歲,於闐、大食、龜茲師王國、西南蕃龍氏、羅氏入貢。宗室子授官者四十六人。



 四年春正月丙戌朔,不受朝。群臣及遼使詣東上閣門拜表賀。班內外學制。庚寅,以阿裏骨子瞎征襲河西軍節度使、邈川首領。甲午,涇原路鈐轄王文振敗夏人於
 沒煙峽。庚戌,李清臣罷。



 二月乙未,以三省言,追貶呂公著為建武軍節度副使,司馬光為清遠軍節度副使,王巖叟為雷州別駕,奪趙瞻、傅堯俞贈謚,追韓維致仕及孫固、範百祿、胡宗愈遺表恩。詔江、淮巡檢依舊法招置土兵。癸亥,於闐來貢,黑汗王攻夏人三州,遣其子以聞。丙寅,夏人寇綏德城。庚午,詔國信使毋得以非例之物遺人使,仍著條禁。癸酉,詔申王佖、端王佶歲賜錢各六千五百緡。丙子,進神宗婉儀宋氏為賢妃。己卯,復元豐
 榷茶法。庚辰,罷《春秋》科。癸未,以三省言,追貶呂大防為舒州團練副使,劉摯為鼎州團練副使,蘇轍為化州別駕,梁燾為雷州別駕,范純仁為武安軍節度副使,安置於循、新、雷、化、永五州;劉奉世為光祿少卿、分司南京;黜韓維以下三十人輕重有差。甲申,降文彥博為太子少保。



 閏月丙戌朔,張天說坐上書詆訕先朝處死。壬寅,以曾布知樞密院事,許將為中書侍郎,蔡卞為尚書左丞,吏部尚書黃履為尚書右丞,翰林學士林希同知樞密院
 事。癸卯,大雨雹。甲辰,蘇軾責授瓊州別駕,移昌化軍安置。範祖禹移賓州安置,劉安世移高州安置,己酉,御集英殿策進士。庚戌,策武舉。



 三月壬戌,夏人犯麟州神堂堡,出兵討之,及進築胡山砦。癸亥,賜禮部奏名進士、諸科及第出身六百九人。甲子,詔武舉謝師古等以遠人賜帛,李惟岳以高年賜帛。丁卯,詔瀘南安撫司、南平軍毋擅誘楊光榮獻納播州疆土。庚午,夏人大至葭盧城下,知石州張構等擊走之。甲戌,幸金明池。丙子,克胡山
 新砦成,賜名平羌砦。辛巳,西上閣門使折克行破夏人於長波川,斬首二千餘級,獲牛馬倍之。壬午,命官編類司馬光等改廢法度論奏事狀。



 夏四月丁亥,令諸獄置氣樓涼窗,設漿飲薦席,杻械五日一浣,系囚以時沐浴,遇寒給薪炭。甲午,熙河築金城關。丙申,詔發解省試添策一道。丁酉,進編臣僚章疏一百四十三帙。己亥,呂大防卒於虔州。庚子,知保安軍李沂伐夏國,破洪州。壬寅,環慶鈐轄張存入鹽州,俘戮甚眾,及還,夏人追襲之,復
 多亡失。甲辰,置克戎砦、平夏城,置靈平砦。丁未,以西邊板築有勞,曲赦陜西、河東路。追貶王珪為萬安軍司戶參軍。己酉,覆文德殿侍從轉對。



 五月丁巳,文彥博薨。辛酉,以皇太妃服藥及亢旱,決四京囚。壬戌,詔陜西添置蕃落馬軍十指揮。丁卯,廢衛州淇水第二馬監、穎昌府單鎮馬監。辛未,韓縝薨。丁丑,貶韓維為崇信軍節度副使。



 六月癸未朔,日明食之。丁亥,太白犯太微垣。戊子,宗楚薨。丙申,詔翰林學士、吏部尚書各舉監察御史二人。
 丁酉,環慶路安疆砦成,詔防托蕃漢官賜帛有差。甲辰,熙河進築青石峽畢工,賜名西平。乙巳,保寧軍觀察留後宗漢為開府儀同三司,徙封安康郡王。己酉,太原地震。太白晝見。



 秋七月壬子朔,太白晝見。



 八月乙酉,封湖州觀察使世開為安定郡王。丙戌,鄜延將王愍復宥州。戊戌,封宗祐為嗣濮王。築威戎城。己酉,彗出西方。



 九月壬子,以星變,避殿減膳,罷秋宴,詔公卿悉心修政,以輔不逮,求中外直言。乙卯,赦天下,出元豐庫緡錢四百萬
 付陜西廣糴,詔歸明人未給田者舍以官舍。戊辰,彗滅。癸酉,謁中太一宮為民祈福。丙子,御殿復膳。命宗景為開府儀同三司。己卯,封婉儀劉氏為賢妃。



 冬十月戊戌,宗景薨。壬寅,廢安國、安陽淇水監及洛陽原武監。



 十一月丁卯,詔諫議大夫以上各舉監察御史一人。癸酉,貶劉奉世為隰州團練副使、郴州安置。丁丑,詔放歸田里程頤涪州編管。十二月癸未,劉摯卒。甲申,曲宴遼使於垂拱殿。乙酉,侍御史董敦逸坐奏對不實,貶秩、知興國
 軍。是歲,兩浙旱饑,詔行荒政,移粟振貸。出宮女二十四人。宣城民妻一產四男子。於闐、西南蕃羅氏入貢。播州夷楊光榮等內附。戶部主戶一千三百六萬八千七百四十一,丁三千三十四萬四千二百七十四;客戶六百三十六萬六千八百二十九,丁三百六萬七千三百三十二。大闢三千一百九十二人。



 元符元年春正月庚戌朔,不視朝。丙寅,咸陽民段義得玉印一紐。甲戌,幸瑞聖園,觀北郊齋宮。



 二月丙戌,白虹
 貫日。庚寅,詔建五王外第。壬辰,復罷翰林侍讀、侍講學士。丁酉,宗祐薨。戊申,知蘭州王舜臣討夏人於塞外。築興平城。



 三月壬子,令三省、樞密吏三歲一試刑法。甲寅,開楚州通漣河。丙辰,米脂砦成。丁巳,五王外第成,賜名懿親宅。戊午,封宗漢為嗣濮王。殺朱崖流人陳衍。壬戌,申王佖、端王佶並為司空。令太常寺與閣門修定刈麥儀。乙丑,詔翰林學士承旨蔡京等辯驗段義所獻玉璽,定議以聞。戊辰,吏部郎中方澤等坐私謁後族宴聚,罰
 金補外。庚午,幸申王府。辛未,幸端王府。甲戌,進封咸寧郡王俁為莘王,普寧郡王似為簡王,祁國公偲為永寧郡王。丙子,築熙河通會關。



 夏四月庚辰,世開薨。甲申,幸睿成宮及莘王、簡王府。丙戌,章惇等進《神宗帝紀》。梁燾卒於化州。壬辰,林希罷。丙申,建顯謨閣,藏《神宗御集》。庚子,幸睿成宮。壬寅,學士院上《寶璽》、《靈光》、《翔鶴》樂章。癸卯,詔學官增習兩經。丁未,曾布上《刪修軍馬敕例》。



 五月戊申朔,御大慶殿,受天授傳國受命寶,行朝會禮。己酉,班
 德音於天下,減囚罪一等,徒以下釋之。癸丑,受寶,恭謝景靈宮。戊午,宴紫宸殿。庚申,詔獻寶人段義為右班殿直,賜絹二百匹。



 六月戊寅朔,改元。丙戌,遣官分詣鄜延、涇原、河東、熙河按驗所築城砦。甲午,蔡京等上《常平免役敕令》。



 秋七月乙卯,詔增置大府丞一員。乙丑,敕大禮五使自今並差執政官,定為令。丁卯,令學官試《三經》。庚午,詔範祖禹移化州安置,劉安世梅州安置,王巖叟、朱光庭諸子並勒停不敘。壬申,京師地震。



 八月丙子朔,熙
 河蘭岷路復為熙河蘭會路。庚辰,詔自今三省、樞密院進擬在京文臣、開封推判官、武臣橫班使副及諸路監司、帥守,並取旨召對。丁亥,詔侍從中書舍人以上各舉所知二人,權侍郎以上舉一人,仍指言所堪職任。



 九月丁未,以霖雨,罷秋宴。庚戌,秦觀除名,移雷州編管。癸亥,賜王安石第於京師。冬十月乙未,詔武官試換文資。丁酉,以河北、京東河溢,遣官振恤。乙亥,夏人寇平夏城。癸卯,附馬都尉張敦禮坐元祐初上疏譽司馬光,奪留後,
 授環衛官。



 十一月壬戌,朝獻景靈宮。癸亥,朝饗太廟。甲子,祀昊天上帝於圜丘,赦天下。是歲,澶州河溢,振恤河北、京東被水者。真定府、祁州野蠶成繭。涇原路禽夏國統軍嵬名阿埋等,高麗、瞎徵、西南蕃張氏、羅氏、程氏入貢。西蕃首領李訛□移、巴詘支、呂承信等內附。



 二年春正月甲辰朔,御大慶殿,以雪罷朝,群臣及遼使詣東上閣門拜表賀。群臣又詣內東門,賀如儀。丁卯,出內金帛二百萬,備陜西邊儲。



 二月甲戌朔,令監司舉本
 路學行優異者各二人。韋蕃入貢。己卯,詔許高麗國王遣士賓貢。辛巳,增置神臂弓,詔自今應被旨舉官,所舉不當,具舉主姓名以聞。甲申,夏人以國母卒,遣使告哀,且謝罪,卻其使不納。戊子,鄜延鈐轄劉安敗夏人於神堆。甲午,大食入貢。乙未,詔吏部:守令課績,從御史臺考察,黜其不實者。



 三月丙辰,遼人遣簽書樞密院事蕭德崇來為夏人請緩師,仍獻玉帶。築環慶路定邊城。丁巳,秦鳳經略司言吳名革率部族、孳畜歸順。詔名革補內
 殿承制,首領李□移補右侍禁,及賜錢帛有差。庚申,知府州折克行獲夏國鈐轄令王皆保。乙丑,祈雨。己巳,莘王俁為司空。



 夏四月庚辰,幸莘王府。令廣西提點刑獄司兼領鹽事。丙戌,築鄜延、河東路暖泉、烏龍砦。丁亥,以旱,減四京囚罪一等,杖以下釋之。辛卯,詔鞫獄,徒以上須結案及錄審覆奏,然後斷遣,不如令者坐之。癸巳,封永嘉郡王偲為睦王。遣中書舍人郭知章報聘於遼。丁酉,築威羌城。



 五月甲辰,太白晝見。庚戌,築鄜延路金湯城。
 癸亥,奉遷真宗神御於萬壽觀延聖殿。曲赦陜西、河東路,減囚罪一等,流以下釋之。建西安州及天都等砦。乙丑,進章惇官五等,曾布三等,許將、蔡卞、黃履皆二等。辛未,詔莘王俁、睦王偲母進封婕妤。



 六月庚辰,賜蘭、會州新砦名會川城。甲午,賜環慶路之字平曰清平關。戊戌,築定邊、白豹城訖工,閣門使張存等轉官、賜金帛有差。



 秋七月乙巳,盛暑,中外決系囚。丁未,放在京工役。庚戌,河北河漲,沒民田廬,遣官振之。甲子,知環州種樸獲夏
 國監軍訛勃囉。丙寅,洮西安撫使王贍復邈川城,西蕃首領欽彪阿成以城降。



 八月癸酉,章惇等進《新修敕令式》。惇讀於帝前,其間有元豐所無而用元祐敕令修立者,帝曰:「元祐亦有可取乎?」惇等對曰:「取其善者。」甲戌,太原地震。戊寅,皇子生。辛巳,降德音於諸路:減囚罪一等,流以下釋之。乙酉,賜熙河路緡錢百萬撫納部族。丁亥,復修會州。癸巳,太白晝見。瞎徵降。甲午,建葭蘆戍為晉寧軍。丙申,保寧軍節度呂惠卿特授檢校司空。



 九月
 庚子朔,夏人來謝罪。癸卯,命御史點檢三省、樞密院,並依元豐舊制。甲辰,幸儲祥宮。乙巳,幸醴泉觀。丁未,立賢妃劉氏為皇后。己未,青唐酋隴拶以城降。壬戌,雨,罷秋宴。甲子,右正言鄒浩論劉氏不當立,特除名勒停、新州羈管。丙寅,御文德殿冊皇后。閏月癸酉,置律學博士員。詔詳議廟制。以青唐為鄯州、隴右節度。邈川為湟州,宗哥城為龍支城,俱隸隴右。戊寅,以廓州為寧砦城。丙戌,果州團練使仲忽進古方鼎,志曰「魯公作文王尊彞」。甲
 午,熒惑犯太微垣左執法。己未,越王茂薨。



 冬十月壬子,詔河北大名二十二州軍置馬步軍指揮,以廣威、保捷為名。甲寅,日有食之,既。



 十一月丁亥,詔綏德城為綏德軍。壬辰,詔河北黃河退灘地聽民耕墾,免租稅三年。乙未,詔諸州置教授者,依太學三舍法考選生徒升補。是月,河中猗氏縣民妻一產四男子。



 三年春正月辛未,帝有疾,不視朝。丁丑,奉安太宗皇帝御容於景靈宮大定殿。戊寅,大赦天下,蠲民租。己卯,帝
 崩。皇太后諭遺制,立弟端王即位於柩前,皇太后權同處分軍國事。



 四月己未,上謚曰欽文睿武昭孝皇帝,廟號曰哲宗。七月丁卯,以謚號冊寶奏告天地、宗廟、社稷。八月壬寅,葬於永泰陵。癸亥,祔太廟。崇寧三年七月,加謚曰憲元繼道世德揚功欽文睿武齊聖昭孝皇帝。政和三年,改謚憲元繼道顯德定功欽文睿武齊聖昭孝皇帝。



 贊曰:折宗以沖幼踐阼,宣仁同政。初年召用馬、呂諸賢,
 罷青苗,復常平,登俊良,闢言路,天下人心,翕然向治。而元祐之政,庶幾仁宗。奈何熙、豐舊奸蘗去未盡,已而媒薛復用,卒假紹述之言,務反前政,報復善良,馴致黨籍禍興,君子盡斥,而宋政益敝矣。籲,可惜哉!



\end{pinyinscope}