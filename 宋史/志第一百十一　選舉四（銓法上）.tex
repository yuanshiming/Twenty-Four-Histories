\article{志第一百十一 選舉四(銓法上)}

\begin{pinyinscope}

 太祖設官分職,多襲五代之
 制,稍損益之。凡入仕,有貢舉、奏蔭、攝署、流外、從軍五等。吏部銓惟注擬州縣官、幕職,兩京諸司六品以下官皆無選;文臣少卿、監以上中書主之,京朝官則審官院主之;武臣剌史、副率以上內職,樞密院主之,使臣則三班院主之。其後,典選之職分為四:文選曰審官東院,曰流內銓,武選曰審官西院,曰三班院。元豐定制而後,銓注之法,悉歸選部:以審官東院為尚書左選,流內銓為侍郎左選,審官西院為尚書右選,三班院為侍郎右選,於是吏部有
 四選之法。文臣寄祿官自朝議大夫、職事官
 自大理正以下,非
 中
 書
 省敕授者,歸尚書左選;武臣升朝官自皇城使、職事官自
 金吾街仗司以下,非樞密院宣授者,歸尚書右選;自初仕至州縣幕職官,歸侍郎左選;自借差、監當至供奉官、軍使,歸侍郎右選。凡應注擬、升移、敘復、蔭補、封贈、酬賞,隨所分隸校勘合格,團甲以上尚書省,若中散大夫、閣門使以上,則列選敘之狀上中書省、樞密院,得畫旨,給告身。



 凡選人階官為七等:其一曰三京府判官,留守判官,節度、觀察判官;即後來承直郎。



 其二曰節度掌書記,觀察支使,防禦、團練判官;即後來儒林郎。其三曰軍事判官,京府、留
 守、節度、觀察推官;即後來文林郎。



 其四曰防禦、團練、軍事推官,軍、監判官;即後來從事郎。



 其五曰縣令、錄事參軍;即後來從政郎。



 其六曰試銜縣令、知錄事;即後來修職郎。



 其七曰三京軍巡判官,司理、戶曹、司戶、法曹、司法參軍,主簿,縣尉。即後來迪功郎。七階選人須三任六考,用奏薦及功賞,乃得升改。



 凡改官,留守、兩府、兩使判官,進士授太常丞,舊亦授正言、監察或太常博士,後多不除。



 餘人太子中允;舊亦授殿中丞。



 支使,掌書記,防禦、團練判官,進士授太子中允,或秘書郎。



 餘人著作佐郎;兩使推官、軍事判官、
 令、錄事參軍,進士授著作佐郎,餘人大理寺丞;初等職官知縣,知錄事參軍,防禦、團練、軍事推官,軍、監判官,進士授大理寺丞,餘人衛尉寺丞;惟判、司、主簿、縣尉七考,進士授大理寺丞,餘人衛尉寺丞。自節、察判官至簿、尉,考不及格者遞降等。



 凡非登科及特旨者,年二十五方注官。凡三班院,二十以上聽差使,初任皆監當,次任為監押、巡檢、知縣。凡流外人,三任七考,有舉者六員,移縣今、通判;有班行舉者三員,與磨勘。凡進納人,六考,有
 職官或縣令舉者四員,移注;四任十考,有改官者五人舉之,與磨勘。



 初定四時參選之制:凡本屬發選解,並以四孟月十五日前達省,自千里至五千里外,為五等日期離本處;若違限及不如式,本判官罰五十直,錄事參軍、本曹官各殿一選;諸州四時具員闕報吏部,逾期及漏誤,判官罰七十直,錄事參軍以下殿一選;在京百司發選解及送闕,違期亦有罰;諸歸司官奏年滿,俟敕下,準格取本司文解赴集,流外銓則據其人自投狀申奏,
 亦依四時取解參選;凡州縣老疾不任事者,許判官、錄事參軍糾舉以聞,判官、錄事參軍則州長吏糾之。藩郡監牧,每遣朝臣攝守,往往專恣。太祖始削外權,命文臣往蒞之;由是內外所授官,多非本職,惟以差遣為資歷。



 建隆四年,詔選朝士分治劇邑,以重其事。大理正奚嶼知館陶,監察御史王祐知魏,楊應夢知永濟,屯田員外郎於繼徽知臨清,常參官宰縣自此始。舊制,畿內縣赤,次赤,畿外三千戶以上為望,二千戶以上為緊,一千戶
 以上為上,五百戶以上為中,不滿五百戶為中下。有司請據諸道所具板圖之數,升降天下縣,以四千戶以上為望,三千戶以上為緊,二千戶以上為上,千戶以上為中,不滿千戶為中下。自是,注擬以為資敘。又詔:「周廣順中應出選門州縣官,於南曹投狀,準格敕考校無礙,與除官;其敘復者,刑部檢勘送銓。」



 先是,選格未備。乾德二年,命陶穀等議:



 凡拔萃、制舉及進士、《九經》判中者,並入初等職官,判下者依常選。初入防禦團練軍事推官、軍
 事判官者,並授將仕郎,試校書郎。周三年得資,即入留守兩府節度推官、軍事判官,並授承奉郎,試大理評事。又周三年得資,即入掌書記、防禦團練判官,並授宣德郎,試大理評事兼監察御史。周二年得資,即入留守、兩府、節度、觀察判官,並授朝散大夫,試大理司直兼監察御史。周一年,入同類職事、諸府少尹。又周一年,送名中書門下,仍依官階,分為四等。已至兩使判官以上、次任入同類職事者,加檢校官或轉憲銜。凡觀察判官以
 上,緋十五年乃賜紫。每任以周三年為限,閏月不預,每周一年,校成一考。其常考,依令錄例,書「中」、「上」;公事闕遺、曾經殿罰者,即降考一等;若校成殊考,則南曹具功績,請行酬獎;或考滿末代,更一周年與成第四考,隨有罷者不赴集;其奏授職事,書校考第,並準新格參選。



 自是銓法漸有倫矣。帝又慮銓曹惟用資歷,而才傑或湛滯,乃詔吏部取赴集選人歷任課績多而無闕失、其材可副升擢者,送中書引驗以聞。時仕者愈眾,頗委積不可
 遣。



 開寶初,令選人應格者,到京即赴集,不必限四時;及成甲次,又給限:南曹八日,銓司旬有五日,門下省七日,自磨勘、注擬及點檢謝詞,總毋逾一月。若別論課績,或負過咎須考驗,行遣如法;及資考未合注擬者,不在此限。



 三年,詔曰:「吏多難以求其治,祿薄末可責其廉,與其冗員重費,不若省官益奉。州縣官宜以戶口為率,差減其員,舊奉月增給五千。西川管內諸州,凡二萬戶,依舊設曹官三員;戶不滿二萬,置錄事參軍、司法參軍各一
 員,司法兼司戶;不滿萬戶,止置司法、司戶,司戶兼錄事參軍;戶不滿五千,止置司戶,兼司法及錄事參軍。縣千戶以上,依舊置令、尉、主簿凡三員;戶不滿千,置令、尉,縣令兼主簿事;戶不滿四百,止置主簿、尉,以主簿兼知縣事;戶不滿二百,止置主簿,兼令、尉。」諸道減員亦仿此制。西川官考滿得代,更不守選。



 嶺表初平,上以其民久困苛政,思惠養之。令吏部銓自襄、荊以南州縣,選見任年未五十者,移為嶺南諸州通判,得攜族之官。以廣南偽
 署官送學士院試書判,稍優則授上佐、令、錄、簿、尉。初,州縣有闕員,差前資官承攝;帝以其紊常制,令所在即上闕員,有司除注。又謂:「諸道攝官或著吏能,悉令罷去,良可惜也。有司按其歷任,三攝無曠敗者以名聞。」



 六年,從流內銓之請,復四時選,而引對者每季一時引對之。時國家取荊、衡,克梁、益,下交、廣,闢土既遠,吏多闕,是以歲常放選。選人南曹投狀,判成送銓,依次注擬。其後選部闕官,即特詔免解,非時赴集,謂之「放選」,習以為常,而取
 解季集之制漸廢。是冬,乃命參知政事盧多遜等,以見行《長定》、《循資格》及泛降制書,乃正違異,削去重復,補其闕漏,參校詳議,取悠久可用者,為書上之,頒為永式,而銓綜之職益有敘矣。



 先是,選人試判三道,其二全通而文翰俱優為上,一道全通而文翰稍堪為中,三道俱不通為下。判上者職事官加一階,州縣官超一資,判中依資,判下入同類,惟黃衣人降一資。至是,增為四等,三道全次、文翰無取者為中下,用舊判下格;全不通而文翰
 又紕繆為下,殿一選。



 太平興國六年,詔京朝官除兩省、御史臺,自少卿、監以下,奉使從政於外受代而歸者,令中書舍人郭贄、膳部郎中兼侍御史知雜事滕中正、戶部郎中雷德驤同考校勞績,論量器材,以中書所下闕員擬定,引對以遣,謂之差遣院。蓋前代常參官,自一品以下皆曰京官,其未常參者曰未常參官;宋目常參者曰朝官,秘書郎而下未常參者曰京官。舊制,京朝官有員數,除授皆云替某官,或云填見闕。京官皆屬吏部,每
 任滿三十月,罷任,則歲校其考第,取解赴集。太祖以來,凡權知諸州,若通判,若監臨物務官,無定員。月限既滿,有司住給奉料,而見厘務者牒有司復支,所厘務罷則已。但不常參,注授皆出中書,不復由吏部。至是,與朝官悉差遣院主之。凡吏部黃衣選人,始許改為白衣選人。



 太宗選用庶僚,皆得引對,觀其敷納可採者超擢之。復慮因緣矯飾,徼幸冒進,乃詔:「應臨軒所選官吏,並送中書門下,考其履歷,審取進止。」舊制,州縣官南曹判成,流
 內銓注擬,其職事官中書除授。然而歷任功過,須經南曹考驗,遂令幕府官罷任,並歸銓曹,其特除拜者聽朝旨。又詔:「獄官關系尤重,新及第人為司理參軍,固未精習,令長吏察視,不勝任者,奏,判、司、簿、尉對易其官。」



 淳化四年,選人以南郊赦免選,悉集京師。帝曰:「並放選,則負罪者幸矣,無罪者何以勸?」乃令經停殿者守常選。又詔:「司理、司法參軍在任有犯,遇赦及書下考者,止與免選,更勿超資。」工部郎中張知白上言:「唐李嶠嘗云:『安人之
 方,須擇郡守。朝廷重內官,輕外任,望於臺閣選賢良分典大州,共康庶績。』鳳閣待郎韋嗣立因而請行,遂以本官出領郡。今江、浙州郡,方切擇人,臣雖不肖,願繼前修。」帝曰:「知白請重親民之官,良可嘉也。」然不允其請。



 淳化以前,資敘未一,及是始定遷秩之制:凡制舉、進士、《九經》出身者,校書郎、正字、寺監主簿、助教並轉大理評事,評事轉本寺丞,任太祝、奉禮郎者轉諸寺監丞,諸寺監丞轉著作佐郎,或特遷太子中允、秘書郎;由大理寺丞轉
 殿中丞,由著作佐郎轉秘書監丞,資淺者或著作郎,優遷者為太常丞;由太子中允、秘書郎轉太常丞,三丞、著作皆遷太常博士,轉屯田員外郎,優者為禮部、工部、祠部、主客;由屯田轉都官,優者為戶部、刑部、度支、金部;由都官轉職方,優者為吏部、兵部、司封、司勛;其轉郎中亦如之。左右司員外郎,太平興國中有之,後罕除者。左右司郎中,惟待制以上當為少卿者即為之。由前行郎中轉太常少卿、秘書少監,由此二官轉右諫議大夫或秘
 書監、光祿卿;諫議轉給事中,資淺者或右轉左;給事中轉工部、禮部侍郎,至兵部、吏部轉左右丞,由左右丞轉尚書。自侍郎以上,或歷曹,或超曹,皆系特旨。



 諸科及無出身者,校書郎、正字、寺監主簿、助教並轉太祝、奉禮郎,太祝、奉禮郎轉大理評事,評事轉諸寺監丞,諸寺監丞轉大理寺丞,大理寺丞轉中舍,優者為左右贊善,資淺者為洗馬。由幕職為著作佐郎者轉太子中允。由中允、贊善、中舍、洗馬皆轉殿中丞,殿中丞轉國子博士,舊除《五經》
 者,至《春秋》博士則轉國子博士,後罕除。



 由國子博士轉虞部員外郎,優者為膳部;由虞部轉比部,優者為倉部;由比部轉駕部,優者為考功;或由水部轉司門,司門轉庫部;為郎中亦如之。至前行郎中轉少卿、監,或一轉,或二三轉,即為諸寺大卿、監,自大卿、監特恩獎擢,或入給諫焉。



 其為臺省官,則正言、監察比太常博士,殿中、司諫比後行員外郎,起居、侍御史比中行員外郎;起居轉兵部、吏部員外郎,侍御史轉職方員外郎,優者為兵部、司封、知制誥;由正言
 以上至郎中,皆敘遷兩資,中行郎中為左右司郎中,若非次酬勞,有遷三資或止一資者;至左右司郎中為知制誥若翰林學士者,遷中書舍人,舊亦有自前行郎中除者,後兵、吏部止遷諫議。



 由中書舍人轉禮部以上侍郎,入丞、郎即越一資以上。內職、學士、待制亦如之。



 御史中丞由諫議轉者遷工部侍郎,由給事轉者遷禮部侍郎,由丞、郎改者約本資焉。



 其學官,司業視少卿,祭酒視大卿。其法官,大理正視中允、贊善。凡正言、監察以上,皆特恩或被舉方除。其任館閣、三司、
 王府職事,開封府判官、推官,江淮發運、諸路轉運使、提點刑獄,皆得優遷,或以勤效特獎者亦如之。兩制、龍圖閣、三館皆不帶御史臺官,樞密直學士、三司副使皆不帶御史臺官及兩省官,待制以上不帶少卿、監。



 其內職,自借職以上皆循資而遷,至東頭供奉官者轉閣門祗候,閣門祗候轉內殿崇班,崇班轉承制,承制轉諸司副使,自副使以上,或一資,或五資、七資,或直為正使者,至正使亦如之。至皇城使者轉昭宣使,昭宣使
 轉宣慶使,宣慶使轉景福殿使。其閣門祗候,特恩轉通事舍人,通事舍人轉西上閣門副使,亦有加諸司副使兼通事者;西上閣門副使轉東上,東上轉引進,引進轉客省,客省轉西上閣門使;自此以上,亦如副使之遷,惟至東上者又轉四方館使。客省使轉內客省使,內客省使轉宣徽使,或出為觀察使。自內客省使以上,非特恩不授。



 武班副率以上至上將軍,其遷歷軍衛如諸司使副焉。由牧伯內職改授,則觀察使以上為上將軍,團練使、閣門使
 以上為大將軍,刺史、諸司使至崇班為將軍,閣門祗候、供奉官為率,殿直以上為副率。



 內侍省、入內內侍省,自小黃門至內供奉官,皆歷級而轉,至內東頭供奉官轉內殿崇班,有轉內侍、常侍者,內常侍亦正轉崇班。



 其銓選之制:兩府司錄,次赤令,留守、兩府、節度、觀察判官,少尹,一選;兩府判、司,兩畿令,掌書記,支使,防禦、團練判官,二選;諸府司、錄,次畿令,四赤簿、尉,軍事判官,留守、兩府、節度、觀察、防禦、團練軍事推官,軍、監判官,進士、制舉,三
 選;諸府司理、判、司,望縣令,《九經》,四選;輔州、大都督府司理、判、司,緊上州錄事參軍,緊上縣令,次赤兩畿簿、尉,《五經》、《三禮》、《三傳》、《三史》、《通禮》、明法,五選;雄望州司理、判、司,中州錄事參軍,中縣令,次畿簿、尉,六選;緊上州司理、判、司,下州、中下州錄事參軍,中下縣、下縣令,緊望縣簿、尉,學究,七選;中州中下州司理、判、司,上縣簿、尉,八選;下州司理、判、司,中縣簿、尉,九選;中下縣下縣簿、尉,十選。太廟齋郎、室長通理九年,郊社齋郎、掌坐通理十一年。



 凡入官,
 則進士入望州判司、次畿簿尉,《九經》入緊州判司、望縣簿尉,《五經》、《三禮》、《通禮》、《三傳》、《三史》、明法入上州判司、緊縣簿尉,學究有出身人入中州判司、上縣簿尉,太廟齋郎入中下州判司、中縣簿尉,郊社齋郎、試銜無出身人入下州判司、中下縣簿尉,諸司入流人入下州判司、下縣簿尉。



 仁宗初,吏員猶簡,吏部奏天下幕職、州縣官期滿無代者八百餘員,而川、廣尤多未代。帝曰:「此豈人情之所樂耶?其亟代之。」帝御後殿視事,或至旰食。中書請如天
 禧舊制,審官、三班院、流內銓日引見毋得過兩人,詔弗許。自真宗朝,試身、言、書、判者第推恩,乃特詔曰:「國家詳核吏治,念其或淹常選,而以四事程其能。騰承統緒,循用舊典,爰命從臣,精加詳考。其令翰林學士李諮與吏部流內銓以成資闕為差擬。」於是咸得遷官,率以為常。後議者以身、言、書、判為無益,乃罷。



 凡磨勘遷京官,始增四考為六考,舉者四人為五人,曾犯過又加一考。舉吏各有等數,得被舉者須有本部監司、長吏按察官,乃得
 磨勘;須到官一考,方許薦任。凡選人年二十五以上,遇郊,限半年赴銓試,命兩制三員鎖試於尚書省,糊名謄錄。習辭業者試論、試詩賦,詞理可採、不違程序為中格,習經業者人專一經,兼試律,十而通五為中格,聽預選。七選以上經三試至選滿,京朝官保任者三人,補遠地判、司、簿、尉,無舉主者補司士參軍,或不赴試、亦無舉者,永不預選。京官年二十五以上,歲首赴試於國子監,考法如選人,中格者調官。兩任無私罪而有部使、州守倅
 舉者五人,入親民;舉者三人,惟與下等厘物務官。



 初,州郡多闕官,縣令選尤猥下,多為清流所鄙薄,每不得調。乃詔吏部選幕職官為知縣,又立舉任法以重令選,敕諸路察縣之不治者。然被舉者日益眾,有司無闕以待之,中書奏罷舉縣令法。未幾,有言親民之任輕,則有害於治,法不宜廢。復令指劇縣奏舉,舉者二人,必一人本部使,既居任,復有舉者,始得遷,否則如常選,毋輒升補。常參官已授外任,勿奏舉。然銓格煩密,府史奸弊尤多,
 而磨勘者待次外州,或經三二歲乃得改官,往往因緣薄勞,求截甲引見。有詔自是弗許。



 神宗欲更制度,建議之臣以為唐銓與今選殊異,雜用其制,則有留礙煩紊之弊。始刊削舊條,務從簡便,因廢南曹而並歸之於銓。初,審官西院與東院對掌文武,尋改從吏部,而左、右選分焉。祖宗以來,中書有堂選,百司、郡縣有奏舉,雖小大殊科,然皆不隸於有司。暨元豐罷奏舉闕,屬之銓曹,而堂選亦不領於中書,一時更制,必欲公天下而詒永久。
 於是除免選之恩,重出官之試,定賞罰之則,酌資蔭之宜。凡設試以待命士而入之銓注者,自蔭補、銓試之外,有進士律義、武臣呈試及試刑法官等,而銓試所受為特廣。中書言:「選人守選,有及三年方遇恩放選者,或適歸選而遽遇恩,既為不均,且蔭補免試注官,以不習事多失職,試者又止試詩,豈足甄才?已受任而無勞績,舉薦及免試恩法,須再試書判三道,然亦虛文。」



 熙寧四年,遂定銓試之制:凡守選者,歲以二月、八月試斷按二,或
 律令大義五,或議三道,後增試經義。差官同銓曹撰式考試,第為三等,上等免選注官,優等升資如判超格,無出身者賜之出身。自是不復試判,仍去免選恩格,若歷任有舉者五人,自與免試注官。任子年及二十,聽赴銓試。其試不中或不能試,選人滿三歲許注官,惟不得入縣令、司理、司法。任子年及三十方許參注,若年及三十授官,已及三年,出官亦不用試。若秩入京朝,即展任監當三年,在任有二人薦之,免展。選人應改官,必對便殿。
 舊制,五日一引,不過二人。至是,待次者多,有逾二年乃得引。帝閔其留滯,詔每甲引四人以便之。



 帝因論郡守,謂宰臣曰:「朕每思祖宗百戰得天下,今州郡付之庸人,常切痛心。卿輩謂何如而得選任之要?」文彥博請擇監司而按察之。陳升之曰:「取難治劇郡,擇審官近臣而責以選才,宜可得也。」



 初置審官西院,磨勘武臣,並如審官院格,而舊審官曰東院。御史中丞呂公著言:「英宗時,文臣磨勘,例展一年,至少卿、監止。武臣橫行以上及使臣,
 猶循舊制,固未嘗如文臣有所節抑也。又仁宗時,嘗著令,正任防禦、團練以上,非邊功不遷。今及十年嘗歷外任,即許轉,亦未如少卿、監之有限止也。」詔兩制詳定。王珪等言:「文武兩選磨勘,已皆均用四年。請今自正任刺史以上,轉官未滿十年,若有顯效者自許特轉,其非次恩惟許改易州鎮,以示旌寵。有過,則比文臣展年。」從之。知審官西院李壽朋言:「皇城使占籍者三十餘員,多領遙郡,而尚得從磨勘,遷刺史、團練防禦使。每進一級,增
 奉錢五萬,廩粟雜給如之,實為無名。請於皇城使上別置二使名,視前行郎中,量給奉祿。其遙郡刺史、團練防禦使,並從朝廷賞功擢用,更不序遷。」詔:「遙郡刺史、團練防禦使,並以十年磨勘,至觀察留後止。應官止而有功若特恩遷者,不以法。」



 諸司使副,每磨勘皆用常制,雖軍功亦無別異,而閣門內侍輩,轉皆七資。帝謂:「左右近習,非勛勞而得超躐,至嘗立功者乃無優遷,非制也。」使副嘗有軍功應轉,許特超七資,閣門通事舍人、帶御器械、
 兩省都知押班、管幹御藥院使臣七資超轉法,皆除之。後客省、引進、四方館各置使二員,東、西上閣門共置使六員,客省、引進、閣門副使共八員。副使靡勘如諸司使法。使有闕,改官及五期者,樞密院檢舉。歷閣門職事有犯事理重者,當遷日除他官;閣門、四方館使七年無私過,未有闕可遷者,加遙郡;特旨與正任者,引進四年轉團練使,客省四年轉防禦使:皆著為定制焉。



 先是,御史乞罷堂選,曾公亮執不可。王安石曰:「中書總庶務,今通
 判亦該堂選,徒留滯,不能精擇,宜歸之有司。」帝曰:「唐陸贄謂:『宰相當擇百官之長,而百官之長擇百官。』今之審官,茍得其人,安有不能精擇百官者哉?」元豐四年,堂選、堂占悉罷。



 初,有司屬職卑者不在吏銓,率命長吏舉奏。都水監主簿李士良言:「沿河干集使臣,凡百六十餘員,悉從水監奏舉,往往不諳水事,干請得之。」乃詔東、西審官及三班院選差。於是悉罷內外長吏舉官法。明年,令吏部始立定選格,其法:各隨所任職事,以入仕功狀,循
 格以俟擬注。如選巡檢、捕盜官,則必因武舉、武學,或緣舉薦,或從獻策得出身之人。他皆仿此。



 自官制行,以舊少卿、監為朝議大夫,諸卿、監為中散大夫,秘書監為中大夫。故事,兩制不轉卿、監官,每至前行郎中,即超轉諫議大夫。前行郎中,於階官為朝請大夫;諫議大夫,於階官為太中大夫。帝謂:「磨勘者,古考績之法,所與百執事共之,而禁近獨超轉,非法也。」於是詔待制以下,並三年一遷,仍轉朝議、中散、中大夫三官。自是遷敘平允。凡開
 府儀同三司至通議大夫,無磨勘法;太中大夫至承務郎,皆應磨勘。待制以上六年遷兩官,至太中大夫止;承務郎以上四年遷一官,至朝請大夫止。朝議大夫以七十員為額,有闕,以次補之。選人磨勘用吏部法,遷京朝官則依新定之制。除授職事官,並以寄錄官品高下為法:凡高一品以上者為行,下一品者為守,二品以下者為試,品同者不用行、守、試。



 哲宗時,御史上官均言:「今仕籍,合文武二萬八千餘員,吏部逆用兩任闕次,而仕者
 七年乃成一任。當清其源,宜加裁抑。」朝廷下其章議之,司諫蘇轍議曰:「祖宗舊法,凡任子,年及二十五方許出官,進士、諸科,初命及已任而應守選者,非逢恩不得放選。先朝患官吏不習律令,欲誘之讀法,乃減任子出官年數,去守選之格,概令試法,通者隨得注官。自是天下爭誦律令,於事不為無補。然人人習法,則試無不中,故蔭補者例減五年,而選人無復選限。吏部員今年已用後四年夏秋闕,官冗至此亦極矣。宜追復祖宗守選舊
 法,而選滿之日,兼行試法之科,此亦今日之便也。」事報聞。



 三省言:「舊經堂除選人,惟嘗歷省府推官、臺諫、寺監長貳、郎官、監司外,悉付吏部銓注,凡格所應入,遞升一等以優之。被邊州軍,其城砦巡檢、都監、監押、砦主、防巡、諸路捕盜官,及三萬緡以上課息場務,凡舊應舉官,員闕,許仍奏舉。」時通議大夫以上,有以特恩、磨勘轉官,而比之舊格,或實轉兩官至三四官者。右正言王覿謂非所以愛惜名器,請官至太中大夫以上,毋用磨勘遷轉。
 詔:「待制、太中大夫應磨勘者,止於通議大夫,餘官止中散大夫。中散以上勞績酬獎,合進官者,止許回授子孫。特命特遷,不拘此制。」



 初,武臣戰功得賞,凡一資,則從所居官遞遷一級。於是以皇城使驟上遙刺,或入橫行;且閣門使以上,等級相比而輕重絕遠。因樞密院言,乃詔「閣門、左藏庫副使得兩資,客省、皇城使得三資,止許一轉,減年者許回授親屬。」又小使臣磨勘轉崇班者,歲毋過八十人。內臣昭宣使以上無磨勘法,惟押班以上則
 取裁,餘理五年磨勘。



 紹聖初,改定《銓試格》,凡攝官初歸選,散官、權官歸司,若新賜第,皆免試。每試者百人,惟取一人入優等,中書奏裁,二人為上等,五人為中等。崇寧以後,又復元豐制,而蔭補者須隸國學一年無過罰,乃試銓,若在學試嘗再入等,即免試;其公、私試嘗居第一,得比銓試推恩。政和間著為令。既而臣僚言:「進士中銓格者,每二百人,得優恩不過五七人,又或闕上等不取。而朝廷取隸國子試格,用之銓注,及今五年,而得上等
 優恩者二百四十人,免試者尚在其外。是蔭補隸學者,優於累試得第之人矣。」於是詔在學嘗魁一試者,許如舊恩,餘止令免試注官。吏部侍郎彭汝礪乞稍責吏部甄別能否,凡京朝官才能事效茍有可錄,尚書暨郎官銓擇以聞。三省分三年考察之,高則引對,次即試用,下者還之本選;若資歷、舉薦應入高而才行不副,許奏而降其等。凡皆略許出法而加升黜,歲各毋過三人。



 初,選人改官,歲以百人為額。元祐變法,三人為甲,月三引見,
 積累至紹聖初,待次者二百八十餘人。詔依元豐五日而引一甲,甲以三人,歲毋過一百四十人,俟待次不及百人,別奏定。又令歷任通及三考,而資序已入幕職、令錄,方許舉之改官。吏部言:「元豐選格,經元祐多所紛更,於是選集後先,路分遠近,資歷功過,悉無區別,逾等超資,惟其所欲。詔旨既復元豐舊制,而闢舉一路尚存,請盡復舊法,以息僥幸。」乃罷闢舉。



 崇寧元年,詔吏部講求元豐本制,酌以時宜,刪成彞格,使才能、閥閱兩當其實。
 吏部言:「堂選窠名及舉官員闕,內外共約三千餘目。元祐法,選人得升資以上賞,及參選射闕,不許遣人代注,今皆罷從元豐法。所當損益者,其知邊近蠻夷州如威、茂、黎、瓊等,及開封府曹掾,平準務,諸路屬官,在京重課場務,京城內外廂官,戶部幹官,曲院,榷貨務,將作監管幹公事,黃河都大,內外榷茶官,凡干刑獄及管庫繁劇,皆不可罷舉。若御史臺主簿、檢法官、協律郎,豈可泛以格授?諸如此類,仍舊闢舉。」從之。惟諸路毋得直牒差待
 闕得替官權攝。



 初,未改官制,大率以職為階官。如以吏部尚書為階官,而同中書門下平章事則其職也。至於選人,則幕職、令錄之屬為階官,而以差遣為職,名實混淆甚矣。元豐未及革正。崇寧二年,刑部尚書鄧洵武極言之,遂定選人七階:曰承直郎,曰儒林郎,曰文林郎,曰從事郎,曰通仕郎,曰登仕郎,曰將仕郎。政和間,改通仕為從政,登仕為修職,將仕為迪功,而專用通仕、登仕、將仕三階奏補未出官人,承直至修職須六考,迪功七考,
 有官保任而職司居其一,乃得磨勘。坐愆犯,則隨輕重加考及舉官有差。



 時權奸柄國,僥幸並進,官員益濫,銓法留礙。臣僚言:「吏員增多,蓋因入流日眾。熙寧郊禮,文武奏補總六百一十一員;元豐六年,選人磨勘改京朝官總一百三十有五員。考之吏部,政和六年,郊恩奏補約一千四百六十有畸,選人改官約三百七十有畸。欲節其濫,惟嚴守磨勘舊法。而今之磨勘,有局務減考第,有川遠減舉官,有用酬賞比類,有因大人特舉,有托事
 到闕不用滿任,有約法違礙許先次而改。凡皆棄法用例,法不能束而例日益繁,茍不裁之,將又倍蓰而未可計也。請詔三省若吏部,舊有止法,自當如故,餘皆毋得用例。」乃詔:「惟川、廣水土惡地,許減舉如制,餘悉用元豐法。」既而又言:「元豐進納官法,多所裁抑。應入令錄及因賞得職官,止與監當,該磨勘者換授降等使臣,仍不免科率,法意深矣。邇者用兵東南,民入金穀皆得補文武官,理選如官戶,與士大夫涇、渭並流,復其戶不受科輸。
 是得數千緡於一日,而失數萬斛於無窮也。況大戶得復,則移其科於下戶,下戶重貧,州縣緩急,責辦何人?此又弊之大者。」不聽。



 初,宗室無參選法,祖宗時,間選注一二,不為常制。徽宗欲優宗室,多得出官,一日參選,即在合選名次之上。而膏粱之習,往往貪恣,出任州縣,黷貨虐民,議者頗陳其害。飲宗即位,臣僚復以為言,始令不注郡守、縣令,仍與在部人通理名次。



 高宗建炎初,行都置吏部。時四選散亡,名籍莫考。始下諸道州、府、軍、監,條
 具屬吏寓官之爵里、年甲、出身、歷仕功過、舉主、到罷月日,編而籍之。然自兵難以來,典籍散失,吏緣為私,申明繁苛,承用踳駁,保任滋眾,阻會無期,參選者苦之。乃令凡文字有不應於今,而桉牘參照明白,從郎官審覆,長貳予決,小不完者聽行,有徇私挾情,則令御史糾之。又詔京畿、京東、河北、京西、河東士夫在部注授,雖銓未中而年及者,皆聽注官。二年,命京官赴行在者,令吏部審量,非政和以後進書頌及直赴殿試之人,乃聽參選。在
 部知州軍、通判、僉判及京朝官知縣、監當以三年為任者,權改為二年。以赴調者萃東南,選法留滯故也。又詔州縣久無正官者,聽在選人申部,審度榜闕差注。



 紹興元年,起居郎胡寅言:「今典章文物,廢墜無幾,百司庶府不可闕者,莫如吏部。姑置侍郎一員,郎官二員,胥吏三十人,則所謂磨勘、封敘、奏薦常程之事,可按而舉矣。」



 詔曰:「六官之長,佐王理邦國者,其惟銓衡乎。亂離以來,士大夫流徙,有徒跣而赴行在者。注授榜闕,奸弊日滋,寒
 士困苦,甚可憫焉。宜令三省議除其弊,嚴立賞禁,仍選能吏以主之,御史臺常加糾察。」於是三省立八事,曰注擬藏闕,申請徼幸,去失問難,刷闕滅裂,關會淹延,審量疑似,給付邀求,保明退難。令長貳機柅之。又詔館職選人到任及一年,通理四考,並自陳,改京官。



 二年,呂頤浩言:「近世堂除,多侵部注,士人失職。宜仿祖宗故事,外自監司、郡守及舊格堂除通判,內自察官省郎以上、館職、書局編修官外,餘闕並寺監丞、法寺官、六院等,武臣自
 準備將領、正副將以上,其部將、巡尉、指使以下,並歸部注。」從之。又覆文臣銓試,以經義、詩賦、時議、斷案、律義為五場,願試一場者聽,榜首循一資。武臣呈試合格者並聽參選。



 三年,右僕射朱勝非等上《吏部七司敕令格式》。自渡江後,文籍散佚,會廣東轉運司以所錄元豐、元祐吏部法來上,乃以省記舊法及續降指揮,詳定而成此書。先是,侍御史沈與求言:「今日矯枉太過,賢愚同滯。」帝曰:「果有豪傑之士,雖自布衣擢為輔相可也;茍未能考
 其實,不若姑守資格。」乃命吏部注授縣令,惟用合格之人。



 五年,詔:「凡注擬,並選擇非老疾及未嘗犯贓與非緣民事被罪之人。」時建議者云:「親民莫如縣令,今率限以資格,雖貪懦之人,一或應格,則大官大邑得以自擇。請詔監司、郡守,條上劇邑,遴選清平廉察之人為之。」既而又詔:「知縣依舊法,止用兩任關升通判資序。」明年,侍御史周秘言:「今有無舉員考第,因近臣薦對,即改官升擢,實長奔競。望詔大臣,自今惟賢德才能之人,餘並依格
 注擬。」廷臣或請以前宰執所舉改官,易以司馬光十科之目,歲薦五員,中書難之。詔「前宰執所舉京削,不理職司」而已。



 三十二年,吏部侍郎凌景夏言:「國家設銓選以聽群吏之治,其掌於七司,著在令甲,所守者法也。今升降於胥吏之手,有所謂例焉。長貳有遷改,郎曹有替移,來者不可復知,去者不能盡告。索例而不獲,雖有強明健敏之才,不復致議;引例而不當,雖有至公盡理之事,不復可伸。貨賄公行,奸弊滋甚。嘗睹漢之公府有辭訟
 比,尚書有決事比,比之為言,猶今之例。今吏部七司宜置例冊,凡換給之期限,戰功之定處,去失之保任,書填之審實,奏薦之限隔,酬賞之用否,凡經申請,或堂白、或取旨者,每一事已,命郎官以次擬定,而長貳書之於冊,永以為例,每半歲上於尚書省,仍關御史臺。如是,則巧吏無所施,而銓敘平允矣。」



 有議減任子者,孝宗以祖宗法令難於遽改,令吏部嚴選試之法。自是,初官毋以恩例免試,雖宰執亦不許自陳回授。舊制,任子降等補文
 學及恩科人皆免,至是悉試焉。凡未經銓中及呈試者,勿堂除;雖墨敕,亦許執奏。舊制,宗室文資與外官文臣參注窠闕,武資則不得與武臣參注,但注添差。至是,始聽注厘務闕。乾道七年,始命銓試不中、年四十,呈試不中、年三十者,令寫家狀,讀律注官。陳師正言:「請令宗室恩任子弟出官日量行銓試,如士夫子弟之法,多立其額而優為之制。」遂詔:「自今宗室曾經應舉得解者,許參選,餘並行銓試,三人取二。其三試終場不中人,聽不拘年限
 調官。」



 淳熙元年,參知政事龔茂良言:「官人之道,在朝廷則當量人才,在銓部則宜守成法。法本無弊,例實敗之。法者,公天下而為之者也;例者,因人而立以壞天下之公者也。昔之患在於用例破法,今之患在於因例立法。諺稱吏部為『例部』。今《七司法》自晏敦復裁定,不無疏略,然守之亦可以無弊。而徇情廢法,相師成風,蓋用例破法其害小,因例立法其害大。法常靳,例常寬,今法令繁多,官曹冗濫,蓋由此也。望令裒集參附法及乾道續降
 申明,重行考定,非大有抵牾者弗去,凡涉寬縱者悉刊正之。庶幾國家成法,簡易明白,賕謝之奸絕,冒濫之門塞矣。」於是重修焉。既而吏部尚書蔡洸以改官、奏薦、磨勘、差注等條法分門編類,名《吏部條法總類》。十一月,《七司敕令格式申明》成書。



 淳熙三年,中書舍人程大昌言:「舊制,選人改秩後兩任關升通判,通判兩任關升知州,知州兩任即理提刑資序。除授之際,則又有別以知縣資序隔兩等而作州者,謂之『權發遣』,以通判資序隔一
 等而作州者,謂之『權知』,上而提刑、轉運亦然。隔等而授,是擇材能也;結銜有差,是參用資格也。今得材能、資格俱應選者為上,其次,則擇第二任知縣以上有課績者許作郡,初任通判以上許作監司,第二任通判以上許作職司,庶幾人法並用。」從之。



 寧宗慶元中,復位《武臣關升格》。先是,初改官人必作令,謂之「須入」。至是,復命除殿試上三名、南省元外,並作邑;後又命大理評事已改官未歷縣人,並令親民一次,著為令。



 紹定元年,臣僚上言:「
 銓曹之患,員多闕少,注擬甚難。自乾道、嘉定以來,嘗命選部職官窠闕,各於元出闕年限之上,與展半年用闕。歷年浸久,入仕者多,即今吏部參注之籍,文臣選人、武臣小使臣校尉以下,不下二萬七千餘員,大率三四人共注一闕,宜其膠滯壅積而不可行。乞命吏部錄參、司理、司法、令、丞、監當酒官,於元展限之上更展半年。」從之。



 淳祐七年,監察御史陳垓建言,乞申戒飭銓法十弊:一曰添差數多,破法耗財;謂倅貳、幕職、參議、機宜、總戎、鈐轄、監押之類。



 二曰抽差員
 眾,州縣廢職;謂監司、帥守幕屬多差見任州縣他官權攝。



 三曰攝局違法,蠹政害民;謂監司、師守徇私差權幕屬等職。



 四曰「須入」不行,僥幸撓法;謂初改官人必作知縣,今多規免,茍圖京局,躐求倅貳,遂使不曾歷縣之人冒當郡寄。



 五曰奏闢不應,奔競日甚;謂在法未經任人不許奏闢,今或以初任或以闕次遠而改闢見次者。



 六曰改任巧捷,紊亂官常;謂在法已授差遣人,不得乾求換易。今既授是官,復謀他職,辭卑居尊,棄彼就此。



 七曰薦舉不公,多歸請托;八曰借補繁多,官資泛濫;九曰□曠職守,役心外求;十曰匿過居官,玩視國法。



 謂曾經罪犯,必俟赦宥。今則既遭彈劾,初未經赦者,經營差遣。



 舊制,軍功補授之人,
 自合從軍,非老疾當汰,無參部及就闢之法。比年諸路奏功不實,寅緣竄名,許令到部,及諸司紛然奏闢,實礙銓法。建炎兵興,雜流補授者眾,有曰上書獻策,曰勤王,曰守御,曰捕盜,曰奉使,其名不一,皆閫帥假便宜承制之權以擅除擢。有進士徑補京官者,有素身冒名即為郎、大夫者。乃詔:「從軍應賞者,第補右選,以清流品。」又有民間願習射者,籍其姓名。守令月一試,取藝優者,如三路保甲法區用。



 紹興初,嘗以兵革經用不足,有司請募
 民入貲補官,帝難之。參知政事張守曰:「祖宗時,授以齋郎,今之將仕郎是也。」知樞密院李回曰:「此猶愈科率於民。」乃許補承節郎、承信郎、諸州文學至進義副尉六等,後又給通直郎、修武郎、秉義郎、承直至迪功郎。其注擬、資考、磨勘、改轉、蔭補、封敘,並依奏補出身法,毋得注令錄及親民官。和議之後,立格購求遺書,亦命以官。凡歿於王事,無遺表致仕格法者,聽奏補本宗異姓親子孫弟侄,文臣將仕郎,武臣承信郎;餘親,上州文學或進武
 校尉,所以褒恤忠義也。又以兩淮、荊襄,其土廣袤,募民力田。凡白身勸民墾田及七十五頃者與副尉,五百頃補承信郎。



 孝宗即位,命帥臣、監司、郡守、嘗任兩府及朝官等遣親屬進貢,等第補授登仕郎、將仕郎,推恩理為選限。淳熙三年,詔罷鬻爵,除歉歲民願入粟賑饑、有裕於眾,聽補官,餘皆停。自是,進納軍功,不理選限,登仕郎、諸州助教不許出官,止於贖罪及就轉運司請解而已。



\end{pinyinscope}