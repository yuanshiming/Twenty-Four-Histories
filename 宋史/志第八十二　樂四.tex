\article{志第八十二 樂四}

\begin{pinyinscope}

 崇寧四年七月,鑄帝鼐、八鼎成。八月,大司樂劉昺言:「大朝會宮架舊用十二熊羆按,金錞、簫、鼓、觱篥等與大樂合奏。今所造大樂,遠稽古制,不應雜以鄭、衛。」詔罷之。又
 依昺改定二舞,各九成,每三成為一變,執鑰秉翟,揚戈持盾,威儀之節,以象治功。庚寅,樂成,列於崇政殿。有旨,先奏舊樂三闕,曲未終,帝曰:「舊樂如泣聲。」揮止之。既奏新樂,天顏和豫,百僚稱頌。九月朔,以鼎樂成,帝御大慶殿受賀。是日,初用新樂,太尉率百僚奉觴稱壽,有數鶴從東北來,飛度黃庭,回翔鳴唳。乃下詔曰:「禮樂之興,百年於此。然去聖愈遠,遺聲弗存。乃者,得隱逸之士於草茅之賤,獲《英莖》之器於受命之邦。適時之宜,以身為度,
 鑄鼎以起律,因律以制器,按協於庭,八音克諧。昔堯有《大章》,舜有《大韶》,三代之王亦各異名。今追千載而成一代之制,宜賜新樂之名曰《大晟》,朕將薦郊廟、享鬼神、和萬邦,與天下共之。其舊樂勿用。」



 先是,端州上古銅器,有樂鐘,驗其款識,乃宋成公時。帝以端王繼大統,故詔言受命之邦,而隱逸之士謂漢津也。朝廷舊以禮樂掌於太常,至是專置大晟府,大司樂一員、典樂二員並為長貳,大樂令一員、協律郎四員,又有制撰官,為制甚備,於
 是禮、樂始分為二。



 五年九月,詔曰:「樂不作久矣!朕承先志,述而作之,以追先王之緒;建官分屬,設府庀徒,以成一代之制。二月,嘗詔省內外冗官,大晟府亦並之禮官。夫舜命夔典樂,命伯夷典禮,禮樂異道,各分所守,豈可同職?其大晟府名可復仍舊。」



 又詔曰:「樂作已久,方薦之郊廟,施於朝廷,而未及頒之天下。宜令大晟府議頒新樂,使雅正之聲被於四海,先降三京四輔,次帥府。」



 大觀二年,詔曰:「自唐以來,正聲全失,無徵角之音,五聲不備,
 豈足以道和而化俗哉?劉詵所上徵聲,可令大晟府同教坊依譜按習,仍增徵、角二譜,候習熟來上。」初,進士彭幾進樂書,論五音,言本朝以火德王,而羽音不禁,徵調尚闕。禮部員外郎吳時善其說,建言乞召幾至樂府,朝廷從之。至是,詵亦上徵聲,乃降是詔。



 三年五月,詔:「今學校所用,不過春秋釋奠,如賜宴闢雍,乃用鄭、衛之音,雜以俳優之戲,非所以示多士。其自今用雅樂。」



 四年四月,議禮局言:「國家崇奉感生帝、神州地祇為大祠,以僖祖、
 太祖配侑,而有司行事不設宮架、二舞,殊失所以尊祖、侑神作主之意。乞皆用宮架、二舞。」詔可。六月,詔近選國子生教習二舞,以備祠祀先聖,本《周官》教國子之制。然士子肄業上庠,頗聞恥於樂舞與樂工為伍、坐作、進退。蓋今古異時,致於古雖有其跡,施於今未適其宜。其罷習二舞,願習雅樂者聽。」



 八月,帝親制《大晟樂記》,命太中大夫劉昺編修《樂書》,為八論:



 其一曰:樂由陽來,陽之數極於九,聖人攝其數於九鼎,寓其聲於九成。陽之數復
 而為一,則寶鼎之卦為《坎》;極而為九,則彤鼎之卦為《離》。《離》,南方之卦也。聖人以盛大光明之業,如日方中,向明而治,故極九之數則曰景鐘,大樂之名則曰《大晟》。日王於午,火明於南,乘火德之運。當豐大之時,恢擴規模,增光前烈,明盛之業,永觀厥成。樂名《大晟》,不亦宜乎?



 其二曰:後世以黍定律,其失樂之本也遠矣。以黍定尺,起於西漢,蓋承《六經》散亡之後,聞古人之緒餘而執以為法,聲既未協,乃屢變其法而求之。此古今之尺所以至於
 數十等,而至和之聲愈求而不可得也。《傳》曰:「萬物皆備於我矣,反身而誠,樂莫大焉!」秬黍云乎哉?



 其三曰:焦急之聲不可用於隆盛之世。昔李照欲下其律,乃曰:「異日聽吾樂,當令人物舒長。」照之樂固未足以感動和氣如此,然亦不可謂無其意矣。自藝祖御極,和樂之聲高,歷一百五十餘年,而後中正之聲乃定。蓋奕世修德,和氣熏蒸,一代之樂,理若有待。



 其四曰:盛古帝王皆以明堂為先務,後世知為崇配、布政之宮,然要妙之旨,秘而不
 傳,徒區區於形制之末流,而不知帝王之所以用心也。且盛德在木,則居青陽,角聲乃作;盛德在火,則居明堂,徵聲乃作;盛德在金,則居總章,商聲乃作;盛德在水,則居玄堂,羽聲乃作;盛德在土,則居中央,宮聲乃作。其應時之妙,不可勝言。一歲之中,兼總五運,凡麗於五行者,以聲召氣,無不總攝。鼓宮宮動,鼓角角應:彼亦莫知所以使之者。則永膺壽考,歷數過期,不亦宜乎?



 其五曰:魏漢津以太極元氣,函三為一,九寸之律,三數退藏,故八
 寸七分為中聲。正聲得正氣則用之,中聲得中氣則用之。宮架環列,以應十二辰;中正之聲,以應二十四氣;加四清聲,以應二十八宿。氣不頓進,八音乃諧。若立春在歲元之後,則迎其氣而用之,餘悉隨氣用律,使無過不及之差,則所以感召陰陽之和,其法不亦密乎?



 其六曰:乾坤交於亥,而子生於黃鐘之宮,故稟於乾,交於亥,任於壬,生於子。自乾至子凡四位,而清聲具焉。漢津以四清為至陽之氣,在二十八宿為虛、昴、星、房,四者居四方
 之正位,以統十二律。每清聲皆有三統:申、子、辰屬於虛而統於子,巳、酉、丑屬於昴而統於丑,寅、午、戌屬於星而統於寅,亥、卯、未屬於房而統於卯。中正之聲分為二十四宿,統於四清焉。



 其七曰:昔人以樂之器有時而弊,故律失則求之於鐘,鐘失則求之於鼎,得一鼎之龠,則權衡度量可考而知。故鼎以全渾淪之體,律呂以達陰陽之情,天地之間,無不統攝,機緘運用,萬物振作,則樂之感人,豈無所自而然耶?



 其八曰:聖上稽帝王之制而成
 一代之樂,以謂帝舜之樂以教冑子,乃頒之於宗學。成周之樂,掌於成均,乃頒之府學、闢雍、太學;而三京藩邸,凡祭祀之用樂者皆賜之,於是中正之聲被天下矣。漢施鄭聲於朝廷,唐升夷部於堂上,至於房中之樂,唯恐淫哇之聲變態之不新也。聖上樂聞平淡之音,而特詔有司制為宮架,施之於禁庭,房中用雅樂,自今朝始云。



 又為圖十二:一曰五聲,二曰八音,三曰十二律應二十八宿,四曰七均應二十八宿,五曰八十四調,六曰十二
 律所生,七曰十二律應二十四氣,八曰十二律鐘正聲,九曰堂上樂,十曰金鐘玉磬,十一曰宮架,十二曰二舞。圖雖不能具載,觀其所序,亦可以知其旨意矣:



 天地相合,五數乃備,不動者為五位,常動者為五行,五行發而為五聲。律呂相生,五聲乃備,布於十二律之間,猶五緯往還於十有二次,五運斡旋於十有二時。其圖五聲以此。



 兩儀既判,八卦肇分。氣盈而動,八風行焉。顓帝乃令飛龍效八風之音,命之曰《承云》。方是時,金、石、絲、竹、匏、土、
 革、木之音未備,後聖有作,以八方之物全五聲者,制而為八音,以聲召氣,八風從律。其圖八音以此。



 上象著明器形,而下以聲召氣,吻合元精。其圖十二律應二十八宿以此。



 斗在天中,周制四方,猶宮聲處中為四聲之綱。二十八舍列在四方,用之於合樂者,蓋樂方七角屬木,南方七徵屬火,西方七商屬金,北方七羽屬水。四方之宿各有所屬,而每方之中,七均備足。中央七宮管攝四氣。故二十八舍應中正之聲者,制器之法也;二十八舍
 應七均之聲者,和聲之術也。其圖七均應二十八宿以此。



 合陰陽之聲而文之以五聲,則九六相交,均聲乃備。黃鐘為宮,是謂天統;林鐘為征,是謂地統;太簇為商,是謂人統。南呂為羽,於時屬秋;姑洗為角,於時屬春;應鐘為變宮,於時屬冬;蕤賓為變徵,於時屬夏。旋相為宮,而每律皆具七聲,而八十四調備焉。其圖八十四調以此。



 自黃鐘至仲呂,則陽數極而為乾,故其位在左;蕤賓至應鐘,則陰數極而為坤,故其位在右。陰窮則歸本,故應
 鐘自生陰律;陽窮則歸本,故仲呂自歸陽位。律呂相生,起於《復》而成於《乾》,終始皆本於陽,故曰「樂由陽來」,六呂則同之而已。相生之位,分則為《乾》、《坤》之爻,合則為《既濟》、《未濟》之卦。自黃鐘至仲呂為《既濟》,故屬陽而居左;自蕤賓至應鐘為《未濟》,故屬陰而居右。《易》始於《乾》、《坤》而終於《既濟》、《未濟》,天地辨位而水火之氣交際於其中,造化之原皆自此出。其圖十二律所生以此。



 二十四氣差之毫厘,則或先天而太過,或後天而不及。在律為聲,在歷為
 氣。若氣方得節,乃用中聲;氣已及中,猶用正律。其圖十二律應二十四氣以此。



 漢津曰:「黃帝、夏禹之法,簡快捷方式直,得於自然,故善作樂者以聲為本。若得其聲,則形數、制度當自我出。今以帝指為律,正聲之律十二,中聲之律十二,清聲凡四,共二十有八」云。其圖十二律鐘正聲以此。



 堂上之樂,以人聲為貴,歌鐘居左,歌磬居右。近世之樂,曲不協律,歌不擇人,有先制譜而後命辭。奉常舊工,村野癃老者斥之。升歌之工,選擇惟艱,故堂上之樂鏗然特
 異焉。其圖堂上樂以此。



 金玉之精,稟氣於乾,故堂上之樂,鐘必以金,磬必以玉。《歷代樂儀》曰:「歌磬次歌鐘之西,以節登歌之句。」即《周官》頌磬也,神考肇造玉磬,聖上紹述先志,而堂上之樂方備,非聖智兼全、金聲而玉振之者,安能與於天道哉?其圖金鐘玉磬以此。



 《大晟》之制,天子親祀圓丘,則用景鐘為君圍,鎛鐘、特磬為臣圍,編鐘、編磬為民圍,非親祀則不用君圍。漢津以謂:「宮架總攝四方之氣,故《大晟》之制,羽在上而以四方之禽,虡在下
 而以四方之獸,以象鳳儀、獸舞之狀。龍簨崇牙,制作華煥。」其圖宮架以此。



 新樂肇興,法夏鑰九成之數:文舞九成,終於垂衣拱手,無為而治;武舞九成,終於偃武修文,投戈講藝。每成進退疾徐,抑揚顧揖,皆各象方今之勛烈。文武八佾,左執鑰,右秉翟。蓋鑰為聲之中,翟為文之華,秉中聲而昌文德。武舞八佾,執干戈而進,以金鼓為節。其圖二舞以此。



 又列八音之器,金部有七:曰景鐘,曰鎛鐘,曰編鐘,曰金錞,曰金鐲,曰金鐃,曰金鐸。其說以謂:



 景鐘乃樂之祖,而非常用之樂也。黃帝五鐘,一曰景鐘。景,大也。鐘,四方之聲,以象厥成。惟功大者其鐘大,世莫識其義久矣。其聲則黃鐘之正,而律呂由是生焉。平時弗考,風至則鳴,鎛鐘形聲宏大,各司其辰,以管攝四方之氣。編鐘隨月用律,雜比成文,聲韻清越。錞、鐲、鐃、鐸,古謂之四金。鼓屬乎陽,金屬乎陰。陽造始而為之倡,故以金鎛和鼓陽動而不知已,故以金鐲節鼓。陽之用事,有時而終,故以金鐃止鼓。時止則止,時行則行,天之道也,
 故以金鐸通鼓。金乃《兌》音,《兌》為口舌,故金之屬皆象之。



 石部有二:曰特磬,曰編磬。其說以謂:「依我磬聲」,以石有一定之聲,眾樂依焉,則鐘磬未嘗不相須也。往者,國朝祀天地、宗廟及大朝會,宮架內止設鎛鐘,惟后廟乃用特磬,若已升祔後廟,遂置而不用。如此,則金石之聲小大不侔。《大晟》之制,金石並用,以諧陰陽。漢津之法,以聲為主,必用泗濱之石,故《禹貢》必曰「浮磬」者,遠土而近於水,取之實難。昔奉常所用,乃以白石為之,其聲沉下,制
 作簡質,理宜改造焉。



 絲部有五:曰一弦琴,曰三弦琴,曰五弦琴,曰七弦琴,曰九弦琴,曰瑟。其說以謂:漢津誦其師之說曰:「古者,聖人作五等之琴,琴主陽,一、三、五、七、九,生成之數也。師延拊一弦之琴,昔人作三弦琴,蓋陽之數成於三。伏羲作琴有五弦,神農氏為琴七弦,琴書以九弦象九星。五等之琴,額長二寸四分,以像二十四氣;岳闊三分,以像三才;岳內取聲三尺六寸,以象期三百六十日:龍斷及折勢四分,以象四時:共長三尺九寸一
 分,成於三,極於九。九者,究也,復變而為一之義也。《大晟》之瑟長七尺二寸,陰爻之數二十有四,極三才之陰數而七十有二,以像一歲之候。既罷箏、築、阮,絲聲稍下,乃增瑟之數為六十有四,則八八之數法乎陰,琴之數則九十有九而法乎陽。」



 竹部有三:曰長笛,曰篪,曰簫。其說以謂:笛以一管而兼律呂,眾樂由焉。三竅成鑰,三才之和寓焉。六竅為笛,六律之聲備焉。篪之制,採竹竅厚均者,用兩節,開六孔,以備十二律之聲,則篪之樂生於律。
 樂始於律而成於簫。律準鳳鳴,以一管為一聲。簫集眾律,編而為器:參差其管,以象鳳翼,簫然清亮,以象鳳鳴。



 匏部有六:曰竽笙,曰巢笙,曰和笙,曰閏餘匏,曰九星匏,曰七星匏。其說以謂:列其管為簫,聚其管為笙。鳳凰于飛,簫則象之;鳳凰戾止,笙則象之。故內皆用簧,皆施匏於下。前古以三十六簧為竽,十九簧為巢,十三簧為和,皆用十九數,而以管之長短、聲之大小為別。八音之中,匏音廢絕久矣。後世以木代匏,乃更其制,下皆用匏,而
 並造十三簧者,以象閏餘。十者,土之成數;三者,木之生數,木得土而能生也。九簧者,以象九星。物得陽而生,九者,陽數之極也。七簧者,以象七星。笙之形若鳥斂翼,鳥,火禽,火數七也。



 土部有一:曰塤。其說以謂:釋《詩》者以塤、篪異器而同聲,然八音孰不同聲,必以塤、篪為況?嘗博詢其旨,蓋八音取聲相同者,惟塤、篪為然。塤、篪皆六孔而以五竅取聲。十二律始於黃鐘,終於應鐘。二者,其竅盡合則為黃鐘,其竅盡開則為應鐘,餘樂不然。故惟塤、篪
 相應。



 革部十有二:曰晉鼓,曰建鼓,曰□鼓,曰雷鼓,曰雷□,曰靈鼓,曰靈□,曰路鼓,曰路□,曰雅鼓,曰相鼓,曰搏拊。其說以謂:凡言樂者,必曰鐘鼓,蓋鐘為秋分之音而屬陰,鼓為春分之音而屬陽。金奏待鼓而後進者,雷發聲而後群物皆鳴也;鼓復用金以節樂者,雷收聲而後蟄蟲坯戶也。《周官》以晉鼓鼓金奏,陽為陰唱也。建鼓,少昊氏所造,以節眾樂。夏加四足,謂之足鼓;商貫之以柱,謂之楹鼓;周縣而擊之,謂之縣鼓。□者,鼓之兆也。天子
 錫諸侯樂,以柷將之;賜伯、子、男樂,以□將之。柷先眾樂,□則先鼓而已。以雷鼓鼓天神,因天聲以祀天也;以靈鼓鼓社祭,以天為神,則地為靈也;以路鼓鼓鬼享,人道之大也。以舞者迅疾,以雅節之,故曰雅鼓。相所以輔相於樂,今用節舞者之步,故曰相鼓。登歌今奏擊拊,以革為之,實之以糠,升歌之鼓節也。



 木部有二:曰柷,曰吾文。其說以謂:柷之作樂。吾文之止樂,漢津嘗問於李良,良曰:「聖人制作之旨,皆在《易》中。《易》曰:『《震》,起也。《艮》,止也。』柷、吾文之義,
 如斯而已。柷以木為底,下實而上虛。《震》一陽在二陰之下,像其卦之形也。擊其中,聲出虛,為眾樂倡。《震》為雷,雷出地奮,為春分之音,故為眾樂之倡,而外飾以山林物生之狀。《艮》位寅,為虎,虎伏則以象止樂。背有二十七刻,三九陽數之窮。戛之以笙,裂而為十,古或用十寸,或裂而為十二,陰數。十二者,二六之數,陽窮而以陰止之。」



 又有度、量、權、衡四法,候氣、運律、教樂、運譜四議,與律歷、運氣或相表裏,甚精微矣,茲獨採其言樂事顯明者。幾為
 書二十卷。說者以謂蔡京使昺為緣飾之,以布告天下云。



 政和二年,賜貢士聞喜宴於闢雍,仍用雅樂,罷瓊林苑宴。兵部侍郎劉煥言:「州郡歲貢士,例有宴設,名曰『鹿鳴』,乞於斯時許用雅樂,易去倡優淫哇之聲。」八月,太常言:「宗廟、太社、太稷並為大祠,今太社、太稷登歌而不設宮架樂舞,獨為未備,請迎神、送神、詣罍洗、歸復位、奉俎、退文舞、迎武舞、亞終獻、望燎樂曲,並用宮架樂,設於北墉之北。」詔皆從之。



 三年四月,議禮局上親祠登歌之制
 大朝會同:



 金鐘一,在東;玉磬一,在西:俱北向。柷一,在金鐘北,稍西;敔一,在玉磬北,稍東。搏拊二:一在柷北,一在吾文北,東西相向。一弦、三弦、五弦、七弦、九弦琴各一,瑟四,在金鐘之南,西上;玉磬之南亦如之,東上。又於午階之東太廟則於泰階之東,宗祀則於東階之西,大朝會則於丹墀香案之東,設笛二、篪一、巢笙二、和笙三,為一列,西上大朝會,和笙在笛南。塤一,在笛南大朝會在篪南。閏餘匏一,簫一,各在巢笙南。又於午階之西太廟則於泰階之西,宗祀則於西階之東,大朝會則於丹墀香案之西,設笛二、篪一、巢笙二、和笙二,為一
 列,東上。塤一,在笛南。七星匏一、九星匏一,在巢笙南。簫一,在九星匏西。鐘、磬、柷敔、博拊、琴、瑟工各坐於壇上太廟、宗祀、大朝會則於殿上,塤、篪、笙、笛、簫、匏工並立於午階之東西太廟則於泰階之東西,宗祀則於兩階之間,大朝會則於丹墀香案之東西。樂正二人在鐘、磬南,歌工四人在敔東,俱東西相向。執麾挾仗色掌事一名,在樂虡之西,東向。樂正紫公服大朝會服絳朝服,方心曲領、緋白大帶、金銅革帶、烏皮履,樂工黑介幘,執麾人平巾幘:並緋繡鸞衫、白絹夾褲、抹帶大朝會同。



 又上親祠宮架之制景靈宮、宣德門、大朝會附:



 四方各設編鐘三、編磬三。東方,編鐘起北,編磬間之,東向。西方,編磬起北,編鐘間之,西向。南方,編磬起西,編鐘間之;北方,編鐘起西,編磬間之:俱北向。設十二鎛鐘、特磬於編架內,各依月律。四方各鎛鐘三、特磬三。東方,鎛鐘起北,特磬間之,東向。西方,特磬起北,鎛鐘間之。西向。南方,特磬起西,鎛鐘間之;北方,鎛鐘起西,特磬間之,皆北向景靈宮、天興殿鎛鐘、編鐘、編磬如每歲大祠宮架陳設。



 植建鼓、鞞鼓、應鼓於四隅,建鼓在中,鞞鼓在左,應鼓在右。設柷、敔於北架內:柷一,
 在道東;敔一,在道西。設瑟五十二朝會五十六。宣德門五十四,列為四行:二行在柷東,二行在敔西。次,一弦琴七,左四右三。次三弦琴一十有八;宣德門二十。



 次五弦琴一十有八宣德門二十。並分左右。次七弦琴二十有三,次九弦琴二十有三,並左各十有二,右各十有一宣德門七弦、九弦各二十五,並左十有三,右十有二。次巢笙二十有八,分左右宣德門三十二。次匏笙三,在巢笙之間,左二、右一。次簫二十有八宣德門、大朝會三十。次竽二十,次篪二十有八宣德門三十六。朝會笛三十三:左十有七,右十有六。次塤一十有八宣德
 門、朝會二十。次笛二十有八,並分左右宣德門笛三十六:朝會三十三,左十有七,右十有六。雷鼓、雷□各一,在左;又雷鼓、雷□各一,在右地祇:靈鼓、靈□各二。太廟:路鼓、路□各二。大朝會晉鼓二。宣德門不設。並在三弦、五弦琴之間,東西相向,晉鼓一,在匏笙間,少南北向。



 副樂正二人,在柷、吾文之前,北向。歌工三十有二宣德門四十。朝會三十有六。次柷、敔,東西相向,列為四行,左右各二行。樂師四人,在歌工之南北,東西相向。運譜二人,在晉鼓之左右,北向。執麾挾仗色掌事一名,在樂虡之右,東向。副樂正同樂正服大朝會同
 樂正朝服,樂師緋公服,運譜緣公服大朝會介幘、絳鞁衣、白絹抹帶,樂工執麾人並同登歌執麾人服朝會同。



 又上親祠二舞之制大朝會同:



 文舞六十四人,執鑰翟;武舞六十四人,執干戚,俱為八佾。文舞分立於表之左右,各四佾。引文舞二人,執纛在前,東西相向。舞色長二人,在執纛之前,分東西若武舞則在執旌之前。引武舞,執旌二人,□二人,雙鐸二人,單鐸二人,鐃二人,持金錞四人,奏金錞二人,鉦二人,相二人,雅二人,各立於宮架之東西,北向,北上,武舞在其後。舞色長帕
 頭、抹額、紫繡袍。引二舞頭及二舞郎,並紫平冕、皂繡鸞衫、金銅革帶、烏皮履大朝會引文舞頭及文舞郎並進賢冠、黃鸞衫、銀褐裙、綠衣盍襠、革帶、烏皮履;引武舞頭及武舞郎並平巾幘、緋鸞衫、黃畫甲身,紫衣盍襠、豹文大口褲、起梁帶,烏皮鞁。



 引武舞人,武弁、緋繡鸞衫、抹額、紅錦臂鞁、白絹褲、金銅革帶、烏皮履大朝會同。



 又上大祠、中祠登歌之制:



 編鐘一,在東;編磬一,在西:俱北向。柷一,在編鐘之北,稍西;敔一,在編磬之北,稍東。搏拊二:一在柷北,一在吾文北,俱東西相向。一弦、三弦、五弦、七弦、九弦琴各一,瑟一,在編鐘之南,西上。編磬
 之南亦如之,東上。壇下午階之東太廟、別廟則於殿下泰階之東,明堂、祠廟則於東階之西,設笛一、篪一、塤一,為一列,西上。和笙一,在笛南;巢笙一,在篪南;簫一,在塤南。午階之西亦如之,東上太廟、別廟則於泰階之西,明堂、祠廟則於西階之東。鐘、磬、柷、吾文、搏拊、琴、瑟工各坐於壇上明堂、太廟、別廟於殿上,祠廟於堂上,塤、篪、笙、笛、簫工並立於午階東西太廟、別廟於太階之東西,明堂、祠廟於兩階之間,若不用宮架,即登歌工人並坐。樂正二人在鐘、磬南,歌工四人在敔東,俱東西相向。執麾挾仗色掌事一名,在樂虡之西,東向。樂正公服,執麾挾仗色掌事
 平巾幘,樂工黑介幘,並緋繡鸞衫、白絹抹帶三京帥府等每歲祭社稷,祀風師、雨師、雷神,釋奠文宣王,用登歌樂,陳設樂器並同,每歲大、中祠登歌。



 又上太祠宮架、二舞之制:



 四方各設鎛鐘三,各依月律。編鐘一,編磬一。北方,應鐘起西,編鐘次之,黃鐘次之,編磬次之,大呂次之,皆北向。東方,太簇起北,編鐘次之,夾鐘次之,編磬次之,姑洗次之,皆東向。南方,仲呂起東,編鐘次之,蕤賓次之,編磬次之,林鐘次之,皆北向。西方,夷則起南,編鐘次之,南呂次之,編磬次之,無射次之,皆西向。設十二特
 磬,各在鎛鐘之內。



 植建鼓、鞞鼓、應鼓於四隅。設柷、敔於北架內,柷在左,敔在右。雷鼓、雷□各二地祇以靈鼓,靈□,太廟、別廟以路鼓、路□。分東西,在歌工之側。瑟二,在柷東。次,一弦、三弦、五弦、七弦、九弦琴各二,各為一列。吾文西亦如之。巢笙、簫、竽、篪、塤、笛各四,為四列,在雷鼓之後若地祇即在靈鼓後,太廟、別廟在路鼓後。晉鼓一,在笛之後:俱北向。副樂正二人在柷、敔之北。歌工八人,左右各四,在柷、敔之南,東西相向。執麾挾仗色掌事一名,在宮架西,北向。副樂正本色公服,執麾挾仗色
 掌事及樂正平巾幘,服同登歌樂工凡軒架之樂三面,其制,宮架之南機;判架之樂二面,其制,又去軒架之北面;特架之樂一面。文武二舞並同親祠,惟二舞郎並紫平冕、皂繡袍、銀褐裙、白絹抹帶,與親祠稍異。



 詔並頒行。



 五月,帝御崇政殿,親按宴樂,召侍從以上侍立。詔曰:「《大晟》之樂已薦之郊廟,而未施於宴饗。比詔有司,以《大晟》樂播之教坊,試於殿庭,五聲既具,無惉懘焦急之聲,嘉與天下共之,可以所進樂頒之天下,其舊樂悉禁。」於是令尚書省立法,新徵、角二調曲譜已經按試
 者,並令大晟府刊行,後續有譜,依此。其宮、商、羽調曲譜自從舊,新樂器五聲、八音方全。塤、篪、匏、笙、石磬之類已經按試者,大晟府畫圖疏說頒行,教坊、鈞容直、開封府各頒降二副。開封府用所頒樂器,明示依式造粥,教坊、鈞容直及中外不得違。今輒高下其聲,或別為他聲,或移改增損樂器,舊來淫哇之聲,如打斷、哨笛、呀鼓、十般舞、小鼓腔、小笛之類與其曲名,悉行禁止,違者與聽者悉坐罪。



 八月,大晟府奏,以雅樂中聲播於宴樂,舊闕徵、
 角二調,及無土、石、匏三音,今樂並已增入。詔頒降天下。九月,詔:「《大晟樂》頒於太學、闢雍,諸生習學,所服冠以弁,袍以素紗、皂緣,紳帶,佩玉。」從劉昺制也。



 昺又上言曰:「五行之氣,有生有克,四時之禁,不可不頒示天下。盛德在木,角聲乃作,得羽而生,以徵為相;若用商則刑,用宮則戰,故春禁宮、商。盛德在火,徵聲乃作,得角而生,以宮為相;若用羽則刑,用商則戰,故夏禁商、羽。盛德在土,宮聲乃作,得征而生,以商為相;若用角則刑,用羽則戰,故季
 夏土王,宜禁角、羽。盛德在金,商聲乃作,得宮而生,以羽為相;若用征則刑,用角則戰,故秋禁征、角。盛德在水,羽聲乃作,得商而生,以角為相;若用宮則刑,用征則戰,故冬禁宮、征。此三代之所共行,《月令》所載,深切著明者也。作樂本以導和,用失其宜,則反傷和氣。夫淫哇殽雜,干犯四時之氣久矣。陛下親灑宸翰,發為詔旨,淫哇之聲轉為雅正,四時之禁亦右所頒,協氣則粹美,繹如以成。」詔令大晟府置圖頒降。



 四年正月,大晟府言:「宴樂諸宮
 調多不正,如以無射為黃鐘宮,以夾鐘為中呂宮,以夷則為仙呂宮之類。又加越調、雙調、大食、小食,皆俚俗所傳,今依月律改定。」詔可。



 六年,詔:「先帝嘗命儒臣肇造玉磬,藏之樂府,久不施用,其令略加磨礱,俾與律合。並造金鐘,專用於明堂。」又詔:「《大晟》雅樂,頃歲已命儒臣著樂書,獨宴樂未有紀述。其令大晟府編集八十四調並圖譜,令劉昺撰以為《宴樂新書》。」十月,臣僚乞以崇寧、大觀、政和所得珍瑞名數,分命儒臣作為頌詩,協以新律,薦
 之郊廟,以告成功。詔送禮制局。



 七年二月,典樂裴宗元言:「乞按習《虞書》賡載之歌,夏《五子之歌》,商之《那》,周之《關雎》、《麟趾》、《騶虞》、《鵲巢》、《鹿鳴》、《文王》、《清廟》之詩。」詔可。中書省言:「高麗,賜雅樂,乞習教聲律、大晟府撰樂譜辭。」詔許教習,仍賜樂譜。



 三月,議禮局言:「先王之制,舞有小大:文舞之大,用羽、鑰;文舞之小,則有羽無鑰,謂之羽舞。武舞之大,用干、戚;武舞之小,則有乾無戚,謂之干舞。武又有戈舞焉,而戈不用於大舞。近世武舞以戈配乾,未嘗用戚。
 乞武舞以戚配乾,置戈不用,庶協古制。」



 又言:「伶州鳩曰:『大鈞有鎛無鐘,鳴其細也;細鈞有鐘無鎛,昭其大也。』然則鐘,大器也;鎛,小鐘也。以宮、商為鈞,則謂之大鈞,其聲大,故用鎛以鳴其細,而不用鐘;以角、征、羽為鈞,則謂之小鈞,其聲細,故用鐘以昭其大,而不用鎛。然後細大不逾,聲應相保,和平出焉。是鎛、鐘兩器,其用不同,故周人各立其官。後世之鎛鐘,非特不分大小,又混為一器,復於樂架編鐘、編磬之外,設鎛鐘十二,配十二辰,皆非是。
 蓋鎛鐘猶之特磬,與編鐘、編磬相須為用者也。編鐘、編磬,其陽聲六,以應律;其陰聲六,以應呂。既應十二辰矣,復為鎛鐘十二以配之,則於義生復。乞宮架樂去十二鎛鐘,止設一大鐘為鐘、一小鐘為鎛、一大磬為特磬,以為眾聲所依。」詔可。



 四月,禮制局言:「尊祖配天者,郊祀也;嚴父配天者,明堂也。所以來天神而禮之,其義一也。則明堂宜同郊祀,用禮天神六變之樂,其宮架赤紫,用雷鼓、雷□。又圜丘方澤,各有大樂宮架,自來明堂就用大慶殿大朝會宮
 架。今明堂肇建,欲行創置。」



 十月,皇帝御明堂平朔左個,始以天運政治頒於天下。是月也,凡樂之聲,以應鐘為宮、南呂為商、林鐘為角、仲呂為閏徵、姑洗為徵、太簇為羽、黃鐘為閏宮。既而中書省言:「五聲、六律、十二管還相為宮,若以左旋取之,如十月以應鐘為宮,則南呂為商、林鐘為角、仲呂為閏徵、姑洗為徵、太簇為羽、黃鐘為閏宮;若以右旋七均之法,如十月以應鐘為宮,則當用大呂為商、夾鐘為角、仲呂為閏徵、蕤賓為徵、夷則為羽、無
 射為閏宮。明堂頒朔,欲左旋取之,非是。欲以本月律為宮,右旋取七均之法。」從之,仍改正詔書行下。



 自是而後,樂律隨月右旋。



 仲冬之月,皇帝御明堂,南面以朝百闢,退,坐於平朔,授民時。樂以黃鐘為宮、太簇為商、姑洗為角、蕤賓為閏徵、林鐘為徵、南呂為羽、應鐘為閏宮。調以羽,使氣適平。



 季冬之月,御明堂平朔右個。樂以大呂為宮、夾鐘為商、仲呂為角、林鐘為閏徵、夷則為徽、無射為羽、黃鐘為閏宮。客氣少陰火,調以羽,尚羽而抑征。



 孟春
 之月,御明堂青陽左個。樂以太簇為宮、姑洗為商、蕤賓為角、夷則為閏徵、南呂為徵、應鐘為羽、大呂為閏宮。客氣少陽相火,與歲運同,火氣太過,調宜羽,致其和。



 仲春之月,御明堂青陽。樂以夾鐘為宮、仲呂為商、林鐘為角、南呂為閏徵、無射為徵、黃鐘為羽、太簇為閏宮。調以羽。



 季春之月,御明堂青陽右個。樂以姑洗為宮、蕤賓為商、夷則為角、無射為閏徵、應鐘為徵、大呂為羽、夾鐘為閏宮。客氣陽明,尚征以抑金。



 孟夏之月,御明堂左個。樂以
 仲呂為宮、林鐘為商、南呂為角、應鐘為閏徵、黃鐘為徵、太簇為羽、姑洗為閏宮。調宜尚征。



 仲夏之月,御明堂。樂以蕤賓為宮、夷則為商、無射為角、黃鐘為閏徵、大呂為徵、夾鐘為羽、仲呂為閏宮。客氣寒水,調宜尚宮以抑之。



 季夏之月,御明堂右個。樂以林鐘為宮、南呂為商、應鐘為角、大呂為閏徵、太簇為徵、姑洗為羽、蕤賓為閏宮。調宜尚宮,以致其和。



 孟秋之月,御明堂總章左個。樂以夷則為宮、無射為商、黃鐘為角、太簇為閏徵、夾鐘為徵、仲
 呂為羽、林鐘為閏宮。調宜尚商。



 仲秋之月,御明堂總章。樂以南呂為宮、應鐘為商、大呂為角、夾鐘為閏徵、姑洗為徵、蕤賓為羽、夷則為閏宮。調宜尚商。



 季秋之月,御明堂總章右個。樂以無射為宮、黃鐘為商、太簇為角、姑洗為閏徵、仲呂為徵、林鐘為羽、南呂為閏宮。調宜尚羽,以致其平。



 閏月,御明堂,闔左扉。樂以其月之律。



 十一月,知永興軍席旦言:「太學、闢雍士人作樂,皆服士服,而外路諸生尚衣衣蘭帕,望下有司考議,為圖式以頒外郡。」



 八年
 八月,宣和殿大學士蔡攸言:「九月二日,皇帝躬祀明堂,合用大樂。按《樂書》:『正聲得正氣則用之,中聲得中氣則用之。』自八月二十八日,已得秋分中氣,大饗之日當用中聲樂。今看詳古之神瞽考中聲以定律,中聲謂黃鐘也,黃鐘即中聲,非別有一中氣之中聲也。考閱前古,初無中、正兩樂。若以一黃鐘為正聲,又以一黃鐘為中聲,則黃鐘君聲,不當有二。況帝指起律。均法一定,大呂居黃鐘之次,陰呂也,臣聲也。今減黃鐘三分,則入大呂律
 矣。易其名為黃鐘中聲,不唯紛更帝律,又以陰呂臣聲僭竊黃鐘之名。若依《樂書》『正聲得正氣則用之,中聲得中氣則用之』,是冬至祀天、夏至祭地,常不用正聲而用中聲也。以黃鐘為正聲,易大呂為中聲之黃鐘,是帝律所起,黃鐘常不用而大呂常用也。抑陽扶陰,退律進呂,為害斯大,無甚於此。今來宗祀明堂,緣八月中氣未過,而用中聲樂南呂為宮,則本律正聲皆不得預。欲乞廢中聲之樂,一遵帝律,止用正聲,協和天人,刊正訛謬,著
 於《樂書》。」詔可。攸又乞取已頒中聲樂在天下者。



 宣和元年四月,攸上書:



 奉詔制造太、少二音登歌宮架,用於明堂,漸見就緒,乞報大晟府者凡八條:



 一,太、正、少鐘三等。舊制,編鐘、編磬各一十六枚,應鐘之外,增黃鐘、大呂、太簇、夾鐘四清聲。今既分太、少,則四清聲不當兼用,止以十二律正聲各為一架。



 其二,太、正、少琴三等。舊制、一、三、五、七、九弦凡五等。今來討論,並依《律書》所載,止用五弦。弦大者為宮而居中央,君也。商張右傍,其餘大小相次,
 不失其序,以為太、正、少之制,而十二律舉無遺音。其一、三、五、七、九弦,太、少樂內更不制造。其三,太、正、少鑰三等。謹按《周官》鑰章之職,龠□以迎寒暑。王安石曰:「鑰,三孔,律呂於是乎生,而其器不行於世久矣。近得古鑰,嘗以頒行。」今如《爾雅》所載,制造太、正、少三等,用為樂本,設於眾管之前。



 其四,太正少笛、塤、篪、簫各三等。舊制,簫一十六管,如鐘磬之制,有四清聲。今既分太、少,其四清聲亦不合兼用,止用十二管。



 其五,大晟匏有三色:一曰七星,二
 曰九星,三曰閏餘,莫見古制。匏備八音,不可闕數,今已各分太、正、少三等,而閏餘尤無經見,唯《大晟樂書》稱「匏造十三簧者,以象閏餘。十者,土之成數;三者,木之生數;木得土而能生也。」故獨用黃鐘一清聲。黃鐘清聲,無應閏之理,今去閏餘一匏,止用兩色,仍改避七星、九星之名,止曰七管、九管。



 其六,舊制有巢笙、竽笙、和笙。巢笙自黃鐘而下十九管,非古制度。其竽笙、和笙並以正律林鐘為宮,三笙合奏,曲用兩調,和笙奏黃鐘曲,則巢笙奏
 林鐘曲以應之,宮、徵相雜。器本宴樂,今依鐘磬法,裁十二管以應十二律,為太、正、少三等,其舊笙更不用。



 其七,柷、吾文、晉鼓、鎛鐘、特磬,雖無太、少,系作止和樂,合行備設。



 其八,登歌宮架有搏拊二器,按《虞書》:「戛擊鳴球,搏拊琴瑟。」王安石解曰:「或戛或擊,或搏或拊。」與《虞書》所載乖戾。今欲乞罷而不用。



 詔悉從之。



 攸之弟絳曰:



 初,漢津獻說,請帝三指之三寸,三合而為九,為黃鐘之律。又以中指之徑圍為容盛,度量權衡皆自是而出。又謂:「有太聲、有
 少聲。太者,清聲,陽也,天道也;少者,濁聲,陰也,地道也;中聲,其間,人道也。合三才之道,備陰陽之奇偶,然後四序可得而調,萬物可得而理。」當時以為迂怪。



 劉昺之兄煒以嘵樂律進,未幾而卒。昺始主樂事,乃建白謂:太、少不合儒書。以太史公《書》黃鐘八寸七分管為中聲,奏之於初氣;班固《書》黃鐘九寸管為正聲,奏之於中氣。因請帝指時止用中指,又不得徑圍為容盛,故後凡制器,不能成劑量,工人但隨律調之,大率有非漢津之本說者。



 及
 政和末,明堂成,議欲為布政調燮事,乃召武臣前知憲州任宗堯換朝奉大夫,為大晟府典樂。宗堯至,則言:太、少之說本出於古人,雖王樸猶知之,而劉昺不用。乃自創黃鐘為兩律。黃鐘,君也,不宜有兩。



 蔡攸方提舉大晟府,不喜佗人預樂。有士人田為者,善琵琶,無行,攸乃奏為大晟府典樂,遂不用中聲八寸七分管,而但用九寸管。又為一律長尺有八寸,曰太聲;一律長四寸有半,曰少聲:是為三黃鐘律矣。律與容盛又不翅數倍。黃鐘既
 四寸有半,則圜鐘幾不及二寸。諸器大小皆隨律,蓋但以器大者為太,小者為少。樂始成,試之於政事堂,執政心知其非,然不敢言,因用之於明堂布政,望鶴愈不至。



 絳又曰:「宴樂本雜用唐聲調,樂器多夷部,亦唐律。徵、角二調,其均自隋、唐間已亡。政和初,命大晟府改用大晟律,其聲下唐樂已兩律。然劉昺止用所謂中聲八寸七分管為之,又作匏、笙、塤、篪,皆入夷部。至於《征招》、《角招》,終不得其本均,大率皆假之以見徵音。然其曲譜頗和美,
 故一時盛行於天下,然教坊樂工嫉之如仇。其後,蔡攸復與教坊用事樂工附會,又上唐譜徵、角二聲,遂再命教坊制曲譜,既成,亦不克行而止。然政和《征招》、《角招》遂傳於世矣。」



 二年八月,罷大晟府制造所並協律官。四年十月,洪州奏豐城縣民鋤地得古鐘,大小九具,狀制奇異,各有篆文。驗之《考工記》,其制正與古合。令樂工擊之,其聲中律之無射。繪圖以聞。七年十二月,金人敗盟,分兵兩道入,詔革弊事,廢諸局,於是大晟府及教樂所、教
 坊額外人並罷。靖康二年,金人取汴,凡大樂軒架、樂舞圖、舜文二琴、教坊樂器、樂書、樂章、明堂布政閏月體式、景陽鐘並虡、九鼎皆亡矣。



\end{pinyinscope}