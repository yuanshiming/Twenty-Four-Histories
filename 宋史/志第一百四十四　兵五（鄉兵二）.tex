\article{志第一百四十四 兵五(鄉兵二)}

\begin{pinyinscope}

 陜西保毅河北忠順河北陜西強人砦戶河北河東強壯河東陜西弓箭手河北等路弓箭社



 鄉兵者,選自戶籍,或土民應募,在所團結訓練,以為防
 守之兵也。周廣順中,點秦州稅戶充保毅軍,宋因之。自建隆四年,分命使臣往關西道,令調發鄉兵赴慶州。咸平四年,令陜西系稅人戶家出一丁,號曰保毅,官給糧賜,使之分番戍守。五年,陜西緣邊丁壯充保毅者至六萬八千七百七十五人。七月,以募兵離去鄉土,有傷和氣,詔諸州點充強壯戶者,稅賦止令本州輸納,有司不得支移之。先是,河北忠烈、宣勇無人承替者,雖老疾不得停籍。至是,詔自今委無家業代替者,放令自便。自是
 以至天禧間,並、代廣銳老病之兵,雖非親屬而願代者聽。河北強壯,恐奪其農時,則以十月至正月旬休日召集而教閱之。忠烈、宣勇、廣銳之歸農而闕員者,並自京差補;戍於河上而歲月久遠者,則特為遷補;貧獨而無力召替者,則令逐處保明放停。



 當是時,河北、河東有神銳、忠勇、強壯,河北有忠順、強人,陜西有保毅、砦戶、強人、強人弓手,河東、陜西有弓箭手,河北東、陜西有義勇,麟州有義兵,川陜有土丁、壯丁,荊湖南、北有弩手、土丁,廣
 南東、西有槍手、土丁,邕州有溪洞壯丁、土丁,廣南東、西有壯丁。



 當仁宗時,神銳、忠勇、強壯久廢,忠順、保毅僅有存者。康定初,詔河北、河東添籍強壯,河北凡二十九萬三千,河東十四萬四千,皆以時訓練。自西師屢衄,正兵不足,乃籍陜西之民,三丁選一,以為鄉弓手。未幾,刺充保捷,為指揮一百八十五,分戍邊州。西師罷,多揀放焉。慶歷二年,籍河北強壯,得二十九萬五千,揀十之七為義勇,且籍民丁以補其不足。河東揀籍如河北法。



 其後,
 議者論「義勇為河北伏兵,以時講習,無待儲廩,得古寓兵於農之意。惜其束於列郡,止以為城守之備。誠能令河北邢、冀二州分東西兩路,命二郡守分領,以時閱習,寇至,即兩路義勇翔集赴援,使其腹背受敵,則河北三十餘所常伏銳兵矣」。朝廷下其議,河北帥臣李昭亮等議曰:「昔唐澤潞留後李抱真籍戶丁男,三選其一,農隙則分曹角射,歲終都試,以示賞罰,三年皆善射,舉部內得勁卒二萬。既無廩費,府庫益實,乃繕甲兵為戰具,遂
 雄視山東。是時,天下稱昭義步兵冠於諸軍,此近代之顯效,而或謂民兵只可城守,難備戰陣,非通論也。但當無事時,便分義勇為兩路,置官統領,以張用兵之勢,外使敵人疑而生謀,內亦搖動眾心,非計之得。姑令在所點集訓練,三二年間,武藝稍精,漸習行陣。遇有警,得將臣如抱真者統馭,制其陣隊,示以賞罰,何敵不可戰哉?至於部分布列,量敵應機,系於臨時便宜,亦難預圖。況河北、河東皆邊州之地,自置義勇,州縣以時按閱,耳目
 已熟,行固無疑。」詔如所議。



 治平元年,宰相韓琦言:「古者籍民為兵,數雖多而贍至薄。唐置府兵,最為近之,後廢不能復。今之義勇,河北幾十五萬,河東幾八萬,勇悍純實,出於天性,而有物力資產,父母妻子之所系,若稍加練簡,與唐府兵何異?陜西嘗刺弓手為保捷,河北、河東、陜西,皆控西北,事當一體。請於陜西諸州亦點義勇,止涅手背,一時不無小擾,終成長利。」天子納其言,乃遣籍陜西義勇,得十三萬八千四百六十五人。



 是時,諫官司
 馬光累奏,謂:「陜西頃嘗籍鄉弓手,始諭以不去鄉里。既而涅為保捷正兵,遣戍邊州,其後不可用,遂汰為民,徒使一路騷然,而於國無補。且祖宗平一海內,曷嘗有義勇哉?自趙元昊反,諸將覆師相繼,終不能出一旅之眾,涉區脫之地。當是時,三路鄉兵數十萬,何嘗得一人之力?議者必曰:『河北、河東不用衣廩,得勝兵數十萬,閱教精熟,皆可以戰;又兵出民間,合於古制。』臣謂不然。彼數十萬者,虛數也;閱教精熟者,外貌也;兵出民間者,名與
 古同而實異。蓋州縣承朝廷之意,止求數多。閱教之日,觀者但見其旗號鮮明,鉦鼓備具,行列有序,進退有節,莫不以為真可以戰。殊不知彼猶聚戲,若遇敵,則瓦解星散,不知所之矣。古者兵出民間,耕桑所得,皆以衣食其家,故處則富足,出則精銳。今既賦斂農民粟帛以給正軍,又籍其身以為兵,是一家而給二家之事也。如此,民之財力安得不屈?臣愚以為河北、河東已刺之民,猶當放還,況陜西未刺之民乎?」帝弗聽。於是三路鄉兵,唯
 義勇為最盛。



 熙寧以來,則尤重蕃兵、保甲之法,餘多承舊制。前史沿革,不復具述,取其有損益者著於篇。南渡而後,土宇雖不及前,而兵制多仍其故,凡其鄉兵、砦兵之可改者,皆附見焉。



 陜西保毅開寶八年,發渭州平涼、潘原二縣民治城隍,因立為保毅弓箭手,分戍鎮砦。能自置馬,免役。逃、死,以親屬代,因周廣順舊制也。



 咸平初,秦州極邊止置千人,分番守戍。上番人月給米六斗,仲冬,賜指揮使至副都
 頭紫綾綿袍,十將以下皂綾袍。五年,點陜西沿邊丁壯充保毅,凡得六萬八千人。給資糧,與正兵同戍邊郡。



 慶歷初,詔悉刺為保捷軍,唯秦州增置及三千人,環、慶、保安亦各籍置。是時,諸州總六千五百十八人,為指揮三十一。



 皇祐五年,涇原都總管程戡上言:「陜西保毅,近歲止給役州縣,無復責以武技。自黥刺為保捷,而家猶不免於保毅之籍,或折賣田產,而得產者以分數助役。今秦州僅三千人,久廢農業,請罷遣。」詔自今敢私役者,計
 傭坐之。治平初,詔置保毅田承名額者,悉揀刺以為義勇。熙寧四年,詔廢其軍。



 環慶砦戶、強人弓手,九年,詔如禁軍法,上其籍,隸於馬軍司,廩給視中禁軍。



 河北忠順自太宗朝以瀛、莫、雄、霸州、乾寧、順安、保定軍置忠順,凡三千人,分番巡徼,隸沿邊戰棹巡檢司。自十月悉上,人給糧二升,至二月輪半營農。慶歷七年,夏竦建議與正兵參戍。八年,以水沴,多逋亡者,權益正兵代其闕額。皇祐四年,權放業農,後不復補。



 河北陜西強人、砦戶、強人弓手名號不一。咸平四年,募河北民諳契丹道路、勇銳可為間伺者充強人,置都頭、指揮使。無事散處田野,寇至追集,給器甲、口糧、食錢,遣出塞偷斫賊壘,能斬首級、奪馬者如賞格。虜獲財畜皆畀之。慶歷二年,環州亦募,涅手背,自備戎械並馬,置押官、甲頭、隊長,戶四等以下免役,上番防守,月給奉廩。三年,涇原路被邊城砦悉置。



 環、慶二州復有砦戶,康定中,以沿邊弓手涅手背充,有警召集防戍,與保毅弓手同。



 大順城、西谷砦有強人弓手,天禧、慶歷間募置,番戍為巡徼斥候,日給糧。人賦田八十畝,能自備馬者益賦四十畝。遇防秋,官給器甲,下番隨軍訓練。為指揮六。



 河北、河東強壯五代時,瀛、霸諸州已置。咸平三年,詔河北家二丁、三丁籍一,四丁、五丁籍二,六丁、七丁籍三,八丁以上籍四,為強壯。五百人為指揮,置指揮使;百人為都,置正、副都頭二人、節級四人。所在置籍,擇善騎射者第補校長,聽自置馬,勝甲者蠲其戶役。五年,募其勇敢,團結附
 大軍為柵,官給鎧甲。景德元年,遣使分詣河北、河東集強壯,借庫兵給糧訓練,非緣邊即分番迭教,寇至悉集守城,寇退營農。



 至康定初,州縣不復閱習,其籍多亡。乃詔二路選補,增其數,為伍保,迭糾游惰及作奸者。二十五人為團,置押官;四團為都,置正、副都頭各一人;五都為指揮,置指揮使,各以階級伏事。年二十系籍,六十免,取家人或他戶代之。歲正月,縣以籍上州,州以籍奏兵部,按舉不如法者。慶歷二年,悉揀以為義勇,不預者釋之,而存其籍,以備守葺城
 池。而河東強壯自此浸廢矣。



 其募于河北者,舊給塘泊河淤之田,力不足以耕,重苦番教,應募者寡。熙寧七年罷之,以其田募民耕,戶兩頃,蠲其賦,以為保甲。



 河東、陜西弓箭手周廣順初,鎮州諸縣,十戶取材勇者一人為之,餘九戶資以器甲芻糧。建隆二年,詔釋之,凡一千四百人。



 景德二年,鎮戎軍曹瑋言:「有邊民應募為弓箭手者,請給以閑田,蠲其徭賦,有警,可參正兵為前鋒,而官無資糧戎械之費。」詔:「人給田二頃,出甲士一人,
 及三頃者出戰馬一匹。設堡戍,列部伍,補指揮使以下,據兵有功勞者,亦補軍都指揮使,置巡檢以統之。」其後,鄜延、環慶、涇原並河東州軍亦各募置。



 慶歷中,諸路總三萬二千四百七十四人,為指揮一百九十二。是時,河東都轉運使歐陽修言:「代州、岢嵐、寧化、火山軍被邊地幾二三萬頃,請募人墾種,充弓箭手。」詔宣撫使範仲淹議,以為便。遂以岢嵐軍北草城川禁地募人拒敵界十里外占耕,得二千餘戶,歲輸租數萬斛,自備弓馬,涅手背
 為弓箭手。既以並州明鎬沮議而止。



 至和二年,韓琦奏訂鎬議非是,曰:「昔潘美患契丹數入寇,遂驅旁邊耕民內徙,茍免一時失備之咎。其後契丹講和,因循不復許人復業,遂名禁地,歲久為戎人侵耕,漸失疆界。今代州、寧化軍有禁地萬頃,請如草城川募弓箭手,可得四千餘戶。」下並州富弼議。弼請如琦奏。詔具為條,視山坡川原均給,人二頃;其租秋一輸,川地畝五升,板原地畝三升,毋折變科徭。仍指揮即山險為屋,以便居止,備征防,無
 得擅役。



 先是,麟、府、豐州亦以閑田募置,人給屋,貸口糧二石,而德順軍靜邊砦壕外弓箭手尤為勁勇。夏人利其地,數來爭占,朝廷為築堡戍守。至治平末,河東七州軍弓箭手總七千五百人,陜西十州軍並砦戶總四萬六千三百人。先是,康定元年,詔麟、府州募歸業人增補義軍,俾耕本戶故地而免其稅租。其制與弓箭手略同,而不給田。



 熙寧二年,兵部上河東七郡舊籍七千五、今籍七千,陜西十郡並砦戶舊籍四萬六千三百,今唯秦
 鳳有砦戶。



 三年,秦鳳路經略使李師中言:「前年築熟羊等堡,募蕃部獻地,置弓箭手。迄今三年,所募非良民,初未嘗團結訓練,竭力田事。今當置屯列堡,為戰守計。置屯之法,百人為屯,授田於旁塞堡,將校領農事,休即教武技。其牛具、農器、旗鼓之屬並官予。置堡之法,諸屯並力,自近及遠築為堡以備寇至,寇退則悉出掩擊。」從之。



 五年,趙離為鄜延路,以其地萬五千九百頃,募漢、蕃弓箭手四千九百人。帝嘉其能省募兵之費,褒賞之。六
 年,離言新募弓箭手頗習武技,請更番代正兵歸京師。詔審度之。十月,詔熙河路以公田募弓箭手,其旁塞民強勇願自占田,出租賦,聯保伍,或義勇願應募,或民戶願受蕃部地者聽。



 七年正月,帶御器械王中正詣熙河路,以土田募弓箭手。所募人毋拘路分遠近,不依常格,差官召募,仍親提舉。三月,王韶言:「河州近城川地招漢弓箭手外,其山坡地招蕃弓箭手,人給地一頃,蕃官兩頃,大蕃官三頃。仍募漢弓箭手等為甲頭,候招及人數,補
 節級人員,與蕃官同管勾。自來出軍,多為漢兵盜殺蕃兵,以為首功。今蕃兵各願於左耳前刺『蕃兵』字。」從之。十月,中書條例司乞五路弓箭手、砦戶,除防拓、巡警及緩急事許差發外,若修城諸役,即申經略安撫、鈐轄司。其有擅差發及科配、和雇者,並科違制之罪。從之。其夔州路義軍、廣南槍手土丁峒丁、湖南弩手、福建鄉丁槍手,依此法。



 八年,詔涇原路七駐泊就糧上下番正兵、弓箭手、蕃兵約七萬餘人分為五將,別置熙河策應將副。十年,知延
 州呂惠卿言:「自熙寧五年,招到弓箭手,只是權行差補,未曾團定指揮。本司見將本路團結將分團成指揮都分,置立將校統轄,即於臨時易為勾集。」從之。



 元豐二年,計議措置邊防所言,以涇原路正兵、漢蕃弓箭手為十一將,分駐諸州。從之。



 三年,詔:「凡弓箭手兵騎各以五十人為隊,置引戰、旗頭、左右傔旗,及以本屬酋首將校為擁隊,並如正軍法。蕃捉生、蕃敢勇、山河戶亦如之。凡募弓箭手、蕃捉生、強人、山河戶,不以等樣,第募有保任、年十
 七已上、弓射七斗、任負帶者。鄜延路新舊蕃捉生、環慶路強人、諸路漢弓箭手、鄜延路歸明界保毅蕃戶弓箭手,皆涅於手背。」



 四年,涇原路經略司言:「本路弓箭手闕地九千七百頃,渭州隴山一帶川原陂地四千餘頃,可募弓箭手二千餘人,或不願應募,乞收其地入官。」熙河路都大經制司言;「乞依熙河舊例,許涇原、秦鳳路、環慶及熙河路弓箭手投換,仍帶舊戶田土,耕種二年,即收入官,別招弓箭手。」皆從之。



 五年正月,鄜延路經略司乞以
 新收復米脂、吳堡、義合、細浮圖、塞門五砦地置漢蕃弓箭手,及春耕種,其約束補職,並用舊條。從之。二月,詔提舉熙河等路弓箭手、營田、蕃部共為一司,隸涇原路制置司。四月,詔:「蕃弓箭手陣亡,依漢弓箭手給賻。弓箭手出戰,因傷及病羸不能自還者,並依軍例賜其家。」七月,提舉熙河路弓箭手營田蕃部司康識、兼提舉舉營田張大寧言:「乞應新收復地差官分畫經界,選知農事廂軍耕佃,頃一人。其部押人員、節級及雇助人工歲入賞罰,
 並用熙河官莊法。餘並招弓箭手營田,每五十頃為一營,差諳農事官一員乾當。」從之。



 六年,鄜延路經略司言:「弓箭手於近里縣置田兩處,立戶及四丁已上,乞取一丁為保甲,一丁為弓箭手,有二丁至三丁,即且令充弓箭手。」詔保甲願充弓箭手者聽,其見充弓箭手與當丁役,毋得退就保甲,陜西、河東亦如之。



 八年,詔罷秦鳳路置場集教弓箭手,令經略司講求土人習教所宜立法。



 元祐元年,詔罷提舉熙河等路弓箭營田蕃部司。三年,
 兵部言:「涇原路隴山一帶系官地,例為人侵冒,略無色役。非自朝廷置局招置摽撥,無以杜絕奸弊。」從之。其後,殿前司副都指揮使劉昌祚奏根括隴山地凡一萬九百九十頃,招置弓箭手人馬凡五千二百六十一,賜敕書獎諭。四年,詔將隴山一帶弓箭手人馬別置一將管幹,仍以涇原路第十二將為名。五年,詔戶部遣官往熙河蘭岷路代孫路措置弓箭手土田。



 紹聖元年,樞密院言:「熙河蘭岷路經略司奏,本路弓箭手,自展置以來,累
 經戰鬥,內有戰功補三班差使已上之人,欲並遣歸所屬差使,仍以其地令親屬承刺,如無,即別召人承之。」三年正月,詔:「自今漢蕃人互投弓箭手者,官司不得收刺,違者杖一百。」五月,詔在京府界、諸路馬軍槍手並改充弓箭手,兼習蕃槍。四年,詔張詢、巴宜專根括安西、金城膏腴地頃畝,可以招置弓箭手若干人,具團結以聞。



 元符元年二月,樞密院言:「鐘傳奏,近往涇原與章楶講究進築天都山、南牟等處。今相度如展置青南訥心,須置
 一將。乞權於熙、秦兩路輟那。新城內土田並招弓箭手,仍置提舉官二員。熙、秦兩路弓箭手,每指揮以三百人為額,乞作二十指揮招置,不一二年間,須得數千民兵,以充武備。」從之。七月,詔:「陜西、河東路新城砦合招弓箭手投換。其元祐八年四月不得招他路弓箭手指揮勿用。」三年,提舉涇原路弓箭手安師文知涇州,罷提舉弓箭手司。



 崇寧元年九月,樞密院勘會:「陜西五路並河東,自紹聖開斥以來,疆土至廣,遠者數百里,近者不減百
 里,罷兵以來,未曾措置。田多膏腴,雖累降詔置弓箭手,類多貧乏,或致逃走。州縣鎮砦污吏豪民冒占沃壤,利不及於平民,且並緣舊疆,侵占新土。今遣官分往逐路提舉措置,應緣新疆土田,分定腴瘠,招置弓箭手,推行新降條法。舊弓箭手如願出佃新疆,亦仰相度施行。」詔湯景仁河東路,董採秦鳳路,陶節夫環慶路,安師文鄜延路,並提舉弓箭手。



 元符三年罷提舉司,今復置。



 崇寧二年十一月,安師文奏:「據權通判德順軍事盧逢原申,根括打量出
 四將地分管下五砦、新占舊邊壕外地共四萬八千七百三十一頃有奇,乞特賜優賞。」詔安師文特授左朝議大夫,差遣如故;盧逢原特授朝請郎。



 二年九月,熙河路都轉運使鄭僅奉詔相度措置熙河新疆邊防利害,僅奏:「朝廷給田養漢蕃弓箭手,本以藩捍邊面,使顧慮家產,人自為力。今拓境益遠,熙、秦漢蕃弓箭手乃在腹裏,理合移出。然人情重遷,乞且家選一丁,官給口糧,團成耕夫使佃官莊。遇成熟日,除糧種外,半入官,半給耕夫,
 候稍成次第,聽其所便。」從之。



 五年三月,趙挺之言:「湟、鄯之復,歲費朝廷供億一千五百餘萬。鄭僅初建官莊之議,朝廷令會計其歲入,凡五莊之入,乃能支一莊之費。蓋鄯、湟乃西蕃之二小國,湟州謂之邈川,鄯州謂之青唐,與河南本為三國,其地濱河,多沃壤。昔三國分據時,民之供輸於其國厚,而又每族各有酋長以統領之,皆衣食贍足,取於所屬之民。自朝廷收復以來,名為使蕃民各占舊地以居,其實屢更戰鬥,殺戮竄逐,所存無幾。
 今兵將官、帥臣、知州多召閑民以居,貪冒者或受金乃與之地,又私取其羊馬駝畜,然無一毫租賦供官。若以昔輸於三國者百分之一入於縣官,即湟州資費有餘矣。」帝深然之。



 翌日,知樞密院張康國入見,力言不可使新民出租,恐致擾動眾情;且言蕃民既刺手背為兵,安可更出租賦。帝因宣諭:「新民不可搖動,兼已令多招弓箭手矣。」廷之奏:「弓箭手,官給以地而不出租,此中國法也。若蕃兵,則其舊俗既輸納供億之物,出戰又人皆為
 兵,非弓箭手之比。今朝廷所費不貲,經營數年,得此西蕃之地,若無一毫之入,而官吏、戍卒饋餉之費皆出於朝廷,何計之拙也!」帝曰:「已令姚雄經畫。」時累詔令雄括空閑地,召人耕墾出課,故深以挺之所奏為然。挺之又云:「鄯、湟之復,羌人屢叛,溪撦羅撒走降夏國,夏國納之,時時寇邊,兵不解嚴而饋運極艱。和糴入粟,鄯州以每石價至七十貫,湟州五十餘貫。蓋倉場利於客人入中乞取,而官吏利於請給觔斗,中官獲利百倍,人人皆富。
 是以上下相蒙,而為朝廷之害。」



 熙寧三年,熙河運司以歲計不足,乞以官茶博糴,每茶三斤易粟一斛,其利甚博。朝廷謂茶馬司本以博馬,不可以博糴,於茶馬司歲額外,增買川茶兩倍茶,朝廷別出錢二百萬給之,令提刑司封樁。又令茶馬官程之邵兼領轉運使,由是數歲邊用粗足。及挺之再相,熙河漕司屢申以軍糧不足為急,乃令會去年拋降錢數共一千一百萬馱,一馱價直三千至四十千,二百馱所轉不可勝計,今年已降撥銀、
 錢、絹等共九百萬,乃令更支兩倍茶一百萬馱。張康國同進呈,得旨,乃密檢元豐以來茶惟用博馬指揮以進。然康國不知兩倍茶自非博馬之數,而何執中、鄧洵武雜然和之。由是兩倍茶更不支給,而鄯、湟兵費不給矣。



 七年,詔:「邊地廣而耕墾未至,膏腴荒閑,芻粟翔踴,歲糴本不貲。昨累降指揮,令涇原路經略司與提舉弓箭手司措置,召人開墾,以助塞下積粟,為備邊無窮之利。訪聞提舉弓箭手司與經略司執見不同,措置議論,不務
 和協。其提舉涇原路弓箭手錢歸善可罷。」



 大觀三年二月,臣僚言:「自復西寧州,饋給每多,而儲積未廣,買價數增,市物隨踴,地利不闢,兵籍不敷,蓋招置之術失講,勸利之法未興也。乞委帥臣、監司講求,或募或招,何為而可足弓箭手之數,以期於不闕;或拘或誘,何為而使蕃部著業而責以耕耘。田既墾則穀自盈,募既充而兵益振,是收班超之功,盡充國之利也。」詔:「熙、河、洮、岷前後收復,歲月深久,得其地而未得其利,得其民而未得其用。
 地利不闢,兵籍不敷,歲仰朝廷供億,非持久之道。可令詳究本末,條畫來上。」



 政和三年,秦鳳路經略安撫使何常奏:



 自古行師用兵,或騎或步,率因地形。兵法曰:「蕃兵惟勁馬奔沖,漢兵惟強弩掎角。」蓋蕃長於馬,漢長於弩也。今則不然。西賊有山間部落謂之「步跋子」者,上下山坡,出入溪潤,最能逾高超遠,輕足善走。有平夏騎兵謂之「鐵鷂子」者,百里而走,千里而期,最能倏往忽來,若電擊雲飛。每於平原馳騁之處遇敵,則多用鐵鷂子以為
 沖冒奔突之兵;山谷深險之處遇敵,則多用步跋子以為擊刺掩襲之用。此西人步騎之長也。我諸路並塞之民,皆是弓箭手地分,平居以田獵騎射為能,緩急以追逐馳騁相尚。又沿邊土兵,習於山川,慣於馳驟。關東戍卒,多是硬弩手及摽牌手,不惟捍賊勁矢,亦可使賊馬驚潰。此中國步騎之利也。



 至道中,王超、丁罕等討繼遷,是時馬上用弩,遇賊則萬弩齊發,賊不能措手足而遁。又元豐間,劉昌祚等趨靈州,賊眾守隘,官軍不能進。於
 是用牌子為先鋒,賊下馬臨官軍,其勢甚盛,昌祚等乃以牌子踢跳閃爍,振以響環,賊馬驚潰。若遇賊於山林險隘之處,先以牌子賊,次以勁弓強弩與神臂弓射賊先鋒,則矢不虛發,而皆穿心達臆矣。或遇賊於平原廣野之間,則馬上用弩攢射,可以一發而盡殪。兼牌子與馬上用弩,皆已試之效,不可不講。前所謂勁馬奔沖,強弩掎角,其利兩得之,而賊之步跋子與鐵鷂子皆不足破也。又步兵之中,必先擇其魁健材力之卒,皆用斬
 馬刀,別以一將統之,如唐李嗣業用陌刀法。遇鐵鷂子沖突,或掠我陣腳,或踐踏我步人,則用斬馬刀以進,是取勝之一奇也。



 詔樞密院札與諸路經略司。



 四年,詔:「西羌久為邊患,乍叛乍服,譎詐不常。頃在先朝,使者在廷,猶或犯境。今植養積歲,屢饑久困,雖誓表已進,羌夷之性不保其往。修備御於無事之時,戒不虞於萃聚之際,正在今日。可令陜西、河東路帥臣訓練兵伍,除治軍器,繕修樓櫓,收積芻糧,常若寇至。不可謂已進誓表,輒或
 弛怠,墮其奸謀。所有弓箭手、蕃兵,常令優恤,逃亡者可速招補,貧乏者亦令貸借。將佐偏裨,如或軟懦失職,具名以聞,或寇至失事,並行軍法。」



 五年二月,詔:「陜西、河東逐路,自紹聖開拓邊疆以來,及西寧、湟、廓、洮州、積石等處新邊,各有包占良田,並合招置弓箭手,以為邊防籬落。至今累年,曠土尚多,應募人數未廣。蓋緣自罷專置提舉官隸屬經略司,事權不專,頗失措置。根括打量、催督開墾、理斷交侵等職事,盡在極邊,帥臣無由親到。即
 今夏人通貢,邊鄙安靜。若不乘此委官往來督責,多方招刺弓箭手墾闢閑田,補助邊計,以寬飛挽之勞,竊慮因循浸久,曠土愈多,銷耗民兵人額,有害邊防大計。兼提舉文臣玩習翰墨,多務安養,罕能沖冒寒暑。可令陜西、河東逐路,並復置提舉弓箭手司,仍各選差武臣一員充,理任、請給、恩數等並依提舉保甲條例施行。每路各置乾當公事使臣二員。仍每歲令樞密院取索逐路招到弓箭手並開墾過地土,比較優劣殿最,取旨黜陟。
 合措置事節,所差官條畫以聞。」



 八月,樞密院言:「欲將近里弓箭手地,但有爭訟侵冒之處,並行打量,庶幾杜絕侵冒之弊。」從之。是月,提舉河東路弓箭手司奏:「本司體訪得沿邊州軍逐處招置弓箭手,多將人戶舊用工開耕之地指射鏟奪,其舊佃人遂至失業。且所出租,僅比佃戶五分之一,於公私俱不便。今欲將系官莊屯田已有人租佃及五年者,並不在招置弓箭手請射之限。其河東路察訪司初不以邊防民兵為重,姑息佃戶,致有
 此弊。欲乞應熙寧八年以前人戶租佃官田,並先取問佃人,如願投刺弓箭手,每出一丁,許依條給見佃田二頃五十畝充人馬地,若不願充弓箭手,及出丁外尚有請占不盡地土,即拘收入官。」從之。



 十一月,邊防司奏:「據提舉熙河蘭湟路弓箭手何灌申:漢人買田常多,比緣打量,其人亦不自安,首陳已及一千餘頃。若招弓箭手,即可得五百人;若納租稅,每畝三斗五升、草二束,一歲間亦可得米三萬五千石、草二十萬束。今相度欲將漢
 人置買到蕃部土田願為弓箭手者,兩頃已上刺一名,四頃已上刺兩名。如願者,依條立定租稅輸納。其巧為影占者,重為禁止。」從之。



 七年三月,詔:「熙、河、鄯、湟自開拓以來,疆土雖廣而地利悉歸屬羌,官兵吏祿仰給縣官,不可為後計。仰本路帥臣相度,以錢糧茶彩或以羌人所嗜之物,與之貿易田土。田土既多,即招置弓箭手,入耕出戰,以固邊圉。」



 宣和六年七月,詔:「已降處分,陜西昨因地震摧塌屋宇,因而死傷弓箭手,內合承襲人,速具
 保明聞奏。」



 靖康元年二月,臣僚言:「陜西恃弓箭手為國藩籬,舊隸帥府,比年始置提舉弓箭手官,務取數多,自以為功。自是選練不精,遂使法制浸壞。欲乞詳酌,罷提舉官,以弓箭手復隸帥司,務求以振邊聲。」詔從之,河東路依此。四月,樞密院奏:「陜西、河東逐路漢弓箭手自來並給肥饒田,近年以來,多將舊人已給田分擘,招刺新人。蓋緣提舉官貪賞欺蔽,務要數多,妄行招刺,無以激勸。朝廷近已罷提舉官,今復隸帥司所轄,況當今邊事
 全藉民兵,若不早計,深慮誤事。」詔令陜西五路制置使錢蓋及陜西、河東逐路帥臣相度措置,將已分擘弓箭手田土,依舊改正撥還,所有新招到人別行給地,務要均濟。仍仰帥臣嚴切奉行。是月,徐處仁又奏,詔並送詳議司。



 熙寧五年,涇原路經略司蔡挺言:「涇原勇敢三百四十四人,久不揀練,徒有虛名。臣委二將領季一點閱,校其騎射能否升除,補有功者以為隊長,募極塞博軍子嘗歷戰陣者補其闕。益募熟戶蕃部以為蕃勇敢,凡
 一千三百八十人,騎一千一百九十四匹,挽弓一石,馳逐擊刺如法。其有功者受勇敢下等奉,餘遇調發,則人給奉三百,益以芻糧。」詔諸路如挺言行之。



 六年,樞密院言:「勇敢效用皆以材武應募從軍,廩食既優,戰馬戎械之具皆出公上,平時又得以家居,以勞效賞者凡四補而至借職,校弓箭手減十資,淹速相遠,甚非朝廷第功均賞之意。請自今河東、鄜延、秦鳳、環慶、熙河路各以三百,涇原路以五百為額。第一等步射弓一石一斗,馬射
 九斗,奉錢千;第二等以下遞減一斗,奉七百至五百。季首閱試於經略司,射親及野戰中者有賞,全不中者削其奉,次季又不中者罷之。戰有功者以八等定賞:一、給公據,二、以為隊長,三、守闕軍將,四、軍將,五、殿侍,六、三班借差,七、差使,八、借職。其弓箭手有功,亦以八等定賞:一、押官,承局;二、將,虞候,十將;三、副兵馬使,軍使;四、副指揮使;五、都虞候;六、都指揮使;七、三班差使;八、借職。即以闕排連者次遷。



 元豐三年,詔涇原路募勇敢如鄜延路,以
 百人為額。自是以後,蕃部益眾,而弓箭手多蕃兵矣。



 弓箭社河北舊有之。熙寧三年十二月,知定州滕甫言:「河北州縣近山谷處,民間各有弓箭社及獵射人,習慣便利,與夷人無異。欲乞下本道逐州縣,並令募諸色公人及城郭鄉村百姓有武勇願習弓箭者,自為之社。每歲之春,長吏就閱試之。北人勁悍,緩急可用。」從之。



 元祐八年十一月,知定州蘇軾言:



 北邊久和,河朔無事。沿邊諸郡,軍政少馳,將驕卒惰,緩急恐不可用;武藝軍裝,皆
 不逮陜西、河東遠甚。雖據即目邊防事勢,三五年間必無警急,然居安慮危,有國之常,備事不素講,難以應變。臣觀祖宗以來,沿邊要害,屯聚重兵,止以壯國威而消敵謀,蓋所謂先聲後實,形格勢禁之道耳。若進取深入,交鋒兩陣,猶當雜用禁旅。至於平日保境,備御小寇,即須專用極邊土人。此古今不易之論也。


晁錯與漢文帝畫備邊策,不過二事:其一曰徙遠方以實空虛,其二曰制邊縣以備敵國。寶元、慶歷中,趙元昊反,屯兵四十餘
 萬,招刺宣毅、保捷二十五萬人,皆不得其用,卒無成功。範仲淹、劉
 \gezhu{
  山成}
 、種世衡等專務整緝蕃漢熟戶、弓箭手,所以封殖其家、砥礪其人者非一道。藩籬既成,賊來無所得,故元昊復臣。今河朔西路被邊州軍,自澶淵講和以來,百姓自相團結為弓箭社,不論家業高下,戶出一人。又自相推擇家資武藝眾所服者為社頭、社副、錄事,謂之頭目。帶弓而鋤,佩劍而樵,出入山板,飲食長技與敵國同。私立賞罰,嚴於官府,分番巡邏,鋪屋相望,若透漏
 北賊及本土強盜不獲,其當番人皆有重罰。遇其警急,擊鼓,頃刻可致千人。器甲鞍馬,常若寇至。蓋親戚墳墓所在,人自為戰,敵深畏之。先朝名臣帥定州者韓琦、龐籍,皆加意拊循其人,以為爪牙耳目之用,而籍又增損其約束賞罰。



 熙寧六年,行保甲法,強壯、弓箭社並行廢罷。熙寧七年,應兩地供輸人戶,除元有弓箭社、強壯並義勇之類並依舊存留外,更不編排保甲。看詳上件兩次聖旨,除兩地供輸村分方許依舊置弓箭社,其餘並
 合廢罷。雖有上件指揮,公私相承,元不廢罷,只是令弓箭社兩丁以上人戶兼充保甲,以至逐捕本界及他盜賊,並皆驅使弓箭社人戶用命捉殺。見今州縣,全藉此等寅夜防拓,灼見弓箭社實為邊防要用,其勢決不可廢。但以兼充保甲之故,召集追呼,勞費失業,今雖名目俱存,責其實用,不逮往日。



 臣竊謂陜西、河東弓箭手,官給良田,以備甲馬。今河朔沿邊弓箭社,皆是人戶祖業田產,官無絲毫之損,而捐軀捍邊,器甲鞍馬與陜西、河
 東無異,苦樂相遠,未盡其用。近日霸州文安縣及真定府北砦,皆有北賊驚劫人戶,捕盜官吏拱手相視,無如之何,以驗禁軍、弓手皆不得力。向使州縣逐處皆有弓箭社,人戶致命盡力,則北賊豈敢輕犯邊砦,如入無人之境?臣已戒飭本路將吏,申嚴賞罰,加意拊循其人,輒復拾用龐籍舊奏約束,稍加增損,別立條目。欲乞朝廷立法,少賜優異,明設賞罰,以示懲勸。今已密切取會到本路極邊定、保兩州、安肅、廣信、順安三軍邊面七縣一
 砦內管自來團結弓箭社五百八十八社,六百五十一火,共計三萬一千四百一十一人。若朝廷以為可行,立法之後,更敕將吏常加拊循,使三萬餘人分番晝夜巡邏,盜邊小寇來即擒獲,不至狃心犬以生戎心。而事皆循舊,無所改作,敵不疑畏,無由生事,有利無害,較然可見。



 奏凡兩上,皆不報。



 政和六年,詔:「河北路有弓箭社縣分,已令解發異等。其逐路縣令佐,俟歲終教閱異等,帥司具優劣之最,各取旨賞罰,以為勸沮。仍具為令。」又高陽
 關路安撫司言:「大觀三年弓箭社人依《保甲法》、《政和保甲格》較最優劣,縣令各減展磨勘年有差。」詔依《保甲格》賞罰施行。



 宣和七年二月,臣僚言:



 往年西路提刑梁揚祖奏請勸誘民戶充弓箭社,繼下東路令仿西路例招誘。原立法之意,不過使鄉民自願入社者閱習武備,為御賊之具爾。奈何邀功生事之人,唯以入社之民眾多為功,厚誣朝廷而斂怨於民,督責州縣急於星火,取五等之籍甲乙而次之,家至戶到,追胥迫脅。悉驅之入社,
 更無免者。法始行於西路,西路既已冒受厚賞,於是東路憲司前後論列,誕謾滋甚。近者東路之奏,數至二十四萬一千七百人,武藝優長者一十一萬六千,且云比之西路僅多一倍。陛下灼知其不然,雖命帥臣與廉訪使者核實,彼安肯以實聞乎?今東路憲司官屬與登、淄兩州當職官,坐增秩者幾二十人,而縣令、佐不及焉。不知出入阡陌間勸誘者誰歟?此其誕謾可知矣。審如所奏,山東之寇,何累月淹時未見殄減哉?則其所奏二十
 四萬與十一萬,殆虛有名,不足以捍賊明矣!大抵因緣追擾,民不堪其勞,則老弱轉徙道路,強壯起為盜賊,此亦致寇之一端也。



 近者仰煩陛下遣將出師,授以方略,又命近臣持詔撫諭,至於發內庫之藏,轉淮甸之粟以振給之,寬免其稅租,蕩宥其罪戾,丁寧纖悉,罔不曲盡。方將歸伏田畝,以為遷善遠罪之民,詎可以其所甚病擾之邪?且私有兵器,在律之禁甚嚴。三路保伍之法,雖於農隙以講武事,然猶事畢則兵器藏於官府。今弓箭
 社一切兵器,民皆自藏於家,不幾於借寇哉?望陛下斷自聖心,罷京東弓箭社之名,所藏兵器悉送之官,使民得免非時追呼迫脅之擾,以安其生。應兩路緣弓箭社推恩者並追奪改正,首議之人重賜黜責,後為奏請誕謾,亦乞特賜施行,庶幾群下悚懼,不敢妄進曲說,以肆其奸,實今日之先務也。



 詔並依奏,梁揚祖落職,兵器並拘入官,弓箭社人依已降指揮放散。


◎兵五
 \gezhu{
  鄉兵二}



 ○河北河東陝西義勇陝西護塞川峽土丁荊湖義軍土丁弩手夔施黔思等處義軍土丁廣南西路土丁廣南東路槍手邕欽溪洞壯丁福建路槍仗手江南西路槍仗手蕃兵



 河北、河東、陝西義勇慶曆二年,選河北、河東強壯並抄民丁涅手背為之。戶三等以上置弩一,當稅錢二千,三等以下官給。各營於其州,歲分兩番訓練,上番給奉廩,犯罪斷比廂軍,下番比強壯。



 治平元年,詔陝西除商、虢二州,餘悉籍義勇。凡主戶三丁選一,六丁選二,九丁選三,年二十至三十材勇者充,止涅手背。以五百人為指揮,置指揮使、副二人,正都頭三人,十將、虞候、承局、押官各五人,歲以十月番上,閱教一月而罷。又詔秦州成紀
 等六縣,有稅戶弓箭手、砦戶及四路正充保毅者,家六丁刺一,九丁刺二;有買保毅田承名額者,三丁刺一,六丁刺二,九丁刺三,悉以為義勇。是歲,詔秦、隴、儀、渭、涇、原、邠、寧、環、慶、鄜、延十二州義勇,遇召集防守,日給米二升,月給醬菜錢三百。蓋慶曆初,河北路總十八萬九千三十一人,河東路總七萬七千七十九人,陝西路治平初總十五萬六千八百七十三人。



 熙寧初,樞密使呂公弼請以河北義勇每指揮揀少壯藝精者百人為上等,手
 背添刺「上等」字,旌別教閱,及數外藝優者亦籍之,俟有闕則補。從之。十二月,詔河北義勇,縣以歲閱;當閱於州者,宜分番,歲以一番;災傷當罷者聽旨。其以指揮分番者,大名府五十三為四番,真定、瀛、洺、邢、滄、定、冀、恩、趙、深、磁、相、博自三十九以及十二並為三番,德、祁、澶、棣、霸、濱、永靜、永寧、懷、衛、乾寧、莫、保、通利自十一以及四並為二番。九指揮已上者再分本番為三,教始十月,止十二月。六指揮已上者再分本番為二,教始十月,止十一月,終
 滿一月罷遣。



 帝嚐問陳升之曰:「侯叔獻言義勇上番何如?」王安石曰:「此事似可為,但少須年歲間議之。」升之曰:「今募兵未已,且養上番義勇,則調度尤不易。」安石曰:「言募兵之害雖多,及用則患少,以民與兵為兩途故也。」十二月,帝言:「義勇可使分為四番出戍。」呂公弼曰:「須先省得募兵,乃可議此。」安石曰:「計每歲募兵死亡之數,乃以義勇補之可也。」陳升之欲令義勇以漸戍近州,安石曰:「陛下若欲變數百年募兵之弊,則宜果斷,詳立法製。不
 然,無補也。」帝以為然,曰:「須豫立定條法,不要宣布,以漸推行可也。」 兩府議上番,或以為一月,或以為一季,且令近戍,文彥博等又言難使遠戍,安石辯之甚力。



 是月,兵部上陝西、河北、河東義勇數:陝西路二十六郡舊籍十五萬三千四百,益以環、慶、延州保毅、弓箭手三千八百,總十五萬六千八百,為指揮三百二十一;河北三十三郡舊籍十八萬九千二百,今籍十八萬六千四百,為指揮四百三十;而河東二十郡,自慶曆後總七萬七千,為
 指揮一百五十九。凡三路義勇之兵,總四十二萬三千五百人。



 三年七月,王安石進呈蔡挺乞以義勇為五番教閱事,帝患密院不肯措置,安石曰:「陛下誠欲行,則孰能禦?此在陛下也。」涇、渭、儀、原四州義勇萬五千人,舊止戍守,經略使蔡挺始令遇上番依諸軍結隊,分隸諸將。選藝精者遷補,給官馬,月廩、時帛、郊賞與正兵同,遂與正兵相參戰守。時土兵有闕,召募三千人。挺奏以義勇點刺累年,雖訓肄以時,而未施於征防,意可以案府兵
 遺法,俾之番戍,以補土兵闕。詔復問以措置遠近番之法。挺即條上,以四州義勇分五番,番三千人,防秋以八月十五日上,十月罷;防春以正月十五日上,三月罷,周而復始。詔從之,行之諸路。九月,秦鳳經略安撫司言:「保毅人數不曾揀充義勇,而其子孫轉易田土,分煙析姓,少有正身。乞令保毅軍已於丁數內揀刺充義勇者,與免承認保毅。」從之。十月,韓絳乞差著作佐郎呂大忠等赴宣撫司,以備提舉義勇,從之。是月,韓絳言:「今將義勇
 分為七路,延、丹、坊為一路,邠、寧、環、慶為一路,涇、原、儀、渭為一路,秦、隴為一路,陝、解、同、河中府為一路,階、成、鳳州、鳳翔府為一路,乾、耀、華、永興軍為一路。逐年將一州之數分為四番,緣邊四路十四州,每年秋冬合用一番屯戍;近裏三路十二州軍,即令依此立定番次,未得逐年差發,遇本處闕少正兵,即得勾抽或那往次邊守戍。」從之。十一月,判延州郭逵言:「陝西起發義勇赴緣邊戰守,今後並令自齎一月糗糧,折本戶稅賦。若不能自備,則
 就所發州軍預請口食一月。」從之。



 十二月,司馬光上疏曰:



 臣以不才,兼領長安一路十州兵民大柄。到官以來,伏見朝廷及宣撫等司指揮,分義勇作四番,欲令以次於緣邊戍守,選諸軍驍銳及募閭裏惡少以為奇兵,造幹糧、炒飯、布囊、力車以備饋運,悉取歲賜趙秉常之物散給緣邊諸路,又竭內地府庫甲兵財物以助之。且以永興一路言之,所發人馬,甲八千副,錢九萬貫,銀二萬三千兩,銀碗六千枚,其餘細瑣之物,不可勝數。動皆迫
 以軍期,上下相驅,急於星火。官吏狼狽,下民驚疑,皆云國家將以來春大舉六師,長驅深入,以討秉常之罪。



 臣以疏賤,不得預聞廟堂之議,未知茲事為虛為實。昨者親承德音,以為方今邊計,惟宜謹嚴守備。其入寇,則堅壁清野,使之來無所得,兵疲食盡,可以坐收其弊。臣退而思念,聖謀高遠,深得王者懷柔遠人之道,實天下之福。及到關中,乃見凡百處置,皆為出征調度。臣不知有司在外,不諭聖意,以致有此張皇,將陛下默運神算不
 令愚賤之臣得聞其實也?臣不勝惶惑,竊為陛下危之。況關中饑饉,十室九空,為賊盜者紛紛已多。縣官倉庫之積,所餘無幾,乃欲輕動大眾,橫挑猛獸,此臣之所大懼也。



 伏望陛下深鑒安危之機,消之於未萌,杜之於未形。速下明詔撫諭關中之民以朝廷不為出征之計,其義勇更不分番於緣邊戍守,亦不選募奇兵。凡諸調發為饋運之具者悉令停罷,愛惜內地倉庫之儲,以備春深賙救饑窮之人。如此,豈惟生民之幸,亦社稷之福也。
 惟陛下裁察。



 再言之甚力,於是永興一路獨得免。



 四年,詔罷陝西路義勇差役。又詔罷陝西諸路提舉義勇官,委本屬州縣依舊分番教閱。



 五年七月,命崇文院校書王安禮專一編修三路義勇條貫。是月,帝問王安石義勇事如何,安石曰:「宜先了河東一路。河東舊制,每年教一月,今令上番巡檢下半月或十日,人情無不悅。又以東兵萬人所費錢糧,且取一半或三分之二,依保甲養恤其人,即人情無不忻願者。」閏七月,執政同進呈河東
 保甲事,樞密院但欲為義勇、強壯,不別名保甲。王安石曰:「此非王安禮初議也。」帝曰:「今以三丁為義勇,兩丁為強壯,三丁遠戍,兩丁本州縣巡檢上番,此即王安禮所奏,但易保丁為強壯。人習強壯久,恐別名或致不安也。」安石曰:「義勇非單丁不替,強壯則皆第五等戶為之。又自置弓弩及箭寄官庫,須上教乃給。今以府界保甲法推之河東,蓋寬利之,非苦之也。」帝曰:「河東義勇、強壯,已成次第。今欲遣官修義勇強壯法,又別令人團集保甲
 如何?」安石曰:「義勇要見丁數,即須隱括,因團集保甲,即一動而兩業就。今既差官隱括義勇,又別差官團集保甲,即一事分為兩事,恐民不能無擾。」帝卒從安石議。彥博請令安石就中書一面施行此事。安石曰:「本為保甲,故中書預議。若止欲作義勇、強壯,即合令樞密院取旨施行。」帝曰:「此大事,須共議乃可。」是月,秦鳳路經略呂公弼乞從本司選差官,自十月初,擇諸州上番義勇材武者以為「上義勇」,免齎送芻糧之役。募養馬者為「有馬上
 義勇」,並免其本戶支移。從之。



 六年九月,詔義勇人員、節級名闕,須因教閱排連遷補。十月,熙河路經略司言:乞許人投換義勇,以地給之,起立稅額。詔以官地招弓箭手,仍許近裏百姓壯勇者占射,依內地起稅,排保甲;即義勇願投充及民戶願受蕃部地者聽之。其頃畝令經略司以肥瘠定數。十一月,詔永興軍、河中府、陝、解、同、華、鄜、延、丹、坊、邠、寧、環、慶、耀十五州軍各依元刺義勇外,商、虢州、保安軍並止團成保甲。七年,詔義勇正身不許應募
 充刺,已應募者召人對替。



 八年四月,詔韓琦等,曰:「河朔義勇民兵,置之歲久,耳目已熟,將校甚整,教習亦良。然團結保甲,一道紛然。義勇舊人十去其七,或撥入保甲,或放而歸農,得增數之虛名,破可用之成法,此又徒起契丹之疑也。」七月,詔應義勇家人投軍後,本戶餘丁數少,合免義勇,並許投軍。十月,詔:「五路義勇每年赴州教,保甲赴縣教,並自十月至次年正月終。義勇不及十指揮、保甲不及十都者,自十二月起教,各據人數分定番
 次,教閱一月,不許拆破指揮、都保。其人數少處,隻作一番、兩番,不須滿所教月分。其年已上番者,止教半月。」十二月,詔五路義勇並與保丁輪充及檢察盜賊,有違犯,依保丁法。



 九年正月,詔義勇、保甲逐年遇閱日比試所習武藝,五路每州以二十分為率取一,分為五等,第一等解發。四月,詔:「河北西路義勇、保甲分三十六番,隨便近村分,於巡檢、縣尉下上番,半月一替。歲於農閑月,並下番人並令所轄巡檢、縣尉擇寬廣處聚教五日。」是月,
 兵部言:「舊條,義勇、保甲所習事藝以十分為率,弓不得過二分,槍刀共不得過二分,餘並習弓弩。」詔槍手依舊專習外,刀牌手令兼習弓弩,仍頒樣下五路施行。九月,詔永興、秦鳳等路義勇,以主戶三丁以上充,不拘戶等。是年,諸路所管義勇:河北東路三萬六千二百一十八人,河北西路四萬五千七百六十六人,永興軍路八萬七千九百七十八人,秦鳳路三萬九千九百八十人,河東路三千五百九十五人,總二十四萬七千五百三十
 七人。


元豐二年,中書、樞密院請河北陝西義勇、保甲皆如諸軍誦教閱法。從之。三年,詔五路轉運、提舉官巡曆所至,按閱見教義勇、保甲,不如法者,牒提點刑獄司施行。四年,蒲宗孟言,乞開封府、五路義勇並改為保甲。自此以次行於諸路矣。
 \gezhu{
  此後義勇改為義勇保甲,載《保甲篇》。}



 陝西護塞慶曆元年,募土人熟山川道路蕃情、善騎射者涅臂充。二百人為指揮,自備戎械,就鄉閭習武技,季一集州閱教。無事放營農,月給鹽茗。有警召集防守,即
 廩給之,無出本路。



 川峽土丁熙寧七年,經製瀘州夷事熊本募土丁五千人,入夷界捕戮水路大小四十六村,蕩平其地二百四十里,募民墾耕,聯其夷屬以為保甲。元祐二年,瀘南沿邊安撫使司言:「請應瀘人因邊事補授班行,自備土丁子弟在本家地分防拓之人,更無廩給酬賞。若遇賊,臨時取旨。其敢邀功生事,重置於法。」從之。



 政和六年,瀘南安撫使孫羲叟奏:「邊民冒法買夷人田,依法盡拘入官,
 招置土丁子弟。見招到二千四百餘人,欲令番上。」從之。



 宣和四年,詔:「茂州、石泉軍舊管土丁子弟,番上守把,不諳射藝。其選施、黔兵善射者各五十人,分任教習,候精熟日遣回。」



 荊湖路義軍土丁、弩手不見創置之始,北路辰、澧二州,南路全、邵、道、永四州皆置。蓋溪洞諸蠻,保據岩險,叛服不常,其控製須土人,故置是軍。皆選自戶籍,蠲免徭賦,番戍砦柵。大率安其風土,則罕嬰瘴毒。知其區落,則可
 製狡獪。其校長則有都指揮使、副都指揮使、指揮使、副指揮使、都頭、副都頭、軍頭、頭首、采斫招安頭首、十將、節級,皆敘功遷補,使相綜領。施之西南,實代王師,有禦侮之備,而無饋餉之勞。其後,荊南、歸、陝、鼎、郴、衡、桂陽亦置。



 慶曆二年,北路總一萬九千四百人,南路總五千一百五十人。番戍諸砦,或以歲,或以季,或以月。上番人給口糧,有功遷補。自都副指揮使歲給綿袍、月給食錢,指揮使給食錢,副指揮使給紫大綾綿袍,都頭已上率有廩給。



 熙
 寧元年,籍荊湖南、北路義軍凡一萬五千人,軍政如舊制。六年,諸路行保甲,司農寺請令全、邵二州土丁、弩手、弩團與本村土人共為保甲,以正、副指揮使兼充都副保正,以都頭、將虞候、頭首、都甲頭兼充保長,以左右節級、甲頭兼充小保長。番上則本鋪土丁、弩手、弩團等同為一保,其隔山嶺不及五大保者亦各置都保正一人。



 元祐七年,選差邵州邵陽、武崗、新化等縣中等以下戶充土丁、弩手,與免科役,七年一替。排補將級,不拘替放
 年,分作兩番邊砦防拓,不得募人。凡上番,依禁軍例教閱武藝及專習木弩。如有私役,並論如《私役禁軍敕》。



 紹聖二年,樞密院言:「荊湖南路安撫、轉運、提刑、常平司奏請,邵州管下緣邊堡砦置弩手一千四百人,乞依元豐六年詔,於五等戶輪差,並半年一替。其上番人如有故,許家人少壯有武藝者代充。」從之。



 崇寧二年,荊湖南路安撫、鈐轄李閎言:「收復綏寧縣上堡裏、臨口砦,合用防拓弩手千人,乞於邵州邵陽、武岡兩縣中等以下戶選
 差,半年一替;遇上番,月支錢米;排補階級,自正副使而下至左右甲頭,依舊為七階;分兩番部轄,令邵州給帖。」從之。



 政和七年,以辰、沅、澧等州更戍土丁與營田土丁名稱重壘,將兵馬都鈐轄司招填土丁改為鼎、澧路營田刀弩手。



 重和元年,辰州招到刀弩手二千一百人,其官吏各轉官,減磨勘年有差。宣和四年,靖州通道縣有邊警,詔添置刀弩手二千人。



 夔州路義軍土丁、壯丁州縣籍稅戶充,或自溪洞歸投。
 分隸邊砦,習山川道路,遇蠻入寇,遣使襲討,官軍但據險策應之。其校長之名,隨州縣補置,所在不一。職級已上,冬賜綿袍,月給食鹽、米麥、鐵錢;其次紫綾綿袍,月給鹽米;其次月給米鹽而已,有功者以次遷。



 施、黔、思三州義軍土丁,總隸都巡檢司。施州諸砦有義軍指揮使、把截將、砦將,並土丁總一千二百八十一人,壯丁六百六十九人。又有西路巡防殿侍兼義軍都指揮使、指揮使、都頭、十將、押番、砦將。黔州諸砦有義軍正副指揮使、兵馬使、都
 頭、砦將、把截將,並壯丁總千六百二十五人。思州、洪杜、彭水縣有義軍指揮使、巡檢將、砦將、科理、旁頭、把截、部轄將,並壯丁總千四百二十二人。



 渝州懷化軍。溱州江津巴縣巡遏將,皆州縣調補。其戶下率有子弟、客丁,遇有寇警,一切責辦主戶。巡遏、把截將歲支料鹽,襖子須三年其地內無寇警乃給,有勞者增之。州縣籍土丁子弟並器械之數,使分地戍守。



 嘉祐中,補涪州賓化縣夷人為義軍正都頭、副都頭、把截將、
 十將、小節級,月給鹽,有功以次遷,及三年無夷賊警擾,即給正副都頭紫小綾綿旋襴一。涪陵、武龍二縣巡遏將,砦一人,以物力戶充,免其役。其義軍土丁,歲以籍上樞密院。



 廣南西路土丁嘉祐七年,籍稅戶應常役外五丁點一為之。凡得三萬九千八百人。分隊伍行陣,習槍、鏢排,冬初集州按閱。後遞歲州縣迭教,察視兵械。以防收刈,改用十一月教,一月罷。



 熙寧七年,知桂州劉彝言:「舊制,宜、
 融、桂、邕、欽五郡土丁,成丁已上者皆籍之。既接蠻徼,自懼寇掠,守禦應援,不待驅策。而近製主戶自第四等以上,三取一以為土丁。而旁塞多非四等以上,若三丁籍一,則減舊丁十之七。餘三分以為保丁,保丁多處內地,又俟其益習武事,則當蠲土丁之籍。恐邊備有闕,請如舊制便。」奏可。



 元豐六年,廣西經略使熊本言:「宜州土丁七千餘人,緩急可用。欲令所屬編排,分作都分,除防盜外,緣邊有警,聽會合掩捕。」從之。



 元符二年,廣西察訪司
 言:「桂、宜、融等用土丁緣邊防拓,差及單丁,乞差兩丁以上之家。」從之。



 廣南東路槍手嘉祐六年,廣、惠、梅、潮、循五州以戶籍置,三等已上免身役,四等以下免戶役,歲以十月一日集縣閱教。治平元年,詔所在遣官按閱,一月罷,有闕即招補,不足,選本鄉有武技者充。



 熙寧元年,詔廣州槍手十之三教弓弩手。是歲,會六郡槍手,為指揮四十一,總一萬四千七百有奇。三年,知廣州王靖言:「東路槍手,自至
 和初立為土丁之額,農隙肄業一月,乃古者寓兵於農之策也。然訓練勸獎之製未備,請比三路義勇軍政教法條上約束。」四年,知封州鄧中立請以本路未置槍手州縣,如廣、惠等五郡例置。奏可。六年,廣東駐泊楊從先言:「本路槍手萬四千,今為保甲,兩丁取一,得丁二十五萬,三丁取一,得丁十三萬。以少計之,猶十倍於槍手。願委路分都監二員,分提舉教閱。」詔司農寺定法以聞。其後,戶四等以上,有三丁者以一為之,每百人為一
 都,五都為一指揮。自十一月至二月,月輪一番閱習,凡三日一試,擇其技優者先遣之。七年,詔廣南東西路舊槍手、土丁戶依河北、陝西義勇法,三丁選一,餘州無槍手、土丁者勿置。九年,兵部言:「廣、惠、循、潮、南恩五郡槍手,請籍主戶第四等以上壯丁,毋過舊額一萬四千,餘以為保甲。」奏可。



 元豐二年,詔:廣、惠、潮、封、康、端、南恩七州皆並邊,外接蠻徼,宜依西路保甲教習武藝。時又詔虔州槍仗手以千五百,撫州、建昌軍鄉丁、關軍、槍仗手各以千七
 百為額。監司以農隙按閱武藝,如廣東製。



 邕、欽溪洞壯丁治平二年,廣南西路安撫司集左、右兩江四十五溪洞知州、洞將,各占鄰迭為救應,仍籍壯丁,補校長,給以旗號。峒以三十人為一甲,置節級,五甲置都頭,十甲置指揮使,五十甲置都指揮使,總四萬四千五百人,以為定額。各置戎械,遇有寇警召集之,二年一閱,察視戎械。有老病並物故名闕,選少壯者填,三歲一上。



 熙寧中,王安石言:「募兵未可全罷,民兵則可漸復,至
 於二廣,尤不可緩。今中國募禁軍往戍南方多死,害於仁政。陛下誠移軍職所得官十二三,鼓舞百姓豪傑,使趨為兵,則事甚易成。」於是,蘇緘請訓練二廣洞丁。舊制,一歲教兩月。安石曰:「訓練之法,當什伍其人,拔其材武之士以為什百之長。自首領以下,各以祿利勸獎,使自勤於閱習,即事藝可成,部分可立,緩急可用。」六年,廣南西路經略沈起言:「邕州五十一郡峒丁,凡四萬五千二百。請行保甲,給戎械,教陣隊。藝出眾者,依府界推恩補
 授。」奏可。



 九年,趙禼征交阯,入辭,帝諭以「用峒丁之法,當先誘以實利,然後可以使人。甘言虛辭,豈能責其效命?比鄜延集教蕃兵,賴卿有以製之,使輕罪可決,重罪可誅。違西夏則其禍遠,違帥臣則其禍速,合於兵法『畏我不畏敵』之義,故能責其效命。王師之南,卿宜選募勁兵數千,擇梟將領之,以脅諸峒,諭以大兵將至,從我者有賞,其不從者按族誅之。兵威既振,先脅右江,右江既附,復脅左江,兩江附則諸蠻無不附者。然後以攻交人劉
 紀巢穴,甚非難也。郭逵性吝嗇,卿宜諭以朝廷兵費無所惜,逵復事崖岸,不通下情,將佐莫敢言者,卿至彼,以朕語詔之。」



 十年,樞密院請:「邕、欽峒丁委經略司提舉,同巡檢總蒞訓練之事,一委分接。歲終上藝優者,與其酋首第受賞。五人為保,五保為隊。第為三等:軍功武藝出眾為上,蠲其徭役;人材矯捷為中,蠲其科配;餘為下。邊盜發則酋長相報,率族眾以捍寇。」十二月,詔邕、欽丁壯自備戎械,貧者假以官錢,金鼓旗幟官給,間歲大閱,畢
 則斂藏之。



 元豐元年,經略司請集兩江峒丁為指揮,權補將校。奏可。二年,廣西經略司言:「團結邕、欽峒丁為指揮一百七十五,籍武藝上等一萬三千六百七人。」詔下諸臣獻議措置峒丁事,付曾布參酌損益,創為規畫,務令詳盡,便於施行。布乃請令鎮砦監押、砦主同管轄兵甲使臣與巡檢等,分定州峒總製,立賞罰懲勸。增置都巡檢使兩員,分提舉。及增首領丁壯,歲閱之,以武藝絕倫者聞,量材補授。詔增都巡檢使二員,餘下熊本擇其
 可者施行之。



 五年,詔:「廣南保甲如戎、瀘故事,自置裹頭無刃槍、竹標排、木弓刀、蒿矢等習武技,遇捕盜則官給器械。



 六年,詔樞密承旨司講議廣西峒丁如開封府界保甲集教、團教法。是年,提點廣西路刑獄彭次雲言:「邕苦瘴癘,請量留兵更戍,餘用峒丁,以季月番上,給禁軍錢糧。」詔許彥先度之,彥先等言:「若盡以代正兵,恐妨農。請計戍兵三之一代以峒丁,季輪二千赴邕州肄習武事。」從之。



 大觀二年,詔:「熙寧團集左、右江峒丁十餘萬眾,
 自廣以西賴以防守。今又二十萬眾來歸。已令張莊依左、右江例相度聞奏。尚慮有司不知先務,措置滅裂,今條畫行下其所修法,入熙河蘭湟、秦鳳路敕遵行之。」



 福建路槍仗手元豐元年,轉運使蹇周輔言:「廖恩為盜,以槍仗手捕殺,乃有冒槍仗手之名,乘賊勢驚擾村落,患有甚於廖恩者。」詔犯者特加刺配。周輔請額定槍仗手人數,歲集閱之。下其章兵部。兵部請依保甲法編排,罷舊法,以隸提刑司。居相近者五人為小保,保有長,五
 小保為一大保長,十大保為一都、副保正。具教閱、捕盜賊、食直等令頒焉。總一萬二百人有奇,以歲之農隙,部使者分閱,依弓手法賞之。二年,立法,聽自置兵械寄於官,遇捕盜乃給,數外置者從私有法。



 元祐元年,御史上官均言:「福建路往年因寇盜召募槍手,多至數百人,少不下一二百人。每歲監司親至按試犒賞,比至閱視,其老弱不閑武技者十七八。監司所至,多先期呼集。既至,往往代名充數,冒受支賞,徒有呼集之勞,而無校試之
 實。欲乞重行考核,不必充滿舊數,庶幾得實。」



 靖康元年,臣僚言:「天下步兵之精,無如福建路槍仗手,出入輕捷,馭得其術,一可當十。乞選官前去召募。」從之。



 江南西路槍仗手熙寧七年,詔籍虔、汀、漳三州鄉丁、槍手等,以製置盜賊司言三州壤界嶺外,民喜販鹽且為盜,非土人不能製故也。



 元豐二年,詔虔州槍仗手千五百三十六人,撫州、建昌軍鄉丁、關軍、槍仗手各千七百七十八人為定額。每歲農隙,輪監司、提舉司官案閱武藝,以備奸
 盜。從前江西轉運副使蔣之奇請也。



 宣和三年,兵部言:「近因江西漕臣謂本路槍仗手,元豐七年以八千三十五人為額,至元祐中減罷七千一百四十二人,元符間雖嚐增立人數,比之元額猶減其七。乞詔諸路監司、帥臣並遵熙寧舊制補足元額。」從之。



 蕃兵者,具籍塞下內屬諸部落,團結以為藩籬之兵也。西北邊羌戎,種落不相統一,保塞者謂之熟戶,餘謂之生戶。陝西則秦鳳、涇原、環慶、鄜延,河東則石、隰、麟、府。其
 大首領為都軍主,百帳以上為軍主,其次為副軍主、都虞候、指揮使、副兵馬使,以功次補者為刺史、諸衛將軍、諸司使、副使、承製、崇班供奉官至殿侍。其充本族巡檢者,奉同正員,月添支錢十五千,米麵傔馬有差。刺史、諸衛將軍請給,同蕃官例。首領補軍職者,月奉錢自三千至三百,又歲給冬服綿袍凡七種,紫綾三種。十將而下皆給田土。



 康定初,趙元昊反,先破金明砦,殺李士彬父子。蕃部既潰,乃破塞門、安遠砦,圍延州。二年,陝西體量
 安撫使王堯臣言:「涇原路熟戶萬四百七十餘帳之首領,各有職名。曹瑋帥本路,威令明著,嚐用之以平西羌。其後,守將失於撫馭,浸成驕黠。自元昊反,鎮戎軍及渭州山外皆被侵擾,近界熟戶亦遭殺掠。蕃族之情,最重酬賽,因其釁隙而激怒之,可復得其用。請遣人募首領願效用者,籍姓名及士馬之數。數及千人,聽自推有謀勇者授班行及巡檢職名,使將領出境。破蕩生戶所獲財畜,官勿檢核。得首級及傷者給賞,仍依本族職名遷
 補增奉。」詔如所請。



 慶曆二年,知青澗城種世衡奏:募蕃兵五千,涅右手虎口為「忠勇」字,隸折馬山族。言者因請募熟戶,給以禁軍廩賜使戍邊。悉罷正兵。下四路安撫使儀,環慶路範仲淹言:「熟戶戀土田,護老弱、牛羊,遇賊力戰,可以藩蔽漢戶,而不可倚為正兵。大率蕃情黠詐,畏強淩弱,常有以製之則服從可用,如倚為正兵必至驕蹇。又今蕃部都虞候至副兵馬使奉錢止七百,悉無衣廩,若長行遽得禁兵奉給,則蕃官必生徼望。況歲罕
 見敵,何用長與廩給?且錢入熟戶,蕃部資市羊馬、青鹽轉入河西,亦非策也。若遇有警,旋以金帛募勇猛,為便。」議遂格。



 治平二年,詔陝西四路駐泊鈐轄秦鳳梁寔、涇原李若愚、環慶王昭明、鄜延韓則順各管勾本路蕃部,團結強人、壯馬,預為經畫,寇至則老弱各有保存之所。仍諭寔等往來蕃帳,受其牒訴,伸其屈抑,察其反側者羈縻之,勿令猜阻以萌釁隙。實等至蕃部召首領,稱詔犒勞,齎以金帛;籍城砦兵馬,計族望大小,分隊伍,給旗幟,
 使各繕堡壘,人置器甲,以備調發。仍約:如令下不集,押隊首領以軍法從事。自治平四年以後,蕃部族帳益多,而撫禦團結之製益密,故別附於其後云:


秦鳳路:砦十三,強人四萬一千一百九十四,壯馬七千九百九十一。
 \gezhu{
  三陽砦,十八門、三十四大部族、四十三姓、一百八十族,總兵馬三千四百六十七。隴城砦,五門、五大部族、三十四小族、三十四姓,總兵馬二千五十四。弓門砦,二大門、十七部族、十七姓、十七小族,總兵馬一千七百四。治坊砦,二大門、二大部族、九姓、九小部族,總兵馬三百六十。┒穰砦,二大門、二大部族、十一姓、十一小族,總兵馬一千八百。靜戎砦,門三,計大部族十、六姓、十六小族,總兵馬六百二十五。
  定西砦,四門、四大部族、十六姓、二十八族,總兵馬六百。伏羌砦,二門、二大部族、三十二姓、三十三小部族,總兵馬一千九百九十二。安遠砦,二十三門、二十三大部族、一百二十六姓、一百二十六小族,總兵馬五千三百五十。來遠砦,八門、八大部族、十九姓、十九小族,總兵馬一千五百七十四。寧遠砦,四門、四大部族、三十六姓、三十六小族,總兵馬七千四百八十。古渭砦,一百七十二門、一百七十一姓、十二大部族、一萬六千九百七十小帳,兵七千七百、馬一千四百九十。}


鄜延路:軍、城、堡、砦十,蕃兵一萬四千五百九十五,官馬二千三百八十二,強人六千五百四十八,壯馬八百十。
 \gezhu{
  永平砦,東路都巡檢所領八族,兵一千七百五十四、馬四百九。青澗城,二族,兵四千五百十、馬七百三十四。
  龍安砦,鬼魁等九族,兵五百九十九、馬一百二十九。西路德靖砦,同都巡檢所領揭家等八族,兵一千一百一十四、馬一百五十。安定堡,東路都巡檢所領十六族,兵一千九百八十九、馬四百六十。保安軍,兩族,兵三百六十一、馬五十。德靖,西路同都巡檢所領二十族,兵七千八百五、馬八百七十七。又小胡等十九族,兵六千九百五十六、馬七百二十五。保安軍,北都巡檢所領厥七等九族,兵一千四百四十一、馬一百六十七。園林堡,兩族,兵八百二十二、馬九十三。肅戎軍,卞移等八族,兵七百四十八、馬一百二十三。}


涇原路:鎮、砦、城、堡二十一,強人一萬二千四百六十六,壯馬四千五百八十六,為一百十甲,總五百五隊。
 \gezhu{
  新城鎮,四族,總兵馬三百四十一,為十六隊。截原砦,六族,總兵馬五百九十六,為六甲二十隊。平安砦,十一族,總兵
  馬二千三百八十四,為十甲四十六隊。開邊砦,十八族,總兵馬一千二百五十四,為九甲四十四隊。新門砦,十二族,總兵馬一千七十三,為三甲二十八隊。西壕砦,三族,總兵馬四百五十四,為四甲二十隊。柳泉鎮,十二族,總兵馬九百八十六,為七甲三十一隊。綏寧、海寧砦,四族,總兵馬七百八十八,為四十甲三十二隊。靖安砦,四族,總兵馬一千九百八十二,為四甲五十九隊。瓦亭砦,四族,總兵馬五百九十一,為四甲十九隊。安國鎮,五族,總兵馬六百三十四,為五甲二十二隊。耀武鎮,一族,總兵馬三十二,為一隊。新砦,兩族,總兵馬一百九。東山砦,四族,總兵馬二百二,為四甲九隊。彭陽城,三族,總兵馬一百八十四,為六甲十二隊。德順軍,強人三千六百七十六,壯馬二千四百八十五,為三十六甲一百三十五隊。本軍二十一族,總兵馬二千五百二,為三十六隊。隆德砦,七族,總兵馬二百五十六,為一十七甲十九隊。靜邊砦,二十四族,
  總兵馬一千八百七,為三十六隊。水洛城,十九族,總兵馬一千三百五十四,為十九甲三十八隊。通邊砦,五族,總兵馬一百七十六,為三隊。}


環慶路:鎮、砦二十八,強人三萬一千七百二十三,壯馬三千四百九十五,總一千一百八十二隊。
 \gezhu{
  安塞砦,四族,強人三百五十一,壯馬三十,為十六隊。洪德砦,二族,強人二百七十三,壯馬五十二,為十隊。肅遠砦,三族,強人一千五百五十九,壯馬二百六十三,為六十隊。烏侖砦,一族,強人六百八十四,壯馬一百一十八,為二十六隊。永和砦,旁家一族計六標,強人一千二百五十五,壯馬二百二,為四十四隊。平遠砦,六族,強人五百四十,壯馬八十七,為二十七隊。安遠砦,六族,強人七百四十八,壯馬一百一十六,為三十隊。合道鎮,十四族,強人一
  千五百六十五,壯馬一百八十三,為五十七隊。木波鎮,十四族,強人二千一百六十九,壯馬一百九十五,為六十一隊。石昌鎮,二族,強人四百六十二,壯馬三十四,為十七隊。馬領鎮,四族,強人一千一十六,壯馬八十,為二十四隊。團堡砦,二族,強人一千二十二,壯馬一百十一,為二十四隊。荔原堡,十三族,強人二千二百二十一,壯馬三百九十四,為八十二隊。大順城,二十三族,強人三千四百九十一,壯馬三百十四,為一百四十一隊。柔遠砦,十二族,強人三千三百八十一,壯馬一千,為九十隊。東穀砦,十六族,強人四百五十九,壯馬五十六,為十四隊。西穀砦,十族,強人一千七百九十四,壯馬一百四十,為六十五隊。淮安鎮,二十七族,強人四千三百六十八,壯馬三百二十一,為一百七十隊。平戎鎮,八族,強人一千八十五,壯馬一百七十一,為四十一隊。五交鎮,十族,強人一千一百七,壯馬七十三,為四十九隊。合水鎮,四族,強人六百三十一,
  壯馬九十五,為二十四隊。鳳川鎮,二十三族,強人八百七十五,壯馬一百四十三,為二十隊。華池鎮,三族,強人二百六十二,壯馬三十八,為十二隊。業樂鎮,十七族,強人一千一百七十二,壯馬六十四,為四十六隊。府城砦,一族,強人二百三十三,壯馬五,為七隊。}



 治平四年,郭逵言:「秦州青雞川蕃部願獻地,請於川南牟穀口置城堡,募弓箭手,以通秦州、德順二州之援,斷賊入寇之路,」閏三月,收原州九砦蕃官三百八十一人,總二百二十九族,七千七百三十六帳,蕃兵萬人,馬千匹。是歲,罷四路內臣主蕃部者,選逐路升朝使臣諳練
 蕃情者為之。



 熙寧元年,議者謂:



 熟羌乃唐設三使所統之党項也。自西夏不臣,種落叛散,分寓南北。為首領者父死子繼,兄死弟襲,家無正親,則又推其旁屬之強者以為族首,多或數百,雖族首年幼,第其本門中婦女之令亦皆信服,故國家因其俗以為法。其大首領,上自刺史,下至殿侍,並補本族巡檢,次首領補軍主、指揮使,下至十將,第受廩給。歲久,主客族帳,混淆莫紀。康定中,嚐遣蔣偕籍之。今逾三十年,主家或以累降失其先職族
 首名品,而客戶或以功為使臣,軍班超處主家之上。軍興調發,有司惟視職名,使號令其部曲,而眾心以非主家,莫肯為用。



 請自今蕃官身歿,秩高者子孫如例降等以為本族巡檢,其旁邊能捍賊者給奉,遠邊者如舊限以歲月;其已降等或三班差使、殿侍身歿無等可降者,子孫不降,充軍主、指揮使者即以為殿侍。如此,則本族蕃官名品常在。或其部曲立功當任官者,非正親毋得為本族巡檢,止增其奉;其軍主至十將,祖、父有族帳兵
 騎者,子孫即承其舊,限年受廩給;能自立功者不用此令。如此,則熟羌之心皆知異日子孫不失舊職,世為我用矣。



 樞密院乃會河東路,蕃部承襲不降資;秦鳳路降兩資,涇原路蕃官告老以門內人承代亦不降資,鄜延、環慶路蕃官使臣比類授職。蕃官副兵馬使以上元無奏到之人,詔鄜延、環慶路蕃官本族首領子孫當繼襲者,若都軍主以下之子孫勿降,殿侍並差使、殿侍之子孫充都軍主,借職、奉職之子孫充殿侍,侍禁、殿直之子
 孫充差使、殿侍、供奉官之子孫補借職,承製以下子孫補奉職;其諸司副使以上子孫合繼襲者,視漢官遺表加恩二等。奏可。



 二月,知青澗城劉怤言:「所隸歸明號箭手八指揮,凡三千四百餘人、馬九百匹,連歲不登,願以丹州儲糧振恤。」詔下其章轉運司行之。



 二年,郭逵奏:「蕃兵必得人以統領之。若專迫以嚴刑,彼必散走山谷,正兵反受其弊。當設六術以用之:曰遠斥堠,曰擇地利,曰從其所長,曰舍其所短,曰利誘其心,曰戰助其力。此用
 蕃兵法也。」詔從之。



 三年,宣撫使韓絳言:「親奉德音,以蕃部子孫承襲者多幼弱,不能統眾,宜選其族人為眾信伏者代領其事。聖算深遠,真得禦邊之要。請下諸路帥臣以詔從事。」



 四年,詔:「蕃官殿侍、三班差使補職,或由殿侍遷差使及十二年,嚐充巡檢或管幹本族公事,或為蕃官指揮,或嚐備守禦之任者,總管司以聞,特與遷改。」



 五年,王韶招納沿邊蕃部,自洮、河、武勝軍以西,至蘭州、馬銜山、洮、岷、宕、疊等州,凡補蕃官、首領九百三十二人,
 首領給飧錢、蕃官給奉者四百七十二人,月計費錢四百八十餘緡,得正兵三萬,族帳數千。



 六年,帝謂輔臣曰:「洮西香子城之戰,官軍貪功,有斬巴氈角部蕃兵以效級者,人極嗟憤。昔李靖分漢蕃兵各為一隊,無用眾於紛亂。」王安石進曰:「李靖非素拊循蕃部者也,故其教兵當如此。今熙河蕃部既為我用,則當稍以漢法治之,使久而與漢兵如一。武王用微、盧、彭、濮人,但為一法。今宜令蕃兵稍與漢同,與蕃賊異,必先錄用其豪傑,漸以化
 之。此用夏變夷之術也。」帝乃詔王韶議其法。



 帝曰:「岷、河蕃部族帳甚眾,儻撫禦鹹得其用,可以坐製西夏,亦所謂以蠻夷攻蠻夷者也。陝西極塞,儻會合訓練,為用兵之勢以愾敵人,彼必隨而聚兵以應我。頻年如此,自致困弊。兵法所謂『佚能勞之』者也。」安石對曰:「 朝廷當先為不可勝,聚糧積財,選兵而已。新附之羌,厚以爵賞,收其豪傑,賜之堅甲利兵,以激其氣,使人人皆有趨赴之誌,待我體強力充,鼓行而西,將無不可者。」馮京、王珪曰:「儻
 如聖策,多方以誤之,彼既疲於點集,而我無攻取之實,久之必不我應。因爾舉兵,若蹈無人之境矣。」帝曰:「此正晉人取吳之策也。夫欲經營四夷,宜無先於此矣。」帝嚐謂:「蕃部未嚐用兵,恐以虛名內附,臨事不可使。」安石對曰:「剛克柔克,所用有宜。王韶以為先以恩信結納其人,有強梗不服者,乃以殺伐加之。大抵蕃部之情,視西夏與中國強弱為向背。若中國形勢強,附中國為利,即不假殺伐,自當堅附。矧蕃部之俗,既宗貴種,又附強國,今
 用木征貴種等三人,又稍以恩信收蕃部,則中國形勢愈強,恐不假殺伐,而所附蕃部自可製使。」帝以為然。是時,王韶拓熙河地千二百里,招附三十餘萬口。安石奏曰:「今以三十萬之眾,漸推文法,當即變其夷俗。然韶所募勇敢士九百餘人,耕田百頃,坊三十餘所。蕃部既得為漢,而其俗又賤土貴貨,漢人得以貨與蕃部易田,蕃人得貨,兩得所欲,而田疇墾,貨殖通,蕃漢為一,其勢易以調禦。請令韶如諸路以錢借助收息,又捐百餘萬緡
 養馬於蕃部,且什伍其人,獎勸以武藝,使其人民富足,士馬強盛,奮而使之,則所向可以有功。今蕃部初附,如洪荒之人,唯我所禦而已。」



 七年,韶言:「討平河州叛蕃,辟土甚廣,已置弓箭手,又以其餘地募蕃兵弓箭手,每砦三指揮或至五指揮,每指揮二百五十人,人給田百畝,以次蕃官二百畝,大蕃官三百畝,仍募漢弓箭手為隊長,稍眾則補將校,暨蕃官同主部族之事。其蕃弓箭手並刺『蕃兵』字於左耳,以防漢兵之盜殺而效首者。」詔如其
 請。十一月,王中正團結熙河界洮、河以西蕃部,得正兵三千八十六人,正、副隊將六十人,供贍一萬五千四百三十人。



 八年五月,詔李承之參定蕃兵法。十一月,詔:「選陝西蕃兵丁壯戶,九丁以上取五,六取四,五取三,三取二,二取一,並年二十以上,涅手背,毋過五丁。每十人置十將一,五十人置副兵馬使一,百人置軍使一、副兵馬使一,二百人置軍使一、副兵馬使三,四百人加軍使一、副兵馬使一,五百人又加指揮使一、副兵馬使一,過五
 百人,每百人加軍使一、副兵馬使一,即一族三十人已上亦置副兵馬使一,不及二十人止置十將。月受奉,仍增給錢,指揮使一千五百至十將有差。」



 十年,樞密院言:「陝西、河東議立團結蕃部法,欲如所奏。」上手詔曰:「夏人所恃以強國者,山界部落數萬之眾爾。按其地誌,朝廷已據有其半。彼用之則並小淩大,所向如欲。在我則徒能含撫豢養,未嚐得其死力,豈惟不能用之,又恐其為患也。故小有悖戾,有司惟能以利說解之,上下相習畏
 憚,任其縱散,久失部勒。其近降之法,固未可信其必行,然以理言之,彼此均有其人,而利害遼遠。今苟循邊人,眾知其說,止於舊法聊改一二,則收功疑亦不異往日。徒為紛紛,無補於事。可再下呂惠卿參詳以聞。」



 元豐六年,詔:「蕃官雖至大使,猶處漢官小使臣之下。朝廷賞功增秩,以為激勸,乃爾卑抑,則孰知遷官之榮?宜定蕃漢官序位。」後河東經略司言:「蕃官部堡塞兵出戰,嚐以漢官驅策,恐不當與漢官序位。 」而兵部請蕃漢非統轄者
 乃令序官,奏可。熙河蘭會路經略製置使李憲言:「治蕃兵,置將領,法貴簡而易行,詳而難犯。臣今酌蕃情立法,凡熙河蘭會五郡,各置都同總領蕃兵將二人;本州諸部族出戰,蕃兵及供贍人馬各置管押蕃兵使臣十人。五郡蕃兵自為一將,出戰則以正兵繼之,旗幟同色。蕃兵以技藝功勞第為四等,蕃官首領推遷如之。」八月,憲又言:「漢蕃兵騎雜為一軍,語言不通,居處飲食悉不便利。昔李靖以蕃落自為一法,臣近以蕃兵自為一將,厘
 漢、蕃為兩軍,相參號令,軍事惟所使焉。」



 七年,瀘南緣邊安撫司言:「羅始黨生界八姓,各願依七姓、十九姓刺充義軍,團結為三十一指揮,凡一萬五千六百六十人。」從之。



 元祐元年,臣僚言:「涇原路蕃兵人馬凡眾,遇臨敵與正兵錯雜,非便。」詔下其章四路都總管詳議,環慶範純粹言:「漢、蕃兵馬誠不可雜用,宜於逐將各選廉勇曉蕃情者一員專充蕃將,令於平日鈐束訓練,遇有調發,即令部領為便。」又言:「頃兵部議乞蕃、漢官非相統轄者,並
 依官序相壓;其城砦等管轄蕃官,即依舊在本轄漢官之下。詔從其請。且諸路蕃官,不問官職高卑,例在漢官之下,所以尊中國,製遠人也。行之既久,忽然更製,便與不相統轄之官依品序位,即邊上使臣及京職官當在蕃官之下十有八九,非人情所能堪。蕃部凶驕,豈可輒啟?宜悉依舊制,並序漢官之下。」從之。



 元符二年三月,涇原經略司言:「乞將東西路蕃兵將廢罷,仍於順便城砦隸屬逐將統領,與漢兵相兼差使。」秦鳳路如之。四月,環
 慶路經略安撫司言:「新築定邊城有西夏來投蕃部甚眾,欲自今將歸順之人,就新城收管給田,仍乞選置總領蕃兵正、副二員。」從之。



\end{pinyinscope}