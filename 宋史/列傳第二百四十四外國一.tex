\article{列傳第二百四十四外國一}

\begin{pinyinscope}

 昔唐承隋後,隋承周、齊,上溯元魏,故西北之疆有漢、晉正朔所不逮者,然亦不過使介之相通、貢聘之時至而
 已。唐德既衰,荒服不至,五季迭興,綱紀自紊,遠人慕義,無所適從。宋祖受命,諸國削平,海內清謐。于是東若高麗、渤海,雖阻隔遼壤,而航海遠來,不憚跋涉。西若天竺、于闐、回鶻、大食、高昌、龜茲、拂林等國,雖介遼、夏之間,筐篚亦至,屢勤館人。党項、吐蕃唃廝囉董氈瞎征諸部,夏國兵力之所必爭者也,宋之威德亦暨其地,又間獲其助焉。交阯、占城、真臘、蒲耳、大理濱海諸蕃,自劉鋹、陳洪進來歸,接踵修貢。宋之待遇亦得其道,厚其委積而不
 計其貢輸,假之榮名而不責以煩縟;來則不拒,去則不追;邊圉相接,時有侵軼,命將致討,服則舍之,不黷以武。先王柔遠之制豈復有加于是哉!南渡以後,朔漠不通,東南之陬以及西鄙,冠蓋猶有至者。交人遠假爵命,訖宋亡而後絕焉。



 女直在宋初屢貢名馬,他日強大,修怨于遼,其索叛臣阿疏,責還所掠宋詔,猶知以通宋為重;及渝海上之盟,尋構大難,宋遂為所絀辱,豈非自取之過乎!前宋舊史有《女直傳》,今既作《金史》,義當削之。
 夏國雖亻面鄉不常,而視金有間,故仍舊史所錄存焉。



 李彝興,夏州人也,本姓拓跋氏。唐貞觀初,有拓跋赤辭者歸唐,太宗賜姓李,置靜邊等州以處之。其後析居夏州者號平夏部。唐末,拓跋思恭鎮夏州,統銀、夏、綏、宥、靜五州地,討黃巢有功,復賜李姓。思恭卒,弟思諫代為定難軍節度使。思諫卒,思恭孫彝昌嗣。梁開平中,彝昌遇害,將士立其族子蕃部指揮仁福。仁福卒,子彝超嗣。事具《五代史》。



 彝興,彝超之弟也,本名彝殷,避宋宣祖諱,改「
 殷」為「興」。初為行軍司馬,清泰二年,彝超卒,遂加定難軍節度使。晉初,加同平章事。開運初,授契丹西南招討使。漢初,加兼侍中。周初,加中書令。顯德初,封西平王。世宗即位,加太保。恭帝初,加太傅。



 宋初,加太尉。北漢劉鈞結代北諸部來寇麟州,彝興遣部將李彝玉會諸鎮兵禦之,鈞衆遂引去。建隆初,獻馬三百匹,太祖大喜,親視攻玉為帶,且召使問曰:「汝帥腹圍幾何?」使言:「彝興腰復甚大。」太祖曰:「汝帥真福人也。」遂遣使以帶賜之。



 乾德五年,
 卒,太祖廢朝三日,贈太師,追封夏王。子克睿立。



 克睿初名光睿,避太宗諱改「光」為「克」。彝興之卒,自權知州事,授檢校太保、定難軍節度使。



 開寶九年,率兵破北漢吳堡砦,斬首七百級,獲牛羊千計,俘砦主侯遇以獻,累加檢校太尉。



 太平興國三年,卒,太宗廢朝二日,贈侍中。子繼筠立。



 繼筠,初為衙內都指揮使、檢校工部尚書。克睿卒,自權知州事,授檢校司徒、定難軍節度觀察留後。太宗征北漢,繼筠遣銀州刺史李光遠、綏州刺史
 李光憲率蕃、漢兵列陣渡河,略太原境以張軍勢。



 太平興國五年,卒,弟繼捧立。



 繼捧立,以太平興國七年率族人入朝。自上世以來,未嘗親覲者,繼捧至,太宗甚嘉之,賜白金千兩、帛千匹、錢百萬。祖母獨孤氏亦獻玉盤一、金盤三,皆厚賚之。繼捧陳其諸父、昆弟多相怨,願留京師。乃遣使夏州護緦麻已上親赴闕,授繼捧彰德軍節度使,並官其昆弟夏州蕃落指揮使克信等十二人有差,遂曲赦銀、夏管內。太宗嘗宴群臣苑中,謂繼捧
 曰:「汝在夏州用何道以制諸部?」對曰:「羌人鷙悍,但羈糜而已,非能制也。」弟權知夏州克文來朝,以唐僖宗所賜其祖思恭鐵券及朱書御劄來上,改博州防禦使。初,繼捧之入也,弟繼遷出奔,及是,數來為邊患。有言繼遷悉知朝廷事,蓋繼捧泄之。乃出為崇信軍節度使,克憲為道州防禦使,克文遣歸博州,並選常參官為通判,以專郡政。



 端拱初,改感德軍節度使。屢發兵討繼遷不克,用宰相趙普計,欲委繼捧以邊事,令圖之。因召赴闕,賜姓
 趙氏,更名保忠,太宗親書五色金花箋以賜之,授夏州刺史,充定難軍節度使、夏銀綏宥靜等州觀察處置押蕃落等使。賜金器千兩、銀器萬兩,並賜五州錢帛、芻粟、田園。保忠辭日,宴于長春殿,賜襲衣、玉帶、銀鞍馬、錦彩三千匹、銀器三千兩,又賜錦袍、銀帶五百,副馬百匹。至鎮數月,上言繼遷悔過歸款,乃授繼遷官,然實無降心也。二年,加保忠特進、同中書門下平章事。



 淳化初,與繼遷戰于安慶澤,繼遷中流矢遁去。保忠乞師禦繼遷,遣
 商州團練使翟守素率兵援之。賜保忠茶百斤、上醞十石。乃獻白鶻,名海東青,以久罷畋獵,詔慰還之。



 五年,繼遷攻靈州,遣侍衛馬軍都指揮使李繼隆討之。保忠先挈其母與妻子壁野外,乃上言與繼遷解怨,獻馬五十匹,乞罷兵。帝覽奏,立遣中使督繼隆進軍。及兵壓境,保忠反為繼遷所圖,欲並其衆,縛牙校趙光祚,襲其營帳。保忠方寢,聞難作,單騎走還城,為大校趙光嗣閉于別室,旦開門迎繼隆,乃執保忠送闕下,待罪崇政殿庭。帝
 詰責數四,釋之,賜冠帶、器幣,並賜其母金銀器以撫之。尋責授右千牛衛上將軍,封宥罪侯,賜第京師。保忠狀貌雄毅,居環列,奉朝請,常怏怏不自得。



 咸平中,丁內艱,以本官起復,遷右金吾衛上將軍,判岳州,移復州。



 景德元年病劇,上言有子永哥不肖,乞配春州。帝以其病語,乃授永州別駕,詔監軍察之。尋卒,贈威塞軍節度使。克文亦死,贈岳州防禦使。



 天禧四年,錄其孫從吉為三班奉職。



 繼遷,繼捧族弟也。高祖思忠,嘗從兄思恭討黃
 巢,拒賊于渭橋,表有鐵鶴,射之沒羽,賊駭之,遂先士卒,戰沒,僖宗贈宥州刺史,祠于渭陽。曾祖仁顏,仕唐,銀州防禦使。祖彝景嗣于晉。父光儼嗣于周。建隆四年,繼遷生于銀州無定河,生而有齒。開寶七年,授定難軍管內都知蕃落使。



 繼捧之歸宋,時年二十,留居銀州,及使至,召緦麻親赴闕,乃詐言乳母死,出葬于郊,遂與其黨數十人奔入地斤澤,澤距夏州東北三百里。



 太平興國八年,知夏州尹憲與都巡檢曹光實偵知,夜襲破之,斬首
 五百級,焚四百餘帳。繼遷與其弟遁免,獲其母與妻。繼遷復連娶豪族,轉遷無常,漸以強大,而西人以李氏世著恩德,往往多歸之。繼遷因語其豪右曰:「李氏世有西土,今一旦絕之,爾等不忘李氏,能從我興復乎?」衆曰:「諾。」遂與弟繼沖、破醜重遇貴、張浦、李大信等起夏州,乃詐降,誘殺曹光實于葭蘆川,遂襲銀州據之,時雍熙二年二月也。三月,破會州,焚毀城郭而去。



 三年,遼以義成公主嫁繼遷,冊為夏國王。四年,知夏州安守忠以三萬
 衆戰于王亭鎮,敗績,繼遷追至城門而返。端拱元年,繼捧之節制夏台,言能歸款,即授洛苑使、銀州刺史。



 淳化初,復與繼捧戰于安慶澤,不利。轉攻夏州,繼捧乞師,及翟守素來,又奉表歸款,授銀州觀察,賜名保吉,子德明管內蕃落使、行軍司馬。



 淳化四年,轉運副使鄭文寶議禁鹽池,用困繼遷。數月,邊人四十二族萬餘騎寇環州,屠小康堡,太宗乃遣錢若水弛其禁,因撫慰之。



 五年正月,繼遷徙綏州民于平夏,部將高文岯等因衆不樂反,攻
 敗之。繼遷復圍堡砦,掠居民,焚積聚,遂攻靈州,詔遣李繼隆等進討。繼遷夜襲保忠,走之,獲其輜重以歸。七月,乃獻馬以謝。又遣弟廷信獻馬、橐駝,太宗撫賚甚厚,遣內侍張崇貴詔諭,賜茶藥、器幣、衣物。



 至道初,遣左都押衙張浦以橐駝、良馬來獻。太宗令衛士翹關、超乘、引強、奪槊于後園,俾浦等觀,且令兵士皆拓兩石弓。帝笑問浦曰:「羌人敢敵否?」浦曰:「羌部弓弱矢短,但見此長大人則已遁矣,況敢敵乎!」繼遷乞禁邊盜掠,詔令謹守疆
 場,還所盜物。遣閣門副使馮訥、中使賈繼隆持詔拜繼遷鄜州節度使,不受。乃以浦為鄭州團練,留京師。繼遷表鄭文寶誘其部長嵬囉、嵬悉,遂貶文寶藍山令。繼遷以千騎攻清遠軍,守臣張延擊退之。



 二年春,命洛苑使白守榮等護送芻粟四十萬于靈州,且令車重先後作三隊,丁夫持弓矢自衛,士卒布方陣以護之,遇敵則戰,可以無失。復令會州觀察使田紹斌率兵應援。而守榮乃並為一運,繼遷邀擊于浦洛河,紹斌不救,衆潰,運饋
 盡為繼遷所得。太宗聞之怒。四月,復命李繼隆為環、慶等州都部署。會四方館使曹璨自河西至,言繼遷衆萬余圍靈武,城中上表告急,為繼遷所得,遂頓兵不去。時朝議或云率輕騎三道搗平夏;或云暑涉旱海無水泉,糧運艱辛,不如靜以待之,帝不聽。九月,親部分諸將,繼隆出環州,丁罕出慶州,範廷召出延州,王超出夏州,張守恩出鄜州,五路進討,直抵平夏。繼隆以環州路迂,乃自青岡峽繞靈武徑趨平夏,兵行數日,與丁罕合,又行
 十餘日無所見,乃引還。張守恩遇之,不戰而遁。王超、範廷召遇之于烏白池,大小數十戰,不利,諸將失期,士卒困乏。繼遷復令軍主史不癿駐屯橐駝口以阻歸宋人,繼隆遣田敏等擊之。



 咸平春,繼遷復表歸順。真宗乃授夏州刺史、定難軍節度、夏銀綏宥靜等州觀察處置押蕃落等使,加邑千戶,實封二百戶,益功臣號,乃放張浦還。復遣押衙劉仁謙表讓恩命,詔不允,賜仁謙錦袍、銀帶。尋遣弟繼瑗來謝恩,授繼瑗亳州防禦使,封繼遷母
 衛慕氏衛國太夫人,子德明為定難軍節度行軍司馬。未幾,復抄邊。



 四年,麟府副部署曹璨率熟戶兵邀繼遷輜重于柳撥川,殺獲甚衆。九月,來攻破定州、懷遠縣及堡靜、永州,清遠軍監軍段義叛,城遂陷。五年三月,繼遷大集蕃部,攻陷靈州,以為西平府。



 六年春,遂都于靈州,詔遣張崇貴、王涉議和,割河西銀、夏等五州與之。六月,復以二萬騎圍麟州,詔金明巡檢李繼周擊之。圍未解,麟州部署請濟師,真宗閱地圖曰:「麟州依險,三面孤絕,
 戮力可守,但城中乏水可憂耳。」乃遣兵走援。繼遷果據水砦,薄城已五日。知州衛居寶出奇兵突戰,縋勇士城下,城上鼓噪,矢石如注,殺傷萬餘人,繼遷乃拔去。遂率衆攻西蕃,取西涼府,都首領潘羅支偽降,繼遷受之不疑。羅支遽集六穀蕃部及者龍族合擊之,繼遷大敗,中流矢。八月,復聚兵浦洛河,聲言攻環州,詔張凝等分兵以待之。



 景德元年正月二日卒,年四十二,子德明立。祥符五年,德明追上繼遷尊號曰應運法天神智仁聖至
 道廣德孝光皇帝。元昊追諡曰神武,廟號太祖,墓號裕陵。



 德明小字阿移,母曰順成懿孝皇后野利氏,即位于柩前,時年二十三。邊臣以德明初立,乞詔撫之,因賜詔令審圖去就。又詔蕃族萬山、萬遇、龐羅逝安、萬子都虞候、軍主吳守正馬尾等,能率部下歸順者,授團練使,銀萬兩、絹萬匹、錢五萬緡、茶五千斤;其有亡命叛去者,釋罪甄錄。既而康奴𡗀移等率屬來降。德明遣牙將王旻奉表歸順,賜旻錦袍、銀帶,遣侍禁夏居厚持詔答
 之,因詔河西羌族各守疆場。德明連歲表歸順。



 三年,復遣牙將劉仁勖奉誓表請藏盟府,且言父有遺命。帝嘉之,乃授特進、檢校太師兼侍中、持節都督夏州諸軍事、行夏州刺史、上柱國,充定難軍節度、夏銀綏宥靜等州管內觀察處置押蕃落等使,西平王,食邑六千戶,食實封一千戶,仍賜推忠保順亮節翊戴功臣。遣內侍左右班都知張崇貴、太常博士趙湘等充旌節官告使,賜襲衣、金帶、銀鞍勒馬、銀萬兩、絹萬匹、錢三萬貫、茶二萬斤,給
 奉如內地。因責子弟入質,德明謂非先世故事,不遣。乃獻御馬二十五匹、散馬七百匹、橐駝三百頭謝恩。



 四年,又獻馬五百匹、橐駝三百頭,謝給奉廩,賜襲衣、金帶、器幣。及請使至京市所需物,從之。五月,母罔氏薨,除起復鎮軍大將軍、右金吾衛上將軍,員外置同正員,餘如故。以殿中丞趙稹為吊贈兼起復官告使,德明以樂迎至柩前,明日釋服,涕泣對使者自陳感恩。及葬,請修供五臺山十寺,乃遣閣門祗候袁瑀為致祭使,護送所供物
 至山。復獻馬五百匹,助修章穆皇后園陵。



 大中祥符元年,以天書降,加賜守正功臣,益食邑一千戶,食實封四百戶。俄境內旱,詔榷場勿禁西人市糧,以振其乏。東封,又遣使來獻,禮成,加兼中書令,益食邑千戶,實封四百戶。時遼亦遣使冊德明為大夏國王。明年,出侵回鶻,恒星晝見,德明懼而還。



 三年,境內饑,上表求粟百萬,朝議不知所出。時王旦為相,請敕有司具粟百萬于京師,詔其來取。德明既得詔,曰:「朝廷有人。」遂止。大起宮室于
 𨫼子山。會旱,西攻河州、甘州宗哥族及秦州緣邊熟戶。遂出大里河,築柵蒼耳平。



 四年,祀汾陰,進中書令。五年,聖祖降,加守太保。七年二月,謁太清宮,遣使來獻方物,加宣德功臣。八年,築堡于石州濁輪穀,將建榷場,詔緣邊安撫司止之。



 九年,因表邊臣違約招納逃亡,云:「自景德中進誓表,朝廷亦降詔書,應兩地逃民,緣邊雜戶不令停舍,皆俾交還。自茲謹守翰垣,頗有倫理。自向敏中歸闕,張崇貴雲亡,後來邊臣,罕守舊制,各務邀功,不虞生
 事,遂致綏、延等界,涇、原以來,擅舉兵甲,入臣境土;其有叛亡部族,劫掠主財,去者百無十回。臣之邊吏,亦務蔽藏,俱失奏論,漸棄盟約。」詔答已令鄜延、涇原、環慶、麟府等路約束邊部,毋相攻劫,其有隱蔽逃亡,畫時勘送。本國亦宜戒部下,毋有藏匿,各遵紀律,以守封疆。



 五年,德明追尊繼遷為太祖應運法天神智仁聖至道廣德光孝皇帝,廟號武宗。七年,甘露降國中。



 天禧元年正月,加守太傅,食邑千戶,實封四百戶。三年春,德明丁繼立母
 憂,除起復如前制,以屯田員外郎上官亻必為吊贈兼起復官告使,閣門祗候常希古為致祭使。冬,郊祀,又加崇仁功臣。



 四年,遼主親將兵五十萬,以狩為言,來攻涼甸,德明帥衆逆拒,敗之。五年,遼復遣金吾衛上將軍蕭孝誠齎玉冊金印,冊為尚書令、大夏國王。



 乾興元年,加純誠功臣。德明自歸順以來,每歲旦、聖節、冬至皆遣牙校來獻不絕。而每加恩賜官告,則又以襲衣五,金荔支帶、金花銀匣副之,銀沙鑼、盆、合千兩,錦彩千匹,金塗銀鞍
 勒馬一匹,副以纓、復,遣內臣就賜之。又遣閣門祗候賜冬服及頒《儀天具注曆》。



 明年,攻慶州柔遠砦。巡檢楊承吉與戰不利,命曹瑋為環、慶、秦州緣邊巡檢安撫使禦備之。德明城懷遠鎮為興州以居。



 仁宗即位,加尚書令。德明娶三姓,衛慕氏生元昊,咩迷氏生成遇,訛藏屈懷氏生成嵬。



 天聖六年,德明遣子元昊攻甘州,拔之。八年,瓜州王以千騎降于夏。火星入南斗。九年十月,德明卒,時年五十一,追諡曰光聖皇帝,廟號太宗,墓號嘉陵。
 宋贈太師、尚書令兼中書令,以尚書度支員外郎朱昌符為祭奠使,六宅副使、內侍省內侍押班馮仁俊副之,賻絹七百匹、布三百匹,副以上醞、羊、米、面。將葬,賜物稱是,皇太后所賜亦如之。帝與皇太后成服于苑中。子曩霄立。



 曩霄本名元昊,小字嵬理,國語謂惜為「嵬」,富貴為「理」。母曰惠慈敦愛皇后衛慕氏。性雄毅,多大略,善繪畫,能創制物始。圓面高准,身長五尺餘。少時好衣長袖緋衣,冠黑冠,佩弓矢,從衛步卒張青蓋。出乘馬,以二旗引,
 百餘騎自從。曉浮圖學,通蕃漢文字,案上置法律,常攜《野戰歌》、《太乙金鑒訣》。弱冠,獨引兵襲破回鶻夜洛隔可汗王,奪甘州,遂立為皇太子。數諫其父毋臣宋,父輒戒之曰:「吾久用兵,疲矣。吾族三十年衣錦綺,此宋恩也,不可負。」元昊曰:「衣皮毛,事畜牧,蕃性所便。英雄之生,當王霸耳,何錦綺為?」德明卒,即授特進、檢校太師兼侍中、定難軍節度、夏銀綏宥靜等州觀察處置押蕃落使、西平王,以工部郎中楊告為旌節官告使,禮賓副使朱允中
 副之。



 既襲封,明號令,以兵法勒諸部。始衣白窄衫,氈冠紅裏,冠頂後垂紅結綬,自號嵬名吾祖。凡六日、九日則見官屬。其官分文武班,曰中書,曰樞密,曰三司,曰御史台,曰開封府,曰翊衛司,曰官計司,曰受納司,曰農田司,曰群牧司,曰飛龍院,曰磨勘司,曰文思院,曰蕃學,曰漢學。自中書令、宰相、樞使、大夫、侍中、太尉已下,皆分命蕃漢人為之。文資則襆頭、鞾笏、紫衣、緋衣;武職則冠金帖起雲鏤冠、銀帖間金縷冠、黑漆冠,衣紫旋襽,金塗銀束
 帶,垂蹀躞,佩解結錐、短刀、弓矢韣,馬乘鯢皮鞍,垂紅纓,打跨鈸拂。便服則紫皂地繡盤球子花旋襽,束帶。民庶青綠,以別貴賤。每舉兵,必率部長與獵,有獲,則下馬環坐飲,割鮮而食,各問所見,擇取其長。初,宋改元明道,元昊避父諱,稱顯道于國中。



 景祐元年,遂攻環慶路,殺掠居人,下詔約束之。是歲,改元開運,逾月,或告以石晉敗亡年號也,乃改廣運。元年,母衛慕氏死,遣使來告哀,起復鎮軍大將軍、左金吾衛上將軍,員外置同正員。以內
 殿崇班、閣門祗候王中庸為致祭使,起居舍人郭勸為吊贈兼起復官告使。慶州柔遠砦蕃部巡檢嵬通攻破後橋諸堡,于是元昊稱兵報仇。緣邊都巡檢楊遵、柔遠砦監押盧訓以兵七百與戰于龍馬嶺,敗績。環慶路都監齊宗矩、走馬承受趙德宣、寧州都監王文援之,次節義峰,伏兵發,執宗矩,久之始放歸。



 二年,加兼中書令。遣其令公蘇奴兒將兵二萬五千攻唃廝囉,敗死略盡,蘇奴兒被執。元昊自率衆攻貓牛城,一月不下。既而詐約
 和,城開,乃大縱殺戮。又攻青唐、安二、宗哥、帶星嶺諸城,唃廝囉部將安子羅以兵絕歸路,元昊晝夜角戰二百餘日,子羅敗,遂取瓜、沙、肅三州。元昊既還,欲南侵,恐唃廝囉制其後,復舉兵攻蘭州諸羌,侵至馬銜山,築城凡川。



 元昊既悉有夏、銀、綏、宥、靜、靈、鹽、會、勝、甘、涼、瓜、沙、肅,而洪、定、威、龍皆即堡鎮號州,仍居興州,阻河依賀蘭山為固。始大建官,以嵬名守全、張陟、張絳、楊廓、徐敏宗、張文顯輩主謀議,以鐘鼎臣典文書,以成逋、克成賞、都臥、
 𡗀如定、多多馬竇、惟吉主兵馬,野利仁榮主蕃學。置十二監軍司,委豪右分統其衆。自河北至午臘蒻山七萬人,以備契丹;河南洪州、白豹、安鹽州、羅落、天都、惟精山等五萬人,以備環、慶、鎮戎、原州;左廂宥州路五萬人,以備鄜、延、麟、府;右廂甘州路三萬人,以備西蕃、回紇;賀蘭駐兵五萬、靈州五萬人、興州興慶府七萬人為鎮守,總五十余萬。而苦戰倚山訛,山訛者,橫山羌,平夏兵不及也。選豪族善弓馬五千人迭直,號六班直,月給米二石。鐵
 騎三千,分十部。發兵以銀牌召部長面受約束。設十六司于興州,以總庶務。元昊自製蕃書,命野利仁榮演繹之,成十二卷,字形體方整類八分,而畫頗重復。教國人紀事用蕃書,而譯《孝經》、《爾雅》、《四言雜字》為蕃語。復改元大慶。



 宋寶元元年,表遣使詣五臺山供佛寶,欲窺河東道路。與諸豪歃血約先攻鄜延,欲自德靖、塞門砦、赤城路三道併入,遂築壇受冊,即皇帝位,時年三十。遣潘七布、昌里馬乞點兵集蓬子山,自詣西涼府祠神。



 明年,遣
 使上表曰:



 臣祖宗本出帝胄,當東晉之末運,創後魏之初基。遠祖思恭,當唐季率兵拯難,受封賜姓。祖繼遷,心知兵要,手握乾符,大舉義旗,悉降諸部。臨河五郡,不旋踵而歸;沿邊七州,悉差肩而克。父德明,嗣奉世基,勉從朝命。真王之號,夙感于頒宣;尺土之封,顯蒙于割裂。臣偶以狂斐,制小蕃文字,改大漢衣冠。衣冠既就,文字既行,禮樂既張,器用既備,吐蕃、塔塔、張掖、交河,莫不從伏。稱王則不喜,朝帝則是從,輻輳屢期,山呼齊舉,伏
 願一垓之土地,建為萬乘之邦家。于時再讓靡遑,群集又迫,事不得已,顯而行之。遂以十月十一日郊壇備禮,為世祖始文本武興法建禮仁孝皇帝,國稱大夏,年號天授禮法延祚。伏望皇帝陛下,睿哲成人,寬慈及物,許以西郊之地,冊為南面之君。敢竭愚庸,常敦歡好。魚來雁往,任傳鄰國之音;地久天長,永鎮邊方之患。至誠瀝懇,仰俟帝俞。謹遣弩涉俄疾、你斯悶、臥普令濟、嵬崖妳奉表以聞。



 詔削奪官爵、互市,揭榜于邊,募人能擒元昊若斬
 首獻者,即為定難軍節度使。又遣賀永年齎嫚書,納旌節及所授敕告置神明匣,留歸孃族而去。



 康定元年,環慶路鈐轄高繼隆、知慶州張崇俊攻後橋,而柔遠砦主武英入自北門,拔之。未幾,夏人攻金明砦,執都監李士彬父子。破安遠、塞門、永平諸砦,圍延州,設伏三川口,執劉平、石元孫、傅偃、劉發、石遜等。又攻鎮戎軍,敗劉繼宗、李緯兵五千。環慶部署任福入白豹城,焚其積聚,破四十一族。



 慶曆元年二月,攻渭州,逼懷遠城。韓琦徼巡邊
 至高平,盡發鎮戎兵及募勇士得萬人,命行營總管任福等並擊之,都監桑懌為前鋒,鈐轄朱觀、都監武英繼之。福申令持重,其夕宿三川,夏人已過懷遠東南。翌日,諸軍躡其後。西路巡檢常鼎、劉肅與夏人對壘于張家堡,懌以騎兵趣之。福分兵,夕與懌為一軍,屯好水川。川與能家川隔在隴山外,觀、英為一軍,屯籠洛川,相離五里。期以明日會兵,不使夏人一騎遁,然已陷其伏中矣。元昊自將精兵十萬,營于川口,候者言夏人有砦,數不
 多,兵益進。詰旦,福與懌循好水川西去,未至羊牧隆城五里,與夏軍遇。懌為先鋒,見道傍置數銀泥合,封襲謹密,中有動躍聲,疑莫敢發,福至發之,乃懸哨家鴿百餘,自合中起,盤飛軍上。于是夏兵四合,懌先犯,中軍繼之,自辰至午酣戰。陣中忽樹鮑老旗,長二丈餘,懌等莫測。既而鮑老揮右則右伏出,揮左則左伏出,翼而襲之,宋師大敗。懌、劉肅及福子懷亮皆戰沒。小校劉進勸福自拔,福不聽,力戰死。初,渭州都監趙津將瓦亭塞騎兵三
 千余為諸將後繼。是日,朱觀、武英兵會能家川與夏人遇,陣合,王珪自羊牧隆城以屯兵四千五百人助觀略陣,陣堅不可動,英重傷,不能出軍戰。自午至申,夏軍益至,東陣步兵大潰,衆遂奔。珪、英、津及參軍耿傅、隊將李簡、都監李禹享、劉均皆死于陣。觀以千餘人保民垣,發矢四射,會暮,夏軍引去。將校士卒死者萬三百人,關右震動。軍須日廣,三司告不足,仁宗為之旰食,宋庠請修潼關以備衝突。秋,夏人轉攻河東,及麟、府,不能下,乃引
 兵攻豐州,城孤無援,遂據之;又破寧遠砦,屯要害,絕麟、府餉道。楊偕始請棄河外,保合河津,帝不許。會張亢管勾麟府軍馬事,破之于柏子,又破之于兔毛川,亢築十餘柵,河外始固。元昊雖數勝,然死亡創痍者相半,人困于點集,財力不給,國中為「十不如」之謠以怨之。元昊乃歸塞門砦主高延德,因乞和,知延州范仲淹為書陳禍福以喻之。元昊使其親信野利旺榮復書,語猶嫚。知延州龐籍言,夏境鼠食稼,且旱,元昊思納款,遂令知保安
 軍劉拯諭旺榮言:「公方持靈、夏兵,倘內附,當以西平茅土分冊之。」知青澗城種世衡又遣王嵩以棗及畫龜為書置蠟丸中遺旺榮,諭以早歸之意,欲元昊得之,疑旺榮。旺榮得之笑曰:「種使君亦長矣,何為此兒戲耶!」囚嵩窖中歲餘。知渭州王沿、總管葛懷敏使僧法淳持書往,而旺榮乃出嵩與教練使李文貴至青澗城,自言用兵以來,資用困乏,人情便于和。籍疑其款吾軍,留之數月。



 二年,復大入,戰于定川,宋師大敗,葛懷敏死之。直抵渭
 州,大焚掠而去。詔籍招納,籍遣文貴還。月余,元昊使文貴與王嵩以其臣旺榮、其弟旺令、嵬名環、臥譽諍三人書議和,然屈強不肯削僭號,且雲「如日方中,止可順天西行,安可逆天東下。」籍以其言未服,乃令自請,而詔籍復書許之。



 明年,遣六宅使伊州剌史賀從勖與文貴俱來,猶稱男邦泥定國兀卒上書父大宋皇帝,更名曩霄而不稱臣。兀卒,即吾祖也,如可汗號。議者以為改吾祖為兀卒,特以侮玩朝廷,不可許。詔遣邵良佐、張士元、張
 子奭、王正倫更往議,且許封冊為夏國主,而元昊亦遣如定、聿舍、張延壽、楊守素繼來。



 四年,始上誓表言:「兩失和好,遂曆七年,立誓自今,願藏盟府。其前日所掠將校民戶,各不復還。自此有邊人逃亡,亦毋得襲逐。臣近以本國城砦進納朝廷,其栲栳、鎌刀、南安、承平故地及他邊境蕃漢所居,乞畫中為界,于內聽築城堡。凡歲賜銀、綺、絹、茶二十五萬五千,乞如常數,臣不復以他相干。乞頒誓詔,蓋欲世世遵守,永以為好。倘君親之義不存,或
 臣子之心渝變,使宗祀不永,子孫罹殃。」詔答曰:「朕臨制四海,廓地萬里,西夏之土,世以為胙。今乃納忠悔咎,表于信誓,質之日月,要之鬼神,及諸子孫,無有渝變。申復懇至,朕甚嘉之。俯閱來誓,一皆如約。」十二月,遣尚書祠部員外郎張子奭充冊禮使,東頭供奉官、閣門祗候張士元副之。仍賜對衣、黃金帶、銀鞍勒馬、銀二萬兩、絹二萬匹、茶三萬斤。冊以漆書竹簡,籍以天下樂錦。金塗銀印,方二寸一分,文曰「夏國主印」,錦綬,塗金銀牌。緣冊法
 物,皆銀裝金塗,覆以紫繡。約稱臣,奉正朔,改所賜敕書為詔而不名,許自置官屬。使至京,就驛貿賣,宴坐朵殿。使至其國,相見用賓客禮。置榷場于保安軍及高平砦,第不通青鹽。然宋每遣使往,館于宥州,終不復至興、靈,而元昊帝其國中自若也。



 是歲,遼夾山部落呆兒族八百戶歸元昊,興宗責還,元昊不遣。遂親將騎兵十萬出金肅城,弟天齊王馬步軍大元帥將騎七千出南路,韓國王將兵六萬出北路,三路濟河長驅。興宗入夏境四
 百里,不見敵,據得勝寺南壁以待。八月五日,韓國王自賀蘭北與元昊接戰,數勝之。遼兵至者日益,夏乃請和,退十里,韓國王不從。如是退者三,凡百餘里矣,每退必赭其地,遼馬無所食,因許和。夏乃遷延,以老其師,而遼之馬益病,因急攻之,遂敗,復攻南壁,興宗大敗。入南樞王蕭孝友砦,擒其鶻突姑駙馬,興宗從數騎走,元昊縱其去。



 元昊五月五日生,國人以其日相慶賀,又以四孟朔為節。凡五娶,一曰大遼興平公主,二曰宣穆惠文皇
 后沒藏氏,生諒祚,三曰憲成皇后野力氏,四曰妃沒𠼪氏,五曰索氏。元昊以慶曆八年正月殂,年四十六。在位十七年,改元開運一年,廣運二年,大慶二年,天授禮法延祚十一年。諡曰武烈皇帝,廟號景宗,墓號泰陵。宋遣開封府判官、尚書祠部員外郎曹穎叔為祭奠使,六宅使、達州刺史鄧保信為吊慰使,賜絹一千匹、布五百端、羊百口、面米各百石、酒百瓶。及葬,仍賜絹一千五百匹,余如初賻。子諒祚立。



 諒祚,景宗長子也,小字甯令
 哥,國語謂「歡嘉」為「寧令」。兩岔,河名也,母曰宣穆惠文皇后沒藏氏,從元昊出獵,至此而生諒祚,遂名焉。以慶曆七年丁亥二月六日生,八年戊子正月,方期歲即位。四月,遣尚書刑部員外郎任顓充冊禮使,供備庫副使宋守約充副使,冊諒祚為夏國主。



 嘉祐元年,母沒藏氏薨,遣祖儒嵬多、聿則慶唐及徐舜卿來告哀,詔以集賢校理馮浩假尚書刑部郎中、直史館為吊慰使,文思副使張惟清假文思使副之,乃獻遺留馬駝以謝。



 諒祚幼養
 于母族訛龐,訛龐因專國政。初,麟州西城枕睥睨曰紅樓,下瞰屈野河,其外距夏境尚七十里,而田腴利厚,多入訛龐,歲東侵不已。至耕獲時,輒屯兵河西,經略使龐籍每戒邊將使毋得過屈野河,然所距屈野河猶二十里。管勾軍馬司賈逵徼循,見所侵田,稍過督邊吏,麟州守王亮懼,始以事聞。詔以殿直張安世、賈恩為同巡檢經制之。訛龐晏然弗革,迫之則格鬥,緩之則歸耕,經略司遣使還所侵田,訛龐專為濫言,無歸意。



 嘉祐二
 年,遂團兵宿境上。逮三月,增至數萬人,守將斂兵弗與戰。知麟州武戡築堡于河西,以為保障。役既興,戡率將吏往按視,遇夏人于沙鼠浪,戡與管勾郭恩等欲止,而走馬承受黃道元以言脅之,遂夜進至臥牛峰,見烽舉,且鼓聲,道元猶不信,比明,至忽里堆,與夏人相去才數十步,遂合戰。自旦至食時,夏人四面合擊,衆大潰,戡走,恩與道元及兵馬監押劉慶等被執。安撫司遣李思道、孫兆往議疆事,而訛龐驁不聽。久之,太原府、代州兵馬鈐轄
 蘇安靜得夏國呂寧、拽浪撩黎來合議,乃築堠九,更新邊禁,要以違約則罷和市,自此始定。諒祚忌訛龐專,或告訛龐將叛,諒祚討殺之,夷其族。已而請去蕃禮,從漢儀。



 嘉祐六年,上書自言慕中國衣冠,明年當以此迎使者。詔許之。明年,又改西壽監軍司為保泰軍,石州監軍司為靜塞軍,韋州監軍司為祥祐軍,左廂監軍司為神勇軍。遣人獻方物,稱宣徽南院使,詔諭非陪臣所宜稱,戒其僭擬,使遵誓詔。表求太宗御制詩章隸書石本,且
 進馬五十匹,求《九經》、《唐史》、《冊府元龜》及宋正至朝賀儀,詔賜《九經》,還所獻馬。



 治平初,求復榷場,不許。既而遣吳宗等來賀英宗即位,詔令門見,使者不從。至順天門,且欲佩魚及儀物自從,引伴高宜禁之,不可,留止廄置一夕,絕其供饋。宗語不遜,宜折之,使如故事,良久,乃聽入。及賜食殿門,又訴于押伴張覲,詔命還赴延州與宜辨。宗度理屈,不復置對。遂詔諒祚懲約之。秋,夏人出兵秦鳳、涇原,抄熟戶,擾邊塞弓箭手,殺掠人畜以萬計。程戡、
 王素、孫長卿諭安諸族首領,防誘脅散叛。遣文思副使王無忌齎詔問之,諒祚遷延弗受,已而因賀正使荔茂先獻表,歸罪宋邊吏。



 三年,遂大舉攻大順城,分兵圍柔遠砦,燒屈乞村,柵段木嶺,州兵、熟戶、蕃官趙明合擊退之。遣西京左藏庫副使何次公詰之。三月,乃獻方物謝罪,賜絹五百匹、銀五百兩。



 神宗即位,乃遣內殿崇班魏璪賜以治平三年冬服、銀絹。供備庫副使高遵裕告哀,並以英宗遣留物賜之。秋,夏國遣使奉慰及進助山陵。
 冬,種諤取綏州,因發兵夜掩嵬名山帳,脅降之。諒祚乃詐為會議,誘知保安軍楊定、都巡檢侍其臻等殺之,邊吏以聞,命韓琦知永興軍,經略西方。諒祚錮送殺定者六宅使李崇貴、右侍禁韓道善及虜去定子仲通。



 十二月,諒祚殂,年二十一。在位二十年,改元延嗣甯國一年,天祐垂聖三年,福聖承道四年,奲都六年,拱化五年。謚曰昭英皇帝,廟號毅宗,墓號安陵。子秉常立。



\end{pinyinscope}