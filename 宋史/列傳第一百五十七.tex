\article{列傳第一百五十七}

\begin{pinyinscope}

 ○餘端禮李璧丘倪思宇文紹節李蘩



 餘端禮,字處恭,衢州龍游人。第進士,知湖州烏程縣。民間賦丁絹錢,率三氓出一縑,不輸絹而折其估,一縑千
 錢,後增至五千,民不勝病。端禮以告於府,事得上聞,又自詣中書陳便宜,歲蠲緡錢六萬。



 召對,時孝宗志在恢復,端禮言:



 謀敵決勝之道,有聲有實。敵弱者先聲後實,以讋其氣;敵強者先實後聲,以俟其機。漢武乘匈奴之困,親行邊陲,威震朔方,而漠南無王庭者,讋其氣而服之,所謂先聲而後實也。越謀吳則不然,外講盟好,內修武備,陽行成以種、蠡,陰結援於齊、晉,教習之士益精,而獻遺之禮益密,用能一戰而霸者,伺其機而圖之,所謂
 先實而後聲也。今日之事異於漢而與越相若。願陰設其備,而密為之謀,觀變察時,則機可投矣。



 古之投機者有四:有投隙之機,有搗虛之機,有乘亂之機,有承弊之機。因其內釁而擊之,若匈奴困於三國之攻而宣帝出師,此投隙之機也。因其外患而伐之,若夫差牽於黃池之役而越兵入吳,此搗虛之機也。敵國不道,因其離而舉之,若晉之降孫皓,此乘亂之機也。敵人勢窮,躡其後而蹙之,若高祖之追項羽,此乘弊之機也。機之未至,不
 可以先;機之已至,不可以後。以此備邊,安若太山,以此應敵,動如破竹,惟所欲為,無不如志。



 上喜曰:「卿可謂通事體矣。」後以薦為監察御史,遷大理少卿,轉太常少卿。



 詔以來歲祈谷上帝,仲春躬耕籍田,令禮官討論明道故事。端禮言:「祈穀之制,合祭天地於圜丘,前期享於太廟,視冬至郊祀之儀,此國朝故事也。若乃明道之制,則以宮中火後考室落成,故於太安殿恭謝天地,此特一時謝災之事耳。今欲祈穀而耕籍,必合祭天地於圜丘,
 必前期朝享於景靈宮、太廟可也。欲如明道之制,行於殿庭不可。」詔太常、禮部集議。中書有可以義起者,端禮曰:「禮固有可義起,至於大體,則不可易。古者郊而後耕,以其於郊,故謂之郊,猶祀於明堂,故謂之明堂。如明道謝災之制,則與祈穀異。今以郊而施之殿庭,亦將以明堂而施之壇壝乎?禮之失自端禮始,端禮死不敢奉詔。」上為之止。



 權兵部侍郎兼太子詹事,進吏部侍郎,出知太平州,奉祠。光宗立,召見,言:「天子之孝不與庶人同。今
 陛下之孝於壽皇,當如舜之於堯,行其道可也,武之於文,繼其志、述其事可也。凡壽皇睿謀聖訓,仁政善教,所嘗施於天下者,願與二三大臣朝夕講求而力行之,則足以盡事親之孝矣。」授集英殿修撰、知贛州,還為吏部侍郎、權刑部尚書兼侍講,以煥章閣直學士知建康府。召拜吏部尚書,擢同知樞密院事。



 興州帥吳挺死,端禮謂樞密趙汝愚曰:「吳氏世握蜀兵,今若復令承襲,將為後患。」汝愚是其言,合辭以奏,光宗意未決,端禮言:「汝愚
 所請為蜀計,為東南計。夫置大將而非其人,是無蜀也,無蜀,是無東南也。今軍中請帥而遲遲不報,人將生心。」不聽。後挺子曦卒以蜀叛,如端禮言。



 上以疾不朝重華宮,孝宗崩,又不能發喪,人情恟然。端禮謂宰相留正曰:「公獨不見唐肅宗朝群臣發哀太極殿故事乎?宜請太皇太后代行祭之禮。」於是宰執以請於太皇太后,留正懼,入臨重華宮,僕地致仕而去。



 太皇太后垂簾,策皇子嘉王即皇帝位,王流涕遜避。端禮奏:「太上違豫,大喪
 乏主,安危之機在於呼吸,太皇太后非為陛下計,乃為太上皇帝計,為宗社計。今堅持退讓,不思國家之大計,是守匹夫之小節而昧天子之大孝也。」寧宗TM然收淚,不得已,側身就御坐之半。端禮與汝愚再拜固請,寧宗乃正御坐,退行禫祭禮。



 進端禮知樞密院事兼參知政事。汝愚去右丞相位,端禮代之。始,端禮與汝愚同心共政,汝愚嘗曰:「士論未一,非餘處恭不能任。」及韓侂胄以傳道之勞,浸竊威柄,汝愚等欲疏斥之,謀洩而汝愚逐。
 端禮不能遏,但長籲而已。



 浙西常平黃灝以放民租竄,知婺州黃度以庇蜀吏褫職罷郡,二人皆侂胄所憾,端禮執奏,竟不免於罪。太府丞呂祖儉坐上書忤侂胄南遷,端禮救解不獲,公議始歸責焉。他日見上,言除從官中書不知,朝綱已紊,禍根已滋。即丐去,不許,進左丞相。



 端禮在相位期年,頗知擁護善類,然為侂胄所制,壹鬱不愜志,稱疾求退,以觀文殿大學士提舉洞霄宮。居頃之,判潭州,移慶元,復帥潭。薨,授少保、郇國公致仕,贈太
 傅,謚忠肅。子嶸,工部尚書。



 李璧字季章,眉之丹稜人。父燾,典國史。壁少英悟,日誦萬餘言,屬辭精博,周必大見其文,異之曰:「此謫仙才也。」孝宗嘗問燾:「卿諸子孰可用?」燾以璧對。以父任入官,後登進士第。召試,為正字。



 寧宗即位,徙著作佐郎兼刑部郎、權禮部侍郎兼直學士院。時韓侂胄專國,建議恢復,宰相陳自強請以侂胄平章國事,遂召璧草制,同禮部尚書蕭達討論典禮,命侂胄三日一朝,序班丞相上。



 璧
 受命使金,行次揚州,忠義人朱裕挾宋師襲漣水,金人憤甚,璧乞梟裕首境上,詔從其請。璧至燕,與金人言,披露肝膽,金人之疑頓釋。璧歸,侂胄用師意方銳,璧言:「進取之機,當重發而必至,毋輕出而茍沮。」既而陳景俊使北還,贊舉兵甚力,錢象祖以沮兵議忤侂胄得罪貶,璧論襄陽形勢,深以腹心為憂,欲待敵先發,然後應之,侂胄意不懌,於是四川、荊、淮各建宣撫而師出矣。



 璧度力不能回,乃入奏:「自秦檜首倡和議,使父兄百世之仇不
 復開於臣子之口。今廟謀未定,士氣積衰,茍非激昂,曷克丕應。臣愚以為宜亟貶秦檜,示天下以仇恥必復之志,則宏綱舉而國論明,流俗變而人心一,君臣上下奮勵振作,拯潰民於殘虐,湔祖宗之宿憤。在今日舉而措之,無難矣。」疏奏,秦檜坐追王爵。議者謂璧不論檜之無君而但指其主和,其言雖公,特以迎合侂胄用兵之私而已。



 初,侂胄召葉適直學士院,草出師詔,適不從,乃以屬璧,由是進權禮部尚書。侂胄既喪師,始覺為蘇師旦
 所誤,一夕招璧飲,酒酣,及師旦事,璧微擿其過,覘侂胄意向,乃極言:「師旦怙勢招權,使明公負謗,非竄謫此人,不足以謝天下。」師旦坐貶官。璧又言:「郭倬、李汝翼僨軍誤國之罪,宜誅之以謝淮民。」拜參知政事。



 金遣使來,微示欲和意,丘崈以聞,璧貽崈書,俾遣小使致書金帥求成,金帥報書以用兵首謀指侂胄,侂胄大恚,不復以和為意。璧言:「張浚以討賊復仇為己任,隆興之初,事勢未集,亦權宜就和。茍利社稷,固難執一。」侂胄不聽,以張巖
 代崈,璧力爭,言丘崈素有人望,侂胄變色曰:「方今天下獨有一丘崈邪!」



 吳曦叛,據蜀稱王,楊巨源、安丙誅之。事聞,璧議須用重臣宣撫,薦制置使楊輔為宣撫使,而使安丙輔之。丙殺楊巨源,輔恐召變,以書舉劉甲自代,侂胄疑輔避事,璧曰:「孝宗聞吳璘病,亟詔汪應辰權宣撫使職事,蜀賴以安,此故事也。」於是命甲權宣撫使。



 方信孺使北歸,言金人欲縛送侂胄,故侂胄忿甚,用兵之意益急。璧方與共政,或勸其速去,毋與侂胄分禍,璧曰:「嘻,
 國病矣,我去誰適謀此?」會禮部侍郎史彌遠謀誅侂胄,以密旨告璧及錢象祖,象祖欲奏審,璧言事留恐洩,侂胄迄誅,璧兼同知樞密院事。御史葉時論璧反復詭譎,削三秩,謫居撫州。後輔臣言誅侂胄事,璧實預聞,乃令自便。復官提舉洞霄宮,久之,復以御史奏削三秩,罷祠。



 越四年,復除端明殿學士、知遂寧府,未至,而潰兵張福入益昌,戕王人,略閬剽果,至遂寧,璧傳檄諭之,福等讀檄泣下,約解甲降。會官軍至挑賊,賊忿,盡燔其城,顧府
 治曰:「李公旦夕來居,此其勿毀。」璧馳書大將張威,使調嘉定黎雅砦丁、牌手來會戰,威夜遣人叩門,來言曰:「賊壘堅不可破,將選死士,梯而登,以火攻之。」璧曰:「審爾,必多殺士卒,盍先斷賊汲路與糧道,使不得食,即自成擒矣。」以長圍法授之,威用其謀,賊遂平。



 璧尋引疾奉祠。嘉定十五年六月卒,進資政殿學士致仕,謚文懿。



 璧嗜學如饑渴,群經百氏搜抉靡遺,於典章制度尤綜練。為文雋逸,所著有《雁湖集》一百卷、《涓塵錄》三卷、《中興戰功錄》
 三卷、《中興奏議》若干卷、內外制二十卷、《援毫錄》八十卷、《臨汝閑書》百五十卷。璧父子與弟𡌴皆以文學知名,蜀人比之三蘇云。



 丘崈字宗卿,江陰軍人。隆興元年進士,為建康府觀察推官。丞相虞允文奇其才,奏除國子博士。孝宗諭允文舉自代者,允文首薦崈。有旨賜對,遂言:「恢復之志不可忘,恢復之事未易舉,宜甄拔實才,責以內治,遵養十年,乃可議北向。」



 時方遣範成大使金,祈請陵寢。崈言:「泛使
 亟遣,無益大計,徒以驕敵。」孝宗不樂,曰:「卿家墳墓為人所據,亦須理索否?」崈對曰:「臣但能訴之,不能請之。」孝宗怒,崈退待罪,孝宗察其忠,不譴也。



 遷太常博士,出知秀州華亭縣。捍海堰廢且百年,咸潮歲大入,壞並海田,蘇、湖皆被其害。崈至海口,訪遺址已淪沒,乃奏創築,三月堰成,三州舄鹵復為良田。除直秘閣、知平江府,入奏內殿,因論楮幣折閱,請公私出內,並以錢會各半為定法。詔行其言,天下便之。



 知吉州,召除戶部郎中,遷樞密院
 檢詳文字。被命接伴金國賀生辰使。金歷九月晦,與《統天歷》不合,崈接使者以恩意,乃徐告以南北歷法異同,合從會慶節正日隨班上壽。金使初難之,卒屈服。孝宗喜謂崈曰:「使人聽命成禮而還,卿之力也。」



 先是,王抃為樞密,崈不少下之。方迓客時,抃排定程頓奏,上降付接伴,令沿途遵執。崈具奏,謂「不可以此啟敵疑心。」不奉詔。抃憾之,訾崈不禮金使,予祠。起知鄂州,移江西轉運判官,提點浙東刑獄,進直徽猷閣、知平江府,升龍圖閣,移
 帥紹興府,改兩浙轉運副使,以憂去。



 光宗即位,召對,除太常少卿兼權工部侍郎,進戶部侍郎,擢煥章閣直學士、四川安撫制置使兼知成都府。崈素以吳氏世掌兵為慮,陛辭,奏曰:「臣入蜀後,吳挺脫至死亡,兵權不可復付其子。臣請得便宜撫定諸軍,以俟朝命。」挺死,崈即奏:「乞選他將代之,仍置副帥,別差興州守臣,並利州西路帥司歸興元,以殺其權。挺長子曦勿令奔喪,起復知和州,屬總領楊輔就近節制諸軍,檄利路提刑楊虞仲往
 攝興州。」朝廷命張詔代挺,以李仁廣副之,遂革世將之患。其後郭杲繼詔復兼利西路安撫。杲死,韓侂胄復以兵權付曦,曦叛,識者乃服崈先見。



 進煥章閣直學士。寧宗即位,赴召,以中丞謝深甫論罷之。居數年,復職知慶元府。既入奏,韓侂胄招以見,出奏疏幾二千言示崈,蓋北伐議也,知崈平日主復仇,冀可與共功名。崈曰:「中原淪陷且百年,在我固不可一日而忘也,然兵兇戰危,若首倡非常之舉,兵交勝負未可知,則首事之禍,其誰任
 之?此必有誇誕貪進之人,攘臂以僥幸萬一,宜亟斥絕,不然必誤國矣。」



 進敷文閣學士,改知建康府。將行,侂胄曰:「此事姑為遲之。」崈因贊曰:「翻然而改,誠社稷生靈之幸,惟無搖於異議,則善矣。」侂胄聞金人置平章,宣撫河南,奏以崈為簽樞,宣撫江、淮以應之。崈手書力論「金人未必有意敗盟,中國當示大體,宜申警軍實,使吾常有勝勢。若釁自彼作,我有辭矣。」宣撫議遂寢。侂胄移書欲除崈內職,宣諭兩淮。崈報曰:「使名雖異,其為示敵人以
 嫌疑之跡則同,且偽平章宣撫既寢,尤不宜輕舉。」侂胄滋不悅。



 升寶文閣學士、刑部尚書、江淮宣撫使。時宋師克泗州,進圖宿、壽,既而師潰,侂胄遣人來議招收潰卒,且求自解之計。崈謂:「宜明蘇師旦、周筠等僨師之奸,正李汝翼、郭倬等喪師之罪。」崈欲全淮東兵力,為兩淮聲援,奏「泗州孤立,淮北所屯精兵幾二萬,萬一金人南出清河口及犯天長等城,則首尾中斷,墮敵計矣。莫若棄之,還軍盱眙。」從之。



 金人擁眾自渦口犯淮南,或勸崈棄
 廬、和州為守江計,崈曰:「棄淮則與敵共長江之險矣。吾當與淮南俱存亡。」益增兵為防。



 進端明殿學士、侍讀,尋拜簽書樞密院,督視江、淮軍馬。有自北來者韓元靖,自謂琦五世孫,崈詰所以來之故,元靖言:「兩國交兵,北朝皆謂出韓太師意,今相州宗族墳墓皆不可保,故來依太師爾。」崈使畢其說,始露講解意。崈遣人護送北歸,俾扣其實。其回也,得金行省幅紙,崈以聞於朝,遂遣王文採持書幣以行。文採還,金帥答書辭順,崈復以聞,遂遣
 陳璧充小使。璧回,具言:「金人詰使介,既欲和矣,何為出兵真州以襲我?然仍露和意也。」崈白廟堂,請自朝廷移書續前議,又謂彼既指侂飯胄為元謀,若移書,宜暫免系銜。侂胄大怒,罷崈,以知樞密院事張巖代之。既以臺論,提舉洞霄宮,落職。



 侂胄誅,以資政殿學士知建康府,尋改江、淮制置大使兼知建康府。淮南運司招輯邊民二萬,號「雄淮軍」,月廩不繼,公肆剽劫,崈乃隨「雄淮」所屯,分隸守臣節制,其西路則同轉運使張穎揀刺為御前武
 定軍,以三萬人為額,分為六軍,餘汰歸農,自是月省錢二十八萬緡,米三萬四千石。武定既成軍伍,淮西賴其力。以病丐歸,拜同知樞密院事。卒,謚忠定。



 崈儀狀魁傑,機神英悟,嘗慷慨謂人曰:「生無以報國,死願為猛將以滅敵。」其忠義性然也。



 倪思,字正甫,湖州歸安人。乾道二年進士,中博學宏詞科。累遷秘書郎,除著作郎兼翰林權直。光宗即位,典冊與尤袤對掌。故事,行三制並宣學士。上欲試思能否,一
 夕並草除公師四制,訓詞精敏,在廷誦嘆。



 權侍立修注官,直前奏:「陛下方受禪,金主亦新立,欲制其命,必每事有以勝之,彼奢則以儉勝之,彼暴則以仁勝之,彼怠惰則以憂勤勝之。」又請增置諫官,專責以諫事。又乞召內外諸將訪問,以知其才否。



 遷將作少監兼權直學士院,兼權中書舍人,升中書舍人兼直學士院、同修國史,尋兼侍講。



 初,孝宗以戶部經費之餘,則於三省置封樁庫以待軍用,至紹熙移用始頻。會有詔發緡錢十五萬入
 內帑備犒軍,思謂實給他費,請毋發,且曰:「往歲所入,約四百六十四萬緡,所出之錢不及二萬,非痛加撙節,則封樁自此無儲。遂定議犒軍歲以四十萬緡為額,由是費用有節。又言:「唐制使諫官隨宰相入閣,今諫官月一對耳,乞許同宰執宣引,庶得從容論奏。」上稱善,除禮部侍郎。



 上久不過重華宮,思疏十上,言多痛切。會上召嘉王,思言:「壽皇欲見陛下,亦猶陛下之於嘉王也。」上為動容。時李皇后浸預政,思進講姜氏會齊侯於濼,因奏:「人
 主治國必自齊家始,家之不能齊者,不能防其漸也。始於褻狎,終於恣橫,卒至於陰陽易位,內外無別,甚則離間父子。漢之呂氏,唐之武、韋,幾至亂亡,不但魯莊公也。」上悚然。趙汝愚同侍經筵,退語人曰:「讜直如此,吾黨不逮也。」



 兼權吏部侍郎,出知紹興府。寧宗即位,改婺州,未上,提舉太平興國宮,召除吏部侍郎兼直學士院。御史姚愈劾思,出知太平州,歷知泉州,建寧府,皆以言者論去。久之,召還,試禮部侍郎兼直學士院。侂胄先以書
 致殷勤,曰:「國事如此,一世人望,豈宜專以潔己為賢哉?」思報曰:「但恐方拙,不能徇時好耳。」



 時赴召者,未引對先謁侂胄,或勸用近例,思曰:「私門不可登,矧未見君乎?」逮入見,首論言路不通:「自呂祖儉謫徙而朝士不敢輸忠,自呂祖泰編竄而布衣不敢極說。膠庠之士欲有吐露,恐之以去籍,諭之以呈槁,誰肯披肝瀝膽,觸冒威尊?近者北伐之舉,僅有一二人言其不可,如使未舉之前,相繼力爭之,更加詳審,不致輕動。」又言:「蘇師旦贓以巨萬計,
 胡不黥戮以謝三軍?皇甫斌喪師襄漢,李爽敗績淮甸,秦世輔潰散蜀道,皆罪大罰輕。」又言:「士大夫寡廉鮮恥,列拜於勢要之門,甚者匍匐門竇,稱門生不足,稱恩坐、恩主甚至於恩父者,諛文豐賂,又在所不論也。」侂胄聞之大怒。



 思既退,謂侂胄曰:「公明有餘而聰不足:堂中剖決如流,此明有餘;為蘇師旦蒙蔽,此聰不足也。周筠與師旦並為奸利,師旦已敗,筠尚在,人言平章騎虎不下之勢,此李林甫、楊國忠晚節也。」侂胄悚然曰:「聞所未聞!」



 司諫毛憲劾思,予祠。侂胄殛,復召,首對,乞用淳熙例,令太子開議事堂,閑習機政。又言:「侂胄擅命,凡事取內批特旨,當以為戒。」



 除權兵部尚書兼侍讀。求對,言:「大權方歸,所當防微,一有干預端倪,必且仍蹈覆轍。厥今有更化之名,無更化之實。今侂胄既誅,而國人之言猶有未靖者,蓋以樞臣猶兼宮賓,不時宣召,宰執當同班同對,樞臣亦當遠權,以息外議。」樞臣,謂史彌遠也。金人求侂胄函首,命廷臣集議,思謂有傷國體。徙禮部尚書。



 史彌
 遠擬除兩從官,參政錢象祖不與聞。思言:「奏擬除目,宰執當同進,比專聽侂胄,權有所偏,覆轍可鑒。」既而史彌遠上章自辨,思求去,上留之。思乞對,言:「前日論樞臣獨班,恐蹈往轍,宗社堪再壞耶?宜親擢臺諫,以革權臣之弊,並任宰輔,以鑒專擅之失。」彌遠懷恚,思請去益力,以寶謨閣直學士知鎮江府,移福州。



 彌遠拜右丞相,陳晦草制用「昆命元龜」語,思嘆曰:「董賢為大司馬,冊文有『允執厥中』一言,蕭咸以為堯禪舜之文,長老見之,莫不心
 懼。今制詞所引,此舜、禹揖遜也。天下有如蕭咸者讀之,得不大駭乎?」仍上省牘,請貼改麻制。詔下分析,彌遠遂除晦殿中侍御史,即劾思藩臣僭論麻制,鐫職而罷,自是不復起矣。



 久之,除寶文閣學士,提舉嵩山崇福宮。嘉定十三年卒,謚文節。



 宇文紹節,字挺臣,成都廣都人。祖虛中,簽書樞密院事。父師瑗,顯謨閣待制。父子皆以使北死,無子,孝宗愍之,命其族子紹節為之後,補官仕州縣。九年,第進士。累遷
 寶謨閣待制、知廬州。



 時侂胄方議用兵,紹節至郡,議修築古城,創造砦柵,專為固圉計。淮西轉運判官鄧友龍譖於侂胄,謂紹節但為城守,徒耗財力,無益於事。侂胄以書讓紹節,紹節復書謂:「公有復仇之志,而無復仇之略;有開邊之害,而無開邊之利。不量國力,浪為進取計,非所敢知。」侂胄得書不樂,乃以李爽代紹節,召還,為兵部侍郎兼中書舍人兼直學士院,以寶文閣待制知鎮江府。



 吳曦據蜀,趣紹節赴闕,任以西討之事。紹節至,謂大
 臣曰:「今進攻,則瞿唐一關,彼必固守;若駐軍荊南,徒損威望。聞隨軍轉運安丙者素懷忠義,若授以密旨,必能討賊成功。」大臣用其言,遣丙所親以帛書達上意,丙卒誅曦。



 權兵部尚書,未幾,除華文閣學士、湖北京西宣撫使、知江陵府。統制官高悅在戍所,肆為殺掠,遠近苦之。紹節召置帳前,收其部曲。俄有訴悅縱所部為寇者,紹節杖殺之,兵民皆歡。升寶文閣學士,試吏部尚書,尋除端明殿學士、簽書樞密院事。



 安丙宣撫四川,或言丙有
 異志,語聞,廷臣欲易丙。紹節曰:「方誅曦初,安丙一搖足,全蜀非國家有,顧不以此時為利,今乃有他耶?紹節願以百口保丙。」丙卒不易。朝廷於蜀事多所咨訪,紹節審而後言,皆周悉事情。



 嘉定六年正月甲午卒,訃聞,上嗟悼,為改日朝享。進資政殿學士致仕,又贈七官為少師,非常典也。謚曰忠惠。



 李蘩,字清叔,崇慶晉原人。第進士,為隆州判官,攝綿州。歲昆,出義倉穀賤糶之,而以錢貸下戶,又聽民以茅秸
 易米,作粥及褚衣,親衣食之,活十萬人。明年又饑,邛蜀彭漢、成都盜賊蜂起,綿獨按堵。知永康軍,移利州,提點成都路刑獄兼提舉常平。歲兇,先事發廩蠲租,所活百七十萬人。知興元府、安撫利州東路。



 漢中久饑,劍外和糴在州者獨多,蘩嘗匹馬行阡陌間訪求民瘼,有老嫗進曰:「民所以饑者,和糴病之也。」泣數行下。蘩感其言,奏免之,民大悅。徙倉部員外郎,總領四川賦財、軍馬、錢糧,升郎中。



 淳熙三年,廷臣上言:「四川歲糴軍糧,名為和糴,
 實科糴也。」詔制置使範成大同蘩相度以聞,蘩奏:「諸州歲糴六十萬石,若從官糴,歲約百萬緡,如於經費之中斟酌損益,變科糴為官糴,貴賤眂時,不使虧毫忽之價;出納眂量,勿務取圭撮之贏,則軍不乏興,民不加賦。」乃書「利民十一事」上之。前後凡三年,蘩上奏疏者十有三,而天子降詔難問者凡八,訖如其議。民既樂與官為市,遠邇歡趨,軍餉坐給,而田里免科糴,始知有生之樂。會歲大稔,米價頓賤,父老以為三十年所無。梁、洋間繪蘩
 像祠之。



 範成大驛疏言:「關外麥熟,倍於常年,實由罷糴,民力稍紓,得以盡於農畝。」孝宗覽之曰:「免和糴一年,田間和氣若此,乃知民力不可重困也。」擢蘩守太府少卿。範成大召見,孝宗首問:「糴事可久行否?」成大奏:「李蘩以身任此事,臣以身保李蘩。」孝宗大悅,曰:「是大不可得李蘩也。」上意方向用,而蘩亦欲奏蠲鹽酒和買之弊,以盡滌民害。會有疾,卒。詔以蘩能官,致仕恩外特與遣表,擇一人庶官,前此所未有。



 初,蘩宰眉山,校成都漕試,念吳
 氏世襲兵柄必稔蜀亂,發策云:「久假人以兵柄,未有不為患者。以武、宣之明,不能銷大臣握兵之禍;以憲、武之烈,不能收藩鎮握兵之權。危劉氏、殲唐室,鮮不由此。」吳挺以為怨。後蘩總餉事,挺謬奏軍食粗惡,孝宗以問蘩,蘩緘其樣以進,挺之妄遂窮。逾三十年,吳曦竟以蜀叛,安丙既誅曦,每語人云:「吾等焦頭爛額耳,孰如李公先見者乎?」蘩講學臨政皆有源委,所著書十八種,有《桃溪集》一百卷。



 論曰:餘端禮平時論議剴正,及為相,受制於韓侂胄,雖有志扶掖善類,而不得以直,遂頗不免君子之論。若李壁、丘崈皆諫侂胄以輕兵召釁之失,及其決意用師,命葉適草詔不從,而壁獨當筆焉,何其所見後先舛迕哉!附會之罪,壁固無以逭於公論矣。倪思直辭劘主,又屢觸權臣,三黜不變其風概,有可尚焉。李蘩所至能舉荒政,蠲苛賦,亦庶幾古所謂惠人也。



\end{pinyinscope}