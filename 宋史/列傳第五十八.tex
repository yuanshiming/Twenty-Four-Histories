\article{列傳第五十八}

\begin{pinyinscope}

 狄棐子遵度郎簡孫祖德張若穀石揚休祖士衡李垂張洞李仕衡李溥胡則薛顏許元鐘離瑾孫沖崔嶧田瑜施昌言



 狄棐,字輔之,潭州長沙人。少隨父官徐州,以文謁路振,振器愛之,妻以女。舉進士甲科,以大理評事知分宜縣。歷開封府司錄,知壁州。道長安,為寇準所厚,準復入相,乃薦通判益州。擢開封府判官,歷京西益州路轉運、江淮制置發運使,累遷太常少卿、知廣州,加直昭文館。代還,不以南海物自隨,人稱其廉。拜右諫議大夫、龍圖閣直學士、權判吏部流內銓,出知滑州,進給事中,徙天雄軍。會給郊賞帛不善,士卒嘩噪趣府門,棐不能治。事聞,
 命侍御史劉夔按視,未及境,眾不自安。棐馳白夔,請紿以行河事。夔至,與轉運使李絳誅首惡數人。棐坐罷懦,降知隨州,徙同州。勾當三班院,進樞密直學士,歷知陜鄭州、河中河南府,復判流內銓。出知揚州,未行,卒。



 有狄國賓者,仁傑之後,分仁傑告身與棐,棐奏錄國賓一官,而自稱仁傑十四世孫。棐在河中時,有中貴人過郡,言將援棐於上前。棐答以他語,退謂所親曰:「吾湘潭一寒士,今官侍從,可以老而自污耶?」其為政愷悌,不為表襮,
 死之日,家無餘貲。



 子遵度,字符規。少穎悟,篤志於學。每讀書,意有所得,即仰屋瞪視,人呼之,弗聞也。少舉進士,一斥於有司,恥不復為。以父任為襄縣主簿,居數月,棄去。好為古文,著《春秋雜說》,多所發明。嘗患時學靡敝,作《擬皇太子冊文》、《除侍御史制》、《裴晉公傳》,人多稱之。尤嗜杜甫詩,賞贊其集。一夕,夢見甫為誦世所未見詩,及覺,才記十餘字,遵度足成之,為《佳城篇》。後數月卒。有集十二卷。



 郎簡,字叔廉,杭州臨安人。幼孤貧,借書錄之,多至成誦。進士及第,補試秘書省校書郎、知寧國縣,徙福清令。縣有石塘陂,歲久湮塞,募民浚築,溉廢田百餘頃,邑人為立生祠。調隨州推官。及引對,真宗曰:「簡歷官無過,而無一人薦,是必恬於進者。」特改秘書省著作佐郎、知分宜縣,徙知竇州。縣吏死,子幼,贅婿偽為券冒有其貲。及子長,屢訴不得直,乃訟於朝。下簡劾治,簡示以舊牘曰:「此爾翁書耶?」曰:「然。」又取偽券示之,弗類也,始伏罪。



 徙藤州,
 興學養士,一變其俗,藤自是始有舉進士者。通判海州,提點利州路刑獄。官罷,知泉州。累遷尚書度支員外郎、廣南東路轉運使,擢秘書少監、知廣州,捕斬賊馮佐臣。入判大理寺,出知越州,復歸判尚書刑部,出知江寧府,歷右諫議大夫、給事中、知揚州,徙明州。以尚書工部侍郎致仕。祀明堂,遷刑部。卒,年八十有九,特贈吏部侍郎。



 簡性和易,喜賓客。即錢塘城北治園廬,自號武林居士。道引服餌,晚歲顏如丹。尤好醫術,人有疾,多自處方以
 療之,有集驗方數十,行於世。一日,謂其子潔曰:「吾退居十五年,未嘗小不懌,今意倦,豈不逝歟?」就寢而絕。幼從學四明朱□,長學文於沉天錫,既仕,均奉資之。後二人亡,又訪其子孫,為主婚嫁。平居宴語,惟以宣上德、救民患為意。孫沔知杭州,榜其里門曰德壽坊。然在廣州無廉稱,蓋為潔所累。潔,終尚書都官員外郎。



 孫祖德,字延仲,濰州北海人。父航,監察御史、淮南轉運。祖德進士及第,調濠州推官、校勘館閣書籍。時校勘官
 不為常職,滿歲而去。改大理寺丞、知榆次縣,上書言刑法重輕。以尚書屯田員外郎通判西京留守司。方冬苦寒,詔罷內外工作,而錢惟演督修天津橋,格詔不下。祖德曰:「詔書可稽留耶?」卒白罷役。



 入為殿中侍御史,遷侍御史。章獻太后春秋高,疾加劇,祖德請還政。已而疾少間,祖德大恐。及太后崩,諸嘗言還政者多進用,遂擢尚書兵部員外郎兼起居舍人、知諫院。言郭皇后不當廢,獲罪,以贖論。久之,遷天章閣待制。



 時三司判官許申因
 宦官閻文應獻計,以藥化鐵成銅,可鑄錢,裨國用。祖德言:「偽銅,法所禁而官自為,是教民欺也。」固爭之,出知兗徐蔡州、永興軍。徙鳳翔府,請置鄉兵。改龍圖閣直學士、知梓州,累遷右諫議大夫、知河中府。歷陳許蔡潞鄆亳州、應天府,以疾得穎州,除吏部侍郎致仕,卒。有《論事》七卷。



 祖德少清約,及致仕,娶富人妻,以規有其財。已而妻悍,反資以財而出之。子珪,江東轉運使。



 張若谷,字德繇,南劍沙縣人。進士及第,為巴州軍事推
 官。會蜀寇掠鄰郡,若谷攝州事,率眾為守禦備,賊乃引去。調全州軍事推官。入見,真宗識其名,顧曰:「是嘗在巴州御賊者耶?」特改大理寺丞、知蒙陽縣。三司言:「廣寧監歲鑄緡錢四十萬,其主監宜擇人。」乃以命若谷。歲餘,所鑄贏三十萬緡。擢知處州,歷江湖淮南益州路轉運、江淮制置發運使。入為三司度支、鹽鐵副使,累遷右諫議大夫、知並州。



 先是,麟、府歲以繒錦市蕃部馬,前守輒罷之。若谷以謂:互市,所以利戎落而通邊情,且中國得戰
 馬;亟罷之,則猜阻不安。奏復市如故,而馬入歲增。提舉諸司庫務,權判大理寺,進樞密直學士,歷知澶州、成德軍、揚州、江寧府,入知審官院,糾察在京刑獄,知通進銀臺司、應天府。改龍圖閣學士,徙杭州。會歲饑,斥餘廩為糜粥賑救之。權判吏部流內銓、知洪州,累官至尚書左丞致仕。



 若谷素為宰相張士遜引拔,然所至亦自有循良跡,不激訐取名云。



 石揚休,字昌言,其先江都人。唐兵部郎中仲覽之後,後
 徙京兆。七代祖藏用,右羽林大將軍,明於歷數,嘗召家人謂曰:「天下將有變,而蜀為最安處。」乃去依其親眉州刺史李滈,遂為眉州人。



 揚休少孤力學,進士高第,為同州觀察推官,遷著作佐郎、知中牟縣。縣當國西門,衣冠往來之沖也,地瘠民貧,賦役煩重,富人隸太常為樂工,僥幸免役者凡六十餘家。揚休請悉罷之。改秘書丞,為秘閣校理、開封府推官,累遷尚書祠部員外郎,歷三司度支、鹽鐵判官。坐前在開封嘗失盜,出知宿州。



 頃之,召
 入為度支判官,修起居注。初,記注官與講讀諸儒,皆得侍坐邇英閣。揚休奏:「史官記言動,當立以侍。」從其言。判鹽鐵勾院,以刑部員外郎知制誥、同判太常寺。初,內出香祠溫成廟,帝誤書名稱臣,揚休言:「此奉宗廟禮,有司承誤不以聞。」帝嘉之。兼勾當三班院,為宗正寺修玉牒官。遷工部郎中,未及謝,卒。



 揚休喜閑放,平居養猿鶴,玩圖書,吟詠自適,與家人言,未嘗及朝廷事。及卒,發楮中所得上封事十餘章,其大略:請增諫官以廣言路,置五
 經博士使學者專其業,出御史按察諸道以防壅蔽,復齒冑之禮以強宗室,擇守令,重農桑,禁奢侈,皆有補於時者。然揚休為人慎默,世未嘗以能言待之也。至於誥命,尤非所長。



 平生好殖財。因使契丹,道感寒毒,得風痺,謁告歸鄉,別墳墓。揚休初在鄉時,衣食不足,徙步去家十八年。後以從官還鄉里,疇昔同貧窶之人尚在,皆曰:「昌言來,必賙我矣。」揚休卒不揮一金,反遍受里中富人金以去。



 祖士衡,字平叔,蔡州上蔡人。少孤,博學有文,為李宗諤所知,妻以兄子。楊億謂劉筠曰:「祖士衡辭學日新,後生可畏也。」舉進士甲科,授大理評事、通判蘄州,再遷殿中丞、直集賢院,改右正言、戶部判官。未幾,提舉在京諸司庫務,遷起居舍人、注釋御集檢閱官,遂知制誥,為史館修撰,糾察在京刑獄,同知通進、銀臺司。天聖初,以附丁謂,落職知吉州。言者又以在郡不修飭,復降監江州稅。士衡兒時過外家,有僧善相,見之,語人曰:「是兒神骨秀
 異,他日有名於時,若年過四十,當位極人臣。」年三十九,卒於官。



 李垂,字舜工,聊城人。咸平中,登進士第,上《兵制》、《將制書》。自湖州錄事參軍召為崇文校勘,累遷著作郎、館閣校理。上《導河形勝書》三卷,欲復九河故道,時論重之。又累修起居注。丁謂執政,垂未嘗往謁。或問其故,垂曰:「謂為宰相,不以公道副天下望,而恃權怙勢。觀其所為,必游朱崖,吾不欲在其黨中。」謂聞而惡之,罷知亳州,遷穎、晉、
 絳三州。明道中,還朝,閣門祗候李康伯謂曰:「舜工文學議論稱於天下,諸公欲用為知制誥,但宰相以舜工未嘗相識,盍一往見之。」垂曰:「我若昔謁丁崖州,則乾興初已為翰林學士矣。今已老大,見大臣不公,常欲面折之,焉能趨炎附熱,看人眉睫,以冀推挽乎?道之不行,命也。」執政知之,出知均州。卒,年六十九。



 五子,仲昌最知名,銳於進取,嘗獻計修六塔河無功,自殿中丞責英州文學參軍。



 張洞,字仲通,開封祥符人。父惟簡,太常少卿。洞為人長大,眉目如畫,自幼開悟,卓犖不群。惟簡異之,抱以訪里之卜者。曰:「郎君生甚奇,必在策名,後當以文學政事顯。」既誦書,日數千言,為文甚敏。未冠,曄然有聲,遇事慷慨,自許以有為。時,趙元昊叛擾邊。關、隴蕭然,困於飛挽,且屢喪師。仁宗太息,思聞中外之謀。洞以布衣求上方略,召試舍人院,擢試將作監主簿。



 尋舉進士中第,調漣水軍判官,遭親喪去,再調穎州推官。民劉甲者,強弟柳使
 鞭其婦,既而投杖,夫婦相持而泣。甲怒,逼柳使再鞭之。婦以無罪死。吏當夫極法,知州歐陽修欲從之。洞曰:「律以教令者為首,夫為從,且非其意,不當死。」眾不聽,洞即稱疾不出,不得已讞於朝,果如洞言,修甚重之。



 晏殊知永興軍,奏管勾機宜文字。殊儒臣,喜客,游其門者皆名士,尤深敬洞。改大理丞、知鞏縣。會殊留守西京,復奏知司錄。殊晚節驟用刑,幕府無敢言。洞平居與殊賦詩飲酒,傾倒無不至,當事有官責,持議甚堅,殊為沮止,洞亦
 自以不負其知。



 樞密副使高若訥、參知政事吳育薦其文學,宜為館職,召試學士院,充秘閣校理、判祠部。時天下戶口日蕃,民去為僧者眾。洞奏:「至和元年,敕增歲度僧,舊敕諸路三百人度一人,後率百人度一人;又文武官、內臣墳墓,得置寺撥放,近歲滋廣。若以勛勞宜假之者,當依古給戶守塚,禁毋樵採而已。今祠部帳至三十餘萬僧,失不裁損,後不勝其弊。」朝廷用其言,始三分減一。知太常禮院,宰相陳執中將葬,洞與同列謚為榮靈,
 其孫訴之,詔孫抃等復議,改曰恭。洞駁奏:「執中位宰相,無功德而罪戾多,生不能正法以黜之,死猶當正名以誅之。」竟從抃等議。



 初,皇后郭氏忤旨得罪廢沒,後仁宗悔之,詔追復其號,二十餘年矣。至是,有司請祔於廟。知制誥劉敞以謂:「《春秋》書『禘於太廟,用致夫人』。致者,不宜致也。且古者不二嫡,當許其號,不許其禮。」洞奏:「後嘗母天下,無大過惡,中外所知。陛下既察其偶失恭順,洗之於既沒,猶曰不許其禮,於義無當。且廢後立後,何嫌於
 嫡?此當時大臣護已然之失,乖正名之典,而敞復引《春秋》『用致夫人』。按《左氏》哀姜之惡所不忍道,而二《傳》有非嫡之辭,敞議非是。若從變禮,尚當別立廟。」不行。轉太常博士,判登聞鼓院。仁宗方響儒術,洞在館閣久,數有建明,仁宗以為知《經》,會覆考進士崇政殿,因賜飛白「善經」字寵之。洞獻詩謝,復賜詔獎諭。



 出知棣州,轉尚書祠部員外郎。河北地當六塔之沖者,歲決溢病民田。水退,強者遂冒占,弱者耕居無所。洞奏一切官為標給,蠲其租
 以綏新集。河北東路民富蠶桑,契丹謂之「綾絹州」,朝廷以為內地不慮。洞奏:「今滄、景,契丹可入之道,兵守多缺,契丹時以販鹽為名,舟往來境上,此不可不察。願度形勢,置帥、增屯戍以控扼之。」



 時天下久安,薦紳崇尚虛名,以寬厚沉默為德,於事無所補,洞以謂非朝廷福。又謂:「諫官持諫以震人主,不數年至顯仕,此何為者。當重其任而緩其遷,使端良之士不亟易,而浮躁者絕意。」致書歐陽修極論之。召權開封府推官。



 英宗即位,轉度支員
 外郎。英宗哀疚,或經旬不御正殿,洞上言:「陛下春秋鼎盛,初嗣大統,豈宜久屈剛健,自比沖幼之主。當躬萬機,攬群材,以稱先帝付畀之意,厭元元之望。」大臣亦以為言,遂聽政。命考試開封進士,既罷,進賦,題曰《孝慈則忠》。時方議濮安懿王稱皇事,英宗曰:「張洞意諷朕。」宰相韓琦進曰:「言之者無罪,聞之者足以戒。」英宗意解。



 詔訊祁國公宗說獄,宗說恃近屬,貴驕不道,獄具,英宗以為辱國,不欲暴其惡。洞曰:「宗說罪在不宥。雖然,陛下將懲惡
 而難暴之,獨以其坑不辜數人,置諸法可矣。」英宗喜曰:「卿知大體。」洞因言:「唐宗室多賢宰相名士,蓋其知學問使然。國家本支蕃衍,無親疏一切厚廩之,不使知辛苦。婢妾聲伎,無多寡之限,至滅禮義,極嗜欲。貸之則亂公共之法,刑之則傷骨肉之愛。宜因秩品立制度,更選老成教授之。」宗室緣是怨洞,痛詆訾言,上亦起藩邸,賴察之,不罪也。



 轉司封員外郎、權三司度支判官。對便殿稱旨,英宗遂欲進用,大臣忌之,出為江西轉運使。江西薦
 饑,徵民積歲賦,洞為奏免之。又民輸綢絹不中度者,舊責以滿匹,洞命計尺寸輸錢,民便之。移淮南轉運使,轉工部郎中。淮南地不宜麥,民艱於所輸,洞復命輸錢,官為糴麥,不逾時而足。洞在棣時,夢人稱敕召者,既出,如拜官然,顧視旌旗吏卒羅於庭。至是,夢之如初。自以年不能永,教諸子部分家事。未幾卒,年四十九。



 李仕衡,字天均,秦州成紀人,後家京兆府。進士及第,調鄠縣主簿。田重進守京兆,命仕衡鞫死囚五人,活者四
 人。重進即其家謂曰:「子有陰施,此門當高大之。」徙知彭山縣,就加大理評事,遷光祿寺丞。父益,以不法誅,仕衡亦坐除名。



 後會赦,寇準薦其材,盡復其官,領渭橋輦運,通判邠州,再遷秘書丞,徙知劍州。王均反,仕衡度州兵不足守,即棄城焚芻粟,輦金帛東守劍門。既而賊陷漢州,攻劍州,州空無所資,即趨劍門。仕衡預招賊眾,得千餘人,待之不疑。賊將至,與鈐轄裴臻迎擊之,斬首數千級。乃乘驛入奏,擢尚書度支員外郎,賜服緋魚。已而使
 者言仕衡嘗棄城,降監虔州稅。



 召還,判三司鹽鐵勾院。度支使梁鼎言:「商人入粟於邊,率高其直,而售以解鹽。商利益博,國用日耗。請調丁夫轉粟,而輦鹽諸州,官自鬻之,歲可得緡錢三十萬。」仕衡曰:「安邊無大於息民,今不得已而調劍之,又增以轉粟挽鹽之役,欲其不困,何可得哉!」不聽,遂行鼎議,而關中大擾。乃罷鼎度支使,以仕衡為荊湖北路轉運使,徙陜西。初,歲出內帑緡錢三十萬,助陜西軍費。仕衡言歲計可自辦,遂罷給。



 真宗謁
 陵寢,因幸洛,仕衡獻粟五十萬斛,又以三十萬斛饋京西。朝廷以為材,召為度支副使。上言:「關右既弛鹽禁,而永興、同華耀四州猶率賣鹽,年額錢請減十之四。」詔悉除之。累遷司封郎中,為河北轉運使。又奏罷內帑所助緡錢百萬。建言:「河北歲給諸軍帛七十萬,而民艱於得錢,悉預假於里豪,出倍償之息,以是工機之利愈薄。方春民不足,請戶給錢,至夏輸帛,則民獲利而官用足矣。詔優其直,仍推其法於天下。



 封泰山,獻錢帛、芻糧各十
 萬,見於行宮,遷右諫議大夫。祀汾陰,又助錢帛三十萬,乃命同林特提舉京西、陜西轉運事。權知永興軍,進給事中。逾月,以樞密直學士知益州。



 頃之,河北闕軍儲,議者以謂仕衡前過助封祀費,真宗聞之,以為河北都轉運使。駕如亳州,又貢絲錦、縑帛各二十萬。後集粟塞下,至鉅萬斛。或言粟腐不可食,朝廷遣使取視之,而粟不腐也。棣州污下苦水患,仕衡奏徙州西北七十里,既而大水沒故城丈餘。南郊,復進錢帛八十萬。先是,每有
 大禮,仕衡必以所部供軍物為貢,言者以為不實。仕衡乃條析進六十萬皆上供者,二十萬即其羨餘。帝不之罪,謂王旦曰:「仕衡應猝有材,人欲以此中之。然朝廷所須,隨大小即辦,亦其所長也。」明年旱蝗,發積粟賑民,又移五萬斛濟京西。



 遷尚書工部侍郎、權知天雄軍。民有盜瓜傷主者,法當死,仕衡以歲饑,奏貸之。盜起淄、青間,遷刑部侍郎、知青州。前守捕群盜妻子置棘圍中,仕衡至,悉縱罷之使去。未幾,其徒有梟賊首至者。入為三司
 使,帝作《寬財利論》以賜之。乃更陜西入粟法,使民得受錢與茶。舊市羊及木,責吏送京師,而羊多道死,木至湍險處往往漂失,吏至破產不能償。仕衡乃許吏私附羊,免其算,使得補死者;聽民自採木輸官,用入粟法償其直。遷吏部侍郎。



 仁宗即位,拜尚書左丞,以足疾,改同州觀察使、知陳州。州大水,築大堤以障水患。徙穎州,復知陳州。曹利用,仕衡婿也。利用被罪,降仕衡左龍武軍大將軍,分司西京。歲餘,改左衛大將軍,卒。其後諸子訴其
 父有勞於國,非意左遷,詔追復同州觀察使。



 仕衡前後管計事二十年,雖才智過人,然素貪,家貲至累鉅萬,建大第長安里中,嚴若官府。



 子丕緒,蔭補將作監主簿。及仕衡歸老,丕緒時為尚書虞部員外郎,請解官就養。朝廷以為郎,故事不許,請削一官,乃聽。未幾,還之。居十餘年,仕衡死,服除,久之不出。大臣為言,起僉書永人軍節度判官事。歷通判永興軍、同州,知解州、興元府、華州,累遷司農卿致仕,卒。丕緒居官廉靜,不為矯激。家多圖書,
 集歷代石刻,為數百卷藏之。



 李溥,河南人。初為三司小吏,陰狡多智數。時天下新定,太宗厲精政事,嘗論及財賦,欲有所更革,引三司吏二十七人對便殿,問以職事。溥詢其目,請退而條上。命至中書,列七十一事以聞,四十四事即日行之,餘下三司議可否。於是帝以溥等為能,語輔臣曰:「朕嘗諭陳恕等,如溥輩雖無學,至於金谷利害,必能究知本末,宜假以色辭,誘令開陳。而恕等強愎自用,莫肯詢問。」呂端對曰:「
 耕當問奴,織當問婢。」寇準曰:「孔子入太廟,每事問。蓋以貴下賤,先有司之義也。」帝以為然,悉擢溥等以官,賜錢幣有差。



 溥為左侍禁、提點三司孔目官,請著內外百官諸軍奉祿為定式。加閣門祗候。催運陜西糧草,赴清遠軍,還,提舉在京倉草場,勾當北作坊。齊州大水,壞民廬舍,欲徙州城,未決,命溥往視,遂徙城而還。又與李仕衡使陜西,增酒榷緡錢歲二十五萬。三遷崇儀使。



 景德中,茶法既弊,命與林特、劉承珪更定法,募人入金帛京師,
 入芻粟塞下,與東南茶皆倍其數,即以溥制置江、淮等路茶鹽礬稅兼發運事,使推行之。歲課緡錢,果增其舊,特等皆受賞。溥時已為發運副使,遷為使,仍改西京作坊使。然茶法行之數年,課復損於舊。江、淮歲運米輸京師,舊止五百餘萬斛,至溥乃增至六百萬,而諸路獄有餘畜。高郵軍新開湖水散漫多風濤,溥令漕舟東下者還過泗州,因載石輸湖中,積為長堤,自是舟行無患。累遷北作坊使。



 時營建玉清昭應宮,溥與丁謂相表裏,盡
 括東南巧匠遣詣京,且多致奇木怪石,以傅會帝意。建安軍鑄玉皇、聖祖,溥典其事,丁謂言溥蔬食者周歲,而溥亦數奏祥應,遂以為迎奉聖像都監、領順州刺史,遷獎州團練使。溥自言江、淮歲入茶,視舊額增五百七十餘萬斤。並言,漕舟舊以使臣若軍大將,人掌一綱,多侵盜,自溥並三綱為一,以三人共主之,使更相司察。大中祥符九年,初運米一百二十五萬石,才失二百石。會溥當代,詔留再任,特遷宮苑使。



 初,譙縣尉陳齊論榷茶法,
 溥薦齊任京官,御史中丞王嗣宗方判吏部銓,言齊豪民子,不可用。真宗以問執政,馮拯對曰:「若用有材,豈限貧富。」帝曰:「卿言是也。」因稱溥畏慎小心,言事未嘗不中利害,以故任之益不疑。然溥久專利權,內倚丁謂,所言輒聽。帝嘗語執政曰:「群臣上書論事,法官輒沮之,云非有大益,無改舊章,然則何以廣言路。」王旦對曰:「法制數更,則詔令抵牾,故重於變易。」因言:「溥嘗請盜販茶鹽者贓仗皆沒官,已可之矣。」帝曰:「此特畏溥之強,不敢退卻,
 自今雖小吏言,亦宜詳究行之。」



 溥既專且貪,繇是浸為不法。發運使黃震條其罪狀以聞,罷知潭州。命御史鞫治,得溥私役兵為姻家林特起第,附官舟販竹木,奸贓十數事。未論決,會赦,貶忠武軍節度副使。仁宗即位,起知淮陽軍,歷光、黃二州,復以贓敗,貶蔡州團練副使。久之,監徐州利國監,以千牛衛將軍致仕,卒。



 胡則,字子正,婺州永康人。果敢有材氣。以進士起家,補許田縣尉,再調憲州錄事參軍。時靈、夏用兵,轉運使索
 湘命則部送芻糧,為一月計。則曰:「為百日備,尚恐不支,奈何為一月邪?」湘懼無以給,遣則遂入奏。太宗因問以邊策,對稱旨,顧左右曰:「州縣豈乏人?」命記姓名中書。後李繼隆討賊,久不解,湘語則曰:「微子幾敗我事。」一日,繼隆移文轉運司曰:「兵且深入,糧有繼乎?」則告湘曰:「彼師老將歸,欲以糧乏為辭耳,姑以有餘報之。」已而果為則所料。湘為河北轉運使,奏改秘書省著作佐郎、僉書貝州觀察判官事。



 後以太常博士提舉兩浙榷茶,就知睦
 州,徙溫州。歲餘,提舉江南路銀銅場、鑄錢監,得吏所匿銅數萬斤,吏懼且死,則曰:「馬伏波哀重囚而縱之,吾豈重貨而輕數人之生乎?」籍為羨餘,不之罪。改江、淮制置發運使,累遷尚書戶部員外郎。真宗幸亳還,擢三司度支副使。



 初,丁謂舉進士,客許田,則厚遇之,謂貴顯,故則驟進用。至是,謂罷政事,出則為京西轉運使,遷禮部郎中。部內民訛言相驚,至遣使安撫乃定。坐是,徙廣西路轉運使。有番舶遭風至瓊州,且告食乏,不能去。則命貸
 錢三百萬,吏白夷人狡詐,又風波不可期。則曰:「彼以急難投我,可拒而不與邪?」已而償所貸如期。又按宜州重闢十九人,為辨活者九人。復為發運使,累遷太常少卿。



 乾興初,坐丁謂黨,降知信州,徙福州,以右諫議大夫知杭州。入權吏部流內銓,坐失舉,復為太常少卿、知池州。未行,復諫議大夫、知永興軍,徙河北都轉運使,以給事中權三司使,通京東西、陜西鹽法,人便之。初,則在河北,殿中侍御史王沿嘗就則假官舟販鹽,又以其子為名
 祈買酒場。至是,張宗誨擿發之,按驗得實,出則知陳州。逾月,授工部侍郎、集賢院學士。劉隨上疏言:「則奸邪貪濫聞天下,比命知池州,不肯行,今以罪去,驟加美職,何以風勸在位?」後徙杭州,再遷兵部侍郎致仕,卒。



 則無廉名,喜交結,尚風義。丁謂貶崖州,賓客隨散落,獨則間遣人至海上,饋問如平日。在福州時,前守陳絳嘗延蜀人龍昌期為眾人講《易》,得錢十萬。絳既坐罪,遂自成都械昌期至。則破械館以賓禮,出俸錢為償之。



 昌期者,嘗注《
 易》、《詩》、《書》、《論語》、《孝經》、《陰符經》、《老子》,其說詭誕穿鑿,至詆斥周公。初用薦者補國子四門助教,文彥博守成都,召置府學,奏改秘書省校書郎,後以殿中丞致仕。著書百餘卷,嘉祐中,詔取其書。昌期時年八十餘,野服自詣京師,賜緋魚,絹百匹。歐陽修言其異端害道,不當推獎,奪所賜服罷歸,卒。



 薛顏字,彥回,河中萬泉人。舉《三禮》中第,為嘉州司戶參軍。代還引見,太宗顧問之,對稱旨,改將作監丞、監華州
 酒稅。以秘書省著作佐郎使夔、峽,疏決刑獄。還,改太子左贊善大夫、知雲安軍,徙渝、閬二州,擢三司鹽鐵判官,河北計置糧草。



 初,丁謂招撫溪蠻,有威惠,部人愛之。留五年,詔謂自舉代,謂薦顏為峽路轉運使,累遷尚書虞部員外郎。始,孟氏據蜀,徙夔州於東山,據峽以拒王師,而民居不便也,顏為復其故城。宜州陳進反,命勾當廣南東、西路轉運司事。賊平,遷金部員外郎,改河東轉運使。



 祀汾陰,徙陜西。河中浮橋歲為水所敗,顏即北岸釃
 上流為支渠,以殺水怒,因取渠水溉其旁田,民頗利之。坊州募人煉礬,歲久課益重,至有破產被系不能償者。顏奏:「罷坊礬,則晉礬當大售。」後如其策。徙河北。歷知河陽、杭徐州,累遷光祿少卿,以少府監知江寧府。邏者晝劫人,反執平人以告。顏視其色動,曰:「若真盜也。」械之,果引伏。轉右諫議大夫、知河南府。



 仁宗即位,遷給事中。丁謂分司西京,以顏雅與善,徙知應天府,又徙耀州。部有豪姓李甲,結客數十人,號「沒命社」,少不如意,則推一人
 以死鬥之,積數年,為鄉人患,莫敢發。顏至,大索其黨,會赦當免,特杖甲流海上,餘悉籍於軍。以光祿卿分司西京,卒於家。



 嘗屬杜衍為墓志,衍卻之。仁宗聞其事,他日,謂衍曰:「薛顏有醜行,卿不欲志其墓,誠清識也。」孫向,自有傳。



 許元,字子春,宣州宣城人。以父蔭為太廟齋郎,改大理寺丞,累遷國子博士,監在京榷貨務,三門發運判官。元為吏強敏,尤能商財利。慶歷中,江、淮歲漕不給,京師乏
 軍儲,參知政事範仲淹薦元可獨倚辦,擢江、淮制置發運判官。至,則悉發瀕江州縣藏粟,所在留三月食,遠近以次相補,引千餘艘轉漕而西。未幾,京師足食,朝廷以為任職,就遷副使。遂以尚書主客員外郎為使,進金部,特賜進士出身,遷侍御史。



 嘗欲與施昌言分行二浙、江南調發軍食。仁宗聞之,語輔臣曰:「東南歲比不登,民力匱乏,嘗詔損歲漕百萬石,而元與昌言乃更欲分道而出,是必誅求疲民以自為功,非朕志也。」下詔戒飭。既而
 元欲專六路財賦,收羨餘以媚三司,憚諸部不從,請以六路轉運司自隸,既可之矣,而轉運使多論其罪,事遂寢。擢天章閣待制,再遷郎中,以疾請還。歷知揚、越、泰州,卒。



 元在江、淮十三年,以聚斂刻剝為能,急於進取,多聚珍奇以賂遺京師權貴,尤為王堯臣所知。發運使治所在真州,衣冠之求官舟者,日數十輩。元視勢家貴族,立榷巨艦與之;即小官惸獨,伺候歲月,有不能得。人以是憤怨,而元自為以當然,無所愧憚。



 鐘離瑾,字公瑜,廬州合肥人。舉進士,為簡州推官,以殿中丞通判益州。建言:「州郡既上雨,後雖兇旱,多隱之以成前奏,請令監司劾其不實者。」擢開封府推官,出提點兩浙刑獄。衢、潤州饑,聚餓者食之,頗廢農作,請發米二萬斛賑給,家毋過一斛。後徙淮南轉運副使,歷京西、河東、河北轉運使,改江、淮制置發運使。殿直王乙者,請自揚州召伯埭東至瓜州,浚河百二十里,以廢二埭。詔瑾規度,以工大不可就,止置閘召伯埭旁,人以為利。累遷
 尚書刑部郎中,為三司戶部副使,除龍圖閣待制、權知開封府。未逾月,得疾,仁宗封藥賜之,使未及門而卒。



 孫沖,字升伯,趙州平棘人。舉明經,歷古田青陽尉、鹽山麗水主簿。嘗並喪父母去官,有司循五代故事,必六年乃聽調,沖援古制,以書乾宰相,不納。後舉進士,登甲科。授將作監丞,歷通判晉、絳、保州,坐與保州守爭事,降監吉州酒,累遷太常博士。



 河決棣州,知天雄軍寇準請徙州治河,命沖往按視。還言:「徙州動民,亦未免治堤,不若
 塞河為便。」遂以沖知棣州,自秋至春,凡四決,沖皆塞之,就除殿中侍御史。準為樞密使,卒徙州陽信。而沖坐守護河堤過嚴,民輸送往來堤上者輒榜之,為使者論奏,徙知襄州。沖復上疏論徙州非便,著《河書》以獻。



 會京西蝗,真宗遣中使督捕,至襄,怒沖不出迎,乃奏蝗唯襄為甚,而州將日置酒,無恤民意。帝怒,命即州置獄。沖得屬縣言歲稔狀,馳驛上之。時使者猶未還,帝悟,為追使者笞之。以侍御史為京西轉運。塞滑州決河,權知滑州。參
 知政事魯宗道總河事,用太常博士李渭策,欲盛夏興役。沖言徒費薪楗,困人力,雖塞必決。遂罷知河陽。累遷刑部郎中,歷湖北、河東轉運使。



 會南郊賞賜軍士,而汾州廣勇軍所得帛不逮他軍,一軍大噪,捽守佐堂下劫之,約與善帛乃免。城中戒備,遣兵圍廣勇營。沖適至,命解圍弛備,置酒張樂,推首惡十六人斬之,遂定。初,守佐以亂軍所約者上聞,詔給善帛。使者至潞,沖促之還,曰:「以亂而得所欲,是愈誘之亂也。」卒留不與。入判登聞鼓
 院,以目疾改兵部郎中、直史館、知河中府,徙潞州,復為河東轉運使,遷太常少卿,擢右諫議大夫,復知潞州,遷翰林院學士。及徙同州,權西京留司御史臺,遷給事中。喪明,卒。



 沖為吏,所至以強乾稱,能任鉤距,多得事情,然無家法,晚節尤寡廉聲。孫永,自有傳。



 崔嶧,字之才,京兆長安人。進士及第,累官尚書職方員外郎、知遂州。建議瞿塘峽置關如劍門,以察奸人。事既施行,徙提點刑獄。嘉陵江歲調民丁治堤堨,嶧更用州
 兵代其役。文州蕃卒數剽攻邊戶,守臣慮生事,多以牛酒和遣。嶧請守臣歲時得行邊,益募勇壯,伺其發,一切捕擊之,後無復內寇。就除轉運使。歷三司戶部判官、河東轉運使。會更錢法,潞州民大擾,推其首惡誅之,人心遂定。



 後為戶部副使,以右諫議大夫為河東都轉運使,遷給事中,還,糾察在京刑獄。諫官、御史言宰相陳執中縱嬖妾殺婢,命按治。嶧以為執中自以婢不恪笞之死,非妾殺之,頗左右執中,即授龍圖閣待制、知慶州。羌井
 坑族亂,潛兵討平。歷知同州、鳳翔府,改工部侍郎、集賢院學士、知河中府。



 嶧所至貪奸,比老益甚。在鳳翔,轉運使薛向按之急,不得已至河中。請老,以刑部侍郎致仕,卒。



 田瑜,字資忠,河南壽安人。舉進士,歷袁、郢、合三州軍事推官,遷大理寺丞,知鹿邑、建陽縣,徙知蒙、江二州,累遷尚書司封員外郎、提點廣南西路刑獄。慶歷中,區希範誘溪洞環州蠻叛,上以瑜習知南方事,就除荊湖北路
 轉運使。瑜檄屬郡募民擊賊,又督轉粟以守要害,故兵所至皆不乏食,賊勢大挫。



 徙兩浙轉運按察使。杭州龍山堤歲決,水冒民居,輒賦芻塞之。瑜與民約,每芻十束,更輸石一尺。率五歲,得石百萬,為石堤,堤固而歲不調民。加直史館、益州路轉運使,改江、淮制置發運使,擢天章閣待制、知廣州,累遷諫議大夫、權三司戶部副使。



 儂智高犯邕,瑜條上用兵禦賊十事。智高平,召對便殿,具言南方山川險要,所以備守之策,乃以為廣南東路體
 量安撫使。還,糾察刑獄,同判吏部流內銓,除龍圖閣直學士、知青州。城中有殺人投尸井中者,吏以其無主名,不以聞。瑜廉得之,大出金帛購賊,後數日,鄰州民執賊以告。屬歲兇多盜,瑜立賞罰、設方略捕格之,境中肅然。徙知澶州,背發疽卒。



 瑜瑾厚少文,而於吏事頗盡心,然御下急,無廉稱。



 施昌言,字正臣,通州靜海人。舉進士高第,授將作監丞、通判滁州。後以太常博士召試館職,不中選,遷尚書屯
 田員外郎、知太平州。上《政論》三十篇。入為殿中侍御史、開封府判官。安撫淮南,還,以禮部員外郎兼侍御史知雜事,遷三司度支副使,除天章閣待制、河北都轉運使。言事者以為濱、棣等六州河可涉,宜有城守如邊,以待契丹。詔昌言與宦官楊懷敏往視。懷敏以為當城如邊,昌言曰:「六州地千里,又河數移徙,城之甚難而無利。契丹未渝盟先自困,非便也。」或請於麟、府立十二砦以拓境,又詔昌言與明鎬、張元度可否,昌言獨以為:「麟、府在
 河外,於國家無毫發入,而至今饋守者,徒以畏蹙國之虛名。今不當又事無利之砦,以重困財力。」就除知慶州。在州所為不法,語徹朝廷。昌言疑通判陳湜言之,追發湜罪,湜坐廢,昌言亦降知華州。



 歷知滄州、河陽,移河北都轉運使。議塞商胡埽決河,令復故道,與北京留守賈昌朝累論。徙江、淮發運使,加龍圖閣直學士、知應天府,又知延州。召還,會塞六塔河,以為都大修河制置使,辭,弗許,加樞密直學士、知澶州,以便役事。河決,奪一官知
 滑州,又知杭州,加龍圖閣學士,復知滑州。以老求罷,乃以知越州。至京師,卒。



 昌言為發運使時,召範仲淹後堂,出婢子為優,雜男子慢戲,無所不言。仲淹怪問之,則皆昌言子也,仲淹大不懌而去。其治家如此。



 論曰:狄棐、郎簡、孫祖德、張若谷、石揚休、祖士衡並以文辭高第,累侍從,歷方州,始為名臣,終鮮大過,考其行事可見也。李垂寧去華近,不肯見宰相;張洞以直言正論為大臣所忌,則其抱負從可知矣。若李仕衡而下十人,
 皆能任劇繁,然或寡廉稱,或有醜行,君子恥之



\end{pinyinscope}