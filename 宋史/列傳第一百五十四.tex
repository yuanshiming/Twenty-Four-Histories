\article{列傳第一百五十四}

\begin{pinyinscope}

 ○樓鑰李大性任希夷徐應龍莊夏王阮王質陸游方信孺王柟



 樓鑰,字大防,明州鄞縣人。隆興元年,試南宮,有司偉其辭藝,欲以冠多士,策偶犯舊諱,知貢舉洪遵奏,得旨以
 冠末等。投贄謝諸公,考官胡銓稱之曰:「此翰林才也。」試教官,調溫州教授,為敕令所刪定官,修《淳熙法》。議者欲降太學釋奠為中祀,鑰曰:「乘輿臨辛,於先聖則拜,武成則肅揖,其禮異矣,可鈞敵乎?」



 改宗正寺主簿,歷太府、宗正寺丞,出知溫州。屬縣樂清倡言方臘之變且復起,邑令捕數人歸於郡。鑰曰:「罪之則無可坐,縱之則惑民。」編隸其為首者,而驅其徒出境,民言遂定。堂帖問故,鑰曰:「蘇洵有言:『有亂之形,無亂之實,是謂將亂。不可以有亂
 急,不可以無亂弛。』」丞相周必大心善之。



 光宗嗣位,召對,奏曰:「人主初政,當先立其大者。至大莫如恢復,然當先強主志,進君德。」又曰:「今之網密甚矣,望陛下軫念元元,以設禁為不得已,凡有創意增益者,寢而勿行,所以保養元氣。」



 除考功郎兼禮部。吏銓並緣為奸,多所壅底。鑰曰:「簡要清通,尚書郎之選。」盡革去之。改國子司業,擢起居郎兼中書舍人。代言坦明,得制誥體,繳奏無所回避。禁中或私請,上曰:「樓舍人朕亦憚之,不如且已。」刑部言,
 天下獄案多所奏裁,中書之務不清,宜痛省之。鑰曰:「三宥制刑,古有明訓。」力論不可。會慶節上壽,扈從班集,乘輿不出。已而玉牒、聖政、會要書成,將進重華,又屢更日。鑰言:「臣累歲隨班,見陛下上壽重華宮,歡動宸極。嘉王日趨朝謁,恪勤不懈,竊料壽皇望陛下之來,亦猶此也。」又奏:「聖政之書,全載壽皇一朝之事。玉牒、會要足成淳熙末年之書,幸速定其日,無復再展,以全聖孝。」於是上感悟,進書成禮。



 試中書舍人,俄兼直學士院。光宗內禪
 詔書,鑰所草也,有云:「雖喪紀自行於宮中,而禮文難示於天下。」薦紳傳誦之。遷給事中。乞正太祖東向之位,別立僖祖廟以代夾室,順祖、翼祖、宣祖之主皆藏其中,祫祭即廟而饗。從之。



 朱熹以論事忤韓侂胄,除職與郡。鑰言:「熹鴻儒碩學,陛下閔其耆老,當此隆寒,立講不便,何如俾之內祠,仍令修史,少俟春和,復還講筵。」不報。趙汝愚謂人曰:「樓公當今人物也,直恐臨事少剛決耳。」及見其持論堅正,嘆曰:「吾於是大過所望矣。」



 寧宗受禪,侂胄
 以知閣門事與聞傳命,頗有弄權之漸,彭龜年力攻之。侂胄轉一官,與在京宮觀,龜年除待制,與郡。鑰與林大中奏,乞留龜年於講筵,或命侂胄以外祠。龜年竟去,鑰遷為吏部尚書,以顯謨閣學士提舉江州太平興國宮。尋知婺州,移寧國府,罷,仍奪職。告老至再,許之。



 侂胄嘗副鑰為館伴,以鑰不附己,深嗛之。侂胄誅,詔起鑰為翰林學士,遷吏部尚書兼翰林侍講。時鑰年過七十,精敏絕人,詞頭下,立進草,院吏驚詫。入朝,陛楯舊班諦視鑰
 曰:「久不見此官矣。」時和好未定,金求韓侂胄函首,鑰曰:「和好待此而決,奸兇已斃之首,又何足恤。」詔從之。



 趙汝愚之子崇憲奏雪父冤,鑰乞正趙師召之罪,重蔡璉之誅,毀龔頤正《續稽古錄》以白誣謗。除端明殿學士、簽書樞密院事,升同知,進參知政事。位兩府者五年,累疏求去,除資政殿學士、知太平州,辭,進大學士,提舉萬壽觀。嘉定六年薨,年七十七,贈少師,謚宣獻。



 鑰文辭精博,自號攻愧主人,有集一百二十卷。



 李大性,字伯和,端州四會人。其先積中,嘗為御史,以直言入元祐黨籍,始家豫章。大性少力學,尤習本朝典故。以父任入官,因參選,進《藝祖廟謨》百篇及公私利害百疏。又言:「元豐制,六察許言事,章惇為相始禁之,乞復舊制,以廣言路。」從臣力薦之,命赴都堂審察,僅遷一秩,為湖北提刑司干官。未幾,入為主管吏部架閣文字。丁母艱,服闋,進《典故辨疑》百篇,皆本朝故實,蓋網羅百氏野史,訂以日歷、實錄,核其正舛,率有據依,孝宗讀而褒嘉
 之。



 擢大理司直,遷敕令所刪定官,添差通判楚州。郡守吳曦與都統劉超合議,欲撤城移他所,大性謂:「楚城實晉義烏間所築,最堅,奈何以脆薄易堅厚乎?」持不可。臺臣將劾其沮撓,不果。會從官送北客,朝命因俾廉訪,具以實聞,遂罷戎帥,召大性除太府寺丞,遷大宗正丞兼倉部郎,尋改工部。



 陳傅良以言事去國,彭龜年、黃度、楊方相繼皆去。大性抗疏言:「朝廷清明,乃使言者無故而去,臣所甚惜也。數人之心,皆本愛君,知其愛君,任其去
 而不顧,恐端人正士之去者將不止此。孟子曰:『不信仁賢,則國空虛。』臣所以為之寒心也。」



 孝宗崩,光宗疾,未能執喪。大性復上疏言:「今日之事,顛倒舛逆,況金使祭奠當引見於北宮素帷,不知是時猶可以不出乎?《檀弓》曰:『成人有兄死而不喪者,聞子皋將為成宰,遂為衰。成人曰:「兄則死而子皋為之衰。」』蓋言成人畏子皋之來方為制服,其服子皋為之,非為兄也。若陛下必待使來然後執喪,則恐貽譏中外,豈特如成人而已哉。」遷軍器少
 監,權司封郎,提舉浙東常平,改浙東提刑兼知慶元府。召為吏部郎中,四遷為司農卿。明年,兼戶部侍郎。



 出知紹興府,甫一歲,召為戶部侍郎,升尚書。朝論將用兵,大性條陳利害,主不宜輕舉之說,忤韓侂胄意,出知平江,移知福州,又移知江陵,充荊湖制置使。江陵當用兵後,殘毀饑饉,繼以疾疫,大性首譏振貸,凡三十八萬緡有奇。前官虛羨,凡十有四萬五千緡,率蠲放不督,民流移新復業者,皆奏免征榷。邊郡武爵,本以勵士,冒濫滋眾,
 大性劾兩路戎司冒受逃亡付身,凡三千四百九十有七道,率繳上毀抹,左選為之一清。江陵舊使銅鏹,錢重楮輕,民持貲入市,有終日不得一錢者。大性奏乞依襄、郢例通用鐵錢,於是泉貨流通,民始復業。除刑部尚書兼詳定敕令,尋遷兵部。



 時金國分裂,不能自存,有舉北伐之議者,大性上疏以和戰之說未定,乞令朝臣集議,從之。尋以端明殿學士知平江府,引疾丐祠,卒於家,年七十七,贈開府儀同三司,謚文惠。



 李氏自積中三世官
 於朝,父子兄弟相師友,而大性與弟大異、大東並躋從列,為名臣云。



 任希夷,字伯起,其先眉州人。四世祖伯雨為諫議大夫,其後仕閩,因家邵武。希夷少刻意問學,為文精苦。登淳熙三年進士第,調建寧府浦城簿。從朱熹學,篤信力行,熹器之曰:「伯起,開濟士也。」



 開禧初,主太常寺簿,奏:「紹熙以來,禮書未經編次,歲月滋久,恐或散亡,乞下本寺修纂。」從之。遷禮部尚書兼給事中。謂:「周惇頤、程顥、程頤為
 百代絕學之倡,乞定議賜謚。」其後惇頤謚元,顥謚純,頤謚正,皆希夷發之。



 進端明殿學士、簽書樞密院事兼權參知政事。史彌遠柄國久,執政皆具員,議者頗譏其拱默。尋提舉臨安洞霄宮,薨,贈少師,謚宣獻。



 徐應龍,字允叔。淳熙二年第進士,調衡州法曹、湖南檢法官。潭獲劫盜,首謀者已系獄,妄指逸者為首,吏信之,及獲逸盜,治之急,遂誣服。吏以成憲讞於憲司,應龍閱實其辭,謂:「首從不明,法當奏。」時周必大判潭州,提刑盧
 彥德不欲反其事,將置逸盜於死,應龍力與之辨。先是,彥德許應龍京削,至是怒曰:「君不欲出我門邪?」應龍曰:「以人命傅文字,所不忍也。」彥德不能奪,聞者多其有守,交薦之。



 改秩,知瑞州高安縣。呂祖儉言事忤韓侂胄,謫死高安,應龍為之經紀其喪,且為文誄之。有勸之避禍者,應龍曰:「呂君吾所敬,雖緣此獲譴,亦所願也。」朱熹貽書應龍曰:「高安之政,義風凜然。」主淮西機宜文字,知南恩州。



 陳自強當國,乃舊同舍,應龍丐雷州而去。召監都
 進奏院,遷國子博士、守工部員外郎,進戶部侍郎,遷國子司業兼實錄院檢討官、崇政殿說書、守秘書少監兼權工部侍郎。



 時金主徙汴,應龍言:「金人窮而南奔,將溢出而蹈吾之境。金亡,更生新敵,尤為可慮。」兼侍講,言:「人主不能盡知天下人材,當責之宰相;宰相不能盡知天下人材,當採之公論。李吉甫為相,號稱得人,而三人之薦,乃出於裴垍之疏。」



 遷吏部侍郎,進刑部尚書兼侍讀。應龍在講筵,多指陳時政。一日讀吳起為卒吮疽事,應
 龍奏:「起恤士卒如此,故能得其死力。今軍將得以賄遷,專事掊克,未免多怨。」上驚曰:「債帥之風,今猶未除邪?」宰相史彌遠聞而惡之,免侍讀。未幾,兼太子詹事。會景獻太子薨,請老,上不許,徙吏部尚書,以煥章閣學士提舉嵩山崇福宮。嘉定十七年卒,贈開府儀同三司,謚文肅。



 子榮叟,官至參知政事,謚文靖;深叟,官終將作監丞;清叟,知樞密院事兼參知政事。各有傳。



 莊夏,字子禮,泉州人。淳熙八年進士。慶元六年,大
 旱,詔求言。夏時知贛州興國縣,上封事曰:「君者陽也,臣者君之陰也。今威福下移,此陰勝也。積陰之極,陽氣散亂而不收,其弊為火災,為旱蝗。願陛下體陽剛之德,使後宮戚里、內省黃門,思不出位,此抑陰助陽之術也。」



 召為太學博士。言:「比年分藩持節,詔墨未乾而改除,坐席未溫而易地,一人而歲三易節,一歲而郡四易守,民力何由裕?」遷國子博士,召除吏部員外郎,遷軍器監,太府少卿。出知漳州,為宗正少卿兼國史院編修官,尋權直學士
 院兼太子侍讀。時流民來歸,夏言:「荊襄、兩淮多不耕之田,計口授地,貸以屋廬牛具。吾乘其始至,可以得其欲;彼幸其不死,可以忘其勞。兵民可合,屯田可成,此萬世一時也。」



 試中書舍人兼太子右庶子、左諭德,言:「今戰守不成,而規模不定,則和好之說,得以乘間而入。今日之患,莫大於兵冗。乞行下將帥,令老弱自陳,得以子若弟侄若婿強壯及等者收刺之,代其名糧。」上曰:「兵卒子弟與召募百姓不同,卿言是也。」除兵部侍郎、煥章閣待制,
 與祠歸。嘉定十年卒。



 王阮,字南卿,江州人。曾祖韶,神宗時,開熙河,擒木徵;祖厚,繼闢湟、鄯;父彥傅,靖康勤王:皆有功。阮少好學,尚氣節。常自稱將種,辭辯奮發,四坐莫能屈。嘗謁袁州太守張栻,栻謂曰:「當今道在武夷,子盍往求之。」阮見朱熹於考亭,熹與語,大說之。登隆興元年進士第。



 時孝宗初即位,欲成高宗之志,首詔經理建業以圖進取,而大臣巽懦幸安,計未決。阮試禮部,對策曰:



 臨安蟠幽宅阻,面湖
 背海,膏腴沃野,足以休養生聚,其地利於休息。建康東南重鎮,控制長江呼吸之間,上下千里,足以虎視吳、楚,應接梁、宋,其地利於進取。建炎、紹興間,敵人乘勝長驅直搗,而我師亦甚憊也。上皇遵養時晦,不得與平,乃駐臨安,所以為休息計也。已三十年來,闕者全,壞者修,弊者整,廢者復,較以曩昔,倍萬不侔。主上獨見遠覽,舉而措諸事業,非固以臨安為不足居也。戰守之形既分,動靜進退之理異也。



 古者立國,必有所恃,謀國之要,必負
 其所恃之地。秦有函谷,蜀有劍閣,魏有成皋,趙有井陘,燕有飛狐,而吳有長江,皆其所恃以為國也。今東南王氣,鐘在建業,長江千里,控扼所會,輟而弗顧,退守幽深之地,若將終身焉,如是而曰謀國,果得為善謀乎?且夫戰者以地為本,湖山回環,孰與乎龍盤虎踞之雄?胥潮奔猛,孰與乎長江之險?今議者從習吳、越之僻固,而不知秣陵之通達,是猶富人之財,不布於通都大邑,而匣金以守之,愚恐半夜之或失也。儻六飛順動,中原在跬
 步間,況一建康耶?古人有言:「千里之行,起於足下。」人患不為爾。



 知貢舉範成大得而讀之,嘆曰:「是人傑也。」



 調南康都昌主簿,以廉聲聞,移永州教授。獻書闕下,請罷吳、楚牧馬之政,而積馬於蜀茶馬司,以省往來綱驛之費、歲時分牧之資,凡數千言。紹熙中,知濠州,請復曹瑋方田,修種世衡射法,日講守備,與邊民親訪北境事宜。終阮在濠,金不敢南侵。改知撫州。



 韓侂胄宿聞阮名,特命入奏,將誘以美官,夜遣密客詣阮,阮不答,私謂所親曰:「
 吾聞公卿擇士,士亦擇公卿。劉歆、柳宗元失身匪人,為萬世笑。今政自韓氏出,吾肯出其門哉?」陛對畢,拂衣出關。侂胄聞之大怒,批旨予祠。阮於是歸隱廬山,盡棄人間事,從容觴詠而已。朱熹嘗惜其才氣術略過人,而留滯不偶雲。嘉定元年卒。



 王質,字景文,其先鄆州人,後徙興國。質博通經史,善屬文。游太學,與九江王阮齊名。阮每云:「聽景文論古,如讀酈道元《水經》,名川支川,貫穿周匝,無有間斷,咳唾皆成
 珠璣。」



 質與張孝祥父子游,深見器重。孝祥為中書舍人,將薦質舉制科,會去國不果。著論五十篇,言歷代君臣治亂,謂之《樸論》。中紹興三十年進士第,用大臣言,召試館職,不就。明年,金主完顏亮南侵,御史中丞汪澈宣諭荊、襄,又明年,樞密使張浚都督江、淮,皆闢為屬。入為太學正。



 時孝宗屢易相,國論未定,質乃上疏曰:



 陛下即位以來,慨然起乘時有為之志,而陳康伯、葉義問、汪澈在廷,陛下皆不以為才,於是先逐義問,次逐澈,獨徘徊康
 伯,難於進退,陛下意終鄙之,遂決意用史浩,而浩亦不稱陛下意,於是決用張浚,而浚又無成,於是決用湯思退。今思退專任國政,又且數月,臣度其終無益於陛下。



 夫宰相之任一不稱,則陛下之志一沮。前日康伯持陛下以和,和不成;浚持陛下以戰,戰不驗;浚又持陛下以守,守既困;思退又持陛下以和。陛下亦嘗深察和、戰、守之事乎?李牧在雁門,法主於守,守乃有戰。祖逖在河南,法主於戰,戰乃有和。羊祜在襄陽,法主於和,和乃有守。
 何至分而不使相合?



 今陛下之心志未定,規模未立。或告陛下,金弱且亡,而吾兵甚振,陛下則勃然有勒燕然之志;或告陛下,吾力不足恃,而金人且來,陛下即委然有盟平涼之心;或告陛下,吾不可進,金不可入,陛下又蹇然有指鴻溝之意。使臣為陛下謀,會三者為一,天下烏有不治哉?



 天子心知質忠,而忌者共讒質年少好異論,遂罷去。會虞允文宣撫川、陜,闢質偕行。一日令草檄契丹文,援毫立就,辭氣激壯。允文起執其手曰:「景文天
 才也。」入為敕令所刪定官,遷樞密院編修官。允文當國,孝宗命擬進諫官,允文以質鯁亮不回,且文學推重於時,可右正言。時中貴人用事,多畏憚質,陰沮之,出通判荊南府,改吉州,皆不行,奉祠山居,絕意祿仕。淳熙十五年卒。



 陸游字務觀,越州山陰人。年十二能詩文,蔭補登仕郎。鎖廳薦送第一,秦檜孫塤適居其次,檜怒,至罪主司。明年,試禮部,主司復置游前列,檜顯黜之,由是為所嫉。檜
 死,始赴福州寧德簿,以薦者除敕令所刪定官。



 時楊存中久掌禁旅,游力陳非便,上嘉其言,遂罷存中。中貴人有市北方珍玩以進者,游奏:「陛下以『損』名齋,自經籍翰墨外,屏而不御。小臣不體聖意,輒私買珍玩,虧損聖德,乞嚴行禁絕。」



 應詔言:「非宗室外家,雖實有勛勞,毋得輒加王爵。頃者有以師傅而領殿前都指揮使,復有以太尉而領閣門事,瀆亂名器,乞加訂正。」遷大理寺司直兼宗正簿。



 孝宗即位,遷樞密院編修官兼編類聖政所檢
 討官。史浩、黃祖舜薦游善詞章,諳典故,召見,上曰:「游力學有聞,言論剴切。」遂賜進士出身。入對,言:「陛下初即位,乃信詔令以示人之時,而官吏將帥一切玩習,宜取其尤沮格者,與眾棄之。」



 和議將成,游又以書白二府曰:「江左自吳以來,未有舍建康他都者。駐蹕臨安出於權宜,形勢不固,饋餉不便,海道逼近,凜然意外之憂。一和之後,盟誓已立,動有拘礙。今當與之約,建康、臨安皆系駐蹕之地,北使朝聘,或就建康,或就臨安,如此則我得以
 暇時建都立國,彼不我疑。」



 時龍大淵、曾覿用事,游為樞臣張燾言:「覿、大淵招權植黨,熒惑聖聽,公及今不言,異日將不可去。」燾遽以聞,上詰語所自來,燾以游對。上怒,出通判建康府,尋易隆興府。言者論游交結臺諫,鼓唱是非,力說張浚用兵,免歸。久之,通判夔州。



 王炎宣撫川、陜,闢為乾辦公事。游為炎陳進取之策,以為經略中原必自長安始,取長安必自隴右始。當積粟練兵,有釁則攻,無則守。吳璘子挺代掌兵,頗驕恣,傾財結士,屢以過
 誤殺人,炎莫誰何。游請以玠子拱代挺。炎曰:「拱怯而寡謀,遇敵必敗。」游曰:「使挺遇敵,安保其不敗。就令有功,愈不可駕馭。」及挺子曦僭叛,游言始驗。



 範成大帥蜀,游為參議官,以文字交,不拘禮法,人譏其頹放,因自號放翁。後累遷江西常平提舉。江西水災。奏:「撥義倉振濟,檄諸郡發粟以予民。」召還,給事中趙汝愚駁之,遂與祠。起知嚴州,過闕,陛辭,上諭曰:「嚴陵山水勝處,職事之暇,可以賦詠自適。」再召入見,上曰:「卿筆力回斡甚善,非他人可
 及。」除軍器少監。



 紹熙元年,遷禮部郎中兼實錄院檢討官。嘉泰二年,以孝宗、光宗兩朝實錄及三朝史未就,詔游權同修國史、實錄院同修撰,免奉朝請,尋兼秘書監。三年,書成,遂升寶章閣待制,致仕。



 游才氣超逸,尤長於詩。晚年再出,為韓侂胄撰《南園閱古泉記》,見譏清議。朱熹嘗言:「其能太高,跡太近,恐為有力者所牽挽,不得全其晚節。」蓋有先見之明焉。嘉定二年卒,年八十五。



 方信孺,字孚若,興化軍人。有雋材,未冠能文,周必大、楊
 萬里見而異之。以父崧卿蔭,補番禺縣尉。盜劫海賈,信孺捕之,盜方沙聚分鹵獲,惶駭欲趨舟,信孺已使人負盜舟去矣,乃悉縛盜,不失一人。



 韓侂胄舉恢復之謀,諸將僨軍,邊釁不已。朝廷尋悔,金人亦厭兵,乃遣韓元靚來使,而都督府亦再遣壯士遺敵書,然皆莫能得其要領。近臣薦信孺可使,自蕭山丞召赴都,命以使事。信孺曰:「開釁自我,金人設問首謀,當何以答之?」侂胄矍然。假朝奉郎、樞密院檢詳文字,充樞密院參謀官,持督帥張
 巖書通問於金國元帥府。



 至濠州,金帥紇石烈子仁之止於獄中,露刃環守之,絕其薪水,要以五事。信孺曰:「反俘、歸幣可也,縛送首謀,於古無之,稱藩、割地,則非臣子所忍言。」子仁怒曰:「若不望生還耶?」信孺曰:「吾將命出國門時,已置生死度外矣。」



 至汴,見金左丞相、都元帥完顏宗浩,出就傳舍。宗浩使將命者來,堅持五說,且謂:「稱藩、割地,自有故事。」信孺曰:「昔靖康倉卒割三鎮,紹興以太母故暫屈,今日顧可用為故事耶?此事不獨小臣不敢言,
 行府亦不敢奏也。請面見丞相決之。」將命者引而前,宗浩方坐幄中,陳兵見之,云:「五事不從,兵南下矣。」信孺辯對不少詘。宗浩叱之曰:「前日興兵,今日求和,何也?」信孺曰:「前日興兵復仇,為社稷也。今日屈己求和,為生靈也。」宗浩不能詰,授以報書曰:「和與戰,俟再至決之。」



 信孺還,詔侍從、兩省、臺諫官議所以復命。眾議還俘獲,罪首謀,增歲幣五萬,遣信孺再往。時吳曦已誅,金人氣頗索,然猶執初議。信孺曰:「本朝謂增幣已為卑屈,況名分地界
 哉?且以曲直校之,本朝興兵在去年四月,若貽書誘吳曦,則去年三月也,其曲固有在矣。如以強弱言之,若得滁、濠,我亦得泗、漣水。若誇胥浦橋之勝,我亦有鳳凰山之捷。若謂我不能下宿、壽,若圍廬、和、楚果能下乎?五事已從其三,而猶不我聽,不過再交兵耳。」



 金人見信孺忠懇,乃曰:「割地之議姑寢,但稱藩不從,當以叔為伯,歲幣外,別犒師可也。」信孺固執不許。宗浩計窮,遂密與定約。復命,再差充通謝國信所參謀官,奉國書誓草及許
 通謝百萬緡抵汴。宗浩變前說,怒信孺不曲折建白,遽以誓書來,有「誅戮禁錮」語。信孺不為動,將命曰:「此事非犒軍錢可了。」別出事目。信孺曰:「歲幣不可再增,故代以通謝錢。今得此求彼,吾有隕首而已。」將命曰:「不爾,丞相欲留公。」信孺曰:「留於此死,辱命亦死,不若死於此。」會蜀兵取散關,金人益疑。



 信孺還,言:「敵所欲者五事:割兩淮一,增歲幣二,犒軍三,索歸正等人四,其五不敢言。」侂胄再三問,至厲聲詰之,信孺徐曰:「欲得太師頭耳。」侂胄大怒,
 奪三秩,臨江軍居住。



 信孺自春至秋,使金三往返,以口舌折強敵,金人計屈情見,然憤其不屈,議用弗就。已而王柟出使,定和議,增幣、函首,皆前信孺所持不可者。柟白廟堂:「信孺辯折敵酋於強愎未易告語之時,信孺當其難,柟當其易。柟每見,金人必問信孺安在,公論所推,雖敵人不能掩也。」乃詔信孺自便。



 尋知韶州,累遷淮東轉運判官兼提刑。知真州,即北山匱水築石堤,袤二十里,人莫知其所為。後金人薄儀真,守將決水匱以退敵,
 城乃獲全。山東初內附,信孺言:「豪傑不可以虛名駕馭,武夫不可以弱勢彈壓,宜選威望重臣,將精兵數萬,開幕山東,以主制客,以重馭輕,則可以包山東,固江北,而兩河在吾目中矣。」坐責降三秩,再奉祠,稍復官。



 信孺性豪爽,揮金如糞土,所至賓客滿其後車。使北時,年財三十。既齟齬歸,營居室巖竇,自放於詩酒。後貲用竭,賓客益落,信孺尋亦死矣。



 王柟,字汝良,大名人。祖倫,同簽書樞密院事。倫使北死,
 孝宗訪求其孫之未祿者三人官之,柟其一也。調通州海門尉。乘輕舟入海濤,捕劇賊小吳郎,並其徒十七人獲之,獄成,不受賞。



 韓侂胄以恢復起兵端,天子思繼好息民,凡七遣使無成。續遣方信孺往,將有成說矣,坐白事忤侂胄得罪。欲再遣使,顧在廷無可者,近臣以柟薦,擢監登聞鼓院,假右司郎中,使持書北行。柟歸白其母,母曰:「而祖以忠死國,故恩及子孫。汝其勉旃,毋以吾老為念。」乃拜命,疾驅抵敵所。



 金將烏骨論等四人列坐,問:「
 韓侂胄貴顯幾年矣?」柟對:「已十餘年,平章國事財二年耳。」又問:「今欲去此人可乎?」柟曰:「主上英斷,去之何難。」四人相顧而笑。有完顏天寵者,袖出文書,云:「王柟雖持韓侂胄書,乃朝廷有旨遣其來元帥府議和,宜詳議以報。」於是金人知侂胄已誅,和議遂決。



 柟持金人牒歸,求函侂胄首,以起居郎許奕為通謝使,柟為通謝所參謀官。柟自軍前再還,議以侂胄首易淮、陜侵地,從之。柟奏:「和約之成,皆方信孺備嘗險阻再三將命之功,臣因人成
 事,乞錄信孺功而蠲其過。」朝論以柟不掩人揚己多之。守軍器少監,知楚州,累官至太府卿。告歸,以右文殿修撰知太平州,加集英殿修撰,致仕。卒,贈寶章閣待制。



 論曰:樓鑰渾厚正大,李大性直言不愧其先,任希夷請謚先儒,徐應龍在經筵多所裨益,莊夏、王阮、王質皆負其有為之才,卒奉祠去國。陸游學廣而望隆,晚為韓侂胄著堂記,君子惜之,抑《春秋》責賢者備也。方信孺年少奉使,而以意氣折金人。王柟北歸,請錄信孺之功,長者
 哉!



\end{pinyinscope}