\article{列傳第六十四}

\begin{pinyinscope}

 楊億弟偉從子紘晁迥子宗愨劉筠薛映



 楊億,字大年,建州浦城人。祖文逸,南唐玉山令。億將生,文逸夢一道士,自稱懷玉山人來謁。未幾,億生,有毛被
 體,長尺餘,經月乃落。能言,母以小經口授,隨即成誦。七歲,能屬文,對客談論,有老成風。雍熙初,年十一,太宗聞其名,詔江南轉運使張去華就試詞藝,送闕下。連三日得對,試詩賦五篇,下筆立成。太宗深加賞異,命內侍都知王仁睿送至中書,又賦詩一章,宰相驚其俊異,削章為賀。翌日,下制曰:「汝方髫齔,不由師訓,精爽神助,文字生知。越景絕塵,一日千里,予有望於汝也。」即授秘書省正字,特賜袍笏。俄丁外艱,服除,會從祖徽之知許州,億
 往依焉。務學,晝夜不息,徽之間與語,嘆曰:「興吾門者在汝矣。」



 淳化中,詣闕獻文,改太常寺奉禮郎,仍令讀書秘閣。獻《二京賦》,命試翰林,賜進士第,遷光祿寺丞。屬後苑賞花曲宴,太宗召命賦詩於坐側;又上《金明池頌》,太宗誦其警句於宰相。明年三月,苑中曲宴,億復以詩獻。太宗訝有司不時召,宰相言:「舊制,未貼職者不預。」即以億直集賢院。表求歸鄉里,賜錢十五萬。至道初,太宗親制九弦琴、五弦阮,文士奏頌者眾,獨稱億為優,賜緋魚。二
 年春,遷著作佐郎,帝知其貧,屢有沾賚,嘗命為越王生辰使。時公卿表疏,多假文於億,名稱益著。



 真宗在京府,徽之為首僚,邸中書疏,悉億草定。即位初,超拜左正言。詔錢若水修《太宗實錄》,奏億參預,凡八十卷,而億獨草五十六卷。書成,乞外補就養,知處州。真宗稱其才長於史學,留不遣,固請,乃許之任。郡人周啟明篤學有文,深加禮待。召還,拜左司諫、知制誥,賜金紫。



 咸平中,西鄙未寧,詔近臣議靈州棄守之事。億上疏曰:



 臣嘗讀史,見漢
 武北築朔方之郡,平津侯諫,以為罷敝中國,以奉無用之地,願罷之。上使辯士朱買臣等發十策以難平津,平津不能對。臣以為平津為賢相,非不能折買臣之舌,蓋所以將順人君之意爾。舊稱朔方,地在要荒之外,聲教不及。元朔中,大將軍衛青奮兵掠地,列置郡縣。今靈州蓋朔方之故墟,僻介西鄙,數百里間無有水草,烽火亭障不相望。當其道路不壅,囊饋無虞,猶足以張大國之威聲,為中原之捍蔽。自邊境屢驚,兇黨猖熾,爵賞之而
 不恭,討罰之而無獲。自曹光實、白守榮、馬紹忠及王榮之敗,資糧屝屨,所失至多,將士丁夫,相枕而死。以至募商人輸帛入谷,償價數倍;孤壤築城,邊民繹騷,國帑匱乏,不能制邊人之命,及濟靈武之急。數年之間,兇黨逾盛。靈武危堞,巋然僅存,河外五城,繼聞陷沒。但堅壁清野,坐食糗糧,閉壘枕戈,茍度朝夕,未嘗出一兵馳一騎,敢與之角。此靈武之存無益,明矣。平津所言罷敝中國以奉無用之地,正今日謂也。



 臣以為存有大害,棄有大
 利,國家挽粟之勞,士卒流離之苦,悉皆免焉。堯、舜、禹,聖之盛者也,地不過數千里,而明德格天,四門穆穆。武丁、成王,商、周之明主也,然地東不過江、黃,西不過氐、羌,南不過蠻荊,北不過太原,而頌聲並作,號為至治。及秦、漢窮兵拓土,肝腦塗地,校其功德,豈可同年而語哉!昔西漢賈捐之建議棄朱崖,當時公卿,亦有異論,元帝力排眾說,奮乎獨見,下詔廢之,人頌其德。故其詔曰:「議者以棄朱崖羞威不行,夫通於時變,即憂萬民之饑鋨,危孰
 大焉。且宗廟之祭,兇年不備,況乎避不嫌之辱哉?」臣以為類於靈武也,必以失地為言,即燕薊八州,河湟五郡,所失多矣,何必此為?



 臣竊惟太祖命姚內斌領慶州,董遵誨領環州,統兵裁五六千,悉付以閫外之事,士卒效命,疆埸晏然,朝廷無旰食之憂,疆埸無羽書之警。臣乞選將臨邊,賜給廩賦,資以策略,許以便宜而行。儻寇擾內屬,撓之以勁兵,示之以大信,懷荒振遠,諭以賞格,彼則奔潰眾叛,安能與大邦為敵哉?若欲謀成廟堂,功在漏
 刻,臣以為彼眾方黠,積財猶豐,未可以歲月破也。直須棄靈州,保環慶,然後以計困之爾。如臣之策,得驍將數人,提銳兵一二萬,給數縣賦以資所用,令分守邊城,則寇可就擒,而朝廷得以無虞矣。



 景德初,以家貧,乞典郡江左,詔令知通進、銀臺司兼門下封駁事。時以吏部銓主事前宜黃簿王太沖為大理評事,億以丞吏之賤,不宜任清秩,即封詔還。未幾,太沖補外。俄判史館,會修《冊府元龜》,億與王欽若同總其事。其序次體制,皆億所定,
 群僚分撰篇序,詔經億竄定方用之。三年,召為翰林學士,又同修國史,凡變例多出億手。大中祥符初,加兵部員外郎、戶部郎中。



 五年,以疾在告,遣中使致太醫視之,億拜章謝,上作詩批紙尾,有「副予前席待名賢」之句。以久疾,求解近職,優詔不許,但權免朝直。億剛介寡合,在書局,唯與李維、路振、刁衎、陳越、劉筠輩厚善。當時文士,咸賴其題品,或被貶議者,退多怨誹。王欽若驟貴,億素薄其人,欽若銜之,屢抉其失;陳彭年方以文史售進,忌
 億名出其右,相與毀訾。上素重億,皆不惑其說。億有別墅在陽翟,億母往視之,因得疾,請歸省,不待報而行。上親緘藥劑,加金帛以賜。億素體羸,至是,以病聞,請解官。有嗾憲官劾億不俟命而去,授太常少卿,分司西京,許就所居養療。嘗作《君可思賦》,以抒忠憤。《冊府元龜》成,進秩秘書監。



 七年,病愈,起知汝州。會加上玉皇聖號,表求陪預,即代還,以為參詳儀制副使,知禮儀院,判秘閣、太常寺。天禧二年冬,拜工部侍郎。明年,權同知貢舉,坐考
 較差謬,降授秘書監。丁內艱,屬行郊禮,以億典司禮樂,未卒哭,起復工部侍郎,令視事。四年,復為翰林學士,受詔注釋御集,又兼史館修撰、判館事,權景靈宮副使。十二月,卒,年四十七。錄其子紘為太常寺奉禮郎。



 億天性穎悟,自幼及終,不離翰墨。文格雄健,才思敏捷,略不凝滯,對客談笑,揮翰不輟。精密有規裁,善細字起草,一幅數千言,不加點竄,當時學者,翕然宗之。而博覽強記,尤長典章制度,時多取正。喜誨誘後進,以成名者甚眾。人
 有片辭可紀,必為諷誦。手集當世之述作,為《筆苑時文錄》數十篇。重交游,性耿介,尚名節。多周給親友,故廩祿亦隨而盡。留心釋典禪觀之學,所著《括蒼武夷穎陰韓城退居汝陽蓬山冠鰲》等集、《內外制》、《刀筆》,共一百九十四卷。弟倚,景德中舉進士,得第三等及第;以億故,升為第二等。億無子,以從子紘為後。弟偉。



 偉字子奇,幼學於億。天禧元年獻頌,召試學士院,賜進士及第。以試秘書省校書郎知衢州龍游縣,再補蘄州
 錄事參軍,國子監薦為直講。駙馬都尉李遵勖守澶州,闢簽書鎮寧軍節度判官事。遷大理寺丞、知河間縣,再遷太常博士。用近臣薦,為集賢校理、通判單州。會巡檢部卒李素合州卒二百餘人,謀殺巡檢使,入鼓角門,州將不敢出。偉挺身往問曰:「若屬何為而反?」俱曰:「將有訴於州,非反也。」偉曰:「持兵來,非反而何?若屬皆有父母妻子,以一朝忿而欲魚肉之乎?」悉令投兵,坐籍首惡得十餘人,斬之。徙知祥符縣、提點開封府界諸縣鎮公事,權
 開封府判官,又判三司開拆司,累遷尚書兵部員外郎、同修起居注。



 偉清慎,無治劇才,常秉小笏以朝。知制誥缺。中書以偉名進。仁宗曰:「此非秉小笏者邪?」遂命知制誥,權諫院。嘗曰:「諫臣宜陳列大事,細故何足論。」然當時譏其亡補。遷刑部郎中,為翰林學士。祀明堂,遷右司郎中、判太常寺,為群牧使兼侍讀學士,進中書舍人。卒,贈尚書禮部侍郎。



 紘字望之,以蔭歷官知鄞縣。鄞濱海,惡少販魚鹽者,群
 居洲島,或掠商人財物入海,吏不能禁。紘至,設方略,使識者質惡少船,及歸,始給還,且戒諭之,由是不敢為盜。以億文獻,賜進士出身。通判越州,知筠州,提點江東刑獄,除轉運、按察使。江東饑,紘開義倉賑之,吏持不可。紘曰:「義倉,為民也,稍稽,人將殍矣。」



 紘御下急,常曰:「不法之人不可貸。去之,止不利一家爾,豈可使郡邑千萬家俱受害邪?」聞者望風解去,或過期不敢之官。與王鼎、王綽號「江東三虎」。坐降知衡州,徙越州。為荊南轉運使,徙福
 建,不赴,知湖州,復為江東轉運使。官至太常少卿,卒。紘性嚴,雖家居,兒女不敢妄言笑。聚書數萬卷,手抄事實,名《窺豹篇》。



 晁迥,字明遠,世為澶州清豐人,自其父佺,始徙家彭門。迥舉進士,為大理評事,歷知嶽州錄事參軍,改將作監丞,稍遷殿中丞。坐失入囚死罪,奪二官。復將作丞,監徐、婺二州稅,遷太常丞。真宗即位,用宰相呂端、參知政事李沆薦,擢右正言、直史館。獻《咸平新書》五十篇,又獻《理
 樞》一篇。召試,除右司諫、知制誥,判尚書刑部。



 帝北征,雍王元份留守京師,加右諫議大夫,為判官,進翰林學士。未幾,知審官院,為明德、章穆二園陵禮儀使,同修國史。知大中祥符元年貢舉。封泰山,祀汾陰,同太常詳定儀注,累遷尚書工部侍郎。使契丹,還,奏《北庭記》,加史館修撰、知通進銀臺司。獻《玉清昭應宮頌》,其子宗操繼上《景靈宮慶成歌》。帝曰:「迥父子同獻歌頌,搢紳間美事也。」



 史成,擢刑部侍郎,進承旨。時朝廷方修禮文之事,詔令多
 出迥手。嘗夜召對,帝令內侍持燭送歸院。方盛暑,為蠲宿直,令三五日一至院;迥辭以非故事,乃聽俟秋還直。遷兵部侍郎,請分司西京,特拜工部尚書、集賢院學士、判西京留司御史臺。賜一子官河南,以就養。



 仁宗即位,遷禮部尚書。居臺六年,累章請老,以太子少保致仕,給全俸,歲時賜賚如學士。天聖中,迥年八十一,召宴太清樓,免舞蹈。子宗愨為知制誥,侍從同預宴。迥坐御史中丞之南,與宰臣同賜御飛白大字。既罷,所以寵賚者甚厚,
 進太子少傅。後復召對延和殿,帝訪以《洪範》雨□昜之應。對曰:「比年變災薦臻,此天所以警陛下。願陛下修飭王事,以當天心,庶幾轉亂而為祥也。」既而獻《斧扆》、《慎刑箴》,《大順》、《審刑》、《無盡燈頌》,凡五篇。及感疾,絕人事,屏醫藥,具冠服而卒,年八十四。罷朝一日,贈太子太保,謚文元。



 迥善吐納養生之術,通釋老書,以經傳傅致,為一家之說。性樂易寬簡,服道履正,雖貴勢無所屈,歷官臨事,未嘗挾情害物。真宗數稱其好學長者。楊億嘗謂迥所作書
 命無過褒,得代言之體。喜質正經史疑義,摽括字類。有以術命語迥,迥曰:「自然之分,天命也。樂天不憂,知命也。推理安常,委命也。何必逆計未然乎?」所著《翰林集》三十卷,《道院集》十五卷,《法藏碎金錄》十卷,《耆智餘書》、《隨因紀述》、《昭德新編》各三卷。子宗愨。



 宗愨字世良,以父蔭為秘書省校書郎。屢獻歌頌,召試,賜進士及第。又除館閣校勘,三遷大理寺丞、集賢校理兼注釋御集檢閱官。迥領西臺,宗愨求便養,通判許
 州。仁宗即位,遷殿中丞、同修起居注。天聖中,百官轉對,宗愨請減上供,墾閑田,擇獄官,令監司舉縣令。累遷尚書祠部員外郎、知制誥。宋綬嘗謂:「自唐以來,唯楊於陵身見其子嗣復繼掌書命,今始有晁氏焉。」父憂,奪喪,管勾會靈觀,入翰林為學士。母亡,又起復,兼龍圖閣學士、權發遣開封府事,辨雪疑獄有能名。



 元昊反,關中久宿師,以宗愨安撫陜西,與夏竦議攻守策。未還,道拜右諫議大夫、參知政事。會朝廷以金飾胡床及金汲器賜唃
 廝羅,宗愨曰:「仲叔于奚辭邑請繁纓,孔子曰:『不如多與之邑。』繁纓,諸侯之馬飾,猶不可與陪臣,況以乘輿之器賜外臣乎?必欲優其禮,不若加賜金帛。」後從帝郊祠感疾,數求罷,除資政殿學士、給事中。數日,卒。贈工部尚書,謚文莊。



 宗愨性敦厚,事父母孝,篤於故舊,凡任子恩皆先其族人。在翰林,一夕草將相五制,褒揚訓戒,人得所宜。嘗密詔訪邊策,陳七事,頗施用之。



 劉筠,字子儀,大名人。舉進士,為館陶縣尉。還,會詔知制
 誥楊億試選人校太清樓書,擢筠第一,以大理評事為秘閣校理。真宗北巡,命知大名府觀察判官事。自邊鄙罷兵,國家閑暇,帝垂意篇籍,始集諸儒考論文章,為一代之典。筠預修圖經及《冊府元龜》,推為精敏。真宗將祀汾睢,屢得嘉獎,召筠及監察御史陳從易崇和殿賦歌詩,帝數稱善。車駕西巡,又命筠纂土訓。是時四方獻符瑞,天子方興禮文之事,筠數上賦頌。及《冊府元龜》成,進左正言、直史館、修起居注。嘗屬疾,予告滿,輒再予,積二
 百日,每詔續其奉。



 遷左司諫、知制誥,加史館修撰,出知鄧州,徙陳州。還,糾察在京刑獄,知貢舉,遷尚書兵部員外郎。復請鄧州,未行,進翰林學士。初,筠嘗草丁謂與李迪罷相制,既而謂復留,令別草制,筠不奉詔,乃更召晏殊。筠自院出,遇殊樞密院南門,殊側面而過,不敢揖,蓋內有所愧也。帝久疾,謂浸擅權,筠曰:「奸人用事,安可一日居此。」請補外,以右諫議大夫知廬州。



 仁宗即位,遷給事中,復召為翰林學士。逾月,拜御史中丞。先是,三院御
 史言事,皆先白中丞。筠榜臺中,御史自言事,毋白丞雜。知天聖二年貢舉,數以疾告,進尚書禮部侍郎、樞密直學士、知穎州。召還,復知貢舉,進翰林學士承旨兼龍圖閣直學士、同修國史、判尚書都省。祀南郊,為禮儀使,請宿齋太廟日,罷朝饗玉清昭應宮,俟禮成,備鑾駕恭謝。從之。筠素愛廬江,遂築室城中,構閣藏前後所賜書,帝飛白書曰「真宗聖文秘奉之閣」。再知廬州,營塚墓,作棺,自為銘刻之。既病,徙於書閣,卒。



 筠,景德以來,居文翰之
 選,其文辭善對偶,尤工為詩。初為楊億所識拔,後遂與齊名,時號「楊劉」。凡三入禁林,又三典貢部,以策論升降天下士,自筠始。性不茍合,遇事明達,而其治尚簡嚴。然晚為陽翟同姓富人奏求恩澤,清議少之。著《冊府應言》、《榮遇》、《禁林》、《肥川》、《中司》、《汝陰》、《三入玉堂》凡七集。一子蚤卒,田廬沒官。包拯少時,頗為筠所知。及拯顯,奏其族子為後,又請還所沒田廬雲。



 薛映,字景陽,唐中書令元超八世孫,後家於蜀。父允中,
 事孟氏為給事中。歸朝,為尚書都官郎中。映進士及第,授大理評事,歷通判綿、宋、升州,累遷太常丞。王化基薦為監察御史、知開封縣。太宗召對,為江南轉運使,改左正言、直昭文館,為江、淮、兩浙茶鹽制置副使。改京東轉運使,徙河東,兼河西隨軍。求便親,知相州。再領漕京東,積遷尚書禮部郎中,擢知制誥,權判吏部流內銓兼制置群牧使。同梁顥安撫河北,還,權判度支。



 映以右諫議大夫知杭州。映臨決蜂銳,庭無留事。轉運使姚鉉移屬
 州:「當直司毋得輒斷徒以上罪。」映即奏:「徒、流、笞、杖,自有科條,茍情狀明白,何必系獄,以累和氣。請詔天下,凡徒流罪於長吏前對辨,無所異,聽遣決之。」朝廷施用其言。與鉉既不協,遂發鉉納部內女口及鬻銅器抑取其直,又廣市綾羅不輸稅。真宗遣御史臺推勘官儲拱劾鉉,得實,貶連州文學。映坐嘗召人取告鉉狀,當贖金,帝特貰之。



 在杭五年,入知通進、銀臺司兼門下封駁事。封泰山,為東京留守判官,遷給事中、勾當三班院,出知河南
 府。祀汾陰還,駐蹕西京,以映有治狀,賜御書嘉獎。



 遷尚書工部侍郎、集賢院學士、判尚書都省,進樞密直學士、知升州。建言:「州以牛賦民出租,牛死,租不得蠲。」帝覽章矍然,曰:「此朝廷豈知邪?」因令諸州條奏,悉蠲之。頃之,糾察在京刑獄,再判都省。歷尚書左丞、知揚州。徙並州,又徙永興軍,拜工部尚書兼御史中丞。仁宗即位,遷禮部,再為集賢院學士、判院事、知曹州,分司南京。卒,贈右僕射,謚文恭。



 映好學有文,該覽強記,善筆札,章奏尺牘,下
 筆立成。為治嚴明,吏不能欺。每五鼓冠帶,黎明據案決事,雖寒暑,無一日異也。子耀卿秘閣校理,孫紳直龍圖閣。



 論曰:自唐末詞氣浸敝,迄於五季甚矣。先民有言:「政厖土裂,大音不完,必混一而後振。」宋一海內,文治日起。楊億首以辭章擅天下,為時所宗,蓋其清忠鯁亮之氣,未卒大施,悉發於言,宜乎雄偉而浩博也。劉筠後出,能與齊名,氣像似爾,至於文體之今古,時習使然,遑暇議
 是哉。晁迥寬易,與物無忤,父子先後典書命,稱為名臣。薛映學藝、吏術俱優,而挾忿以抉人之私,君子病之。



\end{pinyinscope}