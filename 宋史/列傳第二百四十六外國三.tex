\article{列傳第二百四十六外國三}

\begin{pinyinscope}

 高麗,本曰高句驪。禹別九州,屬冀州之地,周為箕子之國,漢之玄菟郡也。在遼東,蓋扶餘之別種,以平壤城為
 國邑。漢、魏以來,常通職貢,亦屢為邊寇。隋煬帝再舉兵,唐太宗親駕伐之,皆不克。高宗命李勣征之,遂拔其城,分其地為郡縣。唐末,中原多事,遂自立君長。後唐同光、天成中,其主高氏累奉職貢。長興中,權知國事王建承高氏之位,遣使朝貢,以建為玄菟州都督,充大義軍使,封高麗國王。晉天福中,復來朝貢。開運二年,建死,子武襲位。漢乾祐末,武死,子昭權知國事。周廣順元年,遣使朝貢,以昭為特進、檢校太保、使持節、玄菟州都督、大義
 軍使、高麗國王。顯德二年,又遣使來貢,加開府儀同三司、檢校太尉,又加太師。



 建隆三年十月,昭遣其廣評侍郎李興祐、副使李勵希、判官李彬等來朝貢。



 四年春,降制曰:「古先哲後,奄宅中區,曷嘗不同文軌于萬方,覃聲教于四海?顧予涼德,猥被鴻名,爰致賓王,宜優錫命。開府儀同三司、檢校太師、玄菟州都督、充大義軍使、高麗國王昭,日邊鐘粹,遼左推雄,習箕子之餘風,撫朱蒙之舊俗。而能占雲候海,奉贄充庭,言念傾輸,實深嘉尚。是
 用賜之懿號,疇以公田,載推柔遠之恩,式獎拱辰之志。於戲!來朝萬里,美愛戴之有孚。柔撫四封,庶混並之無外。永保東裔,聿承天休。可加食邑七千戶,仍賜推誠順化保義功臣。」其年九月,遣使時贊等來貢,涉海,值大風,船破,溺死者七十餘人,贊僅免,詔加勞恤。



 開寶五年,遣使以方物來獻,制加食邑,賜推誠順化守節保義功臣。進奉使內議侍郎徐熙加檢校兵部尚書,副使內奉卿崔鄴加檢校司農卿並兼御史大夫,判官廣評侍郎
 康禮試少府少監,錄事廣評員外郎劉隱加檢校尚書、金部郎中,皆厚禮遣之。



 昭卒,其子伷權領國事。



 九年,伷遣使趙遵禮奉土貢,以父沒當承襲,來聽朝旨。授伷檢校太保、玄菟州都督、大義軍使,封高麗國王。



 太宗即位,加檢校太傅,改大義軍為大順軍。遣左司御副率於延超、司農寺丞徐昭文使其國。伷遣國人金行成入就學於國子監。



 太平興國二年,遣其子元輔以良馬、方物、兵器來貢。其年,行成擢進士第。



 三年,又遣使貢方物、兵器,加
 伷檢校太師,以太子中允直舍人院張洎、著作郎直史館句中正為使。



 四年,復遣供奉官、閣門祗候王僎使其國。五年六月,再遣使貢方物。六年,又遣使來貢。



 七年,伷卒,其弟治知國事,遣使金全奉金銀線罽錦袍褥、金銀飾刀劍弓矢、名馬、香藥來貢,且求襲位。授治檢校太保、玄菟州都督,充大順軍使,封高麗國王,以監察御史李巨源、《禮記》博士孔維奉使。



 雍熙元年,遣使韓遂齡以方物來貢。二年,加治檢校太傅,遣翰林侍書王著、侍讀呂
 文仲充使。



 三年,出師北伐,以其國接契丹境,常為所侵,遣監察御史韓國華齎詔諭之曰:「朕誕膺丕構,奄宅萬方,華夏蠻貊,罔不率俾。蠢茲北裔,侵敗王略,幽薊之地,中朝土疆,晉、漢多虞,夤緣盜據。今國家照臨所及,書軌大同,豈使齊民陷諸獷俗?今已董齊師旅,殄滅妖氛。惟王久慕華風,素懷明略,效忠純之節,撫禮義之邦。而接彼邊疆,罹於蠆毒,舒泄積憤,其在茲乎!可申戒師徒,迭相犄角,協比鄰國,同力蕩平。奮其一
 鼓之雄,戡比垂亡之寇,良時不再,王其圖之!應俘獲生口、牛羊、財物、器械,並給賜本國將士,用申賞勸。」



 先是,契丹伐女真國,路由高麗之界。女真意高麗誘導構禍,因貢馬來訴於朝,且言高麗與契丹結好,倚為勢援,剽略其民,不復放還。洎高麗使韓遂齡入貢,太宗因出女真所上告急木契以示遂齡,仍令歸白本國,還其所俘之民。治聞之憂懼,及國華至,令人言于國華曰:



 前歲冬末,女真馳木契來告,稱契丹興兵入其封境,恐當道未知,宜豫為之備。當道
 與女真雖為鄰國,而路途遐遠,彼之情偽,素知之矣,貪而多詐,未之信也。其後又遣人告曰,契丹兵騎已濟梅河。當道猶疑不實,未暇營救。俄而契丹雲集,大擊女真,殺獲甚衆,餘族敗散逃遁,而契丹壓背追捕,及於當道西北德昌、德成、威化、光化之境,俘擒而去。時有契丹一騎至德米河北,大呼關城戍卒而告曰:「我契丹之騎也,女真寇我邊鄙,率以為常,今則復仇已畢,整兵回矣。」當道雖聞師退,猶憂不測,乃以女真避兵來奔二千餘衆,
 資給而歸之。



 女真又勸當道控梅河津要,築治城壘,以為防遏之備,亦以為然。方令行視興功,不意女真潛師奄至,殺略吏民,驅掠丁壯,沒為奴隸,轉徙他方。以其歲貢中朝,不敢發兵報怨,豈期反相誣構,以惑聖聽。當道世稟正朔,踐修職貢,敢有二心,交通外國?況契丹介居遼海之外,復有大梅、小梅二河之阻,女真、渤海本無定居,從何徑路,以通往復?橫罹讒謗,憤氣填膺,日月至明,諒垂昭鑒。



 間者,女真逃難之衆,罔不存恤,亦有授以官
 秩,尚在當國,其職位高者有勿屈尼於、舟阝元、尹能達、舟阝老正、衛迦耶夫等十數人。欲望召赴京闕,與當道入貢之使庭辯其事,則丹石之誠,庶幾昭雪。



 國華諾之,乃命發兵西會。治遷延未即奉詔,國華屢督之,得報發兵而還,具錄女真之事以奏焉。十月,遣使朝貢,又遣本國學生崔罕、王彬詣國子監肄業。



 端拱元年,加治檢校太尉,以考功員外郎兼侍御史知雜呂端、起居舍人呂祐之
 為使。



 二年,遣使來貢,詔其使選官侍郎韓藺卿、副使兵官郎中魏德柔並授金紫光祿大夫,判官少府丞李光授檢校水部員外郎。先是,治遣僧如可齎表來覲,請《大藏經》,至是賜之,仍賜如可紫衣,令同歸本國。



 淳化元年三月,詔加治食邑千戶,遣戶部郎中柴成務、兵部員外郎直史館趙化成往使。其國俗信陰陽鬼神之事,頗多拘忌,每朝廷使至,必擇良月吉辰,方具禮受詔。成務在館逾月,乃遺書於治曰:「王奕葉藩輔,尊獎王室,凡行大慶,首被徽章。今國家特馳信使,以申殊寵,非止曆川塗
 之綿邈,亦復蹈溟海之艱危,皇朝眷遇,斯亦隆矣。而乃牽於禁忌,泥于卜數,眩惑日者之浮說,稽緩天子之命書。惟典冊之垂文,非卜祝之能曉,是以《書》稱上日,不推六甲之元辰;《禮》載仲冬,但取一陽之嘉會。粲然古訓,足以明稽,所宜改圖,速拜君賜。儻鳳綍無滯,克彰拱極之誠;則龍節有輝,免貽辱命之責。謹以誠告,王其聽之。」治覽書慚懼,遣人致謝焉。會霖雨不止,仍以俟霽為請。成務復遺書以責之,治翌日乃出拜命。



 二年,遣使韓彥恭
 來貢。彥恭表述治意,求印佛經,詔以《藏經》並御制《秘藏詮》、《逍遙詠》、《蓮華心輪》賜之。



 四年正月,治遣使白思柔貢方物並謝賜經及御制。二月,遣秘書丞直史館陳靖、秘書丞劉式為使,加治檢校太師,仍降詔存問軍吏耆老。靖等自東牟趣八角海口,得思柔所乘海船及高麗水工,即登舟自芝岡島順風泛大海,再宿抵甕津口登陸,行百六十里抵高麗之境曰海州,又百里至閻州,又四十里至白州,又四十里至其國。治迎使於郊,盡藩臣禮,
 延留靖等七十餘日而還,遺以襲衣、金帶、金銀器數百兩、布三萬餘端,附表稱謝。



 先是,三年,上親試諸道貢舉人,詔賜高麗賓貢進士王彬、崔罕等及第,既授以官,遣還本國。至是,靖等使回,治上表謝曰:「學生王彬、崔罕等入朝習業,蒙恩並賜及第,授將仕郎、守秘書省校書郎,仍放歸本國。竊以當道薦修貢奉,多曆歲年,蓋以上國天高,遐荒海隔,不獲躬趨金闕,面叩玉階,唯深拱極之誠,莫展來庭之禮。彬、罕等幼從匏系,嗟混跡於嵎夷;不
 憚蓬飄,早賓王於天邑。縕袍短褐,玉粒桂薪,堪憂食貧,若為卒歲。皇帝陛下天慈照毓,海量優容,豐其館穀之資,勖以藝文之業。去歲高懸軒鑒,大選魯儒,彬、罕接武澤宮,敢萌心於中鵠;濫巾英域,空有志於羨魚。陛下以其萬里辭家,十年觀國,俾登名于桂籍,仍命秩於芸台;憫其懷土之心,慰以倚門之望,別垂宸旨,令歸故鄉。玄造曲成,鴻恩莫報,臣不勝感天戴聖之至。」



 又有張仁銓者,進奉使白思柔之孔目吏也,上書獻便宜。思柔意其
 持國陰事以告,仁銓懼不敢歸。上命靖等領以還國,仍詔治釋仁銓罪。治又上表謝曰:「官告國信使陳靖、劉式至,奉傳聖旨,以當道進奉使從行孔目官張仁銓至闕,輒進便宜,翻懷憂懼,今附使臣帶歸本國者。仁銓嵎宅細民,海門賤吏,獲趨上國,敢貢愚誠,罔思狂瞽之尤,輒奏權宜之事,妄塵旒冕,上黷朝廷。今者,仰奉綸言,釋其罪罟。小人趨利,豈虞僭越之求,聖主寬恩,遠降哀矜之命。其張仁銓者已依詔旨放罪,令掌事如故。」又上言
 願賜板本《九經》書,用敦儒教,許之。



 先是,式等復命,治遣使元證衍送之,證衍至安香浦口,值風損船,溺所齎物。詔登州給證衍文據遣還,仍賜治衣段二百匹、銀器二百兩、羊五十口。



 五年六月,遣使元郁來乞師,訴以契丹寇境。朝廷以北鄙甫寧,不可輕動干戈,為國生事,但賜詔慰撫,厚禮其使遣還。自是受制於契丹,朝貢中絕。



 治卒,弟誦立。嘗遣兵校徐遠來候朝廷德音,遠久不至。



 咸平三年,其臣吏部侍郎趙之遴命牙將朱仁紹至登州偵
 之,州將以聞,上特召見仁紹。因自陳國人思慕皇化,為契丹羈制之狀,乃賜誦函詔一道,令仁紹齎還。



 六年,誦遣使戶部郎中李宣古來朝謝恩,且言:「晉割燕薊以屬契丹,遂有路趣玄菟,屢來攻伐,求取不已,乞王師屯境上為之牽制。」詔書優答之。



 誦卒,弟詢權知國事。先是,契丹既襲高麗,遂築六城曰興州、曰鐵州、曰通州、曰龍州、曰龜州、曰郭州於境上。契丹以為貳己,遣使來求六城,詢不許。遂舉兵,奄至城下,焚蕩宮室,剽劫居人,詢徙
 居升羅州以避之。兵退,乃遣使請和。契丹堅以六城為辭,自是調兵守六城。



 大中祥符三年,大舉來伐,詢與女真設奇邀擊,殺契丹殆盡。詢又於鴨綠江東築城,與來遠城相望,跨江為橋,潛兵以固新城。



 七年,方遣告奏使御事工部侍郎尹證古以金線織成龍鳳鞍並繡龍鳳鞍襆各二幅、細馬二匹、散馬二十匹來貢。證古還,賜詢詔書七通並衣帶、銀彩、鞍勒馬等。



 八年,詔登州置館於海次以待使者。其年,又遣御事民官侍郎郭元來貢。元
 自言:「本國城無垣牆,府曰開城,管六縣,民不下三五千。有州軍百餘,置十路轉運司統之。每州管縣五六,小者亦三四,每縣戶三四百。國境南北千五百里,東西二千里。軍民雜處,隸軍者不黥面。方午為市,不用錢,第以布米貿易。地宜粳稻,風俗頗類中國。無羊、兔、橐駝、水牛、驢。氣候少寒,暑差多。有僧,無道士。民家器皿,悉銅為之。樂有二品:曰唐樂,曰鄉樂。三歲一試舉人,有進士、諸科、算學,每試百餘人,登第者不過一二十。每正月一日、五月
 五日祭祖禰廟。又正月七日,家為王母像戴之。二月望,僧俗燃燈如中國上元節。上巳日,以青艾染餅為盤羞之冠。端午有秋千之戲。士女服尚素。地產龍須席、藤席、白︴紙、鼠狼尾筆。」元辭貌恭恪,每受宴賜,必自為謝表,粗有文采,朝廷待之亦厚。九年,辭還,賜詢詔書七函,襲衣、金帶、器幣、鞍馬及經史、曆日、《聖惠方》等。元又請錄《國朝登科記》及所賜御詩以歸,從之。



 天禧元年,遣御事刑官侍郎徐訥奉表獻方物於崇政殿,又賀封建壽春郡
 王。



 三年九月,登州言高麗進奉使禮賓卿崔元信至秦王水口,遭風覆舟,漂失貢物,詔遣內臣撫之。十一月,元信等入見,貢罽錦衣褥、烏漆甲、金飾長刀匕首、罽錦鞍馬、紵布藥物等,又進中布二千端,求佛經一藏。詔賜經還布,以元信覆溺匱乏,別賜衣服、繒彩焉。明州、登州屢言高麗海船有風漂至境上者,詔令存問,給度海糧遣還,仍為著例。



 五年,詢遣告奏使御事禮部侍郎韓祚等一百七十九人來謝恩,且言與契丹修好,又表乞陰陽
 地理書、《聖惠方》,並賜之。



 金行成者,累官至殿中丞,治表乞放還。行成自以筮仕朝廷,不願歸本國。又以父母垂老,在海外旦暮思念,恨祿不及,令工圖其像置正寢,與妻史氏居旁室,晨夕定省上食,未嘗少懈。淳化初,通判安州。被病,知州李範與僚佐數人省之,行成病已篤,泣且言曰:「行成外國人,為朝官,佐郡政,病且死,未有以報主恩,雖瞑目固有遺恨。二子宗敏、宗訥皆幼,家素貧,無他親可依,旦暮委溝壑矣。」未幾,行成死,其妻養二子,誓
 不嫁,織履以給。範表其事,詔以宗敏補太廟齋郎,令安州月給其家錢三緡、米五斛,長吏歲時存問。



 又高麗信州永甯人康戩,字休祐,父允,三世為兵部侍郎。戩少好學,時紇升與契丹交兵,戩從允戰木葉山下,連中二矢,神色不變。後陷契丹,遁居墨斗嶺,又至黃龍府,間道得歸高麗,時允猶在。開寶中,允遣戩隨賓貢肄業國學。太平興國五年,登進士第,解褐大理評事,知湘鄉縣,再遷著作佐朗,知江陰軍、江州。曆官以清白乾力聞,改太常
 博士。蘇易簡在翰林,稱其吏才,命為廣南西路轉運副使,賜緋魚,就遷正使,再轉度支員外郎、戶部判官。出知峽、越二州,連被詔褒其能政。又為京西轉運使,加工部郎中,賜金紫。戩所至好行事,上章多建白,以竭誠自任。景德三年,卒,真宗特以其子希齡為太常寺奉禮郎,給奉終喪。



 乾興元年二月,祚等辭歸國,賜詢如故事。會真宗晏駕,又齎遺物以賜詢。



 天聖八年,詢復遣御事民官侍郎元穎等二百九十三人奉表入見於長春殿,貢金
 器、銀罽刀劍、鞍勒馬、香油、人參、細布、銅器、硫黃、青鼠皮等物。明年二月辭歸,賜予有差,遣使護送至登州。其後絕不通中國者四十三年。



 詢孫徽嗣立,是為文王。



 熙寧二年,其國禮賓省移牒福建轉運使羅拯云:「本朝商人黃真、洪萬來稱,運使奉密旨,令招接通好。奉國王旨意,形於部述。當國僻居暘谷,邈戀天朝,頃從祖禰以來,素願梯航相繼。蕞爾平壤,邇於大遼,附之則為睦鄰,疏之則為勍敵。慮邊騷之弗息,蓄陸詟以靡遑。久困
 羈縻,難圖攜貳,故違述職,致有積年。屢卜雲祥,雖美聖辰於中國;空知日遠,如迷舊路于長安。運屬垂鴻,禮稽展慶。大朝化覃無外,度豁包荒,山不謝乎纖埃,海不辭於支派。謹當遵尋通道,遄赴稿街,但茲千里之傳聞,恐匪重霄之紆眷。今以公狀附真、萬西遷,俟得報音,即備禮朝貢。」徽又自言嘗夢至中華,作詩紀其事。三年,拯以聞,朝廷議者亦謂可結之以謀契丹,神宗許焉,命拯諭以供擬腆厚之意。徽遂遣民官侍郎金悌等百十人來,詔待之
 如夏國使。



 往時高麗人往反皆自登州,七年,遣其臣金良鑒來言,欲遠契丹,乞改塗由明州詣闕,從之。郡縣供頓無舊准,頗擾民,詔立式頒下,費悉官給。又以其不邇華言,恐規利者私與交關,令所至禁止。徽問遺二府甚厚,詔以付市易務售縑帛答之。又表求醫藥、畫塑之工以教國人,詔羅拯募願行者。



 九年,復遣崔思訓來,命中貴人仿都亭西驛例治館,待之寢厚,其使來者亦益多。嘗獻伶官十餘輩,曰:「夷樂無足觀,止欲潤色國史爾。」帝
 以其國尚文,每賜書詔,必選詞臣著撰而擇其善者。



 元豐元年,始遣安燾假左諫議大夫、陳睦假起居舍人往聘。造兩艦於明州,一曰淩虛致遠安濟,次曰靈飛順濟,皆名為神舟。自定海絕洋而東,既至,國人歡呼出迎。徽具袍笏玉帶拜受詔,與燾、睦尤禮,館之別宮,標曰順天館,言尊順中國如天云。徽已病,僅能拜命,且乞醫藥。



 二年,遣王舜封挾醫往診治。徽又使柳洪來謝,海中遇風,失所貢物。洪上章自劾,敕書安慰。尋獻日本所造車,曰:「諸
 侯不貢車服,故不敢與土貢同進。」前此貢物至。輒下有司估直,償以萬縑,至是命勿復估,以萬縑為定數。



 六年,徽卒,在位三十八年,治尚仁恕,為東夷良主。然猶循其俗,王女不下嫁臣庶,必歸之兄弟,宗族貴臣亦然。次子運諫,以為既通上國,宜以禮革故習。徽怒,斥之於外。訃聞,天子閔焉,詔明州修浮屠供一月,遣楊景略、王舜封祭奠,錢勰、宋球吊慰。景略辟李之儀書狀,帝以之儀文稱不著,宜得問學博洽、器宇整秀者召赴中書,試以文乃
 遣。又以遠服不責其備,諭使者以相見之所殿名、鴟吻,皆聽勿避。



 徽子順王勳嗣,百日卒。弟宣王運嗣。運仁賢好文,內行飭備,每賈客市書至,則潔服焚香對之。



 八年,遣其弟僧統來朝,求問佛法並獻經像。



 哲宗立,遣使金上琦奉慰,林暨致賀,請市刑法之書、《太平御覽》、《開寶通禮》、《文苑英華》。詔惟賜《文苑英華》一書,以名馬、錦綺、金帛報其禮。



 運立四年卒,子懷王堯嗣。未閱歲,以病不能為國,國人請其叔父雞林公熙攝政。未幾堯卒,熙乃立,凡數
 歲使不至。



 元祐四年,其王子義天使僧壽介至杭州祭亡僧,言國母使持二金塔為兩宮壽,知州蘇軾奏卻之,語在《軾傳》。熙後避遼主諱,改名顒。顒性貪吝,好奪商賈利,富室犯法,輒久縻責贖,雖微罪亦輸銀數斤。



 五年,復通使,賜銀器五千兩。七年,遣黃宗愨來獻《黃帝針經》,請市書甚衆。禮部尚書蘇軾言:「高麗入貢,無絲發利而有五害,今請諸書與收買金箔,皆宜勿許。」詔許買金箔,然卒市《冊府元龜》以歸。



 元符中,遣士賓貢。



 徽宗立,遣任懿、
 王嘏來吊賀。



 崇寧二年,詔戶部侍郎劉逵、給事中吳拭往使。



 顒卒,子俁嗣。貢使接踵,且令士子金瑞等五人入太學,朝廷為置博士。



 政和中,升其使為國信,禮在夏國上,與遼人皆隸樞密院;改引伴、押伴官為接送館伴。賜以《大晟燕樂》、籩豆、簠簋、尊罍等器,至宴使者于睿謨殿中。



 宣和四年,俁卒。初,高麗俗兄終弟及,至是諸弟爭立,其相李資深立俁子楷。來告哀,詔給事中路允迪、中書舍人傅墨卿奠慰。俁之在位也,求醫於朝,詔使二醫往,
 留二年而歸,楷語之曰:「聞朝廷將用兵伐遼。遼兄弟之國,存之足為邊捍。女真狼虎耳,不可交也。業已然,願二醫歸報天子,宜早為備。」歸奏其言,已無及矣。



 欽宗立,賀使至明州,御史胡舜陟言:「高麗靡敝國家五十年,政和以來,人使歲至,淮、浙之間苦之。彼昔臣事契丹,今必事金國,安知不窺我虛實以報,宜止勿使來。」乃詔留館於明而納其贄幣。明年始歸國。



 自王微以降,雖通使不絕,然受契丹封冊,奉其正朔,上朝廷及他文書,蓋有稱甲
 子者。歲貢契丹至於六,而誅求不已。常云:「高麗乃我奴耳,南朝何以厚待之?」使至其國,尤倨暴,館伴及公卿小失意,輒行ㄏ棰,聞我使至,必假他事來覘,分取賜物。嘗詰其西向修貢事,高麗表謝,其略曰:「中國,三甲子方得一朝;大邦,一周天每修六貢。」契丹悟,乃得免。



 高宗即位,慮金人通於高麗,命迪功郎胡蠡假宗正少卿為高麗國使以間之。蠡之回,史失書。



 二年,浙東路馬步軍都總管楊應誠上言:「由高麗至女真路甚徑,請身使三韓,結
 雞林以圖迎二聖。」乃以應誠假刑部尚書充高麗國信使。浙東帥臣翟汝文奏言:「應誠欺罔,為身謀耳。若高麗辭以金人亦請問津以窺吳、越,其將何辭以對?萬一辱命,取笑遠夷,願毋遣。」應誠聞之,遂與副使韓衍、書狀官孟健由杭州浮海以行。六月,抵高麗,諭其王楷以所欲為,楷曰:「大朝自有山東路,盍不由登州往?」應誠曰:「以貴國路徑耳。」楷有難色,已而命其門下侍郎傳佾至館中,果對如翟汝文言。應誠曰:「女真不善水戰。」佾曰:「彼常於
 海道往來,況女真舊臣本國,今反臣事之,其強弱可見矣。」居數日,復遣其中書侍郎崔洪宰、知樞密院金富軾持前議不變,謂二聖今在燕雲,大朝雖盡納土,未必可得,何不練兵與戰?終不奉詔。應誠留兩月餘,不得已見楷于壽昌門,受其拜表而還。十月,至闕,入對言狀,上以楷負國恩,怒甚。尚書右丞朱勝非曰:「彼鄰金人,與中國隔海,利害甚明。曩時待之過厚,今安能責其報也。」右僕射黃潛善曰:「以巨艦載精兵數萬,徑搗其國,彼寧不懼。」
 勝非曰:「越海興師,燕山之事可為近鑒。」上怒解。十一月,楷遣其臣尹彥頤奉表謝罪,詔以二聖未歸,燕設不宜用樂,乃設幕殿門外,命客省官吳得興伴賜酒食,命中書舍人張澂押伴,如禮遣還。



 三年八月,上謂輔臣曰:「聞上皇遣內臣、宮女各二人隨高麗貢使來,朕聞之悲喜交集。」呂頤浩曰:「此必金人之意,不然高麗必不敢,安知非窺我虛實以報。」於是詔止之,略曰:「王緬守基圖,夙同文軌,乃附乘桴之信,嗣修貢篚之恭。惟忠順之無他,質
 神明而靡愧,屬關聞聽,良用歎嘉。言念晚年,實為多故,舉中原之生聚,遭強敵之震驚,既涉境以冞深,猶稱兵而未已,茲移仗衛,暫駐江湖。如行使之果來,恐有司之不戒,俟休邊警,當問聘期。壞晉館以納車,庶無後悔,閉漢關而謝質,非用前規。想彼素懷,知吾誠意。」



 紹興元年十月,高麗將入貢,禮部侍郎柳約言:「四明殘破之余,荒蕪單弱,恐起戎心,宜屯重兵以俟其至。」十一月,詔柳約奉使高麗,不果行。



 二年閏四月,楷遣其禮部員外郎崔
 惟清、閣門祗候沈起入貢金百兩、銀千兩、綾羅二百匹、人參五百斤,惟清所獻亦三之一。上御後殿引見,賜惟清、起金帶二,答以溫詔遣還。是月,定海縣言,民亡入高麗者約八十人,願奉表還國。詔候到日,高麗綱首卓榮等量與推恩。十二月,聞高麗遣知樞密院事洪彝敘等六十五人來貢,議以臨安府學館其使。言者謂雖在兵間,不可無學,恐為所窺。詔以法惠寺為同文館以待之。既而卒不至。



 六年,高麗持牒官金稚圭至明州,賜銀帛
 遣之,懼其為金間也。



 三十二年三月,高麗綱首徐德榮詣明州言,本國欲遣賀使。守臣韓仲通以聞,殿中侍御史吳芾奏曰:「高麗與金人接壤,昔稚圭之來,朝廷懼其為間,亟遣還。今兩國交兵,德榮之請,得無可疑?使其果來,猶恐不測,萬一不至,貽笑遠方。」詔止之。



 隆興二年四月,明州言高麗入貢。史不書引見日,恐同彝敘之詐。其後使命遂絕。



 慶元間,詔禁商人持銅錢入高麗,蓋絕之也。



 初,高麗入使,明、越困於供給,朝廷館遇燕齎錫予之
 費以钜萬計,饋其主者不在焉。我使之行,每乘二神舟,費亦不貲。三節官吏縻爵捐廩,皆仰縣官。昔蘇軾言於先朝,謂高麗入貢有五害,以此也。惟是國于吳會,事異東都。昔高麗入使,率由登、萊,山河之限甚遠,今直趨四明,四明距行都限一浙水耳。由海道奉使高麗,彌漫汪洋,洲嶼險阻,遇黑風,舟觸礁輒敗,出急水門至群山島,始謂平達,非數十日不至也。舟南北行,遇順風則歷險如夷,至不數日。其國東西二千里,南北五百里,西北接
 契丹,恃鴨綠江以為固,江廣三百步。其東所臨,海水清澈,下視十丈,東南望明州,水皆碧。



 王居開州蜀莫郡,曰開成府。依大山置宮室,立城壁,名其山曰神嵩。民居皆茅茨,大止兩椽,覆以瓦者才十二。以新羅為東州樂浪府,號東京。百濟為金州金馬郡,號南京。平壤為鎮州,號西京。西京最盛。總之,凡三京、四府、八牧、郡百有十八、縣鎮三百九十、洲島三千七百。郡邑之小者,或只百家。男女二百十萬口,兵、民、僧各居其一。地寒多山,土宜松柏,
 有粳、黍、麻、麥而無秫,以粳為酒。少絲蠶,匹縑直銀十兩,多衣麻紵。王出,乘車駕牛,曆山險乃騎。紫衣行前,捧《護國仁王經》以導。出令曰教,曰宣。臣民呼之曰聖上,私謂曰嚴公,後妃曰宮主。百官名稱、階、勳、功臣、檢校,頗與中朝相類。過御史台則下馬,違者有劾。士人以族望相高,柳、崔、金、李四姓為貴種。無宦者,以世族子為內侍六衛。歲十二月朔,王坐紫門小殿注官,外官則付國相。有國子監、四門學,學者六千人。貢士三等,王城曰土貢,郡邑
 曰鄉貢,他國人曰賓貢。間歲試於所屬,再試於學,所取不過三四十人,然後王親試以詩、賦、論三題,謂之簾前重試。亦有制科宏詞之目,然特文具而已。士尚聲律,少通經。



 王城有華人數百,多閩人因賈舶至者,密試其所能,誘以祿仕,或強留之終身,朝廷使至,有陳牒來訴者,則取以歸。



 百官以米為奉,皆給田,納祿半給,死乃拘之。國無私田,民計口授業。十六以上則充軍,六軍三衛常留官府,三歲以選戍西北,半歲而更。有警則執兵,任事
 則服勞,事已復歸農畝。王亦有分地以供私用,王母、妃主、世子皆受湯沐田。



 上下以賈販利入為事。日中為虛,用米布貿易。地產銅,不知鑄錢,中國所予錢,藏之府庫,時出傳玩而已。崇寧後,始學鼓鑄,有「海東通寶」、「重寶」、「三韓通寶」三種錢,然其俗不便也。兵器疏簡,無強弩大刀。



 崇尚釋教,雖王子弟亦常一人為僧。信鬼,拘陰陽,病不相視,斂不撫棺。貧者死,則露置中野。歲以建子月祭天。國東有穴,號礻遂神,常以十月望日迎祭,謂之八關齋,禮
 儀甚盛,王與妃嬪登樓,大張樂宴飲。賈人曳羅為幕,至百匹相聯以示富。三歲大祭祠,遍其封內,因是斂民財,而王與諸臣分取之。祖廟在國門之外,大祭則具車服冕圭親祠。王城有佛寺七十區而無道觀,大觀中,朝廷遣道士往,乃立福源院,置羽流十餘輩。俗不知醫,自王俁來請醫,後始有通其術者。



 人首無枕骨,背扁側。男子巾幘如唐裝,婦人鬢髻垂右肩,餘發被下,約以絳羅,貫之簪。旋裙重疊,以多為勝。男女自為夫婦者不禁,夏月
 同川而浴。婦人、僧、尼皆男子拜。樂聲甚下,無金石之音。既賜樂,乃分為左、右二部:左曰唐樂,中國之音也;右曰鄉樂,其故習也。堂上設席,升必脫屨,見尊者則膝行,必跪,應必唯。其拜無不答,子拜,父猶半答其禮。性仁柔惡殺,不屠宰,欲食羊豕則包以蒿而燔之。



 刑無慘酷之科,唯惡逆及罵父母者斬,餘皆杖肋。外郡刑殺悉送王城,歲以八月減囚死罪,貸流諸島,累赦,視輕重原之。



 自明州定海遇便風,三日入洋,又五日抵墨山,入其境。自墨
 山過島嶼,詰曲礁石間,舟行甚駛,七日至禮成江。江居兩山間,束以石峽,湍激而下,所謂急水門,最為險惡。又三日抵岸,有館曰碧瀾亭,使人由此登陸,崎嶇山谷四十餘里,乃其國都云。



 Category:朝鮮



\end{pinyinscope}