\article{列傳第二百二文苑五}

\begin{pinyinscope}

 ○梅堯臣江休復蘇洵章望之王逢孫唐卿黃庠楊寘附唐庚史伯虎附文同楊傑
 賀鑄劉涇鮑由黃伯思



 梅堯臣,字聖俞,宣州宣城人,侍讀學士詢從子也。工為詩,以深遠古淡為意,間出奇巧,初未為人所知。用詢陰為河南主簿,錢惟演留守西京,特嗟賞之,為忘年交,引與酬倡,一府盡傾。歐陽修與為詩友,自以為不及。堯臣益刻厲,精思苦學,繇是知名於時。宋興,以詩名家為世所傳如堯臣者,蓋少也。嘗語人曰:「凡詩,意新語工,得前
 人所未道者,斯為善矣。必能狀難寫之景如在目前,含不盡之意見於言外,然後為至也。」世以為知言。歷德興縣令,知建德、襄城縣,監湖州稅,簽書忠武、鎮安判官,監永豐倉。大臣屢薦宜在館閣,召試,賜進士出身,為國子監直講,累遷尚書都官員外郎。預修《唐書》,成,未奏而卒,錄其子一人。



 寶元、嘉祐中,仁宗有事郊廟,堯臣預祭,輒獻歌詩,又嘗上書言兵。注《孫子》十三篇,撰《唐載記》二十六卷、《毛詩小傳》二十卷、《宛陵集》四十卷。



 堯臣家貧,喜飲
 酒,賢士大夫多從之游,時載酒過門。善談笑,與物無忤,詼嘲刺譏托於詩,晚益工。有人得西南夷布弓衣,其織文乃堯臣詩也,名重於時如此。



 江休復,字鄰幾,開封陳留人。少強學博覽,為文淳雅,尤善於詩。喜琴、弈、飲酒,不以聲利為意。進士起家,為桂陽監藍山尉,騎驢之官,每據鞍讀書至迷失道,家人求得之。舉書判拔萃,改大理寺丞,遷殿中丞。獻其所著書,召試,為集賢校理,判尚書刑部。與蘇舜欽游,坐預進奏院
 祠神會落職,監蔡州商稅。久之,知奉符縣,通判睦州,徙廬州,復集賢校理,判吏部南曹、登聞鼓院,為群牧判官,出知同州,提點陜西路刑獄,入判三司鹽鐵勾院,修起居注,累遷尚書刑部郎中,卒。



 休復外簡曠而內行甚飾,事孀姑如母,所與游皆一時豪俊。為政簡易。嘗著《神告》一篇,言皇嗣未立,假神告祖宗之意,冀以感悟。又嘗言昭憲太后子孫多流落民間,宜甄錄之。著《唐宜鑒》十五卷、《春秋世論》三十卷、文集二十卷。



 蘇洵,字明允,眉州眉山人。年二十七始發憤為學,歲餘舉進士,又舉茂才異等,皆不中。悉焚常所為文,閉戶益讀書,遂通《六經》、百家之說,下筆頃刻數千言。至和、嘉祐間,與其二子軾、轍皆至京師,翰林學士歐陽修上其所著書二十二篇,既出,士大夫爭傳之,一時學者競效蘇氏為文章。所著《權書》、《衡論》、《機策》,文多不可悉錄,錄其《心術》、《遠慮》二篇。



 《心術》曰:



 為將之道,當先治心,太山覆於前而色不變,麋鹿興於左而目不瞬,然後可以待敵。凡兵
 上義,不義雖利不動。夫惟義可以怒士,士以義怒,可與百戰。凡戰之道,未戰養其財,將戰養其力,既戰養其氣,既勝養其心。謹烽燧,嚴斥堠,使耕者無所顧忌,所以養其財,豐犒而優游之,所以養其力。小勝益急,小挫益厲,所以養其氣。用人不盡其所為,所以養其心。故士當蓄其怒、懷其欲而不盡。怒不盡則有餘勇,欲不盡則有餘貪,故雖並天下而士不厭兵,此黃帝所以七十戰而兵不殆也。



 凡將欲智而嚴,凡士欲愚。智則不可測,嚴則不可犯,故士
 皆委己而聽命,夫安得不愚?夫惟士愚而後可與之皆死。凡兵之動,知敵之主,知敵之將,而後可以動於嶮。鄧艾縋兵於穴中,非劉禪之庸,則百萬之師可以坐縛,彼固有所侮而動也。故古之賢將,能以兵嘗敵,而又以敵自嘗,故去就可以決。



 凡主將之道,知理而後可以舉兵,知勢而後可以加兵,知節而後可以用兵。知理則不屈,知勢則不沮,知節則不窮。見小利不動,見小患不遷,小利小患不足以辱吾技也,夫然後有以支大利大患。夫
 惟養技而自愛者無敵於天下,故一忍可以支百勇,一靜可以制百動。



 兵有長短,敵我一也。敢問:「吾之所長,吾出而用之,彼將不與吾校;吾之所短,吾斂而置之,彼將強與吾角。奈何?」曰:「吾之所短,吾抗而暴之,使之疑而卻;吾之所長,吾陰而養之,使之狎而墮其中。此用長短之術也。」



 善用兵者使之無所顧,有所恃。無所顧則知死之不足惜,有所恃則知不至於必敗。尺箠當猛虎,奮呼而操擊,徒手遇蜥蜴,變色而卻步,人之情也,知此者可以將矣。
 袒裼而按劍,則烏獲不敢逼;冠胄衣甲據兵而寢,則童子彎弓殺之矣。故善用兵者以形固,夫能以形固,則力有餘矣。



 《遠慮》曰:



 聖人之道,有經、有權、有機,是以有民、有群臣而又有腹心之臣。曰經者,天下之民舉知之可也;曰權者,民不可得而知矣,群臣知之可也;曰機者,雖群臣亦不得而知之矣,腹心之臣知之可也。夫使聖人無權,則無以成天下之務,無機,則無以濟萬世之功,然皆非天下之民所宜知;而機者又群臣所不得聞,群臣不
 得聞,則誰與議?不議不濟,然則所謂腹心之臣者,不可一日無也。後世見三代取天下以仁義,而守之以禮樂也,則曰「聖人無機」。夫取天下與守天下,無機不能。顧三代聖人之機,不若後世之詐,故後世不得見。



 其有機也,是以有腹心之臣。禹有益,湯有伊尹,武王有太公望,是三臣者,聞天下之所不聞,知群臣之所不知。禹與湯武倡其機於上,而三臣者和之於下,以成萬世之功。下而至於桓、文,有管仲、狐偃為之謀主,闔廬有伍員,勾踐有
 範蠡、大夫種。高祖之起也,大將任韓信、黥布、彭越,裨將任曹參、樊噲、滕公、灌嬰,游說諸侯任酈生、陸賈、樅公,至於奇機密謀,君臣所不與者,唯留侯、酂侯二人。唐太宗之臣多奇才,而委之深、任之密者,亦不過曰房、杜。夫君子為善之心與小人為惡之心一也,君子有機以成其善,小人有機以成其惡。有機也,雖惡亦或濟,無機也,雖善亦不克,是故腹心之臣不可以一日無也。司馬氏,魏之賊也,有賈充之徒為之腹心之臣以濟,陳勝、吳廣,秦
 民之湯、武也,無腹心之臣以不克。何則?無腹心之臣,無機也,有機而洩也。夫無機與有機而洩者,譬如虎豹食人而不知設陷阱,設陷阱而不知以物覆其上者也。



 或曰:「機者,創業之君所假以濟耳,守成之世,其奚事機而安用夫腹心之臣?」嗚呼!守成之世,能遂熙然如太古之世矣乎?未也,吾未見機之可去也。且夫天下之變,常伏於安,田文所謂「子少國危,大臣未附」,當是之時,而無腹心之臣,可為寒心哉!昔者高祖之末,天下既定矣,而又以
 周勃遺孝惠、孝文;武帝之末,天下既治矣,而又以霍光遺孝昭、孝宣。蓋天下雖有泰山之勢,而聖人常以累卵為心,故雖守成之世,而腹心之臣不可去也。



 《傳》曰:「百官總己以聽於塚宰。」彼塚宰者,非腹心之臣,天子安能舉天下之事委之,三年不置疑於其間邪?又曰:「五載一巡狩。」彼無腹心之臣,五載一出,捐千里之畿,而誰與守邪?今夫一家之中必有宗老,一介之士必有密友,以開心胸,以濟緩急,奈何天子而無腹心之臣乎?近世之君抗
 然於上,而使宰相眇然於下,上下不接,而其志不通矣。臣視君如天之遼然而不可親,而君亦如天之視人,泊然無愛之之心也。是以社稷之憂,彼不以為憂,君憂不辱,君辱不死。一人譽之則用之,一人毀之則舍之。宰相避嫌畏譏且不暇,何暇盡心以憂社稷?數遷數易,視相府如傳舍。百官泛泛於下,而天子惸惸於上,一旦有卒然之憂,吾未見其不顛沛而殞越也。聖人之任腹心之臣也,尊之如父師,愛之如兄弟,執手入臥內,同起居寢食,
 知無不言,言無不盡。百人譽之不加密,百人毀之不加疏,尊其爵,厚其祿,重其權,而後可與議天下之機,慮天下之變。



 宰相韓琦見其書,善之,奏於朝,召試舍人院,辭疾不至,遂除秘書省校書郎。會太常修纂建隆以來禮書,乃以為霸州文安縣主簿,與陳州項城令姚闢同修禮書,為《太常因革禮》一百卷。書成,方奏未報,卒。賜其家縑、銀二百,子軾辭所賜,求贈官,特贈光祿寺丞,敕有司具舟載其喪歸蜀。有文集二十卷、《謚法》三卷。



 章望之,字表民,建州浦城人。少孤,喜問學,志氣宏放。為文辯博,長於議論。初由伯父得象蔭為秘書省校書郎,監杭州茶庫。逾年辭疾去,求舉賢有方正,得象在相位,以嫌扼之,乃上書論時政凡萬餘言,不報。丁母憂,毀瘠過制。服除,浮游江、淮間,犯艱苦,汲汲以營衣食,不自悔,人勸之仕,不應也。其兄拱之知晉江縣,忤其守蔡襄,襄怒,誣以贓,貶。望之號泣,力訴於朝。時襄方貴顯,事久不得直。望之訴不已,章十餘上,起獄數年,朝廷為再劾,卒脫
 拱之冤,復官如初,望之遂不復仕。覃恩遷太常寺太祝、大理評事。翰林學士歐陽修、韓絳、知制誥吳奎劉敞、範鎮同薦其才,宰相欲稍用之,除簽書建康軍節度判官,不赴。又除知烏程縣,趣令受命,固辭,遂以光祿寺丞致仕,卒。



 望之喜議論,宗孟軻言性善,排荀卿、揚雄、韓愈、李翱之說,著《救性》七篇。歐陽修論魏、梁為正統,望之以為非,著《明統》三篇。江南人李覯著《禮論》,謂仁、義、智、信、樂、刑、政皆出於禮,望之訂其說,著《禮論》一篇。其議論多有過人
 者。嘗北游齊、趙,南泛湖、湘,西至汧、隴,東極吳會,山水勝處,無所不歷。有歌詩、雜文數百篇,集為三十卷。



 王逢,字會之,太平州當塗人。其四世祖居巖,仕唐為驍衛長史,遭亂棄官,歸居青山。楊行密據淮南,使人以兵迫起之。居巖散遣其家人,而以一身歸行密,授以湖州別駕,不遣。一日,行密大會,失居巖,亟使人掩其家,無一人在者。其後有人於嵩山見空石室,詢其旁,或云有道人王居巖居此,去而莫知其所終。子孫仕無顯者,至逢,
 博學能屬文,尤長於講說。



 少舉進士不中,去,教授蘇州,學者嘗數百人。晚始登第,補南雄州軍事判官,歸為國子監直講兼隴西郡王宅教授,李瑋從學,事之甚謹。岐國公主既降,瑋為逢求遷官,且有命,逢辭不受。久之,以太常博士通判徐州,卒。逢為人樂易,篤於朋友,與胡瑗最善。喜著書,有《易傳》十卷、《乾德指說》一卷、《復書》七卷。妻陳氏亦有賢行,無子。



 孫唐卿,字希元,青州人。少有學行,年十七,以書謁韓琦,
 琦甚器之。與黃庠、楊寘自景祐以來俱以進士為舉首,有名一時。唐卿初中第,通判陜州,於吏事若素習。民有母再適人而死,及葬其父,恨母之不得祔,乃盜母之喪而同葬之。有司論以法,唐卿時權府事,乃曰:「是知有孝而不知有法爾。」乃釋之以聞。未幾,丁父憂,毀瘠嘔血而卒。詔賻其家。



 黃庠字長善,洪州分寧人。博學強記,超敏過人。初至京師,就舉國子監、開封府、禮部,皆為第一。比引試崇政殿,
 以疾不得入,天子遣內侍即邸舍撫問,賜以藥劑。是時庠名聲動京師,所作程文,傳誦天下,聞於外夷,近世布衣罕比也。歸江南五年,以病卒。



 楊寘字審賢,察之弟。少有雋才,慶歷二年舉進士京師,試國子監、禮部皆第一。既試崇政殿,帝臨軒啟封,見名,喜動於色,謂輔臣曰:「楊寘也。」遂擢第一,公卿相賀為得人。授將作監丞、通判穎州。未至官,持母喪,病羸卒,特詔賻恤其家。先是,其友夢寘作龍首山人,寘自謂:「龍首,我
 四冠多士;山人,無祿位之稱。我其終是乎!」已而果然。



 唐庚,字子西,眉州丹棱人也。善屬文,舉進士,稍為宗子博士,張商英薦其才,除提舉京畿常平。商英罷相,庚亦坐貶,安置惠州。會赦,復官承議郎,提舉上清太平宮。歸蜀,道病卒。年五十一。庚為文精密,通於世務,作《名治》、《察言》、《閔俗》、《存舊》、《內前行》諸篇,時人稱之。有文集二十卷。子文若,自有傳。



 庚兄弟五人,長兄瞻,字望之,後改名伯虎,字長孺。治《易》、《春秋》,皆有家法。元祐三年,其父游瀘南,伯虎
 兄弟居母喪於丹山,伯虎夜半蹴庚曰:「吾夢收父書,發之,得『亟來』二字,吾父得無他乎?吾心動矣。汝奉母奠朝夕,吾趨瀘南。」庚未及應,伯虎奮曰:「吾決矣!」起裹糧,黎明走洪川僦舟,遇江漲,聲搖數十里,客舟皆艤岸不敢動,伯虎徬徨堤上,有漁者持小艇系港中,啖以厚利,不許。伯虎超入艇中,叱僕夫解維,漁者不得已,從之。二日半至瀘南,父果病甚,見伯虎,大驚,問其故,具告之。父嘆曰:「天告汝也!」是日,疾少間,伯虎具舟侍父以歸。居數日,疾
 復作,遂卒。



 元符二年,庚以貢舉事系獄臨邛,語連伯虎,臨邛並械之。凡對吏逾年,掠治無完膚,其詞確然,一不及庚,以故獄久不具,卒會赦,除之。伯虎性真率,無威儀,人多易之,至是皆大服,以為不可及。伯虎仕於四方,每數年一歸,不過旬日復去。後卒於家,有子二人。



 文同,字與可,梓州梓潼人,漢文翁之後,蜀人猶以「石室」名其家。同方口秀眉,以學名世,操韻高潔,自號笑笑先生。善詩、文、篆、隸、行、草、飛白。文彥博守成都,奇之,致書同
 曰:「與可襟韻灑落,如晴雲秋月,塵埃不到。」司馬光、蘇軾尤敬重之。軾,同之從表弟也。同又善畫竹,初不自貴重,四方之人持縑素請者,足相躡於門。同厭之,投縑於地,罵曰:「吾將以為襪。」好事者傳之以為口實。初舉進士,稍遷太常博士、集賢校理,知陵州,又知洋州。元豐初,知湖州,明年,至陳州宛丘驛,忽留不行,沐浴衣冠,正坐而卒。



 崔公度嘗與同同為館職,見同京南,殊無言,及將別,但云:「明日復來乎?與子話。」公度意以「話」為「畫」,明日再往,同
 曰:「與公話。」則左右顧,恐有聽者。公度方知同將有言,非畫也。同曰:「吾聞人不妄語者,舌可過鼻。」即吐其舌,三疊之如餅狀,引之至眉間,公度大驚。及京中傳同死,公度乃悟所見非生者。有《丹淵集》四十卷行於世。



 楊傑,字次公,無為人。少有名於時,舉進士。元豐中,官太常者數任,一時禮樂之事,皆預討論。嘗議玉牒帝系自僖祖而上,世次莫知,則僖祖為始祖無疑,宜以僖祖配感生帝。又請孝惠賀後、淑德尹後、章懷潘後皆祖宗首
 納之後,孝章宋後嘗母儀天下,升祔之禮,久而未講,宜因慈聖光獻崇配之日,升四后神主祔於祖宗祏室,斷天下之大疑,正宗廟之大法。由是四後始得升祔。



 神宗詔秘書監劉幾、禮部侍郎範鎮議樂,幾請命傑同議。傑言大樂七失,並圖上之。神宗下幾、鎮參定,鎮不用傑議,自制。樂成,詔褒之。元豐末,晉州教授陸長愈言:「近封孟軻鄒國公,宜春秋釋奠,與顏子並配。」下太常議,傑與少卿葉均、博士盛陶、王古、辛公佐以謂凡配享從祀,皆孔
 子同時之人,今以孟軻並配非是。禮部復言:「自唐至今,以伏勝、高堂生等二十一賢從祀,豈必同時人?」詔從禮部議。



 哲宗即位,議樂,又用範鎮說。傑復破鎮樂章曲名、宮架加磬、十六鐘磬之非。又論鎮以黑黍用秠制律、銅量,叩之不合黃鐘,以世無真黍,用太府尺為樂尺,下舊樂三律。詳具《樂志》。傑在神宗時與鎮異議,至是復攻之,鎮之樂律卒不用。元祐中,為禮部員外郎,出知潤州,除兩浙提點刑獄,卒,年七十。自號無為子,有文集二十餘卷,《樂
 記》五卷。



 賀鑄,字方回,衛州人,孝惠皇后之族孫。長七尺,面鐵色,眉目聳拔。喜談當世事,可否不少假借,雖貴要權傾一時,小不中意,極口詆之無遺辭,人以為近俠。博學強記,工語言,深婉麗密,如次組繡。尤長於度曲,掇拾人所棄遺,少加隱括,皆為新奇。嘗言:「吾筆端驅使李商隱、溫庭筠常奔命不暇。」諸公貴人多客致之,鑄或從或不從,其所不欲見,終不貶也。



 初,娶宗女,隸籍右選,監太原工作,
 有貴人子同事,驕倨不相下。鑄廉得盜工作物,屏侍吏,閉之密室,以杖數曰:「來,若某時盜某物為某用,某時盜某物入於家,然乎?」貴人子惶駭謝「有之」。鑄曰:「能從吾治,免白發。」即起自袒其膚,杖之數下,貴人子叩頭祈哀,即大笑釋去。自是諸挾氣力頡頏者,皆側目不敢仰視。是時,江、淮間有米芾以魁岸奇譎知名,鑄以氣俠雄爽適相先後,二人每相遇,瞋目抵掌,論辯鋒起,終日各不能屈,談者爭傳為口實。



 元祐中,李清臣執政,奏換通直郎、
 通判泗州,又倅太平州。竟以尚氣使酒,不得美官,悒悒不得志,食宮祠祿,退居吳下,稍務引遠世故,亦無復軒輊如平日。家藏書萬餘卷,手自校讎,無一字誤,以是杜門將遂其老。家貧,貸子錢自給,有負者,輒折券與之,秋毫不以丐人。



 鑄所為詞章,往往傳播在人口。建中靖國時,黃庭堅自黔中還,得其「江南梅子」之句,以為似謝玄暉。其所與交,終始厚者,惟信安程俱。鑄自裒歌詞,名《東山樂府》,俱為序之。嘗自言唐諫議大夫知章之後,且推
 本其初,出王子慶忌,以慶為姓,居越之湖澤所謂鏡湖者,本慶湖也,避漢安帝父清河王諱,改為賀氏,慶湖亦轉為鏡。當時不知何所據。故鑄自號慶湖遺老,有《慶湖遺老集》二十卷。



 劉涇,字巨濟,簡州陽安人。舉進士,王安石薦其才,召見,除經義所檢討。久之,為太學博士,罷知咸陽縣,常州教授,通判莫州、成都府,除國子監丞,知處、虢、真、坊四州。元符末上書,召對,除職方郎中。卒,年五十八。涇為文務奇
 怪語,好進取,多為人排斥,屢躓不伸。



 同時有鄭少微者,字明舉,成都人也,與涇俱以文知名,而仕不偶。



 鮑由,字欽止,處州龍泉人。舉進士。嘗從王安石學,又親炙蘇軾,故其文汪洋閎肆,詩尤高妙。徽宗召對,除工部員外郎,居無何,以不合去,責監泗州轉般倉。歷河東、福建路常平、廣西、淮南轉運判官,復召為郎。以言者罷,提點元封觀。起知明州,又知海州,復奉祠。卒,年五十六。嘗注杜甫詩,有文集五十卷。



 黃伯思,字長睿,其遠祖自光州固始徙閩,為邵武人。祖履,資政殿大學士。父應求,饒州司錄。伯思體弱,如不勝衣,風韻灑落,飄飄有凌雲意。自幼警敏,不好弄,日誦書千餘言。每聽履講經史,退與他兒言,無遺誤者。嘗夢孔雀集於庭,覺而賦之,詞採甚麗。以履任為假承務郎。甫冠,入太學,校藝屢占上游。履將以恩例奏增秩,伯思固辭,履益奇之。元符三年,進士高等,調磁州司法參軍,久不任,改通州司戶。丁內艱,服除,除河南府戶曹參軍,治劇
 不勞而辦。秩滿,留守鄧洵武闢知右軍巡院。



 伯思好古文奇字,洛下公卿家商、周、秦、漢彞器款識,研究字畫體制,悉能辨正是非,道其本末,遂以古文名家,凡字書討論備盡。初,淳化中博求古法書,命待詔王著續正法帖,伯思病其乖偽龐雜,考引載籍,咸有依據,作《刊誤》二卷。由是篆、隸、正、行、草、章草、飛白皆至妙絕,得其尺牘者,多藏弆。



 又二年,除詳定《九域圖志》所編修官兼《六典》檢閱文字,改京秩。尋監護崇恩太后園陵使司,掌管箋奏。以
 修書恩,升朝列,擢秘書省校書郎。未幾,遷秘書郎。縱觀冊府藏書,至忘寢食,自《六經》及歷代史書、諸子百家、天官地理、律歷卜筮之說無不精詣。凡詔講明前世典章文物、集古器考定真贗,以素學與聞,議論發明居多,館閣諸公自以為不及也。逾再考,丁外艱,宿抱羸瘵,因喪尤甚。服除,復舊職。



 伯思頗好道家,自號雲林子,別字霄賓。及至京,夢人告曰:「子非久人間,上帝有命典司文翰。」覺而書之。不逾月,以政和八年卒,年四十。伯思學問慕揚
 雄,詩慕李白,文慕柳宗元。有文集五十卷、《翼騷》一卷。



 二子:詔,右宣教郎、荊湖南路安撫司書寫機宜文字;言乃,右從事郎、福州懷安尉,裒伯思平日議論題跋為《東觀餘論》三卷。



\end{pinyinscope}