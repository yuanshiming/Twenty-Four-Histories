\article{列傳第一百五十三}

\begin{pinyinscope}

 ○胡紘何澹林慄高文虎陳自強鄭丙京鏜謝深甫許及之梁汝嘉



 胡紘字應期,處州遂昌人。淳熙中,舉進士。紹熙五年,以
 京鏜薦,監都進奏院,遷司農寺主簿、秘書郎。韓侂胄用事,逐朱熹、趙汝愚,意猶未快,遂擢紘監察御史。



 紘未達時,嘗謁朱熹於建安,熹待學子惟脫粟飯,遇紘不能異也。紘不悅,語人曰:「此非人情。只雞尊酒,山中未為乏也。」遂亡去。及是,劾趙汝愚,且詆其引用朱熹為偽學罪首。汝愚遂謫永州。



 汝愚初抵罪去國,搢紳大夫與夫學校之士,皆憤悒不平,疏論甚眾。侂胄患之,以汝愚之門及朱熹之徒多知名士,不便於己,欲盡去之,謂不可一一
 誣以罪,則設為偽學之目以擯之。用何澹、劉德秀為言官,專擊偽學,然未有誦言攻熹者。獨稿草疏將上,會改太常少卿,不果。沈繼祖以追論程頤得為察官,紘遂以槁授之。繼祖論熹,皆紘筆也。



 寧宗以孝宗嫡孫行三年服,紘言止當服期。詔侍從、臺諫、給舍集議釋服,於是徙紘太常少卿,使草定其禮。既而親饗太廟。



 紘既解言責,復入疏云:「比年以來,偽學猖獗,圖為不軌,動搖上皇,詆誣聖德,幾至大亂。賴二三大臣、臺諫出死力而排之,故
 元惡殞命,群邪屏跡。自御筆有『救偏建中』之說,或者誤認天意,急於奉承,倡為調停之議,取前日偽學之奸黨次第用之,以冀幸其他日不相報復。往者建中靖國之事,可以為戒,陛下何未悟也。漢霍光廢昌邑王賀,一日而誅群臣一百餘人;唐五王不殺武三思,不旋踵而皆斃於三思之手。今縱未能盡用古法,亦宜且令退伏田里,循省愆咎。」俄遷紘起居舍人。詔偽學之黨,宰執權住進擬,用紘言也。自是學禁益急。進起居郎,權工部侍郎,
 移禮部,又移吏部。坐同知貢舉、考宏詞不當而罷。未幾,學禁漸弛,紘亦廢棄,卒於家。



 何澹,字自然,處州龍泉人。乾道二年進士,累官至國子司業,遷祭酒,除兵部侍郎。光宗內禪,拜右諫議大夫兼侍講。



 澹本周必大所厚,始為學官,二年不遷,留正奏遷之。澹憾必大,及長諫垣,即劾必大,必大遂策免。澹嘗與所善劉光祖言之,光祖曰:「周丞相豈無可論,第其門多佳士,不可並及其所薦者。」澹不聽。



 時姜特立、譙熙載以
 春坊舊恩頗用事。一日,光祖過澹,因語澹曰:「曾、龍之事不可再。」澹曰:「得非姜、譙之謂乎?」既而澹引光祖入便坐,則皆姜、譙之徒也,光祖始悟澹謾諾。明年,澹同知貢舉,光祖除殿中侍御史,首上學術邪正之章。及奏名,光祖被旨入院拆號,與澹席甫逼。澹曰:「近日風採一新。」光祖曰:「非立異也,但嘗為大諫言者,今日言之耳。」既出,同院謂光祖曰:「何自然見君所上章,數夕恍惚,餌定志丸,他可知也。」進御史中丞。



 澹有本生繼母喪,乞有司定所服,
 禮寺言當解官,澹引不逮事之文,乞下給、諫議之。太學生喬嚞、朱有成等移書於澹,謂:「足下自長臺諫,此綱常之所系也。四十餘年以所生繼母事之,及其終也,反以為生不逮而不持心喪可乎?奉常禮所由出,顧以臺諫、給舍議之,識者有以窺之矣。」澹乃去。終制,除煥章閣學士、知泉州,移明州。



 寧宗即位,朱熹、彭龜年以論韓侂胄俱絀,澹還為中丞,怨趙汝愚不援引。汝愚時已免相,復詆其廢壞壽皇良法美意,汝愚落職罷祠。又言:「專門之
 學,流而為偽。願風厲學者,專師孔、孟,不得自相標榜。」除同知樞密院事、參知政事,遷知樞密院。



 吳曦賄通時宰,規圖帥蜀,未及賄澹,韓侂胄已許之,澹持不可。侂胄怒曰:「始以君肯相就,黜偽學,汲引至此,今顧立異耶?」以資政殿大學士提舉洞霄宮。起知福州。澹居外,常怏怏失意,以書祈侂胄,有曰:「跡雖東冶,心在南園。」南園,侂胄家圃也。侂胄憐之。進觀文殿學士,尋移知隆興府。後除江、淮制置大使兼知建康府,移使湖北,兼知江陵。奉祠卒,
 贈少師。



 澹美姿容,善談論,少年取科名,急於榮進,阿附權奸,斥逐善類,主偽黨之禁,賢士為之一空。其怕更化,兇黨俱逐,澹以早退幸免,優游散地幾二十年。



 林慄字黃中,福州福清人。登紹興十二年進士第,調崇仁尉,教授南安軍。宰相陳康伯薦為太學正,守太常博士。孝宗即位,遷屯田員外郎、皇子恭王府直講。



 時金人請和,約為叔侄之國,且以歸疆為請。慄上封事言:「前日之和,誠為非計。然徽宗梓宮、慈寧行殿在彼,為是而屈,
 猶有名焉。今日之和,臣不知其說也。宗廟之仇,而事之以弟侄,其忍使祖宗聞之乎!無唐、鄧,則荊、襄有齒寒之憂;無泗、海,則淮東之備達於真、楊,海道之防遍於明、越矣。議者皆言和戎之幣少,養兵之費多,不知講和之後,朝廷能不養兵乎?今東南民力,陛下之所知也,朝廷安得而不較乎?且非徒無益而已。與之歲幣,是畏之矣。三軍之情,安得不懈弛;歸正之心,安得不攜貳。為今日計,宜停使勿遣,遷延其期。比至來春,別無動息,徐於境上
 移書,諭以兩國誓言。敗之自彼,信不由衷,雖盟無益。自今宜守分界,休息生靈,不煩聘使之往來,各保疆場之無事,焉用疲弊州縣,以奉犬羊之使乎?」



 孝宗懲創紹興權臣之弊,躬攬權綱,不以責任臣下,慄言:「人主蒞權,大臣審權,爭臣議權,王侯、貴戚善撓權者也,左右近習善竊權者也。權在大臣,則大臣重;權在邇臣,則邇臣重;權在爭臣,則爭臣重。是故人主常患權在臣下,必欲收攬而獨持之,然未有能獨持之者也。不使大臣持之,則王
 侯、貴戚得而持之矣;不使邇臣審之,爭臣議之,則左右近習得而議之矣。人主顧謂得其權而自執之,豈不誤哉。是故明主使人持權而不以權與之,收攬其權而不肯獨持之。」至有「以鹿為馬、以雞為鸞」之語。方奉對時,讀至「人主常患權在臣下,必欲收攬而獨持之」,孝宗稱善,慄徐曰:「臣意尚在下文。」執政有訴於孝宗曰:「林慄謂臣等指鹿為馬,臣實不願與之同朝。」乃出知江州。



 有旨省並江州屯駐一軍,慄奏:「辛巳、甲申,金再犯兩淮,賴江州
 一軍分布防托,故舒、蘄、黃三州獨不被寇。本州上至鄂渚七百里,下至池陽五百里;平時屯戍,誠哲無益,萬一有警,鄂渚之戍,上越荊、襄,池陽之師,下流增備,中間千里藩籬,誠為虛闕。無以一夫之議,而廢長江千里之防。」由是軍得無動。



 以吏部員外郎召。冬至,有事南郊,前期十日,百執事聽誓戒;會廢節,有旨上壽不用樂,迨宴金使,乃有權用樂之命。慄以為不可,致書宰相,不聽,乃乞免充舉冊官,以狀申朝廷曰:「若聽樂則廢齋,廢齋則不
 敢以祭。祖宗二百年事天之禮,今因一介行人而廢之。天之可畏,過於外夷遠矣。」不聽。



 兼皇子慶王府直講,有旨令二王非時招延講讀官,相與議論時政,期盡規益。慄以為不可,疏言:「漢武帝為戾太子開博望苑,卒敗太子;唐太宗為魏王泰立文學館,卒敗魏王。古者教世子與吾祖宗之所以輔導太子、諸王,惟以講經讀史為事,他無預焉。若使議論時政,則是對子議父,古人謂之無禮,不可不留聖意。」



 除右司員外郎,遷太常少卿。太廟祫
 享之制,始祖東向,昭南向,穆北向,別廟神主祔於祖姑之下,隨本室南北向而無西向之位。紹興、乾道間,懿節、安穆二後升祔,有司設幄西向。逮安恭皇后新祔,有司承前失,其西向之位,幾與僖祖相對。慄辨正之。



 除直寶文閣、知湖州。慄朝辭,曰:「臣聞漢人賈誼號通達國體,其所上書至於痛哭流涕者,考其指歸,大抵以一身諭天下之勢。其言曰:『天下之勢方病大瘇。非徒瘇也,又苦𧾷灸盭。又類闢,且病痱。』臣每見士大夫好論時事,臣輒舉以
 問之:今日國體,於四百四病之中名為何病?能言其病者猶未必能處其方,不能言其病而輒處其方,其誤人之死,必矣。聞臣之言者不忿則默,間有反以詰臣,即對之曰:今日之病,名為風虛,其狀半身不隨是也。風者在外,虛者在內,真氣內耗,故風邪自外而乘之,忽中於人,應時殭僕,則靖康之變是也。幸而元氣猶存,故僕而復起,則建炎之興是也。然元氣雖存,邪氣尚盛,自淮以北皆吾故壤,而號令不能及,正朔不能加,有異於半身不
 隨者乎?非但半身不隨而已,半身存者,凜凜乎畏風邪之乘而不能以自安也。今日論者,譬如痿人之不忘起,奚必賢智之士,然後與國同其願哉?而市道庸流,口傳耳受,茍欲嘗試以售其方,則蕩熨針石,雜然並進,非體虛之人所宜輕受也。聞之醫曰:『中風偏廢,年五十以下而氣盛者易治。蓋真氣與邪氣相敵,真氣盛則邪氣衰,真氣行則邪氣去。然真氣不充滿於半存之身,則無以及偏廢之體。故欲起此疾者,必禁
 其嗜欲,節其思慮,愛其氣血,養其精神,使半存之身,日以充實,則陽氣周流,脈絡宜暢,將不覺舍杖而行。若急於愈疾而不顧其本,百毒入口,五臟受風,風邪之盛未可卒去,而真氣之存者日以耗亡,故中風再至者多不能救。』臣愚有感於斯言,竊謂賈誼復生,為陛下言,無以易此。」



 知興化軍,又移南劍,除夔路提點刑獄,改知夔州,加直敷文閣。夔屬郡曰施州,其羈縻郡曰思州。施民譚汝翼者,與知思州田汝弼交惡,會汝弼卒,汝翼帥兵二千人伐其喪。汝弼之
 子祖周深入報復,兵交於三州之境,施、黜大震。汝翼復繕甲兵,料丁壯,以重幣借兵諸洞,而乞師於帥府。慄曰:「汝翼實召亂者。」移檄罷兵,乃選屬吏往攝兵職,以漸收汝翼之權。命兵馬鈐轄按閱諸州,密檄至施,就攝州事。汝翼不之覺,已乃皇遽遁入成都。事聞,孝宗親札賜慄及成都制置使陳峴曰:「田氏猶是羈縻州郡,譚氏乃夔路豪族,又且首為釁端,帥閫不能彈壓,縱其至此。如尚不悛,未免加兵,除其元惡。」時汝翼在成都,聞之逃歸,調
 集家丁及役八砦義軍,列陳於沱河橋與官軍戰,潰,汝翼遁去,俘其徒四十有三人,獲甲鎧器仗三萬一千。慄取其巨惡者九人誅之。田祖周由是懼,與其母冉氏謀獻黔江田業,計錢九十萬緡以贖罪,蠻徼遂安。



 既而汝翼入都訴慄受田氏金,詔以汝翼屬吏,省札下夔州。慄親書奏狀繳還,並辨其事。上大怒。會近臣有救解者,尋坐慄身為帥臣,擅格上命,鐫職罷歸。既而理寺追究,事白,貸汝翼死,幽置紹興府。



 居頃之,詔慄累更事任,清
 介有聞,復直寶文閣、廣南西路轉運判官,就改提點刑獄,又改知潭州。除秘閣修撰,進集英殿修撰、知隆興府。召對便殿,奏乞仿唐制置補闕、拾遺左右各一員,不以糾彈為責。從之。除兵部侍郎。朱熹以江西提刑召為兵部郎官,熹既入國門,未就職。慄與熹相見,論《易》與《西銘》不合。至是,慄遣吏部趣之,熹以腳疾請告。慄遂論:「熹本無學術,徒竊張載、程頤之緒餘,為浮誕宗主,謂之道學,妄自推尊。所至輒攜門生十數人,習為春秋、戰國之態,妄
 希孔、孟歷聘之風,繩以治世之法,則亂人之首也。今採其虛名,俾之入奏,將置朝列,以次收用。而熹聞命之初,遷延道途,邀索高價,門生迭為游說,政府許以風聞,然後入門。既經陛對,得旨除郎,而輒懷不滿,傲睨累日,不肯供職,是豈張載、程頤之學教之然也?緣熹既除兵部郎官,在臣合有統攝,若不舉劾,厥罪惟均。望將熹停罷,姑令循省,以為事君無禮者之戒。」



 上謂其言過當,而大臣畏慄之強,莫敢深論。太常博士葉適獨上封事辯之
 曰:「考慄之辭,始末參驗,無一實者。其中『謂之道學』一語,無實最甚。蓋自昔小人殘害良善,率有指名,或以為好名,或以為立異,或以為植黨。近忽創為『道學』之目,鄭丙唱之,陳賈和之。居要路者密相付授,見士大夫有稍務潔修,粗能操守,輒以道學之名歸之,殆如吃菜事魔、影跡犯敗之類。往日王淮表裏臺諫,陰廢正人,蓋用此術。慄為侍從,無以達陛下之德意志慮,而更襲鄭丙、陳賈密相傳授之說,以道學為大罪。文致言語,逐去一熹,固
 未甚害,第恐自此游辭無實,讒言橫生,善良受害,無所不有!願陛下正紀綱之所在,絕欺罔於既形,摧抑暴橫以扶善類,奮發剛斷以慰公言。」於是侍御史胡晉臣劾慄,罷之,出知泉州,又改明州。奉祠以卒,謚簡肅。



 慄為人強介有才,而性狷急,欲快其私忿,遂至攻詆名儒,廢絕師教,殆與鄭丙、陳賈、何澹、劉德秀、劉三傑、胡紘輩黨邪害正者同科。雖疇昔論事,雄辯可觀,不足以蓋晚節之謬也。



 高文虎,字炳如,四明人,禮部侍郎閌之從子。登紹興庚辰進士第,調平江府吳興縣主簿。



 曾幾守官在吳,文虎從之游,故聞見博洽,多識典故。除國子正,遷太學博士。孝宗幸兩學,祭酒林光朝訪文虎具儀注,文虎輯國朝以來臨幸故事授之。兼國史院編修官,與修《四朝國史》。出知建昌軍,擢將作丞兼實錄院檢討官,修《高宗實錄》;又兼玉牒所檢討官,修《神宗玉牒》。自熙寧以來,史氏淆雜,人無所取信。文虎盡取朱墨本刊正繆妄,一一研核。
 既奏御,又修《徽宗玉牒》,考訂宣和、崇、觀以來尤為詳審。



 寧宗即位,遷軍器少監兼將作監,遷國子司業兼學士院權直,遷祭酒、中書舍人,兼直學士院兼祭酒,升實錄院同修撰、同修國史。



 韓侂胄用事,既逐趙汝愚、朱熹,以其門多知名士,設偽學之目以擯之,遂命文虎草詔曰:「向者權臣擅朝,偽邪朋附,協肆奸宄,包藏禍心。賴天之靈,宗廟之福,朕獲承慈訓,膺受內禪,陰謀壞散,國勢復安。嘉與士大夫厲精更始,凡曰淫朋比德,幾其自新,而
 歷載臻茲,弗迪厥化。締交合盟,窺伺間隙,毀譽舛迕,流言間發,將以傾國是而惑眾心。甚至竊附於元祐之眾賢,而不思實類乎紹聖之奸黨。國家秉德康寧,弗汝瑕殄,今惟自作弗靖,意者漸於流俗之失不可復反歟?將狃於國之寬恩而罰有弗及歟?何其未能洗濯以稱朕意也!朕既深詔二三大臣與夫侍從言議之官,益維持正論以明示天下矣,諭告所抵,宜各改視回聽,毋復借疑似之說以惑亂世俗。若其遂非不悔,怙終不悛,邦有
 常刑,必罰毋赦!」



 西掖詞命,舊率以數人共一詞,文虎以為非所以崇訓戒、贊人才也,乃人人各為之。遷兵部侍郎兼中書舍人,又兼祭酒,拜翰林學士兼侍讀、實錄院修撰,修國史。除華文閣學士、知建寧府,力丐祠,提舉太平興國宮。以臺臣言奪職,卒。



 文虎以博洽自負,與胡紘合黨,共攻道學,久司學校,專困遏天下士,凡言性命道德者皆絀焉。



 陳自強者,福州閩縣人,字勉之。登淳熙五年進士第。慶
 元二年,入都待銓。自以嘗為韓侂胄童子師,欲見之,無以自通,適僦居主人出入侂胄家,為言於侂胄。一日,鄉自強,比至,則從官畢集,侂胄設褥於堂,向自強再拜,次召從官同坐。侂胄徐曰:「陳先生老儒,汩沒可念。」明日,從官交薦其才。除太學錄,遷博士,數月轉國子博士,又遷秘書郎。入館半載,擢右正言、諫議大夫、御史中丞。入臺未逾月,遂登樞府,由選人至兩地財四年。嘉泰三年,拜右丞相,歷封祁、衛、秦國公。



 韓侂胄顓朝權,包苴盛
 行,自強尤貪鄙。四方致書饋,必題其緘云:「某物並獻」;凡書題無「並」字,則不開。縱子弟親戚關通貨賄,仕進乾請,必諧價而後予。日押空名刺札送侂胄家,須用乃填,三省不與也。都城火,自強所貯,一夕為煨燼。侂胄首遺之萬緡,執政及列郡聞之,莫不有助。不數月,得六十萬緡,遂倍所失之數。創國用司,自為國用使,以費士寅、張巖為同知國用事,掊克民財,州郡騷動。



 方侂胄欲為平章,猶畏眾議,自強首率同列援典故入奏。詔以侂胄為平章軍
 國事。常語人曰:「自強惟一死以報師王。」每稱侂胄為恩王、恩父,而呼堂吏史達祖為兄、蘇師旦為叔。



 侂胄將用兵,遣使北行審敵虛實,自強薦陳景俊以往。金人有「不宜敗好」之語,景俊歸,自強戒使勿言,侂胄乃決恢復之議。吳曦有逆謀,求歸蜀,厚賂自強。自強語侂胄:「非曦不足以鎮坤維。」乃縱之歸,曦卒受金人命為蜀王。侂胄奸兇,久盜國柄,自強實為之表裏。



 既開邊隙,朝野洶洶,三遣使請和。金人欲縛送首議用兵賊臣,侂胄恚憤,復欲
 用兵,中外大懼。史彌遠建議誅侂胄,詔以自強阿附充位,不恤國事,罷右丞相。未幾,詔追三官,永州居住,又責武泰軍節度副使、韶州安置。中書舍人倪思繳奏,乞遠竄,籍其家,詔從之。再責復州團練副使、雷州安置。後死於廣州。



 鄭丙,字少融,福州長樂人。紹興十五年進士。積官至吏部尚書、浙東提舉。朱熹行部至臺州,奏臺守唐仲友不法事,宰相王淮庇之。熹章十上。丙雅厚仲友,且迎合宰
 相意,奏:「近世士大夫有所謂『道學』者,欺世盜名,不宜信用。」蓋指熹也。於是監察御史陳賈奏:「道學之徒,假名以濟其偽,乞擯斥勿用。」道學之目,丙倡賈和,其後為慶元學禁,善類被厄,丙罪為多。



 嘗知泉州,為政暴急,或勸之尚寬,丙曰:「吾疾惡有素,豈以晚節易所守哉。」聞者哂之。丙官終端明殿學士,卒,謚簡肅。



 京鏜字仲遠,豫章人也。登紹興二十七年進士第。龔茂良帥江西,見之曰:「子廟廓器也。」及茂良參大政,遂薦鏜
 入朝。



 孝宗詔侍從舉良縣令為臺官,給事中王希呂曰:「京鏜蚤登儒級,兩試令,有聲。陛下求執法官,鏜其人也。」上引見鏜,問政事得失。時上初統萬機,銳志恢復,群臣進說,多迎合天子意,以為大功可旦暮致。鏜獨言「天下事未有驟如意者,宜舒徐以圖之。」上善其言。鏜於是極論今日民貧兵驕,士氣頹靡,言甚切至。上說,擢為監察御史,累遷右司郎官。



 金遣賀生辰使來,上居高宗喪,不欲引見,鏜為儐佐,以旨拒之。使者請少留闕下,鏜曰:「信
 使之來,以誕節也。誕節禮畢,欲留何名乎?」使行,上嘉其稱職。轉中書門下省檢正諸房公事。



 金人遣使來吊,鏜為報謝使。金人故事,南使至汴京則賜宴。鏜請免宴,郊勞使康元弼等不從,鏜謂必不免宴,則請徹樂,遺之書曰:「鏜聞鄰喪者舂不相,里殯者不巷歌。今鏜銜命而來,繄北朝之惠吊,是荷是謝。北朝勤其遠而憫其勞,遣郊勞之使,蕆式宴之儀,德莫厚焉,外臣受賜,敢不重拜。若曰而必聽樂,是於聖經為悖理,於臣節為悖義,豈惟貽
 本朝之羞,亦豈昭北朝之懿哉?」相持甚久。鏜即館,相禮者趣就席,鏜曰:「若不徹樂,不敢即席。」金人迫之,鏜弗為動,徐曰:「吾頭可取,樂不可聞也。」乃帥其屬出館門,甲士露刃向鏜,鏜叱退之。金人知鏜不可奪,馳白其主,主嘆曰:「南朝直臣也。」特命免樂。自是恆去樂而後宴鏜。孝宗聞之喜,謂輔臣曰:「士大夫平居孰不以節義自許,有能臨危不變如鏜者乎?」



 使還,入見,上勞之曰:「卿能執禮為國家增氣,朕將何以賞卿?」鏜頓首曰:「北人畏陛下威德,
 非畏臣也。正使臣死於北庭,亦臣子之常分耳,敢言賞乎!」故事,使還當增秩。右相周必大言於上曰:「增秩常典爾,京鏜奇節,今之毛遂也,惟陛下念之。」乃命鏜權工部侍郎。



 四川闕帥,以鏜為安撫制置使兼知成都府。鏜到官,首罷徵斂,弛利以予民。瀘州卒殺太守,鏜擒而斬之,蜀以大治。召為刑部尚書。



 寧宗即位,甚見尊禮,由政府累遷為左丞相。當是時,韓侂胄權勢震天下,其親幸者由禁從不一二歲至宰輔;而不附侂胄者,往往沉滯不
 偶。鏜既得位,一變其素守,於國事謾無所可否,但奉行侂胄風旨而已。又薦引劉德秀排擊善類,於是有偽學之禁。



 後宦者王德謙除節度使,鏜乃請裂其麻,上曰:「除德謙一人而止可乎?」鏜曰:「此門不可啟。節鉞不已,必及三孤;三孤不已,必及三公。願陛下以真宗不予劉承規為法,以大觀、宣、政間童貫等冒節鉞為戒。」上於是謫德謙而黜詞臣吳宗旦,或曰,亦侂胄意也。



 居無何,以年老請免相,薨,贈太保,謚文忠。後以監察御史倪千里言,改
 謚莊定。



 謝深甫,字子肅,臺州臨海人。少穎悟,刻志為學,積數年不寐,夕則置瓶水加足於上,以警困怠。父景之識為遠器,臨終語其妻曰:「是兒當大吾門,善訓迪之。」母攻苦守志,督深甫力學。



 中乾道二年進士第,調嵊縣尉。歲饑,有死道旁者,一嫗哭訴曰:「吾兒也。傭於某家,遭掠而斃。」深甫疑焉,徐廉得嫗子他所,召嫗出示之,嫗驚伏曰:「某與某有隙,賂我使誣告耳。」



 越帥方滋、錢端禮皆薦深甫有
 廊廟才,調昆山丞,為浙曹考官,一時士望皆在選中。司業鄭伯熊曰:「文士世不乏,求具眼如深甫者實鮮。」深甫曰:「文章有氣骨,如泰山喬嶽,可望而知,以是得之。」



 知處州青田縣。侍御史葛邲、監察御史顏師魯、禮部侍郎王藺交薦之。孝宗召見,深甫言:「今日人才,枵中侈外者多妄誕,矯訐沽激者多眩鬻。激昂者急於披露,然或鄰於好誇;剛介者果於植立,而或鄰於太銳;靜退簡默者寡有所合,或鄰於立異。故言未及酬而已齟齬,事未及成
 而已挫抑。於是趣時徇利之人,專務身謀,習為軟熟,畏避束手,因循茍且,年除歲遷,亦至通顯,一有緩急,莫堪倚仗。臣願任使之際,必察其實,既悉其實,則涵養之以蓄其才,振作之以厲其氣,栽培封殖,勿使沮傷。」上嘉納。問當世人才,對曰:「薦士,大臣職也。小臣來自遠方,不足以奉明詔。」上頷之,諭宰臣曰:「謝深甫奏對雍容,有古人風。」除籍田令,遷大理丞。



 江東大旱,擢為提舉常平,講行救荒條目,所全活一百六十餘萬人。光宗即位,以左曹
 郎官借禮部尚書為賀金國生辰使。紹熙改元,除右正言,遷起居郎兼權給事中。知閣門事韓侂胄破格轉遙郡刺史,深甫封還內降云:「人主以爵祿磨厲天下之人才,固可重而不可輕;以法令堤防天下之僥幸,尤可守而不可易。今侂胄驀越五官而轉遙郡,僥幸一啟,攀援踵至,將何以拒之?請罷其命。」



 進士俞古應詔言事,語涉詆訐,送瑞州聽讀。深甫謂:「以天變求言,未聞旌賞而反罪之,則是名求而實拒也。俞古不足以道,所惜者朝廷
 事體耳」右司諫鄧馹論近習,左遷,深甫請還馹,謂:「不可以近習故變易諫官,為清朝累。」



 二年,知臨安府。三年,除工部侍郎。入謝,光宗面諭曰:「京尹寬則廢法,猛則厲民,獨卿為政得寬猛之中。」進兼吏部侍郎,兼詳定敕令官。四年,兼給事中。陳源久以罪斥,忽予內祠,深甫固執不可。姜特立復詔用,深甫力爭,特立竟不得入。張子仁除節度使,深甫疏十一上,命遂寢。每禁庭燕私,左右有希恩澤者,上必曰:「恐謝給事有不可耳。」



 寧宗即位,除煥章
 閣待制、知建康府,改御史中丞兼侍讀。上言:「比年以來,紀綱不立。臺諫有所論擊,不與被論同罷,則反除以外任;給、舍有所繳駁,不命次官書行,則反遷以他官;監司有所按察,不兩置之勿問,則被按者反得美除。以奔競得志者,不復知有廉恥;以請屬獲利者,不復知有彞憲。貪墨縱橫,莫敢誰何;罪惡暴露,無所忌憚。隳壞紀綱,莫此為甚。請風厲在位,革心易慮,以肅朝著。」禮官議祧僖祖,侍講朱熹以為不可。深甫言:「宗廟重事,未宜遽革。朱
 熹考訂有據,宜從熹議。」



 慶元元年,除端明殿學士、簽書樞密院事,遷參知政事,再遷知樞密院事兼參知政事。內侍王德謙建節,深甫三疏力陳不可蹈大觀覆轍,德謙竟斥。進金紫光祿大夫,拜右丞相,封申國公,進岐國公。光宗山陵,為總護使。還,拜少保,力辭,改封魯國公。



 嘉泰元年,累疏乞避位,寧宗曰:「卿能為朕守法度,惜名器,不可以言去。」召坐賜茶,御筆書《說命》中篇及金幣以賜之。



 有餘嘉者,上書乞斬朱熹,絕偽學,且指蔡元定為偽
 黨。深甫擲其書,語同列曰:「朱元晦、蔡季通不過自相與講明其學耳,果有何罪乎?余哲蟣虱臣,乃敢狂妄如此,當相與奏知行遣,以厲其餘。」



 金使入見不如式,寧宗起入禁中,深甫端立不動,命金使俟於殿隅,帝再御殿,乃引使者進書,迄如舊儀。



 拜少保。乞骸骨,授醴泉觀使。明年,拜少傅,致仕。有星隕於居第,遂薨。後孫女為理宗後,追封信王,易封衛、魯,謚惠正。



 許及之,字深甫,溫州永嘉人。隆興元年第進士,知袁州
 分宜縣。以部使者薦,除諸軍審計,遷宗正簿。乾道元年,林慄請增置諫員,乃效唐制置拾遺、補闕,以及之為拾遺,班序在監察御史之上。



 高宗崩,及之言:「皇帝既躬三年之喪,群臣難從純吉,當常服黑帶。」王淮當國久,及之奏:「陛下即位二十七年,而群臣未能如聖意者,以茍且為安榮,以姑息為仁恕,以不肯任事為簡重,以不敢任怨為老成。敢言者指為輕儇,鮮恥者謂之樸實。陛下得若人而相之,何補於治哉!」淮竟罷職予祠。



 光宗受禪,除
 軍器監,遷太常少卿,以言者罷。紹熙元年,除淮南運判兼淮東提刑,以鐵錢濫惡不職,貶秩,知廬州。召除大理少卿。寧宗即位,除吏部尚書兼給事中。及之早與薛叔似同擢遺、補,皆為當時所予。黨事既起,善類一空,叔似累斥逐,而及之謅事侂胄,無所不至。嘗值侂胄生日,朝行上壽畢集,及之後至,閹人掩關拒之,及之俯僂以入。為尚書,二年不遷,見侂胄流涕,序其知遇之意及衰遲之狀,不覺膝屈。侂胄惻然憐之曰:「尚書才望,簡在上心,
 行且進拜矣。」居亡何,同知樞密院事。當時有「由竇尚書、屈膝執政」之語,傳以為笑。



 嘉泰二年,拜參知政事,進知樞密院事兼參政。兵端開,侂胄欲令及之守金陵,及之辭。侂胄誅,中丞雷孝友奏及之實贊侂胄開邊,及守金陵,始詭計免行。降兩官,泉州居住。嘉定二年,卒。



 梁汝嘉,字仲謨,處州麗水人。以外祖太宰何執中任入官,調中山府司議曹事。建炎初,知常州武進縣。守薦其治狀,擢通判州事,加直秘閣,歷官至轉運副使。



 臨安闕
 守,火盜屢作,命汝嘉攝事。汝嘉修火政,嚴巡徼,盜發輒得,火災亦息。遂命為真,加直龍圖閣。以稱職,擢徽猷閣待制,試戶部侍郎兼知臨安府。累遷戶部侍郎,進權尚書兼江、淮、荊、廣經制使。



 汝嘉素善秦檜,殿中侍御史周葵將按之。汝嘉聞,紿中書舍人林待聘曰:「副端將論君。」待聘亟告檜,徙葵起居郎。葵入後省,出疏示待聘曰:「梁仲謨何其幸也。」待聘始知為汝嘉所賣,士大夫以是薄汝嘉。汝嘉求去,以寶文閣直學士提舉太平觀。未幾,升
 學士、知明州,兼浙西沿海制置使,更溫、宣、鼎三郡,復奉祠以歸。紹興二十三年,卒。汝嘉長於吏治,在臨安風績尤著。



 論曰:君子之論人,亦先觀其大者而已矣。忠孝,人之大節也,胡紘導其君以短喪,不得謂之忠;何澹疑所生繼母之服,士論紛紜而後去,不可以為孝。彼於其大者且忍為之,則其協比權奸,誣構善類,亦何憚而不為乎?謝深甫出處,舊史泯其跡,若無可議為者。然慶元之初,韓
 侂胄設偽學之禁,網羅善類而一空之,深甫秉政,適與之同時,諉曰不知,不可也。況於一劾陳傅良,再劾趙汝愚,形於深甫之章,有不可揜者乎?陳自強、鄭丙、許及之輩,狐媚茍合,以竊貴寵,斯亦不足論已。若林慄之有治才,善論事,高文虎之自負該洽,京鏜之仗義秉禮,志信於敵國,抑豈無足稱者。然慄以私忿詆名儒,不為清議所與,而文虎草偽學之詔,以是為非,以正為邪,變亂白黑,以欺當世,其人可知也。鏜暮年得政,朋奸取容,既愧
 其初服矣,況偽學之目,識者以為鏜實發之乎?士君子立身行事,一失其正,流而不知返,遂為千古之罪人,可不懼哉!可不懼哉!



\end{pinyinscope}