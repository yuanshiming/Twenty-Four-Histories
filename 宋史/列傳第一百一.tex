\article{列傳第一百一}

\begin{pinyinscope}

 梁燾王巖叟鄭雍孫永



 梁燾,字況之,鄆州須城人。父蒨,兵部員外郎、直史館。燾以蒨任,為太廟齋郎。舉進士中第,編校秘閣書籍,遷集賢校理、通判明州,檢詳樞密五房文字。



 元豐時久旱,上
 書論時政曰:



 陛下日者閔雨,靖惟政事之闕,惕然自責。丁卯發詔,癸酉而雨,是上天顧聽陛下之德言,而喜其有及民之意也。當四方仰雨十月之久,民刻於新法,嗷嗷如焦,而京師尤甚,闤闠細民,罔不失職,智愚相視,日有大變之憂。陛下既惠以詔旨,又施之行事,講除刻文,蠲損緡錢等,一日之間,歡聲四起。距誕節三日而膏澤降,是天以雨壽陛下之萬年,感聖心於大寤,有以還其仁政也。



 然法令乖戾,為毒於民者,所變才能萬一。人心
 之不解,故天意亦未釋,而雨不再施。陛下亦以此為戒,而夙夜慮之乎?今陛下之所知者,市易事耳。法之為害,豈特此耶?曰青苗錢也,助役錢也,方田也,保甲也,淤田也。兼是數者,而天下之民被其害。青苗之錢未一及償,而責以免役;免役之錢未暇入,而重以淤田;淤田方下,而復有方田;方田未息,而迫以保甲。是徒擾百姓,使不得少休於聖澤。其為害之實,雖一有言之者,必以下主吏,主吏妄報以無是,則從而信之,恬不復問,而反坐言
 者。雖間遣使循行,而茍且寵祿,巧為妄誕,成就其事,至請遍行其法,上下相隱,習以成風。



 臣謂天下之患,不患禍亂之不可去,患朋黨蔽蒙之俗成,使上不得聞所當聞,故政日以敝,而禍亂卒至也。陛下可不深思其故乎?



 疏入,不報。



 內侍王中正將兵出疆,干賞不以法。燾爭之不得,請外,出知宣州。入辭,神宗曰:「樞臣雲卿不肯安職,何也?」對曰:「臣居官五年,非敢不安職,恐不勝任使,故去耳。」神宗曰:「王中正功賞文書,何為獨不可?」曰:「中正罔冒
 僥覬,臣不敢屈法以負陛下。」未幾,提點京西刑獄,哲宗立,召為工部郎中,遷太常少卿、右諫議大夫。有請宣仁後御文德殿服袞冕受冊者,燾率同列諫,引薛奎諫章獻明肅皇后不當以王服見太廟事,宣仁后欣納。又論市易已廢,乞蠲中下戶逋負;又乞欠青苗下戶,不得令保人備償。



 文彥博議遣劉奉世使夏國,御史張舜民論其不當遣,降通判虢州。燾言:「御史持紀綱之官,得以犯顏正論,況臣下過失,安得畏忌不言哉?今御史敢言大
 臣者,天下之公議;大臣不快御史者,一夫之私心。罪天下敢言之公議,便一夫不快之私心,非公朝盛事也。」時同論者傅堯俞、王巖叟、朱光庭、王覿、孫升、韓川,凡七人,悉召至都堂,敕諭以「事當權其輕重,故不惜一新進御史,以慰老臣。」燾又言:「若論年齡爵祿,則老臣為重;若論法度綱紀,則老臣為輕。御史者,天子之法官也,不可以大臣鞅鞅而斥去。願還舜民,以正國體。」章十上,不聽。



 燾又面責給事中張問不能駁還舜民制命,以為失職。坐
 詬同列,出為集賢殿修撰、知潞州,辭不拜,曰:「臣本論張舜民不當罷,如以為非,即應用此受斥。今乃得以微罪冒美職,守劇郡,如此則朝廷命令,不能明辨曲直,以好惡示天下矣。」不報。至潞,值歲饑,不待命發常平粟振民。流人聞之,來者不絕,燾處之有條,人不告病。



 明年,以左諫議大夫召。甫就道,民攀轅不得行,逾太行,抵河內乃已。既對,上書言:「帝富於春秋,未專宸斷;太皇保祐聖主,制政簾帷,奸人易為欺蔽。願正綱紀,明法度,採用忠言,
 講求仁術。」兩宮嘉納焉。



 前宰相蔡確作詩怨謗,燾與劉安世交攻之。燾又言:「方今忠於確者,多於忠朝廷之士;敢為奸言者,多於敢正論之人。以此見確之氣焰兇赫,根株牽連,賊化害政,為患滋大。」確卒竄新州。燾進御史中丞。鄧潤甫除吏部尚書,燾論潤甫柔佞不立,巧為進取。不聽。改權戶部尚書,不拜,以龍圖閣直學士知鄭州。旬日,入權禮部尚書,為翰林學士。



 元祐七年,拜尚書右丞,轉左丞。蔡京帥蜀,燾曰:「元豐侍從,可用者多;惟京輕
 險貪愎,不可用。」又與同列議夏國地界,不能合,遂丐去。哲宗遣近臣問所以去意,且令密訪人才。燾曰:「信任不篤,言不見聽,而詢問人才,非臣所敢當也。」使者再至,乃言:「人才可大任者,陛下自知之。但須識別邪正,公天下之善惡,圖任舊人中堅正純厚有人望者,不牽左右好惡之言以移聖意,天下幸甚。」



 以疾,罷為資政殿學士、同醴泉觀使。故事,非宰相不除使,遂置同使以寵之。力辭,改知穎昌府。既出京師,哲宗遣中貴諭以復用之旨。紹
 聖元年,知鄆州。朋黨論起,哲宗曰:「梁燾每起中正之論,其開陳排擊,盡出公議,朕皆記之。」以故最後責,竟以司馬光黨黜知鄂州。三年,再貶少府監。分司南京。明年,三貶雷州別駕,化州安置。三年卒,年六十四。徙其子於昭州。徽宗立,始得歸。



 燾自立朝,一以引援人物為意。在鄂作《薦士錄》,具載姓名。客或見其書,曰:「公所植桃李,乘時而發,但不向人開耳。」燾笑曰:「燾出入侍從,至位執政,八年之間所薦,用之不盡,負愧多矣。」其好賢樂善如此。



 王巖叟,字彥霖,大名清平人。幼時,語未正已知文字。仁宗患詞賦致經術不明,初置明經科,巖叟年十八,鄉舉、省試、廷對皆第一。調欒城簿、涇州推官,甫兩月,聞弟喪,棄官歸養。



 熙寧中,韓琦留守北京,以為賢,闢管勾國子監,又闢管勾安撫司機宜文字,監晉州折博、煉鹽務。韓絳代琦,復欲留用。巖叟謝曰:「巖叟,魏公之客,不願出他門也。」士君子稱之。後知定州安喜縣,有法吏罷居鄉里,導人為訟,巖叟捕撻於市,眾皆竦然。定守呂公著嘆曰:「此
 古良吏也。」有詔近臣舉御史,舉者意屬巖叟而未及識,或謂可一往見。巖叟笑曰:「是所謂呈身御史也。」卒不見。



 哲宗即位,用劉摯薦,為監察御史。時六察尚未言事,巖叟入臺之明日,即上書論社稷安危之計,在從諫用賢,不可以小利失民心。遂言役錢斂法太重,民力不勝,願復差法如嘉祐時。又言河北榷鹽法尚行,民受其弊,貧者不復食。錄大名刻石《仁宗詔書》以進,又以河北天下根本,自祖宗以來,推此為惠。願復其舊。



 江西鹽害民,詔
 遣使者往視。巖叟曰:「一方病矣,必待使還而後改為,恐有不及被德澤而死者。願亟罷之。」又極陳時事,以為「不絕害本,百姓無由樂生;不屏群邪,太平終是難致。」時下詔求民疾苦,四方爭以其情赴訴,所司憚於省錄,頗成壅滯。巖叟言:「不問則已,言則必行之。不然,天下之人必謂陛下以空言說之,後有詔令,孰肯取信?」李定不持所生母仇氏服,巖叟論其不孝,定遂分司。



 宰相蔡確為裕陵復土使,還朝,以定策自居。巖叟曰:「陛下之立,以子繼
 父,百王不易之道。且太皇太后先定於中,而確敢貪天自伐。章惇讒賊狼戾,罔上蔽明,不忠之罪,蓋與確等。近簾前爭役法,詞氣不遜,無事上之禮。今聖政不出房闥,豈宜容此大奸猶在廊廟!」於是二人相繼退斥。



 遷左司諫兼權給事中。時並命執政,其間有不協時望者,巖叟即繳錄黃,上疏諫。既而命不由門下省以出,巖叟請對,言之益切。退就閣上疏曰:「臣為諫官既當言,承乏給事又當駁,非臣好為高論,喜忤大臣,恐命令斜出,尤損紀
 綱。」疏凡八上,命竟寢。又言:「三省胥吏,月饗厚奉,歲累優秩。而朝廷每舉一事,輒計功論賞,不知平日祿賜,將焉用之?姑息相承,流弊已極。望飭勵大臣,事為之制。」即詔裁抑僥幸,定為十七條。



 遷侍御史。兩省正言久闕,巖叟上疏曰:「國朝仿近古之制,諫臣才至六員,方之先王,已為至少。今復虛而不除,臣所未諭。豈以為治道已清,而無事於言邪?人材難稱,不若虛其位邪?二者皆非臣所望於今日也。願趣補其闕,多進正人以壯本朝;正人進,
 則小人自消矣。」



 諸路水災,朝廷行振貸,戶部限以災傷過七分、民戶降四等始許之。巖叟言:「中戶以上,蓋亦艱食。乞毋問分數、等級,皆得貸,庶幾王澤無間,以召至和矣。」坐張舜民事,改起居舍人,不拜,以直集賢院知齊州。請河北所言鹽法,行之京東。明年,復以起居舍人召。嘗侍邇英講,進讀《寶訓》,至節費,巖叟曰:「凡言節用,非偶節一事便能有濟。當每事以節儉為意,則積久累日,國用自饒。」讀仁宗知人事,巖叟曰:「人主常欲虛心平意,無所偏
 系,觀事以理,則事之是非,人之邪正,自然可見。」



 司馬康講《洪範》,至「乂用三德」,哲宗曰:「止此三德,為更有德。」蓋哲宗自臨御,淵默不言,巖叟喜聞之,因欲風諫,退而上疏曰:「三德者,人君之大本,得之則治,失之則亂,不可須臾去者也。臣請別而言之。夫明是非於朝廷之上,判忠邪於多士之間,不以順己而忘其惡,不以逆己而遺其善,私求不徇於所愛,公議不遷於所憎。竭誠盡節者,任之當勿二;罔上盜寵者,棄之當勿疑。惜紀綱,謹法度,重典
 刑,戒姑息,此人主之正直也。遠聲色之好,絕盤游之樂,勇於救天下之弊,果於斷天下之疑,邪說不能移,非道不能說,此人主之剛德也。居萬乘之尊而不驕,享四海之富而不溢,聰明有餘而處之若不足,俊傑並用而求之如不及,虛心以訪道,屈己以從諫,懼若臨淵,怯若履薄,此人主之柔德也。三者足以盡天下之要,在陛下力行何如耳。」巖叟因侍講,奏曰:「陛下退朝無事,不知何以消日?」哲宗曰:「看文字。對曰:「陛下以讀書為樂,天下幸甚。聖
 賢之學,非造次可成,須在積累。積累之要,在專與勤。屏絕它好,始可謂之專;久而不倦,始可謂之勤。願陛下特留聖意。」哲宗然之。



 巖叟館伴遼賀正旦使耶律寬,寬求觀《元會儀》,巖叟曰:「此非外國所宜知。」止錄《笏記》與之,寬不敢求。進權吏部侍郎、天章閣待制、樞密都承旨。湖北諸蠻互出擾邊,無有寧歲,巖叟請專以疆事委荊南唐義問。遂自草檄文,喻義問以朝廷方敦尚恩信,勿為僥幸功賞之意,後遂安輯。



 初,夏人遣使入貢,及為境上之
 議,故為此去彼來,牽致勞苦,每違期日。巖叟請預戒邊臣,夏違期,一不至則勿復應,自後不復敢違。質孤、勝如二堡,漢趙充國留屯之所,自元祐講和,在蘭州界內,夏以為形勝膏腴之地,力爭之。二堡若失,則蘭州、熙河遂危。延帥欲以二堡與夏,蘇轍主其議。及熙河、延安二捷同報,轍奏曰:「近邊奏稍頻,西人意在得二堡。今盛夏猶如此,入秋可虞,不若早定議。」意在與之也。巖叟曰:「形勢之地,豈可輕棄,不知既與,還不更求否?」太皇太后曰:「然。」
 議遂止。



 夏人數萬侵定西之東、通遠之北,壞七崖匙堡,掠居人,轉侵涇原及河外鄜、府州,眾遂至十萬。熙帥範育偵伺夏右廂種落大抵趣河外,三疏請乘此進堡砦,築龕谷、勝如、相照、定西而東徑隴諾城。朝議未一,或欲以七巉經毀之地,皆以與夏。巖叟力言不可與,彼計得行,後患未已。因請遣官諭熙帥,即以戶部員外郎穆衍行視,築定遠以據要害。其調兵貲費,一從便宜,不必中覆。定遠遂城,皆巖叟之力。



 拜中書舍人。滕甫帥太原,為
 走馬承受所撼,徙穎昌。巖叟封還詞頭,言:「進退帥臣,理宜重慎。今以小臣一言易之,使後人畏憚不自保,此風浸長,非委任安邊之福。」乃止。



 復為樞密都承旨、權知開封府。舊以推、判官二人分左右廳,共治一事,多為異同,或累日不竟,吏疲於咨稟。巖叟創立逐官分治之法,自是署為令。都城群偷所聚,謂之「大房」,每區容數十百人,淵藪詭僻,不可勝究。巖叟令掩捕撤毀,隨輕重決之,根株一空。供備庫使曹續以產貿萬緡,市儈逾年負其半,
 續盡力不可取。一日啟戶,則所負皆在焉。驚扣其故,儈曰:「王公今日知府矣。」初,曹氏之隸韓絢與同隸訟,事連其主,就逮之。曹氏者,慈聖後之族也。巖叟言:「部曲相訟,不當論其主。今不惟長告訐之風,且傷孝治。慈聖仙游未遠,一旦因廝役之過,使其子孫對吏,殆聖情有所不忍。」詔竄絢而絕其獄。巖叟常謂:「天下積欠多名,催免不一,公私費擾,乞隨等第多寡為催法。」朝廷乃定五年十科之令。



 元祐六年,拜樞密直學士、簽書院事。入謝,太
 皇太后曰:「知卿才望,不次超用。」巖叟又再拜謝,進曰:「太后聽政以來,納諫從善,務合人心,所以朝廷清明,天下安靜。願信之勿疑,守之勿失。」復少進而西,奏哲宗曰:「陛下今日聖學,當深辨邪正。正人在朝,則朝廷安,邪人一進,便有不安之象。非謂一夫能然,蓋其類應之者眾,上下蔽蒙,不覺養成禍胎爾。」又進曰:「或聞有以君子小人參用之說告陛下者,不知果有之否?此乃深誤陛下也。自古君子小人,無參用之理。聖人但云:『君子在內,小人
 在外則泰,小人在內、君子在外則否。」小人既進,君子必引類而去。若君子與小人競進,則危亡之基也。此際不可不察。」兩宮深然之。



 上清儲詳宮成,太皇太后謂輔臣曰:「此與皇帝皆出閣中物營之,以成先帝之志。」巖叟曰:「陛下不煩公,不勞民,真盛德事。然願自今以土木為戒。」又以宮成將戒肆赦,巖叟曰:「昔天禧中,祥源成,治平中,醴泉成,皆未嘗赦。古人有垂死諫君無赦者,此可見赦無益於聖治也。」



 哲宗方選後,太皇太后曰:「今得狄諮女,年
 命以便,然為是庶出過房,事須評議。」巖叟進曰:「按《禮經·問名篇》,女家答曰:『臣女,夫婦所生。』及外氏官諱,不識今者狄氏將何辭以進?」議遂寢。哲宗選後既定,太皇太后曰:「帝得賢後,有內助功,不是小事。」巖叟對曰:「內助雖後事,其正家須在皇帝。聖人言:『正家而得天下』。當慎之於始。」太皇太后以是語哲宗者再。巖叟退取歷代後事可為法者,類為《中宮懿範》上之。



 宰相劉摯、右丞蘇轍以人言求避位,巖叟曰:「元祐之初,排斥奸邪,緝熙聖治,摯與
 轍之功居多。原深察讒毀之意,重惜腹心之人,無輕其去就。」兩宮然之。後摯竟為御史鄭雍所擊,巖叟連上疏論救。摯去位,御史遂指為黨,罷為端明殿學士、知鄭州。言者猶未厭,太皇太后曰:「巖叟有大功,今日之命,出不獲已耳。」



 明年,徙河陽,數月卒,年五十一。贈左正議大夫。紹聖初,追貶雷州別駕。司馬光以其進諫無隱,稱之曰:「吾寒心慄齒,憂在不測,公處之自如,至於再三,或累十數章,必行其言而後已。」為文語省理該,深得制誥體。有《
 易》、《詩》、《春秋傳》行於世。



 鄭雍,字公肅,襄邑人。進士甲科,調兗州推官。韓琦上其文,召試秘閣校理、知太常禮院。英宗之喪,論宗室不當嫁娶,與時相忤,通判峽州,知池州,復還太常禮院,歷開封府判官。



 熙寧、元豐間,更制變令,士大夫多違己以求合,雍獨靜默自守。改嘉王、岐王府記室參軍。神宗末年,二王既長,猶居禁中,雍獻四箴規戒,且諷使求出外邸。凡在邸七年,用久次,以轉運使秩留。宣仁後知其賢,及
 臨政,擢為起居郎,進中書舍人。



 鄧潤甫除翰林承旨,雍當制。制未出,言事者五人交章攻之,換為侍讀學士。雍言:二職皆天下精選,以潤甫之過薄,不當革前命;以為奸邪,不當在經幄。今中外咸謂朝廷姑以是塞言者,如此則邪正何由可辨,善惡何由可明?若每事必待言,是賞罰之柄,不得已而行,非所以示信天下之道。」潤甫仍為承旨。周□童乞以王安石配享神宗廟,雍言:「安石持國政,不能上副屬任,非先帝神明,遠而弗用,則其所敗
 壞,可勝言哉!今穜以小臣輒肆橫議,願正其罪。」從之。



 使契丹還,徙右諫議大夫,言:「朝廷重內輕外,選用牧伯,罕輟從班,以閥閱輕淺者充員,不復為來日慮。願自今稍積資望,以慚試之。」吳中大饑,方議振恤,以民習欺誕,敕本部料檢,家至戶到。雍言:「此令一布,吏專料民而不救災,民皆死於饑。今富有四海,奈何謹圭撮之濫,而輕比屋之死乎?」哲宗悟,追止之。



 侍御史賈易沽激自喜,中丞趙彥若懦不自立,雍並論之,遂罷易,左轉彥若,以雍為
 中丞。雍辭曰:「中丞以臣言去而身承其乏,非所以厚風俗也。」不許。時二府禁謁加嚴,雍嘆曰:「旁招俊乂,列於庶位,宅百揆職也。彼有足不及公卿之門者,猶當物色致之,奈何設禁若是!且二府皆天子所改容而體貌之者,乃復防閑其私如此乎?」於是援賈誼廉恥節行之說以諫,詔弛其禁。



 刑部讞囚,宰執論殺之,有司以為可生,不奉詔,得罪。雍言:「是固可罪,然究其用心,在於廣好生之德耳,若遽以為罪,臣恐鄰於嗜殺。今使有司欲殺而朝廷
 生之,猶恐仁恩德意不白於天下,而況反是者哉!」哲宗嘉納,囚遂得生。



 初,邢恕以書抵宰相劉摯,摯答之,有『自愛以俟休復』之語,排岸司茹東濟錄書示雍與殿中侍御史楊畏,雍、畏釋其語曰:「『俟休復』者,俟他日太后復闢也。」遂並以此事論摯威福自恣,乞罷之以收主柄。又論王嚴叟、朱光庭、梁燾等三十人皆為摯黨,以閉其援。及摯出知鄆州,光庭方為給事中,繳還摯麻詞,嚴叟、燾力救之,哲宗以先入之言,不納。雍之攻摯,人以為附左相
 呂大防也。又有請暴摯陰事者,雍曰:「吾為國擊宰相,非仇摯也。彼之陰事,何有於國哉?」置不以聞。



 拜尚書右丞,改左丞。雍在政地,哲宗稱其事上有禮。紹聖初,治元祐眾臣,雍頓首自列,哲宗明其亡他心,諭使勿去。周秩乘隙抵之,謂雍初為侍從時,因徐王私於權臣以進。哲宗怒曰:「此是何言也!使徐王聞之,豈能自安?」黜秩知廣德軍,敕銀臺毋受雍辭去奏章,東府吏毋聽雍妻子輒出,且令學士錢勰善為留詔。二年,始以資政殿學士知陳
 州,徙北京留守。



 初,章惇以白帖貶謫元祐臣僚,安燾爭論不已,哲宗疑之。雍欲為自安計,謂惇曰:「熙寧初,王安石作相,常用白帖行事。」惇大喜,取其案牘懷之,以白哲宗,遂其奸。雍雖以此結惇,然卒罷政,坐元祐黨,奪職知鄭州。數日,改成都府。元符元年,提舉崇福宮,歸,未至而卒,年六十八。政和中,復資政殿學士。



 孫永,字曼叔,世為趙人,徙長社。年十歲孤,祖給事中沖,列為子行,蔭將作監主簿,肄業西學,群試常第一。
 沖戒之曰:「洛陽英雋所萃,汝年少,不宜多上人。」自是不復試。沖卒,喪除,復列為孫,換試銜,擢進士第,調襄城尉、宜城令,至太常博士。御史中丞賈黯薦為御史,以母老不就。韓琦讀其詩,嘆譽之,引為諸王府侍讀。神宗為穎王,出新錄《韓非子》畀宮僚讎定,永曰:「非險薄刻核,其書背《六經》之旨,願毋留意。」王曰:「廣藏書之數耳,非所好也。」及為皇太子,進舍人;即位,擢天章閣待制,安撫陜西。民景詢外叛,詔捕送其孥,勿以赦原。永言:「陛下新御極,曠澤
 流行,惡逆者猶得虧除。今緣坐者弗宥,非所以示信也。」



 歷河北、陜西都轉運使。時邊用不足,以解鹽、市馬別為一司,外臺不得與。永奏曰:「鹽、馬,國之大計,使主者專其柄,既無以統隸,茍為非法,孰從而制之?」



 加龍圖閣直學士、知秦州。王韶以布衣入幕府,建取熙河策,永折之曰:「邊陲方安靜,無故騷動,恐變生不測。」會新築劉家堡失利,眾請戮偏裨以塞責。永曰:「居敵必爭之地,軍孤援絕,兵法所謂不得而守者也。尤人以自免,於我安乎?」竟用
 是降天章閣待制、知和州,以詳定編敕知審官東院召還,神守問:「青苗、助役之法,於民便否?」對曰:「法誠善,然強民出息輸錢代徭,不能無重斂之患。若用以資經費,非臣所知也。」時倉法峻密,庾吏受百錢,則黥為卒,府史亦如之。神宗又問:「此法既下,吏尚為奸乎?」對曰:「強盜罪死,犯者猶眾,況配隸邪?使人畏法而不革心,雖在府史,臣亦不敢必其無犯也。」議復肉刑,事下永。永奏曰:「刻人肌膚,深害仁政,漢文帝所不忍,陛下忍之乎?」神宗曰:「事固
 未決,待卿始定耳。」不果行。



 復學士,知瀛州。河決,於貝、瀛、冀尤甚,民租以災免者,州縣懼常平法,徵催如故。永連章論止,神宗從之,仍命發廩粟以振。白溝巡檢趙用以遼人漁界河,擅引兵北度,蕩其族帳,遼持此兆釁,數暴邊上,神宗遣使問故,永請正用罪以謝,未報,遼屯兵連營互四十里,永好諭之曰:「疆吏冒禁,已置之獄矣,今何為者?」敵意解,但求醪□犒師而旋。



 進樞密直學士、知開封府。呂嘉問言,吏欲使都人列肆輸錢以免直。下府詢
 究,曹椽以為便。永占書紙尾,不暇省。既乃行市易抵當法,貸民錢而為之期,有不能償而死者。神宗頗知之,嘉問妄變其名以罔聽。神宗慮立法未盡,詔永及韓維究實。永奏言:「市算下逮錐刀,為人患苦。」御史張琥劾永棄同即異,罷為提舉中太一宮。



 元豐中,判軍器監。有司病皮革不給,嚴隱匿之科,亡賴輩肆情為訐,至婦人冠飾亦不免。永請聽人以所藏之善者售於官,得貸其餘,訐訟既息,國用亦濟。出知太原,且行,神宗訪以時務,永言:「近
 者造戎器倍常,外間謂將有事於征討。兵非輕用之物,原軫不戢自焚之戒。」神宗曰:「此備豫不虞,若四方安平,豈有輕動之理?卿言是也。」忻、代產鹽,苦惡不堪食,轉運使必欲理之,以盜販闌越之罪罪兵吏。永言:「鹽,民食也,不可禁;兵,武備也,不可闕。顧以惡鹽累防兵,非計也。」詔弛其禁。



 入判將作,進端明殿學士。病不能朝,神宗遣上醫調視,六命近侍問安否,至虛樞密位以待。辭去益力,提舉崇福宮。逾年,起知陳州,徙穎昌。永裕起陵,許、汝當
 運粟數十萬斛於陵下,調民牛數萬,永請而免。哲宗召拜工部尚書。太皇太后下詔求言,永陳保馬、保甲、免役三事最敝,願一切罷去,復修監牧、保伍、差徭之法。太皇太后皆納之。元祐元年,遷吏部,又屬疾,改資政殿學士兼侍讀,提舉中太一宮,未拜而卒,年六十八。贈銀青光祿大夫,賻金帛二千,謚曰康簡。



 永外和內勁,論議常持平,不求詭異。事或悖於理,雖逼以勢,亦不為屈。未嘗以矯亢形於色辭,與人交,終身無怨仇。范純仁、蘇頌皆稱
 之為國器。



 論曰:「宋之衰也,人才尚多。梁燾、王巖叟盡忠事上,凡有過舉,知無不言,雖或從或違,而隱然有虎豹在山之勢矣。第以新州之舉,於是為過。故他日紹聖復以借口,使元祐眾賢皆罹其禍,由是再變而為宣、政之奸臣,國日危矣。鄭雍易其所守,肆擊劉摯,波及者三十人,欲結章惇以取容,然而終亦不免。小人反復,專務自全,竟何益哉?孫永之為人,庶得其中焉。



\end{pinyinscope}