\article{列傳第四十六}

\begin{pinyinscope}

 楊礪宋湜王嗣宗李昌齡從子紘趙安仁父孚子良規孫君錫陳彭年



 楊礪,字汝礪,京兆鄮人。曾祖守信,唐山南西道節度、同平章事,本宦官復恭假子也。祖知禮,後唐均州刺史。父
 仁儼,入蜀仕王氏,為丹棱令。蜀平,補渭南主簿,累遷永和令。礪,建隆中舉進士甲科。父喪,絕水漿數日。服除,以祿不足養母,閑居無仕進意,鄉舊移書敦諭,礪乃赴官。解褐鳳州團練推官,歲餘,又以母疾棄官。開寶九年,詣闕獻書,召試學士院,授隴州防禦推官。入遷光祿寺丞,丁內艱,起就職。久之,轉秘書丞,改屯田員外郎、知鄂州,以善政聞。



 端拱初,真宗在襄邸,遷庫部,充記室參軍,賜金紫。初,廣順中,周世宗節制澶州,礪贄文見之,館接數
 日。世宗入朝,礪處僧舍,夢古衣冠者曰:「汝能從乎?」礪隨往,睹宮衛若非人間,殿上王者秉珪南向,總三十餘。礪升謁之,最上者前有案,置簿錄人姓名,礪見己名居首,因請示休咎。王者曰:「我非汝師。」指一人曰:「此來和天尊,異日汝主也,當問之。」其人笑曰:「此去四十年,汝功成,予名亦顯矣。」礪再拜,寤而志之。礪初名勵,以籍作礪,遂改之。至是,受命謁見藩府,歸謂子曰:「吾今見襄王儀貌,即所夢來和天尊也。」遷水部郎中。真宗尹開封,礪為推官。
 真宗嘗問礪:「何年及第?」礪唯唯不對。後知其唱名第一,自悔失問,謂礪不以科名自伐,甚重之。儲宮建,兼右諭德,轉度支郎中。即位,拜給事中、判吏部銓。未幾,召入翰林為學士。咸平初,知貢舉,俄拜工部侍郎、樞密副使。二年,卒,年六十九。真宗軫悼,謂宰相曰:「礪介直清苦,方當任用,遽此淪謝。」即冒雨臨其喪。礪僦舍委巷中,乘與不能進,步至其第,嗟憫久之。廢朝,贈兵部尚書,中使護葬。



 礪為文尚繁,無師法,每詩一題或數十篇。在翰林,制誥
 迂怪,見者哂之。有文集二十卷。子嶠至祠部郎中,嶧至太常博士,峭至太子中舍。少子嵎,至道初與張庶凝刊校真宗儲邸書籍,真宗即位,皆賜進士出身、直史館。嵎至祠部郎中,庶凝至太常丞。



 宋湜,字持正,京兆長安人。曾祖擇,牟平令。祖贊,萬年令。父溫故,晉天福中進士,至左補闕;弟溫舒,亦進士,至職方員外郎,兄弟皆有時名。湜幼警悟,早孤,與兄泌勵志篤學,事母以孝聞。溫舒典耀州,湜侍行,代作箋奏,詞敏
 而麗。溫舒拊背曰:「此兒真國器,恨吾兄不及見也。」太平興國五年進士,釋褐將作監丞、通判梓州榷鹽院,就遷右贊善大夫。宋準薦其文,拜著作郎、直史館,賜緋。雍熙三年,以右補闕知制誥,與王化基、李沆並命,仍賜白金五百兩、錢五十萬。加戶部員外郎,與蘇易簡同知貢舉,俄判刑部,賜金紫。



 淳化二年,妖尼道安訟大理斷獄不當,湜坐累,降均州團練副使。時母老,湜留其室奉養。移汝州,與王禹偁並召入,為禮部員外郎、直昭文館。五年,
 以職方員外郎再知制誥、判集賢院,知銀壹、通進、封駁司。至道元年,為翰林學士,知審官院、三班。又兼修國史、判昭文史館事,加兵部郎中。



 真宗即位,拜中書舍人。丁內艱,起復。咸平元年冬,改給事中,充樞密副使。真宗北巡,將次大名,以扈從軍列為行陣,親御鎧甲於中,諸王、樞密介胃以從,命湜與王顯分押後陣。駐蹕數日,常召見便殿,方奏事,疾作僕地。內侍掖出,太醫診視,撫問相繼,以疾亟聞。明年正月,真宗臨視,許以先歸,賜衾褥,曰:「
 此朕嘗御者,雖故暗,亦足禦道途之寒。」又遣內侍護送供帳,至澶州,卒,年五十一。廢朝,贈吏部侍郎。以子綸為太祝,純為奉禮郎;弟某為光祿寺丞,湛為大理寺丞;侄孫選同學究出身。真宗再幸河朔,追悼之,加贈刑部尚書,謚曰忠定。



 湜風貌秀整,有醞藉,器識沖遠,好學,美文詞,善談論飲謔,曉音律,妙於弈棋。筆法遒媚,書帖之出,人多傳效。喜引重後進有名者,又好趨人之急,當世士流,翕然宗仰之。有文集二十卷。



 湜兄泌,太平興國二年
 進士,至起居郎、直史館、越王府記室參軍。



 溫舒三子,沆、澥、濤。沆,剛率,喜談兵。太平興國五年進士,歷左正言、京西轉運使、度支判官。淳化二年,呂蒙正罷相,沆坐親黨,貶宜州團練副使,起為太子中允,換如京副使。咸平中,遣與梅詢使西京為安撫使,未行,罷為環慶路都監。與知環州張從古擅發兵襲敵,不與部署葉謀,又士卒有死傷者,責授供奉官。後為文思副使、京西提點刑獄,卒。澥有清節,居長安不仕,與種放、魏野游,多篇什酬唱。濤,
 端拱二年進士,歷殿中丞、知襄城縣,以政績聞,賜緋魚。歷鹽鐵判官,累遷監察御史、知虢州。純及泌子緯皆至殿中丞。



 王嗣宗,字希阮,汾州人。曾祖同節,寶鼎令。祖待價,汾州防禦推官。父夢證,成州軍事判官。嗣宗少力學自奮,游京師,以文謁王祐,頗見優待。開寶八年,登進士甲科,補秦州司寇參軍。侍御史路沖知州事,為政苛急,盜賊群起。嗣宗乘間極言其闕失,沖大怒,系嗣宗於獄,又教無
 賴民被罪者訟嗣宗治獄枉濫。朝廷遣殿中丞王廷範按之,具獲訟者誣罔狀,嗣宗乃得釋。



 太宗征河東,嗣宗陳邊事,召赴行在,授大理寺丞、通判睦州,改右贊善大夫、徙河州。太宗遣武德卒潛察遠方事,嗣宗械送京師,因奏曰:「陛下不委任天下賢俊,猥信此輩以為耳目,臣竊不取。」太宗怒其橫,遣使械嗣宗下吏,削秩。會赦,復官,尋以秘書丞通判澶州,並河東西,植樹萬株,以固堤防。上言:「本州榷酤斗量,校以省鬥不及七升,民犯私釀者
 三石以上坐死,有傷深峻,臣恐諸道率如此制,望詔自今並準省斗定罪。」從之。入為三司開拆推官,以左正言充河北轉運副使。時邊境用兵,崔翰為大將,嗣宗每以苦言激其展效,就賜緋魚。太宗將議親征,嗣宗上疏言契丹必不至之狀,甚見嘉納。改左司諫,賜白金千兩。入為度支判官,改駕部員外郎。妻病,夜抉本司署門取藥,為直官宋鎬所發,坐罷職。頃之,出知興元府,徙京西轉運使。又移河北,賜金紫,貝州驍捷卒五十餘人謀竊發,
 嗣宗率吏悉擒之,優詔嘉獎。遷虞部郎中,賜錢百萬。



 至道初,移河東轉運使,以為政暴率聞。徙知耀州,又知同州,加比部郎中、淮南轉運使、江浙荊湖發運使。揚、楚間有窄家神廟,民有疾不餌藥,但竭致祀以徼福。嗣宗撤其廟,選名方,刻石州門,自是民風稍變。初,漕運經泗州浮橋,舟多覆壤,嗣宗徙置城隅,遂獲安濟。又建議外任官奉薄,貪猥者或致豐給,廉謹者終嬰貧匱,請以公田均賜之。就改職方郎中。



 咸平三年,以漕運稱職,就拜太
 常少卿。逾年,以右諫議大夫充三司戶部使,改鹽鐵使。嘗與度支使梁鼎、戶部使梁顥同對,言曰:「國家經費甚繁,賦入漸少,加以冗食者眾,尤為耗蠹,所宜裁節。若用度不足,即復重擾於民矣。況西北二邊未平,有饋運之煩,臣等會議,事可省者,願條列以聞。」從之。明年,將郊祀,嗣宗因條上應奉諸物以及工作,凡減雜物十萬六千,省工九萬九千。又言計省條奏,事有可紀者,望令判使一員,撰錄送史館。詔以三司務繁,不當日有纂錄,可逐
 季錄送。會罷三部使,改左諫議大夫,知通進、銀臺司兼門下封駁事,出知並州兼並代部署。州境有臥龍王廟,每窮冬,闔境致祭,值風雪寒甚,老幼踣於道,嗣宗亟毀之。轉運使鄭文寶上其政績,有詔褒美。先是,西邊市馬,以給北邊戰士,有瘠弱者即送闕下,署月道遠多死。嗣宗建議,以汾州地涼,接樓煩諸監,美水草,請就牧放,從之。召拜御史中丞。



 大中祥符間,真宗告謁太廟,嗣宗立班失儀,因自首。真宗謂憲官當守禮法,以其性粗略,不
 之責。加兼工部侍郎、權判吏部銓。嗣宗剛果率易,無所畏憚,每進見,極談時事,或及人間細務。頗輕險好進,深詆參知政事馮拯之短,遂結宰相王旦弟旭,使達意於旦以為助。旦疾其醜行,因力庇拯,嗣宗大怒。知制誥王曾從妹適孔冕家,閨門不睦。曾從東封,至冕家啜茗中毒,得良藥乃解。事已暴露,曾密疏方行大禮,願罷推究。宰相亦以冕先聖後,將有褒擢,乃隱其事。嗣宗獨謂曾誣構冕,懼反坐,乃求寢息。會愆雨,嗣宗請對,言:「孔冕為
 王曾所訟,儻朝旨鞫問,加之鍛煉,則冕終負冤枉。又侯德昭援赦敘緋,年考未滿,以欺詐得之,非吏部令史自首,亦無由知。沿堂行首李永錫坐贓除名,復引充舊職,尋送銓授令錄。」真宗亟召王旦等詰之。旦曰:「孔冕之罪,朝議特為容隱,不令按問,誠非冤枉也。德昭據吏部奏驗,乃行制命,及其首露,即已追奪。永錫先為縣吏,坐為本部節度市羊不輸算除名,及沿堂闕人,李沆以其魁梧,因選擬官,復用為副行首。在省祗事四年,陳牒乞班
 敘用,因復送銓。」真宗曰:「止此,乃致旱邪?」嗣宗理屈,復以他辭侵旦,旦不與抗,乃已。明年十月,嗣宗復請對,言:「去歲八月至今年十月不雨,宿麥不登。及秋,兗、鄆苦雨,河溢害稼,刑政有失,致成災沴。孔冕冤枉,播在人口,王曾尚居近班,願示黜退,以正朝典,臣請露章以聞。」真宗語王旦等曰:「曾實無罪,若嗣宗上章,亦須裁處。」旦曰:「冕不善之跡甚眾,但以宣聖之後不欲窮究,謂其冤枉,感傷和氣,恐未近理。」趙安仁曰:「今若再行按問,冕何能免罪?」
 王欽若曰:「臣請審問嗣宗,若再鞫冕,不能自隱,如何區處?」明日,嗣宗復對,且謝前言之失,真宗亦優容之。其強妄多此類。



 將祀汾陰,以永興重地,思得大臣才兼文武者鎮之。因謂宰相曰:「嗣宗嘗自言知武事,可授廉車以當此任,宜召問之。」嗣宗願奉詔,即拜耀州觀察使、知永興軍府。真宗作詩賜之。時種放得告歸山,嗣宗逆於傳舍,禮之甚厚。放既醉,稍倨,嗣宗怒,以語譏放。放曰:「君以手搏得狀元耳,何足道也!」初,嗣宗就試講武殿,搏趙昌
 言帽,擢首科,故放及之。嗣宗愧恨,因上疏言:「所部兼並之家,侵漁眾民,凌暴孤寡,凡十餘族,而放為之首。放弟侄無賴,據林麓樵採,周回二百餘里,奪編甿厚利。願以臣疏下放,賜放終南田百畝,徙放嵩山。」疏辭極於詬辱,至目放為魑魅。真宗方厚待放,令徙居嵩陽避之。



 四年,邠寧陳興擅釋劫盜,徙嗣宗知邠州兼邠寧環慶路都部署。城東有靈應公廟,傍有山穴,群狐處焉,妖巫挾之為人禍福,民甚信向,水旱疾疫悉禱之,民語為之諱「狐」
 音。前此長吏,皆先謁廟然後視事。嗣宗毀其廟,熏其穴,得數十狐,盡殺之,淫祀遂息。徙知鎮州,發邊肅奸贓,肅坐貶。嗣宗嘗言徙種放、掘邠狐、按邊肅,為去三害。



 居二歲,召還,授樞密副使、檢校太保。寇準為使,嗣宗與之不葉,累表解職,授檢校太傅、大同軍節度、知許州。嗣宗嘗游是州,別墅在焉,時人以為榮。移知河南府。天禧初,改感德軍節度,洛下訛言相驚。徙知陜州,再表請老,且求入覲,遣使召還。郊祀,改靜難軍節度。既至闕下,病足,不
 能朝謁,乃求再知許州,不復議休退。寇準為相,素惡之,特命以左屯衛上將軍、檢校太尉致仕。表求面辭,以足疾艱於拜起,特免舞蹈,許其子扶掖之。對數刻,賜錢百萬,還許下。準貶,朝議以嗣宗藩輔舊臣,特令月給奉五十千。嗣宗尤睦宗族,撫諸侄如己子,著遺戒以訓子孫勿得析居,又令以《孝經》、弓劍、筆硯置壙中。五年,卒,年七十八。廢朝,贈侍中。謚曰景莊。錄其子二人、甥二人官。



 嗣宗事三朝,最為宿舊。所至以嚴明御下,尤傲狠,務以醜
 言凌挫群類。為中丞日,嘗忿宋白、郭贄、邢昺七十不請老,屢請真宗敕其休致,又遣親屬諷激之。及嗣宗晚歲疾甚,猶享厚祿,徘徊不去,嘗謂人曰:「僕惟此一事,未能免物議。」眾皆□蚩之。嗣宗好為文,而札尤甚。奉祀之歲,近臣皆為頌記,宰相以嗣宗所撰,不足發揮盛德,慮為後所誚,乃不許刻石。所著有《中陵子》三十卷。



 子堯臣,內殿承制;唐臣,太子中舍。從子舜臣,供奉官、閣門祗候;禹臣,太子中舍。



 李昌齡,字天錫,宋州楚丘人。曾祖確,膠水令。祖譚,邯鄲令。父運,太常卿。昌齡,太平興國三年舉進士,大理評事、通判合州。歷將作監丞、右贊善大夫、通判銀州。京城開金明池,昌齡獻詩百韻,太宗嘉之,擢右拾遺、直史館,賜緋。改右補闕,出知滁州。丁內艱,起為淮南轉運使,轉戶部員外郎、知廣州。



 廣有海舶之饒,昌齡不能以廉自守,淳化二年代還。初,運嘗典許州,有第在城中,昌齡包苴輜重悉留貯焉,其至京城,但藥物藥器而已。會有言其
 貪者,太宗以為誣,召賜金紫,擢禮部郎中,逾月,為樞密直學士。昌齡上言:「廣州市舶,每歲商舶至,官盡增價買之,良苦相雜,少利。自今請擇其良者,官如價給之,苦者恣其賣,勿禁。雷、化、新、白、惠、恩等州山林有群像,民能取其牙,官禁不得賣。自今宜令送官,以半價償之,有敢隱匿及私市與人者,論如法。」詔皆從之。



 是秋,初置審刑院於禁中。凡獄具上奏,先申審刑院,印付大理、刑部斷覆以聞,又下審刑中覆裁決,以付中書,當者行之,否則宰
 相聞以論決。命昌齡知院事。月餘,又權判吏部流內銓,數日,授右諫議大夫,充戶部使。



 三年,改度支使,拜御史中丞。下詔御史臺,合行故事並條奏以聞,獄無大小,自中丞以下皆親臨鞫問,不得專責所司,李繼隆受命河朔征討,不赴臺辭,昌齡糾之,遣吏追還,罰奉。又劾陜西轉運使鄭文寶生事邊境,築城沙磧,輕變禁法,文寶坐貶湖外。



 至道二年,以本官參知政事。占謝便殿,太宗謂曰:「中書政本,當進用善良,博詢眾議,以正道臨之,即怨
 謗無由而生矣。」昌齡居位,頗選心耎無所建明。真宗即位,加戶部侍郎。坐交結王繼恩,貶忠武軍節度行軍司馬。



 咸平二年,起為殿中少監。會詔群臣言邊事,昌齡求面陳事機,不報。王均之亂,命知梓州。知雜御史範正辭劾其廣舶宿犯,亟代還,知河陽。丁外艱,起復,奉朝請,以風恙求領小郡,復得光州,就改光祿卿。疾,不能治事。轉運使以聞,命守本官分司西京。尋請致仕,真宗曰:「昌齡素無清譽。」乃授秘書監,遂其請。大中祥符元年,卒,年七十
 二。廢朝,錄子虞卿試將作監主簿。昌齡兄昌圖至國子博士,弟昌言至太子中舍。昌言子晉卿、仲卿、耀卿,並進士及第,晉卿為秘書丞。從子紘。



 紘字仲綱。父克明,仕至提點廣東刑獄。紘,進士及第,試秘書省校書郎、知歙縣。地產黃金,民輸以代賦,後金竭,責其賦如故。紘奏罷之。歷知於潛、剡縣,治有惠愛。御史知雜呂夷簡薦之,改著作佐郎、監丹陽縣酒稅,知靈池縣。



 劉均、蔡齊舉為御史臺推直官,拜監察御史。時召成
 都府樂工許朝天等補教坊,紘言:「陛下即位,尚未能顯嚴穴之士,而首召伶人,非所以廣德美於天下。」朝天等遂罷歸。遷殿中侍御史。閣門使王遵度領皇城,遣卒刺事,告賈人有為契丹間諜者,捕系皇城司按劾。命紘覆訊,紘悉得其冤,抵卒罪,降遵度曹州兵馬都監。



 判三司開拆司。輔郡旱,流星墜西南有聲,會僧禳於文德殿,紘奏曰:「文德殿布政會朝之正位,每災異,輒聚緇黃贊唄於其間,何以示中外?」改鹽鐵判官,歷梓州、陜西、河北路
 轉運使,遷侍御史。建言:「西北久通好,士習安佚,不知戰陣之法。宜擇良將,練精卒,去冗惰,實倉廩,豐財用,為守禦備。」舉種世衡等數人,及奏罷貢餘物遺近臣。遷知雜事、權同判流內銓。



 為三司度支副使,使契丹。故事,奉使者以皇城卒二人與偕,察其舉措,使者悉姑息以避中傷。前此劉隨為所誣,坐貶,久未復。紘使還,具言其枉,稍徙隨南京。除天章閣待制、河北都轉運使,遷刑部郎中,還,同知通進、銀臺司,進龍圖閣直學士、知秦州,卒。



 紘方
 介有吏材,篤於交游,與劉顏為友,顏死,移任子恩官其子。



 弟緯,起家三班借職,杜衍薦為閣門祗候,鎮戎軍瓦亭砦都監。積勞累遷至河北緣邊安撫副使。韓琦薦知保州,以左騏驥使、榮州刺史知雄州。治兵頗嚴,不事廚傳,數與宦者爭利害。積公使錢貯米三千斛為常平倉,奏下其法他州。遷西上閣門使,留再任,卒。子師中至天章閣待制。



 趙安仁,字樂道,河南洛陽人。曾祖武唐,虢州刺史。父孚
 字大信。周顯德初,舉進士,調補開封尉。乾德中,為浦江令,持父喪,服闋,攝永寧令。會親征太原,部送本邑糧饋,民懷其惠,列狀以聞,即真授其任,擢宗正丞。開寶中,初置衣庫,令孚主之。俄坐事連逮抵罪,語見《趙普傳》。



 太宗即位,起為國子監丞、知袁州。還,知開封府司錄參軍事,受詔與殿中侍御史柴成務、供奉官葛彥恭、殿直郭載行視黃河,分南北岸按行,復遙堤以紓湍決。孚言治遙堤不如分水勢,於是建議於澶、滑二州立分水之制。時
 決河未平,重惜民力而寢焉。朝廷議行封禪,孚上《封禪頌》,召拜秘書丞,賜緋魚。受詔鞫開封獄,得其非辜者,即日授推官。遷監察御史,出知舒州,改殿中侍御史。



 雍熙中,詔詢文武御戎之策。孚奏議曰:「臣愚以為不用干戈,不勞飛挽,為萬世之利者,敢獻其說,惟明主擇之。古者兵交使在其間,雖飛矢在上,走驛在下,蓋信義不可廢也。昔苗民逆命,帝乃誕敷文德,而有苗格。又仲尼曰:『有能一日克己復禮,天下歸仁。』只如並門一方,歷代難取,
 聖襟英斷,一舉成功。當其逆城危於累卵,生聚懷伏,而陛下猶遣通事舍人薛文寶入城諭之。日者北邊未賓,全燕猶梗,再興軍旅,將復土疆。臣竊計屯戍邊陲,故非獲已,暴露原野,豈是願為?欲望朝廷通達國信,近鑒唐高祖之降禮,遠法周古公之讓地。聖人以百姓之心為心,君子見幾而作,諭以禍福,示以恩威,議定邊疆,永息征戰。養民事天,濟時利物,莫過於此。臣又計彼雖嗜好不同,然去危就安,厭勞喜逸,亦人情之所同也。」上嘉之。
 雍熙中,廷策貢士,而安仁預為考會,賜金紫,因顧安仁問孚年幾,安仁曰:「臣父年六十二。」上曰:「孚,名士也。」亟召對,亦賜金紫。明年,卒。



 安仁生而穎悟,幼時執筆能大字,十三通經傳大旨,早以文藝稱。趙普、沈倫、李昉、石熙載咸推獎之。雍熙二年,登進士第,補梓州榷鹽院判官,以親老弗果往。會國子監刻《五經正義》板本,以安仁善楷隸,遂奏留書之。



 歷大理評事、光祿寺丞,召試翰林,以著作佐郎直集賢院,賜緋。時王侯、內戚家多以銘誄為托。
 太宗制九紘琴、五紘阮,時多獻賦頌,上嘉文物之盛,悉閱覽,訂其工拙。時稱安仁、李宗諤、楊億辭雅贍,召詣中書獎諭。翌日,改遷太常丞。



 真宗即位,拜右正言,預重修《太祖實錄》。上出師大名,安仁上疏曰:「臣以為有急務者三,大要者五。急務三者:其一,激勵戎臣,舉勸懲之典;其二,振救邊民,行優恤之惠;其三,車駕還京,重神武之威。大要五者:其一,選將略;其二,持兵勢;其三,求軍謀;其四,修軍政;其五,愛民力。」



 咸平三年,同知貢舉。未幾,知制誥,副
 夏侯嶠巡撫江南,還,知審刑院。嘗有將校笞所部卒死,罪議大闢。安仁以軍中之令,非嚴不整,遂獲免死。繼判尚書刑部兼制置群牧使,同知三班、審官院。景德初,翰林學士梁顥召對,詢及當世臺閣人物,上稱安仁文行。尋顥卒,即以安仁為工部員外郎,充翰林學士。



 初,孚極陳和好之利。至是,安仁從幸澶州,會北邊請盟,首命安仁撰答書,又獨記太祖時聘問書式。遼使韓杞至,道命接伴,凡覲見儀制,多所裁定。館舍夕飲,杞舉橙子曰:「此
 果嘗見高麗貢。」安仁曰:「橙橘產吳、楚,朝廷職方掌天下圖經,凡他國所產靡不知也。今給事中呂祐之嘗使高麗,未聞有橙柚。」杞失於誇誕,有愧色。杞既受襲衣之賜,且以長為解,將辭復左衽。安仁曰:「君將升殿受還書,天顏咫尺,如不衣所賜之衣,可乎?」杞乃服以入。



 及姚東之至,又令安仁接伴。東之談次,頗矜兵強戰勝。安仁曰:「老氏云:『佳兵者不祥之器,聖人不得已而用之。』勝而不美,而美之者,是樂殺人也,樂殺人者不得志於天下。」東之
 自是不敢復言。王繼忠將兵陷沒,不能死節而反事之,東之屢稱其材。安仁曰:「繼忠早事藩邸,聞其稍謹,不知其它。」其敏於酬對,切中事機,類如此。時論翕然,稱其得體,上益器之,自是有意柄用。安仁又集和好以來事宜,及採古事,作《戴鬥懷柔錄》三卷以獻。



 二年春,又與晁迥等同知貢舉。三年,以右諫議大夫參知政事,俄修國史。大中祥符初,議封禪,與王欽若並為泰山經制度置使、判兗州。禮畢,復拜工部侍郎。內外書詔有切要者,必經
 其裁。進秩刑部。五年,以兵部侍郎仍兼修史,奉祀,又同知禮儀院。八年,知貢舉。三典春闈,擇士平允,是故獨無譏誚,上再賜詩嘉之。



 尋知兼宗正卿。舊制,宮闈令,凡有議奏與寺連署。上以安仁舊德,俾知寺,以次列狀取裁。寺掌玉牒屬籍,梁周翰始創其制而未備,安仁重加詳定,又為《仙源積慶圖》,皆統例精簡。奏置修玉牒官,事具《職官志》。國史成,遷右丞。是夏,又為景靈宮副使。屢得對言事,嘗奏曰:「方今治定功成,固軼前代,陛下尚親庶政,
 旰食忘倦,然而君臨之大,所宜分飭有司,為式於天下。」遂詔諸司掌常務有條例者,毋或奏稟。天禧二年,改御史中丞。請給御寶印歷,書三院御史彈糾事。五月,暴疾卒,年六十一。廢朝,贈吏部尚書,謚文定,以其子溫瑜為大理寺丞,良規為奉禮郎,承裕為正字。



 安仁質直純愨,無所矯飾,寬恕謙退,與物無競,雖家人僕使,未嘗見其喜慍。女弟適董氏,早寡,取歸給養。其甥董靈運尚幼,躬自訓導,為畢婚娶。幼少與宋元輿同學,元輿門地貴盛,
 待安仁甚厚。元輿蚤卒,家緒浸替,安仁屢以金帛濟之。善訓諸子,各授一經。尤嗜讀書,所得祿賜,多以購書。雖至顯寵,簡儉若平素。時閱典籍,手自讎校。三館舊闕虞世南《北堂書鈔》,惟安仁家有本,真宗命內侍取之,嘉其好古,手詔褒美。尤知典故,凡近世典章人物之盛,悉能記之。喜誨誘後進,成其聲名,當世推重之。有集五十卷。溫瑜,後為國子博士。



 良規字符甫。父安仁奏為秘書省正字、同判太常寺。張
 知白薦之,召試,賜進士及第。用王曙舉,擢集賢校理兼宗正丞,預修《會要》。坐宗正吏盜太廟神御物,出通判蘄州,徙河南府,知泰、滁二州。歷京西陜西路提點刑獄、荊湖南路轉運使,奏罷馬氏時所賦丁口米數萬石。權判三司開拆司、度支勾院,直集賢院、知廬州,積官至光祿卿,罷職。初與張憲、掌禹錫、齊廓、張子思並為太常少卿兼館職,當進諫議大夫,而執政靳之,止遷卿。故事,卿不兼職,故皆罷。未幾,皆還之。



 改直秘閣、同判宗正事,遷秘
 書監,知同、陜、相三州。陜歲饑,百姓請閣殘稅二分,為官伐芟,以給河埽。或以為須報乃可行,良規曰:「若爾,無及矣。」檄縣遂行,而以擅命自劾。進太子賓客、權判殿中省,遷尚書工部侍郎、判本部、知濠州,卒。良規所至州郡,為政不甚力,然善委任佐屬,祿賜多分贍族人,餘皆輸之酒家。子君錫。



 君錫字無愧。性至孝。母亡,事父良規不違左右,夜則寢於旁。凡衾稠薄厚、衣服寒溫、藥石精粗、飲食旨否、櫛發
 翦爪、整冠結帶,如《內則》所載者,無不親之。及登進士第,以親故不願仕。良規每出,必扶掖上下,至雜立僕御中。嘗從謁文彥博,彥博異其容止,問而知之,語諸子,令視以為法。



 良規沒,調知武強縣。從韓琦大名幕府。彥博及吳充在樞管,更薦之為檢詳吏房文字,徙知大宗正丞,加秘閣校理,改宗正丞。時增諸宗院講書教授官,而逐院自備緡錢為月饋,貧者或不能以時致,宗師輒移文督取。君錫言:「國家養天下士於太學,尚不較其費,安有
 教育宗室令自行束修之理!」詔悉從官給。歷開封府推官。



 元祐初,遷司勛右司郎中、太常少卿,擢給事中。論蔡確、章惇有罪不宜復職;大河不可輕議東回,請亟罷修河司,以省邦費,寬民力。蘇軾出知杭州,君錫言:「軾之文,追攀《六經》,蹈藉班、馬,知無不言。壬人畏憚,為之消縮;公論倚重,隱如長城。今飄然去國,邪黨必謂朝廷稍厭直臣,且將乘隙復進,實系消長之機。不若留之在朝,用其善言則天下蒙福,聽其讜論則聖心開益,行其詔令則
 四方風動,而利博矣。」進刑部侍郎、樞密都承旨,拜御史中丞。即上疏勸哲宗親講學,廣諮問,為躬政之漸。



 君錫素有志行,後隨人低昂,無大建明。初稱蘇軾之賢,遇賈易劾軾題詩怨謗,即繼言「軾負恩懷逆,無禮先帝,願亟正其罪。」宣仁後覽之不悅,曰:「君錫全無執守。」復以吏部侍郎、天章閣待制知鄭陳澶三州、河南府,徙應天。因清明出郊,具奠謁杜衍、張忭、張方平、趙概、王堯臣、蔡抗、蔡挺之塋,邀七家子孫,陪祭於側,時人傳其風義。紹聖中,
 貶少府少監,分司南京,卒,年七十二。紹興六年,贈徽猷閣直學士。



 陳彭年,字永年,撫州南城人。父省躬,鹿邑令。彭年幼好學,母惟一子,愛之,禁其夜讀書。彭年篝燈密室,不令母知。年十三,著《皇綱論》萬餘言,為江左名輩所賞。唐主李煜聞之,召入宮,令子仲宣與之游。金陵平,彭年師事徐鉉為文。太平興國中,舉進士,在場屋間頗有雋名。嘗因京城大酺,跨驢出游構賦,自東華門至闕前,已口占數
 千言。然佻薄好嘲詠,頻為宋白所黜,雍熙二年始中第。



 調江陵府司理參軍。因監決死囚,怖之,換江陵主簿,歷澧、懷二州推官。在懷,深為知州喬惟岳倚任。會樊知古為河北轉運,以親嫌,徙澤州,丁內艱免。御史中丞王化基薦其才,改衛尉寺丞,遷秘書郎,為大理寺詳斷官。坐事出監湖州鹽稅,尋又停官。彭年素貧窶,居喪免職,賴僕人傭販以濟。真宗即位,復為秘書郎。喬惟嶽刺史海州,及知蘇、壽二州,並表彭年通判州事。



 咸平三年,屢上
 疏言事,召試學士院,遷秘書丞、知閬州。未行,改金州。四年,上疏曰:「夫事有雖小而可以建大功,理有雖近而可以為遠計者,其事有五:一曰置諫官,二曰擇法吏,三日簡格令,四曰省冗員,五曰行公舉。此五者,實經世之要道,致治之坦塗也。」會詔舉賢良方正,翰林學士朱昂以彭年聞,召之,辭以貧乏,請終秩。



 景德初,代還,真秘閣。杜鎬、刁衎薦其該博,命直史館兼崇文院檢討。又代潘慎修起居注,賜緋魚。獻《大寶箴》曰:



 二儀之內,最靈者人。生
 民之中,至大者君。民既可畏,天亦無親。



 所輔者德,所歸者仁。恭己御下,輝光益新。載籍斯在,謀猷備陳。



 內綏萬姓,外撫百蠻。治亂所始,言動之間。觀之則易,處之甚難。



 由是先哲,喻彼投艱。茍能慮未,乃可防閑。審求逆耳,無惡犯顏。



 既庶而富,教化乃施。慈儉之政,富庶之基。鰥寡孤獨,人之所悲。



 發號施令,宜先及之。黃發鮐背,心實多知。左右侍從,何尚於茲。



 瞻言百闢,咸代天工。儻無虛授,可建大中。克彰慎柬,惟藉至公。



 知人則哲,聽德則聰。才
 固難備,道亦少同。葑菲罔舍,杞梓乃充。



 不扶自直,惟蓬在麻。非揀莫見,惟金在沙。參備顧問,必辨忠邪。



 獻替以正,裨益無涯。自匿草澤,亦有國華。訪此髦士,可拒朋家。



 三章之立,庶民作程。欽哉恤哉,可以措刑。七代之建,奸孽是平。



 本仁本義,可以弭兵。是為齊禮,亦曰好生。有教無類,自誠而明。



 宗廟社稷,饗之以恭。宮室苑囿,誡之在豐。春鬼秋獮,不廢三農。



 擊石拊石,用格神宗。使人以悅,乃克成功。治國以政,罔或不從。



 濟濟多士,用之有光。硜
 硜小器,謀之弗臧。忠言致益,豈讓膏粱。



 六藝為樂,寧後笙簧。任賢勿貳,堯所以昌。改過不吝,湯所以王。



 六合至廣,萬匯尤多。風俗靡一,嗜欲相摩。如馭朽索,若防決河。



 左契斯執,六轡遂和。導之以德,民免嬰羅。不懈於位,俗乃偃戈。



 先王之訓,罔不咸然。吾君之治,亦取斯焉。小心翼翼,終日幹幹。



 三靈降鑒,百祿無愆。由茲率土,永戴先天。巍巍洪業,億萬斯年。



 頃之,預修《冊府元龜》。三年,遷右正言,充龍圖閣待制,賜金紫。先是,詔諫官御史舉職言
 事,唯彭年與侍御史賈翱數有章奏,建白彈射,真宗令中書置籍記之。加刑部員外郎。與晁迥同知貢舉,請令有司詳定考試條式。真宗因命彭年與戚綸參定,多革舊制,專務防閑。其所取者,不復揀擇文行,止較一日之藝,雖杜絕請托,然置甲等者,或非宿名之士。



 大中祥符中,議建封禪,彭年預詳定儀注,上言辨正包茅之用。禮成,進秩工部郎中,加集賢殿修撰。三年,改兵部郎中、龍圖閣直學士。遷右諫議大夫兼秘書監,詔就賜食廳編
 次《太宗御集》,賜勛上柱國。



 嘗因奏對,真宗謂之曰:「儒術污隆,其應實大,國家崇替,何莫由斯。故秦衰則經籍道息,漢盛則學校興行。其後命歷迭改,而風教一揆。有唐文物最盛,朱梁而下,王風寢微。太祖、太宗丕變弊俗,崇尚斯文。朕獲紹先業,謹導聖訓,禮樂交舉,儒術化成,實二後垂裕之所致也。又君之難,由乎聽受;臣之不易,在乎忠直。其君以寬大接下,臣以誠明奉上,君臣之心皆歸於正。直道而行,至公相遇,此天下之達理,先王之成
 憲,猶指諸掌,孰謂難哉!」彭年曰:「陛下聖言精詣,足使天下知訓,伏願躬演睿思,著之篇翰。」真宗為制《崇儒術》、《為君難為臣不易》二論示之。彭年復請示輔臣,刻石國子監焉。



 六年,召入翰林,充學士兼龍圖閣學士,同修國史。彭年嘗謁王旦,旦辭不見。翌日,見向敏中。敏中以彭年所上文字示旦,旦瞑目不覽,曰:「是不過興建符瑞,圖進取耳。」真宗奉祀亳州太清宮,丁謂為經度制置使,以彭年副之。又與謂同知禮儀院,禮成,加給事中。時謂懇讓
 進秩,彭年亦辭之,不許,又為天書同刻玉副使。國史成,遷工部侍郎。九年,拜刑部侍郎、參知政事,判禮儀院,充會靈觀使。



 天禧大禮,為天書儀衛副使。又為參詳儀制奉寶冊使。正月九日,侍真宗朝天書,將詣太廟,退就中書閣中如廁,眩僕,肩輿還家。遣中使挾醫診療,旦夕存問。進兵部侍郎,表求罷奉,不許。二月,卒,年五十七。真宗親臨,涕泗久之。又睹所居陋弊,嘆息數四。廢朝,贈右僕射,謚曰文僖,錄子佺期大理寺丞,孫彥先太常寺奉禮
 郎。真宗前後賜彭年禦制歌詩凡六篇。彭年妻入謁,出彭年像示之,錫繼甚厚。



 彭年性敏給,博聞強記,慕唐四子為文,體制繁靡。貴至通顯,奉養無異貧約。所得奉賜,惟市書籍。大中祥符間,附王欽若、丁謂,朝廷典禮,無不參預。其儀制沿革、刑名之學,皆所詳練,若前世所未有,必推引依據以成就之。故時政大小,日有諮訪,應答該辯,一無凝滯,皆與真宗意諧。



 及升內閣,李宗諤、楊億皆在後。宗諤卒,億病退,而彭年專任矣。事務既叢,形神皆
 耗,遂舉止失措,顛倒冠服,家人有不記其名者。奉詔同編《景德朝陵地里》、《封禪》、《汾陰》三記,《閣門》、《客省》、《御史臺儀制》,又受詔編御集及宸章,集歷代婦人文集。所著《文集》百卷,《唐紀》四十卷。



 論曰:楊礪遭遇龍飛,致位崇顯,自以夢協其兆,而忠言善政,一無可述。惟棄官侍母,不以科名自伐,蓋有取焉。宋湜懿文多識,名動人主,至與李沆同命。雖去沆遠甚,然樂善好施,士類歸之,亦可尚也。王嗣宗治家能睦,為
 政可稱,所至立徹淫祀,亦人之所難。至於剛復少文,謀害王旦、王曾,與寇準相忤,其餘不足觀也矣。李昌齡累更劇任,遂階大用,黨邪徇貨,遂貽終身之玷,良可醜也。趙安仁言事,切中時弊,及答契丹書,不失祖宗規式,又能以兇器之言折敵,不使矜戰,可謂才辨之臣矣。其孫君錫于元祐反正,論格蔡確、章惇復官之命,庶幾無忝所生。陳彭年以辭藻被遇,上表獻箴,詳練儀制,若可嘉尚。乃附王欽若、丁謂,溺志爵祿,甘為小人之歸,豈不重
 可嘆也哉!



\end{pinyinscope}