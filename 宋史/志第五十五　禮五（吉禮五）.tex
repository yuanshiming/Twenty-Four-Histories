\article{志第五十五 禮五(吉禮五)}

\begin{pinyinscope}

 社稷岳瀆籍田先蠶奏告祈禜



 社稷,自京師至州縣,皆有其祀。歲以春秋二仲月及臘日祭太社、太稷。州縣則春秋二祭,刺史、縣令初獻,上佐、
 縣丞亞獻,州博士、縣簿尉終獻。如有故,以次官攝。若長吏職官或少,即許通攝,或別差官代之。牲用少牢,禮行三獻,致齋三日。其禮器數:正配坐尊各二,籩、豆各八,簠、簋各二,俎三。從祀籩、豆各二,簠、簋、俎各一。太社壇廣五丈,高五尺,五色土為之。稷壇在西,如其制。社以石為主,形如鐘,長五尺,方二尺,剡其上,培其半。四面宮垣飾以方色,面各一屋,三門,每門二十四戟,四隅連飾罘罳,如廟之制,中植以槐。其壇三分宮之一,在南,無屋。慶歷用
 羊、豕各二,正配位籩、豆十二,山罍、簠、簋、俎二,祈報象尊一。



 元豐三年,詳定所言:「社稷祝版、牲幣、饌物,請並瘞於坎,更不設燔燎。又《周禮·大宗伯》『以血祭社稷』,社為陰祀,血者幽陰之物,是以類求神之意。郊天先薦血,次薦腥,次薦爓,次薦熟。社稷、五祀,先薦爓,次薦熟。至於群小祀,薦熟而已。今社稷不用血祭,又不薦爓,皆違經禮。請以埋血為始,先薦爓,次薦熟。古者祭社,君南向於北墉下,所以答陰也,今社稷壝內不設北墉,而有司攝事,乃設
 東向之位,非是。請設北墉,以備親祠南向答陰之位,有司攝事,則立北墉下少西。《王制》曰:『天子社稷皆太牢,諸侯社稷皆少牢。』今一用少牢,殊不應禮。夫為一郡邑報功者,當用少牢;為天下報功者,當用太牢。所有春秋祈報太社、太稷,請於羊、豕外加角握牛二。」又言:「社稷之祭,有瘞玉而無禮玉,《開元禮》:奠太社、太稷,並以兩圭有邸。請下有司造兩圭有邸二,以為禮神之器,仍詔於壇側建齋廳三楹,以備望祭。」



 先是,州縣社主不以石。禮部以
 謂社稷不屋而壇,當受霜露風雨,以達天地之氣,故用石主,取其堅久。又《禮》:諸侯之壇半天子之制。請令州縣社主用石,尺寸廣長亦半太社之制。遂下太常,修入祀儀。元祐中,又從博士孫諤言,祭太社、太稷,皆設登歌樂。大觀,議禮局言:「太社獻官、太祝、奉禮,皆以法服;至於郡邑,則用常服。請下祭服制度於郡縣,俾其自制,弊則聽改造之。」



 紹興元年,以春秋二仲及臘前祭太社、太稷於天慶觀,又望祭於臨安天寧觀。十四年,始築壇壝於觀
 橋之東,立石主,置太社令一員,備牲牢器幣,進熟、望燎如儀。



 岳鎮海瀆之祀。太祖平湖南,命給事中李昉祭南嶽,繼令有司制諸嶽神衣、冠、劍、履,遣使易之。廣南平,遣司農少卿李繼芳祭南海,除去劉鋹所封偽號及宮名,易以一品服。又詔:「岳、瀆並東海廟,各以本縣令兼廟令,尉兼廟丞,專掌祀事。」又命李昉、盧多遜、王祐、扈蒙等分撰岳、瀆祠及歷代帝王碑,遣翰林待詔孫崇望等分詣諸廟
 書於石。六年,遣使奉衣、冠、劍、履,送西鎮吳嶽廟。



 太平興國八年,河決滑州,遣樞密直學士張齊賢詣白馬津,以一太牢沉祠加璧。自是,凡河決溢、修塞皆致祭。秘書監李至言:「按五郊迎氣之日,皆祭逐方岳鎮、海瀆。自兵亂後,有不在封域者,遂闕其祭。國家克復四方,間雖奉詔特祭,未著常祀。望遵舊禮,就迎氣日各祭於所隸之州,長史以次為獻官。」其後,立春日祀東嶽岱山於兗州,東鎮沂山於沂州,東海於萊州,淮瀆於唐州。立夏日祀南
 岳衡山於衡州,南鎮會稽山於越州,南海於廣州,江瀆於成都府。立秋日祀西嶽華山於華州,西鎮吳山於隴州,西海、河瀆並於河中府,西海就河瀆廟望祭。立冬祀北岳恆山、北鎮醫巫閭山並於定州,北鎮就北嶽廟望祭,北海、濟瀆並於孟州,北海就濟瀆廟望祭。土王日祀中嶽嵩山於河南府,中鎮霍山於晉州。



 真宗封禪畢,加號泰山為仁聖天齊王,遣職方郎中沈維宗致告。又封威雄將軍為炳靈公,通泉廟為靈派侯,亭山神廟為廣
 禪侯,嶧山神廟為靈巖侯,各遣官致告。詔泰山四面七里禁樵採,給近山二十戶以奉神祠,社首、徂徠山並禁樵採。車駕次澶州,祭河瀆廟,詔進號顯聖靈源公,遣右諫議大夫薛映詣河中府,比部員外郎丁顧言詣澶州祭告。秘書丞董溫其言:「漢以霍山為南嶽,望令壽州長吏春秋致祭。」禮官言:「雖前漢嘗以霍山為南嶽,緣今岳廟已在衡山,難於改制。其霍山如遇水旱祈求及非時,準別敕致祭,即委州縣奉行。」詔封江州馬當上水府,福
 善安江王;太平州採石中水府,順聖平江王;潤州金山下水府,昭信泰江王。



 及祀汾陰,命陳堯叟祭西海,曹利用祭汾河。車駕至潼關,遣官祠西岳及河瀆,並用太牢,備三獻禮。庚午,親謁華陰西嶽廟,群臣陪位,廟垣內外列黃麾仗,遣官分奠廟內諸神,加號嶽神為順聖金天王。還至河中,親謁奠河瀆廟及西海望祭壇。五月乙未,加上東岳曰天齊仁聖帝,南岳曰司天昭聖帝,西岳曰金天順聖帝,北岳曰安天元聖帝,中岳曰中天崇聖帝。
 命翰林、禮官詳定儀注及冕服制度、崇飾神像之禮。其玉冊制,如宗廟謚冊。帝自作《奉神述》,備紀崇奉意,俾撰冊文。有司設五岳冊使一品鹵簿及授冊黃麾仗、載冊輅、袞冕輿於乾元門外,各依方所。群臣朝服序班、仗衛如元會儀。上服袞冕,禦乾元殿。中書侍郎引五岳玉冊,尚衣奉袞冕升殿,上為之興。奉冊使、副班於香案前,侍中宣制曰:「今加上五岳帝號,遣卿等持節奉冊展禮。」咸承制再拜。奉冊使以次升自東階,受冊御坐前,降西階;副
 使受袞冕輿於丹墀,隨冊使降立丹墀西。玉冊發,至於朝元門外,帝復坐。冊使奉冊升輅,鼓吹振作而行。東嶽、北嶽冊次於瑞聖園,南嶽冊次於玉津園,西嶽、中嶽冊次于瓊林苑。及廟,內外列黃麾仗,設登歌。奉冊於車,奉袞冕於輿,使、副褲褶騎從,遣官三十員前導。及門,奉置幄次,以州長吏以下充祀官,致祭畢,奉玉冊、袞冕置殿內。又加上五岳帝後號:東曰淑明,南曰景明,西曰肅明,北曰靖明,中曰正明。遣官祭告。詔岳、瀆、四海諸廟,遇設
 醮,除青詞外,增正神位祝文。又改唐州上源桐柏廟為淮瀆長源公,加守護者。帝自制五岳醮告文,遣使醮告。即建壇之地構亭立石柱,刻文其上。



 天禧四年,從靈臺郎皇甫融請,凡修河致祭,增龍神及尾宿、天江、天記、天社等諸星在天河內者,凡五十位。



 仁宗康定元年,詔封江瀆為廣源王,河瀆為顯聖靈源王,淮瀆為長源王,濟瀆為清源王,加東海為淵聖廣德王,南海為洪聖廣利王,西海為通聖廣潤王,北海為沖聖廣澤王。皇祐四年,
 又以靈臺郎王大明言,汴口祭河,兼祠箕、斗、奎,與東井、天津、天江、咸池、積水、天淵、天潢、水位、水府、四瀆、九坎、天船、王良、羅堰等十七星在天河內者。五年,以儂智高遁,益封南海洪聖廣利招順王。其五鎮,沂山舊封東安公,政和三年封王;會稽舊封永興公,政和封永濟王;吳山舊封成德公,元豐八年封王;醫巫閭舊封廣寧公,政和封王;霍山舊封應聖公,政和封應靈王。東海,大觀四年,加號助順廣德王。



 紹興七年,太常博士黃積厚言:「岳鎮
 海瀆,請以每歲四立日分祭東西南北,如祭五方帝禮。」詔從之。



 乾道五年,太常少卿林慄言:「國家駐蹕東南,東海、南海,實在封域之內。自渡江以後,惟南海王廟,歲時降御書祝文,加封至八字王爵。如東海之祠,但以萊州隔絕,未嘗致祭,殊不知通、泰、明、越、溫、臺、泉、福,皆東海分界也。紹興中金人入寇,李寶以舟師大捷於膠西,神之助順,為有功矣。且元豐間嘗建廟於明州定海縣,請依南海特封八字王爵,遣官詣明州行禮。」詔可。



 籍田之禮,歲不常講。雍熙四年,始詔以來年正月擇日有事於東郊,行籍田禮。所司詳定儀注:「依南郊置五使。除耕地朝陽門七里外為先農壇,高九尺,四陛,周四十步,飾以青;二壝,寬博取足容御耕位。觀耕臺大次設樂縣、二舞。御耕位在壝門東南,諸侯耕位次之,庶人又次之。觀耕臺高五尺,周四十步,四陛,如壇色。其青城設於千畝之外。」又言:「隋以青箱奉穜稑,唐廢其禮。青箱舊無其制,請用竹木為之而無蓋,兩端設襻,飾以青;中分九
 隔,隔盛一種,覆以青帊。穜稑即早晚之種,不定穀名,請用黍、稷、秫、稻、粱、大小豆、大小麥,陳於箱中。」大禮使李昉言:「按《通禮》,乘耕根車,今請改乘玉輅,載耒耜於耕根車。又前典不載告廟及稱賀之制,今請前二日告南郊、太廟。耕禮畢,百官稱賀於青城。禮有勞酒,合設會於還宮之翼日,望如親祀南郊之制,擇日大宴。」詳定所言:「御耒耜二具,並盛以青絳,準唐乾元故事,不加雕飾。禮畢,收於禁中,以示稼穡艱難之意。其祭先農,用純色犢一,如
 郊祀例進胙,餘並權用大祠之制。皇帝散齋三日,致齋二日,百官不受誓戒。神農、后稷冊,學士院撰文進書。」以鹵簿使賈模等言,復用象輅載耒耜,以重其事。五年正月乙亥,帝服袞冕,執鎮圭,親享神農,以後稷配,備三獻,遂行三推之禮。畢事,解嚴,還行宮,百官稱賀。帝改御大輦,服通天冠、絳紗袍,鼓吹振作而還。禦乾元門大赦,改元端拱,文武遞進官有差。二月七日,宴群臣於大明殿,行勞酒禮。



 景德四年,判太常禮院孫奭言:「來年畫日,正
 月一日享先農,九日上辛祈穀,祀上帝。《春秋傳》曰:『啟蟄而郊,郊而後耕。』《月令》曰:『天子以元日祈穀於上帝。乃擇元辰,親載耒耜,躬耕帝籍。』先儒皆云:元日,謂上辛郊天也;元辰,謂郊後吉亥享先農而耕籍也。《六典》、《禮閣新儀》並云上辛祀昊天,次云吉亥享先農。望改用上辛後亥日,用符禮文。」



 明道元年,詔以來年二月丁未行籍田禮,而罷冬至親郊。遣官奏告天地、宗廟、諸陵、景靈宮,州都就告岳瀆、宮廟。其禮一如端拱之制,而損益之。禮成,遣
 官奏謝如告禮。



 元豐二年,詔於京城東南度田千畝為籍田,置令一員,徙先農壇於中,神倉於東南,取卒之知田事者為籍田兵。乃以郊社令辛公祐兼令。公祐請因舊鏺麥殿規地為田,引蔡河水灌其中,並植果蔬,冬則藏冰,凡一歲祠祭之用取具焉。先薦獻而後進御,有餘,則貿錢以給雜費,輸其餘於內藏庫,著為令。權管幹籍田王存等議,以南郊鏺麥殿前地及玉津園東南羨地並民田共千一百畝充籍田外,以百畝建先農壇兆,開
 阡陌溝洫,置神倉、齋宮並耕作人牛廬舍之屬,繪圖以進。已而殿成,詔以思文為名。



 政和元年,有司議:享先農為中祠,命有司攝事,帝止行耕籍之禮。罷命五使及稱賀、肆赦之類。太史局擇日不必專用吉亥。耕籍所乘,改用耕根車,罷乘玉輅。躬耕之服,止用通天冠、絳紗袍,百官並朝服。仿雍熙儀注,九卿以左右僕射、六尚書、御史大夫攝,諸侯以正員三品官及上將軍攝。設庶人耕位於諸侯耕位之南,以成終畝之禮。備青箱,設九穀,如
 隋之制。尋復以耕籍為大祠,依四孟朝享例行禮,又命禮制局修定儀注。



 孟春之月,太史擇上辛後吉日,皇帝親耕籍田,命有司以是日享先農、后稷於本壇,如常儀。前期,殿中監設御坐於思文殿,儀鸞司設文武官次殿門外之左右。其日早,奉禮郎設御耕褥位於耕籍所,尚舍設觀耕御坐於壇上,南向。典儀設侍耕群臣位於御耕之東西,設從耕群臣位於御耕之東南,西向,北上。奉禮郎設御耒席於三公之北,稍西,南向。太僕設御耕牛
 於御壇之西,稍北;太僕卿位於耕牛之東,稍前,南向。太常設左輔位於御耕之東,稍南,西向;設司農位二,一在左輔之後,一在其南,並西向。籍田令三,皆位司農卿南,少退,北上。奉青箱官位於後。諸執耒耜者位公卿耕者後,侍耕者前,西向。三公、三少、宰臣、親王等每員三人,執政二人,從耕;群官一名助耕,並服絳衣、介幘。三公以次群官耒耜各一具,每一具正副牛二,隨牛二人。庶人耕位在從耕官位之南,西向。庶人百人,並青衣,耕牛二百,
 每兩牛用隨牛一人,耒耜百具,畚五十具,鍤二十五具,以木為刃。耆老百人,常服陪位於庶人位南,西向。司農少卿位二於庶人位前,太社令位司農少卿之西,少退,俱北向。畿內諸令位庶人之東,西向。尚輦局設玉輅於仗內。前期三日,司農以青箱奉九穀穜稑之種進內。前二日,皇后率六宮獻於皇帝,受於內殿。前一日,降出付司農。



 其日質明,左輔奉耒耜載於玉輅訖,耕籍使朝服乘車,用本品鹵簿,以儀仗二千人衛耒耜先詣壇所。尚
 輦奉御設平輦於祥曦殿,皇帝靴袍出自內東門,從駕臣僚禁衛並起居如常儀。將至耕所,文武侍耕、從耕以下及耆老、庶人俱詣籍田西門外立班,再拜奉迎訖,各就次。從耕、陪耕等官服朝服以俟耕。車駕至思文殿,進膳訖,左輔以御耒耜授籍田令,橫執之,詣耕籍所,置於席,遂守之。凡執耒耜者橫執之,受則先其耒、後其耜。諸縣令率終畝庶人、陪耕耆老先就位,司農卿、籍田令、太社令、奉青箱官、諸執耒耜者以次就位。御史臺引殿中
 侍御史一員先入就位,次禮直官、宣贊舍人等分引侍耕、從耕群官各就位。尚輦奉御進輦思文殿。左輔奏請中嚴。少頃,奏外辦。皇帝通天冠、絳紗袍,乘輦出。將至御耕位,尚舍先設黃道,太常請降輦就位。既降輦,太常卿前導至褥位南向立,奏請行禮。禮直官請籍田令進詣御耒席南向,引司農卿詣籍田令東西向,籍田令俯伏跪,執事者以絳受之,籍田令解絳出耒,執耒興,東向立,以授司農卿,司農卿西向立,以授左輔,左輔詣御耕位
 前少東,北向。太常卿奏請受耒耜,左輔執以進,執耒者助執之。皇帝受以三推,左輔前受耒耜,授司農卿,以授籍田令,各復位。籍田令跪而納於絳,執耒興,以授執事者,退復位。



 皇帝初耕,諸執耒耜者以耒耜各授從耕者,禮直官引太常卿詣御位前,北向,奉請皇帝升壇觀耕,復位立。前導官導皇帝升壇,即御坐南向。禮直官、太常博士、太常卿近東,西向北上立。禮直官引三公、三少、宰臣、親王各五推,餘從耕官各九推,訖,執耒耜者前受耒
 耜。禮直官引司農少卿帥庶人以次耕於千畝,候耕少頃,禮直官引左輔詣御坐前跪奏禮畢。降壇,乘輦還思文殿,左輔奏解嚴,侍耕、從耕官皆退。次籍田令以青箱授司農卿,詣耕所,出穜稑播之。次司農少卿帥太社令檢校終畝。次司農卿詣御前北向俯伏跪奏省功畢,退。所司放仗以俟,皇帝常服還內,侍衛如常儀。紹興七年,始舉享先農之禮,以立春後亥日行一獻禮。十六年,皇帝親耕籍田,並如舊制。



 先蠶之禮久廢,真宗從王欽若請,詔有司檢討故事以聞。按《開寶通禮》:「季春吉巳,享先蠶於公桑。前享五日,諸與享官散齋三日,致齋二日。享日未明五刻,設先蠶氏神坐於壇上北方,南向。尚宮初獻,尚儀亞獻,尚食終獻。女相引三獻之禮,女祝讀文,飲福、受胙如常儀。」又按《唐會要》:「皇帝遣有司享先蠶如先農可也。」乃詔:「自今依先農例,遣官攝事。」禮院又言:「《周禮》:『蠶於北郊。』以純陰也。漢蠶於東郊,以春桑生也。請約附故事,築壇東郊,從桑生
 之義。壇高五尺,方二丈,四陛,陛各五尺。一壝,二十五步。祀禮如中祠。」



 慶歷用羊、豕各一,攝事獻官太尉、太常、光祿卿,不用樂。元豐,詳定所言:「季春吉巳,享先蠶氏。唐《月令注》:『以先蠶為天駟。』按先蠶之義,當是始蠶之人,與先農、先牧、先炊一也。《開元享禮》:為瘞坎於壇之壬地。而《郊祀錄》載《先蠶祀文》,有『肇興蠶織』之語,《禮儀羅》又以享先蠶無燔柴之儀,則先蠶非天駟星明矣。今請就北郊為壇,不設燎壇,但瘞埋以祭,餘如故事。」



 政和,禮局言:「《禮》:天
 子必有公桑蠶室,以興蠶事。歲既畢,則奉繭而繅,遂朱綠之,玄黃之,以為郊廟之祭服。今既開籍田以供粢盛,而未有公桑蠶室以供祭服,尚為闕禮。請仿古制,於先蠶壇側築蠶室,度地為宮,四面為墻,高仞有三尺,上被棘,中起蠶室二十七,別構殿一區為親蠶之所。仿漢制,置繭館,立織室於宮中,養蠶於薄以上。度所用之數,為桑林。築採桑壇於先蠶壇南,相距二十步,方三丈,高五尺,四陛。凡七事。置蠶官令、丞,以供郊廟之祭服。又《周官
 內宰》:『詔後帥內外命婦蠶於北郊。』鄭氏謂:『婦人以純陰為尊。』則蠶為陰事可知。《開元禮》享先蠶,幣以黑,蓋以陰祀之禮祀之也。請用黑幣,以合至陰之義。」詔從其議,命親蠶殿以無斁為名。又詔:「親蠶所供,不獨袞服,凡施於祭祀者皆用之。」



 宣和元年三月,皇后親蠶,即延福宮行禮。其儀:季春之月,太史擇日,皇后親蠶,命有司享先蠶氏於本壇。前期,殿中監帥尚舍設坐殿上,南向;前楹施簾,設東西閣殿後之左右。又設內命婦妃嬪以下次於
 殿之左右,外命婦以下次於殿門內外之左右,隨地之宜,量施帷幄。於採桑壇外,四面開門,設皇后幄次於壇壝東門之內道北,南向。



 其日,有司設褥位壇上,少東,東向。設內命婦位壇下東北,南向;設外命婦位壇下東南,北向,俱異位重行西上。內外命婦,一品各二人;二品、三品各一人。又設從採桑內命婦等位於外命婦之東,南向;用內命婦一員充詣蠶室,授蠶母桑以食蠶。設從採桑外命婦等位於外命婦東,北向,俱異位重行西上。設
 執皇后鉤箱者位於內命婦之西,少南,西上。尚功執鉤,司制執箱;內外命婦鉤箱者,各位於後,典制執鉤,女史執箱。又於壇上設執皇后鉤箱位於皇后採桑位之北,稍東,南向,西上。



 前出宮一日,兵部率其屬陳小駕鹵簿於宣德門外,太僕陳厭翟車東偏門內,南向。其日未明,外命婦應採桑及從採桑者,先詣親蠶所幕次,以俟起居,各令其女侍者進鉤箱,載至親蠶所,授內謁者監以授執鉤箱者。前一刻,內命婦各服其服,內侍引內命婦
 妃嬪以下,俱詣殿庭起居訖,內侍奏請中嚴;少頃,又奏外辦。皇后首飾、鞠衣,乘龍飾肩輿如常儀,障以行帷,出內東門至左升龍門。內侍跪奏:「具官臣某言,請降肩輿升厭翟車。」訖,俯伏,興,少退。御者執綏升厭翟車,內侍詣車前奏,請車進發,出宣德東偏門,執事者進鉤箱,載之車。至親蠶所殿門,降車,乘肩輿入殿後西閣門,侍衛如常儀。內侍先引內外命婦及從採桑者俱就壇下位,諸執鉤箱者各就位。內侍奏請中嚴;少頃,奏外辦。皇后首
 飾、鞠衣,乘肩輿,內侍前導至壇東門,華蓋、仗衛止於門外,近侍者從之入。內侍奏請降肩輿,至幄次內,下簾。又內侍至幄次,請行禮,導皇后詣壇,升自南陛,東向立。執鉤箱者自北陛以次升壇就位次,內侍引尚功詣採桑位前西向,奉鉤以進,皇后受鉤採桑,司製奉箱進以受桑,皇后採桑三條,止,以鉤授尚功,尚功受鉤,司製奉箱俱退,復位。



 初,皇后採桑,典制各以鉤授內外命婦,皇后採桑訖,內外命婦以次採桑,女使執箱者受之,內外命
 婦一品各採五條,二品、三品各採九條,止,典制受鉤,與執箱者退,復位。內侍各引內外命婦退,復位。內侍詣皇后前,奏禮畢,退,復位。內侍引皇后降自南陛,歸幄次。少頃,奏請乘肩輿如初。內侍前導,皇后歸殿後閣,內侍奏解嚴。初,皇后降壇,內侍引內命婦詣蠶室,尚功帥執鉤箱者以次從至蠶室,尚功以桑授蠶母,蠶母受桑縷切之,授內命婦食蠶,灑一簿訖,內侍引內外命婦各還次,皇后還宮。



 宣和復位親蠶禮,外命婦、宰執並一品夫人升
 壇侍立,餘品列於壇下。六年閏二月,皇后復行親蠶之禮焉。紹興七年,始以季春吉巳日享先蠶,視風師之儀。乾道中,升為中祀。



 告禮。古者,天子將出,類於上帝,命吏告社稷及圻內山川。又天子有事,必告宗廟,歷代因之。宋制:凡行幸及封泰山、祠后土、謁太清宮,皆親告太廟。三歲郊祀,每歲祈谷上帝,祀感生帝,雩祀,祭方丘,明堂、神州地祇、圜丘,並遣官告祖宗配侑之意。他大事:即位、改元、更御名、上尊
 號、尊太后、立皇后太子、皇子生、籍田、親征、納降、獻俘、朝陵、肆赦、河平及大喪、上謚、山陵、園陵、祔廟、奉遷神主,皆遣官奏告天地、宗廟、社稷、諸陵、岳瀆、山川、宮觀、在京十里內神祠。其儀用犧尊、籩、豆各一,實以酒、脯、醢。宮寺以素饌、時果代,用祝幣,行一獻禮。若車駕出京,則有□爰祭,用羝羊一。所過州郡橋梁、山川、帝王名臣陵廟去路十里內者,各令本州以香、酒、脯祭告。建降元年,太祖平澤、潞,仍祭襖廟、泰山、城隍。徵揚州、河東,並用此禮。四年,修
 葺太廟,遣官奏告四室及祭本廟土神。凡修葺同。如遷神主,修畢奉安。是歲十一月,詔以郊祀前一日,遣官奏告東嶽、城隍、浚溝廟、五龍廟及子張、子夏廟,他如儀。



 太平興國五年十一月,車駕北征。前一日,遣官祭告天地於圜丘,用特牲;太廟、社稷用太牢;望祭岳瀆、名山、大川於四郊,磔風於風伯壇,祀雨師於本壇,禱馬於馬祖壇,祭蚩尤、榪牙於北郊,並用少牢;祭北方天王於北郊迎氣壇,用香、柳枝、燈油、乳粥、酥蜜餅、果。仍遣內侍一人監
 祭。咸平中北征,禮同。八年,滑州合河口畢工,遣官告天地、岳瀆,後天禧中,又遣謝玉清昭應景靈上清太一宮、會靈祥源觀及諸陵。雍熙四年,詔以親耕籍田,遣官奏告外,又祭九龍、黃溝、扁鵲、吳起、信陵、張耳、單雄信七廟,後又增祭德安公、嶽臺諸神廟,為定式。



 淳化三年十二月將郊,常奏告外,又告太社、太稷及文宣、武成等廟。景德二年,契丹遣使修好,遣官奏告諸陵。四年二月次西京,遣告汾陰、中嶽、太行、河、洛、啟母少姨廟,東還,奏告如
 常儀。大中祥符元年,天書降,及封禪,告天地、宗廟、社稷及諸祠廟、宮觀;其在外者,乘傳以往。澶、鄆、兗州、高陽帝嚳、帝堯,亦皆告之。四年,加五岳帝號,告天地、宗廟、社稷。五年,聖祖降,告如封禪禮。六年,宮庭嘉禾生,遣官告廟及玉皇、聖祖天尊大帝。天禧元年,奉迎太祖聖容赴西京,遣官奏告如常儀,及經由五里內並西京城內外神祠。天聖七年,玉清昭應宮火,遣告諸陵。十年,大內火,遣告天地、廟社。明道二年,詔以蟲螣為沴,減尊號四字,告
 天地、宗廟。熙寧七年,南郊雅飾,奏告太廟、後廟。八年,以韓琦配享,告英宗廟。元符三年四月朔,太陽虧,遣官告太社。大觀元年十二月,以恭受八寶,告天地、宗廟、社稷。政和二年冬至,受元圭,禮同。三年二月,以太平告成,冊告諸陵。四年二月,皇長子冠,告天地、宗廟、社稷、諸陵。五年,建明堂,告如上禮,及宮觀、岳瀆。



 高宗建炎已後,事有關於國體者皆告。紹興九年,金人遣使議和割地;十一年,詔撰講和誓文;二十四年,進《徽宗御集》;二十六年,進《
 太后回鑾事實》;二十七年,進《玉牒仙源類譜》;明年,進《神宗寶訓》,進祖宗《仙源積慶圖》,進《徽宗實錄》,進《祐陵迎奉錄》;三十一年,金人叛盟興師;開禧二年,吳曦伏誅;嘉定七年,進《高宗中興經武要略》;十三年,進《宗藩慶系錄》,刊正《憲聖慈烈皇后聖德事跡》,進《光宗玉牒》;十四年,進《孝宗寶訓》;十五年,得玉璽;明年,上玉璽;端平元年,獲完顏守緒函骨;淳祐五年,進《光宗寧宗兩朝寶訓》、《經武要略》、《玉牒》、《日歷》、《會要》;寶祐元年,皇女延昌公主進封瑞國公
 主,又封升國;五年,進《中興四朝史》;景定二年,進《孝宗》、《光宗實錄》,皇女周國公主下降;咸淳四年,安奉《寧宗理宗實錄》、《御集》、《會要》,《經武要略》:皆告天地、宗廟、社稷、攢陵。其餘即位、改元、受禪、冊寶,皇子生、冠及巡幸、納降、獻俘之屬,並仍舊制。



 祈報。《周官》:「太祝掌六祝之辭,以事鬼神,示其福祥。」於是歷代皆有襘禜之事。宋因之,有祈、有報。祈,用酒、脯、醢,郊廟、社稷,或用少牢;其報如常祀。或親禱諸寺觀,或再幸,
 或徹樂、減膳、進蔬饌,或分遣官告天地、太廟、社稷、岳鎮、海瀆,或望祭於南北郊,或五龍堂、城隍廟、九龍堂、浚溝廟,諸祠如子張、子夏、信陵君、段干木、扁鵲、張儀、吳起、單雄信等廟亦祀之。或啟建道場於諸寺觀,或遣內臣分詣州郡,如河中之後土廟、太寧宮,毫之太清、明道宮,兗之會真景靈宮、太極觀,鳳翔之太平宮,舒州之靈仙觀,江州之太平觀,泗州之延祥觀,皆函香奉祝,驛往禱之。凡旱、蝗、水潦、無雪,皆禜禱焉。



 咸平二年旱,詔有司祠雷
 師、雨師。內出李邕《祈雨法》:以甲、乙日擇東方地作壇,取土造青龍,長吏齋三日,詣龍所,汲流水,設香案、茗果、餈餌,率群吏、鄉老日再至祝酹,不得用音樂、巫覡。雨足,送龍水中。餘四方皆如之,飾以方色。大凡日乾及建壇取土之里數,器之大小及龍之修廣,皆以五行成數焉。



 景德三年五月旱,又以《畫龍祈雨法》付有司刊行。其法:擇潭洞或湫濼林木深邃之所,以庚、辛、壬、癸日,刺史、守令帥耆老齋潔,先以酒脯告社令訖,築方壇三
 級,高二尺,闊一丈三尺,壇外二十步,界以白繩。壇上植竹枝,張畫龍。其圖以縑素,上畫黑魚左顧,環以天黿十星;中為白龍,吐雲黑色;下畫水波,有龜左顧,吐黑氣如線,和金銀朱丹飾龍形。又設皂幡,刎鵝頸血置盤中,柳枝灑水龍上,俟雨足三日,祭以一豭,取畫龍投水中。大中祥符二年旱,遣司天少監史序祀玄冥五星於北郊,除地為壇,望告。已而雨足,遣官報謝及社稷。



 初,學士院不設配位,及是問禮官,言:「祭必有配,報如常祀。當設配
 坐。」又諸神祠、天齊、五龍用牛祠,祆祠、城隍用羊一,八籩,八豆。舊制,不祈四海。帝曰:「百穀之長,潤澤及物,安可闕禮?」特命祭之。



 天禧四年四月,大風飛沙折木,晝晦數刻,命中使詣宮觀,建醮禳之。天聖三年九月,帝宣諭:「近內臣南中勾當回,言諸處名山洞府,投送金龍玉簡,開啟道場,頗有煩擾。速令分祈,投龍處不得開建道場。」康定二年三月,以黃河水勢甚淺,致分流入汴未能通濟,遣祭河瀆及靈津廟。又澶州曹村埽方開減水直河,而水
 自流通,遣使祭謝,後修塞,禮同。治平四年十二月,詔以來歲正旦日食,命翰林學士承旨王珪祭社。



 熙寧元年正月,帝親幸寺觀祈雨,仍令在京差官分禱,各就本司先致齋三日,然後行事。諸路擇端誠修潔之士,分禱海鎮、岳瀆、名山、大川,潔齋行事,毋得出謁宴飲、賈販及諸煩擾,令監司察訪以聞。諸路神祠、靈跡、寺觀,雖不系祀典,祈求有應者,並委州縣差官潔齋致禱。已而雨足,復幸西太一宮報謝。九年十二月,以安南行營將士疾病
 者眾,遣同知太常禮儀院王存詣南嶽虔潔致禱,仍建祈福道場一月。又以西江運糧獲應,命本州長吏往祭龍祠。十年四月,以夏旱,內出《蜥蜴祈雨法》:捕蜥蜴數十納甕中,漬之以雜木葉,擇童男十三歲下、十歲上者二十八人,分兩番,衣青衣,以青飾面及手足,人持柳枝沾水散灑,晝夜環繞,誦咒曰:「蜥蜴蜥蜴,興雲吐霧,雨令滂沱,令汝歸去!」雨足。



 元豐元年十月,太皇太后違豫,命輔臣以下分禱天地、宗廟、社稷,及都內諸神祠。又作祈福
 道場於寺觀及五岳、四瀆凡靈跡所在。八年,帝疾,分禱亦如之。又以京城火災,建醮於集禧觀,且為民祈福。元祐元年十二月,以華州鄭縣山摧,命太常博士顏復往祭西嶽。七年,詔:「太皇太后本命歲,正月一日,京師及天下州軍,各齋僧尼、道士、女冠一日,在京宮觀、寺院,開建道場七晝夜,內外獄囚並設食三日。」八年,太皇太后違豫,祈禱如元豐,仍致禱諸陵。又令南京等處長吏,詣祖宗神御所在建置道場。紹興二年三月苦雨,命往天竺
 山祈晴,即日雨止。四年,知樞密院張浚言:「四川自七月以來霖雨、地震,乞制祝文,名山大川祈禱。」上曰:「霖雨、地震之災,豈非兵久在蜀,調發供饋,民怨所致。當修德以應之,又可禱乎?」



 七年正月一日,詔:「朕痛兩宮北狩,道君皇帝春秋益高,念無以見勤誠之意,可遣官往建康府元符萬歲宮修建祈福道場三晝夜,務令嚴潔,庶稱朕心。」又謂輔臣曰:「宣和皇后春秋浸高,朕朝夕思之,不遑安處。已遣人於三茅山設黃菉醮,仰祝聖壽。」是歲七月,
 張浚等言:「雨澤稍愆,乞禱。」上曰:「朕患不知四方水旱之實,宮中種稻兩區,其一地下,其一地高,高者其苗有槁意矣,須精加祈禱,以救旱□。」八年,宰臣奏積雨傷蠶,上曰:「朕宮中自蠶一薄,欲知農桑之候,久雨葉濕,豈不有損?」乃命往天竺祈晴。



 三十二年,太常少卿王普言:「逆亮誅夷,虜騎遁去,兩淮無警,舊疆浸歸。茲者回鑾臨安,當行報謝之禮。」從之。嘉定八年八月,蝗,禱於霍山。九年六月蝗,禱群祀。淳祐七年六月大旱,命待從禱於天竺觀
 音及霍山祠。



\end{pinyinscope}