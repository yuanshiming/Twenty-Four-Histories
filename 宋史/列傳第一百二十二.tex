\article{列傳第一百二十二}

\begin{pinyinscope}

 李
 光子孟傳許翰許景衡張愨張所陳禾蔣猷



 李光,字泰發,越州上虞人。童稚不戲弄。父高稱曰:「吾兒雲間鶴,其興吾門乎!」親喪,哀毀如成人,有致賻者,悉辭
 之。及葬,禮皆中節。服除,游太學,登崇寧五年進士第。調開化令,有政聲,召赴都堂審察,時宰不悅,處以監當,改秩,知平江府常熟縣。朱勉父沖倚勢暴橫,光械治其家僮。沖怒,風部使者移令吳江,光不為屈。改京東西學事司管勾文字。



 劉安世居南京,光以師禮見之。安世告以所聞於溫公者曰:「學當自無妄中入。」光欣然領會。除太常博士,遷司封。首論士大夫諛佞成風,至妄引荀卿「有聽從,無諫諍」之說,以杜塞言路;又言怨嗟之氣,結為妖
 沴。王黼惡之,令部注桂州陽朔縣。安世聞光以論事貶,貽書偉之。李綱亦以論水災去國,居義興,伺光於水驛,自出呼曰:「非越州李司封船乎?」留數日,定交而別。除司勛員外郎,遷符寶郎。



 郭藥師叛,光知徽宗有內禪意,因納符,謂知樞密院蔡攸曰:「公家所為,皆咈眾心。今日之事,非皇太子則國家俱危。」攸矍然,不敢為異。欽宗受禪,擢右司諫。上皇東幸,憸人間兩宮,光請集議奉迎典禮。又奏:「東南財用,盡於朱勉,西北財用,困於李彥,天下根
 本之財,竭於蔡京、王黼。名為應奉,實入私室,公家無半歲之儲,百姓無旬日之積。乞依舊制,三省、樞密院通知兵民財計,與戶部量一歲之出入,以制國用,選吏考核,使利源歸一。」



 金人圍太原,援兵無功。光言:「三鎮之地,祖宗百戰得之,一旦舉以與敵,何以為國?望詔大臣別議攻守之策,仍間道遣使檄河東、北兩路,盡起強壯策應,首尾掩擊。」遷侍御史。



 時言者猶主王安石之學,詔榜廟堂。光又言:「祖宗規摹宏遠,安石欲盡廢法度,則謂人
 主制法而不當制於法;欲盡逐元老,則謂人主當化俗而不當化於俗。蔡京兄弟祖述其說,五十年間,毒流四海。今又風示中外,鼓惑民聽,豈朝廷之福?」



 蔡攸欲以扈衛上皇行宮因緣入都,光奏:「攸若果入,則百姓必致生變,萬一驚犯屬車之塵,臣坐不預言之罪。望早黜責。」時已葺擷景園為寧德宮,而太上皇后乃欲入居禁中。光奏:「禁中者,天子之宮。正使陛下欲便溫凊,奉迎入內,亦當躬稟上皇,下有司討論典禮。」乃下光章,使兩宮臣奏
 知,於是太上皇后居寧德宮。



 金人逼京城,士大夫委職而去者五十二人,罪同罰異,士論紛然,光請付理寺公行之。太原圍急,奏:「乞就委折彥質盡起晉、絳、慈、隰、澤、潞、威勝、汾八州民兵及本路諸縣弓手,俾守令各自部轄。其土豪、士人願為首領者,假以初官、應副器甲,協力赴援。女真劫質親王,以三鎮為辭,勢必深入,請大修京城守禦之備,以伐敵人之謀。」



 又言:「朱勉托應奉脅制州縣,田園第宅,富擬王室。乞擇清強官置司,追攝勉父子及奉
 承監司、守令,如胡直孺、盧宗原、陸寘、王促閔、趙霖、宋晦等,根勘驅磨,計資沒入,其強奪編戶產業者還之。」



 李會、李擢復以諫官召。光奏:「蔡京復用,時會、擢迭為臺官,禁不發一語;金人圍城,與白時中、李邦彥專主避敵割地之謀。時中、邦彥坐是落職,而會、擢反被召用,復預諫諍之列。乞寢成命。」不報。光丐外,亦不報。



 彗出寅、艮間,耿南仲輩皆謂應在外夷,不足憂。光奏:「孔子作《春秋》,不書祥瑞者,蓋欲使人君恐懼修省,未聞以災異歸之外夷也。」
 疏奏,監汀州酒稅。



 高宗即位,擢秘書少監,除知江州;未幾,擢侍御史,皆以道梗不赴。建炎三年,車駕自臨安移蹕建康,除知宣州。時範瓊將過軍,光先入視事,瓊至則開門延勞,留三日而去,無敢嘩者。光以宣密邇行都,乃繕城池,聚兵糧,籍六邑之民,保伍相比,謂之義社。擇其健武者,統以土豪,得保甲萬餘,號「精揀軍」。又柵險要二十三所謹戍之,厘城止為十地分,分巡內外,晝則自便,夜則守城,有警則戰。苗租歲輸邑者,悉命輸郡。初歡言
 不便,及守城之日,贍軍養民,迄賴以濟。事聞,授管內安撫,許便宜從事,進直龍圖閣。



 杜充以建康降,金人奪馬家渡。御營統制王□燮、王□民素不相能,至是,擁潰兵砦城外索鬥。光親至營,諭以先國家後私讎之義,皆感悟解去。時奔將、散卒至者,光悉厚貲給遺。有水軍叛於繁昌,逼宣境,即遣兵援擊,出賊不意,遂宵遁。進右文殿修撰。光奏:「金人雖深入江、浙,然違天時地利,臣已移文劉光世領大兵赴州,並力攻討。乞速委宣撫使周望,約日水
 陸並進。」



 潰將邵青自真州擁舟數百艘,剽當塗、蕪湖兩邑間,光招諭之,遺米二千斛。青喜,謂使者曰:「我官軍也,所過皆以盜賊見遇,獨李公不疑我。」於是秋毫無犯。他日,舟過繁昌,或紿之曰:「宣境也。」乃掠北岸而去。



 劇盜戚方破寧國縣,抵城下,分兵四擊。光募勇敢劫之,賊驚擾,自相屠蹂。朝廷遣統制官巨師古、劉晏兼程來援。賊急攻朝京門,纜竹木為浮梁以濟。須臾,軍傅城,列炮具,立石對樓。光命編竹若簾揭之,炮至即反墜,不能傷。取桱
 木為撞竿,倚女墻以御對樓,賊引卻。劉晏率赤心隊直搗其砦,賊陽退,晏追之,伏發遇害。師古以中軍大破賊,賊遁去。初,戚方圍宣,與其副並馬巡城,指畫攻具。光以書傅矢射其副馬前,言:「戚方窮寇,天誅必加,汝為將家子,何至附賊。」二人相疑,攻稍緩,始得為備,而援師至矣。嘗置匕首枕匣中,與家人約曰:「城不可必保,若使人取匕首,我必死。汝輩宜自殺,無落賊手。」除徽猷閣待制、知臨安府。



 紹興元年正月,除知洪州,固辭,提舉臨安府洞
 霄宮。除知婺州,甫至郡,擢吏部侍郎。光奏疏極論朋黨之害:「議論之臣,各懷顧避,莫肯以持危扶顛為己任。駐蹕會稽,首尾三載。自去秋迄今,敵人無復南渡之意,淮甸咫尺,了不經營,長江千里,不為限制,惴惴焉日為乘桴浮海之計。晉元帝區區草創,猶能立宗社,修宮闕,保江、浙。劉琨、祖逖與逆胡拒戰於並、冀、兗、豫、司、雍諸州,未嘗陷沒也。石季龍重兵已至歷陽,命王導都督中外諸軍以御之,未聞專主避狄如今日也。陛下駐蹕會稽,江、
 浙為根本之地,使進足以戰、退足以守者,莫如建康。建康至姑熟一百八十里,其隘可守者有六:曰江寧鎮,曰□岡砂夾,曰採石,曰大信,其上則有蕪湖、繁昌,皆與淮南對境。其餘皆蘆□之場,或□奇岸水勢湍悍,難施舟楫。莫若預於諸隘屯兵積粟,命將士各管地分,調發旁近鄉兵,協力守御。乞明詔大臣,參酌施行。」



 時有詔,金人深入,諸郡守臣相度,或守或避,令得自便。光言:「守臣任人民、社稷之重,固當存亡以之。若預開遷避之門,是誘之遁
 也,願追寢前詔。」上欲移蹕臨安,被旨節制臨安府見屯諸軍,兼戶部侍郎、督營繕事。光經營撙節,不擾而辦。奏蠲減二浙積負及九邑科配,以示施德自近之意。戚方以管軍屬節制,甚懼,拜庭下。光握手起之,曰:「公昔為盜,某為守,分當相直;今俱為臣子,當共勉力忠義,勿以前事為疑。」方謝且泣。兼侍讀,因奏:「金人內寇,百姓失業為盜賊,本非獲已,尚可誠感。自李成北走,群盜離心,儻因斯時顯用一二酋豪,以風厲其黨,必更相效慕,以次就
 降。」擢吏部尚書。



 大將韓世清本苗傅餘黨,久屯宣城,擅據倉庫,調發不行。光請先事除之,乃授光淮西招撫使。光假道至郡,世清入謁,縛送闕下伏誅。初,光於上前面稟成算,宰相以不預聞,怒之。未至,道除端明殿學士、江東安撫大使、知建康府、壽春滁濠廬和無為宣撫使。時太平州卒陸德囚守臣據城叛,光多設方略,盡擒其黨。



 秦檜既罷,呂頤浩、朱勝非並相,光議論素與不合。言者指光為檜黨,落職奉祠。尋復寶文閣待制、知湖州,除顯
 謨閣直學士,移守平江,除禮部尚書。光言:「自古創業中興,必有所因而起。漢高因關中,光武因河內,駐蹕東南,兩浙非根本所因之地乎?自冬及春,雨雪不已,百姓失業,乞選臺諫察實以聞。兼比歲福建、湖南盜作,範汝為、楊麼相鋌而起,朝廷發大兵誅討,殺戮過當。今諸路旱荒,流丐滿路,盜賊出入。宜選良吏招懷撫納,責諸路監司按貪贓,恤流殍。」



 議臣欲推行四川交子法於江、浙,光言:「有錢則交子可行。今已謂樁辦若干錢,行若干交子,
 此議者欲朝廷欺陛下,使陛下異時不免欺百姓也。若已樁辦見錢,則目今所行錢關子,已是通快,何至紛紛?其工部鑄到交子務銅印,臣未敢給降。」除端明殿學士,守臺州,俄改溫州。



 劉光世、張俊連以捷聞。光言:「觀金人布置,必有主謀。今已據東南形勢,敵人萬里遠來,利於速戰,宜戒諸將持重以老之。不過數月,彼食盡,則勝算在我矣。」除江西安撫、知洪州兼制置大使,擢吏部尚書,逾月,除參知政事。



 時秦檜初定和議,將揭榜,欲籍光名
 鎮壓。上意不欲用光,檜言:「光有人望,若同押榜,浮議自息。」遂用之。同郡楊煒上光書,責以附時相取尊官,墮黠虜奸計,隳平時大節。光本意謂但可因和而為自治之計。既而檜議徹淮南守備,奪諸將兵權,光極言戎狄狼子野心,和不可恃,備不可徹。檜惡之。檜以親黨鄭億年為資政殿學士,光於榻前面折之,又與檜語難上前,因曰:「觀檜之意,是欲壅蔽陛下耳目,盜弄國權,懷奸誤國,不可不察。」檜大怒,明日,光丐去。高宗曰:「卿昨面叱秦檜,
 舉措如古人。朕退而嘆息,方寄卿以腹心,何乃引去?」光曰:「臣與宰相爭論,不可留。」章九上,乃除資政殿學士、知紹興府,改提舉臨安府洞霄宮。



 十一年冬,中丞萬俟離論光陰懷怨望,責授建寧軍節度副使,藤州安置。越四年,移瓊州。居瓊州八年,仲子孟堅坐陸升之誣以私撰國史,獄成;呂願中又告光與胡銓詩賦倡和,譏訕朝政,移昌化軍。論文考史,怡然自適。年逾八十,筆力精健。又三年,始以郊恩,復左朝奉大夫,任便居住。至江州而卒。
 孝宗即位,復資政殿學士,賜謚莊簡。



 孟傳字文授,光幼子也。光南遷之日,才六歲。以光遺表恩,累官至太府丞。韓侂冑願見之,孟傳曰:「行年六十,去計已決,不敢聞也。」由是出知江州。以朝請大夫、直寶謨閣致仕。卒,年八十。有《盤溪詩》二十卷,《文稿》三十卷,《宏辭類稿》十卷,《左氏說》十卷,《讀史》十卷,《雜志》十卷。博學多聞,持身甚嚴,時推能世其家。



 許翰,字崧老,拱州襄邑人。中元祐三年進士第。宣和七
 年,召為給事中。為書抵時相,謂百姓困弊,起為盜賊,天下有危亡之憂。願罷雲中之師,修邊保境,與民休息。高麗入貢,調民開運河,民間騷然。中書舍人孫傅論高麗於國無功,不宜興大役,傅坐罷。翰謂傅不當黜,時相怒,落職,提舉江州太平觀。



 靖康初,復以給事中召。時金人攻京師甫退,翰造闕,即日賜對,除翰林學士,尋改御史中丞。上疏言邊事,因陳決勝之策。陳邦昌為太宰,翰上疏力爭之。種師道罷為中太一宮使,翰言:「師道名將,沉
 毅有謀,山西士卒,人人信服,不可使解兵柄。」欽宗謂其老難用,翰曰:「秦始皇老王翦而用李信,兵辱於楚;漢宣帝老趙充國,而卒能成金城之功。自呂望以來,用老將收功者,難一二數。以古揆今,師道雖老,可用也。」且謂:「金人此行,存亡所系,令一大創,使失利去,則中原可保,四夷可服。不然,將來再舉,必有不救之憂。宜起師道邀擊之。」上不能用。擢中大夫、同知樞密院,論益不合,以病去,除延康殿學士、知亳州。坐言者落職,提舉南京鴻慶宮。



 高宗即位,用李綱薦,召復延康殿學士。既至,拜尚書右丞兼權門下侍郎。時建炎大變之後,河北山東大盜李成、孔彥舟等,聚眾各數十萬,皆以勤王為名,願得張所為帥。所為御史,嘗論黃潛善奸邪不可用,由此得罪。李綱為相,乃以所為河北等路招撫使,率成等眾渡河,號召諸路,為興復計。潛善力沮之。宗澤論車駕不宜南幸,宜還京師,且詆潛善等。潛善等請罷澤,翰極論以為不可。李綱罷,翰言:「綱忠義英發,舍之無以佐中興,今罷綱,
 臣留無益。」力求去,高宗未許。時潛善奏誅陳東,翰謂所親曰:「吾與東,皆爭李綱者。東戮東市,吾在廟堂可乎?」求去益力,章八上,以資政殿大學士提舉洞霄宮。復以言者落職。



 紹興元年,召復端明殿學士、提舉萬壽觀,辭不至。二月,復資政殿學士。三年五月,卒,贈光祿大夫。



 翰通經術,正直不撓,歷事三朝,致位政府,徒以黼、攸、潛善輩熏蕕異味,橫遭口語,志卒不展。綱雖力引之,不旋踵去,翰亦斥逐而死。所著書有《論語解》、《春秋傳》。



 許景衡,字少伊,溫州瑞安人。登元祐九年進士第。宣和六年,召為監察御史,遷殿中侍御史。是時,王黼、蔡攸用事,景衡言:「尚書省比闕長官,而同知樞密院亦久闕。雖三公通治三省,然文昌政事之本,樞密總兵之地,各有攸屬,安可久虛其位?願博採公議,遴選忠賢,以補政府之闕。」遂大忤黼意。朝廷用童貫為河東、北宣撫使,將北伐,景衡論其貪繆不可用者數十事,不報。



 睦寇平,江、浙郡縣殘毀,而茶鹽比較之法如故。景衡奏:「茶鹽之法,當
 以食之眾寡為歲額之高下。今收復之後,戶版半耗,民力蕭然,而茶鹽比較不減於昔。民欲無困得乎?」奏上,詔兩浙、江東路權免茶鹽比較,賊平日仍舊。



 朝廷既興燕雲之師,調度不繼,誅求益急。景衡奏:「財力匱乏在節用,民力困弊在恤民。今不急之務。若營繕諸役,花石綱運,其名不一。吏員猥多,軍額冗濫。又無名功賞,非常賜予,皆夤緣僥幸,干請無厭,宜節以祖宗之制而省去之。」且極論和買、和糴、鹽法之害,不報。會知洋州吳巖夫以私
 書抵執政子,道景衡之賢。因從子婿符寶郎周離亨以達,離亨繆以其書誤致王黼,黼用是中景衡,逐之。



 欽宗即位,以左正言召,旋改太常少卿兼太子諭德,遷中書舍人。侍御史李光、正言程瑀以鯁亮忤執政斥,景衡為辨白,坐落職予祠。



 高宗即位,以給事中召,既至,除御史中丞。宗澤為東京留守,言者附黃潛善等,多攻其短,欲逐去之。景衡奏曰:「臣自浙渡淮,以至行在。聞澤之為尹,威名政事,卓然過人,雖不識其人,竊用嘆慕。臣以為去
 冬京城內,有赤心為國如澤等數輩,其禍變未至如是之酷。今若較其小短,不顧盡忠徇國之節,則不恕已甚。且開封宗廟社稷所在,茍欲罷澤,別遣留守,不識搢紳中威名政事有加於澤者乎?」疏入,上大悟,封以示澤,澤乃安。



 杭州叛卒陳通作亂,權浙西提刑趙叔近招降之,請授以官。景衡曰:「官吏無罪而受誅。叛卒有罪而蒙賞,賞罰倒置,莫此為甚。」卒奏罷之。除尚書右丞。有大政事,必請間極論。潛善、伯彥以景衡異己,共排沮之。或言正、
 二月之交,乃太一正遷之日,宜於禁中設壇望拜。高宗以問景衡,曰:「修德愛民,天自降福,何迎拜太一之有?」



 初,李綱議建都,以關中為上,南陽次之,建康為下。綱既相,遂主南陽之議。景衡為中丞,奏:「南陽無險阻,且密邇盜賊,漕運不繼,不若建康天險可據,請定計巡幸。」潛善等傾綱使去,南陽之議遂格。至是,諜報金人攻河陽、汜水,景衡又奏請南幸建康。已而有詔還京,罷景衡為資政殿大學士、提舉杭州洞霄宮。至瓜洲,得暍疾,及京口卒,
 年五十七,謚忠簡。



 景衡得程頤之學,志慮忠純,議論不與時俯仰。建炎初,李綱議幸南陽,宗澤請還京,景衡乃請幸建康。黃潛善等素惡其異己,暨車駕駐揚州,怵於傳聞,不得已下還京之詔,遂借渡江之議罪之,斥逐而死。既沒,高宗思之曰:「朕自即位以來,執政忠直,遇事敢言,惟許景衡。」詔賜景衡家溫州官舍一區。



 張愨,字誠伯,河間樂壽人。登元祐六年進士第。累遷龍圖閣學士、計度都轉運使。高宗為兵馬大元帥,募諸道
 兵勤王,愨飛挽踵道,建議即元帥府印給鹽鈔,以便商旅。不閱旬,得緡錢五十萬以佐軍。高宗器重之,命以便宜權大名尹兼北京留守、馬步軍都總管。愨初聞二帝北行,率副總管顏岐等三上箋勸進。最後,愨上書,極論中原不可一日無君,高宗為之感悟。



 建炎改元,為戶部尚書,除同知樞密院事、措置戶部財用兼御營副使。建言:「三河之民。怨敵深入骨髓,恨不殲殄其類,以報國家之仇。請依唐人澤潞步兵、雄邊子弟遺意,募民聯以什
 伍,而寓兵於農,使合力抗敵,謂之巡社。」為法精詳,前此論民兵者莫及也。詔集為書行之。遷尚書左丞,官至中書侍郎。



 愨善理財,論錢谷利害,猶指諸掌。在朝諤諤有大臣節,然論議可否,不形辭色,未嘗失同列之歡。卒,謚忠穆。上每念之,謂愨謀國盡忠,遇事敢諫,古之遺直也。



 張所,青州人。登進士第,歷官為監察御史。高宗即位,遣所按視陵寢,還,上疏言:「河東、河北,天下之根本。昨者誤用奸臣之謀,始割三鎮,繼割兩河,其民怨入骨髓,至今
 無不扼腕。若因而用之,則可藉以守;不則兩河兵民,無所系望,陛下之事去矣。」且論還京師有五利,謂國之安危,在乎兵之強弱、將相之賢不肖,不在乎都之遷不遷。又條上兩河利害。上欲以其事付所,會所言黃潛善奸邪不可用,恐害新政。乃罷所御史,改兵部郎中。尋責所鳳州團練副使,江州安置。



 後李綱入相,欲薦所經略兩河,以其嘗言潛善故,難之。一日,與潛善從容言曰:「今河北未有人,獨一張所可用,又以狂言抵罪。不得已抆拭
 用之,使為招撫,冒死立功以贖過,不亦善乎?」潛善許諾,乃借所直龍圖閣,充河北招撫使。賜內府錢百萬緡,給空名告千餘道;以京西卒三千為衛,將佐官屬,許自闢置,一切以便宜從事。所入見,條上利害。上賜五品服遣行,命直秘閣王圭為宣撫司參謀官佐之。



 河北轉運副使張益謙附黃潛善意,奏所置司北京非是;且言自置招撫,河北盜賊愈熾,不若罷之,專以其事付帥司。李綱言:「張所今留京師,招集將佐,尚未及行,益謙何以知其
 擾?朝廷以河北民無所歸,聚而為盜,故置司招撫,因其力而用之,豈由置司乃有盜賊乎?今京東、西群盜公行,攻掠郡縣,亦豈招撫司過耶?時方艱危,朝廷欲有所經理,益謙小臣,乃以非理沮抑,此必有使之者。」上乃命益謙分析,命下樞密院,汪伯彥猶用其奏詰責招撫司。李綱與伯彥爭於上前,伯彥語塞。



 所方招來豪傑,以王彥為都統制,岳飛為準備將,而李綱已罷相。朝廷以王圭代之,所落直龍圖閣,嶺南安置。卒於貶所。子宗本,以岳
 飛奏補官。



 陳禾,字秀實,明州鄞縣人。舉元符三年進士。累遷闢雍博士。時方以傳注記問為學,禾始崇尚義理,黜抑浮華。入對契旨,擢監察御史、殿中侍御史。



 蔡京遣酷使李孝壽窮治章綖鑄錢獄,連及士大夫甚眾,禾奏免孝壽。京子壝為太常少卿,何執中婿蔡芝為將作監,皆疏其罪,罷之。天下久平,武備寬弛,東南尤甚。禾請增戍、繕城壁,以戒不虞。或指為生事,格不下。其後盜起,人服其先見。
 遷左正言,俄除給事中。



 時童貫權益張,與黃經臣胥用事,御史中丞盧航表裏為奸,搢紳側目。禾曰:「此國家安危之本也。吾位言責,此而不言,一遷給舍,則非其職矣。」未拜命,首抗疏劾貫。復劾經臣:「怙寵弄權,誇炫朝列。每云詔令皆出其手,言上將用某人,舉某事,已而詔下,悉如其言。夫發號施令,國之重事,黜幽陟明,天子大權,奈何使宦寺得與?臣之所憂,不獨經臣,此塗一開,類進者眾,國家之禍,有不可遏,願亟竄之遠方。」



 論奏未終,上拂
 衣起。禾引上衣,請畢其說。衣裾落,上曰:「正言碎朕衣矣。」禾言:「陛下不惜碎衣,臣豈惜碎首以報陛下?此曹今日受富貴之利,陛下他日受危亡之禍。」言愈切,上變色曰:「卿能如此,朕復何憂?」內侍請上易衣,上卻之曰:「留以旌直臣。」翌日,貫等相率前訴,謂國家極治,安得此不詳語。盧航奏禾狂妄,謫監信州酒。遇赦,得自便還里。



 初,陳瓘歸自嶺外,居于鄞,與禾相好,遣其子正匯從學。後正匯告京罪,執詣闕,瓘亦就逮。經臣蒞其獄,檄禾取證,禾答
 以事有之,罪不敢逃。或謂其失對,禾曰:「禍福死生,命也,豈可以死易不義耶?願得分賢者罪。」遂坐瓘黨停官。



 遇赦,復起知廣德軍,移知和州。尋遭內艱,服除,知秀州。王黼新得政,禾曰:「安能出黼門下?」力辭,改汝州。辭益堅,曰:「寧餓死。」黼聞而銜之。禾兄秉時為壽春府教授,禾侍兄官居。適童貫領兵道府下,謁不得入,饋之不受。貫怒,歸而譖之,上曰:「此人素如此,汝不能容邪?」久之,知舒州,命下而卒,贈中大夫,謚文介。



 禾性不茍合,立朝挺挺有風
 操。有《易傳》九卷,《春秋傳》十二卷,《論語》、《孟子解》各十卷。



 蔣猷,字仲遠,潤州金壇縣人。舉進士。政和四年,拜御史中丞兼侍讀,有直聲。嘗論士風浮薄,廷臣伺人主意,承宰執風旨向背,以特立不回者為愚,共嗤笑之,此風不可長;輔臣奏事殿上,雷同唱和,略無所可否,非論道獻替之禮;內侍省不隸臺察,紊元豐官制;楊戩不當除節度使;趙良嗣不宜出入禁中。上皆嘉納,至揭其章內侍省,且詔自今無得規圖節鉞。又疏孟昌齡、徐鑄等奸狀。
 遷兵部尚書兼禮制局詳議官。七年,知貢舉,改工部、吏部尚書。



 以徽猷閣直學士知婺州。明年,請祠歸。宣和末,召為刑部尚書兼資善堂翊善。靖康初,奉上表起居太上皇帝於淮陰,且特詔貶童貫。猷奏貫得罪天下,願黜遠之。太上以為然,亟令宣詔,趣貫赴貶所。遂奉太上還京,移兵部尚書,累官正議大夫。引疾,授徽猷閣直學士、提舉嵩山崇福宮。卒。贈特進。



 論曰:夫拯溺救焚之際,必以任人為急。靖康、建炎之禍
 變,亦甚於焚溺矣。當時非乏人才也,然而國恥卒不能雪者,豈非任之之道有所未至歟?夫以李光之才識高明,所至有聲;許翰、許景衡之論議剴切;張愨之善理財;張所之習知河北利害:皆一時之雋也。是數臣者,使其言聽計從,不為讒邪所抑,得以直行其志,其效宜可待也。然或斥遠以死,或用之不竟其才,世之治亂安危,雖非人力所為,君子於此,則不能無咎於時君之失政焉。蔣猷歷仕五朝,當建炎初,避地而終,則無足稱也。陳禾
 引裾盡言,有古諫臣之風,其行事在宣和之前,孝宗以後乃加褒謚雲。



\end{pinyinscope}