\article{志第一百一十三 選舉六(保任 考課)}

\begin{pinyinscope}

 保任之制。銓注有格,概拘以法,法可以制平而不可以擇才,故予奪升黜,品式具在,而又責官以保任之。凡改秩遷資,必視舉任有無,以為應否;至其職任優殊,則又
 隨事立目,往往特詔公卿、部刺史、牧守長官,即所部所知,揚其才識而任其能否。上自侍從、臺諫、館學,下暨錢穀、兵武之職,時亦以薦舉命之,蓋不膠於法矣。



 國初,保任未立限制。建隆三年始詔:「常參官及翰林學士,舉堪充幕職、令、錄者各一人,條析其實,毋以親為避。」既而舉者頗因緣為奸,用知制誥高錫奏:「請許人訐告,得實,則有官者優擢,非仕宦者授以官,或賞緡錢;不實,則反坐之。」自是,或特命陶穀等舉才堪通判者,或詔翰林學士
 及常參官舉京官、幕職、州縣正員堪升朝者。藩鎮奏掌書記多越資敘,則詔歷兩任有文學方得奏。又令諸道節度、觀察使,於部內官選才識優茂、德行敦篤者各二人,防禦、團練使各舉一人,遣詣闕庭,觀其器業而進用焉。凡被舉擢官,於誥命署舉主姓名,他日不如舉狀,則連坐之。



 太宗尤嚴牧守之任,詔諸道使者察部內履行著聞、政術尤異、文學茂異者,州長吏擇判、司、簿、尉之清廉明乾者,具名以聞,驛召引對,授之知縣。又令閱屬部
 司理參軍,廉慎而明於推鞫者,舉之。雍熙二年,舉可升朝者,始令翰林學士、兩省、御史臺、尚書省官舉之。



 淳化三年,令宰相以下至御史中丞,各舉朝官一人為轉運使,乃詔曰:「國家詳求幹事之吏,外分主計之司,雖曰轉輸,得兼按察,總覽郡國,職任尤重,物情舒慘,靡不由之。尚慮徼功,固當責實。凡轉運使厘革庶務,平反獄訟,漕運金谷,成績居最,及有建置之事,果利於民,令歲終以聞。非殊異者不得條奏。」又詔:三司、三館職事官已升擢
 者,不在論薦;其有懷材外任,未為朝廷所知者,方得奏舉。始令內外官,凡所舉薦有變節逾矩者,自首則原其聯坐之罪。



 太宗聽政之暇,每取兩省、兩制清望官名籍,擇其有德譽者,悉令舉官。所舉之人,須析其爵里及歷任殿最以聞,不得有隱。如舉狀者有賞典,無驗者罪之。又嘗謂宰臣曰:「君子小人,趣向不同。君子畏慎,不欺暗室,名節造次靡渝;小人雖善談忠信,而履行頗僻,在官黷貨,罔畏刑罰。如薛智周以侍御史守婺,政以賄成,聚
 斂無已,其土產富於羅,州民謂之『羅端公』,則為治可知矣。卿等職在掄材,今令朝臣舉官,已是逐末,更不擇舉主,何由得人也。」供奉官劉文質嘗入奏,察舉兩浙部內官高輔之、李易直、艾仲孺、梅詢、高鼎、高貽慶、姜嶼、戚綸八人有治跡,並降璽書褒諭。帝曰:「文質所舉,皆良吏也。」特遷文質為西京作坊副使。



 咸平間,秘書丞陳彭年請用唐故事舉官自代。詔樞密直學士馮拯、陳堯叟參詳之。拯等上言:「往制,常參官及節度、觀察、防禦、刺史、少尹、
 畿赤令並七品以上清望官,授訖三日內,於四方館上表讓一人以自代。其表付中書門下,每官闕,以見舉多者量而授之。今官品制度沿革不同,請令兩省、御史臺、尚書省六品以上,諸司四品以上,授訖,具表讓一人自代,於閣門投下,方得入謝。在外者,授訖三月內,具表附驛以聞。」遂著為令。



 真宗初,屢詔舉官,未立常制。大中祥符二年詔:「幕職、州縣官初任,未閑吏事,須三任六考,方得論薦。」三年,始定制:



 自翰林學士以上常參官,歲各舉
 外任京朝官、三班使臣、幕職、州縣官一人,著其治行所宜任,令閣門、御史臺歲終會其數。如無舉狀,即具奏致罰。於冬季以差出,亦須舉官後乃入辭。諸司使副、承制、崇班曾任西北邊、川、廣鈐轄、親民者,亦仿此制。諸路轉運使副、提點刑獄官,知州、通判奏舉部內官屬,則不限人數,具在任勞績,如無可舉及顯有逾濫者,亦須指述,不得顧避。以次年二月二十五日以前到京,違期則都進奏院以名聞,論如不申考帳法。



 三司使副舉在京掌
 事京朝官、使臣。凡被舉者,中書歲置二籍,疏其名銜,下列歷任功過、舉主姓名及薦舉數。一以留中書,一以五月一日進內。明年,籍內仍計向來功過及舉主數,使臣即樞密院置籍。兩省、尚書省、御史臺官凡出使回,須採訪所至及經歷鄰近郡官治跡善惡以聞。轉運使副、提點刑獄官、知州、通判赴闕,各具前任部內官治跡能否,如鄰近及所經州縣訪聞善惡,亦許同奏,先於閣門投進,方得入見。



 凡朝廷須人才,及欲理州縣弊政劇務,即
 籍內視舉任及課績數多而資歷相當者差委,於宣敕內盡列舉主姓名。或任內幹集,特與遷秩,茍不集事,本犯雖不去官,亦移閑慢僻遠地。內外群臣所舉及三人有成績,仰中書、樞密院具姓名取旨甄獎。如並舉三人俱不集事,坐罪不至去官,亦仰奏裁,當行責降。或得失相參,亦與折當。



 天聖六年詔:「審刑院舉常參官在京刑法司者為詳議官;大理寺詳斷、刑部詳覆法直官,皆舉幕職、州縣曉法令者為之。自請試律者須五考,有舉者,
 乃聽試。試律三道,疏二道,又斷中小獄案二道,通者為中格。」時舉官擢人,不常其制。國子監闕講官,則詔諸路轉運使舉經義通明者;或欲不次用人,嘗詔近臣舉常參官歷通判無贓罪而才任繁劇者;欲官諸邊要,亦嘗詔節度使至閣門使、知州軍、鈐轄、諸司使,舉殿直以上材勇堪邊任者,或令三司使下至天章閣待制舉奏之。邊有警,則詔諸路轉運使、提點刑獄舉所部官才堪將帥者;三路知州、通判、縣令,則詔近臣舉廉幹吏選任之,毋
 拘資格。至於文行之士,錢穀之才,刑名之學,各因時所求而薦焉。



 自天聖後,進者頗多,始戒近臣,非受詔毋輒舉官。又下詔風厲,毋以薦舉為阿私。其任用已至部使者,毋得復薦,失舉而已擢用,聽。自言不實,弗為負。初,選人四考,有舉者四人,得磨勘遷京官;始詔增為六考,舉者五人,須有本部使者。御史王端以為:「法,用舉者兩人,得為縣令。為令無過譴,遷職事官、知縣;又無過譴,遂得改京官。乃是用舉者兩人,保其三任也。朝廷初無參伍
 考察之法,偶幸無過,輒信而遷之。是以碌碌之人,皆得自進,因仍弗革,其弊將深。」乃定令:被薦為令,任內復有舉者始得遷,否則如常選,毋輒升補。



 時增設禁限,常參官已授外任,毋得奏舉。京官見任知州、通判,升朝官兵馬都監、諸司副使以上,及在京員外郎嘗任知州、通判,諸司副使嘗任兵馬都監者,乃聽舉,流內銓復裁。內外臣僚歲舉數,文臣待制至侍御史,武臣自觀察至諸司副使,舉吏各有等數,毋得輒過;而被舉者須有本部監
 司、長吏、按察官,乃得磨勘。又限到官一考,方得薦。知雜御史、觀察使以上,歲舉京官不得過二人,其常參官毋得復舉,自是舉官之數省矣。定監司以所部州多少劇易之差,為舉令數,非本部勿舉。其後又增舉主三員。蓋官冗之弊浸極,故保薦之法,大抵初略而後詳也。



 英宗時,御史中丞賈黯又言:「今京朝官至卿、監,凡二千八百餘員,而吏部奏舉磨勘選人,未引見者至二百五十餘人。且以先朝事較之:方天聖中,法尚簡,選人以四考改
 官,而諸路使者薦部吏,未有限數;而在京臺閣及常參官嘗任知州、通判者,雖非部吏皆得薦。時磨勘改官者,歲才數十人,後資考頗增,而知州薦吏,視屬邑多少裁定其數,常參官不許薦士。其條約漸繁,而改官者固已眾矣,然引對猶未有待次者也。皇祐中,始限監司奏舉之數,其法益密,而磨勘待次者已不減六七十人。皇祐及今才十年耳,而猥多至於三倍。向也,法疏而其數省;今也,法密而其數增,此何故哉?正在薦吏者歲限定員,
 務充數而已。如郡守歲許薦五人,而歲終不滿其數,則人人以為遺己。當舉者避謗畏譏,欲止不敢,此薦者所以多,而真才實廉未免混於無能也。宜明詔天下,使有人則薦,不必滿所限之數。」天子納其言,下詔申敕。中外臣僚歲得舉京官者,視元數以三分率之,減一分;舉職官,有舉者三人,任滿選如法。所以分減舉者數,省京官也。



 判吏部流內銓蔡抗又言:「奏舉京官人,度二年引對乃可畢,計每歲所舉,無慮千九百員,被舉者既多,則磨勘者
 愈眾。且今天下員多闕少,率三人而待一闕,若不稍改,除吏愈難。臣以為可罷知雜御史、觀察使以上歲得舉官法。」從之。自是舉官之數彌省矣。故事,初入二府,舉所知者三人,將以觀大臣之能。後來請謁之說勝,而薦者或不以公。四年詔:「中書、樞密院舉人,皆明言才業所長,堪任何事,以副朕為官擇人之意。」



 神宗即位,乃罷兩府初入舉官。凡薦任之法,選人用以進資改秩,京朝官用以升任,舊悉有制。熙寧後,又從而損益之,故舉皆限員,
 而歲又分舉,制益詳矣。定十六路提點刑獄歲舉京官、縣令額。又詔察訪使者得舉官。選人任中都官者,舊無舉薦,始許其屬有選人六員者,歲得舉三員。既而帝以舊舉官往往緣求請得之,乃革去奏舉,而概以定格。詔內外舉官法皆罷,令吏部審官院參議選格。



 元祐初,左司諫王巖叟言:「自罷闢舉而用選格,可以見功過而不可以見人材,中外病之。於是不得已而別為之名,以用其平日之所信,故有『踏逐申差』之目。『踏逐』實薦舉而不
 與同罪,且選才薦能而謂之『踏逐』,非雅名也。況委人以權而不容舉其所知,豈為通術?」遂復內外舉官法。



 及司馬光為相,奏曰:



 為政得人則治。然人之才,或長於此而短於彼,雖皋、夔、稷、契,各守一官,中人安可求備?故孔門以四科論士,漢室以數路得人。若指瑕掩善,則朝無可用之人;茍隨器授任,則世無可棄之士。臣備位宰相,職當選官,而識短見狹,士有恬退滯淹,或孤寒遺逸,豈能周知?若專引知識,則嫌於私;若止循資序,未必皆才。莫
 若使有位達官,各舉所知,然後克協至公,野無遺賢矣。



 欲乞朝廷設十科舉士:一曰行義純固可為師表科,有官、無官人,皆可舉。二曰節操方正可備獻納科,舉有官人。



 三曰智勇過人可備將帥科,舉文武有官人。



 四曰公正聰明可備監司科,舉知州以上資序。五曰經術精通可備講讀科,有官、無官人,皆可舉。



 六曰學問該博可備顧問科,同上。



 七曰文章典麗可備著述科,同上。



 八曰善聽獄訟盡公得實科,舉有官人。



 九曰善治財賦公私俱便科,舉有官人。十曰練習法令能斷請讞科。同上。



 應職事官
 自尚書至給舍、諫議,寄祿官自開府儀同三司至太中大夫,職自觀文殿大學士至待制,每歲須於十科內舉三人,仍具狀保任,中書置籍記之。異時有事須材,即執政案籍視其所嘗被舉科格,隨事試之,有勞,又著之籍。內外官闕,取嘗試有效者隨科授職。所賜告命,仍備所舉官姓名,其人任官無狀,坐以繆舉之罪。所貴人人重慎,所舉得才。



 光又言:「朝廷執政惟八九人,若非交舊,無以知其行能。不惟涉徇私之嫌,兼所取至狹,豈足以盡
 天下之賢才?若採訪毀譽,則情偽萬端。與其聽游談之言,曷若使之結罪保舉?故臣奏設十科以舉士,其『公正聰明可備監司』,誠知請屬挾私所不能無,但有不如所舉,譴責無所寬宥,則不敢妄舉矣。」詔皆從之。



 二年,殿中侍御史呂陶言:「郡守提封千里,生聚萬眾,所系休戚,而不察能否,一以資格用之,凡再為半刺、有薦者三人,則得之矣。不公不明,十郡而居三四,是天下之民,半失其養。請令內外從臣,歲舉可為守臣者各三人,略資序而採
 公言,庶是可以擇才庇民也。」詔:「內外待制、太中大夫以上,歲舉再歷通判資序、堪任知州者一人,籍於吏部。遇三路及一州而四縣者,其守臣有闕,先差本資序人,次案籍以及所薦者。」



 頃之,侍御史韓川言:「近太中大夫以上歲舉守臣,而薦所不及,雖課入優等,皆未預選,此倚薦以為信也。然太中大夫以上,率在京師,唯馳騖請求、因緣宛轉者,常多得之。跡遠地寒,雖歷郡久、治狀著、課入上考,偶以無薦,則反在通判下,不許入三路及四縣州。
 且州以縣之多少而分簡劇,亦未為盡。蓋繁簡在事不在縣,固有縣多而事不繁,亦有縣少而事不簡者。願參以考績之實,著為通令,仍不以縣之多少而為簡劇。」詔吏部立法以聞。已而歲舉積久,吏部無闕以授。四年,遂罷太中大夫以上歲舉法,惟奉詔乃舉焉。



 紹聖元年,右司諫朱勃言:「選人初受任,雖能,法未得舉為京官。而有挾權善請求者,職官、縣令舉員既足,又並改官舉員求之。」詔:「歷任通及三考,而資序已入幕職、令錄,方許舉之
 改官。」



 初,神宗罷薦舉,惟舉御史法不廢。熙寧二年,王安石言:「舉御史法太密,故難於得人。」帝曰:「豈執政者惡言官得人耶?」於是中書悉具舊法以奏。安石曰:「舊法,凡執政聽薦,即不得為御史。執政取其平日所畏者薦之,則其人不復得言事矣,蓋法之弊如此。」帝乃令悉除舊法,一委中丞舉之,而稍略其資格。趙抃曰:「用京官恐非體,又不委知雜,專任中丞,亦非舊制。」帝曰:「唐以布衣馬周為之,用京官何為不可?知雜,屬也,委長為是。」侍御史劉
 述奏曰:「舊制,舉御史必官升京朝,資入通判。眾學士、本臺丞、知雜更互論薦,每一闕上,二人而擇用一人。今專委中丞,則愛憎由己,公道廢於私恩;或受權臣之托,引所親厚,擅竊人主威福,此大不便。」弗聽。既改法,著作佐郎程顥、王子韶、謝景福方為條例司屬官,中丞呂公著薦之,遂以太子中允權監察御史裏行。



 宣仁太后聽政,詔范純仁為諫議大夫,唐叔問、蘇轍為司諫,朱光庭、範祖禹為正言。章惇曰:「故事,諫官皆薦諸侍從,然後大臣
 稟奏,今得無有近習援引乎?」太后曰:「大臣實皆言之,非左右也。」惇曰:「臺諫所以糾大臣之越法者。故事,執政初除,茍有親戚及嘗被薦引者見為臺臣,則皆他徙,防壅蔽也。今天子幼沖,太皇太后同聽萬機,故事不可違。」於是呂公著以範祖禹,韓縝、司馬光以范純仁,皆避親嫌。光曰:「純仁、祖禹實宜在諫列,不可以臣故妨賢,寧臣避位。」惇曰:「縝、光、公著必不私,他日有懷奸當國者,例此而引其親黨,蔽塞聰明,恐非國之福。純仁、祖禹請除他官,
 仍令侍從以上,各得奏舉。」於是,詔尚書、侍郎、給舍、諫議、中丞、待制各舉諫官二員;純仁改除天章閣待制,祖禹為著作佐郎。後又命司諫、正言、殿中侍御史、監察御史,並用升朝官通判資序。



 元祐六年,御史中丞鄭雍言:「舊御史闕,臺官得自薦,所以正名舉職也。自官制行,御史中丞與兩省分舉,而今之兩省官屬,皆與聞門下、中書政事,其自舉非故事,且有嫌。乞專委臺官,若稍涉私,自有黜典。」詔御史中丞舉殿中侍御史二員,翰林學士、中
 書舍人同舉監察御史二員,給事中亦舉二員。雍又言:「風憲之地,責任宜專。若臺屬多由他薦,恐非責任之本意。」詔中丞更舉監察御史二員。八年,侍御史楊畏言:「風憲之任,人主寄耳目焉。御史進用,宰執不得預,顧令兩省屬官舉之,非是。」遂寢前命。



 武臣薦舉立格,有枚別職任而舉之者,有概名材武而入之銓格者,又其上則「謀略膽勇可備統眾」、「諳練兵事可任邊寄」之類。惟邊要任使隸樞密院,餘則審官西院、三班院按格注之。其後,雖
 時有更易,而薦舉之所重輕,選用之所隸屬,多規此立制。



 建炎兵興多事,以中外有文武材略出倫,或淹布衣,或沉下僚,命侍從、監司、郡守搜訪,各舉所知,州縣禮遣赴行在。又詔舉「忠信寬博可使絕域」與「智謀勇毅能將萬眾」者,不以有無官資,並詣登聞檢院自陳,才謀勇略可使者,赴御營司量材錄用。或命庶僚各舉內外官及布衣隱士才堪大用者,擢為輔弼,協濟大功;或命侍從舉可為臺諫者,或舉縣令,或舉宗室;刺史舉忠義之
 士能恢復土疆保護王室者;帥臣、監司、守令舉所部見任寄居待次文武官有智謀及武藝精熟者;及訪求國初功臣後裔,中興以來忠義死節之家子孫。四年,以朝班多闕,詔:「臺諫、左右司郎官已上,各薦士二人,仍令執政同選。在外待從雖在謫籍,無大過而政事才學實可用者,亦與召擢。」



 紹興二年,廷臣言:「今右武之世,雖二三大將,各立俊功,微賤之中,尚多奇士。願廣加薦舉,延問恢復之計。」帝然其言。詔觀察使以上各薦可為將帥者二
 人,樞密籍錄以備選用。又以中原士大夫隔絕滋久,流徙東南者,媒寡援疏,多致沉滯,令侍從搜訪以聞。三年,復司馬光十科,時遣五使宣諭諸道,令訪廉潔清修可以師表吏民者。錄詔宣諭官所薦,並俟終更,令入對升擢,以勸能吏。復用舊制,侍從官受命三日,舉官一員自代,中書、門下省籍記姓名,每闕官,即以舉狀多者進擬。內外武臣,舉忠勇智略可自代者一人,如文臣法。



 五年,命自監察御史至侍從官,舉曾經治縣聲績顯著者為
 監司、郡守,不限員數,遇闕選除;才堪大縣者,通舉二十人,不限資序。十年,以南渡後人材萃於兩浙,而屬吏薦員甚狹,增部使者薦舉改官之額,歲五員。十四年,命守臣終更入見,各舉所部縣令一人。



 二十二年,右諫議大夫林大鼐言:「國初,常參官皆得舉人,不限內外,亦無員數。南渡之初,恩或非泛,人得僥幸,有從軍而改秩者。有捕盜而改秩者,有以登對而改秩者。今朝廷無事,謹惜名器,惟薦舉一路,貪躁者速化,廉靜者陸沉。今欲取考
 第、員數增減以便之,增一任者減一員,九考者用四,十二考者用三,十五考者用二。如減舉法,須實歷縣令,不得仍請岳祠。其或負犯殿選,自如常坐。士有應此格者,行無玷缺,年亦蹉跎,無非孤寒老練安義分之士。望付有司條上,以弭奔競。」二十五年,命侍從舉知州、通判治跡顯著者,以補監司之闕;仍保任終身,犯贓及不職,與同罪。



 二十九年,聞人滋又請:「凡在官歷任及十考以上,無公私罪,雖舉削不及格,許降等升改。或疑其太濫,則
 取吏部累年改官酌中之數,立為限隔,舉狀、年勞,參酌並用。」於是下其議,中書舍人洪遵、給事中王晞亮等上議曰:「本朝立薦舉之法,必使歷任六考,所以遲其歲月而責其赴功,必使之舉官五員,所以多其保任而必其可用。今如議臣所請,則有力者惟圖見次,無材者茍冀終更,出官十餘年,可以坐待京秩。此不可一也。今欲減改官分數以待無舉削者,則當被舉之人,必有失職淹滯之嘆。此不可二也。京官易得,馴至郎位,任子之恩,愈
 不可減,非所以救入流之弊。此不可三也。夫祖宗之法非有大害,未易輕議;今一旦取二百年成法而易之。此不可四也。臣以為如故便。」滋議遂寢。



 三十年,以武臣被薦者眾,命內外大臣所舉統制、統領官各遷一秩,將官以下,所舉者今兩府籍記。右正言何溥言:「比命侍從薦舉縣令,如聞選人不可授大邑,止籍記姓名。夫論人才不拘資格,豈堪為縣令而有小大之別乎?今所舉者才也,非官也。願無拘劇易,早與選除,歲一行之,十年之後,
 天下多賢令矣。」乃詔:「薦舉守令,遇見闕依次除授;如已授差遣者,任滿取旨。」帝謂輔臣曰:「朕有一人材簿,臣下有所薦揚,退則記其姓名。遇有選用,搜而得之,無不適當。」



 孝宗嘗命內外選在任閑居待次官舉可任監司、郡守之人,以資序分二等,一見今可任,一將來可任,注籍於三省,仍作圖進呈,以憑除擢。又以武選之眾,拔擢未廣,立「謀略沉雄可任大計」、「寬猛適宜可使御眾」、「臨陣驍勇可鼓士氣」、「威信有聞可守邊郡」、「思智精巧可治器械」
 凡五等科目,令曾歷軍功觀察使以上各舉三人。其「通習典章可掌朝儀」、「練達民事可任郡寄」、「諳曉財計可裕民力」、「持身廉潔可律貪鄙」、「詞辨不屈可備奉使」五等,令非軍功觀察使以上舉之。並隨類指陳實跡,毋得別撰褒詞。



 隆興二年,廷臣上言,謂:「國朝視文武為一體,故有武臣以文學換授文資,文臣以材略智謀換右職當邊寄者。蓋文武兩塗,情本參商。若文臣總幹戎事,不換武階,則終以氣習相忌,有不樂從者矣。今兵塵未息,方厲
 恢復之圖,願博採中外有材智權略可以臨邊、可以制閫者,仿舊制改授。」從之。乾道以後,又選大將之家能世其武勇者,武舉及第武藝絕倫可為將佐者。會廷臣言曰:「方今國家之兵,東至淮海,西至川蜀,殆百餘萬。其間可為將帥者,不在其上,則在其下,而朝廷未知振其氣、表其才也。今文臣有三人舉主,則為之循資再任,五人則為之改秩,而武臣無有焉。古語曰:『三辰不軌,擢士為相;蠻夷不恭,拔卒為將。』宜令都統制視監司者歲舉武
 臣二人,視郡守者歲舉一人。以智勇俱全為上,善撫士卒、專有膽勇者次之。不拘將校士卒,優以獎擢。被舉人有臨戰不用命者,與文臣犯入己贓者同,並坐舉主。」帝可其奏,仍著為法。



 三年,禮部尚書趙雄請令侍從、臺諫、兩省,於知縣資序以上歲薦堪充郡守,通判資序以上歲薦監司,仍用漢朝雜舉之制,三省詳加考察。詔如所請,仍不以內外,雜舉歲各五人,保舉官及五員以上,列銜共奏。帝曰:「薦舉本欲得人,又恐干請,反長奔競。」龔茂
 良言:「三代良法,亦不免於弊。今欲精選監司、郡守,非薦舉何由知之。」帝曰:「若今雜舉,則須眾論僉允,又經中書考察而後除授,亦博採遴選之道也。」



 吏部請:「武舉軍班武藝特奏名出身,並任巡檢、駐泊、監押、知砦,比附《文臣關升條令》,並實歷六考,有舉主四人,內監司一人,聽關升親民。正副將,兩任、有舉主二人,內一人監司,亦與關升。凡升副將,視文臣初任通判資序;再關升正將,視文臣次任通判資序;關升路分副都監,視文臣初任知州
 資序;小郡州鈐轄,視文臣次任知州資序。」孝宗以歲舉京官數濫,於是內外薦舉改官員數,六部、寺、監長貳,戶部右曹郎官等,三分減一;禮部、國子監長貳,如上條外又減半;前宰執,歲各減二員;諸道轉運、提刑、提舉常平茶鹽學事司,總領茶馬、鑄錢司,安撫、制置司,及諸路州軍,並四分減一。通籍之數彌省矣。



 光宗時,言者謂:「被薦者眾,朝廷疑其私而不信,病其泛而難從,縱有賢才,不免與僥幸者並棄,請條約之。」乃命帥守、監司毋獨員薦
 士。時薦舉固多得人,然有或乏廉聲而舉充廉吏,或素昧平生而舉充所知,或不能文而舉可備著述。遂命臣僚自今有人則薦,無人則闕,其尤繆妄者覺察之。



 嘉泰二年,令內外舉薦並具實跡以聞,自是濫舉之弊稍革。嘉定十二年,命監司、守臣舉十科政績所知自代,露章列薦,並籍記審察。任滿,則取其舉數多、有政績行誼者,升擢之。



 宋初,內外小職任,長吏得自奏闢。熙寧間,悉罷歸選部。然要處職任,如沿邊兵官、防河捕盜、重課額務
 場之類,尋又立專法聽舉,於是闢置不能全廢也。既出常格,則憸人往往因之以行其私。元祐以來,屢行屢止。蓋處心公明,則得以用其所知,固為良法;茍徇私昧理,則才不為用,請屬賄賂,無所不有矣。又孰若付之銓曹而概以公法者哉?



 建炎初,詔河北招撫、河東經制及安撫等使,皆得闢置將佐官屬;行在五軍並御營司將領,亦闢大小使臣。諸道郡縣殘破之餘,官吏解散,諸司誘人填闕,皆先領職而後奏給付身。於是州郡守將,皆假
 軍興之名,換易官屬,有罪籍未敘復、守選未參部者。朝論患之,乃令厘正,使歸部依格注擬。惟陜西五路、兩河、兩淮、京東等路經略安撫司屬官聽舉闢,餘路並罷。四年,初置諸鎮撫使,管內州縣官並許闢置。言者謂遠方之民,理宜綏撫。如峽州四縣,多用軍功或胥吏補知縣,欄吏補監稅,民被其害。遂命取峽州、江陵府、荊門軍、公安軍州縣官闕,委安撫司奏闢。命御史臺仍舊闢舉承務郎已上官充主簿、檢法官,不限資序。



 紹興二年,臣僚
 又以「比年帥守、監司闢官,攙奪部注,朝廷不能奪,銓曹不能違,又多畀以添差不厘務之闕。上自監司、倅貳,下至掾屬、給使,一郡之中,兵官八九員,一務之中,監當六七員,數倍於前日。存無事之官,食至重之祿,所以重困生民。請裁省其闕,否則以宮廟之祿畀之。」遂命自今已就闢差理資任者,毋得據舊闕以妨下次。六年,詔諸道宣撫司,僚屬許本司奏闢,內京官以二年為任,願留再任者,取旨。自兵興,所闢官有經十年不退者,故條約焉。
 二十六年,詔已注知縣、縣令,不許奏闢。



 孝宗初,詔內外有專法,闢闕並仍舊。乾道九年,命監司、帥臣,非有著令,不得創行奏闢;所闢毋得攙已差之闕,違者御史臺察之。淳熙三年,命自今極邊知縣、縣令闕官,專委本州守臣奏闢,毋得仍舊權攝;其見攝官留意民事百姓愛服者,許不以有無拘礙,特行奏闢。七年,詔未中銓、未歷任、初改秩人毋得差闢,著為令。



 理宗寶慶二年,以廣南東、西路通判、幕職、教授等官,法未嘗許闢者,須於各官將
 滿之前具闕。如未有代者,即聽申部出闕,滿三月無人注擬,申省下本路。通判以下京官闕,從諸司奏闢。選人闕,從漕司定差。作邑未滿三年、作倅未滿二考,不許預期奏闢他闕。諸司屬官不許輒自闢置,或久闕正官,許令次官暫攝,待朝命方許奏闢。淳祐十一年,以御史臺申嚴銓法,禁監司、郡守闢親戚為屬吏。又選人無考第、舉主不及三員,及納粟人雖有考第、舉主,並不聽闢為令。寶祐三年,戒諸路監司、帥閫,不應闢而輒闢者,闢主
 及受闢之官,並與鐫秩。



 考課。宋初循舊制,文武常參官各以曹務閑劇為月限,考滿即遷。太祖謂非循名責實之道,罷歲月敘遷之制。置審官院,考課中外職事。受代京朝官引對磨勘,非有勞績不進秩。其後立法,文臣五年、武臣七年,無贓私罪始得遷秩。曾犯贓罪,則文臣七年、武臣十年,中書、樞密院取旨。其七階選人,則考第資歷,無過犯或有勞績者遞遷,謂之「循資」。凡考第之法,內外選人,周一歲為一
 考,欠日不得成考。三考未替,更周一歲,書為第四考,已書之績,不得重計。初著令,州縣戶口準見戶十分增一,刺史、縣令進考,若耗一分,降考一等。建隆三年,又以科賦有欠逾十之一,及公事曠違嘗有制受罰者,皆如耗戶口降考。吏部南曹又舉周制,請州縣官益戶增稅,受代日並書於籍,凡千戶以下能增百戶減一選,減及三選以上,令賜章服,主簿升秩進階。能歸復逋亡之民者,亦如之。



 是年,縣始置尉,頒《捕盜條》,給以三限,限各二十日,
 三限內獲者,令、尉等第議賞;三限外不獲,尉罰一月奉,令半之。尉三罰、令四罰,皆殿一選,三殿停官。令、尉與賊鬥而能盡獲者,賜緋升擢。乾德四年,詔諸縣令、佐有能招攜勸課,以致蕃庶民籍,租額出其元數,減一選,仍進一階。



 太宗勵精圖治,遣官分行郡縣,廉察官吏。河南府法曹參軍高丕等,皆以不勝任免官。復詔諸道察舉部內官,第其優劣為三等:「政績尤異」為上,「職務粗治」為中,「臨事弛慢所蒞無狀」者為下。歲終以聞。先是,諸州掾曹
 及縣令、簿、尉,皆戶部南曹給印紙、歷子,俾州郡長吏書其績用愆過,秩滿,送有司差其殿最。詔有司申明,其諸州別給公據者罷之。判吏部南曹董淳言:「有司批書印歷,多所闕略,令漏書一事殿一選,三事降一資。」自是職事官依州縣給南曹歷子,天下知州、通判、京朝官厘務於外者,給以御前印紙,令書課績。時蔣元振知白州,為政清簡,民甚便之;秩滿,眾輒詣部使乞留,凡十有八年,未受代。姚益恭清白有才幹,知鄆州須城縣,鞭相不施,
 境內大治。淳化初,採訪使各言其狀,下詔褒嘉,賜元振絹三十匹、粟五十石,賜益恭對衣、銀帶、絹五十匹。



 四年,始分置磨勘之司。審官院掌京朝官,考課院掌幕職、州縣官,廢差遣院,令審官總之。乃詔:「郡縣有治行尤異、吏民畏服、居官廉恪、蒞事明敏、鬥訟衰息、倉廩盈羨、寇盜剪滅、部內清肅者,本道轉運司各以名聞,當驛置赴闕,親問其狀加旌賞焉。其貪冒無狀、淹延斗訟、逾越憲度、盜賊競起、部內不治者,亦條其狀以聞,當行貶斥。」



 以翰
 林學士錢若水、樞密直學士劉昌言同知審官院,考覆功過,以定升降;又以判流內銓翰林學士蘇易簡、知制誥王旦等知考課院,重其職也。凡流內銓,主常調選人;考課院,主奏舉及歷任有殿最者。明年,帝親選京朝官三十餘人,自書戒諭之言於印紙曰:「勤政愛民,奉法除奸,方可書為勞績。」且謂錢若水曰:「奉法除奸之言,恐諸臣未喻,因而生事,可語之曰:『除奸之要,在乎奉法。』」至道初,罷考課院,並流內銓。二年,遣使廉察諸道長吏,得八
 人蒞事公正、惠愛及民,皆降璽書獎諭。



 真宗即位,命審官院考京朝官殿最,引對遷秩。京朝官引對磨勘,自此始。先是,每恩慶,百僚多得序進。帝始罷之,惟郊祀恩許加勛、階、爵邑。帝察群臣有聞望者,得刑部郎中邊肅等二十有四人,令閣門再引對,觀其辭氣文藝,並得優升。景德初,令諸道辨察所部官吏能否,為三等:公勤廉干惠及民者為上,幹事而無廉譽、清白而無治聲者為次,畏懦貪猥為下。



 仁宗尤矜憐下吏,以銓法選人有私罪,
 皆未聽磨勘,諭近臣:「凡『門謝弗至』與『對揚失儀』,其毋以為罪。」又曰:「州縣秩卑,而長吏多鉤摭細故,文致之法,使不得自進,朕甚閔焉。」宰相王曾曰:「引對時,陛下酌其輕重而稍擢之,則下無滯才矣。」其後選人,有束鹿縣尉王得說,歷官寡過,書考最多而無保任者。帝察其孤貧,特擢為大理寺丞。天聖時,詔:「文武臣僚,非有勛德善狀,不得非時進秩;非次罷免者,毋以轉官帶職為例。兩省以上,舊法四年一遷官,今具履歷聽旨。京朝官磨
 勘年限,有私罪及歷任嘗有罪,先以情重輕及勤績與舉者數奏聽旨;若無私犯而著最課及有舉者,皆第遷之。自請厘物務於京師,五年一磨勘,因舉及選差勿拘。凡有善政異績,準事大小遷升,選人視此。」又定監物務入親民,次升通判,通判升知州,皆用舉者。舉數不足,毋輒關升。



 慶歷三年,從輔臣範仲淹等奏定磨勘保任之法:自朝官至郎中、少卿,須清望官五人保任,始得遷。其後,知諫院劉元瑜以為適長奔競,非所以養廉恥,乃罷之。



 八年,
 詔近臣論時政。翰林學士張方平言:「祖宗之時,文武官不立磨勘年歲,不為升遷資序。有才實者,從下位立見超擢,無才實者,守一官十餘年不轉。其任監當或知縣、通判、知州,至數任不遷。當時人皆自勉,非有勞效,知不得進。祥符之後,朝廷益循寬大,自監當入知縣,知縣入通判,通判入知州,皆以兩任為限;守官及三年,例得磨勘。先朝始行,未見有弊。及今年深,習以為常,皆謂分所宜得,無賢不肖,莫知所勸。願陛下稍革此制,其應磨勘
 敘遷,必有勞績;或特敕擇官保任者,即與轉遷;如無勞績又不因保任者,更增展年。其保任之法,須選擇清望有才識之人,命之舉官。如此,則是執政之臣舉清望官,委清望官舉親民官。凡官有闕,惟隨員數舉之,庶見急才愛民之意。」



 嘉祐六年,下詔曰:「朕觀古者治世,牧民之吏,多稱其官,而百姓安其業。今求材之路非不廣,責善之法非不詳,而吏多失職,非稱所以為民之意。豈人材獨少而世變殊哉?殆不得久於其官故也。蓋智能才力
 之士,雖有興利除害、禁奸勸善之意,非假以歲月,則亦偷不為用,欲終厥功,其路無由。自今諸州縣守令,有清白不擾、政跡尤異而實惠及民者,本路若州連書同罪保舉,將政跡實狀以聞,中書門下察訪得實,許令再任。」



 英宗治平三年,考課院言:「知磁州李田,再考在劣等。」降監淄州鹽酒稅務。坐考劣降等,自田始。考績,舊,審定殿最格法,自發運使率而下至於知州,皆歸考課院,專以監司所第等級為據;至考監司,則總其甄別部吏能否,
 副以採訪才行,合二事為課,悉書「中等」,無高下。



 神宗即位,凡職皆有課,凡課皆責實。監司所上守臣課不占等者,展年降資;而治狀優異者,增秩賜金帛,以璽書獎勸之。若監司以上,則命御史中丞、侍御史考校。凡縣令之課,以斷獄平允、賦入不擾、均役屏盜、勸課農桑、振恤饑窮、導修水利、戶籍增衍、整治簿書為最,而德義清謹、公平勒恪為善,參考治行,分定上、中、下等。至其能否尤殊絕者,別立優劣二等,歲上其狀,以詔賞罰。其入優劣者,
 賞罰尤峻。繼又令:一路長吏,無甚臧否,不須別為優劣二等,止因上、中、下三等區別以聞。是時,內外官職,各從所隸司以考核,而中書皆置之籍。每歲竟,或有除授,則稽差殿最,取其尤甚者而進退之。



 熙寧五年,遂罷考課院。間遣使察訪,所至州縣,條其吏課。凡知州、通判上中書,縣令上司農,各注籍以相參考。惟侍從出守郡,聽不以考法,朝廷察其治焉。元豐元年,詔因勞效得酬賞,皆分五等,有司受其等而差進之。初一等,京朝官、大小使
 臣皆轉一官,選人資歷深者改京朝官,資淺者循兩資。次二等,隨其官高下升資,或減磨勘年。惟軍功、捕盜皆得改次等。京朝官自三等以下,賞以差減。若一人而該兩賞,許累計其等以遷。三年,詔:「御史臺六察按官,以所糾劾官司稽違失職事多寡為殿最,中書置簿以時書之,任滿,取旨升黜。」



 元祐初,御史中丞劉摯言:「近者,朝廷主察名實,行綜核之政,下乃承之以刻;主行教化,擴寬洪之澤,而下乃為茍簡。先此追罪監司數人,為其掊斂
 害民耳;而昧者矯枉過正,乃欲以緩縱委靡為安靜。請申立監司考績之政,以常賦登耗、郡縣勤惰、刑獄當否、民俗休戚為之殿最,歲終用此以誅賞之。」文彥博又奏:「《唐六典》所載,以德行、才用、勞效三類察在選之士,參辨能否。今之選格特多,舉主、有軍功,斯為上矣。然舉主可求,軍功或妄,何可盡據?請委吏部尚書侍郎依仿三類,第其才德功效,送中書門下覆驗,取其應選者,引對而去留之。」詔令近臣議,議者請用《元豐考課令》,第為高下,以行升黜,歲
 毋過五人。後改立縣令課,有「四善」、「五最」之目,及增損監司、轉運課格,守令為五等減磨勘法。初,元祐嘗立吏、戶、刑三部郎官課。崇寧間,言者乞仿周制,歲終委省、寺、監、六曹之長,各考其屬,稽其官鄙,而三年遂校其勤惰,行賞罰焉。



 大觀元年詔:「國家休養生民,垂百五十年。生齒日繁,而戶部民籍曾不加益,州縣於進丁、入老,收落失實,以故課役不均,皆守令弛職,可申嚴《考課法》。」然其考法,因時所尚,以示誘抑。若勸學、墾田、植桑棗、振貸、葬枯、
 興發坑冶、奉詔無違、誘進道徒、賦稅趣辦、能按贓吏,皆因事而增品目,舊法固不易也。但奉行不皆良吏,以請謁移實者亦多。



 紹興二年,初詔監司、守臣舉行考課之法。時郡縣數罹兵燹,又命以「戶口增否」別立守令課,分上、中、下三等,每等分三甲置籍。守倅考縣令,監司考知州,考功會其已成,較其優劣而賞罰之。五年,立縣令四課:曰糾正稅籍,團結民兵,勸課農桑,勸勉孝悌。三歲,就緒者加旌賞,無善狀者汰之。



 臣僚上言:「守令之治,其略
 有七:一曰宣詔令,二曰厚風俗,三曰勸農桑,四曰平獄訟,五曰理財賦,六曰興學校,七曰實戶口。得人,則七者皆舉。今之監司,實古刺史。比年守令奸貪,監司未嘗按發,玩弛之弊日甚。」而戶部侍郎張致遠亦言之。乃下詔戒飭監司,考察守令而舉按焉。頃之,有請令江、淮官久任,而課其功過者。帝曰:「朕昔為元帥時,見州縣官以三年為任,猶且一年立威信,二年守規矩,三年則務收人情,以為去計。今止以二年為任,雖有葺治之心,蓋亦無
 暇矣,可如所奏。」是時,歲以十五事考校監司,四善、四最考校縣令,違限不實者有罪。又詔監司,一歲再具所部知縣有無「善政顯著」、「繆懦不職」上之省。



 十三年,詔淮東、京西路州縣,逐考批書,若增添戶口、勸課農桑、增修水利,歲終委監司覆實比較。守臣之條有九,通判之條十有四,令佐而下有差。二十五年,以州縣貪吏為虐,監司、郡守不訶察,遂命監司按郡守之縱容,臺諫劾監司之失察,而每歲校其所按之多寡,以為殿最之課。二十
 七年,校書郎陳俊卿言:「古人各守一官終身,使易地而居,未必盡其能也。今監司、帥守,小州換大州,東路易西路;朝廷百執事,亦往往計日待遷,視所居之官,有如傳舍。望令有政術優異者,或增秩賜金,或待終秩而後遷。使久於其職,察其勤惰而升黜之。庶幾人安其分,而萬事舉矣。」詔三省行之。



 隆興元年,命湖南、北路應守令增闢田疇,自一千頃以下轉磨勘有差,虧者展磨勘、降名次。二年,詔淮南、川峽、京西邊郡守令,能安輯流亡、勸課
 農桑首就緒者,本道監司以聞。乾道二年,廷臣上言:「國朝盛時,有京朝官考課,有幕職、州縣官考課,其後為審官院,為考課院,皆命中書或兩制臣僚校其能否,以施賞罰。望遵故事,應監司郡守朝辭日,別給御前歷子。如薦賢才為幾人,若為治錢穀,若為理獄訟,興某利,除某害,各為條目,使之黽勉從事。每考,令當職官吏從實批書,代還,使藉手陛見,然後詔執事精加考核。其風績有聞者,優與增秩;所蒞無狀者,罰之無赦。則賢者效職,而中
 下之才,亦皆強於為善矣。」帝乃命經筵官參照累朝考課之法,講而行之。



 淳熙二年,因臣僚言,沿邊七路,每路以文臣一人充安撫使以治民,武臣一人充都總管以治兵。分舉其職,各奏其功,任必加久,歲考優劣。一年視其規畫,二年視其成效,三年視其大成,重議誅賞。臧否分為三等:治效顯著者為臧,貪刻庸繆者為否,無功無過者為平。時天子留意黜陟,諸道莫敢不奉承。於是得實者皆增秩升擢,而監司、牧伯舉按稽緩者輒降黜。行
 之十餘年,不免有弊,帝因諭輔臣曰:「臧否亦有喜怒之私,如諸司以為臧,一司以為否,必從眾為公,亦在精擇監司,而以臺諫考察之,庶乎其可也。」光宗初,詔罷其令。



 寧宗以郡國按刺,多徇私情,遂仿舊制,於御史臺別立考課一司,歲終各以能否之實聞於上,以詔升黜。其貪墨、昏懦致臺諫奏劾者,坐監司、郡守以容庇之罪。



 度宗咸淳三年,命參酌舊制,凡文武官一是以公勤、廉恪為主,而又職事修舉,斯為上等,公勤、廉恪各有一長為中
 等,既無廉聲又多繆政者考下等。其要則以御史臺總帥閫、監司,監司總守、倅,守、倅總州縣屬官。餘如戎司及屯軍大壘,則總之制司;或無制司,則並各郡總管、鈐轄並總於帥司。或以諸路所部州郡多寡之數,分隸轉運、提舉、提刑三司。守倅月一考州縣屬官,監司會所隸守倅,制司會戎司、軍壘,遵照舊制互用文移,會其兵甲、獄訟、金穀之數,及各司屬官書擬公事、拘榷錢物、招軍備器之數,次月置冊,各申御史臺上之課籍。俟至半年,
 類考較前三年定為三等,中者無所賞罰,上者或轉官、或減磨勘,下者降官、展磨勘,各有等差。



\end{pinyinscope}