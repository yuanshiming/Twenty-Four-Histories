\article{列傳第十九}

\begin{pinyinscope}

 曹翰楊
 信弟嗣贊黨進李漢瓊劉遇李懷忠米信田重進劉廷翰崔翰



 曹翰,大名人。少為郡小吏,好使氣陵人,不為鄉里所譽。乾祐初,周太祖鎮鄴,與語,奇之,以隸世宗帳下。世宗鎮
 澶淵,署為牙校,入尹開封,留翰在鎮。會太祖寢疾,翰不俟召,歸見世宗,密謂曰:「主上不豫,王為塚嗣,不侍醫藥而決事於外廷,失天下望。」世宗悟,即入侍,以府事屬翰總決。



 及世宗即位,補供奉官,從征高平,參豫謀畫。尋遷樞密承旨,護塞決河。世宗征淮南,留鎧甲千數在正陽,既而得降卒八百,部送歸京師。時翰適從京師來詣,過正陽十數里許遇之,慮劫兵器為叛,矯殺之。及見世宗,具言其事,世宗不悅。翰曰:「賊以困歸我,非心服也,所得
 器甲,盡在正陽,茍為所劫,是復生一淮南矣。」因不之罪。從征瓦橋關,會班師,留知雄州。世宗大漸,諭範質等以王著為相,翰為宣徽使。質以著嗜酒,翰飾詐而專,並寢之。改德州刺史。



 宋初,從征澤、潞,還,改濟州刺史。乾德二年,太祖親征西蜀,移刺均州,澗谷深險,翰令鑿石信道,師旋以濟;詔兼西南諸州轉運使,自石門徑趨歸州,餉運不乏,由夔、萬入會王全斌軍,成都以平。時全師雄擁眾十萬餘據郫縣叛,謀窺成都,翰率兵會劉光毅、曹彬
 等討平之。未幾,軍校呂翰殺武懷節,據嘉州以叛,翰及諸將奪其城。諜知賊約三鼓復來攻,翰戒知更使緩,向晨猶二鼓,賊眾不集而潰,因而破之,劍南遂平。師還,遷蔡州團練使。



 開寶二年,從征太原,復為行營都壕砦使。既班師,會河決澶州,令翰董其役,翰出銀器助役,沉所乘白馬以祭;復決陽武,再護役,皆有成績。將征江南,命翰率兵先赴荊南,改行營先鋒使,進克池州。金陵平,江州軍校胡德、牙將宋德明據城拒命。翰率兵攻之,凡五
 月而陷,屠城無□類,殺兵八百。所略金帛以億萬計,偽言欲致廬山東林寺鐵羅漢像五百頭於京師,因調巨艦百艘,載所得以歸。錄功遷桂州觀察使、判穎州。



 太平興國四年,從征太原,為攻城南面都部署。與崔彥進、李漢瓊、劉遇三節度分部攻城,翰攻東北,而劉遇攻西北,與劉繼元直,城尤險固,遇欲與翰易處,翰言:「觀察使班次下,當部東北。」遇堅欲易之,數日不決。上慮諸將不協,遣諭翰曰:「卿智勇無雙,西北面非卿不能當也。」翰乃奉
 詔,築土山瞰城中,數日而就,繼元甚恐。軍中乏水,城西十餘里谷中有娘子廟,翰往禱之,穿渠得水,人馬以給。又從征幽州,率所部攻城東南隅,卒掘土得蟹以獻。翰謂諸將曰:「蟹水物而陸居,失所也。且多足,彼援將至,不可進拔之象,況蟹者解也,其班師乎?」已而果驗。



 五年,從幸大名,拜威塞軍節度,仍判穎州,復命為幽州行營都部署。詔督役開南河,自雄達莫,以通漕運,議築大堤以捍之。翰遣徒數萬,伐巨木於漢境,遣騎五,授五色旗為
 斥候,前遇丘陵、水澤、寇賊、煙火,則各舉其旗以為應,又起烽燧於境上,敵疑不敢近塞,得巨木數萬以濟用,訖事歸鎮。



 翰在郡歲久,徵斂苛酷,政因以弛。上以其有功,每優容之。會汝陰令孫崇望詣闕,訴翰私市兵器,所為多不法。詔遣御史滕中正乘傳鞫之,獄具,當棄市,上貸其罪,削官爵,流錮登州。雍熙二年,起為右千牛衛大將軍、分司西京。四年,召入為左千牛衛上將軍,賜錢五百萬,白金五千兩。淳化三年,卒,年六十九,贈太尉。上命遷
 其四子守謙、守能、守節、守貴官,其六子守讓、守贄、守澄、守恩、守英、守吉皆補殿直。



 翰陰狡多智數,好誇誕,貪冒貨賂,飲酒至數斗不亂。每奏事上前,雖數十條,皆默識不少差。嘗作《退將詩》曰:「曾因國難披金甲,恥為家貧賣寶刀。」翰直禁日,因語及之。上憫其意,故有銀錢之賜。咸平元年,賜謚武毅。



 楊信,瀛州人。初名義。顯德中,隸太祖麾下為裨校。宋初,權內外馬步軍副都軍頭。建隆二年,領賀州刺史。改鐵
 騎、控鶴都指揮使,遷殿前都虞候,領漢州防禦使。乾德初,親郊,為儀仗都部署。四年,信病喑,上幸其第,賜錢二百萬。五年,改靜江軍節度。開寶二年,散指揮都知杜廷進等將為不軌,謀洩,夜啟玄武門,召信逮捕,遲明,十九人皆獲,上親訊而誅之。六年,遷殿前都指揮使,改領建武軍節度。



 太祖嘗令御龍直習水戰於後池,有鼓噪聲,信居玄武門外,聞之,遽入,服皂綈袍以見。上謂曰:「吾教水戰爾,非有他也。」出,上目送之,謂左右曰:「真忠臣也。」九
 年,授義成軍節度。太平興國二年,改鎮寧軍,並領殿前都指揮使。三年春,以瘍疾在告,俄卒,贈侍中。



 信雖喑疾而質實自將,善部分士卒,指顧申儆,動有紀律,故見信任,而終始無疑焉。有童奴田玉者,能揣度其意,每上前奏事,及與賓客談論,或指揮部下,必回顧玉,書掌為字,玉因直達其意無失。信未死前一日,喑疾忽愈,上聞而駭之,遽幸其第。信自言遭遇兩朝,恩寵隆厚,敘謝感慨,涕泗橫集。上加慰勉,錫賚有差。信弟嗣、贊。



 嗣,建隆初以
 信薦為殿直,三遷崇儀副使、大山軍監軍。雍熙四年,就命知軍事。代還,以吏民借留再任,俄遷高陽關戰棹都監。淳化二年,改知保州,門無私謁。轉運使言其治狀,優遷威虜軍,改崇儀使,與曹思進同為靜戎軍、保州、長城、蒲城緣邊都巡檢使。改如京使,再知保州,有戰功。



 真宗即位,加洛苑使。咸平初,領獎州刺史。三年,與敵人戰於廉良,斬首二千級,獲戰馬輜重甚眾,以功真拜保州刺史。召還,授本州團練使。時楊延昭方為刺史,嗣言:「嘗與
 延昭同官,驟居其上,不可,願守舊官。」上嘉其讓,乃遷延昭官。嗣與延昭久居北邊,俱以善戰聞,時謂之「二楊」。嗣以武人治郡,不屑細務,又兼領巡徼,在郡日少,城堞圮壞,有未葺者,詔供備庫副使趙彬代之,改深州團練都巡檢使兼保州鈐轄。



 五年,邊人寇保州,嗣與楊延昭御之,部伍不整,為所襲。士馬多亡失,代還,特宥其罪。明年,與防秋之策,條陳北面利害,以其練達邊事,出為鎮、定、高陽關三路後陣鈐轄,移定州副都部署,留其家京師,
 假官第以居。



 景德初,改鎮州路副都部署。上以嗣耄年總軍政,慮有廢闕,旋命代之。連為趙、貝深三州部署。大中祥符五年,復出為天雄軍副都部署。六年,以左龍武大將軍致仕。明年卒,年八十一。錄其子承憲為侍禁。



 贊稍知書,無異能,以兄故得掌禁旅,累資朝著至牧守焉。



 黨進,朔州馬邑人。幼給事魏帥杜重威,重威愛其淳謹,及壯,猶令與姬妾雜侍。重威敗,進以膂力隸軍伍。周廣順初,補散指揮使,累遷鐵騎都虞候。宋初,轉本軍都校、
 領欽州刺史,遷馬步軍副都軍頭、領虔州團練使,改虎捷右廂都指揮使、領睦州防禦使。建隆二年,改領閬州。乾德初,改龍捷左廂都虞候、領利州觀察使。後四年,權步軍。杜審瓊卒,命進代領其務。五年,領彰信軍節度兼侍衛步軍都指揮使。



 開寶元年,將征太原,以進將河東行營前軍。開寶二年,太祖師臨晉陽、置砦四面,命進主其東偏。師未成列,太原驍將楊業領突騎數百來犯,進奮身從數人逐業;業急入隍中,會援兵至,緣縋入城獲
 免。上激賞之。六年,改侍衛馬軍都指揮使、領鎮安軍節度。九年,又命將河東行營兵征太原,入其境,敗太原軍於城北。太祖崩,召還。太平興國二年,出為忠武軍節度。在鎮歲餘,一日自外歸,有大蛇臥榻上寢衣中,進怒,烹食之。遇疾卒,年五十一,贈侍中。



 進出戎行,形貌魁岸,居常恂恂,每擐甲冑,毛發皆豎。進名進,自稱曰暉,人問之,則曰:「吾欲從吾便耳。」先是,禁中軍校,自都虞候已上,悉書所掌兵數於梃上,如笏記焉。太祖一日問進所掌幾
 何,進不識字,但舉梃以示於上曰:「盡在是矣。」上以其樸直,益厚之。嘗受詔巡京師,聞里間有畜養禽獸者,見必取而縱之,罵曰:「買肉不將供父母,反以飼禽獸乎。」太宗嘗令親吏臂鷹雛於市,進亟欲放之,吏曰:「此晉王鷹也。」進乃戒之曰:「汝謹養視。」小民傳以為笑,其變詐又如此。杜重威子孫有貧困者,進分月俸給之,士大夫或有愧焉。子崇義閑廄使,崇貴閣門祗候。



 李漢瓊,河南洛陽人。曾祖裕,祁州刺史。漢瓊體質魁岸,
 有膂力。晉末,補西班衛士,遷內殿直。周顯德中,從征淮南,先登,遷龍旗直副都知,改左射指揮使。宋初,再遷鐵騎第二軍都校、領饒州刺史,遷控鶴左廂都校、領瀘州刺史,改澄州團練使,轉虎捷左廂都指揮使、領融州防禦使,遷侍衛馬軍都虞候、領洮州觀察使。



 王師征江南,命領行營騎軍兼戰棹左廂都指揮使,自蘄春攻岐口砦,斬首數千級,獲樓船數百艘,沿流拔池州,破銅陵,取當塗,作浮梁於牛渚以濟大軍。分圍金陵,率所部度秦
 淮,取巨艦實葦其中,縱火攻其水砦,拔之。江南平,以功領振武軍節度。



 太平興國二年,出為彰德軍節度。四年,太宗親征太原,改攻城都部署。漢瓊與牛思進主攻城南偏,漢瓊先登,矢集其腦,並中指,傷甚猶力疾戰。上召至幄殿,賜良藥以慰勞之。先是,攻城者以牛革冒木上,士卒蒙之而進,謂之洞子。上欲幸其中,以勞士卒,漢瓊極諫,以為矢石之下,非萬乘之尊所宜輕往,上乃止。太原平,改鎮州兵馬鈐轄。



 契丹數萬騎寇中山,漢瓊與戰
 於蒲城,大敗之,逐至遂城,俘斬萬計,加檢校太尉。車駕幸大名,漢瓊上謁,陳邊事稱旨,命為滄州都部署,加賜戰馬、金甲、寶劍、戎具以寵之。六年,以病還京,賜白金萬兩,月餘卒,年五十五,贈中書令。



 漢瓊性木強,使酒難近,然善戰有功。無子,弟漢贇、漢彬。太平興國初,漢贇補供奉官,嘗監高陽關、平戎軍,乘傳衢、婺二州,捕劇賊程白眉數十人,悉殲焉。累仕崇儀使、知寧州,大中祥符七年卒。漢彬至禮賓副使。



 劉遇,滄州清池人。少魁梧有膂力。周祖鎮大名,隸帳下。廣順初,補控鶴都頭,改副指揮使。宋初,遷御馬直指揮使,俄領漢州刺史,改領眉州。累遷控鶴右廂都指揮使、領瓊州團練使。從征太原,以功遷虎捷右廂,改領蔚州防禦使。開寶六年,轉侍衛步軍都虞候、領洮州觀察使。征江南,領步軍戰棹都指揮使。時吳兵三萬屯皖口,遇會諸路兵破之,擒其將朱令贇、王暉等,獲戎器數萬,金陵以平,錄功加領大同軍節度。車駕雩祀西洛,命率禁
 衛以從。



 太平興國二年,出為彰信軍節度。四年,徵太原,與史珪攻城北面,平之。進攻範陽,師還,坐所部失律,責授宿州觀察使。五年,從幸大名,復保靜軍節度、幽州行營都部署,護築保州、威虜、靜戎、平塞、長城五城。八年,徙鎮滑州。晨興方對客,足有灸瘡痛,其醫謂火毒不去,故痛不止。遇即解衣,取刀割瘡至骨,曰:「火毒去矣。」談笑如常時,旬餘乃差。遇性淳謹,待士有禮,尤善射,太宗待之甚厚。雍熙二年,卒,年六十六,贈侍中,歸葬京師。



 李懷忠,涿州範陽人。初名懷義。太祖掌禁兵時,隸帳下為散都頭,累遷殿前都指揮使、都虞候、領開州刺史。乾德中,授東西班都指揮使,改領富州。開寶中,從太祖征晉陽,累月未下。會盛暑,欲班師以休息士卒,懷忠謂:「賊嬰孤城,內無儲峙,外無援兵,其勢危困,若急攻之,破在旦夕,臣願奮銳為士卒先。」會大熱,戰不利,懷忠中流矢,力疾戰益奮。還授散指揮使,遷富州團練使,改日騎左右廂都指揮使。



 上幸西京,愛其地形勢得天下中正,有
 留都之意。懷忠乘間進曰:「東京有汴渠之漕,歲致江、淮米數百萬斛,禁衛數十萬人仰給於此,帑藏重兵皆在焉。根本安固已久,一旦遽欲遷徙,臣實未見其利。」上嘉納之。



 太宗即位,改領本州防禦使,稍遷侍衛步軍都虞候、領大同軍節度。三年,改步軍都指揮使,五月,卒,贈侍中。錄其子紹宗等三人為供奉官。大中祥符三年,又錄其子德鈞為借職。



 米信,舊名海進,本奚族,少勇悍,以善射聞。周祖即位,隸
 護聖軍。從世宗征高平,以功遷龍捷散都頭。太祖總禁兵,以信隸麾下,得給使左右,遂委心焉,改名信,署牙校。及即位,補殿前指揮使,遷直長。平揚州日,信執弓矢侍上側,有游騎將迫乘輿,射之,一發而斃。遷內殿直指揮使。開寶元年,改殿前指揮使、領郴州刺史。



 太宗即位,轉散都頭指揮使,繼領高州團練使。太平興國三年,遷領洮州觀察使。四年,徵太原,命為行營馬步軍指揮使,與田重進分督行營諸軍。並人潛師來犯,信擊敗之,殺其
 將裴正。並州平,遂移兵攻範陽。師還,以功擢保順軍節度使。時信族屬多在塞外,會其兄子全自朔州奮身來歸,召見,俾乘傳詣代州,伺間迎致其親屬,發勁卒護送之。既而全宿留逾年,邊境斥候嚴,竟不能致。信慷慨嘆曰:「吾聞忠孝不兩立,方思以身徇國,安能復顧親戚哉。」北望號慟,戒子侄勿復言。五年,命與郭守贇等同護定州屯兵。六年秋,遷定州駐泊部署。八年,改領彰化軍節度使。



 雍熙三年,徵幽薊,命信為幽州西北道行營馬步
 軍都部署,敗契丹於新城。契丹率眾復來戰,王師稍卻,信獨以麾下龍衛卒三百御敵,敵圍之數重,矢下如雨,信射中數人,麾下士多死。會暮,信持大刀,率從騎大呼,殺數十人,敵遂小卻,信以百餘騎突圍得免。坐失律,議當死,詔特原之,責授右屯衛大將軍。明年,復授彰武軍節度。



 端拱初,詔置方田,以信為邢州兵馬都部署以監之。二年,改鎮橫海軍。信不知書,所為多暴橫,上命何承矩為之副,以決州事。及承矩領護屯田,信遂專恣不法,
 軍人宴犒甚薄,嘗私市絹附上計吏,稱官物以免關征,上廉知之。四年,召為右武衛上將軍。明年,判左右金吾街仗事。未逾月,吏卒以無罪被捶撻者甚眾。強市人物,妻死買地營葬,妄發居民塚墓。家奴陳贊老病,棰之致死,為其家人所告。下御史鞫之,信具伏。獄未上而卒,年六十七。贈橫海軍節度。子繼豐,內殿崇班、閣門祗候。



 田重進,幽州人。形質奇偉,有武力。周顯德中,應募為卒,隸太祖麾下。從征契丹,至陳橋還,遷御馬軍使,積功至
 瀼州刺史。太平興國四年,從征太原還,錄功擢為天德軍節度使。六年,改侍衛步軍指揮使。八年,改領靜難軍節度使。九年,河決滑州韓、房村,重進總護其役,以劉吉為之副,河遂塞。



 雍熙中,出師北征,重進率兵傅飛狐城下,用袁繼忠計,伏兵飛狐南口,擒契丹驍將大鵬翼及其監軍馬贇、副將何萬通並渤海軍三千餘人,斬首數千級,俘獲以萬計,逐北四十里,連下飛狐、靈州等城。進攻蔚州,其牙校李存璋等殺酋帥蕭啜理、執耿紹忠,率
 吏民來附。會曹彬之師不利,乃命重進董師駐定州,遷定州駐泊兵馬都部署。三年,率師入遼境,攻下岐溝關,殺守城兵千餘及獲牛馬輜重以還。四年春,改彰信軍節度。



 淳化三年,改真定尹、成德軍節度。未幾,移京兆尹、永興軍節度。五年,改知延州,復還鎮。至道三年,卒,年六十九,贈侍中。



 重進不事學,太宗居藩邸時,愛其忠勇,嘗遺以酒炙不受,使者曰:「此晉王賜也,何為不受?」重進曰:「為我謝晉王,我知有天子爾。」卒不受。上知其忠樸,故終
 始委遇焉。子守信六宅使,守吉閣門祗候。



 劉廷翰,開封浚儀人。父紹隱,後唐末隸兵籍。晉天福中,以隊長戍魏博。範延光反,紹隱力戰死焉。周世宗鎮澶淵,廷翰以膂力隸帳下;即位,補殿前指揮使,累從征伐,以戰功再遷至散指揮第一直都知。



 宋初,預平上黨、維揚,遷鐵騎都指揮使、領廉州刺史。太宗即位,遷右廂都指揮使、領本州團練使,遷雲州觀察使。太平興國四年,從征太原,領鎮州駐泊都鈐轄。



 太宗北伐,既班師,上以
 邊備在於得人,乃命廷翰、李漢瓊率兵屯真定,崔彥進屯關南,崔翰屯定州。冬,契丹果縱兵南侵。廷翰先陣於徐河,彥進率師出黑蘆堤北,銜枚躡契丹後,崔翰、漢瓊兵繼至,合擊之,大敗其眾於滿城。廷翰以功領大同軍節度、殿前都虞候。八年,改領彰信軍節度。雍熙四年春,改鎮滑、邢。端拱中,鎮州駐泊馬步軍都部署郭守文卒,上特命廷翰代之。淳化三年,改大名尹、天雄軍節度。三年,以病求解官,還闕,上親臨問,賜賚有加。未幾卒,年七
 十,贈侍中。



 廷翰自衛士至上將,頗以武勇自任,寬厚容眾,雖不事威嚴,而長於御下。為殿前都指揮使,入朝,常行眾中,每歷宮殿門,少識之者。嘗與郊祀恩,當追封三世,廷翰少孤,其大父以上皆不逮事,忘其家諱,上為撰名親書賜之。子贊元,宮苑使、澄州刺史;贊明,皇城使、勤州團練使。



 崔翰,字仲文,京兆萬年人。少有大志,風姿偉秀,太祖見而奇之,以隸麾下。從周世宗征淮南,平壽春,取關南,以
 功補軍使。宋初,遷御馬直副指揮使,從征澤、潞。開寶初,遷河東降民以實陜西地,晉人勇悍,多習武藝,命翰差擇之。及閱試河北鎮兵,取其驍果者以分配天武兩軍。九年,領端州刺史。



 太宗即位,進本州團練使。太平興國二年秋,議武於西郊,時殿前都指揮使楊信病喑,命翰代之。翰分布士伍,南北綿互二十里,建五色旗號令,將卒望其所舉,以為進退,六師周旋如一。上御臺臨觀,大悅,以藩邸時金帶賜之,謂左右曰:「晉朝之將,必無如崔
 翰者。」



 四年,從征太原,命總侍衛馬步諸軍,率先攻城,流矢中其頰,神色不變,督戰益急,上即軍帳撫問之。太原平,時上將有事幽薊,諸將以為晉陽之役,師罷餉匱,劉繼元降,賞賚且未給,遽有平燕之議,不敢言。翰獨奏曰:「所當乘者勢也。不可失者時也,取之易。」上謂然,定議北伐。既而班師,命諸將整暇以還。至金臺驛,大軍南向而潰,上令翰率衛兵千餘止之。翰請單騎往,至則諭以師律,眾徐以定,不戮一人。既復命,上喜,因命知定州,得以
 便宜從事,緣邊諸軍並受節制,軍市租儲,得以專用。



 冬,契丹兵數萬寇蒲城,翰會李漢瓊兵於徐河,河陽節度崔彥進兵自高陽關繼至,因合擊之。契丹投西山坑谷中死者不可勝計,俘馘數萬,所獲他物又十倍焉。以功擢武泰軍節度使。



 初,劉繼元降,上令翰往撫慰,俘略無得出城。時秦王廷美以數十騎將冒禁出,翰呵止之。至是,構於上。明年夏,出為感德軍節度使。至鎮時,盜賊充斥,翰誘其渠魁,戒以禍福,群盜感悟,散歸農畝,境內肅
 然。



 雍熙二年,移知滑州。三年,北伐不利,上追念徐河之功,召翰為威虜軍行營兵馬都部署。四年春,改鎮定國軍。二年,移鎮鎮安軍。淳化三年召還,以疾留京師。稍間,入見上曰:「臣既以身許國,不願死於家,得以馬革裹尸足矣。」上壯之,復令赴鎮,月餘卒,年六十三,贈侍中。



 翰驍勇有謀,所至多立功。輕財好施,死之日家無餘貲。晚年酷信釋氏。子繼顒,虞部員外郎。孫承業,內殿承制、閣門祗候;承祐,內殿崇班。



 論曰:自曹翰而下,嘗任將帥居節鎮者凡十人,其初率由拳勇起家戎行,雖不事問學,而皆精白一心,以立事功。始終匹休,而無韓、彭之禍者,由制御保全之有道也。楊信以篤實,重進以忠樸,劉遇以淳謹,廷翰以武勇稱,故皆終始委遇而不替。漢瓊雖木強使酒,米信所為雖多暴橫,黨進恂恂類懷奸詐,懷忠論遷似昧大體;然以徵太原、平江南、戰徐河觀之,皆不害其為驍果也。至於好謀善戰,輕財好施,所至立功,則未有優於曹翰、崔翰
 者也。然不可與古之良將同日而語者,崔之論奏平燕,未免出於率爾;而曹之殺降卒,屠江州,則又過於忍者也。君子謂功莫優於二子,而過亦莫先於二子,信矣



\end{pinyinscope}