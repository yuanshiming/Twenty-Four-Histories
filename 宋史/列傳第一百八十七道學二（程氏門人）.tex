\article{列傳第一百八十七道學二(程氏門人)}

\begin{pinyinscope}

 ○劉絢李籲謝良佐游酢張繹蘇昞尹焞楊時羅從彥李侗



 劉絢字質夫,常山人。以蔭為壽安主簿、長子令,督公家逋賦,不假鞭撲而集。歲大旱,府遣吏視傷所,蠲財什二,絢力爭不得,封還其楬,請易之。富弼嘆曰:「真縣令也。」元祐初,韓維薦其經明行修,為京兆府教授。王巖叟、朱光庭又薦為太學博士,卒於官。絢力學不倦,最明於《春秋》。程顥每為人言:「他人之學,敏則有矣,未易保也,若絢者,吾無疑焉。」



 李籲字端伯,洛陽人。登進士第。元祐中為秘書省校書
 郎,卒。程頤謂其才器可以大受,及亡也,祭之以文曰:「自予兄弟倡明道學,能使學者視仿而信從者,籲與劉絢有焉。」



 謝良佐字顯道,壽春上蔡人。與游酢、呂大臨、楊時在程門,號「四先生」。登進士第。建中靖國初,官京師,召對,忤旨去。監西京竹木場,坐口語系詔獄,廢為民。良佐記問該贍,對人稱引前史,至不差一字。事有未徹,則顙有泚。與程頤別一年,復來見,問其所進,曰:「但去得一『矜』字爾。」頤
 喜,謂朱光庭曰:「是子力學,切問而近思者也。」所著《論語說》行於世。



 游酢字定夫,建州建陽人。與兄醇以文行知名,所交皆天下士。程頤見之京師,謂其資可以進道。程顥興扶溝學,招使肄業,盡棄其學而學焉。第進士,調蕭山尉。近臣薦其賢,召為太學錄。遷博士,以奉親不便,求知河清縣。范純仁守潁昌府,闢府教授。純仁入相,復為博士。簽書齊州、泉州判官。晚得監察御史,歷知漢陽軍、和舒濠三
 州而卒。



 張繹字思叔,河南壽安人。家甚微,年長未知學,傭力於市,出聞邑官傳呼聲,心慕之,問人曰:「何以得此?」人曰:「此讀書所致爾。」即發憤力學,遂以文名。預鄉里計偕,謂科舉之習不足為,嘗游僧舍,見僧道楷,將祝發從之。時周行己官河南,警之曰:「何為舍聖人之學而學佛?異日程先生歸,可師也。」會程頤還自涪,乃往受業,頤賞其穎悟。讀《孟子》「志士不忘在溝壑,勇士不忘喪其元」,慨然若有
 得。未及仕而卒。頤嘗言「吾晚得二士」,謂繹與尹焞也。



 蘇昞字季明,武功人。始學於張載,而事二程卒業。元祐末,呂大中薦之,起布衣為太常博士。坐元符上書入邪籍,編管饒州,卒。



 尹焞字彥明,一字德充,世為洛人。曾祖仲宣七子,而二子有名:長子源字子漸,是謂河內先生;次子洙字師魯,是謂河南先生。源生林,官至虞部員外郎。林生焞。



 少師事程頤,嘗應舉,發策有誅元祐諸臣議,焞曰:「噫,尚可以
 干祿乎哉!」不對而出,告頤曰:「焞不復應進士舉矣。」頤曰:「子有母在。」焞歸告其母陳,母曰:「吾知汝以善養,不知汝以祿養。」頤聞之曰:「賢哉母也!」於是終身不就舉。焞之從師,與河南張繹同時,繹以高識,焞以篤行。頤既沒,焞聚從洛中,非吊喪問疾不出戶,士大夫宗仰之。



 靖康初,種師道薦焞德行可備勸講,召至京師,不欲留,賜號和靖處士。戶部尚書梅執禮、御史中丞呂好問、戶部侍郎邵溥、中書舍人胡安國合奏:「河南布衣尹焞學窮根本,德
 備中和,言動可以師法,器識可以任大,近世招延之士無出其右者。朝廷特召,而命處士以歸,使焞韜藏國器,不為時用,未副陛下側席求賢之意。望特加識擢,以慰士大夫之望。」不報。



 次年,金人陷洛,焞闔門被害,焞死復蘇,門人舁置山谷中而免。劉豫命偽帥趙斌以禮聘焞,不從則以兵恐之。焞自商州奔蜀,至閬,得程頤《易傳》十卦於其門人呂稽中,又得全本於其婿邢純,拜而受之。紹興四年,止於涪。涪,頤讀《易》地也,闢三畏齋以居,邦人
 不識其面。侍讀範沖舉焞自代,授左宣教郎,充崇政殿說書,以疾辭。範沖奏給五百金為行資,遣漕臣奉詔至涪親遣。六年,始就道,作文祭頤而後行。



 先是,崇寧以來,禁錮元祐學術,高宗渡江,始召楊時置從班,召胡安國居給舍,範沖、朱震俱在講席,薦焞甚力。既召,而左司諫公輔上疏攻程氏之學,乞加屏絕。



 焞至九江,上奏曰:「臣僚上言,程頤之學惑亂天下。焞實師頤垂二十年,學之既專,自信甚篤。使焞濫列經筵,其所敷繹,不過聞於
 師者。舍其所學,是欺君父,加以疾病衰耗,不能支持。」遂留不進。胡安國奉祠居衡陽,上書言:「欲使學者蹈中庸,師孔、孟,而禁不從程頤之學,是入室而不由戶。」



 朱震引疾告去,時趙鼎去位,張浚獨相,於是召安國,俾以內祠兼侍讀,而上章薦焞,言其拒劉豫之節,且謂其所學所養有大過人者,乞令江州守臣疾速津送至國門。復以疾辭,上曰:「焞可謂恬退矣。」詔以秘書郎兼說書,趣起之,焞始入見就職。八年,除秘書少監,未幾,力辭求去。上語
 參知政事劉大中曰:「焞未論所學淵源,足為後進矜式,班列得老成人,亦是朝廷氣象。」乃以焞直徽猷閣,主管萬壽觀,留侍經筵。資善堂翊善朱震疾亟,薦焞自代。輔臣入奏,上慘然曰:「楊時物故,胡安國與震又亡,朕痛惜之。」趙鼎曰:「尹焞學問淵源,可以繼震。」上指奏牘曰:「震亦薦焞代資善之職,但焞微聵,恐教兒費力爾。」除太常少卿,仍兼說書。未幾,稱疾在告,除權禮部侍郎兼侍講。



 時金人遣張通古、蕭哲來議和,焞上疏曰:



 臣伏見本朝有
 遼、金之禍,亙古未聞,中國無人,致其猾亂。昨者城下之戰,詭詐百出,二帝北狩,皇族播遷,宗社之危,已絕而續。陛下即位以來十有二年,雖中原未復,仇敵未殄,然而賴祖宗德澤之厚,陛下勤撫之至,億兆之心無有離異。前年徽宗皇帝、寧德皇后崩問遽來,莫究不豫之狀,天下之人痛心疾首,而陛下方且屈意降志,以迎奉梓宮、請問諱日為事。今又為此議,則人心日去,祖宗積累之業,陛下十二年勤撫之功,當決於此矣。不識陛下亦嘗
 深謀而熟慮乎,抑在廷之臣不以告也?



 《禮》曰:「父母之仇不共戴天,兄弟之仇不反兵。」今陛下信仇敵之譎詐,而覬其肯和以紓目前之急,豈不失不共戴天、不反兵之義乎?又況使人之來,以詔諭為名,以割地為要,今以不戴天之仇與之和,臣切為陛下痛惜之。或以金國內亂,懼我襲己,故為甘言以緩王師。倘或果然,尤當鼓士卒之心,雪社稷之恥,尚何和之為務?



 又移書秦檜言:



 今北使在廷,天下憂憤,若和議一成,彼日益強,我日益怠,侵
 尋朘削,天下有被發左衽之憂。比者,竊聞主上以父兄未返,降志辱身於九重之中有年矣,然亦自是未聞金人悔過,還二帝於沙漠。繼之梓宮崩問不詳,天下之人痛恨切骨,金人狼虎貪噬之性,不言可見。天下方將以此望於相公,覬有以革其已然,豈意為之已。甚乎。



 今之上策,莫如自治。自治之要,內則進君子而遠小人,外則賞當功而罰當罪,使主上孝弟通於神明,道德成於安強,勿以小智孑義而圖大功,不勝幸甚。



 疏及書皆不報,
 於是焞固辭新命。



 九年,以徽猷閣待制提舉萬壽觀兼侍講,又辭,且奏言:



 臣職在勸講,蔑有發明,期月之間,病告相繼,坐竊厚祿,無補聖聰。先聖有言:「陳力就列,不能者止。」此當去者一也。臣起自草茅,誤膺召用,守道之語,形於訓詞,而臣貪戀寵榮,遂移素守,使朝廷非常不次之舉,獲懷利茍得之人。此當去者二也。比嘗不量分守,言及國事,識見迂陋,已驗於今,跡其庸愚,豈堪時用。此當去者三也。臣自擢春官,未嘗供職,以疾乞去,更獲超遷,
 有何功勞,得以祗受。此當去者四也。國朝典法,揆之禮經,年至七十,皆當致仕。今臣年齒已及,加以疾病,血氣既衰,戒之在得。此當去者五也。臣聞聖君有從欲之仁,匹夫有莫奪之志,今臣有五當去之義,無一可留之理,乞檢會累奏,放歸田里。



 疏上,以焞提舉江州太平觀。引年告老,轉一官致仕。



 焞自入經筵,即乞休致,朝廷以禮留之;浚、鼎既去,秦檜當國,見焞議和疏及與檜書已不樂,至是,得求去之疏,遂不復留。十二年,卒。



 當是時,學於程
 頤之門者固多君子,然求質直弘毅、實體力行若焞者蓋鮮。頤嘗以「魯」許之,且曰:「我死,而不失其正者尹氏子也。」其言行見於《涪陵記善錄》為詳,有《論語解》及《門人問答》傳於世。



 楊時字中立,南劍將樂人。幼穎異,能屬文,稍長,潛心經史。熙寧九年,中進士第。時河南程顥與弟頤講孔、孟絕學於熙、豐之際,河、洛之士翕然師之。時調官不赴,以師禮見顥於潁昌,相得甚歡。其歸也,顥目送之曰:「吾道南
 矣。」四年而顥死,時聞之,設位哭寢門,而以書赴告同學者。至是,又見程頤於洛,時蓋年四十矣。一日見頤,頤偶瞑坐,時與游酢侍立不去,頤既覺,則門外雪深一尺矣。關西張載嘗著《西銘》,二程深推服之,時疑其近於兼愛,與其師頤辨論往復,聞理一分殊之說,始豁然無疑。



 杜門不仕者十年,久之,歷知瀏陽、餘杭、蕭山三縣,皆有惠政,民思之不忘。張舜民在諫垣,薦之,得荊州教授。時安於州縣,未嘗求聞達,而德望日重,四方之士不遠千里從
 之游,號曰龜山先生。



 時天下多故,有言於蔡京者,以為事至此必敗,宜引舊德老成置諸左右,庶幾猶可及,時宰是之。會有使高麗者,國主問龜山安在,使回以聞。召為秘書郎,遷著作郎。及面對,奏曰:



 堯、舜曰「允執厥中」,孟子曰「湯執中」,《洪範》曰「皇建其有極」,歷世聖人由斯道也。熙寧之初,大臣文六藝之言以行其私,祖宗之法紛更殆盡。元祐繼之,盡復祖宗之舊,熙寧之法一切廢革。至紹聖、崇寧抑又甚焉,凡元祐之政事著在令甲,皆焚之
 以滅其跡。自是分為二黨,縉紳之禍至今未殄。臣願明詔有司,條具祖宗之法,著為綱目,有宜於今者舉而行之,當損益者損益之,元祐、熙、豐姑置勿問,一趨於中而已。



 朝廷方圖燕云,虛內事外,時遂陳時政之弊,且謂:「燕雲之師宜退守內地,以省轉輸之勞,募邊民為弓弩手,以殺常勝軍之勢。」又言:「都城居四達之衢,無高山巨浸以為阻衛,士人懷異心,緩急不可倚仗。」執政不能用。登對,力陳君臣警戒,正在無虞之時,乞為《宣和會計錄》,以
 周知天下財物出入之數。徽宗首肯之。



 除邇英殿說書。聞金人入攻,謂執政曰:「今日事勢如積薪已然,當自奮勵,以竦動觀聽。若示以怯懦之形,委靡不振,則事去矣。昔汲黯在朝,淮南寢謀。論黯之才,未必能過公孫弘輩也,特其直氣可以鎮壓奸雄之心爾。朝廷威望弗振,使奸雄一以弘輩視之,則無復可為也。要害之地,當嚴為守備,比至都城,尚何及哉?近邊州軍宜堅壁清野,勿與之戰,使之自困。若攻戰略地,當遣援兵追襲,使之腹背
 受敵,則可以制勝矣。」且謂:「今日之事,當以收人心為先。人心不附,雖有高城深池、堅甲利兵,不足恃也。免夫之役,毒被海內,西城聚斂,東南花石,其害尤甚。前此蓋嘗罷之,詔墨未幹,而花石供奉之舟已銜尾矣。今雖復申前令,而禍根不除,人誰信之?欲致人和,去此三者,正今日之先務也。」



 金人圍京城,勤王之兵四集,而莫相統一。時言:「唐九節度之師不立統帥,雖李、郭之善用兵,猶不免敗衄。今諸路烏合之眾,臣謂當立統帥,一號令,示紀
 律,而後士卒始用命。」又言:「童貫為三路大帥,敵人侵疆,棄軍而歸,孥戮之有餘罪,朝廷置之不問,故梁方平、何灌皆相繼而遁。當正典刑,以為臣子不忠之戒。童貫握兵二十餘年,覆軍殺將,馴至今日,比聞防城仍用閹人,覆車之轍,不可復蹈。」疏上,除右諫議大夫兼侍講。



 敵兵初退,議者欲割三鎮以講和,時極言其不可,曰:「河朔為朝廷重地,而三鎮又河朔之要藩也。自周世宗迄太祖、太宗,百戰而後得之,一旦棄之北庭,使敵騎疾驅,貫吾
 腹心,不數日可至京城。今聞三鎮之民以死拒之,三鎮拒其前,吾以重兵躡其後,尚可為也。若種師道、劉光世皆一時名將,始至而未用,乞召問方略。」疏上,欽宗詔出師,而議者多持兩端,時抗疏曰:「聞金人駐磁、相,破大名,劫虜驅掠,無有紀極,誓墨未幹,而背不旋踵,吾雖欲專守和議,不可得也。夫越數千里之遠,犯人國都,危道也。彼見勤王之師四面而集,亦懼而歸,非愛我而不攻。朝廷割三鎮二十州之地與之,是欲助寇而自攻也。聞肅
 王初與之約,及河而返,今挾之以往,此敗盟之大者。臣竊謂朝廷宜以肅王為問,責其敗盟,必得肅王而後已。」時太原圍閉數月,而姚古擁兵逗留不進,時上疏乞誅古以肅軍政,拔偏裨之可將者代之。不報。



 李綱之罷,太學生伏闕上書,乞留綱與種師道,軍民集者數十萬,朝廷欲防禁之。吳敏乞用時以靖太學,時得召對,言:「諸生伏闕紛紛,忠於朝廷,非有他意,但擇老成有行誼者,為之長貳,則將自定。」欽宗曰:「無逾於卿。」遂以時兼國子祭
 酒。首言:「三省政事所出,六曹分治,各有攸司。今乃別闢官屬,新進少年,未必賢於六曹長貳。」又言:



 蔡京用事二十餘年,蠹國害民,幾危宗社,人所切齒,而論其罪者,莫知其所本也。蓋京以繼述神宗為名,實挾王安石以圖身利,故推尊安石,加以王爵,配饗孔子廟庭。今日之禍,實安石有以啟之。



 謹按安石挾管、商之術,飭六藝以文奸言,變亂祖宗法度。當時司馬光已言其為害當見於數十年之後,今日之事,若合符契。其著為邪說以塗學
 者耳目,而敗壞其心術者,不可縷數,姑即一二事明之。



 昔神宗嘗稱美漢文惜百金以罷露臺,安石乃言:「陛下若能以堯、舜之道治天下,雖竭天下以自奉不為過,守財之言非正理。」曾不知堯、舜茅茨土階。禹曰:「克儉於家」,則竭天下以自奉者,必非堯、舜之道。其後王黼以應奉花石之事,竭天下之力,號為享上,實安石有以倡之也。其釋《鳧鷖》守成之詩,於末章則謂:「以道守成者,役使群眾,泰而不為驕,宰制萬物,費而不為侈,孰弊弊然以愛
 為事。」《詩》之所言,正謂能持盈則神祇祖考安樂之,而無後艱爾。自古釋之者,未有泰而不為驕、費而不為侈之說也。安石獨倡為此說,以啟人主之侈心。後蔡京輩輕費妄用,以侈靡為事。安石邪說之害如此。



 伏望追奪王爵,明詔中外,毀去配享之像,使邪說淫辭不為學者之惑。疏上,安石遂降從祀之列。士之習王氏學取科第者,已數十年,不復知其非,忽聞以為邪說,議論紛然。諫官馮澥力主王氏,上疏詆時。會學官中有紛爭者,有旨學
 官並罷,時亦罷祭酒。



 時又言:「元祐黨籍中,惟司馬光一人獨褒顯,而未及呂公著、韓維、范純仁、呂大防、安燾輩。建中初言官陳瓘已褒贈,而未及鄒浩。」於是元祐諸臣皆次第牽復。



 尋四上章乞罷諫省,除給事中,辭,乞致仕,除徽猷閣直學士、提舉嵩山崇福宮。時力辭直學士之命,改除徽猷閣待制、提舉崇福宮。陛辭,猶上書乞選將練兵,為戰守之備。



 高宗即位,除工部侍郎。陛對言:「自古聖賢之君,未有不以典學為務。」除兼侍讀。乞修《建炎會
 計錄》,乞恤勤王之兵,乞寬假言者。連章丐外,以龍圖閣直學士提舉杭州洞霄宮。已而告老,以本官致仕,優游林泉,以著書講學為事。卒年八十三,謚文靖。



 時在東郡,所交皆天下士,先達陳瓘、鄒浩皆以師禮事時。暨渡江,東南學者推時為程氏正宗。與胡安國往來講論尤多。時浮沉州縣四十有七年,晚居諫省,僅九十日,凡所論列皆切於世道,而其大者,則闢王氏經學,排靖康和議,使邪說不作。凡紹興初崇尚元祐學術,而朱熹、張栻之
 學得程氏之正,其源委脈絡皆出於時。



 子迪,力學通經,亦嘗師程頤云。



 羅從彥字仲素,南劍人。以累舉恩為惠州博羅縣主簿。聞同郡楊時得河南程氏學,慨然慕之,及時為蕭山令,遂徒步往學焉。時熟察之,乃喜曰:「惟從彥可與言道。」於是日益以親,時弟子千餘人,無及從彥者。從彥初見時三日,即驚汗浹背,曰:「不至是,幾虛過一生矣。」嘗與時講《易》,至《乾》九四爻,云:「伊川說甚善。」從彥即鬻田走洛,見頤
 問之,頤反覆以告,從彥謝曰:「聞之龜山具是矣。」乃歸卒業。



 沙縣陳淵,楊時之婿也,嘗詣從彥,必竟日乃返,謂人曰:「自吾交仲素,日聞所不聞,奧學清節,真南州之冠冕也。既而築室山中,絕意仕進,終日端坐,間謁時將溪上,吟詠而歸,恆充然自得焉。



 嘗採祖宗故事為《遵堯錄》,靖康中,擬獻闕下,會國難不果。嘗與學者論治曰:「祖宗法度不可廢,德澤不可恃。廢法度則變亂之事起,恃德澤則驕佚之心生。自古德澤最厚莫若堯、舜,向使子孫可
 恃,則堯、舜必傳其子。法度之明莫如周,向使子孫世守文、武、成、康之遺緒,雖至今存可也。」又曰:「君子在朝則天下必治,蓋君子進則常有亂世之言,使人主多憂而善心生,故治。小人在朝則天下亂,蓋小人進則常有治世之言,使人主多樂而怠心生,故亂。」又曰:「天下之變不起於四方,而起於朝廷。譬如人之傷氣,則寒暑易侵;木之傷心,則風雨易折。故內有林甫之奸,則外必有祿山之亂,內有盧杞之奸,則外必有朱泚之叛。」



 其論士行曰:「周、
 孔之心使人明道,學者果能明道,則周、孔之心,深自得之。三代人才得周、孔之心,而明道者多,故視死生去就如寒暑晝夜之移,而忠義行之者易。至漢、唐以經術古文相尚,而失周、孔之心,故經術自董生、公孫弘倡之,古文自韓愈、柳宗元啟之,於是明道者寡,故視死生去就如萬鈞九鼎之重,而忠義行之者難。嗚呼,學者所見,自漢、唐喪矣。」又曰:「士之立朝,要以正直忠厚為本。正直則朝廷無過失,忠厚則天下無嗟怨。一於正直而不忠厚,
 則漸入於刻。一於忠厚而不正直,則流入於懦。」其議論醇正類此。



 朱熹謂:「龜山倡道東南,士之游其門者甚眾,然潛思力行、任重詣極如仲素,一人而已。」紹興中卒,學者稱之曰豫章先生,淳祐間謚文質。



 李侗字願中,南劍州劍浦人。年二十四,聞郡人羅從彥得河、洛之學,遂以書謁之,其略曰:



 侗聞之,天下有三本焉,父生之,師教之,君治之,闕其一則本不立。古之聖賢莫不有師,其肄業之勤惰,涉道之淺深,求益之先後,若
 存若亡,其詳不可得而考。惟洙、泗之間,七十二弟子之徒,議論問答,具在方冊,有足稽焉,是得夫子而益明矣。孟氏之後,道失其傳,枝分派別,自立門戶,天下真儒不復見於世。其聚徒成群,所以相傳授者,句讀文義而已爾,謂之熄焉可也。



 其惟先生服膺龜山先生之講席有年矣,況嘗及伊川先生之門,得不傳之道於千五百年之後,性明而修,行完而潔,擴之以廣大,體之以仁恕,精深微妙,各極其至,漢、唐諸儒無近似者。至於不言而飲
 人以和,與人並立而使人化,如春風發物,蓋亦莫知其所以然也。凡讀聖賢之書,粗有識見者,孰不願得授經門下,以質所疑,至於異論之人,固當置而勿論也。



 侗之愚鄙,徒以習舉子業,不得服役於門下,而今日拳拳欲求教者,以謂所求有大於利祿也。抑侗聞之,道可以治心,猶食之充飽,衣之禦寒也。人有迫於饑寒之患者,皇皇焉為衣食之謀,造次顛沛,未始忘也。至於心之不治,有沒世不知慮,豈愛心不若口體哉,弗思甚矣。



 侗不量
 資質之陋,徒以祖父以儒學起家,不忍墜箕裘之業,孜孜矻矻為利祿之學,雖知真儒有作,聞風而起,固不若先生親炙之得於動靜語默之間,目擊而意全也。今生二十有四歲,茫乎未有所止,燭理未明而是非無以辨,宅心不廣而喜怒易以搖,操履不完而悔吝多,精神不充而智巧襲,揀焉而不凈,守焉而不敷,朝夕恐懼,不啻如饑寒切身者求充饑禦寒之具也。不然,安敢以不肖之身為先生之累哉。



 從之累年,授《春秋》、《中庸》、《語》、《孟》之說。
 從彥好靜坐,侗退入室中,亦靜坐。從彥令靜中看喜怒哀樂未發前氣象,而求所謂「中」者,久之,而於天下之理該攝洞貫,以次融釋,各有條序,從彥亟稱許焉。



 既而退居山田,謝絕世故餘四十年,食飲或不充,而怡然自適。事親孝謹,仲兄性剛多忤,侗事之得其歡心。閨門內外,夷愉肅穆,若無人聲,而眾事自理。親戚有貧不能婚嫁者,則為經理振助之。與鄉人處,飲食言笑,終日油油如也。



 其接後學,答問不倦,雖隨人淺深施教,而必自反身自
 得始。故其言曰:「學問之道不在多言,但默坐澄心,體認天理。若是,雖一毫私欲之發,亦退聽矣。」又曰:「學者之病,在於未有灑然冰解凍釋處。如孔門諸子,群居終日,交相切磨,又得夫子為之依歸,日用之間觀感而化者多矣。恐於融釋而不脫落處,非言說所及也。」又曰:「讀書者知其所言莫非吾事,而即吾身以求之,則凡聖賢所至而吾所未至者,皆可勉而進矣。若直求之文字,以資誦說,其不為玩物喪志者幾希。」又曰:「講學切在深潛縝密,
 然後氣味深長,蹊徑不差。若概以理一,而不察其分之殊,此學者所以流於疑似亂真之說而不自知也。」嘗以黃庭堅之稱濂溪周茂叔「胸中酒落,如光風霽月」,為善形容有道者氣象,嘗諷誦之,而顧謂學者存此於胸中,庶幾遇事廓然,而義理少進矣。



 其語《中庸》曰:「聖門之傳是書,其所以開悟後學無遺策矣。然所謂『喜怒哀樂未發謂之中』者,又一篇之指要也。若徒記誦而已,則亦奚以為哉?必也體之於身,實見是理,若顏子之嘆,卓然若
 有所見,而不違乎心目之間,然後擴充而往,無所不通,則庶乎其可以言《中庸》矣。」其語《春秋》曰:「《春秋》一事各是發明一例,如觀山水,徙步而形勢不同,不可拘以一法。然所以難言者,蓋以常人之心推測聖人,未到聖人灑然處,豈能無失耶?」



 侗既閑居,若無意當世,而傷時憂國,論事感激動人。嘗曰:「今日三綱不振,義利不分。三綱不振,故人心邪僻,不堪任用,是致上下之氣間隔,而中國日衰。義利不分,故自王安石用事,陷溺人心,至今不自
 知覺。人趨利而不知義,則主勢日孤,人主當於此留意,不然,則是所謂『雖有粟,吾得而食諸』也。」



 是時吏部員外郎朱松與侗為同門友,雅重侗,遣子熹從學,熹卒得其傳。沙縣鄧迪嘗謂松曰:「願中如冰壺秋月,瑩徹無瑕,非吾曹所及。」松以謂知言。而熹亦稱同:「姿稟勁特,氣節豪邁,而充養完粹,無復圭角,精純之氣達於面目,色溫言厲,神定氣和,語默動靜,端詳閑泰,自然之中若有成法。平日恂恂,於事若無甚可否,及其酬酢事變,斷以義理,
 則有截然不可犯者。」又謂自從侗學,辭去復來,則所聞益超絕。其上達不已如此。



 侗子友直、信甫皆舉進士,試吏旁郡,更請迎養。歸道武夷,會閩帥汪應辰以書幣來迎,侗往見之,至之日疾作,遂卒,年七十有一。



 信甫仕至監察御史,出知衢州,擢廣東、江東憲,以特立不容於朝雲。



\end{pinyinscope}