\article{列傳第二百四十一世家五}

\begin{pinyinscope}

 世家五○北漢劉氏



 北漢劉繼元,并州太原人。祖崇,漢祖之弟,漢初為太原尹、北京留守。隱帝嗣位,周祖為樞密使,崇謂判官鄭珙
 曰:「吾與郭樞密素不協,朝廷幼弱,郭得誌,吾無類矣。」因泣下。珙遂勸繕完甲兵,招集亡命,為自全計。及聞隱帝遇害,崇欲率兵南向,會漢太后下令遣馮道詣徐州迎崇子贇為漢嗣,崇信之,謂賓佐曰:「吾兒為帝矣,復何慮哉?」少尹李驤曰:「知幾其神,時不可失。揣郭公之心,必不以天下與人,不如領精騎疾度太行,控孟津,以觀其變,徐州位定,然後歸晉陽,即郭公不敢動矣。」崇大怒,罵曰:「腐儒敢離間我父子!」遽令左右曳出斬之。驤曰:「仆負王
 佐才,今日為愚人畫計,死固甘心,但家有病妻,願同戮於市。」崇並殺之,表其事於太后,明無他誌。俄周祖為眾所推,降封贇湘陰公。崇遣使奉書周祖,乞贇歸藩。使還,知贇已死,崇慟哭,為驤立祠。



 遂即皇帝位,國仍號漢,仍稱乾祐年,改名旻。以子鈞為太原尹,判官趙華、鄭珙為宰相,陳光裕為宣徽使。齎重幣結契丹,自言與周有隙,願如晉祖故事,約為父子。契丹主許之,遣政事令燕王耶律述軋、上樞使高勳,策崇為大漢神武皇帝。自是數
 侵晉、絳。高平之敗,崇單騎遁歸,由此喪氣,不敢復出師。顯德元年,崇卒,鈞襲位。



 鈞舊名承鈞,後止名鈞。改元天會,以衛融為相,段常為樞密使,蔚進掌親軍,子繼恩為太原尹。始建七廟於漢祖舊第,號顯聖宮。潛結江南、西川為外援。六年冬,鈞結契丹侵周。明年正月,周恭帝命太祖北征,至陳橋驛,眾推戴太祖即位。鈞與契丹兵皆遁去。



 是夏,李筠以上黨叛,令判官囚監軍周光遜等送於鈞,稱臣求援。鈞自至太平驛與筠會,遣其宣徽使盧
 讚將騎數千隨筠入寇,又遣其河陽節度範守圖援之。及太祖親討,前軍石守信、高懷德破筠眾於澤州,獲守圖,殺鈞兵數千。鈞之沙穀砦又為折德扆所破,斬首五百級。九月,昭義李繼勳率師入鈞平遙,虜獲甚眾。建隆二年冬,繼勳又敗鈞兵,斬首百餘級,獲其遼州刺史傅廷彥弟勳以獻。



 三年二月,鈞侵晉、潞二州,守將擊走之。三月,太祖詔河東降人徙家於邢、洺,計口給粟。四月,太原民四百七十人降。七月,鈞捉生指揮使路貴等十一
 人降,並補內殿直。四年八月,邢州王全贇率師攻樂平,鈞拱衛指揮使王超、散指揮使元威、侯霸榮率所部千八百人降全贇。未幾,鈞侍衛都指揮使蔚進、馬軍都指揮使郝貴超與契丹悉兵來救樂平,三戰皆敗之。遂下其城,詔建為平晉軍,以降兵為效順軍,賜以錢帛,靜陽十八砦遂相率來降。九月,鈞復引契丹攻平晉軍,太祖遣洺州防禦使郭進、濮州防禦使張彥進、客省使曹彬、趙州刺史陳萬通將步騎萬餘救之,未至而鈞遁去。



 乾
 德二年二月,李繼勳與兵馬鈐轄康延沼、馬步軍都軍頭尹訓率兵攻遼州,鈞遣郝貴超來援,戰於城下,大敗。刺史杜延韜危蹙,與拱衛都指揮使冀進、兵馬都監侯美籍部兵三千降於繼勳,賜延韜等襲衣、銀帶、器幣、鞍勒馬,其降兵以效順、懷恩為名。是月,府州擒鈞衛州刺史楊璘以獻。又鈞耀州團練使周審玉等四人降,賜審玉襲衣、金帶、絹千匹、銀五百兩、鞍勒馬,仍賜名承晉,以為左千牛衛大將軍、領汾州團練使。四月,太祖遣馬軍
 都校劉光將兵戍潞,備鈞入侵。五年三月,鈞招收指揮使閻章以石盆砦降鎮州。四月,招收指揮使樊暉殺監軍成昭,以鴻唐砦降鎮州。六年正月,偏成砦招收指揮使任恩等百五十人降晉州。三月,鎮州守將攻破鈞馬鞍山砦。七月,鈞烏玉砦主胡遇等百三十九人降鎮州。



 初,鈞自李筠敗,狼狽而歸,旦夕懼宋師之至。以趙文度為相,召抱腹山人郭無為參議中書事,以五台山僧繼顒為鴻臚卿,參議國事。因事誅段常,契丹主遣使責鈞
 曰:「爾不稟我命,其罪三:擅改年號,一也;助李筠有所覬覦,二也;殺段常,三也。」鈞皇恐曰:「父為子隱,願赦罪。」契丹不報。自是使契丹者被留不遣。終以勢力窘弱,憂憤成疾,是月卒,年四十三。繼恩嗣位。



 初,太祖嚐因界上諜者謂鈞曰:「君家與周氏為世仇,宜其不屈,今我與爾無所間,何為困此一方人也?若有誌中國,宜下太行以決勝負。」鈞遣諜者復命曰:「 河東土地甲兵不足以當中國,然鈞家世非叛者,區區守此,蓋懼漢氏之不血食也。」太祖
 哀其言,笑謂諜者曰:「為我語鈞,開爾一生路。」故終其世不加兵焉。



 繼恩本姓薛。父釗,娶崇女,晉初為護聖營卒。漢祖典禁兵,以釗崇婿,釋其籍,館門下。漢祖後領方鎮,爵位通顯,釗罕得見其妻,居常怏怏。一日乘醉求見,即引佩刀刺妻,妻奮衣得脫,釗乃自剄。繼恩時尚幼,漢祖令鈞養為子,遂冒姓劉。



 八月,太祖詔伐繼恩。以內客省使盧懷忠等二十二人將禁兵赴潞州,昭義節度李繼勳為行營前軍都部署,侍衛步軍都指揮使黨進副之,
 宣徽南院使曹彬為都監;棣州防禦使何繼筠為前鋒部署,懷州防禦使康延沼為都監;建雄軍節度趙讚為汾州路部署,絳州防禦使司超副之,隰州刺史李謙溥為都監。九月,繼勳敗繼恩軍於洞渦河,其左勝軍使李瓊來降,賜襲衣、金帶、鞍勒馬。



 初,鈞謂郭無為曰:「繼恩庸懦,何堪付後事?」無為亦以為然。至是繼恩獨處一室行喪,左右親信皆在太原,無得從者。或勸召之,繼恩猶豫不決。有侯霸榮者,邢州龍岡人。多力善射,走及奔馬,嚐
 為盜並、汾間,鈞用為散指揮使,戍樂平。建隆中,率所部來歸,補內殿直。未幾,復奔太原,鈞署供奉官。至是謀持繼恩首獻太祖,遂乘繼恩無備,白晝挺刃而入,反扃其門,繼恩繞屏環走,霸榮以刃揕胸弑之,年三十四,時立六十日矣。無為遣卒登梯入,殺霸榮,立其弟繼元。



 繼元本姓何。初,薛釗死,崇以女再妻何氏,生繼元。何死,鈞亦養繼元為子。繼元既襲位,改元廣運,復結契丹為援。開寶二年春,太祖詔李繼勳、趙讚、郭進、司超等將兵先赴
 太原,太祖遂親征。以繼元太穀令梁文陟為太子洗馬,祁令張續為右讚善大夫。太祖將至,繼勳敗繼元兵於城下,其憲州推官史昭文以州來降,升本州刺史。乃壅汾水灌其城,又遣海州刺史孫方進圍汾州。繼元方恃契丹為援,守陴者揚言旦夕契丹至。四月,何繼筠敗契丹於陽曲北。太祖命以所獲首級、鎧甲示於城下,城中由是喪氣,知嵐州趙文度遂來降。閏五月,南城為汾水陷,水注城中,太祖幸長堤觀焉。登望樓者見繼元殺其
 相郭無為,城中紛擾。俄而城兵自西長連城出,將焚攻戰具,反為攻兵擊走之,斬首萬餘級。夜半,傳呼壁外繼元降,太祖令衛士擐甲,將開壁門,八作使趙遂曰:「受降如受敵,詎可中夜輕出?」太祖使伺之,果諜者也。



 太常博士李光讚上言曰:「陛下應天順人,體元禦極,戰無不勝,謀無不臧,四方恃險之邦,僭竊帝王之號者,昔日與中國為鄰,今日與陛下為臣。蕞爾晉陽,豈須親討,重勞飛免,久駐師徒。且太原得之未必為多,失之未足為辱。今
 時屬炎蒸,候當暑雨,倘河津泛溢,道路阻艱,輦運稽留,恐勞宸慮。」太祖覽奏甚喜,命宰相趙普撫諭諸將欲班師。禁軍校趙翰等叩頭願乘城急擊,以盡死力,太祖曰:「汝曹我所訓練,無不一當百,以備肘腋、同休戚也。我寧不取太原,豈忍驅汝曹冒鋒鏑而蹈必死之地乎?」士皆感泣,遂班師。



 九年八月,太祖又遣黨進、潘美、楊光美、牛思進、米文義討之。時繼元諜者趙訓為晉州所捕,械送於朝,太祖命釋之,給服裝放歸。又遣郭進入忻代路,郝
 崇信、王政忠入汾州路,閻彥進、齊超入沁州路,孫晏宣、安守忠入遼州路,齊延琛、穆彥璋入石州路。九月,黨進敗繼元兵數千,獲馬千餘。郭進得山北民三萬七千餘。十月,遼州監押馬繼恩入并州境,燔四十餘砦,獲牛羊數千。郭進又破壽陽,得民九千。穆彥璋入并州境,得民二千。黨進又敗繼元兵千餘於城下。是月,太宗即位,召諸將還。



 太平興國二年,繼元胡桃砦指揮使史溫等以其民內附。太宗謂齊王廷美曰:「太原,我必取之。」四年,始
 議討伐,曹彬以為可,太宗意遂決,語在《彬傳》。宰相薛居正曰:「昔周世宗舉兵,太原倚契丹之援,堅壁不戰,以至師老而歸。及太祖破契丹於雁門關南,盡驅其民分布河、洛之間,雖巢穴尚存,而危困已甚,得之不足以辟土,舍之不足以為患,願陛下熟慮之。」太宗曰:「今者事同而勢異,彼弱而我強。昔先皇破契丹,徙其人而空其地者,正為今日事也。朕計決矣,卿勿復言。」遂遣宣徽南院使潘美等率諸將分兵圍汾、沁、嵐諸州,車駕遂親征,以驍
 將郭進扼石嶺關,斷契丹援路。契丹果至,進擊敗之。



 初,繼元遣子續質於契丹,契丹為進所敗。繼元又遣健步間道齎蠟丸帛書求救,進又得之,徇於城下。繼元外援不至,餉道又絕,潘美等兵數十萬長圍四合,自春徂夏,矢石如雨,晝夜不息,城中大懼。會太宗奄至,親督衛士急攻,人百其勇,城無完堞。太宗慮城陷則殺傷者眾,以手詔諭繼元降,詔至城下,守陴者不納,繼元不能知。太宗躬擐甲胄,夜至長連城督諸將攻之,控弦之士數萬
 列陣於前,蹲甲交射,矢集城上如蝟毛,每給矢必數百萬,頃之鹹盡。捕得城中人雲,繼元以十錢購一矢,凡聚百餘萬,太宗笑曰:「此為我畜也。」



 五月庚辰,繼元宣徽使範超來降,攻城者以超為出戰,禽而戮之。繼元遂斬超妻子,投其首城外。壬午,馬軍都指揮使郭萬超逾城降,繼元帳下親信因之漸亡去,城中危急。太宗又自草詔諭之曰:「越王、吳主獻地歸朝,或授以大藩,或列於上將,臣僚、子弟皆享官封。繼元但速降,必保終始富貴,安危
 兩途,爾宜自擇。」至是詔入,諸將銳攻不可遏,太宗臨之,恐城陷害民,麾眾少退。是夕,繼元遣其客省使李勳奉表請降,太宗賜勳襲衣、金帶、銀器、錦彩、銀鞍勒馬,復遣通事舍人薛文寶齎詔答之。夜漏未盡,太宗幸城北,張樂宴從臣於城台,繼元降。遲明,繼元率官屬縞衣紗帽待罪台下,詔釋之,賜襲衣、玉帶、金銀鞍勒馬三匹、金器五百兩、銀器五千兩、錦彩二千段,文武官各賜衣、金銀帶、器幣、鞍勒馬有差。召升台,繼元叩頭言:「臣聞車駕親
 征,即願束身歸罪,蓋亡命者懼死,逼臣不得降爾。」太宗籍軍中亡投繼元者數百人,選其巨室者以從軍法,餘賜服及錢帛,分隸諸將。詔授繼元特進、檢校太師、右衛上將軍,封彭城郡公,館於行在所,給賜甚厚。其相李惲等授官有差,命中使康仁寶監之。繼元獻其宮妓百餘,悉分賜立功將校。又令仁寶護繼元親屬百餘赴京,所過續食,賜京城甲第一區,歲時優加頒賚。六年,加開府儀同三司。雍熙三年,建房州為保康軍,以繼元為節度。



 淳化二年,繼元疾,遣中使護醫診視,及卒,遺奏以其子三豬為托,太宗惻然哀之,贈中書令,追封彭城郡王,賵賻加等,葬事官給。時三豬六歲,賜名守節,授西京作坊副使,家居賜祿。



 初,太宗征繼元,行次澶淵,有太仆寺丞宋捷者掌出納行在軍儲,太宗見其姓名喜,以為師必有捷之兆。及將至太原,太宗遣語攻城諸將曰:「我以端午日當置酒高會於太原城中。」至癸未,繼元降,乃五月五日也。劉崇自周廣順元年稱帝,曆四主二十九年而
 亡。



 繼元性殘忍,在太原,凡臣下有忤意,必族其家。自太祖親征及遣將攻伐,因之殺傷不可勝紀。及窮蹙始降,太宗待遇終保全之,嚐謂近臣曰:「晉司馬昭以劉禪思蜀之對,戲之雲『何乃似卻正之言』,此不仁之甚也。亡國之君皆暗懦所致,苟有遠識,豈至滅亡?此可湣傷,何反戲侮乎?劉繼元朕所虜者,待之若賓客,猶恐不慰其意爾。」



 守節後為崇儀使,改右屯衛將軍。天禧四年,特遷右武衛將軍,改右驍衛將軍。



 衛融字明遠,青州博興人。晉天福初舉進士,調南樂主簿,曆齊澶二州從事、忠武軍掌書記。漢初,為太原觀察支使,劉崇稱帝,授中書侍郎、平章事。



 太祖立,李筠據上黨,遣使降劉鈞,鈞自將兵至太平驛與筠會,遣宣徽使盧讚入潞州監筠軍。讚與筠不協,鈞遣融和解之。會筠敗,融被擒,太祖責之曰:「汝何故勸劉鈞舉兵助李筠反耶?」融曰:「犬吠非其主,臣四十口受劉氏豐衣美食,不忍負之。陛下縱不殺臣,臣亦不為陛下用,終當間道走河
 東爾。」太祖怒,令左右以鐵撾擊其首,曳出將戮之。融大呼曰:「大丈夫死或重於泰山,或輕於鴻毛,今之死正得其所爾。」太祖聞之曰:「此忠臣也。」遽命釋之,召坐御前,以良藥傅其創,賜襲衣、金帶、鞍勒馬。既而欲放融歸,令融先為書諭鈞,言俟周光遜等歸朝,即遣融去。鈞得書久無報,乃授融太府卿,賜第京城。乾德初,郊祀,融獻《郊禋大體賦》,改司農卿,出知陳、舒、黃三州。開寶六年,卒,年六十九。



 子偁、儔,孫齊,並進士及第。



 趙文度,薊州漁陽人。父玉嚐客滄州,依節度判官呂兗。劉守光破滄州,收兗親屬盡戮之,兗子琦年十四,玉負之以逃,至太原,變姓名,丐衣食以給琦,琦後唐同光初為藩郡從事。當是時,燕、趙之士,以玉能存呂氏之孤,翕然稱之。明宗朝,琦至職方員外郎知雜。清泰中,琦為給事中、端明殿學士,玉已卒矣。



 文度入洛舉進士,琦薦於主司馬裔孫,擢甲科,曆徐、兗、陳、許四鎮從事。漢初,為河東掌書記。文度捷給善戲謔,劉崇雅愛之,及稱帝,累官
 至翰林承旨、兵部尚書。天會四年,授中書侍郎、平章事,轉門下侍郎兼樞密使,加司徒。久之,與郭無為不協,出知汾州,徙嵐州。



 太祖開寶二年親征晉陽,遣偏師圍嵐。文度危蹙請降,待罪行宮,太祖命釋之,賜襲衣、玉帶、金鞍勒馬、器幣甚厚,其官屬賜物有差。文度本名弘,以犯宣祖廟諱,賜今名。師還,授檢校太傅、安國軍節度,歲餘徙華州,不宣製而告敕同宣製之例。又徙耀州,凡曆三鎮。七年,卒,年六十一。



 文度善為詩,人多諷誦,有《觀光集》。
 文度之降也,其母在太原,世以不能死節罪之。子昌圖,至內殿崇班、閣門祗候。



 李惲字孟深,開封陽武人。漢乾祐中舉進士,客遊嵐州。會劉崇自立,署州從事,擢知制誥、翰林學士,累至司空、平章事。時母在鄉里,惲不知存亡,居常戚戚,但以弈棋沈飲為務,政事多廢。劉繼元頻以為言,惲不介意。後方與僧弈棋,繼元命近侍直抵惲前,取局焚之。惲怡然,徐詣繼元謝,繼元因切責之,明日別造新局,弈棋如故。太
 宗克太原,為殿中監,始知母亡,表求追服母喪,不許。出知廣州,遷司農卿,連知許、孟二州。以足疾求解,授忠武軍行軍司馬。端拱元年,卒,年七十三。



 惲性疏達,善談名理。年少時好滑稽,及為相,頗事持重。初與王溥、李昉同年登第,太原平,相見敘舊,情好益固,論者美之。子存誠,駕部員外郎;存信,左侍禁、閣門祗候。



 馬峰,并州太原人。仕劉繼元至樞密使、左仆射致仕。太原平,太宗以為將作監,遷太府卿,分司西京。峰善服餌
 養生,體強無疾,性鄙吝,頗好持論。雍熙元年,卒,年八十餘。



 郭無為,青州千乘人。少博學有辭辯,為道士,隱武當山。漢乾祐中,周祖征河中,無為杖策謁於軍門,周祖一見大奇之,將留館門下。左右曰:「無為縱橫家流,今公握重兵,不宜親之。」無為遂拂衣去,隱太原抱腹山。



 會劉鈞將兵援李筠,將發太原,其大臣趙華諫曰:「筠舉動輕易,今起兵應之,未見其可。」鈞怒不顧,遂行。及筠敗,鈞狼狽而
 歸,由是重文學之士。且日夕懼宋師至,頗求有智謀者與之計事。段常薦無為於鈞,鈞以諫議大夫召之。及至,與語大悅,尋遷吏部侍郎、參議中書事。與趙文度同秉政,意好不協,鈞乃出文度知汾州。俄誅段常,遂以無為為左仆射、平章事兼樞密使,機務一以委之。鈞嚐病,與無為語及後事,謂其子繼恩不才,無為亦言其然。繼恩既立,知其事,欲誅無為,畏懦不能決。月餘,侯霸榮弑繼恩,無為使人殺霸榮,並人疑無為初授意於霸榮,後殺
 之以滅口也。



 繼元立,太祖遣李繼勳等討之,仍詔許繼元以青州節度、無為邢州節度,無為得詔色動。一日,繼元宴群臣,契丹使亦在焉,無為慟哭於庭曰:「今日以空城抗大軍,計將安出?」引佩刀欲自刺,繼元遽降階持其手,引無為升坐,蓋無為欲以動眾心也。及太祖親征,長圍既合,無為請自將兵夜出擊圍,欲自拔來歸,值天陰晦而止。閹人衛德貴告其事。會太祖壅汾水浸城,城中人情大懼,繼元乃殺無為以徇。



\end{pinyinscope}