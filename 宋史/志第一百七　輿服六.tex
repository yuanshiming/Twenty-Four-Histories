\article{志第一百七 輿服六}

\begin{pinyinscope}

 寶印符券宮室制度臣庶室屋制度



 寶。秦制,天子有六璽,又有傳國璽,歷代因之。唐改為寶,其制有八。五代亂離,或多亡失。周廣順中,始造二寶:其一曰「皇帝承天受命之寶」,一曰「皇帝神寶」。太祖受禪,傳
 此二寶,又制「大宋受命之寶」。至太宗,又別制「承天受命之寶」。是後,諸帝嗣服,皆自為一寶,以「皇帝恭膺天命之寶」為文。凡上尊號,有司制玉寶,則以所上尊號為文。



 寶用玉,篆文,廣四寸九分,厚一寸二分。填以金盤龍鈕,系以暈錦大綬,赤小綬,連玉環;玉檢高七寸,廣二寸四分,厚四分;玉斗方二寸四分,厚一寸二分:皆飾以紅錦,金裝,裹以紅錦,加紅羅泥金夾帊,納於小盝。盝以金裝,內設金床,暈錦褥,飾以雜色玻璃、碧鈿石、珊瑚、金精石、瑪瑙。
 又盝二重,皆裝以金,覆以紅羅繡帊,載以腰輿及行馬,並飾以金。又有香爐、寶子、香匙、灰匙、火箸、燭臺、燭刀,皆以金為之,是所謂緣寶法物也。



 別有三印:一曰「天下合同之印」,中書奏覆狀、流內銓歷任三代狀用之;二曰「御前之印」,樞密院宣命及諸司奏狀內用之;三曰「書詔之印」,翰林詔敕用之。皆鑄以金,又以瑜石各鑄其一。雍熙三年,北改為寶,別鑄以金,舊六印皆毀之。



 真宗即位,作皇帝受命寶,文曰「皇帝恭膺天命之寶」。大中祥符元年
 五月,詳定所言:「按玉牒、玉冊,用皇帝受命寶印之,納玉匱於石□感,以天下同文之印封之。今封禪泰山,請依舊制,別造玉寶一枚,方寸二分,文同受命寶。其封石□感,用天下同文之印,舊史元無制度,今請用金鑄,大小同御前之寶,以『天下同文之寶』為文。所有緣寶法物,亦請依式制造。」從之。天禧元年十二月,召輔臣於滋福殿,觀新刻「五岳聖帝玉寶』及「皇帝昭受乾符之寶」,命擇日迎導赴會靈觀奉安。其寶並金柙玉鈕,制作精妙。真宗以奏
 章上帝,承前皆用御前之寶,以理未順,故改用昭受乾符之寶。



 乾興元年,仁宗即位,作受命寶,文同真宗。天聖元年,詔以宮城火,重制受命寶及尊號冊寶。慶歷八年十一月,詔刻「皇帝欽崇國祀之寶」。先是,天禧中,真宗刻昭受乾符之寶,而於醮祠表章用之。後經大內火,寶焚,乃用御前之寶。至是,下學士院定其文,命宰臣陳執中書之。皇祐五年七月,詔作「鎮國神寶」。先是,奉宸庫有良玉,廣尺,厚半之。仁宗以為希代之珍,不欲為服玩,因作
 是寶,命宰臣龐籍篆文。寶成,太常禮院引《唐六典》次序曰:「一神寶,二受命寶,冬至祀南郊,大駕儀仗,請以鎮國神寶先受命寶為前導。」自是為定式。至和二年,初,太宗以玉寶二鈕賜太祖之子德芳,其文曰「皇帝信寶」,至是,德芳孫左屯衛大將軍從式上之。



 嘉祐八年,仁宗崩,英宗立,翰林學士範鎮言:「伏聞大行皇帝受命寶及緣寶法物,與平生衣冠器用,皆欲舉而葬之,恐非所以稱先帝恭儉之意。其受命寶,伏乞陛下自寶用之,且示有所
 傳付。若衣冠器玩,則請陳於陵寢及神御殿,歲時展視,以慰思慕。」詔檢討官考索典故,及命兩制、禮官詳議。翰林學士王珪等奏曰:「受命寶者,猶昔傳國璽也,宜為天子傳器,不當改作。古者藏先王衣服於廟寢,至於平生器玩,則前世既不皆納於方中,亦不盡陳於陵寢。謂今宜從省約,以稱先帝恭儉之實。」帝不用其議,乃別造受命寶,命參知政事歐陽修篆文八字。至哲宗立,亦作焉,其文並同。



 紹聖三年,咸陽縣民段義得古玉印,自言於
 河南鄉劉銀村修舍,掘地得之,有光照室。四年,上之,詔禮部、御史臺以下參驗。無符元年三月,翰林學士承旨蔡京及講議官十三員奏:



 按所獻玉璽,色綠如藍,溫潤而澤,其文曰「受命於天,既壽永昌」。其背螭鈕五盤,鈕間有小竅,用以貫組。又得玉螭首一,白如膏,亦溫潤,其背亦螭鈕五盤,鈕間亦有貫組小竅,其面無文,與璽大小相合。篆文工作,皆非近世所為。



 臣等以歷代正史考之,璽之文曰「皇帝壽昌」者,晉璽也;曰「受命於天」者,後魏璽
 也;「有德者昌」,唐璽也;「惟德允昌」,石晉璽也;則「既壽永昌」者,秦璽可知。今得璽於咸陽,其玉乃藍田之色,其篆與李斯小篆體合。飾以龍鳳鳥魚,乃蟲書鳥跡之法,於今所傳古書,莫可比擬,非漢以後所作明矣。



 今陛下嗣守祖宗大寶,而神璽自出,其文曰「受命於天,既壽永昌」,則天之所畀,烏可忽哉?漢、晉以來,得寶鼎瑞物,猶告廟改元,肆眚上壽,況傳國之器乎?其緣寶法物禮儀,乞下所屬施行。



 詔禮部、太常寺按故事詳定以聞。禮官言:五月
 朔,故事當大朝會,宜就行受寶之禮。依上尊號寶冊儀,有司豫制緣寶法物,並寶進入。俟降出,權於寶堂安奉。前三日,差官奏告天地、宗廟、社稷。前一日,帝齋於內殿。翌日,御大慶殿,降坐受寶,群臣上壽稱賀。先期,又詔龍圖、天章閣繼治平元年耀州所獻受命寶玉檢,赴都堂參議。詔以五月朔受傳國寶,命章惇書玉檢,以「天授傳國受命之寶」為文。



 徽宗崇寧五年,有以玉印獻者。印方寸,以龜為鈕,工作精巧,文曰「承天福延萬億永無極」。徽
 宗因次其文,仿李斯蟲魚篆作寶文。其方四寸有奇,螭鈕,方盤,上圓下方,名為鎮國寶。大觀元年,又得玉工,用元豐中玉琢天子、皇帝六璽,疊篆。初,紹聖間,得漢傳國璽,無檢,幅又不闕,疑其一角缺者,乃檢也。有《檢傳》,考驗甚詳,傳於世。帝於是取其文而黜其璽不用,自作受命寶,其方四寸有奇,琢以白玉,篆以蟲魚。鎮國、受命二寶,合天子、皇帝六璽,是為八寶。



 詔曰:「自昔皆有尚符璽官。今雖隸門下後省,遇親祠,則臨時具員,訖事復罷。八寶
 既備,宜重典司之職。可令尚書省置官,如古之制。」又詔曰:「永惟受命之符,當有一代之制,而尚循秦舊,六璽之用,度越百年之久,或未大備。自天申命,地不愛寶,獲全玉於異域,得妙工於編氓,八寶既成,□無前比,殆天所授,非人能為。可以來年元日,御大慶殿恭受八寶。」尚書省言:



 請置符寶郎四員,隸門下省,二員以中人充,掌寶於禁中。按唐八寶,車駕臨幸,則符寶郎奉寶以從;大朝會,則奉寶以進。今鎮國寶、受命寶非常用之器,欲臨幸
 則從六寶,朝會則陳八寶,皆夕納。內符寶郎奉寶出以授外符寶郎,外符寶郎從寶行於禁衛之內,朝則分進於御坐之前。



 鎮國寶、受命寶不常用,唯封禪則用之。皇帝之寶,答鄰國書則用之;皇帝行寶,降御札則用之;皇帝信寶,賜鄰國書及物則用之;天子之寶,答外國書則用之;天子行寶,封冊則用之;天子信寶,舉大兵則用之。應合用寶,外符寶郎具奏,請內符寶郎御前請寶,印訖,付外符寶郎承受。



 從之。二年,詔受命寶之上,加「鎮國」二
 字。



 政和七年,從於闐得大玉逾二尺,色如截肪。徽宗又制一寶,赤螭鈕,文曰「範圍天二,幽贊神明,保合太和,萬壽無疆」。篆以魚蟲,制作之工,幾於秦璽。其寶九寸,檢亦如之,號曰「定命寶」。合前八寶為九,詔以九寶為稱,以定命寶為首。且曰:「八寶者,國之神器;至於定命寶,乃我所自制也。」於是,應行導排設,定命與受命、天子寶在左,鎮國與皇帝寶在右。又詔:「鎮國、受命寶與天子、皇帝之寶,其數有八,蓋非乾元用九之數。比得寶玉於異哉,受定
 命之符於神霄,乃以『範圍天地,幽贊神明,保合太和,萬壽無疆』為文。卜云其吉,篆以蟲魚,縱廣之制,其寸亦九,號曰定命寶。來年元日祗受。」又詔差官奏告天地、宗廟、祖稷。八年正月一日,御大慶殿,受定命寶,百僚稱賀。其後京城之難,諸寶俱失之,惟大宋受命之寶與定命寶獨存,蓋天意也。



 建炎初,始作金寶三:一曰「皇帝欽崇國祀之寶」,祭祀祠表用之;二曰「天下合同之寶」,降付中書門下省用之;三曰「書詔之寶」,發號施令用之。紹興元年,
 又作玉寶一,文曰「大宋受命中興之寶」。又得舊寶二,歷世寶之,凡上太上皇尊號、冊后太子皆用焉。十六年,又作八寶:一曰鎮國神寶,以「承天福延萬億永無極」九字為文;二曰受命寶,以「受命於天既壽永昌」為文;三曰天子之寶;四曰天子信寶;五曰天子行寶;六曰皇帝之寶;七曰皇帝信寶;八曰皇帝行寶。藏之御府,大朝會則陳之;上冊寶尊號、冊后太子、大禮設鹵簿,亦如之。寶之制,用玉尺度,鈕鼻,大小綬,玉環。檢制,舊制如牌,上刻曰某
 寶。皆裹以朱縷,加緋羅泥金帕,納於小盝。盝三重,皆飾以金,內設金床、金寶斗,龍鑰金鎖,覆以緋羅繡帕,載以腰輿、行馬。



 孝宗即位,議上太上皇帝尊號曰光堯壽聖太上皇帝,寶用皇祐中法、黍尺量度。乾道六年,再加十四字尊號,以寶材元系螭龍鈕,止堪改作蹲龍,其鈕高二寸四分五厘,厚一寸一分五厘,竅徑一寸。理宗寶慶三年,加上寧宗皇帝徽號,寶面廣四寸二分,厚一寸二分,蹲龍鈕,通高四寸一分,寶四面鉤碾行龍。



 後妃之寶。哲宗元祐元年,詔:天聖中,章獻明肅皇后用玉寶,方四寸九分,厚一寸二分,龍鈕。今太皇太后權同處分軍國事,宜依章獻明肅皇后故事。二年,又詔:太皇太后玉寶,以「太皇太后之寶」為文;皇太后金寶,以「皇太后寶」為文;皇太妃金寶,以「皇太妃寶」為文。中興之後,後寶用金,方二寸四分,高下隨宜,鼻紐以龜。斗、檢以銀,塗以金。寶盝三重,鈒百花,塗金盤鳳。輿案、行馬、帕褥亦如之。



 皇太子寶。至道元年,制皇太子受冊金寶。方二寸,厚五寸,系以朱組大綬,連玉環,金斗。金檢長五寸,闊二寸,厚二分。裹以紅綿。加紅羅泥金帊,納於小盝。盝以金裝,內設金床。又盝二重,皆覆以紅羅銷金帊。盝及腰輿、行馬皆銀裝金塗。他法物皆銀為之,鈒花塗金。中興,寶,龜鈕;金塗銀檢,上勒「皇太子寶」四字,金塗銀寶斗。黝漆盝三重,並錦拓里,外以金塗銀百花鳳葉子五明裝,鑰以金鎖,載以黝漆腰輿、行馬。



 冊制。用鈱玉,簡長一尺二寸,闊一寸二分;簡數從字之多少。聯以金繩,首尾結帶。前後褾首四枚,二枚畫神,二枚刻龍鏤金,若奉護之狀。藉以錦褥,覆以緋羅泥金夾帊。冊匣長廣取容冊,塗以朱漆,金鏤百花凸起行龍,金鎖、□錔。覆以紅羅繡盤龍蹙金帊,承以金裝長竿床,金龍首,金魚鉤,又以紅絲為絳縈匣。冊案塗朱漆,以銷金紅羅覆之。



 後冊,用鈱,或以象。縷文以鳳,尺寸制度並同帝冊。



 皇太子冊,用鈱簡六十枚,乾道中,用七十五枚,每枚高尺二寸,博一寸二分。前後褾首四枚,長隨簡,博四寸,其二刻神,其二刻龍,為奉護狀。貫以金絲,首尾結為金花,飾以□錔。襯以紅羅泥金夾帕,藉以錦褥,盛以黝漆匣,錦拓里,以金塗銀葉段五明裝,隱起百花鳳。覆以緋羅泥金帕,絡以紅絲結絳,襯以錦褥,載以黝漆腰輿、行馬。



 亡金國寶。理宗端平元年,命孟珙等以兵從大元兵夾攻金人於蔡州,滅之。其年四月丙戌,大理寺言:



 京湖制
 置司以所獲亡金寶物來上,令金臣參知政事張天綱辨識。其玉寶一,文曰「太祖應乾興運昭德定功睿神莊孝仁明大聖武元皇帝尊謚寶」,乃金人上其祖阿骨打謚寶也。其法物有銷金盤龍紅紵絲袍一;透碾雲龍玉帶一,內方八胯結頭一,塌尾一,並玉塗金結頭一,塗金小結攀一;連珠環玉束帶一,垂頭裡拓,上有金龍,帶上玉事件大小一十八;又玉靶鐵剉一,銷金玉事件二,皮茄袋一,玉事件三。



 天綱稱:上項帶,國言謂之「兔鶻」,皆其
 故主冠顏守緒常服之物也。碾玉巾環一,樺皮龍飾角弓一,金龍環刀一,紅紵絲靠枕一,佩玉大環一,皆非臣庶服用之物。制旨冊一本,舊作聖旨,近侍局平日掌此,以承受內降指揮。壬辰四月,故主援東漢光武故事,令上書者不得言「聖」,故避「聖」字不敢當,因改作「制旨」。



 外有臣下虎頭金牌三,銀牌八十四,塗金印三,及諸官署銅印三百一十二顆。法司以守緒函骨及俘囚故寶、法物等,庭引天綱並護尉都尉完顏好海及天綱妻完顏氏
 烏古論栲栳、小女瓊瓊一一審實,件列以聞。



 有旨「完顏守緒遺骸並故寶、法物等,藏大理寺獄庫。天綱、好海、完顏氏烏古論、瓊瓊拘諸殿前司,候朝旨」云。



 印制。兩漢以後,人臣有金印、銀印、銅印。唐制,諸司皆用銅印,宋因之。諸王及中書門下印方二寸一分,樞密、宣徽、三司、尚書省諸司印方二寸。惟尚書省印不塗金,餘皆塗金。節度使印方一寸九分,塗金。餘印並方一寸八分,惟觀察使塗金。諸王、節度、觀察使、州、府、軍、監、縣印,皆
 有銅牌,長七寸五分,諸王廣一寸九分,餘廣一寸八分。諸王、節度、觀察使牌塗以金,刻文云「牌出印入,印出牌入」。其奉使出入,或本局無印者,皆給奉使印。景德初,別鑄兩京奉使印。又有朱記,以給京城及外處職司及諸軍將校等,其制長一寸七分,廣一寸六分。士庶及寺觀亦有私記。



 乾德三年,太祖詔重鑄中書門下、樞密院、三司使印。先是,舊印五代所鑄,篆刻非工。及得蜀中鑄印官祝溫柔,自言其祖思言,唐禮部鑄印官,世習繆篆,即《
 漢書·藝文志》所謂「屈曲纏繞,以模印章」者也。思言隨僖宗入蜀,子孫遂為蜀人。自是,臺、省、寺、監及開封府、興元尹印,悉令溫柔重改鑄焉。



 太宗雍熙元年,詔新除漢南國王錢俶印,宜以「漢南國」為文。四年,詔錢俶新授南陽國王印,宜以「南陽國王之印」為文。真宗咸平三年,賜山前後百蠻王諾驅印,以「大渡河南山前後都鬼王之印」為文。景德四年,鑄交址郡王印,制安南旌節,付廣南轉運司就賜之。



 大中祥符五年,詔諸寺觀及士庶之家所
 用私記,今後並方一寸,雕木為文,不得私鑄。是歲七月,帝覽河西節度使、知許州石普奏狀,用許州觀察使印,以問宰臣王旦。對曰:「節度州有三印:節度印隨本使,使缺則納有司;觀察印,則州長吏用之;州印,晝則付錄事掌用,暮納於長吏。節度使在本鎮,兵仗則節度判官、掌書記、推官書狀,用節度印;田賦則觀察判官、支使、推官書狀,用觀察印;符刺屬縣,則本使判書,用州印。故命帥必曰某軍節度、某州管內觀察等使、某州刺史。言軍,
 則專制其兵旅;言管內,則總察其風俗;言刺史,則蒞其州事。石普獨書奏章,當用河西節度使印。」



 仁宗景祐三年,少府監言:「得篆文官王文盛狀,『在京三司糧料院,頻有人偽造印記,印成旁歷,盜請官物。欲乞鑄造圓印三面,每面闊二寸五分,於外一匝先篆年號及糧料院名,計十二字;次一匝篆寅印十二辰,亦十二字;中心篆正字,上連印鈕,鑄成轉關,以機穴定之。用時逐月分對,年終轉逮十二月,自寅至丑,終始使用。所有轉關正字,次
 月轉定之時,令本院官封押,選差人行使其印。遇改年號,即令別鑄。』詔三司定奪以聞,三司請如文盛奏。後又命知制誥邵必、殿中丞蘇唐卿詳定天下印文,必、唐卿皆通篆籀,然亦無所厘改焉。



 神宗熙寧五年,詔內外官及溪洞官合賜牌印,並令少府監鑄造,送禮部給付。元豐三年,廣西經略司言,知南丹州莫世忍貢銀、香、獅子、馬。遂賜以印,以「西南諸道武盛軍德政官家明天國主印」為文,並以南丹州刺史印賜之,仍詔經略司毀其舊
 印。六年,舊制貢院專掌貢舉,其印曰「禮部貢舉院之印」,以廢貢院,事歸禮部,別鑄「禮部貢舉之印」。是歲十二月,詔自今臣僚所授印,亡歿並賜隨葬,不即隨葬因而行用者,論如律。



 中興仍舊制,惟三省、樞密院用銀印,六部以下用銅印,諸路監司、州縣亦如之。寺監惟長貳給焉,屬則從其長。若倉庫關涉財用,司存,或給之。監司、州縣長官曰印,僚屬曰記。又下無記者,止令本道給以木朱記,文大方寸。或銜命出境者,以奉使印給之,復命則納於
 有司。後以朝命出州縣者,亦如之。新進士置團司,亦假奉使印,結局還之。此常制也。



 南渡之後,有司印記多亡失,彼遺此得,各自收用。尚方重鑄給之,加「行在」二字,或冠年號以別新舊,然欺偽猶未能革。乾道二年,禮部請郡縣假借印記者,悉毀而更鑄。四年,兵部侍郎陳彌作言:「六部印藏於官,以牌出入,而胥史用於戶外,或借用於他廳。近有偽為文符、盜印以支錢糧者,有偽作奏鈔、盜拆御寶而改秩者,皆慢藏有以誨之。」詔三省申嚴戒
 敕。紹熙元年,禮部侍郎李巘言:「文書有印,以示信防奸,給毀悉經省部,具有條制。然州縣沿循,或以縣佐而用東南將印,以掾曹而用司寇舊章,名既不正,弊亦難防。請令有司制州縣官合用印記,舊印非所當用者,毀之。」



 紹興十四年,臣僚又言:「印信事重,凡有官司印記,年深篆文不明,合改鑄者,非進呈取旨,不得改鑄焉。」時更鑄者,成都府錢引,每界以銅朱記給之。行在都茶場會子庫,每界給印二十五:國用印三鈕,各以「三省戶房國用
 司會子印」為文;檢察印五鈕,各以「提領會子庫檢察印」為文;庫印五鈕,各以「會子庫印造會子印」為文;合同印十二鈕,內一貫文二鈕,各以「會子庫一貫文合同印」為文;五百文、二百文準此。



 蕃國效順者,給以銅印。安南國王李天祚乞印,以「安南國王之印」六字為文,方二寸,給牌,皆以銅鑄,金塗。西蕃隴右郡王趙懷恩乞印,以「隴右郡王之印」為文給之。宜州界外諸蠻乞印,以「宜州管下羈縻某州之印」為文,凡六十顆給之。其後文武百司節
 次所鑄,不備載。



 朱記,同舊制。紹興二年,始鑄親賢宅、益王府銅朱記。二十七年,改鑄建康戶部大軍庫記。三十年,鑄馬軍司統制、統領官朱記。三十二年,鑄鄧、恭、慶王直講、贊讀朱記。隆興元年,鑄都督府僉廳記,又鑄寄樁庫記。二年,鑄戶部大軍庫勘合庫子記二鈕,湖廣總領所覆印會子記二鈕。乾道二年,鑄成都錢引務朱記。淳熙十六年,鑄建康榷貨務中門大門之記。凡內外官有請於朝,則鑄給焉。用木者,易之以銅。



 符券。唐有銀牌,發驛遣使,則門下省給之。其制,闊一寸半,長五寸,面刻隸字曰「敕走馬銀牌」,凡五字。首為竅,貫以韋帶。其後罷之。宋初,令樞密院給券,謂之「頭子」。太宗太平興國三年,李飛雄詐乘驛謀亂,伏誅。詔罷樞密院券,乘驛者復制銀牌,闊二寸半,長六寸。易以八分書,上鈒二飛鳳,下鈒二麒麟,兩邊年月,貫以紅絲絳。端拱中,以使臣護邊兵多遺失,又罷銀牌,復給樞密院券。



 仁宗康定元年五月,翰林學士承旨丁度、翰林學士王堯臣、
 知制誥葉清臣等請制軍中傳信牌及兵符事,詔令兩制與端明殿學士李淑詳定,奏聞:



 軍中符信,切要杜絕奸詐,深合機宜。今請下有司造銅兵符,給諸路總管主將,每發兵三百人或全指揮以上即用。又別造傳信朱漆木牌,給應軍中往來之處,每傳達號令、關報會合及發兵三百人以下即用。又檢到符彥卿《軍律》有字驗,亦乞令於移牒、傳信牌上,兩處參驗使用。



 一、銅兵符:漢制,銅鑄,上刻虎形。今聞皇城司見有木魚契,乞令有司用
 木契形狀,精巧鑄造。陜西五路,每路依漢制各給一至二十,計二十面,更換給用,仍以公牒為照驗。



 二、傳信木牌:先朝舊制,合用堅木朱漆為之,長六寸,闊三寸,腹背刻字而中分之,字云某路傳信牌。卻置池槽,牙縫相合。又鑿二竅,置筆墨,上帖紙,書所傳達事。用印印號上,以皮系往來軍吏之項。臨陣傳言,應有取索,並以此牌為信,寫其上。如已曉會施行訖,復書牌上遣回。今乞下有司造牌,每路各給一面為樣,餘令本司依此制造,分給
 諸處,更換使用。城砦分屯軍馬,事須往來關會之處,亦如數給與。



 三、字驗:凡軍行計會,不免文牒,或主司遺失懼罪,單使被擒,軍中所謀,自然洩露。故每分屯軍馬之時,與主將密定字號,各掌一通,不令左右人知其義理。但於尋常公狀文移內,以此字私為契約,有所施行,依此參驗。不得字有重疊,及用兇惡嫌疑之語。每用文牒之上,別行寫此字驗,訖,印其上發往。如所請報,到,許,即依號卻寫印遣回;如不許,即空之。此惟主將自知,他人
 皆不得測。符彥卿元用四十條,以四十字為號;今檢得只有三十七條,內亦有不急之事,今減作二十八字。所貴軍中戎旅之人,事簡易記。



 詔並從之。嘉祐四年,三司使張方平編驛券則例,凡七十四條,賜名《嘉祐驛令》。



 神宗熙寧五年,詔西作坊鑄造諸銅符三十四副,令三司給左契付諸門,右契付大內鑰匙庫。今後諸門輪差人員,依時轉銅契入,赴庫勘同。其鐵牌只請人自執,在外仗止宿。本庫依漏刻發鑰匙,付外仗驗請人鐵牌給付,
 候開門訖,卻執鐵牌納鑰匙,請出銅契。至晚卻依上請納。其開門朝牌六面,亦隨銅契依舊發放。時神宗以京城門禁不嚴,素無符契,命樞密院約舊制,更造銅契,中刻魚形,以門名識之,分左右給納,以戒不虞,而啟閉之法密於舊矣。元豐元年,詳定禮文所言:「舊南郊式,車駕出入宣德門、太廟靈星門、朱雀門、南熏門,皆勘箭。熙寧中,因參知政事王珪議,已罷勘箭,而勘契之式尚存。《春秋》之義,不敢以所不信加之尊者;且雷動天行,無容疑
 貳,必使誰何而後過門,不應典禮。考詳事始,不見於《開寶禮》。咸平中,初載於儀注,蓋當時禮官之失。請自今車駕出入,罷勘契。」從之。



 高宗建炎三年,改鑄虎符,樞密院主之。其制以銅為之,長六寸,闊三寸,刻篆而中分之,以左契給諸路,右契藏之。



 門符制,以繒裹紙版,謂之「號」,皇城司掌之。敕入禁衛號,黃綾八角,三千道;入殿門黃絹以方,一千道;入宮門黃絹以圓,八千道;入皇城門黃絹以長,三千道。紹興二年正月所定也。後更宮門號以緋
 紅絹方,皇城門以緋紅絹圓,遂久用之。後復盡以黃,或方或圓,各隨其制。



 又有檄牌,其制有金字牌、青字牌、紅字牌。金字牌者,日行四百里,郵置之最速遞也;凡赦書及軍機要切則用之,由內侍省發遣焉。乾道末,樞密院置雌黃青字牌,日行三百五十里,軍期急速則用之。淳熙末,趙汝愚在樞筦,乃作黑漆紅字牌,奏委諸路提舉官催督,歲校遲速最甚者,以議賞罰。其後尚書省亦踵行之,仍命逐州通判具出入界日時狀申省。久之,稽緩
 復如故。紹熙末,遂置擺鋪焉。



 宮室。汴宋之制,侈而不可以訓。中興,服御惟務簡省,宮殿尤樸。皇帝之居曰殿,總曰大內,又曰南內,本杭州治也。紹興初,創為之。休兵後,始作祟政、垂拱二殿。久之,又作天章等六閣。寢殿曰福寧殿。淳熙初,孝宗始作射殿,謂之選德殿。八年秋,又改後殿擁舍為別殿,取舊名,謂之延和殿,便坐視事則御之。他如紫宸、文德、集英、大慶、講武,惟隨時所御,則易其名。紫宸殿,遇朔受朝則御焉;
 文德殿,降赦則御焉;集英殿,臨軒策士則御焉;大慶殿,行冊禮則御焉;講武殿,閱武則御焉。其實垂拱、崇政二殿,權更其號而已。二殿雖曰大殿,其修廣僅如大郡之設廳。淳熙再修,止循其舊。每殿為屋五間,十二架,修六丈,廣八丈四尺。殿南簷屋三間,修一丈五尺,廣亦如之。兩朵殿各二間,東西廊各二十間,南廊九間。其中為殿門,三間六架,修三丈,廣四丈六尺。殿後擁舍七間,即為延和,其制尤卑,陛階一級,小如常人所居而已。



 奉太上
 則有德壽宮、重華宮、壽康宮,奉聖母則有慈寧宮、慈福宮、壽慈宮。德壽宮在大內北望仙橋,故又謂之北內,紹興三十二年所造,宮成,詔以德壽宮為名,高宗為上皇御之。重華宮即德壽宮也,孝宗遜位御之。壽康宮即寧福殿也。初,丞相趙汝愚議以秘書省為泰寧宮,已而不果行,以慈懿皇后外第為之。上皇不欲遷,因以舊寧福殿為壽康宮,光宗遜位御之。



 大內苑中,亭殿亦無增,其名稱可見者,僅有復古殿、損齋、觀堂、芙蓉閣、翠寒堂、清華
 閣、欏木堂、隱岫、澄碧、倚桂、隱秀、碧琳堂之類,此南內也。北內苑中,則有大池,引西湖水注之,其上疊石為山,像飛來峰。有樓曰聚遠,禁□周回,四分之。東則香遠、清深、月臺、梅坡、松菊三徑、清妍、清新、芙蓉岡,南則載忻、欣欣、射廳、臨賦、燦錦、至樂、半丈紅、清曠、瀉碧,西則冷泉、文杏館、靜樂、浣溪,北則絳華、旱船、俯翠、春桃、盤松。



 皇太子宮曰東宮。其未出閣,但聽讀於資善堂,堂在宮門內。已受冊,則居東宮,宮在麗正門內。紹興三十二年始置,孝宗
 居之;莊文太子立,復居之。光宗為太子,孝宗謂輔臣曰:「今後東宮不須創建,朕宮中宮殿,多所不御,可移修之。」自是皆不別建。



 淳熙二年,始創射堂一,為游藝之所,圃中有榮觀玉淵清賞等堂、鳳山樓,皆宴息之地也。



 幕殿,即《周官》大、小次也。東都時,郊壇大次謂之青城,祀前一日宿齋詣焉。其制,中有二殿,外有六門:前曰泰禋,後曰拱極,東曰祥曦,西曰景曜,東偏曰承和,西偏曰迎禧。大殿曰端誠,便殿曰熙成。中興後,以事天尚質,屢詔郊壇
 不得建齋宮,惟設幕屋而已。其制,架木而以葦為障,上下四旁周以幄帟,以象宮室,謂之幕殿。及行事,又於壇所設小次。大、小次之外,又有望祭殿,遇雨則行事於中。東都時為瓦屋五間,周圍重廊。中興後,惟設葦屋,蓋仿清廟茅屋之制也。



 臣庶室屋制度。宰相以下治事之所曰省、曰臺、曰部、曰寺、曰監、曰院,在外監司、州郡曰衙。在外稱衙而在內之公卿、大夫、士不稱者,按唐制,天子所居曰衙,故臣下不
 得稱。後在外藩鎮亦僭曰衙,遂為臣下通稱。今帝居雖不曰衙,而在內省部、寺監之名,則仍唐舊也。然亦在內者為尊者避,在外者遠君無嫌歟?私居,執政、親王曰府,餘官曰宅,庶民曰家。



 諸道府公門得施戟,若私門則爵位穹顯經恩賜者,許之。在內官不設,亦避君也。



 凡公宇,棟施瓦獸,門設梐枑。諸州正牙門及城門,並施鴟尾,不得施拒鵲。六品以上宅舍,許作烏頭門。父祖舍宅有者,子孫許仍之。凡民庶家,不得施重栱、藻井及五色文採
 為飾,仍不得四鋪飛簷。庶人舍屋,許五架,門一間兩廈而已。



\end{pinyinscope}