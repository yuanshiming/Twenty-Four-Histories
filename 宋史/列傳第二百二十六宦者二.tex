\article{列傳第二百二十六宦者二}

\begin{pinyinscope}

 ○楊守珍韓守英藍繼宗張惟吉甘昭吉盧守勤王守規李憲張茂則宋用臣王中正李舜舉
 石得一梁從吉劉惟簡



 楊守珍,字仲寶,開封祥符人。為入內黃門,習書史,學兵家方略。善射,家僮過堂下,一發貫髻,人服其精。選為環慶路走馬承受公事。契丹謀入塞,為鎮、定、高陽關行營同押先鋒事。會許民周繼宗為人誣告與外夷交通,乾證者六十人,辭服,遣守珍覆問,悉辦理出之。徙真定、保、趙等州駐泊都監,邕、桂等十州安撫都監。從曹克明降撫水州蠻,築二柵以扼其要。天禧初,擒盜於青灰山。累
 遷西京作坊使、帶御器械、永興軍兵馬鈐轄,徙真定、邠寧路。為內侍省內侍押班,提點內弓箭軍器庫。進內園使、右班都知、領端州刺史。嘗侍仁宗苑中,命乘馬馳射,賞其便習,賜錦袍UJ酒。卒,贈原州防禦使。



 韓守英,字德華,開封祥符人。初為入內高品,從征河東,數奉詔至石嶺關督戰,取隆州,遷殿頭。久之,以西頭供奉官擢入內內侍押班,遷副都知。隨王繼恩招安西川,為先鋒,戰於劍門有功,遷西京作坊使、劍門都監。還,勾
 當三班院,進入內內侍都知。歷定州、鎮定高陽關、並代路兵馬鈐轄。契丹圍岢嵐軍,守英與鈐轄張志言、知府州折惟昌帥所部渡河,抵朔州,以牽賊勢。遂破狼水砦,俘數百人,獲馬牛羊鎧甲以數萬計,賊為解去。賜錦袍、金帶。俄領會州刺史,解都知。再遷昭宣使,復領三班。



 出為鄜延路都鈐轄,徙並代路。建言:「本路宿兵多,百姓困於飛挽,今幸邊鄙無事,請留騎軍千,餘人悉徙內地。」真宗曰:「邊臣能體朝廷恤民之意,宜詔諸路視此行之。」



 提
 舉在京諸司庫務,勾當皇城司,為趙德明官告使。歷宣政、宣慶二使,內侍左班都知,領獎州團練使、雅州防禦使,入內都知,管勾修國史。書成,進景福殿使,又為延福宮使、入內都知,復提舉諸司庫務。卒贈定國軍節度觀察留後。



 藍繼宗,字承祖,廣州南海人。事劉鋹為宦者,歸朝,年十二,遷為中黃門。從征太原,傳詔營陳間,多稱旨。



 秦州並邊有大、小洛門砦,自唐末陷西羌。雍熙中,溫仲舒諭酋
 豪使獻其地,徒眾渭北。言者以為生事,請罷仲舒。太宗遣繼宗往按視,還奏二砦據要害,產良木,不可棄。帝悅,復使繼宗勞賜仲舒。累遷西京作坊副使、勾當內東門。



 元德太后、章穆皇后葬,為按行園陵使。車駕北征,勾當留司、皇城司。車駕謁諸陵,近陵舊乏水,繼宗疏泉陵下,百司從官皆取以濟。擢入內副都知,為天書扶侍都監。詔與李神祐第東封扈從內臣之勞,而入內供奉官範守遜等訴其不公,罷都知。祀汾脽,復為天書扶侍都監,
 再遷東染院使。



 明年,領會州刺史,進崇儀使、勾當皇城司。修玉清昭應宮,與劉承珪典工作。宮成,遷洛苑使、高州團練使,充都監。坐章穆皇后陵隧墊,貶如京使。典修景靈宮,進南作坊使,復修會靈、祥源觀。車駕幸亳州,管勾留司、大內公事,提舉在京諸司庫務,勾當三班院,修國史院。為趙德明加恩使,德明與繼宗射,繼宗每發必中,德明遺以所乘名馬。為內侍省右班都知,遷入內都知。



 仁宗即位,遷左騏驥使、忠州防禦使、永定陵修奉鈐
 轄。歷昭宣、宣政、宣慶使。累上章求致仕,特免入朝拜舞及從行幸。頃之,復固請罷都知,以景福殿使、邕州觀察使家居養疾。卒,贈安德軍節度使,謚僖靖。



 繼宗事四朝,謙謹自持,每領職未久,輒請罷。家有園池,退朝即亟歸,同列或留之,繼宗曰:「我欲歸種花卉、弄游魚為樂爾。」景福殿置使,自大中祥符間至繼宗,授者才三人。養子元用、元震。



 元用終左藏庫使、梓州觀察使。



 元震以兄蔭補入內黃門,轉高班,給事明肅太后。禁中夜火,後擁仁宗
 登西華門,左右未集,元震獨傳呼宿衛,以功遷高品。為三陵都監,條列防守法,其後諸陵以為式。歷群牧都監,監三館秘閣,積官皇城使。累遷入內副都知、忠州防禦使。仙韶院火,元震救護,火以時息。詔褒之,賜襲衣、金帶。卒,贈鎮海軍留後。元震養子五人,不畜閹子。



 張惟吉,字祐之,開封人。初補入內黃門,遷殿頭、高陽關路走馬承受公事。護塞滑州天臺埽役,遷西頭供奉官,監在京榷貨務。知嘉州張約以贓敗,詔與御史王軫往
 劾其獄。還,領內東門司,為修奉章獻、章懿太后二陵承受。時議復用李諮榷茶算緡法,乃以惟吉為內殿崇班,復監榷貨務。凡內侍領內東門,次遷勾當御藥院,而惟吉才進官,眾以為薄,惟吉欣然就職。再期,以羨餘遷承制。



 為趙元昊官告使,還,言元昊驕僭,勢必叛,請預飭邊備。及元昊寇延州,遣按視鄜延、環慶兩路器甲,並訪攻守利害。敵既退,夏竦、韓琦謀自鄜延深入,乘虛擊之,命惟吉募並、汾驍勇,副以土兵,輕齎赴河外。惟吉以為我
 師當持重伺變,不宜馳赴不測以自困,已而元昊果引去。還奏稱旨,領皇城司,遷內侍省押班、群牧都監,簡陜西冗兵,領軍頭引見司,遷供備庫使,盡汰軍頭司軍校之罷癃者。同提舉在京諸司庫務,領恩州刺史,為入內都知。



 商胡決,為澶州修河都鈐轄。轉運使施昌言請亟塞,崔嶧以為歲災民困,役宜緩。命惟吉按視,言河可塞而民誠困,財用不足,宜少待之。從其議。遷如京使、果州團練使,復領皇城司,卒。



 惟吉任事久。頗見親信,而言弗
 阿徇。張貴妃薨,將治喪皇儀殿,諸宦官皆以為可,獨惟吉曰:「此事干典禮,須翌日問宰相。」既而宰相不能執議,惟吉深以為非。贈昭信軍節度觀察留後。逾月,又贈保順軍節度使,謚忠安。



 養子若水,字益之,以惟吉奏補小黃門,給事章惠太后殿,轉入內高品。王師平貝州,徵儂賊,皆以幹敏選為走馬承受。賊平,以勞進官,三遷環慶路鈐轄。討環州解乜臼族復有功,歷帶御器械、內侍押班、副都知。



 熙寧初,造神臂弓成,神宗御延和殿臨閱,置
 鐵甲七十步,俾衛士射,未有中者。若水自請射,連中徹札。建慶壽、寶慈兩宮,典領工作,再遷嘉州防禦使。以病蘄解職,領輝州觀察使,提舉四園苑諸司庫務。卒,贈天平軍留後。



 甘昭吉,字祐之,開封人。初以內侍殿頭為英、韶州巡檢,捕盜有功,再遷內殿崇班、京東路都巡檢。齊州武衛小校馮坦率營卒二百突入州廳事,欲為變,昭吉單騎馳往,戒所從將士操兵在外,先獨見亂卒,諭以福禍,令推
 首惡自贖,眾疑沮不敢動。已而操兵者皆入,即共執十餘人,告曰:「此誘我者也。」昭吉立殺之,縱其餘去,州以無事。特廷供備庫副使、帶御器械。後內侍省押班闕,仁宗記前功,特以授之。遷入內副都知。



 英宗即位之夕,昭吉直禁中,翊衛有勞,自文思副使超遷供備庫使、康州刺史。昭吉奏曰:「臣本孤微,無左右之舉,而先帝知臣樸直,自小官拔用至此,分當從葬,今願得灑掃陵寢足矣。」帝愛其忠,特授永昭陵使,加如京使。還朝,表辭職,以左龍
 武軍大將軍致仕,卒。昭吉敦實慎密,人士稱之。



 盧守勤,字君錫,開封祥符人。自入內內品累遷禮賓使、邠寧環慶路鈐轄,還為入內內侍省押班、領昌州刺史。明道中,改葬章懿太后,而舊藏有水,以守勤嘗典葬事,罷為永興軍兵馬鈐轄,徙鄜延路。再遷六宅使,加貴州團練使,進榮州防禦使兼邠寧環慶路安撫都監。元昊寇保安軍,守勤率兵擊走之,特遷左騏驥使,移陜西鈐轄。



 初,劉平、石元孫被執,守勤撫膺涕泣不敢出,又嘗
 易蕃官馬。延州通判計用章勸範雍棄城,將保鄜州,雍欲遣安撫都監李康伯往說賊,不肯行,賊去而守勤、用章更相論奏。知制誥葉清臣以守勤擁兵觀望,請正其罪,並按二人。守勤奪防禦使,為湖北都監;用章除籍,配雷州本城;康伯,均州都監。



 久之,復恩州防禦使,遷利州觀察使,歷真定府、定州、北京路鈐轄。以左衛大將軍致事,卒,贈保順軍節度使,謚安恪。養子昭序。



 王守規,真定欒城人,入內都都知守忠之弟。守忠事真
 宗,謹願慎密,眷遇最厚。明道時,守規為小黃門,禁中夜半火,守規先覺,自寢殿至後苑皆擊去其鎖,乃奉仁宗及皇太后至延福宮,回視所經處已成煨燼。翌日,執政候起居,帝曰:「非王守規導朕至此,幾不與卿等相見。」以功遷入內殿頭。選治京城水,決汴河於公賈村,決蔡河於四里橋,水患以息。加帶御器械。積官至宣慶使、康州防禦使、內侍右班副都知。卒,年六十七,贈昭武軍留
 後。



 李憲,字子範,開封祥符人。皇祐中,補入內黃門,稍遷供奉官。神宗即位,歷永興、太原府路走馬承受,數論邊事合旨,乾當後苑。王韶上書請復河湟,命憲往視師,與韶進收河州,加東染院使,乾當御藥院。復戰牛精谷,拔珂諾城,為熙河經略安撫司乾當公事。按視鄜延軍制,行至蒲中,會木征合董氈、鬼章之兵攻破踏白城,殺景思立,圍河州,詔趣赴之,憲馳至軍。先是,朝廷出黃旗書敕諭將士,如用命破賊者倍賞。於是憲晨起帳中,張以示
 眾曰:「此旗,天子所賜也,視此以戰,帝實臨之。」士爭呼用命以進。督諸將傍山焚族帳,即日通路至河州。賊餘眾保踏白,官軍出與戰,大破之。進至餘川,又破賊堡十餘,木征率酋長八十餘人詣軍門降。捷聞,以功加昭宣使、嘉州防禦使。還,為入內內侍省押班、乾當皇城司。



 安南叛,副趙UK招討,未行,UK建言:「朝廷置招討副使,軍事須共議,至節制號令即宜歸一。」憲銜之。由是屢紛辨,遂罷憲而令乘驛計議秦鳳、熙河邊事,諸將皆聽節度。於是
 御史中丞鄧潤甫、御史周尹、蔡承禧、彭汝礪極論其不可,又言:「鬼章之患小,用憲之患大;憲功不成其禍小,有成功其禍大。」章再上,弗聽。冷雞樸誘山後生羌擾邊,木征請自效,眾以為不可。憲曰:「何傷乎!羌人天性畏服貴種。」聽之往。木徵盛裝以出,眾聳視,皆無鬥志,師乘之,殺獲萬計,斬冷雞樸。董氈懼,即遣使奉贄效順。加宣州觀察使、宣政使、入內副都知,又遷宣慶使。時用兵連年,度支調度不繼,詔憲兼經制財用,裁冗費什六,歲運西山
 巨木給京師營繕。賜瑞應坊園宅一區。



 元豐中,五路出師討夏國,憲領熙、秦軍至西市新城。復蘭州,城之,請建為帥府。帝又詔憲領兵直趣興、靈,董氈亦稱欲往,宜乘機協助力入掃巢穴,若興、靈道阻,即過河取涼州。乃總兵東上,平夏人於高川石峽。進至屈吳山,營打囉城,趨天都,燒南牟府庫,次葫蘆河而還。



 憲既不能至靈州,董氈亦失期,師無功。憲欲以開蘭、會邀功弭責,同知樞密院孫固曰:「兵法,期而後至者斬。況諸路皆至而憲獨
 不行,不可赦。」帝以憲猶有功,但令詰擅還之由,憲以饋餉不接為辭,釋弗誅。復上再舉之策,兼陳進築五利,且從之。會李舜舉入奏,具陳師老民困狀,乃罷兵。趣憲赴闕,道賜銀帛四千。為涇原經略安撫制置使,給衛三百。進景福殿使、武信軍留後,使復還熙河,仍兼秦鳳軍馬。



 夏人入蘭州,破西關,降宣慶使。憲以蘭州乃西人必爭地,眾數至河外而相羊不進,意必大舉,乃增城守塹壁,樓櫓具備。明年冬,夏人果大入,圍蘭州,步騎號八十萬眾,十
 日不克,糧盡引去。又詔憲遣間諭阿裏骨結等,且選騎渡河,與賊遇,破之。坐妄奏功狀,罷內省職事。



 哲宗立,改永興軍路副都總管,提舉崇福宮。御史中丞劉摯論憲貪功生事,一出欺罔,避興、靈會師之期,頓兵以城蘭州,遺患至今,永樂之圍,逗留不急赴援。降宣州觀察使,又貶右千牛衛將軍,分司南京,居陳州。卒,年五十一。紹聖元年,贈武泰軍節度使,初謚敏恪,改忠敏。



 憲以中人為將,雖能拓地降敵,而罔上害民,終貽患中國云。



 張茂則,字平甫,開封人。初補小黃門,五遷至西頭供奉官,乾當內東門。禁庭夜有盜,茂則首登屋以入,既獲賊,遷領御藥院。



 仁宗不豫,中夜促召,茂則趨入扶衛,左右或欲掩宮門,茂則曰:「事無可慮,何至使中外生疑耶?」帝疾間,欲處以押班,懇求補外,轉宮苑使、果州團練使,為永興路兵馬鈐轄。入為內侍押班,再遷副都知。熙寧初,同司馬光相視恩、冀、深、瀛四州生堤及六塔、二股河利害,進入內都知。



 上元夜,宮中火,督眾即撲滅。詔曰:「宮禁
 不驚,帑藏如故,惟忠與力,予固嘉之。」賜以窄衣金帶。累乞退休,言受國厚恩,廩食過量,積而未請者七年,乞令三司毀券。詔褒之,仍進其官。哲宗即位,遷寧國軍留後,加兩省都都知。卒,年七十九。



 茂則性儉素,食不重味,衣裘累十數年不易。紹聖論元祐人,以茂則嘗預任使,追貶左監門衛將軍,崇寧中入黨籍。



 宋用臣,字正卿,開封人。為人有精思強力,以父蔭隸職內省。神宗建東、西府,築京城,建尚書省,起太學,立原廟,
 導洛通汴,凡大工役,悉董其事。性敏給,善傳詔令,故多訪以外事。同列悉籍以進,朝士之乏廉節者,往往諂附之,權勢震赫一時。積勞至登州防禦使,加宣政使。元祐初,言者論其罪,降為皇城使,謫監滁州、太平州酒稅。四年,主管靈仙觀。紹聖初,召為內侍押班,進瀛州刺史。



 徽宗即位,遷蔡州觀察使、入內副都知。為永泰陵修奉鈐轄,卒陵下,贈安化軍節度使,謚僖敏。謚議謂用臣為廣平宋公,有「天子念公之勞,久徙於外」之語。豐稷論奏,以
 為凡稱公者皆須耆宿、大臣與鄉黨有德之士,其曰:「念公之勞,久徙於外」,斯乃古周公之事,於用臣非所宜言也。止令賜謚,論者是之。



 王中正,字希烈,開封人。因父任補入內黃門,遷赴延福宮學詩書、歷算。仁宗嘉其才,命置左右。慶歷衛士之變,中正援弓矢即殿西督捕射,賊悉就擒,時年甫十八,人頗壯之。遷東頭供奉官,歷乾當御藥院、鄜延、環慶路公事,分治河東邊事。破西人有功,帶御器械。



 神宗將復熙
 河,命之規度。還言:「熙河譬乳虎抱玉,乘爪牙未備,可取也。」遂從王韶入熙河,治城壁守具,以功遷作坊使、嘉州團練使,擢內侍押班。



 吐蕃圍茂州,詔率陜西兵援之,圍解。自石泉至茂州,謂之隴東路,土田肥美,西羌據有之,中正不能討。乃因吐蕃入寇,言:「其路經靜州等族,棒僻不通,邇年商旅稍往來,故外蕃因以乘間。縣至綿與茂,道里均,而龍安有都巡檢,緩急可倚仗。請割石泉隸綿,而窒其故道。」從之,隴東遂不可得。還,使熙河經畫鬼章,
 進昭宣使、入內副都知。



 元豐初,提舉教畿縣保甲將兵捕賊盜巡檢,獻民兵伍保法,請於村畽及縣以時閱習,悉行其言。復往鄜延、環慶經制邊事,詔凡所須用度,令兩路取給,無限多寡。既行,又稱面受詔,所過募禁兵,願從者將之,主者不敢違。



 問罪西夏,以中正簽書涇原路經略司事。詔五路之師皆會靈州,中正失期,糧道不繼,士卒多死,命權分屯鄜延並邊城砦,以俟後舉。自請罷省職,遷金州觀察使、提舉西太一宮,坐前敗貶秩。元祐
 初,言者再論其將王師二十萬,公違詔書之罪,劉摯比中正與李憲、宋用臣、石得一為四兇,又貶秩兩等。久之,提舉崇福宮。紹聖初,復嘉州團練使。卒,年七十一。



 李舜舉,字公輔,開封人。世為內侍,曾祖神福,事太宗以信謹終始。舜舉少補黃門,仁宗使督工冶金為器,既成,有羨數並上之,帝嘉其不欺。出為秦鳳路走馬承受。



 英宗立,奏事京師。會帝不豫,內謁者止之宮門,舜舉曰:「天子新即位,使者從邊方來,不得一見而去,何以慰遠人!」
 謁者以聞,亟召對,帝意良悅。因言:「承受公事,以察守將不法為職,而終更論最,乃使帥臣保任,乞免之。」遂刪舊制。



 熙寧中,歷乾當內東門、御藥院、講筵閣、實錄院。郭逵討交州,以為廣西乾當公事,軍中之政得與講畫,或疾置入朝,稟受成算。會逵貶,亦降左藏庫副使,以文思院使領文州刺史、帶御器械。進內侍押班,制置涇原軍馬。



 五路師出無功,議再舉,李憲督饋糧,言受密詔,自都轉運使以下乏軍興者皆聽斬。民懲前日之役多死於凍
 餒,皆憚行,出錢百緡不能雇一夫,相聚立柵山澤不受調,吏往逼呼,輒毆擊,解州至械縣令以督之,不能集。舜舉入奏其事,乃罷兵。退詣中書,王珪迎勞之曰:「朝廷以邊事屬押班及李留後,無西顧之憂矣。」舜舉曰:「四郊多壘,此卿大夫之辱,相公當國,而以邊事屬二內臣,可乎?內臣正宜供禁庭灑掃之職,豈可當將帥之任!」聞者代珪慚焉。



 轉嘉州團練使。沈括城永樂,遣舜舉計議,被圍急,斷衣襟作奏曰:「臣死無所恨,願朝廷勿輕此賊。」尋以
 死聞,贈昭信軍節度使,謚曰忠敏。



 舜舉資性安重,與人言未嘗及宮省事。頗覽書傳,能文辭筆札。在御藥院十四年,神宗嘗書「李舜舉公忠奉上,恭勤檢身,始終惟一,以安以榮。」十九字賜之。



 石得一,開封人。為內侍黃門,累官內殿承制。神宗時,帶御器械、管幹龍圖天章寶文閣、皇城司,四遷入內副都知。元祐初,領成州團練使,罷內省職。御史劉摯言:「得一頃筦皇城,恣其殘刻,縱遣邏者,所在棋布,張阱設網,以
 無為有,以虛為實。朝廷大吏及富家小人,飛語朝上,暮入狴犴,上下惴恐,不能自保,至相顧以目者殆十年。」坐降左藏庫使,卒。紹聖中,贈隨州觀察使。



 梁從吉,字君祐,開封人。補入內高班。王則反,奉命宣慰,還言:「小寇無多慮,諸將之兵足以翦除,若得重臣統其事,不崇朝可平矣。」於是仁宗以文彥博為安撫招討使。賊平,又奏請分河北為路,每路以一帥府統之,遂建魏、鎮、定、瀛四帥。熙寧初,為邠寧環慶路駐泊兵馬鈐轄。夏
 人寇大順城,圍慶州七砦,從吉率兵八百餘人與戰,獲其酋領。又討平寧州叛卒,以功升都鈐轄,累官皇城使。從高遵裕至靈武,督士卒攻城,身被創甚,進入內押班,遷永州團練使,為副都知。元祐中卒,贈成德軍節度使,謚曰敏恪。



 劉惟簡,開封人,由入內黃門積官至昭宣使、康州刺史、高陽關路兵馬都監,為入內押班。英宗初立,惟簡自河北來朝,請對寢門,內謁者難之,獨引見皇太后。惟簡立
 福寧殿下,雨沾衣不退,帝起坐幃中,望見呼問曰:「諸路如汝者幾人,何以獨來?」對曰:「陛下新即位,臣來自邊塞,未瞻天表,不敢輒還,不知其他。」帝嘆曰:「小臣知所守如此。」識其姓名屏間。他日,神宗覽所題屏,擢幹當延福宮,自是蒙親信。



 交人叛,詔馳驛至桂州審視事勢,還言:「帥臣劉彞貪功生事,罪當誅。乾德狂童,頸不足系。」帝信之。郭逵、趙UK南征,以為行營承受。逵、UK被謫,惟簡亦奪一官。



 陜西五路師還,受命撫犒士卒,以疾先還者不賜。惟
 簡心知其不便,至慶州,疏言:「士卒不幸,以將臣上違聖略,糧食不繼,逃生以歸,其情可貸。今同立庭中而不預賜,恐患生倉卒。」帝用其言,均予之。又使案閱河北保甲,振濟京西水災,參定諸陵薦獻。既而為言者所劾,擯不用。哲宗在藩時,惟簡奔奏服勤,及親政,召至左右。以內侍押班卒,贈昭化軍留後。



\end{pinyinscope}