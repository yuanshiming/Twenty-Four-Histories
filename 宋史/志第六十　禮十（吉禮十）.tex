\article{志第六十 禮十(吉禮十)}

\begin{pinyinscope}

 禘祫



 宗廟之禮。每歲以四孟月及季冬,凡五享,朔、望則上食、薦新。三年一祫,以孟冬;五年一禘,以孟夏,唯親郊、封祀。
 又有朝享、告謝及新主祔謁,皆大祀也。二薦,則行一獻禮。其祔祭,春祀司命及戶,夏祀灶,季夏祀中溜,秋祀門及厲,冬祀行,惟臘享、禘祫則遍祀焉。



 禘祫之禮。真宗咸平二年八月,太常禮院言:「今年冬祭畫日,以十月六日薦享太廟。按《禮》,三年一祫,以孟冬。又《疑義》云:三年喪畢,遭禘則禘,遭袷則袷。宜改孟冬薦享為祫享。」仁宗天聖元年,禮官言:「真宗神主祔廟,已行吉祭,三年之制,又從易月之文,自天禧二年四月禘享,至今已及五年,合行
 禘禮。」遂以孟夏薦享為禘享。八年九月,太常禮院言:「自天聖六年夏行禘享之禮,至此年十月,請以孟冬薦享為祫享。」詔恭依。



 嘉祐四年十月,仁宗親詣太廟行祫享禮,以宰臣富弼為祫享大禮使,韓琦為禮儀使,樞密使宋庠為儀仗使,參知政事曾公亮為橋道頓遞使,樞密副使程戡為鹵簿使。同判宗正寺趙良規請正太祖東向位,禮官不敢決。觀文殿學士王舉正等議曰:「大祫之禮所以合昭穆,辨尊卑,必以受命之祖居東向之位。本
 朝以太祖為受命之君,然僖祖以降,四廟在上,故每遇大袷,止列昭穆而虛東向。魏、晉以來,亦用此禮。今親享之盛,宜如舊便。」



 禮官張洞、韓維言:「國朝每遇禘祫,奉別廟四后之主合食太廟。唐《郊祀志》載禘祫祝文,自獻祖至肅宗所配皆一後,惟睿宗二後,蓋昭成,明皇母也。《續曲臺禮》有別廟皇後合食之文,蓋未有本室,遇祫享即祔祖姑下。所以大順中,三太后配列禘祭,議者議其非禮。臣謂每室既有定配,則餘後不當參列,義當革正。」



 學
 士孫抃等議:「《春秋傳》曰:『大祫者何,合祭也。未毀廟之主皆升合食於太祖。』是以國朝事宗廟百有餘年,至祫之日,別廟後主皆升合食,非無典據。大中祥符中已曾定議,禮官著酌中之論,先帝有『恭依』之詔。他年有司攝事,四後皆預。今甫欲親祫而四後見黜,不亦疑於以禮之煩故邪?宗廟之禮,至尊至重,茍未能盡祖宗之意,則莫若守舊禮。臣等愚以謂如故便。」



 學士歐陽修等曰:「古者宗廟之制,皆一帝一後。後世有以子貴者,始著並祔之
 文,其不當祔者,則有別廟之祭。本朝禘祫,乃以別廟之後列於配後之下,非惟於古無文,於今又四不可。淑德,太宗之元配,列於元德之下;章懷,真宗之元配,列於章懿之下,一也。升祔之後,統以帝樂;別廟之後,則以本室樂章自隨,二也。升祔之後,同牢而祭,牲器祝冊亦統於帝;別廟諸後,乃從專享,三也。升祔之後,聯席而坐;別廟之後,位乃相絕,四也。章獻、章懿在奉慈廟,每遇禘祫,本廟致享,最為得禮。若四後各祭於廟,則其尊自申,是於
 禮無失。以為行之已久,重於改作,則是失禮之舉,無復是正也。請從禮官。」



 詔:「四後祫享依舊,須大禮畢,別加討論。」仍詔:「祫享前一日,皇帝詣景靈宮,如南郊禮,衛士毋得迎駕呼萬歲。」有司言:「諸司奉禮,攝廩犧令省牲,依《通禮》改正祀儀。散齋四日於別殿,致齋二日於大慶殿,一日於太廟。尚舍直殿下,設小次,御坐不設黃道褥位。七室各用一太牢,每坐簠簋二,□鉶三,籩豆為後,無黼扆、席幾。出三閣瑞石、篆書玉璽印、青玉環、金山陳於庭。別
 廟四後合食,牲樂奠拜無異儀。故事,七祀、功臣無牲,止於廟牲肉分割,知廟卿行事。請依《續曲臺禮》,共料一羊,而獻官三員,功臣單席,如大中祥符加褥。」



 十月二日,命樞密副使張忭望告昊天上帝、皇地祇。帝齋大慶殿。十一日,服通天冠、絳紗袍,執圭、乘輿,至大慶殿門外降輿,乘大輦,至天興殿,薦享畢,齋於太廟。明日,帝常服至大次,改袞冕,行禮畢,質明,乘大輦還宮,更服靴袍,御紫宸殿,宰臣、百官賀,升宣德門肆赦。二十一日,詣諸觀寺行
 恭謝禮。二十六日,御集英殿為飲福宴。



 治平元年,有司「準畫日,孟冬薦享改為祫祭。按《春秋》,閔公喪未除而行吉



 禘,《三傳》譏之。真宗以咸平二年六月喪除,至十月乃祫祭。天聖元年在諒陰,有司誤通天禧舊禘之數,在再期內按行禘祭。以理推之,是二年冬應祫,而誤禘於元年夏,故四十九年間九禘八祫,例皆太速。事失於始,則歲月相乘,不得而正。今在大祥內,禮未應袷,明年未禫,亦未應禘,至六月即吉,二月合行祫祭,乞依舊時享,庶
 合典禮。」



 二年二月,翰林學士王珪等上議曰:「同知太常禮院呂夏卿狀:古者新君踐阼之三年,先君之喪二十七月為禫祭,然後新主祔廟,特行禘祭,謂之始禘。是冬十月行袷祭,明年又行禘祭,自此五年,再為禘祫。喪除必有禘祫者,為再大祭之本也。今當袷祭,緣陛下未終三年之制,納有司之說,十月依舊時享。然享廟、袷祭,其禮不同。故事,郊享之年遇祫未嘗權罷,唯罷臘祭。是則孟享與享廟嘗並行於季冬矣。其禘祫
 年數,乞一依太常禮院請,今年十月行祫祭,明年四月行禘祭。仍如夏卿議。」權罷今年臘享。



 熙寧八年,有司言:「已尊僖祖為太廟始祖,孟夏禘祭,當正東向之位。」又言:「太廟禘祭神位,已尊始祖居東向之位,自順祖而下,昭、穆各以南北為序。自今禘祫,著為定禮。」



 元豐四年,詳定郊廟禮文所言:「禘祫之義,存於《周禮》、《春秋》,而不著其名。行禮之年,經皆無文,唯《公羊傳》曰:『五年而再盛祭。』《禮緯》曰:『三年一祫,五年一禘。』而又分為二說:鄭氏則曰:『前三
 後二』,謂禘後四十二月而祫,祫後十八月而禘。徐邈則曰:『前二後三』,謂二祭相去各三十月。以二說考之,惟鄭氏曰:『魯禮,三年喪畢,祫於太廟,明年禘於群廟,自後五年而再盛祭,一祫一禘。』實為有據。本朝慶歷初用徐邈說,每三十月一祭。熙寧八年,既禘而祫,此有司之失也。請今十八月而禘,禘四十二月而祫,庶幾舉禮不煩,事神不瀆。」太常禮院言:「本朝自慶歷以來,皆三十月而一祭。至熙寧五年後,始不通計,遂至八年禘祫並在一
 歲。昨元豐三年四月已行禘禮,今年若依舊例,十月行祫享,即比年頻袷,復踵前失。請依慶歷以來之制,通計年數,皆三十月而祭。」詔如見行典禮。



 詳定所又言:「古者稞獻、饋食,禴祠、烝、嘗,並為先王之享,未嘗廢一時之祭。故孔氏《正義》以為:『天子夏為大祭之禘,不廢時祭之礿;秋為大祭之祫,不廢時祭之嘗。』則王禮三年一袷與禘享,更為時祭。本朝沿襲故常,久未厘正,請每禘祫之月雖已大祭,仍行時享,以嚴天子備禮,所以丕崇祖宗之義。
 其郊禮、親祠準此。」



 又言:「《禮》:不王不禘。虞、夏、商、周四代所禘,皆以帝有天下,其世系所出者明,故追祭所及者遠也。太祖受命,祭四親廟,推僖祖而上所自出者,譜失其傳,有司因仍舊說,禘祫皆合群廟之主,綴食於始祖,失禮莫甚。今國家世系與四代不同,既求其祖之所自出而不得,則禘禮當闕,必推見祖系乃可以行。」神宗謂輔臣曰:「禘者,本以審禘祖之所自出,故禮,不王不禘。秦、漢以後,譜牒不明,莫知其祖之所自出,由禘禮可廢也。」



 已
 而詳定所言:「古者天子祭宗廟,有堂事焉,有室事焉。按《禮》,祝延尸入奧,灌後乃出延牲,延尸主出於室,坐於堂上,始祖南面,昭在東,穆在西,乃行朝踐之禮,是堂事也。設饌於堂,復延主入室,始祖東面,昭南穆北,徙常上之饌於室中,乃行饋食之禮,是室事也。請每行大袷,堂上設南面之位,室中設東面之位。」禮部言:「合食之禮,始祖東面、昭南穆北者,本室中之位也。今設位戶外,祖宗昭、穆別為幄次,殆非合食之義。請自今祫享,即前楹通設
 帳幕,以應室中之位。」



 大觀四年,議禮局請:「每大祫,堂上設南面之位,室中設東南之位,始祖南面則昭穆東西相向,始祖東面則昭穆南北相向,以應古義。」又請:「陳瑞物及代國之寶與貢物可出而陳者,並令有司依嘉祐、元豐詔旨,凡親祠太廟準此。」從之。



 南渡之後,有祫而無禘。高宗建炎二年,祫享於洪州。紹興二年,祫享於溫州。時儀文草創,奉遷祖宗及祧廟神主、別廟神主,各設幄合食於太廟。始祖東向,昭、穆以次南北相向。


五年,吏部
 員外郎董
 \gezhu{
  分廾}
 言:「臣聞戎、祀,國之大事,而宗廟之祭,又祀之大者也。大祀,禘祫為重,祫大禘小,則袷為莫大焉。今戎事方殷,祭祀之禮未暇遍舉,然事有違經戾古,上不當天地神祇之意,下未合億兆黎庶之心,特出於一時大臣好勝之臆說,而行之六十年未有知其非者。顧雖治兵御戎之際,正厥違誤,宜不可緩。仰惟太祖受天明命,混一區宇,即其功德所起,宜祇享以正東向之尊。逮至仁宗,親行祫享,嘗議太祖東向,用昭正統之緒。當時
 在廷之臣,僉謂自古必以受命之祖乃居東向之位,本朝太祖乃受命之君,若論七廟之次,有僖祖以降四廟在上,當時大祫,止列昭穆而虛東向,蓋終不敢以非受命之祖而居之也。暨熙寧之初,僖祖以世次當祧,禮官韓維等據經有請,適王安石用事,奮其臆說,乃俾章衡建議,尊僖祖為始祖,肇居東向。馮京奏謂士大夫以太祖不得東向為恨,安石肆言以折之。已而又欲罷太祖郊配,神宗以太祖開基受命,不許,安石終不以為然。元
 祐之初,翼祖既祧,正合典禮。至於崇寧,宣祖當祧,適蔡京用事,一遵安石之術,乃建言請立九廟,自我作古,其已祧翼祖、宣祖並即依舊。循沿至今,太祖尚居第四室,遇大祫處昭穆之列。今若正太祖東向之尊,委合《禮經》。」


太常寺丞王普又言:「
 \gezhu{
  分廾}
 所奏深得禮意,而其言尚有未盡。臣竊以古者廟制異宮,則太祖居中,而群廟列其左右;後世廟制同堂,則太祖居右,而諸室皆列其左。古者祫享,朝踐於堂,則太祖南向,而昭穆位於東西;饋食於
 室,則太祖東向,而昭穆位於南北。後世祫享一於堂上,而用室中之位,故唯以東向為太祖之尊焉。若夫群廟迭毀,而太祖不遷,則其禮尚矣。臣故知太祖即廟之始祖,是為廟號,非謚號也。惟我太宗嗣服之初,太祖廟號已定,雖更累朝,世次猶近,每於祫享,必虛東向之位,以其非太祖必不可居也。迨至熙寧,又尊僖祖為廟之始祖,百世不遷,祫享東向,而太祖常居穆位,則名實舛矣。儻以熙寧之禮為是,僖祖當稱太祖,而太祖當改廟號。
 然則太祖之名不正,前日之失大矣。今宜奉太祖神主居第一室,永為廟之始祖。每歲五享、告朔、薦新,止於七廟。三年一祫,則太祖正東向之位。太宗、仁宗、神宗南向為昭,真宗、英宗、哲宗北向為穆。五年一禘,則迎宣祖神主享於太廟,而以太祖配焉。如是,則宗廟之事盡合《禮經》,無復前日之失矣。」上曰:「太祖皇帝開基創業,始受天命,祫享宜居東向之位。」宰相趙鼎等奏曰:「三昭三穆,與太祖之廟而七,載在《禮經》,無可疑者。」



 紹熙五年九月,太
 常少卿曾三復亦言:請祧宣祖,就正太祖東向之位,其言甚切。既而吏部尚書鄭僑等亦乞因大行祔廟之際,定宗廟萬世之禮,慰太祖在天之靈,破熙寧不經之論。今太祖為始祖,則太宗為昭,真宗為穆,自是而下以至孝宗,四昭四穆與太祖之廟而九。上參古禮,而不廢崇寧九廟之制,於義為允。又言:「治平四年,僖祖祧遷,藏在西夾室。至熙寧五年,王安石以私意使章衡等議,乃復祔僖祖以為始祖,又將推以配天,欲罷太祖郊配。韓維、
 司馬光等力爭,而安石主其說愈堅。孫固慮其罷太祖配天,建議以僖祖權居東向之位。既曰權居,則當厘正明矣。」詔從之。



 閏十月,權禮部侍郎許及之言:「僖、順、翼、宣四祖,為太祖之祖考,所遷之主,恐不得藏於子孫之廟。今順、翼二祖藏於西夾室,實居太廟太祖之右。遇祫享,則於夾室之前,設位以昭穆焉。」於是詔有司集議,吏部尚書兼侍讀鄭僑等言:「僖祖當用唐興聖之制,立為別廟,順祖、翼祖、宣祖之主皆祔藏焉。如此,則僖祖自居別
 廟之尊,三祖不祔子孫之廟。自漢、魏以來,太祖而上,毀廟之主皆不合食,今遇祫,則即廟而享,於禮尤稱。」諸儒如樓鑰、陳傅良皆以為可,詔從之。



 時朱熹在講筵,獨入議狀,條其不可者四,大略云:「準尚書吏部牒,集議四祖祧主宜有所歸。今詳群議雖多,而皆有可疑。若曰藏之夾室,則是以祖宗之主下藏於子孫之夾室。至於祫祭,設幄於夾室之前,則亦不得謂之祫。欲別立一廟,則喪事即遠,有毀無立。欲藏之天興殿,則宗廟、原廟不可相
 雜。議者皆知其不安,特以其心欲尊奉太祖三年一袷時暫東向之故,其實無益於太祖之尊,而徒使僖祖、太祖兩朝威靈,相與校強弱於冥冥之中。今但以太祖當日追尊帝號之令而默推之,則知今日太祖在天之靈,必有所不忍而不敢當矣。又況僖祖祧主遷於治平,不過數年,神宗復奉以為始祖,已為得禮之正而合於人心,所謂『有其舉之,莫敢廢者』。」又言:「當以僖祖為始祖,如周之後稷,太祖如周之文王,太宗如周之武王,與仁宗
 之廟,皆萬世不祧;昭穆而次,以至高宗之廟亦萬世不祧。」又言:「元祐大儒程頤以為王安石言『僖祖不當祧』,復立廟為得禮。竊詳頤之議論與安石不同,至論此事則深服之,足以見義理人心之所同,固有不約而合者。特以司馬光、韓維之徒皆是大賢,人所敬信,其議偶不出此,而安石乃以變亂穿鑿得罪於公議,故欲堅守二賢之說,並安石所當取者而盡廢之。今以程頤之說考之,則是非可判矣。」



 議既上,召對,令細陳其說。熹先以所論
 畫為圖本,貼說詳盡,至是出以奏陳久之。上再三稱善,且曰:「僖祖自不當祧,高宗即位時不曾祧,壽皇即位,太上即位,亦不曾祧,今日豈可容易?可於榻前撰數語,徑自批出。」熹方懲內批之弊,因乞降出札子,再令臣僚集議,上亦然之。熹既退,即進擬詔意,以上意諭廟堂,則聞已毀四祖廟而遷之矣。



 時宰臣趙汝愚既以安石之論為非,異議者懼其軋己,藉以求勝,事竟不行。熹時以得罪,遺汝愚書曰:「相公以宗子入輔王室,而無故輕納妄
 議,拆祖宗之廟以快其私,欲望神靈降歆,垂休錫羨,以永國祚於無窮,其可得乎?」時太廟殿已為十二室,故孝宗升祔,而東室尚虛。熹以為非所以祝延壽康之意,深不然之,因自劾不堪言語侍從之選,乞追奪待制,不許。及光宗祔廟,遂復為九世十二室。蓋自昌陵祔廟,逾二百年而後正太祖之位。慶元二年四月,禮部太常寺言:「已於太廟之西,別建僖祖廟,及告遷僖、順、翼、宣帝後神主詣僖祖廟奉安。所有今年孟冬祫享,先詣四祖廟室
 行禮,次詣太廟,逐幄次行禮。」



 理宗紹定四年九月丙戌,京師大火,延及太廟。太常少卿度正言:「伏見近世大儒侍講朱熹詳考古禮,尚論宗廟之制,畫而為圖,其說甚備。然其為制,務效於古而頗更本朝之制,故學士大夫皆有異論,遂不能行。今天降災異,火發民家,延及宗廟,舉而行之,莫此時為宜。臣於向來備聞其說,今備員禮寺,適當此變,若遂隱默,則為有負,謹為二說以獻。其一,純用朱熹之說,謂本朝廟制未合於古,因畫為圖,謂僖
 祖如周后稷,當為本朝始祖。夫尊僖祖以為始祖,是乃順太祖皇帝之孝心也。始祖之廟居於中,左昭右穆各為一廟,門皆南向,位皆東向。祧廟之主藏於始祖之廟夾室,昭常為昭,穆常為穆,自不相亂。三年合食,則並出祧廟之主,合享於始祖之廟。始祖東向,群昭之主皆位北而南向,群穆之主皆位南而北向。昭穆既分,尊卑以定。其說合於古而宜於今,盡美盡善。舉而行之,祖宗在天之靈必歆享於此,而垂祐於無窮也。其一說,則因
 本朝之制,而參以朱熹之說。蓋本朝廟制,神宗嘗命禮官陸佃討論,欲復古制,未及施行。渡江以來,稽古禮文之事,多所未暇。今欲驟行更革,恐未足以成其事,而徒為紛紛。或且仍遵本朝之制,自西徂東,並為一列。惟於每室之後,量展一間,以藏祧廟之主。每室之前,量展二間,遇三年袷享,則以帷幄幕之,通為一室,盡出諸廟主及祧廟主並為一列,合食其上。前乎此廟為一室,凡遇袷享,合祭於其室,名為袷享,而實未嘗合。今量展此三間,
 後有藏祧主之所,前有祖宗合食之地,於本朝之制,初無大段更革,而頗已得三年大袷之義。今來朝廷若能舉行朱熹前議,固無以加;如其不然,姑從後說,亦為允當,不失禮意。然宗廟之禮,儻無其故,何敢妄議?今因大火之後,若加損益,亦惟其時,乞賜詳議。」有旨,令侍從、禮部、太常集議,後竟不行。



\end{pinyinscope}