\article{志第五十八 禮八(吉禮八)}

\begin{pinyinscope}

 文宣王廟武成王廟先代陵廟諸神祠



 至聖文宣王。唐開元末升為中祠,設從祀,禮令攝三公行事。朱梁喪亂,從祀遂廢。後唐長興二年,仍復從祀。周
 顯德二年,別營國子監,置學舍。宋因增修之,塑先聖、亞聖、十哲像,畫七十二賢及先儒二十一人像於東西廡之木壁,太祖親撰《先聖》、《亞聖贊》,十哲以下命文臣分贊之。建隆中,凡三幸國子監,謁文宣王廟。太宗亦三謁廟。詔繪三禮器物、制度於國學講論堂木壁。又命河南府建國子監文宣王廟,置官講說及賜《九經》書。



 真宗大中祥符元年,封泰山,詔以十一月一日幸曲阜,備禮謁文宣王廟。內外設黃麾仗,孔氏宗屬並陪位,帝
 服靴袍,行酌獻禮。又幸叔梁紇堂,命官分奠七十二弟子、先儒洎叔梁紇、顏氏。初,有司定儀肅揖,帝特展拜,以表嚴師崇儒之意,親制贊,刻石廟中。復幸孔林,以樹擁道,降輿乘馬,至文宣王墓,設奠再拜,詔追謚曰玄聖文宣王,祝文進署,祭以太牢,修飾祠宇,給便近十戶奉塋廟。仍追封叔梁紇為齊國公,顏氏魯國太夫人,伯魚母開官氏,鄆國夫人。



 二年五月乙卯,詔追封十哲為公,七十二弟子為侯,先儒為伯或贈官。親制《玄聖文宣王贊》,命宰相等撰
 顏子以下贊,留親奠祭器於廟中,從官立石刻名。既以國諱,改謚至聖文宣王。賜孔氏錢帛,錄親屬五人並賜出身,又賜太宗御制、御書一百五十卷,銀器八百兩。詔太常禮院定州縣釋奠器數:先聖、先師每坐酒尊一、籩豆八、簋二、簠二、俎三、罍一、洗一、篚一,尊皆加勺、冪,各置於坫,巾共二,燭二,爵共四,坫。有從祀之處,諸坐各籩二、豆二、簋一、簠一、俎一、燭一、爵一。仁宗再幸國子監,謁文宣王廟,皆再拜焉。



 熙寧七年,判國子監常秩等請立孟
 軻、揚雄像於廟廷,仍賜爵號,又請追尊孔子以帝號。下兩制禮官詳定,以為非是而止。



 京兆府學教授蔣夔請以顏回為兗國公,毋稱先師,而祭不讀祝,儀物一切降殺,而進閔子騫九人亦在祀典。禮官以孔子、顏子稱號,歷代各有據依,難輒更改,儀物祝獻,亦難降殺,所請九人,已在祀典。熙寧祀儀,十哲皆為從祀,惟州縣釋奠未載。請自今二京及諸州春秋釋奠,並準熙寧祀儀。



 詔封孟軻鄒國公。晉州州學教授陸長愈請春秋釋奠,孟子
 宜與顏子並配。議者以謂凡配享、從祀,皆孔子同時之人,今以孟軻並配,非是。禮官言:「唐貞觀以漢伏勝、高堂生、晉杜預、範寧之徒與顏子俱配享,至今從祀,豈必同時?孟子於孔門當在顏子之列,至於荀況、揚雄、韓愈,皆發明先聖之道,有益學者,久未配食,誠闕典也。請自今春秋釋奠,以孟子配食,荀況、揚雄、韓愈並加封爵,以世次先後,從祀於左丘明二十一賢之間。自國子監及天下學廟,皆塑鄒國公像,冠服同兗國公。仍繪荀況等像
 於從祀:荀況,左丘明下;揚雄,劉向下;韓愈,範寧下。冠服各從封爵。」詔如禮部議,荀況封蘭陵伯,揚雄封成都伯,韓愈封昌黎伯,令學士院撰贊文。又詔太常寺修四孟釋菜儀。



 元祐六年,幸太學,先詣國子監至聖文宣王殿行釋奠禮,一獻再拜。



 崇寧初,封孔鯉為泗水侯,孔伋為沂水侯。詔:「古者,學必祭先師,況都城近郊,大闢黌舍,聚四方之士,多且數千,宜建文宣王廟,以便薦獻。」又詔:「王安石可配享孔子廟,位於鄒國公之次。」國子監丞趙子
 櫟言:「唐封孔子為文宣王,其廟像,內出王者袞冕衣之。今乃循五代故制,服上公之服。七十二子皆周人,而衣冠率用漢制,非是。」詔孔子仍舊,七十二子易以周之冕服。又詔闢雍文宣王殿以「大成」為名。帝幸國子監,謁文宣王殿,皆再拜行酌獻禮,遣官分奠兗國公而下。國子司業蔣靜言:「先聖與門人通被冕服,無別。配享、從祀之人,當從所封之爵,服周之服,公之袞冕九章,侯、伯之□冕七章。袞,公服也,達於上。鄭氏謂公袞無升龍,誤矣。考《
 周官》司服所掌,則公之冕與王同;弁師所掌,則公之冕與王異。今既考正配享、從祀之服,亦宜考正先聖之冕服。」於是增文宣王冕為十有二旒。



 大觀二年,從通仕郎侯孟請,繪子思像,從祀於左丘明二十四賢之間。議禮局言:「建隆三年,詔國子監廟門立戟十六,用正一品禮。大中祥符二年,賜曲阜廟桓圭,從上公之制。又《史記·弟子傳》曰,受業身通六藝者七十有七人,自顏回至公孫龍三十五人頗有年名及受業見於書傳,四十二人姓
 名僅存。《家語》曰,七十二弟子皆升堂入室者。按《唐會要》七十七人,而《開元禮》止七十二人,又復去取不一。本朝議臣,斷以七十二子之說,取琴張等五人,而去公夏首等十人。今以《家語》、《史記》參定,公夏首、後處、公肩定、顏祖、鄡單、罕父黑、秦商、原抗、樂欬、廉潔,《唐會要》、《開元禮》亦互見之,皆有伯爵,載於祀典。請追贈侯爵,使預祭享。」詔封公夏首鉅平侯,後處膠東侯,公肩定梁父侯,顏祖富陽候,鄡單聊城侯,罕父黑祈鄉侯,秦商馮翊候,原抗樂平
 侯,樂欬建成侯,廉潔胙城侯。又詔改封曾參武城侯,顓孫師穎川侯,南宮絲舀汶陽侯,司馬耕睢陽侯,琴張陽平侯,左丘明中都伯,穀梁赤睢陵伯,戴聖考城伯,以所封犯先聖諱也。



 政和三年,詔封王安石舒王,配享;安石子雱臨川伯,從祀。《新儀》成,以孟春元日釋菜,仲春、仲秋上丁日釋奠。以兗國公顏回、鄒國公孟軻、舒王王安石配享殿上;瑯邪公閔損、東平公冉耕、下邳公冉雍、臨淄公宰予、黎陽公端木賜並西向,彭城公冉求、河內公仲由、
 丹陽公言偃、河東公卜商、武城侯曾參並東向;東廡。穎川侯顓孫師以下至成都伯揚雄四十九人並西向,西廡,長山侯林放以下至臨川伯王雱四十八人並東向。頒闢雍大成殿名於諸路州學。



 五年,太常寺言:「兗州鄒縣孟子廟,詔以樂正子配享,公孫丑以下從祀,皆擬定其封爵:樂正子克利國侯,公孫丑壽光伯,萬章博興伯,告子不害東阿伯,孟仲子新泰伯,陳臻蓬萊伯,充虞昌樂伯,屋廬連奉符伯,徐闢仙源伯,陳代沂水伯,彭更雷
 澤伯,公都子平陰伯,咸丘蒙須城伯,高子泗水伯,桃應膠水伯,盆成括萊陽伯,季孫豐城伯,子叔承陽伯。」大晟樂成,詔下國子學選諸生肄習,上丁釋奠,奏於堂上,以祀先聖。



 靖康元年,右諫議大夫楊時言王安石學術之謬,請追奪王爵,明詔中外,毀去配享之像,使邪說淫辭不為學者之惑。詔降安石從祀廟廷。尚書傅墨卿言:「釋奠禮饌,宜依元豐祀儀陳設,其《五禮新儀》勿復遵用。」



 時又有算學。大觀三年,禮部、太常寺請以文宣王為先師,
 兗、鄒、荊三國公配享,十哲從祀。自昔著名算數者畫像兩廡,請加賜五等爵,隨所封以定其服。於是中書舍人張邦昌定算學:封風後上谷公,箕子遼東公,周大夫商高鬱夷公,大撓涿鹿公,隸首陽周公,容成平都公,常儀原都公,鬼俞區宜都公,商巫咸河東公,晉史蘇晉陽伯,秦卜徒父穎陽伯,晉卜偃平陽伯,魯梓慎汝陽伯,晉史趙高都伯,魯卜楚丘昌衍伯,鄭裨灶滎陽伯,趙史墨易陽伯,周榮方美陽伯,齊甘德菑川伯,魏石申隆慮伯,漢
 鮮於妄人清泉伯,耿壽昌安定伯,夏侯勝任城伯,京房樂平伯,翼奉良成伯,李尋平陵伯,張衡西鄂伯,周興慎陽伯,單揚湖陸伯,樊英魯陽伯,晉郭璞聞喜伯,宋何承天昌盧伯,北齊宋景業廣宗伯,隋蕭吉臨湘伯,臨孝恭親豐伯,張冑玄東光伯,周王樸東平伯,漢鄧平新野子,劉洪蒙陰子,魏管輅平原子,吳趙逵穀城子,宋祖沖之範陽子,後魏商紹長樂子,北齊信都芳樂城子,北齊許遵高陽子,隋耿詢湖熟子,劉焯昌亭子,劉炫景城子,唐
 傅仁均博平子,王孝通介休子,瞿曇羅居延子,李淳風昌樂子,王希明瑯琊子,李鼎祚贊皇子,邊岡成安子,漢郎顗觀陽子,襄楷隰陰子,司馬季主夏陽男,落下閎閬中男,嚴君平廣都男,魏劉徽淄鄉男,晉姜岌成紀男,張丘建信成男,夏侯陽平陸男,後周甄鸞無極男,隋盧大翼成平男。尋詔以黃帝為先師。



 禮部員外郎吳時言:「書畫之學,教養生徒,使知以孔子為師,此道德之所以一也。若每學建立殿宇,則配食、從祀,難於其人。請春秋釋
 奠,止令書畫博士量率職事生員,陪預執事,庶使知所宗師。醫學亦準此。」詔皆從之。



 其釋奠之禮:景德四年,同判太常禮院李維言:「按《開寶通禮》,諸州釋奠,並刺史致齋三日,從祭之官齋於公館。祭日,刺史為初獻,上佐為亞獻,博士為終獻。今諸州長吏不親行祀,非尊師重教之道。」詔太常禮院檢討以聞。按《五禮精義》,州縣釋奠,刺史、縣令初獻,上佐、縣丞亞獻,州博士、縣主簿終獻。有故,以次官攝之。大中祥符三年,判國子監孫奭言:「上丁釋
 奠,舊禮以祭酒、司業、博士充三獻官,新禮以三公行事,近歲止命獻官兩員臨時通攝,未副崇祀向學之意。望自今備差太尉、太常、光祿卿以充三獻。」又命崇文院刊《釋奠儀注》及《祭器圖》頒之諸路。熙寧五年,國子監言:「舊例遇貢舉歲,禮部貢院集諸州府所貢第一人謁奠先聖,如春秋釋奠儀。況春秋自有釋奠禮,請罷貢舉人謁奠。」崇寧,議禮局言:「太學獻官、太祝、奉禮,皆以法服,至於郡邑,則用常服。望命有司降祭服於州縣,凡獻官、祝、禮,
 各服其服,以盡事神之儀。」詔以衣服制度頒使州縣自造焉。



 其謁先師之禮:建隆二年,禮院準禮部貢院移,按《禮閣新儀》云:「舊儀無貢舉人謁先師之文。開元二十六年,詔諸州貢舉人見訖,就國子監謁先師,官為開講,質問疑義,所司設食。昭文、崇文兩館學士及監內諸舉人亦準此。」自後諸州府貢舉人,十一月朔日正衙見訖,擇日謁先師,遂為常禮。大觀初,大司成強淵明言:「考之禮經,士始入學,有釋菜之儀。請自今每歲貢士始入闢雍,
 並以元日釋菜於先聖。」其儀:獻官一員,以丞或博士;分奠官八員,以博士、正錄;大祝一員,以正錄。應祀官前釋菜一日赴學,各宿其次。至日,詣文宣王殿常服行禮,貢士初入學者陪位於庭,其它亦略仿釋奠之儀。紹興十年,詔與大社、大稷並為大祀。淳熙四年,去王雱畫像。淳祐元年正月,理宗幸太學,詔以周敦頤、張載、程顥、程頤、朱熹從祀,黜王安石。景定二年,皇太子詣學,請以張栻、呂祖謙從祀。從之。



 咸淳三年,詔封曾參郕國公,孔伋沂
 國公,配享先聖。封顓孫師陳國公,升十哲位。復以邵雍、司馬光列從祀。其序:兗國公、郕國公、沂國公、鄒國公,居正位之東面,西向北上,為配位;費公閔損、薛公冉雍、黎公端木賜、衛公仲由、魏公卜商,居殿上東面,西向北上,鄆公冉耕、齊公宰予、徐公冉求、吳公言偃、陳公顓孫師,居殿上西面,東向北上,為從祀;東廡,金鄉侯澹臺滅明、任城侯原憲、汝陽侯南宮適、萊蕪侯曾點、須昌侯商瞿、平輿侯漆雕開、睢陽侯司馬耕、平陰侯有若、東阿侯巫
 馬施、陽谷侯顏辛、上蔡侯曹恤、枝江侯公孫龍、馮翊侯秦祖、雷澤侯顏高、上邽侯壤駟赤、成邑侯石作蜀、鉅平侯公夏首、膠東侯後處、濟陽侯奚容點、富陽侯顏祖、滏陽侯句井疆、鄄城侯秦商、即墨侯公祖句茲、武城侯縣成、汧源侯燕伋俯句侯顏之僕、建成侯樂劾、堂邑侯顏何、林慮侯狄黑、鄆城侯孔忠、徐城侯公西點、臨濮侯施之常、華亭侯秦非、文登侯申棖、濟陰侯顏噲、泗水侯孔鯉、蘭陵伯荀況、睢陵伯穀梁赤、萊蕪伯高堂生、樂壽伯
 毛萇、彭城伯劉向、中牟伯鄭眾、緱氏伯杜子春、良鄉伯盧植、滎陽伯服虔、司空王肅、司徒杜預、昌黎伯韓愈、河南伯程顥、新安伯邵雍、溫國公司馬光、華陽伯張栻,凡五十二人,並西向;西廡,單父侯宓不齊、高密侯公冶長、北海侯公皙哀、曲阜侯顏無繇、共城侯高柴、壽張侯公伯寮、益都侯樊須、鉅野侯公西赤、千乘侯梁鱣、臨沂侯冉孺、沐陽侯伯虔、諸城侯冉季、濮陽侯漆雕哆、高苑侯漆雕徒父、鄒平侯商澤、當陽侯任不齊、牟平侯公良孺、
 新息侯秦冉、梁父侯公肩定、聊城侯鄡單、祁鄉侯罕父黑、淄川侯申黨、厭次侯榮旗、南華侯左人郢、朐山侯鄭國、樂平侯原亢、胙城侯廉潔、博平侯叔仲會、高堂侯邽巽、臨朐侯公西輿如、內黃侯蘧瑗、長山侯林放、南頓侯陳亢、陽平侯琴張、博昌侯步叔乘、中都伯左丘明、臨淄伯公羊高、乘氏伯伏勝、考城伯戴聖、曲阜伯孔安國、成都伯揚雄、歧陽伯賈逵、扶風伯馬融、高密伯鄭玄、任城伯何休、偃師伯王弼、新野伯範寧、汝南伯周敦頤、伊陽
 伯程頤、郿伯張載、徽國公朱熹、開封伯呂祖謙,凡五十二人,並東向。



 昭烈武成王。自唐立太公廟,春秋仲月上戊日行祭禮。上元初,封為武成王,始置亞聖、十哲等,後又加七十二弟子。梁廢從祀之祭,後唐復之。太祖建隆三年,詔修武成王廟,與國學相對,命左諫議大夫崔頌董其役,仍令頌檢閱唐末以來謀臣、名將勛績尤著者以聞。四年四月,帝幸廟,歷觀圖壁,指白起曰:「此人殺已降,不武之甚,
 何受享於此?」命去之。景德四年,詔西京擇地建廟,如東京制。大中祥符元年,加謚昭烈。



 初,建隆議升歷代功臣二十三人,舊配享者退二十二人。慶歷儀,自張良、管仲而下依舊配享,不用建隆升降之次。元豐中,國子司業朱服言:「釋奠文宣王,以國子祭酒、司業為初獻,丞為亞獻,博士為終獻,太祝、奉禮並以監學官充。及上戊釋奠武成王,以祭酒、司業為初獻,其亞獻、終獻及讀祝、捧幣,令三班院差使臣充之。官制未行,武學隸樞密院,學官
 員數少,故差右選。今武學隸國子監,長、貳、丞、簿,官屬已多,請並以本監官充攝行事,仍令太常寺修入《祀儀》。」



 政和二年,武學諭張滋言:「《詩》云『赫赫南仲』、『維師尚父』、『文武吉甫』、『顯允方叔』、『王命召虎』、『程伯休父』,是均為周將,功著聲詩,今昔所尊惟一尚父,而南仲、吉甫之徒不預配食,餘如卻縠之閱禮樂、敦詩書,尉繚以言為學者師法,不當棄而不錄,請並配食。」博士孫宗鑒亦請以黃石公配。後有司討論不定,國子監丞趙子崧復言之。



 宣和五年,
 禮部言:「武成王廟從祀,除本傳已有封爵者,其未經封爵之人,齊相管仲擬封涿水侯,大司馬田穰苴橫山侯,吳大將軍孫武滬瀆侯,越相範蠡遂武侯,燕將樂毅平虜侯,蜀丞相諸葛亮順興侯,魏西河守吳起封廣宗伯,齊將孫臏武清伯,田單昌平伯,趙將廉頗臨城伯,秦將王翦鎮山伯,漢前將軍李廣懷柔伯,吳將軍周瑜平虜伯。」於是釋奠日,以張良配享殿上,管仲、孫武、樂毅、諸葛亮、李績並西向,田穰苴、範蠡、韓信、李靖、郭子儀並東向。
 東廡,白起、孫臏、廉頗、李牧、曹參、周勃、李廣、霍去病、鄧禹、馮異、吳漢、馬援、皇甫嵩、鄧艾、張飛、呂蒙、陸抗、杜預、陶侃、慕容恪、宇文憲、韋孝寬、楊素、賀若弼、李孝恭、蘇定方、王孝傑、王晙、李光弼,並西向;西廡,吳起、田單、趙奢、王翦、彭越、周亞夫、衛青、趙充國、寇恂、賈復、耿弇、段熲、張遼、關羽、周瑜、陸遜、羊祜、王浚、謝玄、王猛、王鎮惡、斛律光、王僧辯、於謹、吳明徹、韓擒虎、史萬歲、尉遲敬德、裴行儉、張仁但、郭元振、李晟,並東向。凡七十二將雲。



 紹興七年五月,太
 常博士黃積厚乞以仲春、仲秋上戊日行禮。十一年五月,國子監丞林保奏:「竊見昭烈武成王享以酒脯而不用牲牢,雖曰時方多事,禮用綿蕝,然非所以右武而勵將士也。乞今後上戊釋奠用牲牢,以管仲至郭子儀十八人祀於殿上。」從之。



 乾道六年,詔武成王廟升李晟於堂上,降李績於李晟位次,仍以曹彬從祀。先是,紹興間,右正言都民望言:「李績邪說誤國,唐祀幾滅,李晟有再造王室之勛;宜升李晟於堂上,置李績於河間王孝恭
 之下。」至是,著作郎傅伯壽言:「武成廟從祀,出於唐開元間,一時銓次,失於太雜。如尹吉甫之伐玁狁,召虎之平淮夷,寔亞鷹揚之烈;陳湯、傅介子、馮奉世、班超之流,皆為有漢之雋功;在晉則謝安、祖逖,在唐則王忠嗣、張巡輩,皆不得預從祀之列。竊聞邇日議臣請以本朝名將從祀,謂宜並詔有司,討論歷代諸將,為之去取,然後與本朝名將,繪於殿廡,亦乞取建隆、建炎以來驍俊忠概之臣,功烈暴於天下者,參陪廟祀。」故有是命。



 先代陵廟及錄名臣後。建隆元年,詔:「前代帝王陵寢、忠臣賢士丘□,或樵採不禁、風雨不芘,宜以郡國置戶以守,隳毀者修葺之。」



 乾德初,詔:「歷代帝王,國有常享,著於甲令,可舉而行。自五代亂離,百司廢墜,匱神乏祀,闕孰甚焉。按《祠令》,先代帝王,每三年一享,以仲春之月,牲用太牢,祀官以本州長官,有故則上佐行事。官造祭器,送諸陵廟。」又詔:「先代帝王,載在祀典,或廟貌猶在,久廢牲牢,或陵墓雖存,不禁樵採。其太昊、炎帝、黃帝、高辛、唐堯、
 虞舜、夏禹、成湯、周文王武王、漢高帝光武、唐高祖太宗,各置守陵五戶,歲春秋祠以太牢;商中宗太戊高宗武丁、周成王康王、漢文帝宣帝、魏太祖、晉武帝、後周太祖、隋高祖,各置三戶,歲一享以太牢;秦始皇帝、漢景帝武帝明帝章帝、魏文帝、後魏孝文帝、唐玄宗憲宗肅宗宣宗、梁太祖、後唐莊宗明宗、晉高祖,各置守陵兩戶,三年一祭以太牢;周桓王景王威烈王、漢元帝成帝哀帝平帝和帝殤帝安帝順帝沖帝質帝獻帝、魏明帝高貴鄉
 公陳留王、晉惠帝懷帝愍帝、西魏文帝、東魏孝靜帝、唐高宗中宗睿宗德宗順宗穆宗代宗敬宗文宗武宗懿宗僖宗昭宗、梁少帝、後唐末帝諸陵,常禁樵採。」尋又禁河南府民耕晉、漢廟□需地。凡諸陵有經開發者,有司造袞冕服、常服各一襲,具棺□郭以葬,掩坎日,所在長吏致祭。



 又詔前代功臣、烈士,詳其勛業優劣以聞。有司言:「齊孫臏晏嬰、晉程嬰公孫杵臼、燕樂毅、漢曹參陳平韓信周亞夫衛青霍去病霍光、蜀昭烈帝關羽張飛諸葛亮、
 唐房玄齡長孫無忌魏徵李靖李績尉遲恭渾瑊段秀實等,皆勛德高邁,為當時之冠;晉趙簡子、齊孟嘗君、趙趙奢、漢邴吉、唐高士廉唐儉岑文本馬周為之次;南燕慕容德、唐裴寂、元稹又次之。」詔孫臏等各置守塚三戶,趙簡子等各二戶,慕容德等禁樵採;其有開毀者,皆具棺□郭、朝服以葬,掩坎日致祭,長吏奉行其事。



 景德元年,詔:「前代帝王陵寢,名臣賢士、義夫節婦墳壟,並禁樵採,摧毀者官為修築;無主者碑碣、石獸之類,敢有壞者論
 如律。仍每歲首所在舉行此令。」鄭州給唐相裴度守墳三戶,賜秦國忠懿王錢俶守墳三戶。加謚太公望昭烈武成王,建廟青州,周公旦追封文憲王,建廟兗州,春秋委長吏致祭。



 熙寧元年,從知濮州韓鐸請:「堯陵在雷澤縣東谷林山,陵南有堯母慶都靈臺廟,請敕本州春秋致祭,置守陵五戶,免其租,奉灑掃。」又以中丞鄧潤甫言,唐諸陵陵已定頃畝外,其餘許耕佃為守陵戶,餘並禁止。先是,仁宗嘗錄唐張九齡九代孫錫,狄仁傑裔孫國
 寶,郭子儀孫元亨,長孫無忌孫宏,皆命以官。神宗又錄魏征孫道嚴,段秀實十二世孫昊、八世孫文酉,仍復其家。



 元祐六年,詔相州商王河但甲塚、沂州費縣顏真卿墓並載祀典。先是,乾德中,定先代帝王配享儀,下諸州以時薦祭,牲用羊、豕,政和議禮局遂為定制。



 紹興元年,命祠禹於越州,及祠越王句踐,以範蠡配。淳熙四年,靜江守臣張栻奏所領州有唐帝祠,其山曰堯山;有虞帝祠,其山曰虞山;請著之祀典。十四年,衡州守臣劉清之
 奏:「史載炎帝陵在長沙茶陵,祖宗時給近陵七戶守視,禁其樵牧,宜復建廟,給戶如故事。」淳祐八年,湖南安撫大使、知潭州陳韡再言,從之。



 初,紹興二年,駕部員外郎李願奏:「程嬰、公孫杵臼於趙最為功臣,神宗皇嗣未建,封嬰為成信侯,杵臼為忠智侯,命絳州立廟,歲時奉祀,其後皇嗣眾多。今廟宇隔絕,祭亦弗舉,宜於行在所設位望祭。」從之。十一年,中書舍人朱翌言:「謹按晉國屠岸賈之亂,韓厥正言以拒之,而嬰、杵臼皆以死匿其孤,卒
 立趙武,而趙祀不絕,厥之功也。宜載之祀典,與嬰、杵臼並享春秋之祀,亦足為忠義無窮之勸。」禮寺亦言:「崇寧間已封厥義成侯,今宜依舊立祚德廟致祭。」十六年,加嬰忠節成信侯,杵臼通勇忠智侯,厥忠定義成侯。後改封嬰疆濟公,杵臼英略公,厥啟侑公,升為中祀。



 諸祠廟。自開寶、皇祐以來,凡天下名在地志,功及生民,宮觀陵廟,名山大川能興雲雨者,並加崇飾,增入祀典。熙寧復詔應祠廟祈禱靈驗,而未有爵號,並以名聞。於
 是太常博士王古請:「自今諸神祠無爵號者賜廟額,已賜額者加封爵,初封侯,再封公,次封王,生有爵位者從其本封。婦人之神封夫人,再封妃。其封號者初二字,再加四字。如此,則錫命馭神,恩禮有序。欲更增神仙封號,初真人,次真君。」大觀中,尚書省言,神祠加封爵等,未有定制,乃並給告、賜額、降敕。已而詔開封府毀神祠一千三十八區,遷其像入寺觀及本廟,仍禁軍民擅立大小祠。秘書監何志同言:「諸州祠廟多有封爵未正之處,如
 屈原廟,在歸州者封清烈公,在潭州者封忠潔侯。永康軍李冰廟,已封廣濟王,近乃封靈應公。如此之類,皆未有祀典,致前後差誤。宜加稽考,取一高爵為定,悉改正之。他皆仿此。」故凡祠廟賜額、封號,多在熙寧、元祐、崇寧、宣和之時。



 其新立廟:若何承矩、李允則守雄州,曹瑋帥秦州,李繼和節度鎮戎軍,則以有功一方者也。韓琦在中山,範仲淹在慶州,孫冕在海州,則以政有威惠者也。王承偉築祁州河堤,工部員外郎張夏築錢塘江岸,則
 以為人除患者也。封州曹覲、德慶府趙師旦、邕州蘇緘、恩州通判董元亨、指揮使馬遂,則死於亂賊者也。其王韶於熙河,李憲於蘭州,劉水扈於水洛城,郭成於懷慶軍,折御卿於嵐州,作坊使王吉於麟州神堂砦,各以功業建廟。寇準死雷州,人憐其忠,而趙普祠中山、韓琦祠相州,則以鄉里,皆載祀典焉。其它州縣岳瀆、城隍、仙佛、山神、龍神、水泉江河之神及諸小祠,皆由禱祈感應,而封賜之多,不能盡錄云。



\end{pinyinscope}