\article{列傳第二百二十方技上}

\begin{pinyinscope}

 ○趙修己王處訥子熙元苗訓子守信馬韶楚芝蘭韓顯符史序周克明劉翰
 王懷隱趙自化馮文智沙門洪蘊蘇澄隱丁少微趙自然



 昔者少皞氏之衰,九黎亂德,家為巫史,神人淆焉。顓頊氏命南正重司天以屬神,北正黎司地以屬民,其患遂息。厥後三苗復棄典常,帝堯命羲、和修重、黎之職,絕地天通,其患又息。然而天有王相孤虛,地有燥濕高下,人事有吉兇悔吝、疾病札瘥,聖人欲斯民趨安而避危,則
 巫醫不可廢也。後世占候、測驗、厭禳、禬,至於兵家遁甲、風角、鳥占,與夫方士修煉、吐納、導引、黃白、房中,一切焄蒿妖誕之說,皆以巫醫為宗。漢以來,司馬遷、劉歆又亟稱焉。然而歷代之君臣,一惑於其言,害於而國,兇於而家,靡不有之。宋景德、宣和之世,可鑒乎哉!然則歷代方技何修而可以善其事乎?「曰:「人而無恆,不可以作巫醫。」漢嚴君平、唐孫思邈呂才言皆近道,孰得而少之哉。宋舊史有《老釋》、《符瑞》二志,又有《方技傳》,多言禨祥。今省
 二志,存《方技傳》云。



 趙修己,開封浚儀人,少精天文推步之學。晉天福中,李守真掌禁軍,領滑州節制,表為司戶參軍,留門下。守真每出征,修己必從,軍中占候多中。奏試大理評事,賜緋。漢乾祐中,守真鎮蒲津,陰懷異志,修己屢以禍福諭之,不聽,遂辭疾歸鄉里。明年,守真果叛,幕吏多伏誅,獨修己得免。朝廷知其能,召為翰林天文。



 周祖鎮鄴,奏參軍謀。會隱帝誅楊邠、史弘肇等,且將害周祖,修己知天命
 所在,密謂周祖曰:「釁發蕭墻,禍難斯作。公擁全師,臨巨屏,臣節方立,忠誠見疑。今幼主信讒,大臣受戮,公位極將相,居功高不賞之地,雖欲殺身成仁,何益於事?不如引兵南渡,詣闕自訴,則明公之命,是天所與也。天與不取,悔何可追!」周祖然之,遂決渡河之計。即位,以為殿中省尚食奉御,賜金紫。改鴻臚少卿,遷司天監。顯德中,累加檢校戶部尚書。嘗遣副翰林學士承旨陶穀,以御衣、金帶、戰馬、器幣賜吳越錢俶。宋初,遷大府卿,判監事,上
 章告老,優詔不許。建隆三年卒,年七十一。



 王處訥,河南洛陽人。少時有老叟至舍,煮洛河石如面,令處訥食之,且曰:「汝性聰悟,後當為人師。」又嘗夢人持巨鑒,星宿燦然滿中,剖腹納之,覺而汗洽,月餘,心胸猶覺痛。因留意星歷、占候之學,深究其旨。晉末之亂,避地太原,漢祖時領節制,闢置幕府。即位,擢為司天夏官正,出補許田令,召為國子《尚書》博士,判司天監事。



 周祖嘗與處納同事漢祖,雅相厚善,及自鄴舉兵入汴,遽命訪
 求處訥,得之甚喜,因問以劉氏祚短事。對曰:「人君未得位,嘗務寬大;既得位,即思復仇。漢氏據中土,承正統,以歷數推之,其大祀猶永。第以高祖得位之後,多報仇殺人及夷人之族,結怨天下,所以運祚不長。」周祖蹶然太息。適發兵圍漢大臣蘇逢吉、劉銖等家,待旦將行孥戮,遽命止之。逢吉已自殺,止誅劉銖,餘悉全活。



 廣順中,遷司天少監。世宗以舊歷差舛,俾處訥詳定。歷成未上,會樞密使王樸作《欽天歷》以獻,頗為精密,處訥私謂樸曰:「
 此歷且可用,不久即差矣。」因指以示樸,樸深然之。



 至建隆二年,以《欽天歷》謬誤,詔處訥別造新歷。經三年而成,為六卷,太祖自製序,命為《應天歷》。處訥又以漏刻無準,重定水秤及候中星、分五鼓時刻。俄遷少府少監。太平興國初,改司農少卿,並判司天事。六年,又上新歷二十卷,拜司天監。歲餘卒,年六十八。子熙元。



 熙元,幼習父業,開寶中,補司天歷算。端拱初,改監丞,累遷太子洗馬兼春官正,加殿中丞,景德中,同判監事。東
 封,隨經度制置使詣祠所。禮畢,授權知司天少監。祠汾陰,真拜少監。奉詔於後苑纘陰陽事十卷上之,真宗為制序,賜名《靈臺秘要》,及作詩紀之。



 初,上所修《儀天歷》,秋官正趙昭益言其二年後必差,又熒惑度數稍謬,後果驗。熙元頗伏其精一。上常對宰相言及歷算事,曰:「歷象,陰陽家流之大者,以推步天道,平秩人時為功。」且言:「昭益能專其業,人鮮及也。」



 玉清昭應宮成,以祗事之勤,授司天監。坐擇日差謬,降為少監。以目疾,改將作監,致仕。
 天禧二年卒,年五十八。



 苗訓,河中人,善天文占候之術。仕周為殿前散員右第一直散指揮使。顯德末,從太祖北征,訓視日上復有一日,久相摩蕩,指謂楚昭輔曰:「此天命也。」夕次陳橋,太祖為六師推戴,訓皆預白其事。既受禪,擢為翰林天文,尋加銀青光祿大夫、檢校工部尚書。年七十餘卒。子守信。



 守信,少習父業,補司天歷算。尋授江安縣主簿,改司天臺主簿,知算造。太平興國中,以《應天歷》小差,詔與冬官
 正吳昭素、主簿劉內真造新歷。及成,太宗命衛尉少卿元象宗與明律歷者同校定,賜號《乾元歷》,頗為精密,皆優賜束帛。雍熙中,遷冬官正。端拱初,改太子洗馬、判司天監。淳化二年,守信上言:「正月一日為一歲之首。每月八日,天帝下巡人世,察善惡。太歲日為歲星之精,人君之象。三元日,上元天官,中元地官,下元水官,各主錄人之善惡。又春戊寅、夏甲午、秋戊申、冬甲子為天赦日,及上慶誕日,皆不可以斷極刑事。」下有司議行。未幾,轉殿
 中丞、權少監事,立本品之下,俄賜金紫。



 至道二年,上以梁、雍宿兵,彌歲兇歉,心憂之,令宰相召守信問以天道咎證所在。守信奏曰:「臣仰瞻玄象,及推驗太一經歷宮分,其荊楚、吳越、交廣並皆安寧。自來五緯陵犯、彗星見及水神太一臨井鬼之間,屬秦、雍分及梁、益之地,民罹其災。水神太一來歲入燕分,歲在房心,正當京都之地,自茲朝野有慶。」詔付史館。明年,真授少監。咸平三年卒,年四十六。子舜卿,為國子博士。



 馬韶,趙州平棘人,習天文三式。開寶中,太宗以晉王尹京,申嚴私習天文之禁,韶素與太宗親吏程德玄善,德玄每戒韶不令及門。九年冬十月十九日,既夕,韶忽造德玄,德玄恐甚,詰其所以來,韶曰:「明日乃晉王利見之辰,韶故以相告。」德玄惶駭,止韶一室,遽入白太宗。太宗命德玄以人防守之,將聞於太祖。及詰旦,太宗入謁,果受遺踐阼。韶以赦獲免。逾月,起家為司天監主簿。太平興國二年,擢太僕寺丞,改秘書省著作佐郎。歷太子中
 允、秘書丞,出為平恩令。歸朝,復守舊任,與楚芝蘭同判司天監事,就遷太常博士。淳化五年,坐事,出為博興令,移長山令。秩滿歸鄉里,卒於家。



 楚芝蘭,汝州襄城人,初習《三禮》,忽自言遇有道之士,教以符天、六壬、遁甲之術。屬朝廷博求方技,詣闕自薦,得錄為學生。以占候有據,擢為翰林天文。授樂源縣主簿,遷司天春官正、判司天監事。占者言五福太一臨吳分,當於蘇州建太一祠。芝蘭獨上言:「京師帝王之都,百神
 所集。且今京城東南一舍地名蘇村,若於此為五福太一建宮,萬乘可以親謁,有司便於祗事,何為遠趨江外,以蘇臺為吳分乎?」輿論不能奪,遂從其議,仍令同定本宮四時祭祀儀及醮法。宮成,特遷尚書工部員外郎,賜五品服。淳化初,與馬韶同判監,俱坐事,芝蘭出為遂平令。卒,年六十。錄其子繼芳為城父縣主簿。



 韓顯符,不知何許人。少習三式,善察視辰象,補司天監生,遷靈臺郎,累加司天冬官正。顯符專渾天之學,淳化
 初,表請造銅渾儀、候儀。詔給用度,俾顯符規度,擇匠鑄之。至道元年渾儀成,於司天監築臺置之,賜顯符雜彩五十匹。顯符上其《法要》十卷,序之云:



 伏羲氏立渾儀,測北極高下,量日影短長,定南北東西,觀星間廣狹。帝堯即位,羲氏、和氏立渾儀,定歷象日月星辰,欽授民時,使知緩急。降及虞舜,測璇璣玉衡以齊七政。《通占》又云:「撫渾儀,觀天道,萬象不足以為多。」是知渾儀者,實天地造化之準,陰陽歷數之元,自古聖帝明王莫不用是精詳
 天象,預知差忒。或鑄以銅,或飾以玉,置之內庭,遣日官近臣同窺測焉。



 自伏羲甲寅年至皇朝大中祥符三年庚戌歲,積三千八百九十七年。五帝之後訖今,明歷象之玄,知渾天之奧者,近十餘朝,考而論之,臻至妙者不過四五。自餘徒誇重於一日,不深圖於久要,致使天象無準,歷算漸差,占候不同,盈虛難定。陛下講求廢墜,爰造渾儀,漏刻星躔,曉然易辨。若人目窺於下,則銅管運於上,七曜之進退盈縮,眾星之次舍遠近,占逆順,明吉
 兇,然後修福俾順其度,省事以退其災,悉由斯器驗之。



 昔漢洛下閎修渾儀,測《太初歷》云:「後五百年必當重制。」至唐李淳風,果合前契。貞觀初。淳風又言前代渾儀得失之差,因令銅鑄。七年,太宗起凝暉閣於禁中,俾侍臣占驗。既在宮掖,人莫得見,後失其處所。玄宗命沙門一行修《大衍歷》,蓋以渾儀為證。又有梁令瓚造渾儀木式,一行謂其精密,思出古人,遂以銅鑄。今文德殿鼓樓下有古本銅渾儀一,制極疏略,不可施用。且歷象之作,非
 渾儀無以考真偽;算造之士,非占驗不能究得失。渾儀之成,則司天歲上細行歷。益可致其詳密。



 其制有九,事具《天文志》。自是顯符專測驗渾儀,累加春官正,又轉太子洗馬。



 大中祥符三年,詔顯符擇監官或子孫可以授渾儀法者。顯符言長子監生承矩善察躔度,次子保章正承規見知算造,又主簿杜貽範、保章正楊惟德皆可傳其學。詔顯符與貽範等參驗之。顯符後改殿中丞兼翰林天文。六年卒,年七十四。又詔監丞丁文泰嗣其事
 焉。



 史序字正倫,京兆人。善推步歷算,太平興國中,補司天學生。太宗親較試,擢為主簿。稍遷監丞,賜緋魚,隸翰林天文院。雍熙二年,廷試中選者二十六人,而序為之首,命知算造,又知監事。



 淳化三年,司天鄭昭宴言:「臣測金、火行度須有相犯。今驗之天,而火行漸南,金度漸北,有若相避,遂不相犯。」序又言:「木、火、金三星初夜在午,木在東,火在中,金最西,漸北行去火尺餘。此國家欽崇天道,
 聖德所感也。」



 序後累遷夏官正、河西、環慶二路隨軍轉運、太子洗馬。修《儀天歷》上之,又嘗纂天文歷書為十二卷以獻,改殿中丞,賜金紫,俄權監事。景德二年遷權知少監,大中祥符初即真。三年卒,年七十六。序慎密勤職,在監三十年,未嘗有過,眾賴稱之。



 周克明字昭文。曾祖德扶,唐司農卿。祖傑,開成中進士,解褐獲嘉尉,歷弘文館校書郎。中和中,僖宗在蜀,傑上書言治亂萬餘言。擢水部員外郎,三遷司農少卿。傑精
 於歷算,嘗以《大衍歷》數有差,因敷衍其法,著《極衍》二十四篇,以究天地之數。時天下方亂,傑以天文占之,惟嶺南可以避地,乃遣其弟鼎求為封州錄事參軍。傑天復中亦棄官攜家南適嶺表。劉隱素聞其名,每令占候天文災變。傑自以年老,嘗策名中朝,恥以星歷事僭偽,乃謝病不出。龑襲位,強起之,令知司天監事,因問國祚脩短。傑以《周易》筮之,得《比》之《復》,曰:「卦有二土,土數生五,成於十,二五相比,以歲言之,當五百五十。」龑大喜,賞賚甚
 厚。龑以梁貞明三年僭號,至開寶四年國滅,止五十五年。蓋傑舉成數以避害爾。大有中,遷太常少卿,卒,年九十餘。傑生茂元,亦世其學,事龑至司天少監,歸宋授監丞而卒,即克明之父也。



 克明精於數術,凡律歷、天官、五行、讖緯及三式、風雲、龜筮之書,靡不究其指要。開寶中授司天六壬,改臺主簿,轉監丞,五遷春官正。克明頗修詞藻,喜藏書。景德初,嘗獻所著文十編,召試中書,賜同進士出身。三年,有大星出氐西,眾莫能辨;或言國皇妖
 星,為兵兇之兆。克明時使嶺表,及還,亟請對,言:「臣按《天文錄》、《荊州占》,其星名曰周伯,其色黃,其光煌煌然,所見之國大昌,是德星也。臣在塗聞中外之人頗惑其事,願許文武稱慶,以安天下心。」上嘉之,即從其請。拜太子洗馬、殿中丞,皆兼翰林天文,又權判監事。屬修兩朝國史,其天文律歷事,命克明參之。大中祥符九年,坐本監擇日差互,例降為洗馬。



 天禧元年夏,火犯靈臺,克明語所親曰:「去歲太白犯靈臺,掌歷者悉被降譴,上天垂象,深
 可畏也。今熒惑又犯之,吾其不起乎!」八月,疽發背,卒,年六十四。克明久居司天之職,頗勤慎,凡奏對必據經盡言。及卒,上頗悼惜,遣內侍諭其婿直龍圖閣馮元,令主喪事,賜賻甚厚。



 初,諸僭國皆有纂錄,獨嶺南闕焉。惟胡賓王、胡元興二家纂述,皆不之備。克明訪耆舊,採碑志,孳孳著撰,裁十數卷,書未成而卒。



 劉翰,滄州臨津人。世習醫業,初攝護國軍節度巡官。周顯德初,詣闕獻《經用方書》三十卷、《論候》十卷、《今體治世
 集》二十卷。世宗嘉之,命為翰林醫官,其書付史館,再加衛尉寺主簿。



 太祖北征,命翰從行。建隆初,加朝散大夫、鴻臚寺丞。時太祖求治,事皆核實,故方技之士必精練。乾德初,令太常寺考較翰林醫官藝術,以翰為優,絀其業不精者二十六人。自後,又詔諸州訪醫術優長者籍其名,仍量賜裝錢,所在廚傳給食,遣詣闕。開寶五年,太宗在藩邸有疾,命翰與馬志視之。及愈,轉尚藥奉御,賜銀器、緡錢、鞍勒馬。



 嘗被詔詳定《唐本草》,翰與道士馬志、
 醫官翟煦、張素、吳復珪圭、王光祐、陳昭遇同議,凡《神農本經》三百六十種,《名醫錄》一百八十二種,唐本先附一百一十四種,有名無用一百九十四種,翰等又參定新附一百三十三種。既成,詔翰林學士中書舍人李昉、戶部員外郎知制誥王祐、左司員外郎知制誥扈蒙詳覆畢上之。昉等序之曰:



 《三墳》之書,神農預其一。百藥即辨,《本草》序其錄。舊經三卷,世所流傳。《名醫別錄》,互為編纂。至梁陶弘景乃以《別錄》參其《本經》,朱墨雜書,時謂明白。而又
 考彼功用,為之注釋,列為七卷,南國行焉。逮乎有唐,別加參校,增藥餘八百味,添注為二十卷。《本經》漏缺則補之,陶氏誤說則證之。然而載歷年祀,又逾四百,朱字墨字,無本得同;舊注新注,其文互闕。非聖主撫大同之運,永無疆之休,其何以改而正之哉!



 乃命盡考傳誤,刊為定本。類例非允,從而革焉。至如筆頭灰,兔毫也,而在草部,今移附兔頭骨之下;半天河、地漿,皆水也,亦在草部,今移附土石類之間;敗鼓皮,移附於獸名;胡桐淚,改從
 於木類;紫鑛,亦木也,自玉石品而改焉;伏翼,實禽也,由蟲魚部而移焉;橘柚,附於果實;食鹽,附於光鹽;生姜、乾姜,同歸一類;至於雞腸、蘩蔞,陸英、蒴藋,以類相似,從而附之。仍採陳藏器《拾遺》、李含光《音義》,或窮源於別本,或傳效於醫家,參而較之,辨其臧否。至如突屈白,舊說灰類,今是木根;天麻根,解似赤箭,今又全異。去非取是,特立新條。自餘刊正,不可悉數。



 下採眾議,定為印板。乃以白字為神農所說,墨字為名醫所傳,唐附今附,各加顯
 注,詳其解釋,審其形性。證謬誤而辨之者,署為今注;考文意而述之者,又為今按。義既判定,理亦詳明。今以新舊藥合九百八十三種,並目錄二十一卷,廣頒天下,傳而行焉。



 翰後加檢校工部員外郎。太平興國四年,命為翰林醫官使,再加檢校戶部郎中。雍熙二年,滑州劉遇疾,詔翰馳往視之。翰還,言遇必瘳,既而即死,坐責授和州團練副使。端拱初,起為尚藥奉御。淳化元年,復為醫官使。卒,年七十二。



 王懷隱,宋州睢陽人。初為道士,住京城建隆觀,善醫診。太宗尹京,懷隱以湯劑祗事。太平興國初,詔歸俗,命為尚藥奉御,三遷至翰林醫官使。三年。吳越遣子惟濬入朝,惟濬被疾,詔懷隱視之。



 初,太宗在藩邸,暇日多留意醫術,藏名方千餘首,皆嘗有驗者。至是,詔翰林醫官院各具家傳經驗方以獻,又萬餘首,命懷隱與副使王祐、鄭奇、醫官陳昭遇參對編類。每部以隋太醫令巢元方《病源候論》冠其首,而方藥次之,成一百卷。太宗御製序,
 賜名曰《太平聖惠方》,仍令鏤板頒行天下,諸州各置醫博士掌之。懷隱後數年卒。



 昭遇本嶺南人,醫術尤精驗,初為醫官,領溫水主簿,後加光錄寺丞,賜金紫。



 趙自化,本德州平原人。高祖常,為景州刺史,後舉家陷契丹。父知嵓脫身南歸,寓居洛陽,習經方名藥之術,又以授二子自正、自化。周顯德中,偕來京師,悉以醫術稱。知嵓卒,自正試方技,補翰林醫學。



 會秦國長公主疾,有薦自化診候者,疾愈,表為醫學,再加尚藥奉御。淳化五
 年,授醫官副使。時召陳州隱士萬適至,館於自化家。會以適補慎縣主簿,適素強力無疾,詔下日,自化怪其色變,為切脈曰:「君將死矣。」不數日,適果卒。



 至道中,有布衣鄭元輔者,嘗依自化之姻吏部令史張崇敏家。元輔時從自化丐索,無所得,心銜之。乃詣檢上書,告自化漏洩禁中語及指斥、非所宜言等事。太宗初甚駭,命王繼恩就御史府鞫之,皆無狀,斬元輔於都市。自化坐交游非類,黜為郢州團練副使。未幾,復舊職。咸平三年,加正使。



 景德初,雍王元份洎晉國長公主並上言:自化藥餌有功。請加使秩,領遙郡。上以自化居太醫之長,不當復為請求,令樞密院召自化戒之。雍王薨,坐診治無狀,降為副使。二年,復舊官。是冬卒,年五十七。遺表以所撰《四時養頤錄》為獻,真宗改名《調膳攝生圖》,仍為制序。



 自化頗喜為篇什,其貶郢州也,有《漢沔詩集》五卷,宋白、李若拙為之序。又嘗纘自古以方技至貴仕者,為《名醫顯秩傳》三卷。



 馮文智,並州人。世以方技為業。太平興國中詣都自陳,召試補醫學,加樂源縣主簿。端拱初,授少府監主簿,逾年轉醫官,加少府監丞。嘗隸並代部署。淳化五年,府州折御卿疾,文智診療獲愈,御卿表薦之,賜緋,加光祿寺丞。咸平三年,明德太后不豫,文智侍醫,既愈,加尚藥奉御,賜金紫。六年,直翰林醫官院。東封,轉醫官副使。祀汾陰,又加檢校主客員外郎。大中祥符五年卒,年六十。



 自建隆以來,近臣皇親、諸大校有疾,必遣內侍挾醫療視,
 群臣中有特被眷遇者亦如之。其有效者,或遷秩、賜服色。邊郡屯帥多遣醫官、醫學隨行,三年一代。出師及使境外、貢院鎖宿,皆令醫官隨之。京城四面,分遣翰林祗候療視將士。暑月,即令醫官合藥,與內侍分詣城門寺院散給軍民。上每便坐閱兵,有被金瘡者,即令醫官處療。



 咸平中,有軍士嘗中流矢,自頰貫耳,眾醫不能取,醫官閻文顯以藥傅之,信宿而鏃出。上嘉其能,命賜緋。



 又有醫學劉贇亦善此術。天武右廂都指揮使韓晸從太
 祖征晉陽,弩矢貫左髀,鏃不出幾三十年。景德初,上遣贇視晸,贇傅以藥出之,步履如故。晸請見,自陳感激,願得死所,又極稱贇之妙。特賜贇白金,遷醫官。



 沙門洪蘊,本姓藍,潭州長沙人。母翁,初以無子,專誦佛經,既而有娠,生洪蘊。年十三,詣郡之開福寺沙門智巴,求出家,習方技之書,後游京師,以醫術知名。太祖召見,賜紫方袍,號廣利大師。太平興國中,詔購醫方,洪蘊錄古方數十以獻。真宗在蜀邸,洪蘊嘗以方藥謁見。咸平
 初,補右街首座,累轉左街副僧錄。洪蘊尤工診切,每先歲時言人生死,無不應。湯劑精至,貴戚大臣有疾者,多詔遣診療。景德元年卒,年六十八。



 又有廬山僧法堅,亦以善醫著名,久游京師,嘗賜紫方袍,號廣濟大師,後還山。景德二年,以雍王元份久被疾,召赴闕,至則元份已薨。法堅復歸山而卒。



 蘇澄隱字棲真,真定人。為道士,住龍興觀,得養生之術,年八十餘不衰老。後唐明宗嘗下詔召之,又令宰相馮
 道致書諭旨,歷清泰、天福中繼有聘命,並辭疾不至。開運末,契丹主兀欲立,求有名稱僧道加以恩命,惟澄隱不受。當時公卿自馮道、李崧、和凝而下,皆在鎮陽,日造其室與談宴,各賦詩以贈。周廣順、顯德中,詔存問之。



 太祖征太原還,駐蹕鎮陽,召見行宮,命中使掖升殿,謂之曰:「京師作建隆觀,思得有道之士居之,師累辭召命,豈懷土耶?」對曰:「大梁帝宅,浩穰繁會,非林泉之士所可寄跡也。」上察其意,亦不強之,賜茶百斤、絹二百匹。又幸其
 觀,問曰:「師年逾八十而氣貌益壯,善養生者也。」因問其術,對曰:「臣之養生,不過精思練氣爾,帝王養生即異於是。老子曰:『我無為而民自化,我無欲而民自正。』無為無欲,凝神太和,昔黃帝、唐堯享國永年,得此道也。」上大悅,賜紫衣一襲、銀器五百兩、帛五百匹。年僅百歲而卒。



 丁少微,毫州真源人。為道士,持齋戒,奉科儀尤為精至。嘗隱華山潼谷,密通陳摶所居,與摶齊名,少微志尚清潔,摶嗜酒適性,其道不同,未嘗相往還。少微善服氣,多
 餌藥,年百餘歲,康強無疾。始,卜居山上,起壇場凈室,通夕朝禮,五十餘年未嘗稍懈。太平興國三年,召赴闕,以金丹、巨勝、南芝、玄芝為獻。留數月,遣還山。七年冬卒。



 趙自然,太平繁昌人,家荻港旁,以鬻茗為業,本名王九。始十三,疾甚,父抱詣青華觀,許為道士。後夢一人狀貌魁偉,綸巾素袍,鬢發班白,自云姓陰,引之登高山,謂曰:「汝有道氣,吾將教汝闢穀之法。」乃出青柏枝令啖,夢中食之。及覺,遂不食,神氣清爽,每聞火食氣即嘔,惟生果
 清泉而已。歲餘,復夢向見老人教以篆書數百字,寤悉能記。寫以示人,皆不能識。或云:「此非篆也,乃道家符籙耳。」嘗為《元道歌》,言修練之要。知州王洞表其事,太宗召赴闕,親問之,賜道士服,改名自然,賚錢三十萬。月餘遣還,住青華觀。後因病,飲食如故。大中祥符二年,詔曰:「如聞自然頗精修養之術。」委發轉使楊覃訪其行跡,命內侍武永全召至闕下,屢得對,賜紫衣,改青華觀曰延禧。自然以母老求還侍養,許之。



 大中祥符中,又有鄭榮者,
 本禁軍,戌壁州還,夜遇神人謂曰:「汝有道氣,勿火食。」因授以醫術救人。七年,賜名自清,度為道士,居上清宮。所傳藥能愈大風疾,民多求之。皆刺臂血和餅給焉。



 又有秦州民家子趙抱一者,常牧羊田間。一夕,有叩門召之者,以杖引行,杖端有氣如煙,其香可悅。俄至山崖絕頂,見數人會飲,音樂交奏,與人間無異。抱一駭而不測。會巡檢司過其下,聞樂聲,疑群盜歡聚,集村民梯崖而上。至則無所睹,抱一獨在,援以下之,具言其故。凡經夕,若
 俄頃。自是不喜熟食,凡火化者未嘗歷口。茹甘菊、柏葉、果實、井泉,間亦飲酒,貌如嬰兒。素不習文墨,口占辭句,頗成篇詠。有道家之趣。遂不親農事,野行露宿。大中祥符四年,至京師,猶丱角,詔賜名,度為道士。自是間歲或一至京師,常令居太一宮,與人言多養生事焉。



\end{pinyinscope}