\article{志第十九 五行四}

\begin{pinyinscope}

 金



 從革,金性也。金失其性,則為變怪。舊說以僭咎、恆暘、詩妖、民訛、毛蟲之孽,白眚、白祥之類皆屬之金,今從之。



 建隆二年七月,晉州神山縣北谷中有鐵隨水流出,方
 二丈三尺,其重七千斤。



 太平興國四年九月,夾江縣民王誼得黑石二,皆丹文,其一云「君王萬歲」,其二云「趙二十一帝」,緘其石來獻。



 至道二年二月,桂陽監熔銀自湧成山峰狀。



 咸平四年十二月,亳州太清宮鐘自鳴。



 乾興元年四月甲戌,修奉山陵總管言:皇堂隧道穿得銅鍋,有兩耳,又於寢宮三門下穿得銅盂一、鐵甕一、鐵甲葉三。



 天聖元年三月庚辰,涪陵縣相軋寺夜有光出阿育王塔之舊址,發之,得金銅像三百二十七。五年七月壬
 寅,遼山縣舊河凌地摧塔,獲古錢一百四十六千五百四十三文。



 明道元年五月壬午,漢州江岸獲古鐘一。



 慶歷四年五月乙亥,金溪縣得生金山,重三百二十四兩。



 皇祐四年,乾寧軍漁人得小鐘二於河濱。五年二月己亥,乾寧軍又進古鐘一。



 至和二年四月甲午,瀏陽縣得古鐘一。



 熙寧元年至元豐元年,橫州共獲古銅鼓一十七。



 元豐三年八月,岳州永慶寺獲銅鐘一、銅磬二。六年,南溪縣穿土得銅錢五萬四千有奇。七年三月,筠州獲
 古銅鐘一。十一月,賓州獲銅鼓一。八年,昌元縣通鹽井得銅鍋九、銅盆一、銅盤一。



 崇寧五年十月,荊南獲古銅鼎。



 政和二年,玄圭始出。晉州上一石,綠色,方三尺餘,當中有文曰「堯天正」,其字如掌大而端楷類手畫者,「堯」字居右,「天正」字綴行於左。都堂驗視,礱石三分而字畫愈明,又於「堯」字之下隱約出一「瑞」字,位置始均,蓋曰「天正堯瑞」云。或謂晉陽,堯都也,方玄圭出,乃有此瑞。四年,府畿、汝、蔡之間,連山大小石皆變為瑪瑙,尚方取為寶
 帶、器玩甚富。五年正月,湖南提舉常平劉欽言:蘆荻沖出生金,重九斤八兩,狀類靈芝祥雲;又淘得碎金四百七兩有奇。十一月,越州民拾生金。湟州丁羊穀金坑僅千餘眼得礦,成金共四等,計一百三十四兩有奇。



 重和元年十二月,孝感縣楚令尹子文廟獲周鼎六。



 宣和四年後,御府所藏,往往復變為石,而色類白骨,此與周寶圭占略同。五年,滎陽縣賈穀山麒麟谷採石修明堂,得一石有文曰「明」,百官表賀。五年四月,又獲甗鼎三。



 崇寧
 四年三月,鑄九鼎,用金甚厚,取九州水土內鼎中。既奉安於九成宮,車駕臨幸,遍禮焉,至北方之寶鼎,忽漏水溢於外。劉炳謬曰:「正北在燕山,今寶鼎但取水土於雄州境,宜不可用。」其後竟以北方致亂。



 建炎元年,南京留守朱勝非夜防城,見南門外火光燭地,掘之,得銅印,有文曰「朱勝私印」。火鑠金,金所畏也。後拜相,有明受之變,卒坐眨。三年,吉州修城,役夫得髑髏棄水中,俄浮一鐘,有銘五十六字,大略云:「唐興元年,吾子沒,瘞廬陵西壘,
 後當火德五九之際,世衰道敗,浙、梁相繼喪亂,章貢康昌之日,吾亦復出是邦,東平鳩工,復使吾子同河伯聽命水官。」郡守命錄其辭,錄畢而鐘自碎。



 紹興十一年三月庚申,長安兵刃皆生火光。二十六年,郫縣地出銅馬,高三尺,制作精好,風雨夜嘶。紹興中,耕者得金甕重二十四鈞於秦檜別業。



 乾道二年三月丙午夜,福清縣石竹山大石自移,聲如雷。石方可九丈,所過成蹊,才四尺,而山之木石如故。



 慶元二年十二月,吳縣金鵝鄉銅錢百萬自飛。



 建隆二年,京師夏旱,冬又旱。三年,京師春夏旱。河北大旱,霸州苗皆焦僕。又河南、河中府、孟、澤、濮、鄆、齊、濟、滑、延、隰、宿等州並春夏不雨。四年,京師夏秋旱。又懷州旱。



 乾德元年冬,京師旱。二年正月,京師旱。夏,不雨。是歲,河南府、陜、虢、麟、博、靈州旱,河中府旱甚。四年春,京師不雨。江陵府、華州、漣水軍旱。五年正月,京師旱;秋,復旱。



 開寶二年夏至七月,京師不雨。三年春夏,京師旱。邠州夏旱。
 五年春,京師旱;冬,又旱。六年冬,京師旱。七年,京師春夏旱;冬又旱。河南府、晉、解州夏旱。滑州秋旱。八年春,京師旱。是歲,關中饑,旱甚。



 太平興國二年正月,京師旱。三年春夏,京師旱。四年冬,京師旱。五年夏,京師旱;秋又旱。六年春夏,京師旱。七年春,京師旱。孟、虢、絳、密、瀛、衛、曹、淄州旱。九年夏,京師旱。秋,江南大旱。



 雍熙二年冬,京師旱。三年冬,京師旱。四年冬,京師旱。



 端拱二年五月,京師旱,秋七月至十一月,旱,上憂形於色,蔬食致禱。是歲,河南、萊、
 登、深、冀、旱甚,民多饑死,詔發倉粟貸之。



 淳化元年正月至四月,不雨,帝蔬食祈雨。河南、鳳翔、大名、京兆府、許、滄、單、汝、乾、鄭、同等州旱。二年春,京師大旱。三年春,京師大旱;冬,復大旱。是歲,河南府、京東西河北河東陜西及亳建淮陽等三十六州軍旱。四年夏,京師不雨,河南府、許汝亳滑商州旱。五年六月,京師旱。



 至道元年,京師春旱。二年春夏,京師旱。



 咸平元年春夏,京畿旱。又江浙、淮南、荊湖四十六軍州旱。二年春,京師旱甚。又廣南西路、
 江、浙、荊湖及曹、單、嵐州、淮陽軍旱。三年春,京師旱。江南頻年旱。四年,京畿正月至四月不雨。



 景德元年,京師夏旱,人多暍死。三年夏,京師旱。



 大中祥符二年春夏,京師旱。河南府及陜西路、潭、邢州旱。三年夏,京師旱。江南諸路、宿州、潤州旱。八年,京師旱。九年秋,京師旱。大名府、澶州、相州旱。



 天禧元年,京師春旱,秋又旱。夏,陜西旱。四年春,利州路旱。夏,京師旱。五年冬,京師旱。



 天聖二年春,不雨。五年夏秋,大旱。六年四月,不雨。



 明道元年五月,畿縣
 久旱傷苗。二年,南方大旱。景祐三年六月,河北久旱,遣使詣北嶽祈雨。



 慶歷元年九月丁未朔,遣官祈雨。二年六月戊寅,祈雨。三年,遣使詣嶽瀆祈雨。四年三月丙寅,遣內侍兩浙、淮南、江南祠廟祈雨。五年二月,詔:天久不雨,令州縣決淹獄,又幸大相國寺、會靈觀、天清寺、祥源觀祈雨。六年四月壬申,遣使祈雨。七年正月,京師不雨。二月丙寅,遣官岳瀆祈雨。三月辛丑,西太乙宮祈雨。



 皇祐元年五月丁未,遣官祈雨。三年,恩、冀諸州旱。三月,分
 遣朝臣詣天下名山大川祠廟祈雨。



 至和二年四月甲午,遣官祈雨。



 嘉祐五年,梓州路夏秋不雨。七年三月甲子,罷春燕,以久旱故也。辛丑,西太乙宮祈雨。



 治平元年春,京師逾時不雨。鄭、滑、蔡、汝、穎、曹、濮、洺、磁、晉、耀、登等州、河中府、慶成軍旱。二年春,不雨。



 熙寧二年三月,旱甚。三年,諸路旱。六月,畿內旱。八月,衛州旱。五年五月,北京自春至夏不雨。七年,自春及夏河北、河東、陜西、京東西、淮南諸路久旱。九月,諸路復旱。時新復洮河亦旱,羌戶多
 殍死。八年四月,真定府大旱。八月,淮南、兩浙、江南、荊湖等路旱。九年八月,河北、京東、京西、河東、陜西旱。十年春,諸路旱。



 元豐二年春,河北、陜西、京東西諸郡旱。三年春,西北諸路旱。五年,亢旱。六年夏,畿內旱。



 元祐元年春,諸路旱。正月,帝及太皇太后車駕分日詣寺觀禱雨。是冬,復旱。二年春,旱。三年秋,諸路旱,京西、陜西尤甚。四年春,京師及東北旱,罷春燕。八年秋,旱。



 紹聖元年春,旱,疏決四京畿縣囚。三年,江東大旱,溪河涸竭。四年夏,兩浙旱。



 元符元年,東南旱。二年春,京畿旱。



 建中靖國元年,衢、信等州旱。



 大觀二年,淮南、江東西諸路大旱,自六月不雨,至於十月。



 政和元年,淮南旱。三年,江東旱。四年旱,詔振德州流民。



 宣和元年二月,詔汝、穎、陳、蔡州饑民流移,常平官勒停。秋,淮南旱。四年,東平府旱。五年夏,秦鳳路旱。是歲,燕山府路旱。



 建炎二年夏,旱。



 紹興二年,常州大旱。帝問致旱之由,中書舍人胡交修奏守臣周祀殘酷所致,尋以屬吏坐贓及殺不辜,竄嶺南。三年四月,旱,至於
 七月,帝蔬食露禱,乃雨。五年五月,浙東、西旱五十餘日。六月,江東、湖南旱。秋,四川郡國旱甚。六年,夔、潼、成都郡縣及湖南衡州皆旱。七年春,旱七十餘日,時帝將如建業,隨所在分遣從臣,有事於名山大川。六月,又旱,江南尤甚。八年冬,不雨。九年六月,旱六十餘日,有事於山川。十一年七月,旱。戊申,有事於岳瀆。乙卯,禱雨於圜丘、方澤、宗廟。十二年三月,旱六十餘日。秋,京西、淮東旱。十二月,陜西旱。十八年,浙東、西旱,紹興府大旱。十九年,常州、
 鎮江府旱。二十四年,浙東、西旱。二十九年二月,旱七十餘日。秋,江、浙郡國旱。三十年春,階、成、鳳、西和州旱。秋,江、浙郡國旱,浙東尤甚。



 隆興元年,江、浙郡國旱,京西大旱。二年,臺州春旱。興化軍、漳、福州大旱,首種不入,自春至於八月。



 乾道三年春,四川郡縣旱,至於秋七月,綿、劍、漢州、石泉軍尤甚。四年夏六月,旱,帝將撤蓋親禱於太乙宮而雨。時襄陽、隆興、建寧亦旱。八月,詔頒皇祐祀龍法於郡縣。五年夏秋,淮東旱,盱眙、淮陰為甚。六年夏,浙東、
 福建路旱,溫、臺、福、漳、建為甚。七年春,江西東、湖南北、淮南、浙、婺秀州皆旱;夏秋,江、洪、筠、潭、饒州、南康、興國、臨江軍尤甚,首種不入。冬,不雨。九年,婺、處、溫、臺、吉、贛州、臨江、南安諸軍、江陵府皆久旱,無麥苗。



 淳熙元年,浙東、湖南郡國旱,臺、處、郴、桂為甚。蜀關外四州旱。二年秋,江、淮、浙皆旱,紹興、鎮江、寧國、建康府、常、和、滁、真、揚州、盱眙、廣德軍為甚。三年夏,常、昭、復隨郢金洋州、江陵德安興元府、荊門漢陽軍皆旱。四年春,襄陽府旱,首種不入。五年,常、綿
 州、鎮江府及淮南、江東西郡國旱,有事於山川群望。六年,衡、永、楚州、高郵軍旱。七年,湖南春旱,諸道自四月不雨,行都自七月不雨,皆至於九月。紹興、隆興、建康、江陵府、臺、婺、常、潤、江、筠、撫、吉、饒、信、徽、池、舒、蘄、黃、和、潯、衡、永州、興國、臨江、南康、無為軍皆大旱,江、筠、徽、婺州、廣德軍、無錫縣尤甚,禱雨於天地、宗廟、社稷、山川群望。八年正月甲戌,積旱始雨。七月,不雨,至於十一月:臨安、鎮江、建康、江陵、德安府、越、婺、衢、嚴、湖、常、饒、信、徽、楚、鄂、復、昌州、江陰、
 南康、廣德、興國、漢陽、信陽、荊門長寧軍及京西、淮郡皆旱。九年夏五月,不雨,至於秋七月,江陵、德安、襄陽府、潤、婺、溫、處、洪、吉、撫、筠、袁、潭、鄂、復、恭、合、昌、普、資、渠、利、閬、忠、涪、萬州、臨江、建昌、漢陽、荊門、信陽、南平、廣安、梁山軍、江山、定海、象山、上虞、、嵊縣皆旱。十年六月旱,至於七月,江淮、建康府、和州、興國軍、恭、涪、瀘、合、金、州、南平軍旱。十一年四月,不雨,至於八月,興元府、吉、贛、福、泉、汀、漳、潮、梅、循、邕、賓、象、金、洋、西和州、建昌軍皆旱,興元、吉尤甚。冬,不雨,至
 於明年二月。十四年五月,旱。六月戊寅,有事於山川群望。甲申,帝親禱於太乙宮。七月己酉,大雩於圜丘,望於北郊,有事於嶽、瀆、海凡山川之神。時臨安、鎮江、紹興、隆興府、嚴、常、湖、秀、衢、婺、處、明、臺、饒、信、江、吉、撫、筠、袁州、臨江、興國、建昌軍皆旱,越、婺、臺、處、江州、興國軍尤甚,至於九月,乃雨。十五年,舒州旱。



 紹熙元年,重慶府、蘄、池州旱。二年五月,真、揚、通、泰、楚、滁、和、普、隆、涪、渝、遂、高郵、盱眙軍、富順監皆旱,簡、資、榮州大旱。三年夏,郢、揚、和州大旱;秋,簡、
 資、普、榮、敘、隆、富順監亦大旱。四年,綿州大旱,亡麥。簡、資、普、渠、合州、廣安軍旱。江、浙自六月不雨,至於八月,鎮江、江陵府、婺、臺、信州、江西、淮東旱。五年春,浙東、西自去冬不雨,至於夏秋,鎮江府、常、秀州、江陰軍大旱,廬、和、濠、楚州為甚,江西七郡亦旱。



 慶元二年五月,不雨。三年,潼、利、夔路十五郡旱,自四月至於九月,金、蓬、普州大旱;四月壬子,禱於天地、宗廟、社稷。六年四月,旱;五月辛未,禱於郊丘、宗社。鎮江府、常州大旱,水竭,淮郡自春無雨,首種
 不入,及京、襄皆旱。



 嘉泰元年五月,旱。丙辰,禱於郊丘、宗社。戊辰,大雩於圜丘。浙西郡縣及蜀十五郡皆大旱。二年春,旱,至於夏秋。七月庚午,大雩於圜丘,祈於宗社。浙西、湖南、江東旱,鎮江、建康府、常、秀、潭、永州為甚。四年五月,不雨,至於七月。浙東西、江西郡國旱。



 開禧元年夏,浙東、西不雨百餘日,衢、婺、嚴、越、鼎、灃、忠、涪州大旱。二年,南康軍、江西、湖南北郡縣旱。三年二月,不雨;五月己丑,禱於郊丘、宗社。



 嘉定元年夏,旱,閏月辛卯,禱於郊丘、宗社。
 二年夏四月,旱,首種不入,庚申,禱於郊丘、宗社。六月乙酉,又禱,至於七月乃雨。浙西大旱,常、潤為甚。淮東西、江東、湖北皆旱。四年,資、普、昌、合州旱。六年五月,不雨,至於七月,江陵、德安、漢陽軍旱。入年春,旱,首種不入。四月乙未,禱於太乙宮。庚子,命輔臣分禱郊丘、宗社。五月康申,大雩於園丘,有事於嶽、瀆、海、至於八月乃雨。江、浙、淮、閩皆旱,建康、寧國府、衢、婺、溫、臺、明、徽、池、真、太平州、廣德、興國、南康、盱眙、安豐軍為甚,行都百泉皆竭,淮甸亦然。十
 年七月,不雨,帝日午曝立,禱於宮中。十一年秋,不雨,至於冬,淮郡及鎮江、建寧府、常州、江陰、廣德軍旱。十四年,浙、閩、廣、江西旱,明、臺、衢、婺、溫、福、贛、吉州、建昌軍為甚。十五年五月,不雨,岳州旱。



 嘉熙元年夏,建康府旱。三年,旱。四年,江、浙、福建旱。



 淳祐七年,旱。十一年,閩、廣及饒州旱。



 咸淳六年,江南大旱。十年,廬州旱,長樂、福清二縣大旱。



 建隆中,京師士庶及樂工、少年競唱歌曰《五來子》。自建隆、開寶,凡平荊、湖、川、廣、江南,五國皆來朝。時西川孟昶
 賦斂無度,射利之家配率尤甚,既乏緡錢,唯仰在質物。乃競書簡札揭於門曰:「今召主收贖。」又每歲除日,命翰林為詞題桃符,正旦置寢門左右。末年,學士幸寅遜撰詞,昶以其非工,自命筆題云:「新年納餘慶,嘉節號長春。」昶以其年正月降王師,即命呂餘慶知成都府,而「長春」乃太祖誕聖節名也,「召」與「趙」、「贖」與「蜀」同音。



 開寶初,廣南劉鋹令民家置貯水桶,號「防火大桶」。又鋹末年,童謠曰:「羊頭二四,白天雨至。」後王師以辛未年二月四日擒鋹。
 識者以為國家以火德王,房為宋分;羊,未神也;雨者,王師如時雨之義也;「防」與「房」、「桶」與「宋」同音。



 周廣順初,江南伏龜山圮,得石函,長二尺,廣八寸,中有鐵銘,云:「維天監十四年秋八月,葬寶公於是。」銘有引曰:「寶公嘗為偈,大事書於版,帛冪之。人欲讀之者,必施數錢乃得,讀訖即冪之。是時,名士陸倕、王筠、姚察而下皆莫知其旨。或問之,雲在五百年後。至卒,乃歸其銘同葬焉。」銘曰:「莫問江南事,江南自有馮。乘雞登寶位,跨犬出金陵。子建司南
 位,安仁秉夜燈。東鄰家道闕,隨虎遇明興。」其字皆小篆,體勢完具,徐鉉、徐鍇、韓熙載皆不能解。及煜歸朝,好事者云:煜丁酉年襲位,即乘雞也;開寶八年甲戌,江南國滅,是跨犬也;當王師圍其城而曹彬營其南,是子建司南位;潘美營其北,是安仁秉夜燈也;其後太平興國三年,淮海王錢俶舉國入覲,即東鄰也;家道闕,意無錢也;隨虎遇,戊寅年也。



 皇祐五年正月戊午,狄青敗儂智高於歸仁鋪。初,謠言「農家種,糴家收」。至是,智高果為青所
 破。



 建炎三年四月,鼎州桃源洞大水,巨石隨流而下,有文曰:「無為大道,天知人情;無為窈冥,神見人形。心言意語,鬼聞人聲;犯禁滿盈,地收人魂。」金石同類,類金為變怪者也。



 紹興二年,李綱帥長沙,道過建寧,僧宗本題邑治之壁曰:「東燒西燒,日月七七。」後數日,江西盜李敦仁入境,焚其邑,七月七日也。



 淳熙中,淮西競歌汪秀才曲曰:「騎驢渡江,過江不得。」又為犬葉舞以和之。後舒城狂生汪格謀不軌,州兵入其家,縛之。其子拒殺,聚惡少數千為
 亂,聲言渡江。事平,格亦伏誅。七年正月,餘杭門外墻壁有詩,其言頗涉怪,後廉得主名,杖遣之。主管城北廂劉君暨以失察異言,坐削秩,其詩不錄。十四年,都城市井歌曰:「汝亦不來我家,我亦不來汝家。」至紹熙二三年,其事始應於兩宮。



 嘉定三年,都城市井作歌詞,末句皆曰「東君去後花無主」,朝廷惡而禁之。未幾,太子詢薨。



 慶元四年三月甲辰,有郵筩置詩達御前者,詔宰臣究其詩,不錄。



 嘉泰四年,越人盛歌《鐵彈子白塔湖曲》。俄有盜金
 十一者自號「鐵彈子」,繆傳其鬥死於白塔湖中,後獲於諸暨縣。



 漢乾祐中,荊南高從誨鑿池於山亭下,得石匣,長尺餘,扃鐍甚固。從誨神之,屏左右,焚香以啟匣,中得石,有文云:「此去遇龍即歇。」及建隆中,從誨孫繼沖入朝,改鎮徐州。「龍」、「隆」音相近。



 太平興國中,京師兒童以木雕合子,中有竅,藏腋下有聲,號云「腋底鬧」。後盧多遜投荒,人以為讖,其在肘腋而司國典也。



 天禧二年五月,西京訛言有
 物如烏帽,夜飛入人家,又變為犬狼狀。人民多恐駭,每夕重閉深處,至持兵器驅逐者。六月乙巳,傳及京師,云能食人。里巷聚族環坐,叫噪達曙,軍營中尤甚,而實無狀,意其妖人所為。有詔嚴捕,得數輩,訊之,皆非。



 政和七年,詔修神保觀,俗所謂「二郎神」者。京師人素畏之,自春及夏,傾城男女負土以獻,揭榜通衢,云某人獻土;又有飾形作鬼使,巡門催納土者。或以為不祥,禁絕之。後金人斡離不圍京師,其國謂之「二郎君」云。



 紹興元年十二
 月,越州連火,民訛言相驚,月幾望當再火。樞密院以軍法禁之,乃定。



 嘉泰二年六月,故循王張俊家火。後旬日,市井訛言相驚,絳衣婦人為火殃下墜。都民徙避,晝夜弗寧,禁之,後亦不火。



 慶元六年十月,瓊州訛言妖星流墮民郭七家,聲如雷。通判曾豐暨瓊山縣令移文驚擾,後皆坐絀。簽書樞密院事林存為似道所擯,道死於漳。漳有富民蓄油煔木甚佳,林氏子弟求之,價高不可得,因撫其木曰:「收取收取,待賈丞相用。」德祐元年,似道謫
 死,郡守與之經營,竟得此木以殮。



 宋初,陳摶有紙錢使不行之說,時天下惟用銅錢,莫喻此旨。其後用交子、會子,其後會價愈低,故有「使到十八九,紙錢飛上天」之謠。似道惡十九界之名,乃名關子,然終為十九界矣,而關子價益低,是紙錢使不行也。



 宋以周顯德七年庚申得天下。圖讖謂「過唐不及漢,一汴、二杭、三閩、四廣」,又有「寒在五更頭」之謠,故宮漏有六更。按漢四百二十餘年,唐二百八十九年。開慶元年,宋衣乍過唐十一年,滿五庚申
 之數;至德祐二年正月降附,得三百一十七年,而見六庚申,如宮漏之數。



 建隆三年,有象至黃陂縣匿林中,食民苗稼,又至安、復、襄、唐州踐民田,遣使捕之。明年十二月,於南陽縣獲之,獻其齒革。乾德二年五月,有象至澧陽、安鄉等縣,又有象涉江入華容縣,直過闤闠門;又有象至澧州澧陽縣城北。



 乾德四年八月,普州兔食禾。五年,有象自至京師。



 雍熙四年,有犀自黔南入萬州,民捕殺之,獲其皮角。



 開
 寶八年四月,平陸縣鷙獸傷人,遣使捕之,生獻十頭。十月,江陵府白晝虎入市,傷二人。



 太平興國三年,果、閬、蓬、集諸州虎為害,遣殿直張延鈞捕之,獲百獸。俄而七盤縣虎傷人,延鈞又殺虎七以為獻。七年,虎入蕭山縣民趙馴家,害八口。



 淳化元年十月,桂州虎傷人,詔遣使捕之。



 至道元年六月,梁泉縣虎傷人。二年九月,蘇州虎夜入福山砦,食卒四人。



 咸平二年十二月,黃州長析村二虎夜鬥,一死,食之殆半,占云:「守臣災。」明年,知州王禹偁
 卒。咸平六年十月乙酉,有狐出皇城東北角樓,歷軍器庫至夾道,獲之。



 大中祥符九年三月,杭州浙江側,晝有虎入稅場,巡檢俞仁祐揮戈殺之。



 天聖九年五月,宿州獲白兔。六月,廬州獲白兔。



 明道二年六月,唐州獲白兔。



 皇祐三年十二月,泰州獲白兔。



 嘉祐三年六月丁卯,交址貢異獸二。初,本國稱貢騏驎,狀如牛身,被肉甲,鼻端有角,食生芻果,必先以杖擊其角,然後食。既至,而樞密使田況辨其非麟,詔止稱異獸。



 熙寧元年九月,撫州獲
 白兔。十二月,嵐州獲白鹿。四年九月,廬州獲白兔。



 政和五年十二月,安化軍獲白兔。六月,泰州軍獲白兔。七年十月,達州獲白兔。



 宣和元年十月,淄州獲黑兔。宣和七年秋,有狐由艮岳直入禁中,據御榻而坐,詔毀狐王廟。



 紹興十一年,海州屬金,悉空其民安江。後二十年,有二虎入城,人射殺之,虎亦搏人。明年,魏勝舉州來歸,亦空其民。漢龔遂曰:「野獸入宮室,宮室將空。」虎豕皆毛孽也。十三年,南康縣雷雨,群貍震死於巖穴中,巖石皆為碎。
 二十二年,劉彭老家貓產數子,皆三足。



 乾道七年,潮州野象數百食稼,農設阱田間,像不得食,率其群圍行道車馬,斂穀食之,乃去。



 淳熙二年,江州馬當山群狐掠人。十年,滁州有熊虎同入樵民舍,夜,自相搏死。



 紹熙元年三月,臨安府民家貓生子一,有八足二尾。四年,鄂州武昌縣虎為人患。五年八月,揚州獻白兔。侍御史章穎劾守臣錢之望以孽為瑞。占曰:「國有憂。」白,喪祥也。是歲,光宗崩。



 慶元三年,德興縣群狐入民舍。



 咸淳九年十一月
 辛卯黎明,有虎出於揚州市,毛色微黑,都撥發官曹安國率良家子數十人射之。制置使李庭芝占曰:「千日之內,殺一大將。」於是臠其肉於城外而厭之。



 紹興六年四月,中京大雪、雷震,犬數十爭赴土河而死,可救者才二三。



 淳熙元年六月,饒州大雷震犬於市之旅舍。



 慶元二年,撫州有犬若人,坐於郡守之坐。未幾,郡守林廷彥卒於官。



 德祐元年五月壬申,揚州禁軍民毋得蓄犬,城中殺犬數萬,輸皮納官。



 乾德三年七月己卯夜,西方起蒼白氣,長五十尺,貫天船、五車,亙井宿,占曰:「主兵動。」六年十月己未旦,西北起蒼白氣三道,長二十尺,趨東散,占曰:「游兵之象。」



 太平興國四年四月己未夜,西北有白氣壓北斗。



 雍熙四年正月癸酉,白氣起角、亢經,太微垣,歷軒轅大星,至月傍散。



 至道二年二月丙子夜,西方有蒼白氣,長短八道,如彗掃稍,經天漢,參錯如交蛇,占曰:「所見之方主兵勝。」



 咸平四年三月丙申,白氣二亙天。五年正月,白氣如虹貫
 日,久而散。七月戊戌,白氣如陣貫東井。六年四月己巳,白氣東西亙天。丁丑,白氣貫日。五月辛亥,白氣出昴至壁沒。六月丙子,白氣出河鼓左右旗,分為數道沒。七月癸卯,白氣如彗,起西南方,占曰:「有兵喪。」



 景德元年五月,白氣貫軒轅,蒼白氣十餘如布亙天。二年二月丁亥,白氣五道貫北斗,占為大風、幸臣憂。十月丙子,白氣出閣道西,孛孛有光,占曰:「宮中憂。」三年三月,白氣貫月。四年三月己未,白氣東西亙天。庚申,白氣出南方,長二丈許,
 久而不散。四月庚午,白氣貫北斗,長十丈,占為大風。庚寅,白氣如布襲月,三丈許。



 大中祥符元年正月丁丑,白氣二,東西亙天。五年二月壬寅,白氣長五丈,出東井,貫北斗魁及軒轅,占為兵、為雷雨。



 明道元年十二月壬戌,西北有蒼白氣亙天。



 慶歷元年八月庚辰夜,東方有白氣長十尺許,在星宿度中,至十日,長丈餘,沖天,九十餘日沒。



 二年八月甲申,白氣貫北斗。三年正月戊戌,中天有白氣長二十尺,向西南行貫日,占曰:「邊兵憂。」四月癸
 卯,白氣二生西北隅,上中天,首尾至濁,東南行,良久散,占曰:「其下有兵寇。」八月壬子夜,白氣貫北斗魁。九月辛巳夜,中天有白氣長二丈許,貫卷舌、南河,東北行,少頃散,占曰:「風雨之候。」



 皇祐四年十一月辛酉夜,白氣起北方近濁,長五丈許,歷北斗,久之散,占曰:「多大風。」



 嘉祐元年三月,彭城縣白鶴鄉地生面,占曰:「地生面,民將饑。」五月,鐘離縣地生面。



 治平二年四月丙午夜,西北方有白氣,漸東南行,首尾至濁,貫角宿,移西北,久方散。占曰:「有兵
 戰疾疫事。」



 熙寧九年四月庚寅夜,白氣長丈,起東北方天市垣。



 元祐三年七月戊辰夜,西北有白氣經天,主兵,宜防西、北二鄙。



 元符二年九月戊辰夜,有白氣十道,各長五尺,主兵及大臣黜。



 崇寧二年五月戊子夜,蒼白氣起東南方,長三丈,貫尾、箕、斗,主蠻夷入貢,舊臣來歸。



 宣和三年九月壬午夜,蒼白氣長三丈,貫月,主其下有亂者。



 靖康元年十二月丙辰,白氣出太微垣。二年二月壬午夜,白氣如虹,自南亙北,須臾,移西南,至東北,天明而
 沒。三月戊子,白氣貫斗。



 建炎二年,杜充為北京留守,天雨紙錢於營中,厚盈寸。明日,與金人戰城下,敗績。紙,白祥也。三年三月,白氣貫日。四年五月壬子夜,北方有白氣十餘道如練。二十六年七月辛酉夜,天雨水銀。



 紹興元年,潭州得白玉於州城蓮花池中,孔彥舟以獻,詔卻之。前史以為玉變近白祥,後彥舟為劇盜。二月己巳夜,東南有白氣。十一年三月庚申,金人居長安,油、酒皆變白色。三十年十一月甲午夜,西南有白氣出危,入昴。十
 二月戊申,白氣出尾,入軫,貫天市垣。三十一年十二月辛丑,白氣如帶,東西亙天,出鬥,歷牛。



 隆興元年十二月壬午夜,白氣見西南方,出危,入昴。二年正月甲寅夜,西南有白氣,亙天如帶。



 乾道元年正月庚午,白氣見西北方,出奎,入參。三月戊辰,白氣如帶,自參及角,東西亙天。四月丁酉夜,白氣見西北方,入天市垣。辛丑夜,白氣入北斗。乙巳夜,白氣入紫微垣。十月己丑夜,蒼白氣見東南方,入翼。十一月丙寅,白氣如帶,出女,入昴,東西亙天。
 三年十二月庚午夜,白氣如帶,東西亙天,出女,入昴。



 淳熙十年正月戊子夜,西南有白氣如天漢而明,南北廣可六丈,東西亙天,歷壁至畢。



 紹熙五年六月壬寅夜,白氣亙天,自紫微至亢、角。己酉日入後,白氣亙天,頃刻而散。



 慶元四年八月庚辰,白氣亙天。五年二月癸酉夜,東北方白氣如帶,自角至參。八月癸亥,東北方有白氣如帶,亙天。



 嘉泰四年十一月辛未,晝有白氣分數道,亙天。



 嘉熙四年二月丙辰,白氣亙天。



 淳祐二年二月甲寅,白
 氣亙天。



 景定三年七月甲申,白氣如匹布,亙天。



 咸三九年,襄陽城中白氣自西而出。



 紹興二年,宣州有鐵佛像,坐高丈餘,自動迭前迭卻若傴而就人者數日,既而郡有火。火氣盛,金失其性而為變怪也。七月,天雨錢,或從石甃中流出,有輪郭,肉好不分明,穿之。碎若沙土。二月,溫州戒福寺銅佛像頂珠自動,光彩激射,經日不少停,數日火作,寺焚。



 淳熙九年春,德興縣民家鏡自飛舞,與日光相射。



 慶元二年正月,泰
 寧縣耕夫得鏡,厚三寸,徑尺有二寸,照見水底,與日爭輝,病熱者對之,心骨生寒,後為雷震而碎。



\end{pinyinscope}