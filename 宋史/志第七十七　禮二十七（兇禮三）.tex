\article{志第七十七 禮二十七(兇禮三)}

\begin{pinyinscope}

 外國喪禮及入吊儀諸臣喪葬等儀



 凡外國喪,告哀使至,有司擇日設次於內東門之北隅,命官攝太常卿及博士贊禮。俟太常卿奏請,即向其國
 而哭之,五舉音而止。皇帝未釋素服,人使朝見,不宣班,不舞蹈,不謝面天顏,引當殿,喝「拜」,兩拜,奏聖躬萬福。又喝「拜」,兩拜,隨拜萬歲。或增賜茶藥及傳宣撫問,即出班致詞,訖,歸位。又喝「拜」,兩拜,隨拜萬歲。喝「祗候」,退。



 大中祥符二年十二月,北朝皇太后兇訃,遣使來告哀。詔遣官迓之,廢朝七日,擇日備禮舉哀成服,禮官詳定儀注以聞。其日,皇帝常服乘輿詣幕殿,俟時釋常服,服素服,白羅衫、黑銀帶、素紗軟腳帕頭。太常卿跪,奏請皇帝為北
 朝皇太后兇訃至掛服,又奏請五舉音。文武百僚進名奉慰,退幕殿。仍遣使祭奠吊慰。



 三年正月,契丹賀正使為本國皇太后成服,所司設幕次、香、酒及衰服、絰、杖等,禮直官引使、副已下詣位,北向再拜。班首詣前,執盞跪奠,俯伏,興,歸位,皆再拜。俟使已下俱衰服、絰、杖,成服訖,禮直官再引各依位北向,舉哭盡哀。班首少前,去杖,跪,奠酒訖,執杖,俯伏,興,歸位。焚紙馬,皆舉哭,再拜畢,各還次,服吉服,歸驛。



 天聖九年六月,契丹使來告哀。禮官詳
 定:北朝兇訃,宜於西上閣門引來使奉書,令閣門使一員跪受承進,宰臣、樞密使已下待制已上,並就都亭驛吊慰。七月一日,使者耶律乞石至,帝與皇太后發哀苑中,使者自驛赴左掖門入,至左升龍門下馬,入北偏門階下,行至右升龍北偏門,入朝堂西偏門,至文德殿門上奉書。太常博士二員與禮直官贊引入文德殿西偏門階下,行至西上閣門外階下,面北跪,進書。閣門使跪受承進。太常博士、禮直官退。使者入西上閣門殿後偏
 門,入宣祐西偏門,行赴內東門柱廊中間,過幕次祗候,朝見訖,赴崇政殿門幕次祗候,朝見皇太后訖,出。三日,近臣慰乞石於驛。



 嘉祐三年正月,契丹告國母哀。使人到闕入見,皇帝問云:「卿離北朝日,侄皇帝悲苦之中,聖躬萬福。」朝辭日,即云:「皇帝傳語北朝侄皇帝,嬸太皇太后上仙,遠勞人使訃告。春寒,善保聖躬。」中書、樞密以下、待制已上赴驛吊慰,云:「竊審北朝太皇太后上仙,伏惟悲苦。」五月,獻遺留物。



 明道元年十一月二十四日敕:夏
 王趙德明薨,特輟朝三日,令司天監定舉哀掛服日辰。其日,乘輿至幕殿,服素服。太常博士引太常卿當御坐前跪,奏請皇帝為夏王趙德明薨舉哀,又奏請十五舉音,又奏請可止。文武百僚進名奉慰。告哀使、副已下朝見,首領並從人作兩班見。先首領見,兩拜後,班首奏聖躬萬福。又兩拜,隨拜萬歲。喝賜例物酒食,跪受。起,又兩拜,隨拜萬歲。喝「各祗候」,退。從人儀同。是日,皇太后至幕殿,釋常服,白羅大袖、白羅大帶,舉哀如皇帝儀。其遣使
 致祭吊慰,如契丹。



 其入吊奠之儀。乾興元年,真宗之喪,契丹遣殿前都點檢崇義軍節度使耶律三隱、翰林學士工部侍郎知制誥馬貽謀充大行皇帝祭奠使、副,左林牙左金吾衛上將軍蕭日新、利州觀察使馮延休充皇太后吊慰使、副,右金吾衛上將軍耶律寧、引進使姚居信充皇帝吊慰使、副。



 所司預於滋福殿設大行皇帝神御坐,又於稍東設御坐。祭奠吊慰使、副並素服,由西上閣門入,陳禮物
 於庭。中書、門下、樞密院並立於殿下,再拜訖,升殿,分東西立。禮直官、閣門舍人贊引耶律三隱等詣神御坐前階下,俟殿上簾卷,使、副等並舉哭,殿上皆哭。再拜訖,引升殿西階,詣神御坐前上香、奠茶酒。貽謀跪讀祭文畢,降階,復位,又舉哭,再拜訖,稍東立。俟皇太后升坐,中書、樞密院起居畢,簾外侍立。舍人引吊慰祭奠使、副朝見。殿上舉哭,左右皆哭。吊慰使、副蕭日新等升殿進書訖,降坐。俟皇帝升坐,中書、樞密院起居畢,升殿侍立。舍人
 引吊慰祭奠使、副朝見。皇帝舉哭,左右皆哭。吊慰使、副耶律寧等升殿進書訖,賜三隱等襲衣、冠帶、器幣、鞍馬,隨行舍利、牙校等衣服、銀帶、器幣有差。吊慰使、副蕭日新等復詣承明殿,俟皇太后升坐,中書、樞密院侍立如儀。舍人引蕭日新等升殿進問聖候書畢,賜銀器、衣著有差。仍就客省賜三隱等茶酒,又令樞密副使張士遜別會三隱等伴宴於都亭驛。



 英宗即位,契丹使來賀乾元節,命先進書奠梓宮,見於東階。放夏國使人見,客省
 以書幣入,後吊慰使見殿門外。契丹祭奠使見於皇儀殿東廂,群臣慰於門外。使人辭於紫宸殿,命坐賜茶。故事賜酒五行,自是,終諒冱,皆賜茶。



 神宗之喪,夏國陳慰使丁努嵬名謨鐸、副使呂則、陳聿精等進慰表於皇儀門外,退赴紫宸殿門,賜帛有差。



 元祐初,高麗入貢,有太皇太后表及進奉物。樞密院請遵故事,惟答以皇帝回諭敕書。已而宣仁聖烈太后崩,禮部、太常、閣門同詳定:高麗奉慰使人於小祥前後到闕,令於紫宸殿門見,客
 省受表以進,賜器物、酒饌,退,並常服、黑帶、不佩魚。候見罷,純吉服。



 淳熙十四年,金國吊祭使到闕,惟皇帝先詣梓宮行燒香禮,及使入門祭訖,皆就幄舉哭外,陳設行事並如先朝舊儀。其奉辭日,有司亦先設神御坐及設香案、茶酒、果食盤臺於幾筵殿上。宰執升殿分東西立,侍從官於殿下西面立。使、副入門,殿上下皆哭,使、副升殿,哭止。使、副詣神坐前一拜,上香、奠茶、三奠酒畢,拜,興,讀祭文官跪讀祭文,一拜,興,殿上下皆哭。使、副俱降,歸
 位立,又再拜訖,退。



 諸臣之喪,國制:諸王、公主、宗室將軍以上有疾,皆乘輿臨問。如小疾在家,或幸其第,有至三四者。其宮邸在禁中,多不時而往。惟宰相、使相、駙馬都尉疾亟,幸其第,或賜勞加禮焉。



 建隆元年七月,宰相範質有疾,太祖親幸其第,賜黃金、銀、絹有差。開寶二年,趙普有疾,帝再往視,賜銀器、絹甚厚。太平興國中,鎮寧軍節度楊信久病喑,忽能言,帝異之,遽幸其第,加賜賚。大中祥符三年三月,
 鎮安軍節度使、駙馬都尉石保吉疾亟,帝將臨視之,其日大忌,宰相言於禮非便,遂遣內侍以諭保吉,明日始臨省焉。六月,幸翰林侍講學士邢昺第視疾,賜白金千兩、衣著千匹、名藥一奩。



 熙寧七年十二月,詔頒新式,凡臨幸問疾者賜銀、絹,宰臣及樞密使帶使相者二千五百兩匹,樞密使、使相二千兩匹,知樞密院事、參知政事、樞密副使、同知樞密院事一千五百兩匹,簽書樞密院事、同簽書樞密院事、宣徽使七百五十兩匹,殿前都指
 揮使一千五百兩匹,駙馬都尉任使相以下者二千五百兩匹,任節度觀察留後以下者一千五百兩匹,並入內內侍省取賜。



 車駕臨奠。《太常新禮》:宰相、樞密、宣徽使、參知政事、樞密副使、駙馬都尉薨,皆臨幸奠酹,及發引,乘輿或再往。咸平二年,工部侍郎、樞密副使楊礪卒,即日冒雨臨其喪。大中祥符元年,殿前都虞候、端州防禦使李繼和卒,真宗將臨其喪,以問宰臣,對曰:「繼和以品秩實無此禮。陛
 下敦序外族,先朝亦嘗臨杜審瓊之喪,於禮無嫌。」帝然之,即日幸其第。



 康定二年,右正言、知制誥吳育奏:「臣竊見車駕每有臨奠臣僚、宗戚之家,皆實時出幸,道路不戒,羽衛不全,從官奔馳,眾目驚異。萬乘法駕,豈慎重之意乎?雖震悼方切於皇慈;而舉動貴合乎經禮。臣竊詳《通禮》舊儀,蓋俟喪家成服,然後臨奠,於事不迫,在禮亦宜。臣愚欲乞今後車駕如有臨奠去處,乞俟本家既斂成服,然後出幸,則恩意容典,詳而得中,警蹕羽儀,備之
 有素。」事下禮官議:「遭喪之家,有出殯日乃成服者,恐至時難行臨奠。請自今聖駕臨奠臣僚、宗戚之家,若奏訃在交未前,即傳宣閣門,只於當日令所屬候儀衛備,奏請車駕出幸;若奏訃在交未後,即次日臨奠。庶使羽衛整肅,於事為宜。」詔可。



 其儀:乘輿自內出,千牛將軍四人執戈,一人執桃,一人執茢,前導。車駕將至所幸之第,贊禮者引喪主哭於大門內,望見乘輿,止哭,再拜,立於庭。皇帝至幕殿,改素服就臨,喪主內外再拜。皇帝哭,十五
 舉音,喪主內外皆哭。皇帝詣祭所三奠酒,喪主已下再拜。皇帝退,止哭。從官進名奉慰。皇帝改常服還內。



 《通禮》著:皇帝臨諸王、妃、主、外祖父母、皇后父母、宗戚、貴臣等喪,出宮服常服,至所臨處變服素服。《天聖喪葬令》:皇帝臨臣之喪,一品服錫衰,三品已上緦衰,四品已下疑衰。皇太子臨吊三師、三少則錫衰,宮臣四品已上緦衰,五品已下疑衰。



 輟朝之制。《禮院例冊》:文武官一品、二品喪,輟視朝二日,
 於便殿舉哀掛服。文武官三品喪,輟視朝一日,不舉哀掛服。然其車駕臨問並特輟朝日數,各系聖恩。一品、二品喪皆以翰林學士已下為監護葬事,以內侍都知已下為同監護葬事。葬日,輟視朝一日,皆取旨後行。慶歷五年四月,禮院奏:「準度支員外郎、集賢校理知院曾公亮奏:『朝廷行輟朝禮,並乞以聞哀之明日輟朝,其假日便以充數,仍為永例。如值其日前殿須坐,則禮有重輕,自可略輕而為重,更不行輟朝之禮。』臣今看詳公亮所奏,
 誠於輟朝之間適宜順變。然慮君臣恩禮之情有所未盡,欲乞除人使見辭、春秋二宴合當舉樂,即於次日輟朝,餘乞依公亮所奏。」詔可。



 太平興國六年,守司空兼門下侍郎平章事薛居正薨,準禮,一品喪合輟二日,詔特輟三日。其後鄧王錢俶、太師趙普、右僕射李沆薨,皆一品,合輟二日,詔並特輟五日。二品、三品者,亦有特輟焉。太平興國九年,右諫議大夫、參知政事李穆卒,準禮,諫議大夫不合輟朝,特輟一日。



 開寶二年,羅彥環、魏仁浦
 薨,以郊祀及軍事不輟朝。景德四年,同平章事王顯薨,以皇帝朝拜諸陵,吉兇難於相干,更不輟朝。康定元年,光祿卿鄭立卒,禮官舉故事輟朝,臺官言:「卿、監職任疏遠,恩禮不稱。」自後遂不輟朝。



 孝宗乾道三年四月一日,太常寺言:「皇伯母秀王夫人薨,輟朝五日,內二日不視事。乞自今月二日為始,輟朝至六日止,其二日、三日並不視事。」從之。



 舉哀掛服。尚舍設次於廣德殿或講武殿、大明殿,其後
 皆於後苑壬地。前一日,所司預設舉哀所幕殿,周以簾帷,色用青素。其日,皇帝常服乘輿詣幕殿,侍臣奏請降輿,俟時釋常服,服素服,白羅衫、黑銀腰帶、素紗軟腳帕頭。太常博士引太常卿當御坐前跪,奏請皇帝為某官薨舉哀,又請舉哭,十五舉音,又奏請可止。中書、門下、文武百官進名於崇政殿門外奉慰。皇帝釋素服,服常服,乘輿還內。



 建隆四年,山南東道節度使慕容延釗卒,太祖素服發哀。其後趙普薨,太宗亦如之。景德元年,李沆
 薨,禮官言:「舉哀品秩,雖載禮典,伏緣國朝惟趙普、曹彬曾行茲禮,今望裁自聖恩。」詔特擇日舉哀。自後宰臣薨,皆用此禮。



 真宗乳母秦國延壽保聖夫人卒,以太宗喪始期,疑舉哀,禮官言:「《通禮》:皇帝為乳母緦麻。按《喪葬令》:皇帝為緦,一舉哀止。秦國夫人保傅聖躬,宜備哀榮。況太宗之喪已終易月之制,今為乳母發哀,合於禮典。」從之。



 鄭國長公主薨,禮官言:「降服大功,擇日成服。緣居大行皇太后大祥之內,衰服未除,典禮舊章,以輕包重,酌
 情順變,禮當厭降,望不成服。皇親諸親,亦不制服。」帝曰:「宗室諸王皆不制服,情所未忍。至期當遣諸王就其第成服,及令皇后臨奠,餘如所請。」皇從弟右監門衛大將軍德鈞卒,以皇帝恭謁陵寢,罷舉哀成服。天禧元年,太尉王旦薨,時季秋大享明堂,其日發哀,真宗疑之。禮官言:「祠事在質明之前,成服於既祠之後,於禮無嫌。」詔可。



 康定二年,皇子壽國公昕薨,年二歲,禮官言:「已有爵命,宜同成人。」遂發哀成服。熙寧十年,永國公薨,系無服之
 殤,詔特舉哀成服。



 元祐元年,王安石薨,在神宗大祥之內;司馬光薨,亦在諒冱中,皆不舉哀成服。高宗於劉光世、張俊、秦檜之喪,皆為臨奠,然設幄舉哀成服之禮,未之行也。孝宗乾道三年,始為皇伯母秀王夫人薨,設幕殿後苑壬地,舉哀成服,復舉行焉。



 皇太后、皇后為本族之喪。孝明皇后姊太原郡君王氏卒,中書門下據太常禮院狀:「準禮例,皇後合出就故彰德軍節度使王饒第發哀成服,文武百僚詣其第進名奉慰。」從之。章穆太后
 母楚國太夫人吳氏薨,太常禮院言:「皇帝為外祖母本服小功,詳《開寶通禮》,即有舉哀成服之文;又緣近儀,大功以上方成服,今請皇太后擇日就本宮掛服,雍王以下為外祖母給假。」其後,太后嫡母韓國太夫人薨,亦用此制焉。章獻明肅皇后改葬父母,前一日,皇后詣攢所,俟時詣成服所改服緦。尚儀奏:「請詣靈柩發哭奠酒,退,六宮內人立班奉慰。掩壙畢,皇后詣墳奠獻,再拜,釋服還宮。外命婦進箋奉慰如儀。」



 輟樂。太平興國七年十月,中書言:「今月七日乾明節,選定二十二日大宴。」二十日,參知政事竇偁卒,明日,皇帝親幸其第,臨喪慟哭,設奠還宮,即令罷宴。有司奏:「伏以百司告備,六樂在庭,睿聖至仁,聞哀而罷,是以顯君父愛慈之道,勵臣子忠孝之心。伏請宣付史館,傳錄美實。」詔可。



 天禧二年九月十一日,宴近臣於長春殿,餞河陽三城節度使張旻赴任,以王旦在殯,不舉樂。嘉祐六年三月五日,宰臣富弼母秦國太夫人薨,十七日春宴,禮
 院上言:「君臣父子,家國均同。元首股肱,相濟成體。貴賤雖異,哀樂則同。一人向隅,滿堂嗟戚。今宰臣新在苫塊,欲乞罷春宴聲樂,以表聖人憂恤大臣之意。」詔下,並春宴寢罷。



 賻贈。凡近臣及帶職事官薨,非詔葬者,如有喪訃及遷葬,皆賜賻贈,鴻臚寺與入內內侍省以舊例取旨。其嘗踐兩府或任近侍者,多增其數,絹自五百匹至五十匹,錢自五十萬至五萬,又賜羊酒有差,其優者仍給米麥
 香燭。自中書、樞密而下至兩省五品、三司三館職事、內職、軍校並執事禁近者亡歿,及父母、近親喪,皆有贈賜。宗室期、功、袒免、乳母、殤子及女出適者,各有常數。其特恩加賜者,各以輕重為隆殺焉。



 建隆元年十月,詔:「有死於矢石者,人給絹三匹,仍復其家三年,長吏存撫之。」慶歷二年,詔:「陣亡軍校無子孫者,賜其家錢,指揮使七萬,副指揮使六萬,軍使、都頭、副兵馬使、副都頭五萬。」



 熙寧七年,參酌舊制著為新式:諸臣喪,兩人以上各該支賜
 孝贈,只就數多者給;官與職各該賻贈者,從多給,差遣、權並同,權發遣並與正同。諸兩府、使相、宣徽使並前任宰臣問疾或澆奠已賜不願敕葬者,並宗室不經澆奠支賜,雖不系敕葬,並支賻贈。餘但經問疾或澆奠支賜或敕葬者,更不支賻贈。前兩府如澆奠只支賻贈,仍加絹一百、布一百、羊酒米面各一十。諸支賜孝贈:在京,羊每口支錢一貫,以折第二等絹充,每匹折錢一貫三百文,余支本色;在外,米支白粳米,面每石支小麥五斗,酒
 支細色,餘依價錢。諸文臣卿監以上,武臣元系諸司使以上,分司、致仕身亡者,其賻贈並依見任官三分中給二,限百日內經所在官司投狀,召命官保關申,限外不給。待制、觀察使以上更不召保。



 元豐五年,詔:「鄜延路沒於王事、有家屬見今在本路欲歸鄉者給賻外,其大使臣以上更支行李錢百千,小使臣五十千,差使、殿侍三十千,其餘比類支給。」



 紹興二十六年,詔:「今後命官實因乾辦公事邂逅非理致死者,並遵依舊法。所有李光申
 請於《紹興條》內添注日限指揮,更不施行。」舊法非理致死者,謂焚溺墜壓之類,通判以上賜銀五百兩,餘三百兩,職司已上取旨。初,紹興二年五月,吏部侍郎李光申明立定折跌骨五十餘日,三十日內身亡之人,並支前項銀數。至是,戶部侍郎宋貺言:「自立定日限,後來多是因他病身故之人,子孫規圖賞給,計會所屬,旋作差出名目,陳乞保奏,誠為期罔。」故有是命。



 詔葬。《禮院例冊》:諸一品、二品喪,敕備本品鹵簿送葬者,
 以少牢贈祭於都城外,加璧,束帛深青二、纁二。諸重:一品柱鬲六,五品已上四,六品已下二。諸銘旌:三品已上長九尺,五品已上八尺,六品已上七尺,皆書某官封姓之柩。諸輀車:三品已上油幰、牛絲絡綱施□,兩廂畫龍,幰竿諸末垂六旒蘇;七品已上油幰、施□,兩廂畫雲氣,垂四旒蘇;九品已上無旒蘇;庶人鱉甲車,無幰、巽、畫飾。諸引、披、鐸、翣、挽歌:三品已上四引、四披、六鐸、六翣、挽歌六行三十六人;四品二引、二披、四鐸、四翣、挽歌者四行
 十六人;五品、六品挽歌八人;七品、八品挽歌六人;六品、九品謂非升朝者



 挽歌四人。其持引、披者,皆布幘、布深衣;挽歌,白練幘、白練鞁衣,皆執鐸、綍,並鞋襪。諸四品已上用方相,七品已上用魌頭。諸纛:五品已上,其竿長九尺;已下五尺已上。諸葬不得以石為棺槨及石室,其棺槨皆不得雕鏤彩畫、施方牖檻,棺內不得藏金寶珠玉。



 又按《會要》:勛戚大臣薨卒,多命詔葬,遣中使監護,官給其費,以表一時之恩。凡兇儀皆有買道、方相、引魂車,香、蓋、紙
 錢、鵝毛、影輿,錦繡虛車,大輿,銘旌;儀棺,行幕,各一;挽歌十六。其明器、床帳、衣輿、結彩床皆不定數。墳所有石羊虎、望柱各二,三品以上加石人二人。入墳有當壙、當野、祖思、祖明、地軸、十二時神、志石、券石、鐵券各一。殯前一日對靈柩,及至墳所下事時,皆設敕祭,監葬官行禮。熙寧初,又著新式,頒於有司。



 乾德三年六月,中書令、秦國公孟昶薨,其母李氏繼亡,命鴻臚範禹偁監護喪事,仍詔禮官議定吉兇儀仗禮例以聞。太常禮院言:「檢詳
 故事,晉天福十二年葬故魏王,周廣順元年葬故樞密使楊邠、侍衛使史弘肇、三司使王章例,並用一品禮。墓方圓九十步,墳高一丈八尺,明器九十事,石作六事,音身隊二十人,當壙、當野、祖明、祖思、地軸、十二時神、蚊廚帳、暖帳各一,□車一,挽歌三十六人;拂一、纛一、翣六、輴車、魂車、儀槨車、買道車、志石車各一;方相氏、鵝毛纛、銘旌、香輿、影輿、蓋輿、錢輿、五穀輿、酒醢輿、衣物輿、庖牲輿各一;黃白紙帳、園宅、象生什物、行幕,並志文、挽歌詞、啟
 攢啟奠祝文,並請下有司修制。其儀:太僕寺革輅,兵部本品鹵簿儀仗,太常寺本品鼓吹儀仗,殿中省傘一、曲蓋二、朱漆團扇四,自第導引出城,量遠近各還。贈玉一、纁二,贈祭少牢禮料,亦請下光祿、太府寺、少府監諸司依禮供應。又楚王母依子官一品例,準令文,外命婦一品侍近二人、青衣六人,偏扇、方扇各十六,行鄣三、坐鄣二,白銅飾犢車駕牛馭人四,從人十六,夾車、從車六,傘一、大扇一、團扇二、戟六十。伏緣久不施用,如特賜施行,
 即合於孟昶吉兇仗內相參排列。」詔並令排列祗應,仍俟導引至城外,分半導至西京墳下及葬,命供奉官周貽慶押奉議軍士二指揮防護至洛陽。又賜子玄哲墳莊一區。



 開寶四年,建武軍節度使何繼筠卒,詔遣中使護葬,仍賜寶劍、甲冑同葬。咸平元年,護國軍節度使、駙馬都尉王承衍葬,鹵簿、鼓吹備而不作,以在太宗大祥忌禁內也。元豐五年,崇信軍節度使、華陰郡王宗旦薨,聽以旌節、牌印葬。尋又詔:不即隨葬者徒二年,因而行
 用者罪之。紹興二十四年,太師、清河郡王張俊葬,上曰:「張俊極宣力,與他將不同,恩數務從優厚。」仍賜七梁額花冠貂蟬籠巾朝服一襲、水銀二百兩、龍腦一百五十兩。其後,楊存中薨,孝宗令諸寺院聲鐘,仍賜水銀、龍腦以斂。



 《熙寧新式》:先是,知制誥曾布言:「竊以朝廷親睦九族,故於死喪之際,臨吊賻恤,至於窀穸之具,皆給於縣官,又擇近臣專董其事,所以深致其哀榮而盡其送終之禮。近世使臣沿襲故常,過取饋遺,故私家之費,往往
 倍於公上。祥符中,患其無節,嘗詔有司定其數。皇祐中,又著之《編敕》,令使臣所受無過五百,朝臣無過三百,有違之者,御史奏劾。伏見比歲以來,不復循守,其取之者不啻十倍於著令。乞取舊例裁定酌中之數,以為永式。」詔令太常禮院詳定,令布裁定以聞。



 嘉祐七年,詔大宗正:自今皇親之喪,五年以上未葬者,不以有無尊親新喪,並擇日葬之。初,龍圖閣直學士向傳式言:「故事,皇親系節度使以上方許承兇營葬,其卑幼喪皆隨葬之。自
 慶歷八年後,積十二年未葬者幾四百餘喪,官司難於卒辦,致濮王薨百日不及葬。請自今兩宅遇有尊屬之喪,不以官品為限而葬之。」下判大宗正司、太常禮儀院、司天監議,而有是詔。元祐中,又詔御史臺:「臣僚父母無故十年不葬,即依條彈奏,及令吏部候限滿檢察。尚有不葬父母,即未得與關升磨勘。如失檢察,亦許彈奏。」



 追封冊命。《通禮》:策贈貴臣,守宮於主人大門外設使、副位,使人公服從朝堂受策,載於犢車,各備鹵簿,至主人
 之門降車。使者稱:「有制。」主人降階稽顙,內外皆哭。讀冊訖,主人拜送之。



 國朝之制:有於私第冊之者,有於本道冊之者。私第冊之者,乾德三年,正衙命使冊贈孟昶尚書令,追封楚王是也。本道冊者,建隆元年,故特進、檢校太師、南平王高保融奉敕贈太尉,端拱元年,故守太師、尚書令、鄧王錢俶特追封秦王是也。其儀與《通禮》大略相類,不復錄。



 定謚。王公及職事官三品以上薨贈官同,本家錄行狀上
 尚書省,考功移太常禮院議定,博士撰議,考功審覆,判都省集合省官參議,具上中書門下宰臣判準,始錄奏聞。敕付所司即考功錄牒,以未葬前賜其家。省官有異議者,聽具議聞。蘊德丘園,聲實明著,雖無官爵,亦奏賜謚曰「先生」。



 太平興國八年,詔增《周公謚法》五十五字,美謚七十一字為一百字,平謚七字為二十字,惡謚十七字為三十字。其沉約、賀琛《續廣謚》盡廢。後以直史館胡旦言:「舊制,文武官臣僚皆以功行上下,各賜謚法。近朝
 以來,遂成闕典。建隆以後,臣僚三品以上合賜謚者百餘人,望令史館編錄行狀,送禮官定謚付館,修入國史。」詔:「今後並令禮官取行狀定謚,送考功詳覆,關送史館,永為定式。」



 直集賢院王皞言:「謚者,行之表也。善行有善謚,惡行有惡謚,蓋聞謚知行,以為勸戒。《六典》:太常博士掌王公以下擬謚,皆跡其功德為之褒貶。近者臣僚薨卒,雖官該擬謚,其家自知父祖別無善政,慮定謚之際,斥其繆戾,皆不請謚。竊惟謚法自周公以來,垂為不刊
 之典,蓋以彰善癉惡,激濁揚清,使其身沒之後,是非較然,用為勸懲。今若任其遷避,則為惡者肆志而不悛。乞自今後不必候其請謚,並令有司舉行,如此,則隱匿無行之人,有所沮勸。若須行狀申乞方行擬謚,考諸方冊,別無明證。惟衛公叔文子卒,其子戍請謚。臣謂春秋之時,禮壞樂闕,公叔之卒,有司不能明舉舊典,故至將葬,始請謚於君。且周制,太史掌小喪賜謚,小史掌卿大夫之家賜謚請誄。以此知有司之職,自當舉行,明矣。」詔下
 有司詳定,如皞請焉。



 禮院更議贈安遠軍節度使馬懷德已葬請謚,乃言:「自古作謚,皆在葬前。唐《開元》,三品以上將葬,既啟殯,告贈謚於柩前;無贈者,設啟奠即告謚。……,
 宋綬建議,令官給酒食。其後,又罷贈遺。自此,既葬請謚者甚眾。歲月浸久,官閥行跡,士大夫所不能知,子孫與其門生故吏,志在虛美隱惡,而有司據以加謚,是廢聖人之法,而褟唐庸有司之議也。」詔:「自今得謚者,令葬前奏請;或其家不請,則尚書、太常合議定謚,前葬牒史館及付其家。即褟私
 謚不以實,論如選舉不以實法。既葬請謚者,不定謚。」



\end{pinyinscope}