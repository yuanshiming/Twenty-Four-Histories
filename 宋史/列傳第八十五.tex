\article{列傳第八十五}

\begin{pinyinscope}

 景泰
 王
 信蔣偕張忠郭恩張岊張君平史方盧鑒李渭王果郭諮田敏侍其曙康德輿張昭遠



 景泰,字周卿,普州人。進士起家,補坊州軍事推官。後以尚書屯田員外郎通判慶州,即上言:「元昊雖稱臣,誠恐包藏禍心。當選主將,練士卒,修城池,儲資糧,以備不虞。」三疏不報。俄元昊反,又上《邊臣要略》二十卷。遷都官、知成州,奏《平戎策》十有五篇。



 會有薦泰知兵者,召對稱旨,換左藏庫使、知寧州。任福敗,徙原州。元昊眾十萬,分二道,一出劉璠堡,一出彭陽城,入攻渭州。葛懷敏援劉璠,戰崆峒北,敗沒,敵騎逾平涼,至潘原。泰率兵五千,從間
 道赴原,而先鋒左班殿直張迥逗遛不進,泰斬以徇。遇敵彭陽西,裨將夏侯觀欲退守彭陽,泰弗許,乃依山而陣。未成列,敵騎來犯,泰陰遣三百騎,分左右翼,張旗幟為疑兵。敵欲遁去,將校請進擊,泰止之,遣士搜山,果得伏兵,與戰,斬首千餘級。以功遷西上閣門使、知鎮戎軍兼兵馬鈐轄。久之,領忠州刺史,徙秦鳳路馬步軍總管,卒。



 子思立,熙寧中屢有戰功,為引進使、忠州防禦使、知河州,與董氈部兵戰,沒,後思忠以左藏庫副使、遂州駐
 泊都監擊瀘州夷人,陷於羅個暮山下。兄弟繼死王事,人皆憐其忠。



 王信,字公亮,太原人。家故饒財,少勇悍。大中祥符中,盜起晉、絳、澤、潞數州,信應募籍軍,與其徒生擒賊七十人,累以功補龍、神衛指揮使。部使者表薦,召閱其藝,遷御前忠佐,領河中府、同乾鄜延丹坊州慶成軍管界捉賊,又遷龍衛都虞候兼鄜延巡檢。



 康定初,劉平、石元孫戰於三川,信以所部兵薄賊,斬首數十級。遷捧日都虞候,
 改西京作坊使、知鎮戎軍,徙保安軍兼鄜延路兵馬都監。始至之夕,敵眾號數萬傅城,軍吏氣懾。信領勁兵二千,夜出南門與戰,失其前鋒,因按軍不動。遲明,潛上東山整軍,乘勢而下,擊走之,獲首級、馬牛居多。遷鈐轄兼經略、安撫、招討都監,領貴州刺史。葛懷敏戰敗,信出兵拒敵,俘斬甚眾。進保州刺史,就遷馬步軍都總管。四路置招討使,遂為本路招討副使。累遷馬步軍都虞候、象州防禦使,徙高陽關路。



 王則反貝州,用安撫使明鎬奏,
 為貝州城下都總管。城破,則遁,信率兵執則而還,餘黨自焚死。拜感德軍節度觀察留後,召為步軍副都指揮使,未至,卒。贈武寧軍節度兼侍中。



 蔣偕,字齊賢,華州鄭縣人。幼貧,有立志。父病,嘗刲股以療,父愈,詰之曰:「此豈孝邪?」曰:「情之所感,實不自知也。」舉進士,補韶州司理參軍,以秘書省著作佐郎為大理寺詳斷官。



 密州豪人王澥使奴殺一家四人,偕當澥及奴皆大闢。宰相陳堯佐欲寬澥,判審刑院宋庠與偕持之
 不從,偕以是知名。



 陜西用兵,數上書論邊事,遷秘書丞、通判同州,計置陜西錢糧。逾年,為沿邊計置青白鹽使。用龐籍、範仲淹薦,改北作坊副使、環慶路兵馬都監,歷知汾、涇二州,徙原州。邊民苦屬戶為鈔盜,偕得數輩,腰斬境上,盜為息。遷北作坊使兼本路鈐轄,明珠、康奴諸族數為寇,偕潛兵伺之,斬首四百,擒酋豪,焚帳落,獲馬、牛、羊千計。所俘皆刳割磔裂於庭下,坐客為廢飲食,而偕語笑自若。徙華州兵馬鈐轄。



 湖南蠻唐和內寇,徙潭
 州鈐轄。賊平,知忻州,徙冀州。坐擅率糧草。降知霸州。逾年,徙恩州,領韶州刺史。屬兵糧乏絕,朝廷方募民入粟,增虛直,給券詣京師射取錢貨,謂之交鈔,患未有應令者,偕使州倉謬為入粟數,輒作鈔,遣屬官持至京師轉貿,得緡錢以補軍食。為御史彈奏,降知坊州。



 儂智高反,除宮苑使、韶州團練使,為廣南東西路鈐轄。賊方圍廣州。偕馳傳十七日至城下。戰士未集,會儂智高徙軍沙頭,安撫楊畋檄偕焚糧儲,退保韶州。坐此,降潭州駐泊
 都監,再降北作坊使、忠州刺史。命未至,軍次賀州太平場,賊夜入營,襲殺之。贈武信軍節度觀察留後。



 初,偕入廣州,即數知州仲簡曰:「君留兵自守,不襲賊,又縱步兵馘平民以幸賞,可斬也。」簡曰:「安有團練使欲斬侍從官?」偕曰:「斬諸侯劍在吾手,何論侍從!」左右解之,乃止。卒以輕肆敗。



 張忠,開封人。初隸龍騎備征,選為教駿。有軍校恣掊斂,忠歐殺之,坐配鼎州。既遁去為盜,復招出。隸龍猛軍,以
 材武補三班借職、陜西總管司指使。數攻破堡砦,殺劇賊張海、郭邈山。從平恩州,功第一。累遷如京使、資州刺史,歷真定府、定州、高陽關、京東西路兵馬鈐轄。



 儂智高反,就移廣東,領英州團練使。初,智高圍廣州,時洪州駐泊都監蔡保恭及知英州蘇緘以兵八千人據邊渡村,扼賊歸路,忠奪而將之。謂其下曰:「我十年前一健兒,以戰功為團練使,若曹勉之。」於是不介騎而前。會先鋒遇賊奔,忠手拉賊帥二人,馬陷濘,不能奮,遂中標槍死。錄
 其父率府副率致仕餘慶為左監門衛大將軍,賜第一區,給半俸終身;封其母為河內郡夫人;弟願遷右班殿直、閣門祗候;官其子永壽、永吉、永德及其婿劉錞凡四人。封長女為清河縣君。



 郭恩,開封人。初隸諸班,出為左侍禁、閣門祗候,歷延州西路都巡檢、環州肅遠砦主,累遷內殿承制、秦鳳路兵馬都監。開古渭州路,為前鋒,斬首九百餘級,擢崇儀副使。會掌烏族叛,又率兵攻討,斬首八十五級,遷六宅副
 使。累勞,補崇儀使,為秦隴路兵馬鈐轄,徙並、代州鈐轄,管勾麟府軍馬事。



 夏人歲侵屈野河西地,至耕獲時,輒屯兵河西以誘官軍。經略使龐籍每戒邊將,斂兵河東毋與戰。嘉祐二年,自正月出屯,至三月然後去。通判並州司馬光行邊至河西白草平,數十里無寇跡。是時,知麟州武戡、通判夏倚已築一堡為候望,又與光議曰:「乘敵去,出不意可更增二堡,以據其地。請還白經略使,益禁兵三千、役兵五百,不過二旬,壁壘可城。然後廢橫戎、
 臨塞二堡,徹其樓櫓,徙其甲兵,以實新堡,列烽燧以通警急。從衙城紅樓之上,俯瞰其地,猶指掌也。有急,則州及橫陽堡出兵救之;敵來耕則驅之,種則蹂踐之;敵盛則入堡以避。如是,則堡外必不敢耕種,州西五六十里之內晏然矣。」籍遂檄麟州如其議。



 五月,恩及武戡、走馬承受公事內侍黃道元等以巡邊為名,往按視之。會言□者言,敵兵盛屯沙黍浪,恩欲止不行。道元怒,以言脅恩,夜率步騎一千四百餘人,不甲者半,循屈野河北而行,
 無復部伍。夏人舉火臥牛峰,戡指以謂恩曰:「敵已知吾軍至矣。」道元曰:「此爾曹故欲沮我師。」及聞鼓聲,道元猶不信。行至谷口,恩欲休軍,須曉乃登山。道元奮衣起曰:「幾年聞郭恩名,今日懦怯與賈逵何殊?」恩亦慍曰:「不過死耳!」乃行。比明,至忽裏堆。敵數十人皆西走,相去數十步,止。恩等踞胡床,遣使騎呼之,敵不應,亦不動。俄而起火,敵騎張左右翼,自南北交至。堆東有塹,其中有梁,謂之「斷道堰」。恩等東據梁口,與力戰,自旦至食。時敵自兩
 旁塹中攀緣而上,四面合擊,恩眾大潰。



 夏倚方在紅樓,見敵騎自西山大下,與推官劉公弼率城中諸軍,閉門乘城。武戡走東山,趣城東,抉門以入。恩、道元及府州寧府砦兵馬都監劉慶皆被執。使臣死者五人,軍士三百八十七人,已馘耳鼻得還者百餘人,亡失器甲甚眾。恩不肯降,乃自殺。贈同州觀察使,封其妻為京兆郡君,錄其子弟有差,給舊俸三年。武戡坐棄軍除名,編管江州。



 張岊,字子云,府州府穀人。以貲為牙將,有膽略,善騎射。
 天聖中,西夏觀察使阿遇有子來歸。阿遇寇麟州,虜邊戶,約還子然後歸所虜。麟州還其子,而阿遇輒背約。安撫使遣岊詰問,岊徑造帳中,以逆順諭阿遇,阿遇語屈,留岊共食。阿遇袖佩刀,貫大臠啖岊,岊引吻就刀食肉,無所憚。阿遇復弦弓張鏃,指岊腹而彀,岊食不輟,神色自若。阿遇撫岊背曰:「真男子也。」翌日,又與岊縱獵,雙兔起馬前,岊發兩矢,連斃二兔。阿遇驚服,遺岊馬、橐駝,悉歸所虜。州將補為來遠砦主。手殺偽首領,奪其甲馬。時
 年十八,名動一軍。



 元昊犯鄜延,詔麟府進兵。岊以都教練使從折繼閔破浪黃、黨兒兩族,射殺數十人,斬偽軍主敖保,以功補下班殿侍、三班差使。



 時敵騎方熾,中人促賜軍衣,至麟州,不得前。康德輿管勾軍馬司事,遣岊馳騎五十往護之。至青眉浪,遇賊接戰,流矢貫雙頰,岊拔矢,鬥愈力,奪馬十二匹而還。賊兵攻府州甚急,城西南隅庳下,賊將登,眾囂曰:「城破矣!」岊乘陴大呼搏賊,賊稍卻,飛矢中右目,下身被三創,晝夜督守。又帥死士開
 關,護州人汲於河,訖圍解,城中水不乏,以勞,遷右班殿直。然賊嘗往來邀奪饋運,以岊為麟、府州道路巡檢。至深柏堰,遇賊數千,分兵追擊,斬首百餘級,奪兵械、馬牛數百。近郊民田,比秋成未敢獲,岊以計干張亢,得步卒九百人護之,大敗賊於龍門川。從諸將通麟州糧道,破賊於柏子砦。改左班殿直。



 內侍宋永誠傳詔砦下,岊護永誠,遇賊三松嶺。賊以精騎挑戰,矢中岊臂,猶躍馬左右馳射,諸將乘勝而進,賊皆棄潰。特改西頭供奉官,又
 遷內殿崇班。賊破豐州,岊與諸將一日數戰,破容州刺史耶布移守貴三砦,俘獲萬計。遷禮賓副使。



 明鎬在河東,以岢嵐軍當云、朔路,奏岊為麟府路駐泊都監兼沿邊都巡檢使,駐岢嵐。張亢修並砦堡障,初議置安豐砦於石臺神,岊以為非要害之地,遂徙砦於生地骨堆以扼賊。左右親信咸曰:「擅易砦地可乎?」岊曰:「茍利國家,得罪無憾也。」卒易之。已而本道上言,左遷絳州兵馬都監。二州未解嚴,復麟府駐泊都監,屯安豐。累遷洛苑使。嘗
 從數騎夜入羌中偵機事,既還,羌覺追之,岊隨羌疾馳,效羌語,與羌俱數里,乃得脫。前後數中流矢,創發臂間,卒。



 張君平,字士衡,磁州滏陽人。以父承訓與契丹戰死,補三班差使殿侍、黔州指揮使。獠兵屢入寇,君平引兵擊破之,以功遷奉職,除駐泊監押,徙容、白等州巡檢。又以捕賊功,遷右班殿直。



 謝德權薦君平河陰窖務,擢閣門祗候,管勾汴口。建言:歲開汴口,當擇其地;得其地,則水
 湍駛而無留沙,歲可省功百餘萬。又請沿河縣植榆柳,為令佐、使臣課最,及瘞汴河流尸。悉從其言。天聖初,議塞滑州決河,以君平習知河事,命以左侍禁簽書滑州事兼修河都監。既而河未塞,召同提點天封府界縣鎮公事。以嘗護滑州堤有功,特遷內殿崇班。君平以京師數罹水災,請委官疏鑿近畿諸州古溝洫,久之,稍完,遂詔畿內及近畿州縣長吏,皆兼管勾溝洫河道。



 自畿至泗州,道路多群寇,君平請兩驛增置使臣,專主捕盜,而
 罷夾河巡檢,於是行者無患。復為滑州修河都監,遷供備庫副使。河平,改西作坊使,就遷鈐轄,卒。



 君平有吏材,尤明於水利,自議塞河,朝廷每訪以利害。河平,君平且死,論者惜之。錄三子官。子鞏,皇祐中,以尚書虞部員外郎為河陰發運判官,管勾汴口,嗣其父職云。



 論曰:孔子謂:「暴虎馮河,死而無悔者,不與也。」老氏曰:「佳兵者不祥。」景泰輩或起書生,或奮行伍,或出亡命,非有將率之材也。泰、信以區區之卒,嘗摧西夏之強鋒,頗知
 持重以制敵耳。蔣、張輕肆自用,竟殞於烏合之寇。恩怵道元之勢,身啖虎口,守義不屈,猶足尚也。岊之驍勇,固非臨事而懼者。君平死戰之子,乃明習水利,以吏材稱,亦可謂善變矣。



 史方,字正臣,開封人。應《周易》學究不中,補西第二班殿侍,再遷三班奉職,為潭、澧、鼎沿邊同巡檢,改右班殿直、閣門祗候。會澧州訴民下溪州蠻侵其土地,遣乘驛往視。自竹疏驛至申文崖,復地四百餘里,得所掠五百餘
 人,又置澧州、武口、楊泉、索溪四砦,以扼賊沖。就知邵州,徙澧州,遷右待禁。



 天禧中,下溪州蠻彭仕漢寇辰州,殺巡檢王文慶。方勒兵入溪洞討捕,降其黨李順同等八百餘人,誅其尤惡者社忽等十九人。遷西頭供奉官、知辰州兼沿邊溪洞都巡檢使,修南、北江五砦,徙夔州。時富、順州蠻田彥晏寇施州,焚暗利砦。方領兵直抵富、順,蕩其巢穴,窮追彥晏至七女柵,降之。遷內殿崇班,改內殿承制,奉使契丹,以供備庫副使知環州、環慶路兵馬
 都監。



 先是,磨媚、浪TA、托校、拔新、兀二、兀三六族內寇,方諭以恩信,乃傳箭牽羊乞和。減禁兵五千,徙內地以省邊費。徙慶州,遷禮賓使兼環慶路兵馬鈐轄,復知環州。歲餘,遷愛州刺史,為益州鈐轄,徙秦鳳路,遷西京作坊使,卒。



 盧鑒,字正臣,金陵人。累舉進士不中,授三班奉職、監坊州酒稅,以右班殿直為鄜延路走馬承受公事。李繼遷寇邊,與總管王榮敗走之;又與鈐轄張崇貴擊賊,焚
 其積聚,斬首級而還。擢閣門祗候,為本路兵馬都監。復出蕩族帳,獲羊牛萬計。徙鳳翔、秦隴、階、成等州提點賊盜公事,尋為都巡檢使,徙利州都監。



 初,繼遷聲言石隕帳前,有文曰:「天誡爾勿為中國患。」鑒時為承受,入奏事,真宗問之,鑒曰:「此詐為之以欺朝廷也,宜益為備。」至是,繼遷陷靈武,帝思其言,特遷右侍禁、知儀州。州有制勝關,最號險要,繼遷欲乘虛襲取之,放言將由此大入。諜者以告。有詔徙老幼、芻粟於內地。鑒曰:「此奸謀也,且示虜
 弱,搖民心,臣不敢奉詔。」卒不徙,已而賊亦不至。再遷西頭供奉官、知利州。



 會歲饑,以便宜發倉粟振民。秩滿,民請留,詔留一年。提點河東路刑獄,歷知保州、廣信軍、原州,就為環慶路都監兼知慶州,徙環州。平磨媚族於合道鎮。坐事徙知丹州。累遷西京左藏庫使、恩州刺史,為環慶路鈐轄兼知環州,改西上閣門使、秦州,卒。



 李渭,字師望,其先西河人,後家河陽。進士起家,為臨穎縣主簿,累官至太常博士。會河決滑州,天聖初,上治河
 十策,參知政事魯宗道奉詔行河。秦渭換北作坊副使,與張君平並為修河都監。未幾皆罷,以渭為鄆州兵馬都監,徙知憲州,又知鳳州兼階、成州鈐轄。



 初,屬戶寇陷階州沙灘砦,渭至,詰所以然者,乃都校趙釗擾之,奏流釗道州,以恩信諭酋帥,復其砦。遷軍器庫副使,歷知原、環、慶三州。時詔舉勇略任邊者,李諮以渭應詔。徙益利路兵馬鈐轄,領惠州刺史,遷東八作使,擢西上閣門使。徙鄜延路,再遷四方館使。



 寶元元年,元昊將山遇率其
 族來歸,且言元昊反狀,渭與知州郭勸謀,卻之。既而元昊果反。又與勸奏,以為元昊表至猶稱臣,可漸屈以禮。朝廷初以渭兼知鄜州,坐是貶為尚食使、知汝州,徙磁州。元昊犯邊,言者益歸罪於渭,復降右監門衛將軍、白波兵馬都監,卒。



 王果字仲武,深州饒陽人。舉明法。歷大理寺詳斷官,遷光祿寺丞,以太子右贊善大夫為審刑院詳議官,遷殿中丞。奏邊策,試舍人院,改衣庫副使、知永寧軍,更尚食
 使、知保州。



 契丹謀致書求關南地,使未至,果購諜者先得其稿,奏之,擢領賀州刺史兼高陽關路兵馬鈐轄。中官楊懷敏領沿邊屯田事,大廣塘水,邊臣莫敢言,果獨抗辨水侵民田,無益邊備。懷敏怒,訴果以不法,左遷青州兵馬都監。歷永興軍兵馬鈐轄、知隴州。



 俄詔還,遷皇城使、河北沿邊安撫副使,徙知定州兼真定路兵馬鈐轄。叛卒據保州,果坐多傷士眾,徙知密州。又知忻州、鄜州,權秦鳳路兵馬總管,遷西上閣門使,徙知滄州,卒。



 郭諮,字仲謀,趙州平棘人。八歲始能言,聰敏過人。舉進士,歷通利軍司理參軍、中牟縣主簿,改大理寺丞、知濟陰縣。建言:「澶、滑堤狹,無以殺大河之怒,故漢以來河決多在澶、滑。且黎陽九河之原,今若引河出汶子山下,穿金堤,與橫□合,以達於海,則害可息。」詔本道使者共議,弗合。部夫坐小法,監通利軍稅。



 洺州肥鄉縣田賦不平,歲久莫治,轉運使楊偕遣諮攝令以往。既至,閉閣數日,以千步方田法四出量括,遂得其數,除無地之租者四
 百家,正無租之地者百家,收逋賦八十萬,流民乃復,偕奏其才,遷殿中丞、知館陶縣。



 康定西征,諮上戰略,獻《拒馬槍陣法》,其制利山川險隘,以騎士試上前,擢通判鎮戎軍,募兵教習。會三司議均稅法,知諫院歐陽修言,惟諮方田法簡而易行,詔諮與孫琳均蔡州上蔡縣稅。以母憂免官。用宰相呂夷簡薦,起為崇儀副使、提舉黃御河堤岸。



 時富弼使契丹,諮入對,陳大水禦戎之要。詔與楊懷敏、鄧保信行河,其議「決黎陽大河,下與胡蘆、滹沱、
 後唐河以注塘泊,混界河,使東北抵於海,上溢鸛鵲陂,下注北當城,南視塘泊,界截虜疆,東至海口,西接保塞。惟保塞正西四十里,水不可到,請立堡砦,以兵戍之。」詔儲用興役,會契丹約和而止。知丹、利二州。



 王則叛,立彥博薦諮知冀州,運糧助攻討。賊平,徙忻州,開渭渠,導汾水,興水利,置屯田。轉運使任顓言諮有巧思,自為兵械皆可用。詔以所作刻漏、圓□盾、獨轅弩、生皮甲來上,帝頗嘉之。除益州路兵馬鈐轄,累遷英州刺史,後為契丹祭
 奠副使、知汾州。未行,言獨轅弩可試,改鄜延路兵馬鈐轄,許置弩五百,募士兵教之。既成,經略使夏安期言其便,詔立獨轅弩軍。以西上閣門使知潞州。言懷、保二郡旁山,可以植稻;定武唐河抵瀛、莫間,可興水田。又作鹿角車、陷馬槍,請廣獨轅弩於他道。詔諮置弩千分給並、潞,諮因上疏曰:「臣自冠武弁,未嘗一日不思御戎之計。頃使契丹,觀幽燕地方不及三百里,無十萬人一年之費,且烏合之眾,非二十萬不敢舉。若以術制之,使舉不
 得利,居無以給,不逾數年,必棄幽州而遁。臣慶歷初經書河北大水,界斷敵疆,乃其術也。臣所創車弩可以破堅甲,制奔沖,若多設之,助以大水,取幽薊如探囊中物爾。」



 時三司議均田租,召還,諮陳均括之法四十條。復上《平燕議》曰:「契丹之地,自瓦橋至古北口,地狹民少。自古北口至中原,屬奚、契丹,自中原至慶州,道旁才七百餘家。蓋契丹疆土雖廣,人馬至少,儻或南牧,必率高麗、渤海、黑水、女真、室韋等國會戰,其來既遠,其糧匱乏。臣聞
 以近待遠,以佚待勞,以飽待饑,用兵之善計。又聞得敵自至者勝,先據便地者佚。以臣所見,請舉慶歷之策,合眾河於塘泊之北界,以限戎馬,然後以景德故事,頓兵自守。步卒十二萬,騎卒三萬,強壯三萬,歲計糧餉百八十三萬六千斛。又傍河郡邑,可以水運以給保州應援。以拒馬車三千,陷馬槍千五百,獨轅弩三萬,分選五將,臣可以備其一,來則戰,去則勿追。幽州糧儲既少,敵不可久留,不半年間,當遁沙漠。則進兵斷古北口,砦松亭
 關,傳檄幽薊,燕南自定。且彼之所恃者,惟馬而已。但能多方致力,使馬不獲伸用,則敵可破,幽燕可取。」帝壯其言,詔置獨轅弩二萬,同提舉百司及南北作坊,以完軍器。



 諮嘗謂:作汴乘索河三十六陂之流,危京師,請自鞏西山七里店孤柏嶺下鑿七十里,導洛入汴,可以四時行運。詔都水監楊佐同往計度。歸,未及論功而卒。



 田敏,字子俊,本易州牙吏。雍熙中,王師討幽薊,曹彬進兵涿州,敵斷其後。王繼恩募勇士持書抵彬,敏應募,間
 行由祁溝關達涿州。彬得詔,選壯士五十人衛敏還,道遇賊,力戰,四十八人死,敏與兩人者,僅以身免。彬上其事,太宗召見,復令繼詔諭彬。師還,補敏易州靜砦指揮使。



 端拱初,以所部兵屯定州。契丹攻北唐河,大將李繼隆遣部將逆戰,為敵所乘。奄至水南。敏以百騎奮擊,敵懼,退水北,遂引去。又出狼山,襲契丹,至滿城,獲首級甚眾。既而敵陷易州,敏失其家所在。帝擢敏本軍都虞候,賜白金三百兩,使間行求其父母,得之以歸。徙屯鎮州,
 而升其指揮為內員僚直。



 李繼隆討夏州,奏隸麾下。敏率兵至靈州橐駝口雙堆西,遇敵,斬首三千級,獲羊馬、橐駝、鎧仗數萬計。繼隆上其功,遷御前忠佐馬步軍副都軍頭。既而又從傅潛於定州。時契丹斷蒲陰路,城中有神勇軍士千餘人。屬敵兵盛,不敢戰,敏率輕銳援出之。真宗幸天雄軍,詔敏隸高瓊,使追賊至寧遠軍,以功領涿州刺史。王均亂西川,從招安使雷有終敗賊於靈池山。賊平,遷馬步軍都軍頭。



 咸平中,契丹復入寇,敏從
 王顯為鎮、定先鋒,大敗契丹於遂城西羊山,斬其酋長。真授單州刺史,後為邢州兵馬鈐轄。未幾,從王起屯定州,遇契丹於望都,逆戰,斬首二千餘級。徙北平砦兵馬鈐轄,領騎兵五千以當其沖。



 先是,兩地供輸民多為契丹鄉導,敏自魚臺北悉驅南徙,凡七百餘戶,送定州。遷北平砦總管,賜御劍,聽以便宜從事。至是,契丹復入寇,復與敵戰楊村,敗之。敏諜知契丹主去北平十里蒲陰駐砦,敏夜率銳兵,襲破其營帳。契丹主大驚,問撻覽曰:「
 今日戰者誰?」撻覽曰:「所謂田廂使者。」契丹主曰:「其鋒銳不可當。」遂引眾去。



 敵攻瀛州不下,欲乘虛犯貝、魏,詔敏與魏能、張凝三路兵,入敵境縱擊,以牽其勢。敏出西路,抵易州南十里,屯師石村,虜獲人畜、鎧仗以萬計。尋詔三路兵還定州,敏遇敵於鎮州之北馬頭嶺,復大破之。契丹請和,乃徙敏鎮定路都鈐轄,遷本州團練使,充鎮定路總管。徙永興軍、陜州,歷鄜延、環慶、鳳翔三路,久之,為環慶路都總管。



 時後橋屬羌數擾邊,敏誅違命者十
 八族,又敗羅骨於三店川,遷鄭州防禦使、涇原路總管,後徙環慶。坐與部豪往還納賂為不法,降左屯衛大將軍、昭州防禦使。既而以虢州圍練使知隰州,復為環慶路都總管、儀州防禦使,卒。敏在邊二十餘年,凡遷授,多以功伐,雖晚不自飭,而朝廷亦優容之。



 侍其曙,字景升。父稹,左監門衛大將軍。曙少舉進士不第,以父任為殿前承旨,改右班殿直。咸平中,以閣門祗候為蘇、杭、湖、秀等州都巡檢使。遷左侍禁,領東西排岸
 司,與謝德權提舉在京倉草場。嘗於倉隙地牧牛羊,為德權所訟。真宗以問德權曰:「牛羊食倉粟邪?」曙聞而自劾,帝勉諭之。它日,召曙問:「汝才孰與德權?」對曰:「德權畏法慎事,臣乃敢於官倉牧牛羊,是不及也。」人多稱之。



 鄂州男子聞人若挫,告其徒永興民李琰將作亂,命曙同度支判官李應機往按之。至則設方略,捕琰黨三十餘人,皆伏法。琰辭連己所不快者數十人,一切不問。青州卒龐德訟其校李緒謀以眾叛,帝疑其誣,又命曙至青
 州,與通判魏德升同至劾,無驗,遂棄德市。知青州張齊賢奏曙擅戮人,帝曰:「不爾,無以安被告者。」曙還,奏德憚緒治軍嚴,故誣之。帝擢緒本軍虞候,而進曙東頭供奉官。初,太宗平河東,建塔於太原故城,塔毀,帝欲新之,遣內待經度,計工二百萬。帝疑,命曙往,減費十九。改內殿崇班。



 祥符二年,黎州夷人為亂,詔曙乘驛往招撫,其酋首納款,殺牲為誓。曙按行鹽井,夷人復叛。曙率部兵百餘,生擒首領三人,斬首數十級。因上言蠻阻險拒命,請
 必加討。詔知慶州孫正辭、環慶駐泊都監張繼勛領陜西兵,同曙俱進,所至皆降。曙又言:王師已至而方出,請誅之。真宗謂王旦曰:「已降而殺之,何以信四夷?」不許。夷人平,遷內殿承制,再遷如京副使、知登州。



 會歲饑,請漕江、淮米以振貧乏,活者甚眾。累遷西京作坊使、惠州刺史、知桂州,徙滑州,遷西上閣門使,徙鄆州,提舉在京諸司庫務,卒。曙為人沉敏,有幹略,善論利害事,朝廷數任使之。



 康德輿,字世基,河南洛陽人。父贊元,嘗以作坊使從曹光實襲李繼遷,獲其母妻,擢崇儀使、武州刺史。贊元死,真宗追其功,錄德輿三班奉職,遷右班殿直、涇原路走馬承受,擢閣門祗候。河嚙陽武埽,詔遣德輿完築。歷開封府西路都巡檢、勾當榷貨務,皆兼領埽事。改巡護開府等六州黃河堤岸。



 天聖中,使夏州,賜趙德明冬服。夏人謂曰:「前康將軍戰靈武者,非先世邪?」德輿懼其復仇,紿曰:「非也。」還,勾當汴口,改西頭供奉官。用樞密使曹
 利用薦,遷內殿崇班、河陰兵馬都監,建沿汴斗門以節水。會積雨,汴水將溢,德輿請自京西導水入護龍河,水得不溢。歷知原州、慶州,益州路兵馬鈐轄,久之,領昭州刺史,徙並代兵馬鈐轄、管勾麟府路軍馬事。



 有蕃部乜羅為殿侍,求錦袍、驛料,德輿不與,乜羅頗出怨言。後有譖乜羅與賊通,戰則反射漢人,乜羅無以自明,乃謀附賊。指揮張岊聞之,召乜羅與飲,乜羅泣曰:「我豈附賊者邪?蓋逃死耳。」岊以告德輿:「乜羅叛,信矣,不可不殺。」元昊
 方屢入寇,德輿不聽,曰:「今日豈殺蕃部時邪?」岊曰:「叛者特乜羅,非眾所欲也,請為君召與飲,僕崖谷中,聲言墮馬死,安知漢殺之?」德輿猶豫不決,以問所親,所親惡岊,短毀之,岊計不得行。



 知府州折繼閔聞賊將至,以告德輿,德輿怒曰:「君不召之,何以知其來也!」賊果以乜羅為響導,自後河川入襲府州。蕃漢欲入城,德輿閉門不納,或降賊,或為賊所殺,不可勝計。賊既圍府州,德輿與馬步軍副總管王元、兵馬鈐轄楊懷忠按兵不出戰,但移
 文轉運司調軍食。轉運副使文彥博籍民輦運,至境以俟,而德輿等終不出。及陷豐州,才出屯州城數里,三日而還。居民望見,以謂寇復至,皆棄其所繼,入保城郭。然朝廷不悉聞,輿與止坐不出戰,降為東染院使、河陽兵馬都監。尋復昭州刺史、知保州,徙真定府定州路總管,歷知代、石、儀三州,大名府路鈐轄,提舉金堤,累遷西上閣門使。



 至和中,河決小吳埽,破東堤頓丘口,居民避水者趨堤上,而水至不得達,德輿以巨船五十,順流以濟
 之,遂免墊溺。復領果州團練使、知冀州,徙趙州。有告雲翼卒謀以上元夜劫庫兵為亂,德輿會賓屬燕飲自若,陰遣人捕首謀誅之。徙陳州鈐轄,卒。



 張昭遠,字持正,滄州無棣人。父凝,殿前都虞候、寧州防禦使。契丹內寇,凝與康保裔伏兵瀛州,陷圍中。昭遠年十八,挺身掖出之,擢左班殿直、寄班祗候。每出使還,奏利害,多稱旨。為忻州都巡檢,改閣門祗候、知狄山軍,管勾河東緣邊安撫司,再遷內殿崇班。



 天禧初,閣門副使
 缺員,樞密院方奏擬人,真宗曰:「朕有人矣。張昭遠知邊略,曹儀習朝儀,可並除西上閣門副使。」俄為河北緣邊安撫副使,尋知瀛州,改東上閣門副使、知定州,以引進副使復知瀛州,遷西上閣門使、知雄州。獻言歲會四榷場入中銀,帝謂輔臣曰:「先朝置榷場,所以通貨,非所以計貿易之利也。」



 會大雨,陂塘大溢,昭遠勒兵築長堤,以捍其沖。徙鄜延路兵馬鈐轄,進都鈐轄,築堡成平川。領忠州刺史、知成德軍,遷四方館使。滹沱河決,壞城郭,乃
 修五關城,外環以堤,民至今為利。擢捧日天武四廂都指揮使、新州防禦使,歷步軍馬軍都虞候、嘉州防禦使,知代州。召還,改莫州防禦使,罷管軍,授左龍武軍大將軍、昭州防禦使,卒。特贈應州觀察使。



 論曰:郭諮以其智巧材略,自見於功利之間,有足稱者。曙,抑其次也,餘皆碌碌者矣。如方之御寇,鑒之料敵,王果持法峭深,治軍嚴辦,茲其長也。田敏屢有戰功,而貪墨敗度,幸容於時。李渭治無遠略,一失機會,關中兵
 禍,數年不解。德輿閉城以棄其民,昭遠計榷場所入,焉知聖人懷柔之意哉。



\end{pinyinscope}