\article{列傳第一百一十七}

\begin{pinyinscope}

 李綱上



 李綱,字伯紀,邵武人也,自其祖始居無錫。父夔,終龍圖閣待制。綱登政和二年進士第,積官至監察御史兼權殿中侍御史,以言事忤權貴,改比部員外郎,遷起居郎。



 宣和元年,京師大水,綱上疏言陰氣太盛,當以盜賊外患為憂。朝廷惡其言,謫監南劍州沙縣稅務。



 七年,為太常少卿。時金人渝盟,邊報狎至,朝廷議避敵之計,詔起師勤王,命皇太子為開封牧,令侍從各具所見以聞。綱上御戎五策,且語所善給事中吳敏曰:「建牧之議,豈非欲委以留守之任乎?巨敵猖獗如此,非傳以位號,不足以招徠天下豪傑。東宮恭儉之德聞於天下,以守宗社可也。公以獻納論思為職,曷不為上極言之。」敏曰:「監國
 可乎?」綱曰:「肅宗靈武之事,不建號不足以復邦,而建號之議不出於明皇,後世惜之。主上聰明仁恕,公言萬一能行,將見金人悔禍,宗社底寧,天下受其賜。」翌日,敏請對,具道所以,因言李綱之論,蓋與臣同。有旨召綱入議,綱刺臂血上疏云:「皇太子監國,典禮之常也。今大敵入攻,安危存亡在呼吸間,猶守常禮可乎?名分不正而當大權,何以號召天下,期成功於萬一哉?若假皇太子以位號,使為陛下守宗社,收將士心,以死捍敵,天下可保。」
 疏上,內禪之議乃決。



 欽宗即位,綱上封事,謂:「方今中國勢弱,君子道消,法度紀綱,蕩然無統。陛下履位之初,當上應天心,下順人欲。攘除外患,使中國之勢尊;誅鋤內奸,使君子之道長,以副道君皇帝付托之意。」召對延和殿,上迎謂綱曰:「朕頃在東宮,見卿論水災疏,今尚能誦之。」李鄴使金議割地,綱奏:「祖宗疆土,當以死守,不可以尺寸與人。」欽宗嘉納,除兵部侍郎。



 靖康元年,以吳敏為行營副使,綱為參謀官。金將斡離不兵渡河,徽宗東幸,
 宰執議請上暫避敵鋒。綱曰:「道君皇帝挈宗社以授陛下,委而去之可乎?」上默然。太宰白時中謂都城不可守,綱曰:「天下城池,豈有如都城者,且宗廟社稷、百官萬民所在,舍此欲何之?」上顧宰執曰:「策將安出?」綱進曰:「今日之計,當整飭軍馬,固結民心,相與堅守,以待勤王之師。」上問誰可將者,綱曰:「朝廷以高爵厚祿崇養大臣,蓋將用之於有事之日。白時中、李邦彥等雖未必知兵,然籍其位號,撫將士以抗敵鋒,乃其職也。」時中忿曰:「李綱莫能
 將兵出戰否?」綱曰:「陛下不以臣庸懦,儻使治兵,願以死報。」乃以綱為尚書右丞。



 宰執猶守避敵之議。有旨以綱為東京留守,綱為上力陳所以不可去之意,且言:「明皇聞潼關失守,實時幸蜀,宗廟朝廷毀於賊手,範祖禹以為其失在於不能堅守以待援。今四方之兵不日雲集,陛下奈何輕舉以蹈明皇之覆轍乎?」上意頗悟。會內侍奏中宮已行,上色變,倉卒降御榻曰:「朕不能留矣。」綱泣拜,以死邀之。上顧綱曰:「朕今為卿留。治兵禦敵之事,專
 責之卿,勿令有疏虞。」綱皇恐受命。未幾,復決意南狩,綱趨朝,則禁衛擐甲,乘輿已駕矣。綱急呼禁衛曰:「爾等願守宗社乎,願從幸乎?」皆曰:「願死守。」綱入見曰:「陛下已許臣留,復戒行何也?今六軍父母妻子皆在都城,願以死守,萬一中道散歸,陛下孰與為衛?敵兵已逼,知乘輿未遠,以健馬疾追,何以御之?」上感悟,遂命輟行。綱傳旨語左右曰:「敢復有言去者斬!」禁衛皆拜伏呼萬歲,六軍聞之,無不感泣流涕。



 命綱為親征行營使,以便宜從事。綱
 治守戰之具,不數日而畢。敵兵攻城,綱身督戰,募壯士縋城而下,斬酋長十餘人,殺其眾數千人。金人知有備,又聞上已內禪,乃退。求遣大臣至軍中議和,綱請行。上遣李梲,綱曰:「安危在此一舉,臣恐李梲怯懦而誤國事也。」上不聽,竟使梲往。金人須金幣以千萬計,求割太原、中山、河間地,以親王、宰相為質。梲受事,自不措一辭,還報。綱謂:「所需金幣,竭天下且不足,況都城乎?三鎮,國之屏蔽,割之何以立國?至于遣質,即宰相當往,親王不當
 往。若遣辯士姑與之議所以可不可者,宿留數日,大兵四集,彼孤軍深入,雖不得所欲,亦將速歸。此時而與之盟,則不敢輕中國,而和可久也。」宰執議不合,綱不能奪,求去。上慰諭曰:「卿第出治兵,此事當徐議之。」綱退,則誓書已行,所求皆與之,以皇弟康王、少宰張邦昌為質。



 時朝廷日輸金幣,而金人需求不已,日肆暑掠。四方勤王之師漸有至者,種師道、姚平仲亦以涇原、秦鳳兵至。綱奏言:「金人貪婪無厭,兇悖已甚,其勢非用師不可。且敵
 兵號六萬,而吾勤王之師集城下者已二十餘萬;彼以孤軍入重地,猶虎豹自投檻阱中,當以計取之,不必與角一旦之力。若扼河津,絕餉道,分兵復畿北諸邑,而以重兵臨敵營,堅壁勿戰,如周亞夫所以困七國者。俟其食盡力疲,然後以一檄取誓書,復三鎮,縱其北歸,半渡而擊之;此必勝之計也。」上深以為然,約日舉事。



 姚平仲勇而寡謀,急於要功,先期率步騎萬人,夜斫敵營,欲生擒乾離不及取康王以歸。夜半,中使傳旨論綱曰:「姚平
 仲已舉事,卿速援之。」綱率諸將旦出封丘門,與金人戰幕天坡,以神臂弓射金人,卻之。平仲竟以襲敵營不克,懼誅亡去。金使來,宰相李邦彥語之曰:「用兵乃李綱、姚平仲,非朝廷意。」遂罷綱,以蔡懋代之。太學生陳東等詣闕上書,明綱無罪。軍民不期而集者數十萬,呼聲動地,恚不得報,至殺傷內侍。帝亟召綱,綱入見,泣拜請死。帝亦泣,命綱復為尚書右丞,充京城四壁守禦使。



 始,金人犯城者,蔡懋禁不得輒施矢石,將士積憤,至是,綱下令
 能殺敵者厚賞,眾無不奮躍。金人懼,稍稍引卻,且得割三鎮詔及親王為質,乃退師。除綱知樞密院事。綱奏請如澶淵故事,遣兵護送,且戒諸將,可擊則擊之。乃以兵十萬分道並進,將士受命,踴躍以行。先是,金帥粘罕圍太原,守將折可求、劉光世軍皆敗;平陽府義兵亦叛,導金人入南北關,取隆德府,至是,遂攻高平。宰相咎綱盡遣城下兵追敵,恐倉卒無措,急徵諸將還。諸將已追及金人於刑、趙間,遽得還師之命,無不扼腕。比綱力爭,復追,而將
 士解體矣。



 詔議迎太上皇帝還京。初,徽宗南幸,童貫、高俅等以兵扈從。既行,聞都城受圍,乃止東南郵傳及勤王之師。道路籍籍,言貫等為變。陳東上書,乞誅蔡京、蔡攸、童貫、朱勉、高俅、盧宗原等。議遣聶山為發運使往圖之,綱曰:「使山所圖果成,震驚太上,此憂在陛下。萬一不果,是數人者,挾太上於東南,求劍南一道,陛下將何以處之?莫若罷山之行,請於太上去此數人,自可不勞而定。」上從其言。



 徽宗還次南都,以書問改革政事之故,且
 召吳敏、李綱。或慮太上意有不測,綱請行,曰:「此無他,不過欲知朝廷事爾。」綱至,具道皇帝聖孝思慕,欲以天下養之意,請陛下早還京師。徽宗泣數行下,問:「卿頃以何故去?」綱對曰:「臣昨任左史,以狂妄論列水災,蒙恩寬斧鉞之誅,然臣當時所言,以謂天地之變,各以類應,正為今日攻圍之兆。夫災異變故,譬猶一人之身,病在五臟,則發於氣色,形於脈息,善醫者能知之。所以聖人觀變於天地,而修其在我者,故能制治保邦,而無危亂之憂。」
 徽宗稱善。又詢近日都城攻圍守御次第,語漸浹洽。徽宗因及行宮止遞角等事,曰:「當時恐金人知行宮所在,非有他也。」綱奏:「方艱危時,兩宮隔絕,朝廷應副行宮,亦豈能無不至者,在聖度燭之耳。」且言:「皇帝仁孝,惟恐有一不當太上皇帝意者,每得詰問之詔,輒憂懼不食。臣竊譬之,家長出而強寇至,子弟之任家事者,不得不從宜措置。長者但當以其能保田園大計而慰勞之,茍誅及細故,則為子弟者,何所逃其責哉?皇帝傳位之初,陛
 下巡幸,適當大敵入攻,為宗社計,庶事不得不小有更革。陛上回鑾,臣謂宜有以大慰安皇帝之心,勿問細故可也。」徽宗感悟,出玉帶、金魚、象簡賜綱,曰:「行宮人得卿來皆喜,以此示朕意,卿可便服之。」且曰:「卿輔助皇帝、捍守宗社有大功,若能調和父子間,使無疑阻,當遂書青史,垂名萬世。」綱感泣再拜。



 綱還,具道太上意。宰執進迎奉太上儀注,耿南仲議欲屏太上左右,車駕乃進。綱言:「如此,是示之以疑也。天下之理,誠與疑、明與暗而已。自
 誠明而推之,可至於堯、舜;自疑暗而推之,其患有不可勝言者。耿南仲不以堯、舜之道輔陛下,乃暗而多疑。」南仲怫然曰:「臣適見左司諫陳公輔,乃為李綱結士民伏闕者,乞下御史置對。」上愕然。綱曰:「臣與南仲所論,國事也。南仲乃為此言,臣何敢復有所辨?願以公輔事下吏,臣得乞身待罪。」章十餘上,不允。



 太上皇帝還,綱迎拜國門。翌日,朝龍德宮,退,復上章懇辭。上手詔諭意曰:「乃者敵在近郊,士庶伏闕,一朝倉猝,眾數十萬,忠憤所激,不
 謀同辭,此豈人力也哉?不悅者造言,致卿不自安,朕深諒卿,不足介懷。巨敵方退,正賴卿協濟艱難,宜勉為朕留。」綱不得已就職。上備邊禦敵八事。



 時北兵已去,太上還宮,上下恬然,置邊事於不問。綱獨以為憂,與同知樞密院事許翰議調防秋之兵。吳敏乞置詳議司檢詳法制,以革弊政,詔以綱為提舉官,南仲沮止之。綱奏:「邊患方棘,調度不給,宜稍抑冒濫,以足國用。謂如節度使至遙郡刺史,本以待勛臣,今皆以戚裏恩澤得之;堂吏轉
 官止於正郎,崇、觀間始轉至中奉大夫,今宜皆復舊制。」執政揭其奏通衢,以綱得士民心,欲因此離之。會守禦司奏補副尉二人,御批有「大臣專權,浸不可長」語。綱奏:「頃得旨給空名告敕,以便宜行事。二人有勞當補官,故具奏聞,乃遵上旨,非專權也。」



 時太原圍未解,種師中戰沒,師道病歸,南仲曰:「欲援太原,非綱不可。」上以綱為河東、北宣撫使。綱言:「臣書生,實不知兵。在圍城中,不得已為陛下料理兵事,今使為大帥,恐誤國事。」因拜辭,不許。
 退而移疾,乞致仕,章十餘上,不允。臺諫言綱不可去朝廷,上以其為大臣游說,斥之。或謂綱曰:「公知所以遣行之意乎?此非為邊事,欲緣此以去公,則都人無辭耳。公堅臥不起,讒者益肆,上怒且不測,奈何?」許翰書:「杜郵」二字遺綱,綱皇恐受命。上手書《裴度傳》以賜,綱言:「吳元濟以區區環蔡之地抗唐室,與金人強弱固不相侔,而臣曾不足以望裴度萬分之一。然寇攘外患可以掃除,小人在朝,蠹害難去。使朝廷既正,君子道長,則所以捍禦
 外患者,有不難也。」因書裴度論元稹、魏洪簡章疏要語以進,上優詔答之。



 宣撫司兵僅萬二千人,庶事未集,綱乞展行期。御批以為遷延拒命,綱上疏明其所以未可行者,且曰:「陛下前以臣為專權,今以臣為拒命,方遣大帥解重圍,而以專權、拒命之人為之,無乃不可乎?願乞骸骨,解樞管之任。」上趣召數四,曰:「卿為朕巡邊,便可還朝。」綱曰:「臣之行,無復還之理。昔範仲淹以參政出撫西邊,過鄭州,見呂夷簡。夷簡曰:『參政豈可復還!』其後果然。
 今臣以愚直不容於朝,使既行之後,進而死敵,臣之願也。萬一朝廷執議不堅,臣當求去,陛下宜察臣孤忠,以全君臣之義。」上為之感動。及陛辭,言唐恪、聶山之奸,任之不已,後必誤國。



 進至河陽,望拜諸陵,復上奏曰:「臣總師出鞏、洛,望拜陵寢,潸然出涕。恭惟祖宗創業守成,垂二百年,以至陛下。適丁艱難之秋,強敵內侵,中國勢弱,此誠陛下嘗膽思報,厲精求治之日,願深考祖宗之法,一一推行之。進君子,退小人,益固邦本,以圖中興,上以慰
 安九廟之靈,下為億兆蒼生之所依賴,天下幸甚!」



 行次懷州,有詔罷減所起兵綱奏曰:「太原之圍未解,河東之勢甚危,秋高馬肥,敵必深入,宗社安危,殆未可知。使防秋之師果能足用,不可保無敵騎渡河之警。況臣出使未幾,朝廷盡改前詔,所團結之兵,悉罷減之。今河北、河東日告危急,未有一人一騎以副其求,甫集之兵又皆散遣,臣誠不足以任此。且以軍法勒諸路起兵,而以寸紙罷之,臣恐後時有所號召,無復應者矣。」疏上,不報。御
 批日促解太原之圍,而諸將承受御畫,事皆專達,宣撫司徒有節制之名。綱上疏,極諫節制不專之弊。



 時方議和,詔止綱進兵。未幾,徐處仁、吳敏罷相而相唐恪,許翰罷同知樞密院而進聶山、陳過庭、李回等,吳敏復謫置涪州。綱聞之,嘆曰:「事無可為者矣!」即上奏丐罷。乃命種師道以同知樞密院事領宣撫司事,召綱赴闕。尋除觀文殿學士、知揚州,綱具奏辭免。未幾,以綱專主戰議,喪師費財,落職提舉亳州明道宮,責授保靜軍節度副使,
 建昌軍安置;再謫寧江。



 金兵再至,上悟和議之非,除綱資政殿大學士,領開封府事。綱行次長沙,被命,即率湖南勤王之師入援,未至而都城失守。先是,康王至北軍,為金人所憚,求遣肅王代之。至是,康王開大元帥府,承制復綱故官,且貽書曰:「方今生民之命,急於倒垂,諒非不世之才,何以協濟事功。閣下學窮天人,忠貫金石,當投袂而起,以副蒼生之望。」



 高宗即位,拜尚書右僕射兼中書侍郎,趣赴闕。中丞顏岐奏曰:「張邦昌為金人所喜,
 雖已為三公、郡王,宜更加同平章事,增重其禮;李綱為金人所惡,雖已命相,宜及其未至罷之。」章五上,上曰:「如朕之立,恐亦非金人所喜。」岐語塞而退。岐猶遣人封其章示綱,覬以沮其來。上聞綱且至,遣官迎勞,錫宴,趣見於內殿。綱見上,涕泗交集,上為動容。因奏曰:「金人不道,專以詐謀取勝,中國不悟,一切墮其計中。賴天命未改,陛下總師於外,為天下臣民之所推戴,內修外攘,還二聖而撫萬邦,責在陛下與宰相。臣自視闕然,不足以副
 陛下委任之意,乞追寢成命。且臣在道,顏岐嘗封示論臣章,謂臣為金人所惡,不當為相。如臣愚蠢,但知有趙氏,不知有金人,宜為所惡。然謂臣材不足以任宰相則可,謂為金人所惡不當為相則不可。」因力辭。帝為出範宗尹知舒州。顏岐與祠。綱猶力辭,上曰:「朕知卿忠義智略久矣,欲使敵國畏服,四方安寧,非相卿不可,卿其勿辭。」綱頓首泣謝,云:



 臣愚陋無取,荷陛下知遇,然今日扶顛持危,圖中興之功,在陛下而不在臣。臣無左右先容,
 陛下首加識擢,付以宰柄,顧區區何足以仰副圖任責成之意?然「靡不有初,鮮克有終」。臣孤立寡與,望察管仲害霸之言,留神於君子小人之間,使得以盡志畢慮,雖死無憾。昔唐明皇欲相姚崇,崇以十事要說,皆中一時之病。今臣亦以十事仰乾天聽,陛下度其可行者,賜之施行,臣乃敢受命。



 一曰議國是。謂中國之御四裔,能守而後可戰,能戰而後可和,而靖康之末皆失之。今欲戰則不足,欲和則不可,莫若先自治,專以守為策,俟吾政
 事修,士氣振,然後可議大舉。



 二曰議巡幸。謂車駕不可不一到京師,見宗廟,以慰都人之心,度未可居,則為巡幸之計。以天下形勢而觀。長安為上,襄陽次之,建康又次之,皆當詔有司預為之備。



 三日議赦令。謂祖宗登極赦令,皆有例程。前日赦書,乃以張邦昌偽赦為法,如赦惡逆及罪廢官盡復官職,皆泛濫不可行,宜悉改正以法。



 四曰議僭逆。謂張邦昌為國大臣,不能臨難死節,而挾金人之勢易姓改號,宜正典刑,垂戒萬世。



 五曰議偽
 命。謂國家更大變,鮮仗節死義之士,而受偽官以屈膝於其庭者,不可勝數。昔肅宗平賊,污為偽者以六等定罪,宜仿之以勵士風。



 六曰議戰。謂軍政久廢,士氣怯惰,宜一新紀律,信賞必罰,以作其氣。



 七曰議守。謂敵情狡獪,勢必復來,宜於沿河、江、淮措置控御,以扼其沖。



 八曰議本政。謂政出多門,紀綱紊亂,宜一歸之於中書,則朝廷尊。



 九曰議久任。謂靖康間進退大臣太速,功效蔑著,宜慎擇而久任之,以責成功。



 十曰議修德。謂上始膺天
 命,宜益修孝悌恭儉,以副四海之望,而致中興。



 翌日,班綱議於朝,惟僭逆、偽命二事留中不出。綱言:



 二事乃今日政刑之大者。邦昌當道君朝,在政府者十年,淵聖即位,首擢為相。方國家禍難,金人為易姓之謀,邦昌如能以死守節,推明天下戴宋之義,以感動其心,敵人未必不悔禍而存趙氏。而邦昌方自以為得計,偃然正位號,處宮禁,擅降偽詔,以止四方勤王之師。及知天下之不與,不得已而後請元祐太后垂簾聽政,而議奉迎。邦昌
 僭逆始末如此,而議者不同,臣請備論而以《春秋》之法斷之。



 夫都城之人德邦昌,謂因其立而得生,且免重科金銀之擾。元帥府恕邦昌,謂其不待征討而遣使奉迎。若天下之憤嫉邦昌者,則謂其建號易姓,而奉迎特出於不得已。都城德之,元帥府恕之,私也,天下憤嫉之,公也。《春秋》之法,人臣無將,將而必誅;趙盾不討賊,則書以殺君。今邦昌已僭位號,敵退而止勤王之師,非特將與不討賊而已。



 劉盆子以漢宗室為赤眉所立,其後以十萬
 眾降光武,但待之以不死。邦昌以臣易君,罪大於盆子,不得已而自歸,朝廷既不正其罪,又尊崇之,此何理也?陛下欲建中興之業,而尊崇僭逆之臣,以示四方,其誰不解體?又偽命臣僚,一切置而不問,何以厲天下士大夫之節?



 時執政中有論不同者,上乃召黃潛善等語之。潛善主邦昌甚力,上顧呂好問曰:「卿昨在圍城中知其故,以為何如?」好問附潛善,持兩端,曰:「邦昌僭竊位號,人所共知,既已自歸,惟陛下裁處。」綱言:「邦昌僭逆,豈可使
 之在朝廷,使道路指目曰『此亦一天子』哉!」因泣拜曰:「臣不可與邦昌同列,當以笏擊之。陛下必欲用邦昌,第罷臣。」上頗感動。伯彥乃曰:「李綱氣直,臣等所不及。」乃詔邦昌謫潭州,吳幵、莫儔而下皆遷謫有差。綱又言:「近世士大夫寡廉鮮恥,不知君臣之義。靖康之禍,能仗節死義者,在內惟李若水,在外惟霍安國,願加贈恤。」上從其請,仍詔有死節者,諸路詢訪以聞。上謂綱曰:「卿昨爭張邦昌事,內侍輩皆泣涕,卿今可以受命矣。」綱拜謝。有旨兼
 充御營使。入對,奏曰:



 今國勢不逮靖康間遠甚,然而可為者,陛下英斷於上,群臣輯睦於下,庶幾靖康之弊革,而中興可圖。然非有規模而知先後緩急之序,則不能以成功。



 夫外禦強敵,內銷盜賊,修軍政,變士風,裕邦財,寬民力,改弊法,省冗官,誠號令以感人心,信賞罰以作士氣,擇帥臣以任方面,選監司、郡守以奉行新政,俟吾所以自治者政事已修,然後可以問罪金人,迎還二聖,此所謂規模也。至於所當急而先者,則在於料理河北、
 河東。蓋河北、河東者,國之屏蔽也。料理稍就,然後中原可保,而東南可安。今河東所失者忻、代、太原、澤、潞、汾、晉,餘郡猶存也。河北所失者,不過真定、懷、衛、浚四州而已,其餘三十餘郡,皆為朝廷守。兩路士民兵將,所以戴宋者,其心甚堅,皆推豪傑以為首領,多者數萬,少者亦不下萬人。朝廷不因此時置司、遣使以大慰撫之,分兵以援其危急,臣恐糧盡力疲,坐受金人之困。雖懷忠義之心,援兵不至,危迫無告,必且憤怨朝廷,金人因得撫而
 用之,皆精兵也。



 莫若於河北置招撫司,河東置經制司,擇有材略者為之使,宣論天子恩德、所以不忍棄兩河於敵國之意。有能全一州、復一郡者,以為節度、防禦、團練使,如唐方鎮之制,使自為守。非惟絕其從敵之心,又可資其禦敵之力,使朝廷永無北顧之憂,最今日之先務也。



 上善其言,問誰可任者,綱薦張所、傅亮。所嘗為監察御史,在靖康圍城中,以蠟書募河北兵,士民得書,喜曰:「朝廷棄我,猶有一張察院能拔而用之。」應募者凡十
 七萬人,由是所之聲震河北。故綱以為招撫河北,非所不可。傅亮者,先以邊功得官,嘗治兵河朔。都城受圍時,亮率勤王之兵三萬人,屢立戰功。綱察其智略可以大用,欲因此試之。上乃以所為河北招撫使,亮為河東經制副使。



 皇子生,故事當肆赦。綱奏:「陛下登極,曠蕩之恩獨遺河北、河東,而不及勤王之師,天下觖望。夫兩路為朝廷堅守,而赦令不及,人皆謂已棄之,何以慰忠臣義士之心?勤王之師在道路半年,擐甲荷戈,冒犯霜露,雖
 未效用,亦已勞矣。加以疾病死亡,恩恤不及,後有急難,何以使人乎?願因今赦廣示德意。」上嘉納。於是兩路知天子德意,人情翕然,間有以破敵捷書至者。金人圍守諸郡之兵,往往引去。而山砦之兵,應招撫、經制二司募者甚眾。



 有許高、許亢者,以防河而遁,謫嶺南,至南康謀變,守倅戮之。或議其擅殺,綱曰:「高、亢受任防河,寇未至而遁,沒途劫掠,甚於盜賊。朝廷不能正軍法,而一守倅能行之,真健吏也。使受命捍賊而欲退走者,知郡縣之
 吏皆得以誅之,其亦少知所戒乎!」上以為然,命轉一官。開封守闕,綱以留守非宗澤不可,力薦之。澤至,撫循軍民,修治樓櫓,屢出師以挫敵。



 綱立軍法,五人為伍,伍長以牌書同伍四人姓名。二十五人為甲,甲正以牌書伍長五人姓名。百人為隊,隊將以牌書甲正四人姓名。五百人為部,部將以牌書隊將正副十人姓名。二千五百人為軍,統制官以牌書部將正副十人姓名。命招置新軍及御營司兵,並依新法團結,有所呼召、使令,按牌以
 遣。三省、樞密院置賞功司,受賂乞取者行軍法,遇敵逃潰者斬,因而為盜賊者,誅及其家屬。凡軍政申明改更者數十條。



 又奏步不足以勝騎,騎不足以勝車,請以車制頒京東、西,制造而教閱之。又奏造戰艦,募水軍,及詢訪諸路武臣材略之可任者以備用。又進三疏:一曰募兵,二曰買馬,三曰募民出財以助兵費。諫議大夫宋齊愈聞而笑之,謂虞部員外郎張浚曰:「李丞相三議,無一可行者。」浚問之,齊愈曰:「民財不可盡括;西北之馬不可
 得,而東南之馬不可用;至於兵數,若郡增二千,則歲用千萬緡,費將安出?齊愈將極論之。」浚曰:「公受禍自此始矣。」



 時朝廷議遣使於金,綱奏曰:「堯、舜之道,孝悌而已,孝悌之至,可以通神明。陛下以二聖遠狩沙漠,食不甘味,寢不安席,思迎還兩宮,致天下養,此孝悌之至,而堯、舜之用心也。今日之事,正當枕戈嘗膽,內修外攘,使刑政修而中國強,則二帝不俟迎請而自歸。不然,雖冠蓋相望,卑辭厚禮,恐亦無益。今所遣使,但當奉表通問兩宮,
 致思慕之意可也。」上乃命綱草表,以周望、傅雱為二聖通問使,奉表以往。且乞降哀痛之詔,以感動天下,使同心協力,相與扶持,以致中興。又乞省冗員,節浮費。上皆從其言。是時,四方潰兵為盜者十餘萬人,攻劫山東、淮南、襄漢之間,綱命將悉討平之。



 一日,論靖康時事,上曰:「淵聖勤於政事,省覽章奏,至終夜不寐,然卒致播遷,何耶?」綱曰:「人主之職在知人,進君子而退小人,則大功可成,否則衡石程書,無益也。」因論靖康初朝廷應敵得失
 之策,且極論金人兩至都城,所以能守不能守之故;因勉上以明恕盡人言,以恭儉足國用,以英果斷大事。上皆嘉納。又奏:「臣嘗言車駕巡幸之所,關中為上,襄陽次之,建康為下。陛下縱未能行上策,猶當且適襄、鄧,示不忘故都,以系天下之心。不然,中原非復我有,車駕還闕無期,天下之勢遂傾不復振矣。」上為詔諭兩京以還都之意,讀者皆感泣。



 未幾,有詔欲幸東南避敵,綱極論其不可,言:「自古中興之主,起於西北,則足以據中原而有
 東南,起於東南,則不能以復中原而有西北。蓋天下精兵健馬皆在西北,一旦委中原而棄之,豈惟金人將乘間以擾內地;盜賊亦將蜂起為亂,跨州連邑,陛下雖欲還闕,不可得矣,況欲治兵勝敵以歸二聖哉?夫南陽光武之所興,有高山峻嶺可以控扼,有寬城平野可以屯兵;西鄰關、陜,可以召將士;東達江、淮,可以運穀粟;南通荊湖、巴蜀,可以取財貨;北距三都,可以遣救援。暫議駐蹕,乃還汴都,策無出於此者。今乘舟順流而適東南,固
 甚安便,第恐一失中原,則東南不能必其無事,雖欲退保一隅,不易得也。況嘗降詔許留中原,人心悅服,奈何詔墨未幹,遽失大信於天下!」上乃許幸南陽,而黃潛善、汪伯彥實陰上巡幸東南之議。客或有謂綱曰:「外論洶洶,咸謂東幸已決。」綱曰:「國之存亡,於是焉分,吾當以去就爭之。」初,綱每有所論諫,其言雖切直,無不容納,至是,所言常留中不報。已而遷綱尚書左僕射兼門下侍郎,黃潛善除右僕射兼中書侍郎。張所乞且置司北京,俟
 措置有緒,乃渡河。北京留守張益謙,潛善黨也,奏招撫司之擾,又言自置司河北,盜賊益熾。綱言:「所尚留京師,益謙何以知其擾?河北民無所歸,聚而為盜,豈由置司乃有盜賊乎?」



 有旨令留守宗澤節制傅亮,即日渡河。亮言:「措置未就而渡河,恐誤國事。」綱言:「招撫、經制,臣所建明,而張所、傅亮,又臣所薦用。今潛善、伯彥沮所及亮,所以沮臣。臣每覽靖康大臣不和之失,事未嘗不與潛善、伯彥議而後行,而二人設心如此,願陛下虛心觀之。」既
 而詔罷經制司,召亮赴行在。綱言:「聖意必欲罷亮,乞以御筆付潛善施行,臣得乞身歸田。」綱退,而亮竟罷,乃再疏求去。上曰:「卿所爭細事,胡乃爾?」綱言:「方今人材以將帥為急,恐非小事。臣昨議遷幸,與潛善、伯彥異,宜為所嫉。然臣東南人,豈不願陛下東下為安便哉?顧一去中原,後患有不可勝言者。願陛下以宗社為心,以生靈為意,以二聖未還為念,勿以臣去而改其議。臣雖去左右,不敢一日忘陛下。」泣辭而退。或曰:「公決於進退,於義得
 矣,如讒者何?」綱曰:「吾知盡事君之道,不可,則全進退之節,患禍非所恤也。



 初,二帝北行,金人議立異姓。吏部尚書王時雍問於吳幵、莫儔,二人微言敵意在張邦昌,時雍未以為然。適宋齊愈自敵所來,時雍又問之,齊愈取片紙書「張邦昌」三字,時雍意乃決,遂以邦昌姓名入議狀。至是,齊愈論綱三事之非,不報。擬章將再上,其鄉人嗛齊愈者,竊其草示綱。時方論僭逆附偽之罪,於是逮齊愈,齊愈不承,獄吏曰:「王尚書輩所坐不輕,然但遷嶺
 南,大諫第承,終不過逾嶺爾。」齊愈引伏,遂戮之東市。張浚為御史,劾綱以私意殺侍從,且論其買馬招軍之罪。詔罷綱為觀文殿大學士、提舉洞霄宮。尚書右丞許翰言綱忠義,合之無以佐中興。會上召見陳東,東言:「潛善、伯彥不可任,綱不可去。」東坐誅。翰曰:「吾與東皆爭李綱者,東戮都市,吾在廟堂,可乎?」遂求去。後有旨,綱落職居鄂州。



 自綱罷,張所以罪去,傅亮以母病辭歸,招撫、經制二司皆廢。車駕遂東幸,兩河郡縣相繼淪陷,凡綱所規
 畫軍民之政,一切廢罷。金人攻京東、西,殘毀關輔,而中原盜賊蜂起矣。



\end{pinyinscope}