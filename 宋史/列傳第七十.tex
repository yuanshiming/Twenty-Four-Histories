\article{列傳第七十}

\begin{pinyinscope}

 晏殊龐籍孫恭孫王隨章得像呂夷簡子公綽公弼公孺張士遜



 晏殊,字同叔,撫州臨川人。七歲能屬文,景德初,張知白安撫江南,以神童薦之。帝召殊與進士千餘人並試廷
 中,殊神氣不懾,援筆立成。帝嘉賞,賜同進士出身。宰相寇準曰:「殊江外人。」帝顧曰:「張九齡非江外人邪?」後二日,復試詩、賦、論,殊奏:「臣嘗私習此賦,請試他題。」帝愛其不欺,既成,數稱善。擢秘書省正字,秘閣讀書。命直史館陳彭年察其所與游處者,每稱許之。



 明年,召試中書,遷太常寺奉禮郎。東封恩,遷光祿寺丞,為集賢校理。喪父,歸臨川,奪服起之,從祀太清宮。詔修寶訓,同判太常禮院。喪母,求終服,不許。再遷太常寺丞,擢左正言、直史館,為
 升王府記室參軍。歲中,遷尚書戶部員外郎,為太子舍人,尋知制誥,判集賢院。久之,為翰林學士,遷左庶子。帝每訪殊以事,率用方寸小紙細書,已答奏,輒並稿封上,帝重其慎密。



 仁宗即位,章獻明肅太后奉遺詔權聽政。宰相丁謂、樞密使曹利用,各欲獨見奏事,無敢決其議者。殊建言:「群臣奏事太后者,垂簾聽之,皆毋得見。」議遂定。遷右諫議大夫兼侍讀學士,太后謂東宮舊臣,恩不稱,加給事中。預修《真宗實錄》。進禮部侍郎,拜樞密副使。
 上疏論張耆不可為樞密使,忤太后旨。坐從幸玉清昭應宮從者持笏後至,殊怒,以笏撞之折齒,御史彈奏,罷知宣州。數月,改應天府,延范仲淹以教生徒。自五代以來,天下學校廢,興學自殊始。召拜御史中丞,改資政殿學士、兼翰林侍讀學士,兵部侍郎、兼秘書監,為三司使,復為樞密副使,未拜,改參知政事,加尚書左丞。太后謁太廟,有請服袞冕者,太后以問,殊以《周官》後服對。太后崩,以禮部尚書罷知亳州,徙陳州,遷刑部尚書,以本官
 兼御史中丞,復為三司使。



 陜西方用兵,殊請罷內臣監兵,不以陣圖授諸將,使得應敵為攻守;及募弓箭手教之,以備戰鬥。又請出宮中長物助邊費,凡他司之領財利者,悉罷還度支。悉為施行。康定初,知樞密院事,遂為樞密使。進同中書門下平章事。慶歷中,拜集賢殿學士、同平章事,兼樞密使。



 殊平居好賢,當世知名之士,如范仲淹、孔道輔皆出其門。及為相,益務進賢材,而仲淹與韓琦、富弼皆進用,至於臺閣,多一時之賢。帝亦奮然有
 意,欲因群材以更治,而小人權幸皆不便。殊出歐陽修為河北都轉運,諫官奏留,不許。孫甫、蔡襄上言:「宸妃生聖躬為天下主,而殊嘗被詔志宸妃墓,沒而不言。」又奏論殊役官兵治僦舍以規利。坐是,降工部尚書、知穎州。然殊以章獻太后方臨朝,故志不敢斥言;而所役兵,乃輔臣例宣借者,時以謂非殊罪。



 徙陳州,又徙許州,稍復禮部、刑部尚書。祀明堂,遷戶部,以觀文殿大學士知永興軍,徙河南府,遷兵部。以疾,請歸京師訪醫藥。既平,復
 求出守,特留侍經筵,詔五日一與起居,儀從如宰相。逾年,病浸劇,乘輿將往視之。殊即馳奏曰:「臣老疾,行愈矣,不足為陛下憂也。」已而薨。帝雖臨奠,以不視疾為恨,特罷朝二日,贈司空兼侍中,謚元獻,篆其碑首曰「舊學之碑」。



 殊性剛簡,奉養清儉。累典州,吏民頗畏其悁急。善知人,富弼、楊察,皆其婿也。殊為宰相兼樞密使,而弼為副使,辭所兼,詔不許,其信遇如此。文章贍麗,應用不窮,尤工詩,閑雅有情思,晚歲篤學不倦。文集二百四十卷,及
 刪次梁、陳以後名臣述作,為《集選》一百卷。



 子知止,為朝請大夫。



 龐籍,字醇之,單州成武人。及進士第,為黃州司理參軍,知州夏竦以為有宰相器。調開封府兵曹參軍,知府薛奎薦為法曹。遷大理寺丞、知襄邑縣。



 預修《天聖編敕》,為刑部詳覆官。擢群牧判官,因轉封言:「舊制不以國馬假臣下,重武備也。樞密院以帶甲馬借內侍楊懷敏,群牧覆奏,乃賜一馬,三日,乃復借之,數日而復罷。樞密掌機
 命,反復乃如此。平時,百官奏事上前,不自批章,止送中書、樞密院。近歲璽書內降,浸多於舊,無以防偏請、杜幸門矣。往者,王世融以公主子毆府吏,法當贖金,特停任。近作坊料物庫主吏盜官物,輒自逃避。以宮掖之親,三司遽罷追究。今日聖斷乃異於昔,臣竊惑焉。祥符令檢下稍嚴,胥吏相率空縣而去,令坐罷免。若是,則清強者沮矣。」



 久之,出知秀州,召為殿中侍御史,章獻太后遺誥:章惠太后議軍國事;籍請下閣門,取垂簾儀制盡燔之。
 又奏:「陛下躬親萬機,用人宜辨邪正、防朋黨,擢進近列,願採公論,毋令出於執政。」孔道輔謂人曰:「言事官多觀望宰相意,獨龐醇之,天子御史也。」為開封府判官,尚美人遣內侍稱教旨免工人市租。籍言:「祖宗以來,未有美人稱教旨下府者,當杖內侍。」詔有司:「自今宮中傳命,毋得輒受。」數劾範諷罪,諷善李迪,皆寢不報,反坐言宮禁事不得實,以祠部員外郎罷為廣南東路轉運使。又言範諷事有不盡如奏,諷坐貶,籍亦降太常博士、知臨江
 軍。尋復官,徙福建轉運使。



 景祐三年,為侍御史,改刑部員外郎、知雜事,判大理寺,進天章閣待制。元昊反,為陜西體量安撫使。坐令開封府吏馮士元市女口,降知汝州。徒同州,就除陜西都轉運使。文彥博鞫黃德和獄,未上,詔籍同案。籍言曰:「德和退怯當誅。劉平力戰而沒,宜加恤其子孫。」又建言:「頻歲災異,天久不雨。宮中費用奢靡,出納不嚴,須索煩多,有司無從鉤校虛實。臣竊謂凡乘輿所費,宮中所用,宜務加裁抑,取則先帝,修德弭災
 之道也。今宿兵西鄙,將士力戰,弗獲功賞;而內官、醫官、樂官,無功勞,享豐賜,天下指目,謂之『三官』。願少裁損,無厚賚予,專勵戰功,寇不足平也。」



 進龍圖閣直學士、知延州,俄兼鄜延都總管、經略安撫緣邊招討使。明年,改延州觀察使,力辭,換左諫議大夫。自元昊陷金明、承平、塞門、安遠、栲栳砦,破五龍川,邊民焚掠殆盡,籍至,稍葺治之。戍兵十萬無壁壘,皆散處城中,畏籍,莫敢犯法。金明西北有渾州川,土沃衍。川尾曰橋子穀,寇出入之隘道。
 使部將狄青將萬餘人,築招安砦於穀旁,數募民耕種,收粟以贍軍。周美襲取承平砦,王信築龍安砦,悉復所亡地,築十一城。及開□□名、平戎道,通永和、烏仁關,更東西陣法為方陣,頗損益兵械。元昊遣李文貴繼野利旺榮書來送款,籍曰:「此詐也。」乃屯兵青澗城。後數月,果大寇定川,籍召文貴開諭之,遣去。既而元昊又以旺榮書來,會帝厭兵,因招懷之,遣籍報書,使呼旺榮為太尉。籍曰:「太尉三公,非陪臣所得稱,使旺榮當之,則元昊不得
 臣矣。今其書自稱『寧令』或『謨寧令』,皆其官名也,於義無嫌。」朝廷從之。



 會敵新破涇原城砦,方議修復。使者往返,逾年,又遣賀從勖來,改名曰曩霄,稱男不稱臣。籍不敢聞,從勖曰:「子事父,猶臣事君也。若得至京師,天子不許,更歸議之。」籍送使者闕下,因陳便宜,言:「羌久不通和市,國人愁怨。今辭理浸順,必有改事中國之心,請遣使者申諭之。」朝廷採用其策。元昊既臣,召籍為樞密副使。籍言:「自陜西用兵,公私俱困,請並省官屬,退近塞之兵就
 食內地。」從之,於是頗省邊費。改參知政事,拜工部侍郎、樞密使,遷戶部,拜同中書門下平章事、昭文館大學士、監修國史。籍初入相,且獨員,而遽為昭文館大學士,出殊拜也。



 儂智高反,師數不利,遣狄青為宣撫使。諫官韓絳謂武人不宜專任,帝以問籍。籍曰:「青起行伍,若以文臣副之,則號令不專,不如不遣。」詔嶺南諸軍,皆受青節度。既而捷書至,帝喜曰:「青破賊,卿之力也。」遂欲以青為樞密使、同平章事,籍力爭之,不聽。嶺南平,二廣舉人推
 恩者六百九十一人,論者以為過。



 頃之,齊州學究皇甫淵以捕賊功,法當賞錢,數上書求用。道士趙清貺與籍姊家親,紿為淵白籍,乃與堂吏共受淵賂。小吏訴之,下開封府,捕清貺,刺配遠州,道死。韓絳言籍陰諷府杖殺清貺以滅口,覆之無狀。言不已,乃罷知鄆州。居數月,加觀文殿大學士。拜昭德軍節度使、知永興軍,改並州。



 仁宗不豫,籍嘗密疏,請擇宗室之賢者為皇子,其言甚切。坐擅聽麟州築堡白草平,而州將武戡等為夏人所敗,
 復為觀文殿大學士、戶部侍郎、知青州。遷尚書左丞,不拜。徙定州,召還京師,上章告老,尋以太子太保致仕,封穎國公。薨,年七十六。時仁宗不豫,廢朝、臨奠皆不果,第遣使吊賻其家。贈司空,加侍中,謚莊敏。



 籍曉律令,長於吏事。持法深峭,軍中有犯,或斷斬刳磔,或累笞至死,以故士卒畏服。治民頗有惠愛,及為相,聲望減於治郡時。子元英,朝散大夫。孫恭孫。



 恭孫字德孺,以蔭,補通判施州。崇寧中,部蠻向文強叛,
 詔轉運使王蘧領州事致討,恭孫說降文強而斬之。蘧上其功,進三秩,知涪州,遂以開邊為己任。誘珍州駱文貴、承州駱世華納土,費不貲。轉運判官朱師古劾恭孫生事,詔黜師古而以恭孫代,於是溱、播、溪、思、費等州相繼降。每開一城,輒褒遷,五年間,至徽猷閣待制。威州守乞通保、霸二州,進恭孫直學士、知成都府,委以招納。未幾,其酋董舜咨、董彥博來納土,詔遣赴闕,皆拜承宣使,賜第京師,更名保州祺州、霸州亨州,使恭孫進築之。言
 者論其貪縱,究治如章,謫保靜軍節度副使。才逾月,起知陳州,復待制,帥瀘州。又以築思州,進學士。前後在西南二十年,所得州縣,多張名簿,實瘠鹵不毛地,繕治轉餉,為蜀人病,無幾時皆廢。宣和中,卒。



 王隨,字子正,河南人。登進士甲科,為將作監丞、通判同州,遷秘書省著作郎、直史館、判三司磨勘司。為京西轉運副使,陛辭,且言曰:「臣父母家洛中,乃在所部,得奉湯藥,聖主之澤也。」真宗因賜詩寵行,以羊酒束帛令過家
 為壽。遷淮南轉運使,父憂,起復。時歲比饑,隨敕屬部出庫錢,貨民市種糧,歲中約輸絹以償,流庸多復業。徙河東轉運使,三遷刑部員外郎兼侍御史知雜事。擢知制誥,以不善制辭,出知應天府。一日,帝謂宰相曰:「隨治南京太寬。」王旦曰:「南京,都會之地,隨臨事汗漫,無以彈壓」。改知揚州。再加右諫議大夫、權知開封府。



 仁宗為太子,拜右庶子,仍領府事。周懷政誅,隨自陳嘗假懷政白金五十兩,奪知制誥,改給事中、知杭州。乾興初,復降秘書
 少監,徙通州。以州少學者,徙孔子廟,起學舍,州人喜,遣子弟就學。母喪,起復光祿卿、知潤州,徙江寧府。歲大饑,轉運使移府發常平倉米,計口日給一升,隨置不聽,曰:「民所以饑者,由兼並閉糴,以邀高價也。」乃大出官粟,平其價。



 復給事中,為龍圖閣直學士、知秦州。秦卒有負罪逃入蕃部者,戎人輒奴畜之,小不如意,復執出求賞,前此坐法多死。隨下教能自歸者免死,聽復隸軍籍,由是多來歸者。又建請增蕃落卒,給廢陷馬地,募民耕種。坐
 事,徙河南府。入為御史中丞,同知禮部貢舉,遷尚書禮部侍郎、翰林侍讀學士。



 明道中,為江淮安撫使,還拜戶部侍郎、參知政事,請與同列日獻前代名臣規諫一事。議者謂非輔弼之職,其事遂寢。加吏部侍郎、知樞密院事,為莊惠皇太后園陵監護使,拜門下侍郎、同中書門下平章事、昭文館大學士、監修國史。自薛居正後,故事,初相無越遷門下侍郎者,學士丁度之失也。



 頃之,以疾在告,詔五日一朝,入中書視事。為相一年,無所建明。與
 陳堯佐、韓億、石中立同執政,數爭事。會災異屢發,諫官韓琦言之,四人俱罷。隨以彰信軍節度使、同中書門下平章事判河陽。薨,贈中書令,謚章惠,後改文惠。



 隨外若方嚴,而治失於寬。晚更卞急,輒嫚罵人。性喜佛,慕裴休之為人,然風跡弗逮也。



 章得像,字希言,世居泉州。高祖仔鈞,事閩為建州刺史,遂家浦城。得像母方娠,夢登山,遇神人授以玉象,及生,父奐復夢家庭積笏如山。長而好學,美姿表,為人莊重。
 進士及第,為大理評事、知玉山縣,遷本寺丞。



 真宗將東封泰山,以殿中丞簽書兗州觀察判官事,知臺州,歷南雄州,徙洪州。楊億以為有公輔器,薦之。或問之,億曰:「閩士輕狹,而章公深厚有容,此其貴也。」得像嘗與億戲博李宗諤家,一夕負錢三十萬,而酣寢自如。他日博勝,得宗諤金一奩;數日博又負,即反奩與宗諤,封識未嘗發也。其度量宏廓如此。



 未幾,召試,為直史館、安撫京東,權三司度支判官,累遷尚書刑部郎中,使契丹,遂以兵部
 郎中知制誥。逾年,為翰林學士,遷右諫議大夫,以給事中為群牧使,遷禮部侍郎兼龍圖閣學士,進承旨兼侍講學士,擢同知樞密院事,遷戶部侍郎,遂拜同中書門下平章事、集賢殿大學士。帝謂得像曰:「向者太后臨朝,群臣邪正,朕皆默識之。卿清忠無所附,且未嘗有所干請,今日用卿,職此也。」



 陜西用兵,加中書侍郎兼工部尚書兼樞密使,辭所加官。明年,以工部尚書為昭文館大學士。慶歷五年,拜鎮安軍節度使、同平章事,封郇國公,
 徙判河南府,守司空致仕,薨。故事,致仕官乘輿不臨奠,帝特往焉。贈太尉兼侍中,謚文憲。皇祐中,改謚文簡。



 得像在翰林十二年,章獻太后臨朝,宦官方熾,太后每遣內侍至學士院,得像必正色待之,或不交一言。在中書凡八年,宗黨親戚,一切抑而不進。仁宗銳意天下事,進用韓琦、範仲淹、富弼,使同得像經畫當世急務,得像無所建明,御史孫抗數言之,得像居位自若。既而章十上請罷,帝不得已,許之。初,閩人謠曰:「南臺江合出宰相。」至
 得像相時,沙湧可涉雲。



 論曰:殊、籍、隨、得像皆起孤生,致位宰相。籍通曉法令,隨練習民事,皆能用其所長。然籍終至絀免,隨數遭譴斥,何其才之難得也。得像渾厚有容,殊喜薦拔人物,樂善不倦,方之諸人,殊其最優乎!



 呂夷簡,字坦夫,先世萊州人。祖龜祥知壽州,子孫遂為壽州人。夷簡進士及第,補絳州軍事推官,稍遷大理寺丞。祥符中,試材識兼茂明於體用科,或言六科所以求
 闕政,今封禪告成,何闕政之求,罷之。通判通州,徙濠州,再遷太常博士。



 河北水,選知濱州。代還奏:「農器有算,非所以勸力本也。」遂詔天下農器皆勿算。擢提點兩浙刑獄,遷尚書祠部員外郎。時京師大建宮觀,伐材木於南方。有司責期會,工徒至有死者,誣以亡命,收系妻子。夷簡請緩其役,從之。又言:「盛冬挽運艱苦,須河流漸通,以卒番送。」真宗曰:「觀卿奏,有為國愛民之心矣。」擢刑部員外郎兼侍御史知雜事。



 蜀賊李順叛,執送闕下,左右稱
 賀。既而屬御史臺按之,非是,賀者趣具順獄,夷簡曰:「是可欺朝廷邪?」卒以實奏,忤大臣意。歲蝗旱,夷簡請責躬修政,嚴飭輔相,思所以共順天意;及奏彈李溥專利罔上。寇準判永興,黥有罪者徙湖南,道由京師,上準變事。夷簡曰:「準治下急,是欲中傷準爾,宜勿問,益徙之遠方。」從之。趙安仁為御史中丞,夷簡以親嫌,改起居舍人、同勾當通進司兼銀臺封駁事。使契丹,還,知制誥。兩川饑,為安撫使,進龍圖閣直學士,再遷刑部郎中、權知開封
 府。治嚴辦有聲,帝識姓名於屏風,將大用之。



 仁宗即位,進右諫議大夫。雷允恭擅徙永定陵地,夷簡與魯宗道驗治,允恭誅,以給事中參知政事,因請以祥符天書內之方中。真宗祔廟,太后欲具平生服玩如宮中,以銀罩覆神主。夷簡言:「此未足以報先帝。今天下之政在兩宮,惟太后遠奸邪,獎忠直,輔成聖德,所以報先帝者,宜莫若此也。」故事,郊祠畢,輔臣遷官,夷簡與同列皆辭之,後為例。遷尚書禮部侍郎、修國史,進戶部,拜同中書門下
 平章事、集賢殿大學士、景靈宮使。玉清昭應宮災,太后泣謂大臣曰:「先帝尊道奉天而為此,今何以稱遺旨哉。」夷簡意其將復營構也,乃推《洪範》災異以諫,太后默然。因奏罷二府兼宮觀使。進吏部,拜昭文館大學士、監修國史,史成,辭進官。



 天聖末,加中書侍郎。章懿太后為順容,薨,宮中未治喪,夷簡朝奏事,因曰:「聞有宮嬪亡者。」太后矍然曰:「宰相亦預宮中事邪?」引帝偕起。有頃獨出,曰:「卿何間我母子也?」夷簡曰:「太後他日不欲全劉氏乎?」太
 后意稍解。有司希太后旨,言歲月葬未利。夷簡請發哀成服,備儀仗葬之。



 大內火,百官晨朝,而宮門不開。輔臣請對,帝御拱辰門,百官拜樓下,夷簡獨不拜。帝使人問其故,曰:「宮庭有變,群臣願一望清光。」帝舉簾見之,乃拜。詔以為修大內使。內成,進尚書右僕射兼門下侍郎,辭僕射,乃兼吏部尚書。



 初,荊王子養禁中,既長,夷簡請出之。太后欲留使從帝誦讀,夷簡曰:「上富春秋,所親非儒學之臣,恐無益聖德。」即日命還邸中。太后崩,帝始親政
 事,夷簡手疏陳八事,曰:正朝綱,塞邪徑,禁貨賂,辨佞壬,絕女謁,疏近習,罷力役,節冗費。其勸帝語甚切。



 帝始與夷簡謀,以張耆、夏竦皆太后所任用者也,悉罷之,退告郭皇后。後曰:「夷簡獨不附太后邪?但多機巧、善應變耳。」由是夷簡亦罷為武勝軍節度使、檢校太傅、同中書門下平章事、判陳州。及宣制,夷簡方押班,聞唱名,大駭,不知其故。而夷簡素厚內侍副都知閻文應,因使為中詗,久之,乃知事由皇后也。歲中而夷簡復相。初,劉渙上疏
 請太后還政,太后怒,使投嶺外,屬太后疾革,夷簡請留之。至是,渙以前疏自言,帝擢渙右正言,顧謂夷簡:「向者樞密院亟欲投渙,賴卿以免。」夷簡謝,因曰:「渙由疏外故敢言,大臣或及此,則太后必疑風旨自陛下,使子母不相安矣。」帝以夷簡為忠。郭后以怒尚美人,批其頰,誤傷帝頸。帝以爪痕示執政大臣,夷簡以前罷相故,遂主廢後議。仁宗疑之,夷簡曰:「光武,漢之明主也,郭后止以怨懟坐廢,況傷陛下頸乎?」夷簡將廢後,先敕有司,無得受
 臺諫章奏。於是御史中丞孔道輔、右司諫範仲淹率臺諫詣閣門請對,有旨令臺諫詣中書,夷簡乃貶出道輔等,後遂廢。宗室子益眾,為置大宗正糾率,增教授員。加右僕射,封申國公。



 王曾與夷簡數爭事,不平,曾斥夷簡納賂市恩。夷簡乞置對,帝問曾,曾語屈,於是二人皆罷。夷簡以鎮安軍節度使、同平章事判許州,徙天雄軍。未幾,以右僕射復入相,逾年,進位司空,辭不拜,徙許國公。時方飭兵備,以判樞密院事,而諫官田況言總判名太
 重,改兼樞密使。



 契丹聚兵幽薊,聲言將入寇,議者請城洛陽。夷簡謂:「契丹畏壯侮怯,遽城洛陽,亡以示威,景德之役,非乘輿濟河,則契丹未易服也。宜建都大名,示將親征以伐其謀。」或曰:「此虛聲爾,不若修洛陽。」夷簡曰:「此子囊城郢計也。使契丹得渡河,雖高城深池,何可恃耶?」乃建北京。



 未幾,感風眩,詔拜司空、平章軍國重事,疾稍間,命數日一至中書,裁決可否。夷簡力辭,復降手詔曰:「古謂髭可療疾,今翦以賜卿。」三年春,帝御延和殿召見,
 敕乘馬至殿門,命內侍取兀子輿以前。夷簡引避久之,詔給扶毋拜。乃授司徒、監修國史,軍國大事與中書、樞密同議。固請老,以太尉致仕,朝朔望。既薨,帝見群臣,涕下,曰:「安得憂國忘身如夷簡者!」贈太師、中書令,謚文靖。



 自仁宗初立,太后臨朝十餘年,天下晏然,夷簡之力為多。其後元昊反,四方久不用兵,師出數敗;契丹乘之,遣使求關南地。頗賴夷簡計畫,選一時名臣報使契丹、經略西夏,二邊以寧。然建募萬勝軍,雜市井小人,浮脆不
 任戰鬥。用宗室補環衛官,驟增奉賜,又加遺契丹歲繒金二十萬,當時不深計之,其後費大而不可止。郭后廢,孔道輔等伏閣進諫,而夷簡謂伏閣非太平事,且逐道輔。其後範仲淹屢言事,獻《百官圖》論遷除之敝,夷簡指為狂肆,斥於外。時論以此少之。



 夷簡當國柄最久,雖數為言者所詆,帝眷倚不衰。然所斥士,旋復收用,亦不終廢。其於天下事,屈伸舒卷,動有操術。後配食仁宗廟,為世名相。始,王旦奇夷簡,謂王曾曰:「君其善交之。」卒與曾
 並相。後曾家請御篆墓碑,帝因慘然思夷簡,書「懷忠之碑」四字以賜之。有集二十卷。



 子公綽、公弼、公著、公孺。公著自有傳。



 公綽字仲裕,蔭補將作監丞、知陳留縣。天聖中,為館閣對讀。召試,直集賢院,辭,改校理,遷太子中允。夷簡罷相,復為直集賢院、同管勾國子監,出知鄭州。嘗問民疾苦,父老曰:「官籍民產,第賦役重輕,至不敢多畜牛,田疇久蕪穢。」公綽為奏之,自是牛不入籍。還判吏部南曹,累遷
 太常博士、同判太常寺。請復太醫局,及請設令、丞、府史如天官醫師。鈞容直假太常旌纛、羽鑰,為優人戲,公綽執不可,遂罷之。糾察在京刑獄。虎翼卒劉慶告變,下吏案驗,乃慶始謀,眾不從,慶反誣眾以邀賞。因言:「京師衛兵百萬,不痛懲之,則眾心搖。」遂斬慶以徇。遷尚書工部員外郎,為史館修撰。



 時夷簡雖謝事,猶領國史,公綽辭修撰。夷簡薨,還兵部員外郎,復為修撰。服除,復同判太常寺兼提舉修祭器。公綽以郊廟祭器未完,制度多違
 禮,請悉更造。故事,薦新諸物,禮官議定乃薦,或後時陳敗。公綽採《月令》諸書,以四時新物及所當薦者,配合為圖。又以歲大、中、小祠凡六十一,禘祫二,祼獻興俯,玉帛尊彞,菁茆醓醢,鐘石歌奏,集為《郊祀總儀》上之。又言:「古者,天地、宗廟、日月、五方、百神之祀,咸有尊罍,五齊三酒,分實其中,加明水、明酒,以達陰陽之氣。今有司徒設尊罍,而酌用一尊,非禮神之意。宜按《周禮》實齊酒,取火於日,取水於月,因天地之潔氣。」又言:「祖宗配郊,當正位,今
 側鄉之,非所以示尊嚴也。」初,謚諸後,皆系祖宗謚,而真宗五後獨曰「莊」。公綽曰:「婦人從夫之謚,真宗謚章聖,而後曰『莊』,非禮也,願更為『章』。」多施行之。



 歷知制誥、龍圖閣直學士、集賢殿修撰、知永興軍,改樞密直學士、知秦州。安遠砦、古渭州諸羌來獻地,公綽顧其屬曰:「天下之大,豈利區落尺寸地以為廣邪?」卻之。弓箭手馬多闕,公綽諭諸砦戶為三等,凡十丁為社,至秋成,募出金帛市馬,馬少,則先後給之。祀明堂,遷刑部郎中,召為龍圖閣學
 士、權知開封府。歲餘,願罷府事,進翰林侍讀學士、知審刑院兼判太常寺。



 初,公綽在開封府,宰相龐籍外屬道士趙清貺受賂,杖脊道死。至是,御史以為公綽受籍旨,杖殺清貺以滅口,左遷龍圖閣學士、知徐州。方杖清貺時,實非公綽所臨。頃之,公綽亦自辨,復侍讀學士,徙河陽,留侍經筵。時久不雨,帝顧問:「何以致雨?」曰:「獄久不決,即有冤者,故多旱。」帝親慮囚,已而大雨。遷右司郎中,未拜,卒。贈左諫議大夫。



 公綽通敏有才,父執政時,多涉干
 請,喜名好進者趨之。嘗漏洩除拜以市恩,時人比之竇申。



 公弼字寶臣。賜進士出身,積遷直史館、河北轉運使。自寶元、慶歷以來,宿師備邊。既西北撤警,而將屯如故,民疲饋餉。公弼始通御河,漕粟實塞下,冶鐵以助經費;移近邊屯兵就食京東;增城卒,給板築;蠲冗賦及民逋數百萬。夷簡之亡也,仁宗思之,問知公弼名,識於殿柱。至是,益材其為。擢都轉運使,加龍圖閣直學士、知瀛州,入
 權開封府。嘗奏事退,帝目送之,謂宰相曰:「公弼甚似其父。」



 改同群牧使,以樞密直學士知渭、延二州,徙成都府。其治尚寬,人疑少威斷。營卒犯法當杖,捍不受,曰:「寧以劍死。」公弼曰:「杖者國法,劍汝自請。」杖而後斬之,軍府肅然。英宗罷三司使蔡襄,召公弼代之。初,公弼在群牧時,帝居藩,得賜馬頗劣,欲易不可。至是,帝謂曰:「卿曩歲不與朕馬,是時固已知卿矣。蔡襄主計,訴訟不時決,故多留事。卿繼其後,將何以處之?」公弼頓首謝,對曰:「襄勤於
 事,未嘗有曠失,恐言之者妄耳。」帝以為長者。拜樞密副使。時言事者數與大臣異議去,公弼諫曰:「諫官、御史,為陛下耳目,執政為股肱。股肱耳目,必相為用,然後身安而元首尊。宜考言觀事,視其所以而進退之。」彗出營室,帝憂之,同列請飭邊備。公弼曰:「彗非小變,陛下宜側身修德,以應天戒,臣恐患不在邊也。」



 神宗立,司馬光劾內侍高居簡,帝未決。公弼曰:「光與居簡,勢不兩立。居簡,內臣耳,而光中執法,願陛下擇其重者。」帝曰:「然則當奈何?」
 公弼曰:「遷居簡一官,而解其近職,光當無爭。」從之進。樞密使。議者欲並環慶、鄜延為一路,公弼曰:「自白草西抵定遠,中間相去千里,若合為一路,猝有緩急,將何以應?」又欲下邊臣使議之,公弼曰:「廟堂之上不處決,而諉邊吏,可乎?」乃止。



 王安石知政事。嗛公弼不附己,白用其弟公著為御史中丞以逼之。公弼不自安,立上章避位,不許。陳升之建議,衛兵年四十以上,稍不中程者,減其牢廩,徙之淮南。公弼以為非人情,帝曰:「是當退為剩員者,今
 故為優假,何所害?」對曰:「臣不敢生事邀名,正恐誤國耳。既使去本土,又削其廩,儻二十萬眾皆反側,為之奈何?」韓絳議復肉刑,公弼力陳不可,帝皆為之止。



 安石立新法,公弼數言宜務安靜,又將疏論之。從孫嘉問竊其稿示安石,安石先白之,帝不樂,遂罷為觀文殿學士、知太原府。韓絳宣撫秦、晉,將取囉兀城,令河東發兵二萬,趣神堂新路。公弼曰:「虜必設伏以待我。永和關雖回遠,可安行無患。」乃由永和。既而新路援兵果遇伏,詔褒之。麟
 州無井,唯沙泉在城外,欲拓城包之,而土善陷,夏人每至圍城,人皆憂渴死。公弼用其僚鄧子喬計,仿古拔軸法,去其沙,實以末炭,墳土於其上,板築立,遂包泉於中。自是城堅不陷,而州得以守。



 俄以疾,請知鄭州。王韶取熙河,朝廷謀秦鳳帥,帝曰:「公弼在河東,方出師倉卒時,有緩御之能,宜使往。」乃拜宣徽西院使、判秦州。帝疑其不肯行,公弼聞命即治裝,帝喜,召之入對,慰勞而遣之。既赴鎮,羌董氈輒治書稱敕,公弼卻之,曰:「藩臣安得妄
 稱敕?」董氈懼,自是不復敢。才旬月,復以疾求解,為西太一宮使。薨,年六十七。贈太尉,謚曰惠穆。



 公孺字稚卿。任為奉禮郎,賜進士出身,判吏部南曹。占對詳敏,仁宗以為可用。知澤、穎、廬、常四州,提點福建、河北路刑獄,入為開封府推官。民鬻薪為盜所奪,逐之遭傷,尹包拯命笞盜。公孺曰:「盜而傷主,法不止笞。」執不從,拯善其守。及使三司,而公孺為判官,事皆咨決之。判都水監,未幾,改陜西轉運使。



 神宗得綏州,遣使議守棄之
 便,久未決。命公孺往,與郭逵議合,遂存綏州。常平法行,公孺請以青苗、免役歸提刑司。徙知渭州,再徙鄆州。坐失人死刑,責知蔡州。



 元豐初,帝召公孺,慰之曰:「長安謀帥,無以易卿。」命知永興軍。徙河陽,洛口兵千人,以久役思歸,奮斧鍤排關,不得入,西走河橋,觀聽洶洶。諸將請出兵掩擊,公孺曰:「此皆亡命,急之,變且生。」即乘馬東去,遣牙兵數人迎諭之曰:「汝輩誠勞苦,然豈得擅還?一度橋,則罪不赦矣!太守在此,願自首者止道左。」皆佇立以
 俟。公孺索倡首者,黥一人,餘復送役所。語其校曰:「若復偃蹇者,斬而後報。」眾帖息。乃自劾專命,詔釋之。



 知審官東院,出知秦州。李憲以詔出兵,欲盡駐原、渭,公孺不可,與憲相論奏,坐徙相州,更陳、杭、鄭、瀛四州。元祐初,加龍圖閣直學士,復以為秦州,固辭,改秘書監。遷刑部侍郎、知開封府,為政明恕。幕人遷黼坐設,毀其角,法當徒,公孺請罪,數十人皆以杖免。原廟亡珠,系治典吏久,公孺曰:「主者番代不一,曷嘗以珠數相授受,歲時諱日,宮嬪
 狎至,奈何顓指吏卒乎?」請之,得釋。擢戶部尚書,以病,提舉醴泉觀。卒,年七十。贈右光祿大夫。



 公孺廉儉,與人寡合。嘗護曹佾喪,得厚餉,辭不受,談者清其節焉。



 張士遜,字順之。祖裕,嘗主陰城鹽院,因家陰城。士遜生百日始啼。淳化中,舉進士,調鄖鄉主簿,遷射洪令。轉運使檄移士遜治郪,民遮馬首不得去,因聽還射洪。安撫使至梓州,問屬吏能否,知州張雍曰:「射洪令,第一也。」改襄陽令,為秘書省著作佐郎、知邵武縣,以寬厚得民。前
 治射洪,以旱,禱雨白崖山陸使君祠,尋大雨,士遜立廷中,須雨足乃去。至是,邵武旱,禱歐陽太守廟,廟去城過一舍,士遜徹蓋,雨沾足始歸。改秘書丞、監折中倉,歷御史臺推直官。



 翰林學士楊億薦為監察御史。貢舉初用糊名法,士遜為諸科巡鋪官,以進士有姻黨,士遜請避去,真宗記名於御屏,自是有親嫌者皆移試,著為令。中書擬人充江南轉運使,再擬輒見卻,帝獨用士遜。再遷侍御史,徙廣東,又徙河北。河侵棣州,詔徙州陽信,議者
 患糧多,不可遷。士遜視瀕河數州方艱食,即計餘以貸貧者,期來歲輸陽信,公私利之。



 仁宗出閣,帝選僚佐,謂宰臣曰:「翊善、記室,府屬也,王皆受拜。今王尚少,宜以士遜為友,令王答拜。」於是以戶部郎中直昭文館,為壽春郡王友,改升王府諮議參軍,遷右諫議大夫兼太子右庶子,改左庶子。士遜言:「詣資善堂,升階列拜,而皇太子猶跪受,宜詔皇太子坐受之。」帝不許。詔士遜等遇太子侍駕出入許陪從。判史館,知審刑院,以太子賓客、樞密
 直學士判集賢院。既而二府大臣皆領東宮官,遂換太子詹事,擢樞密副使,遷給事中兼詹事,累遷尚書左丞,遂拜禮部尚書、同中書門下平章事、集賢殿大學士。



 曹汭獄事起,宦者羅崇勛、江德明方用事,因譖利用。帝疑之,問執政,眾顧望未有對者。士遜徐曰:「此獨不肖子為之,利用大臣,宜不知狀。」太后怒,將罷士遜。帝以其東宮舊臣,加刑部尚書、知江寧府,解通犀帶賜之。後領定國軍節度使、知許州。



 明道初,復入相,進中書侍郎兼兵部
 尚書。明年,進門下侍郎、昭文館大學士、監修國史。是歲旱蝗,士遜請如漢故事冊免,不許。及帝自損尊號,士遜又請降官一等,以答天變,帝慰勉之。群臣上章懿謚冊,退而入慰,士遜與同列過楊崇勛園飲,日中不至。御史中丞範諷劾士遜,以尚書左僕射判河南府,崇勛亦以使相判許州。翌日入謝,班崇勛下。帝問其故,士遜曰:「崇勛為使相,臣官僕射,位當下。」遂為山南東道節度使、同中書門下平章事、判許州,以崇勛知陳州。時士遜罷已
 累日,制猶用宰相銜,有司但奉行制書,不復追改。徙河南府。



 寶元初,復以門下侍郎、兵部尚書入相,封郢國公。士遜與輔臣奏事,帝從容曰:「朕昨放宮人,不獨閔幽閉,亦省浮費也。近復有獻孿女者,朕卻而弗受。」士遜曰:「此盛德事也。」帝徐曰:「近言者至有毀大臣、揭君過者。」士遜曰:「陛下審察邪正,則憸訐之人,宜自戒懼矣。」馮士元獄既具,帝以決獄問士遜。士遜曰:「臺獄阿徇,非出自宸斷,何以愜中外之論邪。」帝曰:「君子小人各有黨乎?」士遜曰:「有
 之,第公私不同爾。」帝曰:「法令必行,邪正有別,則朝綱舉矣。」



 康定初,士遜言禁兵久戍邊,其家在京師,有不能自存者。帝命內侍條指揮使以下為差等,出內藏緡錢十萬賜之。士遜又請遣使安撫陜西,帝命遣知制誥韓琦以行。於是詔樞密院,自今邊事,並與士遜等參議。及簡輦官為禁軍,輦官攜妻子遮宰相、樞密院喧訴,士遜方朝,馬驚墮地。時朝廷多事,士遜亡所建明,諫官韓琦論曰:「政事府豈養病之地邪。」士遜不自安,累上章請老,乃
 拜太傅,封鄧國公致仕。詔朔望朝見及大朝會,綴中書門下班,與一子五品服。士遜辭朝朔望。間遣中使勞問,御書飛白「千歲」字賜之,士遜因建千歲堂。嘗請買城南官園,帝以賜士遜。宰相得謝,蓋自士遜始。就第凡十年,卒,年八十六。帝臨奠,贈太師、中書令,謚文懿,御篆其墓碑曰「舊德之碑」。



 士遜生七日,喪母,其姑育養之。既長,事姑孝謹,姑亡,為行服,徒跣扶柩以葬,追封南陽縣太君。初,陳堯佐罷參知政事,人有挾怨告堯佐謀反,復有誣
 諫官陰附宗室者。士遜曰:「憸人構陷善良,以搖朝廷,奸偽一開,亦不能自保矣。」帝悟,抵告者以罪,誣諫官事亦不下。然曹利用在樞府,藉寵肆威,士遜居其間,無所可否,時人以「和鼓」目之。士遜嘗納女口宮中,為御史楊偕所劾。



 子友真字益之。初補將作監主簿,再遷為丞。士遜為請館閣校勘,仁宗曰:「館閣所以待英俊,不可。」乃令館閣讀書,詔校勘毋得增員。後編三館書籍,遷秘閣校理、同知禮院,賜進士出身,知襄州。坐軍賊張海剽劫不能
 制,罷歸。後除史館修撰,御史何郯言:「史館修撰,故事,皆試知制誥,友直不當得。」改集賢殿修撰。以天章閣待制知陜州,同勾當三班院。侍宴集英殿,猶衣緋衣,仁宗顧見之,乃賜金紫。累遷工部郎中、知越州。州民每春斂財,大集僧道士女,謂之「祭天」,友直下令禁絕,取所斂財建學以延諸生。卒官。士遜嘗記帝東宮舊事,而史官未之見,友直纂為《資善錄》上之。



 幼子友正字義祖,杜門不治家事,居小閣學書,積三十年不輟,遂以書名。神宗評其
 草書,為本朝第一。



 論曰:呂夷簡、張士遜皆以儒學起家,列位輔弼。仁宗之世,天下承平,因時制宜,濟以寬厚,相臣預有力焉。士遜練習民事,風跡可紀,而依違曹利用以取譏。方夷簡在下僚,諸父蒙正以宰相才期之。及其為相,深謀遠慮,有古大臣之度焉。在位日久,頗務收恩避怨,以固權利,郭后之廢,遂成其君之過舉,咎莫大焉。雖然,呂氏更執國政,三世四人,世家之盛,則未之有也。



\end{pinyinscope}