\article{志第九十三 樂十五(鼓吹上)}

\begin{pinyinscope}

 鼓吹者,軍樂也。昔黃帝涿鹿有功,命岐伯作凱歌,以建威武、揚德風、厲士諷敵。其曲有《靈夔競》、《雕鶚爭》、《石墜崖》、《壯士怒》之名,《周官》所謂「師有功則凱歌」者也。漢有《朱鷺》
 等十八曲,短簫鐃歌序戰伐之事,黃門鼓吹為享宴所用,又有騎吹二曲。說者謂列於殿庭者為鼓吹,從行者為騎吹。魏、晉而下,莫不沿尚,始有鼓吹之名。江左太常有鼓吹之樂,梁用十二曲,陳二十四曲,後周亦十五曲。唐制,大駕、法駕、小駕及一品而下皆有焉。



 宋初因之,車駕前後部用金鉦、節鼓、□鼓、大鼓、小鼓、鐃鼓、羽葆鼓、中鳴、大橫吹、小橫吹、觱慄、桃皮觱慄、簫、笳、笛,歌《導引》一曲。又皇太子及一品至三品,皆有本品鼓吹。凡大駕用一
 千五百三十人為五引,司徒六十四人,開封牧、太常卿、御史大夫、兵部尚書各二十三人。法駕三分減一,用七百六十一人為引,開封牧、御史大夫各一十六人。小駕用八百一十六人。太常鼓吹署樂工數少,每大禮,皆取之於諸軍。一品已下喪葬則給之,亦取於諸軍。又大禮,車駕宿齋所止,夜設警場,用一千二百七十五人。奏嚴用金鉦、大角、大鼓,樂用大小橫吹、觱慄、簫、笳、笛,角手取於近畿諸州,樂工亦取於軍中,或追府縣樂工備數。歌《
 六州》、《十二時》,每更三奏之。



 大中祥符六年,以其煩擾,詔罷追集,悉以禁兵充,常隸太常閱集。七年,親享太廟,登歌始作,聞廟外奏嚴,遂詔:行禮之次,權罷嚴警;禮畢,仍復故。



 凡祀前一日,上御青城門觀奏嚴。若車駕巡幸,則夜奏於行宮前,人數減於大禮,凡用八百八十人。真宗崇奉聖祖,亦設儀衛,別作導引曲,今附之。



 《兩朝志》云:「大駕千七百九十三人,法駕千三百五人,小駕千三十四人,人數多於前。



 鑾駕九百二十五人。迎奉祖宗御容或神主祔廟,用小鑾駕三百二十五人,上宗廟謚冊二百人,其曲即隨時更制。」



 自天聖已來,帝郊祀、躬
 耕籍田,皇太后恭謝宗廟,悉用正宮《降仙臺》、《導引》、《六州》、《十二時》,凡四曲。景祐二年,郊祀減《導引》第二曲,增《奉禋歌》。初,李照等撰警嚴曲,請以《振容》為名,帝以其義無取,故更曰《奉禋》。其後袷享太廟亦用之。大享明堂用黃鐘宮,增《合宮歌》。凡山陵導引靈駕,章獻、章懿皇后用正平調,仁宗用黃鐘羽,增《昭陵歌》。神主還宮,用大石調,增《虞神歌》。凡迎奉祖宗御容赴宮觀、寺院並神主祔廟,悉用正宮,惟仁宗御容赴景靈宮改用道調,皆止一曲。



 皇祐中大饗
 明堂,帝謂輔臣曰:「明堂直端門,而致齋於內,奏嚴於外,恐失靖恭之意。」詔禮官議之,咸言:「警場本古之鼛鼓,所謂夜戒守鼓者也。王者師行、吉行皆用之。今乘輿宿齋,本緣祀事,則警場亦因以警眾,非徒取觀聽之盛,恐不可廢。若以奏嚴之音去明堂近,則請列於宣德門百步之外,俟行禮時,罷奏一嚴,亦足以稱虔恭之意。」帝曰:「既不可廢,則祀前一夕邇於接神,宜罷之。」



 熙寧中,親祠南郊,曲五奏,正宮《導引》、《奉禋》、《降仙臺》;祠明堂,曲四奏,黃鐘
 宮《導引》、《合宮歌》:皆以《六州》、《十二時》。永厚陵導引、警場及神主還宮,皆四曲,虞主祔廟、奉安慈聖光獻皇后山陵亦如之。諸後告遷、升祔、上仁宗、英宗徽號,迎太一宮神像,亦以一曲導引,率因事隨時定所屬宮調,以律和之。



 元豐中,言者以鼓吹害雅樂,欲調治之,令與正聲相得。楊傑言:「正樂者,先王之德音,所以感召和氣、格降上神、移變風俗,而鼓吹者,軍旅之樂耳。蓋鼓角橫吹,起於西域,聖人存四夷之樂,所以一天下也;存軍旅之樂,示不
 忘武備也。『鞮鞻氏掌夷樂與其聲歌,祭祀則龠□而歌之,燕亦如之。』今大祀,車駕所在,則鼓吹與武嚴之樂陳於門而更奏之,以備警嚴。大朝會則鼓吹列於宮架之外,其器既異先代之器,而施設概與正樂不同。國初以來,奏大樂則鼓吹備而不作,同名為樂,而用實異。雖其音聲間有符合,而宮調稱謂不可淆混。故大樂以十二律呂名之,鼓吹之樂則曰正宮之類而已。



 乾德中,設鼓吹十二案,制氈床十二,為熊羆騰倚之狀。每案設大鼓、羽葆鼓、金錞各一,歌、簫、笳各二。又有叉手笛,名曰拱宸管,考驗皆與雅音
 相應,列於宮縣之籍,編之令式。



 若以律呂變易夷部宮調,則名混同而樂相紊亂矣。」遂不復行。



 元符三年七月,學士院奏:「太常寺鼓吹局應奉大行皇帝山陵鹵簿、鼓吹、儀仗,並嚴更、警場歌詞樂章,依例撰成。靈駕發引至陵所,仙呂調《導引》等九首,已令樂工協比聲律。」從之。



 政和七年三月,議禮局言:「古者鐃歌、鼓吹曲各易其名,以紀功烈。今所設鼓吹,唯備警衛而已,未有鐃歌之曲,非所以彰休德、揚偉績也。乞詔儒臣討論撰述,因事命名,審協聲律,播之鼓
 吹,俾工師習之。凡王師大獻,則令鼓吹具奏,以聳群聽。」從之。十二月,詔《六州》改名《崇明祀》,《十二時》改名《稱吉禮》,《導引》改名《熙事備成》,六引內者,設而不作。



 紹興十六年,臣僚言:「國家大饗、乘輿齋宿必設警場,肅儀衛而嚴祀事。樂工隸太常,歌詞備三疊,累朝以來皆用之。比者郊廟行事,率代以鉦、鼓,取諸殿司。夫軍旅、祭祀,事既異,宜樂聲清濁,用以殊尚。鉦、鼓、鳴角列於鹵簿中,所以示觀德之盛,宜詔有司更制,兼籍鼓吹樂工以時閱習,遇熙
 事出而用之。」有司請下軍器所造節鼓一,奏嚴鼓一百二十,鳴角亦如之,金鉦二十有四。太常前後部振作通用一千八百五十七人,而鼓吹益盛。



 孝宗隆興二年,兵部言:「奉明詔,大禮乘輿服御,除玉輅、平輦等外,所用人數並從省約。內鼓吹合用八百四十一人,止用五百八十八人;警場合用二百七十五人,止用一百三十人。」淳熙中大閱,帝自祥曦殿戎服而出,皇太子、親王、執政以下並從,諸將皆介冑乘馬導駕,軍器分衛前後,奏隨軍
 鼓管大樂。上尋易金甲,乘馬升將臺,殿帥舉黃旗,諸軍呼拜,奏發嚴,中軍鳴角。馬步簇隊,連三鼓。至四鼓,舉白旗,中軍鼓聲旗應,乃變方陣。別高一鼓,馬步軍出陣。別高一鼓,各歸部隊。五鼓舉黃旗,變員陣。又鼓,舉赤旗,變銳陣;青旗變直陣。收鼓訖,一金止,重鼓鳴角,簇隊放教。此其凡也。



 開寶元年南郊三首



 《導引》



 氣和玉燭,睿化著鴻明,緹管一陽生。郊禋盛禮
 燔柴畢,旋軫鳳凰城。森羅儀衛振華纓,載路溢歡聲。皇圖大業超前古,垂象泰階平。



 歲時豐衍,九土樂升平,睹寰海澄清。道高堯、舜垂衣治,日月並文明。《嘉禾》、《甘露》登歌薦,雲物煥祥經。兢兢惕惕持謙德,未許禪雲、亭。



 《六州》



 嚴夜警,銅蓮漏遲遲。清禁肅,森陛戟,羽衛儼皇闈。角聲勵,鉦鼓攸宜。金管成雅奏,逐吹逶迤。薦蒼璧,郊祀神祇,屬景運純禧。京坻豐衍,群材樂育,諸侯述
 職,盛德服蠻夷。



 殊祥萃,九苞丹鳳來儀。膏露降,和氣洽,三秀煥靈芝。鴻猷播,史冊相輝。



 張四維,卜世永固丕基。敷玄化,蕩蕩無為,合堯、舜文思。混並寰宇,休牛歸馬,銷金偃革,蹈詠慶昌期。



 《十二時》



 承寶運,馴致隆平,鴻慶被寰瀛。時清俗阜,治定功成,遐邇詠《由庚》。嚴郊祀,文物聲明。會天正,星拱奏嚴更,布羽儀簪纓。宸心虔潔,明德播惟馨。動蒼冥,神降享精誠。



 燔柴半,萬乘移天仗,肅鑾輅旋衡。
 千官雲擁,群后葵傾,玉帛旅明庭。《韶》、《濩》薦,金奏諧聲,集休亨。皇澤浹黎庶,普率洽恩榮。仰欽元後,睿聖貫三靈。萬邦寧,景貺福千齡。



 真宗封禪四首



 《導引》



 民康俗阜,萬國樂升平,慶海晏河清。唐堯、虞舜垂衣化,詎比我皇明!九天寶命垂丕貺,雲物效祥英。星羅羽衛登喬嶽,親告禪雲、亭。



 汾陰云:「星羅羽衛臨汾曲,親享答資生。」



 我皇垂拱,惠化洽文明,盛禮慶重行。登封、降禪燔柴
 畢,汾陰云:「告虔睢上皇儀畢。」



 天仗入神京。雲雷布澤遍寰瀛,遐邇振歡聲。巍巍聖壽南山固,千載賀承平。



 《六州》



 良夜永,玉漏正遲遲。丹禁肅,周廬列,羽衛繞皇闈。嚴鼓動,畫角聲齊。金管飄雅韻,遠逐輕颸。薦嘉玉,躬祀神祇,祈福為黔黎。升中盛禮,增高益厚,登封檢玉,《時邁》合《周詩》。



 汾陰云:「方丘盛禮,精嚴越古,陳牲檢玉,《時邁》展鴻儀。」



 玄文錫,慶雲五色相隨。甘露降,醴泉湧,汾陰云:「嘉禾合。」



 三秀發靈芝。皇猷播,史冊光輝。受鴻禧,萬年永固丕基。吾君德,蕩
 蕩巍巍,邁堯、舜文思。從今寰宇,休牛歸馬,耕田鑿井,鼓腹樂昌期。



 《十二時》



 聖明代,海縣澄清,惠化洽寰瀛。時康歲足,治定武成,遐邇賀升平。嘉壇上,昭事神靈。薦明誠,報本禪雲、亭,汾陰云:「蠲潔答鴻寧。」



 俎豆列犧牲。宸心蠲潔,明德薦惟馨。紀鴻名,千載播天聲。燔柴畢,汾陰云:「親祀畢。」



 雲罕回仙仗,慶鑾輅還京。八神卮蹕,四隩來庭,嘉氣覆重城。殊常禮,曠古難行,遇文明。仁恩蘇品匯,沛澤被簪纓。祥
 符錫祚,武庫永銷兵。育群生,景運保千齡。



 告廟《導引》



 明明我後,至德合高穹,祗翼勵精衷。上真紫殿回飆馭,示聖冑延鴻。躬承寶訓表欽崇,慶澤布寰中。告虔備物朝清廟,荷景福來同。



 奉祀太清宮三首



 《導引》



 穹旻錫祐,盛德日章明,見地平天成。垂衣恭己干戈偃,億載祐黎氓。羽旄飾駕當春候,款謁屆殊庭。精衷昭感膺多福,夷夏保咸寧。



 聖君御宇,祗翼奉
 三靈,已偃革休兵。區中海外鴻禧浹,恭館勵虔誠。九斿七萃著聲明,徯後徇輿情。丕圖寶緒承繁祉,率土仰隆平。



 《六州》



 千載運,寶業正遐昌。欽至道,崇明祀,盛禮邁前王。鑾輅動,萬騎騰驤。馳道紛彩仗,瑞日煌煌。奉秘檢,玉羽群翔,非霧滿康莊。躬朝真館,齊心繹思,順風俯拜,奠酒爇蕭薌。



 精衷達,飆輪降格昭彰。回羽旆,駐琱輦,舊地訪睢陽。享清廟,孝德輝光。屆靈場,星羅
 萬國珪璋。陳牲幣,金石鏘洋,景福降穰穰。垂衣法坐,恩覃群品,慶均海宇,聖壽保無疆。



 《十二時》



 乾坤泰,帝壽遐昌,宇縣樂平康。真游降格,寶誨昭彰,宸蹕造仙鄉。崇妙道,精意齊莊。款靈場,潔豆薦芬芳,備樂奏鏗鏘。猶龍垂裕,千古播休光。極褒揚,明號洽徽章。



 朝修展,春豫諧民望,睹文物煌煌。言旋羽衛,肅設壇場,報本達蕭薌。申嚴祀,禮備烝嘗,答穹蒼。純禧沾品匯,慶賚浹窮荒。封人獻壽,德化掩陶
 唐。保綿長,錫祐永無疆。



 亳州回詣玉清昭應宮一首



 《導引》



 秘文鏤玉,金閣奉安時,旌蓋儼仙儀。珠旒俯拜陳章奏,精意達希夷。卿雲鬱鬱曜晨曦,玉羽拂華枝。靈心報貺垂繁祉,寶祚永隆熙。



 親享太廟一首



 《導引》



 躬朝太室,列聖大功宣,彩仗耀甘泉。秘文升輅空歌發,一路覆祥煙。珠旒薦獻極精虔,列侍儼貂蟬。
 穰穰降福均寰宇,垂拱萬斯年。



 南郊恭謝三首



 《導引》



 重熙累盛,睿化暢真風,尊祖奉高穹。林棼彩仗明初日,瑞氣滿晴空。玉鑾徐動出環宮,虔鞏罄宸衷。禮成均慶人神悅,聖壽保無窮。



 《六州》



 承天統,聖主應昌辰。寶菉降,飆游至,瑞命慶惟新。崇大號,仰奉高真。獻歲當初吉,天下皆春。謁秘宇,藻衛星陳,薌靄極紛綸。瓊編焜耀,仙衣綷絲蔡,垂旒俯
 拜,薦獻禮惟寅。芬芳備,精衷上達穹旻。尊道祖,享清廟,助祭萬方臻。升泰畤,縟典彌文。侍群臣,漢庭儒雅彬彬。煙飛火舉,畢嚴禋,天地降氤氳。高臨華闕,恩覃動植,慶延宗社,聖壽比靈椿。



 《十二時》



 亨嘉會,萬宇歡康,聖化邁陶唐。元符錫命,天鑒昭彰,徽號奉琳房。陳縟禮,獻歲惟良。耀旗章,翠輦駐仙鄉,睿意極齊莊。仙衣渥彩,玉冊共熒煌。薦芬芳,飆馭降靈場。



 回雲罕,尊祖趨仙宇,金石韻鏘洋。聿朝
 清廟,躬奠瑤觴,報本國之陽。執籩豆,列侍貂榼,對穹蒼。洪恩霈夷夏,大慶浹家邦。垂衣紫極,聖壽保遐昌。集祺祥,地久與天長。



 天書《導引》七首



 詣泰山



 我皇纘位,覆燾合穹旻,秘菉示靈文。齊居紫殿膺玄貺,降寶命氤氳。奉符讓德事嚴禋,檢玉陟天孫。垂鴻紀號光前古,邁八九為君。



 汾陰云:「後祗坤德宅河、汾,瘞玉考前聞。垂休紀績超唐、漢,光監格鴻動。
 靈臺偃武,書軌慶同文,奄六合居尊。圓穹錫命垂真菉,清曉降金門。升中報本禪云云,汾陰云:「方丘報本務精勤。」



 嚴祀事惟寅。無為致治臻清凈,見反樸還淳。



 詣太清宮



 寶圖熙盛,登格聖功全,瑞命集靈篇。欽修祀典成明察,道祖降雲軿。賴鄉真館宅真仙,朝謁帝心虔。尊崇教父膺鴻福,綿亙萬斯年。



 猶龍勝境,真宇儼靈姿,肅謁展皇儀。寶符先路,嘉祥應,雲物煥金枝。紛紜紫節間黃麾,藻衛極葳蕤。高穹
 報貺延休祉,仁壽協昌期。



 詣玉清昭應宮



 紫霄金闕,重疊降元符,億兆祚皇圖。雲章焜耀傳溫玉,寶閣起清都。奉迎彩仗溢天衢,觀者競歡呼。明君欽翼承鴻蔭,億載御中區。寶符錫祚,慶壽命惟新,俄降格飆輪。巍巍帝德增虔奉,懿號薦穹旻。精齊秘館奉嚴禋,文物耀昌辰。升煙太一修郊報,鴻祉介烝民。



 詣南郊



 聖神纘緒,赫奕帝圖昌,寶錄降穹蒼。宸心
 勵翼修郊報,彩仗列康莊。祥煙瑞靄雜天香,筦磬發聲長。升壇禮畢膺繁祉,睿算保無疆。



 建安軍迎奉聖像《導引》四首



 玉皇大帝



 太霄玉帝,總御冠靈真,威德聳天人。寶文瑞命符皇運,綿遠慶維新。洞開霞館法虛晨,八景降飆輪。含生普洽空鴻福,聖壽比仙椿。



 聖祖天尊



 至真降鑒,飆馭下皇闈,清漏正依依。範金肖像申嚴奉,仙館壯翬飛。萬靈拱衛瑞煙披,岸柳映
 黃麾。九清祚聖鴻基永,堯德更巍巍。



 太祖皇帝



 元符錫命,祗受慶誠明,恭館法三清。開基盛烈垂無極,金像儼天成。奉迎霞布甘泉仗,簫瑟振和聲。靈辰協吉鴻儀畢,萬國保隆平。



 太宗皇帝



 膺乾撫運,垂慶洽重熙,元聖嗣鴻基。發揮寶緒靈仙降,感吉夢先期。良金璀璨範真儀,精意答蕃厘。閟宮神館崇嚴配,萬祀播葳蕤。



 聖像赴玉清昭應宮《導引》四首



 玉皇大帝



 先天氣祖,魄寶御中宸,列位冠高真。綠符錫瑞昭元聖,寶歷亙千春。琳宮壯麗從嚴闉,璇碧照龍津。珍金鑄像靈儀睟,集福庇烝民。



 聖祖天尊



 仙宗靈祖,御氣降中宸,孚宥慶惟新。國工鎔範成金像,儀炳動威神。玉虛聖境絕纖塵,歡抃洽群倫。導迎雲駕歸琳館,恭肅奉高真。



 太祖皇帝



 石文應瑞,真主御寰瀛,慈儉撫群生。巍巍威德超千古,大業保盈成。神皋福地開恭館,靈貺日
 昭明。鑄金九牧天儀睟,紺殿矗千楹。



 太宗皇帝



 乘雲英聖,千載仰皇靈,垂法藹朝經。禹金鎔範肖儀刑,日角煥珠庭。琳宮翠殿鳳文屏,迎奉慶安寧。孝思瞻謁薦惟馨,誠愨貫青冥。



 奉寶冊《導引》三首



 玉清昭應宮



 太霄垂祐,綿宇洽祺祥,秘檢煥雲章。宸心虔奉崇徽號,茂典邁前王。霞明藻衛列通莊,寶冊奉琳房。都人震抃騰謠頌,億載保歡康。



 景靈宮



 明明道祖,金闕冠仙真,清禁降飆輪。遙源始悟垂鴻慶,億兆聳群倫。虔崇徽號盛儀陳,寶冊奉良辰。邦家億載蒙繁祉,聖壽保無垠。



 太廟



 祖宗垂祐,亨會協重熙,德澤被烝黎。虔崇尊謚陳徽冊,藻衛列葳蕤。宸心致孝極孜孜,展禮詔臺司。祥煙瑞靄浮清廟,綿宇被純禧。



 治平四年英宗祔廟一首



 《導引》



 壽原初掩,歸蹕九虞終,億馭更無蹤。思皇攀慕
 追來孝,作廟繼三宗。旌旗居外擁千重,延望相威容。寶輿迎引歸新殿,奏享備欽崇。



 熙寧二年仁宗、英宗御容赴西京會聖宮應天禪院奉安一首



 《導引》



 九清三境,飆馭杳難追,功烈並巍巍。洛都不及西巡到,猶識睟容歸。三條馳道隱金槌,仙仗共逶迤。珠宮紺宇申嚴奉,億載固皇基。



 章惠皇太后神主赴西京一
 首



 《導引》



 祥符盛際,二鄙正休兵,瑞應滿寰瀛。東封西祀鳴鑾輅,從幸見升平。仙游一去上三清,廟食享隆名。寢園松柏秋風起,簫吹想平生。



 中太一宮奉安神像一首



 《導引》



 九霄仙馭,四紀樂西清,游衍遍黃庭。雲駢萬里歸真室,上應泰階平。金輿玉像下瑤京,彩仗擁霓旌。天人感會千年運,福祚永昌明。



 四年英宗御容赴景靈宮奉安一首



 《
 導引》



 鼎湖龍去,仙仗隔蓬萊,輦路已蒼苔。漢家原廟臨清渭,還泣玉衣來。鳳簫鑾扇共徘徊,帳殿倚雲開。春風不向天袍動,空繞翠輿回。



 十年南郊,皇帝歸青城《導引》一首



 《降仙臺》



 清都未曉,萬乘並駕,煌煌擁天行。祥風散瑞靄,華蓋聳旗常,建耀層城。四列兵衛,爟火映金支翠旌。眾樂警作充宮庭,皦繹成。紺幄掀,袞冕明。妥帖壇陛,霄升振珩璜,神格至誠。雲車下冥冥,儲祥降嘏莫
 可名。御端闕,□號敷榮。澤翔施溥,茂祉均被含生。



 元豐二年慈聖光獻皇后發引四首



 儀仗內《導引》一首



 駕班龍,忽催金母,轉仙仗,去瑤宮。絳闕深沉杳無蹤,漸塵空。絲網瓊林,花似怨東風,垂清露啼紅。猶想舊春中,獻萬壽,寶船空。



 警場內三曲



 《六州》



 九龍輿,記春暮,幸蓬壺。瓊囿敞,繡仗趨,年華與逝水俱。瑤京遠,信息斷無。寶津池面落花鋪,愁晚容車來禁途。鳳簫鑾翣,西指昭陵去。舊
 賞蟠桃熟,又見漲海枯。應共靈真母,曳霞裾。宴清都,恨滿山隅,春城翠柏藏烏。扃戶劍,照燈魚,人間一夢覺餘。泉宮窈窕金巢夜龍,銀江澄澹浴仙鳧,煙冷金爐玉殿虛。綠苔新長,雕輦曾行處。夜夜東朝月,似舊照錦疏,侍女盈盈淚珠。



 《十二時》



 治平時,暫垂簾,祐聖子,解危疑。坐安天下,逾歲厭避萬機,退處宸闈。殿開慶,養志入希夷。扶皓日,浴咸池。看神孫撫御,千載重雍累熙,四方欽仰洪慈。
 陰德遠,仁功積,歡養罄九域,禮無違。事難期,乘霞去,乍睹升仙,誥下九圍。泣血漣如,更鸞車動,春晚霧暗翠旗,路指嵩、伊。薤歌鳳吹,悠揚逐風悲。珠殿悄,綱塵垂。空坐濕。罔極吾皇孝思,鏤玉寫音徽。彤管煒,青編紀,寧更羨周《雅》播聲詩。



 《祔陵歌》



 真人地,瑞應待聖時。鞏原西,滎、河會,澗、洛與湹、伊,眾水縈回。嵩高映抱,幾疊屏幃。秀嶺參差,遙山群鳳隨。共瞻陵寢浮佳氣,非煙朝暮飛,龜筮告前期。
 奠收玉斝,筵卷時衣。鑾輅曉駕載龍旗,路逶遲。鈴歌怨,畫翣引華芝,霧薄風微。真游遠,閉寶閣金扉,侍女悲啼。玉階春草滋,露桃結子靈椿翠,青車何日歸!銜恨望西畿。便一房金巢,夜臺曉無期。



 虞主回京四首



 儀仗內《導引》一曲



 龍輿春晚,曉日轉三川,鼓吹慘寒煙。清明過後落花天,望池館依然。東風百寶泛樓船,共薦壽當年。如今又到苑西邊,但魂斷香軿。



 警場內三曲



 《六州》



 慶深恩,寶歷正乾坤。前帝子,後聖孫,援立兩儀軒。西宮大母朝寢門,望椒闥常溫。芳時媚景,有三千宮女,相將奉玉輦金根。上林紅英繁,縹緲鈞天奏梨園。望絕瑤池,影斷桃源。恨難論,開禁閽,春風丹旐翩翩。飛翠蓋,駕琱。轀,容衛入西原。管簫動地清喧,陵上柏煙昏。殘霞弄影,孤蟾浮天外,行人觸目是消魂。問蒼天,塵世光陰去如奔。河、洛潺湲,此恨長存。



 《
 十二時》



 望嵩、邙,永昭陵畔,王氣壓龍岡。鞏、洛靈光,鬱鬱起嘉祥。虛彩帟,轉哀仗,閟幽堂。嘆仙鄉路長,景霞飛松上。珠襦宵掩,細扇晨歸,昆閬茫茫。滿目東郊好,紅葩鬥芳,韶景空駘蕩。對春色,倍淒涼,最情傷。從輦嬪嬙,指瑤津路,淚雨泣千行。翠珥明榼,曾憶薦瓊觴。春又至,人何往,事難忘,向斜陽斷腸。聽鈞天嘹亮,清都風細,朱欄花滿,誰奏清商!紫幄重簾外,時飄寶香。環佩珊珊響,問何日,反琱房!



 《
 虞主歌》



 轉紫芝,指東都帝畿。愁霧裏,簫聲宛轉,輦路逶迤。那堪見,郊原芳菲,日遲遲。對列鳳翣龍旗,輕陰黯四垂。樓臺綠瓦冱琉璃,仙仗歸。壽原清夜,寒月掩褕禕。翠幰琱輪,空反靈螭。憩長岐,嵩峰遠,伊川渺水彌。此時還帝裏,旌幡上下,葆羽葳蕤。天街回,垂楊依依。過端闈,閶闔正闢金扉,觚棱射暖暉。虞神寶篆散輕絲,空涕洟。望陵宮女,嗟物是人非。萬古千秋,煙慘風悲。



 虞主祔廟儀仗內一首



 《導引》



 輕輿小輦,曾宴玉欄秋,慶賞殿宸游。傷心處,獸香散盡,一夜入丹丘。翠簾人靜月光浮,但半卷銀鉤。誰知道,桂華今夜,欲照鵲臺幽。



 五年景靈宮神御殿成,奉迎一首



 《導引》



 新宮翼翼,鉅麗冠神京,金虯蟠繡楹,都人瞻望洪紛處,陸海湧蓬、瀛。仙輿縹緲下圓清,彩仗擁天行。熉黃珠幄承靈德,錫羨永升平。



 慈孝寺彰德殿遷章獻明肅皇后御容赴景靈宮衍慶殿奉安一首



 《導引》



 九清雲杳,飆馭邈難追,功化盛當時。保扶仁聖成嘉靖,彤管載音徽。天都左界抗華榱,仙仗下逶迤。寶楹黼帳承神貺,萬壽永無期。



 八年神宗靈駕發引四首



 《導引》



 金殿晚,注目望宮車,忽聽受遺書。白雲縹緲帝鄉去,抱弓空慕龍湖。瑤津風物勝蓬壺,春色至,望琱
 輿。花飛人寂寂,淒涼一夢清都。



 《六州》



 炎圖盛,六葉正協重光。膺寶瑞,更法度,智通軼超成湯。昭回漢爛文章,震揚威武懾多方,生民帖泰擁殊祥。封人祝頌,萬壽與天長。豈知丹鼎就,龍下五雲旁。飄然真馭,游衍仙鄉。泣彤裳,伊、洛洋洋,嵩峰少室相望。藏弓劍,游衣冠,雋功盛德難忘。泉臺寂,魚燭熒煌。銀海深,鳧雁翱翔。想象平居,謾焚香。望陵人散,翠柏忽成行。獨餘嵩峰月,夜夜照幽堂,千秋陳跡淒涼。



 《
 十二時》



 珍符錫,祐啟真人,儲思在斯民。勤勞日升,萬物皆入陶鈞。收威柄,更法令,鼎從新。東風吹百卉,上苑正青春。流虹節近,衣冠玉帛,交奏嚴宸,萬壽祝堯仁。忽聽宮車晚出,但號慕,瞻雲路,企龍鱗。窮天英冠古精神。杳然上人素,人空望屬車巡。虛仗星陳,畫翣環擁龍輴。泉宮掩,帝鄉遠,邈難親。反琱輪,飛羽蓋,還渡天津。霧迷朱服,風搖細扇,觸目悲辛。列嬪嬙,垂紅淚,
 浥行塵。相將問,何日下青旻?



 《永裕陵歌》



 升龍德,當位富春秋。受天球,膺駿命,玉帛走諸侯。寶閣珠樓臨上苑,百卉弄春柔。隱約瀛洲,旦旦想宸游。那知羽駕忽難留,八馬入丹丘,哀仗出神州。笳聲凝咽,旌旗去悠悠。碧山頭,真人地,龜洛奧,鳳臺幽。繞伊流,嵩峰岡勢結蛟虯。皇堂一閉威顏杳,寒霧帶天愁。守陵嬪御,想象奉龍輈。牙盤赭案肅神休,何日覿雲裘!紅淚滴衣鞁,那堪風點綴柏城秋。



 虞主回京四首



 《導引》



 上林寒早,仙仗轉郊圻,笳鼓入雲悲。逶迤輦路過西池,樓閣鎖參差。都人瞻望意如疑,猶想翠華歸。玉京傳信杳無期,空掩赭黃衣。



 《六州》



 承聖緒,垂意在升平。驅貔虎,策豪英,號令肅天兵。四方無復羽書征,德澤浸群生。睿謀雄雋,絀漢高狹陋,慕三皇二帝登閎,緝樂綴文明。將升岱嶽告功成,玉牒金繩,勝寶飛聲。事難評。軒鼎就,清都一夢俄
 頃。飛霞佩,乘龍馭,羽衛入高清。祥光浮動五色,迎鸞鳳,雜簫笙。因山功就,同軌人至,銘旌畫翣,行背重城。楚笳凝咽,漢儀雄盛,攀慕傷情。惟餘內傳,知向蓬、瀛。



 《十二時》



 太平時,御華夷。躬聽斷,破危疑。春秋鼎盛,絀聲樂游嬉,日升繁機。長駕遠馭,垂意在軒、羲。恢六典,斥三垂。有殊尤絕跡,盛德旁魄周施,方將綴緝聲詩。擴皇綱,明帝典,紹累聖重熙,高拱無為,事難知。春色盛,逼千秋嘉節,忽聞憑玉幾,頒命彤闈,厭世御雲歸。
 翊翠鳳,駕文螭,縹緲難追。侍臣宮女,但攀慕號悲。玉輪動,指嵩、伊。龍鑣日益遠,空游漢廟冠衣。惟盛德巍巍,鏤玉冊,傳青史,昭示無期。



 《虞神》



 復土初,明旌下儲胥。回虛仗,簫笳互奏,旌旆隨驅。豈知飆御在蓬壺,道縈紆。風日慘,六馬躊躇,留恨滿山隅。不堪回首,翠柏已扶疏。帝城漸邇。愁霧金巢天衢。公卿百闢,鱗集云敷,迓龍輿。端門闢,金碧凌虛,此時還帝都。嚴清廟,入空畤,升文物,燦爛極嘉娛。配三
 宗,號稱神古所無。帝德協唐、虞,《九歌》畢奏斐然殊,會軒朱。神具燕喜,錫福集皇居。更千萬祀,祐啟邦圖。



 神主祔廟一首



 《導引》



 歲華婉娩,侍宴玉皇宮,琱輦出房中。豈知軒後丹成去,望絕鼎湖龍。壽原初掩九虞終,歸蹕五雲重。惟餘寶冊書鴻烈,清廟配三宗。



 政和三年追冊明達皇后一首



 《導引》



 來嬪初載,令德冠層城,柔範藹徽聲。熊羆夢應
 芳蘭鬱,佳氣擁雕楹。珠宮縹緲泛蓬、瀛,脫屣世緣輕。空餘寶冊光瓊玖,千古仰鴻名。



 神主祔別廟一首



 《導引》



 柔容懿範,蚤歲藹層闈,蘭夢結芳時。秋風一夜驚羅幕。鸞扇影空回。榮追禕翟盛威儀,遺像掩瑤扉。春來只有芭蕉葉,依舊倚晴暉。



 景靈西宮坤元殿奉安欽成皇后御容一首



 《導引》



 雲軿芝蓋,仙路去難攀,海浪濺三山。重迎遺像
 臨馳道,還似在人間。西宮瑤殿指坤元,璇榜聳飛鸞。移升寶殿從新詔,盛典永流傳。



 別廟一首



 《導引》



 蓬萊邃館,金碧照三山,真境勝人間。秋風又見芭蕉長,遺跡在人寰。雲軒一去杳難攀,斑竹彩輿
 還。深宮舊檻聞簫
 鼓,悵望慘朱顏。



\end{pinyinscope}