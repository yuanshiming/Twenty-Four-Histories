\article{志第五 天文五}

\begin{pinyinscope}

 七曜景星彗孛客星流星妖星星變雲氣日食日變日輝氣月食月變月輝氣



 七曜



 日為太陽之精,君之象,日行一
 度,一年一周天。日月行有道之國,則光明。君道至大,則日色光明;動不失時,則日揚光。至德之萌,日月如連壁。君臣有道,則日含「王」字;君亮天工,則日備五色;有聖人起,則日再中。人君有德,日有四彗,光芒四出;日有二彗,一年再赦。



 《周禮》視祲掌十輝之法;一曰祲,陰陽五色之氣,浸淫相侵;二曰象,雲氣成形象;三曰鑴,日旁氣刺日;四曰監,雲氣臨日上;五曰闇暗,謂蝕及日光脫;六曰瞢,不光明;七曰彌,白虹貫日;八曰序,謂氣若山而在日上,及冠珥背璚重疊次序在於日旁;九曰隮,謂暈及虹也;十曰想,五色有形想。



 凡黃氣
 環在日左右為抱氣;居日上為戴氣、為冠氣;居日下為承氣、為履氣;居日下左右為紐氣、為纓氣。抱氣則輔臣忠,餘皆為喜、為得地,吉。



 一珥在日西則西軍勝,在東則東軍勝,南北亦然;無兵,亦有拜將。兩珥氣圜而小在日左右,主民壽考。三珥色黃白,女主喜;純白,為喪;赤,為兵;青,為疾;黑,為水。四珥主立侯王,有子孫喜。



 日有黃芒,君福昌;多黃輝,王政太平。日無光,為兵、喪,又為臣有陰謀。日旁雲氣白如席,兵眾戰死;黑,有叛臣;如蛇貫之而青,穀多傷;白,為兵;赤,其下有叛;黃,臣下交兵;黑,為水。日始出,黑雲氣貫之,三日有暴雨。青雲在上下,可出兵。有赤氣如死蛇,為饑,為疫。雜氣刺日皆為兵。



 日暈,七日內無風雨,亦為兵;甲乙,憂火;丙丁,臣下忠;戊己,後族盛;庚辛,將利;壬癸,臣專政。半暈,相有謀;黃,則吉;黑,為災。暈再重,歲豐;色青,為兵,穀貴;赤,蝗為災。三重,兵起。四重,臣叛。五重,兵、饑。六重,兵、喪。七重,天下亡。



 日並出,諸侯有謀,無道用兵者亡。日闢,為兵寇。日隕,下
 失
 政。
 日中見飛燕,下有廢主。日中黑子,臣蔽主明。日晝昏,臣蔽君之明,有篡弒。赤如血,君喪臣叛。日夜出,兵起,下陵上,大水。日光四散,君失明。白虹貫日,近臣亂,諸侯叛。日赤如火,君亡。日生牙,下有賊臣。



 日食,為陰蔽陽,食既則大臣憂,臣叛主,兵起。
 日食在正旦,王者惡之。日珥,甲乙,日有二珥四珥而食,白雲中出,主兵;丙丁,黑雲,天下疫;戊己,青雲,兵、喪;庚辛,赤雲,天下有少主;壬癸,黃雲,有土功。日食在甲乙
 日,主四海之外,不占;丙丁,江、淮、海、岱也;戊己,中州河、濟也;庚辛,華山以西;壬癸,常山以北。各以其下所主當之。寅卯辰木,招謀者司徒也。巳午未火,招謀者太子也。申酉戌金,司馬也。亥子丑水,司空也。



 月為太陰之精,女主之象,一月一周天。君明,則依度;臣專,則失道。或大臣用事,兵刑失理,則乍南乍北;或女主外戚專權,則或進或退。月變色,為殃;青,饑;赤,兵、旱;黃,喜;黑,水。晝明,則奸邪作。月旁瑞氣,一珥,五穀登;兩珥,外兵
 勝;四珥及生戴氣,君喜國安。終歲不暈,天下偃兵。



 晦而明見西方,曰朏;朔而明見東方,曰仄匿。朏則政緩,仄匿則政急。六日而弦,臣專政。七日而弦,主勝客。八日而弦,天下安。十日不弦,將死,戰不勝。



 兩月並見,兵起,國亂,水溢。星入月中,亡國破將。白暈貫之,下有廢主。白虹貫之,為大兵起。生齒,則下有叛臣。生足,則後族專政。



 月珥背璚,暈而珥,六十日兵起;珥青,憂;赤,兵;白,喪;黑,國亡;黃,喜。有背璚,臣下馳縱,欲相殘賊,不和之氣。暈三重,兵起;四
 重,國亡;五重,女主憂;六重,國失政;七重,下易主;八重,亡國;九重,兵起亡地;十重,天下更始。



 月食,從上始則君失道,從旁始為相失令,從下始為將失法。歲星犯之,兵、饑、民流。熒惑犯之,大將死,有叛臣,民饑。填星犯之,人臣弒主;合,國饑。月食填星,民流;一曰月犯填,女主憂,民流。太白犯,出月右為陰國有謀,左為陽國有謀;出月下,君死、民流。



 月戴太白,起兵;入月,將死;與太白會,太子危。辰星犯之,天下水。月食辰,水,饑。辰入月,臣叛主。彗星入,或犯
 之,兵期十二年,大饑;貫月,臣叛主。流星犯之,有兵;入無光,有亡國;在月上下,國將亂。月犯列星,其國受兵。星食月,國相死。星見月中,主憂。



 凡月之行,歷二十有九日五十三分而輿日相會,是謂合朔。當朔日之交,月行黃道而日為月所掩,則日食,是為陰勝陽,其變重,自古聖人畏之。若日月同度於朔,月行不入黃道,則雖會而不食。月之行在望與日對沖,月入於闇虛之內,則月為之食,是為陽勝陰,
 其變輕。昔朱熹謂月食終亦為災,陰若退避,則不至相敵而食。所謂闇虛,蓋日火外明,其對必有闇氣,大小與日體同。此日月交會薄食之大略也。日食修德,月食修刑,自昔人主遇災而懼,側身修行者,此也。



 歲星為東方,為春,為木。於人五常,仁也;五事,貌也。超舍而前為贏,退舍為縮。色光明潤,君壽民富。又主福,主大司農,主五穀。石申曰:歲星所在,國不可伐,如歲在卯,不可東征。甘德曰:所去,國兇;所之,國吉;退行,為兇災。主泰
 山、徐青兗及角、亢、氐、房、心、尾、箕。君令不順,則歲星退行;入陰為內事,入陽為外事;行陰道為水,行陽道為旱。星大,則喜;小,則牛馬多死,疾疫。初見小而日益大,所居國利。初出大而日小,國耗。《荊州占》:歲星色黑,為喪;黃,則歲豐;白,為兵;青,多獄;君暴,則色赤。熒惑相犯,為大戰;相去方寸為犯,戰,客勝。食火,國亡。邊侵曰食。守之為賊。居之不去為守。觸火,則國亂。兩體俱動而直曰觸。合鬥,為饑、旱。離復合、合復離曰斗。填星相犯,退,犯填,太子叛。當
 東反西曰退。與填星合,為內亂,民饑。芒角相及同光曰合。守填星,其下城敗。太白相犯,大臣黜,女主喪。觸太白,則四邊來侵。守太白,為四序不調。合鬥,則大將死。辰星相犯,太子憂。觸辰,主憂;守,憂賊。合,則君臣和。晝見,則臣強。他星犯之,主不安。客星犯守,主憂。流星犯之,色蒼黑,大農死;赤,為饑疫;黃,則歲豐。抵之,臣叛主。



 熒惑為南方,為夏,為火。於人五常,禮也;五事,視也。晉灼曰:「常以十月入太微,受制而出,行列宿,司無道,出入無
 常。」二歲一周天。出,則有兵;入,則兵散。逆行一舍二舍,為不祥,所舍國為亂、賊、疾、喪、饑、兵。或環繞勾巳,芒角、動搖、變色,乍前乍後,為殃愈甚。退行一舍,天下有火災;五舍,大臣叛。《星經》曰:「主霍山、揚、荊、交州,又主輿鬼、柳、七星。」又主大鴻臚,又曰主司空,為司馬,主楚、吳、越以南,司天下君臣之過失。東行,則兵聚東方;西行,則兵聚西方。天下安,則行疾。與歲星相犯,主冊太子,有赦。觸歲星,有子;守之,太子危。填星相犯,兵大起。入填星,將為亂;觸之,有刀兵;
 守之,有內賊,太子危。與太白相犯,主亡,兵起;守北,太子憂;南,庶子憂。環繞,偏將死。與辰星相犯,兵敗。與辰星相會,為旱,秋為兵,冬為喪;守之,太子憂,有赦。他星相犯,兵起,祅星犯之,為兵,為火。



 填星為中央,為季夏,為土。於人五常,信也;五事,思也。常以甲辰元始之歲填行一宿,二十八歲而一周天。四星皆失,填為之動。所居,國吉,女子有福,不可伐。去之,失地。天子失信,則填大動。盈則超舍,以德盈則加福,刑盈則
 不復;縮則退舍不及常,德宿則迫戚,刑縮則不育。《星經》曰:「主嵩山、豫州,又主東井。」行中道,則陰陽和調。退行一舍,為水;二舍,海溢河決。經天退行,天下更政,地動。巫咸曰:光明,歲熟。大明,主昌。小暗,主憂。春青,夏赤,女主喜。春色蒼,歲大熟;色赤,饑。有芒,兵。與歲星相犯相鬥,為內亂;合,則野有兵。熒惑相犯,為兵、喪;合,則為兵,為內亂,大人忌之。太白相犯,為內兵,有大戰,一曰王者失地。合於太微,國有大兵,一曰國亡。辰星犯,為兵,為旱。祅星犯,下臣
 謀上。流星犯,則民多事。與月相犯,有兵。



 太白為西方,為秋,為金。於人五常,義也;五事,言也。常以正月甲寅與火晨出東方,二百四十日而入。入四十日又出西方,二百四十日而入。入三十五日而復出東方。出以寅戌,入以丑未也。一年一周天。日方南太白居其南,日方北太白居其北,為贏,侯王不寧,用兵進吉退兇。日方南太白居其北,日方北太白居其南,為縮,侯王有憂,用兵退吉進兇。《星經》曰:「主華陰山、梁、雍、益州,又主奎、
 婁、胃、昴、畢、觜、參。」出西方,失行,外國敗。出東方,失行,中國敗。若經天,天下革,民更主,是謂亂紀,人眾流亡。晝見,與日爭明,強國弱,女主昌,又曰主大臣。巫咸曰:光明見影,戰勝,歲熟。狀炎然而上,兵起。光如張蓋,下有立王。凡與歲星相犯,兵敗失地。犯熒惑,客敗主勝。犯填星,太子不安,失地。犯辰星,主兵。入月,主死,其下兵。犯月角,兵起,在左則中國勝,在右則外國勝。當見不見,失地破軍。他星犯,其事急。祅星犯,邊城有戰。客星犯,主兵將死。凡太白
 至午位,避日而伏,若行至未,即為經天,其災異重也。



 辰星為北方,為冬,為水。於人五常,智也;五事,聽也。常以二月春分見奎、婁,五月夏至見東井,八月秋分見角、亢,十一月冬至見牽牛。出以辰戌,入以丑未,二旬而入。晨候之東方,夕候之西方也。一年一周天。出早為月食,晚為彗星及天祅。一時不出,其時不和。四時不出,天下大饑。《星經》曰:「主常山、冀、並、幽州,又主斗、牛、女、虛、危、室、壁。」又曰主燕、趙、代,主廷尉,以比宰相之象。石申曰:色黃,五穀
 熟;黑,為水;蒼白,為喪。凡與歲星相犯,皇后有謀。熒惑犯,妨太子。填星犯,兵敗;太白亦然。芒角相及同光曰合,他星光曜相逮為害。客星、太陰、流星相犯,主內患。



 凡五星:歲星色青,比參左肩;熒惑色赤,比心大星;填星色黃,比參右肩;太白色白,比狼星;辰星色黑,比奎大星。得其常色而應四時則吉,變常為兇。



 木與土合為內亂,饑;與水合為變謀而更事;與火合為饑,為旱;與金合為白衣之會,合鬥,國有內亂,野有破軍,為水。
 太白在南,歲星在北,名曰牝牡,年穀大熟。太白在北,歲星在南,其年或有或無。火與金合為爍,為喪,不可舉事用兵,從軍為軍憂;離之,軍卻。出太白陰,分地;出其陽,偏將戰。與土合為憂,主孽卿。與水合為北軍,用兵舉事大敗。一曰,火與水合為焠,不可舉事用兵。土與水合為壅沮,不可舉事用兵,有覆軍。一曰,為變謀更事,必為旱。與金合為疾,為白衣會,為內兵,國亡地。與木合國饑。水與金合為變謀,為兵、憂。



 木、火、土、金與
 水斗,皆為戰,兵不在外,皆為內亂。



 三星合,是謂驚立絕行,其國外內有兵與喪,百姓饑乏,改立侯王。四星合,是謂大湯,其國兵、喪並起,君子憂,小人流。五星若合,是謂易行,有德,受慶,改立王者,奄有四方,子孫蕃昌;亡德受殃,離其國家,滅其宗廟,百姓離去,被滿四方。五星皆大,其事亦大;皆小,事亦小。五星俱見,其年必惡。



 凡五星與列宿相去方寸為犯,居之不去為守,兩體俱動而直曰觸,離復合、合復離曰斗,當東反西
 曰退,芒角相及同舍曰合。凡五星東行為順,西行曰逆,順則疾,逆則遲,通而率之,終於東行。不東不西曰留,與日相近而不見曰伏,伏與日同度曰合。



 凡金、水二星,行速而不經天,自始與日合後,行速而先日,夕見西方。去日前稍遠,夕時欲近南方則漸遲,遲極則留,留而近日,則逆行而合日;在於日後,晨見東方。逆極則留,留而後遲,遲極去日稍遠,旦時欲近南方,則遠行以追日,晨伏於東方,復與日合度。此五星合見、
 遲疾、順逆、留行之大端也。



 凡五星之行,古法:周天之數,如歲星謂十二年一周天,乃約數耳。晉灼謂太歲在四仲則行三宿,在四孟、四季則行二宿,故十二年而行周二十八宿。其說亦非。夫二十八宿,度有廣狹,而歲星之行自有盈縮,豈得以十二年一周無差忒乎?唐一行始言歲星自商、周迄春秋季年,率百二十餘年而超一次,因以為常。以春秋亂世則其行速,時平則其行遲,其說尤迂。既
 乃為後率前率之術以求之,則其說自悖矣。今紹興歷法,歲星每年行一百四十五分,是每年行一次之外有餘一分,積一百四十四年剩一次矣。然則先儒之說安可信乎?餘四星之行,固無逆順,中間亦豈無差忒?一行不復詳言,蓋亦知之矣。



 景星



 景星,德星也,一曰瑞星,如半月,生於晦朔,大而中空,其名各異。曰周伯,其色黃,煌煌然,所見之國大昌。曰含譽,
 光耀似彗,喜則含譽射。曰格澤,狀如炎火,下大上銳,色黃白,起地上,見則不種而獲。曰歸邪,兩赤彗向上,有蓋。曰天保星,有音,如炬火下地、野雞鳴。皆五行沖和之氣所生也。其王蓬芮、玄保、昭明、昏昌、旬始、司危、菟昌、地維臧光之類,亦皆為瑞星。然前志以王蓬芮已下星為妖星。又奇星,古無所考,見於仁宗、英宗之時,故附於景星之末云。



 彗孛



 彗星,小者數寸,長者或竟天。見則兵起,大水,除舊布新之兆也。其體無光,傅日而為光。故夕見則東指,晨見則西指。光芒所及則為災。有五色,各依五行本精所生。



 孛星,彗屬。偏指曰彗,芒氣四出曰孛。孛者,孛孛然,非常惡氣之所生也。主大亂,主大兵,災甚於彗。旄頭星,《玉冊》云:亦彗屬也。



 客星



 客星有五:周伯、老子、王蓬絮、國皇、溫星是也。周伯,大而
 黃,煌煌然,所見之國,兵喪,饑饉,民庶流亡。老子,明大純白,出則為饑,為兇,為善,為惡,為喜,為怒。王蓬絮,狀如粉絮,拂拂然,見則其國兵起,有白衣之會。國皇,大而黃白,有芒角,主兵起,水災,人主惡之。溫星,色白,狀如風動搖,常出四隅。皆主兵。此五星錯出乎五緯之間,其見無期,其行無度,各以其所在分野而占之。又四隅各有三星:東南曰盜星,主大盜;西南曰種陵,出則穀貴;西北曰天狗,見則天下大饑;東北曰女帛,主有大喪。



 流星



 流星,天使也。自上而降曰流,東西橫行亦曰流。流星有八,曰天使,曰天暉,曰天雁,曰天保,曰地雁,曰梁星,曰營頭,曰天狗。流星之為天使者,有祥有妖,為天暉、天雁、夜隕而為天保,則祥;若夜隕而為地雁、梁星,晝隕而為營頭,則妖。流星之大者為奔星,夜隕而為天狗,厥妖大。自下而升曰飛。飛星有五,亦有妖祥之分,飛星化而為天刑則祥;為降石,為頓頑,為解銜,為大滑,則為妖。



 妖星



 妖星,五行乖戾之氣也。五星之精,散而為妖星,形狀不同,為殃則一。各以其所見日期、分野、形色,占為兵、饑、水、旱、亂、亡。星長三尺至五尺,期百日,等而上之,至一丈期一年,三丈期三年,五丈期五年,十丈期七年,十丈已上,不出九年。蓋妖星長大則期遠而殃深,短小則期近而殃淺。



 天攙星乃歲之精,主奮爭。天槍如彗,出西方,長二三尺,名天槍,主破國。天猾主招亂。天棓出西方,長數丈,
 主國亂。蚩尤旗類彗而後曲,主兵。天沖狀如人,蒼衣赤首,不動,主下謀上,滅國。國皇大而赤,去地三丈,如炬火,主內寇。及登主夷分,主恣虐,旦見則主弱。昭明如太白,光芒不行,主兵、喪。司危,《天官書》如太白,有目,去地可六丈,大而白,其下有兵,主擊強。五殘如辰星,去地六七丈,其下有兵,主奔亡。六賊去地六丈,大而赤,有光,出非其方,下有兵、喪。獄漢青中赤表,下有三彗,去地可六丈,大而赤,數動。大賁主滅邪暴兵。燭星主滅邪。絀流主伏逃。
 茀星、昴、孛星主災。旬始出北斗旁,如雄雞,見則更主。擊咎主大兵,有反者,大亂。天杵主□羊。天柎主擊殃。伏靈見則世亂。天敗主斗沖。司奸主見怪。天狗有毛,旁有短彗,下如狗形,見則兵饑。天殘主貪殘。卒起有謀反,主驚亡。枉矢色黑,蛇行,望之如有毛目,長數匹者,見則兵起,破女君臣憂,上下亂。拂樞主制時。滅寶主伐亂。繞綎主亂孳。驚理主相屠。大奮祀主招邪。



 天鋒彗象,形似矛鋒,見則兵起,有亂臣。昭星有三彗,兵出,有大盜不成,又主
 滅邪。蓬星大如二斗器,色白,出東南方,東北主旱,或大水。長庚星如一匹布著天,見則兵起。四填大而赤,可二丈,為兵。地維臧光星如月,始出,大而赤,去地二丈,東南,旱;西北,兵;出東北,大水。老子星色白,為善為惡,為饑為兇,為喜為怒。營頭星有雲如壞山墜,所墜下有覆軍流血。積陵出西南,長三丈,主兵,小饑。昏昌出西北,氣青赤色,中赤外青,主國易政。莘星出西北,狀如環,大則諸侯失地。白星如削瓜,主男喪。菟昌有赤青環之,主水,天下
 改易。蒙星赤如牙旗,長短四面,西南最多,亂之象。長星出西方。



 歲星之精,化為天棓、天槍、天猾、天沖、國皇、及登,蒼彗。火星之精,化為昭旦、蚩尤之旗、昭明、司危、天欃,赤彗。土星之精,化為五殘、六賊、獄漢、大賁、昭星、絀流、茀星、旬始、蚩尤,虹蜺、擊咎,黃彗。太白之精,化為天杵、天柎、伏靈、天敗、司奸、天狗、天殘、卒起,白彗。辰星之精,化為枉矢、破女、拂樞、滅寶、繞綎、驚理、大奮祀,黑彗。



 而月旁祅星,亦各有所生。天槍、天荊、真若、天手袁、天樓、天垣,歲星所生也,
 見以甲寅日,有兩青方在其旁。天陰、晉若、官張、天惑、天雀、赤若、蚩尤,熒惑所生也,出在丙寅日,有兩赤方在其旁。天上、天伐、縱星、天樞、天翟、天沸、荊彗,填星所生也,出在戊寅日,有兩黃方在其旁。若星、帚星、若彗、竹彗、墻星、權星、白雚,太白所生也,出在庚寅日,有兩白方在其旁。天美、莒天毚、天社、天林、天庥、天蒿、端下,辰星所生也,出以壬寅日,有兩黑方在其旁,見則為水、旱、兵、喪、饑、亂。



 雲氣



 《
 周禮·保章氏》:「以五雲之物辨吉兇,水旱降豐荒之祲象。」故魯僖公日南至登觀臺以望,漢明帝升靈臺以望元氣,吹時律,觀物變。蓋古者分至主啟閉必書,雲物為備故也。迨乎後世,其法寢備。瑞氣則有慶雲、昌光之屬,妖氣則有虹蜺、牂雲之類,以候天子之符應,驗歲事之豐兇,明賢者之出處,占戰陣之勝負焉。



 日食



 建隆元年五月己亥朔,日有食之。二年四月癸巳朔,日
 有食之。



 乾德三年二月壬寅朔,日當食不食。五年六月戊午朔,日有食之。



 開寶元年十二月己酉朔,日有食之。三年四月辛酉朔,日有食之。四年十月癸亥朔,日有食之。五年九月丁巳朔,日有食之。七年二月庚辰朔,日有食之。八年七月辛未朔,日有食之。



 太平興國二年十一月丁亥朔,日有食之,既。六年九月乙未朔,日有食之。七年三月癸巳朔,日有食之。八年二月戊子朔,日有食之。



 雍熙二年十二月庚子朔,日有食之。三年六月戊戌朔,
 日有食之。



 淳化二年閏二月辛未朔,日有食之。三年二月乙丑朔,日有食之。四年二月己未朔,日有食之。八月丙辰朔,日有食之。五年十二月戊寅朔,日有食之,雲陰不見。



 咸平元年五月戊午朔,日有食之。十月丙戌朔,日有食之。二年九月庚辰朔,日有食之。三年三月戊寅朔,日有食之。五年七月甲午朔,日有食之。



 景德元年十二月庚辰朔,日有食之。三年五月壬寅朔,日有食之,雲陰不見。四年五月丙申朔,日有食之,陰雨不見。



 大中祥符
 二年三月丙辰朔,日有食之,陰雨不見。五年八月丙申朔,日有食之。六年十二月戊午朔,日有食之。七年十二月癸丑朔,日當食不食。八年六月己酉朔,日有食之。



 天禧三年三月戊午朔,日有食之。五年七月甲戌朔,日有食之。



 乾興元年七月甲子朔,日食幾盡。



 天聖二年五月丁亥朔,日當食不食。四年十月甲戌朔,日有食之。六年三月丙申朔,日有食之。七年八月丁亥朔,日有食之。



 明道二年六月甲午朔,日有食之。



 景祐三年四月己酉朔,
 日當食不食。



 寶元元年正月戊戌朔,日有食之。



 康定元年正月丙辰朔,日有食之。



 慶歷二年六月癸酉朔,日有食之。三年五月丁卯朔,日有食之。四年十一月戊午朔,日當食不食。五年四月丁亥朔,日有食之,雲陰不見。六年三月辛巳朔,日有食之。



 皇祐元年正月甲午朔,日有食之。四年十一月壬寅朔,日有食之。五年十月丙申朔,日有食之。



 至和元年四月甲午朔,日有食之。



 嘉祐元年八月庚戌朔,日有食之。三年八月己亥朔,日有食之。四
 年正月丙申朔,日有食之。六年六月壬子朔,日有食之,雲陰不見。



 熙寧元年正月甲戌朔,日有食之。二年七月乙丑朔,日有食之,雲陰不見。六年四月甲戌朔,日有食之,雲陰不見。八年八月庚寅朔,日有食之,雲陰不見。



 元豐元年六月癸卯朔,日當食不食。三年十一月己丑朔,日有食之。四年十一月癸未朔,日當食不食。五年四月壬子朔,日有食之,雲陰不見。六年九月癸卯朔,日有食之。



 元祐二年七月庚戌朔,日有食之,陰雨不見。六年五
 月己未朔,日有食之。



 紹聖元年三月壬申朔,日有食之。二年二月丁卯朔,日當食不食。四年六月癸未朔,日有食之,雲陰不見。



 元符三年四月丁酉朔,日有食之。



 建中靖國元年四月辛卯朔,日有食之,雲陰不見。



 大觀元年十一月壬子朔,日有食之。二年五月庚戌朔,日有食之。四年九月丙寅朔,日有食之。



 政和三年三月壬子朔,日有食之。五年七月戊辰朔,日有食之。



 重和元年五月壬午朔,日有食之。



 宣和元年四月丙子朔,日有食之。五年
 八月辛巳朔,日有食之,陰雲不見。



 建炎三年九月丙午朔,日食於亢。



 紹興五年正月乙巳朔,日食於女。七年二月癸巳朔,日食於室是年當金之天會十五年,《金史》不書日食。八年至十二年,日食多在夜,史蒙蔽不書。十三年十二月癸未朔,日食於牛,陰雲不見。十五年六月乙亥朔,日食於井。十七年十月辛卯朔,日食於氐是年乃金之皇統七年,《金史》不書日食。十八年四月戊子朔,日有食之,陰雲不見。十九年三月癸未朔,日有食之,陰雲不見。二十四年五月癸丑朔,日有食之,
 陰雲不見。二十五年五月丁未朔,日有食之,陰雲不見。二十八年三月辛酉朔,日有食之,陰雲不見。三十年八月丙午朔,日食於翼。三十一年正月甲戌朔,太史言日當食而不食。三十二年正月戊辰朔,日食於女。



 隆興元年六月庚申朔,日食於井。二年六月甲寅朔,日有食之,陰雲不見。



 乾道五年八月甲申朔,日食在翼,陰雲不見。九年五月壬辰朔,日食在井,陰雲不見。



 淳熙元年十一月甲申朔,日食在尾,陰雲不見。三年三月丙午朔,日有
 食之,陰雲不見。四年九月丁酉朔,日有食之,陰雲不見。十年十一月壬戌朔,日食於心。十五年八月甲子朔,日食於翼。十六年二月辛酉朔,日有食之,陰雲不見。



 慶元元年三月丙戌朔,日食於婁。四年正月己亥朔,日有食之,陰雲不見。五年正月癸巳朔,日有食之,陰雲不見。六年六月乙酉朔,日有食之,陰雲不見是年乃金承安五年,《金史》不書日食。



 嘉泰二年五月甲辰朔,日食於畢。三年四月己亥朔,日有食之《金史》不書。



 開禧二年二月壬子朔,日當食,太史言不
 見虧分。



 嘉定三年六月丁巳朔,日有食之。四年十一月己酉朔,日當食,太史言不見虧分。《金史》不書。七年九月壬戌朔,日食於角。九年二月甲申朔,日食於室。十年七月丙子朔,日食於張。十一年七月庚午朔,日有食之。十四年五月甲申朔,日食於畢。十六年九月庚子朔,日食於軫。



 寶慶三年六月戊申朔,日有食之。



 紹定元年六月壬寅朔,日有食之。六年九月壬寅朔,日有食之,陰雲不見。



 端平二年二月甲子朔,日當食不虧。



 嘉熙元年十二月戊
 寅朔,日有食之。



 淳祐二年九月庚辰朔,日有食之。三年三月丁丑朔,日有食之。五年七月癸巳朔,日有食之。六年正月辛卯朔,日有食之。九年四月壬寅朔,日有食之。十二年二月乙卯朔,日有食之。



 寶祐元年二月己酉朔,日有食之。



 景定元年三月戊辰朔,日有食之。二年三月壬戌朔,日有食之。



 咸淳元年正月辛未朔,日有食之。三年五月丁亥朔,日有食之。四年十月戊寅朔,日有食之。六年三月庚子朔,日有食之。七年八月壬辰朔,日有食
 之。八年八月丙戌朔,日有食之。



 德祐元年六月庚子朔,日食,既,星見,雞鶩皆歸。明年,宋亡。



 日變



 周顯德七年正月癸卯,日既出,其下復有一日相掩,黑光摩蕩者久之。



 開寶七年正月丙戌,日中有黑子二。



 景德元年十二月甲辰,日有二影,如三日狀。三年九月戊申,日赤如赭。四年四月甲申,日無光。



 寶元二年十二月庚申,日赤如朱,逾二刻復。



 慶歷八年正月乙未,日赤無
 光。



 熙寧十年二月辛卯,日中有黑子如李,至乙巳散。



 元豐元年閏正月庚子,日中有黑子如李,至二月戊午散。十二月丙午,日中有黑子如李大,至丁巳散。二年二月甲寅,日中有黑子如李,至癸亥散。



 崇寧二年五月癸卯,日淡赤無光。三年十月壬辰,日中有黑子如棗大。



 政和二年四月辛卯,日中有黑子,乍二乍三,如慄大。八年十一月辛亥,日中有黑子如李大。



 宣和二年正月己未,日蒙蒙無光。五月己酉,日中有黑子如棗大。三年十二月
 辛卯,日中有黑子,如李大。四年二月癸巳,日蒙蒙無光。



 靖康元年閏十一月庚申,日赤如火,無光。



 建炎三年三月己卯,日中有黑子,至壬寅始消。



 紹興元年二月己卯,日中有黑子如李大,三日乃伏。六年十月壬戌,日中有黑子如李大,至十一月丙寅始消。七年二月庚子,日中有黑子如李大,旬日始消。四月戊申,日中有黑子,至五月乃消。八年二月辛酉,日中有黑子。十月乙亥,日中有黑子。十五年六月丙午,日中有黑氣往來。丁未,日中有
 黑子,日無光。



 乾道五年正月甲申,日色黃白,昏霧四塞。



 淳熙十二年正月癸巳,日中生黑子,大如棗。戊戌至庚戌,日中皆有黑子。十三年五月庚辰,日中生黑子,大如棗。



 紹熙四年十一月辛未,日中有黑子,至庚辰始消。



 慶元六年八月乙未,日中有黑子如棗大,至庚子始消。十二月乙酉,又生,至乙巳始消。



 嘉泰二年十二月甲戌,日中生黑子,大如棗。丙戌,始消。四年正月癸未,開禧元年四月辛丑,日中皆有黑子大如棗。



 嘉熙二年十月己巳,
 日中有黑子。



 德祐二年二月丁酉朔,日中有黑子,如鵝卵相蕩。



 日輝氣



 建隆元年迄開寶末,凡冠氣七,珥百,抱氣七,承氣六,赤黃氣三,黃白氣三,青氣二,纓一,暈一百五十六,半暈四十五,重暈五十九,重半暈七,交暈一十八,背氣二百三十一,紐氣戟氣三。



 太平興國迄至道末,凡冠氣一十八,戴氣三,抱氣一十三,珥七十七,承氣三,赤黃氣璚氣一,
 青氣三,暈五十九,半暈二十三,重暈一十二,交暈三,背氣四十四,紐氣三,戟氣一,直氣一十五。



 咸平元年迄乾興末,凡重輪二十四,彗一,五色氣一,冠氣二百六十六,珥四十一,戴氣一百九十七,抱氣五十七,承氣一百八十四,直氣七十七,光氣一,黃氣九,赤黃氣四,紫氣五,赤黃交氣二,赤黃綠碧氣二,青赤氣二十一,黃白氣一,黑氣二,白氣五,纓三,戟氣一,紐氣二,背氣二百九十九,暈一千二百三十一,半暈六百五十三,重暈二十七,交暈
 一十三。



 天聖元年訖嘉祐末,凡日黃曜有光一,輝氣一十九,龍鳳云一,慶雲二,五色雲八,紫黃雲五,赤黃雲一,紫雲二,青黃紫暈八百五十五,周暈二十六,重暈一十六,交暈五,連環暈一,珥八百四十七,冠氣一百四十,戴氣二百五十六,承氣一百,重承氣一,抱氣一十八,負氣一,背氣一百七,格氣二,直氣五,白虹貫日四,白氣如繩貫日並暈一。



 治平元年訖四年,凡五色雲八,輝氣一,暈一百二十八,周暈三,重暈十二,交暈二,珥八十九,冠氣
 一十一,戴氣三十九,承氣五,背氣三十三,白虹貫日一,白虹貫珥一。



 治平以後訖元豐末,凡日暈一千三百五十六,周暈二百七十七,重暈七十四,交暈四十九,連環暈一,珥八百八十二,冠氣四十二,戴氣二百七十一,承氣五十,抱氣二,背氣二百四十六,直氣二,戟氣一,纓氣五,璚氣一,白虹貫日九,貫珥三,五色雲二十六。



 自元豐八年三月五日訖元符三年正月十二日,暈五百二十八,周暈二百五十七,重暈六十八,交暈六十七,五色氣
 暈二,珥五百五十六,冠氣六十一,戴氣一百五十,承氣三十三,背氣一百七十四,直氣三,戟氣四,纓氣一,格氣五,白虹貫日一十六,貫珥一,五色雲十二。



 自元符三年正月訖靖康二年四月,凡日暈九,暈戴三,半暈一,暈珥背一,半暈重背一,暈纓一,珥背三,珥十三,暈珥七,冠氣七,暈背四,戴氣六,承氣二,抱氣四,背氣一十七,五色氣暈一,直氣四,環氣戴氣二,戟氣一,履氣二,半暈重履一,半暈再重一。



 建炎三年春、明年二月辛丑,白虹貫日。四
 年十一月癸卯,日生背氣。



 紹興元年正月壬戌,日生背氣。二年四月壬申、五月戊寅,日皆生戴氣。閏四月丙申,日生背氣。三年二月乙卯,日生戴氣。六月甲申朔,日生背氣。四年正月壬子,日生承氣。三月壬戌,日暈於軫。甲子,又暈於婁。辛未,又暈於胃,是日,日生抱氣。五月甲戌,日生背氣。六月壬辰,日暈於井。五年正月庚申,日有戴氣。六年二月丙寅,日暈於婁。三月戊寅,日暈於張。丁亥,又暈於胃。四月己亥,日生戴氣。庚子,復生,仍有承氣。十
 一月庚寅,日左右生珥並背氣。癸巳,日又生背氣。七年二月辛丑,氛氣翳日。八年二月辛巳,白虹貫日。二十一年閏四月壬申,日生赤黃暈周匝。二十七年二月壬寅,白虹貫日。二十八年二月戊申,日生赤黃暈周匝。二十九年正月癸酉,日連暈,上生青赤黃色戴氣,日左右生珥。三十一年四月戊辰,日生赤黃暈周匝。六月辛酉,日上暈外生赤黃色,有背氣。七月辛卯,日上暈外生背氣。



 隆興二年二月壬申,日生赤黃色暈,日左右生青赤黃
 珥。癸未,日生赤黃色暈周匝。三月庚戌,日生赤黃色暈周匝。六月甲子,日有戟氣。七月甲申朔,日生赤黃暈不匝,上生重暈,又生背氣及青珥。丁亥,日生重暈,上生青赤黃色背氣。癸卯,日生赤黃暈不匝,暈外生背氣,赤黃,兩頭向外曲。



 乾道元年六月丁未,日暈周匝,下暈外生格氣,橫在日下。二年二月庚辰,日左生赤黃色直氣長丈餘,及半暈背氣。三年三月丁巳,日暈於婁,外生赤黃承氣。四月辛卯,日暈,赤黃色周匝。五月戊戌朔,日赤黃
 暈周匝。甲辰,日下暈外有青赤黃承氣。六月丙子,日赤黃暈周匝。四年六月丁巳,日赤黃暈周匝。五年正月己巳,日生黃色戴氣承氣。六年三月丁丑,日暈不匝,下生承氣。閏五月壬辰,日半暈再重,生戴氣承氣。丁酉,日左生珥。八年六月辛丑,日暈不匝,左右生珥。壬寅,日暈周匝。丁未,日暈不匝,外生承氣,日下暈。九年二月丙子,日暈於奎。



 淳熙元年三月辛丑,日暈於胃。二年七月甲辰,日生背氣。三年二月庚子,日暈不匝,外日半暈再重。四
 年二月戊子,日暈不匝,日上連暈生戴氣,日下暈外生承氣。五年三月癸卯、四月乙酉、六月庚辰,皆日暈周匝。十二月乙未,日生兩珥,一戴氣。六年二月癸丑,日半暈再重。六月己丑,日暈周匝。十二月辛亥,日暈外生戴氣。八年正月己酉,日生戴氣,後日左生青赤黃珥。閏三月丙申,日暈周匝。七月己卯,日半暈外生背氣。十一年正月戊申,日半暈再重。十三年五月己卯,日暈周匝。十五年二月己卯,日半黃暈周匝。六月丙申,日上生青赤黃
 色背氣。十六年三月壬寅,日半暈再重。



 紹熙元年五月庚辰,日半暈再重。六月甲申,日生赤黃暈周匝。二年二月壬寅,日生戴氣,青赤黃色。三月辛未,日生青赤黃暈周匝。四月癸未,日生戴氣。七月庚申,日暈外生背氣。壬戌,日有背氣。四年二月癸亥,日暈周匝。十一月辛巳,日暈外生背氣。五年四月乙卯,日暈周匝。六月丙午,日上暈外生背氣。



 慶元元年正月丙辰,白虹貫日。二月辛巳,日上暈外生青赤黃背氣。四月己未,日生赤黃色格氣。
 二年五月己丑,日生背氣,其色青黃。



 嘉泰元年六月辛卯,日暈周匝。



 嘉定四年七月己卯巳初刻,日有赤黃暈不匝,至酉初後,日上暈外生青赤黃背氣。六年四月己卯,日赤黃暈周匝。七年三月壬申,日生赤黃暈,外有青赤黃承氣,後暈周匝。十一年二月丙辰,日有赤黃暈,白虹貫日。丙寅,日有戴氣。十五年二月己亥,日暈於婁,周匝,有承氣。十七年六月辛卯,日生背氣。



 寶慶三年十二月己酉,日旁有氣如珥。



 紹定三年二月丙申,日有背氣。
 四年七月己丑,日生承氣。五年三月丁酉,日生抱氣承氣。



 端平元年四月甲申,日生赤暈。六月戊子,日生赤黃暈,上下有格氣。二年六月戊寅,日有承氣。三年二月辛亥,日暈周匝。



 嘉熙元年二月己酉,日暈周匝。三月癸亥、七月壬申,日有背氣。四年二月丙申朔,日生背氣。辛丑,白虹貫日。



 淳祐元年二月戊寅,午後日暈。三年七月甲午,日生格氣。五年五月戊申,日生赤黃暈,外有背氣。六月甲子,日暈周匝。六年三月癸巳,日暈周匝,生珥氣。四
 月丁丑,日暈周匝。七年二月戊申,日暈周匝。八年六月己酉,日暈於井,赤黃,周匝。



 寶祐元年正月戊戌,日生戴氣。二年二月辛酉,日暈周匝。四年三月乙卯,日暈周匝。



 景定四年四月戊辰,日生赤黃暈。五年三月己丑,日暈於婁,周匝,赤黃,自午至申。六月庚午,日生赤黃暈。九月己丑,日生格氣。



 咸淳元年六月壬午,日生承氣。七年春三月辛巳,日暈,赤黃,周匝。



 月食



 開寶元年十一月庚寅,月食。二年十月戊子,月食。三年四月乙酉,月食。五年八月壬寅,月食。七年八月庚寅,月當食不食。



 太平興國二年六月甲辰,月食,既。十一月壬寅,月食。三年十月丙寅,月食,雲陰不見。五年八月乙卯,月食,既。



 雍熙元年正月丙寅,月食。二年七月戊午,月當食不食。四年五月丁丑,月食。



 端拱二年三月丁酉,月當食不食。



 淳化元年正月庚寅,月食。二年八月壬午,月食,既。三年正月癸卯,月食。八月丙子,月食,雲陰不見。五年
 六月乙未,月食。十二月癸巳,月食,既。



 至道元年六月己丑,月食,雲陰不見。十二月丁亥,月食。二年十月辛亥,月食。



 咸平元年十月庚子,月食。二年九月乙未,月食。三年二月壬戌,月食。八月庚申,月食。四年八月甲寅,月食。五年正月辛亥,月食。七月戊申,月食。六年正月甲辰,月食。七月壬寅,月食。



 景德元年十一月乙丑,月食。二年五月壬戌,月食。十月庚寅,月食。三年十一月癸丑,月食。四年五月辛亥,月食,雲陰不見。九月戊寅,月當食不食。



 大中
 祥符元年九月癸酉,月食。二年九月丁卯,月當食不食。三年閏二月甲子,月食。五年正月甲申,月食,陰翳不見。七月庚辰,月食。十二月丁丑,月食。八年十月辛卯,月食。九年四月己丑,月食,雲陰不見。



 天禧元年四月壬午,月食。十月庚辰,月食。三年二月壬寅,月食。四年八月癸巳,月食。



 天聖二年五月壬寅,月當食不食。四年五月戊午,月食。



 慶歷二年六月丁亥,月食。五年四月庚子,月食。九月戊戌,月食。六年九月壬辰,月食。



 皇祐二年七月庚子,
 月食。四年十一月丙辰,月食。五年十月辛亥,月食。



 至和二年九月庚午,月食。



 嘉祐元年八月甲子,月食,既。二年二月壬戌,月食。八月戊午,月食。三年閏十二月辛巳,月食。四年六月戊寅,月食。十二月己亥,月食,既。五年十二月己巳,月食。七年十月己丑,月食。八年十月癸未,月食,既。



 治平元年四月庚辰,月食。四年二月甲午,月食。



 熙寧元年七月乙酉,月食。二年閏十一月丁未,月食。三年五月乙巳,月當食,雲陰不見。四年五月己亥,月食。十一月
 丙戌,月食。六年三月戊午,月食。九月乙卯,月食。七年九月己酉,月食,既。九年正月壬申,月食,雲陰不見。十年正月丙寅,月食。七月癸亥,月食,雲陰不見。



 元豐元年正月庚申,月當食,有雲障之。六月戊午,月食。二年六月壬子,月當食,雲陰不見。三年十月甲戌,月食,雲陰不見。四年四月辛未,月食,既。十月己巳,月食。五年十月癸亥,月食。六年八月丁亥,月當食不食。七年二月乙酉,月食,雲陰不見。八月辛巳,月食,雲陰不見。八年八月丙子,月食,既。



 元祐元年十二月戊戌,月當食,雲陰不見。三年六月庚寅,月食,既。十二月丁亥,月當食,雲陰不見。四年五月甲申,月食,雲陰不見。五年五月戊寅,月食,雲陰不見。六年四月癸卯,月食,雲陰不見。七年三月戊戌,月食,既。八年九月己丑,月食,雲陰不見。



 紹聖三年七月癸卯,月食,雲陰不見。四年正月庚子,月食,雲陰不見。



 元符元年五月壬戌,月當食不食。二年五月丙辰,月食,既。十月甲寅,月食,既。三年十月戊申,月食。



 崇寧二年二月甲子,月食,既。
 八月辛酉,月食,既。三年二月己未,月食。八月丙辰,月食。四年十二月戊寅,月食。五年六月乙亥,月食。十二月壬申,月食,既。



 大觀三年十月丙戌,月食。四年四月甲申,月食,既。九月庚辰,月食,既。



 政和元年三月戊寅,月食。九月甲戌,月食。三年二月丁酉,月食。十月甲午,月食。四年正月辛卯,月食,既。六年十一月乙巳,月食。七年十一月乙亥,月食。



 重和元年五月丙甲,月食。



 宣和二年三月丙辰,月食。六年正月癸亥,月食。十二月戊午,月食,既。



 建炎三
 年二月壬午,月食於軫。



 紹興元年八月己卯,月當食,雲陰不見。二年二月丙子,月未當闕而闕,體如食,色黃白。



 七月甲戌,月食於室,既。三年七月戊辰,月食於危。四年十二月庚寅,月食於井。五年十一月乙酉,月食於井,既。六年五月辛巳,月食於南斗。十一月己卯,月當食,雲陰不見。八年三月辛丑,月當食,雲陰不見。九月丁酉,月當食,雲陰不見。九年九月壬辰,月食於胃,既。十二年七月丙午,月食,雲陰不見。十三年六月庚子,月食,既。十二月
 戊戌,月當食,雲陰不見。十四年六月甲午,月食於女。十五年五月己未,月當食,陰雲不見。十六年四月甲寅,月食。二十一年二月丙辰望,月當食,陰雲不見。二十五年五月壬戌望,月當食,以山色遮映,不見虧分。二十七年九月丁丑,月食。三十年正月甲午望,月當食,陰雲蔽之。



 隆興二年五月己亥,月當食,陰雲蔽之。



 乾道元年四月甲午,月當食,陰雲蔽之。四年二月丁未,月食,既。五年二月辛丑,月當食,陰雲不見。六年十一月辛酉,月當食,陰
 雲不見。八年六月壬子,月當食,陰雲不見。



 淳熙元年四月壬申,月當食,陰雲不見。二年四月丙寅,月食於房,既。九月癸亥,月當食,雲陰不見。三年三月庚申,月當食,雲陰不見。五年二月己卯,月當食,雲陰不見。六年正月甲戌,月食,既。八年十一月丁亥,月食。九年十一月辛巳,月食。十年五月己卯,月食。十二年三月戊戌,月食。九月乙未,月當食,雲陰不見。十三年三月壬辰,月當食,陰雲不見。八月庚寅,月食,既。十四年八月甲申,月當食,陰雲不
 見。十六年十二月辛丑,月當食,雲陰不見。



 紹熙元年六月丁酉,月當食,陰雲不見。十一月乙未,月當食,陰雲不見。二年六月壬辰,月當食,陰雲不見。三年四月乙巳,月當食,陰雲不見。五年九月癸卯,月當食,雲陰不見。



 慶元二年八月壬戌,月食。三年七月己未,月食,既。四年七月庚戌,月食。六年五月庚午,月當食,陰雲不見。



 嘉泰二年五月己未,月當食,陰雲不見。三年三月癸未,月當食,陰雲不見。



 開禧元年三月壬申,月當食,陰雲不見。閏八月
 己巳,月當食,陰雲不見。三年正月壬辰,月食。七月戊子,月食。



 嘉定元年二月丙戌,月當食,陰雨不見。十二月庚辰,月食。二年六月丁丑,月食。三年十一月己亥,月食。五年十月戊子,月食。七年二月庚戌,月食。八月丁未,月食。八年八月辛丑,月食,既。九年二月己亥,月當食,雲陰不見。閏七月乙未,月當食,雲陰不見。十年十二月戊午,月食。十一年六月乙卯,月食。十二月壬子,月食,既。十二年五月庚戌,月當食,既,雲陰不見。十一月丙午,月食。十三
 年五月甲辰,月當食,雲陰不見。十四年十月丙寅,月食。十五年三月癸亥,月當食於氐,既,雲陰不見。十六年三月丁巳,月當食,雲陰不見。



 寶慶元年正月丁丑,月食。七月癸酉,月食,陰雨不見。二年七月戊辰,月食,陰雨不見。



 紹定元年十一月甲申,月食。二年十一月己卯,月食。四年四月庚午,月食。五年三月乙未,月食。六年二月庚寅,月食。



 端平二年十二月癸卯,月食。三年十二月丁酉,月食。



 嘉熙元年六月乙未,月食。三年四月甲寅,月食。四年
 四月戊申,月食。



 淳祐元年九月庚子,月食。四年十月癸丑,月食。五年七月戊申,月食。七年五月丁卯,月食。八年十月己丑,月食。十一年三月乙亥,月食。九月壬申,月食。十二年八月丙寅,月食。



 寶祐二年閏六月丙戌,月食。三年十二月丁丑,月食。五年十月丁酉,月食。六年四月癸巳,月食。十月辛卯,月食。



 開慶元年四月戊子,月食。十月乙酉,月食。



 景定二年七月甲戌,月食。



 咸淳二年六月丁丑,月食。十一月甲辰,月食。四年七月癸亥,月食。五年九
 月丁巳,月食。六年三月乙卯,月食。九月辛亥,月食。九年正月戊辰,月食。十二月壬戌,月食。



 月變



 天禧四年四月乙酉,西南方兩月重見。



 月輝氣



 建隆元年迄開寶末,凡珥一十九,輝氣一十三,暈二十九,重暈一,半暈一十四,交暈二,紐氣二。



 太平興國元年迄至道末,凡冠氣一,珥六,輝氣五,赤氣二,抱氣一,暈八,
 半暈三,背氣一。



 咸平元年迄乾興末,凡重輪三,珥一百二十,冠氣十二,暈氣十二,承氣八,抱氣三,戴氣九,赤黃氣十七,五色氣十一,青赤氣二,黃紅氣一,暈三百九十四,五色重暈二十,背氣一。



 天聖元年訖嘉祐末,凡揚光一,光芒氣一,紅光輝氣一,輝氣五,五色輝氣一,暈二百五十七,周暈三十三,交暈四,連環暈一,珥七十二,冠氣五,戴氣一十三,承氣五,背氣一,白虹貫月一,黃虹貫月二。



 治平元年訖四年,凡五色輝氣一,五色暈氣一,暈
 五十一,珥一十五,冠氣一,戴氣四,背氣二。四年訖元豐末,凡五色輝氣十一,五色暈氣六,暈四百二十三,周暈二百四十七,交暈二,珥一百三十四,冠氣七,戴氣五十,承氣五,背氣一十,白虹貫月五,貫珥一。



 自元豐八年三月五日至元符三年正月十二日,凡五色暈氣九,暈八十九,周暈二百五十一,重暈一,交暈三,珥一百三,冠氣七,戴氣二十七,背氣八,白虹貫月二,貫珥一。



 自元符三年正月迄靖康二年四月,凡暈五,暈珥二,五色暈五,珥二,
 暈冠一,交暈一,重暈一,白虹貫月一,五色雲一。



 建炎四年十月己卯,暈生五色。



 紹興二年四月壬申,暈於軫。五月乙亥,暈生五色。四年六月壬午,暈生珥。五年正月戊午,暈於東井。



 乾道元年三月丁巳,暈周匝,著太微西扇星。三年五月壬午,生黃白暈,左右珥。四年三月壬寅,生黃白暈周匝。五年二月庚子,黃白暈周匝。



 嘉泰三年七月壬午,白虹如半暈貫月中。



 淳祐六年閏四月辛丑,暈五重。十月辛丑,生珥。八年二月戊子,暈生黃白。



 寶祐四
 年三月乙卯,四月庚午,景定三年十月甲子,十二月辛酉,四年二月戊午,暈皆周匝。



 德祐二年正月己卯,暈東井。



\end{pinyinscope}