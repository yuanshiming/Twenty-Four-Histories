\article{志第七十三 禮二十三(賓禮五 附錄)}

\begin{pinyinscope}

 群
 臣上表儀宰臣赴上儀朝省集議班位臣僚上馬之制臣僚呵引之制



 群臣上表儀。《通禮》:守宮設次於朝堂,文東武西,相對為
 首設中書令位於群臣之北。禮曹掾舉表案入,引中書令出,就南面立。禮部郎中取表授中書令,令即受表入奏。



 其禮:凡正、至不受朝,及邦國大慶瑞、上尊號請舉行大禮,宰相率文武群臣暨諸軍將校、蕃夷酋長、道釋、耆老等詣東上閣門拜表,知表官跪授表於宰臣,宰臣跪授於閣門使,乃由通進司奏御。凡有答詔,亦拜受於閣門,獲可,奏者奉表稱賀。其正、至,樞密使率內班拜表長春殿門外,亦閣門使受之。



 又西京留守拜表儀制:留司
 百官每五日一上表起居,質明,並集長壽寺立班,置表於案,再拜以遣。其春、秋賜服及大慶瑞並如之。或令分司官繼詣行在,或止驛付南京留司,約用此制。若巡幸,東京則留司百官亦五日一上表起居,並集大相國寺。



 其制:群臣詣閣拜奏者,首云文武百僚具官臣某等言;常奏御者,止云臣某言,並稱尊號,已有功臣爵邑者具之;狀奏者,前後列銜,不稱尊號,亦云功臣爵邑。其外,又有書疏、奏札、榜子之類。



 乾德二年,令有司詳定表首。太
 常禮院言:「僕射南省官品第二,太子三師官品第一,品位雖高,而南省上臺為尊,合以僕射充首。若專以品秩為定,則諸行侍郎品第四,列於諸司三品卿監之上,不可以品序為準。按唐貞元六年詔,每有慶賀及諸臣上表,並合上公為首,如三公闕,以令、僕行之。中書、門下列貢章表,則僕射是百僚師長,難同宮僚之例。」詔百官集議。翰林學士陶穀等曰:「按唐制:上臺、東宮並是廷臣,當時左右僕射、侍中、中書令為正宰相。貞觀末,帶同中
 書門下三品者方為宰相。今僕射既非宰相,合在大子三師之下,理固不疑。若以宮僚非廷臣,既宰相豈當兼領?今若先二品而後一品,升後列而退前班,紊其等威,事恐非順。請以太子三師為表首。」竇儀等曰:「東宮三師為表首,論討故典,實皆無據。左右僕射當為表首者,其事有六:按《六典》,尚書為百官之本,今自一品至六品常參官,皆以尚書省官為首,則僕射合為表首,一也。又唐制,上表無上公,即尚書令僕以下行之,其嗣王合隨宗正,
 若有班位,合依王品,則嗣王雖一品,不得為表首,二也。僕射位次三公,合為表首,三也。況僕射為百僚師長,東宮三師非師長之任,四也。晉天福中詔,謝賀上表,上公行之,如闕,即令僕射行之,五也。立制之班,卑者先入後出,尊者後入先出。今東宮一品立定,僕射乃入,僕射既退,東宮一品乃出,且在兩省之後,六也。」



 詔從儀等議,以僕射為表首焉。



 宰臣赴上儀。《開元禮》有任官初上相見之儀。宋制:凡牧
 守赴上,多仍州府舊禮。臺省之制,宰相、親王、使相正衙謝訖,出文德殿便門至西廊,堂後官、兩省雜事迎參;至中書便門,兩省官迎班;升都堂,與送上官對揖見任侍中、中書令、同平章事者,降階,又與送上官對拜訖,分東、西升坐於床。兩省雜事讀案,堂後官接案。搢笏頂筆判署,凡三道:一,司天監壽星見;二,開封府嘉禾合穗;三,澶州黃河清。並判準,始謝送上官,訖,三司使、學士、兩省官、待制、三司副使升堂展賀。百官先班中書門外,上事官降階,百官入,直
 省官通班贊致賀,歸後堂,與參知政事、樞密副使、宣徽使相見,會食訖,退。



 建隆三年,中書、門下言:「準唐天成元年詔故事,藩鎮帶平章事,合於都堂視事,刊石以記官族,輸禮錢三千貫。近年頗隳曩制。自今藩鎮帶平章事者,輸禮錢五百千,刻石記歲月。其錢以給兩省公用,望舉行之。」詔自今宰相及樞密使兼平章事、侍中、中書令者,輸禮錢三百千,藩鎮五百千,刻石以記如舊制。增秩者不再輸,舊相復入者輸如其數。



 乾德二年,置參知政
 事,就宣徽院赴上,而樞密使、副止上事於本廳。後以曹彬兼侍中,為樞密使,特令赴中書上事。



 大中祥符中,詔自今宰相官至僕射者,並於中書都堂赴上,不帶平章事亦令赴上。有司上儀注,宰相用常儀。僕射本省上日,郎中、員外班迎於都堂門內,尚書丞、郎於東廊階上稍近班迎揖,金吾將軍升殿展拜賀,禮生贊引,主事讀案。見任中書樞密使相、前任中書門下並不赴,餘如宰相之儀。上訖,與本省御史臺四品、兩省五品、諸司三品以
 上會食。



 右僕射王旦充玉清昭應宮使,有司按故事,宰相凡有吉慶,百官皆班賀。詔以未葺攸司,其班賀權罷。旦赴上修宮所,特賜會,丞、郎、三司副使以上悉預。自是宮觀使副上日皆賜會作樂。



 天禧初,太保、平章事王旦為太尉。國朝以來,三公不兼宰相,無赴上儀。特詔有司詳定,就尚書省赴上,百官班迎,宰相而下悉集。御史大夫、中丞、知雜、三院御史皆僚屬送上,判案三道。中丞以上,即京府尹、赤縣令、諸曹、節度、刺史、皇城、宮苑使悉集。
 翰林學士入院日賜設,惟學士、中書舍人赴坐。又資政、侍讀、侍講、龍圖閣學士、直學士兼秘書監並赴上。秘閣及兩省五品以上任三館學士、判館、修撰者,皆賜設焉。



 朝省集議,前代不載其儀。宋初,刑政典禮之事當集議者,先下詔都省,省吏以告當議之官,悉集都堂,設左、右丞於堂之東北,南向;御史中丞於堂之西北,南向;尚書、侍郎於堂東廂,西向;兩省侍郎、常侍、給事、諫舍於堂之西廂,東向;知名表郎官於堂之東南,北向;監議御史於
 堂之西南,北向。又設左右司郎中、員外於左、右丞之後,三院御史於中丞之後,郎中、員外於尚書、侍郎之後,起居、司諫、正言於諫舍之後。如有僕射、御史大夫,即於左右丞,中丞之前。如更有他官,即諸司三品於侍郎之南,東宮一品於尚書之前,武班二品於諫舍之南,皆重行異位。卑者先就席。左、右丞升廳,省吏抗聲揖群官就坐,知名表郎官以所議事授所司奉詣左、右丞,左、右丞執卷讀訖授中丞、中丞授於尚書、侍郎,以次讀訖,復授知
 名表郎官。將畢,左、右丞奉筆叩頭揖群官,以一副紙書所議事節署字於下,授四坐。監議御史命吏告云:「所見不同者請不署字。」以官高者為表首。如止集本省官,坐如常儀,其知名表郎官、監議御史坐仍北向。惟僕射以上得乘馬至都堂,他官雖同平章,事亦止屏外。



 明道二年,尚書議莊獻、莊懿太后升祔,省官帶內外制、兼三司副使承例移文不赴。



 監議御史段少連以為官帶近職,一時之選,宜有建明,不當反自高異。乃奏議事不集以
 違制論。從之。



 集賢校理趙良規言:「國朝故事,令敕儀制,別有學士、知制誥、待制、三司副使著位,視品與前朝異,固無在朝敘職、入省敘官之說。若全不論職,則後行員外郎兼學士,在朝立丞、郎上,入省居比、駕下;知制誥、待制入朝與侍郎同列,入省分廁散郎;員外郎任三司副使、郎中任判官,在三司為參佐,入本省為正員。所以舊來議事,集尚書省官,帶職者不赴。別詔三省悉集,則及大小兩省;內朝官悉集,則及學士、待制、三司副使;更集
 他官,則諸司三品、武官二品,各次本司長官。故事,尚書省官帶知制誥,中書省奏班簿,是於尚書省、御史臺了不著籍,故有絕曹之語。又凡定學士、舍人、兩省著位,除先後入外,若有升降,皆特稟朝旨,豈有在朝、入省迭為高下?」御史臺、禮院詳定,久不決。



 判禮院馮元等曰:「會議之文,由來非一,或出朝廷別旨,或循官司舊規。故集本省者,即南省官;集學士、兩省、臺官者,容有兩制、給舍、中丞;集學士、臺省及諸司四品以上者,容有卿、監;集文武
 百官者,容有諸衛。蓋謀事有小大,集官有等差,率系詔文,乃該餘職。少連以太常易名之細,考功復議之常,誤謂群司普當會席,列為具奏,嬰以嚴科,遂使絕曹清列,還入本行,分局常員,略無異等。請臣僚擬謚,止集南省官屬,或事緣體大,臨時敕判,兼召三省、臺、寺,即依舊例。」御史臺言:「今尚書省官任兩制者,系臺省之籍,無坐曹之實。論職官之言,正為絕曹者設,豈可受祿則系官定奉,議事則絕曹為辭?況王旦、王化基、趙安仁、晁迥、杜鎬、
 楊億皆嘗預議於尚書省。故相李昉為主客郎中、知制誥日,屢經都省議事。又議大事,僕射、御史大夫入省,唯僕射至廳下馬,於今行之,所以重本省也。故都堂會議,列狀以品,就坐以官,忽此更張,恐非通理。」



 禮官吳育曰:「兩奏各有未安。尚書省制度雖崇,亦天子之有司,在朝廷既殊班列,入有司輒易尊卑,是以朝省為彼我、官職分二事也。兩制近職,若有事議而云絕班不赴,非所以求至當。且知制誥中書省奏班簿,是謂絕班。翰林學士
 亦知制誥,不絕班簿。此因循之制,非確據也。縱絕班有例,而絕官無聞,一人命書,三省連判,而都無所系,止為奉錢,豈命官之禮?今取典故中最明一事,足以質定。祥符五年僕射上事儀:絕班之官,別頭贊引,不與本省官同在迎班。請凡會議,省官帶近職者,別作一行而坐,自為序列,非以相壓。若招兩制、臺省、諸司、諸衛官畢集,則各從其類,自作一行,書議如其位次。」



 詔尚書省議事,應帶職官三司副使以上並不赴,如遇集議大事,令赴,別
 設坐次。



 是歲,紫宸、垂拱殿刊石為百官表位。三司使,內朝班學士右,獨立石位;門外,亦班其上。



 熙寧二年,御史臺、太常禮院詳定臣僚禦路上馬之制:近上臣僚及北使到闕,並於御路上行馬。中書樞密院執政官、宣徽院、御史中丞、知雜御史、左右金吾、攝事官清道者,導從呵止依舊式,其三司副使以上亦許出節。正任觀察使以上與合出節臣僚,並許自宣德門外至天漢橋北禦路上行馬,如從賀出入及宗室內庭諸宮
 院車騎,並不在此限。



 御史臺又言:「舊制:百官臺參、辭謝臣僚於朝堂,先赴三院御史幕次,又赴中丞幕次,得以體按老疾。今止於御史廳一員對拜,不惟有失舊儀,兼恐不能公共參驗。請如舊制朝堂拜揖,遇放常朝,即詣御史臺。」已而詔宰臣、親王、使相、兩府、宣徽使,遇入樞密院門,許至從南第二門外上下馬。又詔:宰臣上馬,樞密院次之,諸司又次之,左、右丞上下馬處並同兩省侍郎。



 御史臺言:「左丞蒲宗孟、右丞王安禮賀僕射上尚書省,
 於都堂下馬。按左、右丞上下馬於本廳。請付有司推治。」安禮爭論上前,以為今日置左、右丞為執政官,不應有厚薄。左、右丞於都堂上下馬自此始。



 尋詔執政官退朝上馬,宰臣於樞密院,餘於隔門外。都堂聚議退,左丞於門下侍郎廳,右丞於中書侍郎廳。品官詣尚書省上下馬依雜壓,大中大夫以上就第一貯廊,監察御史以上就過道,諸六曹尚書、侍郎即大中大夫以上就本廳,監察御史以上就客位,餘並過道門外。



 政和朝參臣僚上
 馬次序:俟皇城門開,樞密入,次三省執政官,次一品二品文臣觕、六曹侍郎、殿中監、開封尹、大司成、侍從官、兩省,次百官,御史臺編欄以次入。



 其宰相罷政,韓琦以司徒、節度判相州,曾公亮以司空、節度為集禧觀使,王安石以觀文殿大學士、吏部尚書知江寧府。曹佾以中書令、節度充景靈宮使,韓絳以觀文殿大學士、吏部侍郎知大名府,致仕太師文彥博來朝,其大朝會班位儀物如之。吳育以觀文殿大學士、吏部尚書為西太一宮使,大
 朝會綴中書、門下班而已。自是,舊相按例重輕以特旨行之。



 治平四年,御史臺言:「慶歷中,有詔詳定武臣出節呵引之制:節度使在尚書下,三節。節度觀察留後在諸行侍郎下,兩節。觀察使在中書舍人下,諸衛大將軍、防禦,團練使在大卿監下,內客省使比諸司大卿,景福殿使比將作監,引進使比庶子,在防禦使上,以上各一節。諸州刺史、諸衛將軍在少卿監下,宣慶、四方館使比少卿,宣
 政、昭宣、閣門使比司天監少監,諸衛將軍上,皇城使以下諸司使比郎中,客省、引進、閣門副使比員外郎,樞密都承旨在司天少監下、閣門使上,副都承旨在閣門使下,樞密副承旨、諸房副承旨在諸司使下,以上並兩人呵引。當時已施行矣,而皇祐編敕刪去此制,請復舉行。」



\end{pinyinscope}