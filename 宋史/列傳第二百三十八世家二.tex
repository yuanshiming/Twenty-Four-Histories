\article{列傳第二百三十八世家二}

\begin{pinyinscope}

 西蜀孟氏



 西蜀孟昶,初名仁贊,及僭位改焉。其先邢州龍岡人。父知祥,事後唐武皇,武皇以弟之子妻之,是為瓊華長公
 主。同光初,知祥為太原尹、知留守事。三年,平蜀。四年,以知祥為劍南西川節度副大使、知節度事。明宗即位,命知祥討平東川,知祥自領兩川節度,明宗即以授之。長興四年,封蜀王,許行墨制。五年,閔帝立,乃稱帝於蜀,改元明德,時清泰元年也。事具五代史。昶母李氏,本莊宗嬪御,以賜知祥,天祐十六年己卯十一月,生昶於太原。初,知祥鎮西川,不及以族行。天成元年,奏遣衙校迎家太原,明宗因令部送長公主及昶與所生母至蜀。公主
 以長興三年卒。



 知祥初署昶西川節度行軍司馬,僭號,以昶為檢校太保、同平章事、崇聖宮使、東川節度。知祥疾,立為皇太子,權監軍國。明德元年七月,知祥卒,昶襲位,年始十六,止稱明德年號,委政於趙季良、張知業、李仁罕等。二年,尊其母李氏為皇太后。四年,改元廣政。後以事誅仁罕、知業,乃親政事。十三年,加號「睿文英武仁聖明孝皇帝」。



 晉末,秦州節度使何建、鳳州防禦使石奉頵俱以城降昶。時契丹亂華,漢祖起幷門,中土蝗旱連
 歲,昶益自大,開貢部,行郊祀禮,自此君臣奢縱。及周世宗克秦、鳳,昶始懼,放還先所獲濮州刺史胡立,致書世宗,稱大蜀皇帝,且言家世邢臺,願敦鄉里之分。世宗怒其無禮,不答。昶愈不自安,乃於劍門、夔、峽多積芻粟,增置師旅。用度不足,遂鑄鐵錢。禁境內鐵,凡器用須鐵為之者,置場鬻之,以專其利。



 立其子玄喆為太子,用王昭遠、伊審徵、韓保正、趙崇韜等分掌機要,總內外兵柄。母李氏謂昶曰:「吾嘗見莊宗跨河與梁軍戰,又見爾父在
 并州捍契丹及入蜀定兩川,當時主兵者非有功不授,故士卒畏服。如昭遠者,出於微賤,但自爾就學之年,給事左右;又保正等皆世祿之子,素不知兵,一旦邊疆警急,此輩有何智略以禦敵?高彥儔是爾父故人,秉心忠實,多所經練,此可委任。」昶不能遵用其言。



 及太祖下荊、楚,昶欲遣使朝貢,昭遠等固止之。太祖詔蜀之邸吏、將卒先在江陵者並放還,仍給賜錢帛以遣。乾德二年,昶遣孫遇、楊蠲、趙彥韜為諜至京師。彥韜潛取昶與并州
 劉鈞蠟丸帛書以告,其書云:「早歲曾奉尺書,遠達睿聽。丹素備陳於翰墨,歡盟已保於金蘭。洎傳吊伐之嘉音,實動輔車之喜色。尋於褒、漢,添駐師徒,只待靈旗之濟河,便遣前鋒而出境。」先是,太祖已有西伐意而未發,及覽書,喜曰:「吾用師有名矣。」即命忠武軍節度王全斌充鳳州路行營前軍兵馬都部署,武信軍節度、侍衛步軍都指揮使崔彥進充副都部署,樞密副使王仁贍充都監,龍捷右廂都指揮使史延德充馬軍都指揮使,虎捷右
 廂都指揮使張萬友充步軍都指揮使,隴州防禦使張凝充先鋒都指揮使,左神武大將軍王繼濤充濠砦使,內染院使康延澤充馬軍都監,翰林副使張煦充步軍都監,供奉官田仁朗充濠砦都監,殿直鄭粲充先鋒都監,步軍都軍頭向韜充先鋒都軍頭,寧江軍節度、侍衛馬步軍都指揮使劉廷讓充歸州路行營前軍兵馬副都部署,內客省使、樞密承旨曹彬充都監,客省使武懷節充戰棹部署,龍捷左廂都指揮使李進卿充步軍都
 指揮使,前階州刺史高彥暉充先鋒都指揮使,右衛將軍白廷誨充濠砦使,御廚副使朱光緒充馬軍都監,儀鸞副使折彥贇充步軍都監,八作副使王令嵒充先鋒都監,供奉官郝守濬充濠砦都監,馬步軍都軍頭楊光美充戰棹左右廂都指揮使,供奉官藥守節充戰棹左廂都監,殿直劉漢卿充戰棹右廂都監,率禁兵三萬人、諸州兵二萬人分路討之。詔令孫遇等指畫江山曲折之狀,及兵砦戍守之處道里遠近,俾畫工圖之,以授全
 斌等。因謂曰:「西川可取否?」全斌等對曰:「臣等仗天威,遵廟算,刻日可定。」龍捷右廂都校史延德前奏曰:「西川一方,倘在天上,人不能到,固無可奈何。若在地上,以今之兵力,到即平矣。」上壯其言,謂之曰:「汝等果敢如此,我何憂乎!」又謂全斌等曰:「凡克城砦,止籍其器甲芻糧,悉以錢帛分給戰士。」



 及兵至,昶遣王昭遠、趙崇韜、韓保正、李進等來拒戰。昭遠等相繼就擒,昶大懼,出金帛募兵,令其子玄喆統之,李廷珪、張惠安為其副,以守劍門。玄喆
 素不習武,廷珪、惠安皆庸懦無識。玄喆離成都,但攜姬妾、樂器及伶人數十輩,晨夜嬉戲,不恤軍政。至綿州,聞宋師已破劍門,遂遁歸東川,所過焚廬舍倉稟而去。昶益惶駭,問計於左右。有老將石斌,對以宋師遠來,勢不能久,請聚兵固守以老之。昶曰:「吾父子以豐衣美食養士四十年,及遇敵,不能為我東向發一矢。今若固壘,何人為我效命?」



 三年正月,昶遣其通奏伊審徵齎表詣全斌請降,且言:「中外骨肉二百餘人,有親年幾七十,願終
 甘旨之養,免賜睽離之責,則祖宗血食庶獲少延。」末援劉禪、陳叔寶故事以請封號。全斌等既受其降,遣馬軍都監康延澤先以百騎入城見昶,諭以恩信,留三日,盡封府庫而還。



 昶又遣其弟仁贄詣闕上表言:



 :「先臣受命唐室,建牙蜀川,因時事之變更,為人心之擁迫。先臣即世,臣方丱年,猥以童昏,繆承餘緒。乖以小事大之禮,闕稱藩奉國之誠,染習偷安,因循積歲。所以上煩宸算,遠發王師,勢甚疾雷,功如破竹。顧惟懦卒,焉敢當鋒?尋束手
 以云歸,止傾心而俟命。



 :今月七日,已令私署通奏使、宣徽南院使伊審徵奉表歸降,以緣路寇攘,前進不得。臣尋更令兵士援送,至十一日,尚恐前表未達,續遣供奉官王茂隆再齎前表。至十二日以後,相次方到軍前,必料血誠,上達睿聽。臣今月十九日,已領親男諸弟,納降禮於軍門,至於老母諸孫,延餘喘於私第。



 :陛下至仁廣覆,大德好生,顧臣假息於數年,所望全軀於此日。今蒙元戎慰恤,監護撫安,若非天地之垂慈,豈見軍民之受
 賜!臣亦自量過咎,尚切憂疑,謹遣親弟詣闕奉表,待罪以聞。」



 太祖詔曰:



 :「朕以受命上穹,臨制中土,姑務保民而崇德,豈思右武以佳兵?至於臨戎,蓋非獲已。矧惟益部,僻處一隅,靡思僭竊之愆,輒肆窺覦之志,潛結幷寇,自啟釁端。爰命偏師,往申吊伐,靈旗所指,逆壘自平。



 :朕嘗中宵憮然,兆民何罪!屢馳馹騎,嚴戒兵鋒,務宣拯溺之懷,以盡招攜之禮。而卿果能率官屬而請命,拜表疏以祈恩,託以慈親,保其宗祀,悉封庫府,以待王師。追咎改
 圖,將自求於多福;匿瑕含垢,當盡滌於前非。朕不食言,爾無他慮。」



 昶乃舉族與官屬由峽江而下,至江陵,上遣皇城使竇思儼迎勞之。四月初,昶與母至襄漢,復遣使齎詔賜茶藥。所賜詔不名,仍呼昶母為國母。昶將至,命太宗勞於近郊。昶率子弟素服待罪闕下,太祖御崇元殿,備禮見之,賜昶襲衣、玉帶、黃金鞍勒馬、金器千兩、銀器萬兩、錦綺千段、絹萬匹;又賜昶母金器三百兩、銀器三千兩、錦綺千匹、絹千匹;子弟及其官屬等襲衣、金玉
 帶、鞍勒馬、車乘、器幣有差;又遣使分詣江陵、鳳翔賜其家屬錢帛,疾病者給以醫藥。即日宴於大明殿。先是,詔有司於右掖門外,臨汴水起大第五百間以待昶,供帳悉備,至是賜之,又為其官屬各營居第。



 翌日,詔曰:



 :「伯禹導川,黑水本梁州之域;《河圖》括象,岷山直井絡之墟。是曰坤維,素為王土。屬中原多故,四海群飛,遂剖裂於山河,競僭竊於位號。朕削平㝢縣,載整皇綱,復周、漢之舊疆,寵綏群后;采唐、虞之大訓,協和萬邦。六年於茲,百揆
 時敘。禮樂征伐之柄,盡出朝廷;蠻夷山海之君,咸修職貢。一昨順長庚而授律,法時雨以興師,先申誕告之文,以慰徯來之眾。



 :咨爾偽蜀主孟昶,克承餘緒,保據一隅,擅正朔以自尊,歷歲時而滋久。屬王師致討,察天道之惡盈,體此綏懷,思於效順,盡率群吏,降於軍門。抗手疏以陳誠,伏天閽而請命。是用昭示大信,盡滌疵瑕,度越彝章,升於崇秩。冠紫微之近署,以奉內朝;剪鶉首之奧區,為之封邑。率從異數,式洽殊私。爾宜欽承,往踐厥
 位。可開府儀同三司、檢校太師兼中書令、秦國公,給上鎮節度使奉祿。餘官除拜有差。」



 昶數日卒,年四十七。太祖廢朝五日,素服發哀於大明殿。賜尚書令,追封楚王,諡「恭孝」,賻布帛千匹,葬事官給。後數日,其母李氏亦卒。初,李氏隨昶至京師,太祖數命肩輿入宮,謂之曰:「母善自愛,無戚戚懷鄉土,異日當送母歸。」李氏曰:「使妾安往?」太祖曰:「歸蜀爾。」李氏曰:「妾家本太原,倘得歸老幷土,妾之願也。」 時晉陽未平,太祖聞其言大喜,曰:「俟平劉鈞,即
 如母所願。」因厚加賜賚。及昶卒,不哭,以酒酹地曰:「汝不能死社稷,貪生以至今日。吾所以忍死者,以汝在爾。今汝既死,吾何生焉!」因不食,數日卒。太祖聞而傷之,賻贈加等。令鴻臚卿范禹偁護喪事,與昶俱葬洛陽,詔發奉義甲士千人護送。



 七月,正衙備禮冊命昶,其文曰:



 :「維乾德三年,歲次乙丑,七月己巳朔,二十四日戊子,皇帝若曰:「咨爾故檢校太師兼中書令、秦國公孟昶,冊贈之典,所以彰世祚而紀勳伐,繼絕之義,所以旌異域而表來庭。
 苟匪全功,寧兼二者。國家乘乾撫運,括地開圖。稽至德於勳、華,體深仁於湯、禹。既定壺關之亂,復剪淮夷之凶,暨荊及衡,洗蕩逋穢。以為君人之道,先德而後刑;王者之師,有征而無戰。兵威震疊,寰宇來同。以至薄伐兩川,徂征三峽。



 :惟爾昶襲乃堂構,據有巴庸,而能祗畏皇靈,保全宗緒,知機識變,委順圖全。馳子牟魏闕之心,奉伯禹塗山之會。朕自聞獻款,良切虛懷。舟車欣至止之初,邸第錫非常之制。封崇異數,祈保永年。景命不融,奄
 然殂謝。



 :於戲!爾有及親之孝,特異常倫;爾有達上之情,所期終養。何高穹之不祐,與幽壤之同歸!斯朕所以當宁興悲,徹縣永歎。詢于史氏,申命禮官,今遣使起復雲麾將軍、檢校太傅、右神武統軍、兼御史大夫、上柱國、平昌縣開國伯食邑七百戶孟仁贄持節,冊贈爾為尚書令,仍追封楚王。於戲!式備哀榮,載光簡牒。南宮峻秩,全楚大邦,併示追崇,夐超彝制。始終之分,朕無愧焉!」



 仍贈昶墳莊一區,給守墳人米千石,錢五萬。



 初,昶在蜀專務
 奢靡,為七寶溺器,他物稱是。每歲除,命學士為詞,題桃符,置寢門左右。末年,學士幸寅遜撰詞,昶以其非工,自命筆題云:「新年納餘慶,喜節號長春。」以其年正月十一日降,太祖命呂餘慶知成都府,而「長春」乃聖節名也。又昶襲位後,民質錢取息者,將徙居,必署其門曰:「召主收贖。」周世宗平淮甸,克關南,即議討蜀而未果,至太祖乃平之。



 昶三子:玄喆、玄珏、玄寶。玄寶先卒,僭贈遂王。昶弟:仁贄、仁裕、仁操。



 昶既降,寧江軍節度、同平章事伊審徵,
 檢校太尉兼侍中韓保正,山南西道節度、同平章事王昭遠,工部侍郎幸寅遜,武信軍節度、保寧軍都巡檢使李廷珪來闕下。審徵授靜難軍節度,昭遠授左領軍衛大將軍,寅遜授右庶子,廷珪授右千牛衛上將軍,韓保正未授官卒。保正、昭遠、廷珪,川中各有田宅,詔各賜錢三百萬。又成都人王處瓊,少孤,有司籍其金寶,昶降,輦送闕下。太祖聞之,令計其直還焉。



 玄喆,字遵聖,幼聰悟,善隸書。年十四,僭封秦王、檢校太尉、同平章事、判六軍
 諸衛事。嘗自書姚崇《口箴》,刻諸石。昶賜以銀器、錦彩。廣政二十一年,領武德軍節度。二十四年,加兼侍中。二十五年,立為皇太子。宋師將至,以玄喆為元帥,精卒萬餘,旌旗用文繡,以錦綢其杠。是日微雨,玄喆慮沾濕,令解去。俄雨止,復旆之,旌幟數千皆倒繫杠上,識者異之。及聞劍門陷,遂奔東川。數日,棄軍遯歸。



 入朝,與昶同日宣制檢校太尉、泰寧軍節度。昶卒,賜玄喆羊五百口、酒五百壺。玄喆獻馬二百匹、白玉水晶鞍勒副之。移鎮貝州,
 在鎮十餘年,亦有治跡。太平興國初,移鎮定州。三年,加開府儀同三司。四年,從平太原,就命為鎮州駐泊兵馬鈐轄。又從征幽州,率所部攻城之西面。會班師,遣與軍器庫使藥可瓊、深州刺史念金鏁、左龍武將軍趙延進、殿前都虞候崔翰、四方館使梁迥、翰林使杜彥圭帥兵歸屯定州。俄契丹入寇,玄喆與諸將校破之徐河。以功封滕國公,入為左龍武軍統軍,判右金吾衛仗。未幾,知滑州。淳化初,病,求換瀕淮一小郡養疾。移知滁州,卒,年
 五十五。贈侍中。



 初,玄喆在貝州,凡民輸稅者皆令出商算,規其餘羨,以備留使之用,人頗苦之。景德中,都官員外郎孔揆使河北,表論其事,詔除之。有子十五人:隆記、隆詁、隆說、隆詮,並進士及第。



 玄珏初封王,與玄喆並日封拜,仍檢校太保。少端敏。常侍昶射,雙箭連中的,昶奇之,賜錢三十萬。時玄珏方就學,為選起居舍人陳鄂為教授。至是,自陳願以錢賜鄂,昶嘉而許焉。鄂嘗仿唐李瀚《蒙求》、高測《韻對》為《四庫韻對》四十卷以獻,玄珏益賞
 之。廣政二十三年,玄珏領閬州保寧軍節度。久之,加檢校太傅。歸朝,為千牛衛上將軍。乾德五年,遷右神武統軍,代玄喆判金吾衛仗。太平興國九年,出為宋、曹、兗、鄆都巡檢,又改右屯衛上將軍。淳化元年四月,復為右神武統軍。六月,出知滑州。三年,卒。



 仁贄,字忠美,初為左威衛將軍同正。廣政十三年,封雅王、檢校太尉。二十年,領閬州保寧軍節度。二十四年,加檢校太尉。及昶降,遣仁贄奉表詣闕,太祖召見廣德殿,
 賜襲衣、玉帶、鞍勒馬。俄授右神武統軍。丁母憂,起復,領大同軍節度、西京都巡檢使。開寶四年,卒,年四十四,贈太子太師。



 仁裕,字鳴謙,初為左威衛將軍同正,與仁贄同日封彭王、檢校太傅。廣政二十年,領黔州武泰軍節度。二十四年,加檢校太尉。歸朝,授檢校太傅、右監門衛上將軍,遷右羽林軍。開寶三年,卒,年四十四,贈太子太傅。



 仁操,初為右領軍衛將軍同正,與仁贄同日封嘉王、檢
 校太傅。廣政二十一年,領果州永寧軍節度。嘗侍昶射於梔子園,仁操連中的者三。二十四年,加檢校太尉。尤奉釋氏,深究其理。歸朝,授右監門衛上將軍,累遷右龍武統軍。雍熙三年,卒。



 伊審徵,字申圖,并州人。父延瓌,隨知祥入蜀。知祥僭位,以女妻延瓌,僭封崇華公主。延瓌歷陵、嘉、眉三州刺史。審徵幼以孝聞,母病,割股肉啖之。以父任,歷蜀州刺史、雲安榷鹽使。廣政十四年,高延昭求解機務,急召為通
 奏使、知樞密院事。久之,領蜀州刺史。秦、鳳興師,命檢校城砦,俄領武泰軍節度。選其子崇度尚公主。又改寧江軍節度、同平章事,與王昭遠俱掌機務。昶事無大小,一以咨之。常自以康濟經略為己任。屬宋師入境,審徵首奉降表詣軍前。昭遠時統軍,敗走。時人笑之。



 審徵歸朝,授靜難軍節度。乾德六年,移鎮延安。開寶末入朝,改右屯衛上將軍。太平興國二年,判右金吾衛仗。雍熙五年,卒,年七十五。



 韓保正,字永吉,潞州長子人。父昭運,從知祥入蜀。及知祥僭號,署珍州刺史。保正初事知祥為押衙,及僭位,以為豐德庫使兼廣義庫使、眉州刺史、樞密副使。復刺漢州,拜宣徽北院使。會鳳翔侯益歸款,以保正為北路行營都監,以圖岐陽。時晉昌趙贊亦謀歸蜀,為王景崇所逼,棄城東奔。偽將李廷珪先退師,保正次陳倉,與大將張虔釗、龐福誠謀議不叶,益亦中變,遂還成都。俄為雄武節度,領兵出新關,至隴州,漢兵固守,保正無功而還。
 復屯雄武。廣政十四年,赴成都,其親吏楊虔範訟保正不法,昶令斬虔範,釋保正不問。俄改夔州寧江軍節度。李昊讓度支,以保正代之。未幾,加宣徽南院使、山南節度、左衛聖步軍節度指揮使,遷奉鸞肅衛馬步軍都指揮使,又選其子崇遂尚主。



 宋初,荊南高繼沖納土,昶聞之,以保正為峽路都指揮制置使,屯夔州,以經畫邊事。遷檢校太尉兼侍中。聞太祖將加兵,以保正為山南節度、興元武定緣邊諸砦屯駐都指揮使。及王全斌至,保
 正棄興元,保西縣。王師進圍之,保正懦懼不敢出,遣人依山背城結陣以自固,為史延德所破。保正以麾下遁,延德追擒之,送全斌。全斌驛置闕下,太祖召升殿勞問,賜袍笏、金帶、茵褥、鞍勒馬,仍賜甲第。未及命官而卒,贈右千牛衛上將軍。



 王昭遠,益州成都人。幼孤貧。年十三,依東郭僧智諲為童子。知祥鎮蜀,一日飯僧於府署,昭遠持巾履從智諲,得入。時昶方就學,知祥見昭遠聰慧,留給事昶左右。昶
 嗣位,以昭遠為卷簾使、茶酒庫使。會樞密使王處回出知梓州,昶以樞密事權太重,乃以昭遠及普豐庫使高延昭為通奏使、知樞密院事,機務一以委之,府庫財帛恣其取不問。加領眉州刺史,出為永平軍節度。不數月,會昭武李繼勳以目疾不能視事,議以閑地處之,昭遠遽以永平讓繼勳。歲餘,為夔州寧江軍節度。昶母常言昭遠不可用,昶不從。未幾,兼領山南西道節度、同平章事。及入謝,求解通奏職,遂以左街使張仁貴為副使、知
 樞密以代之。



 昭遠好讀兵書,頗以方略自許。宋師入境,昶遣昭遠與趙崇韜率兵拒戰。始發成都,昶遣其宰相李昊等餞郊外。昭遠酒酣,攘臂曰:「是行也,非止克敵,當領此二三萬雕面惡少兒,取中原如反掌耳。」及行,執鐵如意指麾軍事,自方諸葛亮。將至漢源,聞劍門已破,昭遠股慄,發言失次。崇韜布陣將戰,昭遠據胡床,皇恐不能起。俄崇韜敗,乃免胄棄甲走投東川,匿倉舍下,悲嗟流涕,目盡腫,惟誦羅隱詩云:「運去英雄不自由」,俄為追
 騎所執,送闕下,太祖釋之,授左領軍衛大將軍。廣南平,奉使交阯。開寶八年,卒。



 趙崇韜,并州太原人。父廷隱,隨知祥入蜀。廷隱拳勇有智略,知祥麾下無及者。東川董璋襲成都,廷隱大破之。璋奔歸,為部下所殺,知祥遂有其地。及僭號,以廷隱總親軍,為衛聖諸軍馬步軍指揮使,累遷至太師、中書令、宋王。卒,諡「忠武」。



 崇韜驍果有父風。昶自置殿直四番,取將家及死事孤子為之,始命李仁罕子繼宏、趙季良子
 元振、張知業子繼昭、侯洪實子令欽及崇韜,分為都知領之。後累遷至客省使。周世宗克秦、鳳,將入蜀境,為崇韜拒退。歷左右衛聖步軍都指揮使。選其子文亮尚公主。加領洋州、武定軍節度、山南武定緣邊諸砦都指揮副使。漢源之戰,獨策馬先登,及蜀軍敗,猶手擊殺十數人,為宋師所擒。



 高彥儔,并州太原人。父暉,宣威軍使。彥儔從知祥入蜀,累歷軍校,為昭武軍監押。昶嗣位,遷邛州刺史,改馬步
 軍使。會漢兵入大散關,克安都砦,彥儔以所部先進。漢人燒砦毀閣遁去,彥儔盡銳追之,復其砦而還。未幾,彥儔領趙州刺史。俄為奉鑾肅衛都指揮副使,改右驍銳馬軍都指揮使,加光聖馬軍都指揮使,真拜源州武定軍節度。



 周顯德初,向訓攻鳳州,昶令彥儔出兵解圍。未至,聞敗軍於唐倉,因潰歸。判官趙玭閉關不納,以城歸朝廷。彥儔遁歸成都,昶不之罪,以為右奉鑾肅衛都指揮使,改功德使。



 廣政二十二年,出授夔州寧江軍都巡
 檢制置招討使,加宣徽北院事、利州昭武軍節度。及宋師至,彥儔謂副使趙崇濟、監軍武守謙曰:「北軍涉遠而來,利在速戰,不如堅壁以待之。」守謙不從,獨領麾下以出。時大將劉廷讓頓兵白帝廟西,遣騎將張廷翰等引兵與守謙戰豬頭鋪,守謙敗走。廷翰等乘勝登其城,廷讓率大軍繼至。彥儔以所部將出拒戰,宋師已乘城而入。彥儔惶駭失次,不知計所出。判官羅濟勸令單騎歸成都,彥儔曰:「我昔已失天水,今復不能守夔州,縱不忍
 殺我,亦何面目見蜀人哉!」濟又勸其降,彥儔曰:「老幼百口在成都,若一身偷生,舉族何負?吾今日止有死耳!」即解符印授濟,具衣冠望西北再拜,登樓縱火自焚。後數日,廷讓得其骨煨燼中,以禮收葬。初,昶母語昶「惟彥儔可任」,及是,果能死難。



 趙彥韜,興州順政人,為本州義軍裨校。乾德中,昶遣與興國軍討擊使孫遇及楊蠲為諜至都下,彥韜潛取昶與并州蠟丸帛書以告,因言伐蜀之狀。太祖並赦遇、蠲,
 出師西討,並以為鄉導。克興州,以為本州馬步軍都指揮使。蜀平,遷本州刺史,移澧州。性凶率,所為不法,部民有訴被盜劫財物,鞫之不實,彥韜手殺之,探取其心肝。民家詣闕訴冤,太祖怒,令杖配蔡州。



 龍景昭,夔州奉節人。少有武勇,事蜀為義軍裨校,以功遷戰棹都將。久之,擢為施州刺史。乾德中,諸將伐蜀,分兵由峽路入,將壓其境。景昭率官吏以牛酒犒宋師,迎入城。太祖聞之,甚悅。蜀平,即授永州刺史。秩滿入朝,改
 右千牛衛將軍。開寶三年,卒。昶之入朝也,為左羽林將軍、景昭弟處瑭等四人隨行,卒於道。太祖憫之,以其男補供奉官殿直。



 幸寅遜,蜀人。初仕昶為茂州錄事參軍。昶好擊毬,雖盛暑不已。寅遜上章極諫,深被賞納。遷新都令,拜司門郎中、知制誥、中書舍人。出知武信軍府,加史館修撰,改給事中,預修《前蜀書》,拜翰林學士,加工部侍郎,判吏部三銓事,領簡州刺史。



 隨昶歸朝,授右庶子。嘗上疏諫獵,太
 祖嘉之,召見賜帛。開寶五年,為鎮國軍行軍司馬。罷職,年九十餘,尚有仕進意,治裝赴闕,未登路而卒。



 李廷珪,并州太原人。七歲隸知祥帳下,後從入蜀。知祥僭號,補軍職,累遷奉鑾肅衛都虞候。賞拔階州之功,領眉州刺史。會圖取鳳翔,令廷珪領兵二萬出子午谷赴援。始出谷,聞趙贊為王景崇所逼,遂退軍。以廷珪權知興元。俄召歸,授捧聖控鶴都指揮使,領蜀州刺史,拜雅州永平軍節度,改右光聖都指揮使,領山南節度,改閬
 州保寧節度、護聖控鶴都指揮使。



 周師攻秦州,以廷珪為北路行營都統。秦、成、階三州竟為周所取,廷珪奉章待罪,昶釋之,以為左右衛聖諸軍馬步軍都指揮使。分衛聖、光聖步騎為左右十軍,以武定節度呂彥珂為之使,並隸廷珪總領之。時論以廷珪不能救援階州,不當復總兵柄,廷珪亦自陳求解,許之。俄加兼侍中、蜀成都巡檢使,改遂州武信軍節度,領本鎮及保寧軍都巡檢使。



 王全斌之下劍關也,昶遣廷珪與其太子玄喆將兵
 來拒宋師,至綿、漢與全斌遇,狼狽而還。玄喆與廷珪謀,所經州縣盡焚其儲蓄。及全斌等入成都,行營都監王仁贍案籍詰所在軍須,廷珪懼,以告馬軍都監康延澤。延澤曰:「王公志在聲色,苟得其所欲,則置而不問矣。」廷珪素儉約,不畜妓樂,遂求於姻戚家,得女妓四人,復假貸金帛直數百萬以遺仁贍,繇是獲免。歸闕,為右千牛衛上將軍。乾德五年,卒。



 先是,廷珪及王昭遠、韓保正川中各自有田宅,昶降後奉表上獻,詔各賜錢三百萬以
 償其直。



 李昊,字穹佐,自言唐相紳之後。祖乾祐,建州刺史。父羔,容管從事。昊生於關中,幼遇唐末之亂,隨父避地至奉天。值昭宗遷洛,岐軍攻破奉天,父及弟妹皆為亂兵所殺。是時年十三,獨得免,遂流寓新平十數年。會劉知俊領岐軍圍州城,昊踰城出,為候騎所得。知俊與語,甚器之,置於門下,以其女妻之。



 知俊歸蜀,偽署遂州武信軍節度,以昊為從事。王建使知俊出師,令昊主留務。會建
 殺知俊,昊亦罷職。王衍襲偽位,授彭州導江令,歷中書舍人、翰林學士。岐軍之難,昊母獨無恙。至是十九年,昊仕獨顯達,乃遣心膂張金、王彥間道迎其母。昊請告境上奉迎,衍賜以金勒名馬。昊至青泥嶺見母,母撫昊首號慟,哀感行路。



 蜀亡入洛,明宗授昊檢校兵部郎中。詔西川孟知祥、三川制置使趙季良同於榷鹽、度支、戶部院間授昊一職,昊至蜀,久無所授。會知祥奏季良為西川節度副使,昊辭歸洛,知祥始辟為觀風推官,遷掌書
 記。知祥稱帝,擢為禮部侍郎、翰林學士。



 昶立,領漢州刺史,遷兵部侍郎,出知武德軍府,加承旨。昶嘗欲命昊二子官,昊固讓,且言:「遂州判官石欽若、蘇涯,前蜀時,同在劉知俊幕下,願回授欽若等子。」昶嘉歎,許之,仍授昊二子官。俄加尚書左丞,拜門下侍郎兼戶部尚書、同平章事、監修國史。因請置史官,乃以給事中郭廷鈞、職方員外郎趙元拱為修撰,雙流令崔崇構、成都主簿王中孚為直館。



 俄加昊左僕射。昶令就知祥真容院圖文武三
 品以上於東西廊,以昊有參佐功,特畫於殿內。自知祥領蜀,凡章奏書檄皆出昊手,至是集為百卷曰《經緯略》以獻,昶齎以珍器、錦彩。俄命判度支戶部。



 廣政十四年,修成昶《實錄》四十卷。昶欲取觀,昊曰:「帝王不閱史,不敢奉詔。」丁母憂,裁百日,起復。俄修《前蜀書》,命昊與趙元拱、王中孚及左諫議大夫喬諷、給事中馮侃、知制誥賈玄珪、幸寅遜、太府少卿郭微、右司郎中黃彬同撰,成四十卷上之。以判使辦集,封趙國公。俄加司空,領遂州武信
 軍節度,出判鹽鐵,加弘文館大學士,修奉太廟禮儀使。



 昶嘗召四孫,悉授太子司儀郎舍人,並賜緋。昊又改判度支使。其子孝連尚昶女鳳儀公主,累遷太常少卿、資州刺史。長子孝逢,給事中。



 蜀平,隨昶入朝,太祖優待之,拜昊工部尚書,賜第。以孝逢為膳部郎中,孝連為將作少監。親屬乘舟自峽下,至夷陵,妻死,昊聞,悲愴成疾而卒,年七十三。贈右僕射。



 昊前後仕蜀五十年。昶之世,位兼將相,秉利權,資貨歲入鉅萬,奢侈尤甚,後堂妓妾曳
 羅綺數百人。昶與江南李景通好,遣其臣趙季札至江南,購得李紳武宗朝入相制書,還以遺昊。昊結綵樓置其中,盡召成都聲妓,昊朝服前迎歸私第,大會賓客宴飲,所費無算。以帛二千匹謝季札。



 初,王衍降莊宗,昊草其表;昶之降也,其表亦昊所為。蜀人潛署其門曰「世修降表李家」,見者哂之。有集二十卷,目為《樞機應用集》。孝連後至司農少卿。昊孫德鏻至國子博士,德錞進士及第。



 毋守素,字表淳,河中龍門人。父昭裔,偽蜀宰相、太子太師致仕。守素弱冠起家,偽授秘書郎,累遷戶部員外郎、知制誥,真拜中書舍人、工部侍郎,出為雲安榷鹽使。召見其二子克溫、克恭,並賜緋;以次子克恭尚昶女,授檢校水部員外郎。



 廣政二十年,拜工部尚書。時昭裔判鹽鐵,衰老不能親職,委其務於判官李光遠,事多留滯。昶患之,命守素代判使務。父子相代,時頗榮之。俄改判度支,領彭州刺史,又判鹽鐵。



 守素奉親頗勤至,雖隆暑暮
 歸,必朝服執簡以申昏定之禮。蜀亡入朝,授工部侍郎,籍其蜀中莊產茶園以獻,詔賜錢三百萬以充其直,仍賜第於京城。歲餘,為兄之子岳州司法正己訟其居父喪娶妾免,正己亦坐奪一官。開寶初,起為國子祭酒。



 太祖征河東,命權知趙州。及平嶺表,移知容州,兼本管諸州水陸轉運使。先是,部民有逋賦者,或縣吏代輸,或於兼並之家假貸,則皆納其妻女以為質。守素表其事,即日降詔禁止。六年,卒,年五十三。



 昭裔性好藏書,在成都
 令門人勾中正、孫逢吉書《文選》、《初學記》、《白氏六帖》鏤板,守素齎至中朝,行於世。大中祥符九年,子克勤上其板,補三班奉職。次子克恭,尚昶女鑾國公主,仕為光祿少卿,歸宋,至左監門衛將軍。



 歐陽迥,益州華陽人。父珏,通泉令。迥少事王衍,為中書舍人。後唐同光中,蜀平,隨衍至洛陽,補秦州從事。知祥鎮成都,迥復來入蜀。知祥僭號,以為中書舍人。廣政十二年,拜翰林學士。明年,知貢舉、判太常寺。遷禮部侍郎,
 領陵州刺史,轉吏部侍郎,加承旨。二十四年,拜門下侍郎兼戶部尚書、平章事、監修國史。嘗擬白居易諷諫詩五十篇以獻,昶手詔嘉美,賚以銀器、錦彩。



 從昶歸朝,為右散騎常侍,俄充翰林學士,就轉左散騎常侍。嶺南平,議遣迥祭南海,迥聞之稱病不出。太祖怒,罷其職,以本官分司西京。開寶四年,卒,年七十六。贈工部尚書。



 迥性坦率,無檢操,雅善長笛。太祖常召於偏殿,令奏數曲。御史中丞劉溫叟聞之,叩殿門求見,諫曰:「禁署之職,典司
 誥命,不可作伶人之事。」上曰:「朕嘗聞孟昶君臣溺於聲樂,迥至宰司尚習此技,故為我所擒。所以召迥,欲驗言者之不誣也。」溫叟謝曰:「臣愚不識陛下鑒戒之微旨。」自是不復召。迥好為歌詩,雖多而不工,掌誥命亦非所長。但在蜀日,卿相以奢靡相尚,迥猶能守儉素,此其可稱也。



\end{pinyinscope}