\article{志第一 天文一}

\begin{pinyinscope}

 儀象極度黃赤
 道中星土圭



 夫不言而信,天之道也。天於人君有告戒之道焉,示之以象而已。故自上古以來,天文有世掌之官,唐虞羲、和,夏昆吾,商巫咸,周史佚、甘德、石申之流。居是官者,專察天象之常變,而述天心告戒之意,進言於其君,以致交修之儆焉。《易》曰:「天垂象,見吉兇,聖人則
 之。」又曰:「觀乎天文,以察時變。」是也。然考《堯
 典》,中星不過正人時以興民事。夏仲康之世,《胤征》之篇:「乃季秋月朔,辰弗集於房。」然後日食之變昉見於《書》。觀其數羲、和以「俶擾天紀」、「昏迷天象」之罪而討之,則知先王克謹天戒,所以責成於司天之官者,豈輕任哉!



 箕
 子《洪範》論休咎之徵曰:「王省惟歲,卿士惟月,師尹惟日。」「庶民惟星,星有好風,星有好雨。」《禮記》言體信達順之效,則以天降膏露先之。至於周《詩》,屢言天變,所謂「旻天疾威,敷於下土」,又所謂「雨無其極,傷我稼穡」,「正月繁霜,我心憂傷」,以及「彼月而微,此日而微」,「燁燁震電,不寧不令」。孔子刪《詩》而存之,以示戒也。他
 日
 約魯史而作《春秋》,則日食、星變屢書而不為煩。聖人以天道戒謹後世之旨,昭然可睹矣。於是司馬遷《史記》而下,歷代皆志天文。第以羲、和既遠,官乏世掌,賴世以有專門之學焉。然其說三家:曰周髀,曰宣夜,曰渾天。宣
 夜先絕,周髀多差,渾天之學遭秦而滅,洛下閎、耿壽昌晚出,始物色得之。故自魏、晉以至隋、唐,精天文之學者犖犖名世,豈世難得其人歟!



 宋之初興,近臣如楚昭輔,文臣如竇儀,號知天文。太宗之世,召天下伎術有能明天文者,試隸司天臺;匿不以聞者幻罪論死。既而張思訓、韓顯符輩以推步進。其後學士大夫如沈括之議,蘇頌之作,亦皆底於幻眇。靖康之變,測驗之器盡歸金人。高宗南渡,至紹興十三年,始因秘書丞嚴抑之請,命太史
 局重創渾儀。自是厥後,窺測占候蓋不廢焉爾。寧宗慶元四年九月,太史言月食於晝,草澤上書言食於夜。及驗視,如草澤言。乃更造《統天歷》,命秘書正字馮履參定。以是推之,民間天文之學蓋有精於太史者,則太宗召試之法亦豈徒哉!今東都舊史所書天文禎祥、日月薄蝕、五緯凌犯、彗孛飛流、暈珥虹霓、精祲雲氣等事,其言時日災祥之應,分野休咎之別,視南渡後史有詳略焉。蓋東都之日,海內為一人,君遇變修德,無或他諉。南渡
 土宇分裂,太史所上,必謹星野之書。且君臣恐懼修省之餘,故於天文休咎之應有不容不縷述而申言之者,是亦時勢使然,未可以言星翁、日官之術有精粗敬怠之不同也。今合累朝史臣所錄為一志,而取歐陽修《新唐書》、《五代史記》為法,凡徵驗之說有涉於傅會,咸削而不書,歸於傳信而已矣。



 儀象



 歷象以授四時,璣衡以齊七政,二者本相因而成。故璣
 衡之設,史謂起於帝嚳,或謂作於宓犧。又云璇璣玉衡乃羲、和舊器,非舜創為也。漢馬融有云:「上天之體不可得知,測天之事見於經者,惟有璣衡一事。璣衡者,即今之渾儀也。」吳王蕃之論亦云:「渾儀之制,置天梁、地平以定天體,為四游儀以綴赤道者,此謂璣也;置望筒橫簫於游儀中,以窺七曜之行,而知其躔離之次者,此謂衡也。」若六合儀、三辰儀與四游儀並列為三重者,唐李淳風所作。而黃道儀者,一行所增也。如張衡祖洛下閎、耿
 壽昌之法,別為渾象,置諸密室,以漏水轉之,以合璇璣所加星度,則渾象本別為一器。唐李淳風、梁令瓚祖之,始與渾儀並用。



 太平興國四年正月,巴中人張思訓創作以獻。太宗召工造於禁中,逾年而成,詔置於文明殿東鼓樓下。其制:起樓高丈餘,機隱於內,規天矩地。下設地輪、地足;又為橫輪、側輪、斜輪、定身關、中關、小關、天柱;七直神,左搖鈴,右扣鐘,中擊鼓,以定刻數,每一晝夜周而復始。又以木為十二神,各直一時,至其時則自執辰
 牌,循環而出,隨刻數以定晝夜短長。上有天頂、天牙、天關、天指、天抱、天束、天條,布三百六十五度,為日、月、五星、紫微宮、列宿、斗建、黃赤道,以日行度定寒暑進退。開元遺法,運轉以水,至冬中凝凍遲澀,遂為疏略,寒暑無準。今以水銀代之,則無差失。冬至之日,日在黃道表,去北極最遠,為小寒,晝短夜長。夏至之日,日在赤道里,去北極最近,為小暑,晝長夜短。春秋二分,日在兩交,春和秋涼,晝夜平分。寒暑進退,皆由於此。並著日月象,皆取仰
 視。按舊法,日月晝夜行度皆人所運行。新制成於自然,尤為精妙。以思訓為司天渾儀丞。



 銅候儀,司天冬官正韓顯符所造,其要本淳風及僧一行之遺法。顯符自著經十卷,上之書府。銅儀之制有九:



 一曰雙規,皆徑六尺一寸三分,圍一丈八尺三寸九分,廣四寸五分,上刻周天三百六十五度,南北並立,置水臬以為準,得出地三十五度,乃北極出地之度也。以釭貫之,四面皆七十二度,屬紫微宮,星凡三十七坐,一百七十有五星,四時常
 見,謂之上規。中一百一十度,四面二百二十度,屬黃赤道內外官,星二百四十六坐,一千二百八十九星,近日而隱,遠而見,謂之中規。置臬之下,繞南極七十二度,除老人星外,四時常隱,謂之下規。



 二曰游規,徑五尺二寸,圍一丈五尺六寸,廣一寸二分,厚四分,上亦刻周天,以釭貫於雙規巔軸之上,令得左右運轉。凡置管測驗之法,眾星遠近,隨天周遍。



 三曰直規,二,各長四尺八寸,闊一寸二分,厚四分,於兩極之間用夾窺管,中置關軸,令其
 游規運轉。



 四曰窺管,一,長四尺八寸,廣一寸二分,關軸在直規中。



 五曰平準輪,在水臬之上,徑六尺一寸三分,圍一丈八尺三寸九分,上刻八卦、十干、十二辰、二十四氣、七十二候於其中,定四維日辰,正晝夜百刻。



 六曰黃道,南北各去赤道二十四度,東西交於卯酉,以為日行盈縮、月行九道之限。凡冬至日行南極,去北極一百一十五度,故景長而寒;夏至日在赤道北二十四度,去北極六十七度,故景短而暑。月有九道之行,歲匝十二辰,正
 交出入黃道,遠不過六度。五星順、留、伏、逆行度之常數也。



 七曰赤道,與黃道等,帶天之紘以隔黃道,去兩極各九十一度強。黃道之交也,按經東交角宿五度少,西交奎宿一十四度強。日出於赤道外,遠不過二十四度。冬至之日行鬥宿,日入於赤道內,亦不過二十四度,夏至之日行井宿;及晝夜分,炎涼等。日、月、五星陰陽進退盈縮之常數也。



 八曰龍柱,四,各高五尺五寸,並於平準輪下。



 九曰水臬,十字為之,其水平滿,北辰正。以置四隅,
 各長七尺五寸,高三寸半,深一寸。四隅水平則天地準。



 唐貞觀初,李淳風於浚儀縣古岳臺測北極出地高三十四度八分,差陽城四分。今測定北極高三十五度以為常準。



 熙寧七年七月,沉括上《渾儀》、《浮漏》、《景表》三議。



 《渾儀議》曰:



 五星之行有疾舒,日月之交有見匿,求其次舍經劘之會,其法一寓於日。冬至之日,日之端南者也。日行周天而復集於表銳,凡三百六十有五日四分日之幾一,而謂之歲。周天之體,日別之謂之度。度之離,其數有
 二:日行則舒則疾,會而均,別之曰赤道之度;日行自南而北,升降四十有八度而迤,別之曰黃道之度。度不可見,其可見者星也。日、月、五星之所由,有星焉。當度之畫者凡二十有八,而謂之舍。舍所以絜度,度所以生數也。度在天者也,為之璣衡,則度在器。度在器,則日月五星可摶乎器中,而天無所豫也。天無所豫,則在天者不為難知也。



 自漢以前,為歷者必有璣衡以自驗跡。其後雖有璣衡,而不為歷作。為歷者亦不復以器自考,氣朔星
 緯,皆莫能知其必當之數。至唐僧一行改《大衍歷法》,始復用渾儀參實,故其術所得,比諸家為多。



 臣嘗歷考古今儀象之法,《虞書》所謂璇璣玉衡,唯鄭康成粗記其法,至洛下閎制圓儀,賈逵又加黃道,其詳皆不存於書。其後張衡為銅儀於密室中,以水轉之,蓋所謂渾象,非古之璣衡也。吳孫氏時王蕃、陸績皆嘗為儀及象,其說以謂舊以二分為一度,而患星辰稠穊,張衡改用四分,而復椎重難運。故蕃以三分為度,周丈有九寸五分寸之
 三,而具黃赤道焉。績之說以天形如鳥卵小橢,而黃、赤道短長相害,不能應法。至劉曜時,南陽孔定制銅儀,有雙規,規正距子午以象天;有橫規,判儀之中以象地;有時規,斜絡天腹以候赤道;南北植幹,以法二極;其中乃為游規、窺管。劉曜太史令晁崇、斛蘭皆嘗為鐵儀,其規有六,四常定,以象地,一象赤道,其二像二極,乃是定所謂雙規者也。其制與定法大同,唯南北柱曲抱雙規,下有縱衡水平,以銀錯星度,小變舊法。而皆不言有黃道,
 疑其失傳也。唐李淳風為圓儀三重:其外曰六合,有天經雙規、金渾緯規、金常規。次曰三辰,轉於六合之內,圓徑八尺,有璇璣規、月游規,所謂璇璣者,黃、赤道屬焉。又次曰四游,南北為天樞,中為游筒可以升降游轉,別為月道,傍列二百四十九交以攜月游。一行以為難用,而其法亦亡。其後率府兵曹梁令瓚更以木為游儀,因淳風之法而稍附新意,詔與一行雜校得失,改鑄銅儀,古今稱其詳確。至道中,初鑄渾天儀於司天監,多因斛蘭、
 晁崇之法。皇祐中,改鑄銅儀於天文院,姑用令瓚、一行之論,而去取交有失得。



 臣今輯古今之說以求數象,有不合者十有三事:



 其一,舊說以謂今中國於地為東南,當令西北望極星,置天極不當中北。又曰:天常傾西北,極星不得居中。臣謂以中國規觀之,天常北倚可也,謂極星偏西則不然。所謂東西南北者,何從而得之?豈不以日之所出者為東,日之所入者為西乎?臣觀古之候天者,自安南都護府至浚儀太嶽臺才六千里,而北極
 之差凡十五度,稍北不已,庸詎知極星之不直人上也?臣嘗讀黃帝《素書》:「立於午而面子,立於子而面午,至於自卯而望酉,自酉而望卯,皆曰北面。立於卯而負酉,立於酉而負卯,至於自午而望南,自子而望北,則皆曰南面。」臣始不諭其理,逮今思之,乃常以天中為北也。常以天中為北,則蓋以極星常居天中也。《素問》尤為善言天者。今南北才五百里,則北極輒差一度以上;而東西南北數千里間,日分之時候之,日未嘗不出於卯半而入
 於酉半,則又知天樞既中,則日之所出者定為東,日之所入者定為西,天樞則常為北無疑矣。以衡窺之,日分之時,以渾儀抵極星以候日之出沒,則常在卯、酉之半少北。此殆放乎四海而同者,何從而知中國之為東南也?彼徒見中國東南皆際海而為是說也。臣以謂極星之果中、果非中,皆無足論者。彼北極之出地六千里之間所差者已如是,又安知其茫昧幾千萬里之外邪?今直當據建邦之地,人目之所及者,裁以為法。不足為法
 者,宜置而勿議可也。



 其二曰:紘平設以象地體,今渾儀置於崇臺之上,下□敢日月之所出,則紘不與地際相當者。臣詳此說雖粗有理,然天地之廣大,不為一臺之高下有所推遷。蓋渾儀考天地之體,有實數,有準數。所謂實者,此數即彼數也,此移赤彼亦移赤之謂也。所謂準者,以此準彼,此之一分,則準彼之幾千里之謂也。今臺之高下乃所謂實數,一臺之高不過數丈,彼之所差者亦不過此,天地之大,豈數丈足累其高下?若衡之低昂,
 則所謂準數者也。衡移一分,則彼不知其數幾千里,則衡之低昂當審,而臺之高下非所當恤也。



 其三曰:月行之道,過交則入黃道六度而稍卻,復交則出於黃道之南,亦如之。月行周於黃道,如繩之繞木,故月交而行日之陰,則日為之虧;入蝕法而不虧者,行日之陽也。每月退交二百四十九周有奇,然後復會。今月道既不能環繞黃道,又退交之漸當每日差池,今必候月終而頓移,亦終不能符會天度,當省去月環。其候月之出入,專以歷
 法步之。



 其四,衡上、下二端皆徑一度有半,用日之徑也。若衡端不能全容日月之體,則無由審日月定次。欲日月正滿上衡之端,不可動移,此其所以用一度有半為法也。下端亦一度有半,則不然。若人目迫下端之東以窺上端之西,則差幾三度。凡求星之法,必令所求之星正當穿之中心。今兩端既等,則人目游動,無因知其正中。今以鉤股法求之,下徑三分,上徑一度有半,則兩竅相覆,大小略等。人目不搖,則所察自正。



 其五,前世皆以
 極星為天中,自祖□恆以璣衡窺考天極不動處,乃在極星之末猶一度有餘。今銅儀天樞內徑一度有半,乃謬以衡端之度為率。若璣衡端平,則極星常游天樞之外;璣衡小偏,則極星乍出乍入。令瓚舊法,天樞乃徑二度有半,蓋欲使極星游於樞中也。臣考驗極星更三月,而後知天中不動處遠極星乃三度有餘,則祖□恆窺考猶為未審。今當為天樞徑七度,使人目切南樞望之,星正循北極樞里周常見不隱,天體方正。



 其六,令瓚以辰刻、
 十干、八卦皆刻於紘,然紘平正而黃道斜運,當子、午之間,則日徑度而道促;卯、酉之際,則日迤行而道舒。如此,辰刻不能無謬。新銅儀則移刻於緯,四游均平,辰刻不失。然令瓚天中單環,直中國人頂之上,而新銅儀緯斜絡南北極之中,與赤道相直。舊法設之無用,新儀移之為是。然當側窺如車輪之牙,而不當衡規如鼓陶,其旁迫狹,難賦辰刻,而又蔽映星度。



 其七,司天銅儀,黃、赤道與紘合鑄,不可轉移,雖與天運不符,至於窺測之時,先
 以距度星考定三辰所舍,復運游儀抵本宿度,乃求出入黃道與去極度,所得無以異於令瓚之術。其法本於晁崇、斛蘭之舊制,雖不甚精縟,而頗為簡易。李淳風嘗謂斛蘭所作鐵儀,赤道不動,乃如膠柱。以考月行,差或至十七度,少不減十度。此正謂直以赤道候月行,其差如此。今黃、赤道度,再運游儀抵所舍宿度求之,而月行則以月歷每日去極度算率之,不可謂之膠也。新法定宿而變黃道,此定黃道而變宿,但可賦三百六十五度
 而不能具餘分,此其為略也。



 其八,令瓚舊法,黃道設於月道之上,赤道又次月道,而璣最處其下。每月移一交,則黃、赤道輒變。今當省去月道,徙璣於赤道之上,而黃道居赤道之下,則二道與衡端相迫,而星度易審。



 其九,舊法:規環一面刻周天度,一面加銀丁。所以施銀丁者,夜候天晦,不可目察,則以手切之也。古之人以璇為之,璇者,珠之屬也。今司天監三辰儀設齒於環背,不與橫蕭會,當移列兩旁,以便參察。



 其十,舊法:重璣皆廣四寸,
 厚四分。其它規軸,椎重樸拙,不可旋運。今小損其制,使之輕利。



 其十一,古之人知黃道歲易,不知赤道之因變也。黃道之度,與赤道之度相偶者也。黃道徙而西,則赤道不得獨膠。今當變赤道與黃道同法。



 其十二,舊法:黃、赤道平設,正當天度,掩蔽人目,不可占察。其後乃別加鉆孔,尤為拙謬。今當側置少偏,使天度出北際之外,自不凌蔽。



 其十三,舊法:地紘正絡天經之半,凡候三辰出入,則地際正為地紘所伏。今當徙紘稍下,使地際與紘
 之上際相直。候三辰伏見,專以紘際為率,自當默與天合。



 又言渾儀制器:



 渾儀之為器,其屬有三,相因為用。其在外者曰體,以立四方上下之定位。其次曰象,以法天之運行,常與天隨。其在內璣衡,璣以察緯,衡以察經。求天地端極三明匿見者,體為之用;察黃道降陟辰刻運徙者,像為之用;四方上下無所不屬者,璣衡為之用。



 體之為器,為圓規者四。其規之別:一曰經,經之規二並峙,正抵子午,若車輪之植。二規相距四寸,夾規為齒,以別
 去極之度。北極出紘之上三十有四度十分度之八強,南極下紘亦如之。對銜二釭,聯二規以為一,釭中容樞。二曰緯,緯之規一,與經交於二極之中,若車輪之倚,南北距極皆九十一度強。夾規為齒,以別周天之度。三曰紘,紘之規一,上際當經之半,若車輪之僕,以考地際,周賦十二辰,以定八方。紘之下有趺,從一衡一,刻溝受水以為平。中溝為地,以受注水。四末建趺,為升龍四以負紘。凡渾儀之屬皆屬焉。龍吭為綱維之四揵以為固。



 象
 之為器,為圓規者四。其規之別:一曰璣,璣之規二並峙,相距如經之度。夾規為齒,對銜二釭,釭中容樞,皆如經之率。設之亦如經,其異者經膠而璣可旋。二曰赤道,赤道之規一刻,璣十分寸之三以銜赤道。赤道設之如緯,其異者緯膠於經,而赤道銜於璣,有時而移,度穿一竅,以移歲差。三曰黃道,黃道之規一,刻赤道十分寸之二以銜黃道,其南出赤道之北際二十有四度,其北入赤道亦如之。交於奎、角,度穿一竅,以銅編屬於赤道。歲差
 盈度,則並赤道徙而西。黃赤道夾規為齒,以別均迤之度。



 璣衡之為器,為圓規二,曰璣,對峙,相距如象璣之度,夾規為齒,皆如象璣。其異者:象璣對銜二釭,而璣對銜二樞,貫於象璣天經之釭中。三物相重而不相膠,為間十分寸之三,無使相切,所以利旋也。為橫簫二,兩端夾樞,屬於璣,其中挾衡為橫一,棲於橫簫之間。中衡為□,以貫橫簫,兩末入於璣之罅而可旋。璣可以左右,以察四方之詳;衡可以低昂,以察上下之祥。



 《浮漏議》曰:



 播水
 之壺三,而受水之壺一。曰求壺、廢壺,方中皆圓尺有八寸,尺有四寸五分以深,其食二斛,為積分四百六十六萬六千四百六十。曰復壺,如求壺之度,中離以為二,元一斛介八斗,而中有達。曰建壺,方尺植三尺有五寸,其食斛有半。求壺之水,復壺之所求也。壺盈則水馳,壺虛則水凝。復壺之肋為枝渠,以為水節。求壺進水暴,則流怒以搖,復以壺,又折以為介。復為枝渠,達其濫溢。枝渠之委,所謂廢壺也,以受廢水。三壺皆所以播水,為水制
 也。自復壺之介,以玉權釃於建壺,建壺所以受水為刻者也。建壺一易箭,則發上室以瀉之。求、復、建壺之洩,皆欲迫下,水所趣也。玉權下水之概寸,矯而上之然後發,則水撓而不躁也。復壺之達半求壺之注,玉權半復壺之達。枝渠博皆分,高如其博,平方如砥,以為水概。壺皆為之冪,無使穢游,則水道不慧。求壺之冪龍紐,以其出水不窮也。復壺士紐,士所以生法者,復壺制法之器也。廢壺鯢紐,止水之沉,鯢所伏也。銅史令刻,執漏政也。冬
 設熅燎,以澤凝也。注水以龍噣直頸附於壺體,直則易浚,附於壺體則難敗。復壺玉為之喙,銜於龍噣,謂之權,所以權其盈虛也。建壺之執窒瓬塗而彌之以重帛,窒則不吐也。管之善利者,水所溲也,非玉則不能堅良以久。權之所出高則源輕,源輕則其委不悍而溲物不利。箭不效於璣衡,則易權、洗箭而改畫,覆以璣衡,謂之常不弊之術。今之下漏者,始嘗甚密,久復先大者管泐也。管泐而器皆弊者,無權也。弊而不可復壽者,術固也。察
 日之晷以璣衡,而制箭以日之晷跡,一刻之度,以賦餘刻,刻有不均者,建壺有眚也。贅者磨之,創者補之,百刻一度,其壺乃善。晝夜已復,而箭有餘才者,權鄙也。晝夜未復,而壺吐者,權沃也。如是,則調其權,此制器之法也。



 下漏必用甘泉,惡其垽之為壺眚也。必用一源泉之冽者,權之而重,重則敏於行,而為箭之情慓;泉之鹵者,權之而輕,輕則椎於行,而為箭之情駑。一井不可他汲,數汲則泉濁。陳水不可再注,再注則行利。此下漏之法也。



 箭一如建壺之長,廣寸有五分,三分去二以為之厚,其陽為百刻,為十二辰。博牘二十有一,如箭之長,廣五分,去半以為之厚。陽為五更,為二十有五籌;陰刻消長之衰。三分箭之廣,其中刻契以容牘。夜算差一刻,則因箭而易牘。鐐匏,箭舟也。其虛五升,重一鎰有半。鍛而赤柔者金之美者也,然後漬而不墨,墨者其久必蝕。銀之有銅則墨,銅之有錫則屑,特銅久灂則腹敗而飲,皆工之所不材也。



 《景表議》曰:



 步景之法,惟定南北為難。古法置
 槷為規,識日出之景與日入之景。晝參諸日中之景,夜考之極星。極星不當天中,而候景之法取晨夕景之最長者規之,兩表相去中折以參驗,最短之景為日中。然測景之地,百里之間,地之高下東西不能無偏,其間又有邑屋山林之蔽,倘在人目之外,則與濁氛相雜,莫能知其所蔽,而濁氛又系其日之明晦風雨,人間煙氣塵坌變作不常。臣在本局候景,入濁出濁之節,日日不同,此又不足以考見出沒之實,則晨夕景之短長未能得
 其極數。



 參考舊聞,別立新術。候景之表三,其崇八尺,博三寸三分,殺一以為厚者。圭首剡其南使偏銳。其趺方厚各二尺,環趺刻渠受水以為準。以銅為之。表四方志墨以為中刻之,綴四繩,垂以銅丸,各當一方之墨。先約定四方,以三表南北相重,令趺相切,表別相去二尺,各使端直。四繩皆附墨,三表相去左右上下以度量之,令相重如一。自日初出,則量西景三表相去之度,又量三表之端景之所至,各別記之。至日欲入,候東景亦如
 之。長短同,相去之疏密又同,則以東西景端隨表景規之,半折以求最短之景。五者皆合,則半折最短之景為北,表南墨之下為南,東西景端為東西。五候一有不合,未足以為正。既得四方,則惟設一表,方首,表下為石席,以水平之,植表於席之南端。席廣三尺,長如九服冬至之景,自表趺刻以為分,分積為寸,寸積為尺。為密室以棲表,當極為溜,以下午景使當表端。副表並趺崇四寸,趺博二寸,厚五分,方首,剡其南,以銅為之。凡景表景薄
 不可辨,即以小表副之,則景墨而易度。



 元祐間蘇頌更作者,上置渾儀,中設渾象,旁設昏曉更籌,激水以運之。三器一機,吻合躔度,最為奇巧。宣和間,又嘗更作之。而此五儀者悉歸於金。



 中興更謀制作,紹興三年正月,工部員外郎袁正功獻渾儀木樣,太史局令丁師仁始請募工鑄造,且言:「東京舊儀用銅二萬餘,今請折半用八千斤有奇。」已而不就,蓋在廷諸臣罕通其制度者。乃召蘇頌子攜取頌遺書,考質舊法,而攜亦不能通也。至十四
 年,乃命宰臣秦檜提舉鑄渾儀,而以內侍邵諤專領其事,久而儀成。三十二年,始出其二置太史局。而高宗先自為一儀置諸宮中,以測天象,其制差小,而邵諤所鑄蓋祖是焉,後在鐘鼓院者是也。



 清臺之儀,後其一在秘書省。按:儀制度:表裏凡三重,其第一重曰六合儀,陽經徑四尺九寸六分,闊三寸二分,厚五分。南北正位,兩面各列周天度數,南北極出入地皆三十一度少,度闊三分。陰緯單環大小如陽經,闊三寸二分,厚一寸八分。
 上置水平池,闊九分,深四分,沿環通流,亦如舊制。內外八干、十二枝,畫艮、巽、坤、乾卦於四維。第二重曰三辰儀,徑四尺三分,闊二寸二分,厚五分。釭釧刻畫如陽經。赤道單環,徑四尺一寸四分,闊一寸二分,厚五分。上列二十八宿、均天度數,闊二分七厘。黃道單環,徑四尺一寸四分,闊一寸二分,厚五分,上列七十二候,均分卦策,與赤道相交,出入各二十四度弱。百刻單環,徑四尺五寸六分,闊一寸二分,厚五分,上列晝夜刻數。第三重曰
 四游儀,徑三尺九寸,闊一寸九分,厚五分。釭釧刻畫如璇璣,度闊二分半。望筒長三尺六寸五分,內圓外方,中通孔竅,四面闊一寸四分七厘,窺眼闊三分,夾窺徑五尺三分。鰲云以負龍柱,龍柱各高五尺二寸。十字平水臺高一尺一寸七分,長五尺七寸,闊五寸二分。水槽闊七分,深一寸二分。若水運之法與夫渾象,則不復設。



 其後朱熹家有渾儀,頗考水運制度,卒不可得。蘇頌之書雖在,大抵於渾象以為詳,而其尺寸多不載,是以難遽
 復云。舊制有白道儀以考月行,在望筒之旁。自熙寧沉括以為無益而去之,南渡更造,亦不復設焉。



 極度



 極度極星之在紫垣,為七曜、三垣、二十八宿眾星所拱,是謂北極,為天之正中。而自唐以來,歷家以儀象考測,則中國南北極之正,實去極星之北一度有半,此蓋中原地勢之度數也。中興更造渾儀,而太史令丁師仁乃言:「臨安府地勢向南,於北極高下當量行移易。」局官呂
 璨言:「渾天無量行更易之制,若用於臨安與天參合,移之他往必有差忒。」遂罷議。後十餘年,邵諤鑄儀,則果用臨安北極高下為之。以清臺儀校之,實去極星四度有奇也。



 黃赤道



 黃赤道占天之法,以二十八宿為綱維,分列四方,南北去極各九十有一度有奇,南低而北昂,去地各三十有六度,一定不易者,名之曰赤道。以日躔半在赤道內,半
 在赤道外,出入內外極遠者皆二十有四度,以其行赤道之中者名之曰黃道。凡五緯皆隨日由黃道行,惟月之行有九道,四時交會歸於黃道而轉變焉,故有青、黑、白、赤四者之異名。



 夫赤道終古不移,則星舍宜無盈縮矣。然自唐一行作《大衍歷》,以儀揆測之,得畢、觜、參、鬼四宿,分度與古不同。皇祐初,日官周琮以新儀測候,與唐一行尤異。紹聖二年,清臺以赤道度數有差,復命考正。惟牛、室、尾、柳四宿與舊法合,其它二十四宿躔度或多
 或寡。蓋天度之不齊,古人特紀其大綱,後世漸極於精密也。



 若夫黃道橫絡天體,列宿躔度自隨歲差而增減。中興以來,用《統元》、《紀元》及《乾道》、《淳熙》、《開禧》、《統天》、《會元》,每一歷更一黃道,其多寡之異有不可勝載者,而步占家亦隨各歷之躔度焉。



 中星



 中星四時中星見於《堯典》,蓋聖人南面而治天下,即日行而定四時,虛、鳥、火、昴之度在天,夷隩析因之候在人,
 故《書》首載之,以見授時為政之大也。而後世考驗冬至之日,堯時躔虛,至於三代則躔於女,春秋時在牛,至後漢永元已在斗矣。大略六十餘年輒差一度。開禧占測已在箕宿,校之堯時,幾退四十餘度。蓋自漢太初至今,已差一氣有餘。而太陽之躔十二次,大約中氣前後,乃得本月宮次。蓋太陽日行一度,近歲《紀元歷》定歲差,約退一分四十餘秒。蓋太陽日行一度而微遲緩,一年周天而微差,積累分秒而躔度見焉。歷家考之,萬五千年
 之後,所差半周天,寒暑將易位,世未有知其說者焉。



 土圭



 土圭《周官》大司徒以土圭之法正日景,以求地中。而馮相氏春夏致日,秋冬致月,以辨四時之



\end{pinyinscope}