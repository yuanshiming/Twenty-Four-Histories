\article{志第九十二 樂十四(樂章八)}

\begin{pinyinscope}

 恭上皇帝皇太后尊號下冊立皇后冊皇



 太子皇子冠鄉飲酒聞喜宴鹿鳴宴



 紹熙元年恭上壽聖皇太后、至尊壽皇聖帝、壽成皇后
 尊號冊寶十四首



 大慶殿發冊寶降殿,《正安》



 帝受內禪,紀元紹熙。欽崇慈親,孝心肅祗。



 乃建顯號,乃蕆丕儀。發冊廣庭,聲歌侑之。



 中書令、侍中奉三宮冊寶詣東階下,用《禮安》



 鐘鼓交作,文物咸備。彤庭玉階,天子是蒞。



 咨爾輔臣,展採錯事。輔臣稽首,敢不率禮!



 冊寶出門,《正安》



 巍巍天宮,洞開閶闔。旗常葳蕤,劍佩
 雜沓。



 寶冊啟行,法駕繼發。鑠哉盛典,快睹胥悅!



 冊寶入重華宮,《正安》



 仰止皇居,九門載闢,麗日重光,非煙五色。



 雷動萬乘,雲從百闢。咫尺重霄,鞠躬屏息。



 至尊壽皇聖帝升坐,《乾安》降同。



 玉璽瑤編,禮容畢具。穆穆至尊,華殿是御。



 德配有虞,紹唐授禹。於萬斯年,受天之祐。



 太傅、中書令奉至尊壽皇聖帝冊升殿,用《聖安》



 慈皇天臨,睟表怡怡。欽哉聖子,親奉玉卮!



 鰲抃嵩呼,歡浹
 華夷。邇臣捧冊,是恪是祗。



 太傅、侍中奉至尊壽皇聖帝寶升殿,用《聖安》



 瑟彼華玉,篆魚鈕龍。與冊並登,咨爾上公。



 詠以歌詩,協之鼓鐘。是陟是降,靡有弗恭。



 太傅、中書令、侍中奉壽聖皇太后冊寶升殿,用《聖安》



 天祐皇家,慶集重闈。寶兮揚名,冊兮流徽。



 金支秀華,盛容祲威。詔我近弼,相禮不違。



 太傅、中書令、侍中奉壽成皇后冊寶升殿,用《聖安》



 大
 哉乾元,既極形容!坤元德至,實與比隆。



 寶冊並登,勒崇垂鴻。相我縟儀,肅肅雍雍。



 皇帝從壽聖皇太后冊寶詣慈福宮,用《正安》



 涓辰協吉,時維春元。上冊三殿,曠古無前。



 思齊重闈,積慶有源。是尊是崇,帝心載虔。



 壽聖皇太后出閣升坐,《坤安》降同。



 丕赫有宋,三聖授受。誰其助之?繄我太母。



 東朝受冊,飲此春酒。聖子神孫,密侍左右。



 內侍官舉壽聖皇太后冊寶詣讀冊寶位,用《聖安》



 坤德益崇,天壽平格。慶流萬世,子孫千億。



 刻玉範金,鋪張赫奕。惟昔姜、任,則莫我匹。



 皇帝詣壽成殿,壽成皇后出閣升坐,《坤安》降同。



 鞠育保護,母道備矣。密贊親傅,德其至矣。



 彩服來朝,慈容有喜。既受鴻名,又多受祉。



 內侍官舉壽成皇后冊寶詣讀冊寶位,用《聖安》



 仰瞻慈闈,登進寶冊。惟時御,祗率厥職。



 曰壽曰名,母兮
 兼得。儷我尊父,億載無極。



 紹熙四年加上壽聖皇太后尊號八首



 大慶殿發冊寶降殿,《正安》



 德厚重闈,沖澹粹穆。何以名之?惟慈惟福。



 寶鏤精鏐,冊鐫華玉。物盛禮崇,丕昭群目。



 中書令、侍中奉壽聖皇太后冊牢詣東階下,《禮安》



 於皇帝室,休運貽孫。重熙疊慶,祗進號榮。



 爰授茲冊,必躬必親。天子聖孝,萬邦儀刑。



 冊寶出門,《正安》



 煌煌冊寶,天子受之。言徐其行,肅展乃儀。



 其儀維何?劍佩黃麾。鸞駕清蹕,聳瞻九逵。



 冊寶入慈福宮殿門,《正安》



 熙辰禮備,濟濟雍雍。言奉斯冊,重親之宮。



 宮帷既敞,協氣感通。皇儀親展,壽祉無窮。



 太傅、中書令、侍中奉壽聖皇太后冊寶升殿,《聖安》



 既肅琨庭,載升金戺。乃導乃陪,威儀濟濟。



 天步繼臨,孝誠備矣。聲容孔昭,中外悅喜。



 冊寶詣宮中,《正安》



 琱輿彩仗,祗詣慈宮。寶冊前奉,龍挾雲從。



 言備茲禮,於宮之中,惟天子孝,於昭祲容。



 壽聖皇太后出閣升御坐,《坤安》降同。



 懿典大冊,陳儀邃深。怡怡愉愉,寶坐是臨。



 重彩儼侍,展肅心。三宮協慶,永播徽音。



 內侍官舉壽聖皇太后冊寶詣讀冊寶位,用《聖安》



 寶冊即奉,祗誦乃言。仁深慶衍,益顯益尊。



 和聲協氣,充溢乾坤。並受伊嘏,聖子神孫。



 慶元二年恭上太皇太后、皇太后、太上皇帝、太上皇后尊號二十四首



 冊寶降殿



 天擁帝家,澤流子孫。三宮燕胥,四海崇尊。



 聲諧《韶》、《濩》,輝燭瑤琨。維皇緝熙,耀德乾坤。



 冊寶授太傅奉詣東階下



 祖後重壽,親闈並崇。駢慶聯休,申景鋪鴻。



 疊璧交輝,多儀煥叢。億萬斯年,福祿攸同。



 冊寶出門



 太任媚姜,塗山翼禹。慈祥曼衍,鴻儀迭舉。



 寶章奕奕,祲宮俁俁。帝用將之,於彼宮所。



 慈福宮寶冊入門



 東朝層邃,端闈靖深。列仗節鑾,鏤玉繩金。



 來奉來崇,載祗載欽。曾孫之慶,世世徽音。



 冊寶升殿



 純祐我宋,母儀四朝。擁翼孫謀,如虞承堯。



 仁覃函夏,喜浮慶霄。福祿萬年,金玉孔昭。



 冊寶詣宮中



 神人和懌,天日淑清。王母來燕,必壽而名。



 琨庭璈音,五雲佩聲。勉勉我皇,遹昭厥成。



 太皇太后出閣升坐



 曾孫致養,五福駢臻。太極所運,
 兩儀三辰。



 輝光日新,啟祐後人。永翼瑤圖,億萬堯春。



 冊寶詣讀冊寶位



 徽光宣華,仁聲流文。曠儀合沓,泰和絪縕。



 慈顏有喜,祚我聖君。珠宮含飴,坐閱來云。



 太皇太后降坐歸閣



 縟儀既登,寶冊既膺。喜洽祥流,雲烝川增。



 天子萬年,鳴玉慈庭。惠我無疆,詵詵繩繩。



 壽慈宮冊寶入門



 新庭靖安,祖後燕怡。有開聖謀,累崇天基。



 典章文明,聲容葳蕤。御於邦家,曰壽曰慈。



 冊寶升殿



 三禮崇容,八鑾警衛。有來辰儀,闡徽媯汭。



 璇宮肅雍,藻景澄霽。文子文孫,本支百世。



 冊寶詣宮門



 堯門疊瑞,姒幄齊輝。重坤靖夷,麗冊華徽。



 天子仁聖,禮文弗違。福壽康寧,同燕層闈。



 皇太后出閣升坐



 文母曼壽,載錫之光。總集瑞命,宜君宜王。



 惠以仁顯,慈以德彰。保祐子孫,受福無疆。



 冊寶詣讀冊寶位



 華鸞編玉,文螭液金。頌德摛英,揚徽嗣音。



 紫幄天開,翠華日臨。歲歲年年,如周大任。



 皇太后降坐歸閣



 宋有明德,天保祐之。以壽繼壽,以
 慈廣慈。



 聲文宣昭,福祉茂綏。神孫之休,燕及華夷。



 壽康宮冊寶入門,《正安》



 大安耽耽,興慶崇崇。維皇之尊,與天比隆。



 非心閑燕,文命延鴻。欲報之恩,禮縟儀豐。



 太上皇帝升御坐,《乾安》



 上帝有赫,百靈效祥。儲祉垂恩,錫年降康。



 皇儀晬溫,帝躬肅莊。三宮齊歡,地久天長。



 太上皇帝冊寶升殿,《聖安》



 夏典稽瑞,禹玉含淳。追琢
 有章,溫潤孔純。



 聖底於安,壽綿於仁。太上立德,自天其申。



 太上皇后冊寶升殿,《聖安》



 父尊母親,天涵地育。燕我翼子,景命有僕。



 得名得壽,如金如玉。子孫千億,成其厚福。



 太上皇帝降御坐,《乾安》



 天地清寧,日月華光。歸尊慈極,嵩呼未央。



 慶函百嘉,壽躋八荒。上皇萬歲,俾熾俾昌。



 冊寶詣宮中,《正安》



 晨趨慈幄,佳氣鬱蔥。受帝之祉,配天其崇。



 璧華金精,禮敷樂充。天子是若,歡聲融融。



 太上皇后出閣升坐,《坤安》



 文物流彩,鑾輅靖陳。龜瑞薦祉,坤儀效珍。



 比皇之壽,翼帝以仁。和氣致祥,與物為春。



 讀冊寶,《聖安》



 黼黻其文,金玉其相。永壽於萬,合德無疆。



 福緒祥源,厥後克昌。天維格斯,祚我聖皇。



 太上皇后降坐歸閣,《坤安》



 榮懷之慶,莫盛於斯。三宮
 四冊,五葉一時。



 德阜而豐,福大而滋。子子孫孫,於時保之。



 嘉泰二年恭上太皇太后尊號八首



 冊寶降殿



 思齊太任,嬪於周京。至哉坤元,萬物資生!



 不可儀測,矧可強名。鏤玉繩金,昭哉號榮!



 冊寶詣東階



 鼓鐘喤□,儀物載陳。儀物陳矣,爛其瑤琨。



 咨爾上公,相予文孫。勿亟勿徐,奉我重親。



 冊寶出門



 蕩蕩天門,金鋪玉戶。採旄翠旌,流蘇葆羽。



 千官影從。乃導乃輔。都人縱觀,填道呼舞。



 壽慈宮冊寶入門



 煌煌寶書,玉篆金縷。曷為來哉?自天子所。



 自天子所,以燕文母。婉嬗祥雲,日正當午。



 冊寶升殿



 文物備矣,三事其承。崇牙高張,樂充宮庭。



 耽耽廣殿,左戚右平。敬爾威儀,攝齊以登。



 冊寶詣宮中



 維壽伊何?聖德日新。維慈伊何?祐於後人。



 乃範斯金,乃縷斯鈱。皇舉玉趾,從於堯門。



 太皇太后升御坐降同。



 侍中版奏,辦外嚴中。出自玉房,
 禕示俞被躬。



 我龍受之,祲威盛容。皇帝聖孝,其樂融融。



 冊寶詣讀冊寶位



 麟趾褭蹄,我寶斯刻。堧磩採致,載備斯冊。



 眉壽萬年,詒謀燕翼。於赫湯孫,克綿永福。



 紹定三年壽明仁福慈睿皇太后冊寶九首



 文德殿冊寶降殿



 思齊聖母,媲于周任。體乾履坤,博厚洪深。



 七表既啟,萬壽自今。昕庭發號,式昭德音。



 冊寶詣東階



 煌煌儀物,繹繹鼓鐘。奉茲寶冊,至於階東。



 上公相儀,列闢盡恭。拜手慈宸,福如華、嵩。



 冊寶出門



 帝闕肅開,天階坦履。霓旌羽蓋,導儀護衛。



 匪誇雕琢,匪矜繁麗。茲謂盛儀,億載千歲。



 慈明殿冊寶入門



 金堅玉純,文鬱禮縟。來從帝所,作瑞王國。



 天開地闢,日熙春燠。茲謂盛事,永燕茀祿。



 冊寶升殿



 皇儀有煒,彩舁次升。沉沉邃殿,穆穆天廷。



 坤德隆。皇圖永寧。咨爾廷臣,攝齊以登。



 冊寶詣宮中



 壽為福先,明燭物表。仁沾動植,福齊穹昊。



 曰慈與睿,並崇丕號。演而申之,萬世永保。



 皇太后升御坐



 邇臣跪奏,嚴辦必恭。乃御禕示俞,升於殿中。



 慈顏雍穆,和氣沖融。芳流清史,傳之無窮。



 冊寶詣讀冊寶位



 徽音孔昭,寶傳斯刻。金昭玉粹,有燁斯冊。



 載祈載祝,以燕以翼。寶之萬年,與宋無極。



 皇太后降御坐



 皇文既舉,慶禮告虔。肇自宮闈,格於幅員。



 子稱母壽,母謂子賢。陟降在茲,隆名際天。



 哲宗發皇后冊寶三首



 皇帝升坐,《乾安》



 既登乃依,如日之升。有嚴有翼,丕顯
 丕承。



 天作之合,家邦其興。朱芾斯皇,子孫繩繩。



 降坐,《乾安》



 我禮嘉成,我駕言旋。降坐而蹕,奏鼓淵淵。



 景命有僕,保祐自天。永錫祚嗣,何千萬年。



 太尉等奉冊寶出入,《正安》



 宣哲維公,就位肅莊。冊寶具舉,丕顯其光。



 出於宸闈,鼓鐘喤□。母儀天下,萬壽無疆。



 紹興十三年發皇后冊寶十三首



 皇帝升坐,《乾安》



 天地奠位,乾坤以分。夫婦有別,父子
 相親。



 聖王之治,禮重婚姻。端冕從事,是正大倫。



 使副入門,《正安》



 天子當陽,群工就列。冊寶既陳,鐘鼓備設。



 上公奉事,容莊心協。克相盛禮,光昭玉牒。



 冊寶出門,《正安》



 穆穆睟容,如天之臨。赫赫明命,如玉之音。



 虔恭出門,禮容兢兢。塗山生啟,夏道以興。



 皇帝降坐,《乾安》



 朝陽已升,熏風習至。樂奏既成,禮容亦備。



 玉佩鏘鳴,帝徐舉趾。壺政穆宣,以聽內治。



 皇后出閣,《乾安》



 猗歟賢後,德本性成!承天致順,溯日
 為明。



 作配儷極,王化以行。萬有千歲,奉祀宗祊。



 冊寶入門,《宜安》



 款承祗事,時惟肅雍。跪奉冊寶,陳於法宮。



 以俯以仰,有儀有容。明神介之,福祿來崇。



 皇后降殿,《承安》



 溫惠之德,禕翟之衣。行中《採薺》,禮無或違。



 降於丹陛,有容有儀。委委蛇蛇,誰其似之!



 皇后受冊寶,《成安》



 鏤蒼玉兮,盛德載揚。鑄南金兮,作鎮椒房。



 虔受賜兮,有燁有光。宜室家兮,朱芾斯皇。



 皇后升坐,《和安》



 禮既行兮,厥位孔安。母儀正兮,容止
 所觀。



 奉東朝兮,常得其歡。求淑女兮,豈樂多般。



 內命婦入門,《惠安》



 素月澄輝,眾星顯列。炳為天文,各有攸別。



 椒房既正,陰教斯設。《關雎》、《麟趾》,應如響捷。



 外命婦入門,《成安》



 窈窕其容,淑其姿。爛其如云,瞻我母儀。



 曰天之妹,作合惟宜。粲然舞抃,疇不肅祗。



 皇后降坐,《徽安》



 寶字煌煌,冊書粲粲。副笄加飾,禕褕有爛。



 祗若帝休,委蛇樂衎。億萬斯年,永膺宸翰。



 皇帝歸閣,《泰安》



 太任徽音,太姒是嗣。則百斯男,周室
 以熾。



 天子萬年,受茲女士。如姒事任,從以孫子。



 淳熙三年發皇后冊寶十三首



 皇帝升坐,《乾安》



 赫赫惟皇,如日之光。肅肅惟後,如月之常。



 禮行一時,明照無疆。天子蒞止,疇敢不莊!



 冊寶入門,《正安》



 卜月惟良,練辰斯臧。臣工在庭,劍佩瑲瑲。



 來汝凝丞,明命是將。有淑其儀,無或怠遑。



 冊寶出門,《正安》



 刻簡以鈱,鑄寶以金。持節伊誰?時惟四鄰。



 自我文德,達之穆清。委蛇委蛇,往迄於成。



 皇帝降坐,《乾安》



 冊行何向?於門東偏。禮備樂成,合扇鳴鞭。



 皇舉玉趾,如天之旋。燕及家邦,億萬斯年。



 皇后出閣,《坤安》



 椒塗蘭馭,河潤山容。副笄在首,禕衣被躬。



 靜女其姝,實翼實從。自彼西閣,聿來殿中。



 冊寶入門,《宜安》



 德隆位尊,禮厚文縟。乃篆斯金,乃鏤斯玉。



 群公盈門,執事有肅。願言保之,永鎮坤軸。



 皇后降殿,《承安》



 規殿沉沉,葉氣旼□。明章婦順,表正人倫。



 躡是左戚,暨於中庭。尚宮顯相,罔有弗欽。



 皇后受冊寶,《成安》



 備物典冊,樂之鼓鐘。拜而受之,極其肅雍。



 司言司寶,各以職從。行地有慶,與天無窮。



 皇后升坐,《和安》



 容典既膺,壺儀既正。羽衛外列,揚顏中映。



 如帝如天,以莊以靚。六宮承式,《二南》流詠。



 內命婦入門,《惠安》



 《葛覃》節用,《樛木》逮下。形為嬪則,夙已心化。



 茲臨長秋,遂正諸夏。以慶以祈,百祥來迓。



 外命婦入門,《咸安》



 碩人其頎,公侯之妻。翟茀以朝,像服是宜。



 如星之共,溯月之輝。母儀既瞻,群心則夷。



 皇后降坐,《徽安》



 窈窕淑女,備六服兮。陟降多儀,聳群目兮。



 內治允備,陰教肅兮。宜君宜王,綏有福兮。



 皇后歸閣,《泰安》



 天監有周,是生太任。亦有太姒,嗣其徽音。



 孰如兩宮,慈愛相承!思齊之盛,復見於今。



 淳熙十六年皇后冊寶十三首



 皇帝升坐,《乾安》



 乾位既正,坤斯順承。日麗於天,月斯溯明。



 惟帝受命,惟帝並登。黼扆尊臨,典冊是行。



 冊寶入門,《正安》



 乃協良辰,維春之宜。乃詔近弼,來汝
 相儀。



 九門洞開,文物華輝。聲詩載歌,於以侑之。



 冊寶出門,《正安》



 有璽範金,有冊鏤瓊。汝使汝介,持節以行。



 禮始文德,達於穆清,是恪是虔,依我和聲。



 皇帝降坐,《乾安》



 鼓鐘喤□,磬筦鏘鏘。劍佩充庭,濟濟洋洋。



 禮典告備,皇心樂康。於萬斯年,受福無疆。



 穆清殿受冊寶,皇后出閣,《坤安》



 懿範柔容,如月斯輝。駕厥翟輅,被以禕衣。



 九御從之,如雲祁祁。典冊是承,心焉肅祗。



 冊寶入門,《宜安》



 華榱璧璫,有馨椒殿。備物來陳,多儀式煥。



 曰冊曰寶,是刻是□彖。並舉以行,皇矣懿典。



 皇后降殿,《承安》



 禕褕盛服,有恪其容。是陟是降,相以尚宮。



 金殿玉階,聿來於中。展詩應律,載詠肅雍。



 皇后受冊寶,《成安》



 帝有顯命,稟於親慈。後德克承,拜而受之。



 人倫既正,王化是基。億載萬年,永祚坤儀。



 皇后升坐,《和安》



 帝慶三宮,膺受寶冊。御於中闈,載欣載惕。



 乃敷陰教,乃明《內則》。翼翼魚貫,罔不承式。



 內命婦入門,《惠安》



 掖庭頒官,於位有四。嘒彼小星,撫以德惠。



 熙焉如春,育焉如地。慶禮聿成,靡弗咸喜。



 外命婦入門,《咸安》



 魚軒鼎來,像服是宜。班於內庭,率禮惟祗。



 化以婦道,時惟母儀。是慶是類,於胥樂兮。



 皇后降坐,《徽安》



 正位長秋,容典備矣。王假有家,人倫至矣。



 儷極俔天,多受祉矣。蟄蟄螽斯,宜孫子矣。



 皇后歸閣,《泰安》



 維天祐宋,盛事相仍。崇號三宮,甫茲浹辰。



 肇正中闈,縟禮載陳。邦家之慶,曠古無倫。



 皇帝升坐,《乾安》



 乾健坤順,群生首資。日常月升,四時葉熙。



 帝嗣天歷,後崇母儀,黼黻承暉,王化是基。



 使副入門,《正安》


熛闕蟺
 \gezhu{
  艸葉}
 ,璧門雲龍。烈文維輔,翊奉有容。



 典章輝明,彞度肅雍。蕆時縟儀,登於璇宮。



 冊寶出門,《正安》



 金晶麗輝,璧葉含春。贊夏之翼,繹虞之嬪。



 樂序《韶》亮,禮文藻新。闢公相成,物採彬彬。



 皇帝降坐,《乾安》



 帟旒雲舒,金秀充庭。璇衛鑾華,蒨佩垂絲呈。



 皇容熙備,柔儀順承。三宮齊歡,萬福昭膺。



 皇后出閣,《坤安》



 驂翟崇容,禕鞠陳衣。戾止蘭殿,夙興椒闈。



 淑正宣華,粹明騰輝。欽若有承,嗣音之徽。



 冊寶入門,《宜安》



 禕帟流光,慶祥增衍。編玉鏤德,螭金溢篆。



 粹猷藻黼,徽文華顯。《二南》聲詩,於時昭闡。



 皇后降殿,《承安》



 翬珩煥採,趨節風韶。陟降戚陛,奉將英瑤。



 闢道承熏,嬪儀揚翹。是敬是祗,德音孔昭。



 皇后受冊寶,《成安》



 帝奉太室,後儀成之。帝養三宮,後志承之。



 德如《關雎》,盛如《螽斯》。宜君宜王,百世本支。



 皇后升坐,《和安》



 肅肅壺彞,雍雍陰教。險詖自防,警戒是效。



 中闈端委,列禦胥告。其思輔順,永翼帝孝。



 內命婦入門,《惠安》



 天子九嬪,王宮六寢。有燁令儀,載秩華品。



 福履綏將,節用躬儉。矢其德音,於以來諗。



 外命婦入門,《咸安》



 象服之文,《鵲巢》之風。化以婦道,覲於內宮。



 採蘋澗濱,採藻澗中。夙夜在公,贊彼累功。



 皇后降坐,《和安》



 光祐晏寧,惠慈燕喜。壽仁並崇,家邦均祉。



 懿文交舉,壺冊嗣美。維億萬年,愛敬惟似。



 皇后歸閣,《泰安》



 天心仁祐,坤德世昭。灼有慈範,著於累朝。



 儉以贊虞,勤以承堯。是用則效,共勵夙宵。



 嘉泰三年皇后冊寶十三首



 皇帝升坐,《乾安》



 茂建坤極,容典聿新。天命所贊,慈訓是遵。



 肅涓穀旦,躬御紫宸。鴻禧累福,駢賚翕臻。



 使副入門,《正安》



 端門曉闢,瑞氣雲凝。有儼良輔,踵武造廷。



 肅肅王命,是將是承。登冊穆清,萬歲永膺。



 冊寶出門,《正安》



 瑤冊玉寶,爛然瑞輝。旁翼絳節,上承
 紫微。



 璆鳴朝佩,徐出獸扉。登進坤極,益彰典徽。



 皇帝降坐,《乾安》



 天臨黼扆,雲集弁纓。金石遞奏,典禮備成。



 玉趾緩步,龍駕翼行。言旋北極,永燕西清。



 皇后出閣,《乾安》



 日熏椒屋,雲靄璧門。有華瑞節,來自帝閽。



 統天惟乾,合德者坤。我龍受之,福祿永繁。



 冊寶入門,《宜安》



 虹輝燦爛,雲篆綢繆。絳節前導,瑞光上浮。



 瑤階玉扉,即集長秋。欽承天寵,永荷帝休。



 皇后降殿,《承安》



 瑤殿清閟,玉戚坦夷。禕衣副珈,陟降
 不遲。



 寶冊聿至,載肅載祗。禮儀昭備,福履永綏。



 皇后受冊寶,《成安》



 日月臨燭,乾坤覆持。明並二曜,德合兩儀。



 光媲宸極,共恢化基。膺受茂典,億載永宜。



 皇后升坐,《和安》



 寶璽瑤冊,既祗既承。繡裀藻席,載躋載升。



 柔儀肅穆,瑞命端凝。永膺多福,如川方增。



 內命婦入門,《惠安》



 服煥盛儀,班分華致。九嬪婦職,六寢內治。



 參差荇菜,求勤寤寐。烝然來思,相禮贊祭。



 外命婦入門,《咸安》



 婦榮於室,通籍禁中。班列有次,車
 服有容。



 佐我《關雎》,《鵲巢》之風。被之僮僮,曷不肅雍!



 皇后降坐,《徽安》



 金石具舉,典禮茂明。淑慎其止,遹觀厥成。



 瓊琚微動,鳳輦翼行。儀光媲極,德邁嬪京。



 皇后歸閣,《泰安》



 寶坐即興,鳳輿戒行。奏解嚴辦,歸燕邃清。



 問安壽慈,奉□宗祊。彌千萬年,內助聖明。



 嘉定十五年皇帝受「恭膺天命之寶」三首



 《恭膺天命》之曲,太簇宮



 我祖受命,恭膺於天。爰作玉寶,載祗載虔。



 申錫無疆,神聖有傳。昭茲興運,於萬斯
 年!



 《舊疆來歸》之曲,太簇宮



 於穆我皇,之德之純。涵濡群生,矧我遺民。



 運齊跨晉,輪貢效珍。土宇日闢,一視同仁。



 《永清四海》之曲,太簇宮



 我祖我宗,德厚澤深。於皇繼序,益單厥心。



 天人協扶,一統有臨。乾坤清夷,振古斯今。



 至道元年冊皇太子二首



 太子出入,《正安》



 主鬯之重,允屬賢明。承華肇啟,上嗣騰英。



 禮修樂舉,育德開榮。一人元良,萬邦以寧。



 群臣稱賀,《正安》



 皇儲既建,聖祚無疆。鸞旌列敘,雞戟分行。



 前星有爛,瑞日重光。際天接聖,溫文允臧。



 天禧三年冊皇太子一首



 太子出入,《明安》



 明《離》之象,少陽之位。固邦為本,體天作貳。



 儀範克溫,禮章斯備。丕宣令猷,恭守宗器。



 乾道元年冊皇太子四首



 皇帝升坐,《乾安》



 宋受天命,聖緒無疆。惟懷永圖,乃登元良。



 涓選休辰,冊書是將。黼坐天臨,穆穆皇皇。



 太子入門,《明安》



 於維皇儲,玉潤金聲。體《震》之洊,重《離》之明。



 冊寶具舉,環佩鏘鳴。守器承祧,惟邦之榮。



 太子出門,《明安》



 樂備既奏,和聲沖融。玉簡金書,翔鸞戲鴻。



 下拜登受,旋於青宮。儀辰作貳,垂休無窮。



 皇帝降坐,《乾安》



 我禮備成,我駕言旋。降坐而蹕,奏鼓淵淵。



 國本既定,保祐自天。克昌厥後,何千萬年。



 乾道七年冊皇太子四首



 皇帝升坐,《乾安》



 建儲以賢,闢宮於東。典冊既備,筮占既從。



 濟濟卿士,鏘鏘鼓鐘。天子戾止,盛哉禮容。



 太子入門,《明安》



 琱鈱瑳□,篆金煌煌。對揚於庭,是承是將。



 星重其暉,日重其光。觀瞻以懌,國有元良。



 太子出門,《明安》



 淵中象德,玉裕凝姿。進退周旋,有肅其儀。



 既定國本,益隆慶基。燕及兩宮,福祿如茨。



 皇帝降坐,《乾安》



 儲副豫定,器之公兮。冊授孔時,禮之
 隆兮。



 天步遲遲,旋九重兮。壽祉萬年,德無窮兮。



 嘉定二年冊皇太子四首



 皇帝升坐



 於皇我宋,受命於天。升儲主鬯,衍慶卜年。



 典冊告備,庭工載虔。萬乘蒞止,端冕邃延。



 太子入門受冊寶



 太極端御,少陽肅祗。鈱簡斯鏤,袞服孔宜。



 式奏備樂,乃陳盛儀。下拜登受,永言保之。



 太子受冊寶出門



 明兩承曜,作貳宣猷。茂德金昭,令譽川流。



 豫定厥本,永貽乃謀。三朝致養,問寢龍樓。



 皇帝降坐



 《震》洊體象,我儲明兮。渙揚顯冊,我禮成兮。



 大駕言旋,警蹕鳴兮。燕祉無疆,邦之榮兮。



 寶祐二年皇子冠二十首



 皇帝將出文德殿,《隆安》



 於皇帝德,乃聖乃神。本支百世,立愛惟親。



 敬共冠事,以明人倫。承天右序,休命用申。



 賓贊入門,《祗安》



 豐芑詒謀,建爾元子。揆禮儀年,筮賓敬事。



 八音克諧,嘉賓至止。於以冠之,成其福履。



 賓贊出門,《祗安》



 禮國之本,冠禮之始。賓升自西,維賓之位。



 於著於阼,維子之義,厥惟欽哉,敬以從事。



 皇帝降坐,《隆安》



 路寢闢門,黼坐恭己。群公在庭,所重維禮。



 正心齊家,以燕翼子。於萬斯年,王心載喜。



 皇子初行



 有來振振,月重輪兮。瑜玉在佩,綦組明兮。



 左徵右羽,德結旌兮。步中《採薺》,矩擭循兮。



 賓贊入門



 我有嘉賓,直大以方。亦既至止,厥德用光。



 冠而字之,厥義孔彰。表裏純備,黃耇無疆。



 皇子詣受制位



 吉圭休成,其日南至。天子有詔,冠爾皇嗣。



 為國之本,隆邦之禮。拜而受之,式共敬止。



 皇子升東階



 茲惟阼階,厥義有在。歷階而升,敬謹將冠。



 經訓昭昭,邦儀粲粲。正纚賓筵,壽考未艾。



 皇子升筵



 秩秩賓筵,籩豆孔嘉。帝子至止,衿纓振華。



 周旋陟降,禮行三加。成人有德,匪驕匪奢。



 初加



 帝子惟賢,懋昭厥德。跪冠於房,玄冠有特。



 鼓鐘喤□,威儀抑抑。百禮既洽,祚我王國。



 初醮



 有賓在筵,有尊在戶。磬管將將,醮禮時舉。



 跪觴祝辭,以永燕譽。寶祚萬年,盤石鞏固。



 再冠



 《復》爻肇祥,《震》維標德。乃共皮弁,其儀不忒。



 體正色齊,維民之則。璇霄眷祐,國壽箕翼。



 再醮



 冠醮之義,匪酬匪酌。於戶之西,敬共以恪。



 金石相宣,冠醮相錯。帝祉之受,施及家國。



 三加



 善頌善禱,三加彌尊。爵弁峨峨,介珪溫溫。



 陽德方長,成德允存。燕及君親,厥祉孔蕃。



 三醮



 席於賓階,禮義以興。受爵執爵,多福以膺。



 匪惟服加,德加愈升。匪惟德加,壽加愈增。



 皇子降



 命服煌煌,跬步中度。慶輯皇闈,化行海宇。



 禮具樂成,惕若戒懼。寶璐厥躬,有秩斯祜。



 朝謁皇帝將出



 皇王烝哉,令聞不已!燕翼有謀,冠醮有禮。



 百僚在庭,遹相厥事。頌聲所同,嘉受帝祉。



 皇子再拜



 青社分封,前星啟焰。繁弱綏章,厥光莫揜。



 容稱其德,蓄學之驗。芳譽敷華,大圭無玷。



 皇子退



 玄袞黼裳,垂徽永世。勉勉成德,是在元子。



 胙土南賓,厥旨孔懿。充一忠字,作百無愧。



 皇帝降坐



 愛始於親,聖盡倫兮。元子冠字,邦禮成兮。



 天步舒徐,皇心寧兮。家人之吉,億萬春兮。



 淳化鄉飲酒三十三章



 鹿鳴呦呦,命侶與儔。宴樂嘉賓,既獻且酬。



 獻酬有序,休祉無疆。展矣君子,邦家之光。



 鹿鳴呦呦,在彼中林。宴樂嘉賓,式昭德音。



 德音愔愔,
 既樂且湛。允矣君子,賓慰我心。



 鹿鳴呦呦,在彼高岡。宴樂嘉賓,吹笙鼓簧。



 幣帛戔戔,禮儀蹡□。樂只君子,利用賓王。



 鹿鳴相呼,聚澤之蒲。我樂嘉賓,鼓瑟吹竽。



 我命旨酒,以燕以娛。何以贈之?玄纁粲如。



 鹿鳴相邀,聚場之苗。我美嘉賓,令名孔昭。



 我命旨酒,以歌以謠。何以置之?大君之朝。



 鹿鳴相應,聚山之荊。我燕嘉賓,鼓簧吹笙。



 我命旨酒,
 以逢以迎。何以薦之?揚於王庭。



 右《鹿鳴》六章,章八句。



 瞻彼南陔,時物嘉良。有泉清泚,有蘭馨香。



 晨飲是汲,夕膳是嘗。慈顏未悅,我心靡遑。



 嬉嬉南陔,眷眷慈顏。和氣怡色,奉甘與鮮。



 事親是宜,事君是思。虔勖忠孝,邦家之基。



 右《南陔》二章,章八句。



 洋洋嘉魚,佇以美罛。君子有道,嘉寶式燕以娛。



 洋洋嘉魚,佇以芳罟。君子有德,嘉賓式歌且舞。



 我有宮沼,龜龍擾之。君子有禮,嘉賓式貴表之。



 我有宮藪,麟鳳來思。君子有樂,嘉賓式慰勤思。



 相彼嘉魚,爰縱之壑。我有旨酒,嘉賓式燕以樂。



 相彼嘉魚,在漢之粱。我有旨酒,嘉賓式燕以康。



 森森喬木,美蔓榮之。我有旨酒,嘉賓式燕宜之。



 喈喈黃鳥,載飛載止。我有旨酒,嘉賓式燕且喜。



 右《嘉魚》八章,章四句。



 崇丘峨峨,動植斯屬。高既自遂,大亦自足。



 和風斯扇,膏雨斯沐。我仁如天,以亭以育。



 崇丘巍巍,動植其依。高大之性,各極爾宜。



 王道坦坦,皇猷熙熙。仁壽之域,烝民允躋。



 右《崇丘》二章,章八句。



 關雎於飛,洲渚之湄。自家刑國,樂且有儀。



 鬱鬱芳蘭,幽人擷之。溫溫恭人,哲後求之。



 求之無斁,寤寐所屬。罄爾一心,受天百祿。



 鬱鬱芳蘭,雨露滋之。
 溫溫恭人,圭組縻之。



 鬱鬱芒蘭,佩服珍之。溫溫恭人,福履綏之。



 關雎蹌蹌,集水之央。好求賢輔,同揚德光。



 蘋蘩芳滋,同誰掇之。願言賢德,靡日不思。



 偶其賢德,輔成已職。永配玉音,服之無斁。



 潔其粢盛,中心匪寧。薦於宗廟,助君德馨。



 賢淑來思,人之表儀。風化天下,何樂如之!



 右《關雎》十章,章四句。



 彼鵲成巢,爾類攸處。之子有歸,瓊瑤是祖。



 彼鵲成
 巢,爾類攸匹。之子有行,錦繡是飾。



 彼鵲成巢,爾類攸共。之子有從,蘭蓀是奉。



 伊鵲成巢,珍禽戾止。婉彼佳人,配於君子。



 伊鵲營巢,珍禽攸處。內助賢侯,弼於明主。



 伊鵲營巢,珍禽輯睦。均養嘉雛,致於蕃育。



 右《鵲巢》六章,章四句。



 大觀聞喜宴六首



 狀元以下入門,《正安》



 多士濟濟,於彼西雍。欽肅威儀,
 亦有斯容。



 烝然來思,自西自東。天畀爾祿,惟王其崇。



 初舉酒,《賓興賢能》



 明明天子,率由舊章。思樂泮水,光於四方。



 薄採其芹,用賓於王。我有好爵,置彼周行。



 再酌,《於樂闢雍》



 樂只君子,式燕又思。服其命服,攝以威儀。



 鐘鼓既設,一朝酬之。德音是茂,邦家之基。



 三酌,《樂育英才》



 聖謨洋洋,綱紀四方。烝我髦士,觀國之光。



 遐不作人,而邦其昌。以燕天子,萬壽無疆。



 四酌,《樂且有儀》



 我求懿德,烝然來思。籩豆靜嘉,式燕
 綏之。



 溫溫其恭,莫不令儀。追琢其章,髦士攸宜。



 五酌,《正安》



 思皇多士,揚於王庭。鐘鼓樂之,肅邕和鳴。



 威儀抑抑,既安且寧。天子萬壽,永觀厥成。



 政和鹿鳴宴五首



 初酌酒,《正安》



 思樂泮水,承流闢雍。思皇多士,賁然來從。



 邕邕濟濟,四方攸同。登於天府,維王是崇。



 再酌,《樂育人才》



 鐘鼓皇皇,磬筦鏘鏘。登降維時,利用賓王。



 髦士攸宜,邦家之光。媚於天子,事舉言揚。



 三酌,《賢賢好德》



 鳴鹿呦呦,載弁俅俅。烝然來思,旨酒思柔。



 之子言邁,泮渙爾游。於彼西雍,對揚王休。



 四酌,《烝我髦士》



 首善京師,灼於四方。烝我髦士,金玉其相。



 飲酒樂曲,吹笙鼓簧。勉戒徒御,觀國之光。



 五酌,《利用賓王》



 遐不作人,天下喜樂。何以況之?鳶飛魚躍。



 既勸之駕,獻酬交錯。利用賓王,縻以好爵。



\end{pinyinscope}