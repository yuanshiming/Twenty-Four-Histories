\article{列傳第一百八十二}

\begin{pinyinscope}

 ○吳
 泳徐範李韶王邁史彌鞏陳塤子蒙趙與TP李大同黃TU楊大異



 吳泳,字叔永,潼川人。嘉定二年進士,歷官為軍器少監,
 行太府寺丞,行校書郎,升秘書丞兼權司封郎官,兼樞密院編修官,升著作郎,時暫兼權直舍人院。



 輪對,言:「願陛下養心,以清明約己,以恭儉進德,以剛毅發強,毋以旨酒違善言,毋以嬖御嫉壯士,毋以靡曼之色伐天性。杜漸防微,澄源正本,使君身之所自立者先有其地。夫然後移所留之聰明以經世務,移所舍之精神以強國政,移所用之心力以恤罷民,移所當省之浮費以犒邊上久戍之士,則不惟可以消弭災變,攘除奸兇,殄滅寇
 賊,雖以是建久安長治之策可也。」



 他日入對,又言:「誦往哲之遺言,進謀國之上策,實不過曰內修政事而已。然所謂內修者,非但車馬器械之謂也。袞職之闕,所當修也;官師之曠,所當修也;出令之所弗清,所當修也;本兵之地弗嚴,所當修也;直言敢諫之未得其職,所當修也;折沖禦侮之弗堪其任,所當修也。陛下退修於其上,百官有司交修於其下,朝廷既正,人心既附,然後申警國人,精討軍實,合內修外攘為一事,神州赤縣,皆在吾指
 顧中矣。」



 火災,應詔上封曰:「京城之災,京城之所見也。四方有敗,陛下亦得而見之乎?夫慘莫慘於兵也,而連年不戢,則甚於火矣。酷莫酷於吏也,而頻歲橫徵,則猛於火矣。閩之民困於盜,浙之民困於水,蜀之民困於兵。橫斂之原既不澄於上,包苴之根又不絕於下。譬彼壞木,疾用無枝,而內涸之形見矣。」



 遷秘書少監,兼權中書舍人,尋遷起居舍人兼權吏部侍郎,兼直學士院。疏言:「世之識治體而憂時幾者,以為天運將變矣,世道將降矣,
 國論將更矣,正人將引去而舊人將登用矣。執持初意,封植正論,茲非砥柱傾頹之時乎?若使廉通敏慧者專治財賦,淑慎曉暢者專御軍旅,明清敬謹者專典刑獄,經術通明使道訓典,文雅麗則使作訓辭,秉節堅厲使備風憲,奉法循理使居牧守,剛直有守者不聽其引去,恬退無競者不聽其里居,功名慷慨者不佚之以祠庭,言論闓爽者不置之於外服,隨才器使,各盡其分,則短長小大,安有不適用者哉!」又言謹政體、正道揆、厲臣節、
 綜軍務四事。



 權刑部尚書兼修玉牒,以寶章閣直學士知寧國府,提舉太平興國宮,進寶章閣學士,差知溫州。赴官,道間聞溫州饑,至處州,乞蠲租科降,救餓者四萬八千有奇,放夏稅一十二萬有奇,秋苗二萬八千有奇,病者復與之藥。事聞,賜衣帶鞍馬。改知泉州,以言罷。所著有《鶴林集》。



 徐範,字彞父,福州候官人。少孤,刻苦授徒以養母。與兄同舉於鄉,入太學,未嘗以疾言遽色先人。



 丞相趙汝愚
 去位,祭酒李祥、博士楊簡論救之,俱被斥逐。同舍生議叩閽上書,書已具,有閩士亦署名,忽夜傳韓侂胄將置言者重闢,閩士怖,請削名,範之友亦勸止之。範慨然曰:「業已書名矣,尚何變?」書奏,侂胄果大怒,謂其扇搖國是,各送五百里編管。範謫臨海,與兄歸同往,禁錮十餘年。



 登嘉定元年進士第。授清江縣尉,闢江、淮制置司準備差遣。屬邊事紛糾,營砦子弟募隸軍籍者未及涅,洶洶相驚。一夕,秉燭招刺千餘人,踴躍爭奮。差主管戶部架
 閣,改太學錄,遷國子監主簿。入對,言:「時平,不急之務、無用之官,猶當痛加裁節,矧多事之秋,所貴全萬民之命,紓一時之急,獨奈何坐視其無救而以虛文自蔽哉!願懲既往之失,廢無用之文,一意養民,以培國本。」



 丐外,添差通判澤州。湖湘大旱,振救多所裨益。知邵武軍,尋召赴行在,言:「功利不若道德,刑罰不若恩厚,雜伯不若純王,異端不若儒術,諛佞不若直諫,便嬖不若正人,奢侈不若詩書,盤游不若節儉,玩好不若宵衣旰食,窮黷不
 若偃兵息民。是非兩立,明白易見。幾微之際,大體所關。積習不移,治道舛矣。」遷國子監丞,徙太常丞,權都官郎官,改秘書丞、著作郎、起居郎、兼國史編修、實錄檢討。以朝奉大夫致仕。卒,贈朝請大夫、集英殿修撰。



 李韶,字元善,彌遜之曾孫也。父文饒,為臺州司理參軍,每謂人曰:「吾司臬多陰德,後有興者。」韶五歲,能賦梅花。嘉定四年,與其兄寧同舉進士。調南雄州教授。校文廣州,時有當國之親故私報所業,韶卻之。調慶元。丞相史
 彌遠薦士充學職,韶不與。袁燮求學宮射圃益其居,亦不與,燮以此更敬韶。



 以廉勤薦,遷主管三省架閣文字,遷太學正,改太學博士。上封事諫濟王竑獄,且以書曉彌遠,言甚懇到。又救太學生寧式,迕學官。丐外,添差通判泉州。郡守游九功素清嚴,獨異顧韶。改知道州。葺周惇頤故居,錄其子孫於學宮,且周其家。紹定四年,行都災,韶應詔言事。提舉福建市舶。會星變,又應詔言事。入為國子監丞,改知泉州兼市舶。



 端平元年,召。明年,轉太
 府寺丞,遷都官郎官,遷尚左郎官。未幾,拜右正言。奏乞以國事、邊防二事專委丞相鄭清之、喬行簡各任責。論汰兵、節財及襄、蜀邊防。又論史嵩之、王遂和戰異議,迄無成功,請出遂於要藩,易嵩之於邊面,使各盡其才。史宅之將守袁州,韶率同列一再劾之。俱不報。乞解言職,拜殿中侍御史,辭,不允。奏曰:「頃同臣居言職者四人,未逾月徐清叟去,未三月杜範、吳昌裔免,獨臣尚就列。清叟昨言『三漸』,臣繼其說,李宗勉又繼之,陛下初不加怒,
 而清叟竟去,猶曰清叟倡之也。今臣與範、昌裔言,未嘗不相表裏,二臣出臺,臣獨留,豈臣言不加切於二臣邪?抑先去二臣以警臣,使知擇而後言邪?清叟所言『三漸』,臣猶以為未甚切。今國柄有陵夷之漸,士氣有委靡之漸,主勢有孤立之漸,宗社有阽危之漸,上下偷安,以人言為諱,此意不改,其禍豈直三漸而已。」



 時魏了翁罷督予祠,韶訟曰:「了翁刻志問學,幾四十年,忠言讜論,載在國史,去就出處,具有本末。端平收召,論事益切。去年督
 府之遣,體統不一,識者逆知其無功。了翁迫於君命,黽勉驅馳,未有大闕,襄州變出肘腋,未可以為了翁罪。樞庭之召,未幾改鎮,改鎮未久,有旨予祠。不知國家四十年來收拾人才,燁然有稱如了翁者幾人?願亟召還,處以臺輔。」又劾奏陳洵益刑餘腐夫,粗通文墨,掃除賤隸,竊弄威權,乞予洵益外祠。劾女冠吳知古在宮掖招權納賄,宜出之禁庭。帝怒,韶還笏殿陛乞歸。會祀明堂,雷電,免二相,韶權工部侍郎、正言,遷起居舍人。復疏洵益、
 知古,不報。辭新命,不許。應詔上封事,幾數千言。帝諭左右曰:「李韶真有愛朕憂國之心。」凡三辭不獲,以生死祈哀乞去。帝蹙額謂韶曰:「曲為朕留。」退,復累疏乞補外,以集英殿修撰知漳州,號稱廉平。朝廷分遣部使者諸路稱提官楮,韶疏極言其敝。



 嘉熙二年,召。明年,上疏乞寢召命云:



 端平以來,天下之患,莫大於敵兵歲至,和不可,戰不能,楮券日輕,民生流離,物價踴貴,遂至事無可為。臣竊論以為必自上始,九重菲衣惡食,臥薪嘗膽,使上
 下改慮易聽,然後可圖。今二患益深,雖欲效忠,他莫有以為說。此其不敢進者一。



 史宅之,故相子,予郡,外議皆謂扳援之徒將自是復用,故嘗論列至再。今聖斷赫然,用舍由己,人才一變矣。環視前日在廷之臣,流落擯棄,臣雖欲貪進,未知所以處其身。此其不敢進者二。



 始臣為郎,蜀受兵方亟,廟堂已遣小使至,特起嵩之於家,而言者攻擊不已。臣妄論以為講和固非策,而首兵亦豈能無罪。故居言路,首乞出高論者付以兵事,使稍知敵
 情者嘗試其說於閫外。不知事勢推移,遂竟罷廢,而款敵無功者,白麻揚廷矣。或者將議臣前日有所附會。此臣重不敢進者三。



 又臣昨彈內侍女冠,不行,退惟聖主高明,必不容其干政。然未幾首相去位,臣亦出臺,傳聞其人謂臣受廟堂風旨,故決意丐外。今臣言迄不行,茍貪君命,竊恐或者譏臣向何所聞而去,今何所見而來。此臣重不敢進者四。



 四年,詔趣赴闕,辭,遷戶部侍郎,再辭,不許。五年,改禮部侍郎,辭,詔不允,令所在州軍護遣
 至闕。嵩之遣人謂詔曰:「毋言濟邸、宮媼、國本。」韶不答。上疏曰:「臣生長淳熙初,猶及見度江盛時民生富樂,吏治修舉。事變少異,政歸私門,紹定之末,元氣索矣。端平更化,陛下初意豈不甚美。國事日壞,其人或罷或死,莫有為陛下任其責者。考論至是,天下事豈非陛下所當自任而力為乎?《左氏》載史墨言:『魯公世從其失,季氏世修其勤。』蓋言所由來者漸矣。陛下臨御日久,宜深思熟念,威福自己,誰得而盜之哉?舍此不為,悠悠玩心妻,乃幾於《
 左氏》所謂『世從其失者。』」蓋以世卿風嵩之也。疏出,嵩之不悅,曰:「治《春秋》人下語毒」當是時,杜範亦在列,二人廉直,中外稱為「李、杜」。



 兼侍講,累辭,兼國史編修、實錄檢討,辭,遷吏部侍郎兼中書舍人,三辭,不許。淳祐二年,疏言:「道揆之地,愛善類不勝於愛爵祿,畏公議不勝於畏權勢。陛下以腹心寄之大臣,大臣以腹心寄之一二都司,恐不能周天下之慮。故以之用人,則能用其所知,豈能用其所不知;以之守法,則能守其所不與,必不能守於
 其所欲與。」又及濟王、國本、宮媼。三上疏乞歸,以寶章閣直學士知泉州,辭,乞畀祠,不許。既歸,三辭,仍舊職提舉鴻慶宮。



 淳祐五年,韶被召,再辭,詔本州通判勸勉赴闕。遷禮部侍郎,三辭,遷權禮部尚書,復三辭,不許。入見,疏曰:「陛下改畀正權,並進時望,天下孰不延頸以覬大治。臣竊窺之,恐猶前日也。君子小人,倫類不同。惟不計近功,不急小利,然後君子有以自見;不惡聞過,不諱盡言,然後小人無以自托。不然,治亂安危,反覆手爾。」



 又曰:「陛下
 所謀者嬪妃近習,所信者貴戚近親。按《政和令》:『諸國戚、命婦若女冠、尼,不因大禮等輒求入內者,許臺諫覺察彈奏。』乞申嚴禁廷之籍,以絕天下之謗。世臣貴戚,牽聯並進,何示人以不廣也。借曰以才選,他時萬一有非才者援是以求進,將何以抑之耶!」



 又曰:「今土地日蹙者未反,人民喪敗者未復,兵財止有此數,旦旦而理之,不過椎剝州縣,朘削裏閭。就使韓、白復生,桑、孔繼出,能為陛下強兵理財,何補治亂安危之數,徒使國家負不韙
 之名。況議論紛然,賢者不過茍容而去,不肖者反因是以媒其身,忠言至計之不行,淺功末利之是計,此君子小人進退機括所系,何不思之甚也!」



 又曰:「聞之道路,德音每下,昆蟲草木咸被潤澤,恩獨不及於一枯胔。威斷出,自公卿大夫莫敢後先,令獨不行於一老媼。小大之臣積勞受爵,皆得以延於世,而國儲君副,社稷所賴以靈長,獨不蚤計而豫定。」又疏乞還,不許。兼侍讀,三辭,不許。又三疏乞歸。



 時游似以人望用,然有牽制之者,韶奏
 云:「人主職論一相而已,非其人不以輕授。始而授之,如不得已,既乃疑之,反使不得有所為,是豈專任責成之體哉!所言之事不必聽,所用之人不必從,疑畏憂沮,而權去之矣。」擢翰林學士兼知制誥、兼侍讀,不拜,詔不許,又三辭,不許。



 嵩之服除,有鄉用之意,殿中侍御史章琰、正言李昂英、監察御史黃師雍論列嵩之甚峻,詔落職予祠。韶同從官抗疏曰:「臣等謹按《春秋》桓公五年書:『蔡人、衛人、陳人從王伐鄭。』春秋之初,無君無親者莫甚於
 鄭莊。二百四十二年之經,未有云『王伐國』者,而書『王』書『伐』,以見鄭之無王,而天王所當聲罪以致討。未有書諸侯從王以伐者,而書三國從王伐鄭,又見諸侯莫從王以伐罪,而三國之微者獨至,不足伸天王之義,初不聞以其嘗為王卿士而薄其伐。今陛下不能正奸臣之罪,其過不專在上,蓋大臣百執事不能輔天子以討有罪,皆《春秋》所不赦。乞斷以《春秋》之義,亟賜裁處。」詔嵩之勒令致仕。既而嵩之進觀文殿大學士,韶上疏爭之甚力。
 未幾,琰、昂英他有所論列,並罷言職。韶復上疏留之。



 七年,韶十上疏丐去,以端明殿學士提舉玉隆宮。時直學士院應人繇、中書舍人趙汝騰拜疏留韶內祠,未報。韶陛辭,疏甚剴切,其略曰:「彼此相視,莫行其志,而



 剸裁庶政,品量人物,相與運於冥冥之中者,不得不他有人焉。是中書之手可束,而臺諫之口可鈐,朝廷之事所當力為,不可枚舉,皆莫有任其責者,甚非所以示四方、一體統。」改提舉萬壽觀兼侍讀,即出國門,力辭,道次三衢,詔趣
 受命,再辭,仍奉祠玉隆。



 八年,被召,辭,不許。再辭,仍舊職奉祠萬壽兼侍讀,令守臣以禮趣行。又辭,不許。九年,仍奉祠玉隆。十一年,祠滿再任。卒,年七十五。韶忠厚純實,平粹簡淡,不溺於聲色貨利,默坐一室,門無雜賓云。



 王邁字貫之,興化軍仙游人。嘉定十年進士,為潭州觀察推官。丁內艱,調浙西安撫司干官。考廷試,詳定官王元春欲私所親置高第,邁顯擿其繆,元春怒,嗾諫官李知孝誣邁在殿廬語聲高,免官。



 調南外睦宗院教授。真
 德秀方守福州,邁竭忠以裨郡政。赴都堂審察,丞相鄭清之曰:「學官掌故,不足浼吾貫之。」俄召試學士院,策以楮幣,邁援據古今,考究本末,謂:「國貧楮多,弊始於兵。乾、淳初行楮幣,止二千萬,時南北方休息也。開禧兵興,增至一億四千萬矣。紹定有事山東,增至二億九千萬矣。議者徒患楮窮,而弗懲兵禍,姑以今之尺籍校之,嘉定增至二十八萬八千有奇。用寡謀之人,試直突之說,能發而不能收,能取而不能守。今無他策,核軍實,窒邊釁,
 救楮幣第一義也。」又言:「修內司營繕廣,內帑宣索多,厚施緇黃,濫予嬪御,若此未嘗裁撙,徒聞有括田、榷鹽之議者。向使二事可行,故相行之久矣。更化伊始,奈何取前日所不屑行者而行之乎?」又因楮以及時事,言:「君子之類雖進,而其道未行;小人之跡雖屏,而其心未服。」真德秀病危,聞邁所對,善之。



 帝再相喬行簡,或傳史嵩之復用,邁上封事曰:「天下之相,不與天下共謀之,是必冥冥之中有為之地者。且舊相奸憸刻薄,天下所知,復用,
 則君子空於一網矣。」又言吳知古、陳洵益撓政。輪對,言:「君不可欺天,臣不可欺君,厚權臣而薄同氣,為欺天之著。」邁由疏遠見帝,空臆無隱,帝為改容。言者劾邁論邊事過實,魏了翁侍經筵,為帝言惜其去,改通判漳州。禋祀雷雨,邁應詔言:「天與寧考之怒久矣。曲蘗致疾,妖冶伐性,初秋逾旬,曠不視事,道路憂疑,此天與寧考之所以怒也。隱、刺覆絕,攸、熹尊寵,綱淪法斁,上行下效,京卒外兵,狂悖迭起,此天與寧考之所以怒也。陛下不是之
 思,方用漢災異免三公故事,環顧在廷,莫知所付。遙相崔與之,臣恐與之不至,政柄他有所屬,此世道否泰,君子小人進退之機也。」於是臺官李大同言邁交結德秀、了翁及洪咨夔以收虛譽,削一秩免。蔣峴劾邁前疏妄論倫紀,請坐以非所宜言之罪,削二秩。久之,復通判贛州,改福州、建康府、信州,皆不行。淳祐改元,通判吉州。右正言江萬里袖疏榻前曰:「邁之才可惜,不即召,將有老不及之嘆。」帝以為然。有尼之者,遂止。



 知邵武軍。在郡,詔
 以亢旱求言,邁驛奏七事,而以徹龍翔宮、立濟王後為先。時鄭清之再相,以左司郎官召,力辭。以直秘閣提點廣東刑獄,亦辭,改侍右郎官,諫官焦炳炎論罷。予祠,卒,贈司農少卿。



 邁以學問詞章發身,尤練世務。易祓戒潭人曰:「此君不可犯。」奪勢家冒占田數百畝以還民。李宗勉嘗論邁,然邁評近世宰輔,至宗勉,必曰「賢相」。徐清叟與邁有違言,邁晚應詔,謂清叟有人望可用。世服其公云。



 史彌鞏,字南叔,彌遠從弟也。好學強記。紹熙四年,入太學,升上舍。時彌遠柄國,寄理不獲試,淹抑十載。嘉定十年,始登進士第。



 時李𡌴開鄂閫,知彌鞏持論不阿,闢咨幕府事。壽昌戍卒失律,欲盡誅其亂者,乃請誅倡者一人,軍心感服。改知溧水縣,首嚴庠序之教。端平初,入監都進奏院。轉對,有君子小人才不才之奏,護蜀保江之奏。嘉熙元年,都城火,彌鞏應詔上書,謂修省之未至者有五。又曰:「天倫之變,世孰無之。陛下友愛之心亦每發
 見。洪咨夔所以蒙陛下殊知者,謂霅川之變非濟邸之本心,濟邸之死非陛下之本心,其言深有以契聖心耳,矧以先帝之子,陛下之兄,乃使不能安其體魄於地下,豈不干和氣,召災異乎?蒙蔽把握,良有以也。」



 出提點江東刑獄。歲大旱,饒、信、南康三郡大侵,謂振荒在得人,俾厘戶為五,甲乙以等第糶,丙為自給,丁糴而戊濟,全活為口一百一十四萬有奇。徽之休寧有淮民三十餘輩,操戈劫人財,逮捕,法曹以不傷人論罪。彌鞏曰:「持兵
 為盜,貸之,是滋盜也。」推情重者僇數人,一道以寧。饒州兵籍溢數,供億不繼,請汰冗兵。令下,營門大噪。乃呼諸校謂曰:「汰不當,許自陳,敢嘩者斬。」咸叩頭請罪,諸營帖然,稟給亦大省。召為司封郎中,以兄子嵩之入相,引嫌丐祠,遂以直華文閣知婺州。時年已七十,丐祠,提舉崇禧觀。里居絕口不道時事。卒,年八十。真德秀嘗曰:史南叔不登宗袞之門者三十年,未仕則為其寄理,已仕則為其排擯,皭然不污有如此。



 五子,長肯之,終刑部郎官,
 能之、有之、胄之俱進士。TV肯之子蒙卿,咸淳元年進士,調江陰軍教授,蚤受業色川陽恪,為學淹博,著書立言,一以朱熹為法。



 陳塤,字和仲,慶元府鄞人。大父叔平與同郡樓鑰友善,死,鑰哭之。塤才四歲,出揖如成人。鑰指盤中銀杏使屬對,塤應聲曰:「金桃。」問何所據?對以杜詩「鸚鵡啄金桃。」鑰竦然曰:「亡友不死矣。」長受《周官》於劉著,頃刻數千百言輒就。試江東轉運司第一,試禮部復為第一。



 嘉定十年,
 登進士第。調黃州教授。喪父毀瘠,考古禮制時祭、儀制、祭器行之。忽嘆曰:「俗學不足學。」乃師事楊簡,攻苦食淡,晝夜不怠。免喪,史彌遠當國,謂之曰:「省元魁數千人,狀元魁百人,而恩數逾等,盍令省元初授堂除教授,當自君始。」塤謝曰:「廟堂之議甚盛,舉自塤始,得無嫌乎?」徑部注處州教授以去,士論高之。



 理宗即位,詔求言,塤上封事曰:「上有憂危之心,下有安泰之象,世道之所由隆。上有安泰之心,下有憂危之象,世道之所由污。故為天下
 而憂,則樂隨之。以天下為樂,則憂隨之。有天下者,在乎善審憂樂之機而已。今日之敝,莫大於人心之不合,紀綱之不振,風俗之不淳,國敝人偷而不可救。願陛下養之以正,勵之以實,蒞之以明,斷之以武。」而塤直聲始著於天下。與郡守高似孫不合,去,歸奉其母。召為太學錄,逾年始至。轉對,言:「天道無親,民心難保。日月逾邁,事會莫留。始之銳,久則怠。始之明,久則昏。垂拱仰成,盛心也,不可因以負有為之志。遵養時晦,至德也,不可因以失
 乘時之機。」上嘉納之。遷太學博士,主宗正寺簿。都城火,塤步往玉牒所,盡藏玉牒於石室。詔遷官,不受。應詔言應上天非常之怒者,當有非常之舉動,歷陳致災之由。又有吳潛、汪泰亨上彌遠書,乞正馮榯、王虎不盡力救火之罪,及行知臨安府林介、兩浙轉運使趙汝憚之罰。人皆壯之。



 遷太常博士,獨為袁燮議謚,餘皆閣筆,因嘆曰:「幽、厲雖百世不改,謚有美惡,豈諛墓比哉?」會朱端常子乞謚,塤曰:「端常居臺諫則逐善類,為藩牧則務刻剝,
 宜得惡謚,以戒後來。」乃謚曰榮願。議出,宰相而下皆肅然改容。考功郎陳耆覆議,合宦者陳洵益欲改,塤終不答。



 李全在楚州有異志,塤以書告彌遠:「痛加警悔,以回群心。蚤正典刑,以肅權綱。大明黜陟,以飭政體。」不納。未幾,賈貴妃入內,塤又言:「乞去君側之蠱媚,以正主德;從天下之公論,以新庶政。」彌遠召塤問之曰:「吾甥殆好名邪?」塤曰:「好名,孟子所不取也。夫求士於三代之上,惟恐其好名;求士於三代之下,惟恐其不好名耳。」力丐去,添
 差通判嘉興府。彌遠卒,召為樞密院編修官。入對,首言:「天下之安危在宰相。南渡以來,屢失機會。秦檜死,所任不過萬俟禼、沈該耳。侂胄死,所任史彌遠耳。此今日所當謹也。」次言:「內廷當嚴宦官之禁,外廷當嚴臺諫之選。」於是洵益陰中之,監察御史王定劾塤,出知常州,改衢州。



 寇卜日發漈坑,遵江山縣而東。塤獲諜者,即遣人致牛酒諭之曰:「汝不為良民而為劫盜,不事耒耜而弄甲兵,今享汝牛酒,冀汝改業,否則殺無赦。」於是自首者日
 以百數,獻器械者重酬之,遂以潰散。改提點都大坑冶,徙福建轉運判官。侍御史蔣峴常與論《中庸》,不合,又劾之。主管崇道觀。逾年,遷浙西提點刑獄。歲旱,盜起,捕斬之,盜懼徙去。安吉州俞垓與丞相李宗勉連姻,恃勢黷貨,塤親按臨之。弓手戴福以獲潘丙功為副尉,宗勉倚之為腹心,盜橫貪害,塤至,福聞風而去。貽書宗勉曰:「塤治福,所以報丞相也。傳間實走丞相,賢輔弼不宜有此。」宗勉答書曰:「福罪惡貫盈,非君不能治。宗勉雖不才,不
 敢庇奸兇。惟君留意。」及獲福豫章,眾皆欲殺之,塤曰:「若是則刑濫矣。」乃加墨徇於市,囚之圜土。以吏部侍郎召,及為國子司業,諸生咸相慶,以為得師。



 未幾,兼玉牒檢討、國史編修、實錄修撰,乃辭兼史館。歷陳境土之蹙,民生之艱,國計之匱,「既無經理圖回之素,惟有感動轉移之策,必有為之本者,本者何?復此心之妙耳」。又言:「履泰安而逸樂者,有習安致危之理。因艱危而克懼者,有慮危圖安之機。明用舍以振紀綱,躬節儉以汰冗濫,屏奸
 妄以厲將士,抑貴近以寬糶,結鄉社以防竊發,黜增創以培根本。今任用混殽,薰蕕同器,遂使賢者恥與同群。」諫議大夫金淵見之,怒。塤乞補外,不許,又辭免和糴轉官賞,亦不許。知溫州,未上,以言罷。



 塤家居,時自娛於泉石,四方學者踵至。輕財急義,明白洞達,一言之出,終身可復。忽臥疾,戒其子抽架上書占之,得《呂祖謙文集》,其《墓志》曰:「祖謙生於丁巳歲,沒於辛丑歲。」塤曰:「異哉!我生於慶元丁巳,今歲在辛丑,於是一甲矣。吾死矣夫!」



 子
 蒙,年十八,上書萬言論國事。吳子良奇之,妻以女。為太府寺主簿。入對,極言賈似道為相時國政闕失,文多不錄。為淮東總領,似道誣以貪污,貶建昌軍簿,錄其家,惟青氈耳。德祐初,禮部侍郎李玨乞放便,以刑部侍郎召,不赴,卒。



 趙與TP,字德淵,太祖十世孫。居湖州。嘉定十三年進士。歷官差主管官告院,遷將作監主簿,差知嘉興府,遷知大宗正兼權樞密院檢詳諸房文字,尋為都官郎官,加
 直寶章閣、兩浙轉運判官。進煥章閣、知慶元府,主管沿海制置司公事,拜司農少卿,仍兼知慶元府兼沿海制置副使。遷浙西提點刑獄,授中書門下省檢正諸房公事,拜司農卿兼知臨安府,主管浙西安撫司公事,權刑部侍郎兼詳定敕令官,權兵部侍郎,遷戶部侍郎,權戶部尚書,時暫兼吏部尚書,尋為真,兼戶部尚書,時暫兼浙西提舉常平,加端明殿學士、提領戶部財用,皆依舊兼知臨安府。與執政恩澤,加資政殿大學士。以觀文殿
 學士知紹興府、浙東安撫使;知平江府兼淮、浙發運使,時暫兼權浙西提點刑獄;授沿江制置使,知建康府、江東安撫使、馬步軍都總管兼行宮留守,節制和州、無為軍、安慶府三郡屯田使;時暫兼權揚州、兩淮安撫制置使,改兼知揚州,尋兼知鎮江府,兼淮東總領,提舉洞霄宮;復為淮、浙發運使,差知平江府,特轉兩官致仕。景定元年八月,卒,特贈少師。與TP所至急於財利,幾於聚斂之臣矣。



 李大同,字從仲,婺州東陽人。嘉定十六年進士。歷官為秘書丞兼崇政殿說書,拜右正言兼侍講。疏言:「趙、冀分野,乃有熒惑犯填星之變,則我師之出,豈無當長慮而卻顧者。故臣願陛下勿以星文為小異而或加忽。一話一語,一政一事,必求有以格天心而弭災變。至於進兵攻討,尤切謹重。」遷太常少卿兼國史編修、實錄檢討,兼侍講,兼權侍立修注官,遷起居郎,拜殿中侍御史,權刑部侍郎兼同修國史、實錄院同修撰,選吏部侍郎,進工
 部尚書,以寶謨閣直學士知平江府,提舉江州太平興國宮。乞致仕,不許,後卒於家。



 黃TU,字子耕,隆興分寧人。嘗從郭雍、朱熹學,熹深期之,而TU亦以道自任,反復論辨,必無所疑然後止。舉太學進士,為瑞昌主簿,監文思院,知盧陽縣,五溪獠獷悍,TU為詩諭之,獠感悅,有公事莫敢違。



 通判處州,經、總制有額無錢,俗號殿最綱,TU會十年中成賦酌取之,閣免逋負,錢額鈞等,獨以最聞。主管官告院、大理寺簿、軍器監
 丞,歲餘三遷,TU乃不樂。間行西湖,慨然曰:「我昔在南、北山,一水一石,無不自題品,今無復情味,何邪?」



 丐外,知臺州。謝良佐子孫居臺者既播越流落,TU求之民間,收而教之。勤苦夙夜,先勸後禁,訟牒銷縮,郡稱平治。為濟糶倉,為抵當庫,葬民之棲寄暴露者為棺千五百,置養濟院,又創安濟坊以居病囚,皆自有子本錢,使不廢。故葉適謂TU條目建置,憂民如家。遷袁州,哭從弟哀甚,得疾卒。所著有《復齋集》。



 楊大異,字同伯,唐天平節度使漢公之後,十世祖祥避地醴陵,因家焉。祥事親孝,親亡哀毀,泣盡繼以血,廬墓終身,有白芝、白烏、白兔之瑞。事聞於朝,褒封至孝公,賜名木植墓道,以旌其孝。大異從胡宏受《春秋》大義。登嘉定十三年進士第。授衡陽主簿,有惠政。經龍泉尉,攝邑令。適歲饑,提刑司遣吏和糴米二萬石於邑,米價頓增,民乏食,大異即以提刑司所糴者如價發糶,民甚德之。提刑趙與TP大怒,捃其罪弗得,坐以方命,移安遠尉。



 邑
 有峒寇擾民,官兵致討,積年弗獲,檄大異往治之。大異以一僕負告身自隨,肩輿入賊峒,傳呼尉至,賊露刃成列以待,徐諭以禍福,皆伏地叩頭,願改過自新。留告身為質,偕其渠魁數輩出降。以賞遷吉州戶曹,改廣西經乾,復以弭盜賞,除四川制置司參議官。北兵入成都,大異從制置使丁黼巷戰,兵敗,身被數創死,闔門皆遇難。詰旦,其部曲竊往瘞之,大異復蘇,負以逃,獲免。進朝奉郎,宰石門縣,就除通判溧陽,攝州事,皆有惠政。去官之
 日,老弱攀號留之,大異易服潛去。擢知登聞鼓院,遷大理寺丞,平反冤獄者七。召對,極言時政得失,迕宰相意,出知澧州。理宗曰:「是四川死節更生者楊大異耶?論事剴切,有用之材也。何遽出之?」對曰:「是人尤長於治民。」命予節兼庾事,進直秘閣、提點廣東刑獄兼庾事。



 時常平司逋負山積,械系追索,奸蠹百出。大異與之約,悉縱遣之,負者如期畢輸,吏無所容其奸。訪張九齡曲江故宅,建相江書院,以祀九齡。改提點廣西刑獄兼漕、庾二司,
 所至奸吏屏息,寇盜絕跡。凡可以為民興利除害者,必奏行之。復建宣成書院祀張栻、呂祖謙。廣海幅員數千里,道不拾遺,報政為最。未六十即丐致仕,不允,章四上,除秘閣修撰、太中大夫,提舉崇禧觀、醴陵縣開國男,食邑三百戶,賜紫金魚袋。歸里第,與居民無異,學者從之,講肄諄諄,相與發明經旨,條析理學。食祠祿者二十四年,卒,年八十二。子霆、霖。霆在《忠義傳》。



 論曰:正論之在天下,未嘗亡也。徐範之於韓侂胄,吳泳、
 李韶、王邁之於史氏,皆能無所回撓,正色直言。至於史彌鞏則彌遠之弟,陳塤其甥也,不以私親而廢天下之公論。抑孟子所謂「寡助之至」者歟?趙與TP揚歷最久,甘為聚斂之臣。李大同以鄉人喬行簡為相,薦起之。黃TU出仕,以恤民尊賢為急,可謂知本。大異節義如此,宜其善政之著稱於世也。



\end{pinyinscope}