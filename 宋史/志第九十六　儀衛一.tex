\article{志第九十六 儀衛一}

\begin{pinyinscope}

 殿庭立仗



 綦天下之貴,一人而已。是故環拱而居,備物而動,文謂之儀,武謂之衛。一以明制度,示等威;一以慎出入,遠危疑也。《書》載弁戈、冕劉、虎賁、車輅,王出入,執盾
 以夾王車。朝儀之制,固已粲然。降及秦、漢,始有周廬、陛戟、鹵簿、金根、大駕、法駕千乘萬騎之盛。歷代因之,雖或損益,然不過為尊大而已。宋初,因唐、五代之舊,講究修葺,尤為詳備。其殿庭之儀,則有黃麾大仗、黃麾半仗、黃麾角仗、黃麾細仗。凡正旦、冬至及五月一日大朝會,大慶、冊、受賀、受朝,則設大仗;月朔視朝,則設半仗;外國使來,則設角仗;發冊授寶,則設細仗。其鹵簿之等有四:一曰大駕,郊祀大饗用之;二曰法駕,方澤、明堂、宗廟、籍田
 用之;三曰小駕,朝陵、封祀、奏謝用之;四曰黃麾仗,親征、省方還京用之。南渡之後,務為簡省。此其大較也。若夫臨時增損,用置不同,則有國史、會要、禮書具在。今取所載。撮其凡為《儀衛志》。



 殿庭立仗,本充庭之制。唐禮,殿庭、屯門,皆列諸衛黃麾大仗。宋興,太祖增創錯繡諸旗並幡氅等,著於《通禮》,正、至、五月一日,御正殿則陳之。青龍、白虎旗各一,分左右;五岳旗五在左,五星旗五在右;五方龍旗二十五在左,
 五方鳳旗二十五在右;紅門神旗二十八,分左右;朱雀、真武旗各一,分左右;皂纛十二,分左右。



 以上金吾。



 天一、太一旗各一,分左右;攝提旗二,分左右;五辰旗五,北斗旗一,分左右;木、火、北斗在左,金、水、土在右。



 二十八宿各一,角宿至壁宿在左,奎宿至軫宿在右。



 風伯、雨師旗各一,分左右;白澤、馴象、仙鹿、玉兔、馴犀、金鸚鵡、瑞麥、孔雀、野馬、犛牛旗各二,分左右;日月合璧旗一,在左;五星連珠旗一,在右;雷公、電母旗各一,分左右;軍公旗六,分左右;黃鹿、飛麟、兕、騶牙、白狼、蒼烏、闢邪、
 網子、貔旗各二,分左右;信幡二十二,分左右;傳教、告止幡各十二,分左右;黃麾二,分左右。



 以上兵部。



 日旗、月旗各一,分左右;君王萬歲旗一在左;天下太平旗一,在右;獅子旗二,分左右;金鸞、金鳳旗各一,分左右;五方龍旗各一。



 青、赤在左,黃、白、黑在右。以上龍墀。



 龍君、虎君旗各五,分左右;赤豹、黃羆旗各五,分左右;小黃龍旗一,在左;天馬旗一,在右;吏兵、力士旗各五,分左右;天王旗四,分左右;太歲旗十二,分左右;天馬旗六,分左右;排欄旗六十,分左右;左右幡氅
 各五行,行七十五;大黃龍旗二,分左右;大神旗六,分左右。



 以上六軍。



 神宗元豐二年,詳定所言:「正旦御殿,合用黃麾仗。案唐《開元禮》,冬至朝會及皇太子受冊、加元服,冊命諸王大臣,朝宴外國,亦皆用之。故事,皇帝受群臣上尊號,諸衛各率其屬,勒所部屯門、殿庭列仗衛。今獨修正旦儀注,而餘皆未及。欲乞冬會等儀,悉加詳定。」詔從之。又言:「御殿儀仗,有黃麾幡三而無黃麾。請制大麾一,注旄於乾首,以取夏制;黃色,以取漢制;用十二幅,以取唐
 制;用一旒,以取今龍墀旗之制。建於當御廂之前,以為表識。其當御廂之後,則建黃麾幡二。」上謂蔡確等曰:「黃麾制度,終有可疑。今鑿而為植於大庭,夷夏共瞻,或致博聞多識者譏議,非善,宜姑闕之。」乃止。三年,詳定所言:「昨定朝會圖,於大慶殿橫街北止陳大輦、逍遙、平輦,而輿未陳也。當大輦之南,增腰輿一,小輿一。古者扇翣,皆編次雉羽或尾為之,故於文從『羽』。唐《開元》改為孔雀,凡大朝會,陳一百五十有六,分居左右。國朝復雉尾之名,
 而四面略為羽毛之形,中繡雙孔雀,又有雙盤龍扇,皆無所本。」遂改制偏扇、團方扇為三等,繡雉。凡朝會,平輦、逍遙並陳於東西龍墀上。



 徽宗政和三年,議禮局上大慶殿大朝會儀衛:



 黃麾大仗五千七十五人。仗首左右廂各二部,絳引幡十。執各一人。



 第一部,左右領軍衛大將軍各一員,第二部,左右領軍衛折沖,掌鼓一人,帥兵官一十人。次執儀刀部十二行,每行持各十人:後部並仗同。



 第一行,黃雞四角氅;凡氅,皆持以龍頭竿。



 第二,儀鍠五色幡;第三,青
 孔雀五角氅;第四,烏戟;第五,緋鳳六角氅;第六,細弓矢;第七,白鵝四角氅;第八,朱縢絡盾刀;第九,皂鵝六角氅;第十,細弓矢;第十一,槊;第十二,綠縢絡盾刀。揭鼓二,掌鼓二人。



 後部同。



 以上排列左右廂。第一部各於軍員之南,居次廂第一部稍前。第二部於第一部之後,並相向。



 次廂左右各三部:第一,左右屯衛;第二,左右武衛,並大將軍;第三,左右衛將軍:各一員。第一,果毅;第二、第三,折沖:各一員。於仗首左右廂第一部之南,相向。持黃麾幡二人,
 在當御廂前分立。當御廂左右各一部,左右衛果毅各一人,於玉輅之前分左右,並北向。



 次後廂左右各三部:第一,左右驍衛將軍;第二,左右領軍衛折沖;第三,左右領軍衛果毅:各一員。第一部,分於當御廂之左右差後;第二部,左在金輅之後西偏,右在象輅之後東偏;第三部,左在革輅之後西偏,右在木輅之後東偏,並北向。



 次左右廂各三部:第一,左右武衛將軍;第二,左右屯衛將軍;第三,左右領軍衛折沖:各一員。各在網子、鶡雞、貔旗
 之前,東西相向。左右廂各步甲十二隊:第一隊,左右衛果毅;第二,左右衛,第四,左右驍衛,第六,左右武衛,第八,左右屯衛,第十、第十二,左右領軍衛,並折沖;第三,左右驍衛,第五,左右武衛,第七,左右屯衛,第九、第十一,左右領軍衛,並果毅:各一員。每隊旗一,貔、鶡雞、仙鹿、金鸚鵡、瑞麥、孔雀、野馬、犛牛、甘露、網子。



 內第十二隊旗同第一隊。



 刀盾、弓矢相間,分十二隊,每隊三十人,五重。第一至第六隊,在仗首第二部北;第七至第十二隊,在仗首第二部南,東西
 相向。



 左右廂後部各十二隊:第一、第二,左右衛;第五至第七,左右武衛;第十至第十二,左右領軍衛:並折沖。第三、第四,左右驍衛;第八、第九,左右屯衛:並果毅。每隊旗二,角、赤熊、兕、太平、馴犀、鵔鸃、轆騶、騶牙、蒼烏、白狼、龍馬、金牛。次弩五人為一列,弓矢十人為二重,槊二十人為四重。以上在大慶殿門外,第一至第四隊在前,第五至第八隊在後,第九至第十二隊又在後,東西相向。



 真武隊:金吾折沖都尉一員,仙童、真武、螣蛇、神龜旗各一,執各一人。



 犦槊二人,弩五人為一列,弓矢二十人為四重,槊二十五人為五重。以上在大慶門外中道,北向排列。



 殿中省尚輦:陳孔雀扇四十於簾外。執各一人。



 陳輦輿於龍墀。大輦在東部,押、執、擎人二百二十有二人;腰輿在南,一十有七人;小輿又在南,二十有五人,皆西向。平輦在西,逍遙在南,共三十七人,皆東向。設傘,扇於沙墀:方傘二,分左右;執傘將校四人。



 團龍扇四,分左右;執扇都將四人。



 方雉扇一百,分傘、扇之後,為五行。執扇長行一百人。



 押當職掌二人,各立團龍
 扇之北。金吾引駕官二人,分立團扇之南。



 文德殿入合之制,唯殿中省細仗,與兩省供奉官班於庭。太宗淳化三年,增黃麾仗二百五十人。神宗熙寧三年,修合門儀制宋敏求言:「本朝惟入合乃御文德殿視朝。今既不用入合儀,即文德殿遂闕視朝之禮。乞下兩制及太常禮院,約唐御宣政殿制裁定,以備朔望正衙視朝之禮。」詔學士院詳定。學士韓維等上其儀:朔前一日,有司供張於文德殿庭。東面,左金吾引駕官一人,四色官二人,各
 帶儀刀。被金甲天武官一人,判殿中省一人,排列官一人。扇二,方傘一。金吾仗碧襴十二,各執儀刀。兵部儀仗排列職掌一人,押隊員僚二人。黃麾幡一,告止幡、傳教幡、信幡各八,龍頭竿、戟各五十。西面,右金吾引駕官以下,皆如東面。天武官東西總百人。門外立仗:其東,青龍旗一,五岳旗五,五龍旗十;其西,白虎旗一,五星旗五,五鳳旗十。御馬,東西皆五匹,每匹人員二人,御龍官四人。設御幄於殿後合。其日,左右金吾將軍常服押本衛仗,
 殿中省官押細仗,東西對列,俟皇帝受朝、降坐、放仗,乃退。



 徽宗政和三年,議禮局上文德殿視朝之制:



 黃麾半仗,共二千二百六十五人。殿內仗首,左右廂各一部,每部一百二十四人,在金吾仗南,東西相向。絳引幡十,執各一人。



 分部之南北,為五重。當御廂左右部同,左部在帥兵官東,右部在帥兵官西,各為十重。左右領軍衛大將軍各一員,居部之中。



 次廂左右第一、第二、第三部同。



 掌鼓一人,次大將軍後。次廂左右第一部並當御廂左右部,次果毅,次廂左右第二、第三部,次折沖,次後廂左右部,次將軍。



 帥
 兵官十人,分部之南北,為五重,北在絳引幡之南,南在絳引幡之北。次廂左右第一、第二、第三部在部之南北,當御廂、次後廂左部在黃氅東,右部在氅西。



 執儀刀部十行,行十人,每色兩行,為五重。



 次廂左右第一、第二、第三部同。當御廂、次後廂左右部,每色一行,為十重。左部以東為首,右部以西為首,並次帥兵官。



 第一行,龍頭竿黃雞四角氅;凡氅皆持以龍頭竿。



 第二,儀鍠五色幡;第三,青孔雀五角氅;第四,烏戟;第五,緋鳳六角氅;第六,細弓矢;第七,白鵝四角氅;第八,朱縢絡盾刀;第九,阜鵝六角氅;第十,
 槊。揭鼓二,掌揭鼓二人。



 分立緋氅、烏戟後當中,次廂左右第一、第二、第三部同,當御廂、次後廂並一在儀鍠、青氅間,一在弓矢、白氅間,與後行齊。



 次廂左右各三部,每部一百一十五人,次左右廂仗首之南,東西相向。第一部,左右屯衛大將軍及果毅各一員。第二部,左右武衛大將軍,第三部,左右衛將軍各一員,折沖各一員。黃麾幡二,分立當御左右廂前中間,北向。當御廂左右各一部,每部一百二十四人,在殿門內中道,分東西,並北向。



 次後廂左右部同。大慶殿列於樂架之南。



 左右衛果毅各一員。左在部西,右在部東。次後左右廂將
 軍準此。



 次後廂左右各一部,每部一百一十四人,次當御廂南,左右驍衛將軍各一員。左右廂各步軍六隊,第一隊,每隊三十三人,第二至第六隊,每隊各二十七人。



 分東西,在仗隊後。第一,左右衛;第三,左右武衛;第五,左右領軍衛:並果毅,各一員。第二,左右驍衛;第四,左右屯衛;第六,左右領軍衛:並折沖,各一員。每隊旗二,貔、金鸚鵡、瑞麥、犛牛、甘露、鶡雞。



 執各一人。



 刀盾、弓矢相間,人數行列同前。左右廂步軍,殿門外左右廂後部各六隊,每隊三十八人,在部下親從後,東西相
 向。第一隊,左右衛;第三,左右武衛;第五,左右領軍衛:並折沖,各一員,第二,左右驍衛;第四、第六,左右屯衛:並果毅,各一員。角、太平、馴犀、騶牙、白狼、蒼烏等旗各二,弩五人,為一列,弓矢十人,為二重,槊二十人,為四重。



 真武隊五十七人,在端禮門內中道,北向。大慶殿於殿門外。



 前有金吾折沖都尉一員,仙童、真武、螣蛇、神龜等旗各一,犦槊二人,弩五人為一列,弓矢二十人為四重,槊二十五人為五重。排列仗隊職掌六人,分立仗隊之間,殿內四人,
 殿外二人。



 殿中省尚輦陳扇二十於簾外,執扇殿侍二十人。陳腰輿、小輿於東、西朵殿,腰輿在東,小輿在西,人員、都將各一人,輦官共四十人。陳傘、扇於殿下,方傘二,團龍扇四,並分左右夾傘。



 執扇各一人,將校或節級。



 方雉扇六十,作三重,在傘、扇之後。輦官長行各一人,金吾左右將軍各一員,在傘、扇之南,稍前。四色官四人,二人立於將軍之南,與傘、扇一列。宣敕放仗二人,在引駕官南。執儀刀引駕官二人,在親從官後。長行二十四人,在四色官之南。
 排列官二人,在長行之南。次金甲天武官二人,在長行南。以上並分東西相向立。設旗於殿門之外,青龍旗一在左,五嶽神旗各一次之,五方色龍旗各一次之,五方色龍旗各一又次之。白虎旗一在右,五星神旗各一次之,五方色鳳旗各一次之,五方色鳳旗各一又次之。



 詔頒行之。大慶殿冊命諸王、大臣,黃麾仗準文德殿視朝。



 政和中,大祀饗立仗:大黃龍負圖旗一,執絲斥二百人,陳於闕庭赤龍旗南少西大黃龍旗之北。宣和冬祀,陳於
 大內前。大黃龍旗一,執絲斥六十人,陳於逐頓宮門外宣德門,次大黃龍負圖旗之南。宣和,此旗下又有日、月、五星連珠、北斗、招搖、蒼龍、白虎、朱雀、玄武、君王萬歲、獅子、金鸞、金鳳、五方龍、天下太平等旗,凡二十一。正、至受朝同。龍墀旗陳於殿庭;太廟,在西欞星門外路南,次赤龍旗少北;青城,在泰禋門外,夏祭大禮在明禋門外。赤龍旗之南。



 宗祀祫饗大禮,不設大黃龍負圖旗、大黃龍旗。



 大神旗六,執絲斥各九十人,宣德門、泰禋門並陳於大黃龍旗之南,東西相望;大
 廟陳於西欞星門外,大黃龍旗之西少南,視赤龍旗為列,南北相望。龍墀旗執絲斥各十二人,左右有日、月旗各一。次君王萬歲旗一,宣德門、泰禋門,在路東;太廟,在門外路南。次獅子旗二,左右有金鸞、金鳳旗各一。次五方龍旗各一:青、黃、赤龍旗,宣德、泰禋門在東,太廟在南;黑、白龍旗,宣德、泰禋門在西,太廟在北。次天下太平旗一,宣德、泰禋門,在路西;太廟,在路北。以上旗皆在車駕前發仗內。執絲斥人並錦帽、五色絁繡寶相花衫、鐵臂鞁、
 革帶。



 政和中,遼使朝紫宸殿,用黃麾角仗,共一千五十六人。殿內黃麾幡二,次四色官之南,分左右。仗首左右廂各一部,每部一百四十人,在朵殿下稍南。絳引幡十,分部之南北,各為五重。左右領軍衛大將軍各一員,在部中稍南。



 次廂左右第一、第二部同。



 掌鼓一人,次大將軍後。次廂左右第一部次果毅,第二部次折沖。



 帥兵官十人,分部之南北,北在絳引幡之南,南在絳引幡之北。次廂左右第一、第二部在部之南北。



 各為五重。執儀刀部九行,每行持各十人。
 第一,龍頭竿黃雞四角氅;皆持以龍頭竿。



 第二,儀鍠五色幡;第三,青孔雀五角氅;第四,烏戟;第五,緋鳳六角氅;第六,細弓矢;第七,白鵝四角氅;第八,槊;第九,阜鵝六角氅。掌揭鼓一人,在緋氅、烏戟之後。



 次廂左右第一、第二部同。



 次廂左右各二部,每部一百五人,次左右廂仗首之南。第一部,左右屯衛大將軍、果毅各一員;第二,左右武衛大將軍、折沖各一員。掌鼓以下至掌揭鼓人數,並同仗首。殿外左右廂各步甲三隊,每隊三十三人。第一,左右衛,第三,左右武
 衛,並果毅;第二,左右驍衛、折沖:並各一員。貔、金鸚鵡、瑞麥旗各二,以次分在三隊。刀盾三十人,為五重。



 內第二隊弓矢。



 左右廂後部各三隊,第一隊每隊三十八人,第二隊每隊三十三人。



 第一,左右衛,第三,左右武衛,並折沖;第二,左右驍衛、果毅。角、太平、馴犀旗各二,以次分在三隊。弩五人,為一列,弓矢十人,為二重,第二、第三隊為一列。



 槊二十人,為四重。排列仗隊職掌二人,次廂第二部之南,分左右。以上殿內外仗隊,東西相向排列。



 殿中省尚輦陳輿、輦於東西朵殿,平輦在東,西
 向;逍遙輦在西,東向。設傘、扇於殿下,方傘二,分左右;團龍扇四,分左右,夾方傘。方雉扇二十四,分左右,各二重,在傘、扇之後。金吾四色官一人。



 政和中,文德殿發冊,用黃麾細仗,共一千四百二人。設日旗、君王萬歲旗、獅子旗、金鸞旗、青龍旗、赤龍旗各一,在殿東階之東,以西為上;月旗、天下太平旗、獅子旗、金鳳旗、白龍旗、黑龍旗各一,在殿西階之西,以東為上;每旗執扯四人。



 俱北向立。押當職掌二人,分左右立於日、月旗南。次
 方傘二,團龍扇四,夾方傘。次金吾上將軍二人,將軍四人,引駕官四人。次金甲二人。次四色官六人,內二人執笏,余執金銅儀刀。次碧襴二十四人,內執金銅儀刀左右各六人,在北。次都押衙二人,立於碧襴之南,少退。次皂纛旗一十二,每旗執扯五人。左右金吾仗司員僚各一人押纛,立於旗南。次青龍旗一在東,白虎旗一在西,每旗執扯六人。



 員僚二人押旗,在旗之北。以上並分左右,東西向。次五方龍旗在東,五方鳳旗在西,各二十五。每五旗相間,各
 依方色排列。次五嶽神旗五在東,五星神旗五在西,各依方位排列。



 每旗執扯三人。



 次朱雀旗一在東,真武旗一在西。每旗執扯六人。



 以上並北向。員僚二人押旗,在旗之南,分左右。次紅門旗二十八,分左右。每旗執扯二人。



 次寅、卯、辰、巳、午、未旗六,在東;申、酉、戌、亥、子、丑旗六,在西。天王旗四,分左右,夾辰旗。次龍君、赤豹、吏兵旗各五,每旗各為一列在東,每列掩尾天馬旗一,以次在東。次虎君、黃熊、力士旗各五,每旗各為一列在西,每列掩尾天馬旗一,以次在西。



 每旗執扯三人。



 員僚六人押仗,各分立旗前。次員僚四人押旗,分左右,東西為一列。每列一員。



 左廂第一隊,鶡雞、白澤、玉馬、貔旗、四瀆旗各一,為一列;下至第九隊旗行列準此。



 第二隊,角、亢、氐、房、心宿旗各一;第三隊,虛、危、室、壁、奎宿旗各一;第四隊,參、井、鬼、柳宿、駃騠旗各一;第五隊,三角獸、黃鹿、苣文、馴象、飛麟旗各一;第六隊,闢邪、玉兔、吉利、仙鹿、祥雲旗各一;第七隊,花鳳、飛黃、野馬、金鸚鵡、瑞麥旗各一;第八隊,孔雀、兕、甘露、網子、角旗各一,並各為一列;第九隊,犛牛旗一,
 設於孔雀旗後。右廂第一隊,同左廂第一;第二隊,尾、箕、斗、牛、女宿旗各一;第三隊,婁、胃、昴、畢、觜宿旗各一;第四隊,星、張、翼、軫、駃騠旗各一;第五至第八隊,並同左廂第五至第八;第九隊,騶牙旗、蒼烏旗各二,相間為一列。



 每旗執扯三人。



 俱北向。員僚二人,押黃麾立於龍鳳旗之北。左右廂五色龍鳳旗之東西,各設黃麾幡二。次告止幡、傳教幡、信幡各五,次絳麾幡二,次絳引幡五。員僚五人,押黃麾立於龍鳳旗北少東。排闌旗三十,自黃麾幡東西排
 列,以次於南,每旗執扯三人。



 俱北向。鐙杖、哥舒各三十,於殿東西兩廂排列。鐙杖起北,哥舒間之,俱東西相向。左右廂執白柯槍各七十五人,東西相向。又於騶牙旗南設大黃龍旗一,在殿門裡少西,執扯二十人。小黃龍旗一,在大黃龍旗後少西,執扯三人。次大神旗六,分左右。衛尉寺押當儀仗職掌四人,排仗通直官二人,大將二人,節級二人,檢察六人,左右金吾仗司押當職掌、排列官各一人。凡大朝會儀衛,有司皆依令式陳
 設。



 初,宋制,有黃麾大仗、半仗、角仗、細仗。南渡後,儀仗尤簡,惟造黃麾半仗、角仗、細仗,而大仗不設。中興大朝會,四朝惟一講,紹興十五年正月朔旦是也。然止以大仗三分減一,用三千三百五十人。自是正旦、冬至俱免大朝賀,以為定例焉。



 黃麾半仗者,大慶殿正旦受朝、兩宮上冊寶之所設也,用二千四百一十五人。其內儀仗官兵等一千八百三人,兵部職掌五人,統制官二人,皆帕頭、
 公裳、腰帶、靴、笏。金吾司碧襴三十二人,帕頭、碧襴衫、銅革帶,執儀刀。將官二人,帕頭、緋抹額、紫繡羅袍、背螣蛇、銅革帶,執儀刀。旁頭一十人,素帽、紫紬衫、纈衫、黃勒帛,執銅仗子。金銅甲二人,兜鍪、甲衫、錦臂衣,執金銅鉞斧。絳引幡十,告止幡、傳教幡、信幡各二,執幡人皆武弁、緋寶相花衫、勒帛。黃麾幡二,執幡人武弁、黃寶相花衫、銅革帶。小行旗三百人,素帽、五色抹額、緋寶相花衫、勒帛。五色小氅三百人,儀鍠四十人,皆纈帽,五色寶相花衫、
 勒帛。金節一十二人,武弁、青寶相花衫、銅革帶。殳叉三十人,素帽、五色寶相花衫、勒帛。綠槊二百一十人,素帽、緋寶相花衫、勒帛。烏戟二百一十人,纈帽、緋寶相花衫、勒帛。白柯槍六十人,素帽子、銀褐寶相花衫、勒帛。儀弓二百七十人,纈帽、青寶相花衫、勒帛。儀弩六十人,平巾幘、緋寶相花衫、勒帛。銅仗子二十人,素帽、紫紬衫、黃勒帛。儀刀百八十四人,平巾幘、緋寶相花衫。內大旗下六百一十二人,大旗三十四,龍旗一十,鳳旗一十,五星旗、
 五岳旗各五,青龍旗、白虎旗、朱雀旗、玄武旗各一,每旗扶拽一十七人,搭材一名,武弁、五色寶相花衫、勒帛。其外殿中輿輦、傘扇百三十三人,逍遙、平輦各一,每輦人員八人,帽子、宜男纈羅單衫、塗金銀柘枝腰帶。輦官二十七人,帕頭、白獅子纈羅單衫、塗金銀海捷腰帶、紫羅裡夾三襜。中道傘扇六十六,輦官七十人,素方傘四十四人,弓腳帕頭、碧襴衫、塗金銅革帶、烏皮履。繡紫方傘六、花團扇十二、十八人,雉扇二十二人,準備四人,皆武
 弁、緋寶相花袍、銅革帶。鳳扇二十二人,黃抹額、黃寶相花袍、黃勒帛。編排儀仗職掌五人,立殿下傘扇後,烏皮介幘、緋羅寬衫、白羅大帶。



 其黃麾小半仗者,大慶殿冊皇太子及穆清殿皇后受冊之所設也,用一千四百九十九人。其內儀仗官兵等八百八十七人,兵部職掌十二人,金吾司碧襴三十人,絳引幡二、告止幡一、傳教幡一、信幡一、用十五人,黃麾幡一、三人。小行旗百八十人,五色小氅子百八十人,金節十二人,儀鍠、斧二十三人,
 綠槊七十五人,烏戟七十五人,白柯槍八十一人,儀弓六十三人,儀弩四十五人,銅仗子一十人,儀刀六十七人。統制官、將官、牽頭、金銅甲,皆與前半仗同。內大旗下六百一十二人,殿中輿輦、傘扇百三十二人,皆同前半仗。



 其黃麾角仗者,大慶殿冬至受朝、紫宸殿即位、兩宮賀節慶壽、紫宸殿受金使朝之所設也,用一千五十六人。內金吾司放仗官二人,統制官一人,攝大將軍六人,旁頭五人,黃麾幡一,三人,絳引幡八,二十四人,金節十
 二人,儀弓七十人,儀弩五十人,儀刀七十人,儀鍠、斧一十三人,白柯槍三十人,綠槊七十人,烏戟七十人,小行旗三百人,五色小氅三百人,銅仗子三十人。



 其黃麾細仗者,大慶殿、文德殿發冊及進國史之所設也。東都用一千四百二人,中興後或用百人至五百人,隨事增損。而其執仗有四,小行旗、五色小氅、儀刀、銅仗子;其服色有四,纈帽子、素帽子、平巾幘、武弁冠,五色寶相花衫、勒帛。



 大朝會之外,有日參、四參、六參、朔參、望參。朔參,
 用厘務、不厘務通直郎已上。望參,用厘務通直郎已上。宣制、非時慶賀以望參官,餘以朔參官。四參官,謂宰執,侍從,武臣正任,文臣卿監、員郎、監察御史已上。四參遇雨則改日參。在京宮觀奉朝請者赴六參。高宗移蹕臨安,殿無南廊,遇雨雪,則日參官於南合內起居。宰執、使相立簷下;侍從、兩省、臺諫官以下立南合內;卿監、郎官、武功大夫以下立東西廊。紹興十二年十月,有司請行正、至朝賀禮,及講求祖宗故實常朝、視朝、正衙、便殿之儀。
 乃討論朔日文德殿視朝,紫宸殿日參、望參,垂拱殿日參、四參,假日崇政殿坐,聖節垂拱、紫宸殿上壽之制。請先御正殿視朝。十一月,禮部侍郎王賞言:「正、至及大慶賀受朝,系禦大慶殿,與文德、紫宸、垂拱殿禮制不同。月朔視朝,則御文德殿,謂之前殿正衙,設黃麾半仗。其餘紫宸、垂拱皆系別殿,不設儀仗。今大慶殿朝會,禮文繁多,欲先舉行文德殿視朝之制。」時行宮止一殿,乃更作祟政、垂拱二殿。御史臺請以射殿為崇政殿,朔望權置帳
 門以為紫宸殿,宣赦書、德音、麻制以為文德殿,群臣拜表、聽御札批答權作文殿德東上合門。其垂拱殿四參,於殿門外設位版。十三年,始視朝於文德殿,設黃麾半仗二千四百十五人。六月,紫宸殿望參,設黃麾角仗一千五十六人。自是,後殿坐及射殿引呈公事,以日景已高,依舊制設衛士、青涼傘十。淳熙十四年,詔引呈射殿公事,殿門外排立御馬,如後殿之儀。



 大朝會儀,舊制,垂拱殿設簾,殿上駐輦,候起居稱賀班絕,乘輦,樞密、知
 合門官、樞密都副承旨、諸房副都承旨前導,管軍引駕至大慶殿後幄降輦,入次更衣。紹興十五年正月朔旦,以二殿經途與東都異,乃以常御殿為垂拱殿,免駐輦,設簾帷,設椅子,稱賀畢,過大慶殿後幄。前期,儀鸞司設御榻於大慶殿中,南向,設東西房於御榻左右稍北,設東西合於殿後左右,殿上前楹施簾,設香案於殿下。太常展宮架樂於殿庭橫街之南。其日,御輦院陳輿輦、傘扇於殿下,東西相向。兵部陳五輅於皇城南門外,俱北
 向。騏驥院列御馬於殿門外,東西相向。兵部帥屬設黃麾仗三千三百五十人於殿門內外。以殿狹,輦出房,不鳴鞭。



 淳熙十六年正旦,行稱賀禮,比政和五禮月朔視朝儀。皇帝御大慶殿,服靴袍,即御坐,皇太子、文武百僚常服稱賀,而設黃麾半仗二千四百十五人。及冬至朝賀,設黃麾角仗一千五十六人。著為令。而大朝會儀。自紹興十五年以後不設。



\end{pinyinscope}