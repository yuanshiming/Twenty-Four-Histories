\article{本紀第五}

\begin{pinyinscope}

 太宗
 二



 二
 年春正月丙辰,以德恭為左武衛大將軍、判濟州,封定安侯;德隆為右武衛大將軍、判沂州,封長寧侯。右補闕劉蒙叟通判濟州,起居舍人韓儉通判沂州。乙丑,賜
 德恭、德隆常奉外支錢三百萬。



 二月戊寅,權交州留後黎桓遣使來貢。乙未,夏州李繼遷誘殺汝州團練使曹光實。己亥,占城遣使來貢。



 三月己未,親試禮部舉人。江南民饑,許渡江自占。



 夏四月乙亥朔,遣使行江南諸州,振饑民及察官吏能否。戊寅,遣忠武軍節度使潘美復屯三交口。己卯,詔以帝所生官舍作啟聖院。己丑,殿前承旨王著坐監資州兵為奸贓,棄市。庚子,甘露降後苑。辛丑,夏州行營破西蕃息利族,斬其代州刺史折羅遇
 並弟埋乞,又破保、洗兩族,降五十餘族。



 五月甲子,幸城南觀麥,賜田夫布帛。天長軍蝝生。



 六月甲戌朔,河西行營言,獲岌羅賦等十四族,焚千餘帳。戊子,復禁鹽、榷酤。



 秋七月庚申,詔諸道轉運使及長吏,宜乘豐儲廩以防水旱。



 八月癸酉朔,遣使按問兩浙、荊湖、福建、江南東西路、淮南諸州刑獄,仍察官吏勤惰以聞。癸巳,西南奉化王子以慈來貢。是月,瀛、莫二州大水。



 九月丙午,以歲無兵兇,除十惡、官吏犯贓、謀故劫殺外,死罪減降,流以下
 釋之,及蠲江、浙諸州民逋租。庚戌,重九,賜近臣飲於李昉第,召諸王、節度使宴射苑中。是夕,楚王宮火。辛亥,廢楚王元佐為庶人、均州安置。丁巳,群臣請留元佐養疾京師,許之。己未,西南蕃王遣使來貢。己巳,禁海賈。



 閏月癸未,太白入南斗。甲申,幸天駟監,賜從臣馬。乙未,禁邕管殺人祭鬼及僧人置妻孥。己亥,均州獻一角獸。



 冬十月辛丑朔,慮囚。丙午,以天竺僧天息災、施護、法天並為朝請大夫、試鴻臚少卿。己酉,汴河主糧胥吏坐奪漕軍
 口糧,斷腕徇於河畔三日,斬之。甲寅,黎邛部蠻王子來貢。



 十一月壬午,狩於近郊,以所獲獻太廟,著為令。戊子,禱雪。辛卯,詔在官丁父母憂者並放離任。十二月庚子朔,日有食之。癸卯,南康軍言,雪降三尺,大江冰合,可勝重載。丁未,遣中使賜緣邊戍卒襦褲。丙辰,門下侍郎兼刑部尚書、平章事宋琪罷守本官。



 三年春正月辛未,右武衛大將軍、長寧侯德隆薨,以其弟德彞嗣侯,仍知沂州。庚辰,夜漏一刻,北方有赤氣如
 城,至明不散。己丑,知雄州賀令圖等請伐契丹,取燕、薊故地。庚寅,北伐,以天平軍節度使曹彬為幽州道行營前軍馬步水陸都部署,河陽三城節度使崔彥進副之;侍衛馬軍都指揮使、彰化軍節度使米信為西北道都部署,沙州觀察使杜彥圭副之,以其眾出雄州;侍衛步軍都指揮使、靜難軍節度使田重進為定州路都部署,出飛狐。戊戌,參知政事李至罷為禮部侍郎。二月壬子,以檢校太師、忠武軍節度使潘美為雲、應、朔等州都部
 署,雲州觀察使楊業副之,出雁門。



 三月癸酉,曹彬與契丹兵戰固安南,克其城。丁丑,田重進戰飛狐北,又破之。潘美自西陘入,與契丹兵遇,追至寰州,執其刺史趙彥辛,辛以城降。辛巳,曹彬克涿州。潘美圍朔州,其節度副使趙希贊以城降。癸未,田重進戰飛狐北,獲其西南面招安使大鵬翼、康州刺史馬頵、馬軍指揮使何萬通。乙酉,曹彬敗契丹於涿州南,殺其相賀斯。丁亥,潘美師至應州,其節度副使艾正、觀察判官宋雄以城降。司門員
 外郎王延範與秘書丞陸坦、戎城縣主簿田辯、術士劉昂坐謀不軌,棄市。庚寅,武寧軍節度使、同平章事、岐國公陳洪進卒。辛卯,田重進攻飛狐,其守將呂行德、張繼從、劉知進等舉城降,以其縣為飛狐軍。占城國遣使來貢。丙申,進圍靈丘,其守將穆超以城降。



 夏四月辛丑,潘美克雲州。田重進戰飛狐北,破其眾。壬寅,曹彬、米信戰新城東北,又破之。己酉,田重進再戰飛狐北,再破之,殺二將。乙卯,重進至蔚州,其牙校李存璋、許彥欽殺大將
 蕭啜理,執其監城使、同州節度使耿紹忠以城降。



 五月庚午,曹彬之師大敗於岐溝關,收眾夜渡拒馬河,退屯易州,知幽州行府事劉保勛死之。丙子,召曹彬、崔彥進、米信歸闕,命田重進屯定州,潘美還代州。徙雲、應、寰、朔吏民及吐渾部族,分置河東、京西。會契丹十萬眾復陷寰州,楊業護送遷民遇之,苦戰力盡,為所禽,守節而死。



 六月戊戌朔,日有食之。甲辰,以御史中丞辛仲甫為參知政事。



 秋七月庚午,貶曹彬為右驍衛上將軍,崔彥進
 為右武衛上將軍,米信為右屯衛上將軍,杜彥圭為均州團練使。應群臣、列校死事及陷敵者,錄其子孫。壬午,徙山後降民至河南府、許汝等州。丁亥,以簽署樞密院事張齊賢為給事中、知代州。癸巳,階州福津縣有大山飛來,自龍帝峽壅江水逆流,壞民田數百里。甲午,詔改陳王元祐為元僖,韓王元休為元侃,冀王元雋為元份。



 八月丁酉朔,以王沔、張宏並為樞密副使。丁未,大雨,遣使禱嶽瀆,至夕雨止。劍州民饑,遣使振之,因督捕諸州
 盜賊。辛亥,降潘美為檢校太保,贈楊業太尉、大同軍節度使。



 九月丙寅朔,減兩京諸州系囚流以下一等,杖罪釋之。賜所徙寰、應、蔚等州民米,升、宣等十四州雍熙二年官所振貸並蠲之。戊寅,賜北征軍士陣亡者家三月糧。



 冬十月甲辰,以陳王元僖為開封尹。壬子,高麗國王遣使來貢。庚申,詔以權靜海軍留後黎桓為本軍節度。



 十一月丙戌,幸建隆觀、相國寺祈雪。十二月乙未朔,大雨雪,宴群臣玉華殿。己亥,定州田重進入契丹界,攻下
 岐溝關。壬寅,契丹敗劉廷讓軍於君子館,執先鋒將賀令圖,高陽關部署楊重進死之。壬子,建房州為保康軍,以右衛上將軍劉繼元為節度使。代州副部署盧漢贇敗契丹於土鐙堡,斬獲甚眾,殺監軍舍利二人。是歲,壽州大水,濮州蝗。



 四年春正月甲子朔,不受朝,群臣詣閣拜表稱賀。己卯,遣使按問西川、嶺南、江浙等路刑獄。丙戌,詔:「應行營將士戰敗潰散者並釋不問,緣邊城堡備御有勞可紀者
 所在以聞。瘞暴骸,死事者廩給其家,錄死事文武官子孫。蠲河北雍熙三年以前逋租,敵所蹂踐者給復三年,軍所過二年,餘一年。」二月丙申,以漢南國王錢俶為武勝軍節度使,徙封南陽國王。丁酉,繕治河北諸州、軍城隍。甲寅,錢俶改封許王。



 三月庚辰,詔申嚴考績。



 夏四月癸巳朔,以御史中丞趙昌言為右諫議大夫、樞密副使。乙未,詔諸州郡暑月五日一滌囹圄,給飲漿,病者令醫治,小罪即決之。丁未,幸金明池觀水嬉,遂習射瓊林苑,
 登樓,擲金錢繒彩於樓下,縱民取之。並水陸發運為一司。



 五月丙寅,遣使市諸道民馬。庚辰,改殿前司日騎為捧日,驍猛為拱辰,雄勇為神勇,上鐵林為殿前司虎翼,腰弩為神射,侍衛步軍司鐵林為侍衛司虎翼。丁亥,詔諸州送醫術人校業太醫署。賜諸將陣圖。



 六月丁酉,以右驍衛上將軍劉廷讓為雄州都部署。戊戌,以彰國軍節度使、駙馬都尉王承衍為貝、冀都部署,郭守文及郢州團練使田欽祚並為北面排陣使。庚子,定國軍節度
 使崔翰復為高陽關兵馬都部署。是月,鄜州獻馬,前足如牛。



 秋七月丙寅,幸講武池觀魚。是月,置三班院。



 八月庚子,免諸州吏所逋京倉米二十六萬七千石。



 九月癸亥,校醫術人,優者為翰林學生。



 冬十月丙午,流雄州都部署劉廷讓於商州。壬子,左僕射致仕沉倫薨。



 十一月庚辰,詔以實數給百官奉。十二月壬寅,幸建隆觀、相國寺祈雪。庚戌,畋近郊。丁巳,大雨雪。



 端拱元年春正月己未朔,不受朝,群臣詣閣拜表稱賀。
 乙亥,親耕籍田。還,御丹鳳樓,大赦,改元。除十惡、官吏犯贓至殺人者不赦外,民年七十以上賜爵一級。癸未,幸玉津園習射。乙酉,禁用酷刑。是月,澶州黃河清。



 二月乙未,改左、右補闕為左、右司諫,左、右拾遺為左、右正言。丙申,禁諸州獻珍禽奇獸。己亥,詔瀛州民為敵所侵暴者賜三年租,復其役五年。庚子,以籍田,開封尹、陳王元僖進封許王,元侃襄王,元份越王,錢俶鄧王,中書門下平章事李昉為尚書右僕射,參知政事呂蒙正同中書門
 下平章事,樞密使王顯加檢校太傅,給事中、許國公趙普守太保兼侍中,參加政事辛仲甫加戶部侍郎,樞密副使趙昌言加工部侍郎,樞密副使王沔為參知政事,御史中丞張宏為樞密副使,餘內外並加恩。甲辰,升建州為建寧軍節度。庚戌,以子元偓為左衛上將軍、徐國公,元偁為右衛上將軍、涇國公。



 三月甲戌,貶樞密副使趙昌言為崇信軍行軍司馬。乙亥,鄭州團練使侯莫陳利用坐不法,配商州禁錮,尋賜死。癸未,幸玉津園習射。
 廢水陸發運司。



 夏四月丁亥,賜京城高年帛。己丑,加高麗國王治、靜海軍節度使黎桓並檢校太尉。



 五月辛酉,置秘閣於崇文院。辛未,感德軍節度使李繼捧賜姓趙氏,名保忠。壬申,以保忠為定難軍節度使。閏五月辛卯,以洺州防禦使劉福為高陽關兵馬都部署,濮州防禦使楊贊為貝州兵馬都部署。乙未,賜諸州高年爵公士。丁酉,交州黎桓遣使來貢。壬寅,親試禮部進士及下第舉人。



 六月丙辰朔,右領軍衛大將軍陳廷山謀反,伏誅。
 丁丑,改湖南節度為武安軍節度。親試進士、諸科舉人。



 秋七月丙午,除西川諸州鹽禁。辛亥,忠武軍節度使潘美知鎮州。



 八月乙卯,壽星見丙地。甲子,以宣徽南院使郭守文為鎮州路都部署。戊寅,太師、鄧王錢俶薨,追封秦國王,謚忠懿。庚辰,幸太學,命博士李覺講《易》,賜帛,遂幸玉津園習射。是月,鳳凰集廣州清遠縣廨合歡樹,樹下生芝三莖。



 九月乙酉朔,以侍衛馬軍都指揮使李繼隆為定州都部署。



 冬十月壬午,以侍衛步軍都指揮使
 戴興為澶州都部署。癸未,詔罷游獵,五方所畜鷹犬並放之,諸州毋以為獻。



 十一月甲申朔,高麗王遣使來貢。己丑,郭守文破契丹於唐河。十二月辛未,以夏州蕃落使李繼遷為銀州刺史,充洛苑使。



 二年春正月癸未朔,不受朝,群臣詣閣拜表稱賀。壬辰,以涪州觀察使柴禹錫為澶州兵馬部署。癸巳,詔議北伐。



 二月壬子朔,令河北東、西路招置營田。癸丑,詔錄將校官吏功及死事使臣、官吏子孫,士卒廩給其家三月。
 平塞、天威、平定、威虜、靜戎、保塞、寧邊等軍,祁、易、保、定、鎮、邢、趙等州民,除雍熙四年正月丙戌詔給復外,更給復二年;霸、代、洺、雄、莫、深等州,平虜、岢嵐軍,更給復一年。戊午,罷乘傳銀牌,復給樞密院牒。以太倉粟貸京畿饑民。癸亥,作方田。戊辰,以國子監為國子學。



 三月辛卯,命高瓊為並、代都部署。壬寅,親試禮部舉人。



 夏四月丁巳,置富順監。辛未,幸趙普第視疾。



 五月戊戌,以旱慮囚,遣使決諸道獄。是夕,雨。



 秋七月甲申,以知代州張齊賢為刑
 部侍郎、樞密副使,鹽鐵使張遜為宣徽北院使、簽署樞密院事。戊子,有彗出東井,上避正殿,減常膳。辛丑,契丹犯威虜軍,崇儀使尹繼倫擊破之,殺其相皮室,大將於越遁去。



 八月丙辰,大赦,是夕,彗不見。癸亥,詔作開寶寺舍利塔成。



 九月壬午,邛部川、山後百蠻來貢。



 冬十月辛未,以定難軍節度使趙保忠同平章事。以歲旱、彗星謫見,詔曰:「朕以身為犧牲,焚於烈火,亦未足以答謝天譴。當與卿等審刑政之闕失、稼穡之艱難,恤物安人,以祈
 玄祐。」十二月辛亥,置三司都磨勘官。丙辰,大雨雪。庚申,詔令四方所上表祗稱皇帝。群臣請復尊號,不許。辛酉,上法天崇道文武皇帝,詔去「文武」二字,餘許之。三佛齊國遣使來貢。



 淳化元年春正月戊寅朔,減京畿系囚流罪以下一等。改元,內外文武官並加勛階爵邑,中書舍人、大將軍以上各賜一子官。賜鰥寡孤獨錢,除逋負。受尊號,改乾明節為壽寧節。戊子,詔作清心殿。



 二月丁未朔,除江南、兩
 浙、淮西、嶺南諸州漁禁。己酉,改大明殿為含光殿。



 三月丙子朔。乙未,幸西京留守趙普第視疾。



 夏四月庚戌,遣中使詣五岳禱雨,慮囚,遣使分決諸道獄。甲寅,詔尚書省四品、兩省五品以上舉轉運使及知州、通判。五溪蠻田漢權來附。戊午,建婺州為保寧軍節度。丙寅,命殿前副都指揮使戴興為鎮州都部署。



 五月甲午,給致仕官半奉。辛卯,置詳覆、推勘官。



 六月丙午,罷中元、下元張燈。庚午,太白晝見。



 秋七月丁丑,太白復見。是月,吉、洪、江、蘄、
 河陽、隴城大水。開封、陳留、封丘、酸棗、鄢陵旱,賜今年田租之半,開封特給復一年。京師貴糴,遣使開廩減價分糶。



 八月乙巳,毀左藏庫金銀器皿。己巳,禁川峽、嶺南、湖南殺人祀鬼,州縣察捕,募告者賞之。庚午,西南蕃主使其子龍漢興來貢。是月,京兆長安八縣旱,賜今年租十之六。蠲舒州宿松等三處魚池稅。



 九月辛巳,熒惑入太微垣。大宴崇政殿。禁川峽民父母在出為贅婿。是月,蠲滄、單、汝三州今年租十之六。



 冬十月甲辰,交州黎桓遣
 使來貢。乙巳,熒惑陵左執法。乙丑,知白州蔣元振、知須城縣姚益恭並以清乾聞,下詔褒諭,賜粟帛。是月,以乾鄭二州、河南壽安等十四縣旱,州蠲今年租十之四,縣蠲其稅。



 十一月戊戌,太白晝見。是月,蠲大名府管內今年租十之七。十二月乙巳,占城遣使來貢。乙卯,高麗國遣使來貢。辛酉,詔中外所上書疏及面奏制可者,並下中書、樞密、三司中覆頒行。是歲,洪、吉、江、蘄諸州水,河陽大水。曹、單二州有蝗,不為災。開封、大名管內及許、滄、單、
 汝、乾、鄭等州,壽安、長安、天興等二十七縣旱。深冀二州、文登牟平兩縣饑。



 二年春正月壬申朔,不受朝,群臣詣閣拜表稱賀。丙子,遣商州團練使翟守素帥兵援趙保忠於夏州。乙酉,置內殿崇班、左右侍禁,改殿前承旨為三班奉職。丙戌,熒惑犯房。己丑,詔陜西諸州長吏設法招誘流亡,復業者計口貸粟,仍給復二年。



 二月癸丑,盡易宮殿彩繪以赭堊。監察御史祖吉坐知晉州日為奸贓,棄市。乙丑,斬夔
 州亂卒謝榮等百餘人於市。



 閏月辛未朔,日有食之。戊寅,禱雨。丁亥,詔內外諸軍,除木槍、弓弩矢外不得蓄他兵器。己丑,詔京城蒲博者,開封府捕之,犯者斬。命近臣兼差遣院流內銓。是月,河水溢,鄄城縣蝗,汴河決。



 三月乙卯,幸金明池,御龍舟,遂幸瓊林苑宴射。己巳,以歲蝗旱禱雨弗應,手詔宰相呂蒙正等:「朕將自焚,以答天譴。」翌日而雨,蝗盡死。



 夏四月庚午,罷端州貢硯。辛巳,以張齊賢、陳恕並參知政事,張遜兼樞密副使,溫仲舒、寇準
 並為樞密副使。是月,河水溢,虞鄉等七縣民饑。



 五月己亥朔,詔減兩京諸州系囚流以下一等,杖罪釋之。庚子,置諸路提點刑獄官。丙辰,左正言謝泌以敢言擢右司諫,賜金紫,錢三十萬。



 六月甲戌,忠武軍節度使、同平章事潘美卒。命張永德為並、代都部署。乙酉,以汴水決浚儀縣,帝親督衛士塞之。庚寅,禁陜西緣邊諸州闌出生口。是月,楚丘、鄄城、淄川三縣蝗,河水、汴水溢。



 秋七月己亥,詔陜西緣邊諸州饑民鬻男女入近界部落者,官贖
 之。李繼遷奉表請降,以為銀州觀察使,賜國姓,改名保吉。是月,乾寧軍蝗,許、雄、嘉三州大水。



 八月己卯,置審刑院。己丑,雅州言登遼山崩。



 九月丁酉朔,戶部侍郎、參知政事王沔,給事中、參知政事陳恕並罷守本官。己亥,中書侍郎兼戶部尚書、平章事呂蒙正罷為吏部尚書,以右僕射李昉、參知政事張齊賢並平章事,翰林學士賈黃中、李沆並為給事中、參知政事。帝飛白書「玉堂之署」四字,以賜翰林承旨蘇易簡。壬寅,邛部川蠻來貢。癸卯,
 罷樞密使王顯為崇信軍節度使。甲辰,以張遜知樞密院事,溫仲舒、寇準同知院事。



 十一月丙申朔,復百官次對。乙巳,罷京城內外力役土功。己酉,幸建隆觀、相國寺祈雪。十二月丙寅朔,行入閣儀。乙亥,賜秦州童子譚孺卿本科出身。癸未,保康軍節度使劉繼元卒,追封彭城郡王。大雨,無冰。是歲,女真表請伐契丹,詔不許,自是遂屬契丹。大名、河中,絳、濮、陜、曹、濟、同、淄、單、德、徐、晉、輝、磁、博、汝、兗、虢、汾、鄭、亳、慶、許、齊、濱、棣、沂、貝、衛、青、霸等州旱。



 三年春正月癸卯,大雨雪。乙巳,詔常參官舉可任升朝官者。丙午,詔宰相、侍從舉可任轉運使者。



 二月乙丑朔,日有食之。



 三月乙未朔,以趙普為太師,封魏國公。戊戌,親試禮部舉人。辛丑,親試諸科舉人。戊午,以高麗賓貢進士四十人並為秘書省秘書郎,遣還。庚申,帝幸金明池觀水戲,縱京城觀者,賜高年白金器皿。



 夏四月丁丑,詔江南、兩浙、荊湖吏民之配嶺南者還本郡禁錮。癸未,上作《刑政》、《稼穡》詩賜近臣。



 五月甲午朔,御文德殿,百官
 入閣。壬寅,詔御史府所斷徒罪以上獄具,令尚書丞郎、兩省給舍一人慮問。丁未,戶部郎中田錫、通判殿中丞郭渭坐稽留刑獄,並責州團練副使,不簽署州事。戊申,詔太醫署良醫視京城病者,賜錢五十萬具藥,中黃門一人按視之。己酉,以旱,遣使分行諸路決獄。是夕,雨。辛亥,置理檢司。甲寅,詔作秘閣。



 六月丁丑,大風,晝晦,京師疫解。戊寅,慮囚。甲申,飛蝗自東北來,蔽天,經西南而去。是夕,大雨,蝗盡死。庚寅,以殿前都虞候王昭遠為並、代
 兵馬都部署。辛卯,置常平倉。



 秋七月己酉,太師、魏國公趙普薨,追封真定王。是月,許、汝、兗、單、滄、蔡、齊、貝八州蝗,洛水溢。



 八月戊辰,以秘閣成,賜近臣宴。壬申,召終南山隱士種放,不至。庚辰,闍婆國遣使來貢。丁丑,釋嶺南東、西路罰作荷校者。



 九月丙申,遣官祈晴京城諸寺觀。甲寅,幸天駟監,賜從臣馬。乙卯,群臣上尊號曰法天崇道明聖仁孝文武皇帝,凡五表,終不許。



 冬十月辛酉朔,折御卿進白花鷹,放之,詔勿復獻。戊寅,始置京朝、幕職、州
 縣官考課,並校三班殿最。戊子,高麗、西南蕃皆遣使來貢。



 十一月己亥,許王元僖薨。甲申,慮囚,降徒流以下一等,釋杖罪。趙保忠貢鶻,號「海東青」,還之。己未,禁兩浙諸州巫師。置三司主轄收支官。是月,蔡州建安大火。十二月丁卯,大雨雪。己卯,占城國王楊陀排遣使來貢。是月,雄州言大火。是歲,潤州丹徒縣饑,死者三百戶。



 四年春正月庚寅朔,享太室,群臣詣齋宮拜表稱賀。辛卯,祀天地於圜丘,以宣祖、太祖配,大赦。乙未,大雨雪。高
 麗國遣使來貢。乙巳,藏才西族首領羅妹以良馬來獻。



 二月己未朔,日有食之。壬戌,召賜京城高年帛,百歲者一人加賜塗金帶。是日,雨雪,大寒,再遣中使賜孤老貧窮人千錢、米炭。置昭宣使。癸亥,廢沿江榷貨八務。乙丑,加高麗國王王治檢校太師,靜海軍節度使黎桓封交阯郡王。己卯,詔以江、浙、淮、陜饑,遣使巡撫。詔分遣近臣巡撫諸道,有可惠民者得便宜行事,吏罷軟、苛刻者上之,詔令有未便者附傳以聞。丙戌,置審官院、考課院。永
 康軍青城縣民王小波聚徒為寇,殺眉州彭山縣令齊元振。是月,商州大雨雪。



 三月壬子,詔權停貢舉。



 四月己卯,諸司奉行公事不得輒稱聖旨。



 五月戊申,罷鹽鐵、戶部、度支等使,置三司使。



 六月戊午朔,詔中丞己下皆親臨鞫獄。丙寅,吏部侍郎、平章事張齊賢罷為尚書左丞。壬申,宣徽北院使、知樞密院事張遜貶右領軍衛將軍,右諫議大夫、同知院事寇準罷守本官。以涪州觀察使柴禹錫為宣徽北院使、知樞密院事,樞密直學士呂端
 參知政事,劉昌言同知樞密院事。戊寅,初復給事中封駁。



 七月丁酉,大雨。戊戌,復沿江務,置諸路茶鹽制置使。



 八月丙辰朔,日有食之。癸酉,以向敏中、張詠始同知銀臺、通進司,視章奏案牘以稽出入。



 九月丙申,詔諸雜除禁錮人,州縣有闕,得次補以責效,能自新勤乾者具聞再敘。乙巳,以給事中封駁隸銀臺、通進司。丙午,命侍從舉任才堪五千戶以上縣令者二人。自七月雨,至是不止。是月,河水溢,壞澶州。江溢,陷涪州。詔溺死者給斂具,
 澶人千錢涪人鐵錢三千,仍發廩以振。



 冬十月壬戌,罷諸路提點刑獄司。庚午,始分天下州縣為十道,兩京為左右計,各署判官領之,置三司使二員。辛未,右僕射、平章事李昉,給事中、參知政事賈黃中、李沆,左諫議大夫、知樞密院事溫仲舒並罷守本官。以吏部尚書呂蒙正平章事,翰林學士蘇易簡為給事中參知政事;樞密都承旨趙鎔為宣徽北院使,樞密直學士向敏中為右諫議大夫,並同知樞密院事。丁丑,以右諫議大夫趙昌言
 為給事中、參知政事。辛巳,遣使按行畿縣,民田被水者蠲其租。是月,河決澶州,西北流入御河。



 閏月辛卯,幸水磑觀魚。己酉,置三司總計度使。



 十一月丁巳,萬安州獻六眸龜。癸酉,還隴西州所獻白鷹。十二月辛丑,大雨雪。戊申,西川都巡檢使張玘與王小波戰江原縣,死之。小波中流矢死,眾推其黨李順為帥。



 五年春正月甲寅朔,不受朝,群臣詣閣拜表稱賀。戊午,李順陷漢州,已未,陷彭州。乙丑,慮囚,流罪以下釋之。己
 巳,李順陷成都,知府郭載奔梓州,順入據之,賊兵四出攻劫州縣。遣使振宋、亳、陳、穎州饑民,別遣決諸路刑獄,應因饑劫藏粟,誅為首者,餘減死。癸酉,以侍衛馬軍都指揮使李繼隆為河西行營都部署,討李繼遷。甲戌,命昭宣使王繼恩為兩川招安使,討李順。詔諸州能出粟貸饑民者賜爵。辛巳,詔除兩京諸州淳化三年逋負。



 二月乙未,李順分攻劍州,都監西京作坊副使上官正、成都監軍供奉官宿翰合擊,大破之,斬馘殆盡。丙午,幸南
 御莊觀稼。己酉,以益王元傑為淮南、鎮江等軍節度使,徙封吳王。辛亥,詔除劍南東西川、峽路諸州主吏民卒淳化五年以前逋負。



 三月乙亥,趙保忠為趙保吉所襲,奔還夏州,指揮使趙光嗣執之以獻,李繼隆帥師入夏州。交阯郡王黎桓遣使來貢。



 夏四月壬午朔,詔除天下主吏逋負。甲申,削趙保吉所賜姓名。丙戌,置起居院,初復起居注。以國子學復為國子監。辛卯,慮囚。大食國王遣使來貢。戊戌,赦諸州,除十惡、故劫殺、官吏犯正贓外,
 降死罪以下囚。己亥,王繼恩帥師過綿州,賊潰走,追殺及溺死者甚眾。庚子,復綿州。內殿崇班曹習破賊於老溪,復閬州。綿州巡檢使胡正遠帥兵進擊,復巴州。壬寅,西川行營擊賊於研口砦,破之,復劍州。癸卯,大雨。



 五月丁巳,西川行營破賊十萬眾,斬首三萬級,復成都,獲賊李順。其黨張餘復攻陷嘉、戎、瀘、渝,涪、忠、萬、開八州,開州監軍秦傳序死之。丙寅,河西行營送趙保忠至闕下,釋其罪,授右千牛衛上將軍,封宥罪侯。己巳,以知梓州張
 雍、都巡檢使盧斌嘗堅守卻賊,斌進擊解閬州圍,遂平蓬州,雍加給事中,斌領成州刺史。以少府監雷有終為諫議大夫、知成都府。庚午,賊攻夔州,峽路都大巡檢白繼贇、夔州巡檢使解守顒大敗其眾於西津口,斬首二萬級,獲舟千餘艘。辛未,降成都府為益州。壬申,右僕射李昉以司空致仕。甲戌,詔利州、興元府、洋州西縣民並給復一年。丙子,磔李順黨八人於鳳翔市。庚辰,初伏,帝親書綾扇賜近臣。



 六月辛卯,詔赦李順脅從詿誤。是月,
 都城大疫,分遣醫官煮藥給病者。賊攻施州,指揮使黃希遜擊走之。戊戌,峽路行營破賊於廣安軍,又破賊張罕二萬眾於嘉陵江口,又破於合州西方溪,俘斬甚眾。戊申,以侍衛步軍都指揮使高瓊為鎮州都部署。賊攻陵州,知州張旦擊破之。高麗遣使,以契丹來侵乞師。



 秋七月辛亥朔,賊攻眉州,知州李簡等堅守逾月,賊引去。癸亥,置江、淮、兩浙發運使。丙寅,除兩浙諸州民錢俶日逋負。甲戌,置威塞軍。乙亥,李繼遷遣使來貢。



 八月甲申,
 詔有司講求大射儀注。癸巳,以內班為黃門。甲午,置宣政使,以宦者昭宣使王繼恩為之。乙未,詔釋劍南、峽路諸州亡命。戊戌,以通遠軍復為環州,置清遠軍。庚子,大雨。貝州言驍捷卒劫庫兵為亂,推都虞候趙咸雍為帥,轉運使王嗣宗率屯兵擊敗之,擒咸雍,磔於市。辛丑,詔遣知益州張詠赴部,得便宜從事。癸卯,以參知政事趙昌言為西川、峽路招安馬步軍都部署,尋詔昌言駐鳳翔,遣內侍押班衛紹欽往行營指揮軍事。峽路行營破
 賊帥張餘,復雲安軍。李繼遷使其弟奉表待罪。



 九月庚戌朔,戶部尚書辛仲甫以太子少保致仕。甲寅,賜三司錢百萬,募能言司事之利便者,量事賞之,盡則再給以備賞。己未,罷諸州榷酤。改黃門院為內侍省,以黃門班院為內侍省內侍班院,入內黃門班院為內侍省入內侍班院。辛酉,遣使分行宋、亳、陳、穎、泗、壽、鄧、蔡等州按行民田,被水及種蒔不及者並蠲其租。壬申,以襄王元侃為開封尹,改封壽王。大赦,除十惡、故謀劫鬥殺、官吏犯
 正贓外,諸官先犯贓罪配隸禁錮者放還。乙亥,以左諫議大夫寇準參知政事。丁丑,以蜀部漸平,下詔罪己。戊寅,西川行營言衛紹欽破賊於學射山,別將楊瓊復蜀州,曹習等又破賊於安國鎮,誅其帥馬太保。



 冬十月庚辰,詔釋殿前司逃軍親屬之禁錮者。西川行營指揮使張嶙殺其將王文壽以叛,遣使招撫其眾,遂共斬嶙首以降。乙未,楊瓊等復邛州。乙巳,改青州平盧軍為鎮海軍,杭州鎮海軍為寧海軍。



 十一月庚戌,遣使諭李繼遷,
 賜以器幣、茶藥、衣服。丙辰,賜近臣飛白書。庚申,詔江南西路及荊湖南北路、嶺南溪洞接連及蕃商、外國使誘子女出境者捕之。癸亥,賊攻眉州,崇儀使宿翰等擊敗之,斬其偽中書令吳蘊。丙寅,幸國子監,賜直講孫奭緋魚,因幸武成王廟,復幸國子監,令奭講《尚書》,賜以束帛。大寒,賜禁衛諸軍緡錢有差。十二月戊寅朔,日當食,雲陰不見。辛巳,命樞密直學土張鑒、西京作坊副使馮守規安撫西川。丙戌,命諸王畋近郊。弛忠、靖二州刑徒。庚
 寅,宿翰等引兵趨嘉州,偽知州王文操以城降。乙未,秘書丞張樞坐知榮州降賊,棄市。辛丑,以三司兩京、十道復歸三部,各置使一員,每部置判官、推官、都監,分勾院為三。



 至道元年正月戊申朔,改元,赦京畿系囚,流罪以下遞降一等,杖罪釋之。蠲諸州逋租,蠲陜西諸州去年秋稅之半。丙辰,詔作上清宮成。丁巳,涼州吐蕃當專以良馬來獻。戊午,占城國王楊陀排遣使來貢。辛酉,上禦乾元
 門觀燈。癸亥,契丹大將韓德威誘黨項勒浪、嵬族自振武犯邊,永安節度使折御卿邀擊,敗之於子河㲼,勒浪等乘亂反擊德威,遂殺其將突厥大尉、司徒、舍利等,獲吐渾首領一人,德威僅以身免。戊辰,以翰林學士錢若水為右諫議大夫、同知樞密院事,樞密副使劉昌言罷為給事中。以宣祖舊第作洞真宮成。甲戌,李繼遷遣使以良馬、橐駝來貢。



 二月甲申,命宰相禱雨。令川峽諸州瘞暴骸。戊戌,以旱慮囚,減流罪以下。丙午,雨。嘉州函賊
 帥張餘首送西川行營,餘黨悉平。蠲襄、唐、均、汝、隨、鄧、歸、峽等州去年逋租。振亳州、房州、光化軍饑,遣使貸之。



 三月庚申,詔求直言。辛酉,以會州觀察使、知清遠軍田紹斌為靈州兵馬都部署。己巳,廢邵武軍歸化縣金坑。



 夏四月癸未,吏部尚書、平章事呂蒙正罷為右僕射,以參知政事呂端為戶部侍郎、平章事。宣徽北院使、知樞密院事柴禹錫罷為鎮寧軍節度使,參知政事蘇易簡為禮部侍郎,以翰林學士張洎為給事中、參知政事。甲申,
 以宣徽北院使、同知樞密院事趙鎔知樞密院事。乙酉,契丹犯雄州,知州何承矩擊敗之,斬其鐵林大將一人。辛丑,遣使分決諸路刑獄,劫賊止誅首惡,降流罪以下一等。壬寅,慮囚。甲辰,大雨,雷電。開寶皇后宋氏崩。



 六月乙酉,購求圖書。丙戌,遣使諭李繼遷,授以鄜州節度使,繼遷不奉詔。丁亥,以銀州左都押衙張浦為銀青光祿大夫、檢校工部尚書、鄭州刺史兼御史大夫,充本州團練使。己亥,許士庶工商服紫。是月,大熱,民有暍死者。



 秋
 七月丙寅,除陳、許等九州及光化軍今年夏稅。



 八月壬辰,詔立壽王元侃為皇太子,改名恆,兼判開封府。大赦,文武常參官子為父後見任官者,賜勛一轉。癸巳,以尚書左丞李至、禮部侍郎李沆並兼太子賓客。癸卯,禁西北緣邊諸州民與內屬戎人昏娶。



 九月丙午,西南蕃牂牁諸蠻來貢,詔封西南蕃主龍漢𤩊為歸化王。丁卯,御朝元殿冊皇太子。庚午,清遠軍言李繼遷入寇,率兵擊走之。



 冬十月甲戌朔,皇太子讓宮僚稱臣,許之。乙丑,陜
 西轉運使鄭文寶坐撓邊,責授藍田縣令。



 十一月己未,閱武使殿。是月,以峰州團練使上官正、右諫議大夫雷有終並為西川招安使,召王繼恩歸闕。十二月甲戌,群臣奉表加上尊號曰法天崇道上聖至仁皇帝,凡五上,不許。契丹犯邊,折御卿率兵御之,卒於師。斬馬步軍都軍頭孫贊於軍中。庚辰,新渾儀成。



 二年春正月辛亥,祀天地於圜丘,大赦,中外文武加恩。丁卯,廢諸州司理判官。



 二月壬申朔,司空致仕李昉薨。
 戊寅,以越王元份為杭州大都督兼領越州,吳王元傑為揚州大都督兼領壽州。己卯,以徐國公元偓為洪州都督、鎮南軍節度使,涇國公元偁為鄂州都督、武清軍節度使。庚辰,以御史中丞李昌齡為給事中、參知政事。辛巳,以呂蒙正為左僕射,宋琪為右僕射。乙未,定任子官制。



 三月丙寅,以京師旱,遣中使禱雨。戊辰,命宰臣祀郊廟、社稷禱雨。



 夏四月甲戌,命侍衛馬軍都指揮使李繼隆為環、慶等州都部署,殿前都虞候範廷召副之,討
 李繼遷。癸未,雨。



 五月癸卯,李繼遷寇靈州。



 六月戊戌,黔州言蠻寇鹽井,巡檢使王惟節戰死。是月,亳州蝗。



 秋七月己亥朔,命殿前都指揮使王超為夏、綏、麟、府州都部署。庚子,詔作壽寧觀成。丙寅,給事中、參知政事寇準罷守本官。戊辰,蠲峽路諸州民去年逋租。是月,汴水決穀熟縣,許、宿、齊三州蝗抱草死。



 閏月庚寅,詔江、浙、福建民負人錢沒入男女者還其家,敢匿者有罪。



 八月辛丑,密州言蝗不為災。



 九月戊寅,右僕射宋琪薨。詔川峽諸州
 民家先藏兵器者,限百日悉送官,匿不以聞者斬。己卯,夏州、延州行營言破李繼遷於烏白池,獲未幕軍主、吃囉指揮使等二十七人,繼遷遁。甲申,會州觀察使、環慶副都部署田紹斌貶右監門衛率府副率、虢州安置。丙戌,秦、晉諸州地晝夜十二震。丙申,詔廢衢州冶。



 冬十月己未,詔以池州新鑄錢監為永豐監。



 十一月丁卯朔,增司天新歷為一百二十甲子。戊寅,置簽署提點樞密、宣徽院諸房公事。辛卯,許州群盜劫郾城縣居民,巡檢李
 昌習鬥死,都巡檢使王正襲擊之,獲賊首宋斌及餘黨,皆斬於市。甲午,禁淮南通行鹽稅。十二月,命宰相以下百官詣諸寺觀禱雪。甲寅,雨雪。大有年。是歲,處州稻再熟。



 三年春正月丙子,以戶部侍郎溫仲舒、禮部侍郎王化基並參知政事,給事中李惟清同知樞密院事,參知政事張洎罷為刑部侍郎。乙酉,孝章皇后陪葬永昌陵。辛卯,以侍衛馬步軍都虞候傅潛為延州路都部署,殿前
 都虞候王昭遠為靈州路都部署。



 二月丙申朔,靈州行營破李繼遷。辛丑,帝不豫。甲辰,降京畿死罪囚,流以下釋之。壬戌,大食、賓同隴國並來貢。



 三月丁卯,占城國來貢。壬辰,不視朝。癸巳,追班於萬歲殿,宣詔令皇太子柩前即位。是日崩,年五十九。在位二十二年,殯於殿之西階。群臣上尊謚曰神功聖德文武皇帝,廟號太宗。十月己酉,葬永熙陵。



 贊曰:帝沈謀英斷,慨然有削平天下之志。既即大位,陳
 洪進、錢俶相繼納土。未幾,取太原,伐契丹,繼有交州、西夏之役。乾戈不息,天災方行,俘馘日至,而民不知兵;水旱螟蝗,殆遍天下,而民不思亂。其故何也?帝以慈儉為寶,服浣濯之衣,毀奇巧之器,卻女樂之獻,悟畋游之非。絕遠物,抑符瑞,閔農事,考治功。講學以求多聞,不罪狂悖以勸諫士,哀矜惻怛,勤以自勵,日晏忘食。至於欲自焚以答天譴,欲盡除天下之賦以紓民力,卒有五兵不試、禾稼薦登之效。是以青、齊耆耋之叟,願率子弟治道
 請登禪者,接踵而至。君子曰:「得乎丘民而為天子」,帝之謂乎?故帝之功德,炳煥史牒,號稱賢君。若夫太祖之崩不逾年而改元,涪陵縣公之貶死,武功王之自殺,宋後之不成喪,則後世不能無議焉。



\end{pinyinscope}