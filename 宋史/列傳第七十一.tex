\article{列傳第七十一}

\begin{pinyinscope}

 韓琦子
 忠彥曾公亮子孝寬孝廣孝蘊陳升之吳充王珪從父罕從兄琪



 韓琦,字稚圭,相州安陽人。父國華,自有傳。琦風骨秀異,弱冠舉進士,名在第二。方唱名,太史奏日下五色雲見,
 左右皆賀。授將作監丞、通判淄州,入直集賢院、監左藏庫。時方貴高科,多徑去為顯職,琦獨滯筦庫,眾以為非宜,琦處之自若。禁中需金帛,皆內臣直批旨取之,無印可驗,琦請復舊制,置傳宣合同司,以相防察。又每綱運至,必俟內臣監蒞,始得受,往往數日不至,暴露廡下。衙校以為病,琦奏罷之。



 歷開封府推官、三司度支判官,拜右司諫。時宰相王隨、陳堯佐,參知政事韓億、石中立,在中書罕所建明,琦連疏其過,四人同日罷。又請停內降,
 抑僥幸。凡事有不便,未嘗不言,每以明得失、正紀綱、親忠直、遠邪佞為急,前後七十餘疏。王曾為相,謂之曰:「今言者不激,則多畏顧,何補上德?如君言,可謂切而不迂矣。」曾聞望方崇,罕所獎與,琦聞其語,益自信。權知制誥。



 益、利歲饑,為體量安撫使。異時郡縣督賦調繁急,市上供綺繡諸物不予直,琦為緩調蠲給之,逐貪殘不職吏,汰冗役數百,活饑民百九十萬。趙元昊反,琦適自蜀歸,論西師形勢甚悉,即命為陜西安撫使。劉平與賊戰,敗,
 為所執,時宰入他誣,收系平子弟,琦辨直其冤。



 進樞密直學士,副夏竦為經略安撫、招討使。詔遣使督出兵,琦亦欲先發以制賊,而合府固爭,元昊遂寇鎮戎。琦畫攻守二策馳入奏,仁宗欲用攻策,執政者難之。琦言:「元昊雖傾國入寇,眾不過四五萬人,吾逐路重兵自為守,勢分力弱,遇敵輒不支。若並出一道,鼓行而前,乘賊驕惰,破之必矣。」乃詔鄜延、涇原同出征。既還營,元昊來求盟。琦曰:「無約而請和者,謀也。」命諸將戒嚴,賊果犯山外。琦
 悉兵付大將任福,令自懷遠城趨德勝砦出賊後,如未可戰,即據險置伏,要其歸。及行,戒之至再。又移檄申約,茍違節度,雖有功,亦斬。福竟為賊誘,沒於好水川。竦使人收散兵,得琦檄於福衣帶間,言罪不在琦。琦亦上章自劾,猶奪一官,知秦州,尋復之。



 會四路置帥,以琦兼秦鳳經略安撫、招討使。慶歷二年,與三帥皆換觀察使,範仲淹、龐籍、王沿不肯拜,琦獨受不辭。未幾,還舊職,為陜西四路經略安撫、招討使,屯涇州。琦與範仲淹在兵間
 久,名重一時,人心歸之,朝廷倚以為重,故天下稱為「韓範」。東兵從宿衛來,不習勞苦,琦奏增土兵以代戍,建德順軍以蔽蕭關、嗚沙之道。方謀取橫山,規河南,而元昊稱臣,召為樞密副使。



 元昊介契丹為援,強邀索無厭,宰相晏殊等厭兵,將一切從之。琦陳其不便,條所宜先行者七事:一曰清政本,二曰念邊計,三曰擢材賢,四曰備河北,五曰固河東,六曰收民心,七曰營洛邑。繼又陳救弊八事,欲選將帥,明按察,豐財利,遏僥幸,進能吏,退不
 才,謹入官,去冗食。謂:「數者之舉,謗必隨之,願委計輔臣,聽其注措。」帝悉嘉納。遂宣撫陜西,討平群盜張海、郭邈山;禁卒羸老不任用者,悉汰之;盡修鄜延城障,須敵悉歸所侵地,乃許和。歸陳西北四策,以為:「今當以和好為權宜,戰守為實務。請繕甲厲兵,營修都城,密定討伐之計。」



 時二府合班奏事,琦必盡言,雖事屬中書,亦指陳其實。同列或不悅,帝獨識之,曰:「韓琦性直。」琦與範仲淹、富弼皆以海內人望,同時登用,中外跂想其勛業。仲淹等
 亦以天下為己任,群小不便之,毀言日聞。仲淹、弼繼罷,琦為辨析,不報。尹洙與劉滬爭城水洛事,琦右洙,朝論不謂然。乃請外,以資政殿學士知揚州,徙鄆州、成德軍、定州。兼安撫使,進大學士,又加觀文殿學士。



 初,定州兵狃平貝州功,需賞賚,出怨語,至欲噪城下。琦聞之,以為不治且亂,用軍制勒習,誅其尤無良者。士死攻戰,則賞賻其家,籍其孤嫠繼稟之,威恩並行。又仿古三陣法,日月訓齊之,由是中山兵精勁冠河朔。京師發龍猛卒戍
 保州,在道為人害,至定,琦悉留不遣,易素教者使之北,又振活饑民數百萬。璽書褒激,鄰道視以為準。



 拜武康軍節度使、知並州。承受廖浩然,怙中貴勢貪恣,既誣逐前帥李昭亮,所為益不法,琦奏還之,帝命鞭諸本省。契丹冒占天池廟地,琦召其酋豪,示以曩日彼所求修廟檄,無以對,遂歸我斥地。既又侵耕陽武砦地,琦鑿塹立石以限之。始,潘美鎮河東,患寇鈔,令民悉內徙,而空塞下不耕,於是忻、代、寧化、火山之北多廢壤。琦以為此皆
 良田,今棄不耕,適足以資敵,將皆為所有矣。遂請距北界十里為禁地,其南則募弓箭手居之,墾田至九千六百頃。久之,求知相州。



 嘉祐元年,召為三司使,未至,迎拜樞密使。三年六月,拜同中書門下平章事、集賢殿大學士。六年閏八月,遷昭文館大學士、監修國史,封儀國公。帝既連失三王,自至和中病疾,不能御殿。中外惴恐,臣下爭以立嗣固根本為言,包拯、範鎮尤激切。積五六歲,依違未之行,言者亦稍怠。至是,琦乘間進曰:「皇嗣者,天
 下安危之所系。自昔禍亂之起,皆由策不早定。陛下春秋高,未有建立,何不擇宗室之賢者,以為宗廟社稷計?」帝曰:「後宮將有就館者,姑待之。」已又生女。



 一日,琦懷《漢書·孔光傳》以進,曰:「成帝無嗣,立弟之子。彼中材之主,猶能如是,況陛下乎。願以太祖之心為心,則無不可者。」又與曾公亮、張升、歐陽修極言之。會司馬光、呂誨皆有請,琦進讀二疏,未及有所啟,帝遽曰:「朕有意久矣,誰可者?」琦皇恐對曰:「此非臣輩所可議,當出自聖擇。」帝曰:「宮中
 嘗養二子,小者甚純,近不慧,大者可也。」琦請其名,帝以宗實告。宗實,英宗舊名也。琦等遂力贊之,議乃定。



 英宗居濮王喪,議起知宗正。琦曰:「事若行,不可中止。陛下斷自不疑,乞內中批出。」帝意不欲宮人知,曰:「只中書行足矣。」命下,英宗固辭。帝復問琦,琦對曰:「陛下既知其賢而選之,今不敢遽當,蓋器識遠大,所以為賢也。願固起之。」英宗既終喪,猶堅臥不起。琦言:「宗正之命初出,外人皆知必為皇子,不若遂正其名。」乃下詔立為皇子。明年,英
 宗嗣位,以琦為仁宗山陵使,加門下侍郎,進封衛國公。



 琦既輔立英宗,門人親客,或從容語及定策事,琦必正色曰:「此仁宗聖德神斷,為天下計,皇太后內助之力,臣子何與焉。」英宗暴得疾,太后垂簾聽政。帝疾甚,舉措或改常度,遇宦官尤少恩。左右多不悅者,乃共為讒間,兩宮遂成隙。琦與歐陽修奏事簾前,太后嗚咽流涕,具道所以。琦曰:「此病固爾,病已,必不然。子疾,母可不容之乎?」修亦委曲進言,太后意稍和,久之而罷。後數日,琦獨見
 上,上曰:「太后待我無恩。」琦對曰:「自古聖帝明王,不為少矣。然獨稱舜為大孝,豈其餘盡不孝耶?父母慈愛而子孝,此常事不足道;惟父母不慈,而子不失孝,乃為可稱。但恐陛下事之未至爾,父母豈有不慈者哉。」帝大感悟。及疾愈,琦請乘輿因禱雨具素服以出,人情乃安。太后還政,拜琦右僕射,封魏國公。



 夏人寇大順,琦議停歲賜,絕和市,遣使問罪。樞密使文彥博難之,或舉寶元、康定事,琦曰:「諒祚,狂童也,非有元昊智計,而邊備過當時遠
 甚。亟詰之,必服。」既而諒祚上表謝,帝顧琦曰:「一如所料。」帝寢疾,琦入問起居,言曰:「陛下久不視朝,願早建儲,以安社稷。」帝頷之,即召學士草制,立穎王。



 神宗立,拜司空兼侍中,為英宗山陵使。琦執政三世,或病其專。御史中丞王陶劾琦不赴文德殿押班為跋扈。琦請去,帝為黜陶。永厚陵復土,琦不復入中書,堅辭位。除鎮安武勝軍節度使、司徒兼侍中、判相州。入對,帝泣曰:「侍中必欲去,今日已降制矣。」賜興道坊宅一區,擢其子忠彥秘閣校
 理。琦辭兩鎮,乃但領淮南。



 會種諤擅取綏州,西邊俶擾,改判永興軍,經略陜西。琦言:「邊臣肆意妄作,棄約基亂,願召二府亟決之。」琦入辭,曾公亮等方奏事,乞與琦同議。帝召之,琦曰:「臣前日備員政府,所當共議。今日,藩臣也,不敢預聞。」又言:「王陶指臣為跋扈,今陛下乃舉陜西兵柄授臣,復有劾臣如陶者,則臣赤族矣。」帝曰:「侍中猶未知朕意邪?」琦初言綏州不當取,已而夏人誘殺楊定,琦復言,賊既如此,綏今不可棄。」樞密院以初議詰之,琦
 具論其故,卒存之。



 熙寧元年七月,復請相州以歸。河北地震、河決,徒判大名府,充安撫使,得便宜從事。王安石用事,出常平使者散青苗錢。琦亟言之。帝袖其疏以示宰臣,曰:「琦真忠臣,雖在外,不忘王室。朕始謂可以利民,今乃害民如此。且坊郭安得青苗,而亦強與之乎?」安石勃然進曰:「茍從其欲,雖坊郭何害。」明日,稱疾不出。當是時,新法幾罷,安石復出,持前議益堅。琦又懇奏,安石下之條例司,令其屬疏駁,刊石頒天下。琦申辨愈切,不克
 從。於是請解四路安撫使,止領一路,安石欲沮琦,即從之。六年,還判相州。



 契丹來求代北地,帝手詔訪琦,琦奏言:



 臣觀近年以來,朝廷舉事,似不以大敵為恤。彼見形生疑,必謂我有圖復燕南意,故引先發制人之說,造為釁端。所以致疑,其事有七:高麗臣屬北方,久絕朝貢,乃因商舶誘之使來,契丹知之,必謂將以圖我。一也。強取吐蕃之地以建熙河,契丹聞之,必謂行將及我。二也。遍植榆柳於西山,冀其成長以制蕃騎。三也。創團保甲。四
 也。諸州築城鑿池。五也。置都作院,頒弓刀新式,大作戰車。六也。置河北三十七將。七也。契丹素為敵國,因事起疑,不得不然。



 臣昔年論青苗錢事,言者輒肆厚誣,非陛下之明,幾及大戮。自此,聞新法日下,不敢復言。今親被詔問,事系安危,言及而隱,死有餘罪。臣嘗竊計,始為陛下謀者,必曰治國之本,當先聚財積穀,募兵於民,則可以鞭笞四夷。故散青苗錢,使民出利;為免役之法,次第取錢;迨置市易務,而小商細民,無所措手。新制日下,更
 改無常,官吏茫然,不能詳記,監司督責,以刻為明。今農怨於甽畝,商嘆於道路,長吏不安其職,陛下不盡知也。夫欲攘斥四夷,以興太平,而先使邦本困搖,眾心離怨,此則為陛下始謀者大誤也。



 臣今為陛下計,謂宜遣使報聘,具言向來興作,乃修備之常,豈有他意;疆土素定,悉如舊境,不可持此造端,以隳累世之好。以可疑之形,如將官之類,因而罷去。益養民愛力,選賢任能,疏遠奸諛,進用忠鯁,使天下悅服,邊備日充。若其果自敗盟,則
 可一振威武,恢復故疆,攄累朝之宿憤矣。



 疏上,會安石再入相,悉以所爭地與契丹,東西七百里,論者惜之。八年,換節永興軍,再任,未拜而薨,年六十八。前一夕,大星隕於治所,櫪馬皆驚。帝發哀苑中,哭之慟。輟朝三日,賜銀三千兩,絹三千匹,發兩河卒為治塚,篆其碑曰「兩朝顧命定策元勛」。贈尚書令,謚曰忠獻,配享英宗廟庭。常令其子若孫一人官於相,以護丘墓。故事,三省長官,惟尚書令為尤重,贈者必兼他官。至琦,乃單贈。後又詔,雖
 當追策,不復更加師保,蓋貴之也。



 琦蚤有盛名,識量英偉,臨事喜慍不見於色,論者以重厚比周勃,政事比姚崇。其為學士臨邊,年甫三十,天下已稱為韓公。嘉祐、治平間,再決大策,以安社稷。當是時,朝廷多故,琦處危疑之際,知無不為。或諫曰:「公所為誠善,萬一蹉跌,豈惟身不自保,恐家無處所。」琦嘆曰:「是何言也。今臣盡力事君,死生以之。至於成敗,天也,豈可豫憂其不濟,遂輟不為哉。」聞者愧服。在魏都久,遼使每過,移牒必書名,曰:「以韓
 公在此故也。」忠彥使遼,遼主問知其貌類父,即命工圖之,其見重於外國也如此。



 琦天資樸忠,折節下士,無賤貴,禮之如一。尤以獎拔人才為急,儻公論所與,雖意所不悅,亦收用之,故得人為多。選飭群司,皆使奉法循理。其所建請,第顧義所在,無適莫心。在相位時,王安石有盛名,或以為可用,琦獨不然之。及守相,陛辭,神宗曰:「卿去,誰可屬國者,王安石何如?」琦曰:「安石為翰林學士則有餘,處輔弼之地則不可。」上不答。其鎮大名也,魏人為
 立生祠。相人愛之如父母,有鬥訟,傳相勸止,曰:「勿撓吾侍中也。」與富弼齊名,號稱賢相,人謂之「富韓」云。徽宗追論琦定策勛,贈魏郡王。子五人:忠彥、端彥、純彥、粹彥、嘉彥。端彥右贊善大夫。純彥官至徽猷閣直學士。粹彥為吏部侍郎,終龍圖閣學士。嘉彥尚神宗女齊國公主,拜駙馬都尉,終瀛海軍承宣使。



 忠彥字師樸,少以父任,為將作監簿,復舉進士。琦罷政,忠彥以秘書丞召試館職,除校理、同知太常禮院,為開
 封府判官、三司鹽鐵判官。出通判永寧軍,召還,為戶部判官。



 琦薨,服除,為直龍圖閣,擢天章閣待制、知瀛州。朝廷以夏人囚廢其主秉常,用兵西方,既下米脂等城砦數十,夏人求救於遼,遼人移書繼至。會遣使賀遼主生辰,神宗以命忠彥,遂以給事中奉使。遼遣趙資睦迓之,語及西事,忠彥曰:「此小役也,何問為?」遼主使其臣王言敷燕於館,言敷問:「夏國胡罪,而中國兵不解?無失兩朝之歡,則善矣。」忠彥曰:「問罪西夏,於二國之好何預乎?」



 使
 還。時官制行,章惇為門下侍郎,奏:「給事中東省屬官,封駁宜先稟而後上。忠彥奏:「朝廷之事,執政之所行也。事當封駁,則與執政固已異矣,尚何稟議之有。」詔從其請。左僕射王珪為南郊大禮使,事之當下者,自從其所畫旨。忠彥以官制駁之曰:「今事於南郊者,大禮使既不從中畫旨,處分出一時者,又不從中書奏審。官制之行,曾未期月,而廟堂自渝之,後將若之何?」乃詔事無鉅細,必經三省而後行。拜禮部尚書,以樞密直學士知定州。



 元
 祐中,召為戶部尚書,擢尚書左丞。弟嘉彥尚主,改同知樞密院事,遷知院事。哲宗親政,更用大臣,言者觀望,爭言垂簾時事。忠彥言:「昔仁宗始政,當時亦多譏斥章獻時事,仁宗惡其持情近薄,下詔戒飭。陛下能法仁祖用心,則善矣。」以觀文殿學士知真定府,移定州。忠彥在西府,以用兵西方非是,願以所取之地棄還之,以息民力。至是,言者以為言,降資政殿學士,改知大名府。徽宗即位,以吏部尚書召拜門下侍郎。忠彥陳四事:一曰廣仁
 恩,二曰開言路,三曰去疑似,四曰戒用兵。逾月,拜尚書右僕射兼中書侍郎。上用忠彥言,數下詔蠲天下逋負,盡還流人而甄敘之,忠直敢言若知名之士,稍見收用。



 進左僕射兼門下侍郎,封儀國公。而曾布為右相,多不協,言事者助布排忠彥,以觀文殿大學士知大名府。又以欽聖欲復廢後,為忠彥罪,再降太中大夫,懷州居住。又論忠彥在相位,不應棄湟州,謫崇信軍節度副使,濟州居住。逮復湟、鄯,又謫磁州團練副使。復太中大夫,遂
 以宣奉大夫致仕。卒,年七十二。子治,徽宗時,為太僕少卿,出知相州。以疾丐祠,命其子肖冑代之,別有傳。



 論曰:琦相三朝,立二帝,厥功大矣。當治平危疑之際,兩宮幾成嫌隙,琦處之裕如,卒安社稷,人服其量。歐陽修稱其「臨大事,決大議,垂紳正笏,不動聲色,措天下於泰山之安,可謂社稷之臣」。豈不信哉!忠彥世濟其美,繼登相位,宜矣。



 曾公亮,字明仲,泉州晉江人。舉進士甲科,知會稽縣。民
 田鏡湖旁,每患湖溢。公亮立斗門,洩水入曹娥江,民受其利。坐父買田境中,謫監湖州酒。久之,為國子監直講,改諸王府侍講。歲滿,當用故事試館職,獨獻所為文,授集賢校理、天章閣侍講、修起居注。擢天章閣待制,賜金紫。先是,待制不改服。仁宗面錫之,曰:「朕自講席賜卿,所以尊寵儒臣也。」遂知制誥兼史館修撰,為翰林學士、判三班院。三班吏叢猥,非賕謝不行,貴游子弟,多倚勢請謁。公亮掇前後章程,視以從事,吏不能舉手。以端明殿
 學士知鄭州,為政有能聲,盜悉竄他境,至夜戶不閉。嘗有使客亡橐中物,移書詰盜,公亮報:「吾境不藏盜,殆從者之廋耳。」索之,果然。復入為翰林學士、知開封府。未幾,擢給事中、參知政事。加禮部侍郎,除樞密使。嘉祐六年,拜吏部侍郎、同中書門下平章事、集賢殿大學士。



 公亮明練文法,更踐久,習知朝廷臺閣典憲,首相韓琦每咨訪焉。仁宗末年,琦請建儲,與公亮等共定大議。密州民田產銀,或盜取之,大理當以強。公亮曰:「此禁物也,取之
 雖強,與盜物民家有間矣。」固爭之,遂下有司議,比劫禁物法,盜得不死。初,東州人多用此抵法,自是無死者。



 契丹縱人漁界河,又數通鹽舟,吏不敢禁,皆謂:與之校,且生事。公亮言:「萌芽不禁,後將奈何?雄州趙滋勇而有謀,可任也。」使諭以指意,邊害訖息。英宗即位,加中書侍郎兼禮部尚書,尋加戶部尚書。帝不豫,遼使至不能見,命公亮宴於館,使者不肯赴。公亮質之曰:「錫宴不赴,是不虔君命也。人主有疾,而必使親臨,處之安乎?」使者即就
 席。神宗即位,加門下侍郎兼吏部尚書。



 熙寧二年,進昭文館大學士,累封魯國公。以老避位,三年九月,拜司空兼侍中、河陽三城節度使、集禧觀使。明年,起判永興軍。先是,慶卒叛,既伏誅,而餘黨越佚,自陜以西皆警備。閱義勇,益邊兵,移內地租賦,人情騷然。公亮一鎮以靜,次第奏罷之,專務裁抑冗費。長安豪喜造飛語,聲言營卒怨減削,謀以上元夜結外兵為亂,邦人大恐。或勸毋出游,公亮不為動,張燈縱觀,與賓佐竟夕乃歸。居一歲,還
 京師。旋以太傅致仕。元豐元年卒,年八十。帝臨哭,輟朝三日,贈太師、中書令,謚曰宣靖,配享英宗廟庭。及葬,御篆其碑首曰「兩朝顧命定策亞勛之碑」。



 公亮方厚莊重,沈深周密,平居謹繩墨,蹈規矩;然性吝嗇,殖貨至鉅萬,帝嘗以方張安世。初薦王安石,及同輔政,知上方向之,陰為子孫計,凡更張庶事,一切聽順,而外若不與之者。嘗遣子孝寬參其謀,至上前略無所異,於是帝益信任安石。安石德其助己,故引擢孝寬至樞密以報之。蘇軾
 嘗從容責公亮不能救正,公亮曰:「上與介甫如一人,此乃天也。」世譏其持祿固寵云。子孝寬,從子孝廣、孝蘊。



 孝寬字令綽,以蔭知桐城縣。選知咸平縣,民詣府訴雨傷麥,府以妄杖之。孝寬躬行田,辨其實,得蠲賦。除秘閣修撰、提點開封府界鎮縣。



 保甲法行,民相驚言且籍為兵。知府韓維上言,乞候農隙行之。孝寬榜十七縣,揭賞告捕扇惑者,民兵不敢訴,維之言不得行。入知審官東院、判刑部。



 熙寧五年,遷樞密都承旨,承旨用文臣,自孝
 寬始。擢拜樞密直學士、簽書樞密院。丁父憂,除喪,以端明殿學士知河陽,徙鄆。鄆有孟子廟,孝寬請於朝,得封鄒國公,配享孔子。連徙鎮,以吏部尚書召,道卒,年六十六。贈右光祿大夫。



 孝廣字仲錫。元豐末,為北外都水丞。元祐中,大臣議復河故道,召孝廣問之,言不可,出通判保州。久之,復為都水丞。前此,班行使臣部木筏至者,須校驗無所失亡,乃得送銓,監吏領賕謝,不時遣。孝廣治籍疏姓名,謹其去
 留,一歲中,歸選者百輩。



 除京西轉運判官,入為水部員外郎。河決內黃,詔孝廣行視,遂疏蘇村,鑿鉅野,導河北流,紓澶、滑、深、瀛之害。遷都水使者。洛水頻歲湓湧,浸嚙北岸,孝廣按河堤,得廢□口遺跡,曰:「此昔人所以殺水勢也。」即日浚決之,累石為防,自是無水患。出提點永興路刑獄,陜西、京西轉運副使,還為左司郎中,擢戶部侍郎,進尚書。坐錢帛不給費,罷為天章閣待制、知杭州。又以前聘契丹失奉使體,奪職。尋復之,移知潭州,加顯謨
 閣直學士、知鄆州。



 孝廣與胡安國、鄒浩善,皆大觀中忤時相,御史論之,復奪職知饒州。逾年,徙廣州,歷成德軍、太原府,得故職以卒,年六十,贈正議大夫。孝廣蒞官以嚴稱,獲盜,輒碎其手焉。



 孝蘊字處善。紹聖中,管幹發運司糶糴事,建言揚之瓜洲,潤之京口,常之SY牛,易堰為閘,以便漕運、商賈。既成,公私便之。提舉兩浙常平,改轉運判官,知臨江軍,召為左司員外郎,遷起居舍人。



 時京邑有盜,徽宗怒,期三日
 不獲,坐尹罪。孝蘊奏:「求盜急則遁益遠,小緩當自出。」從其言,得盜。崇寧建殿中省,擢為監。居數月,言者論其與張商英善,以集賢殿修撰出知襄州,徙江浙荊淮發運。泗州議開直河,以避漲溢沙石之害,孝蘊以淮、汴不相接,不可成。既而工役大集,竟成之,策勛第賞,辭不受。未幾,河果塞,召為戶部侍郎,帝嘗問右曹儲物幾何,疾作不能對。徙工部,以顯謨閣待制知杭州。其後坐累,連削黜,至貶安遠軍節度副使。



 宣和二年,始復天章閣待制、
 知歙州。方臘起青溪,孝蘊約敕郡內,無得奔擾,分兵守厄塞,有避賊來歸者,獲罪,使出境,人稍恃以安。會移青州,既行而歙陷,道改杭州,時賊已破杭,孝蘊單車至城下。城既克復,軍士多殺人,孝蘊下令,脅從者得自首,無輒殺,皆束手不敢驁。論功,進顯謨閣直學士,又加龍圖閣學士。卒,年六十五,贈通議大夫。



 陳升,之字□昜叔,建州建陽人。舉進士,歷知封州、漢陽軍,入為監察御史、右司諫,改起居舍人、知諫院。時俗好藏
 去交親尺牘,有訟,則轉相告言,有司據以推詰。升之謂:「此告訐之習也,請禁止之。」又言:「三館為搢紳華途,近者用人益輕,遂為貴游進取之階,請嚴其選。」詔自今臣僚乞子孫恩者,毋得除館閣。



 著作佐郎王瓘遇殿帥郭承祐於道,訶怒不下馬,執送府。升之言,京官不宜為節度使下馬,因劾承祐驕恣,解其任。張堯佐緣後宮親,為三司使,尋為宣徽使;內侍王守忠領兩鎮留後,求升正班;御史張忭補郡,久不召;彭思永論事,令窮問所從來;唐
 介擊宰相,斥嶺南:升之皆極諫。遷侍御史知雜事。凡任言責五年,所上數十百事,然持論不堅,以故不盡施用。



 擢天章閣待制、河北都轉運使,知瀛州、真定府,加龍圖閣直學士,復知諫院。上言:「天下州縣治否,朝廷不能周知,悉付之轉運使。今選用不精,又無考課,非暗滯罷懦,則凌肆刻薄,所以疾苦愁嘆,雍圩上聞。必欲垂意元元,宜從此始。」乃詔翰林學士承旨孫抃、權御史中丞張忭,與升之同領磨勘轉運使及提點刑獄功務。



 升之初為
 諫官時,嘗請抑絕內降,詔許有司執奏勿下。至是,申言之。詔委三省劾正其罪,仍揭於朝堂。文彥博乞罷相,升之慮樞密使賈昌朝復用,疏論其邪,昌朝卒罷去。遷樞密直學士、知開封府。歲餘,拜樞密副使。於是諫官御史唐介、範師道、呂誨、趙抃、王陶交章論升之陰結宦者,故得大用。仁宗以示升之,升之丐去。帝謂輔臣曰:「朕選用執政,豈容內臣預議邪。」乃兩罷之。以升之為資政殿學士、知定州,徙太原府。



 治平二年,復拜樞密副使。神宗立,
 以母老請郡,為觀文殿學士、知越州。熙寧元年,徙許,中道改大名府,過闕,留知樞密院。故事,樞密使與知院事不並置。時文彥博、呂公著既為使,帝以升之三輔政,欲稍異其禮,故特命之。明年,同制置三司條例司,與王安石共事。數月,拜中書門下平章事、集賢殿大學士。升之既相,遂請免條例司,其說以為宰相無所不統,所領職事,豈可稱司。安石曰:「古之六卿,即今之執政,有司馬、司徒、司空,各名一職,何害於理?」升之曰:「若制置百司條例
 則可,但今制置三司一官,則不可。」由是忤安石,稱疾歸臥逾十旬,帝數敦諭,乃出。會母喪,去位;終制,召為樞密使。足疾不能立朝,七年,冬祀,又不能相禮。拜鎮江軍節度使、同平章事、判揚州,封秀國公。卒,年六十九。贈太保、中書令,謚曰成肅。



 升之深狡多數,善傅會以取富貴。王安石用事,患正論盈庭,引升之自助。升之心知其不可,而竭力為之用,安石德之,故使先己為相。甫得志,即求解條例司,又時為小異,陽若不與之同者。世以是譏之,
 謂之「筌相」。升之初名旭,避神宗嫌名,改焉。



 吳充,字沖卿,建州浦城人。未冠舉進士,與兄育、京、方皆高第。調穀熟主簿,入為國子監直講、吳王宮教授。等輩多與宗室狎,充齒最少,獨以嚴見憚,相率設席受經。充作《六箴》以獻,曰視,曰聽,曰好,曰學,曰進德,曰崇儉。仁宗命繕寫賜皇族,英宗在藩邸,書之坐右。



 除集賢校理、判吏部南曹。選人胡宗堯者,翰林學士宿之子,坐小累,不得改京官。判銓歐陽修為之請,仇家譖修以為黨宿,詔
 出修同州。充言:「修以忠直擢侍從,不宜用讒逐。若以為私,則臣願與修同貶。」於是修復留,而充改知太常禮院。張貴妃薨,治喪越式,判寺王洙命吏以印紙行文書,不令同僚知。充移開封治吏罪,忤執政意,出知高郵軍。還為群牧判官、開封府推官,歷知陜州,京西、淮南、河東轉運使。



 英宗立,數問充所在,會入覲,語其為吳王宮教授時事,嘉勞之。尋權鹽鐵副使。熙寧元年,知制誥。神宗諭以任用意,曰:「先帝知卿久矣。遂同知諫院。言:「士大夫親
 沒,或蒿殯數十年,傷敗風化,宜限期使葬。」詔著為令。河北水災、地震,為安撫使。使還,王安石參知政事,充子安持,其婿也,引嫌解諫職,知審刑院,權三司使,為翰林學士。三年,拜樞密副使。王韶取洮州,蕃酋木征遁去,充請招還故地,縻以爵秩,使自領所部,永為外臣,無庸列置郡縣,殫財屈力。時方以開拓付韶,充言不用。



 八年,進檢校太傅、樞密使。充雖與安石連姻,而心不善其所為,數為帝言政事不便。帝察其中立無與,欲相之,安石去,遂
 代為同中書門下平章事、監修國史。充欲有所變革,乞召還司馬光、呂公著、韓維、蘇頌,乃薦孫覺、李常、程顥等數十人。光亦以充可告語,與之書曰:「自新法之行,中外洶洶。民困於煩苛,迫於誅斂,愁怨流離,轉死溝壑。日夜引領,冀朝廷覺悟,一變敝法,幾年於茲矣。今日救天下之急,茍不罷青苗、免役、保甲、市易,息征伐之謀,而欲求成效,猶惡湯之沸,而益薪鼓橐也。欲去此五者,必先別利害,以悟人主之心。欲悟人主之心,必先開言路。今病
 雖已深,猶未至膏肓,失今不治,遂為痼疾矣。」充不能用。



 王珪與充並相,忌充,陰掣其肘。而充素惡蔡確,確治相州獄,捕安持及親戚、官屬考治,欲鉤致充語,帝獨明其亡他。及確預政,充與議變法於前,數為所詘。安南師出無功,知諫院張璪又謂充與郭逵書,止其進兵,復置獄。充既數遭同列困毀,素病瘤,積憂畏,疾益侵。元豐三年三月,輿歸第,罷為觀文殿大學士、西太一宮使。逾月,卒,年六十。贈司空兼侍中,謚曰正憲。



 充內行修飭,事兄甚
 謹。為相務安靜。性沉密,對家人語,未嘗及國家事,所言於上,人莫知者。將終,戒妻子勿以私事干朝廷,帝益悲之。世謂充心正而力不足,譏其知不可而弗能勇退也。子安詩、安持。安詩在元祐時為諫官、起居郎。安持為都水使者,遷工部侍郎,終天章閣待制。安詩子儲、安持子侔,官皆員外郎,坐與妖人張懷素通謀,誅死。



 王珪,字禹玉,成都華陽人,後徙舒。曾祖永,事太宗為右補闕。吳越納土,受命往均賦,至則悉除無名之算,民皆
 感泣。使還,或言其多弛賦租。帝詰之,對曰:「使新附之邦,蒙天子仁恩,臣雖得罪,死不恨。」帝大悅。



 珪弱歲奇警,出語驚人。從兄琪讀其所賦,唶曰:「騏驥方生,已有千里之志,但蘭筋未就耳。」舉進士甲科,通判揚州。吏民皆少珪,有大校嫚不謹,捽置之法。王倫犯淮南,珪議出郊掩擊之,賊遁去。召直集賢院,為鹽鐵判官、修起居注。接伴契丹使,北使過魏,舊皆盛服入。至是,欲便服,妄云衣冠在後乘。珪命取授之,使者愧謝。遂為賀正旦使。進知制誥、
 知審官院,為翰林學士、知開封府。遭母憂,除喪,復為學士,兼侍讀學士。



 先是,三聖並侑南郊,而溫成廟享獻同太室。珪言:「三後並配,所以致孝也,而瀆乎饗帝。後宮有廟,所以廣恩也,而僭乎饗親。」於是專以太祖侑於郊,而改溫成廟為祠殿。嘉祐立皇子,中書召珪作詔,珪曰:「此大事也,非面受旨不可。」明日請對,曰:「海內望此舉久矣,果出自聖意乎?」仁宗曰:「朕意決矣。」珪再拜賀,始退而草詔。歐陽修聞而嘆曰:「真學士也。」帝宴寶文閣,作飛白書
 分侍臣,命珪識歲月姓名。再宴群王,又使為序,以所御筆、墨、箋、硯賜之。



 英宗立,當撰先帝謚,珪言:「古者賤不誄貴,幼不誄長,故天子稱天以誄之,制謚於郊,若云受之於天者。近制,唯詞臣撰議,庶僚不得參聞,頗違稱天之義。請令兩制共議。」從之。濮王追崇典禮,珪與侍從、禮官合議宜稱皇伯,三夫人改封大國,執政不以為然。其後三夫人之稱,卒如初議。始,珪之請對而作詔也,有密譖之者。英宗在位之四年,忽召至蕊珠殿,傳詔令兼端明
 殿學士,錫之盤龍金盆,諭之曰:「秘殿之職,非直器卿於翰墨間,二府員缺,即出命矣。曩有讒口,朕今釋然無疑。」珪謝曰:「非陛下至明,臣死無日矣。」神宗即位,遷學士承旨。珪典內外制十八年,最為久次,嘗因展事齋宮,賦詩有所感,帝見而憐之。熙寧三年,拜參知政事。九年,進同中書門下平章事、集賢殿大學士。



 元豐官制行,由禮部侍郎超授銀青光祿大夫。五年,正三省官名,拜尚書左僕射兼門下侍郎,以蔡確為右僕射。先是,神宗謂執政
 曰:「官制將行,欲新舊人兩用之。」又曰:「御史大夫,非司馬光不可。」珪、確相顧失色。珪憂甚,不知所出。確曰:「陛下久欲收靈武,公能任責,則相位可保也。」珪喜,謝確。帝嘗欲召司馬光,珪薦俞充帥慶,使上平西夏策。珪意以為既用兵深入,必不召光,雖召,將不至。已而光果不召。永樂之敗,死者十餘萬人,實珪啟之。



 八年,帝有疾,珪白皇太后,請立延安郡王為太子。太子立,是為哲宗。進珪金紫光祿大夫,封岐國公。五月,卒於位,年六十七。特輟朝五
 日,賻金帛五千,贈太師,謚曰文恭。賜壽昌甲第。



 珪以文學進,流輩咸共推許。其文閎侈瑰麗,自成一家,朝廷大典策,多出其手,詞林稱之。然自執政至宰相,凡十六年,無所建明,率道諛將順。當時目為「三旨相公」,以其上殿進呈,云「取聖旨」;上可否訖,云「領聖旨」;退諭稟事者,云「已得聖旨」也。紹聖中,邢恕謗起,黃履、葉祖洽、劉拯交論珪元豐末命事,以為當時兩府大臣,嘗議奏請建儲,珪輒語李清臣云:「他自家事,外庭不當管。」恕又誘教高遵裕
 子士京上奏,言珪欲立雍王,遣士京故兄士充,傳道言語於禁中。珪由是得罪,追貶萬安軍司戶參軍,削諸子籍。徽宗即位,還其官封。蔡京秉政,復奪贈謚。政和中,又復之。珪季父罕,從兄琪。



 罕字師言,以蔭知宜興縣。縣多湖田,歲訴水,輕重失其平。罕躬至田處,列高下為圖,明年訴牒至,按圖標之,某戶可免,某戶不可免,眾皆服。範仲淹在潤,奏下其式於諸道。西方用兵,仍年科箭羽於東南,價踴貴,富室至豫
 貯以待鬻。罕白郡守,倍其直市之,而令民輸錢。旁州聞之,皆願如常州法。累遷戶部判官。修太宗別廟,中貴人大慮材,將一新之。罕白是特歲久丹漆黯暗,但當致飾耳,榱櫨皆如故,唯易一楹,省緡錢十萬。



 出為廣東轉運使。儂智高入寇,罕行部在潮,廣州守仲簡自圍中遣書邀罕,罕報曰:「吾家亦受困,非不欲歸,顧獨歸無益,當求所以相濟者。」遂還惠州。州之惡少年正相率為盜,里落驚擾,惠人要罕出城,及郊,遮道求救護者數千計。罕擇
 父老可語者問以策,曰:「吾屬皆有田客,欲給以兵,使相保聚。」罕曰:「有田客者如是,得矣,無者奈何?」乃呼耆長發里民,補壯丁,每長二百人;又令邑尉增弓手二千。巳時下令,約申而集。募有方略者,許以官秩、金帛,使為甲首。久之,無至者。有婦人訴為僕奪釵珥,捕得之,並執奪攘者十八輩,皆梟首決口置道左,傳曰:「此耆長發為壯丁不肯行者也。」觀者始有怖色。至期,得六百人,尉所部亦至。於是染庫帛為旗;授之。割牛革為盾形,柔之湯中,每
 盾削竹簽十六,穿於革,以木為鼻,使持之自蔽。斷苦竹數千,銛其末,使操為兵。悉出公私戎器。檄告屬城,仿而行之。數日,眾大振,向之惡少年,皆隸行伍,無敢動。乃簡卒三千,方舟建旗,伐鼓作樂,順流而下。將至廣,悉眾登岸,斬木為鹿角,積高數仞,營於南門。智高戴黃蓋臨觀,相去三十步,見已嚴備,不敢犯。罕徐開門而入,智高遂解去。時南道郵驛斷絕,罕上事,不得通;而提點刑獄鮑軻遁處南雄,數具奏。及賊平,軻受賞,罕謫監信州酒。安
 撫使孫沔言罕實有功,復以為西路轉運使。或傳智高不死,走火峒,儂宗旦據險聚眾,邕守蕭注謀擊之。罕呼宗旦子日新謂之曰:「汝父內為交址所仇,外為邊將希賞之餌,非計也。汝歸報,擇利而為之。於是父子俱降。



 徙知潭州。擢戶部、度支副使,復為潭州。為政務適人情,不加威罰。有狂婦數訴事,出言無章,卻之則勃罵,前守每叱逐之。罕獨引至前,委曲徐問,久稍可曉,乃本為人妻,無子,夫死,妾有子,遂逐婦而據家資,屢訴不得直,因憤
 恚發狂。罕為治妾而反其資,婦良愈,郡人傳為神明。監司上治狀,敕書褒諭,賜絹三百。徙知明州。以光祿卿卒,年八十。兄之子珪少孤,罕教養有恩,後珪貴,每予書,必以盛滿為戒云。



 琪字君玉,兒童時已能為歌詩。起進士,調江都主簿。上時務十二事,請建義倉,置營田,減度僧,罷鬻爵,禁錦綺、珠貝,行鄉飲、籍田,復制科,興學校。仁宗嘉之,除館閣校勘、集賢校理。



 帝宴太清樓,命館閣臣作《山水石歌》,琪獨
 蒙褒賞。詔通判舒州。歲饑,奏發廩救民,未報,先振以公租,守以下皆不聽,琪挺身任之。知復州,民毆佃客死,吏論如律。琪疑之,留未決,已而新制下,凡如是者聽減死。歷開封府推官,直集賢院、兩浙淮南轉運使、修起居注、鹽鐵判官、判戶部勾院、知制誥。嘗入對便殿,帝從容謂曰:「卿雅有心計,若三司缺使,當無以易卿。」



 會奉使契丹,因感疾還,上介誣其詐,責信州團練副使。久之,以龍圖閣待制知潤州。轉運使欲浚常、潤漕河,琪陳其不便,詔
 寢役。而後議者卒請廢古城埭,破古涵管而浚之,河反狹,舟不得方行,公私交病。徙知江寧。先是,府多火災,或托以鬼神,人不敢求。琪召令廂邏,具為作賞捕之法,未幾,得奸人,誅之,火患遂息。復知制誥,加樞密直學士、知鄧州,徙揚州,入判太常寺,又出知杭州,復為揚州、潤州。以禮部侍郎致仕。卒,年七十二。



 琪性孤介,不與時合。數臨東南名鎮,政尚簡靜。每疾俗吏飾廚傳以沽名譽,故待賓客頗闊略。間造飛語起謗,終不自恤。葬於真州。詔
 真、揚二州發卒護其窆,蓋異數也。



 論曰:公亮靜重鎮浮,練達典憲,與韓琦並相,號稱老成。升之自為言官,即著直聲。然皆挾術任數,公亮疾琦專任,薦王安石以間之,升之陰助安石,陽為異同,以避清議,二人措慮如此,豈誠心謀國者乎?新法之行,何望其能正救也。及安石去位,充、珪實代之,天下喁喁,思有所休息。充力不逮心,同僚左掣右伺,至鞅鞅以死,傷哉,其不足與有行也。珪容身固位,於勢何所重輕,而陰忌正
 人,以濟其患失之謀,鄙夫可與事君也與哉!



\end{pinyinscope}