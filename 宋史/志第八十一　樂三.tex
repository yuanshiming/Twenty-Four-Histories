\article{志第八十一 樂三}

\begin{pinyinscope}

 元豐三年五月,詔秘書監致仕劉幾赴詳定所議樂,以禮部侍郎致仕範鎮與幾參考得失。而幾亦請命楊傑同議,且請如景祐故事,擇人修制大樂。詔可。



 初,言大
 樂七失:一曰歌不永言,聲不依永,律不和聲。蓋金聲舂容,失之則重;石聲溫潤,失之則輕;土聲函胡,失之則下;竹聲清越,失之則高;絲聲纖微,失之則細;革聲隆大,失之則洪;匏聲叢聚,失之則長;木聲無餘,失之則短。惟人稟中和之氣而有中和之聲,八音、律呂皆以人聲為度,言雖永,不可以逾其聲。今歌者或詠一言而濫及數律,或章句已闋而樂音未終,所謂歌不永言也。請節其煩聲,以一聲歌一言。且詩言人志,詠以為歌。五聲隨歌,是
 謂依詠;律呂協奏,是謂和聲。先儒以為依人音而制樂,托樂器以寫音,樂本效人,非人效樂者,此也。今祭祀樂章並隨月律,聲不依詠,以詠依聲,律不和聲,以聲和律,非古制也。



 二曰八音不諧,鐘磬闕四清聲。虞樂九成,以簫為主;商樂和平,以磬為依;周樂合奏,以金為首。鐘、磬、簫者,眾樂之所宗,則天子之樂用八,鐘、磬、簫,眾樂之本,乃倍之為十六。且十二者,律之本聲;而四者,應聲也。本聲重大為君父,應聲輕清為臣子,故其四聲曰清聲,或
 曰子聲也。李照議樂,始不用四清聲,是有本而無應,八音何從而諧哉?今巢笙、和笙,其管十九,以十二管發律呂之本聲,以七管為應聲。用之已久,而聲至和,則編鐘、磬、簫宜用四子聲以諧八音。



 三曰金石奪倫。樂奏一聲,諸器皆以其聲應,既不可以不及,又不可以有餘。今琴、瑟、塤、篪、笛、簫、笙、阮、箏、築奏一聲,則鎛鐘、特磬、編磬連擊三聲;聲煩而掩眾器,遂至奪倫,則鎛鐘、特磬、編鐘、編磬節奏與眾器同,宜勿連擊。



 四曰舞不像成。國朝郊廟之
 樂,先奏文舞,次奏武舞,而武舞容節六變:一變象六師初舉,所向宜北;二變象上黨克平,所向宜北;三變象維揚底定,所向宜東南;四變象荊湖來歸,所向宜南;五變象邛蜀納款,所向宜西;六變象兵還振旅,所向宜北而南。今舞者發揚蹈厲、進退俯仰,既不足以稱成功盛德,失其所向,而文舞容節尤無法度,則舞不像成也。



 五曰樂失節奏。樂之始,則翕然如眾羽之合;縱之,純如也;節奏明白,皦如也;往來條理,繹如也:然後成。今樂聲不一,混殽無敘,
 則失於節奏,非所謂成也。



 六曰祭祀、饗無分樂之序。蓋金石眾作之謂奏,詠以人聲之謂歌。陽律必奏,陰呂必歌,陰陽之合也。順陰陽之合,所以交神明、致精意。今冬至祀天,不歌大呂;夏至祭地,不奏太簇;春饗祖廟,不奏無射;秋饗後廟,不歌小呂。而四望山川無專祠用樂之制,則何以贊導宣發陰陽之氣而生成萬物哉?



 七曰鄭聲亂雅。然朱紫有色而易別,雅、鄭無象而難知,聖人懼其難知也,故定律呂中正之音,以示萬世。今古器尚存,
 律呂悉備,而學士、大夫不講考擊,奏作委之賤工,則雅、鄭不得不雜。願審調鐘管用十二律還宮均法,令上下通習,則鄭聲莫能亂雅。



 遂為十二均圖,並上之。



 其論以為:「律各有均,有七聲,更相為用。協本均則樂調,非本均則樂悖。今黃鐘為宮,則太簇、姑洗、林鐘、南呂、應鐘、蕤賓七聲相應,謂之黃鐘之均。餘律為宮,同之。宮為君,商為臣,角為民,徵為事,羽為物。君者,法度號令之所出,故宮生徵;法度號令所以授臣而承行之,故徽生商;君臣一
 德,以康庶事,則萬物得所,民遂其生,故商生羽,羽生角。然臣有常職,民有常業,物有常形,而遷則失常,故商、角、羽無變聲。君總萬化,不可執以一方;事通萬務,不可滯於一隅:故宮、征有變聲。凡律呂之調及其宮、樂章,具著於圖。」



 帝取所上圖,考其說,乃下鎮、幾參定。而王樸、阮逸之黃鐘乃當李照之太簇,其編鐘、編磬雖有四清聲,而黃鐘、大呂正聲舛誤;照之編鐘、編磬雖有黃鐘、大呂,而全闕四清聲,非古制也。樸之太簇、夾鐘,則聲失之高,歌
 者莫能追逐,平時設而不用。聖人作樂以紀中和之聲,所以導中和之氣,清不可太高,重不可太下,必使八音協諧、歌者從容而能永其言。鎮等因請擇李照編鐘、編磬十二參於律者,增以王樸無射、應鐘及黃鐘、大呂清聲,以為黃鐘、大呂、太簇、夾鐘之四清聲,俾眾樂隨之,歌工詠之,中和之聲庶可以考。請下樸二律。就太常鐘磬擇其可用者用之,其不可修者別制之。而太常以為大樂法度舊器,乞留樸鐘磬,別制新樂,以驗議者之術。詔
 以樸樂鐘為清聲,毋得銷毀。



 幾等謂:「新樂之成,足以薦郊廟,傳萬世。其明堂、景靈宮降天神之樂六奏:舊用夾鐘之均三奏,謂之夾鐘為宮;夷則之均一奏,謂之黃鐘為角;林鐘之均一奏,謂之太簇為征。姑洗為羽。而《大司樂》『凡樂,圜鐘為宮,黃鐘為角,太簇為征,姑洗為羽。』而『圜鐘者,夾鐘也』。用夾鐘均之七聲,以其宮聲為始終,是謂圜鐘為宮;用黃鐘均之七聲,以其角聲為始終,是謂黃鐘為角;用太簇均之七聲,以其徵聲為始終,是謂太簇
 為徵;用姑洗均之七聲,以其羽聲為始終,是謂姑洗為羽。今用夷則之均一奏,謂之黃鐘為角,林鐘之均二奏,謂之太簇為徵、姑洗為羽,則祀天之樂無夷則、林鐘而用之,有太簇、姑洗而去之矣。唐典,祀天以夾鐘宮、黃鐘角、太簇征、姑洗羽,乃周禮也,宜用夾鐘為宮。其黃鐘為角,則用黃鐘均,以其角聲為始終;太簇為征,則用太簇均,以其徵聲為始終;姑洗為羽,則用姑洗均,以其羽聲為始終。祭地祇,享宗廟,皆視此均法以度曲。」



 幾等又以
 太常磬三等,王樸磬厚,李照磬薄,惟阮逸、胡瑗磬形制精密而聲太高,以磬氏之法摩其旁,輕重與律呂相應。鐘三等,王樸鐘所謂「聲疾而短聞」者也,阮逸、胡瑗鐘所謂「聲舒而遠聞」者也,惟李照鐘有旋蟲之制。鐘磬皆三十有六架,架各十有六,則正律相應,清聲自足。其堂上堂下篪、笛率從新制,而調琴、瑟、阮、築、塤諸器,隨所下律。詔悉從之。乃緝新器用,徙置太常,闢屋以貯藏之。考選樂工,汰其椎鈍癃老,而優募能者補其闕員,立為程度,
 以時習焉。



 初,皇祐中,益州進士房庶論尺律之法,以為嘗得古本《漢書》,言在《律歷志》。範鎮以其說為然,請依法作為尺律,然後別求古樂參考。於是庶奉詔造律管二,尺、量、龠各一,而殿中丞胡瑗以為非。詔鎮與幾等定樂,鎮曰:「定樂當先正律。」帝然之。鎮作律、尺等,欲圖上之。而幾之議律主於人聲,不以尺度求合。其樂大抵即李照之舊而加四清聲,遂奏樂成。第加恩賚,而鎮謝曰:「此劉幾樂也,臣何預焉!」乃復上奏曰:「太常鎛鐘皆有大小、輕
 重之法,非三代莫能為者。禁中又出李照、胡瑗所鑄銅律及尺付太常,按照黃鐘律合王樸太簇律,仲呂律合王樸黃鐘律,比樸樂才下半律,外有損益而內無損益,鐘聲鬱而不發,無足議者。照之律雖是,然與其樂校,三格自相違戾。且以太簇為黃鐘,則是商為宮也。



 方劉幾奏上時,臣初無所預。臣頃造律,內外有損益,其聲和,又與古樂合。今若將臣所造尺律依大小編次太常鎛鐘,可以成一代大典。又太常無雷鼓、靈鼓、路鼓,而以散鼓
 代之。開元中,有以畫圖獻者,一鼓而為八面、六面、四面,明堂用之。國朝郊廟或考或不考,宮架中惟以散鼓,不應經義。又八音無匏、土二音:笙、竽以木鬥攢竹而以匏裹之,是無匏音也;塤器以木為之,是無土音也。八音不具,以為備樂,安可得哉!」不報。



 四年十一月,詳定所言:「『搏拊、琴、瑟以詠』,則堂上之樂,以象朝廷之治;『下管、□鼓』,『合止柷吾文』,『笙、鏞以間』,則堂下之樂,以象萬物之治。後世有司失其傳,歌者在堂,兼設鐘磬;宮架在庭,兼設琴瑟;堂
 下匏竹,寘之於床:並非其序。請親祠宗廟及有司攝事,歌者在堂,不設鐘磬;宮架在庭,不設琴瑟;堂下匏竹,不寘於床。其郊壇上下之樂,亦以此為正,而有司攝事如之。」又言:「以《小胥》宮縣推之,則天子鐘、磬、鎛十二虡為宮縣明矣。故或以為配十二辰,或以為配十二次,則虡無過十二。先王之制廢,學者不能考其數。隋、唐以來,有謂宮縣當二十虡,甚者又以為三十六虡。方唐之盛日,有司攝事,樂並用宮縣。至德後,太常聲音之工散亡,凡郊
 廟有登歌而無宮縣,後世因仍不改。請郊廟有司攝事,改用宮架十二虡。」太常以謂用宮架十二虡,則律呂均聲不足,不能成均。請如禮:宮架四面如辰位,設鎛鐘十二虡,而甲、丙、庚、壬設鐘,乙、丁、辛、癸設磬,位各一虡。四隅植建鼓,以像二十四氣。宗廟、郊丘如之。



 五年正月,開封布衣葉防上書論樂器、律曲不應古法,復下楊傑議。傑論防增編鐘、編磬二十有四為簨制,管簫視鐘磬數,登歌用玉磬,去樂曲之近清聲者,舞不立表,皆非是。其言
 均律差互,與劉幾同。請以晉鼓節金奏。考經、禮,制簨虡教國子、宗子舞,用之郊廟,為何所取?而範鎮亦言:「自唐以來至國朝,三大祀樂譜並依《周禮》,然其說有黃鐘為角、黃鐘之角。黃鐘為角者,夷則為宮;黃鐘之角者,姑洗為角。十二律之於五聲,皆如此率。而世俗之說,乃去『之』字,謂太簇曰黃鐘商,姑洗曰黃鐘角,林鐘曰黃鐘征,南呂曰黃鐘羽。今葉防但通世俗夷部之說,而不見《周禮》正文,所以稱本寺均差互,其說難行。」帝以樂律絕學,防
 草萊中習之尤難,乃補防為樂正。



 六年春正月,御大慶殿,初用新樂。二月,太常言:「郊廟樂虡,若遇雨雪,望祭即設於殿上。」三月,禮部言:「有司攝事,祀昊天舞名。請初獻曰《帝臨嘉至》,亞、終獻曰《神娭錫羨》;太廟初獻曰《孝熙昭德》,亞、終獻曰《禮洽儲祥》。」詔可。九月,禮部言:「《周禮》,凡大祭祀,王出入則奏《王夏》,明入廟門已用樂矣。今既移祼在作樂之前,皇帝詣罍洗奏《乾安》,則入門亦當奏《乾安》,庶合古制。其入景靈宮及南郊壝門,乞如之。」



 七年正月,詔從協
 律郎榮咨道請,於奉宸庫選玉造磬,令太常審定音律。六月,禮部言:「親郊之歲,夏至祀皇地祇於方丘,遣塚宰攝事,禮容樂舞謂宜加於常祀。而其樂虡二十、樂工百五十有二、舞者六十有四,與常歲南北郊上公攝事無異,未足以稱欽崇之意。乞自今準親祠用三十六虡,工人三百有六,舞人百二十有四。」詔可。



 元祐元年,咨道又言:「先帝詔臣制造玉磬,將用於廟堂之上,依舊同編鐘以登歌。今年親祠明堂,請用之,以章明盛典。」從之。三年,
 範鎮樂成,上其所制樂章三、鑄律十二、編鐘十二、鎛鐘一、衡一、尺一、斛一,響石為編磬十二、特磬一,簫、笛、塤、篪、巢笙、和笙各二,並書及圖法。帝與太皇太后御延和殿,詔執政、侍從、臺閣、講讀官皆往觀焉。賜鎮詔曰:「朕惟春秋之後,禮樂先亡;秦、漢以來,《韶》、《武》僅在。散樂工於河、海之上,往而不還;聘先生於齊、魯之間,有莫能致。魏、晉以下,曹、鄶無譏。豈徒鄭、衛之音,已雜華、戎之器。間有作者,猶存典刑。然銖、黍之一差,或宮、商之易位。惟我四朝之老,
 獨知五降之非。審聲知音,以律生尺。覽詩書之來上,閱簨虡之在廷。君臣同觀,父老太息。方詔學士、大夫論其法,工師、有司考其聲。上追先帝移風易俗之心,下慰老臣愛君憂國之志。究觀所作,嘉嘆不忘。」



 鎮為《樂論》,其自敘曰:「臣昔為禮官,從諸儒難問樂之差謬,凡十餘事。厥初未習,不能不小抵牾。後考《周官》、《王制》、司馬遷《書》、班氏《志》,得其法,流通貫穿,悉取舊書,去其抵牾。掇其要,作為八論。」其《論律》、《論黍》、《論尺》、《論量》、《論聲器》,言在《律歷志》。



 《論鐘》
 曰:



 夫鐘之制,《周官·鳧氏》言之甚詳,而訓解者其誤有三:若云:「帶,所以介,其名也介,在於、鼓、鉦、舞、甬、衡之間。」介于、鼓、鉦、舞之間則然,非在甬、衡之上,其誤一也。又云:「舞,上下促,以橫為修,從為廣,舞廣四分。」今亦去徑之二分以為之間,則舞間之方常居銑之四也。舞間方四,則鼓間六亦其方也。鼓六、鉦六、舞四,即言鼓間與舞佾相應,則鼓與舞皆六,所云「鉦六、舞四」,其誤二也。又云:「鼓外二,鉦外一。」彼既以鉦、鼓皆六,無厚薄之差,故從而穿鑿,以遷
 就其說,其誤三也。



 今臣所鑄編鐘十二,皆從其律之長,故鐘口十者,其長十六以為鐘之身。鉦者,正也,居鐘之中,上下皆八,下去二以為之鼓,上去二以為之舞,則鉦居四而鼓與舞皆六。是故於、鼓、鉦、舞、篆、景、欒、隊、甬、衡、旋蟲,鐘之文也,著於外者也;廣、長、空徑、厚、薄、大、小,鐘之數也,起於內者也。若夫金錫之齊與鑄金之狀率按諸《經》,差之毫厘則聲有高下,不可不審。其鎛鐘亦以此法而四倍之。



 今太常鐘無大小、無厚薄、無金齊,一以黃鐘為
 率,而磨以取律之合,故黃鐘最薄而輕。自大呂以降,迭加重厚,是以卑陵尊,以小加大,其可乎?且清聲者不見於《經》,惟《小胥》注云:「鐘磬者,編次之,二八十六枚而在一虡謂之堵。」至唐又有十二清聲,其聲愈高,尤為非是。國朝舊有四清聲,置而弗用,至劉幾用之,與鄭、衛無異。



 《論磬》曰:



 臣所造編磬,皆以《周官·磬氏》為法,若黃鐘股之博四寸五分,股九寸,鼓一尺三寸五分;鼓之博三寸,而其厚一寸,其弦一尺三寸五分。十二磬各以其律之長而
 三分損益之,如此其率也。今之十二磬,長短、厚薄皆不以律,而欲求其聲,不亦遠乎?鐘有齊也,磬,石也,天成之物也。以其律為之長短、厚薄,而其聲和,此出於自然,而聖人者能知之,取以為法,後世其可不考正乎?考正而非是,則不足為法矣。



 特磬則四倍其法而為之。國朝祀天地、宗廟及大朝會,宮架內止設鎛鐘,惟后廟乃用特磬,非也。今已升祔後廟,特磬遂為無用之樂。臣欲乞凡宮架內於鎛鐘後各加特磬,貴乎金石之聲小大相應。



 《
 論八音》曰:



 匏、土、革、木、金、石、絲、竹,是八物者,在天地間,其體性不同而至相戾之物也。聖人制為八器,命之商則商,命之宮則宮,無一物不同者。能使天地之間至相戾之物無不同,此樂所以為和而八音所以為樂也。



 樂下太常,而楊傑上言:「元豐中,詔範鎮、劉幾與臣詳議郊廟大樂,既成而奏,稱其和協。今鎮新定樂法,頗與樂局所議不同。且樂經仁宗命作,神考睿斷,奏之郊廟、朝廷,蓋已久矣,豈可用鎮一說而遽改之?」遂著《元祐樂議》以破
 鎮說。其《議樂章》曰:



 國朝大樂所立曲名,各有成憲,不相淆雜,所以重正名也。故廟室之樂皆以「大」名之,如《大善》、《大仁》、《大英》之類是也。今鎮以《文明》之曲獻祖廟,以《大成》之曲進皇帝,以《萬歲》之曲進太皇太后,其名未正,難以施於宗廟、朝廷。



 《議宮架加磬》曰:



 鎮言:「國朝祀天地、宗廟及大朝會,宮架內止設鎛鐘,惟后廟乃用特磬,非也。今已升後廟,特磬遂為無用之樂,欲乞凡宮架內於鎛鐘後各加特磬,貴乎金石之聲小大相應。」按《唐六典》:天子
 宮架之樂,鎛鐘十二、編鐘十二、編磬十二,凡三十有六虡,宗廟與殿庭同。凡中宮之樂,則以大磬代鐘,餘如宮架之制。今以鎛鐘、特磬並設之,則為四十八架,於古無法。皇帝將出,宮架撞黃鐘之鐘,右五鐘皆應;皇帝興,宮架撞蕤賓之鐘,左五鐘皆應。未聞皇帝出入,以特磬為節。



 《議十六鐘磬》曰:



 鎮謂:「清聲不見於《經》,惟《小胥》注云『鐘磬者,編次之,十六枚而在一虡謂之堵。』至唐又有十二清聲,其聲愈高,尤為非是。國朝舊有四清聲,置而弗用,
 至劉幾用之,與鄭、衛無異。」按編鐘、編磬十六,其來遠矣,豈徒見於《周禮·小胥》之注哉?漢成帝時,犍為郡於水濱得古磬十六枚,帝因是陳禮樂、《雅》《頌》之聲,以風化天下。其事載於《禮樂志》,不為不詳,豈因劉幾然後用哉?且漢承秦,秦未嘗制作禮樂,其稱古磬十六者,乃二帝、三王之遺法也。其王樸樂內編鐘、編磬,以其聲律太高,歌者難逐,故四清聲置而弗用。及神宗朝下三律,則四清聲皆用而諧協矣。《周禮》曰:「鳧氏為鐘,薄厚之所震動,清濁
 之所由出。」則清聲豈不見於《經》哉?今鎮以簫、笛、塤、篪、巢笙、和笙獻於朝廷,簫必十六管,是四清聲在其間矣。自古無十二管之簫,豈《簫韶》九成之樂已有鄭、衛之聲乎?



 禮部、太常亦言「鎮樂法自系一家之學,難以參用」,而樂如舊制。



 四年十二月,始命大樂正葉防撰朝會二舞儀。



 武舞曰《威加四海》之舞:



 第一變:舞人去南表三步,總干而立,聽舉樂,三鼓,前行三步,及表而蹲;再鼓,皆舞,進一步,正立;再鼓,皆持干荷戈,相顧作猛賁速趫之狀;再鼓,皆
 轉身向裏,以干戈相擊刺,足不動;再鼓,皆回身向外,擊刺如前;再鼓,皆正立舉手,蹲;再鼓,皆舞,進一步轉面相向立。干戈各置腰;再鼓,各前進,以左足在前,右足在後,左手執乾當前,右手執戈在腰為進旅;再鼓,各相擊刺;再鼓,各退身復位,整其乾為退旅;再鼓,皆正立,蹲;再鼓,皆舞,進一步正立;再鼓,皆轉面相向,秉乾持戈坐作;再鼓,各相擊刺;再鼓,皆起,收其干戈為克捷之象;再鼓,皆正立,遇節樂則蹲。



 第二變:聽舉樂,依前蹲;再鼓,皆舞,進
 一步正立;再鼓,皆正面,作猛賁趫速之狀;再鼓,皆轉身向裏相擊刺,足不動;再鼓,各轉身向外擊刺如前;再鼓,皆正立,蹲;再鼓,皆舞,進一步,陳其干戈,左右相顧為猛賁趫速之狀;再鼓,皆並入行,以八為四;再鼓,皆兩兩對相擊刺;再鼓,皆回,易行列,左在右,右在左,再鼓,皆舉手,蹲;再鼓,皆舞,進一步正立;再鼓,各分左右;再鼓,各揚其干戈;再鼓,交相擊刺;再鼓,皆總干正立,遇節樂則蹲。



 第三變:聽舉樂則蹲;再鼓,皆舞,進一步轉而相向,再鼓,整
 干戈以象登臺講武;再鼓,皆擊刺於東南;再鼓,皆按盾舉戈,東南向而望,以象漳、泉奉土;再鼓,皆擊刺於正南;再鼓,皆按盾舉戈,南向而望,以象杭、越來朝,再鼓,皆舞,進一步正立;再鼓,皆擊刺於西北;再鼓,皆按盾舉戈,西北向而望,以象克殄並、汾;再鼓,皆擊刺於正西;再鼓,皆按盾舉戈,西向而望,以象肅清銀、夏;再鼓,皆舞,進一步正跪,右膝至地,左足微起;再鼓,皆置干戈於地,各拱其手,象其不用;再鼓,皆左右舞蹈,像以文止武之意;再鼓,
 皆就拜,收其干戈,起而躬立;再鼓,皆舞,退,鼓盡即止,以象兵還振旅。



 文舞曰《化成天下》之舞:



 第一變:舞人立南表之南,聽舉樂則蹲;再鼓,皆舞,進一步正立;再鼓,皆稍前而正揖,合手自下而上;再鼓,皆左顧左揖;再鼓,皆右顧右揖;再鼓,皆開手,蹲;再鼓,皆舞,進一步正立;再鼓,皆少卻身,初辭,合手自上而下;再鼓,皆右顧,以右手在前、左手推後為再辭;再鼓,皆左顧,以左手在前,右手推出為固辭;再鼓,皆合手,蹲;再鼓,皆舞,進一步正立;再鼓,皆
 俯身相顧,初謙,合手當胸;再鼓,皆右側身、左垂手為再謙;再鼓,皆左側身、右垂手為三謙;再鼓,皆躬而授之,遇節樂則蹲。



 第二變:聽舉樂則蹲;再鼓,皆舞,進一步轉面相向;再鼓,皆稍前相揖;再鼓,皆左顧左揖;再鼓,開手,蹲,正立;再鼓,皆舞,進一步,復相向;再鼓,皆卻身為初辭;再鼓,皆舞,辭如上儀;再鼓,皆再辭;再鼓,皆固辭;再鼓,皆合手,蹲,正立;再鼓,皆舞,進一步;再鼓,相向;再鼓,皆顧為初謙;再鼓,皆再謙;再鼓,皆三謙;再鼓,皆躬而授之,正立,遇
 節樂則蹲。



 第三變:聽舉樂則蹲;再鼓,皆舞,進一步兩兩相向;再鼓,皆相趨揖;再鼓,皆左揖如上;再鼓,皆右揖;再鼓,皆開手,蹲,正立;再鼓,皆舞,進一步,復相向;再鼓,皆卻身初辭;再鼓,皆再辭;再鼓,皆固辭;再鼓,皆合手,蹲,正立;再鼓,皆舞,進一步兩兩相向;再鼓,皆相顧初謙;再鼓,皆再謙;再鼓,皆三謙,躬而授之,正立,節樂則蹲。



 凡二舞綴表器及引舞振作,並與大祭祀之舞同。協律郎陳沂按閱,以謂節奏詳備,自是朝會則用之。



 八年,太常博士孫
 諤言:「臣嘗奉社稷之祠,親睹陳設,初疑其闕略而不備,退而考元祐祀儀,乃與所親見者合焉。其登歌之樂,雖有鐘、磬、簨虡、搏拊、柷敔之屬,獨陳太社壇上,而太稷闕焉。夫宮架不備,非所以重社稷也。《周官》制祭祀之法,則有靈鼓以鼓之,有幬帗舞以舞之,有太簇、應鐘、《咸池》以極其歌舞之節,此樂文之備也。唐社稷用二十架,至於開元,亦循三代之遺法,於壇之北,宮架備陳,別異天神,中建靈鼓,歌鐘、歌虡各設二壇,下舞上歌,何其盛也!臣稽
 考典禮,凡祭太社、太稷,宜仿《周官》及《開元禮》文,於壇之北備設宮架,鐘、匏、竹各列二壇,南架之內,更植靈鼓。」於是集侍從、禮官議增稷壇樂,而添用宮架之說不行。



 元符元年十一月,詔登歌、鐘、磬並依元豐詔旨,復先帝樂制也。



 二年正月,詔前信州司法參軍吳良輔按協音律,改造琴瑟,教習登歌,以太常少卿張商英薦其知樂故也。初,良輔在元豐中上《樂書》五卷,其書分為四類,以謂:「天地兆分,氣數爰定。律厥氣數,通之以聲。於是撰《釋
 律》。律為經,聲為緯。律以聲為文,聲以律為質。旋相為宮,七音運生。於是撰《釋聲》。聲生於日,律生於辰,故經之以六律,緯之以五聲。聲律相協,和而無乖。播之八音,八音以生。於是撰《釋音》。四物兼採,八器以成。度數施設,像隱於形。考器論義,道德以明。於是撰《釋器》。」類各有條,凡四十四篇,大抵考之經傳,精以講思,頗益於樂理,文多,故弗著焉。



 崇寧元年,詔宰臣置僚屬,講議大政。以大樂之制訛繆殘闕,太常樂器弊壞,琴瑟制度參差不同,簫笛之
 屬樂工自備,每大合樂,聲韻淆雜,而皆失之太高。箏、築、阮,秦、晉之樂也,乃列於琴、瑟之間;熊羆按,梁、隋之制也,乃設於宮架之外。笙不用匏,舞不像成,曲不協譜。樂工率農夫、市賈,遇祭祀朝會則追呼於阡陌、閭閻之中,教習無成,瞢不知音。議樂之臣以《樂經》散亡,無所據依。秦、漢之後,諸儒自相非議,不足取法。乃博求知音之士,而魏漢津之名達於上焉。



 漢津至是年九十餘矣,本剩員兵士,自雲居西蜀,師事唐仙人李良,授鼎樂之法。皇祐
 中,漢津與房庶以善樂被薦,既至,黍律已成,阮逸始非其說,漢津不得伸其所學。後逸之樂不用,乃退與漢津議指尺,作書二篇,敘述指法。漢津嘗陳於太常,樂工憚改作,皆不主其說。或謂漢津舊嘗執役於範鎮,見其制作,略取之,蔡京神其說而托於李良。



 二年九月,禮部員外郎陳暘上所撰《樂書》二百卷,命禮部尚書何執中看詳,以謂暘欲考定音律,以正中聲,願送講議司,令知音律者參驗行之。暘之論曰:「漢津論樂,用京房二變、四清。
 蓋五聲十二律,樂之正也;二變、四清,樂之蠹也。二變以變宮為君,四清以黃鐘清為君。事以時作,固可變也,而君不可變;太簇、大呂、夾鐘,或可分也,而黃鐘不可分。豈古人所謂尊無二上之旨哉?」壬辰,詔曰:「朕惟隆禮作樂,實治內修外之先務,損益述作,其敢後乎?其令講議司官詳求歷代禮樂沿革,酌古今之宜,修為典訓,以貽永世,致安上治民之至德,著移風易俗之美化,乃稱朕咨諏之意焉。」



 三年正月,漢津言曰:「臣聞黃帝以三寸之器
 名為《咸池》,其樂曰《大卷》,三三而九,乃為黃鐘之律。禹效黃帝之法,以聲為律,以身為度,用左手中指三節三寸,謂之君指,裁為宮聲之管;又用第四指三節三寸,謂之臣指,裁為商聲之管;又用第五指三節三寸,謂之物指,裁為羽聲之管。第二指為民、為角,大指為事、為征,民與事,君臣治之,以物養之,故不用為裁管之法。得三指合之為九寸,即黃鐘之律定矣。黃鐘定,餘律從而生焉。臣今欲請帝中指、第四指、第五指各三節,先鑄九鼎,次鑄
 帝坐大鐘,次鑄四韻清聲鐘,次鑄二十四氣鐘,然後均弦裁管,為一代之樂制。」



 其後十三年,帝一日忽夢人言:「樂成而鳳凰不至乎!蓋非帝指也。」帝寤,大悔嘆,謂:「崇寧初作樂,請吾指寸,而內侍黃經臣執謂『帝指不可示外人』,但引吾手略比度之,曰:『此是也。』蓋非人所知。今神告朕如此,且奈何?」於是再出中指寸付蔡京,密命劉昺試之。時昺終匿漢津初說,但以其前議為度,作一長笛上之。帝指寸既長於舊,而長笛殆不可易,以動人觀聽,於
 是遂止。蓋京之子絳雲。



 秋七月,景鐘成。景鐘者,黃鐘之所自出也。垂則為鐘,仰則為鼎。鼎之大,終於九斛,中聲所極。制煉玉屑,入於銅齊,精純之至,音韻清越。其高九尺,拱以九龍,惟天子親郊乃用之。立於宮架之中,以為君圍。於是命翰林學士承旨張康國為之銘。其文曰:「天造我宋,於穆不已。四方來和,十有二紀。樂象厥成,維其時矣。迪惟有夏,度自禹起。我龍受之,天地一指。於論景鐘,中聲所止。有作於斯,無襲於彼。九九以生,律呂根柢。
 維此景鐘,非弇非侈。在宋之庭,屹然中峙。天子萬年,既多受祉。維此景鐘,上帝命爾。其承伊何,以燕翼子。永言寶之,宋樂之始。」



\end{pinyinscope}