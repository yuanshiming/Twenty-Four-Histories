\article{志第四十六 河渠三}

\begin{pinyinscope}

 黃河下



 汴河上



 元祐八年二月乙卯,三省奉旨:「北流軟堰,並依都水監所奏。」門下侍郎蘇轍奏:「臣嘗以謂軟堰不可施於北流,利害甚明。蓋東流本人力所開,闊止百餘步,冬月河流
 斷絕,故軟堰可為。今北流是大河正溜,此之東流,何止數倍,見今河水行流不絕,軟堰何由能立?蓋水官之意,欲以軟堰為名,實作硬堰,陰為回河之計耳。朝廷既已覺其意,則軟堰之請,不宜復從。」趙偁亦上議曰:「臣竊謂河事大利害有三,而言者互進其說,或見近忘遠,徼幸盜功,或取此舍彼,譸張昧理。遂使大利不明,大害不去,上惑朝聽,下滋民患,橫役枉費,殆無窮已,臣切痛之。所謂大利害者:北流全河,患水不能分也;東流分水,患水
 不能行也;宗城河決,患水不能閉也。是三者,去其患則為利,未能去則為害。今不謀此,而議欲專閉北流,止知一日可閉之利,而不知異日既塞之患,止知北流伏槽之水易為力,而不知闞村方漲之勢,未可並以入東流也。夫欲合河以為利,而不恤上下壅潰之害,是皆見近忘遠,徼幸盜功之事也。有司欲斷北流而不執其咎,乃引分水為說,姑為軟堰;知河沖之不可以軟堰御,則又為決堰之計。臣恐枉有工費,而以河為戲也。請俟漲水
 伏槽,觀大河之勢,以治東流、北流。」



 五月,水官卒請進梁村上、下約,束狹河門,既涉漲水,遂壅而潰。南犯德清,西決內黃,東淤梁村,北出闞村,宗城決口復行魏店,北流因淤遂斷,河水四出,壞東郡浮梁。十二月丙寅,監察御史郭知章言:「臣比緣使事至河北,自澶州入北京,渡孫村口,見水趨東者,河甚闊而深;又自北京往洺州,過楊家淺口復渡,見水之趨北者,才十之二三,然後知大河宜閉北行東。乞下都水監相度。」於是吳安持復兼領都
 水,即建言:「近準朝旨,已堰斷魏店刺子,向下北流一枝斷絕。然東西未有堤岸,若漲水稍大,必披灘漫出,則平流在北京、恩州界,為害愈甚。乞塞梁村口,縷張包口,開青豐口以東雞爪河,分殺水勢。」呂大防以其與己意合,向之,詔同北京留守相視。



 時范純仁復為右相,與蘇轍力以為不可。遂降旨:「令都水監與本路安撫、轉運、提刑司共議,可則行之,有異議速以聞。」紹聖元年正月也。是時,轉運使趙偁深不以為然,提刑上官均頗助之。
 偁之言曰:「河自孟津初行平地,必須全流,乃成河道。禹之治水,自冀北抵滄、棣,始播為九河,以其近海無患也。今河自橫□、六塔、商胡、小吳,百年之間,皆從西決,蓋河徙之常勢。而有司置埽創約,橫截河流,回河不成,因為分水。初決南宮,再決宗城,三決內黃,亦皆西決,則地勢西下,較然可見。今欲弭息河患,而逆地勢,戾水性,臣未見其能就功也。請開闞村河門,修平鄉鉅鹿埽、焦家等堤,浚澶淵故道,以備漲水。」大名安撫使許將言:「度今之利,若
 舍故道,止從北流,則慮河下已湮,而上流橫潰,為害益廣。若直閉北流,東徙故道,則復慮受水不盡,而破堤為患。竊謂宜因梁村之口以行東,因內黃之口以行北,而盡閉諸口,以絕大名諸州之患。俟春夏水大至,乃觀故道,足以受之,則內黃之口可塞;不足以受之,則梁村之役可止。定其成議,則民心固而河之順復有時,可以保其無害。」詔:「令吳安持同都水監丞鄭祐,與本路安撫、轉運、提刑司官具圖、狀保明聞奏,即有未便,亦具利害來
 上。」



 三月癸酉,監察御史郭知章言:「河復故道,水之趨東,已不可遏。近日遣使按視,逐司議論未一。臣謂水官朝夕從事河上,望專委之。」乙亥,呂大防罷相。



 六月,右正言張商英奏言:「元豐間河決南宮口,講議累年,先帝嘆曰:『神禹復生,不能回此河矣。』乃敕自今後不得復議回河閉口,蓋採用漢人之論,俟其泛濫自定也。元祐初,文彥博、呂大防以前敕非是,拔吳安持為都水使者,委以東流之事。京東、河北五百里內差夫,五百里外出錢雇夫,
 及支借常平倉司錢買梢草,斬伐榆柳。凡八年而無尺寸之效,乃遷安持太僕卿,王宗望代之。宗望至,則劉奉世猶以彥博、大防餘意,力主東流,以梁村口吞納大河。今則梁村口淤澱,而開沙堤兩處決口以洩水矣。前議累七十里堤以障北流,今則云俟霜降水落興工矣。朝廷咫尺,不應九年為水官蔽欺如此。九年之內,年年礬山水漲,霜降水落,豈獨今年始有漲水,而待水落乃可以興工耶?乞遣使按驗虛實,取索回河以來公私費錢
 糧、梢草,依仁宗朝六塔河施行。」



 會七月辛丑,廣武埽危急,詔王宗望亟往救護。壬寅,帝謂輔臣曰:「廣武去洛河不遠,須防漲溢下灌京師,已遣中使視之。」輔臣出圖、狀以奏曰:「此由黃河北岸生灘,水趨南岸。今雨止,河必減落,已下水官,與洛口官同行按視,為簽堤及去北岸嫩灘,令河順直,則無患矣。」



 八月丙子,權工部侍郎吳安持等言:「廣武埽危急,刷塌堤身二千餘步處,地形稍高。自鞏縣東七里店至見今洛口,約不滿十里,可以別開新
 河,引導河水近南行流,地步至少,用功甚微。王宗望行視並開井筒,各稱利便外,其南築大堤,工力浩大,乞下各屬官司,躬往相度保明。」從之。



 十月丁酉,王宗望言:「大河自元豐潰決以來,東、北兩流,利害極大,頻年紛爭,國論不決,水官無所適從。伏自奉昭凡九月,上稟成算,自闞村下至栲栳堤七節河門,並皆閉塞。築金堤七十里,盡障北流,使全河東還故道,以除河患。又自闞村下至海口,補築新舊堤防,增修疏浚河道之淤淺者,雖盛夏
 漲潦,不至壅決。望付史官,紀紹聖以來聖明獨斷,致此成績。」詔宗望等具析修閉北流部役官等功力等弟以聞。然是時東流堤防未及繕固,瀕河多被水患,流民入京師,往往泊御廊及僧舍。詔給券,諭令還本土,以就振濟。



 己酉,安持又言:「準朝旨相度開浚澶州故道,分減漲水。按澶州本是河行舊道,頃年曾乞開修,時以東西地形高仰,未可興工。欲乞且行疏導燕家河,仍令所屬先次計度合增修一十一埽所用工料。」詔:「令都水監候來
 年將及漲水月分,先具利害以聞。」



 癸丑,三省、樞密院言:「元豐八年,知澶州王令圖議,乞修復大河故道。元祐四年,都水使者吳安持,因紓南宮等埽危急,遂就孫村口為回河之策。及梁村進約東流,孫村口窄狹,德清軍等處皆被水患。今春,王宗望等雖於內黃下埽閉斷北流,然至漲水之時,猶有三分水勢,而上流諸埽已多危急,下至將陵埽決壞民田。近又據宗望等奏,大河自閉塞闞村而下,及創築新堤七十餘里,盡閉北流,全河之水,
 東還故道。今訪聞東流向下,地形已高,水行不快。既閉斷北流,將來盛夏,大河漲水全歸故道,不惟舊堤損缺怯薄,而闞村新堤,亦恐未易枝梧。兼京城上流諸處埽岸,慮有壅滯沖決之患,不可不豫為經畫。」詔:權工部侍郎吳安持、都水使者王宗望、監丞鄭祐同北外監丞司,自闞村而下直至海口,逐一相視,增修疏浚,不致壅滯沖決。



 丙辰,張商英又言:「今年已閉北流,都水監長貳交章稱賀,或乞付史官,則是河水已歸故道,止宜修緝堤
 埽,防將來沖決而已。近聞王宗望、李仲卻欲開澶州故道以分水,吳安持乞候漲水前相度。緣開澶州故道,若不與今東流底平,則才經水落,立見淤塞。若與底平,則從初自合閉口回河,何用九年費財動眾?安持稱候漲水相度,乃是悠悠之談。前來漲水並今來漲水,各至澶州、德清軍界,安持首尾九年,豈得不見?更欲延至明年,乃是狡兔三窟,自為潛身之計,非公心為國事也。況立春漸近調夫,如是時不早定議,又留後說,邦財民力,何
 以支持?訪聞先朝水官孫民先、元祐六年水官賈種民各有《河議》,乞取索照會。召前後本路監司及經歷河事之人,與水官詣都堂反復詰難,務取至當,經久可行,定議歸一,庶免以有限之財事無涯之功。」二年七月戊午,詔:「沿黃河州軍,河防決溢,並即申奏。」



 元符二年二月乙亥,北外都水丞李偉言:「相度大小河門,乘此水勢衰弱,並先修閉,各立蛾眉埽鎮壓。乞次於河北、京東兩路差正夫三萬人,其它夫數,令修河官和雇。」三月丁巳,偉又
 乞於澶州之南大河身內,開小河一道,以待漲水,紓解大吳口下注北京一帶向著之患。」並從之。



 六月末,河決內黃口,東流遂斷絕。八月甲戌,詔:「大河水勢十分北流,其以河事付轉運司,責州縣共力救護堤岸。」辛丑,左司諫王祖道請正吳安持、鄭祐、李仲、李偉之罪,投之遠方,以明先帝北流之志。詔可。



 三年正月己卯,徽宗即位。鄭祐、吳安持輩皆用登極大赦,次第牽復。中書舍人張商英繳奏:「祐等昨主回河,皆違神宗北流之意。」不聽。商英
 又嘗論水官非其人,治河當行其所無事,一用堤障,猶塞兒口止其啼也。三月,乃以商英為龍圖閣待制、河北都轉運使兼專功提舉河事。商英復陳五事:一曰行古沙河口;二曰復平恩四埽;三曰引大河自古漳河、浮河入海;四曰築御河西堤,而開東堤之積;五曰開木門口,洩徒駭河東流。大要欲隨地勢疏浚入海。會四月,河決蘇村。七月,詔:「商英毋治河,止厘本職,其因河事差闢官吏並罷。」復置北外都水丞司。



 建中靖國元年春,尚書省
 言:「自去夏蘇村漲水,後來全河漫流,今已淤高三四尺,宜立西堤。」詔都水使者魯君貺同北外丞司經度之。於是左正言任伯雨奏:



 河為中國患,二千歲矣。自古竭天下之力以事河者,莫如本朝。而徇眾人偏見,欲屈大河之勢以從人者,莫甚於近世。臣不敢遠引,只如元祐末年,小吳決溢,議者乃譎謀異計,欲立奇功,以邀厚賞。不顧地勢,不念民力,不惜國用,力建東流之議。當洪流中,立馬頭,設鋸齒,梢芻材木,耗費百倍。力遏水勢,使之東
 注,陵虛駕空,非特行地上而已。增堤益防,惴惴恐決,澄沙淤泥,久益高仰,一旦決潰,又復北流。此非堤防之不固,亦理勢之必至也。



 昔禹之治水,不獨行其所無事,亦未嘗不因其變以導之。蓋河流混濁,泥沙相半,流行既久,迄邐淤澱,則久而必決者,勢不能變也。或北而東,或東而北,亦安可以人力制哉!



 為今之策,正宜因其所向,寬立堤防,約欄水勢,使不至大段漫流。若恐北流淤澱塘泊,亦祗宜因塘泊之岸,增設堤防,乃為長策。風聞近
 日又有議者獻東流之計,不獨比年災傷,居民流散,公私匱竭,百無一有,事勢窘急,固不可為;抑亦自高注下,湍流奔猛,潰決未久,勢不可改。設若興工,公私徒耗,殆非利民之舉,實自困之道也。



 崇寧三年十月,臣僚言:「昨奉詔措置大河,即由西路歷沿邊州軍,回至武強縣,循河堤至深州,又北下衡水縣,乃達於冀。又北渡河過遠來鎮,及分遣屬僚相視恩州之北河流次第。大抵水性無有不下,引之就高,決不可得。況西山積水,勢必欲下,
 各因其勢而順導之,則無壅遏之患。」詔開修直河,以殺水勢。



 四年二月,工部言:「乞修蘇村等處運糧河堤為正堤,以支漲水,較修棄堤直堤,可減工四十四萬、料七十一萬有奇。」從之。閏二月,尚書省言:「大河北流,合西山諸水,在深州武強、瀛州樂壽埽,俯瞰雄、霸、莫州及沿邊塘濼,萬一決溢,為害甚大。」詔增二埽堤及儲蓄,以備漲水。是歲,大河安流。



 五年二月,詔滑州系浮橋於北岸,仍築城壘,置官兵守護之。八月,葺陽武副堤。



 大觀元年二月,
 詔於陽武上埽第五鋪開修直河至第十五鋪,以分減水勢。有司言:「河身當長三千四百四十步,面闊八十尺,底闊五丈,深七尺,計役十萬七千餘工,用人夫三千五百八十二,凡一月畢。」從之。十二月,工部員外郎趙霆言:「南北兩丞司合開直河者,凡為里八十有七,用緡錢八九萬。」異時成功,可免河防之憂,而省久遠之費。」詔從之。



 二年五月,霆上免夫之議,大略謂:「黃河調發人夫修築埽岸,每歲春首,騷動數路,常至敗家破產。今春滑州魚
 池埽合起夫役,嘗令送免夫之直,用以買土,增貼埽岸,比之調夫,反有贏餘。乞詔有司,應堤埽合調春夫,並依此例,立為永法。」詔曰:「河防夫工,歲役十萬,濱河之民,困於調發。可上戶出錢免夫,下戶出力充役,其相度條畫以聞。」丙申,邢州言河決,陷鉅鹿縣。詔遷縣於高地。又以趙州隆平下濕,亦遷之。



 六月己卯,都水使者吳玠言:「自元豐間小吳口決,北流入御河,下合西山諸水,至清州獨流砦三叉口入海。雖深得保固形勝之策,而歲月浸
 久,侵犯塘堤,沖壞道路,嚙損城砦。臣奉詔修治堤防,御捍漲溢。然築八尺之堤,當九河之尾,恐不能敵。若不遇有損缺,逐旋增修,即又至隳壞,使與塘水相通,於邊防非計也。乞降旨修葺。」從之。庚寅,冀州河溢,壞信都、南宮兩縣。



 三年八月,詔沉純誠開撩兔源河。兔源在廣武埽對岸,分減埽下漲水也。



 政和四年十一月,都水使者孟昌齡言:「今歲夏秋漲水,河流上下並行中道,滑州浮橋不勞解拆,大省歲費。」詔許稱賀,官吏推恩有差。昌齡又
 獻議導河大伾,可置永遠浮橋,謂:「河流自大伾之東而來,直大伾山西,而止,數里方回南,東轉而過,復折北而東,則又直至大伾山之東,亦止不過十里耳。視地形水勢,東西相直徑易,曾不十餘里間,且地勢低下,可以成河,倚山可為馬頭,又有中水單,正如河陽。若引使穿大伾大山及東北二小山,分為兩股而過,合於下流,因是三山為趾,以系浮梁,省費數十百倍,可寬河朔諸路之役。」朝廷喜而從之。



 五年,置提舉修系永橋所。六月癸丑,降
 德音於河北、京東、京西路,其略曰:「鑿山釃渠,循九河既道之跡;為梁跨趾,成萬世永賴之功。役不逾時,慮無愆素。人絕往來之阻,地無南北之殊。靈祗懷柔,黎庶呼舞。眷言朔野,爰暨近畿,畚鍤繁興,薪芻轉徙,民亦勞止,朕甚憫之。宜推在宥之恩,仍廣蠲除之惠。應開河官吏,令提舉所具功力等第聞奏。」又詔:「居山至大伾山浮橋屬浚州者,賜名天成橋;大伾山至汶子山浮橋屬滑州者,賜名榮光橋。」俄改榮光曰聖功。七月庚辰,禦制橋名,磨
 崖以刻之。方河之開也,水流雖通,然湍激猛暴,遇山稍隘,往往泛溢,近砦民夫多被漂溺,因亦及通利軍,其後遂注成巨濼云。是月,昌齡遷工部侍郎。



 八月己亥,都水監言:「大河以就三山通流,正在通利之東,慮水溢為患。乞移軍城於大伾山、居山之間,以就高仰。」從之。十月丁巳,中書省言冀州棗強埽決,知州辛昌宗武臣,不諳河事,詔以王仲元代之。



 十一月丙寅,都水使者孟揆言:「大河連經漲淤,灘面已高,致河流傾側東岸。今若修閉棗
 強上埽決口,其費不貲,兼冬深難施人力,縱使極力修閉,東堤上下二百餘里,必須盡行增築,與水爭力,未能全免決溢之患。今漫水行流,多堿鹵及積水之地,又不犯州軍,止經數縣地分,迤邐纏御河歸納黃河。欲自決口上恩州之地水堤為始,增補舊堤,接續御河東岸,簽合大河。」從之。乙亥,臣僚言:「禹跡湮沒於數千載之遠,陛下神智獨運,一旦興復,導河三山。長堤盤固,橫截巨浸,依山為梁,天造地設。威示南北,度越前古,歲無解系之
 費,人無病涉之患。大功既成,願申飭有司,以日繼月,視水向著,隨為堤防,益加增固,每遇漲水,水官、漕臣不輟巡視。」詔付昌齡。



 六年四月辛卯,高陽關路安撫使吳玠言冀州棗強縣黃河清,詔許稱賀。七月戊午,太師蔡京請名三山橋銘閣曰纘禹繼文之閣,門曰銘功之門。十月辛卯,蔡京等言:「冀州河清,乞拜表稱賀。」



 七年五月丁巳,臣僚言:「恩州寧化鎮大河之側,地勢低下,正當灣流沖激之處。歲久堤岸怯薄,沁水透堤甚多,近鎮居民例
 皆移避。方秋夏之交,時雨霈然,一失堤防,則不惟東流莫測所向,一隅生靈所系甚大,亦恐妨阻大名、河間諸州往來邊路。乞付有司,貼築固護。」從之。六月癸酉,都水使者孟揚言:「舊河陽南北兩河分流,立中水單,系浮梁。頃緣北河淤澱,水不通行,止於南河修系一橋。因此河項窄狹,水勢沖激,每遇漲水,多致損壞。欲措置開修北河,如舊修系南北兩橋。」從之。九月丁未,詔揚專一措置,而令河陽守臣王序營辦錢糧,督其工料。



 重和元年三月
 己亥,詔:「滑州、浚州界萬年堤,全藉林木固護堤岸,其廣行種植,以壯地勢。」五月甲辰,詔:「孟州河陽縣第一埽,自春以來,河勢湍猛,侵嚙民田,迫近州城止二三里。其令都水使者同漕臣、河陽守臣措置固護。」是秋雨,廣武埽危急,詔內侍王仍相度措置。



 宣和元年九月辛未,蔡京等言:「南丞管下三十五埽,今歲漲水之後,岸下一例生灘,河行中道,實由聖德昭格,神祇順助。望宣付史館。」詔送秘書省。十二月,開修兔源河並直河畢工,降詔獎諭。



 二年九月己卯,王黼言:「昨孟昌齡計議河事,至滑州韓村埽檢視,河流沖至寸金潭,其勢就下,未易御遏。近降詔旨,令就畫定港灣,對開直河。方議開鑿,忽自成直河一道,寸金潭下,水即安流,在役之人,聚首仰嘆。乞付史館,仍帥百官表賀。」從之。



 三年六月,河溢冀州信都。十一月,河決清河埽。是歲,水壞天成、聖功橋,官吏行罰有差。四年四月壬子,都水使者孟揚言:「奉詔修系三山東橋,凡役工十五萬七千八百,今累經漲水無虞。」詔因橋壞
 失職降秩者,俱復之,揚自正議大夫轉正奉大夫。



 七年,欽宗即位。靖康元年二月乙卯,御史中丞許翰言:「保和殿大學士孟昌齡、延康殿學士孟揚、龍圖閣直學士孟揆,父子相繼領職二十年,過惡山積。妄設堤防之功,多張梢樁之數,窮竭民力,聚斂金帛。交結權要,內侍王仍為之奧主,超付名位,不知紀極。大河浮橋,歲一造舟,京西之民,猶憚其役。而昌齡首建三山之策,回大河之勢,頓取百年浮橋之費,僅為數歲行路之觀。漂沒生靈,無
 慮萬計,近輔郡縣,蕭然破殘。所闢官吏,計金敘績,富商大賈,爭注名牒,身不在公,遙分爵賞。每興一役,乾沒無數,省部御史,莫能鉤考。陛下方將澄清朝著,建立事功,不先誅竄昌齡父子,無以昭示天下。望籍其奸贓,以正典刑。」詔並落職:昌齡在外宮觀,揚依舊權領都水監職事,揆候措置橋船畢取旨。翰復請鉤考簿書,發其奸贓。乃詔昌齡與中大夫,揚、揆與中奉大夫。三月丁丑,京西轉運司言:「本路歲科河防夫三萬,溝河夫一萬八千。緣
 連年不稔,群盜劫掠,民力困弊,乞量數減放。」詔減八千人。



 汴河,自隋大業初,疏通濟渠,引黃河通淮,至唐,改名廣濟。宋都大梁,以孟州河陰縣南為汴首受黃河之口,屬於淮、泗。每歲自春及冬,常於河口均調水勢,止深六尺,以通行重載為準。歲漕江、淮、湖、浙米數百萬,及至東南之產,百物眾寶,不可勝計。又下西山之薪炭,以輸京師之粟,以振河北之急,內外仰給焉。故於諸水,莫此為重。
 其淺深有度,置官以司之,都水監總察之。然大河向背不常,故河口歲易;易則度地形,相水勢,為口以逆之。遇春首輒調數州之民,勞費不貲,役者多溺死。吏又並緣侵漁,而京師常有決溢之虞。



 太祖建隆二年春,導索水自旃然,與須水合入於汴。三年十月,詔:「緣汴河州縣長吏,常以春首課民夾岸植榆柳,以固堤防。」



 太宗太平興國二年七月,開封府言:「汴水溢壞開封大寧堤,浸民田,害稼。」詔發懷、孟丁夫三千五百人塞之。三年正月,發軍
 士千人復汴口。六月,宋州言:「寧陵縣河溢,堤決。」詔發宋、亳丁夫四千五百人,分遣使臣護役。四年八月,又決於宋城縣,以本州諸縣人夫三千五百人塞之。



 淳化二年六月,汴水決浚儀縣。帝乘步輦出乾元門,宰相、樞密迎謁。帝曰:「東京養甲兵數十萬,居人百萬家,天下轉漕,仰給在此一渠水,朕安得不顧。」車駕入泥淖中,行百餘步,從臣震恐。殿前都指揮使戴興叩頭懇請回馭,遂捧輦出泥淖中。詔興督步卒數千塞之。日未旰,水勢遂定。帝
 始就次,太官進膳。親王近臣皆泥濘沾衣。知縣宋炎亡匿不敢出,特赦其罪。是月,汴又決於宋城縣,發近縣丁夫二千人塞之。



 至道元年九月,帝以汴河歲運江、淮米五七百萬斛,以濟京師,問侍臣汴水疏鑿之由,令參知政事張洎講求其事以聞。其言曰:



 禹導河自積石至龍門,南至華陰,東至砥柱;又東至於孟津,東過洛汭,至於大伾,即今成皋是也,或云黎陽山也。禹以大河流泛中國,為害最甚,乃於貝丘疏二渠,以分水勢:一渠自舞陽
 縣東,引入漯水,其水東北流,至千乘縣入海,即今黃河是也;一渠疏畎引傍西山,以東北形高敝壞堤,水勢不便流溢,夾右碣石入於渤海。《書》所謂「北過洚水,至於大陸」,洚水即濁漳,大陸則邢州鉅鹿澤。「播為九河,同為逆河,入於海。」河自魏郡貴鄉縣界分為九道,下至滄州,今為一河。言逆河者,謂與河水往復相承受也。齊桓公塞以廣田居,唯一河存焉,今其東界至莽梧河是也。禹又於滎澤下分大河為陰溝,引注東南,以通淮、泗。至大梁
 浚儀縣西北,復分為二渠:一渠元經陽武縣中牟臺下為官渡水;一渠始皇疏鑿以灌魏郡,謂之鴻溝,莨菪渠自滎陽五出池口來注之。其鴻溝即出河之溝,亦曰莨菪渠。



 漢明帝時,樂浪人王景、謁者王吳始作浚儀渠,蓋循河溝故讀也。渠成流注浚儀,故以浚儀縣為名。靈帝建寧四年,於敖城西北壘石為門,以遏渠口,故世謂之石門。渠外東合濟水,濟與河、渠渾濤東注,至敖山北,渠水至此又兼邲之水,即《春秋》晉、楚戰於邲。邲又音汳,即「
 汴」字,古人避「反」字,改從「汴」字。渠水又東經滎陽北,旃然水自縣東流入汴水。鄭州滎陽縣西二十里三皇山上,有二廣武城,二城相去百餘步,汴水自兩城間小澗中東流而出,而濟流自茲乃絕。唯汴渠首受旃然水,謂之鴻渠。東晉太和中,桓溫北伐前燕,將通之,不果。義熙十三年,劉裕西征姚秦,復浚此渠,始有湍流奔注,而岸善潰塞,裕更疏鑿而漕運焉。隋煬帝大業三年,詔尚書左丞相皇甫誼發河南男女百萬開汴水,起滎澤入淮千
 餘里,乃為通濟渠。又發淮南兵夫十餘萬開邗溝,自山陽淮至於揚子江三百餘里,水面闊四十步,而後行幸焉。自後天下利於轉輸。昔孝文時,賈誼言「漢以江、淮為奉地」,謂魚、鹽、穀、帛,多出東南。至五鳳中,耿壽昌奏:「故事,歲增關東谷四百萬斛以給京師。」亦多自此渠漕運。



 唐初,改通濟渠為廣濟渠。開元中,黃門侍郎、平章事裴耀卿言:江、淮租船,自長淮西北溯鴻溝,轉相輸納於河陰、含嘉、太原等倉。凡三年,運米七百萬石,實利涉於此。開
 元末,河南採訪使、汴州刺史齊浣,以江、淮漕運經淮水波濤有沉損,遂浚廣濟渠下流,自泗州虹縣至楚州淮陰縣北八十里合於淮,逾時畢功。既而水流迅急,行旅艱險,尋乃廢停,卻由舊河。



 德宗朝,歲漕運江、淮米四十萬石,以益關中。時叛將李正己、田悅皆分軍守徐州,臨渦口,梁崇義阻兵襄、鄧,南北漕引皆絕。於是水陸運使杜祐請改漕路,自浚儀西十里疏其南涯,引流入琵琶溝,經蔡河至陳州合穎水,是秦、漢故道,以官漕久不由
 此,故填淤不通,若畎流培岸,則功用甚寡;又廬、壽之間有水道,而平岡亙其中,曰雞鳴山,祐請疏其兩端,皆可通舟,其間登陸四十里而已,則江、湖、黔、嶺、蜀、漢之粟,可方舟而下。由是白沙趨東關,經廬、壽,浮穎步蔡,歷琵琶溝入汴河,不復經溯淮之險,徑於舊路二千里,功寡利博。朝議將行,而徐州順命,淮路乃通。至國家膺圖受命,以大梁四方所湊,天下之樞,可以臨制四海,故卜京邑而定都。



 漢高帝云:「吾以羽檄召天下兵未至。」孝文又云:「
 吾初即位,不欲出虎符召郡國兵。」即知兵甲在外也。唯有南北軍、期門郎、羽林孤兒,以備天子扈從藩衛之用。唐承隋制,置十二衛府兵,皆農夫也。及罷府兵,始置神武、神策為禁軍,不過三數萬人,亦以備扈從藩衛而已,故祿山犯關,驅市人而戰;德宗蒙塵,扈駕四百餘騎,兵甲皆在郡國。額軍存而可舉者,除河朔三鎮外,太原、青社各十萬人,邠寧、宣武各六萬人,潞、徐、荊、揚各五萬人,襄、宣、壽、鎮海各二萬人,自餘觀察、團練據要害之地者,
 不下萬人。今天下甲卒數十萬眾,戰馬數十萬匹,並萃京師,悉集七亡國之士民於輦下,比漢、唐京邑,民庶十倍。甸服時有水旱,不至艱歉者,有惠民、金水、五丈、汴水等四渠,派引脈分,咸會天邑,舳艫相接,贍給公私。所以無匱乏,唯汴水橫亙中國,首承大河,漕引江、湖,利盡南海,半天下之財賦,並山澤之百貨,悉由此路而進。然則禹力疏鑿以分水勢,煬帝開畎以奉巡游,雖數湮廢,而通流不絕於百代之下,終為國家之用者,其上天之意
 乎?



 真宗景德元年九月,宋州言汴河決,浸民田,壞廬舍。遣使護塞,逾月功就。三年六月,京城汴水暴漲,詔覘候水勢,並工修補,增起堤岸。工畢,復遣使致祭。



 大中祥符二年八月,汴水漲溢,自京至鄭州,浸道路。詔選使乘傳減汴口水勢。既而水減,阻滯漕運,復遣浚汴口。八年六月,詔自今後汴水添漲及七尺五寸,即遣禁兵三千,沿河防護。八月,太常少卿馬元方請浚汴河中流,闊五丈,深五尺,可省修堤之費。即詔遣使計度修浚。使還,上言:「
 泗州西至開封府界,岸闊底平,水勢薄,不假開浚。請止自泗州夾岡,用功八十六萬五千四百三十八,以宿、亳丁夫充,計減功七百三十一萬,仍請於沿河作頭踏道擗岸,其淺處為鋸牙,以束水勢,使其浚成河道,止用河清、下卸卒,就未放春水前,令逐州長吏、令佐督役。自今汴河淤澱,可三五年一浚。又於中牟、滎澤縣各置開減水河。」並從之。



 天禧三年十二月,都官員外郎鄭希甫言:「汴河兩岸皆是陂水,廣浸民田,堤腳並無流洩之處。今
 汴河南省自明河接澳入淮,望詔轉運使規度以聞。」



 仁宗天聖三年,汴流淺,特遣使疏河注口。四年,大漲,堤危,眾情恟心匈憂京城,詔度京城西賈陂岡地,洩之於護龍河。六年,勾當汴口康德輿言:「行視陽武橋萬勝鎮,宜存斗門。其梁固斗門三宜廢去,祥符界北岸請為別竇,分減溢流。」而勾當汴口王中庸欲增置孫村之石限,悉從其請。七年,德輿言,修河芟地為並灘農戶所侵。詔限一月使自實,檢括以還縣官。皇祐三年,命使詣中牟治堤。
 明年八月,河涸,舟不通,令河渠司自口浚治,歲以為常。舊制,水增七尺五寸,則京師集禁兵、八作、排岸兵,負土列河上以防河。滿五日,賜錢以勞之,曰「特支」;而或數漲數防,又不及五日而罷,則軍士屢疲,而賜予不及。是歲七月,始制防河兵日給錢,薄其數,才比特支十分之一,軍士便之。明年,遣使行河相利害。



 嘉祐六年,汴水淺澀,常稽運漕。都水奏:「河自應天府抵泗州,直流湍駛無所阻。惟應天府上至汴口,或岸闊淺漫,宜限以六十步闊,
 於此則為木岸狹河,扼束水勢令深駛。梢,伐岸木可足也。」遂下詔興役,而眾議以為未便。宰相蔡京奏:「祖宗時已嘗狹河矣,俗好沮敗事,宜勿聽。」役既半,岸木不足,募民出雜梢。岸成而言者始息。舊曲灘漫流,多稽留覆溺處,悉為駛直平夷,操舟往來便之。



 神宗熙寧四年,創開訾家口,日役夫四萬,饒一月而成。才三月已淺澱,乃復開舊口,役萬工,四日而水稍順。有應舜臣者,獨謂新口在孤柏嶺下,當河流之沖,其便利可常用勿易,水大則
 洩以斗門,水小則為輔渠於下流以益之。安石善其議。



 五年,先是,宣徽北院使、中太一宮使張方平嘗論汴河曰:「國家漕運,以河渠為主。國初浚河渠三道,通京城漕運,自後定立上供年額:汴河觔斗六百萬石,廣濟河六十二萬石,惠民河六十萬石。廣濟河所運,止給太康、咸平、尉氏等縣軍糧而已。惟汴河專運粳米,兼以小麥,此乃大倉蓄積之實。今仰食於官廩者,不惟三軍,至於京師士庶以億萬計,太半待飽於軍稍之餘,故國家於漕
 事至急至重。然則汴河乃建國之本,非可與區區溝洫水利同言也。近歲已罷廣濟河,而惠民河觔斗不入大倉,大眾之命,惟汴河是賴。今陳說利害,以汴河為議者多矣。臣恐議者不已,屢作改更,必致汴河日失其舊。國家大計,殊非小事。願陛下特回聖鑒,深賜省察,留神遠慮,以固基本。」方平之言,為王安石發也。



 六年夏,都水監丞侯叔獻乞引汴水淤府界閑田,安石力主之。水既數放,或至絕流,公私重舟不可蕩,有閣折者。帝以人情不
 安,嘗下都水分析,並詔三司同府界提點官往視。十一月,範子奇建議:冬不閉汴口,以外江綱運直入汴至京,廢運般。安石以為然。詔汴口官吏相視,卒用其說。是後高麗入貢,令溯汴赴闕。



 七年春,河水壅溢,積潦敗堤。八月,御史盛陶謂汴河開兩口非便,命同判都水監宋昌言視兩口水勢,檄同提舉汴口官王珫。珫言訾家口水三分,輔渠七分。昌言請塞訾家口,而留輔渠。時韓絳、呂惠卿當國,許之。



 八年春,安石再相,叔獻言:「昨疏浚汴河,
 自南京至泗州,概深三尺至五尺。惟虹縣以東,有礓石三十里餘,不可疏浚,乞募民開修。」詔檢計工糧以聞。七月,叔獻又言:「歲開汴口作生河,侵民田,調夫役。今惟用訾家口,減人夫、物料各以萬計,乞減河清一指揮。」從之。未幾,汴水大漲,至深一丈二尺,於是復請權閉汴口。



 九年十月,詔都水度量疏浚汴河淺深,仍記其地分。十年,範子淵請用浚川杷,以六月興工,自謂功利灼然,請「候今冬疏浚畢,將杷具、舟船等分給逐地分。使臣於閉口
 之後,檢量河道淤澱去處,至春水接續疏導」。大抵皆無甚利。已而清汴之役興。



\end{pinyinscope}