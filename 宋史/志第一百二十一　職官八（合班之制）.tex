\article{志第一百二十一 職官八(合班之制)}

\begin{pinyinscope}

 建隆以後合班之
 制



 中書令侍中同中書門下平章事已上為宰相。



 親王、樞宏使、留守、節度、京尹兼中書令、侍中、同中書門下平章事已上並為使相。



 尚書令太師太尉太傳太保司徒司空舊儀,太師、太傳、太保為三師。
 太尉、司徒、司空為三公。太尉在太保
 下。國朝以來,自太傅除太尉,今依此次序。其三師、三公之稱如舊儀制。



 樞密使知樞密院事參知政事舊在樞密使下。



 樞密副使舊在知院之上。



 同知樞密院事宣徽南院、北院使簽書樞密院事參政以下班位臨時取奏裁



 太子太師、太傳、太保左、右僕射太子少師、少傅、少保諸府牧開封、河南、應天、大名、江陵、興元、真定、江寧、京兆、鳳翔、河中。又有大都督、大都護,今皆領使,無特為者。御史大夫觀文殿大學士舊無此位。



 六尚書吏、兵、戶、刑、禮、工。



 左、右金
 吾衛左、右衛上將軍
 門下、中書侍
 郎舊在尚書下。



 節度使泰寧、武寧、彰信、鎮海、天平、安化、武成。忠武、鎮海、可陽、山南東道、武勝、崇信、昭化、保康、天雄、成德、鎮寧、彰德、永清、安國、威德、靜難、彰化、雄武、保大、淮南、忠正、保信、保靜、集慶、建康、寧國、鎮南、昭信、荊南、寧海、武昌、安遠、武安、鎮東、平江、鎮江、宣德、保寧、康國、威武、建寧、益州、安靜、武信、山南西道、昭武、安德、武定、寧海、寧江、武康、清海、靜江、寧遠、建武、高州定南、密州靜海、涼州西河、沙州歸義、洮州保順、應州彰國、威城、昌化、豐州、天德、朔州振武、雲州大同。



 觀文殿學士舊曰文明殿,若學士官尚書者自從本班。



 資政殿大
 學士三司使與觀文、資政班位臨時取裁。



 玉清昭應宮、景靈宮、會靈觀副使與三司使、翰林學士班位臨時取裁。



 翰林學士承旨翰林學士資政殿學士翰林侍讀、侍講學士龍圖閣學士天章閣學士樞密直學士龍圖直學士天章直學士左、右散騎常侍舊在諸衛上將軍下。



 六統軍左、右龍武左、右羽林、左右神武,



 諸衛上將軍左、右驍衛左、右武衛左、右屯衛左、右領軍衛左、右千牛衛。



 太子賓客太常、宗正卿御史中丞權中丞立中丞磚位。內殿起居日止立本官班。



 左、右丞諸行侍郎節度觀察留後
 給事中左、右諫議大夫中書舍人知制誥龍圖閣待制天章閣待制觀察使秘書監光錄、衛尉、太僕、大理、鴻臚、司農、太府卿內客省使國子祭酒殿中、少府、將作監景福殿使延福宮使客省使開封、河南、應天、大名尹太子詹事諸王傅司天監諸衛大將軍太子左右庶子引進使防禦使齊、濟、沂、登、萊、鄭、汝、蔡、穎、均、郢、懷、衛、博、磁、洺、棣、深、瀛、雄、霸、莫、代、絳、解、龍、和、蘄、舒、復、眉、象、陸、果。



 團練使單、濮、濰、唐、祁、冀、隰、忻、成、鳳、海、鼎。



 三司鹽鐵、
 度支、戶部副使官至諫議大夫已上,從本官。



 玉清昭應宮、景靈宮、會靈觀判官太常寺、宗正少卿秘書少監光錄等寺七寺少卿宣慶使四方館使國子司業殿中、少府、將作少監開封、河南、應天、大名少尹太子少詹事、左右諭德太子家令太子率更令太子僕諸州刺史淄、趙、德、濱、保、並、汾、澤、遼、憲、嵐、石、虢、坊、丹、階、乾、商、寧、原、慶、渭、儀、環、楚、泰、泗、濠、光、滁、通、黃、真、舒、江、池、饒、信、太平、吉、袁、撫、筠、岳、澧、峽、歸、辰、衡、永、全、郴、邵、常、秀、溫、臺、衢、睦、處、南劍、汀、漳、綿、漢、彭、邛、蜀、嘉、簡、黎、雅、維、茂、資、榮、昌、普、渠、合、戎、瀘、興、劍、文、集、壁、巴、蓬、龍、施、萬、開、達、涪、渝、昭、循、潮、連、梅、英、賀、封、南雄、
 端、新、康、恩、春、惠、韶、梧、藤、龔、象、潯、貴、賓、橫、融、化、竇、高、雷、南儀、欽、鬱林、廉、瓊、崖、儋、萬安。



 諸王府長史、司馬司天少監樞密都承旨如客省使以下充者,依本職同班。如合門使充。即在合門使之上。如自見任內客省使以下轉南班官充。亦與同班,仍在舊職之上。如自客省副使以下轉南班官充者,並在合門使之上。



 宣政使昭宣使東上、西上合門使樞密承旨樞密副都承旨諸軍衛將軍起居郎起居舍人知雜御史侍御史諸行郎中左右司吏部兵部司封司勛考功職方駕部庫部度支戶部金部倉部刑部都官比部司門禮部工部祠部主客膳部屯田虞部水部。



 皇城以下
 諸司使皇城洛苑右騏驥尚食左騏驥御廚內藏庫軍器左藏儀鸞南作坊弓箭庫北作坊衣庫莊宅六宅文思東作坊內苑牛羊如京東綾錦香藥崇儀榷易西京左、右藏氈毯西綾錦西京作坊鞍轡庫東染院酒坊本染院法酒庫禮賓翰林醫官供備庫。



 樞密院副承旨、諸房副承旨如帶南班官者,在諸司使之下;不帶南班官者,在皇城副使之上。



 殿中侍御史左、右司諫諸行員外郎客省引進、合門副使左、右正言監察御史太常博士皇城以下諸司副使諸次府少尹大都督府左、右司馬兗、徐、潞、陜、揚、杭、越、福。



 通事舍人國
 子博士《春秋》、《禮記》、《毛詩》、《尚書》、《周易》博士都水使者開封、祥符、河南、洛陽、宋城縣令太常、宗正、秘書丞著作郎殿中丞內殿承制殿中省尚食、尚藥、尚衣、尚舍、尚乘、尚輦奉御大理正太子中允、左右贊善大夫內殿崇班合門祗候太子中舍、洗馬太子諸率府率左、右衛左、右監門左、右清道左右司御。樞密院兵房、吏房、戶房、禮房副承旨東頭、西頭供奉官太子諸率府副率諸衛中郎將左、右金吾左、右衛左、右千牛
 左、右羽林。



 郎將左、右金吾左、右衛。



 左、右侍禁諸王友諸王府諮議參軍官高者從本官。



 司天春官、夏官、中官、秋官、冬官正節度行軍司馬、副使秘書郎左、右班殿直著作佐郎大理寺丞諸寺、監丞大地評事太學、廣文博士太常太祝、奉禮郎秘書省校書郎、正字御史臺、諸寺、監主簿國子助教廣文、太學、四門、書學、算學博士律學助教書、算學無助教。



 司天靈臺郎、保章正、挈壺正三班奉職、借職防禦、團練副使
 留守、京府、節度、觀察推官節度掌書記觀察支使防禦、團練判官留守、京府、節度、觀察推官軍事判官防禦、團練、軍事推官軍、監判官諸軍別駕、長史、司馬司錄、錄事參軍司理參軍三京府軍巡判官在諸曹參軍之下。諸州諸司參軍軍巡判官諸縣令赤縣丞諸縣主簿、尉諸軍文學、參軍、助教。



 元豐以後合班之制



 諸太師舊制,太尉為三公,在太傳上,政和改為三少。太傳太保侍中中書令
 政和二年,改左輔右弼,靖康後復。



 尚書令少師少傳少保舊太尉、司徒、司空,政和二年改。



 尚書左、右僕射政和二年,改太宰、少宰,靖康復舊,元豐令王在左右僕射下。



 開府儀同三司知樞密院事門下、中書侍郎尚書左、右丞同知樞密院事簽書樞密院事元豐罷,元祐復置,政和入雜壓。太子太師大傅太保特進觀文殿大學士太尉舊為三公,政和二年,改為三少,復以太尉為武選一品,位節度使上。



 太子少師少傅少保冀、兗、青、徐、揚、荊、豫、梁、雍州牧元祐復置,政和入雜壓。



 御史大夫觀文殿學士資政、元豐令在節度使下。



 保和政和五年,置宣和殿大學士、學士,宣和元年,改為保和學士。待制同。



 殿大學
 士吏部、戶部、禮部、兵部、刑部、工部尚書金紫、銀青光祿大夫左、右金吾衛上將軍節度使翰林學士承旨翰林學士資政、保和、端明政和四年,改為延康。殿學士龍圖、天章、寶文、元豐二年,增置直學士,待制同。



 顯謨、元豐元年增置。



 徽猷崇寧二年增置。



 閣學士左、右散騎常侍御史中丞舊在直學士下,元豐八年升。



 開封君崇寧三年升。



 尚書列曹侍郎樞密直學士政和四年,改為述古殿直學士。



 龍圖、天章、寶文、顯謨、徽猷閣直學士宣奉、光祐,左光錄大夫。正奉、元祐,右光錄大夫,並大觀二年改置。正議、通奉大夫殿中監舊在秘書監下,崇寧二年升。



 大司成
 崇寧二年增置。



 左右驍衛、武衛、屯衛、領軍衛、監門衛、千牛衛上將軍太子賓客、詹事給事中中書舍人通議大夫承宣使舊節度觀察留後,政和七年始改。



 左、右諫議大夫保和殿待制龍圖、天章、寶文、顯謨、微猷閣待制太中大夫太常卿大司樂崇寧二年增置。宗正卿秘書監殿中少監崇寧二年升。



 觀察使中大夫光錄、衛尉、太僕、大理、鴻臚、司農、太府卿中奉、元祐,左中散大夫,大觀二年改。中散、通侍大夫舊內客省使,政和二年改,橫行、正使、副使、大使臣、小使臣並改。



 樞密都承旨國子祭酒太常少卿典藥崇寧二年增置。



 宗正少
 卿秘書少監正侍、舊延福宮使,政和二年改。



 宣正、履正、協忠、三階系政和六年增置。



 中侍、中亮大夫舊客省使。



 太子左、右庶子中衛、舊引進使。



 翊衛、親衛大夫政和六年增置。



 防禦、團練使諸州刺史左、右金吾以下諸衛大將軍附馬都尉集英殿修撰政和八年置。七寺少卿朝議、奉直大夫元祐,右朝議大夫,大觀二年改置。



 尚書左、右司郎中右文殿修撰舊集賢殿修撰,不入雜壓,政和六年改,增入。國子、闢雍司業崇寧元年增置。



 少府、將作、軍器監都水使者入內內侍省都都知政和,改知入內內侍省事。內侍省都都內知政和,改知內侍省事。



 拱衛大夫舊四
 方館使。



 太子少詹事、左右諭德入內內侍省副都知內侍省副都知政和並改同知省事。左武、右武大夫舊東、西上合門使。



 入內內侍省押班內侍省押班政和並改簽書省事。



 管乾殿中省尚舍、尚藥、尚醞、尚輦、尚衣、尚食局崇寧二年增置。



 樞密副都承旨起居郎起居舍人侍御史尚書左、右司員外郎秘閣修撰政和六年增置。



 開封少尹崇寧三年升。



 尚書吏部、司封、司勛、考功、戶部、度支、金部、倉部、禮部、祠部、主客、膳部、兵部、職方、庫部、駕部、刑部、都官、比部、司門、工部、屯田、虞部、水部郎中開封
 府司錄事舊錄參軍事在兩赤縣令之上,崇寧三年升改。



 直龍圖閣元豐、元祐令,並不入雜壓,政和增入,餘同。



 朝請、朝散、朝奉大夫直天章閣政和六年增入。



 殿中侍御史左、右司諫左、右正言舊在監察御史上,政和升。



 符寶郎大觀元年增置。



 殿中省尚食、尚藥、尚醞、尚輦、尚衣、尚舍典御崇寧三年增置。



 內符寶郎大觀元年增置。



 樞密副承旨元豐令,有知上州在此下,元祐以後並去。



 武功、舊皇城使,自此以下,並政和六年改。



 武德、舊宮苑、左右騏驥、內藏庫使。



 和安、成和、成安、成全、舊翰林、尚食、軍器、儀鸞使。



 武顯、舊左藏、東西作坊使。



 武節、舊莊宅、六宅、文思使。



 平和、舊綾錦使,初改保和,政和五年,以犯殿名,改保痊;宣和六年,又改為平和。



 武略、舊內園、洛苑、如京、崇
 儀使。保安、舊榷易使。



 武經、舊西京左藏庫使。



 武義大夫舊西京作坊、東西染院、禮賓使。翰林良醫舊翰林醫官使。



 武翼大夫舊供備庫使。



 尚書諸司員外郎直寶文閣政和六年增置。



 開封府司六曹事崇寧三年增置。



 樞密院諸房副承旨朝請、朝散、朝奉郎直顯謨閣政和六年增入。



 少府、將作、軍器少監諸衛將軍太子侍讀、侍講正侍、宣正、履正、協忠、自宣正至協忠,並政和六年增置。



 中侍、中亮、中衛、翊衛、親衛、拱衛、左武、右武郎,舊橫行、副使、政和六年改。



 監察御史元豐令,有知中州在此下。



 殿中丞舊秘書丞下,崇寧二年升。



 直徽猷閣政和六年置。



 承議郎武功至武義
 郎翰林醫正武翼郎諸司副使。太子中合太子舍人親王府翊善、贊讀、直講舊侍讀、侍講,政和改。



 太常丞大晟樂令崇寧二年增置。太醫令宗正、大宗正秘書丞直秘閣政和六年置,元豐令,知下州在此下。奉議郎大理正著作郎太史局令直翰林醫官局殿中省六尚奉御舊在大理正之上,政和改。



 太醫丞元祐增置。



 合門宣贊舍人舊合門通事舍人,政和六年改。



 兩赤縣令太子左右衛、司御、清道、監門,內率府率七寺丞秘書郎太常博士陵臺令元祐中增置。著作佐郎殿中省主簿崇寧二年增置。



 國子監丞闢雍丞崇寧二年增置。



 宗
 子、崇寧元年增置。



 國子博士大理司直、評事敦武、舊內殿承制,政和六年改,下同。



 通直郎修武郎內殿崇班。內常侍元豐令,上州通判在此下。



 太史局正少府、將作、軍器、都水監丞開封府參軍事崇寧三年增置。



 太醫局正秘書省校書郎、天字親王府記室元豐,元祐令,有「參軍」字,政和三年除去。



 太史局五官正御史臺檢法官、主簿元豐令在監丞上,元祐在監丞下。



 九寺、大晟府崇寧三年增置。



 主簿合門祗候樞密院逐房副承旨元豐令,中下州通判在此下。供奉官舊內東頭供奉官,政和六年改,下同。



 從義郎東頭供奉官,



 左侍禁內西頭供奉官。



 秉義郎西頭供奉官,



 太子諸率府
 副率乾當左、右廂公事崇寧中增入。



 右侍禁左班殿直殿頭高品。



 忠訓、忠翊、左、右侍禁。



 宣教郎舊宣德郎,政和四年改。



 太學、闢雍、崇寧元年增置。



 武學、律學開封府大觀元年置。



 博士太常寺奉禮郎大晟府協律郎崇寧二年增置。



 太常寺太祝、郊社、籍田令光錄寺太官令元豐、元令,在太學博士上。



 五監、闢雍崇寧元年增置。



 主簿宣義郎成忠、保義、左右班殿直。



 承事。承奉、承務郎宗子、崇濘元年增置。



 國子、太學、闢雍正武學諭崇寧元年置。



 律學正崇寧元年置。



 太醫局丞京府、諸州司錄事承直郎崇寧三年,以留守節度判官改,凡選人七階,儒林至迪功。京畿
 縣令兩赤縣丞三京赤縣令右班殿直高班。



 黃門內品承節、承信郎舊三班奉職、借職。



 京府、諸州司六曹事元豐、元祐令,並六曹參軍。政和三年,除去「參軍」字,為司錄事,司儀曹事,餘曹放此。



 儒林、舊掌書記。



 文林、從事郎三京畿縣令京畿縣丞三京赤縣、畿縣丞兩赤縣主簿、尉諸州上、中、下縣令丞從政郎舊司錄事參軍、縣令。



 京府、諸州掾官修職郎舊知錄事參軍、知縣事。



 京畿縣主簿、尉諸州上、中、下縣主簿尉城砦主簿馬監主簿迪功郎舊巡判官、司理、司法、司戶。諸州司士文學助教舊參軍事。



 唐令,定流內一品至九品,有正從上下階之制。其後,升侍中、中書令為為正二品,御史大夫、散騎常侍、兩省侍郎為正三品,御史中丞正四品。諫議大夫分左、右,改將作大匠為監,太史局為司天監,置大監正三品,少監正四品上,丞正六品上,寺簿正七品上,主事正八品下,五官正五品上,副正正六品,靈臺郎正七品下,保章正從七品上,挈壺正八品上,五官監候正八品下,司歷從八品上,司辰正九品上。又置國子、五經博士為正五品上,左、
 右金吾衛上將軍為從二品,左、右龍武、神武軍大將軍為正三品,將軍為從三品。又置內侍監為為正三品,少監從四品,改諸州府學博士為文學,在參軍上。五代復置尚書令為一品,升右丞為正四品上,降諫議在給事之下。



 宋初,並因其制,唯升宗正卿為正四品,丞為從五品。其軍器監、少監,甲弩坊署令、丞、監作、錄事,昭文館校書郎,司辰、司歷、監候,殿中諸署監事、計官,太常諸陵廟、太醫、太公廟署令丞,醫針博士、助教,按摩、咒禁博士,卜正,
 卜博士,宗正崇玄署令、丞,大理獄丞,鴻臚典客,太府寺平準、左右藏、常平署令丞,都水監舟楫、河渠署令丞,官苑總副監牧監副、丞、主簿,諸園苑司並百工等監、副監及丞,諸倉、諸冶、諸屯、溫湯監及丞,掌漕,諸軍衛錄事諸曹參軍、司階、中候、司戈、執戟、校尉、旅帥、隊正、隊副、正直長、長上、備身、左右備身,左右親、勛、翊衛府中郎將,兵曹三衛,折沖、果毅、別將、長史、兵曹參軍、校尉、旅帥、隊正、隊副,鎮軍司馬、判司,太子詹事府丞、主簿、司直,司議郎,舍
 人,文學,校書,正字,崇文館校書,侍醫,通事舍人,左、右春坊錄事、主事,三寺丞、主簿,諸署令、丞,典倉署園丞,廄牧典乘,內坊典內及丞、典直,率府長史、錄事諸曹參軍、司階、中候、司戈、執戟、校尉、旅帥、隊正、隊副、直長、千牛備身,親、勛、翊府中郎將,兵曹三衛,王府文學,東西合祭酒,掾、屬、主簿、錄事諸曹參軍、行參軍、典簽,典軍、執杖執乘親事、校尉、旅帥、隊正、隊副,國令,大農尉、丞,公主邑令丞、邑司錄事,河南應天及諸次府都督都府功曹、倉、兵曹
 參軍,諸州司功、司倉、司兵參軍,諸縣丞,京縣錄事,諸鎮倉曹、兵曹參軍,戍主、戍副,關津令丞,並門下省城門、符寶郎,太常寺協律郎,軍器監丞、主簿,太常寺郊社、太卜、廩犧,光祿寺太、官珍羞、良醞、掌醢,衛尉寺武器、守宮,太僕寺乘黃、典廄、典牧、車府,鴻臚寺典客、司儀,司農寺上林、太倉、鉤盾、導官,太府寺諸市,少府監中尚、左尚、右尚、織染、掌冶,將作監左校、中校、甄官署令丞、監膳,殿中省六局直長、食醫、侍御、醫司、醫佐、掌輦、奉乘、司廩,太子典膳、
 典藥、內直、典設、宮門郎並局丞,皆存其名而罕除者,綿不祿『,惟常命官者載之。諸司主事、事皆存,而無士人為之。別置中書、樞密、宣徽院、三司及內庭諸司,沿舊制而損益焉。



 建隆三年三月,有司上《合班儀》:「太師,太傳,太保,太尉,司徒,司空,東宮三太,嗣王,郡王,僕射,三少,三京牧,大都督,大都護,御史大夫,六尚書,常侍,門下、中書侍郎,太子賓客,太常、宗正卿,御史中丞,左、右諫議大夫,給事中,中書舍人,左、右丞,諸行侍郎,秘書監,光祿、衛尉、太
 僕、大理、鴻臚、司農、太府卿,國子祭酒,殿中、少府、將作監,前任、見任節度使,開封、河南、太原尹,詹事,諸王傅,司天監,五府尹,國公,郡公,中都督,上都,護,下都督,庶子,五大都督府長史,中都護,副都護,太常、宗正少卿,秘書少監,光祿等七少卿,司業,三少監,三少尹,少詹事,諭德,家令,率更令、僕,諸王府長史、司馬,司天少監,起居郎、舍人,侍御史,殿中侍御史,補關,拾遺,監察御史,郎中,員外郎,太常博士,五府少尹,五大都督府司馬,通事舍人,國子、五經
 博士,都水使者,四赤縣令,太常、宗正、秘書丞,著作郎,殿中丞,六尚奉御,大理正,中允,贊善,中舍,洗馬,諸王友,諮議參軍,司天五官焉,凡雜坐之次,以此為準。



 詔曰:「尚書中臺,萬事之本,而班位率比兩省官;節度使出總方面,其檢校官多至師傅、三公者,而位居九寺卿監之下,甚無謂也。其給事中、諫議、舍人,宜降於六曹侍郎之下;補闕次郎中、拾遺,監察次員外郎、節度使,升於中書侍郎之下。」乾德五年正月朔,乾元殿受朝,升節度使班在龍
 墀內金吾將軍之上。



 淳化三年八月,有司復位《合班儀》,詔升尚書令三師之上。四年,節度使升常侍之上,觀察使在秘書監之上,防禦、團練使在庶子之下,刺史在太子僕之下,又升諸行郎中於殿中侍御史之上,至道三年七月,令節度觀察留後在給事中之上。大中祥符元年八月,升兩省侍郎班常侍之上。



 天禧三年十一月,令節度使班中書侍郎之下。其序班及視品之制,樞密使、副使、參知政事、宣微使並班宰相後。



 樞密使不兼平章事者,立參知政事
 前,在宣微使下。至道三年升在上。大中祥符九年九月,詔自今參知政事、樞密副使並以先後為次。宣徽使同。



 資政殿大學士立文明殿學士之上。舊文明殿學士在樞密副使之上,太平興國五年移在下。



 資政殿學士、翰林侍讀學士在翰林學士下。



 建隆三年,令翰林學士班諸行侍郎下,官至丞、郎者在常侍上,至尚書者依本班。淳化五年,升丞、郎之上。樞密直學士同。



 龍圖閣學士在樞密直學士上,龍圖直學士在其下,仍少退。待制在知制誥之下。



 景德元年,初置待制,赴內朝,其五日起居,止敘本班。大中祥符二年,升侍知制誥,仍在其下。



 權三司使立知制誥上。帶學士職者從本班。三司副使立少卿、監上。官高者從本班,並為內品職。宮觀副使立
 學士班。在翰林學士上,其學士為者,止本班。



 判官立三司副使之下。知制誥以上為者,從本班。



 給、諫權御史中丞者,令正衙立中丞磚位。餘就本班。凡起復,皆如初授,在本官之末,亦有特旨令敘舊班者,內客省使視七寺大卿,景福殿使、客省使視將作監,引進使視庶子。宣慶使、四方館使視少卿,宣政、昭宣、合門使視少監。客省等副使視員外郎。皇城使以下諸司使視郎中,副使視太常博士。內殿承制視殿中丞,崇班及合門祗候視贊善大夫,供奉官視諸衛率,侍禁
 視副率。殿直視著作佐郎,奉職、借職在諸州幕官上。樞密都承旨在合門使下,副承旨、諸房副承旨在諸司使下,逐房副承旨在洗馬下。金吾衛、左右衛上將軍並在節度使上,六統軍、諸衛上將軍在常侍下,乾德二年,令上將軍在中書侍郎之下。淳化四年,升金吾、左右衛在尚書之下,仍於節度使之上敘。



 大將軍在大監下,將軍在少監下。仍在合門使之下,



 金吾立本班上。謂中郎將。



 諸衛率、副率在洗馬下。凡內職,視朝官者在其下,視京官者在其上。



 皇親之制:開寶六年,詔:「晉王位望俱崇,親賢莫二,宜位在宰相之上。」太平興國八年,楚王、廣平郡王出合,令宰相立親王之上。



 天禧四年七月先天節,群臣上壽,宰相闕,命涇王元儼攝太尉。



 景德中,皇侄武信軍節度惟吉加同平章事。時駙馬都尉石保吉先為使相,史館引唐制,宗室在同品官上,遂升惟吉焉。大中祥符元年正月,有司上《都亭驛酺宴位圖》,皇從侄孫內殿崇班守節與從侄右衛將軍惟敘等同一班。上曰:「族子諸父,膊可同列?」乃命重行設位九年正月,興
 州團練使德文言:「男侍禁承顯赴起居,請在惟忠子從恪之上。」時從恪雖侄行,而拜職在前,遂詔宗正寺定《宗室班圖》以聞。宗正言:「按《公式令》:朝參行立,職事同者先爵,爵又同者先齒。今請宗子官同而兄叔次弟侄者,並虛一位而立。」天禧四年五月,左正言、知制誥張師德言:「奉詔知穎州,緣皇弟德雍見任本州防禦使,其署銜望降規式。」中書門下言:「據御史臺稱,每大朝會立班,皇親防禦、團練、刺史次節度使下,稍退序立。」詔師德序署
 位德雍之下。其外官制置、發運、轉運使副使,不限官品,著位並在提點刑獄之上。



 舊止從官,大中祥符七年,詔定其制。



 朝官知令、錄在判官之上,京官在判官之下、推官之上。長史、司馬、別駕在幕府官下、錄事參軍上,見長史庭參。監當朝官殿直以下,在通判、都監之下,判官之上。其通判與都監並依官次。京官奉職、借職監當者,依知令、錄列在判官之下。元豐制行,參以寄祿官品高下,更革既多,別為班序。其後元祐、崇寧、大觀、政和,復有增益更革者,別附於
 其下云。



 至道二年,祠部員外郎主判都省郎官事王炳上言曰:



 尚書省,國家藏載籍、典治教之府,所以周知天下地理廣袤、風土所宜、民俗利害之事。當成周之世,治定制禮,首建六官,漢、唐因之。自唐末亂雜,急於經營,不遑治教,故金穀之政主於三司,曹名雖存,而其實亡矣。謹按:吏部四司,天官之職,掌文官選舉,周知天下吏功過能否,考定升降之類;戶部四司,司徒之職,掌邦五教,周知天
 下戶口之數;禮部四司,宗伯之職,掌國五禮,辨儀式制度,周知天下祠典祠祀之類;兵部四司,司馬之職,掌武人選舉,周知天下兵馬器械之數;刑部四司,司寇之職,掌國法令,周知天下獄訟刑名徒隸之數;工部四司,司空之職,掌國百工,周知天下封疆、城圻、山澤、草木、川瀆、津渡、橋船、陂池之數。凡此二十四司所掌事務,各封圖書,具載名數,藏之本曹,謂之載籍;所以周知天下事,由中制外,如指諸掌。



 今職司久廢,載籍散亡,惟吏部四司
 官曹小具,祠部有諸州僧道文帳,職方有諸州閏年圖經,刑部有詳覆諸州已決大闢案牘及勾禁奏狀,此外多無舊式。欲望令諸州,每年造戶口稅租實行簿帳,寫以長卷者,別寫一本送尚書省,藏於戶部。以此推之,其餘天下官吏、民口、廢置、祠廟、甲兵、徒隸、百工、疆畎、封洫之類,亦可以籍其名數,送尚書省,分配諸司,俾之緘掌;候期歲之後,文籍大備,然後可以振舉官守,興崇治教。望選大僚數人博通治體者,參取古今禮典及諸令式,與
 三司所受金谷、器械、簿賬之類,仍詳定諸州供送二十四司載籍之式。如此,則尚書省備藏天下事物名數之籍,如秘閣藏圖書,太學藏經典,三館藏史傳,皆其職也。



 太宗覽奏,嘉之。詔尚書丞、郎及五品以上集議。



 吏部尚書宋琪等上奏曰:「王者六官,法天地四時之柄,百官之本,典教所出,望委崇文院檢討六曹所掌圖籍,自何年不擊都省,詳其廢置之始,究其損益之源,以期恢復。既而其議亦寢。



 大中祥符九年,真宗與宰相語及尚書省
 制,言事者屢請復二十四司之制。楊礪嘗言:「行之不難,但以郎官、諸司使同領一職,則漸可改作。」王旦曰:「唐設內諸司使,悉擬尚書省:如京,倉部也;莊宅,屯田也;皇城,司門也;禮賓,主客也。雖名品可效,而事任不同。唐朝諸司所領,惟京邑內外耳,諸道兵賦各歸藩鎮,非南宮一郎中、員外所能制也。朝廷所得三分之一,名曰上供,其它留州、留使之名,皆藩臣所有。今之三司即尚書省,故事盡在,但一毫所賦皆歸於縣官而仰給焉,故蠲放則
 澤及下,予賜則恩歸上,此聖朝不易之制也。」



 咸平四年,左司諫、知制誥楊億上疏曰:



 國家遵舊制,並建群司,然徒有其名,不舉其職。只如尚書會府,上法文昌,治本是資,政典攸出,條目皆具,可舉而行。今之存者,但吏部銓擬,秩曹詳覆。自餘租庸筦榷,由別使以總領;尺籍伍符,非本司所校定。職守雖在,或事有所分;綱領雖存,或政非自出。丞轄之名空設而無違可糾,端揆之任雖重而無務可親。周之六官,於是廢矣,且如寺、監素司於掌執,
 臺、閣咸著於規程,昭然軌儀,布在方冊。國家慮銓擬之不允,故置審官之司;憂議讞之或濫,故設審刑之署;恐命令之或失,故建封駁之局,臣以為在於紀綱植立,不在於琴瑟更張。若辨論官材歸於相府,即審官之司可廢矣;詳評刑闢屬於司寇,即審刑之署可去矣;出納詔命關於給事中,好封駁之局可罷矣。至於尚書二十四司各揚其職,寺、監、臺、閣悉復其舊,按《六典》之法度,振百官之遺墜,在我而已,夫豈為難?如此則朝廷益尊,堂陛
 益嚴,品流益清,端拱而天下治者,由茲道也。



 又以唐、虞之時,建官惟百,夏、商官倍,秦、漢益繁。施及有唐,六策咸在,自三公之極貴、九品之至微,著於令文,皆有員數。《傳》云:「官不必備,惟其人。」蓋闕之,斯可矣,若乃員外加置,茍非其材,故「灶下」、「羊頭」,形於嘲詠,「斗量車載」,播厥風謠,國體所先,尤須慎重。竊睹班簿,員外郎及三百餘人,郎中亦及百數,自餘太常國子博士、殿中丞、舍人、洗馬,俱不下數百人,率為常參,皆著引籍,不知職業之所守,多由
 恩澤而序遷。欲乞按唐制,應九品以上官並定員數。



 又念昔者秦之開郡置守,漢以天下為十三部,命刺史以領之。自後因郡為州,以太守為刺史,降及唐氏,亦嘗變更,曾未數年,又仍舊貫。今多命省署之職出為知州,又設通判之官以為副貳,此權宜之制耳,豈可為經久之訓哉?臣欲乞諸州並置刺史,以戶口多少置其奉祿,分下、中、上、緊、望、雄之等級,品秩之制率如舊章,與常參官比視階資,出入更踐,省去通判之目,但置從事之員,建
 廉察之府以統臨,按輿地之圖而區處。昔者興國初,詔廢支郡,出於一時;十國為連,周法斯在,一道署使,唐制可尋。至若號令之行,風教之出,先及於府,府以及州,州以及縣,縣及鄉里。自上而下,由近及遠,譬如身之使臂,臂之使指,提綱而眾目張,振領而群毛理。由是言之,支郡之不可廢也明矣。臣欲乞復置支郡,隸於大府,量地里而分割,如漕運之統臨,名分有倫,官業自舉。



 又睹唐制內外官奉錢之外,有祿米、職田,又給防合、庶僕、親事、
 帳內、執衣、白直、門夫,各以官品差定其數,歲收其課以資於家。本司又有公廨田、食本錢,以給公用。自唐末離亂,國用不充,百官奉錢並減其半,自餘別給一切權停。今君官於半奉之中已是除陌,又於半奉三分之內,其二以他物給之,鬻於市廛十裁得其一二,曾餬口之不及,豈代耕之足雲?昔漢宣帝下詔云「吏能勤事而奉祿薄,欲其無侵漁百姓難矣。遂加吏奉,著於策書。竊見今之結發登朝,陳力就列,其奉也不能致九人之飽,不及
 周之上農;其祿也未嘗有百石之入,不及漢之小吏。若乃左、右僕射,百僚之師長,位莫崇焉,月奉所入,不及軍中千夫之帥,豈稽古之意哉?欲乞今後百官奉祿雜給,並循舊制,既豐其稍入,可責以廉隅。官且限以常員,理當減於舊費,乃唐、虞之制也。



 凡品官,各設資考,課其殿最,歸於有司,或歷階以升,或越次而補。國朝多以郊祀覃慶而稍遷官,考功之黜陟不行,士流之清濁無辨。陛下深鑒其弊,始務惟新。昨有事於明禋,但篇加於階
 爵;雖矯前失,未振舊規。並乞依舊內外官各立考限,復令考功修舉其職,每歲置使考校,以表盡公,資秩改遷,賞罰懲勸,一遵典故,以振滯淹。



 又西漢以來,用秦武功之爵,惟列侯啟封,或逾萬戶,至關內侯,或有食邑,不過數百家。自是因循,以至唐室,但食邑者率為虛設,言實封者歲入有差。迨及聖朝,並無所給,至於除拜之際,猶名數未移,空有食採之稱,真同畫餅之妄。欲乞依元和中所定實封條貫支給,削去虛邑,但行實食,以寵勛臣。
 又國家每屬嚴禋,即覃大慶,敘封追贈,罔限彞章。乃至太醫之微,司歷之賤,率荷蓼蕭之澤,亦疏石窌之封,恩雖出於殊常,職不循於經制。



 又官勛之設,名品實繁,今朝散、銀青,猶闕命服,護軍、柱國,全是虛名,欲乞自今常參官,勛、散俱至五品者許封贈,官、勛俱至三品者許立戟。又五等之爵,施於賢才,雖有啟封之稱,曾無胙土之實。苴茅建社,固不可以遽行,翼子詒孫,亦足稽於舊典。內外官封至伯、子、男者,許蔭子;至公、侯者,許蔭孫;封國公
 者許嫡子、嫡孫一人襲封。



 又當今功臣之稱始於德宗,扈蹕將士並加「奉天定難功臣」之號,因一時之賞典,為萬世之通規。近代以來,將相大臣有加至十餘字者,尤非經據,不可遵行,所宜削除,以明憲度。昔者講求典禮,晉國以清,考核名實,漢朝稱治。當文化誕敷之際,是舊間章咸秩之時,跂見太平,正在今日矣。



 論者嘉之,然以因襲既久,難於驟革。



 既而言者繼請復二十四司之制。神宗既位,始命館閣校《唐六典》,以摹本賜郡臣,而置局詳定
 之。於是凡省、臺、寺、監領空名者,一切易之以階,元豐三年,詳定所上《寄祿格》,會明常禮成,即用新制,遷近臣秩。初,新階尚少,而轉行者得以易。及元祐初,朝議大夫六階以上始分左、右,紹聖中,罷之。崇寧初,自承直至將仕郎,凡換選人七階,又增宣奉至奉直大夫四階。政和末,自從政至迪功郎,又改選人三階,文階始備;而武階亦易正使為大夫,副使為郎。其橫班十二階使、副亦然。繼又增置宣正、履正大夫、郎,凡十階,通為橫班其後,復更
 開封守臣為尹牧,而內侍省悉仿機廷之號,六尚局之修,三降郎之建,及左輔、右弼、太宰、少宰之稱,員既濫冗,名益繁雜,由是官有視秩,元豐之制,至此大壞。及宣和末,王復請修《官制格目》,而邊事起,訖不果成。



 初,太平興國八年五月,太宗作《戒諭百官辭》二通,以付合門。一戒京朝官受任於外者,一戒幕職、州縣官,朝辭對別日,令舍人宣示之,各繕寫歸所治,奉以為訓焉。大中祥符元年,真宗以祥符降錫,述大中清凈為治之道,申誡
 百官,又作《誡諭辭》二道,易舊辭,賜出使京朝官及幕職、州縣官,其後,又作《文》、《武七條》。《文》,賜京朝官任轉運使、提點刑獄、知州府軍監、通判、知縣者:一曰清心,謂平心待物,不為喜怒愛憎之所遷,則庶事自正。二曰奉公,謂公直潔己,則民自畏服。三曰修德,謂以德化人,不必專尚威猛。四曰責實,勿競虛譽。五曰明察,謂勤察民情,勿使賦役不均,刑罰不中。六曰勸課,謂勸諭下民,勤於孝悌之行、農桑之務。七曰革弊,謂求民疾苦而厘革之。《武條》賜
 牧伯洎諸司使而下任部署、鈐轄、知州軍縣、都監、監押、駐泊巡檢者:一曰修身,謂修飭其身,使士卒有所法則,二曰守職,謂不越其職,侵撓州縣民政。三曰公平,謂均撫士卒,無有偏黨。四曰訓習,謂訓教士卒,勤習武藝。五曰簡閱,謂察視士卒,識其勤惰勇怯。六曰存恤,謂安撫士卒,甘苦皆同,當使齊心,無令失所。七曰威嚴,謂制馭士卒,無使越禁。仍許所在刊石或書廳壁,奉以為法。又以《禮記儒行篇》賜親民厘務文臣,其幕職、州縣官使臣
 賜敕戒礪。令崇文院刻板模印,送合門,辭日分給之。



 淳化元年,國子祭酒孔維上言:「中外文、武官稱呼假借,逾越班制,伏請一切禁斷。太宗命翰林學士宋白等議之。白等請:「自今文武臺省官及卿、監、郎中、員外並呼本官,太常博士、大理評事並不得呼『郎中』,諸司使、諸衛將軍未領刺史者、及諸司副使不得呼『太保』,供奉官以下不得呼『司徒』,校書郎以下令、錄事不得呼『員外郎』,判、司、簿、尉不得呼『侍御』,待詔、醫官不得呼『奉御』,其文武職事
 州縣官,如有檢校、兼、試、同正官者,稱之。」



 太宗時,郊祀行慶,群官率多進改。真宗初,右司諫孫何上言曰:「伏見國家撫有多方,並建眾職。外則郡將、通守,朝士代行;關征、榷酤,使者兼掌;下至幕府職掾之微,或自朝廷選補而授。用人既廣,推擇難精。貢部上名,動逾千計;門資入仕,亦及百人。稍著職勞,即升京秩;將命而出,冗長尤多。每躬祀圓丘,誕敷霈澤,無賢不肖,並許敘遷。至使評事、寺丞,才數載而通閨籍;贊善、洗馬,不十年而登臺郎。竊計
 今之班簿,臺、省、宮、寺凡八百員,玉石混淆,名品猥濫。異夫《虞書》考績、《周官》計治之法也。有唐舊制,郊禋慶宥,但進階、勛而已,今若十年之內,肆赦相仍,必恐京僚過於胥徒,朝臣多於州縣,豈惟連車平斗之刺,亦有敗財假器之失。況祿廩所賦,皆自地征所來,須從民力,何必空竭公藏,附益私人。已授者朘削既難,未遷者防閑宜峻,古人所謂『損無用之費,罷不急之官』,正在此也,伏願降詔書,自今郊祀,群官一例不得遷陟,必若績用有聞、
 才名夙著、自可待之不次,豈俟歷階而升。至於省並吏員,上系與奪。」時左司諫耿望亦以為言,故咸平二年親郊,止加階、勛,命有司考其殿最而黜陟之。然三年差遣受代,率皆考課引對,多獲進改,罕有退黜,而官籍浸增矣。



 紹興以後合班之制



 諸太師、太傳、太保左丞相、右丞相少師、少傳、少保王樞密使開府儀同三司知樞密院事參知政事同知樞密
 院事樞密副使簽書樞密院事太子太師、太傳、太保特進觀文殿大學士太尉太子少師、少傅、少保冀、兗、青、徐、揚、荊、豫梁、雍州牧御史大夫觀文殿學士資政、保和殿大學士吏部、戶部、禮部、兵部、刑部、工部尚書金紫、銀青光祿大夫光祿大夫左、右金吾衛上將軍左、右衛上將軍殿前都指揮使節度使翰林學士承旨翰林學士資政、保和、端明殿學士龍圖、天章、寶文、顯謨、徽猷、敷文閣學士左、右散騎常侍權六曹尚書御史中丞開封尹尚
 書列曹侍郎樞密直學士龍圖、天章、寶文、顯謨、徽猷、敷文閣直學士宣奉、正奉、正議、通奉大夫左、右驍衛、武衛、屯衛、領軍衛、監門衛、千牛衛上將軍太子賓客、詹事給事中承宣使中書舍人通議大夫殿前副都指揮使左、右諫議大夫保和殿待制龍圖、天章、寶文、顯謨、微猷、敷文閣待制權六曹侍郎太中大夫觀察使太常卿宗正卿秘書監馬軍都指揮使步軍都指揮使馬、步副都指揮使中大夫光祿、衛尉、太僕、大理、鴻臚、司農、太府卿中奉、中
 散大夫內客省使通侍大夫樞密都承旨國子祭酒太常少卿宗正少卿秘書少監正侍、宣正、履正、協忠大夫中侍、中亮大夫太子左、右庶子中衛、翊衛親衛大夫知合門事殿前都虞候馬軍都虛候步軍都虞候防禦使捧日、天武四廂都指揮使龍、神衛四廂都指揮使團練使諸州刺史左、右金吾以下諸衛大將軍駙馬都尉集英殿修撰七寺少卿朝議、奉直大夫中書門下省檢正諸房公事尚書左、右司郎中右文殿修撰國子司業少
 府、將作、軍器監都水使者入內內侍省、內侍省都知宣政使拱衛大夫太子少詹事、左右諭德入內內侍省、內侍省副都知昭宣使左武大夫同知合門事右武大夫入內內侍省、內侍省押班樞密承旨樞密副都承旨起居郎起居舍人侍御史帶御器械尚書左、右司員外郎樞密院檢詳諸房文字秘閣修撰開封少尹。太子侍讀、侍講尚書吏部、司封、司勛、考功、戶部、度支、金部、倉部、禮部、祠部、主客、膳部、兵部、職方、駕部、庫部、刑部、都官、比
 部、司門、工部、屯田、虞部、水部郎中開封府判官、推官直龍圖閣朝請、朝散、朝奉大夫直天章閣殿中侍御史左、右司諫左、右正言符寶郎內行寶郎樞密副承旨武功、武德、和安、春官、成和、夏官、成安、中官、成全、秋官、武顯、武節、平和、冬官、武略、保安、武經、武義、武翼大夫尚書諸司員外郎直寶文閣開封府司祿參軍事樞密院諸房副承旨朝請、朝散、朝奉郎直顯謨閣少府、將作、軍器少監諸衛將軍正侍、宣正、履正、協忠、中侍、中亮、中衛、翊衛、親
 衛、拱衛、左武、右武郎監察御史直微猷、敷文閣承議郎中郎將翰林良醫武功、武德、和安、成和、成安、成全、武顯、武節、平和、武略、保安、武經、武義、武翼郎太子中舍人太子舍人親王府翊善、贊讀、直講、太常丞判太醫局宗正、大宗正秘書丞直秘閣左右郎將奉議郎大理正著作郎合門舍人宣贊舍人翰林醫官翰林醫效翰林醫痊兩赤縣令太子左右衛、司御、清道監門、內率府率七寺丞秘書郎太常博士樞密院計議、編修官敕令所刪定
 官陵臺令著作佐郎國子監丞諸王宮大小學教授國子博士大理司直、評事訓武、通直、修武郎、內常侍、少府、將作、軍器、都水監丞監尚書六部門開封府功曹倉曹戶兵曹法曹士曹參軍事、左右軍巡使、判官主管太醫局秘書省校書郎、正字親王府記室太史局五官正御史臺檢法官、主簿九寺主簿合門祗候樞密院逐房副承旨從義、秉義郎太子諸率府副率乾辦左、右廂公事忠訓、忠翊、宣教郎太學、武學、律學博士太常寺奉禮郎、
 太祝、郊社令、籍田令光祿寺太官令五監主簿宣義、成忠、保義、承事、承奉、承務郎國子、太學正武學諭國子、太學錄律學正太醫局丞京府判官京府司錄參軍承直郎京畿縣令兩赤縣丞三京赤縣令承節、承信郎節度、觀察判官節度掌書記觀察支使防禦、團練判官京府、節度、觀察推官軍事判官防禦、團練、軍事推官軍、監判官節鎮錄事參軍京府諸曹參軍事軍巡判官儒林、文林、從事郎京畿縣丞三京赤縣丞上、中、下州錄事參
 軍事三京畿縣丞。



 兩赤縣主簿、尉諸州上中下縣令、丞從政郎諸府司理、諸曹參軍事節鎮、上中下州司理、司戶、司法參軍修職郎京畿縣主簿、尉三京赤縣、畿縣主簿、尉諸州上中下縣簿、尉城砦主簿馬監主簿迪功郎諸州司士、文學、助教



 為官職雜壓之序。



 官品紹興、乾道、慶元。先後修定,間有官、勛已從罷省,而令仍不廢,今具載焉。



 諸太師,太傳,太保,左、右丞相,少師,少傳,少保,王,為正一
 品。



 諸樞密使,開府儀同三司,特進,太子太師、太傳、太保,嗣王,郡王,國公,為從一品。



 諸金紫光祿大夫,知樞密院事,參知政事,同知樞密院事,太尉,開國郡公,上柱國,為正二品。



 諸銀青光祿大夫,簽書樞密院事,觀文殿大學士,太子少師、少傳、少保,御史大夫,吏部、戶部、禮部、兵部、刑部、工部尚書,左右金吾衛、左右衛上將軍,冀、兗、青、徐、揚、荊、豫、梁、雍州牧,殿前都指揮使,節度使,開國縣公,柱國,為從二
 品。



 諸宣奉、正奉大夫,觀文殿學士,翰林、資政、保和殿大學士,翰林學士承旨,翰林學士,資政、保和、端明殿學士,龍圖、天章、寶文、顯謨、徽猷、敷文閣學士,樞密直學士,左、右散騎常侍,權六曹尚書,上護軍,為正三品。



 諸正議、通奉大夫,龍圖、天章、寶文、顯謨、徽猷、敷文閣直學士,御史中丞,開封尹,尚書列曹侍郎,諸衛上將軍,太子賓客、詹事,開國侯,護軍,為從三品。



 諸通議大夫,給事中,中書舍人,太常卿,宗正卿,秘書監,諸衛大將軍,殿前副都指揮使,承宣使,開國伯,上輕車都尉,為正四品。



 諸太中大夫,保和殿、龍圖、天章、寶文、顯謨、徽猷、敷文閣侍制,左、右諫議大夫,權六曹侍郎七寺卿,國子祭酒,少府、將作監,諸衛將軍、輕車都尉,為從四品。



 諸中大夫,馬、步軍都指揮使,副都指揮使,觀察使,通侍、正侍、宣正、履正、協忠、中侍大夫,開國子,上騎都尉,為正
 五品。



 諸中奉、中散大夫,太常、宗正少卿,秘書少監,內客省使,延福宮使,景福殿使,太子左、右庶子,樞密都承旨,中亮、中衛、翊衛、親衛大夫,殿前馬、步軍都虞候,防禦使,捧日、天武、龍神衛四廂都指揮使,團練使,諸州刺史,駙馬都尉,開國男,騎都尉,為從五品。



 諸朝議、奉直大夫,集英殿修撰,七寺少卿,中書門下省檢正諸房公事,尚書左、右司郎中,國子司業,軍器監,都
 水使者,太子少詹事、左右諭德,入內內侍省、內侍省都知副都知,宣慶、宣政、昭宣使,拱衛、左武、右武大夫,入內內侍省、內侍省押班,樞密承旨、副承旨,驍騎尉,為正六品。



 諸朝請、朝散、朝奉大夫,起居郎,起居舍人,侍御史,尚書省左、右司員外郎,樞密院檢詳諸房文字,右文殿、秘閣修撰,開封少尹,尚書諸司郎中,開封府判官、推官,少府、將作、軍器少監,和安、成和、成安大夫,陵臺令,飛騎尉,為
 從六品。



 諸朝請、朝散、朝奉郎,殿中侍御史,左、右司諫,尚書諸司員外郎,侍講,直龍圖、天章、寶文閣,開封府司錄參軍事,樞密副承旨,樞密院諸房副承旨,武功至武翼大夫,成全、平和、保安大夫,翰林良醫,太子侍讀、侍講,兩赤縣令,雲騎尉,為正七品。



 諸承議郎,左、右正言,符寶郎,監察御史,直顯謨徽猷、敷文閣,太常、宗正、秘書丞,大理正,著作郎,崇政殿說書,內
 符寶郎,正侍至右武郎,武功至武翼郎,和安至保安郎,翰林醫官,合門宣贊舍人,太子中舍人、舍人、諸率府率,親王府翊善、贊讀、直講,判太醫局令,翰林醫效、醫痊,武騎尉,為從七品。



 諸奉議、通直郎,七寺丞,秘書郎,太常博士,樞密院計議官、編修官,敕令所刪定官,直秘閣,著作佐郎,國子監丞,諸王宮大小學教授,國子博士,大理司直、評事,訓武、修武郎,內常侍,開封府諸曹參軍事、軍巡使、判官,京府判
 官,亦畿縣令,兩赤縣丞,三京赤縣、畿縣令,太史局五官正,中書、門下省錄事,尚書省都事,為正八品。



 諸宣教、宣議郎,御史臺檢法官、主簿,少府、將作、軍器、都水監丞,寺、監主簿,秘書省校書郎、正字,太常寺奉禮郎、太祝,太學、武學、律學博士,主管太醫局,合門祗候,樞密院逐房副承旨,東、西頭供奉官,從義、秉義郎,太子諸率府副率,親王府記室,節度、觀察、防禦、團練、軍事、監判官,節度掌書記,觀察支使,京府、節度、觀察、防禦、團練、軍事
 推官,諸州簽判,節鎮、上中下州錄事參軍,京府諸曹參軍事、軍巡判官,承直、儒林、文林、從事、從政、修職郎,京畿縣丞,三京赤縣、畿縣丞,諸州上中下縣令、丞,兩赤縣主簿,尉,諸府諸曹,節鎮、上州諸司參軍事,節度副使、行軍司馬,防禦、團練副使,太史局丞、直長、靈臺郎、保章正,翰林醫愈、醫證、醫診、醫候,三省樞密院主事,守闕主事、令史、書令史,為從八品。



 諸承事、承奉郎,理親民資序者,從八品,承務郎準此。



 殿頭高品,郊社、籍田、
 太官令,國子太學正、錄,武學諭,律學正,太醫局丞,忠訓、忠翊、成忠、保義郎,挈壺正,京畿縣主簿、尉,三京赤縣主簿、尉,諸州別駕、長史、司馬,樞密院守闕書令史,為正九品。



 諸承務郎,高班,黃門內品,承節、承信、迪功郎,中、下州諸司參軍,諸州上中下縣主簿、尉,城砦、馬監主簿,諸州司士、文學、助教,翰林醫學,為從九品。



\end{pinyinscope}