\article{列傳第一百九十五儒林六}

\begin{pinyinscope}

 ○陳亮鄭樵林霆附李道傳



 陳亮,字同父,婺州永康人。生而目光有芒,為人才氣超邁,喜談兵,論議風生,下筆數千言立就。嘗考古人用兵
 成敗之跡,著《酌古論》。郡守周葵得之,相與論難,奇之,曰:「他日國士也。」請為上客。及葵為執政,朝士白事,必指令揖亮,因得交一時豪俊,盡其議論。因授以《中庸》、《大學》,曰:「讀此可精性命之說。」遂受而盡心焉。



 隆興初,與金人約和,天下忻然幸得蘇息,獨亮持不可。婺州方以解頭薦,因上《中興五論》,奏入,不報。已而退修於家,學者多歸之,益力學著書者十年。



 先是,亮嘗圜視錢塘,喟然嘆曰:「城可灌爾!」蓋以地下於西湖也。至是,當淳熙五年,孝宗即
 位蓋十七年矣。亮更名同,詣闕上書曰:



 臣惟中國天地之正氣也,天命所鐘也,人心所會也,衣冠禮樂所萃也,百代帝王之所相承也。挈中國衣冠禮樂而寓之偏方,雖天命人心猶有所系,然豈以是為可久安而無事也!天地之正氣鬱遏而久不得騁,必將有所發洩,而天命人心,固非偏方所可久系也。



 國家二百年太平之基,三代之所無也;二聖北狩之痛,漢、唐之所未有也。方南渡之初,君臣上下痛心疾首,誓不與之俱生,卒能以奔敗
 之餘,而勝百戰之敵。及秦檜倡邪議以沮之,忠臣義士斥死南方,而天下之氣惰矣。三十年之餘,雖西北流寓皆抱孫長息於東南,而君父之大仇一切不復關念,自非海陵送死淮南,亦不知兵戈為何事也。況望其憤故國之恥,而相率以發一矢哉!



 丙午、丁未之變,距今尚以為遠,而海陵之禍,蓋陛下即位之前一年也。獨陛下奮不自顧,志於殄滅,而天下之人安然如無事。時方口議腹非,以陛下為喜功名而不恤後患,雖陛下亦不能以
 崇高之勢而獨勝之,隱忍以至於今,又十有七年矣。



 昔春秋時,君臣父子相戕殺之禍,舉一世皆安之。而孔子獨以為三綱既絕,則人道遂為禽獸,皇皇奔走,義不能以一朝安。然卒於無所遇,而發其志於《春秋》之書,猶能以懼亂臣賊子。今舉一世而忘君父之大仇,此豈人道所可安乎?使學者知學孔子之道,當道陛下以有為,決不沮陛下以茍安也。南師之不出,於今幾年矣,豈無一豪傑之能自奮哉?其勢必有時而發洩矣。茍國家不能
 起而承之,必將有承之者矣。不可恃衣冠禮樂之舊,祖宗積累之深,以為天命人心可以安坐而久系也。「皇天無親,惟德是輔。民心無常,惟惠之懷」。自三代聖人皆知其為甚可畏也。



 春秋之末,齊、晉、秦、楚皆衰,吳、越起於小邦,遂伯諸侯。黃池之會,孔子所甚痛也,可以明中國之無人矣。此今世儒者之所未講也。今金源之植根既久,不可以一舉而遂滅;國家之大勢未張,不可以一朝而大舉。而人情皆便於通和者,勸陛下積財養兵,以待時
 也。臣以為通和者,所以成上下之茍安,而為妄庸兩售之地,宜其為人情之所甚便也。自和好之成十有餘年,凡今日之指畫方略者,他日將用之以坐籌也;今日之擊球射雕者,他日將用之以決勝也。府庫充滿,無非財也;介胄鮮明,無非兵也。使兵端一開,則其跡敗矣。何者?人才以用而見其能否,安坐而能者不足恃也。兵食以用而見其盈虛,安坐而盈者不足恃也。而朝廷方幸一旦之無事,庸愚齷齪之人皆得以守格令、行文書,以奉
 陛下之使令,而陛下亦幸其易制而無他也。徒使度外之士擯棄而不得騁,日月蹉跎而老將至矣。臣故曰:通和者,所以成上下之茍安,而為妄庸兩售之地也。



 東晉百年之間,南北未嘗通和也,故其臣東西馳騁,多可用之才。今和好一不通,朝野之論常如敵兵之在境,惟恐其不得和也,雖陛下亦不得而不和矣。昔者金人草居野處,往來無常,能使人不知所備,而兵無日不可出也。今也城郭宮室、政教號令,一切不異於中國,點兵聚糧,
 文移往反,動涉歲月。一方有警,三邊騷動,此豈能歲出師以擾我乎?然使朝野常如敵兵之在境,乃國家之福,而英雄所用以爭天下之機也,執事者胡為速和以惰其心乎?



 晉、楚之戰於邲也,欒書以為:「楚自克庸以來,其君無日不討國人而訓之:『於!民生之不易,禍至之無日,戒懼之不可以怠。』在軍,無日不討軍實而申儆之:『於!勝之不可保,紂之百克而卒無後。』」晉、楚之弭兵於宋也,子罕以為:「兵所以威不軌而昭文德也,聖人以興,亂人以
 廢,廢興存亡昏明之術,皆兵之由也。而求去之,是以誣道蔽諸侯也。」夫人心之不可惰,兵威之不可廢,故雖成、康太平,猶有所謂四征不庭、張皇六師者,此李沆所以深不願真宗皇帝之與遼和親也。況南北角立之時,而廢兵以惰人心,使之安於忘君父之大仇,而置中國於度外,徒以便妄庸之人,則執事者之失策亦甚矣。陛下何不明大義而慨然與金絕也?



 貶損乘輿,卻御正殿,痛自克責,誓必復仇,以勵群臣,以振天下之氣,以動中原
 之心,雖未出兵,而人心不敢惰矣。東西馳騁,而人才出矣。盈虛相補,而兵食見矣。狂妄之辭不攻而自息,懦庸之夫不卻而自退縮矣。當有度外之士起,而惟陛下之所欲用矣。是雲合響應之勢,而非可安坐所致也。臣請為陛下陳國家立國之本末,而開今日大有為之略;論天下形勢之消長,而決今日大有為之機,惟陛下幸聽之。



 唐自肅、代以後,上失其柄,藩鎮自相雄長,擅其土地人民,用其甲兵財賦,官爵惟其所命,而人才亦各盡心
 於其所事,卒以成君弱臣強、正統數易之禍。藝祖皇帝一興,而四方次第平定,藩鎮拱手以趨約束,使列郡各得自達於京師。以京官權知,三年一易,財歸於漕司,而兵各歸於郡。朝廷以一紙下郡國,如臂之使指,無有留難。自筦庫微職,必命於朝廷,而天下之勢一矣。故京師嘗宿重兵以為固,而郡國亦各有禁軍,無非天子所以自守其地也。兵皆天子之兵,財皆天子之財,官皆天子之官,民皆天子之民,紀綱總攝,法令明備,郡縣不得以
 一事自專也。士以尺度而取,官以資格而進,不求度外之奇才,不慕絕世之雋功。天子蚤夜憂勤於其上,以義理廉恥嬰士大夫之心,以仁義公恕厚斯民之生,舉天下皆由於規矩準繩之中,而二百年太平之基從此而立。



 然契丹遂得以猖狂恣睢,與中國抗衡,儼然為南北兩朝,而頭目手足渾然無別。微澶淵一戰,則中國之勢浸微,根本雖厚而不可立矣。故慶歷增幣之事,富弼以為朝廷之大恥,而終身不敢自論其勞。蓋契丹徵令,是
 主上之操也;天子供貢,是臣下之禮也。契丹之所以卒勝中國者,其積有漸也。立國之初,其勢固必至此。故我祖宗常嚴廟堂而尊大臣,寬郡縣而重守令。於文法之內,未嘗折困天下之富商巨室;於格律之外,有以容獎天下之英偉奇傑,皆所以助立國之勢,而為不虞之備也。



 慶歷諸臣亦嘗憤中國之勢不振矣,而其大要,則使群臣爭進其說,更法易令,而廟堂輕矣;嚴按察之權,邀功生事,而郡縣又輕矣。豈惟於立國之勢無所助,又從
 而朘削之,雖微章得象、陳執中以排沮其事,亦安得而不自沮哉!獨其破去舊例,以不次用人,而勸農桑,務寬大,為有合於因革之宜,而其大要已非矣。此所以不能洗契丹平視中國之恥,而卒發神宗皇帝之大憤也。



 王安石以正法度之說,首合聖意,而其實則欲籍天下之兵盡歸於朝廷,別行教閱以為強也;括郡縣之利盡入於朝廷,別行封樁以為富也。青苗之政,惟恐富民之不困也;均輸之法,惟恐商賈之不折也。罪無大小,動輒興
 獄,而士大夫緘口畏罪矣。西、北兩邊致使內臣經畫,而豪傑恥於為役矣。徒使神宗皇帝見兵財之數既多,銳然南北征伐,卒乖聖意,而天下之勢實未嘗振也。彼蓋不知朝廷立國之勢,正患文為之太密,事權之太分,郡縣太輕於下而委瑣不足恃,兵財太關於上而重遲不易舉。祖宗惟用前四者以助其勢,而安石竭之不遺餘力,不知立國之本末者,真不足以謀國也。元祐、紹聖一反一復,而卒為金人侵侮之資,尚何望其振中國以威
 四裔哉?



 南渡以來,大抵遵祖宗之舊,雖微有因革增損,不足為輕重有無。如趙鼎諸臣,固已不究變通之理,況秦檜盡取而沮毀之,忍恥事仇,飾太平於一隅以為欺,其罪可勝誅哉!陛下憤王業之屈於一隅,勵志復仇,不免籍天下之兵以為強,括郡縣之利以為富。加惠百姓,而富人無五年之積;不重征稅,而大商無巨萬之藏,國勢日以困竭。臣恐尺籍之兵,府庫之財,不足以支一旦之用也。陛下蚤朝晏罷,冀中興日月之功,而以繩墨取
 人,以文法涖事;聖斷裁制中外,而大臣充位,胥吏坐行條令,而百司逃責,人才日以闒茸。臣恐程文之士,資格之官,不足當度外之用也。藝祖經畫天下之大略,太宗已不能盡用,今其遺意,豈無望於陛下也!陛下茍推原其意而行之,可以開社稷數百年之基,而況於復故物乎!不然,維持之具既窮,臣恐祖宗之積累亦不足恃也。陛下試令臣畢陳於前,則今日大有為之略必知所處矣。



 夫吳、蜀天地之偏氣,錢塘又吳之一隅。當唐之衰,錢
 鏐以閭巷之雄,起王其地,自以不能獨立,常朝事中國以為重。及我宋受命,禘盡以其家入京師,而自獻其土。故錢塘終始五代,被兵最少,而二百年之間,人物日以繁盛,遂甲於東南。及建炎、紹興之間,為岳飛所駐之地,當時論者,固已疑其不足以張形勢而事恢復矣。秦檜又從而備百司庶府,以講禮樂於其中,其風俗固已華靡,士大夫又從而治園囿臺榭,以樂其生於干戈之餘,上下晏安,而錢塘為樂國矣。一隙之地,本不足以容萬乘,
 而鎮壓且五十年,山川之氣蓋亦發洩而無餘矣。故穀粟、桑麻、絲枲之利,歲耗於一歲,禽獸、魚鱉、草木之生,日微於一日,而上下不以為異也。公卿將相,大抵多江、浙、閩、蜀之人,而人才亦日以凡下,場屋之士以十萬數,而文墨小異,已足以稱雄於其間矣。陛下據錢塘已耗之氣,用閩、浙日衰之士,而欲鼓東南習安脆弱之眾,北向以爭中原,臣是以知其難也。



 荊、襄之地,在春秋時,楚用以虎視齊、晉,而齊、晉不能屈也。及戰國之際,獨能與秦
 爭帝。其後三百餘年,而光武起於南陽,同時共事,往往多南陽故人。又二百餘年,遂為三國交據之地,諸葛亮由此起輔先主,荊楚之士從之如云,而漢氏賴以復存於蜀;周瑜、魯肅、呂蒙、陸遜、陸抗、鄧艾、羊祜皆以其地顯名。又百餘年,而晉氏南渡,荊、雍常雄於東南,而東南往往倚以為強,梁竟以此代齊。及其氣發洩無餘,而隋、唐以來,遂為偏方下州。五代之際,高氏獨常臣事諸國。本朝二百年之間,降為荒落之邦,北連許、汝,民居稀少,土
 產卑薄,人才之能通姓名於上國者,如晨星之相望。況至於建炎、紹興之際,群盜出沒於其間,而被禍尤極,以迄於今,雖南北分畫交據,往往又置於不足用,民食無所從出,而兵不可由此而進。議者或以為憂,而不知其勢之足用也。其地雖要為偏方,然未有偏方之氣五六百年而不發洩者,況其東通吳會,西連巴蜀,南極湖湘,北控關洛,左右伸縮,皆足以為進取之機。今誠能開墾其地,洗濯其人,以發洩其氣而用之,使足以接關洛之
 氣,則可以爭衡於中國矣,是亦形勢消長之常數也。



 陛下慨然移都建業,百司庶府皆從草創,軍國之儀皆從簡略,又作行宮於武昌,以示不敢寧居之意。常以江、淮之師為金人侵軼之備,而精擇一人之沈鷙有謀、開豁無他者,委以荊、襄之任,寬其文法,聽其廢置,撫摩振厲於三數年之間,則國家之勢成矣。



 石晉失盧龍一道,以成開運之禍,蓋丙午、丁未歲也。明年,藝祖皇帝始從郭太祖征伐,卒以平定天下。其後契丹以甲辰敗於澶淵,
 而丁未、戊申之間,真宗皇帝東封西祀,以告太平,蓋本朝極盛之時也。又六十年,而神宗皇帝實以丁未歲即位,國家之事於此一變矣。又六十年丙午、丁未,遂為靖康之禍。天獨啟陛下於是年,而又啟陛下以北向復仇之志。今者去丙午、丁未,近在十年間矣。天道六十年一變,陛下不可不有以應其變乎?此誠今日大有為之機,不可茍安以玩歲月也。



 臣不佞,自少有驅馳四方之志,嘗數至行都,人物如林,其論皆不足以起人意,臣是以知
 陛下大有為之志孤矣。辛卯、壬辰之間,始退而窮天地造化之初,考古今沿革之變,以推極皇帝王伯之道,而得漢、魏、晉、唐長短之由,天人之際昭昭然可考而知也。始悟今世之儒士自以為得正心誠意之學者,皆風痺不知痛癢之人也。舉一世安於君父之仇,而方低頭拱手以談性命,不知何者謂之性命乎?陛下接之而不任以事,臣於是服陛下之仁。又悟今世之才臣自以為得富國強兵之術者,皆狂惑以肆叫呼之人也。不以暇時
 謀究立國之本末,而方揚眉伸氣以論富強,不知何者謂之富強乎?陛下察之而不敢盡用,臣於是服陛下之明。陛下厲志復仇足以對天命,篤於仁愛足以結民心,而又仁明足以照臨群臣一偏之論,此百代之英主也。今乃委任庸人,籠絡小儒,以遷延大有為之歲月,臣不勝憤悱,是以忘其賤而獻其愚。陛下誠令臣畢陳於前,豈惟臣區區之願,將天地之神、祖宗之靈,實與聞之。



 書奏,孝宗赫然震動,欲榜朝堂以勵群臣,用種放故事,召令
 上殿,將擢用之。左右大臣莫知所為,惟曾覿知之,將見亮,亮恥之,逾垣而逃。覿以其不詣己,不悅。大臣尤惡其直言無諱,交沮之,乃有都堂審察之命。宰相臨以上旨,問所欲言,皆落落不少貶,又不合。



 待命十日,再詣闕上書曰:



 恭惟皇帝陛下厲志復仇,不肯即安於一隅,是有大功於社稷也。然坐錢塘浮侈之隅以圖中原,則非其地;用東南習安之眾以行進取,則非其人。財止於府庫,則不足以通天下之有無;兵止於尺籍,則不足以兼天
 下之勇怯。是以遷延之計遂行,而陛下大有為之志乖矣。此臣所以不勝忠憤,齋沐裁書,獻之闕下,願得望見顏色,陳國家立國之本末,而開大有為之略;論天下形勢之消長,而決大有為之機,務合於藝祖經畫天下之本旨。然待命八日,未有聞焉。臣恐天下豪傑有以測陛下之意向,而雲合響應之勢不得而成矣。



 又上書曰:



 臣妄意國家維持之具,至今日而窮,而藝祖皇帝經畫天下之大指,猶可恃以長久,茍推原其意而變通之,則恢
 復不足為矣。然而變通之道有三:有可以遷延數十年之策,有可以為百五六十年之計,有可以復開數百年之基。事勢昭然而效見殊絕,非陛下聰明度越百代,決不能一一以聽之。臣不敢洩之大臣之前,而大臣拱手稱旨以問,臣亦姑取其大體之可言者三事以答之。



 其一曰:二聖北狩之痛,蓋國家之大恥,而天下之公憤也。五十年之餘,雖天下之氣銷鑠頹墮,不復知仇恥之當念,正在主上與二三大臣振作其氣,以洩其憤,使人人如
 報私仇,此《春秋》書衛人殺州籲之意也。



 其二曰:國家之規模,使天下奉規矩準繩以從事,群臣救過之不給,而何暇展布四體以求濟度外之功哉!



 其三曰:藝祖皇帝用天下之士人,以易武臣之任事者,故本朝以儒立國。而儒道之振,獨優於前代。今天下之士熟爛委靡,誠可厭惡,正在主上與二三大臣反其道以教之,作其氣而養之,使臨事不至乏才,隨才皆足有用,則立國之規模不至戾藝祖之本旨,而東西馳騁以定禍亂,不必專在
 武臣也。



 臣所以為大臣論者,其略如此。



 書既上,帝欲官之,亮笑曰:「吾欲為社稷開數百年之基,寧用以博一官乎!」亟渡江而歸。日落魄醉酒,與邑之狂士飲,醉中戲為大言,言涉犯上。一士欲中亮,以其事首刑部。侍郎何澹嘗為考試官,黜亮,亮不平,語數侵澹,澹聞而嗛之,即繳狀以聞。事下大理,笞掠亮無完膚,誣服為不軌。事聞,孝宗知為亮,嘗陰遣左右廉知其事,及奏入取旨,帝曰:「秀才醉後妄言,何罪之有!」劃其牘於地,亮遂得免。



 居無何,
 亮家僮殺人於境,適被殺者嘗辱亮父次尹,其家疑事由亮。聞於官,笞榜僮,死而復蘇者數,不服。又囚亮父於州獄。而屬臺官論亮情重,下大理。時丞相淮知帝欲生亮,而辛棄疾、羅點素高亮才,援之尤力,復得不死。



 亮自以豪俠屢遭大獄,歸家益厲志讀書,所學益博。其學自孟子後惟推王通,嘗曰:「研窮義理之精微,辨析古今之同異,原心於秒忽,較禮於分寸,以積累為工,以涵養為正,睟面盎背,則於諸儒誠有愧焉。至於堂堂之陳,正正
 之旗,風雨雲雷交發而並至,龍蛇虎豹變現而出沒,推倒一世之智勇,開拓萬古之心胸,自謂差有一日之長。」亮意蓋指朱熹、呂祖謙等云。



 高宗崩,金遣使來吊,簡慢。而光宗由潛邸判臨安府,亮感孝宗之知,至金陵視形勢,復上疏曰:



 有非常之人,然後可以建非常之功。求非常之功,而用常才、出常計、舉常事以應之者,不待知者而後知其不濟也。秦檜以和誤國二十餘年,而天下之氣索然無餘矣。陛下慨然有削平宇內之志,又二十餘
 年,天下之士始知所向,其有功於宗廟社稷者,非臣區區所能誦說其萬一也。高宗皇帝春秋既高,陛下不欲大舉,驚動慈顏,抑心俯首,以致色養,聖孝之盛,書冊之所未有也。今者高宗既已祔廟,天下之英雄豪傑皆仰首以觀陛下之舉動,陛下其忍使二十年間所以作天下之氣者,一旦而復索然乎?



 天下不可以坐取也,兵不可以常勝也,驅馳運動又非年高德尊者之所宜也。東宮居曰監國,行曰撫軍,陛下何以不於此時而命東宮
 為撫軍大將軍,歲巡建業,使之兼統諸司,盡護諸將,置長史、司馬以專其勞,而陛下於宅憂之餘,運用人才,均調天下,以應無窮之變?此肅宗所以命廣平王之故事也。



 高宗與金有父兄之仇,生不能以報之,則死必有望於子孫,何忍以升遐之哀告諸仇哉!遺留、報謝,三使繼遣,金帛寶貨,千兩連發。而金人僅以一使,如臨小邦,哀祭之辭寂寥簡慢,義士仁人痛切心骨,豈以陛下之聖明智勇而能忍之乎!



 陛下倘以大義為當正,撫軍之
 言為可行,則當先經理建業而後使臨之。縱今歲未為北舉之謀,而為經理建康之計,以振動天下而與金絕,陛下之初志亦庶幾於少伸矣!陛下試一聽臣,用其喜怒哀樂之權鼓動天下。



 大略欲激孝宗恢復,而是時孝宗將內禪,不報。由是在廷交怒,以為狂怪。



 先是,鄉人會宴,末胡椒特置亮羹胾中,蓋村俚敬待異禮也。同坐者歸而暴死,疑食異味有毒,已入大理。會呂興、何念四毆呂天濟且死,恨曰:「陳上舍使殺我。」縣令王恬實其事,臺官
 諭監司選酷吏訊問,無所得,取入大理,眾意必死。少卿鄭汝諧閱其單辭,大異曰:「此天下奇材也。國家若無罪而殺士,上干天和,下傷國脈矣。」力言於光宗,遂得免。



 未幾,光宗策進士,問以禮樂刑政之要,亮以君道、師道對,且曰:「臣竊嘆陛下之於壽皇蒞政二十有八年之間,寧有一政一事之不在聖懷?而問安視寢之餘,所以察辭而觀色,因此而得彼者其端甚眾,亦既得其機要而見諸施行矣。豈徒一月四朝而以為京邑之美觀也哉!」時
 光宗不朝重華宮,群臣更進迭諫,皆不聽,得亮策,乃大喜,以為善處父子之間。奏名第三,御筆擢第一。既知為亮,則大喜曰:「朕擢果不謬。」孝宗在南內,寧宗在東宮,聞知皆喜,故賜第告詞曰:「爾蚤以藝文首賢能之書,旋以論奏動慈宸之聽。親閱大對,嘉其淵源,擢置舉首,殆天留以遺朕也。」授僉書建康府判官廳公事。未至官,一夕,卒。



 亮之既第而歸也,弟充迎拜於境,相對感泣。亮曰:「使吾他日而貴,澤首逮汝,死之日,各以命服見先人於地
 下足矣。」聞者悲傷其意。然志存經濟,重許可,人人見其肺肝。與人言,必本於君臣父子之義,雖為布衣,薦士恐弗及。家僅中產,畸人寒士衣食之,久不衰。卒之後,吏部侍郎葉適請於朝,命補一子官,非故典也。端平初,謚文毅,更與一子官。



 鄭樵,字漁仲,興化軍莆田人。好著書,不為文章,自負不下劉向、楊雄。居夾漈山,謝絕人事。久之,乃游名山大川,搜奇訪古,遇藏書家,必借留讀盡乃去。趙鼎、張浚而下
 皆器之。初為經旨,禮樂、文字、天文、地理、蟲魚、草木、方書之學,皆有論辨,紹興十九年上之,詔藏秘府。樵歸,益厲所學,從者二百餘人。



 以侍講王綸、賀允中薦,得召對,因言班固以來歷代為史之非。帝曰:「聞卿名久矣,敷陳古學,自成一家,何相見之晚耶?」授右迪功郎、禮、兵部架閣,以御史葉義問劾之,改監潭州南嶽廟,給札歸抄所著《通志》。書成,入為樞密院編修官,尋兼攝檢詳諸房文學。請修金正隆官制,比附中國秩序,因求入秘書省翻閱
 書籍。未幾,又坐言者寢其事。金人之犯邊也,樵言歲星分在宋,金主將自斃,後果然。高宗幸建康,命以《通志》進,會病卒,年五十九,學者稱夾水祭先生。



 樵好為考證倫類之學,成書雖多,大抵博學而寡要。平生甘枯淡,樂施與,獨切切於仕進,識者以是少之。



 同郡林霆,字時隱,擢政和進士第,博學深象數,與樵為金石交。林光朝嘗師事之。聚書數千卷,皆自校讎,謂子孫曰:「吾為汝曹獲良產矣。」紹興中,為敕令所刪定官,力詆秦檜和議之非,即掛
 冠去,當世高之。



 李道傳字貫之,隆州井研人。父舜臣,嘗為宗正寺主簿。道傳少莊重,稍長,讀河南程氏書,玩索義理,至忘寢食,雖處暗室,整襟危坐,肅如也。擢慶元二年進士第,調利州司戶參軍,徙蓬州教授。



 開禧用兵,金人窺散關急,道傳以諸司檄計事,道聞吳曦反,痛憤見於形色。遣其客間道持書遺安撫使楊輔,論曦必敗,曰:「彼素非雄才,犯順首亂,人心離怨,因人心而用之,可坐而縛也。誠決此
 舉,不惟內變可定,抑使金知中國有人,稍息窺覬。正使不捷,亦無愧千古矣。」曦黨以曦意脅道傳,道傳以義折之,竟棄官歸。曦平,詔以道傳抗節不撓,進官二等。



 嘉定初,召為太學博士,遷太常博士兼沂王府小學教授。會沂府有母喪,遺表官吏例進秩,道傳曰:「有襄事之勞者,推恩可也,吾屬何與?」於是皆辭不受。遷秘書郎、著作佐郎,見帝,首言:「憂危之言不聞於朝廷,非治世之象。今民力未裕,民心未固,財用未阜,儲蓄未豐,邊備未修,將帥
 未擇,風俗未能知義而不偷,人才未能匯進而不乏。而八者之中,復以人才為要。至於人才盛衰,系學術之明晦,今學禁雖除,而未嘗明示天下以除之之意。願下明詔,崇尚正學,取朱熹《論語》、《孟子集注》、《中庸大學章句》、《或問》四書,頒之太學,仍請以周惇頤、邵雍、程顥、程頤、張載五人從祀孔子廟。」時執政有不樂道學者,以語侵道傳,道傳不為動。兼權考功郎官,遷著作郎。



 時薛拯、胡矩等皆以新進用事,賄賂成風,道傳言:「今名優儒臣,實取材
 吏,刻剝殘忍、誕謾傾危之人進矣。」遂求補郡,於是出知真州。城圮弗治,道傳甓之,築兩石壩以護並江居民,益浚二壕,又堤陳公塘,有警,則決之以為阻,人心始固。除提舉江東路常平茶鹽公事。初至,即按部劾吏之貪縱者十餘人,胥吏為民害者,大黥小逐百餘人,釋獄之濫系者二百餘人,弛負錢一十餘萬緡。夏大旱,道傳應詔言楮幣之換,官民如仇;鈔法之行,商賈疑怨;賦斂增加,軍將推剝,皆切中時病。遂條上荒政,朝廷多從之。與漕
 臣真德秀振饑,道傳分池、宣、徽三州,窮冬行風雪中,雖深村窮穀必至,賴以全活者甚眾。攝宣州守,行朱熹社倉法,上饒、新安、南康諸郡翕然應命,人蒙其利。



 廣德守魏峴劾教官林庠委堂試而任荒政,挾漕臣以凌郡守,且言真德秀輕視朝廷,自專掠美,乞遠之。道傳上疏力辨,峴坐免。會胡矩為吏部侍郎,薦道傳自代。引疾乞去,不許。召令奏事,再辭,又不許,遂入對。上自宮掖,次及朝廷,以至侍從、臺諫闕失,盡言無所諱,帝不以為忤。除兵部
 郎官,辭未就。監察御史李楠覘當路指意,乞授以節鎮蜀,遂出知果州。至九江,得疾卒,年四十八,詔特轉一官致仕,謚文節。



 道傳自蜀來東南,雖不及登朱熹之門,而訪求所嘗從學者與講習,盡得遺書讀之。篤於踐履,氣節卓然。於經史未有論著,曰:「學未至,不敢。」於詩文未嘗茍作,曰:「學未至,不暇。」一日以疾謁告,真德秀造焉,臥榻屏間,大書「喚起截斷」四字,知其用功慎獨如此。居官以惠利為本,振荒遺愛江東,人久而思焉。



 三子:達可、當可、
 獻可。獻可為心傳後。



\end{pinyinscope}