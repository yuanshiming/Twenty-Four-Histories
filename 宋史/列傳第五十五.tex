\article{列傳第五十五}

\begin{pinyinscope}

 韓丕師頏張茂直梁顥子固楊徽之楊澈呂文仲王著呂祐之潘慎修杜鎬查道從兄陶



 韓丕,字太簡,華州鄭人。父杲,晉開運中,為曲陽主簿,契
 丹攻城,陷沒焉。母改適他氏。丕幼孤貧,有志操,讀書於驪山、嵩陽,通《周易》、《禮記》,為人講說。常有山林之志,家雖甚貧,處之晏如。年長,始學文。開寶中,鄭牧知文州,與之偕行,遂薄游兩川。及牧知成都,劉熙古延置門下,掌書奏,以孫女妻之。



 太平興國三年舉進士,聲名籍甚,公卿多薦之者。嘗著《孟母碑》、《返魯頌》,人多諷誦之。解褐大理評事、通判衡州。石熙載薦其文行,代還,以文學試中書,擢著作佐郎、直史館,賜緋魚。未幾,改左拾遺。八年,遷職
 方員外郎、知制誥。雍熙初,加虞部郎中。二年,與賈黃中、徐鉉同知貢舉。丕屬思艱澀,及典書命,傷於稽緩。宰相宋琪性褊急,常加督責,或申以諧謔,丕不能平。又舍人王祐以前輩負氣,每陵轢面折之。丕乃表求外郡,出知虢州,就改職方郎中。端拱初,拜右諫議大夫,賜金紫,知河陽、濠州。



 丕起寒素,以沖澹自處,不奔競於名宦,太宗甚嘉重之。淳化二年,召入為翰林學士,終以遲鈍不敏於用。俄罷職,充集賢殿修撰、知均州。就遷給事中、工部
 侍郎,徙金州。召還,充史館修撰,又出知滁州,就加禮部。大中祥符二年,卒。



 丕純厚畏慎,似不能言者。歷典州郡,雖不優於吏事,能以清介自持,時稱其長者云。



 師頏,字霄遠,大名內黃人。父均,後唐長興二年進士,終永興節度判官,因家關右。頏少篤學,與兄頌齊名。建隆二年舉進士,竇儀典貢舉,擢之上第。釋褐耀州軍事推官,以疾解,久不赴調。開寶中,復為解州推官。太平興國初,召還,遷大理寺丞、陜西河北轉運判官,就改著作佐
 郎。秩滿,遷監察御史、通判永興軍府。坐秦王廷美假公帑緡錢,左授乾州團練副使,尋復舊官。六年,改殿中侍御史、通判邠州。徙知簡州,轉起居舍人。以公累去官,復為殿中侍御史,知資、眉二州。頏所至,以簡靜為治,蜀人便之。代還,遷侍御史、知安州,賜緡錢二十萬。移朗州,超拜工部郎中,命知陜州,賜金紫。



 時西鄙用兵,餫道所出,軍士多亡命,嘯聚山林為盜。頏嚴其巡捕,盜越他境。改刑部郎中,未幾召還。真宗以其舊人,素負才望,而久次
 於外,累召對,詢其文章。頏謙遜自晦,上益嘉之。翌日,命以本官知制誥,兼史館修撰。咸平二年,與溫仲舒、張詠同知貢舉。明年,召入翰林為學士。五年,復與陳恕同典貢部,又知審官院、通進銀臺封駁司。俄卒,年六十七。詔遣官護葬,給其子仲回秘書丞奉終喪。



 頏曠達夷雅,搢紳多慕其操尚。有集十卷。子三人:仲回,端拱元年進士及第,至太常博士;仲宰,國子博士;仲說,殿中丞。



 張茂直,字林宗,兗州瑕丘人。父延升,以經術教授鄉里。
 茂直方弱冠,慕容彥超據州城,驅之守陴。及周師破敵,擁城守者列坐,將斬之。有卒挾刃謂茂直曰:「汝發甚鬒,惜為頸血所污,可先斷之。」茂直許焉。刃未及發,會得釋。後勵志於學。



 開寶中,州將器其為人,首薦之,且給錢五萬,以助其裝。二年,登進士第,解褐海州推官,進司農寺丞、通判泰州。為轉運使韋務升誣奏,徙監梓州富國監。代還,自陳得雪。復通判靜安軍。軍不領縣,城闉之外,即深州之下博,茂直奏割下博隸焉。進秩著作佐郎。扈蒙
 薦其才,改秘書丞。



 會福州民訟田,命茂直按之,將行,留不遣。參知政事李至稱其端實,命入益王元傑府為記室參軍。王好學,多為詩什,遇茂直甚厚。雖受時果之賜,亦分餉焉。王嘗遣使征詩,茂直援筆而就,甚稱賞之。



 端拱元年,召對,賜金紫。數日,改度支員外郎,三遷本曹郎中。真宗居藩時,茂直與朱昂並在諸王府,每預宴集,屢因酬唱識其名。即位,選用舊臣,得茂直及昂,與梁周翰、師頏輩相繼知制誥。茂直既入西閣,會元傑生旦,遣持
 禮幣為賜,復至舊府,時人榮之。



 茂直淳至寡言,晚年多疾,才思梗澀不稱職。改秘書少監,出知穎州。咸平四年,卒,年七十五。子成列,端拱二年進士及第;成務,比部員外郎。



 梁顥,字太素,鄆州須城人。曾祖涓,成武主簿。祖惟忠,以明經歷佐使府,至天平軍節度判官。父文度早世,顥養於叔父。王禹偁始與鄉貢,顥依以為學,嘗以疑義質於禹偁,禹偁拒之不答。顥發憤讀書,不期月,復有所質,禹
 偁大加器賞。初舉進士,不中第,留闕下。獻疏曰:



 臣歷觀史籍,唐氏之御天下也,列聖間出,人文闡耀,尚且渴於共治,旁求多彥,設科之選,逾四十等。當時秉筆之士,彬彬翔集,表著所以。左右前後,有忠有良,導化原、樹治本者,享三百年,得人之由也。



 五代不競,茲制日淪。國家興儒,追風三代。方今科名之設,俊造畢臻,秉筆者如林,趨選者如云。貢於諸侯,考於春官,陛下躬臨慎擇,必盡至公。奈何所取不出於詩賦、策論,簡於心者援而陟之,咈
 於心者推而黜之,寧無濫陟枉黜之失耶?其間闒茸妄進,濫廁科場者,間亦有之。



 若曰陛下嘉惠孤寒沉滯之士,罔計賢否,悉拔而登之,一視同仁。臣竊謂此非確論。蓋聖人在上,則內君子而外小人。若熏蕕同器,甚非所以正人倫、淳風俗也。況丘園之下,豈無宏才茂德之士。陛下誠能設科以擢異等之士,俾陳古今之治亂、君臣之得失、生民之休戚、賢愚之用舍,庶幾有益於治,不特詩賦、論策之小技,以應有司之求而已。



 疏上,不報。



 雍熙
 二年,復舉進士,廷試,方禹中獻賦。太宗召升殿,詢其門第,賜甲科,解褐大名府觀察推官。四年,與梁湛並召為右拾遺、直史館,賜緋。判鼓司、登聞院。顥在大名佐趙昌言。昌言人掌樞密,會翟馬周事,顥坐貶虢州司戶參軍。起知魚臺縣,就加大理評事。召還,遷殿中丞。頃之,復直史館,壓開封府推官、三司關西道判官,轉太常博士,丁內艱,起令赴職,改右司諫。



 真宗初,詔群臣言事,顥時使陜西,途中作《聽政箴》以獻。還為度支判官。咸平元年,與
 楊勵、李若拙、朱臺符同知貢舉。時詔錢若水重修《太祖實錄》,表顥參其事,又同修起居注。扈蹕大名,詔訪群臣邊事,顥上疏曰:



 臣聞自古用兵之道,在乎明賞罰而已。然而賞不可以獨任,罰不可以少失。故《兵法》曰:「罰之不行,譬如驕子之不可用。」又曰:「善為將者,威振敵國,令行三軍。盡忠益時者,雖仇必賞;犯法敗事者,雖親必罰。」故孫武斬隊長而兵皆整,穰苴斬監軍而敵遂退。以此言之,兵法不可不正也。



 昨者命將出師,乘秋備塞,而傅潛
 奉明詔,握重兵,逗撓無謀,守陴玩寇,老精兵於不用。以至蕃馬南牧,邊塵畫驚,河朔之民,流移失所,魏博以北,蹂踐一空。遂至殘妖未殄,鑾輅親征,此所謂以賊遺君父者也。乃或赦而不問,則何以謝橫死之民;或黜而不戮,則何以恢用兵之略。以軍法論之,固合斬潛以徇軍中,降詔以示天下。如此,則協前古之典章,戒後來之將帥,然後擇邊臣之可用者,就委用之。



 臣嘗讀漢史,李廣之屯兵行師也,無部伍行陣,就善水草,人人自便,不擊
 刁斗以自衛,遠於斥候,未嘗遇害,而廣終為名將,士卒樂用。又唐高祖之備北邊也,選頸兵為游騎,不繼軍糧,隨逐水草,遇敵則殺,當時以為得策。願於邊將中,不以名位高卑,但擇其武勇謀略素為眾所推服者,取十人焉。人付騎士五十,器甲完備,輕繼糧糗,逐水草以為利,往復捍禦。不令入郡邑,不許聚處,遇有寇兵,隨時掩捕。仍令烽候相望,交相救應。緣邊州郡守城兵帥,即堅壁以待之。遇游騎近城,掩殺邊寇,內量出兵甲援救。如此,
 則乘城者不堅閉壘門,免坐觀於勝負;捍邊者不茍依郡郭,可行備於寇攘。雖匪良籌,且殊膠柱。



 時論頗稱之。



 三年,與李宗諤、趙安仁並命知制誥,賜金紫,是年冬,王均平,命為峽路安撫使。歸掌三班。韓國華判大理,以斷刑失中,乃選顥以代之。四年,張齊賢使關右安撫,以顥為之副。



 顥有吏才,每進對,詞辯明敏,真宗嘉賞之。凡群臣上封者,悉付顥洎薛映詳閱可否。冬,以河北饑盜,命與映分為東、西路巡檢使。還,拜右諫議大夫,充戶部使。
 會罷三部使,以顥為翰林學士同知審官院、三班。景德元年,權知開封。



 顥美風姿,強力少疾,閨門雍睦。與人交久而無改,士大夫多之。六月,暴病卒,年九十二。上甚軫惻,賜贈加等。所著文集十五卷。子固、述、適。適相仁宗,別有傳。



 固字仲堅。幼有志節,嘗著《漢春秋》,顥器賞之。初,以顥遺蔭,賜進士出身。服闋,詣登聞院讓前命,願赴鄉舉,許之。大中祥符元年,舉服勤詞學科,擢甲第。解褐將作監丞、
 同判密州,就遷著作佐郎。歸朝,改著作郎、直史館,賜緋。歷戶部判官、判戶部勾院。



 為人氣調俊爽,善與人交,疏財慷慨,尚氣義,明於吏道。馬元方領三司,臨事粗率,固摭其曠闕之狀,屢請對條奏。嘗詔鞠獄,時稱平審。天禧大禮成,奏頌甚工。無幾卒,年三十三。有集十卷。



 楊徽之,字仲猷,建州浦城人。祖郜,仕閩為義軍校。家世尚武,父澄獨折節為儒,終浦城令。徽之幼刻苦為學,邑人江文蔚善賦,江為能詩,徽之與之游從,遂與齊名。嘗
 肄業於潯陽廬山,時李氏據有江表,乃潛服至汴、洛,以文投竇儀、王樸,深賞遇之。



 周顯德中,舉進士,劉溫叟知貢部,中甲科。同時登第者十六人,世宗命覆試,惟徽之與李覃、何□嚴、趙鄰幾中選。解褐校書郎、集賢校理。宰相範質深器重之。歷著作佐郎、右拾遺。竇儼纂禮樂書,徽之預焉。



 乾德初,與鄭□並出為天興令,府帥王彥超素知其名,待以賓禮。蜀平,移峨眉令。時宋白宰玉津,多以吟詠酬答。復為著作佐郎、知全州,就遷左拾遺、右補闕。
 太平興國初,代還。太宗素聞其詩名,因索所著。徽之以數百篇奏御,且獻詩為謝,其卒章有「十年流落今何幸,叨遇君王問姓名」語。太宗覽之稱賞,自是聖制多以別本為賜。遷侍御史、權判刑部。嘗屬疾,遣尚醫診療,賜錢三十萬。轉庫部員外郎,賜金紫,判南曹,同知京朝官差遣。會詔李昉等採緝前代文字,類為《文苑英華》,以徽之精於風雅,分命編詩,為百八十卷。歷遷刑、兵二部郎中。獻《雍熙詞》,上賡其韻以賜。



 端拱初,拜左諫議大夫,出知
 許州。入判史館事,加修撰。因次對上言,曰:「自陛下嗣統鴻圖,闡揚文治,廢墜修舉,儒學響臻,乃至周巖野以聘隱淪,盛科選以來才彥,取士之道,亦已至矣。然擅文章者多超遷,明經業者罕殊用,向非振舉,曷勸專勤,師法不傳,祖述安在!且京師四方之會,太學首善之地。今五經博士,並闕其員,非所以崇教化、獎人材、繇內及外之道也。伏望浚發明詔,博求通經之士,簡之朝著,拔自草萊,增置員數,分教冑子,隨其所業,授以本官,廩稍且優,
 旌別斯在。淹貫之士,既蒙厚賞,則天下善類知所勸矣,無使唐、漢專稱得人。」太宗嘉納之,顧謂宰相曰:「徽之儒雅,操履無玷,置於館閣宜矣。」未幾,改判集賢院。嘗詔預觀燈乾元樓,上嘉其精力不衰。



 時劉昌言拔自下位,不逾時參掌機務,懼無以厭人望,常求自安之計。董儼為右計使,欲傾昌言代之,嘗謂徽之曰:「上遇張洎、錢若水甚厚,旦夕將大用。」有直史館錢熙者,與昌言厚善,詣徽之,徽之語次及之。熙遽以告昌言,昌言以告洎。洎方固
 寵,謂徽之遣熙構飛語中傷己,遂白上。上怒,召昌言質其語。出徽之為山南東道行軍司馬,熙落職通判朗州。徽之未行,改鎮安軍行軍司馬。



 真宗尹京,妙選僚佐,驛召為左諫議大夫,與畢士安並充開封府判官,召對便殿,諭以輔導意。東宮建屬,以徽之兼左庶子。嘗出巡田,真宗作詩言懷,因以寄之。遷給事中。即位,拜工部侍郎、樞密直學士,俄兼秘書監。咸平初,加禮部侍郎。二年春,以衰疾求解近職,改兵部,仍兼秘書監,入謝,命坐,勞之
 曰:「圖書之府,清凈無事,俾卿得以養性也。」是秋,特置翰林侍讀學士,命與夏侯嶠、呂文仲並為之,賜宴秘閣,且褒以詩。



 未幾,以足疾請告,上取名藥以賜。郊祀不及扈從,錫繼如侍祠之例。車駕北巡,徽之力疾辭於苑中。上顧謂曰:「卿勉進醫藥,比見,當不久也。」及駐蹕大名,特降手詔存諭。明年春正月,車駕還,又遣使臨問。卒,年八十。贈兵部尚書,賜其家錢五十萬,絹五百匹。錄其外孫宋綬太常寺太祝,侄孫偃、集並同學究出身。



 徽之純厚清
 介,守規矩,尚名教,尤疾非道以干進者。嘗言:「溫仲舒、寇準用搏擊取貴位,使後輩務習趨競,禮俗浸薄。」世謂其知言。徽之寡諧於俗,唯李昉、王祐深所推服,與石熙載、李穆、賈黃中為文義友。自為郎官、御史,朝廷即待以舊德。善談論,多識典故,唐室以來士族人物,悉能詳記。酷好吟詠,每對客論詩,終日忘倦。既沒,有集二十卷留於家,上令夏侯嶠取之以進。徽之無子。後徽之妻王卒,及葬,復以緡帛賜其家。



 澈字晏如,徽之宗人也,世家建陽。父思進,晉天福中北渡海,因家於青州之北海,累佐使幕。澈幼聰警,七歲讀《春秋左氏傳》,即曉大義。周宰相李谷召令默誦,一無遺誤,穀甚異之。年十六,思進為鎮趙從事,會昭慶令缺,使府命澈假其任。時河決鄰郡,府督役甚急。澈部徒數千,徑大澤中,多蘆葦,令採刈為筏,順流而下。既至,執事者訝以後期,俄而葦筏繼至,駭而問之,澈以狀對,乃更嗟賞。



 建隆初,舉進士,時竇儀典貢部,謂澈文詞敏速,可當
 書檄之任。調補河內主簿,再遷青州司戶參軍。知州張全操多不法,澈鞫獄平允,無所阿畏。太祖知其名,召試禁中,改著作佐郎,出知渠州。江南平,改通判虔州,令就大將曹彬分兵以行。既入境,偽帥郭再興擁兵自固,澈單騎直趨其壘,諭以朝廷威信,再興即奉符以代。澈悉料城中軍士之勇壯者,凡五百人為一綱,部送京師。土豪黎、羅二姓,聚眾依山謀亂,澈率兵平之,擒二豪,械送闕下。



 遷右贊善大夫、知淄州。事親以孝聞,求便侍養,徙
 同判青州。三遷祠部員外郎,復知淄州,又知舒州,累轉祠部郎中。咸平初,遴選王府僚佐,以澈為雍王府記室參軍,賜金紫,加度支郎中。



 景德初,車駕幸澶淵,王為東京留守,澈遷兵部郎中,充留守判官。軍巡囚逸,王驚而感疾,及薨,又得閨門殘忍之狀,坐輔導不善免官。未幾,起為祠部郎中。卒,年七十四。子巒,淳化進士,職方員外郎。



 呂文仲,字子臧,歙州新安人。父裕,偽唐歙州錄事參軍。
 文仲在江左,舉進士,調補臨川尉,再遷大理評事,掌宗室書奏。入朝,授太常寺太祝,稍遷少府監丞。預修太平《御覽》、《廣記》、《文苑英華》,改著作佐郎。太平興國中,上每御便殿觀古碑刻,輒召文仲與舒雅、杜鎬、吳淑讀之。嘗令文仲讀《文選》,繼又令讀《江海賦》,皆有賜繼。以本官充翰林侍讀,寓直御書院,與侍書王著更宿。時書學葛湍亦直禁中,太宗暇日,每從容問文仲以書史、著以筆法、湍以字學。雍熙初,文仲遷著作佐郎,副王著使高麗。復命
 改左正言,巡撫福建。未幾,賜金紫,加左諫議大夫。



 淳化中,與陳堯叟並兼關西巡撫使。時內品方保吉專幹榷酤,威制郡縣。民疲吏擾,變易舊法,訟其掊克者甚眾。文仲等具奏其實,太宗怒甚。亟召保吉,將劾之,反為保吉所訟,下御史驗問。文仲所坐皆細事,而素巽懦,且恥與保吉辨對,因自誣伏,遂罷職。既而太宗知其由,復令直秘閣;逾月,再為侍讀。一日,召於崇政殿,讀上草書經史故實數十軸,詔模刻於石。遷起居舍人、兵部員外郎、同
 判吏部銓,知銀臺通進封駁司、審官院。咸平三年,拜工部郎中,充翰林侍讀學士,受詔集太宗歌詩為三十卷,詔書加獎,又知審刑院。六年,授御史中丞。



 景德中,鞫曹州奸民趙諫獄。諫多與士大夫交游,內出姓名七十餘人,令悉窮治。文仲請對,言逮捕者眾,或在外郡,茍悉索之,慮動人聽。上曰:「卿執憲,當嫉惡如仇,豈公行黨庇邪?」文仲頓首曰:「中司之職,非徒繩糾愆違,亦當顧國家大體。今縱七十人悉得奸狀,以陛下之慈仁,必不盡戮,不
 過廢棄而已。但籍其名,更察其為人,置於冗散,或舉選對揚之日擯斥之,未為晚也。」上從其言。三年,遷工部侍郎,復為翰林侍讀學士。



 文仲久居禁近,頗周密兢慎。一日早朝,暴得風疾,請告逾百日,詔續其奉。明年,改刑部侍郎,充集賢院學士,未幾卒,錄其子永為奉禮郎。



 文仲富詞學,器韻淹雅。其使高麗也,善於應對,清凈無所求,遠俗悅之。後有使高麗者,必詢其出處。然性頗齷齪,不為時論所許。有集十卷。



 王著,字知微,文仲同時人。自言唐相石泉公方慶之後,世家京兆渭南。祖賁,廣明中眾僖宗入蜀,遂為成都人。賁仕王建,為雅州刺史。父景環,萬州別駕。著,偽蜀明經及第,歷平泉、百丈、永康主簿。蜀平赴闕,授隆平主簿,凡十一年不代。著善攻書,筆跡甚媚,頗有家法。太宗以字書訛舛,欲令學士刪定,少通習者。太平興國三年,轉運使侯陟以著名聞,改衛寺丞、史館祗候,委以詳定篇韻。六年,召見,賜緋,加著作佐郎、翰林侍書與侍讀,更直於
 御書院。



 太宗聽政之暇,嘗以觀書及筆法為意,諸家字體,洞臻精妙。嘗令中使王仁睿持御札示著,著曰:「未盡善也。」太宗臨學益勤,又以示著,著答如前。仁睿詰其故,著曰:「帝王始攻書,或驟稱善,則不復留心矣。」久之,復以示著。著曰:「功已至矣,非臣所能及。」其後真宗嘗對宰相語其事,且嘉著之善於規益,於侍書待詔中亦無其比。



 雍熙二年,遷左拾遺,使高麗。端拱初,加殿中侍御史。二年,與文仲同賜金紫。明年,卒,特加賵賜,錄其子嗣復為
 奉禮郎。



 呂祐之,字符吉,濟州鉅野人。父文贊,本州錄事參軍。祐之,太平興國初,舉進士,解褐大理評事、通判洋州。改右贊善大夫,出為泰寧軍節度判官,移天雄軍。召拜殿中侍御史,決獄西蜀。還知貝州,換右補闕、直史館、同判吏部南曹,遷起居舍人。



 端拱中,副呂端使高麗,假內庫錢五十萬以辦裝。還,遇風濤,舟欲覆,祐之悉取所得貨沉之,即止。復獻《海外覃皇澤詩》十九首,太宗嘉之,仍蠲
 其所貸。淳化初,判戶部勾院,會分備三館職,以祐之與趙昂、安德裕並直昭文館。俄以本官知制誥,賜金紫,同知貢舉。



 有東野日宣者,祐之以妻族嘗薦舉之,坐鞫獄陳州不實,貶官,祐之亦降授殿中丞,再直史館。未幾,復知制誥。太宗嘗閱班簿,擇近臣舉官,睹祐之姓名,宰相因言其前坐舉無狀。上曰:「此正可令贖過矣。」即取祐之焉。



 至道初,拜右諫議大夫,賜金紫,知審官院。出知襄州,徙壽州。真宗即位,轉給事中,復知襄州,移升州。歲餘,又典
 襄陽。歸,掌吏部選事,知通進、銀臺司,與呂文仲並拜工部侍郎、翰林侍讀學士。自置侍讀、侍講,甚艱其選,至是裁七人。祐之第其名氏,刻石於秘閣。



 祐之純謹長者,不喜趨競,所至無顯譽,備顧問,不能有所啟發。會文仲以疾罷近職,祐之亦出為集賢院學士,仍並遷刑部侍郎。景德四年,卒,年六十一。有集三十卷。



 潘慎修,字成德,泉州莆田縣人。父承祐,仕閩,後歸江南,仕李景,至刑部尚書致仕。慎修少以父任為秘書省正
 字,累遷至水部郎中兼起居舍人。



 開寶末,王師征江南,李煜遣隨其弟從鎰入貢買宴錢,求緩兵。留館懷信驛。旦夕捷書至,邸吏督從鎰入賀。慎修以為國且亡,當待罪,何賀也?自是每群臣稱賀,從鎰即奉表請罪。太祖嘉其得禮,遣呂使慰諭,供帳牢餼悉加優給。煜歸朝,以慎修為太子右贊善大夫。煜表求慎修掌記室,許之。煜卒,改太常博士。歷膳部、倉部、考功三員外,通判壽州,知開封縣,又知湖、梓二州。



 淳化中,秘書監李至薦之,命以本
 官知直秘閣。慎修善弈棋,太宗屢召對弈,因作《棋說》以獻。大抵謂:「棋之道在乎恬默,而取舍為急。仁則能全,義則能守,禮則能變,智則能兼,信則能克。君子知斯五者,庶幾可以言棋矣。」因舉十要以明其義,太宗覽而稱善。俄與直昭文館韓援使淮南巡撫,累遷倉部、考功二部郎中。咸平中,又副邢昺為兩浙巡撫使,俄同修起居注。景德初,上言衰老,求外任。真宗以儒雅宜留秘府,止聽解記注之職。數月,擢為右諫議大夫、翰林侍讀學士。從
 幸澶州,遘寒疾,詔令肩輿先歸。明年正月,卒,年六十九。賻錢二十萬,絹一百匹。



 慎修疾雖亟,精爽不亂,托陳彭年草遺奏,不為諸子干澤,但以主恩未報為恨。上憫之,錄其子汝士為大理評事,汝礪為奉禮郎。令有司給舟載其柩歸洪州。



 慎修風度醞藉,博涉文史,多讀道書,善清談。先是,江南舊臣多言李煜暗懦,事多過實。真宗一日以問慎修,對曰:「煜或懵理若此,何以享國十餘年?」他日,對宰相語及之,且言慎修溫雅不忘本,得臣子之操,深
 嘉獎之。當時士大夫與之游者,咸推其素尚。然頗恃前輩,待後進倨慢,人以此少之。有集五卷。



 汝士至工部員外郎,直集賢院。



 杜鎬,字文周,常州無錫人。父昌業,南唐虞部員外郎。鎬幼好學,博貫經史。兄為法官,嘗有子毀父畫像,為旁親所訟,疑其法不能決。鎬曰:「僧道毀天尊、佛像,可比也。」兄甚奇之。舉明經,解褐集賢校理,入直澄心堂。



 江南平,授千乘縣主簿。太宗即位,江左舊儒多薦其能,改國子監
 丞、崇文院檢討。會將祀南郊,彗星見,宰相趙普召鎬問之。鎬曰:「當祭而日食,猶廢;況謫見如此乎?」普言於上,即罷其禮。翌日,遷著作佐郎,改太子左贊善大夫,賜緋魚。歷殿中丞、國子博士,加秘閣校理。太宗觀書秘閣,詢鎬經義,進對稱旨,即日改虞部員外郎,加賜金帛。又問:「西漢賜與悉用黃金,而近代為難得之貨,何也?」鎬曰:「當是時,佛事未興,故金價甚賤。」又嘗召問天寶梨園事,敷奏詳悉。再遷駕部員外郎,判太常禮院,與朱昂、劉承珪編
 次館閣書籍,虞部郎中,事畢,賜金紫,改直秘閣。會修《太祖實錄》,命鎬檢討故事,以備訪問。



 景德初,置龍圖閣待制,因以命錫鎬,加都官郎中。從幸澶淵,遇懿德皇后忌日,疑軍中鼓吹之禮,時鎬先還備儀仗,命馳騎問之。鎬以武王載木主伐紂,前歌後舞為對。預修《冊府元龜》,改司封郎中。四年,拜右諫議大夫、龍圖閣直學士,賜襲衣、金帶,班在樞密直學士下。時特置此職,儒者榮之。



 大中祥符中,同詳定東封儀注,遷給事中。三年,又置本閣學
 士,遷鎬工部侍郎,充其職。上日,賜宴秘閣,上作詩賜之,進秩禮部侍郎。六年冬,卒,年七十六。錄其子渥為大理寺丞及三孫官。



 鎬博聞強記,凡所檢閱,必戒書吏云:「某事,某書在某卷、幾行。」覆之,一無差誤。每得異書,多召問之,鎬必手疏本末以聞,顧遇甚厚。士大夫有所著撰,多訪以古事,雖晚輩、卑品請益,應答無倦。年逾五十,猶日治經史數十卷,或寓直館中,四鼓則起誦《春秋》。所居僻陋,僅庇風雨,處之二十載,不遷徙。燕居暇日,多挈醪饌
 以待賓友。性和易,清素有懿行,士類推重之。



 查道,字湛然,歙州休寧人。祖文徽,仕南唐至工部尚書。父元方,亦仕李煜,為建州觀察判官。王師平金陵,盧絳據歙州,遣使傳檄至郡,元方斬其使。及絳擒,太祖聞元方所為,優獎之。拜殿中侍御史、知泉州,卒。



 道幼沉嶷不群,罕言笑,喜親筆硯,文徽特愛之。未冠,以詞業稱。侍母渡江,奉養以孝聞。母嘗病,思鱖羹,方冬苦寒,市之不獲。道泣禱於河,鑿冰取之,得鱖尺許以饋。又刲臂血寫佛
 經,母疾尋愈。後數年,母卒,絕意名宦,游五臺,將落發為僧。一夕,震雷破柱,道坐其下,了無怖色,寺僧異之,咸勸以仕。



 端拱初,舉進士高第,解褐館陶尉。曹彬鎮徐州,闢為從事,深被禮遇。改興元觀察推官。寇準薦其才,授著作佐郎。淳化中,蜀寇叛,命道通判遂州。召對,出御書歷,俾錄其課,給以實奉。至道三年,有使兩川者,得道公正清潔之狀以聞,優詔嘉獎。遷秘書丞,俄徙知果州。



 時寇黨尚有伏巖谷依險為柵者,其酋何彥忠集其徒二百
 餘,止西充之大木槽,彀弓露刃。詔書招諭之,未下,咸請發兵殄之。道曰:「彼遇人也,以懼罪,欲延命須臾爾。其黨豈無詿誤邪?」遂微服單馬數僕,不持尺刃,間關林壑百里許,直趨賊所。初悉驚畏,持滿外向。道神色自若,踞胡床而坐,諭以詔意。或識之曰:「郡守也,嘗聞其仁,是寧害我者。」即相率投順羅拜,號呼請罪,悉給券歸農。加賜袍帶驛奏,璽書褒諭。



 咸平四年代歸,賜緋魚。上言曰:「朝廷命轉運使、副,不惟審度金谷,蓋以察廉郡縣,庶臻治平,
 以召和氣。今觀所至,或匪盡公,蓋無懲勸之科,致有因循之弊。望自今每使回日,先令具任內曾薦舉才識者若干,奏絀貪猥者若干,朝廷議其否臧,以為賞罰。」從之。俄出知寧州。會舉賢良方正之士,李宗諤以道名聞,策入第四等,拜左正言、直史館。未幾,出為西京轉運副使。六年,始令三司使分部置副,召入,拜工部員外郎、充度支副使,賜金紫。



 道儒雅迂緩,治劇非所長。卞哀為鹽鐵副使,與道同候對,將升殿,遽出奏牘請道同署。及上詢
 問事本,道素未省視,不能對,遂以本官罷,出知襄州。卒不能自辯,亦無慍色。



 大中祥符元年,歸直史館,遷刑部員外郎,預修《冊府元龜》。三年,進秩兵部,為龍圖閣待制,與張知白、孫奭、王曙並命焉。加刑部郎中、判吏部選事,糾察在京刑獄。奉使契丹,以久次,進右司郎中。真宗退朝之暇,召馮元講《易》便坐,惟道與李虛己、李行簡預焉。



 天禧元年,以耳聵難於對問,表求外任,得知虢州。將行,上御龍圖閣飲餞之。秋,蝗災民歉,道不候報,出官廩米
 賑之,又設粥糜以救饑者,給州麥四千斛為種於民,民賴以濟,所全活萬餘人。二年五月,卒,訃聞,真宗軫惜之。詔其子奉禮郎循之乘傳往治喪事,遷大理評事,賦祿終制。



 道性淳厚,有犯不較,所至務寬恕,胥吏有過未嘗笞罰,民訟逋負者,或出己錢償之,以是頗不治。嘗出按部,路側有佳棗,從者摘以獻,道即計直掛錢於樹布去。兒時嘗戲畫地為大第,曰:「此當分贍孤遺。」及居京師,家甚貧,多聚親族之煢獨者,祿賜所得,散施隨盡,不以屑
 意。與人交,情公切至,廢棄孤露者,待之愈厚,多所周給。



 初,赴舉,貧不能上,親族裒錢三萬遺之。道出滑臺,過父友呂翁家。翁喪,貧窶無以葬,其母兄將鬻女以襄事。道傾褚中錢與之,且為其女擇婿,別加資遣。又故人卒,貧甚,質女婢於人。道為贖之,嫁士族。搢紳服其履行。好學,嗜弈棋,深信內典。平居多茹蔬,或止一食,默坐終日,服玩極於卑儉。嘗夢神人謂曰:「汝位至正郎,壽五十七。」而享年六十四,論者以為積善所延也。有集二十卷,從兄
 陶。



 陶字大均,初事李煜,以明法登科,補常州錄事參軍。歸朝,詔大理評事,試律學,除本寺丞,遷大理正,歷侍御史、權判大理寺,賜緋。斷官仲禹錫訟陶用法非當,陶抗辯得雪。遷工部郎中,俄知臺州,累遷兵部。咸平五年,朱博為大理,議趙文海罪不當,宰相請以陶代。真宗曰:「聞陶亦深文,當加戒勖。」即遷秘書少監、判寺事。時楊億知審刑,陶屢攻其失,又命代之,賜金紫。陶持法深刻,用刑多
 失中,前後坐罰金百餘斤,皆以失入,無誤出者。景德三年,卒,年七十。子拱之,淳化三年進士,後為都官郎中;慶之,太子中舍。



 論曰:典誥命者,以詞章典雅為先;侍講讀者,以道德洽聞為貴。自昔皆難其人,至宋尤重其選。太宗崇尚儒術,聽政之暇,以觀書為樂,置翰林侍讀學士以備顧問。真宗克紹先志,兼置侍講學士,且因內閣以設職名,俾鴻碩之士更直迭宿,相與從容講論。以丕之清介,頑之和
 豫,顥之明敏,茂直之淳厚,俾領詞職,固無忝矣。若文仲之器韻淹雅,慎修之醞藉該貫,杜鎬之博聞強識,查道之純孝篤義,置諸左右,啟沃尤多,豈直講論文義而已哉。若祐之不喜趨競,徽之深疾幸進,風採凝峻,又其卓然者也。徽之嘗謂:「溫仲舒、寇準以搏擊取貴位,使後輩務習趨競,禮俗浸薄。」君子以為名言云



\end{pinyinscope}