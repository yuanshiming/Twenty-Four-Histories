\article{志第一百三十二 食貨下一(會計)}

\begin{pinyinscope}

 宋貨財之制,多因於唐。自天寶以後,天下多事,戶口凋耗,租稅日削,法既變而用不給,故興利者進,而徵斂名額繁矣。方鎮握重兵,皆留財賦自贍,其上供殊鮮。五代
 疆境逼蹙,藩鎮益強,率令部曲主場、院,其屬三司者,補大吏以臨之,輸額之外亦私有焉。



 太祖周知其弊,及受命,務恢遠略,修建法程,示之以漸。建隆中,牧守來朝,猶不貢奉以助軍實。乾德三年,始詔諸州支度經費外,凡金帛悉送闕下,毋或占留。時藩郡有闕,稍命文臣權知所在場務,或遣京朝官廷臣監臨。於是外權始削,而利歸公上,條禁文簿漸為精密。諸州通判官到任,皆須躬閱帳籍所列官物,吏不得以售其奸。主庫吏三年一易。
 市征、地課、鹽曲之類,通判官、兵馬都監、縣令等並親臨之,見月籍供三司,秩滿較其殿最,欺隱者置於法。募告者,賞錢三十萬。而小民求財報怨,訴訟煩擾,未幾,除募告之禁。



 先是,茶鹽榷酤課額少者,募豪民主之。民多增額求利,歲更荒儉,商旅不行,至虧常課,乃籍其貲產以償。太宗始詔以開寶八年為額,既又慮其未均,乃遣使分詣諸州,同長吏裁定。凡左藏及諸庫受納諸州上供均輸金銀、絲帛暨他物,令監臨官謹視之。欺而多取,主
 稱、藏吏皆斬,監臨官亦重置其罪。罷三司大將及軍將主諸州榷課,命使臣分掌。掌務官吏虧課當罰,長吏以下分等連坐。雍熙二年,令三分勾院糾本部陷失官錢,及百千賞以十之一,至五千貫者遷其職。



 淳化元年詔曰:「周設司會之職,以一歲為準;漢制上計之法,以三年為期。所以詳知國用之盈虛,大行群吏之誅賞,斯乃舊典,其可廢乎?三司自今每歲具見管金銀、錢帛、軍儲等簿以聞。」四年,改三司為總計司,左右大計分掌十道財
 賦。令京東西南北各以五十州為率,每州軍歲計金銀、錢、繒帛、芻粟等費,逐路關報總計司,總計司置簿,左右計使通計置裁給,餘州亦如之。未幾,復為三部。



 宋聚兵京師,外州無留財,天下支用悉出三司,故其費浸多。太宗孜孜庶務,或親為裁決。有司嘗言油衣、帟幕損破者數萬段,帝令煮之,染以雜色,制旗幟數千。調退材給窯務為薪,俾擇其可用者造什物數千事。其愛民惜費類此。



 真宗嗣位,詔三司經度茶、鹽、酒稅以充歲用,勿增賦
 斂以困黎元。是時條禁愈密,較課以租額前界遞年相參。景德初,榷務連歲增羨,三司即取多收者為額。帝慮或致掊克,詔凡增額比奏。上封者言:「諸路歲課增羨,知州、通判皆書歷為課最,有虧者則無罰。」乃令諸路茶、鹽、酒稅及諸場務,自今總一歲之課,合為一,以額較之。有虧則計分數,知州、通判減監官一等科罰,州司典吏減專典一等論,大臣及武臣知州軍者止罰通判以下。



 至道末,天下總入緡錢二千二百二十四萬五千八百。三
 歲一親祀郊丘,計緡錢常五百餘萬,大半以金銀、綾綺、絁綢平其直給之。天禧末,上供惟錢帛增多,餘以移用頗減舊數,而天下總入一萬五千八十五萬一百,出一萬二千六百七十七萬五千二百,而贏數不預焉。景德郊祀七百餘萬,東封八百餘萬,祀汾、上寶冊又增二十萬。丁謂為三司使,著《景德會計錄》以獻,林特領使,亦繼為之。凡舉大禮,有司皆籍當時所費以聞,必優詔獎之。



 初,吳、蜀、江南、荊湖、南粵皆號富強,相繼降附,太祖、太
 宗因其蓄藏,守以恭儉簡易。天下生齒尚寡,而養兵未甚蕃,任官未甚冗,佛老之徒未甚熾;外無金繒之遺,百姓亦各安其生,不為巧偽放侈,故上下給足,府庫羨溢。承平既久,戶口歲增,兵籍益廣,吏員益眾。佛老、外國耗蠹中土,縣官之費數倍於昔,百姓亦稍縱侈,而上下始困於財矣。



 仁宗承之,經費浸廣。天聖初,首命有司取景德一歲用度,較天禧所出,省其不急者。自祥符天書一出,齋醮糜費甚眾,京城之內,一夕數處,至是,始大裁損。
 京師營造,多內侍傳旨呼索,費無藝極。帝與太后知其弊,詔自今營造所須,先下三司度功費然後給。又減內外宮觀清衛卒及工匠,分隸諸軍、八作司。舊殿直已上,雖幼未任朝謁,遇乾元、長寧節皆賜服,至是亦罷給。故事,上尊號、謚號,隨冊寶物並用黃命。帝曰:「先帝、太后用黃金,若朕所御,止用塗金。」時洞真宮、壽寧觀相繼災,宰相張知白請罷不急營造,以答天戒。及滑州塞決河,御史知雜王鬷復以為言。既而玉清昭應宮災,遂詔諭中
 外,不復繕修。自是道家之奉有節,土木之費省矣。



 帝天資恭儉,尤務約己以先天下,有司言利者,多擯不取。聞民之有疾苦,雖厚利,舍之無所愛。貢獻珍異,故事有者,或罷之。山林、川澤、陂池之利,久與民共者,屢敕有司毋輒禁止。至於州縣徵取苛細,蠲減蓋不可勝數。



 至寶元中,陜西用兵,調度百出,縣官之費益廣。天章閣侍讀賈昌朝言:「臣嘗治畿邑,邑有禁兵三千,而留萬戶賦輸,僅能取足,郊祀慶賞,乃出自內府。計江、淮歲運糧六百餘
 萬石,以一歲之入,僅能充期月之用,三分二在軍旅,一在冗食,先所蓄聚,不盈數載。天下久無事,而財不藏於國,又不在民,儻有水旱軍戎之急,計將安出?」於是議省冗費。右司諫韓琦言:「省費當自掖庭始。請詔三司取先朝及近歲賜予日費之數,裁為中制,無名者一切罷之。」乃令入內內侍省、御藥院、內東門司裁定,有司不預焉。



 議者或欲損吏兵奉賜。帝謂:「祿廩皆有定制,毋遽變更以搖人心。」尹洙在陜西,請為鬻爵之法,亦不果行。其後
 西兵久不解,財用益屈,內出詔書:「減皇后至宗室婦郊祠半賜,著為式;皇后、嬪御進奉乾元節回賜物皆減半,宗室、外命婦回賜權罷。」於是皇后、嬪御各上奉錢五月以助軍費,宗室刺史已上,亦納公使錢之半。荊王元儼盡納公使錢,詔給其半,後以元儼叔父,全給如故。帝亦命罷左藏庫月進錢一千二百緡。公卿、近臣以次減郊祠所賜銀絹,舊四千、三千者損一千,千損三百,三百損百,百損二十,皆著為式。



 三司使王堯臣取陜西、河北、河
 東三路未用兵及用兵後歲出入財用之數,會計以聞。寶元元年未用兵,三路出入錢帛糧草:陜西入一千九百七十八萬,出二千一百五十一萬;河北入二千一十四萬,出一千八百二十三萬;河東入一千三十八萬,出八百五十九萬。用兵後,陜西入三千三百九十萬,出三千三百六十三萬,蓋視河東、北尤劇,以兵屯陜西特多故也。又計京師出入金帛:寶元元年,入一千九百五十萬,出二千一百八十五萬,是歲郊祠,故出入之數視常
 歲為多;慶歷二年,入二千九百二十九萬,出二千六百一十七萬,而奇數皆不預焉。



 會元昊請臣,朝廷亦已厭兵,屈意撫納,歲賜繒、茶增至二十五萬;而契丹邀割地,復增歲遺至五十萬,自是歲費彌有所加。西兵既罷,而調用無所減,乃下詔切責邊臣及轉運司趣議裁節,稍徙戍兵還內地。命三司戶部副使包拯行河北,與邊臣、轉運司議罷省冗官,汰軍士之不任役者。詔翰林學士承旨王堯臣等較近歲天下財賦出入之數,相參耗登。
 皇祐元年,入一億二千六百二十五萬一千九百六十四,而所出無餘。堯臣等為書七卷上之,送三司,取一歲中數以為定式。初,真宗時,內外兵九十一萬二千,宗室、吏員受祿者九千七百八十五。寶元以後,募兵益廣,宗室蕃衍,吏員歲增。至是,兵一百二十五萬九千,宗室、吏員受祿者萬五千四百四十三,祿廩奉賜從而增廣。及景德中,祀南郊,內外賞賚金帛、緡錢總六百一萬。至是,饗明堂,增至一千二百餘萬,故用度不得不屈。



 至和中,
 諫官範鎮上疏曰:「陛下每遇水旱之災,必露立仰天,痛自刻責,而吏不稱職,陛下憂勤於上,人民愁嘆於下。今歲無麥,朝廷為放稅免役及發倉廩拯貸,存恤之恩不為不至。然人民流離,父母妻子不相保者,平居無事時,不少寬其力役,輕其租賦;歲大熟,民不得終歲之飽;及有小歉,雖加重放,已不及事。此無他,重斂之政在前也。國家自陜西用兵以來,賦役煩重。及近年,轉運使復於常賦外進羨錢以助南郊,其餘無名斂率不可勝計。」



 又
 言:「古者塚宰制國用,今中書主民,樞密主兵,三司主財,各不相知。故財已匱而樞密院益兵不已,民已困而三司取財不已。中書視民之困,而不知使樞密減兵、三司寬財者,制國用之職不在中書也。願使中書、樞密通知兵民財利大計,與三司量其出入,制為國用,則天下民力庶幾少寬。」然自天聖以來,帝以經費為慮,屢命官裁節,而有司不能承上之意,卒無所建明。



 治平中,兵數少損,隸籍者猶百十六萬二千,宗室、吏員視皇祐無慮增
 十之三。英宗以勤儉自飭,然享國日淺,於經紀法度所未暇焉。治平二年,內外入一億一千六百十三萬八千四百五,出一億二千三十四萬三千一百七十四,非常出者又一千一百五十二萬一千二百七十八。是歲,諸路積一億六千二十九萬二千九十三,而京師不預焉。



 神宗嗣位,尤先理財。熙寧初,命翰林學士司馬光等置局看詳裁減國用制度,仍取慶歷二年數,比今支費不同者,開析以聞。後數日,光登對言:「國用不足,在用度大
 奢,賞賜不節,宗室繁多,官職冗濫,軍旅不精。必須陛下與兩府大臣及三司官吏,深思救弊之術,磨以歲月,庶幾有效,非愚臣一朝一夕所能裁減。」帝遂罷裁減局,但下三司共析。



 王安石執政,議置三司條例司,講修錢穀之法。帝因論措置之宜,言:「今財賦非不多,但用不節,何由給足?宮中一私身之奉有及八十千者,嫁一公主至費七十萬緡,沈貴妃料錢月八百緡。聞太宗時宮人惟系皂綢襜,元德皇后嘗用金線緣襜,太宗怒其奢。仁宗
 初定公主奉料,以問獻穆,再三始言初僅得五貫爾,異時中宮月有止七百錢者。」時天下承平,帝方經略四夷,故每以財用不給為憂。日與大臣講求其故,命官考三司簿籍,商量經久廢置之宜,凡一歲用度及郊祀大費,皆編著定式。



 有司請造龍圖、天章閣覆欄檻青氈四百九十。帝謂:「禁中諸殿欄檻率故弊,不必覆也。」既而並延福宮覆檻氈罷之。後呂嘉問復建議省儀鸞司供禁中彩帛。是歲,詔內外勿給土木工作,非兩宮、倉廩、武庫,皆
 罷省。三年,儀鸞司闕氈三千,三司請命河東制之。帝曰:「牛羊司積毛數萬斤,皆同糞壤,三司不取於此,而欲勤遠民乎?」金州歲貢班竹簾,簡州歲貢綿綢,安州市紅花萬斤,梓州市碌二千斤,帝皆以道遠擾民,亟命停罷。



 制置司言:「諸路科置上供羊,民費錢幾倍,而河北榷場博買契丹羊歲數萬,路遠,抵京皆瘦惡耗死,公私費錢四十餘萬緡。」詔著作佐郎程博文訪利害。博文募民有保任者,以產為抵,官預給錢,約期限、口數、斤重以輸。民多
 樂從,歲計充足。凡供御膳及祀祭與泛用者,皆別其牢棧,以三千為額,所裁省冗費十之四。其後,又用呂嘉問、劉永淵之言,治灶藏冰,以省工費。



 帝嘗患增置官司費財。王安石謂增置官司,所以省費。帝曰:「古者什一而稅,今取財百端。」安石謂古非特什一而已。帝又以倉吏給軍食,多侵盜,詔足其概量,嚴立諸倉丐取法。中書因請增諸倉主典、役人祿至一萬八千九百緡,且盡增選人之祿,均其多寡。令、祿增至十五千;司理至簿、尉,防團軍
 監推、判官增至十二千。其後又增中書、審官東西、三班院、樞密院、三司、吏部流內銓、南曹、開封府吏祿,受財者以倉法論。安石蓋欲盡祿天下之吏,帝以役法未就,緩其議。三司上新增吏祿數:京師歲增四十一萬三千四百餘緡,監司、諸州六十八萬九千八百餘緡。時主新法者皆謂吏祿既厚,則人知自重,不敢冒法,可以省刑。然良吏實寡,賕取如故,往往陷重闢,議者不以為善。



 初,陜西用兵,凡費緡錢七百餘萬。帝以問王安石,安石曰:「楚
 建中考沉起簿書,計一道半歲費錢銀綢絹千二百萬貫、匹、兩。」帝因欲知陜西歲用錢穀、金帛及增虧凡數,乃詔薛向條上。王安石以為擾,力請罷之,止詔三司帳司會計熙寧六年天下財用出入之數以聞。



 韓絳既相,建言:「三司總天下財賦,請選官置司,以天下戶口、人丁、稅賦、場務、坑冶、河渡、房園之類租額年課,及一路錢穀出入之數,去其重復,歲比較增虧、廢置及羨餘、橫費。計贏闕之處,使有無相通,而以任職能否為黜陟,則國計大
 綱可以省察。」三司使章惇亦以為言,乃詔置三司會計司,以絳提舉。其後一州一路會計式成,上之,餘未就緒,未幾遂罷。



 元豐官制既行,三司所掌職務散於六曹、諸寺監。元祐初,司馬光言:「今戶部尚書,舊三司使之任,左曹隸尚書,右曹不隸焉。天下之財分而為二,視彼有餘,視此不足,不得移用。宜令尚書兼領左右曹,侍郎分職而治,舊三司所掌錢穀財用事,有散於五曹及諸寺、監者,並歸戶部。」遂詔尚書省立法。



 有司請以府界、諸路在
 京庫務及常平等文帳悉歸戶部。初,熙寧五年,患天下文帳之繁,命曾布刪定法式。布因請選吏於三司顓為一司,帳司之置始此。至元豐三年,首尾七八年,所設官吏僅六百人,費錢三十九萬緡,而勾磨出失陷錢止萬緡。朝廷知其無益,遂罷帳司,使州郡應上省帳皆歸轉運司,惟錢帛、糧草、酒曲、商稅等別為計帳上戶部。至是,令戶部盡收諸路文帳。蘇轍時為諫官,謂徒益紛紛,請如舊為便。不行。



 三年,戶部尚書韓忠彥、侍郎蘇轍、韓宗
 道言:「文武百官、宗室之蕃,一倍皇祐,四倍景德,班行、選人、胥吏率皆增益,而兩稅、征推、山澤之利,與舊無以相過。治平、熙寧之間,因時立政,凡改官者自三歲而為四歲,任子者自一歲一人而為三歲一人、自三歲一人而為六歲一人,宗室自袒免以上漸殺恩禮,此則今日之成法。乞檢會寶元、慶歷、嘉祐故事,置司選官共議。」詔戶部取應乾財用,除諸班諸軍料錢、衣賜、賞給、特支如舊外,餘費並裁省。又詔:「方將裁損八流,以清取士之路。命
 今後遇聖節、大禮、生辰,太皇太后、皇太后、皇太妃所得恩澤,並四分減一。」於是上自宗室貴近,下至官曹胥吏,旁及宮室械器,皆命裁損。久之,事未就。議者謂裁減浮費所細碎苛急,甚損國體。於是已議未行者一切寢之。後乃詔:「元祐裁損除授正任以下奉祿,失朝廷優禮,見條悉除之,循元豐舊制。」



 元豐鉤考隱漏宜錢,督及一分者賞三厘。自元祐改法,賞薄而吏怠,遂復其舊。時議裁損吏祿,隸省、曹、寺、監者,止以元豐三年錢數為額,而吏
 三省者,凡兼領因事別給並舊請並罷。劉摯遂乞悉罷創增吏祿,詔韓維等究度,然不果罷。其後有司計中都吏祿,歲費緡錢三十二萬,詔以坊場稅錢給之。於是吏祿之冗濫者,率多革去矣。然三省吏猶有人受三奉而不改者,故孫升、傅堯俞皆以為言。至紹聖、元符,務反元祐之政,下至六曹吏,亦詔皆給見緡,如元豐之制。



 先是,既罷導洛、堆垛等局,又罷熙河蘭會經制財用司,減放市易欠負及積欠租輸,選官體量茶鹽之法。使者之刻
 剝害民,如吳居厚、呂孝廉、王子京、李琮,內臣之生事斂怨,如李憲、宋用臣等,皆相繼正其罪。既而稍復講修財利。李清臣因白帝,今中外錢穀艱窘,戶部給百官奉,常無數月之備。章惇遂以財用匱乏,專指為司馬光、呂公著、呂大防、蘇轍諸人之罪。左司諫翟思亦奏疏詆:「元祐以理財為諱,利入名額類多廢罷,督責之法不加於在職之臣,財利既多散失,且借貸百出,而熙、豐餘積,用之幾盡。方今內外財用,月計歲會,所入不足給所出。願下
 諸路會元祐以前所儲金谷及異時財利名額、歲入經數,著為成式。」



 建中靖國元年,詔諸路轉運司以歲入財用置都籍,定諸州租額,且計一路凡數;即有贏縮,書其籍。崇寧元年,又令:「歲以錢穀出入名數報提刑司保驗,以上戶部;戶部歲條諸路轉運使財賦虧贏,以行賞罰。諸路無額錢物,立式下提刑司,括三年外未發數,期以一季聞奏。」二年,官吏違負上供錢物,以分數為科罪之等,不及九分者罪以徒,多者更加之。歲首則列次年之
 數,聞於漕司,考實申部。又以督限未嚴,更一季為一月。然國之經費,往往不給。



 五年,詔省罷官局,命戶部侍郎許幾專切提舉措置。裁罷開封府重祿通引官客司並街道司額外兵士,及罷在京料次錢三十八處。



 大觀三年,罷諸路州軍見貢六上局供奉物名件四百四十餘,存者才十一二,減數十二,停貢六。戶部侍郎範坦言:「戶部歲入有限,支用無窮,一歲之入,僅了三季,餘仰朝廷應付。今歲支遣,較之去年又費百萬。」有詔鐫減財賦,命
 御史中丞張克公與吳居厚、許幾等置局議論。克公抗言:「官冗者汰,奉厚者減,今官較之元祐已多十倍,國用安得不乏。乞將節度使下至遙郡刺史,除軍功轉授者,各減奉半,然後閑慢局務、工伎末作,亦宜減省。自貴及賤,自近及遠,行之公當,人自無詞。」時論韙之。



 時諸路轉運司類以乏告,詔戶部編次一歲財用出納之數,諸路州縣各為都籍,以待考較;工部金、銀、銅、鉛、水銀、朱砂等,亦嚴帳籍之法;令諸路各條三十年以還一歲出入及
 泛用之數。初,比部掌勾稽天下文帳,吏習偷惰,自崇寧至政和,稽違積數凡二千六百七十有餘。於是申敕六曹,以拘督一歲多寡為寺、監賞罰。



 政和七年,命戶部參稽熙、豐及今財用有餘不足之數,又立旁通格,令諸路漕司各條元豐、紹聖、崇寧、政和一歲財用出入多寡來上。淮南漕臣張根言:「天下之費,莫大於土木之功。其次如人臣賜第,一第無慮數十萬緡,稍增雄麗,非百萬不可。佐命如趙普,定策如韓琦,不聞峻宇雕墻,僭擬宮省,
 奈何剝民膚髓,為廝役之奉乎?其次如田產、房廊,雖不若賜第之多,然日削月朘,所在無幾。又如金帛以供一時之好賜,有不可已者,而亦不可不節。至如賜帶,其直雖不過數百緡,然天下金寶糜費日久,夫豈易得?今乃賚及僕隸,使混淆公卿間,貴賤、賢不肖,莫之辨也。如以為左右趨走之人,不欲其墨綬,當別為制度,以示等威。」疏奏,不省。



 重和初,罷講畫經費局。有司議勾收白地,禁榷鐵貨,方田增稅,榷酤增價,量收醋息,河北添折稅米
 等。俄慮騷擾,悉罷之,並焚其條約。未幾,又置裕民局,命蔡京提舉,徐處仁詳定。京大不悅,尋亦罷。宣和元年,以左藏庫虧沒一百七十九萬有奇,乃別造都籍,催轄司、太府寺、左藏庫互相鉤考,以絕奸弊。



 帝初即位,思節冗費,中都吏重復增給及泛監員額,並詔裁損。後苑嘗計增葺殿宇,計用金箔五十六萬七千。帝曰:「用金為箔,以飭土木,一壞不可復收,甚亡謂也。」令內侍省罰請者。及蔡京為相,增修財利之政,務以侈靡惑人主,動以《周官》
 惟王不會為說,每及前朝惜財省費者,必以為陋。至於土木營造,率欲度前規而侈後觀。元豐改官制,在京官司供給之數,皆並為職錢,視嘉祐、治平時賦祿優矣。京更增供給、食料等錢,於是宰執皆然。京既罷相,帝惡其變亂法度,將盡更革。命戶部侍郎許幾裁損浮費及百官濫祿,悉循元豐之舊,宰執亦聽辭所增奉。京不便,與其黨倡言:「減奉非治世事。司馬光請聽宰臣辭南郊給賜,神宗卒不允,且增選人及庶人在官者之奉。帝以繼
 述為事,當奉承神宗。」由是官吏奉給並仍舊,而宰執亦增如故。初,宰執掌食亦皆有常數。至是,品目偎多,有公使、乏支之別,臺、省、寺、監又增廚錢。侍御史毛注嘗奏論之,不行。蔡京復得政,言者遂以裁損祿廩為幾罪,幾坐奪職。



 於時天下久平,吏員冗溢,節度使至八十餘員,留後、觀察下及遙郡刺史多至數千員,學士、待制中外百五十員。京又專用豐亨豫大之說,諛悅帝意,始廣茶利,歲以一百萬緡進御,以京城所主之。其後又有應奉司、
 御前生活所、營繕所、蘇杭造作局、御前人船所,其名雜出,大率爭以奇侈為功。歲運花石綱,一石之費,民間至用三十萬緡。奸吏旁緣,牟取無藝,民不勝弊。用度日繁,左藏庫異時月費緡錢三十六萬,至是,衍為一百二十萬。



 又三省、密院吏員猥雜,有官至中大夫,一身而兼十餘俸,故當時議者有「俸入超越從班,品秩幾於執政」之言。又增置兼局,禮制、明堂,詳定《國朝會要》、《九域圖志》、《一司敕令》之類,職秩繁委,廩給無度。侍御史黃葆光論其
 弊,帝善之而未行;俄而詔云:當豐亨豫大之時,為衰亂減損之計」,自是罕敢言者。然吏祿泛冒已極,以史院言之,供檢吏三省幾千人。蔡京又動以筆帖於榷貨務支賞給,有一紙至萬緡者。京所侵私,以千萬計,朝論喧然。乃詔三省、樞密院吏額用元豐法,其歲賜悉裁之,時翕然以為快。臣僚上言:「諸州遇天寧節,除公使外,別給系省錢,充錫宴之用。獨諸路監司許支逐司錢物,一筵之饌,有及數百千者,浮侈相誇,無有藝極。」自是詔:「遇天寧
 節宴,舊應給錢者,發運、監司每司不得過三百貫,餘每司不得過二百貫,以上舊給數少者,止依舊。」



 自崇寧以來,言利之臣殆析秋毫,沿汴州縣創增鎮柵以牟稅利。官賣石炭增二十餘場,而天下市易務,炭皆官自賣。名品瑣碎,則有四腳鋪床、榨磨、水磨、廟圖、淘沙金等錢,不得而盡記也。宣和以後,王黼專主應奉,掊剝橫賦,以羨為功。嶺南、川蜀農民陂罰錢,罷學制學事司贍學錢,皆歸應奉司。所入雖多,國用日匱。



 六年,尚書左丞宇文粹
 中言:



 「近歲南伐蠻獠,北贍幽燕,關陜、綿、茂邊事日起,山東、河北寇盜竊發。賦斂歲入有限,支梧繁伙,一切取足於民。陜西上戶多棄產而居京師,河東富人多棄產而入川蜀。河北衣被天下,而蠶織皆廢;山東頻遭大水,而耕種失時;他路取辦目前,不務存恤。穀麥未登,已先俵糴;歲賦已納,復理欠負。托應奉而買珍異奇寶,欠民積者一路至數十萬計;價上供而織文繡錦綺,役工女者一郡至百餘人。



 陛下勤恤民隱,詔令數下,悉為虛文。民
 不聊生,不惟寇盜繁滋,竊恐災異數起。祖宗之時,國計所仰,皆有實數。有額上供四百萬,無額上供二百萬,京師商稅、店宅務、抵當所諸處雜收錢一百餘萬。三司以七百萬之入,供一年之費,而儲其餘以待不測之用。又有解池鹽鈔、晉礬、市舶遺利,內贍京師,外實邊鄙,間遇水旱,隨以振濟,蓋量入為出,沛然有餘。近年諸局務、應奉等司截撥上供,而繁富路分一歲所入,亦不敷額。然創置書局者比職事官之數為多,檢計修造者比實用
 之物增倍,其它妄耗百出,不可勝數。若非痛行裁減,慮智者無以善其後。」



 久之,乃詔蔡攸等就尚書省置講議財利司,除茶法已有定制,餘並講究條上。攸請:內侍職掌,事乾宮禁,應裁省者,委童貫取旨。時貫以廣陽郡王領右府故也。於是不急之務,無名之費,悉議裁省。帝亦自罷諸路應奉官吏,省六尚歲貢。



 七年,詔諸路帥臣、監司各條所部當裁省凡目以聞。後苑書藝局等月省十九萬緡,歲可省二百二十萬。應奉司所管諸色窠名錢
 數內:兩浙路錢旁定帖息錢,湖、常、溫、秀州無額上供錢,淮南路添酒錢等,並行截節,更不充應奉支用。十二月,詔曰:「比年寬大之詔數下,裁省之令屢行。有司便文而實惠不至,蓋緣任用非人,興作事端,蠹耗邦財。假充上之名,濟營私之欲,漁奪百姓,無所不至。朕夙夜痛悼,思有以撫循慰安之。應茶鹽立額結絕。應奉司兩浙諸路置局及花石綱等,諸路非泛上供拋降物色,延福宮西城所租課,內外修造諸處採斫木植、制造局所,並罷。諸
 局及西城所見管錢物並付有司,其拘收到百姓地上,並給還舊佃人。減掖庭用度,減侍從官以上月廩,及罷諸兼局,以上並令有司據所得數撥充諸路糴本,及樁充募兵賞軍之用。應齋醮道場,除舊法合有外,並罷道官及撥賜宮觀等房錢、田土之類。六尚,並依祖宗法。罷大晟府,罷教學所,罷教坊額外人。罷行幸局,罷採石所,罷待詔額外人。罷都茶場,依舊歸朝廷。河坊非危急泛科、免夫錢並罷。」



 是時天下財用歲入,有御前錢物、朝廷
 錢物、戶部錢物,其措置裒斂、取索支用,各不相知。天下財賦多為禁中私財,上溢下漏,而民重困。言者請令戶部周知大數,而不失盈虛緩急之宜。上至宮禁所須,下逮吏卒廩餼,一切付之有司,格以法度,示天下以至公。詔可。戶部尚書聶山亦請以熙、豐後增置添給,如額外醫官、內中諸閣分位次主管文字等使臣、福源靈應諸觀清衛卒、后妃戚里及文武臣僚之家母妻封國太夫人郡太夫人等請給,並添給食料、茶湯等錢四十萬八
 千九百餘緡,凡熙、豐無法該載者罷之。



 靖康元年,詔曰:「朕托於兆庶之上,永念民惟邦本,思所以閔恤安定之。乃者,減乘輿服御,放宮女,罷苑囿,焚玩好之物,務以率先天下;減冗官,澄濫賞,汰貪吏,為民除害。方詔減上供收買之額,蠲有司煩苛之令,輕刑薄賦,務安元元;而田里之間,愁痛未蘇,儻不蠲革,何以靖民!今詢酌庶言,疏剔眾弊,舉其綱目,以授四方。詔到,監司、郡守其悉力奉行;應民所疾苦,不在此詔,許推類聞奏。」於是凡當時苛
 刻煩細、一切不便於民者皆罷。



 高宗建炎元年,詔:「諸路無額上供錢,依舊法,更不立額。」三年二月,減婺州上供額羅二萬八千匹,著為定制。八月,減福建、廣南路歲買上供銀三分之一。紹興二年,罷鎮江府御服羅,省錢七萬緡,助劉光世軍。四年二月,詔:「諸路州縣天申節禮物,並置場和買,毋得抑配於民。」十有一月,免淮南州軍大禮絹。五年,以四川上供錢帛依舊留以贍軍。十一年,始命四川上供羅復輸內藏,其後綾、紗、絹悉如之。



 四路天申節大
 禮絹及上供綢、綾、錦、綺,共九萬五千八百匹。



 淳熙五年,湖北漕臣劉焞言:「鄂、岳、漢陽自紹興九年所收賦財,十分為率,儲一分充上供始,十三年年增二分。鄂州元儲一分,錢一萬九千五百七十緡,今已增至一十二萬九千餘緡;岳州五千八百餘緡,今增至四萬二千一百餘緡;漢陽三千七百緡,今增至二萬二千三百餘緡。民力凋弊,無所從出。」於是以見增錢數立額,已後權免遞增。詔夔州路九州百姓科買上供金、銀、絹,自淳熙六年為始盡免。十六年,蠲兩
 淮州軍合發上供諸窠名錢物,極邊全免,次邊展免一年。



 紹定元年,江、浙諸州軍折輸上供物帛錢數,除合起輕貨,並用錢、會中半;路不通水,願以銀折輸者聽,兩不過三貫三百文。兩浙、江東共四百一十三萬八千六百一十二貫有奇,並輸送左藏西庫。



 咸淳六年,都省言:「南渡以來,諸路上供數重,自嘉定至嘉熙,起截之數雖減,而州縣猶以大數拘催,害及百姓。」有旨:「自咸淳七年為始,銀、錢、關、會用咸淳三年起截中數拘催,綢、絹、絲、綿、綾、
 羅用咸淳二年起截中數拘催。錢、關、會子二千四百九十五萬八千七百四十八貫,銀一十六萬九千六百四十三兩,綢四萬一千四百三十八匹,絹七十三萬七千八百六十匹,絲九萬五千三百三十三兩,綿一百五萬七千九百二十五兩,綾五千一百七十九匹,羅七千三百五十五匹,戶部遍牒諸路,視今所減定額起催。」



 所謂經總制錢者,宣和末,陳亨伯以發運兼經制使,因以為名。建炎二年,高宗在揚州,四方貢賦不以期至,戶部尚
 書呂頤浩、翰林學士葉夢得等言:「亨伯以東南用兵,嘗設經制司,取量添酒錢及增一分稅錢,頭子、賣契等錢,斂之於細,而積之甚眾。及為河北轉運使,又行於京東西,一歲得錢近二百萬緡,所補不細。今若行於諸路州軍,歲入無慮數百萬計。邊事未寧,茍不出此,緩急必致暴斂。與其斂於倉卒,曷若積於細微。」於是以添酒錢、添賣糟錢、典賣田宅增牙稅錢、官員等請給頭子錢、樓店務增三分房錢,令兩浙、江東西、荊湖南北、福建、二廣收
 充經制錢,以憲臣領之,通判斂之,季終輸送。紹興五年,參政孟庾提領措置財用,請以總制司為名,又因經制之額增析而為總制錢,而總制錢自此始矣。



 財用司言:「諸路州縣出納系省錢所收頭子錢,貫收錢二十三文省,內一十文省作經制起發上供,餘一十三文充本路郡縣並漕司用。今欲令諸路州縣雜稅出納錢貫收頭子錢上,量增作二十三文足。除漕司及州舊合得一十三文省,餘盡入經制窠名帳內,起發助軍。」江西提舉司
 言:「常平錢物,舊法貫收頭子錢五文足。今當依諸色錢例,增作二十三文足,除五文依舊法支用,餘增到錢與經制司別作窠名輸送。」



 九年,諫議大夫曾統上疏言:「經制使本戶部之職,更置一司,無益於事。如創供給酒庫,亦是陰奪省司之利。若謂監司、郡縣違法廢令,別建此司按之,則又不然。夫朝廷置監司以轄州郡,立省部以轄監司,祖宗制也。稅賦失實,當問轉運司;常平錢穀失陷,當問提舉司。若使經制司能事事檢察,則雖戶部版
 曹,亦可廢矣。且自置司以來,漕司之移用,憲司之贓罰,監司之妄支,固未嘗少革其弊。罷之便。」疏奏,不省。十六年,以諸路歲取經總制錢,本路提刑並檢法乾辦官拘催,歲終通紐以課殿最。二十一年,以守、倅同檢察。二十九年,詔專以通判主之。



 乾道元年,詔:「諸路州縣出納,貫添收錢一十三文省,充經總制錢,以所增錢別輸左藏西庫,補助經費。」自是經總制錢每千收五十六文矣。然遇兵兇,亦時有蠲免。三年,復以守、倅共掌之。



 淳熙十六
 年,光宗即位,減江東西、福建、淮東、浙西經總制錢一十七萬一千緡。紹熙二年,詔平江府合發經總制錢歲減二萬緡。嘉定十七年,詔蠲嘉定十五年終以前所虧錢數。端平三年,詔:「諸路州軍因災傷檢放苗米,毋收經總制頭子、勘合朱墨等錢;自今已放苗米,隨苗帶納錢並與除放。」



 所謂月樁錢者,始於紹興之二年。時韓世忠駐軍建康,宰相呂頤浩、朱勝非議今江東漕臣月樁發大軍錢十萬緡,以朝廷上供經制及漕司移用等錢供億。
 當時漕司不量州軍之力,一例均科,既有偏重之弊,上供經制,無額添酒錢並爭利錢,贍軍酒息錢,常平錢,及諸司封樁不封樁、系省不系省錢,皆是朝廷窠名也。



 於是郡縣橫斂,銖積絲累,江東、西之害尤甚。十七年,詔州郡以寬剩錢充月樁,以寬民力,遂減江東、西之錢二十七萬七千緡有奇。



 又有所謂板帳錢者,亦軍興後所創也。如輸米則增收耗剩,交錢帛則多收糜費,幸富人之犯法而重其罰,恣胥吏之受賕而課其入,索盜贓則不償失主,檢財產則不及卑幼,亡僧、絕戶不俟核實而入
 官,逃產、廢田不與消除而抑納,他如此類,不可遍舉。州縣之吏固知其非法,然以版帳錢額太重,雖欲不橫取於民,不可得已。



 凡貨財不領於有司者,則有內藏庫,蓋天子之別藏也。縣官有巨費,左藏之積不足給,則發內藏佐之。宋初,諸州貢賦皆輸左藏庫,及取荊湖,定巴蜀,平嶺南、江南,諸國珍寶、金帛盡入內府。初,太祖以帑藏盈溢,又於講武殿後別為內庫,嘗謂:「軍旅、饑饉當預為之備,不可臨事厚斂於民。



 太宗嗣位,漳泉、吳越相次獻
 地,又下太原,儲積益厚,分左藏庫為內藏庫,令內藏庫使翟裔等於左藏庫擇上綾羅等物別造帳籍,月申樞密院;改講武殿後庫為景福殿庫,俾隸內藏。其後乃令揀納諸州上供物,具月帳於內東門進入,外庭不得預其事。帝因謂左右曰:「此蓋慮司計之臣不能節約,異時用度有闕,復賦率於民,朕不以此自供嗜好也。」



 自乾德、開寶以來,用兵及水旱振給、慶澤賜賚、有司計度之所闕者,必籍其數以貸於內藏,候課賦有餘,即償之。淳化
 後二十五年間,歲貸百萬,有至三百萬者。累歲不能償,則除其籍。



 景德四年,又以新衣庫為內藏西庫。初,劉承珪嘗掌庫,經制多其所置,又推究置庫以來出納,造都帳及《須知》,屢加賞焉。真宗再臨幸,作銘刻石。大中祥符五年,重修庫屋,增廣其地。既而又以香藥庫、儀鸞司屋益之,分為四庫:金銀一庫,珠玉、香藥一庫,錦帛一庫,錢一庫。金銀、珠寶有十色,錢有新舊二色,錦帛十三色,香藥七色。天禧二年,又出內藏緡錢二百萬給三司。



 天聖
 以後,兵師、水旱費無常數,三歲一賚軍士,出錢百萬緡,綢絹百萬匹,銀三十萬兩,錦綺、鹿胎、透背、綾羅紗縠合五十萬匹,以佐三司。又歲入饒、池、江、建新鑄緡錢一百七萬,而斥舊蓄緡錢六十萬於左藏庫,率以為常。異時三司用度不足,必請貸於內藏,輒得之,其名為貸,實罕能償。景祐中,內藏庫主者言:「歲斥緡錢六十萬助三司,自天禧三年始。計明道二年距今才四年,而所貸錢帛九百一十七萬。」在太宗時三司所貸甚眾,久不能償,至
 慶歷中,詔悉蠲之。蓋內藏歲入金帛,皇祐中,二百六十五萬七千一十一;治平一百九十三萬三千五百五十四。其出以助經費,前後不可勝數,至於儲積贏縮,則有司莫得詳焉。



 神宗臨御之初,詔立歲輸內藏錢帛之額,視慶歷上供為數。嘗謂輔臣曰:「比閱內藏庫籍,文具而已,財貨出入,初無關防。舊以龍腦、珍珠鬻於榷貨務,數年不輸直,亦不鉤考。嘗聞太宗時內藏財庫,每千計用一牙錢記之。凡名物不同,所用錢色亦異,他人莫能曉,
 匣而置之御閣,以參驗帳籍中定數。晚年,出其錢示真宗曰:『善保此足矣。』今守內藏臣,皆不曉帳籍關防之法。」即命乾當御藥李舜舉領其事。繼詔諸路金銀輸內藏庫者,歲以帳上三司拘催。元豐以來,又詔諸路金帛、緡錢輸內庫者,委提點刑獄司督趣,若三司、發運司擅留者,坐之。超發坊場錢勿寄市易務,直赴內藏庫寄帳封樁。當輸內庫金帛、緡錢,逾期或他用者,如擅用封樁錢法。



 初,藝祖嘗欲積縑帛二百萬易敵人首,又別儲於景
 福殿。元豐初,乃更景福殿庫名,自制詩以揭之曰:「五季失圖,玁狁孔熾,藝祖造邦,思有懲艾,爰設內府,基以募士,曾孫保之,敢忘厥志。」一字一庫以號之,凡三十二庫。後積羨贏為二十庫,又揭詩曰:「每虔夕惕心,妄意遵遺業,顧予不武姿,何日成戎捷。」



 元祐元年,監察御史上官均言:「自新官制,蓋有意合理財之局總於一司,故以金部右曹主行內藏受納,而奉宸內藏庫受納又隸太府寺。然按其所領,不過關通所入名數,為之拘催而已,支
 用多寡,不得轉質。總領之者,止中官數十人,彼惟知謹扃鑰、塗窗牖,以為固密爾,又安能鉤考其出入多少,與夫所蓄之數哉?宜因官制之意,令戶部、太府寺,於內藏諸庫皆得檢察。」明年,詔內藏庫物聽以多寡相除。置庫百餘年,至是始編閱云。



 崇寧元年,詔:「祖宗置內藏庫貯經費餘財,所以募士威敵,振乏固本,皆有成法。比歲官司懈馳,侵蠹耗減,務在協力遵守,無令偏廢。」於是命倉部郎中丘括行諸路驅磨。三年,中書奏:「熙寧之制,江南
 諸路金銀課利並輸內帑。元祐中,戶部尚書李常於中以三分助轉運司,致內帑漸以虧減。」乃詔諸路新舊坑冶所收課利金銀並輸內帑,如熙寧之舊。後又入於大觀東庫。尋命仍舊以七分輸內帑,餘給轉運司。宣和六年,申截留、借兌內帑錢物之制。



 時又有元豐庫,則雜儲諸司羨餘錢。諸道榷酤場,舊以酬衙前之陪備官費者,熙寧役法行,乃聽民增直以售,取其價給衙前。久之,坊場錢益多,司農請歲發百萬緡輸中都。元豐三年,遂於
 司農寺南作元豐庫貯之,以待非常之用。



 元祐元年,右司諫蘇轍論河北保甲之害,因言:「元豐及內庫財物山委,皆先帝多方蓄藏,以備緩急。若積而不用,與東漢西園錢,唐之瓊林、大盈二庫何異?願以三十萬緡募保甲為軍。」尋用其議。元祐三年,改封樁錢物庫為元祐庫。未幾,分元豐庫為元豐南、北庫。數月,以北庫為司空呂公著廨,封樁並附南庫仍舊。元豐六年,詔歲以內藏庫緡錢五十萬樁元豐庫,補助軍費。崇寧以後,諸路封樁禁
 軍闕額給三路外,與常平、坊場、免役、綢絹、貼輸東北鹽錢,及鬻賣在官田屋錢,應前收樁管封樁權添酒錢、侵占房廊白地錢、公使庫遺利等錢,並輸元豐庫。別又置大觀庫,制同元豐,但分東西之別。最後,建宣和庫,有泉貨、弊餘、服御、玉食、器貢等名,蓋蔡絳欲效王黼以應奉司貢獻要寵,事不足紀。



 靖康元年,詔諸路公使庫及神霄宮金銀器皿,所在盡輸元豐庫。戶部尚書聶山輒取元豐庫北珠,宰相吳敏白帝,言:「朝廷有元豐、大觀庫,猶
 在陛下有內藏庫。朝廷有闕用,需於內藏,必得旨然後敢取,戶部豈可擅取朝廷庫務物哉?若人人得擅取庫物,則綱紀亂矣。」欽宗然之。



 南渡,內藏諸庫貨財之數雖不及前,然兵興用乏,亦時取以為助。其籍帳之詳莫得而考,則以後宋史多闕雲。



\end{pinyinscope}