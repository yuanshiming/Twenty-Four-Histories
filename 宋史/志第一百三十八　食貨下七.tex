\article{志第一百三十八 食貨下七}

\begin{pinyinscope}

 酒坑
 冶礬香附



 酒宋榷酤之法:諸州城內皆置務釀酒,縣、鎮、鄉、閭或許民釀而定其歲課,若有遺利,所在多請官酤。三京官造曲,聽民納直以取。



 陳、滑、蔡、穎、隨、郢、鄧、金、房州、信陽軍舊
 皆不榷。太平興國初,京西轉運使程能請榷之,所在置官吏局署,取民租米麥給釀,以官錢市薪□□及吏工奉料。歲計獲無幾,而主吏規其盈羨,及醞齊不良,酒多醨薄,至課民婚葬,量戶大小令酤,民甚被其害。歲儉物貴,殆不償其費。太宗知其弊,淳化五年,詔募民自釀,輸官錢減常課三之二,使其易辦;民有應募者,檢視其貲產,長吏及大姓共保之,後課不登則均償。是歲,取諸州歲課錢少者四百七十二處,募民自酤,或官賣曲收
 其直。其後民應募者寡,猶多官釀。



 陜西雖榷酤,而尚多遺利。咸平五年,度支員外郎李士衡請增課以助邊費,乃歲增十一萬餘貫。兩浙舊募民掌榷,雍熙初,以民多私釀,遂蠲其禁,其榷酤歲課如曲錢之制,附兩稅均率。二年,詔曰:「有司請罷杭州榷酤,乃使豪舉之家坐專其利,貧弱之戶歲責所輸,本欲惠民,乃成侵擾。宜仍舊榷酒,罷納所均錢。」天禧四年,轉運副使方仲荀言:「本道酒課舊額十四萬貫,遺利尚多。」乃歲增課九萬八千貫。



 川峽承
 舊制,賣曲價重,開寶二年,詔減十之二。既而頗興榷酤,言事者多陳其非便,太平興國七年罷,仍舊賣曲。自是,惟夔、建、開、施、廬、黔、涪、黎、威州、梁山雲安軍,及河東之麟、府州,荊湖之辰州,福建之福、泉、汀、漳州、興化軍,廣南東、西路不禁。



 自春至秋,醞成即鬻,謂之「小酒」,其價自五錢至三十錢,有二十六等;臘釀蒸鬻,候夏而出,謂之「大酒」,自八錢至四十八錢,有二十三等。凡醞用粳、糯、粟、黍、麥等及曲法、酒式,皆從水土所宜。諸州官釀所費穀麥,
 準常糴以給,不得用倉儲。酒匠、役人當受糧者給錢。凡官曲,麥一斗為曲六斤四兩。賣曲價:東京、南京斤直錢百五十五,西京減五。



 咸平末,江、淮制置增榷酤錢,頗為煩刻。景德二年,詔毋增榷,自後制置使不得兼領酒榷。四年,又詔中外不得更議增課以圖恩獎。天禧初,著作郎張師德使淮南,上言:「鄉村酒戶年額少者,望並停廢。」從之。



 至道二年,兩京諸州收榷課銅錢一百二十一萬四千餘貫、鐵錢一是五十六萬五千餘貫,京城賣曲錢四
 十八萬餘貫。天禧末,榷課銅錢增七百七十九萬六千餘貫,鐵錢增一百三十五萬四千餘貫,曲錢增三十九萬一千餘貫。



 五代漢初,犯曲者並棄市;周,至五斤者死。建隆二年,以周法太峻,犯私曲至十五斤、以私酒入城至三斗者始處極刑,餘論罪有差;私市酒、曲者減造人罪之半。三年,再下酒、曲之禁,戶私造差定其罪:城郭二十斤、鄉閭三十斤,棄市;民持私酒入京城五十里、西京及諸州城二十里者,至五斗處死;所定裏數外,有官署
 酤酒而私酒入其地一石,棄市。乾德四年,詔比建隆之禁第減之:凡至城郭五十斤以上、鄉閭百斤以上、私酒入禁地二石、三石以上、至有官署處四石,五石以上者,乃死。法益輕而犯者鮮矣。



 端拱二年令:民買曲釀酒酤者,縣鎮十里如州城二十里之禁。天聖以後,北京售曲如三京法,官售酒、曲亦畫疆界,戒相侵越,犯皆有法。其不禁之地,大概與宋初同,唯增永興軍、大通監、川峽之茂州、富順監。



 時天下承平既久,戶口浸蕃,為酒醪以靡
 穀者益眾。乾興初,言者謂:「諸路酒課,月比歲增,無有藝極,非古者禁群飲、教節用之義。」遂詔:「鄉村毋得增置酒場,已募民主之者,期三年;他人雖欲增課以售,勿聽;主者自欲增課,委官吏度異時不至虧額負課,然後上聞。」既而御史中丞晏殊請酒場利薄者悉禁增課。



 天聖七年,詔:「民間有吉兇事酤酒,舊聽自便,毋抑配。而江、淮、荊湖、兩浙酒戶往往豪制良民,至出引目,抑使多售。其嚴禁止,犯者聽人告,募人代之。」慶歷初,三司言:「陜西用兵,
 軍費不給,尤資榷酤之利。請較監臨官歲課,增者第賞之。」繼令蕭定基、王琪等商度利害。



 初,酒場歲課不登,州縣多責衙前或伍保輸錢以充其數,嘉禧、治平中,數戒止之。治平四年,手詔蠲京師酒戶所負榷錢十六萬緡,又江南比歲所增酒場,強率人酤酒者禁止。皇祐中,酒曲歲課合緡錢一千四百九十八萬六千一百九十六,至治平中,減二百一十二萬三千七百三;而皇祐中,又入金帛、絲纊、芻粟、材木之類,總其數四百萬七百六十,
 治平中,乃增一百九十九萬一千九百七十五。



 熙寧三年,詔諸郡遇節序毋得以酒相饋。初,知渭州蔡挺言:「陜西有醞公使酒交遺,至逾二十驛,道路煩苦。」詔禁之。至是,都官郎中沈行復言:「知莫州柴貽範饋他州酒至九百餘瓶,用兵夫逾一百人。」故並諸路禁焉。



 四年,三司承買酒曲坊場錢率千錢稅五十,儲以祿吏。六月,令式所刪定官周直孺言:「在京曲院酒戶鬻酒虧額,原於曲數多則酒亦多,多則價賤,賤則人戶損其利。為今之法,宜
 減數增價,使酒有限而必售,則人無耗折之患,而官額不虧。請以百八十萬斤為定額,閏年增十五萬斤。舊直,斤百六十八,百以八十五為數,後增為二百,百用足數,以便入出。」七年,諸郡舊不釀酒者許釀,以公使錢率百緡為千石,溢額者以違制論。在京酒戶歲用糯三十萬石。九年,江、浙災傷,米直騰貴,詔選官至所產地預給錢,俟成稔折輸於官。未幾,詔勿行,止以所糴在京新米與已糴米半用之。



 元豐元年,增在京酒戶曲錢,較年額損
 曲三十萬斤,閏年益造萬斤。二年,詔:「在京鬻曲,歲以百二十萬斤為額,斤直錢二百五十,俟鬻及舊額,令復舊價。酒戶負糟、糯錢,更期以二年帶輸,並蠲未請曲數十萬斤。」先是,京師曲法,自熙寧四年更定後,多不能償,雖屢閣未請曲數,及損歲額為百五十萬斤,斤增錢至二百四十,未免逋負。至是,命畢仲衍與周直孺講求利病,請:「損額增直,均給七十店,令日輸錢,周歲而足,月輸不及數,計所負倍罰;其炊醞非時、擅益器量及用私曲,皆
 立告賞法。」悉施行之,而裁其價。三年,詔:「帶輸舊曲錢及倍罰錢,仍寬以半歲,未經免罰者蠲三之一。」五年,外居宗室酒,止許於舊宮院尊長及近屬寄醞。增永興軍乾祐縣十酒場。酒戶負糟、糯錢,更令三年之內增月限以輸,並除限內罰息,其倍罰曲錢已蠲三之一,下戶更免一分。



 元祐元年,刪監司鬻酒及三路饋遺條。紹聖二年,左司諫翟思言:諸郡釀酒,非沿邊並復熙寧之數。詔:「熙寧五年以前,諸郡不釀酒、及有公使錢而無酒者,所釀
 並依《熙寧編敕》數。仍令諸郡所減勿逾百石,舊不及數者如舊,毋得於例外供饋。」後又以陜西沿邊官監酒務課入不足,乃令邊郡非帥府並酌條制定釀酒數,諸將並城砦止許於官務寄釀。



 崇寧二年,知漣水軍錢景允言建立學舍,請以承買醋坊錢給用。詔常平司計無害公費如所請,仍令他路準行之。初,元祐臣僚請罷榷醋,戶部謂本無禁文。後翟思請以諸郡醋坊日息用餘悉歸常平,至是,景允有請,故令常平計之。十月,諸路官監
 酒直,上者升增錢二,中下增一,以充學費,餘裨轉運司歲用。



 大觀四年,以兩浙轉運司之請,官監鬻糟錢別立額比較。又詔:「諸郡榷酒之池,入出酒米,並別遣倉官。賣醋毋得越郡城五里外,凡縣、鎮、村並禁,其息悉歸轉運司,舊屬常平者如故。



 政和二年,淮南發運副使董正封言:「杭州都酒務甲於諸路,治平前歲課三十萬緡,今不過二十萬。請令分務為三,更置比較務二,毋增官吏兵匠,仍請本路諸郡並增務比較。」從之。四年,兩浙轉運司
 亦請置務比較,定課額釀酒收息,以增虧為賞罰。詔:「酒務官二員者分兩務,三員者復增其一,員雖多毋得過四務。內有官雖多而課息不廣者,聽如舊。」是歲,以湖南路諸務糟酵錢分入提舉司,令斤增錢三,為直達糧綱水工之費。立酒匠闕聽選試清務廂軍之法。清務者,本州選刺供踏曲爨蒸之役,闕則募人以充。



 宣和二年,公使庫假用米曲及因耗官課者,以坐贓罪之,監官移替。三年,發運使陳遘奏:「江、淮等路官監酒直,上者升權增
 錢五,次增三,為江、浙新復州縣之用。」其後尚書省請令他路悉行之。詔如其請,所收率十之三以給漕計,餘輸大觀庫。五年,罷夔路榷酤,未幾復舊,以轉運司言新邊城砦藉以供億故也。六年,在任官以奉酒抑賣坊戶轉鬻者,論以違制律。先是,政和末,嘗詔毋得令人置肆以鬻,今並禁之。諸路增酒錢,如元豐法,悉充上供,為戶部用,毋入轉運司。七年,諸路鬻醋息,率十五為公使,餘如鈔旁法,令提刑司季具儲備之數,毋得移用。靖康元年,
 兩浙路酒價屢增,較熙、豐幾倍,而歲稔米曲直賤,民規利,輕冒法,遂令罷所增價。



 渡江後,屈於養兵,隨時增課,名目雜出,或主於提刑,或領於漕司,或分隸於經、總制司,惟恐軍資有所未裕。建炎三年,總領四川財賦趙開遂大變酒法:自成都始,先罷公帑實供給酒,即舊撲買坊場所置隔釀,設官主之,民以米入官自釀,斛輸錢三十,頭子錢二十二。明年,遍下其法於四路,歲遞增至六百九十餘萬緡,凡官槽四百所,私店不預焉,於是東南
 之酒額亦日增矣。四年,以米曲價高,詔上等升增二十文,上等升增十八文,俟米曲價平依舊。



 紹興元年,兩浙酒坊於買撲上添凈利錢五分,季輸送戶部。又增諸酒錢上升二十文,下十文。其諸州軍賣酒虧折,隨宜增價,一分州用,一分漕計,一分隸經制司。先是,酒有定價,每增須上請。是後,郡縣始自增,而價不一矣。五年,令諸州酒不以上下,升增五文,隸制總司。六年,以紹興二年以後三年中一年中數立額,其增羨給郡縣用。罷四川州、
 軍、縣、鎮酒官百七員,其酒息微處並罷之。



 七年,以戶部尚書章誼等言,行在置贍軍酒庫。四川制置使胡世將即成都、潼川、資、普、廣安立清酒務,許民買撲,歲為錢四萬八千餘緡。自趙開行隔槽法,增至十四萬六千餘緡,紹興元年。



 及世將改官監,所入又倍,自後累增至五十四萬八千餘緡,紹興二十五年。



 而外邑及民戶坊場又為三十九萬緡。淳熙二年。



 然隔槽之法始行,聽就務分槽醞賣,官計所入之米而收其課,若未病也。行之既久,醞賣虧欠,則責入
 米之家認輸,不復核其米而第取其錢,民始病矣。



 十年,罷措置贍軍酒庫所,官吏悉歸戶部,以左曹郎中兼領,以點檢贍軍酒庫為名,與本路漕臣共其事。十五年,弛夔路酒禁。以南北十一庫並充贍軍激賞酒庫,隸左右司。十七年,省四川清酒務監官,成都府二員,興元遂寧府、漢、綿、邛、蜀、彭、簡、果州、富順監並漢州綿州、縣各一員。



 二十一年,詔諸軍買撲酒坊監官賞格依舊。



 四萬、三萬貫已上場務:增及一倍,減一年磨勘,二倍減二年磨勘,三倍減三年磨勘,四倍減四年磨勘。二萬、一萬貫已上場務:增及
 一倍,減三季磨勘,二倍減一年磨勘,三倍減三年磨勘,七千貫以上場務:增及一倍,斤三季名次,二倍減一年磨勘,三倍減一年半磨勘,四倍減二年磨勘。七千貫以下場務:增及一萬貫減一年磨勘,二萬貫減二年磨勘,三萬貫減三年磨勘,四萬貫減四年磨勘。



 二十五年,罷諸路漕司寄造酒。二十七年,以隔槽酒擾民,許買撲以便民。罷官監,後復制之。



 三十年,以點檢措置贍軍酒庫改隸戶部。既而戶部侍郎邵大受等言:「歲計賴經、總制,窠名至多,今諸路歲虧二百萬,皆緣諸州公使庫廣造,別置店酤賣,以致酒務例皆敗壞。」詔罷諸州別置酒庫,如軍糧酒庫、防月庫、
 月樁庫之類,並省務寄造酒及帥司激買酒庫。凡未分隸經、總制錢處,並立額分隸,補趁虧額。三十一年,殿帥趙密以諸軍酒坊六十六歸之戶部,見九年。



 同安郡王楊存中罷殿帥,復以私撲酒坊九上之;歲通收息六十萬緡有奇,以十分為率,七分輸送行在,三分給漕計。蓋自軍興以來,諸帥擅榷酤之利,由是,縣官始得資之以佐經費焉。



 孝宗乾道元年,以浙東、西犒賞庫六十四隸三衙,輸課於左藏南庫,餘錢充隨年贍軍及造軍器。二年,
 詔:「臨安府安撫司酒庫悉歸贍軍;並贍軍諸庫及臨安府安撫司酒務,令戶部取三年所收一年中數立額。」日售錢萬緡,歲收本錢一百四十萬,息錢一百六十萬,曲錢二萬,羨餘獻以內藏者又二十萬,其後增為五十萬。四年,立場務官賞格。七年,以淮西總領周閟言,總所庫四,安撫司庫五,都統司庫十八,馬軍司庫一,增置行宮庫一,共為庫二十九,以三年最高年為額;其行宮新庫息錢,除分認諸處錢及縻費,以凈息三分為率,一分輸
 御前酒庫;以提領建康府戶部贍軍酒庫為名,遂鑄印及改庫名。八年,知常德府劉邦翰言:「江北之民困於酒坊,至貧乏家,不捐萬錢則不能舉一吉兇之禮。」乃檢《乾道重修敕令》,申嚴抑買之禁。淳熙三年詔:「四川酒課折估困弊,可減額錢四十七萬三千五百餘緡,令禮部給降度牒六百六十一道,補還今歲減數,明年於四川合給湖廣總所錢補之。」



 寧宗開禧元年,知臨安府兼點檢贍軍激賞庫趙善防、轉運判官提領戶部犒賞酒庫詹
 徽之言,官吏冗費,請諸司官屬兼管。明年,又以都省言課額失陷,依舊闢置。



 初,趙開之立隔釀法也,蓋以紓一時之急,其後行之諸郡,國家贍兵,郡縣經費,率取給於此。故雖罷行、增減,不一而足,而其法卒不可廢云。



 坑冶凡金、銀、銅、鐵、鉛、錫監冶場務二百有一:金產商、饒、歙、撫四州,南安軍。銀產鳳、建、桂陽三州,有三監;饒、信、虔、越、衢、處、道、福、汀、漳、南劍、韶、廣、英、連、恩、春十七州,建昌、邵武、南安三軍,有五十一場;秦、隴、興元三州,有三務。銅產
 饒、處、建、英、信、汀、漳、南劍八州,南安、邵武二軍,有三十五場;梓州有一務。鐵產徐、兗、相三州,有四監;河南、鳳翔、同、虢、儀、蘄、黃、袁、英九州,興國軍,有十二冶;晉、磁、鳳、澧、道、渠、合、梅、陜、耀、坊、虔、汀、吉十四州,有二十務;信、鄂、連、建、南劍五州,邵武軍,有二十五場。鉛產越、建、連、英、春、韶、衢、汀、漳、南劍十州,南安、邵武二軍,有三十六場、務。錫產河南、南康、虔、道、賀、潮、循七州,南安軍,有九場。水銀產秦、階、商、鳳四州,有四場。朱砂產商、宜二州,富順監,有三場。



 開寶三
 年,詔曰:「古者不貴難得之貨,後代賦及山澤,上加侵削,下益雕弊。每念茲事,深疚於懷,未能捐金於山,豈忍奪人之利。自今桂陽監歲輸課銀,宜減三分之一。」民鑄銅為佛像、浮圖及人物之無用者禁之,銅鐵不得闌出蕃界及化外。



 至道二年,有司言:「定州諸山多銀礦,而鳳州山銅礦復出,採煉大獲,而皆良焉。請置官署掌其事。」太宗曰:「地不愛寶,當與眾庶共之。」不許。東、西川監酒商稅課半輸銀帛外,有司請令二分入金。景德三年,詔以非
 土產罷之。



 天聖中,登、萊採金,歲益數千兩。仁宗命獎勸官吏,宰相王曾曰:「採金多則背本趨末者眾,不宜誘之。」景祐中,登、萊饑,詔弛金禁,聽民採取,俟歲豐復故。然是時海內承平已久,民間習俗日漸侈靡,糜金以飭服器者不可勝數,重禁莫能止焉。景祐、慶歷中,屢下詔申敕之,語在《輿服志》。大率山澤之利有限,或暴發輒竭,或採取歲久,所得不償其費,而歲課不足,有司必責主者取盈。仁宗、英宗每降赦書,輒委所在視冶之不發者,或廢
 之,或蠲主者所負歲課,率以為常;而有司有請,亦輒從之,無所吝。故冶之興廢不常,而歲課增損隨之。



 皇祐中,歲得金萬五千九十五兩,銀二十一萬九千八百二十九兩,銅五百一十萬八百三十四斤,鐵七百二十四萬一千斤,鉛九萬八千一百五十一斤,錫三十三萬六百九十五斤,水銀二千二百斤。



 其後,以赦書從事或有司所請,廢冶百餘。既而山澤興發,至治平中,或增冶或復故者六十有八,而諸州坑冶總二百七十一:登、萊、商、饒、
 汀、南恩六州,金之冶十一;登、虢、秦、鳳、商、隴、越、衢、饒、信、虔、郴、衡、漳、汀、泉、建、福、南劍、英、韶、連、春二十三州,南安、建昌、邵武三軍,桂陽監,銀之冶八十四;饒、信、虔、建、漳、汀、南劍、泉、韶、英、梓十一州,邵武軍,銅之冶四十六;登、萊、徐、兗、鳳、翔、陜、儀、邢、虢、磁、虔、吉、袁、信、澧、汀、泉、建、南劍、英、韶、渠、合、資二十四州,興國、邵武二軍,鐵之冶七十七;越、衢、信、汀、南劍、英、韶、春、連九州,邵武軍,鉛之冶三十;商、虢、虔、道、賀、潮、循七州,錫之冶十六;而水銀、丹砂州冶,與至道、天禧之
 時則一,皆置吏主之。是歲,視皇祐金減九千六百五十六,銀增九萬五千三百八十四,銅增一百八十七萬,鐵、錫場百餘萬,鉛增二百萬,又得丹砂二千八百餘斤,獨水銀無增損焉。



 熙寧元年,詔:「天上寶貨坑冶,不發而負歲課者蠲之。」八年,令近坑冶坊郭鄉村並淘採烹煉,人並相為保;保內及於坑冶有犯,知而不糾或停盜不覺者,論如保甲法。



 元豐元年,諸坑冶金總收萬七百一十兩,銀二十一萬五千三百八十五兩,銅千四百六十萬
 五千九百六十九斤,鐵五百五十萬一千九十七斤,鉛九百十九萬七千三百三十五斤,錫二百三十二萬一千八百九十八斤,水銀三千三百五十六斤,朱砂三千六百四十六斤十四兩有奇。



 先是,熙寧七年,廣西經略司言:「邕州右江填乃洞產金,請以鄧闢鑒金場。」後五年,凡得金為錢二十五萬緡,闢遷官者再焉。元豐四年,始以所產薄罷貢,而虔、吉州界鉛悉禁之。七年,戶部尚書王存等請復開銅禁,各展磨勘年有差。是歲,坑冶凡一
 百三十六所,領於虞部。



 紹聖元年,戶部尚書蔡京奏:「岑水場銅額浸虧,而商、虢間苗脈多,陜民不習烹採,久廢不發。請募南方善工詣陜西經畫,擇地興冶。」於是以許天啟同管幹陜西坑冶事。元符三年,天啟罷領坑冶,以其事歸之提刑司。初,新舊坑冶合為一司,而漕司兼領。天啟為同管幹,欲專其事,慮有所牽制,乃請川、陜、京西路坑冶自為一司,許檢束州縣,刺舉官吏,而漕司不復兼坑冶。至是,中書奏天啟所領,首末六歲,總新舊銅止
 收二百六萬餘斤,而兵匠等費繁多,故罷之。



 崇寧元年,提舉江、淮等路銅事游經言:「信州膽銅古坑二:一為膽水浸銅,工少利多,其水有限;一為膽土煎銅,無窮而為利寡。計置之初,宜增本損息,浸銅斤以錢五十為本,煎銅以八十。」詔用其言。諸路坑冶,自川、陜、京西之外,並令常平司同管幹。所收息薄而煩官監者,如元符、紹聖敕立額,許民封狀承買。四年,湖北旺溪金場,以歲收金千兩,乃置監官。廣東漕臣王覺自言嘗領常平,講求山澤
 之利,岑水一場去年收銅,比租額增三萬九千一百斤,較之常年亦增六十六萬一千斤。遂增其秩。是歲,山澤坑冶名數,令監司置籍,非所當收者別籍之,若弛興、廢置、移並,亦令具注,上於虞部。



 大觀二年,詔:「金銀坑發,雖告言而方檢視,私開淘取者以盜論。坑冶舊不隸知縣、縣丞者,並令兼監,賞罰減正官一等。」有冶地,知縣月一行點閱。言者論其職在宣導德澤,平征賦獄訟,不宜為課利走山谷間,遂已之。八月,提舉陜西坑冶司改並入
 轉運司。



 政和元年,張商英言:「湖北產金,非止辰、沅、靖溪峒,其峽州夷陵、宜郡縣,荊南府枝江、江陵縣赤湖城至鼎州,皆商人淘採之地。漕司既乏本錢,提舉司買止千兩,且無專司定額。請置專切提舉買金司,有金苗無官監者,許遣部內州縣官及使臣掌乾。」詔提舉官措畫以聞,仍於荊南置司。廣東漕司復奏:「端州高明、惠州信上立溪場皆宜停閉;韶州曹峒場、英州銀岡場皆並入英之清溪場,惟黃坑場欲權存,俟歲終會所入別奏;惠州
 楊梅東坑、康州雲烈、潮州豐政、連州元魚銅坑黃田白寶、廣州大利宜祿、韶州伍注岑水銅岡、循州大佐羅翊、英州鐘銅凡十六場,請並如舊;循之夜明、英之竹溪、韶之思溪、連之同安請更遣攝官。」從之。



 三年,尚書省言:「陜西路坑冶已遣官吏提轄措置,川路金銀坑治興廢,慮失利源。」詔:「令陜西措置官兼行川路事。坑冶所收金、銀、銅、鉛、錫、鐵、水銀、朱砂物數,令工部置籍簽注,歲半消補,上之尚書省。」自是,戶工部、尚書省皆有籍鉤考,然所憑
 唯帳狀,至有有額而無收,有收而無額,乃責之縣丞、監官及曹、部奉行者,而更督遞年違負之數。九月,措置陜西坑冶蔣彞奏:本路坑冶收金千六百兩,他物有差。詔輸大觀西庫,彞增秩,官屬各減磨勘年。四年,令監司遣官同諸縣丞遍視坑冶之利,為圖籍簽注,監司覆實保奏,議遣官再覆,酌重輕加賞,異同、脫漏者罪之。六年,川、陜路各置提轄措置。坑冶官劉芑計置萬、永州產金,一歲收二千四百餘兩,特與增秩。十二月,廣東漕司言:「本
 路鐵場坑冶九十二所,歲額收鐵二百八十九萬餘斤,浸銅之餘無他用。」詔令官悉市以廣浸,仍以諸司及常平錢給本。尚書省奏:「五路坑冶已有提轄措置專司,及淮南、湖北、廣東西亦監司領,其餘路請並令監司領之。」於是江東西、福建、兩浙漕臣皆領坑冶。



 七年,提舉東南九路坑冶徐禋奏:「太平瑞應,史不絕書。令部內山澤、坑冶,若獲希世珍物及古寶器,請赴書藝局上進。」蓋自政和初,京西漕臣王□奏太和山產水精,知桂州王覺奏
 枕門等處產金及生花金田,提轄京西坑冶王景文奏汝州青嶺鎮界產瑪瑙,其後湟州界蕃官結彪地內金坑千餘,收生熟金四等,凡百三十四兩有奇。蔡京請宣付史館,帥百官表賀,故禋復有是請焉。是時,河北、京東西及徐禋所領九路興修坑冶,類鑿空擾下,抑州縣承額,於是降黜河北提轄官,遣廉訪使者鄭諶並諸路廉訪悉究陳利病真偽。八月,中書奏坑冶浸已即緒,詔京東西、河北路並提舉東南九路坑冶並罷。十一月,尚書
 省言:「徐禋以東南黑鉛留給鼓鑄之餘,悉造丹粉,鬻以濟用。」詔諸路常平司以三十萬輸大觀西庫,餘從所請。



 明年,令諸路鐵仿茶鹽法榷鬻,置爐冶收鐵,給引召人通市。苗脈微者聽民出息承買,以所收中賣於官,私相貿易者禁之。先是,元豐六年,京東漕臣吳居厚奏:「徐、鄆、青等州歲制軍器及上供簡鐵之類數多,而利國、萊蕪二監鐵少不能給。請鐵從官興煽,所獲可多數倍。」自是,官榷鐵造器用以鬻於民,至元祐罷之。其後大觀初,入
 內皇城使裴絢為涇原乾當,奏上渭州通判苗沖淑之言:「石河鐵冶既令民自採煉,中賣於官,請禁民私相貿易。農具、器用之類,悉官為鑄造,其冶坊已成之物,皆以輸官而償其直。」乃禁毋得私相貿易,農具、器用勿禁,官自賣鐵唯許鑄瀉戶市之。



 政和初,臣僚言:「鹽鐵利均,今鹽筴推行已備,而鐵貨尚未講畫。請即冶戶未償之錢,收其已煉之鐵,為器鬻之。兼京東二監所出尤多,河北固鎮等冶並官監,其利不貲,而河東鐵、炭最盛,若官
 榷為器,以贍一路,旁及陜、雍,利入甚廣,且以銷盜鑄之弊。又夏人茶山鐵冶既入中國,乏鐵為器,聞以鹽易鐵錢於邊,若官自為器,則鐵與錢俱重,可伐其謀。請榷諸路鐵,擇其最盛者,可置監設官總之,概諸路不越數十處,餘止為鑄瀉之地,屬之都監或監當官兼領。凡農具、器用皆官鑄造,表以字號,官本之餘,取息二分以上,仍置鐵引以通諸路,儲其錢助三路鈔本。」詔戶部下諸路漕臣詳度。會次年,廣東路請以可監之地如舊法收其凈
 利,苗脈微者召人承買,官不榷取,遂並諸路詳度之旨不行。至是,臣僚復以為言,故嚴貿易之禁,而鐵利盡榷於官,然農具、器用從民鑄造,卒如舊法。



 四月,廣東廉訪黃烈等言:「廣、惠、英、康、韶州、興慶府,政和中,寶貨司立坑冶金銀等歲額,或苗脈微,或無人承買,而浮冗之人虛托其名,發毀民田,騷動邀販。」詔:「政和六年所立額並罷,舊有苗脈可給歲課者如故。」十一月,復諸路元罷提舉坑冶官,其江南路仍令江西漕臣劉蒙同措置。



 宣和元
 年,石泉軍江溪沙磧麩金,許民隨金脈淘採,立課額,或以分數取之。十月,復置相州安陽縣銅冶村監官。先是,詔留邢州綦村、磁州固鎮兩冶,餘創置冶並罷,而常平司謂銅冶村近在河北,得利多,故有是命。六年,詔:「坑冶之利,二廣為最,比歲所入,稽之熙、豐,十不逮一。令漕臣鄭良提舉經畫,分任官屬典掌計置,取元豐以來歲入多數立額,定為常賦,坑冶司毋預焉。」時江、淮、荊、浙等九路,坑冶凡二百四十五,鑄錢院監十八,歲額三百餘萬
 緡。五月,詔:「坑冶舊隸轉運司者,如熙、豐、紹聖法;崇寧以後隸常平司者,如崇寧法;其江、淮等路坑冶官屬,如熙、豐員數,餘路官屬並罷,仍令中書選提點官。」



 靖康元年,諸路坑冶苗礦既微,或舊有今無,悉令蠲損,凡民承買金場並罷。宋初,舊有坑冶,官置場監,或民承買以分數中賣於官。初隸諸路轉運司,本錢亦資焉,其物悉歸之內帑。崇寧已後,廣搜利穴,榷賦益備。凡屬之提舉司者,謂之新坑冶,用常平息錢與剩利錢為本,金銀等物往
 往皆積之大觀庫,自蔡京始。政和間數罷數復,然告發之地多壞民田,承買者立額重,或舊有今無,而額不為損。欽宗即位,詔悉罷之。



 南渡,坑冶廢興不常,歲入多寡不同。今以紹興三十二年金、銀、銅、鐵、鉛、錫之冶廢興之數一千一百七十,及乾道二年鑄錢司比較所入之數附之:



 湖南、廣東、江東西金冶二百六十七,廢者一百四十二;湖南、廣東、福建、浙東、廣西、江東西銀冶一百七十四,廢者八十四;潼川、湖南、利州、廣東、浙東、廣西、江東西、
 福建銅冶一百九,廢者四十五。舊額歲七百五萬七千二百六十斤有奇,乾道歲入二十六萬三千一百六十斤有奇。



 淮西、夔州、成都、利州、廣東、福建、浙東、廣西、江東西鐵冶六百三十八,廢者二百五十一,舊額歲二百一十六萬二千一百四十斤有奇,乾道歲入八十八萬三百斤有奇。



 淮西、湖南、廣東、福建、浙東、江西鉛冶五十二,廢者一十五,舊額歲三百二十一萬三千六百二十斤有奇,乾道歲入一十九萬一千二百四十斤有奇。



 湖南、
 廣東、江西錫冶一百一十八,廢者四十四,舊額歲七十六萬一千二百斤有奇,乾道歲入二萬四百五十斤有奇。



 宋初,諸冶外隸轉運司,內隸金部;崇寧二年,始隸石曹;建炎元年,復隸金部、轉運司。隆興二年,坑冶監官歲收買金及四千兩、銀及十萬兩、銅錫及四十萬斤、鉛及一百二十萬斤者,轉一官;守倅部內歲比租額增金一萬兩、銀十萬兩、銅一百萬斤,亦轉一官;令丞歲收買及監官格內之數,減半推賞。



 慶元二年,宰執言:「封樁銀
 數比淳熙末年虧額幾百五十萬。今務場所入歲不滿三十萬,而歲奉三宮及冊寶費約四十萬,恐愈侵銀額。欲權以三分為率,一分支銀,二分支會子。」上曰:「善。」



 端平三年,赦曰:「諸路州縣坑冶興廢,在觀寺、祠廟、公宇、居民墳地及近墳園林地者,在法不許人告,亦不得受理。訪聞官司利於告發,更不究實,多致擾害。自今許人戶越訴,官吏並訟者重置典憲。及有坑冶停閉、苗脈不發之所,州縣勒令坑戶虛認歲額,提點鑄錢司核實追正。」



 礬唐於晉州置平陽院以收其利。開成三年,度支奏罷之,乃以礬山歸之州縣。五代以來,復創務置官吏,宋因之。



 白礬出晉、慈、坊州、無為軍及汾州之靈石縣,綠礬出慈、隰州及池州之銅陵縣,皆設官典領,有鑊戶鬻造入官市。晉、汾、慈州礬,以一百四十斤為一馱,給錢六十。隰州礬馱減三十斤,給錢八百。博賣白礬價:晉州每馱二十一貫五百,慈州又增一貫五百;綠礬:汾州每馱二十四貫五百,慈州又增五百,隰州每馱四貫六百。散賣白
 礬:坊州斤八十錢,汾州百九十二錢,無為軍六十錢;綠礬,斤七十錢。



 建隆中,詔:「商人私販幽州礬,官司嚴捕沒入之。」繼定私販河東幽州礬一兩以上、私鬻礬三斤、及盜官礬至十斤者,棄市。開寶三年,增私販至十斤、私鬻及盜滿五十斤者死,餘罪論有差。太平興國初,以歲鬻不充,乃詔私販化外礬一兩以上、及私鬻至十斤,並如律論決,再犯者悉配流,還復犯者死。淳化元年,有司言:「慈礬滯積,小民多於山谷僻奧之地私鬻侵利。而綠礬
 價賤,不宜與晉礬均法。」詔同犯私茶罪賞。



 先是,建隆二年,命左諫議大夫劉熙古詣晉州制置礬,許商人輸金銀、布帛、絲綿、茶及緡錢,官償以礬,凡歲增課八十萬貫。太平興國初,歲博緡錢、金銀計一十二萬餘貫,茶計三萬餘貫。端拱初,銀、絹帛二萬餘貫,茶計十四萬貫。」至是,言者謂:「礬直酬以見錢,商人以陳茶入博,有利豪商,無資國用。」詔今後惟聽金銀、見錢入博。



 至道中,白礬歲課九十七萬六千斤,綠礬四十萬五千餘斤,鬻錢一十七萬
 餘貫。真宗末,白礬增二十萬一千餘斤,綠礬增二萬三千餘斤,鬻錢增六萬九千餘貫。天聖以來,晉、慈二州礬募民鬻之,季鬻礬一盆,多者千五、六百斤,少者六、七百斤,四分輸一入官,餘則官市之。無為軍亦置務鬻礬,後聽民自鬻,官置場售之,私售礬禁如私售茶法。六年,詔弛兩蜀榷礬之禁。



 時河東礬積益多,復聽入金帛、芻粟。芻粟虛估高,商人利於入中。麟州粟鬥實直錢百,虛估增至三百六十,礬之出官為錢二萬一千五百,才易粟
 六石,計粟實直錢才六千,而礬一馱已費本錢六十。縣官徒有榷礬之名,其實無利。嘉祐六年,罷入芻粟,復令入緡錢。礬以百四斤為一馱,入錢京師榷貨務者,為錢十萬七千;入錢麟、府州者,又減三千。自是商賈不得專其利矣。皇祐中,晉、慈入礬二百二十七萬三千八百斤,以易芻粟之類,為緡錢十三萬六千六百;無為軍礬售緡錢三萬三千一百。治平中,晉、慈礬損一百九萬六千五百四斤;無為軍礬售錢歲有常課,發運使領之,視皇
 祐數無增損;隰州礬至是入三十九萬六千斤,亦以易緡錢助河東歲糴。



 熙寧元年,命河東轉運司經畫礬、鹽遺利。李師中言:「官積礬三百斤,走鹵消耗,恐後為棄物。」詔令商人入中糧草,即以償之。三年,罷潞州交子務,以妨中納糧草、算請礬鹽故也。知慶州王廣淵言:「河東,礬為利源之最,請河東、京東、河北、陜西別立礬法,專置提舉官。」詔遣光祿丞楊蟠會議以聞。蟠言:「坊州產礬,官雖置場,而商多私售。請置鑊戶,定其數,許於陜西北界黃
 河,東限潼關,南及京西、均、房、襄、鄧、金州、光化軍,令鑊戶遞相保察。或私賣越界,禁如私白礬法,仍增官獲私礬輒以夾雜減斤重之法。」從之。



 元豐元年,定畿內及京東、西五路許賣晉、隰礬;陜西自潼關以西、黃河以南,達於京西均、房、襄、鄧、金州則售坊州礬;礬之出於西山、保霸州者,售於成都、梓州路;出無為軍者,餘路售之。私鬻與越界者,如私礬法。



 自熙寧初,礬法始變。歲課所入,元年為錢三萬六千四百緡有奇,並增者五年,乃取熙寧六
 年中數,定以十八萬三千一百緡有奇為新額;至元豐六年,課增至三十三萬七千九百緡,而無為軍礬歲課一百五十萬斤,用本錢萬八千緡;自治平至元豐數無增損。



 元祐元年,戶部言:「商旅販礬,舊聽其便。乃者發運司請用河東例,令染肆鋪戶連保豫買,頗致抑擾。」詔如舊制。元符三年,崇儀使林像奏:「禁河北土礬非便。若即河北產礬地置場官買,增價出之,罷運晉礬,則官獲凈利,無運載之勞,民資地產,省犯法之弊。」詔下戶部。



 初,熙、
 豐間,東南九路官自賣礬,發運司總之。元祐初通商,紹聖復熙、豐之制。大觀元年,定河北、河東礬額各二十四萬緡,淮南九萬緡,罷官賣,從商販,而河東、河北、淮南各置提舉官。政和初,復官鬻,罷商販如舊制。淮南礬事司罷歸發運司,上供礬錢責以三萬三千一百緡為額。三年,有司奏減河北、河東並淮南礬額,計十六萬緡。四年,礬額復循大觀之制。五年,河北、河東綠礬聽客販於東南九路,民間見用者,依通商地籍之,聽買新引帶賣,大
 率循仿鹽法。宣和中,舉比較增虧賞罰,未幾,以擾民罷。



 建炎三年,措置財用黃潛厚奏許商人販淮南礬入東南諸路,聽輸錢行在,而持引據赴場支礬。



 紹興十一年,以鑄錢司韓球言,撫州青膽礬斤錢一百二十文,土礬斤三十文省,鉛山場所產品高於撫,青膽礬斤作一百五十文,黃礬斤作八十文。二十九年,以淮西提舉司言,取紹興二十四年至二十八年所收礬錢一年中數四萬一千五百八十五緡為定額。其它產礬之所,若潭州
 瀏陽之永興場、韶州之岑水場,皆置場給引,歲有常輸。惟漳州之東,去海甚邇,大山深阻,雖有採礬之利,而潮、梅、汀、贛四州之奸民聚焉,其魁傑者號大洞主、小洞主,土著與負販者,皆盜賊也。



 香宋之經費,茶、鹽、礬之外,惟香之為利博,故以官為市焉。建炎四年,泉州抽買乳香一十三等,八萬六千七百八十斤有奇。詔取赴榷貨務打套給賣,陸路以三千斤、水路以一萬斤為一綱。



 紹興元年,詔:「廣南市舶司抽買
 到香,依行在品答成套,召人算請,其所售之價,每五萬貫易以輕貨輸行在。」六年,知泉州連南夫奏請,諸市舶綱首能招誘舶舟、抽解物貨、累價及五萬貫十萬貫者,補官有差。大食蕃客囉辛販乳香直三十萬緡,綱首蔡景芳招誘舶貨,收息錢九十八萬緡,各補承信郎。閩、廣舶務監官抽買乳香每及一百萬兩,轉一官;又招商入蕃興販,舟還在罷任後,亦依此推賞。然海商入蕃,以興販為招誘,僥幸者甚眾。



 淳熙二年,郴、桂寇起,以科買乳
 香為言。詔:「湖南路見有乳香並輸行在榷貨務,免科降。」十二年,分撥榷貨務乳香於諸路給賣,每及一萬貫,輸送左藏南庫。十五年,以諸路分賣乳香擾民,令止就榷貨務招客算請。



 紹熙三年,以福建舶司乳香虧數,詔依前博買。開禧三年,住博買。嘉定十二年,臣僚言以金銀博買,洩之遠夷為可惜,乃命有司止以絹帛、錦錡、瓷漆之屬博易。聽其來之多寡,若不至則任之,不必以為重
 也。



\end{pinyinscope}