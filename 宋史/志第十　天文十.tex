\article{志第十 天文十}

\begin{pinyinscope}

 流隕一



 流隕



 建隆元年正月戊午,有星出東北方,青赤色,北行,初小後大,尾跡斷續,光燭地。四月,有星出天市垣。六月癸酉,
 有大星赤色,出心大星。甲申,有星赤色,出太微垣,歷上相。乙未,有大星色赤,流虛東北。九月癸亥,有星出昴。甲子,有星如缶,出卯,光明燭地。十二月戊辰,有星青赤色,出參旗西南,慢行而沒,蒼光燭地。三年六月丁酉,有星出天市,入南斗魁。



 乾德元年二月丙午,有星如桃,色赤,出弧矢東南沒,有光明。二年二月乙丑,有星黃白色,出太微五帝南,速行至外廚沒,其體散落,光燭地。三年六月丁巳,有星如桃,色黃赤,出北斗魁,經太微垣北,過角
 宿西,漸大,行五尺餘,沒,尾跡凝天有光明。十二月丁巳,有星出天河,青白色,南行至天倉沒,初小後大,光濁地。四年正月乙未,有星出天社,青白色,速行,尾跡三丈餘,初小後大,沒,有光明。四月甲寅,有星出天乳,青赤色,東南行,貫房沒,光燭地。閏八月己丑,有星出天船,青白色,西北速行,沒於文昌。



 開寶元年七月戊子,有星出大角,青白色,北行沒,明燭地。九月戊子,有星出文昌,赤黃色,東北速行而沒。二年六月己卯,有星出河鼓,慢行,明燭
 地。三年九月庚午,廣州民見眾星皆北流。四年八月辛卯,有星出織女,西北行,尾跡三丈餘,沒,久有聲。五年八月乙巳,有星出王良,西北行,四丈餘,有聲而散。七年九月甲午,有星出室,西北行,星體散落有聲,明燭地。



 太平興國三年十月甲寅,有星出天船,赤黃色,至天棓,星體散落,明燭地。八年三月丙寅,有星晝出西南,當未地,青白色,尾跡二丈餘,沒於東南,有光明。七月辛巳,有星如稱權,沒於婁。八月壬寅,有星出紫微鉤陳東,赤黃色,向
 北速行,近北極沒。



 雍熙元年十月丁酉,有星出昴,赤色,東南蛇行二丈餘,沒。二年正月壬戌,有星出東井,其大倍於金星,入輿鬼沒。四年六月庚戌酉初,有星出西北,色青白,入濁,當戌地,有聲如雷。八月乙亥,有星出天關東,色赤黃,尾貫月。



 端拱元年四月辛亥,有星出天津,赤黃色,蛇行,有聲,明燭地,犯天津東北。閏五月辛亥醜時,有星出奎,如半月,北行而沒。乙卯,有星出紫微鉤陳西,色青,尾跡短,赤光照地,北行而沒。九月癸丑,有星出西
 南,如太白,有尾跡,至中天。旁出一小星,行丈餘,又出一小星,相隨至五車沒。二年四月辛亥戌時,有星出東南,色白,墜於氐、房間。壬申,有星出漸臺,血色赤,東南急行,掩左旗,過河鼓沒。



 淳化元年九月辛巳,有星出羽林,色青,南行,光奪月。十一月壬午,流星出天關,南行,歷天井、郎位、攝提,至大角東北墜於地,光芒四照,聲如隤墻。二年正月丙申,有星出水府西,色赤黃,經參旗分為三星,相從至天苑東沒,光燭地。七月癸酉,有星出雲雨側,色
 青白,緩行三尺餘,沒。二年三月己酉未時,西北方有星西北速行,色青白,有尾跡。四月己卯,有星出文昌,西南速行至柳,分為二星而沒。六月己丑,有星出天市垣屠肆東,色青白,西北慢行丈餘,分為三星,從而沒。四年五月乙未平明,有星東南出南斗,色青白,西北行而沒。五年八月己酉,常星未見,有星出東方,色青白,東北慢行,至濁沒,大約出奎、婁間。九月庚午,有星出昴北,緩行,過卷舌,至礪石沒。



 至道元年四月乙巳,常星未見,有星出
 心北,色青赤,急行而墜。七月癸丑,有星出危,色青白,入羽林沒。二年五月辛丑,有星出紫微北,尾跡丈餘,如彗而有聲,墜於壁、室間。五月己未,日未及地五尺間,有星出中天,色赤黃,有尾跡,東行速行二丈餘,沒。六月己卯,有星出牽牛西,歷狗國,光芒丈餘,墜東南,及地無聲。又有星出翼,貫天廟,墜於稷星東,光燭地。九月丁酉平明,有星出北方,東行三丈餘,分為三星,從而沒。三年九月丁丑,有星二,隕於西南,一出南斗,一出牽牛,有光三丈
 許。



 咸平五年三月丙午,有星晝出心,至南斗沒,赤光丈餘。八月辛巳,有星出營室,色白。丙申,有流星出東方,西南行,大如斗,有聲若牛吼,小星數十隨之而隕。戊戌,又有星十數入輿鬼,至中臺,凡一大星偕小星數十隨之,其間兩星,一至狼星,一至南斗沒。丁未,有星晝出紫微垣,貫北斗沒。壬子,有星出中天,尾跡數道如迸火,西流至狼、弧沒。六年五月乙未,有星出王良西,又出北極稍東北,至垣外沒,有聲如雷。六月庚午,有星晝出東北方,
 色黃白,有尾跡。七月壬辰,有星出昴,尾跡丈餘,色白,隱隱有聲,至狼星沒。十一月癸丑,有星出畢,至屏星北沒,尾跡蛇行,屈曲三丈餘,久方沒。十二月乙酉,威虜軍有星歷城西北,尾跡長數里,光照地,落蕃帳,有聲如雷者三。



 景德元年六月戊午,有星晝出西南方,赤黃,有尾跡,速流丈餘,沒。十月戊申,天雄軍有星出北方,隕於西北,光丈餘。十二月庚辰,有星出文昌,慢行西北,分為數星,至紫微垣東北沒。戊子,有星出昴,至參旗,迸為數星沒。
 二年正月丙子,日未沒,有星速流西南。二月己亥,有星出太微上將,光燭地。四月癸卯,有星北流入天倉,尾跡丈餘。十月戊寅,有星出太微垣內屏北,至翼分為三星,隨而沒,尾跡青白色。十一月壬子,有星出南晝,聲如雷,光燭地。三年五月乙卯,有星出天津東北、紫微垣北,分為四星,隨而沒,赤黃,有尾跡。六月乙亥,有星出雲雨星北,至羽林天軍南,迸為三星沒。丁酉,有星出胃北,入天囷,迸為數星,光燭地。七月庚申,有星出靈臺,有炬彗,聲
 如雷,至南北沒,赤光燭地。十一月辛丑,有星出中臺東北,速流,有聲,光燭地。四年三月庚申,有星晝出南方。六月丙辰,有星出北方,慢流至八穀,迸為數星沒,光燭地。己未,有星出天市,分為三星,至尾沒。



 七月辛卯,有星出敗瓜南,慢流,歷河鼓,入天市,至宗人東北,迸為二星沒,色赤黃,有尾跡。十二月癸巳,有星出弧矢,赤黃色,尾跡丈餘,光燭地,速流入濁。



 大中祥符元年二月戊申,有星十餘,急流入濁,色赤黃,有尾跡。五月辛未,有星如太白,
 出天市垣宗人東南,尾跡丈餘,闊三寸,向北慢流,至女床西,分為數星沒。六月戊申,有星出北斗魁內,赤黃,有尾跡,稍北速行,迸為數星沒。八月己丑,有星晝出中天,如太白,有尾跡,急流東南,近日沒。九月乙丑,有星出天倉,急流東南,星體散落。二年三月己未,有星出天津南,至離珠沒,尾跡五丈餘,照地明。四月丙申,有星出八穀,有尾跡,速流而西,至五車東,迸為數星沒。五月乙亥,有星晝出東方,如太白,尾跡赤黃,流至日北沒。八月丙申,
 有星出北斗杓,西南急行,至郎將西,分為數點。九月乙丑,有星出南河,如桃,色赤,至中臺沒。三年三月丁未,有星出天市宗人東北,尾跡二丈,至左旗,迸為數星沒,光燭地。五月丁亥,有星出北斗魁,如桃,色青白,尾跡二丈餘。六月丁巳,有星出文昌,至上臺沒。乙卯,有星出傳舍,如桃,色赤黃,至紫微沒。壬申,有星出建星,入南斗沒,赤黃,有尾跡。七月庚辰,有星出宗人西,北流入濁,光照地。八月丁未,有星出貫索,至帝座沒,尾跡光明。壬戌,有星
 出文昌,至北極沒,尾跡丈餘。九月庚辰,有星出軒轅左,入太微垣沒。十月庚戌,有星出東方,赤黃,無尾跡,分為數星,稍南沒。四年二月辛亥,有星出東方,尾跡赤黃,二丈餘。四月乙丑,有星出柳,色赤黃,至翼沒。五月戊子,有星出東方,赤黃色。六月壬戌,有星出觜東北,流入濁。七月壬申,有星出紫微宮,速流至天皇沒。戊寅,有星自內階流經文昌,至上臺,迸為數星,隨而沒。十月戊午,有星出東北,入濁。又星出七星南,至天稷沒,尾跡丈餘。五年
 二月戊申,有星出貫索,經庫樓,迸為數星沒。八月戊午,有星大小二十餘,皆有尾跡,北流。又一星光燭地,出紫微垣外,尾丈餘,闊三寸許,東北流,至傳舍沒。庚申,星出天耗北,尾跡十丈餘,明燭地,至文昌沒。六年乙巳,有星晝出南方,赤光迸逸,照地明。十一月丁巳,有星出太微郎位東,色赤黃,有尾跡,至軫北,迸為數星沒。十二月癸亥,有星出西南,色青白,入東北沒。七年三月內戌,有星出南河,大如杯,至玉井沒。四月辛酉,星出鉤陳,尾跡赤
 黃。七月丁未,有星晝出東南方,色黃,急流而北。九月辛亥,有星出軍市,至柳,迸為三星沒。十一月癸未,有星晝出日西南,尾跡二丈餘,闊三寸許,青白色,西流而沒。己丑,有星出南河,至弧矢沒,光燭地。八年二月丁卯,有星出郎將北,迸為三星。四月癸丑,有星出亢西,至右攝提,迸為數星,隨而沒。五月乙酉,有星青白色,出人星,至騰蛇沒,光燭地。丙申,有星西南流,迸為數星沒,明照地。八月己亥,有星出參,南流入濁。九年四月庚子,有星晝出,
 赤黃色,急流西北沒。



 天禧元年四月己巳,有星出軫,至器府北沒,光照地。六月,有星出河鼓,速流至天田,迸為數星沒。十二月癸巳,有星出東北,尾跡赤黃,急流西南沒。二年八月乙卯,有星二,有尾跡,赤黃,一出五車,一出狼北,入濁。戊午,有星出酒旗,至明堂沒,光燭地。九月戊子,有星出西南,至天園沒。十一月辛酉,有星出南河,色赤黃,至柳沒。三年六月乙巳,有星出昴,急流至天倉沒。十二月壬寅,有星出軒轅,尾跡黃,慢流至太微垣,久之,
 有聲如雷。四年正月丁丑,有星出王良,明照地,至騰蛇沒。五年四月丙辰,有星出軒轅前星,大如桃,狀若粉絮,犯次將,入太微垣,歷屏星,凡七十五日,入濁沒。己未,有星出南方,如二升器,色青赤,北流入濁,尾跡三丈許。七月辛巳,有星出文昌,光明燭地。十月乙巳,有星出天津西。



 乾興元年三月庚寅,夜漏未上,星出七星,曳尾緩行,至翼沒。五月己巳,星出天棓,速行入紫微極星西沒。癸酉,星出張,西北入濁。壬午,星出危,赤黃,有尾跡,速行而
 東,炸烈如迸火,隨至羽林軍南沒,明燭地。己丑,星出北河,至軒轅沒。九月己巳,星出羽林,流至芻稿沒。己丑,星出天市垣旁,緩行經天,過天市垣,至營室沒。壬辰,星出營室,行至天倉沒。十月丁酉,星出右旗,如太白,西南速行,至天弁沒,明燭地。十一月壬辰,常星未見,有星出五車,南行至奎沒。



 天聖元年正月丙戌,星出北斗魁西,至八穀沒。三月戊辰,星出貫索,至五車沒。六月戊戌,星出天弁,至建星沒。己丑,星出北斗星,東北入濁沒。庚寅,星
 出五車,至五諸侯沒。閏九月癸巳,星出五車,至參沒。丙申,星出東壁,至天倉沒。甲辰,常星未見,星出營室,至外屏沒。己酉,星出翼,南行入濁。二年辛丑,星出五車,至畢沒。六月丁卯,晝漏上,星出中天,赤黃色,有尾跡,西南緩行入濁。辛巳,星出牽牛,南入濁。九月辛卯,星出太微,沒於右執法。四年正月壬午,星出亢,東南流入濁。丁巳,星出靈臺,至翼沒。丙午,星出北斗魁,近文昌沒。其夜,又有星出箕,南行入濁。四月丙寅,星出太微從官側,南行入
 濁。五月辛巳,星出天市垣市樓側,東北流入濁。閏王月丙辰,星出天船,沒於紫微鉤陳側。六月乙亥,星出土司空,東南入濁。八月乙未,星出天棓,近天倉沒。



 九月丁未,星出王良,西北入濁。十一月丙辰,星出東井,沒於南河側。十二月丁丑,星出鉤陳,沒於天棓側。戊戌,星出太微,至文昌沒。五年正月壬寅,星出天社,西南入濁。九月癸卯,星出天廚,北流入濁。丁未,星出北辰,沒於天床側。甲子,有星出北河,沒於東井。六年四月甲申,夜漏欲盡,有
 星大如斗器,自北方至於西南,光照地,有聲如雷,曳尾跡長數丈,久之,散為蒼白雲。七年二月乙丑,星出天乳,貫天市,入濁。八年二月丁酉,星出軒轅大星側,如杯,速行至器府沒。



 明道元年三月癸巳,星出中臺,貫北河,入東井沒,炸烈有聲,明燭地。食頃,又有星出天市垣宗人側,東流入濁。四月乙巳,星出貫索,大如杯,沒於鉤陳側,光照地。八月癸亥,星出天船,近鉤陳沒,明燭地。乙丑,星出胃,大如杯,有尾跡,西北緩行,迸為六七小星,相隨沒
 於大陵,明燭地。丙寅,星出營室,西南速行,至危沒。良久,又有星出天園,至天社沒,光燭地。九月丙子,星出婁,沒於雲雨側,尾跡久方散。食頃,又有星出天大將軍,近奎沒,尾跡久方散,明燭地。續又星出北辰,西北速行,至內階沒。又有星出天苑,沒於天園,明燭地。



 景祐元年八月己卯,星出東井,行至廁星沒,尾跡久方散,明燭地。乙酉,星出北斗魁,西北速行,入紫微東南垣沒。又有星出文昌,西北速行,至紫微鉤陳沒,尾跡久方散,明燭地。九月
 丁亥,星出天津,如太白,青色,有尾跡,沒於危。良久,星出五車,沒天廩。己丑,星出東井,如太白,赤黃色,有尾跡,向東速行,至柳沒,光照地。其夜,星出婁,至奎沒,明燭地。十一月乙卯,星出軒轅大星側,如太白,赤黃,向東速行,入濁,明照地。二年八月庚申,星出大陵,如太白,赤黃色,東南緩行,沒於昴,尾跡久方散,明燭地。九月丙午,常星未見,星出婺女,緩行,近南斗沒。十一月辛丑,星出五車,至觜觿沒,明燭地。四年閏四月癸未,夜漏未上,星出天津,
 大如杯,東北行入濁。己亥,星出上臺,至軒轅沒。五月辛亥,星出華蓋,至北辰沒。六月壬申,星出天津,入天市垣,至宗人沒。是夜,星出王良。如太白,青白色,有尾跡,東南速行,至婁沒,明燭地。己卯,星出梗河,沒於亢。七月戊申,有星數百皆西南流,其最大者一星至東壁沒,光燭地,久之不散。九月庚子,星出南河,東南速行,近狼星沒,青白色,有尾跡如太白,明燭地。己酉,星出牽牛,如太白,青白色,西南入濁。丁卯,星出紫宮,沒天棓,有尾跡,明燭地。



 寶元元年正月戊戌,星出左攝提,如太白,赤黃色,至天市西垣沒,明燭地。二月甲午,星出河鼓,至七公沒。三月辛丑,星出東井,沒參側。庚戌,星出大角,至氐沒。辛亥,星出北斗魁,如太白,青白色,有尾跡,東北速行入濁,光照地。四月壬申,有星出中臺,如太白,青白色,有尾跡,向北速行入濁,明燭地。又星出天江,如太白,有尾跡,西南速行,至房沒。八月壬申,星出東井,如太白,東北速行,沒輿鬼,明燭地。十月壬午,星出天津,至營室沒。己丑,星出東
 井,如太白,赤黃,有尾跡,至狼側沒,明燭地。十一月癸丑,星出中臺,至軒轅沒。二年正月庚申,星出翼,如太白,行至角沒。三月癸丑,星出右旗,赤黃,有尾跡,向南速行,沒於建星,明燭地。五月庚戌,星出房,至積卒沒。閏十二月甲寅,星出文昌,如太白,有尾跡,西北速行,至五車沒,明燭地。



 康定元年三月戊寅,有星出文昌,如太白,青白色,北行入濁。四月丁未,有星出紫宮東垣上衛側,至北辰沒。癸丑,星出北斗,北行入濁。六月庚戌,星出天弁,西北
 入濁,明燭地。九月戊寅,星出天船,東行,入五車沒。十月壬辰,星出天津,速行至紫宮西垣沒。壬戌,中天有星大如杯,赤黃,有尾跡,西南速行,沒於濁,光照地,良久,有聲如雷。十一月乙亥,星出文昌,北行,明燭地,入濁。



 慶歷元年八月癸未,星出天船,如太白,東北速行入濁,青白色,明燭地。己亥,星出奚仲,大如杯,色青白,西南緩行,沒於天津側,明燭地。辛丑,有星經天廩,東南緩行入濁。乙巳,夜漏未上,星出營室,如太白,東行入濁,青白色。九月己
 酉,星出奎,如太白,有尾跡,西行,沒於東壁,明燭地。丙辰,星出畢,如太白,有尾跡,西北速行,至王良沒。丁卯,星出北辰,如太白,北行入濁,明燭地。戊辰,星出壁壘陣,如太白,赤黃,有尾跡,西南入濁,明燭地。二年二月庚子,星出房,如太白,赤黃,有尾跡,西南速行,入濁沒,明燭地。三月戊寅,星出鉤陳側,如太白,赤黃,有尾跡,西行緩行,至天棓沒,明燭地。四月丁丑,星出貫索,大如醆,青白色,有尾跡,東北慢行,至閣道沒,明燭地。丙申,星出貫索,如太白,
 赤黃色,西北速行,沒於中臺側,明燭地。七月壬寅,星出河鼓,大如杯,青白色,西速行,至牽牛沒,明燭地。己酉,星出婺女,如太白,青白色,有尾跡,東南慢行入濁,明燭地。乙丑,星出天津,如太白,赤黃,向西速行,至貫索沒,尾跡久方散,明燭地。八月壬寅,星出北斗杓,如太白,青白色,西北行,沒於濁。乙亥,夜漏未上,星出箕,南行入濁。又有星出天倉,如太白,東南入濁沒。壬午,星出危,東南行,至濁沒。九月辛亥,星出天船,如太白,東行入濁,青白色,有
 尾跡。庚申,星出婁,至東壁沒。乙丑,星出婁,至天倉沒。丁卯,星出五車,東北流,沒於文昌側。閏九月辛未,星出羽林軍,如太白,赤黃色,西南行入濁。乙亥,星出婁,西行入濁。十二月庚申,有星出弧矢,南行入濁,赤黃,有尾跡,燭地。三年二月壬寅,星出上臺,至軒轅沒,有尾跡,明燭地。



 四月戊申,夜漏未上,中天星出大角,如太白,西行至軒轅沒。辛亥,星出女床,至天市西垣沒。丙辰,星出牽牛,如太白,西南緩行,至天淵沒。七月己卯,星出北斗魁,西北
 行入濁。甲申,星出貫索,如太白,速行至北斗柄沒。甲寅,星出閣道,如太白,東北速行入濁,有尾跡,明燭地。十月戊申,星出柳,如太白,西南速行,至弧矢沒,尾跡久方散。五年五月辛巳,星出紫宮鉤陳側,北行入濁。六月辛酉,星出奎,如太白,西行,至天倉沒,有尾跡,明燭地。壬戌,星出營室,如太白,赤黃色,東南速行,過危,至虛沒,有尾跡,明燭地。七月甲午,星出建星,如太白,向南速行,至濁沒。乙巳,星出牽牛,如太白,南行,至濁沒。八月甲寅,星出八
 穀,東北入濁。少頃,有星出天將軍,如太白,西北速行,至王良沒,有尾跡,其色赤黃。巳卯,星出文昌,大如醆,直北速行入濁,有尾跡,明燭地。壬午,星出北河,至柳沒。十月甲寅,星出畢,東南速行,至天苑沒,赤黃,有尾跡。丙辰,星出張,東南速行,至濁沒。丙寅,星出天津,大如杯,東南速行,至危沒,赤黃,有尾跡,明燭地。六年三月乙未,星出大角,如太白,西南速行,至濁沒。庚戌,星出文昌,如太白,向北速行入濁,青白色,有尾跡,明燭地。六月丁巳,星出營
 室,大如杯,光燭地,有聲,北行,至王良沒。七月癸巳,星出昴,至參沒。九月辛巳,星出王良,如太白,東北速行入濁。乙巳,星出南河,如太白,東北速行,沒於輿鬼側。七年四月己酉,星出營室,東北速行入濁。戊辰,星出郎位,如太白,至梗河沒,有尾跡,明燭地。六月己巳,星出天田,赤黃色,有尾跡,西南緩行,至折威沒。戊辰,星出尾,西南速行入濁。九月乙亥,星出河鼓,入天市垣,至宗人沒。戊寅,星出天苑,如太白,南行,至天園沒,有尾跡,明燭地。庚辰,星
 出東井,沒於狼。丙戌,星出北落師門,西南緩行,至濁沒。十二月癸亥,星出五車,赤黃色,西北速行,至天船沒。八年正月乙酉,星出天廁側,西南速行入濁,有尾跡,明燭地。丁酉,星出柳,直南速行入濁。二月乙酉,星出文昌,青白色,東北速行,至濁沒。四月己巳,星出奎,如太白,東北速行,至婁沒。五月壬寅,星出氐,如太白,向西南速行,入濁沒。戊午,星出房,色赤黃,東南入濁。六月戊寅,星出北落師門,西南速行,沒於濁。己卯,星出北斗,至郎位沒,有
 尾跡,明燭地。癸巳,星出天津,至紫宮西垣沒。七月庚申,星出七公,如太白,西北速行,入濁沒。八月乙亥,星出天市,西南速行入濁,有尾跡,色赤黃。是夜,星出東壁,赤黃色,東北速行,至濁沒。九月壬寅,星出天倉,如太白,東北速行,至胃沒。甲子,星出天苑,西南速行,入濁沒。十月乙酉,星出匏瓜,如太白,向東速行,至天津沒。十二月乙丑,星出南河,如太白,東南行,至弧矢沒。己丑,星出天市垣,東南行,至濁沒。



 皇祐元年三月庚子,星出軫,西南速行,
 沒於翼。四月辛巳,星出織女,向南速行,入天市垣,至宗人沒,明燭地。甲申,星出心,如太白,東南速行入濁。六月丙寅,星出紫宮鉤陳側,如太白,北行入濁。己巳,星出匏瓜,赤黃,有尾跡,向南速行,至建星沒。丁丑,星出造父,如太白,向西南速行,至天棓沒,有尾跡,明燭地。九月壬子,星出閣道,東南速行,至婁沒,有尾跡,明燭地。十一月癸巳,星出文昌,向東速行,至五車沒,有尾跡,明燭地。十二月乙丑,星出亢,赤黃色,向東北緩行,至天市垣西沒。丁
 酉,星出文昌,向北速行,沒於北辰側。二年四月癸未,星出氐,赤黃色,東南速行,至心沒,有尾跡,明燭地。



 五月乙巳,星出貫索,向東速行,至女床沒。七月己丑,星出奎,赤黃色,西南緩行,沒於營室側。九月辛卯,星出織女,如太白,向西速行,入濁沒。十二月丁未,星出庫樓,如太白,赤黃色,至翼沒。三年七月丙辰,星出南斗,赤黃色,尾跡凝天,向南緩行,至濁沒。



 八月庚辰,星出奎,如太白,西北速行,沒於濁。九月癸丑,星出上臺,東北入濁。十月乙巳,星
 出天槍,如太白,西北速行入濁。四年三月庚申,星出郎將,東行,至貫索沒。壬申,星出文昌,沒於五車側。四月辛巳,星出天市垣市樓側,至南斗沒。癸卯,星出東壁,沒於天船側。六月庚子,星出危,如太白,東南速行入濁。壬寅,星出天船,如太白,東北入濁。八月丁酉,星出天倉,如太白,西南速行,至濁沒。戊戌,星出參旗,如太白,西南速行,至天苑沒。九月丙午,星出婁,西南速行入濁。戊申,星出紫宮北辰側,赤黃色,西南速行,至貫索沒,尾跡凝天,明
 燭地。己酉,星出營室,如太白,東南速行入濁。是夜,星出參,如太白,東南速行入濁,尾跡赤黃。甲子,有星出南河,如太白,東北入濁。十月丁丑,星出天棓,西北速行入濁,有尾跡,明燭地。丙申,星出天倉,如太白,西南速行入濁。十一月丙申,星出北河,沒於北斗璇星側。五年正月壬寅,夜漏未上,星出東井,如太白,東北速行,至濁沒,有尾跡,明燭地。五月庚戌,星出北斗魁側,西北速行入濁,尾跡赤黃。庚申,星出大角,如太白,西北行,至中臺沒,青白
 色,有尾跡。六月癸酉,星出紫宮北辰側,赤黃色,北行,至濁沒。七月癸卯,星出王良,至天津沒。甲辰,星出奎,如太白,速行沒於危。是夜,星出紫宮北辰側,色赤黃,西南速行,至天市垣東沒,有尾跡,明燭地。乙巳,星出王良,速行至營室沒。戊午,星出貫索,西南速行,入天市垣,至宦者沒。八月丙戌,星出紫宮北辰側,至王良沒。是夜,又星出危,沒婺女側。癸亥,星出大陵,至營室沒,有尾跡,明燭地。九月乙亥,星出參,如太白,西北速行,至昴沒,有尾跡,明
 燭地。



 至和元年七月壬戌,星出王良,色赤黃,向北速行,至天船沒,有尾跡,明燭地。八月壬寅,星出上臺,東北行入濁。二年七月甲申,星出牽牛,如太白,赤黃色,南行入濁,有尾跡,明燭地。九月己卯,星出弧矢,如太白,西南速行,至丈人沒,尾跡青白。又有星出軒轅,向北速行,至中臺沒。庚辰,星出天廩,東南緩行,至天苑沒。十一月戊辰,星出南河,向南行,至弧矢沒。辛酉,星出弧矢,色赤黃,南行入濁。十二月甲申,星出太微東垣,如太白,赤黃色,東
 南速行,至軫沒。辛卯,星出柳,如太白,赤黃色,直北速行入濁。



 嘉祐元年三月辛酉,星出庫樓,沒於尾。乙亥,星出紫微北辰東,如太白,色赤黃,西南速行,至右攝提沒。壬午,星出張,至東甌沒。九月壬午,星出東井,如太白,赤黃色,向北速行,至文昌沒。二年正月丁酉,星出文昌,如太白,速行入紫宮北辰沒。辛丑,星出華蓋,緩行至北辰沒。甲辰,星出觜觿,緩行至畢沒。二月甲子,星出紫宮東垣,大如杯,東北行入濁。七月乙亥,星出北斗魁西,如太白,
 西北速行入濁。丁丑,星出王良,如太白,赤黃色,西南緩行,至亢沒,有尾跡,明燭地。九月丙子,星出王良,如太白,赤黃色,向西速行,至騰蛇沒,有尾跡,明燭地。丁亥,星出南河子星側。戊戌,晝漏上,中天有星出狼,大如杯,東南速行,至濁沒,尾跡青白。三年正月乙未,星出參,赤黃色,向西速行,至天廩沒。五月甲午,星出河鼓,如太白,赤黃色,東北緩行,至虛沒。七月辛未,星出天船,東北行,至濁沒。乙酉,星出北河,如太白,赤黃色,東南緩行,散為數道,
 至狼沒,尾跡凝天。丁酉,有星出危,西南速行入濁。其夜,又有星出天苑,緩行入濁。八月丙午,星出天綱,東南速行入濁,尾跡赤黃。戊申,星出危,西南速行入濁,有尾跡,明燭地。己未,星出牽牛西,速行至牽牛北沒。癸亥,星出王良,向南速行,至天津沒。夜漏盡,有星出柳,如太白,赤黃色,西北行,至北斗沒。乙丑,星出文昌,向西速行,至北極沒。九月庚午,星出婁,向南速行,至土司空沒。甲申,出天將軍,如太白,青白色,向西速行,至濁沒。庚寅,星出五
 車,如太白,赤黃色,東北速行,至北河沒,有尾跡,明燭地。辛卯,星出王良,北行至鉤陳沒。四年二月己亥,星出翼,入濁。夜漏盡,又有星出營室,沒於鉤陳。癸卯,星出天槍,至郎將沒。乙卯,星出角,西行,至翼沒。五月辛丑,星出右攝提,西行入濁。己酉,星出大角,至軫沒。癸丑,星出營室,大如杯,赤黃色,西南速行,至羽林軍沒,炸烈有聲。六月癸亥,星出天倉,至天苑沒,有尾跡,明燭地。甲子,星出天津,至北辰沒。辛未,星出胃,沒於鉤陳。又星出天船,至王
 良沒。乙亥,星出墳墓,至北落師門沒。又有星出天船,東南速行,至昴沒。癸未,星出氐宿,西南行入濁。己丑,星出畢,速行至五車沒。八月乙亥,夜漏盡,星出輿鬼,速行至五車沒。又星出輿鬼,速行至太微北落。癸未,星出軍市,速行至弧矢沒。己丑,星出天囷,至天倉沒。九月己亥,星出紫宮鉤陳側,大如碗,東北速行,曳尾長五尺,初直後曲,流至北辰東沒,後尾跡凝結如盤,食頃散。又有星出太微西,東北速行入濁。辛丑,星出天津,速行至織女沒。
 癸丑,星四,皆如太白,赤黃色,有尾跡,明燭地:一出天棓,西南速行,至天市垣侯星沒;一出危,西南速行,至女沒;一出畢,南行沒於天苑側;一出五車北,速行至鉤陳沒。十月乙丑,晝漏上,星出天大將軍,西南行,至濁沒,色青白,尾跡凝天,良久散。其夜,星出參,至弧矢沒。丁卯,星出婺女,東南至濁沒。戊辰,星出東井,東行,至柳沒。戊寅,星出狼,南行,至濁沒。丁亥,星出天倉。乙未,星出上臺南,速行至北河沒。十二月甲子,星出貫索,至女床沒。五年正
 月辛卯,星出畢,大如碗,赤黃色,速行至天倉沒,明燭地,尾跡炸烈而散,有聲如雷。四月辛未,星出氐,緩行,東南入濁沒。癸酉,星出婺女,至羽林軍沒。庚辰,夜漏盡,星出大角,西南行,至濁沒,尾跡青白。癸未,星出女床,東行,至河鼓沒。乙酉,星出騎官,西南行,至濁沒。甲午,星出天市東,如太白,向東速行,至河鼓沒,尾跡赤黃。丙申,星出貫索,東北行,至北斗柄沒。辛亥,星出天棓,西南行,入天市,至宦者沒。六月己未,星出婁,東北行,至濁沒。壬戌,星出
 天倉,東南行,至濁沒。辛巳,星出天津,西南行,至天市垣宦者沒。又有星出王良,至土司空沒。癸酉,星出南斗,大如杯,行入濁。八月庚申,星出東壁,東行入濁。丙寅,夜漏未上,星出虛,大如杯,東南入濁。甲午,星出五車,至文昌沒。乙卯,星出天苑,南行入濁。十月乙亥,星出軒轅星北斗魁旁,沒,尾跡赤黃。十一月壬辰,星出五車,至畢沒。十二月壬申,有星出北河,至輿鬼沒。戊寅,星出弧矢,至南河沒。己卯,夜漏未上,星出軫,至氐側沒。六年六月丁巳,
 星出天市垣宦者側,沒於氐。己巳,星出天市垣車肆側,西南行,至尾沒。七月乙酉,星出騰蛇,至危沒。其夜,又有星出婁,大如杯,赤黃色,速行入羽林沒。丙戌,星出天津,至危沒,尾跡赤黃。庚寅,星出文昌,北行,至濁沒。八月丁巳,星出婁,東北速行,至昴沒。戊辰,星出鉤陳,北行入濁。己卯,星出天市垣北,東行,入濁沒。丁卯,星出狼,大如杯,至天社沒,明燭地,尾跡凝天,良久散。九月甲寅,星出營室,西南行入濁。癸亥,星出柳,東行,至翼沒。十一月癸丑,
 星出東北維,去地五丈許,大如碗,向東北緩行入濁,尾跡青白。壬申,星出參旗,至濁沒。丙子,星出狼,大如杯而赤黃,緩行至弧矢沒,有尾跡,明燭地。十二月辛丑,星出貫索,如太白,東北速行,入天市,至候星沒,尾跡青白。七年正月乙亥,星出下臺,至上臺沒。二月己卯,星出北河,大如杯,色赤黃,速行,沒於閣道側,有尾跡,明燭地。壬辰,星出東井,如太白,至畢沒。四月庚子,星出太微郎位,如太白,西南緩行,至張沒,尾跡赤黃。六月丁丑,星出北落
 師門,南行入濁。七月丁未,星出牽牛,至南斗沒。又有星出羽林軍,至北落師門沒。己酉,星出壁壘陣,如太白,向西速行,至敗臼沒,尾跡赤黃。辛酉,星出天紀,西北速行入濁。八月己卯,星出文昌,至下臺沒。乙未,星出天苑,南行入濁,尾跡赤黃。己亥,星出天津,西南入濁。九月丙辰,星出土司空,東南入濁。丁卯,星出東壁,大如杯,西行,至虛沒,有尾跡,赤黃,明燭地。十月丙子,星出昴,如太白,西北速行,至天大將軍沒,尾跡赤黃。丁丑,星出大陵,如太
 白,南行,至天倉沒。庚寅,星出南河,至天社沒,明燭地。丁酉,星出天廟,南入濁。己亥,星出參,如太白,西南行,至天園沒,尾跡青白。八年正月辛酉,星出軫,赤黃色,東南速行,入庫樓沒。三月癸卯,星出匏瓜,東南至危沒,赤黃色,有尾跡,明燭地。癸亥,星出文昌,北行入濁,有尾跡,明燭地。又有星出傳舍,速行至北辰沒。五月癸卯,星出天市垣宗人側,東南速行,至鱉星沒。己亥,星出招搖,赤黃色,行南向,入氐沒。七月乙丑,星數百,縱橫西流。八月庚寅,
 星出閣道,東南速行,入濁沒。甲子,星出上臺,大如杯,赤黃色,向東速行,至下臺沒。



 治平元年二月丁卯,星出紫宮鉤陳側,西北入濁沒,明燭地,尾跡炸烈有聲。六月辛酉,夜漏未上,星出河鼓,東南速行,至危沒。七月癸未,星出危,西南速行,入天市垣沒。八月辛亥,星出北辰,大如杯,速行至鉤陳沒,尾跡青黃。丁巳,星出奎,大如碗,速行至五車沒。壬戌,夜漏盡,星出奎,西南行,至濁沒。九月癸酉,星出北斗魁,大如盞,東北速行,至濁沒,尾跡赤黃。十
 一月癸丑,星出軍市,東南速行,至濁沒。二年二月丁酉,星出太廟,色青白,西南入濁。乙卯,星出中臺,色赤黃,西北慢行,至內階沒。五月壬戌,星出北斗魁,如杯,色青白,北行,至濁沒。六月己丑晝,有星出中天,大如碗,西速行,至濁沒,尾跡赤黃。八月己未,星出河鼓,大如盞,色赤黃,速行至天市垣宗內星沒。丁巳,星出危,至濁沒。九月癸酉,星出北斗魁,東北速行,至濁沒。三年四月癸巳,星出房,至濁沒,明燭地,尾跡炸而散。七月庚申,晝漏未上,星
 出紫宮,西行,曳尾長二丈,沒,尾跡青白。九月丁丑,有星出參,至天倉沒。十一月己卯,星出王良,西北速行,至濁沒,尾跡青黃。



\end{pinyinscope}