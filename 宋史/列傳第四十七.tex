\article{列傳第四十七}

\begin{pinyinscope}

 任中正弟中師周起程琳姜遵範雍孫子奇曾孫坦趙稹任布高若訥孫沔



 任中正,字慶之,曹州濟陰人。父載,右拾遺。中正進士及
 第,為池州推官。歷大理評事、通判邵州,改太府寺丞、通判濮州。以翰林學士錢若水薦,遷秘書省著作佐郎、通判大名府。



 轉運使陳緯徙陜西,舉中正自代,太宗曰:「朕自知之。」召為秘書丞、江南轉運副使。中正軀幹頎長,帝擇大笏,命內臣取緋衣之長者賜之。至部,歲大稔,民出租賦、平糴皆盈羨。發運使王子輿欲悉調餉京師,中正曰:「東南歲輸五百餘萬,而江南所出過半。今歲有餘,或歲少歉則數不登,患及吾民矣。」乃止。



 擢監察御史、兩浙
 轉運使。民饑,中正不俟詔,發官廩振之。按晉州盛梁獄,論如法。遷殿中侍御史、判三司憑由司。既而有與梁善者,密中之,出為荊湖轉運使。遷左司諫、直史館、知梓州。擢樞密直學士,代張詠知益州。在郡五載,遵詠條教,蜀人便之。知審刑院,出知並州。遷給事中、權知開封府。



 大中祥符九年,拜尚書工部侍郎、樞密副使。馬知節知密院,改同知院事。明年,曹利用為樞密使,復為副使,再進兵部侍郎、參知政事。



 仁宗在東宮時,以右丞兼賓客。遷
 工部尚書。帝既即位,乃拜兵部尚書。中正素與丁謂善,謂且貶,左右莫敢言者,中正獨營救謂,降太子賓客、知鄆州。中正弟尚書兵部員外郎、判三司鹽鐵勾院中行,右正言中師,皆坐貶。頃之,以母老徙曹州,遷禮部尚書。卒,贈尚書左僕射,謚康懿。



 初,中正母入謁禁中,與陳彭年、王曾、張知白妻同見真宗,命中正母為班首,且賜坐。中正事親孝,平居簡素,而飲食極豐美。



 中師字祖聖,進士及第,試秘書省校書郎、知平陸縣。真
 宗將祀汾陰,命陳堯叟判河中府,以經制祀事,闢掌箋奏,累遷著作佐郎,歷知千乘、襄邑縣,改秘書丞。以張知白薦,遂為右正言。中正貶,中師亦降太常博士、監宿州酒稅。未幾,通判應天府。



 曹利用闢為群牧判官,徙知滑州,入為開封府判官。累遷尚書度支郎中、直史館、知澶州。以太常少卿、直昭文館知廣州。視事之明日,吏白,故事當謁諸祠廟,而廨有淫祠,中師遽命撤去之。兼市舶使,市舶置使自此始。



 還,為諫議大夫、判尚書刑部。加集
 賢院學士,再知澶州。未行,進龍圖閣直學士、知並州,許便宜從事。改樞密直學士、知益州。先是,轉運使韓瀆急於籠利,自薪芻、蔬果之屬皆有算,而中師盡奏蠲之。



 康定中,任布守河陽,數上書論事,帝欲用之。呂夷簡薦中師才不在任布下,遂並召為樞密副使。明年,建北京,令中師領修建。進給事中,宣撫河東,不行。求補郡,以尚書禮部侍郎、資政殿學士知永興軍。求內徙,得知陳州。



 逾年,上書言:「臣老矣,家本曹人,願得守曹。」遂以知曹州。改
 戶部侍郎。明年,請老,拜太子少傅致仕,進少師。卒,贈太子太傅,謚安惠。中師性樂易,平居自奉甚儉約,晚知養生之術,號大塊翁。



 周起字萬卿,淄州鄒平人。生而豐下,父意異之,曰:「此兒必起吾門。」因名起。幼敏慧如成人。意知衛州,坐事削官,起才十三,詣京師訟父冤,父乃得復故官。舉進士,授將作監丞、通判齊州。擢著作佐郎、直史館,累遷戶部、度支判官。



 真宗北征,領隨軍糧草事。以右正言知制誥,權判
 吏部流內銓。尋為東京留守判官,判登聞鼓院。封泰山,攝御史中丞、考制度副使,所過得採訪官吏能否及民利病以聞。東封還,近臣率頌功德,起獨以居安為戒。進金部員外郎、判集賢院。



 初置糾察刑獄司,因命起,起乃請諸已決而事有所枉及官吏非理榜掠者,並聽受訴,從之。擢樞密直學士、權知開封府。起聽斷明審,舉無留事。真宗嘗臨幸問勞,起請曰:「陛下昔龍潛於此,請避正寢,居西廡。」詔從之,名其堂曰繼照。



 起嘗奏事殿中,適仁
 宗始生,帝曰:「卿知朕喜乎?宜賀我有子矣」即入禁中,懷金錢出,探以賜起。改勾當三班院兼判登聞檢院。從祀汾陰,貿權知河中府,徙永興、天雄軍,所至有風烈,數賜書褒諭。三遷右諫議大夫、知並州。拜給事中、同知樞密院事。進禮部侍郎,為樞密副使。嘗與寇準過同列曹瑋家飲酒,既而客多引去者,獨起與寇準盡醉,夜漏上乃歸。明日入見,引咎伏謝。真宗笑曰:「天下無事,大臣相與飲酒,何過之有?」



 起素善寇準。準且貶,起亦罷為戶部郎
 中、知青州,又降太常少卿、知光州。稍遷秘書監,徙揚、杭二州,又徙應天府。復為禮部侍郎、判登聞鼓院。以疾請知穎州,徙陳州、汝州。卒,贈禮部尚書,謚安惠。



 起性周密,凡奏事及答禁中語,隨輒焚草,故其言,外人無知者。家藏書至萬餘卷。起能書。弟超,亦能書,集古今人書並所更體法,為《書苑》十卷,累官主客郎中。起子:延荷,以孝友聞,官殿中丞;延雋,頗雅厚,官太常少卿。



 程琳,字天球,永寧軍博野人。舉服勤辭學科,補泰寧軍
 節度推官。改秘書省著作佐郎、知壽陽縣,監左藏庫,召試,直集賢院。改太常博士、權三司戶部判官,契丹館伴使。契丹使者謂琳曰:「先皇帝嘗通使承天,太后獨無使,何也?」琳曰:「南北,兄弟也。先皇帝視承天猶從母,故無嫌;今皇太后乃嫂也,禮不通問。」契丹使者語屈。後修《真宗實錄》,而大中祥符以來起居注闕,琳追述上之,遂修起居注,提舉在京諸司庫務,知制誥、判吏部流內銓。



 權三司使範雍使契丹,命琳發遣三司使。太倉贍軍粟陳腐
 不可食,歲且饑,琳盡發以貸民,凡六十萬斛,饑民賴以全活,而軍得善粟。鹽鐵官任布請鑄大錢一當十,度支判官許申請以銅鐵雜鑄,下其議。琳曰:「第五琦用大錢,法卒不可行。乞令申試之。」鑄卒不就。



 契丹遣蕭蘊、杜防來,蘊出位圖標琳曰:「中國使者坐殿上高位,今我位乃下,請升之。」琳曰:「此真宗所定,不可易。」防曰:「大國之卿,可以當小國之君。」琳曰:「南北雖兩朝,無小大之異,卿嘗坐我殿上,我顧小國耶?」防無以對。宰相將許之,琳曰:「許其
 小必啟其大。」



 以右諫議大夫權御史中丞。宰相張知白尤器之,當除命,喜曰:「不辱吾筆矣。」時歲饑,上疏請罷諸土木營造,蠲被災郡縣租賦。改樞密直學士、知益州。上元張燈,州人夜聚游嬉,琳戒曰:「有火則隨救之,毋白也。」已而果有火,終宴人無知者。或告振武軍變,琳曰:「軍中動靜我自知之,茍有謀,不待告也。」



 遷給事中、權知開封府。王蒙正子齊雄捶老卒死,貸妻子使以病告。琳察其色辭異,令有司驗得捶死狀。蒙正連姻章獻太后家,太
 后謂琳曰:「齊雄非殺人者,乃其奴嘗捶之。」琳曰:「奴無自專理,且使令與己犯同。」太后嘿然,遂論如法。外戚吳氏離其夫而挈其女歸,夫訴於府。琳命還女,吳氏曰:「已納宮中矣。」琳請於帝曰:「臣恐天下人有竊議陛下奪人妻女者。」帝亟命出之。笞而歸其妻。



 遷工部侍郎、龍圖閣學士,復為御史中丞。不拜,以翰林侍讀學士兼龍圖閣學士再知開封府。改三司使,出納尤謹,禁中有所取,輒奏罷之。內侍言琳專,琳曰:「三司財賦,皆朝廷有也。臣為陛
 下惜,於臣何有?」帝然之。或請並天下農田稅物名者,琳曰:「合而為一,易於勾校,可也。後有興利之臣,復用舊名增之,是重困民,無已時也。」再遷吏部侍郎,遂參知政事,遷尚書左丞。



 時元昊反,猶遣使來朝,眾請按誅之。琳曰:「遣使,常事也,殺之不祥。」後使者益驕橫,大臣患之。琳曰:「始不殺,無罪也;今既驕橫,可暴其惡誅之,國法也,又何患耶?」又議重賄唃廝囉使討賊,得地即與之。琳曰:「使□角廝囉得地是復生一元昊矣。不若用間,使二羌勢不合,
 中國利也。



 故樞密副使張遜第在武成坊,其曾孫偕才七歲,宗室女生也,貧不自給。乳媼擅出券鬻第,琳欲得之,使開封府吏密諭媼,以偕幼,宜得御寶許鬻乃售。乳媼以宗室女故,入宮見章惠太后。既得御寶,琳乃市取之。又令吏市材木,買婦女。已而吏以贓敗,御史按劾得狀,降光祿卿、知穎州。



 頃之,為戶部侍郎,尋復吏部、知天雄軍。又以左丞為資政殿學士。及建天雄軍為北京,內侍皇甫繼明主營宮室,欲侈大以要賞。琳以為方事邊
 陲,又事土木以困民,不可。既而繼明數有論奏,帝遣御史魚周詢按視,遂罷繼明,命琳獨主之。遷工部尚書,加大學士、河北安撫使。改武昌軍節度使、知永興軍、陜西安撫使。以宣徽北院使判延州,仍為陜西安撫使。



 元昊死,諒祚立,方幼,三大將分治其國。議者謂可因此時,以節度使啖三將,使各有所部分,以弱其勢,可不戰而屈矣。琳曰:「幸人之喪,非所以柔遠人,不如因而撫之。」議者惜其失幾。



 既而遣使冊命,夏人方圍慶陽。琳曰:「彼若貪
 此,可緩慶州之難矣。」具禮幣賜予之數移報之,果喜,即日迎冊使,慶陽之圍亦解。嘗獲戎首,不殺,戒遣之,夏人亦相告毋捕漢民。久之,以五百戶驅牛羊扣邊請降,且言:「契丹兵至衙頭矣,國中亂,願自歸。」琳曰:「彼詐也。契丹至帳下,當舉國取之,豈容有來降者?間聞夏人方捕叛者,此其是邪?不然,誘我也。」拒不受。已而賊果以騎三萬臨境上,以捕降者為辭。琳諜知之,閉壁倒旗,戒諸將勿動,賊疑有備,遂引去。



 拜同中書門下平章事、判大名府。
 琳持重不擾,前後守魏十年,度要害,繕壁壘,增守禦備。植雜木數萬,曰:「異時樓櫓之具,可不出於民矣。」人愛之,為立生祠。改武勝軍,又換鎮安軍節度使。上書曰:「臣雖老,尚能為國守邊。」未報,得疾卒。贈中書令,謚文簡。



 琳為人敏厲深嚴,長於政事,辨議一出,不肯下人。然性嗇於財,而厚自奉養。章獻太后時,嘗上《武后臨朝圖》,人以此薄之。



 姜遵,字從式,淄州長山人。進士及第,為蓬萊尉,就闢登
 州司理參軍,開封府右軍巡判官。有疑獄,將抵死,遵辨出之。遷太常博士,王曾薦為監察御史,殿中侍御史,開封府判官。知吉州高惠連與遵有隙,發遵在廬陵時贓事,按驗無狀,猶降通判延州。復入為侍御史、判戶部勾院。利州路饑,以遵為體量安撫,遷知邢州。



 仁宗即位,徙滑州,為京東轉運使,徙京西。未幾,以刑部郎中兼侍御史知雜事。建言三司、開封府日接賓客,廢事,有詔禁止。歷三司副使,再遷右諫議大夫、知永興軍。奏罷咸陽富
 民元氏歲貢梨。召拜樞密副使,遷給事中,卒。贈吏部侍郎。



 遵長於吏事,為治尚嚴猛,所誅殘者甚眾。在永興,太后嘗詔營浮屠,遵毀漢、唐碑碣代磚甓,既成,得召用。



 範雍,字伯純,世家太原。曾祖仁恕,仕蜀為宰相。祖從龜,刑部侍郎,入朝,改右屯衛將軍,後葬河南,遂為河南人。雍中進士第,為洛陽縣主簿。累官殿中丞、知端州。遷太常博士。寇準闢為河南通判,還,判三司開拆司。河決滑州,選為京東轉運副使。歷河北、陜西轉運使,入為三司
 戶部副使,又徙度支。以尚書工部郎中為龍圖閣待制、陜西都轉運使。還,提舉諸司庫務,勾當三班院。



 環、原州屬羌擾邊,以雍為安撫使。建言:「屬羌因罪罰羊者,舊輸錢,而比年責使出羊,羌人頗以為患。請輸錢如舊,罪輕者以漢法贖金。」從之。遷右諫議大夫、權三司使。



 雍在京東時,平滑州水患。以勞加龍圖閣直學士。明年,拜樞密副使。丁母憂,起復,遷給事中。玉清昭應宮災,章獻太后泣對大臣曰:「先帝竭力成此宮,一夕延燎幾盡,惟一二
 小殿存爾。」雍抗言曰:「不若悉燔之也。先朝以此竭天下之力,遽為灰燼,非出入意;如因其所存,又將葺之,則民不堪命,非所以畏天戒也。」時王曾亦止之,遂詔勿葺。遷尚書禮部侍郎。



 太后崩,罷為戶部侍郎、知陜州,改永興軍。是歲饑疫,關中為甚,雍為振恤。以疾,請近郡,遂知河陽。進吏部侍郎,徙應天府,又改河南府,進資政殿學士。陳安邊六事,又請於天雄軍聚甲兵以備河北,於水興軍、河中府益募土兵以備陜西,即涇原、環慶有警,河中
 援之。



 既而元昊反,拜振武軍節度使、知延州。因言:「延州最當賊沖,地闊而砦柵疏,近者百里,遠者二百里,土兵寡弱,又無宿將為用,而賊出入於此,請益師。」不報。元昊先遣人通款於雍,雍信之,不設備。一日,引兵數萬破金明砦,乘勝至城下。會大將石元孫領兵出境,守城者才數百人。雍召劉平於慶州,平帥師來援,合元孫兵與賊夜戰三川口,大敗,平、元孫皆為賊所執。雍閉門堅守,會夜大雪,賊解去,城得不陷。左遷戶部侍郎、知安州。居一
 歲,復吏部侍郎、知河中府。



 又為資政殿學士、知永興軍兼轉運司事,遷尚書左丞,加大學士。初,完永興城,或言其非便,詔止其役,雍匿詔而趣成之。明年,賊犯定川,邠、岐之間皆恐,而永興獨不憂寇。復徙河南府,又遷禮部尚書,卒。贈太子太師,謚忠獻。



 雍為治尚恕,好謀而少成。在陜西,嘗請於商、虢置監鑄鐵錢,後不可行;又括諸路牛以興營田,亦隨廢。頗知人,喜薦士,後多至公卿者。狄青為小校時,坐法當斬,雍貸之。



 子宗傑,為兵部員外郎、
 直史館,歷陜西轉運使,先雍卒。宗傑子子奇。



 子奇字中濟,階祖雍蔭,簽書並州判官。以唐介薦,神宗賜對,提舉修在京倉。三司使又薦,按覆營繕,匠吏積為欺隱,懼罪,造飛語間之。神宗遣大閹張茂則察其無私,勞之曰:「為吏當如是,無恤人言。」授戶部判官,為湖南轉運副使。建言:「梅山蠻恃險為邊患,宜拓取之。」後章惇開五溪,議由此起。



 入判將作監。使於遼,導者改路回遠,子奇謂曰:「此去雲中有直道,旬日可至,何為出此?」導者又
 欲沮子奇下馬館門外,子奇曰:「異時於中門下馬,今何以輒易?」導者計屈。歷河東、陜西、河北、京東四路轉運使,工部、左司二郎中,加直龍圖閣,使河北。諸郡猶榷鹽,奏罷之。



 元祐初,為將作監、司農卿,復使陜西,以病解。起知鄭州,加集賢殿修撰、知河陽。召權戶部侍郎。刪酒戶苛禁及奴婢告主給賞法。未幾,出知慶州,廣儲蓄,繕城柵,嚴守備,羈黠羌,推誠待下,人樂為用。入為吏部侍郎,以待制致仕,卒,年六十三。子坦。



 坦字伯履,以父任為開封府推官、金部員外郎、大理少卿,改左司員外郎。押伴夏國使,應對合旨,賜進士第,權起居舍人。使於遼,復命,具語錄以獻。徽宗覽而善之,付鴻臚,令後奉使者視為式。遷殿中監,知開封府,再命使遼。時興邊議,非時遣使以觀釁,坦以不宜始禍,辭其行。徽宗怒,責舒州團練副使,稍復集賢殿修撰,知江寧府、洪揚二州。



 召為戶部侍郎,論當十及夾錫錢之弊。以便親請外,知河陽。入辭,徽宗曰:「夾錫錢之害,甚於當十,宜
 速正之,為一道率。」坦至,即奏罷之。政和初,復為戶部,遂改當十錢為當三;罷淮鹽入東北;鬻諸州公田,以實常平。又上疏言:「戶部歲入有限,用則無窮。今節度使八十員,留後至刺史數千員,自非軍功得之,宜減其半奉;及他工技末作,一切裁損。」時以為當。



 時張商英為相,坦多與之合。及商英去,言者論坦助為匱竭之說,以搖眾聽;又言坦建議鬻田、改常平法、廢元符令及罷夾錫錢之罪,貶黃州團練副使,安置韶州。以赦,復徽猷閣待制,卒,
 年六十二。



 趙稹,字表微。其先單父人,後徙宣城。為人誠質寬厚,少好學。吳太府卿田霖退居郡中,名有風鑒,故以女妻稹。擢進士第,歷平定軍判官、臺州推官。改大理寺丞、知昆山縣,通判楚州。遷殿中丞、知通州。召還,同判宗正寺,樞密直學士李浚薦為監察御史,再遷侍御史、判登聞鼓院、開封府判官,徙三司開拆、憑由司。帝祀汾陰,為留守推官。



 遷尚書兵部員外郎、益州路轉運使,真宗諭曰:「蜀
 遠而數亂,其利害朕所欲聞。卿至,悉條上之,祗附常奏,毋著姓名。」稹至,數言部中事,至一日章數上。蒲江縣捕劫盜不得,反逮系平民,楚掠誣服。稹適行部,意其冤,馳入縣獄,問得狀,悉縱之。遷工部郎中。



 召為侍御史知雜事、同判吏部流內銓,糾察在京刑獄。慎從吉知開封府,其子鈞、銳受賕,事連錢惟演。稹與王曾白其奸狀,從吉坐免,惟演亦罷去。



 改三司鹽鐵副使,擢右諫議大夫、集賢院學士、知益州。度支市錦六千匹,召工計歲織裁千
 餘匹,止以歲所織數上供。久之,或言稹不達民情,喜尊大,降知同州,徙鳳翔、京兆府,三遷工部侍郎,復糾察在京刑獄。加樞密直學士、知並州,代還,遷刑部侍郎。



 天聖八年,擢樞密副使,遷吏部侍郎。時,權出宮掖,稹厚結劉美人家婢,以故致位政府。命未出,人馳告稹,稹問曰:「東頭?西頭?」蓋意在中書也。聞者皆以為笑。章獻太后崩,罷為尚書左丞、知河中府,遷禮部尚書。既病,乞骸骨,拜太子少傅致仕。卒,贈太子太保,謚僖質。



 任布,字應之,河南人。後唐宰相圜四世孫也。力學,家貧,嘗從人借書以讀。進士及第,補安肅軍判官,輒刺問虜中事,上疏請飭邊備,仍奏河北利害。後契丹至澶淵,真宗識其名,特改大理寺丞、知安陽縣。通判嘉州,還,知開封府司錄事,通判大名府。初置提點刑獄,選布領荊湖南路。



 入權三司鹽鐵判官,判度支勾院。京城東南有泉湧出,為築祥源觀,男女徒跣奔走瞻拜。布論之曰:「明朝不宜以神怪衒愚俗。」遂忤宰相意。又與徐奭、麻溫其試
 開封府進士,而奭潛發封卷視之。降監鄧州稅,徙知宿州。



 時越州守闕,寇準曰:「越州有職分田,歲入且厚,今爭者頗眾,非廉士莫可予。」乃徙布越州。有祖訟其孫者「醉酒詈我」,已而悔,日哭於庭曰:「我老無子,賴此孫以為命也。」布聞之,貸其死,上書自劾,朝廷亦不之責。



 寇準貶,布亦徙建州,累遷尚書職方員外郎。丁謂既逐,稍用為白波發運使。歲餘,判三司開拆司,出為梓州路轉運使。富順監鹽井,歲久鹵薄而課存,主者至破產,或鬻子孫不
 能償。布奏除之。遷祠部郎中、權戶部判官,擢江、淮制置發運使。前使者多聚山海珍異之物以餉權要,布一切罷去。



 召為三司度支副使,奉使契丹。還,加直史館、知荊南。為鹽鐵副使,命管伴契丹使。歷兵部、刑部郎中,拜右諫議大夫、知真定府。或欲省河北兵,布言:「契丹、西夏方窺伺中國,備未可弛也。」築甬道屬滹沱河,跨絕泥潦。徙滑州,改天雄軍。遷給事中、集賢院學士、知許州。未幾,為龍圖閣直學士,徙澶州。黃德和誣劉平降賊,欲收平家,
 布力言平非降賊者。復徙真定,又徙河南府,未至,召為樞密副使。



 布純約自守,及秉政,無所建明。子遜嘗上書,詆大臣及布皆為不才,御史魚周詢因奏疏曰「布不才,其子能知之。」乃以尚書工部侍郎罷知河陽。議者以周詢引遜語逐其父,為不知體。改蔡州,授太子少保致仕,進少傅。皇祐間,詔陪祀明堂,稱疾不赴。賜一子進士出身,遷少師。



 始,布歸洛中,作五知堂,謂知恩、知道、知命、知足、知幸也。卒,贈太子太傅,謚恭惠。子達,性亦恬遠,尚釋
 氏學,歷官為司封郎中。



 高若訥,字敏之,本並州榆次人,徙家衛州。進士及第,補彰德軍節度推官,改秘書省著作佐郎,再遷太常博士、知商河縣。縣有職分田,而牛與種皆假於民,若訥獨廢不耕。



 御史知雜楊偕薦為監察御史裏行,遷尚書主客員外郎、殿中侍御史裏行。改左司諫、同管勾國子監,遷起居舍人、知諫院。時範仲淹坐言事奪職知睦州,餘靖、尹洙論救仲淹,相繼貶斥。歐陽修乃移書責若訥曰:「仲
 淹剛正,通古今,班行中無比。以非辜逐,君為諫官不能辨,猶以面目見士大夫,出入朝廷,是不復知人間有羞恥事耶!今而後,決知足下非君子。」若訥忿,以其書奏,貶修夷陵令。未幾,加直史館,以刑部員外郎兼侍御史知雜事。



 王蒙正知蔡州,若訥言:「蒙正起裨販,因緣戚裏得官。向徙郴州,物論猶不平,今予之大州,可乎?」詔寢其命。大慶殿設祈福道場,若訥奏曰:「大慶殿非行禮不御,非法服不坐,國之路寢也,豈可聚老、釋為瀆慢?」閻文應為
 入內都知,若訥言其肆橫不法,請出之,遂出文應為相州兵馬鈐轄。又奏三公坐而論道,今二府對才數刻,何以盡萬幾?宜賜坐從容,如唐延英故事。



 擢天章閣待制、知永興軍,留判吏部流內銓,出為河東路都轉運使。召還,兼侍讀、權判尚書刑部。丁母憂,始許行服,給實奉終喪。服除,加龍圖閣直學士、史館修撰,以右諫議大夫權御史中丞。時宰相賈昌朝與參知政事吳育數爭事上前。明年春,大旱,帝問所以然者,若訥曰:「陰陽不和,責在
 宰相。《洪範》,大臣不肅,則雨不時若。」於是昌朝及育皆罷,若訥遂代育為樞密副使。



 王則據貝州,討之,逾月未下。或議招降,若訥言:「河朔重兵所積,今釋不討,後且啟亂階。」及破城,知州張得一送御史臺劾治,有臣賊狀。朝廷議貸死,若訥謂:「守臣不死,自當誅,況為賊屈?」得一遂棄市。



 以工部侍郎、參知政事為樞密使。凡內降恩,若訥多覆奏不行。入內都知王守忠欲得節度使,固執為不可。若訥畏惕少過,而前騶驅路人輒至死,御史奏彈之。皇
 祐五年,罷為觀文殿學士兼翰林侍讀學士、尚書左丞、同群牧制置使、判尚書都省,止命舍人草詞。卒,贈右僕射,謚文莊。



 若訥強學善記,自秦、漢以來諸傳記無不該通,尤喜申、韓、管子之書,頗明歷學。因母病,遂兼通醫書,雖國醫皆屈伏。張仲景《傷寒論訣》、孫思邈《方書》及《外臺秘要》久不傳,悉考校訛謬行之,世始知有是書。名醫多出衛州,皆本高氏學焉。



 皇祐中,詔累黍定尺以制鐘律,爭論連年不決。若訥以漢貨泉度一寸,依《隋書》定尺十
 五種上之。並損益祠祭服器,悉施用。有集二十卷。



 孫沔,字符規,越州會稽人。中進士第,補趙州司理參軍。跌蕩自放,不守士節,然材猛過人。後以秘書丞為監察御史裏行。



 景祐元年,禮院奏用冬至日冊后,沔奏:「喪未祥禫而行嘉禮,非制也。」同安縣尉李安世上書指切朝政,被劾,沔奏:「加罪安世,恐杜天下言者,請勿治。」黜知衡山縣。道上書言時事,再貶永州監酒。移通判潭州、知處州。復為監察御史,再知楚州。所在皆著能跡。召為左正
 言,論事益有直名。遷尚書工部員外郎,提舉兩浙刑獄,遂以起居舍人為陜西轉運使。



 時宰相呂夷簡求罷,仁宗優詔弗許。沔上書言:「自夷簡當國,黜忠言,廢直道,及以使相出鎮許昌,乃薦王隨、陳堯叟代己。才庸負重,謀議不協,忿爭中堂,取笑多士,政事寢廢。又以張士遜冠臺席,士遜本乏遠識,至隳國事。蓋夷簡不進賢為社稷遠圖,但引不若己者為自固之計,欲使陛下知輔相之位非己不可,冀復思己而召用也。陛下果召夷簡還,自
 大名入秉朝政,於茲三年,不更一事。以姑息為安,以避謗為智。西州將帥累以敗聞,契丹無厭,乘此求賂。兵殲貨悖,天下空竭,刺史牧守,十不得一。法令變易,士民怨嗟,隆盛之基,忽至於此。今夷簡以病求退,陛下手和御藥,親寫德音,乃謂『恨不移卿之疾在於朕躬』,四方義士傳聞詔語,有泣下者。夷簡在中書二十年,三冠輔相,所言無不聽,所請無不行,有宋得君,一人而已,未知何以為陛下報?天下皆稱賢而陛下不用者,左右毀之也;皆
 謂憸邪而陛下不知者,朋黨蔽之也。比契丹復盟,西夏款塞,公卿忻忻,日望和平。若因此振紀綱,修廢墜,選賢任能,節用養兵,則景德、祥符之風,復見於今矣。若恬然不顧,遂以為安,臣恐土崩瓦解,不可復救。而夷簡意謂四方已寧,百度已正,欲因病默默而去,無一言啟沃上心,別白賢不肖,雖盡南山之竹,不足書其罪也。」



 書聞,帝不之罪,議者喜其謇切。居兩月,以天章閣待制為都轉運使,又遷禮部郎中,為環慶路都總管、安撫經略使、知
 慶州。元昊死,諸將欲乘其隙,大舉滅之。沔曰:「乘危伐喪,非中國體。」三司所給特支,物惡而估高,軍士有語,優人因戲及之。沔曰:「此朝廷特賜,何敢妄言動眾!」命斬之徇。將佐爭言:「此特戲爾,不足深罪也。」沔徐呼還,杖脊配嶺南,謂之曰:「汝賴戲我前,即私議動眾,汝必死,而告者超遷矣。」明日,給特支,士無敢歡者。



 歷知陜州、河東都轉運使,又知慶州,聚戰亡遺骸葬祭之,軍中感泣。凡三知慶州,邊人服其能。遷龍圖閣直學士,又遷樞密直學士、知
 成都府,未至,以母喪罷。服除,為陜西都轉運使。求知明州,會京東多盜,乃以知徐州,明購賞,嚴誅罰,盜遂止。



 徙秦州,時儂智高反,沔入見,帝以秦事勉之。對曰:「臣雖老,然秦州不足煩聖慮,陛下當以嶺南為憂也。臣睹賊勢方張,官軍朝夕當有敗奏。」明日,聞蔣偕死,帝諭執政曰:「南事誠如沔所料。」宰相龐籍奏遣沔行,以為湖南、江西路安撫使,以便宜從事,加廣南東、西路安撫使。沔請益發騎兵,且增選偏裨二十八人,求武庫精甲五千。參知
 政事梁適折之曰:「毋張皇!」沔曰:「前日惟亡備,故至此。今指期滅賊,非可以僥幸勝,乃欲示鎮靜耶?夫實備不至而貌為鎮靜,危亡之道也。」居二日,促行,才與兵七百。沔憂賊度嶺而北,乃檄湖南、北曰:「大兵且至,其繕治營壘,多具宴犒。」賊疑不敢北侵。會遣狄青為宣撫使,沔與青會。青與智高遇,戰歸仁鋪,智高敗走。青還,沔留治後事,遷給事中。及還,帝問勞,解御帶賜之,以知杭州。至南京,召為樞密副使。



 張貴妃薨,追冊為皇后,命沔讀冊。故事,
 正後,翰林學士讀冊。沔既陳不可用宰相護葬,且曰:「陛下若以臣沔讀冊則可,以樞密副使讀冊則不可。」遂求罷職。以資政殿學士知杭州。遷大學士,徙知青州。又遷觀文殿學士、知並州。而諫官吳及、御史沈起奏沔淫縱無檢,守杭及並所為不法,乃徙壽州。



 詔按其跡,而使者奏:「沔在處州時,於游人中見白牡丹者,遂誘與奸。及在杭州,嘗從蕭山民鄭旻市紗,旻高其直,沔為恨。會旻貿紗有隱而不稅者,事覺,沔取其家簿記,積計不稅者幾
 萬端,配隸旻他州。州人許明有大珠百,沔妻弟邊珣以錢三萬三千強市之。沔愛明所藏郭虔暉畫《鷹圖》,明不以獻。初,明父禱水仙大王廟生明,故幼名『大王兒』。沔即捕按明僭稱王,取其畫鷹,刺配之。及沔罷去,明詣提點刑獄,斷一臂自訟,乃得釋。杭州人金氏女,沔白晝使吏卒輿致,亂之。有趙氏女已許嫁莘旦,沔見西湖上,遂設計取趙女至州宅,與飲食臥起。所刺配人以百數,及罷,盜其按去,後有訴冤者多以無按,不能自解。在並州,私
 役使吏卒,往來青州、麟州市賣紗、絹、綿、紙、藥物。官庭列大梃,或以暴怒擊訴事者,嘗剔取盜足後筋,斷之。」奏至,乃責寧國節度副使,監司坐失察,皆被絀。其後復光祿卿,分司南京,居宿州。會恩,知濠州,以尚書禮部侍郎致仕。



 英宗即位,遷戶部。帝與執政議守邊者,難其人,參知政事歐陽修奏:「孫沔向守環慶,養練士卒,招撫蕃夷,恩信最著。今雖七十,心力不衰,中間曾以罪廢,然宜棄瑕使過。」遂起為資政殿學士、知河中府,又以為觀文殿學
 士、知慶州,徙延州,道卒。



 沔居官以才力聞,強直少所憚,然喜宴游女色,故中間坐廢。妻邊氏悍妒,為一時所傳。初,陜西用兵,朝廷多假邊帥倚以集事,近臣出帥或驕恣越法。及沔廢後,真定路安撫使呂溱繼得罪,自此守帥之權宜微矣。



 論曰:君子惟能立身,而後可以佐國。中正、起自陷朋黨,遵、稹憸邪,沔頗知兵而以污敗。琳有才器,能斷大事,然獻《武后臨朝圖》於章獻,君子鄙之。雍任邊寄而覆軍敗
 將,幾不自保。若訥喜申、韓、管子之書,中師、布少所建明,殆亦未足與議也



\end{pinyinscope}