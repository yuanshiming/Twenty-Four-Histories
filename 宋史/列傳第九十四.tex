\article{列傳第九十四}

\begin{pinyinscope}

 種世衡
 子古諤誼孫樸師道師中



 種世衡,字仲平,放之兄子也。少尚氣節,昆弟有欲析其貲者,悉推與之,惟取圖書而已。以放蔭補將作監主簿,累遷太子中舍。



 嘗知涇陽縣,里胥王知謙以奸利事敗,
 法當徙,遁去。比郊赦輒出,世衡曰「送府則會赦」,杖其脊而請罪於府,知府李諮奏釋之。後通判鳳州。州將王蒙正,章獻後姻家也,所為不法。嘗干世衡以私,不聽,蒙正怒,乃誘知謙訟冤而陰助之,世衡坐流竇州,徙汝州。弟世材上一官以贖,為孟州司馬。久之,龍圖閣直學士李紘為辨其誣,宋綬、狄棐繼言之,除衛尉寺丞,歷監隨州酒,簽書同州、鄜州判官事。



 西邊用兵,守備不足。世衡建言,延安東北二百里有故寬州,請因其廢壘而興之,以當
 寇沖,右可固延安之勢,左可致河東之粟,北可圖銀、夏之舊。朝廷從之,命董其役。夏人屢出爭,世衡且戰且城之。然處險無泉,議不可守。鑿地百五十尺,始至於石,石工辭不可穿,世衡命屑石一畚酬百錢,卒得泉。城成,賜名青澗城。遷內殿崇班、知城事。開營田二千頃,募商賈,貸以本錢,使通貨贏其利,城遂富實。間出行部族,慰勞酋長,或解與所服帶。嘗會客飲,有得敵情來告者,即以飲器予之,繇是屬羌皆樂為用。再遷洛苑副使、知環
 州。



 蕃部有牛家族奴訛者,素屈強,未嘗出謁郡守,聞世衡至,遽郊迎。世衡與約,明日當至其帳,往勞部落。是夕大雪,深三尺。左右曰:「地險不可往。」世衡曰:「吾方結諸羌以信,不可失期。」遂緣險而進。奴訛方臥帳中,謂世衡必不能至,世衡蹴而起,奴訛大驚曰:「前此未嘗有官至吾部者,公乃不疑我耶!」率其族羅拜聽命。



 羌酋慕恩部落最強,世衡嘗夜與飲,出侍姬以佐酒。既而世衡起入內,潛於壁隙中窺之。慕恩竊與侍姬戲,世衡遽出掩之,慕
 恩慚懼請罪。世衡笑曰:「君欲之耶?」即以遺之,由是得其死力。諸部有貳者,使討之無不克。有兀二族,世衡招之不至,即命慕恩出兵誅之。其後百餘帳皆自歸,莫敢貳。因令諸族置烽火,有急則舉燧,介馬以待。



 葛懷敏敗,率羌兵數千人以振涇原,無敢後者。嘗課吏民射,有過失,射中則釋其罪;有辭某事、請某事,輒因中否而與奪之。人人自厲,皆精於射,繇是數年敵不敢近環境。



 遷東染院使、環慶路兵馬鈐轄。範仲淹檄令與蔣偕築細腰城,
 世衡時臥病,即起,將所部甲士晝夜興築,城成而卒。



 初,世衡在青澗城,元昊未臣,其貴人野利剛浪□夌、遇乞兄弟有材謀,皆號大王。親信用事,邊臣欲以謀間之。慶歷二年,鄜延經略使龐籍,兩為保安軍守劉拯書,賂蕃部破醜以達野利兄弟,而涇原路王沿、葛懷敏亦遣人持書及金寶以遣遇乞。會剛浪□夌令浪埋、賞乞、媚娘等三人詣世衡請降,世衡知其詐,曰:「與其殺之,不若因以為間。」留使監商稅,出入騎從甚寵。有僧王光信者,趫勇善
 騎射,習知蕃部山川道路。世衡出兵,常使為鄉導,數蕩族帳,奏以為三班借職,改名嵩。世衡為蠟書,遣嵩遺剛浪□夌,言浪埋等已至,朝廷知王有向漢心,命為夏州節度使,奉錢月萬緡,旌節已至,趣其歸附,以棗綴畫龜,喻其早歸之意。剛浪□夌得書大懼,自所治執嵩歸元昊。元昊疑剛浪□夌貳己,不得還所治,且錮嵩阱中。使其臣李文貴以剛浪□夌旨報世衡,且言不達所遺書意,或許通和,願賜一言。世衡以白籍。時朝廷已欲招拊,籍召文貴
 至,諭以國家寬大開納意,縱使還報。元昊得報,出嵩,禮之甚厚,使與文貴偕來。自是繼遣使者請降,遂稱臣如舊。世衡聞野利兄弟已誅,為文越境祭之。籍疏嵩勞,具言元昊未通時,世衡畫策遣嵩冒艱險間其君臣,遂成猜貳,因此與中國通,請優進嵩官。遷三班奉職。後嵩因對自陳,又進侍禁、閣門祗候。



 世衡死,籍為樞密使。世衡子古上書訟父功,為籍所抑。古復上書,遂贈世衡成州團練使,詔流內銓授古大縣簿尉,押還本貫。籍既罷,古
 復辯理,下御史考驗,以籍前奏王嵩疏為定。詔以其事付史官,聽古從官便郡。



 世衡在邊數年,積穀通貨,所至不煩縣官益兵增饋。善撫養士卒,病者遣一子專視其食飲湯劑,以故得人死力。及卒,羌酋朝夕臨者數日,青澗及環人皆畫像祠之。子古、諤、診,皆有將材。關中號曰「三種」。誼,其幼子也。孫樸、師道、師中。



 古字大質,少慕從祖放為人,不事科舉。當任官,辭以與弟,時稱「小隱君」。世衡卒,錄古為天興尉,累轉西京左藏
 庫副使、涇原路都監、知原州。



 羌人犯塞,古御之。斬級數百。築城鎮戎之北,以據要害。神宗召對,遷通事舍人,官其三弟。與弟診破環州折姜會,斬首二千級,遷西上閣門副使。民有損直鬻田於熟羌以避役者,古按其狀,得良田三千頃,丁四千,悉刺為民兵。歷環慶、永興軍路鈐轄。



 坐訟范純仁不當,奪一官,知寧州,徙鎮戎軍。熙河師十萬道境上,須芻糧,僚佐以他路為言。古曰:「均王師也。」命給之。又徙鄜、隰二州,卒,年七十。



 古明達孝義。弟諤坐
 擅興系獄,乞納官贖其罪。世衡遺張問田千畝,問返之,而世衡死,古終不復受。然世衡受知於範仲淹,因立青澗功,而古以私憾訟純仁,士論少之。



 諤字子正,以父任累官左藏庫副使,延帥陸詵薦知青澗城。



 夏酋令㖫內附,詵恐生事,欲弗納,諤請納之。夏人來索,詵問所以報,諤曰:「必欲令㖫,當以景詢來易。」乃止。詢者,中國亡命至彼者也。



 夏將嵬名山部落在故綏州,其弟夷山先降,諤使人因夷山以誘之,賂以金盂,名山
 小吏李文喜受而許降,而名山未之知也。諤即以聞,詔轉運使薛向及陸詵委諤招納。諤不待報,悉起所部兵長驅而前,圍其帳。名山驚,援槍欲鬥,夷山呼曰:「兄已約降,何為如是?」文喜因出所受金盂示之,名山投槍哭,遂舉眾從諤而南。得酋領三百、戶萬五千、兵萬人。將築城,詵以無詔出師,召諤還。軍次懷遠,晨起方櫛,敵四萬眾坌集,傅城而陳。諤開門以待,使名山帥新附百餘人挑戰,諤兵繼之,鼓行而出。至晉祠據險,使偏將燕達、劉甫
 為兩翼,身為中軍,乃閉壘,悉老弱乘城鼓噪以疑賊。已而合戰,追擊二十里,俘馘甚眾,遂城綏州。詵劾諤擅興,且不稟節制,欲捕治,未果而詵徙秦。言者交攻之,遂下吏,貶秩四等,安置隨州。會侯可以言水利入見,神宗問其事,對曰:「種諤奉密旨取綏而獲罪,後何以使人?」帝亦悔,復其官。



 韓絳宣撫陜西,用為鄜延鈐轄。絳城囉兀,規橫山,令諤將兵二萬出無定川,命諸將皆受節度,起河東兵會銀州。城成而慶卒叛,詔罷師,棄囉兀,責授汝州
 團練副使。再貶賀州別駕,移單州,又移華州。絳再相,訟其前功,復禮賓副使、知岷州。董氈將鬼章聚兵於洮、岷,新羌多叛,諤討襲誅之。從李憲出塞,收洮州,下逋宗、講珠、東宜諸城,掩擊至大河,斬首七千級。



 遷東上閣門使、文州刺史、知涇州,徒鄜延副總管。上言:「夏主秉常為其母所囚,可急因本路官搗其巢穴。」遂入對,大言曰:「夏國無人,秉常孺子,臣往持其臂以來耳。」帝壯之,決意西討,以為經略安撫副使,諸將悉聽節制。諤即次境上,帝以
 諤先期輕出,使聽令於王中正。敵屯兵夏州,諤率本路並畿內七將兵攻米脂,三日未下。夏兵八萬來援,諤御之無定川,伏兵發,斷其首尾,大破之,降守將令介訛遇。捷書聞,帝大喜,群臣稱賀,遣中使諭獎,而罷中正。諤留千人守米脂,進次銀、石、夏州,不見敵。始,被詔當會靈武,諤迂枉不進,士卒饑憊,欲以糧運不繼歸罪轉運使李稷。駐軍麻家平,大校劉歸仁以眾潰,詔令班師。猶遷鳳州團練使、龍神衛四廂都指揮使。



 諤謀據橫山之志未
 已,遣子樸上其策。帝召樸問狀,擢為閣門祗候。將進城橫山,命徐禧、李舜舉使鄜延計議。諤言:「橫山延袤千里,多馬宜稼,人物勁悍善戰,且有鹽鐵之利,夏人恃以為生;其城壘皆控險,足以守御。今之興功,當自銀州始。其次遷宥州,又其次修夏州,三郡鼎峙,則橫山之地已囊括其中。又其次修鹽州,則橫山強兵戰馬、山澤之利,盡歸中國。其勢居高,俯視興、靈,可以直覆巢穴。」而禧與沉括定議移銀州,城永樂,與諤始謀異,乃奏留諤守延。既
 而永樂受圍,諤觀望不救,帝冀其後效,置不問,且虞賊至,就命知延州。疽發背卒,年五十七。



 諤善馭士卒,臨敵出奇,戰必勝,然詐誕殘忍,左右有犯立斬,或先刳肺肝,坐者掩面,諤飲食自若。敵亦畏其敢戰,故數有功。李稷之饋軍也,旦入諤營,軍吏鳴鼓聲喏。諤呼問吏曰:「軍有幾帥?要當借汝頭以代運使。」即叱斬之。稷惶怖遽出。嘗渡河,猝遇敵,紿門下客曰:「事急矣,可衣我衣,乘我馬,從旗鼓千騎,亟趨大軍。」客信之,敵以為諤,追之,幾不免。自
 熙寧首開綏州,後再舉西征,皆其兆謀,卒致永樂之禍。議者謂諤不死,邊事不已。



 誼字壽翁。熙寧中,古入對,神宗問其家世,命誼以官。從高遵裕復洮、岷,又平山後羌,至熙河副將。



 使青唐,董氈遣鬼章迎候境上,取道故為回枉,以誇險遠。誼固習其地里,誚之曰:「爾跳梁坎井間,謂我不知遠近邪?」命趨便道。鬼章怒,脅以兵,誼聲氣不動,卒改塗。外為路都監。自蘭州渡河討賊,斬首六百,累轉西京使。元祐初,知岷州。
 鬼章誘殺景思立,後益自矜,大有窺故土之心,使其子詣宗哥請益兵入寇,且結屬羌為內應。誼刺得其情,上疏請除之。詔遣游師雄就商利害,遂與姚兕合兵出討。羌迎戰,擊走之,追奔至洮州。誼亟進攻,晨霧蔽野,跬步不可辨。誼曰:「吾軍遠來,彼固不知厚薄,乘此可一鼓而下也。」遂親鼓之。有頃,霧霽,先登者已得城,鬼章就執。誼戲問之曰:「別後安否?」不能對,徐謂人曰:「我生惡種使,今日果為所擒。天不使我復有故土,命也。」遂俘以歸。拜西
 上閣門使、康州刺史,徙知鄜州。



 夏人犯延安,趙離使誼統諸將。敵聞誼至,皆潰去。延人謂:「得誼,勝精兵二十萬。」進熙河鈐轄、知蘭州。蘭與通遠皆絕塞,中間保障不相接,腴田多棄不耕,誼請城李諾平以扼沖要。會遷東上閣門使、保州團練使,卒,年五十五。



 誼倜儻有氣節,喜讀書。蒞軍整嚴,令一下,死不敢避;遇敵,度不勝不出,故每戰未嘗負敗。岷羌酋包順、包誠恃功驕恣,前守務姑息,誼至,厚待之。適有小過,叱下吏,將置法,順、誠叩頭伏罪,
 願效命以贖,乃使輸金出之,群羌畏惕。及洮州之役,二人功最多。



 樸以父任右班殿直,積勞,遷至皇城使、昌州刺史,徙熙河蘭會鈐轄兼知河州,安撫洮西沿邊公事。



 河南蕃部叛,屬羌阿章率他族拒官軍,熙帥胡宗回使樸出討。時樸至州才二日,以賊鋒方銳,且盛寒,欲姑徐之,而宗回馳檄至六七,不得已,遂出兵。羌知樸來,伏以待。樸遇伏,首尾不相應,樸殊死戰,為賊所殺,以馬負其尸去。羌乘
 勝追北。師還遇隘,壅迮不得行。偏將王舜臣者善射,以弓卦臂,獨立敗軍後。羌來可萬騎,有七人介馬而先。舜臣念此必羌酋之尤桀黠者,不先殪之,吾軍必盡。乃宣言曰:「吾令最先行者眉間插花。」引弓三發,隕三人,皆中面;餘四人反走,矢貫其背。萬騎□咢眙莫敢前,舜臣因得整眾。須臾,羌復來。舜臣自申及酉,抽矢千餘發,無虛者。指裂,血流至肘。薄暮,乃得逾隘。將士氣奪,無敢復言戰。當是時,微舜臣則師殲矣。事聞,贈樸雄州防禦使,官其
 後十人。



 師道字彞叔。少從張載學,以蔭補三班奉職,試法,易文階,為熙州推官、權同谷縣。縣吏有田訟,彌二年不決。師道翻閱案牘,窮日力不竟,然所訟止母及兄而已。引吏詰之曰:「母、兄,法可訟乎。汝再期擾鄉里足未?」吏叩頭服罪。



 通判原州,提舉秦鳳常平。議役法忤蔡京旨,換莊宅使、知德順軍。又謂其詆毀先烈,罷入黨籍,屏廢十年。以武功大夫、忠州刺史、涇原都鈐轄知懷德軍。夏國畫境,
 其人焦彥堅必欲得故地,師道曰:「如言故地,當以漢、唐為正,則君家疆土益蹙矣。」彥賢無以對。



 童貫握兵柄而西,翕張威福,見者皆旅拜,師道長揖而已。召詣闕,徽宗訪以邊事,對曰:「先為不可勝,來則應之。妄動生事,非計也。」貫議徙內郡弓箭手實邊,而指為新邊所募。帝復訪之,對曰:「臣恐勤遠之功未立,而近擾先及矣。」帝善其言,賜襲衣、金帶,以為提舉秦鳳弓箭手。時五路並置官,帝謂曰:「卿,吾所親擢也。」貫滋不悅,師道不敢拜,以請,得提
 舉崇福宮。久之,知西安州。



 夏人侵定邊,築佛口城,率師往夷之。始至渴甚,師道指山之西麓曰:「是當有水。」命工求之,果得水滿谷。累遷龍神衛四廂都指揮使、洺州防禦使、知渭州。督諸道兵城席葦,土賦工,敵至,堅壁葫蘆河。師道陳於河滸,若將決戰者。陰遣偏將曲克徑出橫嶺,揚言援兵至,敵方駭顧,楊可世潛軍軍其後,姚平仲以精甲衷擊之,敵大潰,斬首五十級,獲橐駝、馬牛萬計,其酋僅以身免。卒城而還。



 又詔帥陜西、河東七路兵征
 臧底城,期以旬日必克。既薄城下,敵守備甚固。官軍小怠,列校有據胡床自休者,立斬之,尸於軍門。令曰:「今日城不下,視此。」眾股慄,噪而登城,城即潰,時兵至才八日。帝得捷書喜,進侍衛親軍馬軍副都指揮使、應道軍承宣使。



 從童貫為都統制,拜保靜軍節度使。貫謀伐燕,使師道盡護諸將。師道諫曰:「今日之舉,譬如盜入鄰家不能救,又乘之而分其室焉,無乃不可乎?」貫不聽。既次白溝,遼人噪而前,士卒多傷。師道先令人持一巨梃自防,
 賴以不大敗。遼使來請曰:「女真之叛本朝,亦南朝之所甚惡也。今射一時之利,棄百年之好,結豺狼之鄰,基他日之禍,謂為得計可乎?救災恤鄰,古今通義,惟大國圖之。」貫不能對,師道復諫宜許之,又不聽,密劾其助賊。王黼怒,責為右衛將軍致仕,而用劉延慶代之。延慶敗績於盧溝,帝思其言,起為憲州刺史、知環州,俄還保靜軍節度使,復致仕。



 金人南下,趣召之,加檢校少保、靜難軍節度使、京畿河北制置使,聽便宜檄兵食。師道方居南
 山豹林谷,聞命即東。過姚平仲,有步騎七千,與之俱北。至洛陽,聞斡離不已屯京城下,或止勿行曰:「賊勢方銳,願少駐汜水,以謀萬全。」師道曰:「吾兵少,若遲回不進,形見情露,祗取辱焉。今鼓行而前,彼安能測我虛實?都人知吾來,士氣自振,何憂賊哉!」揭榜沿道,言種少保領西兵百萬來。遂抵城西,趨汴水南,徑逼敵營。金人懼,徙砦稍北,斂游騎,但守牟駝岡,增壘自衛。



 時師道春秋高,天下稱為「老種」。欽宗聞其至,喜甚,開安上門,命尚書右丞
 李綱迎勞。時已議和,入見,帝問曰:「今日之事,卿意如何?」對曰:「女真不知兵,豈有孤軍深入人境而能善其歸乎?」帝曰:「業已講好矣。」對曰「臣以軍旅之事事陛下,餘非所敢知也。」拜檢校少傅、同知樞密院、京畿兩河宣撫使,諸道兵悉隸焉。以平仲為都統制。師道時被病,命毋拜,許肩輿入朝。金使王汭在廷頡頏,望見師道,拜跪稍如禮。帝顧笑曰:「彼為卿故也。」京城自受圍,諸門盡閉,市無薪菜。師道請啟西、南壁,聽民出入如常。金人有擅過偏將
 馬忠軍者,忠斬其六人。金人來訴,師道付以界旗,使自為制,後無有敢越佚者。又請緩給金幣,使彼惰歸,扼而殲諸河,執政不可。



 種氏、姚氏皆為山西巨室,平仲父古方以熙河兵入援。平仲慮功名獨歸種氏。乃以士不得速戰為言達於上。李綱主其議,令城下兵緩急聽平仲節度。帝日遣使趣師道戰,師道欲俟其弟秦鳳經略使師中至,奏言過春分乃可擊。時相距才八日,帝以為緩,竟用平仲斫營,以及於敗。既敗,李邦彥議割三鎮,師道
 爭之不得。李綱罷,太學諸生、都人伏闕願見種、李,詔趣使彈壓。師道乘車而來,眾褰廉視之,曰:「果我公也。」相率聲喏而散。



 金師退,乃罷為中太一宮使。御史中丞許翰見帝,以為不宜解師道兵柄。上曰:「師道老矣,難用,當使卿見之。」令相見於殿門外。師道不語,翰曰:「國家有急,詔許訪所疑,公勿以書生之故不肯談。」師道始言:「我眾彼寡,但分兵結營,控守要地,使彼糧道不通,坐以持久,可破也。」翰嘆味其言,復上奏謂師道智慮未衰,尚可用。於
 是加檢校少師,進太尉,換節鎮洮軍,為河北、河東宣撫使,屯滑州,實無兵自隨。



 師道請合關、河卒屯滄、衛、孟、滑,備金兵再至。朝論以大敵甫退,不宜勞師以示弱,格不用。既而師中戰死,姚古敗,朝廷震悚,召師道還。太原陷,又使巡邊。次河陽,遇王汭,揣敵必大舉,亟上疏請幸長安以避其鋒。大臣以為怯,復召還。既至,病不能見。十月,卒,年七十六。帝臨奠,哭之慟,贈開府儀同三司。



 京師失守,帝搏膺曰:「不用種師道言,以至於此!」金兵之始退也,
 師道申前議,勸帝乘半濟擊之,不從,曰:「異日必為國患。」故追痛其語。建炎中,加贈少保,謚曰忠憲。



 師中字端孺。歷知環、濱、邠州、慶陽府、秦州,侍衛步軍馬軍副都指揮使、房州觀察使,奉寧軍承宣使。



 金人內侵,詔提秦鳳兵入援,未至而敵退,乃以二萬人守滑。遣副姚古為河北制置使,古援太原,師中援中山、河間。或謂師中自磁、相而北,金人若下太行,則勢不能自還,此段凝師於河上比也。時大臣立議矛盾,樞密主破敵,而三
 省令護出之。師中渡河,即上言:「粘罕已至澤州,臣欲由邢、相間捷出上黨,搗其不意,當可以逞。」朝廷疑不用。乾離不還,師中逐出境。粘罕至太原,悉破諸縣,為鎖城法困之,內外不相通。姚古雖復隆德、威勝,扼南北關,而不能解圍。於是詔師中由井陘道出師,與古掎角,進次平定軍,乘勝復壽陽、榆次,留屯真定。時粘罕避暑雲中,留兵分就畜牧,覘者以為將遁,告諸朝。知樞密院許翰信之,數遣使督師中出戰,且責以逗撓。師中嘆曰:「逗撓,兵
 家大戮也。吾結發從軍,今老矣,忍受此為罪乎!」即日辦嚴,約古及張灝俱進,輜重賞犒之物,皆不暇從行。五月,抵壽陽之石坑,為金人所襲。五戰三勝,回趨榆次,去太原百里,而古、灝失期不至,兵饑甚。敵知之,悉眾攻,右軍潰而前軍亦奔。師中獨以麾下死戰,自卯至巳,士捽發神臂弓射退金兵,而賞繼不及,皆憤怨散去,所留者才百人。師中身被四創,力疾鬥死。



 師中老成持重,為時名將,諸軍自是氣奪。劉韐言:「師中聞命即行,奮不顧身,雖
 古忠臣,不過也。」請加優贈,以勸死國者。詔贈少師,謚曰莊愍。



 論曰:宋懲五季藩鎮之弊,稍用逢掖治邊陲、領介冑。然兵勢國之大事,非素明習,而欲應變決策於急遽危難之際,豈不僕哉。種氏自世衡立功青澗,撫循士卒,威動羌、夏,諸子俱有將材,至師道、師中已三世,號山西名將。徽宗任宦豎起邊釁,師道之言不售,卒基南北之禍。金以孤軍深入,師道請遲西師之至而擊之,長驅上黨;師
 中欲出其背以掩之,可謂至計矣。李綱、許翰顧以為怯緩逗撓,動失機會,遂至大衄,而國隨以敗,惜哉!



\end{pinyinscope}