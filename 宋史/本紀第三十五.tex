\article{本紀第三十五}

\begin{pinyinscope}

 孝宗三



 五年春正月辛丑,侍御史謝廓然乞戒有司毋以程頤、王安石之說取士。從之。癸卯,罷特旨免臣僚及寺觀科徭。庚戌,大風。己未,詔侍從、臺諫、兩省官集議考課法。



 二
 月己巳,置州縣丁稅司。辛未,申嚴武臣呈試法。詔二廣毋以攝官人治獄。丁丑,禁解鹽入京西界。甲申。雨土。庚寅,威州蠻寇邊,討降之。



 三月丁未,李彥穎罷。給辰、沅、澧、靖四州刀弩手田。壬子,以史浩為右丞相。丁巳,幸玉津園。己未,以王淮知樞密院事,趙雄參知政事。是春,黎州蠻出降。



 夏四月乙丑朔,詔葉衡任便居住。丙寅,以禮部尚書範成大參知政事。辛未,知紹興府張津進羨餘四十萬緡,詔以代民輸和買、身丁之半。賜禮部進士姚穎
 以下四百十有七人及第、出身。丁丑,雨土。己卯,以趙思奉使不如禮,罷起居舍人,仍降二官。丁亥,命後省擇中外所言利病不戾成法者以聞。



 五月庚子,置武學國子員。丁未,修臨安府城。禁諸路州軍責屬縣進羨餘。



 六月庚午,飭百官及諸監司毋得請托。乙亥,範成大罷。癸未,詔京西、湖北商人以牛馬負茶出境者罪死。甲申,詔翰林學士、諫議大夫、給事中、中書舍人、侍御史各舉堪御史者二人。以給事中錢良臣簽書樞密院事。己丑,罷諸
 州私置稅場。減四川茶課十五萬餘緡。庚寅,蠲大理寺贓錢三萬九千餘緡。



 閏月丙申,贈強霓、強震官,立廟西和州,賜名旌忠。丁酉,限四川總領會子額。戊戌,罷興州都統司營田官兵,募民耕佃。己亥,復分利州東、西路為二。壬寅,置鎮江、建康府轉般倉。龔茂良卒於英州。乙巳,以魏王愷為永興、成德軍節度使、雍州牧、判明州如故。庚戌,蠲秀州民折帛錢。



 秋七月甲子,太尉、提舉萬壽觀李顯忠薨。癸未,禁砂毛錢。丁亥,以歲豐,命沿江糴米百
 六十萬石,以廣邊儲。



 八月甲午,詔諸路監司戒所部,民稅毋以重價強折輸錢。復制科舊法。丁酉,詔關外四州增募民兵為忠勇軍。戊午,增銓試為五場,呈試為四場。



 九月甲子,定廣西賣鹽賞罰。壬申,幸秘書省。戊寅,賜岳飛謚曰武穆。



 冬十月戊戌,史浩等上《三祖下第六世仙源類譜》、《仁宗玉牒》。庚子,遣宇文價等使金賀正旦。辛亥,金遣張九思等來賀會慶節。乙卯,奉國軍節度使、殿前都指揮使王友直以募兵擾民,降為武寧
 軍承宣使,罷軍職,統制以下奪官有差。軍民嘩呶者,執送大理寺鞫之。戊午,以孫右千牛衛大將軍擴為明州觀察使,封英國公。



 十一月丙寅,詔軍民喧哄者,並從軍法。史浩言民不宜律以軍法,不聽。王友直再降為宜州觀察使、信州居住。浩請罷政。甲戌,浩罷為少傅,還舊節,充醴泉觀使兼侍讀。乙亥,以錢良臣參知政事。丁丑,以趙雄為右丞相,王淮為樞密使。戊寅,以兩川禁卒千人為成都府雄邊軍。庚辰,復監司互察法。十二月庚寅朔,班新定薦舉
 式。辛卯,遣錢沖之等賀金主生辰。丁酉,罷興元都統司營田官兵,募民耕佃。辛丑,復同安、蘄春監。丙午,禁兩淮銅錢,復行鐵錢。丙辰,金遣烏延察等來賀明年正旦。是歲,階、福建興化軍水,通、泰、楚州、高郵軍田鼠傷禾。三佛齊國入貢。



 六年春正月戊辰,振淮東饑民。庚午,復置內侍省合同憑由司。壬申,蠲夔州路上供金銀。丁丑,雨雹。辛巳,復置光州中渡榷場。



 二月己丑朔,幸祐聖觀,召史浩、曾覿賜
 酒。壬辰,錢良臣以失舉贓吏,奪三官,丙申,詔前宰執、侍從有己見利便,聽不時以聞,辛丑,立武臣關升蔭補法。丙午,詔逃軍犯強盜者毋擬貸。癸丑。命州縣毋撓義役。乙卯,詔自今歸正官親赴部授官,以革冒濫。丁巳,裁特奏名試法。



 三月庚申,幸聚景園。丙寅,錄趙鼎、岳飛子孫,賜以京秩。己巳,郴州賊陳峒等破連、道州、桂陽軍諸縣,命湖南帥臣討捕之。置廣西義倉。辛未,再振淮東饑民。壬申,雨雹。丁丑,詔戒勵諸道轉運使。庚辰,幸玉津園。夏
 五月壬戎,裁宗室換官法。庚午,蠲四川鹽課十萬緡。乙亥,郴寇平。癸未,給襄陽歸正忠義人田。



 六月甲午,建豐儲倉。丙申,詔特奏名毋授知縣、縣令。戊戌,蠲郴州運糧丁夫今年役錢之半。辛亥,廣西妖賊李接破鬱林州,守臣李端卿棄城遁,遂圍化州。命經略司討捕之。端卿除名勒停、梅州編管。



 秋七月癸亥,籍郴州降寇。隸荊、鄂軍。戊辰,班《隆興以來寬恤詔令》於諸路。趙雄等上《會要》。乙亥,詔諸軍五口以上增給緡錢。癸未,太白晝見,經天。



 八
 月庚寅,罷諸路監司、帥守便宜行事。壬寅,以知楚州翟畋過淮生事,奪五官、筠州居住。



 九月辛未,合祭天地於明堂,大赦。癸未,詔福建、二廣賣鹽毋擅增舊額。



 冬十月乙酉朔,蠲連州被寇民租稅。辛卯,遣陳峴等使金賀正旦。丙申,詔太學兩優釋褐,與殿試第二人恩例。庚子,四川行當三大錢。再蠲四川鹽課十七萬餘緡。辛丑,除紹興府民逋賦五萬餘緡。乙巳,金遣蒲察鼎壽等來賀會慶節。戊申,廣西妖賊平。



 十一月乙卯朔,帝著論數百言,
 深原用人之弊,因及誅賞之法,命宰執示從臣於都堂。辛酉,裁宗子試法。戊寅,罷金州管內安撫司。壬午,詔宗室有出身人得考試及注教授官。癸未,遣傅淇等賀金主生辰。十二月丙戌,班《重修淳熙敕令格式》。丙申,修百司省記法。己亥,詔自今鞫贓吏,後雖原貸者,毋以失入坐獄官。庚戌,金遣耶律慥等來賀明年正旦。辛亥,蠲臨安府征稅一年。是歲,溫、臺州水,和州旱。



 七年春正月甲子,減廣西諸州歲賣鹽數。乙丑,劉焞以
 以平李接功,擢集英殿修撰,將佐幕屬吏士進官、減磨勘年有差。己卯,詔京西州軍並用鐵錢及會子;民戶銅錢,以鐵錢或會子償之,滿二月不輸官,許告賞。庚辰,蠲淮東民貸常平錢米。



 二月癸未朔,初置廣南煙瘴諸州醫官。丙戌,復置皇太子宮小學教授。辛卯,魏王愷薨。乙未,詔撥廣西兵校五百人隸提刑司。戊戌,罷瓜洲孳生馬監。己亥,出湖南樁積米十萬石,振糶永、邵、郴三州。甲辰,命利州路守、貳、縣令兼領營田。乙巳,限改官員歲毋過
 八十人。封子楝為宜州觀察使、安定郡王。



 三月壬戌,詔舉賢良方正能直言極諫者。庚午,迎太上皇、太上皇后宴翠寒堂。乙亥,減內外官薦舉員。丁丑,再蠲臨安府民身丁錢三年,詔諸州招補軍籍之闕,自今歲以為常。



 夏四月甲申,幸聚景園。丙戌,趙雄等上仁宗、哲宗玉牒。戊子,除明州積欠諸司錢十五萬緡。辛卯,再免沿邊歸正人請占官田賦役三年,甲辰,黎州五部落犯盤佗砦,兵馬都監高晃以綿、潼大軍三千人與戰,敗走,蠻人深入,
 大掠而去。己酉,命蔭補、武舉、宗室、小使臣行三年喪。



 五月戊辰,以吏部尚書周必大參知政事,刑部尚書謝廓然簽書樞密院事。袁州分宜縣大水,捐其稅。戊寅,詔舒、蘄二州鑄錢歲以四十五萬貫為額。己卯,申飭書坊擅刻書籍之禁。庚辰,詔特奏名年六十人毋注縣尉。



 六月丙戌,以特進、觀文殿大學士、判建康府陳俊卿為少保。壬辰,五部落再犯黎州,制置司鈐轄成光延戰敗,官軍死者甚眾,提點刑獄、權州事折知常棄城遁。甲午,制置
 司益兵,遣都大提舉茶馬吳總往平之。壬寅,詔試刑法官增試經義。



 秋七月癸丑,詔二廣帥臣、監司察所部守臣臧否以聞。丁卯,以旱,決系囚,分命群臣禱雨於山川。壬申,移廣西提刑司於鬱林州。



 八月癸未,禁黎州官吏市蕃商物。甲申,以禱雨未應,諭輔臣欲令職事官以上各實封言事。是夕,雨。丁酉,置湖南飛虎軍。戊戌,雨。甲辰,五部落犯黎州塞,興州左軍統領王去惡拒卻之,折知常重賂蠻,使之納款。



 九月癸亥,詔自今常朝毋稱丞相
 名。甲子,命樞密使亦如之。乙丑,詔宰執、使相,給使減年恩數,身後三年者毋收使。丙寅,詔知縣成資始聽監司薦舉。壬申,禁諸路遏糴。癸酉,名省記法為《淳熙重修百司法》。



 冬十月丙戌,詔:「限田太寬,民役煩重,其令臺諫、給舍同戶部長貳詳議以聞。」戊子,遣葉宏等使金賀正旦。乙未,黎州五部落進馬乞降,詔卻獻馬,許其互市。庚子,金遣李佾等來賀會慶節。



 十一月癸丑,詔邊吏存恤江西過淮饑民。丁巳,禁淮南諸司、州郡抑配民酒。辛酉,蠲
 兩淮州軍二稅一年。癸亥,黎州戍軍伍進等作亂,折知常遁去,王去惡誘進等誅之。壬申,南康軍旱,詔出檢放所餘苗米萬石充軍糧。癸酉。遣蓋經等賀金主生辰。十二月庚寅,趙雄等上神宗、哲宗、徽宗、欽宗四朝《國史志》。壬辰,以四川制置使胡元質不備蕃部,致其猖獗,奪兩官罷之。丙申,嗣濮王士輵薨。戊戌,以新除成都府路提點刑獄祿東之權四川制置司,應黎州邊事,隨
 宜措置。癸卯,詔臨安府承宣旨審奏如故事。甲辰,金遣徒單守素等來賀明年正旦。是月,詔以太上皇明年七十有五,議行慶壽禮,太上皇不允,帝進黃金二千兩為壽。是歲,江、浙、淮西、湖北旱,蠲租,發廩貸給,趣州縣決獄,募富民振濟補官。故歲雖兇,民無流殍。安南入貢。



 八年春正月甲寅,停折知常官、汀州居住。丙辰,詔、內侍見帶兵官並與在京宮觀,著為令。乙亥,詔福建歲撥鹽於邵武軍市軍糧。



 二月壬午,詔去歲旱傷郡縣,以義倉米日給貧民,至閏三月半止。黎州土丁張百
 祥等不堪科役為亂,統領官劉大年引兵逆擊之,土丁潰去,大年坐誅。戊子,禁浙西民因旱置圍田者。裁童子試法。己丑,禁廣西諸州科賣亭戶食鹽。庚寅,詔三省、樞密、六部置籍,稽考興利除害等事。戊戌,以保康軍節度使士歆為嗣濮王。



 三月丁未朔,幸祐聖觀。戊午,以潮州賊沉師為亂,趣帥、憲捕之。辛未,幸聚景園。閏月辛巳,命諸路帥臣、監司分州郡臧否為三等,歲終來上。戊子,賜禮部進士黃由以下三百七十有九人及第、出身。庚寅,修揚州城。甲午,幸玉津園。壬寅,減在京及諸路房廊錢什之三,德壽宮所減,月以南庫錢貼進。禁潭、道等州官賣鹽。甲辰,立宗室命繼法。



 夏四月癸丑,修湖南諸州城。丙辰,以臨安疫,分命醫官診視軍民。庚申,復以強盜配隸諸軍重役。丁卯,安定郡王子棟薨。癸酉,立郴州宜章、桂陽軍臨武縣學,以教養峒民子弟。



 五月戊寅,詔監司、守令勸課農桑,以奉行勤怠為賞罰。壬午,詔諸路轉運司趣民間補葺經界簿籍。辛卯,以久雨,減京畿及兩浙囚罪一等,釋杖以下,貸貧民稻種錢。壬寅,以史浩為少師。



 六月己酉,詔放殿前司平江府牧馬草場二萬畝,聽民漁採。戊午,除淳熙七年諸路旱傷檢放米一百三十七萬石、錢二千六萬緡。辛酉,罷諸路坊場監官,聽民承買。戊辰,史浩薦薛叔似、楊簡、陸九淵、陳謙、葉適、袁燮、趙善譽等十六人,詔並赴都堂審察。



 七月癸未,復以許浦水軍隸殿前司。永陽郡王居廣薨,追封永王。辛未,賞監司、守臣修舉荒政者十六人。以不雨、決系囚。
 壬辰,紹興大水,出秀、婺州、平江府米振糶。丁酉,嚴州水,詔被災之家蠲其和買,三等以上戶減半。辛丑,錄範質後。



 八月丙午,以旱,罷招軍。庚戌,趙雄罷。壬子,詔紹興府諸縣夏稅、和市、折帛、身丁錢絹之類,不以名色,截日並令住催。癸丑。以王淮為右丞相兼樞密使。甲寅,以
 謝廓然同知樞密院事。丙辰,更後殿幄次為延和殿。己未,以觀文殿大學士、新四川制置使趙雄知瀘州,戊辰,言者請自今歉歲蠲減,經費有虧,令戶部據實以聞,毋得督趣
 己蠲閣之數。從之。罷諸路補葺經界簿籍。



 九月庚辰,命諸路提舉司貸民麥種。辛巳,錢良臣罷。庚寅,以謝廓然兼權參知政事。



 冬十月己酉,遣施師點等使金賀正旦。辛酉,錄黎州戰歿將士四百三人。甲子,金遣完顏寔等來賀會慶節。詔災傷州縣諭民振糶。



 十一月甲戌,以旱傷,罷喜雪宴。戊寅,蠲富陽、新城、錢塘夏稅。庚寅,前池州守趙粹中誤斬遞卒汪青,落職,仍詔給青家衣糧十五年。辛卯,詔兩省、侍從、臺諫各舉所知。浚行在至鎮江府
 運河。丁酉,遣燕世良賀金主生辰。己亥,振臨安府及嚴州饑民。庚子,再詔臨安府為粥食饑民。辛丑,以淳熙元年減半推賞法募民振糶。十二月癸卯朔,以徽、饒二州民流者眾,罷守臣,官出南庫錢三十萬緡,付新浙東提舉常平朱熹振糶。丁未,禁諸州營造。戊申,謚劉安世曰忠定。辛亥,蠲諸路旱傷州軍明年身丁錢物。甲寅,雨雹。以度僧牒募閩、廣民入米。丙辰,詔縣令有能舉荒政者,監司、郡守以名聞。甲子,下朱熹社倉法於諸路。戊辰,金
 遣魏貞吉等來賀明年正旦。以爭執進書儀,帝還內,遣王抃往諭旨。己巳,貞吉奉書入見。



 是月,廣東安撫鞏湘誘潮賊沉師出降,誅之。是歲,江、浙、兩淮、京西、湖北、潼川、夔州等路水旱相繼,發廩蠲租,遣使按視,民有流入江北者,命所在振業之。



 九年春正月甲戌,詔四孟朝獻分用三日,如在京故事。丁丑,命兩淮戍兵歲一更。癸未,罷樞密都承旨王抃為在外宮觀,因罷諸軍承受,復密院文書關錄兩省舊法,
 以文臣為都承旨。戊子,糴廣南米赴行在。庚寅,詔江、浙、兩淮旱傷州縣貸民稻種,計度不足者貸以樁積錢。



 二月庚戌,遣使訪問二廣鹽法利害。戊辰,四川制置司言獲敘州賊大波浪。三月辛未朔,幸祐聖觀。詔振濟忠、萬、恭、涪四州。癸未,振濟鎮江。壬辰,遣使按視淮南、江、浙振濟。甲午,罷諸路寄招軍兵三年,就揀軍子弟補其闕。



 夏四月甲辰,詔自今盜發所在,親臨帥守、監司論罰,平定有勞者議賞。乙卯,詔諸路提刑,文武臣通置一員。癸亥,
 帝覽陸贄《奏議》,諭講讀官曰:「今日之政,恐有如德宗之弊者,卿等條陳來上,無有所隱。」五月癸酉,以孫手丙為右千牛衛大將軍。丙子,詔輔臣擇監司、郡守,必先才行。



 六月壬寅,詔侍從、臺諫各舉操修端亮、風力強明、可充監司者一二人。甲寅,蠲犒賞庫酒課二十二萬餘緡。汀、漳二州民為沉師蹂踐者,除其賦。丁巳,給臨安府貧民棺瘞錢。戊午,謝廓然薨。庚申,太白晝見。臨安府蝗,詔守臣亟加焚瘞。甲子,太白晝見,經天。



 秋七月甲戌,以江西常
 平、義倉及樁管米四十萬石付諸司,預備振糶。辛己,出南庫錢三十萬緡付浙東提舉朱熹,以備振糶。壬辰,以資政殿學士李彥穎參知政事。詔發所儲和糴米百四十萬石,補淳熙八年振濟之數,於沿江屯駐諸州樁管。



 八月己亥朔,詔紹興民戶去歲已納夏稅應減者三十萬緡,理為今年之數。庚子,減皇后內命婦蔭補碼,立文武臣遇郊奏薦員,限致仕、遣表恩澤,視舊法捐三之一。淮東、浙西蝗。壬子,定諸州官捕蝗之罰。乙卯,復賞修舉
 荒政監司、守臣。



 九月己巳朔,罷諸路科買軍器物料三年。庚午,以王淮為左丞相,梁克家為右丞相。丙子,以子彤為容州觀察使,封安定郡王。辛巳,大享明堂,大赦。乙酉,以錢引十萬緡賜瀘州,備振糶。辛卯,封伯圭為滎陽郡王。以旱減恭、合、渠、昌州今年酒課。癸巳,太白晝見。乙未,禁蕃舶販易金銀,著為令。



 十月戊戌朔,遣王藺等使金賀正旦。丙午,罷軍器所招軍。辛亥,塞四川沿邊支徑。戊午。金遣完顏宗回等來賀會慶節。甲子,蠲諸路旱傷
 州軍淳熙七年八年逋賦,出縣官緡錢以償戶部。



 十一月戊辰朔,禁臣庶之家婦飾僭擬。庚午,振夔路饑。乙酉,進奏院火。丙戌,遣賈選等賀金主生辰。戊子,大風。十二月己亥,更二廣官賣鹽法,復行客鈔,仍出緡錢四十萬以備漕計之闕。癸亥,金遣孛術魯正等來賀明年正旦。



 十年春正月丁丑,以給事中施師點簽書樞密院事。命州縣掘蝗。甲申,李彥穎罷。乙酉,命二廣提舉鹽事官互措置鹽事。丙戌,以施師點兼權參知政事。丁亥,詔終身
 任宮觀人毋得奏子。己丑,詔罷廣南官鬻鹽法。壬辰,罷江東、浙西寄招鎮江諸軍三年。



 二月癸卯,提舉德壽宮陳源有罪,竄建寧府,尋移郴州,仍籍其家貲,進納德壽宮。



 三月戊辰,李燾上《續資治通鑒長編》六百八十七卷。辛未,有司請造第七界會子。辛巳,免四川和糴三年。癸未,幸玉津園。戊子,詔四川類試自今十六人取一人。己丑,除詐稱災傷籍產法。癸巳,復銓試舊法,罷試雜文。



 夏四月丙申,再蠲臨安府民丁身錢三年。己亥,命湖南、廣
 西堙塞溪洞徑路。



 五月丙寅,增皇太子宮小學教授一員。甲戌,以潭州飛虎軍隸江陵都統司。戊寅,幸聚景園。辛卯,詔疏襄陽水渠,以渠傍地為屯田,尋詔民間侵耕者就給之。廢舒州宿松監。



 六月戊戌,監察御史陳賈請禁偽學。乙巳,罷昭州歲貢金。己未,詔諸路監司、帥臣歲舉廉吏。庚申,嚴贓吏禁。秋七月乙丑,以不雨,決系囚。丙寅,幸明慶寺禱雨。甲戌,以夏秋旱□,避殿減膳,令侍從、臺諫、兩省、卿監、郎官、館職各陳朝政闕失,分命群臣禱
 雨於天地、宗廟、社稷、山川。左丞相王淮等以旱乞罷,不許。丁丑,詔除災傷州縣淳熙八年欠稅。甲申,雨。己丑,御殿復膳。



 八月戊申,以施師點參知政事兼同知樞密院事。御史中丞黃洽參知政事。庚戌,以史浩為太保、魏國公致仕,庚申,以左藏南庫隸戶部。



 九月乙丑,長溪、寧德縣大水。丙寅,嚴盜販解監法。丁丑,幸祐聖觀。壬午,蠲諸州逋負內藏庫錢六十萬緡。乙酉,遣餘端禮等使金賀正旦。丁亥,禁內郡行鐵錢。



 冬十月乙未,詔兩浙義役從
 民便。壬子,金遣完顏方等來賀會慶節。



 十一月壬戌朔,日有食之。乙丑,降會子,收兩淮銅錢。甲戌,幸龍山大閱,遂幸玉津園。



 閏月壬寅,詔卻安南獻象。丁巳,遣陳居仁等賀金主生辰。十二月丙子,朝德壽宮,行太上皇后慶壽禮,推恩如太上皇故事。丁亥,金遣完顏婆盧火等來賀明年正旦。是歲,福、漳、臺、信、吉州水,京西、金、澧州、南平、荊門、興國、廣德軍、江陵、建康、鎮江、紹興、寧國府旱。



 十一年春正月辛卯朔,雨土。辛丑,安化蠻蒙光漸等犯
 宜州思立砦,廣西兵馬鈐轄沙世堅出兵討之,獲光漸。丙午,詔江東、西路諸監司,義役、差役從民便。甲寅,雨土。



 二月甲申,詔兩淮、京西、湖北萬弩手令在家閱習,每州許歲上材武者一二人,試授以官,如四川義士之制。



 三月辛卯,詔刑部、御史臺每季以仲月錄囚徒。癸巳,命利路三都統吳挺、郭鈞、彭杲密陳出師進取利害,以備金人。復金州管內安撫司。甲午,以上津、潮陽旱,蠲其稅。辛丑,罷秀州御馬院莊,歸其侵地於民。丁未,禁淮民招溫、
 處州戶口。除職田、官田八年逋租。庚戌,詔御試策有及軍民利害者,考官裒類以聞。辛亥,史浩入謝,賜宴於內殿。



 夏四月甲子,以興元義勝軍移戍襄陽。戊辰,賜禮部進士衛涇以下三百九十四人及第、出身。癸未,重班《紹興申明刑統》。



 五月戊子朔,蠲崇德等十六縣小民淳熙十年欠稅十四萬緡。癸卯,命刑部、大理寺議減刺配法。甲寅,出緡錢三十萬犒給四川久戍將士。乙卯,太白晝見。



 六月戊午朔,詔諸道總領舉偏裨可將帥者。庚申,以
 周必大為樞密使。壬戌,詔在內尚書、侍郎、兩省諫議大夫以上、御史中丞、學士、待制,在外守臣、監司,不限科舉年分,各舉賢良方正能直言極諫一人。己卯,詔諸州歲買稻種,備農民之闕。



 秋七月癸卯,蠲減浙東敗闕坊場酒課。癸丑,以浙西、江東水,禁諸州遏糴。甲寅,築黎州要沖城。



 八月庚申,遣章森使金賀正旦。



 九月丁亥,詔諸路添差官自今毋創置。乙巳,詔殿前軍子弟許權收刺一次。甲寅,再減四川酒課六十八萬餘緡。



 冬十月甲子,初
 命舉改官人犯贓者,舉主降二官。乙丑,遣王信等賀金主生辰。庚午,禁諸州增收稅錢。丙子,金遣張大節等來賀會慶節。盱眙軍言得金人牒,以上京地寒,來歲正旦、生辰人使權止一年。壬午,詔諸以忠義立廟者,兩淮漕臣繕治之。



 十一月壬寅,禁福建民私有兵器。癸卯,助廣西諸州歲計十萬緡。甲寅,令峽州歲時存問處士郭雍。十二月丁巳,修湖南府城。己卯,詔戒監司、州縣毋得於常賦外追取於民。是歲,江東、浙西諸州水,福建、廣東、吉、
 贛州、建昌軍、興元府、金、洋、西和州旱。



 十二年春正月己丑,禁交址鹽入省地。壬辰,四川制置使留正遣人誘青羌奴兒結,殺之。戊戌,日中有黑子。戊申,賜任伯雨謚曰忠敏。庚戌,日中復有黑子。



 二月辛酉,雨雹。乙亥,罷諸軍額外制領將佐。庚辰,置黎州防邊義勇。



 三月乙酉,進孫擴為安慶軍節度使,封平陽郡王。辛卯,禁習渤海樂。辛亥,命侍從、臺諫、兩省、總領、管軍官各舉堪都、副統制者一二人。癸丑,除稅場高等累賞法。



 夏
 四月甲子,幸聚景園。戊辰,班《淳熙寬恤詔令》。丙子,諜言故遼大石林牙假道夏人以伐金,密詔吳挺與留正議之。己卯,幸玉津園。



 五月庚寅,地震。辛卯,福州地震。詔帥臣趙汝愚察守令、擇兵官、防盜賊。



 六月乙卯,立淮東強勇軍效用效士法。壬戌,除諸軍逋欠營運錢。丁丑,詔浙東帥臣、監司不以時上諸州臧否,奪一官。戊寅,太白晝見。



 秋七月丁酉,太白晝見,經天。壬寅,詔二廣試攝官如銓試例,取其半。甲辰,以淮西屯田鹵莽,總領、軍帥、漕
 臣、守臣奪官有差。



 八月癸亥,詔太上皇壽八十,令有司議慶壽禮。乙丑,詔戶部、給舍、臺諫詳官民戶役法以聞。



 九月甲申,復二廣監司以下到罷酬賞法。丙戌,詔恤潮州、臺州被水之家。庚寅,遣王信等使金賀正旦。丁丑,詔諸路總領、軍帥、漕臣、守臣歲上屯田所收之數。



 冬十月辛亥,加上太上皇尊號曰光堯壽聖憲天體道性仁誠德經武緯文紹業興統明謨盛烈太上皇帝、太上皇后曰聖壽齊明廣慈備德太上皇后。甲寅,蠲施、黔州經制
 無額錢。命侍從各舉宗室二三人。癸亥,詔諸路臧否以三月終、四川、二廣以五月終來上。



 十一月丁亥,鄂州大火。戊子,雷。壬辰,遣章森等賀金主生辰。辛丑,合祀天地於圜丘,大赦。十二月庚戌朔,帥群臣奉上太上皇、太上皇后冊寶於德壽宮,推恩如紹興三十二年故事。甲子,以知福州趙汝愚為四川制置使。丙子,金遣僕散守忠等來賀明年正旦。



 十三年春正月庚辰朔,率群臣詣德壽宮行慶壽禮。大
 赦,文武臣僚並理三年磨勘,免貧民丁身錢之半為一百一十餘萬緡,內外諸軍犒賜共一百六十萬緡。癸巳,以史浩為太傅,陳俊卿為少師,嗣濮王士歆為少保。庚子,以昭慶軍節度使士峴為開府儀同三司。



 二月甲寅,詔強盜兩次以上,雖為從,論死。庚申,詔舉歸正、添差、任滿人才藝堪從軍者。



 三月丁酉,詔職事官改官,許在歲額八十員之外。合提舉廣南東、西鹽事司為一。甲辰,幸玉津園。



 夏四月辛亥,詔吳挺結約夏人。戊辰,再蠲四川
 和糴軍糧三年。辛未,幸聚景園。



 五月癸未,日中有黑子。甲申,詔非泛補官及七色補官人、非曾任在朝侍從者,品秩雖高,毋得免役。丙申,賜沖晦處士郭雍號曰頤正先生,仍遣官就問雍所欲言,備錄來上。



 秋七月壬辰,詔內外諸軍主帥各舉堪統制者二三人。壬寅,謚胡銓曰忠簡。



 閏月丙午朔,雨雹。戊申,以敷文閣學士留正簽書樞密院事。己酉,施師點乞免兼同知樞密院事,許之。己未,五星皆伏。



 八月乙亥朔,日、月、五星聚於軫。丙子,以故
 相曾懷鬻奏補恩,追落觀文殿大學士。壬午,新築江陵城成。



 九月乙巳,詔偽造會子凡經行用,並處死。是月,遣李獻等使金賀正旦。



 冬十月甲戌朔,福州火。甲午,金遣完顏老等來賀會慶節。



 十一月戊午,詔四川制置司通知馬政,量收水渠民包占荒田租。庚申,遣張叔椿等賀金主生辰。甲子,王淮等上仁宗、英宗玉牒、神宗、哲宗、徽、宗欽宗四朝《國史列傳》、《皇帝會要》。丙寅,梁克家罷為觀文殿大學士、醴泉觀使兼侍讀。辛未,裁定百司吏額。十二
 月丙子,思州田氏獻納所買黔州民省地,詔償其直。辛己,減汀州鹽價歲萬緡。甲午,陳俊卿薨。乙未,振臨安府城內外貧乏老疾之民。戊戌,大理寺獄空。己亥,金遣耶律子元等來賀明年正旦。辛丑,再賜軍士雪寒錢。是歲,利州路饑,江西諸州旱。



 十四年春正月癸亥,出四川樁積米貸濟金、洋州及關外四州饑民。



 二月丁亥,以周必大為右丞相。戊子,以施師點知樞密院事。



 三月甲子,幸玉津園。



 夏四月己卯,置
 籍考諸路上供殿最,以為賞罰。戊子,賜禮部進士王容以下四百三十五人及第、出身。



 五月乙巳,成都火。己酉,遣官措置汀州經界。



 六月戊寅,以久旱,班畫龍祈雨法。甲申,幸太一宮、明慶寺禱雨。丁亥,梁克家薨。庚寅,臨安府火。辛卯,太白晝見。癸巳,王淮等以旱求罷,不許。詔衡州葺炎帝陵廟。己亥,減兩浙路囚罪一等,釋杖以下。



 秋七月辛丑,罷戶部上供殿最。丙午,詔群臣陳時政闕失及當今急務。丁未,以旱,罷汀州經界。己酉,詔監司條上
 州縣弊事、民間疾苦。辛亥,避殿減膳徹樂。癸丑,命檢正都司看詳群臣封事,有可行者以聞。詔省部、漕臣催理已蠲逋欠者,令臺諫覺察。權減秀州經、總制糴本錢半年。丙辰,命臨安府捕蝗,募民輸米振濟。除紹興新科下戶今年和市布帛二萬八千匹。辛酉,江西、湖南饑,給度僧牒,鬻以糴米備振糶。戊辰,雨。命給、舍看詳監司所條弊事。



 八月辛未,賜度牒一百道、米四萬五千石,備振紹興府饑。甲戌,御殿復膳。癸未,以留正參知政事兼同知
 樞密院事。丙戌,復夔路酬賞法。



 九月癸卯,太上皇不豫。乙巳,詣德壽宮問疾。丙午,遣萬鐘等使金賀正旦。己未,詣德壽宮問疾。乙丑,罷增收木渠民田租。丙寅,除官軍私負。



 冬十月辛未,以太上皇不豫,赦。壬申,詣德壽宮問疾。癸酉,分遣群臣禱於天地、宗廟、社稷。甲戌,以太上皇未御常膳,自來日不視朝,宰執奏事內殿。乙亥,詣德壽宮侍疾,太上皇崩於德壽殿,遺誥太上皇后改稱皇太后。奉皇太后旨,以奉國軍承宣使甘忭主管太上皇喪
 事。丙子,以韋璞等為金告哀使。戊寅,以滎陽郡王伯圭為攢宮總護使。翰林學士洪邁言大行皇帝廟號當稱「祖」,詔有司集議以聞。己卯,詔尊皇太后。辛巳,詔曰:「大行太上皇帝奄棄至養,朕當衰服三年,群臣自遵易月之令,可令有司討論儀制以聞。」甲申,用禮官顏師魯等言,大行太上皇帝上繼徽宗正統,廟號稱「宗」。乙酉,百官五上表請帝還內聽政。丙戌,詔俟過小祥,勉從所請。戊子,帝衰絰御素輦還內。以顏師魯等充金國遣留國信使。
 己丑,金遣田彥皋等來賀會慶節,詔免入見,卻其書幣。甲午,詣德壽宮,自是七日皆如之。



 十一月戊戌朔,詣德壽宮,自是朔望皆如之。己亥,大行太上皇帝大祥,自是帝以白布巾袍禦延和殿。詣德壽宮,衰絰而杖如初。詔皇太子惇參決庶務。庚子,皇太子三辭參決庶務,不許。辛丑,詣德壽宮禫祭,百官釋服。甲辰,群臣三上表請御殿聽政,詔俟過祔廟。戊申。遣胡晉臣等賀金主生辰。辛亥,冬至,詣德壽宮。甲寅,西南方有赤氣隨日入。乙卯。雷。
 戊午,詔皇太子參決庶務於議事堂,在內寺監、在外守臣以下,與宰執同除授訖乃奏。己未,詔三日一朝德壽宮。十二月庚午,大理寺獄空。壬午,東北方有赤氣隨日出。癸巳,金遣完顏崇安等來賀明年正旦,見於垂拱殿之東楹素幄,詔禮物毋入殿,付之有司。是歲,兩浙、江西、淮西、福建旱,振之。



 十五年春正月丁酉朔,詣德壽宮幾筵行禮。戊戌,皇太子初決庶務於議事堂。辛丑,復置左、右補闕、拾遺。乙巳,
 詔免諸州軍會慶節進奉二年。詔自今御內殿,令皇太子侍立。庚申,施師點罷。甲子,以黃洽知樞密院事,吏部尚書蕭燧參知政事。



 二月丁亥,金遣蒲察克忠等來吊祭,行禮於德壽殿,次見帝於東楹之素幄。癸巳,遣京鏜等使金報謝。



 三月庚子,王淮等上大行太上皇謚曰聖神武文憲孝皇帝,廟號高宗。乙巳,上高宗謚冊寶於德壽殿,又上懿節皇后改謚憲節冊寶於別廟本室。丁未,右丞相周必大攝太傅,持節導梓宮。癸丑,用洪邁議,以
 呂頤浩、趙鼎、韓世忠、張俊配饗高宗廟庭,吏部侍郎章森乞用張浚、岳飛,秘書少監楊萬里乞用浚,皆不報。丙寅,權攢高宗於永思陵。



 夏四月壬申,帝親行奉迎虞主之禮,自是七虞、八虞、九虞、卒哭、奉辭皆如之。乙亥,詔洪邁、楊萬里並予郡。甲申,用禮官尤袤請,詔群臣再集議配享臣僚。丙戌,祔高宗神主於太廟,詔曰:「朕比下令欲衰絰三年,群臣屢請御殿易服,故以布素視事內殿。雖詔俟過祔廟,勉從所請,然稽諸典禮,心實未安,行之終
 制,乃為近古。宜體至意,勿復有請。」己丑,詔減臨安、紹興府囚罪一等,釋杖以下,民緣攢宮役者蠲其賦。庚寅,用御史冷世光言,罷再議配享。皇太后有旨,車駕一月四詣德壽宮,如舊禮。



 五月己亥,王淮罷。乙巳,帝既用薛叔似言罷王淮,詔諭叔似等曰:「卿等官以拾遺、補闕為名,不任糾劾。今所奏乃類彈擊,甚非設官命名之意,宜思自警。」丁巳,詔修《高宗實錄》。己未,祁門縣大水。壬戌,始御後殿。詔歲出錢五萬六千餘緡,減廣東十二州折納米
 價錢。



 六月丁卯,雨雹。戊辰,罷敕令所。己巳,以伯圭為少傅,帶御器械夏執中為奉國軍節度使。癸酉,以新江西提點刑獄朱熹為兵部郎官,熹以疾未就職。侍郎林慄劾熹慢命,熹乞奉祠。太常博士葉適論慄襲王淮、鄭丙、陳賈之說,為「道學」之目,妄廢正人。詔熹仍赴江西,熹力辭不赴。庚寅,熒惑犯太微。



 秋七月戊戌,上高宗廟樂曰《大勛》,舞曰《大德》。己未,出兵部侍郎林慄。壬戌,恩平郡王璩薨,追封信王。



 八月甲子朔,日有食之。



 九月庚子夜,南
 方有赤黃氣覆大內。辛丑,大饗明堂,以太祖、太宗配,大赦。癸卯,更試補醫官法。己酉,遣鄭僑等使金賀正旦。甲寅,上皇太后宮名慈福。



 冬十月癸未,金遣王克溫等來賀會慶節,見於垂拱殿東楹。甲申,會慶節,詔北使、百官詣東上閣門拜表起居,免入賀。己丑,再罷諸州科買軍器物料三年。



 十一月庚子,建煥章閣,藏高宗御集。遣何澹賀金主生辰。甲辰,詔百官輪對,毋過三奏。十二月丙寅,追復龔茂良資政殿學士。壬午,命朱熹主管西太一
 宮兼崇政殿說書,辭不至。戊子,金遣田彥皋等來賀明年正旦。是歲,江西、湖北、兩淮、建寧府、徽州水。



 十六年春正月癸巳,金主雍殂,孫璟立。甲午,封孫秉為嘉國公。丙申,黃洽罷。己亥,以周必大為左丞相,留正為右丞相,蕭燧兼權知樞密院事,禮部尚書王藺參知政事,刑部尚書葛邲同知樞密院事。乙巳,蕭燧罷。丙午,皇太后移御慈福宮。戊申,以昭慶軍承宣使郭師禹為保大軍節度使。辛亥,罷淮西屯田。是日,帝始諭二府,以旬
 日當內禪,命周必大留身呈詔草。丙辰,罷拘催錢所。復二廣官般官賣鹽法。己未,更德壽宮為重華宮。謚李綱曰忠定。



 二月辛酉朔,日有食之。壬戌,下詔傳位皇太子。是日,皇太子即皇帝位。帝素服駕之重華宮。辛未,上尊號曰至尊壽皇聖帝,皇后曰壽成皇后。紹熙五年五月壬戌,壽皇聖帝不豫。六月戊戌,崩於重華殿,年六十有八。十有丙辰,謚曰哲文神武成孝皇帝,廟號孝宗。十一月乙卯,權攢於永阜陵。十二月甲戌,祔於太廟。慶元
 三年十一月辛丑,加謚紹統同道冠德昭功哲文神武明聖成孝皇帝。



 贊曰:高宗以公天下之心,擇太祖之後而立之,乃得孝宗之賢,聰明英毅,卓然為南渡諸帝之稱首,可謂難矣哉。即位之初,銳志恢復,符離邂逅失利,重違高宗之命,不輕出師,又值金世宗之立,金國平治,無釁可乘,然易表稱書,改臣稱侄,減去歲幣,以定鄰好,金人易宋之心,至是亦寢異於前日矣。故世宗每戒群臣積錢穀,謹邊
 備,必曰:「吾恐宋人之和,終不可恃。」蓋亦忌帝之將有為也。天厭南北之兵,欲休民生,故帝用兵之意弗遂而終焉。然自古人君起自外藩,入繼大統,而能盡宮庭之孝,未有若帝。其間父子怡愉,同享高壽,亦無有及之者。終喪三年,又能卻群臣之請而力行之。宋之廟號,若仁宗之為「仁」,孝宗之為「孝」,其無愧焉,其無愧焉!



\end{pinyinscope}