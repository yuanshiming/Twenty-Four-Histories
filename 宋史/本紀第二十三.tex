\article{本紀第二十三}

\begin{pinyinscope}

 欽宗



 欽宗恭文順德仁孝皇帝,諱桓,徽宗皇帝長子,母曰恭顯皇后王氏。元符三年四月乙酉生於坤寧殿。初名但,封韓國公,明年六月進封京兆郡王。崇寧元年二月甲
 午,更名烜,十一月丁亥,又改今名。大觀二年正月,進封定王。政和元年三月,講學於資善堂。三年正月,加太保。四年二月癸酉,冠於文德殿。五年二月乙巳,立為皇太子,大赦天下。丁巳,謁太廟。詔乘金輅,設鹵簿,如至道、天禧故事,及宮僚參謁並稱臣,皆辭之。六年六月癸未,納妃朱氏。



 宣和七年十二月戊午,除開封牧。庚申,徽宗詔皇太子嗣位,自稱曰道君皇帝,趣太子入禁中,被以御服。泣涕固辭,因得疾。又固辭,不許。辛酉,即皇帝位,御垂
 拱殿見群臣。是日,日有五色暈,挾赤黃珥,重日相蕩摩久之。乃引道君皇帝出居龍德宮,皇后出居擷景園。以少宰李邦彥為龍德宮使,太保、領樞密院事蔡攸、門下侍郎吳敏副之。是時,金人已分道犯境。壬戌,赦大逆、反叛以下罪,進百官秩一等,賞諸軍,立妃朱氏為皇后,以太子詹事耿南仲簽書樞密院事。癸亥,詔太傅燕王、越王入朝不趨,贊拜不名。詔非三省、樞密院所得旨,有司勿行。甲子,斡離不陷信德府,粘罕圍太原。詔京東、淮西、
 浙募兵入衛。太學生陳東等上書,數蔡京、童貫、王黼、梁師成、李彥、朱勉罪,謂之六賊,請誅之。丙寅,上道君皇帝尊號曰教主道君太上皇帝,皇后曰道君太上皇后。詔改元。



 靖康元年春正月丁卯朔,受群臣朝賀,退詣龍德宮,賀道君皇帝。詔中外臣庶實封言得失。金人破相州。戊辰,破浚州。威武軍節度使梁方平師潰,河北、河東路制置副使何灌退保滑州。己巳,灌奔還,金人濟河,詔親征。道
 君皇帝東巡,以領樞密院事蔡攸為行宮使,尚書右丞宇文粹中副之。詔自今除授、黜陟及恩數等事,並參酌祖宗舊制。罷內外官司、局、所一百五處,止留後苑,以奉龍德宮。以門下侍郎吳敏知樞密院事,吏部尚書李梲同知樞密院事。貶太傅致仕王黼為崇信軍節度副使、安置永州。賜翊衛大夫、安德軍承宣使李彥死,並籍其家。放寧遠軍節度使朱勉歸田里。帝欲親征,以李綱為留守,以李梲為副。給事中王寓諫親征,罷之。庚午,道君
 皇帝如亳州,百官多潛遁。宰相欲奉帝出襄、鄧,李綱諫止之。以綱為尚書右丞。辛未,以李綱為親征行營使,侍衛親軍馬軍都指揮使曹曚副之。太宰兼門下侍郎白時中罷。李邦彥為太宰兼門下侍郎,守中書侍郎張邦昌為少宰兼中書侍郎,尚書左丞趙野為門下侍郎,翰林學士承旨王孝迪為中書侍郎,同知樞密院事蔡懋為尚書左丞。壬申,金人渡河,遣使督諸道兵入援。癸酉,詔兩省、樞密院官制一遵元豐故事。金人犯京師,命尚
 書駕部員外郎鄭望之、親衛大夫康州防禦使高世則使其軍。詔從官舉文武臣僚堪充將帥有膽勇者。是夜,金人攻宣澤門,李綱御之,斬獲百餘人,至旦始退。甲戌,金人遣吳孝民來議和,命李梲使金軍。金人又使蕭三寶奴、耶律忠、張願恭來。以吏部尚書唐恪同知樞密院事。乙亥,金人攻通津、景陽等門,李綱督戰,自卯至酉,斬首數千級,何灌戰死。李梲與蕭三寶奴、耶律忠、王汭來索金帛數千萬,且求割太原、中山、河間三鎮,並宰相、親
 王為質,乃退師,丙子,避正殿,減常膳。括借金銀,籍倡優家財。庚辰,命張邦昌副康王構使金軍,詔稱金國加「大」字。辛巳,道君皇帝幸鎮江。以兵部尚書路允迪簽書樞密院事。金人陷陽武,知縣事蔣興祖死之。壬午,大風走石,竟日乃止。封子諶為大寧郡王。甲申,省廉訪使者官,罷鈔旁定貼錢及諸州免行錢,以諸路贍學戶絕田產歸常平司。統制官馬忠以京西募兵至,擊金人於順天門外,敗之。乙酉,路允迪使粘罕軍於河東。平陽府將劉
 嗣初以城叛。丁亥,靖難軍節度使、河北河東路制置使種師道督涇原、秦鳳兵入援,以師道同知樞密院事,為京畿、河北、河東宣撫使,統四方勤王兵及前後軍。庚寅,盜殺王黼於雍丘。癸巳,大霧四塞。乙未,貶少保、淮南節度使梁師成為彰化軍節度副使,行及八角鎮,賜死。



 二月丁酉朔,命都統制姚平仲將兵夜襲金人軍,不克而奔。戊戌,罷李綱以謝金人,廢親征行營司。金人復來議和。庚子,命附馬都尉曹晟使金軍。辛丑,又命資政殿大
 學士宇文虛中、知東上閣門事王球使之,許割三鎮地。太學諸生陳東等及都民數萬人伏闕上書,請復用李綱及種師道,且言李邦彥等疾綱,恐其成功,罷綱正墮金人之計。會邦彥入朝,眾數其罪而罵。吳敏傳宣,眾不退,遂撾登聞鼓,山呼動地。殿帥王宗濋恐生變,奏上勉從之。遣耿南仲號於眾曰:「已得旨宣綱矣。」內侍朱拱之宣綱後期,眾臠而磔之,並殺內侍數十人。乃復綱右丞,充京城防禦使。壬寅,追封範仲淹魏國公,贈司馬光太
 師,張商英太保,除元祐黨籍學術之禁。詔誅士民殺內侍為首者,禁伏闕上書,廢苑囿宮觀可以與民者。金人使王汭來。癸卯,命肅王樞使金軍。以觀文殿學士、大名尹徐處仁為中書侍郎,宇文虛中簽書樞密院事。蔡懋罷。乙巳,宇文虛中、王球復使金軍。康王至自金軍。金人遣韓光裔來告辭,遂退師,京師解嚴。丙午,康王構為太傅、靜江奉寧軍節度使。省明堂班朔布政官。丁未,日有兩珥。戊申,赦天下。詔諭士民,自今庶事並遵用祖宗舊
 制,凡蠹國害民之事,一切寢罷。己酉,罷宰執兼神霄玉清萬壽宮使及殿中監、符寶郎。詔用祖宗故事,擇武臣得軍心者為同知、簽書樞密院,邊將有威望者為三衙。以金人請和,詔官民昔嘗附金而復歸本朝者,各還其鄉國。庚戌,李邦彥罷,以張邦昌為太宰兼門下侍郎,吳敏為少宰兼中書侍郎,李綱知樞密院事,耿南仲為尚書左丞,李梲為尚書右丞。辛亥,詔監察御史言事如祖宗法。宇文粹中罷知江寧府。癸丑,種師道罷為中太一
 宮使。贈右正言陳瓘為右諫議大夫。甲寅,貶太師致仕蔡京為秘書監、分局南京,太師、廣陽郡王童貫為左衛上將軍,太保、領樞密院事蔡攸為太中大夫、提舉亳州明道宮。先是,粘罕遣人來求賂,大臣以勤王兵大集,拘其使人,且結約余睹以圖之。至是,粘罕怒,及攻太原不克,分兵趣京師,過南、北關,權威勝軍李植以城降。乙卯,陷隆德府,知府張確、通判趙伯臻、司錄張彥遹死之。丙辰,有二流星,一出張宿入濁沒,一出北河入軫。己未,詔
 遙郡承宣使有功應除正任者,自今除正任刺史。辛酉,梁方平坐棄河津,伏誅。王孝迪罷。命給事中王云、侍衛親軍馬軍都指揮使曹曚使金國,鎮洮軍節度使、中太乙宮使種師道為河北、河東路宣撫使,保靜軍節度使、殿前副都指揮使姚古為制置使。乙丑,御殿復膳。丙寅,下哀痛之詔於陜西、河東。是月,金人犯澤州之高平,知州高世由往犒之,乃去。



 三月丁卯朔,遣徽猷閣待制宋煥奉表道君皇帝行宮。詔侍從言事。詔非三省、樞密院所
 奉旨,諸司不許奉行。罷川路歲所遣使。募人掩軍民遺骸,遣使分就四郊致祭。戊辰,李梲罷為鴻慶宮使。己巳,張邦昌罷為中太一宮使。徐處仁為太宰兼門下侍郎,唐恪為中書侍郎,翰林學士何□為尚書右丞,御史中丞許翰同知樞密院事。庚午,宇文虛中罷知青州。癸酉,詣景靈東宮行恭謝禮。命趙野為道君皇帝行宮奉迎使。甲戌,恭謝景靈西宮及建隆觀。乙亥,詣陽德觀、凝祥池、中太乙宮、祐神觀、相國寺。丙子,改擷景園為寧德宮。
 錄司馬光後。己卯,燕王俁、越王偲為太師。壬午,詔金人叛盟深入,其元主和議李邦彥、奉使許地李梲、李鄴、鄭望之悉行罷黜。又詔種師道、姚古、種師中往援三鎮,保塞陵寢所在,誓當固守。癸未,遣李綱迎道君皇帝於南京,以徐處仁為禮儀使。殿中侍御史李擢、左司諫李會罷。乙酉,迎道君皇帝於宜春苑,太后入居寧德宮。丙戌,知中山府詹度為資政殿大學士,知太原府張孝純、知河間府陳遘並為資政殿學士,知澤州高世由直龍圖
 閣,賞城守之勞也。丁亥,朝於寧德宮,詔扈從行宮官吏,候還京日優加賞典,除有罪之人迫於公議已行遣外,餘令臺諫勿復用前事糾言。庚寅,肅王樞為太傅。姚古復隆德府。辛卯,復威勝軍。壬辰,太保景王杞、濟王栩為太傅。有流星出紫微垣。甲午,康王構為集慶、建雄軍節度使,尚書戶部侍郎錢蓋為陜西制置使。命陳東初品官,賜同進士出身,辭不拜。籍朱勉家。乙未,詔金歸朝官民未發遣者,止之。丙申,貶蔡京為崇信軍節度副使。是
 春,夏人取天德、雲內、武州及河東八館。



 夏四月戊戌,夏人陷震威城,攝知城事朱昭死之。己亥,迎太上皇帝入都門。壬寅,朝於龍德宮。癸卯,立子諶為皇太子。耿南仲為門下侍郎。乙巳,置《春秋》博士。戊申,置詳議司於尚書省,討論祖宗法。己酉,乾龍節,群臣上壽於紫宸殿。庚戌,趙野罷。壬子,金人使賈霆、冉企弓來。癸丑,封太師、沂國公鄭紳為樂平郡王。貶童貫為昭化軍節度副使、安置郴州。減宰執俸給三之一及支賜之半。詔開經筵。令吏
 部稽考庶官,凡由楊戩、李彥之公田,王黼、朱勉之應奉,童貫西北之師,孟昌齡河防之役,夔蜀、湖南之開疆,關陜、河東之改幣,及近習所引,獻頌可採,特赴殿試之流,所爵賞,悉奪之。甲寅,種師道加太尉、同知樞密院事、河北河東路宣撫使。乙卯,詔自今假日特坐,百司毋得休務。以平涼軍節度使範訥為右金吾衛上將軍。丙辰,詔有告奸人妄言金人復至以恐動居民者,賞之。戊午,進封南康郡王栻為和王,平陽郡王榛為信王。己未,復
 以詩賦取士,禁用《莊》、《老》及王安石《字說》。壬戌,詔親擢臺諫官,宰執勿得薦舉,著為令。追政和以來道官、處士、先生封贈奏補等敕書。甲子,令在京監察御史、在外監司、郡守及路分鈐轄已上,舉曾經邊任或有武勇可以統眾出戰者,人二員。東兵正將占沆與金人戰於交城縣,死之。乙丑,詔三衙並諸路帥司各舉諳練邊事、智勇過人並豪俊奇傑、眾所推服、堪充統制將領者各五名。貶蔡攸節度副使,安置朱勉於循州。



 五月丙寅朔,朝於龍
 德宮,令提舉官日具太上皇帝起居平安以聞。丁卯,詔天下有能以財穀佐軍者,有司以名聞,推恩有差。以少傅、鎮西軍節度餘深為特進、觀文殿大學士。戊辰,罷王安石配享孔子廟庭。庚午,少傅、安武軍節度使錢景臻,鎮安軍節度使、開府儀同三司劉宗元並為左金吾衛上將軍。保信軍節度使劉敷、武成軍節度使劉敏、向德軍節度使張楙、岳陽軍節度使王舜臣、應道軍節度度使朱孝孫、瀘川軍節度使錢忱並為右金吾衛上將軍。是
 日,寒。辛未,申銅禁。詔:無出身待制已上,年及三十而通歷任實及十年者,乃得任子。監察御史餘應求坐言事迎合大臣,罷知衛州。甲戌,曲赦河北路。乙亥,申銷金禁。丁丑,詔以儉約先天下,澄冗汰貪,為民除害,授監司、郡縣奉行所未及者,凡十有六事。姚古將兵至威勝,聞粘罕將至,眾驚潰,河東大振。河北、河東路制置副使種師中與金人戰於榆次,死之。己卯,借外任官職田一年。開府儀同三司高俅卒。辛巳,損太官日進膳。追削高俅官。
 甲申,罷詳議司。己丑,以河東經略安撫使張孝純為檢校少保、武當軍節度使。壬辰,詔天下舉習武藝、兵書者。乙未,詔姚古援太原。



 六月丙申朔,以道君皇帝還朝,御紫宸殿,受群臣朝賀。詔諫官極論闕失。戊戌,令中外舉文武官才堪將帥者。時太原圍急,群臣欲割三鎮地,李綱沮之,乃以李綱代種師道為宣撫使、援太原。辛丑,以資政殿學士劉韐為宣撫副使,陜西制置司都統制解潛為制置副使。太白犯歲星。壬寅,封鄆國公屋為安康
 郡王,韓國公楗為廣平郡王,並開府儀同三司。詔:「今日政令,惟遵奉上皇詔書,修復祖宗故事。群臣庶士亦當講孔、孟之正道,察安石舊說之不當者,羽翼朕志,以濟中興。」癸卯,以侍衛親軍馬軍副都指揮使、鎮西軍承宣使王稟為建武軍節度使,錄堅守太原之功也。甲辰,路允迪罷為醴泉觀使。乙巳,左司諫陳公輔以言事責監合州酒務。壬子,天狗墜地。有聲如雷。癸丑,慮囚。丙辰,太白、熒感、歲、鎮四星合於張。辛酉,罷都水、將作監承受內
 侍官。熙河都統制焦安節坐不法,李綱斬之。壬戌,姚古坐擁兵逗遛,貶為節度副使、安置廣州。彗出紫微垣。



 秋七月乙丑朔,除元符上書邪等之禁。宋昭政和中上書諫攻遼,貶連州;庚午,詔赴都堂。乙亥,安置蔡京於儋州,攸雷州,童貫吉陽軍。己卯,免借河北、河東、陜西路職田。乙酉,詔蔡京子孫二十三人已分竄遠地,遇赦不許量移。是日,京死於潭州。丁亥,令侍從官共議改修宣仁聖烈皇后謗史。辛卯,遣監察御史張澄誅童貫,廣西轉運
 副使李升之誅趙良嗣,並竄其子孫於海南。壬辰,侍御史李光坐言事貶監當。是月,解潛與金人戰於南關,敗績。劉韐自遼州引兵與金人戰,敗績。



 八月甲午朔,錄陳瓘後。丙申,復命種師道以宣撫使巡邊,召李綱還。庚子,詔以彗星,避殿減膳,令從臣具民間疾苦以聞。河東察訪使張灝與金人戰於文水,敗績。辛丑,詔求民之疾苦者十七事,悉除之。丁未,斡離不復攻廣信軍、保州,不克,遂犯真定。戊申,都統制張思正等夜襲金人於文水
 縣,敗之。己酉,復戰,師潰,死者數萬人,思正奔汾州。都統制折可求師潰於子夏山。威勝、隆德、汾、晉、澤、絳民皆渡河南奔,州縣皆空。金人乘勝攻太原。錄張庭堅後。乙卯,遣徽猷閣待制王云、閣門宣贊舍人馬識遠使於金國,秘書著作佐郎劉岑、太常博士李若水分使其軍議和。戊午,許翰罷知亳州。己未,太宰徐處仁罷知東平,少宰吳敏罷知揚州。以唐恪為少宰兼中書侍郎,何□為中書侍郎,禮部尚書陳過庭為尚書右丞,開封尹聶昌同
 知樞密院事,御史中丞李回簽書樞密院事。庚申,遣王云使金軍,許以三鎮賦稅。是月,福州軍亂,殺其知州事柳庭俊。



 九月丙寅,金人陷太原,執安撫使張孝純,副都總管王稟、通判方笈皆死之。辛未,貶吳敏為崇信軍節度副使、安置涪州。移蔡攸於萬安軍,尋與弟翛及朱勉皆賜死。乙亥,詔編修敕令所取靖康以前蔡京所乞御筆手詔,參祖宗法及今所行者,刪修成書。丁丑,禮部尚書王□為尚書左丞。戊寅,有赤氣隨日出。李綱罷知揚
 州。壬午,梟童貫首於都市。癸未,賜布衣尹焞為和靖處士。甲申,日有兩珥、背氣。丙戌,建三京及鄧州為都總管府,分總四道兵。庚寅,以知大名府趙野為北道都總管,知河南府王襄為西道都總管,知鄧州張叔夜為南道都總管,知應天府胡直孺為東道都總管。又罷李綱提舉洞霄宮。辛卯,遣給事中黃鍔由海道使金國議和。是月,夏人陷西安州。



 冬十月癸巳朔,御殿復膳。貶李鋼為保靜軍節度副使、安置建昌軍。丁酉,金人陷真定,都鈐轄
 劉□死之。有流星如杯。戊戌,金人使楊天吉、王汭來。庚子,日有青、赤、黃恩氣。金人陷汾州,知州張克戩、兵馬都監賈但死之。又攻平定軍。辛丑,下哀痛詔,命河北、河東諸路帥臣傳檄所部,得便宜行事。壬寅,天寧節,率群臣詣龍德宮上壽。甲辰,詔用蔡京、王黼、童貫所薦人。丙午,集從官於尚書省,議割三鎮。召種師道還。丁未,以禮部尚書馮澥知樞密院事。己酉,閱炮飛山營。庚戌,以範訥為寧武軍節度使、河北河東路宣撫使。遼故將小□
 胡□錄攻陷麟州建寧砦,知砦楊震死之。壬子,詔太常禮官集議金主尊號。命尚書左丞王□副康王使斡離不軍,□辭。乙卯,雨木冰。丙辰,金人陷平陽府,又陷威勝、隆德、澤州。丁巳,高麗入貢,令明州遞表以進,遣其使還。戊午,貶王□為單州團練副使,命馮澥代行。庚申,日有兩珥及背氣。侍御史胡舜陟請援中山,不省。辛酉,種師道薨。



 十一月丙寅,夏人陷懷德軍,知軍事劉銓、通判杜翊世死之。籍譚稹家。戊辰,康王未至金軍而還。馮澥罷。己巳,
 集百官議三鎮棄守。庚午,詔河北、河東、京畿清野,令流民得占官舍寺觀以居。辛未,有流星如杯。壬申,禁京師民以浮言相動者。癸酉,右諫議大夫範宗尹以首議棄地罷。金人至河外,宣撫副使折彥質領師十二萬拒之。甲戌,師潰。金人濟河,知河陽燕瑛、西京留守王襄棄城遁。乙亥,命刑部尚書王云副康王使斡離不軍。許割三鎮,奉袞冕、車輅,尊其主為皇叔,且上尊號。丙子,金人渡河,折彥質兵盡潰,提刑許高兵潰於洛口。金人來言,欲
 盡得河北地。京師戒嚴。遣資政殿學士馮澥及李若水使粘罕軍。丁丑,何□罷。以尚書左丞陳過庭為中書侍郎,兵部尚書孫傅為尚書右丞。命成忠郎郭京領選六甲正兵所。簽書樞密院事李回以萬騎防河,眾潰而歸。是日,塞京城門。戊寅,進龍德宮婉容韋氏為賢妃,康王構為安國、安武軍節度使。罷清野。辛巳,以知懷州霍安國為徽猷閣待制,通判林淵直徽猷閣,賞守禦之功也。壬午,斡離不使楊天吉、王汭、勃堇撒離梅來。命耿南仲
 使斡離不軍,聶昌使粘罕軍,許畫河為界。康王至磁州,州人殺王云,止王勿行,王復還相州。甲申,以尚書右丞孫傅同知樞密院事,御史中丞曹輔簽書樞密院事。以京兆府路安撫使範致虛為陜西五路宣撫使,令督勤王兵入援。乙酉,斡離不軍至城下。遣蠟書間行出關召兵,又約康王及河北守將來援。多為邏兵所獲。丁亥,大風發屋折木。李回罷。戊子,金人攻通津門,範瓊出兵焚其砦。己丑,南道總管張叔夜將兵勤王,至玉津園,以叔
 夜為延康殿學士。斡離不遣劉晏來。庚寅,幸東壁勞軍。詔三省長官名悉依元豐舊制。領開封府何□為門下侍郎。



 閏月壬辰朔,金人攻善利門,統制姚仲友御之。奇兵作亂,殺使臣,王宗濋斬數十人乃定。唐恪出都,人欲擊之,因求去,罷為中太一宮使。以門下侍郎何□為尚書右僕射兼中書侍郎。劉韐坐棄軍,降五官予祠。癸巳,京師苦寒,用日者言,借土牛迎春。朱伯友坐棄鄭州,降三官罷。西道總管王襄棄西京去。知澤州高世由以城
 降於金。燕瑛欲棄河陽,為亂兵所殺。河東諸郡,或降或破殆盡。都民殺東壁統制官辛亢宗。罷民乘城,代以保甲。粘罕軍至城下。甲午,時雨雪交作,帝被甲登城,以御膳賜士卒,易火飯以進,人皆感激流涕。金人攻通津門,數百人縋城御之,焚其炮架五、鵝車二。驛召李綱為資政殿大學士、領開封府。金人陷懷州,霍安國、林淵及其鈐轄張彭年、都監趙士□、張諶皆死之。乙未,金人入青城,攻朝陽門。馮澥與金人蕭慶、楊真誥來。丙申,帝幸宣
 化門,以障泥乘馬,行泥淖中,民皆感泣。張叔夜數戰有功,帝如安上門召見,拜資政殿學士。金人執胡直孺,又陷拱州。丁酉,赤氣亙天。以馮澥為尚書左丞。戊戌,殿前副都指揮使王宗濋與金人戰於城下,統制官高師旦死之。庚子,以資政殿學士張叔夜簽書樞密院事。金人攻宣化門,姚仲友御之。辛丑,金人攻南壁,殺傷相當。壬寅,詔河北守臣盡起軍民兵,倍道入援。癸卯,金人攻南壁,張叔夜、範瓊分兵襲之,遙見金兵,奔還,自相蹈藉,溺
 隍死者以千數。甲辰,大雨雪。金人陷亳州。遣間使召諸道兵勤王。乙巳,大寒,士卒噤戰不能執兵,有殭僕者。帝在禁中徒跣祈晴。時勤王兵不至,城中兵可用者惟衛士三萬,然亦十失五六。金人攻城急。丙午,雨木冰。丁未,始避正殿。己酉,遣馮澥、曹輔與宗室仲溫、士言布使金軍請和。命康王為天下兵馬大元帥,速領兵入衛。辛亥,金人來議和,要親王出盟。壬子,金人攻通津、宣化門,範瓊以千人出戰,渡河,冰裂,沒者五百餘人,自是士氣益挫。
 甲寅,大風自北起,俄大雨雪,連日夜不止。乙卯,金人復使劉晏來,趣親王、宰相出盟。丙辰,妖人郭京用六甲法,盡令守禦人下城,大啟宣化門出攻金人,兵大敗。京托言下城作法,引餘兵遁去。金兵登城,眾皆披靡。金人焚南熏諸門。姚仲友死於亂兵,宦者黃經國赴火死,統制官何慶言、陳克禮、中書舍人高振力戰,與其家人皆被害。秦元領保甲斬關遁,京城陷。衛士入都亭驛,執劉晏,殺之。丁巳,奉道君皇帝、寧德皇后入居延福宮。命何□
 及濟王栩使金軍。戊午,何□入言,金人邀上皇出郊。帝曰:「上皇驚憂而疾,必欲之出,朕當親往。」自乙卯雪不止,是日霽。夜有白氣出太微,彗星見。庚申,日赤如火,無光。辛酉,帝如青城。



 十二月壬戌朔,帝在青城。蕭慶入居尚書省。是日,康王開大元帥府於相州。癸亥,帝至自青城。甲子,大索金帛。丙寅,遣陳過庭、劉韐使兩河割地。辛未,定京師米價,勸糶以振民。癸酉,斬行門指揮使蔣宣、李福。乙亥,康王如北京。丙子,尚書省火。庚辰,雨雹。癸未,大
 雪,寒。縱民伐紫筠館花木為薪。庚寅,康王如東平。



 二年春正月辛卯朔,命濟王栩、景王杞出賀金軍,金人亦遣使入賀。壬辰,金人趣召康王還。遣聶昌、耿南仲、陳過庭出割兩河地,民堅守不奉詔,凡累月,止得石州。甲午,詔兩河民開門出降。乙未,有大星出建星,西南流入於濁沒。丁酉,雨木冰。己亥,陰曀,風迅發。夜,西北陰雲中有如火光。庚子,金人索金銀急。何□、李若水勸帝親至軍中,從之,以太子監國而行。乙巳,籍梁師成家。丙午,劉
 韐自經於金軍。太學生徐揆上書,乞守門請帝還闕。金人取至軍中,揆抗論,為所殺。至夜,金人劫神衛營。丁未,大霧四塞。金人下含輝門剽掠,焚五岳觀。



 二月辛酉朔,帝在青城,自如金軍,都人出迎賀。丙寅,金人塹南熏門路,人心大恐。已而金人令推立異姓,孫傅方號慟,乞立趙氏,不允。丁卯,金人要上皇如青城。以內侍鄧述所具諸王孫名,盡取入軍中。辛未,金人逼上皇召皇后、皇太子入青城。庚辰,康王如濟州。癸未,觀文殿大學士唐恪
 仰藥自殺。乙酉,金人以括金未足,殺戶部尚書梅執禮、侍郎陳知質、刑部侍郎程振、給事中安扶。



 三月辛卯朔,帝在青城。丁酉,金人立張邦昌為楚帝。庚子,金人來取宗室,開封尹徐秉哲令民結保,毋藏匿。丁巳,金人脅上皇北行。



 夏四月庚申朔,大風吹石折木。金人以帝及皇后、皇太子北歸。凡法駕、鹵簿,皇后以下車輅、鹵簿,冠服、禮器、法物,大樂、教坊樂器,祭器、八寶、九鼎、圭璧,渾天儀、銅人、刻漏,古器、景靈宮供器,太清樓秘閣三館書、天下
 州府圖及官吏、內人、內侍、技藝、工匠、娼優,府庫畜積,為之一空。辛酉,北風大起,苦寒。



 五月庚寅朔,康王即位於南京,遙上尊號曰孝慈淵聖皇帝。紹興三十一年五月辛卯,帝崩問至。七月己丑,上尊謚曰恭文順德仁孝皇帝,廟號欽宗。三十二年閏二月戊寅,祔於太廟。



 贊曰:帝在東宮,不見失德。及其踐阼,聲技音樂一無所好。靖康初政,能正王黼、朱勉等罪而竄殛之,故金人聞帝內禪,將有卷甲北旆之意矣。惜其亂勢已成,不可救
 藥,君臣相視,又不能同力協謀,以濟斯難,惴惴然講和之不暇。卒致父子淪胥,社稷蕪茀。帝至於是,蓋亦巽懦而不知義者歟!享國日淺,而受禍至深,考其所自,真可悼也夫!真可悼也夫!



\end{pinyinscope}