\article{列傳第二十三}

\begin{pinyinscope}

 薛居正子惟吉沉倫子繼宗盧多遜父億宋琪宋雄



 薛居正,字子平,開封浚儀人。父仁謙,周太子賓客。居正少好學,有大志。清泰初,舉進士不第,為《遣愁文》以自解,
 寓意倜儻,識者以為有公輔之量。逾年,登第。



 晉天福中,華帥劉遂凝闢為從事。遂凝兄遂清領邦計,奏署鹽鐵判官。開運初,改度支推官。宰相李崧領鹽鐵,又奏署推官,加大理寺直,遷右拾遺。桑維翰為開封府尹,奏署判官。



 漢乾祐初,史弘肇領侍衛親軍,威權震主,殘忍自恣,無敢忤其意者。其部下吏告民犯鹽禁,法當死。獄將決,居正疑其不實,召詰之,乃吏與民有私憾,因誣之,逮吏鞫之,具伏抵法。弘肇雖怒甚,亦無以屈。周廣順初,遷比
 部員外郎,領三司推官,旋知制誥。周祖征兗州,詔居正從行,以勞加都官郎中。顯德三年,遷左諫議大夫,擢弘文館學士,判館事。六年,使滄州定民租。未幾,以材幹聞於朝,擢刑部侍郎,判吏部銓。



 宋初,遷戶部侍郎。太祖親征李筠及李重進,並留司三司,俄出知許州。建隆三年,入為樞密直學士,權知貢舉。初平湖湘,以居正知朗州。會亡卒數千人聚山澤為盜,監軍使疑城中僧千餘人皆其黨,議欲盡捕誅之。居正以計緩其事,因率眾剪滅
 群寇,擒賊帥汪端,詰之,僧皆不預,賴以全活。



 乾德初,加兵部侍郎。車駕將親征太原,大發民饋運。時河南府饑,逃亡者四萬家,上憂之,命居正馳傳招集,浹旬間民盡復業。以本官參知政事。五年,加吏部侍郎。開寶五年,兼淮南、湖南、嶺南等道都提舉三司水陸發運使,又兼門下侍郎,監修國史;又監修《五代史》,逾年畢,錫以器幣。六年,拜門下侍郎、平章事。八年二月,上謂居正等曰:「年穀方登,庶物豐盛,若非上天垂祐,何以及斯。所宜共思濟
 物,或有闕政,當與振舉,以成朕志。」居正等益修政事,以副上意焉。



 太平興國初,加左僕射、昭文館大學士。從平晉陽還,進位司空。因服丹砂遇毒,方奏事,覺疾作,遽出。至殿門外,飲水升餘,堂吏掖歸中書,已不能言,但指廡間儲水器。左右取水至,不能飲,偃閣中,吐氣如煙焰,輿歸私第卒,六年六月也,年七十。贈太尉、中書令,謚文惠。



 居正氣貌瑰偉,飲酒至數斗不亂。性孝行純,居家儉約。為相任寬簡,不好苛察,士君子以此多之。自參政至為
 相,凡十八年,恩遇始終不替。



 先是,太祖嘗謂居正曰:「自古為君者鮮克正己,為臣者多無遠略,雖居顯位,不能垂名後代,而身陷不義,子孫罹殃,蓋君臣之道有所未盡。吾觀唐太宗受人諫疏,直詆其非而不恥。以朕所見,不若自不為之,使人無異詞。又觀古之人臣多不終始,能保全而享厚福者,由忠正也。」開寶中,居正與沉倫並為相,盧多遜參知政事,九年冬,多遜亦為平章事。及居正卒,而沉倫責授,多遜南流,論者以居正守道蒙福,果
 符太祖之言。



 居正好讀書,為文落筆不能自休。子惟吉集為三十卷上之,賜名《文惠集》。咸平二年,詔以居正配饗太祖廟庭。



 惟吉字世康,居正假子也。居正妻妒悍,無子,婢妾皆不得侍側,故養惟吉,愛之甚篤。少有勇力,形質魁岸,與京師少年追逐,角抵蹴□匊,縱酒不謹。雅好音樂,嘗與伶人游,居正不能知。蔭補右千牛衛備身,歷太子通奉舍人,改西頭供奉官。



 太宗即位,三相子皆越次拔擢,沉倫、盧多
 遜子並為尚書郎,惟吉以不習文,故為右千牛衛大將軍。及居正卒,太宗親臨,居正妻拜於喪所,上存撫數四,因問:「不肖子安在,頗改行否?恐不能負荷先業,奈何!」惟吉伏喪側,竊聞上語,懼赧不敢起。自是盡革故態,謝絕所與游者,居喪有禮。既而多接賢士大夫,頗涉獵書史,時論翕然稱之。上知其改行,令知澶州,改揚州。上表自陳,遷左千牛衛大將軍。丁內艱,卒哭,起復本官,懇求終制,不許。俄詔知河南府,又知鳳翔府。



 淳化五年,秦州溫
 仲舒以伐木為蕃戶攘奪,驅其部落徙居渭北,頗致騷動。詔擇守臣安撫之,乃命惟吉與仲舒對易其任。未幾,遷左領軍衛大將軍。至道二年,移知延州,未行,卒,年四十二。



 惟吉既知非改過,能折節下士,輕財好施,所至有能聲。然御家無法,及其死,家人爭財致訟,妻子辨對於公庭雲。



 沉倫,字順儀,開封太康人。舊名義倫,以與太宗名下字同,止名倫。少習《三禮》於嵩、洛間,以講學自給。漢乾祐中,
 白文珂鎮陜,倫往依之。



 周顯德初,太祖領同州節度,宣徽使昝居潤與倫厚善,薦於太祖,留幕府。太祖繼領滑、許、宋三鎮,皆署從事,掌留後財貨,以廉聞。及受周禪,自宋州觀察推官召為戶部郎中。奉使吳越歸,奏便宜十數事,皆從之。道出揚、泗,屬歲饑,民多死,郡長吏白於倫曰:「郡中軍儲尚百餘萬斛,儻貸於民,至秋復收新粟,如此則公私俱利,非公言不可。」還具以白。朝論沮之曰:「今以軍儲振饑民,若薦饑無征,孰任其咎?」太祖以問,倫曰:「
 國家以廩粟濟民,自當召和氣,致豐稔,豈復有水旱耶?此當決於宸衷。」太祖即命發廩貸民。



 建隆三年,遷給事中。明年春,為陜西轉運使。王師伐蜀,用為隨軍水陸轉運使。先是,王全斌、崔彥進之入成都也,競取民家玉帛子女,倫獨居佛寺飯疏食,有以珍異奇巧物為獻者,倫皆拒之。東歸,篋中所有,才圖書數卷而已。太祖知之,遂貶全斌等,以倫為戶部侍郎、樞密副使。親征太原,領大內都部署、判留司三司事。



 先是,倫第庳陋,處之晏如。時
 權要多冒禁市巨木秦、隴間,以營私宅,及事敗露,皆自啟於上前。倫亦嘗為母市木營佛舍,因奏其事。太祖笑謂曰:「爾非逾矩者。」知其未葺居第,因遣中使按圖督工為治之。倫私告使者,願得制度狹小,使者以聞,上亦不違其志。



 開寶二年,丁母憂,起復視事。六年,拜中書侍郎、平章事、集賢殿大學士兼提舉荊南、劍南水陸發運事。雩祀西洛,以倫留守東京兼大內都部署。俄召赴行在,令預大禮。



 太平興國初,加右僕射兼門下侍郎,監修國
 史。親征太原,復以倫為留守、判開封府事。師還,加左僕射。五年,史官李昉、扈蒙撰《太祖實錄》五十卷,倫為監修以獻,賜襲衣、金帶。六年,加開府儀同三司。是歲疾作,自是多請告。



 盧多遜事將發,倫已上表求致仕。明年多遜敗,以倫與之同列,不能覺察,詔加切責,降授工部尚書。其子都官員外郎繼宗,本由父蔭,不宜更在朝行,可落班簿。時倫病不能興,上表謝。未幾,倫再奉章乞骸骨,復授左僕射致仕。上以倫國初舊臣,遽復繼宗官以慰其
 心。雍熙四年,卒,年七十九。贈侍中。



 倫清介醇謹,車駕每出,多令居守。好釋氏,信因果。嘗盛夏坐室中,恣蚊蚋噆其膚,童子秉箑至,輒叱之,冀以徼福。在相位日,值歲饑,鄉人假粟者皆與之,殆至千斛,歲餘盡焚其券。



 微時娶閻氏,無子,妾田氏生繼宗。及貴,閻以封邑固讓田,倫乃為閻治第太康,田遂為正室,搢紳非之。



 初,有司議謚倫曰恭惠,繼宗上言曰:「亡父始從冠歲,即事儒業,未遑從賊,遽赴賓招,叨遇明時,陟於相位。伏見國朝故相,薛居
 正謚文惠,王溥謚文獻,此雖近制,實為典常。若以臣父起家不由文學,即嘗歷集賢、修史之職,伏請改謚曰『文』。」



 判太常禮儀院趙昂、判考功張洎駁曰:「沉倫逮事兩朝,早升臺弼,有祗畏謹守之美,有矜恤周濟之心。案《謚法》:不懈於位,與夫謹事奉上、執事堅固、執禮御賓、率事以信、接下不驕、能遠恥辱、賢而不伐、尊賢貴讓、愛民長悌、不懈為德、既過能改,數者皆謂之『恭』。又云:慈民好與,與夫柔質慈民、愛民好柔、寬裕不苛、和質受諫,數者皆謂
 之『惠』。由漢以來,皆為美謚。如唐相溫彥博之出納明允,止謚曰『恭』;竇易直之公舉無避,乃謚曰『恭惠』。而沉倫備位臺衡,出於際會,徒能謹飭以自保全,以『恭』配『惠』,厥美居多。又按《謚法》:道德博聞曰『文』,忠信接禮曰『文』,寬不慢、廉不劌曰『文』,堅強不暴曰『文』,敏而好學、不恥下問曰『文』,德美才秀曰『文』,修治班制曰『文』。昔張說之謚文正,楊綰之謚文簡,人不謂然。蓋行義有所未充,雖蒙特賜,誠非至公。若夫大臣子孫,許其為父陳請,則曲臺、考功之司
 為虛器,而彰善癉惡之義微矣。繼宗以其父曾任集賢殿學士及監修國史之職,輒引薛居正、王溥為比,則彼皆奮跡辭場,歷典誥命,以『文』為謚,允合國章。至於集賢、國史,皆宰相兼領之任,非必由文雅而登。其沉倫謚,伏望如故。」從之。



 繼宗字世卿,倫為樞密副使,以蔭補西頭供奉官。倫作相,授水部員外郎,加朝散大夫。遷都官、職方,知浚儀縣,轉屯田郎中,出知單州。代歸,命使京東計度財賦。濮州
 土貢銀,課民織造,不折省稅;鄆州節度配屬縣納藥物,皆為民病。繼宗歸,歷言於上以除其弊。至道末,領淮南轉運使。



 繼宗貴家子,倦於從吏,既因疾,以將作少監致仕。東封歲,求扈從,復授職方郎中。禮畢,改太僕少卿、判吏部南曹,遷光祿少卿、判三司三勾院。



 繼宗善營產業,厚於養生,不飲酒,不嗜音律,而喜接賓客,終日宴集無倦。大中祥符五年,卒,年五十五。前後錄其子惟溫、惟清、惟恭,並為將作監主簿。惟溫後至秘書丞;惟清娶密王女
 宜都縣主,至內殿承制。



 盧多遜,懷州河內人。曾祖得一、祖真啟皆為邑宰。父億,字子元,少篤學,以孝悌聞。舉明經,調補新鄉主簿。秩滿,復試進士,校書郎、集賢校理。晉天福中,遷著作佐郎,出為鄆州觀察支使。節帥杜重威驕蹇黷貨,幕府賄賂公行,唯億清介自持。會景延廣鎮天平,表億掌書記;留守西洛,又表為判官。時國用窘乏,取民財以助軍,河南府計出二十萬緡,延廣欲並緣以圖羨利,增為三十七萬
 緡。億諫曰:「公位兼將相,既富且貴。今國帑空竭,不得已而取資於民,公何忍利之乎?」延廣慚而止。



 漢初,以魏王承訓為開封尹,授億水部員外郎,充推官。時侍衛諸軍驕恣,朝廷姑息之,軍士成美以驢負鹽入都門,閽者不敢執,反擒平民孟柔送侍衛司。柔自誣伏,論當棄市。億察其冤,言於漢祖而釋之。



 周初,為侍御史。漢末兵亂,法書亡失。至是,大理奏重寫律令格式,統類編敕。乃詔億與刑部員外郎曹匪躬、大理正段濤同加議定。舊本以京
 兆府改同五府,開封、大名府改同河南府,長安、萬年改為次赤縣,開封、浚儀、大名、元城改為赤縣。又定東京諸門熏風等為京城門,明德等為皇城門,啟運等為宮城門,升龍等為宮門,崇元等為殿門。廟諱書不成文,凡改點畫及義理之誤字二百一十有四。又以晉、漢及周初事關刑法敕條者,分為二卷,附編敕,自為《大周續編敕》,詔行之。俄以本官知雜事,加左司員外郎,遷主客度支郎中,並兼弘文館直學士。世宗晏駕,為山陵判官,出為
 河南令。



 宋初,遷少尹。億性恬退,聞其子多遜知制誥,即上章求解。乾德二年,以少府監致仕。



 多遜,顯德初,舉進士,解褐秘書郎、集賢校理,遷左拾遺、集賢殿修撰。建隆三年,以本官知制誥,歷祠部員外郎。乾德二年,權知貢舉。三年,加兵部郎中。四年,復權知貢舉。六年,加史館修撰、判館事。



 開寶二年,車駕征太原,以多遜知太原行府事。移幸常山,又命權知鎮州。師還,直學士院。三年春,復知貢舉。四年冬,命為翰林學士。六年,使江南還,因言
 江南衰弱可圖之狀。受詔同修《五代史》,遷中書舍人、參知政事。丁外艱,數日起復視事。會史館修撰扈蒙請復修時政記,詔多遜專其事。金陵平,加吏部侍郎。



 太平興國初,拜中書侍郎、平章事。四年,從平太原還,加兵部尚書。



 多遜博涉經史,聰明強力,文辭敏給,好任數,有謀略,發多奇中。太祖好讀書,每取書史館,多遜預戒吏令白己,知所取書,必通夕閱覽,及太祖問書中事,多遜應答無滯,同列皆伏焉。



 先是,多遜知制誥,與趙普不協,及在翰
 林日,每召對,多攻普之短。未幾,普出鎮河陽。太宗踐祚,普入為少保。數年,普子承宗娶燕國長公主女,承宗適知澤州,受詔歸闕成婚禮。未逾月,多遜白遣歸任,普由是憤怒。



 初,普出鎮河陽,上言自訴云:「外人謂臣輕議皇弟開封尹,皇弟忠孝全德,豈有間然。矧昭憲皇太后大漸之際,臣實預聞顧命。知臣者君,願賜昭鑒。」太祖手封其書,藏於宮中。至是,普復密奏:「臣開國舊臣,為權幸所沮。」因言昭憲顧命及先朝自訴之事。上於宮中訪得普
 前所上表,因感悟,即留承宗京師。未幾,復用普為相,多遜益不自安。普屢諷多遜,令引退,多遜貪固權位,不能決。



 會有以多遜嘗遣堂吏趙白交通秦王廷美事聞,太宗怒,下詔數其不忠之罪,責授守兵部尚書。明日,以多遜屬吏,命翰林學士承旨李昉、學士扈蒙、衛尉卿崔仁冀、膳部郎中知雜事滕中正雜治之。獄具,召文武常參官集議朝堂,太子太師王溥等七十四人奏議曰:「謹案兵部尚書盧多遜,身處宰司,心懷顧望,密遣堂吏,交結
 親王,通達語言,咒咀君父,大逆不道,干紀亂常,上負國恩,下虧臣節,宜膏斧鉞,以正刑章。其盧多遜請依有司所斷,削奪在身官爵,準法誅斬。秦王廷美,亦請同盧多遜處分,其所緣坐,望準律文裁遣。」



 遂下詔曰:「臣之事君,貳則有闢,下之謀上,將而必誅。兵部尚書盧多遜,頃自先朝擢參大政,洎予臨御,俾正臺衡,職在燮調,任當輔弼。深負倚毗,不思補報,而乃包藏奸宄,窺伺君親,指斥乘輿,交結藩邸,大逆不道,非所宜言。爰遣近臣,雜治其
 事,醜跡盡露,具獄已成,有司定刑,外廷集議,僉以梟夷其族,污瀦其宮,用正憲章,以合經義。尚念嘗居重位,久事明廷,特寬盡室之誅,止用投荒之典,實汝有負,非我無恩。其盧多遜在身官爵及三代封贈、妻子官封,並用削奪追毀。一家親屬,並配流崖州,所在馳驛發遣,縱經大赦,不在量移之限。期周已上親屬,並配隸邊遠州郡。部曲奴婢縱之。餘依百官所議。中書吏趙白、秦王府吏閻密、王繼勛、樊德明、趙懷祿、閻懷忠並斬都門外,仍籍
 其家,親屬流配海島。



 閻密初給事廷美左右,太宗即位,補殿直,仍隸秦邸,恣橫不法。王繼勛尤廷美所親信,嘗使求訪聲妓,繼勛因怙勢以取貨賄。德明素與趙白游處,多遜因之傳達機事,以結廷美。又累遣懷祿私召同母弟軍器庫副使趙廷俊與語。懷忠嘗為廷美使詣淮海國王錢俶遺白金、扣器、絹扇等,廷美又嘗遣懷忠繼銀器、錦彩、羊酒詣其妻父潘璘營宴軍校。至是皆伏罪。多遜累世墓在河面,未敗前,一夕震電,盡焚其林木,聞
 者異之。



 多遜至海外,因部送者還,上表稱謝。雍熙二年,卒於流所,年五十二。詔徙其家於容州,未幾,復移置荊南。端拱初,錄其子雍為公安主簿,還其懷州籍沒先塋。雍卒,諸弟皆特敕除州縣官。



 初,億性儉素,自奉甚薄。及多遜貴顯,賜賚優厚,服用漸侈,愀然不樂,謂親友曰:「家世儒素,一旦富貴暴至,吾未知稅駕之所。」後多遜果敗,人服其識。



 咸平五年,又錄雍弟寬為襄州司士參軍。寬弟察,中景德進士,將廷試,特詔授以州掾。大中祥符二
 年,始改簿尉。三年,察奉多遜喪歸葬襄陽,又詔本州賜察錢三十萬。四年,仍錄其孫又元為襄州司士。



 宋琪,字俶寶,幽州薊人。少好學,晉祖割燕地以奉契丹,契丹歲開貢部,琪舉進士中第,署壽安王侍讀,時天福六年也。幽帥趙延壽闢琪為從事,會契丹內侵,隨延壽至京師。延壽子贊領河中節度,漢初改授晉昌軍,皆署琪為記室。周廣順中,贊罷鎮,補觀城令。世宗征淮南,贊自右龍武統軍為排陣使,復闢琪從征。及金陵歸款,以
 贊鎮廬州,表為觀察判官。部有冤獄,琪辨之,免死者三人,特加朝散大夫。贊仕宋,連移壽陽、延安二鎮,皆表為從事。



 乾德四年,召拜左補闕、開封府推官。太宗為府尹,初甚加禮遇,琪與宰相趙普、樞密使李崇矩善,出入門下,遂惡之,乃白太祖出琪知龍州,移閬州。開寶九年,為護國軍節度判官。



 太宗即位,召赴闕。時程羽、賈琰皆自府邸攀附致顯要,抑琪久不得調。太平興國三年,授太子洗馬,召見詰責,琪拜謝,請悔過自新。遷太常丞,出知
 大通監。五年,召歸,將加擢用,為盧多遜所阻,改都官郎中,出知廣州,將行,復以藩邸舊僚留判三司勾院。七年,與三司使王仁贍廷辨事忤旨,責授兵部員外郎,俄通判開封府事,京府置通判自琪始。



 八年春正月,擢拜右諫議大夫、同判三司。三月,改左諫議大夫、參知政事。是秋,上將以工部尚書李昉參預國政,以琪先入,乃遷琪為刑部尚書。十月,趙普出鎮南陽,琪遂與昉同拜平章事。自員外郎歲中四遷至尚書為相。上謂曰:「世之治亂,在
 賞當其功,罰當其罪,即無不治;謂為飾喜怒之具,即無不亂,卿等慎之。」



 九年九月,上幸景龍門外觀水磑,因謂侍臣曰:「此水出於山源,清冷甘美,凡近河水味皆甘,豈非餘潤之所及乎?」琪等對曰:「實由地脈潛通而然,亦猶人之善惡以染習而成也。」其年冬,郊祀禮畢,加門下侍郎、昭文館大學士。



 一日,上謂琪等曰:「在昔帝王多以崇高自處,顏色嚴毅,左右無敢質言者。朕與卿等周旋款曲,商榷時事,蓋欲通上下之情,無有壅蔽。卿等但直道
 而行,無得有所顧避。」琪謝曰:「臣等非才,待罪相府,陛下曲賜溫顏,令盡愚懇,敢不傾竭以副聖意。」會詔廣宮城,宣徽使柴禹錫有別第在表識內,上言願易官邸,上覽奏不悅。禹錫陰結琪,欲因白請盧多遜舊第,上益鄙之。先是,簡州軍事推官王浣引對,上嘉其雋爽,面授朝官。翼日,琪奏浣經學出身,一任幕職,例除七寺丞。上曰:「吾已許之矣,可與東宮官。」琪執不從,擬大理丞告牒進入,上批曰:「可右贊善大夫。」琪勉從命,上滋不悅。



 初,上令琪
 娶馬仁瑀寡妻高繼沖之女,厚加賜與以助採。廣南轉運王延範,高氏之親也,知廣州徐休復密奏其不軌,且言其依附大臣。上因琪與禹錫入對,問延範何如人,琪未知其端,盛言延範強明忠幹,禹錫旁奏與琪同。上意琪交通,不欲暴其狀,因以琪素好詼諧,無大臣體,罷守本官;禹錫授左驍衛大將軍。琪將罷前數日,有異鳥集琪待漏之所,驅之不去,及是罷相,人以為先兆云。



 端拱初,上親耕籍田,以舊相進位吏部尚書。二年,將討幽薊,
 詔群臣各言邊事。琪疏上謂:



 大舉精甲,以事討除,靈旗所指,燕城必降。但徑路所趨,不無險隘,必若取雄、霸路直進,未免更有陽城之圍。蓋界河之北,陂澱坦平,北路行師,非我所便。況軍行不離于輜重,賊來莫測其淺深。欲望回轅,西適山路,令大軍會於易州,循孤山之北,漆水以西,挾山而行,援糧而進,涉涿水,並大房,抵桑乾河,出安祖砦,則東瞰燕城,裁及一舍,此是周德威收燕之路。



 自易水距此二百餘里,並是沿山,村墅連延,溪澗相
 接,採薪汲水,我占上游。東則林麓平岡,非戎馬奔沖之地,內排槍弩步隊,實王師備御之方,而於山上列白幟以望之,戎馬之來,二十里外可悉數也。



 從安祖砦西北有盧師神祠,是桑乾出山之口,東及幽州四十餘里。趙德君作鎮之時,欲遏西沖,曾塹此水。況河次半有崖岸,不可徑度,其平處築城護之,守以偏師,此斷彼之右臂也。仍慮步奚為寇,可分雄勇兵士三五千人,至青白軍以來山中防遏,北是新州、媯川之間,南出易州大路,其
 桑乾河水屬燕城北隅,繞西壁而轉。大軍如至城下,於燕丹陵東北橫堰此水,灌入高梁河,高梁岸狹,桑水必溢。可於駐掞寺東引入郊亭澱,三五日彌漫百餘里,即幽州隔在水南。王師可於州北系浮梁以通北路,賊騎來援,已隔水矣。視此孤壘,浹旬必克。幽州管內洎山後八軍,聞薊門不守,必盡歸降,蓋勢使然也。



 然後國家命重臣以鎮之,敷恩澤以懷之。奚、□部落,當劉仁恭及其男守光之時,皆刺面為義兒,服燕軍指使,人馬疆土少
 劣於契丹,自被脅從役屬以來,常懷骨髓之恨。渤海兵馬土地,盛於奚帳,雖勉事契丹,俱懷殺主破國之怨。其薊門洎山後雲、朔等州,沙陀、吐渾元是割屬,咸非叛黨。此蕃漢諸部之眾,如將來王師討伐,雖臨陣擒獲,必貸其死,命署置存撫,使之懷恩,但以罪契丹為名。如此則蕃部之心,願報私憾,契丹小醜,克日殄平。其奚、□、渤海之國,各選重望親嫡,封冊為王,仍賜分器、鼓旗、軍服、戈甲以優遣之,必竭赤心,永服皇化。



 俟克平之後,宣布守
 臣,令於燕境及山後雲、朔諸州,厚給衣糧料錢,別作禁軍名額,召募三五萬人,教以騎射,隸於本州。此人生長塞垣,諳練戎事,乘機戰鬥,一以當十,兼得奚、□、渤海以為外臣,乃守在四夷也。



 然自阿保機時至於近日,河朔戶口,虜掠極多,並在錦帳。平盧亦邇柳城,遼海編戶數十萬餘,耕墾千里,既殄異類,悉為王民。變其衣冠,被以聲教,願歸者俾復舊貫,懷安者因而撫之,申畫郊圻,列為州縣,則前代所建松漠、饒落等郡,未為開拓之盛也。



 琪本燕人,以故究知蕃部兵馬山川形勢。俄又上奏曰:



 國家將平燕薊,臣敢陳十策:一、契丹種族,二、料賊眾寡,三、賊來布置,四、備邊,五、命將,六、排陣討伐,七、和蕃,八、饋運,九、收幽州,十、滅契丹。



 契丹,蕃部之別種,代居遼澤中,南界潢水,西距邢山,疆土幅員,千里而近。其主自阿保機始強盛,因攻渤海,死於遼陽。妻述律氏生三男:長曰東丹;次曰德光,德光南侵還,死於殺胡林;季曰自在太子。東丹生永康,永康代德光為主,謀起軍南侵,被殺於
 大神澱。德光之子述律代立,號為「睡王」。二年,為永康子明記所篡。明記死,幼主代立。明記妻蕭氏,蕃將守興之女,今幼主,蕭氏所生也。



 晉末,契丹主頭下兵謂之大帳,有皮室兵約三萬,皆精甲也,為爪牙。國母述律氏頭下,謂之屬珊,屬珊有眾二萬,乃阿保機之牙將,當是時半已老矣。南來時,量分借得三五千騎,述律常留餘兵為部族根本。其諸大首領有太子、偉王、永康、南北王、於越、麻答、五押等。於越,謂其國舅也。大者千餘騎,次者數百
 騎,皆私甲也。



 別族則有奚、□,勝兵亦萬餘人,少馬多步。奚,其王名阿保得者,昔年犯闕時,令送劉琋、崔廷勛屯河、洛者也。又有渤海首領大舍利高模翰步騎萬餘人,並愆發左衽,竊為契丹之飾。復有近界尉厥里、室韋、女真、黨項亦被脅屬,每部不過千餘騎。其三部落,吐渾、沙陀,洎幽州管內、雁門已北十餘州軍部落漢兵合二萬餘眾,此是石晉割以賂蕃之地也。蕃漢諸族,其數可見矣。



 每蕃部南侵,其眾不啻十萬。契丹入界之時,步騎車
 帳不從阡陌,東西一概而行。大帳前及東西面,差大首領三人,各率萬騎,支散游奕,百十里外,亦交相偵邏,謂之欄子馬。契丹主吹角為號,眾即頓舍,環繞穹廬,以近及遠。折木梢屈之為弓子鋪,不設槍營塹柵之備。每軍行,聽鼓三伐,不問昏晝,一匝便行。未逢大敵,不乘戰馬,俟近我師,即竟乘之,所以新羈戰蹄有餘力也。且用軍之術,成列而不戰,俟退而乘之,多伏兵斷糧道,冒夜舉火,土風曳柴,饋餉自繼,退敗無恥,散而復聚,寒而益堅,
 此其所長也。中原所長,秋夏霖霪,天時也;山林河津,地利也;槍突劍弩,兵勝也;財豐士眾,力強也。乘時互用,較然可知。



 王師備邊破敵之計,每秋冬時。河朔州軍緣邊砦柵,但專守境,勿輒侵漁,令彼尋戈,其詞無措。或戎馬既肥,長驅入寇,契丹主行,部落萃至,寒雲翳日,朔雪迷空,鞍馬相持,氈褐之利。所宜守陴坐甲,以逸待勞,令騎士並屯於天雄軍、貝磁相州以來,若分在邊城,緩急難於會合;近邊州府,只用步兵,多屯弩手,大者萬卒,小者
 千人,堅壁固守,勿令出戰。彼以全國之兵,此以一郡之眾,雖勇懦之有殊,慮眾寡之不敵也。國家別命大將,總統前軍,以遏侵軼,只於天雄軍、刑洺貝州以來,設掎戎之備。俟其陽春啟候,虜計既窮,新草未生,陳荄已朽,蕃馬無力,疲寇思歸,逼而逐之,必自奔北。



 前軍行陣之法,馬步精卒不過十萬,自招討以下,更命三五人藩侯充都監、副戎、排陣、先鋒等職,臨事分布,所貴有權。追戎之陣,須列前後,其前陣萬五千騎,陣身萬人,是四十指揮,
 左右哨各十指揮,是二十將。每指揮作一隊,自軍主、都虞候、指揮使、押當,每隊用馬突或刃子槍一百餘,並弓劍、骨朵。其陣身解鐙排之,俟與戎相搏之時,無問厚薄,十分作氣,槍突交沖,馳逐往來,後陣更進。彼若乘我深入,陣身之後,更有馬步人五千,分為十頭,以撞竿,鐙弩俱進,為回騎之舍。陣哨不可輕動,蓋防橫騎奔沖,此陣以都監主之,進退賞罰,便可裁決。後陣以馬步軍八萬,招討董之,與前陣不得過三五里,展梢實心,布常山之
 勢,左右排陣分押之。或前陣擊破寇兵,後陣亦禁其馳驟輕進,蓋師正之律也。



 《牧誓》云:「四伐五伐,乃止齊焉。」慎重之戒也。是以開運中晉軍掎戎,未嘗放散,三四年間,雖德光為戎首,多計桀黠,而無勝晉軍之處,蓋並力禦之。厥後以任人不當,為彥澤之所誤。如將來殺獲驅攘之後,聖人務好生之德,設息兵之謀,雖降志難甘,亦和戎為便。魏絳嘗陳五利,奉春僅得中策,歷觀載籍,前王皆然。《易》稱高宗用伐鬼方,《詩》美宣王薄伐蒨狁,是知
 戎狄侵軼,其來尚矣。然則兵為兇器,聖人不得已而用之。若精選使臣,不辱君命,通盟繼好,弭戰息民,此亦策之得也。



 臣每見國朝發兵,未至屯戍之所,已於兩河諸郡調民運糧,遠近騷然,煩費十倍。臣生居邊土,習知其事。況幽州為國北門,押蕃重鎮,養兵數萬,應敵乃其常事。每逢調發,惟作糗糧之備,入蕃旬浹,軍糧自繼,每人給面斗餘,盛之於囊以自隨。征馬每匹給生穀二斗,作口袋,飼秣日以二升為限,旬日之間,人馬俱無饑色。更以
 牙官子弟,戮力津擎裹送,則一月之糧,不煩饋運。俟大軍既至,定議取舍,然後圖轉餉亦未為晚。臣去年有平燕之策,入燕之路具在前奏,願加省覽。



 疏奏,頗採用之。



 淳化二年,詔百官轉對,琪首應詔,建明堂、闢雍之議。五年,李繼遷寇靈武,命侍衛馬軍都指揮使李繼隆為河西兵馬都部署以討之。西川賊帥李順攻劫州縣,以昭宣使王繼恩為劍南西川招安使。琪又上書言邊事曰:



 臣頃任延州節度判官,經涉五年,雖未嘗躬造夷落,然
 常令蕃落將和斷公事,歲無虛月,蕃部之事,熟於聞聽。大約黨項、吐蕃風俗相類,其帳族有生戶、熟戶,接連漢界、入州城者謂之熟戶,居深山僻遠、橫過寇略者謂之生戶。其俗多有世仇,不相來往,遇有戰鬥,則同惡相濟,傳箭相率,其從如流。雖各有鞍甲,而無魁首統攝,並皆散漫山川,居常不以為患。



 黨項界東自河西銀、夏,西至靈、鹽,南距鄜、延,北連豐、會。厥土多荒隙,是前漢呼韓邪所處河南之地,幅員千里。從銀夏至青、白兩池,地惟沙
 磧,俗謂平夏;拓拔,蓋蕃姓也。自鄜、延以北,多土山柏林,謂之南山;野利,蓋羌族之號也。



 從延州入平夏有三路:一、東北自豐林縣葦子驛至延川縣接綏州,入夏州界;一、正北從金明縣入蕃界,至盧關四五百里,方入平夏州南界;一、西北歷萬安鎮經永安城,出洪門至宥州四五百里,是夏州西境。我師如入夏州之境,宜先招致接界熟戶,使為鄉導,其強壯有馬者,令去官軍三五十里踏白先行。緣此三路,土山柏林,溪穀相接,而復隘狹不
 得成列,躡此鄉導,可使步卒多持弓弩槍鋸隨之,以三二千人登山偵邏,俟見坦途寧靜,可傳號勾馬遵路而行,我皆嚴備,保無虞也。



 長興四年,夏州李仁福死,有男彞超擅稱留後。當時詔延州安從進與李彞超換鎮,彞超據夏州,固不奉詔,朝廷命邠州藥彥稠總兵五萬送從進赴任。時頓兵城下,議欲攻取,軍儲不繼,遽命班師。而振旅之時,不能嚴整,失戈棄甲,遂為邊人之利。



 臣又聞黨項號為小蕃,非是勍敵,若得出山布陣,止勞一戰,
 便可蕩除。深入則饋運艱難,窮追則窟穴幽隱,莫若緣邊州鎮,分屯重兵,俟其入界侵漁,方可隨時掩擊,非為養勇,亦足安邊。凡烏合之徒,勢不能久,利於速斗,以騁兵鋒。莫若持重守疆,以挫其銳。彼無城守,眾乏餱糧,威賞不行,部族分散,然後密令覘其保聚之處,預於麟、府、鄜、延、寧、慶、靈、武等州約期會兵,四面齊進,絕其奔走之路,合勢擊之,可以剪除無□類矣。仍先告語諸軍,擊賊所獲生口、資畜,許為己有,彼為利誘,則人百其勇也。



 靈
 武路自通達軍入青岡峽五百里,皆蕃部熟戶。向來使人、商旅經由,並在部族安泊,所求賂遺無幾,謂之「打當」,亦如漢界逆旅之家宿食之直也。此時大軍或須入其境,則鄉導踏白,當如夏州之法。況彼靈州便是吾土,芻粟儲畜,率皆有備。緣路五七程,不煩供饋,止令逐都兵騎,裹糧輕繼,便可足用。諺所謂「磨鐮殺馬」,劫一時之力也,旬浹之餘,固無闕乏矣。



 又臣曾受任西川數年,經歷江山,備見形勢要害。利州最是咽喉之地。西過桔柏江,
 去劍門百里,東南去閬州,水陸二百餘里,西北通白水、清川,是龍州入川大路,鄧艾於此破蜀,至今廟貌存焉。其外三泉、西縣,興、鳳等州,並為要沖,請選有武略重臣鎮守之。



 奏入,上密寫其奏,令繼隆擇利而行。



 至道元年春,大宴於含光殿,上問琪年,對曰:「七十有九。」上因慰撫久之。二年春,拜右僕射,特令月給實奉一百千,又以其衰老,詔許五日一朝。是年九月被病,令其子貽序秉筆,授辭作《多幸老民敘》,大抵謂《洪範》五福,人所難全,而己
 兼有之,實天幸也。又口占遺表數百字而卒。贈司空,謚惠安。起復貽序為右贊善大夫,貽庥為大理評事,貽廣童子出身。貽序上表乞終喪制,從之。天禧初,錄其孫宗諒試秘書郎。



 琪素有文學,頗諧捷。在使府前後三十年,周知人情,尤通吏術。在相位日,百執事有所求請,多面折之,以是取怨於人。



 貽序嘗預修《冊府元龜》,筆札遒勁。未幾,坐事左遷復州副使,起為殿中丞卒。



 宋雄者,亦幽州人。初與琪齊名燕、薊間,謂之「二宋」。



 雄仕契
 丹為應州從事。雍熙三年,王師北伐,雄與其節度副使艾正以城降,授正本州觀察使,以雄為鴻臚少卿同知州事。改光祿少卿,歷知均、唐二州。未幾,護河陰屯兵,以知河渠利害,因命領護汴口,均節水勢,以達轉漕,京師賴之。改太子詹事,復為光祿少卿,遷將作監。所至職務修舉,公私倚任焉。



 雄涉獵文史,善談論,有氣節,士流多推許之。景德元年,卒,年七十六。錄其子可久為太常寺奉禮郎,賦祿終制。



 論曰:自薛居正而下,嘗居相位者凡四人,其始終出處雖不同,然觀於其行事,概可見矣。初,朗州亡卒嘯聚為盜,監軍使疑城中僧千餘人皆與謀,欲盡殺之,居正緩其事,賊禽而僧不與,卒賴以活。沉倫使吳越還,請以揚、泗軍儲百萬餘斛貸饑民,朝論難之。倫曰:「國家以廩粟濟民,自當召和氣,致豐稔,豈復有水旱?」得請乃已。太祖每取書史館,盧多遜預戒吏令白己,知所取,必通夕閱覽,以是答問多中。宋琪始為程羽、賈琰所抑,繼為多遜
 所忌,其後自員外郎歲中四遷至尚書,居相位。即此而觀,則守道蒙福者非幸致,而投荒竄死者非不幸也。宋雄善持論,有氣節,雖與琪齊名,而爵位不侔者,所遇不同焉爾。嗚呼,自昔懷材抱藝,而抑鬱下僚以終其身者多矣,豈特宋雄為然哉!



\end{pinyinscope}