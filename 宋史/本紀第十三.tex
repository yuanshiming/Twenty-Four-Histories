\article{本紀第十三}

\begin{pinyinscope}

 英宗



 英宗體乾應歷隆功盛德憲文肅武睿聖宣孝皇帝,諱曙,濮安懿王允讓第十三子,母曰仙游縣君任氏。明道元年正月三日生於宣平坊第。初,王夢兩龍與日並墮,
 以衣承之。及帝生,赤光滿室,或見黃龍游光中。四歲,仁宗養於內。寶元二年,豫王生,乃歸濮邸。帝天性篤孝,好讀書,不為燕嬉褻慢,服御儉素如儒者。每以朝服見教授,曰:「師也,敢弗為禮?」時吳王宮教授吳充進《宗室六箴》,仁宗付宗正,帝書之屏風以自戒。景祐三年,賜名宗實,授左監門衛率府副率,累遷右羽林軍大將軍、宜州刺史。皇祐二年,為右衛大將軍、岳州團練使。嘉祐中,宰相韓琦等請建儲,仁宗曰:「宗子已有賢知可付者,卿等其勿
 憂。」時帝方服濮王喪。六年十月辛卯,起為秦州防禦使、知宗正寺,帝以終喪辭。奏四上,乃聽。喪終,復授前命,又辭。七年八月,許罷宗正,復為岳州團練使。戊寅,立為皇子。癸未,改今名。帝聞詔稱疾,益堅辭。詔同判大宗正事安國公從古等往喻旨,即臥內起帝以入。甲辰,見清居殿。自是,日再朝,或入侍禁中。九月,遷齊州防禦使、鉅鹿郡公。



 八年,仁宗崩。夏四月壬申朔,皇后傳遺詔,命帝嗣皇帝位。百官入,哭盡哀。韓琦宣遺制。帝御東楹見百官。
 癸酉,大赦,賜百官爵一等,優賞諸軍,如乾興故事。遣王道恭告哀於契丹。帝欲亮陰三年,命韓琦攝塚宰,宰臣不可,乃止。乙亥,帝不豫。遣韓贄等告即位於契丹。丙子,尊皇后曰皇太后。己卯,詔請皇太后同聽政。壬午,皇太后御小殿垂簾,宰臣覆奏事。乙酉,作受命寶。丁亥,以皇子右千牛衛將軍仲金咸為安州觀察使、光國公。熒惑自七年八月庚辰不見,命宰臣祈禳,至是月己丑見於東方。庚子,立京兆郡君高氏為皇后。五月戊午,以富弼為
 樞密使。戊辰,初御延和殿。以疾未平,命宰臣祈福於天地、宗廟、社稷及寺觀,又祈於岳瀆名山。六月辛卯,契丹遣蕭福延等來祭吊。



 秋七月壬子,初御紫宸殿。帝自六月癸酉不御殿,至是始見百官。癸亥,歲星晝見。乙丑,星大小數百西流。戊辰,百官請大行皇帝謚於南郊。八月癸巳,以生日為壽聖節。



 九月辛亥,以光國公仲金咸為忠武軍節度使、同中書門下平章事、淮陽郡王,改名頊。戊午,上仁宗謚冊於福寧殿。



 冬十月甲午,葬仁宗於永昭
 陵。十一月丙午,祔於太廟。大風霾。己酉,減東西二京罪囚一等,免山陵役戶及靈駕所過民租。辛亥,契丹遣蕭素等來賀即位。



 十二月己巳,初御邇英閣,召侍臣講讀經史。乙亥,淮陽郡王頊出閣。是歲,於闐、西南蕃來貢。



 治平元年春正月丁酉朔,改元。戊戌,太白晝見。己亥,壽聖節,百官及契丹使初上壽於紫宸殿。甲寅,賞知唐州趙尚寬修溝堰、增戶口,進一官,賜錢二十萬。



 三月壬寅,命修秦悼王塚,置守護官。戊午,錄囚。辛酉,雨土。



 夏四月
 癸未,放宮女百三十五人。甲午,祈雨於相國天清寺、醴泉觀。賜諸軍錢有差。



 五月己亥,浚二股河。戊申,皇太后還政。庚戌,初日御前後殿。壬子,詔:「皇太后稱聖旨,出入儀衛如章獻太后故事。其有所須,內侍錄聖旨付有司,覆奏即行。」丙辰,上皇太后宮殿名曰慈壽。己未,熒惑犯太微上將。壬戌,以病愈,命宰臣謝天地、宗廟、社稷及宮觀。



 閏月戊辰,輔臣進爵一等。



 六月己亥,以淮陽郡王頊為穎王,祁國公顥為保寧軍節度使、同中書門下平章
 事、東陽郡王,鄠國公頵為左衛上將軍。增宗室教授。丁未,增同知大宗正事一員。辛亥,作睦親、廣親宅。辛酉,太白晝見。壬戌,歲星晝見。



 八月甲辰,錄周世宗後。甲寅,太白入太微垣。乙卯,遣兵部員外郎呂誨等四人充賀契丹太后生辰、正旦使,刑部郎中章岷等四人充賀契丹主生辰、正旦使。丙辰,內侍都知任守忠坐不法,貶保信軍節度副使、蘄州安置。丁巳,以上供米三萬石振宿、亳二州水災戶。



 九月丁卯,復武舉。庚午,詔夏國精擇使人,
 戒勵毋紊彞章。



 冬十月丙申,詔中外近臣、監司舉治行素著可備升擢者二人。



 十一月乙亥,科陜西戶三丁之一,刺以為義勇軍,凡十三萬八千四百六十五人,各賜錢二千。諫官司馬光累上疏諫之,不允。戊寅,復內侍養子令。十二月乙巳,雨土。丙辰,契丹遣耶律烈等來賀壽聖節,蕭禧等來賀明年正旦。是歲,畿內、宋、亳、陳、許、汝、蔡、唐、穎、曹、濮、濟、單、濠、泗、廬、壽、楚、杭、宣、洪、鄂、施、渝州、光化、高郵軍大水,遣使行視,疏治振恤,蠲其賦租。西蕃瞎氈子
 瞎欺米征內附。二年春正月甲戌,振蔡州。



 二月甲辰,大風,晝冥。丁未,錄囚。是月,賜禮部奏名進士、明經諸科及第出身三百六十一人。



 三月己巳,班《明天歷》。



 夏四月戊戌,詔議崇奉濮安懿王典禮。辛丑,詔監司、知州歲薦吏毋徒充數。丙午,奉安仁宗御容於景靈宮。丁未,白氣起西方。



 五月癸亥,詔以綜核名實勵臣下。丙子,詔自今皇子及宗室屬卑者,勿授以檢校師、傅官。乙酉,詔宗室封王者子孫襲爵。



 六月壬辰,錄囚。己酉,詔尚書集三省、御史臺議奉濮安懿王典禮。甲寅,罷尚書省集議,令有司博求典故,務在合經。詔遣官與契丹定疆界。



 秋七月癸亥,富弼罷。丙寅,詔減乘輿服御。丙子,放宮女百八十人。丁丑,太白晝見。己卯,群臣五上尊號,不允。庚辰,張昪罷,以文彥博為樞密使。



 八月庚寅,京師大雨,水。癸巳,賜被水諸軍米,遣官視軍民水死者千五百八十人,賜其家緡錢,葬祭其無主者。乙未,以雨災,詔責躬乞言。初,學士草詔曰:「執政大
 臣,其惕思天變。」帝書其後曰:「雨災專以戒朕不德,可更曰『協德交修』。」己亥,以水災,罷開樂宴。壬子,以工部郎中蔡抗等充賀契丹生辰使,侍御史趙鼎等充賀契丹正旦使。乙卯,減袞冕制度。丙辰,陜西置壯城兵。



 九月壬戌,雨,罷大宴。己巳,以災異風俗策制舉人。壬午,太白犯南斗。乙酉,以久雨,遣使祈於岳瀆名山大川。



 冬十月乙巳,雨木冰。



 十一月庚午,朝饗景靈宮。辛未,饗太廟。壬申,有事南郊,大赦。上皇太后冊。冊皇后。以齊州為興德軍節
 度。辛巳,加恩百官。十二月辛亥,太白晝見。是歲,蔣、波、繡、雲、龍賜等州來貢。



 三年春正月丙辰朔,契丹遣使耶律仲達等來賀正旦。戊午,契丹遣使蕭惟輔等來賀壽聖節。丙寅,幸降聖院,謁神御殿。癸酉,契丹改國號為遼。己卯,溫州火,燒民屋萬四千間,死者五千人。丁丑,皇太后下書中書門下:「封濮安懿王宜如前代故事,王夫人王氏、韓氏、任氏,皇帝可稱親。尊濮安懿王為皇,夫人為後。」詔遵慈訓。以塋為
 園,置守衛吏,即園立廟,俾王子孫主祠事,如皇太后旨。辛巳,詔臣民避濮安懿王諱,以王子宗懿為濮國公。壬午,黜御史呂誨、范純仁、呂大防。



 二月乙酉朔,白虹貫日。



 三月庚申,彗星晨見於室。辛酉,黜諫官傅堯俞、御史趙鼎、趙瞻。戊辰,上親錄囚。庚午,以彗,避正殿,減膳。辛未,以黜呂誨等詔內外。癸酉。以災異責躬,詔轉運使察獄訟、調役利病大者以聞。辛巳,彗晨見於昴,如太白,長丈有五尺。壬午,孛于畢,如月。



 夏四月丙午,詔有司察所部左
 道、淫祀及賊殺善良不奉令者,罪毋赦。



 五月甲子,罷知雜御史、觀察使以上歲舉人。乙丑,彗至張而沒。戊辰,謂宰相曰:「朕欲與公等日論治道,中書常務有定制者,付有司行之。」六月己酉,錄囚。



 秋七月乙丑,進濮王子孫及魯王孫爵一等。



 八月庚子,遣傅卞等賀遼主生辰,張師顏等賀正旦。



 九月壬子朔,日有食之。癸亥,定待制、諫官、朝官少卿郎中遷選歲月補員格。庚辰,禁妃嬪、公主以下薦服親之夫。



 冬十月壬午朔,以仙游縣君任氏墳域
 為園。乙酉,詔兩日一御邇英閣。丁亥,詔禮部三歲一貢舉。甲午,詔宰臣、參知政事舉才行士可試館職者各五人。



 十一月戊午,帝不豫,禱於大慶殿。己未,宰相始奏事。辛酉,降天下囚死罪一等,流以下釋之。十二月乙未,宰相祈於天地、宗廟、社稷。壬寅,立穎王頊為皇太子。癸卯,大赦。賜文武官子為父後者勛一轉。遼遣蕭靖等來賀正旦、壽聖節。是歲,遣使以違約數寇責夏國,諒詐獻方物謝罪。



 四年春正月庚戌朔,群臣上尊號曰體乾膺歷文武聖孝皇帝。降天下囚罪一等,徒以下釋之。大風霾。辛亥,蠲京師逋曲錢。丁巳,帝崩於福寧殿,壽三十六。謚曰憲文肅武宣孝皇帝,廟號英宗。帝自居睦親宅,孝德著聞。濮安懿王薨,以所服玩物分諸子,帝所得悉以與王府舊人既葬而辭去者。宗室有假金帶而以銅帶歸,主吏以告,帝曰:「真吾帶也。」受之。命殿侍鬻犀帶,直錢三十萬,亡之,帝亦不問。初辭皇子,請潭王宮教授周孟陽作奏,孟
 陽有所勸戒,即謝而拜之。奏十餘不允,始就召,戒舍人曰:「謹守吾舍,上有適嗣,吾歸矣。」既為皇子,慎靜恭默,無所猷為,而天下陰知其有聖德。即位,每命近臣,必以官而不以名,大臣從容以為言,帝曰:「朕雖宮中命小臣,亦未嘗以名也。」一日,語神宗曰:「國家舊制,士大夫之子有尚帝女,皆升行以避舅姑之尊,義甚無謂。朕嘗思此,寤寐不平,豈可以富貴之故,屈人倫長幼之序也?可詔有司革之。」會疾不果,神宗述其事焉。



 贊曰:昔人有言,天之所命,人不能違。信哉!英宗以明哲之資,膺繼統之命,執心固讓,若將終身,而卒踐帝位,豈非天命乎?及其臨政,臣下有奏,必問朝廷故事與古治所宜,每有裁決,皆出群臣意表。雖以疾疹不克大有所為,然使百世之下,欽仰高風,詠嘆至德,何其盛也!彼隋晉王廣、唐魏王泰窺覦神器,矯揉奪嫡,遂啟禍原,誠何心哉!誠何心哉!



\end{pinyinscope}