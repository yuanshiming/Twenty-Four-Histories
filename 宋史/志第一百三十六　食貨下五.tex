\article{志第一百三十六 食貨下五}

\begin{pinyinscope}

 鹽下茶上



 其在福建曰福州長清場,歲鬻十萬三百石,以給本路。天聖以來,福漳泉州、興化軍皆鬻鹽,歲視舊額增四萬八千九百八石。



 熙寧十年,有廖恩者起為盜,聚黨掠州
 郡。恩既平,御史中丞鄧閏甫言:「閩越山林險阻,連亙數千里,無賴奸民比他路為多,大抵盜販鹽耳。恩平,遂不為備,安知無躡恩之跡而起者?」乃詔福建路蹇周輔度利害。周輔言:「建、劍、汀州、邵武軍官賣鹽價苦高,漳、泉、福州、興化軍鬻鹽價賤,故盜多販賣於鹽貴之地。異時建州嘗計民產賦錢買鹽,而民憚求有司,徒出錢或不得鹽。今請罷去,頗減建、劍、汀、邵武鹽價,募上戶為鋪戶,官給券,定月所賣,從官場買之,如是則民易得鹽,盜販不
 能規厚利。又稍興復舊倉,選吏增兵。立法,若盜販、知情囊橐之者,不以赦原;三犯,杖、編管鄰州;己編管復犯者,杖、配犯處本城。」皆行之,歲增賣二十三萬餘斤,而鹽官數外售者不預焉。



 元豐二年,提舉鹽事賈青請自諸州改法酌三年之中數立額。又請捕盜官獲私鹽多者,論賞不限常法。三年,青上所部賣鹽官吏歲課,比舊額增羨。詔曰:「周輔承命創法,青相繼奉行,期年有成,課增盜止,東南賴之。」時周輔已擢三司副使,監司已次被賞者
 凡二十人。



 哲宗即位,御史中丞黃履奏福建多以鹽抑民,詔:「去歲先帝已立分遣御史、郎官察舉監司之法,福建遣御史黃降,江西遣御史陳次升按之。」繼又以命吏部郎中張汝賢並察舉周輔所立鹽法。降言:「福州緣王氏之舊,每產錢一當餘州之十,其科納以此為率,餘隨均定,鹽額亦當五倍,而實減半焉。昨王子京奏立產鹽法,失於詳究,遂概以額增,多寡之間,遼遠絕殊,遠民久無以伸。」詔付汝賢。明年,按察司盡以所察事狀聞,於是
 福建轉運副使賈青、王子京皆坐掊克,謫監湖廣鹽酒稅;刑部侍郎蹇周輔坐議江西鹽法,掊克誕謾,削職知和州;郟但坐倡議運廣鹽江西,張士澄坐附會推行周輔之法,肆志抑擾,並黜官;閩清縣尹徐壽獨用鹽法初行,能守官不撓,民以故不多受課,言於朝加賞焉。汝賢請定福建產賣鹽額,詔從其請;凡抑民為鹽戶及願退不為行者,以徒一年坐之;提舉鹽事官知而不舉,論如其罪。



 已而殿中侍御史呂陶奏:「朝廷以福建、江西、湖南
 等路鹽法之弊,流毒生靈,遣使按視,譴黜聚斂之吏,以慰困窮之民,天下皆知公議之不可廢也。然湖南、江西運賣廣鹽添額之害,京東、河北榷鹽,皆章惇所倡,願付有司根治其罪,使賊民罔上之臣,少知所畏。」監察御史孫升繼言:「江西、湖南鹽法之害,兩路之民,殘虐塗炭,甚於兵火,獨提舉官劉誼乃能上言極其利害,誼坐奪官勒停。」詔復誼官,起守韶州。



 崇寧以後,蔡京用事,鹽法屢變。獨福建鹽於政和初斤增錢七,用熙寧法聽商人轉
 廊算請,依六路所算末鹽錢每百千留十之一,輸請鹽處為鹽本錢。



 建炎間,淮、浙之商不通,而閩、廣之鈔法行;未幾,淮、浙之商既通,而閩、廣之鈔法遂罷。舊法,閩之上四州建、劍、汀、邵行官賣鹽法,閩之下四州福、泉、漳、化行產鹽法。



 隨稅輸鹽也。



 官賣之法既革,產鹽之法亦弊,鈔法一行,弊若可革,而民俗又有不便。故當時轉運、提舉司請上四州依上法,下四州且令依舊。及鈔法既罷,歲令漕司認鈔錢二十萬緡輸行在所榷貨務,自後或減或增,
 卒為二十二萬緡。



 二十七年,常平提舉張汝楫復申明鈔法,上以問宰執。陳誠之奏曰:「建、劍山溪之險,細民冒法私販,雖官賣鹽猶不能革;若使民自賣,其能免私販乎?私販既多,鈔額必虧。」上曰:「中間曾用鈔法,未幾復罷。若可行,祖宗已行之矣。大抵法貴從容,不然不可經久。」淳熙五年,詔泰寧、尤溪兩縣計產買鹽之令,更不施行。



 八年,福建市舶陳峴言:「福建自元豐二年轉運使王子京建運鹽之法,不免有侵盜科擾之弊,且天下州縣皆
 行鈔法,獨福建膺運鹽之害。紹興初,趙不已嘗措置鈔法,而終不可行者,蓋漕司則藉鹽綱為增鹽錢,州縣則藉鹽綱以為歲計,官員則有賣鹽食錢、縻費錢,胥吏則有發遣交納常例錢,公私齟齬,無怪乎不可行也。鈔法未成倫序,而綱運遽罷,百姓率無食鹽,故漕運乘此以為不便,請抱引錢而罷鈔法。鈔法罷而綱運興,官價高而私價賤,民多食私鹽而官不售,科抑之弊生矣。」於是詔峴措置。峴請從榷貨務自立五十斤至百斤,分為五
 等,造大小鈔給買,仍預措置賣鈔,先以本錢畀三倉買鹽,以備商旅請買。九年正月,以福建鹽自來運賣,近為鈔法敷擾害民,於是詔福建轉運司,諸州鹽綱依舊官般官賣。三月,詔轉運傅自得、楊由義廉察官賣鹽未便者,措置以聞。



 淳熙十三年,四川安撫制置趙汝愚言:「汀州民貧,而官鹽抑配視他州尤甚,乞以汀州為客鈔。」事下提舉應孟明及汀州守臣議,孟明等言:「上四州軍有去產鹽之地甚邇者,官不賣鹽則私禁不嚴,民食私鹽
 則客鈔不售,既無翻鈔之地則客賣銷折,所以鈔法屢行而屢罷。四川闊遠,猶不可翻鈔,汀州將何所往?故鈔法雖良,不可行於汀州,惟裁減本州並諸縣合輸內錢,而嚴科鹽之禁,庶幾汀民有瘳矣。」復下轉運趙彥操等措置裁減,以歲運二百萬四千斤會之,總減三萬九千三十八緡有奇,又免其分隸諸司,則汀州六邑歲減於民者三萬九千緡有奇,減於官者一萬緡有奇,所補州用又在外。蓋上四州財賦絕少,所恃者官賣鹽耳。



 又瀕
 海諸郡計產輸錢,官給之鹽以供食,其後遂為常賦,而民不復請鹽矣,此又下四州產鹽之弊也。寧宗嘉定六年,臣僚嘗極言之,於是下轉運司,將福之下四州軍凡二十文產以下合輸鹽五斤之家盡免,其折戶產錢僅及二十文者不輸鹽錢。



 寶慶二年,監察御史梁成大言:「福建州縣半系頻州產鹽之地,利權專屬漕臣,乃其職也。鹽產於福州、興化,而運於建、劍、汀、邵,四郡二十二縣之民食焉。福建提舉司主常平茶事而鹽不預,漕司與
 認凈鏹以助用,近來越職營利,多取綱運,分委屬縣。縣邑既為漕司措辦課鹽,今又增提舉司之額,其勢必盡敷於民,殆甚於青苗之害。望將運鹽盡歸漕司,提舉司不得越職,庶幾事權歸一,民瘼少蘇矣。」從之。



 景定元年九月,明堂赦曰:「福建上四州縣倚鹽為課,其間有招趁失時,月解拖欠,其欠在寶祐五年以前者,並與除放,尚敢違法計口科抑者,監司按劾以聞。」三年,臣僚言:「福建上四州山多田少,稅賦不足,州縣上供等錢銀、官吏宗
 子官兵支遣,悉取辦於賣鹽,轉運司雖拘榷鹽綱,實不自賣。近年創例自運鹽兩綱,後或歲運十綱至二十綱,與上四州縣所運歲額相妨,而綱吏搭帶之數不預焉。州縣被其攙奪,發洩不行,上供常賦,無從趁辦,不免敷及民戶,其害有不可勝言者。」有旨:「福建轉運司視自來鹽法,毋致違戾;建寧府、南劍州、汀州、邵武軍依此施行。」



 廣州東筦、靜康等十三場,歲鬻二萬四千餘石,以給本路及西路之昭桂州、江南之南安軍。廉州白石、石康二
 場,歲鬻三萬石,以給本州及容、白、欽、化、蒙、龔、藤、象、宜、柳、邕、潯、貴、賓、梧、橫、南儀、鬱林州。又高、竇、春、雷、融、瓊、崖、儋、萬安州各鬻以給本州,無定額。天聖以後,東、西海場十三皆領於廣州,歲鬻五十一萬三千六百八十六石,以給東、西二路。而瓊、崖諸州,其地荒阻,賣鹽不售,類抑配衙前。前後官此者,或擅增鹽數,煎鹽戶力不給,有破產者。元豐三年,朱初平奏蠲鹽之不售者,又約所賣數定為煎額,以惠遠民。久之,廣西漕司奏民戶逋鹽稅,其縣令
 監官雖已代,並住奉勒催,須足乃罷。而廣東漕臣復奏嶺外依六路法,以逐州管幹官,提點刑獄兼提舉鹽事,考較賞罰如之。瓊、崖等州復請賦鹽於民,斤重視其戶等,而民滋困矣。



 南渡,二廣之鹽皆屬於漕司,量諸州歲用而給之鹽。然廣東俗富,猶可通商;廣西地廣莫而雕瘁,食鹽有限,商賈難行。自東廣而出,乘大水無灘磧,其勢甚易;自西廣而出,水小多灘磧,其勢甚難。建炎末鬻鈔,未幾復止,然官般、客鈔,亦屢有更革;東、西兩漕,屢有
 分合。



 紹興元年三月,南恩州陽江縣土生堿,募民墾之,置灶六十七,產鹽七十萬八千四百斤,收息錢三萬餘緡。十有二月,復置廣西茶鹽司。八年,詔廣西鹽歲以十分為率,二分令欽、廉、雷、化、高五州官賣,餘八分行鈔法。尋又詔廣東鹽九分行鈔法,一分產鹽州縣出賣。廣南去中州絕遠,土曠民貧,賦入不給,故漕司鬻鹽,以其息什四為州用,可以粗給,而民無加賦。昭州歲收買鹽錢三萬六千緡,以七千緡代潯、貴州上供赴經略司買馬,
 餘為州用。及罷官賣,遂科七千緡於民戶,謂之縻費錢焉。九年,罷廣東官賣,行客鈔法,以其錢助鄂兵之費。



 孝宗乾道四年,罷鹽鈔,令廣西漕司自認漕錢二十萬。且廣西之鹽乃漕司出賣,自乾道元年因曾連請並歸廣東,於是度支唐琢言:「廣西鹽引錢欠幾八千萬緡,緣向來二廣鹽事分東西兩司,而西路鹽常為東路所侵,昔廣西自作一司,故鹽不至於虧減;今既罷西司並入東路,則廣東之鹽無復禁止,廣西坐失一路所入。」故有是
 命。既而宰執進蔣芾之奏:「鹽利舊屬漕司,給諸州歲;自賣鈔鹽之後,漕司遂以苗米高價折錢。今朝廷更不降鹽鈔,只今漕司認發歲額,則漕司自獲鹽息,析米招糴之弊皆去矣。」九年,詔廣州復行官般官賣法。



 淳熙三年,詔廣西轉運司歲收官鹽息錢三分撥諸州,七分充漕計,從經略張栻請也。栻去而漕臣趙公浣增鹽直斤百錢為百六十,欽州歲賣鹽千斛而五增之。六年,侍御史江溥以為言,上黜公浣,詔閩、廣賣鹽自
 有舊額定直,自今毋得擅增。



 九年,詔遣浙西撫干胡廷直訪求利害,與帥、漕、提舉詳議以聞。使還,尋以廷直提舉廣東同措置廣西鹽事。十五年,詔曰:「廣南在數千里外,疾痛難於上聞,朕憫之尤切。蓋鹽者,民資以食,向也官利其贏,轉而自鬻,久為民疾。朕為之更令,俾通販而杜官鬻,民固以為利矣;然利於民者官不便焉,必胥動以浮言,且朕知恤民而已,浮言奚恤?矧置監司、守令以為民,朕有美意,弗廣其推,顧撓而壞之,可乎?自今如或有此,必置之法。」
 於是命詹儀之知靜江府,並廣東、西鹽事為一司,其兩路賣鹽,歲以十六萬五千籮為額。儀之等言:「兩路鹽且以十萬籮為額,俟三數年,視其增虧,乃增其額。所有客鈔東西路通貨錢與免,以便商販。」



 十六年,經略應孟明言:「廣中自行鈔法,五六年間,州縣率以鈔抑售於民,其害有甚於官般。」詔孟明、朱晞顏與提舉廣南鹽事王光祖從長措置經久利便,毋致再有科抑之弊。寶慶元年,以廣州安撫司水軍大為興販,罷其統領尹椿、統轄黃
 受,各降一官。



 鬻堿為鹽,向並州永利監,歲鬻十二萬五千餘石,以給本州及忻、代、石、嵐、憲、遼、澤、潞、麟、府州,威勝、岢嵐、火山、平定、寧化、保德軍,許商人販鬻,不得出境。仁宗時,分永利東、西兩鹽,東隸並州,西隸汾州。籍州民之有堿土者為鐺戶;戶歲輸鹽於官,謂之課鹽;餘則官以錢售之,謂之中賣。鹽法亦與海鹽同,歲鬻視舊額減三千四百三十七石。河東唯晉、絳、慈、隰食池鹽,餘皆食永利鹽。其入官,斤為八錢或六錢,出為錢三十六,歲課緡
 錢十八萬九千有奇。



 自咸平以來,聽商人輦鹽過河西麟府州、濁輪砦貿易,官為下其價予之。後積鹽益多,康定初,罷東監鬻鹽三年。皇祐中,又權罷西監鬻鹽,俟鹽少復故。時議者請募商人入芻粟麟府州、火山軍,予券償以鹽,從之。既而芻粟虛估高,券直千錢,為鹽商所抑,才售錢四百有餘,而出官鹽五十斤,蠹耗縣官。或請罷入芻粟,第令入實錢,轉運司議以為非便而止。大抵堿土或厚或薄,薄則利微,鐺戶破產不能足其課。至和初,
 韓琦請戶滿三歲,地利盡得自言,摘他戶代之。明年,又詔鐺戶輸歲課以分數為率,蠲復有差,遇水災,又聽摘他戶代役,百姓便之。河北、陜西亦有鬻堿為鹽者,然其利薄。明道初,嘗詔廢河中府、慶成軍堿場,禁民鬻鹽以侵池鹽之利。



 熙寧八年,三司使章惇言:「兩監舊額歲課二十五萬餘緡,自許商人並邊中糧草,增饒給鈔支鹽,商人得鈔千錢,售價半之,縣官陰有所亡,坐賈獲利不貲。又私鹽不禁,歲課日減,今才十萬四千餘緡,若計糧
 草虛估,官才得實錢五萬餘緡,視舊虧十之八。請如解鹽例,募商人入錢請買,或官自運,鬻於本路,重私販之禁,歲課且大增,並邊市糧草,一用見錢。」詔如所奏,官自運鬻於本路。



 元豐元年,三司戶部副使陳安石言:「永利東、西監鹽,請如慶歷前商人輸錢於麟、府、豐、代、嵐、憲、忻、岢嵐、寧化、保德、火山等州軍,本州軍給券於東、西監請鹽,以除加饒折糴之弊。仍令商人言占戶所賣地,即鹽已運至場務者,商人買之加運費。如是則官鹽價平而
 商販通。」遂行其說,用安石為河東都轉運使。安石請犯西北青白鹽者,以皇祐敕論罪,首從皆編配;又青白入河東,犯者罪至流,所歷官司不察者罪之。四年,安石自言治鹽歲有羨餘,及增收忻州堿地鐺戶、馬城池鹽課,詔安石遷官,賞其屬。



 元祐元年,右司諫蘇轍言:「異時河東除食解鹽,餘仰東、西永利鹽,未嘗闕。元豐三年後,前宰相蔡確、兄礪等始議創增河東忻州馬城池鹽,夾硝味苦,民不願買。乞下轉運司,茍無妨闕,即止勿收。」詔從
 之。



 四年,陳安石坐為河東轉運使附會時論,興置鹽井,害及一路,降知鄭州。先是,熙寧中,議收熙河蕃部包順鹽井,或以為非宜,王安石謂邊將茍自以情得之,何害?議者不能奪焉。



 六年,詔代州賣鹽年額酌以中數,以八十五萬斤為額,部內多少均裁之。紹聖元年,河東復行官賣法。崇寧三年,以河東三路鈔無定估,本路尤賤,害於糴買,罷給三路鈔,止給見錢鈔,他如河北新降鈔法。四年,詔河東永利兩監土鹽仍官收,見緡鬻之,聽商人
 入納算請,定往河東州軍,罷客販東北鹽入河東者。



 鬻井為鹽,曰益、梓、夔、利,凡四路。益州路一監九十八井,歲鬻八萬四千五百二十二石;梓州路二監三百八十五井,十四萬一千七百八十石;夔州路三監二十井,八萬四千八百八十石;利州路一百二十九井,一萬二千二百石:各以給本路。大為監,小為井,監則官掌,井則土民乾鬻,如其數輸課,聽往旁境販賣,唯不得出川峽。初,川峽承舊制,官自鬻鹽。開寶七年,詔斤減十錢,令干鬻者
 有羨利但輸十之九。



 太平興國三年,石拾遺郭泌上言:「劍南諸州官糶鹽,斤為錢七十。鹽井浚深,鬻鹽極苦,樵薪益貴,輦之甚艱,加之風水之虞,或至漂喪;豪民黠吏,相與為奸,賤市於官,貴糶於民,至有斤獲錢數百,官虧歲額,民食貴鹽。望稍增舊價為百五十文,則豪猾無以規利,民有以給食。」從之。有司言:「昌州歲收虛額鹽萬八千五百餘斤,乃開寶中知州李佩掊斂以希課最,廢諸井薪錢,歲額外課部民鬻鹽,民不習其事,甚以為苦,至
 破產不能償其數,多流入他部,而積年之徵不可免。」詔悉除之,其舊額二萬七千六十斤如故。端拱元年七月,西川食鹽不足,許商販階、文州青白鹽、峽路井鹽、永康軍崖鹽,勿收算。



 川峽諸州自李順叛後,增屯兵,乃募人入粟,以鹽償之。景德二年,權三司使丁謂言:「川峽糧儲充足,請以鹽易絲帛。」詔諸州軍食及二年、近溪洞州三年者,從其請。大中祥符元年,詔滬州南井灶戶遇正、至、寒食各給假三日,所收日額,仍與除放。三年,減滬州淯
 井監課鹽三之一。



 仁宗時,成都、梓、夔三路六監與宋初同,而成都增井三十九,歲課減五萬六千五百九十七石;梓州路增井二十八,歲課減十一萬一十九石;利州路井增十四,歲課減四百九十二石三斗有奇;夔州路井增十五,歲課減三千一百八十四石。各以給一路,夔州則並給諸蠻,計所入鹽直,歲輸緡錢五分,銀、綢絹五分。又募人入錢貨諸州,即產鹽厚處取鹽,而施、黔並邊諸州,並募人入米。



 康定元年,淮南提點刑獄郭維言:「川
 峽素不產銀,而募人以銀易鹽,又鹽酒場主者亦以銀折歲課,故販者趨京師及陜西市銀以歸,而官得銀復輦置京師,公私勞費。請聽入銀京師榷貨務或陜西並邊州軍,給券受鹽於川峽,或以折鹽酒歲課,願入錢,二千當銀一兩。」詔行之。既而入銀陜西者少,議鹽百斤加二十斤予之,並募入中鳳翔、永興。會西方用兵,軍食不足,又詔入芻粟並邊,俟有備而止。芻粟虛估高,鹽直賤,商賈利之。西方既無事,猶入中如故。夔州轉運使蔣
 賁以為入中十餘年,虛費夔鹽計直二十餘萬緡,令陜西用池鹽之利,軍儲有備,請如初。詔許之。



 先是,益、利鹽入最薄,故並食大寧監、解池鹽,商賈轉販給之。慶歷中,令商人入錢貨益州以射大寧監鹽者,萬斤增小錢千緡,小錢十當大錢一。販者滋少,蜀中鹽踴貴,斤為小錢二千二百,知益州文彥博以為言,詔皆復故。



 四路鹽課,縣官之所仰給,然井源或發或微,而積課如舊,任事者多務增課為功,往往貽患後人。時方切於除民疾苦,尤以
 遠人為意,有司上言,輒為蠲減。初,鹽課聽以五分折銀、綢、絹,鹽一斤計錢二十至三十,銀一兩、綢絹一匹折錢六百至一千二百,後詔以課利折金帛者從時估。荊湖之歸、峽二州,州二井,歲課二千八百二十石,亦各以給本州。



 熙寧中,蜀鹽私販者眾;禁不能止。欲盡實私井,運解鹽以足之,議未決。神宗以問修起居注沉括,對曰:「私井既容其撲買,則不得無私易,一切實之而運解鹽,使一出於官售,此亦省刑罰籠遺利之一端;然忠、萬、戎、瀘
 間夷界小井尤多,止之實難,若列候加警,恐所得不酬所費。」議遂寢。九年,劉佐入蜀經度茶事,嘗歲運解鹽十萬席。侍御史周尹奏:「成都府路素仰東川產鹽,昨轉運司商度賣陵井場,遂止東鹽及閉卓筒井,失業者眾,言利之臣,復運解鹽,道險續運甚艱;成都鹽踴貴,東川鹽賤,驅民冒法。乞東川鹽仍入成都,勿閉卓筒井,罷官運解鹽。」詔商販仍舊,賣解鹽依客商例,禁抑配於民。未幾,官運解鹽竟罷。



 元祐元年,詔委成都提點刑獄郭概體
 量鹽事。右司監蘇轍劾概觀望阿附,奏不以實,且言:「四川數州賣邛州蒲江井官鹽,斤為錢百二十,近歲堿泉減耗,多雜沙土;而梓、夔路客鹽及民間販小井白鹽,價止七八十,官司遂至抑配,概不念民朝夕食此貴鹽。」詔遂罷概,今黃廉體量以聞。上封事者言:「有司于稅課外,歲令井輸五十緡,謂之官溪錢。」詔付廉悉蠲之。詔自今溪有鹽井輸課利鹽稅外,毋得更增以租。



 崇寧二年,川峽利、洋、興、劍、蓬、閬、巴、綿、漢、興元府等州,並通行東北鹽。
 四年,梓、遂、夔、綿、漢州、大寧監等鹽仍鬻於蜀,惟禁侵解鹽地。



 紹興二年,四川總領趙開初變鹽法,仿大觀法置合同場,收引稅錢,大抵與茶法相類,而嚴密過之。斤輸引錢二十有五,土產稅及增添約九錢四分,所過稅錢七分,住稅一錢有半,引別輸提勘錢六十六,其後又增貼輸等錢。凡四川四千九百餘井,歲產鹽約千餘萬斤,引法初行,百斤為一擔,又許增十斤勿算以優之,其後遞增至四百餘萬緡。二十九年,減西和州賣鹽直之
 半。



 孝宗淳熙六年,四川制置胡元質、總領程價言:「推排四路鹽井二千三百七十五、場四百五,除井一千一百七十四、場一百五十依舊額煎輸;其自陳或糾決增額者井一百二十五、場二十四,並今渲淘舊井亦願入籍者四百七十九;其無鹽之井,即與鏟除,不敷而抱輸者,即與量減;共減錢引四十萬九千八百八十八道,而增收錢引十三萬七千三百四十九道,庶井戶免困重額。」七年,元質又言:「鹽井推排,所以增有餘補不足,有司務
 求贏餘,盈者過取,涸者略減,盡出私心。今後凡遇推排,以增補虧,不得逾已減之數。」十一年,以京西轉運副使江溥言金州帥司置場拘買商鹽,高價科賣,致商旅坐困,民食貴鹽,詔金州依法聽商人從便買賣,不得置場拘催。



 初,趙開之立榷法也,令商人入錢請引,井戶但如額鬻鹽,輸土產稅而已。然堿脈有盈縮,月額有登耗,間以虛鈔付之,而收其算,引法由是大壞。井戶既為商人所要,因增其斤重予之,每擔有增至百六十斤者。又逃
 絕之井,許增額承認,小民利於得井,界增其額,而不能售,其引息土產之輸,無所從出,由是刎縊相尋,公私病之。



 光宗紹熙三年,吏部尚書趙汝愚言:「紹興間趙開所議鹽法,諸井皆不立額,惟禁私賣,而諸州縣鎮皆置合同場,以招商販,其鹽之斤重,遠近皆平準之,使彼此均一而無相傾奪,貴賤以時而為之翕張。今其法盡廢,宜下四川總所視舊法施行。」時楊輔為總計,去虛額,閉廢井,申嚴合同場法,禁斤重之逾格者,而重私販之罰,鹽
 直於是頓昂。輔又請罷利州東路安撫司所置鹽店六,及津渡所收鹽錢,與西路興州鹽店。後總領陳曄又盡除官井所增之額焉。



 五年,戶部言:「潼川府鹽、酒為蜀重害。鹽既收其土產錢給賣官引,又從而征之,矧州縣額外收稅,如買酒錢、到岸錢、榻地錢之類,皆是創增。」於是申禁成都、潼川、利路諸司。寧宗嘉定七年,詔四川鹽井專隸總所,既而宣撫使安丙言防秋藉此以助軍興,乃復奪之。



 茶宋榷茶之制,擇要會之地,曰江陵府,曰真州,曰海州,曰漢陽軍,曰無為軍,曰蘄州之蘄口,為榷貨務六。初,京城、建安、襄復州皆置務,後建安、襄復州務廢,京城務雖存,但會給交鈔往還,而不積茶貨。在淮南則蘄、黃、廬、舒、光、壽六州,官自為場,置吏總之,謂之山場者十三;六州採茶之民皆隸焉,謂之園戶。歲課作茶輸租,餘則官悉市之。其售於官者,皆先受錢而後入茶,謂之本錢;又民歲輸稅願折茶者,謂之折稅茶。總為歲課八百六十五
 萬餘斤,其出鬻皆就本場。在江南則宣、歙、江、池、饒、信、洪、撫、筠、袁十州,廣德、興國、臨江、建昌、南康五軍;兩浙則杭、蘇、明、越、婺、處、溫、臺、湖、常、衢、睦十二州;荊湖則江陵府、潭、澧、鼎、鄂、岳、歸、峽七州、荊門軍;福建則建、劍二州,歲如山場輸租折稅。總為歲課江南千二十七萬餘斤,兩浙百二十七萬九千餘斤,荊湖二百四十七萬餘斤,福建三十九萬三千餘斤,悉送六榷務鬻之。



 茶有二類,曰片茶,曰散茶。片茶蒸造,實卷模中串之,唯建、劍則既蒸而
 研,編竹為格,置焙室中,最為精潔,他處不能造。有龍、鳳、石乳、白乳之類十二等,以充歲貢及邦國之用。其出虔、袁、饒、池、光、歙、潭、岳、辰、澧州、江陵府、興國臨江軍,有仙芝、玉津、先春、綠芽之類二十六等,兩浙及宣、江、鼎州又以上、中、下或第一至第五為號。散茶出淮南、歸州、江南、荊湖,有龍溪、雨前、雨後之類十一等,江、浙、又有以上、中、下或第一至第五為號者。買臘茶斤自二十錢至一百九十錢有十六等,片茶大片自六十五錢至二百五錢有
 五十五等,散茶斤自十六錢至三十八錢五分有五十九等;鬻臘茶斤自四十七錢至四百二十錢有十二等,片茶自十七錢至九百一十七錢有六十五等,散茶自十五錢至一百二十一錢有一百九十等。



 民之欲茶者售於官,給其日用者,謂之食茶,出境則給券。商賈貿易,入錢若金帛京師榷貨務,以射六務、十三場茶,給券隨所射與之,願就東南入錢若金帛者聽,計直於茶如京師。至道末,鬻錢二百八十五萬二千九百餘貫,天禧末,
 增四十五萬餘貫。天下茶皆禁,唯川峽、廣南聽民自買賣,禁其出境。



 凡民茶折稅外,匿不送官及私販鬻者沒入之,計其直論罪。園戶輒毀敗茶樹者,計所出茶論如法。舊茶園荒薄,採造不充其數者,蠲之。當以茶代稅而無茶者,許輸他物。主吏私以官茶貿易,及一貫五百者死。自後定法,務從輕減。太平興國二年,主吏盜官茶販鬻錢三貫以上,黥面送闕下;淳化三年,論直十貫以上,黥面配本州牢城,巡防卒私販茶,依本條加一等論。凡
 結徒持杖販易私茶、遇官司擒捕抵拒者,皆死。太平興國四年,詔鬻偽茶一斤杖一百,二十斤以上棄市。雍熙二年,民造溫桑偽茶,比犯真茶計直十分論二分之罪。淳化五年,有司以侵損官課言加犯私鹽一等,非禁法州縣者,如太平興國詔條論決。



 茶之為利甚博,商賈轉致於西北,利嘗至數倍。雍熙後用兵,切於饋餉,多令商人入芻糧塞下,酌地之遠近而為其直,取市價而厚增之,授以要券,謂之交引,至京師給以緡錢,又移文江、淮、
 荊湖給以茶及顆、末鹽。端拱二年,置折中倉,聽商人輸粟京師,優其直,給茶鹽於江、淮。



 淳化三年,監察御史薛映、秘書丞劉式等請罷諸榷務,令商人就出茶州軍官場算買,既大省輦運,又商人皆得新茶。詔以三司鹽鐵副使雷有終為諸路茶鹽制置使,左司諫張觀與映副之。四年二月,廢沿江八務,大減茶價。詔下,商人頗以江路回遠非便,有司又以損直虧課為言。七月,復置八務,罷制置使、副。至道初,劉式猶固執前議,西京作坊使楊
 允恭言商人市諸州茶,新陳相糅,兩河、陜西諸州,風土各有所宜,非參以多品則少利,罷榷務令就茶山買茶不可行。太宗欲究其利害之說,命宰相召鹽鐵使陳恕等與式、允恭定議,召問商人,皆願如淳化所減之價,不然,即望仍舊。有司職出納,難於減損,皆同允恭之說,式議遂寢。即以允恭為江南、淮南、兩浙發運兼制置茶鹽使。二年,從允恭等請,禁淮南十二州軍鹽,官鬻之,商人先入金帛京師及揚州折博務者,悉償以茶。自是鬻鹽
 得實錢,茶無滯積,歲課增五十萬八千餘貫,允恭等皆被賞。



 初,商人以鹽為急,趨者甚眾,及禁江、淮鹽,又增用茶,如百千又有官耗,增十年場耗,隨所在饒益。其輸邊粟者,持交引詣京師,有坐賈置鋪,隸名榷貨務,懷交引者湊之。若行商,則鋪賈為保任,詣京師榷務給錢,南州給茶;若非行商,則鋪賈自售之,轉鬻與茶賈。及南北和好罷兵,邊儲稍緩,物價差減,而交引虛錢未改。既以茶代鹽,而買茶所入不補其給,交引停積,故商旅所得茶,
 指期於數年之外,京師交引愈賤,至有裁得所入芻粟之實價,官私俱無利。是年,定監買官虧額自一厘以上罰奉、降差遣之制。



 景德二年,命鹽鐵副使林特、崇儀副使李溥等就三司悉索舊制詳定,而召茶商論議,別為新法:其於京師入金銀、綿帛實直錢五十千者,給百貫實茶,若須海州茶者,入見緡五十五千;河北緣邊入金帛、芻粟,如京師之制,而茶增十千,次邊增五千;河東緣邊次邊亦然,而所增有八千、六千之差;陜西緣邊亦如
 之,而增十五千,須海州茶者,納物實直五十二千,次邊所增如河北緣邊之制。其三路近地所入所給,皆如京師。河北次邊、河東緣邊次邊,皆不得射海州茶。茶商所過,當輸算,令記錄,候至京師並輸之。仍約束山場,謹其出納。議奏,三司皆以為便。五月,以溥為淮南制置發運副使,委成其事。行之一年,真宗慮未盡其要,三年,命樞密直學士李浚等比較新舊法利害。時新法方行,商人頗眩惑,特等請罷比較,從之。



 有司上歲課:元年用舊法,
 得五百六十九萬貫,二年用新法,得四百一十萬貫,三年二百八萬貫。特言「所增蓋官本少而有利」,乃實課也,所虧虛錢耳。四年秋,特等皆遷官,仍詔三司行新法,不得輒有改更。大中祥符二年,特、溥等上編成《茶法條貫》並課利總數二十三策。



 自新法之行,舊有交引而未給者,已給而未至京師者,已至而未磨者,悉差定分數,折納入官。大約商人有舊引千貫者,令新法歲入二百千,候五歲則新舊皆給足。官府有茶充公費者,慮其價賤亂
 法,悉改以他物。山場節其出耗,所過商稅嚴其覺舉。諸榷務所受茶,皆均第配給場務,以交引至先後為次。大商刺知精好之處,日夜走僮使繼券詣官,率多先焉。初,禁淮南鹽,小商已困,至是,益不能行。



 六年,申監買官賞罰之式,凡買到入算茶,及租額遞年送榷務交足而有羨餘者,即理為課績,其不入算者,雖多不在此限。大中祥符五年,歲課二百餘萬貫,六年至三百萬貫,七年又增九十萬貫,八年才百六十萬貫。



 是時數年間,有司以
 京師切須錢,商人舊執交引至場務即付物,時或特給程限,逾限未至者,每十分復令別輸二分見緡,謂之貼納。豪商率能及限,小商或不即知,或無貼納,則賤鬻於豪商。有司徒知移用之便,至存一歲之內文移小改至十數者,商人惑之,顧望不進。乃詔刑部尚書馮拯、翰林學士王曾詳定,拯等深以慎重敦信為言,而上封者猶競陳改法之弊。九年,乃命翰林學士李迪、權御史中丞凌策、侍御史知雜呂夷簡與三司同議條制。時以茶多
 不精,給商人罕有饒益,行商利薄,陜西交引愈賤,鬻於市才八千。知秦州曹瑋請於永興、鳳翔、河中府官出錢市之,詔可。迪等以入中緡錢、金帛,舊從商人所有受之,至是請令十分輸緡錢四五,又定加饒貼納之差。然凡有條奏,多令李溥裁酌,溥務執前制,罕所變革。



 天禧二年,太常博士李垂請放行茶貨。左諫議大夫孫奭言:「茶法屢改,商人不便,非示信之道,望復位經久之制。」即詔奭與三司詳定,務從寬簡。未幾,奭出知河陽,事遂止。三
 司言:「陜西入中芻糧,請依河北例,鬥束量增其直,計實錢給鈔,入京以見錢買之,願受茶貨交引,給依實錢數,令榷貨務並依時價納緡錢支茶,不得更用芻糧文鈔貼納茶貨。」詔每八百千,增五千茶與之,餘從其請。時陜西交引益賤,京師裁直五千,有司惜其費茶。五年,出內庫錢五十萬貫,令閣門祗候李德明於京師市而毀之。



 乾興以來,西北兵費不足,募商人入中芻粟如雍熙法給券,以茶償之。後又益以東南緡錢、香藥、犀齒,謂之三
 說;而塞下急於兵食,欲廣儲偫,不愛虛估,入中者以虛錢得實利,人競趨焉。及其法既弊,則虛估日益高,茶日益賤,入實錢金帛日益寡。而入中者非盡行商,多其土人,既不知茶利厚薄,且急於售錢,得券則轉鬻於茶商或京師交引鋪,獲利無幾;茶商及交引鋪或以券取茶,或收蓄貿易,以射厚利。由是虛估之利皆入豪商巨賈,券之滯積,雖二三年茶不足以償而入中者以利薄不趨,邊備日蹙,茶法大壞。初,景德中丁謂為三司使,嘗計
 其得失,以謂邊糴才及五十萬,而東南三百六十餘萬茶利盡歸商賈。當時以為至論,厥後雖屢變法以救之,然不能亡敝。



 天聖元年,命三司使李諮等較茶、鹽、礬稅歲入登耗,更定其法。遂置計置司,以樞密副使張士遜、參知政事呂夷簡、魯宗道總之。首考茶法利害,奏言:「十三場茶歲課緡錢五十萬,天禧五年才及緡錢二十三萬,每券直錢十萬,鬻之售錢五萬五千,總為緡錢實十三萬,除九萬餘緡為本錢,歲才得息錢三萬餘緡,而官
 吏廩給雜費不預,是則虛數多而實利寡,請罷三說,行貼射法。」其法以十三場茶買賣本息並計其數,罷官給本錢,使商人與園戶自相交易,一切定為中估,而官收其息。如鬻舒州羅源場茶,斤售錢五十有六,其本錢二十有五,官不復給,但使商人輸息錢三十有一而已。然必輦茶入官,隨商人所指予之,給券為驗,以防私害,故有貼射之名。若歲課貼射不盡,或無人貼射,則官市之如舊。園戶過期而輸不足者,計所負數如商人入息。舊
 輸茶百斤,益以二十斤至三十五斤,謂之耗茶,亦皆罷之。其入錢以射六務茶者如舊制。



 先是,天禧中,詔京師入錢八萬,給海州、荊南茶;入錢七萬四千有奇,給真州、無為、蘄口、漢陽並十三場茶,皆直十萬,所以饒裕商人;而海州、荊南茶善而易售,商人願得之,故入錢之數厚於他州。其入錢者,聽輸金帛十之六。至是,既更為十三場法,又募入錢六務,而海州、荊南增為八萬六千,真州、無為、蘄口、漢陽增為八萬。商人入芻粟塞下者,隨所
 在實估,度地里遠近,量增其直。以錢一萬為率,遠者增至七百,近者三百,給券至京,一切以緡錢償之,謂之見錢法;願得金帛、若他州錢、或茶鹽、香藥之類者聽。大率使茶與邊糴,各以實錢出納,不得相為輕重,以絕虛估之敝。朝廷皆用其說。



 行之期年,豪商大賈不能為輕重,而論者謂邊糴償以見錢,恐京師府藏不足以繼,爭言其不便。會江、淮計置司言茶有滯積壞敗者,請一切焚棄。朝廷疑變法之弊,下書責計置司,又遣官行視茶積。諮
 等因條上利害,且言:「嘗遣官視陜西、河北,以鎮戎軍、定州為率,鎮戎軍入粟直二萬八千,定州入粟直四萬五千,給茶皆直十萬。以蘄州市茶本錢視鎮戎軍粟直,反亡本錢三之一,得不償失,敝在茶與邊糴相須為用,故更今法。以新舊二法較之,乾興元年用三說法,每券十萬,茶售錢五萬一千至六萬二千,香藥、象齒售錢四萬一千有奇,東南緡錢售錢八萬三千,而京師實入緡錢五十七萬有奇,邊儲芻二百五萬餘圍,粟二百九十八
 萬石。天聖元年用新法,至二年,茶及香藥、東南緡錢每給直十萬,茶入實錢七萬四千有奇至八萬,香藥、象齒入錢七萬二千有奇,東南緡錢入錢十萬五百,而京師實入緡錢增一百四萬有奇,邊儲芻增一千一百六十九萬餘圍,粟增二百一十三萬餘石。舊以虛估給券者,至京師為出錢售之,或折為實錢給茶,貴賤從其市估。其先賤售於茶商者,券錢十萬,使別輸實錢五萬,共給天禧五年茶直十五萬,小商百萬以下免輸錢,每券十
 萬,給茶直七萬至七萬五千;天禧茶盡,則給乾興以後茶,仍增別輸錢五萬者為七萬,並給耗如舊,俟舊券盡而止。如此又省合給茶及香藥、象齒、東南緡錢總直緡錢一百七十一萬。」二府大臣亦言:「所省及增收計為緡錢六百五十餘萬。時邊儲有不足以給一歲者,至是,多者有四年,少者有二年之蓄,而東南茶亦無滯積之弊。其計置司請焚棄者,特累年壞敗不可用者爾。推行新法,功緒已見。蓋積年侵蠹之源一朝閉塞,商賈利於復
 故,欲有以動搖,而論者不察其實,助為游說。願力行之,毋為流言所易。」於是詔有司榜諭商賈以推行不變之意,賜典吏銀絹有差,然論者猶不已。



\end{pinyinscope}