\article{列傳第一百七十八}

\begin{pinyinscope}

 ○宣繒薛
 極陳貴誼曾從龍鄭性之李鳴復鄒應龍餘天錫許應龍林略徐榮叟別之傑劉伯正金淵李性傳陳韡崔福附



 宣繒,慶元府人。嘉泰三年,太學兩優釋褐。歷官以太學博士召試,為秘書省校書郎。升著作佐郎兼權考功郎官、知吉州、福建提點刑獄。遷考功員外郎,又遷秘書少監。時暫兼權侍立修注官、守起居舍人,為起居郎兼權侍左侍郎,編《孝宗寶訓》。試吏部侍郎,權兵部尚書。嘉定十四年,同知樞密院事兼參知政事。明年,拜參知政事。以資政殿學士奉祠。端平三年召赴闕,升大學士、提舉洞霄宮,以觀文殿大學士致仕。卒,贈少師。詔繒嘗預定
 策,以王堯臣故事贈太師,謚忠靖。



 薛極,字會之,常州武進人。以父任調上元主簿。中詞科,為大理評事、通判溫州,知廣德軍。以參知政事樓鑰薦,遷大理正、刑部郎官,司封郎中、權右司郎中,遷右司郎中兼提領雜賣場、寄樁庫,兼敕令所刪修官,中書門下省檢正諸房公事,兼刪修敕令官。拜司農卿兼權兵部侍郎,尋為真。



 嘉定八年,疏奏:「願陛下深思顧諟之難,益懷兢業之念。勿謂帝德罔愆而怠於進修,勿以天災代
 有而應不以實。政綱雖舉,必求益其所未至;德澤雖布,必思及其所未周。誓以今日遇災警懼之心,永為異時暇逸之戒。將見天心昭格,沛然之澤響應於不崇朝之間。」遷權刑部尚書,尋試戶部尚書兼權吏部尚書,遂為真,時暫兼權戶部尚書。十五年,特賜同進士出身,拜端明殿學士、簽書樞密院事。



 紹定元年,拜參知政事兼同知樞密院事。尋知樞密院事兼參知政事,封毗陵郡公。以觀文殿大學士知紹興府兼浙東安撫使。端平元年,
 加少保、和國公,致仕,卒。



 陳貴誼,字正甫,福州福清人。慶元五年進士,授瑞州觀察推官。丁內外艱,服除,調安遠軍節度掌書記,闢差四川制置司書寫機宜文字。中博學宏詞科,授江南東路安撫司機宜文字。遷太社令。改武學諭、國子錄,遷太學博士。



 時議更楮幣法,貴誼轉對言:「人主令行禁止者,以同民之所好惡。楮券之令,乃使奸惡獲逞,道路咨怨,非所以祈天永命、固結人心。」因援熙寧新法為辭。又言:「明
 銳果敢之才,足以集事而失於剽輕;老成寬博之士,足以厚俗而失於循理。孰若舉之以眾,取之以公。」主更幣之法者,乃摘新法等語激怒時相,且謂「貴誼引類植黨」,人為危之。



 遷太常博士。以兄貴謙兼禮部郎官,引嫌,遷將作監丞兼魏惠憲王府小學教授。轉對,謂:「言路雖開,觸犯忌諱者指為好名,切劘時政者指為玩令。利害關於天下,是非公於人心。一人言之未已,或至累十數人言之,則又指為朋黨。是非易位,忠佞不分。」史彌遠益不
 樂,遷秘書郎,出知江陰軍,提舉江西常平。召赴行在,未至,授禮部郎官。



 屬金人大擾淮、蜀,貴誼言:「人才所以立國,今旁蹊曲徑,幸門四闢。言路所以通下情,今壬MF循默,囊括不言。民力已竭,而科斂之外,饋遺以謀進者未已。軍中恥言敗北,則陣亡者不恤;恥言棄潰,則逃竄者復招。」又言:「婉順巽從者,是災疢也,非愛我也,宜屏之外之;矯拂救正者,是藥石也,愛我也,宜用之聽之。」彌遠滋不樂,諷言者論罷,主管崇禧觀。



 起知徽州,召授司封郎
 官兼翰林權直,兼玉牒所檢討。會有事明堂,首引包拯皇祐中乞因肆赦除聚斂掊克之敝,當察州縣府庫致羨之由。仿成周邦饗必及死王事者之子與漢置羽林孤兒,專取從軍死事之後,教以五兵。



 理宗即位,以為宗正少卿兼侍講,兼權直學士院。尋遷起居舍人。寶慶初,詔舉賢能才識之士。貴誼乃言曰:「世以容嘿滯固為賢,以苛刻生事為能,以褊狹趣辦為才,以輕疏嘗試為識。及茲初政,當求忠實正直、奉公愛民、知禮義廉恥而
 不越防範者,以充中外之選。」又言:「成王之初,元臣故老警以《無逸》者,欲其克壽;勉以敬德者,欲其永命;期以豈弟者,欲其受命之長。則可謂愛君切而慮患深矣。」



 遷中書舍人,升兼直學士院。內侍濫受恩賞,輒封還詔書。將郊,貴誼以:「民生實艱,吏員尚眾,徵斂幾於奪取,公費掩為私藏。宜大明黜陟,庶有以見帝於郊。」遷禮部侍郎,仍兼中書舍人、權刑部尚書。升修玉牒官兼侍讀。為禮部尚書兼給事中、端明殿學士、簽書樞密院事。



 紹定六年
 冬,上始親政,進參知政事。上面諭之曰:「頃聞憂國之言,朕所不忘。」兼同知樞密院事。出師汴、洛時,貴誼已移疾,猶上疏力爭。五上章乞歸,轉四官,加邑封,致仕。卒,贈少保、資政殿大學士。



 曾從龍,字君錫,左僕射公亮四世從孫。初名一龍,慶元五年,擢進士第一,始賜今名。授簽書奉國軍節度判官廳公事。遷兵部員外郎、左司郎中、起居舍人兼太子右諭德。



 使金還,轉官。疏言:「州郡累月闕守,而以次官權攝
 者,彼惟其攝事也,自知非久,何暇盡心於民事?獄訟淹延,政令玩弛,舉一郡之事付之胥吏。幸而除授一人,民望其至如渴望飲,足未及境而復以他故罷去矣。且每易一守,供帳借請少不下萬緡。郡帑所入,歲有常數,而頻年將迎,所費不可勝計。然則輕於易置,公私俱受其病。欲望明詔二三大臣,郡守有闕,即時進擬。其有求避憚行者,悉杜絕其請;其繳劾彈拄者,疾速行之。蓋郡計寬則民力裕,利害常相關故也。」又請已振濟者免其後。



 開禧間丐外,知信州。戍卒行掠境內,從龍置於法,索得婦人衣,命梟於市。召權禮部侍郎兼中書舍人兼太子左諭德。繳還張鎡復官詞頭,以鎡抑令侄女竭資財結姻蘇師旦之子故也。尋兼太子諭德,兼同修國史、實錄院同修撰,兼國子祭酒。為吏部侍郎,仍兼職兼太子右庶子,兼給事中,兼直學士院,權刑部尚書。



 嘉定六年秋,陰雨,乞放系囚。進對,言「修德政,蓄人材,飭邊備」。帝善其言。七年,知貢舉。疏奏:「國家以科目網羅天下之英雋,義
 以觀其通經,賦以觀其博古,論以觀其識,策以觀其才。異時謀王斷國,皆繇此其選。比來循習成風,文氣不振,學不務根祇,辭不尚體要,涉獵未精,議論疏陋,綴緝雖繁,氣象萎TT。願下臣此章,風厲中外,澄源正本,莫甚於斯。」詔從之。



 進端明殿學士、簽書樞密院、太子賓客,改參知政事。疾胡矩憸壬,排沮正論,陳其罪。矩嗾言者劾罷,以前職提舉洞霄宮。起知建寧府。丁內艱,服除,為湖南安撫使。撫安峒獠,威惠並行,興學養士,湘人紀之石。
 改知隆興府,復提舉洞霄宮,改萬壽觀兼侍讀,奉朝請。



 端平元年,授資政殿大學士、沿江制置使兼知建康府兼行宮留守。拜參知政事兼同知樞密院事。時有三京之役,極論南兵輕進易退。未幾言驗。進知樞密院事兼參知政事,以樞密院使督視江淮、荊襄軍馬。疏言:「邊面遼遠,聲援不接,請並建二閫。」詔許之,專畀江淮,以荊襄屬魏了翁。朝論邊用不給,詔從龍、了翁並領督府。及從龍卒,贈少師。弟用虎、天麟、治鳳,皆歷顯任。



 鄭性之字信之,初名自誠,後改今名,福州人。嘉定元年,進士第一,歷官知贛州,改知隆興府。後以寶章閣待制提舉玉隆萬壽宮,進華文閣待制、提舉上清太平宮。進敷文閣待制、知建寧府。



 端平元年,召為吏部侍郎。入對,言:「陛下大開言路,以通壅蔽,心茍愛君,誰不欲言,言不切直,何能感動?譬如積水,久雍一決,其勢必盛,其聲必激。故言者多則易於取厭,言之激則難於樂受。若少有厭倦,動於詞色,則讒諂乘間,或不自知矣。」又言:「願陛下
 明詔百闢,滌去舊污,一以清白相師。權之所在,勢所必趨,恐懼戒謹,尤防其微,以保終譽。毋招謗議。則朝綱肅而國體尊矣。」又曰:「為君者不以堯、舜自期,則無善治;告君者不陳堯、舜之道,則無遠猷。」



 擢左諫議大夫,言:「臺臣交章互詆,願陛下監古今天下安危之變,君子小人消長之機,公以處之,乃得其當。況夫聽言之道,宜以事觀,若言果有關國體,有補治道,有益主德,則言之過激,夫亦何傷。彼雖採名,我實有益。惟虛心納善,若決江河,則
 激者自平矣。」



 拜端明殿學士、簽書樞密院事,進同知樞密院事兼權參知政事。尋拜參知政事兼同知樞密院事。尋知樞密院事兼參知政事,加觀文殿學士,致仕。寶祐二年卒。



 李鳴復,字成叔,瀘州人。嘉定二年進士。歷官權發遣金州兼乾辦安撫司公事。制置使鄭損薦於朝,乞召審察。授司農寺丞,遷駕部員外郎,遷兵部郎中。面對,遷軍器少監、大理少卿,拜侍御史兼侍講。進對,言:「荊襄制臣有
 當戒者三:曰去私、禁暴、懲怒。」權工部尚書兼權吏部尚書。又權刑部尚書兼給事中、簽書樞密院事。端平三年,拜參知政事。以資政殿學士知紹興府。嘉熙元年,復為參知政事。明年,知樞密院事兼參知政事,加資政殿大學士,賜衣帶、鞍馬。淳祐四年,復為參知政事。未幾,出知福州、福建安撫使,尋予祠。監察御史蔡次傳按劾落職,罷宮觀,後卒於嘉興。



 鄒應龍,字景初。慶元二年進士。歷官為起居舍人,以直
 龍圖閣權知贛州,遷江西提點刑獄。尋遷中書舍人兼太子右諭德,復兼太子左庶子、試戶部尚書。使金還,為太子詹事兼中書舍人。遷給事中兼太子詹事。權禮部侍郎兼侍講。權工部尚書兼同修國史、實錄院同修撰。遷刑部尚書。乞祠,以敷文閣學士提舉安慶府真原萬壽宮。以徽猷閣學士起知太平州,以臣僚論罷。以敷文閣學士提舉玉隆萬壽宮,拜禮部尚書兼侍讀。嘉熙元年,拜端明殿學士、簽書樞密院事。進資政殿學士、知慶
 元府兼沿海制置使,依舊職提舉洞霄宮。淳祐四年卒,贈少保。



 餘天錫,字純父,慶元府昌國人。丞相史彌遠延為弟子師,性謹願,絕不預外事,彌遠器重之。是時彌遠在相位久,皇子疆椓惡之,念欲有廢置。會沂王宮無後,丞相欲借是陰立為後備。天錫秋告歸試於鄉,彌遠曰:「今沂王無後,宗子賢厚者幸具以來。」



 天錫絕江與越僧同舟,舟抵西門,天大雨,僧言門左有全保長者,可避雨,如其言
 過之。保長知為丞相館客,具雞黍甚肅。須臾有二子侍立,全曰:「此吾外孫也。日者嘗言二兒後極貴。」問其姓,長曰趙與莒,次曰與芮。天錫憶彌遠所屬,其行亦良是,告於彌遠,命二子來。保長大喜,鬻田治衣冠,心以為沂邸後可冀也,集姻黨且詫其遇以行。



 天錫引見,彌遠善相,大奇之。計事洩不便,遽復使歸。保長大慚,其鄉人亦竊笑之。逾年,彌遠忽謂天錫曰:「二子可復來乎?」保長謝不遣。彌遠密諭曰:「二子長最貴,宜撫於父家。」遂載與歸。天
 錫母朱為沐浴、教字,禮度益閑習。未幾,召入嗣沂王,迄即帝位,是為理宗。



 天錫,嘉定十六年舉進士,歷監慈利縣稅,籍田令,超授起居舍人。遷權吏部侍郎兼玉牒所檢討官,兼崇政殿說書。遷戶部侍郎兼知臨安府、浙西安撫使。試戶部侍郎,權戶部尚書,皆兼知臨安府。升兼詳定敕令官,以寶文閣學士知婺州,仍舊職奉祠。起知寧國府,進華文閣學士、知福州。



 召為吏部尚書兼給事中兼侍讀。疏奏:「臣荷國恩,起家分閫,旋蒙趣覲,躐玷邇
 聯。時權禮部侍郎曹豳實在諫省,蓋嘗抗疏謂用臣大驟。臣與豳父交最久,相知最深,今觀其所論,於君父有陳善之敬,友朋有責善之道。而豳遂遷官,臣竟污要路。豳以不得其言,累疏丐去。夫亟用舊人而遂退二莊士,則將謂之何哉!豳老成之望,直諒多益,置之近班,可以正乃闢,可以儀有位。欲望委曲留行,使之釋然無疑,安於就職,則陛下既昭好賢之美,而微臣亦免妨賢之愧。」帝從之。



 嘉熙二年,拜端明殿學士、同簽書樞密院事。尋拜
 參知政事兼同知樞密院事,封奉化郡公。授資政殿學士、知紹興府、浙東安撫使。以觀文殿學士致仕。朱氏亦封周、楚國夫人,壽過九十。將以生日拜天錫為相,而天錫卒。贈少師,尋加太師,謚忠惠。



 弟天任為兵部尚書。兄弟友愛,方貧時,率更衣以出,終歲同衾。從子晦,歷官尚書,出帥全蜀,嘗置義壯,以贍宗族;然在蜀以違言論知閬州王惟忠死,士論少之。


許應龍,字恭甫,福州閩縣人。五歲通經旨,坐客曰「小兒
 氣食牛」,應龍應聲「丈夫才吐鳳」為對,四坐嘉嘆。入太學,嘉定元年舉進士。調汀州教授,差浙東宣撫司掾,差戶部架閣。遷籍田令、太學博士。時李全、時青輩歸附,應龍入對,有「
 \gezhu{
  艸幵}
 蜂是懲,養虎遺患」之說,後皆如所言。遷國子博士、國子丞、宗學博士。



 理宗即位,應龍首陳:「正心為治國平天下之綱領。」遷秘書郎兼權尚右郎官,遷著作郎。丐外,知潮州。盜陳三槍起贛州,出沒江、閩、廣間,勢熾甚。而盜鐘全相挻為亂,樞密陳韡帥江西任招捕,三路調
 軍,分道追剿。盜逼境上,應龍亟調水軍、禁卒、士兵、弓級,分扼要害。明間諜,守關隘,斷橋開塹,斬木塞塗。點集民兵,激勸隅總,諭以保鄉井、守室廬、全妻子,搜補親兵,日加訓閱。既而橫岡、桂嶼相繼以捷聞。



 招捕司遣統領官齊敏率師由漳趨潮,截贛寇餘黨。應龍諭敏曰:「兵法攻瑕,今鐘寇將窮,陳寇猖獗,若先破鐘,則陳不戰禽矣。」敏惟命,於是諸寇皆平。方未解嚴時,有行旅數人,隅總搜其橐中金銀,指為賊黨。應龍辨其非盜,釋之,皆羅拜感
 泣。始,人疑應龍儒者不閑戎事,及見其區畫事宜,分別齊民,靜練雍容,莫不嘆服。僚屬請上功,應龍曰:「守職捍城保民,何功之云?」距州六七十里曰山斜,峒獠所聚,丐耕土田不輸賦。禁兵與共,應龍平決之,其首感悅,率父老鳴缶擊筒,踴躍詣郡謝。去之日,闔郡遮道攀送。



 端平初,召為禮部郎官。入對,帝謂應龍曰:「卿治潮有聲,與李宗勉治臺齊名。」應龍頓首曰:「民無不可化,顧牧民者如何耳。臣治州幸免曠瘝,皆陛下德化所暨,臣非曰能之。」
 兼榮文恭王府教授,力辭,遷國子司業。祭酒徐僑議學校差職,欲先譽望。應龍以為不若差以資格,資格一定,則僥幸之門杜而造請之風息。僑以為然。時有憑勢干職者,力卻之。



 兼權直舍人院,遷國子祭酒。攝侍右侍郎兼學士院權直。是日,罷鄭清之、喬行簡制,應龍所草也。翼日文德殿宣布畢,帝遣中使召應龍諭之曰:「草制甚善。」應龍復謝曰:「臣聞昔人有言,進人若將加諸膝,退人若將墜諸淵。今二相乞罷機政,與陛下體貌大臣之意,
 兩盡其美可也。」帝善之,就令草敕書戒諭諸閫。權吏部侍郎兼侍講,兼權直學士院。試吏部侍郎,升侍讀,權兵部尚書。



 時楮幣虧甚,行簡主行稱提之說,州縣希旨奉承,貧富猜懼。應龍奏從民便、節用二說,行簡然之。兼吏部尚書,遷兵部兼中書舍人。三上章丐外,不允。兼給事中,兼吏部尚書。請外,詔免兼中書,拜端明殿學士、簽書樞密院事。累辭,會正言郭磊卿有論疏,以端明殿學士提舉洞霄宮。卒年八十有一。贈資政殿學士、銀青光祿
 大夫。應龍不躁不競,不激不隨,不妄薦士,而亦無傷人害物之事。潮州之治,最可紀也。



 林略,字孔英,溫州永嘉人。慶元五年,舉進士。歷饒州大寧監教授,闢乾辦四川茶馬司公事。崔與之帥蜀,目之曰「此臺閣之瑞也」,薦之。遷武學博士、國子監丞、太常寺丞。奉祠,拜宗正少卿兼崇政殿說書。遷右司諫,尋遷左司諫兼侍講,告於帝曰:「虛心以為從諫之本,從諫以為求治之本。」拜殿中侍御史,升侍御史,試右諫議大夫。嘉
 熙三年,以端明殿學士同簽書樞密院事,以言罷,提舉洞霄宮。以資政殿學士致仕。淳祐三年八月卒,特贈宣奉大夫。



 徐榮叟,字茂翁,煥章閣學士應龍之子。嘉定七年,舉進士。歷官通判臨安府,遷太學博士兼崇政殿說書,遷秘書郎,升著作佐郎兼侍左郎官。出為江東提點刑獄,直秘閣、知婺州。遷著作郎兼禮部郎官,以集英殿修撰知靜江府兼廣西經略安撫使。召為行在司諫,復兼說書
 兼侍講。



 嘉熙四年,拜右諫議大夫。入對,言:「自楮幣不通,物價倍長,而民始怨;自米運多阻,粒食孔艱,而民益怨。此見之京師者然也。外而郡邑,苛征橫斂,無所不有,嚴刑峻罰,靡所不施。和糴則科抑以取贏,軍需則並緣而規利,逃亡強令代納,蠲放忍至重催。犯私販者不問多寡,概遭黥徒;逋官課者不恤有無,動輒監系。囹圄充斥,率是乾連;詞訟追呼,莫非枝蔓。如此則民安得而不怨?甚者富家巨室,武斷鄉閭,貴族豪宗,侵牟民庶。茹冤者
 不敢告,負抑者不得伸,怨氣薰蒸,天示之應。此亢陽之所以為沴也。」



 遷權禮部尚書兼權吏部尚書,拜端明殿學士、簽書樞密院事。淳祐二年乞歸田里,以資政殿大學士提舉洞霄宮。六年,轉一官致仕。卒。



 別之傑,字宋才,郢州人。嘉定二年進士。歷官差充京西安撫司參議官,遷太府寺主簿,又遷將作監丞,差知澧州、知德安府。親喪,起復,知德安府。加直寶謨閣、知江陵府、湖北安撫副使。進直煥章閣,言親年八十,乞祠歸養,
 庶幾君親之義兩全。從之。以京湖安撫制置使陳晐論罷,以前職主管崇禧觀。進直敷文閣、知江陵府、湖北安撫使。起復,知真州,改知江寧府、湖北安撫副使,加兵部郎官,差充督視行府參謀官。遷軍器監,加直寶文閣、京西轉運判官兼提點刑獄。加秘閣修撰、知江陵兼京湖制置副使。進寶章閣待制、知太平州。又進寶謨閣學士,依舊沿江制置使兼知建康府、江東安撫使。加兵部尚書兼淮西制置使,邊事聽便行之。加端明殿學士。淳祐
 二年,授同知樞密院事兼權參知政事,進資政殿學士、湖南安撫使兼知潭州。監察御史蔡次傳論罷。七年,拜參知政事。乞歸田里,依前職知紹興府,復以兩浙轉運判官翁甫論罷。寶祐元年卒,特贈少師。



 劉伯正,字直卿,饒州餘干人。父簡,為丞相趙汝愚客,嘗書慶歷四諫奏議授伯正,而伯正以開禧元年舉進士。調太平主簿,通判棗陽軍,闢荊湖制置司機宜、兩浙轉運司主管公事。歷軍器、將作、太府三監主簿,樞密院編
 修官,兵部郎官,監察御史。有事於明堂,雷電忽至,執事者鮮不離次,伯正立殿下,紳笏儼然,聲色不動。帝遂以大任期之。



 遷左司諫,疏言:「兵籍浸廣,糧餉益艱,請豫備軍食。」又言銓選、財計、刑獄之積敝,「乞以願治之心而急董正治官之圖,以勤政之思而嚴察計吏之法」。又言:「所憂非一,而急務之當慮者有三:曰申飭邊備,區處流民,堤防奸盜。」帝皆善其言。升右正言。以華文閣待制知廣州兼廣東經略安撫使。召見,賜金帶鞍馬。改轉運使,以
 寶章閣直學士知太平州。召為禮部侍郎兼中書舍人,遷吏部侍郎兼侍講、同修國史、實錄院同修撰。兼給事中,權刑部尚書兼侍讀。



 淳祐四年,拜端明殿學士、簽書樞密院事兼權參知政事。真拜參知政事。以監察御史孫起予言罷,授資政殿學士、提舉洞霄宮。監察御史蔡次傳言之,降一官,尋復舊官致仕。卒,贈正奉大夫,加少保。時論謂伯正立朝,以靜重鎮浮,不求名譽,善藏其用云。



 金淵,字淵叔,臨安府人。嘉定七年進士。歷官為太學博士,遷太府寺丞、秘書郎。升著作佐郎兼權司封郎官。遷秘書丞,拜右正言兼工部侍郎。遷將作少監兼侍右郎官,兼國子司業,兼國史編修、實錄檢討,兼崇政殿說書。拜監察御史,論曹豳、項寅孫。兼侍講,遷禮部侍郎,尋兼國子祭酒。遷吏部侍郎,拜右諫議大夫,改左諫議大夫。遷禮部尚書兼給事中。淳祐四年,知貢舉,拜端明殿學士、同簽書樞密院事。侍御史劉漢弼論淵尸位妨賢,罷
 政予祠。監察御史劉應起言,落職罷祠。十一年,妻盛氏訴於朝,乞曲加貸宥,少敘官職。詔止量移平江府居住。卒。



 李性傳,字成之,崇正寺主簿舜臣之子也。嘉定四年舉進士。歷乾辦行在諸軍審計司。進對:「有崇尚道學之名,未遇其實。」帝曰:「實者何在?」性傳對曰:「在陛下格物致知,以為出治之本。」遷武學博士。尋為太常博士兼諸王宮大小學教授。升太常寺丞兼權工部郎中,兼權都官郎
 官,遷起居舍人兼侍講。



 疏言:「東周以後,諸侯卿大夫皆以既葬而除服。秦、漢之際,尤為淺促,孝文定為三十六日之制,則視孝惠以前已有加矣。東漢以後又損之為二十七日,謂之以日易月,則薄之至也。千數百年,惟晉武帝、魏孝文為能復古之制,而群臣沮格,未克盡行。惟孝宗通喪三年,近古所獨。陛下繼之,至性克盡,前烈有光。乞以此疏付之史官,庶幾四海聞風,民德歸厚。」



 遷起居郎,兼國史編修、實錄檢討。權刑部侍郎,進禮部侍郎。
 以臣僚言罷。尋以寶章閣待制知饒州,改知寧國府,再知饒州,復以言罷。召為兵部侍郎兼侍講,兼同修國史,兼實錄院同修撰。升兼侍讀,權兵部尚書。進讀《仁皇訓典》,乞讀《帝學》,從之。權吏部尚書。臣僚論舜臣立廟封爵事,落職,提舉太平興國宮。



 淳祐四年,權禮部尚書兼給事中,兼同修國史、實錄院同修撰,兼侍讀。五年,拜端明殿學士、簽書樞密院事兼權參知政事。尋同知樞密院事。未幾,落職與郡。十二年,以資政殿大學士提舉洞霄
 宮。寶祐二年,依舊職提舉萬壽觀兼侍讀。以觀文殿學士致仕。卒,特贈少保。



 陳韡,字子華,福州候官人。父孔碩,為朱熹、呂祖謙門人。韡讓父郊恩與弟韔。登開禧元年進士第,從葉適學。嘉定十四年,賈涉開淮閫,闢京東、河北幹官。韡謂:「山東、河北遺民,宜使歸耕其土,給耕牛農具,分配以內郡之貸死者。然後三分齊地,張林、李全各處其一,其一以待有功者。河南首領以三兩州來歸者,與節度使,一州者守
 其土,忠義人盡還北。然後括淮甸閑田,仿韓琦河北義勇法,募民為兵,給田而薄征之,擇土豪統率;鹽丁又別廩為一軍,此第二重藩籬也。」



 十五年,淮西告捷,韡策金人必專向安豐而分兵綴諸郡,使卞整、張惠、李汝舟、範成進各以其兵屯盧州以待之。金將盧鼓捶新勝於潼關,乘銳急戰,當持久困之,不過十日必遁,設伏邀擊,必可勝。又使時青、夏全候金人深入,以輕兵搗其巢穴,第一策也。其後金人果犯安豐,韡如盱眙犒師。改淮東制
 置司干辦公事。再如盱眙見劉琸,調下整、張惠、範成進、夏全諸軍應援搗虛,皆行韡之策,遂有堂門之捷,俘其四駙馬者。



 遷將作監丞,又遷太府寺丞,差知真州、淮東提點刑獄。加直寶章閣,依舊提點刑獄兼知寶應州。遷宗正寺丞、權工部郎中,改倉部員外郎。入對,言:「臣所陳夏、周、漢、唐數君之事,如布德兆謀、任賢使能、信賞必罰、區處藩鎮、不事姑息,規摹莫大於此。」又言:「人主所以御天下者,賞罰而已。」



 紹定二年冬,盜起閩中,帥王居安屬
 韡提舉四隅保甲,韡有親喪,辭之。轉運使陳汶、提舉常平史彌忠告急於朝,謂非韡莫可平。明年,以寶章閣直學士起復,知南劍州,提舉汀州、邵武軍兵甲公事,福建路兵馬鈐轄,同共措置招捕盜賊兼福建路招捕使。未幾,加提點刑獄。韡籍土民丁壯為一軍。沙縣紫雲臺告急。沙縣破,賊由間道趨城,忠勇軍破之於高橋,賊乃趨邵武,勢益熾。時有議當招不當捕者,韡言:「始者賊僅百計,招而不捕,養之至千,又養之至萬,今復養之,將至於
 無算。求淮西兵五千人可圖萬全。」詔韡兼福建路招捕使。



 賊急攻汀州,淮西帥曾式中調精兵三千五百人由泉、漳間道入汀,擊賊於順昌勝之。六月,兵大合,加福建提點刑獄。七月,韡親提兵至沙縣、順昌、將樂、清流、寧化督捕,所至克捷。九月,分兵進討。十月,進攻五賊營砦,平之。十一月,破潭瓦磜賊起之地,夷其巢穴。十二月,誅汀州叛卒,諭降連城七十有二砦,汀境皆平。四年正月,遣將破下瞿張原砦。二月,躬往邵武督捕餘寇,賊首晏彪
 迎降,韡以其力屈乃降,卒誅之。進右文殿修撰,依舊提點刑獄、招捕使兼知建寧府。衢州寇汪徐、來二破常山、開化,勢張甚。韡命淮將李大聲提兵七百,出賊不意,夜薄其砦,賊出迎戰,見算子旗,驚曰:「此陳招捕軍也!」皆大哭,急擊之,衢寇悉平。



 六年,進寶章閣待制、知隆興府。贛寇陳三槍據松梓山砦,出沒江西、廣東,所至屠殘。韡遣官吏諭降,賊輒殺之。乃謂盜賊起於貪吏,劾其尤者二人。又謂:「寇盜稽誅,以臣下欺誕、事權渙散所致,若決計
 蕩除,數月可畢。」十一月,詔節制江西、廣東、福建三路捕寇軍馬。韡奏遣將劉師直扼梅州、齊敏扼循州,自提淮西兵及親兵搗賊巢穴。十二月,兼知贛州。



 端平元年正月,進華文閣待制、江西安撫使。二月至贛,斬將士張皇賊勢及掠子女貨財者。齊敏、李大聲所至克捷。三月,分兵守大石堡,截賊糧道,遂破松梓山。三槍與餘黨縋厓而遁。韡親督諸將,乘春瘴未生,薄松梓山。賊悉精銳下山迎敵,旗幟服色甚盛。韡軍步騎夾擊,又縱火焚之,士
 皆攀厓上,賊巢蕩為煙埃,賊首張魔王自焚。斬千五百級,禽賊將十二,得所掠婦女、牛馬及僭偽服物各數百計。三槍中箭,與敏軍遇,擊敗之,賊遁。翼日,追及下黃,又敗之。餘眾尚千餘,薙獼略盡。三槍僅以數十人遁至興寧就禽,檻車載三槍等六人,斬隆興市。



 初,賊跨三路數州六十砦,至是悉平。詔曰:「韡忠勤體國,計慮精審,身任討捕之責,江、閩、東廣,訖底寧輯。」乃進權工部侍郎,仍知隆興兼江西安撫使。未幾,為工部侍郎,改江東安撫使、
 知建康府,兼行宮留守。二年,入奏事,帝稱其平寇功,韡頓首言曰:「臣不佞,徒有孤忠,仗陛下威靈,茍逃曠敗耳,何功之有。」遷權工部尚書,又權刑部尚書、沿江制置大使,依舊江東安撫使、知建康府。往來巡視鄂州江面,措置捍禦。三年,加寶謨閣學士。十月,詔選猛將精兵,相視緩急,據地利,遏要沖,以伐奸謀。嘉熙元年,進煥章閣學士。四年,拜刑部尚書,辭免。加徽猷閣學士、知潭州、荊湖南路安撫使。



 淳祐四年,召為兵部尚書,遷禮部尚書兼
 侍讀,兼同修國史、實錄院同修撰。拜端明殿學士、同簽書樞密院事兼參知政事。尋拜參知政事兼同知樞密院事。七年,知樞密院事、湖南安撫大使兼知潭州。九年,以觀文殿學士、福建安撫大使知福州,五上章辭,以舊職提舉洞霄宮。開慶元年,召赴闕,落致仕,充醴泉觀使兼侍讀。景定元年,授福建安撫大使兼知福州。久之,提舉祐神觀,力請致仕。明年卒,年八十有三。贈少師,謚忠肅。



 崔福者,故群盜,嘗為官軍所捕,會夜大雪,方與嬰兒
 同榻,兒寒啼不止,福不得寐,覺捕者至,因以故衣擁兒口,遂逸去。因隸軍籍。初從趙葵,收李全有功,名重江、淮,又累從韡捕賊,積功至刺史、大將軍。



 後從韡留隆興。既而韡移金陵,而福猶在隆興。屬通判與郡僚燕滕王閣,福恚其不見招,道遇民訴冤者,福攜其人直至飲所,責以郡官不理民事,麾諸卒盡碎飲具,官吏皆惴恐竄去,莫敢嬰其鋒。韡知之,遂檄建康,署為鈐轄。福又奪統制官王明鞍馬,及迫逐總領所監酒官親屬。韡戒諭之,不
 聽。



 會淮兵有警,步帥王鑒出師,鑒請福行,韡因厚遣之。福不樂為鑒用,遇敵不擊,托以葬女擅歸,亦不聞於制置司。鑒怒,遂白其前後過惡,請必正其慢令之罪。會韡亦厭忌之,遂坐以軍法,然後聲其罪於朝,且自劾專殺之罪。下詔獎諭,免其罪。



 福勇悍善戰,頗著威聲;其死也,軍中惜之。時論以為良將難得,而韡以私忿殺之。然福跋扈之跡已不可掩,殺身之禍,亦有以自取之也。



 論曰:宋自嘉定以來,居相位者賢否不同,故執政者各
 以其氣類而用之,因其所就而後世得以考其人焉。宣繒、薛極者,史彌遠之腹心也。陳貴誼、曾從龍、鄭性之、李性傳、劉伯正,皆無所附麗。李鳴復、金淵者,史嵩之之羽翼也。鄒應龍無所考見,許應龍治郡見稱循良,林略所謂虛心從諫者,有益於人主矣。徐榮叟父子兄弟皆為名臣,陳韡將帥才也,優於別之傑多矣。



\end{pinyinscope}