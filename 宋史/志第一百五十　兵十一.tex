\article{志第一百五十 兵十一}

\begin{pinyinscope}

 器甲之制



 器甲之制其工署則有南北作坊,有弓弩院,諸州皆有作院,皆役工徒而限其常課。南北作院歲造塗金脊鐵甲等凡三萬二千,弓弩院歲造角弝弓等凡千六百
 五十餘萬,諸州歲造黃樺、黑漆弓弩等凡六百二十餘萬。又南北作坊及諸州別造兵幕、甲袋、梭衫等什物,以備軍行之用。京師所造,十日一進,謂之「旬課」。上親閱視,置五庫以貯之。嘗令試床子弩於郊外,矢及七百步,又令別造步弩以試。戎具精致犀利,近代未有。



 開寶三年五月,詔:「京都士庶之家,不得私蓄兵器。軍士素能自備技擊之器者,寄掌本軍之司;俟出征,則陳牒以請。品官準法聽得置隨身器械。」時兵部令史馮繼升等進火箭
 法,命試驗,且賜衣物、束帛。



 淳化二年,申明不得私蓄兵器之禁。



 至道二年二月,詔:先造光明細鋼甲以給士卒者,初無襯里,宜以紬里之,俾擐者不磨傷肌體。



 咸平元年六月,御前忠佐石歸宋獻木羽弩箭,箭裁尺餘,而所激甚遠,中鎧甲則竿去而鏃存,牢不可拔。詔增歸宋月奉,且補其子為東西班侍。



 三年四月,神騎副兵馬使焦偓獻盤鐵槊,重十五斤,令偓試之,馬上往復如飛,命遷本軍使。八月,神衛水軍隊長唐福獻所制火箭、火
 球、火蒺藜,造船務匠項綰等獻海戰船式,各賜緡錢。先是,相國寺僧法山,本洺州人,強姓,其族百口,悉為戎人所掠。至是,願還俗隸軍伍以效死力,且獻鐵輪撥,渾重三十三斤,首尾有刃,為馬上格戰具。詔補外殿直。



 五年,知寧化軍劉永錫制手炮以獻,詔沿邊造之以充用。



 六年十月,給軍中傳信牌。其制,漆木為牌,長六寸,闊三寸,腹背刻字而中分之,置鑿枘令可合;又穿二竅容筆墨,上施紙札。每臨陣則分而持之,或傳令,則署其言而系軍吏
 之頸,至彼合契,乃書復命。因冀州團練使石普之請也。



 仁宗時,天下久不用兵。天聖四年,詔減諸路歲造兵器之半。是歲,詔作坊造鐵槍一萬五千,給秦、渭、環、慶、延州、鎮戎軍。



 六年,詔:外器甲久不繕,先遣使分詣諸路閱視修治之。



 景祐二年,罷秦州造輸京師弓弩三年。詔:「廣南民家毋得置博刀,犯者並鍛人並以私有禁兵律論。」先是,嶺南為盜者多持博刀,杖罪輕,不能禁,轉運使以為言,故著是令。



 四年,詔作坊制栓子槍、觚槍各五萬。



 康定
 元年四月,詔江南、淮南州軍造紙甲三萬,給陜西防城弓手。又詔河東強壯習弩者聽自置,戶四等以下官給之。八月,詔陜西制柳木旁牌。



 慶歷元年,知並州楊偕遣陽曲縣主簿楊拯獻《龍虎八陣圖》及所制神盾、劈陣刀、手刀、鐵連槌、鐵簡,且言《龍虎八陣圖》有奇有正,有進有止,遠則射,近則擊以刀盾。彼蕃騎雖眾,見神盾之異,必遽奔潰,然後以驍騎夾擊,無不勝者。歷代用兵,未有經慮及此。帝閱於崇政殿,降詔獎諭。其後,言者以為其器重
 大,緩急難用云。



 二年,詔鄜延、環慶、涇原、秦鳳路各置都作院,賜河北義勇兵弓弩箭材各一百萬。



 四年,賜鄜延路總管風羽子弩箭三十萬。



 五年,詔諸路所儲兵械悉報三司,三司歲具須知以聞,仍約為程序預頒之。



 八年,詔:「士庶之家所藏兵器,非法所許者,限一月送官。敢匿,聽人告捕。」



 皇祐元年,御崇政殿,閱知澧州、供備庫副使宋守信所獻沖陣無敵流星弩、拒馬皮竹牌、火鎌石火綱三刃、黑漆順水山字鐵甲、野戰拒馬刀弩、砦腳車、沖
 陣劍輪無敵車、大風翎弩箭八種。



 四年,河北、河東、陜西都總管司言:「郭諮所造獨轅沖陣無敵流星弩,可以備軍陣之用。詔弓弩院如樣制之。除諮為鄜延路鈐轄,許置弩五百,募土民教之。既成,經略夏安期言其便,詔立獨轅弩軍。



 五年,荊南兵馬鈐轄王遂上臨陣拐槍。



 至和元年,詔河北、河東、陜西路每歲夏曝器甲,有損斷者,悉令完備。如復閱視有不堪用者,知州、通判並主兵官並貶秩。



 嘉祐四年,詔京師所制軍器,多不鋒利,其選朝臣、
 使臣各一員揀試之。



 七年,詔江西制置賊盜司,在所有私造兵甲匠並籍姓名,若再犯者,並妻子徙淮南。



 熙寧元年,始命入內副都知張若水、西上閣門使李評料簡弓弩而增修之。若水進所造神臂弓,實李宏所獻,蓋弩類也。以□為身,檀為弰,鐵為□登子槍頭,銅為馬面牙發,麻繩扎絲為弦。弓之身三尺有二寸,弦長二尺有五寸,箭木羽長數寸,射三百四十餘步,入榆木半笴。帝閱而善之。於是神臂始用,而他器弗及焉。



 二年,命河北州軍凡戎
 器分三等以聞,又詔內庫凡器甲擇其良若干條上。



 四年,詔諸路遣官詣州,分庫藏甲兵器為三等如沿邊三路,而川峽不與。



 五年,帝匣斬馬刀以示蔡挺,挺謂制作精而操擊便,乃命中人領工造數萬口賜邊臣,鐔長尺餘,刃三尺餘,首為大環。是歲,詔權三司度支副使沉起詳定軍器制度。起以為一己之見有限,宜令在京及三路主兵官、監官、工匠審度法度所宜,庶可傳久。詔從之。



 時帝欲利戎器,而患有司茍簡。王雱上疏曰:「漢宣帝號
 中興賢主,而史稱技巧工匠,獨精於元、成之時。是雖有司之事,而上系朝廷之政。方今外御邊患,內虞盜賊,而天下歲課弓弩、甲冑入充武庫者以千萬數,乃無一堅好精利實可為備者。臣嘗觀諸州作院兵匠乏少,至拘市人以備役,所作之器,但形質而已。武庫之吏,計其多寡之數而藏之,未嘗責其實用,故所積雖多,大抵敝惡。夫為政如此,而欲抗威決勝,外攘內修,未見其可也。儻欲弛武備,示天下以無事,則金木、絲枲、筋膠、角羽之材,
 皆民力也,無故聚工以毀之,甚可惜也。莫若更制法度,斂數州之作聚為一處,若今錢監之比,擇知工事之臣使專其職;且募天下良工散為匠師,而朝廷內置工官以總制其事,察其精窳而賞罰之,則人人務勝,不加責而皆精矣。聞今武庫太祖時弓尚有如新者,而近世所造往往不可用,此可見法禁之張弛矣。」大抵雱為此言,以迎逢上意,欲妄更舊制也。



 六年,始置軍器監,總內外軍器之政。置判一人、同判一人。屬有丞,有主簿,有管當
 公事。先是,軍器領於三司,至是罷之,一總於監。凡產材州,置都作院。凡知軍器利害者,聽詣監陳述,於是吏民獻器械法式者甚眾。是歲,又置內弓箭南庫。軍器監奏以利器頒諸路作院為式。是年冬,以騎兵據大鞍不便野戰,始制小鞍,皮□毚木□登,長於回旋,馬射得以馳驟,且選邊人習騎者分隸諸軍。



 時周士隆上書論廣西、交址事,請為車以禦象陣,文彥博非之。安石以為自前代至本朝,南方數以象勝中國,士隆策宜可用,因論自古
 車戰法甚辨,請以車騎相當試,以觀其孰利。帝亦謂北邊地平,可用車為營,乃詔試車法,令沿河採車材三千兩,軍器監定法式,造戰車以進。



 七年,判監呂惠卿言:「其所上弓式及其它兵器制度,下殿前、馬、步三司令定奪去取。而逐司不過取責軍校文狀以聞,非獨持其舊說不肯更張,又其智慮未必能知作器之意。臣於朝廷已行之令,非敢言改,乞就一司同議。」帝乃遣管軍郝質赴監定奪,皆曰「便」。時軍器監制器不一,材用滋耗。於是詔
 不以常制選官馳往州縣根括牛皮角筋,能令數羨,次第加獎。是歲,始造箭曰狼牙,曰鴨觜,曰出尖四楞,曰一插刃鑿子,凡四種推行之。



 八年,詔:「河北拒馬,或多以竹為之,不足當敵。令軍器監造三萬具赴北京、澶定州。」又令計河北所少兵器制造,其不急者毋得妄費材力。又詔民戶馬死,舊不以報官者並報,輸皮筋以充用。



 帝慮置監未有實效,而虛用材役,詔中書、樞密院核實其事,令條畫以聞。軍器監奏,置監以來,增造兵器若干,為工
 若干,視前器增而工省。帝復詰之,且令與御前工作所較工孰省,驗器孰良。王韶謂:「如此,恐內外相傾成俗。且往年軍器監檢察內臣折剝弓弩,隙由此生。今令內臣較按軍器監,又如曩日相傾無已。」帝曰:「比累累說軍器監事,若不較見事實,即中外便以為聽小臣譖訴。今令得實行法,所以明曲直也。」安石曰:「誠當如此。若每事分別曲直,明其信誕,使功罪不蔽,則天下之治久矣。」王韶曰:「軍器監事不須比較。」帝曰:「事不比較,無由見枉直。」安
 石曰:「朝廷治事,唯欲直而已。」其後,安石卒以辯口解帝之疑,而軍器監獲免欺冒之罪。冬十月,軍器監欲下河東等路採市曲木為鞍橋,帝以勞民費財,不許。是時,河東、陜西、廣南帥臣邀功不已,請增給兵器,帝各令給與之。至是,有乞以耕牛博買器甲者。



 元豐元年冬,鄜延路經略使呂惠卿乞給新樣刀,軍器監欲下江、浙、福建路制造,帝不許,給以內南庫短刃刀五萬五千口。



 二年,御批有曰:「河東路見運物材於緣邊造軍器,顯為迂費張
 皇,可令軍器監速罷之。」



 三年,吉州奏:「奉詔市箭笴三十萬,非土地所產,且民間不素蓄,乞豫給緡錢,期以一年和市。」從之。



 時西邊用兵久不解。四年春,陜西轉運使李稷奏:「本道九軍,什物之外,一皆無之,乞於永興軍庫以餘財立法營辦。」七月,涇原路奏修渭州城畢,而防城戰具寡少,乞給三弓八牛床子弩、一槍三劍箭,各欲依法式制造。詔圖樣給之。



 五年七月,鄜延路計議邊事所奏乞緡錢百萬、工匠千人、鐵生熟五萬斤、牛馬皮萬張造
 軍器,並給之。八月,詔令沉括以劈陣大斧五千選給西邊諸將。十一月,陜西轉運使李察言:「本路都作院五,宜各委監司提舉。」從之。



 六年二月,詔:熙河路守具有闕,給氈三千領、牛皮萬張,運送之。八月,從環慶路趙離之請,以神臂弓一千、箭十萬給之。未幾,賜蘭會路藥箭二十五萬。



 七年,陜西轉運副使葉康直言:「秦鳳路軍器見闕名物計四百三十餘萬,使一一為之,非十餘年可就,乞自京給賜。」詔量給之。



 帝性儉約。有司造將官皮甲,欲
 以生絲染紅,代犛牛尾為瀝水,帝惜之,代以他毛。於一弓、一矢、一甲、一牌之用,無不盡心焉。弓曰闊閃促張弓,罷長弰舊法。矢曰減指箭。牌以欒竹穿皮為之,以易桐木牌。改素鐵甲為編挨甲。其法精密,乃劉昌祚、尹抃、閻守勤等所定制度雲。



 八年十月,詔內外所造軍器,以見餘物材工匠造之,兵匠、民工即罷遣之。



 元祐元年,詔:三路既罷保甲團教,其器甲各送官收貯,勿得以破損拘民整治。八月,詔太僕少卿高遵惠,會工部及軍器監內
 外作坊及諸州都作院工器之數,以要切軍器立為歲課,務得中道,他非要切,並權住勿造。於是數年之間,督責少弛。



 紹聖三年,有司言:「州郡兵備,全為虛文,恐緩急不足備御。請稍推行熙寧之詔,常令封樁、排垛,依雜隊法。」從之。



 元符元年,詔江、湖、淮、浙六路合造神臂弓三千、箭三十萬。



 二年,臣僚奏乞增造神臂弓,於是軍器監所造歲益千餘弓。是歲,詔河北沿邊州城壁、樓櫓、器械、各務修治,有不治者罪之。



 先是,二廣路土丁令依熙寧指揮
 修置器械。三年,知端州蕭刓上疏,極言傷財害民,其弊非一,乞住買槍手器械。疏奏不報。



 崇寧初,臣僚爭言元祐以來因循弛廢,兵不犀利。詔復令諸路都作院創造修治,官吏考察一如熙寧時矣。時有詔造五十將器械,從工部請,令內外共造,由是都大提舉內外制造軍器所之名立焉。



 初,從邢恕之議,下令創造兵車數十乘,買牛以駕。已而蔡碩又請河北置五十將兵器,且為兵車萬乘。蔡京主其說,奸吏旁緣而因為民害者深矣。



 崇
 寧三年,河北、陜西都轉運司言:「兵車之式,若用許彥圭所定,則車大而費倍。若依往年二十將舊式,則輕小易用,且可省費。」詔卒以許彥圭式行之。時熙河轉運副使李復先奏曰:「今之用兵,與古不同。古者征戰有禮,不為詭遇,多由正途,故車可行而敵不敢輕犯。今之用兵,盡在極邊,下砦駐軍,各以保險為利,車不能上。又戰陣之交,一進一退,車不能及,一被追襲,遂非己有。臣屢觀戎馬之間,雖糧糗、衣服、器械不能為用,況於車乎?臣聞此車
 之造,許彥圭因姚麟以進其說。朝廷以麟熟於邊事,而不知彥圭輕妄、麟立私恩以誤國計。其車比於常法闊六七寸,運不合轍,東來兵夫牽挽不行,以致典賣衣物,自賃牛具,終日而進六七里,棄車而逃者往往而是。夫未造則有配買物材、顧差夫匠之擾;既成,又難運致,則為諸路之患有不可勝言者矣。彥圭但圖一官之得,不知有誤於國,此而不誅,何以懲後!今乞便行罷造,已造者不復運來,以寬民力。」其後,彥圭卒得罪。



 元豐之時,河
 北、河東路軍器,每季終委逐路職司更互考察。元祐罷之。四年,因工部之請,復行之。



 大觀二年,手詔曰:「前東南備御指揮,深慮監郡縣吏急切者倚法害民,廢職者慢令失事,如築城壁、造軍器、收戰馬、習水戰之類,並可量度工力,計以歲月,漸次興作,毋得急遽科斂及差雇百姓,使急不擾民,緩不廢事,然後為稱。」尋詔限十年一切畢工。四月,罷黎、雅等州市犛牛尾,慮為民害。八月,提舉御前軍器所奏,乞如崇寧五年指揮,下諸路買牛角四
 十萬只、筋十萬斤。從之。



 政和二年二月,詔諸路州郡造軍器有不用熙寧法式者,有司議罰,具為令。六月,又詔並用御前軍器所降法式,前二月指揮勿行。



 三年,詔:「馬甲曩用黑髹漆,今易以朱。」是歲,姚古奏更定軍器,曩時甲二副,今拆造三副;曩時手刀太重,今皆令輕便易用;曩時神臂弓硾二石三斗,今硾一石四斗。從之,悉下諸路改造。



 六年,軍器少監鄧之綱奏:「國家諸路為將一百三十有一,訓練士卒,各給軍器,以備不虞。惟河北諸將
 軍器乃熙、豐時造,精利牢密,冠於諸路。臣恐歲久因循,多致損弊。乞自河北、陜西路為首,令諸路一新戎器,仰稱陛下追述先志,儲戎器、壯國威之意。」從之。



 七年,之綱三上奏,一言修武庫,二言整軍器,大省國用。詔升之綱為大監,又遷一官。時宇文粹中賜對崇政殿,奏武庫事,因奏:「武庫有祖宗所御軍器十餘色,乞編入《鹵簿圖志》,遇郊兵重禮,陳於儀物之首,以識武功,且示不忘創業艱難意。」是年,御筆以武庫當修軍器近一億萬,其中箭
 鏃五千餘萬,用平時工料,須七十年餘然後可畢。於是令鄧之綱分給沿流作院,限三年修之,而權住三年上供軍器。



 八年,以之綱奏,諸路歲起上供料買分數,特免三年綱發。然自時厥後,申明郡縣牛皮角筋之禁,紛然為害者,之綱之請也。



 宣和元年,權荊湖南路提點刑獄公事鄭濟奏:「本路惟潭、邵二州,各有年額制造軍器。今年制造已足,躬親試驗,並依法式,不誤施用。」詔加旌賞,以為諸路之勸。然自是歲督
 軍器率用御筆處分,工造不已而較數嘗闕,繕修無虛歲而每稱弊壞。大抵中外相應,一以虛文,上下相蒙,而馴致靖康之禍矣。



 靖康初,兵仗皆闕,詔書屢下,嚴立賞刑,而卒亦無補。時通判河陽、權州事張旗奏曰:「河陽自今春以來,累有軍馬經過,軍士舉隨身軍器若馬甲、神臂弓、箭槍牌之類,於市肆博易熟食,名為寄頓,其實棄遣,避逃征役。拘收三日間,得器械四千二百餘物。此乃太原援師,尚且棄捐器甲,則他路軍馬事勢可知。宜諭民首納,免貽他患。」帝善旗
 奏,賞以一官。



 初,御前軍器監、軍器所萬全軍匠以三千七百為額,東、西作坊工匠以五千為額。紹興初,役兵才千人,久之,增至五千六百餘,又於諸道增二千九百餘,本券外復增給日錢百七十、月米七斗半。於是內庫累歲兵械山積,而諸軍悉除戎器。二十六年,詔:「工匠宜減免,江、浙、福建諸州物料悉蠲之。」有司奏物料減三之一,工匠二千、雜役兵五百為額。



 舊,軍器所得專達。建炎中,嘗以閹官董愨提舉,尋罷之。紹興五年,隸工部,後復以中
 人典領。三十年,工部言非祖宗建官意,詔依條檢察。孝宗受禪,增提點官一員,御史力論其不可,復隸工部焉。



 造車之制。渡江後,東南地多沮洳險隘,不以車為主。宗澤、李綱有戰車法,王大智獻車式,皆不復用,而屬意甲冑、弧矢之利矣。建炎初,上諭宰執曰:「方今戰士無慮三十萬,若皆被堅執銳,加以弧矢之利,雖強敵,無足畏也。造弓必用良工善價。」紹興三年,提舉制造軍器所言:「以七十工造全裝甲一。又長齊頭甲每一甲用工百四
 十一,短齊頭甲用工七十四。乞以本所全裝甲為定式。」席益言:「諸州造馬蝗弩,不若令造弓。」詔並改造弓弩,內馬蝗弩改手射弓。



 紹興四年,軍器所言:「得旨,依禦降式造甲。緣甲之式有四等,甲葉千八百二十五,表裏磨珵。內披膊葉五百四,每葉重二錢六分;又甲身葉三百三十二,每葉重四錢七分;又腿裙鶻尾葉六百七十九,每葉重四錢五分;又兜鍪簾葉三百一十,每葉重二錢五分。並兜鍪一,杯子、眉子共一斤一兩,皮線結頭等重五
 斤十二兩五錢有奇。每一甲重四十有九斤十二兩。若甲葉一一依元領分兩,如重輕差殊,即棄不用,虛費工材。乞以新式甲葉分兩輕重通融,全裝共四十五斤至五十斤止。」詔勿過五十斤。三十二年,詔江東安撫司造木弩五千、箭五十萬。



 隆興元年,御降木羽弩箭式,每路依式制箭百萬。淳熙九年,衢州守臣制到木鶴觜弩二千、箭十萬。又湖北、京西造納無羽箭。上曰:「箭不用羽,可謂精巧,其屋藏之。」淮東總領朱佺言:「鎮江一軍,乃韓世
 忠部曲。世忠造克敵弓,以當敵騎沖突,其發可至百步,其勁可穿重甲,最為利器。往歲調發,弓不免損失,存者歲久亦漸弛壞。今考諸軍見弩手八千八百四十二人,人合用兩弓,一弓一日上教,一弓備出戰,合用弓萬七千六百八十有四,僅存六千五百七十有四,餘皆不堪施教,乞下鎮江都統司足其額。」



 十五年,工部侍郎李昌圖言:「弓矢之利,貴於便疾。神臂弓鬥力及遠,屢獲其用。後又又造神勁弓,及遠雖在神臂弓上,軍中多言其發遲,
 每神臂三矢而神勁方能一發,若臨敵之際,便疾反出神臂下。」上曰:「平原曠野宜用神勁弓,西蜀崇山峻嶺,未知孰利。」詔金州都統司詳議以聞。既而都統制吳挺奏:「神勁弓並彈子頭箭,諸軍用之誠便疾,神臂不及也。」詔從其便。楚州兵馬鈐轄言:「弩之力,勁者三十石,次者十五石,矢之鏃狀若鍬,所發何啻數百步,洞穿數人。江上諸軍有弩式,皆廢不修。」詔兩淮、荊襄沿邊城守,各制二十枝,御前軍器所亦如之。紹熙而後,日造器械,數目山
 積。



 開慶元年,壽春府造𫧝筒木弩,與常弩明牙發不同,箭置筒內甚穩,尤便夜中施發。又造突火槍,以鉅竹為筒,內安子窠,如燒放,焰絕然後子窠發出,如炮聲,遠聞百五十餘步。



 咸淳九年,沿邊州郡,因降式制回回炮,有觸類巧思,別置炮遠出其上。且為破炮之策尤奇。其法,用稻穰草成堅索,條圍四寸,長三十四尺,每二十條為束,別以麻索系一頭於樓後柱,搭過樓,下垂至地,梁梁垂四層或五層,周庇樓屋,沃以泥漿,火箭火炮不能侵,
 炮石雖百鈞無所施矣。且輕便不費財,立名曰「護陴籬索。」是時兵紀不振,獨器甲視舊制益詳。



\end{pinyinscope}