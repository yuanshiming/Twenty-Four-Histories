\article{列傳第二百四十世家四}

\begin{pinyinscope}

 ◎
 世家四○南漢劉氏



 南漢劉鋹,其先蔡州上蔡人,高祖安仁,仕唐為潮州刺史,因家嶺表。安仁生謙,為廣州牙校,累遷封州刺史、賀
 水鎮遏使。謙生隱,謙卒,隱代領其任。唐昭宗以薛王知柔鎮南海,辟為行軍司馬,委以兵柄。及宰相徐彥若代知柔,以為節度副使。時唐室已季,彥若威令不振,事皆決於隱。彥若卒,遺表薦隱自代,昭宗不從,以崔遠代之。遠至江陵,遷延不進,乃以隱為留後,未幾,授以節旄。梁開平初,兼靜海軍節度使,封南海王。隱卒,弟陟襲位。貞明三年,僭帝號,國稱大漢,改元乾亨,行郊祀禮。改名岩,又改龔,終改龑。「龑」讀為「儼」,字書不載,蓋其妄作也。晉天
 福七年,卒,子玢嗣,為弟晟所殺。晟遂自立,性尤酷暴,周顯德五年,卒,事具《五代史》。



 鋹即晟長子也,初名繼興,封衛王,襲父位,改今名,改元大寶。性昏懦,委政宦官龔澄樞及才人盧瓊仙,每詳覽可否,皆瓊仙指之。鋹日與宮人、波斯女等遊戲。內官陳延壽引女巫樊胡入宮,言玉皇遣樊胡命鋹為太子皇帝,乃於宮中施帷幄,羅列珍玩,設玉皇坐。樊胡遠遊冠、紫衣、紫霞裙,坐宣禍福,令鋹再拜聽命;嘗雲瓊仙、澄樞、延壽皆玉皇遣輔太子皇帝,
 有過不得治。又有梁山師、馬媼、何擬之徒出入宮掖。宮中婦人皆具冠帶,領外事。



 初,龑雖寵任中官,其數裁三百餘,位不過掖庭諸局令丞。至晟時千餘人,稍增內常侍、諸謁者之稱。至鋹漸至七千餘,有為三師、三公,但其上加「內」字,諸使名不翅二百,女官亦有師傅、令仆之號。目百官為「門外人」,群臣小過及士人、釋、道有才略可備問者,皆下蠶室,令得出入宮闈。作燒煮剝剔、刀山劍樹之刑,或令罪人鬥虎抵象。又賦斂煩重,邕民入城者人
 輸一錢,瓊州米斗稅四五錢。置媚川都,定其課,令入海五百尺采珠。所居宮殿以珠、玳瑁飾之。陳延壽作諸淫巧,日費數萬金。宮城左右離宮數十,鋹遊幸常至月餘或旬日。以豪民為課戶,供宴犒之費。



 乾德中,太祖命師克郴州,獲其內品十餘人。有餘延業者,人質麼麽,太祖問曰:「爾在嶺南為何官?」對曰:「為扈駕弓箭手官。」命授之弓矢,延業極力控弦不開。太祖因笑問鋹為治之跡,延業備言其奢酷,太祖驚駭曰:「吾當救此一方之民。」



 先是,
 晟因湖南馬氏之亂,襲取桂、郴、賀等州。開寶初,鋹又舉兵侵道州,刺史王繼勳上言。鋹為政昏暴,民被其毒,請討之。太祖難其事,令江南李煜遣使以書諭鋹使稱臣,歸湖南舊地。鋹不從。煜又遣其給事中龔慎儀遺書曰:



 煜與足下叨累世之睦,繼祖考之盟,情若弟兄,義敦交契,憂戚之患,曷嘗不同。每思會麵而論此懷,抵掌而談此事,交議其所短,各陳其所長;使中心釋然,利害不惑,而相去萬里,斯願莫伸。凡於事機不得款會,屢達誠素,
 冀明此心;而足下視之,謂書檄一時之儀,近國梗概之事,外貌而待之,泛濫而觀之,使忠告確論如水投石,若此則又何必事虛詞而勞往復哉?殊非宿心之所望也。



 今則復遣人使罄申鄙懷,又慮行人失辭,不盡深素,是以再寄翰墨,重布腹心,以代會麵之談與抵掌之議也。足下誠聽其言如交友諫爭之言,視其心如親戚急難之心,然後三復其言,三思其心,則忠乎不忠,斯可見矣,從乎不從,斯可決矣。



 昨以大朝南伐,圖復楚疆,交兵已
 來,遂成釁隙。詳觀事勢,深切憂懷,冀息大朝之兵,求契親仁之願,引領南望,於今累年。臨昨使臣入貢大朝,大朝皇帝果以此事宣示曰:「彼若以事大之禮而事我,則何苦而伐之;若欲興戎而爭我,則以必取為度矣。」見今點閱大眾,仍以上秋為期,令弊邑以書復敘前意,是用奔走人使,遽貢直言。深料大朝之心非有唯利之貪,蓋怒人之不賓而已;足下非有不得已之事與不可易之謀,殆一時之忿而已。



 觀夫古之用武者,不顧小大強弱
 之殊而必戰者有四;父母宗廟之仇,此必戰也;彼此烏合,民無定心,存亡之機以戰為命,此必戰也;敵人有進,必不舍我,求和不得,退守無路,戰亦亡,不戰亦亡,奮不顧命,此必戰也;彼有天亡之兆,我懷進取之機,此必戰也。今足下與大朝非有父母宗廟之仇也,非同烏合存亡之際也,既殊進退不舍、奮不顧命也,又異乘機進取之時也。無故而坐受天下之兵,將決一旦之命,既大朝許以通好,又拒而不從,有國家、利社稷者當若是乎?



 夫
 稱帝稱王,角立傑出,今古之常事也;割地以通好,玉帛以事人,亦古今之常事也。盈虛消息、取與翕張,屈伸萬端,在我而已,何必膠柱而用壯,輕禍而爭雄哉?且足下以英明之姿,撫百越之眾,北距五嶺,南負重溟,藉累世之基,有及民之澤,眾數十萬,表裏山川,此足下所以慨然而自負也。然違天不祥,好戰危事,天方相楚,尚未可爭。恭以大朝師武臣力,實謂天讚也。登太行而伐上黨,士無難色;絕劍閣而舉庸蜀,役不淹時。是知大朝之力
 難測也,萬里之境難保也。十戰而九勝,亦一敗可憂;六奇而五中,則一失何補!



 況人自以我國險,家自以我兵強,蓋揣於此而不揣於彼,經其成而末經其敗也。何則?國莫險於劍閣,而庸蜀已亡矣;兵莫強於上黨,而太行不守矣。人之情,端坐而思之,意滄海可涉也,及風濤驟興,奔舟失馭,與夫坐思之時蓋有殊矣。是以智者慮於未萌,機者重其先見;圖難於其易,居存不忘亡,故曰計禍不及,慮福過之。良以福者人之所樂,心樂之,故其望
 也過;禍者人之所惡,心惡之,故其思也忽。是以福或修於慊望,禍多出於不期。



 又或慮有矜功好名之臣,獻尊主強國之議者,必曰:「慎無和也。五嶺之險,山高水深,輜重不並行,士卒不成列;高壘清野而絕其運糧,依山阻水而射以強弩,使進無所得,退無所歸。」此其一也。又或曰:「彼所長者,利在平地,今舍其所長,就其所短,雖有百萬之眾,無若我何。」此其二也。其次或曰:「戰而勝,則霸業可成,戰而不勝,則泛巨舟而浮滄海,終不為人下。」此大
 約皆說士孟浪之談,謀臣捭闔之策,坐而論之也則易,行之如意也則難。



 何則?今荊湘以南、庸蜀之地,皆是便山水、習險阻之民,不動中國之兵,精卒已逾於十萬矣。況足下與大朝封疆接畛,水陸同途,殆雞犬之相聞,豈馬牛之不及?一旦緣邊悉舉,諸道進攻,豈可俱絕其運糧,盡保其城壁?若諸險悉固,誠善莫加焉;苟尺水橫流,則長堤虛設矣。其次曰,或大朝用吳越之眾,自泉州泛海以趣國都,則不數日至城下矣。當其人心疑惑,兵勢
 動搖,岸上舟中皆為敵國,忠臣義士能復幾人?懷進退者步步生心,顧妻子者滔滔皆是。變故難測,須臾萬端,非惟暫乖始圖,實恐有誤壯誌,又非巨舟之可及,滄海之可遊也。然此等皆戰伐之常事,兵家之預謀,雖勝負未知,成敗相半。苟不得已而為也,固斷在不疑;若無大故而思之,又深可痛惜。



 且小之事大,理固然也。遠古之例不能備談,本朝當楊氏之建吳也,亦入貢莊宗。恭自烈祖開基,中原多故,事大之禮,因循未遑,以至交兵,幾
 成危殆。非不欲憑大江之險,恃眾多之力,尋悟知難則退,遂修出境之盟;一介之使才行,萬里之兵頓息,惠民和眾,於今賴之。自足下祖德之開基,亦通好中國,以闡霸圖。願修祖宗之謀,以尋中國之好,蕩無益之忿,棄不急之爭;知存知亡,能強能弱,屈己以濟億兆,談笑而定國家,至德大業無虧也,宗廟社稷無損也。玉帛朝聘之禮才出於境,而天下之兵已息矣,豈不易如反掌,固如太山哉?何必扼腕盱衡,履腸蹀血,然後為勇也。故曰:「德
 如毛,民鮮克舉之,我儀圖之。 」又曰:「知止不殆,可以長久。」又曰:「沈潛剛克,高明柔克。」此聖賢之事業,何恥而不為哉?



 況大朝皇帝以命世之英,光宅中夏,承五運而乃當正統,度四方則咸偃下風;獫狁、太原固不勞於薄伐,南轅返旆更屬在於何人。又方且遏天下之兵鋒,俟貴國之嘉問,則大國之義斯亦以善矣,足下之忿亦可以息矣。若介然不移,有利於宗廟社稷可也,有利於黎元可也,有利於天下可也,有利於身可也。凡是四者無一
 利焉,何用棄德修怨,自生仇敵,使赫赫南國,將成禍機,炎炎奈何,其可向邇?幸而小勝也,莫保其後焉,不幸而違心,則大事去矣。



 復念頃者淮、泗交兵,疆陲多壘,吳越以累世之好,遂首為厲階;惟有貴國情分逾親,歡盟愈篤,在先朝感義,情實慨然,下走承基,理難負德,不能自己,又馳此緘。近奉大朝諭旨,以為足下無通好之心,必舉上秋之役,即命弊邑速絕連盟。雖善鄰之心,期於永保;而事大之節,焉敢固違。恐煜之不得事足下也,是以
 惻惻之意所不能雲,區區之誠於是乎在。又念臣子之情,尚不逾於三諫,煜之極言,於此三矣,是為臣者可以逃,為子者可以泣,為交友者亦惆悵而遂絕矣。



 鋹得書,遂囚慎儀,驛書答煜,言甚不遜,煜上其書。



 開寶三年,太祖命潭州防禦使潘美、朗州團練使尹崇珂討之。八月,師至白霞,



 鋹賀州刺史陳守忠告急於鋹。時舊將多以讒構誅死,宗室翦滅殆盡,掌兵者唯宦人數輩。自晟以來,耽於遊宴,城壁壕隍多飾為宮館池沼,樓艦皆毀,兵
 器又腐,內外震恐。乃遣龔澄樞往賀州,郭崇嶽往桂州,李托往韶州,畫守禦之策。



 九月,美與崇珂圍賀州,澄樞遁歸。鋹遣大將伍彥柔領兵赴賀,美等以奇兵伏南鄉岸。彥柔夜至,艤舟岸側,遲明挾彈登岸,踞胡床指麾。伏兵卒發,彥柔眾大亂,死者千人。擒彥柔斬之,梟首以示城中。翌日,城陷。美等督戰艦,聲言順流趨廣州,鋹令都統潘崇徹將兵五萬屯賀江。十月,美等次昭州,破開建砦,殺卒數百,擒砦將靳暉,昭州刺史田行稠遁去,城遂陷。
 桂州刺史李承進棄城亦奔。十一月,連州陷,招討使盧收率眾退保清遠。十二月,美等攻韶州,都統李承渥以兵數萬陣蓮華山下。初,鋹教象為陣,每象載十數人,皆執兵仗,凡戰必置陣前,以壯軍威。至是與美遇,美盡索軍中勁弩布前以射之,象奔踶,乘象者皆墜,反踐承渥軍,遂大敗,承渥僅以身免。韶州陷,擒刺史辛延渥、諫議大夫卿文遠。鋹始令塹廣州東壕,遣郭崇嶽統兵六萬屯馬逕,列柵以拒之。



 四年正月,美等破英、雄二州,都統
 潘崇徹來降。翌日,次瀧頭,鋹遣使請和,且求緩師。瀧頭山水險惡,美等疑有伏兵,乃挾鋹使速度諸險。二月,過馬逕,去廣城十里,砦於雙女山下。鋹聞之,取舶船十餘艘,載金寶、妃嬪欲入海。未及發,宦官樂範與衛兵千餘盜舶船走。美等將至城,鋹懼,遣其右僕射蕭漼奉表詣軍門乞降。美諭太祖意,語在《美傳》。使者乞部送赴闕,師遂頓城外。鋹又遣其弟保興率百官奉迎,為郭崇嶽所遏。崇嶽無謀勇,但祈禱鬼神,復為拒捍之備。美等乃進
 攻,保興迎戰,大為所敗,美乘風縱火,煙埃坌起,崇嶽死於亂兵。城既破,鋹盡焚其府庫。美擒鋹及龔澄樞、李托、薛崇譽與宗室文武九十七人,同縻於龍德宮。保興逃於民家,亦獲之,悉部送闕下。斬閹工五百餘人。凡得州六十、縣二百十四、戶十七萬。鋹至江陵,邸吏龐師進迎謁,學士黃德昭侍鋹,鋹問師進何人,德昭曰:「本國人也。」鋹曰:「何為在此?」曰:「先主歲貢大朝,輜重比至荊州,乃令師進至邸,於此造車,以給饋運爾。」鋹歎曰:「我在位十四
 年,未嘗聞此言,今日始知祖宗山河及大朝境土也。」因泣涕久之。



 至京,舍於玉津園,太祖遣參知政事呂餘慶問鋹翻覆及焚府庫之罪,鋹歸罪澄樞、托、崇譽。翌日,有司以帛係鋹及其官屬獻太廟、太社。太祖禦明德門,遣攝刑部尚書盧多遜宣詔責鋹,鋹對曰:「臣年十六僭偽位,澄樞等皆先臣舊人,每事臣不得專,在國時臣是臣下,澄樞是國主。」遂伏地待罪。太祖命攝大理卿高繼申引澄樞、托、崇譽斬於千秋門外。釋鋹罪,賜襲衣、冠帶、器
 幣、鞍勒馬,授金紫光祿大夫、檢校太保、右千牛衛大將軍、員外置同正員,封恩赦侯,朝會班上將軍之下。以其弟保興為右監門率府率,左僕射蕭漼為太子中允,中書舍人卓惟休為太僕寺丞,餘並署諸州上佐、縣令、主簿。



 初,龑時嘗召司天監周傑筮之,遇《復》之《豐》,龑問曰:「享年幾何?」傑曰:「凡二卦皆土為應,土之數五,二五,十也,上下各五,將五百五十五乎。」及鋹之敗,果五十五年,蓋傑舉成數以避一時之害爾。又廣州童謠曰:「羊頭二四,白
 天雨至。」識者以羊是未之神,是歲歲在辛未,以二月四日擒鋹。天雨者,王師如時雨之義。又前一年九月八日夕,眾星皆北流,有知星者言,劉氏歸朝之兆也。



 四年,詔鋹月給增錢五萬、米麥五十斛。八年,李煜平,遷左監門衛上將軍,進封彭城郡公。太平興國初,又進衛國公。五年,卒,年三十九。廢朝三日,贈太師,追封南越王。



 鋹體質豐碩,眉目俱竦。有口辯,性絕巧,嘗以珠結鞍勒為戲龍之狀,極其精妙,以獻太祖。太祖詔示諸宮官,皆駭伏,遂
 以錢百五十萬給其直,謂左右臣曰:「鋹好工巧,習以成性,儻能以習巧之勤移於治國,豈至滅亡哉!」



 太祖嘗乘肩輿從十數騎幸講武池,從官未集,鋹先至,賜鋹卮酒。鋹疑為冘,泣曰:「臣承祖父基業,違拒韓廷,勞王師致討,罪固當死,陛下不殺臣,今見太平,為大梁布衣足矣。願延旦夕之命,以全陛下生成之恩,臣未敢飲此酒。」太祖笑曰:「朕推心於人腹,安有此事!」命取鋹酒自飲之,別酌以賜,鋹大慚頓首謝。



 太宗將討晉陽,召近臣宴,鋹預之,
 自言:「朝廷威靈及遠,四方僭竊之主,今日盡在坐中,旦夕平太原,劉繼元又至,臣率先來朝,願得執梃為諸國降王長。」太宗大笑,賞賜甚厚。其詼諧此類也。



 鋹子守節、守正,皆至崇儀副使。守正卒,帝聞其家貧,詔月給萬錢。守素,咸平中為侍禁,亦貧,真宗賜白金百兩,語宰相曰:「諸偽主子孫率多窘迫,蓋僭侈之後不知稼穡艱難所致也。」後至內殿崇班,天禧中,又錄為閣門祗候。守通,供奉官。守正子克昌,為三班奉職;國昌,為借職。



 龔澄樞,廣州南海人。性廉謹,不妄交遊。幼事龑為內供奉官,累遷內給事。晟襲位,任閹人林延遇為甘泉宮使,頗預政事。延遇病將死,言於晟曰:「臣死,惟龔澄樞可用。」即日擢知承宣院兼內侍省,改德陵使兼龍德宮使。鋹嗣位,加特進、開府儀同三司、萬華宮使、驃騎大將軍,改上將軍、左龍虎軍觀軍容使、內太師,軍國之務皆決於澄樞。澄樞與李托、薛崇譽置酷法之具,民甚苦之。



 初,岩改名龔,有術者言不利,名龔,當敗國事,遂改名龑。後鋹
 用澄樞,以其姓卒亡其國,澄樞亦被誅。



 李托,封州封川人。少習騎射,以謹願事龑為內府局令。晟襲位,遷內侍省內侍,充宮闈諸衛押番兼秀華宮使。鋹立,改玩華宮使、內侍監兼列聖、景陽二宮使。托納二女於鋹,鋹以其長為貴妃,次為美人,政事皆訪托而後行。加特進、開府儀同三司、甘泉宮使兼六軍觀軍容使、行內中尉,遷驃騎上將軍、內太師。



 太祖命師伐鋹,既克韶州,統軍使李承渥戰死,節度副使辛延渥間道遣人
 勸鋹降,托堅沮其議。及就擒至許田,太祖遣使問托等:「昨已約降,復率眾來拒戰,及軍敗又縱火焚府庫,誰為之謀也?」托俯首不能對。鋹諫議大夫王珪謂托曰:「昔在廣州,機務並爾輩所專,火又自內起,今天子遣使案問,爾復欲推過何人?」遂唾而批其頰,托乃引伏,後至京斬之。



 薛崇譽,韶州曲江人。善《孫子五曹算》。晟署為內門使兼太倉使。鋹嗣位,遷內中尉、特進、開府儀同三司、簽書點
 檢司事。太祖命師克廣州,崇譽縱火焚倉廩,擒至京,與李托同戮。



 潘崇徹,廣州南海人。事龑為內侍省局丞。頗讀兵書,立戰功。晟嘗遣大將吳懷恩伐桂州平之,懷恩為部下所殺,命崇徹代之。鋹襲位,加西北面都統。歲餘,鋹頗疑崇徹,遣薛宗譽使其軍以察之。崇譽還,遂白崇徹日以伶人百餘衣錦繡、吹玉笛,為長夜之飲,不恤軍政。鋹怒,召歸,奪其兵柄,自是居常怏怏。太祖命師度嶺,鋹復命崇
 徹領兵五萬戍賀江,崇徹不為效命。鋹敗,至京,太祖知其事,特赦之,授汝州別駕,卒。



\end{pinyinscope}