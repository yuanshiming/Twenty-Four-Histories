\article{列傳第一百八十}

\begin{pinyinscope}

 ○楊
 棟姚希得包恢常挺陳宗禮常家鉉翁李庭芝



 楊棟,字元極,眉州青城人。紹定二年進士第二。授簽書劍南西川節度判官廳公事。未上,丁母憂。服除,遷荊南
 制置司,改闢西川,入為太學正。丁父憂,服除,召試授秘書省正字兼吳益王府教授,遷校書郎、樞密院編修官。入對,言:「飛蝗蔽天,願陛下始終一德,庶幾感格天心,消弭災咎。」又言:「邇來中外之臣,如主兵理財,聽其言無非可用,跡其實類皆欺誣,上下相蒙,無一可信。陛下先之以至誠,而後天下之事可為也。」又言:「祖宗立國,不恃兵財法,惟恃民心固結而已。願陛下常存忠厚之意,勿用峻急之人。」理宗悅,以臣僚言奉祠。



 起知興化軍。孔子
 之裔有居水亟頭鎮者,棟為建廟闢田,訓其子弟。遷福建提點刑獄,尋加直秘閣兼權知福州,兼本路安撫使,遷都官郎官,又遷左司郎官,尋為右司郎官兼玉牒所檢討官,除宗正少卿。進對,帝曰:「止是正心修身之說乎?」棟對曰:「臣所學三十年,止此一說。用之事親取友,用之治凋郡、察冤獄,至為簡易。」時有女冠出入宮禁,頗通請謁,外廷多有以為言者。棟上疏曰:「陛下何惜一女冠,天下所側目而不亟去之乎?」帝不謂然。棟曰:「此人密交小人,甚
 可慮也。」又言:「京、襄、兩淮、四川殘破郡縣之吏,多是兵將權攝,科取無藝,其民可矜,非陛下哀之,誰實哀之。」帝從之。



 遷太常少卿、起居郎,差知滁州,以殿中侍御史周坦論罷。起直龍圖閣、知建寧府,不拜。提舉千秋鴻禧觀,遷起居郎兼權侍左侍郎、崇政殿說書,繼遷吏部侍郎兼同修國史、實錄院同修撰兼侍讀,以集英殿修撰兼中書舍人兼侍講,出知太平州,以右補闕蕭泰來論罷,依舊職提舉太平興國宮。起知婺州。召奏事,以舊職奉祠。
 度宗立為太子,帝親擢棟太子詹事。遷工部侍郎,仍為詹事兼同修國史、實錄院同修撰兼中書舍人,兼直學士院,權刑部尚書兼國子祭灑,遷禮部尚書,加端明殿學士、同簽書樞密院事兼太子賓客,進同知樞密院事兼權參知政事,拜參知政事。



 臺州守王華甫建上蔡書院,言於朝,乞棟為山主,詔從之。因卜居於臺。尋授資政殿學士、知建寧府,不拜。以舊職提舉洞霄宮,復依舊職知慶元府、沿海制置使。以監察御史胡用虎言罷,仍
 奉祠。加觀文殿學士知慶元府、沿海制置使,又不拜,仍奉祠。乃以資政殿大學士充萬壽觀使。卒,遺表上,帝輟朝,特贈少保。



 棟之學本諸周、程氏,負海內重望。方賈似道入相,登用故老,列之從官,棟亦預焉。及彗星見,棟乃言蚩尤旗,非彗也,故為世所少云。或謂棟姑為是言,陰告於帝,謀逐似道,似道覺之,遂蒙疑而去。所著有《崇道集》、《平舟文集》。



 姚希得,字逢原,一字叔剛,潼川人,嘉定十六年進士。授
 小溪主簿,待次三年,朝夕討論《六經》、諸子百家之言。調盤石令。會蜀有兵難,軍需調度不擾而集,更調嘉定府司理參軍。改知蒲江縣。巨室挾勢,邑號難治。希得綏強扶弱,聲聞著聞。同知樞密院事游似以希得名聞,召審察,遷行在都進奏院,通判太平州,改福州,徒步至候官,吏不知為通判也。



 召為國子監丞,遷太府寺丞,時暫書擬金部文字兼沂靖王府教授。時帝斥逐權奸,收召名德,舉朝相慶。希得以為外觀形狀,似若清明之朝;內察
 脈息,有類危亡之證。乃上疏言:「堯、舜、三代之時,無危亡之事,而常喜危亡之言;秦、漢以來,多危亡之事,而常諱危亡之言。夫危亡之事不可有,而危亡之言不可亡。後世人主乃履危如履坦,諱言如諱病。」又言:「君子非不收召,而意向猶未調一;小人非不斥逐,而根株猶未痛斷。大權若操握,而不能無旁蹊曲逕之疑;大勢若更張,而未見有長治久安之道。廷臣之所諷諫,封囊之所奏陳,非不激切,而陛下固不之罪,亦不之行。自古甘蹈危亡
 之機,非獨暗主,而明君亦有焉,此臣之所甚懼。朝廷者,萬化之所自出也,實根於人君之一心。夫何大明當天,猶有可議者?內小學之建,人皆知陛下有意建儲也。然歲月逾邁,未睹施行,人心危疑,無所系屬。秦、漢而下,嗣不蚤定,事出倉卒,或宮闈出令,或宦寺主謀,或奸臣首議,此皆足以危人之國也。陛下何憚而不蚤定大計?邸第之盛,人皆知篤於親愛也。然依馮者眾,輕視王法,請托之行,捷於影響。楊干,晉侯弟也,亂行於曲梁,而魏絳
 戮其僕,晉侯始怒而終悔,晉卒以霸。平原君,趙王弟也,不出租稅,而趙奢刑其用事者,趙王賢而用之,趙卒以強。皆足以興人之國也。陛下何為而不少伸國法?今女冠者流,眾所指目;近璫小臣,時竊威福。此皆陛下之心乍明乍晦之所致,豈不謂之危乎?國有善類,猶人有元氣,善類一敗一消,元氣一病一衰。善類能幾,豈堪數消,沙極則國隨之矣。陛下明於知人,公於用人,固無權奸再用之意。然道路之人往往竊議,此元祐、紹聖將分之
 機也。禍根猶伏而未去,不幾於安其危乎?」帝改容曰:「朕決不用史嵩之。」



 遷知大宗正丞兼權金部郎官。李韶以病告,十上疏欲去。希得言:「韶有德望,雖以病告,曷若留奉內祠,侍經幄,亦足為朝廷重。」又言:「財用困竭,民生憔悴,移此不急之費,以實軍儲,以厚民生,敬天莫大於此,豈在崇大宮宇,莊嚴設像哉!」又條救錢楮三策,請置惠民局,帝皆以為可行。



 進秘書丞,尋遷著作郎,授江西提舉常平。役法久壞,臨川富室有賂吏求免者,希得竟罪
 之。遂提點刑獄,加直秘閣。未幾,加度支員外郎,尋直寶章閣,移治贛州。盜有偽號「崔太尉」者,據石壁,連結數郡;劉老龍等聚眾焚掠,一方繹騷。希得指授方略,不五旬平之。以直寶謨閣、廣西轉運判官兼權靜江府。尋授直徽猷閣、知靜江府、主管廣西經略安撫司公事兼轉運判官。母喪,免。召為秘書少監兼中書門下省檢正諸房公事。入對,言君子小人邪正之辯,且曰:「君子犯顏敢諫,拂陛下之意,退甘家食,此乃為國計,非為身計也。小人
 自植朋黨,擠排正人,甘言佞語,一切順陛下之意,遂取陛下官爵,此乃為身計,非為國計也。」遷宗正少卿兼國史編修、實錄檢討兼權給事中,兼權刑部侍郎、同修國史、實錄院同修撰。時西方用兵,有為嵩之復出計者,謂非此人不能辦。帝有意再用,知希得必執之,出旨諭意,希得毅然具疏密奏,不報。又繳鄧泳予祠之命。右正言邵澤、監察御史吳衍、殿中侍御史朱熠相繼論罷。



 久之,以集英殿修撰提點千秋鴻禧觀。未幾,依舊職兩淮宣
 撫使司判官,俄加寶謨閣待制,移京西、湖南北、四川。詔敘復元官。護江陵有功,召為戶部侍郎。帝曰:「姚希得才望可為閫帥。」乃進煥章閣待制、知慶元府、沿海制置使,繼升敷文閣待制。詔增沿海舟師,希得為之廣募水軍,造戰艦,蓄糧食,蠲米一萬二千石、舊逋一百萬。去官,庫餘羨悉以代民輸。召為工部尚書兼侍讀。入侍經筵,帝問慶元之政甚悉。以華文閣直學士、沿江制置使知建康府、江東安撫使、行宮留守。希得按行江上,慰勞士卒,
 眾皆歡說。溧陽饑,發稟勸分,全活者眾。創寧江軍,自建康、太平至池州列砦置屋二萬餘間,屯戍七千餘人。帝聞之,一再降詔獎諭。加寶章閣學士,尋加刑部尚書,依舊任兼淮西總領。



 景定五年,召為兵部尚書兼侍讀。乃言用人才、修政事、治兵甲、惜財用四事。拜端明殿學士、簽書樞密院事兼太子賓客。公星變,上疏引咎,乞解機務。兼權參知政事。度宗即位,授同知樞密院事兼權參知政事,尋授參知政事。以言罷,授資政殿學士、提舉洞
 霄宮。起知潭州、湖南安撫使,以疾甚,辭,乃仍舊職奉祠。請致仕,詔不許,力請,以資政殿大學士、金紫光祿大夫、依舊潼川郡公致仕。咸淳五年,卒。遺表聞,帝輟朝,贈少保。



 希得忠亮平實,清儉自將,好引善類,不要虛譽,蓋有誦薦於上而其人莫之知者。廣西官署以錦為帟幕,希得曰:「吾起身書生安用此!」命以繒纈易之。蜀之親族姻舊相依者數十家,希得廩之終身,昏喪悉損己力,晚年計口授田,各有差。所著有《續言行錄》、《奏稿》、《橘州文集》。



 包恢,字宏父,建昌人。自其父揚、世父約、叔父遜從朱熹、陸九淵學。恢少為諸父門人講《大學》,其言高明,諸父驚焉。嘉定十三年,舉進士。調金谿主簿。邵武守王遂闢光澤主簿,平寇亂。建寧守袁甫薦為府學教授,監虎翼軍,募土豪討唐石之寇。授掌故,改沿海制置司干官。會歲饑,盜起金壇、溧陽之間,恢部諸將為十誅夷之。沿江制置使陳韡闢為機宜,復有平寇功,改知吉州永豐縣,未行,差發運乾官。福建安撫使陳塏檄平寇,遷武學諭、宗
 正寺主簿,添差通判臺州。徐鹿卿討溫寇,闢兼提點刑獄司主管文字,議收捕。改通判臨安府,遷宗正寺主簿、知臺州。有妖僧居山中,號「活佛」,男女爭事之,因為奸利,豪貴風靡,恢誅其僧。



 進左司郎官,未行,改湖北提點刑獄,未行,移福建兼知建寧。閩俗以九月祠「五王」生日,靡金帛,傾市奉之。恢曰:「彼非犬豕,安得一日而五子同生,非不祥者乎?而尊畏之若是。」眾感悟,為之衰止。兼轉運判官,以侍御史周坦論罷。光州布衣陳景夏上書云:「包
 恢剛正不屈之臣,言者污蔑之耳。」又四年,起為廣東轉運判官,權經略使,遷侍右郎官,尋為大理少卿,即日除直顯文閣、浙西提點刑獄。是時海寇為亂,恢單車就道、調許、澉浦分屯建砦,一旦集諸軍討平之。嘉興吏因和糴受賕百萬,恢被旨慮囚,曰:「吾用此消沴氣。」乃減死,斷其手。



 進直龍圖閣、權發運,升秘閣修撰,知隆興府兼江西轉運。沈妖妓於水,化為狐,人皆神之。有母訴子者,年月後狀作「疏」字,恢疑之,呼其子至,泣不言。及得其情,母
 孀居,與僧通,惡其子諫,以不孝坐之,狀則僧為之也。因責子侍養跬步不離,僧無由至。母乃托夫諱日,入寺作佛事,以籠盛衣帛,因納僧於內以歸。恢知之,使人要之,置籠公庫,逾旬,吏報籠中臭達於外,恢命沉於江,語其子曰:「為汝除此害矣。」又姑死者假子婦棺以斂,家貧不能償,婦愬於恢,恢怒,買一棺,紿其婦臥棺中以試,就掩而葬之。改湖南轉運使,罷。



 景定初,拜大理卿、樞密都承旨兼侍講,權禮部侍郎,尋為中書舍人。林希逸奏恢
 守法奉公,其心如水。權刑部侍郎,進華文閣直學士、知平江府兼發運。豪有奪民包舉田寄公租誣上者,恢上疏,指為以小民祈天永命之一事,帝覽奏惻然,罪任事者,即歸民田。召赴闕,辭,改知紹興,又辭。度宗即位,召為刑部尚書,進端明殿學士,簽書樞密院事,封南城縣侯。郊祀禮成,還,以資政殿學士致仕。



 恢歷仕所至,破豪猾,去奸吏,治蠱獄,課盆鹽,理銀欠,政聲赫然。嘗因輪對曰:「此臣心惻隱所以深切為陛下告者,陛下惻隱之心如天
 地日月,其閉而食之者曰近習、曰外戚耳。」參知政事董槐見而嘆曰:「吾等有慚色矣。」他日講官因稱恢疏剴切,願容納。理宗欣然曰:「其言甚直,朕何嘗怒直言!」經筵奏對,誠實懇惻,至身心之要,未嘗不從容諄至。度宗至比恢為程顥、程頤。恢侍其父疾,滌濯拚除之役不命僮僕。年八十有七,臨終,舉盧懷慎臥簀窮約事戒諸子斂以深衣,作書別親戚而後卒,有光隕其地。遺表聞,帝輟朝,贈少保,謚文肅,賻銀絹五百。



 常挺字方叔,福州人。嘉熙二年進士。歷官為太學錄,召試館職,遷秘書省正字兼莊文府教授,升校書郎。輪對,乞以李若水配享高宗。改秘書郎兼考功郎官,出知衢州,拜監察御史兼崇政殿說書。疏言邊閫三事:曰闢實才,曰奏實功,曰招實兵。朝廷二事:曰選良吏,曰擢正人。又言:「願陛下深思宏遠之規模,奮發清明之志氣,立綱陳紀必為萬世之法程,昭德塞違以示百官之憲度。」遷太常少卿兼國子司業,兼國史編修、實錄檢討兼直舍
 人院。遷起居郎,權工部侍郎兼直學士院。遷工部侍郎、給事中。右諫議大夫陳堯道論罷。以寶章閣直學士知漳州,改知泉州,權兵部尚書兼侍讀,權禮部尚書兼同修國史、實錄院同修撰。進《帝學發題》,遷吏部尚書。咸淳三年,授同知樞密院事兼權參知政事,封合沙郡公,拜參知政事。四年,致仕,尋卒,贈少保。



 陳宗禮字立之。少貧力學,袁甫為江東提點刑獄,宗禮往問學焉。淳祐四年,舉進士。調邵武軍判官,入為國子
 正,遷太學博士、國子監丞,轉秘書省著作佐郎。入對,言火不循軌。帝以星變為憂,宗禮曰:「上天示戒,在陛下修德布政以回天意。」又曰:「天下方事於利欲之中,士大夫奔競趨利,惟至公可以遏之。」兼考功郎官,兼國史實錄院校勘,兼景獻府教授,升著作郎,遷尚左郎官兼右司。時丁大全擅國柄,以言為諱。宗禮嘆曰:「此可一日居乎!」陛對,言:「願為宗社大計,毋但為倉廩府庫之小計;願得天下四海之心,毋但得左右便嬖戚畹之心;願寄腹心
 於忠良,毋但寄耳目於卑近;願四通八達以來正人,毋但旁蹊曲逕類引貪濁。」拜太常少卿,以直寶謨閣、廣東提點刑獄進直煥章閣,遷秘書監。以監察御史虞慮言追兩官,送永州居住。



 景定四年,拜侍御史,直龍圖閣、淮西轉運判官,遷刑部尚書。以起居舍人曹孝慶言罷。度宗即位,兼侍講,拜殿中侍御史。疏言:「恭儉之德自上躬始,清白之規自宮禁始,左右之言利者必斥,蹊隧之私獻者必誅。」以《詩》進講,因奏:「帝王舉動,無微不顯,古人所
 以貴於慎獨也。」權禮部侍郎兼給事中。進讀《孝宗聖訓》,因奏:「安危治亂,常起於一念慮之間,念慮少差,禍亂隨見。天下之亂未有不起於微而成於著。」又言:「不以私意害公法,乃國家之福。」帝曰:「孝宗家法,惟賞善罰惡為尤謹。」宗禮言:「有功不賞,有罪不罰,雖堯舜不能治天下,信不可不謹也。」



 遷禮部侍郎,尋權禮部尚書,乞奉祠,帝曰:「豈朕不足與有為耶?」以華文閣直學士知隆興府,再辭,依舊職與待次差遣。逾年,依舊職廣東經略安撫使兼
 知廣州,加端明殿學士、簽書樞密院事,尋兼權參知政事。疏奏:「國所以立,曰天命人心。因其警而加敬畏,天命未有不可回也;因其未墜而加綏定,人心未嘗不可回也。」卒官,遺表上,贈開府儀同三司、盱江郡侯,謚文定。所著有《寄懷斐稿》、《曲轅散木集》、《兩朝奏議》、《經筵講義》、《經史明辨》、《經史管見》、《人物論》。



 常楙字長孺,顯謨閣直學士同之曾孫。入太學。淳祐七年舉進士。調常熟尉。公廉自持,不畏強御,部使者交薦
 之。調婺州推官。疏決滯訟,以剸繁裁劇稱。臨安府尹馬光祖又薦於朝,闢差平江府百萬倉檢察,不受和糴事例,戢吏卒苛取。發運使趙與TP兼提點刑獄,屬楙檢核,雪無錫翟氏冤獄。監江淮茶鹽所蕪湖局,不受商稅贏,光祖益敬之。改知嘉定縣。歲大水,勸分和糴,按籍均敷。發運使王爚、提點刑獄孫子秀俱特薦於朝,簽書臨安府判官,不為權勢撓。有為淮東提舉常平,闢楙提管,楙知其不可與共事,笑而卻之。未幾,政府強楙行,遂拂衣
 去,朝野高之。主管城南廂,聽訟嚴明,豪石益憚之。都城火後,瓦礫充斥,差民船徙運,在籍者百五十家,惟二十有五家應役,餘率為勢要宦官所庇。楙悉追之,不服者杖其人,械於他所,無不聽命。又力拒戶部科買。葉夢鼎、陳昉深期獎焉。添差臨安通判。朝命鞫封樁庫吏範成獄,不肯承廟堂風旨,無辜者悉出之。



 知廣德軍。郡有水災,發社倉粟以活饑民,官吏難之,楙先發而後請專命之罪,置慈幼局,立先賢祠。故事,郡守秋苗例可得米千
 石,乃以代屬縣償大農綱欠。拜監察御史,知無不言。嘗論天變及賈似道家爭田事,論繼皇子竑嗣,觸度宗怒,遷司農卿,尋為兩浙轉運使。禁戢吏奸,不以急符督常賦。海鹽歲為咸潮害稼,楙請於朝,捐金發粟,復輟己帑,大加修築新塘三千六百二十五丈,名曰海晏塘。是秋,風濤大作,塘不浸者尺許,民得奠居,歲復告稔,邑人德之。



 遷戶部侍郎。受四方民詞,務通下情。兼中書門下省檢正諸房公事,兼刑部侍郎。極論檢覆之敝。上進故事,
 首論雷雪非時之變,帝意不悅。丐祠,不許,以集英殿修撰知平江。值旱。故事,郡守合得緡錢十五萬,悉以為民食、軍餉助。蠲苗九萬、稅十三萬、版帳十六萬,又蠲新苗二萬八千,大寬公私之力。飛蝗幾及境,疾風飄入太湖。節浮費,修府庫。既代,有送還事例,自給吏卒外,餘金萬楮,楙悉不受。吏驚曰:「人言常侍郎不愛錢,果然。」改浙東安撫使。值水災,捐萬楮以振之,復請糴於朝,得米萬石,蠲新苗三萬八千。又以諸暨被水尤甚,給二萬楮付
 縣折運,民食不至乏絕。民各祀於家。兩浙及會稽、山陰死者暴露與貧而無以為殮者,乃以十萬楮置普惠庫,取息造棺以給之。尋以刑部侍郎召。申明期赦敘改法,與廟堂爭可否,辨偽關獄,救八倉虧欠免死罪,平反天井巷殺人獄,全活者甚眾。兼給事中,封還隆國夫人從子黃進觀察使錄黃。帝怒,似道以御書令委曲書行,楙迄不奉命。以寶章閣待制提舉太平興國宮。



 德祐元年,拜吏部尚書,以老病辭,累詔不許,專官趣行甚峻。楙入見,
 首言「霅川之變,非其本心,置之死,過矣,不與立後,又過矣。巴陵帝王之胄,生不得正命,死不得血食,沉冤幽憤,鬱結四五十年之久,不為妖為札於冥冥中者幾希。願陛下勿搖浮議,特發神斷,宗社幸甚」。於是詔國史院討論典故以聞。明堂禮成,進端明殿學士、提領戶部財用,特與執政恩數。楙以國步方艱,非臣子貪榮之時,力辭恩數。與廟堂議事不合,以疾謁告。二年春,拜參知政事,為夏士林繳駁,拜疏出關,後六年卒。



 家鉉翁,眉州人。以蔭補官。累官知常州,政譽翕然。遷浙東提點刑獄,入為大理少卿,直華文閣,以秘閣修撰充紹興府長史,遷樞密都丞旨,知建寧府兼福建轉運副使,權戶部侍郎兼知臨安府、浙西安撫使,遷戶部侍郎,權侍右侍郎,仍兼樞密都丞旨。賜進士出身,拜端明殿學士、簽書樞密院事。



 大元兵次近郊,丞相吳堅、賈餘慶檄告天下守令以城降,鉉翁獨不署。元帥遣使至,欲加縛,鉉翁曰:「中書省無縛執政之理。」堅奉表祈請於大元,
 以鉉翁介之,禮成不得命,留館中。聞宋亡,旦夕哭泣不食飲者數月。大元以其節高欲尊官之,以示南服。鉉翁義不二君,辭無詭對。宋三宮北還,鉉翁再率故臣迎謁,伏地流涕,頓首謝奉使無狀,不能感動上衷,無以保存其國。見者莫不嘆息。文天祥女弟坐兄故,系奚官,鉉翁傾橐中裝贖出之,以歸其兄璧。



 鉉翁狀貌奇偉,身長七尺,被服儼雅。其學邃於《春秋》,自號則堂,改館河間,乃以《春秋》教授弟子,數為諸生談宋故事及宋興亡之故,或
 流涕太息。大元成宗皇帝即位,放還,賜號「處士」,錫齎金幣,皆辭不受。又數年以壽終。



 李庭芝,字祥甫。其先汴人,十二世同居,號「義門李氏」,後徙隨之應山縣。金亡,襄、漢被兵,又徙隨。然特以武顯。



 庭芝生時,有芝產屋棟,鄉人聚觀,以為生男祥也,遂以名之。少穎異,日能誦數千言,而智識恆出長老之上。王旻守隨,庭芝年十八,告其諸父曰:「王公貪而不恤下,下多怨之,隨必亂,請徙家德安以避。」諸父勉強從之,未浹旬,
 旻果為部曲挾之以叛,隨民死者甚眾。嘉熙末,江防甚急,庭芝得鄉舉不行,以策干荊帥孟珙請自效。珙善相人,且夜夢車騎稱李尚書謁己,明日庭芝至。珙見其魁偉,顧諸子曰:「吾相人多,無如李生者,其名位當過我。」時四川有警,即以庭芝權施之建始縣。庭芝至,訓農治兵,選壯士雜官軍教之。期年,民皆知戰守,善馳逐,無事則植戈而耕,兵至則悉出而戰。夔帥下其法於所部行之。淳祐初始去,舉進士,中第。闢珙幕中,主管機宜文字。珙
 卒,遺表舉賈似道自代,而薦庭芝於似道,庭芝感珙知己,扶其柩葬之興國,即棄官歸,為珙行三年喪。



 似道鎮京湖,起為制置司參議,移鎮兩淮,與似道議柵清河五河口,增淮南烽百二十。繼知濠州,復城荊山以備淮南。皆切中機會。開慶元年,似道宣撫京湖,留庭芝權揚州。尋以大兵在蜀,奏知峽州,以防蜀江口。朝廷以趙與TP為淮南制置,李應庚為參議官。應庚發兩路兵城南城,大暑中暍死者數萬。李璮窺其無謀,奪漣水三城,渡淮
 奪南城。鄂兵解,庭芝丁母憂去。朝議擇守揚者,帝曰:「無如李庭芝。」乃奪情主管兩淮制置司事。庭芝再破璮兵,殺璮將厲元帥,夷南城而歸。明年,復敗璮於喬村,破東海、石圃等城。又明年,璮降,徙三城民於通、泰之間。又破蘄縣,殺守將。



 庭芝初至揚時,揚新遭火,廬舍盡毀。州賴鹽為利,而亭戶多亡去,公私蕭然。庭芝悉貸民負逋,假錢使為屋,屋成又免其假錢,凡一歲,官民居皆具。鑿河四十里入金沙餘慶場,以省車運。兼浚他運河,放亭戶
 負鹽二百餘萬。亭民無車運之勞,又得免所負,逃者皆來歸,鹽利大興。始,平山堂瞰揚城,大元兵至,則構望樓其上,張車弩以射城中。庭芝乃築大城包之,城中募汴南流民二萬人以實之,有詔命為武銳軍。又大修學,為詩書、俎豆,與士行習射禮。郡中有水旱,即命發廩,不足則以私財振之。揚民德之如父母。劉槃自淮南入朝,帝問淮事,槃對曰:「李庭芝老成謹重,軍民安之。今邊塵不驚,百度具舉,皆陛下委任得人之效也。」



 咸淳五年,北兵
 圍襄陽急,夏貴入援,大敗虎尾州;範文虎總諸兵再入,又敗,文虎以輕舠遁,兵亂,士卒溺漢水死者甚眾。冬,命庭芝以京湖制置大使督師援襄陽。文虎聞庭芝至,貽書似道曰:「吾將兵數萬入襄陽,一戰可平,但無使聽命於京閫,事成則功歸恩相矣。」似道喜,即除文虎福州觀察使,其兵從中制之。文虎日攜美妾,走馬擊球軍中為樂。庭芝屢欲進兵,曰:「吾取旨未至也。」明年六月,漢水溢,文虎不得已始一出師,未至鹿門,中道遁去。庭芝數自
 劾請代,不允,竟失襄陽。陳宜中請誅文虎,似道芘之,止降一官知安慶府,而貶庭芝及部將蘇劉義、範友信廣南。庭芝罷居京口。



 未幾,大元兵圍揚州,制置印應雷暴死,即起庭芝制置兩淮。庭芝請分淮西夏貴,而己得專力淮東,從之。十年,築清河口,詔以為清河軍。十二月,大元兵破鄂,詔天下勤王,庭芝首遣兵為諸道倡。德祐元年春,似道兵潰蕪湖,沿江諸郡或降或遁,無一人能守者。庭芝率所部郡縣城守。有李虎者持招降榜入揚州,
 庭芝誅虎,焚其榜。總制張俊出戰,持孟之縉書來招降,庭芝焚書,梟俊五人於市。而日調苗再成戰其南,許文德戰其北,姜才、施忠戰其中。時出金帛牛酒燕犒將士,人人為之死鬥。朝廷亦以督府金勞之,加庭芝參知政事。七月,以知樞密院事徵入朝,徙夏貴知揚州,貴不至,事遂已。



 十月,大元丞相伯顏入臨安,留元帥阿術軍鎮江以遏淮兵。阿術攻揚久不拔,乃築長圍困之。冬,城中食盡,死者滿道。明年二月,饑益甚,赴濠水死者日數百,
 道有死者,眾爭割啖之立盡。宋亡,謝太后及瀛國公為詔諭之降,庭芝登城曰:「奉詔守城,未聞有詔諭降也。」已而兩宮入朝,至瓜州,復詔庭芝曰:「比詔卿納款,日久未報,豈未悉吾意,尚欲固圉邪?今吾與嗣君既已臣伏,卿尚為誰守之?」庭芝不答,命發弩射使者,斃一人,餘皆退去。姜才出兵奪兩宮,不克,復閉城守。三月,夏貴以淮西降,阿術驅降兵至城下以示之,旌旗蔽野,幕客有以言覘庭芝者,庭芝曰:「吾惟一死而已。」阿術使者持詔來招
 降,庭芝開壁納使者,斬之,焚詔陴上。已而知淮安州許文德、知盱眙軍張思聰、知泗州劉興祖皆以糧盡降。庭芝猶括民間粟以給兵,粟盡,令官人出粟,粟又盡,令將校出粟,雜牛皮、曲蘗以給之。兵有烹子而食者,猶日出苦戰。七月,阿術請赦庭芝焚詔之罪,使之降,有詔從之。庭芝亦不納。是月,益王遣使以少保、左丞相召庭芝,庭芝以朱煥守揚,與姜才將兵七千人東入海,至泰州,阿術將兵追圍之。朱煥既以城降,驅庭芝將士妻子至泰
 州城下,陴將孫貴、胡惟孝等開門降。庭芝聞變,赴蓮池,水淺不得死。被執至揚州,朱煥請曰:「揚自用兵以來,積骸滿野,皆庭芝與才所為,不殺之何俟?」於是斬之。死之日,揚之民皆泣下。



 有宋應龍者為泰州咨議官,泰守孫良臣之弟舜臣自軍中來說降,良臣召應龍與計,應龍極陳國家恩澤,君臣大義,請殺舜臣以戒持二心者,良臣不得已殺之。及泰州降,應龍夫婦自經死。提刑司諮議褚一正置司高郵,督戰被創,沒水死。知興化縣胡拱
 辰,城破亦死。



 論曰:楊棟學本伊、洛,而尼於權臣,速謗召尤,誰之過歟?姚希得藹然君子。包恢以嚴為治,抑以衰世之民非可以縱馳待之耶?常挺、陳宗禮咸通濟,著聲望。常楙晚訟皇子竑事,光明正大,公義炳然。家鉉翁義不二君,足為臣軌。李庭芝死於國難,其可憫哉!



\end{pinyinscope}