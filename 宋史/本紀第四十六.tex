\article{本紀第四十六}

\begin{pinyinscope}

 度宗



 度宗
 端文明武景孝皇帝,諱祺,太祖十一世孫。父嗣榮王與芮,理宗母弟也。嘉熙四年四月九日生於紹興府榮邸。初,榮文恭王夫人全氏夢神言:「帝命汝孫,然非汝
 家所有。」嗣榮王夫人錢氏夢日光照東室,是夕,齊國夫人黃氏亦夢神人採衣擁一龍納懷中,已而有娠。及生,室有赤光。資識內慧,七歲始言,言必合度,理宗奇之。及在位歲久,無子,乃屬意托神器焉。淳祐六年十月己丑,賜名孟啟,以皇侄授貴州刺史,入內小學。七年正月乙卯,授宜州觀察使,就王邸訓習。九年正月乙巳,授慶遠軍節度使,封益國公。十一年正月壬戌,改賜名孜,進封建安郡王。寶祐元年正月庚辰,詔立為皇子,改賜今名。
 癸未,授崇慶軍節度使、開府儀同三司,進封永嘉郡王。二年七月,以宗正少卿蔡抗兼翊善。時資善堂初建,理宗制《堂記》,書以賜王。十月癸酉,進封忠王。十一月壬寅,加元服,賜字邦壽。五年十月庚子,授鎮南、遂安軍節度使。景定元年六月壬寅,立為皇太子,賜字長源,命楊棟、葉夢鼎為太子詹事。七月丁卯,太子入東宮。癸未,行冊禮。時理宗家教甚嚴,雞初鳴問安,再鳴回宮,三鳴往會議所參決庶事。退入講堂,講官講經,次講史,終日手不
 釋卷。將晡,復至榻前起居,率為常。理宗問今日講何經,答之是,則賜坐賜茶;否,則為之反復剖析;又不通,則繼以怒,明日須更覆講。二年正月丁丑,謁孔子於太學,請以張栻、呂祖謙列從祀。十二月癸卯,冊永嘉郡夫人全氏為皇太子妃。



 五年十月丁卯,理宗崩。受遺詔,太子即皇帝位。戊辰,尊皇后謝氏曰皇太后,生日為壽崇節。庚午,宰執、文武百官詣祥曦殿表請聽政,不允。辛未,大赦。



 十一月壬申,宰執以下日表請視朝,不允。丁丑,凡七表,
 始從。丙戌,帝初聽政,御後殿,命馬廷鸞、留夢炎兼侍讀,李伯玉、陳宗禮、範東叟兼侍講,何基、徐幾兼崇政殿說書。詔求直言。又詔先朝舊臣趙葵、謝方叔、程元鳳、馬光祖、李曾伯各上言以匡不逮。召江萬里、王□龠、洪天錫、湯漢等赴闕。詔躬行三年喪。復濟王竑元贈少師、節度使,追封鎮王,謚昭肅,有司討論墳制增修之。加封嗣榮王與芮武康、寧江軍節度使,依前太師、判宗正事。詔撫勞邊防將士。監察御史劾宦官李忠輔、何舜卿等贓罪,並
 竄遠方。戊戌,詔儒臣日侍經筵,輔臣觀講。乙未,命洪天錫以侍御史兼侍讀。十二月辛丑,詔改明年為咸淳元年,行銅錢關子,率貫以七百七十文足。壬寅,戒贓吏絕貢羨餘。甲辰,詔以生日為乾會節。初開經筵,講殿以熙明為名。禮部尚書馬廷鸞進讀《大學衍義序》,陳心法之要。是歲,兩浙、江東西、湖南北、廣東西、福建、成都、京西、潼川、夔、利路戶五百六十九萬六千九百八十九,口一千三百二萬六千五百三十二。大理寺奏大闢三十三人。



 咸淳元年春正月辛未朔,日有食之。丞相賈似道請為總護山陵使,不允,尋下詔獎諭。癸酉,直學士院留夢炎疏留似道。甲戌,諫議大夫朱貔孫等亦請改命,不報。詔臨安免征商三月。丙子,京湖制置使呂文德辭免,不允。



 二月庚申,置籍中書,記諫官、御史言事,歲終以考成績。



 三月癸酉,似道乞解機政,不允。壬午,京湖制司創招鎮邊軍。甲申,葬理宗於永穆陵。夏四月壬寅,賞四川都統昝萬壽云頂山、金堂峽之功,及其將士。丁未,壽崇節,免
 征臨安官私房僦地錢。戊申,乾會節,如上免征,再免在京征商三月。自是祥慶、災異、寒暑皆免。戊午,賈似道特授太師。己未,幸景靈宮,發米八萬石贍京城民。夔路都統王勝以李市、沙平之戰獲功,轉官兩資,將士效力者,上其名推賞。



 五月己巳,追命史彌遠為公忠翊運定策元勛。



 閏月乙巳,久雨,京城減直糶米三萬石。自是米價高即發廩平糶,以為常。丁未,發錢二十萬贍在京小民,錢二十萬賜殿、步、馬司軍人,錢二萬三千賜宿衛。自是
 行慶、恤災,或遇霪雨雪寒,咸賜如上數。以江萬里參知政事,王□龠同知樞密院事、權參知政事,馬廷鸞端明殿學士、簽書樞密院事。丁巳,以錢三十萬命臨安府通變平物賈。丁卯,故成都馬步軍總管張順歿於王事,詔特贈官五轉,其子與八官恩澤。



 六月乙酉,名理宗御制之閣曰顯文,置學士、直學士、待制、直閣等官。戊子,沿海制置使葉夢鼎三辭免,不允。己丑,名理宗原廟殿曰章熙。



 秋七月丁酉,太白晝見。初命迪功郎鄧道為韶州相江
 書院山長,主祀先儒周惇頤。壬寅,參知政事江萬里乞歸田里,不允。戊申,夔路安撫徐宗武城開、達石城,乞推恩,從之。壬戌,督州縣嚴錢法,禁民間用牌帖。癸亥,以諒陰,命宰執類試,阮登炳以下,依廷試例出身。禁在京置窠柵、私系囚。



 八月庚辰,命陳奕沿江按閱軍防,賜錢二十萬給用。丁亥,詔有司收民田租,或掊克無藝,監司其嚴禁戢,違者有刑。甲午,大元元帥阿術帥大軍至廬州及安慶,諸路統制範勝、統領張林、正將高興、副將孟
 興逆戰,沒於陣。詔勝等各官其一子進勇副尉。



 九月己酉,以洪天錫為工部侍郎兼侍讀。壬子,命宰執訪司馬光、蘇軾、朱熹後人,賢者能者,各上其名錄用。癸丑,呂文德言京湖制、帥、策應三司官屬,乞推恩。詔各進一秩。庚申,吏部侍郎李常上七事,曰崇廉恥、嚴鄉舉、擇守令、黜貪污、讞疑獄、任儒師、修役法。



 冬十月壬申,減四川州縣鹽酒課,始自景定四年正月一日,再免征三年。乙亥,減田契稅錢什四。庚辰,江安州、潼川安撫司以攻懷、簡小
 富砦戰圖來上,詔優答以賞。



 十一月乙未,兄少保、保寧軍節度使致仕乃裕薨,贈少傅,追封臨川郡王。



 二年春正月癸丑,江萬里四請歸田、乞祠祿,不允,以為湖南安撫使兼知潭州。



 二月乙巳,侍講範東叟奏正心之要有三:曰進德,曰立政,曰事天。上嘉納焉。戊寅,詔免湖南漕司積年運上峽米耗折逋直。辛卯,詔左、右史循舊制立侍御坐前。



 三月庚子,賞夔路總管張喜等防護開、達軍功,將士進官有差。乙巳,詔郡守兩年為任,方別
 授官。戊申,賜敕書獎諭呂文德。



 夏四月乙丑,洪天錫三請祠,不允,以顯文閣待制知潭州兼湖南安撫使。甲申,侍御史程元嶽上言:「帝王致壽之道在修德,後世怵邪說以求之,往轍可鑒。修德之目有三,曰清心,曰寡欲,曰崇儉,皆致壽之原。」上嘉納之。丁亥,授信州布衣徐直方史館編校。



 五月癸丑,詔諸節制將帥討軍實,節浮費,毋占役兵士,致妨訓練。



 六月丁丑,給羅鬼國化州印。壬午,以衢州饑,命守、令勸分諸藩邸發廩助之。



 秋七月壬辰,
 祈雨,詔以來年正月一日郊。壬寅,禮部侍郎李伯玉言:「人材貴乎善養,不貴速成,請罷童子科,息奔競,以保幼稚良心。」詔自咸淳三年為始罷之。



 八月甲申,安南國遣使賀登位,獻方物。



 九月丙辰,浙西安撫使李芾以臺臣黃萬石等言,削兩秩免。冬十一月辛丑,兩淮制置使李庭芝立城,屯駐武銳一軍,以工役費用及圖來上。詔獎勞之。乙卯,少師致仕趙葵薨,贈太傅,賜謚忠靖。丁巳,利東安撫使、知合州張玨調統制史照、監軍王世昌等復
 廣安大梁城,詔推爵賞有差。



 十二月丁丑,申嚴戢貪之令。甲申,以請先帝謚祭告天地、宗廟、社稷。丙戌,奉冊寶請於南郊,上謚曰建道備德大功復興烈文仁武聖明安孝皇帝,廟號理宗。大理寺奏歲終大闢古十五人。



 三年春正月己丑朔,郊,大赦。丁酉,奉皇太后寶,上尊號曰壽和。辛丑,壽和太后冊、寶禮成,謝堂等二十七人各進一秩,高平郡夫人謝氏等二十二人各進封、特封有差。癸卯,冊命妃全氏為皇后。戊申,帝詣太學謁孔子,
 行舍菜禮,以顏淵、曾參、孔伋、孟軻配享,顓孫師升十哲,邵雍、司馬光升列從祀,雍封新安伯。禮部尚書陳宗禮、國子祭酒陳宜中進讀《中庸》。己酉,執經官宗禮、講經官宜中各進一秩,宜中賜紫章服。太學、武學、宗學、國子學、宗正寺官若醫官、監書庫、門、庖等,各進一秩,諸齋長諭及起居學生推恩有差。乙卯,壽和太后親屬謝奕修、郭自中、黃興在等二十八人各升補一秩。



 二月己未,克復廣安軍,詔改為寧西軍。庚申,馬光祖再乞致仕,不允。乙丑,
 詔賈似道太師、平章軍國重事,一月三赴經筵,三日一朝,治事都堂。丙子,樞密院言:知夔州、夔路安撫徐宗武創立臥龍山堡囿。詔宗武帶行遙郡團練使,以旌其勞。



 三月癸卯,知房州李鑒及將校杜汝隆、夏喜以戰龍光砦有功,優與旌賞。



 夏四月庚申,壽和太后兩次冊、寶,族兄弟謝奕實等十五人、族侄謝在達等四十七人、族侄孫謝鏞等十四人各錫銀十兩、帛十匹。詔:太中大夫全清夫儒科發身,懇陳換班,靖退可尚,特授清遠軍承宣
 使、提舉祐神觀,仍奉朝請。乙酉,張玨護合州春耕,戰款龍溪,以狀言功,詔趣上立功將士姓名。



 五月丁亥朔,日有食之。戊申,詔曰:「比常命有司按月給百官奉,惟官愈卑,去民愈親,仍聞過期弗予,是吏奉吾命不虔也,諸路監司其嚴糾劾。」六月壬戌,加授呂文德少傅,馬光祖參知政事,李庭芝兵部尚書,並職任仍舊。皇后受冊推恩,弟全清夫以下十五人官一轉,全必□□以下十七人補承信郎。癸酉,美人楊氏進封淑妃。戊寅,詔榮王族姻與
 萊等三十四人各轉官有差。



 秋七月丁亥,張玨授正任團練使、帶行左領軍衛大將軍,賜金帶。壬辰,樞密院言:「右武大夫、權鄂州都統汪政鄂城戰御,又焚光化城外積聚,及攻真陽城,皆有功,該轉十二官。」詔轉橫行遙郡。甲午,四川都統昝萬壽調統制趙寶、楊立等率舟師護糧達渠城,以功推賞。己酉,權黎州張午招諭大青羌主歸義,乞用兩林西蕃瑜林例,賜予加優,從之。



 八月辛酉,遣步帥陳奕率馬軍舟師巡邏江防。壬戌,邊報警急,詔
 諭呂文德等申嚴防遏。乙丑,太師、武康、寧江軍節度使、判大宗正事嗣榮王與芮進封福王,主榮王祀事。壬申,久雨,命在京三獄、赤縣、直司、簽廳擇官審決獄訟毋滯。



 九月乙未,詔郡縣折收民田租,毋厚直取贏,違者論罪。癸卯,知邕州總統譚淵、李旺、周勝等繇特磨行大理界,率兵攻建水州,禽其知州阿孱以下三百餘人,獲馬二百餘,焚穀米、器甲、廬舍。師還論功,各轉官三資,軍校補轉有差。



 冬十月庚申,復開州,賜四川策應司錢百萬勞
 軍。甲戌,大雷電。



 十一月丙申,故左丞相吳潛追復光祿大夫。壬寅,賞知房州李鑒調遣路將夏喜、統領馮興等均州武陽壩戰功。十二月丙辰,呂文煥依舊帶行御器械,改知襄陽府兼京西安撫副使,丁卯,臺臣言敘復元官觀文殿學士、提舉洞霄宮皮龍榮貪私傾險,嘗朋附丁大全,乞寢新命。詔予祠祿。



 四年春正月癸未,賜呂師夔紫章服、金帶。己丑,呂文德言知襄陽府兼京西安撫副使呂文煥、荊鄂都統制唐
 永堅蠟書報白河口、萬山、鹿門山北帥興築城堡,檄知郢州翟貴、兩淮都統張世傑申嚴備御。癸巳,故守合州王堅賜廟額曰「報忠」。癸卯,沔州駐扎、潼川安撫副使昝萬壽特升右武大夫、帶行左驍衛大將軍,賜金帶。己酉,印應雷改知慶元府兼沿海制置使。庚戌,詔曰:「邇年近臣無謂引去以為高,勉留再三,弗近益遠,往往相尚,不知其非義也。亦由一二大臣嘗勇去以為眾望,相踵至今。孟子於齊王不遇,故去,是未嘗有君臣之情也,然猶
 三宿出晝,庶幾改之。儒者家法,無亦取此乎。朕於諸賢,允謂無負,其弗高尚,使人疑於負朕。」閏月庚午,賜夏貴金帶。



 夏四月壬午,湯漢三辭免刑部侍郎、福建安撫使。庚寅,乾會節,帝御紫宸殿,群臣稱賀。上曰:「謝方叔托名進香,擅進金器諸物,且以先帝手澤,每系之跋,率多包藏,至以先帝行事為己功,殊失大臣體,宜鐫一秩。」於是盧鉞等相繼論列方叔昨蜀、廣敗事,誤國殄民,今又違制擅進,削一秩罰輕。詔削四秩,奪觀文殿大學士、惠國
 公,罷宰臣恩數,仍追《寶奎錄》並系跋真本來上。丙申,右正言黃鏞言:「今守邊急務,非兵農合一不可。一曰屯田,二曰民兵。川蜀屯田為先,民兵次之,淮、襄民兵為先,屯田次之,此足食足兵良策也。」不報。丁酉,詔故修武郎姚濟死節,立廟,賜額曰「忠壯」。



 五月辛酉,樞密都承旨高達再辭侍衛都虞候,乞歸田里,命孫虎臣代之。壬申,賜陳文龍以下六百六十四人進士及第、出身。丙子,賈似道乞骸骨,不允。



 六月辛巳,葉夢鼎再乞歸田里,不允。詔罷
 浙西諸州公田莊官,募民自耕輸租,租減什三,毋私相易田,違制以盜賣官田論。



 秋七月戊午,有星出氐宿,西北急流入騎官星沒。己未,淑妃楊氏親屬楊幼節以下百三十四人推恩進秩。



 八月壬寅,奉安《寧宗實錄》、《理宗實錄》、《御集》、《日歷》、《會要》、《玉牒》、《經武要略》、《咸淳日歷》、《玉牒》,賈似道、葉夢鼎、馬廷鸞各補轉兩官,諸局官若吏推恩有差。



 九月癸未,太白晝見。大元兵築白河城,始圍襄、樊。



 冬十月戊寅朔,日有食之。子憲生。參知政事常挺六乞歸
 田里,詔予郡。己亥,已減四川州縣鹽酒課,詔自咸淳四年始,再免征三年。



 十一月癸丑,樞密院言:「南平、紹慶六郡鎮撫使韓宣城渝、嘉、開、達、常、武諸州有勞,繇峽州至江陵水陸措置,盡瘁以死,宜視沒於王事加恩。」詔宣守本官致仕,任一子承節郎,仍贈正任承宣使。丁巳,詔知江陵府陳奕,裨將周全、王德等戰西山、南谷口、田家山有功,各以等第推賞。戊午,子鍠生。丙寅,福建安撫使湯漢再辭免,乞祠祿,詔別授職。辛未,以文武官在選,困於
 部吏,隆寒旅瑣可閔。詔吏部長貳、郎官日趣銓注,小有未備,特與放行,違者有刑。自是隆寒盛暑,申嚴誡飭。常挺卒,贈少保。壬申,行義役法。十二月辛卯,以夏貴為沿江制置副使兼知黃州。癸巳,史館狀《理宗實錄》接續起修。張九成、孫象先力學飭行,不墜家聲,其免一解示表厲。命建康府建南軒書院,祠先儒張栻,戊戌,汪立信知潭州兼湖南安撫使,職任依舊。乙巳,詔賞京湖總管張喜、趙萬等石門板堰戰功。



 五年春正月丁未,以李庭芝為兩淮安撫制置大使兼知揚州。壬子,京湖策應司參謀呼延德領諸將張喜等遇北兵,戰於蠻河。癸亥,葉夢鼎累章請老,留之,固辭,依前少保、判福州、福建安撫使,封信國公。以馬廷鸞參知政事兼同知樞密院事。甲戌,以江萬里參知政事。



 二月戊子,江萬里辭免參知政事,不允。



 三月丙午,北帥阿術自白河以兵圍樊城。甲寅,葉夢鼎辭免判福州、福建安撫使,詔不允。乙卯,皇后歸寧,族姻推恩,保信軍節度使
 全清夫以下五十六人各進一秩,咸安郡夫人全氏以下三十二人各特封有差。大元兵城鹿門。己未,詔浙西六郡公田設官督租有差。辛酉,京湖都統張世傑率馬步舟師援襄、樊,戰於赤灘圃。戊辰,以江萬里為左丞相,馬廷鸞為右丞相兼樞密使。己巳,以馬光祖知樞密院事兼參知政事,吳革沿江制置使。



 夏四月丙子,賞張世傑戰功。辛巳,江萬里、馬廷鸞辭免,詔不允。壬午,知渠州張資上蓬州界白土、神山、蒲渡等處今年春戰功。丙戌,
 以安西都統張朝寶、利東路安撫張玨領兵護錢粟餉寧西軍,還至水磑頭,戰有功,詔推賞。己丑,劉雄飛依舊樞密都承旨、知沅州兼常德、澧、辰、沅、靖五郡鎮撫使。癸巳,李庭芝特進一秩。高郵縣夏世賢七世義居,詔署其門。



 五月己酉,馬光祖依舊觀文殿學士、提舉洞霄宮。乙卯,程元鳳薨,贈少師。庚申,有星自斗宿距星東北急流向牛宿,至濁沒。壬戌,詔:信陽諸將婁安邦、朱興戰千石畈,呂文煥、呼延德戰福山,楊青、李忠戰石湫,俱有勞效,
 推賞有差。壬申,京湖制司言:故夔路安撫徐宗武沒於王事,乞優加贈恤。詔致仕恩外,特官其一子承節郎。



 六月庚辰,以呂文福為復州團練使、知濠州兼淮西安撫副使。甲申,皇子是生。辛卯,家鉉翁辭免新命,詔別授職。庚子,李庭芝辭免兼淮東提舉,不允。



 秋七月己酉,觀文殿學士馬光祖乞守本官致仕,詔允所請。庚申,祈雨。壬戌,東南有星自河鼓距星西北急流,至濁沒。



 八月戊寅,詔郡縣收民田租,毋巧計取贏,毋厚直折納,轉運司申
 嚴按劾。詔襄、樊將士戰御宣力,以錢二百萬犒師,趣上其立功姓名補轉官資。



 九月丙午,祈晴。辛酉,祀明堂,大赦。丙寅,明堂禮成,加上壽和聖福皇太后尊號冊、寶,太師、判大宗正事、福王、榮王祀事與芮加食邑一千戶。



 冬十月甲申,子憲授檢校太尉、武安軍節度使,封益國公。己丑,呂文德進封崇國公,加食邑七百戶。以湯漢為顯文閣直學士、提舉玉隆萬壽宮兼象山書院山長。



 十一月戊辰,少傅文德乞致仕,詔特授少師,進封衛國公,
 依所請致仕。十二月癸酉,文德卒,贈太傅,賜謚武忠。己卯,以範文虎為殿前副都指揮使。壽和聖福皇太后尊號冊寶禮成,侄謝堂、侄孫光孫等二十八人各轉一官,餘姻推恩有差。甲申,以錢二百萬命京湖帥臣給犒襄、郢等處水陸戍士。戊子,詔安南國王父陳日煚、國王陳威晃並加食邑一千戶。大元兵築南新城。



 六年春正月壬寅,以李庭芝為京湖安撫制置使兼夔路策應使,印應雷兩淮安撫制置使。己酉,以錢二百萬
 賜夔路策應司備禦賞給。庚戌,以高達為湖北安撫使、知鄂州,孫虎臣起復淮東安撫副使、知淮安州。辛酉,行《成天歷》,丁卯,上制《字民》、《牧民》二訓,以戒百官。戊辰,以江萬里為福建安撫使。



 二月辛未,檢校少保、安德軍節度使與萊加食邑五百戶。丁亥,陳宜中經筵進講《春秋》終篇,賜象簡、金禦仙花帶、鞍馬。丁酉,以呂文福為淮西安撫副使兼知廬州。己亥,朱祀孫權兵部尚書,仍四川安撫制置、總領夔路轉運、知重慶府。



 三月庚子朔,日有食
 之。癸丑,詔曰:「吏以廉稱,自古有之,今絕不聞,豈不自章顯而壅於上聞歟?其令侍從、卿監、郎官,各舉廉吏,將顯擢焉。」癸亥,詔:「贛、吉、南安境數被寇,雖有砦卒,寇出沒無時,莫能相救。宜即要沖立四砦,砦屯兵百,使地勢聯絡,御寇為便。從三郡擇將官領之。」夏四月戊寅,以文天祥兼崇政殿說書。



 五月辛丑,以吳革為沿江制置宣撫使。



 六月庚午,詔《太極圖說》、《西銘》、《易傳序》、《春秋傳序》,天下士子宜肄其文。戊寅,賈似道托疾退辭,疏十數上,上留益
 堅,禮異之,曰師相而不名。馬廷鸞洎省、部、臺諫、學館、諸司,連章請留似道。庚辰,子憲薨。庚寅,詔以襄、郢水陸屯戍將士隆暑露處,出錢二百萬,命京湖制司給賜。



 秋七月,復開州。己亥,更鑄印給之。



 八月甲申,瑞安府樂清縣嘉禾生,詔薦士增四名。壬辰,詔:郡縣行推排法,虛加寡弱戶田租,害民為甚。其令各路監司詢訪,亟除其弊。詔精擇監司、守令,監司察郡守,郡守察縣令,置籍考核,歲終第其治狀來上。癸巳,以夏貴能舉職事,進一秩。詔似
 道十日一朝。



 九月庚戌,以黃萬石為沿海制置使。壬子,臺州大水。



 冬十月丁丑,遣範文虎總統殿司、兩淮諸軍,往襄、樊會合備御,賜錢百五十萬犒師。己卯,詔臺州發義倉米四千石並發豐儲倉米三萬石,振遭水家。甲申,以陳宗禮、趙順孫兼權參知政事,依舊同提舉編修敕令、《經武要略》。閏十月己酉,安吉州水,免公田租四萬四千八十石。戊午,詔殿、步、馬諸軍貧乏陣沒孤遺者多,方此隆寒,其賜錢二十萬、米萬石振之。



 十一月丁丑,嘉興、
 華亭兩縣水,免公田租五萬一千石,民田租四千八百一十石。庚辰,詔襄、郢屯戍將士隆寒可閔,其賜錢二百萬犒師。己丑,都統張世傑領兵江防。乙未,詔陳宗禮進一秩,為資政殿學士,依所請守兼參知政事致仕。十二月戊戌,陳宗禮卒,贈七秩。己亥,詔唐全、張興祖等繼蠟書入襄陽,往復甚艱,各補轉三官,賜錢二千緡。大元兵築萬山城。



 七年春正月乙丑,子是授左衛上將軍,進封建國公。詔
 湯漢、洪天錫赴闕。詔戒貪吏。辛未,紹興府諸暨縣湖田水,免租二千八百石有奇。



 三月戊寅,發屯田租穀十萬石,振和州、無為、鎮巢、安慶諸州饑。辛巳,日暈,赤黃周匝。乙酉,平江府饑,發官倉米六萬石。吉州饑,發和糴米十萬石,皆減直振糶。丙戌,詔減內外百司吏額。戊子,發米一萬石,往建德府濟糶。詔臨江軍宣聖四十七代孫延之子孫,與放國子監試。



 夏四月辛亥,免廣東提舉司鹽籮銀三萬兩。甲寅,禮部侍郎陳宜中再乞補外,以顯文
 閣待制出知福州兼福建路安撫使。



 五月乙酉,賜禮部進士張鎮孫以下五百二人及第、出身。壬辰,發米二萬石,詣衢州振糶。



 六月癸巳,以錢百萬、銀五千兩命知嘉定府昝萬壽修城浚壕,繕甲兵,備御遏。以韓震帶行御器械、知江安州兼潼川東路安撫副使,馬坤帶行御器械、知咸淳府、節制涪、萬州。臺臣劾朱善孫督綱運受贓四萬五千,詔特貸死,配三千里,禁錮不赦。乙未,詔以蜀閫調度浩繁,賜錢二百萬給用。丙申,諸暨縣大雨、暴風、
 雷電,發米振遭水家。瑞州民及流徙者饑,乏食,發義倉米一萬八千石,減直振糶。己亥,詔以陸九淵孫溥補上州文學。己酉,鎮江府轉輸米十萬石於五河新城積貯。癸丑,以隆暑,給錢二百萬賜襄、郢屯戍將士。丙辰,撫州黃震言:「本州振荒勸分,前谷城縣尉饒立積米二百萬,靳不發廩,雖嘗監貸,宜正遏糴之罪。」詔饒立削兩秩、武岡軍居住。洪天錫三辭召命,詔守臣勉諭赴闕。戊午,紹興府饑,振糧萬石。己未,兩淮
 五河築城具完,賜名安淮軍。大元會兵圍襄陽。



 秋七月辛未,樞密院言吳信、周旺繼蠟書入襄城,往復效勞,詔各補官三轉。丁丑,湖南轉運司訪求先儒張栻後人義倫以聞,詔補將仕郎。壬午,四川制置使朱祀孫言:「夏五以來,江水凡三泛溢,自嘉而渝,漂蕩城壁,樓櫓圮壞。又嘉定地震者再,被災害為甚,乞賜黜罷,上答天譴。」詔不允。癸未,詔:城五河,淮東制置印應雷具有勞績,進一秩,宣勞官屬將士皆推恩。



 八月壬辰朔,日有食之。甲午,以錢三百萬,遣京湖制置李
 庭芝詣郢州調遣犒師。丁未,命沿江制置副使夏貴會合策應,以錢二百萬隨軍給用。



 九月乙亥,顯文閣直學士湯漢、顯文閣直學士洪天錫各五辭召命,詔並升華文閣學士,仍予祠祿。己丑,子□生。



 冬十月丙申,少傅、嗣秀王與澤薨,詔贈少師,追封臨海郡王。癸丑,從政郎朱鑒孫進《群經要略》。己未,詔殿、步、馬諸軍貧乏陣沒孤遺者,方此隆寒,其賜錢二十萬、米萬石振之。



 十一月癸亥,詔民有以孝弟聞於鄉者,守、令其具名上聞,將旌異勞
 賜焉。己已。詔湯漢官一轉,端明殿學士,依所請致仕。十二月甲午,詔諸路監司循按刑獄,傔從擾民,御史臺申嚴覺察。丙午,以錢三十萬命四川制司下渠、洋、開州、寧西鎮撫使張朝寶創司犒師。己亥,淮東統領兼知鎮江府趙溍乞祠祿,不允。謝方叔特敘復元官職、惠國公致仕。辛亥,初置士籍。戊午,詔舉廉能材堪縣令者,侍從、臺諫、給舍各舉十人,卿監、郎官各舉五人,制帥、監司各舉六人,知州、軍、監各舉二
 人。



 八年春正月庚申,詔:「朕惟崇儉必自宮禁始,自今宮禁敢以珠翠銷金為首飾服用,必罰無貸。臣庶之家,咸宜體悉。工匠犯者,亦如景祐制,必從重典。」又詔:「有虞之世,三載考績,三考黜陟幽明。漢之為吏者長子孫,則其遺意也。比年吏習偷薄,人懷一切,計日待遷,事未克究,又望而之他。吏胥狎玩,竊弄官政,吾民奚賴焉?繼自今內之郎曹,外之牧守以上,更不數易,其有治狀昭著,自宜獎異。」辛未,子昺生。己丑,湯漢卒,賜謚文清。



 二月癸巳,謝
 方叔卒,贈少師。前知臺州趙子寅歿,無所歸,特贈直秘閣,給沒官宅一區、田三百畝,養其孤遺,以旌廉吏。丙午,以錢二百萬給犒襄、郢水陸戰戍將士。



 三月丙子,同知樞密院事兼權參知政事趙順孫授中大夫。



 夏四月戊子,知合州、利路安撫張玨創築宜勝山城。



 五月己巳,王□龠除觀文殿學士、提舉萬壽觀兼侍讀。大元兵久圍襄、樊,援兵厄關險,不克進。詔荊、襄將帥移駐新郢,遣部轄張順、張貴將死士三千人自上流夜半輕舟轉戰。比明
 達襄城,收軍閱視,失張順。



 六月丙申,皮龍榮徙衡州。丁酉,以章鑒為端明殿學士、同簽書樞密院事、同提舉《經武要略》。以錢千萬命京湖制司糴米百萬石,轉輸襄陽府積貯。乙巳,以家鉉翁兼權知紹興府、浙東安撫、提舉司事,以唐震為浙西提點刑獄。王□龠乞寢新命,不允,勉諭赴闕。辛亥,臺臣言江西推排田結局已久,舊設都官、團長等虛名尚在,占吝常役,為害無窮,又言廣東運司銀場病民。詔俱罷之。癸丑,以錢五百萬命四川制司
 詣湖北糴運上峽入夔米五十萬石。秋七月辛未,知靜江府、廣西經略安撫使兼計度轉運使胡穎乞祠祿,詔勛一轉,依所乞宮觀。



 八月丙戌朔,日有食之。辛丑,詔家鉉翁赴闕。丁未,紹興府六邑水,發米振遭水家。壬子,王□龠辭免明堂大禮陪祠。乙卯,詔福建安撫陳宜中克舉厥職,升寶謨閣待制。



 九月丁卯,詔洪天錫轉端明殿學士,允所請致仕,辛未,明堂禮成,祀景靈宮。還遇大雨,改乘逍遙輦入和寧門,肆赦。庚辰,詔以朱祀孫兼四川屯
 田使。乙酉,洪天錫卒,贈五官,謚文毅。



 冬十月己亥,紹興府言八月一日,會稽、餘姚、上虞、諸暨、蕭山五縣大水,詔減田租有差。丁未,以章鑒兼權參知政事。右丞相馬廷鸞十疏乞骸骨,詔不允。庚戌,以秋雨水溢,詔減錢塘、仁和兩縣民田租什二,會稽湖田租什三,諸暨湖田租盡除之。辛亥,陳宜中兼給事中。



 十一月乙卯,右丞相馬廷鸞累疏乞骸骨,授觀文殿學士、知饒州。詔以隆寒,殿、步、馬司諸軍貧窶並陣沒孤遺者,振以錢粟。丙辰,陳奕以
 殿前都指揮使攝侍衛步軍司、馬軍司。己未,馬廷鸞辭免知饒州,乞祠祿。詔以所請,以觀文殿大學士、鄱陽郡公提舉洞霄宮。壬戌,命阮思聰赴樞密院廩議。己巳,詔明堂禮成,安南國王陳日煚、陳威晃各加食邑一千戶,賜鞭、鞍、馬等物。十二月甲寅,以葉夢鼎為少傅、右丞相兼樞密使。



 九年春正月乙丑,樊城破,範天順、牛富死之。癸未,詔定安豐統制金文彪、朱文廣、王文顯、盛全水戎河、古河、泉河、
 鈱河等處戰功行賞。



 二月甲申,詔鄂州左水軍統制張順沒身戰陣,贈寧遠軍承宣使,官其二子承信郎,立廟京湖,賜額曰忠顯。甲午,朱祀孫撫綏備御,義不辭難,敕書獎諭。丁未,以夏貴檢校少保。庚戌,呂文煥以襄陽府歸大元。癸丑,以朱澗寺戰功,推賞來歸人馬宣、沿江都統王喜等將士千五百七十餘人。



 三月庚申,賈似道言邊遽日聞,請督師以勵將帥。詔不允。四川制司言:「近出師成都,劉整故吏羅鑒自北復還,上整書稿一帙,有
 取江南二策:其一曰先取全蜀,蜀平,江南可定;其二曰清口、桃源,河、淮要沖,宜先城其地,屯山東軍以圖進取。」帝覽奏,亟詔淮東制司往清口,擇利城築以備之。葉夢鼎辭免右丞相,詔不允。庚午,遣金吾衛上將軍阮思聰由平江、鎮江及黃州行視城池,凡合繕修增易者亟條奏。丙子,來歸人方德秀補成忠郎,慄勇、楊林、胡巨川補保義郎,劉全補承信郎。戊寅,賈似道始奏李庭芝表言襄帥呂文煥以城降大元。己卯,加昝萬壽寧遠軍承
 宣使、職任仍舊。庚辰,夏貴辭免檢校少保,不允。壬午,詔建機速房,以革樞密院漏洩兵事、稽違邊報之弊。賈似道累疏請身督師,詔勉留。



 夏四月,詔褒襄城死節,右領衛將軍範天順贈靜江軍承宣使,右武大夫、馬司統制牛富贈金州觀察使,各官其二子承信郎,賜土田、金幣恤其家。甲申,汪立信權兵部尚書、京湖安撫制置使、知江陵府、夔路策應使、湖廣總領,不許辭免。以錢二百萬給立信開閫犒師。葉夢鼎乞致仕,遣官勉諭赴都堂治事。
 辛卯,以趙溍為淮西總領兼沿江制置、建康留守。詔黃萬石赴闕。壬辰,詔:「襄陽六年之守,一旦而失,軍民離散,痛切朕心。今年乾會節,其免集英殿宴,以錢六十萬給沿江制置趙溍江防捍禦。」癸巳,知招信軍陳巖乞祠祿。詔曰:「乃者邊吏弗戒,致有襄難,將士頻歲暴露,邊民蕩析離居,PN傷朕心。爾閫臣專征方面,宜身率諸將,宣揚國威,以賞戮用命不用命。爾守臣有土有民,宜申儆國人,保固封守。爾諸將尚迪果毅,一乃心力,各以其兵,敵
 王所愾。今朕多誥,爾其悉聽明訓,毋懈毋心耎,習於故常,功多有厚賞,爾不克用勸,罰固不得私也。又如中外小大臣僚,有材識超卓、明控御之宜、懷攻守之略者,密具以聞,一如端拱二年制書,朕當虛心以聽。」李庭芝乞解罷,詔赴闕。壬寅,詔復置樞密院都統制、副都統制各一員。丁未,以高達為寧江軍節度使、湖北安撫使、知峽州。詔忠州潛已升咸淳府,刺史王達改授高州刺史。李庭芝辭召赴闕,詔與祠。巳酉,詔:「南歸人復有戰功者予
 優賞,楊春、薛聚成、陳君謨、周海、周興各補成忠郎,蕭成、侯喜、丁甫、劉鑄、鄭歸各補承信郎。」以夏貴兼侍衛馬軍都指揮使。庚戌,詔汪立信賞罰調用悉聽便宜行事。辛亥,呂師夔言:「比賈似道得李庭芝書,報臣叔父文煥以襄城降,臣聞之隕越無地,不能頃刻自安。請以經略安撫、轉運、靜江府印委次官護之,席蒿俟命,容臣歸省偏親,誓當趨事赴功,毀家紓難,以贖門戶之愆,以報君父之造。」詔不允。



 五月乙卯,以黃萬石權戶部尚書兼知臨
 安府、浙西安撫使。四川制司朱祀孫言:「所部諸縣除正闢文臣外,諸郡屬邑,許令本司不拘外縣一體選闢文臣,以幸蜀之士民。」奏可。丙辰,知廬州呂文福言:「從兄文煥以襄陽降,為其玷辱,何顏以任邊寄,乞放罷歸田里。」詔不允。呂師夔五疏乞罷任,詔赴闕。丁卯,申禁奸民妄立經會,私創庵舍,以避征徭,保伍容芘不覺察坐之。辛未,劉雄飛乞致仕。戊寅,孝感縣丞關應庚上書言邊防二十事,詔授武當軍節度推官兼司法,京湖制司量材
 任使。庚辰,馬軍司統制王仙昔在襄、樊緣戰陷陣,今復來歸,特與官五轉,充殿前司正額統制,賜錢一萬。布衣林椿年等上書言邊防十數事,詔諸人上書凡言請以丞相似道督視者不允,餘付機速房。



 六月,刑部尚書兼給事中陳宜中言,樊城之潰,牛富死節尤著,以職卑,贈恤下範天順一階,未愜輿情。詔加贈富寧遠軍承宣使,仍賜土田、金幣恤其家。前四川宣撫司參議官張夢發詣賈似道,上書陳危急三策,曰鎮漢江口岸,曰城荊門
 軍當陽界之玉泉山,曰峽州宜都而下聯置堡砦,以保聚流民,且守且耕,並圖上城築形勢。賈似道不以上聞,下京湖制司審度可否,事竟不行。成都安撫使昝萬壽去冬調將士攻毀成都大城,今春戰碉門,五月遣統制楊國寶領兵至雅州,統領趙忠領兵至眉州,兩路捍禦有勞,詔具將士宣力等第、姓名以聞。呂文福言文煥為人扶擁,以襄陽降非由己心。詔與李庭芝元陳異同,其審核以聞。庭芝表:「向在京湖,來歸人吳旺等備言文煥
 父子降狀,先納莞鑰,旋獻襄城,且陳策攻郢州,請自為先鋒。言人人同,制司案辭可徵,非敢加誣人罪。」詔文福勉力捍禦,毋墜家聲。京湖制司言:「去年冬間,探司總管劉儀、盛聰,總制趙鐸,領精銳至均州文龍崖立砦。呂文煥既降,均城受敵,知郡劉懋偕劉儀等捍禦宣勞。」詔懋升右武大夫、帶行左衛大將軍,仍舊職,儀添差荊湖北路兵馬鈐轄,聰添差鄂州兵馬鈐轄,各官三轉,將士官兩轉。左藏東庫蹇材望上書言邊事大可憂者七,急當
 為者五。不報。丙戌,劉雄飛卒,特贈一官。戊子,京湖制司請給器械,詔內軍器庫選犀利者賜之,仍贈錢百萬備修繕。四川制置朱祀孫言月奉銀計萬兩,願以犒師,向後月免請。詔常祿勿辭。己丑,給事中陳宜中言,乞正範文虎不力援襄之罰,詔文虎降一官、依舊知安慶府。安南國進方物,特賜金五百兩、帛百匹。癸卯,汪立信言:「臣奉命分閫,延見吏民,皆痛哭流涕而言襄、樊之禍,皆由範文虎及俞興父子。文虎以三衙長聞難怯戰,僅從薄
 罰,猶子天順守節不屈,猶或可以少贖其愆。興奴僕庸材,器量褊淺,務復私仇,激成劉整之禍,流毒至今。其子大忠挾多資為父行賄,且自希榮進,今雖寸斬,未足以快天下之忿,乞置重典,則人心興起,事功可圖。」詔俞大忠追毀出身文字,除名、循州拘管。又言守闕進義副尉童明,襄陽破,拔身來歸,且嘗立功開州,乞補轉四官。詔特與官兩轉。



 閏月辛亥,命殿前指揮使陳奕總統舟師備鄂州、黃州江防。癸丑,來歸人郭珍補成忠郎,張進、張
 春、張德林、向德成、王全、婁德、王興各補承信郎。丙辰,朝散郎師顯行進《注皇朝文鑒》。前臨安府司法梁炎午陳攻守之要五事,不報。命大理寺丞鐘蜚英點視沿江堡隘兵船,戊辰,知敘州郭漢傑言,馬湖蠻王汝作、鹿巫蠻王沐丘,帥蠻兵五百餘助官軍民義阻險馬湖,捍禦有功。詔賞汝作、沐丘金帛及其部兵有差。敘州總管曹順一軍,凡在戰陣者,趣具立功等第來上。



 秋七月丁亥,權紹興府節制紫城軍義文榮鼎及將校趙居敬、丁福、孟
 青、蒲祥、白貴、史用、羅宜、王繁等九人,成都之役沒於兵,各追贈官秩,仍官其子。癸巳,知達州趙章、知開州鮮汝忠、知渠州張資等復洋州。戊戌,張玨等復馬□□山。



 八月癸丑,權知均州徐鼎、總管盛聰戰房州胡師峪、板倉。乙卯,知房州李鑒調權竹山縣王國材、統制熊權、總轄馬宗明,戰落馬坪、白羊山,詔有司各以勞效論賞。



 九月辛巳,以章鑒簽書樞密院事兼參知政事,陳宜中同簽書樞密院事。成都安撫使昝萬壽城嘉定烏尤山。乙未,以
 洪燾為浙東安撫使。丙申,以黃萬石為湖南安撫使。



 冬十月己酉,來歸人汪福、許文政各官五轉。癸丑,鎮巢軍、和州、太平州諸將查文、李文用、孟浩等十一人,以射湖岡、萬歲嶺、後港及焦湖北岸戰功,咸賜爵賞。癸亥,雷。四川制司言何炎向失洋州,調知達州趙章等率諸部軍義復之;七月又復洋州、吳勝堡兩城,權檄統轄謝益知洋州,總制趙桂楫知巴州,俾任責吳勝堡戰守之事。至是以功來上,且以二州攝事守臣請命於朝,詔與正授。
 丁丑,兩淮制置使印應雷告老,進二秩致仕。李庭芝兩淮安撫制置使,賜錢二百萬激犒備御。



 十一月壬午,子□授左衛上將軍,封嘉國公。戊子,知泰州龔準遣其將王大顯等捍禦水砦有功,又獲俘民以還,詔水步兩軍將校凡用命者賞激有差。甲午,以夏貴為淮西制置使兼知廬州,陳奕沿江制置使兼知黃州,呂文福知閣門事。詔從李庭芝請分淮東、西制置為兩司,就命庭芝交割淮東,仍兼淮西策應使。乙未,以夏貴為淮西安撫制
 置使,賜錢百萬激犒備御。李庭芝辭免淮西策應使,不允。知安豐軍陳萬以舟師自城西大澗口抵正陽城,遇北兵力戰,詔旌其勞。十二月甲子,以馬廷鸞為浙東安撫使、知紹興府。丙寅,權參知政事章鑒再乞解機政,不允。丁丑,沿江制置使所轄四郡夏秋旱澇,免屯田租二十五萬石。



 十年春正月壬午,城鄂州漢口堡。權總制施忠、部將熊伯明、知泰州龔準以天長縣東橫山、秦潼湖、青蒲口等
 處戰功推賞。戊子,江萬里以疾辭職任,詔依舊觀文殿大學士、提舉洞霄宮。乙丑,以留夢炎知潭州兼湖南安撫使。庚寅,城鄂州沌口西岸堡。京湖制司言襄陽勇信中軍鈐轄吳信隨呂文煥北往,今並妻子冒險來歸。詔吳信赴闕,制司仍存恤其家。丙申,江東沙圩租米,以咸淳九年水災,詔減什四。乙巳,雨土。



 二月己酉,以趙順孫為福建安撫使。辛酉,詔諸制閫就任升除恩數,其告命、衣帶、鞍馬,閣門勿差人給賜,往要厚賂,以失優寵制臣
 之意,違者有刑。



 三月己卯,免郡縣侵負義倉米七十四萬八千餘石。



 夏四月乙卯,子昺授左衛上將軍,進封永國公。詔賞沿江都統王達、黃俁戰黃連寺之功。戊午,以呂文福為常德、辰、沅、澧、靖五郡鎮撫使、知沅州。辛酉,詔賞光州守陳巖、路分李全、許彥德、總管何成、路鈐仰子虎等牛市畈、丁家莊戰功。烏蘇蠻王詣雲南軍前納款大元。



 五月丁亥,以高世傑為湖北安撫副使兼知嶽州,總統出戍軍馬。辛丑,馬廷鸞辭免觀文殿大學士、知紹
 興府、浙東安撫使,詔不允。壬寅,張玨表請城馬□□、虎頭兩山,或先築其一,以據險要。



 六月戊午,以銀二萬兩命壽春府措置邊防。



 秋七月壬午,汪立信乞致仕,不允。癸未,帝崩於福寧殿,遺詔太子□即皇帝位。甲申,臺臣劾內醫蔡幼習,詔奪五秩,送五百里州軍居住,二子並罷閣門職。



 八月己酉,上大行皇帝謚曰端文明武景孝皇帝,廟號度宗。德祐元年正月壬午,葬於永紹陵。



 贊曰:宋至理宗,疆宇日蹙,賈似道執國命。度宗繼統,雖
 無大失德,而拱手權奸,衰敝寢甚。考其當時事勢,非有雄才睿略之主,豈能振起其墜緒哉!歷數有歸,宋祚尋訖,亡國不於其身,幸矣。



\end{pinyinscope}