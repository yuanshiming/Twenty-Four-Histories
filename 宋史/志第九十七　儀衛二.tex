\article{志第九十七 儀衛二}

\begin{pinyinscope}

 宮中導從行幸儀衛太上皇儀衛后妃儀衛



 宮中導從之制,唐已前無聞焉。五代漢乾祐中,始置主輦十六人,捧足一人,掌扇四人,持踏床一人,並服文綾袍、銀葉弓腳帕頭。尚宮一人,寶省一人,高鬢、紫衣。書
 省二人,紫衣、弓腳帕頭。新婦二人,高鬟、青袍。大將二人,紫衣、弓腳帕頭。童子執紅絲拂二人,高鬟髻、青衣。執犀盤二人,帶鬅頭、黃衫。執翟尾二人,帶鬅頭、黃衫。雞冠二人,紫衣,分執金灌器、唾壺。女冠二人,紫衣,執香爐、香盤。分左右以次奉引。



 太宗太平興國初,增主輦二十四人,改服高腳帕頭;輦頭一人,衣紫繡袍,持金塗銀仗以督領之。奉珍珠、七寶、翠毛華樹二人,衣緋袍;奉金寶山二人,衣綠繡袍;奉龍腦合二人,衣緋銷金袍,並高腳帕頭。執
 拂翟四人,鬅頭、衣黃繡袍。舊衣綾袍、紫衣者,悉易以銷金及繡。復增司薄一人,內省一人,司儀一人,司給一人,皆分左右前導,凡一十七行。每正、至御殿,祀郊廟,步輦出入至長春殿用之。其乘輦,則屈右足、垂左足而憑幾,蓋唐制也。真宗時,加四面內官周衛。大中祥符三年,內出繪圖以示宰相。



 行幸儀衛。宋初,三駕皆以待禮事。車駕近出,止用常從以行。其舊儀,殿前司隨駕馬隊,凡諸班直內,殿前指揮
 使全班祗應:左班七十六人,二十四人在駕前左邊引駕,五十二人作兩隊隨駕;右班七十七人,二十四人在駕前右邊引駕,五十三人在駕後作兩隊隨駕,二十七人第一隊,二十六人第二隊。內殿直五十四人,散員六十四人,散指揮六十四人,散都頭五十四人,散祗候五十四人,金槍五十四人,茶酒班祗應殿侍百五十七人,東第二班長入祗候殿侍十八人,駕後動樂三十一人,馬隊弩手分東西八十五人,招箭班三十五人,散直
 百七人,鈞容直三百二十人,御龍直百四十二人,御龍骨朵子直二百二十人,並全班祗應。御龍弓箭直百三十三人,御龍弩直百三十三人,寬衣天武指揮二百一十六人。



 各有都虞候、指揮使、員僚。



 若隨駕不使馬隊,即減內殿直、散員、散指揮、散都頭、散祗候、金槍等直,仍減東西班馬隊駑手八十五人,餘並同上。



 凡皇城司隨駕人數:崇政殿祗應親從四指揮共二百五十二人,執擎骨朵,充禁衛;崇政殿門外快行、祗候、親從第四指揮五十四人;車駕
 導從、兩壁隨行親從親事官共九十六人,並於駕前先行,行幸所到之處,充行宮司把門、灑掃祗應。



 各有正副都頭、節級、十將。



 尚書兵部供黃麾仗內法物:罕畢各一。五色繡氅子並龍頭竿掛,第一,青繡孔雀氅;第二,緋繡鳳氅;第三,青繡孔雀氅;第四,皂繡鵝氅;第五,白繡鵝氅;第六,黃繡雞氅。又六軍儀仗司供儀仗法物,內獅子旗四口,充門旗二口,各一人執,分左右;二口各十人執扯,分左右,扯人執弓箭。又左金吾引駕仗供牙門旗十四口,十口開五
 門,每門二口,每口一人執二人夾,計三十人,並騎,夾人執弓箭。監門校尉二十人,每門四人,並帶儀刀,騎。二口系前步甲第七隊前,二口系前部黃麾第一隊前,二口系後部黃麾第一隊前,二口系後步甲第一隊前,二口系後步甲第七隊前。四口開二門,每門二口,每口一人執二人夾,計十二人,並騎。監門校尉六人,並帶儀刀,騎。二口系兵部班劍儀刀隊後,二口系真武隊前。又右金吾引駕仗供牙門旗十四口,制同左仗。



 仁宗康定元年,
 參知政事宋庠言:「車駕行幸,非郊廟大禮具陳鹵薄外,其常日導從,惟前有駕頭,後擁傘扇而已,殊無禮典所載公卿奉引之盛。其侍從及百司官屬,下至廝役,皆雜行道中。步輦之後,但以親事官百餘人執□以殿,謂之禁衛。諸班勁騎,頗與乘輿相遠;士庶觀者,率隨扈從之人,夾道馳走,喧呼不禁。所過旗亭市樓,垂簾外蔽,士民馮高下瞰,莫為嚴憚。邏司街使,恬不呵止。威令弛闕,玩習為常。非所謂旄頭先驅,清道後行之慎也。且自黃帝
 以神功盛德,猶假師兵營衛,則防微御變,古今一體。案漢魏以降,有大駕、小駕之儀。至唐又分殿中諸衛、黃麾等仗,名數次序,各有施設。國朝承五姓荒殘之弊,事從簡略,每鳴鑾游豫,盡去戈戟、旌旗之制,儀衛寡薄,頗同藩鎮。此皆制度放佚,憚於改作之咎。宜委一二博學近臣,討繹前代儀注及鹵薄令,以乘輿常時出入之儀,比之三駕諸仗,酌取其中,稍增儀物,具嚴法禁,以示尊極,以防未然。革去因循,其在今日。」詔太常禮院與兩制詳
 定,參以舊儀,別加新制。



 兩制同禮官議,略準小駕制度,添清道馬、罕畢、旗氅等物。別為常行禁衛儀,加清道馬百匹,並帶器械,分五行,行二十人。請下殿前司,於諸班內差。



 罕畢各一,分左右,並騎。牙門旗前後各四,分左右,並騎。緋繡鳳氅二十四,分左右,並騎。以上請下殿前司,於諸班內差充。



 雉扇十二,分左右。請於親從官內差充。



 以上新添百六十二人。凡天武官舊二百一十六人,空行,今添執哥舒,為一重。親從官舊百四十五人,今添百五十五人,通為三百人,為一重。殿前指
 揮使舊四十八人,今添百五十二人,通為二百人,或於近上諸班相兼差充,並騎,為一重。以上因舊人數添。



 舊四百九人,新添三百七人,共七百一十六人。



 凡駕前殿前指揮使、親從官為二重,左右相對,各開二門,約二丈,每門並差人員二人押當。第一門與通事舍人相對,第二門與閣門使相對。每有臣僚迎駕起居,並令中道候起居畢,於左右門出。其諸色人止令於牙門旗前道傍起居,不得便入禁衛中。



 每門外重,令殿前指揮使執旗二面以表門,用轉光錯彩旗,通上計五重,皆掩後團轉。



 凡百司祗
 應人於禁衛內無執掌者,及隨駕臣僚除合將入禁衛隨從人數外,餘並令於殿前指揮使行外左右前後行。凡前牙門旗以後,後牙門旗以前,屬禁衛中,不得輒入。凡中書、樞密院臣僚,並於從內第三重寬衣天武內行馬;其餘隨駕文武臣僚,並在從內第四重殿前指揮使內,分左右依官位行馬。



 凡車駕經歷去處,若有樓閣,並不得垂簾障蔽,及止絕士庶不許臨高瞰下,止於街兩傍立觀,即不得夾路喧呼馳走。前牙門以前,後牙門以後,不在此限。



 凡車
 駕未出皇城門,宣德、左右掖、東華、拱宸門及已至所幸處,即自有門禁,不用牙門旗約束。凡車駕已在道,前牙門旗雖行,後牙門旗未行,除止絕閑雜行人外,其隨駕臣僚官司人等,並依常例,次第赴合隨從及行馬去處。凡前牙門旗在清道馬後約十步已來,後牙門旗在駕後殿前指揮使之後。凡街巷寬闊處,儀衛並依新圖排列。如遇窄狹街巷,禁衛止用親從官二重,御龍直二重,雉扇隨輦。其殿前指揮使、天武官,並權分於駕前後隨
 行。後至寬闊處,乘輿徐行,儀仗依舊排列。或駕幸園苑、宮觀、寺院並臣僚宅,即清道馬、儀仗、殿前指揮使、天武官更不入,惟於外排立。其隨駕臣僚及諸司人,自依常例隨從,候駕行,依次排列。或臣僚宅在巷內,前去不通人行處,其儀仗、殿前指揮使等,各於巷口排立,止絕行人,餘並如故。時詳定閱習既畢,或言新制嚴密,慮違犯者眾,因不果行。



 嘉祐六年,先是,幸睦親宅,抱駕頭內臣墜馬,壞駕頭。太常禮院、閣門及整肅禁衛所請自今車
 駕出,以閣門祗候並內臣各二員,分駕頭左右扇筤後編攔,仍以皇城司親從官二十人隨之。



 哲宗紹聖二年,詔:車駕行幸儀衛,駕後東西班殿侍馬兩隊,撥充駕前編攔,分兩壁行於前引行門之前,隨身器械,各別給銀骨朵一。駕後馬隊、殿前指揮使馬,以百人分四隊。不足,據人數均差,仍別差人員六人。內殿直、散員、散指揮、散都頭、散祗候,並增作一百四人,分四隊,內人員各四人。金槍班添一隊,作七十八人,內人員三人。弩手班添兩
 隊,充填撥過東西班殿侍馬兩隊。禁衛御龍直、弓箭直、弩直、長行,仍各添給銀骨朵。禁衛外,添差編攔天武人員、長行共二百人,揀選有行止舊人充,出入止於宣德門外,至行在所,即止於行宮門外。



 南渡後,乘輿出入,初未有儀。高宗將迎韋太后於郊,因制常行儀仗,用黃麾仗二千二百六十五人。孝宗朝德壽宮,減一千人,用殿前司六百二十九人,皇城在內巡檢司三百九十一人,崇政殿四百四十九人,凡一千四百六十九人。四孟詣
 景靈宮,用殿前司八百七十五人,皇城在內巡檢司五百二十八人,崇政殿五百二十一人,凡一千九百二十四人。九年正月,詔:駕出御後殿坐,宰執、百官、儀衛等赴後殿,起居殿上;登輦,出後殿門,駕回,入祥曦殿門。



 太上皇儀衛。隆興元年,孝宗嗣位,詔有司討論德壽宮輿輦儀衛。先是,紹興三十二年六月,詔:「上皇日常朝殿,差御龍直四十三人,執仗排立,並設傘扇,鳴鞭。宰執退朝,仍赴德壽宮起居。如遇行幸,令禁衛所隨以祗應。」兩
 奉上皇旨,卻而不受,故復有是詔。尋有司上言:「漢之未央,唐之興慶,其車輦儀衛不載。今父堯子舜,事親典禮,凡往古來今所未備者,當以義起,極其尊崇,為萬世法。」遂定宰執、百官詣德壽宮起居,則禁衛所依後殿坐儀排列,禁衛二百九十七人祗應。行幸,則禁衛所差行門、禁衛諸班直、天武親從官及傘扇、鳴鞭、燭罩等合五百人,隨行扈從。前引七十人:內行宮殿前崇政殿親從一十人,都下親從二十人,快行親從二十人,殿前指揮使
 二十人。中道六十人:編排禁衛行子一十人,執從物御龍直三十人,執傘扇天武一十人,崇政殿親從攔前一十人。禁衛圍子四重四百人:第一,崇政殿親從一百人;第二,御龍直、骨朵直、弓箭直三十人,東西班七十人;第三,執燭罩都下親從一百人;第四,內殿直一十人,散員、散指揮、散都頭、散祗候、金鎗、銀鎗班各一十人,後從殿前指揮使二十人。



 皇太后儀衛。自乾興元年仁宗即位,章獻太后預政,侍
 衛始盛。用禮儀院奏,制皇太后所乘輿,名之曰「大安輦」。天聖元年,有司言:「皇太后車駕出,合設護衛:御龍直都虞候一人,都頭二人,副都頭一人,十將、長行五十人;骨朵子直都虞候一人,都頭二人,副都頭二人,十將、長行八十人;弓箭直指揮使一人,都頭二人,副都頭二人,十將、長行五十人;弩直指揮使一人,都頭二人,副都頭二人,十將、長行五十人。殿前指揮使兩班:左班都虞候一人,都知一人,行門三人,長行二十人,帶器械;右班
 指揮使一人,都知一人,行門三人,長行二十人,帶器械。皇城司禁衛二百人,寬衣天武二百人,供御輦官六十二人,寬衣天武百人。餘諸司祗應、鳴鞭、侍衛,如乘輿之儀。」詔依。



 嘉祐八年,英宗即位,太常禮院言:「準詔再詳定皇太后出入儀衛:御龍直都頭二人,長行二十五人;骨朵子直都頭二人,長行四十人;弓箭直都頭二人,長行二十五人;弩直都頭二人,長行二十五人。殿前指揮使兩班,各都知一人,行門各二人,長行各一十人,帶器械。
 皇城司禁衛一百人,寬衣天武一百五十人,打燈籠子親事官八十人。入內都知、御藥院官各一員,內東門司使臣二員。御輦院短鐙、教駿、攏馬親事官,入內院子,諸司並入內內侍省祗應內品,入數不定。」詔依。



 治平元年,詔皇太后出入唯不鳴鞭,他儀衛如章獻明肅故事。四年,神宗嗣位,詔太皇太后儀範已定,皇太後合設儀衛:御龍直、骨朵子直差都虞候、都頭、副都頭各一人,十將、長行各共三十人;弓箭直、弩直差指揮使、都頭、副都頭
 各一人,十將、長行各共二十人。皇城司親從官一百人,執骨朵寬衣天武官百五十人,充圍子行宮司人員共一百人,入內院子五十人,充圍子皇城司親事官八十人。打燈籠、短鐙馬、攏馬親從官,金銅車、棕車隨車子祗應人,擎擔子供御輦官,執擎從物等供御、次供御並下都輦直等,人數不定。都知一員,御藥院使臣二員,內東門司使臣二員,內酒坊、御廚、法酒庫、儀鸞司、乳酪院、翰林司、翰林院、車子院、御膳素廚、化成殿果子庫,並從。遇
 出新城門,添差帶器械內臣。



 哲宗即位,元祐元年,詔太皇太后出入儀衛,並依章獻明肅皇后故事。其不可考者,則依慈聖光獻皇后之例。既而又詔:太皇太后出入儀衛,添御龍骨朵子直三十六人,御龍弓箭直四十五人,御龍弩直四十五人,皇城司禁衛五十人,馬隊三百五十人,東西班、茶酒班殿侍共一百人,快行增至二十人。軍頭引見司監官二員,並將帶承局、等子,依隨駕例祗應;鈞容直並動樂殿侍,則候開樂取旨。



 仁、英、哲之世,
 太后臨朝垂簾,儀從亦不崇侈,止曰儀衛,無鹵薄名也。南渡後尤簡,其車以輿不以輦,餘惟傘、扇而已。紹興奉迎太母,極意備禮,然猶曰太后天性樸素,不敢過飾儀從。器物惟塗金,輿前用黃羅傘扇二,緋黃繡雉扇六,紅黃緋金拂扇二,黃羅暖扇二。朝謁景靈宮、太廟,則用禁衛諸班直、天武親從五百人。其前引、中道、圍子,同上皇儀衛而差省焉。



 皇太妃出入儀衛。哲宗紹聖元年,三省、樞密院言:「增崇
 皇太妃出入儀衛:龍鳳扇二十,侍從官入內省都知或押班一員,內侍省都知或押班一員,皇城司、御藥院、內東門司各一員,帶御器械內侍八員,引喝內侍一員。殿前指揮使三十二人,內人員二人,御龍直三十三人,骨朵子直三十三人,弓箭直二十三人,弩直二十三人,天武官一百五十四人,皇城司禁衛一百人,入內院子五十人,行宮司一百人,輦官供御六十二人,次供御四十九人,下都五十八人,燭籠七十,諸司御燎子、茶床、快行
 親從四人。」禮部太常寺又言:「元祐三年,詔皇太妃傘用紅黃羅。參議得皇太后出入兼用紅黃,今皇太妃若亦用黃,則非差降之意。伏請紅黃兼用,從皇太后出入,則止用紅。」



 徽宗崇寧元年,臣僚言:「元符皇后,先帝皇后也,其曲禮宜極褒崇。」於是約聖瑞皇太妃之制,出入由宣德正門,增龍鳳扇二十,御龍直十二人,御龍骨朵子直十七人,御龍弓箭直十二人,御龍弩直二十二人,殿前指揮十三人,皇城司禁衛二十人,快行親從官四人,執
 燭、皇城司親從官、金銅車並棕車,隨時定數供須。行幸藥架一坐,勾當官、吏人二員,封題一員,藥童三人,抬擎藥架輦官十一人,秤、庫子親事官,量差人數祗應。從之。



 二年,臣僚又言:「元符皇后,元符末嘗預定策之勛,以承神宗、哲宗之志。」禮部太常寺奏:「典禮,準聖瑞皇太妃例,侍從官入內內侍省都知或押班一員,皇城、御藥、內東門司官各一員,御輦院輪官隨從,諸司御燎子、茶床、帶御器械內侍十人,引喝內侍一人。輿用龍鳳,傘紅黃兼
 用。出入由宣德東門,今欲出入由宣德正門。龍鳳扇二十柄,今添作三十柄。輦官供御六十二人,次供御四十九人,都下五十八人。御龍直三十三人,今添作四十五人。御龍骨朵子直三十三人,今添作五十人。御龍弓箭直三十三人,今添作四十五人。御龍弩直二十三人,今添作四十五人。殿前指揮三十二人,今添作四十五人。內臣二人。皇城司一百人禁衛,今添作一百二十人。天武官一百五十四人,行宮司一百人,入內院子五十人。
 快行親從官四人,今添作八人。執燭、皇城司親從官、金銅車並棕車,隨時內中批出合要數供須。行幸藥架一坐,勾當官一員,吏人二員,封題一員,藥童三人,抬擎藥架輦官十一人,秤、庫子親事官,量差人數祗應。」從之。



 皇后儀衛,惟東都《政和禮》有鹵簿,他無鹵簿之名,惟曰儀衛而已。中興後,皇太后既尚簡素,後尤簡焉。出入朝謁宮廟,用應奉御輦官一員,人吏三人。供應六十三人:內人員十五人,頭帽、紫羅四衣癸單衫、金塗銀柘枝腰帶;
 肩擎輦官四十八人,帕頭、緋羅單衫、金塗海捷腰帶、紫羅表夾三襜、緋羅看帶。次供應十四人:內人員一人,服同上,惟海捷帶;輦官一十三人,服同肩擎官,惟行獅帶。都下五十四人:內人員一人,帽服同前;輦官五十三人,服同上,輦宮惟雲鶴帶。



\end{pinyinscope}