\article{列傳第一百三十}

\begin{pinyinscope}

 白
 時中徐處仁馮澥王倫宇文虛中湯思退



 白時中,字蒙亨,壽春人。登進士第,累官為吏部侍郎。坐事,降秩知鄆州,已而復召用。政和六年,拜尚書右丞、中
 書門下侍郎。宣和六年,除特進、太宰兼門下,封崇國公,進慶國。



 始,時中嘗為春官,詔令編類天下所奏祥瑞,其有非文字所能盡者,圖繪以進。時中進《政和瑞應記》及《贊》。及為太宰,表賀翔鶴、霞光等事。圜丘禮成,上言休氣充應,前所未有,乞宣付秘書省。時燕山日告危急,而時中恬不為慮。金人入攻,京城修守備,時中謂宇文粹中曰:「萬事須是涉歷,非公嘗目擊守城之事,吾輩豈知首尾邪?」



 欽宗即位,召大臣決策守京師,問誰可將者。李綱
 言:「朝廷高爵厚祿蓄養大臣,蓋將用之有事之日。時中輩雖書生,然撫將士以抗敵鋒,乃其職也。」時中勃然曰:「李綱莫能將兵出戰乎?」綱曰:「陛下儻使臣,當以死報。」於是以綱為右丞,充守禦使。時中尋罷為觀文殿學士、中太一宮使。御中劾時中孱懦不才,詔落職。未幾,卒。



 徐處仁,字擇之,應天府穀熟縣人。中進士甲科,為永州東安縣令。蠻人叛,處仁入峒,開示恩信,蠻感泣,誓不復反。知濟州金鄉縣。以薦者召見,徽宗問京東歲事,處仁
 以旱蝗對。問:「邑有盜賊乎?」曰:「有之。」上謂處仁不欺,除宗正寺丞、太常博士。



 時初置算學,議所祖,或以孔子贊《易》知數。處仁言:「仲尼之道無所不備,非專門比。黃帝迎日推策,數之始也,祖黃帝為宜。」擢監察御史,遷殿中、右正言、給事中。攝開封府,裁決如流,囚系常空。進戶部尚書,繼拜中大夫、尚書右丞。丁母憂,免喪,以資政殿學士知青州,徙知永興軍。



 童貫使陜西,欲平物價,處仁議不合,曰:「此令一傳,則商賈弗行,而積藏者弗出,名為平價,適
 以增之。」轉運使阿貫意,劾其格德音,倡異論,侵辱使者。詔處仁赴闕。尋改知河陽,落職知蘄州。久之,以顯謨閣直學士知穎昌府。民有得罪宮掖者,雖赦不原,處仁為奏上。童貫乘是擠之,奪職,提舉鴻慶宮。復延康殿學士、知汝州,再奉鴻慶祠、知徐州,召為醴泉觀使。



 徽宗訪以天下事,處仁對曰:「天下大勢在兵與民,今水旱之餘,賦役繁重,公私凋弊,兵民皆困,不及今謀之,後將有不勝圖者。」上曰:「非卿不聞此言。」明日,除侍讀。進讀罷,理前語,
 處仁言:「昔周以塚宰制國用,於歲之杪,宜會朝廷一歲財用之數,量入為出,節浮費,罷橫斂,百姓既足,軍儲必豐。」上稱善詔置裕民局討論振兵裕民之法。蔡京不悅,言者謂:「今設局曰『裕民』,豈平日為不裕民哉?」乃罷局,出處仁知揚州。未幾,以疾奉祠歸南都。



 方臘為亂,處仁亟見留守薛昂,為畫守戰之策。因語昂曰:「睢陽蔽遮江、淮,乃國家受命之地,脫有非常,吾助君死守。」語聞於朝,起為應天尹。河北盜起,徙大名尹。前尹王革慘而怯,盜無
 輕重悉抵死,小有警,輒閉城以兵自衛。處仁至,即大開城門,徹牙內甲兵,人情遂安。



 徽宗賜手詔曰:「金人雖約和,然狼子野心,易扇以變,有當行事以聞。」處仁上《備邊御戎》十策。進觀文殿學士,召為寶菉宮使,特升大學士。舊制,大觀文非宰相不除,前二府得除,自處仁始。



 欽宗即位,金人犯京師,處仁儲糧列備,合銳兵萬人勤王;奏乞下詔親征,以張國威。奏至,朝廷適下親征詔書,以李綱為行營使。即移書綱,言備御方略。金人請和而歸,處
 仁奏宜伏兵浚、滑,擊其半濟,必可成功。召為中書侍郎。入見,欽宗問割三鎮,處仁言:「國不競亦陵,且定武陛下之潛藩,不當棄。」與吳敏議合。敏薦處仁可相,拜太宰兼門下侍郎。



 童貫部勝捷軍衛徽宗東巡,貫既貶,軍士有惡言。徽宗將還,都人洶懼,或請為備。處仁曰:「陛下仁孝,思奉晨昏,屬車西還,天下大慶,宜郊迎稱賀。軍士妄言,臣請身任之。」乃以處仁為扈駕禮儀使,統禁旅從出郊,迄二聖還宮,部伍肅然。



 初,處仁為右丞,言:「六曹長貳,
 皆異時執政之選,而部中事一無所可否,悉稟命朝廷。夫人才力不容頓異,豈有前不能決一職而後可共政者乎?乞詔自今尚書、侍郎不得輒以事諉上,有條以條決之,有例以例決之,無條例者酌情裁決;不能決,乃申尚書省。」會處仁以憂去,不果行,及當國,卒奏行之。



 聶山為戶部尚書兼開封尹,庫有美珠,山密語寧德宮宦者,用特旨取之。處仁奏:「陛下鑒近患,事必由三省。今以珠為道君太上皇后壽,誠細故,且美事;然此端一開,則前日
 應奉之徒復縱,臣為陛下惜之。」乃抵主藏吏罪。



 處仁言論,初與吳敏、李綱合,尋亦有異議。嘗與敏爭事,擲筆中敏面,鼻額為黑。唐恪、耿南仲、聶山欲排去二人而代之位,諷言者論之,與敏俱罷,處仁以觀文殿大學士為中太一宮使。尋知東平府,提舉崇福宮。高宗即位,起為大名尹、北道都總管,卒於郡。



 處仁在宣和間,數請寬民力以弭盜賊。尹大名,以剛廉稱。及為首相,無大建明,方進言以金人出境,社稷再安,皆由聖德儉勤,致有天人之
 助。仲師道請合諸道兵屯河陽諸州,為防秋計,處仁謂金人豈能復來,不宜先自擾以示弱。南都受圍時,處仁在圍城中,都人指為奸細,殺其長子庚。幼子度,吏部侍郎。



 馮澥字長源,普州安岳人。父山,熙寧末,為秘書丞、通判梓州,鄧綰薦為臺官,不就,退居二十年,範祖禹薦於朝,官終祠部郎中。澥登進士第,歷官入朝,以言事再謫。



 靖康元年,澥為左諫議大夫。金人圍太原,朝廷命李綱宣
 撫兩河,澥奏罷之。金人要割三鎮,高宗自康邸出使,除澥知樞密院事,充副使,不果行,尋除尚書左丞。金人犯闕,詔宗室郡王為報謝使,澥與曹輔以樞密為副,留金營三日歸,詔暫權門下侍郎。欽宗詣金營,澥扈從。張邦昌僭位,與澥有舊,取之歸,以澥康邸舊臣,命為奉迎使,為總領迎駕儀物使。建炎初,除資政殿學士、知潼川府。言者論澥嘗污偽命,奪職,已而復官。紹興三年,以資政殿學士致仕,卒。



 澥為文師蘇軾,論西事與蔡京忤。郡人
 張庭堅以言事斥象州死,妻子流離,澥力振其家,及入諫省,奏官其一子。然議論主熙、豐、紹聖,而排鄒浩、李綱、楊時,君子少之。



 王倫,字正道,莘縣人,文正公旦弟勖玄孫也。家貧無行,為任俠,往來京、洛間,數犯法,幸免。汴京失守,欽宗御宣德門,都人喧呼不已,倫乘勢徑造御前曰:「臣能彈壓之。」欽宗解所佩夏國寶劍以賜,倫曰:「臣未有官,豈能彈壓?」道自薦其才。欽宗取片紙書曰:「王倫可除兵部侍郎。」倫
 下樓,挾惡少數人,傳旨撫定,都人乃息。宰相何□以倫小人無功,除命太峻,奏補修職郎,斥不用。



 建炎元年,選能專對者使金,問兩宮起居,遷朝奉郎,假刑部侍郎。充大金通問使,閣門舍人朱弁副之,見金左副元帥宗維議事,金留不遣。



 有商人陳忠,密告倫二帝在黃龍府,倫遂與弁及洪皓以金遺忠往黃龍府潛通意,由是兩宮始知高宗已即位矣。久之,粘罕使烏陵思謀即驛見倫,語及契丹時事。倫曰:「海上之盟,兩國約為兄弟,萬世無
 變。雲中之役,我實饋師,贊成厥功。上國之臣,嘗欲稱兵南來,先大聖惠顧盟好,不許。厥後舉兵以禍吾國,果先大聖意乎?況亙古自分南北,主上恭勤,英俊並用,期必復古。盍思久遠之謀,歸我二帝、太母,復我土疆,使南北赤子無致塗炭,亦足以慰先大聖之靈,幸執事者贊之。」思謀沉思曰:「君言是也,歸當盡達之。」已而粘罕至,曰:「比上國遣使來,問其意指,多不能對。思謀傳侍郎語欲議和,決非江南情實,特侍郎自為此言耳。」倫曰:「使事有指,
 不然來何為哉?人定者勝天,天定亦能勝人,惟元帥察之。」粘罕不答。是後,宇文虛中、魏行可、洪皓、崔縱、張邵相繼入使,皆拘之。



 紹興二年,粘罕忽自至館中與倫議和,縱之歸報。是秋,倫至臨安,入對,言金人情偽甚悉,帝優獎之。除右文殿修撰,主管萬壽觀,官其二弟一侄。時方用兵討劉豫,和議中格。三年,韓肖冑使金還,金遣李永壽、王詡繼至。二人驕倨,以倫充伴使,倫與道雲中舊故,驕倨少損,遂拜詔。訖事,倫復請祠。劉光世求倫參議軍
 事,辭。宰相趙鼎請召倫赴都堂稟議,倫陳進取之策,不合,復請祠。



 七年春,徽宗及寧德後訃至,復以倫為徽猷閣待制,假直學士,充迎奉梓宮使,以朝請郎高公繪副之。入辭,帝使倫謂金左副元帥昌曰:「河南地,上國既不有,與其付劉豫,曷若見歸?」倫奉詔以行,因附進太后、欽宗黃金各二百兩,仍以金帛賜宇文虛中、朱弁、孫傅、張叔夜家屬之在金國者。



 倫至睢陽,劉豫館之,疑有他謀,移文取國書。倫報曰:「國書須見金主面納,若所銜命,則
 祈請梓宮也。」豫肋取不已。會迓者至,渡河見撻懶於涿州,具言豫邀索國書無狀,且謂:「豫忍背本朝,他日安保其不背大國。」



 是年冬,豫廢。倫及高公繪還,左副元帥昌送倫等曰:「好報江南,自今道塗無壅,和議可以平達。」倫入對,言金人許還梓宮及太后,又許歸河南地,且言廢豫之謀由己發之。帝大喜,賜予特異。



 初,倫既見昌,昌遣使偕倫入燕見金主但,首謝廢豫,次致使指。金主始密與群臣定議許和,遂遣倫還,且命太原少尹烏陵思謀、
 太常少卿石慶來議事。至行在,倫往來館中計事。八年秋,以端明殿學士再使金國,知閣門事藍公佐為之副,申問諱日,期還梓宮。倫辭,引至都堂授使指二十餘事。既至金國,金主但為設宴三日,遣簽書宣徽院事蕭哲、左司郎中張通古為江南詔諭使,偕倫來。



 朝論以金使肆嫚,抗論甚喧,多歸罪倫。十一月,倫至行在,引疾請祠,不許,趣赴內殿奏事。時哲等驕倨,受書之禮未定。御史中丞勾龍如淵詣都堂與秦檜議,召倫責曰:「公為使通
 兩國好,凡事當於彼中反復論定,安有同使至而後議者?」倫泣曰:「倫涉萬死一生,往來虎口者數四,今日中丞乃責倫如此。」檜等共解之曰:「中丞無他,亦欲激公了此事耳。」倫曰:「此則不敢不勉。」倫見通古,以一二策動之。通古恐,遂議以檜見金使於其館,受書以歸。金許歸梓宮、太母及河南地。



 九年春,賜倫同進士出身、端明殿學士、簽書樞密院事,充迎梓宮、奉還兩宮、交割地界使,既又以倫為東京留守兼開封尹。倫至東京,見金右副元帥
 兀朮,交割地界,兀朮還燕。五月,倫自汴京赴金國議事。初,兀朮還,密言於金主曰:「河南地本撻懶、宗盤主謀割之與宋,二人必陰結彼國。今使已至汴,勿令逾境。」倫有雲中故吏隸兀朮者潛告倫,倫即遣介具言於朝,乞為備。兀朮遂命中山府拘倫,殺宗盤及撻懶。



 十月,倫始見金主於御子林,致使指。金主悉無所答,令其翰林待制耶律紹文為宣勘官,問倫:「知撻懶罪否?」倫對:「不知。」又問:「無一言及歲幣,反來割地,汝但知有元帥,豈知有上國
 邪?」倫曰:「比蕭哲以國書來,許歸梓宮、太母及河南地,天下皆知上國尋海上之盟,與民休息,使人奉命通好兩國耳。」既就館,金主復遣紹文諭倫曰:「卿留雲中已無還期,及貸之還,曾無以報,反間貳我君臣耶?」乃遣藍公佐先歸,論歲貢、正朔、誓表、冊命等事,拘倫以俟報;已而遷之河間,遂不復遣。



 十年,金渝盟,兀朮等復取河南。倫居河間六載,至十四年,金欲以倫為平灤三路都轉運使,倫曰:「奉命而來,非降也。」金益脅以威,遣使來趣,倫拒益
 力。金杖其使,俾縊殺之。倫厚賂使少緩,遂冠帶南向,再拜慟哭曰:「先臣文正公以直道輔相兩朝,天下所知。臣今將命被留,欲污以偽職,臣敢愛一死以辱命!」遂就死,年六十一。於是河間地震,雨雹三日不止,人皆哀之。詔贈通議大夫,賜其家金千兩、帛千匹。子述與從兄遵間入金境,至河間,得倫骨以歸,官給葬事。後謚愍節。



 宇文虛中,字叔通,成都華陽人。登大觀三年進士第,歷官州縣,入為起居舍人、國史編修官、同知貢舉,遷中書
 舍人。



 宣和間,承平日久,兵將驕惰,蔡攸、童貫貪功開邊,將興燕雲之役,引女直夾攻契丹,以虛中為參議官。虛中以廟謨失策,主帥非人,將有納侮自焚之禍,上書言:「用兵之策,必先計強弱,策虛實,知彼知己,當圖萬全。今邊圉無應敵之具,府庫無數月之儲,安危存亡,系茲一舉,豈可輕議?且中國與契丹講和,今逾百年,自遭女真侵削以來,向慕本朝,一切恭順。今舍恭順之契丹,不羈縻封殖,為我蕃籬,而遠逾海外,引強悍之女真以為鄰
 域。女真藉百勝之勢,虛喝驕矜,不可以禮義服,不可以言說誘,持卞莊兩斗之計,引兵逾境。以百年怠惰之兵,當新銳難抗之敵;以寡謀安逸之將,角逐於血肉之林。臣恐中國之禍未有寧息之期也。」王黼大怒,降集英殿修撰,督戰益急。虛中建十一策,上二十議,皆不報。



 斡離不、粘罕分道入侵,童貫聞之,憂懣不知所為,即與虛中及範訥等謀,以赴闕稟議為遁歸之計,以九月至汴京。是日,報粘罕迫太原,帝顧虛中曰:「王黼不用卿言,今金
 人兩路並進,事勢若此,奈何?」虛中奏:「今日宜先降詔罪己,更革弊端,俾人心悅,天意回,則備御之事,將帥可以任之。」即命虛中草詔,略曰:「言路壅蔽,面諛日聞,恩幸持權,貪饕得志,上天震怒而朕不悟,百姓怨懟而朕不知。」又言出宮人、罷應奉等事。帝覽詔曰:「今日不吝改過,可便施行。」虛中再拜泣下。



 時守禦難其人,欲召熙河帥姚古與秦鳳帥種師道,令以本路兵會鄭、洛,外援河陽,內衛京城。帝顧謂虛中曰:「卿與姚古、師道如兄弟,宜以一
 使名護其軍。」遂以虛中為資政殿大學士、軍前宣諭使。虛中檄趣姚古、師道兵馬,令直赴汴京應援。金騎至城下,放兵掠至鄭州,為馬忠所敗,遂收斂為一。西路稍通,師道、姚古及其它西兵並得達汴京。虛中亦馳歸,收合散卒,得東南兵二萬餘人。以便宜起致仕官李邈,令統領於汴河上從門外駐兵。



 會姚平仲劫金營失利,西兵俱潰,金人復引兵逼城下,虛中縋而入。欽宗欲遣人奉使,辨劫營非朝廷意,乃姚平仲擅興兵,大臣皆不肯行。
 虛中承命即往都亭驛,見金使王汭,因持書復議和。渡濠橋,道逢甲騎如水,雲梯、鵝洞蔽地,冒鋒刃而進。既至敵營,露坐風埃,自巳至申,金人注矢露刃,周匝圍繞,久乃得見康王於軍中。次日,侍王至金幕,見二太子者語不遜,禮節倨傲。抵暮,遣人隨虛中入城,要越王、李邦彥、吳敏、李綱、曹晟及金銀、騾馬之類,又欲御筆書定三鎮界至,方退軍。



 令虛中再往,必請康王歸。虛中再出,明日,從康王還,除簽書樞密院事。自是又三往,金人固要三
 鎮,虛中泣下不言,金帥變色,虛中曰:「太宗殿在太原,上皇祖陵在保州,詎忍割棄。」諸酋曰:「樞密不稍空,我亦不稍空。」如中國人稱「脫空」,遂解兵北去。言者劾以議和之罪,罷知青州,尋落職奉祠。建炎元年,竄韶州。



 二年,詔求使絕域者,虛中應詔,復資政殿大學士,為祈請使,楊可輔副之。尋又以劉誨為通問使,王貺為副。明年春,金人並遣歸,虛中曰:「奉命北來祈請二帝,二帝未還,虛中不可歸。」於是獨留。虛中有才藝,金人加以官爵,即受之,
 與韓昉輩俱掌詞命。明年,洪皓至上京,見而甚鄙之。累官翰林學士、知制誥兼太常卿,封河內郡開國公,書金太祖《睿德神功碑》,進階金紫光祿大夫,金人號為「國師」。然因是而知東北之士皆憤恨陷北,遂密以信義結約,金人不覺也。



 金人每欲南侵,虛中以費財勞人,遠征江南荒僻,得之不足以富國。王倫歸,言:「虛中奉使日久,守節不屈。」遂詔福州存恤其家,仍命其子師瑗添差本路轉運判官。檜慮虛中沮和議,悉遣其家往金國以牽制之。
 金皇統四年,轉承旨,加特進,遷禮部尚書,承旨如故。



 虛中恃才輕肆,好譏訕,凡見女真人,輒以「礦鹵」目之,貴人達官,往往積不平。虛中嘗撰宮殿榜署,本皆嘉美之名,惡之者摘其字以為謗訕,由是媒薛成其罪,遂告虛中謀反。鞫治無狀,乃羅織虛中家圖書為反具。虛中曰:「死自吾分。至於圖籍,南來士大夫家家有之,高士談圖書尤多於我家,豈亦反邪?」有司承順風旨,並殺士談。虛中與老幼百口同日受焚死,天為之晝晦。淳熙間,贈開府
 儀同三司,謚肅愍,賜廟仁勇,且為置後,是為紹節,官至簽書樞密院事。開禧初,加贈少保,賜姓趙氏。有文集行於世。



 湯思退,字進之,處州人。紹興十五年,以右從政郎授建州政和縣令,試博學宏詞科,除秘書省正字。自是登郎曹,貳中秘,秉史筆。



 二十五年,由禮部侍郎除端明殿學士、簽書樞密院事,未幾參大政。先是,秦檜當國,惡直醜正,必不異和議,不摘己過,始久於用。時思退名位日進,
 檜病篤,招參知政事董德元及思退至臥內,屬以後事,各贈黃金千兩。德元慮其以我為自外,不敢辭,思退慮其以我期其死,不敢受。高宗聞之,以思退不受金,非檜黨,信用之。二十六年,除知樞密院事。明年,拜尚書右僕射;又二年,進左僕射。明年,侍御史陳俊卿論其「挾巧詐之心,濟傾邪之術,觀其所為,多效秦檜,蓋思退致身,皆檜父子恩也。」遂罷,以觀文殿大學士奉祠。



 隆興元年,符離師潰,召思退復相。諫議大夫王大寶上章論之,不報。
 金帥紇石烈志寧遺書三省、樞密院,索海、泗、唐、鄧四郡。思退欲與和,遣淮西安撫司干辦公事盧仲賢加樞密院計議、編修官,持報書以往。既行,上戒勿許四郡。仲賢至宿州,僕散忠義懼之以威,仲賢皇恐,言歸當稟命,遂以忠義為三省、樞密院書來。上猶欲止割海、泗,思退遽奏以吏部侍郎王之望為通問使,知閣門事龍大淵副之,將割棄四州。張浚在揚州聞之,遣其子栻入奏仲賢辱國無狀。上怒,會侍御周操論仲賢不應擅許四郡,下
 大理究問,召浚赴行在。十二月,拜思退左僕射,浚右僕射。



 二年,浚以金未可與和,請上幸建康,圖進兵。上手批王之望等並一行禮物並回,詔荊、襄、川、陜嚴邊備,竄仲賢郴州。思退恐,奏請以宗社大計,奏稟上皇而後從事。上批示三省曰:「金無禮如此,卿猶欲言和。今日敵勢,非秦檜時比,卿議論秦檜不若。」思退大駭,陰謀去浚,遂令之望、大淵驛疏兵少糧乏,樓櫓、器械未備,人言委四萬眾以守泗州,非計。上頗惑之,乃命浚行邊,還兵罷招納。
 浚力乞罷政,許之。上命思退作書,許金四郡。



 既而金專事殺戮,上意中悔,思退復密令孫造諭敵以重兵脅和。上聞有敵兵,命建康都統王彥等御之,仍命思退督江、淮軍,辭不行。僕散忠義自清河口渡淮,言者極論思退急和撤備之罪,遂罷相,尋責居永州。於是太學生張觀等七十二人上書,論思退、王之望、尹穡等奸邪誤國,招致敵人,請斬之。思退憂悸死。



 思退始終與張浚不合,浚以雪恥復仇為志,思退每借保境息民為口實,更勝迭
 負,思退之計迄行,然終以不免。敵既得海、泗、唐、鄧,又索商、秦,皆思退力也。



 論曰:以白時中之孱佞,徐處仁之奸細,馮澥之邪枉,湯思退之巧詐,而排楊時,誤李綱,異張浚,其識趣可見矣,雖有小善,何足算哉。王倫雖以無行應使,往來虎口,屢被拘留,及金人脅之以官,竟不受,見迫而死,悲夫!較之虛中即受其命,為之定官制、草赦文、享富貴者,大有間矣。卒以輕肆譏諷,覆其家族,真不知義命者哉。雖雲冤
 死,亦自取焉。律以豫讓之言,益可愧哉。



\end{pinyinscope}