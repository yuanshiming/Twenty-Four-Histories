\article{列傳第五十四}

\begin{pinyinscope}

 尹洙孫甫謝絳子景溫葉清臣楊察



 尹洙,字師魯,河南人。少與兄源俱以儒學知名。舉進士,調正平縣主簿。歷河南府戶曹參軍、安國軍節度推官、
 知光澤縣。舉書判拔萃,改山南東道節度掌書記、知伊陽縣,有能名。用大臣薦,召試,為館閣校勘,遷太子中允。會範仲淹貶,敕榜朝堂,戎百官為朋黨。洙上奏曰:「仲淹忠亮有素,臣與之義兼師友,則是仲淹之黨也。今仲淹以朋黨被罪,臣不可茍免。」宰相怒,落校勘,復為掌書記、監唐州酒稅。



 西北久安,洙作《敘燕》、《息戍》二篇,以為武備不可弛。



 《敘燕》曰:



 戰國世,燕最弱。二漢叛臣,持燕挾虜,蔑能自固,以公孫伯珪之強,卒制於袁氏。獨慕容乘石虎
 亂,乃並趙。雖勝敗異術,大概論其強弱,燕不能加趙。趙、魏一,則燕固不敵。唐三盜連衡百餘年,虜未嘗越燕侵趙、魏,是燕獨能支虜也。自燕入於契丹,勢日熾大。顯德世,雖復三關,尚未盡燕南地。國初,始與並合,勢益張,然止命偏師備御。王師伐蜀伐吳,泰然不以兩河為顧,是趙、魏足以制之明矣。並寇既平,悉天下銳專力契丹,不能攘尺寸地。頃嘗以百萬眾駐趙、魏,訖敵退莫敢抗,世多咎其不戰。然我眾負城,有內顧心,戰不必勝,不勝則
 事亟矣,故不戰未嘗咎也。



 原其弊,在兵不分。設兵為三,壁於爭地,掎角以疑其勢,設覆以待其進。邊壘素固,驅民以守之,俾其兵頓堅城之下,乘間夾擊,無不勝矣。蓋兵不分有六弊:使敵蓄勇以待戰,無他枝梧,一也;我眾則士怠,二也;前世善將兵者必問幾何,今以中才盡主之,三也;大眾儻北,彼遂長驅無復顧忌,四也;重兵一屬,根本虛弱,纖人易以干說,五也;雖委大柄,不無疑貳,復命貴臣監督,進退皆由中御,失於應變,六也。兵分則盡
 易其弊,是有六利也。



 勝敗兵家常勢。悉內以擊外,失則舉所有以棄之,苻堅淝水、哥舒翰潼關是也。是則制敵在謀不在眾。以趙、魏、燕南,益以山西,民足以守,兵足以戰。分而帥之,將得專制,就使偏師挫衄,他眾尚奮,詎能系國安危哉?故師覆於外而本根不搖者,善敗也。昔者六國各有地千里,師敗於秦,散而復振,幾百戰猶未及其都,守國之固也。陳勝、項梁舉關東之眾,朝敗而夕滅,新造之勢也。以天下之廣謀其國,不若千里之固,而襲
 新造之勢,僥幸於一戰,庸非惑哉?兵既久弭,士大夫誦習,謂百世不復用,非甚妄者不談。然兵果廢則已,儻後世復用之,鑒此少以悟世主,故跡其勝敗云。



 《息戍》曰:



 國家割棄朔方,西師不出三十年,而亭徼千里,環重兵以戍之。雖種落屢擾,實時輯定,然屯戍之費,亦已甚矣。西戎為寇,遠自周世,西漢先零,東漢燒當,晉氐、羌,唐禿發,歷朝侵軼,為國劇患。興師定律,皆有成功,而勞弊中國,東漢尤甚,費用常以億計。孝安世,羌叛十四年,用二百
 四十億。永和末,復經七年,用八十餘億。及段紀明,用裁五十四億,而剪滅殆盡。今西北涇原、邠寧、秦鳳、鄜延四帥,戍卒十餘萬。一卒歲給,無慮二萬,騎卒與冗卒,較其中者,總廩給之數,恩賞不在焉,以十萬較之,歲用二十億。白靈武罷兵,計費六百餘億,方前世數倍矣。平世屯戍,且猶若是,後雖有他警,不可一日輟去,是十萬眾,有增而無損期也。國家厚利募商入粟,傾四方之貨,然無水漕之運,所挽致亦不過被邊數郡爾。歲不常登,廩有
 常給,頃年亦嘗稍匱矣。儻其乘我薦饑,我必濟師,饋饟當出於關中,則未戰而西垂已困,可不慮哉?



 按唐府兵,上府千二百人,中府千人,下府八百人。為今之計,莫若籍丁民為兵,擬唐置府,頗損其數。又今邊鄙雖有鄉兵之制,然止極塞數郡,民籍寡少,不足備敵。料京兆西北數郡,上戶可十餘萬,中家半之,當得兵六七萬。質其賦無他易,賦以帛名者不易以五穀,畜馬者又蠲其雜徭。民幸於庇宗,樂然隸籍。農隙講事,登材武者為什長、隊
 正,盛秋旬閱,常若寇至。以關內、河東勁兵傅之,盡罷京師禁旅,慎簡守帥,分其統,專其任。分統則兵不重,專任則將益勵,堅其守備,習其形勢,積粟多,教士銳,使虜眾無隙可窺,不戰而懾。《兵志》所謂「無恃其不來,恃吾有以待之」,其廟勝之策乎?



 又為《述享》、《審斷》、《原刑》、《敦學》、《矯察》、《考績》、《廣諫》,凡《雜議》共九篇上之。



 趙元昊反,大將葛懷敏闢為經略判官。洙雖用懷敏闢,尤為韓琦所深知。頃之,劉平、石元孫戰敗,朝廷以夏竦為經略、安撫使,範仲淹、韓
 琦副之,復以洙為判官。洙數上疏論兵,請便殿召對二府大臣議邊事,及講求開寶以前用兵故實,特出睿斷,以重邊計。又請減並柵壘,召募土兵,省騎軍,增步卒。又上鬻爵令。時詔問攻守之計,竦具二策,令琦與洙詣闕奏之。帝取攻策,以洙為集賢校理。洙遂趨延州謀出兵,而仲淹持不可。還至慶州,會任福敗於好水川,因發慶州部將劉政銳卒數千,趨鎮戎軍赴救,未至,賊引去。夏竦奏洙擅發兵,降通判濠州。當時言者謂福之敗,由參
 軍耿傅督戰太急。後得傅書,乃戒福使持重,毋輕進。洙以傅文吏,無軍責而死於行陣,又為時所誣,遂作《憫忠》、《辨誣》二篇。



 未幾,韓琦知秦州,闢洙通判州事,加直集賢院。上奏曰:



 漢文帝盛德之主,賈誼論當時事勢,猶云可為慟哭。孝武帝外制四夷,以強主威,徐樂、嚴安尚以陳勝亡秦、六卿篡晉為戒。二帝不以危亂滅亡為諱,故子孫保有天下者十餘世。秦二世時,關東盜起。或以反者聞,二世怒,下吏;或曰逐捕今盡,不足憂,乃悅。隋煬帝時,
 四方兵起,左右近臣皆隱賊數,不以實聞,或言賊多者,輒被詰。二帝以危亂滅亡為諱,故秦、隋宗社數年為丘墟。陛下視今日天下之治,孰與漢文?威制四夷,孰與漢武?國家基本仁德,陛下慈孝愛民,誠萬萬於秦、隋矣。至於西有不臣之虜,北有強大之鄰,非特閭巷盜賊之勢也。



 自西夏叛命四年,並塞苦數擾,內地疲遠輸。兵久於外而休息無期,卒有乘弊而起。《兵法》所謂「雖有智者,不能善其後」。當此之時,陛下宜夙夜憂懼,所以慮事變而
 塞禍源也。陛下延訪邊事,容納直言,前世人主。勤勞寬大,未有能遠過者。然未聞以宗廟為憂,危亡為懼,此賤臣所以感憤於邑而不已也。何者?今命令數更,恩寵過濫,賜與不節。此三者,戒之慎之,在陛下所行爾,非有難動之勢也。而因循不革,弊壞日甚。臣謂陛下不以宗廟為憂、危亡為懼者,以此。



 未命令者,人主所以取信於下也。異時民間,朝廷降一命令,皆竦視之;今則不然,相與竊語,以為不久當更,既而信然,此命令日輕於下也。命
 令輕,則朝廷不尊矣。又聞群臣有獻忠謀者,陛下始甚聽之,年復一人沮之,則意移矣。忠言者以信之不能終,頗自詘其謀,以為無益,此命令數更之弊也。



 夫爵賞,陛下所持之柄也。近時外戚、內臣以及士人,或因緣以求恩澤,從中而下謂之「內降」。臣聞唐氏政衰,或母后專制,或妃主擅朝,樹恩私黨,名為「斜封」。今陛下威柄自出,外戚、內臣賢而才者,當與大臣公議而進之,何必襲「斜封」之弊哉。且使大臣從之,則壞陛下綱紀;不從,則沮陛下
 德音。壞綱紀,忠臣所不忍為;沮德音,則威柄輕於上。且盡公不阿,朝廷所以責大臣。今乃自以私暱撓之,而欲責大臣之不私,難矣。此恩寵過濫之弊也。



 夫賜予者,國家所以勤功也。比年以來,嬪御及伶官、太醫之屬,賜予過厚。民間傳言,內帑金帛,皆祖宗累朝積聚。陛下用之,不甚愛惜,今之所存無幾。疏遠之人,誠不能知內府豐匱之數,但見取於民者日煩,即知畜於公帑者不厚。臣亦知國家自西方宿兵,用度浸廣,帑藏之積,未必悉為
 賜予所費,然下民不可家至而戶曉,獨見陛下行事感動爾。往歲聞邊將王珪,以力戰賜金,則無不悅服;或見優人所得過厚,則往往憤嘆。人情不可不察,此賜予不節之弊也。



 臣所論三事,皆人人所共知,近臣從諛而不言,以至今日。方今非獨四夷之為患,朝政日弊而陛下不寤,人心日危而陛下不知。故臣願先正於內,以正於外。然後忠謀漸進,紀綱漸舉,國用漸足,士心漸奮。邊境之患,庶乎息矣。惟深察秦、隋惡聞忠言所以亡,遠法漢
 王不諱危亂所以存,日親盛德,與民更始,則天下幸甚。



 仁宗嘉納之。



 改太常丞、知涇州。以右司諫、知渭州兼領涇原路經略公事。會鄭戩為陜西四路都總管,遣劉滬、董士廉城水洛,以通秦、渭援兵。洙以為前此屢困於賊者,正由城砦多而兵勢分也。今又益城,不可,奏罷之。時戩已解四路。而奏滬等督役如故。洙不平,遣人再召滬,不至;命張忠往代之,又不受。於是諭狄青械滬、士廉下吏。戩論奏不已,卒徙洙慶州而城水洛。又徙晉州,遷起
 居舍人、直龍圖閣、知潞州。會士廉詣闕上書訟洙,詔遣御史劉湜就鞫,不得他罪。而洙以部將孫用由軍校補邊,自京師貸息錢到官,亡以償。洙惜其才可用,恐以犯法罷去,嘗假公使錢為償之,又以為嘗自貸,坐貶崇信軍節度副使,天下莫不以為湜文致之也。徙監均州酒稅,感疾,沿牒至南陽訪醫,卒,年四十七。嘉祐中,宰相韓琦為洙言,乃追復故官,及官其子構。



 洙內剛外和,博學有識度,尤深於《春秋》。自唐末歷五代,文格卑弱。至宋初,
 柳開始為古文,洙與穆修復振起之。其為文簡而有法,有集二十七卷。自元昊不庭,洙未嘗不在兵間,故於西事尤練習。其為兵制之說,述戰守勝敗,盡當時利害。又欲訓土兵代戍卒,以減邊費,為御戎長久之策,皆未及施為。而元昊臣,洙亦去而得罪矣。



 孫甫字之翰,許州陽翟人。少好學,日誦數千言,慕孫何為古文章。初舉進士,得同學究出身,為蔡州汝陽縣主簿。再舉進士及第,為華州推官。轉運使李紘薦其材,遷
 大理寺丞、知絳州翼城縣。杜衍闢為永興司錄,凡吏職,纖末皆倚辦甫。甫曰:「待我以此,可以去矣。」衍聞之,不復以小事屬甫。衍與宴語,甫必引經以對,言天下賢俊,歷評其才性所長。衍曰:「吾闢屬官,得益友。」諸生亦多從甫學問。



 徙知永昌縣,監益州交子務,再遷太常博士。蜀用鐵錢,民苦轉貿重,故設書紙代錢,以便市易。轉運使以偽造交子多犯法,欲廢不用。甫曰:「交子可以偽造,錢亦可以私鑄,私鑄有犯,錢可廢乎?但嚴治之,不當以小
 仁廢大利。」後卒不能廢。衍為樞密副使,薦於朝,授秘閣校理。



 是歲,詔三館臣僚言事。甫進十二事,按祖宗故實,校當世之治有所不逮者,論述以為諷諫,名《三聖政範》。改右正言。時河北降赤雪,河東地震五六年不止,甫推《洪範五行傳》及前代變驗,上疏曰:「赤雪者,赤眚也,人君舒緩之應。舒緩則政事弛,賞罰差,百官廢職,所以召亂也。晉太康中,河陰降赤雪。時武帝怠於政事,荒宴後宮。每見臣下,多道常事,不及經國遠圖,故招赤眚之怪,終
 致晉亂。地震者,陰之盛也。陰之象,臣也,後宮也,四夷也。三者不可過盛,過盛則陰為變而動矣。忻州趙分,地震六年。每震,則有聲如雷,前代地震,未有如此之久者。惟唐高宗本封於晉,及即位,晉州經歲地震。宰相張行成言,恐女謁用事,大臣陰謀,宜制於未萌。其後武昭儀專恣,幾移唐祚。天地災變,固不虛應,陛下救紓緩之失,莫若自主威福,時出英斷,以懾奸邪,以肅天下。救陰盛之變,莫若外謹戎備,內制後宮。謹戎備,則切責大臣,使之
 預圖兵防,熟計成敗;制後宮,則凡掖庭非典掌御幸者,盡出之,且裁節其恩,使無過分,此應天之實也。」時契丹、西夏稍強,後宮張修媛寵幸,大臣專政,甫以此諫焉。



 又言:「修媛寵恣市恩,禍漸已萌。夫後者,正嫡也,其餘皆婢妾爾。貴賤有等,用物不宜過僭。自古寵女色,初不制而後不能制者,其禍不可悔。」帝曰:「用物在有司,朕恨不知爾。」甫曰:「世謂諫臣耳目官,所以達不知也。若所謂前世女禍者,載在書史,陛下可自知也。」



 夏國乞盟,甫上一利、
 曰害曰:「宿兵以來,國用空耗。今若與之約和,則邊兵可減,科斂可省。其為利一也。始,契丹聲言,嘗遣使諭西人使臣中國。今和議既成,必恃其功。去歲有割地之請,朝廷已增歲賂,若更有求,將安拒之?其為害一也。自承平四十年,武事不飭,及邊鄙有警,而用不習之將,不練之兵,故久無成功。然比來邊臣中材謀勇健者,往往復出,方在講訓不懈,以張中國之威。一旦因議和弛備,復如曩日,緩急必不可用。其為害二也。自元昊拒命,終不敢
 深入關中者,以唃廝囉等族不附,慮為後患也。今中國與之和,獲歲遺之厚,彼必專力以制二蕃,強大之勢,自茲為始。其為害三也。且朝廷恃久安之勢,法令紀綱,弛而不葺。及西戎累敗,王師始議更張,以救前弊。今見戎人請和,茍貪無事,他時之患,不可救矣。其為害四也。凡利害之機,願陛下熟圖之。」



 又言:「張子奭使夏州回,元昊復稱臣,然乞歲賣青鹽十萬石,兼欲就京師互市諸物,仍求增歲給之數。臣以謂西鹽數萬石,其直不下錢十
 餘萬緡。況朝廷已許歲賜二十五萬,若又許其賣鹽,則與遺契丹物數相當。使契丹聞之,則貪得之心生矣。況自德明之時,累乞放行青鹽,先帝以其亂法,不聽。及請之不已,追德明弟入質而許之,是則以彼難從之事,杜其意也。蓋鹽,中國之大利,又西戎之鹽,味勝解池所出,而出產無窮。既開其禁,則流於民間,無以堤防矣。兼聞張子奭言,元昊自拒命以來,收結人心,鈔掠所得,旋給其眾,兵力雖勝,用度隨窘。當此之時,尤宜以計困之,安
 得汲汲與和,曲徇其請乎?」



 時陜西經略招討副使韓琦、判官尹洙還朝,甫建議請詔琦等,條四路將官能否,為上、中、下三等,黜其最下者。保州兵變前,有告者,大臣不時發之。甫因言樞密使副當得罪,使,乃杜衍也。邊將劉滬城水洛於渭州,總管尹洙以滬違節度,將斬之。大臣稍主洙議,甫以謂:「水洛通秦、渭,於國家為利,滬不可罪。」由是罷洙而釋滬。衍屢薦甫,洙與甫素善者,而甫不少假借,其鯁亮不私如此。



 甫嘗言參知政事陳執中不學
 亡術,不可用。帝難之,由是求補外,不許。其後奏丁度因對求進用,帝曰:「度未嘗請也。」度乞與甫辯,且指甫為宰相杜衍門人。乃以右司諫出知鄧州,徙安州,歷江東、兩浙轉運使。



 範仲俺知杭州,多以便宜從事。甫曰:「範公,大臣也。吾屈於此,則不得伸於彼矣。」一切繩之以法,然退未嘗不稱其賢。再遷尚書兵部員外郎,改直史館、知陜州,徙晉州。為河東轉運使、三司度支副使,遷刑部郎中、天章閣待制、河北都轉運使,留為侍讀。卒,特贈右諫議
 大夫。



 甫性勁果,善持論,有文集七卷,著《唐史記》七十五卷。每言唐君臣行事,以推見當時治亂,若身履其間,而聽者曉然,如目見之。時人言:「終日讀史,不如一日聽孫論也。」《唐史》藏秘閣。



 謝絳,字希深,其先陽夏人。祖懿文,為杭州鹽官縣令,葬富陽,遂為富陽人。父濤,以文行稱,進士起家,為梓州榷鹽院判官。李順反成都,攻陷州縣,濤嘗畫守禦之計。賊平,以功遷觀察推官,權知華陽縣。亂亡之後,田廬荒廢,
 詔有能占田而倍入租者與之,於是腴田悉為豪右所占,流民至無所歸。濤收詔書,悉以田還主。改秘書省著作佐郎、知興國軍。還,以治行召對長春殿,命試學士院。會契丹入寇,真宗議親征,時曹、濮多盜,而契丹聲言趨齊、鄆,以濤知曹州。屬縣賦稅多輸睢陽助兵食,是歲霖潦,百姓苦於轉送,濤悉留不遣。奏曰:「江、淮漕運,日過睢陽,可取以餉軍。願留曹賦繇廣濟河以饋京師。」轉運使論以為不可,詔從濤奏。嘗使蜀還,舉所部官三十餘人。
 宰相疑以為多,濤曰:「有罪,願連坐之。」奉使舉官連坐,自濤始。久之,用馮拯薦,復召試,以尚書兵部員外郎直史館,遂兼侍御史知雜事。真宗山陵靈駕所經道路,有司請悉壞城門、廬舍,以過車輿象物。濤言:「先帝車駕封祀,儀物大備,猶不聞有所毀撤,且遺詔從儉薄。今有司治明器侈大,以勞州縣,非先帝意,願下少府裁損之。」進直昭文館,累官至太子賓客。



 絳以父任試秘書省校書郎,舉進士中甲科,授太常寺奉禮郎、知汝陰縣。善議論,喜談
 時事,嘗論四民失業,累數千言。天禧中,上疏謂宋當以土德王天下。時大理寺丞董行父,請用天為統,以金為德。詔兩制議,皆言:「用土德,則當越唐上承於隋;用金德,則當越五代紹唐。而太祖實受終周室,豈可弗遵傳繼之序?」絳、行父議皆黜不用。



 楊億薦絳文章,召試,擢秘閣校理、同判太常禮院。丁母憂,服除,仁宗即位,遷太常博士。用鄭氏《經》、唐故事議宣祖非受命祖,不宜配享感生帝,請以真宗配之。翰林學士承旨李維以為不可。尋出
 通判常州。天聖中,天下水旱、蝗起,河決滑州,絳上疏曰:



 去年京師大水,敗民廬舍,河渠暴溢,幾冒城郭;今年苦旱,百姓疫死,田穀焦槁,秋成絕望:此皆大異也。按《洪範》、京房《易傳》皆以為簡祭祀,逆天時,則水不順下;政令逆時,水失其性,則壞國邑,傷稼穡;顓事者知,誅罰絕理,則大水殺人;欲德不用,茲謂張,厥災荒;上下皆蔽,茲謂隔,其咎旱:天道指類示戒,大要如此。陛下夙夜勤苦,思有以上塞時變,固宜策告殃咎,變更理化,下罪己之詔,修
 順時之令,宣群言以導壅,斥近幸以損陰。而聖心優柔,重在改作,號令所發,未聞有以當天心者。



 夫風雨、寒暑之於天時,為大信也;信不及於物,澤不究於下,則水旱為沴。近日制命,有信宿輒改,適行遽止,而欲風雨以信,其可得乎?天下之廣,萬幾之眾,不出房闥,豈能盡知?而在廷之臣,未聞被數刻之召,吐片言之善,朝夕左右,非恩澤即佞幸,上下皆蔽,其應不虛。



 昔兩漢日食、地震、水旱之變,則策免三公,以示戒懼。陛下進用丞弼,極一
 時之選,而政道未茂,天時未順,豈大臣輔佐不明邪?陛下信任不篤邪?必若使之,宜推心責成,以極其效;謂之不然,則更選賢者。比來奸邪者易進,守道者數窮,政出多門,俗喜由徑。聖心固欲盡得天下之賢能,分職受業;而宰相方考賢進吏,無敢建白。欲德不用之應,又可驗矣。



 今陽驕莫解,蟲孽漸熾,河水妄行。循故道之跡,行尋常之政,臣恐不足回靈意、塞至戒。古者,穀不登則虧膳,災屢至則降服,兇年不塗塈。願下詔引咎,損太官之膳,避
 路寢之朝,許士大夫斥諱上聞,譏切時病。罷不急之役,省無名之斂,勿崇私恩,更進直道,宣德流化,以休息天下。至誠動乎上,大惠浹於下,豈有時澤之艱哉!



 仁宗嘉納之。



 會修國史,以絳為編修官,史成,遷祠部員外郎、直集賢院。時濤官西京,且老矣,因請便養,通判河南府。又論:「唐室麗正、史官之局,並在大明、華清宮內。太宗皇帝肇修三館,更立秘閣於升龍門左,親為飛白書額,作贊刻石閣下。景德中,圖書浸廣,真宗皇帝益以內帑四庫。
 二聖數嘗臨幸,親加勞問,遞宿廣內者,有不時之召。人人力道術、究藝文,知天子尊禮甚勤,而名臣高位,繇此其選也。往者遭遘延燔,未遑中葺,或引兩省故事,別建外館,直舍卑喧,民簷叢接。大官衛尉,供擬滋削,虧體傷風,莫茲為甚。陛下未嘗迂翠華、降玉趾,寥寥冊府,不聞輿馬之音,曠有日矣。議者以謂慕道不篤於古,待士少損於前。士無延訪之勤,而因循相尚,不自激策,文雅漸弊,竊為聖朝惜之。願闢內館,以恢景德之制。」詔可。



 絳雖
 在外,猶數論事。奏言:「近歲不逞之徒,托言數術,以先生、處士自名,禿巾短褐,內結權幸,外走州邑,甚者矯誣詔書,傲忽官吏。請嚴禁止。嘗以墨敕賜封號者,追還之。」



 還權開封府判官,言:



 蝗亙田野,坌人郛郭,跳擲官寺,井郾皆滿。魯三書螟,《穀梁》以為哀公用田賦虐取於民。朝廷斂弛之法,近於廉平,以臣愚所聞,似吏不甚稱而召其變。凡今典城牧民,有顓方面之執:才者掠功取名,以嚴急為術,或辯偽無實,數蒙獎錄;愚者期會簿書,畏首與
 尾。二者政殊,而同歸於弊。



 夫為國在養民,養民在擇吏,吏循則民安,氣和而災息。願先取大州邑數十百,詔公卿以下,舉任州守者,使得自闢屬縣令長,務求術略,不限資考。然後寬以約束,許便宜從事。期年條上理狀,或徙或留,必有功化風跡,異乎有司以資而任之者焉。漢時,詔問京房災異可息之術,房對以考功課吏。臣願陛下博訪理官,除煩苛之命;申敕計臣,損聚斂之役。勿起大獄,勿用躁人,務靜安,守淵默。《傳》曰:「大侵之禮,百官備
 而不制。言省事也。」如此而沴氣不弭,嘉休不至,是靈意言□讕,而聖言罔惑歟。



 會郭皇后廢,絳陳《詩白華》,引申后、褒姒事以諷,辭甚切至。徙三司度支判官,再遷兵部員外郎。上言:「邇來用物滋侈,賜予過制,禁中須索,去年計為緡錢四十五萬。自今春至四月,已及二十餘萬。比詔裁節費用,而有司移文,但求咸平、景德簿書。簿書不存,則無所措置。臣以謂不若推近及遠,遞考歲用而裁節之,不必咸平、景德為準也。」



 初,詔罷織密花透背,禁人服
 用,且云自掖庭始。既而內人賜衣,復取於有司。又後苑作制玳瑁器,索龜筒於市。龜筒,禁物也,民間不得有,而索不已。絳皆論罷之。又言:「號令數變則虧國體,利害偏聽則惑聰明。請者務欲各行,而守者患於不一。請罷內降,凡詔令皆由中書、樞密,然後施行。」因進《聖治箴》五篇。



 以父憂去,服除,擢知制誥,判吏部流內銓、太常禮院。吏部擬官,舊視職田有無,不問多寡,以是不均。絳為核其實,以多寡為差,其有名而無實者皆不用,人以為便。初
 改判禮院為知禮儀事,自張絳建請。



 使契丹,還,請知鄧州。距州百二十里,有美陽堰,引湍水溉公田。水來遠而少,利不及民;濱堰築新土為防,俗謂之墩者,大小又十數,歲數壞,輒調民增築。奸人蓄薪茭,以時其急,往往盜決堰墩,百姓苦之。絳按召信臣六門堰故跡,距城三里,壅水注鉗廬陂,溉田至三萬頃。請復修之,可罷州人歲役,以水與民,未就而卒,年四十六。



 絳以文學知名一時,為人修潔醞藉,所至大興學舍,嘗請諸郡立學。在河
 南修國子學,教諸生,自遠錠而至者數百人。好施宗族,喜賓客,以故,卒之日,家無餘貲。有文集五十卷。子景初、景溫、景平、景回。景平好學,著詩書傳說數十篇,終秘書丞。景回早卒。



 景溫字師直。中進士第,通判汝、莫二州,江東轉運判官。興宣城百丈圩,議者以為罪,降通判、知漣水軍。神宗初,知諫院邵亢直其前事,徙真州,提點江西刑獄。歷京西、淮南轉運使。



 景溫平生未嘗仕中朝,王安石與之善,又
 景溫妹嫁其弟安禮,乃驟擢為侍御史知雜事。安石方惡蘇軾,景溫劾軾向丁憂歸蜀,乘舟商販。朝廷下六路捕逮篙工、水師窮其事,訖無一實。蘇頌等論李定不持母服,景溫察安石指,為辨於前。已而事下臺,景溫難違眾議,始云定當追服。又言薛向不當得侍從,王韶邊奏誣罔,浸失安石意,然猶以嘗助己,但改直史館兼侍讀。不敢拜,出知鄧州。



 逾年,進陜西都轉運使,以不奉司農約束,改知鄧、襄、澶三州,加直龍圖閣,判將作監。轉右諫
 議大夫、知潭州。章惇開五溪,景溫協力拓築,論功進官,召拜禮部侍郎。復出知洪州、應天府、瀛州。



 元祐初,進寶文閣直學士、知開封府。未滿歲,御史中丞劉摯言其非撥煩吏。右司諫王覿言:「瀛州妖婦李自稱事九仙聖母,能與人通語言,談禍福。景溫在郡為所惑,禮餉甚厚,遣十兵挈之入京。數遣子慥至其處;補李婿為小史,使出入官府,崇大聲勢;至縱嬖妾之弟,醉歐市人。為政若此,尚何惜而不加譴。」於是罷知蔡州。



 三年初,置權六曹尚
 書,以為刑部。劉安世復論之,改知鄆州,再歷永興軍。時章惇為相,景溫言元祐大臣改先帝之政,並西夏人偃蹇終未順命,宜罷分畫,以馬跡所至為境。惇用其說,徙知河陽,卒,年七十七。



 葉清臣,字道卿,蘇州長洲人。父參,終光祿卿。清臣幼敏異,好學善屬文。天聖二年,舉進士,知舉劉筠奇所對策,擢第二。宋進士以策擢高第,自清臣始。授太常寺奉禮郎、簽書蘇州觀察判官事。還為光祿寺丞、集賢校理,通
 判太平州、知秀州。入判三司戶部勾院,改鹽鐵判官。



 上言九事:請遣使循行天下,知民疾苦,察吏能否;興太學,選置博士,許公卿大臣子弟補學生;重縣令;諸科舉人取名大義,責以策問;省流外官,無得入仕;聽武臣終三年之喪;罷度僧;廢讀經一業;訓兵練將,慎出令,簡條約。詞多不載。出知宣州,累遷太常丞,同修起居注,判三司鹽鐵勾院,進直史館。



 是冬,京師地震,上疏曰:「天以陽動,君之道也;地以陰靜,臣之道也。天動地靜,主尊臣卑。易
 此則亂,地為之震。乃十二月二日丙夜,京師地震,移刻而止;定襄同日震,至五日不止,壞廬寺,殺人畜,凡十之六。大河之東,彌千五百里而及都下,誠大異也。屬者熒惑犯南斗,治歷者相顧而駭。陛下憂勤庶政,方夏泰寧,而一歲之中,災變仍見。必有下失民望、上戾天意者,故垂戒以啟迪清衷。而陛下泰然不以為異,徒使內侍走四方,治佛事,修道科,非所謂消復之實也。頃範仲淹、餘靖以言事被黜,天下之人,齰舌不敢議朝政者,行將二
 年。願陛下深自咎責,許延忠直敢言之士,庶幾明威降鑒,而善應來集也。」書奏數日,仲淹等皆得近徙。



 會詔求直言,清臣復上疏言大臣專政,仁宗嘉納之。清臣請外,為兩浙轉運副使。並太湖有民田,豪右據上游,水不得洩,而民不敢訴。嘗建請疏盤龍匯、滬瀆港入於海,民賴其利。以右正言知制誥,知審官院,判國子監。



 時陜西用兵,上言:「當今將不素蓄,兵不素練,財無久積。小有邊警,外無驍將,內無重兵。舉西北二垂觀之,若濩落
 大瓠,外示雄壯,其中空洞,了無一物。脫不幸戎馬猖突,腹內諸城,非可以計術守也。自元昊僭竊,因循至於延州之寇,中間一歲矣。而屯戍無術,資糧不充,窮年畜兵,了不足用,連監牧馬,未幾已虛。使蚩蚩之甿無所倚而安者,此臣所以孜孜憂大瓠之穿也。今羌戎稍卻,變詐亡窮,豈宜乘實時之小安,忘前日之大辱?又將泰然自處,則後日視今,猶今之視前也。」



 元昊圍延州,既解去,鈐轄內侍盧守勤與通判計用章更訟於朝。時內侍用事者,多為
 守勤游說,朝廷議薄守勤罪,而流用章嶺南。清臣上疏曰:「臣聞眾議,延州之圍,盧守勤首對範雍號泣,謀遣李康伯見元昊,為偷生之計。計用章以為事急,不若退保鄜州,李康伯遂有『死難,不可出城見賊』之語。自元昊退,守勤懼金明之失、二將之沒,朝廷歸罪邊將;又思倉卒之言,一旦為人所發,則禍在不測。遂反復前議,移過於人,先為奏陳,冀望取信。正如黃德和誣奏劉平,欲免退走之罪。尋聞計用章亦疏斥守勤事狀,詔文彥博置劾,
 未分曲直,而遽罪用章、康伯,牲赦守勤。此必有議者結中人、惑聖聽,以為方當用師邊陲,不可輕起大獄。臣觀前史,魏尚、陳湯雖有功,尚不免削爵,罰作案驗吏士。何況擁兵自固,觀望不出,恣縱羌賊,破一縣,擒二將。大罪未戮,又自蔽其過,矯誣上奏,此而不按,何罪不容?設用章有退保之言,止坐畏懦;而守勤謀見賊之行,乃是歸款。二者之責,孰重孰輕,望詔彥博鞫正其獄。茍用章之狀果虛,守勤之罪果白,用章更置重科,物論亦允。無容
 偏聽一辭,以虧王道無黨之義。」其後獄具,守勤才降湖北兵馬都監。



 時西師未解,急於經費,中書進擬三司使,清臣初不在選中。帝曰:「葉清臣才可用。」擢為起居舍人、龍圖閣學士、權三司使公事。始奏編前後詔敕,使吏不能欺,簿帳之叢冗者,一切刪去。內東門、御廚皆內侍領之,凡所呼索,有司不敢問,乃為合同以檢其出入。清臣與宋庠、鄭戩雅相善,為呂夷簡所惡,出知江寧府。逾年,入翰林學士,知通進銀臺司、勾當三班院。丁父憂,言
 者以清臣為知兵,請起守邊。及服除,宰相陳執中素不悅之,即除翰林侍讀學士、知邠州。道由京師,因請對,改澶州,進尚書戶部郎中、知青州。徙知永興軍,浚三白渠,溉田逾六千頃。



 仁宗御天章閣,召公卿,出手詔問當世急務。清臣聞之,為條對,極論時政闕失,其言多劘切權貴。且曰:「陛下欲息奔競,此系中書。若宰相裁抑奔競之流,則風俗惇厚,人知止足;宰相用憸佞之士,則貪榮冒進,激成渾波。向有職在管庫,日趨走時相之門。入則取
 街談巷言,以資耳目;出則竊廟謨朝論,以驚流輩。一旦皆擢職司,以酬所任。比日人士競踵此風,出入權要之家,時有『三尸』、『五鬼』之號。乃列館職,或置省曹。且臺諫官為天子耳目,今則不然,盡為宰相肘腋。宰相所惡,則捃以微瑕,公行擊搏;宰相所善,則從而唱和,為之先容。中書政令不平,賞罰不當,則箝口結舌,未嘗敢言。人主纖微過差,或宮闈小事,即極言過當,用為訐直。供職未逾歲時,遷擢已加常等。宋禧為御史,勸陛下宮中畜犬設
 棘,以為守衛。削弱朝體,取笑四夷,不加訶譴,擢為諫官。王達兩為湖南、江西轉運使,所至苛虐,誅剝百姓,徒配無辜,特以宰相故舊,不次拔擢,遂有河北之行。如此,是長奔競也。」其它所列利害甚眾。



 會河決商胡,北道艱食,復以為翰林學士、權三司使。舊制,有三司使、權使公事,而清臣所除,止言「權使」,自是分三等焉。以戶部副使向傳式不職,奏請出之。皇祐元年春,帝御便殿,訪近臣以備邊之策。清臣上對,略曰:



 陛下臨御天下,二十八年,未
 嘗一日自暇自逸。而西夏、契丹頻歲為患者,豈非將相大臣,不得其人,不能為陛下張威德而攘四夷乎?昔王商在廷。單于不敢仰視。郅都臨代,匈奴不敢犯邊。今內則輔相寡謀,綱紀不振;外則兵不素練,將不素蓄。此外寇得以內侮也。慶歷,劉六符來,執政無術略,不能折沖樽俎,以破其謀。六符初亦疑大國之有人,藏奸計而未發。既見表裏,遂肆陸梁。只煩一介之使,坐致二十萬物,永匱膏血,以奉腥膻。此有識之士,所以為國長太息
 也。



 今詔問:「北使詣闕,以伐西戎為名,即有邀求,何以答之?」臣聞誓書所載,彼此無求。況元昊叛邊,累年致討,契丹坐觀金鼓之出,豈有毫發之助?今彼國出師,輒求我助,奸盟違約,不亦甚乎?若使辯捷之人,判其曲直,要之一戰,以破其謀,我直彼曲,豈不憚服。茍不知咎,或肆侵陵,方河朔災傷之餘,野無廬舍,我堅壁自守,縱令深入,其能久居?既無所因之糧,則亟當遁去。然後選擇驍勇,遏絕歸師,設伏出奇,邀擊首尾,若不就禽,亦且大敗矣。



 詔問:「輔翊之能,方面之才,與夫帥領偏裨,當今敦可以任此者。」臣以為不患無人,患有人而不能用爾。今輔翊之臣,抱忠義之深者,莫如富弼。為社稷之固者,莫知範仲淹。諳古今故事者,莫如夏竦。議論之敏者,莫如鄭戩。方面之才,嚴重有紀律者,莫如韓琦。臨大事能斷者,莫如田況。剛果無顧避者,莫如劉渙。宏達有方略者,莫如孫沔。至於帥領偏裨,貴能坐運籌策,不必親當矢石,王德用素有威名,範仲淹深練軍政,龐籍久經邊任,皆其
 選也。狄青、範全頗能馭眾,蔣偕沉毅有術略,張亢倜儻有膽勇,劉貽孫材武剛斷,王德基純愨勁勇,此可補偏裨者也。



 詔謂:「朔方災傷,軍儲缺乏。」此則三司失計置,轉運使不舉職,固非一日。既往固已不咎,來者又復不追,臣未見其可也。且如施昌言承久弊之政,方欲竭思慮、辦職事,一與賈昌朝違戾,遂被移徙,軍儲何由不乏?自去年秋八月,計度市糴,而昌朝執異議,仲春尚未與奪,財賦何緣得豐?先朝置內帑,本備非常。今為主者之吝,
 自分彼我,緩急不以為備,則臣不知其所為也。至如粒食之重,轉徙為難,莫若重立爵等,少均萬數,豪民詿誤,使得入粟,以免杖笞,必能速辦。夫能儉嗇以省費,漸致於從容。德音及此,天下之福也。比日多以卑官躐請厚奉,或身為內供奉而有遙刺之給,或為觀察使便占留後之封,幸門日開,賜予無藝,若令有司執守,率循舊規,庶幾物力亦獲寬弛。



 詔問:「戰馬乏絕,何策可使足用?」臣前在三司,嘗陳監牧之弊,占良田九萬餘頃,歲費錢百
 萬緡。天閑之數,才三四萬,急有徵調,一不可用。今欲不費而馬立辦,莫若賦馬於河北、河東、陜西、京東西五路。上戶一馬,中戶二戶一馬,養馬者復其一丁。如此,則坐致戰馬二十萬匹,不為難矣。



 時清臣以河北乏兵食,自汴漕米繇河陰輸北道者七十餘萬;又請發大名庫錢,以佐邊糴。而安撫使賈昌朝格詔不從,清臣固爭,且疏其跋扈不臣。宰相方欲兩中之,乃徙昌朝鄭州,罷清臣為侍讀學士、知河陽。卒,贈左諫議大夫。



 清臣天資爽邁,
 遇事敢行,奏對無所屈。郭承祐妻舒王元偁女,封郡主,給奉;及承祐為殿前副都指揮使,妻以不加封,請增月給,清臣執奏不可。仁宗曰:「承祐管軍,妻又諸王女,當優之。」清臣曰:「是終為僥幸。」遂卷其奏置懷中,不行。數上書論天下事,陳九議、十要、五利,皆當世可行者。有文集一百六十卷。子均,為集賢校理。



 楊察,字隱甫。其先晉人,從唐僖宗入蜀,家於成都。至其祖鈞,始從孟昶歸朝。鈞生居簡,仕真宗時,至尚書都官
 員外郎,嘗官廬州,遂為合肥人。居簡生察,景祐元年,舉進士甲科,除將作監丞、通判宿州。遷秘書省著作郎、直集賢院,出知穎、壽二州,入為開封府推官,判三司鹽鐵、度支勾院,修起居注,歷江南東路轉運使。屬吏以察年少,易之。及行部,數摘奸隱,眾始畏伏。察在部,專以舉官為急務。人或議之,察曰:「此按察職也,茍掎拾羨餘,則俗吏之能,何必我哉!」召為右正言、知制誥,權判禮部貢院。時上封者請罷有司糊名考士,及變文格,使為放軼以
 襲唐體。察以謂:「防禁一潰,則奔競復起。且文無今昔,惟以體要為宗,若肆其澶漫,亦非唐氏科選之法。」前議遂寢。



 晏殊執政,以妻父嫌,換龍圖閣待制。母憂去職,服除,復為知制誥,拜翰林學士、權知開封府,擢右諫議大夫、權御史中丞。論事無所避。會詔舉御史,建言:「臺屬供奉殿中,巡糾不法,必得通古今治亂良直之臣。今舉格太密,公坐細故,皆置不取,恐英偉之士,或有所遺。」御史何郯以論事不得實,中書問狀。察又言:「御史,故事許風聞;縱
 所言不當,自系朝廷採擇。今以疑似之間,遽被詰問,臣恐臺諫官畏罪緘默,非所以廣言路也。」



 又數以言事忤宰相陳執中。未幾,三司戶部判官楊儀以請求貶官,察坐前在府失出笞罪,雖去官,猶罷知信州。徙揚州,復為翰林侍讀學士,又兼龍圖閣學士、知永興軍,加端明殿學士、知益州。再遷禮部侍郎,復權知開封府,復兼翰林學士、權三司使。



 內侍楊永德毀察於帝,三司有獄,辭連衛士,皇城司不即遣,而有詔移開封府鞫之。察由是乞
 罷三司,乃遷戶部侍郎兼三學士,提連集禧觀,進承旨。逾年,復以本官充三司使。餌鐘乳過劑,病癰卒。贈禮部尚書,謚宣懿。



 察美風儀。幼孤,七歲始能言,母頗知書,嘗自教之。敏於屬文,其為制誥,初若不用意;及稿成,皆雅致有體,當世稱之。遇事明決,勤於吏職,雖多益喜不厭。癰方作,猶入對,商畫財利,歸而大頓,人以為用神太竭雲。有文集二十卷。無子,以兄子庶為嗣。



 弟寘,舉進士第一,通判潤州,以母憂不赴,毀瘠而卒。時人傷之。



 論曰:當仁宗在位時,宋興且百年,海內嘉靖,上下安佚。然法制日以玩弛,僥幸之弊多。自西陲用兵,關中困擾,天子憫勞元元,奮然欲用群材以更內外之治,於時俊傑輩出。尹洙崎嶇兵間,亦頗論天下之事。孫甫馳騁言路,咸以文學、方正知名。絳文詞議論,尤為儒林所宗。朝廷方欲倚用之,不幸死矣。最後,清臣、察繇進士高等,不數年致位侍從,立朝謇謇,無所附麗,為一時名臣。豈非出於上之所自擢,故奮勵不撓,以圖報稱哉?



 。



\end{pinyinscope}