\article{列傳第一百一十四}

\begin{pinyinscope}

 賈
 易董敦逸上官均來之邵葉濤楊畏崔臺符楊汲呂嘉問李南公董必虞策弟奕郭知章



 賈易,字明叔,無為人。七歲而孤。母彭,以紡績自給,日與易十錢,使從學。易不忍使一錢,每浹旬,輒復歸之。年逾冠,中進士甲科,調常州司法參軍。自以儒者不閑法令,歲議獄,唯求合於人情,曰:「人情所在,法亦在焉。」訖去,郡中稱平。



 元祐初,為太常丞、兵部員外郎,遷左司諫。論呂陶不爭張舜民事,與陶交攻,遂劾陶黨附蘇軾兄弟,並及文彥博、范純仁。宣仁后怒其訐,欲謫之,呂公著救之力,出知懷州。御史言其謝表文過,徙廣德軍。明年,提點
 江東刑獄,召拜殿中侍御史。遂疏彥博至和建儲之議為不然,宣仁後命付史館,彥博不自安,竟解平章重事而去。蘇轍為中丞,易引前嫌求避,改度支員外郎,孫升以為左遷。又改國子司業,不拜,提點淮東刑獄。復入,為侍御史。上書言:



 天下大勢可畏者五:一曰上下相蒙,而毀譽不得其真。故人主聰明壅蔽,下情不得上達;邪正無別,而君子之道日消,小人之黨日進。二曰政事茍且,而官人不任其責。故治道不成,萬事隳廢,惡吏市奸而
 自得,良民受弊而無告;愁嘆不平之氣,充溢宇宙,以幹陰陽之和。三曰經費不充,而生財不得其道。故公私困弊,無及時預備之計,衣食之源日蹙;無事之時尚猶有患,不幸倉卒多事,則狼狽窮迫而禍敗至矣。四曰人材廢闕,而教養不以其方。故士君子無可用之實,而愚不肖充牣於朝;污合茍容之俗滋長,背上欺君之風益扇,士氣浸弱,將誰與立太平之基。五曰刑賞失中,而人心不知所向。故以非為是,以黑為白,更相欺惑,以罔其上;
 爵之以高祿而不加勸,僇之以顯罰而不加懼,徼利茍免之奸,冒貨犯義之俗,將何所不有。



 今二聖焦勞念治,而天下之勢乃如此,任事者不可以不憂。是猶寢於積薪之上,火未及然,而自以為安,可不畏乎?



 然則欲知毀譽真偽之情,則莫若明目達聰,使下無壅蔽之患。欲官人皆任其責,則莫若詢事考言,循名責實。欲生財不逆其道,則莫若敦本業而抑末作,崇儉約而戒奢僭。欲教養必以其方,則莫若廣詳延之路,厲廉恥之節,使公卿大
 臣各舉所知,召對延問,以觀其能否,善者用之,不善者罷之。欲人心皆知所向,則莫若賞以勸善,刑以懲惡,不以親疏貴賤為之輕重。則民志一定,而放僻邪侈不為矣。



 其言雖頗切直,然皆老生常談,志於抵厄時事,無他奇畫。



 蘇軾守杭,訴浙西災潦甚苦。易率其僚楊畏、安鼎論軾姑息邀譽,眩惑朝聽,乞加考實。詔下,給事中範祖禹封還之,以謂正宜闊略不問,以活百姓。易遂言:「軾頃在揚州題詩,以奉先帝遺詔為『聞好語』;草《呂大防制》云『
 民亦勞止』,引周厲王詩以比熙寧、元豐之政。弟轍蚤應制科試,文繆不應格,幸而濫進,與軾昔皆誹怨先帝,無人臣禮。」至指李林甫、楊國忠為喻,議者由是薄易,出知宣州。除京西轉運副使,徙蘇州、徐州,加直秘閣。元符中,累謫保靜軍行軍司馬,邵州安置。



 徽宗立,召為太常少卿,進右諫議大夫。陳次升論其為曾布客,改權刑部侍郎,歷工部、吏部,未滿歲為真。以寶文閣待制知鄧州,尋入黨籍。卒,年七十三。



 董敦逸,字夢授,吉州永豐人。登進士第,調連州司理參軍、知穰縣。時方興水利,提舉官調民鑿馬渡港,云可灌田二百頃,敦逸言於朝,以為利不補害,核實如敦逸言。免役夫十六萬,全舊田三千六百頃。徙知弋陽縣,寶豐銅冶役卒多困於誘略,有致死者,敦逸推見本末,縱還鄉者數百人。稍遷梓州路轉運判官。



 元祐六年,召為監察御史,同御史黃慶基言:「蘇軾昔為中書舍人,制誥中指斥先帝事,其弟轍相為表裏,以紊朝政。」宰相呂大防
 奏曰:「敦逸、慶基言軾所撰制詞,以為謗毀先帝。臣竊觀先帝聖意,本欲富國強兵,鞭撻不庭,一時群臣將順太過,故事或失當。及太皇太后與皇帝臨御,因民所欲,隨事救改,蓋事理當然爾。昔漢武帝好用兵,重斂傷民,昭帝嗣位,博採眾議,多行寢罷,明帝尚察,屢興慘獄,章帝改之以寬厚,天下悅服,未有以為謗毀先帝者也。至如本朝真宗即位,弛放逋欠以厚民財;仁宗即位,罷修宮觀以息民力。凡此皆因時施宜,以補助先朝闕政,亦未
 聞當時士大夫有以為謗毀先帝者也。比惟元祐以來,言事官用此以中傷士人,兼欲動搖朝廷,意極不善。」轍復奏曰:「臣昨日取兄軾所撰《呂惠卿告》觀之,其言及先帝者,有曰:『始以帝堯之仁,姑試伯鯀;終然孔子之聖,不信宰予。』兄軾亦豈是謗毀先帝者邪?臣聞先帝末年,亦自深悔已行之事,但未暇改爾。元祐改更,蓋追述先帝美意而已。」宣仁後曰:「先帝追悔往事,至於泣下。」大防曰:「先帝一時過舉,非其本意。」宣仁後曰:「皇帝宜深知。」於是
 敦逸、慶基並罷。敦逸出為湖北運判,改知臨江軍。



 紹聖初,軾、轍失位,劉拯訟敦逸無罪。哲宗記其人,曰:「非前日白須御史乎?」復除監察御史。論常安民為二蘇之黨,凡論議主元祐者,斥去之。改工部員外郎,遷殿中待御史、左司諫、侍御史,入謝曰:「臣再污言路,第恐擠逐,不能久奉彈糾之責。」哲宗曰:「卿能言,無患朕之不能聽;卿言而信,無患朕之不能行也。」



 瑤華秘獄成,詔詣掖庭錄問。敦逸察知冤狀,握筆弗忍書,郝隨從旁脅之,乃不敢異。獄
 既上,於心終不安。幾兩旬,竟上疏,其略云:「瑤華之廢,事有所因,情有可察。詔下之日,天為之陰翳,是天不欲廢之也;人為之流涕,是人不欲廢之也。臣嘗閱錄其獄,恐得罪天下。」哲宗讀之怒,蔡卞欲加重貶,章惇、曾布以為不可,曰:「陛下本以皇城獄出於近習,故使臺端錄問,冀以取信中外。今謫敦逸,何以解天下後世之謗。」哲宗意解而止。明年,用他事出知興國軍,徙江州。



 徽宗即位,加直龍圖閣、知荊南,召入,為左諫議大夫,敦逸極言蔡京、
 蔡卞過惡。遷戶部侍郎。卒,年六十九。



 上官均,字彥衡,邵武人。神宗熙寧親策進士,擢第二,為北京留守推官、國子直講。元豐中,蔡確薦為監察御史裏行。時相州富人子殺人,讞獄為審刑、大理所疑,京師流言法官竇莘等受賕。蔡確引猜險吏數十人,窮治莘等慘酷,無敢明其冤。均上疏言之,乞以獄事詔臣參治,坐是,謫知光澤縣。莘等卒無罪,天下服其持平。有巫托神能禍福人,致貲甚富,均焚像杖巫,出諸境。還,監都進
 奏院。



 哲宗即位,擢開封府推官。元祐初,復為監察御史。議者請兼用詩賦取土,宰相遂欲廢經義。均言:「經術以理為主,而所根者本也,詩賦以文為工,而所逐者末也。今不計本末,而欲襲詩賦之敝,未見其不得也。」自熙寧以來,京師百司有謁禁。均言:「以誠待人,則人思竭忠;以疑遇物,則人思茍免。願除開封、大理外,餘皆釋禁,以明洞達不疑之意。」遂論青苗,以為有惠民之名而無惠民之實,有目前之利而為終歲之患,願罷之而復為常平
 糴糶之法。又言官冗之弊,請罷粟補吏,減任子員,節特奏名之濫,增攝官之舉數,抑胥史之幸進,以清入仕之源。詔有司議,久之不能有所省。復疏言:「今會議之臣,畏世俗之譏評,不計朝廷之利害,閔鄙耄之不進,不思才者之閑滯,非策之善也。」因請對,力陳之,宣仁後曰:「當從我家始。」乃自後屬而下至大夫,悉裁其數。



 又言:「治天下道二,寬與猛而已。寬過則緩而傷義,猛過則急而傷恩。術雖不同,其蠹政害民,一也。間者,監司務為慘核,郡縣
 望風趣辦,不暇以便民為意。陛下臨御,務從寬大,為吏者又復茍簡縱弛,猛寬二者胥失。願明詔四方,使之寬不縱惡,猛不傷惠,以起中和之風。」詔下其章。



 蔡確弟碩盜貸官錢以萬計,獄既上,均論確為宰相,挾邪撓法,當顯正其罪,以厲百官。張璪、李清臣執政,與正人異趣,相繼擊去之。監察御史張舜民論邊事,因及宰相文彥博,舜民左遷。均言:「風憲之任許風聞,所以廣耳目也。舜民之言是,當行之;其言非,當容之。願復舜民職。」不從。臺諫
 約再論,均謂事小不當再論,王巖叟遂劾均反復,巖叟移官。均遷殿中侍御史,內不自安,引義丐去,改禮部員外郎。居三年,復為殿中侍御史。



 西夏自永樂之戰,怙勝氣驕,欲復故地。朝廷用趙離計,棄四砦,至是,又請蘭州為砦地。均上疏曰:「先王之御外國,知威之不可獨立,故假惠以濟威,知惠之不可獨行,故須威以行惠,然後外國且懷且畏,無怨望輕侮之心。今西夏所爭蘭州砦地,皆控扼要路,若輕以予之,恐夏人搗虛,熙河數郡,孤立
 難守。若繼請熙河故地,將何辭以拒之?是傅虎以翼,借寇以兵,不惟無益,祗足為患。不如治兵積穀,畫地而守,使夏人曉然知朝廷意也。」



 時傅堯俞為中書侍郎,許將為左丞,韓忠彥為同知樞密院。三人者,論事多同異,俱求罷。均言:「大臣之任同國休戚,廟堂之上當務協諧,使中外之人,泯然不知有同異之跡。若悻悻然辨論,不顧事體,何以觀視百僚。堯俞等雖有辨論之失,然事皆緣公,無顯惡大過,望令就職。」詔從之。御史中丞蘇轍等尚
 以為言,均上疏曰:「進退大臣當,則天下服陛下之明,而大臣得以安其位。進退不當,則累陛下之哲,而言者自此得以朋黨,合謀並力,以傾搖大臣。天下之事,以是非為主。所論若當,雖異,不害其為善;所論若非,雖同,未免為不善。今堯俞等但不能協和,實無大過。蘇轍乃以許將當時已定議,既而背同列之議,獨上論奏。臣以為善則順之,惡則正之,豈在每事唯命,遂非不改,然後為忠邪?將舍同列之議,上奉聖旨,是能將順其美,不當反以
 為過惡也。若使不忠,雖與同列協和,是乃奸臣爾,非朝廷之利也。」將罷,均又言:「呂大防堅強自任,每有差除,同列不敢異,唯許將時有異同。轍素與大防善,盡力排將,期於心勝。臣恐綱紀法令,自此敗壞矣。」因論:「御史,耳目之任;中丞,風憲之長。轍當公是公非,別白善惡,而不當妄言也。」遂乞罷,出知廣德軍,改提點河北東路刑獄。



 紹聖初,召拜左正言。時大防、轍已罷政,均論大防、轍六罪,並再黜大防,史禍由此起。又奏罷詩賦,專以經術取士。
 宰相章惇欲更政事,專黜陟之柄,陰去異己,出吏部尚書彭汝礪知成都府,召朱服為中書舍人。均言汝礪不可出,服不可用。惇怒,遷均為工部員外郎。尋提點京東、淮東刑獄,歷梓州淮南轉運副使、知越州。



 徽宗立,入為秘書少監,遷起居郎,拜中書舍人、同修國史兼《哲宗實錄》修撰,遷給事中。太學生張寅亮應詔論事,得罪屏斥,均言:「寅亮雖不識忌諱,然志非懷邪。陛下既招其來,又罪其言,恐沮多士之氣。」寅亮得免。時宰相欲盡循熙、豐
 法度為紹述以風均,均曰:「法度惟是之從,無彼此之辨。」由是不協,以龍圖閣待制知永興軍,徙襄州。崇寧初,與元祐黨籍,奪職,主管崇禧觀。政和中,復集賢院修撰、提舉洞霄宮。久之,復龍圖閣待制,致仕。卒,年七十八。



 來之邵,字祖德,開封咸平人。登進士第,由潞州司理參軍為刑部詳斷官。元豐中,改大理評事,御史中丞黃履薦為監察御史。未幾,買倡家女為妾,履劾其污行,左遷將作丞。



 哲宗即位,為太府丞、提舉秦鳳常平、利州成都
 路轉運判官,入為開封府推官,復拜監察御史,遷殿中侍御史。之邵資性奸譎,與楊畏合攻蘇頌,論頌稽留賈易知蘇州之命。又論梁燾緣劉摯親黨,致位丞弼。又論范純仁不可復相,乞進用章惇、安燾、呂惠卿。紹聖初,國事丕變,之邵逆探時指,先劾呂大防。惇既相,擢為侍御史。王安石配食神宗,之邵又請加美謚。疏:「司馬光等畔道逆理,典刑未正,鬼得而誅。獨劉摯尚存,實天以遺陛下。」其阿恣無忌憚如此。



 進刑部侍郎。陽翟民蓋漸以訟
 至有司,之邵二子皆娶蓋氏,誣漸非蓋氏子,以規其貲。諫官張商英論之,以直龍圖閣出知蔡州。卒,年四十八。蔡京為相,特贈太中大夫。



 葉濤,字致遠,處州龍泉人。進士乙科,為國子直講。虞蕃訟起,濤坐受諸生茶紙免官。濤,王氏婿也,即往從安石於金陵,學為文詞。哲宗立,上章自理,得太學正,遷博士。紹聖初,為秘書省正字,編修《神宗史》,進校書郎。曾布薦為起居舍人,擢中書舍人。司馬光、呂公著、王巖叟追貶,
 呂大防、劉摯、蘇轍、梁燾、范純仁責官,皆濤為制詞,文極醜詆。安燾降學士,濤封還命書,云:「燾在元祐時,嘗詆文彥博棄熙河,全先帝萬世之功,不宜加罪。」蔡京劾為黨,罷知光州。又以訴理有過,為範鏜所論,連三黜。曾布引為給事中,居數月而病,以龍閣閣待制提舉崇禧觀,卒。



 楊畏,字子安,其先遂寧人,父徙洛陽。畏幼孤好學,事母孝,不事科舉。黨友交勸之,乃擢進士第。調成紀主簿,不之官,刻志經術,以所著書謁王安石、呂惠卿,為鄆州教
 授。自是尊安石之學,以為得聖人之意。除西京國子監教授,舒但薦為監察御史裏行。時有御史中丞出為郡守,監司薦之,畏言:「侍從賢否,上所素知,監司乃敢妄薦,蓋為異日地爾,乞戒其觀望。」舒但有盜學士院廚錢罪,為王安禮所白,畏抗章辨論,以為可謂之失,未可謂之故。但罷,畏坐左轉宗正丞,出提點夔州路刑獄。



 元祐初,請祠歸洛。畏恐得罪於司馬光,嘗曰:「畏官夔峽,雖深山群獠,聞用司馬光,皆相賀,其盛德如此。」至光卒,畏復曰:「
 司馬光若知道,便是皋、夔、稷、契;以不知道,故於政事未盡也。」呂大防、劉摯為相,俱與畏善,用畏為工部員外郎,除監察御史,擢殿中侍御史。畏助大防攻摯十事,並言梁燾、王巖叟、劉安世、朱光庭皆其死黨,必與為地。既而燾等果救摯,皆不納。摯罷,蘇頌為相,畏復攻頌,以留賈易除書為頌罪。頌罷,畏意欲蘇轍為相。宣仁後外召範純仁為右僕射,畏又攻純仁,不報。畏本附轍,知轍不相,復上疏詆轍不可用。其傾危反復如此,百僚莫不側目。



 遷侍御史,畏言事之未治有四:曰邊疆,曰河事,曰役法,曰內外官政。時有旨令兩省官舉臺官,畏言:「御史與宰執,最為相關之地。宰執既不自差,使其屬舉之,可乎?」太常博士朱彥以議皇地示祭不同,自列乞罷。畏言:「彥據經論理,若彥罷出,恐自是人務觀望,不敢以守官為義。」



 宣仁后崩,呂大防欲用畏諫議大夫,范純仁以畏非端士,不可,大防乃遷畏禮部侍郎。及大防為宣仁後山陵使,畏首背大防,稱述熙寧、元豐政事與王安石學術,哲
 宗信之,遂薦章惇、呂惠卿可大任。廷試進士,李清臣發策有紹述意,考官第主元祐者居上,畏復考,悉下之,拔畢漸以為第一。



 惇入相,畏遣所親陰結之,曰:「畏前日度勢力之輕重,遂因呂大防、蘇轍以逐劉摯、梁燾。方欲逐呂、蘇,二人覺,罷畏言職。畏跡在元祐,心在熙寧,首為相公開路者也。」惇至,徙畏吏部,引以自助。中書侍郎李清臣、知樞密院安燾與惇不合,畏復陰附安、李,惇覺其情;又曾布、蔡卞言畏平日所為於惇,遂以寶文閣待制出
 知真定府。天下於是目為「楊三變」,謂其進於元豐,顯於元祐,遷於紹聖也。



 尋落職知虢州,入元祐黨。後知郢州,復集賢殿修撰、知襄州,移荊南,提舉洞霄宮,居於洛。未幾,知鄧州,再丐祠,以言者論列落職,主管崇禧觀。



 蔡京為相,畏遣子侄見京,以元祐末論蘇轍不可大用等章自明,又因京黨河南尹薛昂致言於京,遂出黨籍。尋復寶文閣待制。政和二年,洛人詣闕,請封禪嵩山,畏上疏累千餘言,極其諛佞。方洽行,得疾卒,年六十九。



 畏頗為
 縱橫學,有才辯而多捭闔,與刑恕締交,其好功名富貴亦同。然恕疏而多失,畏謀必中,其究俱為搢紳禍云。



 論曰:賈易初以剛直名,觀其再劾文彥博、范純仁,而斥蘇軾、蘇轍尤甚,何以剛直為哉?董敦逸於元祐末與黃慶基誣二蘇,以開紹聖之禍,及紹聖則肆詆元祐諸臣,甚至瑤華之冤不能持正,雖終悔而諫,亦何及焉。及見蔡京、蔡卞稔惡,乃論其過惡以自文,杯水不足以救車薪之火也。上官均諫切中時事,及不從紹述之議,其為
 人若可觀,然論呂大防、蘇轍,以之再黜,是亦助紹述者也。楊畏傾危反復,周流不窮,雖儀、秦縱橫,無以尚之,豈徒有三變而已。至於倡紹述以取信哲宗,又謂王安石之學有聖人意,可謂小人無忌憚也哉。來之邵盡擊時賢而進章惇、安燾、呂惠卿,又請加美謚於安石,其流惡不已,乃誣人非其子而欲掩其貲,亦何所不至焉。葉濤在太學,已著污跡,擢第之後,諂安石而從之學,後得曾布之薦,凡元祐名賢貶責制辭,肆筆醜詆,雖有善猶不
 能自滌,況無可述者乎!



 崔臺符,字平叔,蒲陰人。中明法科,為大理詳斷官,校試殿帷,仁宗賜以「盡美」二字。熙寧中,文彥博薦為群牧判官,除河北監牧使,入判大理寺。初,王安石定按問欲舉法,舉朝以為非,臺符獨舉手加額曰:「數百年誤用刑名,今乃得正。」安石喜其附己,故用之。歷知審刑院,判少府監。復置大理獄,拜右諫議大夫,為大理卿。時中官石得一以皇城偵邏為獄,臺符與少卿楊汲輒迎伺其意,所
 在以鍛煉笞掠成之,都人惴慄,至不敢偶語。數年間,麗文法者且萬人。官制行,遷刑部侍郎,官至光祿大夫。元祐初,御史林旦、上官均發其惡,出知潞州,又貶秩徙相州。後兼監牧使。卒,年六十四。



 舊制,武臣至內殿崇班,始蔭其族。臺符言:「文吏州判司猶許用蔭,武臣五歲一遷,自借職四十年乃得通朝籍,輕重不相準。請自供奉官即用蔭。」從之。嘗使遼,至其朝,久立帳前,儐者不贊導。問其故,曰:「太子未至。」臺符誚之曰:「安有君父臨軒而臣子
 偃蹇不至,久立使者禮乎?」儐者懼,贊導如儀。



 楊汲,字潛古,泉州晉江人。登進士第,調趙州司法參軍。州民曹潯者,兄遇之不善,兄子亦加侮焉。潯持刀逐兄子,兄挾之以走,潯曰:「兄勿避,自為侄爾。」既就吏,兄子云:「叔欲紿吾父,止而殺之。」吏當潯謀殺兄,汲曰:「潯呼兄使勿避,何謂謀。若以意為獄,民無所措手足矣。」州用其言,讞上,潯得不死。



 主管開封府界常平,權都水丞,與侯叔獻行汴水淤田法,遂釃汴流漲潦以溉西部,瘠土皆為
 良田。神宗嘉之,賜以所淤田千畝。提點淮西刑獄,提舉西路常平,修古芍陂,引漢泉灌田萬頃。召判都水監,為大理卿,遷刑部、戶部侍郎。元祐初,以寶文閣待制知廬州。崔臺符被劾,汲亦落職知黃州。歷徐、襄、越州。紹聖中,復為戶部侍郎,卒。



 呂嘉問,字望之,以蔭入官。熙寧初,條例司引以為屬,權戶部判官,管諸司庫務,行連灶法於酒坊,歲省薪錢十六萬緡。王安石用魏繼宗議,即京城置市易務,命嘉問
 提舉。上建置十三事,其一欲於律外禁兼並之家輒取利,神宗去之,安石執不可。居二年,連以羨課受賞。神宗聞其擾民。語安石。安石曰:「嘉問奉法不公,以是媒怨。」神宗曰:「免行錢所收細瑣,市易鬻及果實,大傷國體。」安石偽辨自解,至譏神宗為叢脞,不知帝王大略,且曰:「非嘉問,執敢不避左右近習?非臣,孰為嘉問辨?」神宗曰:「即如是,士大夫何故以為不便?」安石請言者姓名,令嘉問條析。



 七年,旱,帝憂心惻怛,語韓維、孫永集市人問之,減坐
 賈錢千萬。安石遂持嘉問條析奏曰:「此皆百姓所願,不如人言也。」嘉問言:「朝廷所以許民輸錢免行者,蓋人情安於樂業,厭於追擾,若一切罷去,則無人祗承。又吏胥祿廩薄,勢不得不求於民,非重法莫禁。以薄廩申重法,則法有時而不行。縣官為給事,則三司經費有限,今取民於鮮,而吏知自重,此臣等推行之本意也。議者乃欲除去,是殆不然。民未嘗不畏吏,方其以行役觸罪,雖欲出錢,亦不可得。今吏祿可謂厚矣,然未及昔日取民所
 得之半,市易所收免行錢,亦未足以償倉法所增之祿,以此推窮,則利害立見矣。」



 初,市易隸三司,嘉問恃勢陵使薛向,出其上。曾布代向,懷不能平。會神宗出手札詢布,布訪於魏繼宗,繼宗憤嘉問掠其功,列其與初議異者。布得實,具上嘉問多收息干賞,挾官府而為兼並之事。神宗將委布考之,安石言二人有私忿,於是詔布與呂惠卿同治。惠卿故憾布,至三司,召繼宗及市賈問狀,其辭同,乃脅繼宗使誣布語言增加,繼宗不從。布言惠
 卿不可共事,神宗欲聽之,安石不可。神宗遂詔中書曰:「朝廷設市易,本為平準以便民,若《周官》泉府者。今顧使中人之家失業,宜厘定其制。」布見神宗曰:「臣每聞德音,欲以王道治天下,今所為駸駸乎間架、除陌矣。嘉問又請販鹽鬻帛,豈不詒四方笑?」神宗頷之。事未決,安石去位,嘉問持之以泣,安石勞之曰:「吾已薦惠卿矣。」惠卿既執政,前獄遂成,布得罪,嘉問亦出知常州。



 明年,安石復相,召檢正中書戶房。安石罷,以知江寧府。歲餘,轉運使
 何琬劾嘉問營繕越法,徙潤州,復坐免。久之,入為吏部郎中、光祿卿。言者交論市易之患,被於天下。本錢無慮千二百萬緡,率二分其息,十有五年之間,子本當數倍,今乃僅足本錢。蓋買物入官,未轉售而先計息取賞;至於物貨苦惡,上下相蒙,虧折日多,空有虛名而已。於是削嘉問三秩,黜知淮陽軍,悉罪前被賞者。



 紹聖中,擢寶文閣待制、戶部侍郎,加直學士、知開封府。專附章惇、蔡卞,多殺不辜,焚去案牘以滅口。嘗薦鄒浩,浩南遷,坐罷
 知懷州。徽宗時,屢暴其宿惡,至分司南京,光州居住,郢州安置。然為蔡氏所右,其婿劉逵蹇序辰、其死友鄧洵武羽翼之,故不久輒起。以龍圖閣學士、太中大夫卒,年七十七,贈資政殿學士。



 初,嘉問竊從祖公弼論新法奏稿,以示王安石,公弼以是斥於外,呂氏號為「家賊」,故不得與呂氏同傳。



 李南公,字楚老,鄭州人。進士及第,調浦江令。郡猾吏恃守以陵縣,不輸負租,南公捕系之。守怒,通判為謝曰:「能
 按郡吏,健令也。」卒置諸法。知長沙縣,有嫠婦攜兒以嫁,七年,兒族取兒,婦謂非前子,訟於官。南公問兒年,族曰九歲,婦曰七歲。問其齒,曰:「去年毀矣。」南公曰:「男八歲而齔,尚何爭?」命歸兒族。熙寧中,提舉京西常平、提點陜西河北刑獄、京西轉運副使,入為屯田員外郎。南公有女皆適人,而同產女弟年三十不嫁,寄他妹家,為御史所論,罷主管崇福宮。



 為河北轉運副使。先是,知澶州王令圖請開迎陽埽舊河,於孫村置約回水東注,南公與範
 子奇以為可行,且欲於大吳北進鋸牙約河勢歸故道。朝廷命使者行視,兩人復以前議為非,云:「迎陽下瞰京師,孫村水勢不便。」又為御史所論,詔罰金。



 加直秘閣、知延安府。夏人犯涇原,南公出師搗其虛,夏人解去。進直龍閣閣,擢寶文閣待制、知瀛州,拜戶部吏部侍郎、戶部尚書。歷知永興軍、成都、真定、河南府、鄭州,擢龍圖閣直學士。



 初,哲宗主入廟,南公修奉,希執政指,請祔東夾室,禮官爭之不得。及更建廟室,坐前議弗當,奪學士,未幾,
 復之,遂致仕。卒,年八十三。



 南公為吏六十年,幹局明銳,然反復詭隨,無特操,識者非之。子譓。



 譓字智甫。第進士。紹聖間,知章丘縣。陜西麥熟,朝廷議遣官諸州,令民平償逋負,譓與餘景在選中。將賜對,曾布言於哲宗曰:「豐兇未可知,言惠、景皆刻薄,必因此暴斂,為民之憂。陛下臨政以來,延見人士未多,如兩人者,懼不足以辱大對。」乃喻使戒飭。使還,為河東轉運判官,徙陜西。進築京師,訖役,除秘閣校理。以母憂去。



 方建永泰
 陵,起使京西。諫官任伯雨言:「祖宗之世,朝廷有大事,邊鄙有兵革,將相大臣召為侍從,乃不得已奪情。今山陵事人皆可辦,何至以一譓隳事體哉?」命遂格。終制,以直龍圖閣知熙州。蔡京使王厚復河湟,譓與之異,召為光祿卿。厚奏功,罷譓守虢。坐嘗言招納未便,停官。



 後數年,為陜西轉運使。京兆麥價踴貴,譓與府縣議從民和市,民弗肯損價。譓移府勒上戶閉糴,府帥徐處仁不聽,且責之。譓怒,上章言處仁沮格詔令,陵毀使者。詔黜處仁,
 而擢譓顯謨閣待制,代其任。鄜延帥錢昂奏:「處仁本以官糴麥損價,與譓爭,乃為民久長之論,不當黜。」詔以昂違道干譽,謫永州。譓又代任鄜延,復徙永興。偽為蟾芝以獻,徽宗疑曰:「蟾,動物也,安得生芝?」命漬盆水,一夕而解。坐罔上,貶散官安置,三年復之。歷數郡,卒。



 董必,字子強,宣州南陵人。嘗謁王安石於金陵,咨質諸經疑義,為安石稱許。登進士第。紹聖中,提舉湖南常平。時相章惇方置眾君子於罪。孔平仲在衡州,以倉粟腐
 惡,乘饑歲,稍損價發之。必即劾其戾常平法,置鞫長沙,以承惇意,無辜系訊多死者。平仲坐徙韶州。



 惇與蔡卞將大誅流人,遣呂升卿往廣東,必往廣西察訪。哲宗既止不治,然必所至,猶以慘刻按脅立威,為五書歸奏。除工部員外郎,中書舍人郭知章封還其命;詔以付趙挺之,權給事中陳次升復封駁不下。必於是訟知章、次升為元祐黨人。坐不當訟言者,出知江州,改湖南轉運判官、提點河北刑獄,召為左司員外郎。



 初,舒但守荊南,起
 邊事,一切詐誕,云徭人款附,實亦不然,必蓋與之謀。及是,但暴卒,加必直龍圖閣往代。乃城信道等六砦,置靖州折博市易,且移飛山營戍。公私煩費,荊人病之。進集賢殿修撰、顯謨閣待制。卒,年五十六,贈龍圖閣待制。



 虞策,字經臣,杭州錢塘人。登進士第,調臺州推官、知烏程縣、通判蘄州。通判蔣之奇以江、淮發運上計,神宗訪東南人才,以策對。王安禮、李常咸薦之,擢提舉利州路常平、湖南轉運判官。



 元祐五年,召為監察御史,進右正
 言。數上書論事,謂人主納諫乃有福,治道以清靜為本。西夏未順命,策言:「今邊備解弛,戎備不修。古之人,善鎮靜者警備甚密,務持重者謀在其中,未有鹵莽闊疏,而曰吾鎮靜、吾持重者。」又乞詔內而省曹、寺監,外而監司、守令,各得以其職陳朝政闕失、百姓疾苦。星文有變,乞順天愛民,警戒萬事,思治心修身之道,勿以宴安為樂。哲宗納後,上《正始要言》。遷左司諫。



 曾肇以議北郊事,與朝論不合,免禮部侍郎,為徐州。策時權給事中,還其命,
 以為肇禮官也,不當以議禮得罪。不從。帝親政,條所當先者五十六事,後多施行。遷侍御史、起居郎、給事中,以龍圖閣待制知青州,改杭州。過闕,留為戶部侍郎。歷刑部、戶部尚書,拜樞密直學士,知永興軍、成都府。



 入為吏部尚書,奏疏徽宗,請均節財用,曰:「臣比在戶部,見中都經費歲六百萬,與天下上供之數略相當。嘗以祖宗故實考之,皇祐所入總三千九百萬,而費才三之一;治平四千四百萬,而費五之一;熙寧五千六十萬,而費盡之。
 今諸道隨一月所須,旋為裒會,汲汲然不能終日。願深裁浮冗,以寬用度。」屬疾祈外,加龍圖閣學士、知潤州,卒於道,年六十六。贈左正議大夫。



 策在元祐、紹聖時,皆居言職。雖不依人取進,亦頗持兩端,故黨議之興,己獨得免。弟奕。



 奕字純臣。第進士。崇寧,提舉河北西路常平,洺、相饑,徙之東路。入對,徽宗問行期,對曰:「臣退即行,流民不以時還,則來歲耕桑皆廢矣。」帝悅。既而西部盜起,復徙提點
 刑獄。時朝廷將遣兵逐捕,奕條上方略,請罷勿用,而自計討賊,不閱月可定。轉運使張摶以為不可,宰相主摶策,數月不效,卒用奕議,悉降之。擢監察御史。親祭北郊,燕人趙良嗣為秘書丞侍祠,奕白其長曰:「今親衛不用三路人,而良嗣以外國降子,顧得預祠事,可乎?」長用其言,具以請,不報。



 陽武民傭於富家,其室美,富子欲私之,弗得,怒殺之,而賂其夫使勿言。事覺,府縣及大理鬻獄,奕受詔鞫訊,皆伏辜。坐漏洩語言罷去。再逾年,還故職,
 提點河北刑獄。自何承矩創邊地為塘濼,有定界。既中貴人典領,以屯田開拓為功,肆侵民田,民上訴,屢出使者按治,皆不敢與直。奕曲折上之,疏其五不可,詔罷屯田。加直秘閣、淮南轉運副使。



 入為開封少尹。故時大理、開封治獄,得請實蔽罪,其後率任情棄法,法益不用。奕言:「廷尉持天下平,京師諸夏本,法且不行,何以示萬國。請自今非情法實不相當,毋得輒請。」從之。遷光祿卿、戶部侍郎。睦州亂,以龍圖閣直學士知鎮江府。寇平,論勞
 增兩秩。還為戶部。內侍總領內藏,予奪顓己,視戶部如僚屬。度支郎方討理滯,奉中旨,令開封尹與總領者來。奕白宰相曰:「計臣不才,當去之而易能者,不可使他人侵其官。」即自劾不稱職。詔為罷內侍,而徙奕工部。



 襲慶守張崇使郡人詣闕請登封,東平守王靚諫以京東歲兇多盜,不當請封。為政者不悅,將罪靚,奕言:「靚憂民愛君,所當獎激,奈何用為罪乎?」靚獲免。未幾卒,年六十,贈龍圖閣學士。



 郭知章,字明叔,吉州龍泉人。第進士,從劉彞廣西幕府,知浮梁、分寧縣。黃履薦為御史,以憂不克拜,知海州、濮州,提點梓州路刑獄。復以鄭雍、顧臨薦,為監察御史。



 哲宗親政,上書請用淳化、天禧詔增諫官員,曰:「館職無所用,朝廷設之不疑;諫官最急,乃常不足。是急於所無用,緩其所當急也。又比歲選授監司,多繇寺監丞,不過知縣資序。外官莫重於部使者,豈宜輕用若是?宜稍限以節。如轉運判官擇實任通判者,提點刑獄擇實任郡守
 者,然後考其治理,簡拔用之。」又言:「自大河東、北分流,生靈被害。今水之趨東者已不可遏,順而導之,閉北而行東,其利百倍矣。」



 遷殿中侍御史。言:「先帝闢地進壤,建策四砦,據高臨下,扼西戎咽喉。元祐用事者委而棄之,願討賾議奏,顯行黜罰。」史院究《神宗實錄》誣罔事,知章請貶治呂大防等。紹聖復制科,知章校試,言:「先朝既策進士,即廢此科,近年復置,誠無所補。」遂復罷。又請復元豐役法,大抵迎合時好。



 進左司員外郎,改左司諫。嘗言:「爵
 祿慶賞,以勸天下之善,願無以假借大臣,使行私恩;刑罰誅戮,以懲天下之惡,願無以假借大臣,使快私忿。忠於陛下者,必見忌大臣;黨於大臣者,必上負陛下。惟明主財察。」權工部侍郎,為中書舍人。



 遼使蕭德崇來為夏人請還河西地,命知章報聘。德崇曰:「兩朝久通好,小國蕞爾疆土,還之可乎?」知章曰:「夏人累犯邊,法當致討,以北朝勸和之故,務為優容。彼若恭順如初,當自有恩旨,非使人所能預知也。」歸未至,坐嘗主導河東流議,以集
 賢殿修撰知和州。



 徽宗立,曾布用為工部侍郎,加寶文閣直學士、知太原府。召拜刑部尚書、知開封府,為翰林學士。言者又論河事,罷知鄧州,旋入黨籍。數年,復顯謨閣直學士。政和初,卒。



 論曰:神宗好大喜功之資,王安石、呂惠卿出而與之遇合,流毒不能止也。哲、徽之世,一變而為蔡確、章惇、曾布,又變而為蔡京、蔡卞,日有甚之,而天下亡矣。乘時起而附之者甚眾,若崔臺符、楊汲以獄殺民;呂嘉問以均輸
 困民;董必肆酷,欲害流人以取悅;李南公以反復詭隨;虞策以心持兩端;郭知章迎合時好,且發實錄之誣。觀諸人所學與其從政,已多可尚,何樂而為此惡哉?不過視一時君相之好尚,將以取富貴而已。設使神宗如仁宗之治,哲、徽承之,必無紹述之禍,雖安石輩亦將有所熏陶,而未必肆其情以至是,況此諸人乎?世道污隆,士習升降,系於人主一念慮之趣向,可不戒哉!可不懼哉!



\end{pinyinscope}