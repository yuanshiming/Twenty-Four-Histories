\article{志第二十七 律歷七}

\begin{pinyinscope}

 明天歷



 《崇天歷》行之至於嘉祐之末,英宗即位,命殿中丞、判司天監周琮及司天冬官正王炳、丞王棟、主簿周應祥、周安世、馬傑、靈臺郎楊得言作新歷,三年而成。琮言:「舊歷
 氣節加時,後天半日;五星之行差半次;日食之候差十刻。」既而司天中官正舒易簡與監生石道、李遘更陳家學。於是詔翰林學士範鎮、諸王府侍講孫思恭、國子監直講劉分文考定是非,上推《尚書》「辰弗集於房」與《春秋》之日食,參今歷之所候,而易簡、道、遘等所學疏闊,不可用,新書為密。遂賜名《明天歷》詔翰林學士王珪序之,而琮亦為義略冠其首。今紀其歷法於後:



 調日法朔餘、周天分、斗分、歲差、日度母附



 造歷之法,必先立元,元正然後定日法,法定然後度周天,以定分、至,三者有程,則歷可成矣。日者,積餘成之;度者,積分成之。蓋日月始離,初行生分,積分成日。自《四分歷》洎古之六歷,皆以九百四十為日法。率由日行一度,經三百六十五日四分之一,是為周天;月行十三度十九分之七,經二十九日有餘,與日相會,是為朔策。史官當會集日月之行,以求合朔。



 自漢太初至於今,冬至差十日,如劉歆《三統》復強於古,故先儒謂之最疏。後漢劉
 洪考驗《四分》,於天不合,乃減朔餘,茍合時用。自是已降,率意加減,以造日法。宋世何承天更以四十九分之二十六為強率,十七分之九為弱率,於強弱之際以求日法。承天日法七百五十二,得一十五強一弱。自後治歷者,莫不因承天法、累強弱之數,皆不悟日月有自然合會之數。



 今稍悟其失,定新歷以三萬九千為日法,六百二十四萬為度母,九千五百為斗分,二萬六百九十三為朔餘,可以上稽於古,下驗於今,反復推求,若應繩準。
 又以二百三十萬一千為月行之餘,月行十三度之餘。



 以一百六十萬四百四十七為日行之餘。日行周天之餘。



 乃會日月之行,以盈不足平之,並盈不足,是為一朔之法。日法也,名元法。



 今乃以大月乘不足之數,以小月乘盈行之分,平而並之,是為一朔之實。周天分也。



 以法約實,得日月相會之數,皆以等數約之,悉得今有之數。盈為朔虛,不足為朔餘。



 又二法相乘為本母,各母互乘,以減周天,餘則歲差生焉,亦以等數約之,即得歲差、度母、周天實用之數。此之一法,理極幽眇,
 所謂反復相求,潛遁相通,數有冥符,法有偶會,古歷家皆所未達。以等數約之,得三萬九千為元法,九千五百為斗分,二萬六百九十三為朔餘,六百二十四萬為日度母,二十二億七千九百二十萬四百四十七為周天分,八萬四百四十七為歲差。



 歲餘:九千五百。古歷曰斗分。



 古者以周天三百六十五度四分度之一,是為斗分。夫舉正於中,上稽往古,下驗當時,反復參求,合符應準,然後施行於百代,為不易之術。自後治歷者,測今冬至日晷,用校古法,過盈,以萬為母,課諸氣分,率二千五百以
 下、二千四百二十八已上為中平之率。新歷斗分九千五百,以萬平之,得二千四百二十五半盈,得中平之數也。而三萬九千年冬至小餘成九千五百日,滿朔實一百一十五萬一千六百九十三,年齊於日分,而氣朔相會。



 歲周:一千四百二十四萬四千五百。以元法乘三百六十五度,內鬥分九千五百,得之,即為一歲之日分,故曰歲周。若以二十四均之,得一十五日、餘八千五百二十、秒一十五,為一氣之策也。



 朔實:一百一十五萬一千六百九十三。本會日月之行,以盈不足平而得二萬六百九十三,是為朔餘,備在調日法術中。



 是則四象全策之餘也。今以元法乘四象全策二十九,總而並之,是為一朔之實也。古歷以一百萬平朔餘之分,得五十三萬六百以下、五百七十已上,是為中平之率。新歷以一百萬平之,得五十三萬五百八十九,得中平之數也。若以四象均之,得七日,餘一萬四千九百二十三、秒,是為弦策也。



 中盈、朔虛分:閏餘附



 日月以會朔為正,氣序以斗建為中,
 是故氣進而盈分存焉。置中節兩氣之策,以一月之全策三十減之,每至中氣,即一萬七千四十、秒十二,是為中盈分。朔退而虛分列焉,置一月之全策三十,以朔策及餘減之,餘一萬八千三百七,是為朔虛分。綜中盈、朔虛分,而閏餘章焉。閏餘三萬五千三百四十五、秒一十三。



 從消息而自致,以盈虛名焉。



 紀法:六十。《易·乾》象之爻九,《坤》象之爻六,《震》、《坎》、《艮》象之爻皆七,《巽》、《離》、《兌》象之爻皆八。綜八卦之數凡六十,又六旬
 之數也。紀者,終也,數終八卦,故以紀名焉。



 天正冬至:大餘五十七,小餘一萬七千。先測立冬晷景,次取測立春晷景,取近者通計,半之,為距至泛日;乃以晷數相減,餘者以法乘之,滿其日晷差而一,為差刻;乃以差刻求冬至,視其前晷多則為減,少則為加,求夏至者反之。



 加減距至泛日,為定日;仍加半日之刻,命從前距日辰,算外,即二至加時日辰及刻分所在。如此推求,則加時與日晷相協。今須積歲四百一年,治平元年甲辰歲,氣積年也。



 則冬至大、小餘與今適
 會。



 天正經朔:大餘三十四,小餘三萬一千。閏餘八十八萬三千九百九十。



 此乃檢括日月交食加時早晚而定之,損益在夜半後,得戊戌之日,以方程約而齊之。今須積歲七十一萬一千七百六十一,治平元年甲辰歲,朔積年也。



 則經朔大、小餘與今有之數,偕閏餘而相會。



 日度歲差:八萬四百四十七。《書》舉正南之星以正四方,蓋先王以明時授人,奉天育物。然先儒所述,互有同異。
 虞喜云:「堯時冬至日短星昴,今二千七百餘年,乃東壁中,則知每歲漸差之所至。」又何承天云:「《堯典》:『日永星火,以正仲夏;宵中星虛,以正仲秋。』今以中星校之,所差二十七八度,即堯時冬至,日在須女十度。」故祖沖之修《大明歷》,始立歲差,率四十五年九月卻一度。虞鄺、劉孝孫等因之,各有增損,以創新法。若從虞喜之驗,昴中則五十餘年日退一度;若依承天之驗,火中又不及百年日退一度。後《皇極》綜兩歷之率而要取其中,故七十五年
 而退一度,此乃通其意未盡其微。今則別調新率,改立歲差,大率七十七年七月,日退一度,上元命於虛九,可以上覆往古,下逮於今。自帝堯以來,循環考驗,新歷歲差,皆得其中,最為親近。



 周天分:二十二億七千九百二十萬四百四十七。本齊日月之行,會合朔而得之。在調日法。



 使上考仲康房、宿之交,下驗姜岌月食之沖,三十年間,若應準繩,則新歷周天,有自然冥符之數,最為密近。



 日躔盈縮定差:張冑玄名損益率曰盈縮數,劉孝孫以盈縮數為朏朒積,《皇極》有陟降率、遲疾數,《麟德》曰先後、盈縮數,《大衍》曰損益、朏朒積,《崇天》曰損益、盈縮積。所謂古歷平朔之日,而月或朝覿東方,夕見西方,則史官謂之朏朒。今以日行之所盈縮、月行之所遲疾,皆損益之,或進退其日,以為定朔,則舒亟之度,乃勢數使然,非失政之致也。新歷以七千一為盈縮之極,其數與月離相錯,而損益、盈縮為名,則文約而義見。



 升降分:《皇極》躔衰有陟降率,《麟德》以日景差、陟降率、日晷景消息為之,義通軌漏。夫南至之後,日行漸升,去極近,故晷短而萬物皆盛;北至之後,日行漸降,去極遠,故晷長而萬物浸衰。自《大衍》以下,皆從《麟德》。今歷消息日行之升降,積而為盈縮焉。



 赤道宿:漢百二年議造歷,乃定東西,立晷儀,下漏刻,以追二十八宿相距於四方,赤道宿度,則其法也。其赤道,斗二十六度及分,牛八度,女十二度,虛十度,危十七度,
 室十六度,壁九度,奎十六度,婁十二度,胃十四度,昴十一度,畢十六度,觜二度,參九度,井三十三度,鬼四度,柳十五度,星七度,張十八度,翼十八度,軫十七度,角十二度,亢九度,氐十五度,房五度,心五度,尾十八度,箕十一度,自後相承用之。至唐初,李淳風造渾儀,亦無所改。開元中,浮屠一行作《大衍歷》,詔梁令瓚作黃道游儀,測知畢、觜、參及輿鬼四宿赤道宿度,與舊不同。畢十七度,觜一度,參十度,鬼三度。



 自一行之後,因相沿襲,下更五代,無所增損。至仁
 宗皇祐初,始有詔造黃道渾儀,鑄銅為之。自後測驗赤道宿度,又一十四宿與一行所測不同。斗二十五度,牛七度,女十一度,危十六度,室十七度,胃十五度,畢十八度,井三十四度,鬼二度,柳十四度,氐十六度,心六度,尾十九度,箕十度。



 蓋古今之人,以八尺圓器,欲以盡天體,決知其難矣。又況圖本所指距星,傳習有差,故今赤道宿度與古不同。自漢太初後至唐開元治歷之初,凡八百年間,悉無更易。今雖測驗與舊不同,亦歲月未久。新歷兩備其數,如淳風從舊之意。



 月度轉分:《洪範傳》曰:「晦而月見西方謂之朏。月未合朔,在日後;今在日前,太疾也。朏者,人君舒緩、臣下驕盈專權之象。朔而月見東方謂之側匿。合朔則月與日合,今在日後,太遲也。側匿者,人君嚴急、臣下危殆恐懼之象。」盈則進,縮則退,躔離九道,周合三旬,考其變行,自有常數。《傳》稱,人君有舒疾之變,未達月有遲速之常也。後漢劉洪粗通其旨。爾後治歷者多循舊法,皆考遲疾之分,增損平會之朔,得月後定追及日之際而生定朔焉。至
 於加時早晚,或速或遲,皆由轉分強弱所致。舊歷課轉分,以九分之五為強率,一百一分之五十六為弱率,乃於強弱之際而求秒焉。新歷轉分二百九十八億八千二百二十四萬二千二百五十一,以一百萬平之,得二十七日五十五萬四千六百二十六,最得中平之數。舊歷置日餘而求朏朒之數,衰次不倫。今從其度而遲疾有漸,月之課驗,稍符天度。



 轉度母:轉法、會周附。



 本以朔分並周天,是為會周。一朔之月常度也,名
 周本母。



 去其朔差為轉終,朔差乃終外之數也。



 各以等數約之,即得實用之數。乃以等數約本母為轉度母,齊數也。



 又以等數約月分為轉法,亦名轉日法也。



 以轉法約轉終,得轉日及餘。本歷創立此數,皆古歷所未有。約得八千一百一十二萬為轉度母,二百九十八億八千二百二十四萬二千二百五十一為轉終分,三百二十億二千五百一十二萬九千二百五十一為會周,一十億八千四百四十七萬三千為轉法,二十一億四千二百八十八萬七千為朔差。



 月離遲疾定差:《皇極》有加減限、朏朒積,《麟德》曰增減率、遲疾積,《大衍》曰損益率、朏朒積,《崇天》亦曰損益率、朏朒
 積。所謂日不及平行則損之,過平行則益之,從陽之義也;月不及平行則益之,過平行則損之,御陰之道也。陰陽相錯而以損益、遲疾為名。新歷以一萬四千八百一十九為遲疾之極,而得五度八分,其數與躔相錯,可以知合食加時之早晚也。



 進朔:進朔之法,興於《麟德》。自後諸歷,因而立法,互有不同。假令仲夏月朔月行極疾之時,合朔當於亥正,若不進朔,則晨而月見東方;若從《大衍》,當戌初進朔,則朔日
 之夕,月生於西方。新歷察朔日之餘,驗月行徐疾,變立法率,參驗加時,常視定朔小餘:秋分後四分法之三已上者,進一日;春分後定朔晨分差如春分之日者,三約之,以減四分之二;定朔小餘如此數已上者,亦進,以來日為朔。俾循環合度,月不見於朔晨;交會無差,明必藏於朔夕。加時在於午中,則晦日之晨同二日之夕,皆合月見;加時在於酉中,則晦日之晨尚見,二日之夕未生;加時在於子中,則晦日之晨不見,二日之夕以生。定晦
 朔,乃月見之晨夕可知;課小餘,則加時之早晏無失。使坦然不惑,觸類而明之。



 消息數:因漏刻立名,義通晷景。《麟德》歷差曰屈伸率。天晝夜者,《易》進退之象也。冬至一陽爻生而晷道漸升,夜漏益減,像君子之道長,故曰息;夏至一陰爻生,而晷道漸降,夜漏益增,像君子之道消,故曰消。表景與陽為沖,從晦者也,故與夜漏長短。今以屈伸象太陰之行,而刻差曰消息數。黃道去極,日行有南北,故晷漏有長短。然
 景差徐疾不同者,句股使之然也。景直晷中則差遲,與句股數齊則差急,隨北極高下,所遇不同。其黃道去極度數與日景、漏刻、昏晚中星反復相求,消息用率,步日景而稽黃道,因黃道而生漏刻,而正中星,四術旋相為中,以合九服之變,約而易知,簡而易從。



 六十四卦:十二月卦出於孟氏,七十二候原於《周書》。後宋景業因劉洪傳卦,李淳風據舊歷元圖,皆未睹陰陽之賾。至開元中,浮屠一行考揚子云《太玄經》,錯綜其數,
 索隱周公三統,糾正時訓,參其變通,著在爻象,非深達《易》象,孰能造於此乎!今之所修,循一行舊義,至於周策分率,隨數遷變。夫六十卦直常度全次之交者,諸侯卦也;竟六日三千四百八、十六秒而大夫受之;次九卿受之;次三公受之;次天子受之。五六相錯,復協常月之次。凡九三應上九,則天微然以靜;六三應上六,則地鬱然而定。九三應上六即溫,六三應上九即寒。上爻陽者風,陰者雨。各視所直之爻,察不刊之象,而知五等與君闢
 之得失、過與不及焉。七十二候,李業興以來迄於《麟德》,凡七家歷,皆以雞始乳為立春初候,東風解凍為次候,其餘以次承之。與《周書》相校,二十餘日,舛訛益甚。而一行改從古義,今亦以《周書》為正。



 岳臺日晷:嶽臺者,今京師嶽臺坊,地曰浚儀,近古候景之所。《尚書·洛誥》稱東土是也。《禮》玉人職:「土圭長尺有五寸以致日。」此即日有常數也。司徒職以圭正日晷,「日至之景,尺有五寸,謂之地中。」此即是地土中致日景與土
 圭等。然表長八尺,見於《周髀》。夫天有常運,地有常中,歷有正像,表有定數。言日至者,明其日至此也。景尺有五寸與圭等者,是其景晷之真效。然夏至之日尺有五寸之景,不因八尺之表將何以得?故經見夏至日景者,明表有定數也。新歷周歲中晷長短,皆以八尺之表測候,所得名中晷常數。交會日月,成象於天,以辨尊卑之序。日,君道也;月,臣道也。謫食之變,皆與人事相應。若人君修德以禳之,則或當食而不食。故太陰有變行以避日,
 則不食;五星潛在日下,為太陰禦侮而扶救,則不食;涉交數淺,或在陽歷,日光著盛,陰氣衰微,則不食;德之休明而有小眚焉,天為之隱,是以光微蔽之,雖交而不見食。此四者,皆德感之所繇致也。按《大衍歷議》:開元十二年七月戊午朔,當食。時自交址至朔方,同日度景測候之際,晶明無雲而不食。以歷推之,其日入交七百八十四分,當食八分半。十三年,天正南至,東封禮畢,還次梁、宋,史官言:「十二月庚戌朔,當食。」帝曰:「予方修先後之職,
 謫見於天,是朕之不敏,無以對揚上帝之休也。」於是徹膳素服以俟之,而卒不食。在位之臣莫不稱慶,以謂德之動天,不俟終日。以歷推之,是月入交二度弱,當食十五分之十三,而陽光自若,無纖毫之變,雖算術乖舛,不宜若是。凡治歷之道,定分最微,故損益毫厘,未得其正,則上考《春秋》以來日月交食之載,必有所差。假令治歷者因開元二食變交限以從之,則所協甚少,而差失過多。由此明之,《詩》云:「此日而微。」乃非天之常數也。舊歷直
 求月行入交,今則先課交初所在,然後與月行更相表裏,務通精數。



 四正食差:正交如累壁,漸減則有差。在內食分多,在外食分少;交淺則間遙,交深則相薄;所觀之地又偏,所食之時亦別。茍非地中,皆隨所在而漸異。縱交分正等同在南方,冬食則多,夏食乃少。假均冬夏,早晚又殊,處南北則高,居東西則下。視有斜正,理不可均。月在陽歷,校驗古今交食,所虧不過其半。合置四正食差,則斜正於
 卯酉之間,損益於子午之位,務從親密,以考精微。



 五星立率:五星之行,亦因日而立率,以示尊卑之義。日周四時,無所不照,君道也;星分行列宿,臣道也。陰陽進退,於此取儀刑焉。是以當陽而進,當陰而退,皆得其常,故加減之。古之推步,悉皆順行,至秦方有金、火逆數。



 《大衍》曰:「木星之行與諸星稍異:商、周之際,率一百二十年而超一次;至戰國之時,其行浸急;逮中平之後,八十四年而超一次,自此之後,以為常率。」其行也,初與日合,一
 十八日行四度,乃晨見東方。而順行一百八日,計行二十二度強,而留二十七日。乃退行四十六日半,退行五度強,與日相望。旋日而退,又四十六日半,退五度強,復留二十七日。而順行一百八日,行十八度強,乃夕伏西方。又十八日行四度,復與日合。



 火星之行:初與日合,七十日行五十二度,乃晨見東方。而順行二百八十日,計行二百一十六度半弱,而留十一日。乃退行二十九日,退九度,與日相望。旋日而退,又
 二十九日,退九度,復留十一日。而順行二百八十日,行一百六十四度半弱,而夕伏西方。又七十日,行五十二度,復與日合。



 土星之行:初與日合,二十一日行二度半,乃晨見東方。順行八十四日,計行九度半強,而留三十五日。乃退行四十九日,退三度半,與日相望。乃旋日而退,又四十九日,退三度少,復留三十五日。又順行八十四日,行七度強,而夕伏西方。又二十一日,行二度半,復與日合。



 金星之行:初與日合,五十八日半行四十九度太,而夕見西方。乃順行二百三十一日,計行二百五十一度半,而留七日。乃退行九日,退四度半,而夕伏西方。又六日半,退四度太,與日再合。又六日半,退四度太,而晨見東方。又退九日,逆行四度半,而復留七日。而復順行二百三十一日,行二百五十一度半,乃晨伏東方。又三十八日半,行四十九度太,復與日會。



 水星之行:初與日合,十五日行三十三度,乃夕見西方。
 而順行三十日,計行六十六度,而留二日,乃夕伏西方。而退十日,退八度,與日再合。又退十日,退八度,乃晨見東方,而復留三日。又順行三十三日,行三十三度,而晨伏東方。又十五日,行三十三度,與日復會。



 一行云:「五星伏、見、留、逆之效,表、里、盈、縮之行,皆系之於時,驗之於政。小失則小變,大失則大變;事微而象微,事章而象章。蓋皇天降譴以警悟人主。又或算者昧於象,占者迷於數,睹五星失行,悉謂之歷舛,以數象相參,兩喪其實。大凡
 校驗之道,必稽古今注記,使上下相距,反復相求,茍獨異常,則失行可知矣。」



 星行盈縮:五星差行,惟火尤甚。乃有南侵狼坐,北入匏瓜,變化超越,獨異於常,是以日行之分,自有盈縮。此乃天度廣狹不等,氣序升降有差,考今升降之分,積為盈縮之數。凡五星入氣加減,興於張子信,以後方士各自增損,以求親密。而《開元歷》別為四象六爻,均以進退,今則別立盈縮,與舊異。



 五星見伏:五星見伏,皆以日度為規。日度之運,既進退不常;星行之差,亦隨而增損。是以五星見伏,先考日度之行,今則審日行盈縮,究星躔進退,五星見伏,率皆密近。舊說,水星晨應見不見在雨水後、穀雨前,夕應見不見在處暑後、霜降前。又云,五星在卯酉南則見遲、0伏早,在卯酉北則見早、伏遲,蓋天勢使之然也。



 步氣朔術



 演紀上元甲子歲,距治平元年甲辰,歲積七十一萬一千七百六十,算外。上驗往古,每年減一算;下算將來,每年加一算。



 元法:三萬九千。



 歲周:一千四百二十四萬四千五百。



 朔實:一百一十五萬一千六百九十三。



 歲周:三百六十五日、餘九千五百。



 朔策:二十九、餘二萬六百九十三。



 望策:一十四、餘二萬九千八百四十六半。



 弦策:七、餘一萬四千九百二十三、秒四半。



 氣策:一十五、餘八千五百二十、秒一十五。



 中盈分:一萬七千四十一、秒一十二。



 朔虛分:一萬八千三百七。



 閏限:一百一十一萬六千三百四十四、秒六。



 歲閏:四十二萬四千一百八十四。



 月閏:三萬五千三百四十八、秒一十二。



 沒限:三萬四百七十九、秒三。



 紀法:六十。



 秒母:一十八。



 求天正冬至:置所求積年,以歲周乘之,為天正冬至氣
 積分;滿元法除之為積日,不滿為小餘。日盈紀法去之,不盡,命甲子,算外,即得所求年前天正冬至日辰及餘。



 求次氣:置天正冬至大、小餘,以氣策加之,即得次氣大、小餘。若秒盈秒母從小餘,小餘滿元法從大餘,大餘滿紀法即去之。



 命大餘甲子,算外,即次氣日辰及餘。餘氣累而求之。



 求天正經朔:置天正冬至氣積分,滿朔實去之為積月,不盡為閏餘;盈元法為日,不盈為餘;以減天正冬至大、小餘,為天正經朔大、小餘。大餘不足減,加紀法;小餘不足減,退大餘,加元法以減之。



 命大餘甲子,算外,即得所求年前天正經朔日辰及餘。



 求弦望及次朔經日:置天正經朔大、小餘,以弦策累加之,命如前,即得弦、望及次朔經日日辰及餘。



 求沒日:置有沒之氣小餘,二十四氣小餘在沒限已上者,為有沒之氣。



 以秒母乘之,其秒從之。



 用減七十一萬二千二百二十五,餘以一萬二百二十五除之為沒日,不滿為除。以沒日加其氣大餘,命甲子,算外,即其氣沒日日辰。



 求減日:置有減經朔小餘,經朔小餘不滿朔虛分者,為有減之朔。



 以三十
 乘之,滿朔虛分為減日,不滿為餘。以減日加經朔大餘,命甲子,算外,即其月減日日辰。



 步發斂術



 候策:五、餘二千八百四十、秒五。



 卦策:六、餘三千四百八、秒六。



 土王策:三、餘一千七百四、秒三。



 辰法:三千二百五十。



 刻法:三百九十。



 半辰法:一千六百二十五。



 秒母:一十八。



 求七十二候:各置中節大、小餘命之,為初候;以候策加之,為次候;又加之,為末候。各命甲子,算外,即得其候日辰。



 求六十四卦:各因中氣大、小餘命之,為公卦用事日;以卦策加之,即次卦用事日;以土王策加諸侯之卦,得十有二節之初外卦用事日。



 求五行用事日:各因四立之節大、小餘命之,即春木、夏火、秋金、冬水首用事日;以土王策減四季中氣大、小餘,
 命甲子,算外,即其月土始用事日也。



 求發斂加時:各置小餘,滿辰法除之為辰數,不滿者,刻法而一為刻,又不滿為分。命辰數從子正,算外,即得所求加時辰時。若以半辰之數加而命之,即得辰初後所入刻數。



 求發斂去經朔:置天正經朔閏餘,以月閏累加之,即每月閏餘;滿元法除之為閏日,不盡為小餘,即得其月中氣去經朔日及餘秒。其閏餘滿閏限,即為置閏,以月內無中氣為定。



 求卦候去經朔:各以卦、候策及餘秒累加減之,中氣前,減;中氣
 後,加。即各得卦、候去經朔日及餘秒。



 步日躔術



 日度母:六百二十四萬。



 周天分:二十二億七千九百二十萬四百四十七。



 周天:三百六十五度。餘一百六十四萬四百四十七,約分二千五百六十四、秒八十二。



 歲差:八萬四百四十七。



 二至限:一百八十二度。餘二萬四千二百五十,約分六千二百一十八。



 一象度:九十一。餘一萬二千一百二十
 五,約分三千一百九。



 求朔弦望入盈縮度:置二至限度及餘,以天正閏日及餘減之,餘為天正經朔入縮度及餘;以弦策累加之,滿二至限度及餘去之,則盈入縮,縮入盈而互得之。



 即得弦、望及次經朔日所入盈縮度及餘。其餘以一萬乘之,元法除之,即得約分。



 求朔弦望盈縮差及定差:各置朔、弦、望所入盈縮度及約分,如在象度分以下者為在初;已上者,覆減二至限,餘為在末。置初、末度分於上,列二至於下,以上減下,餘以下乘上,為積數;滿四千一百三十五除之為度,不滿,
 退除為分,命曰盈縮差度及分。若以四百乘積數,滿五百六十七除之,為盈縮定差。若用立成者,以其度損益率乘度除,滿元法而一,所得,以損益其度下盈縮積,為定差度;其損益初、末分為二日者,各隨其初、末以乘除。其後皆如此例。



 求定氣日:冬、夏二至,盈縮之端,以常為定。餘者以其氣所得盈縮差度及分盈減縮加常氣日及約分,即為其氣定日及分。



 赤道宿度



 斗:二十六



 牛:八



 女:十二



 虛:十及分



 危:十七



 室:十六



 壁:九



 北方七宿九十八度。餘一百六十萬四百四十七,約分二千五百六十四。



 奎:十六



 婁:十二



 胃:十四



 昴:十一



 畢:十七



 觜:一



 參:十



 西方七宿八十一度。



 井:三十三



 鬼:三



 柳:十五



 星:七



 張:十八



 翼:十八



 軫:十七



 南方七宿一百一十一度。



 角:十二



 亢:九



 氐:十五



 房:五



 心:五



 尾:十八



 箕:十一



 東方七宿七十五度。



 前皆赤道度,自《大衍》以下,以儀測定,用為常數。赤道者,常道也,紘於天半,以格黃道。



 求天正冬至赤道日度:以歲差乘所求積年,滿周天分去之,不盡,用減周天分,餘以度母除之,一度為度,不滿為餘。餘以一萬乘之,度母退除為約分。



 命起赤道虛宿六度去之,至不滿宿,
 即所求年天正冬至加時赤道日躔所在宿度及分。



 求夏至赤道加時日度:置天正冬至加時赤道日度,以二至限度及分加之,滿赤道宿度去之,即得夏至加時赤道日度。若求二至昏後夜半赤道日度者,各以二至之日約餘減一萬分,餘以加二至加時赤道日度,即為二至初日昏後夜半赤道日度,每日加一度,滿赤道宿度去之,即得每日昏後夜半赤道日度。



 求赤道宿積度:置冬至加時赤道宿全度,以冬至赤道加時日度減之,餘為距後度及分;以赤道宿度累加之,即各得赤道其宿積度及分。



 求赤道宿積度入初末限,各置赤道宿積度及分,滿九十一度三十一分去之,餘在四十五度六十五分半以下分以日為母。



 為在初限;以上者,用減九十一度三十一分,餘為入末限度及分。



 求二十八宿黃道度:各置赤道宿入初、末限度及分,用減一百一十一度三十七分,餘以乘初、末限度及分,進一位,以一萬約之,所得,命曰黃赤道差度及分;在至後、分前減,在分後、至前加,皆加減赤道宿積度及分,為其
 宿黃道積度及分;以前宿黃道積度減其宿黃道積度,為其宿黃道度及分。其分就近為太、半、少。



 黃道宿度



 斗:二十三半



 牛:七半



 女:十一半



 虛:十少、秒六十四



 危:十七太



 室:十七少



 壁:九太



 北方七宿九十七度半、秒六十四。



 奎:十七太



 婁:十二太



 胃:十四半



 昴:十太



 畢:十六



 觜:一



 參:九少



 西方七宿八十二度。



 井:三十



 鬼:二太



 柳:十四少



 星:七



 張:十八太



 翼:十九半



 軫:十八太



 南方七宿一百一十一度。



 角:十三



 亢:九半



 氐:十五半



 房:五



 心:四



 尾:十七



 箕:十



 東方七宿七十四度太。



 七曜循此黃道宿度,準今歷變定。若上考往古,下驗將
 來,當據歲差,每移一度,乃依法變從當時宿度,然後可步日、月、五星,知其守犯。



 求天正冬至加時黃道日度:以冬至加時赤道日度及分,減一百一十一度三十七分,餘以冬至加時赤道日度及分乘之,進一位,滿一萬約之為度;不滿為分,命曰黃赤道差;用減冬至赤道日度及分,即為所求年天正冬至加時黃道日度及分。



 求冬至之日晨前夜半日度:置一萬分,以其日升分加
 之,以乘冬至約餘,以一萬約之,所得,以減冬至加時黃道日度,即為冬至之日晨前夜半黃道日度及分。



 求逐月定朔之日晨前夜半黃道日度:置其朔距冬至日數,以其度下盈縮積度盈加縮減之,餘以加天正冬至晨前夜半日度,命之,即其月定朔之日晨前夜半日躔所在宿次。



 求每日晨前夜半黃道日度:各置其定朔之日晨前夜半黃道日度,每日加一度,以其日升降分升加降減之,
 滿黃道宿度去之,即各得每日晨前夜半黃道日躔所在宿度及分。若次年冬至小餘滿法者,以升分極數加之。



\end{pinyinscope}