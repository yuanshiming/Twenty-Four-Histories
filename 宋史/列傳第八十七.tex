\article{列傳第八十七}

\begin{pinyinscope}

 李
 清臣安燾張璪蒲宗孟黃履蔡挺兄抗王韶子厚寀薛向子嗣昌章楶



 李清臣,字邦直,魏人也。七歲知讀書,日數千言,暫經
 目輒誦,稍能戲為文章。客有從京師來者,與其兄談佛寺火,清臣從傍應曰:「此所謂災也,或者其蠹民已甚,天固儆之邪?」因作《浮圖災解》。兄驚曰:「是必大吾門。」韓琦聞其名,以兄之子妻之。



 舉進士,調邢州司戶參軍、和川令。歲滿,薦者逾十數,應得京官。適舉將薛向有公事未竟,閡銓格,判銓張掞擿使自陳勿用。清臣曰:「人以家保己而己舍之,薄矣。須待之。」掞離席曰:「君能如是,未可量也。」應材識兼茂科,歐陽修壯其文,以比蘇軾。治平二年,試秘
 閣,考官韓維曰:「荀卿氏筆力也。」試文至中書,修迎語曰:「不置李清臣於第一,則謬矣。」啟視如言。



 時大雨霖,災異數見,論者歸咎濮議。及廷對,或謂曰:「宜以《五行傳》『簡宗廟,水不潤下』為證,必擢上第。」清臣曰:「此漢儒附會之說也,吾不之信。民間豈無疾痛可上者乎?」即條對言:「天地之大,譬如人一身,腹心肺腑有所攻塞,則五官為之不寧。民人生聚,天地之腹心肺腑也;日月星辰,天地之五官也。善止天地之異者,不止其異,止民之疾痛而已。」策
 入等,以秘書郎簽書平江軍判官,名聲籍甚。英宗知之,語王廣淵曰:「韓琦固忠臣,但避嫌太審。如李清臣者,公議皆謂可用,顧以親抑之可乎?」既而詔舉館閣,歐陽修薦之,得集賢校理、同知太常禮院。



 從韓絳使陜西。慶卒亂,家屬九指揮應誅,清臣請於絳,配隸為奴婢。絳坐貶,清臣亦通判海州。久之,還故官,出提點京東刑獄。齊、魯盜賊為天下劇,設耳目方略,名捕且盡。作《韓琦行狀》,神宗讀之曰:「良史才也。」召為兩朝國史編修官,撰《河渠》、《律
 歷》、《選舉》諸志,文直事詳,人以為不減《史》、《漢》。同修起居注,進知制誥、翰林學士。元豐新官制,拜吏部尚書。清臣官右正言,當易承議階,帝曰:「安有尚書而猶承議郎者?」乃授朝奉大夫。六年,拜尚書右丞。哲宗即位,轉左丞。



 時熙、豐法度,一切厘正,清臣固爭之,罷為資政殿學士、知河陽,徙河南、永興。召為吏部尚書,給事中姚勉駁之,改知真定府。班行有王宗正者,致憾於故帥,使其妻詣使者,告前後饋餉過制,囚系數百人。清臣至,立奏解其獄,而
 竄宗正。帝親政,拜中書侍郎,勉復駁之,不聽。



 紹聖元年,廷試進士,清臣發策曰:「今復詞賦之選而士不知勸,罷常平之官而農不加富,可差可募之說紛而役法病,或東或北之論異而河患滋,賜土以柔遠也而羌夷之患未弭,弛利以便民也而商賈之路不通。夫可則因,否則革,惟當之為貴,聖人亦何有必焉。」主意皆絀元祐之政,策言悟其指,於是紹述之論大興,國是遂變。



 范純仁去位,清臣獨顓中書,亟復青苗、免役法,除諸路提舉官。覬為
 相,顧蘇轍軋己,乃擿轍嘗以漢武比先帝激上怒,轍罷。時召章惇未至,清臣心益覬之。已而惇入相,復與為異。惇既逐諸臣,並籍文彥博、呂公著以下三十人,將悉竄嶺表。清臣曰:「更先帝法度,不為無過,然皆累朝元老,若從惇言,必大駭物聽。」帝曰:「是豈無中道耶?合揭榜朝堂,置餘人不問。」鄜延路金明砦主將張輿戰沒,惇怒,議盡戮全軍四千人。清臣曰:「將死亦多端,或先登爭利,或輕身入敵。今悉誅吏士,異時亡將必舉軍降虜矣。」於是
 但誅牙兵十六輩。



 上幸楚王第,有狂婦人遮道叫呼,告清臣謀反,屬吏捕治,本澶州娼而為清臣姑子田氏外婦者。清臣不能引去,用御史言,以大學士知河南,尋落職知真定府。



 初,蔡確子渭上書訴父冤,造奇譖以陷劉摯罪,清臣心知其誣,弗之省,坐奪學士。徽宗立,入為門下侍郎。僕射韓忠彥與之有連,惟其言是聽,出范純禮、張舜民,不使呂希純、劉安世入朝,皆其謀也。尋為曾布所陷,出知大名府而卒,年七十一。贈金紫光祿大夫。



 清
 臣蚤以詞藻受知神宗,建大理寺,築都城,皆命作記,簡重宏放,文體各成一家。為人寬洪,不忮害。嘗為舒但所劾,及在尚書,但以贓抵罪,獨申救之,曰:「但信亡狀,然謂之贓則不可。」再為姚勉所駁,當紹聖議貶,或激使甘心,清臣為之言曰:「勉以議事,所見或不同,豈應以臣故而加重?」帝悟,薄勉罪。起身窮約,以儉自持,至富貴不改。居官奉法,毋敢撓以私。然志在利祿,不公於謀國,一意欲取宰相,故操持悖謬,竟不如願以死。後朝議以復孟後
 罪,追貶武安軍節度副使,再貶雷州司戶參軍。



 安燾,字厚卿,開封人。幼警悟。年十一,從學里中,羞與群兒伍,聞有老先生聚徒,往師之。先生曰:「汝方為誦數之學,未可從吾游,當群試省題一詩,中選乃置汝。」燾無難色。詩成,出諸生上,由是知名。



 登第,調蔡州觀察推官,至太常丞、主管大名府路機宜文字。用歐陽修薦,為秘閣校理、判吏部南曹、荊湖北路轉運判官、提點刑獄兼常平、農田水利、差役事。時方興新法,奉行之吏,或迎合求
 進。司農符檄日夜下,如免役增寬剩,造簿供手實,青苗責保任,追胥苛切,其類旁午。燾平心奉法,列其泰甚於朝。移使京東路,過闕入見,神宗偉其儀觀,留檢正中書孔目房、修起居注。



 元豐初,高麗新通使,假燾左諫議大夫往報之。高麗迎勞,館餼加契丹禮數等,使近臣言:「王遇使者甚敬,出誠心,非若奉契丹茍免邊患而已。」燾笑答曰:「尊中華,事大國,禮一也,特以罕至有加爾。朝廷與遼國通好久,豈復於此較厚薄哉!」使還,帝以為知禮,即
 授所假官,兼直學士院。



 知審刑院,決剖滯訟五百餘案。因言:「每蔽獄上省,輕重有疑,則必致駁,勢既不敵,故法官顧避稽停。請自今以疑獄讞者,皆得輕論。」從之。求知陳州,還,為龍圖閣直學士、判軍器監。



 命館遼使。方宴近郊,使者不令其徒分坐廡下,力爭之,使無以奪。至肆儀將見,又不使綴行分班,使者入,餘皆坐門外,壽請令門見而出,眾始愧悔。逮辭日,悉如儀。或謂細故無足較,燾曰:「契丹喜嘗試人,其漸不可長也。」俄權三司使,改戶部
 尚書。六年。同知樞密院。



 夏人款塞,乞還侵疆。燾言:「地有非要害者固宜予,然羌情無厭,當使知吾宥過而息兵,不應示以厭兵之意。」哲宗立,復仍前議,二府遂欲並棄熙河。燾固爭之,曰:「自靈武而東,皆中國故地。先帝有此武功,今無故棄之,豈不取輕於外夷?」於是但以葭蘆等四砦歸之。



 蔡確輩更用事,燾循循其間,不能有所建明。元祐二年,進知院事。時復洮、河,擒鬼章青宜結,二邊少清,而並塞猶苦寇掠。燾言:「為國者不可好用兵,亦不可
 畏用兵,好則疲民,畏則遺患。今朝廷每戒疆吏,非舉國入寇毋得應之,則固畏用兵矣。雖僅保障戍,實墮其計中,願復講攻擾之策。且乾順幼豎,梁氏擅權,族黨酋渠多反側顧望。若有以離間之,未必不回戈而復怨,此一奇也。」其後夏人自相攜貳,使來修貢,悉如燾策。



 宣仁太后患國用不足,頗裁冗費,宗室奉亦在議中。燾諫曰:「陛下雖痛抑外家,以示至公,然此舉不可不深思而熟計。」太后悟,遂止。



 大河北流,宰相主水官議,必欲回之東注。
 燾以河流入濼澱,久必淤淺,恐河朔無以禦敵,遂上言曰:「自小吳未決之前,河雖屢徙,而盡在中國,故京師得以為北限。今決而西,則河尾益北,如此不已,將南岸遂屬敵界。彼若建橋梁,守以州郡,窺兵河外,可為寒心。今水官之議,不過論地形,較功費;而獻納之臣,不考利害輕重,徒便於治河,而以設險為緩,非至計也。」帝雖然之,而回河之議紛起,東北蕭然煩費,功亦不就。



 三年,同列皆序遷,且新用執政,燾獨如初。詔增其兩秩,燾懇辭曰:「
 是雖有故事,竊意以一時同列超升之故,特用是以慰安其心爾。今日願自臣革之,使朝廷不為姑息,而大臣稍敦廉恥之風,庶或有補。」竟不受。以母憂去,卒喪,拜觀文殿學士、知鄭州,徙穎昌及河南府,入為門下侍郎。



 宣仁之喪,宗室既為三年服,才越歲,章惇拜相,欲革為期。燾爭之曰:「上以先後保祐之久,追崇如恐不盡,茲用明道故實耳。遽改之,播諸天下,非佳聲也。」乃止。燾與惇布衣交,覬其助己,燾不肯少下之。陽翟民蓋漸有財訟,而
 與諫官來之邵交通,開封得其事。惇右之邵,欲薄其罪,燾不可;復欲並劾開封,燾又不可,遂與惇隙。明堂齋祠,為儀仗使,後官有絕馳道穿仗而過者,燾方舉劾,諫官常安民又言,教坊不當於相國寺作樂。帝怒,欲逐安民,燾為救釋。惇遂譖其相表裏,出知鄭州,徙大名。



 父日華,本三班院吏,以燾恩封光祿大夫,至是卒,年九十餘。燾免喪,徽宗立,復知樞密院。舊制,內侍出使,以所得旨言於院,審實乃得行。後多輒去,燾請按治之。都知閻守懃
 領他職,祈罷不以告,亦劾之,帝敕守懃詣燾謝。郝隨得罪,或揣上意且起用,欲援赦為階,亦爭之。



 以老避位,帝將寵以觀文殿大學士,有間之者曰:「是宰相恩典也。」但以學士知河南。將行,上疏曰:「自紹聖、元符以來,用事之臣,持紹述之名,誑惑君父,上則固寵位而快恩仇,下則希進用而肆朋附。彼自為謀則善矣,未嘗有毫發為公家計者也。夫聽言之道,必以其事觀之。臣不敢高談遠引,獨以神考之事切於今者為證。熙寧、元豐之間,中外
 府庫,無不充衍,小邑所積錢米,亦不減二十萬,紹聖以還,傾竭以供邊費,使軍無見糧,吏無月俸,公私虛耗,未有甚於此時,而反謂紹述,豈不為厚誣哉!願陛下監之,勿使飾偏辭而為身謀者復得行其說。」又言:「東京黨禍已萌,願戒履霜之漸。」語尤激切。



 初,建青唐邈川為湟州,戍守困於供億。燾在樞府,因議者以為可棄,奏還之。崇寧元年議其罪,降端明殿學士,再貶寧國軍節度副使,漢陽軍安置。湟州復,又降祁州團練副使。鄯州之復,又
 移建昌軍,然棄鄯州時,燾居憂不預也,終不敢自明。閱再歲,始復通議大夫,還洛卒,年七十五。後五歲,悉還其官職。



 子扶,靖康時為給事中。金人入京師,責取金帛,扶與梅執禮、陳知質、程振皆見殺。



 張璪,初名琥,字邃明,滁州全椒人,洎之孫也。早孤,鞠於兄環,欲任以官,辭不就。未冠登第,歷鳳翔法曹、縉雲令。



 王安石與環善,既得政,將用之,而環已老,乃引璪同編修中書條例,授集賢校理、知諫院、直舍人院。楊繪、劉摯
 論助役,安石使璪為文詰之,辭,曾布請為之,由是忤安石意。神宗欲命璪知制誥,安石薦用布,以璪同修起居注。自縣令至是,才歲餘。坐奏事不實,解三職,已而復之。



 時建議武學,璪言:「古之太學,舞幹習射,受成獻功,莫不在焉。文武之才,皆自此出,未聞偏習其一者也。請無問文武之士,一養於太學。」朝廷既復河、隴,欲因勢戡定夔、蜀、荊、廣諸夷,璪言:「先王務治中國而已。今生財未盡有道,用財未盡有禮,不宜遽及徂征之事。」皆不聽。以集賢
 殿修撰知蔡州,復知諫院兼侍御史知雜事。



 盧秉行鹽法於東南。操持峻急,一人抵禁,數家為黥徙,且破產以償告捕,二年中犯者萬人。璪條列其狀。又言:「行役法以來,最下戶亦每歲納錢,乞度寬羨數均損之,以惠貧弱。」後皆施行。



 鄭俠事起,璪媚呂惠卿,劾馮京與俠交通有跡,深其辭,致京等於罪。判司農寺,出知河陽。元豐初,入權度支副使,遂知制誥、知諫院。判國子監,薦蔡卞可為直講。建增博士弟子員,月書、季考,歲校,以行藝次升,
 略仿《周官》鄉比之法,立齋舍八十二。學官之盛,近代莫比,其議多自璪發之。



 蘇軾下臺獄,璪與李定雜治,謀傅致軾於死,卒不克。詳定郊廟奉祀禮文,議者多以國朝未嘗躬行方澤之禮為非正,詔議更制。璪請於夏至之日,備禮容樂舞,以塚宰攝事。帝曰:「在今所宜,無以易此。」卒行其說。為翰林學士,詳定官制,以寄祿二十四階易前日省、寺虛名,而職事名始正。



 四年,拜參知政事,改中書侍郎。哲宗立,諫官、御史合攻之,謂:「璪奸邪便佞,善窺主
 意,隨勢所在而依附之,往往以危機陷人。深交舒但,數起大獄,天下共知其為大奸。小人而在高位,德之賊也。」疏入,皆不報。最後,劉摯言:「璪初奉安石,旋附惠卿,隨王珪,黨章惇,諂蔡確,數人之性不同,而能探情變節,左右從順,各得其歡心。今過惡既章,不可不速去。」如是逾歲,乃以資政殿學士知鄭州,徙河南、定州、大名府,進大學士,知揚州以卒。贈右銀青光祿大夫,謚曰簡翼。



 蒲宗孟,字傳正,閬州新井人。第進士,調夔州觀察推官。
 治平中,水災地震,宗孟上書,斥大臣及宮禁、宦寺,熙寧元年,改著作佐郎。神宗見其名,曰:「是嘗言水災地震者邪!」召試學士院,以為館閣校勘、檢正中書戶房兼修條例,進集賢校理。



 時三司新置提舉帳司官,祿豐地要,人人欲得之。執政上其員,帝命與宗孟。命察訪荊湖兩路,奏罷辰、沅役錢及湖南丁賦,遠人賴之。呂惠卿制手實法,然猶許災傷五分以上不預。宗孟言:「民以手實上其家之物產而官為注籍,以正百年無用不明之版圖而
 均齊其力役,天下良法也。然災傷五分不預焉。臣以為使民自供,初無所擾,何待豐歲?願詔有司,勿以豐兇弛張其法。」從之,民於是益病矣。



 俄同修起居注、直舍人院、知制誥,帝又稱其有史才,命同修兩朝國史,為翰林學士兼侍讀。舊制,學士唯服金帶,宗孟入謝,帝曰:「學士職清地近,非他官比,而官儀未寵。」乃加佩魚,遂著為令。樞密都承旨張誠一預書局事,頗肆橫,挾中旨以脅同列。宗孟持其語質帝前,皆非是,因叩頭白其奸。帝察其不
 阿,欲大用,拜尚書左丞。



 帝嘗語輔臣,有無人才之嘆,宗孟率爾對曰:「人才半為司馬光邪說所壞。」帝不語,直視久之,曰:「蒲宗孟乃不取司馬光邪!未論別事,只辭樞密一節,朕自即位以來,唯見此一人;他人,則雖迫之使去,亦不肯矣。」宗孟慚懼,至無以為容。僅一歲,御史論其荒於酒色及繕治府舍過制,罷知汝州。逾年,加資政殿學士,徙毫、杭、鄆三州。



 鄆介梁山濼,素多盜,宗孟痛治之,雖小偷微罪,亦斷其足筋,盜雖為衰止,而所殺亦不可勝
 計矣。方徙河中,御史以慘酷劾,奪職知虢州。明年,復知河中,還其職。帥永興,移大名。宗孟厭苦易地,頗默默不樂,復求河中。卒,年六十六。



 宗孟趣尚嚴整而性侈汰,藏帑豐,每旦刲羊十、豕十,然燭三百入郡舍。或請損之,慍曰:「君欲使我坐暗室忍饑邪?」常日盥潔,有小洗面、大洗面、小濯足、大濯足、小大澡浴之別。每用婢子數人,一浴至湯五斛。他奉養率稱是。嘗以書抵蘇軾云:「晚年學道有所得。」軾答之曰:「聞所得甚高,然有二事相勸:一曰慈,
 二曰儉也。」蓋針其失雲。



 黃履,字安中,邵武人。少游太學,舉進士,調南京法曹,又為高密、廣平王二宮教授、館閣校勘,同知禮院。擢監察御史裏行,辭御史,改崇政殿說書兼知諫院。



 神宗嘗詢天地合祭是非,對曰:「國朝之制,冬至祭天圓丘,夏至祭地方澤,每歲行之,皆合於古。猶以有司攝事未足以盡,於是三歲一郊而親行之,所謂因時制宜者也,雖施之方今,為不可易。惟合祭之非,在所當正。然今日禮文之
 失,非獨此也,願敕有司正群祀,為一代損益之制。」詔置局詳定,命履董之,北郊之議遂定。同修起居注,進知制誥、同修國史。遭母憂去,服除,以禮部尚書召對闕中。



 閩省鹽法苦,言者眾,神宗謂履自閩來,恃以為決。履乃陳法甚便,遂不復革,鄉論鄙之。遷御史中丞。履以大臣多因細故罰金,遂言:「賈誼有云:『遇之以禮,則群臣自喜。』群臣且然,況大臣乎?使罪在可議,黜之可也;可恕,釋之可也,豈可罰以示辱哉!」時又制侍郎以下不許獨對,履言:「
 陛下博訪萬務,雖遠外微官,猶令獨對,顧於侍從乃弗得願也。」遂刊其制。御史翟思言事,有旨詰所自來。履諫曰:「御史以言為職,非有所聞,則無以言。今乃究其自來,則人將懲之,臺諫不復有聞矣,恐失開言路之意。」事乃寢。



 哲宗即位,徙為翰林學士。履素與蔡確、章惇、邢恕相交結,每確、惇有所嫌惡,則使恕道風旨於履,履即排擊之。至是,更自謂有定策功。劉安世發其罪,以龍圖閣直學士知越州,坐舉御史不當,降天章閣待制。歷舒、洪、蘇、
 鄂、青州、江寧、應天、穎昌府。紹聖初,復龍圖閣直學士,為御史中丞。極論呂大防、劉摯、梁燾垂簾時事,乞正典刑;又言司馬光變更先朝已行之法為罪。



 先是,北郊之論雖定,猶不果行,履又建言:「陽復陰消,各因其時。上圓下方,各順其體。是以聖人因天祀天,因地祀地,三代至漢,其儀不易。及王莽諂事元後,遂躋地位,同席共牢,歷世襲行,不能全革。逮神宗考古揆今,以正大典,嘗有意於茲矣。今承先志,當在陛下及二三執政。」哲宗詢諸朝,章
 惇以為北郊止可謂之社。履曰:「天子祭天地。蓋郊者交於神明之義,所以天地皆稱郊。故《詩序》云『郊祀天地』。若夫社者,土之神而已,豈有祭大祇亦謂之社乎?」哲宗可之,遂定郊議。拜尚書右丞。



 會正言鄒浩以言事貶新州,履曰:「浩以親被拔擢之故,敢犯顏納忠,陛下遽斥之死地,人臣將視以為戒,誰復敢為陛下論得失乎?乞徙善地。」坐罷知亳州。徽宗立,召為資政殿學士兼侍讀,復拜右丞。未逾年,求去,加大學士、提舉中太一宮,卒。



 論曰:哲宗親政之初,見慮未定,範、呂諸賢在廷,左右弼謨,俾日邇忠讜,疏絕回遹,以端其志向,元祐之治業,庶可守也。清臣怙才躁進,陰覬柄用,首發紹述之說,以隙國是,群奸洞之,沖決莫障,重為薦紳之禍焉。至於興大獄以傾馮京、蘇軾者,璪也;助成手實之法,以壞人材、讕司馬光者,宗孟也;訐垂簾之事,擊呂大防、劉摯等去之者,履也。清臣真小人之靡,三子抑其亞乎。燾論議識趣,有可稱述,雖立朝無附,而依違蔡確、章惇間,無所匡建,
 非大臣之道也。



 蔡挺,字子政,宋城人。第進士,調虔州推官。秩滿,以父希言當官蜀,乞代行,遂授陵州團練推官。王堯臣安撫陜西,闢管勾文字。富弼使遼,奏挺從,至雄州,誓書有所更易,遣挺還白。仁宗欲知契丹事,召對便殿,挺時有父喪,聽以衫帽人。



 範仲淹宣撫陜西、河東,奏挺通判涇州,徙鄜州。河北多盜,精擇諸郡守,以挺知博州。申飭屬縣嚴保伍,得居停奸盜者數人,弛其宿負,補為吏,使之察警,
 盜每發輒得。均博平、聊城二縣稅,歲衍鉅萬。三司下其法於四方,然大抵增賦也。



 為開封府推官、提點府界公事。部修六漯河,用李仲昌議,塞北流,入於六漯。一夕復決,兵夫芟楗漂溺不可計。降知滁州,言者以為輕,乃貶秩停官。



 越數歲,稍起知南安軍,提點江西刑獄,提舉虔州鹽。自大庾嶺下南至廣,驛路荒遠,室廬稀疏,往來無所芘。挺兄抗時為廣東轉運使,乃相與謀,課民植松夾道,以休行者。江閩鹽賊率千百為州縣害,挺諭所部與
 期,使首納器甲,原其罪,得兵械萬計。官鹽惡而價貴,盜鹽善而價且下,故私販日滋。挺簡僚吏至淮轉新鹽,明殿賞,以官數之餘畀之,於是賊黨破散,宿弊遂絕,歲增賣鹽四十萬。



 改陜西轉運副使,進直龍圖閣、知慶州,因上書論攻守大計。夏人大入,挺盡斂邊戶入保,戒諸砦無出戰。諒祚親帥軍數萬攻大順,挺料城堅不可破,而柔遠城惡,亟遣總管張玉將銳師守之。先布鐵蒺藜大順城旁水中,騎渡水多躓,驚言有神。過三日不克,諒祚
 督帳下決戰,挺伏強弩壕外,飛矢貫其鎧,遂引卻。移寇柔遠,玉夜斫營,夏人驚擾潰去。環州熟羌思順舉族投諒祚,倚為鄉導。挺宣言思順且復來,命葺其舊舍,出兵西為迎候之舉。諒祚果疑思順,毒之死。挺築城馬練平為荔原堡,分屬羌三千人守之。



 神宗即位,加天章閣待制、知渭州。舉籍禁兵悉還府,不使有隱占。建勤武堂,五日一訓之,偏伍鉦鼓之法甚備。儲勁卒於行間,遇用奇,則別為一隊。甲兵整習,常若寇至。又分義勇為伍番,番三
 千人,參正兵防秋與春,以八月、正月集,四十五日而罷,歲省粟帛、錢緡十三萬有奇。括並邊生地冒耕田千八百頃,募人佃種,以益邊儲。取邊民闌市蕃部田八千頃,以給弓箭手。又築城定戎軍為熙寧砦,開地二千頃,募卒三千人耕守之。



 諜告夏人候胡盧河,挺出奇兵迎擊之。夏人潰,分諸將躡而討之,蕩其七族。進右諫議大夫,賜金帛三千。夏人復犯諸砦,環慶兵不能御,挺遣張玉以萬人往解其圍。慶州軍變,挺討平之,進龍圖閣直學
 士。廣銳卒徙營,眾憚遷,欲為亂,城中震擾,挺推斬首惡十九人,訖徙營。蕃部歲饑,以田質於弓箭手,過期輒沒。挺為貸官錢,歲息什一,後遂推為蕃漢青苗、助役法。又自以意制渡河大索及兵械鐮槍,皆獲其用。



 熙寧五年,拜樞密副使。帝問挺涇原訓兵之法,召部將按於崇政殿,善之,下以為諸郡法。河州景思立戰死,帝開天章閣訪執政,挺請行。帝曰:「此小事,不足煩卿。河朔有警,卿當行矣。」契丹議雲中地,挺請罷沿邊戍人,示以無事,因
 乞置三十七將,皆行其策。



 七年冬,奏事殿中,疾作而僕,帝親臨賜藥,罷為資政殿學士、判南京留司御史臺。元豐二年,薨,年六十六。贈工部尚書,謚曰敏肅。



 挺譎而多知,人莫能窺其城府。初,為富弼、范仲淹客,頗洩其幾事於呂夷簡以自售。在渭久,鬱鬱不自聊,寓意詞曲,有「玉關人老」之嘆。中使至,則使優伶歌之,以達於禁掖。神宗愍焉,遂有樞密之拜云。



 抗字子直。中進士,調太平州推官。聞父疾,委官去。稍遷
 睦親宅講書。英宗在宮邸,器重之,請於安懿王,願得與游。每見,必衣冠盡禮,義兼師友。再遷太常博士、通判秦州,為秘閣校理,乞知蘇州。州並江湖,民田苦風潮害,抗築長堤,自城屬昆山,亙八十里,民得立塍堨,大以為利。



 徙廣東轉運使。岑水銅冶廢,官給虛券為市,久不償。人無所取資,聚而私鑄,抗盡給之,人得直以止。番禺歲運鹽英、韶,道遠,多侵竊雜惡。抗命十舸為一運,擇攝官主之,歲終會其殿最,增十五萬緡。



 英宗立,召為三司判官。
 廣部去京師遠,不即至,帝見南來者必問之。及入對,諭曰:「卿乃吾故人,朕望於卿者厚,勿以常禮自疏也。」以史館修撰同知諫院。方議安懿王典禮,抗引禮為人後之誼,指陳切至,涕淚被面,帝亦感泣。都城大水,抗請見,帝迎問之,抗推原變異,守前說以對。大臣畏其諫,列白為知制誥,遷龍圖閣直學士、知定州。帝惜其去,曰:「第行,且召矣。」



 郡兵番戍,室家留營多不謹,夫歸輒首原,抗下令悉按以法,戍者感焉。帝不豫,趣命為太子詹事,未至而
 神宗立,改樞密直學士、知秦州。過闕,帝見之,悲慟不自勝,曰:「先帝疾大漸,猶不忘卿。」遂赴鎮。



 秦有質院,質諸羌百餘人,自少至老,扃系之,非死不出,抗皆縱釋,約毋得擅相仇殺。已而有犯者,斬以徇,莫敢奸令。居數日,夢英宗召語,眷如平生,欲退復留。覺為家人言,感念歔欷。及靈駕發引之旦,東望號慟,見僚佐於便室,驟得疾卒,年六十。特贈禮部侍郎。又欲賜謚,吳奎曰:「抗以舊恩,自雜學士贈官,已逾常制。」遂止。



 王韶,字子純,江州德安人。第進士,調新安主簿、建昌軍司理參軍。試制科不中,客游陜西,訪採邊事。



 熙寧元年,詣闕上《平戎策》三篇,其略以為:「西夏可取。欲取西夏,當先復河、湟,則夏人有腹背受敵之憂。夏人比年攻青唐,不能克,萬一克之,必並兵南向,大掠秦、渭之間,牧馬於蘭、會,斷古渭境,盡服南山生羌,西築武勝,遣兵時掠洮、河,則隴、蜀諸郡當盡驚擾,瞎征兄弟其能自保邪?今唃氏子孫,唯董氈粗能自立,瞎徵、欺巴溫之徒,又法所及,
 各不過一二百里,其勢豈能與西人抗哉!武威之南,至於洮、河、蘭、鄯,皆故漢郡縣,所謂湟中、浩亹、大小榆、枹罕,土地肥美,宜五種者在焉。幸今諸羌瓜分,莫相統一,此正可並合而兼撫之時也。諸種既服,唃氏敢不歸?



 唃氏歸則河西李氏在吾股掌中矣。且唃氏子孫,瞎征差盛,為諸羌所畏,若招諭之,使居武勝或渭源城,使糾合宗黨,制其部族,習用漢法,異時族類雖盛,不過一延州李士彬、環州慕恩耳。為漢有肘腋之助,且使夏人無所連
 結,策之上也。」神宗異其言,召問方略,以韶管干秦鳳經略司機宜文字。



 蕃部俞龍珂在青唐最大,渭源羌與夏人皆欲羈屬之,諸將議先致討。韶因按邊,自變量騎直抵其帳,諭其成敗,遂留宿。明旦,兩種皆遣其豪隨以東。久之,龍珂率屬十二萬口內附,所謂包順者也。



 韶又言:「渭源至秦州,良田不耕者萬頃,願置市易司,頗籠商賈之利,取其贏以治田。」帝從其言,改著作佐郎,仍命韶提舉。經略使李師中言:「韶乃欲指占極邊弓箭手地耳,又將
 移市易司於古渭,恐秦州自此益多事,所得不補所亡。」王安石主韶議,為罷師中,以竇舜卿代,且遣李若愚按實。若愚至,問田所在,韶不能對。舜卿檢索,僅得地一頃,既地主有訟,又歸之矣。若愚奏其欺,安石又為罷舜卿而命韓縝。縝遂附會實其事,師中、舜卿皆坐謫,而韶為太子中允、秘閣校理。後帥郭逵上韶盜貸市易錢,安石以為不足校,徙逵涇原。



 帝志復河、隴,築古渭為通遠軍,以韶知軍事。五年七月,引兵城渭源堡及乞神平,破
 蒙羅角、抹耳水巴等族。初,羌保險,諸將謀置陣平地,韶曰:「賊不舍險來鬥,則我師必徒歸。今已入險地,當使險為吾有。」乃徑趣抹邦山,壓敵軍而陣,令曰:「敢言退者斬!」賊乘高下鬥,師小卻。韶躬披甲冑,麾帳下兵逆擊之,羌大潰,焚其廬帳而還,洮西大震。。會瞎徵度洮為之援,餘黨復集。韶命別將由竹牛嶺路張軍聲,而潛師越武勝,遇瞎征首領瞎夔等,與戰破之,遂城武勝,建為鎮洮軍。進右正言、集賢殿修撰。復擊走瞎征,降其部落二萬。更
 名鎮洮為熙州,以熙、河、洮、岷、通遠為一路,韶以龍圖閣待制知熙州。



 六年三月,取河州,遷樞密直學士。降羌叛,韶回軍擊之。瞎徵以其間據河州,韶進破訶諾木藏城,穿露骨山,南入洮州境,道狹隘,釋馬徒行,或日至六七。瞎征留其黨守河州,自將尾官軍,韶力戰破走之,河州復平。連拔宕、岷二州,疊、洮羌酋皆以城附。軍行五十有四日,涉千八百里,得州五,斬首數千級,獲牛、羊、馬以萬計。進左諫議大夫、端明殿學士。七年,入朝,又加資政殿
 學士,賜第崇仁坊。



 還至興平,聞景思立敗於踏白城,賊圍河州,日夜馳至熙。熙方城守,命撤之。選兵得二萬。議所向,諸將欲趨河州。韶曰:「賊所以圍城者,恃有外援也。今知救至,必設伏待我,且新勝氣銳,未可與爭。當出其不意,以攻其所恃,此所謂『批亢搗虛,形格勢禁,則自為解』者也。」乃直扣定羌城,破結河族,斷夏國通路,進臨寧河,分命偏將入南山。瞎徵知援絕,拔柵去。



 初,思立之覆師也,羌勢復熾,朝廷議棄熙河,帝為之旰食,數下詔戒
 韶持重勿出。及是,帝大喜。韶還熙州,以兵循西山繞出踏白後,焚八千帳,瞎征窮蹙丐降,俘以獻。拜韶觀文殿學士、禮部侍郎。資政、觀文學士,非嘗執政而除者,皆自韶始。官其兄弟及兩子,前後賜絹八千匹。未幾,召為樞密副使。



 熙河雖名一路,而實無租人,軍食皆仰給他道。轉運判官馬瑊捃官吏細故,韶欲罷瑊,王安石右瑊,韶始沮,於是與安石異。數以母老乞歸,帝語安石勉留之。



 安南之役,韶言:「決里、廣源之建,臣以為貪虛名而忘
 實禍,執政乃疑臣為刺譏。方舉事之初,臣力爭極論,欲寬民力而省財用,但同列莫肯聽,至以熙河事折臣。臣本意不費朝廷而可以至伊吾盧甘,初不欲令熙河作路,河、岷作州也。今與眾異論,償不求退,必致不容。」韶本鑿空開邊,驟躋政地,乃以勤兵費財歸曲朝廷,帝由是不悅,以故罷職知洪州,又坐謝表怨慢,落職知鄂州。元豐二年,還其職,復知洪州。四年,病疽卒,年五十二。贈金紫光祿大夫,謚曰襄敏。



 韶起孤生,用兵有機略。臨出師,召
 諸將授以指,不復更問,每戰必捷。嘗夜臥帳中,前部遇敵,矢石已交,呼聲震山谷,侍者往往股慄,而韶鼻息自如。在鄂宴客,出家姬奏樂,客張繢醉挽一姬不前,將擁之,姬泣以告。韶徐曰:「本出汝曹娛客,而令失歡如此。」命酌大杯罰之,談笑如故,人亦服其量。韶交親多楚人,依韶求仕,乃分屬諸將,或殺降羌老弱予以首為功級。韶晚節言動不常,頗若病狂狀。既病疽,洞見五臟,蓋亦多殺征雲。子十人,厚、寀最顯。



 厚字處道。少從父兵間,暢習
 羌事,官累通直郎。元祐棄河、湟,厚上疏陳不可,且詣政事堂言之,不聽。紹聖中,用薦者換禮賓副使、乾當熙河公事。



 會羌酋瞎徵、隴拶爭國,河州守將王贍與厚同獻議復故地。元符元年六月,師出塞。七月,下邈川,降瞎征。九月,次青唐,隴拶出迎。遂定湟、鄯。詔賜隴拶姓名曰趙懷德,進厚東上閣門副使、知湟州。既而他種叛,合兵來攻,厚不能支。朝廷度二州不可守,乃以畀懷德,而貶厚右內府率,再貶賀州別駕。



 崇寧初,蔡京復開邊,還厚前
 秩,於是羌人多羅巴奉懷德之弟溪賒羅撒謀復國。懷德畏偪,奔河南,種落更挾之以令諸部。朝廷患眾羌扇結,命厚安撫洮西,遣內客省使童貫偕往。多羅巴知王師且至,集眾以拒。厚聲言駐兵而陰戒行,羌備益弛,乃與偏將高永年異道出。多羅巴三子以數萬人分據險,厚進擊破殺之,唯少子阿蒙中流矢去,道遇多羅巴,與俱遁。遂拔湟州。以功進威州團練使、熙河經略安撫。



 三年四月,厚帥大軍次於湟,命永年將左軍循宗水而北,
 別將張誡將右軍出宗穀而南,自將中軍趨綏遠,期會宗哥川,羌置陳臨宗水,倚北山,溪賒羅撒張黃屋,建大旆,乘高指呼,望中軍旗鼓爭赴之。厚麾游騎登山攻其背,親帥強弩迎射,羌退走,右軍濟水擊之,大風從東南來,揚沙翳羌目,不得視,遂大敗,斬首四千三百餘級,俘三千餘人。羅撒以一騎馳去,其母龜茲公主與諸酋開鄯州降。厚計羅撒必且走青唐,將夜追之,童貫以為不能及,遂止。師下青唐,知羅撒留一宿去,貫始悔之。厚將
 大軍趣廓州,酋落施軍令結以眾降,遂入廓州。超拜厚武勝軍節度觀察留後。



 明年,羅撒復入寇,永年戰死,羌焚大通河橋以叛,新疆大震。厚坐逗遛,降郢州防禦使。已而趙懷德約降未決,厚以書諭之,懷德即納款。還厚舊官。入朝,提舉醴泉觀,卒。贈寧遠軍節度使,謚曰莊敏。



 寀字輔道。好學,工詞章。登第,至校書郎。忽若有所睹,遂感心疾,唯好延道流談丹砂、、神仙事。得鄭州書生,托左道,自言天神可祈而下,下則聲容與人接。因習行其術,
 才能什七八,須兩人共為乃驗。外間歡傳,浸淫徹禁庭。



 徽宗方崇道教,侍晨林靈素自度技不如,願與之游,拒弗許。戶部尚書劉昺,寀外兄也,久以爭進絕還往,神降寀家,使因昺以達,寀言其故,神曰:「第往與之言,汝某年月日在蔡京後堂談某事,有之否?」昺驚駭汗浹,不能對,蓋所言皆陰中傷人者。乃言之帝,即召。寀風儀既高,又善談論,應對合上指。帝大喜,約某日即內殿致天神。靈素求與共事,又弗許。或謂靈素,但勿令
 鄭書生偕,寀當立敗。即白帝曰:「寀父兄昔在西邊,密與夏人謀反國。遲至尊候神,且圖不軌。」帝疑焉。及是日,寀與書生至東華門,靈素戒閽卒獨聽寀入。帝齋潔敬待,越三夕無所聞,乃下寀大理,獄成,棄市,寀竄瓊州。



 薛向,字師正。以祖顏任太廟齋郎,為永壽主簿,權京兆戶曹。有商胡繼銀二篋,出樞密使王德用書,云以與其弟。向適監稅,疑之曰:「烏有大臣寄家問而諉胡人者?」鞫之,果妄。



 為邠州司法參軍。夏人叛,秦中治城,侍御史陳
 洎行邊,向詣洎陳三敝,言:「今板築暴興,吏持斧四出伐木,無問井閭丘隴,民不敢訴。必不得已,宜且葺邊城。函關,秦東塞,今西鄉設守,是為棄關內乎?三司貸龍門富人錢,以百年全盛之天下,一方有警,即稱貸於民,非義也。」洎上其說,悉從之。邠守貪沓,欲因事為邪,並治於城,立表於市以撤屋,冀得賂免,向力爭罷之。



 監在京榷貨務,連歲羨緡錢,當遷秩,移與其兄。三司判官董沔議改河北便糴,行鈔法。向曰:「如此,則都內之錢不繼,茶、鹽、香、
 象將益不售矣。」有司主沔議,既而邊糴滯不行,沔坐黜。



 以向知鄜州。大水冒城郭,沉室廬,死者相枕。郡卒戍延安。詣主將求歸視。弗得,皆亡奔。至,則家人無存者,聚謀為盜,民大恐。向遣吏曉之曰:「冒法以赴急,人之常情,而不聽若輩歸,此武將不知變之過也。亟往收溺尸,貰汝擅還之罪。」眾人庭下泣謝,一境乃安。



 又論河北糴法之弊,以為:「度支歲費錢緡五百萬,所得半直,其贏皆入賈販家。今當有以權之,遇穀貴,則官糴於澶,魏,載以給邊;
 新陳未交,則散糴價以救民乏;軍食有餘,則坐倉收之。此策一行,穀將不可勝食矣。」朝廷是向計,始置便糴司於大名,以向為提點刑獄兼其事。武強有盜殺人而逸,尉捕平民抑使承,向覆其冤,脫六囚於死。



 入為開封度支判官,權陜西轉運副使、制置解鹽。鹽足支十年,而歲調畦夫數千,向奏損其數。兼提舉買馬,監牧沙苑養馬,歲得駒三百,而費錢四千萬,占田千頃。向請斥閑田予民,收租入以市之。乃置場於原、渭,以羨鹽之直市馬,於
 是馬一歲至萬匹。昭陵復土,計用錢糧五十萬貫石,三司不能供億,將移陜西緣邊入鹽中於永安縣。向陳五不可,以為失信商旅,遂舉所闕之數以獻。嘗夜至靈寶縣,先驅入驛,與客崔令孫爭舍。令孫正病臥,驚而死,罷知汝州。甫數月,復以為陜西轉運副使,進為使。厚陵役費,其助如永昭時,凡將漕八年,所入鹽、馬、芻、粟數累萬,民不益賦,其課為最。



 夏將嵬名山以綏州來歸,青澗城主種諤將往迎,詔向與議。諤不俟命,亟率所部出塞,遂
 城之。廷議劾諤擅興,將致法。向言:「諤今者之舉,蓋忘身以徇國,有如不稱,臣請坐之。」諤既貶,向亦罷知絳州,再貶信州,移潞州。張靖使陜西還,陳向制置鹽、馬之失。詔向詣闕與辯,靖辭窮,即罪之。



 神宗知向材,以為江、浙、荊、淮發運使。綱舟歷歲久,篙工利於盜貨,嘗假風水沉溺以滅跡。向募客舟分載,以相督察。官舟有定數,多為主者冒占,悉奪畀屬州,諸運皆詣本曹受遣;以地有美惡,利有重輕,為立等式,用所漕物為誅賞。遷天章閣待制。
 環慶有疆事,帝以向習知地形,召詣中書。舊制,發運使上計毋得出人,唯止都門達章奏。至是,弛其禁。熙寧四年,權三司使。明堂禮成,有司誤遷向右諫議大夫,詔罰吏而向官不奪。河、洮用兵,縣官費不可計,向未嘗乏供給。及解嚴,上疏乞戒將帥裁溢員,汰冗卒、省浮費、節橫賦,手敕褒納。進龍圖閣直學士。



 遼人求代北地,北邊擇牧,加樞密直學士、給事中、知定州。高陽關募兵,敵陰遣人應選。向諜知之,主者覺,縱使亡去,向遣邏捕取之,械
 送瀛州,戮於市。北使久留都亭,數出不遜語,而雲、應點兵,涿、易治道,僉謂必諭盟。向曰:「彼欲疆議速成,故多張虛勢以撼我。使者懼不如其請,故肆嫚言以徼幸取成。兵來不除道,其亦無能為也已。」後皆如向言。遷工部侍郎。向控辭,賜詔弗允。故事,前兩府辭官乃降詔,兩省得詔自向始。元豐元年,召同知樞密院。



 向乾局絕人,尤善商財,計算無遺策,用心至到,然甚者不能無病民,所上課間失實。時方尚功利,王安石從中主之,御史數有言,
 不聽也。向以是益得展奮其材業,至於論兵帝所,通暢明決,遂由文俗吏得大用。及在政地,同列質以西北事,則養威持重,未嘗啟其端,非常所以屬望意。會詔民畜馬,向既奉命,旋知民不便,議欲改為。於是舒但論向反復無大臣體,斥知穎州。又改隨州,卒,年六十六。元祐中,錄其言,謚曰恭敏。子紹彭,有翰墨名,中子嗣昌。



 嗣昌亦以吏材奮。崇寧中,歷熙河轉運判官,梓州、陜西轉運副使,直龍圖閣、集賢殿修撰,入為左司郎中,擢徽
 猷閣待制、陜西都轉運使,知渭州,改慶州。監公使庫皇置坐獄,嗣昌奏請之。遂以監臨自盜責安化軍節度副使,安置郢州。起知相州,復待制、知太原府。論築涇原三倉勞,加顯謨閣直學士;又以撫納西羌功,進延康、宣和殿學士,拜禮部、刑部尚書。坐啟擬反復罷,提舉崇福宮。久之,遷延康殿學士、知延安府,賜第京師。當遷官,丐回授其子昶京秩。



 嗣昌前後因事六七貶,多以欺罔獲罪。至是,言者並論之,降為待制,卒。



 先是,徽宗有意圖北方,
 遣譚稹銜命訪諸帥,韓粹彥、洪中孚皆力雲不可,嗣昌乃潤飾諜詞,以開邊隙。及論事帝前,語至興師,或感激流涕。造亂之咎,人皆歸責焉。



 章楶,字質夫,建州浦城人。祖頻,為侍御史,忤章獻后旨黜官,仁宗欲用之而卒,楶以叔得像蔭,為孟州司戶參軍。應舉入京,聞父對獄於魏,棄不就試,馳往直其冤。還,試禮部第一,擢知陳留縣,歷提舉陜西常平、京東轉運判官、提點湖北刑獄、成都路轉運使,入為考功、吏部、右
 司員外郎。



 元祐初,以直龍圖閣知慶州。時朝廷戢兵,戒邊吏勿妄動,且捐葭蘆、安疆等四砦予夏,使歸其永樂之人。夏得砦益驕。楶言:「夏嗜利畏威,不有懲艾,邊不得休息,宜稍取其土疆,如古削地之制,以固吾圉。然後諸路出兵,據其要害,不一再舉,勢將自蹙矣。」遂乘便出討,以致其師,夏果人圍環州。楶先用間知之,遣驍將折可適伏兵洪德城。夏師過之,伏兵識其母梁氏旗幟,鼓噪而出,斬獲甚眾。又預毒於牛圈瀦水,夏人馬飲者多死。
 召權戶部侍郎。明年,除知同州。紹聖初,知應天府,加集賢殿修撰、知廣州,徙江、淮發運使。



 哲宗訪以邊事,對合旨,命知渭州。至即上言城胡蘆河川,據形勝以逼夏。乃以三月及熙河、秦風、環慶四路之師,陽繕理他堡壁數十所,自示其怯。或以楶怯,請曰:「此夏必爭之地,夏方營石門峽,去我三十里,能奪而有之乎?」楶又陽謝之,陰具板築守戰之備,帥四路師出胡蘆河川,築二城於石門峽江口好水河之陰。二旬有二日成,賜名平夏城、靈平
 砦。方興役時,夏以其眾來乘,楶迎擊敗之。既而環慶、鄜延、河東、熙河皆相繼築城,進拓其境,夏人愕視不敢動。夏主遂奉其母合將數十萬兵圍平夏,疾攻十餘日,建高車臨城,填塹而進,不能克,一夕遁去。夏統軍嵬名阿埋、西壽監軍妹勒都逋皆勇悍善戰,楶諜其弛備,遣折可適、郭成輕騎夜襲,直人其帳執之,盡俘其家,虜馘三千餘、牛羊十萬,夏主震駭。哲宗為御紫宸殿受賀,累擢楶樞密直學士、龍圖閣端明殿學士,進階大中大夫。



 楶
 在涇原四年,凡創州一、城砦九,薦拔偏裨,不間廝役,至於夏降人折可適、李忠傑、朱智用,咸受其馭。夏自平夏之敗,不復能軍,屢請命乞和,哲宗亦為之寢兵。楶立邊功,為西方最。



 時章惇用事,楶與惇同宗,其得興事,頗為世所疑。徽宗立,請老,徙知河南。入見,留拜同知樞密院事,俾其子縡為開封推官以便養。逾年,力謝事罷,授資政殿學士、中太一宮使,未幾,卒。徽宗悼之。贈右銀青光祿大夫,謚曰莊簡,賻恤甚厚。



 楶七子:縡、綜、絲京、綰、綖、演、縝。縡、絲京
 最知名。縡繇推官為戶部員外郎、提點淮南東路刑獄、權知揚州兼提舉香鹽事。時方鑄崇寧大錢,令下,市區晝閉,人持錢買物,至日旰,皇皇無肯售。縡飾市易務致百貨,以小錢收之;且檄倉吏糶米,以大錢予之,盡十日止,民心遂安。未幾,新鈔法行,舊鈔盡廢,一時商賈束手,或自殺。縡得訴者所持舊鈔,為錢以千計者三十萬,上疏言鈔法誤民,請如約以示大信。上怒,罷縡,降兩官。



 絲京第進士,歷陜西轉運判官,入為戶部員外郎。中書
 侍郎劉逵之妻,絲京姊也。逵漸復元祐之政,絲京多贊之。蔡京欲擠逵,且惎絲京不附己,使其黨攻之,出絲京湖州。論者不已,差主管西京崇福宮。



 綜歷通判常州,綰知丹徒縣,綖簽判西安州,演簽判蘇州,楶孫茇承奉郎,藎監蘇州稅,俱列仕顯。



 及京復相,遂興制獄,傾章氏。綖居蘇州,或得私鑄錢數巨罌,京風言者誣綖與州人鬱寶所鑄。詔遣李孝壽、張茂直、沉畸、蕭服更往鞫之,連系數百人,累月卒無實,獄多死者。京大怒,別遣孫傑鞫之,傅致如章,
 綖刺面配沙門島,追毀出身以來文字,除名勒停,籍入其家。竄縡臺州,綜秀州,絲京溫州,綰睦州,演永州,茇處州,藎均州,官司降罷除名者十餘人,時論冤之。



 孫傑擢龍圖閣直學士、知蘇州,張商英入相,始辨前獄,移綖常州,絲京復朝奉郎、通判秀州。頃之,綖改授內殿崇班,絲京秘書省校書郎,遷戶部員外郎,出提點兩浙刑獄,以龍圖閣直學士知越州。譚稹宣撫燕山,請絲京為參謀,加右文殿修撰。金人破蔚州,背歸山後議,稹以錯置乖方罷。絲京落
 職送吏部,會赦恩,上書告老,復龍圖閣直學士致仕,卒。



 論曰:神宗奮英特之資,乘財力之富,銳然欲復河、湟,平靈、夏,而蔡挺、王韶、章楶輩起諸生,委褒衣,樹勛戎馬間。世非無材,顧上所趣尚磨厲奚如耳。觀挺之治兵,韶之策敵,楶之制勝,亦一時良將。薛向雖無三子勞,而董漕邊食襄,不乏仰給,持重樞府,不啟事端,又其善也。若厚之降隴拶、瞎征,取湟、鄯、廓州,功足繼韶。而嗣昌造釁北伐,乃悖於向,可勝誅邪?雖然,佳兵好還,道家所戒,卒之寀
 以左道殺,綖以鑄錢陷,此非其驗也與。



\end{pinyinscope}