\article{列傳第一百四十八}

\begin{pinyinscope}

 尤袤
 謝諤顏師魯袁樞李椿劉儀鳳張孝祥



 尤袤,字延之,常州無錫人。少穎異,蔣偕、施坰呼為奇童。入太學,以詞賦冠多士,尋冠南宮。紹興十八年,擢進士
 第。嘗為泰興令,問民疾苦,皆曰:「邵伯鎮置頓,為金使經行也,使率不受而空厲民。漕司輸蒿秸,致一束數十金。二弊久莫之去。」乃力請臺閫奏免之。縣舊有外城,屢殘於寇,頹毀甚,袤即修築。已而金渝盟,陷揚州,獨泰興以有城得全。後因事至舊治,吏民羅拜曰:「此吾父母也。」為立生祠。



 注江陰學官,需次七年,為讀書計。從臣以靖退薦,召除將作監簿。大宗正闕丞,人爭求之,陳俊卿曰:「當予不求者。」遂除袤。虞允文以史事過三館,問誰可為秘
 書丞者,僉以袤對,亟授之。張栻曰:「真秘書也。」兼國史院編修官、實錄院檢討官,遷著作郎兼太子侍讀。



 先是,張栻說自閣門入西府,士論鼎沸,從臣因執奏而去者數十人,袤率三館上書諫,且不往見。後說留身密奏,於是梁克家罷相,袤與秘書少監陳騤各與郡。袤得臺州,州五縣,有丁無產者輸二年丁稅,凡萬有三千家。前守趙汝愚修郡城工才什三,屬袤成之。袤按行前築,殊鹵莽,亟命更築,加高厚,數月而畢。明年大水,更築之,墉正直水
 沖,城賴以不沒。



 會有毀袤者,上疑之,使人密察,民誦其善政不絕口,乃錄其《東湖》四詩歸奏。上讀而嘆賞,遂以文字受知。除淮東提舉常平,改江東。江東旱,單車行部,核一路常平米,通融有無,以之振貸。



 朱熹知南康,講荒政,下五等戶租五斗以下悉蠲之,袤推行於諸郡,民無流殍。進直秘閣,遷江西漕兼知隆興府。屢請祠,進直敷文閣,改江東提刑。



 梁克家薦袤及鄭僑以言事去國,久於外,當召,上可之。召對,言:「水旱之備惟常平、義倉,願預
 飭有司隨市價禁科抑,則人自樂輸,必易集事。」除吏部郎官、太子侍講,累遷樞密檢正兼左諭德。輸對,又申言民貧兵怨者甚切。



 夏旱,詔求闕失,袤上封事,大略言:「天地之氣,宣通則和,壅遏則乖;人心舒暢則悅,抑鬱則憤。催科峻急而農民怨;關征苛察而商旅怨;差注留滯,而士大夫有失職之怨;廩給朘削,而士卒有不足之怨;奏讞不時報,百久系囚者怨;幽枉不獲伸,而負累者怨;強暴殺人,多特貸命,使已死者怨;有司買納,不即酬價,負
 販者怨。人心抑鬱所以感傷天和者,豈特一事而已。方今救荒之策,莫急於勸分,輸納既多,朝廷吝於推賞。乞詔有司檢舉行之。」



 高宗崩前一日,除太常少卿。自南渡來,恤禮散失,事出倉卒,上下罔措,每有討論,悉付之袤,斟酌損益,便於今而不戾於古。



 當定廟號,袤與禮官定號「高宗」,洪邁獨請號「世祖」。袤率禮官顏師魯、鄭僑奏曰:「宗廟之制,祖有功,宗有德。藝祖規創大業,為宋太祖,太宗混一區夏,為宋太宗,自真宗至欽宗,聖聖相傳,廟制
 一定,萬世不易。在禮,子為父屈,示有尊也。太上親為徽宗子,子為祖而父為宗,失昭穆之序。議者不過以漢光武為比,光武以長沙王後,布衣崛起,不與哀、平相繼,其稱無嫌。太上中興,雖同光武,然實繼徽宗正統,以子繼父,非光武比。將來祔廟在徽宗下而稱祖,恐在天之靈有所不安。」詔群臣集議,袤復上議如初,邁論遂屈。詔從禮官議。眾論紛然。會禮部、太常寺亦同主「高宗」,謂本朝創業中興,皆在商丘,取「商高宗」,實為有證。始詔從初議。建議
 事堂,令皇太子參決庶務。袤時兼侍讀,乃獻書,以為:「儲副之位,止於侍膳問安,不交外事;撫軍監國,自漢至今,多出權宜。乞便懇辭以彰殿下之令德。」



 臺臣乞定喪制,袤奏:「釋老之教,矯誣褻瀆,非所以嚴宮禁、崇幾筵,宜一切禁止。」靈駕將發引,忽定配享之議,洪邁請用呂頤浩、韓世忠、趙鼎、張俊。袤言:「祖宗典故,既祔然後議配享,今忽定於靈駕發引一日前,不集眾論,懼無以厭伏勛臣子孫之心。宜反復熟議,以俟論定。」奏入,詔未預議官詳
 議以聞,繼寢之,卒用四人者。時楊萬里亦謂張浚當配食,爭之不從,補外。進袤權禮部侍郎兼同修國史侍講,又兼直學士院。力辭,上聽免直院。



 淳熙十四年,將有事於明堂,詔議升配,袤主紹興孫近、陳公輔之說,謂:「方在幾筵,不可配帝,且歷舉郊歲在喪服中者凡四,惟元祐明堂用呂大防請,升配神考,時去大祥止百餘日,且祖宗悉用以日易月之制,故升侑無嫌。今陛下行三年之喪,高宗雖已祔廟,百官猶未吉服,詎可近違紹興而
 遠法元祐升侑之禮?請俟喪畢議之。」詔可。



 孝宗嘗論人才,袤奏曰:「近召趙汝愚,中外皆喜,如王蘭亦望收召。」上曰:「然。」一日論事久,上曰:「如卿才識,近世罕有。」次日語宰執曰:「尤袤甚好,前此無一人言之,何也?」兼權中書舍人,復詔兼直學士院,力辭,且薦陸游自代,上不許。時內禪議已定,猶未論大臣也。是日諭袤曰:「旦夕制冊甚多,非卿孰能為者,故處卿以文字之職。」袤乃拜命,內禪一時制冊,人服其雅正。



 光宗即位,甫兩旬,開講筵,袤奏:「願謹
 初戒始,孜孜興念。」越數日,講筵又奏:「天下萬事失之於初,則後不可救。《書》曰:『慎厥終,惟其始。』」又歷舉唐太宗不私秦府舊人為戒。又五日講筵,復論官制,謂:「武臣諸司使八階為常調,橫行十三階為要官,遙郡五階為美職,正任六階為貴品,祖宗待邊境立功者。近年舊法頓壞,使被堅執銳者積功累勞,僅得一階;權要貴近之臣,優游而歷華要,舉行舊法。」姜特立以為議己,言者固以為周必大黨,遂與祠。



 紹熙元年,起知婺州,改太平州,除煥
 章閣待制,召除給事中。既就職,即昌言曰:「老矣,無所補報。凡貴近營求內除小礙法制者,雖特旨令書請,有去而已,必不奉詔。」甫數日,中貴四人希賞,欲自正使轉橫行,袤繳奏者三,竟格不下。



 兼侍講,入對,言:「願上謹天戒,下畏物情,內正一心,外正五事,澄神寡欲,保毓太和,虛己任賢,酬酢庶務。不在於勞精神、耗思慮、屑屑事為之末也。」



 陳源除在京宮觀,耶律適嘿除承宣使,陸安轉遙郡,王成特補官,謝淵、李孝友賞轉官,吳元充、夏永壽遷
 秩,皆論駁之,上並聽納。



 韓侂冑以武功大夫、和州防禦使用應辦賞直轉橫行,袤繳奏,謂:「正使有止法,可回授不可直轉。侂冑勛賢之後,不宜首壞國法,開攀援之門。」奏入,手詔令書行,袤復奏:「侂冑四年間已轉二十七年合轉之官,今又欲超授四階,復轉二十年之官,是朝廷官爵專徇侂冑之求,非所以為摩厲之具也。」命遂格。



 上以疾,一再不省重華宮,袤上封事曰:「壽皇事高宗歷二十八年如一日,陛下所親見,今不待倦勤以宗社付陛
 下,當思所以不負其托,望勿憚一日之勤,以解都人之惑。」後數日,駕即過重華宮。



 侍御史林大中以論事左遷,袤率左史樓鑰論奏,疏入,不報,皆封駁不書黃。耶律適嘿復以手除詔承宣使,一再繳奏,輒奉內批,特與書行。袤言:「天下者祖宗之天下,爵祿者祖宗之爵祿,壽皇以祖宗之天下傳陛下,安可私用祖宗爵祿而加於公議不允之人哉?」疏入,上震怒,裂去後奏,付前二奏出。袤以後奏不報,使吏收閣,命遂不行。



 中宮謁家廟,官吏推賞
 者百七十有二人,袤力言其濫,乞痛裁節,上從之。嘗因登對,專論廢法用例之弊,至是復申言之。除禮部尚書。駕當詣重華宮,復以疾不出,率同列奏言:「壽皇有免到宮之命,願力請而往,庶幾可以慰釋群疑,增光孝治。」後三日,駕隨出,中外歡呼。



 兼侍讀,上封事曰:「近年以來,給舍、臺諫論事,往往不行,如黃裳、鄭汝諧事遷延一月,如陳源者奉祠,人情固已驚愕,至姜特立召,尤為駭聞。向特立得志之時,昌言臺諫皆其門人,竊弄威福,一旦
 斥去,莫不誦陛下英斷。今遽召之,自古去小人甚難,譬除蔓草,猶且復生,況加封植乎?若以源、特立有勞,優以外任,或加錫繼,無所不可。彼其閑廢已久,含憤蓄怨,待此而發,儻復呼之,必將潛引黨類,力排異己,朝廷無由安靜。」



 時上已屬疾,國事多舛,袤積憂成疾,請告,不報。疾篤乞致仕,又不服,遂卒,年七十。遺奏大略勸上以孝事兩宮,以勤康庶政,察邪佞,護善類。又口占遺書別政府。明年,轉正奉大夫致仕。贈金紫光祿大夫。



 袤少從喻樗、汪
 應辰游。樗學於楊時,時,程頤高弟也。方乾道、淳熙間,程氏學稍振,忌之者目為道學,將攻之。袤在掖垣,首言:「夫道學者,堯、舜所以帝,禹、湯、武所以王,周公、孔、孟所以設教。近立此名,詆訾士君子,故臨財不茍得所謂廉介,安貧守分所謂恬退,擇言顧行所謂踐履,行己有恥所謂名節,皆目之為道學。此名一立,賢人君子欲自見於世,一舉足且入其中,俱無得免,此豈盛世所宜有?願徇名必責其實,聽言必觀其行,人才庶不壞於疑似。」孝宗曰:「
 道學豈不美之名,正恐假托為奸,使真偽相亂爾。待付出戒敕之。」袤死數年,侂冑擅國,於是禁錮道學,賢士大夫皆受其禍,識者以袤為知言。



 嘗取孫綽《遂初賦》以自號,光宗書扁賜之。有《遂初小稿》六十卷、《內外制》三十卷。嘉定五年,謚文簡。子棐、概。孫煜,禮部尚書。



 謝諤,字昌國,臨江軍新喻人。幼敏惠,日記千言,為文立成。紹興二十七年,中進士第,調峽州夷陵縣主簿,未上,撫之樂安多盜,監司檄諤攝尉,條二十策,大要使其徒
 相糾而以信賞隨之,群盜果解散。金渝盟,諸軍往來境上,選行縣事,有治辦聲。



 改吉州錄事參軍。囚死者舊瘞以鞂,往往暴骨。諤白郡,取船官棄材以棺斂之。郡民陳氏僮竊其篋以逃,有匿之者。陳於官,詞過其實,反為匿僮者所誣。帥龔茂良怒,欲坐以罪,諤為書白茂良,陳氏獲免,茂良亦以是知之。



 歲大侵,饑民萬餘求廩,官吏罔措。諤植五色旗,分部給糶,頃刻而定。知袁州分宜縣。縣積負於郡數十萬,歲常賦外,又徵緡錢二萬餘,諤乃疏
 其弊於諸監司,請免之。以母憂去。尋丁父憂,服闋,除乾辦行在諸司糧料院。遷國子監簿,尋擢監察御史。奏減袁州分宜、秀州華亭月樁錢。



 諤里居時,創義役法,編為一書,至是上之。詔行其法於諸路,民以為便。



 遷侍御史,再遷右諫議大夫兼侍講。講《尚書》,言於上曰:「《書》,治道之本,故觀經者當以《書》為本。」上曰:「朕最喜伊尹、傅說所學,得事君之道。」諤曰:「伊、傅固然,非成湯、武丁信用之,亦安能致治!」因論及邊事,上有乘機會之諭,諤曰:「機會雖不
 可失,舉事亦不可輕。」上嘗問曰:「聞卿與郭雍游,雍學問甚好,豈曾見程頤乎?」諤奏:「雍父忠孝嘗事頤,雍蓋得其傳於父。」上遂封雍為頤正先生。



 光宗登極,獻十箴,又論二節三近:所當節者曰宴飲,曰妄費;所當近者曰執政大臣,曰舊學名儒,曰經筵列職。除御史中丞,權工部尚書。請祠,以煥章閣直學士知泉州,又辭,提舉太平興國宮而歸。紹熙五年,卒,年七十四,贈通議大夫。



 諤為文仿歐陽修、曾鞏。初居縣南之竹坡,名其燕坐曰艮齋,人稱
 艮齋先生。周必大薦士,及諤姓名,孝宗曰:「是謂艮齋者耶?朕見其《性學淵源》五卷而得之」云。



 顏師魯,字幾聖,漳州龍溪人。紹興中,擢進士第,歷知莆田、福清縣。嘗決水利滯訟,闢陂洫綿四十里。歲大侵,發廩勸分有方而不遏糴價,船粟畢湊,市糴更平。鄭伯熊為常平使,薦於朝,帥陳俊卿尤器重之。召為官告院,遷國子丞,除江東提舉。時天雨土,日青無光,都人相驚,師魯陛辭,言:「田里未安,犴獄未清,政令未當,忠邪未辨,天
 不示變,人主何繇省悟!願詔中外,極陳得失,求所以答天戒,銷患未形。」上韙其言。



 尋改使浙西。役法敝甚,細民至以雞豚罌榻折產力,遇役輒破家。師魯下教屬邑,預正流水籍,稽其役之序,寬比限,免代輸,咸便安之。鹽課歲百鉅萬,本錢久不給,亭灶私鬻,禁不可止,刑闢日繁。師魯撙帑緡,盡償宿負,戒官吏毋侵移,比旁路課獨最。上謂執政曰:「儒生能辦事如此。」予職直秘閣。農民有墾曠土成田未及受租者,奸豪多為己利,師魯奏:「但當正
 其租賦,不應繩以盜種法,失劭農重本意。」奏可,遂著為令。



 入為監察御史,遇事盡言,無所阿撓。有自外府得內殿宣引,且將補御史闕員,師魯亟奏:「宋璟召自廣州,道中不與楊思勖交一談。李墉恥為吐突承璀所薦,堅辭相位不拜。士大夫未論其才,立身之節,當以璟、墉為法。今其人朋邪為跡,人所切齒,縱朝廷乏才,寧少此輩乎?臣雖不肖,羞與為伍。」命乃寢。繼累章論除職帥藩者:「比年好進之徒,平時交結權幸,一紆郡紱,皆掊克以厚包
 苴,故昔以才稱,後以貪敗。」上出其疏袖中,行之。



 十年,繇太府少卿為國子祭酒。初,上諭執政擇老成端重者表率太學,故有是命。首奏:「宜講明理學,嚴禁穿鑿,俾廉恥興而風俗厚。」師魯學行素孚規約,率以身先,與諸生言,孳孳以治己立誠為本,藝尤異者必加獎勸,由是人知飭勵。上聞之喜曰:「顏師魯到學未久,規矩甚肅。」除禮部侍郎,尋兼吏部。



 有旨改官班,特免引見。師魯獻規曰:「祖宗法度不可輕馳,願始終持久,自強不息。」因言:「賜帶多
 濫,應奉微勞,皆得橫金預外朝廷會,如觀瞻何?且臣下非時之賜,過於優隆;梵舍不急之役,亦加錫繼。雖南帑封樁不與大農經費,然無功勞而概與之,是棄之也。萬一有為國制變禦侮,建功立事者,將何以旌寵之?」高宗喪制,一時典禮多師魯裁定,又與禮官尤袤、鄭僑上議廟號,語在《袤傳》。



 詔充遺留禮信使。初,顯仁遺留使至金,必令簪花聽樂。師魯陛辭,言:「國勢今非昔比,金人或強臣非禮,誓以死守。」沿途宴設,力請徹樂。至燕山,復辭簪
 花執射。時孝宗以孝聞,師魯據經陳誼,反復慷慨,故金終不能奪。



 遷吏部侍郎,尋除吏部尚書兼侍講,屢抗章請老,以龍圖閣直學士知泉州。臺諫、侍從相繼拜疏,引唐孔戣事以留行。內引,奏言:「願親賢積學,以崇聖德,節情制欲,以養清躬。」在泉因任,凡閱三年,專以恤民寬屬邑為政,始至即蠲舶貨,諸商賈胡尤服其清。再起知泉州,以紹熙四年卒於家,年七十五。



 師魯自幼莊重若成人,孝友天至。初為番禺簿,喪父以歸,扶柩航海,水程數
 千里,甫三日登於岸,而颶風大作,人以為孝感。常曰:「窮達自有定分,枉道希世,徒喪所守。」故其大節確如金石,雖動與俗情不合,而終翕然信服。嘉泰二年,詔特賜謚曰定肅。



 袁樞,字機仲,建之建安人。幼力學,嘗以《修身為弓賦》試國子監,周必大、劉珙皆期以遠器。試禮部,詞賦第一人,調溫州判官,教授興化軍。



 乾道七年,為禮部試官,就除太學錄,輪對三疏,一論開言路以養忠孝之氣,二論規
 恢復當圖萬全,三論士大夫多虛誕、僥榮利。張說自閣門以節鉞簽樞密,樞方與學省同僚共論之,上雖容納而色不怡。樞退詣宰相,示以奏疏,且曰:「公不恥與噲等伍邪?」虞允文愧甚。樞即求外補,出為嚴州教授。



 樞常喜誦司馬光《資治通鑒》,苦其浩博,乃區別其事而貫通之,號《通鑒紀事本末》。參知政事龔茂良得其書,奏於上,孝宗讀而嘉嘆,以賜東宮及分賜江上諸帥,且令熟讀,曰:「治道盡在是矣。」



 他日,上問袁樞今何官,茂良以實對,上
 曰:「可與寺監簿。」於是以大宗正簿召登對,即因史書以言曰:「臣竊聞陛下嘗讀《通鑒》,屢有訓詞,見諸葛亮論兩漢所以興衰,有『小人不可不去』之戒,大哉王言,垂法萬世。」遂歷陳往事,自漢武而下至唐文宗偏聽奸佞,致於禍亂。且曰:「固有詐偽而似誠實,憸佞而似忠鯁者,茍陛下日與圖事於帷幄中,進退天下士,臣恐必為朝廷累。」上顧謂曰:「朕不至與此曹圖事帷幄中。」樞謝曰:「陛下之言及此,天下之福也。」



 遷太府丞。時士大夫頗有為黨與
 者。樞奏曰:「人主有偏黨之心,則臣下有朋黨之患。比年或謂陛下寵任武士,有厭薄儒生之心,猜疑大臣,親信左右,內庭行廟堂之事,近侍參軍國之謀。今雖總權綱,專聽覽,而或壅蔽聰明,潛移威福。願可否惟聽於國人,毀譽不私於左右。」上方銳意北伐,示天下以所向。樞奏:「古之謀人國者,必示之以弱,茍陛下志復金仇,臣願蓄威養銳,勿示其形。」復陳用宰執、臺諫之術。



 時議者欲制宗室應舉鎖試之額,限添差岳祠,減臣僚薦舉,定文武
 任子,嚴特奏之等,展郊禋之歲,緩科舉之期,樞謂:「此皆近來從窄之論,人君惟天是則,不可行也。」遂抗疏勸上推廣大以存國體。



 兼國史院編修官,分修國史傳。章惇家以其同里,宛轉請文飾其傳,樞曰:「子厚為相,負國欺君。吾為史官,書法不隱,寧負鄉人,不可負天下後世公議。」時相趙雄總史事,見之嘆曰:「無愧古良史。」



 權工部郎官,累遷兼吏部郎官。兩淮旱,命廉視真、楊、廬、和四郡。歸陳兩淮形勢,謂:「兩淮堅固則長江可守,今徒知備江,不
 知保淮,置重兵於江南,委空城於淮上,非所以戒不虞。瓜洲新城,專為退保,金使過而指議,淮人聞而嘆嗟。誰為陛下建此策也?」



 遷軍器少監,除提舉江東常平茶鹽,改知處州,赴闕奏事。樞之使淮入對也,嘗言:「朋黨相附則大臣之權重,言路壅塞則人主之勢孤。」時宰不悅。至是又言:「威權在下則主勢弱,故大臣逐臺諫以蔽人主之聰明;威權在上則主勢強,故大臣結臺諫以遏天下之公議。今朋黨之舊尚在,臺諫之官未正紀綱,言路將
 復荊榛矣。」



 除吏部員外郎,遷大理少卿。通州民高氏以產業事下大理,殿中侍御史冷世光納厚賂曲庇之,樞直其事以聞,人為危之。上怒,立罷世光,以朝臣劾御史,實自樞始。手詔權工部侍郎,仍兼國子祭酒。因論大理獄案請外,有予郡之命,既而貶兩秩,寢前旨。光宗受禪,敘復元官,提舉太平興國宮、知常德府。



 寧宗登位,擢右文殿修撰、知江陵府。江陵瀕大江,歲壞為巨浸,民無所托。楚故城楚觀在焉,為室廬,徙民居之,以備不虞。種木
 數萬,以為捍蔽,民德之。尋為臺臣劾罷,提舉太平興國宮。自是三奉祠,力上請制,比之疏傅、陶令。開禧元年,卒,年七十五。



 自是閑居十載,作《易傳解義》及《辯異》、《童子問》等書藏於家。



 李椿,字壽翁,洺州永年人。父升,進士起家。靖康之難,升翼其父,以背受刃,與長子俱卒。椿年尚幼,蒿殯佛寺,深HC而詳識之;奉繼母南走,艱苦備嘗,竭力以養。以父澤,補迪功郎,歷官至寧國軍節度推官。治豪民偽券,還陳
 氏田,吏才精強,人稱之。



 張浚闢為制司準備差遣,常以自隨。椿奔走淮甸,綏流民,布屯戍,察廬、壽軍情,相視山水砦險要,周密精審,所助為多。



 隆興元年春,諸將有以北討之議上聞者,事下督府,椿方奉檄至巢,亟奏記浚曰:「復讎伐敵,天下大義,不出督府而出諸將,況藩籬不固,儲備不豐,將多而非才,兵弱而未練,議論不定,縱得其地,未易守也。」既而師出無功。



 浚嘗嘆實才之難,椿曰:「豈可厚誣天下無人,唯不惡逆耳而甘遜志,則庶其
 肯來耳。」浚復除右相,椿知事不可為,勸之去。明年春,浚出視師,椿曰:「小人之黨已勝,公無故去朝廷,蹤跡必危。」復申前說甚苦。浚心是之,而自以宗臣任天下之重,不忍決去,未幾果罷。



 監登聞鼓院,有所不樂,請通判廉州以歸。未上,召對,知鄂州。請行墾田,復戶數千,曠士大闢。



 移廣西提點刑獄,獄未竟者,一以平決之,釋所疑數十百人。奏罷昭州金坑,禁仕者毋市南物。移湖北漕,適歲大侵,官強民振糶,且下其價,米不至,益艱食。椿損所強糶
 數而不遏其直,未幾米舟湊集,價減十三。每行部,必前期戒吏具州縣所當問事列為籍,單車以行,所至取吏卒備使令。凡以例致饋,一不受,言事者請下諸道為式。



 召為吏部郎官,論廣西鹽法,孝宗是其說,遂改法焉。除樞密院檢詳。小吏持南丹州莫酋表,求自宜州市馬者,因簽書張說以聞。椿謂:「邕遠宜近,故遷之,豈無意?今莫氏方橫,奈何道之以中國地里之近?小吏妄作,將啟邊釁,請論如法。」說怒,椿因求去,上慰諭令安職。



 遷左司,復
 請外,除直龍圖閣、湖南運副。兼請十三事,同日報可,大者減桂陽軍月樁錢萬二千緡,損民稅折銀之直,民刻石紀之。



 除司農卿。椿會大農歲用米百七十萬斛,而省倉見米僅支一二月,嘆曰:「真所謂國非其國矣。」力請歲儲二百萬斛為一年之蓄。



 擇臨安守,椿在議中,執政或謂其於人無委曲,上曰:「正欲得如此人。」遂兼臨安府,視事三月,竟以幸不便解去。椿在朝,遇事輒言,執政故不悅。及是轉對,又言:「君以剛健為體而虛中為用,臣以
 柔順為體而剛中為用。陛下得虛中之道,以行剛健之德矣。在廷之臣,未見其能以剛中守柔順而事陛下者也。」執政滋不悅,出知婺州。



 會詔市牛筋,凡五千斤。椿奏:「一牛之筋才四兩,是欲屠二萬牛也。」上悟,為收前詔。



 除吏部侍郎,又極言閽寺之盛,曰:「自古宦官之盛衰,系國家興亡。其盛也,始則人畏之,甚則人惡之,極則群起而攻之。漢、唐勿論,靖康、明受之禍未遠,必有以裁制之,不使至極,則國家免於前日之患,宦官亦保其富貴。門禁
 宮戒之外,勿得預外事,嚴禁士大夫兵將官與之交通。」上聞靖康、明受語,蹙頞久之,曰:「幼亦聞此。」因納疏袖中以入。最後極言:「當預邊備,如欲保淮,則楚州、盱眙、昭信、濠梁、渦口、花靨、正陽、光州皆不可以不守;如欲保江,則高郵、六合、瓦梁、濡須、巢湖、北峽亦要地也。」



 以病請祠,不許,面請益力,乃除集英殿修撰、知寧國府,改太平州,賜尚方珍劑以遣。既至,力圖上流之備,請選將練習,緩急列艦,上可以援東關、濡須,下可以應採石。



 年六十九,上章
 請老,以敷文閣待制致仕。越再歲,上念湖南兵役之餘,欲鎮安之,謂椿重厚可倚,命待制顯謨閣、知潭州、湖南安撫使。累辭不獲,乃勉起,至則撫摩凋瘵,氣像一如盛時。復酒稅法,人以為便。歲旱,發廩勸分,蠲租十一萬,糶常平米二萬,活數萬人。



 潭新置飛虎軍,或以為非便,椿曰:「長沙一都會,控扼湖、嶺,鎮撫蠻徭,二十年間,大盜三起,何可無一軍?且已費縣官緡錢四十二萬,何可廢耶?亦在馭之而已。」未滿歲,復告歸,進敷文閣直學士致仕,
 朝拜命,夕登舟,歸老野塘上。



 椿年十五歲避地南來,貧無以為養,不得專力於學。年三十始學《易》,其言於朝廷,措諸行事,皆《易》之用。嶷然有守,存心每主於厚,尤惡佛老邪說。



 淳熙十年,卒,年七十三。朱熹嘗銘其墓,謂其「逆知得失,不假蓍龜」,「不阿主好,不詭時譽」云。



 劉儀鳳,字韶美,普州人。少以文謁左丞馮澥,澥甚推許,遂知名。紹興二年,登進士第。抱負倜儻,不事生產,於仕進恬如也。擢第十年,始赴調,尉遂寧府之蓬溪,監資州
 資陽縣酒稅,為果州、榮州掾。



 紹興二十七年,有旨令侍從薦士,起居郎趙逵舉儀鳳,稱其「富有詞華,恬於進取。」宰執上其名,上曰:「蜀人道遠,文學行義有可用者,不由論薦,何緣知之?前此蜀仕宦者例多隔絕,不得一至朝廷,殊可惜也。」自秦檜專權,深抑蜀士,故上語及之。尋除諸王宮大小學教授。召試館職,辭以久離場屋,改國子監丞。宰相以其名士,遷秘書丞、禮部員外郎。所草箋奏,以典雅稱。



 孝宗受禪,議上「光堯壽聖」尊號冊寶,有欲俟
 欽宗服除者,太常博士林慄謂:「唐憲宗上順宗冊寶在德宗服中,不必避,備樂而不作可也。」儀鳳獨上議曰:「謹按上尊號事屬嘉禮,累朝必俟郊祀慶成然後舉行。太上皇帝為欽宗備禮終制,見於詔書。議者引憲宗故事,考之唐史,自武德以來,皆用易月之制,與本朝事體大相遠也。乞候欽宗終制,檢舉以行,則國家盛美,主上事親情實稱矣。」議者雖是其言,然謂事親當權宜而從厚,竟用慄議,儀鳳復爭辨不已。尋兼國史院編修官兼權
 秘書少監。乾道元年,遷兵部侍郎兼侍講。



 儀鳳在朝十年,每歸即匿其車騎,扃其門戶,客至,無親疏皆不得見,政府累月始一上謁,人尤其傲,奉入,半以儲書,凡萬餘卷,國史錄無遺者。御史張之綱論儀鳳錄四庫書本以傳私室,遂斥歸蜀。



 三年十二月,輔臣進前侍從當復職者,上曰:「劉儀鳳無罪,可與復集英殿修撰。」起知邛州,未上,改漢州、果州,罷歸。淳熙二年十二月丙申,卒,年六十六。



 儀鳳苦學,至老不倦,尤工於詩。然頗慕晉人簡傲之
 風,不樂與庸輩接,故平生多蹭蹬,一跌遂不振云。



 張孝祥,字安國,歷陽烏江人。讀書過一目不忘,下筆頃刻數千言,年十六,領鄉書,再舉冠里選。紹興二十四年,廷試第一。時策問師友淵源,秦塤與曹冠皆力攻程氏專門之學,孝祥獨不攻。考官已定塤冠多士,孝祥次之,曹冠又次之。高宗讀塤策皆秦檜語,於是擢孝祥第一,而塤第三,授承事郎、簽書鎮東軍節度判官。諭宰相曰:「張栻孝祥詞翰俱美。」



 先是,上之抑塤而擢孝祥也,秦檜已
 怒,既知孝祥乃祁之子,祁與胡寅厚,檜素憾寅,且唱第後,曹泳揖孝祥於殿庭,以請婚為言,孝祥不答,泳憾之。於是風言者誣祁有反謀,系詔獄。會檜死,上郊祀之二日,魏良臣密奏散獄釋罪,遂以孝祥為秘書省正字。故事,殿試第一人,次舉始召,孝祥第甫一年得召由此。



 初對,首言乞總攬權綱以盡更化之美。又言:「官吏忤故相意,並緣文致,有司觀望鍛煉而成罪,乞令有司即改正。」又言:「王安石作《日錄》,一時政事,美則歸己。故相信任之
 專,非特安石。臣懼其作《時政記》,亦如安石專用己意,乞取已修《日歷》詳審是正,黜私說以垂無窮。」從之。



 遷校書郎。芝生太廟,孝祥獻文曰《原芝》,以大本未立為言,且言:「芝在仁宗、英宗之室,天意可見,乞早定大計。」遷尚書禮部員外郎,尋為起居舍人、權中書舍人。



 初,孝祥登第,出湯思退之門,思退為相,擢孝祥甚峻。而思退素不喜汪澈,孝祥與澈同為館職,澈老成重厚,而孝祥年少氣銳,往往陵拂之。至是澈為御史中丞,首劾孝祥奸不在廬
 杞下,孝祥遂罷,提舉江州太平興國宮,於是湯思退之客稍稍被逐。



 尋除知撫州。年未三十,蒞事精確,老於州縣者所不及。孝宗即位,復集英殿修撰,知平江府。事繁劇,孝祥剖決,庭無滯訟。屬邑大姓並海囊橐為奸利,孝祥捕治,籍其家得穀粟數萬。明年,吳中大饑,迄賴以濟。



 張浚自蜀還朝,薦孝祥,召赴行在。孝祥既素為湯思退所知,及受浚薦,思退不悅。孝祥入對,乃陳「二相當同心戮力,以副陛下恢復之志。且靖康以來惟和戰兩言,
 遺無窮禍,要先立自治之策以應之。」復言:「用才之路太狹,乞博採度外之士以備緩急之用。」上嘉之。



 除中書舍人,尋除直學士院兼都督府參贊軍事。俄兼領建康留守,以言者改除敷文閣待制,留守如舊。會金再犯邊,孝祥陳金之勢不過欲要盟。宣諭使劾孝祥落職,罷。



 復集英殿修撰、知靜江府、廣南西路經略安撫使,治有聲績,復以言者罷。俄起知潭州,為政簡易,時以威濟之,湖南遂以無事。復待制,徙知荊南、荊湖北路安撫使。築寸金堤,自是
 荊州無水患,置萬盈倉以儲諸漕之運。



 請祠,以疾卒,孝宗惜之,有用才不盡之嘆。進顯謨閣直學士致仕,年三十八。



 孝祥俊逸,文章過人,尤工翰墨,嘗親書奏札,高宗見之,曰:「必將名世。」但渡江初,大議惟和戰,張浚主復仇,湯思退祖秦檜之說力主和,孝祥出入二人之門而兩持其說,議者惜之。



 論曰:尤袤學本程頤,所謂老成典刑者,立朝抗論,與人主爭是非,不允不已,而能令終完節,難矣。謝諤、顏師魯、
 袁樞臨民則以治辨聞,立朝則啟沃忠諫,各舉乃職,為世師表。李椿、劉儀鳳言論節概,著於行事。張孝祥早負才畯,蒞政揚聲,迨其兩持和戰,君子每嘆息焉。



\end{pinyinscope}