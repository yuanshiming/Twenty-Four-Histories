\article{列傳第一百六十六}

\begin{pinyinscope}

 ○杜範楊簡錢時附張虙呂午子沆



 杜範,字成之,黃巖人。少從其從祖熚、知仁游,從祖受學朱熹,至範益著。嘉字定元年舉進士,調金壇尉,再調婺州司法。紹定三年,主管戶部架閣文字。六年,遷大理司直。



 端平元年,改授軍器監丞。明年,入對,言:「陛下親覽大政,兩年於茲。今不惟未睹更新之效,而或者乃有浸不如舊之憂。夫致弊必有原,救弊必有本,積三四十年之蠹習,浸漬薰染,日深日腐,有不可勝救者,其原不過私之一字耳。陛下固宜懲其弊原,使私意凈盡。顧以天位之重而或藏其私憾,天命有德而或濫於私予,天討有罪而或制於私情,左右近習之言或溺於私聽,土木無益之工或侈於私費,隆禮貌以尊賢而用之未盡,溫辭色
 以納諫而行之惟艱,此陛下之私有未去也。和衷之美不著,同列之意不孚,紙尾押敕,事不預知,同堂決事,莫相可否,集義盈庭而施行決於私見,諸賢在列而密計定於私門,此大臣之私有末去也。君相之私容有未去,則教條之頒徒為虛文。近者召用名儒,發明格物致知、誠意正心之學,有好議論者,乃從而詆訾訕笑之,陛下一惑其言,即有厭棄儒學之意。此正賢不肖進退之機,天下安危所系,願以其講明見之施行。」



 改秘書郎,尋拜
 監察御史。奏:「曩者權臣所用臺諫,必其私人,約言已堅,而後出命。其所彈擊,悉承風旨,是以紀綱蕩然,風俗大壞。陛下親政,首用洪咨夔、王遂,痛矯宿弊,斥去奸邪。然廟堂之上,奉制尚多。言及貴近,或委曲回護,而先行丐祠之請;事有掣肘,或彼此調停,而卒收論罪之章。亦有彈墨尚新而已頒除目,沙汰未幾而旋得美官。自是臺諫風採,昔之振揚者日以鑠;朝廷紀綱,昔之漸起者日以壞。」理宗深然之。



 又奏九江守何炳年老不足備風寒,
 事寢不行。範再奏曰:「一守臣之未罷其事小,臺諫之言不行其事大。阻臺諫之言猶可也,至於陛下之旨匿而不行,此豈勵精親政之時所宜有哉!」丞相鄭清之見之大怒,五上章丐去,有「危機將發,朋比禍作」之語;且謂範順承風旨,粉飾擠陷。範遂自劾,言:「宰相之與臺諫,官有尊卑而事關一體,但當同心為國,豈容以私而害公。行之者宰相,言之者臺諫。行之者豈盡合於事宜,言之者或未免於攻詆,清明之朝,此特常事。古者大臣欲扶持
 紀綱,故必崇獎臺諫,聞有因言而待罪者矣,未聞有諱言而含怒者也。曩者柄臣所用臺諫,必其私人;陛下更新庶政,而臺諫皆出於親擢。若廟堂不欲臣言其親故,鉗其口,奪其氣,則與曩者之用私人何以異?不知所謂『承順風旨』者何人?『粉飾擠陷』者何事?乞檢臣前奏,賜之罷黜,以從臣退安田里之欲。」



 時清之妄邀邊功,用師河、洛,兵民死者十數萬,資糧器甲悉委於敵,邊境騷然,中外大困。範率合臺論其事,並言制閫之詐謀罔上。於是
 凡侍從、近臣之不合時望者,監司、郡守之貪暴害民者,皆以次論斥。清之愈忌之,改太常少卿。轉對言:「今日之病,莫大於賄賂交結之風。名譽已隆者賈左右之譽以固寵,宦游未達者惟梯級之求以進身。邊方帥臣,黃金不行於反間,而以探刺朝廷;厚賜不優於士卒,而以交通勢要。以致賞罰顛倒,威令慢褻,罪貶者拒命而不行,棄城者巧計以求免,提援兵者召亂而肆掠,當重任者怙勢而奪攘。下至禁旅,驕悍難制,監軍群聚相剽劫。欲
 望陛下毋以小恩廢大誼,毋以私情撓公法,嚴制宮掖,不使片言得以入於閫;禁約閹宦,不使讒諂得以售其奸。」範自入臺,屢丐祠,至是復五上歸田之請,皆不允。



 遷秘書監兼崇政殿說書。大元兵徇江陵,範乞屯兵蘄、黃以防窺江,且令沿江帥臣兼江、淮制置大使以重其權,令淮西帥臣急調兵撥糧以援江陵。拜殿中侍御史,辭不獲,乃因講筵,奏:「臣嘗冒耳目之寄,輒忤宰相,至煩陛下委曲調護,今又使居向者負芒之地,豈以臣絕私比,
 而其言猶有可取耶?抑以臣巽懦之質,易於調護,而姑使之備數耶?昔人主之於諍臣,非樂而聽之,即勉而從之,否則疏而遠之,未聞有不用其言而復用其人者。陛下自端平親政以來,召用正人以振臺綱,未幾而有委曲調護之弊,其所彈擊,或牽制而不行,其所斥逐,復因緣以求進。臣於入臺之初,固已力言之,不惟不之革,而其弊滋甚,甚至節貼而文理不全,易寫而臺印無有,中書不敢執奏,見者為之致疑。不意聖明之時,其弊一至
 於此。陛下以其言之不可用,又從而超遷之,則是臺諫之官,專為仕途之捷徑。陛下但知崇獎臺諫為盛德,而不知阻抑直言之為弊政,則陛下外有好諫之名,內有拒諫之實,天下豈有虛可以蓋實哉。」範始以不得其言不去為恨,至是遂極言臺諫失職之弊。



 時襄、蜀俱壞,江陵孤危,兩浙震恐,復言:「清之橫啟邊釁,幾危宗祀,及其子招權納賄,貪冒無厭,盜用朝廷錢帛以易貨外國,且有實狀。」並言:「簽書樞密院事李鳴復與史寅午、彭大雅
 以賄交結,曲為之地。鳴復既不恤父母之邦,亦何有陛下之社稷。」帝以清之潛邸舊臣,鳴復未見大罪,未即行,範亦不入臺。帝促之,範奏:「鳴復不去則臣去,安敢入經筵?」方再奏之,鳴復抗疏自辨,言:「臺臣論臣,不知所指何事,豈以臣嘗主和議耶?幸未斥退,則安國家、利社稷,死生以之;否則無家可歸,惟有扁舟五湖耳。」範又極言其寡廉鮮恥,既而合臺劾之,太學諸生亦上書交攻之。鳴復將出關,帝又遣使召回,範復與合臺奏:「鳴復為宰執,
 所交惟史寅午、彭大雅,此等相與陰謀,不過賂近習、蒙上聽,以陰圖相位。臣近見自辨之章,見其交鬥邊臣以啟嫌隙,妄言和戰以肆脅持,且以蜀既破蕩而欲泛舟五湖,又以安國家、利社稷自任,不知鳴復久居政府,今又有何安利之策?欺君罔上,無所不至。如臣等言是,即乞行之;所言若非,早賜罷斥。」改起居郎,範奏:「臣論鳴復,未見施行,忽拜左史之命,則是所言不當,姑示優遷。臣前者嘗奏臺諫但為仕途之捷徑,初無益朝廷之紀綱,
 躬言之,躬蹈之,臣之罪大矣。」即渡江而歸。授江東提點刑獄,尋改浙西提點刑獄,範力辭之,而鳴復亦出守越。



 嘉熙二年,差知寧國府。明年至郡,適大旱,範即以便宜發常平粟,又勸寓公富人有積粟者發之,民賴以安。始至,倉庫多空,未幾,米餘十萬斛,錢亦數萬,悉以代輸下戶糧。兩淮饑民渡江者多剽掠,其首張世顯尤勇悍,擁眾三千餘人至城外。範遣人犒之,俾勿擾以俟處分,世顯乃陰有窺城之意。範以計擒斬之,給其眾使歸。



 四年,
 還朝,首言:



 旱暵荐臻,人無粒食。楮券猥輕,物價騰踴。行都之內,氣象蕭條,左浙近輔,殍死盈道。流民充斥,未聞安輯之政,剽掠成風,已開弄兵之萌,是內憂既迫矣。新興北兵,乘勝而善鬥,中原群盜,假名而崛起。搗我巴蜀,據我荊襄,擾我淮堧,近又由夔、峽以瞰鼎、澧。疆場之臣,肆為欺蔽,勝則張皇而言功,敗則掩覆而不言。脫使乘上流之無備,為飲馬長江之謀,其誰與捍之?是外患既深矣。



 人主上所事者天,下所恃者民。近者天文示變,妖
 彗吐芒,方冬而雷,既春而雪,海潮沖突於都城,赤地幾遍於畿甸,是不得乎天而天已怒矣。人死於干戈,死於饑饉,父子相棄,夫婦不相保,怨氣盈腹,謗言載路,「等死」一萌,何所不至,是不得乎民而民已怨矣。內憂外患之交至,天心人心之俱失,陛下能與二三大臣安居於天下之上乎?陛下亦嘗思所以致此否乎?



 蓋自曩者權相陽進妾婦之小忠,陰竊君人之大柄,以聲色玩好內蠱陛下之心術,而廢置生殺,一切惟其意之所欲為,以致
 紀綱陵夷,風俗頹靡,軍政不修而邊備廢缺。凡今日之內憂外患,皆權相三十年釀成之,如養護癰疽,待時而決耳。端平號為更化,而居相位者非其人,無能改於其舊,敗壞污穢,殆有甚焉。自是聖意惶惑,莫知所倚仗,方且不以彼為仇而以為德,不以彼為罪而以為功。於是天之望於陛下者孤,而變怪見矣,人之望於陛下者觖,而怨叛形矣。



 陛下敬天有圖,旨酒有箴,緝熙有記,使持此一念,振起傾頹,宜無難者。然聞之道路,謂警懼之意,
 祗見於外朝視政之頃;而好樂之私,多縱於內廷燕褻之際。名為任賢,而左右近習或得而潛間;政出於中書,而御筆特奏或從而中出。左道之蠱惑,私親之請托,蒙蔽陛下之聰明,轉移陛下之心術。



 於是範去國四載矣,帝撫勞備至。



 遷權吏部侍郎兼侍講。以久旱,復言:「陛下嗣膺寶位餘二十年,災異譴告,無歲無之,至於今而益甚。陛下求所以應天者,將止於減膳徹樂、分禱群祀而已乎?抑當外此而反求諸躬乎?夫不務反躬悔過,而徒
 覬天怒之釋,天下寧有是理?欲望陛下一灑舊習以新天下,出宮女以遠聲色,斥近習以防蔽欺,省浮費以給國用,薄徵斂以寬民力。且儲貳未立,國本尚虛,乞選宗姓之賢者育之宮中而教導之。」又言銓法之壞:「廟堂既有堂除,復時取部缺以徇人情;士大夫既陷贓濫,乃間以不經推勘而改正。凡此皆徇私忘公之害。」未幾,復上疏曰:



 天災旱暵,昔固有之。而倉廩匱竭,月支不繼,升粟一千,其增未已,富戶淪落,十室九空,此又昔之所無也。
 甚而闔門饑死,相率投江,里巷聚首以議執政,軍伍誶語所不忍聞,此何等氣象,而見於京城眾大之區。浙西稻米所聚,而赤地千里。淮民流離,襁負相屬,欲歸無所,奄奄待盡。使邊塵不起,尚可相依茍活,萬一敵騎沖突,彼必奔迸南來,或相攜從敵,因為之鄉導,巴蜀之覆轍可鑒也。



 竊意陛下宵旰憂懼,寧處弗遑。然宮中宴賜未聞有所貶損,左右嬙嬖未聞有所放遣,貂璫近習未聞有所斥遠,女冠請謁未聞有所屏絕,朝廷政事未聞有
 所修飭,庶府積蠹未聞有所搜革。秉國鈞者惟私情之徇,主道揆者惟法守之侵,國家大政則相持而不決,司存細務則出意而輒行。命令朝更而夕變,紀綱蕩廢而不存,無一事之不弊,無一弊之不極。陛下盍亦震懼自省。



 詔:「中外臣庶思當今急務,如河道未通,軍餉若何而可運?浙右旱歉,荒政若何而可行?財計空匱,糴本若何而可足?流徙失所,遣使若何而可定?敵情叵測,邊圉若何而可固?各務悉力盡思,以陳持危制變之策。」



 拜吏部
 侍郎兼中書舍人,復極言宴賜不節、修造不時、玩寇縱欲數事。兼權兵部尚書,改禮部尚書兼中書舍人。



 淳祐二年,擢同簽書樞密院事。範既入都堂,凡行事有得失,除授有是非,悉抗言無隱情。丞相史嵩之外示寬容,內實忌之。四年,遷同知樞密院事。以李鳴復參知政事,範不屑與鳴復共政,去之。帝遣中使召回,且敕諸城門不得出範。太學諸生亦上書留範而斥鳴復,並斥嵩之。嵩之令諫議大夫劉晉之等論範及鳴復,範遂行。會嵩之
 遭喪謀起復不果,於是拜範右丞相,範以遜游似,不許,遂力疾入覲。帝親書「開誠心,布公道。集眾思,廣忠益」賜之。



 範上五事:「曰正治本,謂政事當常出於中書,毋使旁蹊得竊威福。曰肅宮闈,謂當嚴內外之限,使宮府一體。曰擇人才,謂當隨其所長用之而久於職,毋徒守遷轉之常格。曰惜名器,謂如文臣貼職,武臣閣衛,不當為徇私市恩之地。曰節財用,謂當自人主一身始,自宮掖始,自貴近始,考封村國用出入之數,而補窒其罅漏,求鹽
 策楮幣變更之目,而斟酌其利害。仍乞早定國本以系人心。」



 時親王近戚多求降恩澤,引前朝杜衍例,範皆封還。乞撥堂除闕歸之吏部,以清中書之務,惟留書庫、架閣、京教及要地幹官。人皆以為不便。太學生亦上書言之,帝以示範,範奏曰:「三四十年權臣柄國,以公朝爵祿而市私恩,取吏部之闕以歸堂除,太學諸生亦習於見聞,乃以近年之弊政為祖宗之成法。如以臣言為是,上下堅守,則諛者必多而謗者息矣。」未幾,赴選調者無淹
 滯,合資格者得美闕,眾始服。



 帝命宰執各條當今利病與政事可行者,範上十二事:



 曰公用舍,願進退人才悉參以國人之論,則乘罅抵巇者無所投其間。曰儲材能,內而朝列,則儲宰執於侍從、臺諫,儲侍從、臺諫於卿監、郎官;外而守帥,則以江面之通判為幕府、郡守之儲,以江面之郡守為帥閫之儲;他職皆然,如是則臨時無乏才之憂。曰嚴薦舉,宜詔中外之臣,凡薦舉必明著職業、功狀、事實,不許止為褒詞,朝廷籍記不如所舉,並罰舉
 主,仍詔侍從、臺諫不許與人覓舉。曰懲贓貪,自今有以贓罪案上,即行下勘證,果有贓敗,必繩以祖宗之法,無實跡而監司妄以贓罪誣人者,亦量行責罰,臺諫風聞言及贓罪,亦行下勘證。曰專職任,吏部不可兼給、舍,京尹不可兼戶、吏,經筵亦必專官。曰久任使,內而財賦、獄訟、銓選與其他煩劇之職,必三年而後遷,外而監司、郡守,亦必使之再任,其不能者則亟行罷斥。曰抑僥幸,布告中外,各務職業,朝廷不以弊例而過恩,宮庭不以私
 謁而廢法;勛舊之家,邸第之戚,不以名器而輕假。曰重閫寄。曰選軍實。曰招土豪。曰宜仿祖宗方田之制,疏為溝洫,縱橫經緯,各相灌注,以鑿溝之土,積而為徑,使不得並轡而馳,結陣而前,如曹瑋守陜西之制,則戎馬之來,所至皆有阻限,而溝之內又可以耕屯,勝於陸地多矣。曰治邊、理財,實為當今急務,有明於治邊、善於理財者,搜訪以聞。



 時孟珙權重兵久居上流,朝廷素疑其難制,至是以書來賀。範復之曰:「古人謂將相調和則士豫
 附,自此但相與同心徇國。若以術相籠架,非範所屑為也。」珙大感服。未幾,大元軍大入五河,絕中流,置營柵,且以重兵綴合肥,令不得相援,為必取壽春之計。範命惟揚、鄂渚二帥各調兵東西來應,卒以捷聞。範計功行賞,莫不曲當,軍士皆悅。



 未幾,卒,贈少傅,謚清獻。其所著述,有古律詩歌詞五卷,雜文六卷,奏稿十卷,外制三卷,《進故事》五卷,《經筵講義》三卷。



 楊簡,字敬仲,慈溪人。乾道五年舉進士,授富陽主簿。會
 陸九淵道過富陽,問答有所契,遂定師弟子之禮。富陽民多服賈而不知學,簡興學養士,文風益振。



 為紹興府司理,犴獄必親臨,端默以聽,使自吐露。越陪都,臺府鼎立,簡中平無頗,惟理之從。一府史觸怒帥,令鞫之,簡白無罪,命鞫平日,簡曰:「吏過詎能免,今日實無罪,必擿往事置之法,某不敢奉命。」帥大怒,簡取告身納之,爭愈力。常平使者朱熹薦之。先是,丞相史浩亦以簡薦,差浙西撫幹,白尹張枃,宜因兇歲戒不虞。乃令簡督三將兵,接
 以恩信,出諸葛亮正兵法肄習之,軍政大修,眾大和悅。



 改知嵊縣。丁外艱,服除,知樂平縣,興學訓士,諸生聞其言有泣下者。楊、石二少年為民害,簡置獄中,諭以禍福,咸感悟,願自贖。由是邑人以訟為恥,夜無盜警,路不拾遺。紹熙五年,召為國子博士。二少年大帥縣民隨出境外,呼曰「楊父」。會斥丞相趙汝愚,祭酒李祥抗章辨之,簡上書言:「昨者危急,軍民將潰亂,社稷將傾危,陛下所親見。汝愚冒萬死易危為安,人情妥定,汝愚之忠,陛下所
 心知,不必深辨。臣為祭酒屬,日以義訓諸生,若見利忘義,畏害忘義,臣恥之。」未幾,亦遭斥,主管崇道觀。再任,轉朝奉郎。嘉泰四年,賜緋衣銀魚,朝散郎,權發遣全州,以言罷,主管仙都觀。



 嘉定元年,寧宗更化,授秘書郎,轉朝請郎,遷秘書省著作佐郎兼權兵部郎官。轉對,極言經國之要,弭災厲、消禍變之道,北境傳誦,為之涕泣。詔以旱蝗求直言,簡上封事,言旱蝗根本,近在人心。兼考功郎官,兼禮部郎官,授著作郎、將作少監。入對,答問往復,
 漏過八刻,上目送久之。兼國史院編修官兼實錄院檢討官,以面對所陳未行,求外補,知溫州。移文首罷妓籍,尊敬賢士。私鹺五百為群過境內,分司干官檄永嘉尉及水砦兵捕之。巡尉不白郡,簡驚曰:「是可輕動乎?萬一召亂,貽朝廷憂。兵之節制在郡將,違節制是不嚴天子命,違節制應斬。」建旗立巡尉庭下,召劊手兩行夾立,郡官盛服立西序,數其罪,命斬之,郡官交進為致悔罪意,良久得釋,奏罷分司,其紀律如此。寓官置民田負其直,
 簡追其隸責之而償所負。勢家第宅障官河,即日撤之,城中歡踴,名楊公河。



 帝遣使至郡譏察,使於簡為先世契,出郊迎,不敢當,從間道走州入客位。簡聞之不敢入,往來傳送數四,乃驅車反。將降車,使者趨出立戟門外,簡亦趨出立使者外,頓首言曰:「天使也,某不敢不肅。」使者曰:「契家子,禮有常尊。」簡曰:「某守臣,使者銜天子命,辱臨敝邑,天使也,某不改不肅。」遂從西翼偕進,禮北面東上,簡行則常西,步則後,及階,莫敢升,已乃同升自西階,
 足踧踧莫敢就主席,使者曰:「邦君之庭也,禮有常尊。」簡曰:「《春秋》,王人雖微,例書大國之上,尊天子也。況今天使乎?」持之益堅,使者辭益力,如是數刻,使者知不可變,乃曰:「某不敏,敢不敬承執事尊天子之義。」即揖而出。既就館,簡乃以賓禮見。儀典曠絕,邦人創見之,莫不瞿然竦觀,屏息立。



 簡在郡廉儉自將,奉養菲薄,常曰:「吾敢以赤子膏血自肥乎!」閭巷雍睦無忿爭聲,民愛之如父母,咸畫象事之。遷駕部員外郎,老稚扶擁緣道,傾城哭送。入
 對,言:「盡掃喜順惡逆之私情,善政盡舉,弊政盡除,民怨自銷,禍亂不作。」改工部員外郎,轉對,又以擇賢久任為言。遷軍器監兼工部郎官,轉朝奉大夫,又遷將作監兼國史院編修官兼實錄院檢討官,轉朝散大夫。



 金人大饑,來歸者日以數千、萬計。邊吏臨淮水射之。簡戚然曰:「得土地易,得人心難。薄海內外,皆吾赤子,中土故民,出塗炭,投慈父母,顧靳斗升粟而迎殺之,蘄脫死乃速得死,豈相上帝綏四方之道哉?」即日上奏,哀痛言之,不報。
 會有疾,請去益力,乃以直寶謨閣主管玉局觀。升直寶文閣主管明道宮、秘閣修撰主管千秋鴻禧觀。特授朝請大夫、右文殿修撰主管鴻慶宮,賜紫衣金魚。進寶謨閣待制、提舉鴻慶宮,賜金帶。



 理宗即位,進寶謨閣直學士,賜金帶。寶慶元年,轉朝議大夫、慈溪縣男,尋授華文閣直學士、提舉祐神觀,奉朝請。詔入見,簡屢辭。授敷文閣直學士,累加中大夫,仍提舉鴻慶宮,尋以寶謨閣學士、太中大夫致仕,卒,贈正奉大夫。



 簡所著有《甲稿》、《乙稿》、《
 冠記》、《昏記》《喪禮家記》、《家祭記》、《釋菜禮記》、《石魚家記》,又有《己易》、《啟蔽》等書,其論治務最急者五,其次八。一曰謹擇左右大臣、近臣、小臣;二曰擇賢以久任中外之官;三曰罷科舉而行鄉舉里選;四曰罷設法道淫;五曰治伍法,修諸葛武侯之正兵,以備不虞。其次急者有八:一曰募兵屯田,以省養兵之費;二曰限民田,以漸復井田;三曰罷妓籍,從良;四曰漸罷和買、折帛暨諸無名之賦及榷酤,而禁群飲;五曰擇賢士教之大學,教成,使分掌諸州
 之學,又使各擇井里之士聚而教之,教成,使各分掌其邑里之學;六曰取《周禮》及古書,會議熟講其可行於今者行之;七曰禁淫樂;八曰修書以削邪說。此簡之志也。後咸淳間,制置使劉黻即其居作慈湖書院。門人錢時。



 時字子是,淳安人。幼奇偉不群,讀書不為世儒之習。以《易》冠漕司,既而絕意科舉,究明理學。江東提刑袁甫作象山書院,招主講席,學者興起,政事多所裨益。郡守及新安、紹興守皆厚禮延請,開講郡庠。其學大抵發明人
 心,論議宏偉,指擿痛決,聞者皆有得焉。丞相喬行簡知其賢,特薦之朝,且曰:「時夙負才識,尤通世務,田里之休戚利病,當世之是非得失,莫不詳究而熟知之,不但通詩書、守陳言而已。」



 授秘閣校勘。詔守臣以時所著書來上。未幾,出佐浙東倉幕,太史李心傳奏召史館檢閱。轉對,敷陳剴切,皆聖賢之精微。旋以國史宏綱未畢求去,授江東帥屬,歸。其書有《周易釋傳》、《尚書演義》《學詩管見》、《春秋大旨》、《四書管見》、《兩漢筆記》、《蜀阜集》、《冠昏記》、《百行冠
 冕集》。寶祐間,守季鏞祠於學。



 張虙,字子宓,慈溪人。慶元二年進士。故事,潛邸進士升名,虙不以自陳。授州教授。為浙東帥屬。帥督新昌舊逋,虙手書諫曰:「越人之瘠,宜咻噢撫摩之。今夏稅當寬為之期,使田里久饑之甿,少還已耗之氣血,尚可理舊逋耶?」力辭不行。



 主管戶部架閣文字,改太學正。時新進者多逞小才、害大體,轉對言:「立國有大經,人主當以靜制天下之動。今日之治,或有鄰於鍥薄,而咈人心、傷國體
 者,宜有以革之,使祖宗之意常如一日可也。」帝嘉納焉。



 遷太常博士,又遷國子博士。時金垂亡,因論自治之道,謂:「天下之治,必有根本。城郭所以禦敵也,使溝壑有轉徙之民,則何敵之能禦?儲峙所以備患也,使枵腹盻盻不得食,則何患之能備?今日之吏,能知守邊之務者多,而能明立國之意者少。繕城郭,聚米粟,恃此而不恤乎民,則其策下矣。」



 時以旱求言,即上疏曰:「上天之心即我祖宗之心,數年以來,蓋有為祖宗所不敢為者。凡祖宗
 之時,幾舉而不遂,已行而復寢,始以人言而從,終以國體而回者,今皆處之以不疑矣。凡祖宗長慮卻顧,所以銷惡運、遏亂原、兢兢相與守之者,皆變於目前利便快意之謀矣。議者惟知衰靡之俗不可不振起也,圮壞之風不可不整刷也,抑不知振起整刷之術,最難施於衰靡圮壞之後。何者?元氣已傷而不可再擾,人心方蘇而不可駭動也。且造楮初欲便民,朝廷既以一切之政駴其聽,復以一定之價迫之從,郡縣之間,遂騷然矣。監司、
 郡守老成遲鈍者悉屏而不用,而取夫新進喜功名者為之,見事則風生,臨事則痛決,事未果集而根本已朘,國未有益而民生已困矣。凡此皆有累於祖宗仁厚之德,此旱勢之所以彌甚也。」



 遷國子監丞。轉對,願力主正論,勿使迎合之人得以投吾機。遷秘書郎,預編《寧宗會要》兼吳、益王府教授,改兼莊文府。講《毛詩》終篇,乞以所讀諸子改讀《尚書》,帝曰:「吾固以《詩》、《書》成麟趾之美也。」



 遷著作佐郎兼權都官郎官。轉對言:「邊事有二病,戒敕千
 條,猶患悖繆,指意明白,猶復背違,安有不示其所向而謂可責其成。且言戰則當知彼,言和則當請於彼,惟守則自求諸己而已。儻以為可,則當力主其說,明告天下,日講求其所以守之之策,蓋議論貴合一,而今則病乎雜也。用人不可以嘗試,任人不可以自疑。朝廷惟慮獨任之難勝,彼此互分,不相扶持,人得抗衡,莫有稟屬,制置但存虛器,便宜反出多門。蓋體貴合一,而今則病乎分也。」



 遷秘書丞,改著作郎。以疾乞外,出知南康。至郡,剖
 決滯訟,眾皆悅服。前守陳宓以錢七千緡置濟民庫為築城費,虙至,曰:「不必取贏於民,吾捐萬緡為倡,繼是儻不已,何患事之難成。」轉運使以錢萬二千緡置平糴於郡,虙復出錢萬二千緡以增益之,民賴其利。將增建禁旅,營地屬民者,索質劑視元直償之。徙知處州,移知溫州,力辭,遂直秘閣、主管千秋鴻禧觀。參議制置使幕中,使者尚威力,愎諫自用,虙守正不阿,每濟以寬大。又上書論海防利便。主管玉局觀。



 端平初,召為國子司業兼
 侍講,以《禮記月令》進讀,至「獄訟必端平」之語,因敷暢厥旨。八陵來復,將議修奉,而論者未能協一,虙議曰:「當乘此時遣官肅清威儀,申祗奉故事,如或為其所紿,功未即就,亦足以感動天下忠臣義士之心。」力辭勸講之職,升國子祭酒。以為「《月令》之書雖出於呂不韋,然人主後天而奉天時,此書不為無助」。乃因已講者為十二卷,乞按月而觀之。兼權工部侍郎兼國子祭酒,命下而卒,詔贈四官。



 呂午,字伯可,歙縣人。嘉定四年進士,授烏程主簿,郡守致之幕下,事一決於午。守張忠恕,丞相浚之孫,薦午猶力,時忠恕之母就養,而時時躬至簿聽迎午二親入郡,與午皆衣彩衣奉觴上壽,邦人榮之。



 調當塗縣丞。守吳柔勝謂午有操守,俾其子淵、潛定交焉。會司理攝蕪湖縣,廬州遣兩兵會公事,司理遂以廬兵奪縣民為言。柔勝怒,悉置獄,屬午問之。午謂「廬州有公櫝,不可謂奪民」。柔勝愈怒,再以屬午。明日,午入謁,柔勝先令左右問若
 何,午執前說。柔勝益加怒,謂「我不忍廬兵奪吾百姓」。不出迎午,午坐客位不退,不食。柔勝勉為出,怒不息,欲黥二兵。午徐曰:「廬州初無公櫝則可,有則縣不為處置而反罪廬兵,恐不可。」久之,卒從午請,由是柔勝益知午。



 陳貴誼守太平,屬午安集淮南流民。江東提舉徐僑知午在郡,驚喜,闢為幕屬。午欲盡決遣郡事而後行,帖趣行至十八而不以白貴誼,僑貽書貴誼,午始行。既而僑行部,以田事迕丞相史彌遠,以言罷。午還當塗。監溫州天
 富北監鹽場,改知餘杭縣,亦以言罷,公論大不平,然午自此名益重。浙東提舉章良朋留之幕,旋兼沿海制置司事。海寇未平,良朋問策安在。午廉知調軍出海,糧盡即還,軍獲寇物,官盡拘收,乃與制置司干官施一飛議,糧盡再給,不許擅還,賊舟所有,悉以給軍,海道遂清。



 差知龍陽縣。豪民陶守忠殺人,正其獄誅之。彌遠雖非賢相,猶置人才簿,書賢士大夫以待用,而午治縣之政亦書之。差兩浙轉運司主管文字,彌遠病久不見客,午入
 謁,特出迎。運使罷,故不用人,以午護印半年。或問彌遠,何以不注官?彌遠曰:「爾謂護印官不能耶?」午聞之力辭。



 差監三省樞密院門兼監提轄封樁上庫。丁父憂,免喪,遷大府寺簿。拜監察御史,帝親擢也。鄭清之喪師,至是丁黼死於成都,史嵩之、孟珙在京湖,嵩之尋升督府。陳韡、杜杲在淮西,王鑒在黃州,計用兵十七萬人,圍始解。獨趙葵在淮東不受兵,而坐視不出兵應援。午疏論:「邊閫角立,當協心釋嫌,而乃幸災樂禍,無同舟共濟之心。」
 葵以為午黨京湖制司,而嵩之亦憾午,乃遷宗正少卿兼國史院編修官、實錄院檢討官。出知泉州。初,左丞相李宗勉深以葵之言為疑,會來自淮東者,乃言臺官皆以葵交書,獨呂御史無之,宗勉始以午為賢,語人曰:「呂伯可獨立無黨者。」嵩之得彌遠人才簿,心知敬午而內怨所論邊事。及午移浙東提刑,嵩之令鄧詠嗾重復亨論罷,中外不直嵩之。



 提舉崇禧觀,再移浙東提刑。復為監察御史,入見,帝曰:「卿向來議論甚明切。」兼崇政殿說
 書。嵩之雅不欲午在經筵,時殿中侍御史項容孫子娶午從子,嵩之俾容孫上疏避午,欲撼之去,而於法無避。嵩之乃與言路密謀,以為午嘗劾王瓚姻家史洽,遂以瓚為右正言,午即治裝去。上手詔趣留之,午力辭,不允,由是再留,而議論愈不合。



 遷起居郎兼史院官,官至中奉大夫,間居一紀卒,年七十有七,累贈至華文閣學士、通奉大夫。子沆。



 沆字叔朝,以恩補將仕郎。端平三年,銓試第一,授黃
 巖縣主簿,監西京中嶽廟者二,總領湖廣、江西、京西財賦所準備差遣。改知於潛縣,重囚逸,聞沆至,自歸。淮西總領闢充主管文字。



 通判婺州,朱君章訟爭田四十有二年,吳王府爭墓二十有九年,沆皆決之。特差充提領兩浙轉運鹽事使司主管文字,又差充行在點檢贍軍激賞酒庫,歷四轄、六院之文思官告,書擬尚左右郎官事。



 賈似道議行公田,彗星見,沆請罷公田還民。及理宗崩,似道矯詔廢十七界會子,行關子,沆力言非便。似道大
 怒,調將作監簿,急令言者論寢。久之,與雲臺觀,起知興國軍,未赴,論仍雲臺觀。起知全州,未赴,與仙都觀。德祐元年,三學伏闕上書訟沆屈,召赴行在,沆不復出,卒,年八十有一。



 論曰:杜範在下僚,已有公輔之望,及入相未久而沒。楊簡之學,非世儒所能及,施諸有政,使人百世而不能忘,然雖享高年,不究於用,豈不重可惜也哉。張虙子諒易直,呂午風採凜然,皆有裨於世道者矣。



\end{pinyinscope}