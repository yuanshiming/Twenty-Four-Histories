\article{列傳第四十}

\begin{pinyinscope}

 呂端畢士安曾孫仲衍仲游寇準



 呂端,字易直,幽州安次人。父琦,晉兵部侍郎。端少敏悟好學,以蔭補千牛備身。歷國子主簿、太僕寺丞、秘書郎、直弘文館,換著作佐郎、直史館。太祖即位,遷太常丞、知
 浚儀縣,同判定州。開寶中,西上閣門使郝崇信使契丹,以端假太常少卿為副。八年,知洪州,未上,改司門員外郎、知成都府,賜金紫。為政清簡,遠人便之。



 會秦王廷美尹京,召拜考功員外郎,充開封府判官。太宗征河東,廷美將有居留之命,端白廷美曰:「主上櫛風沐雨,以申吊伐,王地處親賢,當表率扈從。今主留務,非所宜也。」廷美由是懇請從行。尋坐王府親吏請托執事者違詔市竹木,貶商州司戶參軍。移汝州,復為太常丞、判寺事。出知
 蔡州,以善政,吏民列奏借留。改祠部員外郎、知開封縣,遷考功員外郎兼侍御史知雜事。使高麗,暴風折檣,舟人怖恐,端讀書若在齋閣時。遷戶部郎中、判太常寺兼禮院,選為大理少卿,俄拜右諫議大夫。



 許王元僖尹開封,又為判官。王薨,有發其陰事者,坐裨贊無狀,遣御史武元穎、內侍王繼恩就鞫於府。端方決事,徐起候之,二使曰:「有詔推君。」端神色自若,顧從者曰:「取帽來。」二使曰:「何遽至此?」端曰:「天子有制問,即罪人矣,安可在堂上對
 制使?」即下堂,隨問而答。左遷衛尉少卿。會置考課院,群官有負譴置散秩者,引對,皆泣涕,以饑寒為請。至端,即奏曰:「臣前佐秦邸,以不檢府吏,謫掾商州,陛下復擢官籍辱用。今許王暴薨,臣輔佐無狀,陛下又不重譴,俾亞少列,臣罪大而幸深矣!今有司進退善否,茍得穎州副使,臣之願也。」太宗曰:「朕自知卿。」無何,復舊官,為樞密直學士,逾月,拜參知政事。



 時趙普在中書,嘗曰:「吾觀呂公奏事,得嘉賞未嘗喜,遇抑挫未嘗懼,亦不形於言,真臺
 輔之器也。」歲餘,左諫議大夫寇準亦拜參知政事。端請居準下,太宗即以端為左諫議大夫,立準上。每獨召便殿,語必移晷。擢拜戶部侍郎、平章事。



 時呂蒙正為相,太宗欲相端,或曰:「端為人胡塗。」太宗曰:「端小事胡塗,大事不胡塗。」決意相之。會曲宴後苑,太宗作《釣魚詩》,有云:「欲餌金鉤深未達,磻溪須問釣魚人。」意以屬端。後數日,罷蒙正而相端焉。初,端兄餘慶,建隆中以藩府舊僚參預大政,端復居相位,時論榮之。端歷官僅四十年,至是驟
 被獎擢,太宗猶恨任用之晚。端為相持重,識大體,以清簡為務。慮與寇準同列,先居相位,恐準不平,乃請參知政事與宰相分日押班知印,同升政事堂,太宗從之。時同列奏對多有異議,惟端罕所建明。一日,內出手札戒諭:「自今中書事必經呂端詳酌,乃得聞奏。」端愈謙讓不自當。



 初,李繼遷擾西鄙,保安軍奏獲其母。至是,太宗欲誅之,以寇準居樞密副使,獨召與謀。準退,過相幕,端疑謀大事,邀謂準曰:「上戒君勿言於端乎?」準曰:「否。」端曰:「邊
 鄙常事,端不必與知,若軍國大計,端備位宰相,不可不知也。」準遂告其故,端曰:「何以處之?」準曰:「欲斬於保安軍北門外,以戒兇逆。」端曰:「必若此,非計之得也,願少緩之,端將覆奏。」入曰:「昔項羽得太公,欲烹之,高祖曰:『願分我一杯羹。』夫舉大事不顧其親,況繼遷悖逆之人乎?陛下今日殺之,明日繼遷可擒乎?若其不然,徒結怨仇,愈堅其叛心爾。」太宗曰:「然則何如?」端曰:「以臣之愚,宜置於延州,使善養視之,以招來繼遷。雖不能即降,終可以系其
 心,而母死生之命在我矣。」太宗撫髀稱善曰:「微卿,幾誤我事。」即用其策。其母後病死延州,繼遷尋亦死,繼遷子竟納款請命,端之力也。進門下侍郎兼兵部尚書。



 太宗不豫,真宗為皇太子,端日與太子問起居。及疾大漸,內侍王繼恩忌太子英明,陰與參知政事李昌齡、殿前都指揮使李繼勛、知制誥胡旦謀立故楚王元佐。太宗崩,李皇后命繼恩召端,端知有變,鎖繼恩於閣內,使人守之而入。皇后曰:「宮車已晏駕,立嗣以長,順也,今將如何?」
 端曰:「先帝立太子正為今日,今始棄天下,豈可遽違命有異議邪?」乃奉太子至福寧庭中。真宗既立,垂簾引見群臣,端平立殿下不拜,請卷簾,升殿審視,然後降階,率群臣拜呼萬歲。以繼勛為使相,赴陳州。貶昌齡忠武軍司馬,繼恩右監門衛將軍、均州安置,旦除名流潯州,籍其家貲。



 真宗每見輔臣入對,惟於端肅然拱揖,不以名呼。又以端軀體洪大,宮庭階戺稍峻,特令梓人為納陛。嘗召對便殿,訪軍國大事經久之制,端陳當世急務,皆
 有條理,真宗嘉納。加右僕射,監修國史。明年夏,被疾,詔免常參,就中書視事。上疏求解,不許。十月,以太子太保罷。在告三百日,有司言當罷奉,詔賜如故。車駕臨問,端不能興,撫慰甚至。卒,年六十六,贈司空,謚正惠,追封妻李氏涇國夫人,以其子藩為太子中舍,荀大理評事,蔚千牛備身,藹殿中省進馬。



 端姿儀瑰秀,有器量,寬厚多恕,善談謔,意豁如也。雖屢經擯退,未嘗以得喪介懷。善與人交,輕財好施,未嘗問家事。李惟清自知樞密改御
 史中丞,意端抑己,及端免朝謁,乃彈奏常參官疾告逾年受奉者,又構人訟堂吏過失,欲以中端。端曰:「吾直道而行,無所愧畏,風波之言不足慮也。」



 端祖兗,嘗事滄州節度劉守文為判官。守文之亂,兗舉族被害。時父琦方幼,同郡趙玉冒鋒刃紿監者曰:「此予之弟,非呂氏子也。」遂得免。玉子文度為耀帥,文度孫紹宗十餘歲,端視如己子,表薦賜出身。故相馮道,鄉里世舊,道子正之病廢,端分奉給之。端兩使絕域,其國嘆重之,後有使往者,每
 問端為宰相否,其名顯如此。



 景德二年,真宗聞端後嗣不振,又錄蔚為奉禮郎。藩後病足,不任朝謁,請告累年,有司奏罷其奉,真宗特令復舊官,分司西京,給奉家居養病。端不蓄貲產,藩兄弟貧匱,又迫婚嫁,因質其居第。真宗時,出內府錢五百萬贖還之。又別賜金帛,俾償宿負,遣使檢校家事。藩、荀皆至國子博士,蔚至太子中舍。



 畢士安,字仁叟,代州雲中人。曾祖宗昱,本縣令。祖球,本州別駕。父乂林,累闢使府,終觀城令,因家焉。士安少好
 學,事繼母祝氏以孝聞。祝氏曰:「學必求良師友。」乃與如宋,又如鄭,得楊璞、韓丕、劉錫為友,因為鄭人。



 乾德四年,舉進士。邠帥楊廷璋闢幕府,掌書奏。開寶四年,歷濟州團練推官,專掌筦榷,歲課增羨。改兗州觀察推官。太平興國初,為大理寺丞,領三門發運事。吳越錢俶納土,選知臺州,言:「錢氏上圖籍,有司皆張侈賦數,今湖海新民始得天子命吏,宜有安輯,願一用舊籍。」詔從之。明年,遷左贊善大夫,徙饒州,改殿中丞。召還,為監察御史。復出
 知幹州,以母老願降任就養,改監汝州稻田務。



 雍熙二年,諸王出閣,慎擇僚屬。以虞部郎中王龜從兼陳王府記室參軍,水部員外郎王素兼韓王府記室參軍,秘書丞張茂直兼益王府記室參軍,士安遷左拾遺兼冀王府記室參軍。太宗召謂曰:「諸子生長宮庭,未閑外事,年漸成人,必資良士贊導,使日聞忠孝之道,卿等勉之。」賜襲衣、銀帶、鞍勒馬。



 士安本名士元,以「元」犯王諱,遂改焉。遷考功員外郎。端拱中,詔王府僚屬各獻所著文,太宗
 閱視累日,問近臣曰:「其才已見矣,其行孰優?」或以士安對。上曰:「正協朕意。」俄以本官知制誥,王請對願留府邸,不許。淳化二年,召入翰林為學士。大臣以張洎薦,太宗曰:「洎視畢士安詞藝踐歷固不減,但履行遠在下爾。」士安以父名乂林抗章引避,朝議謂二名不偏諱,不聽。



 三年,與蘇易簡同知貢舉,加主客郎中。以疾請外,改右諫議大夫、知穎州。真宗以壽王尹開封府,召為判官。及為皇太子,以兼右庶子遷給事中。登位,命權知開封府事,
 拜工部侍郎、樞密直學士。時近臣有怙勢強取民間定婚女,其家訴於府,士安因對奏,還之。宮府常從為廷職者,每授任於外,必令士安戒勖。



 咸平初,辭府職,拜禮部侍郎,復為翰林學士。詔選官校勘《三國志》、晉、唐書。或有言兩晉事多鄙惡不可流行者。真宗以語宰相,士安曰:「惡以戒世,善以勸後。善惡之事,《春秋》備載。」真宗然之,遂命刊刻。士安以目疾求解,改兵部侍郎,出知潞州,特加月給之數。入為翰林侍讀學士。景德初,兼秘書監。契丹
 謀入境,士安首疏五事應詔,陳選將、餉兵、理財之策,真宗嘉納。



 李沆卒,進士安吏部侍郎、參知政事。入謝,真宗曰:「未也,行且相卿。」士安頓首。真宗曰:「朕倚卿以輔相,豈特今日。然時方多事,求與卿同進者,其誰可?」對曰:「宰相者,必有其器,乃可居其位,臣駑朽,實不足以勝任。寇準兼資忠義,善斷大事,此宰相才也。」真宗曰:「聞其好剛使氣。」又對曰:「準方正慷慨有大節,忘身徇國,秉道疾邪,此其素所蓄積,朝臣罕出其右者,第不為流俗所喜。今天
 下之民雖蒙休德,涵養安佚,而西北跳梁為邊境患,若準者正所宜用也。」真宗曰:「然,當藉卿宿德鎮之。」未閱月,以本官與準同拜平章事。士安兼監修國史,居準上。



 準為相,守正嫉惡,小人日思所以傾之。有布衣申宗古告準交通安王元傑,準皇恐,莫知所自明。士安力辯其誣,下宗古吏,具得奸罔,斬之,準乃安。



 景德元年九月,契丹統軍撻覽引兵分掠威虜、順安、北平,侵保州,攻定武,數為諸軍所卻,益東駐陽城澱,遂攻高陽,不得逞,轉窺貝、
 冀、天雄,兵號二十萬。真宗坐便殿,問策安出。士安與寇準條所以御備狀,又合議請真宗幸澶淵。士安言澶淵之行,當在仲冬;準謂當亟往,不可緩。卒用士安議。



 初,咸平六年,雲州觀察使王繼忠戰陷契丹。至是,為契丹奏請議和。大臣莫敢如何,獨士安以為可信,力贊真宗當羈縻不絕,漸許其成。真宗謂敵悍如此,恐不可保。士安曰:「臣嘗得契丹降人,言其雖深入,屢挫不甚得志,陰欲引去而恥無名,且彼寧不畏人乘虛覆其巢穴,此請
 殆不妄。繼忠之奏,臣請任之。」真宗喜,手詔繼忠,許其請和。



 時已詔巡幸,而議者猶哄哄,二三大臣有進金陵及成都圖者。士安亟同準請對,力陳其不可,惟堅定前計。真宗嚴兵將行,太白晝見,流星出上臺北貫斗魁。或言兵未宜北,或言大臣應之。士安適臥疾,移書準曰:「屢請舁疾從行,手詔不許,今大計已定,唯君勉之。士安得以身當星變而就國事,心所願也。」已而少間,追至澶淵,見於行在。時已聚兵數十萬,契丹大震,猶乘眾掠德清。至
 澶北鄙,為伏弩發射,撻覽死,眾潰遁去。



 會曹利用自契丹使還,具得要領,又與其使者姚東之俱來,講和之議遂定。歲遺契丹銀絹三十萬,朝論皆以為過。士安曰:「不如此,契丹所顧不重,和事恐不能久。」及罷兵,從還,乃按邊要選良守將易置之:雄州以李允則,定州馬知節,鎮州孫全照,保州楊延昭,它所擇用各得其任。令塞上得境外牛馬類者悉還之,通互市,除鐵禁,招流亡,廣儲蓄。未幾,夏州趙德明亦款塞內附。二方既定,中外略安。量
 時制法,次第施行。復置賢良方正、直言極諫等科,以廣取士。



 二年,章七八上,以病求免,優詔不允。遣使敦諭,不得已,復起視事。十月晨朝,至崇政殿廬,疾暴作,真宗步出臨視,已不能言。詔內侍竇神寶以肩輿送歸第,卒,年六十八。車駕臨哭,廢朝五日,贈太傅、中書令,謚文簡。以皇城使衛紹欽治葬,有司給鹵簿。錄其子世長為太子中舍,慶長為大理寺丞,孫從古為將作監主簿。



 士安端方沉雅,有清識,□溫藉,美風採,善談吐,所至以嚴正稱。年
 耆目眊,讀書不輟,手自讎校,或親繕寫。又精意詞翰,有文集三十卷。嘗謂人曰:「僕仕宦無赫赫之譽,但力自規檢,庶幾寡過爾。」凡交游無黨援,唯王祐、呂端見引重,王旦、寇準、楊億相友善,王禹偁、陳彭年皆門人也。禹偁,濟州人。幼時以事至士安官舍,士安識其非常童,留之,教以學,舉業日顯。後遂登科進用,更在士安前。及士安知制誥,其命乃禹偁詞也。



 士安沒後,真宗謂寇準等曰:「畢士安,善人也,事朕南府、東宮,以至輔相。飭躬慎行,有古
 人之風,遽此淪沒,深可悼惜。」及王旦為相,面奏:「陛下前稱畢士安清慎如古人,在位聞之感嘆。仕至輔相,而四方無田園居第,沒未終喪,家用已屈,真不負陛下所知。然使其家假貸為生,宜有以周之者,竊謂當出上恩,非臣敢為私惠。」真宗感嘆,賜白金五千兩。



 子世長至衛尉卿,慶長至大府卿。孫從善光祿少卿,從古駕部郎中,從厚、從誨檢校水部員外郎,從簡博羅令,從道殿中丞,從範山南西道節度推官,從益太常寺太祝,從周朝散郎、
 知洋州。曾孫仲達、仲偃仕至郡守,仲衍、仲游、仲愈。



 仲衍字夷仲,以蔭為陽翟主簿。張忭,縣人也,方鎮許,請於朝,欲興鄉校。既具材計工,又聽民自以其力輸助。邑子馬宏以口舌橫閭里,謾謂諸豪曰:「張公興學,而縣令乃因以取諸民,由十百而至千萬未已也,君將不堪。誠捐百金予我,我能止役。」豪信其能,予百金。宏即詣府宣言:「縣吏盡私為學之費,又將賦於民。」忭果疑焉,敕縣且止,又揭其事於道。令欲上疏辯,仲衍曰:「亡益也,不如取
 宏治之,不辯自直矣。」會攝縣事,即逮捕驗治,五日得其奸,言於忭,流宏鄧州,一縣相賀。給事中張問居里中,謂仲衍曰:「諺云『鋤一惡,長十善』,君之謂也。」



 舉進士中第,調沉丘令。歐陽修、呂公著薦之,入司農為主簿,升丞。吳充引為中書檢正。奉使契丹,宴射連破的,眾驚異之。且偉其姿容,密使人取其衣為度,制服以賜。時預其元會,盡能記其朝儀節奏,圖畫歸獻。後錢勰出使,契丹主猶問:「畢少卿何官?今安在?」



 王珪與充不相能,以仲衍為充所
 用,數求罪過欲傷之,卒無可乘,但留滯不遷。經四年,乃以秘閣校理同知太常禮院,為官制局檢討官,制文字千萬計,區別分類,損益刪補,皆曲盡其當。凡從中問其事,必須仲衍然後報,他人不知也。撰《中書備對》三十卷,士大夫家爭傳其書。



 高麗使入貢,詔館之。上元夕,與使者宴東闕下,作詩誦聖德,神宗次韻賜焉,當時以為寵。官制行,帝自擢起居郎,王珪留除命,謂為太峻,爭於前。帝連稱曰:「是當得爾。」未幾,暴得疾,一夕卒,年四十三。帝
 遣中使唁其家,賻錢五十萬。



 仲游字公叔,與仲衍同登第,調壽丘柘城主簿、羅山令、環慶轉運司干辦公事。從高遵裕西征,運期迫遽,陜西八十縣饋挽之夫三十萬,一旦悉集,轉運使范純粹、李察度受其賦而給之食,必曠日乃可。會僚屬議,皆不知所為,以諉仲游。仲游集諸縣吏,令先效金帛緡錢之最,戒勿啟扃鐍,共簿其名數以為質,預飭其斛量數千,洞撤倉庾墻壁,使贏糧者至其所,人自奭斗概,輸其半而以
 半自給,不終朝霍然而散。翌日,大軍遂行。純粹、察嘆且謝曰:「非君幾敗吾事。」



 元祐初,為軍器衛尉丞。召試學士院,同策問者九人,乃黃庭堅、張耒、晁補之輩。蘇軾異其文,擢為第一。加集賢校理、開封府推官,出提點河東路刑獄。韓縝以故相在太原,按視如列郡,縝奴告有卒剽其衣於公堂之側,縝怒,將置卒於理。仲游曰:「奴衣服鮮薄而敢掠之於帥牙,非人情也。」取以付獄治,卒得免。太原銅器名天下,獨不市一物;懼人以為矯也,且行,買二
 茶匕而去。縝曰:「如公叔可謂真清矣。」



 召拜職方、司勛二員外郎,改秘閣校理、知耀州。是歲大旱,仲游先民之未饑,揭喻境內曰:「郡振施與平糴若干萬碩。」實虛張其數。富室知有備,亦相勸發廩。凡民就食者十七萬九千口,無一人去其鄉。



 徽宗時,歷知鄭、鄆二州,京東、淮南轉運副使。入為吏部郎中,言孔子廟自顏回以降,皆爵命於朝,冠冕居正,而子鯉、孫伋乃野服幅巾以祭,為不稱。詔皆追侯之。



 仲游早受知於司馬光、呂公著,不及用。范純
 仁尤知之,當國時,又適居母喪,故未嘗得尺寸進。然亦墮黨籍,坎NI散秩而終,年七十五。



 仲游為文切於事理而有根柢,不為浮誇詭誕、戲弄不莊之語。蘇軾在館閣,頗以言語文章規切時政。仲游憂其及禍,貽書戒之曰:



 孟軻不得已而後辯,孔子欲無言,古人所以精謀極慮,固功業而養壽命者,未嘗不出乎此。君自立朝以來,禍福利害系身者未嘗言,顧直惜其言爾。夫言語之累,不特出口者為言,其形於詩歌、贊於賦頌、托於碑銘、著於
 序記者,亦語言也。今知畏於口而未畏於文,是其所是則見是者喜,非其所非則蒙非者怨;喜者未能濟君之謀,而怨者或已敗君之事矣。天下論君之文,如孫臏之用兵,扁鵲之醫疾,固所指名者矣。雖無是非之言,猶有是非之疑,又況其有耶?官非諫臣,職非御史,而非是人所未是,危身觸諱以游其間,殆猶抱石而救溺也。



 司馬光為政,反王安石所為,仲游予之書曰:



 昔安石以興作之說動先帝,而患財之不足也,故凡政之可以得民財
 者無不用。蓋散青苗、置市易、斂役錢、變鹽法者,事也;而欲興作、患不足者,情也。茍未能杜其興作之情,而徒欲禁其散斂變置之事,是以百說而百不行。今遂廢青苗,罷市易,蠲役錢,去鹽法,凡號為利而傷民者,一掃而更之,則向來用事於新法者必不喜矣。不喜之人,必不但曰『青苗不可廢,市易不可罷,役錢不可蠲,鹽法不可去』,必操不足之情,言不足之事,以動上意,雖致石人而使聽之,猶將動也。如是,則廢者可復散,罷者可復置,蠲者
 可復斂,去者可復存矣。則不足之情,可不預治哉?



 為今之策,當大舉天下之計,深明出入之數,以諸路所積之錢粟一歸地官,使經費可支二十年之用。數年之間,又將十倍於今日。使天子曉然知天下之餘於財也,則不足之論不得陳於前,然後所論新法者,始可永罷而不可行矣。



 昔安石之居位也,中外莫非其人,故其法能行。今欲救前日之敝,而左右侍從、職司、使者,十有七八皆安石之徒,雖起二三舊臣,用六七君子,然累百之中存基
 十數,烏在其勢之可為也。勢未可為而欲為之,則青苗雖廢將復散,況未廢乎?市易雖罷且復置,況未罷乎?役錢、鹽法亦莫不然。以此救前日之敝,如人久病而少間,其父子兄弟喜見顏色而未敢賀者,以其病之猶在也。



 光、軾得書聳然,竟如其慮。



 仲愈歷國子監丞、諸王府侍講、知鳳翔府,坐兄仲游陷黨籍,例廢黜。徽宗曰:「畢仲衍被遇先帝,可除罪籍。」以仲愈為都官郎中,擢秘書少監,卒。



 寇準,字平仲,華州下邽人也。父相,晉開運中,應闢為魏王府記室參軍。準少英邁,通《春秋》三傳。年十九,舉進士。太宗取人,多臨軒顧問,年少者往往罷去。或教準增年,答曰:「準方進取,可欺君邪?」後中第,授大理評事,知歸州巴東、大名府成安縣。每期會賦役,未嘗輒出符移,唯具鄉里姓名揭縣門,百姓莫敢後期。累遷殿中丞、通判鄆州。召試學士院,授右正言、直史館,為三司度支推官,轉鹽鐵判官。會詔百官言事,而準極陳利害,帝益器重之。
 擢尚書虞部郎中、樞密院直學士,判吏部東銓。嘗奏事殿中,語不合,帝怒起,準輒引帝衣,令帝復坐,事決乃退。上由是嘉之,曰:「朕得寇準,猶文皇之得魏徵也。」



 淳化二年春,大旱,太宗延近臣問時政得失,眾以天數對。準對曰:「《洪範》天人之際,應若影響,大旱之證,蓋刑有所不平也。」太宗怒,起入禁中。頃之,召準問所以不平狀,準曰:「願召二府至,臣即言之。」有詔召二府入,準乃言曰:「頃者祖吉、王淮皆侮法受賕,吉贓少乃伏誅;淮以參政沔之弟,
 盜主守財至千萬,止杖,仍復其官,非不平而何?」太宗以問沔,沔頓首謝,於是切責沔,而知淮為可用矣。即拜準左諫議大夫、樞密副使,改同知院事。



 準與知院張遜數爭事上前。他日,與溫仲舒偕行,道逢狂人迎馬呼萬歲,判左金吾王賓與遜雅相善,遜嗾上其事。準引仲舒為證,遜令賓獨奏,其辭頗厲,且互斥其短。帝怒,謫遜,準亦罷知青州。



 帝顧準厚,既行,念之,常不樂。語左右曰:「寇準在青州樂乎?」對曰:「準得善藩,當不苦也」數日,輒復問。左
 右揣帝意且復召用準,因對曰:「陛下思準不少忘,聞準日縱酒,未知亦念陛下乎?」帝默然。明年,召拜參知政事。



 自唐末,蕃戶有居渭南者。溫仲舒知秦州,驅之渭北,立堡柵以限其往來。太宗覽奏不懌,曰:「古羌戎尚雜處伊、洛,彼蕃夷易動難安,一有調發,將重困吾關中矣。」準言:「唐宋璟不賞邊功,卒致開元太平。疆埸之臣邀功以稔禍,深可戒也。」帝因命準使渭北,安撫族帳,而徙仲舒鳳翔。



 至道元年,加給事中。時太宗在位久,馮拯等上疏乞
 立儲貳,帝怒,斥之嶺南,中外無敢言者。準初自青州召還,入見,帝足創甚,自褰衣以示準,且曰:「卿來何緩耶?」準對曰:「臣非召不得至京師。」帝曰:「朕諸子孰可以付神器者?」準曰:「陛下為天下擇君,謀及婦人、中官,不可也;謀及近臣,不可也;唯陛下擇所以副天下望者。」帝俯首久之,屏左右曰:「襄王可乎?」準曰:「知子莫若父,聖慮既以為可,願即決定。」帝遂以襄王為開封尹,改封壽王,於是立為皇太子。廟見還,京師之人擁道喜躍,曰:「少年天子也。」帝
 聞之不懌,召準謂曰:「人心遽屬太子,欲置我何地?」準再拜賀曰:「此社稷之福也。」帝入語後嬪,宮中皆前賀。復出,延準飲,極醉而罷。



 二年,祠南郊,中外官皆進秩。準素所喜者多得臺省清要官,所惡不及知者退序進之。彭惟節位素居馮拯下,拯轉虞部員外郎,惟節轉屯田員外郎,章奏列銜,惟節猶處其下。準怒,堂帖戒拯毋亂朝制。拯憤極,陳準擅權,又條上嶺南官吏除拜不平數事。廣東轉運使康戩亦言:呂端、張洎、李昌齡皆準所引,端德
 之,洎能曲奉準,而昌齡畏心耎,不敢與準抗,故得以任胸臆,亂經制。太宗怒,準適祀太廟攝事,召責端等。端曰:「準性剛自任,臣等不欲數爭,慮傷國體。」因再拜請罪。及準入對,帝語及馮拯事,自辯。帝曰:「若廷辯,失執政體。」準猶力爭不已,又持中書簿論曲直於帝前,帝益不悅,因嘆曰:「鼠雀尚知人意,況人乎?」遂罷準知鄧州。



 真宗即位,遷尚書工部侍郎。咸平初,徙河陽,改同州。三年,朝京師,行次閿鄉,又徙鳳翔府。帝幸大名,詔赴行在所,遷刑部,權
 知開封府。六年,遷兵部,為三司使。時合鹽鐵、度支、戶部為一使,真宗命準裁定,遂以六判官分掌之,繁簡始適中。



 帝久欲相準,患其剛直難獨任。景德元年,以畢士安參知政事,逾月,並命同中書門下平章事,準以集賢殿大學士位士安下。是時,契丹內寇,縱游騎掠深、祁間,小不利輒引去,徜徉無鬥意。準曰:「是狃我也。請練師命將,簡驍銳據要害以備之。」是冬,契丹果大入。急書一夕凡五至,準不發,飲笑自如。明日,同列以聞,帝大駭,以問準。
 準曰:「陛下欲了此,不過五日爾。」因請帝幸澶州。同列懼,欲退,準止之,令候駕起。帝難之,欲還內,準曰:「陛下入則臣不得見,大事去矣,請毋還而行。」帝乃議親征,召群臣問方略。



 既而契丹圍瀛州,直犯貝、魏,中外震駭。參知政事王欽若,江南人也,請幸金陵。陳堯叟,蜀人也,請幸成都。帝問準,準心知二人謀,乃陽若不知,曰:「誰為陛下畫此策者,罪可誅也。今陛下神武,將臣協和,若大駕親征,賊自當遁去。不然,出奇以撓其謀,堅守以老其師,勞佚
 之勢,我得勝算矣。奈何棄廟社欲幸楚、蜀遠地,所在人心崩潰,賊乘勢深入,天下可復保邪?」遂請帝幸澶州。



 及至南城,契丹兵方盛,眾請駐蹕以覘軍勢。準固請曰:「陛下不過河,則人心益危,敵氣未懾,非所以取威決勝也。且王超領勁兵屯中山以扼其亢,李繼隆、石保吉分大陣以扼其左右肘,四方征鎮赴援者日至,何疑而不進?」眾議畢懼,準力爭之,不決。出遇高瓊於屏間,謂曰:「太尉受國恩,今日有以報乎?」對曰:「瓊武人,願效死。」準復入對,
 瓊隨立庭下,準厲聲曰:「陛下不以臣言為然,盍試問瓊等?」瓊即仰奏曰:「寇準言是。」準曰:「機不可失,宜趣駕。」瓊即麾衛士進輦,帝遂渡河,御北城門樓,遠近望見御蓋,踴躍歡呼,聲聞數十里。契丹相視驚愕,不能成列。



 帝盡以軍事委準,準承制專決,號令明肅,士卒喜悅。敵數千騎乘勝薄城下,詔士卒迎擊,斬獲大半,乃引去。上還行宮,留準居城上,徐使人視準何為。準方與楊億飲博,歌謔歡呼。帝喜曰:「準如此,吾復何憂?」相持十餘日,其統軍撻
 覽出督戰。時威虎軍頭張瑰守床子弩,弩撼機發,矢中撻覽額,撻覽死,乃密奉書請盟。準不從,而使者來請益堅,帝將許之。準欲邀使稱臣,且獻幽州地。帝厭兵,欲羈縻不絕而已。有譖準幸兵以自取重者,準不得已,許之。帝遣曹利用如軍中議歲幣,曰:「百萬以下皆可許也。」準召利用至幄,語曰:「雖有敕,汝所許毋過三十萬,過三十萬,吾斬汝矣。」利用至軍,果以三十萬成約而還。河北罷兵,準之力也。



 準在相位,用人不以次,同列頗不悅。它日,
 又除官,同列因吏持例簿以進。準曰:「宰相所以進賢退不肖也,若用例,一吏職爾。」二年,加中書侍郎兼工部尚書。準頗自矜澶淵之功,雖帝亦以此待準甚厚。王欽若深嫉之。一日會朝,準先退,帝目送之,欽若因進曰:「陛下敬寇準,為其有社稷功邪?」帝曰:「然。」欽若曰:「澶淵之役,陛下不以為恥,而謂準有社稷功,何也?」帝愕然曰:「何故?」欽若曰:「城下之盟,《春秋》恥之。澶淵之舉,是城下之盟也。以萬乘之貴而為城下之盟,其何恥如之!」帝愀然為之不
 悅。欽若曰:「陛下聞博乎?博者輸錢欲盡,乃罄所有出之,謂之孤注。陛下,寇準之孤注也,斯亦危矣。」



 由是帝顧準浸衰。明年,罷為刑部尚書、知陜州,遂用王旦為相。帝謂旦曰:「寇準多許人官,以為己恩。俟行,當深戒之。」從封泰山,遷戶部尚書、知天雄軍。祀汾陰,命提舉貝、德、博、洺、濱、棣巡檢捉賊公事,遷兵部尚書,入判都省。幸亳州,權東京留守,為樞密院使、同平章事。



 林特為三司使,以河北歲輸絹闕,督之甚急。而準素惡特,頗助轉運使李士衡
 而沮特,且言在魏時嘗進河北絹五萬而三司不納,以至闕供,請劾主吏以下。然京師歲費絹百萬,準所助才五萬。帝不悅,謂王旦曰:「準剛忿如昔。」旦曰:「準好人懷惠,又欲人畏威,皆大臣所避。而準乃為己任,此其短也。」未幾,罷為武勝軍節度使、同平章事、判河南府,徙永興軍。



 天禧元年,改山南東道節度使,時巡檢朱能挾內侍都知周懷政詐為天書,上以問王旦。旦曰:「始不信天書者準也。今天書降,須令準上之。」準從上其書,中外皆以為
 非。遂拜中書侍郎兼吏部尚書、同平章事、景靈宮使。



 三年,祀南郊,進尚書右僕射、集賢殿大學士。時真宗得風疾,劉太后預政於內,準請間曰:「皇太子人所屬望,願陛下思宗廟之重,傳以神器,擇方正大臣為羽翼。丁謂、錢惟演,佞人也,不可以輔少主。」帝然之。準密令翰林學士楊億草表,請太子監國,且欲援億輔政。已而謀洩,罷為太子太傅,封萊國公。時懷政反側不自安,且憂得罪,乃謀殺大臣,請罷皇后預政,奉帝為太上皇,而傳位太子,
 復相準。客省使楊崇勛等以告丁謂,謂微服夜乘犢車詣曹利用計事,明日以聞。乃誅懷政,降準為太常卿、知相州,徙安州,貶道州司馬。帝初不知也,他日,問左右曰:「吾目中久不見寇準,何也?」左右莫敢對。帝崩時亦信惟準與李迪可托,其見重如此。



 乾興元年,再貶雷州司戶參軍。初,丁謂出準門至參政,事準甚謹。嘗會食中書,羹污準須,謂起,徐拂之。準笑曰:「參政國之大臣,乃為官長拂須邪?」謂甚愧之,由是傾構日深。及準貶未幾,謂亦南
 竄,道雷州,準遣人以一蒸羊逆境上。謂欲見準,準拒絕之。聞家僮謀欲報仇者,乃杜門使縱博,毋得出,伺謂行遠,乃罷。



 天聖元年,徙衡州司馬。初,太宗嘗得通天犀,命工為二帶,一以賜準。及是,準遣人取自洛中,既至數日,沐浴,具朝服束帶,北面再拜,呼左右趣設臥具,就榻而卒。



 初,張詠在成都,聞準入相,謂其僚屬曰:「寇公奇材,惜學術不足爾。」及準出陜,詠適自成都罷還,準嚴供帳,大為具待。詠將去,準送之郊,問曰:「何以教準?」詠徐曰:「《霍光
 傳》不可不讀也。」準莫諭其意,歸取其傳讀之,至「不學無術」,笑曰:「此張公謂我矣。」



 準少年富貴,性豪侈,喜劇飲,每宴賓客,多闔扉脫驂。家未嘗爇油燈,雖庖匽所在,必然炬燭。



 在雷州逾年。既卒,衡州之命乃至,遂歸葬西京。道出荊南公安,縣人皆設祭哭於路,折竹植地,掛紙錢,逾月視之,枯竹盡生筍。眾因為立廟,歲時享之。無子,以從子隨為嗣。準歿後十一年,復太子太傅,贈中書令、萊國公,後又賜謚曰忠愍。皇祐四年,詔翰林學士孫抃撰神
 道碑,帝為篆其首曰「旌忠」。



 論曰:呂端諫秦王居留,表表已見大器,與寇準同相而常讓之,留李繼遷之母不誅。真宗之立,閉王繼恩於室,以折李後異謀,而定大計;既立,猶請去簾,升殿審視,然後下拜,太宗謂之「大事不胡塗」者,知臣莫過君矣。宰相不和,不足以定大計。畢士安薦寇準,又為之辨誣。契丹大舉而入,合辭以勸真宗,遂幸澶淵,終卻鉅敵。及議歲幣,因請重賄,要其久盟;由是西夏失牽制之謀,隨亦內
 附。景德、咸平以來,天下乂安,二相協和之所致也。準於太宗朝論建太子,謂神器不可謀及婦人、謀及中官、謀及近臣。此三言者,可為萬世龜鑒。澶淵之幸,力沮眾議,竟成雋功,古所謂大臣者,於斯見之。然挽衣留諫,面詆同列,雖有直言之風,而少包荒之量。定策禁中,不慎所與,致啟懷政邪謀,坐竄南裔。勛業如是而不令厥終,所謂「臣不密則失身」,豈不信哉!



\end{pinyinscope}