\article{本紀第三十六}

\begin{pinyinscope}

 光宗



 光宗循道憲仁明功茂德溫文順武聖哲慈孝皇帝,諱惇,孝宗第三子也。母曰成穆皇后郭氏。紹興十七年九月乙丑,生於藩邸。二十年賜今名,授右監門衛率府副
 率,轉榮州刺史。孝宗即位,拜鎮洮軍節度使、開府儀同三司,封恭王。及莊文太子薨,孝宗以帝英武類己,欲立為太子,而以其非次,遲之。乾道六年七月,太史奏:木、火合宿,主冊太子,當有赦。是時,虞允文相,因請蚤建儲貳。孝宗曰:「朕久有此意,事亦素定。但恐儲位既正,人性易驕,即自縱逸,不勤於學,浸有失德。朕所以未建者,更欲其練歷庶務,通知古今,庶無後悔爾。」七年正月丙子朔,孝宗上兩宮尊號冊、寶,禮成。丞相允文復以請,孝宗曰:「
 朕既立太子,即令親王出鎮外藩,卿宜討論前代典禮。」允文尋以聞。二月癸丑,乃立帝為皇太子,慶王愷為雄武、保寧軍節度使、判寧國府,進封魏王。三月丁酉,受皇太子冊。四月甲子,命判臨安府,尋領尹事。帝之為恭王,與講官商較前代,時出意表,講官自以為不及。逮尹臨安,究心民政,周知情偽。孝宗數稱之,且語丞相趙雄曰:「太子資質甚美,每遣人來問安,朕必戒以留意問學。」淳熙十四年十月乙亥,高宗崩。十一月己亥,百官大祥畢,
 孝宗手詔:「皇太子可令參決庶務,以內東門司為議事堂。」十五年二月戊戌,帝始赴議事堂,自是,間日與輔臣公裳系鞋相見,內外除擢,自館職、部刺史以上乃以聞。九月乙巳,又詔:「每遇朝殿,令皇太子侍立。」十一月,丞相周必大乞去,孝宗諭曰:「朕比年病倦,欲傳位太子,卿須少留。」會陳康伯家以紹興傳位御札來上,十二月壬申,孝宗遣中使密持賜必大,因令討論典禮,既又密以禪意諭參知政事留正。十六年正月辛亥,兩府奏事,孝宗
 諭以倦勤,欲禪位皇太子,退就休養,以畢高宗三年之制。因令必大進呈詔草。



 二月壬戌,孝宗吉服御紫宸殿,行內禪禮,應奉官以次稱賀。內侍固請帝坐,帝固辭。內侍扶掖至七八,乃微坐,復興。次丞相率百僚稱賀,禮畢,樞密院官升殿奏事,帝立聽。班退,孝宗反喪服,御後殿,帝侍立,尋登輦,同詣重華宮。帝還內,即上尊號曰至尊壽皇聖帝,皇后曰壽成皇后。壽皇聖帝詔立帝元妃李氏為皇后。甲子,帝率群臣朝重華宮。大赦,百官進秩一
 級,優賞諸軍,蠲公私逋負及郡縣淳熙十四年以前稅役。丙寅,帝率群臣詣重華宮,上尊號冊、寶。以閣門舍人譙熙載、姜特立並知閣門事。庚午,詔五日一朝重華宮。辛未,尊皇太后曰壽聖皇太后。壬申,詔內外臣僚陳時政闕失,四方獻歌頌者勿受。遣羅點等使金告即位。癸酉,詔戒敕將帥。賜前宰執、從官詔,訪以得失。乙亥,詔兩省官詳定內外封章,具要切者以聞。遣諸葛廷瑞等使金吊祭。丙子,詔戒敕官吏。己卯,詔官吏贓罪顯著者,重
 罰毋貸。辛巳,以生日為重明節。丁亥,詔百官輪對。己丑,詔編《壽皇聖政》。庾寅,詔中書舍人羅點縣可為臺諫者,點以葉適、吳鎰、孫逢吉、張體仁、馮震武、鄭湜、劉崇之、沉清臣八人上之。



 三月壬辰,以周必大為少保,留正轉正奉大夫。丙申,遣沉揆等使金賀即位。詔侍從、兩省、臺諫,各舉可任湖廣及四川總領者一人。己亥,子擴進封嘉王。癸卯,金遣王元德等來告哀。戊申,以壽皇卻五日之朝,詔自今月四朝重華宮。甲寅,以史浩為太師,伯圭為
 少師,少保土歆為少傅,昭慶軍節度使士峴為少保。戊午,金遣張萬公等來致遺留物。己未,以左補闕薛叔似為將作監,右拾遺許及之為軍器監。拾遺、補闕官自此罷。詔東宮書籍並賜嘉王。



 夏四月丙寅,有事於太廟。丁卯,四川應起經、總制錢存留三年,代輸鹽酒重額。癸酉,侄秉進封許國公。乙亥,以兩浙犒賞酒庫隸諸州,歲入六十五萬,尋減三十萬。戊寅,金遣徒單鎰等來告即位。以權兵部侍郎何澹為右諫議大夫。丙戌,有事於景靈
 宮。



 五月甲午,以王藺知樞密院事兼參知政事。丙申,周必大罷為觀文殿大學士、判潭州。常德府、辰、沅、靖州大水入其郛。丁酉,詔丞相以下月一朝重華宮。戊戌,罷周必大判潭州之命,許以舊官為醴泉觀使。戊申,以和義郡夫人黃氏為貴妃。右丞相留正論知閣門事姜特立,罷之。



 閏月庚申朔,詔內侍陳源許在外任便居住。免郡縣淳熙十四年以前私負,十五年以後輸息及本者亦蠲之。壬戌,以趙雄為寧武軍節度使、開府儀同三司,進
 封衛國公,仍判江陵府。庚午,詔罷賣浙西常平官田。癸酉,詔季秋有事於明堂,以高宗配。丙子,趙雄疾甚,改判資州。戊寅,蠲郡縣第五等戶身丁錢及臨安第五等戶和買絹各一年,仍出錢二十三萬緡振臨安貧民。己卯,階州大水入其郛。壬午,大理獄空。乙酉,御後殿慮囚。



 六月庚寅,鎮江大水入其郛。癸卯,詔自今臣僚奏請事涉改法者,三省、樞密院詳具以聞。



 秋七月辛酉,儒林郎倪恕等以封事可採,遷官、免文解有差。戊辰,遣謝深甫等
 賀金主生辰。庚辰,下詔恤刑。



 八月甲午,升恭州為重慶府。丙申,減兩浙月樁等錢歲二十五萬五千緡。己亥,王淮薨。癸丑,金遣溫迪罕肅等來賀即位。



 九月癸亥,金遣完顏守真等來賀重明節。減紹興和買絹歲額四萬四千餘匹。乙丑,戒執政、侍從、臺諫,毋移書以薦舉、請托。南劍州火,降其守臣一官,仍令優加振濟。戊辰,詔侍從各舉公正強敏之士,嘗任守令及職事官、材堪御史者一人。甲戌,詔監司、帥守,秩滿到闕,薦所部廉吏一二人。遣
 郭德麟等使金賀正旦。



 冬十月庚子,罷樞密院審察諸軍之制。壬寅,蠲楚州、高郵盱眙軍民負常平米一萬四千餘石。甲寅,大閱。



 十一月庚午,詔改明年為紹熙元年。復置嘉王府翊善,以秘書郎黃裳為之。乙亥,詔陳源毋得輒入國門。丁丑,減江、浙月樁錢額十六萬五千餘緡。十二月壬子,金遣裴滿餘慶等來賀明年正旦。



 紹熙元年春正月丙辰朔,帝率群臣詣重華宮,奉上壽聖皇太后、至尊壽皇聖帝、壽成皇后冊寶。壬申,再蠲臨
 安府民身丁錢三年。壬午,何澹請置《紹熙會計錄》。詔何澹同戶部長貳、檢正、都司稽考財賦出入之數以聞。



 二月丁酉,雨雹。辛亥,殿中侍御史劉光祖言:道學非程氏私言,乞定是非,別邪正。從之。



 三月丁卯,詔秀王襲封,置園廟。班安僖王諱。錄趙普後一人。庚午,以久雨,釋杖以下囚。夏四月乙酉,詔兩淮措置流民。己丑,以伯圭為太保、嗣秀王。丁未,殿中侍御史劉光祖以論帶御器械吳端罷。戊申,賜禮部進士餘復以下五百三十有七人及
 第、出身。



 五月乙卯,趙雄坐所舉以賄敗,降封益川郡公,削食邑一千戶。己未,出吳端為浙西馬步軍副總管。丙寅,修楚州城。丙子,太白晝見。



 六月丁亥,遣丘崇等賀金主生辰。丙申,以上供等錢償廣州放免身丁錢數。甲午,御後殿慮囚。



 秋七月癸丑,詔秀王諸孫並授南班。甲寅,以葛邲參知政事,給事中胡晉臣簽書樞密院事。乙卯,以留正為左丞相,王藺樞密院使。癸酉,建秀王祠堂於行在。



 八月辛卯,立任子中銓人吏部簾試法。己亥,帝率
 君臣上《壽皇聖帝玉牒》、《日歷》於重華宮。己酉,詔造新歷。



 九月丁巳,金遣王修等來賀重明節。己未,升劍州為隆慶府。辛酉,雷。庚午,遣蘇山等使金賀正旦。



 冬十月丁酉,詔內外諸軍自今毋置額外制、領以下官。丙午,詔內外軍帥各薦所部有將才者。庚戌,詔諭郡縣吏奉法愛民。



 十一月甲寅,安南入貢。壬戌,潼川轉運判官王溉撙節漕計,代輸井戶重額錢十六萬緡,詔獎之。十二月辛巳朔,贈左千牛衛大將軍挺為保寧軍節度使。壬午,賜王
 倫謚曰節愍。丙戌,罷王藺樞密使。戊子,以葛邲知樞密院事,胡晉臣參知政事兼同知樞密院事。癸卯,詔歲減廣東官賣鹽。丙午,金遣把德固等來賀明年正旦。戊申,浦城盜張海作亂,詔提點刑獄豐誼捕之。



 二年春正月庚戌朔,命兩淮行義倉法。壬子,詔尊高宗為萬世不祧之廟。庚申,修六合城。辛酉,金主母徒單氏殂。戊寅,雷電,雨雹。



 二月庚辰朔,大雨雪。壬午,遣宋之瑞等使金吊祭。癸未,名新歷曰《會元》。甲申,福建安撫使趙
 汝愚等以盜發所部,與守臣、監司各降秩一等,縣令追停。乙酉,詔以陰陽失時,雷雪交作,令侍從、臺諫、兩省、卿監、郎官、館職,各具時政闕失以聞。出米五萬石賑京城貧民。權罷修皇后家廟。辛卯,布衣餘古上書極諫,帝怒,詔送筠州學聽讀。丁未,金遣完顏回等來告哀。



 三月丙辰,詔監司、郡守互送以贓論。丁巳,詔自今邊事令宰相與樞密院議,仍同簽書。丙寅,詔福建提點刑獄陳公亮、知漳州朱熹同措置漳、泉、汀三州經界。丁卯,增廣州摧
 鋒軍三百人。癸酉,建寧府雨雹,大如桃李,壞民居五千餘家。溫州大風雨,雷電,田苗桑果蕩盡。丙子,出右司諫鄧馹。



 夏四月乙酉,從壽皇聖帝、壽成皇后幸聚景園。丙申,詔侍從、兩省、臺諫及在外侍從之臣,各舉所知嘗任監司、郡守可充郎官、卿監及資歷未深可充諸職事官者各三人。辛丑,徽州火,二日乃滅。



 五月己酉朔,福州水。辛亥,詔六院官許輪對,仍入雜壓。庚申,詔侍從、經筵、翰苑官,自今並不時宣對,庶廣咨詢,以補治道。戊長,金州
 大火。己巳,潼川、崇慶二府、大安、石泉、淮安三軍、興、利、果、合、綿、漢六州大水。



 六月戊寅,詔監司到任半年,條上裕民事,如郡守。庚辰,遣趙NU等賀金主生辰。丁亥,以伯圭判大宗正事。癸巳,詔宰臣、執政,自今不時內殿宣引奏事。



 秋七月丁未朔,詔故容州編管人高登追復元官,仍贈承務郎。己未,出會子百萬緡,收兩淮私鑄鐵錢。乙丑,復置太醫局。己巳,興州大水,漂沒數千家。



 八月戊寅,何澹以本生繼母喪去官。甲申,寬兩浙榷鐵之禁。



 九月壬
 子,金遣完顏兗等來賀重明節。召知福州趙汝愚為吏部尚書。壬戌,禁職田折變。癸亥,遣黃申等使金賀正旦。乙丑,以久雨,命大理、三衙、臨安府及兩浙決系囚,釋杖以下。己巳,詔侍從於嘗任卿監、郎官內,選堪斷刑長貳一二人以聞。



 冬十月丙子朔,詔罷經界。丁丑,築福州外城。庚辰,減百官大禮賜物三之一。甲申,復吳端帶御器械。辛卯,詔守令毋徵斂病民。庚子,下詔撫諭四川被水州軍。



 十一月戊申,安定郡王子肜薨。己巳,冊加高宗徽
 號曰受命中興全功至德聖神武文昭仁憲孝皇帝。辛未,有事於太廟。皇后李氏殺黃貴妃,以暴卒聞。壬申,合祭天地於圜丘,以太祖、太宗配,大風雨,不成禮而罷。帝既聞貴妃薨,又值此變,震懼感疾,罷稱賀,肆赦不御樓。壽皇聖帝及壽成皇后來視疾,帝自是不視朝。十二月庚辰,築荊門軍城。丁亥,帝始對輔臣於內殿。乙未,增楚州更戍兵一千五百人。庚子,復出會子百萬緡,收兩淮鐵錢。辛丑,金遣完顏宗璧等來賀明年正旦。壬寅,資、簡、
 普、榮四州及富順監旱。甲辰,詔慶遠軍承宣使、內侍省都知楊皓懷奸兇恣,刺面杖脊,配吉州;和州防禦使、內侍省押班黃邁私相朋附,決杖、編管撫州。尋送皓撫州、邁常州居住。是歲,建寧府、汀州水,階、成、西和、鳳四州及淮東旱,振之。



 三年春正月乙巳朔,帝有疾,不視朝。庚戌,蠲秀州上供米四萬四千石。歲蠲四川鹽酒重額錢九十萬緡。出度僧牒二百,收淮東鐵錢。丁巳,命夔路轉運使通融漕
 計糴米,以備兇荒。壬戌,罷文州民雜役。詔輔臣代行恭謝之禮。



 二月甲戌朔,復以兩浙犒賞酒庫隸戶部。丁酉,申嚴錢銀過淮之禁。



 閏月丙午,禁郡縣新作寺觀。甲寅,以王藺為端明殿學士、四川安撫制置使,藺辭不行。壬戌,詔州縣未斷之訟,監司毋得移獄,違者許執奏。甲子,成都府路轉運判官王溉以代民輸激賞等絹錢三十三萬緡,詔進一官,仍令再任。詔賣郡縣沒官田屋及營田。



 三月甲戌,修天長縣城。辛巳,帝疾稍愈,始御延和殿聽
 政。以子濤為安定郡王。甲申,罷雅州稅場五。築峽州城。乙酉,留正乞去位,不許。庚寅,宜州蠻寇邊,改知鬱林州沙世堅知宜州討之。辛卯,復監司列薦法。丁酉,罷廣東增收鹽斤錢。己亥,詔技藝補授之人毋得奏補,著為令。庚子,監察御史郭德麟以察事失體,出為湖北提舉常平茶鹽。



 夏四月癸卯,補童子吳鋼官。甲寅,振四川旱傷郡縣。乙卯,以戶部侍郎丘崇為煥章閣直學士、四川安撫制置使。戊午,帝朝重華宮。丁卯,蠲臨安民元年二年
 逋賦。



 五月,帝有疾,不視朝。乙未,命漢陽、荊門軍、復州行鐵錢。己亥,蠲四川水旱郡縣租賦。仍以兩浙犒賞酒庫隸諸州,令戶部郎官提領,歲以四十五萬緡為額,庚子晦,常德府大水入其郛。



 六月辛丑朔,下詔戒飭風俗,禁民奢侈與士為文浮靡、吏茍且飾偽者。以權禮部尚書陳騤同知樞密院事,甲辰,遣錢之望等賀金主生辰。丁未,罷四川諸軍歲起西兵。廢光州定城監。壬子,慮囚。戊午,以伯圭為太師。甲子,增捕獲私鑄銅錢賞格。丙寅,以
 太尉郭師禹為少保。



 秋七月己巳,刺沿邊盜萬人為諸州禁軍。壬申,監文思院常良孫坐贓,配海外。益國公周必大坐繆舉良孫,降滎陽郡公。省廣西郡縣官。甲戌,臺州水。壬午,瀘州騎射卒張信等作亂,殺其帥臣張孝芳。甲申,軍士卞進、張昌擊殺信。增嘉王府講讀官二員。壬辰,修揚州城。



 八月甲寅,詔兩淮行鐵錢交子。戊午,總領四川財賦楊輔奏:已蠲東、西兩川畸零絹錢四十七萬緡、激賞絹六萬六千匹。詔獎之。自是歲以為例。



 九月甲
 戌,修德安府外城。乙亥,金遣僕散端等來賀重明節。戊子,遣鄭汝諧等使金賀正旦。丙申,勸兩淮民種桑。



 冬十月壬寅,修大禹陵廟。丙午,修潭州城。辛亥,帝詣重華宮進香。庚申,會慶節,丞相率百官詣重華宮拜表稱賀。



 十一月壬申,振襄陽府被水貧民。癸酉,減蘄州歲鑄錢二十萬緡。丙戌,日南至,丞相率百官詣重華宮拜表稱賀。兵部尚書羅點、給事中尤袤、中書舍人黃裳皆上疏請帝朝重華宮,吏部尚書趙汝愚亦因面對以請,帝
 開納。辛卯,帝朝重華宮,皇後繼至,都人大悅。癸巳,蠲湖南北、京西、江西郡縣月樁、經總制錢歲二十三萬餘緡。戊戌,詔李純乃皇后親侄,可特除閣門宣贊舍人。十二月癸卯,帝率群臣上《壽皇聖帝玉牒》、《聖政》、《會要》於重華宮。丙午,蠲歸正人賦役三年。辛亥,以留正為少保。乙丑,金遣溫敦忠等來賀明年正旦。是歲,江東、京西、湖北水。



 四年春正月己巳朔,帝朝重華宮。辛卯,蠲臨安府民身丁錢三年。



 二月戊戌朔,詔陳源特與在京宮觀。丙寅,貸
 淮西民市牛錢。出米七萬石振江陵饑民。甲戌,皇孫生。



 三月丙子,帝朝重華宮,皇后從。辛巳,以葛邲為右丞相,胡晉臣知樞密院事,陳騤參知政事,趙汝愚同知樞密院事。甲申,監察御史汪義端奏:汝愚執政,非祖宗故事,請罷之。疏三上,不報。辛卯,義端罷。癸巳,帝從壽皇聖帝、壽成皇后幸聚景園。乙未,修巢縣城。



 夏四月己酉,罷括賣四川沿邊郡縣官田。



 五月丙寅朔,復永州義保。己巳,賜禮部進士陳亮以下三百九十有六人及第、出身。進
 士李僑年五十四,調成都司戶參軍,自以祿不及養,乞以一官回贈父母。帝嘉其志,特詔以本官致仕,父母皆與初品官封。丙子,淮西大水。丙戌,紹興大水。召浙東總管姜特立。丞相留正以論特立不行,乞罷相,不報。壬辰,太尉、利州安撫使吳挺卒。四川制置使丘崇承制以總領財賦楊輔權安撫使,命統制官李世廣權管其軍。



 六月丙申朔,留正出城待罪。振江浙、兩淮、荊湖被水貧民。戊戌,秘書省著作郎沉有開,著作佐郎李唐卿,秘書郎
 範黼、彭龜年,校書郎王奭,正字蔡幼學、顏棫、吳獵、項安世上疏,乞寢姜特立召命。己亥,遣許及之等賀金主生辰。壬寅,詔市淮馬充沿江諸軍戰騎。戊申,胡晉臣薨。己酉,御後殿慮囚。癸丑,蠲臨安增民稅錢八萬餘緡。甲寅,太白晝見。甲子,雨雹。



 秋七月乙丑朔,太白晝見。丙寅,大雨雹。己巳,留正復論姜特立,繳納出身以來文字、待罪於範村。丙子,以不雨,命諸路提刑審斷滯獄。戊寅,命臨安府及三衙決系囚,釋杖以下。壬午,以趙汝愚知樞密
 院事,吏部尚書餘端禮同知樞密院事,陳源為內侍省押班。癸未,禁邕州左、右兩江販鬻生口。乙酉,敘州夷賊沒該落無等寇邊,遣兵討平之。



 八月丙申,蠲紹興丁鹽、茶租錢八萬二千餘緡。丁酉,罷郡縣賣沒官田。癸丑,詔三省議振恤郡縣水旱。丁巳,贈吳挺少保;其子曦落階官,起復濠州圖練使、帶御器械。戊午,振江東、浙西、淮西旱傷貧民。



 九月己巳,金遣董師中等來賀重明節,庚午,重明節,百官上壽。侍從、兩省請帝朝重華宮,不聽。己卯,
 上壽聖皇太后尊號曰壽聖隆慈備福皇太后。壬午,遣倪思等使金賀正旦。甲申,帝將朝重華宮,皇后止帝,中書舍人陳傅良引裾力諫,不聽。戊子,著作郎沉有開、秘書郎彭龜年、禮部侍郎倪思等咸上疏,請朝重華宮。



 冬十月丙午,內教三衙諸軍。己酉,朝獻於景靈宮。夜,地震。庚戌,朝獻於景靈宮。夜,地又震。壬子,秘書省官請朝重華宮,疏三上,不報。甲寅,雨土。工部尚書趙彥逾等上疏重華宮,乞會慶聖節勿降旨免朝。壽皇曰:「朕自秋涼以
 來,思與皇帝相見,卿等奏疏,已令進御前矣。」明日會慶節,帝以疾不果朝,丞相葛邲率百官賀於重華宮。侍從上章,居家待罪,詔不許。嘉王府翊善黃裳上疏,請誅內侍楊舜卿。臺諫張叔椿、章穎上疏,乞罷黜。戊午,太學生汪安仁等二百一十八人上書,請朝重華,皆不報。己未,丞相以下奏事重華宮。庚申,帝將朝重華宮,復以疾不果。丞相以下上疏自劾,請罷政,彭龜年請逐陳源以謝天下,皆不報。



 十一月辛未,日中有黑子。壬申,侍從、兩省
 趙彥逾等十一人同班奏事。癸酉,太白晝見,地生毛,夜有赤雲白氣。戊寅,帝朝重華宮,都人大悅。遣右司郎官徐誼召留正於城外。庚辰,正始入朝,復赴都堂視事。命姜特立還故官。日中黑子滅。癸未,帝率群臣奉上皇太后冊、寶於慈福宮。十二月戊戌,帝朝重華宮。壬寅,右司諫章穎以地震請罷葛邲,疏十餘上,不報。甲辰,命沿邊守臣三年為任。己酉,詔監司、帥守毋獨員薦士。庚戌,趙雄薨。甲寅,復四川鹽合同場舊法。丁巳,振江、浙流民。己
 未,金遣完顏弼等來賀明年正旦。



 五年春正月癸亥朔,帝御大慶殿,受群臣朝,遂朝重華宮,次詣慈福宮,行慶壽禮。推恩如淳熙十年故事。癸酉,壽皇聖帝不豫。丙子,大理獄空。癸未,葛邲罷。丙戌,寬紹興民租稅。



 二月乙未,趙汝愚、餘端禮以奏除西帥不行,居家待罪。戊戌,荊鄂諸軍都統制張詔為成州團練使、興州諸軍都統制。庚戌,禁湖南、江西遏糴。



 三月癸亥,合利州東、西為一路。己巳,壽成皇后生辰,免過宮上壽。



 夏
 四月甲午,帝幸玉津園,皇后及後宮皆從。乙未,壽皇聖帝幸東園。丙申,史浩薨。己亥,朝獻於景靈宮。壬寅,以不雨,使大理、三衙、臨安府及兩浙決系囚,釋杖以下。癸卯,雨土。甲辰,侍從入對,請朝重華宮。己酉,太學生程肖說等以帝未朝,移書大臣。事聞,帝將以癸丑日朝。至期,丞相以下入宮門以俟,日昃,帝復以疾不果出。侍從、館學官上疏,乞罷黜,居家待罪。職事官請去待罪者百餘人,詔不許。丙辰,侍講黃裳、秘書少監孫逢吉等再上疏以
 請。丁巳,起居郎兼權中書舍人陳傅良請以親王、執政或近上宗戚一人充重華宮使。臺諫交章劾內侍陳源、楊舜卿、林億年離間兩宮,請罷逐之。



 五月辛酉朔,辰州徭賊寇邊。甲子,侍從入對,未得見。宰執詣重華宮問疾,不及引。陳傅良繳上告敕,出城待罪。丁卯,以壽皇聖帝疾棘,命丞相以下分禱天地、宗廟、社稷。戊辰,丞相留正等請帝侍疾,正引裾隨帝至福寧殿,久之,乃泣而出。辛未,丞相以下以所請不從,求退,帝命皆退,於是丞相以
 下遂出城待罪。知閣門事韓侂冑請宣押入城,許之。追封史浩為會稽郡王。乙亥。帝將朝重華宮,復不果。戊寅,以壽皇聖帝疾,赦。權刑部尚書京鏜入對,請朝重華宮。庚辰,丞相以下詣重華宮問疾。癸未,起居舍人彭龜年叩頭請奏事,詔令上殿,乃請朝重華宮。甲申,從官列奏以請,嘉王府翊善黃裳、講讀官沉有開、彭龜年奏,乞令嘉王詣重華宮問疾,許之。王至重華宮,壽皇為之感動。丙戌,權戶部侍郎袁說友入對,請朝重華宮。



 六月,遣
 梁總等賀金主生辰。戊戌夜,壽皇聖帝崩,遺誥改重華宮為慈福宮,建壽成皇后殿於宮後,以便定省。以重華宮錢銀一百萬緡賜內外軍。先是,丞相留正、知樞密院事趙汝愚、參知政事陳騤、同知樞密院事餘端禮聞壽皇聖帝大漸,見帝於後殿,力請帝朝重華宮,皇子嘉王亦泣以請,不聽。至是,丞相正等聞壽皇聖帝崩,乃率百官聽遺誥於重華宮。己亥,丞相以下上疏,請詣重華成禮。庚子,遣薛叔似等使金告哀。辛丑,丞相率百官拜表,請
 就喪次成服。壬寅,壽皇大斂。皇子嘉王復入奏事,詔侯疾愈,過宮行禮。丞相以下請皇太后垂簾聽政,不許;請代行祭奠禮,許之。仍有旨:皇帝有疾,聽就內中成服。夜,白氣亙天。乙巳,尊壽聖隆慈備福皇太后為太皇太后,壽成皇后為皇太后。己酉,白氣亙天。乙卯,遣林湜等使金致遺留物。



 秋七月辛酉,丞相留正稱疾,乞罷政,遂逃歸。初,正等屢請立嘉王為皇太子,帝許之。正擬指揮以進,奉御筆:「歷事歲久,念欲退閑。」正得之,大懼,乃謀退焉。甲
 子,太皇太后以皇帝疾未能執喪,命皇子嘉王即皇帝位於重華宮之素幄,尊皇帝為太上皇帝,皇后為壽仁太上皇后,移御泰安宮。慶元元年十一月戊戌,上尊號曰聖安壽仁太上皇帝。六年八月庚寅,太上皇帝不豫。辛卯,崩於壽康宮,年五十有四。十一月丙寅,謚曰憲仁聖哲慈孝皇帝,廟號光宗。嘉泰三年十一月壬申,加謚循道憲仁明功茂德溫文順武聖哲慈孝皇帝。



 贊曰:光宗幼有令聞,向用儒雅。逮其即位,總權綱,屏嬖
 幸,薄賦緩刑,見於紹熙初政,宜若可取。及夫宮闈妒悍,內不能制,驚憂致疾。自是政治日昏,孝養日怠,而乾、淳之業衰焉。



\end{pinyinscope}