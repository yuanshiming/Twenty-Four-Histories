\article{本紀第十七}

\begin{pinyinscope}

 哲宗一



 哲宗憲元繼道顯德定功欽文睿武齊聖昭孝皇帝,諱煦,神宗第六子也,母曰欽聖皇后朱氏。熙寧九年十二月七日己丑生於宮中,赤光照室。初名傭,授檢校太尉、
 天平軍節度使,封均國公。元豐五年,遷開府儀同三司、彰武軍節度使,進封延安郡王。七年三月,神宗宴群臣於集英殿,王侍立,天表粹溫,進止中度,宰相而下再拜賀。八年二月,神宗寢疾,宰相王珪乞早建儲,為宗廟社稷計,又奏請皇太后權同聽政,神宗首肯。三月甲午朔,皇太后垂簾於福寧殿,諭珪等曰:「皇子性莊重,從學穎悟。自皇帝服藥,手寫佛書,為帝祈福。」因出以示珪等,所書字極端謹,珪等稱賀,遂奉制立為皇太子。初,太子宮
 中常有赤光,至是光益熾如火。



 戊戌,神宗崩,太子即皇帝位。己亥,大赦天下常赦所不原者。群臣進秩,賜賚諸軍。遣使告哀於遼。白虹貫日。庚子,尊皇太后曰太皇太后,皇后曰皇太后,德妃朱氏曰皇太妃。命宰臣王珪為山陵使。甲寅,以群臣固請,始同太皇太后聽政。



 己未,賜叔雍王顥、曹王頵贊拜不名。令中外避太皇太后父遵甫名。詔邊事稍重者,樞密院與三省同議以進。庚申,尚書左僕射、郇國公王珪進封岐國公。顥進封揚王,頵為
 荊王,並加太保。弟寧國公佶為遂寧郡王,儀國公佖為太寧郡王,成國公俁為咸寧郡王,和國公似為普寧郡王。高密郡王宗晟、漢東郡王宗瑗、華原郡王宗愈、安康郡王宗隱、建安郡王宗綽並為開府儀同三司。太師、潞國公文彥博為司徒,濟陽郡王曹佾為太保,特進王安石為司空,餘進秩,賜致仕服帶、銀帛有差。辛酉,詔顏子、孟子配享孔子廟庭。



 夏四月丙寅,初御紫宸殿。辛未,蠲元豐六年以前逋賦。甲戌,加李乾德同中書門下平章
 事,董氈檢校太尉。詔曰:「先皇帝臨御十有九年,建立政事以澤天下,而有司奉行失當,幾於煩擾,或茍且文具,不能布宣實惠。其申諭中外,協心奉令,以稱先帝惠安元元之意。」乙亥,詔以太皇太后生日為坤成節。丁丑,召呂公著侍讀。諭樞密、中書通議事都堂。詔遵先帝制,遣官察舉諸路監司之法。庚辰,呂惠卿遣兵入西界,破六砦,斬首六百餘級。辛己,遣使以先帝遺留物遺遼國及告即位,甲申,水部員外郎王諤非職言事,坐罰金。丙戌,
 以蕃官高福戰死,錄其子孫。丁亥,復蠲舊年逋賦。



 五月丙申,詔百官言朝政闕失。資政殿學士司馬光過闕,入見。丁酉,群臣請以十二月八日為興龍節。壬寅,城熙、蘭、通遠軍,賜李憲、趙濟銀帛有差。甲辰,作受命寶。丙午,京師地震。復置遼州。庚戌,王珪薨。改命蔡確為山陵使。丙辰,賜禮部奏名進士、諸科及第出身四百六十一人。戊午,以蔡確為尚書左僕射兼門下侍郎,韓縝為尚書右僕射兼中書侍郎,章惇知樞密院,司馬光為門下侍郎。



 六月庚午,賜楚州孝子徐積絹米。丁亥,詔中外臣庶許直言朝政闕失、民間疾苦。



 秋七月戊戌,以資政殿大學士呂公著為尚書左丞。詔府界、三路保甲罷團教。丙午,遼人來吊祭。丙辰,白虹貫日。吏部侍郎熊本奏歸化儂智會異同,坐罰金。罷沅州增修堡砦。



 八月乙丑,詔按察官所至,有才能顯著者以名聞。己巳,鎮江軍節度使韓絳進開府儀同三司。癸酉,遣使賀遼主生辰、正旦。乙亥,以供奉王英戰死葭蘆,錄其子。



 九月戊戌,以神宗英文
 烈武聖孝皇帝之謚告於天地、宗廟、社稷。己亥,上寶冊於福寧殿。己酉,遣使報謝於遼。



 冬十月甲子,夏國遣使進助山陵馬。癸酉,詔仿《唐六典》置諫官。丁丑,令侍從各舉諫官二人。詔監察御史兼言事,殿中侍御史兼察事。罷義倉。己卯,詔均寬民力,有司或致廢格者,監司、御史糾劾之。河決大名。乙酉,葬神宗皇帝於永裕陵。丙戌,罷方田。以夏國主母卒,遣使吊祭。



 十一月癸巳,詔按問強盜,欲舉自首者毋減。丁酉,祧翼祖,祔神宗於太廟,廟樂
 曰《大明之舞》。辛丑,減兩京、河陽囚罪一等,杖已下釋之,民緣山陵役者蠲其賦。己酉,遼遣使賀即位。十二月壬戌,於闐進獅子,詔卻之。開經筵,講《魯論》,讀《三朝寶訓》。罷《太學保任同罪法》。丙寅,夏人以其母遺留物、馬、白駝來獻。辛未,左僕射蔡確、右僕射韓縝並遷秩、加食邑,揚王顥、荊王頵並為太傅。壬申,章惇、司馬光等進秩有差。申戌,罷後苑西作院。乙亥,詔執政、侍臣講讀。戊寅,罷增置鑄錢監十有四。乙酉,遼遣蕭睦等來賀正旦。是歲,日有
 五色雲者六。高麗、大食入貢。



 元祐元年春正月庚寅朔,改元。丙午,錄在京囚,減死罪以下一等,杖罪者釋之。丁未,詔回賜高麗王鞍馬、服帶、器幣有加。罷陜西、河東元豐四年後凡緣軍興添置官局。丙辰,久旱,幸相國寺祈雨。立神宗原廟。戊午,甘露降。



 二月辛酉,以河決大名,壞民田,民艱食者眾,詔安撫使韓絳振之。乙丑,修《神宗實錄》。丁卯,詔左右侍從各舉堪任監司者二人,舉非其人有罰。庚午,禁邊民與夏人為
 市。辛未,董氈卒,以其子阿裏骨襲河西軍節度使、邈川首領。庚辰,夏人入貢。辛巳,刑部侍郎蹇周輔坐變鹽法落職。



 閏月庚寅,蔡確罷。以司馬光為尚書左僕射、門下侍郎。詔韓維、呂大防、孫永、范純仁詳定役法。壬辰,以呂公著為門下侍郎。丙午,守尚書右丞李清臣為尚書左丞,試吏部尚書呂大防為尚書右丞。白虹貫日。丁未,群臣上太皇太后宮名曰崇慶,殿曰崇慶壽康;皇太后宮曰隆祐,殿曰隆祐慈徽。庚戌,賜於闐國王服帶、器幣。辛
 亥,章惇罷。甲寅,詔侍從、御史、國子司業各舉經明行修、可為學官者二人。乙卯,以吏部尚書范純仁同知樞密院事。丙辰,掩京城暴骸。罷諸州常平管勾官。



 三月辛未,詔毋以堂差沖在選已注官。置訴理所,許熙寧以來得罪者自言。命太學公試,司業、博士主之,如春秋補試法。癸酉,置開封府界提點刑獄一員。乙亥,罷熙河蘭會路經制財用司。己卯,復廣濟河輦運。辛巳,詔民間疾苦當議寬恤者,監司具聞。以程頤為崇政殿說書。乙酉,許職
 事官帶職。



 夏四月己丑,韓縝罷。辛卯,詔諸路旱傷蠲其租。壬辰,以旱慮囚。癸巳,王安石薨。辛丑,詔執政大臣各舉可充館閣者三人。壬寅,以呂公著為尚書右僕射兼中書侍郎,文彥博平章軍國重事。乙巳,詔戶部裁冗費,著為令。李憲等以用兵失利,為劉摯所劾,貶秩奉祠。辛亥,揚王顥、荊王頵並特授太尉。詔遇科舉,令升朝官各舉經明行修之士一人,俟登第日與升甲。罷謁禁之制。知誠州周士隆撫納溪洞民一千三百餘戶,賜士隆銀
 帛。癸丑,定六曹郎官員數。



 五月丁巳朔,以資政殿大學士韓維為門下侍郎。罷諸路重祿,復熙寧前舊制。庚申,夏人來賀即位。壬戌,詔侍從、臺官、監司各舉縣令一人。戊辰,命程頤同修立國子監條制。己巳,幸揚王、荊王第,官其子九人。癸酉,復左、右天廄坊。壬午,詔文彥博班宰相之上。



 六月甲辰,置《春秋》博士。呂惠卿落職,分司南京、蘇州居住。戊申,以富弼配享神宗廟庭。庚戌,太白晝見。甲寅,詔正風俗,修紀綱,勿理隱疵細故。復置通利軍。程
 頤上疏論輔養君德。



 秋七月丁巳,置檢法官。辛酉,設十科舉士法。劉恕同修《資治通鑒》,未沾恩而卒,詔官其子。乙丑,夏國主秉常卒。庚午,夏國遣使賀坤成節。



 八月辛卯,詔常平依舊法,罷青苗錢。壬辰,封弟偲為祁國公。甲午,占城國遣使入貢。壬子,日傍有五色雲。磁州谷異壟同穗。



 九月丙辰朔,司馬光薨。己未,朝獻景靈宮。辛酉,大享明堂,以神宗配,赦天下。丁卯,試中書舍人蘇軾為翰林學士、知制誥。己卯,張璪罷。



 冬十月丙戌,改衍聖公為
 奉聖公。庚寅,太白晝見。壬辰,夏人來告哀。庚子,遣使吊祭。十一月戊午,以尚書左丞呂大防為中書侍郎,御史中丞劉摯為尚書右丞。乙亥,於闐國遣使入貢。庚辰,蠲鹽井官溪錢。



 十二月庚寅,詔將來服除,依元豐三年故事,群臣勿上尊號。戊戌,華州鄭縣小敷谷山崩。戊申,詔以冬溫無雪,決系囚。是歲,河北、楚、海諸州水。



 二年春正月乙丑,封秉常子乾順為夏國主。戊辰,詔舉人程試,主司毋得於《老》、《莊》、《列子》書命題。辛巳,詔蘇轍、劉
 分文編次神宗御制。白虹貫日。



 二月丁亥,遣左司諫朱光庭使河北,振民被災者。詔施、黔、戎、瀘等州保甲監司免歲閱。丁酉,加賜於闐國金帶、錦袍、器幣。己亥,命吏部選人改官,歲以百人為額。辛丑,詔陜西、河東行策應牽制法。是月,代州地震。



 三月壬戌,太皇太后手詔,止就崇政殿受冊。戊辰,詔中外侍從歲舉郡守各一人。令御史臺察民俗奢僭者。夏人遣使入謝。癸酉,奉安神御於景靈宮宣光殿。庚辰,詔內侍省供奉官以下百人為額。



 夏
 四月丙戌,交址入貢。丁亥,鬼章子結齷齪寇洮東。戊子,慮囚。己丑,詔太師文彥博十日一議事都堂。辛卯,詔:「冬夏旱□,海內被災者廣,避殿減膳,責躬思過,以圖消復。」丁酉,以四方牒訴上尚書者,或冤抑不得直,令御史分察之。己亥,太皇太后以旱權罷受冊禮。癸卯,雨。乙丑,以徐州布衣陳師道為亳州司戶參軍。丁未,復制科。戊申,御殿復膳。李清臣罷。



 五月癸丑,夏人圍南川砦。丁卯,以劉摯為尚書左丞,兵部尚書王存為尚書右丞。壬申,於
 闐入貢。丁丑,詔御史官闕,御史中丞、翰林學士、兩省諫議大夫以上雜舉。



 六月辛丑,以安燾知樞密院事。壬寅,有星如瓜出文昌。丙午,邈川首領結藥來降,授三班奉職。



 秋七月辛亥,詔戶部修《會計錄》。韓絳以司空致仕。夏人寇鎮戎軍。詔府界、三路教閱保甲。復課利場務虧額科罰。丙辰,罷諸州數外歲貢。戊午,以遼蕭德崇等賀坤成節,曲宴垂拱殿,始用樂。庚申,進封李乾德為南平王。辛酉,改誠州為渠陽軍。辛未,韓維罷。



 八月辛巳,程頤罷
 經筵,權同管勾西京國子監。癸未,以西蕃寇洮、河,民被害者給錢粟,死者賜帛其家。詔復進納人改官舊法。乙酉,命呂大防為西京安奉神宗御容禮儀使。庚寅,西南蕃遣人入貢。癸巳,以夏國政亂主幼,強臣乙逋等擅權逆命,詔諸路帥臣嚴兵備之。庚子,授西蕃首領心牟欽氈銀州團練使,溫溪心瓜州團練使。辛丑,涇原言夏人寇三川諸砦,官軍敗之。丁未,岷州行營將種誼復洮州,執蕃酋鬼章青宜結。



 九月乙卯,發太皇太后冊寶於大
 慶殿。丙辰,發皇太后、皇太妃冊寶於文德殿。己未,夏人寇鎮戎軍。丁卯,禁私造金箔。



 冬十月壬午,奉安神宗御容於會聖宮及應天院。癸未,日有五色雲。戊子,恭謝景靈宮。辛卯,減西京囚罪一等,杖已下釋之。己亥,西南龍、張蕃遣人入貢。庚子,論復洮州功,種誼等遷秩、賜銀絹有差。



 十一月丙辰,復置漣水軍。庚申,獻鬼章於崇政殿,以罪當死,聽招其子及部屬歸以自贖。乙亥,大雪甚,民凍多死,詔加振恤,死無親屬者官瘞之。罷內殿承制試
 換文資格。丙子,決囚。十二月乙酉,賜諸軍及貧民錢。丙戌,興龍節,初上壽於紫宸殿。己丑,大寒,罷集英殿宴。壬辰,兀徵聲延部族老幼萬人渡河南,遣使廩食之,仍諭聲延勿失河北地。乙未,白虹貫日。壬寅,頒《元祐敕令式》。是冬,始閉汴口。



 三年春正月己酉朔,不受朝。庚戌,復廣惠倉。己未,朝獻景靈宮。庚申,雪寒,發京西谷五十餘萬石,損其直以紓民。辛酉,詔廣南西路朱崖軍開示恩信,許生黎悔過自
 新。壬戌,罷上元游幸。壬申,阿裏骨奉表詣闕謝罪,令邊將無出兵,仍罷招納。甲戌,決囚。



 二月甲申,罷修金明池橋殿。乙酉,德音:減囚罪一等,徒以下釋之,工役權放一年,流民饑貧量與應副。丙戌,詔河東苦寒,量度存恤戍兵。癸巳,罷春宴。乙未,白虹貫日。辛丑,太白晝見。乙巳,廣東兵馬監童政坐擅殺無辜,伏誅。



 三月丙辰,韓絳薨。丁巳,御集英殿策進士。戊午,策武舉。己巳,賜禮部奏名進士、諸科及第出身一千一百二十二人。乙亥,夏人寇德
 靜砦,將官張誠等敗之。



 夏四月戊寅,令諸路郡邑具役法利害以聞。辛巳,以呂公著為司空、同平章軍國事,呂大防為尚書左僕射兼門下侍郎,范純仁為尚書右僕射兼中書侍郎。壬午,以觀文殿學士孫固為門下侍郎,劉摯為中書侍郎,王存為尚書左丞,御史中丞胡宗愈為尚書右丞,戶部侍郎趙瞻簽書樞密院事。癸巳,詔定職事官歲舉升陟人數。丁酉,阿裏骨來貢。庚子,詔天下郡城以地里置壯城兵額,禁勿他役。



 五月癸亥,漢東郡
 王宗瑗薨。



 六月癸未,詔司諫、正言、殿中、監察御史,仿故事,以升朝官通判資序歷一年者為之。辛丑,夏人寇塞門砦。甲辰,五色雲見。



 秋七月戊申,荊王頵薨。戊辰夜,東北方明如晝,俄成赤氣,中有白氣經天。辛未,太白晝見。癸酉,忠州言臨江塗井鎮雨黑黍。



 八月戊寅,阿裏骨入貢。己卯,進封揚王顥為徐王。辛巳,復置荊門軍。丙戌,罷吏試斷刑法。丁酉,渠陽蠻入寇。辛丑,降系囚罪一等,杖以下釋之。



 九月庚申,禁宗室聯姻內臣家。乙丑,阿裏骨
 復遷職,加封邑。詔觀察使以上給永業田。丁卯,御集英殿策賢良方正能直言極諫科。



 十月丙戌,詔罷新創諸堡砦,廢渠陽軍。戊戌,復南、北宣徽院。



 十一月甲辰,遣吏部侍郎範百祿等行河。丁卯,大食麻囉拔國入貢。詔歲以十月給巡城兵衣裘。十二月丁酉,渝州獠人寇小溪。壬寅,白虹貫日。



 閏月癸卯朔,頒《元祐式》。甲辰,範鎮定鑄律、度量、鐘磬等以進,令禮部、太常參定。戊申,減宰執賜予。庚申,置六曹尚書權官。丙寅,詔吏部詳定六曹重復
 利害以聞。是歲,三佛齊、于闐、西南蕃入貢。天下上戶部:主戶二百一十三萬四千七百三十三,丁二千八百五十三萬三千九百三十四。客戶六百一十五萬四千六百五十二,丁三百六十二萬九千八十三。斷大闢二千九百一十五人。



 四年春正月壬申朔,不受朝,群臣及遼使詣東上閣門、內東門拜表賀。丙子,宴遼使於紫宸殿。甲申,以夏人通好,詔邊將毋生事。



 二月甲辰,呂公著薨。庚戌,白虹貫日。
 乙卯,夏人來謝封冊。



 三月己卯,作渾天儀。胡宗愈罷。丁亥,以不雨,罷春宴。己丑,詔自今大禮毋上尊號。辛卯,晝有流星出東方。癸巳,錄囚。乙未,罷幸瓊林苑、金明池。



 夏四月乙巳,呂大防等以久旱求罷,不允。丁未,曹佾薨。戊申,罷大禮使及奏告執政加賜。戊午,立試進士四場法。壬戌,弛在京牧地與民。



 五月癸酉,詔自今侍讀以三人為額。中丞李常、侍御史盛陶坐不論蔡確,改官。辛巳,貶觀文殿學士蔡確為光祿卿。丁亥,復貶確為英州別駕、
 安置新州。丁酉,於闐國來貢。



 六月甲辰,范純仁、王存罷。丙午,以趙瞻同知樞密院事,戶部尚書韓忠彥為尚書左丞,翰林學士許將為尚書右丞。丁未,夏國來貢。癸丑,邈黎國般次泠移、四林慄迷等繼於闐國黑汗王及其國蕃王表章來貢。秋七月丙子,詔復外都水使者。丁丑,遼國使蕭寅等來賀坤成節,曲宴垂拱殿。庚辰,安燾以母憂去位。



 八月壬寅,敕郡守貳以「四善三最」課縣令,吏部歲上監司考察知州狀。辛酉,太皇太后詔:今後明堂
 大禮,毋令百官拜表稱賀。九月戊寅,致齋垂拱殿。己卯,朝獻景靈宮,辛巳,大饗明堂,赦天下,百官加恩,賜繼士庶高年九十以上者。乙酉,加賜韓縝、范純仁器幣有差。乙未,檢舉先朝文武七條,戒諭百官遵守。



 冬十月辛丑,西南程蕃入貢。丁未,龍蕃入貢。戊申,翰林學士蘇轍上《神宗御集》,藏寶文閣。癸丑,御邇英殿,講官進講《三朝寶訓》。



 十一月庚午,敕朝請大夫以下進士為左,餘為右。溪洞彭儒武等進溪洞布。癸未,以孫固知樞密院事,劉摯
 為門下待郎,吏部尚書傅堯俞為中書侍郎。乙酉,有星色赤黃,尾跡燭地。己丑,太皇太后卻元日賀禮,令百官拜表。庚寅,章惇買田不法,降官。辛卯,改發運、轉運、提刑預妓樂宴會徒二年法。十二月庚子,遼使耶律常等賀興龍節,曲宴垂拱殿。癸丑,更定朝儀二舞曰《威加四海》、《化成天下》。甲寅,減鄜延等路戍兵歸營。戊午,以御史闕,令中丞、兩省各舉二人。是歲,夏國、邈黎、大食、麻囉拔國入貢。



 五年春正月丁卯朔,御大慶殿視朝。丁丑,朝獻景靈宮。



 二月丁酉,罷諸州、軍通判奏舉改官。己亥,夏人歸永樂所掠吏士百四十九人。庚子,加溪洞人田忠進等九十二人檢校官有差。辛丑,以旱罷,修黃河。癸卯,禱雨岳瀆,罷浚京城壕。丁未,減天下囚罪,杖以下釋之。庚戌,文彥博以太師充護國軍、山南西道節度等使致仕,令所司備禮冊命。壬子,彥博乞免冊禮,從之。甲子,宴餞文彥博於玉津園。



 三月丙寅朔,趙瞻薨。丁卯,詔賜故孫覺家緡
 錢,令給喪事。壬申,以韓忠彥同知樞密院事,翰林學士承旨蘇頌為尚書左丞。癸未,罷春宴。壬辰,罷幸金明池、瓊林苑。



 夏四月癸卯,詔鄭穆、王巖叟等同舉監察御史二員。甲辰,呂大防等以旱求退,不允。丙午,孫固薨。癸丑,詔講讀官御經筵退,留二員奏對邇英閣。丁巳,詔以旱、避殿減膳,罷五月朔日文德殿視朝。辛酉,以保寧軍節度使馮京為檢校司空。



 五月壬申,詔差役法有未備者,令王巖叟等具利害以聞。乙亥,雨。己卯,御殿復膳。



 六月
 辛丑,錄囚。癸亥,晝有五色雲。



 七月壬申,涇原路經略司言:諸人違制典買蕃部田土,許以免罪,自二頃五十畝以下,責其出刺弓箭手及買馬備邊用各有差。乙酉,夏人來議分畫疆界。



 九月丁丑,詔復置集賢院學士。



 冬十月癸巳,罷提舉修河司。丁酉,詔定州韓琦祠載祀典。



 十二月辛卯朔,許將罷。安康郡王宗隱薨。丙辰,禁軍大閱,賜以銀楪、匹帛,罷轉資。是歲,東北旱,浙西水災。賜宗室子授官者四十四人。斷大闢四千二百六十有一。高麗、
 于闐、龍蕃、三佛齊、阿裏骨入貢。



 六年春正月辛酉朔,不受朝,群臣及遼使詣東上閣門、內東門拜表賀。癸酉,詔祠祭、游幸毋用羔。



 二月辛卯,以劉摯為尚書右僕射兼中書侍郎,龍圖閣待制王巖叟簽書樞密院事。癸巳,以蘇轍為尚書右丞,宗室士俔追封魏國公。庚子,拂箖國來貢。丁丑,授阿裏骨男溪邦彪籛為化外庭州團練使。



 三月癸亥,呂大防上《神宗實錄》。己巳,御集英殿策進士。庚午,策武舉。癸酉,詔御史中丞
 舉殿中侍御史二人,翰林學士至諫議大夫同舉監察御史二人。丙子,呂大防特授右正議大夫。壬午,賜禮部奏名進士、諸科及第出身九百五十七人。丁亥,罷幸金明池、瓊林苑。



 夏四月乙未,復置通禮科。丙申,詔恤刑。辛丑,詔大臣堂除差遣,非行能卓異者不可輕授。仍搜訪遺材,以備擢任。夏人寇熙河蘭岷、鄜延路。壬寅,太白晝見。壬子,賜南平王李乾德袍帶、金帛、鞍馬。



 五月己未朔,日有食之,罷文德殿視朝。庚辰,詔娶宗室女得官者,毋
 過朝請大夫、皇城使。丁亥,後省上《元祐敕令格》。



 六月壬辰,錄囚。甲辰,置國史院修撰官。乙卯,詔以田思利為銀青光祿大夫,充溪洞都巡檢。



 秋七月癸亥,復張方平宣徽南院使致仕。乙丑,復制置解鹽使。己卯,振兩浙水災。



 八月己丑,三省進納後六禮儀制。辛卯,詔御史臺:臣僚親亡十年不葬,許依條彈奏及令吏部檢察。己亥,改宗正屬籍曰《宗藩慶系錄》。令文武臣出入京城門書職位、差遣、姓名及所往。己酉,修《神宗寶訓》。癸丑,詔鄜延路都
 監李儀等以違旨夜出兵入界,與夏人戰死,不贈官,餘官降等。乙卯,夏人寇懷遠砦。



 閏月壬戌,嚴飭陜西、河東諸路邊備。甲子,太白晝見。庚午,詔御史中丞舉殿中侍御史二人,翰林學士、中書舍人、給事中舉監察御史四人。壬申,太子太保致事張方平辭免宣徽使,不允。甲申,刑部侍郎彭汝礪與執政爭獄事,自乞貶逐,詔改禮部侍郎。



 九月丁亥,夏人寇麟、府二州。壬辰,詔州民為寇所掠,廬舍焚蕩者給錢帛,踐稼者振之,失牛者官貸市之。
 癸巳,御集英殿策賢良方正能直言極諫科。丁酉,御試方正王普等,遷官有差。歲出內庫緡錢五十萬以備邊費。甲辰,幸上清儲祥宮。壬子,宮成,減天下囚罪一等,杖以下釋之。癸丑,以執政官行謁禁法非便,詔有利害陳述勿禁。



 冬十月丁卯,有流星晝出東北。庚午,朝獻景靈宮,還,幸國子監,賜祭酒豐稷三品服,監學官賜帛有差。庚辰,令諸宮院建小學。貴妃苗氏薨。癸未,編修神宗御制官轉秩加賞。詔京西提刑司歲給錢物二十萬緡,以
 奉陵寢。



 十一月乙酉朔,劉摯罷。壬辰,作《元祐觀天歷》。尚書右丞蘇轍罷知絳州。辛丑,傅堯俞薨。十二月戊辰,開封府火。壬申,范純仁以前禦敵失策降官。是歲,兩浙水,定州野蠶成繭。高麗、交址、三佛齊入貢。



 七年春正月甲辰,以遼使耶律迪卒,輟朝一日。乙巳,張誠一以穿父墓取犀帶,責授左武衛將軍,提舉亳州明道宮。



 二月丁卯,詔陜西、河東邊要進築守禦城砦。



 三月己亥,錄囚。



 夏四月己未,立皇後孟氏。甲子,命呂大防為
 皇后六禮使。甲戌,立考察縣令課績法。



 五月戊戌,御文德殿冊皇后。庚子,罷侍從官轉對。丙午,王巖叟罷知鄭州。大食進火浣布。



 六月辛酉,以呂大防為右光祿大夫,蘇頌為尚書右僕射兼中書侍郎,韓忠彥知樞密院事,蘇轍為門下侍郎,翰林學士範百祿為中書侍郎,翰林學士梁燾為尚書左丞,御史中丞鄭雍為尚書右丞,戶部尚書劉奉世簽書樞密院事。甲子,置廣文館解額。戊辰,渾天儀像成。甲戌,日旁五色雲見。



 七月癸巳,詔修《神
 宗史》。復翰林侍講學士。己酉,詔諸路安撫鈐轄司及西京、南京各賜《資治通鑒》一部。庚戌,宗室緦麻以上者禁析居。



 八月丙辰,罷監酒稅務增剩給賞法。己未,詔西邊諸將嚴備,毋輕出兵。乙亥,戒邊將毋掊克軍士。前陷交址將吏蘇佐等十七人自拔來歸。



 九月戊戌,詔:「冬至日南郊,宜依故事設皇地祇位。禮畢,別議方澤之儀以聞。」己酉,永興軍、蘭州、鎮戎軍地震。



 冬十月庚戌朔,環州地震。丁巳,陜西有前代帝王陵廟處,給民五家充守陵戶。丁
 卯,夏人寇環州。



 十一月辛巳,太白晝見。甲申,詔太中大夫以上許占永業田。丙戌,於闐入貢。庚寅,帝齋大慶殿。辛卯,朝獻景靈宮。壬辰,饗太廟。癸巳,祀天地於圜丘,赦天下,群臣中外加恩。罷南京榷酒。民罹親喪者,戶以差等與免徭。辛丑,賜徐王劍履上殿。十二月辛亥,阿裏骨、李乾德加食邑實封。甲子,罷飲福宴。庚午,祈雪。是歲,兗州仙源縣生瑞穀。高麗、占城、西南蕃龍氏、羅氏入貢。



 八年春正月己卯朔,不受朝。甲申,蔡確卒。丁亥,御邇英
 閣,召宰臣讀《寶訓》。庚寅,詔復范純仁太中大夫。壬辰,幸太乙宮。庚子,詔頒高麗所獻《黃帝針經》於天下。



 二月己酉,詔西南蕃龍氏遷秩補官。辛亥,禮部尚書蘇軾言:「高麗使乞買歷代史及《策府元龜》等書,宜卻其請不許。」省臣許之,軾又疏陳五害,極論其不可。有旨:「書籍曾經買者聽。」壬子,詔刑部不得分禁系人數,瘐死數多者申尚書省。癸丑,詔大寧郡王以下出就外學。



 三月甲申,蘇頌罷。辛卯,範百祿罷。庚子,詔御試舉人復試賦、詩、論三題。



 夏四月丁未朔,夏人來謝罪,願以蘭州易塞門砦,不許。癸丑,詔恤刑。甲寅,令範祖禹依先朝故事止兼侍講。丁巳,詔南郊合祭天地,罷禮部集官詳議。



 五月癸未,置蘄州羅田縣。丁亥,罷二廣鑄折二錢。己丑,錄囚。辛卯,監察御史董敦逸、黃慶基以論蘇軾、蘇轍,罷為湖北、福建轉運判官。己亥,祁國公人思為開府儀同三司。



 六月戊午,梁燾罷。壬戌,中書後省上《元祐在京通用條貫》。



 秋七月丙子朔,以觀文殿大學士范純仁為尚書右僕射兼中
 書侍郎。戊寅,令陜西沿邊鐵錢、銅錢悉還近地。



 八月丁未,久雨。禱山川。辛酉,以太皇太后疾,帝不視事。壬戌,遣使按視京東西、河南北、淮南水災。癸亥,減京師囚罪一等,徒以下釋之。丁卯,禱於岳瀆、宮觀、祠廟。戊辰,赦天下。庚午,詔陜西復鑄小銅錢。辛未,禱於天地、宗廟、社稷。乙亥,禱於諸陵。



 九月戊寅,太皇太后崩。己卯,詔以太皇太后園陵為山陵。庚辰,遣使告哀於遼。甲申,命呂大防為山陵使。壬辰,詔山陵修奉從約,諸道毋妄有進助。



 冬十
 月戊申,群臣七上表請聽政。戊辰,徐王顥乞解官給喪,詔不允。庚午,復內侍劉瑗等六人。



 十一月丙子,始御垂拱殿。乙未,以雪寒,振京城民饑。壬寅,賜勞修奉山陵兵士。十二月乙巳,范純仁乞罷,不允。甲寅,仿《唐六典》修官制。丁巳,遼人遣使來吊祭。出錢粟十萬振流民。己巳,上太皇太后謚曰宣仁聖烈皇后。是歲,河入德清軍,決內黃口。



\end{pinyinscope}