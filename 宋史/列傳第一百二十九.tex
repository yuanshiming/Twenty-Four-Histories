\article{列傳第一百二十九}

\begin{pinyinscope}

 王友直李寶成閔趙密劉子羽呂祉胡世將鄭剛中



 王友直,字聖益,博州高平人。父佐,以材武稱。友直年十二,隨父游,諳兵法。



 紹興三十一年,金人渝盟,友直結豪
 傑,志恢復。謂其眾曰:「權所以濟事,權歸於正,何害於理。」乃矯制自擬承宣使、河北等路安撫制置使,餘擬官有差,遍諭州縣勤王。未幾,得眾數萬,制為十三軍,軍置都統制、提舉、提點、提轄、訓練統之。九月戊子,進攻大名,一鼓而克,撫定眾庶,諭以紹興年號。乃與王任、馮谷、張升、牛汝霖列奏於朝,欲領眾南歸。時金人尚在揚州,久不報。



 友直將由壽春涉淮而濟,道拜敕書勉以率眾搗敵腹心,掎角應援。除友直檢校少保、天雄軍節度使,王任
 天平軍節度使,馮谷左通議大夫、徽猷閣直學士,張升右朝奉大夫、直秘閣,牛汝霖通直郎、直秘閣,職任各從舊,得便宜行事。時三十二年正月一日也。



 旋與敵遇,相拒淮北;敵兵來益眾,友直即率所部渡淮。既而審金主亮已斃,所遇乃歸師,悔不襲擊之。高宗視師江上,見於金陵,賜金帶、章服,錫賚及二子。友直恥前功不遂,自陳,改復州防禦使,以忠義軍統制隸鎮江都統司。



 越四月,詔偕統制張子蓋援海州。方接戰,友直張一旗,大書「宋
 忠義將河北王九郎」以自表。潛由小徑背敵陣,因其輜重,扼歸道橋,左右枕水。張子蓋知友直已乘敵後,麾軍進擊,敵潰走,盡溺死,圍遂解。轉宜州觀察使。



 孝宗受禪,友直與統制宋寧數出奇轉戰。張浚都督江、淮,一見喜之,闢建康前軍統制。隆興二年九月,金人犯邊,宣諭使王之望命以前軍戍昭關,友直不逾時即行。他軍同戍者,敵至,輒退保和州,友直孤軍堅守。金兵駐黃山,鼓柝相聞,益整暇自持。



 乾道元年,移鎮江御前諸軍統制,俄
 改步司左軍統制兼左驍衛上將軍。初,淮北之戰,友直母子相失,至是,訪得之,乃與其妻李攜二女自淮而還,錫予加厚。又明年,除御前諸軍統制,請祠,手詔慰勞。四年,繇京口入覲,進神、龍衛四廂都指揮使,主管步司公事,遷侍衛親步軍都指揮使。朝廷議遣馬、步二司移屯重地,丞相虞允文欲先發步司,友直請以馬司先。及馬帥李顯忠屯金陵,友直奏馬軍道途轉徙,困斃已甚。有旨免移步司。八年,轉承宣使,旋除殿前副都指揮使。



 淳
 熙元年,授奉國軍節度使。四年,總殿步司大閱於茅灘,鎧仗精明,號令閑肅。明年,進殿前指揮使,賜第中都,賜田平江,燕射咸預。晚節宴安,軍政稍失律,授宜州觀察使。尋罷宮觀,徙居信州。以郊祀恩內徙,三奉祠,復武寧軍承宣使。卒,年六十一,追復節度使,贈檢校少保。



 李寶,河北人。嘗陷金,拔身從海道來歸。金主亮渝盟,淮、浙奸民倪詢、梁簡等教金造舟,且為鄉導。金使蘇保衡造舟於潞河。明年,以保衡為統軍,將繇海道襲浙江。諜
 聞,高宗謂宰臣曰:「李寶頃因召對,詢以北事,歷歷如數。且以一介脫身還朝,陛對無一毫沮懾,是必能事者。」乃授浙西路馬步軍副總管,駐札平江,令與守臣督海舟捍禦。高宗問:「舟幾何?」曰:「堅全可涉風濤者,百二十艘。」「兵幾何?」曰:「僅三千,皆閩、浙弓弩手,非正兵也。旗幟甲仗亦粗備。事急矣,臣願亟發。」賜寶衣帶、鞍馬、尚方弓刀、戈甲及銀絹萬數。



 八月,次江陰,先遣其子公佐,謂曰:「汝為潛伺敵動靜虛實,毋誤。」公佐受命,即與將官邊士寧偕往。
 寶將啟行,軍士爭言西北風力尚勁,迎之非利。寶下令,敢沮大計者斬。遂發蘇州,大洋行三日,風甚惡,舟散不可收。寶慷慨顧左右曰:「天以是試李寶邪?寶心如鐵石,不變矣。」酹酒自誓,風即止。明日,散舟復集。



 士寧自密州回,得敵耗甚悉,且言公佐已挾魏勝得海州。寶喜曰:「吾兒不負乃翁矣。」士氣百倍,趣眾乘機進。適大風復作,海濤如山,寶神色不為動;風少殺,始縱舟泊抵東海。敵已雲合,圍海州,旌麾數十里。寶麾兵登岸,以劍畫地,令曰:「
 此非復吾境,力戰與否在汝等。」因握槊前行,遇敵奮擊,將士賈勇,無不一當十。敵出不意,亟引去。勝出城迎,寶獎其忠義,勉以共立功名,勝感泣。乃維舟犒士,遺辯者四出招納降附,聲振山東。豪傑如王世修輩各署旗,集義勇,爭應援,多者數萬人。寶列名上諸朝,檄所部會密之膠西,命公佐以郡事畀勝,與俱發。



 至膠西石臼島,敵舟已出海口,泊唐島,相距僅一山。時北風盛,寶禱於石臼神。俄有風自柂樓中來,如鐘鐸聲,眾咸奮,引舟握刃
 待戰。敵操舟者皆中原遺民,遙見寶船,紿敵兵入舟中,使不知王師猝至。風駛舟疾,過山薄虜,鼓聲震疊,海波騰躍。敵大驚,掣釘舉帆,帆皆油纈,彌亙數里,風浪卷聚一隅,窘束無復行次。



 寶亟命火箭環射,箭所中,煙焰旋起,延燒數百艘。火所不及者猶欲前拒,寶叱壯士躍登其舟,短兵擊刺,殪之舟中。餘所謂簽軍,盡中原舊民,皆登島垠,脫甲歸命,以故不殺。然倉卒,舟不獲艤,溺死甚眾。俘大漢軍三千餘人,斬其帥完顏鄭家奴等六人,禽
 倪詢等上於朝,獲其統軍符印與文書、器甲、糧斛以萬計。餘物眾不能舉者,悉焚之,火四晝夜不滅。



 寶將乘勢席卷,公佐切諫,以為金主亮方濟淮,聞通、泰已陷,得遠失近,且有腹背憂。乃還軍駐東海,視緩急為表裏援。遣曹洋輕舟報捷。上喜曰:「朕獨用李寶,果立功,為天下倡矣。」詔獎諭,書「忠勇李寶」四字,表其旗幟。除靜海軍節度使、沿海制置使,賜金器、玉帶。



 亮聞膠西之敗,大怒,召諸酋約以三日渡江,於是內變殺亮。向微唐島之捷,則亮
 之死未可期,錢唐之危可憂也。寶之功亦大矣。



 寶戰具精利,宰臣陳康伯取其長槍、克敵弓弩,俾所司為式制之。卒,贈檢校少保。



 成閔,字居仁,邢州人。靖康初,劉韐為真定帥,募勇士捍金兵,閔在麾下。高宗即位,閔領數百騎至揚州。會上南渡,韓世忠追苗傅及襲兀朮、討範汝為,閔皆在戎行,又以力戰卻敵,積功至武功大夫、忠州刺史。



 從世忠入見,世忠指閔曰:「臣在南京,自謂天下當先,使當時見此人,
 亦避一頭矣。」上嘉嘆勞勉。旋以取海州功,擢磁州團練使。召見,賜袍帶、錦帛,加贈玉束帶。時方與金盟,世忠罷兵,入為樞密使,詔進閔棣州防禦使、殿前游奕軍統制,歷遷保寧軍承宣使。



 紹興二十四年,拜慶遠軍節度使。尋丁母憂,詔起復,贈其母鄭國夫人。金主亮將敗盟,詔閔提禁旅三萬鎮武昌,命湖北守、漕創砦屋三萬間以待之,發折帛米錢茶引共百四十餘萬緡、義倉和糶米六十三萬石備軍用,仍賜金器、劍甲臨遣之。閔至鄂,未
 幾,進屯應城縣。



 八月,除湖北、京西制置使,節制兩路軍馬。九月,兼京西、河北招討使。十一月,詔回援淮西。閔喜於得歸,冒雨兼程趨建康,士卒多道死,朝廷所給犒師物奄歸己,不及士卒。士卒有怨言,閔斬之。未幾,除淮東制置使,駐鎮江。既而言者論諸軍皆聚鎮江,恐敵出不意搗上流,於是詔閔發鄂州張成、華旺軍回駐鄂。



 亮死,閔引兵渡江趨揚州。及金人自盱眙渡淮北去,閔列兵南岸,軍士喏聲相聞。金人笑之曰:「寄聲成太尉,有勤護
 送。」時虜氣已奪,日虞王師之至,委棄戈甲、粟米山積,諸軍多仰以給。惟閔軍多浙人,素不食粟,死者甚眾。



 閔至泗州,奏已克復淮東。尋入朝,,凡侍從、卿監、閣門、內侍,皆有賂遺。左正言劉度劾之,猶超拜太尉,主管殿前司公事。尋復為御史論列,罷太尉,婺州居住,奪慶遠節。乾道初,聽自便,歸湖州;尋詔復節,都統鎮江諸軍。九年,請詞,致仕,治園第於平江。



 淳熙元年卒,年八十一。贈開府儀同三司。子十一人。



 趙密字微叔,太原清源人。政和四年,用材武試崇政殿,授河北隊將,戍燕。高宗以大元帥開府,檄統先鋒援京師。



 建炎元年,從張俊討任城寇李昱,俊輕騎先行,遇伏,密奔射斃數人,乃脫。擢閣門祗候。俊置靖勝軍,以密統之。平賊董青、越萬、徐明等,累功轉武節郎、左軍統領。金兵陷揚州,士民隨乘輿渡江,眾數萬,密露立水濱,麾舟濟之。苗傅之變,破赤心軍於臨平。金人犯明州,俊遣密及楊沂中與殊死戰,敗之,進武功大夫,升統制。



 紹興元
 年,李成、馳進擾江、淮,俊復遣密大破之,成、進皆北遁。賜金帶,轉親衛大夫、康州刺史,總管涇原馬步軍。平張莽蕩,尋詔入衛。十年,金犯亳、宿,從俊營合肥,出西路。時水潦暴漲,涉六晝夜始達宿,與敵遇,敗之。



 明年,敵分兵犯滁、濠,密進擊之,且命張守忠以五百騎出全椒縣,伏篁竹間,敵疑,宵遁。密乃引兵出六丈河,斷其歸路,又敗之。進中衛、協忠大夫,和州團練、防禦使。尋拜宣州觀察使,為龍、神衛四廂都指揮使,主管侍衛步軍。



 海寇朱明暴
 橫,密授張守忠方略曰:「海與陸異,窮之則日月相持,非策之善,要在拊定之耳。」守忠用其計,明降。進定江軍承宣使、崇信軍節度使,以年勞轉太尉,拜開府儀同三司。明年,領殿前都指揮使,獻本軍酒方十六所,積錢十萬緡、銀五萬兩助軍用,詔獎之。上疏告老,以萬壽觀使奉朝請。



 隆興二年,進少保致仕。俄報金復犯淮,詔密再為殿前都指揮使。初,敵聲言航海,朝論選從官視舟師,徹禁旅防守,密不為動,迄如所料。和議成,罷為醴泉使。



 乾道元年九月,致仕。卒,年七十一。贈少傅。



 劉子羽,字彥修,建之崇安人,資政殿學士韐之長子也。宣和末,韐帥浙東,子羽以主管機宜文字佐其父。破睦賊,入主太府、太僕簿,遷衛尉丞。韐守真定,子羽闢從。會金人入,父子相誓死守,金人不能拔而去,由是知名。除直秘閣。京城不守,韐死之,既免喪,除秘閣修撰、知池州。



 以書抵宰相,論天下兵勢,當以秦、隴為根本。改集英殿修撰、知秦州。未行,召赴行在,除樞密院檢詳文字。



 建炎
 三年,大將範瓊擁強兵江西,召之弗來,來又不肯釋兵。知樞密院事張浚,與子羽密謀誅之。一日,命張俊以千兵渡江,若備他盜者,使皆甲而來。因召俊、瓊及劉光世赴都堂議事,為設飲食,食已,諸公相顧未發。子羽坐廡下,恐瓊覺,取黃紙趨前,舉以麾瓊曰:「下,有敕,將軍可詣大理置對。」瓊愕不知所為,子羽顧左右擁置輿中,衛以俊兵,送獄。光世出撫其眾,數瓊在圍城中附金人迫二帝出狩狀。且曰:「所誅止瓊爾,汝等固天子自將之兵也。」
 眾皆投刃曰:「諾。」有旨分隸御營五軍,頃刻而定。瓊竟伏誅。浚以此奇其材。



 浚宣撫川、陜,闢子羽參議軍事。至秦州,立幕府,節度五路諸將,規以五年而後出師。明年,除徽猷閣待制。金人窺江、淮急,浚念禁衛寡弱,計所以分撓其兵勢者,遂合五路之兵以進。子羽以非本計,爭之。浚曰:「吾寧不知此?顧今東南之事方急,不得不為是耳。」遂北至富平,與金人遇,戰不利。金人乘勝而前,宣撫司退保興州,人情大震。



 官屬有建策徙治夔州者,子羽叱
 之曰:「孺子可斬也!四川全盛,敵欲入寇久矣,直以川口有鐵山、棧道之險,未敢遽窺耳。今不堅守,縱使深入,而吾僻處夔、峽,遂與關中聲援不相聞,進退失計,悔將何及。今幸敵方肆掠,未逼近郡。宣司但當留駐興州,外系關中之望,內安全蜀之心;急遣官屬出關,呼召諸將,收集散亡,分布險隘,堅壁固壘,觀釁而動。庶幾猶或可以補前愆而贖後咎,奈何乃為此言乎?」浚然子羽言,而諸參佐無敢行者。子羽即自請奉命北出,復以單騎至秦
 州,召諸亡將。諸亡將聞命大喜,悉以其眾來會。子羽命吳玠柵和尚原,守大散關,而分兵悉守諸險塞。金人知有備,引去。



 明年,金人復聚兵來攻,再為玠所敗。浚移治閬州,子羽請獨留河池,調護諸將,以通內外聲援,浚許之。明年,玠以秦鳳經略使戍河池,王彥以金、均、房鎮撫使戍金州。二鎮皆饑,興元帥臣閉糴,二鎮病之。玠、彥皆願得子羽守漢中,浚乃承制拜子羽利州路經略使兼知興元府。子羽至漢中,通商輸粟,二鎮遂安。除寶文閣
 直學士。



 是冬,金人犯金州。三年正月,王彥失守,退保石泉。子羽亟移兵守饒風嶺,馳告玠。玠大驚,即越境而東,日夜馳三百里至饒風,列營拒守。金人悉力仰攻,死傷山積,更募死士,由間道自祖溪關入,繞出玠後。玠遽邀子羽去,子羽不可,而留玠同守定軍山,玠難之,遂西。



 子羽焚興元,退守三泉縣,從兵不滿三百,與士卒取草牙、木甲食之,遺玠書訣別。玠時在仙人關,其愛將楊政大呼軍門曰:「節使不可負劉待制,不然,政輩亦舍節使去
 矣。」玠乃間道會子羽,子羽留玠共守三泉。玠曰:「關外蜀之門戶,不可輕棄。」復往守仙人關。子羽以潭毒山形斗拔,其上寬平有水,乃築壁壘,十六日而成。金人已至,距營十數里。子羽據胡床,坐於壘口。諸將泣告曰:「此非待制坐處。」子羽曰:「子羽今日死於此。」敵尋亦引去。



 自金人入梁、洋,四蜀復大震。張浚欲移潼川,子羽遺浚書,言己在此,金人必不南,浚乃止。撒離曷由斜谷北去,子羽謀邀之於武休,不及,既回鳳翔,遣十人持書旗招子羽,子
 羽盡斬之,而留其一,縱之還,曰:「為我語賊,欲來即來,吾有死爾,何可招也!」先是,子羽預徙梁、洋公私之積,至是,金人深入,饋不繼,又腹背為子羽、玠所攻,死傷十五六,疫癘且作,亟遁去。子羽出師掩擊,墮溪澗死者不可勝計,餘兵不能自拔者,悉降。



 始,金人攻蜀,所選士卒千取百,百取十;戰被重鎧,登山攻險,每一人前,輒二人推其後,前者死,後者被其甲以進,又死,則又代之,其為必取計如此。浚雖衄師,卒全蜀,子羽之力居多。子羽還興元。
 四年,坐富平之役,與浚俱罷。尋為言者所論,責授單州團練副使,白州安置。



 新除川、陜宣撫副使吳玠,始為裨將,未知名。子羽獨奇之,言於浚,浚與語大悅,使盡護諸將。至是,上疏論子羽之功,請納節贖其罪。詔聽子羽自便。明年,復元官,提舉江州太平觀。



 張浚還朝,議合兵大舉,乃請召子羽,令諭旨西帥,以集英殿修撰知鄂州。未幾,權都督府參議軍事,與主管機宜文字熊彥詩同撫諭川、陜。時吳玠屢言軍前乏糧,故令子羽見玠諭指,且
 與都轉運使趙開計事,並察邊備虛實以聞,時五年冬也。明年秋,與彥詩同還朝。子羽言:「金人未可圖,宜益兵屯田,以俟機會。」時張浚以淮西安撫使劉光世驕惰不肅,密奏請罷之,而以其兵屬子羽。子羽辭,乃以徽猷閣待制知泉州。



 七年,淮西酈瓊叛,張浚罷相。八年,御史常同論子羽十罪,上批出「白州安置」。趙鼎曰:「章疏中論及結吳玠事,今方倚玠,恐不自安。」同疏再上,以散官安置漳州。十一年,樞密使張浚薦子羽復元官,知鎮江府兼
 沿江安撫使。金人入寇,子羽建議清野,淮東之人,皆徙鎮江,撫以恩信,雖兵民雜居,無敢相侵者。既而金人不至,浚問子羽,子羽曰:「異時金人入寇,飄忽如風雨,今久遲回,必有他意。」蓋金人以柘皋之敗,欲急和也。未幾,果遣使議和。復徽猷閣待制。秦檜風諫官論罷之,復提舉太平觀。



 十六年,卒。子珙,自有傳。吏部郎朱松以子熹托子羽,子羽與弟子翬篤教之,異時卒為大儒云。



 呂祉,字安老,建州建陽人。宣和初,上舍釋褐。建炎二年,
 為右正言,以論事忤執政,通判明州。



 紹興元年,盜起湖南、北,為荊湖提刑。祉既至,招捕有方,逾年盜平。進直秘閣,尋召赴行在。淮南宣撫使韓世忠將出師,闢祉議軍事,除直徽猷閣,充參議官,辭不行。



 三年,升直龍圖閣、知建康府。祉到官,與通判府事吳若、安撫司準備差遣陳充共議,作《東南防守利便》三卷上之,大略謂:「立國於東南者,當聯絡淮甸、荊、蜀之勢,今臨安僻在海隅,移蹕江上,然後可以系南北離散之心。」



 四年冬,金人攻淮,江左
 戒嚴,獨韓世忠統銳卒在高郵。金既陷漣水,破山陽、盱眙,遂犯承州。祉上章言:「宜遣兵為世忠援。」既而援兵不至,世忠退保鎮江。祉再上言:「置江北于度外,非命帥宣撫兩淮之意,且恐失中原心。唯當急遣諸將,且乞親御六師,庶幾上下協心,可以不戰而勝。」於是降詔親征。車駕至平江,金人退師。



 五年,召為中書門下省檢正諸房文字,尋除兵部侍郎兼戶部侍郎、給事中。六年,遷刑部侍郎、都督府參議軍事,俄遷吏部侍郎。劉豫分道入寇,
 時車駕駐平江,或請回臨安,且令守江防海。祉獨抗言:「士氣當振,賊鋒可挫,不可遽退以示弱。」劉麟眾十萬,已次濠、壽。劉光世在合肥,欲移屯太平州,軍已行,乃命祉馳往軍前,督其還。七年,遷兵部尚書,升督府參謀軍事,往淮西撫諭諸軍。



 浚以劉光世持不戰之論,罷之,乃命行營左護軍前統制王德為都統制,又以統制官酈瓊為之副。瓊與德素不協,祉還朝,瓊與德交訟於都督府及御史臺,乃命德還建康,以其軍隸督府。八月,復命祉
 往廬州節制之。祉至廬州,瓊等復訟德。祉諭之曰:「若以君等為是,則大相誑。然張丞相但喜人向前,儻能立功,雖有大過亦闊略,況此小嫌乎?當力為諸公辨之,保無他慮。」瓊等感泣。



 事小定,祉乃密奏乞罷瓊及統制官靳賽兵權。其書吏漏語於瓊,瓊令人遮祉所遣郵置,盡得祉所言,大怨怒。會朝廷命張俊為淮西宣撫使,置司盱眙;楊存中為淮西制置使,劉錡為副,置司廬州;召瓊赴行在。瓊懼,遂叛。諸將晨謁祉,坐定,瓊袖出文書,示中軍
 統制官張璟曰:「諸兵官有何罪,張統制乃以如許事聞之朝廷邪?」祉見之大驚,欲返走,不及,為瓊所執。璟及兵馬鈐轄喬仲福,統制劉永、衡友死之。瓊遂率全軍四萬人渡淮降劉豫,擁祉次三塔,距淮三十里。祉下馬曰:「劉豫逆臣,我豈可見之?」眾逼祉上馬,祉罵曰:「死則死於此!」又語其眾曰:「劉豫逆臣,爾軍中豈無英雄,乃隨酈瓊去乎?」眾頗感動,凡千餘人環立不行。瓊恐搖動眾心,急策馬先渡,祉遇害。



 時有得祉括發之帛歸吳中者,其妻吳
 氏持帛自縊以徇葬,聞者哀之。慶元間,詔立廟賜額,以旌其忠云。



 胡世將,字承公,常州晉陵人,宿之曾孫。登崇寧五年進士第。範汝為寇閩,以世將為監察御史、福建路撫諭使。入境,韓世忠已平賊。遷尚書右司員外郎,又遷起居郎,遷中書舍人,賜三品服,兼修政局。坐言者落職奉祠。未幾,除徽猷閣待制、知鎮江府,入為禮部侍郎,改刑部,出知洪州,兼江西安撫、制置使。屬建昌兵變,殺守卒,嬰城
 以叛,世將以便宜發兵討平之。除兵部侍郎,復知鎮江。



 未幾,召為給事中兼侍講,直學士院,復遷兵部侍郎。尋以樞密直學士出為四川安撫制置使,兼知成都府。宣撫吳玠以軍無糧,奏請踵至。世將既被命入境,約玠會議。蜀之餉運,溯嘉陵江千餘里,半年始達。於是奏用轉般折運之法,軍儲稍充,公私便之。



 紹興九年,玠卒,以世將為寶文閣學士、宣撫川、陜。時關陜初復,朝廷分軍移屯熙、秦、鄜延諸道。明年夏,金人陷同州,入長安,諸路皆
 震。蜀兵既分,聲援幾絕,乃遣大將吳璘、田晟出鳳翔,郭浩出奉天,楊政由赤穀歸河池。不數日,璘捷於石壁及扶風,金人逡巡不敢度隴,分屯之軍得全師而還。詔除端明殿學士。



 十一年秋,朝廷復用兵。會母喪,命起復。遂復隴州,破岐下諸屯,又取華、虢,兵威稍振。未幾,瘍發於首。除資政殿學士致仕,恩數視簽書樞密院事。卒,年五十八,命有司給葬事。



 鄭剛中,字亨仲,婺州金華人。登進士甲科,累官為監察
 御史,遷殿中侍御史。剛中由秦檜薦於朝,檜主和議,剛中不敢言。移宗正少卿,請去,不許,改秘書少監。



 金歸侵疆,檜遣剛中為宣諭司參謀官;及還,除禮部侍郎。復遣剛中為川、陜宣諭使,諭諸將罷兵,尋充陜西分畫地界使。金使烏陵贊謨入境,欲盡取階、成、岷、鳳、秦、商六州,剛中力爭不從;又欲姑取商、秦,於大散關立界,剛中又堅不從。繼除川、陜宣撫副使。



 兀朮遣人力求和尚原,剛中恐敗和好,以和尚原自紹興四年後不系吳地分,於
 是割秦、商之半,棄和尚原以與金。朝廷命剛中去「陜」字,為四川宣撫副使。剛中治蜀,頗有方略。宣撫司舊在綿、閬間,及胡世將代吳玠,就居河池,饋餉不繼。剛中奏:利州在潭毒關內,與興、洋諸關聲援相接,乞移司利州。自是省費百萬。剛中始至,即欲移屯一軍,大將楊政不從,呼政語之曰:「剛中雖書生,不畏死!」聲色俱厲,政即聽命。



 都統每入謁,必庭參然後就坐。吳璘升檢校少師來謝,語閽吏,乞講鈞敵之禮。剛中曰:「少師雖尊,猶都統制耳,
 儻變常禮,是廢軍容。」行禮如故。



 奏蠲四川雜征,又請減成都府路對糴及宣撫司激賞錢。時剛中於階、成二州營田,抵秦州界,凡三千餘頃,歲收十八萬斛。先是,川口屯兵十萬,分隸三大將:吳璘屯興州,楊政屯興元府,郭浩屯金州,皆建帥節;而統制官知成州王彥、知階州姚仲、知西和州程俊、知鳳州楊從儀亦領沿邊安撫。剛中請分利州為東、西路,以興元府、利閬洋巴劍州、大安軍七郡為東路,治興元,命政為安撫;以興、階、成、西和、文、隴、
 鳳七州為西路,治興州,命璘為安撫;而命浩為金、房、開、達州安撫;諸裨將領安撫者皆罷。從之。弛夔路酒禁,復利州錢監為紹興監。時軍已罷,移屯內郡,剛中言逐路各有漕司,都漕宜罷。從之。



 秦檜怒剛中在蜀專擅,令侍御史汪勃奏置四川財賦總領官,以趙不棄為之,不隸宣撫司。不棄牒宣撫司,剛中怒,由是有隙。不棄頗求剛中陰事言於檜,檜陽召不棄歸,因召剛中。剛中語人曰:「孤危之跡,獨賴上知之耳。」檜聞愈怒,遂罷,責桂陽軍居
 住;再責濠州團練副使,復州安置;再徙封州,卒。



 論曰:自紹興和議成,材武善謀之士,無所用其力。若王友直之矯制起兵,李寶之立功膠西,成閔、趙密皆足以斬將搴旗,劉子羽轉戰屢勝,呂祉不從劉豫,胡世將、鄭剛中威震巴蜀。皆中道以歿,是以知宋不克興復也。



\end{pinyinscope}