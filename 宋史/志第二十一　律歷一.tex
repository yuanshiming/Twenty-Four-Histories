\article{志第二十一 律歷一}

\begin{pinyinscope}

 應天 乾元 儀天曆



 古者,帝王之治天下,以律曆為先;儒者之通天人,至律曆而止。曆以數始,數自律生,故律曆既正,寒暑以節,歲功以成,民事以序,庶績以凝,萬事根本由茲立焉。古
 人自入小學,知樂知數,已曉其原。後世老師宿儒猶或弗習律曆,而律曆之家未必知道,各師其師,岐而二之。雖有巧思,豈能究造化之統會,以識天人之蘊奧哉!是以審律造曆,更易不常,卒無一定之說。治效之不古若,亦此之由,而世豈察及是乎!



 宋初,承五代之季王朴制律曆、作律準,以宣其聲,太祖以雅樂聲高,詔有司考正。和峴等以影表銅臬暨羊頭秬黍累尺制律,而度量權衡因以取正。然累代尺度與望臬殊,黍有鉅細,縱橫容積,
 諸儒異議,卒無成說。至崇寧中,徽宗任蔡京,信方士「聲為律、身為度」之說,始大盭乎古矣。



 顯德欽天曆亦朴所制也,宋初用之。建隆二年,以推驗稍疏,詔王處訥等別造新曆。四年,曆成,賜名應天,未幾,氣候漸差。太平興國四年,行乾元曆,未幾,氣候又差。繼作者曰儀天,曰崇天,曰明天,曰奉元,曰觀天,曰紀元,迨靖康丙午,百六十餘年,而八改曆。南渡之後,曰統元,曰乾道,曰淳熙,曰會元,曰統天,曰開禧,曰會天,曰成天,至德祐丙子,又百五十
 年,復八改曆。使其初而立法吻合天道,則千歲日至可坐而致,奚必數數更法,以求幸合玄象哉!蓋必有任其責者矣。



 雖然,天步惟艱,古今通患,天運日行,左右既分,不能無忒。謂七十九年差一度,雖視古差密,亦僅得其概耳。又況黃、赤道度有斜正闊狹之殊,日月運行有盈縮、朏朒、表裡之異。測北極者,率以千里差三度有奇,晷景稱是。古今測驗,止於岳臺,而岳臺豈必天地之中?餘杭則東南,相距二千餘里,華夏幅員東西萬里,發斂晷
 刻豈能盡諧?又造曆者追求曆元,踰越曠古,抑不知二帝授時齊政之法,畢殫於是否乎?是亦儒者所當討論之大者,諉曰星翁曆生之責可哉?至於儀象推測之具,雖亦數改,若熙寧沈括之議、宣和璣衡之制,其詳密精緻有出於淳風、令瓚之表者,蓋亦未始乏人也。今其遺法具在方冊,惟奉元、會天二法不存。舊史以乾元、儀天附應天,今亦以乾道、淳熙、會元附統元,開禧、成天附統天。大抵數異術同,因仍增損,以追合乾象,俱無以大相
 過,備載其法,俾來者有考焉。



 昔黃帝作律呂,以調陰陽之聲,以候天地之氣。堯則欽若曆象,以授人時,以成歲功,用能綜三才之道,極萬物之情,以成其政化者也。至司馬遷、班固敘其指要,著之簡策。自漢至隋,歷代祖述,益加詳悉。暨唐貞觀迄周顯德,五代隆替,踰三百年,博達之士頗亦詳緝廢墜,而律志皆闕。宋初混一○內,能士畢舉,國經王制,悉復古道。漢志有備數、和聲、審度、嘉量、權衡之目,後代因之,今亦
 用次序以志於篇:



 曰備數。周禮,保氏教國子以六藝,其六曰九數,謂方田、粟米、差分、少廣、商功、均輸、方程、贏朒、旁要,是為九章。其後又有海島、孫子、五曹、張丘建、夏侯陽、周髀、綴術、緝古等法相因而起,歷代傳習,謂之小學。唐試右千牛衛胄曹參軍陳從運著得一算經,其術以因折而成,取損益之道,且變而通之,皆合於數。復有徐仁美者,作增成玄一法,設九十三問,以立新術,大則測於天地,細則極於微妙,雖粗述其事,亦適用於
 時。古者命官屬於太史,漢、魏之世,皆在史官。隋氏始置算學博士於國庠,唐增其員,宋因而不改。



 曰和聲。周禮,典同掌六律六同之和,凡為樂器,以十有二律為之數度。古之聖人推律以製器,因器以宣聲,和聲以成音,比音而為樂。然則律呂之用,其樂之本歟!以其相生損益,數極精微,非聰明博達,則罕能詳究。故歷代而下,其法或存或闕,前史言之備矣。周顯德中,王樸始依周法,以秬黍校正尺度,長九寸,虛徑三分,為黃鐘之管,作律準,以宣
 其聲。宋乾德中,太祖以雅樂聲高,詔有司重加考正。時判太常寺和峴上言曰:「古聖設法,先立尺寸,作為律呂,三分損益,上下相生,取合真音,謂之形器。但以尺寸長短非書可傳,故累秬黍求為準的,後代試之,或不符會。西京銅望臬可校古法,即今司天臺影表銅臬下石尺是也。及以樸所定尺比校,短於石尺四分,則聲樂之高,蓋由於此。況影表測於天地,則管律可以準繩。」上乃令依古法,以造新尺并黃鐘九寸之管,命工人校其聲,果下
 於樸所定管一律。又內出上黨羊頭山秬黍,累尺校律,亦相符合。遂下尚書省集官詳定,眾議僉同。由是重造十二律管,自此雅音和暢。



 曰審度者,本起於黃鐘之律,以秬黍中者度之,九十黍為黃鐘之長,而分、寸、尺、丈、引之制生焉。宋既平定四方,凡新邦悉頒度量於其境,其偽俗尺度踰於法制者去之。乾德中,又禁民間造者。由是尺度之制盡復古焉。



 曰嘉量。周禮,○氏為量。漢志云,物有多少受以量,本起於黃鐘之管容秬黍千二百,而
 龠、合、升、斗、斛五量之法備矣。太祖受禪,詔有司精考古式,作為嘉量,以頒天下。其後定西蜀,平嶺南,復江表,泉、浙納土,并、汾歸命,凡四方斗、斛不中式者皆去之。嘉量之器,悉復昇平之制焉。


曰權衡之用,所以平物一民、知輕重也。權有五,曰銖、兩、斤、鈞、石,前史言之詳矣。建隆元年八月,詔有司按前代舊式作新權衡,以頒天下,禁私造者。及平荊湖,即頒量、衡於其境。淳化三年三月三日,詔曰:「書云:『協時、月,正日,同律、度、量、衡。』所以建國經而立
 民極也。國家萬邦咸乂,九賦是均,顧出納於有司,繫權衡之定式。如聞秬黍之制,或差毫釐,錘鈞為姦,害及黎庶。宜令詳定稱法,著為通規。」事下有司,監內藏庫、崇儀使劉承珪言:「太府寺舊銅式自一錢至十斤,凡五十一,輕重無準。外府歲受黃金,必自毫釐計之,式自錢始,則傷於重。」遂尋究本末,別製法物。至景德中,承珪重加參定,而權衡之制益為精備。其法蓋取漢志子穀秬黍為則,廣十黍以為寸,從其大樂之尺,
 \gezhu{
  秬黍,黑黍也。樂尺,自黃鐘之管而生也。
  謂以秬黍中者為分寸、輕重之制。}
 就成二術,
 \gezhu{
  二術謂以尺、黍而求氂、絫。}
 因度尺而求氂,
 \gezhu{
  度者,丈、尺之總名焉。因樂尺之源,起於黍而成於寸,析寸為分,析分為氂,析氂為毫,析毫為絲,析絲為忽。十忽為絲,十絲為毫,十毫為氂,十氂為分。}
 自積黍而取絫。
 \gezhu{
  從積黍而取絫,則十黍為絫,十絫為銖,二十四銖為兩。錘皆以銅為之。}
 以氂、絫造一錢半及一兩等二稱,各懸三毫,以星準之。等一錢半者,以取一稱之法。其衡合樂尺一尺二寸,重一錢,錘重六分,盤重五分。初毫星準半錢,至稍總一錢半,析成十五分,分列十氂;
 \gezhu{
  第一毫下等半錢,當五十氂,若十五斤稱等五斤也。}
 中毫至稍一錢,析成十分,分列十氂;
 末毫至稍半錢,析成五分,分列十氂。等一兩者,亦為一稱之則。其衡合樂分尺一尺四寸,重一錢半,錘重六錢,盤重四錢。初毫至稍,布二十四銖,下別出一星,等五絫;
 \gezhu{
  每銖之下,復出一星,等五絫,則四十八星等二百四十絫,計二千四百絫為十兩。}
 中毫至稍五錢,布十二銖,列五星,星等二絫;
 \gezhu{
  布十二銖為五錢之數,則一銖等十絫,都等一百二十絫為半兩。}
 末毫至稍六銖,銖列十星,星等絫。
 \gezhu{
  每星等一絫,都等六十絫為二錢半。}
 以御書真、草、行三體淳化錢,較定實重二銖四絫為一錢者,以二千四百得十有五斤為一稱之則。其法,
 初以積黍為準,然後以分而推忽,為定數之端。故自忽、絲、毫、氂、黍、絫、銖各定一錢之則。
 \gezhu{
  謂皆定一錢之則,然後制取等稱也。}
 忽萬為分,
 \gezhu{
  以一萬忽為一分之則,以十萬忽定為一錢之則。忽者,吐絲為忽;分者,始微而著,言可分別也。}
 絲則千,
 \gezhu{
  一千絲為一分,以一萬絲定為一錢之則。}
 毫則百,
 \gezhu{
  一百毫為一分,以一千毫定為一錢之則。毫者,毫毛也。自忽、絲、毫三者皆斷驥尾為之。}
 氂則十,
 \gezhu{
  一十氂為一分,以一百氂定為一錢之則。氂者,氂牛尾毛也,曳赤金成絲為之也。}
 轉以十倍倍之,則為一錢。
 \gezhu{
  轉以十倍,謂自一萬忽至十萬忽之類定為則也。}
 黍以二千四百枚為一兩,
 \gezhu{
  一龠容千二百黍為十二銖,則以二千四百黍定為一兩之則。兩者,以二龠為兩。}
 絫以二百四十,
 \gezhu{
  謂以二百四十絫定為一兩之
  則。}
 銖以二十四,
 \gezhu{
  轉相因成絫為銖,則以二百四十絫定成二十四銖為一兩之則。銖者,言殊異。}
 遂成其稱。稱合黍數,則一錢半者,計三百六十黍之重。列為五分,則每分計二十四黍。又每分析為一十氂,則每氂計二黍十分黍之四。
 \gezhu{
  以十氂分二十四黍,則每氂先得二黍。都分成四十分,則一絫又得四分,是每氂得二黍十分黍之四。}
 每四毫一絲六忽有差為一黍,則氂、絫之數極矣。一兩者,合二十四銖為二千四百黍之重。每百黍為銖,二百四十黍為絫,二銖四絫為錢,二絫四黍為分。一絫二黍重五氂,六黍重二氂五毫,三
 黍重一氂二毫五絲,則黍、絫之數成矣。其則,用銅而鏤文,以識其輕重。新法既成,詔以新式留禁中,取太府舊稱四十、舊式六十,以新式校之,乃見舊式所謂一斤而輕者有十,謂五斤而重者有一。式既若是,權衡可知矣。又比用大稱如百斤者,皆懸鈞於架,植環於衡,鐶或偃,手或抑按,則輕重之際,殊為懸絕。至是,更鑄新式,悉由黍、絫而齊其斤、石,不可得而增損也。又令每用大稱,必懸以絲繩。既置其物,則卻立以視,不可得而抑按。復鑄銅
 式,以御書淳化三體錢二千四百暨新式三十有三、銅牌二十授於太府。又置新式於內府、外府,復頒於四方大都,凡十有一副。先是,守藏吏受天下歲貢金帛,而太府權衡舊式失準,得因之為姦,故諸道主者坐逋負而破產者甚眾。又守藏更代,校計爭訟,動必數載。至是,新制既定,奸弊無所指,中外以為便。
 \gezhu{
  度、量、權、衡皆太府掌造,以給內外官司及民間之用。凡遇改元,即差變法,各以年號印而識之。其印面有方印、長印、八角印,明制度而防偽濫也。}



 宋初,用周顯德欽天曆,建隆二年五月,以其曆推驗稍疏,
 乃詔司天少監王處訥等別造曆法。四年四月,新法成,賜號應天曆。太平興國間,有上言應天曆氣候漸差,詔處訥等重加詳定。六年,表上新曆,詔付本監集官詳定。會冬官正吳昭素、徐瑩、董昭吉等各獻新曆,處訥所上曆遂不行。詔以昭素、瑩、昭吉所獻新曆,遣內臣沈元應集本監官屬、學生參校測驗,考其疏密。秋官正史端等言:「昭吉曆差。昭素、瑩二曆以建隆癸亥以來二十四年氣朔驗之,頗為切準。復對驗二曆,唯昭素曆氣朔稍均,
 可以行用。」又詔衛尉少卿元象宗與元應等,再集明曆術吳昭素、劉內真、苗守信、徐瑩、王熙元、董昭吉、魏序及在監官屬史端等精加詳定。象宗等言:「昭素曆法考驗無差,可以施之永久。」遂賜號為乾元曆。應天、乾元二曆皆御製序焉。



 真宗嗣位,命判司天監史序等考驗前法,研覈舊文,取其樞要,編為新曆。至咸平四年三月,曆成來上,賜號儀天曆。凡天道運行,皆有常度,曆象之術,古今所同。蓋變法以從天,隨時而推數,故法有疏密,
 數有繁簡,雖條例稍殊,而綱目一也。今以三曆參相考校,以應天為本,乾元、儀天附而注之,法同者不復重出,法殊者備列於後。



 建隆應天曆


演紀上元木星甲子,距建隆三年壬戌,歲積四百八十二萬五千五百五十八。
 \gezhu{
  乾元上元甲子距太平興國六年辛巳,積三千五十四萬三千九百七十七。儀天自上元土星甲子至咸平四年辛丑,積七十一萬六千四百九十七。}



 步氣朔


元法:一萬二。
 \gezhu{
  乾元元率九百四十。儀天宗法一萬一百。又總謂之日
  法。}


歲盈:二十六萬九千三百六十五。
 \gezhu{
  乾元歲周二十一萬四千七百六十四。儀天歲周三十六萬八千八百九十七。儀天有周天三百六十五、餘二千四百七十,約餘二千四百四十五;歲餘五萬二千九百七十、餘二千四百七十。應天、乾元無此法,後皆倣此。}


月率:五萬九千七十三。
 \gezhu{
  乾元不置此法。儀天合率二十九萬八千二百五十九。又儀天有歲閏一萬九千八百六十二,月閏九千一百一十五、秒六。}


會日:二十九、小餘五千三百七。
 \gezhu{
  乾元朔策二十九、小餘一千五百六十。儀天會日二十九、小餘五千三百五十七。}


弦策:七、小餘三千八百二十七、秒六。
 \gezhu{
  乾元小餘一千一百二
  十五。儀天小餘三千八百六十四、秒二十七。策並同。}


望策:十四、小餘七千六百五十四、秒一十二。
 \gezhu{
  乾元小餘二千二百五十七。儀天小餘七千七百二十七、秒一十八。策並同。}


氣策:十五、小餘二千一百八十五、秒二十四。
 \gezhu{
  乾元小餘六百四十二半。儀天小餘二千二百七、秒三。策並同。又儀天有氣盈四千四百一十四、秒六。}


朔虛分:四千六百九十五。
 \gezhu{
  乾元一千三百八十。儀天四千七百四十一。}


沒限:七千八百一十六、秒九。
 \gezhu{
  乾元二千二百九十七半。儀天七千八百九十二。又儀天有紀實六十萬六千。}


秒法:二十四。
 \gezhu{
  乾元一百。儀天秒母三十六。}
 紀法:六十。
 \gezhu{
  二曆同。}
 推元積:
 \gezhu{
  乾元、儀天皆謂之求歲積分。}
 置所求年,以歲盈展之為元積。


求天正所盈之日及分并冬至大小餘:以八十四萬一百六十八去元積,不盡者,半而進位,以元法收為所盈日,不滿為小餘。日滿六十去之,不滿者,命從甲子,算外,即冬至日辰、大小餘也。
 \gezhu{
  乾元以歲周乘積年為歲積分,以七萬五百六十去之,不盡,以五因,滿元率收為日,不滿為餘日。儀天以歲周乘積年,進一位,為歲積分;盈宗法而一為積日,不滿為餘日。去
  命並同應天。}


求次氣:以天正冬至大、小餘遍加諸常數,盈六十去之,不盈者,命如前,即得諸氣日辰、大小餘秒也。
 \gezhu{
  乾元置中氣大、小餘,以氣策加之,命如前,即次氣日辰也。儀天置冬至大、小餘,加氣策及餘秒,秒盈秒母從小餘,盈紀法去之,皆命如前法,各得次氣常日辰及餘秒。}


求天正十一月朔中日:
 \gezhu{
  乾元謂之經朔。儀天謂之天正合朔。}
 以月率去元積,不盡者,為天正十一月通餘;以通餘減七十三萬六百三十五,餘,半而進位,以元法收為日,不滿為分,即得
 所求天正十一月朔中日及餘秒。
 \gezhu{
  乾元以一萬七千三百六十四去歲積分,不盡為朔餘;以歲積分為朔積分,又倍五萬二千九百二十,除之,餘以五因,滿元率為日,不滿為分。儀天以合率去歲積分,不盡為閏餘;滿宗法為閏日,不滿為餘,以閏日及餘減天正冬至大、小餘,為天正合朔大、小餘;去命如前,即得合朔日辰、大小餘。}


求次朔望中日:
 \gezhu{
  乾元謂之求弦望經朔。儀天謂之求次朔。}
 置朔中日,累加弦策餘秒,即得弦、望及次朔中日。
 \gezhu{
  乾元以弦策加經朔大、小餘,即得次朔經日;以弦策及餘秒加經朔,得上弦;再加,得望;三之,得下弦。}


求望中月:置朔中月,加半交,盈交正去之,餘為望中月。
 \gezhu{
  二曆不立此法。}


求朔弦望入氣:置朔、望中日,各以盈縮準去,不盡者,為入氣日及分。
 \gezhu{
  二曆不立此法。}


推沒日:置有沒之氣小餘,
 \gezhu{
  其小餘七千八百一十六、秒九以上者求之也。}
 返減元法,餘以八因之,一千九十二、秒一十九半除為沒日,命起氣初,即得沒日辰。其秒不足者,退一分,加二十四秒,然後除之,四分之三以上者進。
 \gezhu{
  乾元置有沒之氣小餘,在二千二百九十七半以上者,以十五乘之,用減四萬四千七百四十二半,餘以六百四十二半除為沒日。儀天以秒母通常氣
  小餘及秒,而從之以減歲周,餘滿五千二百九十七為沒日,去命如前。}


推滅日:以冬至大、小餘,遍加朔日中為上位,有分為下位,在四千六百九十五以下者,為有滅之分也。置有滅之分,進位,以一千五百六十五除為滅日,以滅日加上位,命從甲子,算外,即得月內滅日。
 \gezhu{
  乾元置有滅之經朔小餘,在一千一百八十以下者,以八因之,滿三百六十八除為滅日。儀天經朔小餘在朔虛法以下者,三因,進位,以朔虛分除為滅日。}



 求發斂


候策:五、小餘七百二十八、秒二,母二十四。
 \gezhu{
  乾元候數五、小餘一百一十四、秒十二,秒母七十二。儀天候率五、小餘七百三十五、秒二十五,秒母三十六。}


卦策:六、小餘八百七十四、秒六。
 \gezhu{
  乾元卦位六、小餘二百五十七,秒母六十。儀天卦率六、小餘八百八十三、秒二十。}


土王策:十二、小餘一千七百四十八、秒一十二。
 \gezhu{
  乾元策三、小餘一百二十八半,秒母一百一十。儀天土王率三、小餘四百四十、秒五,秒母同上。}


辰數:八百三十三半。
 \gezhu{
  乾元辰法二百四十五,辰率千五百二十。}
 刻法:一百。
 \gezhu{
  乾元一百四十七。儀天刻三百。}


求七十二候:各因諸氣大、小餘秒命之,即初候日也;各以候策加之,得次候日;又加之,得末候日。
 \gezhu{
  二曆同法。}


求六十四卦:各置諸中氣大、小餘秒命之,即公卦用事日;以卦策加之,得次卦用事日;又加之,得終卦用事日。十有二節之初,皆諸侯外卦用事日。
 \gezhu{
  二曆同法。}


求五行用事:各因四立大、小餘秒命之,即春木、夏火、秋金、冬水首用事日;以土王策加四季之節大、小餘秒,命從甲子,算外。即其月土王用事日。
 \gezhu{
  乾元以土王策減四季中
  氣大、小餘。儀天以土王率加四季大、小餘。}


求二十四氣加時辰刻:
 \gezhu{
  乾元謂之辰刻。儀天謂之求時。}
 各置小餘,以辰數除之為時數,不滿,百收為刻分,命起子正,算外,即所在。
 \gezhu{
  乾元時數同,其不盡,以五因之,以刻法除為刻分。儀天以三因小餘,以辰率除之為時數,不盡者,滿刻率除為刻,餘為分。}



 常數月中節四正卦初候中候末候始卦中卦末卦



 冬至十一月中坎初六蚯蚓結麋角解水泉動公 中孚辟 復侯 屯內



 小寒十二月節坎九二鴈北鄉鵲始巢雉始雊侯 屯外大夫 謙卿 睽



 大寒十二月中坎六三雞始乳鷙鳥厲疾水澤腹堅公 升辟 臨侯 小過內



 立春正月節坎六四東風解凍蟄蟲始振魚上冰侯 小過外大夫 蒙卿 益



 雨水正月中坎九五獺祭魚鴻鴈來草木萌動公 漸辟 泰侯 需內



 驚蟄二月節坎上六桃始華倉庚鳴鷹化為鳩侯 需外大夫 隨卿 晉



 春分二月中震初九玄鳥至雷乃發聲始電公 解辟 大壯侯 豫內



 清明三月節震六二桐始華田鼠化鴽虹始見侯 豫外大夫 訟卿 蠱



 穀雨三月中震六三萍始生鳴鳩拂羽戴勝降桑公 革辟 夬侯 旅內



 立夏四月節震九四螻蟈鳴蚯蚓出王瓜生侯 旅外大夫 師卿 比



 小滿四月中震六五苦菜秀靡草死小暑至公 小畜辟 乾侯 大有內



 芒種五月節震上六螗螂生鵙始鳴反舌無聲侯 大有外大夫 家人卿 井



 夏至五月中離初九鹿角解蜩始鳴半夏生公 咸辟 姤侯 鼎內



 小暑六月節離六二溫風至蟋蟀居壁鷹乃學習侯 鼎外大夫 豐卿 渙



 大暑六月中離九三腐草為螢土潤溽暑大雨時行公 履辟 遯侯 恆內



 立秋七月節離九四涼風至白露降寒蟬鳴侯 恆外大夫 節卿 同人



 處暑七月中離六五鷹乃祭鳥天地始肅禾乃登公 損辟 否侯 巽內



 白露八月節離上九鴻鴈來玄鳥歸羣鳥養羞侯 巽外大夫 萃卿 大畜



 秋分八月中兌初九雷乃收聲蟄蟲壞戶水始涸公 賁辟 觀侯 歸妹內



 寒露九月節兌九二鴻鴈來賓雀入水為蛤菊有黃花侯 歸妹外大夫 無妄卿 明夷



 霜降九月中兌六三豺乃祭獸草木黃落蟄蟲咸俯公 困辟 剝侯 艮內



 立冬十月節兌九四水始冰地始凍雉入大水為蜃侯 艮外大夫 既濟卿 噬嗑



 小雪十月中兌九三虹藏不見天氣上騰地氣下降閉塞成冬公 大過辟 坤侯 未濟內



 大雪十一月節兌上六鶡鳥不鳴虎始交荔挺出侯 未濟外大夫 蹇卿 頤二曆同



 求日躔


天總:七十三萬六百五十八、秒六十四。
 \gezhu{
  乾元軌率二十一萬四千七十
  七、秒七千五百一十、小分七十。儀天乾元數三百六十八萬九千八十八、秒九十九。}


天度:三百六十五、小餘二千五百六十三,微八十八。
 \gezhu{
  乾元周天三百六十五度、小餘二千五百六十三。儀天乾則三百六十五度、小餘二千五百八十八、秒九十九。應天諸法皆在天總數中。乾元、儀天各立其法。乾元周天策一百七萬三千八百五十三、秒七千五百五十三半,會周一萬七千三百六十四,會餘二十一萬四千七百六十四,天中一百八十二、六千二百八十一半。儀天歲差一百一十八、秒九十九,一象度九十一、餘三千一百四十二、秒五十,盈初縮末限分八十九萬七千六百九十九、秒五十,限日八十八、餘八千八百九十九、秒五十,縮初盈末限分九十四萬六千七百八十五、秒十五,限日九十三、餘七千四百八十五、秒五十,盈縮積二萬四千五百四十三,進退率一千八百三十六,秒母一百。}



 常氣盈縮準常數定日損益準先後積



 冬至十四五千四十五 秒十五十五二千一百八十五秒十五十四五千四十五 秒十五損六十四後二十



 小寒一十九一千二百八十六三十四千三百七十一十四六千二百三十六秒十五損六十九先五百二十九



 大寒四十三八千七百五 秒二十一四十五六千五百五十六秒二十一十四七千四百二十五秒十五損七十六先九百七十五



 立春五十八七千三百二十半六十八千七百四十二半十四八千六百一十六秒十五損八十二先一千三百三十五



 雨水七十三七千三百六十三七十六九百二十六十五四十二 秒十五損八十九先一千六百六



 驚蟄八十八八千八百三十四太九十一三千一百一十一太十五一千四百七十秒十五損九十七先一千七百七十一



 春分一百四一千三百三十三九一百六五千二百九十七秒九十五二千八百九十九秒十五益九十七先一千八百一十九



 清明一百十九六千六十一空一百二十一七千四百八十三空十五四千三百二十八秒十五益八十九先一千七百八十



 穀雨一百三十五一千八百一十五十五一百三十六六千六百六十八秒十五十五五千七百五十七秒十五益八十三先一千六百五



 立夏一百五十八千七百六十五 六一百五十二一千八百五十二秒六十五六千九百四十七秒十五益七十八先一千三百五十



 小滿一百六十六六千八百九十七二十一一百六十七四千三十一 秒二十十五八千一百三十六秒十五益七十二先九百九十五



 芒種一百八十二六千二百二十三半一百八十二六千二百三十三半十五九千三百七十二秒十五益六十六先五百四十一



 夏至一百九十八五千五百四十九三一百九十七八千四百九 秒三十五九千三百二十七秒十五損六十五先五



 小暑二百十四三千六百八十三 十八二百十三五百九十二太十五八千一百三十六秒十五損七十二後五百四十九



 大暑二百三十六百二十九九二百二十八二千七百七十八秒九十五八千一百三十六秒十五損七十七後九百八十五



 立秋二百四十五六千三百八十六空二百四十三四千九百六十四空十五五千七百五十六秒十五損八十三後一千三百四十六



 處暑二百六十一七百一十二 十五二百五十八七千二百四十九秒十五十五四千三百二十八秒十五損八十九後一千六百一十一



 白露二百七十六三千六百一十二六二百五十八七千一百四十九秒十五十五四千三百二十八秒十五損九十七後一千七百八十



 秋分二百九十一五千八十三 二十一二百八十九七千五百十八秒五十一十五益九十七後一千八百三十一



 寒露三百六五千一百二十六十二二百四三千七百四半十五四十二 秒十五益八十九後一千七百八十六



 霜降二百二十一三千四百四十一三三百一十九五千八百九十秒三十四八千六百一十六秒三益八十二後一千六百二十一



 立冬三百三十六一千六百六十四十六三百三十四八千七十五太十四七千四百二十五秒十五益七十五後一千三百五十七



 小雪三百五十七千四百十九三百五十三百九十九秒十五十四六千二百三十六秒十五益七十後九百八十八



 大雪三百六十五二千四百四十五三百六十五二千四百四十五十四五千四十五 秒十五益六十四後五百五十



 乾元二十四氣日躔陰陽度



 陰陽分 陰陽度損益率陰陽差



 冬至陽分二千二百七十六卷陽度空益一百七十陽差空



 小寒陽分一千七百八十四卷陽初度二千二百七十六卷益一百三十三卷陽差一百七十



 大寒陽分一千三百四十四卷陽一度一千一百二十益一百一陽差三百三



 立春陽分九百五十六陽一度二千四百六十四卷益七十一陽差四百四



 雨水陽分五百八十一陽二度四百八十益四十三陽差四百七十五卷



 驚蟄陽分二百九十三卷陽二度一千六十一益十四陽差五百一十八卷



 春分陽分一百九十四卷陽二度一千二百五十五卷損十四陽差五百三十二



 清明陽分五百八十一陽二度一千六十一損四十三陽差五百一十八卷



 穀雨陽分九百五十六卷陽二度四百八十損七十一陽差四百七十五卷



 立夏陽分一千三百四十四卷陽一度二千四百六十四卷損一百一陽差四百四



 小滿陽分一千七百八十四卷陽一度一千一百二十損一百三十三陽差三百三



 芒種陽分二千二百七十六卷陽初度二千二百七十六卷損一百七十陽差一百七十



 夏至陰分二千二百七十六卷陰度空益一百七十陰差空



 小暑陰分一千七百八十四卷陰度二千二百七十六卷益一百三十三卷陰差一百七十



 大暑陰分一千三百四十四卷陰一度一千一百二十益一百一陰差三百三



 立秋陰分九百五十六陰一度二千四百六十四卷益七十一陰差四百四



 處暑陰分五百八十一陰二度四百八十益四十三陰差四百七十五卷



 白露陰分一百九十四卷陰二度一千六十一益十四陰差五百一十八卷



 秋分陰分一百九十四卷陰二度一千二百五十五卷損十四陰差五百二十一



 寒露陰分五百八十一陰二度一千六十一損四十三陰差五百一十八卷



 霜降陰分九百五十六卷陰二度四百八十損七十一陰差四百七十五卷



 立冬陰分一千三百四十四卷陰一度二千四百六十四卷損百一陰差四百四



 小雪陰分一千七百八十四卷陰一度一千一百二十損一百三十三陰差三百三



 大雪陰分二千二百七十六卷陰初度二千二百七十六卷損一百七十陰差一百七十


\gezhu{
  應天、乾元二曆,以常氣求其陰陽差,故有二十四氣立成。儀天以盈縮定分、四限直求二十四氣陰陽差,乃更不制二十四氣差法。}


求日躔損益盈縮度:
 \gezhu{
  乾元謂之求每日陰陽差。儀天謂之求入盈縮分先後定數。}
 各置定日及分,以冬至常數相減,百收,通為分,自雨水後十六為法,自霜降後十五為法。除分為氣中率,二相減,
 為合差;半之,加減率為初、末率。
 \gezhu{
  後多者,減為初、加為末;後少者,加為初、減為末。}
 又法,以除合差,為日差;
 \gezhu{
  後少者,日損初率;後多者,日益初率。}
 為每日日躔損益率;累積其數,為盈縮度分。
 \gezhu{
  乾元各置氣數,以一百二十乘之,以一千八百二十六除之,所得為平行率;相減,為合差;初、末並如應天。儀天以宗法乘盈縮積,以其限分除之,為限率分;倍之,為未限平率;日分乘之,亦以限分除之,為日差;半之,加減初、末限平率,在初者減初加末,在末者減末加初,為末定率;乃以日差累加減限初定率,初限以減、末限以加,為每日盈縮定分;各隨其限盈加縮減其下先後數,為每日先後定數;冬至後積盈為先,在縮減之;夏至後,積縮為後,在盈減之。其進退率、昇平積準此求之,即各得其限每日
  進退率、昇平積也。}


求日躔先後定數:
 \gezhu{
  乾元謂之求入氣、求弦望氣入、求日躔陰陽差。}
 各以朔、弦、望入氣日及減本氣定日及分秒通之,下以損益率展,以元法為分,損減益加次氣下先後積為定數。
 \gezhu{
  乾元以其月氣節減經朔大、小餘,即得入氣日及分;又以弦策累加天正朔日入氣大、小餘,滿氣策去之,即得弦、望經朔入氣日及分;以其日損益率乘入氣日餘分,所得,用損益其日陰陽差為定數。儀天法見上。又儀天有求四正節定日,去冬、夏二至盈縮之中,先後皆空,以常為定;其春、秋二分盈縮之極,以一百乘盈縮積,滿宗法為日,先減後加,去命如前,各得定日。若求朔、弦、望盈縮限日,以天正閏日及餘減縮末限日及分,餘為天正十一月經朔加時入限日及餘;以弦策累加之,即得弦、望及後朔初、末限日;各置入限日及餘,以其日進退率乘之,如宗法而一,所得,
  以進退其日下昇平積,即各為定數。}


赤道宿度斗:二十六。牛:八。女:十二。虛:十。
 \gezhu{
  及分。}
 危:十七。室:十六。壁:九。
 \gezhu{
  二曆同。}


北方七宿九十八度。虛分二千五百六十三,秒一十九。
 \gezhu{
  乾元七千五百三十五、秒二十五。儀天二千五百八十八、秒九十九。}


奎:十六。婁:十二。胃:十四。昴:十一。畢:十七。觜:一。參:十。
 西方七宿八十一度。
 \gezhu{
  二曆同。}


井:三十三。鬼:三。柳:十五。星:七。張:十八。翌:十八。軫:十七。南方七宿一百一十一度。
 \gezhu{
  二曆同}


角:十二。亢:九。氐:十五。房:五。心:五。尾:十八。箕:十一。東方七宿七十五度。
 \gezhu{
  二曆同。}


\gezhu{
  又儀天云:「前皆赤道度,自古以來,累依天儀測定,用為常準。赤道者,天中紘帶,儀極攸憑,以格黃道
  也。」}


求赤道變黃道度:
 \gezhu{
  乾元謂之求黃道度。儀天謂之推黃道度。}
 準二至赤道日躔宿次。前後五度為限,初限十二,每限減半,終九限減盡。距二立之宿,減一度少強,又從盡起限,每限增半,九限終於十二。距二分之宿,皆乘限度,身外除一,餘滿百為度分,命曰黃赤道差。二至前後各九限,以差為減;二分前後各九限,以差為加。各加減赤道度為黃道度,有餘分就近收為太、半、少之數。
 \gezhu{
  乾元初率九,每限減一,末率一。儀天初數一百七,每
  限減一十,末率二十七,其餘限數加減並同應天。}



 黃道宿度


斗:二十三度半。牛:七度半。
 \gezhu{
  二曆同。}
 女:十一度太。
 \gezhu{
  二曆並十一度半。}
 虛:十度少強。
 \gezhu{
  二千五百六十三、秒十九。乾元無分。儀天六十三分,九十九秒。}
 危:十七度少。
 \gezhu{
  乾元同。儀天十七度太。}
 室:十六度太。壁:十度。
 \gezhu{
  乾元九度太。儀天同。}


北方七宿九十七度二千五百六十三、秒十九。
 \gezhu{
  乾元九十六度半、儀天九十七度半、六十三、秒九十九。}


奎:十七度半。
 \gezhu{
  二曆同。}
 婁:一
 十二度太。
 \gezhu{
  乾元十三度。儀天同。}
 胃:十四度少。
 \gezhu{
  二曆並十四度太。}
 昴:十一度。
 \gezhu{
  二曆同。}
 畢:十六度半。
 \gezhu{
  乾元同。儀天十六度少。}
 觜:一度。參:九度少。
 \gezhu{
  二曆並同。}
 西方七宿八十二度少。
 \gezhu{
  乾元八十三度。儀天八十二度半。}


井:三十度。鬼:二度太。
 \gezhu{
  二曆並同。}
 柳:十四度半。
 \gezhu{
  乾元、儀天十四度少。}
 星:七度。
 \gezhu{
  乾元、儀天並六度太。}
 張:十八度少。
 \gezhu{
  乾元同。儀天十八度太。}
 翼:十九度少。
 \gezhu{
  乾元十九度。儀天同。}
 軫:十八度太。
 \gezhu{
  二曆同。}
 南方七宿一百一十度半。
 \gezhu{
  乾元一百九度太。儀天同。}
 角:十三度。亢:九度半。
 \gezhu{
  二曆並同。}
 氐:十二度少。
 \gezhu{
  乾元、儀天並十五度半。}
 房:五度。
 \gezhu{
  二曆同。}
 心:五度。
 \gezhu{
  乾元同。儀天四度太。}
 尾:十七度少。
 \gezhu{
  乾元同。儀天十七度。}
 箕:十度
 \gezhu{
  乾元十度太。儀天十度。}
 東方七宿七十五度少。
 \gezhu{
  乾元七十六度。儀天七十四度太。}


求赤道日度:
 \gezhu{
  儀天謂之推日度。}
 以天總除元積,為總數;不盡,半而進位,又以一百收總數從之,以元法收為度,不滿為分秒,命起赤道虛宿四度分。
 \gezhu{
  乾元以軌率去歲積分,餘以五因之,滿軌率收為度,不滿,退除為分,餘同。儀天以乾數去歲積分,宗法收為度,命起盧宿二度,餘同應天。又以一象度及餘秒累加之,滿赤道宿度即去之,各得四正,即初日加時赤道日度也。} 求黃道日度:置冬至赤道日躔宿度,以所入限數乘之,所得,身外除一,滿百為度,不滿為分,用減赤道日度,為冬至加時黃道日度及分。
 \gezhu{
  乾元、儀天亦如其法。乾元即以八十四,儀天以一百一除為度,餘同應天。}


求朔望常日月:
 \gezhu{
  乾元謂之求黃道平朔日度。}
 置朔、望日躔先後定數,進一位,倍之,身外除之,以元法收為度分,先加後減朔望中日、月,為朔望中常日、月度分;用加冬至黃道之宿,命如前,即得朔望常日、月所在。
 \gezhu{
  乾元置會週一萬七千三百六十,以距十一月
  後來月數乘之,所得,減去朔餘,加會餘而半之,以二百九十四收為度,不盡,退除為分。儀天法在後。乾元又有求黃道加時朔日度,置平朔日,以日躔陽加陰減之,又以冬至黃道日度加而命之,即其朔加時黃道日度及分也。若求望日度者,以半朔策加之,即得望日度及分也。用陽度,即依本術。}


每日加時黃道日度:
 \gezhu{
  乾元謂之每日行分。}
 以定朔、望日所在相減,餘以距後日數除之,為平行分;二行分相減,為合差;半之,加減平行分,為初行分;
 \gezhu{
  後平行多,減為初;後平行少,加為初。}
 以距後日數除合差,為日差;後少者損,後多者益,為每日行分;累加朔、望日,即得所求。
 \gezhu{
  乾元同。儀天不立此法。又儀天有求次正定日加時黃道日度,置歲
  差,以限數乘之,退一位,滿一百一為差秒及小分,再析之,乃以加一象度,所得,累加冬至黃道日,滿黃道宿次去之,各得四正,即加時黃道日度也。若求四正定日夜半黃道日度,置其定日小餘副之,以其日盈縮分乘之,滿宗法而一,盈加縮減其副,乃以減其日加時,即為夜半黃道日度。又有求每日夜半日度,因四正初日夜半度,累加一策,以其日盈縮分盈加縮減,滿黃道宿次去之,即得每日夜半日度。又有求定朔、弦、望加時日度,置定朔、望小餘副之,以其日盈縮分乘之,以宗法收之為分,盈加縮減其副,以加其日夜半度,各得其時加日躔所次。如朔、望有進退者,此術不用。}



\end{pinyinscope}