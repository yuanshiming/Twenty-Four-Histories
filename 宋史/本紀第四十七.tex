\article{本紀第四十七}

\begin{pinyinscope}

 瀛國公二王附



 瀛國公名□,度宗皇帝子也,母曰全皇后,咸淳七年九月己丑,生於臨安府之大內。九年十一月授左衛上將軍,封嘉國公。十年七月癸未,度宗崩,奉遺詔即皇帝位
 於柩前,年四歲,謝太后臨朝稱詔。甲申,兄是保康軍節度使、開府儀同三司,進封吉王,加食邑一千戶;弟昺保寧軍節度使、開府儀同三司,進封信王,加食邑一千戶。命平章賈似道依文彥博故事,獨班起居。丙戌,上皇太后尊號曰壽和聖福太皇太后,皇后曰皇太后。又詔以生日為天瑞節。戊子,命臨安府振贍細民。辛卯,以朱祀孫為京湖、四川宣撫使兼知江陵府。壬寅,詔撫三邊將士。命州郡舉遺逸。除浙西安撫司、兩浙轉運司、臨安府
 見追贓賞錢。詔求言。



 八月甲辰,詔乞言於老臣江萬里、葉夢鼎、馬廷鸞、留夢炎、趙順孫、王□龠。李庭芝築清河城,以圖來上,詔庭芝進一秩,宣勞將士,具名推賞。加知鄂州李雷應守軍器監,知太平州孟之縉尚書兵部員外郎,知江州錢真孫直寶章閣,知鎮江軍洪起畏直敷文閣。癸丑,大霖雨,天目山崩,水湧流,安吉、臨安、餘杭民溺死者亡算。甲寅,太皇太后以老不能御正衙,命暫以慈元殿為後殿。辛酉,作度宗廟。戊辰,以全清夫為昭信軍
 節度使,謝堂檢校少保,謝垕保康軍節度使。馬廷鸞乞骸骨歸田里,詔趣之任。



 九月丁丑,資政殿大學士、光祿大夫王□龠乞致仕,詔不允。戊寅,發米振餘杭、臨安兩縣水災。餘杭災甚,再給米二千石。己卯,似道乞免答拜,從之。辛巳,覆試文武舉士人。壬午,覆試文武舉士人。癸未,大元兵大會於襄陽。丙戌,丞相伯顏將一軍趣郢州,元帥唆都將一軍入淮,翟招討將一軍徇荊南。丁亥,大元軍薄郢州。戊子,免被水州縣今年田租。甲午,初開經筵。
 丁酉,天瑞節,免征臨安府公私房賃錢十日。以金符十三、銀符百給夏貴激賞奇功。己亥,試正奏名進士,賜王龍澤以下出身有差。壬寅,有星見西方,委曲如蚓。復州副將翟國榮遇大元兵,戰爛泥湖,死之。閩中旱。



 冬十月丙午,知達州趙章復洋州,加右驍騎尉中郎將。大元兵破渠州禮義城,知州張資自殺。丁未,饒州布衣董聲應進《諸史纂約》、《兵鑒》、《刑鑒》,詔聲應充史館編校文字。癸丑,上度宗謚。廣西經略司權參議官邢友龍擊潮州、漳州
 寇,破之。乙卯,令州縣行義田、義役。丁巳,友龍以下諸將各轉官有差。大元兵攻郢州,都統制張世傑力戰御之,遂去,由藤湖入漢。戊午,郢州副都統趙文義追戰全子湖死,恤其家。庚申,贈翟國榮復州團練使,官其二子,立廟復州。壬戌,以錢百萬給郢城屯戍將士。甲子,詔以明年為德祐元年。乙丑,以章鑒同知樞密院事兼權參知政事,陳宜中簽書樞密院事兼權參知政事。大元兵徇沙洋城,京湖宣撫司遣總管王虎臣援之。丙寅,城破,虎
 臣與守隘官王大用皆被執。熒惑犯鎮星。大元兵至新城。戊辰,總制黃順出降。己巳,副總制仁寧出降。都統制邊居誼力戰,城破,赴火死。知復州翟貴以城降。閩中地震。



 十一月癸酉,以朱祀孫為京湖、四川宣撫使。丁丑,命沿江制置使趙溍巡江策應,賜錢百萬激賞戰功。戊寅,馬廷鸞力辭浙東安撫使、知紹興府,詔依舊觀文殿大學士、提舉洞霄宮。贈趙文義清遠軍節度使,與其兄威武軍節度使文亮共立廟揚州,賜名傳忠。庚辰,以陸秀
 夫為淮東安撫制置司參議官。壬午,削諸班直溢額人。癸未至乙酉,覆試特奏名士人。丙戌,以王□龠為左丞相,章鑒為右丞相,並兼樞密使。似道自九月乞命左右丞相,至是從之。以張晏然兼京湖、四川宣撫司參議官。己丑至庚寅,覆試特奏名士人。壬辰至癸巳,如上覆試。甲午,括邸第戚畹及御前寺觀田,令輸租。丁酉,加安南國王陳日煚寧遠功臣,其子威晃奉正功臣。十二月癸卯朔,命建康府、太平州、池州振避兵淮民。以隆寒,勞賜京
 湖及沿江戍守將士。甲胡,詔淮西四郡水旱,去年屯田未輸之租其勿征。丁未,提舉興國宮呂師夔請募兵江州,詔知州錢真孫同募,尚書省以錢米給之。癸丑,大元兵攻陽邏堡,夏貴以兵力守,武定軍都統制王達戰死。乙卯,大元兵夜以偏師乘雪渡青山磯。丙辰,都統程鵬飛鏖戰,被重創,歸鄂州,都統高邦憲屯馬家渡,棄舟走,被執。大元兵復攻夏貴於陽邏堡,都統制劉成以定海水軍戰死。貴敗,沿江縱兵大掠,歸廬州。朱祀孫將兵至
 鄂州,聞鄂兵敗,夜奔江陵府。己未,權知漢陽軍王儀以城降。呂文煥以北兵攻鄂州。庚申,程鵬飛及權守張晏然以城降。幕僚張山翁不屈,諸將欲殺之,丞相伯顏曰:「義士也,釋之。」詔錢塘、仁和兩縣民年七十至九十已上者,賜帛及酒米。癸亥,詔似道都督諸路軍馬,以步軍指揮使孫虎臣總統諸軍,所闢官屬皆先命後奏。詔天下勤王。甲子,起李芾為湖南提刑。乙丑,以高達為湖北制置使兼安撫、知江陵府。詔:「邊費浩繁,吾民重困,貴戚釋
 道,田連阡陌,安居暇食,有司核其租稅收之。」贈王達清遠軍承宣使。庚午,度宗梓宮發引至浙江上,俟潮漲絕江,潮失期,日晡不至。程鵬飛以北兵徇黃州,知州陳奕遣人請降於壽昌軍。李庭芝以兵勤王。辛未,命州郡節制駐戍經從兵。



 德祐元年春正月癸酉朔,大元兵入黃州。甲戌,陳奕遣人下蘄州,並招其子巖於安東州。丁丑,知蘄州管景模遣人請降於黃州。戊寅,詔浙東邸第出米,減價糶民。壬
 午,葬度宗於永紹陵。大元兵入蘄州。癸未,似道以呂師夔權刑部尚書、都督府參贊軍事,任中流調遣。乙酉,以陳宜中同知樞密院事兼參知政事。呂師夔、錢真孫遣人請降於蘄州。丙戌,大元兵徇江州。知安東州陳巖夜遁。邳州降。知壽昌軍胡夢麟寓治於江州,丁亥,自殺。戊子,知南康軍葉閶遣人請降於江州。似道出師。知德安府來興國以城降。夔路安撫張起巖與其將弋德攻開州,復取之。己丑,知安慶府範文虎遣人以酒饌如江州
 迎師。乙未,附度宗神主於新宮。以孫虎臣為寧武軍節度使。戊戌,赦京畿罪。池州都統張林遣人請降於江州。大元兵入安慶,範文虎降,通判夏椅仰藥死。是月,知達州鮮汝忠以城降。



 二月癸卯,似道以宋京為都督府計議官,使大元軍中。甲辰,以黃萬石為江南西路制置使,加湖北制置副使高達檢校少保。庚戌,大元兵入池州,權守趙卯發自經死。宋京如軍中,請稱臣、奉歲幣,不得請而還。辛亥,贈劉成清遠軍承宣使。乙卯,五郡鎮撫呂
 文福遣所部淮兵入衛,降詔褒之。丙辰,詔勞賈似道,命都督府歲舉改官如史嵩之故事。己未,加張起巖福州觀察使,弋德以下各轉五官。庚申,虎臣與大元兵戰於丁家洲,敗績,奔魯港,夏貴不戰而去。似道、虎臣以單舸奔揚州,諸軍盡潰,翁應龍以都督府印奔臨安。壬戌,大元兵徇饒州,知州唐震死之。故相江萬里赴水死,通判萬道同以城降。沿江制置大使趙溍、知鎮江府洪起畏、知寧國府趙與可、知隆興府吳益皆棄城遁。知和州王
 喜以城降。建康都統翁福出迎大元兵。甲子,大元兵至臨江軍,民盡去,知軍鮑廉死之。似道上書請遷都。乙丑,下公卿雜議,王□龠言己不能與大計,遂去。張世傑將兵入衛臨安,道饒州,復取之,其將謝元、王海、李旺、袁恩、呂再興皆戰死。江西提刑文天祥起兵勤王。丙寅,以天祥為江西安撫副使、知贛州,趣入衛。詔募兵。以謝堂為兩浙鎮撫使,謝至保寧軍節度使,全永堅、謝垕並檢校少保。戊辰,征兩浙、福建諸郡廂禁兵之半入衛。湖南提刑
 李芾以兵勤王。知江陰軍鄭棄城遁,知無為軍劉權、知太平州孟之縉皆以城降。己巳,大元兵攻嘉定九頂山,都統侯興戰死。以陳宜中知樞密院事兼參知政事,曾淵子同知樞密院事、兩浙安撫制置大使兼知臨安府,文及翁簽書樞密院事,倪普同簽書樞密院事。召王□龠為浙西、江東宣撫招撫大使,使居京師,以備咨訪。遣大元國信使郝經等歸。庚午,加夏貴開府儀同三司,令以所部兵入衛。令長吏給經過兵民錢米,一切勿征稅。
 應編配、拘鎖人,除偽造關會、強劫盜放火者,餘悉縱之。放免浙西公田逋米及諸文武官在謫籍者,並放自便與敘復改正,放參親民。加張玨寧遠軍節度使,昝萬壽保康軍節度使,張世傑和州防禦使,令將兵入衛。陳宜中乞誅似道,詔罷似道平章、都督,予祠。趙與可除名,令臨安府捕案之。招似道潰兵。辛未,右丞相章鑒遁。



 三月壬申朔,詔復茶鹽市舶法。似道諸不恤民之政,次第除之,以公田給佃主,令率其租戶為兵。殿前
 指揮使韓震請遷都,陳宜中殺之。震所部兵叛,攻嘉會門,射火箭至大內,急發兵捕之,皆散走。癸酉,都統徐旺榮迎大元兵入建康府,鎮江統制石祖忠請降於建康。命浙西提刑司準備差遣劉經戍吳江,兩浙轉運司準備差遣羅林、浙西安撫司參議官張濡戍獨松關,山陰縣丞徐垓、正將鬱天興戍四安鎮,起趙淮為太府寺丞,戍銀樹東壩。湖北安撫司計議官吳繼明攻通城縣,復取之,執縣令以歸。遣使召章鑒還朝。甲戌,以似道為醴
 泉觀使。大元兵至無錫縣,知縣阮應得出戰,一軍皆沒,應得赴水死。詔發兵戍吳江。乙亥,發兵戍獨松嶺、銅嶺。詔諭呂文煥、陳奕、範文虎使通和議息兵。以王□龠為左丞相兼樞密使。閩中地復大震。丙子,下詔罪己。以陳宜中為特進、右丞相兼樞密使。罷章鑒官,予祠。侍御史陳過請竄賈似道並治其黨人翁應龍等,不俟報而去。監察御史潘文卿、季可乞從過所請,乃命捕應龍下臨安府獄。罷廖瑩中、王庭、劉良貴、游汶、朱浚、陳伯大、董樸。責
 洪起畏鎮江自效。丁丑,知滁州王應龍以城降。己卯,杖翁應龍,刺配吉陽軍。命王□龠、陳宜中並都督諸路軍馬。加呂文福福州觀察使。庚申,贈唐震華文閣待制。削萬道同三官,罷之。壬午,復吳潛、向士璧官。知常州趙與鑒聞兵至遁,常民錢誾以城降。甲申,大元兵至西海州,安撫丁順降。乙酉,知東海州施居文乞降於西海州。知平江府潛說友、通判胡玉、林鏜以城降。加張世傑保康軍承宣使,總都督府諸軍。丙戌,知廣德軍令狐概以城降。
 徙浙西提點刑獄司於平江府。張世傑遣其將閻順、李存進軍廣德,謝洪永進軍平江,李山進軍常州。丁亥,張德以下各轉官有差。謝元等贈十官。有星二斗於中天,頃之,一星隕。己丑,滁人執王應龍歸於揚州,殺之。加呂文福保康軍承宣使,趣入衛。文福至饒州,殺使者,入江州降大元。庚寅,左司諫潘文卿、右正言季可、同知樞密院曾淵子、兩浙轉運副使許自、浙東安撫王霖龍相繼皆遁。簽書樞密院文及翁、同簽書樞密院倪普諷臺臣
 劾己,章未上,亟出關遁。知安東州孫嗣武以城降。雨土。辛卯,命在京文武官並轉兩官,其畔官而遁者,令御史臺覺察以聞。閻順戰安吉縣,復取鳳平。張濡部曲害大元行人嚴忠範於獨松關,執廉希賢至臨安,重創死。壬辰,岳州安撫高世傑軍洞庭中,大元兵攻之,世傑降。癸巳,攻岳州,總制孟之紹以城降。甲午,詔褒諭張世傑、閻順,諸將各轉官有差。乙未,免安吉縣今年夏田租,有戰沒者,縣令、丞恤之。丙申,顧順攻廣德軍,復取之。以陳合
 同簽書樞密院事。丁酉,贈邊居誼利州觀察使。戊戌,赦邊城降將罪,能自拔而歸者錄之,復一州者予知州,復一縣者予知縣,所部僚吏將卒及土豪立功者同賞。罷章鑒祠官並奪宰輔恩數,曾淵子削兩官,奪執政恩數,陳過、陳堅、徐卿孫各削兩官,奪侍從恩數。趙與鑒追兩官罷之,遇赦永不收敘。罷許自、王霖龍。令淮東制置司用標由。庚子,徙淮東總領所於江陰軍。加吳繼明閣門宣贊舍人。



 四月壬寅朔,贈趙卯發華文閣待制。貶陳過
 平江府。雄江軍統制洪福率眾復鎮巢軍。甲辰,贈江萬裏太師,謚文忠,輟視朝二日。乙巳,大元兵入廣德縣,知縣王汝翼與寓居官趙時晦率義兵戰鬥山,路分孟唐老與其二子皆死,汝翼被執,至建康死之。王大用贈三官,王虎臣贈兩官,官其二子。丙午,大元兵破沙市城,都統孟紀死之,監鎮司馬夢求自經死。戊申,京湖宣撫朱祀孫、湖北制置副使高達以江陵降,京湖北路相繼皆下。張起巖提兵保飛山。己酉,命劉師勇戍平江府。辛亥,
 顧順諸將各轉三官,孟唐老贈三官。壬子,以高斯得簽書樞密院事權參知政事。總統張敏與大元兵戰豐城,死之。癸丑,贈五官,官其一子。阮應得贈十官。乙卯,以福王與芮為武康、寧江軍節度使、判紹興府。丙辰,王□龠來,令如文彥博故事,自朝參起居外並免拜。以樞密副使召夏貴提兵入衛。丁巳,總制霍祖勝攻溧陽縣,復取之。戊午,贈張資眉州防禦使,侯興復州團練使。乙未,文及翁、倪普並削一官,奪執政恩數;潛說友削三官,奪侍
 從恩數。庚申,令狐概除名、配鬱林州牢城,籍其家。知金壇縣李成大率義局官含山縣尉胡傳心、陽春主簿潘大同、濠梁主簿潘大PO、進士潘文孫潘應奎攻金壇縣,取之。鎮江統制侯巖、縣尉趙嗣濱復助大元兵來戰,成大二子及大同等皆死,執成大以歸。壬戌,大元兵攻真州,知州苗再成、宗子趙孟錦率兵大戰於老鸛觜。癸亥,加知思州田謹賢、知播州楊邦憲並復州團練使,趣兵入衛。有大星自心東北流入濁沒。乙丑,熒惑犯天江。提
 舉太平興國宮常楙請立濟王後。丁卯,加李庭芝參知政事。戊辰,詔宜興、溧陽民兵助戰有功,特免今年田租。江陰民被兵,其租亦勿收責。庚午,大元兵至揚子橋,揚州都撥發官雷大震出戰死。是月,常德、鼎、澧皆降。



 五月辛未朔,命宰執日赴朝堂治事。旌德縣城守有功,免其民今年田租。癸酉,大元兵至寧國縣,知縣趙與□唐出戰死。甲戌,淮安總制李宗榮、知慶遠府仇子真將兵來勤王。乙亥,加苗再成濠州團練使,趙孟錦揚州都統司計
 議官。以洪福知鎮巢軍。丁丑,詔趙溍統軍民船屯江陰。劉師勇攻常州,復取之,執安撫戴之泰,司戶趙必佮、總管陸春戰死。戊寅,淮東兵馬鈐轄阮克己將兵來勤王,加左驍騎中郎將。己卯,賜婺州處士何基謚文定,王柏承事郎。加張玨檢校少保、四川制置副使、知重慶府。庚辰,贈雷大震保康軍節度使。辛巳,加劉師勇濠州團練使,其將劉圭以下各轉官有差。戊子,贈潘大同等官,餘有功人並轉兩官。辛卯,貶潛說友南安軍,吳益汀州,並
 籍其家。罷李玨,送婺州。籍呂文煥、孟之縉、陳奕、範文虎家。甲午,饒、信州饑,令民入粟補官。罷市舶分司,令通判任舶事。淮東、西官民兵各轉一官。丙申,詔張世傑、張彥、阮克己、仇子真四道出兵、遣使告天地、宗廟、社稷諸陵、宮觀。己亥,勞軍。吳繼明復蒲圻、通城、崇陽三縣,加帶行帶御器械、權知鄂州,令擇險為寓治。贈鮑廉直華文閣,官其一子;趙與□唐直華文閣。



 六月庚子朔,日有食之,既,晝晦如夜。昝萬壽以嘉定及三龜、九頂、紫雲城降。知敘
 州李演將兵援嘉定府,遂解歸,戰羊雅江,兵敗被執。辛丑,太皇太后詔削尊號「聖福」字以應天戒。復魏克愚官,太學生蕭規、唐棣並補承信郎。知嘉興府餘安裕坐聞兵求去,貽書朝中,語涉不道,削一官送徽州。徐卿孫削一官貶吉州。命侍從官已上各舉才堪文武者五人,餘廷臣各舉三人,雖在謫籍,亦聽舉之。丙午,王應麟言:「開慶之禍,始於丁大全,請凡大全之黨,在謫籍者皆勿宥。」從之。己酉,免廣德軍今年田租及諸郡縣未納綱解。王
 應麟繳還章鑒、曾淵子錄黃,言韓震為逆,二人實芘之;且淵子芘翁應龍致有逸罰,又嘗竊府庫金以遁。庚戌,命削鑒一官,放歸田里,淵子再削一官,徙吉州,誅翁應龍,籍其家。辛亥,銓試。甲寅,留夢炎入朝,王□龠請相夢炎,乞以經筵備顧問,陳宜中請相夢炎,乞祠,詔二相毋藉此求閑。以□龠為平章軍國重事,一月兩赴經筵,五日一朝;宜中左丞相兼樞密使,都督諸路兵;夢炎右丞相兼樞密使,都督諸路兵。乙卯,詔求言。知敘州郭漢傑以城
 降。丙辰,疏決在京罪人。免引見。戊午,知瀘州梅應春以城降。己未,以李庭芝知樞密院事兼參知政事。庚申,知富順監王宗義以城降。王應麟復繳還曾淵子貶吉州錄黃,癸亥,貶韶州。丙寅,吳繼明諸將各轉官有差。丁卯,朱祀孫除名,籍其家。



 秋七月庚午朔,江西制置黃萬石移治撫州,詔還隆興府。辛未,張世傑諸軍戰焦山下,敗績。甲戌,徙似道居婺州,廖瑩中除名貶昭州,王庭除名貶梅州,徙曾淵子雷州。寧國吏楊義忠率義兵出戰死,
 乙亥,贈武功大夫。丁丑,徙似道建寧府。太白入東井。庚申,加知高郵軍褚一正閣門宣贊舍人,知懷遠軍金之才帶御器械,知安淮軍高福閣門祗候,知泗州譚與閣門宣贊舍人,知濠州孫立右衛大將軍,賞守邊功。壬午,太白晝見。詔饒州被兵,令免今年田租。路鈐劉用調兵入靖州,知州康玉劫之,通判張起巖入殺玉,復靖州。癸未,拘內司局錢餉兵。丙戌,令權糴公田今年租,每石以錢十貫給佃主,十貫給種戶,其鎮江、常州、江陰被兵者
 勿糴。庚寅,謫似道為高州團練副使、貶循州,籍其家。糴浙西邸第、寺觀田米十之三。追復皮龍榮官。監司、郡守避事不即到官者,令御史臺覺察以聞。辛卯,王□龠子嗾京學生劉九皋等伏闕上書言:宜中擅權,黨似道,芘趙溍、潛說友,使門客子弟交通關節,其誤國將甚於似道。宜中去,遣使四輩召之,皆不至。謝堂乞罷兩浙鎮撫司,不從。張世傑乞濟師,不報。壬辰,下劉九皋等臨安獄,罷王□龠為醴泉觀使。癸巳,以夏貴知揚州,朱煥知廬州。甲午,
 遣使召宜中還朝。乙未,以陳文龍同簽書樞密院事兼權參知政事。通判婺州張鎮孫聞兵遁,罷其官。貶胡玉連州、林鏜韶州,並除名。沿江招討大使汪立信卒。丙申,削李玨兩官、貶潮州。以開慶兵禍,追罪史嵩之奪其謚。戊戌,遣使召宜中還朝。



 八月己亥朔,總制毛獻忠將衢州兵入衛。辛丑,疏決臨安府罪人。壬寅,右正言徐直方遁。加夏貴樞密副使、兩淮宣撫大使,李芾湖南鎮撫大使。總制戴虎破大南砦,轉三官。加張起巖太府寺丞、知
 靖州,劉用以下立功人各轉官有差。大元兵駐巴陵縣黃沙。乙巳,吳繼明復平江縣。戊申,試太學上舍生。己酉,拘閻貴妃集慶寺、賈貴妃演福寺田還安邊所。庚戌,劉師勇攻呂城,破之。癸丑,復《嘉定七司法》。丁巳,遣使召宜中還朝。加張世傑神龍衛四廂都指揮使,總都督府諸兵。戊午,加劉師勇和州防禦使。熒惑犯南斗。趙淇除大理少卿,王應麟封還錄黃,言昔內外以寶玉獻似道,淇兄弟為甚,己未,遂罷之。甲子,以文天祥為浙西、江東制
 置使兼知平江府。乙丑,揚州文武官轉兩官。加吳繼明湖北招討使,朱旺諸將各轉三官。



 九月己巳,陳宜中授觀文殿大學士、醴泉觀使兼侍讀。左司諫陳景行請令講官坐講陪宿直,從之。辛未,加田謹賢福州觀察使,楊邦憲利州觀察使,趣入衛。己卯,陳宜中乞任海防,不允。辛巳,有事於明堂,赦。李成大被執,不屈死,壬午,贈五官。丙戌,命文天祥為都督府參贊官,總三路兵。會稽縣尉鄭虎臣部送似道之貶所,至漳州,殺之。大元兵至泰州,
 知州孫虎臣自殺,庚寅,贈太尉。免靖州今年田租。辛卯,徙李玨梧州。乙未,劉良貴再削兩官、貶信州。張彥與大兵戰敗被執,以城降。



 冬十月己亥,加張世傑沿江招討使,劉師勇福州觀察使,總統出戍兵。壬寅,宜中來。癸卯,玉牒殿災。丁未,以夢炎為左丞相,宜中為右丞相,並兼樞密使、都督。城臨安。辛亥,以張世傑為沿江制置副使兼知江陰軍兼浙西策應使。丁巳,太白會填星。戊午,領戶部財用常懋、中書舍人王應麟請立濟王後。贈夏椅
 直秘閣。征紹興府處士陸應月為史館編校文字。壬戌,大元兵發建康,參政阿刺罕、四萬戶總管奧魯赤將右軍出四安鎮,趣獨松關,參政董文炳、範文虎將左軍出江入,江陰軍,丞相伯顏將中軍入常州。熒惑犯壘壁陣。癸亥,張全、尹玉、麻士龍援常州,士龍戰虞橋死,全奔五牧。朱煥至廬州,貴不內。煥歸,復以為淮東制置副使。陳合坐匿廖瑩中家資,奪執政恩數。甲子,尹玉戰五牧,死之,張全不戰遁。丙寅,趣趙溍、趙與可、鄭所募兵。詔中
 外官有習兵略者,各以書來上。是月,李世修以江陰降。



 十一月丁卯朔,銅關將貝寶、胡巖起攻溧水死,贈寶武翼郎,巖起朝奉郎。庚午,以陳文龍同知樞密院事兼權參知政事,黃鏞同簽書樞密院。命諸制司各舉才堪將帥者十人,不限偏裨士卒,如不隸軍中者,許投匱自薦。辛未,起居舍人曾唯辭官不允,去。癸酉,贈尹玉濠州團練使、麻士龍高州刺史,免張全、朱華臨陳退師罪。丁丑,詔被俘將士能率眾來歸者,以人數補官,能立功者予
 節鉞;諸閫以下官,以所招人多寡行賞。戊寅,大元兵破廣德軍。己卯,破四安鎮,正將胡明等死之。召文天祥入衛。辛巳,曾唯削一官免。太白犯房。壬午,大元兵至隆興府,黃萬石棄撫州遁,轉運判官劉盤以隆興降。癸未,大元兵破興化縣,知縣胡拱辰自殺。甲申,中書舍人王應麟辭免兼給事中,不允。大元兵至常州,招降不聽,攻二日,破之,屠其城。知州姚誾、通判陳照、都統王安節皆死,劉師勇潰圍奔平江。乙酉,改宜興縣為南興軍。禮部侍
 郎陳景行辭官不允,去。丙戌,贈濟王太師、尚書令,進封鎮王,謚昭肅,令福王與芮擇後奉祀,賜田萬畝。丁亥,獨松關告急,趣文天祥入衛。戊子,調民兵出守餘杭、錢塘。己丑,獨松關破,馮驥死之,張濡遁,鄰邑望風皆遁。通判平江府鄭疇遁,庚寅,通判王矩之、都統制王邦傑遣人迎降於常州。辛卯,大元兵趨撫州,都統密祐逆戰於璧邪,兵敗,死之。癸巳,以張世傑為浙西制置副使兼知平江府。甲午,權禮部尚書王應麟遁,黃萬石提兵走建昌
 軍。乙未,左丞相夢炎遁。丙申,遣使召夢炎還朝。賜餘杭、武康、長興縣民錢,並免今年田租。鄭疇降一官,罷通判。撫州施至道以城降。



 十二月丁酉朔,詔許似道歸葬,以其祖田廬還之。戊戌,復趙與可為都督府參議官,放李玨自便。己亥,贈王汝翼朝奉郎。庚子,以吳堅簽書樞密院事,黃鏞兼權參知政事。遣柳嶽奉書詣大元軍中,稱盜殺廉尚書,乞班師修好。癸卯,以陳文龍為參知政事兼權知樞密院事,賜謝堂同進士出身,同知樞密院事。
 甲辰,贈姚誾龍圖閣待制,其父希得贈太師,陳照直寶章閣,馮驥集英殿修撰。嘉興府告急,給封樁庫錢為兵備。命趙與侲戍縉雲縣。復季可官,令如龍泉縣募兵。乙巳,以陳景行為浙東安撫副使,戍處州。起方逢辰戍淳安縣。丙午,追封呂文德和義郡王。丁未,出安邊封樁庫金付浙東諸郡為兵備。大元兵入平江府。起吳君擢為太府少卿,提點臨平民兵。遣使召夢炎、應麟,皆不至。戊申,張世傑入衛,加檢校少保,降詔獎諭。王□龠薨,輟視朝
 二日。乙酉,括臨安府州縣馬。庚戌,柳嶽還。癸丑,遣宗正少卿陸秀夫、刑部尚書夏士林、兵部侍郎呂師孟使軍前。詔呂文煥、趙孟桂通好。己未,方興、丁廣、趙文禮兵皆敗歸。庚申,以柳岳為工部侍郎,洪雷震為右正言,使燕祈請。大元兵破大洪山,知隨州朱端履降。權吏部尚書丁應奎、左侍郎徐宗仁遁。癸亥,遣使召夢炎,不至。



 德祐二年春正月丁卯朔,大元兵自元年十月圍潭州,湖南安撫兼知州李芾拒守三月,大小戰數十合,力盡
 將破,芾闔門死,郡人知衡州尹谷亦舉家自焚,帥司參議楊霆及幕屬陳億孫、顏應焱等皆從芾死。守將吳繼明、劉孝忠以城降。寶慶降,通判曾如驥死之。陸秀夫等至大元軍中,求稱侄納幣,不從;稱侄孫,不從。戊辰,還。太皇太后命用臣禮。己巳,嘉興守劉漢傑以城降。庚午,同簽書樞密院事黃鏞、參知政事陳文龍遁。以謝堂為兩浙鎮撫大使,文天祥知臨安府,全永堅浙東撫諭使。辛未,命吳堅為左丞相兼樞密使,常楙參知政事。日午,宣
 麻慈元殿,文班止六人。諸關兵盡潰。遣監察御史劉岊奉表稱臣,上大元皇帝尊號曰仁明神武皇帝,歲奉銀絹二十五萬,乞存境土以奉蒸嘗。癸酉,左司諫陳孟虎、監察御史孔應得遁。熒惑犯木星。甲戌,大元兵至瑞州,知州姚巖棄城去。乙亥,以賈餘慶知臨安府。丙子,命吉王是、信王昺出鎮。丁丑,以夏士林簽書樞密院事。己卯,加全永堅太尉。參知政事常懋遁。三學生誓死不去,特與放釋褐出身。以楊亮節為福州觀察使,提舉吉王府
 行事;俞如珪為環衛官、提舉信王府行事。大元兵入安吉州,知州趙良淳自經死。月暈東井。庚辰,簽書樞密院夏士林遁。辛巳,祀太乙宮。癸未,升封吉王是為益王,判福州、福建安撫大使;信王昺為廣王,判泉州兼判南外宗正事。以留夢炎為江東西、湖南北宣撫大使。甲申,大元兵至皋亭山,遣監察御史楊應奎上傳國璽降,其表曰:「宋國主臣□謹百拜奉表言,臣眇然幼沖,遭家多難,權奸似道背盟誤國,至勤興師問罪。臣非不能遷避,以
 求茍全,今天命有歸,臣將焉往。謹奉太皇太后命,削去帝號,以兩浙、福建、江東西、湖南、二廣、兩淮、四川見存州郡,悉上聖朝,為宗社生靈祈哀請命。伏望聖慈垂念,不忍臣三百餘年宗社遽至隕絕,曲賜存全,則趙氏子孫,世世有賴,不敢弭忘。」是夜,丞相陳宜中遁,張世傑、蘇劉義、劉師勇各以所部兵去。乙酉,以文天祥為右丞相兼樞密使、都督。丙戌,命天祥同吳堅使大元軍。賜家鉉翁進士出身、簽書樞密院事,賈餘慶同簽書樞密院事、知
 臨安府。戊子,知建德軍方回、知婺州劉怡、知處州梁椅、知臺州楊必大皆降。是月,知臨江軍滕巖瞻遁。



 二月丁酉朔,日中有黑子相蕩,如鵝卵。辛丑,率百官拜表祥曦殿,詔諭郡縣使降。大元使者入臨安府,封府庫,收史館、禮寺圖書及百司符印、告敕,罷官府及侍衛軍。壬寅,猶遣賈餘慶、吳堅、謝堂、劉岊、家鉉翁充祈請使。是日,大元軍軍錢塘江沙上,潮三日不至。



 三月丁丑,入朝。



 五月丙申,朝於上都。降封開府儀同三司、瀛國公。是月,陳宜中
 等立是於福州,後二年四月,是殂於□岡洲,陸秀夫等復立衛王昺,後三年始平之。



 贊曰:司馬遷論秦、趙世系同出伯益。夫稷、契、伯益其子孫皆有天下,至於運祚短長,亦系其功德之厚薄焉。趙宋雖起於用武,功成治定之後,以仁傳家,視秦宜有間矣。然仁之敝失於弱,即文之敝失於僿也。中世有欲自強,以革其敝,用乖其方,馴致棼擾。建炎而後,土宇分裂,猶能六主百五十年而後亡,豈非禮義足以維持君子
 之志,恩惠足以固結黎庶之心歟?瀛國四歲即位,而天兵渡江,六歲而群臣奉之入朝。漢劉向言:「孔子論《詩》至『殷士膚敏,裸將於京。』喟然嘆曰:大哉天命,善不可不傳於後嗣,是以富貴無常。」至哉言乎!我皇元之平宋也,吳越之民,市不易肆。世祖皇帝命征南之帥,輒以宋祖戒曹彬勿殺之言訓之。《書》曰:「大哉王言,一哉王心。」我元一天下之本,其在於茲。


二王者,度宗庶子也。長建國公是,母淑妃楊氏;季永國
 公昺,母修容俞氏。度宗崩,謝太后召賈似道等入宮議所立,眾以為是長當立,似道主立嫡,乃立□而封是為吉王、昺信王。德祐二年正月,文天祥尹臨安,請以二王鎮閩、廣,不從,始命二王出閣。大元兵迫臨安,宗親復以請,乃徙封是為益王、判福州、福建安撫大使,昺為廣王、判泉州兼判南外宗正,以駙馬都尉楊鎮及楊亮節、俞如珪為提舉。大元兵至皋亭山,鎮等奉之走婺州。丞相伯顏入臨安,遣範文虎將兵趣婺,召鎮以王還,鎮得報
 即去,曰:「我將就死於彼,以緩追兵。」亮節等遂負王徒步匿山中七日,其將張全以兵數十人始追及之,遂同走溫州,陸秀夫、蘇劉義繼追及於道。遣人召陳宜中於清澳,宜中來謁,復召張世傑於定海,世傑亦以所部兵來溫之江心寺。高宗南奔時嘗至是,有御座在寺中,眾相率哭座下,奉是為天下兵馬都元帥,昺副之。乃發兵除吏,以秀王與
 \gezhu{
  □幸}
 為福建察訪使兼安撫、知西外宗正,趙吉甫知南外宗正兼福建同提刑,先入閩中撫吏民,諭
 同姓。太皇太后尋遣二宦者以兵八人召王於溫,宜中等沉其兵江中,遂入閩。時汀、建諸州方欲從黃萬石降,聞是將至,即閉城卻使者,萬石將劉俊、宋彰、周文英輩亦多來歸。



 五月乙未朔,宜中等乃立是於福州,以為宋主,改元景炎,冊楊淑妃為太后,同聽政。封信王昺為衛王。宜中為左丞相兼都督,李庭芝為右丞相,陳文龍、劉黻為參知政事,張世傑為樞密副使,陸秀夫為簽書樞密院事。命吳浚、趙溍、傅卓、李玨、翟國秀等分道出兵。改
 福州為福安府,溫州為瑞安府。郊赦。是日黎明,有大聲出府中,眾皆驚僕。文天祥自鎮江亡歸,庚辰,以為右丞相兼知樞密院事。遣其將呂武入江、淮招豪傑,杜滸如溫州募兵。廣東經略使徐直諒遣梁雄飛請降於隆興帥府,乃假雄飛招討使,使徇廣州。既而直諒聞是立,命權通判李性道、摧鋒軍將黃俊等拒雄飛於石門,性道不戰,俊戰敗奔廣州,直諒棄城遁。



 六月丙子,雄飛入廣州,諸降將皆授以官,俊獨不受,遂為眾所殺。吳浚聚兵
 於廣昌,取南豐、宜黃、寧都三縣。翟國秀取秀山,傅卓至衢、信諸縣,民多應之者。命文天祥為同都督。



 七月丁酉,進兵南劍州,欲取江西。是月,吳浚兵敗於南豐,翟國秀聞兵至,遂引還。傅卓兵敗,詣江西元帥府降。平章阿里海牙破嚴關,馬暨退保靜江府。


八月,漳州亂,命陳文龍為閩廣宣撫使以討之。甲戌,秀王與
 \gezhu{
  □幸}
 圍婺州。丙子,聞大兵至,遂解歸。以王積翁為福建提刑、招捕使、知南劍州,備御上三郡;黃佺為同提刑、招捕使、知漳州,備御下
 三郡。張世傑遣兵助吳浚與元帥李恆戰兜零,兵敗,奔寧都。興化石手軍亂。



 九月,復以陳文龍知興化軍。東莞人熊飛為黃世傑守潮、惠二州,聞趙溍至,即以兵應之,攻雄飛於廣州。壬寅,雄飛遁,熊飛遂復韶州。新會令曾逢龍亦帥兵至廣州,李性道出迎謁,飛與逢龍執而殺之。衢州守將魏福興出戰福星橋,死。壬子,趙溍入廣州。是月,招討也的迷失會東省兵於福州。元帥呂師夔、張榮實將兵入梅嶺。


十月壬戌朔,文天祥入汀州。趙溍遣
 曾逢龍就熊飛禦大軍於南雄,逢龍戰死,熊飛奔韶州。大軍圍韶州,守將劉自立以城降,飛率兵巷戰,兵敗,赴水死。十有一月,參政阿刺罕、董文炳將兵至處州,李玨以城降。甲辰,秀王與
 \gezhu{
  □幸}
 逆戰於瑞安,觀察使李世達死之。與
 \gezhu{
  □幸}
 及其弟與慮、子孟備、監軍趙由□葛、察訪使林溫被執,皆死。阿刺罕兵至建寧府,執守臣趙崇鐖,知邵武軍趙時賞、知南劍州王積翁皆棄城去。乙巳,是入海。癸丑,大軍至福安府,知府王剛中以城降。是欲入泉州,招
 撫蒲壽庚有異志。初,壽庚提舉泉州舶司,擅蕃舶利者三十年。是舟至泉,壽庚來謁,請駐蹕,張世傑不可。或勸世傑留壽庚,則凡海舶不令自隨,世傑不從,縱之歸。繼而舟不足,乃掠其舟並沒其貲,壽庚乃怒殺諸宗室及士大夫與淮兵之在泉者。是移潮州。是月,福、興化皆降。英德守臣凌彌堅、徐夢得等亦降。



 十二月辛酉朔,趙溍棄廣州遁。乙丑,制置方興亦遁,吳浚退走入瑞金。戊辰,蒲壽庚及知泉州田真子以城降。知興化軍陳文龍嬰
 城不下,乙酉,通判曹澄孫以城降,文龍被執,不屈死。是次甲子門。



 至元十四年正月,大軍破汀關。癸巳,知循州劉興降。壬寅,吳浚棄瑞金遁,鎮撫孔遵入瑞金,文天祥走漳州,浚尋還汀州,降。戊申,知潮州馬發及其通判戚繼祖降,癸丑,復來歸。丁巳,權知梅州錢榮之以城降。



 二月,大軍至廣州,縣人趙若岡以城降。廣東諸郡皆降。



 三月,文天祥取梅州,陳文龍從子瓚舉兵殺守將林華,據興化軍。



 四月,文天祥取興國縣,廣東制置使張鎮孫襲
 廣州取之,梁雄飛等棄城走韶州。



 五月,張世傑將兵取潮州,文天祥提兵自梅州出江西,入會昌縣,淮民張德興亦起兵殺太湖縣丞王德顒,據司空山,攻下黃州、壽昌軍。丁巳,遇宣慰鄭鼎,戰樊口,鼎墜水死。



 六月辛酉,文天祥取雩都。己卯,入興國縣。



 七月,遣兵取吉、贛諸縣,圍贛州。衡山人趙璠、撫州人何時皆起兵應之。乙巳,張世傑圍泉州,遣將高日新復邵武。淮兵在福州者,欲殺王積翁以應世傑,皆為積翁所戮。江西宣慰李恆遣兵援
 贛州,而自將兵入興國。



 八月,文天祥諸將兵皆敗,乃引兵即鄒洬於永豐,洬兵亦潰。己巳,熒惑掩月,天色赤。壬申,文天祥兵敗於興國。己卯,大軍破司空山,張德興敗,亡走。甲申,天祥至空坑,兵盡潰,遂挺身走循州,諸將皆被執。



 九月,元帥唆都援泉州。戊申,張世傑歸淺灣。左丞塔出將兵入大庾嶺,參政也的迷失將兵復取邵武,入福州。



 十月甲辰,唆都破興化軍,陳瓚死之。進攻潮州,馬發拒之,乃去攻惠州。



 十一月,塔出圍廣州。庚寅,張鎮孫以
 城降。元帥劉深以舟師攻是於淺灣,是走秀山。陳宜中入占城,遂不反。



 十二月丙子,是至井澳,颶風壞舟,幾溺死,遂成疾。旬餘,諸兵士始稍稍來集,死者十四。丁丑,劉深追是至七州洋,執俞如珪以歸。



 十五年正月,大軍夷廣州城。張世傑遣兵攻雷州,不克。己大軍克涪州,執守將王明。



 二月,大軍破潮州,馬發死之。



 三月,文天祥取惠州,廣州都統凌震、轉運判官王道夫取廣州。是欲往居占城,不果,遂駐□岡洲,遣兵取雷州。曾淵子自雷州
 來,以為參知政事,廣西宣諭使。



 四月戊辰,是殂於□岡洲,其臣號之曰端宗。庚午,眾又立衛王昺為主,以陸秀夫為左丞相。是月,有黃龍見海中。



 五月癸未朔,改元祥興。乙酉,升□岡洲為翔龍縣。遣張應科、王用取雷州,應科三戰皆不利,用因降。



 六月丁巳,應科再戰雷州,遂死之。知高州李象祖降。己未,昺徙居崖山,升廣州為翔龍府。己巳,有大星東南流,墜海中,小星千餘隨之,聲如雷,數刻乃已。己卯,都元帥張弘範、李恆征崖山。



 十月,趙與珞與
 謝明、謝富守瓊州,阿里海牙遣馬成旺招之,與珞率兵拒於白沙口。



 十一月癸巳,州民執與珞以降。



 閏月庚戌,王道夫棄廣州遁。壬戌,凌震遁。癸亥,大軍入廣州。十二月壬午,王道夫攻廣州,兵敗被執。凌震兵繼至,亦敗。文天祥走海豐,壬寅,被執於五坡嶺。震兵又敗於芰塘。大軍破南安縣,守將李梓發死之。



 十六年正月壬戌,張弘範兵至崖山。庚午,李恆兵亦來會。世傑以舟師碇海中,棋結巨艦千餘艘,中艫外舳,貫以大索,四周起樓棚如
 城堞,居昺其中。大軍攻之,艦堅不動。又以舟載茅,沃以膏脂,乘風縱火焚之。艦皆塗泥,縛長木以拒火舟,火不能爇。



 二月戊寅朔,世傑部將陳寶降。己卯,都統張達以夜襲大軍營,亡失甚眾。癸未,有黑氣出山西。李恆乘早潮退攻其北,世傑以淮兵殊死戰。至午潮上,張弘範攻其南,南北受敵,兵士皆疲不能戰。俄有一舟檣旗僕,諸舟之檣旗遂皆僕。世傑知事去,乃抽精兵入中軍。諸軍潰,翟國秀及團練使劉俊等解甲降。大軍至中軍,會暮,
 且風雨,昏霧四塞,咫尺不相辨。世傑乃與蘇劉義斷維,以十餘舟奪港而去,陸秀夫走衛王舟,王舟大,且諸舟環結,度不得出走,乃負昺投海中,後宮及諸臣多從死者,七日,浮尸出於海十餘萬人。楊太后聞昺死,撫膺大慟曰:「我忍死艱關至此者,正為趙氏一塊肉爾,今無望矣!」遂赴海死,世傑葬之海濱,已而世傑亦自溺死。宋遂亡。



 贊曰:宋之亡徵,已非一日。歷數有歸,真主御世,而宋之
 遺臣,區區奉二王為海上之謀,可謂不知天命也已。然人臣忠於所事而至於斯,其亦可悲也夫!



\end{pinyinscope}