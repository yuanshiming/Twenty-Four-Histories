\article{志第一百三十三 食貨下二(錢幣)}

\begin{pinyinscope}

 錢幣



 錢有銅、鐵二等,而折二、折三、當五、折十,則隨時立制。行之久者,唯小平錢。夾錫錢最後出,宋之錢法至是而壞。蓋自五代以來,相承用唐舊錢,其別鑄者殊鮮。太
 祖初鑄錢,文曰「宋通元寶」。凡諸州輕小惡錢及鐵鑞錢悉禁之,詔到限一月送官,限滿不送官者罪有差,其私鑄者皆棄市。銅錢闌出江南、塞外及南蕃諸國,差定其法,至二貫者徒一年,三貫以上棄市,募告者賞之。江南錢不得至江北。



 蜀平,聽仍用鐵錢。開寶中,詔雅州百丈縣置監冶鑄,禁銅錢入兩川。太平興國四年,始開其禁,而鐵錢不出境,令民輸租及榷利,鐵錢十納銅錢一。時銅錢已竭,民甚苦之。商賈爭以銅錢入川界與民互市,
 銅錢一得鐵錢十四。



 明年,轉運副使張諤言:「川峽鐵錢十直銅錢一,輸租即十取二。舊用鐵錢千易銅錢四百,自平蜀,沉倫等悉取銅錢上供,及增鑄鐵錢易民銅錢,益買金銀裝發,頗失裁制,物價滋長,鐵錢彌賤。請市夷人銅,斤給鐵錢千,可以大獲銅鑄錢。民租當輸錢者,許且輸銀絹,候銅錢多,即漸令輸。」又詔令市夷人銅,斤給鐵錢五百,餘皆從之。然銅卒難得,而轉運副使聶詠、轉運判官範祥皆言:民樂輸銅錢,請歲遞增一分,後十歲
 則全取銅錢。詔如所請。祥等因以月俸所得銅錢市與民,厚取其直,於是增及三分,民益以為苦,或發古塚、毀佛像器用,才得銅錢四五,坐罪者甚眾。知益州辛仲甫具言其弊,內使臣吳承勛馳傳審度。仲甫集諸縣令、佐問之,多潛持兩端,莫敢正言。仲甫以大誼責之,乃皆言其不便。承勛運命。二年,遂令川峽輸租榷利勿復征銅錢。宋詠等皆坐罪免。既而又從西川轉運使劉度之請,官以鐵錢四百易銅錢一百,後竟罷之。



 平廣南、江南,赤
 德雄州舊錢,如川蜀法。初,南唐李因鑄錢,一工為錢千五百,得三十萬貫。太宗即位,詔升州置監鑄錢,令轉運使按行所部,凡小山之出銅者悉禁民採,並以給官鑄焉。太平興國二年,樊若水言:「江南舊用鐵錢,於民非便。今諸州銅錢尚六七十萬緡,虔、吉等州未有銅錢,各發六七萬緡,俾市金帛輕貨上供及博糴穀麥。於則、免饒等州產銅之地,大鑄銅錢,銅錢既不渡江,益出新錢,則民間錢愈多,鐵錢自當不用,悉熔鑄為農器什物,以給
 江北流民之歸附者。除銅錢渡江之禁。」從之。



 自唐天祐中,兵亂窘乏,以八十五錢為百。後唐天成中,減五錢,漢乾祐初,復減三錢。宋初,凡輸官者亦用八十或八十五為百,然諸州私用則各隨其俗,至有以四十八錢為百者。至是,詔所在用七十七錢為百。



 西北邊內屬戎人,多繼貨帛於秦、階州易銅錢出塞,銷鑄為器。乃詔吏民闌出銅錢百已上論罪,至五貫以上送闕下。



 舊饒州永平監歲鑄錢六萬貫,平江南,增為七萬貫,而銅、鉛、錫常不
 給。轉運使張齊賢訪求得南唐承旨丁釗,能知饒、信等州山谷產銅、鉛、錫,乃便宜調民採取;且詢舊鑄法,惟永平用唐開元錢料最善,即詣闕面陳。八年,詔增市鉛、錫、炭價,於是得銅八十一萬斤、鉛二十六萬斤、錫十六萬斤,歲鑄錢三十萬貫。補釗殿前承旨,領三州銅山。然民間猶雜用舊大小錢。是時,以福建銅錢數少,令建州鑄大鐵錢並行,尋罷鑄,而官私所有鐵錢十萬貫,不出州境,每千錢與銅錢七百七十等,外邑鄰兩浙者亦不用。



 雍熙初,令江南諸州官庫所貯雜錢,每貫及四斤半者送闕下,不及者銷毀。民間惡錢尚多,復申乾德之禁,稍峻其法。京城居民蓄銅器者,限兩月悉送官。



 端拱元年,內侍蕭延皓使嶺南還,以民間私鑄三等錢來上,且言多與蠻人貿易,侵敗禁法。因詔察民私鑄及銷熔好錢作薄惡錢者,並棄市;輒以新惡錢與蠻人博易者,抵罪。



 江北諸州所用錢非甚薄惡者,新舊大小兼用。江南雖用舊大錢,淳化四年,乃詔每貫及前詔斤數、有官監字
 號者皆許用,不分新舊。



 先是,淳化二年,宗正少卿趙安易言:嘗使蜀,見所用鐵錢至輕,市羅一匹,為錢二萬。堅請改鑄一當十大錢,御書錢式,遣詣川峽路諸州冶鑄,所在並為御書錢監;諸州舊貯小鐵錢悉輦送官。民間小錢許送監,計數給以大錢;若改鑄未集,許民大小兼用。既而一歲才成三千餘貫,眾皆以為不便。會安易入奏事,因留不遣,遂罷冶鑄。五年,安易復請,不許。第令川峽仍以銅錢一當鐵錢十。



 荊湖、嶺南民輸稅須大錢,民
 以小錢二或三易大錢一,官屬以奉錢易於民以規利。詔自今吏受民輸,但常所通行錢勿卻,官吏毋得以奉錢換易。至道二年,始禁道、賀州錫,官益其價市之,以給諸路鑄錢。



 咸平初,又申新小錢之禁,令官置場盡市之。舊犯銅禁,七斤以上處死,奏裁多蒙減斷,然待報常淹緩。四年,詔滿五十斤以上取裁,餘從第減。



 景德四年,詔曰:「鼓鑄錢刀,素有程限,憫其勞苦,特示矜寬。自今五月一日至八月一日止收半功,本司每歲量支率分錢以
 備醫藥。」十二月,令鑄匠每旬停作一日。天禧三年,詔:犯銅、瑜石,悉免極刑。



 時銅錢有四監:饒州曰永平,池州曰永豐,江州曰廣寧,建州曰豐國。京師、升鄂杭州、南安軍舊皆有監,後廢之。凡鑄錢用銅三斤十兩,鉛一斤八兩,錫八兩,得錢千,重五斤。唯建州增銅五兩,減鉛如其數。至道中,歲鑄八十萬貫;景德中,增至一百八十三萬貫。大中祥符後,銅坑多不發。天禧末,鑄一百五萬貫。



 鐵錢有三監:邛州曰惠民,嘉州曰豐遠,興州曰濟眾。益州、雅
 州舊亦有監,後並廢。大錢貫十二斤十兩,以準銅錢。嘉、邛二州所鑄錢,貫二十五斤八兩,銅錢一當小鐵錢十兼用。後以鐵重,多盜熔為器,每二十五斤鬻之直二千。大中祥符七年,知益州凌策言:「錢輕則易繼,鐵少則熔者鮮利。」於是詔減景德之制,其見使舊錢仍用如故。歲鑄總二十一萬貫,諸路錢歲輸京師,四方由此錢重而貨輕。



 景祐初,詔三司以江東、福建、廣南盛輸緡錢合三十餘萬易為金帛,錢流民間。



 許申為三司度支判官,建
 議以藥化鐵與銅雜鑄,輕重如銅錢法,銅居三分,鐵六分,皆有奇贏,亦得錢千,費省而利厚。詔申用其法鑄於京師。大率鑄錢雜鉛、錫,則其液流速而易成,申雜以鐵,流澀而多不就,工人苦之。初命申鑄萬緡,逾月裁得萬錢。申性詭譎,少成事,自度言無效,乃求為江東轉運使,欲用其法於江州。朝廷從之,因詔申即江州鑄百萬緡,毋漏其法。中外知其非是,而宰相主之,卒無成功。



 初,太宗改元太平興國,更鑄:「太平通寶」,淳化更鑄,又親書「淳
 化元寶」,作真、行、草三體。後改元更鑄,皆曰「元寶」,而冠以年號,至是改元寶元,文當曰「寶元元寶」,仁宗特命以「皇宋通寶」為文,慶歷以後,復冠以年號如舊。



 自天聖以來,毀錢鑄鐘及為銅器,皆有禁。慶歷初,闌出銅錢,視舊法第加其罪,錢千,為首者抵死。



 五年,泉州青陽鐵冶大發,轉運使高易簡不俟詔,置鐵錢務於泉,欲移銅錢於內地;梓州路轉運使崔輔、判官張固亦請即廣安軍魚子鐵山採礦炭,置監於合州,並銷舊小錢以鑄減輕大錢,
 未得報,先移合州相地置監。州以上聞,朝廷以易簡、輔、固為擅鑄錢,皆坐貶。



 軍興,陜西移用不足,始用知商州皮仲容議,採洛南縣紅崖山、虢州青水冶青銅,置阜民、朱陽二監鑄錢。既而陜西都轉運使張奎、知永興軍範雍請鑄大銅錢與小錢兼行,大錢一當小錢十;又請因晉州積鐵鑄小錢。及奎徙河東,又鑄大鐵錢於晉、澤二州,亦以一當十,助關中軍費。未幾,三司奏罷河東鑄大鐵錢,而陜西復採儀州竹尖嶺黃銅,置博濟監鑄大錢。
 因敕江南鑄大銅錢,而江、池、饒、儀、虢又鑄小鐵錢,悉輦致關中。數州錢雜行,大約小銅錢三可鑄當十大銅錢一,以故民間盜鑄者眾,錢文大亂,物價翔踴,公私患之。於是奎復奏晉、澤、石三州及威勝軍日鑄小鐵錢,獨留用河東。河東鐵錢既行,盜鑄獲利什六,錢輕貨重,患如陜西。知並州鄭戩請河東鐵錢以二當銅錢一,行之一年,又以三當一或以五當一,罷官爐日鑄,且行舊錢。而契丹亦鑄鐵錢,易並邊銅錢。



 慶歷末,葉清臣為三司使,
 與學士張方平等上陜西錢議,曰:「關中用大錢,本以縣官取利太多,致奸人盜鑄,其用日輕。比年以來,皆虛高物估,始增直於下,終取償於上,縣官雖有折當之虛名,乃受虧損之實害。救弊不先自損,則法未易行。請以江南、儀商等州大銅錢一當小錢三,小鐵錢三當銅錢一,河東小鐵錢如陜西,亦以三當一,且罷官所置爐。」自是奸人稍無利,猶未能絕濫錢。其後,詔商州罷鑄青黃銅錢,又令陜西大銅錢、大鐵錢皆以一當二,盜鑄乃止。然
 令數變,兵民耗於資用,類多咨怨,久之始定。方大錢之行,有劉羲叟者語人曰:「是於周景王所鑄無異,上其感心腹之疾乎。」已而果然,語在本傳。



 時興元府西縣增置濟遠監。而韶州天興銅大發,歲採二十五萬斤,詔即其州置永通監。後濟遠監廢,儀州博濟監既廢復置。



 皇祐中,饒、池、江、建、韶五州鑄錢百四十六萬緡,嘉、邛、興三州鑄大鐵錢二十七萬緡。至治平中,饒、池、江、建、韶、儀六州鑄錢百七十萬緡,而嘉、邛以率買鐵炭為擾,自嘉祐四
 年停鑄十年,以休民力。至是,獨興州鑄錢三萬緡。



 熙寧初,同、華二州積小鐵錢凡四十萬緡,詔賜河東,以鐵償之。四年,陜西轉運副使皮公弼奏:「自行當二錢,銅費相當,盜鑄衰息。請以舊銅鉛盡鑄。」詔聽之。自是折二錢遂行於天下。京西轉運使吳幾復建議:郢、唐、均、房、金五州多林木,而銅鉛積於淮南,若由襄、郢轉致郢、唐等州置監鑄錢,可以紓錢重之弊。神宗是之,而王安石沮之,其議遂寢。後乃詔京西、淮南、兩浙、江西、荊湖五路各置鑄
 錢監,江西、湖南十五萬緡、餘路十萬緡為額,仍申熟錢斤重之限。又以興國軍、睦衡舒鄂惠州既置監六,通舊十六監,水陸回遠,增提點之官。



 時諸路大率務於增額:韶惠州永通、阜民監舊額八十萬,至七年,增三十萬,及折二凡五十萬;後衛州黎陽監歲增折二凡五萬緡,西京阜財監歲增市易本錢凡十萬緡,興州濟眾監歲增七萬二千餘緡,陜西三銅錢監各歲增五萬緡。而睦州則置神泉,徐州則置寶豐,梧州以鉛錫易得,萬州以多
 鐵礦,皆置監。又詔秦鳳等路即鳳翔府斜谷置監,已而所鑄錢青銅夾錫,脆惡易毀,罷之。然私錢往往雜用,不能禁,至是法弊,乃詔禁私錢,在官惡錢不堪用者,別為模以鑄。商、虢、洛南增三監,耀、鄜權置兩監,通永興、華、河中、陜舊監為九,以給改鑄。永興、鄜、耀、河中、陜去鐵冶遠,聽改鑄一年罷;商、洛南、華、虢最近鐵冶,聽久置;鄜州等五監候罷改鑄,並其工作歸永興等四監,專鑄大錢,所鑄大鐵錢約補及所廢偽錢,及可以待交子所用而止。



 八年,詔河東鑄錢七十萬緡外,增鑄小錢三十萬緡。於是知太原韓絳請仿陜西令本重模精,以息私鑄之弊。



 初,薛向鑄鐵錢於陜西,後許彥先鑄於廣南。既而民不便用,神宗欲遂罷之,王安石固爭,乃詔京師畿內並罷,其行於四方蓋如故。元豐以後,西師大舉,邊用匱闕,徐州置寶豐下監,歲鑄折二錢二十萬緡,轉移陜府。



 於時,同、渭、秦、隴等州錢監,廢置移徙不一,銅鐵官多建言鑄錢,事不盡行,而又自弛錢禁,民之銷毀與夫闌出境外
 者為多。張方平嘗極諫曰:「禁銅造幣,盜鑄者抵罪至死,示不與天下共其利也。故事,諸監所鑄錢悉入於王府,歲出其奇羨給之三司,方流布於天下。然自太祖平江南,江、池、饒、建置爐,歲鼓鑄至百萬緡。積百年所入,宜乎貫朽於中藏,充足於民間矣。比年公私上下並苦乏錢,百貨不通,人情窘迫,謂之錢荒。不知歲所鑄錢,今將安在。夫鑄錢禁銅之法舊矣,令敕具載,而自熙寧七年頒行新敕,刪去舊條,削除錢禁,以此邊關重車而出,海舶
 飽載而回,聞沿邊州軍錢出外界,但每貫收稅錢而已。錢本中國寶貨,今乃與四夷共享,又自廢罷銅禁,民間銷毀無復可辦。銷熔十錢得精銅一兩,造作器用,獲利五倍。如此則逐州置爐,每爐增數,是猶畎澮之益,而供尾閭之洩也。」



 元豐八年,哲宗嗣位,復申錢幣闌出之禁,如嘉祐編敕;罷徐州寶豐鼓鑄;詔戶部條諸監之可減者,凡增置鑄錢監十四皆罷之。



 陜西行鐵錢,至陜府以東即銅錢地,民以鐵錢換易,有輕重不等之患。元祐六
 年,乃議限東行,有稅物者以十分率之,止許易二分,人毋得過五千。八年,命公私給納、貿易並專用鐵錢,而官帑銅錢以時計置,運致內郡,商旅願於陜西內郡入便銅錢,給據請於別路者聽。仍定加饒之數,每百緡,河東、京西加饒三千,在京、餘路四千。



 先是,太祖時取唐飛錢故事,許民入錢京師,於諸州便換。其法:商人入錢左藏庫,先經三司投牒,乃輸於庫。開寶三年,置便錢務,令商人入錢詣務陳牒,即輦致左藏庫,給以券,仍敕諸州凡
 商人繼券至,當日給付,違者科罰。至道末,商人入便錢一百七十餘萬貫,天禧末,增一百一十三萬貫。至是,乃復增定加饒之數行焉。



 折二銅錢又定鉤致之法。初欲復舊,止行於本路。議者謂:「關東諸路既已通行,奪彼予此,理亦非便。且陜右所用折二鐵錢,止當一小銅錢,即折二銅錢盡歸陜西,不直般運費廣,猝難鉤致,且與鐵錢一等,慮鐵錢轉更加輕。」乃令折二銅錢寬所行地,聽行於陜西一路,及河東晉、絳、石、慈、隰州,京西西京、河陽、
 許、汝、鄭、金、房、均、鄧等州,餘路則禁。仍限二年毋更用,在民間者聽以輸買納,在官帑者以輸上供,即非沿流地或數無上供者,所隸運司移發輸京師。尋詔更鑄小銅錢。河東安撫、提刑司言:「頃絳州垣曲縣置監鼓鑄銅錢,費且不給,今已廢監,又禁折二銅錢不通行,非便。」乃聽行使如舊。



 供備庫使鄭價使契丹還,言其給輿箱者錢,皆中國所鑄。乃增嚴三路闌出之法。



 熙、豐間銅鐵錢嘗並行,銅錢千易鐵錢千五百,未聞輕重之弊。及後銅錢
 日少,鐵錢滋多。紹聖初,銅錢千遂易鐵錢二千五百,鐵錢浸輕。元符二年,下陜西諸路安撫司博究利害。於是詔陜西悉禁銅錢,在民間者令盡送官,而官銅悉取就京西置監。永興帥臣陸師閔言:「既揀毀私錢,禁銅罷冶,則物價當減。願下陜西州縣,凡有市買,並準度銅錢之直,以平其價。」詔用其言,而豪賈富家多不便。



 徽宗嗣位,通判鳳州馬景夷言:「陜西自去年罷使銅錢,續追官措置錢法,未聞有深究錢弊輕重灼見利害者。銅錢流注
 天下,雖千百年未嘗有輕重之患。獨鐵錢局於一路,所可通交易有無者,限以十州之地,欲無滯礙,安可得乎?又諸州錢監鼓鑄不已,歲月增多,以鼓鑄無窮之錢,而供流轉有限之用,更數十年,積滯一隅,暴如丘山,公私為害,又倍於今日矣。謂宜弛其禁界,許鄰近陜西、河東等路特不入京城外,凡解鹽地州縣並許通行折二鐵錢。如此則流注無窮,久遠自無輕重之患。」繼而言者謂:「鐵錢重滯,難以繼遠,民間皆願復用銅錢。當公私匱乏
 之時,諸路州縣官私銅錢積貯萬數,反無所用。」乃詔銅鐵錢聽民間通行,而銅錢止用糴買。



 建中靖國元年,陜西轉運副使孫傑以鐵錢多而銅錢少,請復鑄銅錢,候銅鐵錢輕重稍均,即聽兼鑄。崇寧元年,前陜西轉運判官都貺復請權罷陜西鑄鐵錢。戶部尚書吳居厚言:「江、池、饒、建錢額不敷,議減銅增鉛、錫,歲可省銅三十餘萬斤,計增鑄錢十五萬九千餘緡。所鑄光明堅韌,與見行錢不異。」詔可。然課猶不登。二年,居厚乃請檢用前後上
 供鑄錢條約,視其登耗之數,別定勸沮之法。



 會蔡京當政,將以利惑人主,托假紹述,肆為紛更。有許天啟者,京之黨也,時為陜西轉運副使,迎合京意,請鑄當十錢。五月,始令陜西及江、池、饒、建州,以歲所鑄小平錢增料改鑄當五大銅錢,以「聖宋通寶」為文,繼而並令舒、睦、衡、鄂錢監,用陜西式鑄折十錢,限今歲鑄三十萬緡,鐵錢二百萬緡。募私鑄人一為官匠,並其家設營以居之,號鑄錢院,謂得昔人招天下亡命即山鑄錢之意。所鑄銅錢
 通行諸路,而陜西、河東、四川系鐵錢地者禁之,第鑄於陜西鐵錢地而已。



 自熙寧以來,折二錢雖行民間,法不許運致京師,故諸州所積甚多。至是,發運司因請以官帑所有折二錢改鑄折十錢。三年,遂罷鑄小平錢及折五錢。置監於京城所,復徐州寶豐、衛州黎陽監,並改鑄折二錢為折十,舊折二錢期一歲勿用。大嚴私鑄之令,民間所用瑜石器物,並官造鬻之,輒鑄者依私有法加二等。命諸路轉運司於沿流順便地,隨宜增置錢監,俾
 民以所有折二錢換納於官,運致所增監改鑄折十錢。二廣產鐵,令鼓鑄小鐵錢,止行於兩路;其公私銅錢兌換運輸元豐庫,仍於潯州置鐵錢監,依陜西料例鑄當二錢。



 四年,立錢綱驗樣法。崇寧監以所鑄御書當十錢來上,緡用銅九斤七兩有奇,鉛半之,錫居三之一。詔頒其式於諸路,令赤仄及烏背,書畫分明。時趙挺之為門下侍郎,繼拜右僕射,與蔡京議多不合,因極言當十錢不便,私鑄浸廣。乃令提刑司歲較巡捕官一路所獲多
 寡,繼令福建、廣南毋行用,第鑄以上供及給他路。凡為人附帶若封識影庇私鑄錢者,悉論以法,毋得蔭贖。其置鑄錢院,蓋將以盡收所在亡命盜鑄之人,然犯法者不為止。乃命荊湖南北、江南東西、兩浙並以折十錢為折五,舊折二錢仍舊。慮冒法入東北也,今以江為界,淮南重寶錢亦作當五用焉。



 五年,兩浙盜鑄尤甚,小平錢益少,市易濡滯。遂命以折五、折十上供,小平錢留本路;江、池、饒、建、韶州錢監,歲課以八分鑄小平錢,二分鑄當
 十錢。俄詔廣南、江南、福建、兩浙、荊湖、淮南用折二錢改鑄折十錢皆罷,其創置鑄錢院及招置錢戶並停。繼復罷鑄當十二分之令,盡鑄小平錢。荊湖、江南、兩浙、淮南重寶錢作當三,在京、京畿、京東西、河東、河北、陜西、熙河作當五。通寶錢所鑄未多,在官者悉封樁,在民間者以小平錢納換。旋復詔京畿、京東西、河北、河東、陜西、熙河當十錢仍舊,兩浙作當三,江南、淮南、荊湖作當五。時錢幣苦重,條序不一,私鑄日甚。御史沈畸奏曰:「小錢便民
 久矣。古者軍興,錫賞不繼,或以一當百,或以一當千。此權時之宜,豈可行於太平無事之日哉?當十鼓鑄,有數倍之息,雖日漸之,其勢不可遏。」未幾,詔當十錢止行於京師、陜西、河東、河北。俄並畿內用之,餘路悉禁。期一季送官,償以小錢,換納到者輸於元豐、崇寧庫,而私錢亦限一季自致,計銅直增二分,償以小錢,隱藏者論如法。尋詔鄭州、西京亦聽用折十錢,禁貿易為二價者。東南諸監增鑄小平錢,以待償錢,而私錢亦改鑄焉。



 折十錢
 為幣既重,一旦更令,則民驟失厚利。又諸路或用或否,往往不盡輸於官,冒法私販。始令四輔、畿內、開封府許搜索舟車,賞視舊法增倍。水陸所由,官司失察者皆停替,而受納不揀選、容私錢其間者,以差定罪法。又以私錢猥多,不能悉禁,乃令外路每一私錢,計小平錢三,以小錢易於官,在京以四小平錢易之。京師出納及民間貿易,並大小錢參用,而私鑄小平錢輒行用。立搜索告捕罪賞,越江、淮入汴錢至京者,一依當十錢法。御史張
 茂直請嚴私販當十之令,綱舟載卸,皆選官監索,保無藏匿,舟車兜擔,即疑慮私販者,並聽搜索;而福建民或私鑄轉入淮、浙、京東等路者,所由州縣官司皆治漏逸之罪,不以赦免。法滋密矣。



 大觀元年,張茂直復言:「州縣督捕加峻,私小黃錢投委江河,不敢復出。請令東南州縣置水匱封鍵於闤闠中,聽民以私錢自投,如自首法。當三、當五錢,舟船附帶者,亦多棄之江河,請下諸路撈漉。」



 時蔡京復相,再主用折十錢。二月,首鑄御書當十錢,
 以京畿錢監所得私錢改鑄。尋興復京畿兩監,以轉運使宋喬年領之,用提舉京畿鑄錢司為名。喬年鑄烏背漉銅錢來上,詔以漉銅式頒行諸路。



 京之初為折十錢,人不以為便,帝亦知之。故崇寧四年以後,稍更其法,及京去位,遂詔諭中外。京再得政復行之,知盜鑄者必眾,將威以刑。會有告蘇州章綖盜鑄數千萬緡,遂興大獄。初遣李孝壽,又遣沉畸、蕭服,末以命知蘇州孫傑、發運副使吳擇仁。綖坐刺流海島,連坐者十餘人,時皆冤之。
 於是頒行大觀新修錢法於天下,申命開封府尹少、外路監司,各分州郡舉行,按舉能否,月檢會法令,使民知禁。用孫傑言,盜鑄依淮東重法地,囊橐強盜之家,籍其財以待賞,居停鄰保並均備告驗;私錢依私茶法;給隨行物;州常樁盜鑄賞錢五千饒,州縣稽於施行,監司失察,不以赦原。是歲,京畿既置錢監,乃專鑄當十大錢,而小平錢則鑄於諸路。既而當十錢少,復置真州鑄錢監,以本路所換錢不依式者及諸司當二見緡,用舊式改
 鑄當十錢。



 明年,令江、池、饒、建州錢監,自來歲以鑄當十五分鑄小平錢。申嚴私鑄之法,即托權要事勢,度越關津,拒捍搜索者,雖輕以違制論,載御物者同之。初,崇寧五年,始禁陜西鐵錢行於興元府等界。至是,又以鐵錢猥多,禁陜西鐵錢入蜀。有董奎者,為走馬承受,遂令以鐵錢三折銅錢一。事聞,責奎以妄肆胸臆,致幣輕物重,奎遂即罪。



 三年,申當十錢行使之令,益以京東、京西,而河北並邊州縣鎮砦、四榷場及登、萊、密州緣海縣鎮等
 皆禁。時蔡京復罷政矣。四年,詔:「鼓鑄當十錢多,慮法隨以弊,其止鑄舊額小平錢。」張商英為相,奏言:「當十錢為害久矣。舊小平錢有出門之禁,故四方客旅之貨,交易得錢,必大半入中末鹽鈔,收買告牒,而餘錢又流布在市井,此上下內外交相養。自當十錢行,以一夫而負八十千,小車載四百千,錢既為輕繼之物,則告牒為滯貨,鹽鈔非得虛抬之息則不行。臣今欲借內庫並密院諸司封樁綢絹、金銀並鹽鈔,下令折十錢限民半年所在
 送官,十千給銀絹各一匹兩,限竟毋更用。俟錢入官,擇其惡者鑄小平錢,存其好者折三行用。如此則錢法、鈔法不相低昂,可以復舊。」



 利州路提刑司言:「舊銅鐵錢輕重相尋,以大鐵錢一折小銅錢二;今大鐵錢五止當一銅錢,比舊輕十倍。又流入川界,錢輕物重,頗類陜西。欲將折二大鐵錢以一折一,雖稍減錢數,錢必稍重。」詔許陜西鐵錢入蜀仍舊,盡釋其禁,且命以今物價量宜裁之。



 政和元年詔:「錢重則物輕,錢輕則物重,其勢然也。今
 諸路所鑄小平錢,行之久而無弊,多而不壅,為利博矣。往歲圖利之臣鼓鑄當十錢,茍濟目前,不究悠久,公私為害,用之幾十年,其法日弊而不勝。奸猾之民規利冒法,銷毀當二、小平錢,所在盜鑄,濫錢益多,百物增價。若不早革,即弊無已時。其官私見在當十錢,可並作當三,以為定制。尚慮豪猾憚於折閱,胥動浮言,可內自京尹,外逮監司、郡縣,悉心開諭。」



 自當十錢行,抵冒者多。大觀四年,星變,赦天下。凡以私錢得罪,有司上名數,亡慮十
 餘萬人。蔡京罔上毒民,可謂烈矣。時御府之用日廣,東南錢額不敷,宣和以後尤甚。乃令饒、贛錢監鑄小平錢,每緡用鐵三兩,而倍損其銅,稍損其鉛。繼又令江、池、饒錢監,盡以小平錢改鑄當二錢,以紓用度,然有司猶數告之。靖康元年,罷政和敕陜西路用銅錢斷徒二年配千里法。



 初,蔡京主行夾錫錢,詔鑄於陜西,亦命轉運副使許天啟推行。其法以夾錫錢一折銅錢二,每緡用銅八斤,黑錫半之,白錫又半之。既而河東轉運使洪中孚
 請通行於天下,京欲用其言,會罷政。大觀元年,京復相,遂降錢式及錫母於鑄錢之路,鑄錢院專用鼓鑄,若產銅地始聽兼鑄小平錢。復用轉運司及提刑司參領其事,衡州熙寧、鄂州寶泉、舒州同安監暨廣南皆鑄焉。二年,江南東西、福建、兩浙許鑄使鐵錢。三年,京復罷政,詔以兩浙鑄夾錫錢擾民,凡東南所鑄皆罷。明年,並河北、河東、京東等路罷之,所在監、院皆廢。惟河東三路聽存舊監,以鑄銅、鐵錢;產銅郡縣聽存,用改鑄小平錢。



 政和
 元年,錢輕物重,細民艱食,詔:「應陜西舊行使鐵錢地,並依元豐年大鐵錢折二,公私通行,夾錫錢同之,毋得分別。見存鐵錢,毋改更鑄夾錫,河東官私折二、夾錫錢同之。」



 童貫宣撫陜西,以詔亟平物價,帥臣徐處仁切責其非,坐貶。錢即經略鄜延,抗疏言:「詳考詔旨,謂鐵錢復行,與夾錫並用。慮奸民妄作輕重,欲維持推行,俾錢物相直,非欲以威力脅制百姓,頓減物價於一兩月之間。今宣撫司裁損米穀、布帛、金銀之價,殆非人情。徐處仁言
 雖未盡,所見為長,望速詢其實。如臣言乖謬,願同處仁貶。」詔即妄有建明,毀辱使命,謫置偏州。尋亦罷行夾錫錢,且禁裁物價,民商貿易,各從其便。繼而童貫復請與舊法鐵錢並折二通行。知閿鄉縣論九齡俄坐以銅錢一估夾錫錢七八,並知州王寀、轉運副使張深俱被劾。時關中錢甚輕,夾錫欲以重之,其實與鐵錢等,物價日增,患甚於當十。



 二年,蔡京復得政,條奏廣、惠、康、賀、衡、鄂、舒州昨鑄夾錫錢精善,請復鑄如故。廣西、湖北、淮東如
 之,且命諸路以銅錢監復改鑄夾錫,遂以政和錢頒式焉。夾錫錢既復推行,錢輕不與銅等,而法必欲其重,乃嚴擅易抬減之令。凡以金銀、絲帛等物貿易,有弗受夾錫、須要銅錢者,聽人告論,以法懲治。市井細民朝夕鬻餅餌熟食以自給者,或不免於告罰。未幾,以夾錫錢不以何路所鑄,並聽通行。



 陜西用「政和通寶」舊大鐵錢,與夾錫錢雜。慮流轉諸路,四年,詔毋更行用,致令諸監改鑄夾錫錢,在民間者赴官換綢。鄭居中、劉正夫為相,以
 為不便,令淮南夾錫錢期三日官私俱禁不用,仍罷鼓鑄,夾錫錢悉輦樁關中。尋詔河東、陜西外,餘路並罷;俄詔並河東罷鑄夾錫錢,止用舊法鼓鑄。重和元年,權罷京西鑄夾錫錢,繼以關中糴買,用之通流,復命鼓鑄,專給關中。夾錫行,小民往往以藥點染,與銅錢相亂,河北漕臣張翬等嘗坐貶焉。



 先是,江池饒州、建寧府四監,歲鑄錢百三十四萬緡,充上供;衡、舒、嚴、鄂、韶、梧州六監,歲鑄錢百五十六萬緡,充逐路支用。建炎經兵,鼓鑄皆廢。
 紹興初,並廣寧監於虔州,並永豐監於饒州,歲鑄才及八萬緡。以銅、鐵、鉛、錫之入,不及於舊,而官吏稍廩工作之費,視前日自若也,每鑄錢一千,率用本錢二千四百文。時範汝為作亂,權罷建州鼓鑄,尋復舊,泉司供給銅、錫六十五萬餘斤。



 六年,斂民間銅器,詔民私鑄銅器者徒二年。贛、饒二監新額錢四十萬緡,提點官趙伯瑜以為得不償費,罷鼓鑄,盡取木炭銅鉛本錢及官吏闕額衣糧水腳之屬,湊為年計。十三年,韓球為使,復鑄新錢,
 興廢坑治,至於發塚墓,壞廬舍,籍冶戶姓名,以膽水盛時浸銅之數為額。



 浸銅之法:以生鐵鍛成薄片,排置膽水槽中浸漬數日,鐵片為膽水所薄,上生赤煤,取刮鐵煤入爐,三煉成銅。大率用鐵二斤四兩,得銅一斤。饒州興利場、信州鉛山場各有歲額,所謂膽銅也。



 無銅可輸者,至熔錢為銅,然所鑄亦才及十萬緡。



 二十四年,罷鐵錢司歸之漕司。二十七年,出版曹錢八萬緡為鑄本,歲權以十五萬緡為額。復饒、贛、韶鑄錢監,以漕臣往來措置,通判主之。殿中侍御史王珪言泉司不可廢,復以戶部侍郎榮薿提領,許置官屬二員。二十
 八年,出御府銅器千五百事付泉司,大索民間銅器,得銅二百餘萬斤,寺觀鐘、磬、鐃、鈸既籍定投稅外,不得添鑄。二十九年,令命官之家留見錢二萬貫,民庶半之,餘限二年聽轉易金銀,算請茶、鹽、香、礬鈔引之類,越數寄隱,許人告。



 以李植提點鑄錢公事,植言:「歲額內藏庫二十三萬緡,右藏庫七十餘萬緡,皆至道以後數也。紹興以來,歲收銅二十四萬斤,鉛二十萬斤,錫五百斤,僅可鑄錢一十萬緡。諸道拘到銅器二百萬斤,附以鉛、錫,可
 鑄六十萬緡。然拘者不可以常,唯當據坑冶所產。」下工部,權以五十萬緡為額。又明年,才鑄及十萬緡。今泉司歲額增至十五萬緡,小平錢一萬八千緡,折二錢六萬六千緡。歲費鑄本及起綱糜費約二十六萬緡,司屬之費又約二萬緡,東南十一路一百一十八州之所供,有坑冶課利錢、木炭錢、錫本錢,約二十一萬緡,比歲所收不過十五六萬緡耳。歲額:金一百二十八兩,銀無額,以七分入內庫,三分歸本司,銅三十九萬五千八百斤,鉛
 三十七萬七千九百斤,錫一萬九千八百七十五斤,鐵二百三十二萬八千斤,比歲所榷十無二三。每當二錢千,重四斤五兩;小平錢千,重四斤十三兩;視舊制,銅少鉛多,錢愈鍥薄矣。



 孝宗隆興元年,詔鑄當二、小平錢,如紹興之初。乾、淳迄於嘉泰、開禧皆如之。



 乾道六年,並鑄錢司歸發運司,尋復置。八年,饒州、贛州復名置提點官。以新鑄錢殽雜,提點鑄錢及永平監官、左藏西庫監官、戶部工部長貳官責降有差。九年,大江之西及湖、廣間多毀錢,夾以
 沙泥重鑄,號「沙尾錢」,詔嚴禁之。淳熙二年,並贛司歸饒州。慶元三年,復禁銅器,期兩月鬻於官,每兩三十。湖州舊鬻監,至是官自鑄之。



 二年,禁銷錢為銅器者,以違制論,爐戶決配海外。



 復神泉監,以所括銅器鑄當三大錢,隸工部。



 舊額,內帑歲收新錢一百五萬,江、池、饒、建四監。



 而每年退卻六十萬,三年一郊,又以三百萬輸三司,是內帑年才得十一萬六千餘緡,而左藏得九十三萬三千餘緡。今歲額止十五萬,而隸封樁者半,內藏者半,左藏咸無焉。



 又自置市舶於浙、於
 閩、於廣,舶商往來,錢寶所由以洩,是以自臨安出門,下江海,皆有禁。淳熙九年,詔廣、泉、明、秀漏洩銅錢,坐其守臣。嘉定元年,三省言:「自來有市舶處,不許私發番船。紹興末,臣僚言:泉、廣二舶司及西、南二泉司,遣舟回易,悉載金錢。四司既自犯法,郡縣巡尉其能誰何?至於淮、楚屯兵,月費五十萬,見緡居其半,南北貿易緡錢之入敵境者,不知其幾。於是沿邊皆用鐵錢矣。」



 淮南舊鑄銅錢。乾道初,詔兩淮、京西悉用鐵錢。荊門隸湖北,以地接襄、
 峴,亦用鐵錢。六年,先是,以和州舊有錢監,舒州山口鎮亦有古監,詔司農丞許子中往淮西措置。於是子中以舒、蘄、黃皆產鐵,請各置監,舒州同安監,蘄州新春監,黃州齊安監。



 且鑄折二錢。以發運司通領四監。江之廣寧監,興國之大冶監,臨江之豐餘監,撫之裕國監。



 子中所領三監,歲合認三十萬貫,其大小鐵錢,令兩淮通行。七年,舒、蘄守臣皆以鑄錢增美遷官,然淮民為之大擾。八年,以江州、興國軍鐵冶額虧,守貳及大冶知縣各降一官。



 淳熙五年,詔舒州歲增鑄十萬貫,以三十萬
 貫為額;蘄州增鑄五萬貫,以十五萬貫為額。如更增鑄,優與推賞。御史黃洽言:「興天下之利者,不窮天下之力。舒、蘄歲鑄四十五萬不易為也。又有增鑄之賞,恐其難繼。」詔除之。八年,以舒州水遠,薪炭不便,減額五萬貫。明年,又減十萬貫,與蘄州並以十五萬貫為額。十年,並舒州之宿城監入同安監。十二年,詔舒、蘄鑄鐵錢,並增五萬貫,以「淳熙通寶」為文。



 光宗紹熙二年,減蘄春、同安兩監歲鑄各十萬貫。嘉泰三年,罷舒、蘄鼓鑄;開禧三年,復
 之。



 嘉定五年,臣僚言江北以銅錢一折鐵錢四,禁之。時銅錢之在江北者,自乾道以來,悉以鐵錢易之,或以會子一貫易銅錢一貫。其銅錢輸送行在及建康、鎮江府。凡沿江私度及邊徑嚴禁漏洩,及於邊界三里內立堠,如出界法;其易京西銅錢,如兩淮例。京西、湖北之鐵錢,則取給於漢陽監及興國富民監,後並富民監於漢陽監,以二十萬為額。



 前宋時,川、陜皆行鐵錢,益、利、夔皆即山冶鑄。紹興九年,詔陜西諸路復行鐵錢。十五年,置利
 州紹興監,鑄錢十萬緡以救錢引。二十二年,復嘉之豐遠、邛之惠民二監,鑄小平錢。二十三年,詔利州並鑄折二錢,後又鑄折二錢。淳熙十五年,四川餉臣言:「諸州行使兩界錢引,全籍鐵錢稱提,止有利州紹興監歲鑄折三錢三萬四千五百貫有奇,邛州惠民監歲鑄折三錢一萬二千五百貫。今大安軍淳熙、新興、迎恩三爐,出生鐵四十九萬三千斤,利之昭化、嘉川縣亦有爐,新產鐵三十餘萬斤。乞從鼓鑄。」嘉定元年,即利州鑄當五大錢。
 三年,制司欲盡收舊引,又於紹興、惠民二監歲鑄三十萬貫,其料並同當三錢。若四川銅錢,淳熙間易送湖廣總所儲之,後又交卸於江陵。



 寶慶元年,新錢以「大宋元寶」為文。端平元年,以膽銅所鑄之錢不耐久,舊錢之精致者洩於海舶,申嚴下海之禁。嘉熙元年,新錢當二並小平錢並以「嘉熙通寶」為文,當三錢以「嘉熙重寶」為文。



 淳祐四年,右諫議大夫劉晉之言:「巨家停積,猶可以發洩;銅器鈹銷,猶可以上遏;唯一入海舟,往而不返。」於是
 復申嚴漏洩之禁。



 八年,監察御史陳求魯言:「議者謂楮便於運轉,故錢廢於蟄藏;自稱提之屢更,圜法為無用。急於扶楮者,至嗾盜賊以窺人之閫奧,峻刑法以發人之窖藏,然不思患在於錢之荒,而不在於錢之積。夫錢貴則物宜賤,今物與錢俱重,此一世之所共憂也。蕃舶巨艘,形若山嶽,乘風駕浪,深入遐陬。販於中國者皆浮靡無用之異物,而洩於外夷者乃國家富貴之操柄。所得幾何,所失者不可勝計矣。京城之銷金,衢、信之瑜器,
 醴、泉之樂具,皆出於錢。臨川、隆興、桂林之銅工,尤多於諸郡。姑以長沙一郡言之,烏山銅爐之所六十有四,麻潭鵝羊山銅戶數百餘家,錢之不壞於器物者無幾。今京邑瑜銅器用之類,鬻賣公行於都市。畿甸之近,一繩以法,由內及外,觀聽聿新,則鈹銷之奸知畏矣。香、藥、象、犀之類異物之珍奇可悅者,本無適用之實,服御之間昭示儉德,自上化下,風俗丕變,則漏洩之弊少息矣。此端本澄源之道也。」有旨從之。



 十年,以會價低減,復申嚴
 下海之禁。十二年,申嚴鈹銷之禁及偽造洩之法。咸淳元年,復申嚴鈹銷、漏禁。寶祐元年,新錢以「皇宋元寶」為文。



\end{pinyinscope}