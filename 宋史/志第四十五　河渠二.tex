\article{志第四十五 河渠二}

\begin{pinyinscope}

 黃河中



 熙寧四年七月辛卯,北京新堤第四、第五埽決,漂溺館陶、永濟、清陽以北,遣茂則乘驛相視。八月,河溢澶州曹村,十月,溢衛州王供。時新堤凡六埽,而決者二,下屬恩、
 冀,貫御河,奔沖為一。帝憂之,自秋迄冬,數遣使經營。是時,人爭言導河之利,茂則等謂:「二股河地最下,而舊防可因,今堙塞者才三十餘里,若度河之湍,浚而逆之,又存清水鎮河以析其勢,則悍者可回,決者可塞。」帝然之。



 十二月,令河北轉運司開修二股河上流,並修塞第五埽決口。五年二月甲寅,興役,四月丁卯,二股河成,深十一尺,廣四百尺。方浚河則稍障其決水,至是,水入於河,而決口亦塞。



 六月,河溢北京夏津。閏七月辛卯,帝語執
 政:「聞京東調夫修河,有壞產者,河北調急夫尤多;若河復決,奈何?且河決不過占一河之地,或西或東,若利害無所校,聽其所趨,如何?」王安石曰:「北流不塞,占公私田至多,又水散漫,久復澱塞。昨修二股,費至少而公私田皆出,向之瀉鹵,俱為沃壤,庸非利乎。況急夫已減於去歲,若復葺理堤防,則河北歲夫愈減矣。」



 六年四月,始置疏浚黃河司。先是,有選人李公義者,獻鐵龍爪揚泥車法以浚河。其法:用鐵數斤為爪形,以繩系舟尾而沈之
 水,篙工急棹,乘流相繼而下,一再過,水已深數尺。宦官黃懷信以為可用,而患其太輕。王安石請令懷信、公義同議增損,乃別制浚川杷。其法:以巨木長八尺,齒長二尺,列於木下如杷狀,以石壓之;兩旁系大繩,兩端釘大船,相距八十步,各用滑車絞之,去來撓蕩泥沙,已又移船而浚。或渭水深則杷不能及底,雖數往來無益;水淺則齒礙沙泥,曳之不動,卒乃反齒向上而曳之。人皆知不可用,惟安石善其法,使懷信先試之以浚二股,又謀
 鑿直河數里以觀其效。且言於帝曰:「開直河則水勢分。其不可開者,以近河,每開數尺即見水,不容施功爾。今第見水即以杷浚之,水當隨杷改趨直河,茍置數千杷,則諸河淺澱,皆非所患,歲可省開浚之費幾百千萬。」帝曰:「果爾,甚善。聞河北小軍壘當起夫五千,計合境之丁,僅及此數,一夫至用錢八緡。故歐陽修嘗謂開河如放火,不開如失火,與其勞人,不如勿開。」安石曰:「勞人以除害,所謂毒天下之民而從之者。」帝乃許春首興工,而賞
 懷信以度僧牒十五道,公義與堂除;以杷法下北京,令虞部員外郎、都大提舉大名府界金堤範子淵與通判、知縣共試驗之,皆言不可用。會子淵以事至京師,安石問其故,子淵意附會,遽曰:「法誠善,第同官議不合耳。」安石大悅。至是,乃置浚河司,將自衛州浚至海口,差子淵都大提舉,公義為之屬。許不拘常制,舉使臣等;人船、木鐵、工匠,皆取之諸埽;官吏奉給視都水監丞司;行移與監司敵體。



 當是時,北流閉已數年,水或橫決散漫,常虞
 壅遏。十月、外監丞王令圖獻議,於北京第四、第五埽等處開修直河,使大河還二股故道,乃命範子淵及朱仲立領其事。開直河……



 七年,都水監丞劉璯言:「自開直河,閉魚肋,水勢增漲,行流湍急,漸塌河岸,而許家港、清水鎮河極淺漫,幾於不流。雖二股深快,而蒲泊已東,下至四界首,退出之田,略無固護,設遇漫水
 出岸,牽回河頭,將復成水患。宜候霜降水落,閉清水鎮河,築縷河堤一道以遏漲水,使大河復循故道。又退出良田數萬頃,俾民耕種。而博州界堂邑等退背七埽,歲減修護之費,公私兩濟。」從之。是秋,判大名文彥博言:「河溢壞民田,多者六十村,戶至萬七千,少者九村,戶至四千六百,願蠲租稅。」從之。又命都水詰官吏不以水災聞者。外都水監丞程昉以憂死。



 十月,安石去位,吳充為相。十年五月,滎澤河堤急,詔判都水監俞光往治之。是歲
 七月,河復溢衛州王供及汲縣上下埽、懷州黃沁、滑州韓村;已丑,遂大決於澶州曹村,澶淵北流斷絕,河道南徙,東匯於梁山、張澤濼,分為二派,一合南清河入於淮,一合北清河入於海,凡灌郡縣四十五,而濮、齊、鄆、徐尤甚,壞田逾三十萬頃。遣使修閉。



 八月,又決鄭州滎澤。於是文彥博言:「臣正月嘗奏:德州河底淤澱,洩水稽滯,上流必至壅遏。又河勢變移,四散漫流,兩岸俱被水患,若不預為經制,必溢魏、博、恩、澶等州之境。而都水略無施
 設,止固護東流北岸而已。適累年河流低下,官吏希省費之賞,未嘗增修堤岸,大名諸埽,皆可憂虞。謂如曹村一埽,自熙寧八年至今三年,雖每計春料當培低怯,而有司未嘗如約,其埽兵又皆給他役,實在者十有七八。今者果大決溢,此非天災,實人力不至也。臣前論此,並乞審擇水官。今河朔、京東州縣,人被患者莫知其數,嗷嗷籲天,上軫聖念,而水官不能自訟,猶汲汲希賞。臣前論所陳,出於至誠,本圖補報,非敢激訐也。」



 元豐元年四
 月丙寅,決口塞,詔改曹村埽曰靈平。五月甲戌,新堤成,閉口斷流,河復歸北。初議塞河也,故道堙而高,水不得下,議者欲自夏津縣東開簽河入董固以護舊河,袤七十里九十步;又自張村埽直東築堤至龐家莊古堤,袤五十里二百步。詔樞密都承旨韓縝相視。縝言:「漲水沖刷新河,已成河道。河勢變移無常,雖開河就堤,及於河身創立生堤,枉費功力。惟增修新河,乃能經久。」詔可。



 十一月,都水監言:「自曹村決溢,諸埽無復儲蓄,乞給錢二
 十萬緡下諸路,以時市梢草封樁。」詔給十萬緡,非朝旨及埽岸危急,毋得擅用。



 二年七月戊子,範子淵言:「因護黃河岸畢工,乞中分為兩埽。」詔以廣武上、下埽為名。



 三年七月,澶州孫村、陳埽及大吳、小吳埽決,詔外監丞司速修閉。初,河決澶州也,北外監丞陳祐甫謂:「商胡決三十餘年,所行河道,填淤漸高,堤防歲增,未免泛濫。今當修者有三:商胡一也,橫□二也,禹舊跡三也。然商胡、橫□故道,地勢高平,土性疏惡,皆不可復,復亦不能持久。
 惟禹故瀆尚存,在大伾、太行之間,地卑而勢固。故秘閣校理李垂與今知深州孫民先皆有修復之議。望召民先同河北漕臣一員,自衛州王供埽按視,訖於海口。」從之。



 四年四月,小吳埽復大決,自澶注入御河,恩州危甚。六月戊午,詔:「東流已填淤不可復,將來更不修閉小吳決口,候見大河歸納,應合修立堤防,令李立之經畫以聞。」帝謂輔臣曰:「河之為患久矣,後世以事治水,故常有礙。夫水之趨下,乃其性也,以道治水,則無違其性可也。
 如能順水所向,遷徙城邑以避之,復有何患?雖神禹復生,不過如此。」輔臣皆曰:「誠如聖訓。」河北東路提點刑獄劉定言:「王莽河一徑水,自大名界下合大流注冀州,及臨清徐曲御河決口、恩州趙村壩子決口兩徑水,亦注冀州城東。若遂成河道,即大流難以西傾,全與李垂、孫民先所論違背,望早經制。」詔送李立之。



 八月壬午,立之言:「臣自決口相視河流,至乾寧軍分入東、西兩塘,次入界河,於劈地口入海,通流無阻,宜修立東西堤。」詔覆計
 之。而言者又請:「自王供埽上添修南岸,於小吳口北創修遙堤,候將來礬山水下,決王供埽,使直河注東北,於滄州界或南或北,從故道入海。」不從。



 九月庚子,立之又言:「北京南樂、館陶、宗城、魏縣,淺口、永濟、延安鎮,瀛州景城鎮,在大河兩堤之間,乞相度遷於堤外。」於是用其說,分立東西兩堤五十九埽。定三等向著:河勢正著堤身為第一,河勢順流堤下為第二,河離堤一里內為第三。退背亦三等:堤去河最遠為第一,次遠者為第二,次近
 一里以上為第三。立之在熙寧初已主立堤,今竟行其言。



 五年正月己丑,詔立之:「凡為小吳決口所立堤防,可按河勢向背應置埽處,毋虛設巡河官,毋橫費工料。」六月,河溢北京內黃埽。七月,決大吳埽堤,以紓靈平下埽危急。八月,河決鄭州原武埽,溢入利津、陽武溝、刀馬河,歸納梁山濼。詔曰:「原武決口已引奪大河四分以上,不大治之,將貽朝廷巨憂。其輟修汴河堤岸司兵五千,並力築堤修閉。」都水復言:「兩馬頭墊落,水面闊二十五
 步,天寒,乞候來春施工。」至臘月竟塞雲。九月,河溢滄州南皮上、下埽,又溢清池埽,又溢永靜軍阜城下埽。十月辛亥,提舉汴河堤岸司言:「洛口廣武埽大河水漲,塌岸,壞下閘斗門,萬一入汴,人力無以枝梧。密邇都城,可不深慮。」詔都水監官速往護之。丙辰,廣武上、下埽危急,詔救護,尋獲安定。



 七年七月,河溢元城埽,決橫堤,破北京。帥臣王拱辰言:「河水暴至,數十萬眾號叫求救,而錢穀稟轉運,常平歸提舉,軍器工匠隸提刑,埽岸物料兵卒
 即屬都水監,逐司在遠,無一得專,倉卒何以濟民?望許不拘常制。」詔:「事幹機速,奏覆牒稟所屬不及者,如所請。」戊申,命拯護陽武埽。



 十月,冀州王令圖奏:「大河行流散漫,河內殊無緊流,旋生灘磧。宜近澶州相視水勢,使還復故道。會明年春,宮車晏駕。



 大抵熙寧初,專欲導東流,閉北流。元豐以後,因河決而北,議者始欲復禹故跡。神宗愛惜民力,思順水性,而水官難其人。王安石力主程昉、範子淵,故二人尤以河事自任;帝雖藉其才,然每抑
 之。其後,元祐元年,子淵已改司農少卿,御史呂陶劾其「修堤開河,縻費巨萬,護堤壓埽之人,溺死無數。元豐六年興役,至七年功用不成。乞行廢放。」於是黜知兗州,尋降知峽州。其制略曰:「汝以有限之材,興必不可成之役,驅無辜之民,置之必死之地。」中書舍人蘇軾詞也。



 八年三月,哲宗即位,宣仁聖烈皇后垂簾。河流雖北,而孫村低下,夏、秋霖雨,漲水往往東出。小吳之決既未塞,十月,又決大名之小張口,河北諸郡皆被水災。知澶州王令
 圖建議浚迎陽埽舊河,又於孫村金堤置約,復故道。本路轉運使範子奇仍請於大吳北岸修進鋸牙,擗約河勢。於是回河東流之議起。



 元祐元年二月乙丑,詔:「未得雨澤,權罷修河,放諸路兵夫。」九月丁丑,詔秘書監張問相度河北水事。十月庚寅,又以王令圖領都水,同問行河。



 十一月丙子,問言:「臣至滑州決口相視,迎陽埽至大、小吳,水勢低下,舊河淤仰,故道難復。請於南樂大名埽開直河並簽河,分引水勢入孫村口,以解北京向下水
 患。」令圖亦以為然,於是減水河之議復起。既從之矣,會北京留守韓絳奏引河近府非是,詔問別相視。



 二年二月,令圖、問欲必行前說,朝廷又從之。三月,令圖死,以王孝先代領都水,亦請如令圖議。



 右司諫王覿言:「河北人戶轉徙者多,朝廷責郡縣以安集,空倉廩以振濟,又遣專使察視之,恩德厚矣。然耕耘是時,而流轉於道路者不已;二麥將熟,而寓食於四方者未還。其故何也,盍亦治其本矣。今河之為患三:泛濫渟滀,漫無涯涘,吞食民
 田,未見窮已,一也;緣邊漕運獨賴御河,今御河淤澱,轉輸艱梗,二也;塘泊之設,以限南北,濁水所經,即為平陸,三也。欲治三患,在遴擇都水、轉運而責成耳。今轉運使範子奇反復求合,都水使者王孝先暗繆,望別擇人。」



 時知樞密院事安燾深以東流為是,兩疏言:「朝廷久議回河,獨憚勞費,不顧大患。蓋自小吳未決以前,河入海之地雖屢變移,而盡在中國;故京師恃以北限強敵,景德澶淵之事可驗也。且河決每西,則河尾每北,河流既益
 西決,固已北抵境上。若復不止,則南岸遂屬遼界,彼必為橋梁,守以州郡;如慶歷中因取河南熟戶之地,遂築軍以窺河外,已然之效如此。蓋自河而南,地勢平衍,直抵京師,長慮卻顧,可為寒心。又朝廷捐東南之利,半以宿河北重兵,備預之意深矣。使敵能至河南,則邈不相及。今欲便於治河而緩於設險,非計也。」



 王巖叟亦言:「朝廷知河流為北道之患日深,故遣使命水官相視便利,欲順而導之,以拯一路生靈於墊溺,甚大惠也。然昔者
 專使未還,不知何疑而先罷議;專使反命,不知何所取信而議復興。既敕都水使者總護役事,調兵起工,有定日矣,已而復罷。數十日間,變議者再三,何以示四方?今有大害七,不可不早為計。北塞之所恃以為險者在塘泊,黃河堙之,猝不可浚,浸失北塞險固之利,一也。橫遏西山之水,不得順流而下,蹙溢於千里,使百萬生齒,居無廬,耕無田,流散而不復,二也。乾寧孤壘,危絕不足道,而大名、深、冀腹心郡縣,皆有終不自保之勢,三也。滄州
 扼北敵海道,自河不東流,滄州在河之南,直抵京師,無有限隔,四也。並吞御河,邊城失轉輸之便,五也。河北轉運司歲耗財用,陷租賦以百萬計,六也。六七月之間,河流交漲,占沒西路,阻絕遼使,進退不能,兩朝以為憂,七也。非此七害,委之可,緩而未治可也。且去歲之患,已甚前歲,今歲又甚焉,則奈何?望深詔執政大臣,早決河議而責成之。」太師文彥博、中書侍郎呂大防皆主其說。



 中書舍人蘇轍謂右僕射呂公著曰:「河決而北,先帝不能
 回,而諸公欲回之,是自謂智勇勢力過先帝也。盍因其舊而修其未備乎?」公著唯唯。於是三省奏:「自河北決,恩、冀以下數州被患,至今未見開修的確利害,致妨興工。」乃詔河北轉運使、副,限兩月同水官講議聞奏。



 十一月,講議官皆言:「令圖、問相度開河,取水入孫村口還復故道處,測量得流分尺寸,取引不過,其說難行。」十二月,張景先復以問說為善,果欲回河,惟北京已上、滑州而下為宜,仍於孫村浚治橫河舊堤,止用逐埽人兵、物料,並
 年例客軍,春天漸為之可也。朝廷是其說。



 三年六月戊戌,乃詔:「黃河未復故道,終為河北之患。王孝先等所議,已嘗興役,不可中罷,宜接續工料,向去決要回復故道。三省、樞密院速與商議施行。」右相范純仁言:「聖人有三寶:曰慈,曰儉,曰不敢為天下先。蓋天下大勢,惟人君所向,群下競趨如川流山摧,小失其道,非一言一力可回,故居上者不可不謹也。今聖意已有所向而為天下先矣。乞諭執政:『前日降出文字,卻且進入。』免希合之臣,妄
 測聖意,輕舉大役。」尚書王存等亦言:「使大河決可東回,而北流遂斷,何惜勞民費財,以成經久之利。今孝先等自未有必然之論,但僥幸萬一,以冀成功,又預求免責,若遂聽之,將有噬臍之悔。乞望選公正近臣及忠實內侍,覆行按視,審度可否,興工未晚。」



 庚子,三省、樞密院奏事延和殿,文彥博、呂大防、安燾等謂:「河不東,則失中國之險,為契丹之利。」范純仁、王存、胡宗愈則以虛費勞民為憂。存謂:「今公私財力困匱,惟朝廷未甚知者,賴先帝
 時封樁錢物可用耳。外路往往空乏,奈何起數千萬物料、兵夫,圖不可必成之功?且御契丹得其道,則自景德至今八九十年,通好如一家,設險何與焉?不然,如石晉末耶律德光犯闕,豈無黃河為阻,況今河流未必便沖過北界耶?」太后曰:「且熟議。」



 明日,純仁又畫四不可之說,且曰:「北流數年未為大患,而議者恐失中國之利,先事回改;正如頃西夏本不為邊患,而好事者以為不取恐失機會,遂興靈武之師也。臣聞孔子論為政曰:『先有
 司。』今水官未嘗保明,而先示決欲回河之旨,他日敗事,是使之得以借口也。」



 存、宗愈亦奏:「昨親聞德音,更令熟議。然累日猶有未同,或令建議者結罪任責。臣等本謂建議之人,思慮有所未逮,故乞差官覆按。若但使之結罪,彼所見不過如此,後或誤事,加罪何益。臣非不知河決北流,為患非一。淤沿邊塘泊,斷御河漕運,失中國之險,遏西山之流。若能全回大河,使由孫村故道,豈非上下通願?但恐不能成功,為患甚於今日。故欲選近臣按視:
 若孝先之說決可成,則積聚物料,接續興役;如不可為,則令沿河踏行,自恩、魏以北,塘泊以南,別求可以疏導歸海去處,不必專主孫村。此亦三省共曾商量,望賜詳酌。存又奏:「自古惟有導河並塞河。導河者順水勢,自高導令就下;塞河者為河堤決溢,修塞令入河身。不聞乾引大河令就高行流也。」於是收回戊戌詔書。



 戶部侍郎蘇轍、中書舍人曾肇各三上疏。轍大略言:



 黃河西流,議復故道。事之經歲,役兵二萬,聚梢樁等物三十餘萬。方
 河朔災傷困弊,而興必不可成之功,吏民竊嘆。今回河大議雖寢,然聞議者固執來歲開河分水之策。今小吳決口,入地已深,而孫村所開,丈尺有限,不獨不能回河,亦必不能分水。況黃河之性,急則通流,緩則淤澱,既無東西皆急之勢,安有兩河並行之理?縱使兩河並行,未免各立堤防,其費又倍矣。



 今建議者其說有三,臣請折之:一曰御河湮滅,失饋運之利。昔大河在東,御河自懷、衛經北京,漸歷邊郡,饋運既便,商賈通行。自河西流,御
 河湮滅,失此大利,天實使然。今河自小吳北行,占壓禦河故地,雖使自北京以南折而東行,則御河湮滅已一二百里,何由復見?此御河之說不足聽也。二曰恩、冀以北,漲水為害,公私損耗。臣聞河之所行,利害相半,蓋水來雖有敗田破稅之害,其去亦有淤厚宿麥之利。況故道已退之地,桑麻千里,賦役全復,此漲水之說不足聽也。三曰河徙無常,萬一自契丹界入海,邊防失備。按河昔在東,自河以西郡縣,與契丹接境,無山河之限,邊臣
 建為塘水,以捍契丹之沖。今河既西,則西山一帶,契丹可行之地無幾,邊防之利,不言可知。然議者尚恐河復北徙,則海口出契丹界中,造舟為梁,便於南牧。臣聞契丹之河,自北南注以入於海。蓋地形北高,河無北徙之道,而海口深浚,勢無徙移,此邊防之說不足聽也。



 臣又聞謝卿材到闕,昌言:「黃河自小吳決口,乘高注北,水勢奔決,上流堤防無復決怒之患。朝廷若以河事付臣,不役一夫,不費一金,十年保無河患。」大臣以其異已罷歸,
 而使王孝先、俞瑾、張景先三人重畫回河之計。蓋由元老大臣重於改過,故假契丹不測之憂,以取必於朝廷。雖已遣百祿等出按利害,然未敢保其不觀望風旨也。願亟回收買梢草指揮,來歲勿調開河役兵,使百祿等明知聖意無所偏系,不至阿附以誤國計。



 肇之言曰:「數年以來,河北、京東、淮南災傷,今歲河北並邊稍熟,而近南州軍皆旱,京東、西、淮南饑殍瘡痍。若來年雖未大興河役,止令修治舊堤,開減水河,亦須調發丁夫。本路不
 足,則及鄰路,鄰路不足,則及淮南,民力果何以堪?民力未堪,則雖有回河之策,及梢草先具,將安施乎?」



 會百祿等行視東西二河,亦以為東流高仰,北流順下,決不可回。即奏曰:



 往者王令圖、張問欲開引水簽河,導水入孫村口還復故道。議者疑焉,故置官設屬,使之講議。既開撅井筒,折量地形水面尺寸高下,顧臨、王孝先、張景先、唐義問、陳祐之皆謂故道難復。而孝先獨叛其說,初乞先開減水河,俟行流通快,新河勢緩,人工物料豐備,徐
 議閉塞北流。已而召赴都堂,則又請以二年為期。及朝廷詰其成功,遽云:「來年取水入孫村口,若河流順快,工料有備,便可閉塞,回復故道。」是又不俟新河勢緩矣。回河事大,寧容異同如此!蓋孝先、俞瑾等知合用物料五千餘萬,未有指擬,見買數計,經歲未及毫厘,度事理終不可為,故為大言。



 又云:「若失此時,或河勢移背,豈獨不可減水,即永無回河之理。」臣等竊謂河流轉徙,乃其常事;水性就下,固無一定。若假以五年,休養數路民力,沿
 河積材,漸浚故道,葺舊堤,一旦流勢改變,審議事理,釃為二渠,分派行流,均減漲水之害,則勞費不大,功力易施,安得謂之一失此時,永無回河之理也?



 四年正月癸末,百祿等使回入對,復言:「修減水河,役過兵夫六萬三千餘人,計五百三十萬工,費錢糧三十九萬二千九百餘貫、石、匹、兩,收買物料錢七十五萬三百餘緡,用過物料二百九十餘萬條、束,官員、使臣、軍大將凡一百一十餘員請給不預焉。願罷有害無利之役,那移工料,繕築
 西堤,以護南決口。」未報。己亥,乃詔罷回河及修減水河。



 四月戊午,尚書省言:「大河東流,為中國之要險。自大吳決後,由界河入海,不惟淤壞塘濼,兼濁水入界河,向去淺澱,則河必北流。若河尾直注北界入海,則中國全失險阻之限,不可不為深慮。」詔範百祿、趙君錫條畫以聞。



 百祿等言:



 臣等昨按行黃河獨流口至界河,又東至海口,熟觀河流形勢;並緣界河至海口鋪砦地分使臣各稱:界河未經黃河行流已前,闊一百五十步下至五十
 步,深一丈五尺下至一丈;自黃河行流之後,今闊至五百四十步,次亦三二百步,深者三丈五尺,次亦二丈。乃知水性就下,行疾則自刮除成空而稍深,與《前漢書》大司馬史張戎之論正合。



 自元豐四年河出大吳,一向就下,沖入界河,行流勢如傾建。經今八年,不舍晝夜,沖刷界河,兩岸日漸開闊,連底成空,趨海之勢甚迅。雖遇元豐七年八年、元祐元年泛漲非常,而大吳以上數百里,終無決溢之害,此乃下流歸納處河流深快之驗也。



 塘濼有
 限遼之名,無御遼之實。今之塘水,又異昔時,淺足以褰裳而涉,深足以維舟而濟,冬寒冰堅,尤為坦途。如滄州等處,商胡之決即已澱淤,今四十二年,迄無邊警,亦無人言以為深憂。自回河之議起,首以此動煩聖聽。殊不思大吳初決,水未有歸,猶不北去;今入海湍迅,界河益深,尚復何慮?藉令有此,則中國據上游,契丹豈不慮乘流擾之乎?



 自古朝那、蕭關、雲中、朔方、定襄、雁門、上郡、太原、右北平之間,南北往來之沖,豈塘濼界河之足限哉。
 臣等竊謂本朝以來,未有大河安流,合於禹跡,如此之利便者。其界河向去只有深闊,加以朝夕海潮往來渲蕩,必無淺澱,河尾安得直注北界,中國亦無全失險阻之理。且河遇平壤灘漫,行流稍遲,則泥沙留淤;若趨深走下,湍激奔騰,惟有刮除,無由淤積,不至上煩聖慮。



 七月己巳朔,冀州南宮等五埽危急,詔撥提舉修河司物料百萬與之。甲午,都水監言:「河為中國患久矣,自小吳決後,泛濫未著河槽,前後遣官相度非一,終未有定論。
 以為北流無患,則前二年河決南宮下埽,去三年決上埽,今四年決宗城中埽,豈謂北流可保無虞?以為大河臥東,則南宮、宗城皆在西岸;以為臥西,則冀州信都、恩州清河、武邑或決,皆在東岸。要是大河千里,未見歸納經久之計,所以昨相度第三、第四鋪分決漲水,少紓目前之急。繼又宗城決溢,向下包蓄不定,雖欲不為東流之計,不可得也。河勢未可全奪,故為二股之策。今相視新開第一口,水勢湍猛,發洩不及,已不候工畢,更撥沙
 河堤第二口洩減漲水,因而二股分行,以紓下流之患。雖未保冬夏常流,已見有可為之勢。必欲經久,遂作二股,仍較今所修利害孰為輕重,有司具析保明以聞。」



 八月丁未,翰林學士蘇轍言:



 夏秋之交,暑雨頻並。河流暴漲出岸,由孫村東行,蓋每歲常事。而李偉與河埽使臣因此張皇,以分水為名,欲發回河之議,都水監從而和之。河事一興,求無不可,況大臣以其符合己說而樂聞乎。



 臣聞河道西行孫村側左,大約入地二丈以來,今所
 報漲水出岸,由新開口地東入孫村,不過六七尺。欲因六七尺漲水,而奪入地二丈河身,雖三尺童子,知其難矣。然朝廷遂為之遣都水使者,興兵功,開河道,進鋸牙,欲約之使東。方河水盛漲,其西行河道若不斷流,則遏之東行,實同兒戲。



 臣願急命有司,徐觀水勢所向,依累年漲水舊例,因其東溢,引入故道,以紓北京朝夕之憂。故道堤防壞決者,第略加修葺,免其決溢而已。至於開河、進約等事,一切毋得興功,俟河勢稍定然後議。不過
 一月,漲水既落,則西流之勢,決無移理。兼聞孫村出岸漲水,今已斷流,河上官吏未肯奏知耳。



 是時,吳安持與李偉力主東流,而謝卿材謂「近歲河流稍行地中,無可回之理」,上《河議》一編。召赴政事堂會議,大臣不以為然。癸丑,三省、樞密院言:「繼日霖雨,河上之役,恐煩聖慮。」太后曰:「訪之外議,河水已東復故道矣。」



 乙丑,李偉言:「已開撥北京南沙河直堤第三鋪,放水入孫村口故道通行。」又言:「大河已分流,即更不須開淘。因昨來一決之後,東
 流自是順快,渲刷漸成港道。見今已為二股,約奪大河三分以來,若得夫二萬,於九月興工,至十月寒凍時可畢。因引導河勢,豈止為二股通行而已,亦將遂為回奪大河之計。今來既因擗拶東流,修全鋸牙,當迤邐增進一埽,而取一埽之利,比至來年春、夏之交,遂可全復故道。朝廷今日當極力必閉北流,乃為上策。若不明詔有司,即令回河,深恐上下遷延,議終不決,觀望之間,遂失機會。乞復置修河司。」從之。



 五年正月丁亥,梁燾言:「朝廷
 治河,東流北流,本無一偏之私。今東流未成,邊北之州縣未至受患,其役可緩;北流方悍,邊西之州縣,日夕可憂,其備宜急。今傾半天下之力,專事東流,而不加一夫一草於北流之上,得不誤國計乎!去年屢決之害,全由堤防無備。臣願嚴責水官,修治北流埽岸,使二方均被惻隱之恩。」



 二月己亥,詔開修減水河。辛丑,乃詔三省、樞密院:「去冬愆雪,今未得雨,外路旱□闊遠,宜權罷修河。」



 戊申,蘇轍言:「臣去年使契丹,過河北,見州縣官吏,訪以
 河事,皆相視不敢正言。及今年正月,還自契丹,所過吏民,方舉手相慶,皆言近有朝旨罷回河大役,命下之日,北京之人,歡呼鼓舞。惟減水河役遷延不止,耗蠹之事,十存四五,民間竊議,意大臣業已為此,勢難遽回。既為聖鑒所臨,要當迤邐盡罷。今月六日,果蒙聖旨,以旱災為名,權罷修黃河,候今秋取旨。大臣覆奏盡罷黃河東、北流及諸河功役,民方憂旱,聞命踴躍,實荷聖恩。然臣竊詳聖旨,上合天意,下合民心。因水之性,功力易就,天
 語激切,中外聞者或至泣下,而臣奉行,不得其平。由此觀之,則是大臣所欲,雖害物而必行;陛下所為,雖利民而不聽。至於委曲回避,巧為之說,僅乃得行,君權已奪,國勢倒植。臣所謂君臣之間,逆順之際,大為不便者,此事是也。黃河既不可復回,則先罷修河司,只令河北轉運司盡將一道兵功,修貼北流堤岸;罷吳安持、李偉都水監差遣,正其欺罔之罪,使天下曉然知聖意所在。如此施行,不獨河事就緒,天下臣庶,自此不敢以虛誑
 欺朝廷,弊事庶幾漸去矣。」



 八月甲辰,提舉東流故道李偉言:「大河自五月後日益暴漲,始由北京南沙堤第七鋪決口,水出於第三、第四鋪並清豐口一並東流。故道河槽深三丈至一丈以上,比去年尤為深快,頗減北流橫溢之患。然今已秋深,水當減落,若不稍加措置,慮致斷絕,即東流遂成淤澱。望下所屬官司,經畫沙堤等口分水利害,免淤故道,上誤國事。」詔吳安持與本路監司、北外丞司及李偉按視,具合措置事連書以聞。



 九月,中
 丞蘇轍言:「修河司若不罷,李偉若不去,河水終不得順流,河朔生靈終不得安居。乞速罷修河司,及檢舉六年四月庚子敕,竄責李偉。」



 七年三月,以吏部郎中趙偁權河北轉運使。偁素與安持等議不協,嘗上《河議》,其略曰:「自頃有司回河幾三年,功費騷動半天下,復為分水又四年矣。故所謂分水者,因河流、相地勢導而分之。今乃橫截河流,置埽約以扼之,開浚河門,徒為淵潭,其狀可見。況故道千里,其間又有高處,故累歲漲落輒復自斷。
 夫河流有逆順,地勢有高下,非朝廷可得而見,職在有司,朝廷任之亦信矣,患有司不自信耳。臣謂當繕大河北流兩堤,復修宗城棄堤,閉宗城口,廢上、下約,開闞村河門,使河流湍直,以成深道。聚三河工費以治一河,一二年可以就緒,而河患庶幾息矣。願以河事並都水條例一付轉運司,而總以工部,罷外丞司使,措置歸一,則職事可舉,弊事可去。」



 四月,詔:「南、北外兩丞司管下河埽,今後令河北、京西轉運使、副、判官、府界提點分認界至,
 內河北仍於銜內帶『兼管南北外都水公事』。」



 十月辛酉,以大河東流,賜都水使者吳安持三品服,北都水監丞李偉再任。



\end{pinyinscope}