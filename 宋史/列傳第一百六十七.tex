\article{列傳第一百六十七}

\begin{pinyinscope}

 ○吳昌裔汪綱陳宓王霆



 吳昌裔,字季永,中江人。蚤孤,與兄泳痛自植立,不肯逐時好,得程頤、張載、朱熹諸書,輒研繹不倦。嘉定七年舉進士,聞漢陽守黃幹得熹之學,往從之。



 調閩中尉。利路
 轉運使曹彥約聞其賢,俾司糴場。時歲饑,議糴上流,昌裔請發本倉所儲數萬而徐糴以償,從之。調眉州教授。眉士故尚蘇軾學,昌裔取諸經為之講說,祠周惇頤及顥、頤、載、熹,揭《白鹿洞學規》,仿潭州釋奠儀,簿正祭器,士習丕變。制置使崔與之薦之,改知華陽縣。修學宮,來四方士,斥羨錢二十萬緡,買良田備旱。通判眉州,著《苦言》十篇,慮蜀甚悉。攝郡事,御軍有紀律。尋權漢州,故事比攝官,奉饋皆如真,昌裔命削其半。核兵籍,興社倉,郡政
 畢舉。興元帥趙彥吶議東納武仙,西結秦、鞏,人莫敢言,昌裔獨奮筆力辨其非。未幾,武仙敗,二州之民果叛。



 端平元年,入為軍器監簿,改將作監簿。改太常少卿。徐僑於人少許可,獨賢之。兼皇后宅教授,昌裔以祖宗舊典無以職事官充者,力辭,改吳、益王府教授。轉對,首陳六事,其目曰:「天理未純,天德未健,天命未敕,天工未亮,天職未治,天討未公。」凡君臣之綱,兄弟之倫,舉世以為大戒而不敢言者,皆痛陳之。至於邊臣玩令,陟罰無章,尤
 拳拳焉。拜監察御史,彈劾無所避,且曰:「今之朝綱果無所撓乎?言及親故則為之留中,言及私暱則為之訖了,事有窒礙則節帖付出,情有嫌疑則調停寢行。今日遷一人,曰存近臣之體,明日遷一人,曰為遠臣之勸。屈風憲之精採,徇人情之去留,士氣銷耎,下情壅滯,非所以糾正官邪,助國脈也。」



 臺臣故事,季詣獄點檢。時有爭常州田萬四千畝,平江亦數百畝,株逮百餘人,視其牘,乃趙善湘之子汝櫄、汝榟也,州縣不敢決,昌裔連疏劾罷
 之。冬洊雷,春大雨雪,昌裔居齋宮秉燭草疏,凡上躬缺失,宮庭嬖私,廟堂除授,皆以為言。又言:「將帥方命,女寵私謁,舊黨之用,邊疆之禍,皆此陰類。」且曰:「今大昕坐朝,間有時不視事之文;私第謁假,或有時不入堂之報。上有耽樂慆逸之漸,下無協恭和衷之風。內則嬖御懷私,為君心之蠹;外則子弟寡謹,為朝政之累。游言噂沓,寵賂章聞,欲《簫》、《勺》大和,得乎?」



 又念蜀事阽危,條四事以進:實規橅,審功賞,訪軍實,儲帥才。時有果、閬州守臣逃遁
 而進職,有知遂寧李煒父子足跡不至邊庭而受賞,僨軍之趙楷、棄城之朱揚祖皆不加罰;又帥臣趙彥吶年老智衰,其子淫刑黷貨,士卒不用命,安癸仲恥遭抨彈,經營復用,欲起謫籍以代帥垣,昌裔皆抗疏彈擊。



 又歷言三邊之事曰:「今朝廷之上,百闢晏然,言論多於施行,浮文妨於實務。後族王宮之冗費,列曹坐局之常程,群工閑慢之差除,諸道非泛之申請,以至土木經營,時節宴游,神霄祈禳,大禮錫賚,藻飾治具,無異平時。至於治
 兵足食之方,脩車備馬之事,乃缺略不講。」且援靖康之敝,痛哭言之。



 出為大理少卿,屢疏引去,不許。會杜範再入臺,擊參政李鳴復,謂昌裔與範善,必相為謀者,數讒之,以權工部侍郎出參贊四川宣撫司軍事。人曰:「此李綱救太原也。太原不可救,特以綱主戰,故出之耳。」昌裔曰:「君命也,不可不亟行。」慷慨褸襆被出關,忽得疾,中道病甚,帝聞之,授秘閣修撰,改嘉興府。昌裔曰:「吾以疾不能歸救父母,上負聖恩,下負此心,若舍遠就近,舍危就安,
 人其謂我何?」辭至四五,而言者以避事論矣。



 改贛州,辭,以右文殿修撰主管鴻慶宮。遷浙東提刑,辭,改知婺州。婺告旱,民日夜望之,乃不忍終辭,減騶從供帳,遣僚佐召邑令周行阡陌,蠲粟八萬一千石、錢二十五萬緡有奇。加集英殿修撰,卒,以寶章閣待制致仕。



 昌裔剛正莊重,遇事敢言,典章多所閑習。嘗輯至和、紹興諸臣奏議本末。名《儲鑒》。又會粹周、漢以至宋蜀道得失,興師取財之所,名《蜀鑒》。有奏議、《四書講義》、《鄉約口義》、《諸老記聞》、《容
 臺議禮》、文集行於世。



 初,昌裔與徐清叟、杜範一日並入臺,皆天下正士,四方想聞風採,人至和《三諫詩》以侈之。然才七閱月以遷,故莫不惋惜云。後謚忠肅。



 汪綱,字仲舉,黟縣人,簽書樞密院勃之曾孫也。以祖任入官,淳熙十四年中銓試,調鎮江府司戶參軍。



 馬大同鎮京口,強毅自任,綱言論獨不詭隨。議者欲以兩淮鐵錢交子行於沿江,廷議令大同倡率行之,綱貽書曰:「邊面行鐵錢,慮銅寶洩於外耳。私鑄盛行,故錢輕而物重。
 今若場務出納不以鐵錢取息,堅守四色請買舊制,冶鑄定額不求餘羨,重禁以戢私鑄,支散邊戍與在軍中半者無異,不以鐵錢準折,則淮民將自便之,何至以敝內郡邪?」大同始悟。試湖南轉運司,又中,綱笑曰:「此豈足以用世澤物耶?」乃刻意問學,博通古今,精究義理;覃思本原。



 調桂陽軍平陽縣令,縣連溪峒,蠻蜒與居,綱一遇以恩信。科罰之害既三十年,綱下車,首白諸臺,罷之。桂陽歲貢銀二萬九千餘兩,而平陽當其三分之二。綱謂
 向者銀礦坌發價輕,故可勉以應,今地寶已竭,市於他郡,其價倍蓰,願力請痛蠲損之。歲饑,旁邑有曹伍者,群聚惡少入境,強貸發廩,眾至千餘,挾界頭、牛橋二砦兵為援,地盤踞萬山間,前後令未嘗一涉其境,不虞綱之至也,相率出迎。綱已夙具酒食,令之曰:「汝何敢亂,順者得食,亂者就誅。」夜宿砦中,呼砦官詰責不能防守狀,皆皇恐伏地請死,杖其首惡者八人,發粟振糶,民賴以安。



 改知金壇縣,親嫌,更弋陽縣。父義和為侍御史主管祐
 神觀。尋丁父喪,服除,知蘭溪縣,決擿如神。歲旱,郡倚辦勸分,綱謂勸分所以助義倉,一切行之,非所謂安富恤貧也,願假常平錢為糶本,使得循環迭濟。又躬勸富民浚築塘堰,大興水利,餓者得食其力,全活甚眾。郡守張抑及部使者列綱為一道荒政之冠。以言去,邑人相率投軌直其事,綱力止之。



 繼知太平縣,主管兩浙轉運司文字,未赴,罹內艱,擢監行在左藏西庫。屬金人殺其主允濟自立,遣使來告襲位,議者即欲遣幣,綱言:『使名不
 遜,當止之境上,姑命左帑視例計辦,或且留京口總司,令盱眙諭之曰:『紀年名節,皆犯先朝避忌,歲幣乃爾前主所增,今既易代,當復隆興、大定之舊。』俟此議定,而後正旦、生辰之使可遣。遲以歲月,吾擇邊將葺城堡,簡軍實,儲峙糗糧,使沿邊屹然有不可犯之勢,聽其自相攻擊,然後以全力制其後。」廟堂韙之。



 提轄東西庫,又乾辦諸司審計司。以選知高郵軍,陛辭,言:「揚、楚二州當各屯二萬人,壯其聲勢,而以高郵為家計砦。高郵三面阻水,
 湖澤奧阻,戎馬所不能騁,獨西南一路直距天長,無險可守,乃去城六十里隨地經畫,或浚溝塹,或備設伏,以扼其沖。」又慮湖可以入淮,招水卒五千人造百艘列三砦以戒非常。興化民田濱海,昔範仲淹築堰以障寫鹵,守毛澤民置石䃮函管以疏運河水勢,歲久皆壞,綱乃增修之。部使者聞於朝,增一秩,提舉淮東常平。淮米越江有禁,綱念「淮民有警則室廬莫保,歲兇則轉徙無歸,豐年可以少蘇,重以苛禁,自分畛域,豈為民父母意哉!
 請下金陵糴三十萬以通淮西之運,京口糴五十萬以通淮東之運。」又言:「兩淮之積不可多,昇、潤之積不可少。平江積米數百萬,陳陳相因,久而紅腐,宜視其收貯近久,取餉輦下百司、諸軍。江上歲餫當至京者,貯之京口、金陵轉漕。兩淮、中都諸倉,亦當廣糴以補其數。」



 制置使訪綱備御孰宜先,綱言:「淮地自昔號財賦淵藪,西有鐵冶,東富魚稻,足以自給。淮右多山,淮左多水,足以自固。誠能合兩淮為一家,兵財通融,聲勢合一,雖不假江、浙
 之力可也。祖宗盛時,邊郡所儲足支十年;慶歷間,中山一鎮尚百八十萬石。今宜上法先朝,令商旅入粟近塞,而算請錢貨於京師。入粟拜爵,守之以信,則輸者必多,邊儲不患不豐。州郡禁兵本非供役,乃就糧外郡耳,今不為戰鬥用,乃使之共力役,緩急戍守,專倚大軍,指日待更,不安風土,豈若土兵生長邊地,墳墓室家,人自為守邪?當精擇伉壯,廣其尺籍,悉隸御前軍額,分擘券給以助州郡衣糧之供,大率如山陽武鋒軍制,則邊面不
 必抽江上之戍,江上不必出禁闈之師。生券更番,勞費俱息。」



 時有獻言制司廣買荒田開墾,以為營田,綱以為「荒瘠之地不難辦,而工力、水利非久不可,棄產欺官,良田終不可得,耗費公帑,開墾難就。曷若勸民盡耕閑田,甽澮堙塞則官為之助,變瘠為沃,使民有餘蓄。晁錯入粟之議,本朝便糴之法,在其中矣。」制司知其無益,乃止。



 淮東煮鹽之利,本居天下半,歲久敝滋,鹽本日侵,帑儲空竭,負兩總司五十餘萬,亭戶二十八萬,借撥於朝廷
 五十萬,又會餉所復鹽鈔,舊制弗許商人預供貼鈔錢,鹽司坐是窘不能支。綱抉擿隱伏,凡虛額無實,詭為出內,飛走移易,事制曲防,課乃更羨。既盡償所負,又贏金三十萬緡,為樁辦庫,以備鹽本之闕。添置新灶五十所,諸場悉視乾道舊額三百九十萬石,通一千三百萬緡,課官吏之殿最。綱約己率下,辭臺郡之互饋,獨增場官奉以養其廉。



 擢戶部員外郎、總領淮東軍馬財賦。時邊面多生券,山東歸附月餉錢糧,以緡計增三十有三萬,
 米以石計增六萬,真、楚諸州又新招萬弩手,皆仰給總所,而浙西鹽利積負至七十餘萬緡,諸州漕運不以時至。綱核名實,警稽慢,區畫處分,餉事賴以不乏。



 移疾乞閑,得直秘閣、知婺州,改提點浙東刑獄,皆屢辭不得請。慮囚,至婺,有奴挾刃欲戕其主,不遇而殺其子,瞞讕妄牽連,徑出斬之。釋衢囚之冤者。臺盜鐘百一非共盜,尉覬賞,躐申制司,綱謂:「治盜雖尚嚴,豈得鍛煉傅會以成其罪邪?」於是得減死。禱雨龍瑞宮,有物蜿蜒朱色,盤旋
 壇上者三日。綱曰:「吾欲雨而已,毋為異以惑眾。」言未竟,雷雨大至,歲以大熟。



 進直煥章閣、知紹興府、主管浙東安撫司公事兼提點刑獄。訪民瘼,罷行尤切。蕭山有古運河,西通錢塘,東達臺、明,沙漲三十餘里,舟行則膠。乃開浚八千餘丈,復創閘江口,使泥淤弗得入,河水不得洩,於塗則盡甃以達城闉。十里創一廬。名曰「施水」,主以道流。於是舟車水陸,不問晝夜暑寒,意行利涉,歡欣忘勩。屬邑諸縣瀕海,而諸暨十六鄉瀕湖,蕩濼灌溉之利
 甚博,勢家巨室率私植埂岸,圍以成田,湖流既束,水不得去,雨稍多則溢入邑居,田閭浸蕩。瀕海藉塘為固,堤岸易圮,咸鹵害稼,歲損動數十萬畝,蠲租亦萬計。以綱言,詔提舉常平司發田園,奇援巧請,一切峻卻,而湖田始復;郡備緡錢三萬專備修築,而海田始固。綱謂:「是邦控臨海道,密拱都畿,而軍籍單弱。」乃招水軍,刺叉手,教習甚專,不令他役。創營千餘間,寬整堅密,增置甲兵,威聲赫然。兼權司農卿,尋直龍圖閣,因任。



 理宗即位,詔為
 右文殿修撰,加集英殿修撰,復因任,又加寶謨閣待制。寶慶三年大水,綱發粟三萬八千餘、緡錢五萬振之,蠲租六萬餘石,捐瘠頓蘇,無異常歲。越有經總制窠名四十一萬,其中二十五,則紹興以來虛額也,前後帥懼負殿,以修奉欑宮之資偽增焉。綱謂:「負殿之責小,罔上之罪大」。摭其實以聞。詔免九萬五千緡,而宿敝因是著明矣。



 紹定元年,召赴行在,綱入見,言:「臣下先利之心過於徇義,為身之計過於謀國,媮惰退縮,奔競貪黷,相與為
 欺,宜有以轉移之。」帝曰:「聞卿治行甚美,越中民力如何?」對曰:「去歲水潦,諸暨為甚,今歲幸中熟,十年之間,千里晏安,皆朝廷威德所及,臣何力之有。」權戶部侍郎。越數月,上章致仕,特畀二秩,守戶部侍郎,仍賜金帶。卒,越人聞之多墮淚,有相率哭於寺觀者。



 綱學有本原,多聞博記,兵農、醫卜、陰陽、律歷諸書,靡不研究;機神明銳,遇事立決。在越佩四印,文書山積,而能操約御詳,治事不過二十刻,公庭如水。卑官下吏,一言中理,慨然從之。為文
 尤長於論事,援據古今,辨博雄勁。服用不喜奢麗,供帳車剩,雖敝不更。所著有《恕齋集》、《左帑志》、《漫存錄》。



 陳宓,字師復,丞相俊卿之子。少嘗及登朱熹之門,熹器異之。長從黃乾游。以父任歷泉州南安鹽稅,主管南外睦宗院、再主管西外,知安溪縣。



 嘉定七年,入監進奏院。時無敢慷慨盡言者,宓上封事言:「宮中宴飲或至無節,非時賜予為數浩穰,一人蔬食而嬪御不廢於擊鮮,邊事方殷而樁積反資於妄用,此宮闈儀刑有未正也。大
 臣所用非親即故,執政擇易制之人,臺諫用慎默之士,都司樞掾,無非親暱,貪吏靡不得志,廉士動招怨尤,此朝廷權柄有所分也。鈔鹽變易,楮幣秤提,安邊所創立,固執己見,動失人心,敗軍之將躐躋殿巖,庸鄙之夫久尹京兆,宿將有守成之功,以小過而貶,三牙無汗馬之勞,托公勤而擢,此政令刑賞多所舛逆也。若能交飭內外,一正紀綱,天且不雨,臣請伏面謾之罪。」奏入,丞相史彌遠不樂,而中宮慶壽,三牙獻遺,至是為之罷卻。尋遷
 軍器監簿。九年,轉對言:



 人主之德貴乎明,大臣之心貴乎公,臺諫之言貴乎直。陛下臨政雖勤而治功未舉,奉身雖儉而財用未豐,愛民雖仁而實惠未遍。良由上下相蒙,務於欺蔽。匭奏囊封,有懷畢吐,陛下付近臣差擇,是有意於行其言也。而有司惟取專攻上躬與移咎牧守之章,騰播中外,以答觀聽。今赤地千里,蝗飛蔽天,如此其可畏,猶或諱晦以旱不為災、蝗不害稼,其他誣罔,抑又可知。臣故曰人主之德貴乎明。



 大臣施設,浸異厥
 初。凡建議求言之人,則以他事逐,諫官言事稍直,則以他職徙。忠憤者指為不靖,切直者目曰沽名,眾怨所萃則相繼超升,物論所歸則以次疏外。某人之遷,是嘗重人罪以快同列之私忿者;某人之擢,是嘗援古事以文邇日之天變者。直節重望以私嫌而久棄,老奸宿臧以巧請而牽復。使大臣果能杜幸門、塞邪徑,則舉錯當而人心服。臣故曰大臣之心貴乎公。



 臺諫平居未嘗立異,遇事不敢盡言。有如金人再通,最關國體,近而侍從,下
 至生徒,莫不力爭,冀裨廟算,獨於言責,不出一辭。輦轂之下,乾沒巨萬,莫之誰何;州縣之間,罪僅毫發,摭以塞責。大臣所欲為之事則遂之,所不右之人則排之。仁宗時,有宰相奉行臺諫風旨之譏,今乃有臺諫不敢違中書之誚,豈祖宗設官之初意哉?臣故曰臺諫之言貴乎直。



 三者機括所系,願陛下幡然悔悟,昭明德以照臨百官。大臣、臺諫,亦宜公心直節,以副望治之意。



 指陳敝事,視前疏尤剴切焉。



 宓遂請罷,歸。在告日,擢太府丞,不拜,
 出知南康軍。詣史彌遠別,彌遠曰:「子言甚切當,第愚昧不能行,殊有愧耳。」至官,歲大侵,奏蠲其賦十之九。會流民群集,宓就役之,築江堤,而給其食。時造白鹿洞,與諸生討論。改知南劍州。時大旱疫,蠲逋賦十數萬,且弛新輸三之一,躬率僚吏持錢粟藥餌戶給之。創延平書院,悉仿白鹿洞之規。



 知漳州,未行,聞寧宗崩,嗚咽累日。亡何,請致仕。寶慶二年,提點廣東刑獄,章復三上,迄不就。直秘閣,主管崇禧觀,宓拜祠命而辭職名。卒,進職一等
 致仕。三學諸生以起宓為請,而沒已閱月矣。



 初,宓之在朝也,寺丞丁焴往使金,宓嘆曰:「世仇未復,何以好為?」餞詩有「百年中國豈無人」之句。後數年,聞關外不靖,以書抵焴曰:「蜀口去關外雖遠,實如一身。近事可寒心,皆士大夫之罪,豈非賄道不絕之故耶?」焴服其言。



 宓天性剛毅,信道尤篤,嘗為《朱墨銘》,謂朱屬陽,墨屬陰,以驗理欲分寸之多寡。自言居官必如顏真卿,居家必如陶潛,而深愛諸葛亮身死家無餘財,庫無餘帛。庶乎能蹈其語
 者,端平初,殿中侍御史王遂首言:「宓事先帝有論諫之直,而不及俟聖化之更,宜褒其身後,以勸天下之為臣者。」帝為感動,詔贈直龍圖閣。所著書有《論語注義問答》、《春秋三傳抄》、《讀通鑒綱目》、《唐史贅疣》之稿數十卷,藏於家。



 王霆,字定叟,東陽人。高大父豪,帥眾誅方臘,以功補官。霆少有奇氣,試有司不偶,去就武舉,嘉定四年,中絕倫異等。喬行簡考藝別頭,喜曰:「吾為朝廷得一帥才矣。」



 授
 承節郎,從軍於鄂,帥鐘興嗣戍邊,請於樞密院,以霆為隨軍都錢糧官。總領綦奎委霆專一教閱總效軍,尋委帥師守禦黃州。沿江制置副使李𡌴闢置幕下,淮右兵叛,遣霆招諭之。霆於軍事知無不言,謂:「招募良家子,不可以寅緣關節冒濫其間,防守江面,全藉正軍,若義勇、民兵,特可為聲援耳。而所謂大軍,羸病者多,兵械損舊,豈不敗事。調兵防江,當於江岸創屋居之,使之專心守御。諸軍伍法既廢,平居則無以稽其虛籍冒請之敝,無
 以糾其竄逸生事之人,緩急則無以稽其並力向敵之志,無以連其逃陳不進之心。此尉繚子所以著束部伍之令,太公謂伍法為要者謂此也。用兵不以人數多寡為勝負,惟教習之精否,則勝負之形可見矣。」



 理宗即位,特差充浙西副都監、湖州駐札。時潘甫等起兵,事甫定,霆因綏撫之。鎮江都統趙勝闢為計議官,時李全寇鹽城,攻海陵,勝出戍揚州,屬官多憚從行,霆慨然曰:「此豈臣子辭難之日!」至揚子橋,人言賊兵昨日在南門,去將
 安之,霆竟至南門,以帥憲之命董三城事。勝次第出城接戰,霆必身先士卒,大小十八戰,無一不利。奪賊壕,築土城,焚城門,賊氣為懾。差知應州兼沿邊都巡檢使,樞密院命節制黃莆後營,彈壓諸道軍馬。諸道兵二十萬將往收復楚州,霆帥所部為掎角之助。



 大帥薦之,召試為閣門舍人。入對言:「恢復之說有二:曰規TO,曰機會。顧今日之規TO安在哉?守令所以牧民,而惠養之未加;將帥所以御軍,而拊循之未至。邦財未裕,而楮券之敝浸
 深;軍儲未豐,而和糴之害徒慘。官有土地而荒蕪,民因賦役而破蕩,獄訟類成冤抑,銓曹率多淹留。薦舉無反坐,貪徒得以引類而通班;按刺不徇公,微官易以迕意而連譴。以言郡計,則紛耗於囊橐包苴;以言戰功,則多私於親暱故舊。至如降卒中處,養虎遺患,輕敵開邊,以肉餧虎。夫以規TO之切要者而不滿人意如此,臣敢輕進恢復之說以誤上聽哉?凡臣之所陳者,誠播告中外之臣,悉懲其舊而圖其新。規TO既立,然後義旗一麾,諸
 道並進,臣力尚壯,願效前驅。惟陛下堅定而勉圖之。」帝稱其言可採。升武功大夫,出知濠州,賜金帶。至州,節浮費,糴粟買馬,以備不虞。尋差知安豐軍,臣僚上言:「王霆在濠,人甚安之,不宜輕易。」詔再任濠,職事修舉,特轉橫班。諸使交薦之。



 北兵至浮光,其民奔遁,相屬於道,朝論以為霆可守之,乃知光州兼沿邊都巡檢使。冒雪夜行,倍道疾馳至州,分遣間探,整飭戰守之具,大戰於謝令橋,光人遂安。督府魏了翁以書來慰安之,以緡錢十萬
 勞其軍。霆以召,尋為吉州刺史,仍知光州。霆固辭,丞相鄭清之、制置使史嵩之皆數以書留霆,霆不從,且曰:「士大夫當以世從道,不可以道從世也。」



 再授閣門舍人,尋為達州刺史、右屯衛大將軍兼知蘄州,不赴。尋遷淮西馬步軍副總管兼淮西游擊軍副都統制。論游擊軍十事,不報。提舉崇禧觀。知高郵軍,流民邦傑聚眾三千人為盜,霆剿其渠魁,餘黨悉散。時議出師,和者甚多,霆以為:「莫若遣間探覘敵情,如不得已然後行之;否則無
 故自蕩其根本,是外兵未至而內兵先慘烈也。」諸軍畢行,惟高郵遲之,境內賴以安全。由是與時迕,而讒者益眾。



 提舉雲臺觀。執政期論邊事,且謂朝廷即有齊安之命。霆曰:「秋防已急,邊守不宜臨時更易,盍少需之。」乃授帶行左領軍衛大將軍,充沿江制置副使司計議官,霆乃撰《沿江等邊志》一編上之。制置使董槐、鄧泳交薦之,差知壽昌軍,改蘄州,建學舍,祠忠臣。嘗嘆曰:「兩淮藩籬也,大江門戶也,三輔堂奧也。藩籬不固則門戶且危,門
 戶既危則堂奧豈能久安乎?」於是貽書丞相杜範,乞瞰江審察形勢,置三新城:蘄春置於龍眼磯,安慶置於孟城,滁陽置於宣化。不報。卒。



 初,其父析業,霆獨以讓其兄。處宗族有恩意,嘗訓其子弟曰:「窮理盡性,學之本也。」有《玉溪集》行於世。



 論曰:吳昌裔訪道東南,一何勤哉!故其造深醇,見諸事功者,足以知其學無雜也。汪綱之遺愛在越,先民所謂擇賢久任者,固不我欺矣。陳宓以宰相子,論諫之直,於
 今有光。王霆通兵家言,而謂不可以道從世,此古人謀帥貴乎「說《禮》、《樂》而敦《詩》、《書》」也。



\end{pinyinscope}