\article{列傳第七十三}

\begin{pinyinscope}

 範
 仲淹子純祐純禮純粹范純仁子正平



 範仲淹,字希文,唐宰相履冰之後。其先邠州人也,後徙家江南,遂為蘇州吳縣人。仲淹二歲而孤,母更適長山朱氏,從其姓,名說。少有志操,既長,知其世家,乃感泣辭
 母,去之應天府,依戚同文學。晝夜不息,冬月憊甚,以水沃面;食不給,至以糜粥繼之,人不能堪,仲淹不苦也。舉進士第,為廣德軍司理參軍,迎其母歸養。改集慶軍節度推官,始還姓,更其名。



 監泰州西溪鹽稅,遷大理寺丞,徙監楚州糧料院,母喪去官。晏殊知應天府,聞仲淹名,召寘府學。上書請擇郡守,舉縣令,斥游惰,去冗僭,慎選舉,撫將帥,凡萬餘言。服除,以殊薦,為秘閣校理。仲淹泛通《六經》,長於《易》,學者多從質問,為執經講解,亡所倦。嘗
 推其奉以食四方游士,諸子至易衣而出,仲淹晏如也。每感激論天下事,奮不顧身,一時士大夫矯厲尚風節,自仲淹倡之。



 天聖七年,章獻太后將以冬至受朝,天子率百官上壽。仲淹極言之,且曰:「奉親於內,自有家人禮,顧與百官同列,南面而朝之,不可為後世法。」且上疏請太后還政,不報。尋通判河中府,徙陳州。時方建太一宮及洪福院,市材木陜西。仲淹言:「昭應、壽寧,天戒不遠。今又侈土木,破民產,非所以順人心、合天意也。宜罷修寺
 觀,減常歲市木之數,以蠲除積負。」又言:「恩幸多以內降除官,非太平之政。」事雖不行,仁宗以為忠。



 太后崩,召為右司諫。言事者多暴太后時事,仲淹曰:「太后受遺先帝,調護陛下者十餘年,宜掩其小故,以全後德。」帝為詔中外,毋輒論太后時事。初,太后遺誥以太妃楊氏為皇太后,參決軍國事。仲淹曰:』太后,母號也,自古無因保育而代立者。今一太后崩,又立一太后,天下且疑陛下不可一日無母後之助矣。」



 歲大蝗旱,江、淮、京東滋甚。仲淹請
 遣使循行,未報。乃請間曰:「宮掖中半日不食,當何如?」帝側然,乃命仲淹安撫江、淮,所至開倉振之,且禁民淫祀,奏蠲廬舒折役茶、江東丁口鹽錢,且條上救敝十事。



 會郭皇后廢,率諫官、御史伏閣爭之,不能得。明日,將留百官揖宰相廷爭,方至待漏院,有詔出知睦州。歲餘,徙蘇州。州大水,民田不得耕,仲淹疏五河,導太湖注之海,募人興作,未就,尋徙明州,轉運使奏留仲淹以畢其役,許之。拜尚書禮部員外郎、天章閣待制,召還,判國子監,遷
 吏部員外郎、權知開封府。



 時呂夷簡執政,進用者多出其門。仲淹上《百官圖》,指其次第曰:「如此為序遷,如此為不次,如此則公,如此則私。況進退近臣,凡超格者,不宜全委之宰相。」夷簡不悅。他日,論建都之事,仲淹曰:「洛陽險固,而汴為四戰之地,太平宜居汴,即有事必居洛陽。當漸廣儲蓄,繕宮室。」帝問夷簡,夷簡曰:「此仲淹迂闊之論也。」仲淹乃為四論以獻,大抵譏切時政。且曰:「漢成帝信張禹,不疑舅家,故有新莽之禍。臣恐今日亦有張禹,
 壞陛下家法。」夷簡怒訴曰:「仲淹離間陛下君臣,所引用,皆朋黨也。」仲淹對益切,由是罷知饒州。



 殿中侍御史韓瀆希宰相旨,請書仲淹朋黨,揭之朝堂。於是秘書丞餘靖上言曰:「仲淹以一言忤宰相,遽加貶竄,況前所言者在陛下母子夫婦之間乎?陛下既優容之矣,臣請追改前命。」太子中允尹洙自訟與仲淹師友,且嘗薦己,願從降黜。館閣校勘歐陽修以高若訥在諫官,坐視而不言,移書責之。由是,三人者偕坐貶。明年,夷簡亦罷,自是朋
 黨之論興矣。仲淹既去,士大夫為論薦者不已。仁宗謂宰相張士遜曰:「向貶仲淹,為其密請建立皇太弟故也。今朋黨稱薦如此,奈何?」再下詔戒敕。



 仲淹在饒州歲餘,徙潤州,又徙越州。元昊反,召為天章閣待制、知永興軍,改陜西都轉運使。會夏竦為陜西經略安撫、招討使,進仲淹龍圖閣直學士以副之。夷簡再入相,帝諭仲淹使釋前憾。仲淹頓首謝曰:「臣鄉論蓋國家事,於夷簡無憾也。」



 延州諸砦多失守,仲淹自請行,遷戶部郎中兼知延
 州。先是,詔分邊兵:總管領萬人,鈐轄領五千人,都監領三千人。寇至御之,則官卑者先出。仲淹曰:「將不擇人,以官為先後,取敗之道也。」於是大閱州兵,得萬八千人,分為六,各將三千人,分部教之,量賊眾寡,使更出禦賊。時塞門、承平諸砦既廢,用種世衡策,城青澗以據賊沖,大興營田,且聽民得互市,以通有無。又以民遠輸勞苦,請建鄜城為軍,以河中、同、華中下戶稅租就輸之。春夏徙兵就食,可省糴十之三,他所減不與。詔以為康定軍。



 明
 年正月,詔諸路入討,仲淹曰:「正月塞外大寒,我師暴露,不如俟春深入,賊馬瘦人饑,勢易制也。況邊備漸修,師出有紀,賊雖猖獗,固已懾其氣矣。鄜、延密邇靈、夏,西羌必由之地也。第按兵不動,以觀其釁,許臣稍以恩信招來之。不然,情意阻絕,臣恐偃兵無期矣。若臣策不效,當舉兵先取綏、宥,據要害,屯兵營田,為持久計,則茶山、橫山之民,必挈族來歸矣。拓疆禦寇,策之上也。」帝皆用其議。仲淹又請修承平、永平等砦,稍招還流亡,定堡障,通
 斥候,城十二砦,於是羌漢之民,相踵歸業。



 久之,元昊歸陷將高延德,因與仲淹約和,仲淹為書戒喻之。會任福敗於好水川,元昊答書語不遜,仲淹對來使焚之。大臣以為不當輒通書,又不當輒焚之,宋庠請斬仲淹,帝不聽。降本曹員外郎、知耀州,徙慶州,遷左司郎中,為環慶路經略安撫、緣邊招討使。初,元昊反,陰誘屬羌為助,而環慶酋長六百餘人,約為鄉道』事尋露。仲淹以其反復不常也,至部即奏行邊,以詔書犒賞諸羌,閱其人馬,為
 立條約:「若仇已和斷,輒私報之及傷人者,罰羊百、馬二,已殺者斬。負債爭訟,聽告官為理,輒質縛平人者,罰羊五十、馬一。賊馬入界,追集不赴隨本族,每戶罰羊二,質其首領。賊大入,老幼入保本砦,官為給食;即不入砦,本家罰羊二;全族不至,質其首領。」諸羌皆受命,自是始為漢用矣。



 改邠州觀察使,仲淹表言:「觀察使班待制下,臣守邊數年,羌人頗親愛臣,呼臣為『龍圖老子』。今退而與王興、朱觀為伍,第恐為賊輕矣。」辭不拜。慶之西北馬鋪
 砦,當後橋川口,在賊腹中。仲淹欲城之,度賊必爭,密遣子純祐與蕃將趙明先據其地,引兵隨之。諸將不知所向,行至柔遠,始號令之,版築皆具,旬日而城成,即大順城是也。賊覺,以騎三萬來戰,佯北,仲淹戒勿追,已而果有伏。大順既城,而白豹、金湯皆不敢犯,環慶自此寇益少。



 明珠、滅臧勁兵數萬,仲淹聞涇原欲襲討之,上言曰:「二族道險,不可攻,前日高繼嵩已喪師。平時且懷反側,今討之,必與賊表裏,南入原州,西擾鎮戎,東侵環州,邊
 患未艾也。若北取細腰、胡蘆眾泉為堡障,以斷賊路,則二族安,而環州、鎮戎徑道通徹,可無憂矣。」其後,遂築細腰、胡蘆諸砦。



 葛懷敏敗於定川,賊大掠至潘原,關中震恐,民多竄山谷間。仲淹率眾六千,由邠、涇援之,聞賊已出塞,乃還。始,定川事聞,帝按圖謂左右曰:「若仲淹出援,吾無憂矣。」奏至,帝大喜曰:「吾固知仲淹可用也。」進樞密直學士、右諫議大夫。仲淹以軍出無功,辭不敢受命,詔不聽。



 時已命文彥博經略涇原,帝以涇原傷夷,欲對徙
 仲淹,遣王懷德喻之。仲淹謝曰:「涇原地重,第恐臣不足當此路。與韓琦同經略涇原,並駐涇州,琦兼秦鳳、臣兼環慶。涇原有警,臣與韓琦合秦鳳,環慶之兵,掎角而進;若秦鳳、環慶有警,亦可率涇原之師為援。臣當與琦練兵選將,漸復橫山,以斷賊臂,不數年間,可期平定矣。願詔龐籍兼領環慶,以成首尾之勢。秦州委文彥博,慶州用滕宗諒總之。孫沔亦可辦集。渭州,一武臣足矣。」帝採用其言,復置陜西路安撫、經略、招討使,以仲淹、韓琦、龐
 籍分領之。仲淹與琦開府涇州,而徙彥博帥秦,宗諒帥慶,張亢帥渭。



 仲淹為將,號令明白,愛撫士卒,諸羌來者,推心接之不疑,故賊亦不敢輒犯其境。元昊請和,召拜樞密副使。王舉正懦默不任事,諫官歐陽修等言仲淹有相材,請罷舉正用仲淹,遂改參知政事。仲淹曰:「執政可由諫官而得乎?」固辭不拜,願與韓琦出行邊。命為陜西宣撫使,未行,復除參知政事。會王倫寇淮南,州縣官有不能守者,朝廷欲按誅之。仲淹曰:「平時諱言武備,
 寇至而專責守臣死事,可乎?」守令皆得不誅。



 帝方銳意太平,數問當世事,仲淹語人曰:「上用我至矣,事有先後,久安之弊,非朝夕可革也。」帝再賜手詔,又為之開天章閣,召二府條對,仲淹皇恐,退而上十事:



 一曰明黜陟。二府非有大功大善者不遷,內外須在職滿三年,在京百司非選舉而授,須通滿五年,乃得磨勘,庶幾考績之法矣。二曰抑僥幸。罷少卿、監以上乾元節恩澤;正郎以下若監司、邊任,須在職滿二年,始得蔭子;大臣不得薦子弟
 任館閣職,任子之法無冗濫矣。三曰精貢舉。進士、諸科請罷糊名法,參考履行無闕者,以名聞。進士先策論,後詩賦,諸科取兼通經義者。賜第以上,皆取詔裁。餘優等免選注官,次第人守本科選。進士之法,可以循名而責實矣。四曰擇長官。委中書、樞密院先選轉運使、提點刑獄、大藩知州;次委兩制、三司、御史臺、開封府官、諸路監司舉知州、通判;知州通判舉知縣、令。限其人數,以舉主多者從中書選除。刺史、縣令,可以得人矣。五曰均公田。
 外官廩給不均,何以求其為善耶?請均其入,第給之,使有以自養,然後可以責廉節,而不法者可誅廢矣。六曰厚農桑。每歲預下諸路,風吏民言農田利害,堤堰渠塘,州縣選官治之。定勸課之法以興農利,減漕運。江南之圩田,浙西之河塘,隳廢者可興矣。七曰修武備。約府兵法,募畿輔強壯為衛士,以助正兵。三時務農,一時教戰,省給贍之費。畿輔有成法,則諸道皆可舉行矣。八曰推恩信。赦令有所施行,主司稽違者,重置於法;別遣使按
 視其所當行者,所在無廢格上恩者矣。九曰重命令。法度所以示信也,行之未幾,旋即厘改。請政事之臣參議可以久行者,刪去煩冗,裁為制敕行下,命令不至於數變更矣。十曰減徭役。戶口耗少而供億滋多,省縣邑戶少者為鎮,並使、州兩院為一,職官白直,給以州兵,其不應受役者悉歸之農,民無重困之憂矣。



 天子方信向仲淹,悉採用之,宜著令者,皆以詔書畫一頒下;獨府兵法,眾以為不可而止。



 又建言:「周制,三公分兼六官之職,漢
 以三公分部六卿,唐以宰相分判六曹。今中書,古天官塚宰也,樞密院,古夏官司馬也。四官散於群有司,無三公兼領之重。而二府惟進擢差除,循資級,議賞罰,檢用條例而已。上非三公論道之任,下無六卿佐王之職,非治法也。臣請仿前代,以三司、司農、審官、流內銓、三班院、國子監、太常、刑部、審刑、大理、群牧、殿前馬步軍司,各委輔臣兼判其事。凡官吏黜陟、刑法重輕、事有利害者,並從輔臣予奪:其體大者,二府僉議奏裁。臣請自領兵賦
 之職,如其無補,請先黜降。」章得像等皆曰不可。久之,乃命參知政事賈昌朝領農田,仲淹領刑法,然卒不果行。



 初,仲淹以忤呂夷簡,放逐者數年,士大夫持二人曲直,交指為朋黨。及陜西用兵,天子以仲淹士望所屬,拔用之。及夷簡罷,召還,倚以為治,中外想望其功業。而仲淹以天下為己任,裁削幸濫,考核官吏,日夜謀慮興致太平。然更張無漸,規摹闊大,論者以為不可行。及按察使出,多所舉劾,人心不悅。自任子之恩薄,磨勘之法密,
 僥幸者不便,於是謗毀稍行,而朋黨之論浸聞上矣。



 會邊陲有警,因與樞密副使富弼請行邊。於是,以仲淹為河東、陜西宣撫使,賜黃金百兩,悉分遺邊將。麟州新罹大寇,言者多請棄之,仲淹為修故砦,招還流亡三千餘戶,蠲其稅,罷榷酤予民。又奏免府州商稅,河外遂安。比去。攻者益急,仲淹亦自請罷政事,乃以為資政殿學士、陜西四路宣撫使、知邠州。其在中書所施為,亦稍稍沮罷。



 以疾請鄧州,進給事中。徙荊南,鄧人遮使者請留,仲淹
 亦願留鄧,許之。尋徙杭州,再遷戶部侍郎,徙青州。會病甚,請穎州,未至而卒,年六十四。贈兵部尚書,謚文正。初,仲淹病,帝常遣使賜藥存問,既卒,嗟悼久之。又遣使就問其家,既葬,帝親書其碑曰「褒賢之碑。」



 仲淹內剛外和,性至孝,以母在時方貧,其後雖貴,非賓客不重肉。妻子衣食,僅能自充。而好施予,置義莊里中,以贍族人。泛愛樂善,士多出其門下,雖里巷之人,皆能道其名字。死之日,四方聞者,皆為嘆息。為政尚忠厚,所至有恩,邠、慶二
 州之民與屬羌,皆畫像立生祠事之。及其卒也,羌酋數百人,哭之如父,齋三日而去。四子:純祐、純仁、純禮、純粹。



 純祐字天成,性英悟自得,尚節行。方十歲,能讀諸書;為文章,籍籍有稱。父仲淹守蘇州,首建郡學,聘胡瑗為師。瑗立學規良密,生徒數百,多不率教,仲淹患之。純祐尚未冠,輒白入學,齒諸生之末,盡行其規,諸生隨之,遂不敢犯。自是蘇學為諸郡倡。寶元中,西夏叛,仲淹連官關陜,皆將兵。純祐與將卒錯處,鉤深擿隱,得其才否。由是
 仲淹任人無失,而屢有功。仲淹帥環慶,議城馬鋪砦,砦逼夏境,夏懼扼其沖,侵撓其役。純祐率兵馳據其地,夏眾大至,且戰且役,數日而成,一路恃之以安。純祐事父母孝,未嘗違左右,不應科第。及仲淹以讒罷,純祐不得已,蔭守將作監主簿,又為司竹監,以非所好,即解去。從仲淹之鄧,得疾昏廢,臥許昌。富弼守淮西,過省之,猶能感慨道忠義,問弼之來公耶私耶,弼曰「公」。純祐曰「公則可」。凡病十九年卒,年四十九。子正臣,守太常寺太祝。



 純禮字彞叟,以父仲淹蔭,為秘書省正字,簽書河南府判官,知陵臺令兼永安縣。永昭陵建,京西轉運使配木石磚甓及工徒於一路,獨永安不受令。使者以白陵使韓琦,琦曰:「范純禮豈不知此?將必有說。」他日,眾質之,純禮曰:「陵寢皆在邑境,歲時繕治無虛日,今乃與百縣均賦,曷若置此,使之奉常時用乎。」琦是其對。還朝,用為三司鹽鐵判官,以比部員外郎出知遂州。



 滬南有邊事,調度苛棘,純禮一以靜待之,辨其可具者,不取於民。民圖
 像於廬,而奉之如神,名曰「範公庵」。草場火,民情疑怖,守吏惕息俟誅。純禮曰:「草濕則生火,何足怪!」但使密償之。庫吏盜絲多罪至死,純禮曰:「以棼然之絲而殺之,吾不忍也。」聽其家趣買以贖,命釋其株連者。除戶部郎中、京西轉運副使。



 元祐初,入為吏部郎中,遷左司。又遷太常少卿、江淮荊浙發運使。以光祿卿召,遷刑部侍郎,進給事中。純禮凡所封駁,正名分紀綱,皆國體之大者。張耒除起居舍人,病未能朝,而令先供職。純禮批敕曰:「臣僚
 未有以疾謁告,不赴朝參先視事者。耒能供職,豈不能見君?壞禮亂法,所不當為。」聞者皆悚動。御史中丞擊執政,將遂代其位,先以諷純禮。純禮曰:「論人而奪之位,寧不避嫌邪?命果下,吾必還之。」宰相即徙純禮刑部侍郎,而後出命。轉吏部,改天章閣待制、樞密都承旨,去知亳州、提舉明道宮。



 徽宗立,以龍圖閣直學士知開封府。前尹以刻深為治,純禮曰:「寬猛相濟,聖人之訓。今處深文之後,若益以猛,是以火濟火也。方務去前之苛,猶慮未
 盡,豈有寬為患也。」由是一切以寬處之。中旨鞫享澤村民謀逆,純禮審其故,此民入戲場觀優,歸途見匠者作桶,取而戴於首曰:「與劉先生如何?」遂為匠擒。明日入對,徽宗問何以處之,對曰:「愚人村野無所知,若以叛逆蔽罪,恐辜好生之德,以不應為杖之,足矣。」曰:「何以戒後人?」曰:「正欲外間知陛下刑憲不濫,足以為訓爾。」徽宗從之。



 拜禮部尚書,擢尚書右丞。侍御史陳次升乞除罷言官並自內批,不由三省進擬,右相曾布力爭不能得,乞降
 黜次升。純禮徐進曰:「次升何罪?不過防柄臣各引所親,且去不附己者爾。」徽宗曰:「然。」乃寢布議。



 呂惠卿告老,徽宗問執政,執政欲許之。純禮曰:「惠卿嘗輔政,其人固不足重,然當存國體。」曾布奏:「議者多憂財用不足,此非所急也,願陛下勿以為慮。」純禮曰:「古者無三年之蓄,曰國非其國。今大農告匱,帑庾枵空,而曰不足慮,非面謾邪?」因從容諫曰:「邇者朝廷命令,莫不是元豐而非元祐。以臣觀之,神宗立法之意固善,吏推行之,或有失當,以致
 病民。宣仁聽斷,一時小有潤色,蓋大臣識見異同,非必盡懷邪為私也。今議論之臣,有不得志,故挾此借口。以元豐為是,則欲賢元豐之人;以元祐為非,則欲斥元祐之士,其心豈恤國事?直欲快私忿以售其奸,不可不深察也。」



 又曰:「自古天下汨亂,系於用人。祖宗於此,最得其要。太祖用呂餘慶,太宗用王禹偁,真宗用張知白,皆從下列置諸要途。人君欲得英傑之心,固當不次飭拔。必待薦而後用,則守正特立之士,將終身晦跡矣。」左司諫
 江公望論繼述事當執中道,不可拘一偏。徽宗出示其疏,純禮贊之曰:「願陛下以曉中外,使知聖意所向,亦足以革小人徇利之情。乞褒遷公望,以勸來者。」



 純禮沉毅剛正,曾布憚之,激駙馬都尉王詵曰:「上欲除君承旨,範右丞不可。」詵怒。會詵館遼使,純禮主宴,詵誣其輒斥御名,罷為端明殿學士、知穎昌府,提舉崇福宮。崇寧中,啟黨禁,貶試少府監,分司南京。又貶靜江軍節度副使,徐州安置,徙單州。五年,復左朝議大夫,提舉鴻慶宮。卒,年
 七十六。



 純粹字德孺,以蔭遷至贊善大夫、檢正中書刑房,與同列有爭,出知滕縣,遷提舉成都諸路茶場。元豐中,為陜西轉運判官。時五路出師伐西夏:高遵裕出環慶,劉昌祚出涇原,李憲出熙河,種諤出鄜延,王中正出河東。遵裕怒昌祚後期,欲按誅之,昌祚憂恚病臥,其麾下皆憤焉。純粹恐兩軍不協,致生他變,勸遵裕往問昌祚疾,其難遂解。神宗責諸將無功,謀欲再舉。純粹奏:「關陜事力
 單竭,公私大困,若復加騷動,根本可憂。異時言者必職臣是咎,臣寧受盡言之罪於今日,不忍默默以貽後悔。」神宗納之,進為副使。



 吳居厚為京東轉運使,數獻羨賦。神宗將以徐州大錢二十萬緡助陜西,純粹語其僚曰:「吾部雖急,忍復取此膏血之餘?」即奏:「本路得錢誠為利,自徐至邊,勞費甚矣。」懇辭弗受。入為右司郎中。哲宗立,居厚敗,命純粹以直龍圖閣往代之,盡革其苛政。時蘇軾自登州召還,純粹與軾同建募役之議,軾謂純粹講
 此事尤為精詳。



 復代兄純仁知慶州。時與夏議分疆界,純粹請棄所取夏地,曰:「爭地未棄,則邊隙無時可除。如河東之葭蘆、吳堡,鄜延之米脂、羲合、浮圖,環慶之安疆,深在夏境,於漢界地利形勢,略無所益。而蘭、會之地,耗蠹尤深,不可不棄。」所言皆略施行。純粹又言:「諸路策應,舊制也。自徐禧罷策應,若夏兵大舉,一路攻圍,力有不勝,而鄰路拱手坐觀,其不拔者幸爾。今宜修明戰守救援之法。」朝廷是之。及夏侵涇原,純粹遣將曲珍救之,曰:「
 本道首建應援牽制之策,臣子之義,忘軀徇國,無謂鄰路被寇,非我職也。」珍即日疾馳三百里,破之於曲律,搗橫山,夏眾遁去。元祐中,除寶文閣待制,再任,召為戶部侍郎,又出知延州。



 紹聖初。哲宗親政,用事者欲開邊釁,御史郭知章遂論純粹元祐棄地事,降直龍圖閣。明年,復以寶文閣待制知熙州。章惇、蔡卞經略西夏,疑純粹不與共事,改知鄧州。歷河南府、滑州,旋以元祐黨人奪職,知均州。徽宗立,起知信州,復故職,知太原,加龍圖閣
 直學士,再臨延州。改知永興軍。尋以言者落職,知金州,提舉鴻慶宮。又責常州別駕,鄂州安置,錮子弟不得擅入都。會赦,復領祠。久之,以右文殿修撰提舉太清宮。黨禁解,復徽猷閣待制,致仕。卒,年七十餘。



 純粹沉毅有幹略,才應時須,嘗論賣官之濫,以為:「國法固許進納取官,然未嘗聽其理選。今西北三路,許納三千二百緡買齋郎,四千六百緡買供奉職,並免試注官。夫天下士大夫服勤至於垂死,不沾世恩,其富民猾商,捐錢千萬,則可
 任三子,切為朝廷惜之。」疏上,不聽。凡論事剴切類此。



 純仁字堯夫,其始生之夕,母李氏夢兒墮月中,承以衣裾,得之,遂生純仁。資警悟,八歲,能講所授書。以父任為太常寺太祝。中皇祐元年進士第,調知武進縣,以遠親不赴;易長葛,又不往。仲淹曰:「汝昔日以遠為言,今近矣,復何辭?」純仁曰:「豈可重於祿食,而輕去父母邪?雖近,亦不能遂養焉。」仲淹門下多賢士,如胡瑗、孫復、石介、李覯之徙,純仁皆與從游。晝夜肄業,至夜分不寢,置燈帳中,
 帳頂如墨色。



 仲俺沒,始出仕,以著作佐郎知襄城縣。兄純祐有心疾,奉之如父,藥膳居服,皆躬親時節之。賈昌朝守北都,請參幕府,以兄辭。宋庠薦試館職,謝曰:「輦轂之下,非兄養疾地也。」富弼責之曰:「臺閣之任豈易得?何庸如是。」卒不就。襄城民不蠶織,勸使植桑,有罪而情輕者,視所植多寡除其罰,民益賴慕,後呼為「著作林」。兄死,葬洛陽。韓琦、富弼貽書洛尹,使助其葬,既葬,尹訝不先聞。純仁曰:「私室力足辦,豈宜慁公為哉?」



 簽書許州觀察
 判官、知襄邑縣。縣有牧地,衛士牧馬,以踐民稼,純仁捕一人杖之。牧地初不隸縣,主者怒曰:「天子宿衛,令敢爾邪?」白其事於上,劾治甚急。純仁言:「養兵出於稅畝,若使暴民田而不得問,稅安所出?」詔釋之,且聽牧地隸縣。凡牧地隸縣,自純仁始。時旱久不雨,純仁籍境內賈舟,諭之曰:「民將無食,爾所販五穀,貯之佛寺,候食闕時吾為糴之。」眾賈從命,所蓄十數萬斛。至春,諸縣皆饑,獨境內民不知也。



 治平中,擢江東轉運判官,召為殿中侍御史,
 遷侍御史。時方議濮王典禮,宰相韓琦、參知政事歐陽修等議尊崇之。翰林學士王珪等議,宜如先朝追贈期親尊屬故事。純仁言:「陛下受命仁宗而為之子,與前代定策入繼之主異,宜如王珪等議。」繼與御史呂誨等更論奏,不聽。純仁還所授告敕,家居待罪。既而皇太后手書尊王為皇,夫人為後。純仁復言:「陛下以長君臨御,奈何使命出房闈,異日或為權臣矯托之地,非人主自安計。」尋詔罷追尊,起純仁就職。純仁請出不已,遂通判安
 州,改知蘄州。歷京西提點刑獄、京西陜西轉運副使。



 召還,神宗問陜西城郭、甲兵、糧儲如何,對曰:「城郭粗全,甲兵粗修,糧儲粗備。」神宗愕然曰:「卿之才朕所倚信,何為皆言粗?」對曰:「粗者未精之辭,如是足矣。願陛下且無留意邊功,若邊臣觀望,將為他日意外之患。」拜兵部員外郎,兼起居舍人、同知諫院。奏言:「王安石變祖宗法度,掊克財利,民心不寧。《書》曰:『怨豈在明,不見是圖。』願陛下圖不見之怨。」神宗曰:「何謂不見之怨?」對曰:「杜牧所謂『天下
 之人,不敢言而敢怒』是也。」神宗嘉納之,曰:「卿善論事,宜為朕條古今治亂可為監戒者。」乃作《尚書解》以進,曰:「其言,皆堯、舜、禹、湯、文、武之事也。治天下無以易此,願深究而力行之。」加直集賢院、同修起居注。



 神宗切於求治,多延見疏逖小臣,咨訪闕失。純仁言:「小人之言,聽之若可採,行之必有累。蓋知小忘大,貪近昧遠,願加深察。」富弼在相位,稱疾家居。純仁言:「弼受三朝眷倚,當自任天下之重,而恤己深於恤物,憂疾過於憂邦,致主處身,二者
 胥失。弼與先臣素厚,臣在諫省,不錄私謁以致忠告,願示以此章,使之自省。」又論呂誨不當罷御史中丞,李師中不可守邊。



 及薛向任發運使,行均輸法於六路。純仁言:「臣嘗親奉德音,欲修先王補助之政。今乃效桑羊均輸之法,而使小人為之,掊克生靈,斂怨基禍。安石以富國強兵之術,啟迪上心,欲求近功,忘其舊學。尚法令則稱商鞅,言財利則背孟軻,鄙老成為因循,棄公論為流俗,異己者為不肖,合意者為賢人。劉琦、錢顗等一言,便
 蒙降黜。在廷之臣,方大半趨附。陛下又從而驅之,其將何所不至。道遠者理當馴致,事大者不可速成,人材不可急求,積敝不可頓革。儻欲事功亟就,必為憸佞所乘,宜速還言者而退安石,答中外之望。」不聽。遂求罷諫職,改判國子監,去意愈確。執政使諭之曰:「毋輕去,已議除知制誥矣。」純仁曰:「此言何為至於我哉,言不用,萬鐘非所顧也。」



 其所上章疏,語多激切。神宗悉不付外,純仁盡錄申中書,安石大怒,乞加重貶。神宗曰:「彼無罪,姑與一
 善地。」命知河中府,徙成都路轉運使。以新法不便,戒州縣未得遽行。安石怒純仁沮格,因讒者遣使欲捃摭私事,不能得。使者以他事鞭傷傳言者,屬官喜謂純仁曰:「此一事足以塞其謗,請聞於朝。」純仁既不奏使者之過,亦不折言者之非。後竟坐失察僚佐燕游,左遷知和州,徙邢州。未至,加直龍圖閣、知慶州。



 過闕入對,神宗曰:「卿父在慶著威名,今可謂世職。卿隨父既久,兵法必精,邊事必熟。」純仁揣神宗有功名心,即對曰:「臣儒家,未嘗學
 兵,先臣守邊時,臣尚幼,不復記憶,且今日事勢宜有不同。陛下使臣繕治城壘,愛養百姓,不敢辭;若開拓侵攘,願別謀帥臣。」神宗曰:「卿之才何所不能,顧不肯為朕悉心爾。」遂行。



 秦中方饑,擅發常平粟振貸。僚屬請奏而須報,純仁曰:「報至無及矣,吾當獨任其責。」或謗其所全活不實,詔遣使按視。會秋大稔,民歡曰:「公實活我,忍累公邪?」晝夜爭輸還之。使者至,已無所負。邠、寧間有叢塚,使者曰:「全活不實之罪,於此得矣。」發塚籍骸上之。詔本路
 監司窮治,乃前帥楚建中所封也。朝廷治建中罪,純仁上疏言:「建中守法,申請間不免有殍死者,已坐罪罷去。今緣按臣而及建中,是一罪再刑也。」建中猶贖銅三十斤。環州種古執熟羌為盜,流南方,過慶呼冤,純仁以屬吏,非盜也。古避罪讕訟,詔御史治於寧州。純仁就逮,民萬數遮馬涕泗,不得行,至有自投於河者。獄成,古以誣告謫。亦加純仁以他過,黜知信陽軍。



 移齊州。齊俗兇悍,人輕為盜劫。或謂:「此嚴治之猶不能戢,公一以寬,恐不
 勝其治矣。」純仁曰:「寬出於性,若強以猛,則不能持久;猛而不久,以治兇民,取玩之道也。」有西司理院,系囚常滿,皆屠販盜竊而督償者。純仁曰:「此何不保外使輸納邪?」通判曰:「此釋之,復紊,官司往往待其以疾斃於獄中,是與民除害爾。」純仁曰:「法不至死,以情殺之,豈理也邪?」盡呼至庭下,訓使自新,即釋去。期歲,盜減比年大半。



 丐罷,提舉西京留司御史臺。時耆賢多在洛,純仁及司馬光,皆好客而家貧,相約為真率會,脫粟一飯,酒數行,洛中
 以為勝事。復知河中,諸路閱保甲妨農,論救甚力。錄事參軍宋儋年暴死,純仁使子弟視喪,小殮,口鼻血出。純仁疑其非命,按得其妾與小吏奸,因會,寘毒鱉肉中。純仁問食肉在第幾巡,曰:「豈有既中毒而尚能終席者乎?」再訊之,則儋年素不食鱉,其曰毒鱉肉者,蓋妾與吏欲為變獄張本,以逃死爾。實儋年醉歸,毒於酒而殺之。遂正其罪。



 哲宗立,復直龍圖閣、知慶州。召為右諫議大夫,以親嫌辭,改天章閣待制兼侍講,除給事中。時宣仁後
 垂簾,司馬光為政,將盡改熙寧、元豐法度。純仁謂光:「去其太甚者可也。差役一事,尤當熟講而緩行,不然,滋為民病。願公虛心以延眾論,不必謀自己出;謀自己出,則諂諛得乘間迎合矣。役議或難回,則可先行之一路,以觀其究竟。」光不從,持之益豎。純仁曰:「是使人不得言爾。若欲媚公以為容悅,何如少年合安石以速富貴哉。」又云:「熙寧按問自首之法,既已行之,有司立文太深,四方死者視舊數倍,殆非先王寧失不經之意。」純仁素與光
 同志,及臨事規正,類如此。初,種古因誣純仁停任。至是,純仁薦為永興軍路鈐轄,又薦知隰州。每自咎曰:「先人與種氏上世有契義,純仁不肖,為其子孫所訟,寧論曲直哉。」



 元祐初,進吏部尚書,數日,同知樞密院事。初,純仁與議西夏,請罷兵棄地,使歸所掠漢人,執政持之未決。至是,乃申前議,又請歸一漢人予十縑。事皆施行。邊俘鬼章以獻,純仁請誅之塞上,以謝邊人,不聽。議者欲致其子,收河南故地,故赦不殺。後又欲官之,純仁復固爭,
 然鬼章子卒不至。



 三年,拜尚書右僕射兼中書侍郎。純仁在位,務以博大開上意,忠篤革士風。章惇得罪去,朝廷以其父老,欲畀便郡,既而中止。純仁請置往咎而念其私情。鄧綰帥淮東,言者斥之不已。純仁言:「臣嘗為綰誣奏坐黜,今日所陳為綰也,左降不宜錄人之過太深。」宣仁後嘉納。因下詔:「前日希合附會之人,一無所問。」



 學士蘇軾以發策問為言者所攻,韓維無名罷門下侍郎補外。純仁奏軾無罪,維盡心國家,不可因譖黜官。及王
 覿言事忤旨,純仁慮朋黨將熾,與文彥博、呂公著辨於簾前,未解。純仁曰:「朝臣本無黨,但善惡邪正,各以類分。彥博、公著皆累朝舊人,豈容雷同罔上。昔先臣與韓琦、富弼同慶歷柄任,各舉所知。常時飛語指為朋黨,三人相繼補外。造謗者公相慶曰:『一綱打盡。』此事未遠,願陛下戒之。」因極言前世朋黨之禍,並錄歐陽修《朋黨論》以進。



 知漢陽軍吳處厚傅致蔡確安州《車蓋亭詩》,以為謗宣仁後,上之。諫官欲寘於典憲,執政右其說,唯純仁與
 左丞王存以為不可。爭之未定,聞太師文彥博欲貶於嶺嶠,純仁謂左相呂大防曰:「此路自乾興以來,荊棘近七十年,吾輩開之,恐自不免。」大防遂不敢言。及確新州命下,純仁於宣仁後簾前言:「聖朝宜務寬厚,不可以語言文字之間曖昧不明之過,誅竄大臣。今舉動宜與將來為法,此事甚不可開端也。且以重刑除惡,如以猛藥治病,其過也,不能無損焉。」又與王存諫於哲宗,退而上疏,其略云:「蓋如父母之有逆子,雖天地鬼神不能容貸,
 父子至親,主於恕而已。若處之必死之地,則恐傷恩。」確卒貶新州。



 大防奏確黨人甚盛,不可不問。純仁面諫朋黨難辨,恐誤及善人。遂上疏曰:「朋黨之起,蓋因趣向異同,同我者謂之正人,異我者疑為邪黨。既惡其異我,則逆耳之言難至;既喜其同我,則迎合之佞日親。以至真偽莫知,賢愚倒置,國家之患,率由此也。至如王安石,正因喜同惡異,遂至黑白不分,至今風俗,猶以觀望為能,後來柄臣,固合永為商鑒。今蔡確不必推治黨人,旁及
 枝葉。臣聞孔子曰:『舉直錯諸枉,能使枉者直。』則是舉用正直,而可以化枉邪為善人,不仁者自當屏跡矣。何煩分辨黨人,或恐有傷仁化。」司諫吳安詩、正言劉安世交章擊純仁黨確,純仁亦力求罷。



 明年,以觀文殿學士知穎昌府。逾年,加大學士、知太原府。其境土狹民眾,惜地不葬。純仁遣僚屬收無主燼骨,別男女異穴,葬者三千餘。又推之一路,葬以萬數計。夏人犯境,朝廷欲罪將吏。純仁自引咎求貶。秋,有詔貶官一等,徙河南府,再徙穎
 昌。



 召還,復拜右僕射。因入謝,宣仁後簾中諭曰:「或謂卿必先引用王覿、彭汝礪,卿宜與呂大防一心。」對曰:「此二人實有士望,臣終不敢保位蔽賢,望陛下加察。」純仁將再入也,楊畏不悅,嘗有言,純仁不知。至是,大防約畏為助,欲引為諫議大夫。純仁曰:「諫官當用正人,畏不可用。」大防曰:「豈以畏嘗言公邪?」純仁始知之。後畏叛大防,凡有以害大防者,無所不至。宣仁後寢疾,召純仁曰:「卿父仲淹,可謂忠臣。在明肅皇后垂簾時,唯勸明肅盡母道;
 明肅上賓,唯勸仁宗盡子道。卿當似之。」純仁泣曰:「敢不盡忠。



 宣仁后崩,哲宗親政,純仁乞避位。哲宗語呂大防曰:「純仁有時望,不宜去,可為朕留之。」且趣入見,問:「先朝行青苗法如何?」對曰:「先帝愛民之意本深,但王安石立法過甚,激以賞罰,故官吏急切,以致害民。」退而上疏,其要以為「青苗非所當行,行之終不免擾民也」。



 是時,用二三大臣,皆從中出,侍從、臺諫官,亦多不由進擬。純仁言:「陛下初親政,四方拭目以觀,天下治亂,實本於此。舜舉
 皋陶,湯舉伊尹,不仁者遠。縱未能如古人,亦須極天下之選。」又群小力排宣仁後垂簾時事,純仁奏曰:「太皇保祐聖躬,功烈誠心,幽明共監,議者不恤國事,一何薄哉。」遂以仁宗禁言明肅垂簾事詔書上之。曰:「望陛下稽仿而行,以戒薄俗。」



 蘇轍論殿試策問,引漢昭變武帝法度事。哲宗震怒曰:「安得以漢武比先帝?」轍下殿待罪,眾不敢仰視。純仁從容言:「武帝雄才大略,史無貶辭。轍以比先帝,非謗也。陛下親事之始,進退大臣,不當如訶叱奴
 僕。」右丞鄧潤甫越次曰:「先帝法度,為司馬光、蘇轍壞盡。」純仁曰:「不然,法本無弊,弊則當改。」哲宗曰:「人謂秦皇、漢武。」純仁曰:「轍所論,事與時也,非人也。」哲宗為之少霽。轍平日與純仁多異,至是乃服謝純仁曰:「公佛地位中人也。」轍竟落職知汝州。



 全臺言蘇軾行呂惠卿告詞,訕謗先帝,黜知英州。純仁上疏曰:「熙寧法度,皆惠卿附會王安石建議,不副先帝愛民求治之意。至垂簾之際,始用言者,特行貶竄,今已八年矣。言者多當時御史,
 何故畏避不即納忠,今乃有是奏,豈非觀望邪?」御史來之邵言高士敦任成都鈐轄日不法事,及蘇轍所謫太近。純仁言:「之邵為成都監司,士敦有犯,自當按發。轍與政累年,之邵已作御史,亦無糾正,今乃繼有二奏,其情可知。」



 純仁凡薦引人材,必以天下公議,其人不知自純仁所出。或曰:「為宰相,豈可不牢籠天下士,使知出於門下?」純仁曰:』但朝廷進用不失正人,何必知出於我邪?」哲宗既召章惇為相,純仁堅請去,遂以觀文殿大學士加右正議
 大夫知穎昌府。入辭,哲宗曰:「卿不肯為朕留,雖在外,於時政有見,宜悉以聞,毋事形跡。」徙河南府,又徙陳州。初,哲宗嘗言:「貶謫之人,殆似永廢。」純仁前賀曰:「陛下念及此,堯、舜用心也。」



 既而呂大防等竄嶺表,會明堂肆赦,章惇先期言:「此數十人,當終身勿徙。」純仁聞而憂憤,欲齋戒上疏申理之。所親勸以勿為觸怒,萬一遠斥,非高年所宜。純仁曰:「事至於此,無一人敢言,若上心遂回,所系大矣。不然,死亦何憾。」乃疏曰:「大防等年老疾病,不習水
 土,炎荒非久處之地,又憂虞不測,何以自存。臣曾與大防等共事,多被排斥,陛下之所親見。臣之激切,止是仰報聖德。向來章惇、呂惠卿雖為貶謫,不出里居。臣向曾有言,深蒙陛下開納,陛下以一蔡確之故,常軫聖念。今赴彥若已死貶所,將不止一蔡確矣。願陛下斷自淵衷,將大防等引赦原放。」疏奏,忤惇意,詆為同罪,落職知隨州。



 明年,又貶武安軍節度副使、永州安置。時疾失明,聞命怡然就道。或謂近名,純仁曰:「七十之年,兩目俱喪,萬
 里之行,豈其欲哉?但區區之愛君,有懷不盡,若避好名之嫌,則無為善之路矣。」每戒子弟毋得小有不平,聞諸子怨章惇,純仁必怒止之。江行赴貶所,舟覆,扶純仁出,衣盡濕。顧諸子曰:「此豈章惇為之哉?」既至永,韓維責均州,其子訴維執政日與司馬光不合,得免行。純仁之子欲以純仁與光議役法不同為請,純仁曰:「吾用君實薦,以至宰相。昔同朝論事不合則可,汝輩以為今日之言,則不可也。有愧心而生者,不若無愧心而死。」其子乃止。



 居三年,徽宗即位,欽聖顯肅後同聽政,即日授純仁光祿卿,分司南京,鄧州居住。遣中使至永賜茶藥,諭曰:「皇帝在藩邸,太皇太后在宮中,知公先朝言事忠直,今虛相位以待,不知目疾如何,用何人醫之。」純仁頓首謝。道除右正議大夫、提舉崇福宮。不數月,以觀文殿大學士、中太一宮使詔之。有曰:「豈唯尊德尚齒,昭示寵優;庶幾鯁論嘉謀,日聞忠告。」純仁以疾,捧詔而泣曰:「上果用我矣,死有餘責。」徽宗又遣中使賜茶藥,促入覲,仍宣渴見
 之意。



 純仁乞歸許養疾,徽宗不得已許之。每見輔臣問安否,乃曰:「范純仁,得一識面足矣。」遂遣上醫視疾。疾小愈,丐以所得冠帔改服色酬醫。詔賜醫章服,令以冠帔與族侄。疾革,以宣仁後誣謗未明為恨。呼諸子口占遺表,命門生李之儀次第之。其略云:「蓋嘗先天下而憂,期不負聖人之學,此先臣所以教子,而微臣資以事君。」又云:「惟宣仁之誣謗未明,致保祐之憂勤不顯。」又云:「未解疆埸之嚴,幾空帑藏之積。有城必守,得地難耕。」凡八事。
 建中靖國改元之旦,受家人賀。明日,熟寐而卒。年七十五。詔賻白金三十兩,敕許、洛官給其葬,贈開府儀同三司,謚曰忠宣,御書碑額曰:「世濟忠直之碑」。



 純仁性夷易寬簡,不以聲色加人,誼之所在,則挺然不少屈。自為布衣至宰相,廉儉如一,所得奉賜,皆以廣義莊;前後任子恩,多先疏族。沒之日,幼子、五孫猶未官。嘗曰:「吾平生所學,得之忠恕二字,一生用不盡。以至立朝事君,接待僚友,親睦宗族,未嘗須臾離此也。」每戒子弟曰:「人雖至愚,
 責人則明;雖有聰明,恕己則昏。茍能以責人之心責己,恕己之心恕人,不患不至聖賢地位也。」又戒曰:「《六經》,聖人之事也。知一字則行一字。要須『造次顛沛必於是』,則所謂『有為者亦若是』爾。豈不在人邪?」弟純粹在關陜,純仁慮其於西夏有立功意。與之書曰:「大輅與柴車爭逐,明珠與瓦礫相觸,君子與小人鬥力,中國與外邦校勝負,非唯不可勝,兼亦不足勝,不唯不足勝,雖勝亦非也。」親族有請教者,純仁曰:「惟儉可以助廉,惟恕可以成德。」
 其人書於坐隅。有文集五十卷,行於世。子正平、正思。



 正平字子夷,學行甚高,雖庸言必援《孝經》、《論語》。父純仁卒,詔特增遺澤,官其子孫,正平推與幼弟。紹聖中,為開封尉,有向氏於其墳造慈雲寺。戶部尚書蔡京以向氏後戚,規欲自結,奏拓四鄰田廬。民有訴者,正平按視,以為所拓皆民業,不可奪;民又撾鼓上訴,京坐罰金二十斤,用是蓄恨正平。



 及當國,乃言正平矯撰父遺表。又謂李之儀所述《純仁行狀》,妄載中使蔡克明傳二聖虛佇
 之意,遂以正平逮之儀、克明同詣御史府。正平將行,其弟正思曰:「議《行狀》時,兄方營窀穸之事,參預筆削者,正思也,兄何為哉?」正平曰:「時相意屬我,且我居長,我不往,兄弟俱將不免,不若身任之。」遂就獄,捶楚甚苦,皆欲誣服。獨克明曰:「舊制,凡傳聖語,受本於御前,請寶印出,注籍於內東門。」使從其家得永州傳宣聖語本有御寶,又驗內東門籍皆同。其遺表八事,諸子以朝廷大事,防後患,不敢上之,繳申穎昌府印寄軍資庫。自穎昌取至,亦
 實。獄遂解。正平羈管象州,之儀羈管太平州。正平家屬死者十餘人。



 會赦,得歸穎昌。唐君益為守,表其所居為忠直坊,取所賜「世濟忠直」碑額也。正平告之曰:「此朝廷所賜,施於金石,揭於墓隧,假寵於範氏子孫則可;若於通途廣陌中為往來之觀,以聳動庸俗,不可也。」君益曰:「此有司之事,君家何預焉?」正平曰:「先祖先君功名,人所知也。十室之邑,必有忠信,異時不獨吾家詒笑,君亦受其責矣。」竟撤去之。正平退閑久,益工詩,尤長五言,著《荀
 里退居編》,以壽終。



 論曰:自古一代帝王之興,必有一代名世之臣。宋有仲淹諸賢,無愧乎此。仲淹初在制中,遺宰相書,極論天下事,他日為政,盡行其言。諸葛孔明草廬始見昭烈數語,生平事業備見於是。豪傑自知之審,類如是乎!考其當朝,雖不能久,然先憂後樂之志,海內固已信其有弘毅之器,足任斯責,使究其所欲為,豈讓古人哉!」純仁位過其父,而幾有父風。元祐建議攻熙、豐太急,純仁救蔡確
 一事,所謂謀國甚遠,當世若從其言,元祐黨錮之禍,不至若是烈也。仲淹謂諸子,純仁得其忠,純禮得其靜,純粹得其略。知子孰與父哉!



\end{pinyinscope}