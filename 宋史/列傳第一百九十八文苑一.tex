\article{列傳第一百九十八文苑一}

\begin{pinyinscope}

 ○宋白
 梁周翰朱昂趙鄰幾何承裕附鄭起郭昱馬應和峴弟蒙附馮吉



 自古創業垂統之君,即其一時之好尚,而一代之規橅,
 可以豫知矣。藝祖革命,首用文吏而奪武臣之權,宋之尚文,端本乎此。太宗、真宗其在藩邸,已有好學之名,及其即位,彌文日增。自時厥後,子孫相承,上之為人君者,無不典學;下之為人臣者,自宰相以至令錄,無不擢科,海內文士,彬彬輩出焉。國初,楊億、劉筠猶襲唐人聲律之體,柳開、穆修志欲變古而力弗逮。廬陵歐陽修出,以古文倡,臨川王安石、眉山蘇軾、南豐曾鞏起而和之,宋文日趨於古矣。南渡文氣不及東都,豈不足以觀世變
 歟!作《文苑傳》。



 宋白,字太素,大名人。年十三,善屬文。多游鄠、杜間,嘗館於張瓊家,瓊武人,賞白有才,遇之甚厚。白豪俊,尚氣節,重交友,在詞場中稱甚著。



 建隆二年,竇儀典貢部,擢進士甲科。乾德初,獻文百軸,試拔萃高等,解褐授著作佐郎,廷賜襲衣、犀帶。蜀平,授玉津縣令。開寶中,閻丕、王洞交薦其才,宜預朝列。白以親老祈外任,連知蒲城、衛南二縣。



 太宗潛藩時,白嘗贄文,有襲衣之賜。及即位,擢為
 左拾遺,權知兗州,歲餘召還。泰山有唐玄宗刻銘,白摹本以獻,且述承平東人望幸之意。預修《太祖實錄》,俄直史館,判吏部南曹。從征太原,判行在御史臺。劉繼元降,翌日,奏《平晉頌》,太宗夜召至行宮褒慰,且曰:「俟還京師,當以璽書授職。」白謝於幄中。尋拜中書舍人,賜金紫。



 太平興國五年,與程羽同知貢舉,俄充史館修撰、判館事。八年,復典貢部,改集賢殿直學士、判院事。未幾,召入翰林為學士。雍熙中,召白與李昉集諸文士纂《文苑英華》
 一千卷。端拱初,加禮部侍郎,又知貢舉。白凡三掌貢士,頗致譏議,然所得士如蘇易簡、王禹偁、胡宿、李宗諤輩,皆其人也。是時,命復舊制,專委有司,白所取二十八人,罷退既眾,群議囂然。太宗遽召已黜者臨軒覆試,連放馬國祥、葉齊等八百餘人焉。



 白嘗過何承矩家,方陳倡優飲宴。有進士趙慶者,素無行檢,游承矩之門,因潛出拜白,求為薦名,及掌貢部,慶遂獲薦,人多指以為辭。又女弟適王沔,淳化二年,沔罷參知政事。時寇準方詆訐
 求進,故沔被出,復言白家用黃金器蓋舉人所賂,其實白嘗奉詔撰錢惟濬碑,得塗金器爾。



 張去華者,白同年生也,坐尼道安事貶。白素與去華厚善,遂出為保大軍節度行軍司馬。逾年,抗疏自陳,有「來日苦少,去日苦多」之語,太宗覽而憫之,召還,為衛尉卿,俄復拜為禮部侍郎,修國史。至道初,為翰林學士承旨。二年,遷戶部侍郎,俄兼秘書監。真宗即位,改吏部侍郎、判昭文館。



 先是,白獻擬陸贄《榜子集》,上察其意,欲求任用,遂命知開封府
 以試之,既而白倦於聽斷,求罷任。咸平四年,擢王欽若、馮拯、陳堯叟入掌機要,以白宿舊,拜禮部尚書。



 白學問宏博,屬文敏贍,然辭意放蕩,少法度。在內署久,頗厭番直,草辭疏略,多不愜旨。景德二年,與梁周翰俱罷,拜刑部尚書、集賢院學士、判院事。舊三館學士止五日內殿起居,會錢易上言,悉令赴外朝。白羸老步梗,就班足跌。未幾,抗表引年。上以舊臣,眷顧未允。再上表辭,乃以兵部尚書致仕,因就宰臣訪問其資產,虞其匱乏,時白繼
 母尚無恙,上東封,白肩與辭於北苑,召對久之,進吏部尚書,賜帛五十匹。



 大中祥符三年,丁內艱。五年正月,卒,年七十七。贈左僕射,錄其孫懿孫為將作監主簿,孝孫試秘書省校書郎,從子唐臣試正字。



 白善談謔,不拘小節,贍濟親族,撫恤孤藐,世稱其雍睦。聚書數萬卷,圖畫亦多奇古者。嘗類故事千餘門,號《建章集》。唐賢編集遺落者,白多纘綴之。後進之有文藝者,必極意稱獎,時彥多宗之,如胡旦、田錫,皆出其門下。陳彭年舉進士,輕俊
 喜嘲謗,白惡其為人,黜落之,彭年憾焉,後居近侍,為貢舉條制,多所關防,蓋為白設也。會有司謚白為文憲,內出密奏言白素無檢操,遂改文安。有集百卷。



 子憲臣,國子博士;得臣,賜進士及第,至太常丞;良臣,為太子中舍;忠臣,殿中丞。



 梁周翰,字元褒,鄭州管城人。父彥溫,廷州馬步軍都校。周翰幼好學,十歲能屬詞。周廣順二年舉進士,授虞城主簿,辭疾不赴。宰相範質、王溥以其聞人,不當佐外邑,
 改開封府戶曹參軍。宋初,質、溥仍為相,引為秘書郎、直史館。



 時左拾遺、知制誥高錫上封,議武成王廟配享七十二賢,內王僧辯以不令終,恐非全德。尋詔事部尚書張昭、工部尚書竇儀與錫重銓定,功業終始無瑕者方得預焉。周翰上言曰:



 臣聞天地以來,覆載之內,聖賢交騖,古今同流,校其顛末,鮮克具美。周公,聖人也,佐武王定天下,輔成王致治平,盛德大勛,蟠天極地。外則淮夷構難,內則管、蔡流言。疐尾跋胡,垂至顛頓;偃禾僕木,僅
 得辨明。此可謂之盡美哉?臣以為非也。孔子,聖人也,刪《詩》、《書》,定《禮》、《樂》,祖述堯、舜,憲章文、武。卒以棲遲去魯,奔走厄陳,雖試用於定、哀,曾不容於季、孟。又嘗履盜跖之虎尾,聞南子之珮聲,遠辱慎名,未見其可。此又可謂其盡善者哉?臣以為非也。自餘區區後賢,瑣瑣立事,比於二聖,曾何足雲?而欲責其磨涅不渝、始卒如一者,臣竊以為難其人矣。



 昉自唐室,崇祀太公。原其用意,蓋以天下雖大,不可去兵;域中有爭,未能無戰。資其祐民之道,立
 乎為武之宗,覬張國威,遂進王號。貞元之際,祀典益修,因以歷代武臣陪饗廟貌,如文宣釋奠之制,有弟子列侍之儀,事雖不經,義足垂勸。況於曩日,不乏通賢,疑難討論,亦云折中。今若求其考類,別立否臧,以羔袖之小疵,忘狐裘之大善,恐其所選,僅有可存。



 只如樂毅、廉頗,皆奔亡而為虜;韓信、彭越,悉菹醢而受誅。白起則錫劍杜郵,伍員則浮尸江澨。左車亦僨軍之將,孫臏實刑餘之人。穰苴則僨卒齊庭,吳起則非命楚國。周勃稱重,有
 置甲尚方之疑;陳平善謀,蒙受金諸將之謗。亞夫則死於獄吏,鄧艾則追於檻車。李廣後期而自剄,竇嬰樹黨而喪身。鄧禹敗於回溪,終身無董戎之寄;馬援死於蠻徼,還尸闕遣奠之儀。其餘諸葛亮之儔,事偏方之主;王景略之輩,佐閏位之君。關羽則為仇國所禽,張飛則遭帳下所害。凡此名將,悉皆人雄,茍欲指瑕,誰當無累?或從澄汰,盡可棄捐。況其功業穹隆,名稱烜赫。樵夫牧稚,咸所聞知;列將通侯,竊年思慕。若一旦除去神位,擯出
 祠庭,吹毛求異代之疵,投袂忿古人之惡,必使時情頓惑,竊議交興。景行高山,更奚瞻於往躅;英魂烈魄,將有恨於明時。



 況伏陛下方厲軍威,將遏亂略,講求兵法,締構武祠,蓋所以勸激戎臣,資假陰助。忽使長廊虛邈,僅有可圖之形;中殿前空,不見配食之坐。似非允當,臣竊惑焉。深惟事貴得中,用資體要,若今之可以議古,恐來者亦能非今。願納臣微忠,特追明敕,乞下此疏,廷議其長。



 不報。



 乾德中,獻《擬制》二十編,擢為右拾遺。會修大內,
 上《五鳳樓賦》,人多傳誦之。五代以來,文體卑弱,周翰與高錫、柳開、範杲習尚淳古,齊名友善,當時有「高、梁、柳、範」之稱。初,太祖嘗識彥溫於軍中,石守信亦與彥溫舊故。一日,太祖語守信,將用周翰掌誥,守信微露其言,周翰遽上表謝。太祖怒,遂寢其命。



 歷通判綿、眉二州,在眉州坐杖人至死,奪二官。起授太子左贊善大夫。開寶三年,遷右拾遺,監綾綿院,改左補闕兼知大理正事。會將郊祀,因上疏曰:「陛下再郊上帝,必覃赦宥。臣以天下至大,
 其中有慶澤所未及、節文所未該者,所宜推而廣之。方今賦稅年入至多,加以科變之物,名品非一,調發供輸,不無重困。且西蜀、淮南、荊、潭、廣、桂之地,皆以為王土。陛下誠能以三方所得之利,減諸道租賦之入,則庶乎均德澤而寬民力矣。」俄坐杖錦工過差,為其所訴。太祖甚怒,責之曰:「爾豈不知人之膚血與己無異,何乃遽為酷罰!」將杖之,周翰自言:「臣負天下才名,不當如是。」太祖乃解,止左授司農寺丞。逾年,為太子中允。



 太平興國中,知
 蘇州。周翰善音律,喜蒱博,惟以飲戲為務。州有伶官錢氏,家數百人,日令百人供妓,每出,必以殽具自隨。郡務不治,以本官分司西京。逾月,授左贊善大夫,仍分司。俄除楚州團練副使。雍熙中,宰相李昉以其名聞,召為右補闕,賜緋魚,使江、淮提點茶鹽。



 周翰以辭學為流輩所許,頻歷外任,不樂吏事。會翰林學士宋白等列奏其有史才,邅回下位,遂命兼史館修撰。會太宗親試貢士,周翰為考官,面賜金紫,因語宰相,稱其有文,尋遷起居
 舍人。淳化五年,張佖建議復置左右史之職,乃命周翰與李宗諤分領之。周翰兼起居郎,因上言:「自今崇政、長春殿皇帝宣諭之言,侍臣諭列之事,望依舊中書修為時政記。其樞密院事涉機密,亦令本院編纂,每至月終送史館。自餘百司凡於對拜、除改、沿革、制置之事,悉條報本院,以備編錄。仍令郎與舍人分直崇政殿,以記言動,別為起居注,每月先進御,後降付史館。」從之。起居注進御,自周翰等始也。周翰蚤有時譽,久擯廢,及被除擢,尤洽時
 論。



 會考課京朝官,有敢隱前犯者,皆除名為民。周翰被譴尤多,所上有司偶遺一事,當免。判館楊徽之率三館學士詣相府,以為周翰非故有規避,其實所犯頻繁,不能悉記,於是止罰金百斤。



 先是,趙安易建議於西川鑄大鐵錢,以一當十,周翰上言:「古者貨、幣、錢三者兼用,若錢少於貨、幣,即鑄大錢,或當百,或當五十,蓋欲廣其錢而足用爾。今不若使蜀民貿易者,凡鐵錢一止作一錢用,官中市物即以兩錢當一。又西川患在少鹽,請於益
 州置榷院,入物交易,則公私通濟矣。」至道中,遷工部郎中。



 真宗在儲宮,知其名,徵之,時為左庶子,因令取其所為文章,周翰悉纂以獻,上答以書。及即位,未行慶,首擢為駕部郎中、知制誥,俄判史館、昭文館。咸平三年,召入翰林為學士,受詔與趙安易同修屬籍。唐末喪亂,籍譜罕存,無所取則,周翰創意為之,頗有倫貫。車駕幸澶淵,命判留司御史臺,周翰懇求扈從,從之。明年,授給事中,與宋白俱罷學士。大中祥符元年,遷工部侍郎。逾年,被
 疾卒,年八十一。真宗憫之,錄其子忠寶為大理評事,給奉終喪。



 周翰性疏雋卞急,臨事過於嚴暴,故多曠敗。晚年才思稍減,書詔多不稱旨。有集五十卷及《續因話錄》。



 朱昂,字舉之,其先京兆人,世家渼陂。唐天復末,徙家南陽。梁祖篡唐,父葆光與唐舊臣顏蕘、李濤數輩挈家南渡,寓潭州。每正旦夕至,必序立南嶽祠前,北望號慟,殆二十年。後濤北歸,葆光樂衡山之勝,遂往家焉。



 昂少與熊若谷、鄧洵美同學。朱遵度好讀書,人號之為「朱萬卷」,
 目昂為「小萬卷」。昂嘗間行經廬陵,道遇異人,謂之曰:「中原不久當有真主平一天下,子仕至四品,安用南為?」遂北游江、淮。時周世宗南征,韓令坤統兵至揚州,昂謁見,陳治亂方略,令坤奇之,署權知揚州揚子縣。適兵革之際,逃亡過半,昂便宜綏輯,復逋亡者七千餘家,令坤即表授本縣令。



 宋初,為衡州錄事參軍,嘗讀陶潛《閑情賦》而慕之,因廣其辭曰:



 維稟氣兮清濁,獨得意兮虛徐。耳何聰兮無瑱,衣何散兮無裾。務冥懷於得喪,寧勤體乎
 菑畬。將使同方姬、孔,抗跡孫、蘧。精騖廣漠,心游太虛。傲朝曦兮南榮,溯夕飆兮北疏。非道之病,惟情之舒。



 繇是含穎懷粹,凝和習懿。器CO淪兮幽憂,德芬馨兮周比。井無渫兮泉融,珠潛輝兮川媚。又何必陋雄之尚《玄》,笑奕之心醉,悲墨之素絲,嘆展之下位?茍因時之明揚,乃斯文之不墜。



 睇煙景兮飄飄,心懸旌兮搖搖。感朝榮而夕落,嗟響蛩而鳴蜩。



 姑藏器以有待,因寄物而長謠。願在首而為弁,束玄發而未衰。



 會名器之有得,與纓珥兮相
 宜。願在足而為舄,何坎險之罹憂。



 欲效勤於豎亥,思追踵於浮丘。願在服而為袂,傳繒素而飾躬。



 異化緇之色涅,寧拭面而道窮。願在目而為鑒,分妍醜於崇朝。



 驚青陽之難久,庶白首以見招。願在地而為簟,當暑溽而冰寒。



 伊膚革之尚疚,胡寤寐以求安?願在觴而為醴,不亂德而溺真。



 體虛受之為器,革譎性以歸淳。願在握而為劍,每輔衽而保裾。



 殊鉛銛之效用,比硎刃而有餘。願在橐而為矢,美筈羽之斯全。



 疇懋勛而錫晉,射窮壘而衄
 燕。願在體而為裘,托針縷以成功。



 非珍華而取飾,將被服而有容。願在軒而為篁,貫歲寒而不改。



 挺介節以自持,廓虛心而有待。



 人之願兮實繁,我之心兮若此。蓄為志兮璞藏,發為文兮霧委。既持瑾兮掌瑜,復擷蘭兮藝芷。始無言兮植杖,終俯首兮嗟髀。振襟兮自適,覿物兮解頤。雲無心兮遐舉,蘿倚幹兮叢滋。想陵穀之變地,況玄黃之易絲。人可汰而可鍛,己不磷而不緇。茍一鳴而驚人,何五鼎而勿飴?



 已而擁膝清嘯,傾懷自寬。樞桑戶
 蓽兮差樂,鳩飛梭躍兮胡難。指夜蟾兮為伍,仰疏籟兮邀歡。何孫牧而伊耕?何巢箕而呂磻?滌我慮兮綠綺。清我眠兮瑯玕。周旋兮有則,徙倚兮可觀。終卷舒兮自得,契休哉於《考槃》。



 李昉知州事,暇日多召語,且以文為贄,昉深所嗟賞。歷宜城令。開寶中,拜太子洗馬、知蓬州,徙廣安軍。會渠州妖賊李仙眾萬人劫掠軍界,昂設策禽之。自餘果、合、渝、涪四州民連結為妖者,置不問,蜀民遂安。宰相薛居正稱其能,遷殿中丞、知泗州。



 嘗作《隋河辭》,
 謂浚決之病民,游觀之傷財,乃天意之所以亡隋也。使隋不興役費財以害其民,則安得有今日之利哉!



 嘗聚淮水流尸三千,為塚瘞之。有戍卒謀亂,昂誅其首惡,凡支黨之詿誤者悉貰之。就遷監察御史、江南轉運副使。太平興國二年,知鄂州,加殿中侍御史,為峽路轉運副使,就改庫部員外郎,遷轉運使。端拱二年,以本官直秘閣,賜金紫。久之,出知復州,表求謝事,不許。遷水部郎中,復請老,召還,再直秘閣,尋兼越王府記室參軍。



 直宗即
 位,遷秩司封郎中,俄知制誥,判史館,受詔編次三館秘閣書籍,既畢,加吏部。咸平二年,召入翰林為學士。逾年,拜章乞骸骨,召對,敦諭,請彌確,乃拜工部侍郎致仕。翌日,遣使就第賜器幣,給全奉,詔本府歲時存問,章奏聽附驛以聞。命其子正辭知公安縣,以便侍養,許歸江陵。舊制,致仕官止謝殿門外,昂特延見命坐,恩禮甚厚。令俟秋涼上道,遣中使賜宴於玉津園,兩制三館皆預,仍詔賦詩餞行,縉紳榮之。



 昂前後所得奉賜,以三之一
 購奇書,以諷誦為樂。及是閑居,自稱退叟,著《資理論》三卷上之,詔以其書付史館。弟協以純謹著稱,仕至主客郎中、雍王府翊善。昂以書招之,協亦告老歸。兄弟皆眉壽,時人比漢之二疏。知府陳堯咨署其居曰東、西致政坊。昂於所居建二亭:曰知止,曰幽棲。頗好釋氏書。晚歲自為墓志。景德四年卒,年八十三,門人謚曰正裕先生。詔加賻贈,錄其孫適出身。



 昂好學,純厚有清節,澹於榮利,為洗馬十五年,不以屑意。居內署,非公事不至兩府。在
 王邸時,真宗居儲宮,知其素守,故每加褒進,然昂未嘗有所私請,進退存禮,士類多之。有集三十卷。子正彞、正辭並登進士第,正基虞部員外郎。



 趙鄰幾,字亞之,鄆州須城人,家世為農。鄰幾少好學,能屬文,嘗作《禹別九州賦》,凡萬餘言,人多傳誦。



 周顯德二年舉進士,解褐秘書省校書郎,歷許州、宋州從事。太平興國初,召為左贊善大夫、直史館,改宗正丞。四年,郭贄、宋白授中書舍人,告謝日交薦之,俄而鄰幾獻頌,上覽
 而嘉之,遷左補闕、知制誥,數月卒,年五十九。中使護葬。



 鄰幾體貌尪弱,如不勝衣。為文浩博,慕徐、庾及王、楊、盧、駱之體,每構思,必斂衣任危坐,成千言始下筆。屬對精切,致意縝密,時輩咸推服之。及掌誥命,頗繁富冗長,不達體要,無稱職之譽。



 常欲追補唐武宗以來實錄,孜孜訪求遺事,殆廢寢食,會疾革,唯以書未成為恨。至淳化中,參知政事蘇易簡因言及鄰幾追補《唐實錄》事,鄰幾一子東之,以蔭補郎山主簿,部送軍糧詣北邊,沒焉,其家
 屬寄居睢陽。太宗遣直史館錢熙往取其書,得鄰幾所補《會昌以來日歷》二十六卷及文集三十四卷,所著《鯫子》一卷、《六帝年略》一卷、《史氏懋官志》五卷,並他書五十餘卷來上,皆塗竄之筆也。詔賜其家錢十萬。



 時又有何承裕者,晉天福末擢進士第,有清才,好為歌詩,而嗜酒狂逸。初為中都主簿,桑維翰鎮兗州,知其直率,不責以吏事。累官至著作佐郎、直史館,出為盩啡、咸陽二縣令,醉則露首跨牛趨府,府尹王彥超以其名士而容之,然
 為治清而不煩,民頗安焉。每覽牒訴,必戲判以喻曲直,訴者多心伏引去。往往召豪吏接坐,引滿,吏因醉挾私白事,承裕悟之,笑曰:「此見罔也,當受杖。」杖訖,復召與飲。其無檢多類此。



 開寶三年,自涇陽令入為監察御史,後歷侍御史,累知忠、萬、商三州,太平興國中卒。



 鄭起,字孟隆,不知何許人。少游京、洛間,佻薄無檢操。聞襄州雙泉寺僧能為黃金,往依焉,遂削發為侍者。久之,知其誑耀,乃反初服。舉進士,時舉子多尚詩賦,惟起有
 文七軸,歌詩尤清麗。周廣順初,調補尉氏主簿,秩滿,以書乾宰相範質,薦為右拾遺、直史館。恭帝初,遷殿中侍御史。



 乾德初,出掌泗州市征。刺史張延範檢校司徒,官吏呼以「太保」。起貧,常乘騾。一日,從延範出近郊送客,延範揖起曰:「請策馬令進。」起曰:「此騾也,不當過呼耳。」以譏延範,延範深銜之,密奏起嗜酒廢職。



 初,顯德末,起見太祖握禁兵,有人望,乃上書範質,極言其事。又嘗遇太祖於路,橫絕前導而過,太祖亦弗之怒。及延範奏至,出為
 河西令。會蜀平,當徙遠官,起不欲往,乃炙烙其足,因是成疾而卒。



 起負才倨傲,多所詆訐,數為群小窘辱,終亦不改。



 時有郭昱者,好為古文,狹中詭僻。周顯德中登進士第,恥赴常選,獻書於宰相趙普,自比巢、由,朝議惡其矯激,故久不調。後復伺普,望塵自陳,普笑謂人曰:「今日甚榮,得巢、由拜於馬首。」開寶末,普出鎮河陽,昱詣薛居正上書,極言謗普,居正奏之,詔署襄州觀察推官。潘美鎮襄陽,討金陵,以昱隨軍。昱中夜被酒號叫,軍中皆驚,
 翌日,美遣還。歲餘,坐盜用官錢除名,因居襄陽,游索樊、鄧間,雍熙中卒。



 又有馬應者,薄有文藝,多服道士衣,自稱「先生」。開寶初效元結《中興頌》作《勃興頌》,以述太祖下荊、湖之功,欲刊石於永州結《頌》之側,縣令惡其誇誕,不以聞。太平興國初,登第,授大理評事,坐事除名,羈旅積年。淳化中,以詩乾同年殿中丞牛景,景因奏上,太宗覽而嘉之,復授大理評事,未幾卒。



 又有穎贄、董淳、劉從義善為文章,張翼、譚用之善為詩,張之翰善箋啟。贄拔
 萃登科,至太子中允。淳為工部員外郎、直史館,奉詔撰《孟昶紀事》。從義多藏書,嘗纘長安碑文為《遺風集》二十卷。餘皆官不達。



 和峴,字晦仁,開封浚儀人。父凝,晉宰相、太子太傅、魯國公。峴生之年,適會凝入翰林、加金紫、知貢舉,凝喜曰:「我平生美事,三者並集,此子宜於我也。」因名之曰三美。七歲,以門蔭為左千牛備身,遷著作佐郎。漢乾祐初,加朝散階。十六,登朝為著作郎。丁父憂,服闋,拜太常丞。



 建隆
 初,授太常博士,從祀南郊,贊導乘輿,進退閑雅。太祖謂近侍曰:「此誰氏之子,熟於贊相?」左右即以峴門閥對。俄拜刑部員外郎兼博士,仍判太常寺。



 乾德元年十一月甲子,有事於南郊。丁丑冬至,有司復請祀昊天上帝,詔峴議其禮,峴以祭義戒於煩數,請罷之。二年,議孝明、孝惠二後神主祔於別廟,峴以舊禮有二後同廟之文,無各殿異室之說,今二後同祔別廟,亦宜共殿別室。孝明皇后嘗母儀天下,宜居上室。孝惠皇后止以追尊,當居
 次室。從之。三年春,初克夔州,以內衣庫使李光睿權知州,峴通判州事。代還,是歲十二月十四日戊戌臘,有司以七日辛卯蠟百神,峴獻議正之。四年,南郊,峴建議望燎位置爟火。



 又嘗言:「依舊典,宗廟殿庭設宮縣三十六架,加鼓吹熊羆十二案,朝會登歌用五瑞,郊廟奠獻用四瑞,回仗至樓前奏《採茨》之曲,御樓奏《隆安》之曲,各用樂章。」復舉唐故事,宗廟祭科處別設珍膳,用申孝享之意。又謂「《八佾》之舞以象文德武功,請用《玄德升聞》、《天下
 大定》二舞」。並從其議。事具《禮》、《樂志》。



 先是,王樸、竇儼洞曉音樂,前代不協律呂者多所考正。樸、儼既沒,未有繼其職者。會太祖以雅樂聲高,詔峴講求其理,以均節之,自是八音和暢,上甚嘉之。語具《律志》。樂器中有叉手笛者,上意欲增入雅樂,峴即令樂工調品,以諧律呂,其執持之狀如拱揖然,請目曰「拱辰管」,詔備於樂府。



 開寶初,遷司勛員外郎、權知泗州,判吏部南曹,歷夔、晉二州通判。九年,江南平,受詔採訪。太宗即位,遷主客郎中。太平興
 國二年,知兗州,改京東轉運使。



 峴性苛刻鄙吝,好殖財,復輕侮人,嘗以官船載私貨販易規利。初為判官鄭同度論奏,既而彰信軍節度劉遇亦上言,按得實,坐削籍,配隸汝州。



 六年,起為太常丞,分司西京,復階勛章服。端拱初,上躬耕籍田,峴奉留司賀表至闕下,因以其所著《奉常集》五卷、《秘閣集》二十卷、《注釋武成王廟贊》五卷奏御,上甚嘉之,復授主客郎中,判太常寺兼禮儀院事。



 是秋得暴疾,卒,年五十六。弟㠓。



 㠓字顯仁,凝第四子也。生五六歲,凝教之誦古詩賦,一歷輒不忘。試令詠物為四句詩,頗有思致,凝嘆賞而奇之,語峴曰:「此兒他日必以文章顯,吾老矣,不見,汝曹善保護之。」



 太平興國八年擢進士第,釋褐霍丘主簿。雍熙初,知崇仁縣,就拜大理評事。江南轉運楊緘以其材干奏,移知南昌縣。代還,刑部取為詳覆官,遷光祿寺丞。



 先是,凝嘗取古今史傳聽訟斷獄、辨雪冤枉等事著為《疑獄集》,㠓因增益事類,分為三卷,表上之。俄獻所著文賦
 五十軸,召試中書,擢為太子中允。先是,馮起撰《御前登第三榜碑》以獻,上甚稱獎,命直史館。淳化初,㠓又撰《七榜題名記》,並補注凝所撰《古今孝悌集成》十卷以獻,遂以本官直集賢院,中謝日,賜緋魚。三年春,獻《觀燈賦》,詔付史館,遷右正言。



 是歲,太宗親試貢士,㠓預考校,作歌以獻,上對宰相稱賞之,召問年幾何。時摹印《儒行篇》,以賜新及第人及三館、臺省官,皆上表稱謝。上時御便坐,出表以示宰相,而㠓與張洎尤稱上旨,因謂李昉曰:「㠓
 ,宰相子,勤學自立,有文章,能荷堂構,如㠓者不可多得也。」遂以本官知制誥。不逾年,加水部員外郎、知理檢院。至道元年,賜金紫,與王旦同判吏部銓。是秋,晨起將朝,風眩暴作而卒,年四十五。上聞之驚嘆,遣中使就家問疾狀,並恤其孤,賵賻加等。長子珙才十歲,即授大理評事。次子璬,補太廟齋郎。



 㠓好修飾容儀,自五鼓張燈燭至辨色,冠帶方畢。雖幼能屬文,殊少警策。每草制,必精思討索而後成,拘於引類偶對,頗失典誥之體。上以其
 貴家子,能業文,甚寵待之,欲召入翰林,謂近臣曰:「㠓眸子眊眊然,胸中必不正,不可以居近侍也。」其命遂寢。



 㠓弟嵲始為三班奉職,淳化中,獻文求試,上以故相之後,改授大理評事。



 馮吉,字惟一,河南洛陽人。父道,周太師、中書令,追封瀛王。吉,晉天福初以父任秘書省校書郎,遷膳部、金部、職方員外郎,屯田、戶部、司勛郎中,累階金紫。周顯德中,遷太常少卿。



 吉嗜學,善屬文,工草隸,議者以掌誥許之。然
 性滑稽無操行,每中書舍人缺,宰相即欲用吉,終以佻薄而止。



 雅好琵琶,尤臻其妙,教坊供奉號名手者亦莫能及。父常戒令勿習,吉性所好,亦不能改。道欲辱之,因家宴,令吉奏琵琶為壽,賜以束帛,吉置於肩,左抱琵琶,按膝再拜如伶官狀,了無怍色,家人皆大笑。



 及為少卿,頗不得意,以杯酒自娛。每朝士宴集,雖不召,亦常自至,酒酣即彈琵琶,彈罷賦詩,詩成起舞。時人愛其俊逸,謂之「三絕」。



 宋初,受詔撰述《明憲皇太后謚議》,見稱於時。建
 隆四年卒,年四十五。



\end{pinyinscope}