\article{列傳第四十一}

\begin{pinyinscope}

 李沆
 弟維王旦向敏中



 李沆,字太初,洺州肥鄉人。曾祖豐,泰陵令。祖滔,洺州團練判官。父炳,從邢帥薛懷讓闢,為觀察支使。懷讓徙同州,又為掌書記,歷邠州、鳳翔判官,拜殿中侍御史、知舒
 州。太祖征金陵,緣淮供億,惟舒尤甚,以勞加侍御史,卒。



 沆少好學,器度宏遠,炳嘗語人曰:「此兒異日必至公輔。」太平興國五年,舉進士甲科,為將作監丞、通判潭州,遷右贊善大夫,轉著作郎。相府召試約束邊將詔書,既奏御,太宗甚悅,命直史館。雍熙三年,右拾遺王化基上書自薦,太宗謂宰相曰:「李沆、宋湜,皆嘉士也。」即命中書並化基召試,並除右補闕、知制誥。沆位最下,特升於上,各賜錢百萬。又以沆素貧,多負人錢,別賜三十萬償之。四
 年,與翰林學士宋白同知貢舉。謗議雖眾,而不歸咎於沆。遷職方員外郎,召入翰林為學士。



 淳化二年,判吏部銓。嘗侍曲宴,太宗目送之曰:「李沆風度端凝,真貴人也。」三年,拜給事中、參知政事。四年,以本官罷,奉朝請。未幾,丁內艱,起復,遂出知升州。未行,改知河南府。真宗升儲,遷禮部侍郎兼太子賓客,詔東宮待以師傅禮。真宗即位,遷戶部侍郎、參知政事。咸平初,以本官平章事,監修國史,改中書侍郎。



 會契丹犯邊,真宗北幸,命沆留守,京
 師肅然。真宗還,沆迎於郊,命坐置酒,慰勞久之。累加門下侍郎、尚書右僕射。真宗問治道所宜先,沆曰:「不用浮薄新進喜事之人,此最為先。」問其人,曰:「如梅詢、曾致堯等是矣。」後致堯副溫仲舒安撫陜西,於閣門疏言仲舒不足與共事。輕銳之黨無不稱快,沆不喜也,因用他人副仲舒,罷致堯。帝嘗語及唐人樹黨難制,遂使王室微弱,蓋奸邪難辨爾。沆對曰:「佞言似忠,奸言似信,至如盧杞蒙蔽德宗,李勉以為真奸邪是也。」真宗曰:「奸邪之跡,
 雖曰難辨,然久之自敗。」



 一夕,遣使持手詔欲以劉氏為貴妃,沆對使者引燭焚詔,附奏曰:「但道臣沆以為不可。」其議遂寢。駙馬都尉石保吉求為使相,復問沆,沆曰:「賞典之行,須有所自。保吉因緣戚里,無攻戰之勞,臺席之拜,恐騰物議。」他日再三問之,執議如初,遂止。帝以沆無密奏,謂之曰:「人皆有密啟,卿獨無,何也?」對曰:「臣待罪宰相,公事則公言之,何用密啟?夫人臣有密啟者,非讒即佞,臣常惡之,豈可效尤?」



 時李繼遷久叛,兵眾日盛,有圖
 取朔方之意。朝廷困於飛挽,中外咸以為靈州乃必爭之地,茍失之,則緣邊諸郡皆不可保。帝頗惑之,因訪於沆。沆曰:「繼遷不死,靈州非朝廷有也。莫若遣使密召州將,使部分軍民空壘而歸,如此,則關右之民息肩矣。」方眾議各異,未即從沆言,未幾而靈州陷,帝由是益重之。



 沆為相,王旦參政事,以西北用兵,或至旰食。旦嘆曰:「我輩安能坐致太平,得優游無事耶?」沆曰:「少有憂勤,足為警戒。他日四方寧謐,朝廷未必無事。」後契丹和親,旦問
 何如,沆曰:「善則善矣,然邊患既息,恐人主漸生侈心耳。」旦未以為然。沆又日取四方水旱盜賊奏之,旦以為細事不足煩上聽。沆曰:「人主少年,當使知四方艱難。不然,血氣方剛,不留意聲色犬馬,則土木、甲兵、禱祠之事作矣。吾老,不及見此,此參政他日之憂也。」沆沒後,真宗以契丹既和,西夏納款,遂封岱祠汾,大營宮觀,搜講墜典,靡有暇日。旦親見王欽若、丁謂等所為,欲諫則業已同之,欲去則上遇之厚,乃以沆先識之遠,嘆曰:「李文靖真
 聖人也。」當時遂謂之「聖相」。



 寇準與丁謂善,屢以謂才薦於沆,不用。準問之,沆曰:「顧其為人,可使之在人上乎?」準曰:「如謂者,相公終能抑之使在人下乎?」沆笑曰:「他日後悔,當思吾言也。」準後為謂所傾,始伏沆言。



 沆為相,接賓客,常寡言。馬亮與沆同年生,又與其弟維善,語維曰:「外議以大兄為無口匏。」維乘間達亮語,沆曰:「吾非不知也。然今之朝士得升殿言事,上封論奏,了無壅蔽,多下有司,皆見之矣。若邦國大事,北有契丹,西有夏人,日旰條
 議所以備御之策,非不詳究。薦紳如李宗諤、趙安仁,皆時之英秀,與之談,猶不能啟發吾意。自餘通籍之子,坐起拜揖,尚周章失次,即席必自論功最,以希寵獎,此有何策而與之接語哉?茍屈意妄言,即世所謂籠罩。籠罩之事,僕病未能也。」沆又嘗言:「居重位實無補,惟中外所陳利害,一切報罷之,此少以報國爾。朝廷防制,纖悉備具,或徇所陳請,施行一事,即所傷多矣,陸象先曰『庸人擾之』是已。憸人茍一時之進,豈念厲民耶?」沆為相,常讀《
 論語》。或問之,沆曰:「沆為宰相,如《論語》中『節用而愛人,使民以時』,尚未能行。聖人之言,終身誦之可也。」



 景德元年七月,沆待漏將朝,疾作而歸,詔太醫診視,撫問之使相望於道。明日,駕往臨問,賜白金五千兩。方還宮而沆薨,年五十八。上聞之驚嘆,趣駕再往,臨哭之慟,謂左右曰:「沆為大臣,忠良純厚,始終如一,豈意不享遐壽!」言終又泣下。廢朝五日,贈太尉、中書令,謚文靖。錄其弟國子博士贄為虞部員外郎,光祿寺丞源為太子中舍,屯田員
 外郎、直集賢院維為戶部員外郎。子宗簡為大理評事。甥蘇昂、妻兄之子朱濤並同進士出身。乾興元年,仁宗即位,詔配享真宗廟庭。



 沆性直諒,內行修謹,言無枝葉,識大體。居位慎密,不求聲譽,動遵條制,人莫能幹以私。公退,終日危坐,未嘗跛倚。治第封丘門內,廳事前僅容旋馬。或言其太隘,沆笑曰:「居第當傳子孫,此為宰相廳事誠隘,為太祝、奉禮廳事已寬矣。」至於垣頹壁損,不以屑慮。堂前藥闌壞,妻戒守舍者勿葺以試沆,沆朝夕見
 之,經月終不言。妻以語沆,沆曰:「豈可以此動吾一念哉!」家人勸治居第,未嘗答。弟維因語次及之,沆曰:「身食厚祿,時有橫賜,計囊裝亦可以治第,但念內典以此世界為缺陷,安得圓滿如意,自求稱足?今市新宅,須一年繕完,人生朝暮不可保,又豈能久居?巢林一枝,聊自足耳,安事豐屋哉?」



 沆與諸弟友愛,尤器重維,暇日相對宴飲清言,未嘗及朝政,亦未嘗問家事。沆沒後,或薦梅詢可用,真宗曰:「李沆嘗言其非君子。」其為信倚如此。



 維字仲方,第進士,為保信軍節度推官。真宗初,獻《聖德詩》,召試中書,擢直集賢院,以沆相,避知歙州。至郡,興學舍,歲時行鄉射之禮。沆沒,入為戶部員外郎。



 契丹請和,以為賀正旦使。真宗方幸西京,維還詣行在,具言其待遇禮厚,必保盟好。擢兵部員外郎、知制誥。自是每北使至,多命維主之。擢為翰林學士,累遷中書舍人,以疾辭,出知許州。復入翰林為學士承旨,加史館修撰。仁宗初,再遷為尚書左丞兼侍讀學士,預修《真宗實錄》,遷工部
 尚書。會塞下傳契丹將絕盟,復遣維往使。其主隆緒重維名,館勞加禮,使賦《兩朝悠久詩》。詩成,大喜。既還,帝欲用為樞密副使,或斥維賦詩自稱小臣,乃寢。遷刑部尚書,辭不拜,引李士衡故事求換官,除州觀察使,為諫官劉隨所詆,知亳州。請赴本鎮,改河陽。久之還朝,復出知陳州,卒。



 維博學,少以文章知名,至老手不廢書。景德以後,巡幸四方,典章名物,多維所參定。嘗預定《七經正義》,修《續通典》、《冊府元龜》。性寬易,喜慍不見於色,獎借後
 進,嗜酒善謔,而好為詩。常曰:「人生觴詠自適,餘何營哉?」既沒,家無餘貲。景祐元年,贈尚書右僕射。子師錫,虞部員外郎;公謹,太子中舍。



 王旦,字子明,大名莘人。曾祖言,黎陽令。祖徹,左拾遺。父祐,尚書兵部侍郎,以文章顯於漢、周之際,事太祖、太宗為名臣。嘗諭杜重威使無反漢,拒盧多遜害趙普之謀,以百口明符彥卿無罪,世多稱其陰德。祐手植三槐於庭,曰:「吾之後世,必有為三公者,此其所以志也。」



 旦幼沉
 默,好學有文,祐器之曰:「此兒當至公相。」太平興國五年,進士及第,為大理評事、知平江縣。其廨舊傳有物怪憑戾,居多不寧。旦將至前夕,守吏聞群鬼嘯呼云:「相君至矣,當避去。」自是遂絕。就改將作監丞。趙昌言為轉運使,以威望自任,屬吏屏畏,入旦境,稱其善政,以女妻之。代還,命監潭州銀場。何承矩典郡,薦入為著作佐郎,預編《文苑英華》、《詩類》。遷殿中丞、通判鄭州。表請天下建常平倉,以塞兼並之路。徙濠州。淳化初,王禹偁薦其才任轉
 運使,驛召至京,旦不樂吏職,獻文。召試,命直史館。二年,拜右正言、知制誥。



 初,祐以宿名久掌書命,旦不十年繼其任,時論美之。錢若水有人倫鑒,見旦曰:「真宰相器也。」與之同列,每曰:「王君凌霄聳壑,棟梁之材,貴不可涯,非吾所及。」李沆以同年生,亦推重為遠大之器。明年,與蘇易簡同知貢舉,加虞部員外郎、同判吏部流內銓、知考課院。趙昌言參機務,旦避嫌,引唐獨孤鬱、權德輿故事辭職。太宗嘉其識體,改禮部郎中、集賢殿修撰。昌言出
 知鳳翔,即日以旦知制誥,仍兼修撰、判院事,面賜金紫,擇牯犀帶寵之,又令冠西閣。至道元年,知理檢院。二年,進兵部郎中。



 真宗即位,拜中書舍人,數月,為翰林學士兼知審官院、通進銀臺封駁司。帝素賢旦,嘗奏事退,目送之曰:「為朕致太平者,必斯人也。」錢若水罷樞務,得對苑中,訪近臣之可用者,若水言:「旦有德望,堪任大事。」帝曰:「此固朕心所屬也。」咸平三年,又知貢舉,鎖宿旬日,拜給事中、同知樞密院事。逾年,以工部侍郎參知政事。



 契
 丹犯邊,從幸澶州。雍王元份留守東京,遇暴疾,命旦馳還,權留守事。旦曰:「願宣寇準,臣有所陳。」準至,旦奏曰:「十日之間未有捷報,時當如何?」帝默然良久,曰:「立皇太子。」旦既至京,直入禁中,下令甚嚴,使人不得傳播。及駕還,旦子弟及家人皆迎於郊,忽聞後有騶訶聲,驚視之,乃旦也。二年,加尚書左丞。三年,拜工部尚書、同中書門下平章事、集賢殿大學士、監修《兩朝國史》。



 契丹既受盟,寇準以為功,有自得之色,真宗亦自得也。王欽若忌準,欲
 傾之,從容言曰:「此《春秋》城下之盟也,諸侯猶恥之,而陛下以為功,臣竊不取。」帝愀然曰:「為之奈何?」欽若度帝厭兵,即謬曰:「陛下以兵取幽燕,乃可滌恥。」帝曰:「河朔生靈始免兵革,朕安能為此?可思其次。」欽若曰:「唯有封禪泰山,可以鎮服四海,誇示外國。然自古封禪,當得天瑞希世絕倫之事,然後可爾。」既而又曰:「天瑞安可必得?前代蓋有以人力為之者,惟人主深信而崇之,以明示天下,則與天瑞無異也。」帝思久之,乃可,而心憚旦,曰:「王旦得
 無不可乎?」欽若曰:「臣得以聖意喻之,宜無不可。」乘間為旦言,旦黽勉而從。帝猶猶豫,莫與籌之者。會幸秘閣,驟問杜鎬曰:「古所謂河出圖、洛出書,果何事耶?」鎬老儒,不測其旨,漫應之曰:「此聖人以神道設教爾。」帝由此意決,遂召旦飲,歡甚,賜以尊酒,曰:「此酒極佳,歸與妻孥共之。」既歸發之,皆珠也。由是凡天書、封禪等事,旦不復異議。



 大中祥符初,為天書儀仗使,從封泰山,為大禮使,進中書侍郎兼刑部尚書。受詔撰《封祀壇頌》,加兵部尚書。四
 年,祀汾陰,又為大禮使,遷右僕射、昭文館大學士。仍撰《祠壇頌》,將復進秩,懇辭得免,止加功臣。俄兼門下侍郎、玉清昭應宮使。五年,為玉清奉聖像大禮使。景靈宮建,又為朝修使。七年,刻天書,兼刻玉使,選御廄三馬賜之。玉清昭應宮成,拜司空。京師賜酺,旦以慘恤不赴會,帝賜詩導意焉。《國史》成,遷司空。旦為天書使,每有大禮,輒奉天書以行,恆邑邑不樂。凡柄用十八年,為相僅一紀。



 會契丹修和,西夏誓守故地,二邊兵罷不用,真宗以無
 事治天下。旦謂祖宗之法具在,務行故事,慎所變改。帝久益信之,言無不聽,凡大臣有所請,必曰:「王旦以為如何?」旦與人寡言笑,默坐終日,及奏事,群臣異同,旦徐一言以定。歸家,或不去冠帶,入靜室獨坐,家人莫敢見之。旦弟以問趙安仁,安仁曰:「方議事,公不欲行而未決,此必憂朝廷矣。」



 帝嘗示二府《喜雨詩》,旦袖歸曰:「上詩有一字誤寫,莫進入改卻否?」王欽若曰:「此亦無害。」而密奏之。帝慍,謂旦曰:「昨日詩有誤字,何不來奏?」旦曰:「臣得詩未
 暇再閱,有失上陳。」惶懼再拜謝,諸臣皆拜,獨樞密馬知節不拜,具以實奏,且曰:「王旦略不辨,真宰相器也。」帝顧旦而笑焉。天下大蝗,使人於野得死蝗,帝以示大臣。明日,執政遂袖死蝗進曰:「蝗實死矣,請示於朝,率百官賀。」旦獨不可。後數日,方奏事,飛蝗蔽天,帝顧旦曰:「使百官方賀,而蝗如此,豈不為天下笑耶?」



 宮禁火災,旦馳入。帝曰:「兩朝所積,朕不妄費,一朝殆盡,誠可惜也。」旦對曰:「陛下富有天下,財帛不足憂,所慮者政令賞罰之不當。臣
 備位宰府,天災如此,臣當罷免。」繼上表待罪,帝乃降詔罪己,許中外封事言得失。後有言榮王宮火所延,非天災,請置獄劾,當坐死者百餘人。旦獨請曰:「始火時,陛下已罪己詔天下,臣等皆上章待罪。今反歸咎於人,何以示信?且火雖有跡,寧知非天譴耶?」當坐者皆免。



 日者上書言宮禁事,坐誅。籍其家,得朝士所與往還占問吉兇之說。帝怒,欲付御史問狀。旦曰:「此人之常情,且語不及朝廷,不足罪。」真宗怒不解,旦因自取嘗所占問之書進
 曰:「臣少賤時,不免為此。必以為罪,願並臣付獄。」真宗曰:「此事已發,何可免?」旦曰:「臣為宰相執國法,豈可自為之,幸於不發而以罪人。」帝意解。旦至中書,悉焚所得書。既而復悔,馳取之,而已焚之矣。由是皆免。仁宗為皇太子,太子諭德見旦,稱太子學書有法。旦曰:「諭德之職,止於是耶?」張士遜又稱太子書,旦曰:「太子不在應舉,選學士不在學書。



 契丹奏請歲給外別假錢幣。旦曰:「東封甚近,車駕將出,彼以此探朝廷之意耳。」帝曰:「何以答之?」旦曰:「
 止當以微物而輕之。」乃以歲給三十萬物內各借三萬,仍諭次年額內除之。契丹得之,大慚。次年,復下有司:「契丹所借金幣六萬,事屬微末,今仍依常數與之,後不為比。」西夏趙德明言民饑,求糧百萬斛。大臣皆曰:「德明新納誓而敢違,請以詔責之。」帝以問旦,旦請敕有司具粟百萬於京師,而詔德明來取之。德明得詔,慚且拜曰:「朝廷有人。」



 寇準數短旦,旦專稱準。帝謂旦曰:「卿雖稱其美,彼專談卿惡。」旦曰:「理固當然。臣在相位久,政事闕失必
 多。準對陛下無所隱,益見其忠直,此臣所以重準也。」帝以是愈賢旦。中書有事送密院,違詔格,準在密院,以事上聞。旦被責,第拜謝,堂吏皆見罰。不逾月,密院有事送中書,亦違詔格,堂吏欣然呈旦,旦令送還密院。準大慚,見旦曰:「同年,甚得許大度量?」旦不答。寇準罷樞密使,托人私求為使相,旦驚曰:「將相之任,豈可求耶!吾不受私請。」準深憾之。已而除準武勝軍節度使、同中書門下平章事。準入見,謝曰:「非陛下知臣,安能至此?」帝具道旦所
 以薦者。準愧嘆,以為不可及。準在藩鎮,生辰,造山棚大宴,又服用僭侈,為人所奏。帝怒,謂旦曰:「寇準每事欲效朕,可乎?」旦徐對曰:「準誠賢能,無如騃何。」真宗意遂解,曰:「然,此正是騃爾。」遂不問。



 翰林學士陳彭年呈政府科場條目,旦投之地曰:「內翰得官幾日,乃欲隔截天下進士耶?」彭年皇恐而退。時向敏中同在中書,出彭年所留文字,旦瞑目取紙封之。敏中請一覽,旦曰:「不過興建符瑞圖進爾。」後彭年與王曾、張知白參預政事,同謂旦曰:「每
 奏事,其間有不經上覽者,公批旨奉行,恐人言之以為不可。」旦遜謝而已。一日奏對,旦退,曾等稍留,帝驚曰:「有何事不與王旦來?」皆以前事對。帝曰:「旦在朕左右多年,朕察之無毫發私。自東封後,朕諭以小事一面奉行,卿等謹奉之。」曾等退而愧謝,旦曰:「正賴諸公規益。」略不介意。



 帝欲相王欽若,旦曰:「欽若遭逢陛下,恩禮已隆,且乞留之樞密,兩府亦均。臣見祖宗朝未嘗有南人當國者,雖古稱立賢無方,然須賢士乃可。臣為宰相,不敢沮抑
 人,此亦公議也。」真宗遂止。旦沒後,欽若始大用,語人曰:「為王公遲我十年作宰相。」欽若與陳堯叟、馬知節同在樞府,因奏事忿爭。真宗召旦至,欽若猶嘩不已,知節流涕曰:「願與欽若同下御史府。」旦叱欽若使退。帝大怒,命付獄。旦從容曰:「欽若等恃陛下厚顧,上煩譴訶,當行朝典。願且還內,來日取旨。」明日,召旦前問之,旦曰:「欽若等當黜,未知坐以何罪?」帝曰:「坐忿爭無禮。」旦曰:「陛下奄有天下,使大臣坐忿爭無禮之罪,或聞外國,恐無以威遠。」
 帝曰:「卿意如何?」旦曰:「願至中書,召欽若等宣示陛下含容之意,且戒約之。俟少間,罷之未晚也。」帝曰:「非卿之言,朕固難忍。」後月餘,欽若等皆罷。



 旦嘗與楊億評品人物,億曰:「丁謂久遠當何如?」旦曰:「才則才矣,語道則未。他日在上位,使有德者助之,庶得終吉;若獨當權,必為身累爾。」後謂果如言。



 旦為兗州景靈宮朝修使,內臣周懷政偕行,或乘間請見,旦必俟從者盡至,冠帶出見於堂皇,白事而退。後懷政以事敗,方知旦遠慮。內臣劉承規以
 忠謹得幸,病且死,求為節度使。帝語旦曰:「承規待此以瞑目。」旦執不可,曰:「他日將有求為樞密使者,奈何?」遂止。自是內臣官不過留後。



 旦為相,賓客滿堂,無敢以私請。察可與言及素知名者,數月後,召與語,詢訪四方利病,或使疏其言而獻之。觀才之所長,密籍其名,其人復來,不見也。每有差除,先密疏四三人姓名以請,所用者帝以筆點之。同列不知,爭有所用,惟旦所用,奏入無不可。丁謂以是數毀旦,帝益厚之。故參政李穆子行簡,以將
 作監丞家居,有賢行,遷太子中允。使者不知其宅,真宗命就中書問旦,人始知行簡為旦所薦。旦凡所薦,皆人未嘗知。旦沒後,史官修《真宗實錄》,得內出奏章,始知朝士多旦所薦雲。諫議大夫張師德兩詣旦門,不得見,意為人所毀,以告向敏中,為從容明之。及議知制誥,旦曰:「可惜張師德。」敏中問之,旦曰:「累於上前言師德名家子,有士行,不意兩及吾門。狀元及第,榮進素定,但當靜以待之爾。若復奔競,使無階而入者當如何也。」敏中啟以
 師德之意,旦曰:「旦處安得有人敢輕毀人,但師德後進,待我薄爾。」敏中固稱:「適有闕,望公弗遺。」旦曰:「第緩之,使師德知,聊以戒貪進、激薄俗也。」



 石普知許州不法,朝議欲就劾。旦曰:「普武人,不明典憲,恐恃薄效,妄有生事。必須重行,乞召歸置獄。」乃下御史按之,一日而獄具。議者以為不屈國法而保全武臣,真國體也。薛奎為江、淮發運使,辭旦,旦無他語,但云:「東南民力竭矣。」奎退而曰:「真宰相之言也。」張士遜為江西轉運使,辭旦求教,旦曰:「朝
 廷榷利至矣。」士遜迭更是職,思旦之言,未嘗求利,識者曰:「此運使識大體。」張詠知成都,召還,以任中正代之,言者以為不可。帝問旦,對曰:「非中正不能守詠之規。他人往,妄有變更矣。」李迪、賀邊有時名,舉進士,迪以賦落韻,邊以《當仁不讓於師論》以「師」為「眾」,與注疏異,皆不預。主文奏乞收試,旦曰:「迪雖犯不考,然出於不意,其過可略。邊特立異說,將令後生務為穿鑿,漸不可長。」遂收迪而黜邊。



 旦任事久,人有謗之者,輒引咎不辨。至人有過失,
 雖人主盛怒,可辨者辨之,必得而後已。素羸多疾,自東魯復命,連歲求解,優詔褒答,繼以面諭,委任無貳。天禧初,進位太保,為兗州太極觀奉上寶冊使,復加太尉兼侍中,五日一赴起居,入中書,遇軍國重事,不限時日入預參決。旦愈畏避,上疏懇辭,又托同列奏白。帝重違其意,止加封邑。一日,獨對滋福殿,帝曰:「朕方以大事托卿,而卿疾如此。」因命皇太子出拜,旦皇恐走避,太子隨而拜之。旦言:「太子盛德,必任陛下事。」因薦可為大臣者十
 餘人,其後不至宰相惟李及、凌策二人,亦為名臣。旦復求避位,帝睹其形瘁,憫然許之。以太尉領玉清昭應宮使,給宰相半奉。



 初,旦以宰相兼使,今罷相,使猶領之,其專置使自旦始焉。尋又命肩輿入禁,使子雍與直省吏挾扶,見於延和殿。帝曰:「卿今疾亟,萬一有不諱,使朕以天下事付之誰乎?」旦曰:「知臣莫若君,惟明主擇之。」再三問,不對。時張詠、馬亮皆為尚書,帝歷問二人,亦不對。因曰:「試以卿意言之。」旦強起舉笏曰:「以臣之愚,莫如寇準。」
 帝曰:「準性剛褊,卿更思其次。」旦曰:「他人,臣所不知也。臣病困,不能久侍。」遂辭退。後旦沒歲餘,竟用準為相。



 旦疾甚,遣內侍問者日或三四,帝手自和藥,並薯蕷粥賜之。旦與楊億素厚,延至臥內,請撰遺表。且言:「忝為宰輔,不可以將盡之言,為宗親求官,止敘生平遭遇,願日親庶政,進用賢士,少減焦勞之意。」仍戒子弟:「我家盛名清德,當務儉素,保守門風,不得事於泰侈,勿為厚葬以金寶置柩中。」表上,真宗嘆之,遂幸其第,賜白金五千兩。旦作
 奏辭之,蒿末,自益四句云:「益懼多藏,況無所用,見欲散施,以息咎殃。」即舁至內闥,詔不許。還至門,旦已薨,年六十一。帝臨其喪慟,廢朝三日,贈太師、尚書令、魏國公,謚文正,又別次發哀。後數日,張旻赴鎮河陽,例宜飲餞,以旦故,不舉樂。錄其子、弟、侄、外孫、門客、常從,授官者十數人。諸子服除,又各進一官。已而聞旦奏蒿自益四句,取視,泣下久之。旦有文集二十卷。乾興初,詔配享真宗廟廷。及建碑,仁宗篆其首曰:「全德元老之碑。」



 旦事寡嫂有
 禮,與弟旭友愛甚篤。婚姻不求門閥。被服質素,家人欲以繒錦飾氈席,不許。有貨玉帶者,弟以為佳,呈旦,旦命系之,曰:「還見佳否?」弟曰:「系之安得自見?」旦曰:「自負重而使觀者稱好,無乃勞乎!」亟還之。故所服止於賜帶。家人未嘗見其怒,飲食不精潔,但不食而已。嘗試以少埃墨投羹中,旦惟啖飯,問何不啜羹,則曰:「我偶不喜肉。」後又墨其飯,則曰:「吾今日不喜飯,可別具粥。」旦不置田宅,曰:「子孫當各念自立,何必田宅,徒使爭財為不義爾。」真宗
 以其所居陋,欲治之,旦辭以先人舊廬,乃止。宅門壞,主者徹新之,暫於廡下啟側門出入。旦至側門,據鞍俯過,門成復由之,皆不問焉。三子:雍,國子博士;沖,左贊善大夫;素,別有傳。



 向敏中,字常之,開封人。父瑀,仕漢符離令。性嚴毅,惟敏中一子,躬自教督,不假顏色。嘗謂其母曰:「大吾門者,此兒也。」敏中隨瑀赴調京師,有書生過門,見敏中,謂鄰母曰:「此兒風骨秀異,貴且壽。」鄰母入告其家,比出,已不見
 矣。及冠,繼丁內外憂,能刻厲自立,有大志,不屑貧窶。



 太平興國五年進士,解褐將作監丞、通判吉州,就改右贊善大夫。轉運使張齊賢薦其材,代還,為著作郎。召見便殿,占對明暢,太宗善之,命為戶部推官,出為淮南轉運副使。時領外計者,皆以權寵自尊,所至畏憚,敏中不尚威察,待僚屬有禮,勤於勸勖,職務修舉。或薦其有武幹者,召入,將授諸司副使。敏中懇辭,仍獻所著文,加直史館,遣還任。以耕籍恩,超左司諫,入為戶部判官、知制誥。
 未幾,權判大理寺。



 時沒入祖吉贓錢,分賜法吏,敏中引鐘離意委珠事,獨不受。妖尼道安構獄,事連開封判官張去華,敏中妻父也,以故得請不預決讞。既而法官皆貶,猶以親累落職,出知廣州。入辭,面敘其事,太宗為之感動,許以不三歲召還。翌日,遷職方員外郎,遣之。是州兼掌市舶,前守多涉譏議。敏中至荊南,預市藥物以往,在任無所須,以清廉聞。就擢廣南東路轉運使,召為工部郎中。太宗飛白書敏中洎張詠二名付中書,曰:「此二
 人,名臣也,朕將用之。」左右因稱其材,並命為樞密直學士。



 時通進、銀臺司主出納書奏,領於樞密院,頗多壅遏,或至漏失。敏中具奏其事,恐遠方有失事機,請別置局,命官專蒞,校其簿籍,詔命敏中與詠領其局。太宗欲大任敏中,當途者忌之。會有言敏中在法寺時,皇甫侃監無為軍榷務,以賄敗,發書歷詣朝貴求為末減,敏中亦受之。事下御史,按實,嘗有書及門,敏中睹其名,不啟封遣去。俄捕得侃私僮詰之,云其書尋納筒中,瘞臨江傳
 舍。馳驛掘得,封題如故。太宗大驚異,召見,慰諭賞激,遂決於登用。未幾,拜右諫議大夫、同知樞密院事。自郎中至是百餘日,超擢如此。時西北用兵,樞機之任,專主謀議,敏中明辨有才略,遇事敏速,凡二邊道路、斥堠、走集之所,莫不周知。至道初,遷給事中。



 真宗即位,敏中適在疾告,力起,見於東序,即遣視事。進戶部侍郎。會曹彬為樞密使,改為副使。咸平初,拜兵部侍郎、參知政事。從幸大名,屬宋湜病,代兼知樞密院事。時大兵之後,議遣重
 臣慰撫邊郡,命為河北、河東安撫大使,以陳堯叟、馮拯為副,發禁兵萬人翼從。所至訪民疾苦,宴犒官吏,莫不感悅。四年,以本官同平章事,充集賢殿大學士。



 故相薛居正孫安上不肖,其居第有詔無得貿易,敏中違詔質之。會居正子惟吉嫠婦柴將攜貲產適張齊賢,安上訴其事,柴遂言敏中嘗求娶己,不許,以是陰庇安上。真宗以問敏中,敏中言近喪妻不復議婚,未嘗求婚於柴,真宗因不復問。柴又伐鼓,訟益急,遂下御史臺,並得敏中
 質宅之狀。時王嗣宗為鹽鐵使,素忌敏中,因對言,敏中議娶王承衍女弟,密約已定而未納採。真宗詢於王氏,得其實,以敏中前言為妄,罷為戶部侍郎,出知永興軍。



 景德初,復兵部侍郎。夏州李繼遷兵敗,為潘羅支射傷,自度孤危且死,屬其子德明必歸宋,曰:「一表不聽則再請,雖累百表,不得,請勿止也。」繼遷卒,德明納款,就命敏中為鄜延路緣邊安撫使,俄還京兆。



 是冬,真宗幸澶淵,賜敏中密詔,盡付西鄙,許便宜從事。敏中得詔藏之,視
 政如常日。會大儺,有告禁卒欲倚儺為亂者,敏中密使麾兵被甲伏廡下幕中。明日,盡召賓僚兵官,置酒縱閱,無一人預知者。命儺入,先馳騁於中門外,後召至階,敏中振袂一揮,伏出,盡擒之,果各懷短刃,即席斬焉。既屏其尸,以灰沙掃庭,張樂宴飲,坐客皆股慄,邊藩遂安。時舊相出鎮,不以軍事為意。寇準雖有重名,所至終日游宴,則以所愛伶人或付富室,輒厚有得。張齊賢倜儻任情,獲劫盜或至縱遣。帝聞之,稱敏中曰:「大臣出臨四方,
 惟敏中盡心於民事爾。」於是有復用之意。二年,又以德明誓約未定,徙敏中為鄜延路都部署兼知延州,委以經略,改知河南府兼西京留守。



 大中祥符初,議封泰山,以敏中舊德有人望,召入,權東京留守。禮成,拜尚書右丞。



 時吏部選人多稽滯者,命敏中與溫仲舒領其事。俄兼秘書監,又領工部尚書,充資政殿大學士,賜御詩褒寵。祀汾陰,復為留守。敏中以厚重鎮靜,人情帖然,帝作詩遣使馳賜之。拜刑部尚書。五年,復拜同平章事,充集
 賢殿大學士,加中書侍郎。尋充景靈宮使,宮成,進兵部尚書,為兗州景靈宮慶成使。



 天禧初,加吏部尚書,又為應天院奉安太祖聖容禮儀使。進右僕射兼門下侍郎,監修國史。是日,翰林學士李宗諤當對,帝曰:「朕自即位,未嘗除僕射,今命敏中,此殊命也,敏中應甚喜。」又曰:「敏中今日賀客必多,卿往觀之,勿言朕意也。」宗諤既至,敏中謝客,門闌寂然。宗諤與其親徑入,徐賀曰:「今日聞降麻,士大夫莫不歡慰相慶。」敏中但唯唯。又曰:「自上即位,
 未嘗除端揆,非勛德隆重,眷倚殊越,何以至此。」敏中復唯唯。又歷陳前世為僕射者勛德禮命之重,敏中亦唯唯,卒無一言。既退,使人問庖中,今日有親賓飲宴否,亦無一人。明日,具以所見對。帝曰:「向敏中大耐官職。」徙玉清昭應宮使。以年老,累請致政,優詔不許。三年重陽,宴苑中,暮歸中風眩,郊祀不任陪從。進左僕射、昭文館大學士,奉表懇讓,又表求解,皆不許。明年三月卒,年七十二。帝親臨,哭之慟,廢朝三日,贈太尉、中書令,謚文簡。五
 子、諸婿並遷官,親校又官數人。



 敏中姿表瑰碩,有儀矩,性端厚豈弟,多智,曉民政,善處繁劇,慎於採拔。居大任三十年,時以重德目之,為人主所優禮,故雖衰疾,終不得謝。及追命制入,帝特批曰:「敏中淳謹溫良,宜益此意。」其恩顧如此。有文集十五卷。



 子傳正,國子博士;傳式,龍圖閣直學士;傳亮,駕部員外郎;傳師,殿中丞;傳範,娶南陽郡王惟吉女安福縣主,為密州觀察使,謚惠節。



 傳亮子經,定國軍留後,謚康懿。經女即欽聖憲肅皇后也,以
 後族贈敏中燕王、傳亮周王、經吳王。敏中餘孫繹、絳,並官太子中書。



 論曰:宋至真宗之世,號為盛治,而得人亦多。李沆為相,正大光明,其焚封妃之詔以格人主之私,請遷靈州之民以奪西夏之謀,無愧宰相之任矣。沆嘗謂王旦,邊患既息,人主侈心必生,而聲色、土木、神仙祠禱之事將作,後王欽若、丁謂之徒果售其佞。又告真宗不可用新進喜事之人,中外所陳利害皆報罷之,後神宗信用安石
 變更之言,馴至棼擾。世稱沆為「聖相」,其言雖過,誠有先知者乎!王旦當國最久,事至不膠,有謗不校,薦賢而不市恩,救罪輒宥而不費辭。澶淵之役,請於真宗曰:「十日不捷,何以處之?」真宗答之曰:「立太子。」契丹逾歲給而借幣,西夏告民饑而假糧,皆一語定之,偉哉宰相才也。惟受王欽若之說,以遂天書之妄,斯則不及李沆爾。向敏中恥受贓物之賜以遠其污,預避市舶之嫌以全其廉,堅拒皇甫侃之書以免其累,拜罷之際,喜慍不形,亦可
 謂有宰相之風焉



\end{pinyinscope}