\article{列傳第一百八十四}

\begin{pinyinscope}

 ○劉
 應
 龍潘牥洪芹趙景緯馮去非徐霖徐宗仁危昭德陳塏楊文仲謝枋得



 劉應龍,字漢臣,瑞州高安人。嘉熙二年進士。授零陵主
 簿,饒州錄事參軍。有毛隆者,務剽掠殺人,州民被盜,遙呼盜曰:「汝毛隆也?」盜亦曰:「我毛隆也。」既,訟於官,捕隆置獄,應龍曰:「盜誠毛隆,其肯自謂?」因言於州,州不可,乃委它官,隆誣伏抵死,未幾盜敗,應龍繇是著名。改知崇仁縣。淮西失守,江西諸州有殘破者,縣佐貳聞變先遁,應龍固守不去。



 先是,理宗久未有子,以弟福王與芮之子為皇子,丞相吳潛有異論,帝已不樂。大元兵度江,朝野震動,逐丞相丁大全,復起潛為相,帝問潛策安出,潛對
 曰:「當遷幸。」又問卿如何,潛曰:「臣當死守於此。」帝泣下曰:「卿欲為張邦昌乎?」潛不敢復言。未幾北兵退,帝語群臣曰:「吳潛幾誤朕。」遂罷潛相。帝怒潛不已,應龍朝受命,帝夜出象簡書疏稿授應龍,使劾潛,應龍謂:「潛本有賢譽,獨論事失當,臨變寡斷。祖宗以來,大臣有罪未嘗輕肆誅戮。欲望姑從寬典,以全體貌。」帝大怒。乃按劾丁大全,請加竄斥,疏言:「內莫急於蘇民瘼以固國本,外莫急於討軍實以振國威。」又言時政四事,廣發稟以振民饑,通
 商販以助民食,勸分富室以助官糴,嚴等第以核民數,稽檢放以蘇民窮,嚴戢盜以除民害。賈似道素忌潛,會京師米貴,應龍為《勸糶歌》宦者取以上聞,帝問知應龍所作,問似道米價高,當亟處之,似道訪其由,亦怒應龍。遷司農少卿,尋以右諫議大夫孫附鳳言,遂去國。



 景定三年,湖南饑,起提舉常平。以救荒功,遷直寶章閣、廣南東路轉運判官。遷秘書監兼國史編修、實錄檢討。知隆興府兼江西轉運副使,奏免和糴二十萬石。擢權戶部
 侍郎兼侍講。時似道當國,百官奏對稍切直者輒黜,應龍言:「臣觀今日之事,可言者多矣。邇日以來,靖恭自守者以論事為忌,指陳稍切者聯翩引去,豈兩省繳駁過甚,重其疑歟?抑廷臣奏對咈意,速其畏歟?朝廷清明之時,而言者已懷疑畏,臣恐正臣奪氣,鯁臣吃舌,宜非盛世所有。」遂迕當路,自侍從、兩省以下無不切齒。未幾,以集英殿修撰知建寧府,亟辭,中書舍人盧鉞希指封還錄黃。久之,起為江東轉運使,辭。



 南海寇作,朝廷患之,乃
 以顯謨閣待制知廣州、廣東紗略安撫使。寇聞應龍至,遁去。應龍剿逐之,南海大治。特旨屢召,拜戶部侍郎仍兼侍讀,七上奏辭免。德祐元年,遷兵部尚書、寶章閣直學士、知贛州,兼江西兵馬鈐轄、青海軍節度使,力辭,隱九峰。



 子元高亦舉進士,知候官縣。沒,洪天錫嘆曰:「朝廷失一御史矣。」



 潘牥字庭堅,福州閩人。端平二年策進士,牥對曰:「陛下承休上帝,皈德匹夫,何異為人子孫,身荷父母劬勞之
 賜,乃指豪奴悍婢為恩私之地。欲父母無怒,不可得也。」又曰:「陛下手足之愛,生榮死哀,反不得視士庶人。此如一門之內,骨肉之間未能親睦,是以僮僕疾視,鄰里生侮。宜厚東海之恩,裂淮南之土,以致人和。」時對者數百人,庭堅語最直。



 會殿中侍御史蔣峴劾方大琮、劉克莊、王邁前倡異論,並誣牥姓同逆賊,策語不順,請皆論以漢法。牥調鎮南軍節度推官、衢州推官,歷浙西提舉常平司。遷太學正,旬日,出通判潭州。日食,應詔上封事曰:「
 熙寧初元日食,詔郡縣掩骼,著為令。故王一手不淺土,其為暴骸亦大矣。請以王禮改葬。」又移書丞相游似申言之,似心善其言,方將收用之,而牥卒。



 洪芹,尚書右僕射適之曾孫,以大父澤入官,甫更調,登進士第。自南平司法改欽州教授。部使者愛其才,先後並薦之,有旨召審察。丁內外艱。入主省架閣,遷太學博士。輪對,發明絜矩之道。擢國子博士,出通判南劍,入為太常博士,累遷將作少監。屬詞臣無當上意,慨然思得
 天下士,丞相程元鳳言當今地望無逾洪芹者,進兼翰林,權直秘書少監。



 開慶元年,升直學士院,繼權禮部侍郎、中書舍人。屬兵興,帝悟柄任非人,自貽國禍,詔書所至,聞者奮激,蓋芹所草也。丁大全罷相,出典鄉郡。芹遷禮部侍郎,繳奏:「大全鬼蜮之資,穿窬之行,暴戾淫黷,引用兇惡,陷害忠良,遏塞言路,濁亂朝綱。乞盡從諫臣所請,追官遠竄,以伸國法,以謝天下。」沈炎乘上怒,攻丞相吳潛,芹獨繳奏曰:「方國本多虞,潛星馳赴闕,理紛鎮浮,
 陳力為多。一旦視為弁髦,得無如《詩》所謂『將安將樂女轉棄予』乎?」慷慨敢言,天下義之。



 遷禮部侍郎,帝銳意鄉用而以論去,退寓永嘉,怡然自適。咸淳初,起知寧國府。卒。有文集。



 趙景緯字德父,臨安府於潛人。少勤學,弱冠得周惇頤、程顥兄弟諸書讀之,恨不及登朱熹之門。熹門人葉味道謂之曰:「度正,吾黨中第一人。」遂往見,首誨以求放心為本。由是往來味道、正之間,研索益精。入太學,登淳祐
 元年進士第。授江陰軍教授,諸生守其矩度。丁母憂,以祿不逮養,服闋不調。作《讀易庵懸霤山》。江東提點刑獄吳勢卿闢為乾辦公事,不就。召為史館檢閱,辭,不許;乞換待次教授,不許;乞岳祠,又不許;乞致仕,不報。有旨特與改合入官,主管崇道觀,三辭,不許。景定元年,特授秘書郎,兩辭,不許。遷著作郎,辭,不許。以疾丐祠,差主管祐神觀兼史館校勘。史成,兩乞外祠,進直秘閣,與在外宮觀,辭職名,不許。差主管崇禧觀。



 臺州守王華甫建上蔡
 書院,禮景緯為堂長,以疾辭。依舊職差知臺州,兩辭,不許,趣命益嚴。至郡,以化民成俗為先務,首取陳述古《諭俗文》書示諸邑,且自為之說,使其民更相告諭、諷誦、服行,期無失墜。約束官吏擾民五事。取《孝經庶人章》為四言詠贊其義,使朝夕歌之,至有為之感涕者。舉遺逸車若水、林正心於朝。旌孝行,作《訓孝文》以勵其俗。平重刑,懲嘩訐,治豪橫。建黃巖縣社倉六十有六。浚河道九十里,築堤路三十里。節浮費,為下戶代輸秋苗。奏蠲五邑
 坊河渡錢。



 斯年之內,乞歸田里者再。進考功郎官,再辭,不許。兼沂靖惠王府教授,辭,不許。是冬,四辭新命,且乞祠,皆不許。乃乞於赤城、桐柏之間採藥著書,庶幾有補後學,使病廢之身不為無用於聖世,不許。御筆兼崇政殿說書,三辭,不許。乃造朝,侍緝熙殿,以《易》進講,論「聖人體元之妙在惟幾,人君得此,則天下有治而無亂,人事有吉而無兇矣」。又曰:「惕厲祗懼,乃天心之所存。聖人先處於憂,故能無憂,先處以危,故能無危;若乃先自處於
 安樂,則憂危乘之矣。」又論監司守令,其說曰:「知人之難,自古已然。人才乏使,莫今為甚。或觀望而撓於勢,或阿私而徇於情,或是非不公而以枉為直,或毀譽失實而以污為廉。遂使舉刺不當,不足以服天下之心。與其糾劾於有罪之後,而未必盡得其情;孰若精擇於未用之先,而使之各稱其職。」



 彗出於柳,景緯應詔上封事曰:



 今日求所以解天意者,不過悅人心而已。百姓之心即天心也。錮私藏而專天下之同欲,則人不悅。保私人而違
 天下之公議,則人不悅。閭閻之糟糠不厭,而燕私之供奉自如,則人不悅。百姓之膏血日朘,而符移之星火愈急,則人不悅。不公於己而欲絕天下之私,則人不悅。不澄其源而欲止天下之貪,則人不悅。夫必有是數者,斯足以召怨而致災。



 願陛下捐內帑以絕壅利之謗;出嬪嬙以節用度之奢。弄權之貂寺素為天下之所共惡者,屏之絕之;毒民之恩澤侯嘗為百姓之所憤者,黜之棄之。擇忠鯁敢言之士置之臺諫,以通關鬲之壅;選慈惠
 忠信之人使為守宰,以保元氣之殘。又必稽乾、淳以來,凡利源窠名之在百司庶府者,悉隸其舊,以濟經用之急;公田派買不均之敝,聽民自陳,隨宜通變,以安田里之生。則人心悅、天意解矣。人之常情,懼心每發於災異初見之時,不能不潛移於諂諛交至之後。萬一過聽左右寬譬之言,曲為它說以自解,毛舉細故以塞責,而恐懼之初心弛,則下拂人心,上違天意,國之安危或未可知。



 又曰:「損玉食,不若損內帑、卻貢奉之為實。避正朝,不
 若塞幸門、廣忠諫之為實。肆大眚固所以廣仁恩,又不若擇循良、黜貪暴之為實。蓋天意方回而未豫,人心乍悅而旋疑,此正陰陽勝復之會,眷命隆替之機也。」兼國史院編修官、實錄院檢討官,辭,不許。轉對,言:「願明辨義利之限,力破系吝之私,以天自處而絕內外之分,以道制欲而黜耳目之累。毋以閨闥之賤幹公議,毋以戚畹之私紊國常。」乞歸田里,不許。拜太府少卿,兼職仍舊,再辭,不許。復上疏乞歸,不許。



 以直敷文閣知嘉興府,辭,仍
 乞奉祠,皆不許。咸淳元年至郡,首以護根本、正風俗為先務。三乞辭,不許。拜宗正少卿,御筆兼侍講,辭,不許。乃還家,三乞祠,御筆趣行,猶乞寬告,不許。至國門,御筆兼權工部侍郎,時暫兼權中書舍人,三辭,不許。以《禮記》進講,開陳敬恕之義。封還濫恩詞頭,帝從之。又言:「損德害身之大莫過於嗜欲,而窒嗜欲之要莫切於思。居處則思敬,動作則思禮,祭祀則思誠,事親則思孝。每御一食,則思天下之饑者。每服一衣,則思天下之寒者。嬪嬙在
 列,必思夏桀以嬖色亡其國。飲燕方歡,必思商紂以沈湎喪其身。念起而思隨之,則念必息。欲萌而思制之,則欲必消。志氣日以剛健,德性日以充實,豈不盛哉。」



 又曰:「雷發非時,竊跡今日之事而有疑焉。內批疊降而名器輕,宮閫不嚴而主威褻,橫恩之濫已收而復出,戢貪之詔方嚴而隨弛。宮正什伍之令所以防奇邪,而或縱於乞憐之卑祠。緇黃出入之禁所以嚴宸居,而間惑於禬禳之小數。以至彈墨未幹,而抆拭之旨已下;駁奏未幾,
 而捷出之徑已開。命令不疑,則陽縱而不收。主意不堅,則陰閉而不密。陛下可不思致災之由,而亟求所以正之哉?願清其天君,以端出治之源;謹其號令,以肅紀綱之本。毋牽於私恩而撓公法,毋遷於邇言而亂舊章,去讒而遠色,賤貨而貴德,則人心悅而天意得,可以開太平而兆中興也。」



 進權禮部侍郎兼修玉牒,再辭,不許。升兼侍讀,辭,不許。進《聖學四箴》:一曰惜日力以致其勤,二曰精體認以充其知,三曰屏嗜好以專其業,四曰謹行
 事以驗其用。五乞歸田里,帝勉留之,請益力。特授集英殿修撰、知建寧府,辭,不許,乃還家。召為中書舍人,三辭,不許,請益力。進顯文閣待制,依所乞予祠,辭職名,不許,遂差提舉玉隆萬壽宮。有疾,謝醫卻藥,曰:「使我清心以順天命,毋重惱我懷。」拱手三揖乃卒。詔特贈四官至中奉大夫,謚文安。景緯天性孝友,雅志沖淡,親沒無意仕進,故其立朝之日不久云。



 馮去非字可遷,南康都昌人。父椅字儀之,家居授徒,所
 注《易》、《書》、《詩》、《語》、《孟》、《太極圖》,《西銘輯說》,《孝經章句》,《喪禮小學》,《孔子弟子傳》,《讀史記》及詩文、志錄,合二百餘卷。



 去非,淳祐元年進士。嘗乾辦淮東轉運司,治儀真,歐陽脩東園在焉,使者黃濤欲以為佛寺,時已許薦,去非力爭不得,寧不受使者薦,謁告而去。寶祐四年,召為宗學諭。丁大全為左諫議大夫,三學諸生叩閽言不可。帝為下詔禁戒,詔立石三學,去非獨不肯書名碑之下方。監察御史吳衍、翁應弼劾諸生下獄,去非復調護宗學生之就逮
 者。未幾,大全簽書樞密院事、參知政事,蔡抗去國,去非亦以言罷。歸舟泊金、焦山,有僧上謁,去非不虞其為大全之人也,周旋甚款。僧乘間致大全意,願毋遽歸,少俟收召,誠得尺書以往,成命即下。去非奮然正色曰:「程丞相、蔡參政牽率老夫至此,今歸吾廬山,不復仕矣,斯言何為至我!」絕之,不復與言。



 徐霖字景說,衢州西安人。年十三,有志聖人之道,取所作文焚之,研精《六經》之奧,控賾先儒心傳之要。淳祐四
 年,試禮部第一。知貢舉官入見,理宗曰:「第一名得人。」嘉獎再三。登第,授沅州教授。



 時宰相史嵩之挾邊功要君,植黨顓國。霖上疏歷言其奸深之狀,以為:「其先也奪陛下之心,其次奪士大夫之心,而其甚也奪豪傑之心。今日之士大夫,嵩之皆變化其心而收攝之矣。且其變化之術甚深,非章章然號於人使之為小人也。常於善類擇其質柔氣弱易以奪之者,親任一二,其或稍有異已,則潛棄而擯遠之,以風其餘。彼以名節之尊不足以易
 富貴之願,義利之辨亦終暗於妻妾宮室之私,則亦從之而已。」疏奏,見者吐舌,為霖危之。未幾,嵩之匿父喪求起復,君子並起而攻之,上大感悟。



 丞相範鐘進所召試館職二人,上思霖之忠,親去其一,易霖名。及試,則曰:「人主無自強之志,大臣有患失之心,故元良未建,兇奸未竄。」是時,丞相杜範已薨,而鐘雖得位,畏奸人覆出為己禍故也。擢秘書省正字,霖辭不獲命,遂就職。會日食,霖應詔上封事曰:「日,陽類也,天理也,君子也。吾心之天理
 不能勝乎人欲,朝廷之君子不能勝乎小人。宮闈之私暱未屏,瑣闥之奸邪未辨,臺臣之討賊不決,精祲感浹,日為之食。」又數言建立太子。遷校書郎。七年夏,大旱,霖應詔言:「諫議大夫不易則不雨,京兆尹不易則不雨。」不報,去國。上遣著作郎姚希得留之,不還。御筆改合入官,乃改宣教郎。霖屢辭,曰:「向為身死而不敢欺其君父,今以官高而自眩於平生,失其本心,何以暴其忠志?」又曰:「志貴乎潔,忠尚乎精,即有取,則自蹈於垢污矣。」



 八年夏,
 添差通判信州,霖皆力辭,竟未拜,改秩之命故也。尋令守臣勉諭之,特改宣教郎、主管雲臺觀,霖乃拜受。十二年,遷秘書省著作郎,累辭,不許。兼國史編修、實錄檢討,上曰:「今日所當言者,當備陳之。」霖復以正太子名為言,又奏:「萬化之本在心,存心之法在敬。」兼權尚左郎官,兼崇政殿說書。乃上疏言:「葉大有陰柔奸黠,為群憸冠,不宜久長臺諫,乞斥去。」不報。兼權左司。霖知無不言,於是讒嫉者思以中傷,而上亦不悅。乞補外,知撫州。祠先賢,
 寬租賦,振饑窮,誅悍將,建營砦,幾一月而政舉化行。以言去,士民遮道,不得行,及暝,始由徑以出。



 寶祐元年,差知衡州。三年,當之官,遂辭,差知袁州。五年,丁外艱,哀毀號絕,水漿不入口七日。明年開慶元年,差主管崇禧觀。景定二年,知汀州。明年,卒。將終,語其長子心亨曰:「有生必有死,自古聖賢皆然,吾復何憾。」尚書省請加優異,詔與一子恩澤。度宗賜祭田百畝,以旌直臣。霖間居衢,守游鈞築精舍,聘霖為學者講道,是日聽者三千餘人。



 徐宗仁字求心,信之永豐人。淳祐十年進士。歷官為國子監主簿。開慶元年,伏闕上書曰:



 賞罰者,軍國之綱紀。賞罰不明,則綱紀不立。今天下如器之欹而未墜於地,存亡之機,固不容發。兵虛將惰,而力匱財殫,環亮四境,類不足恃;而所恃以維持人心、奔走豪傑者,惟陛下賞罰之微權在耳。權在陛下,而陛下不知所以用之,則未墜者安保其終不墜乎?臣為此懼久矣。



 陛下當危急之時,出金幣賜土田,授節鉞,分爵秩,尺寸之功,在所必賞。
 故當悉心效力,圖報萬分可也。而自乾腹之兵越江逾廣以來,凡閱數月,尚未聞有死戰陣、死封疆、死城郭者,豈賞罰不足以勸懲之耶?今通國之所謂佚罰者,不過丁大全、袁玠、沈翥、張鎮、吳衍、翁應弼、石正則、王立愛、高鑄之徒,而首惡則董宋臣也。是以廷紳抗疏,學校叩閽,至有欲借尚方劍為陛下除惡。而陛下乃釋而不問,豈真欲愛護此數人而重咈千萬人之心?天下之事勢急矣,朝廷之紀綱壞矣。若誤國之罪不誅,則用兵之士不
 勇。今東南一隅天下,已半壞於此數人之手,而罰不損其豪毛。彼方擁厚貲,挾聲色,高臥華屋,而使陛下與二三大臣焦心勞思,可乎?三軍之在行者,豈不憤然不平曰:「稔禍者誰歟,而使我捐軀兵革之間?」百姓之罹難者,豈不群然胥怨曰:「召亂者誰歟,而使我流血鋒鏑之下?」陛下亦嘗一念及此乎?



 又極論邊事,謂惠褻而威不振。論董宋臣盤固日久,蒙蔽日久。又請「使有言責者皆得以盡其言,則國論伸而國威振,臣雖屏處山林,亦有生
 氣」。遷國子監丞、秘書省著作佐郎,主管崇禧觀。遷考功郎官兼崇政殿說書,進讀《敬天圖》。遷太府少卿兼侍講、兼侍立修注官,遷太常少卿兼國史編修、實錄檢討。知寧國府。監察御史郭閶論罷。



 德祐元年,起授吏部侍郎兼中書門下檢正諸房公事,兼提領豐儲倉所,兼同修國史、實錄院同修撰,侍左待郎。乞假督府名稱往本州同守臣防拓,不允。權禮部尚書兼益王府贊讀。衛益王走海上,厓山兵敗,死焉。



 危昭德,邵武人。寶祐元年進士。歷官為史館檢閱校勘、武學諭、宗正寺簿兼崇政殿說書,遷秘書郎。疏言:「國之命在民,民之命在士大夫。士大夫不廉,朘民膏血,為己甘腴,民不堪命矣。」又言:「願陛下與二三大臣察利害之實,究安危之本,明詔郡國,申嚴號令,俾急其所急,凡荒政之當舉者,不可一日而置念;緩其可緩,凡苛賦之肆擾者,易為此時之寬征。固結人心,乃所以延天命也。」又言:「願陛下舉考課之事,內以責諸彈糾之職,外以責諸
 監司、郡守之計。貪濁昏庸,固在必懲。廉能正直,尤當示勸。察之精則黜陟之咸服,行之力則觀聽之具孚,而課吏之實得矣。」



 進兼侍講。又言:「民者,邦之命脈,欲壽國脈,必厚民生,欲厚民生,必寬民力。」且條上厲民四敝。又言:「願陛下為萬世根本之慮,為一時倉卒之防,必求安節之亨,毋招不節之咎,節之又節,則宮闈之費差省,帑藏之積自充,上用足而下不匱矣。」又乞「察欣瘁休戚之故,酌利害損益之宜,孰為當因,孰為當革,孰為可罷,孰為
 可行,則折衷泉貨而遠近便,開通關梁而商賈行。下修身奉法之詔,而吏得自新;出輸倉助貸之令,而民免貴糴;窒墨敕之門,而無官府黜陟之異;止輸臺之議,而無疆界彼此之分,則氣脈蘇醒、意向翕合矣」。



 遷起居舍人兼國史編修、實錄檢討,尋遷殿中侍御史、侍御史。諫作宗陽宮。權工部侍郎兼同修國史實錄院,乞致仕,特轉一官。昭德在經筵,以《易》、《春秋》、《大學衍義》進講,反覆規正者甚多。所著《春山文集》。



 子徹孫,咸淳元年進士。



 陳塏字子爽,嘉興人。歷京湖制置使司主管機宜文字,差知德安府,加直寶謨閣、江西提點刑獄,改直敷文閣、提舉千秋鴻禧觀,轉司農寺丞、主管崇道觀、知安慶府。召赴闕,加直顯謨閣、湖南提點刑獄。再召為右司郎官,加直寶文閣知隆興府、江西安撫使,改知江州,主管江西安撫司事。召為右司郎官,進直龍圖閣、浙西提點刑獄,遷司農少卿,以秘閣修撰知慶元府兼沿海制置副使,遷大理卿,進右文殿修撰、知平江府兼淮、浙發運使。



 戶部侍郎趙必願舉塏最,詔特轉一官,遷太府卿、司農卿,權工部侍郎兼同詳定敕令官,兼中書門下省檢正諸房公事。入奏,言:「願陛下轉移世道之樞機,砥礪士大夫之廉恥,使知名義為重,利祿為輕。久去國以恬退聞者召之,久立朝以更迭請者從之,甘言容悅者必斥,真情丐閑者勿留。如此,則君臣上下皆以真實相與,四維既張,士大夫難進易退之風,當見於聖世,人才幸甚!」又言:「請以從官仿古昔人從出藩之意,其從臣為諸路憲
 漕,則以提點刑獄使、轉運使系銜,假之『使』名,示與庶官別,仍乞除授自臣始。」自是屢言於帝前,不許。以言罷。



 未幾,進集英殿修撰、知婺州,改知太平州兼江東轉運副使。請蠲放諸郡災傷。加戶部侍郎、淮東總領,尋提領江、淮茶鹽所兼知太平州。發公帑代三縣輸折絲帛錢五十萬九千三百六十餘貫。又作浮淮書堂以處兩淮之民而教之。進顯謨閣待制、知廣州,權兵部尚書,又進寶章閣直學士、知婺州,遷權戶部尚書,尋為真,時暫兼吏
 部尚書,以寶文閣學士知潭州兼湖南安撫使。召赴闕,以舊職提舉太平興國宮,加龍圖閣學士,依舊宮觀。久之,加端明殿學士。咸淳四年,卒,謚清毅。



 塏屢歷麾節,軍民愛戴,幕客盛多,而塏又樂薦士。所著《可齋瓿稿》二十卷。



 楊文仲字時發,眉州彭山人。七歲而孤。母胡,年二十有八,守節自誓,教養諸子。文仲既冠,以《春秋》貢,其母喜曰:「汝家至汝,三世以是經收效矣。」



 淳祐七年,文仲以胄試
 第一入太學。九年,又以公試第一升內舍。時言路頗壅,因季冬雷震,首帥同舍叩閽極言時事,有曰:「天本不怒,人激之使怒。人本無言,雷激之使言。」一時爭傳誦之。升上舍,為西廊學錄。丞相謝方叔嘗問文仲曰:「今日何事最急?」對曰:「國本未建,莫大於此。上意未喻,當以死請可也。」寶祐元年,登進士第。丁母憂,釋服,屬從叔父棟守婺州罷歸,寓餘杭,文仲往問伊、洛之學。



 調復州學教授。轉運使印應飛闢入幕。明嫠婦冤獄,應飛悉從文仲議,且
 薦之。荊湖宣撫使趙葵署文仲佐分司幕。姚希得、江萬里合薦文仲學為有用。闢四川宣撫司準備差遣,添差沿海制置司干辦公事。召為戶部架閣,遷太學正,升博士。時棟為祭酒,講學益詣精邃。遷國子博士。丐外,添差通判臺州。故事,守貳尚華侈,正月望,取燈民間,吏以白,文仲曰:「為吾然一燈足矣。」劭農東郊,守因欲泛湖,文仲即先馳歸。添差通判揚州。牙契舊額歲為錢四萬緡,累政增至十六萬,開告訐以求羨。文仲曰:「希賞以擾民,吾
 不為也。」卒增十八界一而已。制置使李庭芝檄主管機宜文字。時有沙田,使者欲舉行之,文仲力爭,以為:「事不可妄興,蓋與民之惠有限,不擾之惠無窮。江北風寒之地,民力竭矣,為利幾何,安忍重擾吾民乎!」事遂不行。



 召為宗學博士。郊祀,攝圜壇子階監察御史。近輔兵變水患,輪對,言:「皇天眷命,垂四百年,天命久熟之餘,國脈癃老之候,此豈非一大喜懼之交乎?願陛下一初清明,自作主宰。」又曰:「春多沈陰,豈但麥秋之憂。於時為《夬》,尤軫
 莧陸之慮。天目則洪水發焉,蘇、湖則弄兵興焉。峨冠於於,而每見大夫之乏使;佩印累累,而常慮貪瀆之無厭。將習黃金橫帶之娛,兵疲赤籍掛虛之穴。蚩蚩編氓,得以輕統府;瑣瑣警遽,輒以憂朝廷。設不幸事有大於此者,國何賴焉?」帝竦聽,顧問甚至。遷太常丞,尋兼權倉部郎官,兼崇政殿說書,遷將作少監,又遷將作監。



 文仲在講筵,每以積成感動,嘗進讀《春秋》,帝問五霸何以為三王罪人,文仲奏云:「齊桓公當王霸升降之會,而不能為
 向上事業,獨能開世變厲階。臣考諸《春秋》,桓公初年多書『人』,越二十年,伐楚定世子之功既成,然後書『侯』之辭迭見,此所以為尊王抑伯之大法。然王豈徒尊哉?蓋欲周王子孫率修文、武、成、康之法度,以扶持文、武、成、康之德澤,則王跡不熄,西周之美可尋,如此方副《春秋》尊王之意。」帝曰:「先帝聖訓有曰:『絲竹之亂耳,紅紫之眩目,良心善性,皆本有之。』又曰:『得聖賢心學之指要,本領端正,家傳世守,以是而君國子民,以是而祈天永命,以是而
 貽謀燕翼。』大哉先訓,朕朝夕服膺。」時帝以疾連不視朝,文仲奏:「聲色之事,若識得破,元無可好。」帝斂容端拱久之。



 盛夏,建宗陽宮,壞徙民居,畿甸騷然。文仲疏諫:「移閭閻之聚,為香火之庭,不得為善計矣。陛下紹祖宗之位,豈以黃、老之居為輕重哉。」翼日面奏,益懇至,丞相賈似道怒曰:「楊文仲多言!」紹卿監以上薦人才,文仲薦陳存、呂折、鐘季玉等十有八人,名士二人,金華王柏、天臺車若水也。兼國子司業,兼侍立修注官。又以救太學教諭
 彭成大迕似道,主管崇禧觀,出知衡州。運餉有法而民不擾,以所當得米八千石立思濟倉。召為秘書少監,尋兼崇政殿說書。以疾乞致仕,不許。兼國史院編修官、實錄院檢討官,遷太常少卿兼國子司業,遷起居舍人。



 瀛國公即位,授權工部侍郎兼權侍右郎官,尋兼給事中。有事明堂,議以上公攝行,文仲議曰:「今祗見天地之始,雖在幼沖,比即喪次,已勝拜跪,執禮無違,所當親饗。」時丞相王爚、陳宜中不協,文仲上疏言:「事危且急矣。祖宗
 所深賴,億兆所寄命,在乎二相,茍以不協之故,今日不戰,明日不征,時不再來,後悔何及!」尋兼國子祭酒。請謚金華何基及柏。時大元兵度江,畿甸震動,朝士多棄去者,侍從班惟文仲一人,詔旌在列不去者二階。文仲疾益甚,丐祠,以集英殿修撰知漳州,三上章乞致仕,改知泉州。因將家逾嶺南待次,卒,而宋亡矣。有《見山文集》焉。



 謝枋得,字君直,信州弋陽人也。為人豪爽。每觀書,五行
 俱下,一覽終身不忘。性好直言,一與人論古今治亂國家事,必掀髯抵幾,跳躍自奮,以忠義自任。徐霖稱其「如驚鶴摩霄,不事籠縶。」



 寶祐中,舉進士,對策極攻丞相董槐與宦官董宋臣,意擢高第矣,及奏名,中乙科。除撫州司戶參軍,即棄去。明年復出,試教官,中兼經科,除教授建寧府。未上,吳潛宣撫江東、西,闢差乾辦公事。團結民兵,以捍饒、信、撫,科降錢米以給之。枋得說鄧、傳二社諸大家,得民兵萬餘人,守信州,暨兵退,朝廷核諸軍費,幾
 至不免。



 五年,彗星出東方,枋得考試建康,擿似道政事為問目,言:「兵必至,國必亡。」漕使陸景思銜之,上其稿於似道,坐居鄉不法,起兵時冒破科降錢,且訕謗,追兩官,謫居興國軍。咸淳三年,赦,放歸。德祐元年,呂文煥導大元兵東下鄂、黃、蘄、安慶、九江,凡其親友部曲皆誘下之,遂屯建康。枋得與呂師夔善,乃應詔上書,以一族保師夔可信,乞分沿江諸屯兵,以之為鎮撫使,使之行成,且願身至江州見文煥與議。從之,使以沿江察訪使行,會
 文煥北歸,不及而反。



 以江東提刑、江西招諭使知信州。明年正月,師夔與武萬戶分定江東地,枋得以兵逆之,使前鋒呼曰:「謝提刑來。」呂軍馳至,射之,矢及馬前。枋得走入安仁,調淮士張孝忠逆戰團湖坪,矢盡,孝忠揮雙刀擊殺百餘人。前軍稍卻,後軍繞出孝忠後,眾驚潰,孝忠中流矢死。馬奔歸,枋得坐敵樓見之,曰:「馬歸,孝忠敗矣。」遂奔信州。師夔下安仁,進攻信州,不守。枋得乃變姓名,入建寧唐石山,轉茶阪,寓逆旅中,日麻衣躡履,東鄉
 而哭,人不識之,以為被病也。已而去,賣卜建陽市中,有來卜者,惟取米屢而已,委以錢,率謝不取。其後人稍稍識之,多延至其家,使為弟子論學。天下既定,遂居閩中。



 至元二十三年,集賢學士程文海薦宋臣二十二人,以枋得為首,辭不起。又明年,行省丞相忙兀臺將旨詔之,執手相勉勞。枋得曰:「上有堯、舜,下有巢、由,枋得名姓不祥,不敢赴詔。」丞相義之,不強也。二十五年,福建行省參政管如德將旨如江南求人材,尚書留夢炎以枋得薦,
 枋得遺書夢炎曰:「江南無人材,求一瑕呂飴甥、程嬰、杵臼廝養卒,不可得也。紂之亡也,以八百國之精兵,而不敢抗二子之正論,武王、太公凜凜無所容,急以興滅繼絕謝天下。殷之後遂與周並立。使三監、淮夷不叛,武庚必不死,殷命必不黜。夫女真之待二帝亦慘矣。而我宋今年遣使祈請,明年遣使問安。王倫一市井無賴、狎邪小人,謂梓宮可還,太后可歸。終則二事皆符其言。今一王倫且無之,則江南無人材可見也。今吾年六十餘矣,
 所欠一死耳,豈復有它志哉!」終不行。郭少師從瀛國公入朝,既而南歸,與枋得道時事,曰:「大元本無意江南,屢遣使使頓兵,令毋深入,待還歲幣即議和,無枉害生靈也。張宴然上書乞斂兵從和,上即可之。兵交二年,無一介行李之事,乃挈數百年宗社而降。」因相與痛哭。



 福建行省參政魏天祐見時方以求材為急,欲薦枋得為功,使其友趙孟TW來言,枋得罵曰:「天祐仕閩,無毫發推廣德意,反起銀冶病民,顧以我輩飾好邪?」及見天祐,又傲
 岸不為禮,與之言,坐而不對。天祐怒,強之而北。枋得即日食菜果。



 二十六年四月,至京師,問謝太后欑所及瀛國所在,再拜慟哭。已而病,遷憫忠寺,見壁間《曹娥碑》,泣曰:「小女子猶爾,吾豈不汝若哉!」留夢炎使醫持藥雜米飲進之,枋得怒曰:「吾欲死,汝乃欲生我邪?」棄之於地,終不食而死。伯父徽明以特奏恩為當陽尉,攝縣事,時天基節上壽,大元兵奄至,徽明出兵戰死,二子趨進抱父尸,亦死。



 論曰:劉應龍不附賈似道,馮去非不附丁大全,潘牥論皇子竑事,坎壈以終。洪芹訟吳潛,偉哉。趙景緯。醇儒也,而無躁競之心。徐霖進則直言於朝,退則講道於里。徐宗仁國亡與亡,異乎懷二心以事其君者也。危昭德經筵進對之言,悉載諸故史。陳塏能以意氣感人,楊文仲當搶攘之時,猶能薦士,謝枋得嶔崎以全臣節,皆宋末之卓然者也。



\end{pinyinscope}