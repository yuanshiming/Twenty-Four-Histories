\article{列傳第二十}

\begin{pinyinscope}

 李
 瓊郭瓊陳承昭李萬超白重贊王仁鎬陳思讓孫若拙焦繼勛子守節劉重進袁彥祁廷訓張鐸李萬全田景咸王暉附



 李瓊,字子玉,幽州人。祖傳正,涿州刺史。父英,涿州從事。瓊幼好學,涉獵史傳。挾策詣太原,會唐莊宗屬募勇士,即應募,與周祖等十人約為兄弟。一日會飲,瓊熟視周祖,知非常人。因舉酒祝曰:「凡我十人,龍蛇混合,異日富貴無相忘,茍渝此言,神降之罰。」皆刺臂出血為誓。周祖與瓊情好尤密,嘗造瓊,見其危坐讀書,因問所讀何書,瓊曰:「此《閫外春秋》,所謂以正守國,以奇用兵,較存亡治亂,記賢愚成敗,皆在此也。」周祖令讀之,謂瓊曰:「兄當教
 我。」自是周祖出入常袖以自隨,遇暇輒讀,每問難瓊,謂瓊為師。及討河中,乃解瓊兵籍,令參西征軍事。賊平,表於朝,授朝散大夫、大理司直。歲中,遷太子洗馬。周祖鎮鄴,表為大名少尹。



 廣順初,拜將作監,充內作坊使,賜金紫。連知亳、陜二州,改濟州刺史。世宗初,遷洺州團練使,改安州防禦使,治郡寬簡,民請立碑頌德,詔中書舍人竇儀撰文賜之。宋初,召為太子賓客。建隆三年,上章請老,改右驍衛上將軍致仕。瓊信釋氏,明年四月八日,詣
 佛寺,遇疾歸,至暮卒,年七十三,贈太子少師。



 郭瓊,平州盧龍人。祖海,本州兩冶使。父令奇,盧臺軍使。瓊少以勇力聞,事契丹,為蕃漢都指揮使。後唐天成中,挈其族來歸,明宗以為亳州團練使,改刺商州,遷原州。清泰初,移階州,城壘未葺,蜀人屢寇,瓊患之,因徙城保險,民乃無患。受詔攻文州,拔二十餘砦,生擒數百人。



 晉天福中,移刺警州,屬羌、渾騷動,朔方節度張希崇表瓊為部署,將兵共討平之。連領滑、坊、虢、衛四州。開運初,為
 北面騎軍排陣使。陽城之役,戰功居多。改沂州刺史,充荊口砦主兼東面行營都虞候。擒莫州刺史趙思以獻,改刺懷州。俄為北面先鋒都監。契丹陷中原,盜賊蜂起,山東為甚,契丹主命瓊復刺沂州以御盜,瓊即日單騎赴郡。盜聞瓊威名,相率遁去。



 漢乾祐中,淮人攻密州,以為行營都部署,未至,淮人解去。會平盧節度劉銖恃佐命之舊,稱疾不朝,將相大臣,懼其難制,先遣瓊與衛州刺史郭超以所部兵屯青州。銖不自安,置酒召瓊,伏壯
 士幕下,欲害瓊。瓊知其謀,屏去從者,從容就席,略無懼色,銖不敢發。瓊因為陳禍福,銖感其言,遂治裝。俄詔至,即日上道。瓊改穎州團練使,又加防禦使。時朗州結荊、淮、廣南合兵攻湖南,詔瓊以州兵合王令溫大軍攻光州,尋以內難不果。罷歸朝,遣詣河北計度兵甲芻糧。



 周祖祀南郊,召權知宗正卿事。世宗征劉崇,為北面行營都監,歷絳、蔡、齊三州防禦使。在齊州,民饑,瓊以己俸賑之。人懷其惠,相率詣闕頌其德政,詔許立碑。



 宋建隆三
 年,告老,加右領軍衛上將軍致仕,歸洛陽。乾德二年,卒,年七十二。瓊雖起卒伍,而所至有惠政,尊禮儒士,孜孜樂善,蓋武臣之賢者也。



 陳承昭,江表人。始事李景為保義軍節度,周世宗征淮南,景以承昭為濠、泗、楚、海水陸都應援使。世宗既拔泗州,引兵東下,命太祖領甲士數千為先鋒,遇承昭於淮上擊敗之,追至山陽北,太祖親禽承昭以獻。世宗釋之,授右監門衛上將軍,賜錦袍、銀帶,改右領軍衛上將軍,
 分司西京。宋初入朝,太祖以承昭習知水利,督治惠民、五丈二河以通漕運,都人利之。建隆二年,河成,賜錢三十萬。承昭言其婿王仁表在南唐,帝為致書於李景,令遣歸闕,歷左右神武統軍。



 四年春,大發近甸丁壯數萬,修畿內河堤,命承昭董其役。又令督諸軍子弟數千,鑿池於朱明門外,以習水戰。從征太原,承昭獻計請壅汾水灌城,城危甚,會班師,功不克就。乾德五年,遷右龍武軍統軍。開寶二年,卒,年七十四。贈太子太師,中使護喪。
 大中祥符元年,錄其孫宗義為三班借職。



 李萬超,並州太原人。幼孤貧,負販以養母,晉祖起並門,萬超應募隸軍籍。戰累捷,稍遷軍校。從李守貞討楊光遠於青州,奮勇先登,飛石中其腦,氣不屬者久之。開運中,從杜重威拒契丹於陽城,流矢貫手,萬超拔矢復戰,神色自若。以功遷肅銳指揮使。



 契丹入中原,時萬超以本部屯潞州,主帥張從恩將棄城歸契丹,會前驍衛將軍王守恩服喪私第,從恩即委以後事,遁去。及契丹使
 至,專領郡務,守恩遂無所預。萬超奮然謂其部下曰:「我輩垂餌虎口,茍延旦夕之命,今欲殺使,保其城。非止逃生,亦足建勛業,汝曹能乎?」眾皆躍然喜曰:「敢不唯命。」遂率所部大噪入府署,殺其使,推守恩為帥,列狀以聞。漢祖從其請,仍命史弘肇統兵先渡河至潞,見萬超,語之曰:「得復此州,公之力也。吾欲殺守恩,以公為帥,可乎?」萬超對曰:「殺契丹使以推守恩,蓋為社稷計爾。今若賊害於人,自取其利,非宿心也。」弘肇大奇之,表為先鋒馬步
 軍都指揮使,路經澤州,刺史翟令奇堅壁拒命,萬超馳至城下,諭之曰:「今契丹北遁,天下無主,並州劉公仗大義,定中土,所向風靡,後服者族,盍早圖之。」令奇乃開門迎納。弘肇即留萬超權州事,漢祖遂以為刺史。及徵李守貞,以萬超為行營壕砦使。河中平,拜懷州刺史。



 周祖開國,從征慕容彥超,又為都壕砦使,以功授洺州團練使,預收秦、鳳,改萊州。從平淮南,連移蘄、登二州,所至有善政。屬有詔重均田租,前牟平令馬陶,籍隸文登縣,隱
 苗不通,命系之,將斬而後聞。陶懼遁去,由是境內肅然。宋初,入為右武衛大將軍,遷左驍衛大將軍。開寶八年,卒,年七十二。



 白重贊,憲州樓煩人,其先沙陀部族。重贊少從軍,有武勇。漢初,自散員都虞候三遷護聖都指揮使。乾祐中,李守貞據河中叛,隱帝以重贊為行營先鋒都指揮使。河中平,以功領端州刺史。周初,轉護聖左廂都指揮使。未幾,出為鄭州防禦使,改相州留後。廣順中,授義成軍節
 度。在鎮日,河屢決,重贊親部丁壯,塞大程、六合二堤,詔書褒美。



 世宗征劉崇,以重贊為河東道行營馬軍都指揮使,重贊與李重進居陣西偏,樊愛能、何徽居陣東偏。既合戰,愛能與徽皆遁走,惟重贊與重進率所部力戰,世宗自督親軍合勢薄之,並人大敗。既誅愛能等,重贊以功授保大軍節度使。及世宗征太原,以河陽劉詞為隨駕都部署,命重贊副之。其忻州監軍殺刺史趙皋及契丹大將楊耨姑,以城降,而契丹兵猶盛,命重贊及符
 彥卿擊走之。世宗還京,改河陽三城節度、檢校太尉。及征淮南,命重贊率親兵三千軍於穎上。未幾,改淮南道行營馬步軍都虞候。俄遷彰義軍節度。



 宋初,加檢校太師,改鎮涇州。有馬步軍教練使李玉,本燕人,兇狡,與重贊有隙。遂與部下閻承恕謀害重贊,密遣人市馬纓,偽造制書云重贊構逆,令夷其族。乃自持偽制並馬纓,以告都校陳延正曰:「使者致而去矣。」延正具白重贊,重贊封其書以聞。太祖大駭,令驗視之,率皆誕謬,遂命六宅
 使陳思誨馳赴涇州,禽玉及承恕鞫問,伏罪棄市。延正擢領刺史以賞之,仍詔諸州,凡被制書有關機密,則詳驗印文筆跡。俄改泰寧軍節度。乾德四年,又為定國軍節度。開寶二年,改左千牛衛上將軍,奉朝請。三年,卒,年六十二。



 王仁鎬,邢州龍岡人。後唐明宗鎮邢臺,署為牙校,即位,擢為作坊副使,累遷西上閣門使。清泰中,改右領軍衛將軍。晉天福中,青州楊光遠將圖不軌,以仁鎬為節度
 副使,伺其動靜。歷二年,或譖仁鎬於朝,改護國軍行軍司馬。仁鎬至河中數月,光遠反書聞。漢乾祐中,歷昭義、天雄二軍節度副使。



 周祖鎮鄴,表仁鎬為副留守。及起兵,仁鎬預其謀。周祖即位,仁鎬為王峻所忌,出為唐州刺史,遷棣州團練使,入為右衛大將軍,充宣徽北院使兼樞密副使。顯德初,出為永興軍節度使。世宗嗣位,移河中。會殿中丞上官瓚使河中還,言河中民多匿田租,遂遣瓚按視均定。百姓苦之,多逃亡他郡,仁鎬抗論其
 事,乃止。丁繼母憂,去官。



 五年,拜安國軍節度,制曰:「眷惟襄國,實卿故鄉。分予龍節之權,成爾錦衣之美。」郡民扶老攜幼,迎於境上,有獻錦袍者四,仁鎬皆重衣之,厚酬以金帛。視事翌日,省其父祖之墓,周視松檟,涕泗嗚咽,謂所親曰:「仲由以為不如負米之樂,信矣。」時人美之。郡有群盜,仁鎬遣使遺以束帛,諭之,悉遁去,不復為盜。恭帝嗣位,移山南東道節度。



 宋初,加檢校太師。建隆二年,以疾召還,次唐州,卒於傳舍,年六十九。



 仁鎬性端謹儉
 約,崇信釋氏,所得俸祿,多奉佛飯僧,每晨誦佛經五卷,或至日旰方出視事從事劉謙責仁鎬曰:「公貴為藩侯,不能勤恤百姓,孜孜事佛,何也?」仁鎬斂容遜謝,無慍色。當時稱其長者。



 陳思讓,字後己,幽州盧龍人。父審確,仕後唐至晉,歷檀、順、涿、均、沁、唐、祁、城八州刺史。預征蜀,權利州節度,終金州防禦使。思讓初隸莊宗帳下,即位,補右班殿直。晉天福中,轉東頭供奉官,再遷作坊使。安從進叛於襄陽,以
 思讓為先鋒右廂都監,從武德使焦繼勛領兵進討。遇從進之師於唐州花山下,急擊大破之,從進僅以身免。以功領獎州刺史。從進平,授坊州刺史。



 八年冬,契丹謀入寇,以思讓監澶州軍,賜鞍勒馬、器帛。討楊光遠於青州也,又為行營右廂兵馬都監,兵罷,改磁州刺史。會符彥卿北征契丹,思讓表求預行。未幾,改衛州。連丁內外艱。時武臣罕有執喪禮者,思讓不俟詔,去郡奔喪,聞者嘉之。起復隨州刺史。



 漢初,移淄州,罷任歸朝。會淮南與
 朗州馬希灊合兵淮南,攻湖南,馬希廣來乞師,旋屬內難,又周祖北征,乃分兵令思讓往郢州赴援,兵未渡而希廣敗。思讓留於郢。



 周祖即位,遣供奉官邢思進召思讓及所部兵還。劉崇僭號太原,周祖思得方略之士以備邊,遣思讓率兵詣磁州,控扼澤、潞。未幾,授磁州刺史,充北面兵馬巡檢。未行,升磁州為團練,即以思讓充使。



 廣順元年九月,劉崇遣大將李瑰領馬步軍各五都,鄉兵十都,自團柏軍於鷂子店。思讓與都監向訓、張仁謙
 等率龍捷、吐渾軍,至虒亭西,與瑰軍遇,殺三百餘人,生禽百人,獲崇偏將王璠、曹海金,馬五十匹。俄遣王峻援晉州,以思讓與康延昭分為左右廂排陣使,令率軍自烏嶺路至絳州與大軍合。崇燒營遁去,思讓又與藥元福襲之。俄命權知絳州。明年春,遷絳州防禦使。



 顯德元年九月,改亳州防禦使,充昭義軍兵馬鈐轄,屢敗並人及契丹援兵,遷安國軍節度觀察留後,充北面行營馬步軍排陣使。五年,敗並軍千餘於西山下,斬五百級。是
 秋,邢州官吏、耆艾邢銖等四十人詣闕,求借留思讓,詔褒之。十二月,改義成軍節度觀察留後。



 六年春,世宗將北征,命先赴冀州以俟命。及得瓦橋關,為雄州,命思讓為都部署,率兵戍守。世宗不豫還京,留思讓為關南兵馬都部署。恭帝嗣位,授廣海軍節度。



 宋初,加檢校太傅。乾德二年,又為保信軍節度。時皇子興元尹德昭納思讓女為夫人。開寶二年夏,改護國軍節度、河中尹。七年,卒,年七十二。贈侍中。



 思讓累歷方鎮,無敗政,然酷信釋
 氏,所至多禁屠宰,奉祿悉以飯僧,人目為「陳佛子」。身沒之後,家無餘財。弟思誨,至六宅使。子欽祚,累遷至香藥庫使、長州刺史。欽祚子若拙。



 若拙字敏之。幼嗜學,思讓嘗令持書詣晉邸,太宗嘉其應對詳雅,將縻以軍職,若拙懇辭。太平興國五年,進士甲科,解褐將作監丞、通判鄂州,改太子右贊善大夫、知單州。以能政,就改太常丞,遷監察御史,充鹽鐵判官。益州系囚甚眾,太宗覽奏訝之,召若拙面諭委以疏決,遷
 殿中侍御史、通判益州。淳化三年,就命為西川轉運副使,未幾,改正使,召歸。會李至守洛都,表若拙佐治,改度支員外郎,通判西京留司。久之,柴禹錫鎮涇州,復奏為通判,遷司封員外郎,部送芻糧至塞外,優詔獎之。



 入為鹽鐵判官,轉工部郎中。與三司使陳恕不協,求徙他局,改主判開拆司。車駕北巡,命李沆留守東京,以若拙為判官。河決鄆州,朝議徙城以避水患,命若拙與閻承翰往規度,尋命權京東轉運使,因發卒塞王陵口,又於齊
 州浚導水勢,設巨堤於採金山,奏免六州所科梢木五百萬,民甚便之。河平,真授轉運使。召還,拜刑部郎中、知潭州。時三司使缺,若拙自謂得之。及是大失望,因請對,言父母年老,不願遠適,求納制命。上怒,謂宰相曰:「士子操修,必須名實相副,頗聞若拙有能幹,特遷秩委以藩任,而貪進擇祿如此。往有黃觀者,或稱其能,選為西川轉運使,輒訴免,當時黜守遠郡。今若拙復爾,亦須譴降。凡用人,豈以親疏為間,茍能盡瘁奉公,有所樹立,何患
 名位之不至也。」乃追若拙所授告敕,黜知處州,徙溫州。代還,復授刑部郎中,再為鹽鐵判官,改兵部郎中、河東轉運使,賜金紫。



 會親祀汾陰,若拙以所部緡帛、芻粟十萬,輸河中以助費,經度制置使陳堯叟言其干職,擢拜右諫議大夫,徙知永興軍府。時鄰郡歲饑,前政拒其市糴,若拙至,則許貿易,民賴以濟。又移知鳳翔府,入拜給事中、知澶州。蝗旱之餘,勤於政治,郡民列狀乞留。天禧二年,卒,年六十四。錄其子映為奉禮郎。



 若拙多誕妄,寡
 學術,當時以第二人及第者為榜眼,若拙素無文,故目為「瞎榜」云。



 焦繼勛,字成績,許州長社人。少讀書有大志,嘗謂人曰:「大丈夫當立功異域,取萬戶侯。豈能孜孜事筆硯哉?」遂棄其業,游三晉間為輕俠,以飲博為務。晉祖鎮太原,繼勛以儒服謁見,晉祖與語,悅之,留帳下。天福初,授皇城兼宮苑使,遷武德使。安重榮反鎮州,安從進自襄陽舉兵為應。晉祖命繼勛督諸將進討。至唐州南,遇從進軍
 萬餘,設伏擊敗之,禽其牙將安洪義、鮑洪等五十餘人,得山南東道印,從進單騎奔還。從進弟從貴率兵千餘人,援均州刺史蔡行遇,繼勛殺其眾七百,生禽百,獲從貴,斷腕放入城中,從進自此不能復鎮。繼勛以功就拜齊州防禦使。少帝即位,從進平,藉繼勛威名鎮之,徙襄陽防禦使。歲餘,入為右千牛衛大將軍,拜宣徽北院使,遷南院使。



 西人寇邊,朝議發師致討,繼勛抗疏請行,拜秦州觀察使兼諸蕃水陸轉運使。既至,推恩信、設方略
 招誘,諸郡酋長相率奉玉帛、牛酒乞盟,邊境以安。俄徙知陜州,就遷保義軍兵馬留後。



 漢初,鳳翔軍校陽彥昭據城叛,命繼勛率師討之,以功授保大軍節度。召入,會漢祖幸大名,留為京城右廂巡檢使,俄改右羽林統軍。隱帝末,命繼勛領兵北征。及周祖舉兵向闕,繼勛奉隱帝逆戰於留子陂,戰不利,遂歸周祖。



 廣順初,改右龍武統軍。世宗征淮南,為左廂排陣使,又改右羽林統軍、左屯衛上將軍,以戰功拜彰武軍節度。



 宋初,召為右
 金吾衛上將軍,改右武衛上將軍。乾德三年,權知延州。四年,判右街仗杜審瓊卒,命繼勛代之。時向拱為西京留守,多飲燕,不省府事,群盜白日入都市劫財,拱被酒不出捕逐。太祖選繼勛代之,月餘,京城肅然。太祖將幸洛,遣莊宅使王仁珪、內供奉官李仁祚部修洛陽宮,命繼勛董其役。車駕還,嘉其幹力,召見褒賞,以為彰德軍節度,仍知留府事。仁珪領義州刺史,仁祚為八作副使。繼勛以太平興國三年卒,年七十八,贈太尉。



 繼勛獵涉
 史傳,頗達治道,所至有善政。然性吝嗇,多省公府用度,時論少之。子守節。



 守節字秉直,初補左班殿直,選為江、淮南路採訪。還奏稱旨,擢閣門祗候。李順餘黨擾西川,命與上官正討平之。高、溪州蠻內寇,又命往圖方略,守節言:「山川回險,非我師之利。」詔許招納。



 咸平中,置江淮南、荊湖路兵馬都監,首被選擢。又討施、夔州叛蠻,以大義諭其酋長,皆悔過內附,因為之畫界定約。還遷閣門通事舍人,監香藥
 榷易院,三司言歲課增八十餘萬。時守節已為衣庫副使,當遷閣門副使,真宗謂輔臣曰:「守節緣財利羨餘而遷橫行,何以勸邊陲效命者?」止以為宮苑副使。



 奉使契丹,館伴丁求說指遠山謂曰:「此黃龍府也。」守節應聲曰:「燕然山距此幾許?」求說慚服。久之,遷皇城副使,管勾軍頭引見司。坐以白直假樞密院副承旨尹德潤治第,免所居官。三遷東上閣門使,加榮州刺史。數請補外,歷知襄、鄧、汝三州,遷四方館使,以右神武大將軍致仕卒。



 劉重進,幽州人,本名晏僧。梁末隸軍籍。晉初,以習契丹語,應募使北邊,改右班殿直,因賜是名。遷西頭供奉官,再使契丹。契丹主以其敏慧,留為帳前通事;俄南侵,署重進忠武軍節度。



 漢初,移鎮鄧州。漢法,禁牛革甚嚴,州民崔彥、陳寶選八人自本鎮持革詣漢祖廟鞔鼓,重進杖遣之。判官史在德謂重進不善用法,宜置極典。及大理、刑部詳覆,重進所斷為是。在德坐故入,杖死之。



 乾祐末,罷鎮來朝。周祖起兵至封丘,詔重進與左神武統軍
 袁義率兵拒之,重進望塵退走。周廣順初,從征兗州。未幾,封薛國公。俄召為右神武統軍,累加檢校太師。世宗南征,為右廂排陣使。顯德三年,世宗聞揚州無備,遣宣祖、韓令坤與重進等往襲取之,又為先鋒都部署,進克泰州。初,楊行密子孫居海陵,號永寧宮,周師渡淮,盡為李景所殺。重進入其家,得玉硯、玉杯盤、水晶盞、瑪瑙碗、翡翠瓶以獻。俄命判廬州行府事兼行營都部署,敗淮人千餘於州境,又敗五百眾於白城湖。及世宗再巡,吳
 師潰於紫金山,有至東山口者,重進殺三千餘眾。及下壽州,以功授武勝軍節度。淮南平,改鎮邠州。世宗北征,為先鋒都指揮使。恭帝即位,封開府。



 宋初,進封燕國公。建隆二年秋,授右羽林統軍。乾德五年,改左領軍衛上將軍。重進徒善譯語,無他才能,值契丹入中原,遂至方鎮。及在環衛,嘗從幸玉津園,太祖召與語。既退,謂左右曰:「觀重進應對不逮常人,前朝以為將帥,何足重耶?」六年,卒,年七十。



 袁彥,河中河東人。少以趫勇應募從事,隸奉國營。漢乾祐中,周祖領軍討李守貞,以彥置麾下,及鎮鄴,以為部直小將。周廣順中,世宗在澶淵,遷為親事都校。世宗尹京,改開封府步直指揮使。顯德初,授內外步軍都軍頭,領泉州刺史。未幾,改嶽州防禦使。從征壽州,為城北造竹龍都部署。竹龍者,以竹數十萬竿,圍而相屬,上設版屋,載甲士數百人,以攻其城。又命於渦口修橋,橋成,世宗幸焉,因立為鎮淮軍。李繼勛以淮上失律,罷軍職,命
 彥為武信軍節度,權侍衛步軍都指揮使。又命為淮南道行營馬步軍副都指揮使,賜衣服、金帶、鞍勒馬、鎧甲、器仗,遣赴軍前。



 太祖下滁陽,禽皇甫暉、姚鳳,彥皆有勞績,詔褒之。又令率師屯下蔡以逼壽春。及劉仁贍降,從世宗攻濠、泗,又禽南唐將許文績、邊鎬等以獻。師還,真授步軍都指揮使,領彰信軍節度。六年春,發近畿丁壯浚五丈河,命彥董其役。恭帝嗣位,移保義軍節度。



 宋初,加檢校太尉。是秋來朝,改鎮曹州。乾德六年,為靜難軍
 節度。開寶二年,移鄜州。五年,罷鎮歸闕,卒,年六十六。景德四年,特詔錄其孫昭慶為借職。大中祥符八年,昭慶上彥周朝所受告敕有二聖名諱者,特遷殿直。



 祁廷訓,本名廷義,避太宗舊名改焉。河南洛陽人。父珪,梁左監門衛大將軍。廷訓善書計、騎射,隸周祖帳下。廣順中,歷東西班右蕃行首、鐵騎都虞候。世宗即位,改東西班都指揮使,遷內殿直都指揮使,繼領蘭、睦二州刺史。從征淮南,賜以明光細甲,令董舟師巡江界。吳人伏
 兵三江口葭荻中,掩擊廷訓,廷訓力戰大破之,俘馘千人,餘黨遁去。江北平,以功遷吉州團練使,領鐵騎左廂都指揮使。月餘,遷嵐州防禦使,領龍捷右廂都指揮使。



 宋初,為安遠軍節度觀察留後,是秋,改河陽。乾德二年,又改彰德軍節度留後,俄權知鄧州。五年,就拜義武軍節度。開寶二年,太祖征太原,以廷訓為北面副都部署。太平興國元年來朝。二年冬,改左領軍衛上將軍。五年,坐私販竹木貴鬻入官,責本衛大將軍。未幾,復舊官。
 六年,卒,年五十八。



 廷訓形質魁岸,無才略,臨事多規避,時人目為「祁橐駝」,以其龐大而無所取也。



 張鐸,河朔人,少以材武應募隸軍籍。漢初,為奉國右第六軍都指揮使,領澧州刺史。周祖以樞密使鎮鄴,鐸以所部從行,及起兵,鐸預焉。廣順初,鐸為奉國左廂都指揮使,韓通為右廂都指揮使;俄並兼防禦使,鐸領永州,通領睦州。會改奉國為虎捷,鐸仍領其職。是冬,出為密州防禦使,改亳州。三年,授鎮國節度。郊祀畢,加檢校太
 傅。世宗初,移彰義軍,未幾,加檢校太尉。顯德三年,又移河中尹、護國軍節度。



 宋初,加檢校太師,俄復鎮涇州。州官歲市馬,鐸厚增其直而私取之,累至十六萬貫,及擅借公帑錢萬餘緡,侵用官曲六千四百餅。事發,召歸京師,本州械系其子保常及親吏宋習。太祖以鐸宿舊,釋不問,罷鎮為左屯衛上將軍,奉朝請而已。其所盜用,仍蠲除之,保常、習亦得釋。鐸又嘗假晉邸錢百六十萬,太宗即位,詔貰之。俄命判左金吾街仗。及駕征河東,以鐸
 為京城內外都巡檢,鄜州刺史高繼充、閑廄副使張守明分為里城左右廂巡檢。雍熙三年,卒,年七十二。贈太傅。



 子熙載至左千牛衛大將軍。熙載子禹珪字天錫,粗知書,有方略,幼事太宗藩邸,即位,補東西班承旨,改殿直,帶御器械。以材勇擢居禁衛,殿前散祗候都虞候。咸平初,授內殿直都虞候,領恩州刺史。三年,出為滁州刺史,知洺、瀛、霸三州。並兼兵馬鈐轄,徙嵐州。西人勒厥麻誘眾叛,禹珪率眾討之,俘六千餘人,獲名馬孳畜甚眾。



 景德初,授高陽關行營副都部署。契丹既請和,帝思守臣有武幹能鎮靜邊郡者,親錄十餘人名付中書,禹珪預焉。遂知石州,徙代、兗州,又移澶州,頗勤政治,以瑞麥生、獄空,連詔嘉獎。會河堤決溢,禹珪率徒塞之,宰相王旦使兗州還,言其狀,優詔褒之。就拜洺州團練使,尋知廣信軍。天禧初,復為高陽關副都部署兼知瀛州。明年召還,將授四廂之職,卒,年五十九。錄其二子。



 李萬全,吐谷渾部人。善左右射,隸護聖軍為騎士,累遷
 至本軍都校,與田景咸、王暉等從周祖入汴,號十軍主。顯德中,為彰武軍節度。宋初,加檢校太尉、橫海軍節度。乾德中代歸,太祖數召於苑中宴射。萬全無將略,惟挽強弓,老而不衰,帝亦以此賞之。



 田景咸、王暉,皆太原人。景咸仕漢,為奉國右廂都校,從周祖入汴,為龍捷左廂都校,改安國軍留後。俄真拜,升本軍節度。世宗時,拜武勝軍節度。宋初,為左驍衛上將軍。開寶三年卒。



 景咸性鄙吝,務聚斂,每使命至,惟設肉
 一器,賓主共食。後罷鎮,常忽忽不樂。妻識其意,引景咸遍閱囊儲,景咸方自釋。在邢州日,使者王班至,景咸勸班酒曰:「王班請滿飲。」典客曰:「是使者姓名也。」景咸悟曰:「我意『王班』是官爾,何不早諭我。」聞者笑之。



 暉性亦吝嗇,貲甚富,而妻子飯疏糲,縱部曲誅求,民甚苦之。世宗以先朝功臣,知而弗問焉,至右神武統軍。建隆四年,終右領軍衛上將軍。



 論曰:太祖事漢、周,同時將校多聯事兵間,及分藩立朝,
 位或相亞。宋國建,皆折其猛悍不可屈之氣,俛首改事,且為盡力焉。揚雄有言:「御之得其道,則狙詐咸作使。」此太祖之英武而為創業之君也歟!



\end{pinyinscope}