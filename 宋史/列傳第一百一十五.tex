\article{列傳第一百一十五}

\begin{pinyinscope}

 劉拯
 錢遹石豫左膚附許敦仁吳執中吳材劉昺宋喬年子忭強淵明蔡居厚劉嗣明蔣靜賈偉節崔鶠張根弟樸任諒周常



 劉拯,字彥修,宣州南陵人。進士及第。知常熟縣,有善政,縣人稱之。元豐中,為監察御史,歷江東淮西轉運判官、提點廣西刑獄。



 紹聖初,復為御史,言:「元祐修先帝實錄,以司馬光、蘇軾之門人範祖禹、黃庭堅、秦觀為之,竄易增減,誣毀先烈,願明正國典。」又言:「蘇軾貪鄙狂悖,無事君之義,嘗議罪抵死,先帝赦之,敢以怨忿形於詔誥,醜詆厚誣。策試館職,至及王莽、曹操之事,方異意之臣,分據要略,而軾問及此,傳之四方,忠義之士,為之寒心扼
 腕。願正其罪,以示天下。」時祖禹等已貶,軾謫英州,而拯猶鷙視不愜也。進右正言累至給事中。



 徽宗立,欽聖後臨朝,而欽慈後葬,大臣欲用妃禮。拯曰:「母以子貴,子為天子,則母乃後也,當改園陵為山陵。」又言:「門下侍郎韓忠彥,雖以德選,然不可啟貴戚預政之漸。」帝疑其阿私觀望,黜知濠州。改廣州,加寶文閣待制,以吏部侍郎召還。帝稱其議欽慈事,褒進兩秩,遷戶部尚書。



 蔡京編次元祐奸黨,拯言:「漢、唐失政,皆分朋黨,今日指前人為黨,
 安知後人不以今人為黨乎?不若定為三等,某事為上,某事為中,某事為下,而不斥其名氏,」京不樂。又言戶部月賦入不足償所出。京益怒,徙之兵部。旋罷知蘄州,徙潤州。



 張商英入相,召為吏部尚書。拯已昏憒,吏乘為奸,又左轉工部,以樞密直學士知同州。時商英去位,侍御史洪彥升並劾之,削職,提舉鴻慶宮,卒。



 錢遹,字德循,婺州浦江人。以進士甲科調洪州推官,累通判越州。至校書郎。徽宗立,擢殿中侍御史。中丞豐稷
 論其回邪不可任風憲,不報。稷復言「必用遹則願罷臣」,乃以提舉湖北常平。崇寧初,召為都官員外郎、殿中侍御史。劾曾布援元祐奸黨,擠紹聖忠賢,布去。遷侍御史,閱兩月,進中丞。乞治元符末大臣嘗乞復孟後而廢劉後事,韓忠彥、曾布、李清臣、黃履及議者曾肇、豐稷、陳瓘、龔□皆坐貶。遂與殿中侍御史石豫、左膚言:「元祐皇后得罪先朝,昭告宗廟,天下莫不知。哲宗上賓,太母聽政。當國大臣盡欲變亂紹聖之事,以逞私欲,因一布衣何
 大正狂言,復還廢後位號。當時物議固已洶洶,乃至疏逖小臣,詣闕上書,忠議激切,則天下公議從可知矣。今朝廷既已貶削忠彥等,及追褫大正誤恩,則元祐皇后義非所安。孔子曰:『必也正名乎,名不正則言不順。』夫在先朝則曰後,今日則謂之元祐皇后,於名為不正;先朝廢而陛下復,於事為不順。考之典禮,則古昔所無;稽之本朝,則故實未有;詢之師言,則大以為不然。況既為先朝所廢,則宗廟祭告,歲時薦饗,人事有嫌疑之跡,神靈
 萌厭斁之心,萬世之後,配祔將安所施。宜蚤正厥事,斷以大義,無牽於流俗非正之論,以累聖朝。」



 明日,又言:「典禮所在,實朝廷治亂之所系,雖人主之尊不得而擅,又況區區臣下,敢輕變易者哉?元祐皇后得罪先朝,廢處瑤華,制誥一頒,天下無間然者。並后匹嫡,《春秋》譏之,豈宜明盛之朝,而循衰世非禮之事?」於是尚書右僕射京、門下侍郎將、中書侍郎尚書左丞挺之、右丞商英言:「元祐皇后再復位號,考之典禮,將來宗廟不可從享,陵寢
 不可配祔。揆諸禮制,皆所未安,請如紹聖三年九月詔書旨。」後由是復廢。遹、豫遂言元符皇后名位未正,乃冊為崇恩太后。



 遹章所言小臣上書者,昌州推官馮澥也。其書以謂:「先帝既終,則後無單立之義;稽之逆順,陛下無立嫂之禮;要之終始,皇太后亦不得伸慈婦之恩。雖已遂之事,難復之失,然感悟追正,何有不可?」澥用是得召對,除鴻臚主簿。



 蔡京謀取青唐,遹助成其議。會籍元祐黨,遹以為多漏略,給事中劉逵駁之,左轉戶部侍郎,
 俄遷工部尚書兼侍讀。逾年,以樞密直學士知穎昌府。言者疏其罪,黜為滁州,稍復顯謨閣待制、直學士,徙宣州。復為工部尚書,舉馮澥自代,謂:「澥趣操端勁,古人與稽,嘗建明典禮,忠義凜凜,搢紳嘆服。」言者又疏其罪,以待制知秀州;中書舍人侯綬封還之,又奪待制。久之,還故職,改述古殿直學士。屏居十五年,方臘陷婺,遹逃奔蘭溪,為賊所殺,年七十二。



 石豫者,寧陵人。第進士。以安惇薦,為監察御史。與左膚
 鞠鄒浩獄,文致重比,又使廣東鐘正甫逮治浩,欲致之死。豫論邊事,謂中國與四夷,相交為君臣,相與為賓客。徽宗以其言無倫理,且辱國,出為淮南轉運判官。陳瓘又追論羅織鄒浩事,降通判亳州。崇寧元年,召拜殿中侍御史。遂同錢遹造廢元祐皇后議,亟遷待御史,至中丞。請削去景靈宮繪像臣僚,自文彥博、司馬光、呂公著、呂大防、范純仁、劉摯、範百祿、梁燾、王巖叟以下。既,以論罷軍器監蔡碩,碩訟豫平生交通狀,黜知陳州,徙鄧州。
 過闕,留為工部侍郎,進戶部,兼侍讀。以調度不繼,降秩一等,徙刑部。祖母死,用嫡孫承重去官,服未闋而卒。



 膚,廬州人,亦用安惇薦為御史,履歷大略與石豫同。遷侍御史,累至刑、兵、戶三尚書,以樞密直學士知河南府,改永興軍,卒。



 許敦仁,興化人。第進士。崇寧初,入為校書郎。蔡京以州里之舊,擢監察御史,亟遷右正言、起居郎,倚為腹心。敦仁凡所建請,悉受京旨,言:「元符之末,奸臣用事,內外制
 詔,類多誣實。乞自今日以前,委中書舍人或著作局討論刪正。」起居郎、舍人,異時遇車駕行幸,惟當直者從,敦仁始請悉扈蹕。遷殿中監,拜御史中丞。甫視事,即上章請五日一視朝。徽宗以其言失當,乖宵旰圖治之意,命罰金,仍左遷兵部侍郎;他日,為朱諤言,且欲逐敦仁,而京庇之甚力,敦仁亦處之自如。後二年卒。靖康中,諫官呂好問論蔡京使敦仁請五日一視朝,欲顓竊國命,蓋指此也。



 吳執中,字子權,建州松溪人。登嘉祐進士第,歷官州縣。同門婿呂惠卿方貴盛,不肯附以取進。凡三十餘年,始提舉河南常平,連徙河東、淮南、江東轉運判官,提點廣東刑獄,入為庫部、吏部、右司郎中。



 大觀初,擢兵部侍郎。二年,進御史中丞,論開封府、內侍省、京畿、秦鳳違法干請,詔獎其得風憲體。又言:「開封之治事,大理之決獄,將作之營繕,榷貨之入中,皆職所當為,乃妄以為功,一歲遷官至五六,宜行抑損。」遂詔自今但賜束帛。鄭居中知
 樞密院,執中言外戚不宜在政地,帝還其章,而諭所以用居中之意。



 初,蔡京忌張康國,故引執中居言路。執中先劾劉炳兄弟、宋喬年父子,皆京客也。帝嘗語執政,嘉其不阿。康國曰:「是乃為逐臣地耳。」已而章果至。帝怒,黜知滁州。未幾,徙越州。石公弼以為執中反復得罪,未宜殿大府。改提舉洞霄宮,以集賢殿修撰知揚州,加顯謨閣待制、知河南府。道過都,復拜中丞。



 帝以星變逐蔡京,言者未已,執中謂進退大臣,當全體貌,於是為京下詔,
 京得不重貶。龐恭孫、趙遹適開梓、夔諸夷州,執中乞正其罪。又言:「八行之舉,所得皆鄉曲常人,不足以為士,願下太學,考其道藝而進退之。」所論多施行。遷禮部尚書。



 張商英罷,御史張克公言,執中與商英皆由郭天信以進,除樞密直學士、知越州。尋降待制,又奪職。卒於家。



 吳材,字聖取,處州龍泉人。中進士第,歷青溪主簿、咸平尉、知江都縣。入為太學博士,以趙挺之薦,擢右正言,遷左司諫。黨論復起,材首論范純禮為朋附黨與,前日大
 臣變更神考法度,故引之執政,不宜復其職;程之元為蘇軾心腹,不宜亞九卿;張舜民當初政時,猖狂無所顧忌,不宜以從官處鄉郡。其後受曾布指,與王能甫疏言:「元符之末,變神考之美政,逐神考之人材者,韓忠彥實為之首。」忠彥遂罷。



 材鷙忍,疾視善類,所排逐最多。進起居郎,以憂去。蔡京用為給事中、吏部侍郎,陛見,有所陳,京不悅。以天章閣待制知光州。挺之作相,召拜工部侍郎,卒。



 論曰:紹述說行,權臣顓假以攻元祐正士;網既盡矣,復假以攻異己。鷹犬外搏,鬼蜮內狙,宜小人得志而空朝廷也。故劉拯摭實錄以肆詆,錢遹斥孟後以遍刺,石豫指繪像以削諸賢,吳材擿黨論以揃善類;許敦仁五日一朝之請,吳執中體貌大臣之言,俱蔡京腹心計也。讒說殄行,虞帝攸SW;似是而非,孔聖惡佞。有國家者,可不監夫!



 劉昺,字子蒙,開封東明人,初名炳,賜今名。元符末,進士
 甲科,起家太學博士,遷秘書省正字、校書郎。



 兄煒,通樂律。煒死,蔡京擢昺大司樂,付以樂正。遂引蜀人魏漢津鑄九鼎,作《大晟樂》。昺撰《鼎書》、《新樂書》,皆漢津妄出己意,而為緣飾,語在《樂志》。累遷給事中。京置局議禮,昺又領之。為翰林學士,改工部尚書。提舉《紀元歷》,有所損益,為吳執中所論,以顯謨閣直學士知陳州。



 昺與弟煥皆侍從,而親喪不葬,坐奪職罷郡,復以事免官。京再輔政,召為戶部尚書。昺嘗為京畫策,排鄭居中,故京力援昺,
 由廢黜中還故班。御史中丞俞慄發其奸利事,京徙慄他官。



 徽宗所儲三代彞器,詔昺討定,凡尊爵、俎豆、盤匜之屬,悉改以從古,而載所制器於祀儀,令太學諸生習肄雅樂。閱試日,昺與大司成劉嗣明奏,有鶴翔宮架之上。再為翰林學士,東宮建,為太子賓客,又還戶部。



 大理議戶絕法,若祖有子未娶而亡,不得養孫為嗣。昺曰:「計一歲諸路戶絕,不過得錢萬緡。使歲失萬緡而天下無絕戶,豈不可乎?」詔從其議。加宣和殿學士,知河南府,積
 官金紫光祿大夫。與王寀交通,事敗,開封尹盛章議以死,刑部尚書範致虛為請,乃長流瓊州。死,年五十七。



 宋喬年,字仙民,宰相庠之孫也。父充國,刻意問學,以鄉書試禮部;既,自謂宰相子,輒罷舉。仁宗知之,召試學士院,賜進士出身,簽書河南判官,判登聞鼓院,知太常禮院。英宗祔廟,議者欲祧僖祖藏夾室,充國請配感生帝為宋始祖,從之。東西府建,上二箴以戒大臣,大臣不懌。會廟饗宿齋,其妻遣兩妾至寺,充國自劾,罷禮院,遂致
 仕。充國性剛介,孝於奉親,平居得微物,必先薦家廟,乃敢嘗。官至太中大夫,卒。



 喬年用父蔭監市易,坐與倡女私及私役吏失官,落拓二十年。女嫁蔡京子攸。京當國,始復起用。崇寧中,提舉開封縣鎮、府界常平,改提點京西北路刑獄。賜進士第,加集賢殿修撰、京畿轉運副使,進顯謨閣待制,為都轉運使,改開封尹,以龍圖閣學士知河南府。京罷相,諫議大夫毛注、御史中丞吳執中交擊之,貶保靜軍節度副使,蘄州安置。京復相,還舊官,知
 陳州。政和三年,卒,年六十七,謚曰忠文。子忭。



 忭字景裕。崇寧初,由譙縣尉為敕令刪定官,數年,至殿中少監。時喬年尹京,父子依憑蔡氏,陵轢士大夫,陰交諫官蔡居厚,使為鷹犬。以徽猷閣待制知陳州。喬年貶,忭亦謫少府少監,分司南京,未幾,知應天府。



 喬年卒,起復為京西都轉運使,蒞葺西宮及修三山新河,擢至顯謨閣學士,方是時,徽宗議謁諸陵,有司預為西幸之備。忭治宮城,廣袤十六里,創廊屋四百四十間,費不可勝。
 會髹漆,至灰人骨為胎,斤直錢數千。盡發洛城外二十里古塚,凡衣冠壟兆,大抵遭暴掘。用是遷正議大夫、殿中監,又奉命補治三陵洩水坑澗,計役四百九十萬工。未幾,卒,贈金紫光祿大夫、延康殿學士,謚曰恭敏。



 強淵明,字隱季,杭州錢塘人。父至,以文學受知韓琦,終祠部郎中。淵明進士第,調海州司法參軍,歷濟、杭二州教授,知蔡州確山縣,通判保定軍。入為太府丞、軍器少監、國子司業。與兄浚明及葉夢得締蔡京為死交,立元
 祐籍,分三等定罪,皆三人所建,遂濟成黨禍。淵明以故亟遷秘書少監、中書舍人、大司成、翰林學士。



 大觀三年,京罷相,以龍圖閣直學士知永興軍,徙鄭、越二州。召為禮部尚書,復拜學士,進承旨。翰林廣直廬,帝書「摛文堂」榜賜之。兼太子賓客。以疾,改延康殿學士、提舉醴泉觀兼侍讀、監修國史。卒,贈金紫光祿大夫、資政殿學士,謚曰文憲。浚明早死。



 蔡居厚,字寬夫,熙寧御史延禧子也。延禧嘗擊呂惠卿
 兄弟,有直名。居厚第進士,累官吏部員外郎。大觀初,拜右正言,奏疏曰:「神宗造立法度,曠古絕儗,雖符、祐之黨力起相軋,而終不能搖者,出於人心理義之所在也。陛下繼志廣聲,政事具舉,願如明詔敕有司勒為成書,以明一代之制。」遷起居郎,進右諫議大夫。論東南兵政七弊,及言學官書局皆為要塗,宜公選實學多聞之士,無使庸常之徒。得以幸進。



 河北、河東群盜起,太原、真定守皆以不能擒捕罪去。居厚言:「將帥之才,不儲養於平時,
 故緩急無所可用,宜令觀察使以上,各舉所知。」又言:「比來從事於朝者,皆姑息胥吏,吏強官弱,浸以成風。蓋輦轂之下,吏習狡獪,故怯懦者有所畏,至用為耳目,倚為鄉導,假借色辭,過為卑辱,浸淫及於侍從。今廟堂之上,稍亦為之,願重為之制。」改戶部侍郎。言者論其在諫省時,為宋喬年父子用,以集賢殿修撰知秦州。降羌在州者逸入京師訴事,坐失察,削職罷。



 蔡京再相,起知滄、陳、齊三州,加徽猷閣待制,為應天、河南尹。初建神霄宮,度
 地污下,為道士交訴,徙汝州。久之,知東平府。復以戶部侍郎召,未至,又以知青州。病不能赴,未幾卒。



 劉嗣明,開封祥符人。入太學,積以試藝,名出諸生右。崇寧中,車駕幸學,解褐補承事郎,歷校書郎至給事中。



 張商英居相位,惡其不附己。時鄭居中雖以嫌去樞密,然陰殖黨與,窺伺益固。嗣明與之合,計傾商英。門下省吏張天忱貶秩,嗣明駁弗下,商英爭之。詔御史臺蔽曲直,商英以是罷。嗣明遂論商英引李士觀、尹天民入政典
 局,矯為敕語,共造奸謀,三人俱坐責。



 嗣明遷大司成。士子肄雅樂被恩,嗣明亦升班與學士等。已而言者論其取悅權貴,妄升國子生預舍法以抑寒士,黜知穎州。未幾,入為工部侍郎、翰林學士、工部尚書。卒,贈資政殿學士、太中大夫。



 蔣靜,字叔明,常州宜興人。第進士,調安仁令。俗好巫,疫癘流行,病者寧死不服藥,靜悉論巫罪,聚其所事淫像,得三百軀,毀而投諸江。知陳留縣,與屯將不協,罷去。



 徽
 宗初立,求言,靜上言,多詆元祐間事,蔡京第為正等,擢職方員外郎;中書舍人吳伯舉封還之,京怒,黜伯舉。明年,遷國子司業。帝幸太學,命講《書·無逸篇》,賜服金紫,進祭酒,為中書舍人。以顯謨閣待制知壽州,徙江寧府。



 茅山道士劉混康以技進,賜號「先生」。其徒倚為奸利,奪民葦場,強市廬舍,詞訟至府,吏觀望不敢治,靜悉抵於法。徙睦州,移病,提舉洞霄宮。越九年,召為大司成,出知洪州。復告歸,加直學士。卒,年七十一,贈通議大夫。



 賈偉節,開封人。第進士,累擢兩浙轉運判官。條上民間利病,加直秘閣,為江、淮發運副使。蔡京壞東南轉般法為直達綱,偉節率先奉承,歲以上供物徑造都下,籍催諸道逋負,造巨船二千四百艘,非供奉物而輒運載者,請論以違制。花石、海錯之急切,自此而興。論功進秩,遂拜戶部侍郎,改刑部。歲餘,以顯謨閣直學士提舉醴泉觀,卒。



 論曰:善乎歐陽修之論朋黨也,其言曰:「君子以同道為
 真朋,小人以同利為偽朋,同道則同心相益而共濟,小人見利則爭先,利盡則疏而相賊害矣。」蘇軾續修說,謂:「君子不得志則奉身而退,樂道不仕;小人不得志則僥幸復用,唯怨之報,此所以不勝也。」秦觀亦言:「君子小人,不免有黨。人主不辨邪正,必至兩廢;或言兩存,則小人卒得志,君子終受害。」其說明甚,徽宗弗之察也。唯蔽於紹述之說,崇奸貶正,黨論滋起。於是紹聖指元祐為黨,崇寧指元符為黨,而鄭居中、張商英、蔡京、王黼諸人互指
 為黨,不復能辨。始以黨敗人,終以黨敗國,衣冠塗炭,垂三十年,其禍汰於東都、白馬,蓋至是而三子之言效焉。彼劉昺、強淵明、宋喬年、劉嗣明直斗筲耳,亦使攘臂恣睢,撼撞無忌,小人之為術蹙矣。嗚呼!朋黨之說,真能空人之國如此哉。



 崔鶠字德符,雍丘人,父毗,徙居穎州,遂為陽翟人。登進士第,調鳳州司戶參軍、筠州推官。徽宗初立,以日食求言,鶠上書曰:



 臣聞諫爭之道,不激切不足以起人主意,
 激切則近訕謗。夫為人臣而有訕謗之名,此讒邪之論所以易乘,而世主所以不悟,天下所以卷舌吞聲,而以言為戒也。臣嘗讀史,見漢劉陶曹、鸞、唐李少良之事,未嘗不掩卷興嗟,矯然有山林不反之意。比聞國家以日食之異,詢求直言,伏讀詔書,至所謂「言之失中,朕不加罪」,蓋陛下披至情,廓聖度,以來天下之言如此,而私秘所聞,不敢一吐,是臣子負陛下也。



 方今政令煩苛,民不堪擾,風俗險薄,法不能勝,未暇一二陳之,而特以判左
 右之忠邪為本。臣生於草萊,不識朝廷之士,特怪左右之人,有指元祐之臣為奸黨者,必邪人也。使漢之黨錮,唐之牛、李之禍,將復見於今日,甚可駭也。



 夫毀譽者,朝廷之公議。故責授朱崖軍司戶司馬光,左右以為奸,而天下皆曰忠;今宰相章惇,左右以為忠,而天下皆曰奸。此何理也?臣請略言奸人之跡:夫乘時抵□戲以盜富貴,探微揣端以固權寵,謂之奸可也;包苴滿門,私謁踵路,陰交不逞,密結禁廷,謂之奸可也;以奇伎淫巧蕩上心,
 以倡優女色敗君德,獨操賞刑,自報恩怨,謂之奸可也;蔽遮主聽,排斥正人,微言者坐以刺譏,直諫者陷以指斥,以杜天下之言,掩滔天之罪,謂之奸可也。凡此數者,光有之乎?惇有之乎?



 夫有其實者名隨之,無其實而有其名,誰肯信之?《傳》曰:「謂狐為貍,非特不知狐,又不知貍。」是故以佞為忠,必以忠為佞,於是乎有繆賞濫罰。賞繆罰濫,佞人徜徉,如此而國不亂,未之有也。



 光忠信直諒,聞於華夷,雖古名臣,未能遠過,而謂之奸,是欺天下也。
 至如惇狙詐兇險,天下士大夫呼曰「惇賊」。貴極宰相,人所具瞻,以名呼之,又指為賊,豈非以其孤負主恩,玩竊國柄,忠臣痛憤,義士不服,故賊而名之,指其實而號之以賊邪。京師語曰「大惇小惇,殃及子孫」,謂惇與御史中丞安惇也。小人譬之蝮蠍,其兇忍害人,根乎天性,隨遇必發。天下無事,不過賊陷忠良,破碎善類;至緩急危疑之際,必有反復賣國、跋扈不臣之心。



 比年以來,諫官不論得失,御史不劾奸邪,門下不駁詔令,共持喑默,以為得
 計。昔李林甫竊相位十有九年,海內怨痛,而人主不知。頃鄒浩以言事得罪,大臣拱而觀之,同列無一語者,又從而擠之。夫以股肱耳目,治亂安危所系,而一切若此,陛下雖有堯、舜之聰明,將誰使言之,誰使行之。



 夫日者陽也,食之者陰也。四月正陽之月,陽極盛、陰極衰之時,而陰乾陽,故其變為大。惟陛下畏天威、聽明命,大運乾剛,大明邪正,毋違經義,毋鬱民心,則天意解矣。若夫伐鼓用幣,素服徹樂,而無懿德善政之實,非所以應天也。



 帝覽而善之,以為相州教授。



 後蔡京條籍上書人,以鶠為邪等,免所居官。久之,調績溪令。移病歸,始居郟城,治地數畝,為婆娑園。屏處十餘年,人無貴賤長少,悉尊師之。



 宣和六年,起通判寧化軍,召為殿中侍御史。既至而欽宗即位,授右正言。上疏曰:



 六月一日詔書,詔諫臣直論得失,以求實是,有以見陛下求治之切也。數十年來,王公卿相,皆自蔡京出。要使一門生死,則一門生用;一故吏逐,則一故吏來。更持政柄,無一人立異,無一人
 害己者,此京之本謀也。安得實是之言聞於陛下哉?



 諫議大夫馮澥近上章曰:「上無異論,太學之盛也。」澥尚敢為此奸言乎!王安石除異己之人,著《三經》之說以取士,天下靡然雷同,陵夷至於大亂,此無異論之效也。京又以學校之法馭士人,如軍法之馭卒伍,一有異論,累及學官。若蘇軾、黃庭堅之文,範鎮、沉括之雜說,悉以嚴刑重賞,禁其收藏,其苛錮多士,亦已密矣。而澥猶以為太學之盛,欺罔不已甚乎?原京與澥罪,乃天地否泰所系,國
 家治亂,由之以分,不可忽也。



 仁宗、英宗選敦樸敢言之士以遺子孫,安石目為流俗,一切逐去。司馬光復起而用之,元祐之治,天下安於泰山。及章惇、蔡京倡為紹述之論,以欺人主。紹述一道德,而天下一於諂佞;紹述同風俗,而天下同於欺罔;紹述理財而公私竭;紹述造士而人材衰;紹述開邊而塞塵犯闕矣。元符應詔上書者數千人,京遣腹心考定之,同己為正,異己為邪,澥與京同者也,故列於正。京之術破壞天下,於茲極矣,尚忍使
 其餘蠹再破壞邪?京奸邪之計大類王莽,而朋黨之眾則又過之,願斬之以謝天下。



 累章極論,時議歸重。



 忽得攣疾,不能行。三求去,帝惜之,不許。呂好問、徐秉哲為言,乃以龍圖閣直學士主管嵩山崇福宮,命下而卒。鶠平生為文至多,輒為人取去,篋無留者。尤長於詩,清峭雄深,有法度。無子,婿衛昂集其遺文,為三十卷,傳於世。



 張根,字知常,饒州德興人。少入太學,甫冠,第進士。調臨江司理參軍、遂昌令。當改京秩,以四親在堂,冀以父母
 之恩封大父母,而貤妻封及母,遂致仕,得通直郎,如其志。時年三十一。鄉人之賢者彭汝礪序其事,自以為不及。



 屏處十年,曾布、曾肇、鄒浩及本道使者上其行義,徽宗召詣闕。為帝言:「人主一日萬幾,所恃者是心耳。一累於物,則聰明智慮且耗,賢不肖混洧,綱紀不振矣。願陛下清心省欲,以窒禍亂之原。」遂請罷錢塘制造局。帝改容嘉美,以為親賢宅教授。



 未幾,通判杭州,提舉江西常平。內侍走馬承受舉劾一路以錢半給軍衣非是,自轉
 運使、郡守以下皆罷。根言:「東南軍法與西北殊,此事行之百五十年矣。帥守、監司,分朝廷憂,顧使有罪,猶當審處,豈宜以小奄尺紙空十郡吏哉?」詔皆令復還。又言:「本道去歲蠲租四十萬,而戶部責償如初。祖宗立發運上供額,而給本錢數百萬緡,使廣糴以待用。比希恩者乃獻為羨餘,故歲計不足,至為無名之斂。」詔貸所蠲租,而以糴本錢還之六路。洪州失官錫,系治兵吏千計。根曰:「此有司失於幾察之過也。今羅取無罪之人,責以不可
 得之物,何以召和氣?」乃罷其獄。



 大觀中,入對言:「陛下幸滌煩苛,破朋黨,而士大夫以議論不一,觀望茍且,莫肯自盡。陛下毀石刻,除黨籍,與天下更始,而有司以大臣仇怨,廢錮自如。為治之害,莫大於此,願思所以勵敕之。」即命為轉運副使,改淮南轉運使,加直龍圖閣。上書請:「常平止聽納息,以塞兼並;下戶均出役錢,以絕奸偽,市易惟取凈利,以役商賈。雖名若非正,然與和賈不讎其直什一,而使之倍輸額外無名無數之斂,有間矣。」又請:「
 分舉官為三科:一縣令,二學官,三縣丞曹。州郡亦分三等。明言其人某材堪充某州、某官、某縣令,吏部據以注擬,則令選稍清,視平配硬差遠矣。」詔吏部、戶部相度以聞。根又以水災多,乞蠲租賦,散活口米、常平青苗米,振貸流民。詔褒諭之。



 徙兩浙,辭不行,乃具疏付驛遞奏。大略謂:「今州郡無兼月之儲,太倉無終歲之積,軍須匱乏,邊備缺然。東南水旱、盜賊間作,西、北二國窺伺日久,安得不豫為之計?」因條列茶鹽、常平等利病之數,遂言:「為
 今之計,當節其大者,而莫大於土木之功。今群臣賜一第,或費百萬。臣所部二十州,一歲上供財三十萬緡耳,曾不足給一第之用。以寵元勛盛德,猶慮不稱,況出於閭閻干澤者哉。雖趙普、韓琦佐命定策所未有,願陛下靳之。其次如田園、邸店,雖不若賜第之多,亦願日削而月損之。如金帛好賜之類,亦不可不節也。又其次如錫帶,其直雖數百緡,亦必斂於數百家而後足,今乃下被僕隸,使混淆公卿間,賢不肖無辨。如以其左右趨走,不欲墨綬,當別為制度,以示等威可也。」書奏,權幸側目,
 謀所以中傷之者,言交上,帝察根誠,不之罪也。



 尋以花石綱拘占漕舟,官買一竹至費五十緡,而多入諸臣之家。因力陳其弊,益忤權幸,乃擿根所書奏牘注切草略,為傲慢不恭,責監信州酒。既又言根非詆常平之法,以搖紹述之政,再貶濠州團練副使,安置郴州。尋以討淮賊功,得自便。以朝散大夫終於家,年六十。



 根性至孝,父病蠱戒鹽,根為食淡。母嗜河豚及蟹,母終,根不復食。母方病,每至雞鳴則少蘇,後不忍聞雞聲。子燾,自有傳。弟
 樸。



 樸字見素。第進士。歷耀、淄、宿三州教授、太學錄、升博士,改禮部員外郎。高麗遣子弟入學肄業,又兼博士,遷光祿、太常少卿,擢侍御史。



 鄭居中去位,樸言:「朋黨分攻,非朝廷福,若不揃其尤,久則難圖。」於是宇文黃中、賈安宅等六人皆罷,凡蔡京所惡,亦指為居中黨而逐。時郎員冗濫,至五十五人。徽宗喻樸使論列,乃擿其庸繆者十六人,疏斥諸外。



 徐處仁議置裕民局,以京提舉,京不樂,
 樸言:「國家法令明具,何嘗不裕民乎?今置局非是」,卒罷之。起復修制大樂局管勾官田為大晟府典樂,樸論為「貪濫不法,物論弗齒,且典樂在太常少卿之上,修制冗官不當超逾」,乃罷為樂令。未幾,復前命,樸爭不已,改秘書少監。蔡攸引為道史檢討官,召試中書舍人,卒。



 任諒,字子諒,眉山人,徙汝陽。九歲而孤,舅欲奪母志,諒挽衣泣曰:「豈有為人子不能養其親者乎!」母為感動而止。諒力學自奮,年十四,即冠鄉書。登高第,調河南戶曹。
 以兵書謁樞密曾布,布使人邀詣闕,既見,覺不能合,徑去。布為相,猶欲用之。諒予書,規以李德裕事,布始怒。蔣之奇、章楶在樞府,薦為編修官,布持其奏不下,為懷州教授。徽宗見其所作《新學碑》,曰:「文士也。」擢提舉夔路學事,歷京西、河北、京東,改轉運判官。著《河北根本籍》,凡戶口之升降,官吏之增損,與一歲出納奇贏之數,披籍可見,上之朝。張商英見其書,謂為天下部使者之最。



 提點京東刑獄。梁山濼漁者習為盜,蕩無名籍,諒伍其家,刻
 其舟,非是不得輒入。他縣地錯其間者,鑱石為表。盜發,則督吏名捕,莫敢不盡力,跡無所容。加直秘閣,徙陜西轉運副使。降人李訛哆知邊廩不繼,陰闕地窖粟而叛,遺西夏統軍書,稱定邊可唾手取。諒諜知其謀,亟輸粟定邊及諸城堡,且募人發所窖,得數十萬石。訛哆果入寇,失藏粟,七日而退。他日,復圍觀化堡,而邊儲已足,訛哆遂解去。



 加徽猷閣待制、江淮發運使。蔡京破東南轉般漕運法為直達綱,應募者率游手亡賴,盜用乾沒,漫
 不可核,人莫敢言。諒入對,首論之,京怒。會汴、泗大水,泗州城不沒者兩板。諒親部卒築堤,徙民就高,振以米粟。水退,人獲全,京誣以為漂溺千計,坐削籍歸田里。執政或言:「水災守臣職,發運使何罪?」帝亦知其枉,復右文殿修撰、陜西都轉運使。尋復徽猷閣待制,進直學士。童貫更錢法,必欲鐵錢與銅錢等,物價率十減其九。詔諒與貫議,諒言為六路害,寢其策。加龍圖閣直學士、知京兆府,徙渭州。以母憂去。



 宣和七年,提舉上清寶菉宮、修國
 史。初,朝廷將有事於燕,諒曰:「中國其有憂乎。」乃作書貽宰相曰:「今契丹之勢,其亡昭然,取之當以漸,師出不可無名。宜別立耶律氏之宗,使散為君長,則我有存亡繼絕之義,彼有瓜分輻裂之弱,與鄰崛起之金國,勢相萬也。」至是,又言郭藥師必反。帝不聽,大臣以為病狂,出提舉嵩山崇福宮。是冬,金人舉兵犯燕山,藥師叛降,皆如諒言。乃復起諒為京兆,未幾,卒,年五十八。



 周常,字仲修,建州人。中進士第。以所著《禮·檀弓義》見王
 安石、呂惠卿,二人稱之,補國子直講、太常博士。以養親,求教授揚州。年未五十即致仕。



 久之,御史中丞黃履薦其恬退,起為太常博士,辭。元符初,復申前命,兼崇政殿說書,遷著作佐郎。疏言:「祖宗諸陵器物止用塗金,服飾又無珠玉,蓋務在質素,昭示訓戒。自裕陵至宣仁後寢宮,乃施金珠,願收貯景靈殿,以遵遺訓。」詔置之奉宸庫。擢起居舍人。鄒浩得罪,常於講席論救,貶監郴州酒。徽宗立,召為國子祭酒、起居郎,從容言:「自古求治之主,未
 嘗不以尚志為先。然溺於富貴逸樂,蔽於諂諛順適,則志隨以喪,不可不戒。元祐法度互有得失,人才各有所長,不可偏棄。」



 時以天暑,令記注官卯漏正即勿奏事,仍具為令。常言:「本朝記注類多兼諫員,故凡言動,得以所聞見論可否。神宗皇帝時,修注官雖不兼諫職,亦許以史事於崇政、延和殿直前陳述。陛下於炎暾可畏之候,暫停進對,亦人情之常。若著為定令,則必記於日錄,傳之史筆,使後人觀之,將以為倦於聽納,而忘先帝之美
 意矣。」事遂寢。進中書舍人、禮部侍郎。蔡京用事,不能容,以寶文閣待制出知湖州。尋又奪職,居婺州。復集賢殿修撰。卒,年六十七。



 論曰:徽宗荒於治,嬖幸塞朝,柄移權奸,不鳴者進,習為□熟。鶠、根、諒、常氣節侃侃,指切時敝,能盡言不諱。卒不勝讒舌,根、常死外,鶠、諒甫用而病奪之,可悲也己!金兵既舉,郭藥師已叛,朝廷猶弗知,矧能先見禍機哉,毋惑乎狂諒之言也。



\end{pinyinscope}