\article{列傳第五十}

\begin{pinyinscope}

 吳育宋綬子敏求從子昌言李若谷子淑孫壽朋復圭王博文子疇王鬷



 吳育,字春卿,建安人也。父待問,與楊億同州里,每造億,億厚禮之。門下少年多易之,億曰:「彼他日所享,非若曹
 可望也。」累官光祿卿,以禮部侍郎致仕。



 育少奇穎博學,舉進士,試禮部第一,中甲科。除大理評事,遷寺丞。歷知臨安、諸暨、襄城三縣。自秦悼王葬汝後,子孫從葬,皆出宦官典護。歲時上塚者,往來呼索擾州縣。育在襄城,請凡官所須,具成數,毋容使者妄索,羊豕悉出大官,由是民省供費殆半。宦官過者銜之,或中夜叩縣門,索牛駕車,育拒不應。異時宗子所過,縱鷹犬暴民田,入襄城境,輒相戒約,毋敢縱者。



 舉賢良方正,擢著作郎、直集賢院、
 通判蘇州。還知太常禮院,奏定禮文,名《太常新禮慶歷祀儀》。改右正言,歷三司鹽鐵、戶部二判官。尋以本官供諫職。



 元昊僭號,議出兵討之。群臣曰:「元昊,小醜也,旋即誅滅矣。」育獨建言:「元昊雖稱蕃臣,其尺賦斗租,不入縣官,且服叛不常,請置之,示不足責。且已僭輿服,勢必不能自削,宜援國初江南故事,稍易其名,可以順拊而收之。」不報。復上言:「宜先以文誥告諭之,尚不賓,姑嚴守御,不足同中國叛臣亟加征討。且征討者,貴在神速;守御
 者,利於持重。羌人剽悍多詐,出沒不時,我師乘銳,見小利小勝,必貪功輕進,往往墮賊計中。第嚴約束,明烽候,堅壁清野,以挫其鋒。」時方銳意討之,既而諸將多覆軍者,久之無功,卒封元昊為夏國主,如育所議。



 育又上言:「天下久安,務因循而厭生事,政令紀綱,邊防機要,置不復修。一有邊警,則倉皇莫知所為,殆稍安靜,則又無敢輒言者。若政令修,紀綱肅,財用富,恩信給,賞罰明,將帥練習,士卒精銳,則四夷望風,自無他志。若一不備,則乘
 間而起矣。」



 又曰:「漢通西域諸國,斷匈奴右臂。諸戎內附,雖有桀黠,不敢獨叛。唐太宗嘗賜回鶻可汗並其相手書,納其貢奉,厚以金帛。真宗命潘羅支攻殺李繼遷,而德明乃降。元昊第見朝廷比年與西域諸戎不通朝貢,乃得以利啖鄰境,固其巢穴,無肘腋之患。跳梁猖獗,彼得以肆而不顧矣。請募士諭唃廝囉及他蕃部,離散其黨與,使並力以攻,而均其恩賜,此伐謀之要也。」因錄上真宗時通西域諸蕃事跡。除同修起居注,遂知制誥,進
 翰林學士,累遷禮部郎中。



 契丹與元昊構兵,元昊求納款。契丹使來請勿納元昊,朝廷未知所答。育因上疏曰:「契丹受恩,為日已久。不可納一叛羌,失繼世兄弟之歡。今二蕃自斗,鬥久不解,可觀形勢,乘機立功。萬一過計亟納元昊,臣恐契丹窺兵趙、魏,朝廷不得元昊毫發之助,而太行東西,且有煙塵之警矣。宜使人諭元昊曰:『契丹汝世姻,一旦自絕,力屈而歸我,我所疑也,若無他者,當順契丹如故,然後許汝歸款。』告契丹曰:『已詔元昊,如
 能投謝轅門,即聽內附;若猶堅拒,當為討之。』如此,則彼皆不能歸罪我矣。」於是召兩制,出契丹書,令兩制同上對,不易育議。



 尋知開封府。居數日,發大奸吏一人,流嶺外。又得巨盜,積贓萬九千緡,獄具而輒再變,帝遣他吏按之,卒伏法。時歲饑多盜,育嚴賞功之法,嘗得盜而未賞者,一切賞之,以明不欺。



 慶歷五年,拜右諫議大夫、樞密副使。居數月,改參知政事。山東盜起,帝遣中使按視,還奏:「盜不足慮。兗州杜衍、鄆州富弼,山東人尊愛之,此
 可憂也。」帝欲徙二人於淮南。育曰:「盜誠無足慮者,小人乘時以傾大臣,禍幾不可御矣。」事遂寢。章獻、章懿太后升祔真宗廟,議者請覃恩,且優賜軍士。育曰:「無事而啟僥幸,誰為陛下建此議者,請治之。」已而外人多怨執政者,帝以語輔臣。育曰:「此必建議者欲動搖上聽,臣以身許國,何憚此耶?」



 向綬知永靜軍,為不法,疑通判江中立發其陰事,因構獄以危法中之,中立自經死。綬宰相子,大臣有營助,欲傅輕法。育曰:「不殺綬,無以示天下。」卒減
 死一等,流南方。御史唐詢請罷制科,帝刊其名付中書,育奏疏駁議,帝因諭輔臣曰:「彼上言者,乞從內批行下,今乃知欺罔也。」育曰:「非睿聽昭察,則挾邪蠹國,靡所不為。願出姓名按劾,以明國法。」



 育在政府,遇事敢言,與宰相賈昌朝數爭議上前,左右皆失色。育論辨不已,乃請曰:「臣所辨者,職也;顧力不勝,願罷臣職。」乃復以為樞密副使。明年大旱,御史中丞高若訥曰:「大臣喧爭為不肅,故雨不時若。」遂罷昌朝,而育歸給事中班。未幾,出知許
 州,徙蔡州。設伍保法,以檢制盜賊。時京師有告妖人千數聚確山者,詔遣中使往召捕者十人。至,則以巡檢兵往索之,育曰:「使者欲得妖人還報邪?」曰:「然。」曰:「育在此,雖不敏,聚千人境內,毋容不知。此特鄉民用浮圖法相聚,以利錢財爾,一弓手召之,可致也。今以兵往,人相驚疑,請留毋往。」中使以為然。頃之,召十人者至,械送闕下,皆無罪釋之。而告者伏辜。



 尋以資政殿學士知河南府,徙陜州。上字論詔獄曰:「先王凝旒黈纊,不欲聞見人之過
 失也。設有罪,即屬之有司。楊儀嘗為三司判官,近自御史臺移劾都亭驛,械縛過市,人人不測為何等大獄。及聞案具,乃止請求常事。使道路眾口紛紛竊議,朝廷之士,人皆自危,豈養廉恥、示敦厚之道哉。」



 遷禮部侍郎、知永興軍,召兼翰林侍讀學士。以疾辭,且請便郡。帝語大臣曰:「吳育剛正可用,第嫉惡太過耳。」因命知汝州,遣內侍賜以禁中良藥。會疾不已,又請居散地,以集賢院學士判西京留司御史臺。外臺舊不領民事,時張堯佐知
 河陽,民訟久不決,多詣育訴。育為辨曲直,判書狀尾,堯佐畏懼奉行。復為資政殿學士兼翰林侍讀學士、知陜州,進資政殿大學士。召還,判尚書都省。



 一日,侍讀禁中,帝因語及「臣下毀譽,多出愛憎,卿所當慎也。」育曰:「知而形之言,不若察而行之事。聖主之行,如日月之明。進一人,使人皆知其善,出一人,使人皆曉其惡,則陰邪不能構害,公正可以自立,百王之要道也。」帝數欲大用,為諫官劉元瑜誣奏育在河南嘗貸民出息錢。久之,除宣徽
 南院使、鄜延路經略安撫使、判延州。



 夏人既稱臣,而並邊種落數侵耕為患。龐籍守並州,欲築堡備之。育謂:「要契未明而亟城,則羌人必爭,爭而受患者必麟府也。」移文河東,又遺籍手書及疏於朝,不報。既而夏人果犯河外,陷驍將郭恩,而太原將佐皆得罪去。疾復作,辭不任邊事,求解宣徽使,復以為資政殿大學士、尚書左丞、知河中府,徙河南。病革,視事如平日,因閱囚辨非罪,竄舞文吏二人。已而卒,年五十五。贈吏部尚書,謚正肅。



 育性
 明果,所至作條教,簡疏易行而不可犯。遇事不妄發,發即人不能撓。辨論明白,使人聽之不疑。



 初尹開封,範仲淹在政府,因事與仲淹忤。既而仲淹安撫河東,有奏請,多為任事者所沮,育取可行者固行之。其在二府,待問以列卿奉朝請,育不自安,請罷去,不聽。及出帥永興,時待問尚亡恙,肩輿迎侍,時人榮之。晚年在西臺,與宋庠相唱酬,追裴、白遺事至數百篇。體素羸,少時力學,得心疾。後得古方,和丹砂餌之,大醉,一夕而愈。後數發,每
 發數十日乃已。有集五十卷。弟充,為宰相,自有傳。



 宋綬,字公垂,趙州平棘人。父皋,尚書度支員外郎、直集賢院。綬幼聰警,額有奇骨,為外祖楊徽之所器愛。徽之無子,家藏書悉與綬。綬母亦知書,每躬自訓教,以故博通經史百家,文章為一時所尚。



 初,徽之卒,遺奏補太常寺太祝。年十五,召試中書,真宗愛其文,遷大理評事,聽於秘閣讀書。大中祥符元年,復試學士院,為集賢校理,與父皋同職。後賜同進士出身,遷大理寺丞。及祀汾陰,
 召赴行在,與錢易、陳越、劉筠集所過地志、風物、故實,每舍止即以奏。將祠亳州太清宮,以簽書亳州判官事,入為左正言、同判太常禮院。久之,判三司憑由司。建言:「比歲下赦令釋逋負,後期未報者六十八州。請於諸路選官考核,欺半月以聞。」於是脫械系三千二百人,蠲積負數百萬。



 擢知制誥、判吏部流內銓兼史館修撰、玉清昭應宮判官。累遷戶部郎中、權直學士院,同修《真宗實錄》,進左司郎中,遂為翰林學士兼侍讀學士、勾當三班院。
 始詔讀唐史,固求解三班以顓進講。同修國史,遷中書舍人。昭應宮災,罷二學士。逾年,復翰林學士。史成,遷尚書工部侍郎兼侍讀學士。



 時太后猶稱制,五日一御承明殿,垂簾決事,而仁宗未嘗獨對群臣也。綬奏言:「唐先天中,睿宗為太上皇,五日一受朝,處分軍國重務,除三品以下官,決徒刑。宜約先天制度,令群臣對前殿,非軍國大事,除拜皆前殿取旨。」書上,忤太后意,改龍圖閣學士,出知應天府。太后崩,帝思綬言,召還,將大用,而宰相
 張士遜沮止之,復加翰林侍讀學士。詔定章獻明肅、章懿太后祔廟禮,綬援《春秋》考仲子之宮、唐坤儀廟故事,請別築宮曰奉慈廟以安神主,事多採用。



 始置端明殿學士,以命綬,綬固辭。又言:「帝王御天下,在總攬威柄。而一紀以來,令出簾帷。自陛下躬親萬務,內外延首,思見聖政,宜懲違革弊,以新百姓之耳目。而賞罰號令,未能有過於前日,豈非三事大臣不能推心悉力,以輔陛下之治耶?頃太后朝多吝除拜,而邪幸或徑取升擢,議者
 謂恩出太后。今恩賞雖行,又謂自大臣出,非大臣朋黨罔上,何以得此。朋黨之為朝廷患,古今同之。或窺測帝旨,密令陳奏;或附會己意,以進退人。大官市恩以招權,小人趨利以售進,此風浸長,有蠹邦政。太宗嘗曰:『國家無外憂必有內患。外憂不過邊事,皆可預防;奸邪共濟為內患,深可懼也。』真宗亦曰:『唐朋黨尤盛,王室遂卑。』願陛下思祖宗之訓,念王業艱難,整齊綱紀,正在今日。」張士遜罷,乃拜綬參知政事。



 初,有詔罷修寺觀,而章惠太后
 以舊宅為道觀,諫官、御史言之。帝曰:「此太后奩中物也,諫官、御史欲邀名邪?」綬進曰:「彼豈知太后所為哉,第見興土木違近詔,即論奏之。且事有疑似,彼猶指為過,或陛下有大闕失,近臣雖不言,然傳聞四方,為聖政之累,何可忽也。太祖嘗謂唐太宗為諫官所詆,不以為愧。何若動無過舉,使無得而言哉?」



 郭皇后廢,帝命綬作詔曰:「當求德閥,以稱坤儀。」既而左右引富人陳氏女入宮,綬曰:「陛下乃欲以賤者正位中宮,不亦與前日詔語戾
 乎?」後數日,王曾入對,又論奏之。帝曰:「宋綬亦如此言。」時大臣繼有論者,卒罷之。



 帝春秋富,天下久無事,綬慮宴樂有漸,乃言:「人心逸於久安,而患害生於所忽。故立防於無事,銷變於未萌。事至而應,不亦殆歟?臣願飭勵群司,不以承平自怠。」又上:「馭下之道有三:臨事尚乎守,當機貴乎斷,兆謀先乎密。能守則奸不能移,能斷則邪不能惑,能密則事不能撓。願陛下念之!至若深居燕間,聲味以調六氣,節宣以順四時,保養聖躬,宗社之休也。」再
 遷吏部侍郎。



 時宰相呂夷簡、王曾論議數不同。綬多是夷簡,而參知政事蔡齊間有所異,政事繇此依違不決,於是四人者皆罷。綬以尚書左丞、資政殿學士留侍講筵,權判尚書都省。歲餘,加資政殿大學士,以禮部尚書知河南府。



 元昊反,劉平、石元孫敗沒,帝以手詔賜大臣居外者,詢攻守之策。綬畫十事以獻。復召知樞密院事,遷兵部尚書、參知政事。時綬母尚在,綬既得疾,不視事,猶起居自力,區處後事。尋卒,贈司徒兼侍中,謚宣獻。



 綬
 性孝謹清介,言動有常。為兒童時,手不執錢。家藏書萬餘卷,親自校讎,博通經史百家,其筆札尤精妙。朝廷大議論,多綬所財定。楊億稱其文沉壯淳麗,曰:「吾殆不及也。」及卒,帝多取所書字藏禁中。初,郊祀,綬攝太僕卿。帝問儀物典故,占對辨洽,因上所撰《鹵簿圖》十卷。子敏求。



 敏求字次道,賜進士及第,為館閣校勘。預蘇舜欽進奏院會,出簽書集慶軍判官。王堯臣修《唐書》,以敏求習唐事,奏為編修官。持祖母喪,詔令居家修書。卒喪,同知太
 常禮院。



 石中立薨,子繼死,無他子。其孫祖仁疑所服,下禮官議。敏求謂宜為服三年,當解官,斬衰。同僚援據不一,判寺宋祁是其議,遂定為令。加集賢校理。從宋庠闢,通判西京。為群牧度支判官。墜馬傷足,出知亳州。治平中,召為《仁宗實錄》檢討官,同修起居注、知制誥、判太常寺。



 英宗在殯,有言宗室服疏者可嫁娶,敏求以為大行未發引,不可。逾年,又有言者。敏求言宗室義服,服降而練,可嫁娶矣。坐前後議異,貶秩知絳州。王珪、範鎮乞留
 之,使成《實錄》。神宗曰:「典禮,國之所重,而誤謬如是,安得無責。」然敏求議初不誤,曾公亮惡禮院劉瑾附敏求為說,故因是去之。是歲,即詔還。



 徐國公主以夫兄為侄奏官,敏求疏其亂天倫,執正之。王安石惡呂公著,誣其言韓琦欲因人心,如趙鞅興晉陽之甲,以逐君側之惡,出之穎州。敏求當草制,安石諭旨使明著罪狀,敏求但言敷陳失實。安石怒白於帝,命陳升之改其語,敏求請解職,未聽。



 會李定自秀州判官除御史,敏求封還詞頭,遂
 以本官右諫議大夫奉朝請。策試賢良方正,孔文仲對語切直,擢置優等,安石愈怒,罷文仲。人為敏求懼,帝獨全護之,除史館修撰、集賢院學士。鄧潤甫為帝言:「比群臣多尚告訐,非國家之美,宜登用敦厚之士,以變薄俗。」乃加敏求龍圖閣直學士,命修《兩朝正史》,掌均國公箋奏。元豐二年,卒,年六十一。特贈禮部侍郎。



 敏求家藏書三萬卷,皆略誦習,熟於朝廷典故,士大夫疑議,必就正焉。補唐武宗以下《六世實錄》百四十八卷,它所著書甚
 多,學者多咨之。嘗建言:「河北、陜西、河東舉子,性樸茂,而辭藻不工,故登第者少。請令轉運使擇薦有行藝材武者,特官之,使人材參用,而士有可進之路。又州郡有學舍而無學官,故士輕去鄉里以求師,請置學官。」後頗施行之。族弟昌言。



 昌言字仲謨,以蔭為澤州司理參軍。州有殺人獄,昌言疑其冤,堅請跡捕,果得真犯者。稍遷河陰發運判官。自濟源之官,見道上棄尸若剮剝狀者甚眾,竊嘆郡縣之
 不治。既至河陰,得兇盜六輩,殺人而鬻之,如是十餘年,掩其家,猶得執縛未殺者七人。縣吏與市井少年共為胠橐,昌言窮治其淵藪,皆法外行之,而流其家人。擢都水監丞。



 熙寧初,河決棗強而北。昌言建議,欲於二股河口西岸新灘,立土約障水,使之東流。候稍深,即斷北流,縱出葫盧下流,以除恩、冀、深、瀛水患。詔從之。提舉河渠王亞以為不可成,不如修生堤。朝廷遣翰林學士司馬光往視,如昌言策。不兩月,決口塞。光奏昌言獨有功,若
 與同列均受賞,恐不足以勸。詔理提點刑獄資序,遷開封府推官、同判都水監。汴水漲,昌言請塞訾家口。已而汴流絕,監丞侯叔獻唱為昌言罪,昌言懼,求知陜州。歷濮、冀二州。河決曹村,召判都水監,往護河堤。靈平埽成,轉少府監。卒,贈絹二百匹。



 李若谷,字子淵,徐州豐人。少孤游學,依姻家趙況於洛下,遂葬父母緱氏。舉進士,補長社縣尉。州葺兵營,課民輸木,檄尉受之,而吏以不中程,多退斥,欲苛苦輸者,因
 以取賕;若穀度財,別其長短、大小為程,置庭中,使民自輸。



 改大理寺丞、知宜興縣。官市湖洑茶,歲約戶稅為多少,率取足貧下,若穀始置籍備勾檢。茶惡者舊沒官,若谷使歸之民,許轉貿以償其數。知連州。真宗將朝謁太清宮,選通判亳州。累遷度支員外郎、權三司戶部判官,出為京東轉運使。會河決白馬,調取芻楗,同列盧士倫協三司意,趣刻擾州縣,而若谷寬之。士倫不悅,構於朝,徙知陜州。盜聚青灰山久不散,遣牙吏持榜招諭之,盜
 殺其黨與自歸。改梓州。



 天聖初,判三司戶部勾院。使契丹,陛辭,不俟垂簾請對,乃遽詣長春殿奏事,罷知荊南。士族元甲恃蔭屢犯法,若穀杖之,曰:「吾代若父兄訓之爾。」王蒙正為駐泊都監,挾太后姻橫肆,若谷繩以法。監司右蒙正,奏徙若谷潭州。



 洞庭賊數邀商人船殺人,輒投尸水中。嘗捕獲,以尸無驗,每貸死,隸他州。既而逃歸,復功劫,若谷擒致之,磔於市。自是寇稍息。累遷太常少卿、集賢殿修撰、知滑州。河嚙韓村堤,夜馳往,督兵為大
 埽,至旦堤完。以右諫議大夫知延州。州有東西兩城夾河,秋、夏水溢,岸輒圮,役費不可勝紀。若穀乃制石版為岸,押以巨木,後雖暴水,不復壞。官倉依山而貯穀少,若谷使作露囤,囤可貯二萬斛,他郡多取法焉。遷給事中、知壽州。豪右多分占芍陂,陂皆美田,夏雨溢壞田,輒盜決。若谷摘冒占田者逐之,每決,輒調瀕陂諸豪,使塞堤,盜決乃止。



 加集賢院學士、知江寧府。卒挽舟過境,寒瘠甚者,留養視之,須春溫遣去。民丐於道者,以分隸諸僧
 寺,助給舂爨。還,勾當三班院,進龍圖閣直學士、知河南府。貴人多葬洛陽,敕使須索煩擾,若谷奏令鴻臚預約所調移府,逆為營辦。改樞密直學士、知並州。民貧失婚姻者,若谷出私錢助其嫁娶。贅婿、亡賴委妻去,為立期,不還,許更嫁。並多降人,喜盜竊,籍累犯者,以三人為保,有犯,並坐之,悛者削去籍名。



 進尚書工部侍郎、龍圖閣直學士、知開封府,拜參知政事。建言:「風俗ND惡,在上之人作而新之。君子小人,各有其類,今一目以朋黨,恐正
 人無以自立矣。」帝悟,為下詔諭中外。以耳疾,累上章辭位,罷為資政殿大學士、吏部侍郎、提舉會靈觀事。以太子少傅致仕,卒,年八十。贈太子太傅,謚康靖。



 若谷性資端重,在政府,論議常近寬厚。治民多智慮,愷悌愛人,其去,多見思。少時與韓億為友,及貴顯,婚姻不絕焉。子淑。



 淑字獻臣,年十二,真宗幸亳,獻文行在所。真宗奇之,命賦詩,賜童子出身。試秘書省校書郎,寇準薦之,授校書郎、館閣校勘。



 乾興初,遷大理評事。修《真宗實錄》,為檢
 討官。書成,改光祿寺丞、集賢校理,為國史院編修官。召試,賜進士及第,改秘書郎,進太常丞、直集賢院、同判太常寺,擢史館修撰,再遷尚書禮部員外郎,上時政十議。改知制誥、勾當三班院,為翰林學士,進吏部員外郎。會若穀參知政事,改侍讀學士,加端明殿學士。若谷罷,進本曹郎中,典豫王府章奏。



 以右諫議大夫知許州。歲饑,取民所食五種上之,帝惻然,為蠲其賦。權知開封府,復為翰林學士、中書舍人。言者指其在開封多褻近吏人,改
 給事中、知鄭州。徙河陽,轉尚書禮部侍郎,復為翰林學士。罷端明殿學士,判流內銓,復加端明殿學士。



 初,在鄭州,作《周陵詩》。國子博士陳求古以私隙訟其議訕朝廷,除龍圖閣學士,出知應天府。累表論辨,不報,乃請侍養。明年,復端明、侍讀二學士,判太常寺。父喪免官,終喪起復,再為翰林學士。諫官包拯、吳奎等言淑性奸邪,又嘗請侍養父而不及其母,罷翰林學士,以端明、龍圖閣學士奉朝請。丁母憂,服除,為端明、侍讀二學士。遷戶部侍
 郎,復為翰林學士,而御史中丞張升等又論奏之,不拜,除兼龍圖閣學士。由是臺鬱不得志,出知河中府,暴感風眩,卒。贈尚書右丞。



 淑警慧過人,博習諸書,詳練朝廷典故,凡有沿革,帝多諮訪。制作誥命,為時所稱。其它文多裁取古語,務為奇險,時人不許也。



 初,宋郊有學行,淑恐其先用,因密言曰:「『宋』,國姓;而『郊』者交,非善應也。」又宋祁作《張貴妃制》,故事,妃當冊命,祁疑進告身非是,以淑明典故問之,淑心知其誤,謂祁曰:「君第進,何疑邪?」祁遂
 得罪去,其傾側險陂類此。嘗修《國朝會要》、《三朝訓鑒圖》、《閣門儀制》、《康定行軍賞罰格》,又獻《系訓》三篇,所著別集百餘卷。子壽朋、復圭。



 壽朋字延老。慶歷初,與弟復圭同試學士院,賜進士出身,判吏部南曹。使行諸陵,奏言:「昭憲皇后誕育二聖,為國文母,獨以合葬安陵,不及時祭,請更其禮。」從之。遷群牧判官,擊斷敏甚。皇城卒邏其縱游無度,出知汝州。盡推職田之入歸前守楊畋;畋死,又經理其家。以饑歲營
 州廨勞民,降為荊門軍。



 歷開封府推官、戶部判官、知鳳翔府滄州。滄地震,壞城郭帑庾。壽朋以席為屋,督吏採繕葺,未數月,復其舊。括蕪田三萬頃,縱民耕,擇其壯者使習兵。河方北湧,隨塞之,故道狹,壽朋度必東潰,諭居人徙避,後三縣四鎮果墊焉。司馬光出使。薦其能,加直史館。入直舍人院、同修起居注,進戶部、鹽鐵副使。性疏雋任俠,奉祠西太一宮,飲酒食肉如常時,暴得疾卒。詔中使撫其孥,賜白金三百兩。



 復圭字審言。通判澶州。北使道澶,民主驛率困憊。豪杜氏十八家,詭言唐相如晦後,每賕吏脫免,復圭按籍役之。知滑州。兵匠相忿鬩,揮所執鐵椎,椎殺爭者於廳事,立斬之。徙知相州。



 自太宗時,聚夏人降者五指揮,號「廳子馬」,子弟相承,百年無它役。復圭斥不如格者,選能騎射士補之。為度支判官、知涇州。始時二稅之入,三司移折已重,轉運使又覆折之,復圭為奏免,民立生祠。歷湖北、兩浙、淮南、河東、陜西、成都六轉運使。浙民以給衙前
 役,多破產,復圭悉罷遣歸農,令出錢助長名人承募,民便之。瀕海人賴蛤沙地以生,豪家量受稅於官而占為己有,復圭奏蠲其稅,分以予民。



 熙寧初,進直龍圖閣、知慶州。夏人築壘於其境,不犯漢地。復圭貪邊功,遣大將李信帥兵三千,授信以陳圖,使自荔原堡夜出襲擊,敗還,復圭斬信自解。又欲澡前恥,遣別將破其金湯、白豹、西和市,斬首數千級。後七日,秉常舉國入寇。御史謝景溫劾復圭擅興,致士卒死傷,邊民流離,謫保靜軍節度
 副使。歲餘,知光化軍。張商英言:「夏人謀犯塞之日久矣,與破金湯適相值,非復圭生事。」乃召判吏部流內銓,知曹、蔡、滄州,還為鹽鐵副使,以集賢殿修撰知荊南,卒。



 復圭臨事敏決,稱健吏,與人交不以利害避。然輕率躁急,無威重,喜以語侵人,獨為王安石所知,故既廢即起。



 王博文,字仲明,曹州濟陰人。祖諫,給事太宗藩邸,為西京作坊副使。博文年十六,善屬文,舉進士開封府,以回文詩百篇為公卷,人謂之「王回文」。淳化三年,太宗親試進
 士,以年少罷歸。後諫卒官廬州,州守劉蒙叟為言,召試舍人院,為安豐主簿,歷南豐尉,有能名。調南劍州軍事推官,改大理寺丞,監荊南榷貨務,遷殿中丞。陳堯咨薦之,試中書,賜進士第,擢知濠州,歷真州。真宗幸亳,權江、淮制置司事。改監察御史、梓州路轉運使。以疾,請出知海州,徙密州。負海有鹽場,歲饑,民多盜鬻,吏捕之輒抵死。博文請弛鹽禁,候歲豐乃復,從之。除殿中侍御史。



 天禧中,朱能、王先在長安偽為《乾祐天書》,事覺,能既敗死,
 先與其徒就禽,詔博文乘驛按劾。博文唯治首惡,脅從者七人,得以減論。還為開封府判官,丁母憂。



 始,博文幼喪父,其母張氏改適韓氏。及博文在朝,謂子無絕母禮,請得以恩封之。母死,又謂古之為父後者不為出母服,以廢宗廟祭也。今喪者皆祭,無害於行服。乃請解官持服,然議者以喪而祭為非禮。服除,為三司戶部判官。出為河北轉運使,遷侍御史、陜西轉運使。



 屬羌撒逋渴以族落數千帳叛,既又寇原州柳泉鎮、環州鵓鴿泉砦,
 梧州刺史杜澄、內殿崇班趙世隆戰沒。博文劾奏內侍都知周文質、押班王懷信為涇原、環慶兩路鈐轄,提重兵駐大拔砦,玩寇逗留,耗用邊費,請用曹瑋、田敏代。既而文質、懷信坐法,遂以瑋知永興軍,使節制邊事。會瑋病不行,又用敏為涇原路總管,寇遂平。



 遷尚書兵部員外郎,為三司戶部副使,再遷戶部郎中、龍圖閣待制、判吏部流內銓、權發遣三司使事。與監察御史崔暨、內侍羅崇勛同鞫真定府曹汭獄。及還,權知開封府,進龍圖
 閣直學士、知秦州。為走馬承受賈德昌所毀,徙鳳翔府,又徙永興軍。明年,德昌以贓敗,改樞密直學士,復知秦州。



 初,沿邊軍民之逃者必為熟戶畜牧,又或以遺遠羌易羊馬,故常沒者數百人。其禽生羌,則以錦袍、銀帶、茶絹賞之。間有自歸,而中道為夏人所得,亦不能辨,坐法皆斬。博文乃遣習知邊事者,密持信紙往招,至則悉貸其罪,由是歲減殊死甚眾。朝廷下其法旁路。



 又言河西回鶻多緣互市家秦、隴間,請悉遣出境,戒守臣使譏察
 之。再遷右諫議大夫,以龍圖閣學士復知開封府。都城豪右邸舍侵通衢,博文制表木按籍,命左右判官分撤之,月餘畢。出知大名府,遷給事中。召權三司使,遂同知樞密院事,逾月而卒。帝臨奠,贈尚書吏部侍郎。



 博文以吏事進,多任劇繁,為政務平恕,常語諸子曰:「吾平生決罪,至流刑,未嘗不陰擇善水土處,汝曹志之。」然治曹汭獄,議者多謂博文希太后旨,縱崇勛傅致其罪。子疇。



 疇字景彞,以父蔭補將作監主簿。中進士第,累遷太常
 博士。翰林學士宋祁提舉諸司庫務,薦疇勾當公事。時有宦官同提舉者,疇辭於中書曰:「翰林先進,疇恐不得事也。然以朝士大夫而為閹人指使,則疇實恥之。」



 用賈昌朝薦,改編修《唐書》。仁宗獵近郊,疇引十事以諫。皇祐中,手詔禁貴戚近習私謁者,疇獻《聖政惟公頌》。召試,直秘閣,為開封府推官。宦者李允良訴其叔父死,疑為仇家所毒,請發棺驗視,眾欲許之,疇獨不可。曰:「茍無實,是無故而暴尸,且安知非允良有奸?」窮治,果與其叔父家
 有怨。歷三司度支判官、修起居注、知制誥、權判吏部流內銓,以右諫議大夫權御史中丞。



 時陳升之拜樞密副使,諫官、御史唐介等奏彈升之不當大用,朝廷持不行,介等爭數月不已,乃兩罷之。而論者謂介等為眾人游談所誤。疇疏言:「浮華險薄之徒,往來諫官、御史家,掎摭人罪,浸以成俗,請出詔戒勵。」從之。遷給事中。



 英宗既即位,感疾,皇太后垂簾聽政。其後帝疾平,猶未御正殿,疇上疏請御朝聽政。及永昭陵復土,祭仁宗虞主於集英
 殿,以宗正卿攝事。疇奏曰:「人子之葬其親,送形而往,迎神而返,故虞祭所以安神也。位尊者禮重,禮重者祭多,故天子之虞數至於九。今山陵,嗣君不得親往,則道路五虞,理可命宗正攝事。若神主既至,則四虞之祭,雖或聖躬未寧,亦宜勉強。況陛下在藩邸,以好古知禮、仁孝聰明聞於中外,此先帝所以托天下也。臣願始終令德,以全美名。」



 帝既視朝前後殿,而於聽事猶持謙抑。疇復上疏曰:「廟社擁祐陛下,起居安平,臨朝以時,僅逾半載,
 而未聞開發聽斷,德音遏塞,人情缺然。伏望思太祖、太宗艱難取天下之勞,真宗、仁宗憂勤守太平之力,勉於聽決大政,以慰母後之慈。勿為疑貳謙抑,自使盛德暗然不光。」



 未幾,又上疏曰:



 董仲舒為武帝言天人之際曰:「事在勉強而已。勉強學問,則聞見廣而智益明;勉強行道,則德日起而大有功。」陛下起自列邸,光有天命,然而祖宗基業之重,天人顧享之際,所以操心治身、正家保國者,尤在於勉強力行也。陛下昔在宗藩,已能務德好
 學,語言舉動未嘗越禮,是天性有聖賢之資。自疾平以來,於茲半歲,而臨朝高拱,無所可否。群臣關白軍國之政者日益至,其請人主財決者日益多,然猶聖心盤桓,無所是非者,何也?得非以初繼大統,或慮未究朝廷之事,故謙抑而未皇耶?或者聖躬尚未寧,而不欲自煩耶?抑有所畏忌而不言耶?茍為謙抑而未皇,則國家萬務,日曠月廢,其勢將趨於禍亂無疑也。若聖躬未能寧,則天下之名醫良工,日可召於前。而方技不試,藥石不進,
 養疾於身,坐俟歲月,非求全之道也。茍有所畏忌而不言,則又過計之甚也。



 今中外之事,無可疑畏,臣嘗為陛下力言之矣。陛下何不坦心布誠、廓開大明以照天下,外則與執政大臣講求治體,內則於母後請所未至。延禮賢俊,諮訪忠直,廣所未見,達所未聞。若陛下朝行之,則眾心夕安矣。況陛下向居藩邸,日夕於側者,惟一二講學之師,與左右給使之人耳。修身行己,德業日新,而知者無幾,則是為善多而得名常少也;然而終能德成
 行尊,美名遠聞,此先帝之所以屬心也。今處億兆之上,有一言動則天下知之,簡冊書之,比之於昔,是善行易顯而美名易成也。然而尚莫之聞者,是不為爾,非不能也。有始有終者,聖賢之能事,在陛下勉強而已。



 疇又上疏欲車駕行幸,以安人心。時大臣亦有請,帝乃出禱雨,都人瞻望歡呼。數日,皇太后還政,疇又上疏:「請詔二府大臣講求所以尊崇母後之禮。若朝廷嚴奉之體,與歲時朔望之儀,車服承衛之等威,百司拱擬之制度,它時
 尊稱之美號,外家延賞之恩典,凡可以稱奉親之意者,皆宜優異章大,以發揚母後之功烈,則孝德昭於天下矣。」



 時詔近臣議仁宗配祭。故事,冬、夏至祀昊天上帝、皇地祗,以太祖配;正月上辛祈穀,孟夏雩祀,孟冬祀神州地祇,以太宗配;正月上辛祀感生帝,以宣祖配;季秋大饗明堂、祀昊天上帝,以真宗配。而學士王珪等與禮官上議,以謂季秋大饗,宜以仁宗配,為嚴父之道。知制誥錢公輔獨謂仁宗不當配祭。疇以謂珪等議遺真宗
 不得配,公輔議遺宣祖、真宗、仁宗俱不得配,於禮意未安。乃獻議曰:「請依王珪等議,奉仁宗配饗明堂,以符《大易》配考之說、《孝經》嚴父之禮。奉遷真宗配孟夏雩祀,以仿唐貞觀、顯慶故事。太宗依舊配正月上辛祈穀、孟冬祀神州祗,餘依本朝故事。如此,則列聖並侑;對越昊穹,厚澤流光,垂裕萬祀。必如公輔之議,則陷四聖為失禮,導陛下為不孝,違經戾古,莫此為甚。」自此公輔不悅,而朝廷以疇論事有補,帝與執政大臣皆器異之。



 遷翰林學
 士、尚書禮部侍郎、同提舉諸司庫務。數月,拜樞密副使。於是公輔言疇望輕資淺,在臺素餐,不可大用,又頗薦引近臣可為輔弼者。公輔坐貶。疇在位五十五日,卒。帝甚悼惜之,臨哭,賜白金三千兩,贈兵部尚書,謚忠簡。



 疇名臣子,性介特,厲風操,喜言朝廷事。好治容服,坐立嶷然,言必文,未嘗慢戲,吏治審密,文辭嚴麗。其執政未久、終於位及所享壽,類其父云。



 王鬷字總之,趙州臨城人。七歲喪父,哀毀過人。既長,狀
 貌奇偉。舉進士,授婺州觀察推官。代還,真宗見而異之,特遷秘書省著作佐郎、知祁縣,通判湖州。再遷太常博士、提點梓州路刑獄,權三司戶部判官。使契丹還,判都磨勘司。以尚書度支員外郎兼侍御史知雜事。上言:「方調兵塞決河,而近郡災歉,民力雕敝,請罷土木之不急者。」改三司戶部副使。樞密使曹利用得罪,鬷以同里為利用所厚,出知湖州,徙蘇州。還為三司鹽鐵副使。



 時龍圖閣待制馬季良方用事,建言京師賈人常以賤價居茶
 鹽交引,請官置務收市之。季良挾章獻姻家,眾莫敢迕其意,鬷獨不可,曰:「與民競利,豈國體耶!」擢天章閣待制、判大理寺、提舉在京諸司庫務,安撫淮南,權判吏部流內銓,累遷刑部。



 益、利路旱饑,為安撫使,以左司郎中、樞密直學士知益州。戍卒有夜焚營、殺馬、脅軍校為亂者,鬷潛遣兵環營,下令曰:「不亂者斂手出門,無所問。」於是眾皆出,命軍校指亂者,得十餘人,即戮之。及旦,人莫知也。其為政有大體,不為苛察,蜀人愛之。拜右諫議大夫、同
 知樞密院事。景祐五年,參知政事。明年,遷尚書工部侍郎、知樞密院事。



 天聖中,鬷嘗使河北,過真定,見曹瑋,謂曰:「君異日當柄用,願留意邊防。」鬷曰:「何以教之?」瑋曰:「吾聞趙德明嘗使人以馬榷易漢物,不如意,欲殺之。少子元昊方十餘歲,諫曰:『我戎人,本從事鞍馬,而以資鄰國易不急之物,已非策,又從而斬之,失眾心矣。』德明從之。吾嘗使人覘元昊,狀貌異常,他日必為邊患。」鬷殊未以為然也。比再入樞密,元昊反,帝數問邊事,鬷不能對。及
 西征失利,議刺鄉兵,又久未決。帝怒,鬷與陳執中、張觀同日罷,鬷出知河南府,始嘆瑋之明識。未幾,得暴疾卒。贈戶部尚書,謚忠穆。



 鬷少時,館禮部尚書王化基之門,樞密副使宋湜見而以女妻之。宋氏親族或侮易之,化基曰:「後三十年,鬷富貴矣。」果如所言。



 論曰:吳育剛毅不撓,而設施無聞,其才不逮志者與?宋綬博洽明敏,若穀務長厚,博文習吏事,當仁宗時,先後與政,僅能恭慎寡過,保有祿位,施及後嗣。敏求、淑俱練
 達典故,傅以文採,而淑以傾險敗德,視疇之介特,數建忠謀,則賢不肖之相去遠矣。王鬷不留意曹瑋之言,卒以昧於邊事見黜,宜哉!



\end{pinyinscope}