\article{本紀第三十二}

\begin{pinyinscope}

 高宗九



 三十一年春正月甲戌朔,以日食,不受朝。丁丑,雷。丁亥,免湖州增丁所輸絹。夜,風雷雨雪交作。辛卯,詔江、浙官民戶均輸和市絁帛。壬辰,劉寶落節鉞、福建路居住。丙
 申,大雨雪,給三衙衛士、行在貧民錢及薪炭,命常平振給輔郡細民,諸路監司決獄。己亥,放張浚、胡銓自便。庚子,禁淮南拘籍戶馬。



 二月戊申,復置邛州惠民監。癸丑,以趙密領殿前都指揮使。甲寅,罷楊存中殿前都指揮使,進太傅,為醴泉觀使,封同安郡王。丙辰,置行在會子務。乙丑,復鬻僧道度牒。詔分經義、詩賦為兩科。丙寅,詔通進司承受內降文字,並囊封送三省、樞密院。辛未,秦熹卒,贈太傅。



 三月甲戌朔,命破敵軍統制陳敏部兵屯
 太平州。己卯,官勛臣魏仁浦、馬知節、餘靖、寇瑊諸孫各一人。選文臣宗室主西、南外兩宗司。庚辰,禁兩淮抑民附種。以利州西路御前諸軍都統制吳拱知襄陽府,部兵三千戍之。壬午,以兵部尚書楊椿參知政事。丁亥,奪秦熹贈官及遺表恩賞。庚寅,以陳康伯為尚書左僕射,朱倬右僕射,並同中書門下平章事。辛卯,復李光左中大夫,官其子孫二人。壬辰,地震。庚子,以前徽猷閣待制張宇發死節,贈四官,錄其子孫。



 夏四月丁巳,以久雨傷
 蠶麥,盜賊間發,命侍從、臺諫條上弭災除盜之策。出天申節銀十萬兩加充戶部糴本。辛未,遣周麟之使金賀遷都。壬申,權減荊南上供錢銀絹絲米之半,用招填禁軍。是月,金主亮率文武群臣如汝、洛。



 五月癸酉朔,給兩淮民兵荒田。乙亥,增築禁城。戊寅,詔吳拱視緩急退守荊南。己丑,命沿淮州郡毋納北人。辛卯,金遣高景山、王全來賀天申節。全揚言無禮,致其主亮語,求淮、漢地及指取將相近臣計事,且以欽宗皇帝訃聞。壬辰,選兩浙、
 江東、福建諸州禁軍弓弩手之半,部送樞密院按試。甲午,宰執召同安郡王楊存中及三衙帥趙密等至都堂議舉兵。詔以王全語諭諸路統制、帥守、監司,隨宜應變,毋失機會。是日,為欽宗皇帝發喪,特詔持斬衰三年。乙未,以吳璘為四川宣撫使,仍命制置使王剛中同處置軍事。丙申,命主管馬軍司成閔部兵三萬人戍鄂州。庚子,命兩浙、江、湖、福建諸州起禁軍弓弩手,部送明州、平江府、江、池、太平三州、荊南府軍前。殿中侍御史陳俊卿
 言,內侍張去為竊權撓政,乞斬之以作士氣。



 六月乙巳,以群臣三上表,始聽政。丙午,劉錡乞即日移軍渡江,詔錡進發,騎兵屯揚州。丁未,出宮女三百九十人。蠲臨安府禁軍闕額錢五年。乙酉,以御史中丞汪澈為湖北、京西宣諭使。辛亥,金主亮遣大懷正至盱眙,語送伴使呂廣問云:將以六月遷汴京。令其歸奏。癸丑,罷教坊,並敕令所歸刑部。乙卯,以劉錡為淮南、江東西、浙西制置使。戊午,命帶御器械劉炎同提舉措置沿淮盜賊。庚申,彗
 出角。遣步軍司都統制戚方提總江上諸軍策應軍馬,聽劉錡節制。諭吳拱嚴備襄陽,視緩急,合田師中、成閔兵以援之。甲子,始御正殿。乙丑,放女樂二百餘人。丙寅,聽淮南諸州移治清野。戊辰,以周麟之辭使北,命樞密都承旨徐哲代行。淮北民兵崔唯夫、董臻等率眾萬餘來歸。



 秋七月丙子,命兩浙、江東濱海諸州預備敵兵。詔諸路帥臣教閱土兵、弓手。戊寅,命雷州守臣節制高、容、廉、化四州軍馬。時雷州軍賊凌鐵作亂,東南第十二將
 高居弁會五州巡尉官兵討平之。戊子,周麟之分司、筠州居住。辛卯,振給淮南歸正人。壬辰,徐哲等至盱眙,金主亮以非所指取之人,諭遣亟還。癸巳,詔:「四川財賦,自當專任總領所。如遇警急,調發不及申奏,則令宣、制司隨宜措置,先舉後聞。」乙未,行新造會子於淮、浙、湖北、京西諸州。是月,金主亮徙都汴京,命其臣劉萼由唐、鄧瞰荊、襄,張中彥、王彥章據秦、鳳、窺巴、蜀,蘇保衡、完顏鄭家奴由海道趨兩浙。



 八月辛丑朔,忠義人魏勝復海州,李
 寶承制以勝知州事。丙午,蠲諸路逋欠經總制錢、江浙等路上供米。丁未,以婉容劉氏妄預國政,廢於家。蠲淮南、京西、湖北民秋稅之半。辛亥,以劉婉容事連坐,昭慶軍承宣使王繼先福州居住,停子孫官,籍其貲。甲寅,李寶率舟師三千發江陰,大風,退泊明州關澳,聚兵復進。乙卯,劉錡引兵屯揚州,遣統制王剛以兵五千屯寶應。丁巳,召田師中赴行在。尋以吳拱為鄂州諸軍都統制。壬戌,復用資政殿學士張燾落致仕、知建康府。癸亥,分
 處歸正人於淮南諸州,能自存者從便,願為兵者籍之。乙丑,詔便宜選補戰功人,後勿遞減。丙寅,出內帑錢七萬緡,犒戍兵之家,仍悉除軍債。乙巳,起復成閔為湖北、京西制置使,節制兩路軍馬。



 九月庚午朔,命大臣朝饗太廟。辛未,宗祀徽宗於明堂,以配上帝,大赦。甲戌,金人犯黃牛堡,守將李彥堅拒卻之,金兵遂扼大散關,吳璘駐青野原,遣將高松等援之。庚辰,以給事中黃祖舜同知樞密院事。壬午,流星晝隕。乙酉,詔劉錡、王權、李顯忠、
 戚方嚴備清河、穎河、渦河口。丁亥,成閔渡江,屯應城縣,遣吳拱戍郢州。博州民王友直聚兵大名,自稱河北安撫制置使,以其徒王任為副,遣軍師馮穀入朝奏事。吳璘遣將彭青至寶雞渭河,夜劫金人橋頭砦,破之。庚寅,成閔遣統制趙撙部兵五千駐德安。辛卯,金國趣使臣書至楚州,守臣以聞,其辭多悖慢。壬辰,監盱眙軍淮河渡夏俊復泗州。癸巳,金人犯通化軍,守將張超拒卻之。甲午,冊謚大行皇帝曰恭文順德仁孝皇帝,廟號欽宗。
 吳璘遣將劉海復秦州,金守將蕭濟降。乙未,金人犯信陽軍。丙申,吳璘遣將曹水休復洮州。戊戌,劉錡發揚州。詔以金人背盟,降敕榜招諭中原軍民。己亥,蘭州漢軍千戶王宏殺其刺史溫敦烏乜來降。吳璘遣將彭青復隴州。是月,金主亮以尚書右丞李通為大都督,造浮梁於淮水之上,遂自將來攻,兵號百萬,遠近大震。



 冬十月庚子朔,詔將親征。魏勝攻沂州,敗,還海州,金人圍之。李寶以舟師至東海縣,金人解圍去,寶遂入海州。辛丑,金人
 自渦口渡淮。癸卯,以吳璘兼陜西、河東招討使,劉錡兼京東、河北東路招討使,成閔兼京西、河北西路招討使。金人陷蔣州。李顯忠遣統制孔福與金人戰於大人洲,敗之。乙巳,金人復犯海州,魏勝、李寶擊卻之。劉錡引兵次淮陰,金人將自清河口入淮,錡列兵於運河岸以扼之。丁未,命宣撫制置司傳檄契丹、西夏、高麗、渤海諸國及河北、河東、陜西、京東、河南諸路,諭出師共討金人。是日,金人立其東京留守葛王褒為皇帝,改元大定。戊申,
 王權聞金兵大至,自廬州引兵遁,屯昭關。己酉,知均州武鉅招納北界杜海等二萬人來歸。庚戌,復置機速房。知廬州龔濤聞金兵將至,棄城走。辛亥,金將蕭琦陷滁州,守臣陸廉棄城走。壬子,改建王瑋為鎮南軍節度使。劉錡遣統制王剛等擊敗金人於清河口,金人復來戰,剛失利。吳拱遣將侯俊、郝敦書復唐州。癸丑,借江、浙、荊湖等路坊場凈利錢三百八十萬緡以備賞軍。金人圍廬州,都監、權州事楊春率兵突陣出,守水砦。金人又攻
 海州,李寶力戰敗之,解圍去。甲寅,金人攻樊城,吳拱遣守將翟貴、王進與戰,貴、進俱戰死,金兵亦退。劉錡遣兵渡淮及金人戰,死者十七八。金主亮以大軍至廬州城北之五里,築土城以居。戚方遣將張寶復蔣州。乙卯,以金人渝盟告於天地、宗廟、社稷。命州縣諭富民捐貲助國。劉錡聞王權遁,自淮陰引兵歸揚州。丙辰,金主亮入廬州,王權自昭關遁,金人追至尉子橋,破敵軍統制姚興戰死,權退保和州。金州都統制王彥遣統制任天錫
 出洵陽,復豐陽縣。丁巳,帝聞王權敗,召楊存中同宰執議於內殿,陳康伯贊帝定議親征。武鉅遣將荀琛復鄧州。戊午,任天錫復商洛縣。命吳璘趣出兵漢中,葉義問督視江、淮軍馬,中書舍人虞允文參謀軍事。金人犯真州,步軍司統制邵宏淵逆戰於胥浦橋,兵敗,真州陷。金人不入城,遂犯揚州。己未,任天錫復商州,執其守完顏守能。趙撙引兵渡淮。庚申,以楊存中為御營宿衛使。趙撙復褒信縣。王權自和州遁歸,屯於東採石。辛酉,復湯
 思退觀文殿大學士、充醴泉觀使兼侍讀。分行在官吏三之一扈從,餘留行遣常事。金人陷和州。壬戌,以將士勞於征討,避殿減膳。劉錡退軍瓜州鎮,金人陷揚州,淮東安撫使劉澤棄城奔泰州。以戶部侍郎劉岑為御營隨軍都轉運使,李顯忠為御營先鋒都統制屯蕪湖,主管步軍司李捧為前軍都統制。趙撙復新蔡縣。癸亥,募諸州豪民招槍杖、弓箭手赴行在。金人入揚州。王權自採石夜還建康,



 尋復如採石。甲子,復張浚觀文殿大學
 士、判潭州。吳璘遣統制吳挺、向起等及金人戰於德順軍之治平砦,敗之。趙撙復平興縣。乙丑,金人趨瓜州,劉錡遣統領員琦拒之於皂角林,大敗之,斬其統軍高景山。丙寅,李寶遇金舟師於膠西縣陳家島,大敗之,斬完顏鄭家奴等五人。劉錡還鎮江府。趙撙復蔡州,斬其總管楊寓。分御營宿衛為五軍。金人攻秦州,向起、吳挺擊卻之。丁卯,葉義問至鎮江。詔起江、浙、福建諸州強丁赴江上諸軍。武鉅復虢州盧氏縣,任天錫復朱陽縣。戊辰,
 殿中侍御史杜莘老劾內侍張去為,帝不悅,去為致仕,出莘老知遂寧府。



 十一月己巳朔,邵宏淵遣統領崔皋及金人戰於定山,敗之。任天錫復虢州,守將蕭信遁去。庚午,通州守臣崔邦弼棄城去。辛未,成閔引兵發應城縣,援淮西。遣權吏部侍郎汪應辰詣浙東措置海道。壬申,以張浚判建康府。召王權赴行在,以李顯忠代將。邵宏淵為池州都統制。金人犯瓜州,鎮江中軍統制劉汜戰敗走,權都統制李橫亦遁。金人鐵騎奄至江上,統制
 魏俊、王方死之。葉義問惶怖欲退走,復趨建康。金人游騎至無為軍,守臣韓髦棄城走。癸酉,淮寧府民陳亨祖執同知完顏耶魯,以其城來歸。趙撙引兵去,蔡州復陷。甲戌,池州統制官崔定等復入無為軍。乙亥,金主亮臨江築壇,刑馬祭天,期以翌日南渡。丙子,虞允文督建康諸軍統制官張振、王琪、時俊、戴皋等以舟師拒金主亮於東採石,戰勝,卻之。崔定復巢縣,任天錫復上津、商洛二縣。丁丑,虞允文遣水軍統制盛新以舟師擊金人於
 楊林河口,又敗之。金主亮焚其舟而去。戊寅,王彥遣將楊堅復欒川縣。己卯,以湯思退為行宮留守。虛恨蠻犯嘉州籠蓬堡,官軍大敗,副將鄭祥等為所殺。庚辰,金主亮引軍趨淮東。癸未,吳璘病,自仙人原還興州,留姚仲節制軍事,虞允文自採石率李捧一軍及戈船如鎮江備敵。甲申,贈姚興、魏俊、王方官。金主亮至揚州。乙酉,貸劉汜死、英州編管。江州統制李貴、忠義首領孟俊復順昌府,金州將邢進復華州。丙戌,賜戰士帛,給其家薪炭。
 任天錫復陜州。丁亥,劉錡以疾罷,以御營宿衛中軍統制劉銳權鎮江都統制。成閔自京西還建康,遂如鎮江。戊子,吳璘復力疾上仙人原。己丑,王權貸死、瓊州編管。李寶泛海南歸。金人復攻陜州,任天錫破走之。復犯襄陽,統制官李勝等拒卻之,復通化軍。王彥遣將楊堅、黨清至西京長水縣及金人戰,敗之。庚寅,復長水縣。癸巳,以成閔為鎮江都統制、淮東制置使、京東西路河北東路淮北泗宿州招討使,李顯忠為淮西制置使、京畿河
 北西路淮北壽亳州招討使,吳拱為湖北京西制置使、京西北路招討使。甲午,武鉅遣鄉兵總轄杜隱等復嵩州。乙未,金人陷泰州。是日,金人弒其主亮於揚州龜山寺。戊戌,金都督府遣人持檄詣鎮江軍中議和。



 十二月己亥朔,趙撙夜襲蔡州,復入其城。王彥遣兵復福昌縣。庚子,楊存中及虞允文渡江至瓜州察金兵。金人犯漢南之茨湖,鄂州軍士史俊登其舟,獲一將,諸軍繼進,遂擊卻之。楊椿夜攻金人,殺其帥高定山,復廬州。辛丑,以
 李寶為靖海軍節度使、浙西通泰海州沿海制置使、京東東路招討使。金統軍劉萼聞茨湖敗,亦退師。王彥遣將閻□巳復澠池縣。壬寅,天有白氣。以趙密為行宮在城都總管。成閔渡江之揚州。癸卯,命諸路招討司率兵進討,互相應援,沿江諸大帥條陳恢復事宜。復岳州舊名。右軍統領沙世堅入泰州。甲辰,虞允文自鎮江入見。均州統領昝朝復鄧州。乙巳,張浚至慈湖,命李顯忠引兵渡江。丙午,淮東統制王選復楚州。丁未,杜隱等入河南
 府。吳拱遣統制牛宏入汝州。戊申,帝發臨安,建王從行。庚戌,金人渡淮北去。壬子,次平江。罷督視府。虞允文還至鎮江。癸丑,淮東統制劉銳、陳敏引兵入泗州。鄂州統制楊欽以舟師追敗金人於洪澤鎮。乙卯,江北金兵盡去,李顯忠復入和州。吳璘遣將復水洛城。金人復破汝州,牛宏敗走。戊午,次鎮江府。庚申,吳璘遣將拔金人治平砦。壬戌,曲赦新復州軍。甲子,降淮南、京西、湖北雜犯死罪以下囚。賞採石功,進統制張振、時俊等官。金穎、壽
 二州巡檢高顯以壽春府來降。丁卯,命諸道籍鄉兵。初,王友直、王任聚兵,嘗命友直為天雄軍節度使,任為天平軍節度使。金主褒既立,下令散其眾,友直等自壽春來歸。是月,金主知亮已死,遂趨燕京。



 三十二年春正月戊辰朔,日有食之。帝在鎮江。己巳,金人犯壽春府,忠義將劉泰戰死,金兵引去。庚午,發鎮江府。壬申,至建康府,張浚入見。丙子,祧翼祖主於夾室。己卯,李顯忠引兵還建康。庚辰,罷郡守年七十者。壬午,金
 人復犯蔡州,趙撙力戰卻之。乙酉,權知東平府耿京遣其將賈瑞、掌書記辛棄疾來奏事。己丑,金主遣其臣高忠建等來告嗣位。以耿京為天平軍節度使、知東平府。庚寅,詔新復州縣搜訪仗節死義之士。丙申,以楊存中為江、淮、荊、襄路宣撫使,虞允文副之。給事中金安節、中書舍人劉珙繳奏再上,乃改命存中措置兩淮。



 二月戊戌朔,罷借兩浙、江、淮坊場凈利錢。以虞允文為兵部尚書、川陜宣諭使,措置招軍市馬及與吳璘議事。庚子,興州
 統領惠逢等復河州。振兩淮饑民。壬寅,金人犯汝州,守將王宣逆戰,敗之。癸卯,帝發建康。惠逢復積石軍,又克來羌城。丁未,劉錡薨。己酉,王宣及金人再戰於汝州。庚戌,金人全師來攻,宣敗績,棄去。辛亥,金人復犯順昌府,孟新拒卻之,尋亦棄去。壬子,賞蔡州功,趙撙等進官有差。乙卯,至臨安府。興元都統制姚仲攻鞏州不下,退守甘谷城,遂引兵圍德順軍。丙辰,金人犯蔡州。趙撙擊卻之。戊午,復引兵來攻,撙又敗之,金兵遁去。王彥遣將馬
 貴斷河中南橋,金兵來攻,貴戰敗之。壬戌,詔軍士戰死者祿其家一年,傷重而死於營者半之。乙丑,王宣及右軍副將汲靖敗金人於蔡州確山縣。趙撙棄蔡州。丙寅,金人復取之。姚仲遣副將趙銓攻下鎮戎軍,金同知渭州秦弼及其子嵩來歸。王彥遣兵救陜州,遇金人於虢州東,敗之,金兵引去。丁卯,吳珙遣將復永安軍、永寧、福昌、長水三縣。



 閏月癸酉,金人破河州,屠其城。乙亥,命楊存中、李顯忠固守新復州軍,量度進討。丙子,姚仲遣將
 復原州。戊寅,祔欽宗主於太廟。癸未,振淮南歸正人。金人犯虢州。吳璘遣楊從儀等攻拔大散關,分兵據和尚原,金人走寶雞。丙戌,給張浚錢十九萬緡,造沿江諸軍戰艦。庚寅,王剛破金人於海州。辛卯,楊椿罷。壬辰,姚仲攻德順軍,敗金人於瓦亭砦、新店。是月,張安國等攻殺耿京,李寶將王世隆攻破安國,執之以獻。



 三月壬寅,更定金使入境接伴、館伴舊儀。癸卯,成閔遣統制杜彥救淮寧,擊敗金人於項城縣。甲辰,罷扈從官吏賞典。乙巳,
 錄商、虢之功,加吳璘少傅、王彥為保平軍節度使。戊申,吳璘復德順軍,又遣將嚴忠取環州。辛亥,命兵部侍郎陳俊卿、工部侍郎許尹經畫兩淮堡砦屯田。癸丑,金人圍淮寧府,守臣陳亨祖死之。甲寅,吳璘自德順軍復還河池。金人犯鎮戎軍。丁巳,遣洪邁等賀金主即位。戊午,忠義軍統制、知蘭州王宏拔會州。金人陷淮寧府,統領戴規戰死。成閔歸自淮東。辛酉,金人攻原州。丙寅,詔舉賢良。



 夏四月丁卯朔,姚仲遣兵救原州。己巳,使侍從、臺
 諫條上防秋足食足民策。遣左武大夫都飛虎結約河東。壬申,賞御營宿衛將士四萬餘人進官有差。癸酉,蠲淮東殘破州軍上供銀絹、米麥及經、總制錢一年。蒙城縣民倪震率丁口數千來歸。甲戌,募民耕淮東荒田,蠲其徭役及租稅七年。戊寅,以御史中丞汪澈參知政事。金人圍海州。戊子,洪邁等辭行,報聘書用敵國禮。是月,大雨,淮水暴溢數百里,漂沒廬舍,人畜死者甚眾。



 五月戊戌,吳璘自河池如鳳翔巡邊,姚仲遣兵救原州,數敗金
 人。庚子,復置提舉秦州買馬監,命四川總領官兼權其職。壬寅,姚仲及金人戰於原州北嶺,敗績。戊申,復以楊存中為醴泉觀使,奉朝請。罷御營宿衛司。辛亥,鎮江都統制張子蓋救海州,遇金人於石湫堰,大敗之,金人解去。甲寅,命張浚專一措置兩淮事務兼節制淮東西、沿江州郡軍馬。乙卯,知順昌軍孟昭率部曲來歸。己未,吳璘遣將復熙州。壬戌,禁諸軍互招逃亡。加鄭藻太尉。振東北流民。命張浚置御前萬弩營,募淮民為之。甲子,詔
 立建王瑋為皇太子,更名慎。加成閔太尉、主管殿前司,李顯忠為太尉、主管馬軍司。籍諸州歸正人,願為農者給官田,復租十年;願為兵者赴軍中。



 六月丙寅朔,吳璘次大幽嶺,檄召姚仲至軍前,下河池獄,命夔路安撫使李師顏代將其兵。戊辰,名新宮曰德壽。庚午,以吳珙主管步軍司。罷三招討司。甲戌,加贈兄子為太師、中書令,追封秀王,謚安僖;妻張氏封王夫人。乙亥,朱倬罷。丙子,詔皇太子即皇帝位。帝稱太上皇帝,退處德壽宮,皇
 后稱太上皇后。孝宗即位,累上尊號曰光堯壽聖憲天體道性仁誠德經武緯文紹業興統明謨盛烈太上皇帝。淳熙十四年十月乙亥,崩於德壽殿,年八十一。謚曰聖神武文憲孝皇帝,廟號高宗。十六年三月丙寅,攢於會稽之永思陵。光宗紹熙二年,加謚受命中興全功至德聖神武文昭仁憲孝皇帝。



 贊曰:昔夏后氏傳五世而後羿篡,少康復立而祀夏;周傳九世而厲王死於彘,宣王復立而繼周;漢傳十有一
 世而新莽竊位,光武復立而興漢;晉傳四世有懷、愍之禍,元帝正位於建鄴;唐傳六世有安、史之難,肅宗即位於靈武;宋傳九世而徽、欽陷於金,高宗纘圖於南京:六君者,史皆稱為中興,而有異同焉。夏經羿、浞,周歷共和,漢間新室、更始,晉、唐、宋則歲月相續者也。蕭王、瑯琊皆出疏屬,少康、宣王、肅宗、高宗則父子相承者也。至於克復舊物,則晉元與宋高宗視四君者有餘責焉。高宗恭儉仁厚,以之繼體守文則有餘,以之撥亂反正則非其
 才也。況時危勢逼,兵弱財匱,而事之難處又有甚於數君者乎?君子於此,蓋亦有憫高宗之心,而重傷其所遭之不幸也。然當其初立,因四方勤王之師,內相李綱,外任宗澤,天下之事宜無不可為者。顧乃播遷窮僻,重以苗、劉群盜之亂,權宜立國,確虖艱哉。其始惑於汪、黃,其終制於奸檜,恬墮猥懦,坐失事機。甚而趙鼎、張浚相繼竄斥,岳飛父子竟死於大功垂成之秋。一時有志之士,為之扼腕切齒。帝方偷安忍恥,匿怨忘親,卒不免於來
 世之誚,悲夫!



\end{pinyinscope}