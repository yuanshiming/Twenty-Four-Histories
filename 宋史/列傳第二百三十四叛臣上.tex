\article{列傳第二百三十四叛臣上}

\begin{pinyinscope}

 宋失其政,金人乘之,
 俘其人民,遷其寶器,效遼故事,立其臣為君,冠屨易位,莫甚斯時。高宗南渡,國勢弗振,悍僕狂奴,欺主衰敗,易動於惡。兵雖凶器,尤忌殘忍,將用忍人,先無仁心,視背君親猶反掌耳。世將之子使握重兵,居之阨塞之地,豈
 非召亂之道乎?大義昭明,旋踵殄滅,蓋天道也。扶綱常,遏亂略,作《叛臣傳》。



 張邦昌,字子能,永靜軍東光人也。舉進士,累官大司成,以訓導失職,貶提舉崇福宮,知光、汝二州。政和末,由知洪州改禮部侍郎。首請取崇寧、大觀以來瑞應尤殊者,
 增製旗物,從之。宣和元年,除尚書右丞,轉左丞,遷中書侍郎。欽宗即位,拜少宰。



 金人犯京師,朝廷議割三鎮,俾康王及邦昌為質于金以求成。會姚平仲夜斫金人營,斡離不怒責邦昌,邦昌對以非出朝廷意。俄進太宰兼門下侍郎。既而康王還,金人復質肅王以行,仍命邦昌為河北路割地使。



 初,邦昌力主和議,不意身自為質,及行,乃要欽宗署御批無變割地議,不許;又請以璽書付河北,亦不許。時粘罕兵又來侵,上書者攻邦昌私敵,社
 稷之賊也。遂黜邦昌為觀文殿大學士、中太一宮使,罷割地議。其冬,金人陷京師,帝再出郊,留青城。



 明年春,吳𠦅、莫儔自金營持文書來,令推異姓堪為人主者從軍前備禮冊命。留守孫傅等不奉命,表請立趙氏。金人怒,復遣𠦅、儔促之,劫傅等召百官雜議。衆莫敢出聲,相視久之,計無所出,乃曰:「今日當勉強應命,舉在軍前者一人。」適尚書員外郎宋齊愈至自外,衆問金人意所主,齊愈書「張邦昌」三字示之,遂定議,以邦昌治國事。孫傅、張
 叔夜不署狀,金人執之置軍中。



 王時雍時為留守,再集百官詣秘書省,至即閉省門,以兵環之,俾范瓊諭衆以立邦昌,衆意唯唯。有太學生難之,瓊恐沮衆,厲聲折之,遣歸學舍。時雍先署狀,以率百官。御史中丞秦檜不書,抗言請立趙氏宗室,且言邦昌當上皇時,專事宴遊,黨附權姦,蠹國亂政,社稷傾危實由邦昌。金人怒,執檜。𠦅、儔持狀赴軍前。



 邦昌入居尚書省,金人趣勸進,邦昌始欲引決,或曰:「相公不前死城外,今欲塗炭一城耶?」適金
 人奉冊寶至,邦昌北向拜舞受冊,即偽位,僭號大楚,擬都金陵。遂升文德殿,設位御床西受賀,遣閤門傳令勿拜,時雍率百官遽拜,邦昌但東面拱立。



 外統制官、宣贊舍人吳革恥屈節異姓,首率內親事官數百人,皆先殺其妻孥,焚所居,謀舉義金水門外。范瓊詐與合謀,令悉棄兵仗,乃從後襲殺百餘人,捕革並其子皆殺之,又擒斬十餘人。



 是日,風霾,日暈無光。百官慘沮,邦昌亦變色。唯時雍、𠦅、儔、瓊等欣然鼓舞,若以為有佐命功云。即以
 時雍權知樞密院事領尚書省,𠦅權同知樞密院事,儔權簽書樞密院事,呂好問權領門下省,徐秉哲權領中書省。下令曰:「比緣朝廷多故,百官有司皆失其職。自今各遵法度,御史臺覺察以聞。」見百官稱「予」,手詔曰「手書」。獨時雍每言事邦昌前,輒稱「臣啟陛下」,邦昌斥之;勸邦昌坐紫宸、垂拱殿,呂好問爭之,乃止。邦昌以嗣位之初,宜推恩四方,以道阻,先赦京城,選郎官為四方密諭使。



 金人將退師,邦昌詣金營祖別,服柘袍,張紅蓋,所過設
 香案,起居悉如常儀,時雍、秉哲、𠦅、儔皆從行,士庶觀者無不感愴。二帝北遷,邦昌率百官遙辭于南薰門,衆慟哭有仆絕者。



 金師既還,邦昌降手書赦天下。呂好問謂邦昌曰:「人情歸公者、劫于金人之威耳,金人既去,能復有今日乎?康王居外久,衆所歸心,曷不推戴之?」又謂曰:「為今計者,當迎元祐皇后,請康王早正大位,庶獲保全。」監察御史馬伸亦請奉迎康王。邦昌從之。王時雍曰:「夫騎虎者勢不得下,所宜熟慮,他日噬臍,悔無及已。」徐秉
 哲從旁贊之,邦昌弗聽,乃冊元祐皇后曰「宋太后」,入御延福宮。遣蔣師愈齎書于康王自陳:「所以勉循金人推戴者,欲權宜一時以紓國難也,敢有他乎?」王詢師愈等,具知所由,乃報書邦昌。邦昌尋遣謝克家獻大宋受命寶,復降手書請元祐皇后垂簾聽政,以俟復辟。書既下,中外大說。太后始御內東門小殿,垂簾聽政。邦昌以太宰退處內東門資善堂。尋遣使奉乘輿服御物至南京,既而邦昌亦至,伏地慟哭請死,王撫慰之。



 王即皇帝位,
 相李綱,徙邦昌太保、奉國軍節度使,封同安郡王。綱上書極論:「邦昌久典機政,擢冠宰司。國破而資之以為利,君辱而攘之以為榮。異姓建邦四十餘日,逮金人之既退,方降赦以收恩。是宜肆諸市朝,以為亂臣賊子之戒。」時黃潛善猶左右之。綱又力言:「邦昌已僭逆,豈可留之朝廷,使道路目為故天子哉?」高宗乃降御批曰:「邦昌僭逆,理合誅夷,原其初心,出於迫脅,可特與免貸,責授昭化軍節度副使、潭州安置。」



 初,邦昌僭居內庭,華國靖恭夫
 人李氏數以果實奉邦昌,邦昌亦厚答之。一夕,邦昌被酒,李氏擁之曰:「大家,事已至此,尚何言?」因以赭色半臂加邦昌身,掖入福寧殿,夜飾養女陳氏以進。及邦昌還東府,李氏私送之,語斥乘輿。帝聞,下李氏獄,詞服。詔數邦昌罪,賜死潭州,李氏杖脊配車營務。時雍、秉哲、𠦅、儔等先已遠竄,至是,並誅時雍。



 劉豫,字彥游,景州阜城人也。世業農,至豫始舉進士,元符中登第。豫少時無行,嘗盜同舍生白金盂、紗衣。政和
 二年,召拜殿中侍御史,為言者所擊,帝不欲發其宿醜,詔勿問。未幾,豫累章言禮制局事,帝曰:「劉豫河北種田叟,安識禮制?」黜豫兩浙察訪。宣和六年,判國子監,除河北提刑。



 金人南侵,豫棄官避亂儀真。豫善中書侍郎張愨,建炎二年正月,用愨薦除知濟南府。時盜起山東,豫不願行,請易東南一郡,執政惡之,不許,豫忿而去。是冬,金人攻濟南,豫遣子麟出戰,敵縱兵圍之數重,郡倅張柬益兵來援,金人乃解去。因遣人啖豫以利,豫懲前忿,
 遂畜反謀,殺其將關勝,率百姓降金,百姓不從,豫縋城納款。三年三月,兀朮聞高宗渡江,乃徙豫知東平府,充京東西、淮南等路安撫使,節制大名開德府、濮濱博棣德滄等州,以麟知濟南府,界舊河以南,俾豫統之。



 四年七月丁卯,金人遣大同尹高慶裔、知制誥韓昉册豫為皇帝,國號大齊,都大名府。先是,北京順豫門生瑞禾,濟南漁者得鳣,豫以為己受命之符,遣麟持重寶賂金左監軍撻辣求僭號。撻辣許之,遣使即豫所部咨軍民所宜
 立,衆未及對,豫鄉人張浹越次請立豫,議遂決,乃命慶裔、昉備璽綬寳冊以立之。九月戊申,豫即偽位,赦境內,奉金正朔,稱天會八年。以張孝純為丞相,李孝揚為左丞,張柬為右丞,李儔為監察御史,鄭億年為工部侍郎,王瓊為汴京留守,子麟為太中大夫、提領諸路兵馬兼知濟南府。孝純始堅守太原,頗懷忠義,高宗以王衣雅厚孝純,俾衣招之,會粘罕遣人自雲中送歸豫,遂失節於賊。



 豫還東平,升為東京。改東京為汴京,降南京為歸
 德府。以弟益為北京留守,尋改汴京留守。復降淮寧、潁昌、順昌、興仁府悉為州。自以生景州,守濟南,節制東平,僭位大名,乃起四郡丁壯數千人,號「雲從子弟」。下偽詔求直言。十月,冊其母翟氏為皇太后,妾錢氏為皇后。錢氏,宣和內人也,習宮掖事,豫欲有所取則,故立之。十一月,改明年元阜昌。



 方豫未僭號時,數遣人說東京副留守上官悟,及賂悟左右喬思恭與共說悟令降金,悟並斬之。又招知楚州趙立,立不發書,斬其使。復遣立友人
 劉偲以榜旗誘之,且曰:「吾君之故人也。」立曰:「我知有君父,不知有故人。」燒殺偲。博州判官劉長孺以書勸豫反正,豫囚之十旬,不屈;欲官之,不受。豫大索宋宗室,承務郎閻琦匿之,豫杖死琦。召迪功郎王寵,不至。文林郎李喆、尉氏令姚邦基皆棄官去。朝奉郎趙俊書甲子不書僭年,豫亦無如之何。洪皓久陷於金,粘罕勸皓仕豫,不從,竄皓冷山。處士尹惇聞豫召,逃山谷間,走蜀中。國信副使宋汝為以呂頤浩書勉豫忠義,豫曰:「獨不見張邦
 昌乎?業已然,尚何言哉!」滄州進士邢希載上豫書乞通宋朝,豫殺希載。



 是月,豫立陳東、歐陽澈廟於歸德,如唐張巡、許遠雙廟制。



 紹興元年五月,張俊討李成,敗之,成逃歸豫。雄州大儈王友直嘗抵豫書招李成,謂劉光世、呂頤浩非中興將相才,後為人所訴,詔鞫而刑之。六月,豫以麟為兵馬大總管、尚書左丞相。置招受司於宿州,誘宋逋逃。金人既立豫,以舊河為界,恐兩河民之陷沒者逃歸,下令大索,或轉鬻諸國,或繫送雲中,實防豫也。
 十月,豫入寇,遣其將王世沖以蕃、漢兵攻廬州,守臣王亨誘斬世沖,大敗其衆。十一月,帥臣葉夢得招降豫將王才。偽秦鳳帥郭振入寇,王彥、關師古敗之。偽知海州薛安靖及通判李匯以州來歸。



 二年二月,知商州董先以商、虢二州叛附於豫。襄陽鎮撫使桑仲上疏請正豫罪。朝廷尋命仲兼節制應援京城軍馬,量度事勢,復豫所陷郡。仍命河南翟興、荊南解潛、金房王彥、德安陳規、蘄黃孔彥舟、廬壽王亨相為應援,毋失事機。三月,仲為
 其將霍明所殺,高宗聞之,授仲二子將仕郎。河南鎮撫使翟興屯伊陽山,豫患之,使人招興,許以王爵。興焚偽詔並戮其使。豫乃陰結興麾下楊偉圖之。偉殺興,持興首降豫。



 四月丙寅,豫遷都汴。因奉祖考于宋太廟,尊其祖曰徽祖毅文皇帝,父為衍祖睿仁皇帝。親巡郊社。是日,暴風卷旗,屋瓦皆震,士民大恐。豫曲赦汴人,與民約曰:「自今不肆赦,不用宦官,不度僧道。文武雜用,不限資格。」時河、淮、陝西、山東皆駐北軍,麟籍鄉兵十餘萬為皇
 子府十三軍。分置河南、汴京淘沙官,兩京塚墓發掘殆盡。賦斂煩苛,民不聊生。



 五月,豫聞桑仲死,遣人招隨州李道、鄧州李橫,皆不受,執其使以聞。六月,蘄黃鎮撫使孔彥舟叛降豫,其將陳彥明率衆千餘來歸。直徽猷閣淩唐佐、尚書郎李亘、國信副使宋汝為留偽庭,久謀疏豫虛實蠟書以聞,事泄,豫殺唐佐,亘亦遇害。豫以知東平府李鄴為尚書右丞,河南鎮撫司都統制董先為大總管府先鋒將。十二月,襄陽鎮撫使李橫敗豫兵于揚石,
 乘勝趣汝州,偽守彭玘以城降。豫遣劉夔與金帥撒離曷侵蜀。執進士薛筇送豫,筇勉豫:「早圖反正,庶或全宗,孰與他日並妻子磔東市?」豫怒,欲兵之,賴張孝純獲免。



 三年正月庚申,李橫破潁順軍,偽守蘭和降。壬戌,敗豫兵于長葛。甲子,橫引兵至潁昌府,偽安撫趙弼固守,急攻下之,弼遁,復潁昌。二月,河南鎮撫司統制官李吉敗豫將梁進于伊陽臺,殪之。三月,豫聞橫入潁昌,求援于金人。粘罕遣兀朮赴之,豫亦遣將李成率師二萬逆戰
 於京城西北之牟駝岡。橫敗績,復陷潁昌。橫軍本群盜,恃勇無律,勝則爭取子女金帛,故及於敗。四月,陷虢州。鎮撫司統制官謝皐指腹示賊曰:「此吾赤心也!」自剖心以死。皐,開封人。是月,明州守將徐文以所部海舟六十艘、官軍四千餘人浮海抵鹽城,輸款於豫。文言沿海無備,二浙可襲取。豫大喜,以文知萊州,益海艦二十,俾寇通、泰間。



 五月,朝廷遣韓肖胄、胡松年使偽齊。豫欲以臣禮見,肖胄無以應,松年曰:「均為宋臣。」遂長揖不拜,豫不
 能屈。因問主上如何,松年曰:「聖主萬壽。」復問帝意所向,松年曰:「必欲復故疆耳。」豫有慚色。



 時豫悉有梁、衛之地,翟琮屯伊陽之鳳牛山,不能孤立,突圍奔襄陽。九月,楊政遣川陝將官吳勝破豫兵於蓮花城。十月己亥,賊將李成陷鄧州,以齊安守之;癸卯,陷襄陽,李橫奔荊南,知隨州李道棄城走。成據襄陽,以王嵩知隨州。甲辰,陷郢州,守臣李簡遁,豫以荊超知州事。賊將王彥先自亳引兵至壽春,將窺江南。劉光世駐軍建康,扼馬家渡,遣酈
 瓊領所部駐無為軍,為濠、壽聲援,賊乃還。



 十二月,金人遣李永壽、王翊來報聘。永壽等驕倨,請還豫俘及西北士民之流寓者,復要畫江以益豫。監廣州鹽稅吳伸上書請討豫,謂「金人雖強,實不足慮,賊豫雖微,實為可憂。今敵使在廷,宜陽許而陰圖之,乘其不疑,可一戰擒也。」



 四年正月,翰林學士綦崇禮言:「豫父子倚重金人,且永壽等從豫所來,畫江之請必出於豫。觀其姦謀,在窺吾境土。恐既通使,人情必解弛,宜戒將帥愈益置守。縱和
 議成,亦未可馳備。」既而朝廷遣章誼使金,至雲中。粘罕答書約毋駐軍淮南,誼不屈,還過汴,豫欲留之,以計獲免。熙河路馬步軍總管關師古與豫兵戰于左要嶺,敗績,遂降賊。洮、岷之地盡歸豫矣。



 二月,豫策進士。五月,知壽春府羅興叛降豫。舒、蘄等州制置使岳飛復襄陽,李成遁,尋復唐州。六月,復隨州,磔偽守王嵩於襄陽市。七月,復鄧州,語在《飛傳》。豫聞岳飛取襄、鄧,遂乞師于金人。偽奉議郎羅誘上南征策,豫大喜。奪民舟五百載戰具,
 以徐文為前軍,聲言攻定海。九月,豫下偽詔,有「混一六合」之言,遣子麟入寇,及誘金人宗輔、撻辣、兀朮分道南侵,步兵自楚、承進,騎兵由泗趨滁。復遣偽知樞密院盧緯請師于金主,金主集諸將議,粘罕、希尹難之,獨宗輔以為可。乃以宗輔權左副元帥,撻辣權右副元帥,調渤海漢軍五萬應豫。以兀朮嘗渡江,習知險易,俾將前軍。豫以麟領東南道行臺尚書令。朝廷震恐。或勸帝他幸,趙鼎曰:「戰而不捷,去未晚也。」張俊曰:「避將安之?」遂決意
 親征。壬申,豫兵與金人分道渡淮,楚州守臣樊序棄城走,淮東宣撫使韓世忠自承州退保鎮江。



 十月丙子朔,詔張俊援世忠,劉光世移軍建康。世忠復還揚州。起張浚為侍讀。戊子,韓世忠戰於大儀,己丑,解元戰於承州,皆捷。丙申,豫露榜有窺江之言。戊戌,帝發臨安。十一月壬子,下詔討豫,始暴豫罪惡,士氣大振,欲濟江決戰。趙鼎曰:「退固不可,渡江亦非策。豫猶不親來,至尊豈可與逆雛決勝負哉?」淮西將王師晟、張琦合兵復南壽春府,
 執偽知州王靖。十二月壬辰,岳飛遣將牛臯、徐慶敗金人於廬州。庚子,金人退師,遣使告麟,麟棄輜重宵遁,語在《世忠傳》。



 五年正月,淮西將酈瓊復光州,偽守許約降。閏二月,豫將商元攻信陽軍,知軍事舒繼明死之。七月,豫廢明堂為講武殿,暴風連日。八月,陷光州。十月,豫令民鬻子依商稅法許貫陌而收其算。豫獻《海道圖》及戰船木樣于金主亶。



 六年正月,豫聚兵淮陽,韓世忠引兵急圍之。賊守將連舉六烽,兀朮與劉猊合兵來援,皆為
 世忠所敗。六月,築劉龍城以窺淮西,王師晟破之,執華知剛,俘其衆而還。九月,豫罷沿海互市。張孝純謂豫曰:「聞南人久治舟,一旦乘風北濟,將不利於我。」豫懼,故罷之。



 豫聞帝親征,告急于金主亶,領三省事宗磐曰:「先帝立豫者,欲豫辟疆保境,我得按兵息民也。今豫進不能取,退不能守,兵連禍結,休息無期。從之則豫收其利,而我實受弊,奈何許之!」金主報豫自行,姑遣兀朮提兵黎陽以觀釁。



 豫於是以麟領東南道行臺尚書令,李鄴行
 臺右丞,馮長寧行臺戶部,許清臣兵馬大總管,李成、孔彥舟、關師古為將,籍民兵三十萬,分三道入寇。麟總中路兵,由壽春犯廬州;猊率東路兵,取紫荊山出渦口以犯定遠;西兵趨光州,寇六安,彥舟統之。十月,猊兵阻韓世忠不得前,還順昌。麟兵從淮西繫三浮橋以濟,賊衆十萬次濠、壽間。江東安撫使張俊拒戰,詔並以淮西屬俊,命殿帥楊沂中至泗州與俊合,比至濠而劉光世已棄合肥矣。張浚遣人星馳采石諭光世曰:「敢濟者斬。」光
 世不得已還廬州,與沂中相應。統制王德、酈瓊出安豐,遇賊三將軍皆敗之。猊衆數萬過定遠,欲趨宣化犯建康。沂中遇猊兵於越家坊,破之;又遇於藕塘,大破之。猊遁,麟聞亦拔砦走,麟兵有自書鄉貫姓名而縊者,豫由此失人心。金人聞麟等敗,詰豫罪狀,始有廢豫意矣。豫覺,請立麟為太子,以覘其意。金人乃答豫曰:「徐當遣人咨訪河南百姓。」



 七年春,豫策進士。遣諜縱火淮甸,燔劉光世帑藏。二月,又焚鎮江。豫自麟敗,意沮氣奪。中原遺
 民,日望王師。三月,帝進駐建康。八月,統制酈瓊執呂祉,以兵三萬叛降豫,尋殺祉。豫聞瓊降大喜,御文德殿見之,授瓊靜難軍節度使、知拱州。瓊勸豫入寇,豫復乞師金人,且言瓊欲自效。金人恐豫兵衆難制,欲以計除之,乃佯言瓊降恐詐,命散其兵。



 金人業已廢豫,而豫日益請兵,遂以女真萬戶束拔為元帥府左都監屯太原,渤海萬戶大撻不也為右都監屯河間。於是尚書省奏豫治國無狀,當廢。十一月丙午,廢豫為蜀王。



 初,金主先令
 撻辣、兀朮偽稱南侵至汴,紿麟出至武城,麾騎翼而擒之,因馳至城中。豫方射講武殿,兀朮從三騎突入東華門,下馬執其手,偕至宣德門,強乘以羸馬,露刃夾之,囚于金明池。翼日,集百官宣詔責豫,以鐵騎數千圍宮門,遣小校巡閭巷間,揚言曰:「自今不僉汝為軍,不取汝免行錢,為汝敲殺貌事人,請汝舊主少帝來此。」由是人心稍安。置行臺尚書省於汴,以張孝純權行臺左丞相。偽丞相張昂知孟州,李鄴知代州,李成、孔彥舟、酈瓊、關師
 古各予一郡。以女真胡沙虎為汴京留守,李儔副之。諸軍悉令歸農,聽宮人出嫁。得金一百二十餘萬兩、銀一千六百餘萬兩、米九十餘萬斛、絹二百七十萬匹、錢九千八百七十餘萬緡。



 豫求哀,撻辣曰:「昔趙氏少帝出京,百姓然頂煉臂,號泣之聲聞於遠邇。今汝廢,無一人憐汝者,何不自責也。」豫語塞,迫之行,願居相州韓琦宅,許之。後並其子麟徙於臨潢,封豫為曹王,賜田以居之。紹興十三年六月卒,是年金皇統三年也。豫僭號凡八年,
 廢時年六十五。先是,齊地數見怪異,有梟鳴于後苑,龍撼宣德門滅「宣德」二字,有星隕于平原鎮。識者謂禍不出百日,豫怒殺之。未幾果廢。



 初,偽麟府路經略使折可求以事抵雲中,左監軍撒離曷密諭可求代豫。後撻辣有歸疆之議,恐可求𡙇望,鴆殺之。



 豫之僭逆也,馬定國進《君臣名分論》,祝簡獻《遷都》、《國馬賦》,語多指斥;又如許清臣毀景靈宮,孟邦雄發永安陵,蹠犬吠堯,蓋無責焉。



 苗傅,上黨人。大父授,父履。授在元豐中為殿前都指揮
 使。康王建元帥府,信德守臣梁揚祖以兵萬人至,傅與張俊、楊沂中、田師中皆隸麾下。隆祐太后南渡,傅為統制官,以所部八千人扈衛,駐於杭州。



 有劉正彥者,不知何許人。父法,政和間為熙河路經略使,死王事。正彥由閤門祗候易文資至朝奉大夫,後以事責降。會法部曲王淵為御營都統制,正彥歸之。淵以法故,薦正彥於朝,復為武德大夫、知濠州,擢御營右軍副都統制,淵分精兵三千與之。以平丁進功,進武功大夫、威州刺史。初,正
 彥討進,請劉晏偕行。晏本嚴陵人,陷遼登第,宣和中率衆來歸。正彥用晏計,易旗幟為疑兵,遂降進。晏自通直郎遷朝請郎,正彥恥己賞薄而晏獲峻遷,由是𡙇望,乃散所賜金帛與將士,尋被命從六宮、皇子至杭州。



 建炎三年二月壬戌,高宗從王淵議,由鎮江幸杭州。時諸大將,如劉光世、張俊、楊沂中、韓世忠分守要害,扈衛者獨苗傅。



 先是,王淵裝大船十數,自維揚來杭,杭人相謂曰:「船所載,皆淵平陳通時殺奪富民家財也。」內侍省押班
 康履頗用事,威福由己出;其徒奪民居,肆為暴橫。傅等恨之,曰:「天子顛沛至此,猶敢爾耶!」其黨張逵復激怒諸軍曰:「能殺淵及內侍,則人人可富,朝廷豈能徧罪哉!」



 三月辛巳,拜王淵同簽書樞密院事。初,淵建幸杭州議,內侍實左右之。及淵躐躋樞筦,衆謂薦由內侍。傅自負宿將,疾淵驟貴。正彥雖由淵進,淵檄取所予兵,亦怨之。於是傅積不能平,與王世脩、張逵、王鈞甫、馬柔吉等謀作亂。鈞甫等皆燕人,所將號「赤心軍」。傅部分既定,乃紿淵
 以臨安縣有盜,意欲使淵出其兵於外。



 康履得黃卷小文書,有兩統制作「田」、「金」字署卷末,田乃苗,金乃劉也。於是頗泄賊謀,以告淵,淵伏兵天竺。明日,賊黨亦伏兵城北橋下,俟淵退朝,誣以結宦官謀反,正彥手殺淵,以兵圍履第,分捕內官,凡無須者盡殺之,揭淵首,引兵犯闕。中軍統制吳湛守宮門,潛與傅通,導其黨入奏曰:「苗傅不負國,止為天下除害。」



 知杭州康允之聞變,率從官扣閽,請帝御樓,百官皆從。殿帥王元大呼聖駕來,傅見黃
 屋,猶山呼而拜。帝憑闌呼二賊問故,傅厲聲曰:「陛下信任中官,軍士有功者不賞,私內侍者即得美官。黃潛善、汪伯彥誤國,猶未遠竄。王淵遇敵不戰,因友康履得除樞密。臣立功多,止作遙郡團練。已斬淵首,更乞斬康履、藍珪、曾擇以謝三軍。」帝諭以當流海島,可與軍士歸營,且曰:「已除傅承宣使、御營都統制,正彥觀察使、御營副都統制。」



 賊不退。帝問百官計安出,浙西安撫司主管機宜文字時希孟曰:「禍由中官,不悉除之,禍未已也。」帝曰:「
 朕左右可無給使耶?」軍器監葉宗諤曰:「陛下何惜康履。」遂命吳湛捕履,得於清漏閣承塵中。傅即樓下腰斬履。



 傅猶肆惡言,謂「帝不當即大位,淵聖來歸,何以處也?」帝使朱勝非縋樓下曲諭之。傅請隆祐太后同聽政及遣使與金議和。帝許諾,即下詔請太后垂簾。賊聞詔不拜,曰:「自有皇太子可立。」張逵曰:「今日之事,當為百姓社稷計。」時希孟曰:「宜率百官死社稷,否則從三軍之請。」通判杭州事章誼叱之曰:「何可從三軍邪!」帝徐謂勝非曰:「朕
 當退避,須太后命。」勝非謂不可。顏岐曰:「得太后親諭之,則無詞矣。」



 時寒甚,門無簾幃,帝坐一竹椅。既請太后,即起立楹側。太后御肩輿出立樓前,二賊拜曰:「今日百姓無辜,肝腦塗地,望太后主張。」太后曰:「道君皇帝任蔡京、王黼,更祖宗法,童貫起邊釁,所以致金人之禍。今皇帝聖孝,無失德,止為黃潛善、汪伯彥所誤,已加竄逐,統制獨不知邪?」傅曰:「臣等定議,必欲立皇子。」后曰:「今強敵在外,使吾一婦人簾前抱三歲兒,何以令天下?」正彥等號
 泣固請,因呼其衆曰:「太后既不允,吾當受戮。」遂作解衣狀,后諭止之。傅曰:「事久不決,恐三軍生變。」顧謂勝非曰:「相公何無一言?」勝非不能答。適顏岐至自帝前,奏曰:「皇帝令臣奏知太后,已決意從傅請矣,乞太后宣諭。」后猶不許,傅等語益不遜。



 太后還入門,帝遣人奏禪位,勝非泣曰:「臣義當死,乞下詰二凶。」帝屏左右語曰:「當為後圖,事不成,死未晚。」勝非曰:「王鈞甫,賊腹心也,適語臣曰:『二將忠有餘,學不足。』此可為後圖耳。」



 是日,帝幸顯忠寺。甲
 申,太后垂簾,降赦,號帝為睿聖仁孝皇帝,以顯忠寺為睿聖宮,留內侍十五人,餘悉編置。



 丙戌,赦至平江府,張浚知有變,不拜。丁亥,至江寧,制置呂頤浩遺浚書,痛述事變。浚乃舉兵。戊子,御營前軍統制張俊至平江,浚諭以起兵,俊泣奉命。



 初,勝非奏,垂簾當二臣同對,今屬時艱,乞許獨對。恐賊疑,乃日引其徒一人與俱。傅入對,后勞勉之。賊喜,無所疑,故臣僚入對,得謀復辟。



 勝非深結王世修,將處以從官,俾通二凶。



 傅欲改元,正彥欲遷都
 建康,太后謂勝非曰:「二事如俱不允,恐賊有他變。」己丑,改元明受。張浚遺書二凶,獎其忠義以慰安之。庚寅,百官朝睿聖宮。以傅為武當軍節度使。



 辛卯,張浚遣進士馮轓赴行在,請帝親總要務。復抵書馬柔吉、王鈞甫宜早反正,以解天下之惑。



 浚既遣轓,即檄諸路,約呂頤浩、劉光世會平江。傅以堂帖趣張浚赴秦州,命趙哲領俊軍,哲不從;改命陳思恭,思恭亦不從。



 壬辰,以諫議大夫鄭瑴為御史中丞。賊以武功大夫王彥為御營司統制,
 瑴面折二凶,彥佯狂,即日致仕。



 癸巳,韓世忠引兵至常熟。辛道宗謂張浚曰:「賊萬一邀駕入海,何以為計!」浚乃聲言防遏海寇,奏道宗為節制司參議官,措置海船以避賊。



 甲午,貶曾擇、藍珪于嶺南,傅追斬擇。賊欲以所部代禁衛守睿聖宮,又欲邀帝幸徽、越,張澄、勝非曲諭止之。



 馮轓說二凶反正,傅按劍瞋目視轓,正彥解之,曰:「須張侍郎來,乃可。」即遣歸朝官趙休與轓共招浚。



 乙未,呂頤浩勤王兵至丹陽,劉光世引所部來會。丙申,韓世忠
 兵至平江,即欲進兵。浚曰:「已遣馮轓甘言誘賊矣。投鼠忌器,不可太亟。」



 賊遣張彥、王德聲言防淮,德伺彥醉,並其軍,自采石濟江歸劉光世,彥尋為人所殺。戊戌,浚以世忠兵少,分張俊兵二千益之,發平江。



 馮轓至平江,浚復遣入責賊以大義,諭以禍福,期雖死無悔。傅等初聞浚集兵,未之信,及得浚書,始悟見討。奏請誅浚以令天下。詔責浚黃州團練副使,郴州安置。鄭瑴上疏謂浚不當責,密遣所親謝嚮變姓名告浚宜持重緩進,賊當自遁,
 浚然之。



 是日,賊遣苗瑀、馬柔吉將赤心隊及王淵舊部曲駐臨平,以拒勤王之師。馮轓至臨平,見馬柔吉,同縋入城。詰朝,與傅等議,傅曰:「爾尚敢來邪?」欲拘轓。浚逆知之,謬為書遺轓,言客自杭來,知二公于朝廷初無異心,殊悔前書失於輕易。賊得浚遺轓書,大喜,乃釋轓。



 壬寅,浚得謫命,恐將士解體,紿曰:「趣召之命也。」是日,呂頤浩至平江,與浚對泣曰:「事不諧,不過赤族。」乃命幕客李承造草檄告四方討賊。賊聞勤王之兵大集,即呼馮轓、勝
 非議復辟。癸卯,張俊發平江,劉光世繼之。賊亦遣兵三千屯湖州小林。丙午,頤浩、浚以大兵發平江。詔以浚為知樞密院事。



 丁未,勝非召二凶至都堂議復辟,率百官三上表以請。夏四月戊申朔,帝還宮,都人大說。帝御前殿,詔尊太后曰隆祐皇太后,立嗣君為皇太子。辛酉,徙傅淮西制置使,正彥副之。庚戌,詔復建炎號。



 是日,頤浩、浚軍次臨平,苗翊、馬柔吉以兵阻河。韓世忠率先鋒力戰,俊、光世乘之,翊敗走。勤王兵進北關。二凶詣都堂,趣
 得所賜鐵券,引精兵二千,夜開湧金門遁。辛亥,頤浩、浚引勤王兵入城。世忠手執王世修以屬吏。



 苗傅犯富陽,統制官喬仲福追擊之。癸丑,犯桐廬。甲寅,斬吳湛。時希孟編管吉陽軍。丙辰,傅等至白沙渡,所過燔橋以阻官軍。丁巳,犯壽昌縣,黥民充軍。庚申,犯衢州,守臣胡唐老拒卻之。丙寅,犯常山。世忠請任討賊。丁卯,以世忠為江浙制置使,自衢、信追擊賊。戊辰,賊犯玉山縣。辛未,賊屯沙溪鎮。統制巨師古自江東討賊還,與喬仲福、王德會
 信州。賊聞之,還屯衢、信間。



 五月戊寅朔,世忠發杭州。庚辰,賊黨張翼斬鈞甫及柔吉父子首以降,江浙制置使周望受之以聞。賊寇浦城縣,夾溪而屯,據險設伏,以邀官軍,統制官馬彥溥死之。賊乘勝犯中軍,世忠瞋目大呼,揮兵直前,正彥墮馬,生禽之。賊將江池殺孟皋、禽苗翊降,衆悉解甲。張逵收餘兵入崇安,喬仲福追殺之。



 傅棄軍變姓名夜遁建陽,土豪詹標覺之,執送世忠,檻車赴行在。壬寅,詔班師。



 秋七月辛巳,世忠軍還,俘傅、正彥
 以獻,磔于建康市。張逵、苗瑀及傅二子俱已前死。詔釋餘黨。



 杜充,字公美,相人也。喜功名,性殘忍好殺,而短于謀略。紹聖間,登進士第,累遷考功郎、光祿少卿,出知滄州。靖康初,加集英殿修撰,復知滄州。時金人南侵,郡中僑寓皆燕人來歸者,充慮為敵內應,殺之無噍類。



 建炎元年,進天章閣待制、北京留守,遷樞密直學士。提刑郭永嘗畫三策以獻充,充不省。永誚之曰:「人有志而無才,好名
 而無實,驕蹇自用而得聲譽,以此當大任,鮮克有終矣。」二年,宗澤卒,充代為留守兼開封尹。三年,以戶部尚書兼侍讀召,未至,改資政殿學士,節制淮南、京東西路,依前京城留守,尋知宣武軍節度使。



 七月,以同知樞密院召還,至,即拜尚書右僕射、同平章事、御營使。初,宗澤要結豪傑,圖迎二帝。澤卒,充短於撫御,人心疑阻,兩河忠義之士往往皆引去,留守判官宗穎嘗疏其失。朝廷謂充有威望,可屬大事,呂頤浩、張浚亦薦之,故有是命。時諸路
 各擁重兵,率驕蹇不用命。張俊方白事,謁未入,俊遽前,充怒戮其使,諸將稍稍慴服。



 高宗將幸浙西,命韓世忠屯太平,王𤫉屯常州。以充為江淮宣撫使,留建康,使盡護諸將。光世、世忠憚充嚴急,不樂屬充。詔移光世江州、世忠常州。時江、浙倚充為重,而充日事誅殺,無制敵之方,識者寒心。


金人窺江,充遣裨將王民、張超分守諸渡,乘高據岸,以神臂弓射卻之。金人復逼●
 \gezhu{
  缺字:石冏}
 砂,時以輕舟薄南岸,官軍奮擊,或沉其舟。一日當晝,金人對江列陣
 而佯退,衆信之,守益懈。敵諜知無備,夜乃乘數十舟橫江直濟,衆不能御,敵遂登岸。充亟命統制官陳淬盡領岳飛諸裨校合二萬人邀擊于馬家渡,約王𤫉俱进。敵氣銳甚,淬戰沒,𤫉引兵遁,充軍潰。



 金人陷建康,充渡江保真州。充嘗痛繩諸將,諸將銜之,伺其敗,衆將甘心焉。充不敢歸,乃北約泗州劉位、徐州趙立,欲合兵邀敵歸路。詔遣內侍任源賜親劄激厲,俾為後圖。源至常州,道阻未得進,募健士先達上意,充詭詞自飭以報源。



 充居
 真州長蘆寺,守臣向子忞勸充由通、泰入浙,欲與偕行,充畜異志,不聽。始,京畿提刑淩唐佐在南京,守臣孟庾歸朝,以府事委之,唐佐遂降于金為所用。唐佐雅善充,以書招之。完顏宗弼復遣人說充曰:「若降,當封以中原,如張邦昌故事。」充遂叛降金。事聞,高宗謂輔臣曰:「朕待充不薄,何乃至是哉?」下制削充爵,徙其子嵩、巖、崑、婿韓汝惟于廣州。



 是冬,充至雲中,粘罕薄之,久之,命知相州。充猜阻肆威,同列多不協。紹興二年,其孫自徙所間走
 歸充,其副胡景山誣充陰通朝廷。粘罕下充吏,炮掠備至,不服,釋之,因問充曰:「汝欲復歸南朝邪?」充曰:「元帥敢歸,充不敢也。」粘罕哂之。七年,命充為燕京三司使。八年,同簽書燕京行臺尚書省事。九年,遷行臺右丞相。十一年,和議成而充死矣。



 吳曦,信王璘之孫,節度挺之中子。以祖任補右承奉郎。淳熙五年,換武德郎,除中郎將,後省言其太驟,改武翼郎。累遷高州刺史。紹熙四年,挺卒,起復濠州團練使。慶
 元元年冬,由建康軍馬都統制除知興州兼利西路安撫使。四年,憲聖園陵成,以勞遷武寧軍承宣使。六年,光宗攢陵成,遷太尉。



 會韓侂胄謀開邊,曦潛畜異志,因附侂胄求還蜀。樞密何澹覺其意,力沮之。陳自強納曦厚賂,陰贊侂胄,遂命曦興州駐紮御前諸軍都統制,兼知興州、利州西路安撫使。從政郎朱不棄上侂胄書,謂曦不可主西師,侂胄不報。曦至鎮,譖副都統制王大節,罷之,更不除副帥,而兵權悉歸於曦。開禧二年,朝廷議出
 師,詔曦為四川宣撫副使,仍知興州,聽便宜行事。自紹興末,王人出總蜀賦,移牒宣司,勢均禮敵。而侂胄以總計隸宣司,副使得節制按劾,而財賦之權又歸於曦。未幾,兼陝西、河東招撫使。



 曦與從弟晛及徐景望、趙富、米修之、董鎮共為反謀,陰遣客姚淮源獻關外階、成、和、鳳四州于金,求封為蜀王。侂胄日夜望曦進兵,曦陽為持重,按兵河池不進,潛為金人地以困王師,侂胄不之覺。會正使程松至,曦不庭參,松不敢詰;曦復多摘取松衛
 兵,松亦不悟。



 金人犯西和,王喜、魯翼拒之。戰方急,曦傳令退保黑谷,軍遂潰。乃焚河池,退壁青野原。曦時已布腹心于金,將士未之知,猶力戰,敵人竊笑之。曦退壁魚關,招集忠義,厚賜以收衆心。興元都統制毋思以重兵守大散關,曦因撤驀關之戍,敵由版閘谷繞出思後,思遁。金遂陷大散關,曦退屯罝口。舉人陳國飾投匭上書,言曦必叛,侂胄不省。



 十二月,興州見兩日相摩。金遣吳端持詔書、金印至罝口,封曦蜀王,曦密受之。李好義敗
 金人于七方關,曦不上其捷,還興州。是夜,天赤如血,光燭地如晝。翌日,曦召幕屬諭意,謂東南失守,車駕幸四明,今宜從權濟事,衆失色。王翼、楊騤之抗言曰:「如此,則相公八十年忠孝門戶,一朝掃地矣!」曦曰:「吾意已決。」即詣甲仗庫,集兵將官語故,祿禧、褚青、王喜、王大中等皆稱賀聽命。曦北向受印。遣徐景望為四川都轉運使、褚青為左右軍統制,趨益昌,奪總領所倉庫。程松聞變,棄興元去。



 三年正月,曦遣將利吉引金兵入鳳州,以四郡付
 之,表鐵山為界。曦乘黃屋左纛,僭王位於興州,即治所為行宮,稱是月為元年。使人告其伯母趙氏,趙怒絕之。叔母劉晝夜號泣,罵不絕口,曦扶出之。族子僎為興元統制,見偽檄,色甚不平。



 曦既僭位,議行削髮左衽之令。遣董鎮至成都治宮殿,將徙居之。曦所統軍七萬並程松軍三萬,分隸十統帥。遣祿祁、房大勳戍萬州,泛舟下嘉陵江,聲言約金人夾攻襄陽。祁尋至夔,遣兵扼巫山得勝、羅護等砦,以遏王師。侂胄聞曦反,不知所為,或勸
 不如因而封之,侂胄納其說。吳晛為曦謀,宜收用蜀名士以係民心。於是陳咸自髡其發,史次秦塗其目,楊震仲飲藥卒,王翊、家拱辰皆不受偽命,楊修年、詹久中、家大酉、李道傳、鄧性善、楊泰之悉棄官去。薛九齡謀舉義兵。



 興州合江倉官楊巨源倡義討逆,未有以發,遂與隨軍轉運安丙共謀誅曦。會李好義與兄好古、李貴等皆有謀,交相結納。二月甲戌夜,漏盡,巨源、好義首率勇敢七十人斧門以入。李貴即曦室斬其首,裂其屍。丙分遣
 將士收其二子及叔父柄、弟晫、从弟晛、賊黨姚淮源、李珪、郭仲、米修之、郭澄等皆誅之。時吳端猶臥後閣,亦伏誅。徐景望、趙富、吳曉、董鎮、郭榮、祿禧等皆在外,遣人就誅之。函曦首獻於朝。



 詔曦妻子處死,親昆弟除名勒停,吳璘子孫並徙出蜀,吳玠子孫免連坐,通主璘祀。曦敗時年四十六。



\end{pinyinscope}