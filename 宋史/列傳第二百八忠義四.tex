\article{列傳第二百八忠義四}

\begin{pinyinscope}

 崔縱
 吳安國附林沖之子鬱從子震霆附滕茂實魏行可郭元邁附閻進朱績附趙師檟
 易青胡斌範旺馬俊楊震仲史次秦郭靖附高稼曹友聞陳寅賈子坤劉銳蹇彞何充附許彪孫張桂金文德曹顏胡世全龐彥海江彥清附陳隆之史季儉附王翊李誠之秦鉅附



 崔縱,字元矩,撫州臨川人。登政和五年進士第。歷確山主簿、仙居丞,累遷承議郎、乾辦審計司。二帝北行,高宗
 將遣使通問,廷臣以前使者相繼受系,莫肯往。縱毅然請行,乃授朝請大夫、右文殿修撰、試工部尚書以行。比至,首以大義責金人,請還二帝,又三遺之書。金人怒,徙之窮荒,縱不少屈。久之,金人許南使自陳而聽其還,縱以王事未畢不忍言。又以官爵誘之,縱以恚恨成疾,竟握節以死。洪皓、張邵還,遂歸縱之骨。詔以兄子延年為後。



 吳安國字鎮卿,處州人。太學進士,累官遷考功郎官。以
 太常少卿使金,值金人渝盟,拘留脅服之,安國毅然正色曰:「我首可得,我節不可奪,惟知竭誠死王事,王命烏敢辱?」金人不敢犯,遣還。後知袁州,卒。



 林沖之,字和叔,興化軍莆田人。元符三年進士,歷御史臺檢法官、大宗正丞,都官、金部郎,滯省寺者十年。出守臨江、南康。



 靖康初,召為主客郎中。金人再來侵,詔副中書侍郎陳過庭使金,同被拘執。初猶給乳酪,迨宇文虛中受其命,金人亦以是邀之,沖之奮厲見詞色,金人怒,
 徙之奉聖州。既二年,過庭卒,金人逼沖之仕偽齊,不屈;徙上京,又不屈;置顯州極北水互寒之地,幽佛寺十餘年。漸便飲茹,以義命自安,髭發還黑。病亟,語同難者曰:「某年七十二,持忠入地無恨,所恨者國仇未復耳。」南向一慟而絕。僧窆之寺隅。洪皓還朝以聞,詔與二子官。子鬱,從子震、霆。



 鬱字襲休,宣和三年進士,再調福建茶司干官。建州勤王卒自京師還,求卸甲錢,郡守逃匿,卒鼓噪取庫兵為亂,殺轉運使毛奎、轉運判官曾仔、主管文字
 沈升。鬱聞變急入諭卒,遇害。事聞,詔各與一子官。



 震字時旉,崇寧元年進士,仕至秘書少監。以不附二蔡有聲崇寧、大觀間。



 霆字時隱,政和五年進士,敕令所刪定官。詆紹興和議,謂不宜置二帝萬里外不通問,即掛冠出都門,權臣大恚怒,亦廢放以死,莆人稱為「忠義林氏」。寶慶三年,即其所居立祠。寶祐中,又給田百畝,使備祭享以勸忠義云。



 滕茂實字秀穎,杭州臨安人。政和八年進士。靖康元年,
 以工部員外郎假工部侍郎,副路允迪出使,為金人所留。時茂實兄陶通判代州,已先降金。粘罕素聞茂實名,乃遷之代州,又自京師取其弟華實同居,以慰其意。



 欽宗自離都城,舊臣無敢候問起居者。茂實聞欽宗將至,即自為哀詞,且篆「宋工部侍郎滕茂實墓」九字,取奉使黃幡裹之,以授其友人朔寧府司理董詵。欽宗及郊,茂實具冠幘迎謁,拜伏號泣。金人諭之曰:「國破主遷,所以留公,蓋將大用。」迫令易服,茂實力拒不從,見者墮淚。茂
 實請從舊主俱行,金人不許,憂憤成疾,卒云中。詵拔歸,錄所為哀詞言於張浚,浚以詵為陜西轉運判官,上其事。紹興二年,贈龍圖閣直學士,官其家三人。



 魏行可,建州建安人。建炎二年,以太學生應募奉使,補右奉議郎,假朝奉大夫、尚書禮部侍郎,充河北金人軍前通問使,仍命兼河北、京畿撫諭使。時河北紅巾賊甚眾,行可始懼為所攻,既而見使旌,皆引去。行可渡河見金人於澶淵,金人知其布衣借官,待之甚薄,因留不遣。
 行可嘗貽書金人,警以「不戢自焚」之禍:「大國舉中原與劉豫,劉氏何德?趙氏何罪?若亟以還趙氏,賢於奉劉氏萬萬也。」



 紹興六年,卒。十三年,張邵來歸,言行可執節沒於王事,行可父通直郎伯能亦訴於朝,遂贈朝奉郎、秘閣修撰,先已官其二子一弟,至是,復官其一孫。



 行可之使也,吳人郭元邁以上舍應募,補右武大夫、和州團練使為之副,不肯髡發換官,亦卒於北焉。



 閻進,隸宣武。建炎初,遣使通問,進從行。既至雲中府,金
 人拘留使者散處之,進亡去。追還,留守高慶裔問:「何為亡?」進曰:「思大宋爾。」又問:「郎主待汝有恩,汝亡何故?」進曰:「錦衣玉食亦不戀也。」慶裔義而釋之。凡三亡乃見殺。臨刑,進謂行刑者:「吾南向受刃,南則我皇帝行在也。」行刑者曳其臂令面北,進踴身直起,盤旋數四,卒南鄉就死。



 進武校尉朱績亦從之,分在粘罕所。績見粘罕數日,遽求妻室。粘罕喜,令擇所虜內人妻之,績取最醜者,人莫諭其意。不半月亡去,追之還,粘罕大怒,績含笑死梃下。
 蓋績求妻者,所以固粘罕也。



 趙師檟以罪拘管西外宗正司,福建提刑王夢龍以智勇可用,屬制軍器。會寇逼尤溪,令師檟統卒數百往戍。既行,大書於旗曰:「不與賊俱生。」人皆壯之。賊兵至,師檟迎敵於林嶺,身為先鋒。戰十餘合,賊至益眾,師檟所乘馬適陷田中,賊斷其左臂,師檟以右手拔背刀斬七級。力盡,部曲欲引遁,師檟仰天大呼曰:「師檟報國死於此矣。」遂沒焉。尤溪之民為之立廟戰處。樞密王野請加褒
 贈,乃贈武節郎,與一子恩澤。



 易青者,為都督行府摧鋒軍效用。初,廣東賊曾袞本軍士也,已受招復叛。紹興六年十月,經略使連南夫與摧鋒軍統制韓京會於惠州,督諸兵討之。京募敢死士七十三人夜劫袞營,青在行中,為所執。賊驅至後軍趙續砦外,謂續曰:「汝大軍為我所擒者甚眾。」青大呼曰:「勿信,所擒者我爾。」賊又言:「吾不汝殺,第令經略持黃榜來招安。」青又呼曰:「勿聽,任賊殺我,我惟以一死報國。」賊怒焚
 之,青死,罵不絕口。青無妻子。事聞,特贈保義郎、閣門祗候,官為薦祭焉。



 胡斌,為殿前司將官。童德興提禁旅戍邵武,江、閩寇作,知邵武有備,未敢犯。會招捕司檄德興稟議,獨留斌將弱卒數百留城中。紹定三年閏月己卯,盜眾大至,他將士皆遁,獨斌奮身迎戰,所格殺甚眾。賊益生兵,官軍所存僅數十人,或告以眾寡不敵,盍避之!斌曰:「郡民死者以萬計,賴生者數千人由東門而出,我不綴其勢,使得
 脫走,則賊躡其後,無噍類矣。」遂巷戰,大呼曰:「我死救百姓。」兵盡矢窮,卒遇害,其尸殭立,移時始僕。事聞,贈武節大夫,錄其後一人。樞密院編修官王野言邵武民即斌戰地立廟,請就以「武節」為廟額,從之。



 範旺,南劍州順昌縣巡檢司軍校也。初,順昌盜俞勝等作亂,官吏皆散,土軍陳望素樂禍,與射士張袞謀舉砦應之,旺叱之曰:「吾等父母妻子皆受國家廩食以活,今力不能討,反更助為虐,是無天地也。」兇黨忿,剔其目而
 殺之。



 一子曰佛勝,年二十,以勇聞,賊詐以父命召之,至則俱死。其妻馬氏聞之,行且哭,賊脅污之,不從,節解之。



 賊既平,旺死跡在地,隱隱不沒,邑人驚異,為設像城隍廟,歲時祭享。紹興六年,轉運使以狀聞,詔贈承信郎,更立祠,號忠節。二十八年,復詔立愍節廟以祠之。



 馬俊或曰進,太平州慈湖砦兵也。紹興二年,砦軍陸德、周青、張順等據州叛,青為謀主,約翌日盡黥城中少壯,而屠其老弱,然後擁眾渡江。俊隸青左右,得其謀,陰結
 其徒十人殺賊,然後諭眾開門,其徒許之。俊歸語其妻孫氏,與之訣,至南門,伺青出上馬,斫中頰,九人懼不敢前。俊與妻子皆遇害。青被傷臥旬日,賊黨散,官軍至,德、青遂伏誅。三年,贈俊修武郎,為立祠,號登勇。



 楊震仲字革父,成都府人。蚤負氣節,雅有志當世。登淳熙二年進士第。知閬州新井縣,以惠政聞。



 闢興元府通判,權大安軍。吳曦叛,素聞震仲名,馳檄招之,震仲辭疾不行。時軍教授史次秦亦被檄,謀於震仲,震仲曰:「大安
 自武興而來,為西蜀第一州,若首從其招,則諸郡風靡矣。顧力不能拒,義死之。教授非城郭臣,且有母在,未可死,脫去為宜。」因屬次秦曰:「吾死,以匹絹纏身,斂以小棺足矣。」曦遣興州都統司機宜郭鵬飛代震仲,趣其行益急。鵬飛宴震仲,終飲不見顏色。歸舍,然燭獨坐,夜漏至三鼓,呼左右索湯,比至,震仲飲毒死矣。次秦如其言,斂而置於蕭寺,闔郡為之流涕。



 震仲之未死,先遺家人書曰:「武興之事,從之則失節,何面目在世間?不從禍立見。
 我死,禍止一身,不及妻子矣。人孰無死,死而有子能自立,即不死。」自震仲死,蜀之義士感慨奮發,始有協謀誅逆者。明年,曦伏誅,蜀帥安丙、楊輔以聞,贈朝奉大夫、直寶謨閣,官二子,表其里曰義榮。吳獵宣諭西蜀,為之請廟與謚,名其廟旌忠,謚曰節毅。



 史次秦,眉山人。及進士第。吳曦叛,招次秦甚遽,次秦遷延固避,偽知大安軍郭鵬飛迫之行,乃以石灰桐油塗兩目,末生附子傅之,比至目益腫。次秦母年高而賢,聞
 次秦為曦所招,即命家人以疾篤馳報,且曰:「恐病不足取信,以訃聞可也。」曦乃聽還。曦誅,蜀帥上其事,改秩為利路主管文字,仕至合州太守。



 有郭靖者,高橋土豪巡檢也。吳曦叛,四州之民不願臣金,棄田宅,推老稚,順嘉陵而下。過大安軍,楊震仲計口給粟,境內無餒死者。曦盡驅驚移之民使還,皆不肯行。靖時亦在遣中,至白崖關,告其弟端曰:「吾家世為王民,自金人犯邊,吾兄弟不能以死報國,避難入關,今為曦所逐,吾不忍棄漢衣冠,
 願死於此,為趙氏鬼。」遂赴江而死。



 高稼,字南叔,邛州蒲江人。真德秀一見以國士期之。嘉定七年進士。調成都尉,轉九隴丞。丁內艱,免喪,闢潼川府路都鈴轄司干辦公事。制置使崔與之聞其名,改闢本司干辦公事。



 稼持論不阿,憂世甚切,及鄭損為制置使,即求去。朝廷以稼贊閫有勞,未幾,改知綿谷縣。制置司以總領所擅十一州會子之利,請盡廢之,此蓋紹興、隆興之間得旨為之者。令下,民疑,為之罷市。稼亟出私
 錢以給中下戶。稼弟定子時為總領所主管文字,相與徵其誤而力救之,得存其半,公私僅濟。歲大饑,有司置弗聞,稼捐橐中裝,市粟以食之,全活甚眾。損之入蜀也,稼同產弟了翁誦言於朝,謂必敗事。損銜之,遂劾稼罷。



 寶慶三年,元兵至武階,損棄沔而遁。桂如淵鎮蜀,闢通判沔州,尋檄兼幕職。稼首言:「蜀以三關為門戶,五州為藩籬,自前帥棄五州,民無固志,一旦敵至,又有因糧之利,或遂留不去。今亟當申理,俾緩急有所保聚。」如淵然
 之,乃創山砦八十有四,且募義兵五千人,與民約曰:「敵至則官軍守原堡,民丁保山砦,義兵為游擊,庶其前靡所掠,後弗容久。」



 北兵由東道以入,如淵憂之,闢稼知洋州。稼日夜為守禦計,以洋居平地,無一卒以守,議移金州帥司軍千人駐洋州,而自任其餉給。李心傳為言諸朝,不報。及鳳州破,制置司始從稼請,調金州兵赴之,而兵不時至。漢中陷,梁、洋之民數十萬盡趨安康。稼乃移屯黃金渡,收散卒,招忠義,以制置司之命,致故將陳昱
 於安康,委以收復之任。昱部分諸軍,召青座、華陽諸關守將,皆以兵來會,凡得三千人,稼竭洋之帑廩贍之。以州事付通判,而自假節制軍馬,督諸將繼進。沔州破,北兵迫大安,益昌大震,稼亟命趨沔,自至西縣援之。



 如淵以便宜命稼利路提刑司兼權興元府,制置司檄其守米倉,稼移書曰:「今日之事如弈棋,所校者先後爾。茍以分水、三泉、米倉為可保,敵兵若自宕昌、清川以入,將孰禦之?盍以興、沔、利三戎司分駐鳳州,俾制司已招之忠
 義、關表復仇之豪傑,聯司以進,兵氣奪矣。」如淵遲疑不決。逮天水、同慶被屠,西和圍益急,始會軍民之眾萬人援之,道梗不得前,而城已破矣。俄報砦窠、七方之師皆潰,稼率遺民駐廉水縣,召集保甲,分布間道,以保巴山。當是時,文臣之在軍中者惟稼一人。



 如淵既罷,李𡌴代之,以稼久勞,請改畀內郡,差知榮州。殿中侍御史汪剛中,如淵黨也,欲使稼分其罪,乃謂蜀之敗實由稼,遽罷之,又削二官。李心傳見上,訟稼無罪,不當罷。



 宣撫使黃
 伯固闢稼知閬州。未幾,伯固去官,制置使趙彥吶以參議官闢之。制置司近漢中,稼言漢中蕩無藩籬,宜經理仙人原以為緩急視師之地。彥吶以委稼,稼至原,繕營壘,峙芻糧,比器甲,開泉源,守禦之規,罔不備具。會召還,彥吶密奏留稼,以直秘閣知沔州、利州提點刑獄兼參議官。始至,告於神曰:「郡當兵難之後,生聚撫摩,所當盡力,去之日,誓垂橐以入劍門。」乃葺理創殘,招集流散,民皆襁負來歸。



 北兵入西和,薄階州,稼贊彥吶登原督戰。
 知天水軍曹友聞等兵大戰。進稼三官,為朝請大夫兼關外四州安撫司公事,措置西路屯田。稼嘗代彥吶論蜀事利害,上嘉覽之。



 北兵自鳳州入,東軍不能御,遂搗河池,至西池谷,距沔九十里。吏民率逃,議欲退保大安。稼白彥吶曰:「今日之事,有進無退,能進據險地,以身捍蜀,敵有後顧,必不深入;若倉皇召兵,退守內地,敵長驅而前,蜀事去矣。」彥吶曰:「吾志也。」已而竟行,留稼守沔。



 北兵自白水關入六股株,距沔六十里。沔無城,依山為阻,
 稼升高鼓噪,盛旗鼓為疑兵。彥吶至置口,輟帳前總管和彥威,以軍還沔,召小將楊俊、何璘悉以兵會,又調總管王宣精兵千人益之。璘軍無紀律,稼捕其縱火者三人,誅之。未幾,北兵大至,璘遁。其眾皆潰,遂下沔州。



 先是,友聞戍七方,知沔不可守,勸稼移保山砦,而自將所部助之。稼曰:「七方要地,不可棄,吾郡將也,城亦不可棄。即事不濟,有死而已。」先二日,子斯得侍,以時危任重為憂,稼舉田承君「五日不汗」之言語之,且曰:「吾得死所,何憾!」
 又以書告李心傳曰:「稼必堅守沔,無沔則無蜀矣。自謂此舉可以無負知己。」及事迫,參議楊約勸稼姑保大安,稼厲聲曰:「我以監司守城郭,爾以幕客往來應援,各行其志。」常平司屬官馮元章率吏士力請稼少避,稼不為動。城既陷,眾擁稼出戶,稼叱之不能止,兵騎四集圍之,遂死焉。詔進稼七官,為正議大夫、龍圖閣直學士,謚曰忠。後以子斯得執政,累贈太師。



 稼為人慷慨有大志,聞人有善,稱之不容口;不善,面折無所避。推轂人士,常恐
 不及,視財如糞土。死之日,聞者莫不於邑流涕。所著有《縮齋類槁》三十卷。斯得自有傳。



 曹友聞,字允叔,同慶慄亭人。武惠王彬十二世孫也。少有大志,與仲弟友諒不遠千里尋師取友。登寶慶二年進士。授綿竹尉,改闢天水軍教授。



 城已被圍,友聞單騎夜入,與守臣張維糾民厲戰。兵退,制置使制大旗,書「滿身膽」以旌之。已而兵復至,友聞罄家財招集忠義,得健士五千人。制置使李𡌴檄管忠義,領所部守仙人關,且
 行且戰,至峽口據險。前軍統制屈信率所部突陣,還所掠四州人畜。至秦填,遣左軍統制杜午迎擊,力不能敵。友聞令諸軍乘高據險,身冒矢石,為士卒先。信與統制張安國領兵出戰。兵退,制置使檄捍七方關。



 北兵東破武休關,已而破七方,遂入沔州金牛,至大安,又分兵自嘉陵江木皮口突出何進軍後,進戰敗死之,遂長驅入劍門。友聞與弟萬各率所部,取間道過氈帽山,至青蒿埧,戰於白水江中流。兵退,制置司檄駐閬州。叛將魯珍
 為陳隆之所斬,珍部曲肆焚劫,友聞討斬其將郭虎、藺廣、楊仲等,餘黨散去。檄知天水軍。



 北兵入鳳州,略河池,抵同慶。友聞密遣統制王漢臣、統領張祥,授以方略出戰。兵至城下,友聞部分諸將各守一門,偃旗伏鼓,戒士卒,俟漸近,鳴鼓張旗,矢石並發。又命漢臣等取間道出戰,自提重兵尾敵後,大戰有功。端平初,友聞遣萬與忠義總管時當可分兵碎石頭、青蒿谷,前後大戰數合。制置使上其功,特授承務郎,權發遣天水軍。



 北兵又自西
 和至階州,友聞曰:「階雖非吾境,豈可坐視而不救。」遂引兵與諸軍會。命前軍統制全貴領所部為先鋒,統制夏用出其左,張成出其右,總管陳庚及萬、友諒往來督戰。有功,制置使趙彥吶俾節制利帥司軍馬,任責措置邊面,換武翼大夫、閣門宣贊舍人,差權利州駐扎御前諸軍都統制,駐扎石門,控扼七方關。



 明年,北兵破武休關,入沔陽,利路提刑高稼死之。制置使進屯青野原,被圍,友聞曰:「青野為蜀咽喉,不可緩。」遣萬領兵自冷水口度
 嘉陵江至六股株,屢戰有功。夜銜枚由間道直趨青野原,制置使奇萬之勇,令督諸軍戰守。兵退,友聞引精兵亦趨至原下,夜半截戰,圍遂得解。特授武德大夫、左驍騎大將軍,依舊利州駐扎御前諸軍統制。



 北兵破沔州,搗大安,友聞遣摧鋒軍統制王資、踏白軍統制白再興速趨雞冠隘,左軍統制王進據陽平關。友聞登溪嶺,手執五方旗,指麾甫畢,兵數萬突至陽平關,遂遣進及游奕部將王剛出戰,又親帥帳兵及背嵬軍突出陣前,左
 右馳射。兵退,友聞謂忠義總管陳庚及當可曰:「敵必旋兵攻雞冠隘,宜急援之。」既而果以步騎萬餘攻隘,庚以騎兵五百直前決戰,當可將步兵左右翼並進,王資、白再興又自隘出戰,蹀血十餘里,兵乃解去。特授友聞眉州防禦使,依舊左驍衛大將軍、利州駐扎御前諸軍統制,兼沔州駐扎,兼管關外四州安撫,權知沔州,節制本府屯戍軍馬。弟萬差知同慶府、四川制置司帳前總管,仍舊總管忠義軍馬,節制屯戍軍馬,董仙駐扎,專與沔、
 利兩司同共任責措置邊面。



 明年,友聞引兵扼仙人關。諜聞北兵合西夏、女真、回回、吐蕃、渤海軍五十餘萬大至,友聞語萬曰:「國家安危,在此一舉,眾寡不敵,豈容浪戰。惟當乘高據險,出奇匿伏以待之。」北兵先攻武休關,敗都統李顯忠軍,遂入興元,欲沖大安。制置使趙彥吶檄友聞控制大安以保蜀口。友聞馳書彥吶曰:「沔陽,蜀之險要,吾重兵在此,敵有後顧之憂,必不能越沔陽而入蜀。又有曹萬、王宣首尾應援,可保必捷。大安地勢平
 壙,無險可守,正敵騎所長,步兵所短,況眾寡不敵,豈可於平地控御。」彥吶不以為然,一日持小紅牌來速者七。友聞議為以寡擊眾,非乘夜出奇內外夾擊不可。乃遣萬、友諒引兵上雞冠隘,多張旗幟,示敵堅守。友聞選精銳萬人夜渡江,密往流溪設伏。約曰:「敵至,內以鳴鼓舉火為應,外呼殺聲。」北兵果至,萬出逆戰,敵將八都魯擁萬餘眾,達海帥千人往來搏戰,矢石如雨。萬身被數創,令諸軍舉烽。友聞遣選鋒軍統制楊大全、游奕軍統制
 馮大用引本部出東菜園,擊敵後隊;敢勇軍總管夏用、知西和州神勁軍總管趙興帥所部出水嶺,擊敵中隊;知天水軍安邊軍總管呂嗣德、陳庚率所部出龍泉頭,擊敵前隊。友聞親帥精兵三千人,疾馳至隘下,先遣保捷軍統領劉虎帥敢死士五百人沖前軍,前軍不動,大兵伏三百騎道旁,虎眾銜枚突戰。會大風雨,諸將請曰:「雨不止,淖濘深沒足,宜俟少霽。」友聞斥曰:「敵知我伏兵在此,緩必失機。」遂擁兵齊進。友聞入龍尾頭,萬聞之,五
 鼓出隘口,與友聞會。內外兩軍皆殊死戰,血流二十里。西軍素以綿裘代鐵甲,經雨濡濕,不利步門。黎明,大兵益增,乃以鐵騎四面圍繞,友聞嘆曰:「此殆天乎!吾有死而已。」於是極口詬罵,殺所乘馬以示必死。血戰愈厲,與弟萬俱死,軍盡沒,北兵遂長驅入蜀。



 秦鞏人汪世顯素服友聞威望,嘗以名馬遺友聞,還師過戰地,嘆曰:「蜀將軍真男兒漢也。」盛禮祭之。事聞,特贈龍圖閣學士、大中大夫,賜廟褒忠,謚曰毅節,官其二子承務郎,婿迪功郎。萬
 特贈武翼大夫,二子成忠郎。



 陳寅,寶謨閣待制咸之子。漕司兩貢進士,以父恩補官,歷官州縣。紹定初,知西和州。西和極邊重地,寅以書生義不辭難。北兵入境,屬都統何進出守大安,獨統制官王銳與忠義千人城守而已。寅誓與其民共守此土。居民始以進留家城中,恃以為固,已而進徙它郡,遂無固志。寅獨留其二子並闔門二十八口,曰:「人各顧其家,將誰共守。」乃散資財以結忠義,為必守之計。



 北兵十萬攻
 城東南門,以降者為先驅。寅草檄文喻之,自執旗鼓,激厲將士,迎敵力戰,矢石如雨。師退,詰旦,增兵復來,寅帥忠義民兵與敢死士力戰,晝夜數十合,兵退。制置司以寅功遍告列郡。北兵伐木為攻具,增兵至數十萬,圍州城。進素與寅不協,寅有功,尤為諸將所忌。至是求援甚急,久之,制置司才遣劉銳及忠義人陳瑀等往救,率皆觀望不進,銳甫進七方關,瑀未及仇池,皆以路梗告。寅率民兵晝夜苦戰,援兵不至,城遂陷。



 寅顧其妻杜氏曰:「
 若速自為計。」杜厲聲曰:「安有生同君祿,死不共王事者?」即登高堡自飲藥。二子及婦俱死母傍。寅斂而焚之,乃朝服登戰樓,望闕焚香,號泣曰:「臣始謀守此城,為蜀藩籬,城之不存,臣死分也。臣不負國!臣不負國!」再拜伏劍而死。賓客同死者二十有八人。一子後至,亦欲自裁,軍士抱持之曰:「不可使忠臣無後。」與俱縋城,亦折足死。制置司以聞,詔特贈朝議大夫、右文殿修撰,賜錢三千緡,即其所居鄉、所守州立廟。久之,加贈華文閣待制,謚襄
 節。



 賈子坤字伯厚,潼川懷安軍人。嘉定十三年進士。為西和推官,攝通判。關外被兵,子坤與郡守陳寅誓死城守。城陷,子坤朝服與其家十二口死之。追贈承議郎,封其父崧承務郎。官其子仲武宣教郎、隆州簽判,改奉議郎、果州通判,卒。



 仲武子昌忠、純孝,同登咸淳七年進士第。純孝揚州教授,受知帥李庭芝,調江、淮總幕。北兵下江南,二王在福州,以史館檢閱召,辭。會丞相文天祥闢佐
 其幕,尋授秘書丞,擢吏部郎中。丁母憂,起復為右司,轉朝散郎。崖山師敗,純孝抱二女偕妻牟同蹈海死。



 劉銳,知文州。嘉熙元年,北兵來攻,銳與通判趙汝曏乘城固守,率軍民七千餘人晝夜搏戰,殺傷甚多。拒守兩月餘,援兵不至,城中無水,取汲於江。會陳昱以去歲失守沔,編置此州,夜逾城出降,獻女大將,告以虛實,敵遂增兵攻城甚急,一夕移江流於數里外。銳度不免,集其家人,盡飲以藥,皆死,乃聚其尸及公私金帛、告命焚之。
 家素有禮法,幼子同哥才六歲,飲以藥,猶下拜受之,左右為之感慟。



 汝曏宣城人,善射。城破被執,先斷其兩臂,而後臠殺之。銳及其二子自刎死,軍民死者數萬人。



 蹇彞,潼川通泉人。嘉定二年進士。累官通判金州。端平三年,北兵攻蜀,彞堅守,戰不能敵,被擒,不屈而死。其子永叔復力戰,城破,舉家死焉。弟維之,紹定五年進士。利州都統王宣闢行參軍事,亦迎敵力戰而死,特官其子。



 何充,漢州德陽人。秘書監耕之孫。通判黎州,攝州事,預
 為備御計。及宋能之至,建議急於邛崍創大小兩關倉及砦屋百間,親督程役。俄關破,充自刺不死,大軍帥呼之語,許以不殺。充曰:「吾三世食趙氏祿,為趙氏死不憾。」帥設帟幄環坐諸將,而虛其賓席,呼充曰:「汝能降,即坐此。」充踞坐地求死,遂罷。它日又呼之,欲辮其發而髡其頂。曰:「可殺不可髡。」又使署招民榜,充曰:「吾監州也,可聚吾民使殺之耶?即一家有死而已,榜必不可署。」大將遺以酒茗羊牛肉,皆卻之。自是水飲絕不入口。敵知其不
 可強,將剮之,大將曰:「此南家好漢也,使之即死。」於是斬其首。



 充妻陳罵不絕口。初,充之見呼也,陳必以一家往。帥曰:「不呼汝,何以來?」陳曰:「吾求死爾。」及充死,東望再拜曰:「臣夫婦雖死,可以對趙氏無愧矣。」眾以石擊殺之。



 方充夫婦之嬰禍也,親戚勸其茍免,充正色曰:「我夫婦與兒婦義同死,汝等自求生可也。」於是上下感泣,願同死者四十餘人。男士麟、孫駒行、從子仲桂先充而死,惟長子士龍得免。



 許彪孫,顯謨閣學士奕之子也。為四川制置司參謀官。景定二年,劉整叛,召彪孫草降文,以潼川一道為獻。彪孫辭使者曰:「此腕可斷,此筆不可書也。」即閉門與家人俱仰藥死。



 整既降,遂引兵襲都統張桂營,桂及統制金文德戰死。納溪曹贛闔門死之。景定四年,沔州都統胡世全護糧運至虎象山,遇敵兵戰敗死。咸淳二年,北兵取開州,守將龐彥海死之。德祐元年,瀘守梅應春殺判官李丁孫、推官唐奎瑞以城降,珍州守將江彥清巷戰
 死之。



 陳隆之,不知所仕履。為四川制置使。淳祐元年十一月,成都被圍,守彌旬,弗下。部將田世顯乘夜開門,北兵突入,隆之舉家數百口皆死。檻送隆之至漢州,命諭漢州守臣王夔降,隆之呼夔語之曰:「大丈夫死爾,毋降也。」遂見殺。後五年,提刑袁簡之上其事,特贈徽猷閣待制,合得恩澤外,特與兩子恩澤,賜謚立廟。



 又有史季儉者,威州棋城主簿也。成都之陷,子良震與婿楊城夫爭相為
 死,各特贈兩官,與一子下州文學。



 王翊,字公輔,郫縣人。寶慶元年進士。吳曦嘗招之入幕,及曦以蜀叛,抗節不拜,為陳大義。曦怒,囚翊,欲烹之,曦誅而免。



 嘉熙元年,制置使丁黼闢為參議官,先遣其家歸鄉里,為文訣先墓,誓以身死報國。及北兵至,帳前提舉官成駒先走,黼倉卒迎敵,敗死。翊與司理王璨、運司干官李日宣等募兵拒守。兵入公署,見翊朝服危坐,問為何人,曰:「小官食天子之祿,臨難不能救,死有餘罪,可
 速殺我。」又問何以不走,曰:「願與此城俱亡。」北兵相謂曰:「忠臣也。」戒勿殺。敵縱火大掠,翊以朝服赴井死。兵後,其家出其尸井中,衣冠儼如也。轉運副使蒲東卯死之。



 兵屠漢州,權州事劉當可、判官邵復、錄事參軍羅由、司戶參軍趙崇啟、知雒縣羅君文皆不屈而死。復,雍六世孫也。入眉州,知丹棱縣馮仲燁死之。取簡州,簡守李大全死之。邛守趙晨親率雅州牌手出戰,力盡而死。



 文州守劉銳、通判趙汝曏相誓死守,更迭出戰,被圍旬有五
 日,汲道絕,兵民水不入口者半月,至吮妻子之血,卒無叛志。城垂陷,汝曏猶提雙刃入陣,中十六矢,被執以死。銳先殺其妻,父子三人登文王臺自刎死。師至遂寧,民兵趙朋拒戰,左臂已斷,而戰不休。



 至重慶,進士胡天啟負母而逃,兵欲殺其母,天啟妻張哀號願以身代,不聽,卒殺之。天啟與其妻呼天大罵,大將奇天啟貌,欲活之,謂之曰:「汝從我,當共富貴。」天啟愈奮罵,於是夫婦同死。事聞,翊、汝曏皆立廟賜謚,餘褒恤有差。



 寶祐六年,北兵
 拔吉平隘,守將楊禮、周德榮死之。拔長寧,守將王佐父子俱死。至閬州,推官趙廣死之。至蓬州,轉運使施擇善死之。至順慶,。帥守段元鑒城守,麾下劉淵殺之以降。



 李誠之字茂欽,婺州東陽人。受學呂祖謙。鄉舉第一,後入太學,舍選亦第一。慶元初,釋褐為饒州教授。丁父母憂,廬墓終喪。乾辦福建安撫司公事,遷刑、工部架閣,擢國子學錄,以言罷。



 起為江西轉運司干辦。使稱提會子,第其物力高下輸錢以斂之,誠之以為擾。使者不悅曰:「
 商君之令,猶能必行,今乃齟齬如此。」誠之愀然曰:「使君儒者,而欲效商君之所為乎?」遂辭去。使者遜謝,罷令而後止。



 改通判常州,知郢州。知金人必敗盟,大修邊防戰攻守禦之具。移知蘄州。蘄自南渡以來,未嘗被兵,誠之曰:「備御無素,長驅而來,將若之何?」相視城壁而增益之,備樓櫓,築羊馬墻,教閱廂禁民兵,激之以賞,積粟四萬。先是,酒庫月解錢四百五十千以獻守,誠之一無所受,寄諸公帑,以助兵食。



 嘉定十四年二月,金人犯淮南。時
 誠之已逾滿,代者不至,欲先遣其孥歸,聞難作而止。喟然謂其僚曰:「吾以書生再任邊壘,行年七十,抑又何求,獨欠一死爾。當與同僚戮力以守,不濟則以死繼之。」乃選丁壯分布城守,募死士迎擊,遇於橫槎橋,大破之。居數日,金人擁眾臨沙河,欲渡,又破之。明日,金兵大至,決湟水,焚戰樓,又拒退之。明日,金移兵要沖,為必渡計,蘄兵直前奮擊,殺其酋帥。金人雖屢挫,然謀益巧,攻益力。未幾,傅城下,圍之數重,遂燔木柵。誠之出兵御之,又殺
 其將卒數十人,奪所佩印。三月朔,金人攻西門,射卻之。俄造望樓以窺城,誠之為疑兵以示之。又使持書來脅降,誠之戮之,而還其書。越二日,金人以攻具進,誠之設械御之,夜出搗其營。料敵應變若熟知兵者,金人卒不得志。



 會黃州失守,並兵為一,凡十餘萬。池陽、合肥援兵敗走,朝命馮榯援二郡,榯至境,遷延不進。誠之激厲將士,勉以忠義。城陷,率兵巷戰,殺傷相當。子士允力戰死,誠之引劍將自剄,呼其孥曰:「城已破,汝等宜速死,無辱!」
 妻許及婦若孫皆赴水死。事聞,贈朝散大夫、秘閣修撰,封正節侯,立廟於蘄,賜名褒忠,賻銀絹二百,仍賜爵迪功郎者三,贈其妻令人,士允通直郎,子婦及孫女之沒於難者皆贈安人。從誠之之死者,通判州事秦鉅。



 秦鉅字子野,丞相檜曾孫。通判蘄州。金人犯境,與郡守李誠之協力捍禦。求援於武昌、安慶,月餘,兵不至。策應兵徐揮、常用等棄城遁。城破,鉅與誠之各以自隨之兵巷戰,死傷略盡。鉅歸署,疾呼吏人劉迪,令火諸倉庫,乃
 赴一室自焚。有老卒見煙焰中著白戰袍者,識其鉅也,冒火挽出之。鉅叱曰:「我為國死,汝輩可自求生。」制衣就焚而死。次子浚先往四祖山,兵至亟還,與弟水翬從父俱死。特贈鉅五官、秘閣修撰,封義烈侯,與誠之皆立廟蘄州,賜額褒忠,贈浚、瀈通直郎,賻以銀絹各二百。



 州學教授阮希甫贈通直郎,防禦判官趙汝標、蘄春主簿寧時鳳、錄事參軍兼司戶杜諤俱贈承務郎,監蘄州都大監轄蘄口鎮倉庫嚴剛中贈承事郎。



 時統制官孫中,小將
 江士旺、陳興、曹全、兵卞,軍士李斌等皆鬥死。司理參軍趙與裕先率民兵百餘人奪關出外求援,僅以身免,而全家十六人皆沒。淳祐十二年,特封鉅義烈顯節侯。黃州之陷,守臣何大節亦投江死焉。



\end{pinyinscope}