\article{列傳第一百一十六}

\begin{pinyinscope}

 何
 灌李熙靖王云譚世績梅執禮程振劉延慶



 何灌,字仲源,開封祥符人。武選登第,為河東從事。經略使韓縝雖數試其材,而常沮抑之,不假借。久乃語之曰:「
 君奇士也,他日當據吾坐。」為府州、火山軍巡檢。盜蘇延福狡悍,為二邊患,灌親梟其首。賈胡□有泉,遼人常越境而汲,灌親申畫界堠,遏其來,忿而舉兵犯我。灌迎高射之,發輒中,或著崖石皆沒鏃,敵驚以為神,逡巡斂去。後三十年,契丹蕭太師與灌會,道曩事,數何巡檢神射,灌曰:「即灌是也。」蕭矍然起拜。



 為河東將,與夏人遇,鐵騎來追,灌射皆徹甲,至洞胸出背,疊貫後騎,羌懼而引卻。知寧化軍、豐州,徙熙河都監,見童貫不拜,貫憾焉。張康
 國薦於徽宗,召對,問西北邊事,以笏畫御榻,指坐衣花紋為形勢。帝曰:「敵在吾目中矣。」



 提點河東刑獄,遷西上閣門使、領威州刺史、知滄州。以治城鄣功,轉引進使。詔運粟三十萬石於並塞三州,灌言:「水淺不勝舟,陸當用車八千乘,沿邊方登麥,願以運費增價就糴之。」奏上,報可。安撫使忌之,劾雲板築未畢而冒賞,奪所遷官,仍再貶秩,罷去。



 未幾,知岷州,引邈川水溉間田千頃,湟人號廣利渠。徙河州,復守岷,提舉熙河蘭湟弓箭手。入言:「漢金
 城、湟中穀斛八錢,今西寧、湟、廓即其地也,漢、唐故渠尚可考。若先葺渠引水,使田不病旱,則人樂應募,而射士之額足矣。」從之。甫半歲,得善田二萬六千頃,募士七千四百人,為他路最。童貫用兵西邊,灌取古骨龍馬進武軍,加吉州防禦使,改知蘭州。又攻仁多泉城,炮傷足不顧,卒拔城,斬首五百級。尋改廓州防禦使。



 宣和初,劉法陷於敵,震武危甚,熙帥劉仲武使灌往救。灌以眾寡不敵,但張虛聲駭之,夏人宵遁。灌恐覘其實,遽反兵,仲武
 猶奏其逗遛,罷為淮西鈐轄。從平方臘,獲賊帥呂師囊,遷同州觀察使、浙東都鈐轄,改浙西。



 童貫北征,檄統制兵馬,涿、易平,以知易州,遷寧武軍承宣使、燕山路副都總管,又加龍、神衛都指揮使。夔離不取景州,圍薊州。貫諉以兵事,即復景城,釋薊圍。郭藥師統蕃、漢兵,灌白:「頃年折氏歸朝,朝廷別置一司,專部漢兵,至於克行,乃許同營。今但宜令藥師主常勝軍,而以漢兵委灌輩。」貫不聽。召還,管幹步軍司。



 陪遼使射玉津園,一發破的,再發
 則否。客曰:「太尉不能耶?」曰:「非也,以禮讓客耳。」整弓復中之,觀者誦嘆,帝親賜酒勞之。遷步軍都虞候。



 金師南下,悉出禁旅付梁方平守黎陽。灌謂宰相白時中曰:「金人傾國遠至,其鋒不可當。今方平掃精銳以北,萬有一不枝梧,何以善吾後,盍留以衛根本。」不從,明日,又命灌行,辭以軍不堪戰,強之,拜武泰軍節度使、河東河北制置副使。未及行而帝內禪,灌領兵入衛。鄆王楷至門欲入,灌曰:「大事已定,王何所受命而來?」導者懼而退。灌竟行,
 援兵二萬不能足,聽募民充數。



 靖康元年正月二日,次滑州,方平南奔,灌亦望風迎潰。黃河南岸無一人禦敵,金師遂直叩京城。灌至,乞入見,不許,而令控守西隅。背城拒戰凡三日,被創,沒於陣,年六十二。帳下韓綜、雷彥興,奇士也,各手殺數人,從以死。欽宗哀悼,賜金帛,命官護葬。已而言者論其不守河津,追削官秩。



 長子薊,至閣門宣贊舍人。從父戰,箭貫左臂,拔出之,病創死。紹興四年,中子蘚以灌事泣訴於朝,詔復履正大夫、忠正軍承
 宣使。



 李熙靖,字子安,常州晉陵人,唐衛公德裕九世孫也。祖均、父公弼皆進士第。公弼,崇寧初通判潞州,以議三舍法不便,使者劾其沮格詔令,坐削黜以死。熙靖擢第,又中詞學兼茂,選為闢雍錄、太學正,升博士。以父老丐外,除提舉淮東學事便養,命下,乃得河東;而為淮東者,臧祐之也。蓋省吏取祐之賂,輒易之。或教使自言,熙靖曰:「事君不擇地,吾其可發人之私,求自便也?」宰相聞而賢
 之,留為兵部員外郎。遭父憂去,還,為右司員外郎。



 王黼以太宰領應奉司,又方事燕云,立經撫房於中書獨專之,他執政皆不得預。熙靖與言曰:「應奉之職,非宰相所當預。尚書、樞密皆有兵房,足以治疆事,經撫何為者哉?」黼積不樂。同列五人皆躐躋禁從,獨滯留四年。都水丞失職,移過於熙靖,貶其兩秩,又將左轉為國子司業,執政交言不可,僅遷太常少卿。黼罷,乃拜中書舍人,蔡攸又惡之,出知拱州。



 越兩月,復以故官召,入對言:「燕山雖
 定,宜益謹思患豫防之戒。」徽宗曰:「《詩》所謂『迨天之未陰雨,徹彼桑土,綢繆牖戶』者是也。」熙靖進曰:「孔子云:『為此詩者,其知道乎!能治其國家,誰敢侮之?』願陛下為無疆之計。」帝嘉之。



 靖康初,同譚世績事龍德宮,改顯謨閣待制、提舉醴泉觀。道君待之甚厚,常從容及內禪事,曰:「外人以為吳敏功,殊不知此自出吾意耳,吾茍不欲,人言且滅族,誰敢哉?或謂吾似唐睿宗上畏天戒,故為之,吾有此心久矣。」熙靖再拜賀。敏聞而忌之,以進對不時受
 罰。



 既拒張邦昌之命,憂憤廢食,家人進粥藥寬譬之,終無生意。故人視其病,相持啜泣,索筆書唐王維所賦「百官何日再朝天」之句,明日遂卒,年五十三,與世績同贈端明殿學士。



 王云,字子飛,澤州人。父獻可,仕至英州刺史、知瀘州。黃庭堅謫於涪,獻可遇之甚厚,時人稱之。雲舉進士,從使高麗,撰《雞林志》以進。擢秘書省校書郎,出知簡州,遷陜西轉運副使。宣和中,從童貫宣撫幕,入為兵部員外郎、
 起居中書舍人。



 靖康元年,以給事中使斡離不軍,議割三鎮以和。使還,傳道斡離不之意,以為黏罕得朝廷所與余睹蠟書,堅云中國不可信,欲敗和約。執政以為不然,罷為徽猷閣待制、知唐州。



 金人陷太原,召拜刑部尚書,再出使,許以三鎮賦入之數。雲至真定,遣從吏李裕還言:「金人不復求地,但索五輅及上尊號,且須康王來,和好乃成。」欽宗悉從之,且命王及馮澥往。未行,而車輅至長垣,為所卻,雲亦還。澥奏言云誕妄誤國,云言:「事勢
 中變,金人必欲得三鎮,不然,則進兵取汴都。」中外震駭,詔集百官議,雲固言:「康王舊與斡離不結歡,宜將命。」帝慮為所留,雲曰:「和議既成,必無留王之理,臣敢以百口保之。」王遂受命,而云以資政殿學士為之副。



 頃云奉使過磁、相,勸兩郡徹近城民舍,運粟入保,為清野之計,民怨之。及是,次磁州,又與守臣宗澤有憾。於是王出謁嘉應神祠,雲在後,民遮道諫曰:「肅王已為金人所留,王不宜北去。」厲聲指雲曰:「清野之人,真奸賊也。」王出廟行,或
 發雲笥,得烏絁短巾,蓋云夙有風眩疾,寢則以護首者。民益信其為奸,噪而殺之。王見事勢洶洶,乃南還相州。是役也,云不死,王必北行,議者以是驗天命云。建炎初,贈觀文殿學士。



 雲兄霽,崇寧時,為謀議司詳議官,上書告蔡京罪,黥隸海島。欽宗復其官,從種師中戰死。



 譚世績字彥成,潭州長沙人。第進士,教授郴州。時王氏學盛行,世績雅不喜。或問之,曰:「說多而屢變,無不易之論也。」置其書不觀。又中詞學兼茂科,除秘書省正字。時
 相蔡京子攸領書局,同舍郎多翕附以取貴仕。世績獨坐直廬,翻書竟日。梁師成之客與為鄰居,數致師成願交意,謝不答。



 在館六年不遷,京罷,用久次為司門員外郎。又三年,遷吏部。京復相,嫌不附己,罷提點太平宮。久之,復還吏部。幸臣妄引恩澤任子,持不與。吏白有某例,世績曰:「豈當以暫例破成法!」已而取中旨行之。進少府監,擢中書舍人,以謹命令、惜名器、廣言路、吝賜予、正上供、省浮費六事言於上,又為當路所嫉。以徽猷閣待制
 知婺州,未行,復留之。



 徽宗禪位東幸,且還,使與李熙靖副執政奉迎,遂同主管龍德宮。請辨正宣仁國史之謗,述欽聖遺旨以復瑤華,大享神祖仍用富弼侑食,釋奠先聖不當以王安石配,後皆施行。



 秋七月,彗出東方,大臣或謂此四夷將衰之兆,世績面奏:「垂象可畏,當修德以應天,不宜惑諛說。」進給事中兼侍讀。內侍喧爭殿門,詔以贖論,世績駁其不恭,因言:「童貫輩初亦甚微,小惡不懲,將馴至大患。」疏入,同類側目。何慄建議分外郡為
 四道,置都總管,事得顓決。世績言:「裂天下以付四人,而王畿所治者才十六縣,獨無尾大不掉之慮乎?」慄不樂。改禮部侍郎。



 金騎駸駸南下,世績言:「守邊為上策;今邊不得守,守河則京畿自固,中策也;巡幸江、淮,會東南兵以捍敵,下策也。金人既渡河,又請遣大將秦元以所部京畿保甲,分護國門,使兵勢連屬,首尾相援,即金人不敢逼。孫傅深然之,又格於慄議。再扈車駕至金帥帳,以十害說其用事者,言講解之利,詞意忠激,金人聳聽。



 張
 邦昌僭國,令與李熙靖同直學士院,皆稱疾臥不起,以憂卒,年五十四。建炎初,褒其守節,贈端明殿學士。



 梅執禮,字和勝,婺州浦江人。第進士,調常山尉未赴,以薦為敕令刪定官、武學博士。大司成強淵明賢其人,為宰相言,相以未嘗識面為慊。執禮聞之曰:「以人言而得,必以人言而失,吾求在我者而已。」卒不往謁。



 歷軍器、鴻臚丞,比部員外郎,比部職勾稽財貨,文牘山委,率不暇經目。苑吏有持茶券至為錢三百萬者,以楊戩旨意迫
 取甚急。執禮一閱,知其妄,欲白之,長貳疑不敢,乃獨列上,界詐也。改度支、吏部,進國子司業兼資善堂翊善,遷左司員外郎,擢中書舍人、給事中。



 林攄以前執政赴闕宿留,冀復故職,執禮論去之。孟昌齡居鄆質人屋,當贖不肯與,而請中旨奪之,外郡卒留役中都者萬數,肆不逞為奸,詔悉令還,楊戩占不遣;內侍張祐董葺太廟,僭求賞:皆駁奏弗行。遷禮部侍郎。



 素與王黼善,黼嘗置酒其第,誇示園池妓妾之盛,有驕色。執禮曰:「公為宰相,當
 與天下同憂樂。今方臘流毒吳地,瘡痍未息,是豈歌舞宴樂時乎?」退又戒之以詩。黼愧怒,會孟饗原廟後至,以顯謨閣待制知蘄州,又奪職。



 明年,徙滁州,復集英殿修撰。時賦鹽虧額,滁亦苦抑配。執禮曰:「郡不能當蘇、杭一邑,而食鹽乃倍粟數,民何以堪?」請於朝,詔損二十萬,滁人德之。



 欽宗立,徙知鎮江府,召為翰林學士,道除吏部尚書,旋改戶部。方軍興,調度不足,執禮請以禁內錢隸有司,凡六宮廩給,皆由度支乃得下。嘗有小黃門持中
 批詣部取錢,而封識不用璽,既悟其失,復取之。執禮奏審,詔責典寶夫人而杖黃門。



 金人圍京都,執禮勸帝親征,而請太上帝後、皇后、太子皆出避,用事者沮之。洎失守,金人質天子,邀金帛以數百千萬計,曰:「和議已定,但所需滿數,則奉天子還闕。」執禮與同列陳知質、程振、安扶皆主根索,四人哀民力已困,相與謀曰:「金人所欲無藝極,雖銅鐵亦不能給,盍以軍法結罪,儻窒其求。」而宦者挾宿怨語金帥曰:「城中七百萬戶,所取未百一,但許
 民持金銀換粟麥,當有出者。」已而果然。酋怒,呼四人責之,對曰:「天子蒙塵,臣民皆願致死,雖肝腦不計,於金繒何有哉?顧比屋枵空,亡以塞命耳。」酋問官長何在,振恐執禮獲罪,遂前曰:「皆官長也。」酋益怒,先取其副胡舜陟、胡唐老、姚舜明、王俁,各杖之百。執禮等猶為之請,俄遣還,將及門,呼下馬撾殺之,而梟其首,時靖康二年二月也。是日,天宇晝冥,士庶皆隕涕憤嘆。



 初,車駕再出,執禮與宗室子昉、諸將吳革等謀集兵奪萬勝門,夜搗金帥
 帳,迎二帝以歸。而王時雍、徐秉哲使範瓊洩其謀,故不克。死時,年四十九。高宗即位,詔贈通奉大夫、端明殿學士。議者以為薄,復加資政殿學士。



 程振,字伯起,饒州樂平人。少有軼材,入太學,一時名輩多從之游。徽宗幸學,以諸生右職除官,為闢雍錄,升博士,遷太常博士,提舉京東、西路學事。請立廟於鄒祀孟軻,以公孫丑、萬章、樂正克等配食,從之。



 提舉京西常平,入為膳部員外郎、監察御史、闢雍國子司業、左司員外
 郎兼太子舍人。始至,即言:「古者大祭禮登餕受爵,必以上嗣,既《禮經》所載,且元豐彞典具存。昨天子展事明堂,而殿下不預,非所以尊宗廟、重社稷也。」太子矍然曰:「宮僚初無及此者。」由是特加獎異。



 方臘起,振謂王黼宜乘此時建革天下弊事,以上當天意,下順人心。黼不懌,曰:「上且疑黼挾寇,奈何?」振知黼忌其言,趨而出,然太子薦之甚力,遂擢給事中。黼白振資淺,且雅長書命,請以為中書舍人。侍郎馮熙載出知亳州,黼怨熙載,欲振詆以
 醜語,振不肯。黼使言者劾為黨,罷提舉沖祐觀。居三年,復還故官。



 靖康元年,進吏部侍郎,為欽宗言:「柄臣不和,論議多駁,詔令輕改,失於事幾。金人交兵半歲,而至今不解者,以和戰之說未一故也。裁抑濫賞,如白黑易分,而數月之間,三變其議,以私心不除,各蔽其黨故也。今日一人言之,以為是而行;明日一人言之,以為非而止。或聖斷隃度而不暇疇咨,或大臣偏見而遂形播告,所以動未必善,處未必宜,乃輒為之反汗,其勢不得不爾
 也。」



 時金兵至河北,振請糾諸道兵掎角擊之,曰:「彼猖獗如此,陛下尚欲守和議,而不使之少有懲艾乎?」上嗟味其言,而牽於外廷,不能用。拜開封尹。故時,大闢有情可矜,多奏取原貸;崇寧以來,議者謂輦轂先彈壓,率便文殺之。振請復舊制。詔捕亡命卒,得數千人,振請以隸步軍而除其罪。步軍司欲論如法,振曰:「方多事之際,而一日殺數千人,必大駭觀聽。」乃盡釋之。改刑部侍郎。



 金騎在郊,邀車駕出城,振為何慄言:「宜思所以折之之策。」慄
 不從。未幾,及於難,年五十七。金人去,從子庭訪得其首歸葬之。初,王黼使其客沉積中圖燕,振戒以後禍,積中懼而言不可。既而振乃用是死,聞者痛之。



 初,宣和崇道家之說,振侍坐東宮,從容言:「孔子以《鴟鴞》之詩為知道,其詞不過曰『迨天之未陰雨,綢繆牖戶』而已。老子亦云:『為之於未有,治之於未亂。』今不固根本於無事之時,而事目前區區,非二聖人意。」他日,太子為徽宗道之。徽宗寤,頗欲去健羨,疏左右近習,而宦寺楊戩輩方大興宮
 室,懼不得肆,因讒家令楊馮,以為將輔太子幸非常。徽宗震怒,執馮誅之,而太子之言亦廢。振尹京時,兩宮方困於惎間,振極意彌縫,治龍德梁忻獄,寬其罪,不使有纖介可指。



 高宗即位,進秩七等,仍官其子及親屬三人,又贈端明殿學士。端平初,曾孫東請謚,賜謚剛愍。同時死者禮部侍郎陳知質,失其傳;給事中安扶,附見父《安燾傳》。



 劉延慶,保安軍人。世為將家,雄豪有勇,數從西伐,立戰
 功,積官至相州觀察使、龍神衛都指揮使、鄜延路總管。遷泰寧軍節度觀察留後,改承宣使。破夏人成德軍,擒其酋賞屈,降王子益麻黨征。拜保信軍節度使、馬軍副都指揮使。從童貫平方臘,節度河陽三城。又從北伐,以宣撫都統制督兵十萬,渡白溝。



 延慶行軍無紀律,郭藥師扣馬諫曰:「今大軍拔隊行而不設備,若敵人置伏邀擊,首尾不相應,則望塵決潰矣。」不聽。至良鄉,遼將蕭乾帥眾來,延慶與戰,敗績,遂閉壘不出。藥師曰:「乾兵不過
 萬人,今悉力拒伐,燕山必虛,願得奇兵五千,倍道襲取,令公之子三將軍簡師為後繼。」延慶許之,遣大將高世宣與藥師先行,即入燕城,幹舉精甲三千巷戰。三將軍者,光世也。



 渝約不至,藥師失援敗走,世宣死之。延慶營於盧溝南,乾分兵斷餉道,擒護糧將王淵,得漢軍二人,蔽其目,留帳中,夜半偽相語曰:「聞漢軍十萬壓吾境,吾師三倍,敵之有餘。當分左右翼,以精兵沖其中,左右翼為應,殲之無遺。」陰逸其一人歸報。明旦,延慶見火起,以
 為敵至,燒營而奔,相蹂踐死者百餘里。自熙、豐以來,所儲軍實殆盡。退保雄州,燕人作賦及歌誚之。朝議延慶喪師,不可不行法,坐貶率府率,安置筠州。契丹知中國不能用兵,由是輕宋。



 未幾,復為鎮海軍節度使。靖康之難,延慶分部守京城,城陷,引秦兵萬人奪開遠門以出,至龜兒寺,為追騎所殺。光世自有傳。



 論曰:靖康之變、執禮、振不忍都人塗炭,拒強敵無厭之欲,親逢其兇。熙靖、世績不肯以一身事二姓,悲不食以
 終。灌、延慶戰敗而沒。此數人者,其所遭不同,至於死國難則一而已。雲之死,雖其有以取之,殆亦天未欲絕宋祀也;不然,是行也,康王其危哉!



\end{pinyinscope}