\article{志第一百六十一 藝文七}

\begin{pinyinscope}

 集類四:一曰楚辭類,二曰別集類,三曰總集類,四曰文史類。



 奏議《楚辭》十六卷楚屈原等撰。



 洪興祖《補注楚
 辭》十七卷



 《考異》一卷



 《楚辭》十七卷後漢王逸章句。



 周紫芝《竹坡楚辭贅說》一卷晁補之《續楚辭》二十卷



 朱熹《楚辭集注》八卷《辨證》一卷



 又《變離騷》一卷



 黃銖《楚辭協韻》一卷



 黃伯思《翼騷》一卷



 《離騷》一卷錢杲之集傳。右楚辭十二部,一百四卷。



 《董仲舒集》一卷



 《枚乘集》一卷



 《劉向集》五卷



 《王褒集》五卷



 《揚雄集》六卷



 又《二十四箴》二卷



 《李尤集》二卷



 《張衡集》六卷



 《張超集》三卷



 《蔡邕集》十卷



 《諸葛亮集》十四卷



 《曹植集》
 十卷



 《魏文帝集》一卷



 《王粲集》八卷



 《陳琳集》十卷



 《嵇康集》十卷



 《阮林集》十卷



 《張華集》二卷



 又《詩》一卷



 《江統集》一卷



 《傅玄集》一卷



 《束皙集》一卷



 《張敏集》二卷



 《潘岳集》七卷



 《索靖集》一卷



 《劉琨集》十卷



 《陸機集》十卷



 《陸雲集》十卷



 《郭璞集》六卷



 《蘭亭詩》一卷



 《陶淵明集》十卷



 《謝莊集》一卷



 《顏延之集》五卷



 《謝靈運集》九卷



 《謝惠連集》五卷



 《王僧達集》十卷



 《鮑昭集》十卷



 《江淹集》十卷



 《王融集》七卷



 《孔稚圭集》十卷



 《謝朓集》十卷



 又《詩》一卷



 顏之推《稽聖賦》一卷



 《梁簡文
 帝集》一卷



 《昭明太子集》五卷



 《沉約集》九卷



 又《詩》一卷



 《劉孝綽集》一卷



 《劉孝威集》一卷



 《吳均詩集》三卷



 《何遜詩集》五卷



 《庾肩吾集》二卷



 《任昉集》六卷



 《庾信集》二十卷



 又《哀江南賦》一卷



 《陳後主集》一卷



 《江總集》七卷



 《沉炯集》七卷



 《徐陵詩》一卷



 《張正見集》一卷



 《唐太宗詩》一卷



 《玄宗詩》一卷



 《王績集》五卷



 《許敬宗集》十卷



 《任敬臣集》十卷



 《宋之問集》十卷



 《沉牷期集》十卷



 《崔融集》十卷



 《李嶠詩》十卷



 《蘇味道詩》一卷



 《杜審言詩》一卷



 《徐鴻詩》一卷



 《王勃詩》八卷



 又《文集》三十卷



 《雜序》一卷



 《楊炯集》二十卷



 又《拾遺》四卷



 《盧照鄰集》十卷



 《駱賓王集》十卷



 《陳子昂集》十卷



 《劉希夷詩》四卷



 《趙彥昭詩》一卷



 《崔湜詩》一卷



 《武平一詩》一卷



 《李乂詩》一卷



 《孫逖集》二十卷



 《張說集》三十卷



 又《外集》二卷



 《蘇頲集》三十卷《張九齡集》二十卷



 《李白集》三十卷



 嚴從《中黃子》三卷



 《毛欽一集》三十卷



 《梁肅集》二十卷



 《李翰集》一卷



 《孟浩然詩》三卷《王昌齡集》十卷



 《崔顥詩》一卷



 《廬象詩》一卷



 《李適詩》一卷



 《陶翰詩》一卷



 《皇甫曾詩》一卷



 《皇
 甫冉集》二卷



 《嚴維詩》一卷



 《祖詠詩》一卷



 《丘為詩》一卷



 《常建詩》一卷



 《岑參集》十卷



 《崔國輔詩》一卷



 《則天中興集》十卷



 又《別集》一卷



 《太宗御集》一百二十卷



 《真宗御集》三百卷《目》十卷



 又《御集》一百五十卷



 《仁宗御集》一百卷《目錄》三卷



 《英宗御制》一卷



 《神宗御筆手詔》二十一卷



 又《御集》一百六十卷



 《哲宗御製前後集》共二十七卷



 《征宗御制崇觀宸奎集》一卷



 又《宮詞》一卷



 駱賓王《百道判》二卷



 《阮籍集》十卷



 李嶠《新詠》一卷



 《阮咸集》一卷



 《吳筠一作集》十一卷



 王道珪注《哀江南賦》一卷



 《杜甫小集》六卷



 《張庭芳注》《哀江南賦》一卷



 薛蒼舒《杜詩刊誤一卷》



 陸淳《東皋子集略》二卷



 元結《元子》十卷



 《魏文正公時務策》五卷又《琦玕子》一卷



 郭元振《九諫書》一卷



 《常袞詔集》二十卷又《安邦策》三卷



 賀知章《人道表》一卷



 李靖《霸國箴》一卷



 《鮑防集》五卷



 王起注《崔融寶圖贊》一卷又《雜感詩》一卷



 《許恭集》十卷



 令狐楚《梁苑文類》三卷



 《任希古集》十卷



 《李司空論事》十七卷



 王勃《舟中纂序》五卷



 《馮宿集》十卷



 廬照鄰《幽優子》三卷



 《
 邵說集》十卷



 杜元穎《五題》一卷



 《李紳批答》一卷



 劉軻《翼孟》三卷



 李德裕《窮愁志》三卷



 又《雜賦》二卷



 《平泉草木記》一卷《段全緯集》五卷



 《薛逢別集》九卷



 《李虞仲制集》四卷



 《柳冕集》四卷



 《李程表狀》一卷



 《李群玉後集》五卷



 又《詩集》二卷



 《令狐綯表疏》一卷



 夏侯韞《與涼州書》一卷



 商璠《丹陽集》一卷



 《舒元輿文》一卷



 《譚正夫文》一卷



 《張□果一作文》一卷



 來擇《秣陵子集》一卷



 又《集》三卷



 《齊夔文》一卷



 《暢當詩》一卷



 皇甫松《大隱賦》一卷



 《於武陵詩》一
 卷



 陸希聲《頤山錄詩》一卷



 《陸鸞集》一卷



 沈棲遠《景臺編》十卷



 《袁皓集》一卷



 黃滔《編略》十卷



 《賈島小集》八卷



 《費冠卿詩》一卷《孟遲詩》一卷



 《王德輿詩》一卷



 鄭谷《宜陽集》一卷



 鬱渾《百篇》一卷



 《周濆詩》一卷



 薛瑩《洞庭詩》一卷



 《李洞詩集》三卷



 《丁棱詩》一卷



 《朱鄴賦》三卷



 又《詩》三卷



 《廬延讓詩集》一卷



 《楊弇詩》一卷



 《賀蘭明吉集》一卷



 《徐融集》一卷



 《韋說詩》一卷



 《劉綺莊集》十卷



 《張琳集》十卷



 《徐杲集》八卷



 《宗嚴集》一卷



 《薛逢賦》四卷



 又《別紙》十三卷



 《宋言賦》一卷



 郭賁《體物集》一卷



 楊復恭《行朝詩》一卷



 《韓偓詩》一卷



 又《人翰林後詩》一卷



 馮涓《懷秦賦》一卷



 又《集》十三卷



 《龍吟集》三卷



 《長樂集》一卷



 朱樸《荊山子詩集》四卷



 又《雜表》一卷



 《孫合小集》三卷



 楊士達《擬諷諫集》五卷



 《陳光詩》一卷



 《吳仁璧詩》一卷



 戚同文《孟諸集》二十卷



 《王振詩》一卷



 嚴虔崧《寶囊》五卷



 又《表狀》五卷



 《倪明基詩》一卷



 《李洪皋集》二卷



 又《表狀》一卷



 《韋文靖箋表》一卷



 崔升魯史分門屬類賦》一卷



 《韋鼎詩》一卷



 《孫該詩》一卷



 《衛單詩》一卷



 《蔡融
 詩》一卷



 《來鵬詩》一卷



 《謝璧賦》一卷



 又《詩集》四卷



 《策林》十卷



 《詠高士詩》一卷



 《沃山焦山賦》一卷



 扈蒙《鰲山集》二十卷



 《毛欽一文》二卷



 《張友正文》一卷



 《南卓集》一卷



 《陳陶文錄》十卷



 封鰲《翰蒿》八卷



 《胡會集》十卷



 《李商隱賦》一卷



 又《雜文》一卷



 《劉鄴集》四卷



 又《叢事》三卷



 《陳一作劉黯集》一卷



 陳汀《五源文集》三卷



 又《賦》一卷



 《張次宗集》六卷



 劉三復《景臺雜編》十卷



 又《問遺集》三卷



 《別集》一卷



 《王嘏集》十卷



 倪曙《獲蒿集》三卷



 又《賦》一卷



 《皮日休別集》七卷



 《陸龜蒙
 詩編》十卷又《賦》一卷



 《錢珝制集》十卷又《舟中錄》二十卷



 《楊夔集》五卷又《賦》一卷



 《冗書》十卷



 《冗餘集》十卷



 鄭昌士《白巖集》五卷



 又《詩集》十卷



 《程遜集》十卷



 溫庭筠《漢南真蒿》十卷



 又《集》十四卷



 《握蘭集》三卷



 《記室備要》三卷



 《詩集》五卷



 崔嘏《管記集》十卷



 蔣文彧《記室定名集》三卷



 盧肇《愈風集》十卷



 又《大統賦注》六卷



 《海潮賦》一卷



 《通屈賦》一卷



 鄭賓一作寶《行宮集》十卷



 張澤《飲河集》十五卷



 《劉宗一作『榮』望制集》八卷



 陸真《禁林集》七卷



 《張玄晏信集》二卷



 《
 高駢集》三卷



 《顧雲集遺》十卷



 又《賦》二卷



 《啟事》一卷



 《苕一作『昭』亭雜筆》五卷



 《篡新文苑》十卷



 《苕一作『昭』川總載》十卷



 康軿《九筆雜編》十五卷



 《樂朋龜集》七卷



 又《綸閣集》十卷



 《徐寅別集》五卷



 《吳融賦集》五卷



 崔致遠《筆耕集》二十卷



 又《別集》一卷



 《崔遘集》二卷



 《羅袞集》二卷



 《李山甫雜賦》二卷



 《李磎集》四卷



 《羊昭業集》十五卷



 章震《肥川集》十卷



 又《磨盾集》十卷



 李景略《南燕染翰》二十卷



 孫合《孫子文篡》四十卷



 《汪文蔚集》三卷



 劉韜美《從軍集》四十
 卷



 《郭子儀表奏》五卷



 《顏真卿集》十五卷



 《元結集》十卷



 《李峴詩》一卷



 《常袞集》三十三卷又《集》十卷



 《韋應物集》十卷



 《高適詩集》十二卷



 《李嘉祏詩》一卷



 《張謂詩》一卷



 《盧綸詩》一卷



 《李端詩》三卷



 《耿緯詩》三卷



 《司空文明集》一卷



 《韓翃詩》五卷



 《錢起詩》十二卷



 《郎士元詩》二卷



 《張繼詩》一卷



 《陸贄集》二十卷



 《王仲舒制集》二卷



 《羊士諤詩》一卷



 《雍裕之詩》一卷



 《裴度集》二卷



 《武元衡詩》三卷



 《權德輿集》五十卷



 《韓愈集》五十卷又《遺文》一卷



 《昌黎文集序傳碑記》一卷



 《
 西掖雅言》五卷



 祝充《韓文音義》五十卷



 朱熹《韓文考異》十卷



 樊汝霖《譜注韓文》四十卷



 洪興祖《韓文年譜》一卷



 《韓文辨證》一卷



 方菘卿《韓集舉正》一卷



 《柳宗元集》三十卷



 張敦頤《柳文音辨》一卷



 《劉禹錫集》三十卷又《外集》十卷



 《呂溫集》十集



 《李觀集》五卷



 《李賀集》一卷又《外集》一卷



 《歐陽詹集》一卷



 《歐陽袞集》一卷



 《張籍集》十二卷



 《孟東野詩集》十卷



 《李翱集》十二卷



 《皇甫湜集》八卷



 《賈島詩》一卷



 《盧仝詩》一卷



 《劉叉詩》一卷



 《沉亞之詩》十二卷



 《樊宗師集》
 一卷



 《吳武陵詩》一卷



 《張碧詩》一卷又《歌行》一卷



 《包幼正詩》一卷



 《朱放詩》二卷



 《符載集》二卷



 《鮑溶歌詩》五卷



 《李益詩》一卷



 《李約詩》一卷



 《熊孺登詩》一卷



 《蔣防集》一卷



 《崔元翰集》十卷



 《張登集》六卷



 《竇叔向詩》一卷



 《竇鞏詩》一卷



 《穆員集》九卷



 《殷堯藩詩》一卷



 《獨孤及集》二十卷



 《張仲素詩》一卷



 《劉言史詩》十卷



 《章孝標集》七卷



 《藏南傑《雜歌行》一卷



 《朱灣詩》一卷



 《張祐詩》十卷



 《李絳文集》六卷



 《元稹集》四十八卷又《元相逸詩》二卷



 《趙陽詩》一卷



 白居易《長慶集》
 七十一卷



 《袁不約詩》一卷



 《施肩吾集》十卷



 《李甘集》一卷



 《朱慶餘詩》一卷



 《李程集》一卷



 王涯《翰林歌詞》一卷



 《令狐楚表奏》二卷又《歌詩》一卷



 《李涉詩》一卷



 《楊巨源詩》一卷



 《喻鳧詩》一卷



 《薛瑩詩》一卷



 《牛僧孺集》五卷



 《李德裕集》二十卷又《別集》十卷



 《記集》二卷



 《姑臧集》五卷德裕翰苑所作。



 《杜牧集》二十卷



 《溫庭筠集》七卷



 《段成式集》七卷



 《薛能詩集》十卷



 《崔嘏制誥》十卷



 《薛逢詩》一卷



 《馬載詩》一卷



 《姚鵠詩》一卷



 《顧況集》十五卷



 《顧非熊詩》一卷



 《裴夷
 直詩》二卷



 《項斯詩》一卷



 劉駕《古風詩》一卷



 《李廓詩》一卷



 《韓宗詩》一卷



 《李遠詩》一卷



 曹鄴《古風詩》二卷



 《許渾詩集》十二卷



 《姚合詩集》十卷



 《李頻詩》一卷



 《李郢詩》一卷



 《雍陶詩集》三卷



 《於鄴詩》十卷



 《陸暢集》一卷



 《劉得仁詩集》一卷



 趙嘏《編年詩》二卷



 《孫樵集》三卷



 《儲嗣宗詩》一卷



 《李鍇詩》一卷



 《鄭巢詩》一卷



 鄭嵎《津陽門詩》一卷



 李殷《古風詩》一卷



 盧肇《文標集》三卷



 《李商隱文集》八卷又《四六甲乙集》四十卷



 《別集》二十卷



 《詩集》三卷



 《劉滄詩》一卷



 《於鵠詩》一
 卷



 《鄭畋集》五卷又《詩集》一卷



 《論事》五卷



 皮日休《文藪》十卷



 《胥臺集》一卷



 《吊江都賦》一卷



 《劉蛻集》十卷



 《李昌符詩》一卷



 侯圭《江都賦》一卷



 《沉光詩集》一卷



 《陸龜蒙集》四卷



 《喻坦之集》一卷



 《周賀詩》一卷



 《曹唐詩》三卷



 《許棠詩集》一卷



 獨孤霖《玉堂集》二十卷



 《李山甫詩》一卷



 胡曾《詠史詩》三卷又《詩》一卷



 《張喬詩》一卷



 《王棨詩》一卷



 于濆《古風詩》一卷



 《聶夷中詩》一卷



 《林寬詩》一卷



 薛廷珪《鳳閣書詞》十卷



 羅虯《比紅兒詩》十卷



 《羅鄴詩》一卷



 羅隱《湘南應用集》
 三卷



 又《淮海寓言》七卷



 《甲乙集》三卷



 《外集詩》一卷



 《啟事》一卷



 《言毚本》三卷



 《言毚書》五卷



 《崔道融集》九卷



 《高駢詩》一卷



 《顧云編蒿》十卷又《鳳策聊華》三卷



 司空圖《一鳴集》三十卷



 《崔塗詩》一卷



 《崔魯詩》一卷



 《林嵩詩》一卷



 《王駕詩》六卷



 《唐彥謙詩集》二卷



 《方乾詩》二卷



 《徐凝詩》一卷



 《周樸詩》一卷



 《陳陶詩》十卷



 《王貞白集》七卷



 陸希聲《君陽遁叟山集記》一卷



 《鄭渥詩》一卷



 鄭雲叟《手疑峰集》二卷



 《杜甫詩》二十卷又《外集》一卷



 《杜詩標題》三卷題鮑氏,不知名。



 《王維集》十卷



 《賈
 至集》十卷又《詩》一卷



 《儲光義集》二卷



 《綦毋潛詩》一卷



 《劉長卿集》二十卷



 《蕭穎士集》十卷



 《李華集》二十卷



 秦系《秦隱君詩》一卷



 《張鼎詩》一卷



 《程晏集》十卷



 《李華集》二十卷



 《張南史詩》一卷



 《陳黯集》一卷



 杜荀鶴《唐風集》二卷



 《嚴郾詩》一卷



 《李溪奏議》一卷



 《吳融集》五卷



 《褚載詩》一卷



 《曹松詩》一卷



 《翁承贊詩》一卷



 《張詩》一卷



 《孫合集》二卷



 《秦韜玉集》三卷



 《鄭谷詩》三卷又《詩》一卷



 《外集》一卷



 韓偓《香奩小集》一卷又《別集》三卷



 《王詩》三卷



 《裴說詩》一卷



 《李雄
 詩》三卷



 《說李中集》三卷



 《李善夷集》六卷



 《黃璞集》五卷



 孫元晏《六朝詠史詩》一卷



 《竇永賦》一卷



 《閻防詩》一卷



 《王季友詩》一卷



 《林藻集》一卷



 《劉憲詩》一卷



 《朱景玄詩》一卷



 《蘇拯詩》一卷



 《王建集》十卷



 《楊炎集》十卷



 《唐於公異奏記》一卷



 《麥信陵詩》一卷



 《劉商集》十卷



 《戎昱集》五卷



 《戴叔倫述蒿》十卷



 《張韋詩》一卷



 《陳羽詩》一卷



 《李慎詩》一卷



 《劉威詩》一卷



 《邵謁詩》一卷



 鄭昌士《四六集》一卷



 《柳倓詩》一卷



 《任翻詩》一卷



 《楊衡詩》一卷



 《文丙詩》一卷



 《皮氏玉笥集》一卷不知
 作者。



 黃滔《莆陽黃御史集》二卷



 《黃寺丞詩》一卷不著名,題唐人。



 《蘆中詩》二卷不知作者。



 李琪《金門集》十卷



 韋莊《浣花集》十卷



 《諫草》一卷



 殷文圭《冥搜集》二十卷又《登龍集》十五卷



 《孫晟集》五卷



 李崧《真珠集》一卷



 高輦《昆玉集》一卷



 《馬幼昌集》四卷



 林鼎《吳江應用》二十卷



 王睿《炙SA子》三卷又《聊珠集》五卷



 周延禧《百一集》二十卷



 《沈文昌集》二十卷



 張沉《一飛集》三卷



 呂述《東平小集》三卷



 《信陵詩》一卷



 《馮道集》六卷又河間集》五卷



 《詩集》十卷



 李松《錦囊集》三卷又《別集》一卷



 王仁
 裕《乘輅集》五卷又《紫閣集》十二卷



 《紫泥集》十二卷



 《紫泥後集》四十卷



 《詩集》十卷



 公乘億《珠林集》四卷又《華林集》三卷



 《集》七卷



 《賦》十二卷



 王超《洋源集》十卷又《鳳鳴集》三卷



 《孫開物集》十六卷



 李琪《應用集》三卷



 《崔拙集》二卷



 李愚《白沙集》十卷又《五書》一卷



 《丘光業詩》一卷



 錢鏐《吳越石壁記》一卷



 孫光憲《荊臺集》四十卷又《筆傭集》十卷



 《紀遇詩》十卷



 《鞏湖編玩》三卷



 《橘齊集》二卷



 和凝《演論集》三卷又《游藝集》五十卷



 《紅藥編》五卷



 賈緯《草堂集》二十卷
 又《續草堂集》十二卷



 張正《西掖集》三十卷



 《陳九疇集》五卷



 《韋莊諫疏箋表》四卷



 楊懷玉《忘筌集》三卷



 《王倓後集》十卷



 《喬諷集》十卷



 《李洪茂集》十卷



 毛文晏《昌城後寓集》十五卷又《西閣集》十卷



 《東壁出言》三卷



 杜光庭《廣成集》一百卷又《壺中集》三卷



 庚傳昌《金行啟運集》二十卷



 李堯夫《梓潼集》二十卷



 勾令言《玄舟集》二十卷



 童九齡《潼江集》二十卷



 王樸《翰苑集》十卷



 李瀚《丁年集》十卷



 《塗昭良集》八卷



 李昊《蜀祖經緯略》一百卷又《樞機集》二十卷



 商文圭《從軍蒿》二十卷又《鏤冰錄》二十卷



 《筆耕詞》二十卷



 游恭《東里集》三卷



 又《廣東里集》二十卷



 《短兵集》三卷



 朱潯《昌吳啟霸集》三十卷



 沈松《錢金集》八卷



 郭昭度《蕓閣集》五十卷



 《李氏金臺鳳藻集》五十卷



 李為光《斐然集》五卷



 程簡之《金鏤集》十二卷



 沉顏《陵陽集》五卷又《聱書》十卷



 《解聱》十五卷



 程柔《安居雜著》十卷



 陳浚《揖讓錄》七卷



 《李煜集》十卷又《集略》十卷



 《詩》一卷



 宋齊丘《祀玄集》三卷



 孫晟《續古闕文》一卷



 陳致雍《曲臺奏議集》二十卷



 孟拱
 辰《鳳苑集》三卷



 湯筠《戎機集》五卷



 喬舜《擬謠》十卷



 《張安石詩》一卷



 《趙摶歌詩》二卷



 方納《遠華集》一卷



 《韋藹詩》一卷



 《張傑詩》一卷



 謝磻隱《雜感詩》二卷



 戴文一作又《回文詩》一卷



 《守素先生遺榮詩集》三卷



 《譚藏用詩》一卷



 羅紹威《政餘詩集》一卷



 《章碣詩》一卷



 商緒《潛陽詩集》三卷



 熊惟簡《湘西詩集》三卷



 《李明詩集》五卷



 《郭鵬詩》一卷



 孟寶子《金鰲詩集》二卷



 《李叔文一作父詩》一卷



 《王希羽詩》一卷



 《廖光圖詩集》二卷



 《廖凝詩集》七卷



 《廖邈詩集》二卷



 《廖融詩
 集》四卷



 《王梵志詩集》一卷



 《左紹沖集》三卷



 熊皦《屠龍集》五卷



 《章一作辛郾詩》一卷



 朱存《金陵覽古詩》二卷



 《韓溉詩》一卷



 《高蟾詩》二卷



 《孫魴詩集》三卷



 《成文乾詩集》五卷



 吳蛻《一字至七字詩》二卷



 羅浩源《廬山雜詠詩》一卷



 王遒一作遵《詠史》一卷冀訪《詠史》十卷



 孫玄晏《覽北史》三卷



 崔道融《申唐詩》三卷



 杜輦《詠唐史》十卷



 趙容一作穀《刺賢詩》一卷



 閻承琬《詠史》三卷



 《六朝詠史》六卷



 童汝為《詠史》六卷



 陸元皓《詠劉子詩》三卷



 《高邁賦》一卷



 《謝觀賦集》八
 卷



 《蔣防賦集》一卷



 《俞嚴賦集》一卷



 《侯圭賦集》一卷



 《鄭瀆賦》二卷



 《王翃賦集》二卷



 《賈嵩賦集》三卷



 《蔣凝賦集》三卷



 《桑維翰賦》二卷



 林絢《大統賦》二卷



 《大紀賦》三卷



 李希運《兩京賦》一卷



 崔葆《數賦》十卷



 毛濤一作鑄《渾天賦》一卷



 劉惲《悲甘陵賦》一卷張龍泉、章孝標注。



 盧獻卿《愍征賦》一卷



 張瑩一作策《吊梁『梁』下或有『郊』字賦》一卷



 王樸《樂賦》一卷



 魯褒《錢神論》一卷



 潘詢注《才命論》一卷



 錢棲業《太虛潮論》一卷



 杜光庭《三教論》一卷



 《大寶論》一卷
 丁友亮《唐興替論》一卷



 丘光庭《海潮論》一卷



 趙昌嗣《海潮論》一卷



 《九證心戒》一卷



 杜嗣先《兔園策》十卷



 鄭寬《百道判》一卷



 《吳康仁判》一卷



 《崔銳判》一卷



 《趙璘表狀》一卷



 《李善夷表集》一卷



 《鄭嵎表狀略》三卷



 《彭霽啟狀》一卷



 《鄭氏貽孫集》四卷



 《張浚表狀》一卷



 趙鄰幾《禹別九州賦》三卷



 《李巨川啟狀》二卷



 鄭準《渚宮集》四卷



 李翥《魚化集》一卷



 《樊景表狀集》一卷



 《羅貫啟狀》二卷



 《梁震表狀》一卷



 趙仁拱《潛龍筆職》三卷



 《黃臺江西表狀》二卷



 《周慎辭表狀》五卷



 郭洪《記室袖中備要》三
 卷



 《金臺倚馬集》九卷



 《擬狀制集》三卷



 《章表分門》一卷



 《兩制珠璣集》二卷



 《搢紳集》三卷



 《蓬壺集》一卷



 《忘機子》五卷並不知作者。



 張昭《嘉善集》五十卷



 高錫《簪履編》七卷



 《王祐集》二十卷



 羅處約《東觀集》十卷



 郭贄《文懿集》三十卷



 陳摶《釣潭集》二卷



 《王溥集》二十卷



 《趙止交集》二十卷



 《薛居正集》三十卷



 寶儀《端揆集》四十五卷



 《白稹集》十卷



 徐鉉《質論》一卷



 《蘇易簡章表》十卷



 《李昉集》五十卷



 《朱昂集》三十卷



 《王旦集》二十卷



 《鞠常集》二十卷



 《李瑩集》十卷



 梁周翰《
 翰苑制草集》二十卷



 王禹備《制誥集》十二卷



 《韓□奏議》三卷



 楊億《虢略集》七卷



 《劉宜集》一卷



 《楊徽之集》五卷



 趙師民《儒林舊德集》三十卷



 《丘旭詩》一卷又《賦》一卷



 曾致堯《直言集》一卷



 《張翼詩》一卷



 韋文化《韶程詩》一卷



 趙晟《金山詩》一卷



 李度《策名詩》一卷



 《楊日嚴集》十卷



 趙抃《成都古今集》三十卷



 宋敏求《書闈前後集》《西垣制詞文集》四十八卷



 《呂惠卿文集》一百卷又《奏議》一百七十卷



 《龍鼎臣諫草》三卷



 《程師孟文集》二十卷又《奏議》十五卷



 《楊
 繪文集》八十卷



 張方平《玉堂集》二十卷



 王洙《昌元集》十卷



 《承乾文集》十卷



 《田況文集》三十卷



 鄧綰《治平文集》三十卷又《翰林制集》十卷



 《西垣制集》三卷



 《奏議》二十卷



 《雜文詩賦》五十卷



 劉彞《明善集》三十卷又《居易集》二十卷



 《趙世繁歌詩》十卷



 《張詵文集》十卷又《奏議》三十卷



 《韓絳文集》五十卷又《內外制集》十三卷



 《奏議》三十卷



 《龐元英文集》三十卷



 《李常文集》六十卷又《奏議》二十卷



 《孫覺文集》四十卷又《奏議》二十卷



 《外集》十卷



 《呂公孺詩集奏議》
 二十卷



 《熊本文集》三十卷又《奏議》二十卷



 《傳堯俞奏議》十卷



 《葉康直文集》十卷



 《李承之文集》三十卷又《奏議》二十卷



 《盧秉文集》十卷又《奏議》三十卷



 晃補之《雞肋集》一百卷



 《王庠文集》五十卷



 《劉紋集》六十卷



 《孔文仲文集》五十卷



 《孔武仲奏議》二卷



 《蒲宗孟文集奏議》七十卷



 《張利一奏議》三卷



 《喬執中古律詩賦》十五卷又《雜文碑志》十卷



 《趙仲庠內外制》十卷又《雜文》五十卷



 《制誥表章》十卷



 《趙仲銳文集》十卷



 《李之純文集》二十卷又《奏議》五卷



 趙
 世逢《英華集》十卷



 《李清臣文集》一百卷又《奏議》三十卷



 又《奏議》三十卷



 《李新集》四十卷



 《逃洙文集》十卷



 《杜紘文集》二十卷



 又奏議十卷



 《後山集》三十卷



 曾肇《元祐制集》十二卷又《曲阜外集》三十卷



 張舜民《書墁集》一百卷



 《王存文集》五十卷



 《李昭集》三十卷



 蔣之奇《荊溪前後集》八十九卷又《別集》九卷



 《北扉集》九卷



 《西樞集》四卷



 《卮言集》五卷



 《芻言》五十篇



 《舒但文集》一百卷



 《龔原文集》七十卷又《穎川唱和詩》三卷



 《安燾文集》四十卷又《奏議》十卷



 《張商英文集》一百
 卷



 《蔡肇文集》三十卷



 《劉跂集》二十卷



 《秦敏學集》二卷



 《曾孝廣文集》二十卷



 《張閣文集》二十卷



 《吳居厚文集》一百卷又《奏議》一百二十卷



 《呂益柔文集》五十卷又《奏議》一卷



 《姚祐文集》六十卷又《奏議》二十卷



 《上官均文集》五十卷又《奏議》十卷



 葉煥《繼明集》一卷



 趙仲御《東堂集》一卷



 李長民《汴都賦》一卷



 《鮑慎由文集》五十卷



 《游酢文集》十卷



 《劉安世文集》二十卷



 《許安國詩》三卷



 《唐恪文集》八十卷



 《譚世績文集》三十卷又《奏議》二十一卷



 《外制》五卷



 《師
 陶集》二卷



 孫希廣《樵漁論》三卷



 寶夢證《東堂集》三卷



 《恭翔集》十卷



 《盧文度集》二卷



 《崔氏乾CM錄》六卷



 《李慎儀集》十二卷



 《唐鴻集》五卷



 《青蕪編集》一卷



 《陳光圖集》七卷



 《李洪源集》二卷



 《酈炎文》四篇



 沈彬《閑居集》十卷



 《羅隱後集》二十卷



 又《汝江集》三卷



 《芻言》五十篇



 《歌詩》十四卷



 《吳越掌書記集》三卷



 熊皎《南金集》二卷



 《龔霖詩》一卷



 《倪曉賦》一卷



 《譚用之詩》一卷



 《扈載集》五卷



 《南唐李後主集》十卷



 《宋齊丘文傳》十三卷



 《徐鍇集》十五卷



 馮延巳《陽春錄》
 一卷



 《田霖四六》一卷



 潘祐《榮陽集》二十卷



 左偃《鐘山集一卷》



 《張為詩》一卷



 徐寅《探龍集》五卷



 張麟《答輿論》三卷



 楊九齡《桂堂編事》二十卷



 《蔡昆詩》一卷



 《廖正圖詩》一卷



 《劉昭禹詩》一卷



 《孫魴詩》五卷



 《李建勛集》二十卷



 杜田注《杜詩補遺正繆》十二卷



 薛舒《杜詩補遺》五卷



 《續注杜詩補遺》八卷



 洪興祖《杜詩辨證》二卷



 範質集》三十卷



 《趙普奏議》一卷



 《李瑩集》一卷



 《陶谷集》十卷



 王祐《襄陽風景古遺跡詩》一卷



 《柳開集》十五卷



 《徐鉉集》三十二卷



 《湯悅集》
 三卷



 《宋白集》一百卷又《柳枝詞》一卷



 《賈黃中集》三十卷



 《李至集》三十卷



 《張洎集》五十卷



 《李諮集》二十卷



 《楊樸詩》一卷



 《潘閬詩》一卷



 《羅處約詩》一卷



 《李光輔集》一卷



 《王操詩》一卷



 盧稹《曲肱編》六卷



 《趙湘集》十二卷



 《古成之集》三卷



 《章士廉集》二卷



 《廖氏家集》一卷



 王禹備《小畜集》三卷又《外集》二十卷



 《承明集》十卷



 《別集》十六卷



 潘祐《榮陽集》二十卷



 《田錫集》五十卷又《別集》三卷



 《奏議》二卷



 魏野《草堂集》二卷又《鉅鹿東觀集》十卷



 《張詠集》
 十卷



 《寇準詩》三卷又《巴東集》一卷



 《丁謂集》八卷又《虎丘錄》五十卷



 《刀筆集》二卷



 《青衿集》三卷



 《知命集》一卷



 《胡旦集》十六卷



 《陳靖集》十卷



 晃迥《昭德新編》三卷



 《穆修集》三卷



 《熊知至集》一卷



 《劉隨諫草》二十卷



 《林逋詩》七卷又《詩》二卷



 《柴慶集》十卷



 《劉夔應制》一卷



 《謝伯初詩》一卷



 《呂祐之集》二十卷



 錢惟演《擁旄集》五卷



 陳堯佐《愚丘集》二卷又《潮陽新編》一卷



 《石介集》二十卷



 《夏竦集》一百卷又《策論》十三卷



 宋庠《緹巾集》十二卷又《操縵集》六卷



 《
 王隨集》二十卷



 《宋郊文集》四十四卷



 《刀筆集》二十卷



 《西川猥蒿》三卷



 《鄭文寶集》三十卷



 楊億《蓬山集》五十四卷



 又《武夷新編集》二十卷



 《穎陰集》二十卷



 《刀筆集》二十卷



 《別集》十二卷



 《汝陽雜編》二十卷



 《鑾坡遺札》十二卷



 劉筠《冊府應言集》十卷又《榮遇集》二十卷



 《中山刀筆集》三卷



 《表奏》六卷



 《肥川集》四卷



 《韓丕詩》三卷



 《種放集》十卷



 李介《種放江南小集》二卷



 《柴成務集》二十卷



 《孫何集》四十卷



 《孫
 僅詩》一卷



 《許申集》一卷



 《錢易集》六十卷



 《高弁集》三卷



 《錢昭度詩》一卷



 《唐異詩集》一卷



 《江為詩》一卷



 《李畋集》十卷



 《張餗集》三卷



 《張景集》二十卷



 《郭震集》四卷



 《鄭修集》一卷



 《許允豹詩》一卷



 劉若中《永昌應制集》三卷



 《陳漸集》十五卷



 陳充《民士編》二十卷



 錢彥遠《諫垣集》三十卷又《諫垣遺蒿》五卷



 《齊唐集》三十卷又《策論》十卷



 《鮑當集》一卷又《後集》一卷



 何涉《治道中術》六卷



 《仲訥集》十二卷



 《梅堯臣集》六十卷又《後集》二卷



 《畢田詩》一卷



 楊備《姑蘇百題詩》
 三卷



 宋綬《常山祐殿集》三卷



 又《托居集》五卷



 《常山遺札》三卷



 《許推官吟》一卷



 袁陟《廬山四游詩》一卷又《金陵訪古詩》一卷



 《魯交集》三卷



 《鄭伯玉詩》一卷



 《顏太初集》十卷



 《範太初集》二十卷



 《範仲淹集》二十卷又《別集》四卷



 《尺牘》二卷



 《奏議》十五卷



 《丹陽編》八卷



 《呂申公試卷》一卷



 《杜衍詩》一卷



 丘浚《觀時感事詩》一卷



 《困編》一卷



 《晏殊集》二十八卷又《臨川集》三十卷



 《詩》二卷



 《二府集》十五卷



 《二府別集》十二卷



 《北海新編》六卷



 《平臺集》一卷



 《胡宿集》七十卷又《制詞》四卷



 《包拯
 奏議》十卷



 《戴真詩》二卷



 《錢藻賢良策》五卷



 《蘇舜欽集》十六卷



 張伯玉《蓬萊詩》二卷



 《孫復集》十卷



 周曇《詠史詩》八卷



 《尹洙集》二十八卷



 崔公度《感山賦》一卷



 《燕肅詩》二卷



 《尹源集》六卷



 又《幕中集》十六卷



 《葉清臣集》十六卷



 李淑《書殿集》二十卷



 又《筆語》十五卷



 《龍昌期集》八卷



 《田況策論》十卷



 《蔣康叔小集》一卷



 《張俞集》二十六卷



 《寇隨詩》一卷



 《王琪詩》二十卷



 《狄遵度集》十卷



 《黃亢集》十二卷



 《李問詩》一卷



 李祺《刀筆集》十五卷



 又《象臺
 四六集》七卷



 陳亞《藥名詩》一卷



 《黃通集》三卷



 《湛俞詩》一卷



 《江休復集》四十卷



 《王回集》十卷



 《蘇洵集》十五卷



 又《別集》五卷



 李泰伯《直講集》三十三卷



 又《後集》六卷



 《黃庶集》六卷



 《劉輝集》八卷



 《王同集》十卷



 《王令集》二十卷



 又《廣陵文集》六卷



 《餘靖集》二十卷



 又《諫草》三卷



 《孫沔集》十卷



 《劉敞集》七十五卷



 《蔡襄集》六十卷



 又《奏議》十卷



 《歐陽備集》五十卷



 又《別集》二十卷



 《六一集》七卷



 《奏議》十八卷



 《內外制集》八卷



 《從諫集》八卷



 《韓琦集》五十卷



 又《諫垣存蒿》
 三卷



 《富弼奏議》十二卷



 又《札子》十六卷



 《呂誨集》十五卷



 又《章奏》二十卷



 趙抃《南臺諫垣集》二卷又《清獻盡言集》二卷



 元絳《玉堂集》二十卷又《玉堂詩》十卷



 《鄭獬集》五十卷



 《王陶詩》三十卷



 又《集》五卷



 宋敏求《東觀絕筆》二十卷



 《晃端友詩》十卷



 程師孟《長樂集》一卷



 《陶弼集》四十卷



 《強至集》四十卷



 《邵雍集》二十卷



 《張載集》十卷



 《張先詩》二十卷



 《陳襄集》二十五卷



 又《奏議》一卷



 曾鞏《元豐類蒿》五十卷



 又《別集》六卷



 《續蒿》四十卷



 《揚蟠詩》二十卷



 《袁思正集》
 六卷



 《晃端忠詩》一卷



 《章望之集》四十卷



 又《集》十一卷



 《吳頎詩》一卷



 《劉渙詩》十二卷



 《吳孝宗集》二十卷



 呂南公《灌園集》三十卷



 《王韶奏議》六卷



 《李師中詩》三卷



 《楊繪諫疏》七卷



 《傅翼之集》一卷



 《任大中集》三卷



 《方子通詩》一卷



 王震《元豐懷遇集》七卷



 《張徽集》三卷



 又《北閩詩》一卷



 《王無咎集》十五卷



 《司馬光集》八十卷



 又《全集》八十卷



 《龔鼎臣集》五十卷



 《文彥博集》三十卷



 又《顯忠集》二卷



 《王安石集》一百卷



 《張方平集》四十卷



 又《進策》九卷



 《王珪集》一
 百卷



 範鎮《諫垣集》十卷



 又《奏議》二卷



 《程顥集》四卷



 《朱光庭奏議》三卷



 《範祖禹集》五十五卷



 《王嚴叟集》四十卷



 《趙瞻集》二十卷



 又《奏議》十卷



 《揚傑集》十五卷



 又《別集》十卷



 《鮮於先集》二卷



 《蘇頌集》七十二卷



 又《略集》一卷



 《劉分文集》六十卷



 《王剛中文集》六卷



 《顏復集》十三卷



 孔平仲《詩戲》一卷



 《
 李清臣集》八十卷



 又《進策》五卷



 《程頤集》二十卷



 蘇軾《前後集》七十卷



 《奏議》十五卷



 《補遺》三卷



 《南征集》一卷



 《詞》一卷



 《南省說書》一卷



 《應詔集》十卷



 《內外制》十三卷



 《別集》四十六卷



 《黃州集》二卷



 《續集》二卷



 《和陶詩》四卷



 《北歸集》六卷



 《但耳手澤》一卷



 《年譜》一卷王宗稷編。



 蘇轍《樂城集》八十四卷



 《應詔集》十卷



 《策論》十卷



 《均陽雜著》一卷



 《黃庭堅集》三十卷



 《樂府》二卷



 《
 外集》十四卷



 《書尺》十五卷



 《陳師道集》十四卷



 又《語業》一卷



 《秦觀集》四十又《奏議》



 《蔣之奇集》一卷



 《曾布集》三十卷



 《呂惠卿集》五十卷



 《曾肇集》五十卷



 又《奏議》十二卷



 《西垣集》十二卷



 《庚辰外制集》三卷



 《內制集》五卷



 《張來集》七十卷



 又《進卷》十二卷



 《李昭玘集》三十卷



 《晁補之集》七十卷



 《李廌集》三十卷



 《蔡肇集》六卷



 又《詩》三卷



 《呂陶集》六十卷



 《張舜民集》一百卷



 《張商英集》十三卷



 《鄭俠集》二十卷



 錢惟演《伊川集》五卷



 《陳簡能集》一
 卷



 又《治說》十卷



 《應制策論》一卷



 《金君卿集》十卷



 劉輝《東歸集》十卷



 《王安國集》六十卷



 又《序言》八卷



 《王安禮集》二十卷



 《范純仁忠宣集》二十卷



 又《彈事》五卷



 《國論》五卷



 韓維《南陽集》三十卷



 又《穎邸記室集》一卷



 《奏議》一卷



 李復《潏水集》四十卷



 《傅堯俞集》十卷



 《丁奏議》二十卷



 又《奏議》一卷



 《陳師錫奏議》一卷



 彭汝礪《鄱陽集》四十卷



 《龍夬奏議》一卷



 範百祿《榮國集》五十卷



 又《奏議》六卷



 《內制》五卷



 《外制》五卷



 鄒浩《文卿集》四十卷



 《郭祥正集》
 三十卷



 《陳瓘集》四十卷



 又《責沉》一卷



 《諫垣集》三卷



 《四明尊堯集》五卷



 《了齊親筆》一卷



 《尊堯餘言》一卷



 《李新集》四十卷



 吳栻《蜀道紀行詩》三卷



 又《庵峰集》一卷



 《徐積集》一卷



 任伯雨《戇草》二卷



 又《乘桴集》三卷



 《葛次仲集句詩》三卷



 《鄭少微策》六卷



 石柔《橘林集》十六卷



 《謝逸集》二十卷又《溪堂詩》五卷



 《謝過集》十卷



 《陸純集》十卷



 《張勵詩》二十卷



 《廖正一集》八卷



 《韓筠集》一卷



 《張勸詩》二卷



 王寀《南陔集》一卷



 《楊天惠集》六十卷



 《劉跋集》二十卷王家撰。



 《唐庚
 集》二十二卷



 《馬存集》十卷



 又《經濟集》十二卷



 《朱服集》十三卷



 《毛滂集》十五卷



 《李樵詩》二卷



 《朱淢集》十二卷



 《劉玨奏議》一卷



 《崔鶠集》三十卷



 《李若水集》十卷



 《梅執禮集》十五卷



 《晁說之集》二十卷



 《楊時集》二十卷



 又《龜山集》三十五卷



 《李樸集》二十卷



 《王安中集》二十卷



 《徐俯集》三卷



 《呂本中詩》二十卷



 《翟汝文集》三十卷



 《汪藻集》六十卷



 《程俱集》三十四卷



 《李綱文集》十八卷



 趙鼎《得全居士集》二卷



 又《忠正德文集》十卷



 《朱勝非奏議》十五卷



 禮《北海集》
 六十卷



 葉夢得《石林集》一百卷又《奏議》十五卷



 《建康集》八卷



 孫覿《鴻慶集》四十二卷



 《汪伯彥後集》二十五卷



 又《續編》一卷



 胡銓《澹庵集》七十卷



 《李光前後集》三十卷



 張澄《澹(山嚴)集》四十卷



 李邴《草堂後集》二十六卷



 饒節《倚松集》十四卷



 《吳則禮集》十卷



 韓駒《陵陽集》十五卷



 又《別集》三卷



 《傅察集》三卷



 趙鼎臣《竹隱畸士集》四十卷



 趙育《酒隱集》三卷



 《曾匪集》十六卷



 《陳東奏議》一卷



 《章誼奏議》二卷



 又《文集》二十卷



 劉安世《元城盡言集》十三卷



 許景衡《
 橫塘集》三十卷



 《田書集》二卷



 劉弇《龍雲集》三十二卷



 《慕容彥逢集》三十卷



 李端叔《姑溪集》五十卷



 又《後集》二十卷



 米芾《山林集拾遺》八卷



 倪濤《玉溪集》二十二卷



 張彥寶《東窗集》四十卷



 又《詩》十卷



 劉一止《苕溪集》五十五卷



 王賞《玉臺集》四十卷



 馮時行《縉雲集》四十三卷



 高登《東溪集》十二卷



 仲並《浮山集》十六卷



 王洋《東牟集》二十九卷



 《關注集》二十卷



 葛立方《歸愚集》二十卷



 曹勛《松隱集》四十卷



 《辛次膺奏議》二十卷



 又《箋表》十卷



 周麟之《海陵
 集》二十三卷



 《王鎡集》二十三卷



 任古《拙齊遺蒿》三卷



 任正言《小丑集》十二卷



 又《續集》五卷



 張積《鶴鳴先生集》四十一卷



 呂大臨《玉溪先生集》二十八卷



 胡恭《政議進蒿》一卷



 《葉訪所業》二卷



 勾滋《遠齊文集》七卷



 《吳正肅制科文集》十卷



 王發《元祐進本制舉論》十卷



 呂頤浩《忠穆文集》十五卷



 張元乾《蘆川詞》二卷



 《三雇隱客文集》十一卷



 《文選精理》二十卷



 岳陽黃氏《靈仙集》十五卷以上不知名。



 《宋初梅花千詠》二卷



 《易安居士文集》七卷宋李格非女撰。



 又《易
 安詞》六卷



 《辛棄疾長短句》十二卷



 又《稼軒奏議》一卷



 《吳楚紀行》一卷宋峽州守吳氏撰,不知名。



 劉子翬《屏山集》九十卷



 《劉珙集》九十卷



 又《附錄》四卷



 鄧良能《書潛集》三十卷



 游桂《畏齊集》二十二卷



 王十朋《南游集》二卷



 又《後集》一卷



 史浩《真隱漫錄》五十卷



 洪適《盤洲集》八十卷



 洪遵《小隱集》七十卷



 洪邁《野處猥蒿》一百四卷



 又《瓊野錄》三卷



 劉儀鳳《奇堂集》三十卷



 又《樂府》一卷



 《羅願小集》五卷



 張嵲《紫微集》三十卷



 周紫芝《太倉稊米集》七十卷



 毛開《樵隱集》十
 五卷



 張行成《觀物集》三十卷



 倪文舉《綺川集》十五卷



 張嗣良《敞帚集》十四卷



 韓元吉《愚戇錄》十卷



 又《南澗甲乙蒿》七十卷



 宋汝為《忠嘉集》一卷



 又《後集》一卷



 《陳熙甫奏札》一卷



 陳康伯《葛溪集》三十卷



 陳恬《澗上卷》三十卷



 汪中立《符桂錄》三卷



 王萊《龜湖集》十卷



 何遲《蒙野集》四十九卷



 曹彥章《箕穎集》一十卷



 孫應時《燭湖集》十卷



 沉興求《龜溪集》十二卷



 呂祖儉《大愚集》十一卷



 顏師魯文集》四十四卷



 陳峴《東齊表奏》二卷



 聶冠卿《蘄春集》十卷



 《沈
 夏文集》二十卷



 陳正伯《書舟雅詞》十一卷



 《劉給事文集》一卷



 《鄧忠臣文集》十二卷



 賀鑄《慶湖遺老集》二十九卷



 《林慄集》三十卷



 又《奏議》五卷



 龔茂良《靜泰堂集》三十九卷



 周必大《詞科舊蒿》三卷



 又《掖垣類蒿》七卷



 《玉堂類蒿》二十卷



 《政府應制蒿》一卷



 《歷官表奏》十二卷



 《省齊文蒿》四十卷



 《別蒿》十卷



 《平園續蒿》四十卷



 《承明集》十卷



 《奏議》十二卷



 《雜著述》二十三卷



 《書蒿》十五卷



 《附錄》五卷



 朱松《韋齊集》十二卷



 又《小集》一卷



 《朱熹前集》四十卷



 《後集》九
 十一卷



 《續集》十卷



 《別集》二十四卷



 張栻《南軒文集》四十八卷



 《呂祖謙集》十五卷又《別集》十六卷



 《外集》十六卷



 《附錄》三卷



 汪應辰《翰林詞章》五卷



 《鄭伯集》三十卷



 《鄭伯英集》二十六卷



 陸九淵《象山集》二十八卷



 又《外集》四卷



 《潘良貴集》十五卷



 《林待聘內外制》十五卷



 吳鎰《敬齊集》三十二卷



 沈樞《宜林集》三十卷



 吳芾《湖山集》四十三卷又《別集》一卷



 《和陶詩》三卷



 《附錄》三卷



 《當塗小集》八卷



 吳天驥《鳳山集》十二卷



 雍焯《過溪前集》二十卷



 又《後集》三卷



 趙彥端《介庵集》十卷



 又《外集》三卷



 《介庵詞》四卷



 龐謙孺《白蘋集蒿》四卷



 《李迎遺蒿》一卷



 謝諤《江行雜著》三卷



 曾豐《樽齊緣督集》十四卷



 陳傅良《止齊集》五十二卷



 《陳亮集》四十卷



 又《外集詞》四卷



 蔡幼學《育德堂集》五十卷



 曾煥《毅齊集》十八卷



 又《臺城丙蒿》四卷



 《南城集》十八卷



 《曾習之詩文》二卷



 《蘇元老文集》三十二卷



 彭克《玉壺梅花三百詠》一卷



 《王景文集》四十卷



 《劉安上文集》四十卷



 《劉安節文集》五卷



 《周博士文集》十卷不知名。



 黃季岑《三餘集》十
 卷



 吳億《溪園自怡集》十卷



 周邦彥《清真居士集》十一卷



 《程大昌文集》二十卷



 蘇《雙溪集》十一卷



 楊椿《蕓室文集》七十五卷



 蔣邁《桂齊拙蒿》二卷



 又《施正遺蒿》二卷



 《丘崇文集》十卷



 羅適《赤城先生文集》十卷



 王灼《頤室文集》五十七卷



 餘安行《石月老人文集》五十七卷



 陸游《劍南續蒿》二十一卷



 又渭南集》五十卷



 費氏《蕓山居士文集》二十一卷不知名。



 李正民《大隱文集》三十卷



 杜受言《碔砆集》十三卷



 鄧肅《栟櫚集》二十六卷



 胡安國《武夷集》二
 十二卷



 胡寅《斐然集》二十卷



 程敦儒《寵堂集》六十八卷



 又《後集》二十卷



 朱翌集》四十五卷



 又《詩》三卷



 廖剛《高峰集》十七卷



 趙令畤《安樂集》三十卷



 《陸九齡文集》六卷



 周孚《金公刀編》三十二卷



 玉堂梅林文集》二十卷又《雲溪類集》三十卷



 李璜《薛庵文集》十二卷



 江公望《釣臺棄蒿》十四卷



 吳沉《環溪集》八卷



 《月湖信筆》三卷不知作者。



 《趙雄奏議》二十卷



 許開《志隱類蒿》二十卷



 項安世《丙辰海蒿》四十七卷



 趙逵《棲雲集》二十五卷



 《黃策集》四十卷



 《連寶學奏
 議》二卷不知名。



 《衛膚敏諫議遺蒿》二卷



 姜夔《白石叢蒿》十卷



 陳伯魚《澹齊草紙目錄》四十二卷



 彭龜年《止堂集》四十七卷



 彭鳳《梅坡集》五卷



 李彌遜《筠溪集》二十四卷



 龔日華《北征讜議》十二卷



 蕭之敏《直諒集》三卷



 李士美《北門集》四卷



 《劉清之文集》二十三卷



 《葉適文集》二十八卷



 周南《山房集》五卷



 王秬復齊制表》一卷



 《倪思奏議》二十六卷



 又《歷官表奏》十卷



 《翰林奏草》一卷



 《翰林前蒿》二十卷



 《翰林後蒿》二卷



 《畢仲游文集》五十卷



 王之道《相山居
 士文集》二十五卷



 又《相山長短句》二卷



 王從三《近齊餘錄》五卷



 謝伋《藥寮叢蒿》二十卷



 《羅點奏議》二十三卷



 《李蘩奏議》二卷



 《詹儀之奏議》二卷



 胡孰《萬石書》一卷



 《周行己集》十九卷



 《鮑欽止集》二十卷



 《黃裳集》六十卷



 《林敏功集》十卷



 《方孝能文集》一卷



 《王庠集》五十卷



 《秦敏學集》二卷



 姚述堯《簫臺公餘》一卷



 蒙泉居士《韓文英華》二卷



 蘇過《斜川集》十卷



 王彥輔《鳳臺子和杜詩》三卷



 《杜甫詩詳說》二十八卷不知作者。



 郭《南湖詩》八卷



 《陸長翁文集》四十
 卷



 詹叔義《狂夫論》十二卷



 《朱敦儒陳淵集》二十六卷



 又《詞》三卷



 《王寔集》三十卷



 《蘇庠集》三十卷



 李師稷《皇華編》一卷



 《劉一止集》五十卷苕溪集》多五卷。張攀《書目》以此本為《非有齊類蒿》。



 《葛勝仲集》八十卷



 《傅崧卿集》六十卷



 又《奏議》二卷《制誥》三卷



 《勾龍如淵雜著》一卷



 《洪皓》三卷



 《胡宏集》一卷



 《曾惇詩》一卷



 《黃邦後集》三卷



 又《強記集》八卷



 《江袤集》二十卷



 《盛浟策論》一卷



 潘《集杜詩句》一卷



 《林震集句》二句



 《湓江集》六卷不知作者。



 《周總集》一卷



 《張守集》五十卷



 又《奏議》二十五
 卷



 又十八卷



 範成大《石湖居士文集》卷亡。



 又《石湖別集》二十九卷



 《石湖大全集》一百三十六卷



 許翰《襄陵文集》二十二卷



 《樓金龠文集》一百二十卷



 張宰《蓮社文集》五卷



 《胡世將集》十五卷



 又《忠獻胡公集》六十卷



 《洪龜父詩》一卷



 柯夢得《抱甕集》十五卷



 姜如晦《月溪集》三十二卷



 《錢聞詩文集》二十八卷



 又《廬山雜著》三卷



 芮暉《家藏集》七卷



 王咨《雪齊文集》四十卷



 《李燾文集》一百二十卷



 薛齊誼《六一先生事證》一卷告詞附。



 王大昌《六一先生在滁詩》一
 卷



 王居正《竹西文集》十卷



 李觀《顯親集》六卷



 陳汝錫《鶴溪集》十二卷



 陳逢寅《山谷詩注》二十卷



 朱熹校《昌黎集》五十卷



 王洙注《杜詩》三十六卷



 方醇道《類集杜甫詩史》三十卷



 僧道翹《寒山拾得詩》一卷



 傅自得《至樂齊集》四十卷



 俞汝尚《溪堂集》四卷



 《劉燾詩集》二十卷



 《方惟深集》十卷又《錄》一卷



 王庭《雲壑集》三卷



 蔡柟《浩歌集》一卷



 王庭珪《盧溪集》十卷



 邵緝《荊溪集》十卷



 吳氏《符川集》一卷不知名。



 陳克《天臺詩》十卷



 又《外集》四卷



 劉綺《清溪詩集》三
 卷



 王質《雪山集》三卷


蕭德藻《千
 \gezhu{
  山嚴}
 擇蒿》七卷



 又《外編》三卷



 楊萬里《江湖集》十四卷



 又《荊溪集》十卷



 《西歸集》八卷



 《南海集》八卷



 《朝天集》十一卷



 《江西道院集》三卷



 《朝天續集》八卷



 《江東集》十卷



 《退休集》十四卷



 《危稹文集》二十卷



 林憲《雪巢小集》二卷



 葉鎮《曾稽覽古詩》一卷



 《邵博文集》五十七卷



 《鄭剛中文集》八卷



 《李浩文集》二卷



 《許及之文集》三十卷



 又《涉齊課蒿》九卷



 《黃乾文集》十卷



 《錦屏先生文集》十一卷不知名。



 祝充《韓文音義》五十卷



 宋德之《青城
 遺蒿》二卷



 《沈渙文集》五卷



 《王述文集》二十卷



 《毛友文集》四十卷



 王性之《雪溪集》八卷



 範浚《香溪文集》二十二卷



 胡嶧《如村冗蒿》二十卷



 唐文若《遯庵文集》三十卷



 黃公度《莆陽知稼翁集》十二卷



 《方有聞文集》一卷



 《陳興義詩》十卷



 又《岳陽紀詠》一卷



 張文伯《江南凱歌》二十卷



 《曾幾集》十五卷



 《張孝祥文集》四十卷



 又《詞》一卷



 《古風律詩絕句》三卷


石行正《玉
 \gezhu{
  畾土}
 題詠》九卷



 何耕《勸戒詩》一卷



 孫稽仲《穀橋愚蒿》十卷



 《臨邛計用章集》十
 二卷



 李縝《梅百詠詩》一卷



 倪正甫《兼山小集》三十卷



 黃當《復齊漫蒿》二卷



 丁逢《南征詩》一卷



 《京鏜詩》七卷



 又《詞》二卷



 趙時逢《山窗斐蒿》一卷



 《王秤詩》四卷



 徐璣《泉山詩蒿》一卷



 《黃虒詩蒿》一卷



 黃景說《白石丁蒿》一卷



 《吳賦之文集》一卷



 曾布之《丹丘使君詩詞》一卷



 朱存《金陵詩》一卷



 《石召集》一卷



 《潘咸詩》一卷



 《文史聊珠》十三卷不知作者。



 《得全居士詞》一卷不知名。



 汪遵《詠史詩》一卷



 《韓遂詩》一卷



 《張安石集》一卷



 《盧士衡詩》一卷



 《葉楚詩》一卷



 《陳三思詩》一
 卷



 《丁棱詩》一卷



 《江漢編》七卷不知作者。



 晉惠遠《廬山集》十卷



 《僧棲白詩》一卷



 《僧子蘭詩》一卷



 《僧懷浦詩集》一卷



 僧安綬《SF蕩山集》一卷



 《僧虛中詩》一卷



 《僧貫休集》三十卷



 《僧清塞集》一卷



 《僧齊已集》十卷



 又《白蓮華或無『華』字編外集》十卷



 《僧義現集》三卷



 《僧應之集》一卷



 《僧承訥集》一卷



 《僧無願集》一卷



 《僧靈穆集》一卷



 《僧靈護筠源集》十卷


僧可朋《玉
 \gezhu{
  畾土}
 集》十卷



 《僧自牧《括囊集》十卷



 《僧賓付集》一卷



 僧尚顏《荊門集》五卷



 僧域《龍華集》十卷



 《僧文雅集》一卷



 僧
 光白《蓮社集》二十卷



 又《虎溪集》十卷



 《僧處默詩》一卷



 僧希覺《擬江東集》五卷



 《僧鴻漸詩》一卷



 《僧智遲詩》一卷



 《僧康白詩》十卷



 《僧惠宗詩》三卷



 僧文暢《碧雲集》一卷



 《僧楚巒詩》一卷



 《僧皎然詩》十卷



 《僧無可詩》一卷



 《僧靈澈詩》一卷



 《僧修睦詩》一卷



 《僧SH徵集》三卷



 《僧本先集》一卷



 《僧文彧詩》一卷



 《僧祐演集》二卷



 《僧保暹集》二卷



 僧智圓《間居編》五十一卷



 《僧大容集》一卷



 《僧來鵬詩》一卷



 僧可尚《揀金集》九卷



 《僧惠澄詩》一卷



 《僧有鵬詩》一卷



 《僧警淳詩》一
 卷



 《僧靈一詩》一卷



 止禪師《青谷集》二卷



 僧惠洪《物外集》二卷



 又《石門文字禪》三十卷



 《僧祖可詩》十三卷



 道士主父果詩》一卷



 《魚玄機詩集》



 《李季蘭詩集》一卷唐女道士李裕撰。



 勾臺符《臥雲編》三卷



 《石仲元詩》二卷



 《謝希孟詩》二卷



 又《採蘋詩》一卷



 《曹希蘊歌詩後集》二卷



 《蒲氏玉清編》一卷



 《吳氏南宮詩》二卷



 《王尚恭詩》一卷王亢女。



 《徐氏閨秀集》一卷



 《王氏詩》一卷



 王綸《瑤臺集》二卷



 《許氏詩》一卷許彥國母。



 楊吉《登瀛集》五卷



 《劉京集》四十卷



 右別集類一千八百二十四部,二萬三千六百四卷



\end{pinyinscope}