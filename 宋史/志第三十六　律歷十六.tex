\article{志第三十六 律歷十六}

\begin{pinyinscope}

 紹興統元乾道淳熙會元歷



 演紀上元甲子,距紹興五年乙卯,歲積九千四百二十五萬一
 千五百九十一。《乾道》上元甲子,距乾道三年丁亥,歲積九千一百六十四萬五千八百二
 十三。《淳熙》
 上元甲子,距淳熙三年丙申,歲積五千二百四十二萬一千九百七十二。《會元》上元甲子,距紹熙三年辛亥,歲積二千五百四十九萬四千七百六十七。



 步氣朔



 元法:六千九百三十。《乾道》三萬。《淳熙》五千六百四十。《會元》統率三萬
 八千七百。



 歲周:二百五十三萬
 一千一百三十八;歲周日:三百六十五、餘一千六百八十八。《乾道》期實一千九十五萬七千三百八,歲周三百六十五、餘七千三百八。《淳熙》歲實二百五萬九千九百七十四,歲周日三百六十
 五、餘一千三百七十四。《會元》氣率
 一千四百一十三萬四千九百二十二。



 氣策:一十五日、餘一千五百一十四、秒十五。《乾道》餘六千五百五
 十四半。《淳熙》餘一千二百二十二、秒二十五。《會元》餘八千四百五十五半。



 朔實:二十萬四千六百四十七。《乾道》八十八萬五千九百一十七、秒七十六。《淳熙》一十六萬六千五百五十二、秒五十六。《會元》朔率一百一十四萬二千八百三十四。



 歲閏:七萬五千三百七十四。《乾道》三十二萬六千二百九十四、秒八十八,又有閏限八十五萬八千七百二十六、秒五十二,月閏二萬七千一百九十一、秒二十四。《會元》四十二萬九百二十四,又有閏限七十二萬一千九百一十。《乾道》又有沒限二萬二千四百四十五半。《淳熙》四千四百七、秒七十五。《會元》三萬二百四十四半。



 朔策:二十九日、餘三萬六千七十七。《乾道》餘一萬五千九百一十七、秒七
 十六。《淳熙》餘三千九百九十二、秒五十六。《會元》餘二萬五百三十四,約分五十三、秒五。



 望策:十四日、餘五千三百三半。《乾道》餘一萬二千九百五十八、秒八十八。《淳熙》餘四千三百一十六、秒二十八。《會元》餘二萬九千六百一十七。



 弦策:七日、餘二千六百五十一太。《乾道》餘一萬一千四百七十九,秒四十四。《淳熙》餘二千一百五十八、秒一十四。《會元》餘一萬四千八百八半。



 中盈分:三千三百二十八、秒三十。《乾道》一萬三千一百九。《淳熙》二千四百六十四、秒五十。《會元》一萬六千九百一十一。



 朔虛分:三千二百五十三。《乾道》一萬四千八十二、秒二十四。《淳熙》二千六百四十七、
 秒四十四。《會元》一萬八千一百六十六。



 旬周:四十一萬五千八百。漙《乾道》一百八十萬。《淳熙》三十三萬八千四百、秒一。《會元》二百三十二萬二千。



 紀法:六十。三歷同。



 推天正冬至:置距所求積年,以歲周乘之,為氣積分;以旬周去之,不盡,總法約之為大餘,不滿為小餘。大餘命甲子,算外,即得所求年天正冬至日辰及餘。其小餘總法退除為約分,即百為母。



 求次氣:置冬至大、小餘,以氣策及餘秒加之,秒盈秒法從一小餘,小餘滿總法從一大餘,滿紀法去,命甲子,算外,合得次氣日辰及餘秒。



 求天正經朔:置天正冬至氣積分,以朔實去之,不盡為閏餘;以減冬至氣積分,餘為天正十一月經朔加時朔積分;以旬周去之,不滿,總法約之為大餘,不滿為小餘。命甲子,算外,即得所求年天正十一月經朔加時朔積分。以旬周去之,不滿,總法約之為大餘,不滿為小餘。大
 餘命甲子,算外,即得所求天正十一月經朔日辰及餘。



 求弦望及次朔經日:置天正十一月經朔大、小餘,以弦策加之,為上弦;累加之,去命如前,各得弦、望及次月朔經日及餘也。



 求沒日:置有沒之氣小餘,以一百八十乘之,秒從之,用減一百二十六萬五千五百六十九,餘以一萬八千一百六十九除之為日,不滿為餘。命其氣初日,算外,即得其氣辰。凡二十四氣,小餘五千四百一十五、秒一百六十五。



 求滅日:置有滅經朔小餘,三十乘之,滿朔虛分除為日,不滿為餘。命經朔初日,算外,即得其月滅日辰。經朔小餘不滿朔虛分者,為有滅之朔。



 步發斂



 候策:五日、餘五百四、秒一百二十五。《乾道》餘二千一百八十四、秒二十五。《淳熙》餘四百一十、秒七十五。《會元》餘二千八百一十二、秒五十。



 卦策:六日、餘六百五、秒一百一十四。《乾道》餘二千六百二十一、秒二十四。《淳熙》餘四百九十二、秒九十。《會元》餘三千三百八十二、秒二十。



 土王策:三日、餘三百二、秒一百四十七。《乾道》餘二千三百一十、秒二十七。《淳熙》餘二百四十六、秒四十五。《會元》一千六百九十一、秒一十。



 辰法:五百七十七半。《乾道》二千五百。《淳熙》四百七十。《會元》三千二百二十五。



 半辰法:二百八十八太。《乾道》一千二百五十。《淳熙》二百三十五。《會元》一千六百一十二半。



 刻法:六百九十三。《乾道》三百。《淳熙》五百六十四。《會元》三百八十七。



 秒法:一百八十。《乾道》三十。《淳熙》、《會元》同一百。《淳熙》又有月閏五千一百一十一、秒九十四。



 求六十四卦、五行用事日、二十四氣、七十二候。四歷俱與前歷
 同,此不載。



 求發斂去經朔日:置天正閏餘,以中盈及朔虛分累益之,即每月閏餘;滿總法除之為閏日,不盡為小餘,即各得其月中氣去經朔日辰。因求卦候者,各以卦、候、土王策依次累加減之,中氣前減,中氣後加。



 各得其月卦、候去經朔日辰。



 求發斂加時:置所求小餘,以辰法除之為辰數,不滿,進一位,以刻法而一為刻,不盡為刻分。其辰數命子正,算外,
 各得加時所在辰、刻及分。加辰刻即命起子初。



 步日躔



 周天分:二百五十三萬一千二百二十六、秒八十七。《乾道》分一千九十五萬七千七百一十七、秒五。



 歲差:八十八、秒八十七。《乾道》四百九、秒五。《淳熙》一萬一千五百一十三。《會元》軌差五百二十五、秒一十三。



 周天度:三百六十五、約分二十五、秒六十四。三歷同。



 乘法:五十五。《乾道》八十七。《淳熙》一百一十九。《會元》一百一十九。



 除法:八百三十七。《乾道》一千三百二十四。《淳熙》一千八百一十二。《會元》一千八百一十一。



 秒法:一
 百。三歷同。



 《乾道》又有象限九十一度、分三十一、秒九,《淳熙》、《會元》同。《淳熙》又有乾實三億九百萬七千六百一十三,半周天一百八十二度、分二十五、秒七十二。《會元》半周天度同、分六十二、秒八十六。



 表略



 求每月盈縮分,朔、弦、望入氣朏朒定數,赤道宿度,冬至赤道日度,赤道宿積度入初末限,二十八宿黃道度,天正冬至加時黃道日度,二十四氣加時黃道日度,二十
 四氣初日晨前夜半黃道日躔宿次,晨前夜半黃道日躔宿次,太陽入宮日時刻及分。法同前歷,此不載。



 步月離



 轉周分:一十九萬九百五十三、秒二千五百六十三。《乾道》八十二萬六千六百三十七、秒七千三百九十五。《淳熙》一十五萬五千四百七、秒九千七百四十。《會元》轉率一百六萬六千三百六十一、秒七千三百一十。



 轉周日:二十七、餘三千八百四十三、秒二千五百六十三。《乾道》餘一萬六千六百三十七、秒七千三百九十五。《淳熙》餘三千一百二十七、秒九千七百四。《會元》餘三
 萬一千四百六十一、秒七千三百一十。



 朔差日:一、餘六千七百六十三、秒七千四百三十七。《乾道》餘二萬九千二百八十、秒二百五。《淳熙》餘五千五百四十、秒五千八百六十。《會元》餘三萬七千七百七十二、秒二千六百一十。



 望策:一十四、餘五千三百三、秒五千。



 弦策:七、餘二千六百五十一、秒七千五百。《乾道》餘一萬一千四百七十九、秒四千四百。《淳熙》餘二千一百五十八、秒一千四百。《會元》一萬四千八百八、秒五十。



 七日:初數六千一百五十八,約分八十九;末數七百七
 十二,約分一十一。



 十四日:初數五千三百八十七,約分七十八;末數一千五百四十三,約分二十二。



 二十一日:初數四千六百一十五,約分六十七;末數二千三百一十五,約分三十三。



 二十八日:初數三千八百四十三,約分五十五;末數空。



 以上秒母一萬。



 以下秒母一百。



 上弦:九十一度三十一分、秒四十一。三歷同。



 望:一百八十二度六十二分、秒八十二。三歷秒八十六。



 下弦:二百七十三度九十四分、秒二十三。三歷秒二十九。



 平行分:一十三度三十六分、秒八十七半。



 推天正十一月經朔入轉,經弦、望及次朔入轉。法同前歷,此不載。



 表略



 《乾道》又有七日初數二萬六千六百五十九、初約八十九,末數三千三百四十一、末約一千一;十四日初數二萬二千三百一十九、初約七十八,末數六千六百八十一,末約二十三;二十一日初數一萬九千九百九十八、初約六十七,末數一萬二十二、末約三十三;二十八日初數一萬六千六百三十七、初約五十五,末數空、末約空。《淳熙》七日初數五千一十一,末數六百二十,初約八十九、末約一千二;十四日初數四千三百八十三、末數一千二百五十七,初約七十八、末約二十二;二十一日初數三千七百五十五、末數一千
 八百八十五,初約六十七、末約三十三;二十八日初數三千一百二十七、初約五十五。《會元》七日初數三萬四千三百九十、安約八十九,末數四千三百一十、末約一十一;十四日初數三萬八千、初約七十八,末數八千六百二十、末約二十一;二十一日初數二萬五千七百七十二、初約六十七,末數一萬二千九百二十九、末約三十三;二十八日初數二萬一千四、初約五十五,末數一百六十一。



 求朔、弦、望入轉朏朒定數:朔、弦、望定日朔、弦、望加時日所在度;推月行九道;求九道宿度,月行九道平交入氣,平定入轉,朏朒定數,正交入氣,正交加時黃道日度,正交加時月離九道宿度,定朔、弦、望月所在宿度,定朔夜
 半入轉,次朔夜半入轉,月晨昏度,朔、弦、望晨昏定程,轉定度,晨昏月,天正十一月經朔加時平行月,天正十一月定朔日晨前夜半平行月,次朔夜半平行月,定弦、望夜半平行月,天正定朔夜半入轉,弦、望及後朔定日夜半入轉,定朔、弦、望夜半月度。法同前歷,此不載。



 步晷漏



 二至限:一百八十二、六十三分。《乾道》分同,秒一十八。《淳熙》、《會元》同。



 象限:九十一、三十一分。三歷同,秒九。



 消息法:一萬二千二百十一。



 辰法:五百七十七半,計八刻二百三十一分。《乾道》餘一百。《淳熙》餘一百八十八。《會元》餘一百二十九。



 昏明刻:三百四十六半。《乾道》餘一百五十。《淳熙》餘二百八十二。



 昏明餘數:一百七十三少。《乾道》昏明分七百五十。《淳熙》昏明分一百四十一。《會元》九百六十七半。



 冬至嶽臺晷景:一丈二尺八寸三分。



 夏至嶽臺晷景:一尺五寸六分。



 冬至後初限夏至後末限:六十二日。分空。



 夏至後初限冬至後末限:一百二十日六十二分。



 求每日消息定數:黃道去極度及赤道內、外度,晨昏日出、入分及半晝分,每日距中度,夜半定漏,晝、夜刻及日出、入辰刻,更籌辰刻,昏、明度,五更攢點中星,九服距差日,九服晷景,九服所在晝、夜漏刻。法與前歷同,此不載。



 步交會



 交終分:一十八萬八千五百八十、秒六千四百五十七。《
 乾道》八十一萬六千三百六十六、秒六千三十四。《淳熙》交實一十五萬三千四百七十六、秒九千五百四十六。《會元》交率一百五萬三千一百一十二、秒二千一百四十。



 交終日:二十七、餘一千四百七十、秒六千四百五十七。《乾道》餘六千三百六十六、秒六千三十四。《淳熙》餘一千一百九十六、秒九千五百四十二。《會元》餘八千二百一十三、秒二千一百四十。



 交中日:一十三、餘四千二百、秒三千二百二十八半。《乾道》餘一萬八千一百八十三、秒三十七。《淳熙》餘三千四百一十八、秒四千七百七十一半。《會元》餘二萬三千四百五十六、秒六千七十。



 朔差:二日、餘二千二百六、秒三千五百四十三。《乾道》餘九千五百五十一、秒一千五百六十六。《淳熙》餘一千七百九十五、秒六千五十七。《會元》餘一萬二千三百二十、秒七千八百六十。



 後限:一日、餘一千一百十三、秒一千七百七十一半。《乾道》餘四千七百七十五、秒五千七百八十三。



 前限:十二日、餘三千九十七、秒一千四百五十。《乾道》餘一萬三千四百九、秒七千二百三十四。



 望策:十四日、餘五千三百三、秒五十。《乾道》餘二萬二千九百五十八、
 秒八千八百。《淳熙》餘四千三百一十六、秒二千八百。《會元》餘二萬九千六百三十七。



 交率:四十二。《乾道》八十。《淳熙》六十一。《會元》五百七。



 交數:五百三十五。《乾道》一千一十九。《淳熙》七百七十七。



 交終度:三百六十三度七十六分。《乾道》分七十九、秒四十。《淳熙》同。《會元》分同、秒四十四。



 交象度:九十度九十四分。《乾道》分同、秒八十五。《淳熙》同。《會元》分同、秒八十六。



 半交象度:一百八十一度八十八分。《乾道》度四十五、分四十七、秒四十二半。《淳熙》同。《會元》秒四十
 二。



 陽歷食限:二千七百四十五。《乾道》一萬四千四百。《淳熙》二千六百三十。《會元》一萬八千。



 陽歷定法:二百七十四半。《乾道》一千四百四十。《淳熙》二百六十三。



 陰歷食限:四千五百八十五。《乾道》一萬八千。《淳熙》三千二百四十。《會元》二萬二千五百。



 陰歷定法:四百五十八半。《乾道》三百二十四。



 《乾道》又有月食限二萬九千一百,《淳熙》五千四百六十,《會元》三萬六千。《乾道》月食定法一千八百,《淳熙》三百五十六。《乾道》月食既限一萬一千一百。《淳熙》月食既限一千九百。



 推天正十一月加時入交泛日:求次朔及望入交泛日,定朔、望夜半交泛,次朔夜半入交泛日,朔、望加時入交常日,朔望加時入交定日,月行陰陽歷,朔、望加時入陰陽歷積度,朔、望加時月去黃道度,食甚定餘,日月食甚入氣,日月食甚中積、氣差、刻差,日入食限,日入食分,日食泛用分,月入食限,月入食分,月食泛用分,日月食定用分,日月食虧初、復滿小餘,月食既內、外分,日月食所起,月食更、點定法,月食入更點,日月帶食出入所見分
 數,日月食甚宿次。法同前歷,此不載。



 步五星



 五星會策:一十五度、二十一分、秒九十。



 木星終率:二百七十六萬四千二百三十八、秒三十二。《乾道》一千一百九十六萬六千五百八十一、秒五十五。《淳熙》周實二百二十四萬九千七百一十五、秒六十五。《會元》周率一千五百四十三萬六千八百三十四、秒九十八。



 終日:三百九十八、約分八十八、秒七十九。《乾道》分八十八、秒六十。《淳熙》約分八十八、秒五十七。《會元》分八十八、秒四十六。



 歲差:六十七、秒九十八。《乾道》周差一百萬八千八百六十四、秒五十。《淳熙》一十八萬九千七百四十一、秒六十五。



 伏見度:一十三。



 《乾道》歷率一千九十五萬七千二百四十九、秒九。《淳熙》二百五萬九千九百八十一、秒一十。《會元》一千四百一十三萬五千四百五十六、秒九。《乾道》歷中度一百八十三、分六十三、秒二十四。《淳熙》同。《會元》秒八十六。《乾道》歷策度一十五、分二十一、秒八十五。《淳熙》同。《會元》秒九
 十。



 表略



\end{pinyinscope}