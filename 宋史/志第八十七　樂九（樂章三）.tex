\article{志第八十七 樂九(樂章三)}

\begin{pinyinscope}

 太廟常享禘袷加上徽號郊前朝享皇后別廟



 建隆以來祀享太廟一十六
 首



 迎神,《禮安》



 肅肅清廟,奉祠來詣。格思之靈,如在之祭。



 克謹威儀,載嚴容衛。降福孔皆,以克永世。



 皇帝行,《隆安》



 工祝升階,賓尸在位。祗達孝思,允修毖祀。



 顯相有儀,克恭乃事。儼恪其容,通此精意。



 奠瓚用《瑞木》



 木符啟瑞,著象成文。於昭大號,協應明君。



 靈命有屬,鴻禧洞分。歌以升薦,休嘉洽聞。



 又《馴象》



 嘉彼馴象,來歸帝鄉。南州毓質,中區效祥。



 仁格巨獸,德柔遐荒。有感斯應,神化無方。



 又《玉烏》



 素烏爰止,淳精允臧。名符瑞牒,色應金方。



 潔白容與,翹英奮揚。孝思攸感,皇德逾張。



 奉俎,《豐安》



 維犧維牲,以炰以烹。植其□鼓,潔彼鉶羹。



 孔碩茲俎,於穆厥聲。肅雍顯相,福祿來成。



 酌獻僖祖室,《大善》



 湯湯洪河,經啟長源。鬱鬱嘉木,挺生
 本根。



 大哉崇基,出乎慶門。發祥垂裕,永世貽孫。



 順祖室,《大寧》



 元鐘九千,生於仲呂。崇臺九層,起於累土。



 赫日之升,《明夷》為主。孝孫作帝,式由祖武。



 翼祖室,《大順》



 明明我祖,積德攸宜。肇繼瓜瓞,將隆本支。



 爰資慶緒,式昭帝基。於穆清廟,永洽重熙。



 宣祖室,《大慶》



 艱難積行,綿長鐘慶。同人之時,得主乃定。



 既敘宗祧,乃修舞詠。經武開先,永昭丕命。



 太祖室,《大定》



 猗歟太祖,受命於天!化行區宇,功溢簡
 編。



 武威震耀,文德昭宣。開基垂統,億萬斯年。



 太宗室,《大盛》



 赫赫皇運,明明太宗。四隩咸暨,一變時雍。



 睿文炳煥,聖備溫恭。千齡萬祀,永播笙鏞。



 飲福,《禧安》



 嘉粟旨酒,博腯牲牷。神鑒孔昭,享茲吉蠲。



 夙夜毖祀,孝以奉先。永錫純嘏,功格於天。



 亞獻,《正安》



 已像文治,乃觀武成。進退可度,威儀
 克明。



 終獻,《正安》



 《常武》徂征,詩人所稱。總干山立,厥象伊疑。



 徹豆,《豐安》



 肥腯之牲,既析既薦,鬱鬯之酒,已酌已獻。



 祝辭亦陳,和奏斯遍。享禮具舉,徹其有踐。



 攝事十三首



 降神,《理安》



 肅肅清廟,昭事祖檷。粢盛苾芬,四海來祭。



 皇靈格思,令
 容有睟。降福孔皆,以克永世。



 太尉行,《正安》



 稞鬯溥將,賓尸在位。帝德升聞,孝思光被。



 公卿庶正,傅御師氏。至誠感神,福祿來暨。



 奠瓚,《瑞安》



 淳清育物,瑞木成文。元氣陶冶,非煙鬱氛。



 玄貺昭格,至和所熏。登歌稞獻,肸蠁如聞。



 奉俎,《豐安》



 麗碑割牲,以炰以烹。博碩肥腯,薦羞神明。



 祖考來格,享於克誠。如聞謦咳,式燕以寧。



 酌獻僖祖室,《大善》



 肅肅藝祖,肇基鴻源。權輿光大,燕翼貽孫。



 載祀惟永,慶流後昆。威靈在天,顧我思存。



 順祖室,《大寧》



 思文聖祖,長發其祥。錫羨蕃衍,德厚流光。



 眷命自天,卜世聿昌。祗肅孝享,降福無疆。



 翼祖室,《大順》



 明明我祖,積德累仁。居晦匿曜,邁種惟
 勤。



 帝圖天錫,輝光日新。寢廟繹繹,昭事同寅。



 宣祖室,《大慶》



 洸洸我祖,時惟鷹揚。潛德弗耀,發源靈長。



 肆類配天,永思不忘。來顧來享,百福是將。



 太祖室,《大定》



 赫赫太祖,受命於天。赤符啟運,威加八埏。



 神武戡難,功無間然。翼翼丕承,億萬斯年。



 太宗室,《大盛》



 穆穆太宗,與天合德。昧旦丕顯,幹幹翼翼。



 敷祐下民,時帝之力。永懷聖神,孝思罔極。



 真宗室,《大明》



 煌煌真宗,善繼善承。經武耀德,臻於治平。



 封祀禮樂,丕昭鴻名。陟配文廟,皇圖永寧。



 徹豆,《豐安》



 鼎俎既陳,豆籩既設。金石在庭,工師就列。



 備物有嚴,著誠致潔。孝惟時思,禮以《雍》徹。



 送神,《理安》



 神之來兮風肅然,神之去兮升九天。排凌兢兮還恍惚,羽旄紛兮蕭燔煙。



 真宗御制二首



 奠瓚用《萬國朝天》



 鴻源浚發,睿圖誕彰。高明錫羨,累洽延祥。



 巍巍藝祖,溥率賓王。煌煌文考,區宇大康。



 珍符昭顯,寶歷綿長。物性茂遂,民俗阜昌。



 甫田多稼,禾黍穰穰。含生嘉育,鳥獸蹌蹌。



 八紘統域,九服要荒。沐浴惠澤,祗畏典常。



 隔谷分壤,望鬥辨方。並襲冠帶,來奉圭璋。



 峨峨雙闕,濟濟明堂。諸侯執帛,天後當陽。



 何以辨等?袞衣繡裳。何以褒德?輅車乘黃。



 聲明煥赫,雅
 頌汪洋。啟茲丕緒,祐我無疆。



 大統斯集,大樂斯揚。俯隆宗祏,仰繼穹蒼。



 亞獻、終獻《平晉樂》



 五代衰替,六合攜離。封疆竊據,兵甲競馳。



 天顧黎獻,塗炭可悲。帝啟靈命,浚哲應期。



 皇祖丕變,金鉞俄麾。率土執贄,獷俗來儀。



 瞻彼大鹵,竊此餘基。獨迷文告,莫畏天威。



 神宗繼統,璇圖有輝。尚安蠢爾,罔懷格思。



 六飛夙駕,萬旅奉辭。徯來發詠,不陣行師。



 雲旗先路,壺漿塞岐。天臨日照,宸慮通微。



 前歌後舞,人心悅隨。要領自得,智力何施。



 風移僭冒,政治淳熙。書文混一,盛德咸宜。



 干戈倒載,振振言歸。誕昭七德,永定九圍。



 真宗告饗六首



 告受天書,《瑞安》



 寶命自天,鴻禧錫祚。昭晰緣文,氤氳黃素。



 玄感薦彰,靈休誕布。寅奉珍符,聿懷永慕。



 太祖、太宗加上尊謚,《顯安》



 報貺陟封,聿昭典禮。讓德穹厚,歸功祖彌。



 丕顯尊稱,盡善盡美。寅威孝思,以介
 蕃祉。



 東封畢躬謝酌獻,《封安》



 奕奕清廟,錫羨詒謀。升中神岳,顯允皇猷。



 歸格藝祖,昭報靈休。奉先追遠,盛德益修。



 祀汾陰畢躬謝酌獻,《顯安》



 於昭列聖,休德清明。威靈如在,享於克誠。



 報功厚載,馨薦惟精。歸格飲至,禮備樂成。



 聖祖降親告,《瑞安》



 於赫聖祖,景靈在天。神游來暨,睟
 容穆然。



 誨言昭示,帝冑開先。齊明欽若,延鴻億年。



 六室加謚,《顯安》



 欽崇太霄,肅奉徽冊。大禮克誠,鴻猷有赫。



 令芳爰薦,明靈斯格。昭謝垂祥,永懷何極。



 景祐親享太廟二首



 迎神,《興安》



 追養奉先,納孝練主。金奏鳳鳴,《關雎》樂舞。



 奠鬯恭神,肥腯展俎。積慶聰明,降景寰宇。



 酌獻真宗室,《大明》



 於穆真皇,宅心道粹。和戎偃革,煥乎文治。



 操瑞拜圖,封天祀地。盛德為宗,烝嘗萬世。



 至和袷享三首



 迎神,《興安》



 濡露降霜,永懷孝思。祫食諦敘,再閏之期。



 歌德詠功,八音播之。歆神惟始,靈其格茲。



 奠瓚,《嘉安》



 昭穆親祖,自室徂堂。禮備樂成,肅然稞將。



 瑟瓚黃流,條鬯芬芳。氣達淵泉,神孚來享。



 送神,《興安》



 四祖基慶,三後在天。薦侑備成,靈娭其旋。



 孝孫應嘏,受福永年。送之懷之,明發惻然。



 嘉祐袷享二首



 迎神,《懷安》



 躬茲孝享,禮備樂成。神登於俎,祝導於祊。



 展牲肥腯,奏格和平。靈其昭格,肅人愛凝情。



 送神,《懷安》



 靈神歸止,光景肅然。福祥裕世,明威在天。



 孝孫有慶,駿烈推先。祐茲基緒,彌萬斯年。



 熙寧以後享廟五首



 酌獻英宗室,《大英》



 在宋五世,天子嗣昌。躬發英斷,若干之剛。



 聲容涷□,被於八荒。垂千萬年,永烈有光。



 送神,《興安》



 鐘鼓惟旅,籩豆孔時。衎我祖宗,既右享之。



 神亟來止,孝孫之喜。神保聿歸,孝孫之思。



 禘祫孟享、臘享,宗正卿升殿,《正安》



 進退有容,服章有儀。匪亟匪遲,降登孔時。



 祫享仁宗,《大和》



 於穆仁廟,聖澤滂流。華夷用乂,動植蒙休。



 徽名冠古,奕世垂謀。帝躬稞獻,盛典昭修。



 英宗,《大康》



 赫赫英皇,總提邦紀。浚發神功,恢張聖理。



 仙馭雖遙,鴻徽不弭。永言孝思,竭誠躬祀。



 常祀五享三首



 迎神,《興安》九變



 奕奕清廟,昭穆定位。霜露增感,粢盛潔祭。



 神靈來格,福祉攸暨。追孝奉先,本支百世。



 太尉奠瓚,《嘉安》



 有秩時祀,匪怠匪瀆。有來宗主,載祗載肅。



 厥作稞將,流黃瓚玉。是享是宜,永綏多福。



 送神,《興安》



 皇祖皇考,配帝配天。駿奔顯相,神保言旋。



 祝以孝告,嘏以慈宣。去來永慕,宗事惟虔。



 紹興以後時享二十五首



 迎神,《興安》



 黃鐘為宮



 奉先嚴祀,率禮大經。時思致享,
 肅薦芳馨。



 竭誠備物,樂奏和聲。真馭來止,熙事克成。



 大呂為角



 聖靈在天,九關崇深。風馬雲車,紛其顧臨。



 擁祥儲休,昭答孝心。孝孫受祉,萬福是膺。



 太簇為征



 嘉承和平,秩祀為先。乃練休辰,祝史告虔。



 內心齊明,祀具吉蠲。交際恍惚,如在後前。



 應鐘為羽



 道信於神,神靈燕娭。酒有嘉德,物惟其時。



 緩節安歌,樂奏具宜。欣欣樂康,福祿綏之。



 奉俎,《豐安》



 王假有廟,子孫保光。奉牲以告,玉俎膏香。



 專精厲意,神其迪嘗。休承靈意,申錫無疆。



 初獻盥洗,《正安》



 恪恭祀典,涓選休成。設洗致潔,直於東榮。



 嘉觴祗薦,明德惟馨。祖考來格,享茲孝誠。



 升殿,《正安》



 冠佩雍容,時惟上公。享於清廟,陟降彌恭。



 籩豆靜嘉,粢盛潔豐。孝孫有慶,萬福來同。



 僖祖室酌獻,《基命》



 於穆文獻,自天發祥。肇基明命,錫羨無疆。



 子孫千億,宗社靈長。神之格思,如在洋洋。



 宣祖室酌獻,《天元》



 天啟炎歷,集我大命。長發其祥,篤
 生上聖。



 夷亂芟荒,乾坤以定。時禮聿修,孝孫有慶。



 太祖室酌獻,《皇武》



 赫赫藝祖,受天明命。威加八紘,德垂累聖。



 祀事孔明,有嚴笙磬。對越在天,延休錫慶。



 太宗室酌獻,《大定》



 明明在上,時維太宗。允武允文,丕基紹隆。



 於肅清廟,昭報是豐。皇靈格思,福祿來同。



 真宗室酌獻,《熙文》



 於穆真皇,維烈有光。丕承二後,奄奠萬方。



 威加戎狄,道格穹蒼。歆時禋祀,降福無疆。



 仁宗室酌獻,《美成》



 至哉帝德,乃聖乃神!恭己南面,天
 下歸仁。



 歷年長久,垂裕後人。禮修舊典,寶命維新。



 英宗室酌獻,《治隆》



 炎基克鞏,赫赫英宗。紹休前烈,仁化彌隆。



 篤生聖子,堯、湯比蹤,烝嘗萬世,福祿來崇。



 神宗室酌獻,《大明》



 於昭神祖,運撫明昌。肇新百度,克配三王。



 遐荒底績,聖武維揚。永言《執競》,上帝是皇。



 哲宗室酌獻,《重光》



 於皇浚哲,遹駿有聲。率時昭考,丕顯儀刑。



 功光大業,道協三靈。永綏厥後,來燕來寧。



 徽宗室酌獻,《承元》



 天錫神聖,徽柔懿恭。垂衣拱手,遵
 制揚功。



 配天立極,體道居中。祐我烈考,萬福攸同。



 欽宗室,《端慶》



 於皇欽宗,道備德宏。允恭允儉,克類克明。



 孝遵前烈,仁翊函生。歆茲肆祀,永燕宗祊。



 高宗室,《大德》



 於皇時宋,自天保定。高宗受之,再僕景命。



 紹開中興,翼善傳聖。何千萬年,永綏厥慶。



 孝宗室,《大倫》



 聖人之德,無加於孝。思皇孝宗,履行立教。



 始終純誠,非曰笑貌。於萬斯年,是則是效。



 光宗室,《大和》



 維宋洽熙,帝繼於理。萬姓厚生,三辰順
 軌。



 對時天休,以燕翼子。肅唱和聲,神其有喜。



 文舞退、武舞進,《正安》



 肅肅清廟,於顯維德。我祀孔時,我奏有翼。



 秉翟載駿,有來干戚。神之燕娭,休祥允格。



 亞、終獻,《文安》



 觀德宗祏,奕世烈光。有嚴祀典,粵循舊章。



 樂諧九變,獻舉重觴。燕娭如在,戩穀穰穰。



 徹豆,《恭安》



 禮備樂成,物稱誠竭。相維闢公,神人以說。



 歌《雍》一章,諸宰斯徹。天子萬世,無競維烈。



 送神,《興安》



 霜露既降,時思展禋。在天之御,睠然顧歆。



 樂成禮備,言歸靡停。既安既樂,福祿來成。



 袷享八首



 迎神,《興安》



 黃鐘宮



 時維孟冬,霜露既零。合食盛禮,以時以行。



 孝心翼翼,惟神來寧。肅倡斯舉,神具是聽。



 大呂角



 於穆孝思,嘉薦維時。誠通茲格,咸來燕娭。



 神之聽之,申錫蕃厘。於萬斯年,永保丕基。



 太簇征



 於昭孝治,通乎神明。寒暑不忒,熙事備成。



 牲牷孔碩,黍稷惟馨。以享以祀,來燕來寧。



 應鐘羽



 苾芬孝祀,薦灌肅雍。致力於神,明信咸通。



 靈之妥留,惠我龐鴻。廣被萬寓,福祿攸同。



 初獻順祖,酌獻,《大寧》



 於赫皇祖,浚發其祥。德盛流遠,奕世彌昌。



 孝孫有慶,嘉薦令芳。神保是享,錫羨無疆。



 翼祖酌獻,《興安》



 上天眷命,祐我丕基。翼翼皇祖,不耀其輝。



 積厚流長,福祿攸宜。祀事孔時,曾孫篤之。



 光宗室酌獻,《大承》



 於皇光宗,握符御極。昭哉嗣服,惟仁與德!



 勤施於民,靡有暇逸。萬年之思,永奠宗祏



 送神,《興安》



 合祭大事,因時發天。翼翼孝思,三獻禮虔。



 神兮樂康,飆馭言旋。永神後人,於千萬年。



 上仁宗、英宗徽號一首



 入門升殿,《顯安》



 於穆仁祖,寵綏萬方。執競英考,迄用成、康。



 圖徽寶冊,有烈其光。庶幾億載,與天無疆。



 上英宗尊號一首



 入門,《正安》



 在宋五世,天子神明。群公奉冊,乃揚鴻名。



 金書煌煌,遹昭厥成。思皇多祜,與天同聲。



 增上神宗徽號一首哲宗朝制



 升殿,《顯安》



 於惟檷廟,乃聖乃神。秉文之士,作起惟新。



 建宮稽古,一視同仁。庶幾備號,以享天人。



 紹興十四年奉上徽宗冊寶三首



 冊賓入門,《顯安》



 於鑠徽考,如天莫名。迨茲丕揚,擬純粹精。



 溫玉鏤文,來至於祊。有嚴奕奕,禮備樂成。



 冊寶升殿,《顯安》



 金字煌煌,瑤光燦燦。群工奉之,登此寶殿。



 對越祖宗,式遵成憲。威靈在天,來止來燕。



 上徽號,《顯安》



 惟精惟一,乃聖乃神。鴻名克揚,茂實斯賓。



 如禹之功,如堯之仁。孝思永慕,用詔無垠。



 淳熙十五年上高宗徽號三首



 冊寶入門,《顯安》



 於穆高皇,功德兼隆。稱天以誄,初謚未崇。



 載稽禮典,揚徽垂鴻。涓日之良,登進廟宮。



 冊寶升殿,《顯安》



 有□彖斯寶,有編斯冊。導以麾仗,奏以金石。



 祲威盛容,煌煌赫赫。臣工奉之,高靈來格。



 上徽號,《顯安》



 中興之烈,高掩商宗。揖遜之美,放勛比
 隆。



 字十有六,擬諸形容。威靈在天,裕後無窮。



 慶元三年奉上孝宗徽號三首



 冊寶入門,《顯安》



 巍巍孝廟,聖德天通。同符藝祖,克紹高宗。



 有儀有冊,載推載崇。鏤玉繩金,登奉祏宮。



 冊寶升殿,《顯安》



 文金晶熒,冊玉輝潤。統紹乎堯,德全於舜。



 勤崇推高,子孝孫順。冠德百王,萬年垂訓。



 上徽號,《顯安》



 金石充庭,珩璜在列。繪畫乾坤,形容日月。



 巍巍功德,顯顯謨烈。垂億萬年,鴻徽昭揭。



 高宗郊祀前朝享太廟三十首



 皇帝入門,《乾安》後還前殿並同



 於皇我後,祗戒專精。假於有廟,祖考是承。



 趨進惟肅,僾思惟誠。神之聽之,來燕來寧。



 皇帝升殿、《乾安》詣室、降殿並同



 皇皇大宮,丕顯於穆。休德昭清,元氣回復。



 芝葉蔓茂,桂華馮翼。孝孫假斯,受茲介福。



 盥洗,《乾安》



 維皇齊精,鬷假於廟。觀盥之初,惟以潔告。



 衎承祖宗,恤祀昭孝。誠心有孚,介福斯報。



 迎神,《興安》



 秬鬯既將,黃鐘具奏。肅我祖考,祗慄以俟。



 監觀於茲,雲車來下。



 尚書奉俎,《豐安》



 有碩其牲,登於大房。肅展以享,庶幾迪嘗。



 匪腯是告,我民其康。保艾爾後,垂休無疆。



 皇帝再盥洗,《乾安》



 盥至於再,潔誠愈孚。帝用祗薦,靈咸嘉虞。



 騰歌臚歡,會於軒朱。觀厥顒若,受福之符。



 僖祖室酌獻,《基命》



 思文僖祖,基德之元。皇武大之,受
 命於天。



 積厚流光,不已其傳。曾孫篤之,於萬斯年。



 翼祖室酌獻,《大順》



 天命有開,維仁是依。乃睠冀邦,於以顧之。



 其顧伊何?發祥肇基。施於孫子,虔奉孝思。



 宣祖室,《天元》



 昭哉皇祖,源深流長!雕戈圭瓚,休有烈光。



 天祐潛德,繼世其昌。永懷積累,嘉薦令芳。



 太祖室,《皇武》



 為民請命,皇祖赫臨。天地並貺,億萬同心。



 造邦以德,介福宜深。挹彼惟旨,真游居歆。



 太宗室,《大定》



 皇矣太宗,嗣服平成!益奮神旅,再徵不
 庭。



 文武秉德,仁孝克明。以聖傳聖,對越紫清。



 真宗室,《熙文》



 思文真宗,體道之崇。憺威赫靈,遵制揚功。



 真符鼎來,告成登封。盛德百世,於昭無窮。



 仁宗室,《美成》徽宗御制



 仁德如天,遍覆無偏。功濟九有,恩涵八埏。



 齊民受康,朝野晏然。擊壤歌謠,四十二年。



 英宗室,《治隆》



 穆穆英宗,持盈守成。世德作求,是纘是承。



 齊家睦族,偃武恢文。於薦清酤,酌之欣欣。



 神宗室,《大明》



 烝哉維后,繼明體神!稽古行道,文物一
 新。



 潤色鴻業,垂裕後人。靈斿沛然,來燕來寧。



 哲宗室,《重光》



 明哲煌煌,照臨無疆。紹述先志,寔宣重光。



 詒謀燕翼,率由舊章。苾芬孝祀,降福穰穰。



 徽宗室,《承元》禦制



 於皇烈考,道化聖神。堯聰舜孝,文恬武忻。



 命子出震,遺駿上賓。罔極之哀,有古莫倫。



 降殿,《乾安》



 明德惟馨,進止回復。裼襲安恭,嚴若惟穀。



 誠意昭融,群工袂屬。成此祲容,生乎齊肅。



 入小次,《乾安》



 於皇我後,祗戒專精。躬制聲詩,文思聰
 明。



 雍容戾止,玉立端誠。神聽如在,福祿來寧。



 文舞退、武舞進,《正安》



 八音諧律,綴兆充庭。進旅退旅,肅恭和平。



 盛薦祖宗,靈監昭升。像功崇德,遹觀厥成。



 亞獻,《正安》



 威神在天,享於克誠。申以貳觴,式昭德馨。



 籩豆孔嘉,樂舞具陳。庶幾是聽,福祿來成。



 終獻,《正安》



 疏冪三舉,誠意一純。孰陪予祀,公族振振。



 神具醉止,燕娭窈冥。於萬斯年,綏我思成。



 皇帝出小次,《乾安》



 夙戒告備,禮節俯成。妥侑惟乾,氛
 氳夜澄。



 有嚴有翼,列聖靈承。於穆清閟,肅肅無聲。



 皇帝再升殿詣飲福位,《乾安》



 維皇親享,至再至三。禮備樂奏,層陛森嚴。



 粢盛芳潔,酒醴旨甘。雲車風馬,從衛觀瞻。



 飲福,《禧安》



 赫赫明明,維祖維宗。鑒於文孫,維德之同。



 日靖四方,亦同其功。億萬斯年,以承家邦。



 還位,《乾安》



 帝既臨享,步武鳴鸞。陟降規矩,顒昂周旋。



 登歌一再,典禮莫愆。神之聽之,祉福綿綿。



 尚書徹豆,《豐安》



 熙事即成,嘉籩告徹。洋洋來臨,藹藹布列。



 配帝其功,在天對越。允集叢厘,萬邦和悅。



 送神,《興安》



 神之來游,風馬雲車。淹留徬佛,顧瞻欷歔。



 神之還歸,鈞天帝居。監觀於下,何福不除!



 降殿,《乾安》



 於皇上天,欽哉成命。集於沖人,丕承列聖。



 爰熙紫壇,於廟告慶。肸蠁潛通,休祥薦應。



 還大次,《乾安》



 盛德豐功,一祖六宗。欽翼燕詒,禋享是崇。



 厲意齊精,假廟惟恭。率禮周旋,福祿來同。



 寧宗朝享三十五首



 皇帝入門,《乾安》



 王假有廟,四極駿奔。鼎俎宵嚴,虡簨雲屯。



 積厚流廣,德隆慶蕃。是則是繩,保我子孫。



 升殿,《乾安》



 於穆清宮,奕奕孔碩。芝莖蔓秀,桂華馮翼。



 八簋登列,六瑚賁室。皇代擁慶,啟祐千億。



 盥洗,《乾安》



 天一以清,地一以寧。維皇精專,承神明靈。



 娥禦墮津,瀆祗揚溟。盥事允嚴,先祖是聽。



 詣室,《乾安》



 丹楹雲深,芳勺宵奠。樂華淳鬯,禮文炳絢。



 有容有儀,載肅載見。維時緝熙,世世以燕。



 還位,《乾安》



 旅楹有閑,人神允葉。福以德昭,饗以誠接。



 六樂宣揚,百禮煒燁。對越在天,流祚萬葉。



 迎神,《興安》九變。



 黃鐘為宮



 《咸》、《英》備樂,簋席列斝。詩歌安世,聲葉皇雅。



 翠旗羽蓋,雲車風馬。神其來兮,以燕以下。



 大呂為角



 勾陳旦闢,閶闔夜分。軫風挾月,車駟凌雲。



 瑞景晻靄,神光耀熅。神其來兮,以留以忻。



 太簇為征



 穆穆紫幄,璜璜清宮。《旱麓》流詠,《鳧鷖》葉工。



 道閎詒燕,業綿垂鴻。神其來兮,以康以崇。



 應鐘為羽



 文以謨顯,武以烈承。聖訓之保,祖武之繩。



 有肅孝假,式嚴衎烝。神其來兮,以宜以寧。



 捧俎,《豐安》



 簋豆薦牲,鉶籩實饋。其俎孔庶,吉蠲為饎。



 惟德達馨,以忱以貴。神既祐享,祉貺來暨。



 再詣盥洗,《乾安》



 精粹象天,明清鑒月。再御茲盥,益致其潔。



 齊容顒若,誠意洞徹。百禮允洽。率禮不越。



 真宗室,《熙文》



 天地熙泰,躋時升平。闡符建壇,聲容文明。



 君臣賡載,夷夏肅清。本支百世,持盈守成。



 仁宗室,《美成》



 在宋四世,天子聖神。用賢致治,約已裕民。



 海內富庶,裔夷肅賓。四十二年,堯、舜之仁。



 英宗室,《治隆》



 明明英後,仁孝儉恭。丕顯丕承,增光祖宗。



 繼志述事,遵制揚功。萬邦作孚,盛德形容。



 神宗室,《大明》



 厲精基治,大哉乾剛!信賞必罰,內修外攘。



 禮樂法理,號令文章。作新之功,度越百王。



 哲宗室,《重光》



 於皇我宋,世有哲明。元祐用人,遹駿有聲。



 紹述先志,思監於成。受天之祜,王配於京。



 徽宗室,《承元》



 帝撫熙運,晏粲協期。禮明樂備,文恬武嬉。



 道光授受,謀深燕詒。駿命不易,子孫保之。



 欽宗室,《端慶》



 顯顯令主,輝光日新。奉親以孝,綏下以仁。



 兢兢業業,誕保庶民。於穆不已,之德之純。



 高宗室,《大德》



 昊天有命,中興復古。治定功成,修文偃武。



 德隆商宗,業閎漢祖。付托得人,系堯之緒。



 孝宗室,《大倫》



 藝祖有孫,聰睿神武。紹興受禪,歸尊於父。



 行道襲爵,百度修舉,聖德曰孝,光於千古。



 光宗室,《大和》



 維宋洽熙,帝繼於理。萬姓厚生,三辰順軌。



 對時天休,以燕翼子。肅唱和聲,神其有喜。



 還位,《乾安》



 在周之庭,設業設虡。酒醴惟醹,爾殽伊脯。



 帝觴畢勺,天步旋舉。丕顯丕承,念茲皇祖。



 降殿,《乾安》


黼幄蟬
 \gezhu{
  艸葉}
 ,飆斿寧燕。尊彞獻裸,瑚簋陳薦。



 視儀天旋,淳音《韶》變。遹求厥寧,福祿流羨。



 入小次,《乾安》



 皇容肅祗,天步舒遲。對越惟恭,敬事不遺。



 陟降蒞止,永言孝思。上帝臨女,日監於茲。



 文舞退、武舞進,《正安》



 明庭承神,□磬柷敔。玉梢飾歌,佾綴維旅。



 既肖厥文,復象乃武。祖德宗功,惟帝時舉。



 亞獻,《正安》



 尊斝星陳,罍冪雲舒。來貳變觴,玉佩瓊琚。



 相予嚴祀,秉德有初。對揚王休,何福不除!



 終獻,《正安》



 秉德翼翼,顯相肅邕。疏冪三舉,誠意益恭。



 光燭黼繡,和流笙鏞。子孫眾多,福祿來從。



 出小次,《乾安》



 廟楹邃嚴,夜景藻清。文物炳彪,禮儀熙成。



 帷宮載敞,佩珩有聲。帝復對越,將受厥明。



 再升殿,《乾安》



 明明維后,詒厥孫謀。系隆我漢,陳錫哉周。



 以孝以饗,世德作求。介以繁祉,萬邦咸休。



 飲福,《乾安》



 玉瓚黃流,有飶其香。來假來享,降福穰穰。



 我應受之,湯孫之將。有百斯男,福祿無疆。



 還位,《乾安》



 聖圖廣大,宗祊光輝。假於有廟,帝命不違。



 僾若有慕,夙夜畏威。嘉樂君子,福祿祁祁。



 徹豆,《豐安》



 升饌有章,卒食攸序。庭鏘金奏,凱收鉶筥。



 其獻惟成,其餕維旅。禮洽慶流,皇祖之祜。



 送神,《興安》



 珠幄熉黃,神既燕娭。監觀於下,福祿來宜。



 雲車風馬,神保聿歸。啟祐我後,福祿來為。



 降殿,《乾安》



 聖有謨訓,詒謀燕翼。奉天酌祖,萬世維則。



 維皇孝熙,乾乾夕惕。禮既式旋,惟福之錫。



 還大次,《乾安》



 王假有廟,對越在天。帷宮旋御,率禮不愆。



 泰畤展祠,雲陽奉瑄。齊居精明,益用告虔。



 理宗朝享三首



 皇帝升降,《乾安》



 於皇祖宗,清廟奕奕。威靈在天,不顯惟德。



 垂裕鴻延,詒謀燕翼。孝孫格斯,受祉罔極。



 迎神,《興安》,九奏



 秬鬯既將,黃鐘具奏。瞻望真游,人愛若有慕。



 於皇列聖,在帝左右。雲車具來,以妥以侑。



 寧宗室,《大安》



 帝德之休,恭儉淵懿。三十一年,謹終如始。



 升祔在宮,祖功並美。民懷有仁,何千萬世。



 高宗祀明堂前朝享太廟二十一首



 皇帝入門,《乾安》



 於皇我後,祗戒專精。齊肅有容,祖考是承。



 造次匪懈,孝思純誠。神聽有格,福祿來寧。



 升殿,《乾安》



 肅哉清宮,熉珠照幄!神之來思,八音振作。



 赤舄龍章,奉玉惟恪。匪今斯今,先民時若。



 盥洗,《乾安》



 於皇維后,觀盥之初。精意昭著,既順既愉。



 圭鬯承祀,卿士咸趨。目視心化,四方其孚。



 迎神,《興安》



 涓選休成,祖考是享。夙夜專精,求諸惚恍。



 洋洋在上,惟神之仰。鬯矣清明,應之如響。



 捧俎,《豐安》



 來相於庭,鳴鋗鏘鏘。奉牲而告,登彼雕房。



 非牲之備,民庶是康。神依民聽,上帝斯皇。



 僖祖室酌獻,《基命》



 何慶之長?實兆於商。由商太戊,子孫其昌。



 皇基成命,宋道用光。詒厥孫謀,膺受四方。



 翼祖室,《大順》



 上帝監觀,維仁是依。繼世修德,皇心顧之。



 其顧伊何?在彼冀方。施於子孫,降福穰穰。



 宣祖室酌獻,《天元》



 昭哉皇祖,駿發其祥!雕戈圭瓚,盛烈載揚。



 天錫寶符,俾熾而昌。神聖應期,赫然垂光。



 太祖室,《皇武》



 猗歟皇祖,下民攸歸!膺帝之命,龍翔太微。



 戎車雷動,天地清夷。峨峨奉璋,萬世無違。



 太宗室,《大定》



 煌煌神武,再御戎軒。時惠南土,旋定太原。



 車書混同,聲教布宣。維天祐之,億萬斯年。



 真宗室,《熙文》



 於皇真宗,體道之崇。游心物外,應跡寰中。



 四方既同,化民以躬。清凈無為,盛德之容。



 仁宗室曲同郊祀。送神亦同。



 英宗室,《治隆》



 噫我大君,嗣世修文!維文維武,諟繼虞
 勛。



 天錫丕祚,施於後昆。於薦清酤,酌之欣欣。



 神宗室,《大明》



 烝哉維后,繼明體神!憲章文、武,宜民宜人。



 經世之道,功格於天。子孫嚴祀,無窮之傳。



 哲宗室,《重光》



 明哲煌煌,照臨無疆。丕承先志,嘉靖多方。



 朝廷尊榮,民庶樂康。珍符來應,錫茲重光。



 徽宗室,《承元》



 聖考巍巍,光紹丕基。禮隆樂備,時維純熙。



 天仁兼覆,皇化無為。功成弗處,心潛希夷。



 文舞退、武舞進,《正安》



 作樂合祖,簨虡在庭。眾奏具舉,
 肅邕和鳴。



 神靈來格,庶幾是聽。皦繹以終,永觀厥成。



 亞獻,《正安》



 威神在天,來格於誠。既載清酤,有聞無聲。



 相予熙事,時賴宗英。肅肅邕邕,允協思成。



 終獻,《正安》



 疏冪三舉,誠意一純。孰陪予祀,公族振振。



 明靈來娭,樂舞具陳。奉神所祐,昭孝息民。



 飲福,《禧安》



 赫赫明明,德與天通。施於孫子,福祿攸同。



 日靖四方,民和年豐。有秩斯祜,申錫無窮。



 徹豆,《豐安》



 歆我齊明,威德如存。牲牷是享,圭玉其溫。



 群公執事,亦既駿奔。禮成告徹,咸福黎元。



 還大次,《乾安》



 神明既交,恍若有承。欽翼齊莊,福祿具膺。



 王業是興,祖武是繩。祐我億年,以莫不增。



 孝宗明堂前享太廟三首



 徽宗室酌獻,《承元》



 明明徽祖,撫世升平。制禮作樂,發政施仁。



 聖靈在天,德澤在民。億萬斯年,保祐後人。



 高宗室,《大德》



 於皇時宋,自天保定。高宗受之,再僕景命。



 紹開中興,翼善傅聖。何千萬年,永綏厥慶。



 還大次,《乾安》



 禮既行矣,樂既成矣。維祖維妣,安且寧矣。



 皇舉玉趾,佩鏘鳴矣。拜貺總章,於厥明矣。



 理宗明堂前朝享二首



 寧宗室奠幣,《定安》



 皇矣昭考,聖靈在天!稱秩宗祀,有嚴恭先。



 奉幣以薦,見之人愛然。仁深澤厚,厥光以延。



 酌獻,《考安》



 假哉皇考,必世後仁!嘉靖我邦,與物皆春。



 之純之德,克配穹旻。餘慶淵如,祐我後人。



 皇后廟十五首



 迎神,《肅安》



 閟宮翼翼,雅樂洋洋。牲器肅設,幾筵用張。



 飾以明備,秩其令芳。神兮來格,風動雲翔。



 太尉行,《舒安》



 服章觀象,山龍是則。容止蹌蹌,威儀翼翼。



 司徒捧俎,《豐安》徹同



 格恭奉祀,祗薦犧牲。九成爰奏,有俎斯盈。



 酌獻孝明皇后室,《惠安》



 祀事孔明,廟室惟肅。鉶登籩豆,金石絲竹。



 既灌既薦,允恭允穆。奉神如在,以介景
 福。



 孝惠皇后室,《奉安》



 初陽作配,內助惟賢。柔順中積,英徽外宣。



 神宮有侐,明祀惟虔。歆誠降祐,於萬斯年。



 孝章皇后室,《懿安》



 猗那淑聖,像應資生。配天作合,與日齊明。



 椒宮垂範,彤史揚名。聿修毖祀,永奉粢盛。



 懿德皇后室,《順安》



 王門稟慶,帝族惟賢。功存內治,德協靜專。



 流芳圖史,垂範紘綖。新廟有侐,祀禮昭然。



 淑德皇后室,《嘉安》



 明明英媛,備備椒庭。籩豆有踐,黍
 稷匪馨。



 靜嘉致薦,容與昭靈。精意以達,顧享來寧。



 莊穆皇后室,《理安》



 曾孫襲慶,柔祗育德。正位居體,其儀不忒。



 教被宮壺,化行邦國。祝史正辭,垂裕無極。



 莊懷皇后室,《永安》



 淑德昭著,至樂和平。登豆在列,膋香薦誠。



 六變合禮,八音諧聲。穰穰景福,祐我休明。



 元德皇后廟,《興安》



 為太宗後,為天下母。誕聖繼明,膺乾作主。



 玉振金相,蘭芬桂芳。於萬斯年,永奉烝嘗。



 飲福,《禧安》



 彞尊鬯酒,慶祐遂行。介以純嘏,允答明誠。



 亞獻,《恭安》



 宗臣率禮,步玉鏘鏘。吉蠲斯獻,百祿是將。



 終獻,《順安》



 薦獻有終,禮容斯穆。以奉嘉觴,以膺多福。



 送神,《歸安》



 明禋告畢,靈輅難留。升雲杳邈,整馭優游。



 誠深嘉慄,禮罄欽修。豐融垂祐,以永洪休。



 景祐以後樂章六首



 章獻明肅皇太后室奠瓚,《達安》



 肅肅閟宮,順時薦事。鬱鬯馨香,如見於位。



 酌獻,《厚安》



 祥標曾麓,德合方儀。萬方展養,九御蒙慈。



 孝恭祊祏,美播聲詩。淑靈顧享,申錫維祺。



 章懿皇太后室奠瓚,《報安》



 青金玉瓚,稞將於京。永懷罔極,夙夜齊明。



 酌獻,《衍安》



 翊祐先朝,章明壺教。淑順謙勤,徽音在劭。



 樹風不止,劬勞匪報。黍稷令芳,嘏茲乃告。



 奉慈廟章惠皇太后室奠瓚,《翕安》



 稞圭既陳,酌鬯斯醇。音容徬佛,奠獻惟寅。



 酌獻,《昌安》



 內輔先猷,夙昭壺則。保祐之勞,慈惠其德。



 榮養有終,芳風無極。享獻閟宮,載懷淒惻。



 真宗汾陰禮畢,親謝元德皇后室三首



 迎神,《肅安》



 閟宮奕奕,《韶》樂洋洋。牲幣虔布,幾筵肅張。



 醴泉淳美,嘉肴潔香。俟神來格,降彼帝鄉。



 奉俎,《豐安》



 樂鏗金石,俎奉犧牲。九成斯奏,五教爰行。



 送神,《理安》



 鸞驂復整,鶴駕難留。白雲縹緲,紫府深幽。



 廟雖載止,神無不游。垂祐皇宋,以永鴻休。



 元德皇后升祔一首



 《
 顯安》之曲



 顯矣皇妣,德侔柔祗!升祔太室,協禮之宜。



 耀彼實冊,列之尊彞。惟誠是厚,永祐慶基。



 崇恩太后升祔十四首



 入門,《顯安》



 俔天生德,作配元符。儀刑壺則,輔佐帝圖。



 登崇廟祏,勒號璠璵。烝嘗億載,皇極之扶。



 神主升殿,《顯安》



 曰嬪於京,天作之配。進賢審官,克勤其志。



 於穆清廟,本仁祖義。億萬斯年,神靈攸暨。



 迎神,《興安》四章



 黃鐘宮二奏



 閟宮有侐,堂筵屹崇。
 靈徽匪遐,精誠感通。



 苾芬維時,登茲明祀。冷然雲車,有來其馭。



 大呂角二奏



 羽旌風翔,翠蕤飄舉。儼其音徽,登茲位處。



 笙鏞始奏,合止柷敔。是享是宜,永求伊祜。



 太簇征二奏



 枚枚閟宮,鼎俎肆陳。烝畀明靈,登其嘉新。



 鼓鐘既戒,旨酒既醇。攸介攸止,純禧薦臻。



 應鐘羽二奏



 旨酒嘉肴,於登於豆。是享是宜,樂既合奏。



 衎我懿德,執事溫恭。靈兮允格,有翼其從。



 罍洗,《嘉安》



 列爵陳俎,芬芳和羹。摐金擊石,洋洋和聲。



 禮行伊始,我德惟明。既盥而往,於昭斯誠。



 升降殿,《熙安》



 笙簫紛如,陟彼廟庭。鏘鏘佩玉,懷茲先靈。



 神保聿止,音容杳冥。繁禧是介,萬年惟寧。



 酌獻,《茲安》



 邕邕玉佩,清酤惟良。粢盛具列,有飶其香。



 懷其徽範,德洽無疆。於茲燕止,降福穰穰。



 亞獻,《神安》



 嬪於潛邸,爰正坤儀。《關雎》化被,《思齊》名垂。



 柔德益茂,家邦以熙。皇心追崇,永羞牲粢。



 退文舞、進武舞,《昭安》



 翩然幹戚,揚庭陳階。文以經緯,武以威懷。



 其張其弛,節與音諧。迄茲獻享,妥靈綏來。



 終獻,《儀安》



 珩璜之貴,禕褕之尊。天作之合,內治慈溫。



 元良鐘慶,祉福乾坤。以享以祀,事亡如存。



 徹豆,《成安》



 鏘洋純繹,於論鼓鐘。周旋陟降,齊莊肅容。



 維罍既旨,維籩伊豐。歌徹以《雍》,介福來崇。



 送神,《興安》



 黍稷維馨,虡業充庭。既欽既戒,靈心是承。



 顧予烝嘗,言從之邁。申錫無疆,是用大介。



 上冊寶十三首



 冊寶入門,《隆安》



 威儀皇止,庶尹在庭。爰舉徽章,遹觀厥成。



 勒崇揚休,寫之瓊瑛。迄於萬祀,發聞惟馨。



 冊寶升殿,《崇安》



 有猶有言,順承天則。聿崇號名,再揚典冊。



 朱英寶函,左右翼翼。千秋萬歲,保茲無極。



 迎神,《歆安》



 黃鐘宮



 籩豆大房,犧尊將將。馨香既登,明靈迪嘗。



 其樂伊何?吹笙鼓簧。靈來燕娭,降福無疆。



 大呂角二奏



 吉蠲惟時,禮儀既備。奉璋峨峨,群公在
 位。



 神之格思,永錫爾類。展彼令德,於焉來暨。



 太簇征二奏



 雍雍在宮,翼翼在庭。顯相休嘉,肅雍和鳴。



 神嗜飲食,明德惟馨。綏我思成,式燕以寧。



 應鐘羽二奏



 犧牲既成,籩豆有楚。摐金擊石,式歌且舞。



 追懷懿德,令聞令儀。靈兮來格,是享是宜。



 罍洗,《嘉安》



 嘉肴旨酒,潔粢豐盛。既盥而往,以我齊明。



 有孚顒若,黍稷非馨。神之格思,享於克誠。



 升降,《熙安》



 佩玉鏘鏘,其來雍雍。陟降孔時,步武有容。



 恪茲祀事,神罔時恫。綏我邦家,福祿來崇。



 酌獻,《明安》



 旨酒嘉慄,有飶其香。衎我淑靈,歆此令芳。



 德貽彤管,號正椒房。神具醉止,降福穰穰。



 退文舞、進武舞,《昭安》



 鑰翟既陳,干戚斯揚。進旅退旅。一弛一張。



 其儀不忒,容服有光。以宴以娭,德音不忘。



 亞、終獻,《和安》



 望高六宮,位應四星。輔佐君子,警戒相成。



 禕衣褒崇,琛冊追榮。於以奠之,有椒其馨。



 徹豆,《成安》



 濯濯其英,殖殖其庭。有來群工,繼我思成。



 嘉肴既將,旨酒既清。《雍》徹不遲,福祿來寧。



 送神,《歆安》



 禮儀既備,神保聿歸。洋洋在上,不可度思。



 神之來兮,肸蠁之隨。神之去兮,休嘉是貽。



 上欽成皇后冊賓六首



 入門升殿,《顯安》



 上帝錫羨,寔生婉淑。輔佐神皇,寵膺天祿。



 誕育泰陵,劬勞顧復。於昭徽音,久而彌鬱。



 迎神,《歆安》



 於顯惟德,徽柔懿明。嬪於初載,有聞惟馨。



 肆我鼓鐘,萬舞在庭。神保是格,來止
 來寧。



 盥洗,《嘉安》



 有煒柔儀,率履不越。惠於初終,既明且達。



 我將我享,相盥乃登。胡臭但時,攸介攸寧!



 升降,《熙安》



 苾苾其芳,殽核維旅。陟降孔時,有秩斯所。



 雍容內化,維神之明。明則不渝,綏我思成。



 酌獻,《明安》



 天維顯思,有相於內。右賢去邪,夙夜儆戒。



 猗歟追冊,重翟禕衣。既右享之,百世是儀。



 亞、終獻,《和安》



 酌彼玉瓚,有椒其馨。鬷假無言,雍容在庭。



 生莫與崇,於赫厥聲。祀事孔明,神格是聽。



 上明達皇后冊寶五首



 迎神,《歆安》



 恭儉宜家,柔順承天。德昭彤管,憂在進賢。



 寶冊禕翟,追榮壽原。四時稞享,何千萬年。



 酌獻,《明安》



 清宮有嚴,廣樂在庭。鐘鼓管磬,九變既成。



 縮茅以獻,潔秬惟馨。靈游可想,來燕來寧。



 退文舞、進武舞,《昭安》



 秉翟竣事,萬舞摐金。總干揮戚,節以鼓音。



 禮容有煒,肸蠁來歆。淑靈是聽,雅奏愔愔。



 徹豆,《成安》



 登獻罔愆,俎豆斯徹。神具醉止,禮終樂闋。



 御事既退,珊珊佩玦。介我繁祉,歆此蠲潔。



 送神,《歆安》



 備成熙事,虛徐翠楹。神保聿歸,雲車夙征。



 鑒我休德,神交惚恍。留祉降祥,千秋是享。



 紹興別廟樂歌五首



 升殿,《崇安》



 新廟肅肅,蕆事以時。陟降階戚,雍容有儀。



 鞠躬周旋,罔敢不祗。祝史正辭,靈其格思。



 奉俎,《肅安》



 肇嚴廟祀,爰圖遺芳。物必稱德,或陳或將。



 有縟其儀,有苾其香。靈兮來下,割烹是嘗。



 懿節皇后室酌獻,《明安》



 曾沙表慶,正位椒庭。徽音杳邈,宮壺儀刑。



 虔修祀事,清酌惟馨。縮以包茅,昭格明靈。



 亞、終獻,《嘉安》



 霄漢月墮,郊原露晞。徽音如在,延佇來歸。



 有酒既清,累觴載祗。神具醉止,燕衎怡怡。



 徹豆,《寧安》



 仙馭弗返,邈清都。薦此嘉殽,即豐既腴。



 奠享有成,鼓樂愉愉。徹我豆籩,率禮無逾。



 乾道別廟樂歌三首



 詣廟,《乾安》



 涓選休辰,於秋之杪。既齊既戒,爰假祖廟。



 有侐儀坤,舊章是效。享祀奚為?天子純孝。



 升殿,《乾安》



 宗祀九筵,先薦閟宮。陟自東階,煌煌袞龍。



 於穆聖善,監茲禮容。是享是宜,介福無窮。



 懿節皇后室酌獻,《歆安》



 丕顯文母,厚德維坤。仙馭雖邈,徽音固存。



 瑟彼玉瓚。酌此鬱尊,簡簡穰穰,裕我後昆。



 紹熙別廟二首



 安穆皇后室酌獻,《歆安》



 祥發俔天,符彰夢日。有懷慈容,孝享廟室。



 泰尊是酌,旨酒嘉慄。靈其格思,祚以元吉。



 安恭皇后室酌獻,《歆安》



 美詠河洲,德嬪媯汭。徽音如存,肇修祀事。



 縮以包茅,酌以醴齊。靈來顧歆,降福攸備。



 紹興二十九年顯仁皇后祔廟一首



 酌獻,《歆安》



 恭惟聖母,躋祔孔時。陳羞宗祏,徼福坤儀。



 鐘鼓惟序,牲玉載祗。於皇來格,永介丕基。



 開禧三年成肅皇后祔廟一首



 酌獻,《歆安》



 天合重華,內治昭融。承承繼繼,保祐恩隆。



 歸從阜陵,登祔太宮。燕我後人,福祿來崇。



\end{pinyinscope}