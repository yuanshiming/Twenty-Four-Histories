\article{本紀第三十七}

\begin{pinyinscope}

 寧宗一



 寧
 宗法天備道純德茂功仁文哲武聖睿恭孝皇帝,諱擴,光宗第二子也,母曰慈懿皇后李氏。光宗為恭王,慈懿夢日墜於庭,以手承之,已而有娠。乾道四年十月丙
 午,生於王邸,五年五月,賜今名。十一月乙丑,授右千牛衛大將軍。七年,光宗為皇太子。淳熙五年十月戊午,遷明州觀察使,封英國公。七年二月,初就傅。九年正月,始冠。十年九月己巳,始預朝參。十一年,當出閣,兩宮愛之,不欲令居外,乃建第東宮之側,以十月甲戌遷焉。十二年三月乙酉,遷安慶軍節度使,封平陽郡王。八月辛酉,納夫人韓氏。十六年二月壬戌,光宗受禪。三月己亥,拜少保、武寧軍節度使,進封嘉王。帝自弱齡,尊師重傅,至
 是,始置翊善,以沉清臣為之。紹熙元年春,宰相留正請立帝為儲嗣。



 五年六月戊戌,孝宗崩,光宗以疾不能出。壬寅,宰臣請太皇太后垂簾聽政,不許;請代行祭奠之禮,從之。丁未,宰臣奏云:「皇子嘉王,仁孝夙成。宜正儲位,以安人心。」越六日,奏三上,從之。明日,遂擬旨以進。是夕,御批付丞相云:「歷事歲久,念欲退閑。」七月辛酉,留正以疾辭去。知樞密院事趙汝愚見正去,乃遣韓侂冑因內侍張宗尹以禪位嘉王之意請於太皇太后,不獲。遇提
 舉重華宮關禮,侂冑因其問,告之。禮繼入內,泣請於太皇太后,太皇太后乃悟,令諭侂冑曰:「好為之!」侂冑出,告汝愚,命殿帥郭杲夜分兵衛南北內。翌日禫祭,汝愚率百官詣大行柩前,太皇太后垂簾,汝愚率同列再拜,奏:「皇帝疾,不能執喪,臣等乞立皇子嘉王為太子,以安人心。」乃奉御批八字以奏。太皇太后曰:「既有御筆,卿當奉行。」汝愚曰:「內禪事重,須議一指揮。」太皇太后允諾。汝愚袖出所擬以進,云:「皇帝以疾,未能執喪,曾有御筆,欲自退
 閑,皇子嘉王擴可即皇帝位。尊皇帝為太上皇,皇后為太上皇后。」太皇太后覽畢,曰「甚善。」汝愚出,以旨諭帝,帝固辭曰:「恐負不孝名。」汝愚曰:「天子當以安社稷、定國家為孝,今中外憂亂,萬一變生,置太上皇何地!」眾扶入素幄,披黃袍,方卻立未坐,汝愚率同列再拜。帝詣幾筵殿。哭盡哀。須臾立仗訖,催百官班,帝衰服出,就重華殿東廡素幄立,內侍扶掖,乃坐。百官起居訖,乃入行禫祭禮。詔建泰安宮,以奉太上皇、太上皇后。汝愚即喪次請召



 丁卯,侍御史張叔椿劾留正擅去相位,詔以叔椿為吏部侍郎。戊辰,詔求直言。遣鄭湜使金告禪位。己巳,以趙汝愚兼參知政事。庚午,召秘閣修撰、知潭州朱熹詣行在。壬申,建泰安宮。乙亥,以趙汝愚為右丞相,參知政事陳騤知樞密院事,餘端禮參知政事,仍兼同知樞密院事。汝愚辭不
 拜。賜前宰執、侍從詔,訪以得失。丙子,大風。戊寅,詔:秋暑,太上皇帝未須移御,即以寢殿為泰安宮。以殿前都指揮使郭杲為武康軍節度使,庚辰,率群臣拜表於泰安宮。辛巳,以趙汝愚為樞密使,保大軍節度使郭師禹為攢宮總護使。壬午,侍御史章穎等劾內侍林億年、陳源、楊舜卿,詔億年、源與在外宮觀,舜卿在京宮觀。韓侂冑落階官,為汝州防禦使。癸未,餘端禮辭兼同知樞密院事。甲申,以兵部尚書羅點簽書樞密院事。詔兩省官詳
 定應詔封事,具要切者以聞。戊子,詔百官輪對。罷楊舜卿在京宮觀,林億年常州居住,陳源撫州居住。



 八月己丑朔,安定郡王子濤薨。辛卯,初御行宮便殿聽政。癸巳,以朱熹為煥章閣待制兼侍講。甲午,增置講讀官,以給事中黃裳、中書舍人陳傅良、彭龜年等為之。丁酉,以生日為天祐節。己亥,率群臣朝泰安宮。辛丑,詔諸道舉廉吏、糾污吏。壬寅,詔經筵官開陳經旨,救正闕失。進封弟許國公秉為徐國公。癸卯,加嗣濮王士歆少師,郭師禹
 少傅,夏執中少保。乙巳,詔晚講官會講。丁未,復罷經筵坐講,命三省議振恤諸路郡縣水旱。乙卯,加安南國王李龍[A147]思忠功臣。詔歲減廣西鹽額十萬緡。丙辰,留正罷,以觀文殿大學士判建康府。以趙汝愚為右丞相。丁巳,詔侍從、兩省、臺諫各舉通亮公清、不植黨與、曾任知縣者二人。



 九月己巳,命趙汝愚朝獻景靈宮。庚子,命嗣秀王伯圭朝饗太廟。是日,羅點薨。辛未,合祭天地於明堂,大赦。壬申,以刑部尚書京鏜簽書樞密院事。甲戌,下
 詔撫諭諸將。改天祐節為瑞慶節。



 冬十月己丑,右諫議大夫張叔椿再劾留正擅去相位,詔落正觀文殿大學士。庚寅,更泰安宮為壽康宮。辛卯,命四川制置司銓量諸州守臣。癸巳,雷。乙未,詔以陰陽謬盭,雷電非時,令臺諫、侍從,各疏朝政闕失以聞。戊戌。復許武舉人試換文資。庚子,以久雨,命大理、三衙、臨安府、兩浙州縣決系囚,釋杖以下。辛丑,減兩浙、江東西路和市折帛錢,蠲兩浙路丁鹽、身丁錢一年。雅州蠻寇邊,土丁拒退之,尋出降。
 甲辰,以朱熹言,趣後省看詳應詔封事。乙巳,上大行至尊壽皇聖帝謚曰哲文神武成孝皇帝,廟號孝宗。丙午,復以朱熹奏請,卻瑞慶節賀表。庚戌,改上安穆皇后謚曰成穆皇后,安恭皇后謚曰成恭皇后。壬子,遣曾三復使金賀正旦。丙辰,上孝宗皇帝冊寶於重華殿,成穆皇后、成恭皇后冊寶於本室。



 是月,建福寧殿。



 閏月庚申,以吏部尚書鄭僑等奏請祧僖、宣二祖,正太祖東向之位,尋立僖祖別廟,以藏順、翼、宣三祖之主。乙丑,遣林季友
 使金報謝。戊辰,金遣使來吊祭。戊寅,侍講朱熹以上疏忤韓侂冑罷,趙汝愚力諫,不聽;臺諫、給舍交章請留朱熹,亦不聽。詔兩省、臺諫、侍從各舉宗室有文學器識者二人。壬午,詔改明年為慶元元年。



 十一月甲午,復加安南國王李龍[A147]濟美功臣。丙午,帝自重華宮還大內。庚戌,以宜州觀察使韓侂冑兼樞密都承旨。辛亥,雨木冰。詔行孝宗三年喪制,命禮官條具典禮以聞。升明州為慶元府。乙卯,權攢孝宗皇帝於永阜陵。十二月丁巳朔,
 禁民間妄言宮禁事。乙丑,吏部侍郎彭龜年上疏言韓侂冑假托聲勢,竊弄威福,乞黜之,以解天下之疑。詔罷龜年,進侂冑一官,與在京宮觀。趙汝愚請留龜年,不聽。御史中丞謝深甫劾陳傅良,罷之。戊辰,以陳康伯配饗孝宗廟庭。己巳,陳騤罷。庚午,以餘端禮知樞密院事,京鏜參知政事,鄭僑同知樞密院事。辛未,監察御史劉德秀劾起居舍人劉光祖,罷之。癸酉,金遣使來賀登位。上孝宗廟樂曰《大倫之舞》。甲戌,祔孝宗神主於太廟。丁丑,
 減臨安、紹興二府死罪以下囚,釋杖以下。蠲民緣攢宮役者賦。戊寅,加郭師禹少師,進封永寧郡王。癸未,金遣使來賀明年正旦。是歲,兩浙、淮南、江東西路水旱,振之,仍蠲其賦。



 慶元元年春正月丁巳朔,蠲兩淮租稅。壬寅,黎州蠻寇邊,官軍戰卻之。乙巳,蠲臺、嚴、湖三州貧民身丁、折帛錢一年。詔兩浙、淮南、江東路荒歉諸州收養遺棄小兒。辛亥,以久雨,振給臨安貧民。丙辰,白虹貫日。



 二月丁巳朔,
 詔兩淮諸州勸民墾闢荒田。壬戌,詔嗣秀王伯圭贊拜不名。癸亥,以久雨,釋大理、三衙、臨安府、兩浙路杖以下囚。丁卯,詔帥臣、監司歲終考察郡守臧否以聞。戊寅,以右正言李沐言,罷趙汝愚為觀文殿大學士、知福州。己卯,雨土。以餘端禮兼參知政事。庚辰,兵部侍郎章穎以黨趙汝愚罷。甲申,謝深甫等再劾汝愚,詔與宮觀。



 三月丙戌朔,日有食之。庚寅,太白經天。辛亥,詔四川歲發西兵詣行在,如舊制。癸丑,命侍從、臺諫、兩省集議江南沿
 江諸州行鐵錢利害。甲寅,國子祭酒李祥、博士楊簡以黨趙汝愚罷。



 夏四月丁巳,太府寺丞呂祖儉坐上疏留趙汝愚及論不當黜朱熹、彭龜年等,忤韓侂冑,送韶州安置。己未,以餘端禮為右丞相,京鏜知樞密院事,鄭僑參知政事,謝深甫簽書樞密院事。庚申,太學生楊宏中等六人以上書留趙汝愚、章穎、李祥、楊簡,請黜李沐,詔宏中等各送五百里外編管。中書舍人鄧馹上疏救之,不聽。戊辰,臨安大疫,出內帑錢為貧民醫藥、棺斂費及
 賜諸軍疫死者家。



 五月戊子,呂祖儉改送吉州安置。戊戌,詔戒百官朋比。丙午,詔諸路提舉司置廣惠倉,修胎養令。辛亥,減大理、三衙、臨安府雜犯死罪以下囚,釋杖以下。



 六月丁巳,復留正觀文殿大學士,充醴泉觀使。右正言劉德秀請考核真偽,以辨邪正。己未,遣汪義端賀金主生辰。庚午,詔三衙、江上諸軍主帥、將佐,初除舉自代一人,歲薦所知二人。癸酉,以韓侂冑為保寧軍節度使、提舉萬壽觀。



 秋七月壬辰,加周必大少傅。丁酉,落趙
 汝愚觀文殿大學士,罷宮觀。己亥,太白晝見。



 八月己巳,詔內外諸軍主帥條奏武備邊防之策以聞。



 九月壬午朔,蠲臨安府水災貧民賦。乙酉,以久雨,決系囚。丙戌,災惑入太微。甲辰,遣黃艾使金賀正旦。己酉,蠲臺、嚴、湖三州被災民丁絹。



 冬十月己卯,詔三省、樞密院條上合教諸軍例。乙丑,升秀州為嘉興府,舒州為安慶府,嘉州為嘉定府,英州為英德府。戊辰,金遣吳鼎樞來賀瑞慶節。壬申,封子恭為安定郡王。



 十一月己丑,雨土。庚寅,以弟
 徐國公秉為昭慶軍節度使。戊戌,加上壽聖隆慈備福太皇太后尊號曰壽聖隆慈備福光祐太皇太后,壽成皇太后曰壽成惠慈皇太后,太上皇曰聖安壽仁太上皇,太上皇后曰壽仁太上皇后。丙午,以監察御史胡紘言,責授趙汝愚寧遠軍節度副使、永州安置。丁未,命宰執大閱。十二月癸亥,置楚州弩手效用軍。丙子,命朱熹為煥章閣待制,辭。丁丑,金遣紇石烈正來賀明年正旦。



 二年春正月庚寅,以餘端禮為左丞相,京鏜為右丞相,
 鄭僑知樞密院事,謝深甫參知政事,御史中丞何澹同知樞密院事,庚子,趙汝愚卒於永州。甲辰,右諫議大夫劉德秀劾留正引用偽學之黨,詔落正觀文殿大學士,罷宮觀。



 二月辛酉,詔追復趙汝愚官,許歸葬,以中書舍人吳宗旦言,罷之。辛未,再蠲臨安府民身丁錢三年。



 三月丙申,命諸軍射鐵簾。己亥,進封弟秉為吳興郡王。丙午,有司上《慶元會計錄》。



 夏四月甲子,餘端禮罷。壬申,以何澹參知政事,吏部尚書葉翥簽書樞密院事。乙亥,增
 置監察御史一員。



 五月辛巳,以旱,禱於天地、宗廟、祖稷。詔大理、三衙、臨安府、兩浙州縣決系囚。乙酉,申嚴獄囚瘐死之罰。辛卯,賜禮部進士鄒應龍以下四百九十有九人及第、出身。甲午,減諸路和市折帛錢三年。建華文閣,以藏孝宗御集。甲辰,更慈福宮為壽慈宮。



 六月庚戌,遣吳宗旦賀金主生辰,乙丑,命監司、帥守臧否縣令,分三等,丙子,子峻生。秋七月癸未,饗於太廟。丙戌,減諸路死罪囚,釋流以下。戊子,量徙流人呂祖儉等於內郡。詔
 檢正、都司考核諸路守臣便民五事以聞。戊戌,以韓侂冑為開府儀同三司、萬壽觀使。



 八月癸丑,奉安孝宗皇帝、成穆皇后、成恭皇后神御於景靈宮。丙辰,以太常少卿胡紘請,權住進擬偽學之黨。壬戌,子峻薨,追封兗王,謚沖惠。



 九月丁亥,復分利州為東西、路。癸巳,嗣濮王士歆薨,追封韶王。甲午,流星晝隕。丁酉,遣張貴謨使金賀正旦。



 冬十月戊申,率群臣奉上壽聖隆慈備福光祐太皇太后、壽成惠慈皇太后、聖安壽仁太上皇、壽仁太上
 皇后冊寶於慈福、壽康宮。辛亥,冊皇后。壬戌,金遣張嗣來賀瑞慶節。甲戌,大閱。



 十一月庚寅,詣壽康宮,上《太上皇帝寬恤詔令》。壬辰,京鏜等上《孝宗皇帝寬恤詔令》。癸卯,賞宜州捕降峒寇功。十二月辛未,金遣完顏崇道來賀明年正旦。是月,監察御史沈繼祖劾朱熹,詔落熹秘閣修撰,罷宮觀。竄處士蔡元定於道州。



 三年春正月壬寅,鄭僑罷。癸卯,以謝深甫兼知樞密院事。



 二月己酉,京鏜等上《神宗玉牒》、《高宗實錄》。丁巳,以大
 理司直邵褎然請詔大臣自今權臣、偽學之黨,勿除在內差遣。詔下其章。



 三月乙未,建東華門。庚子,禁浙西州軍圍田。壬寅,詔自今有司奏讞死罪不當者,論如律。」夏四月丙午,雨土。命不□去為嗣濮王。壬子,以旱禱於天地、宗廟、社稷。乙丑,雨雹。



 六月戊辰,頒《淳熙寬恤詔令》。



 閏月甲戌,內出銅器付尚書省毀之,命申嚴私鑄銅器之禁。乙亥,遣衛涇賀金主生辰。甲午,詔留正分司西京、邵州居住。是夏,廣東提舉茶鹽徐安國遣人捕私鹽於大奚
 山,島民遂作亂。



 秋七月庚午,監察御史沈繼祖錄淹囚四百餘條來上,詔進二官。



 八月戊子,復置嚴州神泉監。辛卯,知廣州錢之望遣兵入大奚山,盡殺島民。甲午,均諸路職田。



 九月壬寅,以四川旱,詔蠲民賦。辛酉,遣曾炎使金賀正旦。乙丑,申嚴帥臣、監司臧否郡守之制。是月,詔監司、帥守薦舉改官,勿用偽學之人。



 冬十月癸酉,雷。丙戌,金遣完顏愈來賀瑞慶節。丙申,以太皇太后違豫,赦。



 十一月辛丑,加孝宗皇帝謚曰紹統同道冠德昭功
 哲文神武明聖成孝皇帝。太皇太后吳氏崩。壬寅,朝獻於景靈宮。癸卯,朝饗於太廟。甲辰,祀天地於圜丘,大赦。乙巳,詔為大行太皇太后服期。丁未,遣趙介使金告哀。十二月丙子,始御正殿。丁丑,以大行太皇太后攢宮,蠲紹興府貧民明年身丁、折帛綿絹。庚辰,罷文武官納官告綾紙錢。甲申,雷,雨土。乙未,金遣奧屯忠孝來賀明年正旦。丁酉,以知綿州王沇請,詔省部籍偽學姓名。



 四年正月己卯,上欽宗皇后謚曰仁懷皇后。丙寅,以
 葉翥同知樞密院事。丁卯,詔有司寬恤兩浙、江淮、荊湖、四川流民。



 二月辛未,詔兩省、侍從、臺諫各舉所知一二人,毋薦宰執親黨。丙子,上大行太皇太后謚曰憲聖慈烈皇后。



 三月甲子,權攢憲聖慈烈皇后於永思陵。乙丑,金遣烏林答天益來吊祭。



 夏四月丙戌,祔仁懷皇后、憲聖慈烈皇后神主於太廟。己丑,蠲臨安、紹興二府租稅有差。丙申,始御正殿。是月,右諫議大夫張釜請下詔禁偽學。遣湯碩使金報謝。



 五月己亥,加韓侂冑少傅,賜玉
 帶。己酉,詔禁偽學。



 六月己巳,遣楊王休賀金主生辰。癸酉,以弟吳興郡王秉為開府儀同三司。



 秋七月辛酉,葉翥罷。



 八月丁卯朔,以久雨,決系囚。丙子,以謝深甫知樞密院事兼參知政事,吏部尚書許及之同知樞密院事。庚辰,白氣亙天。丙戌,詔以太上皇聖躬清復,率群臣上壽。尋不克行。



 九月壬寅,太白晝見。癸卯,太白經天。丁未,頒《慶元重修敕令格式》。庚申,遣馬覺使金賀正旦。是月,詔造新歷。



 冬十月戊子,金遣孫鐸來賀瑞慶節。



 十二月
 丙戌,再蠲臨安府民身丁錢三年。己丑,金遣楊庭筠來賀明年正旦。



 五年春正月庚子,樞密院直省官蔡璉訴趙汝愚定策時有異謀,詔下大理捕鞫彭龜年、曾三聘等,以實其事。中書舍人範仲藝力爭之於韓侂冑,事遂寢。張釜等復請窮治,詔停龜年、三聘官。壬戌,建玉堂。



 二月癸酉,白氣亙天。乙酉,張釜劾劉光祖附和偽學,詔房州居住。



 三月甲午,罷監司臧否郡守之制。夏五月壬辰朔,新歷成,賜
 名曰《統天》。戊戌,賜禮部進士曾從龍以下四百十有一人及第、出身。戊申,以久雨,民多疫,命臨安府振恤之。壬子,詔諸路州學置武士齋,選官按其武藝。



 六月癸亥,遣李大性賀金主生辰。



 秋七月甲寅,禁高麗、日本商人博易銅錢。



 八月乙亥,白氣亙天。辛巳,太祖廟楹生芝,率群臣詣壽康宮上壽,始見太上皇,成禮而還。甲申,以過宮上壽禮成,中外奉表稱賀。丙戌,詔減諸路流囚,釋杖以下,推恩如慶壽故事。丁亥,進京鏜等官一級。戊子,立沿
 邊諸州武舉取士法。



 九月庚寅朔,加韓侂冑少師,封平原郡王。丙辰,遣朱致知使金賀正旦。



 冬十月庚申朔,封郭師禹為廣陵郡王。丙子,金遣僕散琦來賀瑞慶節。



 十一月己丑朔,詔復右司一員。十二月辛酉,嗣濮王不□去薨。庚午,命廣東水土惡弱諸州建安仁宅、惠濟倉庫,給士大夫死不能歸者。己亥,奉安仁懷皇后、憲聖慈烈皇后神御於景靈宮。甲申,金遣範楫來賀明年正旦。是歲,饒、信、江撫、嚴、衢、臺七州、建昌、興國軍、廣東諸州皆水,振
 之。



 六年春正月己亥,子坦生。



 二月戊辰,減諸路雜犯死罪囚,釋徒以下。己巳,雨土。己卯,率群臣奉上《聖安壽仁太上皇玉牒》、《聖政》、《日歷》、《會要》於壽康宮。甲申,封婕妤楊氏為貴妃。



 閏月庚寅,以京鏜為左丞相,謝深甫為右丞相,何澹知樞密院事兼參知政事。乙巳,復留正少保、觀文殿大學士致仕。丁未,雨土。辛亥,以殿前副都指揮使吳曦為昭信軍節度使。



 三月甲子,朱熹卒。辛未,從壽成惠
 慈皇太后幸聚景園。己卯,安定郡王子恭薨。



 夏四月己酉,命不璺為嗣濮王。



 五月丙辰,以旱,決中外系囚。除茶鹽賞錢。有司上《慶元寬恤詔令》、《役法撮要》。癸亥,避正殿,減膳。丙寅,詔大理、三衙、臨安府及諸路闕雨州縣釋杖以下囚。戊辰,詔侍從、臺諫、兩省、卿監、郎官、館職疏陳闕失及當今急務。辛未,以久不雨,詔中外陳朝廷過失及時政利害。壬申,雨。丁丑,詔三省、樞密院擇臣僚封事可行者以聞。



 六月乙酉朔,日有食之。丁亥,以太上皇后違
 豫,赦。戊子,太上皇后李氏崩。壬辰,遣趙善義賀金主生辰,吳旴使金告哀。戊申,許及之以母憂去位。



 秋七月己未,初御後殿。丁卯,以御史中丞陳自強簽書樞密院事。



 八月庚寅,以太上皇違豫,赦。辛卯,太上皇崩。甲午,遣李寅仲使金告哀。乙未,日中有黑子。丙申,上大行太上皇后謚曰慈懿皇后。丁酉。京鏜薨。壬寅,子坦薨,追封邠王,謚沖溫。癸卯,權攢慈懿皇后於臨安府南山之修吉寺。



 九月乙卯,祔慈懿皇后神主於太廟。甲子,婺州布衣呂
 祖泰上書,請誅韓侂冑、蘇師旦,逐陳自強等,以周必大代之。詔杖祖泰,配欽州牢城。己巳,命謝深甫朝獻景靈宮。庚午,命嗣濮王不璺朝饗太廟。辛未,合祭天地於明堂。大赦。丙子,遣丁常任為金國遺留國信使。



 冬十月丙戌,加韓侂冑太傅。戊子,遣林桷使金賀正旦。庚子,復加安南國王李龍。保節功臣。辛丑,雨土。



 十一月癸丑朔,詔宗子與願更名曮,為福州觀察使。己未,皇後韓氏崩。癸亥,子增生。丙寅,東北地震。上大行太上皇謚曰憲仁
 聖哲慈孝皇帝,廟號光宗。乙亥,上大行皇后謚曰恭淑皇后。十二月癸未朔,子增薨,追封郢王,謚沖英。乙酉,日中有黑子。辛卯,雨土。權攢憲仁聖哲慈孝皇帝於永崇陵。己亥,金遣烏古論誼來吊祭。壬寅,權攢恭淑皇后於臨安府南山之廣教寺。癸卯,祔光宗皇帝神主於太廟。遣虞儔使金報謝。詔改明年為嘉泰元年。乙巳,日中黑子滅。蠲臨安、紹興二府民緣攢宮役者賦。戊申,金遣紇石烈忠定來賀明年正旦。己酉,加吳曦太尉。庚戌,祔恭
 淑皇后神主於太廟。詔罷四川總領所所增關外四州營田租。是歲,建寧府、徽、嚴、衢、婺、饒、信、南劍七州水,建康府、常、潤、楊、楚、通、泰和七州、江陰軍旱,振之。



\end{pinyinscope}