\article{列傳第一百七十二}

\begin{pinyinscope}

 ○趙汝
 談趙汝讜趙希趙彥吶趙善湘趙與歡趙必願



 趙汝談,字履常,生而穎悟,年十五,以大父恩補將仕郎。登淳熙十一年進士第。丞相周必大得其文異之,語參
 知政事施師點曰:「是子他日有大名於世。」調汀州教授,改廣德軍,添差江西安撫司干辦公事。嘗從朱熹訂疑義十數條,熹嗟異之。



 佐丞相趙汝愚定大策,汝愚欲驟以詞掖處之,力辭去。持祖母服。汝愚去國,其弟汝讜力上疏乞留汝愚、斬侂胄,聞者吐舌。兄弟罹黨禍斥去。尋調安慶府教授,添差浙東安撫司干辦公事。丁母憂,免喪,召為太社令。



 時侂胄用事熾甚,汝談痛憤,登壇讀祝,大呼侂胄及陳自強名。自強不能堪,它日指汝談曰:「末
 坐白皙者何人?」汝談不為動。以參知政事李壁薦,召試館職,擢正字。是時吳曦叛,上下束手,或請就以曦為王,其人造汝談,汝談詰之曰:「孰欲王曦者,可斬!」其人面發赤不能對,遂以言去,主管崇道觀。添差通判嘉興府,與郡守王介志合。改知無為軍,與光州守柴中行、安豐守陸峻俱稱循吏。



 時金人內變,有旨令獻料敵、備邊二策。其料敵之策曰:「禍亂猶在河北,未遽至河南,蓋豪雄擇形勢,大盜窺貨寶,金帛重器俱聚河北,河南無大川為
 之險,欲起安所憑?且金素以河南近我,置守多完顏氏親黨,其下亦令蕃漢錯居,所以防慮備盡。縱彼喪亂,守將欲畔則自畔,何至相率盡反。然有天下者,自不容易一日廢備,豈以金人存亡之候為吾緩急哉!」其備邊之策曰:「今邊州大抵無城,缺兵少糧,鎧仗不足。若使自辦,何所取資?丐諸朝廷,安得力給?若仿古藩封,拔用英傑守郡,則並租稅市榷之利盡與之,免其共貢,上不置監臨,下悉聽選闢,民得自賦,兵得自募,凡百悉聽所為。其
 有功者亦不遽徙,就峻爵秩,增異車服,給美田宅,官其子孫,凡可優寵,無不極至,使內為公卿,雖貴曾不如守邊之樂。如此則有才者爭自奮勵,緩急必能出死力報上。」於後河南二十餘年猶為金守,宋沿邊諸郡權大削,兵事無肯任責者,汝談之言若蓍龜然。



 改湖北提舉常平,振饑盡力。知溫州,改知外宗正,作詩勉其族屬,皆望風而化。遷江西提舉常平。寧宗崩,以哀痛得疾。賀理宗表,力寓勸戒。陳碩曰:「此諫書也。」數丐祠,授江西轉運判
 官,辭不獲命,之官一月,以言者罷。



 先是,汝談因疾去官,言者謂其傲睨軒冕,不樂為世用。至是彌遠不與祠,乃杜門著述。



 端平初,以禮部郎官召,入對言:「倚用老成,廣集忠智,訪求眾敝之原,闢取可行之策,以飭積蠹之蠱,而成終泰之功者,願加聖心焉。」又言:「大佞似忠,大奸似聖,未免信向而擢任之。始未見甚失,久乃浸至差訛,則綱維之臣將不能不執,議論之士將不得不言。執之堅,寧不疑其侵權?言之數,寧不意其賣直?至是則不特是
 非邪正易位,而黜陟予奪失中多矣。」又曰:「外之得以窒吾聽、雜吾目、擾吾天君者,以吾未得虛一而靜之理也。茍得之,導我聲色而不能入,投我寶貨而不能中,扇我以功名而不能動,凝然湛然,孰得乾之哉。」改秘書少監兼權直學士院。時集議出師,汝談反覆言不可輕戰,而和尤非計。既而三京收復,雖前言用兵不便者亦喜,汝談獨有憂色。未幾,洛師敗,朝論始服其先見。



 遷宗正少卿,兼權直,兼編修國史、檢討實錄,兼崇政殿說書。因講《
 論語》而言漢元帝恭儉無過,惟以剛不克改,明不能繹,優柔不斷,而漢業遂衰。權吏部侍郎,升侍讀,兼直學士院,兼同修國史院同修撰,以所注《易》進講。時朝議履畝稱楮,汝談言非便,迕時宰意。京師軍變,宰相乞貶秩,上已允,汝談奏恐失體,持不可。草答詔,以為貶秩易,審舉措難,宰相滋不悅。以言去國,提舉崇禧觀。起知婺州,四辭不允。至郡,力丐祠。召赴行在,四辭。



 權禮部侍郎兼學士院,力辭兼直。時金兵新破,三閫增秩,稱提官楮,四郡
 獲賞。汝談獨蹙頞,登對,首疏言:「邊面無可倚仗,乞超越拘攣,簡拔俊傑,如吳用周瑜、魯肅,晉任祖逖、陶侃故事,使之各分方面,連數十城,推轂授權,盡歸賜履。巴蜀一人,荊襄一人,兩淮各一人,一切便宜行事,不復更從中御,庶幾伸縮由己,機用出心。」蓋推廣鄉者備邊之策。且曰:「臣之此策,行於開禧未用兵之前,決不至罹今日之患。」其論楮法,尤中時敝,上稱嘆久之,且謂:「卿文學高世,宜代予言,力辭何為?」卒以老祈免,章四上,免兼直,改侍
 講。數日,仍兼直學士院,五辭。權給事中,權刑部尚書,及卒,轉兩官。遣表上,又轉四官。



 汝談天資絕人,沈思高識,自少至老,無一日去書冊。其論《易》,以為為占者作;書《堯》、《舜》二典宜合為一,禹功只施於河洛,《洪範》非箕子之作;《詩》不以《小序》為信;《禮記》雜出諸生之手;《周禮》宜傅會女主之書。要亦卓絕特立之見。為文章有天巧。篤於倫誼而忘仇怨,御史王益祥嘗劾之,後汝談官其鄉,益祥愧不敢見,汝談乃數過之,相得歡甚。嘗論議韓非、李斯皆
 有荀卿之才,惟其富貴利欲之心重,故世得而賤之,惟卿獨能守其身,不茍希合,士何可不自重哉。所著有《易》、《書》、《詩》、《論語》、《孟子》、《周禮》、《禮記》、《荀子》、《莊子》、《通鑒》、《杜詩注》。



 趙汝讜,字蹈中,少俶儻有軼材,智略出人上。龍泉葉適嘗過其家,汝讜年少,衣短後衣,不得避。適勸之曰:「名門子安可不學。」汝讜慚,自是終身不衣短後衣。折節讀書,與兄汝談齊名,天下稱為「二趙」。以祖遺恩補承務郎,歷泉州市舶務、利州大軍倉屬。從臣薦宗室之賢者,監行
 在右藏西庫。



 韓侂胄謀逐趙汝愚,汝讜兄弟昌言非是,且上言訟汝愚冤。侂胄懼其詞直,使其黨胡紘再攻汝愚,以汝讜兄弟受汝愚厚恩,私屬為之畫策,惑亂天聽為言,斥使去國。坐廢十年,調華亭浦東鹽場,棄職去。闢浙西安撫司幕官,調簽書昭慶軍節度判官,皆不赴。以前官改鎮東軍。登嘉定元年進士第,為太社令,遷將作監簿、大理司農丞。與史彌遠不合,請外,改湖南提舉常平,易江西,尋提點刑獄。瑞州大姓幸氏貪徐氏田不可
 得,強取其禾,終不與,誣以殺婢,置徐獄。徐訴其冤,汝讜以反坐法黥竄幸氏,籍其家。幸氏走,告急於中宮,徙汝讜湖南。既至,則表直臣龔夬墓。瀏陽有豪民羅氏奪民田,汝讜復懲以法。遷知溫州,卒。



 汝讜常言:「宗子不忘君,孝子不辱身,臨難則功業當如朱虛,立身當如子政。」



 趙希錧,字君錫,舊名希哲,登慶元二年進士第,改賜今名。少扶父喪歸,道遇寇,左右駭散,希錧拊棺慟哭不懾,寇義而去。學於陳傅良、徐誼,既舉進士,調汀州司戶。峒
 寇李元礪方起,汀人震懼,郡會僚佐議守城,希錧下坐無一語,守異之曰:「不言得無有所見乎?」希錧曰:「守城非策也,距城三十里有關曰古城,若悉精銳以扼其沖,賊不足慮矣。」守以付希錧,人為危之。希錧至關,審形明間,申令謹候,分畫粗定,賊已遣諜窺關。希錧得諜詰之,縱其舉火相示,而羸師以誤之。夜半,賊數百銜枚突至,希錧嚴兵以待。賊且至,始命矢石俱下,賊無一免,餘黨聞風而遁。希錧引還,老稚羅拜相屬,希錧繇他道以避之。
 事聞,詔升州推官,治疑獄,決滯訟,攝下邑,弭亂卒。去之日,軍民遮道泣送者數十里。



 調主管夔州路轉運司帳司,疏大寧鹽井利病,使者上諸朝,民便之。改知玉山縣,未行。召對,希錧首言民力困於貪吏,軍力困於僨帥,國家之力則外困於歸附之卒,內困於浮冗之費;次論四蜀銓科舉之弊;次論大寧鹽井本末。寧宗嘉納之。



 授大理寺丞,遷大宗正丞,權工部郎官。宗姓多貧,而始生有訓名,為人後有過禮,吏受賕亡藝,莫敢自陳,希錧白其
 長推行之。會朝議,燕邸近屬赴朝參者少,命希錧易班,希錧力辭,弗克。特換授吉州刺史、提舉祐神觀。未幾,廷臣言宗姓換班人嘗舉進士,請視朝士,聽輪對。於是希錧次對時首論:「今日多事之際,而未有辦事之人。朝紳,清選也,以緘默為清重,以刻薄為舉職,以無所可否為識體。閫寄,重任也,以大言為有志,以使過為知恩。臣非敢厚誣天下以為無人,患在選擇未得其道、器使未當其才爾。」授成州團練使,賜金帶,令服系。以寶璽推恩,進
 和州防禦使。



 理宗即位,進潭州觀察使,以公族近邸,恩特加厚。又進安德軍承宣使。希錧引對,言:「初政急務,莫先於明道,總治統,收人心。」上為動容。越明年,論祠祭不蠲,禁衛不肅。慈明宮上壽,升節度,封信安郡公。卒,遺奏聞,上震悼輟視朝,賜含斂,贈以金幣。



 希錧風資凝重,胸抱魁壘,揚人之善,不記人之過,急人之難,不忘人之恩。居官,祁寒盛暑未嘗謁告,衣食取裁足而已。追封信安郡王。



 趙彥吶,字敏若,彭州人。登四川類試第。少以材稱。吳曦叛,以祿禧偽守夔,彥吶結義士殺之,遂顯名。



 嘉定十二年,關外西和州新被兵,制使安丙檄使經理,金人再至,戰卻之。因請修州北水關,募民耕戰以守;又勸丙盡捐關外四州租,結民兵使各自為守。皆不行。在州五年,得軍民心,轉提點刑獄,尋帥沔,時譽甚都。及崔與之代丙,始察其大言無實,謂他日誤事省必此人,請廟堂毋付以邊藩。尋奪其節制。



 寶慶元年,乃移帥興元。三年,會鄭
 損棄四州,退保三關,彥吶力爭不勝,罷歸家者五年。紹定四年,桂如淵代損,起彥吶於副使,更李𡌴、黃伯固,皆彥吶副之。端平元年,遂升正使,丞相鄭清之趣其出兵,以應入洛之役,不從。秦、鞏之豪汪世顯久求內附。至是彥吶為力請數四,清之亦汔不從。三年,金人大入至三泉,彥吶大敗,眨衡州,其子洸夫用事亦竄嶺南,史嵩之留之江陵兩年,卒。



 趙善湘字清臣,濮安懿王五世孫。父武翼郎不陋,從高
 宗渡江,聞明州多名儒,徙居焉。善湘以恩補保義郎,轉成忠郎、監潭州南嶽廟,轉忠翊郎,又轉忠訓郎。慶元二年舉進士,以近屬轉秉義郎,換承事郎,調金壇縣丞。五年,知餘姚縣。



 開禧元年,添差通判婺州。嘉定元年,以招茶寇功,赴都堂審察,提轄文思院。出判無為軍兼淮南轉運判官、淮西提點刑獄。四年,改知常州。八年,主管武夷山沖祐觀。十年,知湖州。十一年,丁內艱,明年起復,知和州,三辭不獲命。遷知大宗正丞兼權戶部郎官,改知
 秘閣、淮南轉運判官,兼淮西提舉常平,兼知無為軍。進直徽猷閣、主管淮南制置司公事,兼知廬州,兼本路安撫,仍兼轉運判官、提舉常平。



 十三年,進直寶文閣。以平固始寇功,賜金帶,許令服系。十四年,進直龍圖閣、知鎮江府。十七年,拜大理少卿,進右文殿修撰、知鎮江府,封祥符縣男,賜食邑。寶慶二年,進集英殿修撰,拜大理卿兼權刑部侍郎,進寶章閣待制、沿海制置使兼知建康府、江東安撫使兼主管行宮留守司公事。賜仙花金
 帶,進封子,加食邑。



 紹定元年,以創防江軍、寧淮軍及平楚州畔寇劉慶福等功,皆升其官,進龍圖閣待制,仍任,兼江東轉運副使。三年,進煥章閣直學士,仍任,進封伯,加食邑。以李全犯淮東,進煥文閣學士、江淮制置使,乃命專討,許便宜從事。四年,進封侯,加食邑。及戮全,善湘遣使以露布上,乃進兵部尚書,仍兼任。



 時善湘見範、葵進取,慰藉殷勤,饋問接踵,有請必應。遣諸子屯寶應以從,範、葵亦讓功督府,凡得捷,皆汝櫄等握筆草報。善湘
 季子汝楳,丞相史彌遠婿也,故奏報無不達。以平閩寇功,轉江淮安撫制置使。五年,復泰州淮安州、鹽城淮陰縣四城,及策應京湖功,進端明殿學士,與執政恩例,仍任,升留守,加食邑。以受金樞密副使納合買住降,復盱眙軍、泗壽二州功,進資政殿學士,加食邑,遣使賜手詔、金器等物。九疏丐歸,皆不許。請愈力,進大學士、提舉洞霄宮,封天水郡公,加食邑。監察御史劾奏善湘,御筆以善湘有討逆復城之功,寢其奏。



 嘉熙二年,授四川宣撫
 使兼知成都府,未拜,改沿海制置使兼知慶元府。即丐祠,改知紹興府兼浙東安撫使。三年,兩請休致,四乞歸田,復提舉洞霄宮。淳祐二年,帝手詔求所解《春秋》,進觀文殿學士,守本官致仕,卒。遺表聞,帝震悼輟視朝,贈少師,賻贈加等。所著有《周易約說》八卷,《周易或問》四卷,《周易續問》八卷,《周易指要》四卷,《學易補過》六卷,《洪範統論》一卷,《中庸約說》一卷,《大學解》十卷,《論語大意》十卷,《孟子解》十四卷,《老子解》十卷,《春秋三傳通議》三十卷,詩詞雜
 著三十五卷。



 趙與歡,字悅道,燕懿王八世孫。嘉定七年進士,調會稽尉,改建寧司戶參軍。中明法科,攝浦城縣。丁父憂,作《善慶五規》示子孫。免喪,授大理評事。轉對,言天變、民情、國威三事,又言:「死囚以取會駁勘,動涉歲時,類瘐死,而干證者多斃逆旅,宜精擇憲臣使詳覆,果可疑則親往鞫正,必情法輕重可閔,始許審奏。」



 遷籍田令。久之,拜宗正寺簿,歷軍器監、司農寺丞,遷宗正丞兼權都官郎官,
 改倉部,權度支,以直寶章閣知安吉州。郡計仰榷醋,禁綱峻密,與歡首捐以予民。設銅鉦縣門,欲訴者擊之,冤無不直。有富民訴幼子,察之非其本心,姑逮其子付獄,徐廉之,乃二兄強其父析業。與歡曉以法,開以天理,皆忻然感悟。又嫠媼僅一子,亦以不孝告,留之郡聽,日給饌,俾親饋,晨昏以禮,未周月,母子如初。二家皆畫像事之。喪母,朝廷屢起之,不可,議使守邊,授淮西提點刑獄,弗能奪。再期,以刑部郎官召,乞終禫,奉祠,復半載,乃趨
 朝。



 自恢復退師,又議納使,與歡言:「在朝迎合,政出多門,必得智識氣節之士,布列中外可也。」兼權檢正,遷宗正少卿兼權戶部侍郎,尋兼知臨安府、浙西安撫使,同詳定,剖決明暢,罪者咸服。郊祀之夕,大風雷,與歡言國本未定,又陣弭盜固本之策。有以刑罰術數言於帝者,與歡言:「導民有本。如臣待罪天府,豈遽能及民,惟其真實相孚,待以不擾,數月而庭訟彌寡。人心本善,有感必從。或謂厲以威、待以術者,非知本之論。」且言:「朝令夕改,非
 以示作新;旁蹊曲徑,非以肅紀綱。」帝為悚然。又建言:「秦刻頌有『端平法度』語。」



 明年改元嘉熙,襄、蜀殘破,或望風棄地,召見便殿,言:「韓琦當仁宗朝,猶晝夜泣血。今主憂臣辱矣。」因具言防邊之道,其後多見施行。與歡招刺三千人為忠毅軍,又言:「禁衛虛籍及京口諸郡,悉宜募兵,統以郡將,財先贍軍,餘始上供,乞省不急之費。」薦文武士四十人。遷戶部侍郎兼權兵部尚書,論邊事至為深切。



 星變,上章請罷。大火,力言災變之烈,謂:「臣罪擢發莫
 數,猶欲以去國為言,少悟上聽。願祗畏天威,思以實德及民,始自上躬,痛加節約,廣推振恤。」五請竄。於是中書方大琮言:「與歡素自潔修,疏財輕爵,人所共知,不幸遇此,觀其待罪之章,懇切至到,未嘗不嘆其知義也。乞俞所請,使小大之臣,皆知引咎。」乃收一階。尋復之。與歡請先敘復同降官屬,又言:「艱難不可為之時,當慷慨厲志,深為人才兵力思。」遷戶部尚書兼權吏部,累丐祠,不許。



 論楮幣自嘉定以一易二,失信天下,嘗出內帑收換,屢
 稱提而折閱益甚。嘗請兩界並展十年勿議造新,責州縣毋以損污抑沮,至是遂請不立界限以絕其疑,所以區畫者甚備。其後詔宰相遍詢侍從,與歡又以前說陳之。有欲以端平錢當五行使,與歡謂:「開禧嘗以二當三,何救於楮。」且曰:「士大夫不清白奉法,恪意扶持,雖日易一法,無救於楮,而國非其國矣。法削國弱,能獨享富貴乎?」每言「端平以來,竄贓吏,禁包苴,戒奔競,戢橫斂,而風俗沈痼自若。或口仁義而身市井,率以欺君為常,肥家
 為樂,遂臨事乏使,而小人得從旁乘間竊取官爵矣。」疏乞:「別邪正,警偷惰,獎用恬退質直之士,以絕躁競浮靡之習。內廷有關於除授者必斥,暗室有涉於謗議者必思,清心寡欲,以革酣歌黷貨之風,其機皆自陛下始。」又言:「軍政弛而尺籍不明,總兵者或緣功賞開嫌隙,內則班行惟求速化,守牧類多貪庸,楮事日非,浮冗不節,指陳無虛日。」



 大風震雷數見,因具陳邊事,且言:「人才國用,民力兵威,願乘此機,加意根本,勿徒困精神於除授,老
 歲月於行移,委公道於私情,付事功於無可奈何也。」遷吏部尚書。講筵言:「膏雨不降,星變頻仍。在京物價騰踴,民訛士噪;在外兵權渙散,流民充斥。登崇元老,並建宰輔,謂宜風採振揚,而事勢猶若此,士大夫未必任天下之責,天下未必知陛下之志。」力求歸田,會潮汐嚙堤,執政道帝意留治之,手詔云:「忠正廉勤,無如卿者。」授端明殿學士、知臨安府、浙西安撫使。江堤竣事,獄空,力丐罷。依舊端明殿學士,提舉萬壽觀。提領戶部財用兼侍讀
 兼修國史、實錄院修撰。奉朝請,出關,遣使趣還。



 會饑民相攜溺死,帝仍付臨安府事,恩例視執政。與歡涕泣奉詔,亟榜諭曰:「今申奏振救,宜忍死須臾各全性命,佇沐聖恩。」都人相謂毋死。與歡上則祈哀公朝,下則推誠勸分,甘雨隨至,米商來集,流移至者有以濟之。力求納祿,授資政殿學士、提舉萬壽觀兼侍讀、監修國史、實錄院修撰。奉朝請,與歡至浙江,上召還,即日絕江去,帝為悵然。與歡三為府尹,盡力民事,都人稱「趙端明」,必以手加
 額曰「趙佛子」也。



 久之,以舊職知溫州,政事必親,吏不敢欺,創水砦,修貢院。以侍讀召,辭,不許。入對,言爵祿之濫,因及國本事。五丐歸,又不許。進《春秋解》,升大學士,薦士六十人。史嵩之將復入相,而人言不已,帝以問與歡。言:「嵩之老師費財,私暱貪富,過立名譽,必不宜復用。」時嵩之猶子璟卿誦言其過忽斃,而杜範、劉漢弼、徐元傑三賢暴死,人皆疑嵩之致毒。與歡請優恤漢弼、元傑家,帝從之,而優恤手詔,則與歡所擬入也。



 又請以兵財分
 任輔臣。在講筵言:「以壞證付庸醫,僅支殘息,徒運巧心,天下事尚堪再誤耶?」時相忌之。尋授安德軍節度使、開府儀同三司、萬壽觀使。日食,應詔言事益切。月賜內帑,與歡辭不取。帝書「安貧樂道,植節秉忠」字賜之。建儲未定,乃申言之,又言:「人才乏使,贓吏不悛,民昔流而南,今流而北,盜昔伏於遠,今伏於近,體認不真,賢否無別,國將誰與立邪?願富一代之儲,使小人無間可投,以絕隱伏之禍。」帝為改容。



 袁士宋斌少從黃乾、李燔登朱熹之
 門,學禁方嚴,羈旅困沮,年且八十,與歡延之,事以父行,奏乞用旌禮布衣故事,死葬西湖上,歲一祭焉。帝逐二諫臣,與歡力爭之。五乞免朝請,三乞致仕,俱不允,賜《泰卦詩》、《忠邪辨》。自是,國事皆縷縷言之,有不勝書,蓋其愛君憂國,本諸天性。拜少傅,卒,遺表猶不忘規正。帝震悼輟朝,賻贈有加,詔有司治葬,贈少師,追封奉化郡王,謚清敏,累贈太師。



 手注《六經》及《仁皇訓典詳釋》,又有《高宗寶訓要釋》、奏議、詩文百卷。與歡嘗謂:「士大夫有貪聲,則
 雖奇才奧學,徒以蠹國害民爾。」故斂之夕,而金帶猶質錢民家云。



 趙必願,字立夫,廣西經略安撫崇憲之子也。未弱冠,丁大母憂,哀毀骨立。服闋,以大父汝愚遺表,補承務郎。



 開禧元年,銓監平江府糧料院,調常熟丞。嘉定七年舉進士,知崇安縣,剖判如流,吏不能困。修學政,立催科法,列戶名為三等,以三期為約,足者旌之,未足者寬以趣之,逾期不納者里胥程督之,民皆感懌願輸。革胥吏鬻鹽
 之敝。擅發光化社倉活饑民,帥怒,逮吏欲懲之,必願曰:「芻牧職也,吏何罪。」束簷俟譴,帥無以詰而止。舊有均惠倉,無所儲,必願捐緡錢增糴,至二千石。力主義役之法,鄉選善士,任以推排,入資買田助役,則勉有產之家,有感化者,出己田以倡,遂遍行一邑,上下便之。臺府以聞,下其式八郡四十八縣。秩滿,民共立祠刻石。



 授湖、廣總所乾辦公事。丁父憂,居喪盡禮,貽書問學於黃幹。服除,差充兩浙運司主管文字。再考,特差充提領安邊所主
 管文字。差知全州,陛辭,奏乞下道、江二州訪周惇頤之後。知常州,改知處州,陳折帛納銀之害,皆得請。移泉州,罷白土課及免差吏榷鐵,諷諸邑行義役。秋旱,力講行荒政,乞撥永儲、廣儲二倉米振救。差主管官告院。越五日,詔依舊主管官告院兼知臺州,一循大父之政,察民疾苦,撫摩凋瘵,修養濟院,建陳瓘祠,政教兼舉。



 端平元年,以直秘閣知婺州。至郡,免催紹定六年分小戶綾羅錢三萬緡有奇。立淳良、頑慢二籍,勸懲人戶。措置廣惠
 倉及諸倉積穀。奏乞寬減內帑綾羅,申省免用舊例,預解諸色窠名錢,罷開化稅場。遷太府寺丞,尋遷度支郎中。詔以汝愚配享寧宗,從必願請也。兼右司郎中,引見,疏言:



 陛下英明密運,斷出於獨,固欲一切轉移之。然而大權若在我,或者猶有下移之疑;眾正若已開,或者猶有旁徑之疑。策免二相,銷天變也,去者固難以復留,留者恐終於引去。虛鼎席以待故老,疑者或意其未必來,而況在數千里之外;責次補以任大政,疑者或意其不
 敢專,而況於不安其位。中書,政之本也。今果何時,尚可含糊意向以啟天下之疑乎?親擢臺諫,開言路也,用之未久者,何為輕於易去?去之未幾,何為使之復來?召於外服者,不知果能用之而必堅;除目周行者,不知果能聽之而無諱乎?



 朝廷除授,軍國賞罰,本至公也,今有姓名未達於廟堂,而遷擢忽由於中出,斥逐三衙,竟不指名罪狀,而人始得以疑陛下矣。一除目之頒,一號令之出,雖未必由於閹宦,而人或疑於閹宦;雖未必由於私
 謁,而人或疑於私謁;雖未必由於戚畹宗邸,而人或疑於戚畹宗邸。夫天下者,祖宗之天下也,非陛下所私有也,陛下雖有去敝之心,而動涉可疑之跡,陛下亦何樂於此。



 時論偉之。



 三京兵敗,邊事甚亟,詔條上守禦計,必願言十事:下哀痛之詔,合江淮之兵,救江陵之急,節財用之宜,縻議和之使,撫無歸之民,處北來之眾,置鎮撫之使,擇帥閫之代,拔未用之將,皆切於邊要。政府議楮幣日輕,欲令諸州再用印及他為稱提之法,必願力爭
 不可。嘉熙元年,貽書政府,論邊防事宜,授右司郎中。



 火災,必願應詔上封事,曰:「開邊稔禍之刑,牽制而未行;激變棄城之戮,姑息而未舉。京、襄淪沒,祖宗之基業莫能保;淮、蜀蹂躪,赤子之冤魂無所依。履畝之令下而加以抑配,稱提之法嚴而重以告訐。民無蓋藏,每有轉壑之憂;士不宿飽,常有思亂之志。」又曰:「臺諫、給舍骨鯁之論莫容;左右便嬖浸潤之言易入。春夏常享,闊略於原廟之尊;節鉞隆恩,殷勤於邸第之貴。」又曰:「必也正故相專
 國之罪,嚴貪夫徇國之誅,思室鬼高明之瞰。先編氓,後親貴,去木妖競治之釁;尚堅固,革奢華,戒宴殿無度之宴酣,節內庭不急之營繕。」又論濟王及國本事。



 遷左司郎中,又遷司農少卿兼左司。轉對,言:「正氣日消月沮,馴至今日,非惟搢紳不肯論事,下至草茅之士,皆結舌矣。端平初年,沉痾方去,新病未作,陛下猶勤於咨訪,如恐不及。今疾攻心腹,決裂將潰,乃不求瞑眩之劑以起其殆,甚可惑也。」又曰:「毋使人臣以指斥懷疑,毋致陛下以
 厭言得謗。」時直士相繼去,故必願及之。兼敕令所刪修官,拜司農卿,兼職如故。翼日,改宗正少卿,仍兼刪修敕令兼國史編修實錄檢討,尋兼左司,遷太府卿,仍兼編修、檢討,遷宗正少卿。詔依舊太府卿,仍兼職,且兼中書門下檢正諸房公事。轉對,言:「中才庸主,惟其無所知覺,故言不可入,而敗亡隨之。陛下作敬天之圖,朝夕對越,謂宜天意可回,而熒惑失度,鬱攸煽災,迫近禁門,幾毀左藏。煙埃方息,白晝隕星,貫日之虹,脅陽之雹,疊見層
 出。陛下觀時察變,何由致此?今日之事,動無良策,惟在側身修行,祈天永命而已。」遷起居舍人,兼職仍舊。



 大水,上封事曰:「海潮毀隘,侵迫禁城,災異之來,理不虛發,必上畏天戒,下修人事,易沴召和,轉移於陛下方寸間耳。」又曰:「《周官》國有大事,則舉大詢之理。今日之事迫矣,謂宜合眾謀,屈君策,上而搢紳,下而芻蕘,各陳所見,擇其可用之策,以授任事之臣,庶幾千慮一得,以成天下人不因之意。」暫兼權右郎官。言:「財非天雨鬼輸,豈可輕施
 妄用。長此不已,必至顛覆,異時或得罪。今之大夫不能為國生財,程異、皇甫鏄之徒乘間捷出,推敲克剝,以術相勝,鑿空取辦,以計巧取,事掊斂,獻羨餘,間架緡錢之令下,而唐祚愈促矣。願陛下精思熟慮,約已愛民,必如勾踐之臥薪嘗膽,必如衛文公之帛衣布冠,可也。」權吏部右侍郎,乞免兼檢正,從之。兼國史修撰。



 時邊事急,必願應詔言:「宜敕彭大雅自重慶領王青之兵東下以復夔,責李安民及歸、峽二守以自效,調一將督中流之師,
 以伐其順流之謀,調一將自間道出鼎、澧之後,以折其搗虛之鋒,調一將助芮興之勢,以備江陵之急。又宜下湖南遣飛軍及團結民兵之類守沅江、益陽江,以防沖突長沙,盡收江上民船,毋資敵用。」區畫皆中事機。暫兼權侍左侍郎。李宗勉每稱其平允。暫兼權戶部侍郎,兼同詳定敕令。請立國本,請親禱雨。遷戶部侍郎,暫兼給事中。



 先是,錢相嘗繳陳洵益贈節使不行,必願復繳奏曰:「李韶向為殿中侍御史,疏論洵益,乞予外祠,以絕窺
 伺,陛下不行其言,復奪其職,韶不能自安,徑求外補。今召之不至,正以此故。若超贈洵益,又繳駁不行,韶愈無來期矣。陛下忍於去一賢從官,而不忍於沮一已死之內侍,則何以興起治功,振揚國勢?欲望寢洵益節鉞,趣韶供職。」於是必願三以疾乞祠,不許。



 權戶部尚書,疏言:「端平元年,洛師輕出。明年,德安失,襄陽失。又明年,固始失,定遠失,六安失,郢、復、荊門失,蜀道蹂,成都破。又明年,夔、峽徙,浮光降。又明年,滁陽殲。越二年,壽春棄。明年,真
 陽擾,安豐危,成都遺燼,靡有孑遺。」又曰:「去冬安豐危而復安,特天幸爾。君臣動色,太平自賀。雷作於雪宴之先期,蜀警於大宴之朌命,戒心一弛,赫鑒已隨之矣。」又乞「諭太府丞,核戶部收支數目,庶見多寡盈虛之實,有餘則儲之以待朝廷之取撥,闕則助之以示宮府之一體。」二疏迕丞相史嵩之,乞免官、乞祠,皆不許。以司諫鄭起潛論列,以寶謨閣直學士奉祠;辭職名,不許。淳祐五年,以華文閣直學士知福州、福建安撫使,三辭,不許。閩人
 聞必願至,欣然嘆羨。



 必願平易以近民,忠信以厚俗,惻怛以勤政,行鄉飲酒,旌退士,獎高年,裁僧寺實封之數。尤留意武事,甫入境,即以軍禮見戎帥,申明左翼軍節制事宜,措置海道修水,教士卒知勸。」居官四年,累乞歸,及命召,又三辭,皆不許。卒,遺表上,贈銀青光祿大夫。



 必願才周器博,心平量廣,而又蚤聞家庭忠孝之訓、師友正大之言,故所立卓然可稱云。



 論曰:宋之公族,往往亦由科第顯用,各能以術業自見,
 汝談、汝讜、希金官是已。彥吶帥邊而墮功,亦由廟算之短。善湘父子克平大盜。與歡以長者稱。必願世濟其美,可謂信厚之公子矣。



\end{pinyinscope}