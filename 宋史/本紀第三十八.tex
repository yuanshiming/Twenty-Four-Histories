\article{本紀第三十八}

\begin{pinyinscope}

 寧宗二



 嘉泰元年春正月戊午,申嚴福建科鹽之禁。壬戌,謝深甫等薦士三十有五人,詔籍名中書,以待選擢。丁卯,命路鈐按閱諸州兵士,毋受饋遺及擅招軍。違者置諸法。
 庚午,以葛邲配饗光宗廟庭。丙子,金遣完顏充來吊祭。



 二月戊子,詔求明歷之士,壬辰,開資善堂。遣俞烈使金報謝。癸巳,監察御史施康年劾少傅、觀文殿大學士致仕周必大首倡偽學,私植黨與,詔降為少保。修《光宗實錄》。乙未,續修《吏部七司法》。己亥,初置教官試於四川。辛丑,雨土。



 三月丙寅,雨雹。戊辰,復雨雹。頒《慶元寬恤詔令》、《役法撮要》。己巳,雨雹。戊寅,臨安大火,四日乃滅。



 夏四月辛巳,詔有司振恤被災居民,死者給錢瘞之。壬午,下詔
 自責。詔樞密院核禁衛班直及諸軍營柵焚毀之數。癸未,避正殿,減膳。甲申,命臨安府察奸民縱火者,治以軍法。內降錢十六萬緡,米六萬五千餘石,振被災死亡之家。辛卯,龍州蕃部寇邊,遣官軍討之。詔以風俗侈靡,災後官軍營造,務遵法制。內出銷金鋪翠,焚之通衢,禁民無或服用。丁酉,御正殿,復膳。戊戌,以潛邸為開元宮。丙午,詔文武臣無寓居州任厘務官,著為令。



 五月戊午,以旱,禱於天地、宗廟、社稷,詔大理、三衙、臨安府、兩浙州縣
 決系囚。癸亥,釋諸路杖以下囚,除茶鹽賞錢。丁卯,命有司舉行寬恤之政十有六條。乙亥,監太平惠民局夏允中請用文彥博故事,以韓侂冑平章軍國重事。韓侂冑上疏請致仕,不許。免允中官。丙子,雨。丁丑,雨雹。



 六月辛巳,遣陳宗召賀金主生辰。丙午,太白經天。



 秋七月乙卯,何澹罷。丁巳,以旱,復禱於天地、宗廟、社稷。壬戌,釋大理、三衙、臨安府及諸路闕雨州縣杖以下囚。癸亥,雨雹。甲子,以陳自強參知政事兼同知樞密院事,張釜簽書樞
 密院事。丁卯,復振被火貧民。己巳,以吳曦為興州都統制兼知興州。



 八月己卯,減奏薦恩。甲申,張釜罷,以陳自強兼知樞密院事,給事中張巖參知政事,右諫議大夫程松同知樞密院事。丙戌,復詔侍從、臺諫、兩省集議沿江八州行鐵錢利害。



 九月辛亥,遣朝臣二人決浙西圍田。己未,雨土。辛未,遣李景和使金賀正旦。甲戌,令禮官纂集孝宗一朝典禮。



 冬十月甲申,詔免瑞慶節諸道入貢。丙戌,起居郎王容請以韓侂冑定策事跡付史館,從
 之。甲午,金遣徒單懷忠來賀瑞慶節。甲辰,編《光宗御集》。



 十一月庚申,蠲潭州民舊輸黃河鐵纜錢。丙寅,太白晝見。十二月己卯,太白經天。庚寅,復免臨安府民身丁錢三年,辛丑,雨土。癸卯,金遣紇石烈真來賀明年正旦。是歲,浙西、江東、兩淮、利州路旱,振之,仍蠲其賦。真裏富國獻馴象二。



 二年春正月癸亥,以知閣門事蘇師旦兼樞密都承旨。丁卯,陳自強等上《高宗實錄》。



 二月甲申,追復趙汝愚資
 政殿學士。丁亥,修《高宗正史》、《寶訓》。戊子,頒《治縣十二事》以風厲縣令。癸巳,禁行私史。



 三月辛亥,詔宰執各舉可守邊郡者二三人。己未,初命諸路提刑以五月按部理囚。己巳,詔諸路帥臣、總領、監司舉任將帥者與本軍主帥列上之。



 夏四月庚寅,雨雹。



 五月甲辰朔,日有食之。己巳,賜禮部進士傅行簡以下四百九十有七人及第、出身。



 六月丙子,遣趙不艱賀金主生辰。己卯,臨安火。壬午,浚浙西運河。辛卯,禁都民以火說相驚者。庚子,大雨雹。



 秋七月辛亥,封子覿為安定郡王。癸亥,以旱,釋諸路杖以下囚。己巳,命有司舉行寬恤之政七條。庚午,禱於天地、宗廟、社稷。復行寬恤四事。



 八月丙子,以吏部尚書袁說友同知樞密院事。癸未,建寶謨閣,以藏《光宗御集》。己丑,詔作壽慈宮,請太皇太后還內。甲午,謝深甫等上《慶元條法事類》。



 九月己酉,朝壽慈宮。甲寅,修《皇帝會要》。壬戌,奉安光宗皇帝、慈懿皇后神御於景靈宮、萬壽觀。丙寅,嗣秀王伯圭薨,追封崇王,謚曰憲靖。庚午,臨安府野
 蠶成繭。



 冬十月乙亥,上壽成惠慈太皇太后尊號曰壽成惠聖慈祐太皇太后。戊子,金遣完顏瑭來賀瑞慶節。乙未,遣魯使金賀正旦。是月,追復朱熹煥章閣待制致仕。



 十一月甲辰,始御正殿。乙巳,重修《吏部七司法》。庚戌,以陳自強知樞密院事,前同知樞密院事許及之參知政事。丁巳,右文殿楹生芝。十二月甲戌,日中有黑子。率群臣奉上壽成惠聖慈祐太皇太后冊寶於壽慈宮。甲申,立貴妃楊氏為皇后。加韓侂冑太師。庚寅,大閱。



 閏
 月丁未,詔講官有當開釋者,隨事開陳。乙卯,以福州觀察使□嚴為威武軍節度使,封衛國公。丁卯,金遣徒單公弼來賀明年正旦。是月,復周必大少傅、觀文殿大學士。是冬,子坰生,未逾月薨,追封華王,謚沖穆。是歲,建寧府、福、汀、南劍、瀘四州水,邵州旱,振之。



 三年春正月庚辰,謝深甫罷。壬午,置湖南溪洞總首。戊子,龍州蕃部復寇邊,遣官軍討之。甲午,張巖罷。丙申,以陳自強兼參知政事。戊戌,幸太學,謁大成殿,御化原堂,命
 國子祭酒李寅仲講《尚書·周官》篇。遂幸武學,謁武成殿。監學官進秩一級,諸生推恩、賜帛有差。以袁說友參知政事,權翰林學士、知制誥傅伯壽簽書樞密院事,伯壽辭不拜。



 二月乙巳,御文德殿冊皇后。以吏部尚書費士寅簽書樞密院事。



 三月丁丑,以久雨,詔大理、三衙、臨安府決系囚。乙酉,幸聚景園。



 夏四月己亥朔,日有食之。壬寅,福州瑞麥生。丙午,出封樁庫兩淮交子一百萬,命轉運司收民間鐵錢。乙卯,陳自強等上《徽宗玉牒》、孝宗、光
 宗《實錄》。辛酉,詔宰執、臺諫子孫毋就試。



 五月戊寅,以陳自強為右丞相,許及之知樞密院事,仍兼參知政事。庚辰,以旱,詔大理、三衙、臨安府釋杖以下囚。癸未,命有司搜訪舊聞,修三朝正史,以書來上者賞之。是月,以蘇師旦為定江軍承宣使。



 六月壬寅,遣劉甲賀金主生辰。己酉,減大理、三衙、臨安府囚罪一等,釋杖以下。癸亥,太白經天。



 秋七月辛未,頒《慶元條法事類》。命殿前司造戰艦。壬午,權罷同安、漢陽、蘄春三監鑄錢。白虹貫日。癸未,禁
 江、浙州縣抑納逃賦。乙未,加光宗皇帝謚曰循道憲仁明功茂德溫文順武聖哲慈孝皇帝。



 八月壬寅,增置襄陽騎軍。戊申,置四川提舉茶馬二員,分治茶、馬事。丙辰,陳自強等上《皇帝會要》。甲子,詔刑部歲終比較諸路瘐死之數,以為殿最。



 九月庚午,袁說友罷。壬申,以宗子希琪為莊文太子嗣,更名搢,授右千牛衛將軍。癸酉,命坑冶鐵冶司毋得毀私錢改鑄。己丑,詔南郊加祀感生帝,太子、庶子星,宋星。遣張孝曾使金賀正旦。



 冬十月庚子,詔宥
 呂祖泰。癸卯,以費士寅參知政事,華文閣學士、知鎮江府張孝伯同知樞密院事。丙午,命兩淮諸州以仲冬教閱民兵萬弩手。丁未,大風。戊申,龍州蕃部出降。壬子,金遣完顏奕來賀瑞慶節。



 十一月壬申,上光宗冊寶於太廟。癸酉,朝獻於景靈宮。甲戌,朝饗於太廟。乙亥,祀天地於圜丘,大赦。癸未,大風。己丑,安定郡王子覿薨。更定選人薦舉改官法。庚寅,復置福田、居養院,命諸路提舉常平司主之。十二月丙辰,命四川提舉茶馬通治茶馬事。
 辛酉,下詔戒敕將帥掊克。金遣獨吉思忠來賀明年正旦。是冬,金國多難,懼朝廷乘其隙,沿邊聚糧增戍,且禁襄陽榷場。邊釁之開,蓋自此始。



 四年春正月乙亥,大風。浚天長縣濠。癸未,日中有黑子。壬辰,雨雹。瓊州西浮洞逃軍作亂,寇掠文昌縣,遣兵討平之。



 二月丁酉,置莊文太子府小學教授。辛亥,命內外諸軍射鐵帖轉資。壬子,蠲臨安府逋負酒稅。己未,立《試刑法避親格》。庚申,夜有赤氣亙天。



 三月丁卯,臨安大火,
 迫太廟,權奉神主於景靈宮。己巳,避正殿。庚午,命臨安府振焚室。辛未,詔修太廟。甲戌,下詔罪己。乙亥,詔百官疏陳時政闕失。庚寅,復御正殿。



 夏四月甲午朔,立韓世忠廟於鎮江府。命內外諸軍詳度純隊法。甲辰,許及之罷。振恤江西水旱州縣。乙巳,以費士寅兼知樞密院事,張孝伯參知政事,吏部尚書錢象祖賜出身,同知樞密院事。丙辰,詔革選舉之弊。



 五月乙亥,詔諸軍主帥各舉部內將材三人,不如所舉者坐之。癸未,追封岳飛為鄂
 王。



 六月癸巳,遣張嗣古賀金主生辰。丙申,置諸軍帳前雄校,以軍官子孫補之。壬寅,詔侍從、臺諫、兩省集議裁抑濫賞。壬子,詔諸路監司核實諸州樁積錢米。沿江、四川軍帥簡練軍實。丁巳,增廬州強勇軍為千人。



 秋七月甲子,以旱,詔大理、三衙、臨安府、兩浙及諸路決系囚。戊辰,禱於天地、宗廟,社稷。己巳,命諸路提刑從宜斷疑獄。蠲內外諸軍逋負營運息錢。辛未,蠲兩浙闕雨州縣逋租。戊子,命諸路提刑、提舉司措置保伍法。



 八月己亥,陳
 自強等上《皇帝玉牒》。癸丑,詔自今以恩賞進秩,歲毋過二官。蠲紹興府攢宮所在民身丁錢絹綿鹽。丙辰,除靜江府、昭州折布錢。戊午,張孝伯罷。



 九月乙丑,得四圭,有邸玉一,詔藏於太常。壬午,遣鄧友龍使金賀正旦。丙戌,戒飭兩淮州縣遵守寬恤舊法。



 冬十月庚子,以資政殿大學士、淮東安撫使張巖參知政事。壬寅,金遣完顏昌來賀瑞慶節。十一月乙未朔,詔兩淮、荊襄諸州值荒歉奏請不及者,聽先發廩以聞。庚午,封伯栩為安定郡王。
 壬申,白氣亙天。庚辰,修六合縣城。



 十二月癸巳,詔總核內外財賦,以陳自強及兼國用使,費士寅、張巖同知國用事。己亥,詔改明年為開禧元年。壬寅,禁州縣挾私籍沒民產。甲辰,再蠲臨安府民身丁錢三年。乙卯,金遣烏林答毅來賀明年正旦。



 開禧元年春正月癸酉,初置澉浦水軍。壬午,雨霾。



 二月癸巳,奪徐安國三官。癸卯,詔國用司立考核財賦之法。丙午,蠲臨安府逋負酒稅。



 三月庚申,太白晝見。辛未,申
 嚴民間生子棄殺之禁,仍令有司月給錢米收養。辛巳,以淮西安撫司所招軍為強勇軍。癸未,費土寅罷。



 夏四月戊子朔,以錢象祖參知政事兼同知樞密院事,吏部尚書劉德秀簽書樞密院事。辛卯,以江陵副都統李奕為鎮江都統,皇甫斌為江陵副都統兼知襄陽府。戊戌,修《憲聖慈烈皇后聖德事跡》。辛丑,日中有黑子。甲寅,武學生華嶽上書,諫朝廷不宜用兵,恐啟邊釁。以忤韓侂冑,送建寧府編管。乙卯,大風。



 五月己巳,賜禮部進士毛
 自知以下四百三十有三人及第、出身。復淳熙薦舉改官法。乙亥,詔以衛國公□嚴為皇子,進封榮王。甲申,鎮江都統戚拱遣忠義人朱裕結弓手李全焚漣水縣。是月,金國以邊民侵掠及增邊戍來責渝盟。



 六月戊子,罷廣東稅場八十一墟。辛卯,詔內外諸軍密為行軍之計。戊戌,命諸路安撫司教閱禁軍。己亥,遣李壁賀金主生辰。庚子,進程松資政殿大學士,為四川制置使。辛丑,淮東安撫鄭挺坐擅納北人牛真及劫漣水軍事敗,奪二官
 罷。壬寅,天鳴有聲。復同安、漢陽、蘄春三監。己巳,熒惑犯太微右執法。陳自強等上《新修淳熙以後吏部七司法》。壬子,陳自強侍御史鄧友龍等請用本朝故事,以韓侂冑平章軍國事。減大理、三衙、臨安府囚罪一等,釋杖以下。



 秋七月庚申,詔韓侂冑平章軍國事,立班丞相上,三日一朝,赴都堂治事。命興元都統司增招戰兵。丙寅,以蘇師旦為安遠軍節度使,領閣門事。丁卯,詔侍從、兩省、臺諫、在外待制、學士已上及內外文武官,各舉將帥
 邊守一二人。戊辰,贈趙汝愚少保。己卯,韓侂冑等上《高宗御集》。壬午,詔諸路提刑、提舉司措置保甲。癸未,以韓侂冑兼國用使。以旱,詔大理、三衙、臨安府、兩浙州縣及諸路決系囚。



 八月丙戌朔,蠲兩浙闕雨州縣贓賞錢。丁亥,命湖北安撫司增招神勁軍。癸巳,雨。乙巳,以殿前副都指揮使郭倪為鎮江都統兼知揚州。是月,贈宇文虛中少保。追封劉光世為鄜王。



 閏月戊寅,韓侂冑等上《欽宗玉牒》、《憲聖慈烈皇后聖德事跡》。



 九月丁亥,劉德秀罷。
 庚子,詔官吏犯贓追還所受如舊法。丁未,遣陳景俊使金賀正旦。庚戌,大風。



 冬十月甲子,江州守臣陳鑄以歲旱圖獻瑞禾,詔奪一官。丙寅,升嘉定府為嘉慶軍。庚午,金遣紇石烈子仁來賀瑞慶節。復置和州馬監。



 十一月乙酉,置殿前司神武軍五千人屯揚州。乙未,申嚴告訐之禁。十二月癸丑朔,修孝宗、光宗《御集》。庚午,詔兩淮京西監司、帥守講行寬恤之政。增刺馬軍司弩手。癸酉,詔永除兩浙身丁錢絹。戊寅,金遣趙之傑來賀明年正旦,
 入見,禮甚倨。韓侂冑請帝還內,詔使人更以正旦朝見。著作郎朱質上書請斬金使,不報。是歲,真裏富國獻瑞象。江浙、福建、二廣諸州旱,兩淮、京西、湖北諸州水,振之。



 二年春正月癸未朔,蠲兩浙路身丁紬綿。癸巳,再給軍士雪寒錢。發米振給貧民。以金使悖慢,館伴使、副以下奪官有差。乙未,增太學內舍生為百二十人。辛丑,更名國用司曰國用參計所。己酉,雷雨雹。辛亥,詔坑戶毀錢為銅者不赦,仍籍其家,著為令。是月,雅州蠻高吟師寇
 邊,遣官軍討之。



 二月癸丑,壽慈宮火。甲寅,太皇太后移居大內,車駕月四朝。乙卯,以火災,避正殿,徹樂。丁巳,以久雨,詔大理、三衙、臨安府及諸路決系囚。己卯,復御正殿。



 三月癸巳,以程松為四川宣撫使,吳曦為宣撫副使。甲午,頒《開禧重修七司法》。丁酉,詔諸路監司歲十一月按部理囚,如五月之制。己亥,從太皇太后幸聚景園。乙巳,錢象祖罷,以張巖兼知樞密院事。丙午,以錢象祖懷奸避事,奪二官、信州居住,己酉,知處州徐邦憲入見,請
 立太子,因以肆赦弭兵,侍御史徐柟劾罷之。



 夏四月己未,雅州蠻作亂,焚碉門砦,官軍失利。庚申,四川宣撫司復調御前大軍往討之。甲子,以薛叔似為兵部尚書、湖北京西宣撫使,鄧友龍為御史中丞、兩淮宣撫使。下納粟補官之令。戊辰,以吳曦兼陜西、河東路招撫使。己巳,調三衙兵增戍淮東。庚午,追奪秦檜王爵,命禮官改謚。乙亥,以郭倪兼山東、京東路招撫使,鄂州都統趙淳兼京西北路招撫使,皇甫斌兼京西北路招撫副使,丁丑,
 吳曦遣其客姚淮源獻關外四州於金,求封蜀王。鎮江都統制陳孝慶復泗州,江州統制許進復新息縣。戊寅,光州忠義人孫成復褒信縣。



 五月辛巳朔,陳孝慶復虹縣。吳興郡王秉薨,追封沂王,謚曰靖惠。癸未,禁邊郡官吏擅離職守。丙戌,江州都統王大節引兵攻蔡州,不克,軍大潰。丁亥,下詔伐金。癸巳,以伐金告於天地、宗廟、社稷。皇甫斌引兵攻唐州,敗績。興元都統秦世輔出師至城固縣,軍大亂。甲午,賜宗室希瞿子名均,命為沂王秉
 後,補千牛衛將軍。以池州副都統郭倬、主管馬軍行司公事李汝翼會兵攻宿州,敗績。壬寅,太白晝見。簡荊襄、兩淮田卒以備戰兵。癸卯,郭倬等還至蘄縣,金人追而圍之,倬執馬軍司統制田俊邁以與金人,乃得免。



 六月壬子,王大節除名、袁州安置,尋徙封州。癸丑,建康都統李爽攻壽州,敗績。甲寅,鄧友龍罷。以江南東路安撫使丘崇為刑部尚書、兩淮宣撫使。乙卯,雅州蠻高吟師出降,官軍殺之。丁巳,減大理、三衙、臨安府囚罪一等,釋杖
 以下。奪郭倬、李汝翼二官。辛酉,奪皇甫斌三官。甲子,李爽罷。丁卯,曲赦泗州,減雜犯死罪囚,餘皆除之,蠲其租稅三年。建康副都統田琳復壽春府。戊辰,雅州蠻復寇邊。甲戌,奪李爽三官、汀州居住。再奪皇甫斌五官、南安軍安置。丙子,奪鄧友龍三官、興化軍居住,戊寅,蘇師旦罷。是月,命丘崇至揚州部署諸將,悉三衙江上軍分守江、淮要害。金人封吳曦為蜀王。



 秋七月辛巳,復紹興邊郡賞。奪蘇師旦三官、衡州居住,仍籍其家。罷旱傷州軍
 比較租賦一年。詔侍從、臺諫、兩省、卿監、郎官、監司、郡守、前宰執侍從、各舉人材二三人。壬午,雅州蠻出降,庚子,蘇師旦除名、韶州安置。癸卯,以張巖知樞密院事,禮部尚書李壁參知政事。乙巳,置沂王府小學教授。



 八月丙寅,有司上《開禧刑名斷例》。斬郭倬丁鎮江。戊辰,再奪李爽三官、南雄州安置。辛未,詔諸州無證有佐之獄毋奏裁。壬申,以淮東安撫司所招軍為御前強勇軍。



 九月壬午,金兵攻奪和尚原。己丑,朝獻於景靈宮。庚寅,朝饗於
 太廟。辛卯,合祭天地於明堂,大赦。乙巳,賞復泗州功。



 冬十月戊申朔,詔內外軍帥各舉智勇可將帥者二人。辛酉,以將士暴露,罷瑞慶節宴。丙子,金人自清河口渡淮,遂圍楚州。



 十一月庚辰,命主管殿前司公事郭杲領兵駐真州以援兩淮。辛巳,金人破棗陽軍。甲申,以丘崇簽書樞密院事,督視江、淮軍馬。金人犯神馬坡,江陵副都統魏友諒突圍趨襄陽。乙酉,趙淳焚樊城。戊子,金人犯廬州,田琳拒退之。癸巳,以金人犯淮告於天地、宗廟、社
 稷。乙未,避正殿,減膳。以湖廣總領陳謙為湖北、京西宣撫副使。丙申,金人去廬州。丁酉,金人犯舊岷州,守將王喜遁去。戊戌,金人圍和州,守將周虎拒之。金人破信陽軍。辛丑,金人圍襄陽。壬寅,金人破隨州。癸丑,太皇太后賜錢一百萬緡犒賞軍士。詔諸路招填禁軍以待調遣。甲辰,金人犯真州。乙巳,金人破西和州。是月,濠州、安豐軍及邊屯皆為金人所破。十二月戊申,金人圍德安府,守將李師尹拒之。庚戌,金人破成州,守臣辛□□之遁去。
 吳曦焚河池縣,退屯青野原。辛亥,釋大理、三衙、臨安府杖以下囚。癸丑,金人去和州。甲寅,金人攻六合縣,郭倪遣前軍統制郭僎救之,遇於胥浦橋,大敗,倪棄揚州走。丁巳,金人破大散關。戊午,熒惑守太微。癸亥,魏友諒軍潰於花泉,走江陵。丁卯,金人犯七方關,興州中軍正將李好義拒卻之。戊辰,吳曦還興州。金人自淮南退師,留一軍據濠州。己巳,罷郭倪,奪三官,責授果州團練副使、南康軍安置。庚午,薛叔似、陳謙罷。以荊湖北路安撫使
 吳獵為湖北、京西宣撫使。復兩浙圍田,募兩淮流民耕種。癸酉,吳曦始自稱蜀王。甲戌,以鎮江副都統畢再遇為鎮江都統、權山東京東路招撫司公事。乙亥,四川宣撫使程松遁。



 三年春正月丁丑朔,丘崇罷。己卯,命知樞密院事張巖督視江、淮軍馬。庚辰,以陳自強兼樞密使。癸未,金人破階州。丁亥,子圻生。庚寅,詔建康府給淮民裝錢,遣歸業。辛卯,吳曦招通判興元府、權大安軍事楊震仲,震仲不
 屈,死之。癸巳,命兩淮帥守、監司招集流民。甲午,吳曦僭位於興州。甲辰,奪池州都統陳孝慶三官罷。



 二月壬子,以金師退,御正殿,復膳。甲寅,削奪福建路總管兼延祥水軍統制商榮官爵、柳州安置。己未,罷程松四川宣撫使,以成都府路安撫使楊輔為四川制置使,沿江制置使葉適兼江、淮制置使。庚申,以旱,詔大理、三衙、臨安府決系囚。癸亥,子圻薨,追封順王,謚沖懷。甲子,振給旱傷州縣貧民。命諸路提刑司從宜斷疑獄。丁卯,罷江、浙、荊
 湖、福建招軍。戊辰,子址生。庚午,金人去襄陽。辛未,以旱,禱於天地、宗廟、社稷。命有司舉行寬恤之政八條,蠲兩淮被兵諸州今年租賦。乙亥,釋兩浙路杖以下囚。四川宣撫副使司隨軍轉運安丙及興州中軍正將李好義、監四川總領所興州合江倉楊巨源等共誅吳曦,傳首詣行在,獻於廟社,梟三日,四川平。並誅曦妻子,家屬徙嶺南,奪其父挺官,遷吳璘子孫出蜀,存其廟祀,玠子孫免連坐。



 三月丙子朔,蠲兩淮被兵州郡役錢。丁丑,斬偽
 四川都轉運使徐景望於利州。壬辰,興州將劉昌國引兵至階州,金人退去。癸巳,李好義復西和州。丁酉,金人去成州。庚子,詔以楊輔為四川宣撫使,安丙為端明殿學士、四川宣撫副使,起居舍人許奕為四川宣諭使。落程松資政殿大學士,奪六官、筠州安置。忠義統領張翼復鳳州。辛丑,曲赦四川,減雜犯死罪囚,釋杖以下。壬寅,責授程松順昌軍節度副使、澧州安置。



 夏四月戊申,以吳獵兼四川宣諭使。子址薨,追封申王,謚沖懿。癸丑,赦
 兩淮、湖北、京西被兵諸州,減雜犯死罪囚,釋流以下。蠲湖北、京西諸郡今年租賦。四川忠義人復大散關。己未,奉使金國通謝、國信所參議官方信孺發行在。庚申,以兵部尚書宇文紹節知江陵府,權湖北、京西宣撫使。壬戌,詔吳獵與宣撫司議,分興州都統司軍之半屯利州。丁卯,召楊輔詣行在,以吳獵為四川制置使。戊辰,以資政殿學士錢象祖參知政事。己巳,改興州為沔州。庚午,贈楊震仲官,仍官其子一人。癸酉,金人復破大散關。甲
 戌,赦西和、階、成、鳳四州。



 五月丁丑,賞誅吳曦功。戊寅,用四川宣撫司奏,吳曦黨人張伸之等一十六人除名,編配兩廣及湖南諸州。己丑,以旱,禱於天地、宗廟、社稷。辛卯,以太皇太后謝氏有疾,赦,是日崩。四川宣撫副使司參贊軍事楊巨源與金人戰於長橋,敗績。戊戌,詔四川宣撫、制置司分治兵民。庚子,復置沔州副都統制,以李好義為之。辛丑,李好義襲秦州,敗還。



 六月甲寅,賞守襄陽功。己未,李好義遇毒死。癸亥,以林拱辰為金國通謝
 使,遣富管使金告哀,劉彌正賀金主生辰。癸酉,安丙殺其參議官楊巨源。



 秋七月己卯,命不儔為嗣濮王。乙酉,以災傷,下詔罪己。



 八月己巳,上大行太皇太后謚曰成肅皇后。



 九月丁丑,詔諸路帥臣申儆邊備。辛巳,召張巖詣行在。壬午,方信孺以忤韓侂冑,坐用私覿物擅作大臣饋遺金將,奪三官、臨江軍居住。甲申,減極邊官吏舉主員。乙酉,權攢成肅皇后於永阜陵。丙戌,命淮西轉運司措置雄淮軍。辛卯,以趙淳為殿前副都指揮使兼江、淮
 制置使。乙未,張巖罷。辛丑,遣王□冉持書赴金國都副元帥府。壬寅,祔成肅皇后神主於太廟。



 冬十月乙巳,減臨安、紹興二府囚罪一等,蠲民緣攢宮役者賦。丙午,更殿前司純隊法。乙卯,復珍州遵義軍。丙辰,詔以邊事諭軍民。



 十一月甲戌,詔:韓侂冑輕啟兵端,罷平章軍國事;陳自強阿附充位,罷右丞相。乙亥,禮部侍郎史彌遠等以密旨命權主管殿前司公事夏震誅韓侂冑於玉津園。以錢象祖兼知樞密院事,李壁兼同知樞密院事。以誅
 韓侂冑詔天下。丁丑,以夏震為福州觀察使、主管殿前司公事,將士行賞有差。奪陳自強三官、永州居住。戊寅,責授蘇師旦武泰軍節度副使、韶州安置;己卯,斬之。詔:「奸臣竄殛,當首開言路,以來忠讜。中外臣僚,各具所見以聞。」辛巳,再奪鄧友龍五官、南雄州安置,尋除名,徙循州。乙酉,置御前忠銳軍。丙戌,以御史中丞衛涇簽書樞密院事兼參知政事。丁亥,詔立皇子榮王□嚴為皇太子,更名懤。戊子,郭倪除名、梅州安置,郭僎除名,連州安置:
 仍籍其家。奪李壁三官、撫州居住。癸巳,奪張巖二官、徽州居住。己亥,以立皇太子,大赦。十二月癸卯,以丘崇為江、淮制置大使。罷山東、京東招撫司。以許奕為金國通問使。乙巳,太白晝見。丁未,罷京西北路招撫司。己酉,落葉適寶文閣待制。蠲兩淮州軍稅一年。庚戌,奪許及之二官、泉州居住。奪薛叔似二官、福州居住。再奪皇甫斌五官、英德府安置。癸丑,金人復破隨州。辛酉,以錢象祖為右丞相兼樞密使,衛涇及給事中雷孝友並參知政
 事,吏部尚書林大中簽書樞密院事。乙丑,以禮部尚書史彌遠同知樞密院事。丙寅,贈呂祖儉朝奉郎、直秘閣,官其子一人。丁卯,詔改明年為嘉定元年。是歲,浙西旱蝗,沿江諸州水。



\end{pinyinscope}