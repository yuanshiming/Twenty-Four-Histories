\article{志第四十二 地理五}

\begin{pinyinscope}

 福建路成都府路潼川府路利州路夔州路



 福建路。州六:福,建,泉,南劍,漳,汀。軍二:邵武,興化。縣四十七。南渡後,升建州為府。紹興三十二年,戶一百三十九萬五百六十五,口二百八十二萬八千八百五十二。



 福州,大都督府,長樂郡,威武軍節度。舊領福建路鈐轄,建炎三年升帥府。崇寧戶二十一萬一千五百五十二。貢荔枝、鹿角菜、紫菜。元豐貢紅花蕉布。縣十二:閩,望。



 侯官,望。



 福清,望。



 古田,望。唐縣。有寶興銀場、古田金坑。



 永福,望,有黃洋、保德二銀場。



 長溪,望。有玉林銀場及鹽場。長樂,緊。有海壇山鹽場。



 羅源,中。舊永貞縣。閩清,中。



 寧德,中。王審知時置。



 懷安,望。太平興國五年,析閩縣置。



 連江。望。



 建寧府,上,本建州,建安郡。舊軍事,端拱元年,升為建寧軍節度;紹興三十二年,以孝宗舊邸,升府。崇寧戶一十
 九萬六千五百六十六。貢火箭、石乳、龍茶。元豐貢龍鳳等茶、練。縣七:建安,望。漢縣。有北苑茶焙、龍焙監庫及石舍、永興、丁地三銀場。



 浦城,望。有餘生、蕉溪、斤竹三銀場。



 嘉禾,望。本建陽縣。有瞿嶺四銀場。景定元年改今名。



 松溪,緊。



 崇安,望。淳化五年,升崇安場為縣。



 政和,緊。咸平三年,升關隸鎮為縣。政和五年,改關隸為政和縣。有天受銀場。



 甌寧。望。熙寧三年廢,元祐四年復。



 監一:豐國。咸平二年置,鑄銅錢。



 泉州,望,清源郡。太平興國初,改平海軍節度。本上郡,大觀元年,升為望郡。崇寧戶二十萬一千四百六。貢松子。元豐貢綿、蕉、葛。縣七:晉江,望。有鹽亭一百六十一。



 南安,中。



 同安,
 中。有安仁、上下馬欄、莊板四鹽場。



 惠安,望。太平興國六年,析晉江置縣。有鹽亭一百二十九。



 永春,中。閩桃源縣,有倚洋一鐵場。



 安溪,下。有青陽鐵場。



 德化。下。有赤水鐵場。



 南劍州,上,劍浦郡,軍事。太平興國四年,加「南」字。崇寧戶一十一萬九千五百六十一。貢土茴香。元豐貢茶。縣五:劍浦,緊。舊龍津縣,南唐改。有大演、石城二銀場,雷、大熟等五茶焙。



 將樂,上。太平興國四年,自建州來隸。有石牌、安福二銀場。



 順昌,上。南唐升永順場為縣。沙,中。有龍泉銀場。



 尤溪。上。有尤溪,寶應等九銀場。



 漳州,下,漳浦郡,軍事。崇寧戶一十萬四百六十九。貢甲
 香、鮫魚皮。縣四:龍溪,望。有吳慣、沐犢、中柵三鹽場。



 漳浦,望。有黃敦鹽場。



 龍巖,望。有大濟、寶興二銀場。



 長泰。望。太平興國五年,自泉州來隸。



 汀州,下,臨汀郡,軍事。淳化五年,以上杭、武平二場並為縣,元符元年,析長汀、寧化置清流縣。崇寧戶八萬一千四百五十四。貢蠟燭。縣五:長汀望。有上寶錫場,歸禾、拔口二銀務,莒溪鐵務。



 寧化,望。有龍門新舊二銀坑。上杭,上。有鐘寮金場。天聖二年,徙治鐘寮場東,乾道四年徙治郭下。



 武平,上。



 清流。南渡後,增縣一:蓮城。本長汀蓮城堡,紹興三年升縣。



 邵武軍,同下州。太平興國五年,以建州邵武縣建為軍,
 仍以歸化、建寧二縣來屬。崇寧戶八萬七千五百九十四。貢絲寧。縣四:邵武,望。有黃土等三鹽場,龍須銅場,寶積等三鐵場。



 光澤,望。太平興國六年,析邵武置縣。有太平銀場、新安鐵場。



 泰寧,望。南唐歸化縣。元祐元年,改為泰寧。有□累砌金場、江源銀場。



 建寧。望。有龍門等三銀場。



 興化軍,同下州。太平興國四年,以泉州游洋、百丈二鎮地置太平軍,尋改。戶六萬三千一百五十七,貢綿、葛布。縣三:莆田,望。自泉州與仙游同來隸。仙游,望。



 興化。中。太平興國四年,析莆田置縣。



 福建路,蓋古閩越之地。其地東南際海,西北多峻嶺抵
 江。王氏竊據垂五十年,三分其地。宋初,盡復之。有銀、銅、葛越之產,茶、鹽、海物之饒。民安土樂業,川源浸灌,田疇膏沃,無兇年之憂。而土地迫狹,生籍繁伙;雖磽確之地,耕耨殆盡,畝直浸貴,故多田訟。其俗信鬼尚祀,重浮屠之教,與江南、二浙略同。然多向學,喜講誦,好為文辭,登科第者尤多。



 成都府路。府一:成都。州十二:眉,蜀,彭,綿,漢,嘉,邛,簡,黎,雅,茂,威。軍二:永康,石泉。監一:仙井。縣五十八。南渡後,府三:
 成都,崇慶,嘉定。州十一:眉,彭,綿,漢,邛,黎,雅,茂,簡,威,隆。軍二:永康,石泉。淳熙二年,戶二百五十八萬,口七百四十二萬。



 成都府,次府,本益州,蜀郡,劍南西川節度。太平興國六年,降為州。端拱元年,復為劍南西川成都府。淳化五年,降為益州,罷節度。嘉祐五年,復為府。六年,復節度。舊領成都府路兵馬鈐轄。建炎三年,罷兼利州路。紹興元年,領成都路安撫使。五年,兼四路安撫、制置大使。十年置宣撫,罷
 制置司,知府帶本路安撫使。十八年,罷宣撫,復制置司;乾道六年,又罷,並歸安撫司,知府仍帶本路安撫使。淳熙二年,復制置司,罷宣撫司。開禧元年,置宣撫,罷制置司。未幾,兩司並置;後罷宣撫,仍置制置大使。嘉定七年,去「大」字。崇寧戶一十八萬二千九十,口五十八萬九千九百三十。貢花羅、錦、高絲寧布、箋紙。縣九:成都,次赤。



 華陽,次赤。新都,次畿。



 郫,次畿。熙寧五年,省犀浦為鎮入焉。



 雙流,次畿。溫江,次畿。



 新繁,次畿。漢繁縣,前蜀改。



 廣都,次畿。熙寧五年廢陵州,以貴平、籍二縣為鎮入焉。



 靈泉。次畿。舊名靈池,天聖四年
 改。



 眉州,上,通義郡,至道二年,升為防禦。崇寧戶七萬二千八百九,口一十九萬二千三百八十四。貢麩金、巴豆。縣四:眉山,望。隋通義縣。太平興國初改。



 彭山,望。



 丹棱,望。



 青神。緊。



 崇慶府,緊,本蜀州,唐安郡,軍事。紹興十四年,以高宗潛藩,升崇慶軍節度。淳熙四年,升府。崇寧戶六萬七千八百三十五,口二十七萬三千五十。貢春羅、單絲羅。縣四:晉源,望。



 新津,望



 江原,望。唐唐安縣。開寶四年改。



 永康。望。蜀析青
 城地置縣。



 彭州,緊,蒙陽郡,軍事。崇寧戶五萬七千五百二十四。貢羅。縣三:九隴,望。唐縣。熙寧二年置堋口縣,四年,省為鎮入焉。有鹿角砦,堋口、木頭二茶場。



 崇寧,望。唐昌縣。崇寧元年改。



 蒙陽。望。



 綿州,上,巴西郡,軍事。紹興三年,以知州事兼綿、威、茂州、石泉軍沿邊安撫使,節制屯戍軍馬。五年,川、峽宣撫副使移司綿州,六年罷。二十一年,罷沿邊安撫使。嘉熙元年,為四川制置副使治所。崇寧戶一十二萬二千九百一十五,口二十三萬四百九。貢綾、絲寧布。縣五:巴西,望。



 彰
 明,望



 魏城,緊。



 羅江,緊。



 鹽泉。中。



 漢州,上,德陽郡,軍事。戶一十二萬九百,口五十二萬七千二百五十二。貢紵布。縣四:雒,望。



 什邡,望



 綿竹,望。



 德陽。望。



 嘉定府,上,本嘉州,犍為郡,軍事。乾德四年,廢綏山、羅目、玉津三縣。慶元二年,以寧宗潛邸,升府。開禧元年,升嘉慶軍節度。崇寧戶七萬一千六百五十二,口二十一萬四百七十二。貢麩金。縣五:龍游,上。宣和元年,改曰嘉祥,後復故。熙寧五年,省平
 羌縣入焉。



 洪雅,上。淳化四年,自眉州來隸。



 夾江,中



 峨眉,中。



 犍為。下。大中祥符四年,移治懲非鎮。



 監一:豐遠。鑄鐵錢。



 邛州,上,臨邛郡,軍事。崇寧戶七萬九千二百七十九,口一十九萬三千三十二。貢絲布。縣六:臨邛,望。熙寧五年,省臨溪縣為鎮入焉,並入依政、蒲江、火井。



 依政,望。



 安仁,望。有延貢砦。



 大邑,望。有思安砦。



 蒲江,上。有鹽井監、鹽井砦。



 火井。中。開寶三年,移治平樂鎮,至道三年復舊。



 監一:惠民。鑄鐵錢。建炎三年罷。



 簡州,下,陽安郡,軍事。崇寧戶四萬一千八百八十八,口
 九萬五千六百一十九。貢綿紬、麩金。縣二:陽安,上。



 平泉。中。



 黎州,上,漢源郡,軍事。崇寧戶二千七百二十二,口九千八十。貢紅椒。縣一:漢源。下。慶歷六年,廢通望縣入焉。舊廢飛越縣有博易務。



 領羈縻州五十四。羅巖州、索古州、秦上州、合欽州、劇川州、輒榮州、蓬口州、柏坡州、博盧州、明川州、胣□皮州、蓬矢州、大渡州、米川州、木屬州、河東州、諾筰州、甫嵐州、昌化州、歸化州、粟川州、叢夏州、和良州、和都州、附木州、東川州、上貴州、滑川州、北川州、吉川州、甫萼州、北地州、蒼榮州、野川州、邛陳州、貴林州、護川州、牒琮州、浪彌州、郎郭州、上欽州、時蓬州、儼馬州、橛查州、邛川州、護邛州、腳川州、開望州、上蓬州、北蓬州、剝重州、久護州、瑤劍
 州、明昌州。



 雅州,上,盧山郡,軍事。崇寧戶二萬七千四百六十四,口六萬二千三百七十八。貢麩金。縣五:嚴道,中。有碉門砦。



 盧山,上。有靈關砦。



 名山,中。熙寧五年,省百丈縣為鎮入焉,元祐二年復。



 榮經,中下。



 百丈。中下。州城內一茶場。熙寧九年置。



 領羈縻州四十四。當馬州、三井州、來鋒州、名配州、鉗泰州、隸恭州、畫重州、羅林州、籠羊州、林波州、林燒州、龍蓬州、敢川州、驚川州、禍眉州、木燭州、百坡州、當品州、嚴城州、中川州、鉗矣州、昌磊州、鉗並州、百頗州、會野州、富仁州、推梅州、作重州、禍林州、金林州、諾祚州、三恭州、布嵐州、欠馬州、羅蓬州、論川州、讓川州、遠南州、卑盧州、夔龍州、、輝川州、金川州、東嘉梁州、西嘉梁州。



 茂州,上,通化郡,軍事。熙寧九年,即汶川縣置威戎軍使,以石泉縣隸綿州。崇寧戶五百六十八,口一千三百七十七。貢麝香。縣一:汶山。下砦一:鎮羌。熙寧九年置。關一:雞宗。熙寧九年置。南渡後,增縣一:汶川。下。有博馬場。領羈縻州十。璫州、直州、時州、塗州、遠州、飛州、乾州、可州、向州、居州。春祺城,本羈縻保州,政和四年,建為祺州,縣曰春祺,宣和三年,廢為城,隸茂州。壽寧砦,本羈縻直州,政和六年,建壽寧軍,在大皂江外,距茂州五里,八年,廢為砦,宣和三年,廢砦為堡,又廢敷文關為敷
 文堡。延寧砦,本威戎軍,熙寧間所建,政和六年,湯延俊等納土,重築軍城,改名延寧,宣和三年,廢為砦,隸茂州,四年,又廢砦及壽寧堡入汶川縣。



 威州,下,維川郡,軍事。本維州。景祐三年,以與濰州聲相亂,改今名。崇寧戶二千二十,口三千一十三。貢當歸、羌活。縣二:保寧,下。唐薛城縣,南唐改。通化。下。天聖元年,改金川。景祐四年復。治平三年,省通化軍隸縣。有博易場。領羈縻州二。保州、霸州。嘉會砦,本羈縻霸州,政和四年,建為亨州,縣曰嘉會,宣和三年,廢州,以縣為砦,隸
 威州。通化軍,熙寧間所建,在保、霸二州之間。政和三年,董舜咨納土,因舊名重築軍城,宣和三年,省軍使為監押,隸威州。



 永康軍,同下州。本彭州導江縣灌口鎮。唐置鎮靜軍。乾德四年,改為永安軍,以蜀州之青城及導江縣來隸。太平興國三年,改為永康軍。熙寧五年,廢為砦;九年,復即導江縣治置永康軍使,隸彭州。元祐初,復故。縣二:導江,望。乾德中,自彭州來隸。熙寧五年,軍廢,復隸彭州,後復於此置軍。有博馬場。青城。望。乾德中,自蜀州來隸。熙
 寧五年軍廢,還隸蜀州,不知何年復來隸。



 仙井監,同下州。本陵州。至道三年,升為團練。咸平四年廢,始建縣。熙寧五年,廢為陵井監。宣和四年,改為仙井監。隆興元年,改為隆州。崇寧戶三萬二千八百五十三,口一十萬四千五百四十五。貢苦藥子、續隨子。縣二:仁壽,中井研。中下。南渡後,增縣二:貴平,中下。熙寧五年,廢入廣都。乾道六年復。籍。中下。廢復同上。鎮一:大安。舊永安鎮。崇寧二年改。鹽井一。



 石泉軍,本綿州石泉縣。政和七年,建為軍,割蜀之永康、
 綿之龍安、神泉來隸。宣和三年,降為軍使,縣皆還舊隸。宣和七年,復為軍額。縣三:石泉,下。



 神泉,上。有石關砦。



 龍安。上。有三盤砦及茶場。宣和元年,改龍安曰安昌,後復故。寶祐後,為軍治所。



 堡九。重和元年置。會同、靖安、嘉平、通津、橫望、平隴、凌霄、聳翠、連雲。



 潼川府路。府二:潼川,遂寧。州九:果,資,普,昌,敘,瀘,合,榮,渠。軍三:長寧,懷安,廣安。監一:富順。紹興三十二年,戶八十萬五千三百六十四,口二百六十三萬六千四百七十六。



 潼川府,緊,梓潼郡,劍南東川節度。本梓州。乾德四年,改靜戎軍,置東關縣。太平興國中,改安靜軍。端拱二年,為東川;元豐三年,復加「劍南」二字。重和元年,升為府。舊兼提舉梓州果、渠、懷安、廣安軍兵馬巡檢盜賊公事。乾道六年,升瀘南為潼川府路安撫使。崇寧戶一十萬九千六百九,口四十四萬七千五百六十五。貢綾、曾青、空青。縣十:郪望。有三十四鹽井。



 中江,望。隋玄武縣。大中祥府五年改。有鹽井。



 涪城,望。有四鎮、二十七鹽井。



 射洪,緊。有鹽井。



 鹽亭,緊。熙寧五年,省永泰縣為鎮入焉。有六鹽井。



 通泉,上。有三鐵
 治。



 飛烏,中。有五鹽井。



 銅山,中。有銅冶。



 東關,中下。有四鹽井。



 永泰。中下。本尉司,南渡後為縣。



 遂寧府,都督府,遂寧郡,武信軍節度。本遂州。政和五年,升為府。宣和五年,升大藩。端平三年,兵亂,權治蓬溪砦。崇寧戶四萬九千一百三十二,口一十萬二千五百五十五。貢樗蒲綾。縣五:小溪,望。隋方義縣。太平興國初改。



 蓬溪,望



 長江,緊。端平三年,以下三縣俱廢。



 青石,緊。



 遂寧。中。唐縣。熙寧六年,省青石縣入焉。七年,復置。



 順慶府,中,本果州,南充郡,團練。寶慶三年,以理宗初潛
 之地,升府,隸劍南東路。端平三年,兵亂。淳祐九年,徙治青居山。崇寧戶五萬五千四百九十三,口一十三萬三百一十三。貢絲布、天門冬。縣三:南充,望。熙寧六年,省流溪縣為鎮入焉;紹興二十七年,復為縣。



 西充,望。



 流溪。望。



 資州,上,資陽郡,軍事。乾德五年,廢月山、丹山、銀山、清溪四縣。宣和二年,改龍水為資川,後復故,淳祐三年廢。崇寧戶三萬二千二百八十七,口四萬七千二百一十九。貢麩金。縣四:盤石,緊。有一十八鹽井、一鐵治。



 資陽,緊。



 龍水,中下。



 內江。下。有
 六十六鹽井。



 普州,上,安岳郡,軍事。乾德五年,廢崇龕、普慈二縣。端平三年,兵亂。淳祐三年,據險置治。寶祐以後廢。崇寧戶三萬二千一百一十八,口七萬三千二百二十一。貢葛、天門冬。縣三:安岳,中下。熙寧五年,廢普康縣入焉。



 安居,中。



 樂至。下。



 昌州,上,昌元郡,軍事。崇寧戶三萬六千四百五十六,口九萬二千五十五。貢麩金、絹。縣三:大足,上。



 昌元,上。咸平四年,移治羅市。



 永川。上。



 敘州,上,南溪郡,軍事。乾德中,廢開邊、歸順二縣。本戎州,政和四年改。咸淳三年,徙治登高山。崇寧戶一萬六千四百四十八,口三萬六千六百六十八。貢葛。縣四:宜賓,中。唐義賓縣。太平興國元年改。熙寧四年,省宜賓入僰道為鎮。政和四年,改僰道為宜賓。



 南溪,中。乾德中,移治舊奮城。有鹽井。



 宣化。唐義賓縣。太平興國元年改。熙寧四年,改為鎮,隸僰道。宣和元年,復以鎮為縣,改今名。



 慶符。本敘州徼外地。政和三年,建為祥州,置慶符、來附二縣。宣和三年,州廢,並來附縣入慶符縣,隸敘州。砦五:柔遠、樂從、清平、石門、懷遠。靖康元年,廢柔遠、樂從二砦隸懷遠。



 羈縻州三十。建州、照州、獻州、南州、洛州、盈州、德州、為州、移州、扶德州、播浪州、筠州、武昌州、志州,已上皆在南廣溪洞;商州、馴州、浪
 川州、騁州,已上皆在馬湖江;協州、切騎州、靖州、曲江州、哥陵州、品州、牁違州、碾衛州、滈州、從州、播陵州、鉗州,已上皆在石門路。



 瀘州,上,瀘川郡,瀘川軍節度。本軍事州。宣和元年,賜軍額。乾道六年,升本路安撫使。嘉熙三年,築合江之榕山,再築江安之三江磧,四年,又築合江之安樂山為城。淳祐三年,又城神臂崖以守。景定二年,劉整以城歸大元,後復取之,改江安州。崇寧戶四萬四千六百一十一,口九萬五千四百一十。貢葛。縣三:乾德五年,廢綿水,富義置上監州。治平四年,廢
 羊羝砦。元豐二年,廢白芀砦。三年,廢平夷堡,於羅池改築安遠砦;廢大硐、武寧二砦。五年,復置武寧砦,隸長寧軍。



 瀘川,中。



 江安,中。有寧遠、安夷、西寧遠、南田、武寧、安遠等砦。



 合江。中。有遙埧、青山、安溪、小溪、帶頭、使君六砦。大觀三年,以安溪砦為縣,隸純州;後廢純州,復為砦。宣和三年,廢遙壩;四年,復。



 南渡後,增縣一:納溪。皇祐三年,納溪口置砦。紹定五年,升為縣。



 監一:南井。城三:樂共城,元豐四年置。



 堡砦四:江門砦、鎮溪堡、梅嶺堡、大洲堡。九支城,大觀二年,建純州,置九支、安溪兩縣及美利城。宣和三年,廢純州及九支縣為九支城,以安溪、美利城為砦,改慈竹砦為堡。



 武都城。大觀三年,建滋州,置承流、仁懷兩縣。宣和三年,廢州為武都城,以仁懷為堡,承流縣並入仁懷。



 安遠砦,元豐三年置。大觀四年廢。政和五年復。



 博望砦,政和七年置。板
 橋堡,政和堡,政和六年置。



 綏遠砦。前隸武都城,宣和三年隸州。



 領羈縻州十八。納州、薛州、晏州、鞏州、奉州、悅州、思峨州、長寧州、能州、淯州、浙州、定州、宋州、順州、藍州、溱州、高州、姚州。



 長寧軍,本羈縻州。熙寧八年,夷人得個祥獻長寧、晏、奉、高、薛、鞏、淯、思峨等十州,因置淯井監隸瀘州。政和四年,建為長寧軍。領砦堡六:梅洞砦,政和五年置。



 清平砦,舊隸祥州,政和二年建築,賜今名。宣和三年廢祥州,以砦隸軍。



 武寧砦,熙寧七年置,舊名小溪口。十年,改今名。元豐四年廢。五年復置。政和四年,建長寧軍,以武寧為倚郭縣。宣和二年,廢縣為堡。四年,復為砦。



 寧遠砦,皇祐元年,置三江砦。三年,改今名。宣和二年,以砦為堡。四年,復為砦。



 安夷砦,熙寧六年置,舊名婆娑。大觀四
 年廢。政和六年復置。



 石筍堡。政和五年置。初名梅賴,後賜今名。



 南渡後,縣一:安寧。嘉定四年,升安夷砦為縣。有武寧、寧遠二砦。



 合州,中,巴川郡,軍事。淳祐三年,移州治於釣魚山。崇寧戶四萬八千二百七十七,口八萬四千四百八十四。貢牡丹皮、白藥子。縣五:石照,中。魏石鑒縣。乾德三年改。



 漢初,中。



 巴川,中。



 赤水,中下。



 銅梁。中下。熙寧四年,省赤水入焉;七年,復置。



 榮州,下,和義郡,軍事。乾德五年,廢和義縣。端平三年,擇地僑治。寶祐後廢。崇寧戶一萬六千六百六十七,口五
 萬二千八十七。貢斑布。縣四:榮德,中下。舊名旭川。治平四年改。熙寧四年,省公井縣為鎮入焉。有鹽監一,端平三年廢。



 威遠,中。資官,中。有鹽井。



 應靈。中下。有鹽井。



 渠州,下,鄰山郡,軍事。寶祐三年,徙治禮義山。崇寧戶三萬二千八百七十七,口六萬三千八百三十。貢綿紬、買子木。縣三:流江,緊。西魏縣。景祐三年,廢大竹縣入焉;紹興三年,復分置。



 鄰水,下。唐縣。乾德四年,移治昆樓鎮。



 鄰山。下。梁縣。乾德三年,移治故鄰州城。



 南渡後,增縣一:大竹。



 懷安軍,同下州。乾德五年,以簡州金水縣建為軍。崇寧戶二萬九千六百二十五,口一十七萬四千九百八十
 五。貢紬。縣二:金水,望。



 金堂。望。乾德五年,自漢州來隸。



 寧西軍,本廣安軍,同下州。開寶二年,以合州儂洄、渠州新明二鎮建為軍。淳祐三年,城大良平為治所。寶祐末,歸大元。景定初,復取之。咸淳二年,改軍名。崇寧戶四萬七千五十七,口一十一萬一千七百五十四。貢絹。縣三:渠江,中。開寶二年,自渠州來隸。



 岳池,緊。開寶二年,自果州來隸。



 新明。中。開寶二年,自合州來屬。六年,移治單溪鎮。



 南渡後,增縣一:和溪。開禧三年,升鎮為縣。



 富順監,同下州。本瀘州之富義縣。掌煎鹽。乾德四年,升
 為富義監。太平興國元年改。治平元年,置富順縣;熙寧元年,省。嘉熙元年,蜀亂監廢。咸淳元年,徙治虎頭山。崇寧戶一萬一千二百四十一,口二萬三千七百一十六。貢葛。領鎮十三,鹽井一。



 利州路。府一:興元。州九:利,洋,閬,劍,文,興,蓬,政,巴。縣三十八。關一:劍門。南渡後,府三:興元,隆慶,同慶。州十二:利,金,洋,閬,巴,沔,文,蓬,龍,階,西和,鳳。軍二:大安,天水。紹興三十二年,戶三十七萬一千九十七,口七十六萬九千八百
 五十二。



 興元府,次府,梁州,漢中郡,山南西道節度。舊兼提舉利州路兵馬巡檢事。建炎二年,升本路鈐轄。四年,兼本路經略、安撫使。後分利州路為東、西路:興元、劍、利、閬、金、洋、巴、蓬、大安為東路,治興元;階、成、西和、鳳、文、龍、興為西路,治興州。又置利州路階、成、西和、鳳州制置使,涇原、秦鳳路經略、安撫使。乾道四年,合為一路,興元帥兼領之;淳熙二年,復分;三年,又合;五年,復分;紹熙五年,再合;慶元二年,
 又分;嘉定三年,復合。崇寧戶六萬二百八十四,口一十二萬三千五百四十。貢胭脂、紅花。縣四:南鄭,次赤。



 城固,次畿。



 褒城,次畿。



 西。次畿。至道二年,割隸大安軍;三年,還隸。有錫冶一務。



 茶場一。熙寧八年置。



 南渡後,增縣一:廉水。次畿。紹興四年,析南鄭縣置,以廉水為名。



 利州,都督府,益川郡,寧武軍節度。舊昭武軍,景祐四年改。紹興十四年,分東、西兩路;紹熙五年,復合為一;慶元二年,復分;嘉定三年,復合;十一年,又分。端平三年,兵亂廢。崇寧戶二萬五千三百七十三,口五萬一千五百三
 十九。貢金、鋼鐵。縣四:綿谷,中。



 葭萌,中。



 嘉川,中下。咸平五年,自集州來隸。熙寧三年,省平蜀縣入焉。



 昭化。下。後周益昌縣。開寶五年改。



 洋州,望,洋川郡,武康軍節度。舊武定軍,景祐四年改。建炎以後,嘗置蓬、巴、洋州安撫使,尋罷。崇寧戶四萬五千四百九十,口九萬八千五百六十七。貢隔織。縣三:興道,望。



 西鄉,上。



 真符。中。



 閬州,上,閬中郡。乾德四年,改安德軍節度。紹興十四年,隸東路。端平三年,兵亂。淳祐三年,移治大獲山。崇寧戶
 四萬三千九百三十六,口一十萬九百七。貢蓮綾。縣七:閬中,望。閬水迂曲,繞縣三面,故名。紹興十八年,省玉井鎮入焉。



 蒼溪,緊



 南部,緊



 新井,緊



 奉國,中。熙寧四年,省岐平縣為鎮入焉。



 新政,中。



 西水。中下。熙寧四年,省晉安縣為鎮入焉。



 隆慶府,本劍州,上,普安郡,軍事。乾德五年,廢永歸縣。隆興二年,以孝宗潛邸,升普安軍節度。紹熙元年,升府。端平三年,兵亂。崇寧戶三萬五千二十三,口一十萬七千五百七十三。貢巴戟。縣六:普安,中。熙寧五年,省臨津縣為鎮入焉。



 梓潼,
 上。



 陰平,中。



 武連,中。



 普成,中下。



 劍門。中下。熙寧五年,以劍門關劍門縣復隸州。有小劍、白綿、□巴砍、糧穀、龍聚、托溪六砦。



 巴州,中,清化郡,軍事。乾德四年,廢盤道、歸仁、始寧三縣。咸平五年,以清化屬集州。熙寧五年,廢集州,又廢壁州,以其縣來隸。建炎三年,兼管內安撫。嘉熙四年,兵亂民散。崇寧戶二萬三千三百三十七,口四萬一千五百一十六。貢綿紬。縣五:化城,中下。省集州清化縣為鎮入焉。



 難江,上。舊隸集州。



 恩錫,中下。熙寧三年,省七盤縣為鎮入焉。



 曾口,下。熙寧五年,省其章縣為鎮入焉。



 通江。下。省壁州
 白石、符陽二縣入焉。



 文州,中下,陰平郡,軍事。建炎後,帶沿邊管內安撫,尋罷,隸利西路。紹定末,置司成都。端平後,兵亂州廢。崇寧戶一萬二千五百三十一,口二萬二千七十八。貢麝香。縣一:曲水。中下。西魏縣。有重石、毗谷、張添、磨蓬、留券、羅移、思村、戎門、披波、綏南十砦,水銀務一。



 沔州,下,順政郡,軍事。本興州。紹興十四年,為利西路治所。開禧三年,吳曦僭改開德府。曦誅,改沔州。崇寧戶一萬二千四百三十,口一萬九千六百七十三。貢蜜、蠟。縣
 二:順政,中。開禧三年,改為略陽。



 長舉。中下。



 監一:濟眾。鑄鐵錢。



 蓬州,下,咸安郡,軍事。乾德三年,廢宕渠縣。淳祐三年,置司古渝縣。崇寧戶二萬七千八百二十七,口五萬一千四百七十二。貢絲寧絲綾、綿紬。縣四:蓬池,中。



 儀隴,中。



 營山,中。唐朗池縣。大中祥符五年改。熙寧三年,省蓬山縣為鎮入焉。



 伏虞。中下。熙寧五年,省良山縣為鎮入焉。



 南渡後,增縣二:良山,中下。建炎二年復。



 相如。望。以南有司馬相如故宅而名。嘉熙間,兵亂。寶祐六年,自果州來屬。



 政州,下,江油郡,軍事。本龍州。政和五年,改為政州。紹興
 元年,復為龍州。端平三年,兵亂。寶祐六年,徙治雍村。崇寧戶三千五百二十三,口九千二百九十四。貢麩金、羚羊角、天雄。縣二:江油,中。有乾坡砦。



 清川。下,本馬盤,唐改今名。康定初,增戍兵。端平三年,兵亂地廢。



 大安軍,中,本三泉縣。舊屬興元府。乾德三年,平蜀,以縣直屬京。至道二年,建為大安軍。三年,軍廢,縣仍舊屬京。紹興三年,復升軍。崇寧戶六千七十五,口一萬八百九十一。領鎮二:金牛,青烏。南渡後,復置三泉縣,隸軍。



 金州,上,安康郡,昭化軍節度。前宋隸京西南路,惟此一州未沒於金。建炎四年,屬利州。紹興元年,置金、均、房州鎮撫使。六年,復隸京西南路。九年,隸西川宣撫司。十年,置金、房、開、達安撫使。十三年,隸利州路,又以商州上津、豐陽兩縣來屬。乾道四年,兼管內安撫。縣五:西城,下。



 漢陰,中下。紹興二年,遷治新店,以舊縣為鎮,嘉定三年,升激口鎮為縣。有饒風鎮。



 洵陽,中。



 石泉,下。



 平利。下。



 南渡後,增縣一:上津。中下。本平利縣地。紹興十六年,以鶻嶺關卓馱平為界。



 階州,中下,武都郡,軍事。本隸秦鳳路。紹興初,陜西地盡入於金,惟階、成、岷、鳳、洮五郡、鳳翔府和尚原、隴州方山原存。紹興初,以楊家崖為家計砦。



 縣二:福津,中下。將利。中下。



 同慶府,中下,同谷郡,軍事。本成州,隸秦鳳路,紹興十四年來屬。寶慶元年,以理宗潛邸,升同慶府。縣二:同谷,中。



 慄亭。中。



 西和州,下,和政郡,團練。本隸秦鳳路。紹興元年,入於金,改祐州。舊名岷州。十二年,與金人和,以岷犯金太祖嫌
 名,改西和州,因郡名和政云。以淮西有和州,故加「西」字。開禧二年,又入於金。縣三:長道,緊。



 大潭,中。



 祐川。



 鳳州,下,河池郡,團練。舊屬秦鳳路,紹興十四年來隸。縣三:梁泉,上。



 兩當,上。



 河池。緊。



 天水軍,同下州。紹興初,秦州入於金,分置南、北天水縣。十三年,隸成州。後以成紀之太平社、隴城之東阿社來屬。嘉定元年升軍,九年,移於天水縣舊治。仍置縣一:天水。紹興十五年,廢成紀、隴城二縣來入。



 夔州路。州十:夔,黔,施,忠,萬,開,達,涪,恭,珍。軍三:雲安,梁山,南平。監一:大寧。縣三十二。南渡後,府三:重慶,咸淳,紹慶。州八:夔,達,涪,萬,開,施,播,思。軍三:雲安,梁山,南平。監一:大寧。紹興三十二年,戶三十八萬六千九百七十八,口一百一十三萬四千三百九十八。



 夔州,都督府,雲安郡,寧江軍節度。州初置在白帝城,景德三年,徙城東。建炎三年,升夔、利兵馬鈐轄。淳熙十五年,帥臣帶歸、峽州兵馬司。元豐戶一萬一千二百一十三。
 貢蜜、蠟。縣二:奉節,中。



 巫山。中下



 紹慶府,下,本黔州,黔中郡,軍事,武泰軍節度。紹定元年,升府。紹熙三年,移巡檢治增潭。元豐戶二千八百四十八。貢朱砂、蠟。縣二:彭水,中。嘉祐八年,廢洪杜、洋水、都濡、信寧四縣入焉。有洪杜、小洞、界山、難溪四砦。紹興二年,以元隸珍州戶四十九還隸。



 黔江。下。有白石、門闌、佐水、永安、安樂、雙洪、射營、右水、蠻塚、浴水、潛平、鹿角、萬就、六堡、白水、土溪、小溪、石柱、高望、木孔、東流、李昌、僕射、相陽、小村、石門、茆田、木柵、虎眼二十九砦。



 羈縻州四十九。南寧州、遠州、犍州、清州、蔣州、知州、蠻州、襲州、峨州、邦州、鶴州、勞州、義州、福州、儒州、令州、郝州、普寧州、緣州、那州、鸞州、絲州、邛州、敷州、晃州、侯州、焚州、添州、瑤州、雙城州、訓州、鄉州、
 茂龍州、整州、樂善州、撫水州、思元州、逸州、思州、南平州、勛州、姜州、棱州、鴻州、和武州、暉州、毫州、鼓州、懸州。



 南渡後,羈縻州五十六。



 施州,下,清江郡,軍事。元豐戶一萬九千八百四。貢黃連、木藥子。縣二:清江,中下。有歌羅、永寧、細沙、寧邊、尖木、夷平六砦。熙寧六年五月,省施州永興砦,置夷平砦。元豐三年七月,廢永寧砦,置行廊、安確二砦。



 建始。中下。有連天一砦。



 監一:廣積。紹聖三年置,鑄鐵錢。



 咸淳府,下,本忠州,南賓郡,軍事。咸淳元年,以度宗潛邸,升府。元豐戶三萬五千九百五十。貢綿紬。縣三:臨江,中下。



 墊江,中下。熙寧五年,省桂溪縣入焉。



 南賓。下。



 南渡後,增縣二:豐都,下。



 龍渠。下。



 萬州,下,南浦郡,軍事。開寶三年,以梁山為軍。元豐戶二萬五百五十五。貢金、木藥子。縣二:南浦,下。有平雲砦。



 武寧。下。



 開州,下,盛山郡,軍事。崇寧戶二萬五千。貢白絲寧、車前子。縣二:開江,上。慶歷四年,廢新浦縣入焉。



 清水。中。



 舊名萬歲縣,後改。



 達州,上,通川郡,軍事。本通州。乾德三年改。乾德五年,廢閬英、宣漢二縣。熙寧六年,省三岡縣;七年,省石鼓縣,分
 隸通川、新寧、永睦三縣。元豐戶四萬六百四十。貢紬。縣五:通川,中。



 巴渠,中。



 永睦,下。隋永穆縣。今改。



 新寧,下。



 東鄉。下。



 南渡後,增縣一:通明。下。舊通明院。



 涪州,下,涪陵郡,軍事。熙寧三年,廢溫山縣為鎮。大觀四年,廢白馬砦。咸淳二年,移治三臺山。元豐戶一萬八千四百四十八。貢絹。縣三:涪陵,下。有白馬鹽場。



 樂溫,下。



 武龍。下。宣和元年,改武龍縣為枳縣。紹興元年依舊。



 重慶府,下,本恭州,巴郡,軍事。舊為渝州。崇寧元年,改
 恭州,後以高宗潛藩,升為府。舊領萬壽縣,乾德五年,廢。雍熙中,又廢南平縣。慶歷八年,以黔州羈縻南、溱二州來隸。皇祐五年,以南州置南川縣。熙寧七年,以南川縣隸南平軍。元豐戶四萬二千八十。貢葛布、牡丹皮。縣三:巴,中。有石英、峰玉、藍溪、新興四鎮。



 江津,中下。乾德五年,移治馬□□鎮。



 壁山。下。



 羈縻州一。溱州,領榮懿、扶歡二縣。以酋首領之,後隸南平軍。



 雲安軍,同下州。開寶六年,以夔州雲安縣建為軍。建炎三年為軍使。元豐戶一萬一千七十五。貢絹。縣一:雲安。
 望。有思問、捍技、平南三砦,玉井鹽場、團云鹽井。



 監一:雲安。熙寧四年,以雲安監戶口析置安義縣。八年,戶還隸雲安縣,復為監。



 梁山軍,同下州,高梁郡。開寶二年,以萬州石氏屯田務置軍,撥梁山縣來隸。熙寧五年,又析忠州桂溪地益軍。元祐元年,還隸萬州,尋復故。元豐戶一萬二千二百七十七。貢綿。縣一:梁山。中下。



 南平軍,同下州。熙寧八年,收西番部,以恭州南川縣銅佛壩地置軍。領縣二:南川,中下。熙寧八年,省入隆化。元豐元年復置。有榮懿、開邊、通
 安、安穩、歸正五砦,溱川堡。



 隆化。下。熙寧八年,自涪州來隸。有七渡水砦,大觀四年砦廢。



 溱溪砦,本羈縻溱州,領榮懿、扶歡二縣。熙寧七年,招納,置榮懿等砦隸恭州,後隸南平軍。大觀二年,別置溱州及溱溪、夜郎兩縣。宣和二年,廢州及縣,以溱溪砦為名,隸南平軍。



 大寧監,同下州。開寶六年,以夔州大昌縣鹽泉所建為監。元豐戶六千六百三十一。貢蠟。縣一:大昌。中下。端拱元年,自夔州來隸。舊在監南六十里,嘉定八年,徙治水口監。



 珍州,唐貞觀中開山洞置,唐末沒於夷。大觀二年,大駱解上下族帥獻其地,復建為珍州。宣和三年,承州廢,以綏陽縣來隸。縣二:樂源、綏陽,本羈縻夷州,大觀三年,酋長獻其地,建為承州,領綏陽、都上、義泉、寧夷、洋川五縣;宣和三年,廢州及都上等縣,以綏陽隸珍州。遵義砦,大觀二年,播州楊文貴獻其地,建遵義軍及遵義縣;宣和三年廢軍及縣,以遵義砦為名,隸珍州。



 思州,政和八年建,領務川、邛水、安夷三縣。宣和四年,廢
 州為城及務川縣,以務川城為名;邛水、安夷二縣皆作堡,並隸黔州。紹興元年,復為思州。縣三:務川,安夷,邛水。宣和四年並廢,隸黔州。紹興二年復。



 播州,樂源郡。大觀二年,南平夷人楊文貴等獻其地,建為州,領播川、瑯川、帶水三縣。宣和三年,廢為城,隸南平軍。端平三年,復以白綿堡為播州,三縣仍廢,嘉熙三年,復設播州,充安撫使。咸淳末,以珍州來屬。縣一,樂源。中。有遵義砦,開禧三年升軍,嘉定十一年復為砦。



 川、峽四路,蓋《禹貢》梁、雍、荊三州之地,而梁州為多。天文與秦同分。南至荊峽,北控劍棧,西南接蠻夷。土植宜柘,繭絲織文纖麗者窮於天下,地狹而腴,民勤耕作,無寸土之曠,歲三四收。其所獲多為遨游之費,踏青、藥市之集尤盛焉,動至連月。好音樂,少愁苦,尚奢靡,性輕揚,喜虛稱。庠塾聚學者眾,然懷土罕趨仕進。涪陵之民尤尚鬼俗,有父母疾病,多不省視醫藥,及親在多別籍異材。漢中、巴東,俗尚頗同,淪於偏方,殆將百年。孟氏既平,聲
 教攸暨,文學之士,彬彬輩出焉。



\end{pinyinscope}