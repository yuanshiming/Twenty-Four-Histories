\article{列傳第七十四}

\begin{pinyinscope}

 韓億,
 字
 宗魏,其先真定靈壽人,徙開封之雍丘。舉進士,為大理評事、知永城縣,有治聲。他邑訟不決者,郡守皇甫選輒屬億治之。通判陳州,會河決,治堤費萬計,億不
 賦民而營築之。真宗嘗欲召試,而與王旦有親嫌,特召見,改一官知洋州。州豪李甲,兄死迫嫂使嫁,因誣其子為他姓,以專其貲。嫂訴於官,甲輒賂吏掠服之,積十餘年,訴不已。億視舊牘未嘗引乳醫為證,召甲出乳醫示之,甲亡以為辭,冤遂辨。累遷尚書屯田員外郎、知相州。河北旱,轉運使不以實聞,億獨言歲饑,願貸民租。有誣其子綱請求受金者,億請自置獄按之,事雖辨,猶降通判大名府。尋為殿中侍御史,遷侍御史,安撫淮、浙,除開
 封府判官,出為河北轉運使。



 仁宗初,進直史館、知青州,以司封員外郎兼侍御史知雜事,判大理寺丞。吳植知臨江軍,使人納金於宰相王欽若,因牙吏至京師,審之,語頗泄,欽若知不可掩,執吏以聞。詔付臺治,而植自言未嘗納金,反誣吏誤以問所親語達欽若。億窮治之,蓋植以病懼廢,金未達而事已露也。植乃除名,並按欽若,詔釋不問。三司更茶法,歲課不登,億承詔劾之,由丞相而下皆坐失當之罰,其不撓如此。自薛奎後,億獨掌臺
 務者逾年。



 除龍圖閣待制,奉使契丹。時副使者,章獻外姻也,妄傳皇太后旨於契丹,諭以南北歡好傳示子孫之意,億初不知也。契丹主問億曰:「皇太后即有旨,大使何獨不言?」億對曰:「本朝每遣使,皇太后必以此戒之,非欲達於北朝也。」契丹主大喜,曰:「此兩朝生靈之福也。」人謂副使既失辭,而億更以為恩意,甚推美之。



 知亳州,召知審刑院,再遷兵部郎中、同判吏部流內銓,以右諫議大夫、樞密直學士知益州。故事,益州歲出官粟六萬石,
 振糶貧民。是歲大旱,億倍數出粟,先期予民,民坐是不饑。又疏九升江口,下溉民田數千頃。維、茂州地接羌夷,蕃部歲至永康官場鬻馬,億慮其覘兩川,奏徙場黎州境上。拜御史中丞,請如唐制,置御史裏行。



 景祐二年,以尚書工部侍郎同知樞密院事。時承平久,武備不戒,乃請二府各列上才任將帥者數十人,稍試用之。又言武臣宜知兵,而書禁不傳,請纂其要授之。於是帝親集《神武秘略》,以賜邊臣。



 唃廝囉與趙元昊相攻,來獻捷。朝廷
 議加唃廝囉節制,億曰:「彼皆蕃臣也,今不能諭令解仇,乃因而加賞,非所以綏御四方也。」議遂寢。元昊歲遣人至京師,出入民間無他禁,億請下詔為除館舍禮之,官主貿易,外雖若煩擾,實羈防之。



 知開封府范仲淹獻《百官圖》,指宰相呂夷簡差除不平,而陰薦億可用。仲淹既貶,帝以諭億,億曰:「仲淹舉臣以公,臣之愚陛下所知;舉臣以私,則臣委質以來,未嘗交託於人。」遂除戶部、參知政事。會忻州地大震,諫官韓琦言宰相王隨、陳堯佐非
 輔弼才,又言億子綜為群牧判官,不當自請以兄綱代之。遂與宰相皆罷,知應天府,尋加資政殿學士、知成德軍。改澶州,復知亳州,官至尚書左丞,以太子少傅致仕。卒,贈太子太保,諡「忠憲」。



 億性方重,治家嚴飭,雖燕居,未嘗有惰容。見親舊之孤貧者,常給其昏葬。每見天下諸路有奏攟拾官吏小過者,輒顏色不懌,曰:「天下太平,聖主之心,雖昆蟲草木,皆欲使之得所。今仕者大則望為公卿,次亦望為侍從、職司一千石,其下亦望京朝、幕職,
 奈何錮之於盛世?」八子:綱、綜、絳、繹、維、縝、緯、緬。



 綱,尚書水部員外郎。慶歷中,知光化軍,性苛急,不能撫循士卒。會盜張海剽劫至境上,綱帥禁兵乘城,給餅餌多不時,民具酒食犒軍,輒收其羊豕,市錢制兵器,士皆憤怒。又嘗命軍校作陣圖,不成,將斬之,眾益駭。一日,士方食,軍校邵興叱眾起勿食。綱怒,執數人繫獄。興懼,帥眾劫庫兵為亂,欲殺綱。綱攜妻子縋城,由漢江而下。興等遂縱火掠城中,引眾趨蜀道,為官兵所敗,遂斬之,余黨悉誅。綱
 坐棄城除名,編管英州。



 綜,字仲文。蔭補將作監主簿,遷大理評事。舉進士中第,通判鄧州、天雄軍。會河溢金堤,民依丘冢者數百家。綜令曰:「能濟一人,予千錢。」民爭操舟筏以救,已而丘冢多潰。呂夷簡自北京入相,薦為集賢校理、同知太常院。歷開封府推官,數月,遷三司戶部判官、同修起居注。



 使契丹,契丹主問其家世,綜言億在先朝嘗持禮來,契丹主喜曰:「與中國通好久,父子俱使我,宜酌我酒。」綜率同使
 者五人起為壽,契丹主亦離席酬之,歡甚。既還,陳執中以為生事,出知滑州,徙許州。



 殿前指揮使許懷德從妹亡,有別產在陽翟,以無子,籍於官,懷德欲私有之,訟未決。因楊儀為書屬綜,書至而轉運使已徙獄他州矣。綜坐得書不以聞,奪集賢校理,知袁州。未幾,復為江東轉運使。還,再修起居注,累遷刑部員外郎、知制誥,卒。



 綜嘗為契丹館伴使,使者欲為書稱北朝而去契丹號。綜曰:「自古未有建國而無號者。」使慚,遂不復言。其後朝廷擇
 館伴契丹使者,帝曰:「孰有如韓綜者乎?」子宗道,為戶部侍郎、寶文閣待制。



 綱子宗彥,字欽聖。蔭補將作監主簿。舉進士甲科,累遷太常博士。以大臣薦,召試,為集賢校理。歷提點京西、京東刑獄。應天府失入平民死罪,獄成未決,通判孫世寧辨正之。獄吏當坐法,而尹劉沆縱弗治;宗彥往按舉,沆復沮止之。宗彥疏沆於朝,抵吏罪。仁宗春秋高,未有嗣。宗彥上書曰:「漢章帝詔諸懷妊者賜胎養穀,人三斗,復其夫勿算一歲,著為令。臣考尋世次,
 帝八子,長則和帝,而質、安以下諸帝皆其系胄,請修胎養之令。」且曰:「人君務蕃毓其民,則天亦昌衍其子孫矣。」以尚書兵部員外郎判三司鹽鐵勾院,卒。



 綜子宗道,歷官至戶部侍郎、寶文閣待制。



 韓絳,字子華。舉進士甲科,通判陳州。直集賢院,為開封府推官。有男子冷青,妄稱其母頃在掖庭得幸,有娠而出生己,府以為狂,奏流汝州。絳言,留之在外將惑眾。追責窮治,蓋其母嘗執役宮禁,嫁民冷緒,生一女,乃生青,
 遂論棄市。



 歷戶部判官。江南飢,為體量安撫使,行便民事數十條;宣州守廖詢貪暴不法,下吏寘諸理,民大悅。使還,同修起居注,擢右正言。仁宗謂絳曰:「用卿出自朕,卿凡論事,不宜過激,當存朝廷大體,要令可行,毋使朕為不聽諫者。」



 入內都都知王守忠兼判內侍省,絳言:「判名太重,且國朝以來,未有兼判兩省者。」詔自今勿復除。道士趙清貺出入宰相龐籍家,以賂敗,開封杖流之,道死。絳言籍諷府殺之,籍與尹俱謫去。未幾復進,絳力爭不
 得,遂解言職。明年,知制誥,乞守河陽,召判流內銓。河決商胡,用李仲昌議,開六塔河而患滋甚,命絳安撫河北。時宰主仲昌,人莫敢異。絳劾其蠹國害民,罪不可貸,仲昌遂竄嶺表。遷龍圖閣直學士、知瀛州。歐陽修率同列言:「絳宜在朝廷,瀛非所處也。」留知諫院,糾察在京刑獄。為翰林學士、御史中丞。



 帝禱茅山求嗣,絳草祝辭,因勸帝汰出宮人,及限內臣養子,以重絕人之世,皆從之。掖庭劉氏通請謁為奸,絳以告帝,帝曰:「非卿言,朕無由知。」不
 數日,出劉氏及他不謹者。真定守呂溱犯法,從官通章請貰之,絳曰:「法行當自貴者始,更相請援,則公道廢矣。」並劾諸請者,溱遂絀。富弼用張茂實掌禁兵,絳言:「人謂茂實為先帝子,豈宜用典宿衛?」不報,闔門待罪,自言不敢復稱御史中丞。詔召之,及出,不秉笏穿朝堂,諫官論之,罷知蔡州。



 數月,以翰林侍讀學士知慶州。熟羌據堡為亂,即日討平之。加端明殿學士、知成都府。張詠鎮蜀日,春糶米,秋糶鹽,官給券以惠貧弱,歷歲久,權歸豪右;
 中人奉使至蜀,使酒吏主貿易,因附益以取悅,絳悉奏罷之。召知開封府,為三司使。請以川、陝職田穀輸常平倉,而隨其事任道里差次給直。帝嘆曰:「眾方姑息,卿獨不能徇時邪!」即行之。內諸司吏數干恩澤,絳輒執不可。為帝言:「身犯眾怒,懼有飛語。」帝曰:「朕在藩邸日,頗聞有司以國事為人情。卿所守固善,何憚於讒?」



 神宗立,韓琦薦絳有公輔器,拜樞密副使。始請建審官西院,掌武臣升朝者,以息吏奸。神宗嘗問天下遺利,絳請盡地力。因
 言差役之弊,願更定其法,役議自此始矣。代陳升之同制置三司條例,王安石每奏事,必曰:「臣見安石所陳非一,皆至當可用,陛下宜省察。」安石恃以為助。



 熙寧三年,參知政事。夏人犯塞,絳請行邊,安石亦請往。絳曰:「朝廷方賴安石,臣宜行。」乃以為陝西宣撫使。既又兼河東,幾事不可待報者,聽便宜施行,授以空名告敕,得自除吏。十二月,即軍中拜同中書門下平章事、昭文館大學士,開幕府於延安。絳素不習兵事,注措乖方,選蕃兵為七
 軍,用知青澗城种諤策,欲取橫山,令諸將聽命於諤,厚賞犒蕃兵,眾皆怨望;又奪騎兵馬以與之,有抱馬首以泣者。既城囉兀,又冒雪築撫寧堡,調發騷然。已而二城陷,趣諸道兵出援,慶卒遂作亂。議者罪絳,罷知鄧州。明年,以觀文殿學士徙許州,進大學士,徙大名府。七年,復代王安石相。既顓處中書,事多稽留不決,且數與呂惠卿爭論,乃密請帝再用安石。安石至,頗與絳異。有劉佐者,坐法免,安石欲抆拭用佐,絳不可。議帝前未決,即再
 拜求去。帝驚曰:「此小事,何必爾?」對曰:「小事尚不伸,況大事乎!」帝為逐佐。未幾,絳亦出知許州。



 元豐元年,拜建雄軍節度使、知定州。入為西太一宮使。六年,知河南府。夏,大雨,伊、洛間民被溺者十五六。絳發廩振恤,環城築堤,數月,水復至,民賴以免。哲宗立,更鎮江軍節度使、開府儀同三司,封康國公,為北京留守。河決小吳,都水議傍魏城鑿渠東趨金堤,役甚棘。絳言:「功必不成,徒耗費國力,而使魏人流徙,非計也。」三奏,訖罷之。元祐二年,請老,
 以司空、檢校太尉致仕。明年,卒,年七十七。贈太傅,諡曰「獻肅」。



 絳臨事果敢,不為後慮。好延接士大夫,數薦司馬光可用,終以黨王安石復得政,是以清議少之。



 子宗師,字傳道,以父任歷州縣職。既登第,王安石薦為度支判官、提舉河北常平。累官至集賢殿修撰、知河中府,卒。初,宗師在神宗朝,數賜對,常弗忍去親側,屢辭官不拜,世以孝與之。



 韓維,字持國。以進士奏名禮部,方億輔政,不肯試大廷,
 受蔭入官。父沒後,閉門不仕。宰相薦其好古嗜學,安於靜退,召試學士院,辭不就。富弼辟河東幕府,史館修撰歐陽修薦為檢討、知太常禮院。禮官議祫享東向位,維請虛室以待太祖。溫成后立廟用樂,維以為不如禮,請一切裁去。議陳執中諡,以為張貴妃治喪皇儀殿、追冊位號,皆執中所建,宜曰「榮靈」,詔諡曰「恭」,維曰:「責難於君謂之恭,執中何以得此?」議訖不行,乞罷禮院。以秘閣校理通判涇州。



 神宗封淮陽郡王、潁王,維皆為記室參軍。
 王每事咨訪,維悉心以對,至拜起進趨之容,皆陳其節。嘗與論天下事,語及功名。維曰:「聖人功名,因事始見,不可有功名心。」王拱手稱善。聞維引疾請郡,上章留之。時禁中遣使泛至諸臣家,為王擇妃。維上疏曰:「王孝友聰明,動履法度,方向經學,以觀成德。今卜族授室,宜歷選勲望之家,謹擇淑媛,考古納采、問名之義,以禮成之,不宜苟取華色而已。」



 左、右史闕,英宗訪除授例,執政曰:「用館閣久次及進士高第者。」帝曰:「第擇人,不必專取高科。」
 執政以維對,遂同修起居注、侍邇英講。帝初免喪,簡默不言。維上疏曰:「邇英閣者,陛下燕閒之所也。侍於側者,皆獻納論思之臣。陳於前者,非經則史。可以博咨訪之義,窮仁義之道,究成敗之原。今禮制終畢,臣下傾耳以聽玉音,陛下之言,此其時也。臣請執筆以俟。」進知制誥、知通進銀臺司。



 御史呂誨等以濮議得罪,維諫曰:「誨等審議守職,不過欲陛下盡如先王之法而止爾。請追還前詔,令百官詳議,以盡人情;復誨等職任,以全政體。」既
 而責命不由門下,維又言:「罷黜御史,事關政體,而不使有司與聞,紀綱之失,無甚於此,乞解銀臺司。」不從,遂闔門待罪。有詔舉臺官二人,維言:「呂誨、范純仁有已試之效,願復其職。」翰林學士范鎮作批答不合旨,出補郡。維言:「鎮所失只在文字,當涵容之。前黜錢公輔,中外以為太重,連退二近臣,而眾莫知其所謂,自此誰敢盡忠者?」



 潁王為皇太子,兼右庶子。神宗即位,維進言:「百執事各有職位,當責任,若代之行事,最為失體。天下大事不可
 猝為,人君設施,自有先後。」因釋滕文公問孟子居喪之禮,推後世禮文之變,以伸規諷,帝皆嘉納。除龍圖閣直學士。



 御史中丞王陶彈宰相韓琦為跋扈,罷為翰林學士。維言:「中丞之言是,宰相安得無罪?若其非是,安得止罷臺職?今為學士,是遷也。」參知政事吳奎論陶事,出知青州。維言進退大臣,不當如是。詔遷奎官。維又言:「執政罷免,則為降黜;今復遷官,則為褒進。二者理難並行,此與王陶罷中丞而加學士何以異?」章上,奎還就職。維援
 前言求去,知汝州。數月,召兼侍講、判太常寺。



 初,僖祖主已遷,及英宗祔廟,中書以為僖祖與稷、契等,不應毀其廟。維言:「太祖戡定大亂,子孫遵業,為宋太祖,無可議者。僖祖雖為高祖,然仰跡功業,非有所因,若以所事稷、契事之,懼有未安,宜如故便。」王安石方主初議,持不行。熙寧二年,遷翰林學士、知開封府。明年,為御史中丞,以兄絳在樞府,力辭之。安石亦惡其言保甲事,復使為開封,始分置八廂決輕刑,轂下清肅。時吳充為三司使,帝
 曰:「維、充以文學進,及任煩劇,而皆稱職,可謂得人矣。」兼侍讀學士,充群牧使。考試制舉人,孔文仲對策入等,以切直罷歸。維言:「陛下毋謂文仲為一賤士,黜之何損。臣恐賢俊解體,忠良結舌,阿諛苟合者將窺隙而進,為禍不細。」安石益惡之。



 樞密使文彥博求去,帝曰:「密院事劇,當除韓維佐卿。」明日,維奏事殿中,以言不用,請郡。帝曰:「卿東宮舊人,當留以輔政。」對曰:「使臣言得行,賢於富貴;若緣攀附舊恩以進,非臣之願也。」遂出知襄州,改許州。



 七年二月,召為學士承旨。入對,帝曰:「天久不雨,朕日夜焦勞,奈何?」維曰:「陛下憂閔旱災,損膳避殿,此乃舉行故事,恐不足以應天變。當痛自責己,廣求直言。」退,又上疏曰:「近畿內諸縣,督索青苗錢甚急,往往鞭撻取足,至伐桑為薪以易錢貨,旱災之際,重罹此苦。若夫動甲兵,危士民,匱財用於荒夷之地,朝廷處之不疑,行之甚銳;至於蠲除租稅,寬裕逋負,以救愁苦之民,則遲遲而不肯發。望陛下奮自英斷行之,過於養人,猶愈過於殺人也。」
 上感悟,即命維草詔求直言。其略曰:「意者聽納不得於理與?獄訟非其情與?賦斂失其節與?忠言讜論郁於上聞,而阿諛壅蔽以成其私者眾與?」詔出,人情大悅。有旨體量市易、免行利病,權罷力田、保甲,是日乃雨。



 王安石罷,會絳入相,加端明殿學士、知河陽,復知許州。帝幸舊邸,進資政殿學士。曾鞏當制,稱其純明亮直,帝令改命詞。維知帝意,請提舉嵩山崇福宮。帝崩,赴臨闕庭。宣仁后手詔勞問,維對曰:「人情貧則思富,苦則思樂,困則思
 息,鬱則思通。誠能常以利民為本,則民富;常以憂民為心,則民樂;賦役非人力所堪者去之,則勞困息;法禁非人情所便者蠲之,則鬱塞通。推此而廣之,盡誠而行之,則子孫觀陛下之德,不待教而成矣。」



 未幾,起知陳州,未行,召兼侍讀,加大學士。嘗言:「先帝以夏國主秉常廢,故興問罪之師。今既復位,有蕃臣禮,宜還其故地。」因陳兵不可不息者三,地不可不棄者五。又言:「仁宗選建儲嗣,一時忠勲皆被寵祿;范鎮首開此議,賞獨不及,願褒顯
 其功。」鎮於是復起用。



 元祐更役法,命維詳定。時四方書疏多言其便,維謂司馬光曰:「小人議論,希意迎合,不可不察。」成都轉運判官蔡曚附會定差,維惡而劾之。執政欲廢王安石《新經義》,維以當與先儒之說並行,論者服其平。拜門下侍郎。御史張舜民以言事罷,王岩叟救之,折簡密詢上官均。語泄,詔岩叟分析。維曰:「臣下折簡聚談,更相督責,乃是相率為善,何害於理?若瑣瑣責善,懼於國事無益也。」



 維處東省逾年,有忌之者密為讒訴,詔
 分司南京。尚書右司王存抗聲簾前曰:「韓維得罪,莫知其端,臣竊為朝廷惜。」乃還大學士、知鄧州。兄絳為之請,改汝州。久之,以太子少傅致仕,轉少師。



 紹聖中,坐元祐黨,降左朝議大夫,再謫崇信軍節度副使,均州安置。諸子乞納官爵,聽父里居。哲宗覽奏惻然,許之。元符元年,以幸睿成宮,復左朝議大夫,是歲卒。年八十二。徽宗初,悉追復舊官。



 韓縝,字玉汝。登進士第,簽書南京判官。仁宗以水災求
 直言,縝上疏曰:「今國本未立,無以系天下心,此陰盛陽微之應。」詞極剴切。劉沆薦其才,命編修三班敕。前此,武臣不執親喪。縝建言:「三年之服,古今通制;晉襄衰墨從戎,事出一時。」遂著令,自崇班以上聽持服。為殿中侍御史。參知政事孫抃持祿充位;權陝西轉運副使薛向赴闕,樞密院輒畫旨除為真;劉永年以外戚除防禦使;內侍史志聰私役皇城親從:縝皆極論之。帝為罷抃,寢向與永年之命,而正志聰罪。遷侍御史、度支判官,出為兩
 浙、淮南轉運使,移河北。



 夏諒祚死,子秉常嗣,遣使求封冊。朝廷方責夏人不修職貢,欲擇人詰其使。縝適陛辭,神宗命之往。縝至驛問罪,使者引服,迨夜,奏上。帝喜,改使陝西。入知審官西院、直舍人院。以兄絳執政,改集賢殿修撰、鹽鐵副使,以天章閣待制知秦州。嘗宴客夜歸,指使傅勍被酒,誤隨入州宅,與侍妾遇,縝怒,令軍校以鐵裹杖箠殺之。勍妻持血衣,撾登聞鼓以訴,坐落職,分司南京。秦人語曰:「寧逢乳虎,莫逢玉汝。」其暴酷如此。久
 之,還待制、知瀛州。



 熙寧七年,遼使蕭禧來議代北地界。召縝館客,遂報聘,令持圖牒致遼主,不克見而還。知開封府,禧再至,復館之。詔乘驛詣河東,與禧分畫,以分水嶺為界。覆命,賜襲衣、金帶,為樞密都承旨,還龍圖閣直學士。元豐五年,官制行,易太中大夫、同知樞密,進知院事。



 哲宗立,拜尚書右僕射兼中書侍郎。首相蔡確與章惇謀誣東朝,及確為山陵使,縝暴其奸狀,由是東朝及外廷悉知之。確使還,欲以其屬高遵惠、張璡、韓宗文為
 美官。宣仁后以訪縝,縝曰:「遵惠為太后從父;璡者,中書郎璪之弟;宗文,臣侄也。今擢用非次,則是君臣各私其親,何以示天下?」乃止。



 元祐元年,御史中丞劉摯、諫官孫覺、蘇轍、王覿,論縝才鄙望輕,在先朝為奉使,割地六百里以遺契丹,邊人怨之切骨,不可使居相位。章數十上,罷為觀文殿大學士、知潁昌府。移永興、河南,拜安武軍節度使、知太原府,易節奉寧軍。請老,為西太一宮使,以太子太保致仕。紹聖四年卒,年七十九。贈司空,諡曰「莊
 敏」。



 縝外事莊重,所至以嚴稱。雖出入將相而寂無功烈,厚自奉養,世以比晉[[W:何曾
 
 何曾]]云。子宗武。



 宗武,第進士,韓宗彥鎮瀛州,辟為河間令。值河溢,增堤護城,吏率兵五百伐材近郊,雖墓木亦不免,父老遮道泣,宗武入府白罷之。徽宗即位,為秘書丞,因日食上疏言:「近世事有微漸而不可不察者五:大臣不畏公論,小臣趨利附下,一也。人主怠於政事,威柄下移,怨讟歸上,二也。左右無輔拂之士,守邊無禦侮之臣,三也。開境土
 以速邊患,耗賦財以弊民力,四也。歲穀不登,倉庾空竭,民人流亡,盜賊數起,五也。根治朋黨,追復私怨。正士黜廢,耆老殲亡,旋起大獄,害及善類。文章號令,衰於前世。大河決溢,饑饉薦臻。執政大臣,人懷異意,排去舊怨,以立新黨,徒為紛紛,無憂國忘家之慮。誠願躬攬權綱,收還威柄,敷言奏功,考察名實,不以侍御之好、鐘鼓之娛為樂。仁祖惻怛至誠,以收天下之心;神宗厲精不息,以舉天下之事;皆所宜法。」不報。



 哲宗將祔廟,中旨索省中
 書畫甚急。宗武言:「先帝祔廟,陛下哀慕方深,而丹青之玩,取索不已,播之於外,懼損聖德。陛下踐祚,如日初升,當講劘典訓,開廣聖學,好玩易志,正古人所戒也。」疏入,皇太后見之,怒曰:「是皆內侍數輩所為爾!」欲盡加罰,帝委曲申救,乃已。明日,太后對宰相獎嘆,令俟諫官員闕即用之。尋除都官員外郎,改開封府推官。丐外,為淮南轉運判官。前使者貸上供錢,禁庭遣使來索。宗武奏具狀,詞極鯁切,坐貶秩,罷歸。久之,蔡京欲以知潁州。帝語
 秘書事,京不敢復言,遂致仕。官累太中大夫,年八十二卒。



 論曰:王稱曰:「昔袁安未嘗以贓罪鞫人,史氏以其仁心,足以覃乎後昆。韓億不悅捃人小過,而君子知其後必大,皆盛德事也。億有子位公府,而行各有適。絳適於同,維適於正,縝適於嚴。嗚呼,維其賢哉!」



\end{pinyinscope}