\article{志第二十四 律歷四}

\begin{pinyinscope}

 道體為一,天地之元,萬物之祖也。散而為氣,則有陰有陽;動而為數,則有奇有偶;凝而為形,則有剛有柔;發而為聲,則有清有濁,其著見而為器,則有律、有呂。凡禮樂、
 刑法、權衡、度量皆出於是。自周衰樂壞,而律呂候氣之法不傳。西漢劉歆、揚雄之徒,僅存其說。京房作準以代律,分六十聲,始於南事,終於去滅。然聲細而難分,世不能用。歷晉及隋、唐,律法微隱。《宋史》止載律呂大數,不獲其詳。今掇仁宗論律及諸儒言鐘律者記於篇,以補續舊學之闕。



 仁宗著《景祐樂髓新經》,凡六篇,述七宗二變及管分陰陽、剖析清濁,歸之於本律。次及間聲,合古今之樂,參之以六壬遁甲。



 其一、釋十二均,曰:「黃鐘之宮為
 子、為神后、為土、為雞緩、為正宮調,太簇商為寅、為功曹、為金、為般頡、為大石調,姑洗角為辰、為天剛、為木、為嗢沒斯、為小石角,林鐘徵為未、為小吉、為火、為雲漢、為黃鐘征,南呂羽為酉,為從魁、為水、為滴、為般涉調,應鐘變宮為亥、為登明、為日、為密、為中管黃鐘宮,蕤賓變徵為午、為勝先、為月、為莫、為應鐘征。大呂之宮為大吉、為高宮,夾鐘商為大沖、為高大石,仲呂角為太一、為中管小石調,夷則徵為傳送、為大呂徵,無射羽為河魁、為高般
 涉,黃鐘變宮為正宮調,林鐘變徵為黃鐘征。太簇之宮為中管高宮,姑洗商為高大石,蕤賓角為歇指角,南呂徵為太簇征,應鐘羽為中管高般涉,大呂變宮為高宮,夷則變徵為大呂徵。夾鐘之宮為中呂宮,仲呂商為雙調,林鐘角在今樂亦為林鐘角,無射徵為夾鐘征,黃鐘羽為中呂調,太簇變宮為中管高宮,南呂變徵為太簇征。姑洗之宮為中管中呂宮,蕤賓商為中管商調,夷則角為中管林鐘角,應鐘徵為姑洗徵,大呂羽為中管中呂
 調,夾鐘變宮為中呂宮,無射變徵為夾鐘征。仲呂之宮為道調宮,林鐘商為小石調,南呂角為越調,黃鐘徵為中呂徵,太簇羽為平調,姑洗變宮為中管中呂宮,應鐘變徵為姑洗徵。蕤賓之宮為中管道調宮,夷則商為中管小石調,無射角為中管越調,大呂徵為蕤賓征,夾鐘羽為中管平調,中呂變宮為道調宮,黃鐘變徵為仲呂徵,林鐘之宮為南呂宮,南呂商為歇指調,應鐘角為大石調,太簇微為林鐘征,姑洗羽為高平調,蕤賓變宮為
 中管道調宮,大呂變徵為蕤賓征。夷則之宮為仙呂,無射商為林鐘商,黃鐘角為高大石調,夾鐘徵為夷則徵,仲呂羽為仙呂調,林鐘變宮為南呂宮,太簇變徵為林鐘征。南呂之宮為中管仙呂宮,應鐘商為中管林鐘商,大呂角為中管高大石角,姑洗徵為南呂徵,蕤賓羽為中管仙呂調,夷則變宮為仙呂宮,夾鐘變徵為夷則徵。無射之宮為黃鐘宮,黃鐘商為越調,太簇角為變角,仲呂徵為無射徵,林鐘羽為黃鐘羽,南呂變宮為中管仙
 呂宮,姑洗變徵為南呂徵。應鐘之宮為中管黃鐘宮,大呂商為中管越調,夾鐘角為中管雙角,蕤賓徵為應鐘征,夷則羽為中管黃鐘羽,無射變宮為黃鐘宮,仲呂變徵為無射徵。」



 二、明所主事,調五聲為五行、五事、四時、五帝、五神、五岳、五味、五色,為生數一二三四五、成數六七八九十,為五藏、五官及五星。



 三、辯音聲,曰:「宮聲沉厚粗大而下,為君,聲調則國安,亂則荒而危。合口通音謂之宮,其聲雄洪,屬平聲,西域言『婆陀力』。一曰婆陀力。



 商聲勁凝
 明達,上而下歸於中,為臣,聲調則刑法不作,威令行,亂則其宮壞。開口吐聲謂之商,音將將、倉倉然,西域言『稽識』。『稽識』,猶長聲也。角聲長而通徹,中平而正,為民,聲調則四民安,亂則人怨。聲出齒間謂之角,喔喔、確確然,西域言『沙識』,猶質直聲也。徵聲抑揚流利,從下而上歸於中,為事,聲調則百事理,亂則事隳。齒合而唇啟謂之徵,倚倚、□戲□戲然,西域言『沙臘』。『沙臘』,和也。羽聲喓喓而遠徹,細小而高,為物,聲調則倉稟實、庶物備,亂則匱竭。齒開
 唇聚謂之羽,詡、雨、酗、芋然。西域言『般瞻』。變宮,西域言『侯利箑』,猶言『斛律』聲也。變徵聲,西域言『沙侯加濫』,猶應聲也。」



 其四、明律呂相生,祭天地宗廟,配律陽之數,曰:「太空,育五太:太易、太初、太始、太素、太極也。分為七政,陽數七,所以齊律呂、均節度,不可加減也。以育六甲,六甲,天之使,行風雹,筴鬼神。為歲日時有善惡,故為九宮。九者,陽數變化之道也。為四正卦、五行、十乾,陰陽錯綜,律呂相葉,命宮而商者應,修下而高者降,下生隔八,上生隔六,
 皆圖於左。」



 其五、著十二管短長。



 其六、出度量衡,辯古今尺龠。律呂真聲,本陰陽之氣,可以感格天地,在於符合尺寸短長,宜因聲以定之。因聲定律,則庶幾為得;以尺定聲,則乖隔甚矣。



 初,馮元等上《新修景祐廣樂記》時,鄭保信、阮逸、胡瑗等奏造鐘律,詔翰林學士丁度、知制誥胥偃、右司諫高若訥、韓琦,取保信、逸、瑗等鐘律詳考得失。度等上議曰:「保信所制尺,用上黨秬黍圓者一黍之長,累而成尺。律管一,據尺裁九十黍之長,空徑三分,空
 圍九分,容秬黍千二百。遂用黍長為分,再累成尺,校保信尺、律不同。其龠、合、升、斗深闊,推以算法,類皆差舛,不合周、漢量法。逸、瑗所制,亦上黨秬黍中者累廣求尺,制黃鐘之律。今用再累成尺,比逸、瑗所制,又復不同。至於律管、龠、合升、斗、斛、豆、區、釜亦率類是。蓋黍有圓長、大小而保信所用者圓黍,又首尾相銜,逸等止用大者,故再考之即不同。尺既有差,故難以定鐘、磬。謹詳古今之制,自晉至隋,累黍之法,但求尺裁管,不以權量參校,
 故歷代黃鐘之管容黍之數不同。惟後周掘地得古玉斗,據斗造律,兼制權量,亦不同周、漢制度。故《漢志》有備數、和聲、審度、嘉量、權衡之說,悉起於黃鐘。今欲數器之制參互無失,則《班志》積分之法為近。逸等以大黍累尺、小黍實龠,自戾本法。保信黍尺以長為分,雖合後魏公孫崇所說,然當時已不施用,況保信今尺以圓黍累之,及首尾相銜,有與實龠之黍再累成尺不同。其量器,分寸既不合古,即權衡之法不可獨用。」詔悉罷之。



 又詔度等
 詳定太府寺並保信、逸、瑗所制尺,度等言:



 尺度之興尚矣,《周官》璧羨以起度,廣徑八寸,袤一尺。



 《禮記》布手為尺,《淮南子》十二粟為一寸,《孫子》十厘為分,十分為寸,雖存異說,其可適從。《漢志》,元始中,召天下通知鐘律者百餘人,使劉歆典領之。是時,周滅二百餘年,古之律度當有考者。以歆之博貫藝文,曉達歷算,有所制作,宜不凡近。其審度之法云:「一黍之廣為分,十分為寸,十寸為尺。」先儒訓解經籍多引以為義,歷世祖襲,著之定法。然而歲有豐儉,
 地有磽肥,就令一歲之中,一境之內,取以校驗,亦復不齊。是蓋天物之生,理難均一,古之立法,存其大概爾。故前代制尺,非特累黍,必求古雅之器以雜校焉。晉泰始十年,荀勖等校定尺度,以調鐘律,是為晉之前尺。勖等以古物七品勘之,一曰姑洗玉律,二曰小呂玉律,三曰西京銅望臬,四曰金錯望臬,五曰銅斛,六曰古錢,七曰建武銅尺。當時以勖尺揆校古器,與本銘尺寸無差,前史稱其用意精密。《隋志》所載諸代尺度,十有五等,然以
 晉之前尺為本,以其與姬周之尺、劉歆銅斛尺、建武銅尺相合。



 竊惟周、漢二代,享年永久,聖賢制作,可取則焉。而隋氏銷毀金石,典正之物,罕復存者。夫古物之有分寸,明著史籍,可以酬驗者,惟有法錢而已。周之圜法,歷載曠遠,莫得而詳。秦之半兩,實重八銖;漢初四銖,其文亦曰半兩。孝武之世始行五銖,下暨隋朝,多以五銖為號,既歷代尺度屢改,故大小輕重鮮有同者,惟劉歆置銅斛。世之所鑄錯刀並大泉五十,王莽天鳳元年改鑄
 貨布、貨泉之類,不聞後世復有兩者。臣等檢詳《漢志》、《通典》、《唐六典》云:「大泉五十,重十二銖,徑一寸二分。錯刀環如大泉,身形如刀,長二寸。貨布重二十五銖,長二寸五分,廣一寸,首長八分有奇,廣八分,足股長八分,間廣二分,圍好徑二分半。貨泉重五銖,徑一寸。」今以大泉、錯刀、貨布、貨泉四物相參校,分寸正同。或有大小輕重與本志微差者,蓋當時盜鑄既多,不必皆中法度,但當較其首足、肉好長廣、分寸,皆合正史者用之,則銅斛之尺從
 可知矣。況經籍制度皆起周世,以劉歆術業之博,祖沖之算數之妙,荀勖揆較之詳密,校之既合周尺,則最為可法。兼詳隋牛弘等議,稱後周太祖敕蘇綽造鐵尺,與宋尺同,以調中律,以均田度地。唐祖孝孫云,隋平陳之後,廢周玉尺,用此鐵尺律,然比晉前尺長六分四犛。今司天監影表尺,和峴所謂西京銅望臬者,蓋以其洛都舊物也。晉荀勖所用西京銅望臬者,蓋西漢之物,和峴謂洛陽為西京,乃唐東都爾。



 今以貨布、錯刀、貨泉、大泉等校之,則景表尺長六分有奇,略合
 宋、周、隋之尺。由此論之,銅斛、貨布等尺寸昭然可驗。有唐享國三百年,其間制作法度,雖未逮周、漢,然亦可謂治安之世矣。



 今朝廷必求尺之中,當依漢錢分寸。若以為太祖膺圖受禪,創制垂法,嘗詔和峴等用影表尺與典修金石,七十年間,薦之郊廟,稽合唐制,以示詒謀,則可且依影表舊尺,俟有妙達鐘律之學者,俾考正之,以從周、漢之制。王樸律準尺比漢錢尺寸長二分有奇,比影表尺短四分,既前代未嘗施用,復經太祖朝更易。其
 逸、瑗、保信及照所用太府寺等尺,其制彌長,出古遠甚,又逸進《周禮度量法議》,欲且鑄嘉量,然後取尺度權衡,其說疏舛,不可依用。謹考舊文,再造影表尺一、校漢錢尺二並大泉、錯刀、貨布、貨泉總十七枚上進。



 詔度等以錢尺、影表尺各造律管,比驗逸、瑗並太常新舊鐘磬,考定音之高下以聞。



 度等言:「前承詔考太常等四尺,定可用者,止按典故及以《漢志》古錢分寸參校影表尺,略合宋、周、隋之尺,謂宜準影表尺施用。今被旨造律管驗音
 高下,非素所習,乞別詔曉音者總領校定。」詔乃罷之。而若訥卒用漢貨泉度尺寸,依《隋書》定尺十五種上之,藏於太常寺:一、周尺,與《漢志》劉歆銅斛尺、後漢建武中銅尺、晉前尺同;二、晉田父玉尺,與梁法尺同,比晉前尺為一尺七犛;三、梁表尺,比晉前尺為一尺二分二犛一毫有奇;四、漢官尺,比晉前尺為一尺三分七毫;五、魏尺,杜夔之所用也,比晉前尺為一尺四分七犛;六、晉后尺,晉江東用之,比晉前尺為一尺六分三厘;七、魏前尺,比晉
 前尺為一尺一寸七厘;八、中尺,比晉前尺為一尺二寸一分一厘;九、后尺,同隋開皇尺、周氏尺,比晉前尺為一尺二寸八分一厘;十、東魏后尺,比晉前尺為一尺三寸八毫;十一、蔡邕銅龠尺,同後周玉尺,比晉前尺為一尺一寸五分八厘;十二、宋氏尺,與錢樂之渾天儀尺、後周鐵尺同。比晉前尺為一尺六分四厘;十三、太府寺鐵尺,制大樂所裁造尺也;十四、雜尺,劉曜渾儀土圭尺也,比晉前尺為一尺五分;十五、梁朝俗尺,比晉前尺為一尺
 七分一厘。太常所掌,又有後周王樸律準尺,比晉前尺長二分一厘,比梁表尺短一厘;有司天監影表尺,比晉前尺長六分三厘,同晉后尺;有中黍尺,亦制樂所新造也。



 其後宋祁、田況薦益州進士房庶曉音,祁上其《樂書補亡》三卷,召詣闕。庶自言賞得古本《漢志》,云:『度起於黃鐘之長,以子穀秬黍中者一黍之起,積一千二百黍之廣,度之九十分,黃鐘之長,一為一分。』今文脫『之起積一千二百黍』八字,故自前世以來,累黍為尺以制律,是律生
 於尺,尺非起於黃鐘也。且《漢志》『一為一分』者,蓋九十分之一,後儒誤以一黍為分,其法非是。當以秬黍中者一千二百實管中,黍盡,得九十分,為黃鐘之長,九寸加一以為尺,則律定矣。」直秘閣範鎮是之,乃為言曰:「照以縱黍累尺,管空徑三分,容黍千七百三十;瑗以橫黍累尺,管容黍一千二百,而空徑三分四厘六毫:是皆以尺生律,不合古法。今庶所言,實千二百黍於管。以為黃鐘之長,就取三分以為空徑,則無容受不合之差,校前二說
 為是。蓋累黍為尺,始失之於《隋書》,當時議者以其容受不合,棄而不用。及隋平陳,得古樂器,高祖聞而嘆曰:『華夏舊聲也!』遂傳用之。至唐祖孝孫、張文收,號稱知音,亦不能更造尺律,止沿隋之古樂,制定聲器。朝廷久以鐘律未正,屢下詔書,博訪群議,冀有所獲。今庶所言,以律生尺,誠眾論所不及,請如其法,試造尺律,更以古器參考,當得其真。」乃詔王洙與鎮同於修制所如庶說造律、尺、龠:律徑三分,圍九分,長九十分;龠徑九分,深一寸;尺
 起黃鐘之長加十分,而律容千二百黍。初,庶言太常樂高古樂五律,比律成,才下三律,以為今所用黍,非古所謂一稃二米黍也。尺比橫黍所累者,長一寸四分。



 庶又言:「古有五音,而今無正徵音。國家以火德王,徵屬火,不宜闕。今以五行旋相生法,得徵音。」又言:「《尚書》『同律、度、量、衡』,所以齊一風俗。今太常、教坊、鈞容及天下州縣,各自為律,非《書》同律之義。且古者帝王巡狩方岳,必考禮樂同異,以行誅賞。謂宜頒格律,自京師及州縣,毋容輒異,
 有擅高下者論之。」帝召輔臣觀庶所進律尺、龠,又令庶自陳其法,因問律呂旋相為宮事,令撰圖以進。其說以五正、二變配五音,迭相為主,衍之成八十四調。舊以宮、征、商、羽、角五音,次第配七聲,然後加變宮、變徵二聲,以足其數。推以旋相生之法謂五行相戾非是,當改變徵為變羽,易變為閏,隨音加之,則十二月各以其律為宮,而五行相生,終始無窮。詔以其圖送詳定所。庶又論吹律以聽軍聲者,謂以五行逆順,可以知吉兇,先儒之說
 略矣。



 是時瑗、逸制樂有定議,乃補庶試秘書省校書郎,遣之。鎮為論於執政日:



 今律之與尺所以不得其真,累黍為之也。累黍為之者,史之脫文也。古人豈以難曉不合之法,書之於史,以為後世惑乎?殆不然也。易曉而必合也,房庶之法是矣。今庶自言其法,依古以律而起尺,其長與空徑、與容受、與一千二百黍之數,無不合之差。誠如庶言,此至真之法也。



 且黃鐘之實一千二百黍,積實分八百一十,於算法圓積之,則空徑三分,圍九分,長
 九十分,積實八百一十分,此古律也。律體本圓。圓積之是也。今律方積之,則空徑三分四厘六毫,比古大矣。故圍十分三厘八毫,而其長止七十六分二厘,積實亦八百一十分。律體本不方,方積之,非也。其空徑三分,圍九分,長九十分,積實八百一十分,非外來者也,皆起於律也。以一黍而起於尺,與一千二百黍之起於律,皆取於黍。今議者獨於律則謂之索虛而求分,亦非也。其空徑三分,圍九分,長九十分之起於律,與空徑三分四厘六
 毫,圍十分三厘八毫,長七十六分二厘之起於尺,古今之法,疏密之課,其不同較然可見,何所疑哉?若以謂工作既久而復改為,則淹引歲月,計費益廣,又非朝廷制作之意也。其淹久而計費廣者,為之不敏也。今庶言太常樂無姑洗、夾鐘、太簇等數律,就令其律與其說相應,鐘磬每編才易數三,因舊而新,敏而為之,則旬月功可也,又向淹久而廣費哉?



 執政不聽。



 四年,鎮又上書曰:



 陛下制樂以事天地、宗廟,以揚祖宗之休,茲盛德之事也。
 然自下詔以來,及今三年,有司之論紛然未決,蓋由不議其本而爭其末也。竊惟樂者,和氣也。發和氣者,聲音也。聲音之生,生於無形,故古人以有形之物傳其法,俾後人參考之,然後無形之聲音得而和氣可道也。有形者,秬黍也,律也,尺也,龠也,釜也,斛也,算數也,權衡也,鐘也,磬也,是十者必相合而不相戾,然後為得,今皆相戾而不相合,則為非是矣。有形之物非是,而欲求無形之聲音和,安可得哉?謹條十者非是之驗,惟裁擇焉!



 按《詩》「
 誕降嘉種,維秬維秠。」誕降者,天降之也。許慎云:「秬,一稃二米。」又云:「一秬二米。」後漢任城縣產秬黍二斛八斗,實皆二米,史官載之,以為嘉瑞。又古人以秬黍為酒者,謂之秬鬯。宗廟降神,惟用一尊;諸侯有功,惟賜一卣,以明天降之物,世不常有而可貴也。今秬黍取之民間者,動至數百斛,秬皆一米,河東之人謂之黑米。設有真黍,以為取數至多,不敢送官,此秬黍為非是,一也。



 又按先儒皆言律空徑三分,圍九分,長九十分,容千二百黍,積實
 八百一十分。今律空徑三分四厘六毫,圍十分二厘八毫,是為九分外大其一分三厘八毫,而後容千二百黍,除其圍廣,則其長止七十六分二厘矣。說者謂四厘六毫為方分,古者以竹為律,竹形本圓,今以方分置算,此律之為非是,二也。



 又按《漢書》,分、寸、尺、丈、引本起黃鐘之長,又云九十分黃鐘之長者,據千二百黍而言也。千二百黍之施於量,則曰黃鐘之龠;施於權衡,則曰黃鐘之重;施於尺,則曰黃鐘之長。今遺千二百之數,而以
 百黍為尺,又不起於黃鐘,此尺之為非是,三也。



 又按《漢書》言龠,其狀似爵,爵謂爵□戔,其體正圓。故龠當圓徑九分,深十分,容千二百黍,積實八百一十分,與律分正同。今龠乃方一寸,深八分一厘,容千二百黍,是亦以方分置算者,此龠之非是,四也。



 又按《周禮》釜法:方尺,圓其外;深尺,容六斗四升。方尺者,八寸之尺也;深尺者,十寸之尺也。何以知尺有八寸、十寸之別?按《周禮》:「璧羨度尺,好三寸以為度。」璧羨之制,長十寸,廣八寸,同謂之度尺。以為
 尺,則八寸、十寸俱為尺矣。又《王制》云:「古者以周尺八尺為步,今以六尺四寸為步。」八尺者,八寸之尺也;六尺四寸者,十寸之尺也。同謂之周尺者,是周用八寸、十寸尺明矣。故知八寸尺為釜之方,十寸尺為釜之深,而容六斗四升,千二百八十龠也。積實一百三萬六千八百分。今釜方尺,積千寸,此釜之非是,五也。



 又按《漢書》斛法:方尺,圓其外,容十斗,旁有庣焉。當隋時,漢斛尚在,故《隋書》載其銘曰:「律嘉量斛,方尺圓其外,庣旁九厘五毫,冪
 百六十二寸,深尺,容一斛。」今斛方尺,深一尺六寸二分,此斛之非是,六也。



 又按算法,圓分謂之徑圍,方分謂之方斜,所謂「徑三、圍九、方五、斜七」是也。今圓分而以方法算之,此算數非是,七也。



 又按權衡者,起千二百黍而立法也。周之釜,其重一鈞,聲中黃鐘;漢之斛,其重二鈞,聲中黃鐘。釜、斛之制,有容受,有尺寸,又取其輕重者,欲見薄厚之法,以考其聲也。今黍之輕重未真,此權衡為非是,八也。



 又按:「鳧氏為鐘:大鐘十分,其鼓間之,以其一為
 之厚;小鐘十分,其鉦間之,以其一為之厚。」今無大小薄厚,而一以黃鐘為率,此鐘之非是,九也。



 又按:「磬氏為磬,倨句一矩有半,其博為一,股為二,鼓為三。」蓋各以其律之長短為法也。今亦以黃鐘為率,而無長短厚薄之別,此磬之非是,十也。



 前此者,皆有形之物也,可見者也。使其一不合,則未可以為法,況十者之皆相戾乎?臣固知其無形之聲音不可得而和也。請以臣章下有司,問黍之二米與一米孰是?律之空徑三分與三分四厘六毫
 孰是?律之起尺與尺之起律孰是?龠之圓制與方制孰是?釜之方尺圓其外,深尺與方尺孰是?斛之方尺圓其外,庣旁九厘五毫與方尺深尺六寸二分孰是?算數之以圓分與方分孰是?權衡之重以二米秬黍與一米孰是?鐘磬依古法有大小、輕重、長短、薄厚而中律孰是?是不是定,然後制龠、合、升、斗、釜、斛以校其容受;容受合,然後下詔以求真黍;真黍至,然後可以為量、為鐘磬;量與鐘磬合於律,然後可以為樂也。今尺律本末未定,而詳定、修制
 二局工作之費無慮千萬計矣,此議者所以雲雲也。然議者不言有司論議依違不決,而願謂作樂為過舉,又言當今宜先政令而禮樂非所急,此臣之所大惑也。儻使有司合禮樂之論,是其所是,非其所非,陛下親臨決之,顧於政令不已大乎。



 昔漢儒議鹽鐵,後世傳《鹽鐵論》。方今定雅樂以求廢墜之法,而有司論議不著盛德之事,後世將何考焉?顧令有司,人人各以經史論議條上,合為一書,則孰敢不自竭盡,以副陛下之意?如以臣議
 為然,伏請權罷詳定、修制二局,俟真黍至,然後為樂,則必得至當而無事於浮費也。



 詔送詳定所。鎮說自謂得古法,後司馬光數與之論難,以為弗合。世鮮鐘律之學,卒莫辯其是非焉。



 宋興百餘年,司天數改歷,其說曰:「歷者歲之積。歲者月之積,月者日之積,日者分之積,又推餘分置閏,以定四時,非博學妙思弗能考也。夫天體之運,星辰之動,未始有窮,而度以一法,是以久則差,差則敝而不可用,歷之
 所以數改造也。物銖銖而較之,至石必差,況於無形之數哉?」乾興初,議改歷,命司天役人張奎運算,其術以八千為日法,一千九百五十八為半分,四千二百九十九為朔,距乾興元年壬戌,歲三千九百萬六千六百五十八為積年。詔以奎補保章正。又推擇學者楚衍與歷官宋行古集天章閣,詔內侍金克隆監造歷,至天聖元年八月成,率以一萬五百九十為樞法,得九鉅萬數。既上奏,詔翰林學士晏殊制序而施行焉,命曰《崇天歷》。歷法
 曰演紀上元甲子,距天聖二年甲子,歲積九千七百五十五萬六千三百四十。上考往古,歲減一算;下驗將來,歲加一算。



 步氣朔



 《崇天》樞法:一萬五百九十。



 歲周:三百八十六萬七千九百四十。



 歲餘:五萬五千五百四十。



 氣策:一十五、餘五千三百一十四、秒六。



 朔實:三十一萬二千七百二十九。



 歲閏:一十一萬五千一百九十二。



 朔策:二十九、餘五千六百一十九。



 望策:一十四、餘八千一百四、秒一十八。



 弦策:七、餘四千五十二、秒九。



 中盈分:四千六百二十八、秒一十二。



 朔虛分:四千九百七十一。



 閏限:三十萬三千一百二十九、秒二十四。



 秒法:三十六。



 旬周:六十三萬五千四百。



 紀法:六十。



 推天正冬至:置距所求積年,以歲周乘之,為氣積分;滿旬周去之,不盡,以樞法約之為大餘,不滿為小餘。大餘命甲子,算外,即所求年天正冬至日辰及餘。若以後合用約分,即以樞法退除為分秒,各以一百為母。



 求次氣:置天正冬至大、小餘,以氣策秒累加之,秒盈秒法從小餘,小餘滿樞法從大餘,滿紀法去之,不盡,命甲子,算外,即各得次氣日辰及餘秒。



 推天正十一月經朔:置天正冬至氣積分,朔實去之,不盡為閏餘;以減天正冬至氣積分,為天正十一月經朔加時積分;滿旬周去之,不盡,以樞法約之為大餘,不滿為小餘。大餘命甲子,算外,即所求年天正十一月經朔日辰及餘。



 求弦望及次朔經日:置天正十一月經朔大、小餘,以弦策累加之,去命如前,即各弦、望及次朔經日及餘秒。



 求沒日:置有沒之氣小餘,三百六十乘之,其秒進一位,從
 之,用減歲周,餘滿歲餘為日,不滿為餘。命其氣初日,算外,即其氣沒日日辰。凡二十四氣小餘滿八千二百六十五、秒三十以上為有沒之氣。



 求減日:置有減經朔小餘,三十乘之,滿朔虛分為日,不滿為餘。命經朔初日,算外,即為其朔減日日辰。凡經朔小餘不滿朔虛分為有減之朔。



 步發斂



 候策:五、餘七百七十一、秒一十四。



 卦策:六、餘九百二十五、秒二十四。



 土王策:三、餘四百六十二、秒三十。



 辰法:八百八十二半。



 刻法:一千五十九。



 秒法:三十六。



 推七十二候:各因中節大、小餘命之,為其氣初候日也;以候策加之,為次候;又加之,為末候。



 求六十四卦:各因中氣大、小餘命之,為公卦用事日;以卦策加之,得次卦用事日;以土王策加諸侯之卦,得十
 有二節之初外卦用事之日。



 推五行用事日:各因四立日大、小餘命之,即春木、夏火、秋金、冬水首用事日;以土王策減四季中氣大、小餘,命甲子,算外,即其月土始用事日。



 七十二候及卦日與《應天》同。



 求發斂去經朔:置天正十一月閏餘,以中盈及朔虛分累益之,即每月閏餘;滿樞法除之為閏日,不盡為小餘,即各得其月中氣去經朔日及餘秒。其餘閏滿閏限至閏,仍先見定朔大
 小。其月內無中氣,乃為閏月。



 求卦候去經朔:各以卦、候策及餘秒累加減之,中氣前以減,中氣後以加。即各得卦、候去經朔日及餘秒。



 求發斂加時:置小餘,以辰法除之為辰數,進一位,滿刻法為刻,不滿為刻分。其辰數命子正,算外,即各加時所在辰、刻及分。



\end{pinyinscope}