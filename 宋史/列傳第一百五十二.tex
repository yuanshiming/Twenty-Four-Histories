\article{列傳第一百五十二}

\begin{pinyinscope}

 ○彭龜
 年黃裳羅點黃度周南附林大中陳騤黃黼詹體仁



 彭龜年,字子壽,臨江軍清江人。七歲而孤,事母盡孝。性穎異,讀書能解大義。及長,得程氏《易》讀之,至忘寢食,從
 朱熹、張栻質疑,而學益明。登乾道五年進士第,授袁州且春尉、吉州安福丞。鄭僑、張枃同薦,除太學博士。



 殿中侍御史劉光祖以論帶御器械吳端,徙太府少卿,龜年上疏乞復其位,貽書宰相云:「祖宗嘗改易差除以伸臺諫之氣,不聞改易臺諫以伸幸臣之私。」兼魏王府教授,遷國子監丞。以侍御史林大中薦,為御史臺主簿。改司農寺丞,進秘書郎兼嘉王府直講。



 光宗嘗親郊,值暴風雨感疾,大臣希得進見。久之,疾平,猶疑畏不朝重華宮。
 龜年以書譙趙汝愚,且上疏言:「壽皇之事高宗,備極子道,此陛下所親睹也。況壽皇今日止有陛下一人,聖心拳拳,不言可知。特遇過宮日分,陛下或遲其行,則壽皇不容不降免到宮之旨,蓋為陛下辭責於人,使人不得以竊議陛下,其心非不願陛下之來。自古人君處骨肉之間,多不與外臣謀,而與小人謀之,所以交鬥日深,疑隙日大。今日兩宮萬萬無此。然臣所憂者,外無韓琦、富弼、呂誨、司馬光之臣,而小人之中,已有任守忠者在焉,
 惟陛下裁察。」



 又言:「使陛下虧過宮定省之禮,皆左右小人間諜之罪。宰執侍從但能推父子之愛,調停重華;臺諫但能仗父子之義,責望人生。至於疑間之根,盤固不去,曾無一語及之。今內侍間諜兩宮者固非一人,獨陳源在壽皇朝得罪至重,近復進用,外人皆謂離間之機必自源始。宜亟發威斷,首逐陳源,然後肅命鑾輿,負罪引慝,以謝壽皇,使父子歡然,宗社有永,顧不幸歟?」居亡何,光宗朝重華,都人歡悅。尋除起居舍人,入謝,光宗曰:「
 此官以待有學識人,念非卿無可者。」



 龜年述祖宗之法為《內治聖鑒》以進。光宗曰:「祖宗家法甚善。」龜年曰:「臣是書大抵為宦官、女謁之防,此曹若見,恐不得數經御覽。」光宗曰:「不至是。」他日,龜年奏:「臣所居之官,以記注人君言動為職,車駕不過宮問安,如此書者又數十矣,恐非所以示後。」有旨幸玉津園,龜年奏:「不奉三宮,而獨出宴游,非禮也。」又言:「陛下誤以臣充嘉王府講讀官,正欲臣等教以君臣父子之道。臣聞有身教,有言教,陛下以身
 教,臣以言教者也,言豈若身之切哉。」



 紹熙五年五月,壽皇不豫,疾浸革,龜年連三疏請對,不獲命。屬上視朝,龜年不離班位,伏地扣額久不已,血漬鶖甓。光宗曰:「素知卿忠直,欲何言?」龜年奏:「今日無大於不過宮。」光宗曰:「須用去。」龜年言:「陛下屢許臣,一入宮則又不然。內外不通,臣實痛心。」同知樞密院餘端禮曰:「扣額龍墀,曲致忠懇,臣子至此,為得已邪?」上云:「知之。」



 孝宗崩,寧宗受禪,是夕召對,寧宗蹙額云:「前但聞建儲之義,豈知遽踐大位,泣
 辭不獲,至今震悸。」龜年奏:「此乃宗祏所系,陛下安得辭,今日但當盡人子事親之誠而已。」因擬起居札子,乞日進一通。又與翊善黃裳同奏往朝南內,因定過宮之禮,乞先一日入奏,率百官恭謝。寧宗朝泰安宮,至則寢門已閉,拜表而退。



 時議欲別建泰安宮,而光宗無徙宮之意。龜年言:「古人披荊棘立朝廷,尚可布政出令,況重華一宮豈為不足哉?陛下居狹處,太上居寬處,天下之人必有諒陛下之心者。」於是宮不果建。遷中書舍人。劉慶
 祖已帶遙郡承宣使,而以太上隨龍人落階官,龜年繳奏,寧宗批:「可與書行。」龜年奏:「臣非為慶祖惜此一官,為朝廷惜此一門耳。夫『可與書行』,近世弊令也,使其可行,臣即書矣,使不可行,豈敢因再令而遂書哉?」寧宗嘗謂:「退朝無事,恐自怠惰,非多讀書不可。」龜年奏:「人君之學與書生異,惟能虛心受諫,遷善改過,乃聖學中第一事,豈在多哉!」



 一日,御筆書朱熹、黃裳、陳傅良、彭龜年、黃由、沈有開、李巘、京鏜、黃艾、鄧馹十人姓名示龜年云:「十人
 可充講官否?」龜年對曰:「陛下若招來一世之傑如朱熹輩,方厭人望,不可專以潛邸學官為之。」尋除侍講,遷吏部侍郎,升兼侍讀。龜年知事勢將變,會暴雨震雷,因極陳小人竊權、號令不時之弊。遣充金國吊祭接送伴使。



 初,朱熹與龜年約共論韓侂胄之奸,會龜年護客,熹以上疏見絀,龜年聞之,附奏云:「始臣約熹同論此事。今熹既罷,臣宜並斥。」不報。迨歸,見侂胄用事,權勢重於宰相,於是條數其奸,謂:「進退大臣,更易言官,皆初政最關大
 體者。大臣或不能知,而侂胄知之,假托聲勢,竊弄威福,不去必為後患。」上覽奏甚駭,曰:「侂胄朕之肺腑,信而不疑,不謂如此。」批下中書,予侂胄祠,已乃復入。



 龜年上疏求去,詔侂胄與內祠,龜年與郡,以煥章閣待制知江陵府、湖北安撫使。龜年丐祠,慶元二年,以呂棐言落職;已而追三官,勒停。嘉泰元年,復元官。起知贛州,以疾辭,除集英殿修撰、提舉沖祐觀。開禧二年,以待制寶謨閣致仕,卒。



 龜年學識正大,議論簡直,善惡是非,辨析甚嚴,其
 愛君憂國之忱,先見之識,敢言之氣,皆人所難。晚既投閑,悠然自得,幾微不見於顏面。自偽學有禁,士大夫鮮不變者,龜年於關、洛書益加涵泳,扁所居曰止堂,著《止堂訓蒙》,蓋始終特立者也。聞蘇師旦建節,曰:「此韓氏之陽虎,其禍韓氏必矣。」及聞用兵,曰:「禍其在此乎?」所著書有《經解》、《祭儀》、《五致錄》、奏議、外制。



 侂胄誅,林大中、樓錀皆白其忠,寧宗詔贈寶謨閣直學士。章穎等請易名,賜謚忠肅。上謂穎等曰:「彭龜年忠鯁可嘉,宜得謚。使人人如
 此,必能納君於無過之地。」未幾,加贈龍圖閣學士,而擢用其子欽。



 黃裳,字文叔,隆慶府普成人。少穎異,能屬文。登乾道五年進士第,調巴州通江尉。益務進學,文詞迥出流輩,人見之曰:「非復前日文叔矣。」



 時蜀中餉師,名為和糴,實則取民。裳賦《漢中行》,諷總領李蘩,蘩為罷糴,民便之。改興元府錄事參軍。以四川制置使留正薦,召對,論蜀兵民大計。遷國子博士,以母喪去。宰相進擬他官,上問裳安
 在,賜錢七十萬。除喪,復召。



 時光宗登極,裳進對,謂:「中興規模與守成不同,出攻入守,當據利便之勢,不可不定行都。富國強兵,當求功利之實,不可不課吏治。捍內御外,當有緩急之備,不可不立重鎮。」其論行都,以為就便利之勢,莫若建康。其論吏治,謂立品式以課其功,計資考以久其任。其論重鎮,謂自吳至蜀,綿亙萬里,曰漢中,曰襄陽,曰江陵,曰鄂渚,曰京口,當為五鎮,以將相大臣守之,五鎮強則國體重矣。除太學博士,進秘書郎。



 遷嘉
 王府翊善,講《春秋》「王正月」曰:「周之王,即今之帝也。王不能號令諸侯,則王不足為王;帝不能統御郡鎮,則帝不足為帝。今之郡縣,即古諸侯也。周之王惟不能號令諸侯,故《春秋》必書『王正月』,所以一諸侯之正朔。今天下境土,比祖宗時不能十之四,然猶跨吳、蜀、荊、廣、閩、越二百州,任吾民者,二百州守也,任吾兵者,九都統也,茍不能統御,則何以服之?」王曰:「何謂九都統?」裳曰:「唐太宗年十八起義兵,平禍亂。今大王年過之,而國家九都統之說
 猶有未知,其可不汲汲於學乎?」



 他日,王擢用東宮舊人吳端,端詣王謝,王接之中節。裳因講《左氏》「禮有等衰」,問王:「比待吳端得重輕之節,有之乎?」王曰:「有之。」裳曰:「王者之學,正當見諸行事。今王臨事有區別,是得等衰之義矣。」王意益向學。於是作八圖以獻:曰太極,曰三才本性,曰皇帝王伯學術,曰九流學術,曰天文,曰地理,曰帝王紹運,以百官終焉,各述大旨陳之。每進言曰:「為學之道,當體之以心。王宜以心為嚴師,於心有一毫不安者,不
 可為也。」且引前代危亡之事以為儆戒。王謂人曰:「黃翊善之言,人所難堪,惟我能受之。」他日,王過重華宮,壽皇問所讀書,王舉以對,壽皇曰:「數不太多乎?」王曰:「講官訓說明白,忱心樂之,不知其多也。」壽皇曰:「黃翊善至誠,所講須諦聽之。」



 裳久侍王邸,每歲誕節,則陳詩以寓諷。初嘗制渾天儀、輿地圖,侑以詩章,欲王觀象則知進學,如天運之不息,披圖則思祖宗境土半陷於異域而未歸。其後又以王所講三經為詩三章以進。王喜,為置酒,手
 書其詩以賜之。王嘗侍宴宮中,從容為光宗誦《酒誥》,曰:「此黃翊善所教也。」光宗詔勞裳,裳曰:「臣不及朱熹,熹學問四十年,若召置府寮,宜有裨益。」光宗嘉納。裳每勸講,必援古證今,即事明理,凡可以開導王心者,無不言也。



 紹熙二年,遷起居舍人。奏曰:「自古人君不能從諫者,其蔽有三:一曰私心,二曰勝心,三曰忿心。事茍不出於公,而以己見執之,謂之私心;私心生,則以諫者為病,而求以勝之;勝心生,則以諫者為仇,而求以逐之。因私而
 生勝,因勝而生忿,忿心生,則事有不得其理者焉。如潘景珪,常才也,陛下固亦以常人遇之,特以臺諫攻之不已,致陛下庇之愈力,事勢相激,乃至於此。宜因事靜察,使心無所系,則聞臺諫之言無不悅,而無欲勝之心,待臺諫之心無不誠,而無加忿之意矣。」



 三年,試中書舍人。時武備寢弛,裳上疏曰:「壽皇在位三十年,拊循將士,士常恨不得效死以報。陛下誠能留意武事,三軍之士孰不感激願為陛下用乎?」又論:「荊、襄形勢居吳、蜀之中,其地
 四平,若金人搗襄陽,據江陵,按兵以守,則吳、蜀中斷,此今日邊備之最可憂也。宜分鄂渚兵一二萬人屯襄、漢之間,以張形勢而壯重地。」時朝廷方宴安,裳所言多不省。



 未幾,除給事中。趙汝愚除同知樞密院,監察御史汪義端言祖宗之法,宗室不為執政,再疏醜詆汝愚,汝愚乞免官。裳奏:「汝愚事父孝,事君忠,居官廉。憂國愛民,出於天性,如青天白日,奴隸知其清明。義端所見,皆奴隸之不如,不可以居朝列。」於是義端與郡。



 裳在瑣闥甫一
 月,封駁無慮十數。韓侂胄落階官,鄭汝諧除吏部侍郎,裳皆繳其命。改兵部侍郎,不拜,遂以顯謨閣待制充翊善。先是,光宗以憂疑成疾,不過重華宮,裳入疏請五日一朝,至是復苦言之。上曰:「內侍楊舜卿告朕勿過宮。」裳請斬舜卿,且以八事之目為奏,曰念恩,釋怨,辨讒,去疑,責己,畏天,防亂,改過。不報。



 裳嘗病疽,及是憂憤,創復作,又奏:



 陛下之於壽皇,未盡孝敬之道,意者必有所疑也。臣竊推致疑之因,陛下毋乃以焚廩、浚井之事為憂乎?夫
 焚廩、浚井,在當時或有之。壽皇之子惟陛下一人,壽皇之心,托陛下甚重,愛陛下甚至,故憂陛下甚切。違豫之際,焫香祝天,為陛下祈禱。愛子如此,則焚廩、浚井之心,臣有以知其必無也,陛下何疑焉?又無乃以肅宗之事為憂乎?肅宗即位靈武,非明皇意,故不能無疑。壽皇當未倦勤,親挈神器授之陛下,揖遜之風,同符堯、舜,與明皇之事不可同日而語明矣,陛下何疑焉?又無乃以衛輒之事為憂乎?輒與蒯聵,父子爭國。壽皇老且病,乃頤
 神北宮,以保康寧,而以天下事付之陛下,非有爭心也,陛下何疑焉?又無乃以孟子責善為疑乎?父子責善,本生於愛,為子者能知此理,則何至於相夷。壽皇願陛下為聖帝,責善之心出於忠愛,非賊恩也,陛下何疑焉?



 此四者,或者之所以為疑,臣以理推之,初無一之可疑者。自父子之間,小有猜疑,此心一萌,方寸遂亂。故天變則疑而不知畏,民困則疑而不知恤,疑宰執專權則不禮大臣,疑臺諫生事則不受忠諫,疑嗜欲無害則近酒色,
 疑君子有黨則庇小人。事有不須疑者,莫不以為疑。乃若貴為天子,不以孝聞,敵國聞之,將肆輕侮,此可疑也,而陛下則不疑;小人將起為亂,此可疑也,而陛下則不疑;中外官軍,豈無他志,此可疑也,而陛下則不疑。事之可疑者,反不以為疑,顛倒錯亂,莫甚於此,禍亂之萌,近在旦夕。宜及今幡然改過,整聖駕,謁兩宮,以交父子之歡,則四夷向風,天下慕義矣。



 會壽皇不豫,中外憂危,裳抗聲諫。上起入宮,裳挽其裾隨之至宮門,揮涕而出。乃
 連章請外,謂:「臣職有三:曰待制,曰侍講,曰翊善。今使供待制之職乎?則當日夕求對以救主失,今不過宮,有虧子道,前後三諫而不加聽,是待制之職可廢也。將使供侍講之職乎?則當引經援古,勸君以孝,今不問安,不視疾,大義已喪,復講何書乎?是侍講之職可廢也。將使供翊善之職乎?當究義理,教皇子以孝,陛下不能以孝事壽皇,臣將何說以勸皇子乎?是翊善之職可廢也。」因出關待命。及聞壽皇遺詔,乃亟入臨。



 寧宗即位,裳病不能
 朝。改禮部尚書,尋兼侍讀。力疾入謝,奏曰:



 孔子曰:「有始有卒者,其惟聖人乎?」又《詩》曰:「靡不有初,鮮克有終。」所謂「有始有卒」者,由其持心之一也;所謂「鮮克有終」者,由其持心之不一也。陛下今日初政固善矣,能保他日常如此乎?請略舉已行之事論之。



 陛下初理萬機,委任大臣,此正得人君持要之道。使大臣得人,常如今日,則陛下雖終身守之可也。臣恐數年之後,亦欲出意作為,躬親聽斷,左右迎合,因謂陛下事決外庭,權不歸上,陛下能
 不咈然於心乎?臣恐是時委任大臣,不能如今日之專矣。夫以萬機之眾,非一人所能酬酢,茍不委任大臣,則必借助左右,小人得志,陰竊主權,引用邪黨,其為禍患,何所不至,臣之所憂者一也。



 陛下獎用臺諫,言無不聽,此正得祖宗設官之意。使臺諫得人,常如今日,則陛下終身守之亦可也。然臣恐自今以往,臺諫之言日關聖聽,或斥小人之過,使陛下欲用之而不能,或暴近習之罪,使陛下欲親之而不可。逆耳之言,不能無厭,左右迎
 合,因謂陛下獎用臺諫,欲聞讜論,而其流弊,致使人主不能自由,陛下能不咈然於心乎?臣恐是時獎用臺諫,不能如今日之重矣。夫朝廷所恃以分別善惡者,專在臺諫,陛下茍厭其多言,則為臺諫者,將咋舌閉口,無所論列。君子日退,小人日進,而天下亂矣,臣之所憂者二也。



 二事,朝廷之大者。又以三事之切於陛下之身言之:曰篤於孝愛,勤於學問,薄於嗜好。陛下今皆行之矣,未知數年之後,能保常如今日乎?



 又引魏征十漸以為戒,
 懇懇數千言。又奏言:「陛下近日所為頗異前日,除授之際,大臣多有不知,臣聞之憂甚而病劇。」蓋是時韓侂胄已潛弄威柄,而宰相趙汝愚未之覺,故裳先事言之。及疾革,時時獨語,曰:「五年之功,無使一日壞之,度吾已不可為,後之君子必有能任其責者。」遂口占遺表而卒,年四十九。上聞之驚悼,贈資政殿學士。



 裳為人簡易端純,每講讀,隨事納忠,上援古義,下揆人情,氣平而辭切,事該而理盡。篤於孝友,與人言傾盡底蘊。恥一書不讀,一
 物不知。推賢樂善,出乎天性。所為文,明白條達。有《王府春秋講義》及《兼山集》,論天人之理,性命之源,皆足以發明伊、洛之旨。嘗與其鄉人陳平父兄弟講學,平父,張栻之門人也,師友淵源,蓋有自來云。嘉定中,謚忠文。子瑾,大宗正丞兼刑部郎官。孫子敏,刑部郎官。



 羅點字春伯,撫州崇仁人。六歲能文。登游熙三年進士第,授定江節度推官。累遷校書郎兼國史院編修官。歲旱,詔求言,點上封事,謂:「今時奸諛日甚,議論凡陋。無所
 可否,則曰得體;與世浮沈,則曰有量;眾皆默,己獨言,則曰沽名;眾皆濁,己獨清,則曰立異。此風不革,陛下雖欲大有為於天下,未見其可也。自旱嘆為虐,陛下禱群祠,赦有罪,曾不足以感動。及朝求讜言,夕得甘雨,天心所示,昭然不誣。獨不知陛下之求言,果欲用之否乎?如欲用之,則願以所上封事,反覆詳熟,當者審而後行,疑者咨而後決,如此則治象日著,而亂萌自消矣。」遷秘書郎兼皇太子宮小學教授。



 寧宗時以皇孫封英國公,點兼
 教授,入講至晡時不輟,左右請少憩,點曰:「國公務學不休,奈何止之。」又摭古事勸戒,為《鑒古錄》以進。高宗崩,孝宗在諒暗,皇太子參決庶務,點時以戶部員外郎兼太子侍講,出使浙右,遷起居舍人,改太常少卿兼侍立修注官,被命使金告登寶位。會金有國喪,迫點易金帶,點曰:「登位吉事也,必以吉服從事。有死而已,帶不可易。」又詰點不當稱「寶位」,點曰:「聖人大寶曰位,不加『寶」字,何以別至尊。」金人不能奪。



 上嘗謂點:「卿舊為宮僚,非他人比,
 有所欲言,毋憚啟告。」點言:「君子得志常少,小人得志常多。蓋君子志在天下國家,而不在一己,行必直道,言必正論,往往不忤人主,則忤貴近,不忤當路,則忤時俗。小人志在一己,而不在天下國家,所行所言,皆取悅之道。用其所以取忤者,其得志鮮矣;用其所以取悅者,其不得志亦鮮矣。若昔明主,念君子之難進,則極所以主張而覆護之;念小人之難退,則盡所以燭察而堤防之。」



 皇子嘉王年及弱冠,點言:「此正親師友、進德業之時,宜擇
 端良忠直之士,參侍燕間。」遂除黃裳為翊善。又言:「人主憂勤,則臣下協心;人主偷安,則臣下解體。今道塗之言,皆謂陛下每旦視朝,勉強聽斷,意不在事。宰執奏陳,備禮應答,侍從庶僚,備禮登對,而宮中燕游之樂,錫齎奢侈之費,已騰於眾口。強敵對境,此聲豈可出哉!」



 紹熙三年十一月日長至,車駕將朝賀重華宮,既而中輟。點言:「自天子達庶人,節序拜親,無有闕者,三綱五常,所系甚大,不當以為常事而忽之。」上過宮意未決,點奏:「陛下已
 涓日過宮,壽皇必引領以俟陛下。常人於朋友且不可以無信,況人主之事親乎?今陛下久闕溫凊,壽皇欲見不可得,萬一憂思感疾,陛下將何以自解於天下?」



 嘗召對便殿,點言:「近者中外相傳,或謂陛下內有所制,不能遽出,溺於酒色,不恤政事,果有之乎?」上曰:「無是。」點曰:「臣固知之。竊意宮禁間或有攖拂之事,姑以酒自遣耳。夫閭閻匹夫,處閨門逆境,容有縱酒自放者。人主宰制天下,此心如青天白日,當風雨雷電既霽之餘,湛然虛明,
 豈容復有纖芥停留哉?」上猶未過宮。點又奏:「竊聞嘉王生朝,稱壽禁中,以報劬勞之德,父子歡洽,寧不動心,上念兩宮延望之意。」十一月,點以言不見聽,求去,不許。十二月,試兵部尚書。



 五年四月,上將幸玉津園,點請先過重華,又奏曰:「陛下為壽皇子,四十餘年一無閑言,止緣初郊違豫,壽皇嘗至南內督過,左右之人自此讒間,遂生憂疑。以臣觀之,壽皇與天下相忘久矣。今大臣同心輔政,百執事奉法循理,宗室、戚里、三軍、萬姓皆無貳志,
 設有離間,誅之不疑。乃若深居不出,久虧子道,眾口謗讟,禍患將作,不可以不慮。」上曰:「卿等可為朕調護之。」黃裳對曰:「父子之親,何俟調護。」點曰:「陛下一出,即當釋然。」上猶未行。點乃率講官言之,上曰:「朕心未嘗不思壽皇。」對曰:「陛下久闕定省,雖有此心,何以自白乎?」及壽皇不豫,點又隨宰執班進諫。閣門吏止之,點叱之而入。上拂衣起,宰執引上裾,點亟前泣奏曰:「壽皇疾勢已危,不及今一見,後悔何及。」群臣隨上入至福寧殿,內侍闔門,眾
 慟哭而退。越三日,點隨宰執班起居,詔獨引點入。點奏:「前日迫切獻忠,舉措失禮,陛下赦而不誅,然引裾亦故事也。」上曰:「引裾可也,何得輒入宮禁乎?」點引辛毗事以謝,且言:「壽皇止有一子,既付神器,惟恐見之不速耳。」



 壽皇崩,點請上奔喪,許而不出,拜遺詔於重華宮。前後與侍從列奏諫請帝過宮者凡三十五疏,自上奏者又十六章,而奏疏重華,上書嘉王及面對口奏不預焉。寧宗嗣位,人心始定。拜點端明殿學士、簽書樞密院事。上有
 事明堂,點扈從齋宮,得疾卒,年四十五。贈太保,謚文恭。



 點天性孝友,無矯激崖異之行,而端介有守,義利之辨皎如。或謂天下事非才不辦,點曰:「當先論其心,心茍不正,才雖過人,果何取哉!」宰相趙汝愚嘗泣謂寧宗曰:「黃裳、羅點相繼淪謝,二臣不幸,天下之不幸也。」



 黃度,字文叔,紹興新昌人。好學讀書,秘書郎張淵見其文,謂似曾恐。隆興元年進士,知嘉興縣。入監登聞鼓院,行國子監簿。言:「今日養兵為巨患,救患之策,宜使民屯
 田,陰復府衛以銷募兵。」具《屯田》、《府衛》十六篇上之。



 紹熙四年,守監察御史。蜀將吳挺死,度言:「挺子曦必納賂求襲位,若因而授之,恐為他日患,乞分其兵柄。」宰相難之。後曦割關外四州賂金人求王蜀,果如度言。



 光宗以疾不過重華宮,度上書切諫,連疏極陳父子相親之義,且言:「太白晝見犯天關,熒惑、勾芒行入太微,其占為亂兵入宮。」以諫不聽,乞罷去。又言:「以孝事君則忠。臣父年垂八十,菽水不親,動經歲月,事親如此,何以為事君之
 忠。」蓋托已為諭,冀因有以感悟上心。



 又與臺諫官劾內侍陳源、楊舜卿、林億年三人為今日禍根,罪大於李輔國。又言:「孔子稱『天下有道,則庶人不議。』夫人主有過,公卿大夫諫而改,則過不彰,庶人奚議焉。惟諫而不改,失不可蓋,使閭巷小人皆得妄議,紛然亂生,故勝、廣、黃巢之流議於下,國皆隨以亡。今天下無不議聖德者,臣竊危之。」上猶不聽。遂出修門,上諭使安職。度奏:「有言責者,不得其言則去,理難復入。」寧宗即位,詔復為御史,改右正
 言。



 韓侂胄用事,丞相留正去國,侂胄知度嘗與正論事不合,欲諷使擠之。度語同列曰:「丞相已去,擠之易耳,然長小人聲焰可乎?」侂胄驟竊政柄,以意所好惡為威福。度具疏將論其奸,為侂胄所覺,御筆遽除度直顯謨閣、知平江府。度言:「蔡京擅權,天下所由以亂。今侂胄假御筆逐臣,使俯首去,不得效一言,非為國之利也。」固辭。丞相趙汝愚袖其疏入白,詔以沖祐祿歸養。俄知婺州,坐不發覺縣令張元弼臟罪,降罷。自是紀綱一變,大權
 盡出侂胄,而黨論起矣。然侂胄素嚴憚度,不敢加害。起知泉州,辭,乃進寶文閣,奉祠如故。



 侂胄誅,天子思而召之,除太常少卿,尋兼國史院編修官、實錄院檢討官。朝論欲函侂胄首以泗州五千人還金,度以為辱國非之。權吏部侍郎兼修玉牒、同修國史、實錄院同修撰,屢移疾,以集英殿修撰知福州,遷寶謨閣待制。始至,訟牒日千餘,度隨事裁決,日未中而畢。



 進龍圖閣,知建康府兼江、淮制置使,賜金帶以行。至金陵,罷科糴輸送之擾,活
 饑民百萬口,除見稅二十餘萬,擊降盜卞整,斬盜胡海首以獻,招歸業者九萬家。侂胄嘗募雄淮軍,已收刺者十餘萬人,別屯數千人未有所屬,度憂其為患,人給錢四萬,復其役遣之。



 遷寶謨閣直學士。度以人物為己任,推挽不休,每曰:「無以報國,惟有此耳。」十上引年之請,不許,為禮部尚書兼侍讀。趣入覲,論藝祖垂萬世之統,一曰純用儒生,二曰務惜民力。上納其言。謝病丐去,遂以煥章閣學士知隆興府。歸越,提舉萬壽宮。嘉定六年十
 月卒,進龍圖閣學士,贈通奉大夫。



 度志在經世,而以學為本。作《詩》、《書》、《周禮說》。著《史通》,抑僭竊,存大分,別為編年,不用前史法。至於天文、地理、井田、兵法,即近驗遠,可以據依,無迂陋牽合之病。又有《藝祖憲監》、《仁皇從諫錄》、《屯田便宜》、《歷代邊防》行於世。婿周南。



 周南字南仲,平江人。年十六,游學吳下,視時人業科舉,心陋之。從葉適講學,頓悟捷得。為文詞,雅麗精切,而皆達於時用,每以世道興廢為己任。登紹熙元年進士第,為池州教授。會度以
 言忤當路,御史劾度,並南罷之。度與南俱入偽學黨。開禧三年,召試館職。南對策詆權要,言者劾南,罷之,卒於家。



 南端行拱立,尺寸有程準。自賜第授文林郎,終身不進官,兩為館職,數月止。既絕意當世,弊衣惡食,挾書忘晝夜,曰:「此所以遺吾老,俟吾死也。」



 林大中,字和叔,婺州永康人。入太學,登紹興三十年進士第,知撫州金谿縣。郡督輸賦急,大中請寬其期,不聽,納告敕投劾而歸。已而主太常寺簿。



 光宗受禪,除監察
 御史。大中謂:「國之大事在祀,沿襲不正,非所以嚴典禮,妥神明。」上疏言:「臣昨簿正奉常,實陪廟祀,見其祝於神者,或舛於文;稱於神者,或訛其字;所宜厚者,或簡不虔;所宜先者,或廢不用;更制器服,或歲月太疏;夙興行事,或時刻太早:是皆禮意所未順,人情所未安也。」一日,御札示大中,謂言事覺察,宜遵舊例。大中曰:「臺臣不當逾分守,固如聖訓,然必抗直敢言,乃為稱職。」



 遷殿中侍御史。奏言:「進退人才,當觀其趣向之大體,不當責其行事
 之小節。趣向果正,雖小節可責,不失為君子;趣向不正,雖小節可喜,不失為小人。」又論:「今日之事,莫大於仇恥未復。此事未就,則此念不可忘。此念存於心,於以來天下之才,作天下之氣,倡天下之義。此義既明,則事之條目可得而言,治功可得而成矣。」陳賈以靜江守臣入奏,大中極論其「庸回亡識,嘗表裹王淮,創為道學之目,陰廢正人。儻許入奏,必再留中,善類聞之,紛然引去,非所以靖國。」命遂寢。



 紹熙二年春,雷電交作,有旨訪時政
 闕失。大中以事多中出,乃上疏曰:「仲春雷電,大雪繼作,以類求之,則陰勝陽之明驗也。蓋男為陽,而女為陰,君子為陽,而小人為陰。當辨邪正,毋使小人得以間君子。當思正始之道,毋使女謁之得行。」



 司諫鄧馹以言事移將作監,大中言:「臺諫以論事不合而遷,臣恐天下以陛下為不能容。」守侍御史兼侍講。知潭州趙善俊得旨奏事,大中上疏劾善俊,而言宗室汝愚之賢當召。上用其言,召汝愚而出善俊與郡。



 時江、淮、荊、襄為國巨屏,而權
 任頗輕。大中言:「宜選行實材略之人,付以江、淮、荊、襄經理之任。舊制河北、陜西分為四路,以文臣為大帥,武臣副之。中興初,沿江置制置使。自秦檜罷三大將兵權,專歸武臣,而江東、荊、襄帥臣不復領制置之職。宜仍舊制置,而以諸將為副,久其任,重其權,則邊防立而國勢張矣。」



 江、浙四路民苦折帛和買重輸,大中曰:「有產則有稅,於稅絹而科折帛,猶可言也,如和買折帛則重為民害。蓋自咸平馬元方建言於春預支本錢濟其乏絕,至夏
 秋使之輸納,則是先支錢而後輸絹。其後則錢鹽分給,又其後則直取於民,今又令納折帛錢,以兩縑折一縑之直,大失立法初意。」朝廷以其言為減所輸者三歲。



 馬大同為戶部,大中劾其用法峻。上欲易置他部,大中曰:「是嘗為刑部,固以深刻稱。」章三上不報。又論大理少卿宋之瑞,章四上,又不報。大中以言不行,求去,改吏部侍郎,辭不拜,乃除大中直寶謨閣,而大同、之瑞俱與郡。



 初,占星者謂朱熹曰:「某星示變,正人當之,其在林和叔耶?」
 至是,熹貽書朝士曰:「聞林和叔入臺,無一事不中的,去國一節,風義凜然,當於古人中求之。」給事中尤袤、中書舍人樓鑰上疏云:「大中言官,當與被論者有別。」尋命知寧國府,又移贛州。寧宗即位,召還,試中書舍人,遷給事中,尋兼侍講。知閣門事韓侂胄來謁,大中救之,無他語,陰請內交,大中笑而卻之,侂胄怨由此始。



 會吏部侍郎彭龜年抗論侂胄,侂胄轉一官與內祠,龜年除煥章閣待制與郡。大中同中書舍人樓鑰繳奏曰:「陛下眷禮僚
 舊,一旦龍飛,延問無虛日。不三數月間,或死或斥,賴龜年一人尚留,今又去之,四方謂其以盡言得罪,恐傷政體。且一去一留,恩意不侔。去者日遠,不復侍左右。留者內祠,則召見無時。請留龜年經筵,而命侂胄以外任,則事體適平,人無可言者。」有旨:「龜年已為優異,侂胄本無過尤,可並書行。」大中復同奏:「龜年除職與郡以為優異,則侂胄之轉承宣使非優異乎?若謂侂胄本無過尤,則龜年論事實出於愛君之忱,豈得為過?龜年既以決去,
 侂胄難於獨留,宜畀外任或外祠,以慰公議。」不聽。



 太府寺丞呂祖儉以上書攻侂胄,謫置韶州,大中捄之。汪義端頃為御史,以論趙汝愚去,至是侂胄引為右史,大中駁之。改吏部侍郎,不拜,以煥章閣待制知慶元府。城南民田,潮溢不可種,大中捐公帑治石築之,民不知役而蒙其利。郡訛言夜有妖,大中謂此必黠賊所為,立捕黥之,人情遂安。丐祠,得請。給事中許及之繳駁,遂削職。後提舉沖祐觀。乞休致,復元職。監宗御史林採論列,再落
 職,尋復之。



 大中罷歸,屏居十二年,未嘗以得喪關其心,作園龜潭之上,客至,擷杞菊,取溪魚,觴酒賦詩,時事一不以掛口。客或勸大中通侂胄書,大中曰:「吾為夕郎時,一言承意,豈閑居至今日耶?」客曰:「縱不求福,盍亦免禍。」大中曰:「福不可求而得,禍可懼而免耶?」侂胄既召兵畔,大中謂:「今日欲安民,非息兵不可;欲息兵,非去侂胄不可。」



 及侂胄誅,即召見,落致仕,試吏部尚書,言:「呂祖儉以言侂胄得罪,死於瘴鄉,雖贈官畀職,而公議未厭。彭
 龜年面奏侂胄過尤,朱熹論侂胄竊弄威柄,皆為中傷,降官鐫職,卒以老死,宜優加旌表。其他因譏切侂胄以得罪者,望量其輕重而旌別之,以伸被罪者之冤。」除端明殿學士、簽書樞密院事。



 嘉定改元,兼太子賓客。嘗議講和事,上曰:「朕不憚屈己為民,講和之後,亦欲與卿等革侂胄弊政作家活耳。」大中頓首曰:「陛下言及此,宗社生靈之福也。」每語所親云:「吾年垂八十,豈堪勞勩,徒以和議未成,思體承聖訓,以革弊幸為經久之計。儻初志
 略遂,即乞身而歸矣。」是年六月卒,年七十有八,贈資政殿學士、正奉大夫,謚正惠。



 大中清修寡欲,退然如不勝衣,及其遇事而發,凜乎不可犯。自少力學,趣向不凡。所著有奏議、外制、文集三十卷。



 陳騤,字叔進,臺州臨海人。紹興二十四年,試春官第一,秦檜當國,以秦塤居其上。累官遷將作少監、守秘書少監兼太子諭德。太子尹臨安,騤謂:「儲宮下親細務,不得專於學,非所以毓德也。」太子矍然,亟辭。崔淵以外戚張
 說進,除秘書郎兼金部郎,騤封還詞頭。



 未幾,出知贛州,易秀州。召還,首言:「陛下銳意圖治,群下急於自媒,爭獻強兵理財之計,及畀以職,報效蔑聞。宜杜邪諂之路。」再歸故官,遷秘書監兼崇政殿說書。淳熙五年,試中書舍人兼侍講、同修國史。



 上欲採晉、宋以下興亡理亂之大端,約為一書,謂騤曰:「惟卿與周必大可任此事。」言者忌而攻之,上留章不下,授提舉太平興國宮。起知寧國府,改知太平州,加集英殿修撰。以言者罷。起知袁州。光宗
 受禪,召試吏部侍郎。紹熙元年,同知貢舉兼侍講。



 二年春,雷雪,詔陳時政得失,騤疏三十條,如宮闈之分不嚴,則權柄移;內謁之漸不杜,則明斷息;謀臺諫於當路,則私黨植;咨將帥於近習,則賄賂行;不求讜論,則過失彰;不謹舊章,則取舍錯;宴飲不時,則精神昏;賜予無節,則財用竭。皆切於時病。



 三年三月,權禮部尚書。六月,同知樞密院事。四年二月,參知政事。光宗以疾不朝垂華宮,會慶節稱壽又不果往。騤三入奏,廷臣上疏者以百數,
 上感悟,以冬至日朝重華。五年正月朔旦,稱壽於慈福宮。孝宗崩,光宗以疾未臨喪,騤請正儲位以安人心。七月,攝行三省事。



 寧宗即位,知樞密院事兼參知政事。趙汝愚為右丞相,騤素所不快,未嘗同堂語。汝愚擬除劉光祖侍御史,騤奏曰:「劉光祖舊與臣有隙,光祖入臺,臣請避之。」汝愚愕而止。



 時韓侂胄恃傳言之勞,潛竊國柄。吏部侍郎彭龜年論侂胄將為國患,不報。於是龜年、侂胄俱請祠,騤曰:「以閣門去經筵,何以示天下?」龜年竟外
 補。侂胄語人曰:「彭侍郎不貪好官,固也,元樞亦欲為好人耶?」遂以資政殿大學士與郡,辭,詔提舉洞霄宮。



 慶元二年,知婺州。告老,授觀文殿學士、提舉洞霄宮。嘉泰三年卒,年七十六。贈少傅,謚文簡。



 黃黼,字元章,臨安餘杭人也。少游太學,第進士,累遷太常博士。輪對,言:「周以輔翼之臣出任方伯,漢以牧守之最擢拜公卿,唐不歷邊任,不拜宰相,本朝不為三司等屬,不除清望官。仁宗時,韓琦、範仲淹、龐籍皆嘗經略西
 事,久歷邊任,始除執政。邊奏復警,範仲淹至再請行。貝州之變,文彥博親自討賊。乞於時望近臣中,擇才略謀慮可以任重致遠者,或畀上流,或委方面,習知邊防利害,地形險厄,中外軍民亦孚其恩信,熟其威名。天下無事則取風績顯著者不次除拜,以尊朝廷。邊鄙有警,則任以重寄,俾制方面。出將入相,何所不可。」上嘉獎曰:「如卿言,可謂盡用人之道。」



 行太常丞,進秘書郎、提舉江東常平茶鹽,召為戶部員外郎。尋除直秘閣、兩浙路轉運
 判官,進直龍圖閣,升副使,辭,改直顯謨閣。浙東瀕海之田,以旱澇告,常平儲蓄不足,黼捐漕計貸之。毗陵饑民取糠粃雜草根以充食,郡縣不以聞,黼取民食以進,乞捐僧牒、緡錢振濟,所全活甚眾。



 除中書門下檢正諸房公事,守殿中侍御史兼侍講,遷侍御史,行起居郎兼權刑部侍郎。以劉德秀論劾,奉祠而卒。



 詹體仁,字元善,建寧浦城人。父綎,與胡宏、劉子翬游,調贛州信豐尉。金人渝盟,綎見張浚論滅金秘計,浚闢為
 屬。體仁登隆興元年進士第,調饒州浮梁尉。郡上體仁獲盜功狀當賞,體仁曰:「以是受賞,非其願也。」謝不就。為泉州晉江丞。宰相梁克家,泉人也,薦於朝。入為太學錄,升太學博士、太常博士,遷太常丞,攝金部郎官。



 光宗即位,提舉浙西常平,除戶部員外郎、湖廣總領,就升司農少卿。奏蠲諸郡賦輸積欠百餘萬。有逃卒千人入大冶,因鐵鑄錢,剽掠為變。體仁語戎帥:「此去京師千餘里,若比上請得報,賊勢張矣。宜速加誅討。」帥用其言,群黨悉
 散。



 除太常少卿,陛對,首陳父子至恩之說,謂:「《易》於《家人》之後次之以《睽》,《睽》之上九曰:『見豕負塗,載鬼一車,先張之弧,後說之弧,匪寇婚媾,往,遇雨則吉。』夫疑極而惑,凡所見者皆以為寇,而不知實其親也。孔子釋之曰:『遇雨則吉,群疑亡也』。蓋人倫天理,有間隔而無斷絕,方其未通也,湮鬱煩憤,若不可以終日;及其醒然而悟,泮然而釋,如遇雨焉,何其和悅而條暢也。伏惟陛下神心昭融,聖度恢豁,凡厥疑情,一朝渙然若揭日月而開雲霧,。丕
 敘彞倫,以承兩宮之歡,以塞兆民之望」。時上以積疑成疾,久不過重華宮,故體仁引《易》睽弧之義,以開廣聖意。



 孝宗崩,體仁率同列抗疏,請駕詣重華宮親臨祥祭,辭意懇切。時趙汝愚將定大策,外庭無預謀者,密令體仁及左司郎官徐誼達意少保吳琚,請憲聖太后垂簾為援立計。寧宗登極,天下晏然,體仁與諸賢密贊汝愚之力也。



 時議大行皇帝謚,體仁言:「壽皇聖帝事德壽二十餘年,極天下之養,諒陰三年,不御常服,漢、唐以來未之
 有,宜謚曰『孝』。」卒用其言。孝宗將復土,體仁言:「永阜陵地勢卑下,非所以妥安神靈。」與宰相異議,除太府卿。尋直龍圖閣、知福州,言者竟以前論山陵事罷之。退居霅川,日以經史自娛,人莫窺其際。



 始,體仁使浙右,時蘇師旦以胥吏執役,後倚侂胄躐躋大官,至是遣介通殷勤。體仁曰:「小人乘君子之器,禍至無日矣,烏得以污我!」未幾,果敗。



 復直龍圖閣、知靜江府,閣十縣稅錢一萬四千,蠲雜賦八千。移守鄂州,除司農卿,復總湖廣餉事。時歲兇
 艱食,即以便宜發廩振捄而後以聞。



 侂胄建議開邊,一時爭談兵以規進用。體仁移書廟堂,言兵不可輕動,宜遵養俟時。皇甫斌自以將家子,好言兵,體仁語僚屬,謂斌必敗,已而果然。開禧二年卒,年六十四。



 體仁穎邁特立,博極群書。少從朱熹學,以存誠慎獨為主。為文明暢,悉根諸理。周必大當國,體仁嘗疏薦三十餘人,皆當世知名士。郡人真德秀早從其游,嘗問居官蒞民之法,體仁曰:「盡心、平心而已,盡心則無愧,平心則無偏。」世服其
 確論云。



 論曰:彭龜年、黃裳、羅點以青宮師保之舊,盡言無隱。黃度、林大中亦能守正不阿,進退裕如。此數臣者,皆能推明所學,務引君以當道,可謂粹然君子矣。陳騤論事頗切時病,詹體仁深於理學,皆有足稱者。然騤嘗詆譏呂祖謙,至視趙汝愚、劉光祖為仇,而體仁乃能以朱熹、真德秀為師友,即其所好惡,而二人之邪正,於是可知焉。



\end{pinyinscope}