\article{列傳第九}

\begin{pinyinscope}

 石守信子保興保吉孫元孫王審琦子承衍承乾孫克臣等高懷德韓重斌子崇訓崇業張令鐸羅彥環王彥升



 石守信,開封浚儀人。事周祖,得隸帳下。廣順初,累遷親衛都虞候。從世宗征晉,遇敵高平,力戰,遷親衛左第一軍都校。師還,遷鐵騎左右都校。從征淮南,為先鋒,下六合,入渦口,克揚州,遂領嘉州防禦使,充鐵騎、控鶴四廂都指揮使。從征關南,為陸路副都部署,以功遷殿前都虞候,轉都指揮使、領洪州防禦使。恭帝即位,加領義成軍節度。



 太祖即位,遷侍衛馬步軍副都指揮使,改領歸德軍節度。李筠叛,守信與高懷德率前軍進討,破筠
 眾於長平,斬首三千級。又敗其眾三萬於澤州,獲偽河陽節度範守圖,降太原援軍數千,皆殺之。澤、潞平,以功加同平章事。李重進反揚州,以守信為行營都部署兼知揚州行府事。帝親征至大儀頓,守信馳奏:「城破在朝夕,大駕親臨,一鼓可平。」帝亟赴之,果克其城。建隆二年,移鎮鄆州,兼侍衛親軍馬步軍都指揮使,詔賜本州宅一區。



 乾德初,帝因晚朝與守信等飲酒,酒酣,帝曰:「我非爾曹不及此,然吾為天子,殊不若為節度使之樂,吾終
 夕未嘗安枕而臥。」守信等頓首曰:「今天命已定,誰復敢有異心,陛下何為出此言耶?」帝曰:「人孰不欲富貴,一旦有以黃袍加汝之身,雖欲不為,其可得乎。」守信等謝曰:「臣愚不及此,惟陛下哀矜之。」帝曰:「人生駒過隙爾,不如多積金、市田宅以遺子孫,歌兒舞女以終天年。君臣之間無所猜嫌,不亦善乎。」守信謝曰:「陛下念及此,所謂生死而肉骨也。」明日,皆稱病,乞解兵權,帝從之,皆以散官就第,賞繼甚厚。



 已而,太祖欲使符彥卿管軍,趙普屢諫,
 以為彥卿名位已盛,不可復委以兵權,太祖不從。宣已出,普復懷之,太祖迎謂之曰:「豈非符彥卿事耶?」對曰:「非也。」因奏他事。既罷,乃出彥卿宣進之,太祖曰:「果然,宣何以復在卿所?」普曰:「臣托以處分之語有侏人離者,復留之。惟陛下深思利害,勿復悔。」太祖曰:「卿苦疑彥卿,何也?朕待彥卿厚,彥卿豈負朕耶。」普對曰:「陛下何以能負周世宗?」太祖默然,事遂中止。



 開寶六年秋,加守信兼侍中。太平興國初,加兼中書令。二年,拜中書令,行河南尹,充西
 京留守。三年,加檢校太師。四年,從征範陽,督前軍失律,責授崇信軍節度、兼中書令,俄進封衛國公。七年,徙鎮陳州,復守中書令。九年,卒,年五十七,贈尚書令,追封威武郡王,謚武烈。



 寧信累任節鎮,專務聚斂,積財鉅萬。尤信奉釋氏,在西京建崇德寺,募民輦瓦木,驅迫甚急,而傭直不給,人多苦之。子保興、保吉。



 保興字光裔,本名保正,太祖取興守之義改之。建隆初,年十四,以蔭補供奉官。明年,遷尚食副使。太祖嘗召功
 臣子弟詢以時事,保興年最少,應對明白,太祖奇之,拜如京使。開寶中,領順州刺史。太宗征河東,為御砦四面都巡檢。太平興國八年,出為高陽關監軍。守信卒,起復,領本州團練使。雍熙初,契丹擾邊,與戴興、楊守一並為澶州前軍駐泊。



 李繼遷入鈔,徙銀、夏、綏府都巡檢使。嘗巡按罨子砦,並黑水河,趣谷中,夏人知之,以數千騎據險,渡河求戰。保興所部不滿二千人,乃分短兵伏於河滸,俟其半渡,急擊之,斬首百餘級,追北數十里。優詔褒
 美。



 共中,知平戎軍,徙莫州,俄為西京都巡檢使。淳化五年,真拜蘄州團練使,為永興軍鈐轄,改夏、綏、麟、府州鈐轄。至道二年,徙延州都巡檢使兼署州事,改本路副都部署,與範重召等五路討賊。有岌伽羅膩數族率眾來拒,保興選敢死士數百人銜枚夜擊,殲之。自是吳移、越移諸族歸降。還,至烏、白池,賊又為方陣來拒。保興麾眾出入陣中,會乘馬中流矢,挺身持滿,易騎奮呼,且行且鬥,凡三日四十二戰,賊遂引去。



 咸平二年,知威虜軍。
 會夏人入鈔,保興發官帑錢數萬緡分給戰士,主者固執不可。保興曰:「城危如此,安暇中覆,事定,覆而不允,願以家財償之。」夏人退,驛置以聞,真宗貸而不問。



 三年,就拜棣州防禦使。徙知邢州,改澶州。在郡頗峻刑罰,每捶人,令緩施其杖,移晷方畢。五年,以疾求歸京師。未幾卒,年五十八。子元孫。



 保興世豪貴,累財鉅萬,悉為季弟保從之子所廢。



 保吉字祐之,初以蔭補天平軍衙內都指揮使。開寶四
 年,召見,賜襲衣、玉帶、金鞍勒馬。選尚太祖第二女延慶公主,拜左衛將軍、駙馬都尉,俄領愛州刺史。太平興國初,遷本州防禦使。五年,坐遣親吏市竹木秦、隴間,矯制度關,為王仁贍所發,罰一季奉。七年,改朔州觀察使。守信卒,起復,為威塞軍節度。雍熙三年,出知河陽。四年,召入,復命知大名府兼兵馬都部署,連改橫海、安國二鎮節度。



 真宗即位,加檢校太尉、保平軍節度。車駕北巡,命為河北諸路行營都部署,屯定州。景德初,改武寧軍節
 度、同平章事。冬,幸澶淵,命民李繼隆分為駕前東西面都排陣使,軍於北門外。遼騎數萬驟至城下,保吉不介馬而馳當其鋒,遼人引去。俄而請盟,錫宴射於行宮後苑。帝謂繼隆等曰:「自古北邊為患,今其畏威服義,息戰安民,卿等力也。」保吉進曰:「臣受命御患,上稟成算。至於布列行陣,指授方略,皆出於繼隆。」繼隆曰:「宣力用心,躬率將士,臣不及保吉。」帝曰:「卿等協和,共致太平,軍旅之事,朕復何憂。」歡甚,賜以襲衣、金帶、鞍勒馬。



 二年,改鎮安
 軍節度。未幾,自治所來朝,願奉朝請,從之。四年,部民上治狀,乞還鎮所,詔獎諭之,仍從其請。大中祥符初,從東封,攝司徒,封祀壇奉俎,加檢校太師還鎮。冬,公主疾,詔歸視,主薨。明年,保吉卒,年五十七,贈中書令,謚莊武。



 保吉姿貌環碩,頗有武幹。累世將相,家多財,所在有邸舍、別墅,雖饌品亦飾以彩繢。好治生射利,性尤驕倨,所至峻暴好殺,待屬吏不以禮。鎮大名也,葉齊、查道皆知名士,嘗械以運糧。初,程能為京西轉運,保吉托治其私負,
 能不從。至是,其子宿為屬邑吏,將辱之,會有闢召乃止。又染家貸錢,息不盡入,質其女,其父上訴,真宗亟命遣還。嘗有僕侵盜私積,不時求對,懇請配隸,帝曰:「是有常法,不可。」保吉請不已,帝戒勖之。



 善弋獵,畜鷙禽獸數百,令官健羅鳥雀飼之,人有規勸者輒怒之。在陳州,盛飾廨舍以迓貴主。因完葺城壘,疏牖於上,以瞰衢路,如箭窗狀。未嘗上聞,賓佐諫之不聽,頗涉眾議。初,守信鎮陳,五十七年卒,及保吉繼是鎮,壽亦止是,談者異之。



 保吉
 子貽孫,任崇儀使、帶御器械,坐事免官。孝孫,西京左藏庫使。



 元孫字善良,始名慶孫,避章獻太后祖諱易之。以守信蔭為東頭供奉官、閣門祗候,累遷如京副使。



 仁宗即位,改文思副使、勾當法酒庫。吏盜酒,坐失察,追二官,復如京副使。為澶州巡檢,徙知莫州,有治跡,以禮賓使再任。又徙保州,領廉州刺史,兼廣信、安肅軍緣邊都巡檢。時開屯田,鑿塘水,有訟元孫擅污民田者,遣官按視,訟者
 以誣服,即賜白金五百兩,詔褒諭之。再遷西上閣門使、並代州兵馬鈐轄,歷侍衛親軍步軍殿前都虞候、鄜延副都總管、緣邊安撫使,遷邕州觀察使。



 康定初,夏人寇延州,元孫與戰於三川口,軍敗見執。傳者以為已死,贈忠正軍節度使兼太傅,錄其子孫七人。及元昊納款,縱元孫歸。諫官御史奏:元孫軍敗不死,辱國,請斬塞下。賈昌朝獨言曰:「在春秋昌,晉獲楚將穀臣,楚獲晉將知□,亦還其國不誅。」因入對,探袖出《魏志於禁傳》以奏曰:「前
 代將臣敗覆而還,多不加罪。」帝乃貸元孫,安置全州。以升祔赦,內徙襄州。侍御史劉湜言:「元孫失軍辱命,朝廷貸而不誅,若例從量移,無以勸用命之士。」元孫遂不徙。從徙許州,還京師卒。



 王審琦字仲寶,其先遼西人,後徙家洛陽。漢乾祐初,隸周祖帳下,性純謹,甚親任之。從平李守貞,以功署廳直左番副將。廣順中,歷東西班行首、內殿直都知、鐵騎指揮使,從世宗征劉崇,力戰有功,遷東西班都虞候,改鐵
 騎都虞候,轉本軍右第二軍都校。世宗召禁軍諸校宴射苑中,審琦連中的,世宗嘉之,賞繼有加。俄領勤州刺史。



 親征淮南,舒州堅壁未下,詔以郭令圖領刺史,命審琦中超以精騎攻其城,一夕拔之,擒其史,獲鎧仗軍儲數十萬計。令圖既入城,審琦等遂救黃州,數日,令圖為舒人所逐。審琦選輕騎銜枚夜發,信宿至城下,大敗舒人,令圖得復還治所。世宗嘉之,授散員都指揮使。又能上能下南唐軍於紫金山,先登,中流矢,轉控鶴右廂都校、
 領虔州團練使。世宗圍濠州,審琦率敢死士數千人拔其水砦,奪月城,濠州遂降。及攻楚州,為南面巡檢,城將陷,審琦意淮人必遁,設伏待之。少頃,城中兵果鑿南門而潰,伏兵擊之,斬數千級,系五千餘人,獻於行在,賜名馬、玉帶、錦彩數百匹。淮南平,改鐵騎右廂都校。又從平瓦橋關。恭帝即位,遷殿前都虞候、領睦州防禦使。



 襟初,擢為殿前都指揮使、領泰寧軍節度。從征李筠,為御營前洞屋都部署,為飛石所傷,車駕臨視。澤、潞平,改領武
 成軍節度。李重進叛,副石守信為前軍部署討之。



 建隆二年,出為忠正軍節度。在鎮八年,為政寬簡。所部邑令以罪停其錄事吏,幕僚白令不先咨府,請按之。審琦曰:「五代以來,諸侯強橫,令宰不得專縣事。今天下治平,我忝守藩維,而部內宰能斥去黠吏,誠可嘉爾,何按之有?」聞者嘆服。



 開寶二年,從征太原,為御營四面都巡檢。三年,改鎮許州,賜甲第,留京師。太祖嘗召審琦宴射苑中,連中的,賜御馬、黃金鞍勒。六年,與高懷德並加同平章
 事。七年,卒,年五十。



 初,審琦暴疾,不能語,帝親臨視,及卒,又幸其第,哭之慟。賜中書令,追封瑯琊郡王,賻贈加等。葬日,又為廢朝。



 審琦重厚有方略,尤善騎射。鎮壽春,歲得租課,量入為出,未嘗有所誅求。素不能飲,嘗侍宴,太祖酒酣仰祝曰:「酒,天之美祿;審琦,朕布衣交也。方興朕共享富貴,何靳之不令飲邪?」祝畢,顧謂審琦曰:「天必賜卿酒量,試飲之,勿憚也。」審琦受詔,飲十杯無苦。自此侍宴常引滿,及歸私家即不能飲,或強飲輒病。



 子承衍、承
 乾、承德、承祐、承俊、承偓、承僎、承僅、承休。承西上閣門使、會州刺史,承9244至如京使,承俊、承僎至內殿崇班,承偓至閣門祗候,承僅至左神武將軍致仕,承休至內殿承制。



 承衍字希甫,幼端謹。審琦鎮袞、滑、壽春,皆署以牙職。開寶初,補內殿供奉官都知。三年,尚太祖女昭慶公主,授右衛將軍、附馬都尉,仍充都知。逾年,領恩州刺史,加本州防禦使。太平興國初,遷應州觀察使。二年春,太宗幸
 其第,賜宴,承衍以金器、名馬為壽,詔賜銀萬兩、錦彩五千匹。三年,加檢校太保。坐市竹木秦、隴,矯制免稅算,罰一季奉。七年,授前國軍節度。



 雍熙中,出知天雄軍府兼都部署。時契丹擾鎮陽,候騎至冀州,去魏二百餘里。鄰境戒嚴,城中大恐,屬上元節,承衍下令市中及佛寺然設樂,與賓佐宴游達旦,人賴以安。明年召還,復為貝冀都部署。端拱初,換永清軍節度,再知天雄軍。吏民千餘詣監軍,請為本道節帥,詔褒之。



 真宗即位,改河中尹、
 護國軍節度,加檢校太尉。咸平六年,以疾求罷節鋮,三抗表不許。帝自臨問,至臥內慰勉久之,賜予甚厚,擇尚醫數人迭宿其第。卒,年五十二。車駕臨,贈中書令,給鹵簿葬,謚恭肅。其後公主請置守塚五戶,從之。



 承衍善騎射,曉單律,頗涉學藝,好吟詠。以功臣子尚主貴顯,擁富貲,自奉甚厚。



 子世安、世隆、世雄、世融。世安至崇儀副使、通事舍人。世隆字本支,以公主子為如京副使,歷洛苑、六宅二使、領平州刺史。性驕恣,每坐諸叔之上,人皆
 嗤之。景德初卒,特贈泰州防禦使。召見其三子,賜名克基、克緒、克忠,皆面授供奉官。世雄至內殿崇班。世融為內殿承制。世安子克正殿中丞。克基、克忠並為西染院副使兼閣門通事舍人。克緒至內殿承制。世隆幼子克明為西上閣門副使。



 承乾字希悅,開寶中,授閑廄使,面賜紫袍、金帶,才十二歲。太平興國中,出監徐州軍,又為西京水南巡檢使,改如京使。表求治郡自效,命知潭州,遷六宅使、領昭州刺
 史,俄知澶州,加莊宅使。咸平中,兩賜川峽傳詔,慰撫官吏,經略蠻洞。連知延、代、並三州,皆兼兵馬鈐轄,改尚食使。鳳翔張雍病,命承乾代之,徙涇州,授下閣門使,改領永州刺史。景德中,真宗以天水近邊,蕃漢雜處,擇守臣撫治,擢承乾知秦州,徙知天雄軍。大中祥符初,進秩東上閣門使。承乾病足,在大名不能騎,政多廢馳,及代,賜告家居,表求解職,不允。以久不朝請,求近郡,改左武衛大將軍,知壽州。二年,卒,年四十九。詔遣其弟承倓巽馳
 往護喪。



 承乾頗涉學,喜為詩,所至為一集。曉音律,多與士大夫游,意豁如也。初,審琦鎮壽春,承乾生於郡廨,至卒亦於其地,人咸異之。



 子世京為閣門祗候,世文內殿崇班。



 克臣字子難。祖承衍尚秦國賢穆公主。克臣第景祐進士,仁宗閱其文,顧侍臣曰:「穆有孫登科,可喜也。」仕累通判壽州。鼓角卒夜入州廨,擊郡將,既就擒,而使所部被甲操刃立庭中,官吏駭觀。克臣徐言曰:「此不過
 為盜耳。」立遣甲者去,戒兒卒勿妄引他人,眾嘆服。是日天貺節,率掾屬朝謁如常儀,人賴以安,猶坐貶監潭州稅。



 熙寧中,為開封、度支二判官,遷鹽鐵副使。時鄭俠以上書竄嶺表,克臣嘗薦俠,且饋之白金,又坐奪官。復為戶部副使,以集賢殿修撰知鄆州。京東多盜,克臣請以便宜處決,遂下諸郡使械送尤桀者斬以徇,盜為少河決曹村,克臣亟築堤城下,或曰:「河決澶淵,去鄆為遠,且州徙於高,八十年不知有水患,安事此。」克臣不聽,役
 愈急,堤成,水大至,不沒者才尺餘。復起甬道,屬之東平王陵埽,人得趨以避水。事寧,皆繪像祀之。



 進天章閣侍制,徙知太原。王中正西討罔功,而誣克臣姑息士卒,使無固志,黜為單州。



 明年,拜工部侍郎。至是,神宗幸尚書省,至部舍止輦,獎其治力,以為雖少者不及。顧其子附馬都尉師約使入覲。元祐四年,以龍圖閣直學士、太中大夫卒,年七十六。



 師約字君授,少習進士業。英宗穀求儒生為主婿,命宰相召克臣諭旨,令師約持所為文至第。明日,獻賦一編,即坐中賦《大人繼明詩》,遂賜對,選為附馬都尉,尚徐國公主。授左衛將軍,面賜玉帶。又賜《九經》、筆硯,勉之進學。



 神宗即位,拜嘉州刺史,遷成州團練使。國朝故事主婿未嘗居職,帝始令師約同管當三班院,試其才。明年,主就館乃罷,遷汝州防禦使。始制附馬都尉七年考績法。轉晉州觀察使。



 哲宗立,遷鎮安軍節度觀察留後。宣仁
 後臨朝,師約屢上書言事。元符初,議者以為職不當上言,褫其秩。徽宗即位,乃復保平軍留後,又為樞密都承旨,未幾復罷。崇寧元年,卒,年五十九。



 師約善射,嘗陪遼使燕射玉津園,一發中鵠,發必破的,屢受金帶及鞍勒馬之賜。



 子殊,主所生,至閬州觀察使。



 高懷德字藏用,真定常山人,周天平節度齊王行周之子。懷德忠厚倜儻,有武勇。行周歷延、潞二鎮及留守洛都,節制宋、亳,皆署以牙職。晉開運初,遼人侵邊,以行周
 為北面前軍都部署。懷德始冠,白行周願從北征。行周壯之,許其行,至戚城遇遼軍,被圍數重,援兵不至,危甚。懷德左右射,縱橫馳突,眾皆披靡,挾父而出。以功領羅州刺史,賜珍裘、寶帶、名馬以寵異之。及行周移鎮鄆州,改信州刺史,仍領牙校。又遷信州刺史,從行周再鎮宋州。



 晉末,契丹南侵,以行周為邢趙路都部署御之,留懷德寧睢陽。會杜重威降契丹,京東諸州群盜大起,懷德堅壁清野,敵不能入。行周率兵歸鎮,敵遂解去。漢初,行
 周移鎮魏博,及再領天平,以懷德為忠州刺史領職如故。周祖征慕容彥超,還過汶上,宏賜行周甚厚,並賜懷德衣帶、彩繒、鞍勒馬。



 行周卒,召懷德為東西班都指揮使、領吉州刺史,改鐵騎都指揮使。太原劉崇入寇,世宗討之,以懷德為先鋒虞候。高平克捷,以功遷鐵騎右廂都指揮使、領果州團練使。



 從征淮南,知廬州行府事,充招安使。戰廬州城下,斬首七百餘級。尋遷能捷左廂都指揮使、領兵州防禦使,賜駿馬七匹。南唐將劉仁贍
 據壽春,舒元據紫金山,置連珠砦為援,以抗周師。世宗命懷德率帳下親信數十騎覘其營壘。懷德夜涉淮,遲明,賊始覺來戰,懷德以少擊眾,擒其裨將以還,盡偵知其形勢強弱,以白世宗。世宗大喜,賜襲衣、金帶、器幣、銀鞍勒馬。世宗一日因按轡準□需以觀賊勢,見一將追擊賊眾,奪槊以還,令左右問之,乃懷德也。召至行在慰勞,許以節鋮。



 世宗北征,命與韓通率兵先抵滄州。初得關南,又命副陳思讓為雄州兵馬都部署,克瓦橋關,降姚
 內斌以。恭帝嗣位,擢為侍衛馬軍都指揮使、領江寧軍節度,又為北面行營馬軍都指揮使。



 太祖即位,拜殿前副都點檢,移鎮滑州,充關南副都部署,尚宣祖女燕國長公主,加附馬都尉。李筠叛上黨,帝將親征,先令懷德率所與石守信進攻,破筠眾於澤州南。事平,以功遷忠武軍節度、檢校太尉。從平揚州。建隆二年,改歸德軍節度。開寶六年秋,加同平章事;冬,長公主薨,去附馬都尉號。



 太宗即位,加兼侍中,又加檢校太師。太平興國
 三年春,被病,詔太醫王元祐、道士馬志就第療之。四年,從平太原,改鎮曹州,封冀國公。七年,改武勝軍節度。是年七月,卒,年五十七,贈中書令,追封渤海郡王,謚武穆。



 懷德將家子,練習戎事,不喜讀書,性簡率,不拘小節。善音律,自為新聲,度曲極精妙。好射獵,嘗三五日露宿野次,獲狐兔累數百,或對客不揖而起,由別門自變量十騎從禽於郊。



 子處恭,歷莊宅使至右監門衛大將軍致仕。處俊至西京作坊使。



 韓重贇,磁州武安人。少以武勇隸周太祖麾下。廣順初,補左班殿直副都知。從世宗戰高平,以功遷鐵騎指揮使。從征淮南,先登中流矢,轉都虞候。俄遷控鶴軍都指揮使、領虔州刺史。



 宋初,以翊戴功,擬為龍捷左廂都校、領永州防禦使。從征澤、潞還,命代張光翰為侍衛馬步軍都指揮使、領江寧軍節度。討李重進,為行營馬步軍都虞候。建隆二年,改殿前都指揮使、領義成軍節度。三年,發京畿丁壯數千,築皇城東北隅,且令有司繪洛陽
 宮殿,按圖修之,命重贇其役。乾德三年秋,河決澶州,命重贇督丁壯數十萬塞之。



 四年,太祖郊祀,以為儀仗都部署。時有譖贇私取親兵為腹心者,太祖怒,欲誅之。趙普諫曰:「親兵,陛下必不自將,須擇人付之。若重贇以讒誅,即人人懼罪,誰復為陛下將親兵者。」太祖納其言,重贇得不誅。後聞普嘗救己,即詣普謝,普拒不見。



 五年二月,出為彰德軍節度。開寶二年,太祖征太原,過其郡,重贇迎謁於王橋頓,召赴燕飲。帝曰:「契丹知我是行,
 必率眾來援,彼意鎮、定無備,必由此路入。卿為我領兵倍道兼行,出其不意,破之必矣。」乃命為北面都部署。重贇令軍士銜枚夜發,果遇契丹兵於定州,見重贇旗幟,大駭欲引去,重贇乘之,大破其眾,獲馬數百匹。太祖大喜,優詔褒美。七年,卒,贈侍中。



 重贇信奉釋氏,在安陽六七年,課氏採木為寺,郡內苦之。子崇訓、崇業。



 重贇與張光翰、趙彥徽分領諸軍節度,嘉其翊戴功也。光翰,後唐山南節度使虔劍兄子,及卒,贈侍中。彥徽,真定安喜人,
 與太祖同事世宗,太祖兄事之,及卒,贈侍中。



 崇訓子知禮,乾中,以蔭補供奉官,遷西京作坊副使,出為澶州河南北都巡檢使。從太宗征河東,還,以貝、冀等州都巡檢使權知麟州。



 雍熙中,李繼遷寇夏州,崇訓領兵赴援,大敗之。徙監夏州軍。歷知越、泉、登、莫四州,徙知威虜軍,改如京使。咸平初,出知石州。屬繼遷犯境,崇訓追襲之,至賀蘭山而還。二年,再知麟州,又敗繼遷於城下。



 崇訓由河西徙閩、越,再移北邊,凡二十五年,以勞
 擢西上閣門使、邠寧環慶清遠軍都巡檢使。徙鎮、定、高陽關行營鈐轄,屯鎮州,兼河北都轉運使事。契丹兵至方順河,將寇威虜軍,崇訓陳兵唐河,折其要路。敵遣別騎冠赤堠驛崇,崇訓分兵擒戮之。既而值霖雨,敵兵饑乏不敢進,遂遁去。移並、代鈐轄,權知並州。從產中署張進領兵由王門會大將王超,襲破契丹於定州。六年,授四方館使、樞密都承旨。又命為鎮、定、高馬步軍都鈐轄,屯定州。



 景德初,契丹入寇至唐河,崇訓陳兵河南。翌日,又
 與王超追襲至鎮州。既而都部署桑贊逗留不進,崇訓帥兵獨往。時車駕幸澶州,召崇訓,乃還。三年春,拜檢校太傅。大中祥符二年,授右龍武軍大將軍,領韶州防禦使,以本官分司西京卒,年五十六。



 崇訓為人長厚謙畏,未嘗忤物。



 子允恭,禮賓副使,有謀略,好學,人以為能世其家云。



 崇業字繼源,以蔭補供奉官,選尚秦王美女雲陽公主,授左臨門衛將軍、附馬都尉。廷美得罪,降為右千牛
 衛率府率,分司西京,俄削秩,去附馬之號,從貶房陵。廷美卒,起為靜難軍行軍司馬。雍熙三年,授寧州刺史。公主卒,葬州境。真宗初,始得入朝。咸平四年,改左屯衛大將軍、領高州團練使,追封公主為虢國長公主。五年十月,卒,年四十一。



 子允升為內殿承制、閣門祗候。



 張令鐸,棣州厭次人。少以勇力隸軍伍。後唐清泰中,補寧衛小校。晉初,改隸奉國軍。漢乾祐中,從周太祖平河中,以功遷奉國軍指揮使。廣順初,遷控鶴指揮使。累遷
 本軍左廂都指揮使、領虔州團練使。從世宗征淮南,移領虎捷左廂,加常州防禦使。再征壽春,命與龍捷右廂柴貴分為京城左右廂巡檢。世宗將北征,命與韓通、高懷德領兵先赴滄州,又副韓令坤為霸州部署,率兵戍守。恭帝即位,授侍衛親軍步軍都指揮使、領武信軍節度使。令鐸本名鐸,以與河中張鐸同姓名,故賜今名。



 宋初,遷馬步軍都虞候、領陳州節制。太祖征李筠,以令鐸為東京卓城內都巡檢。建隆二年,出為鎮寧軍節度。帝
 為皇弟興元尹光美娶其第三女。開寶二年,來朝被病,車駕臨問,賜帛五千匹、銀五千兩,並賜其家人甚厚。明年春,卒於京師,年六十。帝甚悲悼,贈侍中。



 令鐸性仁恕,嘗語人曰:「我從軍三十年,大小四十餘戰,多摧堅陷敵,未嘗妄殺一人。」及卒,人多惜之。



 子守正,至內園使。守恩,淳化中,累至崇儀副使,稍遷崇儀使,領錦州刺史。景德初,知原州,就加西上閣門使、知泰州,卒。錄其子奉禮郎永安為大理評事,後至殿中丞。



 羅彥環,並州太原人。父全德,晉泌州刺史,彥環得補內殿直。



 少帝在澶州,欲命使宣慰大名府,時河北契丹騎充斥,遂募軍中驍勇士十人從行,彥環備選。銜枚夜發,往返如期,由是補興順指揮使。開運末,契丹主至汴,擢為護聖指揮使。赴幽薊。彥環至元氏,聞漢祖建號太原,以為歸漢,漢祖嘉之。及入汴,擢為護聖指揮使。周初,遷散員都虞候,坐樞密使王浚黨,出為鄧州教練使。世宗嗣位,召為伴飲指揮使,改馬步軍都軍頭。從向訓收
 秦、鳳有功,遷散指揮都虞候。



 顯德末,太祖自陳橋入歸公署,見宰相範質等,未及言,彥不挺劍而前曰:「我輩無主,今日須天子。」質等由是降階聽命。擢為控鶴左廂都指揮使,改內外馬軍都軍頭、領眉州防禦使。從平澤、潞還,命代趙彥徽為侍衛步軍都指揮使、領武信軍節度。建隆二年,出為彰德軍節度。乾德二年,改安國軍節度,與昭義軍節度李繼勛大破契丹。四年春,又與閣門使田欽祚殺太原軍千餘人於靜陽,禽其將鹿英等,獲
 馬三百匹。明年,移鎮華州。開寶二年,卒,年四十七。



 王彥升字光烈,性殘忍多力,善擊劍,號「王劍兒」。本蜀人,後唐同光中,蜀平,徙家洛陽。



 初事宦官驃騎大將軍孟漢瓊,漢瓊以其趫勇,言於明宗,補東班承旨。晉天福中,轉內殿直。開運初,契丹圍大名,少帝幸澶州,募勇敢士繼詔納城中,彥升與羅彥環應之。一夕突圍而入,以功遷護聖指揮使。周廣順中,從向拱破太原兵虒亭南,斬敵帥王璋於陣,以功遷龍捷右第九軍都虞候。累轉鐵
 騎右第二軍都校、領合州刺史。世宗征淮南,從劉崇進、宗偓破金牛水砦,禽偽軍校閻承旺、範橫。又從李重進捍吳兵於盛唐,斬二千人個利益級。又從張永德攻瀛州,下束城,改散員都指揮使。



 太祖北征,至陳橋,為眾推戴。彥升以所部先入京,遇韓通於路,逐至第殺之。初,太祖誓軍入京不得有秋毫犯,及聞通死,意甚不樂。以建國之始,不及罪彥升,拜恩州團練使、領鐵騎左廂都指揮使。



 後為京城巡檢,中夜詣王溥第,溥驚悸而出,既坐,乃曰:「此
 夕巡警甚困,聊就公一醉耳。」彥升意在求賄,溥佯不悟,置酒數行而罷。翌日,溥密奏其事,乃出為唐州刺史。



 乾德初,遷申州團練使。開寶二年,改防州防禦使,是冬,又移原州。西人有犯漢法者,彥升不加刑,召僚屬飲宴,引所犯以手捽斷其耳,大嚼,卮酒下之。其人流血被體,股心慄不敢動。前後啖者數人。西人畏之,不敢犯塞。七年,以病代還,次乾州卒,年五十八。太祖以其奪殺韓通,終身不授節鋮。



 論曰:石守信而下,皆顯德舊臣,太祖開懷信任,獲其忠力。一日以黃袍之喻,使自解其兵柄,以保其富貴,以遺其子孫。漢光武之於功臣,豈過是哉。然守信之貨殖鉅萬,懷德之馳逐敗度,豈非亦因以自晦者邪。至於審琦之政成下蔡,重贇之功宣廣陵,卓乎可稱。令鐸身四十餘戰,未嘗妄殺,可主胃勇者之仁矣。彥環於革命之日,首挺劍以語範質,於宋則未必功以眾先,於周則其過不在人後矣。王彥升殺韓通,太祖雖不加罪,而終身不授
 節鋮,是足垂訓後人矣。保吉、承衍咸以帝婿致位落鎮,其被驅策、著功,則保吉為優,況推功李繼隆,尤為不伐而有讓,然械役名士,縱意禽荒,累德多矣



\end{pinyinscope}