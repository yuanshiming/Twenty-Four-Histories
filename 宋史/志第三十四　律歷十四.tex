\article{志第三十四 律歷十四}

\begin{pinyinscope}

 中原既失,禮
 樂淪亡。高宗時,胡銓著《審律論》,曰:



 臣聞司馬遷有言曰:「六律為萬事根本,其於兵械尤所重,望敵知吉兇,聞聲效勝負,百王不易之道也。」臣嘗深愛遷之
 言律於兵械為尤重,而深惜後之談兵者止以戰鬥、擊刺、奇謀,此律之所以汨陳而學者未嘗道也。



 夫律、度、量、衡,古也淵源於馬遷,濫觴於班固,劉昭挹其流,孟康、京房、錢樂之之徒汨其泥而揚其波。遷之言曰:「黃鐘之實八十一以為宮,而以九為法,實如法,得長一寸,則黃鐘為九寸矣。黃鐘之實十七萬七千一百四十七,而以一萬九千六百八十三為法,實如法,亦得長一寸,亦黃鐘為九寸也。然則十七萬七千一百四十七與夫所謂八
 十一者,雖多少之不同,而其實一也;萬九千六百八十三與夫所謂九者,雖多少之不同,而其法一也。又曰,丑二,寅八,卯十六,辰六十四。夫醜與卯,陰律也;寅與辰,陽律也。生陰律者皆二,所謂下生者倍其實;生陽律者皆四,所謂上生者四其實。遷之言財數百,可謂簡矣,而後之言律者祖焉,是不亦淵源於馬遷乎?



 固之言曰:黃鐘之實,八百一十分。蓋遷意也。然以林鐘之實五百四十,而乃以為六百四十,林鐘、太蔟之實以其長自乘,則聲
 雖有,小同於黃鐘之宮耳。然則魏柴玉制律,而與黃鐘商、征不合,其失兆此矣。夫自子一分,終於亥十七萬七千一百四十七分,蓋遷術也。而固亦曰,太極元氣,函三為一,始動於子,參之於丑,歷十二辰之數,而得黃鐘之實,以為陰陽合德,化生萬物。其說蓋有本矣。然其言三分蕤賓損一,下生大呂,而不言夫所謂濁倍之變何?夫蕤賓之比於大呂,則蕤賓清而大呂濁,今又損二分之一以生大呂,則大呂之聲乃清於蕤賓,是不知夫倍大
 呂之濁。然則蕭衍之論,至於夾鐘而裁長三寸七分,其失兆此矣。是不亦濫觴於班固乎?



 昭之言曰:推林鐘之實至十一萬八千九十八、太蔟之實至十五萬七千四百六十四,二乘而三約之者,為下生之實;四乘而三約之者,為上生之實。此遷、固之意,昭則詳矣。然以蕤賓為上生大呂,而大呂乃下生夷則,何也?蓋昭之說陽生陰為下生,陰生陽為上生。今以蕤賓為上生大呂,則是陽生陰,乃上生也;以大呂為下生夷則,是陰生陽,乃下生
 也。其蔽亦由不知夫大呂有濁倍之變,則其視遷、固去本遠矣。是不亦挹其流於劉昭乎?



 若夫孟康、京房、錢樂之之徒,則又大不然矣。夫班固以八十一分為黃鐘之實,起十二律之周徑,度其長以容其實,初末嘗有徑三圍九之說也。康之徒惑於八十一分之實,以一寸為九十分,而不察方圜之異,於是有徑三圍九之論興焉。天律之形圜,如以為徑三圍九,則刓其四用之方,而不足於九分之數,以之容黍,豈能至於千二百哉!然則所謂
 圍九,方分也。何以知之?知龠之方,則知黃鐘之分亦方也。固雖無明說,其論洛下閎起歷之法曰:「律容一龠,積八十一寸,則一日之分也。」夫八十一寸者,是乃八百一十分,以千二百黍納之龠中,則不搖而自滿,是無異黃鐘之容也。龠之制,方寸而深八分。一龠之方,則黃鐘之分,安得而不方哉!圍九方分而圜之,則徑不止於三分矣。故夫徑三圍九之說,孟康為之也。



 然由律生呂,數十有二,止矣;京氏演為六十,錢樂之廣為三百六十,則與
 黃帝之說悖矣。蓋樂之用《淮南》之術,一律而生五音,十二律而為六十音,而六之,故三百六十音,以當一歲之日。以黃鐘、太蔟、姑洗、林鐘、南呂生三十有四,以大呂、夾鐘、中呂、蕤賓、夷則、無射生二十有七,應鐘生二十有八,始於包育,而終於安運。然由黃鐘迄於壯進百有五十,則三分損一焉以下生;由依行迄於億兆二百有九,則三分益一焉以上生;惟安運為終而不生。其言與黃帝之法大相抵牾。自遷、固而下,至是雜然莫適為主,至五
 季王樸而後議少定,沉括、蔣之奇論之當矣。是不亦汨其泥而揚其波乎?



 嗚呼!律也者,固以實為本而法為末,陛下修其實於上,而有司方定其法於下,以協天地中和之聲,則夫數子者,其說有可考焉,臣敢輕議哉!



 淳熙間,建安布衣蔡元定著《律呂新書》,朱熹稱其超然遠覽,奮其獨見,爬梳剔抉,參互考尋,推原本根,比次條理,管括機要,闡究精微。其言雖多出於近世之所未講,而實無一字不本於古人之成法。其書有《律呂本原》、《律呂證
 辨》。《本原》者,《黃鐘》第一,《黃鐘之實》第二,《黃鐘生十二律》第三,《十二律之實》第四,《變律》第五,《律生五聲圖》第六,《變聲》第七,《八十四聲圖》第八,《六十調圖》第九,《候氣》第十,《審度》第十一,《嘉量》第十二,《謹權量》第十三。《證辨》者,《造律》第一,《律長短圍徑之數》第二,《黃鐘之實》第三,《三分損益上下相生》第四,《和聲》第五。權臣既誣元定以偽學,貶死舂陵,雖有其書,卒為空言,嗚呼惜哉!



 久之,宜春歐陽之秀復著《律通》,其自序曰:



 自律呂之度數不見於經,而釋經者
 反援《漢志》以為據,蓋濫觴於《管子》、《呂氏春秋》,流衍於《淮南子》、司馬遷之書,而波助於劉歆、京房之學。班固《漢志》,盡歆所出也;《司馬彪志》,盡房所出也。後世協律者,類皆執守以為定法。歷代合樂,不為無人,而終不足以得天地陰陽之和聲,所以不能追還於隆古之盛者,大抵由三分損益之說拘之也。夫律固不能舍損益之說以求之,由其有損有益,而後有上生下生之異。至其專用三分以為損益之法則失之,未免乎聲與數之不相合,有
 非天成之自然耳。



 蓋嘗因其損益、上下生之義,而去其專用三分之蔽,乃多為分法以求之,自黃鐘以往,其下生者盈十,而上生者止一而已。此其數之或損或益,出於自然,而與舊法固不侔矣。若謂相生之法,一下必一上,既上而復下,則其法之窮也,於蕤賓、大呂間見之。夫黃鐘而降,轉以相生,至於姑洗則下生應鐘,而應鐘之上生蕤賓者,法也。今乃蕤賓之生大呂,又從而上生焉,此《班志》所載,所以變其說為下生大呂,而大呂之長遂
 用倍法矣。夫律之相生而用倍法,猶為有理,獨專用三分以為損益,則律之長短,不中乎天地自然之數爾。



 生律之分,蓋不止於三分損益之一端,以一律而分為三,此生律之極數,特一求徵聲之法耳。茍以三分損益,一下生而一上生,則聲律殆無窮矣,何至於十二而止也乎。夫十二律之生也,十律皆下生,一律獨上生。唯其下生者,損之極也,而後上生者益焉。上生則律窮矣,此窮上反下、窮下反上之理也。琴一弦之間具十二律,皆用
 下生之法,而末以上生法終之。若以七弦而緊慢之為旋宮之法,則應鐘一均之律,宮聲之外,多用倍法生一律矣。此天地聲音自然而然,不可拘於一而不知通變也。故正律止於十二而已。



 竊意十二律之度數,當具於《周禮》之《冬官》,如《考工記》鳧氏為鐘、磬氏為磬之類,各有一職。然《冬官》一篇既亡,則世無以考其度數之詳,而三分損益之說散見於書傳者,恐或得之目擊而不及識其全,或得之口授而未能究其誤,或求諸耳決而不能
 究其真,因是遂著為定論。夫人皆以為法之盡善矣,豈知三分損益所生之律,乃僅得其聲之近似而未真。蓋非師曠之聰,則耳不能齊,其聲之近似者,足以惑人之聽,是以不復求其法之未盡善者。此蔡邕所以不如耳決之明者,亦不能盡信其法也。



 後世之制樂者,不知律法之固有未善,而每患其聲音高下之不協,以至取古昔遺亡之器而求之,蓋亦不知本矣。聲以數而傳,數以聲而定,二者皆有自然之則。如侈者聲必咋,弇者聲必
 鬱,高者數必短,下者數必長。侈弇者,數也,未聞其聲而已知其有咋鬱之分;高下者,聲也,未見其數已知其有長短之異。故不得其自然之聲,則數不可得而考;不得其自然之數,則聲不可得而言。今之制律者,不知出此,而顧先區區於秬黍之縱橫、古尺之修短、觔斗之廣狹、鐘磬之高下謀之,是何足以得其聲之和哉!



 邵雍曰:「世人所見者,漢律歷耳。」然則三分損益之法為未善,亦隱然矣。近世蔡元定特著一書,可謂究心,然其說亦有可
 用與否。其可用者,多其所自得,而又有證於古,凡載於吾書者可見矣;其否者,皆由習熟於三分上下生之說,而不於聲器之近似者察之也。豈嘗察之而未有法以易之乎?此《律通》之所以作也。



 蓋律之所以長短,不止乎三分損益之一端,自四分以往,推而至於有二十分之法。管之所以廣狹,必限於千二百黍之定數,因其容受有方分、圜分之異,與黍體不相合,而遂分辨其空龠有實積、隙積之理。其還相為宮之法,有以推見其為一陰
 一陽相繼之道,而非一上一下相生之謂也。



 嗟乎!觀吾書者,能知其數之出於自然而然,則知由先漢以前至於今日,上下幾二千年,凡史傳所述三分損益一定之說者,可以刪而去之矣。使其說之可用也,則累世律可協、樂可和,何承天、劉焯輩不改其法矣。故京房六十律不足以和樂,而況錢樂之衍為三百六十之非法,徒增多而無用乎?是其數非出於自然之無所加損,而徒欲傅會於當期之日數云爾。



 古之聖人所以定律止於十二
 者,自然之理數也。茍不因自然之理數,則以三分損益之法衍之,聲律殆不特三百六十而已也,而況京房之六十乎!且房之律,吾意其自為之也,而托言受之焦延壽,以欺乎人,以售其說。使律法之善,何必曰受諸人?律法不善矣,雖焦延壽何益哉!所謂善不善者,亦顧其法之可用與否耳。曩者,魏漢津嘗創用指尺以制律,乃竊京房之故智,上以取君之信,下以遏人之議,能行之於一日,豈能使一世而用之乎?



 今《律通》之作,其數之損益
 可以互相生,總為百四十四以為之體,或變之,又可得二百一十有六以為之用,乾坤之策具矣。世不用則已,用則聲必和,亦因古黃鐘九寸法審之,以人物之聲而稍更定之耳。或曰:律止十二,胡為復衍百四十四律乎?」應之曰:「十二者,正聲也;百四十四者,變聲也。使不為百四十四者,何以見十二宮七聲長短之有定數,而宮、商、角、征、羽清濁之有定分乎?其要主於和而已。故有正聲則有變聲也,通其變然後可與論律矣。」



 《律通》上、下二篇:《
 十二律名數》第一,《黃鐘起數》第二,《生律分正法》第三,《生律分變法》第四,《正變生律分起算法》第五,《十二宮百四十四律數》第六,《律數傍通法》第七,《律數傍通別法》第八;《九分為寸法辨》第九、第十,《五十九律會同》第十一,《空圍龠實辨》第十二,《十二律分陰陽圖說》第十三,《陽聲陰聲配乾坤圖》第十四,《五聲配五行之序》第十五,《七聲配五行之序》第十六,《七聲分類》第十七,《十二宮七聲倡和》第十八,《六十調圖說》第十九,《辨三律聲法》第二十。真德
 秀、趙以夫皆盛稱之。



 舒州桐城縣丞李如篪作《樂書》,評司馬光、範鎮所論律,曰:



 鎮得蜀人房庶言尺法,庶言:「嘗得古本《漢書》,云:『度起於黃鐘之長,以子穀秬黍中者,一黍之起,積一千二百黍之廣,度之九十分,黃鐘之長,一為一分。』今文脫去『之起積一千二百黍』八字,故自前世累黍為之,縱置之則太長,橫置之則太短。今新尺橫置之不能容一千二百黍,則大其空徑四厘六毫,是以樂聲太高,皆由儒者誤以一黍為一分,其法非是。不若以千
 二百黍實管中,隨其短長斷之,以為黃鐘九寸之管九十分,其長一為一分,取三分以度空徑,數合則律正矣。」鎮盛稱此論,以為先儒用意皆不能到。其意謂制律之法,必以一千二百黍實黃鐘九寸之管九十分,其管之長一為一分,是度由律起也。光則據《漢書》正本之「度起於黃鐘之長。以子穀秬黍中者,一黍之廣,度之九十分,黃鐘之長,一為一分。」本無「之起積一千二百黍」八字。其意謂制律之法,必以一黍之廣定為一分,九十分則得黃鐘
 之長,是律由度起也。



 《書》云:「同律、度、量、衡。」先言律而後及度、量、衡,是度起於律,信矣。然則鎮之說是,而光之說非也。然庶之論積一千二百黍之廣之說則非,必如其說,則是律非起於度而起於量也。光之說雖非先王作律之本,而後之為律者,不先定其分寸,亦無以起律。又其法本之《漢志》之文,則光之說亦不得謂其非是也。



 故嘗論之,律者,述氣之管也。其候氣之法,十有二月,每月為管,置於地中。氣之來至,有淺有深,而管之入地者,有短
 有長。十二月之氣至,各驗其當月之管,氣至則灰飛也。其為管之長短,與其氣至之淺深,或不相當則不驗。上古之聖人制為十二管,以候十二辰之氣,而十二辰之音亦由之而出焉。以十二管較之,則黃鐘之管最長,應鐘之管至短;以林鐘比於黃鐘,則短其三分之一;以太簇比之林鐘,則長其三分之一;其餘或長或短,皆上下於三分之一之數。其默符於聲氣自然之應者如此也,當時惡睹所謂三分損益哉!又惡睹夫一千二百黍實
 黃鐘容受之量與夫一黍之廣一為一分之說哉!古之聖人既為律矣,欲因之以起度、量、衡之法,遂取秬黍之中者以實黃鐘之管,滿龠傾而數之,得黍一千有二百,因以制量;以一黍之廣而度之,得黃鐘管九十分之一,因以起度;以一龠之黍之重而兩之,因以生衡。去古既遠,先王作律之本始,其法不傳,而猶有所謂一千二百黍為一龠容受之量與夫一黍之廣一為一分者可考也。推其容受而度其分寸,則律可得而成也。先王之本
 於律以起度、量、衡者,自源而生流也;後人以度、量、衡而起律者,尋流而及源也。



 光、鎮爭論往復,前後三十年不決,大概言以律起度,以度起律之不同。鎮深闢光以度起律之說,不知後世舍去度數,安得如古聖人默符聲氣之驗,自然而成律也哉?至若庶之增益《漢志》八字以為脫誤,及其它紛紛之議,皆穿鑿以為新奇,雖鎮力主之,非至當之論有補於律法者也。



 如篪書曰《樂本》,曰《樂章》。



 沙隨程迥著《三器圖議》,曰:「體有長短,所以起度也;受
 有多寡,所以生量也;物有輕重,所以用權也。是器也,皆準之上黨羊頭山之秬黍焉。以之測幽隱之情,以之達精微之理。推三光之運,則不失其度;通八音之變,則可召其和。以辨上下則有品,以分隆殺則有節。凡朝廷之出治,生民之日用,未有頃刻不資焉者也。古人以度定量,以量定權,必參相得,然後黃鐘之律可求,八音五聲從之而應也。皇祐中,阮逸、胡瑗累黍定尺,既大於周尺,姑欲合其量也,然竟於權不合,乃謂黍稱二兩,已得官
 稱一兩,反疑史書之誤。及韓琦、丁度詳定,知逸、瑗之失,亦莫能以三器參相考也。」



 先是,鎮上封事曰:「樂者,和氣也;發和氣者,音聲也。音聲生於無形,故古人以有形之物傳其法,俾後人參考之。有形者何?秬黍也、律也、尺也、龠也、釜也、斛也、算數也、權稱也、鐘也、磬也,是十者必相合而不相戾,而後為得也。」迥謂:「以黍定三器,則十者無不該。三者,尺為之本。周尺也者,先儒考其制,吻合者不一。至宋祁取《隋書》大業中歷代尺十五等,獨以周尺為
 之本,以考諸尺。韓琦嘉祐累黍尺二,其一亦與周尺相近。司馬備刻之於石。光舊物也。茍以是定尺,又以是參定權量,以合諸器,如挈裘而振其領,其順者不可勝數也。」



 迥博學好古,朱熹深禮敬之。其後江陵府學教授廬陵彭應龍,既注《漢·律歷志》,設為問答,著《鐘律辨疑》三卷,至為精密,發古人所未言者。



 宋歷在東都凡八改,曰《應天》、《乾元》、《儀天》、《崇天》、《明天》、《奉元》、《觀天》、《紀元》。星翁離散,《紀元歷》亡,紹興二年,高宗重購得之,六月甲午,語輔臣曰:「歷
 官推步不精,今歷差一日,近得《紀元歷》,自明年當改正,協時月正日,蓋非細事。」是歲,始議制渾儀。十一月,工部言,《渾儀法要》當以子午為正,今欲定測樞極,合差局官二員。詔差李繼宗等充測驗定正宮,俟造畢進呈日,同參詳指說制度官丁師仁、李公謹入殿安設。三年正月壬戌,進呈渾儀木樣。壬申,太史局令丁師仁等言,省識東都渾儀四座:在測驗渾儀刻漏所曰至道儀,在翰林天文局曰皇祐儀,在太史局天文院曰熙寧儀,在合臺
 曰元祐儀,每座約銅二萬餘斤,今若半之,當萬餘斤。且元祐制造,有兩府提舉。時都司覆實,用銅八千四百斤。詔工部置物料,臨安府傭工匠,仍令工部長、貳提舉。



 五年,日官言,正月朔旦日食九分半,虧在辰正。常州布衣陳得一言:當食八分半,虧在巳初。其言卒驗。侍御史張致遠言:「今歲正月朔日食,太史所定不驗,得一嘗為臣言,皆有依據。蓋患算造者不能通消息、盈虛之奧,進退、遲疾之分,致立朔有訛。凡定朔小餘七千五百以上者,
 進一日。紹興四年十二月小餘七千六百八十,太史不進,故十一月小盡;今年五月小餘七千一百八十,少三百二十,乃為進朔,四月大盡。建炎三年定十一月三十日甲戌為臘,陰陽書曰:臘者,接也,以故接新,在十二月近大寒前後戌日定之,若近大寒戌日在正月十一日,若即用遠大寒戌日定之,庶不出十二月。如宣和五年十二月二十七日丙午大寒,後四日庚戌,雖近,緣在六年正月一日,此時以十九日戊戌為臘。得一於歲旦日
 食,嘗預言之,不差厘刻。願詔得一改造新歷,委官專董其事。仍盡取其書,參校太史有無,以補遺闕。擇歷算子弟粗通了者,授演撰之要,庶幾日官無曠,歷法不絕。」二月丙子,詔秘書少監朱震即秘書省監視得一改造新歷。八月,歷成,震請賜名《統元》,從之。詔翰林學士孫近為序,以六年頒行,遷震一秩,賜得一通微處士,官其一子。道士裴伯壽等受賞有差。



 得一等上推甲子之歲,得十一月甲子朔夜半冬至日度起於虛中以為元。著《歷經》
 七卷,《歷議》二卷,《立成》四卷,《考古春秋日食》一卷,《七曜細行》二卷,《氣朔入行草》一卷,詔付太史氏,副藏秘府。



 紹興九年,史官重修神宗正史,求《奉元歷》不獲,詔陳得一、裴伯壽赴闕補修之。



 十四年,太史局請制渾儀,工部員外郎謝伋言:「臣嘗詢渾儀之法,太史官生論議不同,鑄作之工,今尚闕焉。臣愚以為宜先詢訪制度,敷求通曉天文歷數之學者,參訂是非,斯合古制。」蘇頌之子應詔赴闕,請訪求其父遺書,考質制度。宰相秦檜曰:「在廷之臣,
 罕能通曉。」高宗曰:「此闕典也,朕已就宮中制造,範制雖小,可用窺測,日以晷度、夜以樞星為則,非久降出,第當廣其尺寸爾。」於是命檜提舉。時內侍邵諤善運思,專令主之,累年方成。



 《統元歷》頒行雖久,有司不善用之,暗用《紀元》法推步,而以《統元》為名。乾道二年,日官以《紀元歷》推三年丁亥歲十一月甲子朔,將頒行,裴伯壽詣禮部陳《統元歷》法當進作乙丑朔,於是依《統元歷》法正之。



 光州士人劉孝榮言:「《統元歷》交食先天六刻,火星差天二
 度。嘗自著歷,期以半年可成,願改造新歷。」禮部謂:「《統元歷》法用之十有五年,《紀元歷》法經六十年,日月交食有先天分數之差,五星細行亦有二三度分之殊。算造歷官拘於依經用法,致朔日有進退,氣節日分有誤,於時宜改造。」伯壽言:「造歷必先立表測景驗氣,庶幾精密。」判太史局吳澤私於孝榮,且言銅表難成、木表易壞以沮之。乃詔禮部尚書周執羔提領改造新歷,執羔亦謂測景驗氣,經涉歲月。孝榮乃採萬分歷,作三萬分以為日
 法,號《七曜細行歷》,上之。三年,執羔以歷來上,孝宗曰:「日月有盈縮,須隨時修改。」執羔對曰:「舜協時月正日,正為積久不能無差,故協正之。」孝宗問曰:「今歷與古歷何如?」對曰:「堯時冬至日在牽牛,今冬至日在斗一度。」



 孝榮《七曜細行歷》自謂精密,且預定是年四月戊辰朔日食一分,日官言食二分,伯壽並非之,既而精明不食。孝榮又定八月庚戌望月食六分半,候之,止及五分。又定戊子歲二月丁未望月食九分以上,出地,其光復滿。伯壽言:「
 當食既,復滿在戌正三刻。」



 侍御史單時言:「比年太史局以《統元歷》稍差而用《紀元歷》,《紀元》浸差,邇者劉孝榮議改歷,四月朔日食不驗,日官兩用《統元》、《紀元》以定晦朔,二歷之差,歲益已甚,非所以明天道、正人事也。如四月朔之日不食,雖為差誤,然一分之說,猶為近焉。八月望之月食五分,新歷以為食六分,亦為近焉。聞欲以明年二月望月食為驗,是夜或有陰晦風雨,願令日官與孝榮所定七政躔度其說異同者,俟其可驗之時,以渾象
 測之,察其稍近而屢中者,從其說以定歷,庶幾不致甚差。」詔從之。十一月,詔國子司業權禮部侍郎程大昌、監察御史張敦實監太史局驗之。時孝宗務知歷法疏密,詔太史局以高宗所降小渾儀測驗造歷。四年二月十四日丁未望,月食生光復滿,如伯壽言。



 時等又言:「去年承詔,十二月癸卯、乙巳兩夜監測太陰、太白,新歷為近。今年二月十四日望月食,臣與大昌等以渾儀定其光滿,則舊歷差近,新歷差遠。若遽以舊歷為是,則去年所
 測四事皆新歷為近,今者所定月食,乃復稍差,以是知天道之難測。儒者莫肯究心,一付之星翁歷家,其說又不精密。願令繼宗、孝榮等更定三月一日內七政躔度之異同者,仍令臣等往視測驗而造歷焉。」三月,詔時與大昌同驗之。太史局止用《紀元歷》與新歷測驗,未嘗參以《統元歷》。臣等先求判太史局李繼宗、天文官劉孝榮等《統元》、《紀元》、新歷異同,於三月初九日夜、十一日早、十四日夜、二十日早詣太史局,召三歷官上臺,用銅儀窺
 管對測太陰、木、火、土星昏晨度經歷度數,參稽所供,監視測驗。初九日昏度:舊歷太陰在黃道張宿十二度八十七分,在赤道張宿十度;新歷在黃道張宿十四度四十分,在赤道張宿十五度太。臣等驗得在赤道張宿十五度半。今考之新歷稍密,舊歷皆疏。十一日早晨度:木星在黃道室宿十五度七分,在赤道室宿十三度少;土星在黃道虛宿七度三分,在赤道虛宿七度強。新歷木星在黃道室宿十五度四十四分,在赤道室宿十四度
 少弱;土星在黃道虛宿六度二十一分,在赤道虛宿六度少弱。臣等驗得五更三點,土星在赤道虛宿六度弱;五更五點,木星在赤道室宿十四度。今考之新歷稍密,舊歷皆疏。十二日,都省令定驗《統元》、《紀元》及新歷疏密。《統元歷》昏度,太陰在黃道氐宿初度九十四分,在赤道氐宿三度少;《紀元歷》在黃道氐宿初度八十三分,在赤道氐宿二度太;新歷在黃道亢宿八度七十一分,在赤道亢宿九度少弱。三歷官以渾儀由南數之,其太陰北
 去角宿距星二十一度少弱。新舊歷官稱昏度亢宿未見,祗以窺管測定角宿距星,復以歷書考東方七宿,角占十二度,亢占九度少;既亢宿未見,當除角宿十二度,即太陰此時在赤道亢宿九度少弱。今考之新歷全密,《紀元》、《統元歷》皆疏。二十日早晨度:《統元歷》太陰在黃道鬥宿十一度九十一份,在赤道鬥宿十二度少;火星在黃道危宿七度九十一分,在赤道危宿七度少;土星在黃道虛宿八度八十二分,在赤道虛宿八度太強。《紀元
 歷》太陰在黃道鬥宿十一度四十分,在赤道鬥宿十一度半;火星在黃道危宿六度,在赤道危宿六度太;土星在黃道虛宿七度三十九分,在赤道虛宿七度半弱。新歷太陰在黃道鬥宿十度六十一分,在赤道鬥宿十度少;火星在黃道危宿七度二十分,在赤道危宿六度;土星在黃道虛宿六度五十三分,在赤道虛宿六度半。三歷官驗得太陰在赤道鬥宿十度,火星在赤道危宿六度強,土星在赤道虛宿六度半。今考之太陰,《紀元歷》
 疏;火星,新歷、《紀元歷》全密,《統元歷》疏;土星,新歷全密,《紀元》、《統元歷》疏。」



 又詔時與尚書禮部員外郎李燾同測驗,時等言:「先究《統元》、《紀元》、新歷異同,召三歷官上臺,用銅儀窺管對測太陰、土、火、木星晨度經歷度數,參稽所供,監視測驗。二十四日早晨度:《統元歷》太陰在黃道危宿十一度九十分,在赤道危宿九度;木星在黃道室宿十八度一十五分,在赤道壁宿初度少;火星在黃道危宿十度七十分,在赤道危宿十度;土星在黃道虛宿八度
 九十五分,在赤道虛宿九度。《紀元歷》太陰在赤道危宿十度五十三分,在赤道危宿八度半;木星在黃道室宿十七度六十八分,在赤道室宿十四度少;火星在黃道危宿九度八十四分,在赤道危宿九度;土星在黃道留在虛宿七度四十分,在赤道虛宿七度半。新歷太陰在黃道危宿十三度五分,在赤道危宿十二度;木星在黃道室宿十八度一十分,在赤道室宿十六度半強;火星在黃道危宿十度八分,在赤道危宿九度;土星在黃道
 虛宿六度六十分始留,在赤道虛宿六度半強始留。三歷官驗得太陰在赤道危宿十度,木星在赤道室宿十六度太,火星在赤道危宿九度半,土星在赤道虛宿六度半弱。今考之太陰,《統元歷》精密、《紀元歷》、新歷皆疏;木星,新歷稍密,《紀元》、《統元歷》皆疏;火星,《紀元》、新歷皆稍密,《統元歷》疏;土星,新歷稍密,《紀元》、《統元歷》皆疏。二十七日早晨度:《統元歷》木星在黃道壁宿初度四十六分,在赤道壁宿初度太強;火星在黃道危宿十二度九十二分,
 在赤道危宿十二度強;土星留在黃道虛宿八度九十八分,在赤道虛宿九度。《紀元歷》木星在黃道壁宿初度二十五分,在赤道壁宿初度分空;火星在黃道危宿十二度九十七分,在赤道危宿十一度;土星留在黃道虛宿七度四十八分,在赤道虛宿七度半。新歷木星在黃道壁宿初度四十四分,在赤道壁宿初少強;火星在黃道危宿十二度二十二分,在赤道危宿十一度半;土星留在黃道虛宿六度六十分,在赤道虛宿六度半強。三
 歷官驗得木星在赤道壁宿初度少,火星在赤道危宿十一度,土星在赤道虛宿六度半。今觀木星,新歷稍密,《紀元》、《統元歷》皆疏;火星,《紀元歷》全密,《統元》、新歷皆疏;土星,新歷稍密,《紀元》、《統元歷》皆疏。」



 由是朝廷始知三歷異同,乃詔太史局以新舊歷參照行之。禮部言:「新舊歷官互相異同,參照實難,新歷比之舊歷稍密。」詔用新歷,名以《乾道歷》,己丑歲頒行。



 孝榮有《考春秋日食》一卷,《漢魏周隋日月交食》一卷,《唐日月交食》一卷,《宋朝日月交食》
 一卷,《氣朔入行》一卷,《強弱日法格數》
 一
 卷。



\end{pinyinscope}