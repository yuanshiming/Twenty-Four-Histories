\article{志第九十九 儀衛四}

\begin{pinyinscope}

 政和大駕鹵簿並宣和增減小駕附



 政和大駕鹵簿。像六,分左右。次六引:開封令、開封牧、大司樂、少傅、御史大夫、兵部尚書。各用本品鹵簿。



 次金吾纛、槊。左右皂纛各六,執、托各一人,絲斥四人。



 押衙四人,並騎。



 犦槊八,執各一人。



 本衛
 上將軍、將軍各四人,本衛大將軍二人,並騎。



 犦槊四,夾大將軍。執各一人,夾二人,並騎。



 法駕,犦槊減二,本衛上將軍、將軍各減二人。



 次朱雀旗隊。並騎。



 金吾衛折沖都尉一人引隊,犦槊二,夾都尉;執旗一人,引、夾各二人。凡仗內引、夾、執人數準此。



 弩四,弓矢十六,槊二十,左右金吾衛果毅都尉二人押隊。法駕,弩減二,弓矢減六,槊減八。宣和,引隊改天武都指揮使,押隊改天武指揮使。



 次龍旗隊。大將軍一員檢校,騎;引旗十二人,並騎。



 風伯、雨師、雷公、電母旗各一,五星旗五,左、右攝提
 旗二,北斗旗一,護旗十二人,副竿二。執人並騎。



 法駕,引旗、護旗人各減四。宣和,檢校改左右衛大將軍,雷公、電母旗去「公」、「母」二字。



 次指南、記里鼓車各一,駕馬各四,駕士各三十人,白鷺、鸞旗、崇德、皮軒車各一,駕士各十八人。法駕,無白鷺、崇德車。宣和,有青旌、青雀、鳴鳶、飛鴻、虎皮、貔貅六車,在記里鼓之下、崇德之前;減白鷺、鸞旗、皮軒三車,駕士之數如前。



 次金吾引駕,騎;本衛果毅都尉二人,儀刀、弩、弓矢、槊各減二。宣和,改都尉為神勇都指揮使。



 次大晟府前部鼓吹。令二人,府史四人,管押指揮使一人,□鼓、金鉦各十二,帥兵官八人領。



 大鼓一百二十,帥兵官二十人領。



 長鳴一百二十,帥兵官六人領。



 鐃鼓十二,帥兵官四人領。



 歌工、拱宸管、簫、茄各二十四,大橫吹一百二十,帥兵官十人領。



 節鼓二,笛、簫、觱篥、笳、桃皮觱篥各二十四;□鼓、金鉦各十二,帥兵官四人領。



 小鼓、中鳴各一百二十,帥兵官八人領。



 羽葆鼓十二,帥兵官四人領。



 歌工、拱宸管、簫、笳各二十四。法駕,前後□鼓、金鉦各減四,大鼓減四十,長鳴減四十,鐃鼓減四,拱宸管後簫、笳各
 減八,大橫吹減四十,節鼓後笛、簫、觱篥、笳、桃皮觱篥各減八,小鼓、中鳴各減四十,羽葆鼓減四,最後簫、笳各減八,帥兵共減十八人。



 次太史相風、行漏等輿。太史令及令史各一人,並騎。



 相風烏輿一。輿士四人。



 交龍鉦、鼓各一,輿士各六人。



 司辰、典士各一人,並騎。漏刻生四人,鼓樓、鐘樓、行漏輿各一,輿士各一百人。太史正一人,清道二人,十二神輿一。輿士十四人。



 法駕,行漏輿一,輿士各十四人。



 神輿一,輿士多大駕二人。



 宣和,鼓、鐘樓並改為輿,太史正前有捧日副指揮使二人,捧日節
 級十人,神輿輿士增十。



 次持鈒前隊。左右武衛果毅都尉二人引隊,左右武衛校尉二人。絳引幡一,絲斥二人。



 左右有金節十二,執人並騎。



 罕、畢各一,朱雀幢、叉、導蓋,青龍、白虎幢各一,叉三。執人並騎。



 稱長一人,鈒戟二百八十八,左右武衛將軍二人檢校,左右武衛校尉四人押隊。法駕,金節減四,鈒戟減七十二。宣和,引隊改驍騎都指揮使,武衛校尉改驍騎軍使,增朱雀旗後之叉一,去龍虎旗後之叉三,檢校改用左右驍騎將軍。



 次黃麾幡一。執一人,騎;絲斥二人。



 法駕,前有殿中侍御史二員。次六軍儀仗。左右神武軍、左右羽林軍、左右龍武軍,各有統軍二員,都頭二人羽林又有節級二人。



 押仗,本軍旗各一,排闌旗各二十分夾,吏兵、力士旗各五,掩尾天馬旗二,羽林有赤豹、黃熊旗,龍武有龍君、虎君旗各一。



 白柯槍五十,哥舒棒十,鐙仗八。法駕,神武軍減排闌旗十,羽林、龍武軍各減四,吏兵、力士旗各減一。宣和,統軍改軍將,神武軍旗改熊虎,排闌旗改平列,哥舒棒改戈戟,鐙杖改矛戟,羽林隊無節級,黃熊旗改黃羆,龍武旗改熊虎。



 次引駕旗。天王旗二,排仗通直官二人押旗,十二辰旗各一。法駕,同。次龍墀旗。天下太平旗一,排仗大將二人夾旗,五方龍旗各一,金鸞、金鳳旗各一,師子旗二,君王萬歲旗一,日、月旗各一。法駕,減鸞、鳳、師子旗。次御馬二十四。控馬每匹天武二人,御馬直二人,為十二重。法駕,減八,為八重。宣和,御馬直改為習馭。次中道隊。大將軍一員檢校。法駕,同。宣和,大將軍改為左右驍衛大將軍。次日月合璧旗一,苣文旗二,五星連珠旗一,
 祥雲旗二,長壽幢二。宣和,苣文改慶雲,祥雲改祥光。



 次金吾細仗。青龍、白虎旗各一,五嶽神旗、五方神旗、五方龍旗、五方鳳旗各五。已上執各一人,絲斥各三人。



 法駕,五方龍、鳳旗各減二。宣和,改校尉為使臣,五嶽神旗去「神」字。



 次八寶。鎮國神寶、皇帝之寶、皇帝行寶、皇帝信寶在左,受命寶、天子之寶、天子行寶、天子信寶在右,為四重。香案八,各以二列於寶輿之前。碧襴二十四人,符寶郎行於碧襴之間。法駕,減碧襴八人。宣和,增引寶職掌二人,香案職掌六人,援衛傳
 喝親從一百人。奉寶輦官每寶二十八人,節級一人,奉寶一十二人,舁香案、行馬、執燭籠各四人,持席褥、油衣共三人,香案、寶輿各九,燭籠三十六,碧襴之數同前。



 次方傘二,大雉尾扇四夾。執傘、扇各一人,以下準此。



 法駕,同。次金吾四色官六人,押仗二人。法駕,減押仗。次金甲二人。宣和,改為銅甲。次太僕寺進馬四人。並騎。



 次引駕千牛衛上將軍一員,千牛八人,中郎將二人,並乘珂馬。



 千牛二人。並騎。



 宣和,引駕改為千牛衛大將軍,中郎將改為捧日都虞候。次長
 史二人。並騎。



 宣和,無。



 次金吾引駕官四人。並騎。



 次導駕官。執政以上人從六人,待制、諫議、防禦使以上五人,監察御史、刺史、諸衛將軍以上四人。



 次傘扇、輿輦。大傘二,中雉尾扇四夾,腰輿一,小雉尾扇四夾,應奉人員一人,十將、將、虞候、節級二人,長行十六人。排列官二人,中雉尾扇十二,華蓋二,執各二人。



 香鐙一。執擎八人。



 小輿一,應奉人,逍遙、平輦下人,長行二十四人。



 逍遙子一,應奉人,十將、將、虞候、節級共九人,長行二十六人。



 平輦一。應奉人員七人,餘同上。



 法駕,排列官後中雉尾扇減四。宣和,去小雉尾扇四,腰輿一,添管押人員二人,都將四人,僉押小輿排列官二人。



 小
 輿一,奉輿二十四人,都將九人。



 逍遙子改為逍遙輦,奉輦一十六人。



 平輦一,奉輦人同上,後有上輦奉御二人,騎。



 小輿前又有大輅一。駕馬六,太僕卿御,駕士一百二十人。



 次駕前東第五班。開道旗一,皂纛旗十二。引駕六十二人,鈞容直三百人。引駕同作樂。



 五方色龍旗五,門旗四十,御龍四直步執門旗六十。天武駕頭下一十二人,茶酒班執從物一十一人,御龍直仗劍六人,天武把行門八人。麋旗一,殿前班擊鞭一十人,簇輦龍旗八,日、月、麟、鳳旗四,青、白、赤、黑龍旗各一。御龍直四十人,踏路馬二,夾
 輅大將軍二員,進輅職掌二員,部押二人,教馬官二員。法駕,同。宣和,無鈞容直,開道旗內增押班一人,殿侍二人。皂纛旗十二,殿侍十二人執。



 引駕人員二人,長行六十人。五方色吉字旗,殿侍三人,管押十人。門旗,殿侍二人,管押四十人,叉八,門旗六十,御龍直一十二人,骨朵直十二人,御龍弓箭直、弩直各十八人,御龍直仗劍六人,執麋旗殿侍二人,管押龍旗人員二人,都知、副都知各一人,執骨朵殿侍十六人,內大將軍改為千牛衛大將軍,朝服
 步從。



 將軍二人,朝服陪乘。



 掌輦四人。



 皇帝乘玉輅,駕青馬六,駕士一百二十八人,扶駕八人,骨朵直一百三十四人,行門三十五人,分左右,陪乘將軍二員。法駕,同。宣和,駕士增為二百三十四人。



 次奉宸隊。御龍直,左廂骨朵子直、右廂弓箭直,弩直,御龍四直,並以逐班直所管人數列為五重。天武骨朵、大劍三百一十人。次駕後東第五班。大黃龍旗一,鈞容直三十一人。扇筤下天武二十人,茶酒班簇輦三十一人,招箭班三十三人。法駕,同。宣和,止
 用黃龍旗,餘並無。



 次副玉輅一,駕青馬六,駕士四十人。法駕,無。宣和,駕士一百人,內人員二人。次大輦一,掌輦四人,應奉人員十二人,十將、將、虞候、節級共十人,長行三百五十五人。尚輦奉御二人,殿中少監、供奉職官二員,令史四人,書令史四人。法駕,同。宣和增奉輦為九十人。次太僕御馬二十四,為十二重。法駕,減八,為八重。宣和。無太僕。



 次持鈒後隊。左右武衛旅帥二人。法駕,同。宣和,改為神勇都指揮使。次重輪旗二,大傘二,大雉尾扇
 四,小雉尾扇、朱團扇各十二,華蓋二,叉二,睥睨十二,御刀六,真武幢一,絳麾二,叉一,細槊十二。法駕,小雉尾扇、朱團扇、睥睨、槊各減四,華蓋減一,御刀減二。宣和,真武幢改為玄武。次左右金吾衛果毅都尉二人,並騎。



 總領大角一百二十。法駕,減四十。宣和改都尉為驍騎都指揮使。



 次大晟府後部鼓吹。丞二人,典事四人,管轄指揮使一人,羽葆鼓十二,帥兵官四人領。



 歌工、拱宸管、簫、笳各二十四,帥兵官二人領。



 鐃鼓十二,帥兵官四人領。



 歌工、簫、笳各二十四,小橫吹
 一百二十,帥兵官八人領。



 笛、簫、觱篥、笳、桃皮觱篥各二十四。法駕,羽葆鼓減四,簫、笳、笛、觱篥、桃皮觱篥各減八,鐃鼓減四,小橫吹減四十。帥兵官並減二人。



 宣和,帥兵官改為天武、神勇、宣武、虎翼四都頭。



 次黃麾一,執、絲斥人數同前部,法駕亦同,有殿中侍御史二員在黃麾前。芳亭輦一,奉輦六十人。



 鳳輦一,奉輦五十人。



 法駕,去鳳輦。宣和,芳亭奉輦六十二人。



 次金、象、革、木四輅,並有副輅。金輅踏路赤馬二,正副各駕赤馬六,駕士六十人。餘輅正副駕馬數同而色異,像輅以赭白,革輅以騮,木輅以
 黑,駕士各四十人。法駕,無副輅。宣和,駕馬之色又異,金以騮,像以赤,革以赭白,木以烏;駕士五百五十人,副一百人,管押人員各二人。耕根車一,駕青馬六,駕士四十人。法駕,同。宣和,無。進賢車一,駕士二十四人;明遠車一,駕士四十人。法駕,無。宣和,各增駕馬四。次屬車十二乘,每乘駕牛三,駕士十人。法駕,減四乘。宣和,增衙官二人,管押節級一人。次門下、中書、秘書、殿中四省局官各二員。法駕,同。次黃鉞車、豹尾車各一,各駕赤馬二,駕士十五
 人。法駕,除進賢、明遠車外,並同。宣和,有黃鉞天武副都頭及神勇副都頭各一。



 次掩後隊。左右威衛折沖都尉二人領隊,大戟、刀盾、弓矢、槊各五十。法駕,各減十六。宣和,押隊改用宣武都指揮使二人。次真武隊。金吾衛折沖都尉一人,犦槊二,仙童旗一,真武旗一,螣蛇、神龜旗各一,槊二十五,弓矢二十,弩五。法駕,槊減六,弓矢減五,弩減一。宣和,改為玄武隊。改真武為玄武,又聖仙童、龜、蛇旗,改都尉為虎翼都指揮使。



 政和大駕外仗。清游隊。
 次第六引外仗,白澤旗二,左右金吾衛折沖都尉二人,弩八,弓矢三十二,槊四十。法駕,次第三引外仗,弩減二,弓矢減八,槊減十。宣和,改都尉為捧日都指揮使。左右金吾各十六騎,帥兵官二人,弩八,弓矢、槊各十二。法駕,金吾騎及弓矢、槊各減四。宣和,改金吾為天武都頭。



 次佽飛隊。左右金吾衛果毅都尉二人分領,並騎。



 虞候佽飛四十八人,並騎。



 鐵甲佽飛二十四人。並甲騎。



 法駕,前減十八人,後減八人。宣和,改金吾衛為拱聖都指揮使,改都尉為
 都指揮使。



 次前隊殳仗。左右領軍衛將軍二人檢校,並騎。



 犦槊四。殳叉分五隊:第一,一百六十人;第二,八十人;第三,一百人;第四、第五各八十人。逐隊有帥兵官左右領軍衛、左右威衛、左右武衛、左右驍衛、左右衛各四人。法駕,殳叉第一隊減六十,第二、第三各減三十,第四、第五各減二十。宣和,改檢校為左右衛將軍,領軍衛為天武都頭,威衛為神勇都頭,武衛為宣武都頭,驍衛為虎翼都頭;殳叉第一隊減六十,增第二隊至第五隊為一
 百。



 次後隊殳仗。殳叉分五隊:第一、第二,八十人;第三,一百人;第四,八十人;第五,一百六十人。帥兵官,左右衛、左右驍衛、左右武衛、左右威衛、左右領軍衛。凡前後隊殳仗,前接中道北斗旗,後盡鹵簿後隊。法駕,殳叉第一、第二隊各減二十四,第三、第四各減三十,第五減六十。宣和,殳叉各一百,天武、神勇、宣武、虎翼、廣勇都頭。



 次前部馬隊。凡十二,皆以都尉二人分領。第一,前左右金吾衛折沖領,角、亢、斗、牛宿旗四,弩十,弓矢二十,槊四十。第二,
 氐、房、女、虛宿旗四;第三,心、危宿旗,第四,尾、室宿旗各二。以上三隊,各以左右領軍衛果毅領。第五,箕、壁宿旗,第六,奎、井宿旗各二,各以左右威衛折沖領。第七,婁、鬼宿旗,第八,胃、柳宿旗,第九,昴、星宿旗各二,各以左右武衛果毅領。第十,畢、張宿旗,第十一,觜、翼宿旗,第十二,參、軫宿旗各二,各以左右驍衛折沖領。弩、弓矢、槊人數,同第一隊。法駕,分二十八宿旗為十隊,逐隊弩減四,弓矢減六,槊減二十。宣和,捧日、拱聖、神勇、驍衛、宣武五都指揮使,分
 領上十隊,以虎翼、廣勇都指揮使,分領下二隊。



 次步甲前隊。凡十二,左右領軍衛將軍二人檢校,並騎。犦槊四,逐隊皆有都尉二人分領。第一、第三各以左右領軍衛,第五以左右威衛,第七以左右武衛,第九以左右驍衛,第十一以左右衛,並折沖;第二、第四各以左右領軍衛,第六以左右威衛,第八以左右武衛,第十以左右驍衛,第十二以左右衛,並果毅。內有鶡、貔、玉馬、三角獸、黃鹿、飛麟、駃騠、鸞、麟、馴象、玉兔、闢邪等旗各二,以序居都尉之
 後。逐隊有弓矢、刀盾相間,各六十人,居旗之後。法駕,止十隊,每隊弓矢各減二十。宣和,檢校改用左右衛將軍,又去皞槊,分領並改為都指揮使:第一、第二並捧日,第三、第四並天武,第五、第六並拱聖,第七、第八並神勇,第九驍騎,第十宣武,第十一虎翼,第十二廣勇。



 次前部黃麾仗。絳引幡二十,下分六部:第一,左右威衛;第二,左右領軍衛;第三,左右威衛;第四,左右武衛;第五,左右驍衛;第六,左右衛。諸部各有殿中侍御史兩員,本衛大將軍
 二人檢校,本衛折沖都尉二人分領。又各有帥兵官二十人。龍頭竿六重,重各二十;揭鼓三重,重各二;儀鍠五色幡、小戟、槊各一重,重各二十;弓矢二重,重各二十;朱綠縢絡盾並刀二重,重各二十。法駕,止五部,絳引幡、帥兵官、龍頭竿、幡、戟、弓矢、盾刀、槊並減六。宣和,六部:驍衛、武衛、屯衛、領軍衛、監門衛、千牛衛,皆左右上將軍;天武、神勇、宣武、虎翼、廣勇,皆都指揮、都頭;逐部上將軍、都頭各一人。



 次青龍、白虎旗各一,左右衛果毅都尉二人,分
 押旗及領後七十騎,弩八,弓矢二十二,槊四十。法駕,減後騎三十,弩減二,弓矢減八,槊減二十。宣和,改都尉為虎翼都指揮使。



 次班劍、儀刀隊。並騎。左右衛將軍二人分領,郎將二十四人,左右親衛、勛衛各四人,每衛班劍二百二十人;諸翊衛左右衛六人,領儀刀四百八人;左右驍衛二人,領儀刀一百三十六人。左右武衛、左右威衛、左右領軍衛、左右金吾衛各二人。法駕,親、勛衛班劍減八十四人,翊衛儀刀減一百三十二人,增左右驍衛
 四人,班劍、儀刀九十二仕。宣和,分領改左右武衛將軍及捧日、天武指揮四人,拱聖六人,神勇、驍騎、驍勝、宣武、虎翼指揮使各二人。



 次親勛、散手、驍衛翊衛隊。並騎。



 左右衛供奉中郎將四人,分領親勛翊衛四十八人;左右衛郎將二人,分領散手翊衛六十人;左右驍衛郎將二人,分領驍衛翊衛五十六人。法駕,親勛減十六人,散手、驍衛各減二十人。宣和,改為中衛、翊衛、親衛隊,中衛郎四人,分領衛兵四十八人;翊衛郎二人,分領衛兵六十人;
 親衛郎二人,分領衛兵五十六人。



 次左右驍衛翊衛三隊。並騎。



 各有二人分領,第一本衛大將軍,第二本衛將軍,第三本衛郎將;花鳳、飛黃、吉利旗各二,分為三隊;逐隊弩十,弓矢二十,槊四十。法駕,弩減四,弓矢、槊各減半。宣和,分領第一、第二隊,左右驍衛大將軍、將軍;第三,廣勇指揮使。改花鳳旗為雙蓮旗。



 次來轂隊。凡六,逐隊都尉二人檢校,第一、第四左右衛折沖,第二、第三、第五、第六並左右衛果毅。逐隊刀盾各六十人,內第一、第四有寶符旗
 二。法駕,各減刀盾二十。宣和,檢校改為捧日、天武、拱聖三指揮使。



 次捧日隊。逐隊引一人,押二人,長行殿侍二十八人,旗頭三人,槍手五人,弓箭手二十人,左右廂天武約攔各一百五十五人。法駕,同。



 次後部黃麾仗。分六部:左右衛、左右驍衛、左右武衛、左右威衛、左右領軍衛、左右武衛。部內殿中侍御史、大將軍、都尉、帥兵官、絳引幡、龍頭竿等,並同前部。法駕,減第六部,絳引幡減六。宣和,六部:第一改為左右驍衛大將軍,自二至六改為天
 武、神勇、宣武、虎翼、廣勇五指揮。



 次步甲後隊。凡十二,皆有都尉二人分領。第一以左右衛,第三以左右驍衛,第五以左右武衛,第七以左右威衛,第九、第十一各以左右領軍衛,以上並果毅;第二以左右衛,第四以左右驍衛,第六以左右武衛,第八以左右威衛,第十、第十二各以左右領軍衛,以上並折沖。內有貔、鶡雞、仙鹿、金鸚鵡、瑞麥、孔雀、野馬、犛牛、甘露、網子、祥光、翔鶴等旗各二,以序居都尉之後。逐隊有弓矢、刀盾相間,各六十人,居旗
 之後。法駕,止十隊。宣和,自第七隊以下,分領改用都指揮使,七、八並神勇,九驍騎,十宣武,十一虎翼,十二廣勇。旗亦改其半,七天正堯瑞,八日有戴承,十翔鶴,十一紅光,十二文石。



 次後部馬隊。凡十二,皆以都尉二人分領。第一、第二各以左右衛,第五、第六、第七各以左右武衛,第十至十一、十二各以左右領軍衛,並折沖;第三、第四各以左右驍衛,第八、第九各以左右威衛,並果毅。內有角、赤熊、兕、天下太平、馴犀、鵔鸃、轆□蜀、騶牙、蒼烏、白狼、
 龍、虎、金牛等旗各二,以序居都尉之後。每隊弩十,弓矢二十,槊四十。法駕,止十隊。弩減四,弓矢減六,槊減十二。宣和,改都尉為指揮使,一、二並以捧日,三、四並以天武,五、六並以拱聖,七、八並以神勇,九以驍騎,十以宣武,十一以虎翼,十二以廣勇。內六有芝禾並秀旗,七有萬年連理木旗。



 以上鹵簿,凡門有六,中道之門二:第一門居日月合璧等旗之後,法駕,居龍墀旗之後;第二門居掩後隊之後,法駕,同。各有金吾衙門旗四,監門校尉六
 人。左右道之門四:第一,居步甲前隊第六隊之後;第二,居第十二隊之後;第三,居夾轂隊之後;第四,居步甲後隊第六隊之後。法駕,同。各有監門校尉四人。宣和,改校尉為使臣。



 政和小駕,減大駕六引及象、木、革輅,五副輅,小輿,小輦,又減指南、記里、白鷺、鸞旗、崇德、皮軒、耕根、進賢、明遠、黃鉞、豹尾、屬車等十一,餘並減大駕之半。



\end{pinyinscope}