\article{列傳第二百二十三外戚中}

\begin{pinyinscope}

 ○王貽永李昭亮李用和子璋瑋珣李遵勖子端懿端願端愨端願子評曹佾從弟偕子評誘高遵裕弟遵惠從侄士林士林子公紀公紀子世則
 向傳範從侄經綜經子宗回宗良張敦禮任澤



 王貽永字季長,溥之孫也。性清慎寡言,頗通書,不好聲技。初生十餘歲時,其舅魏咸信見而奇之,曰:「後當類我。」



 咸平中,尚鄭國公主,授右衛將軍、駙馬都尉。從封泰山,領高州刺史,再遷右監門衛大將軍、獎州團練使。求外補,得知單州。真宗戒之曰:「和眾靜治,卿所當先也。」真拜洺州團練使、徙徐州。河決滑州,徐大水,貽永作堤城南以御之。改衛州團練使,進懷州防禦使,知澶、定二州,徙
 成德軍。



 會有告曹訥變者,貽永奏治之。遷耀州觀察使,復知澶州。歷彰化、武定軍節度使觀察留後,拜安德軍節度使。出知天雄軍,徙保寧軍節度使、知鄆州。州自咸平中徙城,而故治為通衢,介梁山,春夏多水患,貽永相度地勢,為築東西道三十餘里,民便之。復徙定州,又徙成德軍。擢同知樞密院事,改副使,加宣徽南院使,進樞密院使。久之,拜同中書門下平章事,遂加兼侍中。



 徙節鎮海,以疾求罷,手詔撫諭,遣上醫診視。帝臨問,頒尚方
 珍藥,手取糜粥食之。貽永自言寵祿過盛,願罷樞筦,解使相還第。帝冀其愈也,乃聽罷侍中,徙彰德節度使,同平章事、樞密使如故。疾稍間,入見,命其子道卿掖登垂拱殿。仍賜五日一朝,遇朝參起居,許休於殿側。至和初,復以疾辭,拜尚書右僕射、檢校太師兼侍中、景靈宮使。卒,贈太師、中書令,謚康靖。



 當時無外姻輔政者,貽永能遠權勢,在樞密十五年,迄無過失,人稱其謙靜。



 子道卿,西上閣門使。



 李昭亮,字晦之,明德太后兄繼隆子也。四歲,補東頭供奉官,許出入禁中。繼隆北征契丹,遣昭亮持詔軍中。問方略及營陣眾寡之勢,昭亮年雖少,還奏稱旨。累遷西上閣門使。出為潞州兵馬鈐轄,徙領麟府路軍馬事,尋為管勾軍頭引見司兼三司衙司。軍士有逃死而冒請官廩者數百人,昭亮按發之。領高州刺史,知代州。以四方館使復領麟府路軍馬事。遷引進使,領賀州團練使。歷知瀛定二州、成州團練使、寧州防禦使、延州觀察使、
 感德軍節度觀察留後。擢殿前都虞候、秦鳳路馬步軍副都總管、經略招討副使。徙永興路馬步軍副都指揮使、並代州路副都總管、安撫招討副使。未幾,守代州,再徙真定路都總管。



 保州兵叛,殺官吏,詔遣王果招降之,叛者乘埤呼曰:「得李步軍來,我降矣。」於是遣昭亮,昭亮從輕騎數十人,不持甲盾弓矢,叩城門呼城上曰:「爾輩第來降,我保其無虞也。不爾,幾無噍類矣。」卒稍稍縋城下。明日,相率開城門降。改淮康軍節度觀察留後,復知
 定州,敕使存勞,賜黃金三百兩,給節度使奉,以褒其功。都轉運使歐陽修言:「昭亮入保州,以叛卒女口分隸諸軍,有輒私入其家者。」置不問。



 明年,拜武寧軍節度使,代李用和為殿前副都指揮使。時承平久,將士多因循樂縱弛。昭亮本將家子,雖以恩澤進,然習軍中事,既統宿衛,政尚嚴,多所建請。萬勝、龍猛軍蒲博爭勝負,徹屋椽相擊,士皆惶駭,昭亮捕斬之,杖其主者,諸軍為之股心慄。帝祠南郊,有騎卒亡所挾弓,會赦,當釋去,昭亮曰:「宿衛
 不謹,不可貸。」卒配隸下軍,禁兵自是頓肅。



 以宣徽北院使判河陽,徙延州。以南院使判澶州,徙並州、成德軍,拜同中書門下平章事,判大名府。仁宗以塗金紋羅書曰:「李昭亮親賢勛舊。」命其子惟賢持以賜。徙定州,改天平、彰信、泰寧軍節度使。在定州數言老疾不任邊事,願還京師,乃以為景靈宮使,又改昭德軍節度使。卒,贈中書令,謚良僖。



 昭亮為人和易,練習近事,於吏治頗通敏,善委任僚佐,以故數更藩鎮無他過。昭亮妻早亡,內嬖三
 妾迭預家政,莫能制也。



 子惟賢,字寶臣,以父蔭為三班奉職,後為閣門祗候、通事舍人。累遷西上閣門使,尋領高州刺史、知莫州,州倉粟陳腐,戍兵大噪,弗肯受,州人皆恐,惟賢馳往諭曰:「邊兵眾則積粟多,廩數多且積久,能無陳腐乎?欲盡取新,則陳者何所歸?」遂斬首惡一人,流十人,軍中帖然。召還,提舉諸司庫務,領榮州團練使、知冀州。會遷補禁軍,自隸籍後犯贓污者皆絀為下軍,惟賢曰:「武士何可責以廉節?且抵罪在昔,今不可以新
 令繩之。」帝為更其制,徙恩州,後遷四方館使,卒。惟賢善宣辭令,習朝儀,仁宗頗愛之。



 李用和,字審禮,章懿皇太后弟也。少窮困,居京師鑿紙錢為業,劉美求用和於民間,奏為三班奉職。累遷右侍禁、閣門祗候、權提點在京倉草場、考城縣兵馬都監。



 太后崩,詔赴喪。既葬,遷禮賓副使,領八作司。遷禮賓使,同領皇城司。遷崇儀使、賀州刺史。改葬太后於永安,領捧日、天武兵護梓宮。



 明年春,又詔乘傳行太后
 陵。還,授寧州刺史。歷遷澤州團練、慶州防禦、鄜州觀察使。既而擢殿前都虞候、鄜延路馬步軍副都總管。未行,拜永清軍節度觀察留後,改真定府、定州路。舊制,刺史以上所賜公使錢得私入,而用和悉用為軍費。歷侍衛親軍步軍馬軍副都指揮使,拜建武軍節度使、殿前副都指揮使。以老乞罷軍職,拜宣徽北院使。逾月,改彰信軍節度使、同中書門下平章事、景靈宮使。以疾告,仁宗臨問,賜銀飾肩輿,進兼侍中。



 初,未有居第,詔寓館芳林園,用和固
 辭,又假以惠寧坊之官第。病革,帝入見臥內,擢其次子珣為閣門使,賜所居第,並日給官舍僦錢五千。既卒,帝哭之慟,贈太師、中書令、隴西郡王,輟朝五日,制服禁中,謚恭僖,帝撰神道碑,書曰「親賢之碑」。其妻卒,亦輟朝成服。



 初,仁宗以太后不逮養,故外家褒寵特厚。用和列位將相,能小心靜默,推遠權勢,論者以此稱之。子璋。



 璋字公明,以章懿皇后恩,補三班借職,積官為天平軍節度觀察留後,知澶州。護塞商胡,會河漲,訛言水且至,
 璋據廳事自若,人心乃安,河亦不溢。徙曹州觀察使,累遷武勝軍節度使、殿前都指揮使。仁宗書「忠孝李璋」字並秘書賜之。宴近臣群玉殿,酒半,命大盞二,飲韓琦及璋,如有所屬。帝崩,執政欲增京城甲士,璋曰:「例出累代,不宜輒易。」時禁衛相告乾興故事,內給食物中有金,既而果賜食,眾視食中,璋曰:「天子未臨政已優賞,汝何功復云云,敢喧者斬!」眾乃定。



 以武成軍節度使知鄆州。京東盜白日殺縣令,略人道中,璋信賞罰擒捕,盜為衰止。
 歲大雨水,競以船筏邀利,多溺死者,璋一切籍之,約所勝載如黃河法。發卒城州西關,調夫修路數十里,夾道植柳,人指為「李公柳」。知鄧州,坐失舉,改節振武軍,知郢州。還朝,道卒,年五十三。贈太尉,謚曰良惠。弟瑋、珣。



 瑋,選尚兗國公主,積官濮州團練使。以樸陋與主不協,所生母又忤主意,主入訴禁中,瑋皇恐自劾,坐罰金。後數年,終不協,主還宮。瑋自安州觀察使降建州,落駙馬都尉,知衛州。未幾,主徙封岐國,復瑋都尉。主薨,以奉主
 亡狀,貶郴州團練使、陳州安置。遇赦還京師,至建武軍節度使、檢校太師,卒。哲宗臨奠,哭之,贈太師、中書令。



 珣字公粹,以蔭為閣門祗候。時兄璋為閣門副使,珣又求通事舍人,仁宗曰:「爵賞所以與天下共也,儻盡用親戚,何以待勛舊乎?」後一年乃命之。



 車駕視用和疾,自西上閣門副使累遷均州防禦使,知相州,賜禦制詩、飛白字寵其行。未幾,遷相州觀察使。時劉永年亦同除官,知制誥楊畋以為不可開僥幸之門,詔他舍人草制,御史
 範鎮復論之,命遂寢。



 使契丹,預釣魚會,獲多。契丹遺以金器,使還,悉上之,更賜黃金及「李珣忠孝」字。



 熙寧中,遷宣州觀察使、知潁州,哲宗初,進泰寧軍留後,提舉萬壽觀。故事,正任遇覃恩止移鎮,唯宗室乃遷官。至是,珣與李端愨皆特遷,戚里一覃恩遷官自此始。復知相州,卒,年七十四。



 李遵勖字公武,崇矩孫,繼昌子也。生數歲,相者曰:「是當以姻戚貴。」少學騎射,馳冰雪間,馬逸,墜崖下,眾以為死,
 遵勖徐起,亡恙也。



 及長,好為文詞,舉進士。大中祥符間,召對便殿,尚萬壽長公主。初名勖,帝益「遵」字,升其行為崇矩子。授左龍武將軍、駙馬都尉,賜第永寧里。主下嫁,而所居堂甃或瓦甓多為鸞鳳狀,遵勖令UI去;主服有龍飾,悉屏藏之,帝嘆喜。



 領澄州刺史,坐私主乳母,謫均州團練使,徙蔡州。逾年,起為太子左衛率府副率,復左龍武軍將軍,領宏州團練使,真拜康州團練使,給觀察使祿。時繼昌官刺史,遵勖請班其下,許之。後繼昌守涇州,
 暴感風眩,遵勖馳省不俟命,帝遣使令乘驛赴之。既還,上表自劾,帝使輔臣慰諭之。



 遷澤州防禦使,又遷宣州觀察使。求補郡自試,出知澶州,賜宴長春殿。在郡,會河水溢,將壞浮梁,遵勖督工徒,七日而堤成。遷昭德軍節度觀察留後,拜寧國軍節度使,徙鎮國軍、知許州。水軍多不練習而隸籍,遵勖命部校按劾,拔去十七八。復以疾請援唐韋嗣立故事,求山林號,詔不許。



 初,天聖間,章獻太后屏左右問曰:「人有何言?」遵勖不答。太后固問之,
 遵勖曰:「臣無他聞,但人言天子既冠,太后宜以時還政。」太后曰:「我非戀此,但帝少,內侍多,恐未能制之也。」嘗上三說五事以論時政。晉國夫人林氏,以太后乳母,多干預國事,太后崩,遵勖密請置之別院,出入伺察之,以厭服眾論。其補助居多類此。



 所居第園池冠京城。嗜奇石,募人載送,有自千里至者。構堂引水,環以佳木,延一時名士大夫與宴樂。師楊億為文,億卒,為制服。及知許州,奠億之墓,慟哭而返。又與劉筠相友善,筠卒,存恤
 其家。通釋氏學,將死,與浮圖楚圓為偈頌。卒,贈中書令,謚曰和文。有《間宴集》二十卷,《外館芳題》七卷。子端懿。



 端懿字元伯,性和厚,喜問學,頗通陰陽、醫術、星經、地理之學。七歲,授如京副使。侍真宗東宮,尤所親愛,嘗解方玉帶賜之。稍長,出入宮禁如家人。



 七遷濟州防禦使,為群牧副使。杜衍為樞密,擇外戚子弟試外官,乃以端懿知冀州。為政循法度,民愛其不擾。轉運使移州捕妖人李教,教已死。恩州王則據城叛,人有言教不死,在賊軍
 中。遂降單州團練使、知均州,改滑州兵馬鈐轄。賊平,實無李教者,乃以為汝州防禦使、提舉在京諸司庫務。



 遷蔡州觀察使、同勾當三班院。徙華州觀察使。以母喪,起復為鎮國軍節度觀察留後,願終制,許之,仍給全奉。服除,提舉集禧觀,出知鄆州兼京東西路安撫使。是歲,京東水,民多饑,大發倉廩以賑之。置弓手局,教以戰鬥,遂如精兵。治汶陽堤百餘里,以卻水患,民便之。



 尋除寧遠軍節度使、知澶州。御史中丞韓絳奏端懿無功,不當
 得旄節,不拜。以留後赴澶州,數月卒。訃聞,帝方宴禁中,為徹樂,贈其家黃金三百兩,贈感德軍節度使,謚良定,再贈兼侍中。



 端懿能自刻厲,聞善士,傾身下之,以故士大夫與之游,甚得名譽。弟端願。



 端願字公謹,以穆獻公主恩,七歲授如京副使,四遷為恩州團練使。仁宗以歲旱,御便殿慮囚,放宮女。端願上疏,謂:「縱釋有罪,小人之幸;放宮女為宦者專制,反失所歸,何以弭災變?」



 累進邢州觀察使、鎮東軍留後,知襄、郢
 二州。本路轉運使獻羨財數十萬被賞,端願言常賦三折,其民不堪,即上其事。帝怒,奪轉運使賞,申折變之禁。移廬州,富弼謂曰:「肥上之政何以減於襄陽?」端願曰:「初官喜事,飾廚傳以於名,則譽者至;更事既久,知抑豪強、制猾吏,故毀隨之。」弼深然其言。



 英宗初,同提舉在京諸司庫務。帝以疾拱默,端願求對,進曰:「陛下當躬攬權綱,以系人心,不宜退托,失天下望。」拜武康軍節度使、知相州。請歸,除醴泉觀使。



 神宗即位,遣使就其家錄取異時
 章奏,賜詔褒之。河東城囉兀,端願手寫趙普《諫太宗北伐疏》以聞。



 連年請老,以太子少保致仕。凡大禮成,賜金帶、器幣,品數視執政。哲宗嗣位,進太子太保。欽聖皇后以甥舅之故,嘗幸其第,致禮於獻穆祠堂,命近侍掖端願勿拜。元祐六年,卒,帝輟朝臨奠,賻典加等,贈開府儀同三司。弟端愨,子評。



 端愨字守道,官左藏庫使,執獻穆喪,辭起復,詔特給奉。累遷東上閣門使、乾辦三班院。嘗侍宴群玉殿,仁宗
 獨賜珠花、飛白字,寵顧特異。知邢、冀、衛三州,至蔡州觀察使。元祐中,以安德軍留後卒,贈昭德軍節度使,謚曰恭敏。



 兄端懿,在嘉祐時嘗密請建儲,人無知者,卒於澶淵,端愨走護其喪以歸。元豐間因進對,袖舊稿上之,神宗嘆曰:「近世之賢戚也。」由是端愨之名益著。



 評字持正,由東頭供奉官八遷皇城使。以父告老,授西上閣門使,為樞密都承旨。出使陜西、河東,還,言鄜延之人皆謂城囉兀非便,乞速毀撤,解一路之患。師出安南,
 調兵及河東,又言王師南征,而取卒於西北,使蠻聞之,得以窺我。所論事頗多,或見施行。然天資刻薄,招權不忌,多布耳目,採聽外事自效以為忠。僥幸進用,中外仄目。



 以榮州刺史出知穎州,還,乾當三班院。副韓縝報聘契丹,且分畫河東地界,凡二年乃決。賜袍帶、金帛以賞勞。進成州團練使、知蔡州。卒,年五十二。贈冀州觀察使,賜白金千兩。



 評少涉書傳,嘗以公主遺奏召試學士院,改殿中丞,意不滿,辭之。後二年再召試,復止遷一官,愈
 不悅,至上書辨論。及卒,人無憐者。



 曹佾字公伯,韓王彬之孫,慈聖光獻皇后弟也。性和易,美儀度,通音律,善奕射,喜為詩。自右班殿直累進殿前都虞候、安化軍留後。言者謂年未四十毋典軍,出知澶、青、許三州,徙河陽。以建武軍節度使為宣徽北院使,知鄆州,改保靜、保平軍節度使、同中書門下平章事、景靈宮使,加兼侍中,封濟陽郡王。



 神宗每咨訪以政,然退朝終日,語不及公事。帝謂大臣曰:「曹王雖用近親貴,而端
 拱寡過,善自保,真純臣也!」進對未嘗名。元豐中以疾告,既愈,入謝,帝曰:「舅久不覿太皇太后,宜少憩內東門,朕當自啟。」已而召入,歷上下儒釋道五閣、大椿蟠桃亭,再升殿乃退。以護國軍節度使、司徒兼中書令為中太一宮使,給朱衣雙引騎吏前馬。



 慈聖喪終,請郡,帝曰:「時見舅如面慶壽宮,奈何欲遠朕,得無禮遇有不至乎?」佾皇恐。即城南為園池,給八作兵庀役,疏惠民河水灌之,且將為築三百楹第,固辭乃止。高麗獻玉帶,為秋蘆白鷺紋
 極精巧,詔後苑工以黃金仿其制為帶,賜佾。生日,賚予如宰相、親王,用教坊樂工服色衣侑酒,以示尊寵。



 哲宗即位,加少保。坤成節獻壽,特綴宰相班,優詔減拜。卒,年七十二,贈太師,追封沂王。從弟偕,子評、誘。



 偕字光道,少讀書知義,以節俠自喜。為許州都監,幕客史沆傾險,劫持為不法,上下畏之。偕從容置酒,對客數沆十罪,將擊殺之,沆起拜謝,偕罵曰:「復不改,必殺汝。」沆為斂跡。累遷東上閣門使、帶御器械、知雄州。議者欲廢
 塘濼為田,偕曰:「何承矩、李允則營此累年,所以限契丹,廢之不可。」進華州防禦使、知相州,徙河陽總管,卒。嘗從梅堯臣學詩,堯臣稱之,為序其詩。



 評字公正,以父任累官至引進使,知審官西院,積遷溫州防禦使。元祐中,提舉萬壽觀,丐外,樞密院白為真定路鈐轄,哲宗曰:「先帝待慈聖家極厚,其以為總管。」徽宗即位,遷相州觀察使,歷龍神衛捧日天武都指揮使、殿前都虞候、馬步軍副都指揮使、寧遠軍留後、平海軍節
 度使、祐神觀使。使契丹者四,館伴者十二。在閣門十二年,預修儀制,多所增損。



 性喜文史,書有楷法。慈聖命書屏以奉,神宗即賜玉帶旌其能。尤善射,左右手如一,夜或滅燭能中。伴契丹使者射,嘗雙破的,客驚竦。在戚里號為湛厚。卒,年六十六,贈開府儀同三司。



 誘字公善,以蔭至左藏庫副使。熙寧中,父佾以疾告入謝,神宗面授誘閣門通事舍人。元祐中,以東上閣門使為真定府、定州路兵馬鈐轄,遷文州刺史。



 使契丹,至其
 宮門,館客者下馬邀誘同入,誘曰:「北朝使至,及朝堂門,兩朝積好久,無妄生事。」卒乘馬入。使還,為樞密副都承旨。徽宗時,進都承旨。歷慶州團練、恩州防禦、晉州觀察使,保慶軍留後。大觀中,進安德軍節度使、醴泉觀使。與兄評同日拜,立雙節堂於家,戚裏榮之。



 性謹密,習熟典故。卒,年六十五,贈開府儀同三司,謚曰忠定。



 高遵裕,字公綽,忠武軍節度使瓊之孫也。以父任累遷供備庫副使、鎮戎軍駐泊都監。夏人寇大順城,諒祚中
 矢遁。會英宗晏駕,遣遵裕告哀,抵宥州下宮,夏人遣王盥受命,以吉服至,遵裕切責之,遂易服。既而具食上宮,語及大順城事,盥曰:「剽掠輩耳。」遵裕曰:「若主寇邊,扶傷而循,斯言非妄邪!」夏人以為辱,亟遣人代對,終食不敢發口,輒忿怒曰:「王人蔑視下國,弊邑雖小,控弦十數萬,亦能躬執橐鞬,與君周旋。」遵裕瞋目曰:「主上天縱神武,毋肆狂蹶,以干誅夷。」時諒祚覘於屏間,搖手使止。神宗聞而嘉之,擢知保安軍。



 橫山豪欲向化,帝使遵裕諭種
 諤圖之。諤遂取綏州。帥怒諤擅發兵,欲正軍法,諤懼,稱得密旨於遵裕,故諤被罪,遵裕亦降為乾州都監。遷通事舍人,主管西路羌部,駐古渭砦,分所部羌兵為三等,教為軍法。



 熙寧初,朝廷用王韶復洮、隴,命為秦鳳路沿邊安撫,以遵裕副之。尋以古渭為通遠軍,命知軍事。明年,持附順羌部圖籍及繪青唐、武勝形勢入獻,擢引進副使、帶御器械,俾歸治師。師次慶平堡,夜行,晨至野人關,羌人旅拒,引親兵一鼓破之。進營武勝城下,羌眾逃
 去,遂據其城。詔建為鎮洮軍,又命知軍事。尋以熙、河、洮、岷、通遠為一路,進西上閣門使、榮州刺史,充總管,復知通遠軍。



 明年,韶欲取河州,遵裕曰:「古渭舉事,先建堡砦,以漸而進,故一舉拔武勝。今兵與糧未備,一旦越數舍圖人之地,使彼阻要害,我軍進退無所矣。」韶與李憲笑曰:「君何遽相異邪?」檄使守臨洮。韶攻河州,果不克。帝善遵裕議,令專管洮、岷、疊、巖未款附者。



 遵裕以俞龍珂地有鹽井,遂築鹽川砦。瞎吳叱率諸羌脅青唐,欲擾邊,詔
 遣張玉攻討。遵裕曰:「青唐無罪,第為生羌所脅耳。」遣裨將與龍珂率眾御之。青唐人見龍珂泣訴,瞎吳叱知不附己,潰去。從韶取岷州,下之,令士眾曰:「生獲老幼與得級同。」全活者以數萬。捷聞,加岷州刺史。



 明年,羌乘景思立之敗,圍河、岷二州,道路不通者幾月。或請退保,遵裕曰:「敢議此者斬!」岷城軍缺,守者恐,遵裕登西門,命將縱擊,別選精騎由南門噪而出,合擊之,羌敗走。時朝廷以岷城遠難守,議棄之。詔至,賊已潰矣。以功進團練使、龍
 神衛都指揮使、知熙州。坐薦張穆之為轉運使,而穆之有罪,罷知潁州,未幾,徙慶州,又坐事黜知淮陽軍。



 元豐四年,復知慶州。詔與諸路討夏國。請濟師,得東兵十一將,騎不足用,以群牧馬益之。又令節制涇原兵,劉昌祚先至靈州,幾得城,遵裕嫉之,故不用其計,遂以潰歸,語在《昌祚傳》。貶郢州團練副使。



 哲宗即位,復右屯衛將軍,主管中嶽廟。卒,年六十,贈永州團練使。紹聖中,崇贈奉國軍節度觀察留後。從弟遵惠。



 遵惠字子育,以蔭為供奉官。熙寧中,試經義中選,換大理評事。歷三班院主簿、軍器丞。



 元祐初,上疏言:「法度更張,事有當否,如先帝所施設,未可輕議。」擢太僕少卿,上太府卿,出知河中府,改河北路都轉運使,未行,拜工部侍郎,以集賢殿修撰知鄆州、河南、穎昌府,加寶文閣待制、知成德軍。召為戶部侍郎,以龍圖閣學士知慶州。卒,年五十八,贈樞密直學士。



 方宣仁後臨朝,繩檢族人一以法度,乃舉家事付遵惠,遵惠躬表率之,人無間言。亦
 能遠嫌自保,故不罹紹聖之禍。從侄士林。



 士林字才卿,宣仁聖烈皇后之弟也。累官內殿崇班、殿直,英宗書「謹守法律」四字誨之曰:「能此則為良吏矣。」每欲進擢,後屢辭輒止。喜儒學,涉閱經史,通大義,尤有巧智。嘗監揚州召伯閘稅,木舊用火印,士林改刃其印文,鑿以為識,尤簡便,傍郡皆效焉。卒,贈德州刺史。神宗立,加贈昭德軍節度使。紹興初,追封普安郡王。子公紀。



 公紀字君正,歷閣門祗候、通事舍人,累進寧州刺史、團
 練使、永州防禦使、集慶留後。性儉約,珍異聲伎無所好,奉祿多以給諸族,得任子恩,均及孤遠。持宣仁後喪未終,卒,贈感德軍節度使,謚曰懷僖。紹興初,追封新興郡王。子世則。



 世則字仲貽,幼以恩補左班殿直,至內殿崇班。復用父遺表恩為閣門祗候,後除親衛郎。以通經典,轉內殿承制。累遷康州防禦使,知西上閣門事。



 宣和末,金泛使至,徽宗命世則掌客。世則記問該洽,應對有據,帝聞,悅之,
 自是掌客多命世則。金人軍城下,又命世則使其軍,還,進秩二等,遷知東上閣門使。金遣燕人吳孝民請和,孝民邀宰執、親王詣軍前議事,高宗在康邸,請行。是日,世則入對,遂除計議副使以從。康王復使河北,世則改華州觀察使,充參議官。召對,賜金帶。



 當高宗艱難中,世則嘗在左右,寢處不少離。大元帥府建,改元帥府參議官,因請布檄諸路,以定人心。進遙郡承宣使,不拜。高宗承制,轉越州觀察使。及即位,除保靜軍承宣使,提舉萬壽
 觀。詔令編類元帥府事跡付史館,召為樞密都承旨兼提舉京畿監牧,再提舉萬壽觀。



 世則居溫州,帝遣中使諭守臣以時給奉祿,凡積二萬緡,因請以裨郡費。常病瘍,艱於據鞍,又以舊所御肩輿賜焉。帝每念宣仁聖烈皇後保祐三朝,中遭誣詆,外家班秩無顯者,制以為感德軍節度使,充萬壽觀使,進開府儀同三司,奉朝請,賜第臨安。除景靈宮使,兼判溫州。尋以病丐罷,後為萬壽觀使。十四年,召入覲,進少保,懇求還。卒,年六十五,贈太
 傅,賜田三十頃,謚曰忠節。



 向傳範,字仲模,尚書左僕射敏中之子。以父任為衛尉丞。娶南陽郡王惟吉女,改內殿崇班、帶御器械,歷知相、恩、邢三州。入管幹客省、閣門、皇城司。知陜州,仁宗賜詩以寵其行。



 熙寧初,知鄆州兼京東西路安撫使。諫官楊繪言:「傳範領安撫使,無以杜外戚僥求之源。」樞密使文彥博曰:「傳範累典郡,非緣外戚。」神宗曰:「得諫官如此言,甚善,可以止他日妄求者。」以密州觀察使卒,賜昭德軍節
 度使,謚曰惠節。



 傳範,宰相子,聯戚里,所至有能稱。以橐中貲千餘萬葬族人在殯者六十四喪。從侄經、綜。



 經字審禮,以蔭至虞部員外郎。神宗為穎王,選經女為妃,改莊宅使。帝即位,妃為皇后,進光州團練使。



 以濰州防禦使知陳州,歲中閱囚,活重闢三人。西華令掠人至死,誣以疾,吏畏令,莫敢言。經得其情,卒窮治如法。歲大雪,輒弛公私僦錢以寬民,有司持不可,經曰:「上使我守陳,民窮蓋我責,我自為此,不爾累也。」方鎮別賜公使錢,
 例私以自奉,去則盡入其餘,經獨斥歸有司,唯以供享勞賓客軍師之用。知河陽,會旱蝗,民乏食,經度官廩歲用無餘,乃先以圭田租入振救之,富人爭出粟,多所濟活。



 徙徐州,遷明州觀察使。召還,提舉景靈宮。進定國軍留後,復出知青州。既行,官給車徒,三宮皆遣使送之,車馬相屬於道。未逾歲,得疾還,卒於淄州,年五十四。詔內侍迎其喪,皇后出哭於新昌第。喪至,慶壽、寶慈宮交遣謁者予菁,後臨於國門之外。贈侍中,謚曰康懿。將葬,遣
 近臣典護穿復土,給太常鹵簿。帝出郊奠之,周視其柩。葬三日,後臨於墓下,賜篆碑首曰「忠勤懿戚」。



 經所至勤吏治,事皆自省決,頗欲以才見於用,故數請外補。嘗因太祖忌日,百官班開元殿下,後召經見行幄,勉以盡忠朝廷,經亦以善事三宮為言,不及其家事。子宗回、宗良。



 綜字君章,知歙縣,籍閭里惡少年,有盜發,用以推跡輒得。通判桂州、常州,知隨、鼎、漳、汾、密、棣、沂七州。沂阻山多盜,綜請用重法繩禁,歲斷大闢減半。兵久惰,會初置官
 提舉,教之急,眾不悅,監兵夜排闥告變,綜疑有他謀,就寢自若。明日大閱,申嚴號令,賞其高強,罰其不進者,卒亦無事。性寬裕,善治劇,於奸惡不少恕。官累中散大夫,卒。



 宗回字子發,累官相州觀察使。徽宗立,進彰德軍留後。歷安國、保信、鎮南、保平軍節度使,檢校司空,封永陽、寧海、安康、漢東郡王,開府儀同三司。崇寧初,有告其陰事者,詔開封府鞫實,御史中丞吳執中臨問,宗回惶懼,上
 還印綬,以太子少保致仕。言者不已,削官爵流郴州。行二日,聽家居省咎。逾年,盡還其故官。



 宗回少驕恣,有小才,嘗權群牧都監,數以蕃息被賞。出知蔡州,擒劇賊,殲其黨類。歲饑,發廩興力役,饑者得濟,而官舍帑廩一新。欽聖後服除,起奉朝請,繼命止朝朔望。卒,年六十二,帝制服苑中,贈檢校少師,謚曰榮縱。



 宗良字景弼,歷秀州刺史、利州觀察使、昭信軍留後,奉國、清海、鎮東、武寧、寧海軍節度使,永嘉郡王,開府儀同
 三司。欽聖後臨朝時,嘗為陳瓘論其與蔡京相結。及預政事,亦能恪共自守。宣和中,卒,年六十六,贈少保。



 張敦禮,熙寧元年選尚英宗女祁國長公主,授左衛將軍、駙馬都尉,遷密州觀察使。元祐初,疏言:「變法易令,始於王安石,成於蔡確。近者退確進司馬光,以臣觀之,所得多矣。」進武勝軍留後。



 章惇為政,言:「敦禮忘德犯分,醜正朋邪。密封章疏,詆毀先烈。引譽罪首,謂當褒崇,欲其黨儔盡見收用。」乃責授左千牛衛大將軍,勒止朝參。徽
 宗立,有司以敦禮在貴籍,奏審恩賜,帝與欽聖後皆以為當與。惇等執前疏,欽聖曰:「戚里何必預知朝廷事,當時罰亦太重矣。」復和州防禦使,進保信軍留後。



 崇寧初,拜寧遠軍節度使。諫官王能甫言:「敦禮以匹夫之賤,一日而富貴具焉。神宗親愛隆厚,禮遇優渥,而敦禮詆毀盛德,罪大謫輕。今復與之節鉞,無乃傷陛下『紹述』之志乎!」乃奪節,仍為集慶軍留後。大觀初,復節度寧遠軍,徙雄武。卒,贈開府儀同三司。



 任澤,字天錫,仙游夫人母弟也。英宗入繼大統,召至延和殿,授西頭供奉官,賜第一區,寵賚甚厚。神宗時,累遷皇城使,領昌州刺史。護仙游柩遷祔於濮園,真拜嘉州刺史。卒,賜崇信軍節度使,謚曰恭僖,賜墓寺,寺額為「旌孝」。澤起田里,際會恩寵,能自安繩檢。帝欲廣其居,固辭。當任子,弗請,其篤謹如此。



\end{pinyinscope}