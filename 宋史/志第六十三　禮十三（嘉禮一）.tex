\article{志第六十三 禮十三(嘉禮一)}

\begin{pinyinscope}

 上尊號儀高宗
 內禪儀上皇太后皇太妃冊寶儀



 舊史以飲食、婚冠、賓射、饗宴、脤膰、慶賀之禮為嘉
 禮,又以歲時朝會、養
 老、宣赦、拜表、臨軒命官附之,今依《政和禮》,分朝會為賓禮,餘如其舊云。



 尊號之典,唐始載於禮官。宋每大祀,群臣詣東上閣門,拜表請上尊號,或三上,或五上,多謙抑弗許。如允所請,
 即奏
 命
 大臣撰冊文及書冊寶。其受冊多用祀禮畢日,御正殿行禮,禮畢,有司以冊寶詣閣門奉進入內。建隆四年,群臣三上表上尊號,詔俟郊畢受冊。前三日,遣官奉告天地、宗廟、社稷,遂為定制。



 其儀:有司宿設崇元殿仗衛,文武百官並集朝堂之次,攝太尉奉冊於案,吏部
 侍郎一員押,司徒奉寶於案,禮部侍郎一員押,以五品、六品清資官充舉冊、舉寶官,皆承之以匣,覆之以帕,俱詣殿門外之東、太尉之前。大樂令帥工人入就位,諸侍衛官及宰執、兩制、供奉官等立於殿階下香案前左右,如常入閣儀。侍中奏中嚴外辦,所司承旨索扇,扇上,皇帝袞冕,御輿出自西房,樂作,即御坐,扇開,樂止。符寶郎奉寶如常儀,禮直官、通事舍人分引太尉以下文武群官應北面位者,各就橫行位,太常卿於冊案前導至丹
 墀西階上少東,北面置訖。太尉、司徒、吏部禮部侍郎各入本班立定,典儀贊百官再拜舞蹈,三稱萬歲,又再拜起居訖,又再拜,分班序立。禮直官引太常卿隨行,吏部侍郎押冊案以次序行,太尉從之,禮部侍郎次押寶案行,司徒從之,詣西階,至解劍褥位。其讀冊中書令、讀寶侍中,候冊案將至,先升於前楹間第一柱北對立。太尉解劍、脫舄訖,吏部侍郎押冊案先升,太尉從升,當御坐前。太尉搢笏,北面奉冊案稍前跪置訖,俯伏,興,少退,東
 向立;中書令進當冊案前,讀冊訖,俯伏,興,又搢笏,奉冊於褥,東向冊函,北向進跪置御坐前,與舉冊官降還侍立位,太尉亦降,納舄、帶劍。禮部侍郎押寶案升,司徒隨升,北面跪置,侍中讀寶訖,置冊之南,俱復位,其納舄、帶劍、俯伏,一如上儀。典儀贊在位官皆再拜,禮直官、通事舍人引太尉至西階下,解劍、舄升,當御坐前跪賀,其詞中書門下撰。



 賀訖復位,皆再拜,如讀冊寶儀。侍中升至御坐前承旨,退,臨階西向稱「有制」,典儀贊再拜訖,宣曰:「朕以鴻
 儀昭舉,保命會昌,迫於群情,祗膺顯號。退循寡昧,惕懼增深。所賀知。」宣訖復位,典儀贊再拜舞蹈,三稱萬歲,又再拜訖。侍中升階奏禮畢,降復位,扇上,樂作,帝降坐,御輿入自東房,扇開,樂止。侍中版奏解嚴,中書侍郎帥奉案官升殿,跪奉冊置於案,次門下侍郎奉寶如奉冊禮,通事舍人贊引詣東上閣門狀進,所司承旨放仗,百官再拜訖,退如常儀。自後受冊皆如之。禮畢,賜百官食於朝堂。



 熙寧元年,宰臣曾公亮等上表請加尊號,詔不允。
 先是,翰林學士司馬光言:「尊號起唐武后、中宗之世,遂為故事。先帝治平二年,辭尊號不受,天下莫不稱頌聖德。其後佞臣建言,國家與契丹常有往來書,彼有尊號而中國獨無,足為深恥。於是群臣復以非時上尊號,論者甚為朝廷惜之。今群臣以故事上尊號,臣愚以為陛下聰明睿知,雖宜享有鴻名,然踐祚未久,又在亮陰之中,考之事體,似未宜受。陛下誠能斷以聖意,推而不居,仍令更不得上表請,則頌嘆之聲將洋溢四海矣。」詔賜
 光曰:「覽卿來奏,深諒忠誠。朕方以頻日淫雨,甲申地震,天威彰著,日虞傾禍。被此鴻名,有慚面目,況在亮陰,亦難當是盛典。今已批降指揮,可善為答辭,使中外知朕至誠慚懼,非欺眾邀名。」其後,宰臣數上表請,終不允。



 徽宗內禪,欽宗上尊號曰教主道君太上皇帝,居龍德宮。靖康元年正月朔,朝賀畢,車駕詣龍德宮賀,百官班門外,宰執進見如儀。



 高宗內禪。紹興三十二年六月十日御札:「皇太子可即皇帝位,朕稱太上皇帝,退處德壽宮,
 皇后稱太上皇后。應軍國事並聽嗣君處分。」



 十一日,行內禪之禮。有司設仗紫宸殿,宰臣、文武百僚立班,皇帝出宮,鳴鞭,禁衛諸班直、親從儀仗並內侍省執骨朵使臣等並迎駕,自贊常起居。皇帝升御坐,知閣門官以下並內侍都知、御帶以下一班起居,次管軍一班起居,次宰執以下常起居訖,左僕射陳康伯、知樞密院事葉義問、參知政事汪澈、同知樞密院事黃祖舜升殿奏曰:「臣等不才,輔政累年,罪戾山積,乃蒙容貸,不賜誅責。今陛
 下超然獨斷,高蹈堯、舜之舉,臣等心實欽仰。但自此不獲日望清光,犬馬之情,不勝依戀。」因再拜辭,相與泣下,幾至號慟。帝亦為之流涕曰:「朕在位三十六年,今老且病,久欲閑退,此事斷自朕心,非由臣下開陳,卿等當悉力以輔嗣君。」康伯等復奏曰:「皇太子仁聖,天下所共知,似聞謙遜太過,未肯便御正殿。」帝曰:「朕前此固嘗與之言,早來禁中又面諭之,即步行徑趨側殿門,欲還東宮,已再三敦勉邀留,今在殿後矣。」宰執降階,皇帝降坐,鳴
 鞭還內。宰臣文武百僚並退,立班,聽宣詔訖,再拜舞蹈,三稱萬歲,再拜訖,班權退,復追班入,詣殿下立班。



 少頃,新皇帝服履袍,涕泣出宮。禁衛諸班直、親從儀仗等迎駕,起居,鳴鞭。內侍扶掖皇帝至御榻,涕泣再三,不坐,內侍傳太上皇帝聖旨,請皇帝升御坐,皇帝升御坐東側坐。知閣門官以下一班起居、稱賀,次管軍官一班起居、稱賀,次文武百僚橫行北向立,舍人當殿稱文武百僚宰臣陳康伯以下起居、稱賀,皇帝降御坐,側身西向不坐。
 俟宰臣以下再拜舞蹈、三稱萬歲、起居、稱賀畢,康伯等升殿奏:「臣等言:願陛下即御坐,以正南面,上副太上皇帝傳授之意。」帝愀然曰:「君父之命出於獨斷,此大位,懼不敢當,尚容辭避。」康伯等再奏:「茲者伏遇皇帝陛下應天順人,龍飛寶位,第以駑下之材,恐不足以仰輔新政,然依乘風雲千載之遇,實與四海蒼生不勝幸慶。」再拜賀畢,奏事而退。宰執下殿,皇帝還內,鳴鞭。宰執文武百僚赴祥曦殿,候太上皇帝登輦,扈從至德壽宮而退。



 翌
 日,詣德壽宮朝見。前期,儀鸞司設大次於德壽宮門內,小次於殿東廊西向。其日,俟皇帝出即御坐,從駕臣僚、禁衛等起居如常儀。皇帝降御坐,乘輦至德壽宮,文武百僚詣宮門外迎駕,起居訖,前導官、太常卿、閣門官、太常博士、禮直官先入,詣大次前,分左右立定,俟皇帝降輦入,次御史臺、閣門、太常寺報文武百僚入,詣殿庭北向立定。前導官導皇帝入小次,簾降,俟太上皇帝即御坐,小次簾卷,前導官導皇帝升殿東階,詣殿上折檻前,
 奏請拜,皇帝再拜訖,前導官導皇帝稍前,躬奏聖躬萬福訖,復位,再拜訖,導皇帝詣太上皇帝御坐之東,西向立。殿下在位官皆再拜,搢笏,三舞蹈,三叩頭,出笏就拜,又再拜,班首不離位,奏聖躬萬福,又再拜,班退,前導官以次退,從駕官歸幕次,以俟從駕。太上皇帝駕興,皇帝從,入見太上皇后,如宮中之儀。皇帝還內,如來儀。每遇正旦、冬至及朔望,並依上儀。



 十二日,帝詣德壽宮,以雨,百僚免入見,上就宮中行禮。自後詣宮,若行宮中禮,即
 不集百官陪位。十三日,詔令宰臣率百官於初二日、十六日詣德壽宮起居。又詔:「朕欲每日一朝德壽宮,修晨昏之禮。面奉慈訓,恐廢萬機,勞煩群下,不蒙賜許。禮官宜復位其期,如前代朝朔望,甚為疏闊,朕不敢取。」於是禮部、太常寺言:「《漢書》高皇帝五日一朝太上皇,乞依此故事,每五日一次詣德壽宮朝見,如宮中禮。」



 帝始御後殿,宰臣陳康伯等奏:「臣等朝德壽宮,太上皇宣諭,車駕每至宮,必於門外降輦,已再三勉諭,既行家人之禮,自
 宜至殿上降輦。」帝曰:「太上有旨不須五日一朝,只朝朔望,朕心未安,宜令有司詳議。如宮門降輦,臣子禮所當然。」於是禮部、太常言:「除朝朔望外,乞於每月初八、二十三日詣德壽宮起居,如宮中儀。」自後皆遵此制,如值雨、盛暑、祁寒,臨期承太上特旨乃免。



 十一月冬至,上詣德壽宮稱賀上壽,禮畢,入見太后,如宮中禮。自後冬至並同。隆興元年正月朔,帝率百官詣德壽宮,如冬至儀。自後正旦並同。



 乾道元年二月朔,帝詣德壽宮,恭請太上、
 太后至延祥觀燒香,太上與帝乘馬,太后於後乘輿;次幸聚景園,次幸玉津園。自後帝詣德壽宮恭請太上、太后至南內,或幸延祥觀靈隱寺天竺寺、恭進太上聖政、冊命皇太子,起居稱謝。遇游幸,則宰執以下從駕至游幸所,除管軍、環衛官等俟駕還護從還內,宰執以下並免護從,先退。



 淳熙十六年,孝宗內禪,皇太子即皇帝位;紹熙五年,光宗內禪,皇子嘉王即皇帝位,並如紹興三十二年故事。



 太皇太后、皇太后、皇太妃冊禮。建隆元年,詔尊母南陽郡太夫人為皇太后,仍令所司追冊四親廟。後不果行。至道三年四月,尊太宗皇后李氏為皇太后,宰臣等詣崇政殿門表賀皇帝,又詣內東門表賀皇太后。乾興元年,真宗遺制尊皇後劉氏為皇太后,淑妃楊氏為皇太妃,亦不果行冊禮。



 天聖二年,宰臣王欽若等五表請上皇太后尊號。十一月,郊祀畢,帝御天安殿受冊,百官稱賀畢,再序班。侍中奏中嚴外辦,禮儀使奏發冊寶,帝服
 通天冠、絳紗袍,秉珪以出。禮儀使、閣門使導帝隨冊寶降自西階,內臣奉至殿庭,置橫街南東向褥位,冊在北,寶在南,帝立殿庭北向褥位,奉冊寶官奉冊寶案,太常卿、吏部、禮部侍郎引置當中褥位。禮儀使奏請皇帝再拜,在位官皆再拜。太尉、司徒就冊寶位,帝搢珪跪,奉冊授太尉,又奉寶授司徒,皆搢笏東向跪受,興,奉冊寶案置於近東西向褥位。禮儀使奏請皇帝歸御幄,易常服,乘輿赴文德殿後幄,百官班退赴朝堂,太尉、司徒奉冊
 寶至文德殿外幄,太尉以下各就次以俟。



 侍中奏中嚴外辦,太后服儀天冠、袞衣以出,奏《隆安》之樂,行障、步障、方團扇,侍衛垂簾,即御坐,南向,樂止。太常卿導冊案至殿西階下,各歸班,在位者皆再拜。太尉押冊案,司徒奉冊,中書令讀冊訖,侍中押寶案,司徒奉寶,侍中讀寶畢,太尉、司徒詣香案前,分班東西序立。尚宮贊引皇帝詣皇太后坐前,帝服靴袍,簾內行稱賀禮,跪曰:「嗣皇帝臣某言:皇太后陛下顯崇徽號,昭煥寰瀛,伏惟與天同壽,率土
 不勝欣抃。」俯伏,興,又再拜,尚宮詣御坐承旨,退,西向稱:「皇太后答曰:皇帝孝思至誠,貫於天地,受茲徽號,感慰良深。」帝再拜,尚宮引歸御幄,太尉率百官稱賀,奏《隆安》之樂,太后降坐還幄,樂止。侍中奏解嚴,所司放仗,百官再拜退。太后還內,內外命婦稱賀太后、皇帝於內殿,在外命婦及兩京留司官並奉表稱賀。自是,上皇太后尊號禮皆如之。



 熙寧二年,神宗尊皇太後曹氏為太皇太后,詣文德殿跪奉玉冊授攝太尉曾公亮、金寶授攝司徒
 韓絳,又跪奉皇太后高氏玉冊授攝太尉文彥博、金寶授攝司徒趙抃,禮畢,百官稱賀。



 哲宗即位,詔尊太后高氏為太皇太后,皇後向氏為皇太后,德妃朱氏為皇太妃。禮部議:「皇太妃生日節序物色,其冠服之屬如皇后例,稱慈旨,慶賀用箋。太皇太后、皇太后於皇太妃稱賜,皇帝稱奉,百官不稱臣。皇帝問皇太妃起居用箋,皇太妃答皇帝用書。」宰臣請特建太皇太后宮曰崇慶,殿曰崇慶、曰壽康;皇太后宮曰隆祐,殿曰隆祐、曰慈徽。



 元祐
 二年,詔太皇太后受冊依章獻明肅皇后故事,皇太后受冊依熙寧二年故事,皇太妃與皇太后同日受冊,令太常禮官詳定儀注。右諫議大夫梁燾請對文德殿,太皇太后曰:「大臣欲行此禮,予意謂必難行。」燾對曰:「誠如聖慮,願堅執勿許。且母後權同聽政,蓋出一時不得已之事,乞速罷之。」中書舍人曾肇亦言:「太皇太后聽政以來,止於延和殿,受遼使朝見,亦止於御崇政殿,未嘗踐外朝。今皇帝述仁祖故事,以極崇奉之禮,太皇太后儻
 以此時特下明詔,發揚皇帝孝敬之誠,而固執謙德,止於崇政殿受冊,則皇帝之孝愈顯,太皇太后之德愈尊,兩義俱得,顧不美歟?」太皇太后欣然納之,乃詔將來受冊止於崇政殿。尋以天旱權罷。未幾,太師文彥博等以時雨溥澍,秋稼有望,請舉行冊禮,凡三請乃從。九月六日,發太皇太后冊寶於大慶殿,發皇太后、太妃冊寶於文德殿,行禮如儀。



 紹聖元年,詔:「奉太皇太后旨,皇太妃特與立宮殿名,坐六龍輿,張傘,出入由宣德正門。」有司
 請應宮中並依稱臣妾,外命婦入內準此;百官拜箋稱賀,稱殿下。



 徽宗即位,加哲宗太妃號曰聖瑞,既又御文德殿,冊命元符皇後劉氏為太后,並依皇后禮制。



 建炎元年五月,冊元祐皇后為隆祐太后,令所司擇日奉上冊寶,時方巡幸,不克行禮;遙尊韋賢妃為宣和皇后。紹興七年三月,詔略曰:「宣和皇后夙擁慶羨,是生眇沖,乃骨肉之至親,偕父兄而時邁。十年地阻,懷《陟岵》、《凱風》之思;萬里使還,奉上皇、寧德之諱。宜尊為皇太后,令所司
 擇日奉上冊寶。」太常寺言:「請依祖宗故事,俟三年之喪終制,然後行禮。」時翰林學士朱震言:「唐德宗建中上太后沈氏尊號時,沈太后莫知所在,猶供張含元殿,具袞冕,出左序,立東方,再拜奉冊。今太后聖體無恙,信使相望,豈可不舉揚前憲?臣又聞,三年之制,惟天地、社稷越紼行事。德宗以大歷十四年即位,明年改元建中,時行易月之制,故以冕服行事。今陛下退朝之服,盡如禮制,謂當供張別殿,遣三公奉冊,藏於有司,恭俟來歸。願下禮
 官講明。」詔從之。禮部、太常言:「寶文欲乞以『皇太后寶』四字為文,合差撰冊文官一員,書冊文官一員,書篆寶文官一員,並差執政。」十年,營建皇太后宮,以慈寧為名。十二月,帝自常御殿詣慈寧殿遙賀皇太后,奉上冊寶。



 十二年八月,皇太后還慈寧宮,十月十八日,奉進冊寶。其日張設慈寧殿,設坐殿中,皇太后服禕衣即御坐,本殿官設冊寶於殿下,慈寧宮事務官並本殿官並朝服詣殿下,再拜,搢笏,舉冊寶奉進;先進冊,次進寶,進畢,降坐,
 易禕衣,服常服。皇帝詣慈寧殿賀,如宮中儀,次宰臣率百僚拜表稱賀。



 三十二年六月,詔上太上皇帝、太上皇后尊號,集議以聞。左僕射陳康伯等言:「五帝之壽,惟堯最高,百王之聖,惟堯獨冠。今茲高世之舉,視堯有光,恭請上太上皇帝尊號曰光堯壽聖太上皇帝,太上皇后尊號曰壽聖太上皇后。」詔恭依,仍令禮部、太常討論禮儀以聞。左僕射陳康伯撰太上皇帝冊文,兼禮儀使、參政汪澈書冊文並篆寶,知樞密院葉義問撰太上皇后
 冊文,同知樞密院事黃祖舜書冊文。



 八月十四日,奉上冊寶。是日,陪位文武百僚、太傅以下行事官,並朝服入詣大慶殿下立班。皇帝自內服履袍入御幄,服通天冠、絳紗袍出至大慶殿,詣冊寶褥位前再拜,在位官皆再拜訖,皇帝行發冊寶授太傅之禮如儀。禮畢,皇帝還幄,服履袍還內,文武百僚退。



 儀仗鼓吹,備而不作。



 護衛冊寶,太傅以下行事官導從冊寶至德壽宮。皇帝自祥曦殿服履袍乘輦,至德壽宮大次降輦,陪位文武官入殿庭立
 班定,太傅以下行事官從冊寶入殿,皇帝服通天冠、絳紗袍升殿,詣西向褥位立,太上皇帝自宮服履袍即坐,皇帝北向四拜起居訖,次太傅以下皆四拜起居。



 次行奉冊之禮,中書令、參知政事史浩讀冊,攝侍中葉義問讀寶,讀訖,退復位。皇帝再拜稱賀曰:「皇帝臣某稽首言:伏惟光堯壽聖太上皇帝陛下冊寶告成,鴻名肇正,與天同壽,率土均歡。」皇帝再拜,次侍中承旨宣答曰:「皇帝孝通天地,禮備古今,勉受鴻名,良深感慰。」皇帝再拜訖,
 西向立,次太傅以下再拜稱賀致詞曰:「攝太傅、尚書左僕射臣康伯等稽首言:伏惟光堯壽聖太上皇帝陛下肅臨寶位,誕受丕稱,獨推天父之尊,普慰帝臣之願。」奏訖,再拜舞蹈。次侍中承旨宣答曰:「光堯壽聖太上皇帝聖旨:倦勤滋久,佚老是圖,勉受嘉名,但增感慰。」又再拜舞蹈。次太上皇帝降坐入宮,皇帝後從壽聖太上皇后冊寶入宮。



 皇帝詣太上皇后坐前北向立,太上皇后升坐,皇帝四拜起居,行奉上冊寶之禮,讀冊官陳子常讀
 冊,讀寶官梁康民讀寶,讀訖復位,皇帝再拜稱賀致詞曰:「皇帝臣某稽首言:伏惟壽聖太上皇后殿下德茂坤元,禮崇大號,寶書翕受,歡抃無疆。」皇帝再拜,次宣答官承旨宣答曰:「壽聖太上皇后教旨:皇帝祲容載蕆,顯號來膺,誠孝通天,但深感惕。」皇帝再拜訖,太上皇后降坐入宮。次太傅以下文武百僚就德壽殿下拜箋稱賀以俟,皇帝服履袍乘輦還內。十六日,宰臣率文武百僚詣文德殿拜表稱賀。



\end{pinyinscope}