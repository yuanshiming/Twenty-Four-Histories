\article{列傳第二百三十姦臣一}

\begin{pinyinscope}

 《易》曰:「陽卦多陰,陰卦多陽。」君子雖多,小人用事,其象為陰;小人雖多,君子用事,其象為陽。宋初,五星聚奎,占
 者以為人才眾多之兆。然終宋之世,賢哲不乏,姦邪亦多。方其盛時,君子秉政,小人聽命,為患亦鮮。及其衰也,小人得志,逞其狡謀,壅閼上聽,變易國是,賊虐忠直,屏棄善良,君子在野,無捄禍亂。有國家者,正邪之辨,可不慎乎!作《姦臣傳》。



 蔡確,字持正,泉州晉江人,父徙陳。確有智數,尚氣,不謹細行。第進士,調邠州司理參軍,以賄聞。轉運使薛向行部,欲按治,見其儀觀秀偉,召與語,奇之,更加延譽。韓絳宣撫陝西,見所制樂語,以為材,薦于弟開封尹維,辟管幹右廂公事,維去而確至。舊制當庭參,確不肯,後尹劉庠責之,確曰:「唐藩鎮自置掾屬,故有是禮。今輦轂下比肩事主,雖故事不可用。」遂乞解職。



 王安石薦確,徙為三班主簿。用鄧綰薦,為監察御史裏行。王韶開熙河,多貸公錢,秦帥郭逵劾其罪,詔使杜純鞫治得實。安石卻其牘,更遣確,確希意直韶,逵、純獲譴。確善觀人主意,與時上下,知神宗已厭安石,因安石乘馬入宣德門與衛士競,即疏其
 過以賈直。加直集賢院,遷御史知雜事。



 范子淵浚
 河之役,知制誥熊本按行以為非是,為子淵所訟,確劾本附文彥博,黜之,代為知制誥、知諫院兼判司農寺。
 三司使沈括謁宰相吳充論免役法,確言括為近臣,見朝廷法令未便,不公言之而私語執政,意王安石既去,新法可搖耳。括坐黜知宣州。



 開封鞫相州民訟,事連判官陳安民,安民令其甥文及甫求援於充之子安持,及甫,充婿也。確言事關大臣,非開封可了,遂移御史
 臺。時獄起皇城,卒事多
 不讎。中丞鄧潤甫,御史上官均按之,與府獄同。王珪奏遣確詣臺參治,確鍛煉為
 獄,潤甫、均不能制,密奏確慘掠諸囚。確伺知之,即劾二人庇有罪,且詐使吏為使者慮問,囚稱冤,輒苦辱之。帝頗疑其濫,連遣諫官及內侍審直,皆怖畏,言不冤,由是潤甫、均皆罷,而確得中丞,猶領司農,凡常平、免役法皆成其手。



 太學生虞蕃訟學官,確深探其獄,連引朝士,自翰林學士許將以下皆逮捕械繫,令獄
 卒與同寢處飲食,旋溷共為一室,設大盆於前,凡羹飯餅胾舉投其中,以杓混攪,分飼之如犬豕。久繫不問,幸而得問,無一事不承。遂劾參知政事元絳有所屬請,絳出知亳州;確代其位。確自
 知制誥為
 御史中丞、
 參知政事,皆以起獄奪人位而居之,士大夫交口咄罵,而確自以為得計也。



 吳充數為帝言新法不便,欲稍去其甚者,確曰:「曹參與蕭何有隙,至代為相,一遵何約束。今陛下所自建立,豈容一人挾
 怨而壞之。」法遂不變。



 元豐五年,拜尚書右僕射兼中書侍郎。時富弼在西京,上言蔡確小人,不宜大用。確既相,屬興羅織之獄,縉紳士大夫重足而立矣。初議官制,蓋仿《唐六典》,事無大小,並中書取旨,門下審覆,尚書受而行之,三省分班奏事,柄歸中書。確說王珪曰:「公久在相位,必得中書令。」珪信不疑。確乃言於帝曰:「三省長官位高,不須置令,但令左右僕射分兼兩省侍郎足矣。」帝以為然。故確名為次相,實顓大政,珪以左僕射兼門下,拱
 手而已。帝雖以次敘相珪、確,然不加禮重,屢因微失罰金,每罰輒門謝。宰相罰金門謝,前此未有,人皆恥之。



 哲宗立,轉左僕射。韓縝入相中書,用其兩侄為列卿,確風御史中丞黃履劾縝。始詔三省,凡取旨事及臺諫官章疏,並執政同進擬,不專屬中書。蓋確畏失權,又復改制也。



 為永裕山陵使,靈駕發引之夕,不宿於次,在道又不扈從,還,又不丐去。御史劉摯、王巖叟連擊之,言確有十當去:「在熙寧、元豐時,冤獄苛政,首尾預其間。及至今日,
 稍語於人曰:『當時確豈敢言。』此其意欲固竊名位,反歸曲於先帝也」。司馬光、呂公著進用,蠲除煩苛,確言皆己所建白,公論益不容,太皇太后猶不忍即退斥。元祐元年閏二月,始罷為觀文殿學士、知陳州。明年,坐弟碩事奪職,徙安州,又徙鄧。



 初,神宗疾革,王珪議建儲事,確與同列皆在側,知狀。確自見得罪于世,陰與章惇、邢恕等合志邪謀,謂珪實懷異意,賴己擁護,故不得逞。確奉使陵下,韓縝白發其端,事寢籍籍。既失勢,愈怨望,恕又益
 為往來造言,識者以為憂,未有以發也。



 確在安陸,嘗游車蓋亭,賦詩十章,知漢陽軍吳處厚上之,以為皆涉譏訕,其用郝處俊上元間諫高宗欲傳位天后事,以斥東朝,語尤切害。於是左諫議大夫梁燾、右諫議大夫范祖禹、左司諫吳安詩、右司諫王巖叟、右正言劉安世,連上章乞正確罪。詔確具析,確自辯甚悉。安世等又言確罪狀著明,何待具析,此乃大臣委曲為之地耳。遂貶光祿卿、分司南京,再責英州別駕、新州安置。宰相范純仁、左
 丞王存坐廉前出語救確,御史李常、盛陶、翟恩、趙挺之、王彭年坐不舉劾,中書舍人彭汝礪坐封還詞命,皆罷去。確後卒於貶所。



 紹聖元年,馮京卒,哲宗臨奠。確子渭,京婿也,於喪次中闌訴。明日,詔復正議大夫。二年,贈太師,諡曰「忠懷」,遣中使護其葬,又賜第京師。崇寧初,配饗哲宗廟庭。蔡京請徽宗書「元豐受遺定策殊勳宰相蔡確之墓」賜其家。京與太宰鄭居中不相能,居中以憂去,京懼其復用,而居中,王珪婿也。時渭更名懋,京使之重
 理前事,以沮居中,遂追封確清源郡王,御製其文,立石墓前。擢懋同知樞密院事,次子莊為從官,弟碩,贈待制,諸女超進封爵,諸婿皆得官,貴震當世。



 高宗即位,下詔暴群姦之罪,貶確武泰軍節度副使,竄懋英州,凡所與濫恩,一切削奪,天下快之。



 吳處厚者,邵武人,登進士第。仁宗屢喪皇嗣,處厚上言:「臣嘗讀《史記》,考趙氏廢興本末,當屠岸賈之難,程嬰、公孫杵臼盡死以全趙孤。宋有天下,二人忠義未見褒表,宜訪其墓域,建為其祠。」帝覽
 其疏矍然,即以處厚為將作丞,訪得兩墓于絳,封侯立廟。



 始,蔡確嘗從處厚學賦,及作相,處厚通箋乞憐,確無汲引意。王珪用為大理丞。王安禮、舒亶相攻,事下大理,處厚知安禮與珪善,論亶用官燭為自盜。確密遣達意救亶,處厚不從,確怒欲逐之,未果。珪請除處厚館職,確又沮之。珪為永裕山陵使,辟掌箋奏。確代使,出知通利軍,又徙知漢陽,處厚不悅。



 元祐中,確知安州,郡有靜江卒當戍漢陽,確固不遣,處厚怒曰:「爾在廟堂時數陷我,
 今比郡作守,猶爾邪?」會得確《車蓋亭詩》,引郝甑山事,乃箋釋上之,云:「郝處俊封甑山公,會高宗欲遜位武后,處俊諫止,今乃以比太皇太后。且用滄海揚塵事,此蓋時運之大變,尤非佳語。譏謗切害,非所宜言。」確遂南竄。擢處厚知衛州,然士大夫由此畏惡之,未幾卒。紹聖間,追貶歙州別駕。



 邢恕,字和叔,鄭州陽武人。博貫經籍,能文章,喜功名,論古今成敗事,有戰國縱橫氣習。從程顥學,因出入司
 馬光、呂公著門。登進士第,補永安主簿。公著薦於朝,得崇文院校書。王安石亦愛之,因賓客諭意,使養晦以待用,恕不能從,而對其子雱語新法不便。安石怒,諫官亦言新進士未歷官而即處館閣,開奔競路,出知延陵縣。縣廢不復調,浮沉陝、洛間者七年,復為校書。



 吳充用為館閣校勘,歷史館檢討、著作佐郎。蔡確代充相,盡逐充所用人,恕深居懼及。神宗見其《送文彥博詩》,稱於確,乃進職方員外郎。帝有復用光、公著意,確以恕于兩人為門
 下客,亟結納之。恕亦深自附託,乃為確畫策,稍收召名士,于政事微有更革,自是相與如素交。



 帝不豫,恕與確成謀,密語宣仁后之侄公繪、公紀曰:「家有白桃著華,道書言可療上疾。」邀與歸視之。至則執其手曰:「蔡丞相令布腹心,上疾不可諱,延安沖幼,宜早有定論,雍、曹皆賢王也。」公繪驚曰:「此何言?君欲禍吾家邪!」急趨出。恕計不行,則反宣言太后屬意雍王,與王珪表裏。導確約珪入問疾,陽鉤致珪語,使知開封府蔡京伏劍士於外,須珪
 小持異則執而誅之。既而珪言上自有子,定議立延安。恕益無所施,猶自謂有定策功,傳播其語。



 哲宗立,遷右司員外郎、起居舍人。又為公繪具奏,乞尊崇朱太妃,為高氏異日計。后詰之曰:「汝素不識字,誰為之者?」公繪不得隱,以恕對,且上其稿。時恕方召試中書,遂黜知隨州,改汝、襄、河陽。恕久斥外,蓄怒憤,間道謁確于鄧,緒成前惡,紿司馬光子康手書,持以取信。會確得罪,恕亦責監永州酒。



 紹聖初,擢寶文閣待制、知青州。章惇、蔡卞得
 政,將甘心元祐諸人,引恕自助,召為刑部侍郎,再遷吏部尚書兼侍讀,改御史中丞。恕既處風憲,遂誣宣仁后有廢立謀,引司馬光言北齊婁后宣訓事,訹高遵裕之子士京追訟其父在日,王珪令其兄士充來謀立雍王,遵裕非之。又教蔡懋上文及甫私牘為廋詞,歷詆梁燾、劉摯,云陰圖不軌,且加司馬光、呂公著以凶悖名。惇使蔡京置獄于同文館,組織萬端,將悉陷諸人於族罪,既而無所得,乃已。



 恕內懷猜猾,而外持正論。嘗於經筵讀寶
 訓,至仁宗諭輔臣,以為人君當修舉政事,則日月薄食、星文變見為不足慮。恕言仁宗之旨雖合于荀卿書,然自古帝王孰肯自謂不修政事者,如此則天變遂廢矣。帝嘉納之,數登對。惇恐其大用,切忌之。恕亦揣帝稍薄惇,屢白其短,竟為惇所陷,出知汝州。未幾,徙應天府。惇復摭其曩過,移知南安軍。徽宗初,言者論其矯誣,責為少府少監、分司西京,居均州。



 蔡京當國,經營湟、鄯,以開邊隙,欲使恕立方面之勳,起為鄜延經略安撫使,旋改
 涇原,擢至龍圖閣學士。恕乞築蕭關,采其里人許彥圭車戰法,為淺攻計。又欲使熙河造船,直抵興、靈,以空夏國巢穴,其謀皆迂誕。轉運使李復言恕所為類兒戲,不可用,帝亦燭其妄,京力主之。已而夏人寇鎮戎,欲趨渭州,警奏至京師日五六,京懼,始徙恕太原,連徙永興、潁昌、真定,尋奪職。久之,復顯謨閣待制。卒,年七十。



 恕本從程門得游諸公間,一時賢士爭與之交。恕善為表襮,蚤致聲名,而天資反覆,行險冒進,為司馬光客即陷光,附
 章惇即背惇,至與三蔡為腹心則之死弗替。上謗母后,下誣忠良,幾於禍及宗廟。建炎元年,與蔡確同追貶,而恕為常德軍節度副使。子:居實、倞。



 居實有異材,八歲為《明妃引》,黃庭堅、晁補之、張耒、秦觀、陳師道皆見而愛之。從恕守隨,作《南征賦》,蘇軾讀之,歎曰:「此足以藉手見古人矣。」卒時年十九,有遺文曰《呻吟集》。



 倞及恕在時為司農丞,靖康初至少卿,奉詔館金國使。是時,肅王使斡離不軍,為所質,朝廷議亦留其使以相當,於是踰月不遣。
 都管趙倫,燕人也,性猾獪,懼不得歸,乃詐以情告倞曰:「金國有餘睹金吾者,尚領契丹精銳甚眾,貳于金人,願歸大國,可結之以圖二酋。」倞以聞,大臣信之,即為賜餘睹詔書授倫,納衣領中,厚與倫金帛。倫獻其書粘罕,粘罕大怒,以聞金主,報令深入攻討,遂復提兵南下。倞時出知岳州,詔責其始禍,削籍停官,既而京闕失守云。



 呂惠卿,字吉甫,泉州晉江人。父璹習吏事,為漳浦令。縣處山林蔽翳間,民病瘴霧蛇虎之害,璹教民焚燎而耕,
 害為衰止。通判宜州,儂智高入寇,轉運使檄璹與兵會,或勸勿行,不聽。將二千人躡賊後以往,得首虜為多。為開封府司錄,鞫中人史志聰役衛卒伐木事,吏多為之地,璹窮治之,志聰以謫去。終光祿卿。



 惠卿起進士,為真州推官。秩滿入都,見王安石,論經義,意多合,遂定交。熙寧初,安石為政,惠卿方編校集賢書籍,安石言於帝曰:「惠卿之賢,豈特今人,雖前世儒者未易比也。學先王之道而能用者,獨惠卿而已。」及設制置三司條例司,以為
 檢詳文字,事無大小必謀之,凡所建請章奏皆其筆。擢太子中允、崇政殿說書、集賢校理,判司農寺。



 司馬光諫帝曰:「惠卿憸巧非佳士,使安石負謗於中外者皆其所為。安石賢而愎,不閑世務,惠卿為之謀主,而安石力行之,故天下並指為姦邪。近者進擢不次,大不厭眾心。」帝曰:「惠卿進對明辨,亦似美才。」光曰:「惠卿誠文學辨慧,然用心不正,願陛下徐察之。江充、李訓若無才,何以能動人主?」帝默然。光又貽書安石曰:「諂諛之士,於公今日誠
 有順適之快,一旦失勢,將必賣公自售矣。」安石不悅。



 會惠卿以父喪去,服除,召為天章閣侍講,同修起居注,進知制誥,判國子監,與王雱同修《三經新義》。又知諫院,為翰林學士。安石求去,惠卿使其黨變姓名,日投匭上書留之。安石力薦惠卿為參知政事,惠卿懼安石去,新法必搖,作書遍遺監司、郡守,使陳利害。又從容白帝下詔,言終不以吏違法之故,為之廢法。故安石之政,守之益堅。議罷制科,馮京爭之不得。



 弟升卿無學術,引為侍講。
 又用弟和卿計,制五等丁產簿,使民自供首實,尺椽寸土,檢括無遺,至雞豚亦遍抄之。隱匿者許告,而以貲三之一充賞,民不勝其困。又因保甲正長給散青苗,使結甲赴官,不遺一人,上下騷動。



 鄭俠疏惠卿朋姦壅蔽,惠卿怒,又惡馮京異己,而安石弟安國惡惠卿姦諂,面辱之。於是乘勢並陷三人,皆獲罪。安石以安國之故,始有隙。惠卿既叛安石,凡可以害王氏者無不為。韓絳為相不能制,請復用安石。安石至,猶與共事。御史蔡承禧論
 其惡,鄧綰又言其兄弟強借秀州富民錢買田,出知陳州。久之,以資政殿學士知延州。



 始,陝西緣邊漢蕃兵各自為軍,每戰則以蕃部為先鋒,而漢兵城守,伺便乃出戰。惠卿始合之為一,先搜補守兵而出其選以戰,隨屯置將,具條約上之,邊人及議者多言不可。路都監高永亨,老將也,爭之力,奏斥之。蕃部屈全乜將入寇,惠卿以近世帥臣多養威持重,乃將牙兵按邊,啟師於東郊,遂趨綏德,抵無定河,歷十有八日而還。



 俄丁母憂,詔於本
 奉外特給五萬,惠卿更請添支萬五千,御史劾之,將下揚州取奉歷,帝曰:「惠卿固貪冒,然嘗為執政,治之傷體,姑責以義可也。」但削其誤奉,惠卿猶自辨,御史又論其方居喪,不應有言,詔勿問。



 元豐五年,加大學士、知太原府。入見,將使仍鎮鄜延。惠卿云:「陝西之師,非唯不可以攻,亦不可以守,要在大為形勢而已。」帝曰:「如惠卿言,是為陝西可棄也,豈宜委以邊事?」數其輕躁矯誣之罪,斥知單州,明年復知太原。哲宗即位,敕疆吏勿侵擾外界。
 惠卿遣步騎二萬襲夏人於聚星泊,斬首六百級,夏人遂寇鄜延。



 惠卿見正人匯進,知不容于時,懇求散地。於是右司諫蘇轍條奏其姦曰:「惠卿懷張湯之辨詐,有盧杞之姦邪,詭變多端,敢行非度。王安石強佷傲誕,於吏事宜無所知,惠卿指擿教導,以濟其惡。又興起大獄,欲株連蔓引,塗汙公卿。賴先帝仁聖,每事裁抑,不然,安常守道之士無噍類矣。安石于惠卿有卵翼之恩,父師之義。方其求進則膠固為一,及勢力相軋,化為敵仇,發其
 私書,不遺餘力。犬彘之所不為,而惠卿為之。昔呂布事丁原則殺丁原,事董卓則殺董卓;劉牢之事王恭則反王恭,事司馬元顯則反元顯,故曹操、桓玄終畏而誅之。如惠卿之惡,縱未正典刑,猶當投畀四裔,以禦魑魅。」中丞劉摯數其五罪,以為大惡。乃貶為光祿卿、分司南京。再責建寧軍節度副使、建州安置。中書舍人蘇軾當制,備載其罪於訓詞,天下傳訟稱快焉。



 紹聖中,復資政殿學士、知大名府,加觀文殿學士、知延州。夏人復入寇,將
 以全師圍延安,惠卿修米脂諸砦以備。寇至,欲攻則城不可近,欲掠則野無所得,欲戰則諸將按兵不動,欲南則懼腹背受敵,留二日即拔柵去,遂陷金明。惠卿求詣闕,不許。以築威戎、威羌城,加銀青光祿大夫,拜保寧、武勝兩軍節度使。



 徽宗立,易節鎮南。因曾布有宿憾,徙為杭州,而用范純粹帥延,治其上功罔冒事,奪節度。布去位,復武昌節度使、知大名。數歲,又以上表引喻失當,還為銀青光祿大夫,令致仕。崇寧五年,起為觀文殿學士、
 知杭州。坐其子淵聞妖人張懷素言不告,淵配沙門島,惠卿責祁州團練副使,安置宣州,再移廬州。復觀文殿學士,為醴泉觀使,致仕。卒,贈開府儀同三司。



 始,惠卿逢合安石,驟致執政,安石去位,遂極力排之,至發其私書於上。安石退處金陵,往往寫「福建子」三字,蓋深悔為惠卿所誤也。雖章惇、曾布、蔡京當國,咸畏惡其人,不敢引入朝。以是轉徙外服,訖於死云。



 章惇,字子厚,建州浦城人,父俞徙蘇州。起家至職方郎
 中,致仕,用惇貴,累官銀青光祿大夫,年八十九卒。



 惇豪俊,博學善文。進士登名,恥出侄衡下,委敕而出。再舉甲科,調商洛令。與蘇軾游南山,抵仙遊潭,潭下臨絕壁萬仞,橫木其上,惇揖軾書壁,軾懼不敢書。惇平步過之,垂索挽樹,攝衣而下,以漆墨濡筆大書石壁曰:「蘇軾、章惇來。」既還,神彩不動,軾拊其背曰:「君他日必能殺人。」惇曰:「何也?」軾曰:「能自判命者,能殺人也。」惇大笑。召試館職,王陶劾罷之。



 熙寧初,王安石秉政,悅其才,用為編修三司
 條例官,加集賢校理、中書檢正。時經制南、北江群蠻,命為湖南、北察訪使。提點刑獄趙鼎言,峽州群蠻苦其酋剝刻,謀內附,辰州布衣張翹亦言南、北江群蠻歸化朝廷,遂以事屬惇。惇募流人李資、張竑等往招之,資、竑淫於夷婦,為酋所殺,遂致攻討,由是兩江扇動。神宗疑其擾命,安石戒惇勿輕動,惇竟以三路兵平懿、洽、鼎州。以蠻方據潭之梅山,遂乘勢而南。轉運副使蔡燁言是役不可亟成,神宗以為然,專委于燁,安石主惇,爭之不已。
 既而燁得蠻地,安石恨燁沮惇,乃薄其賞,進惇修起居注,以是兵久不決。



 召惇還,擢知制誥、直學士院、判軍器監。三司火,神宗御樓觀之,惇部役兵奔救,過樓下,神宗問知為惇,明日命為三司使。呂惠卿去位,鄧綰論惇同惡,出知湖州,徙杭州。入為翰林學士。元豐三年,拜參知政事。朱服為御史,惇密使客達意於服,為服所白。惇父冒占民沈立田,立遮訴惇,惇繫之開封。坐二罪,罷知蔡州,又歷陳、定二州。五年,召拜門下侍郎。豐稷奏曰:「官府
 肇新而惇首用,非稽古建官意。」稷坐左遷。諫官趙彥若又疏惇無行,不報。



 哲宗即位,知樞密院事。宣仁后聽政,惇與蔡確矯唱定策功。確罷,惇不自安,乃駁司馬光所更役法,累數千言。其略曰:「如保甲、保馬一日不罷,有一日害。若役法則熙寧之初遽改免役,後遂有弊。今復為差役,當議論盡善,然後行之,不宜遽改,以貽後悔。」呂公著曰:「惇所論固有可取,然專意求勝,不顧朝廷大體。」光議既行,暴憤恚爭辨簾前,其語甚悖。宣仁后怒,劉摯、蘇
 轍、王覿、朱光庭、王巖叟、孫升交章擊之,黜知汝州。七八年間,數為言者彈治。



 哲宗親政,有復熙寧、元豐之意,首起惇為尚書左僕射兼門下侍郎,於是專以「紹述」為國是,凡元祐所革一切復之。引蔡卞、林希、黃履、來之邵、張商英、周秩、翟思、上官均居要地,任言責,協謀朋姦,報復仇怨,小大之臣,無一得免,死者禍及其孥。甚至詆宣仁后,謂元祐之初,老姦擅國。又請發司馬光、呂公著塚,斫其棺。哲宗不聽,惇意不愜,請編類元祐諸臣章疏,識者
 知禍之未弭也。遂治劉安世、范祖禹諫禁中雇乳媼事,又以文及甫誣語書導蔡渭,使告劉摯、梁燾有逆謀,起同文館獄,命蔡京、安惇、蹇序辰窮治,欲覆諸人家。又議遣呂升卿、董必察訪嶺南,將盡殺流人。哲宗曰:「朕遵祖宗遺制,未嘗殺戮大臣,其釋勿治。」然重得罪者千餘人,或至三四謫徙,天下冤之。



 惇用邢恕為御史中丞,恕以北齊婁太后宮名「宣訓」,嘗廢孫少主,立子常山王演,託司馬光語范祖禹曰:「方今主少國疑,宣訓事猶可慮。」又
 誘高士京上書,言父遵裕臨死屏左右謂士京曰:「神宗彌留之際,王珪遣高士充來問曰:『不知皇太后欲立誰?』我叱士充去之。」皆欲誣宣仁后,以此實之。惇遂追貶司馬光、王珪,贈遵裕奉國軍留後。結中官郝隨為助,欲追廢宣仁后,自皇太后、太妃皆力爭之。哲宗感悟,焚其奏,隨覘知之,密語惇與蔡卞。明日惇、卞再言,哲宗怒曰:「卿等不欲朕入英宗廟乎?」惇、卞乃已。



 惇又以皇后孟氏,元祐中宣仁后所立,迎合郝隨,勸哲宗起掖庭秘獄,託以
 左道,廢居瑤華宮。其後哲宗頗悔,乃歎曰:「章惇壞我名節。」惇又結劉友端相表裏,請建劉賢妃于中宮。



 初,神宗用王安石之言,開熙、河,謀靈、夏,師行十餘年不息。迨聞永樂之敗,神宗當宁慟哭,循致不豫,故元祐宰輔推本其意,專務懷柔外國。西夏請故地,以非要害城砦還之。惇以為蹙國棄地,罪其帥臣,遂用淺攻撓耕之說,肆開邊隙,絕夏人歲賜,進築汝遮等城,陝西諸道興役五十餘所,敗軍覆將,復棄青唐,死傷不可計。知天下怨己,欲
 塞其議,請詔中外察民妄語者論如律。優立賞邏,告訐之風浸盛。民有被酒狂訛者,詔貸其死,惇竟論殺之。用刑愈峻,然不能遏也。



 哲宗崩,皇太后議所立,惇厲聲曰:「以禮律言之,母弟簡王當立。」皇太后曰:「老身無子,諸王皆是神宗庶子。」惇復曰:「以長則申王當立。」皇太后曰:「申王病,不可立。」惇尚欲言,知樞密院事曾布叱之曰:「章惇,聽太后處分。」皇太后決策立端王,是為徽宗,遷惇特進,封申國公。



 為山陵使,靈轝陷澤中,踰宿而行。言者劾其
 不恭,罷知越州,尋貶武昌軍節度副使、潭州安置。右正言任伯雨論其欲追廢宣仁后,又貶雷州司戶參軍。初,蘇轍謫雷州,不許占官舍,遂僦民屋,惇又以為強奪民居,下州追民究治,以僦券甚明,乃已。至是,惇問舍於是民,民曰:「前蘇公來,為章丞相幾破我家,今不可也。」徙睦州,卒。



 惇敏識加人數等,窮凶稔惡,不肯以官爵私所親,四子連登科,獨季子援嘗為校書郎,餘皆隨牒東銓仕州縣,訖無顯者。



 妻張氏甚賢,惇之入相也,張病且死,屬
 之曰:「君作相,幸勿報怨。」既祥,惇語陳瓘曰:「悼亡不堪,奈何?」瓘曰:「與其悲傷無益,曷若念其臨絕之言。」惇無以對。



 政和中,追贈觀文殿大學士。紹興五年,高宗閱任伯雨章疏,手詔曰:「惇詆誣宣仁后,欲追廢為庶人,賴哲宗不從其請,使其言施用,豈不上累泰陵?貶昭化軍節度副使,子孫不得仕於朝。」詔下,海內稱快,獨其家猶為《辨誣論》,見者哂之。



 曾布,字子宣,南豐人。年十三而孤,學於兄鞏,同登第,調
 宣州司户参軍、懷仁令。



 熙寧二年,徙開封,以韓維、王安石薦,上書言為政之本有二,曰云厲風俗、择人才。其要有八,曰勸農桑、理財賦、興學校、審選舉、責吏課、叙宗室、修武備、制遠人。大率皆安石指也。



 神宗召見,論建合意,授太子中允、崇政殿說書,加集賢校理,判司農寺,檢正中書五房。凡三日,五受敕告。與吕惠卿共創青苗、助役、保甲、農田之法,一時故臣及朝士多争之。布疏言:「陛下以不世出之資,登延碩學遠識之臣,思大有為於天下,而
 大臣玩令,倡之於上,小臣横議,和之於下。人人窺伺間隙,巧言醜詆,以譁眾罔上。是勸沮之術未明,而威福之用未果也。陛下誠推赤心以待遇君子而厲其氣,奮威斷以屏斥小人而消其萌,使四方曉然皆知主不可抗,法不可侮,則何為而不可,何欲而不成哉?」布欲堅神宗意,使專任安石以威脅眾,使毋敢言。故驟見拔用,遂修起居注、知制誥,為翰林學士兼三司使。韓琦上疏極論新法之害,神宗頗悟,布遂為安石條析而駁之,持之愈
 固。



 七年,大旱,詔求直言,布論判官吕嘉問市易掊克之虐,大概以为:「天下之財匱乏,良由貨不流通;貨不流通,由商賈不行;商賈不行,由兼并之家巧为摧抑。故設市易於京師以售四方之貨,常低抑其價,使高於兼并之家而低於倍蓰之直,官不失二分之息,則商賈自然無滞矣。今嘉問乃差官於四方買物貨,禁客旅無得先交易,以息多寡為誅賞殿最,故官吏、牙駔惟恐裒之不尽而息之不夥,則是官自为兼并,殊非市易本意也。」事下
 两制議,惠卿以为沮新法,安石怒,布遂去位。



 惠卿参大政,置獄舉劾,黜布知饒州,徙潭州。復集賢院學士、知廣州。元豐初,以龍圖閣待制知桂州,進直學士、知秦州,改歷陳、蔡、慶州。元豐末,復翰林學士,遷户部尚書。司馬光為政,諭令增損役法,布辭曰:「免役一事,法令纖悉皆出己手,若令遽自改易,義不可為。」元祐初,以龍圖閣學士知太原府,歷真定、河陽及青、瀛二州。紹聖初,徙江寧,過京,留為翰林學士,遷承旨兼侍讀,拜同知樞密院,進知
 院事。



 初,章惇為相,布草制極其稱美,冀惇引为同省執政,惇忌之,止薦居樞府,故稍不相能。布贊惇「绍述」甚力,請甄賞元祐臣庶論更役法不便者,以勸敢言。惇遂興大獄,陷正人,流貶鐫廢,略無虚日,布多陰擠之。掖庭詔獄成,付執政蔽罪,法官謂厭魅事未成,不當處極极典。布曰:「驢媚蛇霧,是未成否?」眾皆瞿然,於是死者三人。



 惇以士心不附,詭情飾過,薦引名士彭汝礪、陳瓘、張庭堅等,乞正所奪司馬光、吕公著贈謚,勿毁墓仆碑,布以为無
 益之事。又奏:「人主操柄,不可倒持,今自丞弼以至言者,知畏宰相,不知畏陛下。臣如不言,孰敢言者?」其意盖欲傾惇而未能。會哲宗崩,皇太后召宰執問誰可立,惇有異議,布叱惇使從皇太后命。



 徽宗立,惇得罪罷,遣中使召蔡京鏁院,拜韓忠彦左僕射。京欲探徽宗意,徐请曰:「麻詞未審合作專任一相,或作分命两相之意。」徽宗曰:「專任一相。」京出,宣言曰:「子宣不復相矣。」已而復召曾肇草制,拜布右僕射,其制曰:「東西分臺,左右建輔。」忠彦雖
 居上,然柔懦,事多决於布,布猶不能容。時議以元祐、紹聖均為有失,欲以大公至正消釋朋黨,明年,乃改元建中靖國,邪正雜用,忠彦遂罷去。布獨當國,漸進「绍述」之說。



 明年,又改元崇寧,召蔡京為左丞,京與布異。會布擬陳祐甫為户部侍郎,京奏曰:「爵禄者,陛下之爵禄,奈何使宰相私其親?」布婿陳迪,祐甫子也。布忿然争辨,久之,聲色稍厲。温益叱布曰:「曾布,上前安得失禮?」徽宗不悦而罷。御史遂攻之,罷為觀文殿大學士、知潤州。



 京積憾
 未已,加布以贓賄,令開封吕嘉問逮捕其諸子,鍛鍊訊鞫,誘左證使自誣而貸其罪。布落職,提舉太清宫、太平州居住。又降司農卿、分司南京。又以嘗薦學官趙諗而諗叛,責散官、衡州安置。又以棄湟州,責賀州别駕,又責廉州司户。凡四年,乃徙舒州,復太中大夫、提舉崇福宫。大觀元年,卒于潤州,年七十二。后贈觀文殿大學士,謚曰「文肅」。



 安惇,字處厚,廣安軍人。上舍及第,調成都府教授。上書
 論學制,召對,擢監察御史。哲宗初政,許察官言事,諫議大夫孫覺請汰其不可者,詔劉摯推擇,罷惇為利州路轉運判官,歷夔州、湖北、江東三路。



 紹聖初,召為國子司業,三遷諫議大夫。章惇、蔡卞造同文謗獄,使蔡京與惇雜治,二人肆其忮心,上言:「司馬光、劉摯、梁燾、呂大防等交通陳衍之徒,變先帝成法,懼陛下一日親政,必有欺君之誅,乃密為傾搖之計。於是疏隔兩宮,斥隨龍內侍,以去陛下之腹心;廢顧命大臣,以翦陛下之羽翼。縱釋
 先帝之所罪,收用先帝之所棄。無君之惡,同司馬昭之心;擅事之跡,過趙高指鹿為馬。比詢究本末,得其情狀,大逆不道,死有餘責。」帝曰:「元祐人果如是乎?」惇、京曰:「誠有是心,特反形未具耳。」帝為誅衍,錮摯、燾子孫。遷御史中丞。



 劉后之受冊也,百官仗衛陳於大庭,是日天氣清晏,惇巍立班中,倡言曰:「今日之事,上當天心,下合人望。」朝士皆笑其姦佞。又鞫鄒浩事,檄廣東使者鐘正甫攝治之於新州,士大夫或千里會逮,踵蹇序辰初議,閱訴
 理書牘,被禍者七八百人,天下怨疾,為二蔡、二惇之謠。徽宗雅惡之。鄒浩還朝,惇言:「浩若復用,慮彰先帝之失。」帝曰:「立后,大事也。御史中丞不言而浩獨敢言之,何為不可復用?」惇懼而退。陳瓘請曰:「陛下欲開正路,取浩既往之善,惇乃詿惑主聽,規騁其私,若明示好惡,當自惇始。」乃以寶文閣待制知潭州,尋放歸田里。



 蔡京為相,復拜工部侍郎、兵部尚書。崇寧初,同知樞密院。卒,贈特進。



 長子郊,後坐指斥誅。流其次子邦于涪而追貶惇單州
 團練副使,其祀遂絕。人以為惇平生數陷忠良之報云。



\end{pinyinscope}