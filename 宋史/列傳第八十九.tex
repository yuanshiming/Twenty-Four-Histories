\article{列傳第八十九}

\begin{pinyinscope}

 任顓李參郭申錫傅求張景憲竇卞張瑰孫瑜許遵盧士宗錢象先韓□杜純弟紘杜常謝麟王宗望
 王吉甫



 任顓,字誠之,青州壽光人。舉進士,得同學究出身。至衛尉丞。上其文,乃賜第,擢鹽鐵判官。陜西鑄康定大銅錢,顓曰:「壞五為一,以一當十,恐犯者眾。」卒如其言。



 夏人納款,遣使要請十一事,甚者欲去臣稱男。顓押伴,一切曉以義,辭折而去。又再遣使來欲自買賣,且通青鹽,增歲賜。詔許置榷場,其議多顓所發。出為京西轉運使,奏計京師。元昊為下所殺,遣楊守素來告哀。守素,乃始為元
 昊謀不稱臣、納賜節者也,仁宗記嘗屈其使者,復使押伴。顓問守素其主所以死,不能對,訖去,不敢肆。改知鳳翔府。帝語輔臣,顓宜備朝廷委任,留判三司恁由司。為諒祚冊禮使,採摭西夏風物、山川、道里、出入攻取之要,為《治戎精要》三篇上之。



 進直史館,遷河東轉運使。帝嘗以禁帑金帛賜河北,亦欲與河東。顓辭曰:「受委制財用,而先有求,不敢。」顓為使者,每行部,必擇僚佐之賢者一人與俱,凡事必與議,未嘗以胥吏自隨,人安其政。入為
 鹽鐵副使,擢天章閣待制。



 儂賊犯嶺外,以知潭州。宣撫司以宣毅卒有功,檄補軍校,顓察其色動,曰:「必有異志。」執按之,具服為賊內應。搜其家,得所記潭事甚悉,梟首以徇。詔書褒激,賜白金五百兩,進龍圖閣直學士、知渭州。坐在潭日賤市死商珠,降為待制。時四路以邊警聞,渭獨無所上,朝廷疑斥候不密,顓力言無他虞,帝使覘之,信。乃還學士,徙徐州,以太子賓客致仕。積官戶部侍郎,卒,年七十八。



 李參,字清臣,鄆州須城人。以蔭知鹽山縣。歲饑,諭富室出粟,平其直予民,不能糴者,給以糟籺,所活數萬。



 通判定州,都部署夏守恩貪濫不法,轉運使使參按之,得其事,守恩謫死。知荊門軍,荊門歲以夏伐竹,並稅簿輸荊南造舟,積日久多蠹惡不可用,牙校破產不償責。參請冬伐竹,度其費以給,餘募商人與為市,遂除其害。



 歷知興元府,淮南、京西、陜西轉運使。部多戍兵,苦食少。參審訂其闕,令民自隱度麥粟之贏,先貸以錢,俟穀熟還之
 官,號「青苗錢」。經數年,廩有羨糧。熙寧青苗法,蓋萌於此矣。



 朝廷患邊費益廣,參建議輦錢邊郡,以平估糴,權罷入中法。比其去,省榷貨錢千萬計。召為鹽鐵副使,以右諫議大夫為河北都轉運使。與安撫使郭申錫相視決河,議不協;又與真定呂溱相惡,二人皆得罪,參移使河東,知荊南。



 嘉祐七年,召為三司使,參知政事孫抃曰:「參為主計,外臺將承風刻剝天下,天下之民困矣。」乃改群牧使。詔王安石、王陶置局經度國計,參言:「官各有職,臣
 若不任事,當從廢黜。不然,乞罷此局。」從之。



 治平初,加集賢院學士、知瀛州,賜黃金百兩,帥臣有賜自參始。再遷樞密直學士、知秦州。蕃酋藥家族作亂,討平之,得良田五百頃,以募弓箭手。居鎮閱歲,未嘗以邊事聞。英宗遣使問故,對曰:「將在邊,期於無事而已,不敢妄以寇貽主憂。」以疾解邊任,判西京御史臺,起知曹、濮二州。神宗久知其才,書姓名於殿柱。以知永興軍,不行,卒,年七十四。



 參無學術,然剛果嚴深,喜發擿奸伏,不假貸,事至即決,
 雖簿書纖悉不遺,時稱能吏。



 郭申錫,字延之,魏人。自言唐代公元振之後。第進士,為晉陵尉。民訴弟為人所殺,申錫察其色懼而哭不哀,曰:「吾得賊矣,非汝乎?」執而訊之,果然。久之。知博州。州兵出戍,有欲脅眾為亂者,申錫戮一人,黥二人,乃定。奏至,仁宗曰:「小官臨事如此,豈易得?」即為御史臺推直官。數上疏論事,大臣不便。鞫獄慶州。京東盜執濮州通判井淵,遷知州事,未閱月,悉擒兇黨,斬以徇。



 召為侍御史,遂知
 雜事。張貴妃追冊、起園陵,張堯佐為使相,陳執中嬖妾殺婢,餘靖引胡恢有醜行,高若訥引範祥啟邊釁,申錫皆奏劾之,屢詆權幸無所避,帝謂之曰:「近世士大夫,方未達時,好指陳時事,及被進用則不然,是資言以進耳,卿勿為也。」



 諜稱契丹遣泛使,命體量安撫河北,還為鹽鐵副使。相視決河,坐訟李參失實,黜知濠州。帝明榜朝堂,稱其欺誣,以儆在位。旋加直史館、知江寧府,再副鹽鐵,進天章閣待制、知鄧州河中。



 種諤取綏州,申錫曰:「邊
 患將自此始。」及諒祚死,請捐前故,聽其子襲爵,且言曰:「二虜賴歲幣甚厚,渝平豈其所利,必有以致之。但得重將守邊,不要功生事,則善矣。」著《邊鄙守御策》。以給事中致仕,卒,年七十七。



 傅求,字命之,考城人。進士甲科,通判泗州。淮水溢,毀城。朝廷遣中使護築,絕淮取土,道遠,度用兵六十萬。求相汴堤旁有高埠,夷之得土,載以回舟,省工費殆半。



 徙大名府,府守呂夷簡委以事。夷簡入相,薦其才,擢知宿州,
 提點江西、益州刑獄,為梓州路轉運使。夷獠寇合江,鈐轄司會兵掩擊,求馳往按所以狀,乃縣吏冒取播州田,獠故恐而叛。即黥吏置嶺南,夷人聞之,散去。益州文彥博上其狀,進秩,徙陜西。



 關中行當十鐵錢,盜鑄不可計,求請變法。時州縣已散二百八十萬緡,亟下令更為當三。民出不意,蕩產失業,多自經死,然盜鑄遂止。自康定用兵,移稅輸邊,民力大困,求令輸本州,而轉錢以供邊糴,民受其惠,而兵食亦足。召為戶部副使。



 隴右蕃酋蘭
 氈獻古渭州地,秦州範祥納之,請繕城屯兵,又括熟戶田,諸羌靳之,相率叛。夏人欲得渭地,又移文來索。後帥張忭以祥貪利生事,請棄之。詔求往視,求以為城已訖役,且已得而棄,非所以強國威。乃詔諭羌眾,反其田,報夏人以渭非其有,不應索,正其封疆而還,兵遂解。進天章閣待制,陜西都轉運使,加龍圖閣直學士、知慶州。



 環之定邊砦蕃官蘇恩,以小過疑懼而遁,將佐議致討。涇原既出師境上,求謂恩非素攜二者,乘以兵,必起邊患。
 但遣裨將從十數卒扣其帳,開以禍福,恩感泣,還砦如初。入判太常寺,權發遣開封府,遷樞密直學士、知定州,復以龍圖閣學士權開封。



 求本有吏能幹局,至是,春秋浸高,且病聵。三司大將錢吉密殺妹,為鄰所告,求不能決,反坐告者;又斷獄數差失。御史言其不勝任,出知兗州。卒,年七十一。



 張景憲,字正國,河南人。以父師德任淮南轉運副使。山陽令鄭昉贓累巨萬,親戚多要人,景憲首案治,流之嶺
 外,貪吏望風引去。徙京西、東轉運使。王逵居鄆,專持吏短長,求請賄謝如所欲,景憲上其惡,編置宿州。熙寧初,為戶部副使。



 韓絳築撫寧、囉兀兩城,帝命景憲往視。始受詔,即言城不可守,固不待到而後知也。未幾,撫寧陷。至延安,又言:「囉兀邈然孤城,鑿井無水,將何以守,臣在道,所見師勞民困之狀非一,願罷徒勞之役,廢無用之城,嚴飭邊將為守計。令邊郡召生羌,與之金帛、官爵,恐黠羌多詐,緩急或為內應,宜亟止之。」陜西轉運司議,欲
 限半歲令民悉納錢於官,而易以交子。景憲言:「此法可行於蜀耳,若施之陜西,民將無以為命。」其後卒不行。



 加集賢殿修撰,為河東都轉運使。議者欲分河東為兩路,景憲言:「本道地肥磽相雜,州縣貧富亦異,正宜有無相通,分之不便。」議遂寢,改知瀛州,上言:「比歲多不登,民債逋欠。今方小稔,而官督使並償,道路流言,其禍乃甚於兇歲。願以寬假。」帝從之,仍下其事。



 元豐初年,知河陽。時方討西南蠻,景憲入辭,因言:「小醜跳梁,殆邊吏擾之耳。
 且其巢穴險阻,若動兵遠征,萬一饋餉不繼,則我師坐困矣。」帝曰:「卿言是也,然朝廷有不得已者。」明年,徙同州,以太中大夫卒,年七十七。



 景憲在仁宗朝為部使者,時吏治尚寬,獨多舉刺;及熙寧以來,吏治峻急,景憲反濟以寬。方新法之行,不劾一人。居官不畏強御,非公事不及執政之門。自負所守,於人少許可,母卒,一夕須發盡白,世以此稱之。



 竇卞,字彥法,曹州冤句人,進士第二,通判汝州。秦悼王
 葬汝,宗室來汝者眾,役兵三千。郡守林濰以汝與其鄉近,因使輦薪芻、鐵石致其家。眾怨憤,謀殺濰,會日暮門閉,不果,遂挾大校叛。卞啟關招諭之,曰:「汝曹特醉酒狂呼爾,毋恐。」眾少定,乃密推首惡羈之,請於朝,詔濰致仕,悉配徙亂者。



 加集賢校理、知太常院,知絳州,開封府推官。方禁銷金為衣,皇城卒捕得之,屬卞治,以中禁為言。奏曰:「真宗行此制,自掖廷始,今不正以法,無以示天下,且非祖宗立法意。」英宗曰:「然。文王『刑於寡妻,至於兄弟,
 以御於家邦』正謂是也。」從其請。



 出知深州。熙寧初,河決滹沱,水及郡城,地大震。流民自恩、冀來,踵相接,卞發常平粟食之。吏白擅發且獲罪,卞曰:「俟請而得報,民死矣。吾寧以一身活數萬人。」尋以請,詔許之。外間訛言水大至,卞下令敢言者斬。一日,復報大水且至,吏請閉門,卞不可,既而果妄。時發六州卒築武強,陳卒惰,主者笞之,不服。卞曰:「廂兵犯將校,法不至重,然興役聚工,不可拘以常法。」命斬之以聞,有詔嘉獎。還為戶部判官、同修起
 居注,進天章閣待制,判昭文館、將作監。



 始,卞官汝時,與殿直王永年者相接頗厚,及在京師,永年求監金曜門庫,卞為禱提舉揚繪,繪薦為之。永年置酒於家,延繪、卞至,出其妻侑飲,且時致薄餉。永年以事系獄死,御史發其私,卞坐奪職,提舉靈仙觀。卒,年四十五。



 張瑰,字唐公,洎之孫也。舉進士,以婦父王飲若嫌,召試學士院,賜第,除秘閣校理、同知太常禮院。謚錢惟演曰文墨,其子撾登聞鼓上訴,仁宗使問狀,瑰條奏甚切,朝
 廷不能奪,乃賜謚曰思。溫成廟祠享如神御,請殺其禮。



 判吏部南曹,為開封府推官、知洪州。營校督役苛急,其徒三百人將以夜殺之。求不獲,持鍤噪於門,請易校。瑰召問諭遣,明日,推治黠十人,不為易校。積閥當遷,十年不會課,文彥博為言,特遷之。徙兩浙轉運使,加直史館、知穎州、揚州,即拜淮南轉運使。



 三司下諸道責羨財,淮南獨上金九錢,三司使怒,移文譙切,瑰以賦數民貧對。入修起居注、知制誥。草故相劉沆贈官制,頗言其附會
 取顯位。沆子瑾帥子弟婦女衰絰詣闕,哭訴瑰挾私怨,且醜詆其人。執政以褒贈乃恩典,瑰不當為貶詞,出知黃州,然瑾亦竟不敢請父謚。還判流內銓。英宗時,論第在先朝乞蚤定儲副者,進左諫議大夫、翰林侍讀學士。劉瑾又訟其判銓日調其子不應法,復出濠州。歷應天府、河南、河陽,請為太平州。



 瑰平生薦士,後雖不如所舉,未嘗以令自首,故再坐削階。當官遇事輒言,觸忤勢要,至屢黜,終不悔。卒,年七十。



 孫瑜,字叔禮,博平人。以父任為將作監主簿,賈昌朝薦為崇文檢討、同知禮院、開封府判官。



 使契丹,適西討捷書至,館伴要入賀,啖以厚餉,瑜辭以奉使有指,不肯賀。加秘閣校理、兩浙轉運使。入辭,仁宗訪其家世,謂曰:「卿孫奭子邪?奭,大儒也,久以道輔朕。」因面賜金紫。



 先是,郡縣倉庾以斗斛大小為奸,瑜奏均其制,黜吏之亡狀者,民大喜。有言其變新器非便,下遷知曹州。尋有言瑜所作量法均一誠便者,乃還其元資,徙知蔡州,毀吳元濟
 像,以其祠事裴度。大水緣城隙入,瑜使囊沙數千捍其沖,城得弗壞。更相、兗、濰、單四州,累官工部侍郎,卒,年七十九。



 始,奭之亡,朝廷錄其子孫,時瑜之子為諸孫長,瑜曰:「吾忍因父喪而官吾子乎?」以兄之孤上之。瑜天資整敏,齊家以嚴稱。善與人交,一受知終身不易。所薦士有過,或教使自言,曰:「已知之而復擠之。吾不為也。」



 論曰:「宋至神宗,承平百餘年,風行政成,士皆守官稱職,雖上之化,亦下之氣習使然也。當時仕於朝廷,出守方
 岳,持節一道,專對四方者,各有其人,其政跡且多可紀,自顓至瑜是已。顓能折夏人,屈元昊使者;參擊貪除害,乃心邊事;申錫除兇黨,詆權幸;求黥黠吏,禁盜鑄;卞以身活人;瑰不貢羨財;景憲因母死而發白;孫瑜不忍以父喪而得官:此其行尤昭昭者歟。



 許遵,字仲途,泗州人,第進士,又中明法,擢大理寺詳斷官、知長興縣。水災,民多流徙,遵募民出米振濟,竟以無患。益興水利,溉田甚博,邑人便利,立石紀之。



 為審刑院
 詳議官,知宿州、登州。遵累典刑獄,強敏明恕。及為登州,執政許以判大理,遵欲立奇以自鬻。會婦人阿雲獄起。初,云許嫁未行,嫌婿陋,伺其寢田舍,懷刀斫之,十餘創,不能殺,斷其一指。吏求盜弗得,疑云所為,執而詰之,欲加訊掠,乃吐實。遵按雲納採之日,母服未除,應以凡人論,讞於朝。有司當為謀殺已傷,遵駁言:「雲被問即承,應為按問。審刑、大理當絞刑,非是。」事下刑部,以遵為妄,詔以贖論。未幾,果判大理。恥用議法坐劾,復言:「刑部定議
 非直,雲合免所因之罪。今棄敕不用,但引斷例,一切按而殺之,塞其自守之路,殆非罪疑惟輕之義。」詔司馬光、王安石議。光以為不可,安石主遵,御史中丞滕甫、侍御史錢顗皆言遵所爭戾法意,自是廷論紛然。安石既執政,悉罪異己者,遂從遵議。雖累問不承者,亦得為按問。或兩人同為盜劫,吏先問左,則按問在左;先問右,則按問在右。獄之生死,在問之先後,而非盜之情,天下益厭其說。



 熙寧間,出知壽州,再判大理寺,請知潤州,又請提
 舉崇福宮。尋致仕,累官中散大夫。卒,年八十一。



 盧士宗,字公彥,濰州昌樂人。舉《五經》,歷審刑院詳議、編敕刪定官,提點江西刑獄。侍講楊安國以經術薦之,仁宗御延和殿,詔講官悉升殿聽其講《易》。明日,復命講《泰卦》,又召經筵官及僕射賈昌朝聽之。授天章閣侍講,賜三品服,加直龍圖閣、天章閣待制、判流內銓。



 李參、郭申錫有決河訟,詔士宗劾之。士宗言兩人皆為時用,有罪當驗問,不宜逮鞫。於是但黜申錫為州。進龍圖閣直學
 士、知審刑院、通進銀臺司。



 仁宗神主祔廟,禮院請以太祖、太宗為一世,而增一室以備天子事七世之禮。詔兩制與禮官考議,孫抃等欲如之。士宗以為:「在禮,太祖之廟,萬世不毀;其餘昭穆,親盡即毀,示有終也。自漢以來,天子受命之初,太祖尚在三昭、三穆之次,祀四世或六世,其以上之主,屬雖尊於太祖,親盡則遷。故漢元帝之世,瘞太上廟主於國,魏明帝遷處士主於國邑,晉武、惠祔廟,遷征西、豫章府君。大抵過六世則遷其主,蓋太祖
 已正東向之位,則並三昭三穆為七世矣。唐高祖初祀四世,太宗增祀六世,太宗祔廟則遷弘農府君,高宗祔廟又遷宣宗,皆前世成法,惟明皇九廟祀八世,於事為不經。今大行祔廟,僖祖親盡當遷,於典禮為合,不當添展一室。」詔抃等再議,卒從八室之說,議者咎之。



 出知青州,入辭,英宗曰:「學士忠純之操,朕所素知,豈當久處外。」命再對,及見,論知人安民之要,勸帝守祖宗法。御史言其罕通吏事,且衰病,改沂州。



 熙寧初,以禮部侍郎致仕,
 卒,年七十一。士宗以儒者長刑名之學,而主於仁恕,故在刑部審刑,前後十數年。



 錢象先,字資元,蘇州人。進士高第,呂夷簡薦為國子監直講,歷權大理少卿、度支判官、河北、江東轉運使,召兼天章閣侍講。詳定一路敕成,當進勛爵,仁宗以象先母老,欲慰之,獨賜紫章服。進待制、知審刑院,加龍圖閣直學士,出知蔡州。



 象先長於經術,侍邇英十餘年,有所顧問,必依經以對,反復諷諭,遂及當世之務,帝禮遇甚渥。
 故事,講讀官分日迭進,像先已得蔡,帝猶諭之曰:「大夫行有日矣,宜講徹一編。」於是同列罷進者浹日。徙知河南府、陳州,復兼侍講、知審刑院。



 象先旁通法家說,故屢為刑官,條令多所裁定。嘗以為犯敕者重,犯令者輕,請移敕文入令者甚眾。又議告捕法,以為罪有可去,有可捕,茍皆許捕,則奸人將倚法以害善良,因削去許捕百餘事。其持心平恕類此。復知許、穎、陳三州,以吏部侍郎致仕。卒,年八十一。



 韓璹,字君玉,衛州汲人。登進士第,知定州安喜縣。為政強力,能使吏不賄,守韓琦稱其才。為開封司錄。嘉祐寬恤諸道,分遣使者。□曰:「京師諸夏本,顧獨不蒙惠乎?」乃具徭役利害上之,詔司馬光、陳洙詳定條式,遂革大姓漁並之弊。提點利州路、河北刑獄,以開封府判官迎契丹使。使問:「南朝不聞打圍,何也?」□曰:「我後仁及昆蟲,非時不為耳。」



 熙寧初,為梓州路轉運使。朝廷命諸道議更役法,璹首建並綱減役之制,綱以數計者百二十有八,
 衙前以人計者二百八十有三,省役人五百。又請裁定諸州衙簿,於是王安石言:「□所言皆久為公私病,監司背公養譽,莫之或恤,而獨能體上意,宜加賞。」乃下褒詔,且賜帛二百。入為鹽鐵副使,以右諫議大夫知澶州。坐失舉,降太常少卿。河決,晝夜捍禦。神宗念其勞,復故官太中大夫,判將作監,轉正議大夫致仕。卒,年七十七。



 □吏事絕人,閱按牘,終身不忘,澶州民懷思之。他日,郡守或欲有所為,民必曰:「此已經韓太中矣。」以故輒止。



 杜純,字孝錫,濮州鄄城人。少有成人之操,伯父沒官南海上,其孤弱,樞不能還。純白父請往,如期而喪至。



 以蔭為泉州司法參軍。泉有蕃舶之饒,雜貨山積。時官於州者私與為市,價十不償一,惟知州關詠與純無私買,人亦莫知。後事敗,獄治多相牽系,獨兩人無與。詠猶以不察免,且檄參對。純憤懣,陳書使者為訟冤,詠得不坐。



 熙寧初,以河西令上書言政,王安石異之,引置條例司,數與論事,薦於朝,充審刑詳議官。或議復肉刑,先以刖代
 死刑之輕者,純言:「今盜抵死,歲不減五十,以死懼民,民常不畏,而況於刖乎?人知不死,犯者益眾,是為名輕而實重也。」事遂寢。



 秦帥郭逵與其屬王韶成訟,純受詔推鞫,得韶罪。安石主韶,變其獄,免純官。韓絳為相,以檢詳三司會計。安石再來,乃請監池州酒。久之,為大理正。上言:「朝廷非不惡告訐,而有覘事者以擿抉隱微,蓋京師聚萬姓,易以宿奸,於計當然,非擾人也。比來或徒隸觖望,或民相怨仇,或意冒告賞,但泛云某有罪,某知狀,官
 不識所逮之囚,囚不省見逮之故。若許有司先計其實,而坐為欺者以誣告,當無不竟矣。」



 隰州商尹奇貿溫泉礬有羨數,雲官潤之,寺欲械訊河東。純曰:「奇情止爾,若傅致其罪,恐自是民無復敢貨礬,則數百萬之儲,皆為土石。請姑沒其羨而釋其人。」曹州民王坦避水患,以車載貨入京,征商者以為匿稅,寺議黥坦,純復爭之,卿楊汲奏為立異,又廢於家。



 元祐元年,范純仁、韓維、王存、孫永交薦之,除河北轉運判官。初更役書,司馬光稱其論
 議詳盡,予之書曰:「足下在彼,朝廷無河北憂。」純因建言:「河防舊隸轉運,今乃領屬都水外丞,計其決溢之變,前日不加多,今日不加少。然出財之司。則常憂費而緩不急;用財之官,則寧過計而無不及,不如使之歸一。」後如其言。



 召為刑部員外郎、大理少卿,擢侍御史。言者詆其不由科第,改右司郎中,尋知相州,徙徐州,陜西轉運使。還,拜鴻臚、光祿卿,權兵部侍郎,謝病,以集賢院學士提舉崇福宮,改修撰。卒,年六十四,弟紘。



 紘字君章,起進士,為永年令。歲荒,民將他往,召諭父老曰:「令不能使汝必無行,若留,能使汝無饑。」皆喜聽命。乃官給印券,使稱貸於大家,約歲豐為督償,於是咸得食,無徙者。明年稔,償不愆素。神宗聞其材,用為大理詳斷官、檢詳樞密刑房,修《武經要略》。以職事對,帝翌日語宰相,嘉其論奏明白,未果用。



 紘每議獄,必傅經誼。民間有女幼許嫁,未行而養於婿氏,婿氏殺以誣人,吏當如昏法。紘曰:「禮,婦三月而廟見,未廟見而死,則歸葬於家,示
 未成婦也。律,定昏而夫犯,論同凡人。養婦雖非禮律,然未成婦則一也。」議乃定。又論:「天下囚應死,吏懦不行法,輒以疑讞。夫殺人而以疑讞,是縱民為殺之道也。請治妄讞者。」不從。



 擢刑部郎中。元祐初,為夏國母祭奠使。時夏人方修貢,入其國,禮猶倨,迓者至衣毛裘,設下人坐,蒙以黲,且不跪受詔。紘責之曰:「天王吊禮甚厚,今不可以加禮。」夏人畏懼加敬。他日,夏使至,請歸復侵疆。紘逆之至館,使欲入見有所陳,紘止之,答語頗不遜。紘曰:「國
 主設有請,必具表中,此大事也,朝廷肯以使人口語為可否乎?」隨語連拄之,乃不敢言。



 遷右司郎中、大理卿,以直秘閣知齊、鄧二州,復為大理卿,權刑部侍郎,加集賢殿修撰,為江淮發運使、知鄆州。獄系囚三百人,紘至之旬日,處決立盡。又以刑部召,未至,還之鄆。



 嘗有揭幟城隅,著妖言其上,期為變,州民皆震。俄而草場白晝火,蓋所揭一事也,民又益恐。或請大索城中,紘笑曰:「奸計正在是,冀因吾膠擾而發,奈何墮其術中?彼無能為也。」居
 無何,獲盜,乃奸民為妖如所揣,遂按誅之。徙知應天府,卒,年六十二。



 紘事兄純禮甚備。在鄆州聞訃,泣曰:「兄教我成立,今亡不得臨,死不瞑矣。」適詣闕,迎其柩於都門,哀動行路。悉以奉錢給寡嫂,推其子恩,官其子若孫一人。宦京師時,里人馬隨調選,病臥逆旅,紘載與歸,醫視之。隨竟死,為治喪第中。或以為嫌,不自恤,其風義蓋天性云。



 杜常,字正甫,衛州人,昭憲皇后族孫也。折節學問,無戚
 裏氣習。嘗跨驢讀書,驢嗜草失道,不之覺,觸桑木而墮,額為之傷。中進士第,調河陽司法參軍事,富弼禮重之。積遷河東轉運判官,提點河北刑獄,歷兵部左司郎中、太常少卿、太僕太府卿、戶工刑吏部侍郎,出知梓州、青、鄆、徐州、成德軍。



 崇寧中,至工部尚書,以龍圖閣學士知河陽軍。苦旱,及境而雨,大河決,直州西上埽,勢危甚。常親護役,徙處埽上,埽潰水溢,及常坐而止。於是役人盡力,河流遂退,郡賴以安。卒,年七十九。



 謝麟,字應之,建州甌寧人。登第,調會昌令。民被酒夜與仇鬥,既歸而所親殺之,因誣仇。麟知死者無子,所親利其財,一訊得實。再調石首令,縣苦江水為患,堤不可御,麟疊石障之,自是人得安堵,號「謝公堤」。



 通判辰州。章惇使湖湘,拓沅州,薦麟為守,由太常博士改西上閣門副使。



 賊犯辰溪,麟且捕且招,一方以寧。詔使經制宜州獠,降其種落四千八百人,納思廣洞民千四百室,得鎧甲二萬,褒賜甚渥。加果州刺吏,知荊南、涇、邠二州。



 元祐
 初,復以朝議大夫、直秘閣知潭州,加直龍圖閣,歷徙江寧、鳳翔府、渭桂二州。融江有夷警,將吏議致討,麟以計平之。戍兵從北來,不能水土,麟部土人使極南,而北兵止屯近郡,賴以全者甚眾。卒於官。



 王宗望,字磻叟,光州固始人。以蔭累擢夔州路轉運副使。哲宗即位,行赦賞軍,萬州彌旬不給。庖卒朱明因眾怒,白晝入府宅,傷守臣,左右驚散,他兵籍籍謀兆亂。宗望聞變,自夔疾驅至,先命給賞,然後斬明以徇,且竄視
 守傷而不救者。乃自劾,朝廷嘉之。歷倉部郎中、司農少卿、江淮發運使。



 楚州沿淮至漣州,風濤險,舟多溺。議者謂開支氏渠引水入運河,歲久不決,宗望始成之,為公私利。代吳安持為都水使者。自大河有東、北流之異,紛爭十年,水官無所適從。宗望謂回河有創立金堤七十里,索緡錢百萬,詔從之。右正言張商英論其誕謾,而宗望奏已有成績,遂增秩三等,加直龍圖閣、河北都轉運使,擢工部侍郎,以集賢殿修撰知鄆州。卒,年七十七。元
 符中,治其導河東流事,以為附會元祐,追所得恩典云。



 王吉甫,字邦憲,同州人。舉明經,練習法律,試斷刑入等,為大理評事,累遷丞、正、刑部員外郎、大理少卿。



 舒但以官燭引至第,執政欲坐以自盜。吉甫謂不可,執政怒,移獄他所,吉甫亦就辨。但乃用飲食論罪,不以燭也。南郊起幔城,役卒急於畢事,董役者責之曰:「此殆類白露屋耳。」卒訴之。吏當非所宜言論死。吉甫謂非咒詛不應死,遂求對。神宗怒曰:「得非為白露屋事來邪?」吉甫從容敷
 陳,不少懾,帝為霽怒,其人得釋。蘇軾南遷,所過,郡守有延館之者,走馬使上聞,詔鞫之。吉甫議當笞,宰相章惇不悅。吉甫曰:「法如是,難以增加成罪。」卒從笞。太倉火,議誅守者十餘人,亦爭之,皆得不死。其持論寬平,大抵類此。



 請知齊州、梓州。梓在東川為壯藩,戶口最盛。轉運使欲增折配以取羨餘。吉甫謂其僚曰:「民力竭矣,一增之後,不可復減,吾寧貽使者怒,忍為國斂怨、為民基禍哉。」竟卻之。歷提點梓州路京畿刑獄、開封少尹、知同、刑、漢
 三州,以中大夫卒,年七十。



 吉甫老於為吏,廉介不回,但一於用法,士恨其少緣飾云。



 論曰:宋取士兼習律令,故儒者以經術潤飾吏事,舉能其官。遵惠政及民,而緩登州婦獄,君子謂之失刑。士宗、象先皆執經勸講,其為刑官,論法平恕,宜哉,□吏事絕人,民懷其德。純以微官能著清節,紘議獄必傅經誼,風義藹然。常坐護危埽,麟定、獠,宗望弭萬州之變,皆靖至難之事於談笑間。吉甫一於用法,而廉介不回,有足
 稱云。



\end{pinyinscope}