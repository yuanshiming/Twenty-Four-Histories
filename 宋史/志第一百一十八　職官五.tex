\article{志第一百一十八 職官五}

\begin{pinyinscope}

 大理寺鴻臚寺司農寺太府寺國子監少府監將作監軍器監都水監司天監



 大理寺舊置判寺一人,兼少卿事一人。建隆三年,以工部尚書竇儀判寺事。凡獄訟之事,隨官司決劾,本寺不復聽訊,但掌斷天下奏獄,送審刑院詳汔,同署以上於朝。詳斷官八人,以京官充,國初,大理正、丞、評事皆有定員,分掌斷獄。其後,擇他官明法令者,若常參官則兼正,未常參則兼丞,謂之詳斷官。舊六人,後加至十一人,又去兼正、丞之名。咸平二年始定置。



 法直官二人,以幕府、州縣官充,改京官則為檢法官。



 元豐官制行,置卿一人,少卿二人,正
 二人,推丞四人,斷
 丞六人,司直六人,評事十有二人,主簿二人。卿掌折獄、詳刑、鞫讞之事。同職務分左右:天下奏劾命官、將校及大闢囚以下以疑請讞者,隸左斷刑,則司直、評事詳斷,丞議之,正審之。若在京百司事當推,或特旨委勘及系官之物應追究者,隸右治獄,則丞專推鞫。蓋少卿分領其事,而卿總焉。凡刑獄應審議者,上刑部。被旨推鞫及情犯重者,卿同所隸官請封奏裁。若獄空或斷絕,則御史按實以聞。分案十有一,置吏
 六十有九。



 先是舊制,大理寺讞天下奏案而不治獄。熙寧五年,增詳斷官二為十員。七年,置詳斷習學官十
 四,詳覆習學官六。
 九年,詔以「京師官寺,凡有獄皆系開封府司錄司及左右軍巡三院,囚逮猥多,難於隔訊,又暑多瘐死,因緣流滯,動涉歲時。稽參故事,宜屬理官,可復置大理獄。」始命崔臺符為知卿
 事,蹇周輔、楊汲為少
 卿,各舉丞及檢法官。初,神宗謂國初廢大理獄非是,以問孫洙,洙對合旨,至是,命官起寺,十七日而成。元豐二年手詔:「大理寺近舉墜典,俾治獄事,推輪規摹,皆以義起,不少寬假,必懷顧忌,稽留弊害,無異前日。宜依推制院及御史臺例,不供報糾察司。」三年,詔依舊供報。凡官屬依御史臺例,謁有禁。又詔糾察司察訪本寺
 斷徒以上出入不當者,索案點檢。五年,詔毋以大理寺官為試官。六年,又詔:「凡斷公案,先上正看詳當否,論難改正,簽印注日,然後過議司復議;如有批難,具記改正,長貳更加審定,然後判成錄奏。」又刑部言:「應吏部補授大理寺左斷刑官,先與刑部、大理寺長貳同議可否,然後注擬。仍取經試得循資以上人充,正闕以丞補,丞闕以評事補。」
 詔刑部、吏部同著為令。八年,詔大理寺推斷事應奏及上尚書省者,更不先申本曹。



 元祐元年,以右治獄勘斷公事全少,並左右兩推為一司。三年,三省請罷右治獄,依三司舊例置推勘檢法官於戶部,從之。又詔大理寺並置長貳。四年,從刑部請,改本寺條,任大理官失斷徒已上五人或死罪二人,不在選限。



 舊條,失斷徒已上三人或死罪一人。



 紹聖元年,詔斷刑獄
 官依元豐元年選試法。二年,復置右治獄,置官屬如元豐制。左右推事有翻異者互送,再有異者,朝廷委官審問,或送御史臺治之。元符元年,應大理寺、開封府承受內降公事,不得奏請移送。又詔應奏斷公事,依開封府專條,不許諸處取索。



 崇寧四年,詔大理寺官諸
 司
 輒奏闢者,以違制論。政和二年,詔法官任滿,擇職事修舉、人材可錄者奏舉再任,仍許就任關升,理本等資序。五年,依熙、豐故事,復置習學公事四員,長、貳立課程,正、丞同指教。宣和七年,評事以上並差試中刑法人。又詔大理寺、開封府承受公事依法斷遣,不得乞降特旨。中興並省官寺,惟大理寺不並。



 紹興初,詔正與丞並堂除。評事
 闕,則委本寺長、貳選擇應格人赴刑部議定,申朝廷差填。如無應格,即選諳習刑法人權充。又立比較法以懲差失。隆興二年,評事鞏衍言:「評事檢斷,躬自節案。親書斷語,最為勞若。」詔增置,以八員為額。淳熙末,嚴寺官出謁之禁,以防請托、漏洩之弊。紹熙初,除試中刑法評事八員外,司直、主簿選用有出身曾歷任人,各兼評事系銜。將八評事已擬斷文字,分兩廳點檢。或有未安,則述所見與長、貳商量。慶元四年,定逐委仲月定日斷絕之
 法。嘉定八年,申嚴紹熙指揮,重司直、主簿之選,增選試取人數以勸法科。



 左斷刑分案三:曰磨勘,掌批會吏部等處改官事;曰宣黃,掌凡斷訖命官指揮;曰分簿,掌行分探諸案文字。設司有四:曰表奏議,掌拘催詳斷案八房斷議獄案,兼旬申月奏;曰開拆;曰知雜;曰法司。又有詳斷案八房,專定斷諸路申奏獄案等。又有敕庫,掌收管架閣文書。吏額;胥長一人,胥史三人,胥佐三十人,貼書六人,楷書十四人。



 隆興共減七人。



 右治獄分案有四:曰左右
 寺案,掌斷訖公事案後收理追贓等;曰驅磨,掌驅磨兩推官錢、官物、文書;曰檢法,掌檢斷左右推獄案並供檢應用條法;曰知雜。又有開拆、表奏二司;有左右推,主鞫勘諸處送下公事及定奪等。吏額;前司胥史一人、胥佐九人,表奏司一人、貼書三人,左右推胥史二人、胥佐八人、般押推司四人、貼書四人。



 隆興共減五人。



 鴻臚寺舊置判寺事一人,以朝官以上充。元豐官制行,置卿一人,少卿一人,丞、主簿各一人。卿掌四夷朝貢、
 宴勞、給賜、送迎之事,及國之兇儀、中都祠廟、道釋籍帳除附之禁令,少卿為之貳,丞參領之。凡四夷君長、使價朝見,辨其等位,以賓禮待之,授以館舍而頒其見辭、賜予、宴設之式,戒有司先期辦具;有貢物,則具其數報四方館,引見以進。諸蕃封冊,即行其禮命。若崇義公承襲,則辨其嫡庶,具名上尚書省。其周嵩、慶、懿陵廟,命官以時致享,若兇儀之節,宗室以服,臣僚以品,辨其喪紀而詔奠臨賻贈之制。禮儀成服,則卿掌贊導之儀,葬則預
 戒有司具鹵簿儀物。分案四,置吏九。其官屬十有二:往來國信所,掌大遼使介交聘之事。都亭西驛及管幹所,掌河西蕃部貢奉之事。禮賓院,掌回鶻、吐蕃、黨項、女真等國朝貢館設,及互市譯語之事。懷遠驛,掌南蕃交州,西蕃龜茲、大食、于闐、甘、沙、宗哥等國貢奉之事。中太一宮、建隆觀等各置提點所,掌殿宇齋宮、器用儀物、陳設錢幣之事。在京寺務司及提點所,掌諸寺葺治之事。傳法院,掌譯經潤文。左、右街僧錄司,
 掌寺院僧尼帳籍及僧官補授之事。同文館及管勾所,掌高麗使命。已上並屬鴻臚寺。中興後,廢鴻臚不置,並入禮部。



 司農寺舊置判寺事二人,以兩制、朝官以上充;主簿一人,以選人充。掌供藉田九種,大中小祀供豕及蔬果、明房油,與平糶、利農之事。



 元豐官制行,始正職掌,置卿、少卿、丞、主簿各一人。卿掌分儲委積之政令,總苑囿庫務之事而謹其出納,少卿為之貳,丞參領之。凡京都官
 吏錄廩,辨其精粗而為之等;諸路歲運至京師,遣官閱其名色而分納於倉庚,蒿秸則歸諸場,歲具封樁、月具見存之數奏聞;給兵食則進呈糧樣,因出納而受賂刻取者,嚴其禁;有負欠者,計其虧數上於倉部。凡諸路奏雨雪之闕與過多者皆籍之。凡苑囿行幸排比及薦饗進御、頒賜植藏之物,戒有司先期辦具。造曲薛、儲薪炭以待給用。天子親耕藉田,有事於先農,則卿奉耒耜,少卿率屬及庶人以終千畝。分案六,置吏十有八。



 初,熙寧
 二年,置制置條例司,立常平斂散法,遣諸路提舉官推行之。三年五月,詔制置司均通天下之財,以常平新法付司農寺,增置丞、簿,而農田水利、免役、保甲等法,悉自司農講行。初以太子中允呂惠卿判司農寺,改同判寺胡宗愈為兼判。四年,以御史知雜鄧綰判寺,曾布同判,詔諸路提舉常平官課績,田寺考校升絀,管幹官令提舉司保明,計功賞之。六年,以司農間遣屬官出視諸路,力有不給,乃置乾當公事官,以葉康直等四人為之。七
 年,本寺言:「所主行農田水利、免役、保甲之法,措置未盡,官吏推行多違法意,欲榜諭官私,使人陳述,有司違法,從寺按察。」九年,以乾當公事官所至輒用喜怒,罷之,從熊本請也。元豐四年,減丞一,主簿三。官制行,寺監不治外事,司農事舊職務悉歸戶部右曹。



 元祐三年,詔司農寺置長、貳。五年,以本寺主簿兼檢法。八年,復置提轄修倉所;紹聖元年,詔罷官屬,以其事歸將作監。四年,罷主簿,添丞一員。



 政和六年,浙西諸州各置排岸一員,從兩浙
 運副應安道請也。所隸官屬凡五十,倉二十有五,掌九穀廩藏之事,以給官吏、軍兵祿食之用。凡綱運受納及封樁支用,月具數以報司農。草場十有二,掌受京畿芻秸,以給牧監飼秣。排岸司四,掌水運綱船輸納雇直之事。園苑四:玉津、瑞聖、宜春、瓊林苑,掌種植蔬蒔以待供進,修飭亭宇以備游幸宴設。下卸司,掌受納綱運。都曲院,掌造曲,以供內酒庫酒醴之用,及出鬻以收其直。水磨務,掌水磑磨麥,以供尚食及內外之
 用。內柴炭庫,掌諸薪炭,以給宮城及宿衛班直軍士薪炭席薦之物。炭場,掌儲炭以供百司之用。



 建炎三年,罷司農寺,以事務並隸倉部。紹興三年,復置丞二員。凡有合行事務,申戶部施行。四年,復置寺,仍置卿、少。十年,復置簿。隆興元年,並省主簿一員。明年,詔如舊制。乾道三年,詔糧綱有欠,從本寺斷遣監納,情理重者,大理寺推勘。分案五,南北省倉、草料場、和糴場隸焉。監倉官分上、中、下界,司其出納。諸場皆置監官。外有監門官,
 交量則有檢察斛面官,綱運下卸有排岸司官,各分其事以佐本寺。豐儲倉所,置監官二員,監門官一員。初,紹興以上供米餘數,樁管別廩,以為水旱之助,後又增廣收糴,淳熙間,命右司為之提領,後以屬檢正,非奉朝廷指揮不許支撥。別置赤歷,提領官結押,不許袞同司農寺收支經常米數。凡外州軍起到樁管米,從司農寺差官盤量,據納到數報本所樁管。監官、監門官遇考任滿,所屬批書外,仍於本所批書,視其有無欠折,以定其
 功過。在外,則鎮江、建康置倉焉。



 太府寺舊置判寺事一人,以兩制或帶職朝官充;同判寺一人,以京朝官充。凡廩藏貿易、四方貢賦、百官奉給,時皆隸三司,本寺但掌供祠祭香幣、帨巾、神席,及校造斗升衡尺而已。



 元豐官制行,始正職掌,置卿、少卿各一人,丞、主簿各二人。卿掌邦國財貨之政令,及庫藏、出納、商稅、平淮、貿易之事,少卿為之貳,丞參領之。凡四方貢賦之輸於京師者,辨其名物,視其多寡,別而受之。儲
 於內藏者,以待非常之用;頒於左藏者,以供經常之費。凡官吏、軍兵奉祿賜予,以法式頒之,先給歷,從有司檢察,書其名數,鉤覆而後給焉。供奉之物,則承旨以進,審奏得晝,乃聽除之。若春秋授軍衣,則前期進樣,定其頒日,畿內將校營兵支請,月具其數以聞。凡商賈之賦,小賈即門征之,大賈則輸於務。貨之不售者,平其價鬻於平淮,乘時賒貸,以濟民用;若質取於官,則給用多寡,各從其抵。歲以香、茶、鹽鈔募人入豆穀實邊。即京都闕用
 物,預報度支。凡課入,以盈虧定課最、行賞罰。大祀,晨裸則卿置幣,奠玉則入陳玉帛,餘祀供其帨巾。分案九,置吏六十有五。



 元祐初,以倉部郎官印發文鈔,三年,復歸本寺。又詔太府置長、貳。五年,令長貳每月分巡所轄庫務。元符元年,增置丞一員。三年,改市易案為平淮,其市易務亦如之。崇寧中,置藥局七所,添丞一員點檢。宣和三年減罷。靖康元年,詔內外官司局所依熙寧法,錢物並納左藏庫,凡省一百五所。又詔戶部、太府寺長貳當
 職官及本庫官吏俸錢,候在京官吏支散並足,方許支給,從戶部尚書梅執禮之請也。



 所隸官司二十有五:左藏東西庫,掌受四方財賦之入,以待邦國之經費,給官吏、軍兵奉祿賜予。舊分南北兩庫,政和六年修建新庫,以東西庫為名。西京、南京、北京各置左藏庫、內藏庫,掌受歲計之餘積,以待邦國非常之用。奉宸庫,掌供內庭,凡金玉、珠寶、良化賄藏焉。祗候庫,掌受錢帛、器皿、衣服,以備傳詔頒給及殿庭賜予。元豐庫,掌
 受諸路積剩及常平錢物,凡封樁者皆入焉。



 神宗常憤契丹倔強,慨然有恢復幽燕之志,聚金帛內帑,自制四言詩一章,曰:「五季失國,獫狁孔熾。藝祖造邦,思有懲艾。爰設內府,基以募士,曾孫保之,敢忘闕志。」每庫以詩一字目之,儲積皆滿。又別置庫,賦詩二十字,分揭於庫,曰:「每虔夕惕心,妄意遵遺業,顧予不武姿,何日成戎捷。」微宗朝,又有崇寧庫、大觀庫。



 布庫,掌受諸道輸納之布,辨其名物,以待給用。茶庫,掌受江、浙、荊湖、建、劍茶茗,以給翰林諸司及賞賚、出鬻。雜物庫,掌受內外雜輸之物,以備支用。糧料院,掌以法式頒廩祿,凡文武百官、諸司、諸軍奉料,以卷淮給。審計司,掌審其
 給受之數,以法式驅磨。都商稅務,掌收京城商旅之算,以輸於左藏。汴河上下鎖、蔡河上下鎖,掌舟船木筏之徵。都提舉市易司,掌提點貿易貨物,其上下界及諸州市易務、雜買務、雜賣場皆隸焉。市易上界,掌斂市之不售、貨之滯於民用者,乘時貿易,以平百物之直。市易下界,掌飛錢給券,以通邊糴。雜買務,掌和市百物,凡宮禁、官府所需,以時供納。雜賣場,掌受內外幣餘之物,計直以待出貨,或淮折支用。榷貨務,掌
 折博觔斗、金帛之屬。交引庫,掌給印出納交引錢鈔之事。抵當所,掌以官錢聽民質取而濟其緩急。和劑局、惠民局,掌修合良藥,出賣以濟民疾。店宅務,掌管官屋及邸店,計置出僦及修造之事。石炭場,掌受納出賣石炭。香藥庫,掌出納外國貢獻及市舶香藥、寶石之事。



 建炎詔罷太府寺,以其所掌職務撥隸金部。紹興元年,復以章億守太府寺丞,措置印給茶鹽鈔引,續添置丞二員。四年,復置卿、少各一員。十年,復置主
 簿。十一年,詔交引庫書押鈔引寺丞兩員。遇合推賞。各與減磨勘二年。尋詔三丞一體行之。隆興元年,並省主簿一員,明年如舊制,設案七,以序次分管。監交案,隨逐丞簿赴左藏庫監交看驗綱運錢物。中興後,所隸惟有糧料院、審計司、左藏東西庫、交引庫、祗候庫、和劑局、惠民局如前制所置。左藏南庫,系樁



 管御前激賞庫改。以侍從官提領,又置提轄檢察官一員,編估局、打套局,二局系揀選市舶香藥雜物等第,會其直以待貿易。



 寄樁庫,掌發賣香藥、匹帛,拘其直歸於左藏南庫。置監官
 提領二人。



 國子監舊置判監事二人,以兩制或帶職朝官充,凡監事皆總之。直講八人,以京官、選人充,掌以經術教授諸生,舊以講書為名,無定員。淳化五年,判監李至奏為直講,以京朝官充。其後,又有講書、說書之名,並以幕職、州縣官充。其熟於講說而秩滿者,稍遷京官。皇祐中,始以八人為額,每員各專一經,並選擇進士並《九經》及第之人,相參薦舉。丞一人,以京朝官或選人充,掌錢穀出納之事。主簿一人,以京官或選人充,掌文簿以勾考其出納。



 舊制,祭酒闕,始置判監事。



 監生無定員。



 並有蔭及京畿人,初隸監授業,後補監生;或隨屬游官,以
 久離本貫,不克赴鄉薦,而文藝可稱,亦許隸補試。廣文教進士,太學教《九經》、《五經、》《三禮》、《三傳》學究,律學館教明律,餘不常置。



 元豐官制行,始置察酒、司業、丞、主簿各一人,太學博士十人,舊系國子監直講,元豐三年,詔改為太學博士,每經二人。



 正、錄各五人,武學博士二人,律學博士、正各一人。



 祭酒掌國子、太學、武學、律學、小學之政令,司業為之貳,丞參領監事。凡諸生之隸於太學者,分三舍。始入學,驗所隸州公據,以試補中者充外舍。齊長、諭月書其行藝於籍。行謂率教不戾規矩,藝謂治經程文,季終考於學諭,次學錄,次學
 正,次博士,然後考於長貳。歲終校定,具注于籍以俟覆試,視其校定之數,參驗而序進之。凡私試,孟月經義,仲月論,季月策。公試,初場以經義,次場以論、策。試上舍如省試法。凡內舍行藝與所試之等俱優者,為上舍上等,取旨命官;一優一平為中,以俟殿試;一優一否或俱平為下,以俟省試。唯國子生不預考選。凡課試、升黜、教導之事,長、貳皆總焉。車駕幸學,則率官屬諸生班迎,即行在距學百步亦如之。凡釋奠於先聖、先師及武成王,則
 率官屬諸生共薦獻之禮。歲計所隸三舍生升降多寡之數,以為學官之殿最賞罰。



 博士,掌分經講授,考校程文,以德行道藝訓導學者。正、錄,掌舉行學規,凡諸生之戾規矩者,待以五等之罰,考校訓導如博士之職。職事學錄五人,掌與正、錄通掌學規。學諭二十人,掌以所授經傳諭諸生,直學四人,掌諸生之籍及幾察出入。凡八十齊,齊置長、諭各一人,掌表率齋生,凡戾規矩者,糾以齊規五等之罰,仍月考齋生行藝,著於籍。武
 學博士、學諭各二人,掌以兵書、弓馬、武藝訓誘學者。律學博士二人,掌傳授法律及校試之事。小學,置職事教諭二人,掌訓導及考校責罰。學長二人,掌充齒位、糾不如儀者。集正二人,掌籍諸生名氏,糾程課不逮者。



 熙寧初,詔用經術取士,廣闊黌舍。分為三學,增置生徒,總二千八百人。隸籍有數,給食有等,庫書有官,治疾有醫。分案八,置吏十。元豐三年,詔自今奏舉太學博士,先以所業進呈。五年,詔國子監官差承務郎以上,闕即
 差選人充正官,立行、守、試請奉法。八年,詔罷太學保任同罪法。



 元祐元年,詔太學每歲公試,以司業、博士主之,如春秋補試法。左司諫王嚴叟言:「太學生補中人,乞並許應舉,罷一年之限。」詔國子監立法。又詔給事中孫覺、秘書少監顧臨、崇政殿說書程頤、國子監長貳看詳修立國子監條例。又詔置《春秋》博士一員,二年,增司業一員。又詔內外學官選年三十以上歷任人充。三年,詔國子監置長貳。四年,詔太學正、錄依熙寧法,選上舍生充,
 闕則以內舍生。五年,殿中侍御史岑象求言:「國子監無叩問師資之益,學官不以訓導為己任,補試伺察不嚴,有假手之弊。」詔禮部相度以聞,本部言:「生員遇有請益,許見長貳。仍詔生員以所納齊課於講堂上指諭,並委博士逐月巡所隸齊,詢考生員所業。凡私試不鎖宿,欲令不罷講說。」從之。



 紹聖元年,監察御史劉拯言:「太學復行元豐中三舍推恩注官、免省試、免解試之制。夫舊法欲行,必先嚴考察。請自今太學長貳、博士、正錄,選學行
 純備、從所推服者為之,有弛慢不公,考察不實,則重加譴責。差職掌長諭改正如元豐舊制。」從之。又詔:「內外學官非制科、進士出身及上舍生入官者,並罷。」又詔:「太學正、錄依元豐舊制,各置五人。」又詔:「太學三舍生並依元豐學制,重行考察,依舊條推恩。」左司諫翟思言:「元豐《太學令》訓迪糾禁亦具矣,今追復經義取士,乞令有司看詳,依舊頒行。」詔送國子監,又詔:「內外學官選進士出身及經明行修人。」又詔學官並召試,國子監長貳、臺諫官、
 外監司皆許薦舉。三年,司業龔原言:「公試依元豐舊制,以長貳監試,輪差博士五員考試,乞朝廷更差官五員參考。」從之。元符元年,詔有官人許入太學充監生,毋過四十人。三年,復置《春秋》博士。



 崇寧元年省罷。



 寧元年,宰臣蔡京言:「有詔天下皆興學貢士,以三舍考選法遍行天下,聽每三年貢入太學。上合試仍別為考,分為三等,若試中上等,補充太學上舍,試中中等、下等者,補充內舍,餘為外舍生,仍建外學於國之南,待其歲考行藝,升之太
 學。其外學官屬:司業一人,丞一人,博士十人,學正五人,學錄五人;職事人系學生充;學錄五人,學諭十人,直學二人,齊長、齊諭各一人。外舍生三千人,太學上舍一百人,內舍三百人,候將來貢試到合格者,即上合以二百人、內舍以六百人為額。處上舍、內舍於太學,處外舍於外學。外學置齊一百,講堂四,每齊三十人。太學自訟齊移於外學。諸路貢士並入外學,候依法考選校試合格,升之太學為上舍、內舍生。見為太學外舍生,依舊在太
 學,候外學成日取旨。外學並依太學敕、令、格、式。」從之。二年,罷《春秋》博士。三年,詔闢雍置司成、司業各一員。四年,詔:「闢雍待四方貢士,在國之郊,太學教養上舍生,在王城之內,內上既殊,高下未倫;闢雍有司成在侍郎之次,國子有祭酒、司業列於卿、少,事體不順,合行厘正。」改闢雍司成為太學司成,總國子監及內外學事,凡學之事,皆許專達。仍立學官謁禁。



 大觀元年,置國子博士四員,國子正、錄各二員。太學、闢雍博士共置二十員,國子、太
 學每經一員,闢雍二員。從薛昂之請也。三年,詔諸路贍學餘錢並起發充在京學事支用。四年,詔省國子、闢雍博士五員,太學命官學錄一員,闢雍二員,國子命官正、錄及命官直學、國子監書庫官等官,並省罷,依紹聖格,毋用謄錄。政和元年,詔兩學博士、正、錄依舊制選試,朝廷除授。七年,新提舉河東路學事王格言:「崇寧初,建闢雍於郊,以處貢士及外舍生,立太學於國,以處上舍、內舍。由州、郡而貢之闢雍,由闢雍而升之太學。法行之初,
 上、內舍之選未眾,故外舍有校定者留太學,無校定者出闢雍。比年上、內舍人日增,而太學又有國子隨行親及小學生,人數已我,居處迫隘,乞以外舍生有無校定,並居闢雍,升補上、內舍乃入太學。」從之。八年,詔兩博士、正、錄並諸州教授兼用元豐試法,仍止試一經。吏部具到元豐法:進士第一甲,或省試十名內,或府、監發解五名內,或太學公、私試三名內,或季試兩次為第一人,或上舍、內舍生,或曾充經論以上職掌,或投所業乞試,並聽試,入上等注博士,中下等注正、錄,即人多闕少,原注諸州教授者聽。



 宣和三年,詔罷天下三舍,太學以三舍考選,開
 封府及諸路以科舉取士。州學未行三舍以前,應置宮及養士去處,依元豐舊制。太學生並撥填舊額,闢雍正額入太學者,撥入額外,依舊制遇填闕。諸內舍上等校定人願入太學者,與免補試。闢雍官屬並罷。又詔國子博士、正、錄改充太學正、錄。七年,臣僚言:「熙、豐間,博士未嘗除代,近年以來,席未暖而代者已至當從正、錄第進。新除太學博士胡世將、周利建乞改除正、錄,候將來升為博士。」從之。



 靖康元年,諫議大夫馮澥言:「朝廷罷元
 祐學術之禁,不專王氏之學,《六經》之旨,其說是者取之,今學校或主一偏之說,執一偏之見,願詔有司考校,敢私好惡去取,重行黜責。」又詔太學博士替成資闕。



 建炎三年,詔國子監並歸禮部。未幾,詔復養生徒,置博士。紹興十二年,置祭酒、司業各一人。十三年,太學成,增置博士、正、錄。參用元祐、紹聖監學法,修立監學新法。詔國子博士、正、錄通治諸齊。學官闕,從本監選舉。其後,監學博士、正、錄增減不齊,兼攝並置不一。至隆興以後,正、錄不
 兼權,祭酒、司業並置,復書庫官;又定國子博士一員,太學博士三員,正、錄共四員,學官之制始定。淳熙四年,置監門官一員,兼管石經閣,以不厘務使臣充,以後相承不改。



 武學慶歷三年,詔置武學於武成王廟,以阮逸為教授。八月,罷武學,以議者言「古名將如諸葛亮、羊祜、杜預等,豈專學系、吳」故也。熙寧五年,樞密院言:「古者出師受成於學,文武弛張,其道一也,乞復置武學。」詔於武成王廟置學。元豐官制行,改教授為博士,紹興十六年,詔
 修建武學,武博、武諭以兵書、弓馬、武藝誘誨學者。紹興二十六年,詔武學博士、學諭各置一員,內博士於文臣有出身或武舉出身曾預高選棄,其學諭差武學人,後又除文臣之有出身者。



 宗學元豐六年,宗室令鑠乞建宗學,詔從之,既而中輟,建中靖國元年復置。崇寧初,立月書、季考法。南渡初,建學。嘉定更新置四齊,後再增三齊。宗學博士,舊諸王宮大、小學教授也。至道元年,太宗將為皇侄等置師傅,執政謂環衛之官非新王比,當有
 降,乃以教授為名。咸平初,遂命諸王府官分兼南、北宅教授。南宮者,太祖、太宗諸王之子孫處之,所謂睦親宅也。崇寧五年,又改稱某王宮宗子博士,位國子博士之上。靖康之亂,宗學遂廢。紹興四年,始復置諸王宮大小學教授二員,隆興省其一。喜定九年十二月,始復置宗學,改教授為博士,又置宗學諭一員,並隸宗正寺,在太常博士之下,諭在國子正之上,奉給、賞典依國子博士及正例,於是宗室疏遠者皆得就學。旋有旨復存諸王
 宮大小學授一員。



 書庫官淳化五年,判國子監李志言:「國子監舊有印書錢物所,名為近俗,乞改為國子監書庫官。」始置書庫監官,以京朝官充。掌印經史群書,以備朝廷宣索賜予之用,及出鬻而收其直以上於官。元豐三年省。中興後,並國子監入禮部。紹興十三年,復置一員;三十一年,罷。隆興初,詔主簿兼書庫。乾道七年,復置一員。



 少府監舊制,判監事一人。以朝官充。凡進御器玩、后
 妃服飾、雕文錯彩工巧之事,分隸文思院、後苑造作所,本監但掌造門戟、神衣、旌節,郊廟諸壇祭玉、法物,鑄牌印諸記,百官拜表案、褥之事。凡祭祀,則供祭器、爵、瓚、照燭。



 元豐官制行,始置監、少監、丞、主簿各一人。監掌百工伎巧之政令,少監為之貳,丞參領之。凡乘輿服御、寶冊、符印、旌節、度量權衡之制,與夫祭祀、朝會展採備物,皆率其屬以供焉。庀其工徒,察其程課、作止勞逸及寒暑早晚之節,視將作匠法,物勒工名,以法式察其良窳。凡金
 玉、犀象、羽毛、齒革、膠漆、材竹,辨其名物而考其制度,事當損益,則審其可否,議定以聞。少府所掌,舊有主名,其工作之事,則監自親之。



 熙寧中,已厘歸有司,官制行,皆復舊。元豐元年,工部言:「文思院上下界諸作工料條格,該說不盡,功限例各寬剩,乞委官檢照前後料例功限,編為定式。」從之。又詔:「文思監官除內侍外,令工部、少府監同議選差。」崇寧三年,詔:「文思院兩界監官,立定文臣一員、武臣二員。並朝廷選差,其內侍乾當官並罷。」



 分案
 四,置吏八。所隸官屬五:文思院,掌造金銀、犀玉工巧之物,金採、繪素裝鈿之飾,以供輿輦、冊寶、法物凡器服之用。綾錦院,掌織絲任錦繡,以供乘輿凡服飾之用。染院,掌染絲枲幣帛。裁造院,掌裁制服飾。文繡院,掌纂繡,以供乘輿服御及賓客祭祀之用,崇寧三年置,招繡工三百人。



 舊置南郊祭器庫監官二人,太廟祭器法物庫監官二人,掌祠祭器服之名物,各有專典。旌節官二人,鑄印篆文官二人。諸州鑄錢監監官各一人。以上並
 屬少府監。



 將作監舊制,判監事一人,以朝官以上充。凡土木工匠之政、京都繕修隸三司修造案,本監但掌祠祀供省牲牌、鎮石、炷香、盥手、焚版幣之事。



 元豐官制行,始正職掌。置監、少監各一人,丞、主簿各二人。監掌宮室、城郭、橋梁、舟車營繕之事,少監為之貳,丞參領之,凡土木工匠板築造作之政令總焉。辨其才幹器物之所須,乘時儲積以待給用,庀其工徒而授以法式;寒暑蚤暮,均其
 勞逸作止之節。凡營造有計帳,則委官覆視,定其名數,驗實以給之。歲以二月治溝渠,通壅塞。乘輿行幸,則預戒有司潔除,均布黃道。凡出納籍帳,歲受而會之,上於工部。熙寧初,以嘉慶院為監,其官屬職事,稽用舊典,已而盡追復之。元祐七年,詔頒將作監修成《營造法式》八年,又詔本監營造橙計畢,長貳隨事給限,丞、簿覆檢。無符元年,三省言:「將作監主簿二員,乞將先到任一員改充乾當公事。候成資替罷。」從之。崇寧五年,詔將作監,應
 承受前後特旨應副外,路並府、監修造差撥人工物料,遵執無豐條格,不得應副。宣和五年,詔罷營繕所歸將監。



 分案五,置吏二十有七。所隸官屬十:修內司,掌宮城、太廟繕修之事。東西八作司,掌京城內外繕修之事。竹木務,掌修諸路水運材植及抽算諸河商販竹木,以給內外營造之用。事材場,掌計度材物,前期樸斫,以給內外營造之用。麥場,掌受京畿諸縣夏租,以給圬墁之用。窯務,掌陶為磚瓦,以給繕營
 及瓶缶之器。丹粉所,掌燒變丹粉,以供繪飾。作坊物料庫第三界,掌儲積材物,以備給用。通材場,掌受京城內外退棄材木,掄其長短有差,其曲直中度者以給營造,余備薪爨。簾箔場,掌抽算竹木、蒲葦,以供簾箔內外之用。



 建炎三年,詔將作監並歸工部。紹興三年,復置丞,仍兼總少府之事。十年,置主簿一員。十一年,詔依司農、太府寺,置長、貳一員。隆興初,宮室無所營繕,職務簡省,百工器用屬之文思院,以隸工部;本監惟置
 丞一員,餘官虛而不除。乾道以後,人材甚多,監、少、丞、簿無闕,凡臺省之久次與郡邑之有聲者,悉寄俓於此,自是號為儲才之地,而營繕之事,多俾府尹、畿漕分任其責焉。



 軍器監國初,戎器之職領於三司胃案,官無專職。熙寧六年,廢胃案,乃按唐令置監,以從官總判,元豐正名,始置監、少監各一人,丞二人,主簿一人。監掌監督繕治兵器什物,以給軍國之用,少監為之貳,丞參領之。凡
 利器以法式授工徒,其弓矢、干戈、甲冑、劍戟戰守之具,因其能而分任之,量用給材,旬會其數以考程課,而輸於武庫,委遣官詣所隸檢察。凡用膠漆、筋革、材物必以時,課百工造作,勞逸必均,歲終閱其良否多寡之數,以詔賞罰。器成則進呈便殿,俟閱試而頒其樣式於諸道。即要會州建都作院分造器械,從本監比較而進退其官吏焉,元祐三年,省丞一員,紹聖中復置。政和三年,應御前軍器監所頒降軍器樣制,非長、貳當職官不得省
 閱,及傳寫漏洩,論以違制。



 分案五,置吏十有三,所隸官屬四:東西作坊,掌造兵器、旗幟、戎帳、什物,辨其名色,謹其繕作,以輸於受藏之府,兵校工匠,其役有程,視精粗利鈍以為之賞罰。作坊物料庫,掌收鐵錫、羽箭、油漆之屬。皮角場,掌收皮革、筋角,以供作坊之用。南渡置御前軍器所,建炎三年,詔軍器監並歸工部,東西作坊、都作院並入軍器所。紹興三年,復置丞一員,令工部相度合管職事歸之。十一年,詔復置長、貳各一員。十
 四年,以朝奉大夫趙子厚守軍器監,宗室為寺監長、貳自此始。



 隆興初,詔置造軍器,已有軍器所隸工部,本監惟置丞一員。乾道五年,復置少監及簿,六年,以少監韓玉往建康點檢物馬,以奉使軍器少監為名。是年,復置監一員。淳熙初元,詔戎器非進入毋輒出所,由是呈驗浸省。二年,錢良臣以少監總領淮東財賦;八年,沉撥復以監長長。諸監長貳自是始許總餉外帶,然二人實初兼版曹職事。嘉定十四年,岳珂獨以軍器監總餉淮東。
 是後,戎所、作坊已備官於下,宥府、起部並提綱於上,監居其間,事務稀簡,特為儲才之所焉。



 都水監舊隸三司河渠案,嘉祐三年,始專置監以領之。判監事一人,以員外郎以上充,同判監事一人,以朝官以上充;丞二人,主簿一人,並以京朝官充。輪遣丞一人出外治河埽之事,或一歲再歲而罷,其有諳知水政,或至三年。置局於澶州,號曰外監。



 元豐正名,置使者一人,丞二人,主簿一人。使者掌中外川澤、河渠、津梁、堤堰
 疏鑿浚治之事,丞參領之。凡治水之法,以防止水,以溝蕩水,以澮寫水,以陂池瀦水。凡江、河、淮、海所經郡邑,皆頒其禁令。視汴、洛水勢漲涸增損而調節之。凡河防謹其法禁,歲計茭揵之數,前期儲積,以時頒用,各隨其所治地而任其責。興役以後月至十月止,民功則隨其先後毋過一月,若導水溉田及疏治壅積為民利者,定其賞罰。凡修堤岸、植榆柳,則視其勤惰多寡以為殿最。南、北外都水丞各一人,都提舉官八人,監埽官百三十有
 五人,皆分職蒞事;即乾機速,非外丞所能治,則使者行視河渠事。



 元豐八年,詔提舉汴河堤岸司隸本監。先是,導洛入汴,專置堤岸司。至是,亦歸之有司。元祐四年,復置外都水使者。五年,詔南、北外都水丞並以三年為任。七年,方議回河東流,乃詔河北、京西漕臣及開封府界提點,各兼南、北外都水事,紹聖元年罷。元符三年,詔罷北外都水丞,以河事委之漕臣;三年,復置。重和元年,工部尚書王詔言,乞選差曾任水官諳練者為南、北兩外
 丞,從之。宣和三年,詔罷南、北外都水丞司,依元豐法,通差文武官一員。



 分案七,置吏三十有七。所隸有:街道司,掌轄治道路人兵,若車駕行幸,則前期修治,有積水則疏導之。



 建炎三年,詔都水監置使者一員。紹興九年,復置南、北外都水丞各一員,南丞於應天府,北丞於東京置司。十年,詔都水事歸於工部,不復置官。



 司天監監少監丞主簿春官正夏官正中官正秋官正冬官正靈臺郎保章正
 挈壺正各一人。掌察天文祥異,鐘鼓漏刻,寫造歷書,供諸壇祀察告神名版位畫日。監及少監闕,則置判監事二人。以五官正充。



 禮生四人,歷生四人,掌測驗渾儀,同知算造三式。元豐官制行,罷司天監,立太史局,隸秘書省。



\end{pinyinscope}