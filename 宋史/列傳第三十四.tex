\article{列傳第三十四}

\begin{pinyinscope}

 劉福安守忠孔守正譚延美元達常思德尹繼倫薛超丁罕趙□附郭密傅思讓李斌附田仁朗劉謙



 劉福,徐州下邳人。少倜儻,魁岸有膂力。周顯德中,世宗
 征淮南,福徒步謁見於壽春。世宗奇之,因留麾下。每出戰,則令福率衛士為先鋒,與破紫金山砦。淮南平,錄功授懷德指揮使。



 宋初,遷橫海指揮使,率所部隸步帥劉光毅,由峽路征蜀。比至成都,孟昶已降。大將王全斌部送降卒歸京師,至綿州,降卒盜庫兵,劫蜀舊將全師雄以叛,焚廬舍,剽財貨以去。刺史成彥饒以同、華兵百餘人守其城,全斌遣米光緒將七百騎及福所部以屯護之。光緒盡殺師雄妻孥,師雄領叛卒,益聚村民十餘萬
 眾,攻城益急。會龍捷指揮使田紹斌率精銳百騎,由東山西北行,福領所部由山南行,出賊不意,夾擊之。賊眾大潰,斬首及溺江死者以萬計,以功授虎捷都虞候。繼隸曹彬麾下,平江南。還,授指揮使,領蔚州刺史。從太宗克並、汾,遷馬步都軍頭、武州團練使。端拱初,出為洺州防禦使。二年,改雄州防禦使兼本州兵馬部署。雄州地控邊塞,常屯重兵。福至部,按行城壘,調鎮兵以給繕完,出私錢以資宴犒,寇雖大至,而恃以無恐矣。淳化初,遷
 涼州觀察使、判雄州事。二年,卒,年六十四。贈太傅。



 福雖不學,而御下有方略,為政簡易,人甚德之。領雄州五年,郡境寧謐。福既貴,諸子嘗勸起大第,福怒曰:「我受祿厚,足以僦舍以庇。汝曹既無尺寸功以報朝廷,豈可營度居室,為自安計乎?」卒不許。既死,上聞其言,賜其子白金五千兩,使市第宅。



 安守忠,字信臣,並州晉陽人。父審琦,為周平盧軍節度,封陳王。晉天福八年,審琦出領山南東道,以守忠為牙
 內指揮使,領繡州刺史。周顯德四年春,改鞍轡庫使。會淮南初下,命守忠馳往宣諭。時藩臣驕蹇,遇朝使多簡傲,守忠抗以正禮,無所辱命。未幾,改衛州刺史。



 宋初,入為左衛將軍。建隆四年,湖南初平,命為永州刺史。乾德中,護河陰屯兵。蜀平,太祖知遠俗苦苛虐,南鄭為走集之地,故特命守忠知興元府以撫綏之。四年,改漢州刺史。時寇難甫平,使車旁午,公帑不足,守忠出私錢以給用。每遣使,太祖必戒之曰:「安守忠在蜀,能律已以正,汝
 行見之,當效其為人也。」開寶初,改濮州刺史。會河決澶州,命守忠副穎州團練使曹翰護役,河決遂塞。五年,知遼州。民有陰召並寇謀內應者,事洩,守忠悉斬以徇。九年,命將征太原,守忠受詔與孫晏宣由遼州入,既而與路羅砦監押馬繼恩遇,乃相與會兵入賊境,燔砦四十餘,獲牛羊數千。議將深入,會上崩,乃班師。



 太平興國初,移知靈州,在官凡七年。雍熙二年,改知易州,徙夏州。每西戎犯邊,戰無不捷,錄功就拜濮州團練使。端拱中,知
 滄州,改瀛州,兼高陽關駐泊部署,遷瀛州防禦使。初,守忠嘗夢一「濮」字方丈餘,及領是郡幾二十年,於是始悟。淳化二年,徙知雄州。方與僚佐宴飲,有軍校謀變,擐甲及閽,閽者倉卒入白。守忠言笑自若,徐顧坐客曰:「此輩酒狂爾,擒之可也。」人服其量焉。明年,加耀州觀察使,兼判雄州。未幾召還,條陳邊事,敷奏稱旨,賜錢五百萬。五年,又知滄州。至道初,移雄州。三年,復知滄州。拜感德軍節度觀察留後,徙宋州,兼制置營田使。威德兼著,吏民
 不忍其去。咸平三年入覲,遣還未行,暴卒,年六十九,贈太尉。錄其子繼昌為供備庫副使,婿王世及為光祿寺丞。



 守忠謹愨淡薄,為治簡靜。太祖居藩日,素相厚善,及受禪後,每優任之,守忠處之益謙。從征太原,多與謀略,人罕知之者。所至藩郡,樂施予,豐宴犒,且喜與士大夫游從,故時論多與之。初,審琦以愛妾故,為隸人所戕。守忠終身不畜妓妾,而喜佞佛,蓋有所懲云。



 孔守正,開封浚儀人。幼事後唐明宗子許王從益。漢初,
 為東西班承旨,事魏王承訓。周世宗征淮南,以材勇選為東班承旨。



 宋初,補內殿直,兼領驍雄、吐渾指揮。從劉廷翰平蜀,還,遷驍雄副指揮使。開寶中,太祖征太原,守正隸何繼筠麾下。會契丹遣兵來援晉陽,守正接戰於石嶺關,大敗之,斬首萬級,獲其將王破得。時宋師之陷敵者數百人,守正以騎軍馳之,盡奪以還。



 太平興國中,累遷日騎東西班指揮使。太宗親征晉陽,守正分主城西洞屋,領步卒大呼先登,繼與內侍蔡守恩等率騎兵
 力戰,晉軍遂潰。從征範陽,至金臺驛,詔與劉仁蘊先趨岐溝關。時城未下,守正夜超垣,度鹿角,臨機橋,以大軍將至,說關使劉禹使降。禹解懸橋,守正遂入城,撫諭其軍民,以城守屬綦廷朗,而已赴行在。時契丹兵在涿州東,守正與傅潛率御前東西班分兩陣馳擊之,逐北二十餘里,降其羽林兵數百人。繼與高懷德、劉廷翰合兵追之至桑乾河,契丹自是不敢近塞。以勞再遷日騎都指揮使,領濡州刺史。



 端拱初,遷龍衛都指揮使,領長州
 團練使,出鎮真定。是年秋,出為穎州防禦使。未幾,太宗以其練習戎旅,特置龍衛、神衛四廂都指揮使以授之。改領振州防禦使。明年,拜殿前都虞候,領容州觀察使。一日,侍宴北苑,上入玄武門,守正大醉,與王榮論邊功於駕前,忿爭失儀,侍臣請以屬吏,上弗許。翌日,俱詣殿廷請罪,上曰:「朕亦大醉,漫不復省。」遂釋不問。俄命為定州行營副部署,受詔詣保州軍開道,遇敵於曹河,與戰數合,梟首三十餘,獲馬五十匹,上聞而壯之。



 淳化初,擢
 高陽關副都部署。軍中小將有詈其校長者,守正械送闕下,取裁於上,未嘗專決焉。明年,護浚惠民河,塞澶州決河,就命知州軍。改慎州觀察使,還,領代州部署,連移並代、夏綏、麟府三鎮。與李繼遷戰大橫岡,援範廷召出塞,破賊於白池,至行莊,焚掠甚眾,改代、夏二州部署。



 真宗即位,復徙代州。咸平初,授昌化軍節度觀察留後。守正上言:「四任雁門,邊亭久安,願徙東北以自效。」會夏人入寇,改定州行營副都部署。四年,移彰德軍留後,以風
 疾妨政,改安化軍留後。景德初,復以不任職代。時議防秋北鄙,守正猶屢表請行。上閔之,不許。無何,卒,年六十六,贈泰寧軍節度使。



 譚延美,大名朝城人。軀幹壯偉。少不逞,遇群盜聚謀將行剽劫,延美即趨就之。及就捕,法皆抵死,延美以與盜素不相識,獲免。自後往來澶、魏間,為盜於鄉里,鄉里患之。周世宗鎮澶淵,募置帳下。即位,補殿前散都頭。從征淮南,以勞遷控鶴軍副指揮使。又從克三關。時太祖領
 禁兵,留督牙隊。



 建隆元年,補控鶴指揮使,稍遷都虞候、馬步副都軍頭。征湖南,與解暉分領行營戰棹都指揮使。時汪端寇攻朗州甚急,招討慕容延釗遣延美率兵赴之,大敗賊眾,擒端以還。擢鐵騎副指揮使,領睦州刺史,四遷至內殿直都知。



 太平興國初,為蘄州刺史,連徙廬、壽、濠、光州軍巡檢使,劇賊之為害者悉就捕。六年,徙知威虜軍。雍熙三年,舉兵北伐,命延美為幽州西面行營都監,與田重進出飛狐北。俄遇敵,延美曰:「彼恃眾易
 我,宜出其不意先攻之。」即麾騎軍直進,敵兵將潰,大軍繼至,遂敗之,斬首五百,獲其將大鵬翼以獻,以功擢本州防禦使。逾年,改亳州,出為鎮州鈐轄。



 端拱元年,徙知寧遠軍。一旦,契丹兵抵城下,延美開門以示之,不敢入。圍城數日,開門如故,民出取芻糧者無異平日,契丹卒疑之,遂引去。二年,進邕州觀察使、判亳州,兼知代州。是時任邊郡者,皆令兼領內地一州,處其家屬。徙知潞、陜、涇州。咸平四年,以左領軍衛上將軍致仕。六年,卒,年八
 十三,贈建武軍節度。子繼倫,至崇儀副使;雍,虞部員外郎。



 元達,初名守旻,洺州雞澤人。身長八尺餘,負膂力,善射。家業農,不任作苦,委耒耜,慨嘆而去之。事任俠,縱酒。嘗醉,見道旁槐樹,拔劍斬之,樹立斷。達私喜曰:「吾聞李將軍射石虎飲羽,今樹為我斷,豈神助歟?」嘗從少年數十百人欲起為盜,里中父老交戒之,乃止。時郡以戶籍調役,達當送徒闕下,行數舍,乃悉縱之,曰:「吾觀汝曹,亦丈
 夫也,豈樂為是哉?可善自為計,吾亦從此逝矣!」已而郡遣追捕,至則達援弓引滿待之,追者不敢近。由是亡命山林間,為鄉里患。



 太宗居晉邸時,達求見,得隸帳下。嘗侍太宗習射園亭,命之射,達射四發不中的,已而連中。上喜,為更其名曰達。及即位,補御龍直隊長。雍熙初,累遷媯州刺史,繼領本州團練使。進州郡部送亡命者至闕,左右諷殺之,達奏曰:「此類竄匿者眾,豈能盡殺之哉?不如赦之,以開其自新之路,且以成好生之德。」上悅,因
 悉原之。端拱二年,擢侍衛步軍都虞候,領幽州刺史。歷北面行營都部署,由常山鎮入為京城巡檢。淳化四年卒,年四十二,贈昭化軍節度。



 達雖奮自草野,歷職戎署,至交士夫,能折節盡禮,人以是稱之。



 常思德,開封人。周顯德初,以材勇應募,隸天武軍,累遷神衛都虞候。雍熙初,從曹彬徵幽州,因署牙校。尋鎮威虜軍。端拱初,以弓箭直都虞候領溪州刺史。淳化中,李順叛蜀,命往夔、峽招捕,師次達州新寧縣,調近州土兵
 掩殺賊徒三千餘人於梁山。時雷有終領大軍抵合州境上,賊眾二萬來拒。思德與尹元、裴莊等合擊之,合州遂平。賊帥田奉正、蘇榮據果州,思德因其遁而追捕之,斬首八百。果州既定,餘賊保渠州,及走廣安、梁山。乃分兵為二:抵廣安、梁山者,思德領之;趣渠州者,元、莊領之。合力進討,盡殲其黨。自是川、峽賴以安靜,無復寇患,以功真授汝州刺史。



 初,曹彬北征不利,至涿州,左右皆潰散,獨思德以所部護至易州。語人曰:「既備戎行,則與主
 帥同死生可也。若視利害以為去就,將何面目以見君父乎?」太宗嘗聞其言,至是陛辭,深加慰勞,且諭之曰:「為臣以忠實為本,汝少壯時,既以驍勇自效,且能盡心於主將,事朕之日雖久,而忠實如一。今雖老,亦當盡心乃職,庶無負乎朕之委寄也。」



 未幾,移慶州路副都部署、屯邠州。咸平初,與李繼隆同部芻糧赴靈州。以疾改陳留都監,換左神武大將軍。二年,卒,年六十五。



 尹繼倫,開封浚儀人。父勛,郢州防禦使。嘗內舉繼倫以
 為可用,太祖以補殿直,權領虎捷指揮,預平嶺表,下金陵。太宗即位,改供奉官。從征太原,還,遷洛苑使,充北面緣邊都巡檢使。



 端拱中,威虜軍糧饋不繼,契丹潛議入寇。上聞,遣李繼隆發鎮、定兵萬餘,護送輜重數千乘。契丹將於越諜知之,率精銳數萬騎,將邀於路。繼倫適領兵巡徼,路與寇直。於越徑趨大軍,過繼倫軍,不顧而去。繼倫謂其麾下曰:「寇蔑視我爾。彼南出而捷,還則乘勝驅我而北,不捷亦且洩怒於我,將無遺類矣。為今日計,
 但當卷甲銜枚以躡之。彼銳氣前趣,不虞我之至,力戰而勝,足以自樹。縱死猶不失為忠義,豈可泯然而死,為胡地鬼乎!」眾皆憤激從命。繼倫令軍中秣馬,俟夜,人持短兵,潛躡其後。行數十里,至唐河、徐河間。天未明,越去大軍四五里,會食訖將戰,繼隆方陣於前以待,繼倫從後急擊,殺其將皮室一人。皮室者,契丹相也。皮室既擒,眾遂驚潰。於越方食,失箸,為短兵中其臂,創甚,乘善馬先遁。寇兵隨之大潰,相蹂踐死者無數,餘黨悉引去。契
 丹自是不敢窺邊,其平居相戒,則曰:當避「黑面大王」,以繼倫面黑故也。以功領長州刺史,仍兼巡檢。



 淳化初,著作佐郎孫崇諫自契丹逃歸,太宗詢以邊事,極言徐河之戰契丹為之奪氣,故每聞繼倫名,則倉皇不知所措。於是遷繼倫尚食使,領長州團練使,以勵邊將。淳化五年,李繼隆奉詔討夏州,以繼倫為河西兵馬都監。未幾,以深州團練使領本州駐泊兵馬部署。



 至道二年,分遣將帥為五道,以討李繼遷。時大將李繼隆由靈環路往,
 逗撓不進。上怒,急召繼倫至京師,授靈、慶兵馬副都部署,欲以夾輔繼隆也。時繼倫已被病,強起受詔。上素聞其嗜酒,以上尊酒賜而遣之。即日乘驛赴行營,至慶州卒,年五十。上聞之嗟悼,賻賵加等,遣中使護其喪而歸葬焉。



 薛超,遼州平城人。少有勇力。乾德初,應募為虎捷卒。從崔彥進伐蜀平,錄功補虞候,遷十將。太平興國初,四遷至天武指揮使。從征太原,領游騎千人備御鎮、定境上,
 以張軍勢。及車駕還,契丹頻寇鎮、定,侵掠無已。超從大將劉廷翰率兵至徐河,賊將領騎十餘出挑戰,超躍馬直前,連射數人斃,敵勢遂卻。大軍乘之奮擊,斬首萬餘級。以功加步軍都軍頭,遷神衛軍都校,領敘州刺史。



 雍熙三年,從潘美北征,至雁門、西陘,路與契丹遇,又戰敗之。追至寰州,斬首五百餘級,其將趙彥辛以城降。超連被創,流血濡甲縷,部分軍士自若,乘勝抵應州,其節度副使艾正以城降。還,加馬步軍都軍頭。淳化初,屯鎮州,
 遷天武指揮使,領澄州團練使。至道元年卒,年五十七。



 丁罕者,穎州人。應募補衛士,累遷指揮使。從劉廷翰戰徐河,以奪橋功,遷本軍都虞候。累遷天武指揮使,領獎州團練使。淳化三年,出為澤州團練使、知霸州。會河溢壞城壘,罕以私錢募築,民咸德之。五年,以容州觀察使領靈環路行營都部署,與李繼遷戰,斬首俘獲以數萬計。至道中,率兵從大將李繼隆出青岡峽,賊聞先遁,追十日程,不見而返。三年,真拜密州觀察使、知威虜軍,徙
 貝州。咸平二年,卒。子守德,能世其家。



 趙□者,貝州清河人。由衛士累遷龍衛指揮使。亦以徐河戰功,加鎮州團練使,至兵馬部署。至道二年卒於官,年七十。贈歸義軍節度使。



 郭密,貝州經城人。軀幹雄偉,膂力絕人。幼孤,隨母適同郡王乙,因冒姓王氏。以知瀛州馬仁瑀薦,隸晉王帳下,給事左右。太宗即位,補指揮使,復姓郭氏。至淳化間,凡八遷,移貝州駐泊兵馬部署。會夏人寇邊,以密有武略,
 擢領安州觀察使,充靈州兵馬都部署。訓練土卒,號令嚴肅,夏人畏服,邊境賴以寧謐。至道二年卒,年五十八。贈保順軍節度。



 傅思讓者,冀州信都人。少無賴,有勇力,善騎射。太宗居晉邸,補親事都校。即位,補衛士直長,累遷至平州刺史。奉詔破契丹兵於唐興口。端拱中,四遷為容州觀察使、知莫州,移隴州。上命殿中丞林特同判州事,以夾輔之,以思讓所為多不法故也。至道二年卒,年七十四。贈保
 順軍節度。



 李斌者,青州人。太宗在晉邸,聞其狀貌魁偉,召置左右。即位,補御龍直副指揮使。太平興國中,以天武指揮使領鄭州刺史。七年,坐嘗受秦王廷美饋遺,貶曹州都校。雍熙三年,遷營州刺史。四年,領溪州團練使,連為貝、冀二州駐泊都監。淳化中,繼領萊州、洺州團練使。勤於政理,人服其清慎,轉運使陳緯以狀聞於朝。至道初,拜桂州觀察使,仍判洺州,徙滄州。及代,吏民不忍其去,鄰境
 亦上其善狀,詔書褒美之。咸平三年卒,年六十一。



 田仁朗,大名元城人。父武,仕晉昭義軍節度使。仁朗以父任西頭供奉官。太祖即位,從討李重進,攻城有功,還,與右神武統軍陳承昭浚五丈河,以通漕運。



 乾德中,討蜀,命仁朗為鳳州路壕砦都監。伐木除道,大軍以濟,錄功遷染院副使。太祖征太原,與陳承昭壅汾水灌城。城將陷,會班師。俄遷內染院使,數日,改左藏庫使。為中官所讒,太祖怒,立召詰之,至殿門,命去冠帶。仁朗神色不
 撓,從容曰:「臣嘗從破蜀,秋毫無犯,陛下固知之。今主藏禁中,豈復為奸利以自污?」太祖怒釋,止停其職。



 開寶六年,起為榷易使。七年,以西北邊內侵,選知慶州。仁郎至,率麾下往擊之,短兵將接,前鋒稍卻,仁朗斬指揮使二人,軍中震恐,爭乞效命,遂大破之。其酋長相率請和,仁朗烹牛置酒與之約誓,邊境乃寧,璽書褒美。



 太平興國初,秦州羌為寇,命仁朗屯兵清水。會李飛雄事敗,召為西上閣門使。四年,徵太原,命仁朗與閣門祗候劉緒按
 行太原城四面壕砦,閱視攻城梯沖、器械。太原平,留仁朗為兵馬鈐轄,閑廄使武再興、軍器庫副使賈湜並為巡檢。俄命仁朗與再興役民築榆次新城。從幸大名,又命為滄州鈐轄,俄遷東上閣門使、知秦州。九年,判四方館事。會議東封,命仁朗自京抵泰山,督役治道。



 李繼遷為亂,命仁朗率兵巡銀、夏,歲餘召還。未幾,繼遷攻麟州,誘殺曹光實,遂圍三族砦。命仁朗與閣門使王侁、副使董願、宮苑使李繼隆,馳傳發邊兵數千擊之。仁朗次綏
 州,奏請益兵,留月餘俟報。會三族砦將折遇乜殺監軍使者,與繼遷合。太宗聞之大怒,亟遣軍器庫使劉文裕自三交乘疾置代仁朗。繼遷乘急攻撫寧砦,仁朗不知為文裕所代,喜謂諸將曰:「敵人逐水草散保巖險,常烏合為寇,勝則進,敗則走,無以窮其巢穴。今繼遷嘯聚羌、戎數萬,盡銳以攻孤壘,撫寧小而固,兵少而精,未可以旬浹破。當留信宿,俟其困,以大兵臨之,分強弩三百,邀其歸路,必成擒矣。」仁朗部署已定,欲示閑暇,日縱其樗
 博,不恤軍事。上知之,遣使召仁朗赴闕,下御史按問仁朗請益兵及陷三族狀。仁朗對曰:「所召銀、綏、夏兵,其州皆留防城,不遣。所部有千餘人,皆曹光實舊卒,器甲不完,故請益兵。況轉輸芻粟未備,三族砦與綏相去道遠,非元詔所救。昨臣已定擒繼遷策,會詔代臣,其謀不果。」因言:「繼遷得部落情,願降優詔懷來之,或以厚利啖諸酋長密圖之。不爾,恐他日難制,大為邊患。」御史以其狀聞,上大怒,切責憲府官吏曰:「仁朗不恤軍政,得為過乎?」
 大理遂當仁朗乏軍興及征人違期二十日以上,坐死,上特貸之,下詔責授商州團練副使,馳驛發遣。



 是役也,仁朗計已決,為王侁等所構,逗撓不進軍,故及於貶。後數月,上知其無罪,召拜右神武軍大將軍。部修河北東路諸州城池,數月而就。留知雄州,加領澄州刺史。時河北用兵,大藩多用節將,朝議以通判權位不倫,選諸司使有吏乾者佐之,以仁朗知定州節度副使事。俄召赴闕,未聞命而卒,年六十,時端拱二年也。



 仁朗性沉厚,有
 謀略。頗涉書傳,所至有善政。雅好音律,尤臻其妙。時內職中咸以仁朗為稱首,故死之日人多惜之。



 劉謙,博州堂邑人。曾祖直,以純厚聞於鄉黨,里有盜其衣者,置不問。州將廉知,俾人故竊其衣,亦不訴理,即召詰前盜衣者,俾還之。直紿云:「衣乃自以遺少年,非竊也。」州將義之,賜以金帛,不受而去。父仁罕,輕俠自任。五代末,寇盜充斥,仁罕率眾斷澶州浮橋以潰賊,因誘獲數十人,出芻粟給官軍。補內黃鎮將。嘗因事至酒家,遇群
 寇暴集,以計悉梟其首,攜詣西京留守向拱,補汜水鎮將,俄為散都頭。宋初,遷許州龍衛副指揮使。會王師征廣南,為前鋒。還,改同州都校,卒。



 謙少感概,不拘小節。初詣嶺表省父,仁罕資以金帛,令北歸行商。還堂邑舊墅,嘗為鄉里惡少所辱,謙不勝怒,毆殺之。亡命京師,遂應募從軍,補衛士,稍遷內殿直都知。至道初,真宗升儲邸,增補宮衛,太宗御便坐,親選諸校,授謙西頭供奉官、東宮親衛都知,賜袍笏、靴帶、器幣。真宗即位,擢授洛苑使。
 謙起行伍,不樂禁職,求換秩,改殿前左班指揮使,給諸司使奉料。咸平初,遷御前忠佐馬步軍都軍頭,領勤州刺史,加殿前右班都虞候。上幸大名,至北苑,屬謙有疾,遣歸將護,謙懇請從行。既俾其二子隨侍,仍挾尚醫以從,御廚調膳以給之。疾瘳,毀所服鞍勒以遺中使,上聞,賜白金二百兩。駕還,改捧日左廂都指揮使,領本州團練使。四年,遷捧日、天武四廂都指揮使,領本州防禦使,權殿前都虞候。



 時高翰為天武左廂都校,有卒負債殺
 人,瘞尸翰營中,累日,發土得之。上怒翰失檢察,執見於便殿。謙即前奏:「翰職在巡邏及閱教諸軍,不時在營,本營事宜責之軍頭。」上為釋翰罪。



 景德初,加侍衛馬軍都虞候,改領潯州防禦使,俄權步軍都指揮使。明年冬,制授殿前副都指揮使、振武軍節度。先是,謙久權殿前都虞候,俄擢曹璨正授,謙頗形慨嘆。至是,璨副馬軍,而升謙領禁衛焉。河北屯兵,常以八月給冬衣。謙上言邊城早寒,請給以六月,後以為例。無何,以足疾求典郡,上召
 見,敦勉之。



 大中祥符初,從東封,上升泰山,詔都總山下馬步諸軍,與西京左藏庫副使趙守倫閱視山門,設施有法,著籍者乃得上焉。禮成,進授都指揮使,移領保靜軍節度。明年八月卒,年六十,贈侍中。初,謙將應募,與同軍王仁德訊於日者。日者指謙謂仁德曰:「爾當為此人廄吏。」及謙帥殿前,仁德果隸役廄中。



 子懷懿,後為東染院副使。懷詮,內殿崇班、閣門祗候。



 論曰:宋初諸將,率奮自草野,出身戎行,雖盜賊無賴,亦
 廁其間,與屠狗販繒者何以異哉?及見於用,皆能卓卓自樹,由御之得其道也。劉福御下有方略,所至著績,受祿雖厚,而不為燕安之謀,可謂國爾忘家者矣。守忠練達邊事,示是身謙慎,弭卒校之變於談笑之頃,非善於行權者不能也。仁朗沉毅有謀,累從征討,綏州之役,不惟無功,而反坐逗撓,豈其計之不善哉?特為讒邪所構爾。自餘諸子,皆積戰功以取通侯。若延美之開門示敵,思德之翼衛主帥,繼倫之襲擊契丹,薛超之裹創赴戰,元
 達之請赦亡命,郭密之訓撫士卒,斯皆忠義仁勇,有足稱者。罕、□、思讓,若斌、若謙,雖乏奇功,而亦克共乃職,能寡過者也。守正素練戎旅,累任邊要,而矜勞肆忿,視於勞謙之君子,能無愧乎



\end{pinyinscope}