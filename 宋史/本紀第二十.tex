\article{本紀第二十}

\begin{pinyinscope}

 徽宗二



 四年春正月庚午朔,改熙河蘭會路為熙河蘭湟路。丙戌,築溪哥城。壬辰,詔察諸路監司貪虐者論其罪。丙申,詔京畿路改置轉運使、提點刑獄官。蔡卞罷。立武學法。
 丁酉,秦鳳蕃落獻邦、潘、疊三州。以內侍童貫為熙河蘭湟、秦鳳路經略安撫制置使。



 二月乙巳,築御謀城。己酉,置親衛、勛衛、翊衛郎、中郎等官,以勛戚近臣之兄弟子孫有官者試充。甲寅,以張康國知樞密院事,兵部尚書劉逵同知樞密院事,吏部尚書何執中為尚書左丞。乙卯,班方田法。庚申,詔西邊用兵能招納羌人者,與斬級同賞。壬戌,升趙州為慶源軍。甲子,雨雹。乙丑,改三衛郎為侍郎。



 閏月壬申,復元豐銓試斷按法。令州縣仿尚
 書六曹分六案。甲申,置陜西、河東、河北、京西監,鑄當二夾錫鐵錢。己丑,御端門,受趙懷德降,授感德軍節度使,封安化郡王。壬辰,曲赦熙河蘭湟路。



 三月壬寅,置青海馬監。甲辰,以趙挺之為尚書右僕射兼中書侍郎。丙午,詔建王口砦為懷遠軍。庚戌,令呂惠卿致仕。戊午,復銀州。乙丑,詔州縣屬鄉聚徒教授者,非經書、子、史毋習。丁卯,牂牁、夜郎首領以地降。是月,夏人攻塞門砦。



 夏四月辛未,遼遣蕭良來,為夏人求還侵地及退兵。戊寅,夏人
 攻臨宗砦。辛巳,詔諸路走馬承受毋得預軍政及邊事。己丑,夏人寇順寧砦,鄜延第二副將劉延慶擊破之;復攻湟州北蕃市城,知州辛叔獻等擊卻之。



 五月戊申,除黨人父兄子弟之禁。壬子,遣林攄報聘於遼。賜張繼先號虛靜先生。癸丑,罷轉運司檢察鉤考法。辛酉,命官分部決獄。



 六月丙子,復解池鹽。占城入貢。丁丑,慮囚。辛巳,罷陜西、河東力役。甲申,曲赦熙河、陜西、河東、京西路。戊子,趙挺之罷。



 秋七月丙申朔,罷三京國子監官,各置司
 業一員。辛丑,置熒惑壇。置四輔郡,以穎昌府為南輔,襄邑縣為東輔,鄭州為西輔,澶州為北輔。甲寅,詔奪元祐輔臣墳寺。丁巳,還上書流人。戶部尚書曾孝廣坐錢帛皆闕,出知杭州。



 八月戊辰,以德妃王氏為淑妃。庚午,以王、江、古州歸順,置提舉溪洞官二員,改懷遠軍為平州。丙子,以東輔為拱州。甲申,奠九鼎於九成宮。乙酉,詣宮酌獻。辛卯,賜新樂名《大晟》,置府建官。壬辰,遣劉正夫使遼。



 九月己亥,赦天下。乙巳,詔元祐人貶謫者以次徙近
 地,惟不得至畿輔。詔京畿、三路保甲並於農隙時教閱。乙卯,賜上舍生三十五人及第。丙辰,詔自今非宰臣毋得除特進。



 冬十月,自七月雨,至是月不止。甲申,以左、右司所編紹聖、元符以來申明斷例班天下,刑名例班刑部、大理司。丁亥,升武岡縣為軍。戊子,詔上書進士未獲者,限百日自陳免罪。壬辰,日中有黑子。



 十一月戊戌,安定郡王世雍薨。丙辰,置諸路提舉學事官。己未,章惇卒。十二月癸酉,升拱州為保慶軍。甲申,分平州置允州、格
 州。是歲,蘇、湖、秀三州水,賜乏食者粟。泰州禾生□魯。



 五年春正月戊戌,彗出西方,其長竟天。庚子,復置江、湖、淮、浙常平都倉。甲辰,以吳居厚為門下侍郎,劉逵為中書侍郎。乙巳,以星變,避殿損膳,詔求直言闕失。毀《元祐黨人碑》。復謫者仕籍,自今言者勿復彈糾。丁未,太白晝見,赦天下,除黨人一切之禁。權罷方田。戊申,詔侍從官奏封事。己酉,罷諸州歲貢供奉物。庚戌,詔崇寧以來左降者,各以存歿稍復其官,盡還諸徙者。辛亥,御殿復膳。
 壬子,罷圜土法。丁巳,罷書、畫、算、醫四學。壬戌,復書、畫、算學。



 二月甲子朔,詔監司條奏民間疾苦。丙寅,蔡京罷為開府儀同三司、中太一宮使。以觀文殿大學士趙挺之為特進、尚書右僕射兼中書侍郎。庚午,詔翰林學士、兩省官及館閣自今並除進士出身人。壬申,省內外冗官,罷醫官兼宮觀者。蒲甘國入貢。丁丑,以前後所降御筆手詔模印成冊,班之中外。州縣不遵奉者,監司按劾,監司推行不盡者,諸司互察之。



 三月丙申,詔星變已消,罷
 求直言。辛丑,改威德軍為石堡砦。封眉州防禦使世福為安定郡王。癸卯,御集英殿策進士。丁未,罷諸州武學。乙卯,廢銀州為銀川城。丙辰,蔡王似薨。己未,賜禮部奏名進士及第、出身六百七十一人。



 夏四月丁丑,停免兩浙水災州郡夏稅。



 五月丁未,班《紀元歷》。辛亥,封子栩為魯國公。乙卯,罷闢舉,盡復元豐選法。



 六月癸亥,立諸路監司互察法,庇匿不舉者罪之,仍令御史臺糾劾。改格州為從州。甲子,詔求隱逸之士,令監司審核保奏,其緣
 私者,御史察之。丁卯,詔輔臣條具東南守備策。壬申,慮囚。



 秋七月庚寅朔,日當食不虧。壬寅,詔改明年元。



 九月辛丑,河南府嘉禾與芝草同本生。



 冬十月己卯,升澶州為開德府。庚辰,降德音於開德府:減囚罪一等,徒以下釋之。



 十一月辛卯,陳王佖薨。乙巳,詔立武士貢法。辛亥,並京畿提刑入轉運司。十二月戊午朔,日當食不虧,群臣稱賀。己未,劉逵罷。壬戌,詔臣僚休日請對,特御便殿。己巳,詔監司按事,有懷奸挾情不盡實者,流竄不敘。是
 歲,廣西黎洞韋晏鬧等內附。



 大觀元年春正月戊子朔,赦天下。甲午,以蔡京為尚書左僕射兼門下侍郎。戊戌,幸興德禪院。復廢官。庚子,復置議禮局於尚書省。甘露降於帝鼎內,群臣稱賀。壬寅,吳居厚罷。戊申,進封衛王俁為魏王,定王偲為鄧王。壬子,以何執中為中書侍郎,鄧洵武為尚書左丞,戶部尚書梁子美為尚書右丞。乙卯,封仲損為南康郡王,仲御為汝南郡王。



 二月壬戌,以向宗回為開府儀同三司,徙封
 安康郡王。甲子,以黎洞納土,曲赦廣西。乙亥,復醫學。己卯,復行方田。丙戌,以平昌郡君韋氏為才人。



 三月丁酉,趙挺之罷。以何執中為門下侍郎,鄧洵武為中書侍郎,梁子美為尚書左丞,吏部尚書朱諤為右丞。甲辰,立八行取士科。癸丑,趙挺之卒。



 夏四月乙丑,以淑妃王氏為貴妃。



 五月己丑,封子棫為楊國公。朝散郎吳儲、承議郎吳侔坐與妖人張懷素謀反,伏誅。貶呂惠卿為祁州團練副使。庚寅,鄧洵武罷。甲午,詔班新樂於天下。癸卯,詔
 自今凡總一路及監司之任,勿以元祐學術及異意人充選。以安化蠻犯邊,益兵赴廣西討之。乙巳,子構生。



 六月己未,以梁子美為中書侍郎。壬戌,詔景靈宮建僖祖殿室。甲子,以黎人地為庭、孚二州。癸酉,賜上舍生二十九人及第。乙亥,朱諤卒。丁丑,慮囚。甲申,以才人韋氏為婕妤。



 秋七月乙酉朔,伊、洛溢。戊子,詔括天下漏丁。壬寅,班祭服於州郡。乙巳,賢妃武氏薨。



 八月乙卯,曾布卒。丁巳,封子構為蜀國公。庚申,以戶部尚書徐處仁為尚書
 右丞,吏部尚書林攄同知樞密院事。己巳,降德音於淮、海、吳、楚二十六州:減囚罪一等,流以下釋之。



 九月庚寅,建顯烈觀於陳橋。己酉,加上僖祖謚曰立道肇基積德起功懿文憲武睿和至孝皇帝,朝獻景靈宮。庚戌,饗太廟。辛亥,大饗明堂,赦天下。升永興軍為大都督府。章綖坐冒法,竄海島。李景直等四人以上書觀望罪,並編管嶺南。



 冬十月己未,詔士有才武絕倫者,歲貢準文士上舍上等法。辛酉,蘇州地震。乙丑,貶張商英為安化軍節度
 副使。己巳,大雨雹。



 閏月丙戌,以林攄為尚書左丞,資政殿學士鄭居中同知樞密院事。乙未,詔守令以戶口為殿最。升桂州為大都督府,鎮州為靖海軍節度。壬寅,禁用翡翠。乙巳,升太原府、鄆州並為大都督府。



 十一月壬子朔,日有食之,蔡京等以不及所當食分,率群臣稱賀。乙丑,置符寶郎。己巳,升瀛州為河間府、瀛海軍節度。戊寅,南丹州刺史莫公佞降。徐處仁以母憂去位。十二月庚寅,以蔡京為太尉,進何執中以下官二等。癸巳,以江
 寧、荊南、揚、杭、越、洪、福、潭、廣、桂並為帥府。置黔南路。丁酉,置開封府府學。己亥,以婉容喬氏為賢妃。開潩河。是歲,秦鳳旱。京東水,河溢,遣官振濟,貸被水戶租。廬州雨豆。汀、懷二州慶雲見。乾寧軍、同州黃河清。於闐、夏國入貢。涪州夷駱世葉、駱文貴內附。



 二年春正月壬子朔,受八寶於大慶殿,赦天下,文武進位一等。蔡京表賀符瑞。乙卯,以婉儀劉氏為德妃。己未,蔡京進太師;加童貫節度使,仍宣撫。庚申,進封魏王俁
 為燕王,鄧王偲為越王,並為太尉;京兆郡王桓為定王,高密郡王楷為嘉王,並為司空;吳國公樞為建安郡王,冀國公杞為文安郡王,楚國公栩為安康郡王,楊國公棫為濟陽郡王,蜀國公構為廣平郡王,並為開府儀同三司。甲子,以神宗德妃宋氏、劉氏為淑妃,賢妃喬氏為德妃。庚午,徙封仲損為齊安郡王,仲御為華陽郡王,孝騫為晉康郡王,孝參為豫章郡王,並開府儀同三司;封仲增為信安郡王,仲忽為普安郡王,仲癸為咸安郡王,
 仲僕為同安郡王,仲糜為淮安郡王。戊寅,徙封向宗回為漢東郡王,向宗良為開府儀同三司。仲損薨。河東、北盜起。



 二月甲申,置諸州曹掾官。甲午,詔建徽猷閣,藏《哲宗御集》,置學士、直學士、待制官。己亥,以安德軍節度使錢景臻為開府儀同三司。庚戌,以婕妤韋氏為修容。



 三月庚申,班《金菉靈寶道場儀範》於天下。甲子,封子材為魏國公。乙亥,封子模為鎮國公。戊寅,賜上舍生十三人及第。升乾寧軍為清州。詔監司歲舉所部郡守二人、縣
 令四人赴三省審察。夏四月甲辰,復洮州。



 五月庚戌朔,日有食之。辛亥,慮囚。以復洮州功,賜蔡京玉帶,加童貫檢校司空,仍宣撫。甲寅,復諸路歲貢供奉物。壬戌,溪哥王子臧征撲哥降,復積石軍。戊辰,詔官蔡京子孫一人,進執政官一等。



 六月乙酉,以涪夷地為珍州。甲午,以平夏城為懷德軍。乙未,以殿中六尚、算學、太官局、翰林儀鸞司皆隸六察。



 秋七月庚戌,罷建僖祖殿室。乙卯,以婉容王氏為賢妃。



 八月辛巳,邢州河水溢,壞民廬舍,復被
 水者家。丙申,中書侍郎梁子美罷知鄆州。己亥,置保州敦宗院。



 九月辛亥,以林攄為中書侍郎,吏部尚書餘深為尚書左丞。壬戌,貶向宗回為太子少保致仕。壬申,封子植為吳國公。癸酉,皇后王氏崩。削向宗回官爵。丙子,曲赦熙河蘭湟、秦鳳、永興軍路。冬十一月丁未朔,太白晝見。乙丑,上大行皇后謚曰靖和。



 十二月壬寅,陪葬靖和皇后於永裕陵。是歲,同州黃河清。出宮女七十有七人。於闐、夏國入貢。涪夷任應舉、楊文貴,湖南徭楊再光
 內附。



 三年春正月乙卯,祔靖和皇后神主於別廟。己未,減兩京、河陽、鄭州囚罪一等,民緣園陵役者蠲其賦。丁卯,以涪夷地為承州。甲戌,升湟州為向德軍節度。



 二月丙子朔,播州楊文貴納土,以其地置遵義軍。丁丑,韓忠彥致仕。



 三月丙午,立海商越界法。庚戌,御集英殿策進士。辛酉,詔四川郡守並選內地人任之。壬戌,並黔南入廣西路。乙丑,賜禮部奏名進士及第、出身六百八十五人。壬
 申,張康國卒。



 夏四月戊寅,林攄罷。戊子,以淑妃劉氏為貴妃。癸巳,以鄭居中知樞密院事,吏部尚書管師仁同知樞密院事。癸卯,以餘深為中書侍郎,兵部尚書薛昂為尚書左丞,工部尚書劉正夫為尚書右丞。



 五月乙巳朔,孟翊獻所畫卦象,謂宋將中微,宜更年號、改官名、變庶事以厭之。帝不樂,詔竄遠方。丙辰,令闢雍宴用雅樂。丁巳,慮囚。戊辰,大雨雹。辛未,以德妃喬氏為貴妃。



 六月甲戌朔,詔修《樂書》。管師仁罷。丁丑,蔡京罷。辛巳,以何執
 中為特進、尚書左僕射兼門下侍郎。以瀘夷地為純、滋二州。庚寅,冀州河水溢。



 秋七月丁未,詔謫籍人除元祐奸黨及得罪宗廟外,餘並錄用。丙辰,詔罷都提舉茶事司,在京令戶部、在外令轉運司主之。



 八月乙酉,封子樸為雍國公。己丑,嗣濮王宗漢薨。甲午,以仲增為開府儀同三司,封嗣濮王。丙申,升融州為清遠軍節度。己亥,韓忠彥薨。



 九月癸丑,封子棣為徐國公。己未,賜天下州學藏書閣名「稽古」。



 冬十月癸巳,減六尚局供奉物。



 十一月
 丁未,詔算學以黃帝為先師,風後等八人配饗,巫咸等七十人從祀。己巳,蔡京進封楚國公致仕,仍提舉《哲宗實錄》,朝朔望。十二月己亥,罷東南鑄夾錫錢。是歲,江、淮、荊、浙、福建旱,秦、鳳、階、成饑,發粟振之,蠲其賦。陜州、同州黃河清。闍婆、占城、夏國入貢。瀘州夷王募弱內附。



 四年春正月癸卯,罷改鑄當十錢。辛酉,詔士庶拜僧者,論以大不恭。丁卯,夏國入貢。二月庚午朔,禁然頂、煉臂、刺血、斷指。庚辰,罷京西錢監。甲申,詔自今以賞進秩者
 毋過中奉大夫。己丑,以餘深為門下侍郎。資政殿學士張商英為中書侍郎,戶部尚書侯蒙同知樞密院事。壬辰,罷河東、河北、京東鑄夾錫鐵錢。



 三月庚子,募饑民補禁卒。詔醫學生並入太醫局,算入太史局,書入翰林書藝局,畫入翰林畫圖局,學官等並罷。甲寅,敕所在振恤流民。癸亥,詔:罪廢人稍加甄敘,能安分守者,不俟滿歲,各與敘進,以責來效。丙寅,賜上舍生十五人及第。戊辰,詔上書邪下等人可依無過人例,今後改官升任並免
 檢舉。



 夏四月己卯,班樂尺於天下。癸未,蔡京上《哲宗實錄》。丙申,立感生帝壇。丁酉,詔修《哲宗史》。



 五月壬寅,停僧牒三年。丁未,彗出奎、婁。甲寅,立詞學兼茂科。丙辰,詔以彗見,避殿減膳,令侍從官直言指陳闕失。戊午,赦天下。壬戌,改廣西黔南路為廣南西路。癸亥,治廣西妄言拓地罪,追貶帥臣王祖道為昭信軍節度副使。甲子,貶蔡京為太子少保。丙寅,餘深罷。



 六月庚午,御殿復膳。乙亥,以張商英為尚書右僕射兼中書侍郎。壬辰,復向宗回
 為開府儀同三司、漢東郡王。乙未,慮囚。丙申,薛昂罷。



 秋七月辛丑,復罷方田。戊申,封子□咢為冀國公。



 八月乙亥,以劉正夫為中書侍郎,侯蒙為尚書左丞,翰林學士承旨鄧洵仁為尚書右丞。戊寅,省內外冗官。庚辰,以資政殿學士吳居厚為門下侍郎。丁亥,行內外學官選試法。



 閏月辛丑,詔諸路事有不便於民者,監司條奏之。癸卯,改陵井監為仙井監。辛酉,詔戒朋黨。以張閣知杭州,兼領花石綱。



 九月丙寅朔,日有食之。



 冬十月丁酉,立貴妃
 鄭氏為皇后。鄭居中罷。戊戌,太白晝見。以吳居厚知樞密院事。



 十一月乙丑朔,朝景靈宮。丙寅,饗太廟。丁卯,祀昊天上帝於圜丘,赦天下,改明年元。丙戌,罷拱州為襄邑縣。十二月庚戌,改謚靖和皇后為惠恭。是歲,夔州江水溢。海水清。出宮女四百八十六人。南丹州首領莫公晟內附。



 政和元年春正月己巳,以賢妃王氏為德妃。壬申,毀京師淫祠一千三十八區。戊寅,封子□共為定國公。丙戌,廢
 白、龔二州。壬辰,詔百官厲名節。



 二月壬寅,冊皇后。乙巳,詔陜西、河東復鑄夾錫錢。丙午,以太子少師鄭紳為開府儀同三司。



 三月己巳,詔監司督州縣長吏勸民增植桑柘,課其多寡為賞罰。癸酉,以吏部尚書王襄同知樞密院事。



 夏四月乙卯,罷陜西、河東鑄夾錫錢。丙辰,慮囚。立守令勸農黜陟法。丁巳,以淮南旱,降囚罪一等,徒以下釋之。



 五月癸亥,詔四川羨餘錢物歸左藏庫。戊辰,改當十錢為當三。己卯,東南有星晝隕。丁亥,解池生紅鹽。



 六月甲寅,復蔡京為太子少師。



 秋七月壬申,以疾愈,赦天下。癸未,廢平、從二州為砦。



 八月乙未,復蔡京為太子太師。丁巳,張商英罷。戊午,詔:「監司部內官吏,一歲中有犯罪至三人以上,雖不及三人而或有曾薦舉者,罪及監司。」九月戊寅,王襄罷。丁亥,封子栻為黃國公。是月,鄭允中、童貫使遼,以李良嗣來,良嗣獻取燕之策,詔賜姓趙。



 冬十月辛卯,以用事之臣多險躁朋比,下詔申儆。庚戌,封昭化軍節度使宗粹為信安郡王。辛亥,貶張商英
 為崇信軍節度副使。



 十一月任戌,以上書邪等及曾經入籍人並不許試學官。丙子,封子榛為福國公。十二月己酉,詔臺諫以直道核是非,毋憚大吏,毋比近習。辛亥,廢鎮州,升瓊州為靖海軍。是歲,虔州芝草生。蔡州瑞麥連野。河南府嘉禾生,野蠶成繭。出宮女八十人。交趾、夏國入貢。



\end{pinyinscope}