\article{志第一百八 選舉一(科目上)}

\begin{pinyinscope}

 自
 敷奏以言,明試以功,三載考績,三考黜陟幽明,始於《舜典》。司徒以鄉三物興賢能,太宰以三歲計吏治,詳於《周官》。兩漢而下,選舉之制不同,歸於得賢而已。考其大
 要,不過入仕則有貢舉之科,服官則有銓選之格,任事則有考課之法。然歷代之議貢舉者每曰:「取士以文藝,不若以德行。就文藝而參酌之,賦論之浮華,不若經義之實學。」議銓選者每曰:「以年勞取人,可以絕超躐,而不無賢愚同滯之嘆;以薦舉取人,可以拔俊傑,而不無巧佞捷進之弊。」議考課者每曰:「拘吏文,則上下督察,浸成澆風;通譽望,則權貴請托,徒開利路。」於是議論紛紜,莫之一也。



 宋初承唐制,貢舉雖廣,而莫重於進士、制科,其
 次則三學選補。銓法雖多,而莫重於舉削改官、磨勘轉秩。考課雖密,而莫重於官給歷紙,驗考批書。其它教官、武舉、童子等試,以及遺逸奏薦、貴戚公卿任子親屬與遠州流外諸選,委曲瑣細,咸有品式。其間變更不常,沿革迭見,而三百餘年元臣碩輔,鴻博之儒,清強之吏,皆自此出,得人為最盛焉。今輯舊史所錄,臚為六門:一曰科目;二曰學校試;三曰銓法;四曰補蔭;五曰保任;六曰考課。煩簡適中,隱括歸類,作《選舉志》。



 宋之科目,有進士,有諸科,有武舉。常選之外,又有制科,有童子舉,而進士得人為盛。神宗始罷諸科,而分經義、詩賦以取士,其後遵行,未之有改。自仁宗命郡縣建學,而熙寧以來,其法浸備,學校之設遍天下,而海內文治彬彬矣。今以科目、學校之制,各著於篇。



 初,禮部貢舉,設進士、《九經》、《五經》、《開元禮》、《三史》、《三禮》、《三傳》、學究、明經、明法等科,皆秋取解,冬集禮部,春考試。合格及第者,列名放榜於尚書省。凡進士,試詩、賦、論各一首,策五道,帖《論
 語》十帖,對《春秋》或《禮記》墨義十條。凡《九經》,帖書一百二十帖,對墨義六十條。凡《五經》,帖書八十帖,對墨義五十條。凡《三禮》,對墨義九十條。凡《三傳》,一百一十條,凡《開元禮》,凡《三史》,各對三百條。凡學究,《毛詩》對墨義五十條,《論語》十條,《爾雅》、《孝經》共十條,《周易》、《尚書》各二十五條。凡明法,對律令四十條,兼經並同《毛詩》之制。各間經引試,通六為合格,仍抽卷問律,本科則否。諸州判官試進士,錄事參軍試諸科,不通經義,則別
 選官考校,而判官監之。試紙,長官印署面給之。試中格者,第其甲乙,具所試經義,朱書通、否,監官、試官署名其下。進士文卷,諸科義卷、帖由,並隨解牒上之禮部。有篤廢疾者不得貢。貢不應法及校試不以實者,監官、試官停任。受賂,則論以枉法,長官奏裁。



 凡命士應舉,謂之鎖廳試。所屬先以名聞,得旨而後解。既集,什伍相保,不許有大逆人緦麻以上親,及諸不孝、不悌、隱匿工商異類、僧道歸俗之徒。家狀並試卷之首,署年及舉數、場第、鄉
 貫,不得增損移易,以仲冬收納,月終而畢。將臨試期,知舉官先引問聯保,與狀僉同而定焉。凡就試,唯詞賦者許持《切韻》、《玉篇》,其挾書為奸,及口相受授者,發覺即黜之。凡諸州長吏舉送,必先稽其版藉,察其行為;鄉里所推,每十人相保,內有缺行,則連坐不得舉。故事,知舉官將赴貢院,臺閣近臣得薦所知之負藝者,號曰「公薦」。太祖慮其因緣挾私,禁之。



 自唐以來,所謂明經,不過帖書、墨義,觀其記誦而已,故賤其科,而「不通」者其罰特重。乾
 德元年,詔曰:「舊制,《九經》一舉不第而止,非所以啟迪仕進之路也;自今依諸科許再試。」是年,諸州所薦士數益多,乃約周顯德之制,定諸州貢舉條法及殿罰之式:進士「文理紕繆」者殿五舉,諸科初場十「不」殿五舉,第二、第三場十「不」殿三舉,第一至第三場九「不」並殿一舉。殿舉之數,朱書於試卷,送中書門下。三年,陶谷子邴擢上第,帝曰:「穀不能訓子,安得登第?」乃詔:「食祿之家,有登第者,禮部具姓名以聞,令覆試之。」自是,別命儒臣於中書
 覆試,合格乃賜第。時川蜀、荊湖內附,試數道所貢士,縣次往還續食。開寶三年,詔禮部閱貢士及十五舉嘗終場者,得一百六人,賜本科出身。特奏名恩例,蓋自此始。



 五年,禮部奏合格進士、諸科凡二十八人,上親召對講武殿,而未及引試也。明年,翰林學士李昉知貢舉,取宋準以下十一人,而進士武濟川、《三傳》劉睿材質最陋,對問失次,上黜之。濟川,昉鄉人也。會有訴昉用情取舍,帝乃籍終場下第人姓名,得三百六十人,皆召見,擇其一
 百九十五人,並準以下,乃御殿給紙筆,別試詩賦。命殿中侍御史李瑩等為考官,得進士二十六人,《五經》四人,《開元禮》七人,《三禮》三十八人,《三傳》二十六人,《三史》三人,學究十八人,明法五人,皆賜及第,又賜錢二十萬以張宴會。昉等尋皆坐責。殿試遂為常制。帝嘗語近臣曰:「昔者,科名多為勢家所取,朕親臨試,盡革其弊矣。」八年,親試進士王式等,乃定王嗣宗第一,王式第四。自是御試與省試名次,始有升降之別。時江南未平,進士林松、雷
 說試不中格,以其間道來歸,亦賜《三傳》出身。



 太宗即位,思振淹滯,謂侍臣曰:「朕欲博求俊彥於科場中,非敢望拔十得五,止得一二,亦可為致治之具矣。」太平興國二年,御殿覆試,內出賦題,賦韻平側相間,依次而用。命李昉、扈蒙第其優劣為三等,得呂蒙正以下一百九人。越二日,覆試諸科,得二百人。並賜及第。又閱貢藉,得十舉以上至十五舉進士、諸科一百八十餘人,並賜出身;《九經》七人不中格,亦憐其老,特賜同《三傳》出身。凡五百餘
 人,皆賜袍笏,錫宴開寶寺,帝自為詩二章賜之。甲、乙第進士及《九經》,皆授將作監丞、大理評事,通判諸州,其餘亦優等授官。三年九月,廷試舉人。故事,惟春放榜,至是秋試,非常例也。是冬,諸州舉人並集,會將親征北漢,罷之。自是,間一年或二年乃貢舉。



 五年,覆試進士。有顏明遠、劉昌言、張觀、樂史四人,以見任官舉進士,特授近藩掌書記。有趙昌國者,求應百篇舉,謂一日作詩百篇。帝出雜題二十,令各賦五篇,篇八句。日旰,僅成數十首,
 率無可觀。帝以是科久廢,特賜及第,以勸來者。



 八年,進士、諸科始試律義十道,進士免帖經。明年,惟諸科試律,進士復帖經。進士始分三甲。自是錫宴就瓊林苑。上因謂近臣曰:「朕親選多士,殆忘饑渴,召見臨問,觀其才技而用之,庶使田野無遺逸,而朝廷多君子爾。」雍熙二年,廷試初唱名及第,第一等為節度推官。是年及端拱初,禮部試已,帝慮有遺才,取不中格者再試之,於是由再試得官者數百人。凡廷試,帝親閱卷累日,宰相屢請宜
 歸有司,始詔歲命官知舉。



 舊制,既鎖院,給左藏錢十萬資費用。端拱元年,詔改支尚書祠部,仍倍其數,罷御廚、儀鸞司供帳。知貢舉宋白等定貢院故事:先期三日,進士具都榜引試,借御史臺驅使官一人監門,都堂簾外置案,設銀香爐,唱名給印試紙。及試中格,錄進士之文奏御,諸科惟籍名而上;俟制下,先書姓名散報之,翌日,放傍唱名。既謝恩,詣國學謁先聖先師,進士過堂合下告名。聞喜宴分為兩日,宴進士,請丞郎、大兩省;宴諸科,
 請省郎、小兩省。綴行期集,列敘名氏、鄉貫、三代之類書之,謂之小錄。醵錢為游宴之資,謂之酺。皆團司主之。制下,而中書省同貢院關黃覆奏之,俟正敕下,關報南曹、都省、御史臺,然後貢院寫春關散給。



 籍而入選謂之春關。



 登科之人,例納朱膠綾紙之直,赴吏部南曹試判三道,謂之關試。



 淳化三年,諸道貢士凡萬七千餘人。先是,有擊登聞鼓訴校試不公者。蘇易簡知貢舉,受詔即赴貢院,仍糊名考校,遂為例。既廷試,帝諭多士曰:「爾等各負志業,效
 官之外,更勵精文採,無墜前功也。」詔刻《禮記·儒行篇》賜之。每科進士第一人,天子寵之以詩,後嘗作箴賜陳堯叟,至是,並賜焉。先是,嘗並學究、《尚書》、《周易》為一科,始更定本經日試義十道,《尚書》、《周易》各義五道,仍雜問疏義六道,經注四道。明法舊試六場,更定試七場:第一、第二場試律,第三場試令,第四、第五場試小經,第六場試令,第七場試律,仍於試律日雜問疏義六、經注四。凡《三禮》、《三傳》、《通禮》每十道義分經注六道、疏義四道,以六通為
 合格。



 自淳化末,停貢舉五年。真宗即位,復試,而高句麗始貢一人。先是,國子監、開封府所貢士,與舉送官為姻戚,則兩司更互考試,始命遣官別試。



 咸平三年,親試陳堯咨等八百四十人,特奏名者九百餘人,有晉天福中嘗預貢者。凡士貢於鄉而屢絀於禮部,或廷試所不錄者,積前後舉數,參其年而差等之,遇親策士則別籍其名以奏,徑許附試,故曰特奏名。又賜河北進士、諸科三百五十人及第、同出身。既下第,願試武藝及量才錄用者,
 又五百餘人,悉賜裝錢慰遣之,命禮部敘為一舉。較藝之詳,推恩之廣,近代所未有也。



 舊制,及第即命以官。上初復廷試,賜出身者亦免選,於是策名之士尤眾,雖藝不及格,悉賜同出身。乃詔有司,凡賜同出身者並令守選,循用常調,以示甄別。又定令:凡試卷,封印院糊名送知舉官考定高下,復令封之送覆考所,考畢然後參校得失,不合格者,須至覆場方落。諭館閣、臺省官,有請屬舉人者密以聞,隱匿不告者論罪。仍詔諸王、公主、近臣,
 毋得以下第親族賓客求賜科名。



 景德四年,命有司詳定《考校進士程序》,送禮部貢院,頒之諸州。士不還鄉里而竊戶他州以應選者,嚴其法。每秋賦,自縣令佐察行義保任之,上於州;州長貳覆審察得實,然後上本道使者類試。已保任而有缺行,則州縣皆坐罪;若省試而文理紕繆,坐元考官。諸州解試額多而中者少,則不必足額。



 尋又定《親試進士條制》。凡策士,即殿兩廡張帟,列幾席,標姓名其上。先一日表其次序,揭示闕外,翌旦拜闕
 下,仍入就席。試卷,內臣收之,付編排官,去其卷首鄉貫狀,別以字號第之;付封彌官謄寫校勘,用御書院印,付考官定等畢,復封彌送覆考官再定等。編排官閱其同異,未同者再考之;如復不同,即以相附近者為定。始取鄉貫狀字號合之,即第其姓名、差次,並試卷以聞。其考第之制凡五等:學識優長、詞理精絕為第一;才思該通、文理周率為第二;文理俱通為第三;文理中平為第四;文理疏淺為第五。然後臨軒唱第,上二等曰及第,三等
 曰出身,四等、五等曰同出身。餘如貢院舊制。



 大中祥符五年,詔士曾預南省試者,犯公罪聽贖罰。令禮部取前後詔令經久可行者,編為條制。諸科三場內有十「不」、進士詞理紕繆者各一人以上,監試、考試官從違制失論,幕職、州縣官得代日殿一選,京朝官降監場務,嘗監當則與遠地;有三人,則監試、考試官亦從違制失論,幕職、州縣官沖替,京朝官遠地監當;有五人,則監試以下皆停見任;舉送守倅,諸科五十人以上有一人十「不」,即罰銅與免殿
 選監當,進士詞理紕繆亦如之。後又詔:「試鎖廳者,州長吏先校試合格,始聽取解;至禮部不及格,停其官,而考試及舉送者,皆重置罪。」八年,始置謄錄院,令封印官封試卷付之,集書吏錄本,監以內侍二人。詔:「進士第一人,令金吾司給七人導從,聽引兩節。著為令。」



 天聖初,宋興六十有二載,天下乂安。時取才唯進士、諸科為最廣,名卿鉅公,皆繇此選,而仁宗亦向用之,登上第者不數年,輒赫然顯貴矣。其貢禮部而數詘者,得特奏名,或因循
 不學,乃詔曰:「學猶殖也,不學將落,遜志務時敏,厥修乃來。朕慮天下之士或有遺也,既已臨軒較得失,而憂其屢不中科,則衰邁而無所成,退不能返其里閭,而進不得預於祿仕。故常數之外,特為之甄採。而狃於寬恩,遂隳素業,茍簡成風,甚可恥也。自今宜篤進厥學,無習僥幸焉。」時晏殊言:「唐明經並試策問,參其所習,以取材識短長。今諸科專記誦,非取士之意,請終場試策一篇。」詔近臣議之,咸謂諸科非所習,議遂寢。舊制,鎖廳試落輒
 停官,至是始詔免罪。



 景祐初,詔曰:「向學之士益蕃,而取人路狹,使孤寒棲遲,或老而不得進,朕甚憫之。其令南省就試進士、諸科,十取其二。凡年五十,進士五舉、諸科六舉;嘗經殿試,進士三舉、諸科五舉;及嘗預先朝御試,雖試文不合格,毋輒黜,皆以名聞。」自此率以為常。士有親戚仕本州,或為發解官,及侍親遠宦,距本州二千里,令轉運司類試,以十率之,取三人。於是諸路始有別頭試。其年,詔開封府、國子監及別頭試,封彌、謄錄如禮
 部。



 初,貢士踵唐制,猶用公卷,然多假他人文字,或傭人書之。景德中,嘗限舉人於試紙前親書家狀,如公卷及後所試書體不同,並駁放;其假手文字,辨之得實,即斥去,永不得赴舉。賈昌朝言:「自唐以來,禮部採名譽,觀素學,故預投公卷;今有封彌、謄錄法,一切考諸試篇,則公卷可罷。」自是不復有公卷。



 寶元中,李淑侍經筵,上訪以進士詩、賦、策、論先後,俾以故事對。淑對曰:「唐調露二年,劉思立為考功員外郎,以進士試策滅裂,請帖經以觀
 其學,試雜文以觀其才。自此沿以為常。至永隆二年,進士試雜文二篇,通文律者,始試策。天寶十一年,進士試一經,能通者試文賦,又通而後試策,五條皆通,中第。建中二年,趙贊請試以時務策五篇,箴、論、表、贊各一篇,以代詩、賦。大和三年,試帖經,略問大義,取精通者,次試論、議各一篇。八年,禮部試以帖經口義,次試策五篇,問經義者三,問時務者二。厥後變易,遂以詩賦為第一場,論第二場,策第三場,帖經第四場。今陛下欲求理道而不
 以雕琢為貴,得取士之實矣。然考官以所試分考,不能通加評校,而每場輒退落,士之中否,殆系於幸不幸。願約舊制,先策,次論,次賦及詩,次帖經、墨義,而敕有司並試四場,通較工拙,毋以一場得失為去留。」詔有司議,稍施行焉。



 既而知制誥富弼言曰:「國家沿隋、唐設進士科,自咸平、景德以來,為法尤密,而得人之道,或有未至。且歷代取士,悉委有司,未聞天子親試也。至唐武后始有殿試,何足取哉?使禮部次高下以奏,而引諸殿廷,唱名
 賜第,則與殿試無以異矣。」遂詔罷殿試。而議者多言其輕上恩,隳故事,復如舊。



 時範仲淹參知政事,意欲復古勸學,數言興學校,本行實。詔近臣議,於是宋祁等奏:「教不本於學校,士不察於鄉里,則不能核名實。有司束以聲病,學者專於記誦,則不足盡人材。參考眾說,擇其便於今者,莫若使士皆土著,而教之於學校,然後州縣察其履行,則學者修飭矣。」乃詔州縣立學,士須在學三百日,乃聽預秋試,舊嘗充試者百日而止。試於州者,令相
 保任,有匿服、犯刑、虧行、冒名等禁。三場:先策,次論,次詩賦,通考為去取,而罷帖經、墨義,士通經術願對大義者,試十道。仲淹既去,而執政意皆異。是冬,詔罷入學日限。言初令不便者甚眾,以為詩賦聲病易考,而策論汗漫難知;祖宗以來,莫之有改,且得人嘗多矣。天子下其議,有司請如舊法。乃詔曰:「科舉舊條,皆先朝所定也,宜一切如故,前所更定令悉罷。」



 會張方平知貢舉,言:「文章之變與政通。今設科選才,專取辭藝,士惟道義積於中,英
 華發於外,然則以文取士,所以叩諸外而質其中之蘊也。言而不度,則何觀焉。邇來文格日失其舊,各出新意,相勝為奇。朝廷惡其然,屢下詔書戒飭,而學者樂於放逸,罕能自還。今賦或八百字,論或千餘字,策或置所問而妄肆胸臆,漫陳他事,驅扇浮薄,重虧雅俗,豈取賢斂才備治具之意邪?其踵習新體,澶漫不合程序,悉已考落,請申前詔,揭而示之。」



 初,禮部奏名,以四百名為限,又諸科雜問大義,僥幸之人,悉以為不便。知制誥王珪奏
 曰:「唐自貞觀訖開元,文章最盛,較藝者歲千餘人,而所收無幾。咸亨、上元增其數,亦不及百人。國初取士,大抵唐制,逮興國中,貢舉之路浸廣,無有定數。比年官吏猥眾,故近詔限四百人,以懲其弊。且進士、明經先經義而後試策,三試皆通為中第,大略與進士等,而諸科既不問經義,又無策試,止以誦數精粗為中否,則其專固不達於理,安足以長民冶事哉?前詔諸科終場問本經大義十道,《九經》、《五經》科止問義而不責記誦,皆以著於令。
 言者以為難於遽更,而圖安於弊也。惟陛下申敕有司,固守是法,毋輕易焉。」



 嘉祐二年,親試舉人,凡與殿試者始免黜落。時進士益相習為奇僻,鉤章棘句,浸失渾淳。歐陽修知貢舉,尤以為患,痛裁抑之,仍嚴禁挾書者。既而試榜出,時所推譽,皆不在選。澆薄之士,候修晨朝,群聚詆斥之,街司邏卒不能止,至為祭文投其家,卒不能求其主名置於法,然自是文體亦少變。待試京師者恆六七千人,一不幸有故不應詔,往往沉淪十數年,以此
 毀行干進者,不可勝數。



 王洙侍邇英閣講《周禮》,至「三年大比,大考州里,以贊鄉大夫廢興。」上曰:「古者選士如此,今率四五歲一下詔,故士有抑而不得進者,孰若裁其數而屢舉也。」下有司議,咸請:「易以間歲之法,則無滯才之嘆。薦舉數既減半,主司易以詳較,得士必精。且人少則有司易於檢察,偽濫自不能容,使寒苦藝學之人得進。」於是下詔:「間歲貢舉,進士、諸科悉解舊額之半。增設明經,試法:凡明兩經或三經、五經,各問大義十條,兩經
 通八,三經通六,五經通五為合格,兼以《論語》、《孝經》,策時務三條,出身與進士等。而罷說書舉。」



 時以科舉既數,而高第之人驟顯,欲稍裁抑。遂詔曰:「朕惟國之取士,與士之待舉,不可曠而冗也。故立間歲之期,以勵其勤;約貢舉之數,以精其選。著為定式,申敕有司,而高第之人,嘗不次而用。若循舊比,終至濫官,甚無謂也。自今制科入第三等,與進士第一,除大理評事、簽書兩使幕職官;代還,升通判;再任滿,試館職。制科入第四等,與進士第
 二、第三,除兩使幕職官;代還,改次等京官。制科入第五等,與進士第四、第五,除試銜知縣;代還,遷兩使職官。鎖廳人視此。若夫高才異行,施於有政而功狀較然者,當以異恩擢焉。」仁宗之朝十有三舉,進士四千五百七十人;其甲第之三人凡三十有九,其後不至於公卿者,五人而已。英宗即位,議者以間歲貢士法不便。乃詔禮部三歲一貢舉,天下解額,取未行間歲之前四之三為率,明經、諸科毋過進士之數。



 神宗篤意經學,深憫貢舉之
 弊,且以西北人材多不在選,遂議更法。王安石謂:「古之取士俱本於學,請興建學校以復古。其明經、諸科欲行廢罷,取明經人數增進士額。」乃詔曰:「化民成俗,必自庠序;進賢興能,抑由貢舉。而四方執經藝者專於誦數,趨鄉舉者狃於文辭,與古所謂『三物賓興,九年大成』,亦已盭矣。今下郡國招徠雋賢,其教育之方,課試之格,令兩制、兩省、待制以上、御史、三司、三館雜議以聞。」議者多謂變法便。直史館蘇軾曰:



 得人之道,在於知人,知人之法,
 在於責實。使君相有知人之明,朝廷有責實之政,則胥吏、皂隸,未嘗無人,雖用今之法,臣以為有餘;使無知人之明,無責實之政,則公卿、侍從,常患無人,況學校貢舉乎?雖復古之制,臣以為不足矣。



 時有可否,物有興廢,使三代聖人復生於今,其選舉亦必有道,何必由學乎?且慶歷間嘗立學矣,天下以為太平可待,至於今惟空名僅存。今陛下必欲求德行道藝之士,責九年大成之業,則將變今之禮,易今之俗。又當發民力以治宮室,斂民
 財以養游士,置學立師;以又時簡不帥教者,屏之遠方,徒為紛紛,其與慶歷之際何異?至於貢舉,或曰鄉舉德行而略文章;或曰專取策論而罷詩賦;或欲舉唐故事,採譽望而罷封彌;或欲變經生帖、墨而考大義,此數者皆非也。



 夫欲興德行,在於君人者修身以格物,審好惡以表俗,若欲設科立名以取之,則是教天下相率而為偽也。上以孝取人,則勇者割股,怯者廬墓。上以廉取人,則弊車、羸馬、惡衣、菲食,凡可以中上意者無所不至。自
 文章言之,則策論為有用,詩賦為無益;自政事言之,則詩賦、論策均為無用。然自祖宗以來莫之廢者,以為設法取士,不過如此也。近世文章華麗,無如楊億。使億尚在,則忠清鯁亮之士也。通經學古,無如孫復、石介。使復、介尚在,則迂闊誕謾之士也。矧自唐至今,以詩賦為名臣者,不可勝數,何負於天下,而必欲廢之?



 帝讀軾疏曰:「吾固疑此,得軾議,釋然矣。」他日問王安石,對曰:「今人材乏少,且其學術不一,異論紛然,不能一道德故也。一道
 德則修學校,欲修學校,則貢舉法不可不變。若謂此科嘗多得人,自緣仕進別無他路,其間不容無賢;若謂科法已善,則未也。今以少壯時,正當講求天下正理,乃閉門學作詩賦,及其入官,世事皆所不習,此科法敗壞人材,致不如古。」



 既而中書門下又言:「古之取士,皆本學校,道德一於上,習俗成於下,其人才皆足以有為於世。今欲追復古制,則患於無漸。宜先除去聲病偶對之文,使學者得專意經術,以俟朝廷興建學校,然後講求三代
 所以教育選舉之法,施於天下,則庶幾可以復古矣。」於是改法,罷詩賦、帖經、墨義,士各占治《易》、《詩》、《書》、《周禮》、《禮記》一經,兼《論語》、《孟子》。每試四場,初大經,次兼經,大義凡十道,後改《論語》、《孟子》義各三道。



 次論一首,次策三道,禮部試即增二道。中書撰大義式頒行。試義者須通經、有文採乃為中格,不但如明經墨義粗解章句而已。取諸科解名十之三,增進士額,京東西、陜西、河北、河東五路之創試進士者,及府、監、他路之舍諸科而為進士者,乃得所增之額以
 試。皆別為一號考取,蓋欲優其業,使不至外侵,則常慕向改業也。



 又立新科明法,試律令、《刑統》,大義、斷桉,所以待諸科之不能業進士者。未幾,選人、任子,亦試律令始出官。又詔進士自第三人以下試法。或言:「高科任簽判及職官,於習法豈所宜緩。昔試刑法者,世皆指為俗吏,今朝廷推恩既厚,而應者尚少,若高科不試,則人不以為榮。」乃詔悉試。帝嘗言:「近世士大夫,多不習法。」吳充曰:「漢陳寵以法律授徒,常數百人。律學在六學之一,後來
 縉紳,多恥此學。舊明法科徒誦其文,罕通其意,近補官必聚而試之,有以見恤刑之意。」



 熙寧三年,親試進士,始專以策,定著限以千字。舊特奏名人試論一道,至是亦制策焉。帝謂執政曰:「對策亦何足以實盡人材,然愈於以詩賦取人爾。」舊制,進士入謝,進謝恩銀百兩,至是罷之。仍賜錢三千,為期集費。諸州舉送、發解、考試、監試官,凡親戚若門客毋試於其州,類其名上之轉運司,與鎖廳者同試,率七人特立一額。後復令存諸科舊額十之一,
 以待不能改業者。



 元祐初,知貢舉蘇軾、孔文仲言:「每一試,進士、諸科及特奏名約八九百人。舊制,禮部已奏名,至御試而黜者甚多。嘉祐始盡賜出身,近雜犯亦免黜落,皆非祖宗本意。進士升甲,本為南省第一人,唱名近下,方特升之,皆出一時聖斷。今禮部十人以上,別試、國子、開封解試、武舉第一人,經明行修進士及該特奏而預正奏者,定著於令,遞升一甲。則是法在有司,恩不歸於人主,甚無謂也。今特奏者約已及四百五十人,又許
 例外遞減一舉,則當復增數百人。此曹垂老無他望,布在州縣,惟務黷貨以為歸計。前後恩科命官,幾千人矣,何有一人能自奮厲,有聞於時?而殘民敗官者,不可勝數。以此知其無益有損。議者不過謂宜廣恩澤,不知吏部以有限之官待無窮之吏,戶部以有限之財祿無用之人,而所至州縣,舉罹其害。乃即位之初,有此過舉,謂之恩澤,非臣所識也。願斷自聖意,止用前命,仍詔考官量取一二十人,誠有學問,即許出官。其餘皆補文學、長
 史之類,不理選限,免使積弊增重不已。」遂詔定特奏名考取數,進士入四等以上、諸科入三等以上,通在試者計之,毋得取過全額之半,是後著為令。



 時方改更先朝之政,禮部請置《春秋》博士,專為一經。尚書省請復詩賦,與經義兼行,解經通用先儒傳注及己說。又言:「新科明法中者,吏部即注司法,敘名在及第進士之上。舊明法最為下科,然必責之兼經,古者先德後刑之意也。欲加試《論語》大義,仍裁半額,注官依科目次序。」詔近臣集議。
 左僕射司馬光曰:「取士之道,當先德行,後文學;就文學言之,經術又當先於詞採。神宗專用經義、論策取士,此乃復先王令典,百王不易之法。但王安石不當以一家俬學,今天下學官講解。至於律令,皆當官所須,使為士者果能知道義,自與法律冥合;何必置明法一科,習為刻薄,非所以長育人材、敦厚風俗也。」



 四年,乃立經義、詩賦兩科,罷試律義。凡詩賦進士,於《易》、《詩》、《書》、《周禮》、《禮記》、《春秋左傳》內聽習一經。初試本經義二道,《語》、《孟》義各一道,
 次試賦及律詩各一首,次論一首,末試子、史、時務策二道。凡專經進士,須習兩經,以《詩》、《禮記》、《周禮》、《左氏春秋》為大經,《書》、《易》、《公羊》、《穀梁》、《儀禮》為中經,《左氏春秋》得兼《公羊》、《穀梁》、《書》,《周禮》得兼《儀禮》或《易》,《禮記》、《詩》並兼《書》,願習二大經者聽,不得偏占兩中經。初試本經義三道,《論語》義一道,次試本經義三道,《孟子》義一道,次論策,如詩賦科。並以四場通定高下,而取解額中分之,各占其半。專經者用經義定取舍,兼詩賦者以詩賦為去留,其名次高下,
 則於策論參之。自復詩賦,上多向習,而專經者十無二三,諸路奏以分額各取非均,其後遂通定去留,經義毋過通額三分之一。



 光又請:「立經明行修科,歲委升朝文臣各舉所知,以勉勵天下,使敦士行,以示不專取文學之意。若所舉人違犯名教及贓私罪,必坐舉主,毋有所赦,則自不敢妄舉。而士之居鄉、居家者,立身行己,不敢不謹,惟懼玷缺外聞。所謂不言之教,不肅而成,不待學官日訓月察,立賞告訐,而士行自美矣。」遂立科,許各舉
 一人。凡試進士者,及中第唱名日,用以升甲。後分路別立額六十一人,州縣保任上之監司,監司考察以聞,無其人則否。預薦者不試於州郡,惟試禮部。不中,許用特奏名格赴廷試,後以為常。既而詔須特命舉乃舉,毋概以科場年上其名。



 六年,詔復通禮科。初,開寶中,改鄉貢《開元禮》為《通禮》,熙寧嘗罷,至是始復。凡禮部試,添知舉官為四員,罷差參詳官,而置點檢官二十人,分屬四知舉,使協力通考;諸州點檢官專校雜犯,亦預考試。



 八年,
 中書請御試復用祖宗法,試詩賦、論、策三題。且言:「士子多已改習詩賦,太學生員總二千一百餘人,而不兼詩賦者才八十二人。」於是詔:「來年禦試,習詩賦人復試三題,專經人且令試策。」自後概試三題。帝既親政,群臣多言元祐所更學校、科舉制度非是,帝念宣仁保祐之功,不許改。紹聖初,議者益多,乃詔進士罷詩賦,專習經義,廷對仍試策。初,神宗念字學廢缺,詔儒臣探討,而王安石乃進其說,學者習焉。元祐禁勿用。至是,除其禁。四年,
 詔禮部,凡內外試題悉集以為籍,遇試,頒付考官,以防復出。罷《春秋》科,凡試,優取二《禮》,兩經許占全額之半,而以其半及他經。既而復立《春秋》博士,崇寧又罷之。



 徽宗設闢雍於國郊,以待士之升貢者。臨幸,加恩博士弟子有差。然州郡猶以科舉取士,不專學校。崇寧三年,遂詔:「天下取士,悉由學校升貢,其州郡發解及試禮部法並罷。」自此,歲試上舍,悉差知舉,如禮部試。五年,詔:「大比歲更參用科舉取士一次,其亟以此意使遠士即聞之。」時
 州縣悉行三舍法,得免試入學者,多當官子弟,而在學積歲月,累試乃得應格,其貧且老者甚病之,故詔及此,而未遽廢科舉也。大觀四年五月,星變,凡事多所更定。侍御史毛注言:「養士既有額,而科舉又罷,則不隸學籍者,遂致失職。天之視聽以民,士,其民之秀者,今失職如此,疑天亦譴怒。願以解額之歸升貢者一二分,不絕科舉,亦應天之一也。」遂詔更行科舉一次。臣僚言:「場屋之文,專尚偶麗,題雖無兩意,必欲厘而為二,以就對偶;其
 超詣理趣者,反指以為澹泊。請擇考官而戒飭之,取其有理致而黜其強為對偶者,庶幾稍救文弊。」



 宣和三年,詔罷天下三舍法,開封府及諸路並以科舉取士;惟太學仍存三舍,以甄序課試,遇科舉仍自發解。六年,禮部試進士萬五千人,詔特增百人額,正奏名賜第者八百餘人,因上書獻頌直令赴試者殆百人。有儲宏等隸大閹梁師成為使臣或小史,皆賜之第。梁師成者,於大觀三年嘗中甲科。自設科以來,南宮試者,無逾此年之盛。
 然雜流閹宦,俱玷選舉,而祖宗之良法蕩然矣。凡士不繇科舉若三舍而賜進士第及出身者,其所從得不一。凡遺逸、文學,吏能言事或奏對稱旨,或試法而經律入優,或材武、或童子而皆能文,或邊臣之子以功來奏,其得之雖有當否,大較猶可取也。崇寧、大觀之後,達官貴冑既多得賜,以上書獻頌而得者,又不勝紀矣。



\end{pinyinscope}