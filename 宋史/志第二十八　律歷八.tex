\article{志第二十八 律歷八}

\begin{pinyinscope}

 明天歷



 步晷漏術



 二至限:一百八十一日六十二分。



 一象
 度:九十
 一度
 三
 十一分。



 消息法:一萬六百八十九。



 辰法:三千二百五十。



 刻法:三百九十。



 半辰法:一千六百二十五。



 昏明刻分:九百七十五。



 昏明:二刻一百九十五分。



 冬至嶽臺晷景常數:一丈二尺八寸五分。



 夏至嶽臺晷景常數:一尺五寸七分。



 冬至後初限、夏至後末限:四十五日六十二分。



 夏至後初限、冬至後末限:一百三十七日。



 求嶽臺晷景入二至後日數:計入二至後來日數,以二至約餘減之,仍加半日之分,即為入二至後來日午中積數及分。



 求嶽臺晷景午中定數:置所求午中積數,如初限以下者為在初;已上者,覆減二至限,餘為在末。其在冬至後初限、夏至後末限者,以入限日減一千九百三十七半,
 為泛差;仍以入限日分乘其日盈縮積,盈縮積在日度術中。



 五因百約之,用減泛差,為定差;乃以入限日分自相乘,以乘定差,滿一百萬為尺,不滿為寸、為分及小分,以減冬至常晷,餘為其日午中晷景定數。若所求入冬至後末限、夏至後初限者,乃三約入限日分,以減四百八十五少,餘為泛差;仍以盈縮差減極數,餘者若在春分後、秋分前者,直以四約之,以加泛差,為定差;若春分前、秋分後者,以去二分日數及分乘之,滿六百而一,以減泛差,
 餘為定差;乃以入限日分自相乘,以乘定差,滿一百萬為尺,不滿為寸、為分及小分,以加夏至常晷,即為其日午中晷景定數。



 求每日消息定數:置所求日中日度分,如在二至限以下者為在息;以上者去之,餘為在消。又視入消息度加一象以下者為在初;以上者,覆減二至限,餘為在末。其初、末度自相乘,以一萬乘而再折之,滿消息法除之,為常數。乃副之,用減一千九百五十,餘以乘其副,滿八千
 六百五十除之,所得以加常數,為所求消息定數。



 求每日黃道去極度及赤道內外度:置其日消息定數,以四因之,滿三百二十五除之為度,不滿,退除為分,所得,在春分後加六十七度三十一分,在秋分後減一百一十五度三十一分,即為所求日黃道去極度及分。以黃道去極度與一象度相減,餘為赤道內、外度。若去極度少,為日在赤道內;若去極度多,為日在赤道外。



 求每日晨昏分及日出入分:以其日消息定數,春
 分後加六千八百二十五,秋分後減一萬七百二十五,餘為所求日晨分;用減元法,餘為昏分。以昏明分加晨分,為日出分;減昏分,為日入分。



 求每日距中距子度及每更差度:置其日晨分,以七百乘之,滿七萬四千七百四十二除為度,不滿,退除為分,命曰距子度;用減半周天,餘為距中度。若倍距子度,五除之,即為每更差度及分。若依司辰星漏歷,則倍距子度,減去待旦三十六度五十二分半,餘以五約之,即每更差度。



 求每日夜半定漏:置其日晨分,以刻法除之為刻,不滿
 為分,即所求日夜半定漏。



 求每日晝夜刻及日出入辰刻:倍夜半定漏,加五刻,為夜刻。用減一百刻,餘為晝刻。以昏明刻加夜半定漏,滿辰法除之為辰數,不滿,刻法除之為刻,又不滿,為刻分。命辰數從子正,算外,即日出辰刻;以晝刻加之,命如前,即日入辰刻。若以半辰刻加之,即命從辰初也。



 求更點辰刻:倍夜半定漏,二十五而一,為點差刻;五因之,為更差刻。以昏明刻加日入辰刻,即甲夜辰刻;以更
 點差刻累加之,滿辰刻及分去之,各得更點所入辰刻及分。若同司辰星漏歷者,倍夜半定漏,減去待旦一十刻,餘依術求之,即同內中更點。



 求昏曉及五更中星:置距中度,以其日昏後夜半赤道日度加而命之,即其日昏中星所格宿次,其昏中星便為初更中星;以每更差度加而命之,即乙夜所格中星;累加之,得逐更中星所格宿次。又倍距子度,加昏中星命之,即曉中星所格宿次。若同司辰星漏歷中星,則倍距子度,減去待旦十刻之度三十六度五十二分半,餘約之為五更,即同內中更點中星。



 求九服距差日:各於所在立表候之,若地在嶽臺北,測冬至後與嶽臺冬至晷景同者,累冬至後至其日,為距差日;若地在嶽臺南,測夏至後與嶽臺晷景同者,累夏至後至其日,為距差日。



 求九服晷景:若地在嶽臺北冬至前後者,以冬至前後日數減距差日,為餘日;以餘日減一千九百三十七半,為泛差;依前術求之,以加嶽臺冬至晷景常數,為其地其日中晷常數。若冬至前後日多於距差日,乃減去距
 差日,餘依前術求之,即得其地其日中晷常數。若地在嶽臺南夏至前後者,以夏至前後日數減距差日,為餘日;乃三約之,以減四百八十五少,為泛差;依前術求之,以減嶽臺夏至晷景常數,即其地其日中晷常數。如夏至前後日數多於距差日,乃減嶽臺夏至常晷,餘即晷在表南也。若夏至前後日多於距差日,即減去距差日,餘依前術求之,各得其地其日中晷常數。若求定數,依立成以求午中晷景定數。



 求九服所在晝夜漏刻:冬、夏二至各於所在下水漏,以定其地二至夜刻,乃相減,餘為冬、夏至差刻。置嶽臺其日消息定數,以其地二至差刻乘之,如嶽臺二至差刻二十而一,所得,為其地其日消息定數。乃倍消息定數,滿刻法約之為刻,不滿為分,乃加減其地二至夜刻,秋分後、春分前,減冬至夜刻;春分後、秋分前,加夏至夜刻。



 為其地其日夜刻;用減一百刻,餘為晝刻。其日出入辰刻及距中度五更中星,並依前術求之。



 步月離術



 轉度母:八千一百一十二萬。



 轉終分:二百九十八億八千二百二十四萬二千二百五十一。



 朔差:二十一億四千二百八十八萬七千。



 朔差:二十六度。餘三千三百七十六萬七千,約餘四千一百六十二半。



 轉法:一十億八千四百四十七萬三千。



 會周:三百二十億二千五百一十二萬九千二百五十一。



 轉終:三百六十八度。餘三十八萬二千二百五十一,約餘三千七百八。



 轉終:二十七日。餘六億一百四十七萬一千二百五十一,約餘五千五百四十六。



 中度:一百八十四度。餘一千五百四萬一千一百二十五半,約餘一千八百五十四。



 象度:九十二度。餘七百五十二萬五百六十二太,約分九百二十七。



 月平行:十三度。餘二千九百九十一萬三千,約分三千六百八十七半。



 望差:一百九十七度。餘三千一百九十二萬四千六百二十五半,約分三千九百三十四。



 弦差:九十八度。餘五千六百五十二萬二千三百一十二太,約分六千九百六十七。



 日衰:一十八、小分九。



 求月行入轉度:以朔差乘所求積月,滿轉終分去之,不盡為轉餘。滿轉度母除為度,不滿為餘,其餘若以一萬乘之,滿轉度母除之,即得約分;若以轉法除轉餘,即為入轉日及餘。



 即得所求月加時入轉度及餘。若以弦度及餘累加之,即得上弦、望、下弦及後朔加時入轉度及分;其度若滿轉終度及餘去之。



 其入轉度如在中度以下為月行在疾歷;如在中度以上者,乃減去中度及餘,為月入遲歷。



 求月行遲疾差度及定差:置所求月行入遲速度,如在象度以下為在初。以上,覆減中度,餘為在末。其度餘用約分百為
 母。



 置初、末度於上,列二百一度九分於下,以上減下,餘以下乘上,為積數;滿一千九百七十六除為度,不滿,退除為分,命曰遲疾差度。在疾為減,在遲為加。



 以一萬乘積數,滿六千七百七十三半除之,為遲疾定差。疾加、遲減,若用立成者,以其度下損益率乘度餘,滿轉度母而一,所得,隨其損益,即得遲疾及定差。其遲疾、初末損益分為二日者,各加其初、末以乘除。



 求朔弦望所直度下月行定分:置遲疾所入初、末度分,進一位,滿七百三十九除之,用減一百二十
 七,餘為衰差。乃以衰差疾初遲末減、遲初疾末加,皆加減平行度分,為其度所直月行定分。其度以百命為分。



 求朔弦望定日:各以日躔盈縮、月行遲疾定差加減經朔、弦、望小餘,滿若不足,進退大餘,命甲子,算外,各得定日日辰及餘。若定朔幹名與後朔幹名同者月大,不同月小,月內無中氣者為閏月。凡注歷,觀定朔小餘,秋分後四分之三已上者,進一日;若春分後,其定朔晨分差如春分之日者,三約之,以減四分之三;如定朔小餘及此數已上者,進一日;朔或當交有食,初虧在日入已前者,其朔不進。弦、望定小餘不滿日出分者,退一日;其望或當交有食,初虧在日出
 已前,其定望小餘雖滿日出分者,亦退之。又月行九道遲疾,歷有三大二小;日行盈縮累增損之,則有四大三小,理數然也。若循其常,則當察加時早晚,隨其所近而進退之,使月之大小不過連三。舊說,正月朔有交,必須消息前後一兩月,移食在晦、二之日。且日食當朔,月食當望,蓋自然之理。夫日之食,蓋天之垂誡,警悟時政,若道化得中,則變咎為祥。國家務以至公理天下,不可私移晦朔,宜順天誡。故《春秋傳》書日食,乃糾正其朔,不可專移食於晦、二。其正月朔有交,一從近典,不可移避。



 求定朔弦望加時日度:置朔、弦、望中日及約分,以日躔盈縮度及分盈加縮減之,又以元法退除遲疾定差,疾加遲減之,餘為其朔、弦、望加時定日。以天正冬至加時
 黃道日度加而命之,即所求朔、弦、望加時定日所在宿次。朔、望有交,則依後術。



 求月行九道:凡合朔所交,冬在陰歷,夏在陽歷,月行青道。冬至、夏至後,青道半交在春分之宿,當黃道東。立夏、立冬後,青道半交在立春之宿,當黃道東南;至所沖之宿亦如之。



 冬在陽歷,夏在陰歷,月行白道。冬至、夏至後,白道半交在秋分之宿,當黃道西;立冬、立夏後,白道半交在立秋之宿,當黃道西北;至所沖之宿亦如之。



 春在陽歷,秋在陰歷,月行朱道。春分、秋分後,朱道交在夏至之宿,當黃道南;立春、立秋後,朱道半交在立夏之宿,當黃道西南:至所沖之宿亦如之。



 春在陰歷,秋在陽歷,月行黑
 道。春分、秋分後,黑道半交在冬至之宿,當黃道正北。立春、立秋後,黑道半交在立冬之宿,當黃道東北;至所沖之宿亦如之。



 四序離為八節,至陰陽之所交,皆與黃道相會,故月行九道。各視月所入正交積度,視正交九道宿度所入節候,即其道、其節所起。



 滿象度及分去之餘,入交積度及象度並在交會術中。



 若在半象以下為在初限。以上,覆減象度及分,為在末限。用減一百一十一度三十七分,餘以所入初、末限度及分乘之,退位,半之,滿百為度,不滿為分,所得為月行與黃道差數。距半交後、正交前,以差數減;距正交後、半交前,以
 差數加。此加減出入六度,單與黃道相較之數,若較之赤道,隨數遷變不常。



 計去二至以來度數,乘黃道所差,九十而一,為月行與黃道差數。凡日以赤道內為陰,外為陽;月以黃道內為陰,外為陽。故月行宿度,入春分交後行陰歷,秋分交後行陽歷,皆為同名;若入春分交後行陽歷,秋分交後行陰歷,皆為異名。其在同名,以差數加者加之,減者減之;其在異名,以差數加者減之,減者加之。皆加減黃道宿積度,為九道宿積度;以前宿九道宿積度減其宿九道宿積度,餘為
 其宿九道宿度及分。其分就近約為太、半、少三數。



 求月行九道入交度:置其朔加時定日度,以其朔交初度及分減之,餘為其朔加時月行入交度及餘。其餘以一萬乘之,以元法退除之,即為約餘。



 以天正冬至加時黃道日度加而命之,即正交月離所在黃道宿度。



 求正交加時月離九道宿度:以正交度及分減一百一十一度三十七分,餘以正交度及分乘之,退一等,半之,滿百為度,不滿為分,所得,命曰定差。以定差加黃道宿
 度,計去冬、夏至以來度數,乘定差,九十而一,所得,依同異名加減之,滿若不足,進退其度,命如前,即正交加時月離九道宿度及分。



 求定朔弦望加時月離所在宿度:各置其日加時日躔所在,變從九道,循次相加。凡合朔加時,月行潛在日下,與太陽同度,是為加時月離宿次。先置朔、弦、望加時黃道宿度,以正交加時黃道宿度減之,餘以加其正交加時九道宿度,命起正交宿次,算外,即朔、弦、望加時所當九道宿度。其合朔加時若非正近,則日在黃道、月在九道各入宿度,雖多少不同,考其去極,若應繩準。故云月行潛在日下,與太陽
 同度。



 各以弦、望度及分加其所當九道宿度,滿宿次去之,各得加時九道月離宿次。



 求定朔夜半入轉:以所求經朔小餘減其朔加時入轉日餘,其經朔小餘,以二萬七千八百七乘之,即母轉法。



 為其經朔夜半入轉。若定朔大餘有進退者,亦進退轉日,無進退則因經為定。其餘以轉法退收之,即為約分。



 求次月定朔夜半入轉:因定朔夜半入轉,大月加二日,小月加一日,餘、分皆加四千四百五十四,滿轉終日及
 約分去之,即次月定朔夜半入轉;累加一日,去命如前,各得逐日夜半入轉日及分。



 求定朔弦望夜半月度:各置加時小餘,若非朔、望有交者,有用定朔、弦、望小餘。



 以其日月行定分乘之,滿元法而一為度,不滿,退除為分,命曰加時度。以減其日加時月度,即各得所求夜半月度。



 求晨昏月:以晨分乘其日月行定分,元法而一,為晨度;用減月行定分,餘為昏度。各以晨昏度加夜半月度,即
 所求晨昏月所在宿度。



 求朔弦望晨昏定程:各以其朔昏定月減上弦昏定月,餘為朔後昏定程;以上弦昏定月減望昏定月,餘為上弦後昏定程;以望晨定月減下弦晨定月,餘為望後晨定程;以下弦晨定月減次朔晨定月,餘為下弦後晨定程。



 求轉積度:計四七日月行定分,以日衰加減之,為逐日月行定程;乃自所入日計求定之,為其程轉積度分。其四
 七日月行定分者,初日益遲一千二百一十,七日漸疾一千三百四十一,十四日損疾一千四百六十一,二十一日漸遲一千三百二十八,乃觀其遲疾之極差而損益之,以百為分母。



 求每日晨昏月:以轉積度與晨昏定程相減,餘以距後程日數除之,為日差。定程多為加,定程少為減。



 以加減每日月行定分,為每日轉定度及分。以每日轉定度及分加朔、弦、望晨昏月,滿九道宿次去之,即為每日晨、昏月離所在宿度及分。凡注歷,朔後注昏,望後注晨。



 已前月度,並依九道所推,以究算術之精微。若注歷求其速要者,即依後術以推黃道
 月度。



 求天正十一月定朔夜半平行月:以天正經朔小餘乘平行度分,元法而一為度,不滿,退除為分秒,所得,為經朔加時度。用減其朔中日,即經朔晨前夜半平行月積度。若定朔有進退,以平行度分加減之。



 即為天正十一月定朔之日晨前夜半平行月積度及分。



 求次月定朔之日夜半平行月:置天正定朔之日夜半平行月,大月加三十五度八十分六十一秒,小月加二
 十二度四十三分七十三秒半,滿周天度分即去之,即每月定朔之晨前夜半平行月積度及分秒。



 求定弦望夜半平行月、計弦、望距定朔日數,以乘平行度及分秒,以加其定朔夜半平行月積度及分秒,即定弦、望之日夜半平行月積度及分秒。亦可直求朔望,不復求度,從簡易也。



 求天正定朔夜半入轉度:置天正經朔小餘,以平行月度及分乘之,滿元法除為度,不滿,退除為分秒,命為加時度;以減天正十一月經朔加時入轉度及約分,餘為
 天正十一月經朔夜半入轉度及分。若定朔大餘有進退者,亦進退平行度分,即為天正十一月定朔之日晨前夜半入轉度及分秒。



 求次月定朔及弦望夜半入轉度:因天正十一月定朔夜半入轉度分,大月加三十二度六十九分一十七秒,小月加十九度三十二分二十九秒半,即各得次月定朔夜半入轉度及分。各以朔、弦、望相距日數乘平行度分以加之,滿轉終度及秒即去之,如在中度以下者為
 在疾;以上者去之,餘為入遲歷,即各得次朔、弦、望定日晨前夜半入轉度及分。若以平行月度及分收之,即為定朔、弦、望入轉日。



 求定朔弦望夜半定月:以定朔、弦、望夜半入轉度分乘其度損益衰,以一萬約之為分,百約之為秒,損益其度下遲疾度,為遲疾定度。乃以遲加疾減夜半平行月,為朔、弦、望夜半定月積度。以冬至加時黃道日度加而命之,即定朔、弦、望夜半月離所在宿次。若有求晨昏月,以其日晨昏分乘其日月行定分,元法而一,所得為晨昏度;以加其夜半定月,即得朔、弦、望晨昏月度。



 求朔弦望定程:各以朔、弦、望定月相減,餘為定程。若求晨昏定程,則用晨昏定月相減,朔後用昏,望後用晨。



 求朔弦望轉積度分:計四七日月行定分,以日衰加減之,為逐日月行定分;乃自所入日計之,為其程轉積度分。其四七日月行定分者,初日益遲一千二百一十,七日漸疾一千三百四十一,十四日損疾一千四百六十一,二十一日漸遲一千三百二十八,乃視其遲疾之極差而損益之,分以百為母。



 求每日月離宿次:各以其朔、弦、望定程與轉積度相減,餘為程差。以距後程日數除之,為日差。
 定程多為益差,定程少為損差。



 以日差加減月行定分。為每日月行定分;以每日月行定分累加定朔、弦、望夜半月在宿次,命之,即每日晨前夜半月離宿次。如晨昏宿次,即得每日晨昏月度。



 步交會術



 交度母:六百二十四萬。



 周天分:二十二億七千九百二十萬四百四十七。



 朔差:九百九十萬一千一百五十九。



 朔差:一度、餘三百六十六萬一千一百五十九。



 望差:空度、餘四百九十五萬五百七十九半。



 半周天:一百八十二度。餘三百九十二萬二百二十三半,約分六千二百八十二。



 日食限:一千四百六十四。



 月食限:一千三百三十八。



 盈初限縮末限:六十度八十七分半。



 縮初限盈末限:一百二十一度七十五分。



 求交初度:置所求積月,以朔差乘之,滿周天分去之,不盡,覆減周天分,滿交度母除之為度,不滿為餘,即得所
 求月交初度及餘;以半周天加之,滿周天去之,餘為交中度及餘。若以望差減之,即得其月望交初度及餘;以朔差減之,即得次月交初度及餘;以交度母退除,即得餘分。若以天正黃道日度加而命之,即各得交初、中所在宿度及分。



 求日月食甚小餘及加時辰刻:以其朔、望月行遲疾定差疾加遲減經朔望小餘,若不足減者,退大餘一,加元法以減之;若加之滿元法者,但積其數。



 以一千三百三十七乘之,滿其度所直月行定分除之,為月行差數;乃以日躔盈定差盈加縮減之,餘為其朔、望食甚小餘。凡加減滿若不足,進退其日,此朔望加時以究月行遲疾之數,若非有交
 會,直以經定小餘為定。



 置之,如前發斂加時術入之,即各得日、月食甚所在晨刻。視食甚小餘,如半法以下者,覆減半法,餘為午前分;半法已上者,減去半法,餘為午後分。



 求朔望加時日月度:以其朔、望加時小餘與經朔望小餘相減,餘以元法退收之,以加減其朔、望中日及約分,經朔望少,加;經朔望多,減。



 為其朔、望加時中日。乃以所入日升降分乘所入日約分,以一萬約之,所得,隨以損益其日下盈縮積,為盈縮定度;以盈加縮減加時中日,為其朔、望加
 時定日;望則更加半周天,為加時定月;以天正冬至加時黃道日度加而命之,即得所求朔、望加時日月所在宿度及分。



 求朔望日月加時去交度分:置朔望日月加時定度與交初、交中度相減,餘為去交度分。就近者相減之,其度以百通之為分。



 加時度多為後,少為前,即得其朔望去交前、後分。交初後、交中前,為月行外道陽歷;交中後、交初前,為月行內道陰歷。



 求日食四正食差定數:置其朔加時定日,如半周天以
 下者為在盈。以上者去之,餘為在縮。視之,如在初限以下者為在初。以上者,覆減二至限,餘為在末。置初、末限度及分,盈初限、縮末限者倍之。



 置於上位,列二百四十三度半於下,以上減下,餘以下乘上,以一百六乘之,滿三千九十三除之,為東西食差泛數。用減五百八,餘為南北食差泛數。其求南北食差定數者,乃視午前、後分,如四分法之一以下者覆減之,餘以乘泛數。若以上者即去之,餘以乘泛數,皆滿九千七百五十除之,為南北食差定數。
 盈初縮末限者,食甚在卯酉以南,內減外加;食甚在卯酉以北,內加外減。



 縮初盈末限者,食甚在卯酉以南,內加外減;食甚在卯酉以北,內減外加。



 其求東西食差定數者,乃視午前、後分,如四分法之一以下者以乘泛數;以上者,覆減半法,餘乘泛數,皆滿九千七百五十除之,為東西食差定數。盈初縮末限者,食甚在子午以東,內減外加;食甚在子午以西,內加外減。



 縮初盈末限者,食甚在子午以東,內加外減;食甚在子午以西,內減外加。



 即得其朔四正食差加減定數。



 求日月食去交定分:視其朔四正食差,加減定數,同名
 相從,異名相消,餘為食差加減總數;以加減去交分,餘為日食去交定分。其去交定分不足減、乃覆減食差總數、若陽歷覆減入陰歷,為入食限;若陰歷覆減入陽歷,為不入食限。凡加之滿食限以上者,亦不入食限。



 其望食者,以其望去交分便為其望月食去交定分。



 求日月食分:日食者,視去交定分,如食限三之一以下者倍之,類同陽歷食分。以上者,覆減食限,餘為陰歷食分。皆進一位,滿九百七十六除為大分,不滿,退除為小分,命十為限,即日食之大、小分。月食者,視去交定分,如
 食限三之一以下者,食既;以上者,覆減食限。餘進一位,滿八百九十二除之為大分,不滿,退除為小分,命十為限,即月食之大、小分。其食不滿大分者,雖交而數淺,或不見食也。



 求日食泛用刻分:置陰、陽歷食分於上,列一千九百五十二於下,以上減下,餘以乘上,滿二百七十一除之,為日食泛用刻、分。



 求月食泛用刻分:置去交定分,自相乘,交初以四百五十九除,交中以五百四十除之,所得,交初以減三千九
 百,交中以減三千三百一十五,餘為月食泛用刻、分。



 求日月食定用刻分:置日月食泛用刻、分,以一千三百三十七乘之,以所直度下月行定分除之,所得為日月食定用刻、分。



 求日月食虧初復滿時刻:以定用刻分減食甚小餘,為虧初小餘;加食甚,為復滿小餘;各滿辰法為辰數,不盡,滿刻法除之為刻數,不滿為分。命辰數從子正,算外,即得虧初、復末辰、刻及分。若以半辰數加之,即命從時初也。



 求日月食初虧復滿方位:其日食在陽歷者,初食西南,甚於正南,復於東南;日在陰歷者,初食西北,甚於正北,復於東北。其食過八分者,皆初食正西,復於正東。其月食者,月在陰歷,初食東南,甚於正南,復於西南;月在陽歷,初食東北,甚於正北,復於西北。其食八分已上者,皆初食正東,復於正西。此皆審其食甚所向,據午正而論之,其食餘方察其斜正,則初虧、復滿乃可知矣。



 求月食更點定法:倍其望晨分,五而一,為更法;又五而
 一,為點法。若依司辰星注歷,同內中更點,則倍晨分,減去待旦十刻之分,餘,五而一,為更法;又五而一,為點法。



 求月食入更點:各置初虧、食甚、復滿小餘,如在晨分以下者加晨分,如在昏分以上者減去昏分,餘以更法除之為更數,不滿,以點法除之為點數。其更數命初更,算外,即各得所入更、點。



 求月食既內外刻分:置月食去交分,覆減食限三之一,不及減者為食不既。



 餘列於上位;乃列三之二於下,以上減下,餘
 以下乘上,以一百七十除之,所得,以定用刻分乘之,滿泛用刻分除之,為月食既內刻分;用減定用刻分,餘為既外刻、分。



 求日月帶食出入所見分數:視食甚小餘在日出分以下者,為月見食甚、日不見食甚;以日出分減復滿小餘,若食甚小餘在日出分已上者,為日見食甚、月不見食甚;以初虧小餘減日出分,各為帶食差;若月食既者,以既內刻分減帶食差,餘乘所食分,既外刻分而一,不及減者,即帶食既出入也。



 以乘所食之分,滿定用
 刻分而一,即各為日帶食出、月帶食入所見之分。凡虧初小餘多如日出分為在晝,復滿小餘多如日出分為在夜,不帶食出入也。



 若食甚小餘在日入分以下者,為日見食甚、月不見食甚;以日入分減復滿小餘,若食甚小餘在日入分已上者,為月見食甚、日不見食甚;以初虧小餘減日入分,各為帶食差;若月食既者,以既內刻分減帶食差,餘乘所差分,既外刻分而一,不及減者,即帶食既出入也。



 以乘所食之分,滿定用刻分而一,即各為日帶食入、月帶食出所見之分。凡虧初小餘多如日入分為在夜,復滿小餘少如日入分為在晝,並不帶食出入也。



 步五星術



 木星終率:一千五百五十五萬六千五百四。



 終日:三百九十八日。餘三萬四千五百四,約分八千八百四十七。



 歷差:六萬一千七百五十。



 見伏常度:一十四度。



 火星終率:
 三千四十一
 萬七千五百三十六。



 終日:七百七十九日。餘三萬六千五百三十六,約分九千三百六十八。



 歷差:六萬一千二百四十。



 見伏常度:
 一十八度。



 土星終率:一千四百七十四萬五千四百四十六。



 終日:三百七十八。餘三千四百四十六,約分八百八十三。



 歷差:六萬
 一千三百五十。



 見伏常度:一十八
 度半。



 金星終率:二千二百七十七萬二千一百九十六。



 終日:五百八十三日。餘三萬五千一百九十六,約分九千二十四。



 見伏常度:一十一度少。



 水星終率:四百五十一萬九千一百八十四。改九千一百九十四。



 終日:一百一十五日。餘三萬四千一百八十四,約分八千七百六十五。



 見伏常度:一十八度。



 求五星天正冬至後諸段中積中星:置氣積分,各以其星終率去之,不盡,覆減終率,餘滿元法為日,不滿,退除為分,即天正冬至後其星平合中積。重列之為中星,因命為前一段之初,以諸段變日、變度累加減之,即為諸段中星。變
 日加減中積,變度加減中星。



 求木火土三星入歷:以其星歷差乘積年,滿周天分去之,不盡,以度母除之為度,不滿,退除為分,命曰差度;以減其星平合中星,即為平合入歷度分;以其星其段歷度加之,滿周天度分即去之,各得其星其段入歷度分。金、水附日而行,更不求歷差。其木、火、土三星前變為晨,
 後變為
 夕。
 金、水二星前變為夕,後變為晨。



 求木火土三星諸段盈縮定差:木、土二星,置其星其段入歷度分,如半周天以下者為在盈。以上者,減去半周天,餘為在縮。置盈縮度分,如在一象以下者為在初限。以上者,覆減半周天,餘為在末限。置初、末限度及分於上,列半周天於下,以上減下,以下乘上,木進一位,土九因之。



 皆滿百為分,分滿百為度,命曰盈縮定差。其火星,置盈縮度
 分,如在初限以下者為在初。以上者,覆減半周天,餘為在末。以四十五度六十五分半為盈初、縮末限度,以一百三十六度九十六分半為縮初、盈末限度分。



 置初、末限度於上,盈初、縮末三因之。



 列二百七十三度九十三分於下,以上減下,餘以下乘上,以一十二乘之,滿百為度,不滿,百約為分,命曰盈縮定差。若用立成法,以其度下損益率乘度下約分,滿百者,以損益其度下盈縮差度為盈縮定差,若在留退段者,即在盈縮泛差。



 求木火土三星留退差:置後退、後留盈縮泛差,各列其星盈縮極度於下,木極度,八度三十三分;火極度,二十二度五十一分;土極度,七度五十分。



 以上減下,餘以下乘上,水、土三因之,火倍之。



 皆滿百為度,命曰留退差。後退初半之,後留全用。



 其留退差,在盈益減損加、在縮損減益加其段盈縮泛差,為後退、後留定差。因為後遲初段定差,各須類會前留定差,觀其盈縮,察其降差也。



 求五星諸段定積:各置其星其段中積,以其段盈縮定差盈加縮減之,即其星其段定積及分;以天正冬至大餘及約分加之,滿紀法去之,不盡,命甲子,算外,即得日辰。其五星合見、伏,即為推算段定日;後求見、伏合定日,即歷注其日。



 求五星諸段所在月日:各置諸段定積,以天正閏日及約分加之,滿朔策及分去之,為月數;不滿,為入月以來日數及分。其月數命從天正十一月,算外,即其星其段入其月經朔日數及分。定朔有進退者,亦進退其日,以日辰為定。若以氣策及約分去定積,命從冬至,算外,即得其段入氣日及分。



 求五星諸段加時定星:各置其星其段中星,以其段盈縮定差盈加縮減之,即五星諸段定星。若以天正冬至加時黃道日度加而命之,即其段加時定星所在宿次。
 五星皆以前留為前退初定星,後留為後順初定星。



 求五星諸段初日晨前夜半定星:木、火、土三星,以其星其段盈縮定差與次度下盈縮定差相減,餘為其度損益差;以乘其段初行率,一百約之,所得,以加減其段初行率,在盈,益加損減;在縮,益減損加。



 以一百乘之,為初行積分;又置一百分,亦依其數加減之,以除初行積分,為初日定行分。以乘其段初日約分,以一百約之,順減退加其段定星,為其段初日晨前夜半定星;以天正冬至加時黃道日
 度加而命之,即得所求。金、水二星,直以初行率便為初日定行分。



 求太陽盈縮度:各置其段定積,如二至限以下為在盈;以上者去之,餘為在縮。又視入盈縮度,如一象以下者為在初;以上者,覆減二至限,餘為在末。置初、末限度及分,如前日度術求之,即得所求。若用立成者,直以其度下損益分乘度餘,百約之,所得,損益其度下盈縮差,亦得所求。



 求諸段日度率:以二段日晨相距為日率,又以二段夜半定星相減,餘為其段度率及分。



 求諸段平行分:各置其段度率及分,以其段日率除之,為其段平行分。



 求諸段泛差:各以其段平行分與後段平行分相減,餘為泛差;並前段泛差,四因之,退一等,為其段總差。五星前留前、後留後一段,皆以六因平行分,退一等,為其段總差,水星為半總差。其在退行者,木、火、土以十二乘其段平行分,退一等,為其段總差。金星退行者,以其段泛差為總差,後變則反用初、末。水星退行者,以其段平行分為總差,若在前後順第一段者,乃半次段總差,為其段總差。



 求諸段初末日行分:各半其段總差,加減其段平行分,
 為其段初、末日行分。前變加為初,減為末;後變減為初,加為末。其在退段者,前則減為初,加為末;後則加為初,減為末。若前後段行分多少不倫者,乃平注之;或總差不備大分者,亦平注之:皆類會前後初、末,不可失其衰殺。



 求諸段日差:減其段日率一,以除其段總差,為其段日差。後行分少為損,後行分多為益。



 求每日晨前夜半星行宿次:置其段初日行分,以日差累損益之,為每日行分。以每日行分累加減其段初日晨前夜半宿次,命之,即每日星行宿次。



 徑求其日宿次:置所求日,減一,以乘日差,以加減初日行分,後少,減之;後多,加之。



 為所求日行分;乃加初日行分而半之,以所求日數乘之,為徑求積度;以加減其段初日宿次,命之,即徑求其日星宿次。



 求五星定合定日:木、火、土三星,以其段初日行分減一百分,餘以除其日太陽盈縮餘為日,不滿,退除為分,命曰距合差日及分。以差日及分減太陽盈縮分,餘為距合差度。以差日、差度盈減縮加。金、水二星平合者,以百
 分減初日行分,餘以除其日太陽盈縮餘為日,不滿,退除為分,命曰距合差日及分。以減太陽盈縮分,餘為距合差度。以差日、差度盈加縮減。金、水星再合者,以初日行分加一百分,以除其日太陽盈縮分為日,不滿,退除為分,命曰再合差日;以減太陽盈縮分,餘為再合差度。以差日、差度盈加縮減。差度則反其加減。



 皆以加減定積,為再合定日。以天正冬至大餘及約分加而命之,即得定合日辰。



 求五星定見伏:木、火、土三星,各以其段初日行分減一百分,餘以除其日太陽盈縮分為日,不滿,退除為分,以盈減縮加。金、水二星夕見、晨伏者,以一百分減初日行分,餘以除其日太陽盈縮分為日,不滿,退除為分,以盈加縮減。其在晨見、夕伏者,以一百分加其段初日行分,以除其日太陽盈縮分為日,不滿,退除為分,以盈減縮加。皆加減其段定積,為見、伏定日。以加冬至大餘及約分,滿紀法去之,命從甲子,算外,即得五星見、伏定日日
 辰。



 琮又論歷曰:「古今之歷,必有術過於前人,而可以為萬世之法者,乃為勝也。若一行為《大衍歷》,議及略例,校正歷世,以求歷法強弱,為歷家體要,得中平之數。劉焯悟日行有盈縮之差。舊歷推日行平行一度,至此方悟日行有盈縮,冬至前後定日八十八日八十九分,夏至前後定日九十三日七十四分,冬至前後日行一度有餘,夏至前後日行不及一度。



 李淳風悟定朔之法,並氣朔、閏餘,皆同一術。舊歷定朔平注一大一小,至此以日行盈縮、月行遲疾加減朔餘,餘為定朔、望加時,以定大小,不過三數。自此後日食在朔,月食在望,更無晦、
 二之差。舊歷皆須用章歲、章月之數,使閏餘有差,淳風造《麟德歷》,以氣朔、閏餘同歸一母。



 張子信悟月行有交道表裏,五星有入氣加減。北齊學士張子信因葛榮亂,隱居海島三十餘年,專以圓儀揆測天道,始悟月行有交道表裏,在表為外道陽歷,在里為內道陰歷。月行在內道,則日有食之,月行在外道則無食。若月外之人北戶向日之地,則反觀有食。又舊歷五星率無盈縮,至是始悟五星皆有盈縮、加減之數。



 宋何承天始悟測景以定氣序。景極長,冬至;景極短,夏至。始立八尺之表,連測十餘年,即知舊《景初歷》冬至常遲天三日。乃造《元嘉歷》,冬至加時比舊退減三日。



 晉姜岌始悟以月食所沖之宿,為日所在之度。日所在不知宿度,至此以月食之宿所沖,為日所在宿度。



 後漢劉洪作《乾象歷》,始悟月行有
 遲疾數。舊歷,月平行十三度十九分度之七,至是始悟月行有遲疾之差,極遲則日行十二度強,極疾則日行十四度太,其遲疾極差五度有餘。



 宋祖沖之始悟歲差。《書·堯典》曰:「日短星昴,以正仲冬;宵中星虛,以殷仲秋。」至今三千餘年,中星所差三十餘度,則知每歲有漸差之數,造《大明歷》率四十五年九月而退差一度。



 唐徐升作《宣明歷》,悟日食有氣、刻差數。舊歷推日食皆平求食分,多不允合,至是推日食,以氣刻差數增損之,測日食分數,稍近天驗。



 《明天歷》悟日月會合為朔,所立日法,積年有自然之數,及立法推求晷景,知氣節加時所在。自《元嘉歷》後所立日法,以四十九分之二十六為強率、以十七分之九為弱率,並強弱之數為日法、朔餘,自後諸歷效之。殊不知日月會合為朔,並朔餘虛分為日法,蓋自然
 之理。其氣節加時,晉、漢以來約而要取,有差半日,今立法推求,得盡其數。



 後之造歷者,莫不遵用焉。其疏謬之甚者,即苗守信之《乾元歷》、馬重績之《調元歷》、郭紹之《五紀歷》也。大概無出於此矣。然造歷者,皆須會日月之行,以為晦朔之數,驗《春秋》日食,以明強弱。其於氣序,則取驗於《傳》之南至。其日行盈縮、月行遲疾、五星加減、二曜食差、日宿月離、中星晷景、立數立法,悉本之於前語。然後較驗,上自夏仲康五年九月「辰弗集於房」,以至於今,其星辰氣朔、日月交食等,使三千年
 間若應準繩。而有前有後、有親有疏者,即為中平之數,乃可施於後世。其較驗則依一行、孫思恭,取數多而不以少,得為親密。較日月交食,若一分二刻以下為親,二分四刻以下為近,三分五刻以上為遠。以歷注有食而天驗無食,或天驗有食而歷注無食者為失。其較星度,則以差天二度以下為親,三度以下為近,四度以上為遠;其較晷景尺寸,以二分以下為親,三分以下為近,四分以上為遠。若較古而得數多,又近於今,兼立法、立數,
 得其理而通於本者為最也。」琮自謂善歷,嘗曰:「世之知歷者甚少,近世獨孫思恭為妙。」而思恭又嘗推劉羲叟
 為知
 歷焉。



\end{pinyinscope}