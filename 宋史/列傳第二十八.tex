\article{列傳第二十八}

\begin{pinyinscope}

 陶
 穀扈蒙王著王祐子旭孫質楊昭儉魚崇諒張澹高錫從子冕



 陶穀,字秀實,邠州新平人。本姓唐,避晉祖諱改焉。歷北齊、隋、唐為名族。祖彥謙,歷慈、絳、澧三州刺史,有詩名,
 自號鹿門先生。父渙,領夷州刺史,唐季之亂,為邠帥楊崇本所害。時穀尚幼,隨母柳氏育崇本家。



 十餘歲,能屬文,起家校書郎、單州軍事判官。嘗以書乾宰相李崧,崧甚重其文。時和凝亦為相,同奏為著作佐郎、集賢校理。改監察御史,分司西京,遷虞部員外郎、知制誥。會晉祖廢翰林學士,兼掌內外制。詞目繁委,穀言多委愜,為當時最。少帝初,賜緋袍、靴、笏、黑銀帶。天福九年,加倉部郎中。



 初,崧從契丹以北,高祖入京師,以崧第賜蘇逢吉,而崧
 別有田宅在西京,逢吉皆取之。崧自北還,因以宅券獻逢吉,逢吉不悅,而崧子弟數出怨言。其後逢吉乃誘告崧與弟嶼、㠖等下獄,崧懼,移病不出。崧族子昉為秘書郎,嘗往候崧,崧語昉曰:「邇來朝廷於我有何議?」昉曰:「無他聞,唯陶給事往往於稠人中厚誣叔父。」崧嘆曰:「穀自單州判官,吾取為集賢校理,不數年擢掌誥命,吾何負於陶氏子哉?」及崧遇禍,昉嘗因公事詣谷,穀問昉:「識李侍中否?」昉斂衽應曰:「遠從叔爾。」穀曰:「李氏之禍,穀出力
 焉。」昉聞之汗出。



 穀性急率,嘗與兗帥安審信集會,杯酒相失,為審信所奏。時方姑息武臣,穀坐責授太常少卿。嘗上言:「頃蒞西臺,每見臺司詳斷刑獄,少有實時決者。至於閭閻夫婦小有爭訟,淹滯積時,坊市死亡喪葬,必俟臺司判狀,奴婢病亡,亦須檢驗。吏因緣為奸,而邀求不已,經旬不獲埋瘞。望申條約以革其弊。」從之。俄拜中書舍人。嘗請教習樂工、停二舞郎,及禁民伐桑棗為薪,並從其請。開運三年,賜金紫。



 契丹主北歸,脅穀令從行。穀
 逃匿僧舍中,衣布褐,陽為行者狀。軍士意其詐,持刃陵脅者日數四。穀頗工歷數,謂同輩曰:「西南五星連珠,漢地當有王者出。契丹主必不得歸國。」及耶律德光死,有孛光芒指北,穀曰:「自此契丹自相魚肉,永不亂華矣。」遂歸漢,為給事中。乾祐中,令常參官轉對。谷上言曰:「五日上章,曾非舊制。百官敘對,且異昌言。徒浼天聰,無益時政,欲乞停轉對。在朝群臣有所聞見,即許不時詣闕聞奏。」從之。



 仕周為右散騎常侍,世宗即位,遷戶部侍郎。
 從征太原,時魚崇諒迎母後至,穀乘間言曰:「崇諒宿留不來,有顧望意。」世宗頗疑之。崇諒又表陳母病,詔許歸陜州就養,以穀為翰林學士。



 世宗嘗謂宰相曰:「朕觀歷代君臣治平之道,誠為不易。又念唐、晉失德之後,亂臣黠將,僭竊者多。今中原甫定,吳、蜀、幽、並尚未平附,聲教未能遠被,宜令近臣各為論策,宣導經濟之略。」乃命承旨徐臺符以下二十餘人,各撰《為君難為臣不易論》、《平邊策》以進。其策率以修文德、來遠人為意,惟穀與竇儀、楊
 昭儉、王樸以封疆密邇江、淮,當用師取之。世宗自克高平,常訓兵講武,思混一天下。及覽其策,忻然聽納,由是平南之意益堅矣。



 顯德三年,遷兵部侍郎,加承旨。世宗留心稼穡,命工刻木為耕夫、織婦、蠶女之狀,置於禁中,思廣勸課之道,穀為贊辭以進。顯德六年,加吏部侍郎。



 宋初,轉禮部尚書,依前翰林承旨。穀在翰林,與竇儀不協,儀有公望,慮其軋己,嘗附宰相趙普與趙逢、高錫輩共排儀,儀終不至相位。



 乾德二年,判吏部銓兼知貢舉。
 再為南郊禮儀使,法物制度,多穀所定。時範質為大禮使,以鹵簿清游隊有甲騎具裝,莫知其制度,以問於穀。穀曰:「梁貞明丁丑歲,河南尹張全義獻人甲三百副、馬具裝二百副。其人甲以布為里,黃泬表之,青綠畫為甲文,紅錦綠青泬為下裙,絳韋為絡,金銅玦,長短至膝。前膺為人面二目,背連膺纏以紅錦騰蛇。馬具裝蓋尋常馬甲,但加珂拂於前膺及後秋爾。莊宗入洛,悉焚毀。」質命有司如穀說,造以給用。又乘輿大輦,久亡其制,穀創
 意造之,後承用焉。明德門成,詔穀為之記。



 乾德中,命庫部員外郎王貽孫、《周易》博士奚嶼同考試品官子弟。谷屬其子鄑於嶼,珝書不通,以合格聞,補殿中省進馬。俄為人所發,下御史府案問,嶼責授乾州司戶,貽孫責授左贊善大夫,奪穀奉兩月。谷後累加刑部、戶部二尚書。開寶三年,卒,年六十八。贈右僕射。



 穀強記嗜學,博通經史,諸子佛老,咸所總覽;多蓄法書名畫,善隸書。為人雋辨宏博,然奔競務進,見後學有文採者,必極言以譽之;
 聞達官有聞望者,則巧詆以排之,其多忌好名類此。初,太祖將受禪,未有禪文,穀在旁,出諸懷中而進之曰:「已成矣。」太祖甚薄之。嘗自曰:「吾頭骨法相非常,當戴貂蟬冠爾。」蓋有意大用也,人多笑之。子邴,至起居舍人。天禧四年,錄谷孫寔試秘書省校書郎。



 扈蒙,字日用,幽州安次人。曾祖洋,涿州別駕。祖智周,盧龍軍節度推官。父曾,內園使。蒙少能文,晉天福中,舉進士,入漢為鄠縣主簿。趙思綰叛,遣郭從義討之。郡縣
 吏供給皆戎服趨事,蒙寇服褒博,舉止舒緩,從義頗訝之。轉運使李谷謂曰:「蒙文學名流,不習吏事。」遂不之問。周廣順中,從歸德軍節度趙暉為掌書記,召為右拾遺、直史館、知制誥。蒙從弟載時為翰林學士,兄弟並掌內外制,時號「二扈」。



 宋初,由中書舍人遷翰林學士,坐請托於同年仇華,黜為太子左贊善大夫,稍遷左補闕,掌大名市征。六年,復知制誥,充史館修撰。開寶中,受詔與李穆等同修《五代史》,詳定《古今本草》。五年,連知貢舉。



 七年,蒙
 上書言:「昔唐文宗每召大臣論事,必命起居郎、起居舍人執筆立於殿側,以紀時政,故《文宗實錄》稍為詳備。至後唐明宗,亦命端明殿學士及樞密直學士輪修日歷,送史官。近來此事都廢,每季雖有內殿日歷,樞密院錄送史館,然所記者不過臣下對見辭謝而已。帝王言動,莫得而書。緣宰相以漏洩為虞,昧於宣播,史官疏遠,何得與聞。望自今凡有裁制之官,優恤之言,發自宸衷、可書簡策者,並委宰臣及參知政事每月輪知抄錄,以備
 史官撰集。」從之,即以參知政事盧多遜典其事。



 九年正月,受朝乾元殿,降王在列,聲明大備。蒙上《聖功頌》,以述太祖受禪、平一天下之功,其詞誇麗,有詔褒之。為盧多遜所惡,出知江陵府。



 太宗即位,召拜中書舍人,旋復翰林學士。與李昉同修《太祖實錄》。太平興國四年,從征太原還,轉戶部侍郎,加承旨。雍熙三年,被疾,以工部尚書致仕。未幾,卒,年七十二。贈右僕射。



 自張昭、竇儀卒,典章儀注,多蒙所刊定。初,太祖受周禪,追尊四廟,親郊,以宣
 祖配天。及太宗即位,禮官以為舜郊嚳,商郊冥,周郊後稷,王業所因興也。若漢高之太公,光武之南頓君,雖有帝父之尊,而無預配天之祭。故自太平興國三年、六年再郊,並以太祖配,於禮為允。太宗將東封,蒙定議曰:「嚴父莫大於配天,請以宣祖配天。」自雍熙元年罷封禪為郊祀,遂行其禮,識者非之。



 蒙性沉厚,不言人是非,好釋典,不喜殺,縉紳稱善人。有笑疾,雖上前不自禁。多著述,有《鰲山集》二十卷行於世。載字仲熙,有傳,見《五代史》。



 王著,字成象,單州單父人。性豁達,無城府。幼能屬文,漢乾祐中,舉進士。周祖鎮大名,世宗侍行,聞著名,召置門下,因得謁見周祖。廣順中,世宗鎮澶州,闢觀察支使。隨世宗入朝,遷殿中丞;即位,拜度支員外郎。顯德三年,充翰林學士。六年,丁家艱,起復。南唐李景使其弟從善來貢,會恭帝嗣位,命著伴送至睢陽,加金部郎中、知制誥,賜金紫。世宗靈駕赴慶陵,符後從行,公務悉資於著。



 宋初,加中書舍人。建隆二年,知貢舉。時亳州獻紫芝,鄆州獲
 白兔,隴州貢黃鸚鵡,著獻頌,因以規諫。太祖甚嘉其意,下詔褒之。四年春,宿直禁中,被酒,發倒垂被面,夜扣滋德殿門求見。帝怒,發其醉宿倡家之過,黜為比部員外郎。乾德初,改兵部員外郎。二年,復知制誥。數月,加史館修撰、判館事。三年,就轉戶部郎中。六年,復為翰林學士,加兵部郎中,再知貢舉。開寶二年冬,暴卒,年四十二。



 著少有俊才,世宗以幕府舊僚,眷待尤厚,常召見與語,命皇子出拜,每呼學士而不名。屢欲相之,以其嗜酒,故遲
 留久之。及世宗疾大漸,太祖與範質入受顧命,謂質等曰:「王著藩邸舊人,我若不諱,當命為相。」世宗崩乃止。著善與人交,好延譽後進,當世士大夫稱之。有傳,見《五代史》。



 王祐,字景叔,大名莘人。祖言,仕唐黎陽令。父徹,舉後唐進士,至左拾遺。



 祐少篤志詞學,性倜儻有俊氣。晉天福中,以書見桑維翰,稱其藻麗,由是名聞京師。鄴帥杜重威闢為觀察支使。漢初,重威移鎮睢陽,反側不自安,祐
 嘗勸之,使無反漢,不聽。祐坐是貶沁州司戶參軍,因作書貽鄉友以見志,辭氣俊邁,人多稱之。仕周,歷魏縣、南樂二令。



 太祖受禪,拜監察御史,由魏縣移知光州,遷殿中侍御史。乾德三年,知制誥。六年,加集賢院修撰,轉戶部員外郎。



 太祖征太原,已濟河。諸州饋集上黨城中,車乘塞路,上聞之,將以稽留罪轉運使。趙普曰:「六師方至,而轉運使以獲罪聞,敵必謂儲峙不充,有以窺我矣,非威遠之道也。俾能治劇者,往蒞其州足矣。」即命祐知潞
 州。及至,饋餉無乏,路亦無壅,班師,召還。



 會符彥卿鎮大名,頗不治,太祖以祐代之,俾察彥卿動靜,謂曰:「此卿故鄉,所謂畫錦者也。」祐以百口明彥卿無罪,且曰:「五代之君,多因猜忌殺無辜,故享國不永,願陛下以為戒。」彥卿由是獲免,故世謂祐有陰德。



 繼以用兵嶺表,徙知襄州。湖湘平,移知潭州。召還,攝判吏部銓。時左司員外郎侯陟自揚州還,復判銓,祐判門下省,陟所注擬,祐多駁正。盧多遜與陟善,陟因訴之,多遜素惡祐不比己,遂出祐
 為鎮國軍行軍司馬。



 太平興國初,移知河中府。入為左司員外郎,拜中書舍人,充史館修撰。未幾,知開封府,以病請告。太宗謂祐文章、清節兼著,特拜兵部侍郎。月餘卒,年六十四。



 初,祐掌誥,會盧多遜為學士,陰傾趙普,多遜累諷祜比己,祜不從。一日,以宇文融排張說事勸釋之,多遜滋不悅。及普再入,多遜果敗,與宇文融事頗類,識者服其先見。



 祐子三人:曰懿,曰旦,曰旭。旦自有傳。初,祐知貢舉,多拔擢寒俊,畢士安、柴成務皆其所取也。後
 與其子旦同入兩制,居中書。懿字文德,勵志為學,舉進士,嘗知袁州,有政績,卒,年四十九。



 旭字仲明。嚴於治內,恕以接物,尤篤友義。以蔭補太祝,嘗知緱氏縣。時官鄰邑者多貪猥,民有「永寧三钁,緱氏一鎌」之謠。又知雍丘縣。



 真宗尹京時,素聞其能,及踐阼,三遷至殿中丞。自旦居宰府,旭以嫌不任職。王矩嘗薦旭材堪治劇,真宗召旦謂曰:「前代弟兄同居要地者多矣,朝廷任才,豈以卿故屈之邪?」命授京府推官,旦固辭,
 改判南曹。由判國子監出知穎州,荒政修舉。



 大中祥符間,旦既薨,揚歷中外,卓有政績,由兵部郎中出知應天府。卒,年六十八。懿子睦,旭子質,皆能其官。



 質字子野。少謹厚淳約,力學問,師事楊億,億嘆以為英妙。伯父旦見其所為文,嗟賞之。以蔭補太常寺奉禮郎。後獻文召試,賜進士及第,被薦為館閣校勘,改集賢校理,累遷尚書祠部員外郎。丁父憂,與諸弟飯脫粟茹蔬。終喪,通判蘇州,州守黃宗旦少質,嘗因爭事,宗旦曰:「少
 年乃與丈人抗邪?」質曰:「事有當爭,職也。」卒不為屈。宗旦得盜鑄錢者百餘人,下獄治,退告質曰:「吾以術鉤致得之。」喜見於色。質曰:「以術鉤人置之死而又喜,仁者之政,固如是乎?」宗旦慚沮,為薄其罪。還判尚書刑部、吏部南曹,知蔡州。州人歲時祀吳元濟廟,質曰:「安有逆醜而廟食於民者。」毀之,為更立狄仁傑、李醞像而祠之,蔡人至今號「雙廟」。以本曹郎中召為開封府推官。時兄雍為三司判官,質不欲兄弟並居省府,懇辭,得知壽州,徙廬州。
 盜殺其徒,並貲而遁,捕得之。質論盜死,大理以謂法不當死,質曰:「盜殺其徒,自首者原之,所以疑懷其黨,且許之自新,此法意也。今殺人取貲而捕獲,貸之,豈法意乎?」疏上,不報,降監舒州靈仙觀。採古今煉形攝生之術,撰《寶元總錄》百卷。逾年,韓琦知審刑院,請盜殺其徒,非自首者勿原。著為令。於是鄭戩、葉清臣皆言質非罪,且稱其材,起知泰州,遷度支郎中,徙荊湖北路轉運使。



 嘗攝江陵府事,或訴民約婚後期,民言貧無貲以辦,故違約。
 質問其費幾何,出私錢予之。吏捕盜人衣者,盜叩頭曰:「平生不為過,迫饑寒而至於此。」質命取衣衣之,遣去。加史館修撰、同判吏部流內銓。擢天章閣待制,出知陜州,卒。



 質家世富貴,兄弟習為驕侈,而質克己好善,自奉簡素如寒士,不喜畜財,至不能自給。初,旦為中書舍人,家貧,與昆弟貸人息錢,違期,以所乘馬償之。質閱書得故券,召子弟示之曰:「此吾家素風,爾曹當毋忘也。」範仲淹貶饒州,治朋黨方急,質獨載酒往餞。或以誚質,質曰:「範
 公賢者,得為之黨,幸矣。」世以此益賢之。



 楊昭儉,字仲寶,京兆長安人。曾祖嗣復,唐門下侍郎、平章事、吏部尚書。祖授,唐刑部尚書。父景,梁左諫議大夫。



 昭儉少敏俊,後唐長興中,登進士第。解褐成德軍節度推官。歷鎮、魏掌書記,拜左拾遺、直史館,與中書舍人張昭遠等同修《明宗實錄》。書成,遷殿中侍御史。



 天福初,改禮部員外郎。晉祖命宰相馮道為契丹冊禮使,以昭儉為介,授職方員外郎,旋加虞部郎中,俄以本官知制誥。
 不逾月三拜命,時人榮之。又為荊南高從誨生辰國信使,賜金紫。使回,拜中書舍人,又為翰林學士。



 時驕將張彥澤鎮涇原,暴殺從事張式,朝廷不加罪。昭儉與刑部郎中李濤、諫議大夫鄭受益抗疏論列,請置之法。疏奏不報。會有詔令朝臣轉對,或有封事,亦許以不時條奏。昭儉復上疏曰:「天子君臨四海,日有萬機,懋建諍臣,彌縫其闕。今則諫臣雖設,言路不通,藥石之論不達於聖聰,而邪佞之徒取容於左右。御史臺紀綱之府,彈糾之
 司,銜冤者固當昭雪,為蠹者難免放流。陛下臨御以來,寬仁太甚,徒置兩司,殆如虛器。遂令節使慢侮朝章,屠害幕吏,始訴冤於丹闕,反執送於本藩。茍安跋扈之心,莫恤冤抑之苦。願回睿斷,誅彥澤以謝軍吏。」由是權臣忌之。會請告洛陽,不赴晉祖喪,為有司所糾,停官。



 未幾,起為河南少尹,改秘書少監,尋復中書舍人。時河決數郡,大發丁夫,以本部帥董其役,既而塞之。晉少主喜,詔立碑記其事。昭儉表諫曰:「陛下刻石紀功,不若降哀痛
 之詔;摛翰頌美,不若頒罪己之文。」言甚切至,少主嗟賞之,卒罷其事。周世宗愛其才,復召入翰林為學士。歲餘,改御史中丞,多振舉臺憲故事。未幾,以鞫獄之失,與知雜御史趙礪、侍御史張糾並出為武勝軍節度行軍司馬。



 開寶二年,入為太子詹事,以眼疾求退。六年,以工部尚書致仕。太宗即位,就加禮部尚書。太平興國二年,卒,年七十六。



 昭儉美風儀,善談名理,事晉有直聲。然利口喜譏訾,執政大臣懼其構謗,多曲徇其意。



 魚崇諒,字仲益,其先楚州山陽人,後徙於陜。崇諒初名崇遠,後避漢祖諱改之。幼能屬文,弱冠,相州刺史闢為從事。會魏帥楊師厚卒,建相州為昭德軍,分魏郡州縣之半以隸之。魏人不便,裨校張彥及帳下,囚節度使賀德倫歸款莊宗,崇諒奔歸陜。



 明宗即位,秦王從榮表為記室。從榮誅,坐除籍,流慶州。清泰初,移華州。俄以從榮許歸葬,放還陜。三年,起為陜州司馬。仕晉,歷殿中侍御史,鳳翔李儼表為觀察支使。奉方物入貢,宰相薦為
 屯田員外郎、知制誥。開運末,契丹入汴,契丹相張礪薦為翰林學士。契丹主北歸,留崇諒京師。



 漢祖之入,盡索崇諒所受契丹詔敕,焚於朝堂,復令知制誥。俄拜翰林學士,就加中書舍人。隱帝即位,崇諒以母老求就養,除保義軍節度副使,領臺州刺史,食郡奉。會舉師討三叛,節度使白文珂在軍前,崇諒知後事。凡供軍儲、備調發,皆促期而辦,近鎮賴之。崇諒親屬盡在鳳翔城中,逾年城破,李谷為轉運使,庇護崇諒家數十口,皆無恙。崇諒請
 告,自岐迎居於陜。未幾,王仁裕罷內職,朝議請召崇諒為學士。



 周祖踐祚,書詔繁委,皆崇諒為之。廣順初,加工部侍郎,充職。會兗州慕容彥超加封邑,彥超已懷反側,遣崇諒充使賜官告,仍慰撫之。時多進策人,命崇諒就樞密院引試,考定升降。



 崇諒以母老思鄉里,求解官歸養。詔給長告,賜其母衣服、繒帛、茶藥、緡錢,假滿百日,令本州月給錢三萬,米面十五斛。俄拜禮部侍郎,復為學士。詔令侍母歸闕,崇諒再表以母老病乞終養,優詔不
 允。世宗征高平,崇諒尚未至,陶穀乘間言曰:「魚崇諒逗留不來,有顧望意。」世宗頗疑之。崇諒又表陳母病,詔許歸陜州就養。訖太祖朝不起。



 太宗即位,詔授金紫光祿大夫、尚書兵部侍郎致仕。歲餘卒。



 張澹,字成文,其先南陽人,徙家河南。澹幼而好學,有才藻。晉開運初,登進士第。宰相桑維翰器之,妻以女。解褐校書郎,直昭文館,再遷秘書郎,充鹽鐵推官,歷左拾遺、禮部員外郎,並充史館修撰。出為洛陽令,秩滿,授吏部
 員外,復充史館修撰。周恭帝初,拜右司員外郎、知制誥。



 建隆二年,加祠部郎中。會秘書郎張去華上書自薦有文藝,願與澹及祠部員外郎知制誥盧多遜、殿中侍御史師頌並試,核定優劣。太祖令並試於講武殿,澹所對不應策問,責授左司員外郎。未幾,通判泰州兼海陵鹽監副使。蜀平,通判梓州,復拜祠部郎中。



 開寶初,就轉倉部郎中。四年冬,以本官復知制誥。六年,會李昉責授,盧多遜使江南,內署闕學士,太祖令澹權直學士院。七年長
 春節,攝殿中監,進酒,命賜金紫。六月,權點檢三司事。不逾旬,疽發背卒,年五十六。太祖聞其無子,甚愍之,命中使護葬於洛陽。



 澹美風儀,善談論,歷官厘務,所至皆治。初與詞臣校藝,黜居郎署,頗怏怏。晚年附會盧多遜,方再獲進用。



 淳化中,太宗論及文士,曰:「澹典書命而試以策,非其所長,此蓋陶谷、高錫黨、張去華以沮澹爾。若使穀輩出其不意而遽試之,豈有不失律者邪?」



 高錫,字天福,河中虞鄉人。家世業儒,幼穎悟,能屬文。漢
 乾祐中,舉進士。王晏鎮徐州,闢掌書記;留守西洛,又闢河南府推官。坐按獄失實奪官,遷置涇州,會赦得歸。周顯德初,劉崇入寇,宰相請選將拒之。世宗銳意親征,破崇高平,誅敗將樊愛能等,由是政無大小悉親決之,不復責成有司。錫徒步詣招諫匭上書,請擇賢任官,分治眾職,疏奏不報。世宗嘗令翰林學士及兩省官分撰俳優詞,付教坊肄習,以奉游宴。錫復上疏諫。後為蔡州防禦推官。



 宋初,棄官歸京師,詣匭上疏,請禁兵器,疏入不
 報。建隆五年,又以書乾宰相範質,質奏用為著作佐郎。明年春,遷監察御史。秋,拜左拾遺、知制誥,加屯田員外郎。



 乾德初,賜緋。太宗尹京,石熙載在幕中,錫弟銑應進士舉,乾熙載,望首薦。銑辭藝淺薄,熙載不許,錫深銜之,數於帝前言熙載裨贊無狀。帝具以語太宗,且曰:「當為汝擇人代之。」太宗曰:「熙載勤於乃職,聞高錫嘗求薦其弟,熙載拒之,慮為錫所構。」帝大悟,雖怒之,未有以發。會使青州,私受節帥郭崇賂遺;又嘗致書澧州刺史為僧
 求紫衣,為人所告。事下御史府核實,責貶萊州司馬。遇赦,改均州別駕,移陳州。太平興國八年,卒。



 兄子冕。冕字子莊,周顯德中,詣闕上書,稱旨,擢為諫議大夫。宰相範質以為超擢太過,詔特授將仕郎,守右補闕,賜賚加等。宋初,由膳部都官員外郎累至膳部郎中,出知益州。雍熙二年,卒,年五十。贈右諫議大夫,錄其子垂休為固始主簿。



 論曰:自唐以來,翰林直學士與中書舍人對掌訓辭,頌
 宣功德,箴諫闕失,不專為文墨之職也。宋興,亦採詞藻以備斯選,若谷之才雋,著之敏達,澹之治跡,錫之策慮,冕之敦質,咸有可觀。然豫成禪代之詔,見薄時君,終身不獲大用。及夫險詖忌前,酣鳷少檢,附勢希榮,構讒謀己,皆無取焉。蒙博洽長厚,繼竇儀裁定儀制,惜乎南郊之議,請去太祖以宣祖配天,為識者所非。昭儉抗論跋扈,志除驕將,而多言歷詆,自取惡名,抑好訐為直者與?崇諒奉親篤至,反罹間毀,終身歸養,而不復起,後蒙旌
 賁之典,則為善者聳動矣。祐以百口明符彥卿無他志,且言以猜忌殺無辜者享國不長,因以杜太宗之他疑,又卻盧多遜之傾趙普,以致被黜,仁者有後,宜乎子旦為宋元臣焉



\end{pinyinscope}