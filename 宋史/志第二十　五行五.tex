\article{志第二十 五行五}

\begin{pinyinscope}

 土



 稼穡作甘,土之性也。土失其性,則為災兇。舊說以恆風、脂夜之妖,華孽、臝蟲之孽,牛禍、黃眚、黃祥,皆屬之土,今從之。



 建隆元年,河南諸州乏食。



 乾德元年,齊、隰等州饑。二年,州府二十二饑。



 開寶四年,州府六水、一旱,諸州民乏食。五年,大饑。六年,水,民饑。九年,州府十二饑。



 太平興國四年,太平州饑。



 淳化元年,開封、河南等九州饑。五年,京東西、淮南、陜西水潦,民饑。



 咸平五年,河北及鄭、曹、滑饑。



 景德元年,江南東、西路饑。二年,淮南、兩浙、荊湖北路饑。三年,京東西、河北、陜西饑。



 大中祥符三年,陜西饑。四年,河北、陜西、劍南饑。五年,河北、淮南饑。七年,淮南、江、浙饑。八
 年,陜西州府五饑。



 天禧元年,饑。三年,江、浙及利州路饑。



 天聖三年,晉、絳、陜、解饑。



 明道元年,京東、淮南、江東饑。二年,淮南、江東、西川饑。



 寶元二年,益、梓、利、夔路饑。



 嘉祐三年,夔州路旱,饑。



 熙寧三年,河北、陜西旱。四年,河北旱,饑。六年,淮南、江東、劍南西川、潤州饑。七年,京畿、河北、京東西、淮西、成都、利州、延、常、潤、府州、威勝、保安軍饑。八年,兩河、陜西、江南、淮、浙饑。九年,雄州饑。十年,漳泉州、興化軍饑。元豐元年,河北饑。四年,鳳翔府、鳳階州饑。七年,河東
 饑。



 元符二年,饑。



 崇寧元年,江、浙、熙河饑。



 大觀三年,秦、鳳、階、成饑。



 重和元年,京西饑。五年,河北、京東、淮南饑。



 建炎元年,汴京大饑,米升錢三百,一鼠直數百錢,人食水藻、椿槐葉,道殣,骼無餘胔。三年,山東郡國大饑,人相食。時金人陷京東諸郡,民聚為盜,至車載乾尸為糧。



 紹興元年,行在、越州及東南諸路郡國饑。淮南、京東西民流常州、平江府者多殍死。二年春,兩浙、福建饑,米斗千錢。時餫餉繁急,民益艱食。三年,吉、郴、道州、桂陽監饑。五年,湖
 南大饑,殍死、流亡者眾。夏,潼川路饑,米斗二千,人食糟糠。興元饑,民流於果、閬。秋,溫、處州饑。六年春,浙東、福建饑,湖南、江西大饑,殍死甚眾,民多流徙,郡邑盜起。夏,蜀亦大饑,米斗二千,利路倍之,道殣枕藉。是歲,果州守臣宇文彬獻《禾粟九穗圖》,吏部侍郎晏敦復言:「果、遂饑民未蘇,不宜導諛。」坐黜爵。七年夏,欽、廉、邕州饑。九年,江東西、浙東饑,米斗千錢,饒、信州尤甚。十年,浙東、江南薦饑,人食草木。十一年,京西、淮南饑。十八年冬,浙東、江、淮郡
 國多饑,紹興尤甚。民之仰哺於官者二十八萬六千人,不給,乃食糟糠、草木,殍死殆半。十九年春、夏,紹興府大饑,明、婺州亦如之。二十四年,衢州饑。二十八年,平江府饑。二十九年,紹興府薦饑。



 隆興元年,紹興府大饑,四川尤甚。平江、襄陽府、隨、泗州、棗陽、盱眙軍大饑,隨、棗間米斗六七千。二年,平江府、常、秀州饑,華亭縣人食秕糠。行都及鎮江府、興化軍、臺、徽州亦艱食。淮民流徙江南者數十萬。



 乾道元年春,行都、平江、鎮江、紹興府、湖、常、秀州
 大饑,殍徙者不可勝計。是歲,臺、明州、江東諸郡皆饑。夏,亡麥。二年夏,亡麥。三年九月,不雨,麥種不入。四年春,蜀、邛、綿、劍、漢州、石泉軍大饑,邛為甚。盜延八郡,漢饑民至九萬餘。五年夏,饒、信州薦饑,民多流徙。徽州大饑,人食蕨葛。臺、楚州、盱眙軍亦饑。秋冬不雨,淮郡麥種不入。六年冬,寧國府、廣德軍、太平、湖、秀、池、徽、和州皆饑。七年秋,江東西、湖南十餘郡饑,江、筠州、隆興府為甚。人食草實,流徙淮甸,詔出內帑收育棄孩。淮郡亦薦饑,金人運麥
 於淮北岸易南岸銅鏹,斗錢八千。江西饑,民流光、濠、安豐間,皆效淮人私糴,錢為之耗。荊南亦饑。八年,江西亡麥。隆興府薦饑,南昌、新建縣饑民仰給者二八千餘。九年春,成都、永康、邛三州饑。秋,臺州饑,溫、婺州亦饑。



 淳熙元年,浙東、湖南、廣西、江西、蜀關外皆饑,臺、處、郴、桂、昭、賀尤甚。二年,淮東西、江東饑,滁、真、揚州、盱眙軍、建康府為甚。是歲,鎮江、寧國府、常州、廣德軍亦艱食。詔獎建康留守劉珙振濟有方。三年,淮甸饑。夏,臺州亡麥。冬,復、施、隨、
 郢州、荊門軍、襄陽、江陵、德安府大饑;四年春,尤饑。六年冬,和州饑。泰、通、楚州、高郵軍大饑,人食草木。七年,鎮江府、臺州、無為、廣德軍民大饑。是歲,江、浙、荊、湘、淮郡皆饑。八年春,江州饑,人採葛而食,詔罷守臣章騂。冬,行都、寧國、建康府、嚴、婺、太平州、廣德軍饑,徽、饒州大饑,流淮郡者萬餘人,浙東常平使者朱熹進對論荒政,請蠲田賦、身丁錢,詔江、浙、淮、湖北三十八郡並免之。九年春,大亡麥。行都饑,於潛、昌化縣人食草木。紹興府、衢、婺、嚴、明、臺、
 湖州饑。徽州大饑,



 穜稑亦絕。湖北七郡薦饑。蜀潼、利、夔三路郡國十八皆饑,流徙者數千人。十年,合、昌州薦饑,民就振相蹂死者三千餘人。十一年,泉、汀、漳州、興化軍亡禾。邕、賓、象州饑。十二年,福建饑,亡麥。江西、廣東西饑。金州饑,有流徙者。十四年,金、洋、階、成、鳳、西和州人乏食。七月,秀州饑,有流徙者。臨安府九縣饑。十六年夏,成州亡麥。冬,際、成、鳳、西和州薦饑。



 紹熙二年,蘄州饑。夔路五郡饑,渝、涪為甚。階、成、鳳、西和州亡麥。三年,資、榮州亡麥,
 普、敘、簡、隆州、富順監皆大饑,亡麥,殍死者眾,民流成都府至千餘人,威遠縣棄兒且六百人。揚州亦多饑。四年,簡、資、普州饑,綿州亡麥。夏,紹興府亡麥。安豐軍大亡麥。五年冬,亡麥苗。行都、淮、浙西東、江東郡國皆饑,常、明州、寧國、鎮江府、廬滁、和州為甚,人食草木。



 慶元元年春,常州饑,民之死徙者眾。楚州饑,人食糟粕。淮、浙民流行都。三年,浙東郡國亡麥,臺州大亡麥,民饑多殍。襄、蜀亦饑。四年秋,浙東西薦饑,多道殣。六年冬,常州大饑,仰哺者
 六十萬人。潤、揚、楚、通、泰州、建康府、江陰軍亦乏食。



 嘉泰元年,浙西郡國薦饑,常州、鎮江、嘉興府為甚。二年,四川饑,廣安、懷安軍、潼川府大亡麥。衡、郴州、武岡、桂陽軍乏食。三年春,邵、永州大饑,死徙者眾,民多剽盜。夏,行都艱食。四年春,撫、袁州、隆興府、臨江軍大饑,殍死者不可勝瘞,有舉家二十七人同赴水死者。



 開禧二年,紹興府、衢、婺州亡麥。湖北、京西、淮東西郡國饑,民聚為剽盜。南康軍、忠、涪州皆饑。



 嘉定元年,淮民大饑,食草木,流於江、浙
 者百萬人。先是淮郡罷兵,農久失業,米斗二千,殍死者十三四,炮人肉、馬矢食之。詔所至郡國振恤歸業,時邦儲既匱,郡計不支,去者多死,亦有俘掠而北者。是歲,行都亦饑,米斗千錢。二年春,兩淮、荊、襄、建康府大饑,米斗錢數千,人食草木。淮民刲道殣食盡,發瘞胔繼之,人相搤噬。流於揚州者數千家,度江者聚建康,殍死日八九十人。是秋,諸路復大歉,常、潤尤甚。冬,行都大饑,殍者橫市,道多棄兒。三年春,建康府大饑,人相食。五月,衢州饑,
 頗聚為剽盜。七年,臺州大亡麥。八年,淮、浙、江東西饑,都昌縣為盜者三十六黨。九年,行都饑,閭巷有殍。十年,臺、衢、婺、饒、信州饑,剽盜起,臺為甚。蜀石泉軍饑,殍死殆萬餘人。十一年秋,淮、浙、江東饑饉,亡麥苗。十二年春,潼川府饑而不害。十三年春,福州饑,人食草根。十六年春,海州新附山東民饑,京東、河北路新附山西民亦饑。湖南永、道州大饑。是歲,行都、江、淮、閩、浙郡國皆亡麥禾。十七年春,餘杭、錢塘、仁和三縣饑,鎮江府饑,真、鄂州亦乏食。



 嘉
 熙四年,紹興府薦饑,臨安府大饑,嚴州饑。



 咸淳七年,江南大饑。八年冬,襄陽饑,人相食。



 德祐二年正月,揚州饑。三月,揚州穀價騰踴,民相食。



 乾德二年五月,揚州暴風,壞軍營舍僅百區。三年六月,揚州暴風,壞軍營舍及城上敵棚。



 開寶二年三月,帝駐太原城下,大風,一夕而止。八年十月,廣州颶風起,一晝夜,雨水二丈餘,海為之漲,飄失舟楫。九年四月,宋州大風,壞甲仗庫、城樓、軍營凡四千五百九十六區。



 太平興
 國二年六月,曹州大風,壞濟陰縣廨及軍營。四年八月,泗州大風,浮梁竹笮、鐵索斷,華表石柱折。六年九月,高州大風雨,壞廨宇及民舍五百區。七年八月,瓊州颶風,壞城門、州署、民舍殆盡。八年九月,太平軍颶風拔木,壞廨宇、民舍千八十七區。十月,雷州颶風壞稟庫、民舍七百區。九年八月,白州颶風,壞廨宇、民舍。



 端拱二年,京師暴風起東北,塵沙曀日,人不相辨。



 淳化二年五月,通利軍大風害稼。三年六月丁丑,黑風自西北起,天地晦暝,
 雷震,有頃乃止。先是京師大熱,疫死者眾,及北風至,疫疾遂止。



 至道二年八月,潮州颶風,壞州廨、營砦。



 咸平元年八月,涪州大風,壞城舍。四年八月丙子,京師暴風。



 景德二年六月甲午,大風吹沙折木。八月,福州海上有颶風,壞廬舍。三年七月丙寅,京師大風。四年三月甲寅夕,京師大風,黃塵蔽天,自大名歷京畿,害桑稼,唐州尤甚。



 大中祥符二年四月乙未,大風起京師西北,連日不止。五年八月,京師大風。七年三月戊辰,京師大風,揚沙礫。
 是日,百官習儀恭謝壇,有陷僕者。八年六月辛亥,京師風起巳位,吹沙揚塵。



 天禧二年正月,永州大風,發屋拔木,數日止。三年五月,徐州利國監大風起西南,壞廬舍二百餘區,壓死十二人。四年四月丁亥,大風起西北,飛沙折木,晝晦數刻。五月乙卯,暴風起西北,有聲,折木吹沙,黃塵蔽天,占並主陰謀奸邪。是秋,內侍周懷政坐妖亂伏誅。



 天聖九年十二月辛酉,大風三日止。



 景祐元年六月己巳,無錫縣大風發屋,民被壓死者眾。九月甲寅
 夜漏上,風自醜起有聲,擺木鳴條。二年六月戊寅平明,風自未來,占者以為百穀豐衍之候。



 皇祐四年七月丁巳,大風起西北方,拔木。



 嘉祐二年正月元日平旦,有風從東北來,遍天有蒼黑雲,占云:「大熟多雨。」



 熙寧四年二月辛巳,京東自濮州至河北旁邊,大風異常,百姓驚恐,六年四月,館陶縣黑風。九年十一月,海陽潮陽二縣颶風、潮,害民居田稼。十年六月,武城縣大風,壞縣廨,知縣李愈妻、主簿寇宗奭妻之母壓死。七月,溫州大風雨,漂
 城樓、官舍。



 元豐四年六月,邕州颶風,壞城樓、官私廬舍。七月甲午夜,泰州海風作,繼以大雨,浸州城,壞公私廬舍數千間。靜海縣大風雨,毀官私廬舍二千七百六十三楹。丹陽縣大風雨,溺民居,毀廬舍。丹徒縣大風潮,飄蕩沿江廬舍,損田稼。六月,邕州颶風,壞城樓、官私廬舍。五年六月,朱崖軍颶風,毀廬舍。



 元祐八年,福建、兩浙海風駕潮,害民田。



 紹聖元年秋,蘇、湖、秀等州有風害民田。



 靖康元年正月望夜,大風起西北,有聲,吹沙走石,盡明
 日乃止。二月戊申,大風起東北,揚塵翳空。三月己巳夜五更,大風乍緩乍急,聲如叫怒。十一月丁亥,大風發屋折木。閏十一月甲寅,大風起北方,雪作,盈數尺,連夜不止。二年正月巳亥,天氣昏噎,狂風迅發,竟日夜,西北陰雲中如有火光,長二丈餘,闊數尺,民時時見之。庚戌,大風雨。二月乙酉,大風折木,晚尤甚。三月巳亥,大風。四月庚申朔,大風吹石折木。辛酉,北風益甚,苦寒。



 建炎元年正月丁酉,大風吹石折木。十二月乙酉,大風拔木。



 紹興
 二十八年七月壬戌,平江府大風雨駕潮,漂溺數百里,壞田廬。三十二年七月戊申,大風拔木。溫州大風,壞屋覆舟。



 隆興元年,浙東、西郡國風水傷稼。二年八月,大風雨,漂蕩田廬。



 乾道二年八月丁亥,溫州大風雨駕海潮,殺人覆舟,壞廬舍。五年十月,臺州大風水,壞田廬。八年六月丙辰,惠州颶風,壞海艦三十餘。時樞密院調廣東經略司水軍,四艦覆其三,死者百三十餘人。



 淳熙三年六月,大風連日。四年九月,明州大風駕海潮,壞定海、鄞
 縣海岸七千六百餘丈及田廬、軍壘。六月乙巳夜,福清縣、興化軍大風雨,壞官舍、民居、倉庫及海口鎮,人多死者。五年正月庚戌,大風。六年十一月,鄂州大風覆舟,溺人甚眾。七年二月,江陵府大風,火及舟,焚溺死者尤眾。十年八月辛酉,雷州颶風大作,駕海潮傷人,禾稼、林木皆折。



 紹熙二年三月癸酉,瑞安縣大風,壞屋拔木殺人。四年七月,興化軍海風害稼。五年六月丙子,大風。七月乙亥,行都大風拔木,壞舟甚眾。紹興府、秀州大風駕海潮,
 害稼。秋,明州颶風駕海潮,害稼。十月甲戌,行都大風拔木。



 慶元二年六月壬申,臺州暴風雨駕海潮,壞田廬。六年三月甲子,大風拔木。



 嘉泰三年十月丁未,暴風。十一月癸未,大風,四年正月乙亥亦如之。



 開禧元年四月乙卯、九月庚戌,大風。



 嘉定元年九月乙丑,大風。二年二月戊子,大風。七月壬辰,臺州大風雨駕海潮,壞屋殺人。三年八月癸酉,大風拔木,折禾穗,墮果實。寧宗露禱,至於丙子乃息。後御史朝陵於紹興府,歸奏風壞陵殿宮墻
 六十餘所、陵木二千餘章。四年閏月丁未,大風。六年十二月,餘姚縣風潮壞海堤,亙八鄉。七年正月庚辰,江州放鐙,黑雲暴風忽作,游人相踐,死者二十餘。十年正月乙未,大風拔木。十一月丁丑,大風。十一年二月甲寅,大風。十月戊午,大風。十三年十一月庚戌、壬子,大風。十二月戊午,大風。十四年六月辛巳,大風。十六年秋,大風拔木害稼,十七年秋,福州颶風大作,壞田損稼。冬,鄂州暴風,壞戰艦二百餘,壽昌軍壞戰艦六十餘,江州、興國亦
 如之。



 嘉熙二年,風雹。三年,風雹。



 淳祐十一年,泰州風。



 景定四年十一月,福州颶風。



 咸淳四年閏月丁巳,大風雷雨,居民屋瓦皆動。七年五月甲申,紹興府大風。十年四月,紹興府大風拔木。



 端拱二年,京師暴風起東北,塵沙曀日,人不相辨。



 淳化三年六月丁丑,黑風自西北起,天地晦冥,雷震,有頃乃止。



 大中祥符二年九月,無為軍城北暴風,晝晦不可辨,拔木,壞城門、營壘、民舍。



 天聖六年二月庚辰,大風晝暝。



 康定元年三月丙子,大風晝暝,經刻乃復。



 嘉祐八年十一月丙午,大風霾。



 治平二年二月乙巳,大風晝晦。四年正月庚辰朔,大風霾。是日,上尊號,廷中仗衛皆不能整。時帝已不豫,後七日崩。



 熙寧四年四月癸亥,京師大風霾。



 元祐八年二月,京師風霾。



 靖康二年正月己亥,天氣昏曀,風迅發竟日。三月丁酉,風霾。



 建炎元年正月辛卯朔,大風霾。丁酉,風霾,日色薄而有暈。二月丁酉,汴京風霾,日無光。是日,張邦昌僭位。二年七月癸未,風雨晝晦。
 是日,東京留守宗澤薨。



 紹興十一年三月庚申,金人居長安,晝晦。



 乾道五年正月甲申,晝霾四塞。



 淳熙五年四月丁丑,塵霾晝晦,日無光。



 慶元九年十二月乙未,天雨霾。



 開禧元年正月壬午,雨霾。



 嘉定十年正月乙未,晝霾。二月癸巳,日無光。



 德祐元年六月庚子朔,日有食之,既,天地晦冥,咫尺不辨人,雞鶩歸淒,自巳至申,其明始復。



 至道二年秋九月,環、慶州梨生花,占有兵。明年,契丹擾北邊。



 景德元年二月,保順軍城壕冰,陷起文為桃李花、
 雜樹、人物之狀。



 大中祥符九年正月,霸州渠冰有如華葩狀。



 大觀二年十月乙巳,龔丘縣檜生花,萼如蓮實。



 紹興七年十二月,中書、門下省檢正官張宗元出撫淮西軍,寓建康。盤冰有文如畫,佳卉茂木,華葉相敷,日易以冰,變態奇出,春暄乃止。二十七年四月,徽州祁門縣圃桃已實復華。



 淳熙初,秀州呂氏家冰瓦有文,樓觀、車馬、人物、芙蓉、牡丹、萱草、藤蘿之屬,經日不釋。淳熙中,興化軍仙游縣九巫山古木末生花,臭如蘭。



 建隆二年九月,渭南縣孑□蟲傷稼。三年七月,兗州、濟、德、磁、洺蝝生。



 乾德六年七月,階州孑□蟲生。



 太平興國二年六月,磁州青黑蟲群飛食桑,夜出晝隱,食葉殆盡。七月,邢州鉅鹿、沙河二縣步屈蟲食桑麥殆盡。五年七月,濰州孑□蟲生,食稼殆盡。七年九月,邠州孑□蟲生,食稼。九年七月,泗州蠓蟲食桑。



 雍熙二年四月,天長軍蠓蟲食苗。



 端拱二年七月,施州孑□蟲生,害稼。



 淳化二年四月,中都縣蜴蟲生。七月,單州蜴蟲生,遇雨死。



 景德
 元年八月,陜、賓、棣州蟲暝害稼。



 大中祥符四年八月,兗州孑□蟲生,有蟲青色隨嚙之,化為水。六年九月,陜西同、華等州孑□蟲食苗。



 天聖五年五月戊辰,磁州蟲食桑。



 景祐四年五月,滑州靈河縣民黃慶家蠶自成被,長二丈五尺,闊四尺。



 嘉祐五年,深州野蠶成繭,被於原野。



 熙寧九年五月,荊湖南路地生黑蟲,化蛾飛去。金州生黑蟲食苗,黃雀來,食之皆盡。



 元祐六年閏八月,定州七縣野蠶成繭。七年五月,北海縣蠶自織如絹,成領帶。



 元
 符元年七月,蒿城縣野蠶成繭。八月,行唐縣野蠶成繭。九月,深澤縣野蠶成繭,織紝成萬匹。二年六月,房陵縣野蠶成繭。



 政和元年九月,河南府野蠶成繭。四年,相州野蠶成繭。五年,南京野蠶成繭,織紬五匹,綿四十兩,聖繭十五兩。



 紹興二十九年秋,浙東、江東西郡縣螟。三十年十月,江、浙郡國螟蝝。



 隆興元年秋,浙東西郡國螟,害穀,紹興府、湖州為甚。二年,臺州螟。



 乾道三年八月,江東郡縣螟螣。淮、浙諸路多言青蟲食穀穗。六年秋,浙西、江
 東螟為害。九年秋,吉、贛州、臨江、南安軍螟。



 淳熙二年秋,浙、江、淮郡縣螟。四年秋,昭州螟。五年,昭州薦有螟螣。七年秋,永州螟。八年秋,江州螟。十二年八月,平江府有蟲聚於禾穗,油灑之即墮,一夕,大雨盡滌之。十四年秋,江州、興國軍螟。十六年秋,溫州螟。



 慶元三年秋,浙東蕭山、山陰縣、婺州,浙西富陽、鹽官、淳安、永興縣、嘉興府皆螟。四年秋,鉛山縣蟲食穀,無遺穗。



 嘉定十四年,明、臺、溫、婺、衢蟊螣為災。十五年秋,贛州螟。十六年:永、道州螟。



 紹定
 三年,福州螟。



 端平元年五月,當塗縣螟。



 淳祐二年五月,兩淮螟。



 景定三年八月,浙東、西螟。



 乾德三年,眉州民王進牛生二犢。四年,南充縣民馬全信及相如縣民彭秀等家牛生二犢。



 開寶二年,九隴縣民王達牛生二犢。



 太平興國三年,流溪縣民白延進牛生二犢。五年,溫江縣民趙進牛生二犢。六年,廣都縣趙全牛生二犢。七年,什邡縣民王信、華陽縣民袁武等牛生二犢。八年,彭州民彭延、閬州民陳則、安樂縣民王公
 泰牛生二犢。九年七月,知乾州衛升獻三角牛。



 雍熙三年,果州民李昭牛生二犢。四年郪縣民鮮於志鮮於皋、眉山縣海羅參、仁壽縣民陰饒、成都縣民李本、成紀縣民王和敏牛生二犢。



 端拱元年,眉州民陳希簡、晉原縣民張昭鬱、魏城縣民鮮於郜、羅江縣民袁族、河陽縣民李美、曲水縣民曾虔、梓潼縣民文光懿、永泰縣民羅德、綿竹縣民陳洪牛生二犢。



 淳化元年,綿竹縣民李昌遠薄逸、閬州民和中、忠州民民王欽、眉州王圖、九隴縣民
 楊皋、玄武縣民羊邁達牛生二犢。二年,永川縣民梁行良、仁壽縣民梁



 至道二年,新都縣民蹇成美牛生二犢。穎陽縣民馮延密牛生二犢,其二額有白。三年,新津縣民文承富、赤水縣民蘇福、廣安軍吏胥仁迪牛生二犢。



 咸平元年,眉山縣民向瓊玖陳元寶、丹
 棱縣民劉承鶚、通泉縣民王居中、曲水縣民楊漢成楊景歡王師讓、眉山縣民陳彥宥牛生三犢。二年,蒙陽縣民杜摯、九隴縣民楊太、眉山縣民蘇仁義、洪雅縣吏陸文贊牛生二犢。三年,敘浦縣民戴昌蘊牛生二犢。四年,流溪縣民何承添、晉原縣民頗全、永昌縣民曾嗣、犀浦縣民何福、彰明縣民王□巳牛生二犢。六年,渠江縣民王德進、魏城民蒲諫王信、石照縣民仲漢宗、大足縣民劉武牛生二犢。



 景德元年,魏城縣民閻明、彭州蒙陽縣
 民郭琮牛生二犢。二年,三泉縣民李景順、東海縣民時祐、小溪縣民劉可、赤水縣民羅永並牛生二犢。三年,長江縣民於承琛牛生二犢。四年,相如縣民楊漢暉、邛州安仁縣民羅瑩、九隴縣民白彥成、渠江縣民王繼豐家及順安軍屯田務牛生二犢。



 大中祥符元年,龔丘縣民李起牛生四犢,判州王欽若圖以獻。二年,立山縣民盧仁依、銅山縣民勾熙正、什邡縣民杜族、南康縣陳邦並牛生二犢。三年,犍為縣民陳知進牛生二犢。四年,東
 關縣民陳知進牛生二犢。五年,富順監些井場官楊守忠、曲水縣民向平、蓬溪縣民蹇知密牛生二犢。六年,廣安軍依政縣民李福、貴溪縣民徐志元牛生二犢。七年,雙流縣民姚彥信、涪城縣民張禮、嘉州龍游縣民張正、夾江縣民郭升、天水縣民王吉牛生二犢。八年,仁壽縣民何志、通泉縣民罷永泰、成都縣民張進、華陽縣民楊承珂牛生二犢。九年,平定軍平定縣民範訓、臨邛縣民楊暉牛生二犢。



 天禧元年,開江縣民冉津及澧州石門
 縣層山院牛生二犢。二年,臨邛縣民王道進、臨溪縣民王勝、西縣民韓光緒牛生二犢。四年,貴溪縣民葉政牛生二犢。五年,巴西縣民向知道牛生二犢。



 自天聖迄治平,牛生二犢者三十二,生三犢者一。



 自熙寧二年距元豐八年,郡國言民家牛生二犢者三十有五,生三角者一。



 元祐元年距元符三年,郡國言民家牛生二犢者十有五。



 大觀元年,閬州、達州言牛生二犢。四年三月,帝謂起居舍人宇文粹中曰:「牛產二犢,亦載之起居注中,豈
 若野蠶成繭之類,民賴其利,乃為瑞邪?」自是史官不復盡書。



 政和五年七月,安武軍言,郡縣民範濟家牛生麒麟。



 重和元年三月,陜州言牛生麒麟。



 宣和二年十月,尚書省言,歙州歙縣民鮑珙家牛生麒麟。三年五月,梁縣民邢喜家牛生麒麟。



 紹興元年,紹興府有牛戴刃突入城市,觸馬,裂腹出腸。時衛卒多犯禁屠牛,牛受刃而逸,近牛禍也。十六年,靜江府城北二十里,有奔犢以角觸人於壁,腸胃出,牛狂走,兩日不可執,卒以射死。十八年
 五月,依政縣牛生二犢。二十一年七月,遂寧府牛生二犢者三。二十五年八月,漢中牛生二犢。



 淳熙十二年,仁和縣良渚有牛生二首,七日而死。餘杭縣有犢二首。十六年三月,池州池口鎮軍屯牛狂走,觸人死。



 慶元三年,樂平縣田家牛生犢如馬,一角,鱗身肉尾,農以不祥殺之,或惜其為麟;同縣萬山牛生犢,人首。



 淳化三年正月乙卯,京師雨土,占曰:「小人叛。」自後李順盜據益州。



 景德元年七月辛亥,黃氣出壁,長五尺餘,占
 曰:「兵出。」二年正月丙寅,黃白氣環之。



 大中祥符元年正月癸亥朔,黃氣出於艮,占曰:「主五穀熟。」二年九月戊午,黃氣如柱起東南方,長五丈許。



 天禧五年,襄州鳳林鎮道側地湧起,高三尺,闊八尺,知州夏辣以聞。



 明道元年十月庚子夜,黃白氣五,貫紫微垣。



 景祐元年八月壬戌夜,有黃白氣如彗,長七尺餘,出張、翼之上,凡三十有三日不見。



 治平元年三月壬戌,雨土。十二月己亥,雨黃土。



 熙寧五年十二月癸未、七年三月戊午,並雨黃土。八年
 五月丁丑,雨黃土兼細毛。



 元豐二年十一月丁亥、五年三月乙巳、六年四月辛未,雨土。



 元祐七年正月戊午,天雨塵土,主民勞苦。



 宣和元年三月庚午,雨土著衣,主不肖者食祿。



 紹興十一年三月庚申,涇州雨黃沙。十八年十一月壬辰,肆赦,天有雲赤黃,近黃祥也,太史附秦檜旨奏瑞。



 乾道四年三月己丑,雨土若塵。



 淳熙四年二月戊戌,雨土,五年二月壬午、甲申、四月丁丑、六年十一月乙丑、十一年正月辛卯、甲寅、十三年正月壬寅,亦如之。
 十五年九月庚子,南方有赤黃氣。



 紹熙四年十月甲寅,雨土,五年四月癸卯亦如之。十月乙未,天有赤黃色,占曰:「是為天變。」色先赤後黃,近黃赤祥也。十一月辛亥,雨土。



 慶元元年二月己卯、十一月己丑,天雨塵土。三年正月丙子、四月丙午、十二月甲申,天雨塵土。六年正月己巳、閏月丁未、十月己丑,雨土。九月辛丑、十一月辛卯,天雨塵土。



 嘉泰元年六月己卯、九月己未、十二月辛丑,天雨塵土。



 嘉定三年正月丙午,天雨塵土。八年二月己未、
 五月辛未,天雨塵土。九年十二月癸巳,天雨土。十年二月癸巳,雨土。十二年二月癸巳,天雨塵土。十三年三月辛卯,天雨塵土。十六年二月戊子,天雨塵土。



 紹定三年三月丁酉,雨土。



 嘉熙二年四月甲申,雨土。三年三月辛卯,天雨塵土。



 淳祐五年二月丙寅朔,天雨塵土。十一年三月乙亥,天雨塵。



 寶祐三年三月己未,雨土。六年二月壬辰,天雨塵土。



 開慶元年三月辛酉,雨土。



 景定五年二月辛未,雨土。



 德祐元年三月辛巳,終日黃沙蔽天,或曰「
 喪氛」。



 乾德三年,京師地震史失日月。五年十一月,許州開元觀老君像自動,知州宋偓以聞。六年正月,簡州普通院毗廬佛像自動。



 至道二年十月,潼關西至靈州、夏州、環慶等州地震,城郭廬舍多壞,占云:「兵饑。」是時,西夏寇靈州,明年,遣將率兵援糧以救之,關西民饑。



 咸平二年九月,常州地震,壞鼓角樓、羅務、軍民廬舍甚眾。四年九月,慶州地震者再。六年正月,益州地震。



 景德元年正月丙申夜,
 京師地震;癸卯夜,復震;丁未夜,又震,屋皆動,有聲,移時方止。癸丑,冀州地震,占云:「土工興,有急令,兵革興。」是年,契丹犯塞。二月,益、黎、雅州地震。三月,邢州地震不止。四月己卯夜,瀛州地震。五月,邢州地復震不止。十一月壬子,日南至,京師地震。癸丑,石州地震。四年七月丙戌,益州地震。己丑,渭州瓦亭砦地震者四。



 大中祥符二年三月,代州地震。四年六月,昌、眉州並地震。七月,真定府地震,壞城壘。天聖五年三月,秦州地震。七年,京師地震。



 景
 祐四年十二月甲子,京師地震。甲申,忻、代、並三州地震,壞廬舍,覆壓吏民。忻州死者萬九千七百四十二人,傷者五千六百五十五人,畜擾死者五萬餘;代州死者七百五十九人,並州千八百九十人。



 寶元元年正月庚申,並、忻、代三州地震。十二月甲子,京師地震。



 慶歷三年五月九日,忻州地大震,說者曰:「地道貴靜,今數震搖,兵興民勞之象也。」四年五月庚午,忻州地震,西北有聲如雷。五年七月十四日,廣州地震。六年二月戊寅,青州地震。
 三月庚寅,登州地震,岠嵎山摧。自是震不已,每震,則海底有聲如雷。五月甲申,京師地震。七年十月乙丑,河陽、許州地震。



 皇祐二年十一月丁酉夜,秀州地震,有聲自北起如雷。



 嘉祐二年,雄州北界、幽州地大震,大壞城郭,覆壓者數萬人。五年五月己丑,京師地震。



 治平四年秋,漳、泉、建州、邵武、興化軍等處皆地震,潮州尤甚,拆裂泉湧,壓覆州郭及兩縣屋宇,士民、軍兵死者甚眾。八月己巳,京師地震。



 熙寧元年七月甲申,地震。乙酉、辛卯,再震;
 八月壬寅、甲辰,又震。是月,須城、東阿二縣地震終日,滄州清池、莫州亦震,壞官私廬舍、城壁。是時,河北復大震,或數刻不止,有聲如雷,樓櫓、民居多摧覆,壓死者甚眾。九月戊子,莫州地震,有聲如雷。十一月乙未,京師及莫州地震。十二月癸卯,瀛州地大震。丁巳,冀州地震。辛酉,滄州地震,湧出沙泥、船板、胡桃、螺蚌之屬。是月,潮州地再震。是歲,數路地震,有一日十數震,有逾半年震不止者。二年十月庚戌,南郊,東壝門內地陷,有天寶十三年
 古墓。



 元豐元年,邕州佛像動搖。初,像動而夏人入寇,又動而州大火,其後儂智高叛,復動,於是知州錢師孟投其像於江中。八年二月甲戌,賓州嶺方縣地陷。五月丙午,京師地震。



 元祐二年二月辛亥,代州地震有聲。四年春,陜西、河北地震。七年九月己酉,蘭州、鎮戎軍、永興軍地震,十月庚戌朔,環州地再震。



 紹聖元年十一月丙戌,大原府地震。二年十月、十一月,河南府地震。是歲,蘇州自夏迄秋地震。三年三月戊戌夜,劍南東川地震。九月
 己酉,滁州、沂州地震。四年六月己酉,太原府地震有聲。



 元符元年七月壬申夜,雲陰蔽天,地震良久。二年正月壬申,恩州地震。八月甲戌,太原府地震;三年五月己巳,太原府又震。



 建中靖國元年十一月辛亥,太原府、潞、晉、隰、代、石、嵐等州岢嵐威勝保化寧化軍地震彌旬,晝夜不止,壞城壁、屋宇,人畜多死。自後有司方言祥瑞,郡國地震多抑而不奏。



 政和七年六月,詔曰:「熙河、環慶、涇原路地震經旬,城砦、關堡、城壁、樓櫓、官私廬舍並皆摧塌,
 居民覆壓死傷甚眾,而有司不以聞,其遣官按視之。」



 宣和四年,北方用兵,雄州地大震。玄武見於州之正寢,有龜大如錢,蛇若朱漆箸,相逐而行,宣撫使焚香再拜,以銀奩貯二物。俄俱死。六年正月,京師連日地震,宮殿門皆動有聲。七年七月己亥,熙河路地震,有裂數十丈者,蘭州尤甚。陷數百家,倉庫俱沒。河東諸郡或震裂。



 建炎二年正月戊戌,長安地大震,金將婁宿圍城,彌旬無外援,乘地震而入,城遂陷。



 紹興三年八月甲申,地震,平江
 府、湖州尤甚。是歲,劉豫陷鄧、隨等州,金人犯蜀。四年,四川地震。五年五月,行都地震。六年六月乙巳夜,地震自西北,有聲如雷,餘杭縣為甚。是冬,劉麟、猊犯順,寇濠、壽州。七年,地震。二十四年正月戊寅,地震。二十五年三月壬申,地震。二十八年八月甲寅夜,震。三十一年三月壬辰,地震。三十二年七月戊申,地震。



 隆興元年十月丁丑,地震;六月甲寅,又震。



 乾道二年九月丙午,地震自西北方。四年十二月壬子,石泉軍地震三日,有聲如雷,屋瓦
 皆落,時綿竹有冤獄云。



 淳熙元年十二月戊辰,地震自東北方。九年十二月壬寅夜,地震。十年十二月丙寅,地震。十二年五月庚寅,地震。



 慶元六年九月,東北地震。十一月甲子,地震東北方。



 嘉定六年四月,行都地震。六月丙子,淳安縣地震。九年二月辛亥,東、西川地大震四日。十年二月庚申,地震自東南。十二年五月,地震。六月,西川地震。十四年正月乙未夜,地震,大雷。五月丙申,西川地震。



 寶慶元年八月己酉,地震。



 嘉熙四年十二月丙辰,
 地震。



 淳祐元年十二月庚辰夜,地震。



 寶祐三年,蜀地震。



 咸淳七年,嘉定府城震者三。



 雍熙三年,階州福津縣常峽山圮,壅白江水,逆流高十許丈,壞民田數百里。



 淳化二年五月,名山縣大風雨,登遼山圮,壅江水逆流入民田,害稼。



 咸平元年七月庚午,寧化軍汾水漲,壞北水門,山石摧圮,軍士有壓死者。二年七月庚寅,靈寶縣暴雨崖圮,壓居民,死者二十二戶。三年三月辛丑夜,大澤縣三陽砦大雨崖摧,壓死者六
 十二人。四年正月,成紀縣山摧,壓死者六十餘人。



 景德四年七月,成紀縣崖圮,壓死居民。



 熙寧五年九月丙寅,華州少華山前阜頭峰越八盤領及穀,摧陷於石子坡。東西五里,南北十里,潰散墳裂,湧起堆阜,各高數丈,長若堤岸。至陷居民六社,凡數百戶,林木、廬舍亦無存者。並山之民言:「數年以來,峰上常有云,每遇風雨,即隱隱有聲。是夜初昏,略無風雨,山上忽霧起,有聲漸大,地遂震動,不及食頃而山摧。」



 元祐元年十二月,鄭縣界小敷
 谷山摧,傷居民。



 紹興十二年十二月,陜西不雨,五穀焦枯,涇、渭、灞、滻皆竭。時秦民以饑離散,壯者為北人所買,郡邑遂空。



 紹熙四年秋,南岳祝融峰山自摧。劍門關山摧。五年十二月,臨安府南高峰山自摧。



 慶元二年六月辛未,臺州黃巖縣大雨水,有山自徙五十餘里,其聲如雷,草木、塚墓皆不動,而故址潰為淵潭。時臨海縣清潭山亦自移。



 嘉泰二年七月丁未,閩建安縣山摧,民廬之壓者六十餘家。



 嘉定六年六月丙子,嚴州淳安縣長樂
 鄉山摧水湧。九年,黎州山崩。



 咸淳十年,天目山崩。



 熙寧元年,荊、襄間天雨白犛如馬尾,長者尺餘,彌漫山谷。三月丁酉,潭州雨毛。八年五月丁丑,雨黃毛。



 紹熙四年十一月癸酉,地生毛。



 咸淳九年,江南平地產白毛,臨安尤多。



\end{pinyinscope}