\article{列傳第一百二十七}

\begin{pinyinscope}

 王德王彥魏勝張憲楊再興牛皋胡閎休



 王德,字子華,通遠軍熟羊砦人。以武勇應募,隸熙帥姚古。會金人入侵,古軍懷、澤間,遣德諜之,斬一酋而還。補
 進武校尉。古曰:「能復往乎?」德從十六騎徑入隆德府治,執偽守姚太師,左右驚擾,德手殺數十百人,眾愕眙莫取前。古械姚獻於朝,欽宗問狀,姚曰:「臣就縛時,止見一夜叉耳。」時遂呼德為「王夜叉」。



 建炎元年,以勤王師倍道趨闕,改隸劉光世,平濟南寇李昱、池陽寇張遇。光世將先鋒討李成,德以百騎覘賊,至蔡州上蔡驛口橋,賊疑為誘騎,擁眾欲西。德麾騎大呼曰:「王師大至矣!」賊駭遁,追殺甚眾。成奔新息,收散卒復戰。賊見光世張蓋行陳,
 不介冑,知為主帥,並兵圍之。德突圍擁光世還軍,遂襲敗李成。授武略大夫。



 三年春,遷前軍統領,屯天長。金人攻揚州,西軍多潰,德趨宣化。會叛將張昱、張彥圍和州,太守張績求援於德,德兵傅城下,賊不意其至,大潰。遲明接戰,斬昱,俘其兵騎萬數,濟自採石。



 光世方謀討苗、劉之逆,迎至建康,謂德曰:「江都之擾,諸軍不竄則盜。公可仗義夜涉大江,徇國急變。」遂以軍屬光世。會苗、劉走閩中,詔德追擊,隸韓世忠。德欲自致功名,而世忠必欲
 德為之使,遣親將陳彥章邀德於信州。彥章拔佩刀擊德,德殺彥章,尸諸市。德至浦城,斬苗瑀,擒馬柔吉送行在。世忠訟其擅殺,下臺獄,侍御史趙鼎按德當死,帝命特原之,編管郴州。



 時光世屯九江,得楊惟忠所失空頭黃敕,即以便宜復德前軍統制,遣平信州妖賊王念經。行次饒州,會賊劉文舜圍城,德引兵赴之,文舜請降。德納而誅之,自餘不戮一人。謂諸校曰:「念經聞吾宿留,必不為備。」倍道而趨,一鼓擒之,獻俘於朝。詔還舊秩,加武
 顯大夫、榮州刺史。



 四年,光世鎮京口,以德為都統制。金兵復南,光世將退保丹陽,德請以死捍江,諸將恃以自強。分軍扼險,渡江襲金人,收真、揚數郡。既而又遇敵於揚州北,有被重鎧突陣者,德馳叱之;重鎧者直前刺德,德揮刀迎之,即墮馬。眾褫駭,因麾騎乘之,所殺萬計。



 紹興元年,平秀州水賊邵青。初,德與戰於崇明沙,親執旗麾兵拔柵以入,青軍大潰。他日,餘黨復索戰,諜言將用火牛,德笑曰:「是古法也,可一不可再,今不知變,此成擒
 耳。」先命合軍持滿,陳始交,萬矢齊發,牛皆返奔,賊眾殲焉。青自縛請命,德獻俘行在。帝召見便殿問勞,褒賞特異。遷中亮大夫、同州觀察使。



 三年,光世宣撫江、淮,當移屯建康,命韓世忠代之。德從數十騎自京口逆世忠,度將及麾下,徒步立道左,抗言曰:「擅殺陳彥章,王德迎馬頭請死。」世忠下馬握其手曰:「知公好漢,鄉來纖介不足置懷。」乃設酒盡歡而別。是冬,知鞏州、熙河蘭廓路兵馬鈐轄。



 明年春,知蘭州,徙屯池陽及當塗,為行營左護軍
 前軍統制。金兵掠江北,破滁州。德越江襲奪之,追至桑根,擒女真萬戶盧孛一人,千戶十餘人。五年,改環慶副總管。



 六年冬,劉豫遣麟、猊驅鄉兵三十萬,分東西道入寇,中外甚恐,議欲為保江計。殿帥楊沂中、統制張宗顏、田師中及德等分兵御之,大敗猊兵於藕塘,猊挺身走;麟在順昌聞之,亦拔砦遁。德追至壽春,弗及,獲其糧舟四百艘。第功,除武康軍承宣使,真拜相州觀察使。



 七年,改熙河蘭廓路副總管、行營左護軍都統制,駐師合肥。
 會光世罷宣撫,詔德盡護其眾,以酈瓊副之。瓊與德故等夷,恥屈其下,率眾叛從劉豫。八年,命隸張俊,名其軍曰「銳勝」。



 十年,解穎昌圍,俊檄德就取宿州。德倍道自壽春馳至蘄縣,與敵游騎遇,遂入城,偃旗臥鼓,騎引去。因潛師宿州,夜半,薄賊營。敵將高統軍詰朝壓汴而陳,偽守馬秦、同知耶律溫以三千人陰水邀戰。德策馬先濟,步騎從之。遙謂賊曰:「吾與金人大小百戰,雖名王貴酋,莫不糜碎,爾何為者。」賊遂投兵降。馬秦、耶律溫馳入,閉
 門城守。德至,呼秦諭以逆順,乃自縋而下。德叱其子順先登,秦率溫降,遣詣行在。德乘勝趨亳州,俊會於城父。時叛將酈瓊屯亳,聞德至,謂三路都統制曰:「夜叉未易當也。」遂遁。德入亳州,白俊曰:「今兵威已振,請乘破竹之勢,進取東都。」俊難之,乃班師。策功第一,拜興寧軍承宣使、龍神衛四廂都指揮使,再遷侍衛親軍馬步軍都虞候,封隴西郡侯。



 十一年,金人自合肥入侵,游騎及江。俊議分軍守南岸,德曰:「淮者,江之蔽也,棄淮不守,是謂唇
 亡齒寒也。敵數千里遠來,餉道決不繼,及其未濟急擊之,可以奪氣;若遲之,使稍安,則淮非吾有矣」俊猶豫未許。德請益堅,曰:「願父子先越江,俟和州下,然後宣撫北渡。」俊乃許德即渡採石,俊督軍繼之。宿江中,德曰:「明旦,當會食歷陽。」已而夜拔和州,晨迎俊入。敵退保昭關,又擊走之,追至柘皋,與金人夾河而軍。



 諸將帥皆集,惟張俊後至,統制田師中欲待之,德怒曰:「事當機會,復何待!」徑上馬。兀朮以鐵騎十餘萬夾道而陣,德曰:「賊右陣堅,
 我當先擊之。」麾軍渡橋,首犯其鋒。一酋被甲躍馬始出,德引弓一發而斃;乘勝大呼,令萬兵持長斧,如墻而進。敵大敗,退屯紫金山,德復尾擊之。劉錡謂德曰:「昔聞公威略如神,今果見之,請以兄禮事。」召拜清遠軍節度使、建康府駐札御前諸軍都統制,歷浙東福建總管、荊南副都統制。二十五年,卒,贈檢校少保,再贈少傅。二子琪、順,亦以驍勇聞。



 王彥,字子才,上黨人。性豪縱,喜讀韜略。父奇之,使詣京
 師,隸弓馬子弟所。徽宗臨軒閱試,補下班祗應,為清河尉。從涇原路經略使種師道兩入夏國,有戰功。



 金人攻汴京,彥慨然棄家赴闕,求自試討賊。時張所為河北招撫使,異其才,擢為都統制。使率裨將張翼、白安民、岳飛等十一將,部七千人渡河,與金人戰。敗之,復衛州新鄉縣,傳檄諸郡。



 金人以為大軍至,率數萬眾薄彥壘,圍之數匝。彥以眾寡不敵,潰圍出。諸將散歸,彥獨保共城西山,遣腹心結兩河豪傑,圖再舉。金人購求彥急,彥慮變,
 夜寢屢遷。其部曲覺之,相率刺面,作「赤心報國,誓殺金賊」八字,以示無他意。彥益感勵,撫愛士卒,與同甘苦。未幾,兩河響應,忠義民兵首領傅選、孟德、劉澤、焦文通等皆附之,眾十餘萬,綿亙數百里,皆受彥約束。金人患之,召其首領,俾以大兵破彥壘。首領跪而泣曰:「王都統砦堅如鐵石,未易圖也。」金人乃間遣勁騎撓彥糧道,彥勒兵待之,斬獲甚眾。益治兵,刻日大舉,告期於東京留守宗澤。



 澤召彥會議,乃將兵萬餘渡河,金人以重兵襲其
 後而不敢擊。既至汴京,澤大喜,令彥宿兵近甸,以衛根本。彥即以所部兵馬付留守司,量帶親兵趨行在。時已遣宇文虛中為祈請使議和。彥見黃潛善、汪伯彥,力陳兩河忠義延頸以望王師,願因人心,大舉北伐。言辭憤激,大忤時相意,遂降旨免對,以彥為武翼郎、閣門宣贊舍人,差充御營平寇統領。時範瓊為平寇前將軍,彥知瓊有逆節,稱疾不就,乞致仕,許之。



 知樞密院事張浚宣撫川、陜,奏彥為前軍統制。浚與金酋婁宿相持於富平,
 欲大舉,初至漢中,會諸將議,彥獨以為不可,曰:「陜西兵將上下之情,皆未相通,若少不利,則五路俱失。不若且屯利、閬、興、洋,以固根本,敵入境,則檄五路兵來援,萬一不捷,未大失也。」浚幕府不然其言。彥即請為利路鈐轄,俄改金均房州安撫使、知金州。



 時中原盜賊蜂起,加以饑饉,無所資食;惟蜀富饒,巨盜往往窺覬。桑仲既陷淮安、襄陽,乘勢西向,均、房失守,直搗金州白土關,眾號三十萬。仲,彥舊部曲也,以申櫝請於彥曰:「仲於公無敢犯,
 願假道入蜀就食耳。」彥乃遣統領官門立為先鋒擊之。賊銳甚,立戰死。將士失色,或請避之。彥叱曰:「樞相張公方有事關陜,若仲越金而至梁、洋,則腹背受敵,大事去矣。敢言避者斬!」即勒兵趨長沙平,阻水據山,設伏以待。賊見官軍少,蟻附搏戰。彥執幟一麾,士殊死鬥,賊敗走。彥休士進擊,追奔至白磧,復房州。



 紹興元年九月,權京西南路副總管李忠反,擾京西,遂攻金州諸關。賊眾皆河朔人,驍果善戰,彥與戰不利,關陷。彥退屯秦郊,令將
 士盡伏山谷間,焚秦郊積聚,偽若遁者。秦郊距郡城二十里,路坦夷,彥募敢死士易麾幟,設奇以待。閱再宿,賊至秦郊,官軍逆戰,大敗之,追襲至秦嶺,遂復乾祐縣以歸。忠走降劉豫。



 初,桑仲既敗還襄陽,乃鳩集散亡陷鄧州,兇焰復熾。南攻德安,西據均陽,分眾三道:一攻住口關,一出馬郎嶺,一搗洵陽,前軍去金州不三十里。彥曰:「仲以我寡彼眾,故分三道以離吾勢,法當先破其堅,則脆者自走。」遣副將焦文通御住口,自以親兵營馬郎。相
 持一月,大戰六日,賊大敗,仲為其下所殺。又有王闢、董貴、祁守中阻兵窺蜀,勢雖不及桑仲,然小者猶不減數萬,彥悉討平之。



 是冬,偽齊秦鳳經略使郭振以數千騎掠白石鎮,彥與關師古並兵御之,賊大敗,獲振,復秦州。張浚承制以彥節制商、虢、陜、華州軍馬。



 三年正月,兀朮入侵,浚召彥與吳玠、劉子羽會於興元。撒離曷自上津疾馳,不一日至洵陽。統制官郭進死之,彥退保石泉縣。金人入金、均,彥趨西鄉。二月,金人攻饒風關,彥與吳玠
 御之,不能卻,關破,彥收餘兵奔達州。五月,彥遣兵至漢陰縣,與劉豫將周貴戰,大敗之,復金州。浚承制進彥保康軍承宣使兼宣撫司參議,彥不受。



 五年四月,差知荊南府,充歸、峽、荊門公安軍安撫使。彥因荊南曠土措置屯田,自蜀買牛千七百頭,授官兵耕,營田八百五十頃,分給將士有差。六年二月,知襄陽府、京西南路安撫使,彥以岳飛嫌辭。浚奏彥為行營前護副軍都統制、督府參謀軍事。



 六月,以八字軍萬人赴行在。至鎮江,聞母喪,
 上疏乞解官,不許。詔免喪服,趣入對,遂以為浙西、淮東沿海制置副使,以所部屯通州之料角。七年正月,彥因遣將捕亡者於解潛軍中,軍士交鬥於市,言者論其軍政不肅,貶秩二等。彥不自安,乞終餘服。二月,復洪州觀察使、知邵州。彥入辭,帝撫勞甚厚,曰:「以卿能牧民,故付卿便郡,行即召矣。」九年,卒於官,年五十。



 彥稱名將,當建炎初,屢破大敵,威聲振河朔。時方撓於和議,遽召之還,又奪其兵柄而使之治郡,士議惜之。彥事親孝,居官廉,
 子弟有戰功,不與推賞。將死,召其弟侄,以家財均給之。



 魏勝,字彥威,淮陽軍宿遷縣人。多智勇,善騎射,應募為弓箭手,徙居山陽。紹興三十一年,金人將南侵,聚芻糧,造器械,籍諸路民為兵。勝躍曰:「此其時也。」聚義士三百,北渡淮,取漣水軍,宣布朝廷德意,不殺一人,漣水民翕然以聽。



 遂取海州。郡守渤海高文富聞勝起,遣兵來捕勝。距海州南八十里大伊,與金兵遇,勝迎擊走之,追至城下。眾驚傳水陸悉有兵,城中大恐,文富閉門守,驅民
 上城御之。勝令城外多張旗幟,舉煙火為疑兵;又遣人向諸城門,諭以金人棄信背盟,無名興師,本朝寬大愛民之意。城上民聞之,即開門,勝遣勇銳者登城樓,餘自門入,莫有御者。獨文富與其子安仁率牙兵拒守,勝整軍與安仁父子戰譙門內,殺安仁及州兵千餘,擒文富,民皆按堵。



 勝權知州事,遣人諭朐山、懷仁、沐陽、東海諸縣,皆定。乃蠲租銳,釋罪囚,發倉庫,犒戰士;分忠義士為五軍,紀律明肅,部分如宿將。勝自兼都統制,益募忠義
 以圖收復,遠近聞之響應,旬日,得兵數千。即具其事報境上帥守,冀給軍裝器甲。時帥守雖知金人將渝盟,未有發其端者,莫敢以聞。



 左軍統制董成謀出西北取沂州,勝先遣間還,知金兵數萬至沂,以我軍器甲未備,戒成勿動。成不從勝,率所部千餘人直入沂州巷戰,殺其守及軍士三千餘,眾悉降,得器甲數萬。金人生兵復集,競登屋擲瓦擊之,成軍幾敗。勝欲斬成,以其驍勇,釋之。



 金人遣同知海州事蒙恬鎮國以兵萬餘取海州,抵州
 北二十里新橋。勝帥兵出迎之,設伏於隘,陣以待。眾殊死戰,伏發,賊大敗,殺鎮國,馘千人,降三百人,軍聲益振。山東之民咸欲來附,勝傳檄招諭,結集以待王師之至。


沂民壁蒼山者數十萬,金人圍之,久不下,砦首滕
 \gezhu{
  曰狄}
 告急於勝。勝提兵往救之,陣於山下。金人多伏兵,勝兵遇伏,皆赴砦。金人襲之,勝單騎而殿,以大刀奮擊。金人望見勝,知其為將也,以五百騎圍之數重。勝馳突四擊,金陣開復闔。戰移時,身被數十槍,冒刃出圍。金兵追之,馬
 中矢踣,步而入砦,無敢當者。金人又急攻,絕其水,砦中食乾□,殺牛馬飲血,勝默禱而雨驟作。



 金人攻盆急,周山為營,勝度其必復攻海州,因間出砦越城中。金人果解蒼山圍,自新橋抵城下,勝出戰皆捷。金分兵四面攻之,勝募士登城以御,矢石如雨者七日,金兵死傷多,遁去。勝嘗出戰,矢中鼻貫齒,不能食,猶親御戰。



 勝起義久,朝廷尚未知。沿海制置使李寶遣其子公佐由海道覘敵,至州,始遣忠義將朱震、褚道詣行在,白勝姓名於執
 政,始知勝之功焉。



 金主亮舉兵渡淮,慮勝睨其後,分軍數萬來攻。會李寶帥舟師往膠西,破金人舟艦,勝遣人邀之,同擊金人於新橋,大敗之。金兵未退,寶知金舟將遁,復以兵登舟備海道。金主初命造海艦,欲分軍入蘇、杭,悉以中原民操舟楫。民家送衣裘者相告語,俟王師至即背之。及寶舟入島中,適北風勁,舟不進。有頃反風,金人艤舟於岸,操舟者望見寶舟,謬云此金國兵也,俾皆入舟中。舟忽至,金人不知,寶縱火焚其舟。舟以赤油
 絹為帆,風順火熾,操舟者皆登岸走。金兵在舟中者,坐以待縛,載之檻車,悉獲其舟。



 寶既捷,勝亦還州為捍禦計。金兵至,營於城北砂巷,列陣將攻關門,先遣人說勝使降。勝開門出諭之曰:「汝主叛盟失信,無故興兵,我朝以仁義之師,來復舊疆,汝主渡淮必敗。爾等宜早來歸,必獲爵賞。」時金兵已逼關,勝登關門張樂飲酒,犒軍士,令固守勿出戰。金兵攻之逾時,乃少遣士出,憑險隘擊之。金人知不可攻,率軍轉而渡河,襲關後。勝斂兵入城,
 金兵追將及,勝獨乘馬逐之,叱曰:「魏勝在此!」聞之皆闢易,士卒後入者不復敢追。



 勝軍已入城,金兵徑趨城東,欲過砂堰環城為營。勝先已據堰備之,金軍不得過,拒戰竟日,終不能近。有新募士守河者,不知兵,金兵遽過河,勝恐絕河路,亟收軍入城。金兵追至東門外黃土板,勝單騎逐之,大叱之,金兵五百皆望風退。勝又追十數里,士得入城;有不得入者,由城南入西門。金兵復自西南來襲,勝從後叱之,金兵駭散,手殺數人。奏功授閣門祗
 候,差知海州兼山東路忠義軍都統。遣其子昌同峒峿山首領張榮,持旗榜往結山東忠義。



 金兵自新橋、關子門、砂堰之敗,殺傷者眾。一日黎明,乘昏霧,四面薄城急攻。勝激厲士卒,竭力捍禦,矢石交下。城上鎔金液,投火牛。金兵不能前,多死傷,乃拔砦走。距海州為長垣,包州城於中,使不能出。及亮死,乃解去。



 勝善用大刀,能左右射,旗揭曰「山東魏勝」,金人望見即退走。勝為旗十數,書其姓名,密付諸將,遇鏖戰即揭之,金兵悉避走。初,勝起
 義時,無州郡糧餉之給,無府庫倉廩之儲。勝經畫市易,課酒榷鹽,勸糶豪右。環海州度視敵兵攻取處,築城浚隍,塞關隘,在軍,未嘗一日懈弛,恆如寇至。方糾集遠邇,犒勞士卒,期約有日,會金主亮被弒,金兵北歸,王師亦南還矣。



 初,亮聞勝在海州,知不可取,曰:「少須,他時取之易耳。」亮既殞,勝益得自治軍旅,人皆精銳。獲金諜者,犒以酒食,厚賂遣還。有自北方來歸者,與之同臥起,共飲食,示以不疑;周其窶貧,使之感激。自是山東、河北歸附
 者眾,得金人虛實,悉以上聞。又第其忠義士功能,假授官資,因李寶轉達於朝,悉如所請。



 金人遣山東路都統、總管以兵十萬攻海州。時寶帥海舟水陸並進,抵城北砂巷,勝率眾合寶軍大破之,斬首不可計,堰水為之不流,餘悉奔潰。勝獨率兵追北二十里,至新橋,又破之,盡獲其鞍馬器甲。寶亦駐海州,為進取計。



 金人復遣五斤太師發諸路兵二十餘萬來攻海州,先遣一軍自州西南斷勝軍餉道。勝驛勇悍士三千餘騎,拒於石闥堰,金
 軍不能進。逮夜始還,留千人備險隘。金兵十萬來奪,勝率眾鏖戰,殺數千人,餘皆遁去,下令守險勿追。報寶,寶以防海道,登舟,不復發兵。金兵盛集,勝力拒之,自旦至暮,金兵不能奪。勝令步卒整隊前行,自為殿。



 時百姓以寶既登舟,懼金兵大至,皆欲入城,統制郭蔚閉城門不納。人民牛馬蔽野,呼號動地,城中亦懼。勝入城,諭以賊勢退怯之狀,固守可保無虞,乃開門盡納之。居無何,金兵環城圍數重,勝與郭蔚分兵備御,偃旗僕鼓,寂若無
 人。金軍驚疑,數日不敢攻,已乃植雲梯,置炮石,四面合圍,負土填壕。勝俟其近城,鳴鼓張旗,矢石俱發,繼以火牛、金液,凡三晝夜,金兵竟不能近。於是罷攻,修營壘,絕河道,謀為固守。勝俟其不備掩擊,或獨出擾之,使不得休息。又間夜發兵劫其營,或焚其攻具。



 既而金人並力急攻,勝告急於李寶。寶以聞,還報城中,已命張子蓋率兵來解圍。金人亦知子蓋軍且至,已有退意。頃之,子蓋先帥騎兵至,勝出與子蓋議戰事,且促其步卒。勝出軍
 城北砂巷,與金軍大戰,斬首不可計,追數十里,餘兵皆遁。勝與子蓋議進討,子蓋曰:「受詔解圍,不知其它。」遂率軍還。城中疑懼,欲隨王師出,勝親邀於道而諭之,至漣水軍,與偕還。



 時都督張浚在建康,招勝,詢以軍務。轉閣門宣贊舍人,差充山東路忠義軍都統制兼鎮江府駐札御前前軍統制,仍知海州。勝還。



 隆興元年,詔以鎮江御前同統制魏全來守海州,督府亦遣賈和仲充山東、河北路招撫使,節制本路軍馬,海州駐札。和仲忌勝,陰
 誘忠義軍使不安。勝與辨是非,和仲又讒勝於都督,惑之。呼勝至鎮江計事,罷其職,改京東路馬步軍副總管、都督府統制,建康府駐札。既而督府知和仲所誣,罷之,復勝舊職,仍遣鎮江御前後軍屯海州,代前軍還鎮江。



 勝既還海州,鎮撫一方,民安其政。改忠州刺史。海州城西南枕孤山,敵至,登山瞰城中,虛實立見,故西南受敵最劇。勝築重城,圍山在內,寇至則先據之,不能害。



 勝嘗自創如意戰車數百兩,炮車數十兩,車上為獸面木牌,
 大槍數十,垂氈幕軟牌,每車用二人推轂,可蔽五十人。行則載輜重器甲,止則為營,掛搭如城壘,人馬不能近;遇敵又可以御箭簇。列陣則如意車在外,以旗蔽障,弩車當陣門,其上置床子弩,矢大如鑿,一矢能射數人,發三矢可數百步。炮車在陣中,施火石炮,亦二百步。兩陣相近,則陣間發弓弩箭炮,近陣門則刀斧槍手突出,交陣則出騎兵,兩響掩擊,得捷拔陣追襲,少卻則入陣間稍憩。士卒不疲,進退俱利。伺便出擊,慮有拒遏,預為解
 脫計,夜習不使人見。以其制上於朝,詔諸軍遵其式造焉。



 二年,以議和撤海州戍,命勝知楚州,以本州官吏及部兵赴新治。詔勝同淮東路安撫使劉寶、知高郵軍劉敏措置盱眙軍、楚州一帶,勝專一措置清河口。時和議尚未決,金兵乘其懈,以舟載器甲糗糧自清河出,欲侵邊。勝覘知之,身帥忠義士拒於清河口。金兵詐稱欲運糧往泗州,由清河口入淮。勝知其謀,欲御之,都統制劉寶以方議和,不許。金騎軼境,勝率諸軍拒於淮陽,自卯
 至申,勝負未決。金軍增生兵來,勝與之力戰,又遣人告急於寶。寶在楚州,相距四十里,堅謂方講和,決無戰事,迄不發一兵。勝矢盡,救不至,猶依士阜為陣,謂士卒曰:「我當死此,得脫者歸報天子。」乃令步卒居前,騎為殿,至淮陰東十八里,中矢,墜馬死,年四十五。



 事聞,贈保寧軍節度使,謚忠壯。時淮南未平,詔於鎮江府江口鎮立廟,賜號褒忠,仍俟事定更祠於戰沒處。且令有司刻木以斂,葬於鎮江。官其二子,郊武功大夫、忠州刺史,昌承信
 郎。賜銀千兩,絹千匹,宅一區,田百頃。其後使者過淮東,始得其詳,還言於朝。以劉寶不出救兵,削兩鎮節鉞,沒入家貲,貶瓊州死。勝所糾集忠義,有為賈和仲誘隸別屯及撤戍隔絕者,尚五千餘人,入京口屯駐前軍。



 郊,添差揚州兵馬鈐轄。淳熙十五年,孝宗語樞臣曰:「魏勝之子,當與優異。」又曰:「人材須用而後見,使魏勝不因邊釁,何以見其才?」詔郊添差兩浙西路馬步軍副總管。



 張憲,飛愛將也。飛破曹成,憲與徐慶、王貴招降其黨二
 萬。有郝政率眾走沅州,首被白布,為成報仇,號「白巾賊」,憲一鼓擒之。



 飛遣憲復隨州,敵將王嵩不戰而遁。進兵鄧州,距城三十里,遇賊兵數萬迎戰。與王萬、董先各出騎突擊,賊眾大潰,遂復鄧州。



 十年,金人渝盟入侵,憲戰穎昌、戰陳州皆大捷,復其城。兀朮頓兵十二萬於臨穎縣,楊再興與戰,死之。憲繼至,破其潰兵八千,兀朮夜遁。憲將徐慶、李山復捷於臨穎東北,破其眾六千,獲馬百匹,追奔十五里,中原大震。



 會秦檜主和,命飛班師,憲亦
 還。未幾,檜與張俊謀殺飛,密誘飛部曲,以能告飛事者,寵以優賞,卒無人應。聞飛嘗欲斬王貴,又杖之,誘貴告飛。貴不肯,曰:「為大將寧免以賞罰用人,茍以為怨,將不勝其怨。」檜、俊不能屈,俊劫貴以私事,貴懼而從。時又有王俊者,善告訐,號「雕兒」,以奸貪屢為憲所裁。檜使人諭之,俊輒從。



 檜、俊謀以憲、貴、俊皆飛將,使其徒自相攻發,因及飛父子,庶主上不疑。俊自為狀付王俊,妄言憲謀還飛兵,令告王貴,使貴執憲。憲未至,俊預為獄以待之。
 屬吏王應求白張俊,以為密院無推勘法。俊不聽,親行鞫煉,使憲自誣,謂得雲書,命憲營還兵計。憲被掠無全膚,竟不伏。俊手自具獄成,告檜械憲至行在,下大理寺。



 檜奏召飛父子證憲事。帝曰:「刑所以止亂,勿妄追證,動搖人心。」檜矯詔召飛父子至。萬俟離誣飛使於鵬、孫革致書憲、貴,令虛申警報以動朝廷,雲與憲書規還飛軍。其書皆無有,乃妄稱憲、貴已焚之矣,但以眾證具獄。語在飛《傳》。憲坐死,籍家貲。紹興三十二年,追復龍神衛四
 廂都指揮使、閬州觀察使,贈寧遠軍承宣使,錄其家。



 楊再興,賊曹成將也。紹興二年,岳飛破成,入莫邪關。第五將韓順夫解鞍脫甲,以所虜婦人佐酒。再興率眾直入其營,官軍卻,殺順夫,又殺飛弟翻。成敗,再興走躍入澗,張憲欲殺之,再興曰:「願執我見岳公。」遂受縛。飛見再興,奇其貌,釋之,曰:「吾不汝殺,汝當以忠義報國。」再興拜謝。



 飛屯襄陽以圖中原,遣再興至西京長水縣之業陽,殺孫都統及統制滿在,斬五百餘人,俘將吏百人,餘黨
 奔潰。明日,再戰於孫洪澗,破其眾二千,復長水,得糧二萬石以給軍民,盡復西京險要。又得偽齊所留馬萬匹,芻粟數十萬。中原響應。復至蔡州,焚賊糧。



 飛敗金人於郾城,兀朮怒,合龍虎大王、蓋天大王及韓常兵逼之。飛遣子云當敵,鏖戰數十合,敵不支。再興以單騎入其軍,擒兀朮不獲,手殺數百人而還。兀術憤甚,並力復來,頓兵十二萬於臨穎。再興以三百騎遇敵於小商橋,驟與之戰,殺二千餘人,及萬戶撒八孛堇、千戶百人。再興戰
 死,後獲其尸,焚之,得箭鏃二升。



 牛皋,字伯遠,汝州魯山人。初為射士,金人入侵,皋聚眾與戰,屢勝,西道總管翟興表補保義郎。杜充留守東京,皋討劇賊楊進於魯山,三戰三捷,賊黨奔潰。累遷榮州刺史、中軍統領。金人再攻京西,皋十餘戰皆捷。加果州團練使。京城留守上官悟闢為同統制兼京西南路提點刑獄。金人攻江西者,自荊門北歸,皋潛軍於寶豐之宋村,擊敗之。轉和州防禦使,充五軍都統制。又與孛堇
 戰魯山鄧家橋,敗之。轉西道招撫使。偽齊乞師於金入寇,皋設伏要地,自屯丹霞以待。敵兵悉眾來,伏發,俘其酋豪鄭務兒。遷安州觀察使,尋除蔡唐州信陽軍鎮撫使、知蔡州。遇敵戰輒勝,加親衛大夫。



 會岳飛制置江西、湖北,將由襄、漢規中原,命皋隸飛軍。飛喜甚,即闢為唐鄧襄郢州安撫使,尋改神武後軍中部統領。偽齊使李成合金人入寇,破襄陽六郡。敵將王嵩在隨州,飛遣皋行,裹三日糧。糧未盡,城已拔,執嵩斬之,得卒五千,遂復
 隨州。李成在襄陽,飛遣皋以騎兵擊破之,復襄陽。



 金人攻淮西,飛遣皋渡江,自提兵與皋會。時偽齊驅甲騎五千薄廬州,皋遙謂金將曰:「牛皋在此,爾輩胡為見犯?」眾皆愕然,不戰而潰。飛謂皋曰:「必追之,去而復來,無益也。」皋追擊三十餘里,金人相踐及殺死者相半,斬其副都統及千戶五人,百戶數十人,軍聲大振。



 廬州平,進中侍大夫。從平楊麼,破之。麼技窮,舉鐘子儀投於水,繼乃自僕。皋投水擒麼,飛斬首函送都督行府。除武泰軍承宣
 使,改行營護聖中軍統制,尋充湖北、京西宣撫司左軍統制,加龍、神衛四廂都指揮使。



 金人渝盟,飛命皋出師戰汴、許間,以功最,除捧日天武四廂都指揮使、成德軍承宣使,樞密行府以皋兼提舉一行事務。宣撫司罷,改鄂州駐札御前左軍統制,升真定府路馬步軍副統總管,轉寧國軍承宣使、荊湖南路馬步軍副總管。



 紹興十七年上巳日,都統制田師中大會諸將,皋遇毒,亟歸,語所親曰:「皋年六十一,官至侍從,幸不啻足。所恨南北通
 和,不以馬革裹尸,顧死牖下耳。」明日卒。或言秦檜使師中毒皋雲。



 初,檜主和,未幾,金渝盟入侵,帝手札賜飛從便措置。飛乃命皋及王貴、董先、楊再興、孟邦傑、李寶等經略東西京、汝、鄭、穎、陳、曹、光、蔡諸郡;又遣梁興渡河,糾合忠義社取河東北州縣。未幾,李寶捷於曹州,捷於宛亭,捷於渤海廟;董先、姚政捷於穎昌;劉政捷於中牟。張憲復穎昌、淮寧府;王貴之將楊成復鄭州;張應、韓清復西京。皋及傅選捷於京西,捷於黃河上。孟邦傑復永安
 軍,其將楊遇復南城軍,又與劉政捷於西京。梁興會太行忠義及兩河豪傑趙雲、李進、董榮、牛顯、張峪等破金人於垣曲,又捷於沁水,追至孟州之邵原,金張太保、成太保等以所部降,又破金高太尉兵於濟源。喬握堅等復趙州;李興捷於河南府,捷於永安軍;梁興在河北取懷、衛二州,大破兀朮軍,斷山東、河北金帛馬綱之路,金人大擾。未幾,岳飛還朝,下獄死,世以為恨云。



 胡閎休,字良㢸,開封人。宣和初,入太學。時方諱兵,閎休
 著《兵書》二卷。靖康初,創知兵科,閎休應試,中優等,補承信郎。



 金人圍城,閎休分地而守。二帝詣金營,閎休欲結義士劫之,何□禁止之。二帝北遷,範瓊散勤王師,閎休曰:「勤王師可進不可退。」檄令隨軍而無靖康年號,閎休得之泣下,懷檄而走,從辛道宗勤王。南渡,以忠義進兩官。湖湘盜起,或曰招之便,或曰討之便,閎休作《致寇》、《御寇》二篇,言天地之氣,先春後秋,招之不伏則討之。於是以岳飛為招討使,飛闢閎休為主管機宜文字。以誅鐘
 子儀功,進成忠郎。



 飛被誣死,閎休發憤杜門,佯疾十年,卒。有《勤王忠義集》藏於家。孫照,德安太守。



 論曰:王德素有威略,蚤隸劉光世,審其不可恃;晚從張俊,竟以功名顯,其知所擇哉。王彥棄家赴國,累破堅敵,威振河朔;晚奪兵柄,使之治郡,用違其材,惜矣。魏勝崛起,無甲兵糧餉之資,提數千烏合之眾,抗金人數十萬之師,卒完一州,名震當時,壯哉!然見忌於諸將,無援而戰死,亦可惜矣。張憲等五人皆岳飛部將,為敵所畏,亦
 一時之傑也;然或以戰沒,或以憤卒,而憲以不證飛獄冤死,悲夫!



\end{pinyinscope}