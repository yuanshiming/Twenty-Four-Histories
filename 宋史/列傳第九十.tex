\article{列傳第九十}

\begin{pinyinscope}

 孫長卿周沆李中師羅拯馬仲甫王居卿孫構張詵蘇寀馬從先沈遘弟遼從弟括李大臨呂夏卿祖無擇程師孟張問陳舜
 俞樂京劉蒙附苗時中韓贄楚建中張頡盧革子秉



 孫長卿,字次公,揚州人。以外祖朱巽任為秘書省校書郎。天禧中,巽守雍,命隨所取浮圖像入見。仁宗方權聽天下事,嘉其年少敏占對,欲留侍東宮,辭以母疾。詔遷官知楚州糧料院。郡倉積米五十萬,陳腐不可食,主吏皆懼法,毋敢輕去,長卿為酌新舊均渫之,吏罪得免。



 通判河南府。秋,大雨,軍營壞,或言某眾將叛,洛中嘩然。長
 卿馳諭之曰:「天雨敗屋廬,未能葺,汝輩豈有欲叛意,得無有乘此動吾軍者邪?」推首惡一人誅之,留宿其所,眾遂定。詔汰三陵奉先卒,汰者群噪府下,長卿矯制使還,而具言不可汰之故,朝廷為止。知和州,民訴人殺弟,長卿察所言無理,問其資,曰:「上等也。」「家幾人?」曰:「惟此弟爾。」曰:「然則汝殺弟也。」鞫之,服,郡人神明之。



 提點益州路刑獄,歷開封鹽鐵判官、江東淮南河北轉運使、江浙荊淮發運使。歲漕米至八百萬,或疑其多,長卿曰:「吾非欲事
 羨贏,以備饑歲爾。」議者謂楚水多風波,請開盱眙河,自淮趣高郵,長卿言:「地阻山回繞,役大難就。」事下都水。調工數百萬,卒以不可成,罷之。時又將弛茶禁而收其征,召長卿議,長卿曰:「本祖宗榷茶,蓋將備二邊之糴,且不出都內錢,公私以為便。今之所行,不足助邊糴什一,國用耗矣。」乃條所不便十五事,不從。



 改陜西都轉運使。逾年,知慶州。州據險高,患無水,蓋嘗疏引澗谷汲城中,未幾復絕。長卿鑿百井,皆及泉。泥陽有羅川、馬嶺,上構危
 棧,下臨不測之淵,過者惴恐。長卿訪得唐故道,闢為通塗。加集賢院學士、河東都轉運使,拜龍圖閣直學士、知定州。



 熙寧元年,河北地大震,城郭倉庾皆隤,長卿盡力繕補。神宗知其能,轉兵部侍郎,留再任。明年,卒,年六十六。



 長卿無文學,而長於政事,為能臣。性潔廉,不以一毫取諸人。定州當得園利八十萬,悉歸之公。既沒,詔中使護其喪歸葬。



 周沆,字子真,青州益都人。第進士,知渤海縣。歲滿,縣人
 請留,既報可,而以親老求監州稅。通判鳳翔,初置轉運判官。沆使江西,求葬親,改知沂州。歷開封府推官。



 湖南蠻唐、盤二族寇暴,殺居民,官軍數不利,以沆為轉運使。沆言:「蠻驟勝方驕,未易鬥力,宜須秋冬進兵。且其地險氣毒,人驍悍,善用金延盾,北軍不能確。請選邕、宜、融三州卒三千人習知山川技藝者,徑搗其巢,布餘兵絡山足,出則獵取之。俟其勢窮力屈,乃可順撫。」朝廷用其策,二族皆降。加直史館、知潭州。他道兵來戍者,率兩期乃代,
 多死瘴癘,沆清以期為斷,戍人便之。



 徙河東轉運使。民盜鑄鐵錢,法不能禁,沆高估錢價,鑄者以無利,自息。入為度支副使。



 儂智高亂定,仁宗命安撫廣西,諭之曰:「嶺外地惡,非賊所至處,毋庸行。」對曰:「君命,仁也;然遠民罹塗炭,當布宣天子德澤。」遂往,遍行郡邑。民避寇棄業,吏用常法,滿半歲則聽人革佃。沆曰:「是豈與兇年詭征役者同科?」奏申其期。擢天章閣待制、陜西都轉運使,改河北。



 李仲昌建六塔河之議,以為費省而功倍。詔沆行視,
 沆言:「近計塞商胡,本度五百八十萬工,用薪芻千六百萬;今才用功一萬,薪芻三百萬。均一河也,而功力不相侔如是,蓋仲昌先為小計,以來興役爾。況所規新渠,視河廣不能五之一,安能容受?此役若成,河必泛溢,齊、博、濱、棣之民其魚矣。」既而從初議,河塞復決,如沆言。



 又徙河東轉運使,遷龍圖閣直學士、知慶州。,召知通進銀臺司、判太常寺。英宗既即位,契丹賀乾元節使至,沆館客,欲取書柩前,使者以非典故,不可。沆折之曰:「昔貴國有
 喪,吾使至柳河即反,今聽於幾筵達命。恩禮厚矣,尚何云?」使者立授書。朝廷未知契丹主年,沆乘間雜他語以問,得其實,使者悔之曰:「今復應兄事南朝矣。」



 進樞密直學士、知成德軍。俗方棄親事佛。沆閱按,斥數千人還其家。以戶部侍郎致仕,卒,年六十九。



 李中師,字君錫,開封人。舉進士,陳執中薦為集賢校理、提點開封府界。境多盜,中師立賞格,督吏分捕,盡得之。進秩,辭不受,乃擢度支判官,為淮南轉運使。兩浙饑,移
 淮粟振贍,僚屬議勿與,中師曰:「朝廷視民,淮、浙等爾。」卒與之。徙河東,入為度支副使,拜天章閣待制、陜西都轉運使,知澶州、河南府。召權三司使、龍圖閣直學士,復為河南。前此多大臣居守,委事掾幕,吏習弛緩,中師一以嚴整齊之。號為治辦。然用法刻深,煩碎無大體,唯厚結中人。



 初,神宗嘗對宰相稱其治狀,富弼曰:「陛下何從知之?」帝默然。中師銜弼沮已,及再至,弼已老,乃籍其戶,令出免役錢與富民等。又希司農指,多取餘,視他處為重,
 洛人怨之。朝廷以中師率先推行,召為群牧使。乞廢河南、北監牧,省國費,而養馬於民,不報。後竟行其說,民不堪命。權發遣開封府,卒,年六十一。有女嫁陳執中子世儒,坐夫事誅死。



 羅拯,字道濟,祥符人。第進士,歷官知榮州。州介兩江間,每江漲,輒犯城郭,拯作東西二堤除其患。選知秀州,為江西轉運判官、提點福建刑獄。泉州興化軍水壞廬舍,拯請勿征海運竹木,經一年,民居皆復其舊。



 遷轉運使。
 邵武之光澤不榷酒,以課賦民,號「黃曲錢」,拯均之他三邑,人以為便。改江、淮發運副使。江、淮故無積倉,漕船系岸下,俟糴入乃得行,蓋官吏以淮南不受陳粟為逃譴計。拯始請凡米至而不可上供者,以廩軍;又貯浙西米於潤倉以時運,自是漕增而費省。轉為使。



 拯使閩時,泉商黃謹往高麗,館之禮賓省,其王云自天聖後職貢絕,欲命使與謹俱來。至是,拯以聞,神宗許之,遂遣金悌和主貢。高麗復通中國自茲始。加天章閣待制。居職七年,徙
 知永興軍、青、穎、秦三州,卒,年六十五。



 拯性和柔,不與人校曲直。為發運使時,與副皮公弼不協。公弼徙他道,御吏劾其貸官錢,拯力為辨理。錢公輔為諫官,嘗論拯短,而公輔姻黨多在拯部內,往往薦進之。或譏以德報怨,拯曰:「同僚不協,所見異也;諫官所言,職也。又何怨乎?」時論服其長者。



 馬仲甫,字子山,廬江人,太子少保亮之子也。舉進士,知登封縣。轅轅道險厄,遂傭民鑿平為坦塗,人便其行,為
 刻石頌美。通判趙州,知臺州,為度支判官。



 內侍楊永德言漕舟淮、汴間,惟水遞鋪為便。詔仲甫偕往訂可否,還言其害十餘條,議遂格。出為夔路轉運使。歲饑,盜粟者當論死,仲甫請罪減一等,詔須奏裁。復言:「饑羸拘囚,比得報,死矣,請決而後奏。」



 徙使淮南。真、揚諸州地狹,出米少,官糴之多,價常踴登,濱江米狼戾,而農無所售。仲甫請移糴以紓其患,兩益於民,從之。遂繇戶部判官為發運使。自淮陰徑泗上浮長淮,風波覆舟,歲罹其患。仲甫
 建議鑿洪澤渠六十里,漕者便之。



 拜天章閣待制、知瀛州秦州。古渭介青唐之南,夏人在其北,中通一徑,小警則路絕。仲甫得篳慄城故址,自雞川砦築堡,北抵南谷,環數百里為內地,詔賜名甘谷堡。故時羌人入城貿易,皆僦邸,仲甫設館處之,陽示禮厚,實閑之也。



 熙寧初,守亳、許、揚三州,糾察在京刑獄,知通進、銀臺司,復為揚州,提舉崇禧觀,卒。



 王居卿,字壽明,登州蓬萊人,以進士至知齊州,提舉夔
 路京東刑獄、鹽鐵判官。建言商賈轉百貨市塞上者,聽以家貲抵於官,為給長券,至賣所,並輸征稅直,公私便之。



 出知揚州,改京東轉運使。青州河貫城中,苦泛溢為病,居卿即城立飛梁,上設樓櫓,下建門,以時閉啟,人誦其智。徙河北路。河決曹村,居卿立軟橫二埽以遏怒流,而不與水爭。朝廷賞其功,建以為都水法。召拜戶部副使、提舉市易,擢天章閣待制、河北都轉運使。知秦州、太原府,卒,年六十二。居卿俗吏,特以言利至從官。



 孫構字紹先,博平人。中進士第,為廣濟軍判官,歲入圭田粟六百石,構止受百石,餘以畀學官。久之,知黎州,夷年墨數擾邊,用間殺之。蜀帥呂公弼上其事,擢知真州。兇歲得盜,令名指黨伍,悉置諸法,境內為清。



 遷度支判官。夔州部夷梁承秀、李光吉、王兗導生獠入寇,轉運判官張詵請誅之。選構為使,倍道之官,至則遣渝州豪杜安行募千人往襲,自督官軍及黔中兵擊其後,斬承秀,入討二族,火其居。餘眾保黑崖嶺,黔兵從間道夜噪而進,
 光吉墜崖死,兗自縛降。以其地建南平軍。錄功加直昭文館。



 徙湖北轉運使。章惇興南、北江蠻事,構諭降懿、洽二州,納歸附州十四。初,渡辰溪,舟毀而溺,得援者僅免,神宗憫之。賜帛三百。北江酋彭師晏常持向背,構知向水酋彭儒武與有隙,檄使攻之。師晏降,得其下溪州地,五溪皆平。進集賢殿修撰,賜三品服。交□止入寇,拜右諫議大夫、知桂州,聲言將掎角搗其巢穴,寇聞引去。以疾提舉崇福宮,換太中大夫,卒,年六十四。



 構喜功名,勇於
 建立,西南邊事自此始云。



 張詵,字樞言,建州浦城人。第進士,通判越州。民患苦衙前役,詵科別人戶,籍其當役者,以差人錢為雇人充,皆以為便。知襄邑縣,擢夔路轉運判官。錄闢土之功,加直集賢院,改陜西轉運副使。召對,帝曰:「朕未識卿,每閱章奏,獨卿與蔡挺有所論請,使人了然。尋當以帥事相屬。」及入辭,賜服金紫。



 明年,直龍圖閣、知秦州。前此將吏貪功,多從羌地獵射,因起邊患。詵至,申令毋得犯,得一人,
 斬諸境上,群羌感悅,遷天章閣待制、知熙州。董氈遣鬼章逼岷州,詵往討,董氈迎戰,破之於錯鑿城,斬首萬級。



 元豐初,加龍圖閣直學士、知成都府,徙杭州。將行,復命權經略熙河事,趣使倍道行。時倉卒治戎,有司計產調夫,戶至累首,民多流亡。詵中塗訴其狀,乞敕劍外招攜之,不報。會靈武師罷,乃赴杭,道過京師,帝訪以西事,對曰:「彼勢雖弱,而我師未銳,邊備未飭,願以歲月圖功。」累官正議大夫,卒,年七十二。



 詵性孝友,廉於財,平生不殖
 田業。既建拓瀘夷地被進用後,雖有善言可紀,終不逭清議云。



 蘇寀字,公佐,磁州滏陽人。擢第,調兗州觀察推官,受知於守杜衍。為大理詳斷官。民有母改嫁而死,既葬,輒盜其柩歸祔,法當死。寀曰:「子取母祔父,豈與發塚取財等?」請而生之。



 遷審刑院詳議、御史臺推真官,知單州,提點梓州益州路刑獄、利路轉運使。文州歲市羌馬,羌轉買蜀貨,猾駔上下物價,肆為奸漁。寀議置折博務,平貨直
 以易馬,宿弊頓絕。



 入判大理寺,為湖北、淮南、成都路轉運使,擢侍御史知雜事,判刑部。使契丹,還及半道,聞英宗晏駕,契丹置宴仍用樂,寀謂送者曰:「兩朝兄第國家,君臣之義,吾與君等一也。此而可忍,孰不可忍。」遂為之徹樂。



 進度支副使,以集賢殿修撰知鳳翔。還,糾察在京刑獄,又出知潭州、廣州,累轉給事中,知河南府,無留訟。入知審刑院,卒。寀長於刑名,故屢為法官,數以讞議受詔獎焉。



 馬從先,字子野,祥符人。少盡力於學。父當任子,推以與其弟。由進士累官太常少卿、知宿州。宿在淮、汴間,素難治,從先以囊博者、重坐者厚賞以求盜。禁屠牛、鑄錢,嚴甚。大水,發廩振流亡,全活數十萬。代還,知壽州,以老辭,英宗諭遣之曰:「聞卿治行籍甚,壽尤重於宿,姑為朕往。」既至,治如曩時。由太子賓客轉工部侍郎致仕。從先性整嚴,雖盛夏不袒跣。晚學佛,預言其終時,年七十六而卒。



 論曰:長卿性務廉潔,以能臣稱,中師用法刻深,以治辨稱,雖均為材吏,而優劣自見。拯及仲甫俱能為國興利除害。構始開西南邊,詵遂拓瀘夷被進用,雖有他善。而不能逭清議。至於沆決河議,綏遠民,折鄰使,歷有可稱述者,其最優歟。



 沈遘,字文通,錢塘人,以蔭為郊社齋郎。舉進士,廷唱第一,大臣謂已官者不得先多士,乃以遘為第二。通判江寧府,歸,奏《本治論》。仁宗曰:「近獻文者率以詩賦,豈若此
 十篇之書為可用也。」除集賢校理。頃之,修起居注,遂知制誥。以父扶坐事免,求知越州,徙杭州。



 為人疏雋博達,明於吏治,令行禁止。民或貧不能葬,給以公錢,嫁孤女數百人,倡優養良家子者,奪歸其父母。善遇僚寀,皆甘樂傾盡為之耳目,刺閭巷長短,纖悉必知,事來立斷。禁捕西湖魚鱉,故人居湖上,蟹夜入其籬間,適有客會宿,相與食之,旦詣府,遘迎語曰:「昨夜食蟹美乎?」客笑而謝之。小民有犯法,情稍不善者,不問法輕重,輒刺為兵,奸
 猾屏息。提點刑獄鞫真卿將按其狀,遘為稍弛,而刺者復為民。



 嘉祐遺詔至,為次於外,不飲酒食肉者二十七日。召知開封府,遷龍圖閣直學士,治如在杭州。蚤作視事,逮午而畢,出與親舊還往,從容燕笑,沛然有餘暇,士大夫交稱其能。拜翰林學士、判流內銓。丁母憂,英宗閔其去,賚黃金百兩,仍命扶喪歸蘇州。既葬,廬墓下,服未竟而卒,年四十,世咨惜之。弟遼,從弟括。



 遼字睿達,幼挺拔不群,長而好學尚友,傲睨一世。讀左
 氏、班固書,小摹仿之。輒近似,乃鋤植縱舍,自成一家。趣操高爽,縹縹然有物外意,絕不喜進取。用兄任監壽州酒稅。吳充使三司,薦監內藏庫。熙寧初,分審官建西院,以為主簿,時方重此官,出則奉使持節。遼故受知於王安石,安石嘗與詩,有「風流謝安石,瀟灑陶淵明」之稱。至是當國,更張法令,遼與之議論,浸浸咈意,日益見疏,於是坐與其長不相能,罷去。



 久之,以太常寺奉禮郎監杭州軍資庫,轉運使使攝華亭縣。他使者適有夙憾,思中以
 文法,因縣民忿爭相牽告,辭語連及,遂文致其罪。下獄引服,奪官流永州,遭父憂不得釋。更赦,始徙池州。留連江湖間累年,益偃蹇傲世。既至池,得九華、秋浦間,玩其林泉,喜曰:「使我自擇,不過爾耳。」既築室於齊山之上,名曰雲巢,好事者多往游。



 遼追悔平生不自貴重,悉謝棄少習,杜門隱幾,雖筆硯亦埃塵竟日。間作為文章,雄奇峭麗,尤長於歌詩,曾鞏、蘇軾、黃庭堅皆與唱酬相往來,然竟不復起,元豐末,卒,年五十四。



 括字存中,以父任為沭陽主簿。縣依沐水,乃職方氏所書「浸曰沂、沭」者,故跡漫為污澤,括新其二坊,疏水為百渠九堰,以播節原委,得上田七千頃。



 擢進士第,編校昭文書籍,為館閣校勘,刪定三司條例。故事,三歲郊丘之制,有司按籍而行,藏其副,吏沿以干利。壇下張幔,距城數里為園囿,植採木、刻鳥獸綿絡其間。將事之夕,法駕臨觀,御端門、陳仗衛以閱嚴警,游幸登賞,類非齋祠所宜。乘輿一器,而百工侍役者六七十輩。括考禮沿革,為
 書曰《南郊式》。即詔令點檢事務,執新式從事,所省萬計,神宗稱善。



 遷太子中允、檢正中書刑房、提舉司天監,日官皆市井庸販,法象圖器,大抵漫不知。括始置渾儀、景表、五壺浮漏,招衛樸造新歷,募天下上太史占書,雜用士人,分方技科為五,後皆施用。加史館檢討。



 淮南饑,遣括察訪,發常平錢粟,疏溝瀆,治廢田,以救水患。遷集賢校理,察訪兩浙農田水利,遷太常丞、同修起居注。時大籍民車,人未諭縣官意,相手延為憂;又市易司患蜀鹽之
 不禁,欲盡實私井而輦解池鹽給之。言者論二事如織,皆不省,括侍帝側,帝顧曰:「卿知籍車乎?」曰:「知之。」帝曰:「何如?」對曰:「敢問欲何用?」帝曰:「北邊以馬取勝,非車不足以當之。」括曰:「車戰之利,見於歷世。然古人所謂兵車者,輕車也,五御折旋,利於捷速。今之民間輜車重大,日不能三十里,故世謂之太平車,但可施於無事之日爾。」帝喜曰:「人言無及此者,朕當思之。」遂問蜀鹽事,對曰:「一切實私井而運解鹽,使一出於官售,誠善。然患萬、戎、瀘間夷
 界小井尤多,不可猝絕也,勢須列候加警,臣恐得不足償費。」帝頷之。明日,二事俱寢。擢知制誥,兼通進、銀臺司,自中允至是才三月。



 為河北西路察訪使。先是,銀冶,轉運司置官收其利,括言:「近寶則國貧,其勢必然;人眾則囊橐奸偽何以檢頤?朝廷歲遺契丹銀數千萬,以其非北方所有,故重而利之。昔日銀城縣、銀坊城皆沒於彼,使其知鑿山之利,則中國之幣益輕,何賴歲餉,鄰釁將自茲始矣。」



 時賦近畿戶出馬備邊,民以為病,括言:「北地
 多馬而人習騎戰,猶中國之工強弩也。今舍我之長技,強所不能,何以取勝。」又邊人習兵,唯以挽強定最,而未必能貫革,謂宜以射遠入堅為法。如是者三十一事,詔皆可之。



 遼蕭禧來理河東黃嵬地,留館不肯辭,曰:「必得請而後反。」帝遣括往聘。括詣樞密院閱故牘,得頃歲所議疆地書,指古長城為境,今所爭蓋三十里遠,表論之。帝以休日開天章閣召對,喜曰:「大臣殊不究本末,幾誤國事。」命以畫圖標禧,禧議始屈。賜括白金千兩使行。至
 契丹庭,契丹相楊益戒來就議,括得地訟之籍數十,預使吏士誦之,益戒有所問,則顧吏舉以答。他日復問,亦如之。益戒無以應,謾曰:「數里之地不忍,而輕絕好乎?」括曰:「師直為壯,曲為老。今北朝棄先君之大信,以威用其民,非我朝之不利也。」凡六會,契丹知不可奪,遂舍黃嵬而以天池請。括乃還,在道圖其山川險易迂直,風俗之純龐,人情之向背,為《使契丹圖抄》上之。拜翰林學士、權三司使。



 嘗白事丞相府,吳充問曰:「自免役令下,民之詆
 訾者今未衰也,是果於民何如?」括曰:「以為不便者,特士大夫與邑居之人習於復除者爾,無足恤也。獨微戶本無力役,而亦使出錢,則為可念。若悉弛之,使一無所預,則善矣。」充然其說,表行之。



 蔡確論括首鼠乖刺,陰害司農法,以集賢院學士知宣州,明年,復龍圖閣待制、知審官院,又出知青州,未行,改延州。至鎮,悉以別賜錢為酒,命廛市良家子馳射角勝,有軼群之能者,自起酌酒以勞之,邊人歡激,執弓傅矢,唯恐不得進。越歲,得徹札超
 乘者千餘,皆補中軍義從,威聲雄他府。以副總管種諤西討援銀、宥功,加龍圖閣學士。朝廷出宿衛之師來戍,賞賚至再而不及鎮兵。括以為衛兵雖重,而無歲不戰者,鎮兵也。今不均若是,且召亂。乃藏敕書,而矯制賜緡錢數萬,以驛聞。詔報之曰:「此右府頒行之失,非卿察事機,必擾軍政。」自是,事不暇請者,皆得專之。蕃漢將士自皇城使以降,許承制補授。



 諤師次五原,值大雪,糧餉不繼,殿直劉歸仁率眾南奔,士卒二萬人皆潰入塞,居民
 怖駭。括出東郊餞河東歸師,得奔者數千,問曰:「副都總管遣汝歸取糧,主者為何人?」曰:「在後。」即諭令各歸屯。及暮,至者八百,未旬日,潰卒盡還。括出按兵,歸仁至,括曰:「汝歸取糧,何以不持軍符?」歸仁不能對,斬以狗。經數日,帝使內侍劉惟簡來詰叛者,具以對。



 大將景思誼、曲珍拔夏人磨崖葭蘆浮圖城,括議築石堡以臨西夏,而給事中徐禧來,禧欲先城永樂。詔禧護諸將往築,令括移府並塞,以濟軍用。已而禧敗沒,括以夏人襲綏德,先往
 救之。不能援永樂,坐謫均州團練副使。元祐初,徙秀州,繼以光祿少卿分司,居潤八年卒,年六十五。



 括博學善文,於天文、方志、律歷、音樂、醫藥、卜算,無所不通,皆有所論著。又紀平日與賓客言者為《筆談》,多載朝廷故實、耆舊出處,傳於世。



 李大臨,字才元,成都華陽人。登進士第,為絳州推官。杜衍安撫河東,薦為國子監直講、睦親宅講書。文彥博薦為秘閣校理。考試舉人,誤收失聲韻者,責監滁州稅。未
 幾,還故職。



 仁宗嘗遣使賜館閣官御書,至大臨家,大臨貧無皂隸,方自秣馬,使者還奏,帝曰:「真廉士也。」以親老,請知廣安軍,徙邛州。還,為群牧判官、開封府推官。



 神宗雅知其名,擢修起居注,進知制誥、糾察在京刑獄。言青苗法有害無益,王安石怒。會李定除御史,宋敏求、蘇頌相繼封還詞命,次至大臨,大臨亦還之。帝批:「去歲詔書,臺官不拘官職奏舉,後未審更制也。」頌、大臨合言:「故事,臺官必以員外郎、博士,近制但不限此,非謂選人亦許
 之也。定以初等職官超朝籍,躐憲臺,國朝未有。幸門一開,名器有限,安得人人滿其意哉。」復詔諭數四,頌、大臨故爭不已,乃以累格詔命,皆歸班,大臨以工部郎中出知汝州。



 辰溪貢丹砂,道葉縣,其二篋化為雙雉,斗山谷間。耕者獲之,人疑為盜,械送於府。大臨識其異,訊得實,釋耕者。徙知梓州,加集賢殿修撰,復天章閣待制。甫七十,致仕七年而卒。



 大臨清整有守,論議識大體,因爭李定後名益重,世並宋敏求、蘇頌稱為「熙寧三舍人」云。



 呂夏卿,字縉叔,泉州晉江人。舉進士,為江寧尉。編修《唐書》成,直秘閣、同知禮院。仁宗選任大臣,求治道,夏卿陳時務五事,且言:「天下之勢,不能常安,當於未然之前救其弊;事至而圖之,恐無及已。」朝廷頗採其策。



 英宗世,歷史館檢討、同修起居注、知制誥。帝嘗訪以政,對曰:「兩朝不惜金帛以和二邊,脫民鋒鏑之禍,古未有也。願勿失前好。」出知穎州,得奇疾,身體日縮,卒時才如小兒,年五十三。



 夏卿學長於史,貫穿唐事,博採傳記雜說數百家,
 折衷整比。又通譜學,創為世系諸表,於《新唐書》最有功云。



 祖無擇,字擇之,上蔡人。進士高第。歷知南康軍、海州,提點淮南廣東刑獄、廣南轉運使,入直集賢院。時封孔子後為文宣公,無擇言:「前代所封曰宗聖,曰奉聖,曰崇聖,曰恭聖,曰褒聖;唐開元中,尊孔子為文宣王,遂以祖謚而加後嗣,非禮也。」於是下近臣議,改為衍聖公。



 出知袁州。自慶歷詔天下立學,十年間其敝徒文具,無命教之
 實。無擇首建學官,置生徒,郡國弦誦之風。由此始盛。同修起居注、知制誥,加龍圖閣直學士、權知開封府,進學士,知鄭、杭二州。



 神宗立,知通進、銀臺司。初,詞臣作誥命,許受潤筆物。王安石與無擇同知制誥,安石辭一家所饋不獲,義不欲取,置諸院梁上。安石憂去,無擇用為公費,安石聞而惡之。熙寧初,安石得政,乃諷監司求無擇罪。知明州苗振以貪聞,御史王子韶使兩浙,廉其狀,事連無擇。子韶,小人也,請遣內侍自京師逮赴秀州獄。蘇
 頌言無擇列侍從,不當與故吏對曲直,御史張戩亦救之,皆不聽。及獄成,無貪狀,但得其貸官錢、接部民坐及乘船過制而已。遂謫忠正軍節度副使。安石猶為帝言:「陛下遣一御史出,即得無擇罪,及知朝廷於事但不為,未有為之而無效者。」尋復光祿卿、秘書監、集賢院學士,主管西京御史臺,移知信陽軍,卒。



 無擇為人好義,篤於師友,少從孫明復學經術,又從穆修為文章。兩人死,力求其遺文匯次之,傳於世。以言語政事為時名卿,用小
 累鍛煉放棄,訖不復振,士論惜之。



 論曰:沈遘以文學致身,而長於治才。沉括博物洽聞,貫乎幽深,措諸政事,又極開敏。呂夏卿號稱史才,尤精譜諜之學。宋之縉紳,士各精其能,學不茍且,故能然也。李大臨官居繳駁,克舉其職;祖無擇治郡所至,能修校官,是皆班班可紀者。然大臨以論李定絀,無擇以忤安石廢棄終身,即是亦足以知二人之賢矣。



 程師孟,字公闢,吳人。進士甲科。累知南康軍、楚州,提點
 夔路刑獄。瀘戎數犯渝州,邊使者治所在萬州,相去遠,有警率浹日乃至,師孟奏徙于渝。夔部無常平粟,建請置倉,適兇歲,振民不足,即矯發他儲,不俟報。吏懼,白不可。師孟曰:「必俟報,餓者盡死矣。」竟發之。



 徙河東路。晉地多土山,旁接川谷,春夏大雨,水濁如黃河,俗謂之「天河」,可溉灌。師孟出錢開渠築堰,淤良田萬八千頃,裒其事為《水利圖經》,頒之州縣。為度支判官,知洪州,積石為江堤,浚章溝,揭北閘以節水升降,後無水患。



 判三司都磨
 勘司。接伴契丹使,蕭惟輔曰:「白溝之地當兩屬,今南朝植柳數里,而以北人漁界河為罪,豈理也哉?」師孟曰:「兩朝當守誓約,涿郡有案牘可覆視,君舍文書,滕口說,遽欲生事耶?」惟輔愧謝。



 出為江西轉運使。盜發袁州,州吏為耳目,久不獲。師孟械吏數輩送獄,盜即成擒。加直昭文館、知福州。築子城,建學舍,治行最東南。徙廣州,州城為儂寇所毀,他日有警,民駭竄,方伯相踵至,皆言土疏惡不可築。師孟在廣六年,作西城。及交址陷邕管,聞廣
 守備固,不敢東。時師孟已召還,朝廷念前功,以為給事中、集賢殿修撰、判都水監。



 賀契丹生辰,至涿州,契丹命席,迎者正南向,涿州官西向,宋使介東向。師孟曰:「是卑我也。」不就列。自日昃爭至暮,從者失色,師孟辭氣益厲,叱儐者易之,於是更與迎者東西向。明日,涿人餞於郊,疾馳過不顧;涿人移雄州,以為言,坐罷歸班。復起知越州、青州,遂致仕,以光祿大夫卒,年七十八。



 師孟累領劇鎮,為政簡而嚴,罪非死者不以屬吏。發隱擿伏如神,得
 豪惡不逞跌宕者,必痛懲艾之,至剿絕乃已,所部肅然。洪、福、廣、越為生立祠。



 張問,字昌言,襄陽人也。進士起家,通判大名府。群牧地在魏,歲久冒入於民,有司按舊籍括之,地數易主,券不明,吏茍趣辦,持詔書奪人田,至毀室盧、發丘墓。問至,則曰:「是豈朝廷意耶?」其上以聞。仁宗諭大臣曰:「吏用心悉如問,何患赤子之不安也。」立罷之。



 擢提點河北刑獄。大河決,議築小吳,問言:「曹村、小吳南北相直,而曹村當水
 沖,賴小吳堤薄,水溢北出,故南堤無患。若築小吳,則左強而右傷,南岸且決,水並京畿為害,獨可於孫、陳兩埽間起堤以備之耳。」詔付水官議,久不決,小吳卒潰。



 徙江東、淮南轉運使,加直集賢院、戶部判官,復為河北轉運使。所部地震,河再決,議者欲調京東民三十萬,自澶築堤抵乾寧。問言:「堤未能為益,災傷之餘,力役勞民,非計也。」神宗從之。問十年不奏考課,詔特遷其官,入為度支副使,拜集賢殿修撰、河東轉運使。坐誤軍須,貶知光化
 軍,未幾,復使河北。諸葛公權之亂,郡縣株蔓,連逮至數百千人,問上疏申理,止誅首惡。



 熙寧末,知滄州。自新法行,問獨不阿時好。歲饑,為帝言民茍免常平、助役之苦,反以得流亡為幸,語切直驚人。元豐定官制,王安禮薦問可任六曹侍郎,帝以其好異論,不用。歷知河陽、潞州。元祐初,為秘書監、給事中,累官正議大夫,卒,年七十五。



 問處己廉潔、嘗仕鄜延幕府,與種世衡善,父喪,世稀遺汝州田十頃,辭弗受。使歸,未至而世衡卒。其子古,用父
 治命,亦不納田,蕪穢者三十年。後汝守請以給學,朝廷命反諸種氏。



 熙寧時,有陳舜俞、樂京、劉蒙,亦以役法廢黜。



 舜俞,字令舉,湖州烏程人。博學強記。舉進士,又舉制科第一。熙寧三年,以屯田員外郎知山陰縣,詔俟代還試館職。舜俞辭曰:「爵祿名器,砥礪多士,宜示以至神,烏可要期如付劑契?」繳中書帖上之。



 青苗法行,舜俞不奉令,上疏自劾曰:「民間出舉財物,取息重止一倍,約償緡錢,
 而穀粟、布縷、魚鹽、薪蔌、櫌鋤、釜錡之屬,得雜取之。朝廷募民貸取,有司約中熟為價,而必償緡錢,欲如私家雜償他物不可得,故愚民多至賣田宅、質妻孥。有識耆老,戒其鄉黨子弟,未嘗不以貰貸為苦。祖宗著令,以財物相出舉,任從書契,官不為理。其保全元元之意,深遠如此。今誘之以便利,督之以威刑,方之舊法,異矣。詔謂振民乏絕而抑兼並,然使十戶為甲,浮浪無根者毋得給俵,則乏絕者已不蒙其惠。此法終行,愈為兼並地爾。何
 以言之?天下之有常平,非能人人計口受餉,但權穀價貴賤之柄,使積貯者不得深藏以邀利爾。今散為青苗,唯恐不盡,萬一饑饉薦至,必有乘時貴糶者,未知將何法以制之?官制既放錢取息,富室藏鏹,坐待鄰里逋欠之時,田宅妻孥隨欲而得,是豈不為兼並利哉。雖分為夏秋二科,而秋放之月與夏斂之期等,夏放之月與秋斂之期等,不過展轉計息,以給為納,使吾民終身以及世世,每歲兩輸息錢,無有窮已。是別為一賦以敝海內,
 非王道之舉也。」奏上,責監南康軍鹽酒稅,五年而卒。



 舜俞始嘗棄官歸,居秀之白牛村,自號白牛居士。已而復出,遂貶死。蘇軾為文哭之,稱其「學術才能,兼百人之器,慨然將以身任天下之事,而人之所以周旋委曲、輔成其天者不至。一斥不復,士大夫識與不識,皆深悲之」云。



 京,荊南人。為布衣時,鄉里稱其行義,事母至孝。妻張氏家絕,挾女弟自隨,京未嘗見其面。妻死,京寢食於外,為嫁之。嘉祐初,詔訪遺逸,以薦聞,得校書郎,為湖陽、赤水
 二縣令。神宗求言,京上疏以畏天保民為請。知長葛縣。助役法行,京曰:「提舉常平官言不便。」使之條析,又不報,且不肯治縣事,自列丐去。提舉官劾之,詔奪著作佐郎。經十年,乃復官,監黃州酒稅,以承議郎致仕。元祐初,召赴闕,不至,終於家。



 蒙字子明,渤海人。恥為詞賦,不肯舉進士;習茂才異等,又不欲自售。都轉運使劉庠舉遺逸,召試第一,知湖陽縣。常平使者召會諸縣令議免役法,蒙為不便,不肯與
 議,退而條上其害,即投劾去,亦奪官。歸鄉教授,養親講學,從游甚眾。元豐二年,卒,才年四十。門人朋友誄其行,號曰正思先生。元祐初,賜其家帛五十匹。



 苗時中,字子居,其先自壺關徙宿州。以蔭主寧陵簿。邑有古河久堙,請開導以溉田,為利甚博,人謂之苗公河。



 調潞州司法參軍。郡守欲入一囚於死,執不可。守怒,責甚峻,時中曰:「寧歸田里,法不可奪。」守悟而聽之。熙寧中,以司農丞使梓州路,密薦能吏十人,後皆進用,人卒莫
 之知。



 交人犯邊,擢廣西轉運副使。師討交人罪,次富良江,久不進。時中曰:「師無進討意,賊必從間道來,乘我不備,冀萬一之勝,勢窮然後降耳。」密備之,既而果從上流來,戰敗,始納款。



 徙梓州轉運副使。韓存寶討蠻乞弟,逗遛不行。時中曰:「師老矣,將士暴露,非計之善者。」存寶不聽,卒坐誅。林廣代存寶。乞弟既降,復逸去,將士相視失色。及暮,刁鬥不鳴,時中問廣,廣曰:「既失賊,故縱兵追之,不暇恤爾。」時中曰:「天子以十萬眾相付,豈以一死為勇
 耶。今入異境,變且不測。」廣悟,亟止追者,整軍以進。會得詔班師,軍行,時中以糧道遠,創為手贊運法,食以不乏。遷兩階,為發運副使、河東轉運使,加直龍圖閣、知桂州,進寶文閣待制,至戶部侍郎,卒。



 韓贄,字獻臣,齊州長山人。登進士第,至殿中侍御史。坐微累,黜監江州稅。道除知睦州,復為侍御史。荊湖災,出持節安撫。湘中自馬氏擅國,計丁輸米,身死產竭不得免,贄奏除之。改知諫院,進天章閣待制。宰相梁適以私
 容奸,狄青起卒伍、位樞密,內侍王守忠遷官不次,皆舉劾無所諱。



 出知滄、瀛二州,遷龍圖閣直學士、河北都轉運使。河決商胡而北,議者欲復之。役將興,贄言:「北流既安定,驟更之,未必能成功。不若開魏金堤使分注故道,支為兩河,或可紓水患。」詔遣使相視,如其策,才役三千人,幾月而畢。入判都水監,權開封府,政簡而治。知河南府,建永厚陵,費省而不擾,神宗稱之。還知審刑院、糾察在京刑獄,知徐州,以吏部侍郎致仕。



 贄性行淑均,平居
 自奉至約,推所得祿賜買田贍族黨,賴以活者殆百數。退休十五年,謝絕人事,讀書賦詩以自娛。年八十五,卒。



 楚建中,字正叔,洛陽人。第進士,知榮河縣。民苦鹽稅不平,建中約田多寡以為輕重。主管鄜延經略機宜文字。夏人來正土疆,往蒞其事。眾暴至,兩騎傅矢引滿向之,建中披腹使射,曰:「吾不憚死。」騎即去,眾服其量。元昊歸款,建中白府請築安定、黑水八堡以控東道,夏人果來,聞有備,不敢入。累遷提點京東刑獄、鹽鐵判官。昭陵建,
 命裁定調度,省數十萬計。歷夔路、淮南、京西轉運使,進度支副使。



 神宗用事西鄙,以建中嘗為邊臣所薦,召欲用之,言不合旨,出知滄州。久之,為天章閣待制、陜西都轉運使,知慶州、江寧、成德軍,以正議大夫致仕。元祐初,文彥博薦為戶部侍郎,不拜。卒,年八十一。



 張頡,字仲舉,其先金陵人,徙鼎州桃源。第進士,調江陵推官。歲旱饑,朝廷遣使安撫,頡條獻十事,活數萬人。知益陽縣,縣接梅山溪峒,多蠻獠出沒,頡按禁地約束,召
 徭人耕墾,上其事,不報。累遷開封府判官、提點江西刑獄、廣東轉運使。



 熙寧中,章惇取南江地,建沅、懿等州,克梅山,與楊光僭為敵。頡居憂於鼎,移書朝貴,言南江殺戮過甚,無辜者十八九,浮尸蔽江,民不食魚者數月。惇疾其說,欲分功啖之。乃言曰:「頡昔令益陽,首建梅山之議,今日成功,權輿於頡。」詔賜絹三百匹。尋擢江、淮制置發運副使,改知荊南,復徙廣西轉運使。時建廣源為順州,將城之,頡謂無益,朝廷從其議。坐捽罵參軍沉竦罷
 歸。



 未幾,以直集賢院知齊、滄二州,進直龍圖閣、知桂州。入覲,帝首言:「卿鄉者論順州不可守,信然。」時有獻言者謂:「海南黎人陳被蓋五洞酋領,異時盛強,且為中國患。今請出兵自效,宜有以撫納之。」命頡處其事。頡使一介往呼之。出,補以牙校,喜而去。詔問何賞之薄,對曰:「荒徼蠻蜒無他覬,得是足矣。」尋罷兵,海外訖無事。



 久之,轉運使馬默劾其經理宜州蠻事失宜,罷職知均州。哲宗立,還故職,知鳳翔、廣州,召為戶部侍郎。



 頡所歷以嚴致理,
 而深文狡獪。右司諫蘇轍論其九罪,執政以頡雖無德而才可用,不報。逾年,以寶文閣待制出為河北都轉運使,徙知瀛州。湖北溪猺畔,朝廷托頡素望,復徙知荊南,至都門,暴卒。



 盧革,字仲辛,湖州德清人。少舉童子,知杭州馬亮見所為詩,嗟異之。秋,貢士,密戒主司勿遺革。革聞,語人曰:「以私得薦,吾恥之。」去弗就。後二年,遂首選;至登第,年才十六。



 慶歷中,知龔州。蠻入寇,桂管騷動,革經畫軍須,先事
 而集。移書安撫使杜杞,請治諸郡城,及易長吏之不才者。又言:「嶺外小郡,合四五不當中州一大縣,無城池甲兵之備,將為賊困,宜度遠近並省之。」後儂智高來,九郡相繼不守,皆如革慮。



 知婺、泉二州,提點廣東刑獄、福建湖南轉運使。復請外,神宗謂宰相曰:「革廉退如是,宜與嘉郡。」遂為宣州。以光祿卿致仕。用子秉恩轉通議大夫,退居於吳十五年。秉為發運使,得請歲一歸覲。後帥渭,乞解官終養。帝數賜詔慰勉,時以為榮。卒,年八十二。



 秉字仲甫,未冠,有雋譽。嘗謁蔣堂,坐池亭,堂曰:「亭沼粗適,恨林木未就爾。」秉曰:「亭沼如爵位,時來或有之;林木非培植根株弗成,大似士大夫立名節也。」堂賞味其言,曰:「吾子必為佳器。」



 中進士甲科,調吉州推官、青州掌書記、知開封府倉曹參軍,浮湛州縣二十年,人無知者。王安石得其壁間詩,識其靜退,方置條例司,預選中。奉使淮、浙治鹽法,與薛向究索利病,出本錢業鬻海之民,戒不得私鬻,還奏,遂為定制。



 檢正吏房公事,提點兩浙、淮
 東刑獄,顓提舉鹽事,持法苛嚴,追胥連保,罪及妻孥,一歲中犯者以千萬數。進制置發運副使。東南饑,詔損上供米價以糴。秉言:「價雖賤,貧者終艱得錢,請但償糴本,而以其餘振贍。」是歲上計,神宗問曰:「聞滁、和民捕蝗充食,有諸?」對曰:「有之,民饑甚,殍死相枕籍。」帝惻然曰:「前此獨趙抃為朕言之耳。」先是,發運使多獻餘羨以希恩寵,秉言:「職在董督六路財賦,以時上之,安得羨。今稱羨者,率正數也。請自是罷獻,獨以七十萬緡償三司逋。」



 加集
 賢殿修撰、知謂州。五路大出西討,唯涇原有功,進寶文閣待制。夏境胡盧川距塞二百里,恃險遠不設備,秉遣將姚麟、彭孫襲擊之。俘斬萬計。遷龍圖閣直學士。夏酋仁多嵬丁舉國入寇,犯熙河定西城,秉治兵瓦亭,分兩將駐靜邊砦,指夏人來路曰:「吾遲明坐待捷報矣。」及明果至,見宋師,驚曰:「天降也。」縱擊之,皆奔潰。或言嵬丁已死,有識其衣服者,諸將請以聞。秉曰:「幕府上功患不實,吾敢以疑似成欺乎?」他日物色之,嵬丁果死,詔褒賜服
 馬、金幣,且使上所獲器甲。



 秉守邊久,表父革年老,乞歸。移知湖州,行三驛,復詔還渭,慰藉優渥。革聞,亦以義止其議。已而革疾亟。乃得歸。元祐中,知荊南。劉安世論其行鹽法虐民,降待制、提舉洞霄宮,卒。



 論曰:宋室之人才亦盛矣。青苗法始行,滿朝耆壽故臣、法家拂士,引古今通誼,盡力爭之而不能止,往往多自引去。及數年之後,憲令既成,天下亦莫如之何。已而間守遠郡,尚能懇懇為民有言。舜俞、京、蒙俱以區區一縣
 令,力抗部使者,視棄其官如弊屣,類非畏威懷祿者能之。師孟活饑羸,興水利,擿奸誅惡,所歷可稱;逮使契丹,正坐席禮,毅然不少屈。時中止林廣縱兵追蠻,深達兵家之變。贄居諫省,舉劾無所避,允有直臣之風。建中雅量卻敵,辭嚴氣正,尤為奇偉。頡雖有才,而深文狡獪,豈其天性然。革始終廉退,秉不免於阿徇時好,行鹽法以虐民,父子之相遠哉。



\end{pinyinscope}