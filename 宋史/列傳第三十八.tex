\article{列傳第三十八}

\begin{pinyinscope}

 王繼忠,開封人。父充,為武騎指揮使,戍瓦橋關,卒。繼忠
 年六歲,補東班殿侍。真宗在藩邸,得給事左右,以謹厚被親信。即位,補內殿崇班,累遷至殿前都虞候,領雲州觀察使,出為深州副都部署,改鎮、定、高陽關三路鈐轄兼河北都轉運使,遷高陽關副都部署,俄徙定州。



 咸平六年,契丹數萬騎南侵,至望都,繼忠與大將王超及桑讚等領兵援之。繼忠至康村,與契丹戰,自日失至乙夜,敵勢小卻。遲明復戰,繼忠陣東偏,為敵所乘,斷餉道,超、讚皆畏縮退師,竟不赴援。繼忠獨與麾下躍馬馳赴,
 服飾稍異,契丹識之,圍數十重。士皆重創,殊死戰,且戰且行,旁西山而北,至白城,遂陷於契丹。真宗聞之震悼,初謂已死,優詔贈大同軍節度,賵賻加等,官其四子。



 景德初,契丹請和,令繼忠奏章,乃知其尚在。朝廷從之,自是南北戢兵,繼忠有力焉。歲遣使至契丹,必以襲衣、金帶、器幣、茶藥賜之,繼忠對使者亦必泣下。嘗附表懇請召還,上以誓書約各無所求,不欲渝之,賜詔諭意。契丹主遇繼忠甚厚,更其姓名為耶律顯忠,又改名宗信,封
 楚王,後不知其所終。子懷節、懷敏、懷德、懷政。



 真宗宮邸攀附者,繼忠之次有王守俊至濟州刺史,蔚昭敏至殿前都指揮使、保靜軍節度,翟明至洺州團練使,王遵度至磁州團練使,楊保用至西上閣門使、康州刺史,鄭懷德至御前忠佐馬步軍都軍頭、永州團練使,張承易至禮賓使,吳延昭至供備庫使,白文肇至引進使、昭州團練使,彭睿至侍衛馬軍副都指揮使、武昌軍節度,靳忠至侍衛馬軍都虞候、端州防禦使,郝榮至安國軍節
 度觀察留後,陳玉至冀州刺史,崔美至濟州團練使,高漢美至鄭州團練使,楊謙至御前忠佐馬步軍副都軍頭、河州刺史。



 傅潛,冀州衡水人。少事州將張廷翰。太宗在藩邸,召置左右。即位,隸殿前左班,三遷東西班指揮使。征太原,一日,再中流矢。又從征范陽,先到涿州,與契丹戰,生擒五百餘人。翌日,上過其所,見積屍及所遺器仗,嘉歎之。師旋,擢為內殿直都虞候。上對樞密言:「潛從行有勞,賞薄。」
 復加馬步都軍頭、領羅州刺史,改捧日右廂都指揮使、領富州團練使,遷日騎、天武左右廂都指揮使,領雲州防禦使。



 雍熙三年,命大將曹彬北征,以潛為幽州道行營前軍馬步軍都指揮使。師敗於拒馬河,責授右領軍衛大將軍,自檢校司徒降為右僕射,仍削功臣爵邑。明年,起為內外馬步都軍頭、領藩州防禦使,尋拜殿前都虞候、領容州觀察使。端拱初,加殿前副都指揮使、領昭化軍節度,出為高陽關都部署。淳化二年四月,拜侍衛
 馬步軍都虞候、領武成軍節度。至道中,出為延州路都部署,改鎮州。



 真宗即位,領忠武軍節度,數月召還。咸平二年,復出為鎮、定、高陽關三路行營都部署。契丹大入,緣邊城堡悉飛書告急,潛麾下步騎凡八萬餘,咸自置鐵撾、鐵棰,爭欲奮擊。潛畏懦無方略,閉門自守,將校請戰者,則醜言罵之。無何,契丹破狼山砦,悉銳攻威虜,略寧邊軍及祁、趙遊騎出邢、洺,鎮、定路不通者逾月。朝廷屢間道遣使,督其出師,會諸路兵合擊,范廷召、桑讚、秦
 翰亦屢促之,皆不聽。廷召等怒,因詬潛曰:「公恇怯乃不如一嫗爾。」潛不能答。都鈐轄張昭允又屢勸潛,潛笑曰:「賊勢如此,吾與之角,適挫吾銳氣爾。」然不得已,分騎八千、步二千付廷召等,於高陽關逆擊之,仍許出兵為援。洎廷召等與契丹血戰而潛不至,康保裔遂戰死。



 及車駕將親征,又命石保吉、上官正自大名領前軍赴鎮、定與潛會。潛卒逗遛不發,致敵騎犯德、棣,渡河湊淄、齊,劫人民,焚廬舍。上駐大名而邊捷未至,且諸將屢請益
 兵,潛不之與。有戰勝者,潛又抑而不聞。上由是大怒,乃遣高瓊單騎即軍中代之,令潛詣行在。至,則下御史府,命錢若水同劾按,一夕獄具。百官議法當斬,從駕群臣多上封請誅之。上貸其死,下詔削奪潛在身官爵,並其家屬長流房州。潛子內殿崇班從範亦削籍隨父流所,仍籍沒其貲產。五年,會赦,徙汝州。景德初,起為本州團練副使,改左千牛衛上將軍,分司西京。大中祥符四年,車駕西巡至洛,因令從駕還京,遷左監門大將軍,還其宅。
 久之,判左金吾街仗。天禧元年,卒。



 張昭允者,字仲孚,衛州人。以父秉蔭,試大理評事。潘美妻以女,奏換右班殿直,以久次,遷通事舍人。端拱初,契丹內擾,命為雄州監軍。敵騎乘秋掠境上,昭允與知州田仁朗選銳卒襲其帳,敗走之。進西上閣門副使,提總左右藏金銀錢帛。



 昭允以諸州絹常度外長數尺,請裂取付工官備他用,歲獲羨餘。既而土卒受冬服,度之不及程,出怨言,昭允坐免官。俄起為崇儀副使,累遷西上
 閣門使、河西馬步軍鈐轄,屯石州。會討李繼遷,王超出夏、綏州路,領後陣,超深入數百里,逾白池,道阻糧絕,昭允以所部援之,戎人大敗。



 真宗即位,以昭允章懷皇后姊婿,頗被親信。咸平二年,命為鎮、定、高陽關行營馬步都鈐轄。時傅潛為都部署,畏愞城守,昭允屢勸其出兵,潛按兵不動。潛既得罪,昭允亦削奪官爵,長流道州。景德二年,起為楚州團練副使,改右神武將軍。大中祥符元年,卒。



 昭允喜筆劄,習射,曉音律。子正中、居中。



 戴興,開封雍丘人。年十餘歲,以勇力聞里中。及長,身長七尺餘,美髭髯,眉目如畫。太宗在藩邸,興詣府求見,奇之,留帳下。即位,補禦馬左直,遷直長,再遷禦龍直副指揮使。從征太原,先登,中流矢,補禦龍弓箭直指揮使,遷都虞候。一日,帝問興曰:「汝頗有尊屬否?」對曰:「臣父延正、兄進皆力田。」即召延正為諸衛將軍,進為天武軍使。俄以興領嚴州刺史,改天武左廂都指揮使、領勝州團練使。



 雍熙三年,曹彬等北征失律,諸將多坐黜免,以興為
 侍衛步軍都虞候,領雲州防禦使。契丹撓邊,命興屯澶州以備非常,改本州觀察使,充天雄軍副都部署。



 端拱初,遷步軍都指揮使、領鎮武軍節度,賜襲衣、金帶、鞍勒馬。曆澶州,天雄軍都部署,改殿前副都指揮使,出帥鎮、定二州。時盜賊群起,會五巡檢兵討之,逾月不能克。興陰勒所部潛出擊之,擒戮殆盡。未幾,徙高陽關,遷殿前都指揮使,領定國軍節度,賜白金萬兩,歲加給錢七百萬。



 淳化五年,出為定武軍節度,歲加給錢千萬。西北未
 平,徙夏州路行營都部署、知州事。時五路討李繼遷,興所部深入千餘里,不見賊。會太宗崩,三上表求赴國哀,不俟報上道。及至京師,以擅離所部,左遷左領衛上將軍。咸平初,兼判左金吾街仗,俄出知京兆府,卒。贈太尉,遣中使護其喪歸葬鄉里。錄其子永和、永豐。



 王漢忠,字希傑,徐州彭城人。少豪蕩,有膂力,形質魁岸,善騎射。節帥高繼衝欲召至帳下,漢忠不往。因毆殺里中少年,遂亡。經宿復蘇,其父遣人追及於蕭縣,漢忠不
 肯還,西至京師。太宗在藩邸,召見,奇其材力,置左右。即位,補殿前指揮使,累遷內殿直都知。從征太原,先登,流矢中眸,戰益急,上壯之,遷東西班指揮使。劉繼元降,以所部安撫城中。師還,改殿前左班指揮使,三遷右班都虞候、領涿州刺史。雍熙中,改馬步軍都軍頭。端拱初,出為賓州團練使,曆冀、貝二州部署,徙天雄軍。二年,入為侍衛馬軍都虞候、領洮州觀察使、高陽關副都部署。契丹南侵,漢忠合諸軍擊敗之,斬馘甚眾。淳化初,徙定州。
 五年,遷殿前都虞候。



 真宗即位,自中山召歸。俄復出為高陽關都部署,進領威塞軍節度。咸平三年,又為涇原、環慶兩路都部署兼安撫使,遷侍衛馬軍都指揮使,改鎮、定、高陽關都部署、三路都排陣使。契丹掠中山,漢忠率諸將陣於野,契丹遁,追斬甚眾,獲其貴將。加殿前副都指揮使,改領保靜軍節度。



 五年,罷西面經略使,命漢忠為邠寧、環慶兩路都部署,李允正、宋沆為鈐轄,領戍兵二萬五千人,委漢忠分道控制。數月召還,坐違詔無
 功,責為左屯衛上將軍、出知襄州,常奉外增歲給錢二百萬。未上道,暴得疾卒。贈太尉,以其長子內殿崇班從吉為閣門祗候,次子從政、從益為左右侍禁。



 漢忠有識略,軍政甚肅,每行師,詰旦,必行香祝曰:「願軍民無犯吾令,違者一毫不貸。」故所部無盜。性剛果,不務小節,輕財樂施。好讀書,頗能詩。喜儒士,待賓佐有禮,名稱甚茂,以是自矜尚,群帥不悅。



 漢忠沒後,其子從吉詣闕上書訟父冤,因歷詆群臣有行賂樹黨及蒙蔽邊防屯戍艱苦
 之事。真宗命樞密王繼英等問狀,從吉止誦狀中語,他無所對。上以從吉付御史,具伏,乃進士楊逢為之辭。從吉坐除名,配隨州;逢杖配春州。



 王能,廣濟定陶人。初事州將袁彥,太宗在晉邸,召置左右。即位,補內殿直,六遷至殿前左班指揮使,進散員都虞候。久之,領潘州刺史,再遷殿前右班都虞候兼御前忠佐馬步軍都軍頭。咸平初,自捧日右廂都指揮使出為濟州團練使、知靜戎軍。建議決鮑河,斷長城口,北注
 雄州塘水,為戎馬限,方舟通漕,以實塞下。又開方田,盡靜戎、順安之境。北邊來寇,能擊走之。



 初,真宗詢軍校勤勇者,委以方面,因語宰相曰:「聞王能、魏能頗宣力公家,陳興、張禹珪亦有聲於時,才固難全,拔十得五,亦有助也。」景德初,擢本州防禦使,與魏能、張凝並命出為邢洺路都部署,俄改鎮、定、高陽關三路行營都部署、押策先鋒。護城祁州,躬率丁夫,旦暮不離役所,宴犒周洽。會詔使自北至者言之,手詔褒飭,連徙天雄軍、高陽關二部
 署,改定州副都部署。



 大中祥符二年,詔合鎮、定兩路部署為一,命能領之。明年召入,拜侍衛步軍副都指揮使、領曹州觀察使。祀汾陰,留為京城巡檢兼留司殿前司事。禮成,加領振武軍節度,復為鎮、定副都部署兼知定州。八年,表求入覲,許之。



 先是,節帥陛見,必飲於長春殿,掌兵者則不預。至是,特令用藩臣例。有司言:「能既赴坐,則殿前馬軍帥皆當侍立。」由是特令諸帥預坐,自是掌兵者率以為例。俄還屯所,改領靜江軍節度。天禧元年,
 轉都指揮使、領保靜軍節度。是冬代還,入見,以足疾免舞蹈,賜宴。累表求解,特與告醫療。二年,制授彰信軍節度,罷軍職赴鎮,以地近其鄉里,寵之也。明年,卒,年七十八。贈太尉,而錄其子守信等官。



 張凝,滄州無棣人。少有武勇,倜儻自任。鄉人趙氏子以材稱,凝恥居其下,因挾弓與角勝負。約築土百步射之,凝一發洞過,矢激十許步,抵大樹而止,觀者歎服。節帥張美壯之,召置帳下。太宗在藩邸,聞其名,以隸親衛。即
 位,補殿前指揮使,稍遷散祗候班都虞候。



 淳化初,以其有材幹,與王斌、王憲並授洛苑使,凝領繡州刺史,賜襲衣、金帶,每頒賚必異等。出為天雄軍駐泊都監,移貝州,改高陽關行營鈐轄、六宅使。真宗踐祚,加莊宅使,遷北作坊使。



 咸平初,契丹南侵,凝率所部兵設伏於瀛州西,出其不意,腹背奮擊,挺身陷敵。凝子昭遠,年十六,從行。即單騎疾呼,突入陣中,掖凝出,左右披靡不敢動。明年,契丹兵大至,車駕幸大名,凝與范廷召於莫州東分據
 要害,斷其歸路。契丹宵遁,凝縱兵擊之,盡奪所掠生口、資畜。徙鎮、定、高陽關路前陣鈐轄,遷趙州刺史。



 四年,召還,代潘璘為邠寧環慶靈州路副部署兼安撫使。時斥堠數擾,轉運使劉綜懼飛免不給,問計於凝。凝曰:「今當深入,因敵資糧,不足慮也。」乃自白豹鎮率兵入敵境,生擒賊將,燒蕩三百餘帳、芻糧八萬,斬首五千餘,獲牛馬、器甲二萬,降九百餘人。慶州蕃族胡家門等桀黠難制,凝因襲破之。又熟戶與生羌錯居,頗為誘脅,凝引兵至
 八州原、分水嶺、柔遠鎮,降峇<者多>等百七十餘族,合四千戶,邊境獲安。就加寧州團練使。



 景德初,遷本州防禦使,代楊嗣為定州路行營副部署,徙保州駐泊,又兼北面安撫使。時王超為總帥,以大兵頓中山,朝議擇凝與魏能、田敏、楊延昭分握精騎,俟契丹至,則深入以牽其勢。超嘗請四人悉隸所部,上以本設奇兵撓敵之心腹,若復取裁大將,則無以責效,乃令凝等不受超節度。時魏能逗撓,退保城堡,眾皆憤悱,責讓能,凝獨默然。或問之,
 凝曰:「能粗材險愎,既不為諸君所容,吾復切言之,使其心不自安,非計也。」上聞而嘉其有識。



 車駕觀兵澶淵,凝率眾抵易州。既而契丹受盟北歸,所過猶侵剽不已,遂以凝為緣邊安撫使,提兵躡其後,契丹乃不敢略奪。改高陽關部署。明年議勞,就加殿前都虞候,卒。



 凝忠勇好功名,累任西北,善訓士卒,繕完器仗,前後賞賜多以犒師,家無餘資,京師無居第。真宗悼惜之,贈彰德軍節度,遣中使護喪還京,官給葬事,厚恤其家。子昭遠。



 魏能,鄆人也。少應募,隸雲騎軍,後選補日騎左射,又隸殿前班,七遷散員左班都知。舊制,諸軍辭見,才器勇敢或迥異出群者,許將校交舉以任,使毋枉其志。能時戍外藩,咸未有舉者。太宗曰:「能材勇過人,朕可自保。」由是進用之。



 端拱二年,加御前忠佐馬軍副都軍頭,歷殿前左班都虞候、領溪州刺史,加秩轉馬步軍都軍頭。咸平三年,真拜黃州刺史。明年,為鎮、定、高陽關三路前陣鈐轄。五年,知鄭州團練使,復任威虜軍。



 契丹入寇,能當
 城西,與諸將合戰,無憚色,大敗其眾,斬首二萬級。契丹統軍鐵林相公來薄陣,能發矢殪之,並其將十五人,奪甲馬、兵械益眾。契丹復入,能率州軍逆戰南關門,遣其子正與都監劉知訓間道絕敵行勢,戰數十合,退薄西山下,破走之,獲器甲十八萬。契丹嘗謀入鈔,能偵知,即發兵逆擊,生擒酋帥,殄滅殆盡。



 六年,改威虜軍部署、知軍事。士民詣闕下乞留能,詔嘉之。能建言戍卒逸邊境者,請沒其妻與子為奴婢。上慮嚴迫,聽緩期自新,違以法
 坐。會浚順安軍營田河道以扼寇,徙莫州路部署。石普屯兵順安之西境,詔能與楊延昭、田敏掎角為備。景德初,破敵長城口,追越陽山,斬首級、獲兵器益眾,詔賜錦袍、金帶。復以所部禦寇於順安。



 六月,召拜防禦使,復出為寧邊軍路部署。詔推能果略,再任以威虜,使副精兵伺敵動止。邊人百餘掠居民,樹蕃僧為帥,能與田敏、楊勳合兵設伏擊之,擒其帥。賊來逼城,能出兵拒之,少衄,即卻陣入城,張凝以兵擊卻之。會詔能與凝領偏師分
 道入幽、易,牽制契丹之勢,能畏愞不前,且不戢所部,多俘奪人馬。俄徙屯定州,及遣凝躡跡北行,能粗險,自度無功,心愧,多怨辭,以訕聞。朝議謂能剛猾少檢,不可專任,乃命綦政敏為鈐轄,俾同職焉。



 明年,師還大名。時王能、曹璨各領兵歸闕,即城下,鈐轄孫全照遣能、璨之師由北門分道先入,能師繼之。能怒全照之後己,即疾驅競入,全照射之,能嚄唶不堪,奪全照弓以去。翌日,詣判府王欽若誣全照射傷押隊閣門楊凝,詞頗紛競。全照
 密疏能摧兵退縮,師緩失期,及師旋不整狀。上初聞能逗遛,微怒。會全照奏,乃質實於張凝、白守素等,即責授右羽林將軍,出為鞏縣都監。明年,以自陳,特改官右驍衛大將軍、虢州都監,累遷加領康州團練使。大中祥符八年,卒。錄其子正為閣門祗候,靖為三班奉職。



 陳興,澶州衛南人。開寶中應募為卒,得隸禦龍右直。太宗征河東,幸幽陵,興常從,特被賞賜,累遷天武指揮使。端拱中,改御前忠佐步軍副都軍頭。王超為並、代部署,
 奏興隨軍,遣戍汾州。明年,李繼隆行營河西,興隸麾下,部清朔、龍衛諸軍,克綏、夏、銀州,繼隆命權知夏州。尋還屯所,受詔提轄河東緣邊城池、器甲、芻糧。至道初,繼隆薦其材幹,召補禦龍弩直都虞候。咸平初,為馬軍都軍頭、領蒙州刺史。三年,真授憲州刺史、知霸州,徙滄州副都部署,移石、隰駐泊。會城綏州,詔與錢若水往視利害,事具《'''[[宋史/卷266
 
 若水傳]]'''》。



 又徙涇原儀渭鎮戎軍部署。上言鎮戎軍去渭州瓦亭砦七十餘里,中有二堡,請留兵三百人戍
 之。俄與曹瑋、秦翰領兵抵鎮戎軍西北武延鹹泊川,掩擊蕃寇章埋族帳,斬二百餘級,生擒三百餘人,奪鎧甲、牛羊、駝馬三萬計。詔書嘉獎,賜金帶、錦袍、器幣。繼遷所部康奴族,往歲鈔劫靈州援糧,恃險與眾,尤桀黠難制。復與秦翰等合眾進討,窮其巢穴,俘老幼、獲器畜甚眾,盡焚掘其窖藏。復詔褒之,仍加賜賚。其年,六穀大首領潘羅支言,欲率諸蕃擊賊,請會兵靈州。上以道遠難刻師期,詔興侯羅支報至,即勒所部過天都山以援,勿須
 奏命。會繼遷死,事寢。景德三年,遷本州團練使、知徐州。



 興起行伍,有武略,所至頗著聲績。真宗言軍校之材,必以興為能。大中祥符初,召為龍神衛四廂都指揮使、領登州防禦使,出為邠寧環慶路副都部署兼知邠州。坐擅釋劫盜,罷軍職,改敘州防禦使、知懷州。六年,卒。



 許均,開封人。父邈,太常博士。均,建隆中應募為龍捷卒,征遼州,以功補武騎十將,賜錦袍、銀帶。開寶中,遷武騎副兵馬使。從曹彬征金陵,率眾陷水砦,流矢貫手。改本
 軍使。從征河東,攻隆州城,先登,陷之,中八創。遷副指揮使,前後屢被賞賚。出屯杭州,妖僧紹倫結黨為亂,均從巡檢使周瑩悉擒殺之。



 端拱初,補指揮使。從李繼隆、秦翰赴夏州。擒趙保忠,令均率兵衛守。改龍衛第四指揮使,俄屯夏州,賊來犯境,一日十二戰,走之。又從石普擊賊於原州牛欄砦,深入,獲牛羊、漢生口甚眾。普表上其功,遷第三軍指揮使。



 咸平初,以御前忠佐馬軍都軍頭戍秦州。王均之亂,遣乘傳之蜀,隸雷有終麾下,守魚橋
 門,又從秦翰追殺賊黨於廣都,降其眾七千餘。驛召授東西班都虞候、領順州刺史。五年,稍遷散員都虞候。嘗召見,訪以北面邊事,翌日,真拜磁州刺史、深州兵馬鈐轄。六年,改涇州駐泊部署。數月,知鎮戎軍。嘗出巡警,至隴山木峽口,真宗以其無故離城,慮有狂寇奔突,詔書戒敕。俄以其不明吏治,用曹瑋代之,徙為邠州駐泊部署,改永興軍部署。車駕將巡澶淵,詔均與知府向敏中及鳳翔梁鼎同提總諸州巡檢捕盜事,至河陽,召赴行
 在。



 時有王長壽者,本亡命卒,有勇力,多計慮,聚徒百餘。是春,抵陳留剽劫,縣民捕之不獲,朝廷遣使益兵,逐之澶、濮間。會契丹南侵,夾河民庶驚擾,長壽結黨愈眾,人皆患之。均至胙城,長壽與其徒五千餘人入縣鈔掠,均部下徒兵裼袒與鬥。均以方略誘之,生擒長壽,斬獲惡黨皆盡。上以方禦敵,未欲因捕賊獎均。但賞均部下卒,被傷者賜帛遷級焉。明年,追敘前勞,擢為本州團練使,尋出知代州。四年秋,均被疾,以米銳代還,未至而均卒。
 錄其子懷忠為奉禮郎,懷信為侍禁。幼子懷德,自有[[宋史/卷324許懷德
 
 傳]]。



 張進,兗州曲阜人,拳勇善射,挽強及石餘。應募曹州,隸鎮兵。太祖親選勇士,奇進才力,以補控鶴官,積勞至禦龍弩直都虞候、領恩州刺史。至道中,兼御前忠佐步軍都軍頭。太宗嘗幸內廄,進以親校執鉞前導,體質魁岸,迥出儕輩。太宗熟視異之,擢為天武右廂都指揮使、領賀州團練使。



 咸平初,遷昭州防禦使,充龍神衛四廂都指揮使、京城左右廂巡檢。未幾,遷捧日、天武四廂都指
 揮使。二年秋,閱武近郊,進與殿前都指揮使王超親執金鼓,節其進退,軍容甚肅。從上北征,又與超管勾大陣及先鋒策應。三年,權殿前都虞候,遷侍衛步軍都虞候、鎮州副部署,徙天雄軍部署。會河決鄆州王陵口,發數州丁男塞之,命進董其役,凡月餘畢,詔褒之。移並、代副都部署。



 李繼遷寇麟州,州將遣單介間道乞師太原。諸將以無詔旨,猶豫未決,進獨抗議,發兵赴援,既至而圍解,手詔褒美。契丹侵中山,命進率廣銳二萬騎,由土門
 會兵鎮、定,未至而敵退,復歸晉陽。景德元年,卒。上遣中使護喪還京,官給葬事。子元晉,至內殿崇班、閣門祗候。天禧末,錄其次子元素為三班借職。



 李重貴,孟州河陽人。姿狀雄偉,善騎射。少事壽帥王審琦,頗見親信,以甥妻之,補合流鎮將。鎮有群盜,以其尚少,謀夜入劫鈔。重貴知之,即築柵課民習射,盜聞之潰去。太宗在藩邸,知其勇幹,召隸帳下。即位,補殿前指揮使,累遷至龍衛左第四軍都指揮使、領河州刺史,改捧
 日右廂都指揮使、領蠻州團練使。



 至道二年,出為衛州團練使。未行,會命將五路討李繼遷,以重貴為麟府州濁輪砦路都部署。得對便殿,因言:「賊居沙磧中,逐水草牧畜,無定居,便戰鬥,利則進,不利則走。今五路齊入,彼聞兵勢太盛,不來接戰,且謀遠遁。欲追則人馬乏食,將守則地無堅壘。賊既未平,臣輩何顏以見陛下?」太宗善之,出御劍以賜,又累遣使撫勞。既而諸將果無大功。及還,命為代、並副都部署。真宗即位,加本州防禦使,徙高
 陽關行營副都部署。



 咸平二年,契丹南侵,議屯兵楊疃,張凝領先鋒遇敵,重貴率策應兵酣戰,全軍而還。范廷召自定州至,遇契丹兵交戰,康保裔大陣為敵所覆,重貴與凝赴援,腹背受敵,自申至寅,疾力戰,敵乃退。時諸將頗失部分,獨重貴與凝全軍還屯。凝議上將士功狀,重貴喟然曰:「大將陷沒而吾曹計功,何面目也!」上聞而嘉之。



 明年春,以勞進階及食邑,徙知貝州,召至勞問,復遣入郡。是冬,徙滄州駐泊副都部署兼知州事。以疾求
 還京就醫藥,既愈,連為邢州、天雄軍二部署,又知冀州。景德初,車駕幸澶淵,召還,為大內都部署。明年春,出知鄭州,以疾甚,授左武衛大將軍、領潘州防禦使,改左羽林軍大將軍致仕。大中祥符三年,卒。



 呼延贊,并州太原人。父琮,周淄州馬步都指揮使。贊少為驍騎卒,太祖以其材勇,補東班長,入承旨,遷驍雄軍使。從王全斌討西川,身當前鋒,中數創,以功補副指揮使。太平興國初,太宗親選軍校,以贊為鐵騎軍指揮使。
 從征太原,先登乘城,及堞而墜者數四,面賜金帛獎之。七年,從崔翰戍定州,翰言其勇,擢為馬軍副都軍頭,稍遷內員寮直都虞候。



 雍熙四年,加馬步軍副都軍頭。嘗獻陣圖、兵要及樹營砦之策,求領邊任。召見,令之作武藝。贊具裝執馳騎,揮鐵鞭、棗槊,旋繞廷中數四,又引其四子必興、必改、必求、必顯以入,迭舞劍盤槊。賜白金數百兩及四子衣帶。



 端拱二年,領富州刺史。俄與輔超並加都軍頭。淳化三年,出為保州刺史、冀州副都部署。
 至屯所,以無統御材,改遼州刺史。又以不能治民,復為都軍頭、領扶州刺史,加康州團練使。



 咸平二年,從幸大名,為行宮內外都巡檢。真宗嘗補軍校,皆敘己功,或至喧嘩,贊獨進曰:「臣月奉百千,所用不及半,忝幸多矣。自念無以報國,不敢更求遷擢,將恐福過災生。」再拜而退,眾嘉其知分。三年,元德皇太后園陵,命掌護儀衛,及還而卒。



 贊有膽勇,鷙悍輕率,常言願死於敵。遍文其體為「赤心殺賊」字,至於妻孥僕使皆然,諸子耳後別刺字曰:「
 出門忘家為國,臨陣忘死為主。」及作破陣刀、降魔杵,鐵折上巾,兩旁有刃,皆重十數斤。絳帕首,乘騅馬,服飾詭異。性復鄙誕不近理,盛冬以水沃孩幼,冀其長能寒而勁健。其子嘗病,贊刲股為羹療之。贊卒後,擢必顯為軍副都軍頭。



 劉用,相州人。祖萬進,河中府馬步軍都指揮使。父守忠,左驍衛大將軍致仕。用曉音律,善騎射,事太宗於晉邸。即位,補軍職,累遷散都頭都虞候。端拱初,為馬步軍副
 都軍頭、領涼州刺史、鎮定招安使,轉捧日都指揮使。李順亂蜀,為西路行營鈐轄。賊平,遷祁州刺史。至道初,為河西、烏白池都鈐轄,斬首千餘級,奪馬五百匹,改高陽關副都部署。



 真宗即位,加本州團練使、并州副都部署。咸平中,徙貝州,俄知瀛州,復為高陽關副都部署。時烽堠數警,用建議益邊兵,俟其南牧,即率驍銳出東路以牽制其勢,因圖上地形。上召宰相閱視,可其奏,且令轉運使於保州、威虜、靜戎、順安軍預備資糧。



 六年,命將三
 路出師扞敵,詔用與劉漢凝、田思明領兵五千,由東路會石普、孫全照掎角攻之。未幾,換鎮州副部署。景德初,為邢州部署。車駕北征,用以城守之勞,進爵邑,歷知齊、陳、潞三州,大中祥符二年卒。



 耿全斌,冀州信都人。父顥,懷順軍校。全斌少豐偉,顥攜謁陳摶,摶謂有藩侯相。顥戍西蜀,全斌往省,乘舟溯江,夜大風失纜,漂七十里,至曙風未止,舟忽泊岸,人頗異之。後遊京師,屬太宗在藩邸,全斌候拜於中衢,自薦材
 幹,得召試武藝,以善左射,隸帳下。即位,補東班承旨,稍遷驍猛副兵馬使。



 從征太原,還,遇契丹於蒲陰,追擊至徐河,因據水口要害。遷補日騎副兵馬使、雲騎軍使,屯瀛州。與契丹戰,所乘馬兩中流矢死,凡三易乘,戰不卻,契丹為引去。端拱初,擊蕃部於宥州,敗之。曆雲騎指揮使、御前忠佐馬軍副都軍頭,改馬軍都軍頭,戍深州,累轉散直都虞候、領順州刺史,改殿前左班都虞候、馬步軍都軍頭。



 全斌在軍中有能名。真宗嘗召問邊事,全斌
 口陳利害,甚稱旨。因謂輔臣曰:「元澄、鄭誠、耿全斌,人多稱之。觀其詞氣,若有志操,止在宿衛,無以見其才,宜以邊郡試之。」遂拜雄州刺史、知深州,徙石、隰部署以備河西。繼遷死,全斌率兵入伏落關,誘蕃部來歸者數千人。俄知安肅軍,嘗繪山川險易為圖以獻。



 契丹來侵,自山北抵河滸,全斌遣子從政焚橋砦,分率精兵擊走之。改冀州刺史、高陽關鈐轄,擢從政為侍禁、寄班祗候。大中祥符初,封禪泰山,以為濮州鈐轄。其年還京師,卒。



 周仁美,深州人。開寶中,應募隸貝州驍捷軍。關南李漢超選備給使,屢捕獲契丹諜者。從漢超戰於西嘉山,身中重創,補隊長。漢超上其功,隸殿前班,賜衣帶、鞍勒馬、什物、奴婢、器械。命王繼恩引入縱觀,過祗候庫,太祖問其力能負錢幾許,仁美曰:「臣可勝七八萬。」太祖曰:「可惜壓死。」止命負四萬五千,因賜之。稍遷右班都知、御前忠佐馬軍副都軍頭,戍環州。



 時牛耶泥族累歲為寇,仁美與陳德玄、宋思恭往擊之,斬首三千級,獲牛羊三百餘,
 發戎族囷窖以餉師。又與思恭討募窟泉岌拖族,格鬥,斬八十餘級。至道初,石昌牛耶泥族復叛,德玄令仁美提兵撫輯之。仁美謂石昌鎮主和文顯曰:「此賊不除,邊患未弭。」因厚設殽酒,召酋長二十八人縛送州獄,自是諸族懾畏。



 二年,又與馬紹忠、白守榮、田紹斌部芻糧趣清遠軍,仁美為先鋒。至岐子平,與虜角,走之。明日,又戰於浦洛河,自巳至戌,戰數十合,進壁乾河。紹忠、守榮皆敗走,紹斌退止浦洛,獨仁美所部不滿三千,身中八創,
 護芻糧、官吏直抵清遠。紹斌繼至,深歎其勇幹,表上其功。



 時運糧民道路被傷者相繼,仁美領徒援護,悉抵環州。又遇虜於橐駝路,擊走之。先是,諸蕃每貢馬京師,為繼遷邀擊,仁美領騎士為援,賊不敢犯。補澶州龍衛軍都虞候,部署李繼隆奏留麾下,選軍中伉健者千人,令仁美領之,屢入敵境,戰有功。



 俄還澶州。召見,會令諸軍射,仁美自陳筋力未衰,願對殿廷發二矢,上許之。既而前奏曰:「臣老於戎門,多戍外郡,罕曾入覲京闕。前後征
 行,體被三十餘創,今日得對萬乘,千載之幸。儻或備員宿衛,立殿庭下一日足矣。」上顧傅潛而笑,潛亦稱其武幹,力留,補馬步軍副都軍頭。



 潛屯北面,常以自隨。契丹攻蒲陰,仁美領萬騎解其圍。又從王超屯鎮、定、儀、渭,累遷龍衛軍都指揮使、領順州刺史,復屯鎮、定。時州有亡命卒聚盜,剽村閭為患,王超委仁美招捕。仁美選勇敢卒,詐亡命趣賊所,得其要領,即自往諭以禍福,留賊中一日。超忽失仁美,求之甚急。詰旦,仁美至,具道其事,乃
 出庫錢付仁美為賞。不數日,賊悉降,凡得二百餘人,以隸軍籍。



 景德中,徙屯陳州,入掌軍頭引見司。大中祥符元年,從駕泰山,命檢視山下諸壇牲牢祭饌。明年,出為磁州團練使、知衛州,俄改滄州部署,移高陽關副部署。八年,擢為龍神衛四廂都指揮、領獎州防禦使,遷捧日、天武四廂都指揮使,改領端州防禦使,權京新城內都巡檢。先是,巡兵捕亡卒盜賊,不獲皆有罰,而獲者無賞。仁美因差立賞格以聞,詔從其請。天禧三年,卒。



 論曰:繼忠臨陣赴敵,以死自效,其生也亦幸而免,然在朔庭貴寵用事,議者方之李陵,而大節固已虧矣。潛為三路帥,握兵八萬餘,大敵在前,逗撓畏縮,致康保裔以無援戰沒,此而不誅,宋於是乎失刑矣。興、均輩或由藩邸進,或自行伍起,一時際會,出則書勳轅門,入則拱扈岩陛,求其如古名將,則未之見也。



\end{pinyinscope}