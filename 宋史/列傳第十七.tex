\article{列傳第十七}

\begin{pinyinscope}

 曹彬
 子璨瑋琮潘美李超附



 曹彬,字國華,真定靈壽人。父蕓,成德軍節度都知兵馬使。彬始生周歲,父母以百玩之具羅於席,觀其所取。彬左手持干戈,右手持俎豆,斯須取一印,他無所視,人皆
 異之。及長,氣質淳厚。漢乾祐中,為成德軍牙將。節帥武行德見其端懿,指謂左右曰:「此遠大器,非常流也。」周太祖貴妃張氏,彬從母也。周祖受禪,召彬歸京師。隸世宗帳下,從鎮澶淵,補供奉官,擢河中都監。蒲帥王仁鎬以彬帝戚,尤加禮遇。彬執禮益恭,公府燕集,端簡終日,未嘗旁視。仁鎬謂從事曰:「老夫自謂夙夜匪懈,及見監軍矜嚴,始覺己之散率也。」



 顯德三年,改潼關監軍,遷西上閣門使。五年,使吳越,致命訖即還。私覿之禮,一無所受。
 吳越人以輕舟追遺之,至於數四,彬猶不受。既而曰:「吾終拒之,是近名也。」遂受而籍之以歸,悉上送官。世宗強還之,彬始拜賜,悉以分遺親舊而不留一錢。出為晉州兵馬都監。一日,與主帥暨賓從環坐於野,會鄰道守將走價馳書來詣,使者素不識彬,潛問人曰:「孰為曹監軍?」有指彬以示之,使人以為紿己,笑曰:「豈有國戚近臣,而衣弋綈袍、坐素胡床者乎?」審視之方信。遷引進使。



 初,太祖典禁旅,彬中立不倚,非公事未嘗造門,群居燕會,亦
 所罕預,由是器重焉。建隆二年,自平陽召歸,謂曰:「我疇昔常欲親汝,汝何故疏我?」彬頓首謝曰:「臣為周室近親,復忝內職,靖恭守位,猶恐獲過,安敢妄有交結?」遷客省使,與王全斌、郭進領騎兵攻河東平樂縣,降其將王超、侯霸榮等千八百人,俘獲千餘人。既而賊將蔚進率兵來援,三戰皆敗之。遂建樂平為平晉軍。乾德初,改左神武將軍。時初克遼州,河東召契丹兵六萬騎來攻平晉,彬與李繼勛等大敗之於城下。俄兼樞密承旨。



 二年冬,
 伐蜀,詔以劉光毅為歸州行營前軍副部署,彬為都監。峽中郡縣悉下,諸將咸欲屠城以逞其欲,彬獨申令戢下,所至悅服。上聞,降詔褒之。兩川平,全斌等晝夜宴飲,不恤軍士,部下漁奪無已,蜀人苦之。彬屢請旋師,全斌等不從。俄而全師雄等構亂,擁眾十萬,彬復與光毅破之於新繁,卒平蜀亂。時諸將多取子女玉帛,彬橐中唯圖書、衣衾而已。及還,上盡得其狀,以全斌等屬吏。謂彬清介廉謹,授宣徽南院使、義成軍節度使。彬入見,辭曰:「
 征西將士俱得罪,臣獨受賞,恐無以示勸。」上曰:「卿有茂功,又不矜伐,設有微累,仁贍等豈惜言哉?懲勸國之常典,可無讓。」



 六年,遣李繼勛、黨進率師征太原,命為前軍都監,戰洞渦河,斬二千餘級,俘獲甚眾。開寶二年,議親征太原,復命為前軍都監,率兵先往,次團柏谷,降賊將陳廷山。又戰城南,薄於濠橋,奪馬千餘。及太祖至,則已分砦四面,而自主其北。六年,進檢校太傅。



 七年,將伐江南。九月,彬奉詔與李漢瓊、田欽祚先赴荊南發戰艦,潘
 美帥步兵繼進。十月,詔以彬為升州西南路行營馬步軍戰棹都部署,分兵由荊南順流而東,破峽口砦,進克池州,連克當塗、蕪湖二縣,駐軍採石磯。十一月,作浮梁,跨大江以濟師。十二月,大破其軍於白鷺洲。



 八年正月,又破其軍於新林港。二月,師進次秦淮,江南水陸十餘萬陳於城下,大敗之,俘斬數萬計。及浮梁成,吳人出兵來御,破之於白鷺洲。自三月至八月,連破之,進克潤州。金陵受圍,至是凡三時,吳人樵採路絕,頻經敗衄,李煜
 危急,遣其臣徐鉉奉表詣闕,乞緩師,上不之省。先是,大軍列三砦,美居守北偏,圖其形勢來上。太祖指北砦謂使者曰:「吳人必夜出兵來寇,爾亟去,令曹彬速成深溝以自固,無墮其計中。」既成,吳兵果夜來襲,美率所部依新溝拒之,吳人大敗。奏至,上笑曰:「果如此。」



 長圍中,彬每緩師,冀煜歸服。十一月,彬又使人諭之曰:「事勢如此,所惜者一城生聚,若能歸命,策之上也。」城垂克,彬忽稱疾不視事,諸將皆來問疾。彬曰:「余之疾非藥石所能愈,惟
 須諸公誠心自誓,以克城之日,不妄殺一人,則自愈矣。」諸將許諾,共焚香為誓。明日,稍愈。又明日,城陷。煜與其臣百餘人詣軍門請罪,彬慰安之,待以賓禮,請煜入宮治裝,彬以數騎待宮門外。左右密謂彬曰:「煜入或不測,奈何?」彬笑曰:「煜素心耎無斷,既已降,必不能自引決。」煜之君臣,卒賴保全。自出師至凱旋,士眾畏服,無輕肆者。及入見,刺稱「奉敕江南干事回」,其謙恭不伐如此。



 初,彬之總師也,太祖謂曰:「俟克李煜,當以卿為使相。」副帥潘美
 預以為賀。彬曰:「不然,夫是行也,仗天威,遵廟謨,乃能成事,吾何功哉,況使相極品乎!」美曰:「何謂也?彬曰:「太原未平爾。」及還,獻俘。上謂曰:「本授卿使相,然劉繼元未下,姑少待之。」既聞此語,美竊視彬微笑。上覺,遽詰所以,美不敢隱,遂以實對。上亦大笑,乃賜彬錢二十萬。彬退曰:「人生何必使相,好官亦不過多得錢爾。」未幾,拜樞密使、檢校太尉、忠武軍節度使。



 太宗即位,加同平章事。議徵太原,召彬問曰:「周世宗及太祖皆親征,何以不能克?」彬曰:「
 世宗時,史彥超敗於石嶺關,人情驚擾,故班師;太祖頓兵甘草地,會歲暑雨,軍士多疾,因是中止。」太宗曰:「今吾欲北征,卿以為何如?」彬曰:「以國家兵甲精銳,剪太原之孤壘,如摧枯拉朽爾,何為而不可。」太宗意遂決。太平興國三年,進檢校太師,從征太原,加兼侍中。八年,為弭德超所誣,罷為天平軍節度使。旬餘,上悟其譖,進封魯國公,待之愈厚。



 雍熙三年,詔彬將幽州行營前軍馬步水陸之師,與潘美等北伐,分路進討。三月,敗契丹於固安,
 破涿州,戎人來援,大破之於城南。四月,又與米信破契丹於新城,斬首二百級。五月,戰於岐溝關,諸軍敗績,退屯易州,臨易水而營。上聞,亟令分屯邊城,追諸將歸闕。



 先是,賀令圖等言於上曰:「契丹主少,母后專政,寵幸用事,請乘其釁,以取幽薊。」遂遣彬與崔彥進、米信自雄州,田重進趣飛狐,潘美出雁門,約期齊舉。將發,上謂之曰:「潘美之師但先趣雲、應,卿等以十萬眾聲言取幽州,且持重緩行,不得貪利。彼聞大兵至,必悉眾救範陽,不暇
 援山後矣。」既而,美之師先下寰、朔、雲、應等州,重進又取飛狐、靈丘、蔚州,多得山後要害地,彬亦連下州縣,勢大振。每奏至,上已訝彬進軍之速。及彬次涿州,旬日食盡,因退師雄州以援餉饋。上聞之曰:「豈有敵人在前,反退軍以援芻粟,失策之甚也。」亟遣使止彬勿前,急引師緣白溝河與米信軍會,案兵養銳,以張西師之勢;俟美等盡略山後地,會重進之師而東,合勢以取幽州。時彬部下諸將,聞美及重進累建功,而已握重兵不能有所攻
 取,謀議蜂起。彬不得已,乃復裹糧再往攻涿州。契丹大眾當前,時方炎暑,軍士乏困,糧且盡,彬退軍,無復行伍,遂為所躡而敗。



 彬等至,詔鞫於尚書省,令翰林學士賈黃中等雜治之,彬等具伏違詔失律之罪。彬責授右驍衛上將軍,彥進右武衛上將軍,信右屯衛上將軍,餘以次黜。四年,起彬為侍中、武寧軍節度使。淳化五年,徙平盧軍節度。真宗即位,復檢校太師、同平章事。數月,召拜樞密使。



 咸平二年,被疾。上趣駕臨問,手為和藥,仍賜白
 金萬兩。問以後事,對曰:「臣無事可言。臣二子材器可取,臣若內舉,皆堪為將。」上問其優劣,對曰:「璨不如瑋。」六月薨,年六十九。上臨哭之慟,對輔臣語及彬,必流涕。贈中書令,追封濟陽郡王,謚武惠;且贈其妻高氏韓國夫人;官其親族、門客、親校十餘人。八月,詔彬與趙普配饗太祖廟庭。



 彬性仁敬和厚,在朝廷未嘗忤旨,亦未嘗言人過失。伐二國,秋毫無所取。位兼將相,不以等威自異。遇士夫於途,必引車避之。不名下吏,每白事,必冠而後見。
 居官奉入給宗族,無餘積。平蜀回,太祖從容問官吏善否,對曰:「軍政之外,非臣所聞也。」固問之,唯薦隨軍轉運使沈倫廉謹可任。為帥知徐州日,有吏犯罪,既具案,逾年而後杖之,人莫知其故。彬曰:「吾聞此人新娶婦,若杖之,其舅姑必以婦為不利,而朝夕笞詈之,使不能自存。吾故緩其事,然法亦未嘗屈焉。」北征之失律也,趙昌言表請行軍法。及昌言自延安還,被劾,不得入見。彬在右府,為請於上,乃許朝謁。



 子璨、珝、瑋、玹、□、珣、琮。珝娶秦王
 女興平郡主,至昭宣使。玹左藏庫副使,□尚書虞部員外郎,珣東上閣門使,琮西上閣門副使。□之女,即慈聖光獻皇后也。蕓,累贈魏王。彬,韓王。□,吳王,謚曰安僖。□之子佾、傅。佾見《外戚傳》。傅,後兄也,榮州刺史,謚恭懷。



 璨字韜光,性沉毅,善射,以蔭補供奉官。常從彬征討,得與計議,彬以為類己,特鐘愛焉。



 遷宮苑副使,出為高陽關及鎮、魏、並、代、趙五州都監。雍熙中,命知定州,改尚食使。淳化二年,領富州刺史,徙知代州。明年,擢為鎮州行
 營鈐轄,徙綏、銀、夏、麟、府等州鈐轄。契丹入寇,屢戰有功。諸將多欲窮追,璨慮有伏,力止之。至道初,遷四方館使、知靈州,徙河西鈐轄,改引進使。範廷召將兵出塞,命璨為之副。丁外艱,起復,為鄜延路副都部署,拜趙州刺史,領武州團練使,充麟、府、濁輪副部署。出蕃兵邀繼遷,俘馘甚眾。入為樞密都承旨,改領亳州團練使。



 契丹入寇,命為鎮、定、高陽關三路行營都鈐轄,領康州防禦使,再知定州。明年冬,拜侍衛馬軍副都指揮使、天德軍節度。
 入為東京舊城都巡檢使,連拜彰國、保靜、武寧、忠武等軍節度使。在禁衛十餘年,未嘗忤旨。天禧三年春,以足疾授河陽節度使、同平章事。卒,年七十,贈中書令,謚武懿。



 璨起貴冑,以孝謹稱,能自奮厲,以世其家。習知韜略,好讀《左氏春秋》,善撫士卒,兼著威愛。雖輕財不逮其父,而敬人和厚,亦有父風。子儀,官至耀州觀察使。



 瑋字寶臣。父彬,歷武寧、天平軍節度使,皆以瑋為牙內都虞候,補西頭供奉官、閣門祗候。沉勇有謀,喜讀書,通《
 春秋三傳》,於《左氏》尤深。李繼遷叛,諸將數出無功,太宗問彬:「誰可將者?」彬曰:「臣少子瑋可任。」即召見,以本官同知渭州,時年十九。



 真宗即位,改內殿崇班、知渭州。馭軍嚴明有部分,賞罰立決,犯令者無所貸。善用間,周知虜動靜,舉措如老將。彬卒,請持喪,不允,改閣門通事舍人。遷西上閣門副使,徙知鎮戎軍。李繼遷虐用其國人,瑋知其下多怨,即移書諸部,諭以朝廷恩信,撫養無所間,以動諸羌。由是康奴等族請內附。繼遷略西蕃還,瑋邀
 擊於石門川,俘獲甚眾。以鎮戎軍據平地,便於騎戰,非中國之利,請自隴山以東,循古長城塹以為限。又以弓箭手皆土人,習障塞蹊隧,曉羌語,耐塞苦,官未嘗與兵械資糧,而每戰輒使先拒賊,恐無以責死力,遂給以境內間田。春秋耕斂,州為出兵護作,而蠲其租。



 繼遷死,其子德明請命於朝。瑋言:「繼遷擅河南地二十年,兵不解甲,使中國有西顧之憂。今國危子弱,不即捕滅,後更強盛,不可制。願假臣精兵,出其不意,禽德明送闕下,復河
 西為郡縣,此其時也。」帝方以恩致德明,不報。既而西延家、妙俄、熟魏數大族請拔帳自歸,諸將猶豫不敢應。瑋曰:「德明野心,不急折其翮,後必揚去。」即日,將其士薄大都山,受降者內徙,德明不敢拒。遷西上閣門使,為環慶路兵馬都鈐轄,兼知邠州。封泰山,進東上閣門使。



 帝以瑋習知河北事,乃以為真定路都鈐轄,領高州刺史。瑋嘗上涇原、環慶兩道圖。至是,帝以示左右,曰:「華夷山川城郭險固出入戰守之要,舉在是矣。」因敕別繪二圖,以
 一留樞密院,一付本道,俾諸將得按圖計事。復為涇原路都鈐轄兼知渭州,與秦翰破章埋族於武延川,分兵滅撥臧於平涼,於是隴山諸族皆來獻地。瑋築堡山外,為籠竿城,募士兵守之。曰:「異時秦、渭有警,此必爭之地也。」祀汾陰,進四方館使。逾年,上表還州事,願專督軍旅。帝不欲遽更守臣,以密詔敦諭之。改引進使、英州團練使,復知秦州,兼涇、原儀、渭、鎮戎緣邊安撫使。



 時唃廝嵒強盛,立遵佐之。立遵乃上書求號「贊普。」瑋言:「贊普,可汗
 號也。立遵一言得之,何以處唃廝嵒邪?且復有求,漸不可制。」乃以立遵為保順軍節度使,恩如廝鐸督。西羌將舉事,必先定約束,號為「立文法」。唃廝嵒使其舅賞樣丹與廝敦立文法於離王族,謀內寇。瑋陰結廝敦,解寶帶予之。廝敦感激,求自效,間謂瑋曰:「吾父何所使?欲吾首,猶可斷以獻。」瑋曰:「我知賞樣丹時至汝帳下,汝能為我取賞樣丹首乎?」廝敦愕然應之。後十餘日,果斷其首來。廝敦因獻南市地。南市者,秦、渭之厄也,瑋城之,表廝敦
 為順州刺史。



 初,張佶知秦州,置四門砦,侵奪羌地,羌人多叛去,畏得罪不敢出。瑋招出之,令入馬贖罪,還故地,至者數千人,每送馬六十匹,給彩一端。築弓門、冶坊、床穰、靜戎、三陽、定西、伏羌、永寧、小洛門、威遠十砦,浚壕三百八十里,皆役屬羌廂兵,工費不出民。伏羌首領廝雞波、李磨論私立文法,瑋潛兵滅其帳。其年,唃廝嵒率眾數萬大入寇,瑋迎戰三都谷,追奔三十里,斬首千餘級,獲馬牛、雜畜、器仗三萬餘。遷客省使、康州防禦使。馬波
 叱臘立柵野吳谷,瑋選募神武軍二百人,斬柵,獲生口、孳畜甚眾。



 宗哥大首領甘遵治兵於任奴川,瑋遣間殺遵,及破魚角蟬所立文法於吹麻城。既而河州、洮蘭、安江、妙敦、邈川、黨逋諸城皆納質為熟戶。時瑋作塹抵拶嵒嚨。拶嵒嚨,西蕃要害地也。先是,瑋遣小吏楊知進護賜物通甘州可汗王,還過宗哥界,立遵邀知進,語曰:「秦州大人直以兵入拶嵒嚨來,幸為我言,願罷兵,歲入貢,約蕃漢為一家。」因使種人黨失畢陵從知進來獻馬。自
 是唃廝嵒勢蹙,退保磧中不出。秦人請刻石紀功,有詔褒之。



 天禧三年,德明寇柔遠砦,都巡檢楊承吉與戰不利。以瑋為華州觀察使、鄜延路副都總管、環、慶、秦等州緣邊巡檢安撫使。委乞、骨咩、大門等族聞瑋至,歸附者甚眾。拜宣徽北院使、鎮國軍節度觀察留後、簽書樞密院事。



 宰相丁謂逐寇準,惡瑋不附己,指為準黨。除南院使、環慶路都總管安撫使。乾興初,謫左衛大將軍、容州觀察使、知萊州。瑋以宿將為謂所忌,即日上道,從弱卒
 十餘人,不以弓韔矢箙自隨。謂敗,復華州觀察使、知青州,徙天雄軍,以彰化軍節度觀察留後知永興軍。拜昭武軍節度使、知天雄軍。以疾守河陽,數月,為真定府、定州都總管,改彰武軍節度使。卒,贈侍中,謚武穆。



 瑋用士,得其死力。平居甚閑暇,及師出,多奇計,出入神速不可測。一日,張樂飲僚吏,中坐失瑋所在,明日,徐出觀事,而賊首已擲庭下矣。嘗稱疾,加砭艾,臥閣內不出。會賊至,瑋奮起裹創,被甲跨馬,賊望見,皆遁去。將兵幾四十年,
 未嘗少失利。唃廝嵒聞瑋名,即望瑋所在,東向合手加顙。契丹使過天雄,部勒其下曰:「曹公在此,毋縱騎馳驅也。」真宗慎兵事,凡邊事,必手詔詰難至十數反,而瑋守初議,卒無以奪。後雖他將論邊事者,往往密付瑋處之。



 渭州有告戍卒叛入夏國者,瑋方對客弈棋,遽曰:「吾使之行也。」夏人聞之,即斬叛者,投其首境上。羌殺邊民,入羊馬贖罪。瑋下令曰:「羌自相犯,從其俗;犯邊民者,論如律。」自是無敢犯。



 環、慶屬羌田多為邊人所市,致單弱不
 能自存,因沒彼中。瑋盡令還其故田,後有犯者,遷其家內地。所募弓箭手,使馳射,較強弱,勝者與田二頃。再更秋獲,課市一馬,馬必勝甲,然後官籍之,則加五十畝。至三百人以上,團為一指揮。要害處為築堡,使自塹其地為方田環之。立馬社,一馬死,眾出錢市馬。降者既多,因制屬羌百帳以上,其首領為本族軍主,次為指揮使,又其次為副指揮使,不及百帳為本族指揮使。其蕃落將校,止於本軍敘進,以其習知羌情與地利,不可徙他軍
 也。開邊壕,率令深廣丈五尺;山險不可塹者,因其峭絕治之,使足以限敵,後皆以為法。天雄卒有犯盜者,眾謂獄具必殺之,瑋乃處以常法。人或以為疑,瑋笑曰:「臨邊對敵,斬不用命者,所以令眾吾,非好殺也。治內郡,安事此乎?」



 初守邊時,山東知名士賈同造瑋,客外舍。瑋欲按邊,即同舍,邀與俱。同問:「從兵安在?」曰:「已具。」既出就騎,見甲士三千環列,初不聞人馬聲。同歸,語人曰:「瑋殆名將也。」瑋為將不如其父寬,然自為一家。嘉祐八年,詔配享
 仁宗廟庭。



 琮字寶章。兄珝,娶秦王女興平郡主。琮幼時,從主入禁中,太宗置膝上,拊其背曰:「曹氏有功我家,此亦佳兒也。」



 及彬領鎮海軍節度使,補衙內都指揮使。彬卒,時遷西頭供奉官、閣門祗候、勾當騏驥院、群牧估馬司,市馬課有羨,再遷西上閣門副使。與曹利用連姻,利用貶,出為河陽兵馬都監,領內軍器庫,遷東上閣門使、榮州刺史。仁宗冊琮兄女為後,禮皆琮主辦,除衛州團練使。琮因
 奏曰:「陛下方以至公屬天下,臣既備後族,不宜冒恩澤,亂朝廷法。族人敢因緣請托,願致於理。」時論稱之。



 出為環慶路馬步軍總管、知邠州,遷秦州防禦使、秦鳳路副都總管兼知秦州。度羨材為倉廩,大積穀古渭、冀城。生羌屢入鈔邊,琮懷以恩信,擊牛釃酒犒之,多請內屬。



 寶元初南郊,召入侍祠。會元昊反,拜同州觀察使,復知秦州,上攻、守、御三策。久之,兼同管勾涇原路兵馬、定國軍節度觀察留後。劉平、石元孫敗,關輔震恐。琮請籍民為
 義軍,以張兵勢,於是料簡鄉弓手數萬人。賊寇山外,還天都,劫儀、秦屬戶。琮發騎士,設伏以待之,賊遂引去。琮欲誘吐蕃犄角圖賊,得西川舊賈,使諭意。而沙州鎮王子遣使奉書曰:「我本唐甥,天子實吾舅也。自黨項破甘、涼,遂與漢隔。今願率首領為朝廷擊賊。」帝善琮策,改陜西副都總管、經略安撫招討副使,拜步軍副都指揮使。與夏竦屯鄜州,還為馬軍副都指揮使,以疾卒。帝臨奠,後並出臨喪,就第成服。贈安化軍節度使兼侍中,謚忠
 恪。



 琮小心謹畏,善贊謁,御軍整嚴,死時家無餘貲。子牷,皇城使、嘉州防禦使。牷子詩,尚魯國大長公主。



 潘美,字仲詢,大名人。父璘,以軍校戍常山。美少倜儻,隸府中典謁。嘗語其里人王密曰:「漢代將終,兇臣肆虐,四海有改卜之兆。大丈夫不以此時立功名、取富貴,碌碌與萬物共盡,可羞也。」會周世宗為開封府尹,美以中涓事世宗。及即位,補供奉官。高平之戰,美以功遷西上閣門副使。出監陜州軍,改引進使。世宗將用師隴、蜀,命護
 永興屯兵,經度西事。



 先是,太祖遇美素厚,及受禪,命美先往見執政,諭旨中外。陜帥袁彥兇悍,信任群小,嗜殺黷貨,且繕甲兵,太祖慮其為變,遣美監其軍以圖之。美單騎往諭,以天命既歸,宜修臣職,彥遂入朝。上喜曰:「潘美不殺袁彥,能令來覲,成我志矣。」



 李重進叛,太祖親征,命石守信為招討使,美為行營都監以副之。揚州平,留為巡檢,以任鎮撫,以功授秦州團練使。時湖南叛將汪端既平,人心未寧,乃授美潭州防禦使。嶺南劉鋹數寇
 桂陽、江華,美擊走之。溪峒蠻獠自唐以來,不時侵略,頗為民患。美窮其巢穴,多所殺獲,餘加慰撫,夷落遂定。乾德二年,又從兵馬都監丁德裕等率兵克郴州。



 開寶三年,徵嶺南,以美為行營諸軍都部署、朗州團練使,尹崇珂副之。進克富川,鋹遣將率眾萬餘來援,遇戰大破之,遂克賀州。十月,又下昭、桂、連三州,西江諸州以次降。美以功移南面都部署,進次韶州。



 韶,廣之北門也,賊眾十餘萬聚焉。美揮兵進乘之,韶州遂拔,斬獲數萬計。鋹窮
 蹙,四年二月,遣其臣王珪詣軍門求通好,又遣其左僕射蕭漼、中書舍人卓惟休奉表乞降。美因諭以上意,以為彼能戰則與之戰,不能戰則勸之守,不能守則諭之降,不能降則死,不能死則亡非此五者他不得受。美既令殿直冉彥袞部送漼等赴闕。



 鋹復遣其弟保興率眾拒戰,美即率厲士卒倍道趨柵頭,距廣州百二十里。鋹兵十五萬依山谷堅壁以待,美因築壘休士,與諸將計曰:「彼編竹木為柵,若攻之以火,彼必潰亂。因以銳師夾
 擊之,萬全策也。」遂分遣丁夫數千人,人持二炬,間道造其柵。及夜,萬炬俱發,會天大風,火勢甚熾。鋹眾驚擾來犯,美揮兵急擊之,鋹眾大敗,斬數萬計。長驅至廣州,鋹盡焚其府庫,遂克之,擒鋹送京師,露布以聞。即日,命美與尹崇珂同知廣州兼市舶使。五月,拜山南東道節度。五年,兼嶺南道轉運使。土豪周思瓊聚眾負海為亂,美討平之,嶺表遂安。



 八年,議征江南。九月,遣美與劉遇等率兵先赴江陵。十月,命美為升州道行營都監,與曹彬
 偕往,進次秦淮。時舟楫未具,美下令曰:「美受詔,提驍果數萬人,期於必勝,豈限此一衣帶水而不徑度乎?」遂麾以涉,大軍隨之,吳師大敗。及採石磯浮梁成,吳人以戰艦二十餘鳴鼓溯流來趨利。美麾兵奮擊,奪其戰艦,擒其將鄭賓等七人,又破其城南水砦,分舟師守之。奏至,太祖遣使令亟徙置戰棹,以防他變。美聞詔即徙軍。是夜,吳人果來攻砦,不能克。進薄金陵,江南水陸十萬陳於城下,美率兵襲擊,大敗之。李煜危甚,遣徐鉉來乞緩
 師,上不之省,仍詔諸將促令歸附。煜遷延未能決,夜遣兵數千,持炬鼓噪來犯我師。美率精銳以短兵接戰,因與大將曹彬率士晨夜攻城,百道俱進。金陵平,以功拜宣徽北院使。



 秋,命副黨進攻太原,戰於汾上,破之,且多擒獲。太平興國初,改南院使。三年,加開府儀同三司。四年,命將征太原,美為北路都招討,判太原行府事。部分諸將進討,並州遂平。繼征範陽,以美知幽州行府事。及班師,命兼三交都部署,留屯以捍北邊。三交西北三百
 里,地名固軍,其地險阻,為北邊咽喉。美潛師襲之,遂據有其地。因積粟屯兵以守之,自是北邊以寧。美嘗巡撫至代州,既秣馬蓐食,俄而遼兵萬騎來寇,近塞,美誓眾銜枚奮擊,大破之。封代國公。八年,改忠武軍節度,進封韓國公。



 雍熙三年,詔美及曹彬、崔彥進等北伐,美獨拔寰、朔、雲、應等州。詔內徙其民。會遼兵奄至,戰於陳家谷口,不利,驍將楊業死之。美坐削秩三等,責授檢校太保。明年,復檢校太師。知真定府,未幾,改都部署、判並州。加
 同平章事,數月卒,年六十七。贈中書令,謚武惠。咸平二年,配饗太宗廟庭。



 子惟德至宮苑使,惟固西上閣門使,惟正西京作坊使,惟清崇儀使,惟熙娶秦王女,平州刺史。惟熙女,即章懷皇后也。美后追封鄭王,以章懷故也。



 惟吉,美從子,累資為天雄軍駐泊都監。雖連戚里,能以禮法自飭,揚歷中外,人咸稱其勤敏云。



 李超者,冀州信都人。為禁卒,常從潘美軍中,主刑刀。美好乘怒殺人,超每潛緩之。美怒解,輒得釋,以是全者甚
 眾,人謂其有陰德。



 子浚字德淵。中進士,累擢秘書、知康州。咸平中,入為刑部詳覆、御史臺推直官。屢上書言事,遷開封府推官,賜緋魚。景德初,拜虞部員外郎兼侍御史知雜事,賜金紫。從幸澶淵,頗上疏言便宜。師還,命與陳堯咨安撫河北。逾年,判吏部銓。浚居憲府,未再歲,帝寵待之,擢樞密直學士。宰相王旦言:「浚雖有剸劇才,然驟歷清切,時望未允。」真宗曰:「朕業已許之矣。」尋知開封,能檢察隱微,京師稱之。累遷至右司郎中,出知秦州,暴
 疾卒。浚與李宗諤同歲同月後一日生,其卒也亦後一日,眾以為異。



 論曰:曹彬以器識受知太祖,遂膺柄用。平居,於百蟲之蟄猶不忍傷,出使吳越,籍上私饋,悉用施予,而不留一錢;則其總戎專征,而秋毫無犯,不妄戮一人者,益可信矣。潘美素厚太祖,信任於得位之初,遂受征討之托。劉鋹遣使乞降,觀美所喻,辭義嚴正,得奉辭伐罪之體;則其威名之重,豈待平嶺表、定江南、徵太原、鎮北門而後
 見哉?二人皆謚武惠,皆與配饗,兩家子孫,皆能樹立,享富貴。而光獻、章懷皆稱賢後,非偶然也。君子謂仁恕清慎,能保功名,守法度,唯彬為宋良將第一,豈無意哉?若李浚者,亦以材干自結主知,遂歷清顯。謂為陰德所致,理或然也



\end{pinyinscope}