\article{志第六十九 禮十九(賓禮一)}

\begin{pinyinscope}

 大朝會儀常朝儀



 《周官》:司儀掌九儀賓客擯相,詔王南鄉以朝諸侯。「大行人掌大賓之禮、大客之儀,以親諸侯」。蓋君臣之際體統
 雖嚴,然而接以仁義,攝以威儀,實有賓主之道焉。是以《小雅·鹿鳴》燕其臣下,皆以嘉賓稱之。宋之朝儀,政和詳定五禮,列為賓禮。今修《宋史》,存其舊云。



 大朝會。宋承前代之制,以元日、五月朔、冬至行大朝會之禮。太祖建隆二年正月朔,始受朝賀於崇元殿,服袞冕,設宮縣、仗衛如儀。仗退,群臣詣皇太后宮門奉賀。帝常服御廣德殿,群臣上壽,用教坊樂。五月朔,受朝賀於崇元殿,帝服通天冠,絳紗袍,宮縣、儀仗如元會儀。乾德
 三年冬至,受朝賀於文明殿,四年於朝元殿,賀畢,常服御大明殿,群臣上壽,始用雅樂登歌、二舞,群臣酒五行罷。



 太宗淳化三年正月朔,命有司約《開元禮》定上壽儀,皆以法服行禮,設宮縣、萬舞,酒三行罷。



 真宗咸平三年五月朔,雨,命放仗,百官常服,起居於長春殿,退詣正衙,立班宣制。



 仁宗天聖四年十二月,詔明年正月朔先率百官赴會慶殿,上皇太后壽,酒畢,乃受朝天安殿,仍令太常禮院修定儀制。



 五年正月朔,曉漏未盡三刻,宰臣、
 百官與遼使、諸軍將校,並常服班會慶殿。內侍請皇太后出殿後幄,鳴鞭,升坐;又詣殿後皇帝幄,引皇帝出。帝服靴袍,於簾內北向褥位再拜,跪稱:「臣某言:元正啟祚,萬物惟新。伏惟尊號皇太后陛下,膺時納祐,與天同休。」內常侍承旨答曰:「履新之祐,與皇帝同之。」帝再拜,詣皇太后御坐稍東。內給事酌酒授內謁者監進,帝跪進訖,以盤興,內謁者監承接之,帝卻就褥位,跪奏曰:「臣某稽首言:元正令節,不勝大慶,謹上千萬歲壽。」再拜,內常侍
 宣答曰:「恭舉皇帝壽酒。」帝再拜,執盤侍立,教坊樂止,皇帝受虛盞還幄。通事舍人引百官橫行,典儀贊再拜、舞蹈、起居。太尉升自西階,稱賀簾外,降,還位,皆再拜、舞蹈。侍中承旨曰:「有制。」皆再拜,宣曰:「履新之吉,與公等同之。」皆再拜、舞蹈。閣門使簾外奏:「宰臣某以下進壽酒。」皆再拜。太尉升自東階,翰林使酌御酒盞授太尉,執盞盤跪進簾外,內謁者監跪接以進,太尉跪奏曰:「元正令節,臣等不勝慶抃,謹上千萬歲壽。」降,還位,皆再拜。宣徽使承
 旨曰:「舉公等觴。」皆再拜。太尉升,立簾外,樂止。內謁者監出簾授虛盞。太尉降階,橫行,皆再拜、舞蹈。宣徽使承旨宣群臣升殿,再拜,升,及東西廂坐,酒三行,侍中奏禮畢,退。樞密使以下迎乘輿於長春殿,起居稱賀。百官就朝堂易朝服,班天安殿朝賀,帝服袞冕受朝。禮官、通事舍人引中書令、門下侍郎各於案取所奏文,詣褥位,脫劍舄,以次升,分東西立。諸方鎮表、祥瑞案先置門外,左右令史絳衣對舉,給事中押祥瑞、中書侍郎押表案入,
 分詣東、西階下對立。既賀,更服通天冠、絳紗袍,稱觴上壽,止舉四爵。乘輿還內,恭謝太后如常禮。



 神宗元豐元年,詔龍圖閣直學士、史館修撰宋敏求等詳定正殿御殿儀注,敏求遂上《朝會儀》二篇、《令式》四十篇,詔頒行之。其制:



 元正、冬至大朝會,有司設御坐大慶殿,東西房於御坐之左右少北,東西閣於殿後,百官、宗室、客使次於朝堂之內外。五輅先陳於庭,兵部設黃麾仗於殿之內外。大樂令展宮架之樂於橫街南。鼓吹令分置十二案於
 宮架外。協律郎二人,一位殿上西階之前楹,一位宮架西北,俱東向。陳輿輦、御馬於龍墀,傘扇於沙墀,貢物於宮架南冬至不設貢物,餘則列大慶門外。陳布將士於街。左、右金吾六軍諸衛勒所部,列黃麾大仗於門及殿庭。百僚、客使等俱入朝。文武常參官朝服,陪位官公服,近仗就陳於閣外。大樂令、樂工、協律郎入就位。中書侍郎以諸方鎮表案、給事中以祥瑞俟於大慶門外之左右冬至不設給事中位、祥瑞案。諸侍衛官各服其器服。



 輦出,至西閣降輦,
 符寶郎奉寶詣閣門奉迎,百官、客使、陪位官俱入就位。侍中版奏中嚴,又奏外辦。殿上鳴鞭,宮縣撞黃鐘之鐘,右五鐘皆應。內侍承旨索扇,扇合,帝服通天冠、絳紗袍。御輿出,協律郎舉麾奏《乾安》樂,鼓吹振作。帝出自西房,降輿即坐,扇開,殿下鳴鞭。協律郎偃麾樂止,爐煙升。符寶郎奉寶置御坐前,中書侍郎、給事中押表案、祥瑞案入,詣東西階下對立,百官、宗室及遼使班分東西,以次入,《正安》樂作,就位。樂止,押樂官歸本班,起居畢,復案位。
 三師、親王以下及御史臺、外正任、遼使俱就北向位。典儀贊拜,在位者皆再拜,起居訖,太尉將升,中書令、門下侍郎俱降至西階下立凡太尉行則樂作,至位樂止。太尉詣西階下,解劍脫舄升殿。中書令、門下侍郎各於案取所奏之文詣褥位,解劍脫舄以次升,分東西立以俟。太尉詣御坐前,北向跪奏:「文武百寮、太尉具官臣某等言:元正啟祚,萬物咸新冬至易為「晷運推移,日南長至」。伏惟皇帝陛下應乾納祐,與天同休。」俯伏,興,降階,佩劍納舄餘官準此。還位,在位官俱再拜、
 舞蹈,三稱萬歲,再拜。侍中進當御坐前承旨,退臨階,西向,稱制宣答曰:「履新之慶冬至易曰「履長之慶」,與公等同之。」贊者曰「拜」,舞蹈,三稱萬歲。橫行官分班立。中書令、門下侍郎升詣御坐前,各奏諸方鎮表及祥瑞訖,戶部尚書就承制位俯伏跪奏諸州貢物,請付所司。禮部尚書奏諸蕃貢物如之。司天監奏雲物祥瑞,請付史館,皆如上儀。侍中進當御坐前奏禮畢,殿上承旨索扇,殿下鳴鞭,官縣撞蕤賓之鐘,左五鐘皆應,協律郎舉麾,宮縣奏《乾安》樂,
 鼓吹振作,帝降坐,御輿入自東房,扇開,偃麾樂止。侍中奏解嚴,百官退還次。客使、陪位官並退。



 有司設食案,大樂令設登歌殿上,二舞入,立於架南。預坐當升殿者位御坐之前,文武相向,異位重行,以北為上,非升殿者位於東西廊下。尚食奉御設壽尊於殿東楹少南,設坫於尊南,加爵一。有司設上下群臣酒尊於殿下東西廂。侍衛官及執事者各立其位,仗衛仍立俟,上壽百官立班如朝賀儀。



 侍中版奏中嚴、外辦,聞鳴鞭,索扇,帝服通天
 冠、絳紗袍,御輿出東房,樂作。帝即坐,扇開,樂止。贊拜畢,光祿卿詣橫街南,跪奏:「具官臣某言,請允群臣上壽。」興,侍中承旨稱「制可」,少退。舍人曰「拜」,光祿卿再拜訖,復位。三師以下就位,贊者曰「拜」,在位者皆拜舞,三稱萬歲。太尉升殿,詣壽尊所,北向,尚食奉御酌御酒一爵授太尉,搢笏執爵詣前跪進,帝執爵,太尉出笏,俯伏,興,少退,跪奏:「文武百寮、太尉具官臣某等稽首言:元正首祚,臣等不勝大慶,謹上千萬壽。」俯伏,興,降,復位。贊者曰「拜」在位者
 皆再拜,三稱萬歲,侍中承旨退,西向宣曰:「舉公等觴。」贊者曰「拜」,在位者皆再拜,三稱萬歲,北向,班分東西序立。太尉自東階侍立。帝舉第一爵,《和安》樂作,飲畢,樂止。太尉受虛爵復於坫,降階。三師以下再拜、舞蹈,稱萬歲如上儀。



 侍中進奏:「侍中具官臣某言,請延公王等升殿。」俯伏,興,降,復位,侍中承旨退,稱「有制」,贊者曰「拜」,在位者皆再拜。宣曰:「延公王等升殿。」贊者曰「拜」,在位者皆再拜。公王等詣東西階,升立於席後。尚食奉御進酒,殿中監省酒以進。
 帝舉第二爵,登歌作《甘露》之曲。飲訖,殿中監受爵,樂止。群臣升殿,就橫行位。舍人曰:「各賜酒。」贊者曰「拜」,群官皆再拜,三稱萬歲。舍人曰:「就坐。」太官令行酒,群官搢笏受酒,宮縣作《正安》之樂,文舞入,立宮架北。觴行一周。凡行酒訖,並太官令奏巡周,樂止。尚食進食升階,以次置御坐前。又設群官食,訖,太官令奏食遍。太樂丞引《盛德升聞》之舞入,作三變,止,出。殿中監進第三爵,群官立席後,登歌作《瑞木成文》之曲。飲訖,樂止。殿中丞受虛爵,舍人曰:「就
 坐。」群官皆坐。又行酒、作樂、進食,如上儀。太樂丞引《天下大定》之舞,作三變,止,出。殿中監進第四爵,登歌奏《嘉禾》之曲,如第三爵。太官令行酒又一周,樂止,舍人曰:「可起。」百寮皆立席後,侍中進御坐前跪奏,禮畢,俯伏,興,與群官俱降階復位,贊者曰:「拜。」皆再拜、舞蹈,三稱萬歲,起,分班立。殿上索扇,扇合,殿下鳴鞭,太樂令撞蕤賓之鐘,左右鐘皆應。協律郎俯伏,舉麾。太樂令令奏《乾安》之樂,鼓吹振作。帝降坐御輿入自東房,扇開,樂止。侍中奏解嚴,
 所司承旨放仗。百寮再拜,相次退。



 舊制,朝賀、上壽,帝執鎮圭,至是始罷不用。



 元祐八年,太常博士陳祥道言:「貴人賤馬,古今所同。故覲禮馬在庭,而侯氏升堂致命。聘禮馬在庭,而賓升堂私覿。今元會儀,御馬立於龍墀之上,而特進以下立於庭,是不稱尊賢才、體群臣之意。請改儀注以御馬在庭,於義為允。」



 舊制,五月朔受朝,熙寧二年詔罷之。元符元年四月,得傳國受命寶,禮官言:「五月朔於故事當大朝會,乞就是日行受寶之禮,依上尊
 號寶冊儀。」前一日,帝齋於殿內,翼日,服通天冠、絳紗袍,御大慶殿,降坐受寶,群臣上壽稱賀。其後,徽宗以元日受八寶及定命寶、冬至日受元圭,皆於大慶殿行朝賀禮。



 《新儀》成,改《元豐儀》太尉為上公,侍中為左輔,中書令為右弼,太樂令為大晟府,《盛德升聞》為《天下化成》之舞,《天下大定》為《四夷來王》之舞,及增刑部尚書奏「天下斷絕,請付史館」,餘並如舊儀。凡遇國恤則廢,若無事不視朝,則下敕云:「不御殿。」群臣進表稱賀於閣門。



 紹興十二
 年十月,臣僚言:「竊以元正一歲之首,冬至一陽之復,聖人重之,制為朝賀之禮焉。自上世以來,未之有改也。漢高祖以五年即位,而七年受朝於長樂宮。我太祖皇帝以建隆元年即位,受朝於崇元殿。主上臨御十有六年,正、至朝賀,初未嘗講,艱難之際宜不遑暇。茲者太母還宮,國家大慶,四方來賀,但惟其時。欲望自今元正、冬至舉行朝賀之禮,以明天子之尊,庶幾舊典不至廢墜。」禮部太常寺考定朝會之禮,依國故事,設黃麾、大仗、車輅、
 法物、樂舞等,百寮服朝服,再拜上壽,宣王公升殿,間飲三周。詔:「自來年舉行。」十一月,權禮部侍郎王賞等言:「朝會之制,正旦、冬至及大慶受朝受賀,系禦大慶殿。其文德、紫宸、垂拱殿禮制各有不同,月朔視朝則御文德殿,謂之前殿正衙,仍設黃麾半仗;紫宸、垂拱皆系側殿,不設儀仗。元正在近,大慶殿之禮事務至多,乞候來年冬至別行取旨。」詔從之。



 明年,閣門言:「依汴京故事,遇行大禮,則冬至及次年正旦朝會皆罷。」



 十四年九月,有司言:「
 明年正旦朝會,請權以文德殿為大慶殿,合設黃麾大仗五千二十七人,欲權減三分之一;合設八寶於御坐之東西,及登歌、宮架、樂舞、諸州諸蕃貢物。行在致仕官、諸路貢士舉首,並令立班。」詔從之。十五年正旦,御大慶殿受朝,文武百官朝賀如儀。



 常朝之儀。唐以宣政為前殿,謂之正衙,即古之內朝也。以紫宸為便殿,謂之入閣,即古之燕朝也。而外又有含元殿,含元非正、至大朝會不御。正衙則日見,群臣百官
 皆在,謂之常參,其後此禮漸廢。後唐明宗始詔群臣每五日一隨宰相入見,謂之起居,宋因其制。皇帝日御垂拱殿。文武官日赴文德殿正衙曰常參,宰相一人押班。其朝朔望亦於此殿。五日起居則於崇德殿或長春殿,中書、門下為班首。長春即垂拱也。至元豐中官制行,始詔侍從官而上日朝垂拱,謂之常參官。百司朝官以上,每五日一朝紫宸,為六參官。在京朝官以上,朔望一朝紫宸,為朔參官、望參官,遂為定制。



 正衙常參。國朝之制:兩省、臺官、文武百官每日赴文德殿立班,宰臣一員押班。常朝官有詔旨免常朝,及勾當更番宿者不赴。遇假並三日以上,即橫行參假。宰相、參知政事及免常朝者悉集事務急速,赴橫行不及者,牒報臺。如遇親王、使相過正衙,則取別旨。群官見、謝、辭者,皆赴正衙。其日,文武班尚書、上將軍以下,並先敘立於殿門之外,東西相向文班一品、二品不敘立。正衙見、謝、辭官立於大班之南,右巡使立正衙位南,北向。臺官大夫、中丞、三院御史各就揖,班位再揖三院不全即不
 揖。揖訖,臺官與左巡使先入,各就位左右巡使立鐘鼓樓下,左巡使奏武班,右巡使奏文班。如只巡使一員,即就入班南立,單奏。如俱闕,即於臺官或員外郎以下差攝。次兩班及右巡使入,次見、謝、辭官入,次兩省官入兩省官自殿西偏門入,於右勤政門北偏門立,候文武班將至,循午階就位,次文班一品、二品入。次宰臣出東上閣門,就位,通事舍人一員立於閣門外,北向,四色官立其後。舍人通承旨奉敕不坐,四色官應喏急趨至放班位宣敕,在位官皆再拜而退。其應橫行者班定,通事舍人揖群官轉班北向,舍人揖再拜復位,如常朝
 之儀兩省官幕次舊在中書門外,近制就使權就朝堂門南上將軍幕次。凡見、謝、辭官新受、加恩、出使到闕者,宰臣、親王、使相俟班定,引贊引出東上閣門,至押班位,西向立定,先赴午階南中書門下正衙位再拜,卻還押班位、樞密使、副使、知院、同知院、簽書院事、參知政事、宣徽使、宗室節度使以下至刺史將軍俟班定,四方館吏引出東上閣門,至殿庭,由東黃道赴正衙位,北向,以西為首,將軍以東為首。正衙畢,宰臣、樞密出西便門,親王宗室入東上閣門,觀文殿大學士、資政殿大學士、觀文殿學士、三司使、翰林資政侍講、侍讀學士、直學士、知制誥、待制直學士以上集丞郎幕次,待制集上將軍幕次。俟班定,四方館吏引入殿西便門赴班,於大夫、中丞
 前出,門下,中書侍郎至正言四方館吏引先集勤政門北,俟班定,於一品、二品官未就位前先就位,放班訖,由西偏門出,御史大夫至御史序班如常朝,三師、三公、僕射,東宮三師、三少班入殿門,朝堂吏引入殿東便門赴班,於兩省、臺官前出,尚書丞郎、左右金吾上將軍至將軍序班如堂朝,節度使至刺史、軍職四廂都指揮使以上,三司副使、文班京朝官、武官郎將以上,分司官、樞密都承旨、諸使副、醫官帶正員官者並文東武西相向,重行序立,餘如常朝,其權三司使、開封府,吏部銓、秘書監、修撰、直館閣校理檢討、三司判官、主判官、開封府
 判官、推官、宮僚、內職、軍校領郡者,內客省使至通事舍人,節度行軍司馬至團練副使,幕職上佐州縣官,諸司勒留官新受者,京朝官改賜章服者,致仕、責授、降授、並謝行軍副使仍辭。京朝官、貢舉發解畢者亦見準儀制,知貢舉官合謝辭。近歲皆實時鎖宿,故謝辭皆停。



 垂拱殿起居,則內侍省都知、押班,率內供奉官以下並寄班等先起居;次客省、閣門使以下呈進目者,次三班使臣節度、觀察、防禦、團練、刺史等子弟充供奉官、侍禁、殿直,有旨令內朝起居者,次內殿當直諸
 班殿前指揮使、左右班都虞候以下、內殿直、散員、散指揮、散都頭、金槍班等,次長入祗候、東西班殿侍,次御前忠佐,次殿前都指揮使率軍校至副指揮使,次駙馬都尉任刺史以上者綴本班,次諸王府僚,次殿前諸軍使、都頭,次皇親將軍以下至殿直,次行門指揮使率行門起居以上並內侍贊喝。如傳宣前殿不坐,即宰相、樞密使、文明殿學士、三司使、翰林樞密直學士、中書舍人、三司副使、知起居注、皇城內監庫藏朝官、諸司使副、內殿崇班、供奉官、侍禁、殿直、翰林醫官、待詔等同班入;次親
 王、侍衛親軍馬步軍都指揮使率軍校至副都指揮使,次使相,次節度使,次統軍,次兩使留後、觀察使,次團練、防禦使、刺史,次侍衛馬步軍使、都頭,起居畢,見、謝班入。如御崇德殿即紫宸殿也。即樞密使以下先就班,候升坐諸司使副以下至殿直,分東西對立,餘皆北面。長春殿皆北面,宰相、參知政事最後入以上並閣門贊喝。日止再拜,朔望及三日假,樞密使以下皆舞蹈。早朝,則宰相、樞密、宣徽使起居畢,升殿問聖體。宰相奏事,樞密、宣徽使退候。宰相對畢,樞密使復入奏事。次三
 司、開封府、審刑院及群臣以次登殿大兩省以上領務京師有公事,許實時請對。自餘受使出入要切者,欲回奏事,則聽先進取旨。其見、謝、辭官,以次入於庭。凡見者先之,謝次之,辭又次之出使閑慢或未升朝官,或止拜於門外,自秘書監、上將軍、觀察使、內客省使以上得拜殿門階上,及升殿止拜御坐前,餘皆庭中班次。惟宰相、親王、使相赴崇德殿,即宣徽使通喚,餘皆側立候通,再拜舞蹈;致辭,即不舞蹈;見,即將相升殿問聖體。其賜分物酒食及收進奉物,皆舞蹈稱謝凡收進奉物皆入謝。幕職、州縣官謝、辭,即判銓官引對,兼於殿門外宣辭戒勵。凡國有大慶
 瑞及出師勝捷,樞密使率內職軍校入賀致辭,閣門使宣答;宰相致辭,宣徽使宣答。如賜酒,即預坐官後入,作樂送酒,如曲宴之儀。晚朝則宰相、樞密、翰林學士當直者,洎近侍執事之臣皆赴。



 乾德六年九月,始以旬假日御講武殿又名崇政,近臣但赴早參宰相以下靴笏,諸司使以下系帶。其節假及大祀,並令如式。



 開寶九年四月,詔旬休日不視事。及太宗即位,復如舊視朝。退進食訖,則易服,御崇政殿。先群臣告謝,次軍頭引見司奏事於殿下,次三班、審官院、
 流內銓、刑部及諸司引見官吏。如假日起居辭見畢,即移御坐,臨軒視事。既退,復有奏事,或閱器物之式者,謂之後殿再坐。



 淳化三年,令有司申舉十五條:常參文武官或有朝堂行私禮,跪拜,待漏行立失序,談笑喧嘩,入正衙門執笏不端,行立遲緩,至班列行立不正,趨拜失儀,言語微喧,穿班仗,出閣門不即就班,無故離位,廊下食、行坐失儀,入朝及退朝不從正衙門出入,非公事入中書。犯者奪奉一月;有司振舉,拒不伏者,錄奏貶降。



 景
 德二年,光祿寺丞錢易言:「竊睹文德殿常朝班不及三四十人,蓋以凡掌職務止赴五日起居,頗違舊章。望令並赴朝參。」乃詔應三館、秘書閣、尚書省二十四司、諸司寺監朝臣內殿起居外,並赴文德殿常參。其審刑院、大理寺、臺直官、開封府判官推官司錄兩縣令、司天監、翰林天文、監倉場庫務等仍免。



 大中祥符二年,御史知雜趙湘言:「伏見常參官每日趨朝,多不整肅。舊制,並早赴待漏院,候開內門齊入。伏緣每日迨辰以朝,以故後時方
 入。又風雨寒暑,即多稱疾,宜令知班驅使官視其入晚者申奏。疾者遣醫親視。」



 天禧四年十月,中書、門下言:「唐朝故事:五日一開延英,只日視事,雙日不坐。方今中外晏寧,政刑清簡,望準舊事,三日、五日一臨軒聽政,只日視事,雙日不坐。至於刑章、錢穀事務,遣差臣僚,除急切大事須面對外,餘並令中書、樞密院附奏。」詔禮儀院詳定,雙日前後殿不坐,只日視事;或於長春殿,或於承明殿,應內殿起居群臣並依常日起居;餘如中書、門下之
 議。俄又請只日承明殿常朝,依假日便服視事,不鳴鞭。詔可。



 康定初,詔中書、樞密、三司,大節、大忌給假一日,小節、旬休並後殿奏事,前後毋得過五班,余聽後殿對,御廚給食。假日,崇政殿辰漏,上入內進食,俟再坐復對。



 神宗即位,御史中丞王陶以《皇祐編敕》宰臣押班儀制移中書,謂「天子新即位,大臣不應隳廢朝儀」,不報。舊制:祖宗以來,日御垂拱殿,待制、諸司使以上俱赴,而百官班文德殿,曰常朝;五日皆入,曰起居。平時,宰相垂拱殿奏
 事畢,赴文德殿押班,或日昃未退,則閣門傳宣放班,多不復赴。王陶以韓琦、曾公亮違故事不押班為不恭,劾之。琦、公亮上表待罪,且言:「唐及五代《會要》,月九開延英,則餘日宰相當押正衙班。及延英對日,未御內殿前,傳宣放班,則宰相不押正衙班明矣。自祖宗繼日臨朝,宰相奏事。至祥符初,始詔循故事,押文德班。以妨職浸廢,乃至今日。請下太常禮院詳定。」陶坐絀。司馬光代為中丞,請令宰相遵國朝舊制押班,不須詳定。尋詔:「宰相春
 分辰初、秋分辰正,垂拱殿未退,聽勿赴文德殿,令御史臺放班。」光又言:「垂拱奏事畢,春分以後鮮有不過辰初,秋分以後鮮有不過辰正,然則自今宰臣常不至文德殿押班。請春分辰正、秋分巳初,奏事未畢,即如今詔,庶幾此禮不至遂廢。」乃詔春秋分率以辰正。



 熙寧六年正月,西上閣門副使張誠一言:「垂拱殿常朝,先內侍唱內侍都知以下至宿衛行門計一十八班起居,後通事舍人引宰執、樞密使以下大班入,次親王,次侍衛馬步軍
 都指揮以下,次皇親使相以下十班入,方引見、謝、辭。或遇百官起居日,自行門後,通事舍人引樞密以下,次親王、使相以下至刺史十班入,方奏兩巡使起居。立定,方引兩省官入,次閣門引宰臣以下大班入。起居畢,候百官出絕,兩省班出,次兩巡使出,中書、樞密方奏事,已是日高。況大班本不分別丞郎、給諫、臺省及常參官,今獨使相以下曲為分別,虛占時刻。請遇垂拱殿百官起居日,將親王以下十班合為四班,親王為一班,侍衛馬
 步軍都指揮使為一班,皇親使相以下至刺史重行異位為兩班,可減六班,如垂拱殿常朝不系百官起居,或紫宸殿百官起居,其親王、使相以下班,並依舊儀序入起居。」從之。九月,引進使李端愨言:「近朔望禦文德殿視朝,祁寒盛暑數煩清蹕,而紫宸之朝歲中罕御。請朔日御文德,既望坐紫宸,庶幾正衙、內殿朝儀並舉。」從之。



 元豐八年二月,詔諸三省、御史臺、寺監長貳、開封府推判官六參,職事官、赤縣丞以上、寄祿升朝官在京厘務者望
 參,不厘務者朔參。



 哲宗元祐四年十月,以戶部尚書呂公孺言,詔朔參官兼赴望參,望參官兼赴六參。五年,詔權侍郎並日參。



 紹聖四年十月,御史臺言:「外任官到闕朝見訖,並令赴朔、望參。」尋又言:「元豐官制,朝參班序有日參、六參、望參、朔參,已著為令。元祐中,改朔參兼赴望參,望參兼赴六參,有失先朝分別等差之意。止依元豐儀令。」從之。



 政和詳定《五禮新儀》,有《文德殿月朔視朝儀》、《紫宸殿望參儀》、《垂拱殿四參儀》、《紫宸殿日參儀》、《垂拱殿
 日參儀》、《崇政殿再坐儀》、《崇政殿假日起居儀》,其文不載。中興仍舊制。



 乾道二年九月,閣門奏:垂拱殿四參四參官謂宰執、侍從、武臣正任、文臣卿監員郎監察御史已上,皇帝坐,先讀奏目。知閣以下,次御帶、環衛官以下,次忠佐、殿前都指揮使以下,次殿前司員僚,次皇太子,次行門已上,逐班並常起居。次樞密、學士、待制、樞密都承旨以下,知閣並祗應武功大夫以下,通班常起居。次親王,次馬步軍都指揮使,次使相,次馬步軍員僚已上,逐班並常起居。次殿中侍御史入
 側宣大起居訖,歸侍立位。次宰執以下,並兩省官、文武百官入,相向立定,通班面北立,大起居訖凡常起居兩拜,大起居七拜,三省升殿侍立。次兩省官出,次殿中侍御史對揖出,三省、樞密院奏事,次引見、謝、辭,次引臣僚奏事訖,皇帝起。詔:「今後遇四參日,分起居班次,可移殿中侍史及宰執以下百官班,令次樞密以下班起居。卻令親王並殿前都指揮使以下殿前司員僚,逐班於宰執以下班後起居,餘並從之。」



 淳熙七年九月,詔:「自今垂拱殿日參,
 宰臣特免宣名。」



 嘉定十二年正月,臣僚奏:「竊見皇帝御正殿,或御後殿,固可間舉,四參官亦有定日。近者每日改常朝為後殿,四參之禮亦多不講,正殿、後殿、四參間免。陛下臨朝之日固未嘗輟,而外廷不知聖意,或謂姑從簡便,非所以肅百執事也。常朝之禮止於從臣,後殿之儀從臣不與,四參止及卿郎,而乃累月僅或一舉。咫尺天威,疏簡至此,非所以尊君上而勵百闢也。伏願陛下嚴常朝、後殿、四參之禮,起群下肅謹之心,彰明時厲
 精之治,豈不偉哉。」從之。



 初,群臣見、辭、謝皆赴正衙。淳化二年,知雜御史張鬱言:「正衙之設謂之外朝,凡群臣辭、見及謝,先詣正衙,見訖,御史臺具官位姓名以報,閣門方許入對,此國家舊制也。自乾德後,始詔先赴中謝,後詣正衙。而文武官中謝後,次日並赴正衙,內諸司遙領刺史、閣門通事舍人以上新授者亦赴正衙辭謝,出使急速免衙辭者亦具狀報臺,違者罰奉一月。其內諸司職官並將校至刺史以上新授者,欲望同百官例,赴正
 衙謝。」從之。元豐既定朝參之制,侍御史知雜事滿中行上言:「文德正衙之制,尚存常朝之虛名,襲橫行之謬例,有司失於申請,未能厘正。兩省、臺官、文武百官赴文德殿,東西相向對立,宰臣一員押班,聞傳不坐,則再拜而退,謂之常朝。遇休假並三日以上,應內殿起居官畢集,謂之橫行。自宰臣、親王以下應見、謝、辭者,皆先赴文德殿,謂之過正衙。然在京厘務之官例以別敕免參,宰臣押班近年已罷,而武班諸衙本朝又不常置。故今之赴
 常朝者,獨御史臺官與審官、待次階官而已。今垂拱內殿宰臣以下既已日參,而文德常朝仍復不廢,舛謬倒置,莫此為甚。至於橫行參假,與夫見、謝、辭官先過正衙,雖沿唐之故事,然必俟天子御殿之日行之可也。」詔下詳定官制所。言:「今天子日聽政於垂拱,以接執政官及內朝之臣,而更於別殿宣敕不坐,實為因習之誤。兼有執事升朝官五日一赴起居,而未有執事者反謂之參,疏數之節尤為未當。又辭、見、謝,自已入見天子,則前殿
 正衙對拜為虛文。其連遇朝假,則百官司赴大起居,不當復有橫行參假。宜如中行言。」於是常朝、正衙、橫行之儀俱罷。



\end{pinyinscope}