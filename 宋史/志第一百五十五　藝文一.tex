\article{志第一百五十五 藝文一}

\begin{pinyinscope}

 《易》曰:「觀乎天文,以察時變;觀乎人文,以化成天下。」文之有關於世運,尚矣。然書契以來,文字多而世代日降;秦火而後,文字多而世教日興,其故何哉?蓋世道升降,人
 心習俗之致然,非徒文字之所為也。然去古既遠,茍無斯文以範防之,則愈趨而愈下矣。故由秦而降,每以斯文之盛衰,占斯世之治忽焉。



 宋有天下先後三百餘年,考其治化之污隆,風氣之離合,雖不足以擬倫三代,然其時君汲汲於道藝,輔治之臣莫不以經術為先務,學士搢紳先生,談道德性命之學,不絕於口,豈不彬彬乎進於周之文哉!宋之不競,或以為文勝之弊,遂歸咎焉,此以功利為言,未必知道者之論也。



 歷代之書籍,莫厄
 於秦,莫富於隋、唐。隋嘉則殿書三十七萬卷。而唐之藏書,開元最盛,為卷八萬有奇。其間唐人所自為書,幾三萬卷,則舊書之傳者。至是蓋亦鮮矣。陵遲逮於五季,干戈相尋,海寓鼎沸,斯民不復見《詩》、《書》、《禮》、《樂》之化。周顯德中,始有經籍刻板,學者無筆札之勞,獲睹古人全書。然亂離以來,編帙散佚,幸而存者,百無二三。



 宋初,有書萬餘卷。其後削平諸國,收其圖籍,及下詔遣使購求散亡,三館之書,稍復增益。太宗始於左升龍門北建崇文院,
 而徙三館之書以實之。又分三館書萬餘卷別為書庫,目曰「秘閣」。閣成,親臨幸觀書,賜從臣及直館宴。又命近習侍衛之臣縱觀群書。



 真宗時,命三館寫四部書二本,置禁中之龍圖閣及後苑之太清樓,而玉宸殿、四門殿亦各有書萬餘卷。又以秘閣地隘,分內藏西庫以廣之,其右文之意,亦云至矣。已而王宮火,延及崇文、秘閣,書多煨燼。其僅存者,遷於右掖門外,謂之崇文外院,命重寫書籍,選官詳覆校勘,常以參知政事一人領之,書成,
 歸於太清樓。



 仁宗既新作祟文院,命翰林學士張觀等編四庫書,仿《開元四部錄》為《崇文總目》,書凡三萬六百六十九卷。神宗改官制,遂廢館職,以崇文院為秘書省,秘閣經籍圖書以秘書郎主之,編輯校定,正其脫誤,則主於校書郎。



 徽宗時,更《崇文總目》之號為《秘書總目》。詔購求士民藏書,其有所秘未見之書足備觀採者,仍命以官。且以三館書多逸遺,命建局以補全校正為名,設官總理,募工繕寫。一置宣和殿,一置太清樓,一置秘閣。
 自熙寧以來,搜訪補輯,至是為盛矣。



 嘗歷考之,始太祖、太宗、真宗三朝,三千三百二十七部,三萬九千一百四十二卷。次仁、英兩朝,一千四百七十二部,八千四百四十六卷。次神、哲、徽、欽四朝,一千九百六部,二萬六千二百八十九卷。三朝所錄,則兩朝不復登載,而錄其所未有者。四朝於兩朝亦然。最其當時之目,為部六千七百有五,為卷七萬三千八百七十有七焉。迨夫靖康之難,而宣和、館閣之儲蕩然靡遺。高宗移蹕臨安,乃建秘書
 省於國史院之右,搜訪遺闕,屢優獻書之賞,於是四方之藏,稍稍復出,而館閣編輯,日益以富矣。當時類次書目,得四萬四千四百八十六卷。至寧宗時續書目,又得一萬四千九百四十三卷,視《崇文總目》,又有加焉。自是而後,迄於終祚,國步艱難,軍旅之事,日不暇給,而君臣上下,未嘗頃刻不以文學為務,大而朝廷,微而草野,其所制作、講說、紀述、賦詠,動成卷帙,壘而數之,有非前代之所及也。雖其間鈲裂大道,疣贅聖謨,幽怪恍惚,繁瑣
 支離有所不免,然而瑕瑜相形,雅鄭各趣,譬之萬派歸海,四瀆可分,繁星麗天,五緯可識,求約於博,則有要存焉。



 宋舊史,自太祖至寧宗,為書凡四。志藝文者,前後部帙,有亡增損,互有異同。今刪其重復,合為一志,蓋以寧宗以後史之所未錄者,仿前史分經、史、子、集四類而條列之,大凡為書九千八百十九部,十一萬九千九百七十二卷雲。



 經類十:一曰《易》類,二曰《書》類,三曰《詩》類,四曰《禮》類,五曰《樂》類,六曰《春秋》類,七曰《孝經》類,八曰《論語》類,
 九曰經解類,十曰小學類。



 《周易古經》一卷



 薛貞注《歸藏》三卷



 《易傳》十卷題卜子夏傳



 《周易上下經》六卷



 《系辭說卦序卦雜卦》三卷韓康伯注。



 鄭玄《周易文言注義》一卷



 王弼《略例》一卷



 《易辨》一卷



 阮嗣宗《通易論》一卷



 乾寶《易傳》十卷



 《易髓》八卷晉人撰,不知姓名



 孔穎達《正義》十四卷



 《玄談》六卷



 《易正義補闕》七卷



 任正一《甘棠正義》三十卷



 關朗《易傳》一卷



 王肅《傳》十一卷



 陸德明《釋文》一卷



 衛元嵩《周易元包》十卷蘇源明傳,李江注



 李鼎祚《集解》
 十卷



 史文徽《易口訣義》六卷



 成玄英《流演窮寂圖》五卷



 蔡廣成《啟源》十卷



 又《周易外義》三卷



 沙門一行《傳》十二卷



 王隱《要削》三卷



 陸希聲《傳》十三卷



 郭京《舉正》三卷



 東鄉助《物象釋疑》一卷



 邢璹《補闕周易正義略例疏》三卷



 李翱《易詮》七卷



 張弧《周易上經王道小疏》五卷



 張韓《啟玄》一卷



 青城山人《揲蓍法》一卷



 王昭素《易論》三十三卷



 縱康乂《周易會通正義》三十三卷



 陰洪道《周易新論傳疏》十卷



 陳摶《易龍圖》一卷



 範諤昌《大易源流圖》一卷



 又《
 證墜簡》一卷



 胡旦《易演聖通論》十六卷



 石介《口義》十卷



 冀震《周易義略》十卷



 代淵《周易旨要》二十卷



 何氏《易講疏》十三卷不著名



 陸秉《意學》十卷



 《古易》十三卷出王洙家



 王洙《言象外傳》十卷



 劉牧《新注周易》十一卷



 又《卦德通論》一卷



 《易數鉤隱圖》一卷



 吳秘《周易通神》一卷



 黃黎獻《略例》一卷



 又《室中記師隱訣》一卷



 龔鼎臣《補注易》六卷



 彭汝礪《易義》十卷



 趙令湑《易發微》十卷



 喬執中《易說》十卷



 趙仲銳《易義》五卷



 謝湜《易義》十二卷



 譚世績《易傳》十卷



 陸
 太易《周易口訣》七卷



 冀珍《周易闡微詩》六卷



 李贊《周易說》九卷



 張杲《周易罔象成名圖》一卷



 裴通《周易玄解》三卷



 邵雍《皇極經世》十二卷



 又《敘篇系述》二卷



 《觀物外篇》六卷門人張愍記雍之言



 《觀物內篇解》二卷雍之子伯溫編



 邵伯溫《周易辨惑》一卷



 常豫《易源》一卷



 徐庸《周易意蘊凡例總論》一卷



 又《卦變解》二卷



 宋咸《易訓》三卷



 又《易補注》十卷



 又《劉牧王弼易辨》二卷



 皇甫泌《易解》十九卷



 鄭揚庭《時用書》二十卷



 又《明用書》九卷



 《易傳辭》三卷



 《易傳辭後語》一
 卷



 陳良獻《周易發隱》二十卷



 石汝礪《乾生歸一圖》十卷



 鮑極《周易重注》十卷



 葉昌齡《圖義》二卷



 胡瑗《易解》一十二卷



 《口義》十卷



 《系辭說卦》三卷



 歐陽修《易童子問》三卷



 阮逸《易筌》六卷



 王安石《易解》十四卷



 尹天民《易論要纂》一卷



 又《易說拾遺》二卷



 司馬光《易說》一卷



 又三卷



 《系辭說》二卷



 鮮于侁《周易聖斷》七卷



 蘇軾《易傳》九卷



 程頤《易傳》九卷



 又《易系辭解》一卷



 張載《易說》十卷



 呂大臨《易章句》一卷



 龔原《續解易義》十七卷



 又《易傳》十卷



 李平西《河圖
 傳》一卷



 李遇《刪定易圖序論》六卷



 張弼《易解義》十卷



 顧叔思《周易義類》三卷



 劉概《易系辭》十卷



 晁說之《錄古周易》八卷



 晁補之《太極傳》五卷



 《因說》一卷



 《太極外傳》一卷



 游酢《易說》一卷



 耿南仲《易解義》十卷



 安泳《周易解義》一部卷亡



 陳瓘《了齋易說》一卷



 鄒浩《系辭纂義》二卷



 張根《易解》九卷



 《周易六十四卦賦》一卷題穎川陳君作,名亡



 林德祖《易說》九卷



 陳禾《易傳》十二卷



 李授之《易解通義》三十卷



 朱震《易傳》十一卷



 《卦圖》三卷



 《易傳叢說》一卷



 張汝明《易索》十
 三卷



 郭忠孝《兼山易解》二卷



 又《四學淵源論》三卷



 任奉古《周易發題》一卷



 陳高《八卦數圖》二卷



 林壝《易說》十二卷



 《變卦》八卷



 《變卦纂集》一卷



 凌唐佐《集解》六卷



 袁樞《學易索隱》一卷



 夏休《講義》九卷



 郭雍《傳家易解》十一卷



 沈該《易小傳》六卷



 都絜《易變體》十六卷



 鄭克《揲蓍古法》一卷



 吳沆《易璇璣》三卷



 李椿年《易解》八卷



 《疑問》一卷



 李光《易說》十卷



 李衡《易義海撮要》十二卷



 洪興祖《易古經考異釋疑》一卷



 張行成《元包數總義》二卷



 《述衍》十八卷



 《通
 變》四十八卷



 晁公武《易詁訓傳》十八卷



 胡銓《易傳拾遺》十卷



 程大昌《易原》十卷



 又《易老通言》十卷



 楊萬里《易傳》二十卷



 林慄《易經傳集解》三十六卷



 李舜臣《易本傳》三十三卷



 曾穜《大易粹言》十卷



 呂祖謙《定古易》十二篇為一卷



 又《音訓》二卷



 《周易系辭精義》二卷



 朱熹《易傳》十一卷



 又《本義》十二卷



 《易學啟蒙》三卷



 《古易音訓》二卷



 張浚《易傳》十卷



 倪思《易訓》三十卷



 趙善譽《易說》二卷



 劉文鬱《易宏綱》八卷



 吳仁傑《古易》十二卷



 又《周易圖說》二卷



 《集
 古易》一卷



 王日休《龍舒易解》一卷



 劉翔《易解》六卷



 胡有開《易解義》四十卷



 鄒巽《易解》六卷



 鄭剛中《周易窺餘》十五卷



 楊簡《己易》一卷



 潘夢旗《大易約解》九卷



 麻衣道者《正易心法》一卷



 鄭東卿《易說》三卷



 項安世《周易玩辭》十六卷



 程迥《易章句》十卷



 又《外編》一卷



 《占法》



 《古易考》一卷



 林至《易裨傳》一卷



 葉適《習學記言·周易述釋》一卷



 李椿《觀畫》二卷



 王炎《筆記》八卷



 鄭汝諧《易翼傳》二卷



 湯羲《周易講義》三卷



 樂只道人《羲文易論微》六卷姓名亡



 朱氏《三
 宮易》一卷名亡



 劉烈《虛谷子解卦周易》三卷



 劉牧、鄭夫注《周易》七卷



 楊文煥《五十家易解》四十三卷



 孫份《周易先天流衍圖》十二卷程敦厚序



 劉半千《羲易正元》一卷



 馮椅《易學》五十卷



 商飛卿《講義》一卷



 《周易卦類》三卷



 《易辭微》三卷



 《易正經明疑錄》一卷



 《易傳》四卷



 《口義》六卷



 《易樞》十卷



 《系辭要旨》一卷



 並不知作者



 《易乾鑿度》三卷



 《易緯》七卷



 《易緯稽覽圖》一卷



 《易通卦驗》二卷



 並鄭玄注



 《流演通卦驗》一卷不知作者



 王柏《讀易記》十卷



 又《涵古易說》一卷



 《大象衍義》一卷



 曾幾《易釋象》五卷



 劉禹偁《易解》十卷



 程達《易解》十卷



 戴溪《易總說》二卷



 趙汝談《易說》三卷



 真德秀《復卦說》一卷



 吳如愚《易說》一卷



 李光《易傳》十卷



 李燾《易學》五卷



 又《大傳雜說》一卷



 朱承祖《易摭卦總論》一十卷



 林起鰲《易述古言》二卷



 方實孫《讀易記》八卷



 魏了翁《易集義》六十四卷



 又《易要義》一十卷



 鄭子厚《大易觀象》三十二卷張野補注



 右《易》類二百十三部,一千七百四十卷。王柏《讀易記》以下不著錄十九部,一百八十六卷



 《
 尚書》十二卷漢孔安國傳



 《古文尚書》二卷孔安國隸



 伏勝《大傳》三卷鄭玄注



 《汲塚周書》十卷晉太康中,於汲郡得之。孔晁注



 陸德明《釋文音義》一卷



 孔穎達《正義》二十卷



 馮繼先《尚書廣疏》十八卷



 又《尚書小疏》十三卷



 尹恭初《尚書新修義疏》二十六卷



 胡旦《尚書演聖通論》七卷



 胡瑗《洪範口義》一卷



 蘇洵《洪範圖論》一卷



 程頤《堯典舜典解》一卷



 王安石《新經書義》十三卷



 又《洪範傳》一卷



 蘇軾《書傳》十三卷



 《書說》一卷程頤門人記



 孔武仲《書說》十三卷



 曾肇《書講義》八卷



 陳諤《開寶
 新定尚書釋文》三卷



 孟先《禹貢治水圖》一卷



 《尚書洪範五行記》一卷



 王晦叔《周書音訓》十二卷



 司馬康等《無逸講義》一卷



 吳安詩等《無逸說命解》二卷



 劉彞《洪範解》六卷



 曾旼等《講義》三十卷



 葉夢得《書傳》十卷



 張綱《解義》三十卷



 吳孜《大義》三卷



 吳棫《裨傳》十二卷



 張九成《尚書詳說》五十卷



 洪興祖《口義發題》一卷



 陳鵬飛《書解》三十卷



 程大昌《書譜》二十卷



 又《禹貢論》五卷



 《禹貢論圖》五卷



 《禹貢後論》一卷



 晁公武《尚書詁訓傳》四十六卷



 史浩《講義》
 二十二卷



 呂祖謙《書說》三十五卷



 黃度《書說》七卷



 李舜臣《尚書小傳》四卷



 吳仁傑《尚書洪範辨圖》一卷



 陳伯達《翼範》一卷



 朱熹《書說》七卷黃士毅集



 林之奇《集解》五十八卷



 陳經《詳解》五十卷



 康伯成《書傳》一卷



 夏僎《書解》十六卷



 王炎《小傳》十八卷



 孫泌《尚書解》五十二卷



 蔡沉《書傳》六卷



 胡瑗《尚書全解》二十八卷



 成申之《四百家集解》五十八卷



 楊王集《尚書義宗》三卷



 《三墳書》三卷元豐中毛漸所得



 《尚書治要圖》五卷



 《尚書解題》一卷



 《渾灝發旨》一卷



 並不知作者



 王柏《讀書記》十卷



 又《書疑》九卷



 《書附傳》四十卷



 袁燮《書鈔》十卷



 袁覺《讀書記》二十三卷



 黃倫《尚書精義》六十卷



 趙汝談《書說》二卷



 卞大亨《尚書類數》二十卷



 胡銓《書解》四卷



 李燾《尚書百篇圖》一卷



 劉甄《書青霞集解》二十卷



 應鏞《書約義》二十五卷



 魏了翁《書要義》二十卷



 右《書》類六十部,八百二卷。王柏《讀書記》以下不著錄十三部,二百四十四卷



 《韓詩外傳》十卷漢韓嬰傳



 《毛詩》二十卷漢毛萇為詁訓傳,鄭玄箋



 鄭玄《詩譜》三卷



 陸璣《草木鳥獸蟲魚疏》二卷



 孔穎達《正義》四十
 卷



 陸德明《詩釋文》三卷



 成伯璵《毛詩指說統論》一卷



 又《毛詩斷章》二卷



 張欣《別錄》一卷



 《毛詩正數》二十卷



 《毛詩釋題》二十卷



 《毛詩小疏》二十卷



 鮮于侁《詩傳》六十卷



 李常《詩傳》十卷



 魯有開《詩集》十卷



 胡旦《毛詩演聖通論》二十卷



 宋咸《毛詩正紀》三卷



 又《外義》二卷



 劉宇《詩折衷》二十卷



 蘇子才《毛詩大義》三卷



 周軾《箋傳辨誤》八卷



 丘鑄《周詩集解》二十卷



 歐陽修《詩本義》十六卷



 又《補注毛詩譜》一卷



 蘇轍《詩解集傳》二十卷



 彭汝礪《詩義》二十卷



 趙
 令湑《講義》二十卷



 喬執中《講義》十卷



 毛漸《詩集》十卷



 沉銖《詩傳》二十卷



 孔武仲《詩說》二十卷



 王商範《毛詩序義索隱》二卷



 王安石《新經毛詩義》二十卷



 《舒王詩義外傳》十二卷



 《新解》一卷程頤門人記其師之說



 張載《詩說》一卷



 趙仲銳《詩義》三卷



 游酢《詩二南義》一卷



 範祖禹《詩解》一卷



 楊時《詩辨疑》一卷



 茅知至《周詩義》二十卷



 蔡卞《毛詩名物解》二十卷



 董逌《廣川詩故》四十卷



 吳良輔《詩重文說》七卷



 劉孝孫《正論》十卷



 吳景山《十五國風咨解》一卷



 劉泉《毛
 詩判篇》一卷



 吳棫《毛詩葉韻補音》十卷



 李樗《毛詩詳解》四十六卷



 晁公武《毛詩詁訓傳》二十卷



 呂祖謙《家塾讀詩記》三十二卷



 鄭樵《詩傳》二十卷



 又《辨妄》六卷



 範處義《詩學》一卷



 又《解頤新語》十四卷



 《詩補傳》三十卷



 朱熹《詩集傳》二十卷



 《詩序辨》一卷



 張貴謨《詩說》三十卷



 鄭諤《毛詩解義》三十卷



 黃度《詩說》三十卷



 吳氏《詩本義補遺》二卷名亡



 戴溪《續讀詩記》三卷



 錢文子《白石詩傳》一十卷



 又《詩訓詁》三卷



 黃邦彥《講義》三卷



 鮮於戣《詩頌解》三卷



 黃
 椿《詩解》二十卷



 《總論》一卷



 林岊《講義》五卷



 《三十家毛詩會解》一百卷吳純編,王安石解義



 《毛詩釋篇目疏》十卷



 《詩疏要義》一卷



 《毛詩玄談》一卷



 《毛詩章疏》三卷



 《毛詩提綱》一卷



 《毛詩名物性門類》八卷



 《義方》二十卷



 《釋文》二十卷



 《通義》二十卷



 《毛鄭詩學》十卷



 《詩關雎義解》一部



 《比興窮源》一卷



 並不知作者



 陳寅《詩傳》十卷



 許奕《毛詩說》三卷



 李燾《詩譜》三卷



 王應麟《詩考》五卷



 又《詩地理考》五卷



 《詩草木鳥獸蟲魚廣疏》六卷



 輔廣《詩說》一部



 嚴粲《詩集》一部



 王質《詩總
 聞》二十卷



 魏了翁《詩要義》二十卷



 王柏《詩辨說》二卷



 又《詩可言》二十卷



 高端叔《詩說》一卷



 曹粹中《詩說》三十卷



 項安世《毛詩前說》一卷



 又《詩解》二十卷



 鄭庠《詩古音辨》一卷



 右《詩》類八十二部,一千一百二十卷。陳寅《詩傳》以下不著錄十四部,二百四十五卷



 《儀禮》十七篇高堂生傳



 《大戴禮記》十三卷戴德纂



 《禮記》二十卷戴聖纂



 鄭玄《古禮注》十七卷



 又《周禮注》十二卷



 《禮記注》二
 十卷



 《禮記月令注》一卷



 崔靈恩《三禮義宗》三十卷



 成伯璵《禮記外傳》十卷張幼倫注



 韋彤《五禮精義》十卷



 又《五禮緯書》二十卷



 丘光庭《兼明書》四卷



 杜肅《禮略》十卷



 陸德明《音義》一卷



 又《古禮釋文》一卷



 賈公彥《儀禮疏》五十卷



 又《禮記疏》五十卷



 《周禮疏》五十卷



 孔穎達《禮記正義》七十卷



 聶崇義《三禮圖集注》二十卷



 楊逢殷《禮記音訓指說》二十卷



 上官均《曲禮講義》二十卷



 歐陽丙《三禮名義》五卷



 魯有開《三禮通義》五卷



 殷介集《五禮極義》一卷



 孫玉汝《
 五禮名義》十卷



 餘希文《井田王制圖》一卷



 《胡先生中庸義》一卷盛喬纂集



 李洪澤《直禮》一卷



 張詵《喪禮》十卷



 《禮粹》二十卷不知作者



 王愨《中禮》八卷



 程顥《中庸義》一卷



 呂大臨《大學》一卷



 又《中庸》一卷



 《禮記傳》十六卷



 喬執中《中庸義》一卷



 游酢《中庸解義》五卷



 王安石《新經周禮義》二十二卷



 王昭禹《周禮詳解》四十卷



 陸佃《禮記解》四十卷



 又《禮象》十五卷



 《述禮新說》四卷



 《儀禮義》十七卷



 何洵直《禮論》一卷



 陸佃《大裘議》一卷



 郭忠孝《中庸說》一卷



 龔原《周禮圖》
 十卷



 郭雍《中庸說》一卷



 陳詳道《批注儀禮》三十二卷



 又《禮例詳解》十卷



 《禮書》一百五十卷



 陳暘《禮記解義》十卷



 李格非《禮記精義》十六卷



 楊時《周禮義辨疑》一卷



 又《中庸解》一卷



 喻樗《大學解》一卷



 司馬光等《六家中庸大學解義》一卷



 江與山《周禮秋官講義》一卷



 馬希孟《禮記解》七十卷



 《四先生中庸解義》一卷程頤、呂大臨、游酢、楊時撰



 方愨《禮記解義》二十卷



 王普《深衣制度》一卷



 夏休《周禮井田譜》二十卷



 《破禮記》二十卷



 周燔《儀禮詳解》十七卷



 李如圭《儀
 禮集釋》十七卷



 史浩《周官講義》十四卷



 鄭諤《周禮解義》二十二卷



 黃度《周禮說》五卷



 徐煥《周官辨略》十八卷



 陳傅良《周禮說》一卷



 徐行《周禮微言》十卷



 易祓《周禮總義》三十六卷



 朱熹《儀禮經傳通解》二十三卷



 又《大學章句》一卷



 《或問》二卷



 《中庸章句》一卷



 《或問》二卷



 《中庸輯略》二卷



 《十先生中庸集解》二卷朱熹序



 《三家冠婚喪祭禮》五卷司馬光、程頤、張載定



 吳仁傑《禘祫綿蕞書》三卷



 劉彞《周禮中義》十卷



 張九成《中庸說》一卷



 《大學說》一卷



 戴溪《曲禮口義》二
 卷



 《學記口義》二卷



 司馬光《中庸大學廣義》一卷



 錢文子《中庸集傳》一卷



 胡銓《禮記傳》十八卷



 又《周禮傳》十二卷



 《二禮講義》一卷



 倪思《中庸集義》一卷



 汪應辰《二經雅言》二卷



 張淳《儀禮識誤》一卷



 俞庭椿《周禮復古編》三卷



 黃幹《續儀禮經傳通解》二十九卷



 又《儀禮集傳集注》十四卷



 林椅《周禮綱目》八卷



 《摭說》一卷



 鄭景炎《周禮開方圖說》一卷



 李心傳《丁丑三禮辨》二十三卷



 鄭伯謙《太平經國書統集》七卷



 鄭氏《三禮名義疏》五卷不著名



 又《三禮圖》
 十二卷



 《江都集禮圖》五十卷



 《三禮圖駁議》二十卷



 《儀禮類例》十卷



 《周禮類例義斷》二卷



 《二禮分門統要》三十六卷



 《禮記小疏》二十卷



 並不知作者



 石(敦山)《中庸集解》二卷



 項安世《中庸說》一卷



 又《周禮丘乘圖說》一卷



 衛湜《禮記集說》一百六十卷



 楊簡《孔子閑居講義》一卷



 鄭樵《鄉飲禮》七卷



 張虙《月令解》十二卷



 晁公武《中庸大傳》一卷



 楊復《儀禮圖解》十七卷



 魏了翁《儀禮要義》五十卷



 又《禮記要義》三十三卷



 《周禮折衷》二卷



 《周禮要義》三十卷



 趙順孫《中
 庸纂疏》三卷



 袁甫《中庸詳說》二卷



 陳堯道《中庸說》十三卷



 又《大學說》十一卷



 真德秀《大學衍義》四十二卷



 謝興甫《中庸大學講義》三卷



 王與之《周禮訂義》八十卷



 王應麟《集解踐祚篇》一冊



 右《禮》類一百十三部,一千三百九十九卷。石(敦山)《中庸集解》以下不著錄二十六部,四百六十九卷



 蔡琰《胡笳十八拍》四卷



 孔衍《琴操引》三卷



 謝莊《琴論》一卷



 梁武帝《鐘律緯》一卷



 陳僧智匠《古今樂錄》十三卷



 趙
 邦利《彈琴手勢譜》一卷



 又《彈琴右手法》一卷



 唐玄宗《金風樂弄》一卷



 太宗《九弦琴譜》二十卷



 《琴譜》六卷



 《唐宗廟用樂儀》一卷



 《唐肅明皇后廟用樂儀》一卷



 崔令欽《教坊記》一卷



 吳兢《樂府古題要解》二卷



 王昌齡《續樂府古解題》一卷



 劉貺《大樂令壁記》三卷



 《大樂圖義》一卷不知作者



 田琦《聲律要訣》十卷



 薛易簡《琴譜》一卷



 段安節《琵琶錄》一卷



 又《樂府雜錄》二卷



 《樂府古題》一卷



 陸鴻漸《教坊錄》一卷



 李勉《琴說》一卷



 陳拙《琴籍》九卷



 徐景安《新纂樂書》三
 十卷



 趙惟簡《琴書》三卷



 宋仁宗《明堂新曲譜》一卷



 又《景祐樂髓新經》一卷



 《審樂要記》二卷



 徽宗《黃鐘徵角調》二卷



 沉括《樂論》一卷



 又《樂器圖》一卷



 《三樂譜》一卷



 《樂律》一卷



 馮元、宋祁《景祐廣樂記》八十一卷



 宋祁《大樂圖》一卷



 聶冠卿《景祐大樂圖》二十卷



 劉次莊《樂府集》十卷



 《樂府集序解》一卷



 《大周正樂》八十八卷五代周竇儼訂論



 《蜀雅樂儀》三十卷



 房庶《補亡樂書總要》三卷



 《真館飲福等》一卷



 蔡攸《燕樂》三十四冊



 範鎮《新定樂法》一卷



 崔遵度《琴箋》一
 卷



 李宗諤《樂纂》一卷



 陳康士《琴調》三卷



 又《琴調》十七卷



 《琴書正聲》十卷



 《琴調》十七卷



 《琴譜記》一卷



 《琴調譜》一卷



 《楚調五章》一卷



 《離騷譜》一卷



 李約《琴曲東杓譜》一卷



 《琴調廣陵散譜序》一卷



 獨孤寔《九調譜》一卷



 齊嵩《琴雅略》一卷



 僧辨正《琴正聲九弄》九卷



 朱文齊《琴雜調譜》十二卷



 蕭祐一作「祜」



 《無射商九調譜》一卷



 呂渭一作「濱」



 《廣陵止息譜》一卷



 張淡正《琴譜》一卷



 蔡翼《琴調》一卷



 僧道英《琴德譜》一卷



 王邈《琴譜》一卷



 沉氏《琴書》一卷失名



 《琴譜調》
 八卷李翱用指法



 《琴略》一卷



 《琴式圖》一卷



 《琴譜纂要》五卷



 胡瑗《景祐樂府奏議》一卷



 又《皇祐樂府奏議》一卷



 阮逸《皇祐新樂圖記》三卷



 陳暘《樂書》二百卷



 僧靈操《樂府詩》一卷



 吳良輔《琴譜》一卷



 又《樂書》五卷



 《樂記》三十六卷



 楊傑《元豐新修大樂記》五卷



 劉昺《大晟樂書》二十卷



 又《樂論》八卷



 《運譜四議》二十卷



 《政和頒降樂曲樂章節次》一卷



 《政和大晟樂府雅樂圖》一卷



 鄭樵《系聲樂譜》二十四卷



 李南玉《古今大樂指掌》三卷



 郭茂倩《樂府詩集》一百卷



 李昌
 文《阮咸弄譜》一卷



 滕康叔《韶武遺音》一卷



 曲瞻《琴聲律》二卷



 又《琴圖》一卷



 令狐揆《樂要》三卷



 王大方《琴聲韻圖》一卷



 《昭微古今琴樣》一卷



 劉籍《琴義》一卷



 沉建《樂府廣題》二卷



 馬以良《琴譜三均》三卷



 喻修樞《阮咸譜》一卷



 吳仁傑《樂舞新書》二卷



 蔡元定《律呂新書》二卷



 李如篪《樂書》一卷



 《琴說》一卷



 《古樂府》十卷



 趙德先《樂說》三卷



 又《樂書》三十卷



 《歷代樂儀》三十卷



 《樂苑》五卷



 《琴箋知音操》一卷



 《樂府題解》一卷



 《大樂署》三卷



 《歷代歌詞》六卷



 《律呂圖》
 一卷



 《仿蔡琰胡笳十八拍》



 並不知作者



 右《樂》類一百十一部,一千七卷。



 《春秋》七卷正經



 杜預《春秋左氏傳經傳集解》三十卷



 又《春秋釋例》十五卷



 何休《公羊傳》十二卷



 又《左氏膏肓》十卷



 範寧《穀梁傳》十二卷



 董仲舒《春秋繁露》十七卷



 《汲塚師春》一卷師春純集疏《左傳》卜筮事



 荀卿《公子姓譜》二卷一名《帝王歷紀譜》



 劉炫《春秋述議略》一卷



 又《春秋義囊》二卷



 孔穎達《春秋左氏傳正義》三十六卷



 《公羊疏》三十卷



 楊士勛《春秋穀梁
 疏》十二卷



 黃恭密《春秋指要圖》一卷



 李瑾《春秋指掌圖》十五卷



 陳岳《春秋折衷論》三十卷



 《春秋災異錄》六卷



 《春秋謚族圖》五卷



 陸德明《三傳釋文》八卷



 陸希聲《春秋通例》三卷



 趙匡《春秋闡微纂類義統》十卷



 陸淳《集傳春秋纂例》十卷



 又《春秋辨疑》七卷



 《集注春秋微旨》三卷



 盧仝《春秋摘微》四卷



 楊蘊《春秋公子譜》一卷



 左丘明《春秋外傳國語》二十一卷韋昭注



 柳宗元《非國語》二卷



 葉真《是國語》七卷



 馮繼先《春秋名號歸一圖》



 又《春秋名字同異錄》
 五卷



 杜預《春秋世譜》七卷



 張暄《春秋龜鑒圖》一卷



 馬擇言《春秋要類》五卷



 徐彥《公羊疏》三十卷



 葉清臣《春秋纂類》十卷



 孫復《春秋尊王發微》十二卷



 《春秋總論》一卷



 李堯俞《春秋集議略論》二卷



 王沿《春秋集傳》十五卷



 章拱之《春秋統微》二十五卷



 王哲《春秋通義》十二卷



 又《皇綱論》五卷



 丁副《春秋演聖統例》二十卷



 《春秋三傳異同字》一卷



 朱定序《春秋索隱》五卷



 杜諤《春秋會義》二十六卷



 胡瑗《春秋口義》五卷



 劉敞《春秋傳》十五卷



 又《春秋權衡》
 十七卷



 《春秋說例》十一卷



 《春秋意林》二卷



 蘇轍《春秋集傳》十二卷



 王安石《左氏解》一卷



 楊彥齡《左氏春秋年表》二卷



 又《左氏蒙求》二卷



 沉括《春秋機括》二卷



 趙瞻《春秋論》三十卷



 又《春秋經解義例》二十卷



 唐既濟《春秋邦典》二卷



 孫覺《春秋經社要義》六卷



 《春秋經解》十五卷



 《春秋學纂》十二卷



 晁補之《左氏春秋傳雜論》一卷



 劉分文《內傳國語》十卷



 《春秋人譜》一卷孫子平、練明道同撰。



 朱長文《春秋通志》二十卷



 家安國《春秋通義》二十四卷



 張大亨《春秋通訓》
 十六卷



 又《五禮例宗》十卷



 陸佃《春秋後傳》二十卷



 又《補遺》一卷



 程頤《春秋傳》一卷



 黎錞《春秋經解》十二卷



 王裴《春秋義解》十二卷



 張冒德《春秋傳類音》十卷



 韓臺《春秋左氏傳口音》三卷



 陳德寧《公羊新例》十四卷



 又《穀梁新例》六卷



 陰洪道《注春秋敘》一卷



 張翰一作「斡」



 《春秋排門顯義》十卷



 李撰《春秋總要》十卷



 袁希一作「孝」



 政《春秋要類》五卷



 張德昌《春秋傳類》十卷



 沉緯《春秋諫類》二卷



 郭翔《春秋義鑒》三十卷



 王仲孚《春秋類聚》五卷



 黃彬《春秋敘鑒》二
 卷



 《春秋精義》三十卷



 洪勛《春秋圖鑒》五卷



 《春秋加減》一卷



 王叡《春秋守鑒》一卷



 《春秋龜鑒》一卷



 張傑《春秋指玄》十卷



 塗昭良《春秋科義雄覽》十卷



 《春秋應判》三十卷



 丁裔昌《春秋解問》一卷



 邵川《春秋括義》三卷



 劉英《春秋列國圖》一卷



 《春秋十二國年歷》一卷



 謝璧《春秋綴英》二卷



 李塗《春秋事對》五卷蔡延龜注



 《春秋扶懸》三卷



 《春秋比事》三卷



 《春秋要義》十卷



 《春秋策問》三十卷



 《春秋夾氏》三十卷



 李融《春秋樞宗》十卷



 姜虔嗣《春秋三傳纂要》二十卷



 惠
 簡《春秋通略全義》十五卷



 元保宗《春秋事要》十卷



 鞏浚一作「潛」



 《春秋琢瑕》一卷



 張傳靖《左傳編紀》十卷



 崔升《春秋分門屬類賦》三卷楊均注



 裴光輔《春秋機要賦》一卷



 尹玉羽《春秋音義賦》十卷冉遂良注



 又《春秋字源賦》二卷楊文舉注



 李象《續春秋機要賦》一卷



 玉霄《春秋括囊賦集注》一卷



 王鄒彥《春秋蒙求》五卷



 張傑《春秋圖》五卷



 《春秋指掌圖》二卷



 蹇遵品《左傳引帖斷義》十卷



 《春秋纂類義統》十卷本十二卷,第二、第四闕



 《春秋通義》十二卷



 《春秋新義》十卷



 《春秋十二
 國年歷》一卷一名《春秋齊年》



 《春秋文權》五卷



 魯有開《春秋指微》十卷



 《國語音義》一卷



 宋庠《國語補音》三卷



 林概《辨國語》三卷



 崔表《世本圖》一卷



 楊蘊《春秋年表》一卷



 謝湜《春秋義》二十四卷



 又《總義》三卷



 崔子方《春秋經解》十二卷



 《春秋本例例要》二十卷



 呂奎《春秋要旨》十二卷



 吳元緒《左氏鼓吹》一卷



 劉易《春秋經解》二卷



 吳孜《春秋折衷》十二卷



 範柔中《春秋見微》五卷



 鄒氏《春秋總例》一卷



 謝子房《春秋備對》十三卷



 朱振《春秋指要》一卷



 又《春秋正名
 頤隱要旨》十二卷



 《春秋正名頤隱旨要敘論》一卷



 《春秋講義》三卷



 沉滋仁《春秋興亡圖鑒》一卷



 陳禾《春秋傳》十二卷



 又《春秋統論》一卷



 任伯雨《春秋繹聖新傳》十二卷



 鄭昂《春秋臣傳》三十卷



 鄧驥《春秋指蹤》二十一卷



 石公孺《春秋類例》十二卷



 王當《春秋列國諸臣傳》五十一卷



 張根《春秋指南》十卷



 李棠《春秋時論》一卷



 葉夢得《春秋讞》三十卷



 又《春秋考》三十卷



 《春秋傳》二十卷



 《石林春秋》八卷



 《春秋指要總例》二卷



 胡安國《春秋傳》三十卷



 又《通
 例》一卷



 《通旨》一卷



 餘安行《春秋新傳》十二卷



 韓璜《春秋人表》一卷



 範沖《春秋左氏講義》四卷



 黃叔敖《春秋講義》五卷



 洪皓《春秋紀詠》三十卷



 胡銓《春秋集善》十三卷



 鄧名世《春秋四譜》六卷



 《辨論譜說》一卷



 劉本《春秋中論》三十卷



 畢良史《春秋正辭》二十卷



 環中《左氏春秋二十國年表》一卷



 《春秋列國臣子表》十卷



 鄭樵《春秋地名譜》十卷



 又《春秋傳》十二卷



 《春秋考》十二卷



 周彥熠《春秋名義》二卷



 毛邦彥《春秋正義》十二卷



 王日休《春秋孫復解辨
 失》一卷



 又《春秋公羊辨失》一卷



 《春秋左氏辨失》一卷



 《春秋穀梁辨失》一卷



 《春秋名義》一卷



 董自任《春秋總鑒》十二卷



 夏沐《春秋素志》三百一十五卷



 又《春秋麟臺獨講》十一卷



 《延陵先生講義》二卷



 呂本中《春秋解》二卷



 晁以武《春秋故訓傳》三十卷



 王炫《春秋門例通解》十卷



 林慄《經傳集解》三十三卷



 時瀾《左氏春秋講義》十卷



 徐得之《左氏國紀》二十卷



 蕭楚《春秋經辨》十卷



 胡定《春秋解》十二卷



 林拱辰《春秋傳》三十卷



 陳傅良《春秋後傳》十二卷



 又《左氏章指》三十卷



 王汝猷《春秋外傳》十五卷



 程迥《春秋顯微例目》一卷



 又《春秋傳》二十卷



 朱臨《春秋私記》一卷



 《春秋外傳》十卷



 王葆《東宮春秋講義》三卷



 《春秋集傳》十五卷



 呂祖謙《春秋集解》三十卷



 又《左傳類編》六卷



 《左氏博議》二十卷



 《左氏說》一卷



 《左氏博議綱目》一卷祖謙門人張成招標注



 《左氏國語類編》二卷祖謙門人所編



 沉棐《春秋比事》二十卷



 李明復《春秋集義》五十卷



 又《集義綱領》二卷



 任公輔《春秋明辨》十一卷



 楊簡《春秋解》十卷



 戴溪《春秋講
 義》四卷



 程公說《春秋分記》九十卷



 《春秋釋疑》二十卷



 《春秋考異》四卷



 《春秋加減》四卷



 《春秋直指》三卷



 《左氏紀傳》五十卷



 《春秋四傳》二十卷



 《春秋類》六卷



 《春秋例》六卷



 《春秋表記》一卷



 《王侯世系》一卷



 《春秋釋例地名譜》一卷



 《春秋本旨》五卷



 《左氏摘奇》十二卷



 並不知作者。



 李浹《左氏廣誨蒙》一卷



 章沖《左氏類事始末》五卷



 王柏《左氏正傳》一十卷



 高端叔《春秋義宗》一百五十卷



 黎良能《左氏釋疑》、《譜學》各一卷



 沉棐《春秋比事》二十卷



 吳曾《春秋考異》四卷



 又《
 左氏發揮》六卷



 方淑《春秋直音》三卷



 石朝英《左傳約說》一卷



 又《百論》一卷



 黃仲炎《春秋通說》一十三卷



 辛次膺《屬辭比事》五卷



 李孟傳《左氏說》十卷



 程大昌《演繁露》六卷



 李燾《春秋學》十卷



 王應麟《春秋三傳會考》三十六卷



 楊士勛《春秋公穀考異》五卷



 陸宰《春秋後傳補遺》一卷



 趙震揆《春秋類論》四十卷



 宇文虛中《春秋紀詠》三十卷



 王夢應《春秋集義》五十卷



 李心傳《春秋考義》十三卷



 魏了翁《春秋要義》六十卷



 陳藻、林希逸《春秋三傳正附論》
 十三卷



 右《春秋》類二百四十部,二千七百九十九卷。王柏《左氏正傳》以下不著錄二十三部,四百八十八卷



 《古文孝經》一卷凡二十二章



 鄭氏注《孝經》一卷



 唐明皇注《孝經》一卷



 元行沖《孝經疏》三卷



 蘇彬《孝經疏》一卷



 邢昺《孝經正義》三卷



 司馬光《古文孝經指解》一卷



 又《古文孝經指解》一卷



 趙克孝《孝經傳》一卷



 任奉古《孝經講疏》一卷



 張元老《講義》一卷



 範祖禹《古文孝經說》一卷



 呂惠卿《孝
 經傳》一卷



 吉觀國《孝經新義》一部卷亡



 家滋《解義》二卷



 王文獻《詳解》一卷



 林椿齡《全解》一卷



 沈處厚《解》一卷



 趙湘《孝經義》一卷



 張師尹《通義》三卷



 張九成《解》四卷



 朱熹《刊誤》一卷



 黃幹《本旨》一卷



 項安世《孝經說》一卷



 馮椅《古孝經輯注》一卷



 《古文孝經解》一卷



 袁甫《孝經說》三卷



 王行《孝經同異》三卷



 右《孝經》類二十六部,三十五卷。袁甫《孝經說》以下不著錄二部,六卷



 《論語》十卷何晏等集解



 皇侃《論語疏》十卷



 韓愈《筆解》二卷



 陸
 德明《釋文》一卷



 馬總《論語樞要》十卷



 陳銳《論語品類》七卷



 《論語井田圖》一卷



 邢昺《正義》十卷



 周武《集解辨誤》十卷



 宋咸《增注》十卷



 王令《注》十卷



 紀但《論語摘科辨解》十卷



 王安石《通類》一卷



 王雱《解》十卷



 孔武仲《論語說》十卷



 呂惠卿《論語義》十卷



 蔡申《論語纂》十卷



 蘇軾《解》四卷



 蘇轍《論語拾遺》一卷



 程頤《論語說》一卷



 劉正容《重注論語》十卷



 陳禾《論語傳》十卷



 晁說之《講義》五卷



 楊時《解》二卷



 謝良佐《解》十卷



 範祖禹《論語說》二十卷



 游酢《雜解》一
 卷



 龔原《論語解》一部卷亡



 呂大臨《解》十卷



 尹焞《論語解》十卷



 又《說》一卷



 侯仲良《說》一卷



 鄒浩《解》十卷



 汪革《直解》十卷



 葉夢得《釋言》十卷



 黃祖舜《解義》十卷



 張九成《解》十卷



 吳棫《續解》十卷



 又《考異》一卷



 《說例》一卷



 喻樗《玉泉論語學》四卷



 張栻《解》十卷



 湯烈《集程氏說》二卷



 倪思《論語義證》二十卷



 葉隆古《解義》十卷



 洪興祖《論語說》十卷



 史浩《口義》二十卷



 薛季宣《論語小學》二卷



 林慄《論語知新》十卷



 朱熹《論語精義》十卷



 又《集注》十卷



 《集義》十
 卷



 《或問》二十卷



 《論語注義問答通釋》十卷



 鄭汝《解義》十卷



 張演《魯論明微》十卷



 《意原》十卷



 錢文子《論語傳贊》二十卷



 王汝猷《論語歸趣》二十卷



 徐煥《論語贅言》二卷



 曾幾《論語義》二卷



 陳儀之《講義》二卷



 姜得平《本旨》一卷



 《論語指南》一卷黃祖舜、沈大廉、胡宏辨論



 戴溪《石鼓答問》三卷



 《東谷論語》一卷不知作者



 陳耆卿《論語記蒙》六卷



 《孔子家語》十卷魏王肅注



 《論語玄義》十卷



 《論語要義》十卷



 《論語口義》十卷



 《論語展掌疏》十卷



 《論語閱義疏》十卷



 《論語世譜》三卷



 並不知作者。



 王居正《論語
 感發》十卷



 畢良史《論語探古》二十卷



 黃幹《論語通釋》十卷



 又《論語意原》一卷



 卞圜《論語大意》二十卷



 高端叔《論語傳》一卷



 真德秀《論語集編》一十卷



 魏了翁《論語要義》一十卷



 右《論語》類七十三部,五百七十九卷。王居正《論語感發》以下不著錄八部,八十二卷



 《周公謚法》一卷即《汲塚周書·謚法篇》



 班固《白虎通》十卷



 沉約《謚法》十卷



 賀琛《謚法》三卷



 晉陽方《五經鉤沉》五卷



 王彥威《續
 古今謚法》十四卷



 劉迅《六經》五卷



 《春秋謚法》一卷即杜預《春秋釋例·謚法篇》



 陸德明《經典釋文》三十卷



 馬光極《九經釋難》五卷



 章崇業《五經釋題雜問》一卷



 僧十朋《五經指歸》五卷



 蘇鶚《演義》十卷



 劉餗《六說》五卷



 《兼講書》五卷



 《授經圖》三卷



 胡旦《演聖通論》六十卷



 劉敞《七經小傳》五卷



 黃敏求《九經餘義》一百卷



 丘光庭《兼明書》二卷



 李肇《經史釋題》二卷



 顏師古《刊謬正俗》八卷



 李涪《刊誤》二卷



 《九經要略》一卷



 《敘元要略》一卷



 《謚法》三卷



 《六家謚法》二十卷範鎮、周沆
 編



 程頤《河南經說》七卷



 又《五言集解》三卷



 蘇洵《嘉祐謚法》三卷



 《皇祐謚錄》二十卷



 楊會《經解》三十三卷



 劉彞《七經中義》一百七十卷



 蔡攸《政和修定謚法》八十卷



 楊時《三經義辨》十卷



 王居正《辨學》七卷



 鄭樵《謚法》三卷



 李舜臣《諸經講義》七卷



 張九成《鄉黨少儀咸有一德論孟子拾遺》共一卷



 張載《經學理窟》三卷



 項安世《家說》十卷



 《附錄》四卷



 黃幹《六經講義》一卷



 《六經疑難》十四卷不知作者



 許奕《九經直音》九卷



 又《正訛》一卷



 《諸經正典》十卷



 《論語尚
 書周禮講義》十卷



 楊甲《六經圖》六卷



 林觀過《經說》一卷



 戴勛《西齋清選》二卷



 葉仲堪《六經圖》七卷



 俞言《六經圖說》十二卷



 張貴謨《泮林講義》三卷



 周士貴《經括》一卷



 游桂《經學》十二卷



 《九經經旨策義》九卷不知作者



 姜得平《詩書遺意》一卷



 沉貴瑤《四書要義》七篇



 張九成《中庸大學孝經說》各一卷



 又《四書解》六十五卷



 張綱《六經辨疑》五卷



 又《確論》十卷



 李燾《五經傳授》一卷



 王應麟《六經天文編》六卷



 陳應隆《四書輯語》四十卷



 劉元剛《三經演義》一
 十一卷孝經、論、孟



 右經解類五十八部,七百五十三卷。沉貴瑤《四書要義》以下不著錄九部,一百四十六卷、篇



 《爾雅》三卷郭璞注



 孔鮒《小爾雅》一卷



 楊雄《方言》十四卷



 史游《急就章》一卷



 劉熙《釋名》八卷



 許慎《說文解字》十五卷



 孫炎《爾雅疏》十卷



 高璉《爾雅疏》七卷



 徐鍇《說文解字系傳》四十卷



 又《說文解字韻譜》十卷



 《說文解字通釋》四十卷



 僧曇棫《補說文解字》三十卷



 錢承志《說文正隸》三十
 卷



 張揖《廣雅音》三卷



 呂忱《字林》五卷



 曹憲《博雅》十卷



 顧野王《玉篇》三十卷



 韋昭《辨釋名》一卷



 王僧虔《評書》一卷



 梁武帝《評書》一卷



 《千字文》一卷梁周興嗣次韻



 顏之推《證俗音字》四卷



 又《字始》三卷



 虞荔《鼎錄》一卷



 蕭該《漢書音義》三卷



 陸法言《廣韻》五卷



 唐玄宗《開元文字音義》二十五卷



 庾肩吾《書品論》一卷



 陸德明《經典釋文》三十卷



 又《爾雅音義》二卷



 顏元孫《干祿字書》一卷



 李嗣真《書後品》一卷



 《續古今書人優劣》一卷



 王之明《述書後品》一卷



 張懷瓘《
 書詁》一卷



 又《評書藥石論》一卷



 《六體論》一卷



 《古文大篆書祖》一卷



 《書斷》三卷



 顏真卿《筆法》一卷



 又《韻海鑒源》十六卷



 朱禹善《書評》一卷



 又《有唐名書贊》一卷



 林罕《字源偏傍小說》三卷



 《金華苑》二十卷



 張參《五經文字》五卷



 李商隱《蜀爾雅》三卷



 顏師古《急就篇注》一卷



 虞世南《筆髓法》一卷



 唐玄度《九經字樣》一卷



 又《十體書》一卷



 張彥遠《法書要錄》十卷



 杜林岳《集備要字錄》二卷



 王僧虔《圖書會粹》六卷



 呂總《續古今書人優劣》一卷



 蔡希宗《法書論》
 一卷



 劉伯莊《史記音義》二十卷



 裴瑜《爾雅注》五卷



 僧守溫《清濁韻鈐》一卷



 黃伯思《東觀餘論》二卷



 竇儼《義訓》十卷



 崔逢《玉璽譜》一卷嚴士元重修,宋魏損潤色



 郭忠恕《佩觿》三卷



 又《汗簡集》七卷



 《辨字圖》四卷



 《歸字圖》一卷



 《正字賦》一卷



 孫季昭《決疑賦》二卷



 徐玄《三家老子音義》一卷



 鄭文寶《玉璽記》一卷



 《景德韻略》一卷戚倫等詳定



 宋高宗《評書》一卷亦名《翰墨志》



 邢昺《爾雅疏》十卷



 歐陽融《經典分毫正字》一卷



 沉立《稽正辨訛》一卷



 唐耜《字說集解》三十冊卷亡



 錢惟演《飛
 白書敘錄》一卷



 周越《古今法書苑》十卷



 祝充《韓文音義》五十卷



 李舟《切韻》五卷



 丘世隆《切韻搜隱》五卷



 劉熙古《切韻拾玉》五卷



 胡元質《西漢字類》五卷



 陳天麟《前漢通用古字韻編》五卷



 陳彭年等《重修廣韻》五卷



 《韻詮》十四卷



 僧師悅《韻關》一卷



 丘雍《校定韻略》五卷



 《韻選》五卷



 《韻源》一卷



 孫愐《唐韻》五卷



 《天寶元年集切韻》五卷



 釋猷智《辨體補修加字切韻》五卷



 丁度《集韻》十卷



 又《景祐禮部韻略》五卷



 《墨藪》一卷不知作者



 賈昌朝《群經音辨》三卷



 夏竦《
 重校古文四聲韻》五卷



 又《聲韻圖》一卷



 司馬光《切韻指掌圖》一卷



 又《類編》四十四卷



 劉溫潤《羌爾雅》一卷



 宋祁《摘粹》一卷



 歐陽修《集古錄跋尾》六卷



 句中正《雍熙廣韻》一百卷《序例》一卷



 又《三體孝經》一卷



 楊南仲《石經》七十五卷



 又《三體孝經》一卷



 燕誨《字傍辨誤》一卷



 道士謝利貞《玉篇解疑》三十卷



 《象文玉篇》二十卷



 石懷德《隸書賦》一卷



 褚長文《範指論》一卷



 李訓《節金錄》一卷



 《翰林隱術》一卷



 荊浩《筆法》一卷



 韋氏《筆寶兩字》五卷



 徐浩《
 書譜》一卷



 又《古跡記》一卷



 宋敏求《寶刻叢章》三十卷



 劉敞《先秦古器圖》一卷



 李行中《引經字源》二卷



 朱長文《續書斷》二卷



 王安石《字說》二十四卷



 米芾《書評》一卷



 又《寶章待訪集》一卷



 呂大臨《考古圖》十卷



 李公麟《古器圖》一卷



 陸佃《爾雅新義》二十卷



 《埤雅》二十卷



 蔡京《崇寧鼎書》一卷



 張有《復古編》二卷



 《政和甲午祭禮器款識》一卷



 王楚《鐘鼎篆韻》二卷



 吳棫《韻補》五卷



 董衡《唐書釋音》二十卷



 竇莘《唐書音訓》四卷



 《宣和重修博古圖錄》三十卷



 趙
 明誠《金石錄》三十卷



 又別本三十卷



 薛尚功《重廣鐘鼎篆韻》七卷



 《歷代鐘鼎彞器款識法帖》二十卷



 張孟《押韻》十卷



 許冠《韻海》五十卷



 吳《童訓統類》一卷



 鄭樵《石鼓文考》一卷



 又《字始連環》二卷



 《象類書》十一卷



 《論梵書》三卷



 《爾雅注》三卷



 《書考》六卷



 《通志六書略》五卷



 郟升卿《四聲類韻》二卷



 又《聲韻類例》一卷



 《淳熙監本禮部韻略》五卷



 劉球《隸韻略》七卷



 潘緯《柳文音義》三卷



 僧應之《臨書關要》一卷



 呂本中《童蒙訓》三卷



 周燔《六經音義》十三卷



 李盛《六經釋文》二卷



 黃瑰《班書韻編》五卷



 張《石經注文考異》四十卷



 洪適《隸釋》二十七卷



 《隸續》二十一卷



 史浩《童丱須知》三卷



 朱熹《小學之書》四卷



 又《四子》四卷



 程端蒙《小學字訓》一卷



 呂祖謙《少儀外傳》二卷



 陳淳《北溪字義》二卷



 婁機《班馬字類》二卷



 《漢隸字源》六卷



 《廣干祿字書》五卷



 《古鼎法帖》五卷



 楊師復《漢隸釋文》二卷



 馬居易《漢隸分韻》七卷



 翟伯壽《籀史》二卷



 胡寅《注敘古千文》一卷



 呂氏《敘古千文》一卷



 《慶元嘉定古器圖》六卷



 僧妙
 華《互注集韻》二十五卷



 羅點《清勤堂法帖》六卷



 李從周《字通》一卷



 遼僧行均《龍龕手鑒》四卷



 黃伯思《法帖刊誤》一卷



 釋元沖《五音韻鏡》一卷



 施宿《大觀法帖總釋》二卷



 又《石鼓音》一卷



 蔡氏《口訣》一卷名亡



 《書錄》一卷



 《書隱法》一卷



 《筆陣圖》一卷



 《西漢字類》一卷



 《纂注禮部韻略》五卷



 《翰林禁經》三卷



 《臨汝帖》三卷



 《筆苑文詞》一卷



 《法帖字證》十卷



 《正俗字》十卷



 《書斷例傳》五卷



 《洪韻海源》二卷



 《互注爾雅貫類》一卷



 《諸家小學總錄》二卷



 《集古系時》十卷



 《蕃漢
 語》一卷



 並不知作者



 劉紹祐《字學摭要》二卷



 洪邁《次李翰蒙求》三卷



 集齋彭氏《小學進業廣記》一部



 王應麟《蒙訓》四十四卷



 又《小學紺珠》十卷



 《小學諷詠》四卷



 《補注急就篇》六卷



 右小學類二百六部,一千五百七十二卷。劉紹祐《字學摭要》以下不著錄六部,六十九卷



 凡經類一千三百四部,一萬
 三千六百
 八卷



\end{pinyinscope}