\article{列傳第一百四}

\begin{pinyinscope}

 劉安世鄒浩田晝王
 回曾誕附陳瓘任伯雨



 劉安世,字器之,魏人。父航,第進士,歷知虞城、犀浦縣。虞城多奸猾,喜寇盜;犀浦民弱而馴。航為政,寬猛急緩不同,兩縣皆治。知宿州。押伴夏使,使者多所要請,執禮不
 遜,且欲服球文金帶入見,航皆折正之。以群牧判官為河南監牧使。持節冊夏主秉常,凡例所遺寶帶、名馬,卻弗受。還,上《御戎書》,大略云:「辨士好為可喜之說,武夫徼冀不貲之寵,或為所誤,不可不戒。」為河北西路轉運使。熙寧大旱求言,航論新政不便者五,又上書言:「人主不可輕失天下心,宜乘時有所改為,則人心悅而天意得矣。」不報。乃請提舉崇福宮,起知涇、相二州。王師西征,徙知陜府。時倉卒軍興,饋餉切急,縣令佐至荷校督民,民
 多棄田廬,或至自盡。航獨期會如平日,事更以辦。終太僕卿。



 安世少時持論已有識。航使監牧時,文彥博在樞府,有所聞,每呼安世告之。安世從容言:「王介甫求去,外議謂公且代其任。」彥博曰:「安石壞天下至此,後之人何可為?」安世拱手曰:「安世雖晚進,竊以為未然。今日新政,果順人所欲而為人利乎?若不然,公當去所害,興所利,反掌間耳。彥博默不應,他日見航,嘆獎其堅正。



 登進士第,不就選。從學於司馬光,咨盡心行己之要,光教之以
 誠,且令自不妄語始。調洺州司法參軍,司戶以貪聞,轉運使吳守禮將按之,問於安世,安世云:「無之。」守禮為止。然安世心常不自安,曰:「司戶實貪而吾不以誠對,吾其違司馬公教乎!」後讀揚雄《法言》「君子避礙則通諸理」,意乃釋。



 光入相,薦為秘書省正字。光薨,宣仁太后問可為臺諫於呂公著,公著以安世對。擢右正言。時執政頗與親戚官,安世言:「祖宗以來,大臣子弟不敢受內外華要之職。自王安石秉政,務快私意,累聖之制,掃地不存。今
 廟堂之上,猶習故態。」因歷疏文彥博以下七人,皆耆德魁舊,不少假借。



 章惇以強市昆山民田罰金,安世言:「惇與蔡確、黃履、邢恕素相交結,自謂社稷之臣,貪天之功,僥幸異日,天下之人指為『四兇』。今惇父尚在,而別籍異財,絕滅義理,止從薄罰,何以示懲?」會吳處厚解釋確《安州詩》以進,安世謂其指斥乘輿,犯大不敬,與梁燾等極論之,竄之新州。宰相范純仁至於御史十人,皆緣是去。



 遷起居舍人兼左司諫,進左諫議大夫。有旨暫罷講筵,
 民間歡傳宮中求乳婢,安世上疏諫曰:「陛下富於春秋,未納後而親女色。願太皇太后保祐聖躬,為宗廟社稷大計,清閑之燕,頻御經帷,仍引近臣與論前古治亂之要,以益聖學,無溺於所愛而忘其可戒。」哲宗俯首不語。後曰:「無此事,卿誤聽爾。」明日,後留呂大防告之故。大防退,召給事中範祖禹使達旨。祖禹固嘗以諫,於是兩人合辭申言之甚切。



 鄧溫伯為翰林承旨,安世言其「出入王、呂黨中,始終反復。今之進用,實系君子小人消長之
 機。乞行免黜。」不報。遂請外,改中書舍人,辭不就。以集賢殿修撰提舉崇福宮,才六月,召為寶文閣待制、樞密都承旨。



 范純仁復相,呂大防白後欲令安世少避。後曰:「今既不居言職,自無所嫌。」又語韓忠彥曰:「如此正人,宜且留朝廷。」乃止。呂惠卿復光祿卿,分司,安世爭以為不可,不聽。出知成德軍。章惇用事,尤忌惡之。初黜知南安軍,再貶少府少監,三貶新州別駕,安置英州。



 同文館獄起,蔡京乞誅滅安世等家,讒雖不行,猶徙梅州。惇與蔡卞
 將必置之死,因使者入海島誅陳衍,諷使者過安世,脅使自裁。又擢一土豪為轉運判官,使殺之。判官疾馳將至梅,梅守遣客來勸安世自為計。安世色不動,對客飲酒談笑,徐書數紙付其僕曰:「我即死,依此行之。」顧客曰:「死不難矣。」客密從僕所視,皆經紀同貶當死者之家事甚悉。判官未至二十里,嘔血而斃,危得免。



 昭懷後正位中宮,惇、卞發前諫乳婢事,以為為後設。時鄒浩既貶,詔應天少尹鼛孫以檻車收二人赴京師。行數驛而徽宗
 即位赦至,鼛乃還。凡投荒七年,甲令所載遠惡地無不歷之。移衡及鼎,然後以集賢殿修撰知鄆州、真定府,曾布又忌之,不使入朝。蔡京既相,連七謫至峽州羈管。稍復承議郎,卜居宋都。宣和六年,復待制,中書舍人沈思封還之。明年卒,年七十八。



 安世儀狀魁碩,音吐如鐘。初除諫官,未拜命,入白母曰:「朝廷不以安世不肖,使在言路。倘居其官,須明目張膽,以身任責,脫有觸忤,禍譴立至。主上方以孝治天下,若以老母辭,當可免。」母曰:「不然,
 吾聞諫官為天子諍臣,汝父平生欲為之而弗得,汝幸居此地,當捐身以報國恩。正得罪流放,無問遠近,吾當從汝所之。」於是受命。在職累歲,正色立朝,扶持公道。其面折廷爭,或帝盛怒,則執簡卻立,伺怒稍解,復前抗辭。旁侍者遠觀,蓄縮悚汗,目之曰「殿上虎」,一時無不敬懾。



 家居未嘗有惰容,久坐身不傾倚,作字不草書,不好聲色貨利。其忠孝正直,皆則象司馬光。年既老,群賢凋喪略盡,巋然獨存,而名望益重。梁師成用事,能生死人,心
 服其賢,求得小吏吳默嘗趨走前後者,使持書來,啖以即大用,默因勸為子孫計,安世笑謝曰:「吾若為子孫計,不至是矣。吾欲為元祐全人,見司馬光於地下。」還其書不答。死葬祥符縣。後二年,金人發其塚,貌如生,相驚語曰:「異人也!」為之蓋棺乃去。



 鄒浩,字志完,常州晉陵人。第進士,調揚州、穎昌府教授。呂公著、范純仁為守,皆禮遇之。純仁屬撰樂語,浩辭。純仁曰:「翰林學士亦為之。」浩曰:「翰林學士則可,祭酒、司業
 則不可。」純仁敬謝。



 元祐中,上疏論事,其略曰:「人材不振,無以成天下之務。陛下視今日人材,果有餘邪,果不足邪?以為不足,則中外之百執事未嘗不備。以為有餘,則自任以天下之重者幾人?正色昌言不承望風旨者幾人?持刺舉之權以肅清所部者幾人?承流宣化而使民安田里者幾人?民貧所當富也,則曰水旱如之何;官冗所當澄也,則曰民情不可擾;人物所當求也,則曰從古不乏材;風俗所當厚也,則曰不切於時變,是皆不明義
 理之過也。」



 蘇頌用為太常博士,來之邵論罷之。後累歲,哲宗親擢為右正言。有請以王安石《三經義》發題試舉人者,浩論其不可而止。陜西奏邊功,中外皆賀,浩言:「先帝之志而陛下成之,善矣。然兵家之事,未戰則以決勝為難,既勝則以持勝為難,惟其時而已。茍為不然,將棄前功而招後患。願申敕將帥,毋狃屢勝,圖惟厥終。」



 京東大水,浩言:「頻年水異繼作,雖盈虛之數所不可逃,而消復之方尤宜致謹。《書》曰:『惟先格王正厥事。』不以為數之
 當然,此消復之實也。」



 蹇序辰看詳元祐章奏,公肆詆欺,輕重不平。浩言:「初旨但分兩等,謂語及先帝並語言過差而已;而今所施行,混然莫辨。以其近似難分之跡,而典刑輕重隨以上下,是乃陛下之威福操柄下移於近臣。願加省察,以為來事之監。」



 章惇獨相用事,威虐震赫,浩所言每觸惇忌,仍上章露劾,數其不忠侵上之罪,未報。而賢妃劉氏立,浩言:



 立後以配天子,安得不審。今為天下擇母,而所立乃賢妃,一時公議,莫不疑惑,誠以國
 家自有仁祖故事,不可不遵用之爾。蓋郭后與尚美人爭寵,仁祖既廢後,並斥美人,所以示公也。及立後,則不選於妃嬪而卜於貴族,所以遠嫌,所以為天下萬世法也。陛下之廢孟氏,與郭後無以異。果與賢妃爭寵而致罪乎,抑其不然也?二者必居一於此矣。孟氏罪廢之初,天下孰不疑立賢妃為後。及讀詔書,有「別選賢族」之語;又聞陛下臨朝慨嘆,以為國家不幸;至於宗景立妾,怒而罪之,於是天下始釋然不疑。今竟立之,豈不上累聖
 德?



 臣觀白麻所言,不過稱其有子,及引永平、祥符事以為證。臣請論其所以然,若曰有子可以為後,則永平貴人未嘗有子也,所以立者,以德冠後宮故也。祥符德妃亦未嘗有子,所以立者,以鐘英甲族故也。又況貴人實馬援之女,德妃無廢後之嫌,迥與今日事體不同。頃年冬,妃從享景靈宮,是日雷變甚異。今宣制之後,霖雨飛雹,自奏告天地宗廟以來,陰淫不止。上天之意,豈不昭然!考之人事既如彼,求之天意又如此,望不以一時改
 命為難,而以萬世公議為可畏,追停冊禮,如初詔行之。



 帝謂:「此亦祖宗故事,豈獨朕邪?」對曰:「祖宗大德可法者多矣,陛下不之取,而效其小疵,臣恐後世之責人無已者紛紛也。」帝變色,猶不怒,持其章躊躇四顧,凝然若有所思,付外。明日,章惇詆其狂妄,乃削官,羈管新州。蔡卞、安惇、左膚繼請治其祖送者王回等,語在他傳。



 徽宗立,亟召還,復為右正言,遷左司諫。上疏謂:「孟子曰:『左右諸大夫皆曰賢,未可也;國人皆曰賢,然後察之,見賢焉,然
 後用之。左右諸大夫皆曰不可,勿聽;國人皆曰不可,然後察之,見不可焉,然後去之。』於是知公議不可不恤,獨斷不可不謹。蓋左右非不親也,然不能無交結之私;諸大夫非不貴也,然不能無恩仇之異。至於國人皆曰賢,皆曰不可,則所謂公議也。公議之所在,概已察之,必待見賢然後用,見不可然後去,則所謂獨斷也。惟恤公議於獨斷未形之前,謹獨斷於公議已聞之後,則人君所以致治者,又安有不善乎?伏見朝廷之事,頗異於即位
 之初,相去半年,遽已如是,自今以往,將如之何?願陛下深思之。」



 改起居舍人,進中書舍人。又言:「陛下善繼神宗之志,善述神宗之事,孝德至矣。尚有五朝聖政盛德,願稽考而繼述之,以揚七廟之光,貽福萬世。」遷兵、吏二部侍郎,以寶文閣待制知江寧府,徙杭、越州。



 初,浩還朝,帝首及諫立後事,獎嘆再三,詢諫草安在。對曰:「焚之矣。」退告陳瓘,瓘曰:「禍其在此乎。異時奸人妄出一緘,則不可辨矣。」蔡京用事,素忌浩,乃使其黨為偽疏,言劉後殺卓
 氏而奪其子。遂再責衡州別駕,語在《獻愍太子傳》。尋竄昭州,五年始得歸。



 初,浩除諫官,恐貽親憂,欲固辭。母張氏曰:「兒能報國,無愧於公論,吾顧何憂?」及浩兩謫嶺表,母不易初意。稍復直龍圖閣。瘴疾作,危甚。楊時過常,往省之。TC然僅存餘息,猶眷眷以國事為問,語不及私。卒,年五十二。高宗即位,詔曰:「浩在元符間,任諫爭,危言讜論,朝野推仰。」復其待制,又贈寶文閣直學士,賜謚忠。



 誥所與游田晝、王回、曾誕,皆良士也。



 晝字承君,陽翟人。樞密使況之從子,以任為校書郎。調磁州錄事參軍,知西河縣,有善政,民甚德之。議論慨慷,有前輩風。



 與鄒浩以氣節相激勵。元符中,浩為諫官,晝監京城門,往見浩曰:「平生與君相許者何如,今君為何官?」浩曰:「上遇群臣,未嘗假以辭色,獨於浩差若相喜。天下事固不勝言,意欲待深相信而後發,貴有益也。」晝然之。既而以病歸許,邸狀報立後,晝謂人曰:「志完不言,可以絕交矣。」浩得罪,晝迎諸塗。浩出涕,晝正色責曰:「使
 志完隱默官京師,遇寒疾不汗,五日死矣。豈獨嶺海之外能死人哉?願君毋以此舉自滿,士所當為者,未止此也。」浩茫然自失,嘆謝曰:「君之贈我厚矣。」



 建中靖國初,入為大宗正丞。曾布數羅致之,不為屈;欲與提舉常平官,亦辭。請知淮陽軍,歲大疫,日挾醫問病者藥之,遇疾卒。淮陽人祀以為土神云。



 回字景深,仙游人。第進士,調松滋令。荊、沔俗用人祭鬼,回捕治甚嚴,其風遂革。知鹿邑縣,入為宗正寺簿。元符
 中,葉祖洽薦為睦親宅講書。與鄒浩友善,皇後劉氏立,浩將論之,密告回,回曰:「事寧有大於此者乎?子雖有親,然移孝為忠,亦太夫人素志也。」



 浩南遷,人莫敢顧。回斂交游錢與治裝,往來經理,且慰安其母。邏者以聞,逮詣詔獄,眾為之懼,回居之晏然。御史詰之,對曰:「實嘗預議,不敢欺也。」因誦浩所上章,幾二千言。獄上,除名停廢。即徒步出都門,行數十里,其子追及,問以家事,不答。祖洽亦坐黜。



 徽宗立,召還舊官,擢監察御史。數日卒,年五十
 三。岑象求、王覿、賈易上章,乞錄其子,恤其家,以獎勸忠義。詔除子渙老郊社齋郎,蔡京為相,奪之,仍列名黨籍。



 誕,公亮從孫也。孟後之廢,誕三與浩書,勸力請復後,浩不報。及浩以言南遷,誕著《玉山主人對客問》以譏之,其略曰:「客問:鄒浩可以為有道之士乎?主人曰:浩安得為知道。雖然,予於此時議浩,是天下無全人也。言之尚足為來世戒。《易》曰:『知幾其神乎?』又曰:『知進退存亡而不失其正者,其惟聖人乎?』方孟後之廢,人莫不知劉氏之將
 立,至四年之後而冊命未行,是天子知清議之足畏也。使當其時,浩力言復後,能感悟天子,則無今日劉氏之事,貽朝廷於過舉,再三言而不聽,則義亦當矣。使是時得罪,必不若是酷以貽老母之憂矣。嗚呼!若浩者,雖不得為知幾之士,然百世之下,頑夫廉,懦夫有立志,尚不失為聖人之清也。」其書既出,識者或以比韓愈《諫臣論》。誕仕亦不顯。



 陳瓘,字瑩中,南劍州沙縣人。少好讀書,不喜為進取學。
 父母勉以門戶事,乃應舉,一出中甲科。調湖州掌書記,簽書越州判官。守蔡卞察其賢,每事加禮,而瓘測知其心術,常欲遠之,屢引疾求歸,章不得上。檄攝通判明州。卞素敬道人張懷素,謂非世間人,時且來越,卞留瓘小須之,瓘不肯止,曰:「子不語怪力亂神,斯近怪矣。州牧既信重,民將從風而靡。不識之,未為不幸也。」後二十年而懷素誅。明州職田之入厚,瓘不取,盡棄於官以歸。



 章惇入相,瓘從眾道謁。惇聞其名,獨邀與同載,詢當世之務,瓘
 曰:「請以所乘舟為喻:偏重可行乎?移左置右,其偏一也。明此,則可行矣。天子待公為政,敢問將何先?」惇曰:「司馬光奸邪,所當先辨,勢無急於此。」瓘曰:「公誤矣。此猶欲平舟勢而移左以置右,果然,將失天下之望。」惇厲色曰:「光不務纘述先烈,而大改成緒,誤國如此,非奸邪而何?」瓘曰:「不察其心而疑其跡,則不為無罪;若指為奸邪,又復改作,則誤國益甚矣。為今之計,唯消朋黨,持中道,庶可以救弊。」意雖忤惇,然亦驚異,頗有兼收之語。至都,用
 為太學博士。會卞與惇合志,正論遂絀。卞黨薛昂、林自官學省,議毀《資治通鑒》,瓘因策士題引神宗所制序文以問,昂、自意沮。



 遷秘書省校書郎。紹述之說盛,瓘奏哲宗言:「堯、舜、禹皆以『若稽古』為訓。『若』者,順而行之;『稽』者,考其當否,必使合於民情,所以成帝王之治。天子之孝,與士大夫之孝不同。」帝反復究問,意感悅,約瓘再入見。執政聞而憾之,出通判滄州,知衛州。徽宗即位,召為右正言,遷左司諫。瓘論議持平,務存大體,不以細故借口,未
 嘗及人晻昧之過。嘗云:「人主托言者以耳目,誠不當以淺近見聞,惑其聰明。」惟極論蔡卞、章惇、安惇邢恕之罪。



 御史龔□擊蔡京,朝廷將逐□,瓘言:「紹聖以來,七年五逐言者,常安民、孫諤、董敦逸、陳次升、鄒浩五人者,皆與京異議而去。今又罷□,將若公道何。」遂草疏論京,未及上,時皇太后已歸政,瓘言外戚向宗良兄弟與侍從希寵之士交通,使物議籍籍,謂皇太后今猶預政。由是罷監揚州糧料院。瓘出都門,繳四章奏之,並明宣仁誣謗
 事。帝密遣使賜以黃金百兩,後亦命勿遽去,畀十僧牒為行裝,改知無為軍。



 明年,還為著作郎,遷右司員外郎兼權給事中。宰相曾布使客告以將即真,瓘語子正匯曰:「吾與丞相議事多不合,今若此,是欲以官爵相餌也。若受其薦進,復有異同,則公議私恩,兩有愧矣。吾有一書論其過,將投之以決去就,汝其書之。但郊祀不遠,彼不兼容,則澤不及汝矣,能不介於心乎?」正匯願得書。旦持入省,布使數人邀相見,甫就席,遽出書,布大怒。爭辯
 移時,至箕踞誶語,瓘色不為動,徐起白曰:「適所論者國事,是非有公議,公未可失待士禮。」布矍然改容。信宿,出知泰州。崇寧中,除名竄袁州、廉州,移郴州,稍復宣德郎。



 正匯在杭,告蔡京有動搖東宮跡。杭守薿執送京師,先飛書告京俾為計。事下開封府制獄,並逮瓘。尹李孝稱逼使證其妄,瓘曰:「正匯聞京將不利社稷,傳於道路,瓘豈得預知?以所不知,忘父子之恩而指其為妄,則情有所不忍;挾私情以符合其說,又義所不為。京之奸邪,
 必為國禍。瓘固嘗論之於諫省,亦不待今日語言間也。」內侍黃經臣蒞鞫,聞其辭,失聲嘆息,謂曰:「主上正欲得實,但如言以對可也。」獄具,正匯猶以所告失實流海上,瓘亦安置通州。



 瓘嘗著《尊堯集》,謂紹聖史官專據王安石《日錄》改修《神宗史》,變亂是非,不可傳信;深明誣妄,以正君臣之義。張商英為相,取其書,既上,而商英罷,瓘又徙臺州。宰相遍令所過州出兵甲護送;至臺,每十日一徙告;且命兇人石悈知州事,執至庭,大陳獄具,將脅以
 死。瓘揣知其意,大呼曰:「今日之事,豈被制旨邪!」悈失措,始告之曰:「朝廷令取《尊堯集》爾。」瓘曰:「然則何用許。使君知『尊堯』所以立名乎?蓋以神考為堯,主上為舜,助舜尊堯,何得為罪?時相學術淺短,為人所愚。君所得幾何,乃亦不畏公議,干犯名分乎?」悈慚,揖使退。所以窘辱之百端,終不能害。宰相猶以悈為怯而罷之。



 在臺五年,乃得自便。才復承事郎,帝批進目,以為所擬未當,令再敘一官,仍與差遣,執政持不行。卜居江州,復有譖之者,至不
 許輒出城。旋令居南康,才至,又移楚。瓘平生論京、卞,皆披擿其處心,發露其情慝,最所忌恨,故得禍最酷,不使一日少安。宣和六年卒,年六十五。



 瓘謙和不與物競,閑居矜莊自持,語不茍發。通於《易》,數言國家大事,後多驗。靖康初,詔贈諫議大夫,召官正匯。紹興二十六年,高宗謂輔臣曰:「陳瓘昔為諫官,甚有讜議。近覽所著《尊堯集》,明君臣之大分,合於《易》天尊地卑及《春秋》尊王之法。王安石號通經術,而其言乃謂『道隆德駿者,天子當北面
 而問焉』,其背經悖理甚矣。瓘宜特賜謚以表之。」謚曰忠肅。



 任伯雨,字德翁,眉州眉山人。父孜,字遵聖,以學問氣節推重鄉里,名與蘇洵埒,仕至光祿寺丞。其弟伋,字師中,亦知名,嘗通判黃州,後知滬州。當時稱「大任」、「小任」。



 伯雨自幼,已矯然不群,邃經術,文力雄健。中進士第,調施州清江主簿。郡守檄使蒞公庫,笑曰:「里名勝母,曾子不入,此職何為至我哉?」拒不受。知雍丘縣,御吏如束濕,撫民
 如傷。縣枕汴流,漕運不絕,舊苦多盜,然未嘗有獲者,人莫知其故。伯雨下令網舟無得宿境內,始猶不從,則命東下者斧斷其纜,趣京師者護以出,自是外戶不閉。



 使者上其狀,召為大宗正丞,甫至,擢左正言。時徽宗初政,納用讜論,伯雨首擊章惇,曰:「惇久竊朝柄,迷國罔上,毒流搢紳,乘先帝變故倉卒,輒逞異意,睥睨萬乘,不復有臣子之恭。向使其計得行,將置陛下與皇太后於何地!若貸而不誅,則天下大義不明,大法不立矣。臣聞北使
 言,去年遼主方食,聞中國黜惇,放箸而起,稱甚善者再,謂南朝錯用此人。北使又問,何為只若是行遣?以此觀之,不獨孟子所謂『國人皆曰可殺』,雖蠻貊之邦,莫不以為可殺也。」章八上,貶惇雷州。繼論蔡卞六大罪,語在《卞傳》。



 建中靖國改元,當國者欲和調元祐、紹聖之人,故以「中」為名。伯雨言:「人才固不當分黨與,然自古未有君子小人雜然並進可以致治者。蓋君子易退,小人難退,二者並用,終於君子盡去,小人獨留。唐德宗坐此致播遷
 之禍,建中乃其紀號,不可以不戒。」



 時議者欲西北典郡專用武臣,伯雨謂:「李林甫致祿山之亂者,此也。」又論鐘傅、王贍生湟、鄯邊事,失與國心,宜棄其地,以安邊息民;張耒、黃庭堅、晁補之、歐陽棐、劉唐老等宜在朝廷。上書皇太后,乞暴蔡京之惡,召還陳瓘,以全定策之勛。



 時以正月朔旦有赤氣之異,詣火星觀以禳之,伯雨上疏言:「嘗聞修德以弭災,未有禳祈以消變。《洪範》以五事配五行,說者謂視之不明,則有赤眚、赤祥。乞攬權綱以信賞
 罰,專威福以殊功罪,使皇明赫赫,事至必斷,則乖氣異象,轉為休祥矣。」又言:「比日內降浸多,或恐矯傳制命。漢之鴻都賣爵,唐之墨敕斜封,此近監也。」



 王覿除御史中丞,仍兼史官,伯雨謂:「史院宰相監修,今中丞為屬,非所以重風憲,遠嫌疑。」已而覿除翰林,伯雨復論曰:「學士爵秩位序,皆在中丞上。今覿為之,是諫官論事,非特朝廷不行,適足以為人遷官爾。」



 伯雨居諫省半歲,所上一百八疏,大臣畏其多言,俾權給事中,密諭以少默即為真。
 伯雨不聽,抗論愈力,且將劾曾布。布覺之,徙為度支員外郎,尋知虢州。崇寧黨事作,削籍編管通州。為蔡卞所陷,與陳瓘、龔□、張庭堅等十三人皆南遷,獨伯雨徙昌化。奸人猶未甘心,用匿名書復逮其仲子申先赴獄,妻適死於淮,報訃俱至。伯雨處之如平常,曰:「死者已矣,生者有負於朝廷,亦當從此訣。如其不然,天豈殺無辜耶!」申先在獄,鍛煉無所傅致,乃得釋,居海上三年而歸。宣和初,卒,年七十三。



 長子象先,登世科,又中詞學兼茂舉,
 有司啟封,見為黨人子,不奏名,調秦州戶曹掾。聞父謫,棄官歸養。王安中闢燕山宣撫幕,勉應之,道引疾還,終身不復仕。申先以布衣特起至中書舍人。



 紹興初,高宗詔贈伯雨直龍圖閣,又加諫議大夫,採其諫章,追貶章惇、蔡卞、邢恕、黃履,明著誣宣仁事以告天下。淳熙中,賜謚忠敏。



 論曰:劉安世覆文彥博之言,時年尚少,然其言即元祐之初政,而司馬光之用心也。鄒浩諫立劉後,反復曲折,
 極人所難言。二人除言官,俱入白其母,母俱勉以盡忠報國,無分毫顧慮後患意。鳴呼,賢哉!陳瓘、任伯雨抗跡疏遠,立朝寡援,而力發章惇、曾布、蔡京、蔡卞群奸之罪,無少畏忌,古所謂剛正不撓者歟!



\end{pinyinscope}