\article{志第十七 五行二下}

\begin{pinyinscope}

 火下



 乾德二年十月,眉州獻《禾生九穗圖》。四年四月,府州、尉氏縣、雲陽縣並有麥兩歧。五月,魚臺縣麥秀三歧。六月,南充縣民何約田禾一莖十三穗,一莖十一穗;七月,又生一莖九穗。



 開寶二年
 五
 月,梓、
 蜀二州;六年四月,東明縣;八年五月,鄭州、梓州、合州巴川縣並獻瑞麥。



 太平興國元年九月,隰州獻合穗禾,長尺餘。十月,渝州獻九穗禾。三年四月,夏縣,五月舒州,六月閬州,麥並秀兩歧。四年七月,洺州獻嘉禾。邛、資二州禾並九穗。八月,涇州民田並有嘉禾。九月,知溫州何士宗獻《嘉禾九穗圖》。五年七月,蓬萊縣民王明田谷隔□合穗,相去一尺許。八月,知慈州張愈獻合穗禾。九月,流溪縣,六年五月汝陰縣,
 九年五月施州,麥並秀兩歧。



 端拱元年五月,陳州;淳化元年四月魏城縣;七月閬州;二年四月,蔡州;五月,陳州、陵州仁壽縣;四年五月,達州;五年四月,永城縣,並獻瑞麥。



 至道元年六月,嘉禾生眉山縣蕭德純田,一本二十四穗。七月,金水縣胥羅翊田禾生九穗。舒州監軍吳光謙廨粟畦兩本,歧分十穗。臨渙縣民侯正家二禾合成一穗。八月,綿竹縣禾生九穗。夏州團練使趙光嗣獻嘉禾一函。十月,濠州獻《瑞穀圖》。二年五月,泗州獻瑞麥。三
 年二月,洋州嘉禾合穗,知州施翊以聞。四月,唐州、遂州、盤石縣並獻瑞麥。五月,黃州、建昌軍麥秀二三穟。八月,雅州禾一莖十四穗。雄州嘉禾生。九月,知代州李允正獻嘉禾穗一匣。



 咸平元年五月,曲水縣麥秀二三穗。七月,嘉禾生後苑,一莖二十四穗。百丈縣民李文寶禾生一莖十七穗。八月,蘇州廨後園、邠州民田並禾生合穗。平夷縣民王義田禾兩穗合為一。化城縣民張美田禾九穗。二年五月,華州麥秀二三穗。七月,資官縣吏董昭
 美禾一莖九穗者各一。棣、洺二州嘉禾合穗。彭城縣民張福先田粟一莖分四穗。八月,郪縣趙範粟一莖九穗。玄武縣民李知進田粟一莖,上分五苗,成二十一穗。榆次縣民周貴田禾三莖共穗。三年五月,酇縣、海陵縣並麥秀二三穗。七月,真定府禾三莖一穗。達州民李國清田禾一苗九穗。八月,辰州公田禾生一莖三穗者四。隰州嘉禾合穗,圖以獻。四年八月,舒州嘉禾生。九月,知河中府郭堯卿獻《嘉禾合穗圖》。五年八月,臨汾縣民吉
 遇、洪洞縣民範思安田並禾生隔二隴上合為一。六年七月,涉縣民連罕田隔四□同穎。銅梁縣民楊彥魯禾一莖九穗。



 景德元年正月,寧晉縣民耿待問田禾合穗者三本,知州王用和圖以獻。二年七月,獲鹿縣禾合穗。八月,滎陽縣及相州嘉禾異畝同穎。九月,並州貢《嘉禾圖》。三年八月,大名府、滄州並嘉禾生。真定府禾異畝同穎。九月,榮州禾一莖十八穗。四年六月,南雄州保昌民田禾一本九穗,以圖來獻。七月,神泉縣民張篆田禾一
 苗九穗。貝、兗二州嘉禾合穗。九月,衛、德二州、廣安軍並上《嘉禾圖。》大中祥符元年,曲水縣、南鄭縣並麥秀二三穗。七月,乾封縣奉高鄉民田禾異□同穎,判州王欽若以聞。八月,鄆州獻嘉禾。淳化縣民賀行滿田禾隔四□,相去四尺許,合為一穗。新平縣民尹遇田禾合穗者二本。真定府粟生二穗。九月,澧州嘉禾一莖十穗。虢州團練使綦興獻合穗禾。嘉州民潘德麟田禾二莖各九穗。麟州嘉禾生。二年六月,簡州民集若寧家禾九穗。七月,黔
 州嘉禾異畝合穗。八月,嘉州廨有一莖十四穗生庭中,岐山縣禾異畝同穎,知州施護以聞。三年四月,同州麥秀二三穗。七月,冀、淄、昭三州嘉禾多穗,異畝同穎。八月,寧化軍嘉禾合穗;寶鼎縣民張知友田禾隔四□,相去二尺許合穗,判府陳堯叟以聞。樓煩縣民田禾異本同穎。劍州嘉禾生,一莖九穗。四年三月辛巳,帝至西京,福昌縣民朱懿文嘉禾一本七穗。昌元縣民舒元晃田禾一莖九穗,知州柴德方以聞。金水縣民田禾一莖三十
 六穗。四月,六安縣麥秀二三穗。五月,唐、汝、廬、宿、泗、濠州麥自生。八月,蜀州禾一莖九穗。長壽縣民常自天田禾合穗者二。蒲縣禾異畝同穎。九月,知虢州李昭獻合穗禾。五年四月,遂州麥秀兩穗或三穗。七月,華州禾一莖兩穗。真定府四縣嘉禾合穗。八月,京兆府嘉禾生。九月,巴州禾一莖二十四穗,一莖十七穗。六年三月,邕州麥秀兩穗或三穗。七月,益州嘉禾九穗至十穗。朝邑縣民田禾八莖同穎。己未,召近臣觀嘉穀於後苑,有七穗至
 四十八穗,繪以示百官。八月,龍門縣、永寧軍博野縣民田並嘉禾生合穗。瀛州嘉禾生,知州馮守信以聞。忻州秀容、定襄二縣民田禾合穗。保定軍公田、大通監並嘉禾生。九月,京兆府獻《長安縣嘉禾圖》,一枝雙穗。七年,通泉縣尉劉定辭官廨禾一本六穗。邯鄲縣民馬文田禾隔□合穗者二本。滁州榷酒署內禾一莖二穗。晉原、平原二縣民田禾並一本十三穗。三月,郾城縣麥秀兩穗、三穗。八月,知亳州李迪獻禾一莖三穗至十穗。府谷縣
 民劉善田禾隔三□合成一穗。嵐州牙吏燕青田禾一莖八穗,一莖五穗。遼州平城民田禾隔二□合穗,有十三本或二十一本合為一者。九月,施州禾一莖九穗至十二穗。真定、貝州並嘉禾合穗。八年,湖陽縣麥秀兩穗、三穗。四月,旭川縣民任慶和田禾一莖九穗。閏六月,眉山縣民楊文繼、邛州李義田禾並一莖九穗。七月,永靜軍禾隔□合穗者二,監軍使仲甫以聞。八月,桂陽監民何文勝田粟一本二穗。九年四月,建初縣麥秀兩穗或
 三穗。八月,判大名府魏咸信獻合穗禾。永靜軍阜城縣民田谷隔三□合穗者二本。廣州嘉禾生。安化縣民吳景延田禾穗長尺五寸。九月,知鳳翔府趙湘、知邠州王守斌並獻《嘉禾圖》。



 天禧元年七月,流江縣禾一莖九穗。二年九月,河北安撫副使張昭遠獻穀穗三,各長尺餘。資州禾一莖九穗。三年七月,饒陽縣民楊宣田禾二□,相去二尺許,合為一穗。益州嘉禾一莖九穗。四年八月,內出《玉宸殿瑞穀圖》示近臣,每本有九穗、十穗者。九月,
 郪縣民岑貫田禾一莖九穗,知州蘇維甫以聞。五年四月,河南府民田嘉禾合穗,知府五欽若以聞。七月,導江縣民趙元賞、青城縣民王偉田禾並一莖九穗。



 乾興元年五月,南劍州麥一本五穗。綿州麥秀兩歧。八月,洋州嘉禾合穗。十一月,高陵縣嘉禾合穗。



 天聖二年八月乙酉,寧化軍嘉禾異畝同穎。四年九月,榮州禾一本九穗。五年,資州禾一本九穗。六年,忻州禾異本同穎。五月乙未,陳州瑞麥一莖二十穗。六月,陳州獻《瑞麥圖》。七年七
 月,河南府嘉禾合穗。八年八月壬午,召近臣觀瑞穀於元真殿。九年,膚施縣禾異畝同穎。



 景祐元年七月,磁州嘉禾合穗。八月,大名府嘉禾合穗。九月,涇州、磁州、保德軍並嘉禾合穗。十月,孝感、應城二縣稻再熟。成德軍禾一本九穗。三年五月,榮州禾一莖九穗。四年七月己巳,臨清縣穀異畝同穎者六十本。



 康定元年六月,蜀州、懷安軍並禾九穗。



 慶歷二年,壽安縣嘉禾合穗。六年五月,昭化縣禾一莖兩歧。八月,趙州、懷州並嘉禾異畝同穎。
 九月,定襄縣嘉禾隔二壟合穗。長江縣禾一莖十穗。十二月,石照諸縣野穀穭生。七年九月,邠州、榮州、德州並嘉禾合穗。



 皇祐元年,密州禾合穗者五本。永康軍禾一莖九穗。二年九月,延州、石州並嘉禾異畝合穗。永康軍嘉禾一莖九穗。十二月,密州禾十莖合一穗。石州四莖合一穗。三年五月,彭山縣上《瑞麥圖》,凡一莖五穗者數本。帝曰:「朕賞禁四方獻瑞,今得西川麥秀圖,可謂真瑞矣!其賜田夫束帛以勸之。」是月,滁州麥一莖五穗。四年
 八月,嘉州、蜀州並嘉禾一莖九穗。九月,南劍州有禾一本,雙莖二十穗。五年三月,資州嘉禾一莖九穗。閏六月,資州麥秀兩歧。七月,鄆州、祁州禾異畝同穎。九月,成德軍嘉禾異畝同穎。綿州禾一莖九穗。



 至和元年十二月,蜀州嘉禾一莖九穗。二年五月,亳州麥秀兩歧。六月,應天府貢大麥一本七十穗,小麥一本二百穗。八月,邛州嘉禾一莖九穗。



 嘉祐三年六月,綿州麥一穗兩歧。七月,泰山上《瑞麥圖》,凡五本五百一穗。四年六月,彰明縣有
 麥兩歧百餘本。五年三月,崇安縣嘉禾一本九十莖。七年,陵州禾一莖九穗。九月。平遙縣禾異畝合穗。



 熙寧元年,永興軍禾一莖四穗。眉州禾一莖九穗。四年,乾寧軍禾二莖合穗。成德軍、晉州、汾州禾異壟同穗。六年,南溪縣禾一苗九穗。八年,懷安軍、瀘州、渠州各麥秀兩歧。安喜縣禾二本間五壟合穗。平山縣禾合穗者二。保塞縣禾七本間一壟或兩壟合穗。潞城縣禾合穗者二。九年,火山軍禾間五壟,束鹿、秀容二縣間四壟,渤海縣皆異
 壟同穎。流江縣禾一苗九穗。譙縣麥一本三穗。尉氏縣、湖陽縣、彭城縣麥一本兩穗。渠州大麥一穗兩歧,或三歧、四歧者。陽翟縣麥秀兩歧。天興、寶雞二縣皆麥秀兩歧,仍一本有三四穗或六穗者。石州、安州麥秀兩歧。十年,磁州禾合穗。眉州禾生九穗。亳州禾生二穗。



 元豐元年,武康軍禾一莖十一穗。汝州禾合穗。寧江軍禾一莖十穗。邢州麥秀兩歧。夔州麥一本三穗。二年,簡州、安德軍麥秀兩歧。曹州生瑞禾。北京、安武軍、懷州、鎮戎軍禾
 合穗。鎮戎軍、懷州禾皆異畝同穎。袁州禾一莖八穗至十一穗,皆層出,長者尺餘。府州禾異畝同穗。三年,眉州禾一本九穗。齊州禾一莖五穗。趙州禾二本合穗。安州麥一本三穗至五穗,凡十四莖。深州麥秀兩歧,或三四穗,凡四十畝。眉州麥秀兩歧。四年,徐州麥一本百七十二穗。代州禾合穗。襄邑縣禾一本九穗。五年,高邑縣禾一莖五穗。青州、安肅軍、憲州禾皆異畝同穎。六年,洪州七縣稻已獲再生,皆實。威勝軍武鄉縣禾二本間五壟
 合穗。歷城縣禾二本合穗。趙州禾間三壟合穗。唐州禾二穗者四。瀘州禾九穗。懷、青、濰三州禾皆異壟同穗。府州、陜州保平軍禾皆合穗。七年,蜀州禾生九穗。青州禾異畝同穎者十一。同州禾異畝同穎。合州麥秀兩歧。八年,亳州麥一莖二穗,一莖三穗,一莖四穗。鎮潼軍秫禾苗異壟同穗。岷州禾皆四穗。泰寧軍禾異本同穎者三。是歲秋、冬,保、澤、趙、鄂、隰、滄、濰、密、簡、饒、諸州、威勝軍禾合穗,或異畝同穎。



 元祐元年,簡州禾合穗。石州禾異畝同
 穎。二年,忻、隰、磁、灘、懷州禾異畝同穎。趙、忻州禾合穗。三年,祁、保、彭州禾異畝同穎。瀛、磁、代、豐州、安國軍禾合穗。劍州、安國軍麥秀兩歧。夔州麥一本十二穗。四年,泰寧軍麥異畝同穎。流江縣禾一本二穗。榮德縣禾一本九穗。青、鄭、濟、趙州禾合穗及有一本三穗。峨嵋縣禾異畝同穎,又禾登一百五十三穗。五年,冀州、安武軍、大名府、成德軍禾合穗。永寧軍禾二本隔五壟合穗。平定軍禾異畝同穗。汀州禾生三十六穗。劍州禾一本八穗。普州
 麥一莖雙穗。夔州麥秀五歧。六年,汝陽縣、美原縣、兗州鄒縣麥一莖數穗。南劍州粟一本三十九穗。瀛、定、懷、汝、晉、昌州、平定、永康軍禾合穗。七年,均、兗、祁、滄、資、華、柳州禾合穗。鄂州禾一本一枝兩穗,三本三枝兩穗。仙源縣禾異壟合穗。耀州粟二莖隔兩壟合為一穗。梁山軍禾一莖九穗。固始縣麥有雙穗。定陶縣、丹陽縣麥秀兩歧。



 紹聖元年,博野縣麥一本五穗。漢陽軍麥秀兩歧。樂壽縣麥一本兩穗或三穗。懷安軍禾一本九穗。二年,青、濰、
 果、冀、德、濱、嵐、濮、達州禾合穗。三年,安武軍禾合穗。嵐州禾兩根合穗者二。普、相、青、齊、嵐州、永康軍禾異畝同穎,合秀至九穗。泉州粟二本五穗、八穗。瑕丘縣、武陟縣、陜城、小溪四縣麥合穗。良原縣、沉丘縣、長子縣麥秀兩歧。四年,河中府麥秀三穗。虹縣、雲安縣麥秀兩歧。茂州一枝兩穗。汶山縣一枝三穗至六穗。西京、鄆、齊、隰州禾合穗。穎昌府禾一莖四穗至五穗。



 元符元年,慶州禾異本同穎。青、晉、潞州、荊南府、永寧、鎮戎軍等一十一處禾
 合穗。邢州禾異壟合穗。南劍州、嘉州禾一莖九穗。內鄉縣麥一莖兩穗。符離、靈壁、臨渙、蘄、虹五縣麥秀兩穗。兩當縣麥秀三穗。安平縣生瑞麥。二年,漣水軍麥合穗。鄧、岷州、鎮戎軍禾合穗。十一月,岷州宕昌砦生瑞麥。



 建中靖國元年,沛縣、晉州禾合穗。崇寧元年,淄州禾合穗。二年,晉寧軍、忻州禾合穗。五年,河南府、保德軍、慶、蘭、潭、冀、府州、岢嵐軍禾合穗。淮西路民田既割,復生實。



 大觀元年,蜀州粟一莖九穗。二年,鞏州粟一莖六穗。鎮潼軍、隆德
 府、保德軍、慶、蘭州禾合穗。武信軍禾一莖九穗。簡州麥秀兩歧。三年,武信軍、瀘、遂、普州麥秀兩歧。四年,蔡州麥一莖兩歧至七八歧者九十畝。九月,尚書左僕射張商英表上《袁州瑞禾圖》及宋大雅獻《修者禾》十有三章,賜詔褒答。商英請並寫置中書省右僕射廳壁,許之,仍許三省、樞密院同觀。



 政和元年,知河南府鄧洵武言:「秋禾大稔,自雙穗至十穗以上,嘉禾無雙。」榮州粟一莖九穗。蔡州麥一莖兩歧或三五歧至八九畝近約十畝,遠或
 連野。二年,知定州梁士野奏嘉禾合穗,一科相隔五壟,計六尺三寸,生為一穗,並中間壟內一科三莖,上生粟三穗。五年,鄧州、仙井監嘉禾合穗。是冬,臺州進寧海縣早禾一稃二米者凡三石。時方修明堂,遂協成典禮,詔許拜表賀。自是史官多記奇祥異瑞,謂麥禾為常事不書。惟宣和末,郭藥師言嘉禾合穗,以新收復,特書之。



 建隆三年七月,南唐李景獻鳳卵。



 雍熙四年十月,知潤州程文慶獻鶴,頸毛如垂纓。



 端拱元年八月,清遠縣廨
 舍有鳳集柏樹,高六尺,眾禽隨之東北去,知州李昌齡圖以獻。



 至道元年九月,京師自旦至酉,群鳥百餘萬,飛翔有聲,識者云「突厥雀」。



 景德元年五月庚寅午時,白州有三鳳自東來,入城中,眾禽圍繞至萬歲寺,棲百尺木上。身長九尺,高五尺,文五色,冠如金杯。申時北向而去。畫圖以聞。



 大中祥符元年春,升州見黃雀群飛蔽日,有從空墜者,占主民有役事。是歲火。



 寶元二年,長舉縣有白鵲,觜腳紅,不類常鵲。



 治平四年五月,太子右贊善大
 夫陳世修獻白烏。



 熙寧七年六月乙未,增城縣鳳凰見。



 元豐三年八月戊寅,平棘縣獲白鵲。九月丙午,趙州獲白烏。六年七月壬申,丹州生白鵲。



 政和三年九月,大饗明堂,有鶴回翔堂上,明日,又翔於上清宮。是時,所在言瑞鶴,宰臣等表賀不可勝紀。



 宣和元年九月戊午,蔡京等表賀赤烏,又賀白鵲。政和後,禁苑多為村居野店,又聚珍禽、野獸、麀鹿、駕鵝、禽鳥數百實其中。至宣和間,每秋風夜靜,禽獸之音四徹,宛若深山大澤陂野之間,識
 者以為不祥。宣和末,南郊禮畢,御郊宮端拱殿。天未明,百闢方稱賀,忽有鴞止鳴於殿屋,若與贊拜聲相應和,聞者駭之。時已報女真背盟,未逾月,內禪。明年有陷城之難。



 建炎三年,高宗在揚州。二月辛亥早朝,有禽翠羽,飛鳴行殿三匝,一再止於宰臣汪伯彥朝冠。冠,尊服,飛鳥踐之不祥;翠羽,又青祥也。劉向以為「野鳥入宮,宮室將空」。一曰敗亡之應。是月,金人入揚州,有倉卒渡江之變。未幾,伯彥罷相,尋坐貶。四年正月丁巳,金人圍陜州,
 有鳶、鴉數萬飛噪城上,與戰聲相亂。金將婁宿曰:「城當陷,急攻之。」遂失守,近羽孽也。七年,梟鳴於劉豫後苑,又群鳥鳴於內庭,如曰「休也」。豫惡之,募人獲一梟者予錢五千。是歲,偽齊亡。十七年二月,有白烏六集於高禖壇上,府尹沉該以瑞奏。二十七年,饒州番陽縣有妖烏,鳧身雞尾,長喙方足赤目,止於民屋數日,彈矢不能中。



 乾道六年,邵武軍泰寧縣有雀飛鳴,立死於瑞寧佛剎香爐。先是紹興初,是邑有雀立死於丹霞佛剎香爐,皆羽
 孽也,而浮圖氏因謂之雀化。



 慶元三年春,池州銅陵縣鴛鴦雄化為雌。



 紹興五年,江東、西羊大疫。十七年,汀州羊無角。



 嘉定九年,信州玉山縣羊生駢首。



 端拱元年十一月戊午夜,西北方有赤氣如日腳,高二丈。



 咸平六年六月辛未,赤氣出婁,貫天廩,占曰:「倉稟有火災。」



 景德三年三月丙辰,北方赤氣亙天。



 大中祥符三年十二月癸亥,青赤氣貫紫微。



 慶歷三年十二月二十
 六日,天雄軍、德、博州天降紅雪,盡,血雨。



 熙寧二年十一月,每夕有赤氣見西北隅,如火,至人定乃滅。



 元祐三年七月丁卯夜,東北方明如晝,俄成赤氣,內有白氣經天。



 建中靖國元年正月朔夕,有赤氣起東北,彌亙西方,久之,中出白氣二,及赤氣將散,復有黑氣在其傍。



 宣和元年四月丙子夜,四北赤氣數十道亙天,犯紫宮北斗。仰視,星皆若隔絳紗,拆裂有聲,間以白黑二氣,自西北俄入東北,延及東南,迨曉乃止。



 靖康元年九月戊寅,有赤
 氣隨日出。



 建炎元年八月庚午,東北方有赤氣,占曰「血祥」。四年五月,洞庭湖夜赤光如火見東北,亙天,俄轉東南,此血祥也。壬子夜,西北方有赤氣彌天,貫以白氣如練者十數,犯北斗、文昌、紫微,由東南而散。



 紹興七年正月乙酉夜,北方有赤氣達旦。辛卯,斗、牛間赤氣如火。十一月癸卯,南方有赤氣,東北皆赤雲,自日入至於甲夜。八年九月甲申,赤氣出紫微垣。十八年八月丁亥、九月甲寅,皆有赤氣如火。二十年十一月,建昌軍新城縣永
 安村大風雪,夜半若數百千人行聲,語笑歌哭,雜擾匆遽,而凝寒陰塞,咫尺莫辨。明旦,雪中有人、畜、鳥、獸蹄跡,流血污染十餘里,入山乃絕。二十七年三月乙酉,赤氣出紫微垣。七月壬申赤氣隨日入。十月壬寅,赤氣如火;三十年二月壬申,亦如之。三十二年春,淮水溢,中有赤氣如凝血。



 隆興二年十一月庚寅,日入後,赤雲隨之。



 乾道元年八月壬午,赤氣中天,自日入至於甲夜。六年十月庚午,赤氣隨日出。十一月丁丑,赤雲隨日入,至於甲
 夜。七年七月壬寅、十月乙巳、丙午、淳熙三年八月丁酉、戊戌,皆有赤氣隨日入出。十三年,行都民家有血自地中出,濺染污人衣。十四年十一月癸丑、甲寅,有赤氣隨日入出。



 紹熙三年春,潼川路久旱,日、月、星皆有赤氣。四年十一月甲戌,赤雲夜見,白氣間之。



 慶元六年十月,赤氣夜發橫天。



 嘉泰四年二月庚辰夜,有赤雲間以白氣,東北亙天,後八日,國有大火,占者以為火祥。



 嘉定六年十月乙卯,赤氣隨日出。十一月辛卯,赤氣隨日出。



 端平
 三年七月甲申,天雨血。



 寶祐二年,蜀雨血。



 開寶七年六月,棣州有火自空墮於城北,有物如龍。



 端拱元年九月,瀘州鹽井竭,遣匠劉晚入視,忽有聲如雷,火焰突出,晚被傷。



 建炎元年正月辛卯夜,西北陰雪中有如火光。



 紹興三十二年,建昌軍新城縣有巨室,篋中時有火光,燔衣帛過半而篋不焚,近火孽也。



\end{pinyinscope}