\article{本紀第四}

\begin{pinyinscope}

 太宗一



 太宗神功聖德文武皇帝諱炅,初名匡乂,改賜光義,即位之二年改今諱,宣祖第三子也,母曰昭憲皇后杜氏。初,後夢神人捧日以授,已而有娠,遂生帝於浚儀官舍。
 是夜,赤光上騰如火,閭巷聞有異香,時晉天福四年十月七日甲辰也。帝幼不群,與他兒戲,皆畏服。及長,隆準龍顏,望之知為大人,儼如也。性嗜學,宣祖總兵淮南,破州縣,財物悉不取,第求古書遺帝,恆飭厲之,帝由是工文業,多藝能。仕周至供奉官都知。太祖即位,以帝為殿前都虞候,領睦州防禦使。親征澤、潞,帝以大內點檢留鎮,尋領泰寧軍節度使。征李重進,為大內都部署,加同平章事、行開封尹,再加兼中書令。徵太原,改東都留
 守,別賜門戟,封晉王,序班宰相上。



 開寶九年冬十月癸丑,太祖崩,帝遂即皇帝位。乙卯,大赦,常赦所不原者咸除之。丙辰,群臣表請聽政,不許。丁巳,宰相薛居正等固請,乃許,即日移御長春殿。庚申,以弟廷美為開封尹兼中書令,封齊王;先帝子德昭為永興軍節度使兼侍中,封武功郡王;德芳為山南西道節度使、興元尹、同平章事。薛居正加左僕射,沉倫加右僕射,盧多遜為中書侍郎,曹彬仍樞密使,並同平章事。楚昭輔
 為樞密使,潘美為宣徽南院使,內外官進秩有差。詔茶、鹽、榷酤用開寶八年額。十一月癸亥朔,帝不視朝。甲子,追冊故尹氏為淑德皇后,越國夫人符氏為懿德皇后。戊辰,罷州縣奉戶。庚午,詔諸道轉運使察州縣官吏能否,第為三等,歲終以聞。命諸州大索知天文術數人送闕下,匿者論死。乙亥,命權知高麗國事王胄為高麗國王。癸未,幸相國寺。己丑,遣著作郎馮正、佐郎張□巳使契丹告哀。詔文武官由譴累不齒者,有司毋得更論前過。十二月己亥,置
 直舍人院。甲寅,禦乾元殿受朝,樂縣而不作。大赦,改是歲為太平興國元年。命太祖子及齊王廷美子並稱皇子,女並稱皇女。丁巳,置三司副使。戊午,契丹遣使來賻。己未,幸講武池,遂幸玉津園。庚申,節度使趙普、向拱、張永德、高懷德、馮繼業、張美、劉廷讓來朝。



 二年春正月壬戌,以大行殯,不視朝。丙寅,禁居官出使者行商賈事。戊辰,親試禮部舉人。甲戌,上大行皇帝謚曰英武聖文神德,廟號太祖。丙子,幸相國寺,還,御東華
 門觀燈。庚辰,閱禮部貢士十舉至十五舉者百二十人,並賜出身。戊子,命邕州廣源州酋長坦坦綽儂民富為檢校司空、御史大夫、上柱國。辛卯,幸講武池。置江南榷茶場。二月甲午,契丹遣使來賀即位及正旦。吳越國遣使來貢。罷南唐鐵錢。庚子,帝改名炅。壬寅,大宴崇德殿,不作樂。乙巳,幸新鑿池,遂幸講武池,宴射玉津園。丁未,占城國遣使來貢。己酉,令江南諸州鹽先通商處悉禁之。戊午,幸太平興國寺,遂幸造船務。還,幸建隆觀。三月
 壬戌朔,始立試官銜選限。己卯,以河陽節度使趙普為太子少保。己丑,幸開寶寺。置威勝軍。禁江南諸州銅。許契丹互市。夏四月辛卯,大食國遣使來貢。丁酉,契丹遣使來會葬。乙卯,葬太祖於永昌陵。五月壬戌,河南法曹參軍高丕、伊闕縣主簿翟嶙、鄭州滎澤令申廷溫坐不勤事,並免。癸亥,向拱、張永德、張美、劉廷讓皆罷節鎮,為諸衛上將軍。乙丑,幸新水磑,遂幸玉津園宴射。丙寅,詔繼母殺子及婦者同殺人論。庚午,宴崇德殿,不作樂。遣
 辛仲甫使契丹。甲戌,以十月七日為乾明節。己卯,祔太祖神主於廟,以孝明皇后王氏配,又以懿德皇后符氏、淑德皇后尹氏祔別廟。庚辰,詔作北帝宮於終南山。癸未,幸新水磑,遂宴射玉津園。六月辛卯朔,白龍見邠州要策池中。乙卯,幸開寶寺,遂幸飛龍院,賜從官馬。是月,磁州保安等縣墨蟲生,食桑葉殆盡。穎州大水。秋七月庚午,詔諸庫藏敢變權衡以取羨餘者死。癸未,鉅鹿、沙河步屈食桑、麥,河決滎澤、頓丘、白馬、溫縣。閏月己亥,
 幸白鶻橋,臨金水河。己酉,河溢開封等八縣,害稼。甲寅,詔發潭州兵擊梅山洞賊。丁巳,有司上閏年輿地版籍之圖。令支郡得專奏事。八月癸亥,黎州兩林蠻來貢。乙丑,平海軍節度使陳洪進來朝。癸酉,以觀燈,遂幸相國寺。戊寅,詔作祟聖殿。是月,陜、澶、道、忠、壽諸州大水,鉅鹿步蝻生,景城縣雹。九月乙未,幸弓箭院,遂幸新修三館。壬寅,幸新水磑,遂幸西御園宴射。丁未,渤尼國遣使來貢,山後兩林蠻來獻馬。辛亥,幸講武臺大閱。容州初貢
 珠。乙卯,鎮海、鎮東軍節度使錢惟浚來朝。丙辰,狩近郊。丁巳,吳越王遣使乞呼名,不允。是月,興州江水溢,濮州大水,汴水溢。



 冬十月戊午朔,賜百官及在外將校、長吏冬服。辛酉,契丹來賀乾明節。己巳,幸京城西北,觀衛士與契丹使騎射,遂宴苑中。己巳,群臣請舉樂,表三上,從之。丙子,詔禁天文卜相等書,私習者斬。辛巳,畋近郊。初榷酒酤。十一月丁亥朔,日有食之,既。庚寅,日南至,帝始受朝。甲午,遣李瀆等賀契丹正旦。丁酉,禁江南諸州新
 小錢,私鑄者棄市。癸丑,幸御龍弓箭直營,賜軍士錢帛有差。十二月丁巳朔,試諸州所送天文術士,隸司天臺,無取者黥配海島。庚午,畋近郊。癸酉,詔定晉州礬法,私煮及私販易者罪有差。辛巳,幸新水磑。高麗國王使其子元輔來賀即位。



 三年春正月丙戌朔,不受朝,群臣詣閣賀。庚寅,殿直霍瓊坐募兵劫民財,腰斬。甲午,浚汾河。雅州西山野川路蠻來朝。戊戌,開襄、漢漕渠,渠成而水不上,卒廢。己亥,光
 祿丞李之才坐擅入酒邀同列飲殿中,除名。庚子,罷陳州蔡河舟算。辛丑,浚廣濟、惠民及蔡三河,治黃河堤。乙巳,浚汴口。己酉,命修《太祖實錄》。辛亥,命群臣禱雨。癸丑,京畿雨足。二月丙辰,幸鄭國公主第。以三館新修書院為崇文院。丁巳,詔班諸州錄事、縣令、簿尉歷子合書式。甲子,罷昌州七井虛額鹽。丙寅,泗州錄事參軍徐璧坐監倉受賄出虛券,棄市。辛未,幸西綾錦院,命近臣觀織室機杼,還,幸崇文院觀書。詔鑿金明池。甲申,禁沿邊諸
 郡闌出銅錢。制西京新修殿名。三月乙酉朔,貝州清河民田祚十世同居,詔旌其門閭,復其家。辛丑,監海門戍、殿直武裕坐奸贓棄市。壬寅,秦州言,戎酋王泥豬寇八狼戍,巡檢劉崇讓擊敗之,梟其首以徇。己酉,吳越國王錢俶來朝。壬子,幸開寶寺。是月,壽州甘露降。夏四月乙卯朔,命群臣禱雨。召華山道士丁少微。丙辰,禁民自春及秋毋捕獵。庚午,幸建隆觀,遂幸西染院,又幸造船務。乙亥,置諸道轉運判官。己卯,陳洪進獻漳、泉二州,凡得
 縣十四、戶十五萬一千九百七十八、兵萬八千七百二十七。庚辰,幸城南觀麥,遂幸玉津園宴射。辛巳,侍御史趙承嗣坐監市征隱官錢,棄市。癸未,以陳洪進為武寧軍節度使、同平章事。錢俶乞罷所封吳越國王,及解天下兵馬大元帥,並寢書詔不名之命,歸其兵甲,求還,不許。是月,河決獲嘉縣。五月乙酉,赦漳、泉,仍給復一年。錢俶獻其兩浙諸州,凡得州十三、軍一、縣八十六、戶五十五萬六百八十、兵一十一萬五千三十六。丁亥,封錢俶
 為淮海國王,其子惟浚徙淮南軍節度使,惟治徙鎮國軍節度使。戊子,赦兩浙,給復如漳、泉。癸巳,遣李從吉等使契丹。乙未,占城國遣使獻方物。壬寅,定難軍節度使李克睿卒,子繼筠立。乙巳,以繼筠襲定難軍節度使。幸殿前都指揮使楊信第視疾。戊申,以秦州節度判官李若愚子飛雄矯制乘驛至清水縣,縛都巡檢周承□及劉文裕、馬知節等七人,將劫守卒據城為叛,文裕覺其詐,禽縛飛雄按之,盡得其狀,詔誅飛雄及其父母妻子
 同產,而哀若愚宗奠無主,申戒中外臣庶,自今子弟有素懷兇險、屢戒不悛者,尊長聞諸州縣,錮送闕下,配隸遠處,隱不以聞,坐及期功以上。六月戊午,復給乘驛銀牌。壬午,秦州清水監軍田仁朗擊破西羌,斬獲甚眾。癸未,詔太平興國元年十月乙卯以來諸職官以贓致罪者,雖會赦不得敘,永為定制。是月,泗州大水,汴水決寧陵縣。秋七月乙酉,大雨震電,西窯務蒿聚焚。壬辰,右千牛衛上將軍李煜卒,追封吳王。戊戌,金鄉縣民李光
 襲十世同居,詔旌其門。庚戌,改明德門為丹鳳門。壬子,中書令史李知古坐受賕擅改刑部所定法,杖殺之。八月癸丑,幸南造船務,遂幸玉津園宴射。滑州黃河清。丙辰,詔兩浙發淮海王緦麻以上親及管內官吏赴闕。辛未,夷州蠻任朗政來貢。癸酉,詹事丞徐選坐贓,杖殺之。甲戌,群臣請上尊號曰應運統天聖明文武皇帝,許之。九月甲申,親試禮部舉人。壬子,以布衣張遁為襄邑縣主簿,張文旦濮陽縣主簿。冬十月癸丑朔,契丹遣使來賀
 乾明節。高麗國王遣使來貢。庚申,幸武功郡王德昭邸,遂幸齊王邸,賜齊王銀萬兩、絹萬匹,德昭、德芳有差。辛酉,復兗州曲阜縣襲封文宣公家。庚午,畋近郊。是月,河決靈河縣。十一月丙申,祀天地於圜丘,大赦。禦乾元殿受尊號。庚子,幸齊王邸。丙午,以郊祀,中外文武加恩。十二月乙丑,幸講武臺觀機石連弩。庚午,畋近郊。戊寅,契丹遣使來賀正旦。己卯,置三司推官、巡官。



 四年春正月丁亥,命太子中允張洎、著作佐郎句中正
 使高麗,告以北伐。遣官分督諸州軍儲輸太原行營。庚寅,以宣徽南院使潘美為北路都招討制置使,分命節度使河陽崔彥進、彰德李漢瓊、彰信劉遇、桂州觀察使曹翰,副以衛府將直,四面進討。侍衛馬軍都虞候米信、步軍虞候田重進並為行營指揮使,將其軍以從,西上閣門使郭守文、順州團練使梁迥監護之。辛卯,命雲州觀察使郭進為太原石嶺關都部署,以斷燕薊援師。癸巳,置簽署樞密院事,以石熙載為之。乙未,宴潘美等
 於長春殿,賜以襲衣、金帶、鞍馬。癸卯,新渾儀成。二月壬子,幸國子監,遂幸玉津園宴射。甲寅,以齊王廷美子德恭為貴州防禦使。丙辰,以中書侍郎、尚書右僕射、同平章事沈倫為東京留守兼判開封府事,宣徽北院使王仁贍為大內都部署,樞密承旨陳從信副之。癸亥,賜扈從近臣鞍馬、衣服、金玉帶有差。甲子,帝發京師。戊寅,次澶州,觀魚於河。三月庚辰朔,次鎮州。丁亥,郭進破北漢西龍門砦,禽獲甚眾。乙未,郭進大破契丹於關南。庚子,
 左飛龍使史業破北漢鷹揚軍,俘百人來獻。乙巳,夏州李繼筠乞帥所部助討北漢。詔泉州發兵護送陳洪進親屬赴闕。夏四月己酉朔,嵐州行營與北漢軍戰,破之。庚戌,盂縣降。以石熙載為樞密副使。辛酉,以孟玄哲、劉廷翰為兵馬都鈐轄,崔翰總馬步軍,並駐泊鎮州。壬戌,帝發鎮州。折御卿克岢嵐軍,獲其軍使折令圖。乙丑,克隆州,獲其招討使李詢等六人。己巳,折御卿克嵐州,殺其憲州刺史郭翊,獲夔州節度使馬延忠。庚午,次太原,
 駐蹕汾東行營。辛未,幸太原城,詔諭北漢主劉繼元使降。壬申夜,帝幸城西,督諸將發機石攻城。甲戌,幸諸砦。乙亥,幸連城,視攻城諸洞。五月己卯朔,攻城西南,遂陷羊馬城,獲其宣徽使範超,斬纛下。辛巳,攻城西北。壬午,其騎帥郭萬超來降,遂移幸城南,手詔賜繼元。癸未,進攻將士盡奮,若將屠之。是夜,繼元遣使納款。甲申,繼元降,北漢平,凡得州十、縣四十、戶三萬五千二百二十。命祠部郎中劉保勛知太原府。乙酉,赦河東常赦所不原
 者,命錄死事將校子孫,瘞戰士。戊子,以榆次縣為新並州。優賞歸順將校,盡括僧道隸西京寺觀,官吏及高貲戶授田河南。北漢節度使蔚進盧遂以汾州降。己丑,以繼元為右衛上將軍、彭城郡公。帝作《平晉詩》,令從臣和。辛卯,繼元獻官妓百餘,以賜將校。乙未,築新城。送劉繼元緦麻以上親赴闕。丙申,幸城北,御沙河門樓。盡徙餘民於新城,遣使督之,既出,即命縱火。丁酉,以行宮為平晉寺,帝作《平晉記》刻寺中。廢隆州,隳其城。庚子,發太原。
 丁未,次鎮州。六月甲寅,以將伐幽、薊,遣發京東、河北諸州軍儲赴北面行營。庚申,帝復自將伐契丹。丙寅,次金臺頓,募民為鄉導者百人。丁卯,次東易州,刺史劉宇以城降,留兵千人守之。戊辰,次涿州,判官劉厚德以城降。己巳,次鹽溝頓,民得近界馬來獻,賜以束帛。庚午,次幽州城南,駐蹕寶光寺。契丹軍城北,帝率眾擊走之。壬申,命節度使定國宋偓、河陽崔彥進、彰信劉遇、定武孟玄哲四面分兵攻城。以潘美知幽州行府事。契丹鐵林廂
 主李札盧存以所部來降。癸酉,移幸城北,督諸將進兵,獲馬三百。幽州神武廳直並鄉兵四百人來降。乙亥,範陽民以牛酒犒師。丁丑,帝乘輦督攻城。秋七月庚辰,契丹建雄軍節度使、知順州劉廷素來降。壬午,知薊州劉守恩來降。癸未,帝督諸軍及契丹大戰於高梁河,敗績。甲申,班師。庚寅,命孟玄哲屯定州,崔彥進屯關南。乙巳,帝至自範陽。八月壬子,西京留守石守信坐從征失律,貶崇信軍節度使。甲寅,彰信軍節度使劉遇貶宿州觀
 察使。癸亥,命潘美屯河東三交口。甲戌,汴水決宋城縣。武功郡王德昭自殺。詔作太清樓。是月,秦州大水。九月己卯,河決汲縣。丁亥,置皇子侍讀。己亥,幸新城,觀鐵林軍人射強弩。庚子,華山道士丁少微詣闕,獻金丹及巨勝、南芝、玄芝。癸卯,山後兩林蠻以名馬來獻。丙午,鎮州都鈐轄劉廷翰及契丹戰於遂城西,大敗之,斬首萬三百級,獲三將、馬萬匹。冬十月乙亥,以平北漢功,齊王廷美進封秦王,薛居正加司空,沈倫加左僕射,盧多遜兼
 兵部尚書,曹彬兼侍中,白進超、崔翰、劉廷翰、田重進、米信並領諸軍節度使,楚昭輔、崔彥進、李漢瓊並加檢校太尉,潘美加檢校太師,王仁贍加檢校太傅,石熙載加刑部侍郎,文武從臣進秩有差。十一月庚辰,放道士丁少微歸華山。己丑,畋近郊。辛卯,忻州言與契丹戰,破之。關南言破契丹,斬首萬餘級。十二月丁未,占城國遣使來貢。丁卯,畋近郊。置諸州司理判官。



 五年春正月庚辰,詔宣慰河東諸州。壬午,新作天駟左、
 右監,以左、右飛龍使為左、右天廄使,閑廄使為崇儀使。庚寅,改端明殿學士為文明殿學士。二月戊辰,斬徐州妖賊李緒等七人。廢順化軍。三月戊子,會親王、宰相、淮海國王及從臣蹴鞠大明殿。己丑,左監門衛上將軍劉鋹卒,追封南越王。癸巳,代州言宣徽南院使潘美敗契丹之師於雁門,殺其駙馬侍中蕭咄李,獲都指揮使李重誨。閏三月丙午,幸水磑,因觀魚。甲寅,親試禮部舉人。丁巳,親試諸科舉人。庚午,幸講武池觀習樓船。辛未,甘、
 沙州回鶻遣使以橐駝名馬來獻。夏四月癸未,親試應百篇舉趙昌國,賜及第。壅汾河晉祠水灌太原,隳其故城。是月,壽州風雹,冠氏縣雨雹。五月癸卯朔,大霖雨。辛酉,命宰相祈晴。六月壬午,高麗國王遣使來貢。是月,穎州大水,徐州白溝溢入城。秋七月丁未,討交州黎桓,命蘭州團練使孫全興、八作使張浚、左監門衛將軍崔亮、寧州刺史劉澄、軍器庫副使賈湜、閣門祗候王僎並為部署。全興、浚、亮由邕州,澄、湜、僎由廉州,各以其眾致討。
 庚申,北海孑□生。八月甲申,西南蕃主龍瓊琚使其子羅若從並諸州蠻來貢。九月癸卯,黎桓遣使為丁璇上表求襲位。甲辰,史館上《太祖實錄》。壬戌,畋近郊。冬十月戊寅,大發兵屯關南及鎮、定州。己丑,發京師,至雄州,民治道。甲午,命侍衛馬軍都指揮使米信護定州屯兵。十一月庚子朔,安南靜海軍節度行軍司馬、權知州事丁璇上表求襲位,不報。丙午,以秦王廷美為東京留守,王仁贍為大內都部署,陳從信副之。己酉,帝伐契丹。壬子,
 發京師。癸丑,次長垣縣。關南與契丹戰,大破之。以河陽三城節度使崔彥進為關南都部署。戊午,駐蹕大名府。諸軍及契丹大戰於莫州,敗績。十二月甲戌,大閱,遂宴幄殿。衛士有盜獲獐者當坐,詔特釋之。戊寅,以保靜軍節度使劉遇、威塞軍節度使曹翰為幽州東、西路部署。庚辰,發大名府,因校獵。乙酉,帝至自大名府。交州行營與賊戰,大破之。



 六年春正月癸卯,置平塞、靜戎二軍。辛亥,易州破契
 丹數千眾。丙寅,改靜戎軍為安靜軍。二月己卯,命宰臣禱雨。三月己酉,興元尹德芳薨,追封岐王。癸丑,詔令諸路轉運使察官吏賢否以聞。丙辰,置破虜、平戎二軍。丁巳,高昌國遣使來貢。壬戌,交州行營破賊於白藤江口,獲戰艦二百艘,知邕州侯仁寶死之。會炎瘴,軍士多死者,轉運使許仲宣驛聞,詔班師。詔斬劉澄、賈湜於軍中,征孫全興下獄。令諸州長吏五日一慮囚。夏四月辛未,幸太平興國寺禱雨。丙戌,高麗國遣使來貢。禁西川諸州
 白衣巫師。罷湖州織羅,放女工。五月己未,雨。降死罪囚,流以下釋之。平塞軍與契丹戰,破之。六月甲戌,司空、平章事薛居正薨。七月丙午,詔渤海琰府王助討契丹。是月,延州、鄜、寧、河中大水,宋州蝗。九月乙未朔,日有食之。甲辰,左拾遺田錫上疏極諫,詔嘉獎之。丙午,置京朝官差遣院,初令中書舍人郭贄等考校課績。辛亥,以趙普為司徒,石熙載為樞密使。壬子,詔求直言。丙辰,易州言破契丹。斬綿州妖賊王禧等十人。



 冬十月癸酉,群臣三
 奉表上尊號曰應運統睿文英武大聖至明廣孝皇帝,許之。甲申,以河陽三城節度使崔彥進為關南都部署,侍衛馬軍都指揮使米信為定州都部署。丙戌,校歷代醫書。甲午,詔作蘇州太一宮成。十一月丁酉,監察御史張白坐知蔡州日假官錢糴糶,棄市。甲辰,改武德司為皇城司。女真遣使來貢。辛亥,祀天地於圜丘,大赦。禦乾元殿受尊號,內外文武加恩。壬子,令諸州監臨官有所聞見傳聞須面陳者,俟報。丁巳,交州行營部署孫全
 興棄市。辛酉,以樞密使楚昭輔為左驍衛上將軍。十二月癸酉,購求醫書。己卯,畋近郊。己丑,諸道節度州置觀察支使,奉料同掌書記,仍不得並置。辛卯,禁民私市近界部落馬。



 七年春正月甲午朔,不受朝,群臣詣閣稱賀。壬戌,定輿服等差及婚取喪葬儀制。二月甲申,改關南為高陽關,徙並州治唐明鎮。乙酉,特貰廬州管內逋米萬七千二百四十石。三月癸巳朔,日有食之。乙未,以秦王廷美為
 西京留守。乙巳,以旱分遣中黃門遍禱方岳。交州以王師致討,遣使來謝。壬子,賜秦王襲衣、通犀帶、錢十萬。是月,舒州上玄石有白文曰「丙子年出趙號二十一帝」。宣州雪霜,殺桑害稼。北陽縣蝗,飛鳥數萬食之盡。夏四月甲子,以樞密直學士竇偁、中書舍人郭贄並參知政事,如京使柴禹錫為宣徽北院使兼樞密副使。戊辰,中書侍郎兼兵部尚書、平章事盧多遜罷為兵部尚書。丁丑,西京留守、秦王廷美罷歸第,復其子德恭、德隆名皇侄,
 女韓氏婦落皇女、雲陽公主之號。盧多遜褫職流崖州,並徙其家,期周以上親悉配遠裔。庚辰,左僕射、平章事沉倫罷為工部尚書。禁河南諸州私鑄鉛錫惡錢及輕小錢。是月,潤州大水。五月辛丑,崔彥進敗契丹於唐興。戊申,慮囚。己酉,夏州留後李繼捧獻其銀、夏、綏、宥四州。辛亥,三交行營言,潘美敗契丹之師於雁門,破其壘三十六。丙辰,秦王廷美降封涪陵縣公、房州安置。以崇儀副使閻彥進知房州,監察御史袁廓通判軍州事,各賜
 白金三百兩。己未,府州破契丹於新澤砦,獲其將校以下百人。是月,陜州蝗,蕪湖縣雨雹。六月乙亥,遣使發李繼捧緦麻巳上親赴闕,其弟繼遷奔地斤澤。丙子,置譯經院。是月,河決臨濟縣。漢陽軍大水。



 秋七月甲午,以子德崇為檢校太保、同平章事,封衛王;德明為檢校太保、同平章事,封廣平郡王。乙卯,工部尚書沈倫以左僕射致仕。是月,河決範濟口。淮水、漢水、易水皆溢。陽谷縣蝗。關、陜諸州大水。



 八月庚申朔,太子太師王溥薨。己卯,詔
 川、峽諸州官織錦綺、鹿胎、透背、六銖、欹正、龜殼等悉罷之,民間勿禁。



 九月己丑朔,西京諸道系籍沙彌,令祠部給牒。甲寅,貴妃孫氏薨。邠州蝗。



 冬十月癸亥,詔河南吏民不得闌出邊關侵撓略奪,違者論罪,有得羊馬生口者還之。戊辰,幸金明池,御龍舟觀習水戰。河決武德縣,蠲臨河民租。己卯,左諫議大夫、參知政事竇偁卒。癸卯,《乾元歷》成。是月,岳州田鼠食稼。



 十一月己酉,以李繼捧為彰德軍節度使。禁民喪葬作樂。十二月戊午朔,日有
 食之。庚午,蠲兩浙諸州太平興國六年以前逋租。戊寅,高麗國王胄卒,其弟治遣使求襲位,詔立治為高麗國王。



 閏月戊子朔,豐州與契丹戰,破之,獲其天德軍節度使蕭太。占城國獻馴象。丙申,狩近郊。辛亥,詔赦銀、夏等州常赦所不原者。諸州置農師。



 八年春正月己卯,以東上閣門使王顯為宣徽南院使,酒坊使弭德超為北院使,並兼樞密副使。癸未,詔令州、縣長吏延問高年耆德。



 二月戊子朔,日有食之。丁酉,禁
 內屬部落私市女口。



 三月庚申,以右諫議大夫宋琪為參知政事。豐州破契丹兵,降三千餘帳。癸亥,分三司,各置使。癸酉,幸金明池,觀習水戰。丙子,親試禮部舉人。甲申,除福建諸州鹽禁。



 夏四月壬寅,班《外官戒諭辭》。壬子,流樞密副使弭德超於瓊州,並徙其家。乙卯,幸樞密使石熙載第視疾。



 五月丁卯,詔作太一宮於都城南。黎桓自稱三使留後,遣使來貢,並上丁璇讓表。詔諭桓送璇母子赴闕,不聽。丁亥,流威塞軍節度使曹翰於登州。乙
 亥,詔長吏誘致關、隴流亡。是月,河決滑州,過澶、濮、曹、濟,東南入於淮。相州風雹。



 六月己亥,以王顯為樞密使,柴禹錫為宣征南院使兼樞密副使。己酉,兗州泰山父老及瑕丘等七縣民詣闕請封禪。是月,穀、洛、瀍、澗溢,壞官民舍萬餘區,溺死者以萬計,鞏縣壞殆盡。



 秋七月辛未,參知政事郭贄罷為秘書少監。庚辰,加宋琪刑部尚書,以工部尚書李昉參知政事。是月,河、江、漢、滹沱及祁之資、滄之胡盧、雄之易惡池水皆溢為患。



 八月壬辰,以大
 水故,釋死罪以下。丁酉,山後兩林蠻來貢。溪、錦、敘、富四州蠻來附。庚戌,以樞密使石熙載為右僕射。辛亥,增《謚法》。詔軍國政要令參知政事李昉及樞密院副使一人錄送史館。



 九月癸丑朔,占城國獻馴象。初置水陸路發運於京師。是月,睢溢,浸田六十里。



 冬十月戊戌,改衛王德崇名元佐,廣平郡王德明名元祐,德昌名元休,德嚴名元雋,德和名元傑。已酉,進元佐為楚王、元祐陳王,封元休韓王、元雋冀王、元傑益王,並檢校太保、同平章事。
 司徒兼侍中趙普罷為武勝軍節度使。



 十一月壬子朔,以參知政事宋琪、李昉並平章事。癸丑,除川、峽民祖父母父母在別籍異財棄市律。己未,太一宮成。壬申,以翰林學士李穆、呂蒙正、李至並參知政事,樞密直學士張齊賢、王沔並同簽署樞密院事。庚辰,置侍讀官。十二月壬午朔,詔綏、銀、夏等州官吏招誘沒界外民歸業,仍給復三年。丁亥,賜河北、河東緣邊戍卒襦,京城諸軍米。淮海國王錢俶三上表乞解兵馬大元帥、國王、尚書中書
 令、太師等官。罷元帥名,餘不許。西人寇宥州,巡檢使李謁擊走之。是月,醴泉縣水中草變為稻。滑州河決。



 雍熙元年春正月壬子朔,不受朝,群臣詣閣拜表稱賀。戊午,右僕射石熙載薨。壬戌,購逸書。丁卯,涪陵縣公廷美薨,追封涪陵王。壬申,蠲諸州民去年官所貸粟。癸酉,左諫議大夫、參知政事李穆卒。



 三月丁巳,滑州河決既塞,帝作《平河歌》賜近臣,蠲水所及州縣今年租。癸未,以涪陵王子德恭、德隆為刺史,婿韓崇業為靜難軍司馬。
 是月,甘露降太一宮庭。



 夏四月乙酉,泰山父老詣闕請封禪。戊子,群臣表請凡三上,許之。甲午,幸金明池,觀習水戰,因幸講武臺觀射,賜武士帛。五月庚戌朔,除江南鹽禁。辛亥,幸城南觀麥,賜刈者錢帛。罷諸州農師。壬子,西州回鶻與波斯外道來貢。丁丑,乾元、文明二殿災。己卯,以京官充堂後官。六月丁亥,詔求直言。己丑,遣使按察兩浙、淮南、西川、廣南獄訟。鎮安軍節度使、守中書令石守信薨。庚子,令諸州長吏十日一慮囚。壬寅,詔罷封
 泰山。甲辰,禁邊臣境外種蒔。



 秋七月壬子,改乾元殿為朝元殿,文明殿為文德殿,丹鳳門為乾元門;改匭院為登聞鼓院,東延恩匭為崇仁檢院,南招諫匭為思諫檢院,西申冤匭為申明檢院,北通玄匭為招賢檢院。



 八月丁酉,親祠太一宮。壬寅,河水溢。是月,淄州大水。



 九月壬戌,群臣表三上尊號曰應運統天睿文英武大聖至仁明德廣孝皇帝,不許,宰相叩頭固請,終不許。丙寅,幸並河新倉。



 冬十月甲申,賜華山隱士陳摶號希夷先生。夏
 州言掩擊李繼遷,獲其母妻,俘千四百餘帳,繼遷走。壬辰,禁布帛不中度者。癸巳,嵐州獻牝獸,一角,並瑞物六十三種圖付史館。戊戌,忠州錄事參軍卜元乾坐受賕枉法,杖殺之。



 十一月壬子,高麗國王遣使來貢。丁巳,祀天地於圜丘,大赦,改元,中外文武官進秩有差。癸酉,以浦城童子楊億為秘書省正字。十二月庚辰,淮海國王錢俶徙封漢南國王。癸未,賜京畿高年帛。丁亥,罷嶺南採珠場。壬辰,立德妃李氏為皇后。丙申,禦乾元門,賜京
 師大酺三日。戊戌,大雨雪。



\end{pinyinscope}