\article{列傳第一百三十二}

\begin{pinyinscope}

 朱
 弁鄭望之張邵洪皓子適遵邁



 朱弁,字少章,徽州婺源人。少穎悟,讀書日數千言。既冠,入太學,晁說之見其詩,奇之,與歸新鄭,妻以兄女。新鄭介汴、洛間,多故家遺俗,弁游其中,聞見日廣。靖康之亂,
 家碎於賊,弁南歸。



 建炎初,議遣使問安兩宮,弁奮身自獻,詔補修武郎,借吉州團練使,為通問副使。至雲中,見粘罕,邀說甚切。粘罕不聽,使就館,守之以兵。弁復與書,言用兵講和利害甚悉。



 紹興二年,金人忽遣宇文虛中來,言和議可成,當遣一人詣元帥府受書還,虛中欲弁與正使王倫探策決去留,弁曰:「吾來,固自分必死,豈應今日覬幸先歸。願正使受書歸報天子,成兩國之好,蚤申四海之養於兩宮,則吾雖暴骨外國,猶生之年也。」倫
 將歸,弁請曰:「古之使者有節以為信,今無節有印,印亦信也。願留印,使弁得抱以死,死不腐矣。」倫解以授弁,弁受而懷之,臥起與俱。



 金人迫弁仕劉豫,且訹之曰:「此南歸之漸。」弁曰:「豫乃國賊,吾嘗恨不食其肉,又忍北面臣之,吾有死耳。」金人怒,絕其餼遺以困之。弁固拒驛門,忍饑待盡,誓不為屈。金人亦感動,致禮如初。久之,復欲易其官,弁曰:「自古兵交,使在其間,言可從從之,不可從則囚之、殺之,何必易其官?吾官受之本朝,有死而已,誓不
 易以辱吾君也。」且移書耶律紹文等曰:「上國之威命朝以至,則使人夕以死,夕以至則朝以死。」又以書訣後使洪皓曰:「殺行人非細事,吾曹遭之,命也,要當舍生以全義爾。」乃具酒食,召被掠士夫飲,半酣,語之曰:「吾已得近郊某寺地,一旦畢命報國,諸公幸瘞我其處,題其上曰『有宋通問副使朱公之墓』,於我幸矣。」眾皆泣下,莫能仰視。弁談笑自若,曰:「此臣子之常,諸君何悲也?」金人知其終不可屈,遂不復強。



 王倫還朝,言弁守節不屈,帝為官
 其子林,賜其家銀帛。會粘罕等相繼死滅,弁密疏其事及金國虛實,曰:「此不可失之時也。」遣李發等間行歸報。其後,倫復歸,又以弁奉送徽宗大行之文為獻,其辭有曰:「嘆馬角之未生,魂消雪窖;攀龍髯而莫逮,淚灑冰天。」帝讀之感泣,官其親屬五人,賜吳興田五頃。帝謂丞相張浚曰:「歸日,當以禁林處之。」八年,金使烏陵思謀、石慶充至,稱弁忠節,詔附黃金三十兩以賜。



 十三年,和議成,弁得歸。入見便殿,弁謝且曰:「人之所難得者時,而時之
 運無已;事之不可失者幾,而幾之藏無形。惟無已也,故來遲而難遇;惟無形也,故動微而難見。陛下與金人講和,上返梓宮,次迎太母,又其次則憐赤子之無辜,此皆知時知幾之明驗。然時運而往,或難固執;幾動有變,宜鑒未兆。盟可守,而詭詐之心宜嘿以待之;兵可息,而銷弭之術宜詳以講之。金人以黷武為至德,以茍安為太平,虐民而不恤民,廣地而不廣德,此皆天助中興之勢。若時與幾,陛下既知於始,願圖厥終。」帝納其言,賜金帛
 甚厚。弁又以金國所得六朝御容及宣和禦書畫為獻。秦檜惡其言敵情,奏以初補官易宣教郎、直秘閣。有司校其考十七年,應遷數官。檜沮之,僅轉奉議郎。十四年,卒。



 弁為文慕陸宣公,援據精博,曲盡事理。詩學李義山,詞氣雍容,不蹈其險怪奇澀之弊。金國名王貴人多遣子弟就學,弁因文字往來說以和好之利。及歸,述北方所見聞忠臣義士朱昭、史抗、張忠輔、高景平、孫益、孫谷、傅偉文、李舟、五臺僧寶真、婦人丁氏、晏氏、小校閻進、朱
 績等死節事狀,請加褒錄以勸來者。有《聘游集》四十二卷、《書解》十卷、《曲洧舊聞》三卷、《續骫骳說》一卷,《雜書》一卷、《風月堂詩話》三卷、《新鄭舊詩》一卷、《南歸詩文》一卷。



 鄭望之,字顧道,彭城人,顯謨閣直學士僅之子也。望之少有文名,山東皆推重。登崇寧五年進士第,自陳留簿累遷樞密院編修官,歷開封府儀、工、戶曹,以治辦稱。臨事勁正,不受請托。宦寺有強占民田者,奏歸之。蔡京子欲奪人妾,使人諭意,望之拒不受。除駕部員外郎兼金
 部。



 靖康元年,金人攻汴京,假尚書工部侍郎,俾為軍前計議使。既還,金人遣吳孝民與望之同入見。望之言金人意在金幣,且要大臣同議,乃命同知樞密院事李梲與望之再使,斡離不以朝廷受歸朝官及賜平州張覺手詔為辭,遣蕭三寶奴偕梲等還,以書求割三鎮,欲得宰相交地,親王送大軍過河。



 時高宗在康邸慷慨請行,遂與張邦昌乘筏渡濠,自午至夜分,始達金砦。又除望之戶部侍郎,同梲再至金營,仍以珠玉遺金人。金人拘
 留望之逾旬。會姚平仲夜劫砦不克,斡離不以用兵詰責諸使者,邦昌恐懼涕泣,王不為動。金人遂不欲留王,更請肅王,乃以兵送望之詣國王砦詰問。會再遣宇文虛中持割地詔至,望之得還,因盛言敵勢強大,我兵削弱,不可不和。既而金兵退,朝廷以議和非策,罷望之提舉亳州明道宮。



 建炎初,李綱以望之張皇敵勢,沮損國威,以致禍敗,責海州團練副使,連州居住。綱罷,詔望之為戶部侍郎,尋轉吏部侍郎。論王云之冤,帝為感動,復
 云元官,與七子恩澤。尋兼主管御營司參贊軍事。論航海不便,忤旨,以集英殿修撰再領亳州明道宮。起知宣州,逾年,以言章罷。



 紹興二年,會赦,復徽猷閣待制致仕。七年,落致仕,召赴行在。望之以衰老辭,帝謂大臣曰:「望之,朕故人也。」於是升徽猷閣直學士,復致仕。三十一年,卒,年八十四。贈中大夫。



 張邵,字才彥,烏江人。登宣和三年上舍第。建炎元年,為衢州司刑曹事。會詔求直言,邵上疏曰:「有中原之形勢,
 有東南之形勢。今縱未能遽爭中原,宜進都金陵,因江、淮、蜀、漢、閩、廣之資,以圖恢復,不應退自削弱。」



 三年,金人南侵,詔求可至軍前者,邵慨然請行,轉五官,直龍圖閣,假禮部尚書,充通問使,武官楊憲副之,即日就道。至濰州,接伴使置酒張樂,邵曰:「二帝北遷,邵為臣子,所不忍聽,請止樂。」至於三四,聞者泣下。翌日,見左監軍撻攬,命邵拜,邵曰:「監軍與邵為南北朝從臣,無相拜禮。」且以書抵之曰:「兵不在強弱,在曲直。宣和以來,我非無兵也,帥
 臣初開邊隙,謀臣復啟兵端,是以大國能勝之。厥後偽楚僭立,群盜蜂起,曾幾何時,電掃無餘,是天意人心未厭宋德也。今大國復裂地以封劉豫,窮兵不已,曲有在矣。」撻攬怒,取國書去,執邵送密州,囚於祚山砦。



 明年,又送邵於劉豫,使用之。邵見劉豫,長揖而已,又呼為「殿院」,責以君臣大義,詞氣俱厲,豫怒,械置於獄,楊憲遂降。豫知邵不屈,久之,復送於金,拘之燕山僧寺,從者皆莫知所之。後又作書,為金言「劉豫挾大國之勢,日夜南侵,不
 勝則首鼠兩端,勝則如養鷹,飽則揚去,終非大國之利」,守者密以告,金取其書去,益北徙之會寧府,距燕三千里。金嘗大赦,許宋使者自便還鄉,人人多占籍淮北,冀幸稍南。惟邵與洪皓、朱弁言家在江南。



 十三年,和議成,及皓、弁南歸。八月,入見,奏前後使者如陳過庭、司馬樸、滕茂實、崔縱、魏行可皆歿異域未褒贈者,乞早頒恤典。邵並攜崔縱柩歸其家。升秘閣修撰,主管祐神觀。左司諫詹大方論其奉使無成,改臺州崇道觀。移書時相,勸
 其迎請欽宗與諸王后妃。十九年,以敷文閣待制提舉江州太平興國宮。知池州,再奉祠卒,年六十一。累贈少師。



 邵負氣,遇事慷慨,常以功名自許,出使囚徙,屢瀕於死。其在會寧,金人多從之學。喜誦佛書,雖異域不廢。初,使金時,遇秦檜于濰州。及歸,上書言檜忠節,議者以是少之。後弟祁下大理獄,將株連邵,會檜死得免。有文集十卷。



 子孝覽、孝曾、孝忠。孝曾後亦以出使歿於金,金人知為邵子,尚憐之。



 洪皓,字光弼,番易人。少有奇節,慷慨有經略四方志。登政和五年進士第。王黼、朱勉皆欲婚之,力辭。宣和中,為秀州司錄。大水,民多失業,皓白郡守以拯荒自任,發廩損直以糶。民坌集,皓恐其紛競,乃別以青白幟,涅其手以識之,令嚴而惠遍。浙東綱米過城下,皓白守邀留之,守不可,皓曰:「願以一身易十萬人命。」人感之切骨,號「洪佛子」。其後秀軍叛,縱掠郡民,無一得脫,惟過皓門曰:「此洪佛子家也。」不敢犯。



 建炎三年五月,帝將如金陵,皓上
 書言:「內患甫平,外敵方熾,若輕至建康,恐金人乘虛侵軼。宜先遣近臣往經營,俟告辦,回鑾未晚。」時朝議已定,不從,既而悔之。他日,帝問宰輔近諫移蹕者謂誰,張浚以皓對。時議遣使金國,浚又薦皓於呂頤浩,召與語,大悅。皓方居父喪,頤浩解衣巾,俾易墨衰絰入對。帝以國步艱難、兩宮遠播為憂。皓極言:「天道好還,金人安能久陵中夏!此正春秋邲、郢之役,天其或者警晉訓楚也。」帝悅,遷皓五官,擢徽猷閣待制,假禮部尚書,為大金通問
 使,龔



 □副之。令與執政議國書,皓欲有所易,頤浩不樂,遂抑遷官之命。



 時淮南盜賊踵起,李成甫就招,即命知泗州羈縻之。乃命皓兼淮南、京東等路撫諭使,俾成以所部衛皓至南京。比過淮南,成方與耿堅共圍楚州,責權州事賈敦詩以降敵,實持叛心。皓先以書抵成,成以汴涸,虹有紅巾賊,軍食絕,不可往。皓聞堅起義兵,可撼以義,遣人密諭之曰:「君數千里赴國家急,山陽縱有罪,當稟命於朝;今擅攻圍,名勤王,實作賊爾。」堅意動,遂強
 成斂兵。



 皓至泗境,迎騎介而來,龔□曰:「虎口不可入。」皓遂還,上疏言:「成以朝廷饋餉不繼,有『引眾建康』之語。今靳賽據揚州,薛慶據高郵,萬一三叛連衡,何以待之?此含垢之時,宜使人諭意,優進官秩,畀之以京口綱運,如晉明帝待王敦可也。」疏奏,帝即遣使撫成,給米伍萬石。頤浩惡其直達而不先白堂,奏皓托事稽留,貶二秩。皓遂請出滁陽路,自壽春由東京以行。至順昌,聞群盜李閻羅、小張俊者梗穎上道。皓與其黨遇,譬曉之曰:「自古
 無白頭賊。」其黨悔悟,皓使持書至賊巢,二渠魁聽命,領兵入宿衛。



 皓至太原,留幾一年,金遇使人禮日薄。及至雲中,粘罕迫二使仕劉豫,皓曰:「萬里銜命,不得奉兩宮南歸,恨力不能磔逆豫,忍事之邪!留亦死,不即豫亦死,不願偷生鼠狗間,願就鼎鑊無悔。」粘罕怒,將殺之。旁一酋唶曰:「此真忠臣也。」目止劍士,為之跪請,得流遞冷山。流遞,猶編竄也。惟□至汴受豫官。



 雲中至冷山行六十日,距金主所都僅百里,地苦寒,四月草生,八月已雪,穴
 居百家,陳王悟室聚落也。悟室敬皓,使教其八子。或二年不給食,盛夏衣粗布,嘗大雪薪盡,以馬矢然火煨面食之。或獻取蜀策,悟室持問皓,皓力折之。悟室銳欲南侵,曰:「孰謂海大,我力可干,但不能使天地相拍爾。」皓曰:「兵猶火也,弗戢將自焚,自古無四十年用兵不止者。」又數為言所以來為兩國事,既不受使,乃令深入教小兒,非古者待使之禮也。悟室或答或默,忽發怒曰:「汝作和事官,而口硬如許,謂我不能殺汝耶?」皓曰:「自分當死,顧
 大國無受殺行人之名,願投之水,以墜淵為名可也。」悟室義之而止。



 和議將成,悟室問所議十事,皓條析甚至。大略謂封冊乃虛名,年號本朝自有;金三千兩景德所無,東南不宜蠶,絹不可增也;至於取淮北人,景德載書猶可覆視。悟室曰:「誅投附人何為不可?」皓曰:「昔魏侯景歸梁,梁武帝欲以易其侄蕭明於魏,景遂叛,陷臺城,中國決不蹈其覆轍。」悟室悟曰:「汝性直不誑我,吾與汝如燕,遣汝歸議。」遂行。會莫將北來,議不合,事復中止。留燕
 甫一月,兀朮殺悟室,黨類株連者數千人,獨皓與異論幾死,故得免。



 方二帝遷居五國城,皓在雲中密遣人奏書,以桃、梨、粟、面獻,二帝始知帝即位。皓聞祐陵訃,北向泣血,旦夕臨,諱日操文以祭,其辭激烈,舊臣讀之皆揮涕。紹興十年,因諜者趙德,書機事數萬言,藏故絮中,歸達於帝。言:「順昌之役,金人震懼奪魄,燕山珍寶盡徙以北,意欲捐燕以南棄之。王師亟還,自失機會,今再舉尚可。」十一年,又求得太后書,遣李微持歸,帝大喜曰:「朕不
 知太后寧否幾二十年,雖遣使百輩,不如此一書。」是冬,又密奏書曰:「金已厭兵,勢不能久,異時以婦女隨軍,今不敢也。若和議未決,不若乘勢進擊,再造反掌爾。」又言:「胡銓封事此或有之,金人知中國有人,益懼。張丞相名動異域,惜置之散地。」又問李綱、趙鼎安否,獻六朝御容、徽宗御書。其後梓宮及太后歸音,皓皆先報。



 初,皓至燕,宇文虛中已受金官,因薦皓。金主聞其名,欲以為翰林直學士,力辭之。皓有逃歸意,乃請於參政韓昉,乞於真
 定或大名以自養。昉怒,始易皓官為中京副留守,再降為留司判官。趣行屢矣,皓乞不就職,昉竟不能屈。金法,雖未易官而曾經任使者,永不可歸,昉遂令皓校云中進士試,蓋欲以計墮皓也。皓復以疾辭。未幾,金主以生子大赦,許使人還鄉,皓與張邵、朱弁三人在遣中。金人懼為患,猶遣人追之,七騎及淮,而皓已登舟。



 十二年七月,見於內殿,力求郡養母。帝曰:「卿忠貫日月,志不忘君,雖蘇武不能過,豈可舍朕去邪!」請見慈寧宮,帟人設簾,
 太后曰:「吾故識尚書。」命撤之。皓自建炎己酉出使,至是還,留北中凡十五年。同時使者十三人,惟皓、邵、弁得生還,而忠義之聲聞於天下者,獨皓而已。皓既對,退見秦檜,語連日不止,曰:「張和公金人所憚,乃不得用。錢塘暫居,而景靈宮、太廟皆極土木之華,豈非示無中原意乎?」檜不懌,謂皓子適曰:「尊公信有忠節,得上眷。但官職如讀書,速則易終而無味,須如黃鐘、大呂乃可。」八月,除徽猷閣直學士、提舉萬壽觀兼權直學士院。



 金人來取趙
 彬等三十人家屬,詔歸之。皓曰:「昔韓起謁環於鄭,鄭,小國也,能引義不與。金既限淮,官屬皆吳人,宜留不遣,蓋慮知其虛實也。彼方困於蒙兀,姑示強以嘗中國,若遽從之,謂秦無人,益輕我矣。」檜變色曰:「公無謂秦無人。」既而復上疏曰:「恐以不與之故,或致渝盟,宜告之曰:『俟淵聖及皇族歸,乃遣。』」又言:「王倫、郭元邁以身徇國,棄之不取,緩急何以使人?」檜大怒,又因言室捻寄聲,檜怒益甚,語在《檜傳》。翌日,侍御史李文會劾皓不省母,出知饒州。



 明年,大水,中官白鍔宣言:「燮理乖盭,洪尚書名聞天下,胡不用?」檜聞之愈怒,系鍔大理獄,尋流嶺表。諫官詹大方遂論皓與鍔為刎頸交,更相稱譽,罷皓提舉江州太平觀。鍔初不識皓,特以從太后北歸,在金國素知皓名爾。



 尋居母喪,他言者猶謂皓睥睨鈞衡。終喪,除饒州通判。李勤又附檜誣皓作欺世飛語,責濠州團練副使,安置英州。居九年,始復朝奉郎,徙袁州,至南雄州卒,年六十八。死後一日,檜亦死。帝聞皓卒,嗟惜之,復敷文閣直學
 士,贈四官。久之,復徽猷閣直學士,謚忠宣。



 皓雖久在北廷,不堪其苦,然為金人所敬,所著詩文,爭鈔誦求鋟梓。既歸,後使者至,必問皓為何官、居何地。性急義,當艱危中不少變。懿節後之戚趙伯璘隸悟室戲下,貧甚,皓賙之。範鎮之孫祖平為傭奴,皓言於金人而釋之。劉光世庶女為人豢豕,贖而嫁之。他貴族流落賤微者,皆力拔以出。惟為檜所嫉,不死於敵國,乃死於讒慝。



 皓博學強記,有文集五十卷及《帝王勇要》、《姓氏指南》、《松漠紀聞》、《金國
 文具錄》等書。子適、遵、邁。



 適字景伯,皓長子也。幼敏悟,日誦三千言。皓使朔方,適年甫十三,能任家事。以皓出使恩,補修職郎。紹興十二年,與弟遵同中博學宏詞科。高宗曰:「父在遠方,子能自立,此忠義報也,宜升擢。」遂除敕令所刪定官。後三年,弟邁亦中是選,由是三洪文名滿天下。改秘書省正字。



 甫數月,皓歸,忤秦檜,出知饒州,適亦出為臺州通判。垂滿,皓謫英州,適復論罷,往來嶺南省侍者九載。檜死皓還,
 道卒,服闋,起知荊門軍。應詔上寬恤四事:輕茶額錢,它州代貢禮物,闢試闈以復舊額,蠲官田令不種者輸租。改知徽州,尋提舉江東路常平茶鹽,首言役法不均之弊。



 會完顏亮來侵,上親征,適覲金陵,言:「本路旱,百姓逐食於淮,復遭金兵,今各懷歸而田產為官鬻,請聽其估贖之。」及亮斃,適上疏曰:「大定僭號,諸國未必服從,宜多遣密詔傳諭中原義士,各取州縣,因以畀之。王師但留屯淮、泗,募兵積粟,以為聲援。俟蜀、漢、山東之兵數道皆
 集,見可而進,庶幾兵力不頓,可以萬全。」升尚書戶部郎中,總領淮東軍馬錢糧。孝宗即位,海州解圍,符離用兵,饋餉繁多,適究心調度,供億無闕。遷司農少卿。



 隆興二年二月,召貳太常兼權直學士院。上欲除諸將環衛官,詔討論其制。適具唐及本朝沿革十一條上之,且言:「太祖、太宗朝,常以處諸將及降王之君臣,自後多以皇族為之,故國史以為官存而事廢。陛下修飭戎備,不必遠取唐制,祖宗故事蓋可法則。今徑行換授,恐有減奉之
 患,乞如閣職兼帶節度,至刺史帶上將軍,橫行遙郡帶大將軍,正使帶將軍,副使帶中郎將,又以下則帶左右郎將,其官府人吏,令有司相度以聞。」除中書舍人。時金人再犯淮,羽檄沓至,書詔填委,盜訪酬答率稱上旨,自此有大用意。金既尋盟,首為賀生辰使。金遣同簽書樞密院事高嗣先接伴,自言其父司空有德於皓,相與甚歡,得其要領以歸。



 乾道元年五月,遷翰林學士,仍兼中書舍人。秦塤久廢,忽予祠,適奏曰:「李林甫死後,諸子皆流
 配嶺南。秦檜稔惡自斃,不肖之孫官職仍舊,可謂幸矣。宮觀雖小,塤得之,則人以除用之漸,恐檜黨牽連而進。」其命遂寢。時巫伋復召,莫汲擢樞密院編修官,餘堯弼復龍圖閣學士,適謂其皆檜黨也,隨命繳之。



 六月,除端明殿學士、簽書樞密院事。上諭參政錢端禮、虞允文曰:「三省事與洪適商量。」東西府始同班奏事。八月,拜參知政事。諫議大夫林安宅以銅錢多入北境,請禁之,即蜀中取鐵錢行之淮上。事既行,適言其不可。上問之,適曰:「
 今每州不得千緡,一州以萬戶計之,每家才得數百,恐民間無以貿易。且客旅無回貨,鹽場有大利害。」上以為然,乃寢前命,但於蜀中取十五萬緡,行之廬、和二州而已。



 十二月,拜尚書右僕射、同中書門下平章事兼樞密使。未幾,春霖,適引咎乞退,林安宅抗疏論適,既而臺臣復合奏。三月,除觀文殿學士、提舉江州太平興國宮。尋起知紹興府、浙東安撫使。再奉祠。淳熙十一年薨,年六十八,謚文惠。



 適以文學聞望,遭時遇主,自兩制一月入
 政府,又四閱月居相位,又三月罷政,然無大建明以究其學。家居十有六年,兄弟鼎立,子孫森然,以著述吟詠自樂,近世備福鮮有及之。或謂適黨湯思退,又謂適來自淮東,言張浚妄費,浚以此罷相,子九人:槻、柲、□□、□修、灌、桴、楹、□、梠。



 遵字景嚴,皓仲子也。自兒時端重如成人,從師業文,不以歲時寒暑輟。父留沙漠,母亡,遵孺慕攀號。既葬,兄弟即僧舍肄詞業,夜枕不解衣。以父蔭補承務郎,與兄適
 同試博學宏詞科,中魁選,賜進士出身。高宗以皓遠使,擢為秘書省正字。中興以來,詞科中選即入館,自遵始。宰相秦檜子熹為官長,謦欬為人輕重,遵恬然不附麗。二年弗遷。



 皓南還,與朝論異,出守。遵遂乞外,通判常、婺、越三州。紹興二十五年,湯思退薦之,復入為正字。八月,兼權直學士院。湯鵬舉副臺端,密薦為御史。方賜對而父訃聞。二十八年,免喪,召對,極陳父冤,曰:「先臣與龔□同出疆,□仕於劉豫,以妄殺兵官為豫所誅,而秦檜贈
 以節旄,擢用其子。先臣拒金人之命,留十五歲乃得歸,顧南竄嶺外,臣兄弟屏跡在外。檜不分忠逆如此。」高宗悉為道謗語所起,且曰:「卿再登三館,嘗典書命,今以修注處卿。」遂拜起居舍人。



 奏乞以經筵官除罷及封章進對、宴會錫予、講讀問答等事,萃為一書,名之曰《邇英記注》。其後乾道間又有《祥曦殿記注》,實自遵始。又因面對,論鑄錢利害,帝嘉納之。遷起居郎兼權樞密院都承旨。舊制,修注官、經筵官許留身奏事,而近例無有。遵奏請
 復舊制,且言起居注未修者十五年,請除見修月進外,每月帶修,皆從之。



 二十九年,拜中書舍人。殿前裨將輔逵轉防禦使,王綱轉團練使,遵言:「近制管軍官十年始一遷,今兩人不滿歲,安得爾?」時勛臣子孫多躐居臺省,遵極言乞明有所止。高宗曰:「正立法,自今功臣子孫序遷至侍從,並令久任在京宮觀。」遵曰:「侍從,朝廷高選,非如磨勘階官,安有遷序之制?」退而上奏言:「今內外將家無慮二十人,若以序遷,不出十年,西清次對皆可坐致。
 太祖開國功臣子孫不過諸司,惟曹彬之子琮、瑋以功名自奮,遂為節度,初不聞有遞遷侍從之例。今旨一出,使穆清之地類皆將種,非所以示天下。望收還前詔。」又言:「瑞昌、興國之間茶商失業,聚為盜賊。望揭榜開諭,許其自新,願充軍者填刺,願為農者放還。」上皆可其奏。



 論者欲復鄱陽永平、永豐兩監鼓鑄,詔給、舍議,遵曰:「唐有鼓鑄使,國朝或以漕臣兼領,或分道置使,厘為三司。自中興來,置都大提點,官屬太多,動為州縣之害。間者亟
 行廢罷,又無一定之論,初委運使,又委提刑,又委郡守、貳,號令不一,鼓鑄益少。竊以為復置便。」



 三十年正月,試吏部侍郎。異時選人詣曹改秩,吏倚為市,亳毛不中節,必巧生沮閡,須賂餉滿欲乃止。遵明與約,茍於大體無害,先行後審,薦員有定限,而舉者周遮重復,或同時一章而巧為兩牘,或當薦五員而輒逾十數,或當舉職官而詭為京狀,或身系常調而妄稱職司,或東西分曹而交錯攙補,或已予復奪而指云事故,件析枚數,請凡如
 是者得通劾之。舊制,致仕任子,隨所在審敕牒即請行。是時,從議者請,必令於元州判奏。遵言:「士大夫或游宦粵、蜀,數千里外,不幸以死。臨終謝事,其家獲歸故里已為至難,今復因此齟齬,反復稽延,是明與惡吏為地也。」乃止仍舊貫。



 平江、湖、秀三州水,無以輸秋苗,有司抑令輸麥。遵言:「麥價珠不在米下,民困如是,奈何指夏以為秋,衍一以為二,使擠溝壑乎?願量取其半,而被水害者悉免之。」金人來索絳陽郭小的、安化劉孝恭二百家,遵
 以蜀之李特可為至戒,願以根集未足為解,淹引日月報之。遷翰林學士兼吏部尚書。汪澈論湯思退罷相,遵行制無貶詞,澈以為言。遂丐去,以徽猷閣直學士提舉太平興國宮。



 三十一年,金主完顏亮命其尚書蘇保衡由海道窺二浙,朝廷以浙西副總管李寶御之。寶駐兵平江,守臣朱翌素與寶異,朝議以遵嘗薦寶,乃命遵知平江。及寶以舟師搗膠西,凡資糧、器械、舟楫皆遵供億,寶成功而歸,遵之助為多。車駕幸金陵,禁衛士丐索無
 藝,它郡隨與不厭。至吳,乃相告曰:「內翰在此,汝毋復然。」先是,朝廷慮商舶為賊得,悉拘入官,既而不返,並海縣團萃巨艦及募水手、民兵,皆縶留未得去。遵因對論之,以船還商,而聽水手自便,吳人德之。



 孝宗即位,拜翰林學士承旨兼侍讀。詔問宰執、侍從、臺諫曰:「敵人來索舊禮,從之則不忍屈,不從則邊患未已。中原歸正人源源不絕,納之則東南力不能給,否則絕向化之心。宜指陳定論以聞。」遵與給事中金安節、中書舍人唐文若、起居
 郎周必大共為一議,其略謂:「不宜直情徑行,亦未可遽為之屈,謂宜遺金繒如前日之數,或許稍歸侵地如海、泗之類,則彼亦可借口而來議矣。」



 知隆興元年貢舉,拜同知樞密院事。壽康殿產金芝十二,同列議表賀,遵引李文靖奏災異故事風止之。薦眉山李燾、永嘉鄭伯熊及林光朝,未及用,會湯思退為左相,而次相張浚罷,御史周璪策遵且超遷,上章致劾,上亟徙置他官。遵不能安位,連章乞免,訖與御史俱去。是年七月,以端明殿學
 士提舉太平興國宮。



 乾道六年,起知信州。徙知太平州。前守周璪以嘗論遵,聞遵來,不俟合符馳去。遵追餞至十里,勞苦如平時,曰:「君當官而行,我何怨?」聞者以為盛德。圩田壞,民失業,遵鳩民築圩凡萬數。方冬盛寒,遵躬履其間,載酒食親餉饁,恩意傾盡,人忘其勞。運使張松忌功,妄奏圩未嘗決,民未嘗轉徙,必責圩戶自閼築,且裁省募工錢米之半。遵連疏爭,至酒遣朝臣覆按。於是將作少監馬希言、監察御史陳舉善狎至,黜松言,圩遂
 成,合四百五十有五。松無所洩其忿,則別治溧水永豐圩,來調丁、米、木,數甚廣。遵曰:「郡當歲儉,方振恤流移,勸分乞糴,如自刲其股以充喉,不暇食,況能飽他人腹哉。」執不從。



 楚地旱,旁縣振贍者慮不早,施置失後先,或得米而亡以炊,或闔戶莩藉而廩不至。遵簡賓佐,隨遠近壯老以差賦給,蠲租至十九,又告糴於江西,得活者不啻萬計。戍兵乘時盜利,曹伍剽於野,盡執拘以歸其軍。故當大札瘥而邑落晏然。徙知建康府、江東安撫使兼
 行宮留守。孝宗諭當制舍人範成大,褒其治績,且許入覲。



 時虞允文當國,有北征志。先調侍衛馬軍出屯,其在府者五軍,悉送其孥,謀築營砦,無慮萬灶。張松用不能罷,特敕遵同宰執赴選德殿奏事。遵奏外臣不敢尾二府後,願需班退別引,上弗許。進資政殿學士以行。至則揭榜,民苗米唯輸正不輸耗,聽民自持斛概,庾人不能輕重其手。遍行郊野卜砦地,求不妨民居、不夷塚墓者,逾年始得之。營卒醉,妄言搖眾,斬之,磔於市,三軍無敢
 嘩。有晝入旗亭挺刃椎壚者,械付獄,驛上奏未下,統帥懼得譴,請自治之。孝宗怒,罷統帥,遵亦坐貶兩秩。未幾,五營成,復元官,仍拜資政殿學士。淳熙元年,提舉洞霄宮。十一月,薨,年五十有五。謚文安。



 邁字景盧,皓季子也。幼讀書日數千言,一過目輒不忘,博極載籍,雖稗官虞初,釋老傍行,靡不涉獵。從二兄試博學宏詞科,邁獨被黜。紹興十五年始中第,授兩浙轉運司干辦公事,入為敕令所刪定官。皓忤秦檜投閑,檜
 憾未已,御史汪勃論邁知其父不靖之謀,遂出添差教授福州。累遷吏部郎兼禮部。



 上居顯仁皇后喪,當孟饗,禮官未知所從,邁請遣宰相分祭,奏可。除樞密檢詳文字。建議令民入粟贖罪,以紓國用,又請嚴法駕出入之儀。



 三十一年,議欽宗謚,邁曰:「淵聖北狩不返,臣民悲痛,當如楚人立懷王之義,號懷宗,以系復仇之意。」不用。吳璘病篤,朝論欲徙吳拱代之。邁曰:「吳氏以功握蜀兵三十年,宜有以新民觀聽,毋使尾大不掉。知樞密院事葉
 義問出視師,奏以邁參議軍事,至鎮江,聞瓜洲官軍與金人相持,遑遽失措。會建康走驛告急,義問遽欲還,邁力止之曰:「今退師,無益京口勝敗之數,而金陵聞返旆,人心動搖,不可。」遷左司員外郎。



 三十二年春,金主褒遣左監軍高忠建來告登位,且議和,邁為接伴使,知閣門張掄副之。上謂執政曰:「向日講和,本為梓宮、太后,雖屈己卑辭,有所不憚。今兩國之盟已絕,名稱以何為正,疆土以何為準,朝見之儀,歲幣之數,所宜先定。」及邁、掄入
 辭,上又曰:「朕料此事終歸於和,欲首議名分,而土地次之。」邁於是奏更接伴禮數,凡十有四事。自渡江以來,屈己含忍多過禮,至是一切殺之,用敵國體,凡遠迎及引接金銀等皆罷。既而高忠建有責臣禮及取新復州郡之議,邁以聞,且奏言:「土疆實利不可與,禮際虛名不足惜。」禮部侍郎黃中聞之,亟奏曰:「名定實隨,百世不易,不可謂虛。土疆得失,一彼一此,不可謂實。」兵部侍郎陳俊卿亦謂:「先正名分,名分正則國威張,而歲幣亦可損矣。」



 進起居舍人。時議遣使報金國聘,三月丁巳,詔侍從、臺諫各舉可備使命者一人。初,邁之接伴也,既持舊禮折伏金使,至是,慨然請行。於是假翰林學士,充賀登位使,欲令金稱兄弟敵國而歸河南地。夏四月戊子,邁辭行,書用敵國禮,高宗親札賜邁等曰:「祖宗陵寢,隔闊三十年,不得以時灑掃祭祀,心實痛之。若彼能以河南地見歸,必欲居尊如故,正復屈己,亦何所惜。」邁奏言:「山東之兵未解,則兩國之好不成。」至燕,金閣門見國書,呼曰:「不
 如式。」抑令使人於表中改陪臣二字,朝見之儀必欲用舊禮。邁初執不可,既而金鎖使館,自旦及暮水漿不通,三日乃得見。金人語極不遜,大都督懷忠議欲質留,左丞相張浩持不可,乃遣還。七月,邁回朝,則孝宗已即位矣。殿中侍御史張震以邁使金辱命,論罷之。明年,起知泉州。



 乾道二年,復知吉州。入對,遂除起居舍人,直前言:「起居注皆據諸處關報,始加修纂,雖有日歷、時政記,亦莫得書。景祐故事,有《邇英延義二閣注記》,凡經筵侍臣
 出處、封章進對、宴會賜予,皆用存記。十年間稍廢不續,陛下言動皆罔聞知,恐非命侍本意。乞令講讀官自今各以日得聖語關送修注官,令講筵所牒報,使謹錄之,因今所御殿名曰《祥曦記注》。」制可。



 三年,遷起居郎,拜中書舍人兼侍讀、直學士院,仍參史事。父忠宣、兄適、遵皆歷此三職,邁又踵之。邁奏:「三省事無鉅細,必先經中書書黃,宰執書押,當制舍人書行,然後過門下,給事中書讀,如給、舍有所建明,則封黃具奏,以聽上旨。惟樞密院
 既得旨,即書黃過門下,例不送中書,謂之『密白』,則封駁之職似有所偏,況今宰相兼樞密,因而厘正,不為有嫌。望詔樞密院。凡已被制敕,並關左右省依三省書黃,以示重出命之意。」報可。



 六年,除知贛州,起學宮,造浮梁,士民安之。郡兵素驕,小不如欲則跋扈,郡歲遣千人戍九江,是歲,或怵以至則留不復返,眾遂反戈。民訛言相驚,百姓恟懼。邁不為動,但遣一校婉說之,俾歸營,眾皆聽,垂橐而入,徐詰什五長兩人,械送潯陽,斬於市。辛卯歲
 饑,贛適中熟,邁移粟濟鄰郡。僚屬有諫止者,邁笑曰:「秦、越瘠肥,臣子義耶?」尋知建寧府。富民有睚眥殺人衷刃篡獄者,久拒捕,邁正其罪,黥流嶺外。



 十一年,知婺州,奏:「金華田多沙,勢不受水,五日不雨則旱,故境內陂湖最當繕治。命耕者出力,田主出谷,凡為公私塘堰及湖,總之為八百三十七所。」婺軍素無律,春給衣,欲以緡易帛,吏不可,則群呼嘯聚於郡將之治,郡將惴恐,姑息如其欲。邁至,眾狃前事,至以飛語榜譙門。邁以計逮捕四十
 有八人,置之理,黨眾相嗾,哄擁邁轎,邁曰:「彼罪人也,汝等何預?」眾逡巡散去。邁戮首惡二人,梟之市,餘黥撻有差,莫敢嘩者。事聞,上語輔臣曰:「不謂書生能臨事達權。」特遷敷文閣待制。



 明年,召對,首論淮東邊備六要地:曰海陵,曰喻洳,曰鹽城,曰寶應,曰清口,曰盱眙。謂宜修城池,嚴屯兵,立游樁,益戍卒。又言:「許浦宜開河三十六里,梅裏鎮宜築二大堰,作斗門,遇行師,則決防送船。」又言:「馮湛創多槳船,底平檣浮,雖尺水可運。今十五六年,修
 葺數少,不足用。」謂宜募瀕海富商入船予爵,招善操舟者以補水軍,上嘉之。以提舉祐神觀兼侍講、同修國史。



 邁初入史館,預修《四朝帝紀》,進敷文閣直學士、直學士院。講讀官宿直,上時召入,談論至夜分。十三年九月,拜翰林學士,遂上《四朝史》,一祖八宗百七十八年為一書。



 紹熙改元,進煥章閣學士、知紹興府。過闕奏事,言新政宜以十漸為戒。上曰:「浙東民困於和市,卿往,為朕正之。」邁再拜曰:「誓盡力。」邁至郡,核實詭戶四萬八千三百有
 奇,所減絹以匹計者,略如其數。提舉玉隆萬壽宮。明年,再上章告老,進龍圖閣學士。尋以端明殿學士致仕,是歲卒,年八十。贈光祿大夫,謚文敏。



 邁兄弟皆以文章取盛名,躋貴顯,邁尤以博洽受知孝宗,謂其文備眾體。邁考閱典故,漁獵經史,極鬼神事物之變,手書《資治通鑒》凡三。有《容齋五筆》、《夷堅志》行於世,其它著述尤多。所修《欽宗紀》多本之孫覿,附耿南仲,惡李綱,所紀多失實,故朱熹舉王允之論,言佞臣不可使執筆,以為不當取覿
 所紀云。



 論曰:孔子云:「使於四方,不辱君命,可謂士矣。」當建炎、紹興之際,凡使金者,如探虎口,能全節而歸,若朱弁、張邵、洪皓其庶幾乎,望之不足議也。皓留北十五年,忠節尤著,高宗謂蘇武不能過,誠哉。然竟以忤秦檜謫死,悲夫!其子適、遵、邁相繼登詞科,文名滿天下,適位極臺輔,而邁文學尤高,立朝議論最多,所謂忠議之報,詎不信夫。



\end{pinyinscope}