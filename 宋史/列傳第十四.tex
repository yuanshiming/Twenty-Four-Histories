\article{列傳第十四}

\begin{pinyinscope}

 郭崇楊廷璋宋偓向拱王彥超張永德王全斌曾孫凱康延澤王繼濤高彥暉附



 郭崇,應州金城人。重厚寡言,有方略。初名崇威,避周祖
 名,止稱崇。父祖俱代北酋長。崇弱冠以勇力應募為卒。後唐清泰中。為應州騎軍都校。



 晉祖割雲應地入為契丹,崇恥事之,奮身南歸,歷鄆、河中、潞三鎮騎軍都校。開運中,戍太原。會漢祖起義,以崇為前鋒。入汴,改護聖左第六軍都校、領郢州刺史,改領富州。



 從周祖平河中,以功遷果州防禦使、領護聖右廂都指揮使。周祖鎮鄴,以崇領行營騎軍兼天雄軍都巡檢使。



 乾祐三年冬,崇從周祖平國難,與李筠拒慕容彥超於劉子陂,走之,以崇
 補侍衛馬軍都指揮使。遣馮道等迎湘陰公斌於徐州,將立之。會契丹南侵,周祖北征,次於澶州,為六軍推戴。樞密使王峻在京師聞變,遣崇率七百騎東拒斌,遇於睢陽。崇陣於牙門外,斌懼,登門樓呼崇曰:「汝等何遽至此?」崇曰:「澶州軍變,遣崇等來衛乘輿,非有他也。」斌召崇升樓,崇未敢登,即遣道下與語,崇乃登,具言軍情有屬,天命已定,斌執崇手泣,俛首久之。俄而斌所領衛兵都校張令超以眾歸崇,斌親將賈、王等數怒目視道,將害
 之。斌曰:「汝輩勿草草,此非關令公事。」崇即送斌就館舍。



 廣順初,領定武軍節度,又為京城都巡檢使、修城都部署兼知步軍公事。未幾,復升陳州為節鎮,以穎州隸焉,命崇為節度。周祖親郊,加同平章事,出鎮澶州。周祖不豫,促還鎮所。



 世宗立,並人侵潞州,命崇與符彥卿出固鎮以御之。世宗親征,又副彥卿為行營都部署。師還,加兼侍中。冬,移真定尹、成德軍世度。四年,世宗征淮南,契丹出騎萬乘餘掠邊,崇率師攻下束鹿縣,斬數百級,俘
 獲甚眾。五年,天清節,崇來朝,表求致政,不允,賜襲衣、金帶、器幣、鞍勒馬,遣之。世宗平關南,至靜安軍,崇來朝。恭帝嗣位,加檢校太師。



 宋初,加兼中書令。崇追感周室恩遇,時復泣下。監軍陳思誨密奏其狀,因言:「常山近邊,崇有異心,宜謹備之。」太祖曰:「我素知崇篤於恩義,蓋有所激發爾。」遣人覘之,還言崇方對賓屬坐池潭小亭飲博,城中晏然。太祖笑曰:「果如聯言。」未幾來朝。時命李重進為平盧軍節度,重進叛,改命崇為節制。乾德三年,卒,年
 五十八。太祖聞之震悼,贈太師。



 子守璘至洛苑副使,妻即明德皇后之姊也。子允恭,以父任授殿直,至崇儀副使、知常州卒。次女為仁宗皇后。天聖三年,詔贈崇尚書令兼中書令,守璘太尉、寧國軍節度,允恭太傅、安德軍節度。六年,又詔追封崇英國公,加贈守璘康清軍節度兼中書令,允恭忠武軍節度兼侍中。允恭子中庸,左侍禁、閣門祗候、副使;中和,娶穎川郡王德彞女,為西染院副使。



 楊廷璋字溫玉,真定人。家世素微賤,有姊寡居京師,周祖微時,欲聘之,姊不從,令媒氏傳言恐逼,姊以告廷璋。廷璋往見周祖,歸謂姊曰:「此人姿貌異常,不可拒。」姊乃從之。



 周祖從漢祖鎮太原,廷璋屢省其姊,周祖愛其純謹。姊卒,留廷璋給事左右。及出討三叛,入平國難,廷璋數獻奇計。即位,追冊廷璋姊為淑妃,擢廷璋為右飛龍使,廷璋固辭不拜,願推恩其父洪裕。即令召洪裕赴闕,以老病辭,就拜金紫光祿大夫、真定少尹。廷璋歷皇城
 使、昭義兵馬都監、澶州巡檢使。



 世宗自澶淵還京,言廷璋有幹材,遷客省使。俄為河陽巡檢、知州事。涇帥史懿稱疾不朝,周祖命廷璋往代之。將行,謂之曰:「懿不就命,即圖之。」廷璋至,屏左右,以詔書示懿,諭以禍福,懿即日載路。俄聞周主崩,廷璋嘔血不食者數日。



 世宗立,拜左驍衛大將軍,充宣徽北院使。征劉崇,以為建雄軍節度。在鎮數年,頗有惠愛。前後率兵入太原境,拔仁義、高壁等砦,獲刺史、軍校數十人,俘其民數千戶,獲兵器羊馬
 數萬計。並人棄沁州二百里,退保新城,廷璋遂置保安、興同、白壁等十餘砦。



 會隰州刺史孫議卒,廷璋遣監軍李謙溥領州事。謙溥至,並人來攻其城,議者以為宜速救之。廷璋曰:「隰州城壁堅完,並人奄至,未能為攻城具,當出奇以破之。」乃募敢死士百餘人,許以重賞,由間道遣人約謙溥為內應。既至,即銜枚夜擊,城中鼓噪以出,並人大潰,追北數十里,斬首千餘級,獲器甲萬計。奏至,世宗喜曰:「吾舅真能禦寇。」詔褒之。



 世宗自河東還,加檢
 校太保。顯德六年夏,率所部入河東界,下堡砦十三,降巡檢使靳漢晁等三人。恭帝即位,加檢校太傅。



 宋初,加檢校太尉。吏民詣闕,請立碑頌功德。太祖命盧多遜撰文賜之。李筠叛,潛遣親信使繼蠟書求援粼境,廷璋獲之,械送京師,因上攻取之策,即下詔委以經略。及車駕親征,詔廷璋率所部入陰地,分賊勢。賊平,歸鎮。是秋來朝,改鎮邠州。乾德四年,移鄜州。開寶二年,召為右千牛衛上將軍。四年,卒,年六十。賻帛二百匹。



 廷璋美髯,長上
 短下,好修容儀,雖見小吏,未嘗懈惰。善待士,幕府多知名人。在晉州日,太祖命荊罕儒為鈐轄。罕儒以廷璋周朝近親,疑有異志,每入府中,從者皆持刀劍,欲圖廷璋。廷璋推誠待之,殊不設備,罕儒亦不敢發,終亦無患。議者以廷璋在涇州保全史懿,陰德之報也。



 洪裕少時,嘗漁於境貂裘陂,忽有馳騎至者,以二石雁授洪裕,一翼掩左,一翼掩右,曰:「吾北嶽使者也。」言訖,忽不見。是年生淑妃,明年生廷璋,家遂昌盛。



 廷璋子七人,皆不為求官,惟
 表其孤甥安崇勛得西頭供奉官。崇勛,後唐樞密使重誨子也。廷璋子坦、塤皆進士及弟。坦至屯田員外郎,鹽鐵副使、判官,塤為都官郎中。



 宋偓河南洛陽人。謙恭下士。祖瑤,唐天德軍節度兼中書令。父廷浩,尚後唐莊宗女義寧公主,生偓。廷浩歷石、原、房三州刺史;晉初,為汜水關使,張從賓之叛,力戰死之。偓年十一,以父死事補殿直,遷供奉官。



 晉祖嘗事莊宗,每偓母入見,詔令勿拜,因從容謂之曰:「朕於主家誠
 無所靳,但朝廷多事,府庫空竭,主所知也。今主居輦下,薪米為憂,當奉主居西洛以就豐泰。」命偓分司就養,敕有司供給,至於醯醢,率有加等。



 漢祖在晉陽,遣其子承訓至洛,奉書偓母,與偓結昏,即永寧公主也。累授北京皇城使。漢乾祐初,拜右金吾衛大將軍、駙馬都尉。隱帝即位,授昭武軍節度,移鎮滑州。



 周祖舉兵向闕,時偓在鎮,開門迎謁,周祖深德之。偓率所部兵從周祖,至劉子陂,隱帝衛兵悉走投周祖。周祖謂偓曰:「至尊危矣,公近
 親,可亟去擁衛,無令驚動。」偓策馬及御營,軍已亂矣。廣順初,丁內艱,服除,授左監門衛上將軍。



 世宗征淮南,令偓與左龍武統軍趙贊、右神武統軍張彥超、前景州刺史劉建於壽州四面巡檢。師還,以偓為右神武統軍,充行營右廂都排陣使,又為廬州城下副部署。吳人大發舟師。次東水布洲,斷蘇、杭之路。世宗遣偓領戰艦數百艘襲之,又遣大將慕容延釗率步騎而進,水陸合勢大破之。



 世宗嘗次於野,有虎逼乘輿,偓引弓射之,一發而斃。
 及江北諸州悉平,畫江為界。世宗駐迎鑾,命偓率舟師三千溯江而上,巡警諸郡。師還,復授滑州節制,又移鎮鄧州。恭帝即位,加開府儀同三司。



 宋初,加檢校太師,遣領舟師巡撫江徼,舒州團練使司超副之。李重進謀以揚州叛,偓察其狀,飛章以聞。太祖令偓屯海陵,以觀重進去就。遂從征揚州,為行營排陣使。及平,以功改保信軍節度。來朝,徙鎮華州。會鑿池都城南,命偓率舟師數千以習水戰,東駕數臨觀焉。五年,改忠武軍節度。



 開寶
 初,太祖納偓長女為後。偓本名延渥,以父名下字從「水」,開寶初,上言改為偓。三年,徙邠州。太平興國初,加同平章事。二年,移定國軍節度。四年,從平太原,又從征幽州。詔偓與尚食使侯昭願領兵萬餘,攻城南面。師還歸鎮。



 五年冬,車駕幸大名,召偓詣行在,詔知滄州。六年,封邢國公。俄遷同州。九年,又為右衛上將軍。雍熙中,曹彬等北伐,班師,命偓知霸州,歸闕。端拱二年,卒,年六十四。廢朝,贈侍中,謚莊惠,中使護葬。



 偓,莊宗之外孫,漢祖之婿,
 女即孝章皇后,近代貴盛,鮮有其比。子元靖至供備庫使,元度至供備庫副使,元載、元亨並至左侍禁、閣門祗候。初,孝章寢疾,語晉國長公主曰:「我瞑目無他憂,惟慮族屬不敦睦,貽笑於人。」景德中,偓幼子元翰果詣京府,求析家財。



 元度子惟簡,為殿直,惟易為奉職。



 向拱字星民,懷州河內人。始名訓,避周恭帝諱改焉。少倜儻負氣。弱冠,聞漢祖在晉陽招致天下士,將往依之。
 中途遇盜,見拱狀貌雄偉,意為富家子,隨之,將劫其財。拱覺,行至石會關,殺所乘驢市酒會里中豪傑,告其故,咸出丁壯護拱至太原。以策干漢祖,漢祖不納,客於周祖門下。及周祖領節鎮,署拱知客押牙。



 周祖即位,授宮苑使。廣順中,遷皇城使,出監昭義屯軍。並人領馬步十五都來侵,拱與巡檢陳思讓逆戰於虎亭南,殺三百餘人,擒百人,獲其帥王璠、曹海金,又敗其軍於壺關。師還,會征慕容彥超,命為都監,賜以六銖、袍帶、鞍勒馬、器仗,
 即日遣行。賊平,命為陜州巡檢。未幾,改客省使、知陜州。



 會延州高允權卒,其寺紹基欲求繼襲,即自領使務。朝廷益禁兵戍守,命拱權知州事,俄遷內客省使。嘗請禁州民賣軍裝兵器於西人,從之。所屬部落有侵盜漢戶者,拱招其酋帥犒之,令誓不敢侵犯。召拜左神武大將軍、宣徽南院使。



 劉崇人寇,遣馬軍樊愛能、步軍何徽赴澤州,令拱監護之。世宗親征,拱以精騎居陣中。高平之捷,以功兼義成軍節度、河東行營前軍都監。師還,出鎮
 陳州。



 先是,晉末,秦州節度何建以秦、成、階三州入蜀,蜀人又取鳳州。至是,宰相王溥薦拱討之,乃召拱與鳳翔王景並率兵出大散關,連下城砦。復命拱為西南面行營都監。蜀入聞鳳州急,發卒五千餘出鳳州北堂倉鎮路,行至黃花穀,將絕周師糧道。拱與王景偵知之,命排陣使張建雄領兵二千直抵黃花穀,又遣別將領勁卒千人出敵後,截其歸路。敵果為建雄所敗,奔堂倉,又為勁卒所逼,合勢掩擊,擒其監軍王巒、孫韜等千五百餘。
 由是劍門之下,州邑營砦,望風宵遁,秦、鳳、階、成平。召歸,宴於金祥殿。賜襲衣、金帶、銀器、繒帛、鞍勒馬。



 顯德二年,世宗親征淮南,以拱權東京留守兼判開封府事。時揚州初平,南唐令境上出師,謀收復。韓令坤有棄城之意,即驛召拱赴行在,拜淮南節度,依前宣徽使兼緣江招討使,以令坤為副。時周師久駐淮陽,都將趙晁、白廷遇等驕恣橫暴,不相稟從,惟務貪濫,至有劫人妻女者。及拱至,戮其不奉法者數輩,軍中肅然。六月,追敘秦、鳳功,
 加檢校太尉。



 時周師圍壽春經年未下,江、淮草寇充斥,吳援兵柵於紫金山,與城中烽火相應。而舒、蘄、和、泰復為吳人所據。拱上言欲且徙揚州之師並力攻壽春,俟其城下,然後改圖進取。世宗從之。拱乃封庫,付揚州主者;復遣本府牙將分部按巡城中。秋毫不犯,軍民感悅。及師行,吳人有負糗糧以送者。至壽春,與李重進合勢以攻其城,改淮南道招討都監,敗淮南軍二千於黃蓍砦。



 世宗再幸壽州,召拱宴賜甚厚,以為武寧軍節度,命
 領其屬駐鎮淮軍。及克壽州,以功加同平章事、領武寧軍節度。四年,徙歸德軍節度。淮南平,改山南東道節度,俄充西南面水陸發運招討使。恭帝即位。加檢校太師、河南尹、西京留守。



 宋初,加兼侍中。太祖征李筠,拱迎謁至汜水,言於上曰:「筠逆節久著,兵力日盛,陛下宜急濟大河,逾太行,乘其未集而誅之,緩則勢張,難為力矣。」帝從其言,卷甲倍道趨之。筠果率兵南向,聞車駕至,惶駭走澤州城守,遂見擒。乾德初,從郊祀畢,封譙國公。



 拱尹
 河南十餘年,專治園林第舍,好聲妓,縱酒為樂,府政廢弛,群盜晝劫。太祖聞之怒,移鎮安州,命左武衛上將軍焦繼勛代之,謂繼勛曰:「洛久不治,選卿代之,無復效拱為也。」



 太平興國初,進封秦國公,來朝,授左衛上將軍。八年,代王彥超判左金吾街仗事。表獻西京長夏門北園,詔以銀五千兩償之。雍熙三年,卒,年七十五。贈中書令。



 咸平初,真宗聞拱之後有寒餒流離者,錄其孫懌為國子助教。拱子德明,至洛苑使;昱,大中祥符八年進士出
 身。德明子悅,為虞部郎中。



 王彥超,大名臨清人。性溫和恭謹,能禮下土。少事後唐魏王繼岌,從繼岌討蜀,還至渭南。會明宗即位,繼岌遇害,左右遁去,彥超乃依鳳翔重雲山僧舍暉道人為徒。暉善觀人,謂彥超曰:「子,富貴人也,安能久居此?」給資帛遣之。



 時晉祖帥陜,乃召至帳下,委以心腹。及移鎮太原,將引兵南下,遣從事桑維翰求援契丹,以彥超從行。天福初,累遷奉德軍校,再轉殿前散指揮都虞候、領蒙州
 刺史。漢初,領岳州防禦使兼護聖左廂都校,出為復州防禦使。



 周祖平內難後,北征契丹,以彥超為行營馬步左廂都排陣使,從周祖入汴。時自彭門迎湘陰公入纘位,會軍變,周祖革命,即命彥超權知徐州節度。未行,湘陰公舊校鞏廷美據州叛,真拜彥超武寧軍節度,命討之。彥超督戰艦破其水砦,乘勝拔之。



 又與樞密使王峻拒劉崇於晉州,彥超以騎兵進,崇遁去,授建雄軍節度。復以所部追賊至霍邑,賊步騎墮崖谷,死者甚眾。彥超
 歸鎮所,俄改河陽三城節度,移鎮河中。



 顯德初,加同平章事。劉崇南寇,命彥超領兵取晉州路東向邀擊,從戰高平。彥超自陰地關與符彥卿會兵圍汾州,諸將請急攻,彥超曰:「城已危矣,旦暮將降,我士卒精銳,儻驅以先登,必死傷者眾,少待之。」翌日,州將董希顏果降。遂引兵趣石州,彥超親鼓士乘城,躬冒矢石,數日下之,擒其守將安彥進,獻行在。師還,改忠武軍節度,加兼侍中。詔率所部浚胡蘆河,城李晏口。工未畢,遼人萬餘騎來侵,彥
 超擊敗之,殺傷甚眾。



 宰相李谷征淮南,以彥超為前軍行營副部署,敗淮南軍二千於壽州城下。吳兵水陸來援,谷退保正陽,吳人躡其後。會李重進兵至,合勢急擊,大敗吳人三萬餘眾,追北二十餘里。還,改京兆尹、永興軍節度。六年夏,移鎮鳳翔。恭帝嗣位,加檢校太師、西面緣邊副都部署。



 宋初,加兼中書令,代還。太祖與彥超有舊,因幸作坊,召從臣宴射,酒酣,謂彥超曰:「卿昔在復州,朕往依卿,何不納我?」彥超降階頓首曰:「勺水豈能止神
 龍耶!當日陛下不留滯於小郡者,蓋天使然爾。」帝大笑。彥超翌日奉表待罪,帝遣中使慰諭,令赴朝謁。



 未幾,復以為永興軍節度。又以其父光祿卿致仕重霸為太子少傅致仕。乾德二年,復鎮鳳翔。三年,丁外艱,起復。開寶二年,為右金吾衛上將軍判街仗事。



 太平興國六年,封邠國公。七年,彥超語人曰:「人臣七十致仕,古之制也。我年六十九,當自知止。」明年,表求致仕,加太子太師,給金吾上將軍祿。彥超既得請,盡斥去僕妾之冗食者,居處
 服用,咸遵儉約。雍熙三年,卒,年七十三。贈尚書令。



 開寶初,彥超自鳳翔來朝,與武行德、郭從義、白重贊、楊廷璋俱侍曲宴。太祖從容謂曰:「卿等皆國家舊臣,久臨劇鎮,王事鞅掌,非朕所以優賢之意。」彥超知旨,即前奏曰:「臣無勛勞,久冒榮寵,今已衰朽,願乞骸骨歸丘園,臣之願也。」行德等竟自陳夙昔戰功及履歷艱苦,帝曰:「此異代事,何足論?」翌日,皆罷行德等節鎮。時議以此許彥超。



 初,彥超將致政,每戒諸子曰:「吾累為統帥,殺人多矣,身死
 得免為幸,必無陰德以及後,汝曹勉為善事以自庇。」及卒,諸子果無達者。宣化門內有大第,園林甚盛,不十餘年,其家已鬻之矣。孫克從,咸平元年進士及第,亦止於州縣。



 張永德字抱一,並州陽曲人。家世饒財。曾祖丕,尚氣節。後唐武皇鎮太原,急於用度,多嚴選富家子掌帑庫。或調度不給,即坐誅,沒入貲寧。丕為之潢滿歲,府財有餘。宗人政當次補其任,率族屬泣拜,請丕濟其急,丕又為代
 掌一年,鄉里服其義。父穎事晉至安州防禦使。



 永德生四歲,母馬氏被出,育於祖母,事繼母劉,以孝聞。周祖初為侍衛吏,與穎善,乃以女妻永德。永德迎其母妻詣宋州。時寇賊充斥,乃易弊衣,毀容儀,居委巷中。有賊過,即邀乞焉,給曰:「此悲田院耳。」賊即舍去,繇是免禍。周祖為樞密使。表永德授供奉官押班。



 乾祐中,命賜潞帥常遇生辰禮幣。遇,周祖之外兄弟也。時周祖鎮鄴,被讒,族其家。永德,在潞州,聞有密詔授遇,永德探知其意,謂遇
 曰:「得非泣殺永德耶?永德即死無怨,恐累君侯家耳。」遇愕然曰:「何謂也?」永德曰:「奸邪蠹政,郭公誓清君側,願且以永德屬吏,事成足以為德,不成死未晚。」遇以為然,止令壯士嚴衛,然所以饋之甚厚。親問之曰:「君視丈人事得成否?」永德曰:「殆必成。」未幾,周祖使至,遇賀且謝曰:「老夫幾誤大事。」



 初,魏人柴翁以經義教裏中,有女,後唐莊宗時備掖庭,明宗入洛,遣出宮。柴翁夫妻往迎之,至鴻溝,遇雨甚,逾旬不能前。女悉取裝具,計直千萬,分其半
 以與父母。令歸魏,曰:「兒見溝旁郵舍隊長,項黵黑為雀形者,極貴人也,願事之。」問之,乃周祖也。父母大愧,然終不能奪。他日,語周祖曰:「君貴不可言,妾有緡錢五百萬資君,時不可失。」周祖因其資,得為軍司。



 柴翁好獨寢,人傳其能司冥間事。一日晨起,大笑不已,妻問之,不對。翁好飲,其妻逼令飲,極醉,因漏言曰:「花項漢作天子矣。」其妻頗露之,遇亦微有聞,未深言。至是,永德故以此諷遇,遇送永德歸周祖。



 周祖登位,封永德妻為晉國公主,授
 永德左衛將軍、內殿直小底四班都知,加駙馬都尉、領和州刺史。逾年,擢為殿前都虞候、領恩州團練使,俄遷殿前都指揮使、泗州防禦使,時年二十四。



 顯德元年,並州劉崇引契丹來侵。世宗親征,戰於高平,大將樊愛能、何徽方戰退衄。時太祖與永德各領牙兵二千,永德部下善左射,太祖與永德厲兵分進,大捷,降崇軍七千餘眾。及駐上黨,世宗晝臥帳中,召永德語曰:「前日高平之戰,主將殊不用命,樊愛能而下,吾將案之以法。」永德曰:「
 陛下欲固守封疆則已,必欲開拓疆宇,威加四海,宜痛懲其失。」世宗擲枕於地,大呼稱善。翌日,誅二將以徇,軍威大振。進攻太原,師薄城下,永德與符彥卿、史彥超北控忻口以斷契丹援路。太原城四十里,周師去城三百步,圍之二匝。自四月至六月,攻之不克。契丹援兵果至,彥超戰沒,繼敗其眾二千,餘眾遁去。以永德領武信軍節度。師還,徙義成軍節度。



 時永德父穎為隸人曹澄等所害,因奔南唐。會議南征,永德請行自效,許之。師至壽
 春,劉仁瞻堅壁不下。永德出疲兵誘之,傍伏精騎,每戰陽不利,北退三十里,伏兵突起夾攻,大敗之,仁贍僅以身免。



 三年,世宗親征,至壽州城下,仁贍執澄等三人檻送行在,意求緩師,詔賜永德,俾其甘心。太祖與永德領前軍至紫金山,吳人列十八砦,戰備嚴整。敵壘西偏有高隴,下瞰其營中,永德選勁弓強弩伏隴旁,太祖麾兵直攻第一砦,戰陽不勝,淮人果空砦出鬥,永德亟登隴,發伏馳入據之,敵眾散走。翌日,又攻第二砦,鼓噪而進,
 始攻北門,淮人開南門而遁。時韓令坤在揚州。復為吳人所逼,欲退師。世宗怒,遣永德率師援之,又敗泗州軍千餘於曲溪堰,俄屯下蔡。



 時吳人以周師在壽春攻圍日急,又恃水戰,乃大發樓般蔽江而下,泊於濠、泗,周師頗不利。吳將林仁肇帥眾千餘,水陸齊進,又以船數艘載薪,乘風縱火,將焚周浮梁,周人憂之。俄而風反,吳人稍卻,永德進兵敗之。又夜使習水者沒其船下,縻以鐵金巢,引輕舠急擊。吳人既不得進,溺者甚眾,奪其巨艦數
 十艘。永德解金帶,賞習水者。乃距浮梁十餘步,以鐵索千餘尺橫截長淮,又維巨木,自是備禦益堅矣。俄又敗千餘眾於淮北岸,獲戰船數十艘,吳人多溺死。詔褒美之。



 冬,擢為殿前都點檢。四年,從克壽州還,制授檢校太尉、領鎮寧軍節度。五年夏,契丹擾邊,命永德率步騎二萬拒之。從世宗北伐,還駐澶淵,解兵柄,加檢校太尉、同中書門下平章事。恭帝嗣位,移忠武軍節度。



 太祖即位,加兼侍中。永德入朝,授武勝軍節度。入覲,召對後苑,道
 舊故,飲以巨觥,每呼駙馬不名。時並、汾未下,太祖密訪其策。永德曰:「太原兵少而悍,加以契丹為援,未易取也。臣以每歲多設游兵,擾其農事,仍發間使以諜契丹,絕其援,然後可下也。」帝然之。俄歸本鎮。



 會出師討金陵,永德以己資造戰船數十艘,運糧萬斛,自順陽沿漢水而下。富民高進者,豪橫莫能禁,永德乃發其奸,置於法。進潛詣闕,誣永德緣險固置十餘砦,圖為不軌。太祖命樞密都承旨曹翰領騎兵察之,詰其砦所,進曰:「張侍中誅
 我宗黨殆盡,希中以法,報私憤爾。」翰以進授永德,永德遽解縛就市,笞而釋之。時稱其長者。



 太平興國二年來朝,拜左衛上將軍。五年,坐市秦、隴竹木所過矯制免關市算,降為本衛大將軍。數月,復舊秩。六年,進封鄧國公。雍熙中,連知滄、雄、定三州。



 端拱元年,拜安化軍節度。召還,為河北兩路排陣使,屯定州。嘗與契丹戰,斬獲甚眾。二年,丁內艱,起復。淳化初,又代田重進知鎮州。二年,改泰寧軍節度兼侍中,出判並州兼並代都部署。



 永德明
 天文術,嘗與僚佐會食,有報遼兵寇州境者,永德用《太白萬勝訣》占之,語坐客曰:「彼雖以年月便利,乘金而來,反值歲星對逆,兵家大忌,必敗。」未幾,折御卿捷報至,眾始歡伏。



 自五代用兵,多姑息,藩鎮頗恣部下販鬻。宋初,功臣猶習舊事。太宗初即位,詔群臣乘傳出入,不得繼貨邀利,及令人諸處圖回,與民爭利。永德在太原,嘗令親吏販茶規利,闌出徼外市羊,為轉運使王嗣宗所發,罷為左衛上將軍。



 真宗即位,進封衛國公。未幾,判左金
 吾街仗事。咸平初,屢表請老,授太子太師,分司西京,仍以其孫大理寺丞文蔚厘務洛下,以便就養。



 二年冬,契丹入邊,帝將北巡,以永德宿將,召入對便殿,賜坐,訪以邊要。以老不可從行,留為東京內外都巡檢使。三年,制授檢校太師、彰德軍節度、知天雄軍。俄以衰耄,命還本鎮。是秋卒,年七十三。遣內園使馮守規護柩還京師、贈中書令。諸孫遷秩者五人。



 永德出母,後適安邑劉祚。及永德鎮南陽,祚已卒,迎母歸州廨,起二堂,與繼母劉並
 居。劉卒,馬預中參,時年八十一,太宗勞之,賜冠帔,封莒國太夫人。同母弟劉再思,署子城使,於市西里起大第,聚劉族。



 初,永德寓睢陽,有書生鄰居臥疾,永德療之獲愈。生一日就永德求汞五兩,既得,即置鼎中煮之,成中金。自是日與永德游,一日,告適淮上,語永德曰:「後當相遇於彼。」永德曰:「吳境不通,子何可去?」生曰:「吾自有術。」永德送行數舍,懇求藥法,生曰:「君當大貴,吾不吝此,慮損君福。」言訖而去。及永德屯下蔡,牙帳前後隊部曲八百
 人,皆金銀刀槊,繡旗幟。永德善騎射,左右分掛十的,握十矢,疾馳互發,發必中。淮民環觀,有一僧睥睨,永德遽召之,乃睢陽書生也。夜宿帳中,復求汞法。僧曰:「始語君貴,今不謬矣。終能謹節,當保五十年富貴,安用此為?然能降志禮賢,當別有授公藥法者。」永德由此益罄家資,延致方士,故太祖以方外待之。



 初,睢陽書生嘗言太祖受命之兆,以故永德潛意拱響。太祖將聘孝明皇后也,永德出緡錢金帛數千以助之,故盡太祖朝而恩渥不
 替。



 孫文蔚虞部員外郎,文炳殿中丞。



 王全斌,並州太原人。其父事莊宗,為岢嵐軍使,私畜勇士餘人,莊宗疑其有異志。召之,懼不敢行。全斌時年十二,謂其父曰:「此蓋疑大人有他圖,願以全斌為質,必得釋。」父從其計,果獲全,因以隸帳下。



 及莊宗入洛,累歷內職。同光末,國有內難,兵入宮城,近臣宿將皆棄甲遁去,惟全斌與符彥卿等十數人居中拒戰。莊宗中流矢,扶掖至絳霄殿,全斌慟哭而去。明宗即位,補禁軍列校。
 晉初,從侯益破張從賓於汜水,以功遷護聖指揮使。周廣順初,改護聖為龍捷,以全斌為右廂都指揮使。及討慕容彥超於兗州,為行營馬步都校。顯德中,從向訓平秦、鳳,遂領恩州團練使。俄遷領泗州防禦使。從世宗平淮南,復瓦橋關,改相州留後。



 宋初,李筠以潞州叛,全斌與慕容延釗由東路會大軍進討,以功拜安國軍節度。詔令完葺西山堡砦,不逾時而就。建隆四年,與洺州防禦使郭進等率兵入太原境,俘數千人以歸,進克樂平。



 乾德二年冬,又為忠武軍節度。即日下詔伐蜀,命全斌為西川行營前軍都部署,率禁軍步騎二萬、諸州兵萬人由鳳州路進討。召示川峽地圖,授以方略。



 十二月,率兵拔幹渠渡、萬仞燕子二砦,遂下興州,蜀刺史藍思綰退保西縣。敗蜀軍七千人,獲軍糧四十餘萬斛。進拔石圌、魚關、白水二十餘砦,先鋒史延德進軍三泉,敗蜀軍靈數萬,擒招討使韓保正、副使李進,獲糧三十餘萬斛。既而崔彥進、康延澤等逐蜀軍過三泉,遂至嘉陵,殺虜甚
 眾。蜀人斷閣道,軍不能進,全斌議取羅川路以入,延澤潛謂彥進曰:「羅川路險,軍難並進,不如分兵治閣道,與大軍會於深渡。」彥進以白全斌,全斌然之。命彥進、延澤督治閣道,數日成,遂進擊金山砦,破小漫天砦。全斌由羅川趣深渡,與彥進會。蜀人依江列陣以待,彥進遣張萬友等奪其橋。會暮夜,蜀人退保大漫天砦。詰朝,彥進、延澤、萬友分三道擊之,蜀人悉其精銳來逆戰,又大破之,乘勝拔其砦,蜀將王審超、監軍崇渥遁去,復與
 三泉監軍劉延祚、大將王昭遠、趙崇韜引兵來戰,三戰三敗,追至利州北。昭遠遁去,渡桔柏江,焚梁,退守劍門。遂克利州,得軍糧八十萬斛。



 自利州趨劍門,次益光。全斌會諸將議曰:「劍門天險,古稱一夫荷戈,萬夫莫前,諸君宜各陳進取之策。」待衛軍頭向韜曰:「降卒牟進言:『益光江東,越大山數重,有狹徑名來蘇,蜀人於江西置砦,對岸有渡,自此出劍關南二十里,至清強店,與大路合。可於此進兵,即劍門不足恃也。』」全斌等即欲卷甲赴之,
 康延澤曰:「來蘇細徑,不須主帥親往。且人屢敗,並兵退守劍門,若諸帥協力進攻,命一偏將趨來蘇,若達清強,北擊劍關與大軍夾攻,破之必矣。」全斌納其策,命史延德分兵趨來蘇,造浮梁於江上,蜀人見梁成,棄砦而遁。昭遠聞延德兵趨來蘇,至清強,即引兵退,陣於漢源坡,留其偏將守劍門。全斌等擊破之,昭遠、崇韜皆遁走,遣輕騎進獲,傳送闕下,遂克劍州,殺蜀軍萬餘人。



 四年正月十三日,師次魏城,孟昶遣使奉表來降,全斌等
 入成都。旬餘,劉廷讓等始自峽路至。昶饋遺廷讓等及犒師,並同全斌之至。及詔書頒賞,諸軍亦無差降。由是兩路兵相嫉,蜀人亦構,主帥遂不協。全斌等先受詔,每制置必須諸將僉議,至是,雖小事不能即決。



 俄詔發蜀兵赴闕,人給錢十千,未行者,加兩月廩食。全斌等不即奉命,由是蜀軍憤怨,人人思亂。兩路隨軍使臣常數十百人,全斌、彥進及王仁贍等各保庇之,不令部送蜀兵,但分遣諸州牙校。蜀軍至綿州果叛,劫屬邑,眾至十餘
 萬,自號「興國軍」。有蜀文州刺史全師雄者,嘗為將,有威惠,士卒畏服。適以其族赴闕下。綿州遇亂,師雄恐為所脅,乃匿其家於江曲民舍。後數日為亂兵所獲,推為主帥。



 全斌遣都監米光緒往招撫之,光緒盡滅師雄之族,納其愛女及橐裝。師雄聞之,遂無歸志,率眾急攻綿州,為橫海指揮使劉福、龍捷指揮使田紹斌所敗;遂攻彭州,逐刺史王繼濤,殺都監李德榮,據其城。成都十縣皆起兵應師雄,師雄自號「興蜀大王」,開幕府,置僚屬,署節
 帥二十餘人,令分據灌口、導江、郫、新繁、青城等縣。彥進與張萬友、高彥暉、田欽祚同討之,為師雄所敗,彥暉戰死,欽祚僅免,賊眾益盛。全斌又遣張廷翰、張煦往擊之,不利,退入成都。師雄分兵綿、漢間,斷閣道,緣江置砦,聲言欲攻成都。自是,邛、蜀、眉、雅、東川、果、遂、渝、合、資、簡、昌、普、嘉、戎、榮、陵十七州,並隨師雄為亂。郵傳不通者月餘,全斌等甚懼。時城中蜀兵尚餘二萬,全斌慮其應賊,與諸將謀,誘致夾城中,盡殺之。



 未幾,劉廷讓、曹彬破師雄之
 眾於新繁,俘萬餘人。師雄退保郫縣,全斌、仁贍又攻破之。師雄走保灌口砦。賊勢既衄,餘黨散保州縣。有陵州指揮使元裕者,師雄署為刺史,眾萬餘,仁贍生擒之,磔於成都市。



 俄虎捷指揮使呂翰為主將所不禮,因殺知嘉州客省使武懷節、戰棹都監劉漢卿,與師雄黨劉澤合,眾至五萬,逐普州刺史劉楚信,殺通判劉沂及虎捷都校馮紹。又果州指揮使宋德威殺知州八作使王永昌及通判劉渙、都監鄭光弼,遂州牙校王可鐐率州民
 為亂。仁贍等討呂翰於嘉州,翰敗走入雅州。師雄病死於金堂,推謝行本為主,羅七君為佐國令公,與賊將宋德威、唐陶鱉據銅山,旋為康延澤所破。仁贍又敗呂翰於雅州,翰走黎州,為下所殺,棄尸水中。後丁德裕等分兵招輯,賊眾始息。



 全斌之入蜀也,適屬冬暮,京城大雪,太祖設氈帷於講武殿,衣紫貂裘帽以視事,忽謂左右曰:「我被服若此,體尚覺寒,念西征將沖犯霜雪,何以堪處!」即解裘帽,遣中黃門馳賜全斌,仍諭諸將,以不遍及
 也。全斌拜賜感泣。



 初,成都平,命參知政事呂餘慶知府事,全斌但典軍旅。全斌嘗語所親曰:「我聞古之將帥,多不能保全功名,今西蜀既平,欲稱疾東歸,庶免悔吝。」或曰:「今寇盜尚多,非有詔旨,不可輕去。」全斌猶豫未決。



 會有訴全斌及彥進破蜀日,奪民家子女玉帛不法等事,與諸將同時召還。太祖以全斌等初立功,雖犯法,不欲辱以獄吏,但令中書問狀,全斌等具伏。詔曰:「王全斌、王仁贍、崔彥進等被堅執銳,出征全蜀,彼畏威而納款,尋
 馳詔以申恩。用示哀矜,務敦綏撫,應孟昶宗族、官吏、將卒、士民悉令安存,無或驚擾;而乃違戾約束,侵侮憲章,專殺降兵,擅開公帑,豪奪婦女,廣納貨財,斂萬民之怨嗟,致群盜之充斥。以至再勞調發,方獲平寧。洎命旋歸,尚欲含忍,而銜冤之訴,日擁國門,稱其隱沒金銀、犀玉、錢帛十六萬七百餘貫。又擅開豐德庫,致失錢二十八萬一千餘貫。遂令中書門下召與訟者質證其事。而全斌等皆引伏。其令御史臺於朝堂集文武百官議其罪。」



 於是百官定議,全斌等罪當大闢,請準律處分。乃下詔曰:「有征無戰,雖舉於王師;禁暴戢兵,當崇於武德。蠢茲庸蜀,自敗奸謀,爰伐罪以宣威,俄望風而歸命。遽令按堵,勿犯秋毫,庶德澤之涵濡,俾生聚之寧息。而忠武軍節度王全斌、武信軍節度崔彥進董茲銳旅,奉我成謀,既居克定之全功,宜體輯柔之深意。比謂不日清謐,實時凱旋,懋賞策勛,抑有彞典。而罔思寅畏,速此悔尤,貪殘無厭,殺戮非罪,稽於偃革,職爾玩兵。尚念前勞,特從
 寬貸,止停旄鉞,猶委藩宣。我非無恩,爾當自省。全斌可責授崇義軍節度觀察留後,彥進可責授昭化軍節度觀察留後,特建隨州為崇義軍、金州為昭化軍以處之。仁贍責授右衛大將軍。」開寶開,車駕幸洛陽郊祀,召全斌侍祠,以為武寧軍節度。謂之曰:「朕以江左未平,慮征南諸將不尊紀律,故抑卿數年,為朕立法。今已克金陵,還卿節鉞。」仍以銀器萬兩、帛萬匹、錢千萬賜之。全斌至鎮數月卒,年六十九。贈中書令。天禧二年,錄其孫永昌
 為三班奉職。



 全斌輕財重士,不求聲譽,寬厚容眾,軍旅樂為之用。黜居山郡十餘年,怡然自得,識者稱之」



 子審鈞,崇儀使、富州刺史、廣州兵馬鈐轄;審銳,供奉官、合門祗候。曾孫凱。



 凱字勝之。祖審鈞,嘗為永興軍駐泊都監,以擊賊死,遂家京兆。饒於財,凱散施結客,日馳獵南山下,以踐蹂民田,捕至府。時寇準守長安,見其狀貌奇之。為言:「全斌取蜀有勞,而審鈞以忠義死,當錄其孤。」遂以為三班奉職、
 監鳳翔盩厔稅。歷左右班殿直、監益州市買院、慶州合水鎮兵馬監押、監在京草場。



 先是,守卒掃遺稈自入,凱禁絕,而從欲害之。事覺,他監官皆坐故縱,凱獨得免。自右侍禁、雄州兵馬監押,擢合門祗候、定邢趙都巡檢使。



 元昊反,徙麟州都監。嘗出雙烽橋、染枝谷,遇夏人,破之。又破龐青、黃羅部,再戰於伺候烽,前後斬首三百餘級,獲區落馬牛、橐駝、器械以數千計。夏人圍麟州,乘城拒鬥,晝夜三十一日,始解去。特遷西頭供奉官。



 代遷,邊寇
 猶鈔掠,以為內殿崇班、麟州路緣邊都巡檢使,與同巡檢張岊護糧道於青眉浪,寇猝大至,與岊相失。乃分兵出其後夾擊之。復與岊合,斬首百餘級。又入兔毛川,賊眾三萬,凱以後六千陷圍,流矢中面,鬥不解,又斬首百餘級,賊自蹂踐,死者以千數。遷南作坊副使,後為並、代州鈐轄,管勾麟府軍馬事。夏人二萬寇青塞堡,凱出鞋邪谷,轉戰四十里,至杜□古川,大敗之,復得所掠馬牛以還。



 經略使明鎬言凱在河外九年,有功,遂領資州刺史。
 久之召還,未及見,會甘陵盜起,即命領兵赴城下。賊平,拜澤州刺史、知邠州。未幾,為神龍衛四廂都指揮使、澤州團練使,歷環慶、並代、定州路副都總管,捧日天武四廂、綿州防禦使,累遷侍衛親軍步軍副都指揮使、涇州觀察使。又徙秦鳳路,辭日,帝諭以唃氏木征,交易阻絕,頗有入寇之萌,宜安靜以處之。凱至,與主帥以恩信撫接,遂復常貢。召拜武勝軍節度觀察留後、侍衛親軍馬軍副都指揮使。卒,年六十六。贈彰武軍節度使,謚莊恪。



 凱治軍有紀律,善撫循士卒,平居與均飲食,至臨陣援枹鼓,毅然不少假。故士卒畏信,戰無不力,前後與敵遇,未嘗挫衄。兔毛川之戰,內侍宋永誠哭於軍中,凱劾罷之。尤篤好於故舊。



 子緘。緘子詵,字晉卿,能詩善畫,尚蜀國長公主,官至留後。



 康延澤,父福,晉護國軍節度兼侍中。延澤,天福中,以蔭補供奉官。周廣順二年,永興李洪信入覲,遣延澤往巡檢,遷內染院副使。



 宋初,從慕容延釗、李處耘平湖湘。時
 荊南高保融卒,其子繼沖嗣領軍事,命延澤繼書幣先往撫之。且察其情偽。及還,盡得其機事,因前導大軍入境,遂下荊峽。以勞授正使。



 乾德中,征蜀,為鳳州路馬軍都監,破白水、合子二砦,進擊西縣、三泉,獲韓保正。由來蘇路會大軍,克劍門。及孟昶降,延澤以百騎先入成都,安撫軍民,盡封府庫而還。就命為成都府都監。會全師雄復亂,徙為普州刺史。時有降兵二萬七千,諸將懼為內應,欲盡殺之。延澤請簡老幼疾病七千人釋之,餘以
 兵衛還,浮江而下,賊若來劫奪,即殺之未晚。諸將不能用。俄出兵。敗賊黨劉澤三萬人。復有王可鐐率數郡賊兵來戰,延澤擊走之,追北至合州。又破可鐐餘黨謝行本等,擒羅七君。事平,優詔嘉獎,就命為東川七州招安巡檢使。



 全斌等得罪,延澤亦坐貶唐州教練使。開寶中,起為供奉官,遷左藏庫副使。坐與諸侄爭家財失官,居西洛卒。



 兄延沼,幼隸後唐明宗帳下。仕晉祖,為尚食使,改散指揮使都虞候、興聖軍都指揮使,出為隨、澤二州
 刺史。



 周祖北征,延沼與白文遇、李彥崇、曹奉金並從。廣順中,為侍衛馬步軍都頭、領信州刺史。從世宗征劉崇,率兵攻遼州,轉龍捷右廂都校、領岳州防禦使,真拜蔡齊鄭楚四州防禦使、晉潞二州兵馬鈐轄。



 宋初,李重進叛,以延沼為前軍馬軍都指揮使。建隆四年,改懷州防禦使。乾德六年,命李繼勛等征河東,以延沼為先鋒都監。太祖親征太原,以延沼宿將,熟練邊事,詔領兵屯潞州,會以疾歸郡。開寶二年,卒,年五十八。



 王繼濤,河朔人,少給事漢祖左右。乾祐初,補供奉官,歷諸司副使。仕周,為右武衛大將軍。淮南平,為天長軍使。顯德五年,遷和州刺史。



 宋初,為左驍騎大將軍,再遷左神武大將軍,乾德二年,命護徒治安陵隧道。



 大軍伐蜀,為鳳州路砦使。興元降,王全斌命繼濤權府事。孟昶降,全斌又遣繼濤與供奉官王守訥部送昶歸闕。守訥白全斌,言繼濤問昶求宮妓、金帛,全斌遂留繼濤,止令守訥送昶俄詔以繼濤為彭州刺史。



 綿州軍亂,劫全師
 雄為帥,率眾攻彭州,繼濤與都監李德榮拒之,德榮戰死,繼濤身被八槍,單騎走至成都。



 素與通事舍人田欽祚有隙,會欽祚入朝,乃誣奏繼濤以他事。太祖驛召繼濤,將面質之,道病卒。詔曰:「故彭州刺史王繼濤,先登擊賊,身被重創,優典未加,繼志而歿。故階州刺史高彥暉,帥師討賊,奮不顧命,垂老之年,殞身鋒鏑。永言痛悼,不忘於懷。宜各賜其家粟帛。」



 高彥暉,薊州漁陽人。仁契丹為瀛州守將。世宗北征,以
 城來降,遷耀、階二州刺史。



 王師伐蜀,為歸州路先鋒都指揮使。全師雄之亂,崔彥進遣彥暉與田欽祚共討之。至導江,與賊遇,賊據隘路,設伏竹箐中,官軍至,遇伏發,遂不利。彥暉謂欽祚曰:「賊勢張大,日將暮,請收兵,詰朝與戰。」欽祚欲遁,慮賊曳其後,乃紿之曰:「公食厚祿,遇賊畏縮,何也?」彥暉復麾兵進。欽祚潛遁去。彥暉獨與部下十餘騎力戰,皆死之,時年七十餘。



 彥暉老將,練習邊事,上聞其歿,甚痛惜,故並命優恤之。



 論曰:「郭崇感激昔遇,發於垂涕。太祖察其忠厚,亟焚思誨之奏。雖魏文不強於楊彪,宋武無猜於徐廣,何以加之。廷璋開懷以待罕孺,宋偓抗章以察重進,向拱獻謀以平上黨,乘時建功,各奮所長,有足尚者。王彥超起自戎昭,歷典藩服,引年高蹈,武夫之貞;至於自悔多殺,垂戒後裔,近乎仁人之用心。張永德前朝勛伐,夙識太祖,潛懷尊奉,雖有橋公祖之知,而非人臣之不二心者矣。乾德伐蜀之師,未七旬而降款至,諸將之功,何可泯也。
 王全斌黷貨殺降,尋啟禍變,太祖罪之,而從八議之貸,斯得馭功臣之道。延澤能相地險,豫謀屯備。繼濤、彥暉,先登重傷,殞沒無避,咸可稱焉



\end{pinyinscope}