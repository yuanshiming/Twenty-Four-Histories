\article{列傳第一百八十一}

\begin{pinyinscope}

 林勳,賀州人。政和五年進士,為廣州教授。建炎三年八
 月,獻《本政書》十三篇,言:「國家兵農之政,率因唐末之故。今農貧而多失職,兵驕而不可用,是以饑民竄卒,類為盜賊。宜仿古井田之制,使民一夫占田五十畝,其有羨田之家,毋得市田,其無田與游惰末作者,皆驅之使為隸農,以耕田之羨者,而雜紐錢穀,以為十一之稅。宋二稅之數,視唐增至七倍。今本政之制,每十六夫為一井,提封百里,為三千四百井,率稅米五萬一千斛、錢萬二千緡;每井賦二兵、馬一匹,率為兵六千八百人、馬三千
 四百匹,歲取五之一以為上番之額,以給征役。無事則又分為四番,以直官府,以給守衛。是民凡三十五年而役使一遍也。悉上則歲食米萬九千餘斛,錢三千六百餘緡,無事則減四分之三,皆以一同之租稅供之。匹婦之貢,絹三尺,綿一兩。百里之縣,歲收絹四千餘匹,綿三千四百斤。非蠶鄉則布六尺、麻二兩,所收視絹綿率倍之。行之十年,則民之口算,官之酒酤,與凡茶、鹽、香、礬之榷,皆可弛以予民。」



 其說甚備。書奏,以勳為桂州節度掌
 書記。



 其後,勳又獻《比較書》二篇,大略謂:「桂州地東西六百里,南北五百里,以古尺計之,為方百里之國四十,當墾田二百二十五萬二千八百頃,有田夫二百四萬八千,出米二十四萬八千斛,祿卿大夫以下四千人,祿兵三十萬人。今桂州墾田約萬四十二頃,丁二十一萬六千六百一十五,稅錢萬五千餘緡,苗米五萬二百斛有奇,州縣官不滿百員,官兵五千一百人。蓋土地荒蕪而遊手末作之人眾,是以地利多遺,財用不足,皆本政不
 修之故。」



 朱熹甚愛其書。東陽陳亮曰:「勳為此書,考古驗今,思慮周密,可謂勤矣。世之為井地之學者,孰有加於勳者乎?要必有英雄特起之君,用於一變之後,成順致利,則民不駭而可以善其後矣。」



 劉才邵,字美中,吉州廬陵人。其上世鶚,太宗召見,未及用而卒。嘗憤五季文辭卑弱,仿楊雄《法言》,著《法語》八十一篇行於世。才邵以大觀二年上舍釋褐,為贛、汝二州教授,復為湖北提舉學事管幹文字。宣和二年,中宏詞
 科,遷司農寺丞。靖康元年,遷校書郎。



 高宗即位,以親老歸侍,居閑十年。御史中丞廖剛薦之,召見,遷祕書丞,歷駕部員外郎,遷吏部員外郎,典侍右選事。先是,宗室注宮觀、岳廟,例須赴部,遠者或難於行。才邵言許經所屬以聞於部,依條注擬,行之而便。遷軍器監,既而遷起居舍人,未幾,為中書舍人兼權直學士院。帝稱其能文,時宰忌之,出知漳州。即城東開渠十有四,為閘與斗門以瀦匯決,溉田數千畝。民甚德之。兩奉祠。紹興二十五年,
 召拜工部侍郎兼直學士院,尋權吏部尚書。以疾請祠,加顯謨閣直學士。卒,贈通奉大夫。



 才邵氣和貌恭,方權臣用事之時,雍容遜避,以保名節。所著《檆溪居士集》行世。



 許忻,拱州人。宣和三年進士,高宗時,為吏部員外郎,有旨引見。是時,金國使人張通古在館,忻上疏極論和議不便,曰:「臣兩蒙召見,擢置文館,今茲復降睿旨引對。今見陛下於多故之時,欲采千慮一得之說以廣聰明,是
 臣圖報萬分之秋也,故敢竭愚而效忠。



 臣聞金使之來,陛下以祖宗陵寢廢祀,徽宗皇帝、顯肅皇后梓宮在遠,母后春秋已高,久闕晨昏之奉,淵聖皇帝與天族還歸無期,欲屈己以就和,遣使報聘。茲事體大,固已詔侍從、臺諫各具所見聞矣,不知侍從臺諫皆以為可乎?抑亦可否雜進,而陛下未有所擇乎?抑亦金已恭順,不復邀我以難行之禮乎?是數者,臣所不得而聞也。請試別白利害,為陛下詳陳之。



 夫金人始入寇也,固嘗云講和矣。
 靖康之初,約肅王至大河而返,已而挾之北行,訖無音耗。河朔千里,焚掠無遺,老稚係累而死者億萬計,復破威勝、隆德等州。淵聖皇帝嘗降詔書,謂金人渝盟,必不可守。是歲又復深入,朝廷制置失宜,都城遂陷。敵情狡甚,懼我百萬之眾必以死爭也,止我諸道勤王之師,則又曰講和矣。乃邀淵聖出郊,次邀徽宗繼往,追取宗族,殆無虛日,傾竭府庫,靡有孑遺,公卿大臣類皆拘執,然後偽立張邦昌而去。則是金人所謂講和者,果可信乎?



 此已然之禍,陛下所親見。今徒以王倫繆悠之說,遂誘致金人責我以必不可行之禮,而陛下遂已屈己從之,臣是以不覺涕泗之橫流也。而彼以『詔諭江南』為名而來,則是飛尺書而下本朝,豈講和之謂哉?我躬受之,真為臣妾矣。陛下方寢苫枕塊,其忍下穹廬之拜乎?臣竊料陛下必不忍為也。萬一奉其詔令,則將變置吾之大臣,分部吾之諸將,邀求無厭,靡有窮極。當此之時,陛下欲從之則無以立國,不從之則復責我以違令,其何以
 自處乎?況犬羊之群,驚動我陵寢,戕毀我宗廟,劫遷我二帝,據守我祖宗之地,塗炭我祖宗之民,而又徽宗皇帝、顯肅皇后鑾輿不返,遂致萬國痛心,是謂不共戴天之仇。彼意我之必復此仇也,未嘗頃刻而忘圖我,豈一王倫能平哉?方王倫之為此行也,雖閭巷之人,亦知其取笑外夷,為國生事。今無故誘狂敵悖慢如此,若猶倚信其說而不寢,誠可慟哭,使賈誼復生,謂國有人乎哉,無人乎哉?



 古之外夷,固有不得已而事之以皮幣、事之
 以珠玉、事之以犬馬者,曷嘗有受其詔,惟外夷之欲是從,如今日事哉?脫或包羞忍恥,受其詔諭,而彼所以許我者不復如約,則徒受莫大之辱,貽萬世之譏;縱使如約,則是我今日所有土地,先拱手而奉外夷矣,祖宗在天之靈,以謂如何?徽宗皇帝、顯肅皇后不共戴天之仇,遂不可復也,豈不能痛哉!陛下其審思之,斷非聖心所能安也。自金使入境以來,內外惶惑,儻或陛下終以王倫之說為不妄,金人之詔為可從,臣恐不惟墮外夷之姦
 計,而意外之虞,將有不可勝言者矣。此眾所共曉,陛下亦嘗慮及於此乎?



 國家兩嘗敗外夷於淮甸,雖未能克復中原之地,而大江之南亦足支吾。軍聲粗震,國勢粗定,故金人因王倫之往復,遣使來嘗試朝廷。我若從其所請,正墮計中;不從其欲,且厚攜我之金幣而去,亦何適而非彼之利哉!為今之計,獨有陛下幡然改慮,布告中外,以收人心,謂祖宗陵寢廢祀,徽宗皇帝、顯肅皇后梓宮在遠,母后、淵聖、宗枝族屬未還,故遣使迎請,冀遂
 南歸。今敵之來,邀朝廷以必不可從之禮,實王倫賣國之罪,當行誅責,以釋天下之疑。然後激厲諸將,謹捍邊陲,無墮敵計,進用忠正,黜遠姦邪,以振紀綱,以修政事,務為實效,不事虛名,夕慮朝謀,以圖興復,庶乎可矣。



 今金使雖已就館,謂當別議區處之宜。臣聞萬人所聚,必有公言。今在廷百執事之臣,與中外一心,皆以金人之詔為不可從,公言如此,陛下獨不察乎?若夫謂粘罕之已死,外夷內亂,契丹林牙復立,故今金主復與我平等
 語,是皆行詐款我師之計,非臣所敢知也。或者又謂金使在館,今稍恭順。如臣之所聞,又何其悖慢於前,而遽設恭順於後?敵情變詐百出,豈宜惟聽其甘言,遂忘備預之深計,待其禍亂之已至,又無所及?此誠切於事情。今日之舉,存亡所繫,愚衷感發,不能自己,望鑒其惓惓之忠,特垂采納,更與三二大臣熟議其便,無貽異時之悔,社稷天下幸甚。」



 疏入,不省。後忻託故乞從外補,乃授荊湖南路轉運判官。謫居撫州,起知邵陽,卒。



 應孟明,字仲實,婺州永康人。少入太學,登隆興元年進士第。試中教官,調臨安府教授,繼為浙東安撫司幹官、樂平縣丞。侍御史葛邲、監察御史王藺薦為詳定一司敕令所刪定官。



 輪對,首論:「南北通好,疆場無虞,當選將練兵,常如大敵之在境,而可以一日忽乎?貪殘苛酷之吏未去,吾民得無不安其生者乎?賢士匿於下僚,忠言壅於上聞,無乃眾正之門未盡開,而兼聽之意未盡孚乎?君臣之間,戒懼而不自持,勤勞而不自寧,進君子,退
 小人,以民隱為憂,以邊陲為警,則政治自修,紀綱自張矣。」孝宗曰:「朕早夜戒懼,無頃刻忘,退朝之暇,亦無它好,正恐臨朝或稍晏,則萬幾之曠自此始矣。」次乞申嚴監司庇貪吏之禁,薦舉徇私情之禁,帝嘉獎久之。它日,宰相進擬,帝出片紙於掌中,書二人姓名,曰:「卿何故不及此?」其一則孟明也。乃拜大理寺丞。



 故大將李顯忠之子家僮溺死,有司誣以殺人,逮繫幾三百家。孟明察其冤,白於長官,釋之。出為福建提舉常平,陛辭,帝曰:「朕知卿
 愛百姓,惡贓吏,事有不便於民,宜悉意以聞。」因問當世人才,孟明對曰:「有才而不學,則流為刻薄,惟上之教化明,取舍正,使回心向道,則成就必倍於人。」帝曰:「誠為人上者之責。」孟明至部,具以臨遣之意咨訪之。帝一日御經筵,因論監司按察,顧謂講讀官曰:「朕近日得數人,應孟明,其最也。」尋除浙東提點刑獄,以鄉部引嫌,改使江東。



 會廣西謀帥,帝謂輔臣曰:「朕熟思之,無易應孟明者。」即以手筆賜孟明曰:「朕聞廣西鹽法利害相半,卿到任,
 自可詳究事實。」進直祕閣、知靜江府兼廣西經略安撫。初,廣西鹽易官般為客鈔,客戶無多,折閱逃避,遂抑配於民。行之六年,公私交病,追逮禁錮,民不聊生。孟明條具驛奏除其弊,詔從之。禁卒朱興結集黨侶,弄兵雷、化間,聲勢漸長,孟明遣將縛致轅門斬之。



 光宗即位,遷浙西提點刑獄,尋召為吏部員外郎,改左司,遷右司,再遷中書門下省檢正諸房公事。寧宗即位,拜太府卿兼吏部侍郎。慶元初,權吏部侍郎,卒。



 孟明以儒學奮身受知
 人主,官職未嘗倖遷。韓侂胄嘗遣其密客誘以諫官,俾誣趙汝愚,孟明不答,士論以此重之。



 曾三聘,字無逸,臨江新淦人。乾道二年進士。調贛州司戶參軍,累遷軍器監主簿。有旨造划車弩,三聘謂:「划車弩六人挽之,而箭之所及止二百六十步。今所用克敵弓較之,工費不及十之三,一人挽之而射可及三百六十步,利害曉然。」乃不果造。



 光宗不朝重華宮,中外疑懼,三聘以書抵丞相留正。正未及言,會以它事不合求去。
 三聘謂:「丞相今泯默而退耶,亦將取今日所難言者別白言之而後退?凡今闕庭之內,閨門衽席之間,父子夫婦之際,群臣莫敢深言者,避嫌遠罪耳。丞相身退計決,言之何嫌乎?」遷祕書郎。帝欲幸玉津園,三聘上疏言:「今人心既離,大亂將作,小大之臣震怖請命,而陛下安意肆志而弗聞知,萬一敵人諜知,馳一介之使,問安北宮,不知何以答之?姦宄窺間,傳一紙之檄,指斥乘輿,不知何以禦之?望亟備法駕朝謁,不然,臣實未知死所也。」



 孝
 宗病革,復上疏言:「道路流言,洶洶日甚,臣恐不幸而有狂夫姦人,託忠憤以行詐,假曲直以動眾,至此而後悔之,則恐無及矣。」帝意為動。及孝宗崩,帝疾不能執喪,朝論益震洶,三聘謂今日事勢,莫若建儲。或戒之曰:「前日臺諫諸公謂汝奪其職,今復有疏耶?」三聘曰:「此何時而可避煩言也。」



 寧宗立,兼考功郎,後知郢州。會韓侂胄為相,指三聘為故相趙汝愚腹心,坐追兩官。久之,復元官與祠。差知郴州,改提點廣西、湖北刑獄,皆辭不赴。侂胄
 誅,諸賢遭竄斥者相繼召用,三聘祿不及,終不自言。嘉熙間,三聘已卒,有旨特贈三官,直龍圖閣,賜謚「忠節」。



 徐僑,字崇甫,婺州義烏人。蚤從學於呂祖謙門人葉邽。淳熙十四年,舉進士。調上饒主簿,始登朱熹之門,熹稱其明白剛直,命以「毅」名齋。入為祕書省正字、校書郎兼吳、益王府教授。直寶謨閣、江東提點刑獄,以迕丞相史彌遠劾罷。寶慶初,葛洪、喬行簡代為請祠,迄不受祿。紹定中,告老,得請。



 端平初,與諸賢俱被召,遷祕書少監、太
 常少卿。趣入覲,手疏數千言,皆感憤剴切,上劘主闕,下逮群臣,分別黑白,無所回隱。帝數慰諭之,顧見其衣履垢敝,愀然謂曰:「卿可謂清貧。」僑對曰:「臣不貧,陛下乃貧耳。」帝曰:「朕何為貧?」僑曰:「陛下國本未建,疆宇日蹙;權幸用事,將帥非材;旱蝗相仍,盜賊並起;經用無藝,帑藏空虛;民困於橫斂,軍怨於掊克;群臣養交而天子孤立,國勢阽危而陛下不悟:臣不貧,陛下乃貧耳。」又言:「今女謁、閹宦相為囊橐,誕為二豎,以處國膏肓,而執政大臣又
 無和緩之術,陛下此之不慮而耽樂是從,世有扁鵲,將望見而卻走矣。」時貴妃閻氏方有寵,而內侍董宋臣表裏用事,故僑論及之。帝為之感動改容,咨嗟太息。明日,手詔罷邊帥之尤無狀者,申儆群臣以朋黨為之戒,命有司裁節中外浮費,而賜僑金帛甚厚。僑固辭不受。



 侍講,開陳友愛大義,用是復皇子竑爵,請從祀周敦頤、程顥、程頤、張載、朱熹,以趙汝愚侑食寧宗,帝皆如其請。金使至,僑以無國書宜館之於外,如叔向辭鄭故事,迕丞
 相意,力丐休致,帝諭留甚勤。遷工部侍郎,辭益堅,遂命以內祠侍讀,不得已就職。遇事盡言。以疾申前請,乃以寶謨閣待制奉祠。卒,謚「文清」。



 僑嘗言:「比年熹之書滿天下,不過割裂掇拾,以為進取之資,求其專精篤實,能得其所言者蓋鮮。」故其學一以真踐實履為尚。奏對之言,剖析理欲,因致勸懲。弘益為多。若其守官居家,清苦刻厲之操,人所難能也。



 度正,字周卿,合州人。紹熙元年進士。歷官為國子監丞。
 時士大夫無賢愚,皆策李全必反而不敢言,正獨上疏極言之,且獻斃全之策有三,其言鯁亮激切。



 遷軍器少監。輪對,言:「陛下推行聖學,當自正家始。」進太常少卿。適太廟災,為二說以獻,其一則用朱熹之議,其一則因宋朝廟制而參以熹之議:「自西徂東為一列,每室之後為一室,以藏祧廟之主。如僖祖廟以次祧主則藏之,昭居左,穆居右,後世穆之祧主藏太祖廟,昭之祧主藏太宗廟。仁宗為百世不遷之宗,後世昭之祧主則藏之。高宗
 為百世不遷之宗,後世穆之祧主則藏之。室之前為兩室;三年祫享,則帷帳幕之通為一室,盡出諸廟主及祧廟主並為一列,合食其上。往者此廟為一室,凡遇祫享合祭於室,名為合享,而實未嘗合享。合增此三室,後有藏祧主之所,前有祖宗合食之地,於本朝之制初無更革,而頗已得三年大祫之義。」



 遷權禮部侍郎兼侍右郎官,兼同修國史、實錄院同修撰。遷禮部侍郎,轉一官,守禮部侍郎致仕。卒,贈四官,賻銀絹三百。所著有《性善堂
 文集》。



 程珌,字懷古,徽州休寧人。紹熙四年進士。授昌化主簿,調建康府教授,改知富陽縣,遷主管官告院。歷宗正寺主簿、樞密院編修官,權右司郎官、秘書監丞,江東轉運判官。陛辭,寧宗謂宰臣曰:「程珌豈可容其補外?」遂復舊職。



 遷浙西提舉常平,又遷祕書丞,升祕書省著作郎,尋為軍器少監兼權左司郎官。遷國子司業兼國史編修、實錄檢討,兼權直舍人院,遷起居舍人,兼職依舊。權吏
 部侍郎,直學士院兼同修國史、實錄院同修撰,兼權中書舍人。遷禮部侍郎仍兼侍讀,權刑部尚書,封休寧縣男。授禮部尚書兼同修國史、實錄院同修撰,兼權吏部尚書,拜翰林學士、知制誥,兼修玉牒官,進封子。五上疏丐祠,以煥章閣學士、知建寧府,授福建路招捕使。以舊職提舉玉隆萬壽宮,進封伯。進敷文閣學士、知寧國府,改知贛州,皆不赴。進封新安郡侯,加寶文閣學士、知福州兼福建安撫使。再奉祠,又加龍圖閣學士。以端明殿
 學士致仕,卒,年七十有九,贈特進、少師。



 珌十歲詠冰,語出驚人。直學士院時,寧宗崩,丞相史彌遠夜召珌,舉家大驚。珌妻丞相王淮女也,泣涕,疑有不測,使人瞷之,知彌遠出迎,而後收涕。彌遠與珌同入禁中草矯詔,一夕為制誥二十有五。初許珌政府,楊皇后緘金一囊賜珌,珌受之不辭,歸視之,其直不貲。彌遠以是銜之,卒不與共政云。



 牛大年,字隆叟,揚州人。慶元二年進士。歷官將作監主
 簿。入對,言:「人主所當先者,要以天命人心之所繫致念焉。夫以人主居富貴崇高之位,重而承宗社之托,尊而為臣辟之戴,一指意而眾莫敢違,一動作而人孰敢議,然而天心靡常,則可畏也。」又言:「今日士氣亦久靡矣,宜體立國之意以振起之。夫有扶持作興之意,而後縉紳無貪名嗜利之習;無貪名嗜利之習,而後有持正秉義之操。國家之休戚,在士大夫之風俗,而風俗之善惡在朝廷。惟陛下為之振起,機括一運,天下轉移,而風俗易
 矣。」



 遷軍器監主簿、大宗正丞、四川提舉茶馬兼權總領、知黎州兼管內安撫司公事、節制黎雅州屯戍軍馬,加直寶章閣,為工部郎官。入對,請懲貪吏。遷侍左郎中,進直華文閣、浙東提點刑獄,遷守秘書少監、宗正少卿,升祕書監,遷起居舍人,升起居郎兼崇政殿說書。以寶章閣待制提舉太平興國宮,卒,特贈四官。大年清操凜然,所至以廉潔自將。



 陳仲微,字致廣,瑞州高安人。其先居江州,旌表義門。嘉
 泰二年,舉進士。調莆田尉,會守令闕,通判又疲軟不任,臺閫委以縣事。時歲兇,部卒並饑民作亂,仲微立召首亂者戮之。籍閉糶,抑強糴,一境以肅。囊山浮屠與郡學爭水利,久不決,仲微按法曰:「曲在浮屠。」它日沿檄過寺,其徒久揭其事鐘上以為冤,旦暮祝詛,然莫省為仲微也。仲微見之曰:「吾何心哉?吾何心哉?」質明,首僧無疾而死。寓公有誦仲微於當路而密授以薦牘者,仲微受而藏之。踰年,其家負縣租,竟逮其奴。寓公有怨言,仲微還
 其牘,緘封如故,其人慚謝,終其任不敢撓以私。



 遷海鹽丞。鄰邑有疑獄十年,郡命仲微按之,一問立決。改知崇陽縣,寢食公署旁,日與父老樵豎相爾汝,下情畢達,吏無所措手。通判黃州,職兼餉餽,以身律下,隨事檢柅,軍興賴以不乏。制置使上其最,辭曰:「職分也,何最之有?」復通判江州,遷幹辦諸司審計事,知贛州、江西提點刑獄,迕丞相賈似道,監察御史舒有開言罷。久之,起知惠州,遷太府寺丞兼權侍右郎官。輪對,言:「祿餌可以釣天下
 之中才,而不可啖嘗天下之豪傑;名航可以載天下之猥士,而不可以陸沉天下之英雄。」似道怒,又諷言者罷奪其官。久之,敘復。



 時國勢危甚,仲微上封事,其略曰:「誤襄者,老將也。夫襄之罪不專在於庸閫、疲將、孩兵也,君相當分受其責,以謝先皇帝在天之靈。天子若曰『罪在朕躬』,大臣宜言『咎在臣等』,宣布十年養安之往繆,深懲六年玩寇之昨非,救過未形,固已無極,追悔既往,尚愈於迷。或謂覆護之意多,克責之辭少;或謂陛下乏哭師
 之誓,師相飾分過之言,甚非所以慰恤死義,祈天悔禍之道也。往往代言乏知體之士,翹館鮮有識之人,吮旨茹柔,積習成痼,君道相業,兩有所虧。方今何時,而在廷無謀國之臣,在邊無折沖之帥。監之先朝宣和未亂之前、靖康既敗之後,凡前日之日近冕旒,朱輪華轂,俯首吐心,奴顏婢膝,即今日奉賊稱臣之人也;強力敏事,捷疾快意,即今日畔君賣國之人也。為國者亦何便於若人哉!迷國者進慆憂之欺以逢其君,託國者護恥敗之
 局而莫敢議,當國者昧安危之機而莫之悔。臣嘗思之,今之所少不止於兵。閫外之事,將軍制之,而一級半階,率從中出,斗粟尺布,退有後憂,平素無權,緩急有責,或請建督,或請行邊,或請京城,創聞駭聽。因諸閫有辭於緩急之時,故廟堂不得不掩惡於敗闕之後,有謀莫展,有敗無誅,上下包羞,噤無敢議。是以下至器仗甲馬,衰颯厖涼,不足以肅軍容;壁壘堡柵,折樊駕漏,不足以當衝突之騎。號為帥閫,名存實亡也。城而無兵,以城與敵;
 兵不知戰,以將與敵;將不知兵,以國與敵。光景蹙近目睫矣!惟君相幡然改悟,天下事尚可為也。轉敗為成,在君相一念間耳。」



 乃出仲微江東提點刑獄。



 德祐元年,遷祕書監,尋拜右正言、左司諫、殿中侍御史。益王即位海上,拜吏部尚書、給事中。厓山兵敗,走安南。越四年卒,年七十有二。



 其子文孫與安南王族人益稷出降,鄉導我師南征。安南王憤,伐仲微墓,斧其棺。



 仲微天稟篤實,雖生長富貴,而惡衣菲食,自同窶人。故能涵飫《六經》,精研
 理致,於諸子百家、天文、地理、醫藥、卜筮、釋老之學,靡不搜獵云。



 梁成大,字謙之,福州人。開禧元年進士。素茍賤亡恥,作縣滿秩,諂事史彌遠家幹萬昕,昕言真德秀當擊,成大曰:「某若入臺,必能辯此事。」昕為達其語。通判揚州,尋遷宗正寺簿。



 寶慶元年冬,轉對,首言:「大佞似忠,大辯若訥,或好名以自鬻,或立異以自詭,或假高尚之節以要君,或飾矯偽之學以欺世。言若忠鯁,心實回衺,一不察焉,
 薰蕕同器,涇渭雜流矣。言不達變,謀不中機,或巧辯以為能,或詭訐以市直,或設奇險之說以駴眾聽,或肆妄誕之論以惑士心。所行非所言,所守非所學,一不辯焉,枘鑿不侔,矛盾相激矣。」



 越六日,拜監察御史。尋奏:「魏了翁已從追竄,人猶以為罪大罰輕。真德秀狂僭悖繆,不減了翁,相羊家食,宜削秩貶竄,一等施行。」章既上,不下者兩月,或傳德秀有衡陽之命,時宰於帝前及之。帝曰:「仲尼不為已甚。」遂止鐫三秩。明年三月,又奏楊長孺寢
 新命,徐瑄追三秩移象州居住,胡夢昱移欽州編管。是冬,拜右正言。紹定元年,進左司諫。四年正月,遷宗正少卿。五年二月,權刑部侍郎,明年十月,帝夜降旨黜之,提舉千秋鴻禧觀。莫澤時兼給事中,急於別異,上疏駁之,遂寢祠命。端平初,洪咨夔、吳泳交章論駁,鐫兩秩。泳復上疏,送泉州居住。會王遂論亦上,再鐫秩,徙潮州。



 成大天資暴狠,心術嶮巇,凡可賊忠害良者,率多攘臂為之。四方賂遺,列置堂廡,賓至則導之使觀,欲其效尤也。尤
 嗜豪奪,冒占宇文氏賜第。既擯歸,訟之者不下百數。竄之日,朝命毀其廬,雖小人如李知孝亦曰:「所不堪者,他日與成大同傳耳。」



 李知孝,字孝章,參知政事光之孫。嘉定四年進士。嘗為右丞相府主管文字,不以為恥。差充幹辦諸司審計司,拜監察御史。



 寶慶元年八月,上疏:「士大夫汲汲好名,正救之力少而附和沽激之意多,扶持之意微而詆訾扇搖之意勝。既慮君上之或不能用,又恐朝廷之或不能
 容,姑為激怒之辭,退俟斥逐之命。始則慷慨而激烈,終則懇切而求去,將以樹奇節而求令名,此臣之所未解。」蓋陰詆真德秀等。又奏洪咨夔鐫三秩、放罷,胡夢昱追毀、除名、勒停,羈管象州。知孝猶語魏了翁曰:「此所論咨夔等,乃府第付出全文。」其情狀變詐如此。



 越月,復言:「近年以來,諸老凋零,後學晚出,不見前輩,不聞義理,不講綱常,識見卑陋,議論偏詖,更唱叠和,蠱惑人心,此風披扇,為害實深。乞下臣章,風厲內外,各務靖共,以杜亂萌。」
 拜右正言。又言:「德秀節改聖語,繆謄牒示,導信邪說,簧鼓同流,其或再有妄言,當追削流竄,以正典刑。」疏既上,遂鏤榜播告天下。又言:「趣召之人,率皆遲回,久而不至,以要君為高致,以共命為常流,可行而固不行,不疾而稱有疾,比比皆是,相扇成風,欲求難進易退之名,殊失尊君親上之義。願將趣召之人計其程途,限以時日,使之造朝;其有衰病者,早與改命。」時召傅伯成、楊簡、劉宰等皆不至,故知孝詆之。又奏張忠恕落職、鐫秩、罷郡。



 知
 孝拜殿中侍御史,升侍御史。紹定元年,遷右司諫,進右諫議大夫。五年,遷工部尚書兼侍讀。越月,進兵部。明年,理宗親政,以寶謨閣直學士出知寧國,後省駁之,令提舉嵩山崇福宮。端平初,監察御史洪咨夔、權直舍人院吳泳交章論駁,鐫秩罷祠。泳復封駁,繼送婺州居住。殿中侍御史王遂且論之,再鐫秩,徙瑞州。



 知孝起自名家,茍於仕進,領袖庶頑,懷諼迷國,排斥諸賢殆盡。時乘小輿,謁醉從官之家,侵欲斂積,不知紀極。紹定末,猶自乞
 為中丞,世指知孝及梁成大、莫澤為三兇。卒以貶死,天下快之。



 論曰:讀《本政書》,然後知林勳之於井地,可謂密矣。劉才邵能全名節於權姦之時。許忻之論和議,最為忠懇,卒以是去國,尤足悲夫。應孟明、曾三聘之不污韓侂胄,孔子所謂「歲寒然後知松柏之後凋也」。徐僑之清節,度正之淳敏,牛大年之廉正,陳仲微之忠實,然皆不至於大用,非可惜哉!若乃程珌之竊取富貴,梁成大、李知孝甘
 為史彌遠鷹犬,遺臭萬年者也。



\end{pinyinscope}