\article{志第一百二十七 食貨上二(方田 賦稅)}

\begin{pinyinscope}

 方田神宗患田賦不均,熙寧五年,重修定方田法,詔司農以《均稅條約並式》頒之天下。以東西南北各千步,當四十一頃六十六畝一百六十步,為一方;歲以九月,縣委令、佐分地計量,隨陂原平澤而定其地,因赤淤黑壚而辨其色;方量畢,以地及色參定肥瘠而分五等,以定稅則;至明年三月畢,揭以示民,一季無訟,即書戶帖,連莊帳付之,以為地符。



 均稅之法,縣各以其租額稅數為限,舊嘗收蹙奇零,如米不及十合而收為升,絹不滿十分而收為寸之類,今不得用其數均攤增展,致溢舊額,凡越額增數皆禁。若瘠鹵不毛,及眾所食利山林、陂塘、溝路、墳墓,皆不立稅。



 凡田方之角,立土為峰,植其野之所宜木以封表之。有方帳,有莊帳,有甲帖,有戶帖;其分煙析產、典賣割移,官給契,縣置簿,皆以今所方之田為正。令既具,乃以濟州鉅野尉王曼為指教官,先自京東路行之,諸路仿焉。六年,詔土色分
 五等,疑未盡,下郡縣物其土宜,多為等以其均當,勿拘以五。七年,京東十七州選官四員,各主其方,分行郡縣,以三年為任。每方差大甲頭二人、小甲頭三人,同集方戶,令各認步畝,方田官驗地色,更勒甲
 頭、方戶同定。諸路及開封府界秋田
 災傷
 三
 分以
 上縣權罷,餘候農隙。河北西路提舉司乞通一縣災傷不及一分勿罷。



 元豐五年,開封府言:「方田法,取稅之最不均縣先行,即一州而定五縣,歲不過兩縣,今府界十九縣,準此行之,十年乃定。請歲方五縣。」從之。其後歲稔農隙乃行,而縣多山林者或行或否。八年,帝知官吏擾民,詔罷之。天下之田已方而見於籍者,至是二百四十八萬四千三百四十有九頃云。



 崇寧三年,宰臣蔡京等言:「自開阡陌,使民得以田私相貿易,富者
 恃其有餘,厚立價以規利,貧者迫於不足,薄移稅以速售,而天下之賦調不平久矣。神宗講究方田利害,作法而推行之,方為之帳,而步畝高下丈尺不可隱;戶給之帖,而升合尺寸無所遺;以賣買,則民不能容其巧;以推收,則吏不能措其奸。今文籍具在,可舉而行。」詔諸路提舉常平官選官習熟其法,諭州縣官吏各以豐稔日推行,自京西、北兩路始。四年,指教官每三縣加一員,點檢官每路二員。未幾,詔諸路添置指教官不得過三員,又
 不專差點檢官,從提舉司於本路見任人內選差。五年,詔罷方田。大觀二年,復詔行之,四年罷其稅賦依未方舊則輸納。十一月,詔:「方田官吏非特妄增田稅,又兼不食之山方之,俾出芻草之直,民戶因時廢業失所。監司其悉改正,毋失其舊。」



 政和三年,河北西路提舉常平司奏:「所在地色極多,不下百數,及至均稅,不過十等。第一等雖出十分之稅,地土肥沃,尚以為輕;第十等只均一分,多是瘠鹵,出稅雖少,猶以為重。若不入等,則積多而至一頃,
 止以柴蒿之直,為錢自一百而至五百,比次十等,全不受稅;既收入等,但可耕之地便有一分之稅,其間下色之地與柴蒿之地不相遠,乃一例每畝均稅一分,上輕下重。欲乞土色十等如故外,折十等之地再分上、中、下三等,折畝均數。謂如第十等地每十畝合折第一等一畝,即十等之上,受稅十一,不改元則;十等之中,數及十五畝,十等之下,數及二十畝,方比上等受一畝之稅,庶幾上下輕重皆均。」詔諸路概行其法。五年,福建、利路茶
 戶山園,如鹽田例免方量均稅。



 宣化元年,臣僚言:「方量官憚於跋履,並不躬親,行繵拍峰、驗定土色,一付之胥吏。致御史臺受訴,有二百餘畝方為二十畝者,有二頃九十六畝方為一十七畝者,虔之瑞金縣是也。有租稅十有三錢而增至二貫二百者,有租稅二十七錢則增至一貫四百五十者,虔之會昌縣者是也。詔望常平使者檢察。」二年,遂詔罷之。民因方量流徙者,守令招誘歸業;荒閑田土,召人請佃。自今諸司毋得起請方田。諸路
 已方量者,賦稅不以有無訴論,悉如舊額輸納;民逃移歸業,已前逋欠稅租,並與除放。



 賦稅自唐建中初變租庸調法作年支兩稅,夏輸毋過六月,秋輸毋過十一月,遣使分道按率。其弊也,先期而苛斂,增額而繁征,至於五代極矣。



 宋制歲賦,其類有五:曰公田之賦,凡田之在官,賦民耕而收其租者是也。曰民田之賦,百姓各得專之者是也。曰城郭之賦,宅稅、地稅之類是也。曰丁口之賦,百姓歲輸身丁錢米是也。曰
 雜變之賦,牛革、蠶鹽之類,隨其所出,變而輸之是也。歲賦之物,其類有四:曰谷,曰帛,曰金、鐵,曰物產是也。穀之品七:一曰粟,二曰稻,三曰麥,四曰黍,五曰穄,六曰菽,七曰雜子。帛之品十:一曰羅,二曰綾,三曰絹,四曰櫬,五曰絁,六曰綢,七曰雜折,八曰絲線,九曰綿,十曰布葛。金鐵之品四:一曰金,二曰銀,三曰鐵、鑞,四曰銅、鐵錢。物產之品六:一曰六畜,二曰齒、革、翎毛,三曰茶、鹽,四曰竹木、麻草、芻菜,五曰果、藥、油、紙、薪、炭、漆、蠟,六曰雜物。其輸有常
 處,而以有餘補不足,則移此輸彼,移近輸遠,謂之「支移」。其入有常物,而一時所輸則變而取之,使其直輕重相當,謂之「折變」。其輸之遲速,視收成早暮而寬為之期,所以紓民力。諸州歲奏戶帳,具載其丁口,男夫二十為丁,六十為老。兩物折科物,非土地所宜而抑配者,禁之。



 五代以來,常檢視見墾田以定歲租。吏緣為奸,稅不均適,繇是百姓失業,田多荒蕪。太祖即位,詔許民闢土,州縣毋得檢括,止以見佃為額。選官分蒞京畿倉瘐,及詣諸
 道,受民租調,有增羨者輒得罪,多入民租者或至棄市。



 舊諸州收稅畢,符屬縣追吏會鈔,縣吏厚斂里胥以賂州之吏,里胥復率於民,民甚苦之。建炎四年,乃下詔禁止。令諸州受租籍不得稱分、毫、合、龠、銖、厘、絲、忽,錢必成文,絹帛成尺,粟成升,絲綿成兩,薪蒿成束,金銀成錢。綢不滿半匹、絹不滿一匹者,許計丈尺輸直,無得三戶、五戶聚合成匹,送納煩擾。民輸夏稅,所在遣縣尉部弓手於要路巡護,後聞擾民,罷之,止令鄉耆、壯丁防援。



 諸州
 稅籍,錄事參軍按視,判官振舉。形勢戶立別籍,通判專掌督之,二稅須於三限前半月畢輸。歲起納二稅,前期令縣各造稅籍,具一縣戶數、夏稅秋苗畝桑功及緣科物為帳一,送州覆校定,用州印,藏長吏廳,縣籍亦用州印,給付令佐。造夏稅籍以正月一日,秋稅籍以四月一日,並限四十五日畢。



 開封府等七十州夏稅,舊以五月十五日起納,七月三十日畢。河北、河東諸州氣候差晚,五月十五日起納,八月五日畢。穎州等一十三州及淮
 南、江南、兩浙、福建、廣南、荊湖、川峽五月一日起納,七月十五日畢。秋稅自九月一日起納,十二月十五日畢,後又並加一月或值閏月,其田蠶亦有早晚不同,有司臨時奏裁。繼而以河北、河東諸州秋稅多輸邊郡,常限外更加一月。江南、兩浙、荊湖、廣南、福建土多粳稻,須霜降成實,自十月一日始收租。掌納官吏以限外欠數,差定其罰。限前畢,減選,升資。民逋租逾限,取保歸辦,母得禁系。中國租二十石輸牛革一,準錢千。川蜀尚循舊制,牛
 驢死,革盡入官,乃詔蠲之,定民租二百石輸牛革一,準錢千五百。



 太平興國二年,江西轉運使言:「本路蠶桑數少,而金價頗低。今折征,絹估少而傷民,金估多而傷官。金上等舊估兩十千,今請估八千;絹上等舊估匹一千,今請估一千三百,餘以次增損。」從之。



 咸平三年,以刑部員外、直史館陳靖為京畿均田使,聽自擇京朝官,分縣據元額定稅,不得增收剩數;逃戶別立籍,令本府招誘歸業;桑功更不均檢,民戶廣令種植。尋聞居民弗諭朝
 旨,翦伐桑柘,即詔罷之。六年,罷廣南西路轉運使馮漣上言:「廉、橫、賓、白州民雖墾田,未嘗輸送,已命官檢括,令盡出常租。」帝曰:「遠方之民,宜省徭賦。」亟命停罷。知袁州何蒙請以金折本州二稅,真宗曰:「若是,將盡廢耕農矣。」不許。



 大中祥符初,連歲豐稔,邊儲有備,河北諸路稅賦,並聽於本州軍輸納。二年,頒《幕職州縣官招徠戶口旌賞條制》。舊制,縣吏能招增戶口者,縣即升等,乃加其奉;至有析客戶為主戶者,雖登於籍,而賦稅無所增。四年,
 詔禁之。雍熙初,嘗詔荊湖等路民輸丁錢,未成丁、已入老並身有廢疾者,免之。至是,又除兩浙、福建、荊湖、廣南舊輸身丁錢,歲凡四十五萬四百貫。九年,詔諸路支移稅賦勿至兩次,仍許以粟、麥、蕎、菽互相折輸。



 凡歲賦,穀以石計,錢以緡計,帛以匹計,金銀、絲綿以兩計,蒿秸、薪蒸以圍計,他物各以其數計。至道末,總七千八十九萬三千;天禧五年,視至道之數有增有減,總六千四百五十三萬。其折變及移輸比壤者,則視當時所須焉。



 宋克
 平諸國,每以恤民為先務,累朝相承,凡無名苛細之斂,常加鏟革,尺縑斗粟,未聞有所增益。一遇水旱徭役,則蠲除倚格,殆無虛歲,倚格者後或兇歉,亦輒蠲之。而又田制不立圳畝轉易,丁口隱漏,兼並冒偽,未嘗考按,故賦入之利視前代為薄。丁謂嘗言:二十而稅一者有之,三十而稅一者有之。仁宗嗣位,首寬畿縣田賦,詔三等以下戶毋遠輸。河中府、同華州請免支移,帝以問輔臣,對曰:「西鄙宿兵,非移用民賦則軍食不足。」特詔量減支
 移。



 福州王氏時有田千餘頃,謂之「官莊「,自太平興國中授券予民耕,歲使輸賦。至是,發運使方仲荀言:「此公田也,鬻之可得厚利。」遣尚書屯田員外郎幸惟慶領其事,凡售錢三十五萬餘緡,詔減緡錢三之一,期三年畢償。監察御史朱諫以為傷民,不可。即而期盡,未償者猶十二萬八千餘緡,詔悉蠲之。後又詔公田重復取賦者皆罷。天聖時,貝州言:「民析居者例加稅,謂之『罰稅』,他州無此比。」詔除之。自是,州縣有言稅之苛細無名者,蠲損甚
 眾。



 自唐以來,民計田輸賦外,增取他物,復折為賦,謂之「雜變」,亦謂之「沿納」。而名品煩細,其類不一。官司歲附帳籍,並緣侵優,民以為患。明道中,帝躬耕籍田,因詔三司以類並合。於是悉除諸名品,並為一物,夏秋歲入,第分粗細二色,百姓便之。



 州縣賦入有籍,歲一置,謂之空行簿,以待歲中催科;閏年別置,謂之實行簿,以藏有司。天聖初,或言實行簿無用,而率民錢為擾,罷之。景祐元年,侍御史韓瀆言:「天下賦入之繁,但存催科一簿,一有散
 亡,則耗登之數無從鉤考。請復置實行簿。」詔再閏一造。至慶歷中復故。



 時患州縣賦役之煩,詔諸路上其數,俾二府大臣合議蠲減。又詔曰:「稅籍有偽書逃徙,或因推割,用幸走移,若請占公田而不輸稅。如此之類,縣令、佐能究見其弊,以增賦入,量數議賞。」既而諫官王素言:「天下田賦輕重不等,請均定。」而歐陽修亦言:「秘書丞孫琳嘗往洺州肥鄉縣,與大理寺丞郭諮以千步方田法括定民田,願詔二人得任之。」三司亦以為然,且請於亳、壽、
 蔡、汝四州擇庀不均者均之。於是遣諮蔡州。諮首括一縣,得田二萬六千九百三十餘頃,均其賦於民。既而諮言州縣多逃田,未可盡括,朝廷亦重勞人,遂罷。



 陜西、河東用兵,民賦率多支移,因增取地里腳錢,民不能堪。五年,詔陜西特蠲之,且令後勿復取。既而詔河東亦然。又令諸路轉運司:「支移、折變,前期半歲書於榜以諭民,有未便者聽自言,主者裁之。」皇祐中,詔:「廣西賦布,匹為錢二百。如聞有司擅損其價,重困遠人,宜令復故。」州郡歲
 常先奏雨足歲豐,後雖災害,不敢上聞,故民賦罕得蠲者,乃下詔申飭之。又損開封諸縣田賦,視舊額十之三,命著於法。



 支移、折變,貧弱者尤以為患。景祐初,嘗詔戶在第九等免之,後孤獨戶亦皆免。至是,因下赦書,責轉運司裁損,歲終條上。其後赦書數以為言,又令折科為平估,毋得害農。久之,復詔曰:「如聞諸路比言折科民賦,多以所折復變他物,或增取其直,重困良農。雖屢戒敕,莫能奉宣詔令。自今有此,州長吏實時上聞。」然有司規
 聚斂,罕能承帝意焉。



 初,湖、廣、閩、浙因舊制歲斂丁身錢米,大中祥符間,詔除丁錢,而米輸如故。至天聖中,始並除婺、秀二州丁錢。後龐籍請罷漳、泉、興化軍丁米,有司持不可。皇祐三年,帝命三司首減郴永州、桂陽監丁米,以最下數一歲為準,歲減十餘萬石。既而漳、泉、興化亦第損之。嘉祐四年,復命轉運司裁定郴、永、桂陽、衡、道州所輸丁米及錢絹雜物,無業者馳之,有業者減半;後雖進丁,勿復增取。時廣南猶或輸丁錢,亦命轉運司條上。
 自是所輸無幾矣。



 自郭諮均稅之法罷,論者謂朝廷徒恤一時之勞,而失經遠之慮。至皇祐中,天下墾田視景德增四十一萬七千餘頃,而歲入九穀乃減七十一萬八千餘石,蓋田賦不均,其弊如此。後田京知滄州,均無棣田,蔡挺知博州,均聊城、高塘田;歲增賦穀帛之類,無棣總一千一百五十二,聊城、高塘總萬四千八百四十七,而滄州之民不以為便,詔輸如舊。嘉祐五年,復詔均定,遣官分行諸路,而秘書丞高本在遣中,獨以為不可
 均,才均數郡田而止。



 景德中,賦入之數總四千九百一十六萬九千九百,至皇祐中,增四百四十一萬八千六百六十五,治平中,又增一千四百一十七萬九千三百六十四。其以赦令蠲除以便於民,若逃移、戶絕不追者,景德中總六百八十二萬九千七百,皇祐中三十三萬八千四百五十七,治平中一千二百二十九萬八千七百。每歲以災害蠲除者,又不在是焉。



 神宗留意農賦,湖、廣之民舊歲輸丁米,大中祥符以後屢裁損,猶不均。熙
 寧四年,乃遣屯田員外郎周之純往廣東相度均之。元豐三年,詔:諸路支移折稅,並具所行月日,上之中書。初,熙寧八年,詔支移二稅於起納錢半歲諭民,使民宿辦,無倉卒勞費。時有司往往緩期,故申約之。州縣又或今民輸錢,謂之「折斛錢」,而糴賤頗用傷農。海南四州軍稅籍殘缺,吏多增損,輒移稅入他戶,代輸者類不能自明。瓊州、昌化軍丁稅米,歲移輸朱崖軍,道遠,民以為苦。至是,用體量安撫朱初平等議,根括四州軍稅賦舊額,存
 其正數;二州丁稅米止令輸錢於朱崖自糴以便民。



 權發遣三司戶部判官李琮根究逃絕稅役,江、浙所得逃戶凡四十萬一千三百有奇,為書上之。明年,除琮淮南轉運副使。兩路凡得逃絕、詭名挾佃、簿籍不載並闕丁凡四十七萬五千九百有奇,正稅並積負凡九十二萬二千二百貫、石、匹、兩有奇。琮蓋用貫石萬數立賞,以誘所委之吏,增加浩大,三路之民,大被其害。而唐州亦增民賦,人情騷然。六年,御史翟恩言:「始,趙尚寬為唐守,勸民
 墾田,高賦繼之,流民自占者眾,凡百畝起稅四畝而已。稅輕而民樂輸,境內殆無曠土。近聞轉運司闢土百畝增至二十畝,恐其勢再致轉徙。望戒飭使者,量加以寬民。」帝每遇水旱,輒輕馳賦租;或因赦宥,又蠲放、倚閣未嘗絕;賦輸遠方不均,皆遣使按之,率以為常。



 哲宗嗣位,宣仁太后同聽政,務行裕民之政,凡民有負,多所寬減。患天下積欠名目煩多,法令不一,王巖叟為開封,請隨等第立貫百為催法。兗州鄒令張文仲議其不便,遂令
 十分為率,歲隨夏秋料帶納一分,是為五年十料之法。



 陜西轉運使呂太忠令農戶支移,鬥輸腳錢十八。御史劾之,下提刑司體量,均其輕重之等。以稅賦戶籍在第一等、第二等者支移三百里,三等、四等者二百里,五等一百里。不願支移而願輸道里腳價者,亦酌度分為三等,以從其便。河東助軍糧草,支移毋得逾三百里。災傷五分以上者免折變,折變皆循舊法。



 紹聖中,嘗詔郡縣貨物用足錢、省陌不等,折變宜用中等。俄以所在時估
 實值多寡不齊,難概立法,命仍舊焉。言者謂:「欲民不流,不若多積穀;欲多積穀,不若推行折納糶糴之法。今常平雖有折納之法,止用中價,故民不樂輸。若依和糴以實價折之,則無損於民。」



 崇寧二年,諸路歲稔,遂行增價折納之法,支移、折變、科率、配買,皆以熙寧法從事,民以穀菽、物帛輸積負零稅者聽之。大觀二年詔:「天下租賦科撥支折,當先富後貧,自近及遠。乃者漕臣失職,有不均之患,民或受害,其定為令。支移本以便邊餉,內郡罕
 用焉。間有移用,則賃民以所費多寡自擇,故或輸本色於支移之地,或輸腳費於所居之邑。而折變之法,以納月初旬估中價準折,仍視歲之豐歉,以定物之低昂,俾官吏毋得私其輕重。」七月,詔曰:「比聞慢吏廢期,凡輸官之物,違期促限,蠶者未絲,農者未獲,追胥旁午,民無所措。自今前期督輸者,加一等坐之;致民逃徙者,論更加等。」舊凡以赦令蠲賦,雖多不過三分。四年,乃詔:天下逋賦,五年外戶口不存者,悉蠲之。



 京西舊不支移,崇寧中,
 將漕者忽令民曰:「支移所宜同,今特免;若地里腳費,則宜輸。」自是歲以為常。腳費,斗為錢五十六,比元豐即當正稅之數,而反復紐折,數倍於昔。民至鬻牛易產猶不能繼,轉運司乃用是以取辦理之譽,言者極論其害。政和元年,遂詔應支移而所輸地里腳錢不及鬥者,免之。尋詔五等戶稅不及鬥者,支移皆免。



 時天下戶口類多不實,雖嘗立法比較鉤考,歲終會其數,按籍隱括脫漏,定賞罰之格,然蔡攸等計德、霸二州戶口之數,率三戶
 四口,則戶版訛隱,不待校而知。乃詔諸路凡奏戶口,令提刑司及提舉常平司參考保奏。而終莫能拯其弊,故租稅亦不得而均焉。



 是時,內外之費浸以不給,中官楊戩主後苑作,有言汝州地可為稻田者,因用其言,置務掌之,號「稻田務』。復行於府畿,易名公田。南暨襄、唐,西及澠池,北逾大河,民田有溢於初券步畝者,輒使輸公田錢。政和末,又置營繕所,亦為公田。久之,後苑、營繕所公田皆並於西城所,盡山東、河朔天荒逃田與河堤退灘
 租稅舉入焉,皆內侍主其事。所括為田三萬四千三百餘頃,民輸公田錢外,正稅不復能輸。



 重和元年,獻言者曰:「物有豐匱,價有低昂,估豐賤之物,俾民輸送,折價既賤,輸官必多,則公私之利也。而州縣之吏,但計一方所乏,不計物之有無,責民所無,其費無量。至於支移,徙豐就歉,理則宜然。豪民賕吏,故徙歉以就豐,繼挾輕貨,以賤價輸官,其利自倍;而貧下戶各免支移,估值既高,更益腳費,視富戶反重。因之逋負,困於追胥。」詔申戒焉。



 宣
 和初,州縣主吏催科失職,逋租數廣,令轉運司察守貳勤惰,聽專達於內侍省。浙西逃田、天荒、草田、葑茭蕩、湖濼退灘等地,皆計籍召佃立租,以供應奉。置局命官,有「措置水利農田」之名,部使者且自督御前租課。



 三年,言者論西蜀折科之弊,其略謂:「西蜀初稅錢三百折絹一匹,草十圍計錢二十。今本路絹不用本色,匹折草百五十圍,圍估錢百五十,稅錢三百輸至二十三千。東蜀如之。仍支移新邊,謂之遠倉,民破產者眾。」七年,言者又論:「
 非法折變,既以絹折錢,又以錢折麥。以絹較錢,錢倍於絹;以錢較麥,麥倍於錢。展轉增加,民無所訴。」



 唐、鄧、襄,汝等州,自治平後,開墾歲增,然未定稅額。元豐中,以所墾新田差為五等輸稅,元祐元年罷之。大觀三年,用轉運副使張徽言之請,復元豐舊制,俄又以訴者而罷。政和三年,轉運使王□復言官失租賦,詔依元豐法,第折以見錢,凡得三十萬緡。欽宗立,詔蠲焉。舊稅租加耗,轉運司有拋樁明耗,州縣有暗樁暗耗之名,諸倉場受納,又
 令民輸頭子錢。熙寧以後,給納並收,其數益增焉,至是悉罷。



 高宗建炎元年五月庚寅,詔二稅並依舊法,凡百姓欠租、閣賦及應天府夏稅,悉蠲之。庚子,詔被虜之家蠲夏秋租稅及科配。



 紹興元年五月詔:「民力久困,州縣因緣為奸,今頒式諸路,凡因軍期不得已而貸於民者,並許計所用之多寡,度物力之輕重,依式開具,使民通知,毋得過數科率。」八月,減大觀稅額三分之一。十有一月,言者論:「浙西科斂之害,農末殆不聊生。鬻田而償,則
 無受者;棄之而遁,則質其妻孥。上下相蒙,民無所措手足。利歸貪吏,而怨歸陛下。願重科斂之罪,嚴貪墨之刑。」詔漕司究實以聞。二年正月,知紹興府陳汝錫違詔科率,謫漳州。四月,建盜範汝為平,詔蠲本路今年二稅及夏科役錢。既而手詔:「訪聞州縣以為著令不過三分,甚非所以稱朕惠恤之意,可以赦並免。」十有一月,焚州縣已蠲稅薄,示民以不疑也。五年二月,詔諸路轉運司以增收租數上戶部,課賞罰。



 六年八月,預借江、浙來年夏
 稅綢絹之半,盡令折米:兩浙綢絹各折七千,江南六千有半,每匹折米二石。九月,右司諫王搢言:「諸寺院之多產者,類請求貴臣改為墳院,冀免科斂,則所科歸之下戶。」詔戶部申嚴禁之。十有二月,詔淮西殘破州縣更免租稅二年。是月戊申,詔曰:「朕惟養兵之費,皆取於民,吾民甚苦;而吏莫之恤,夤緣軍須,掊斂無藝,朕甚悼之。監司郡守,朕所委寄以惠養元元者也,今漫不加省,復何賴焉!其各勤乃職,察民之侵漁納賄者,按劾以聞。茍庇
 覆弗治,朕不汝貸。」是歲,兩浙轉運李迨取婺秀湖州、平江府歲計寬剩錢二十二萬八千緡有奇,依折帛錢限起發。自是以為例。



 七年二月,詔:駐蹕及所過州縣欠紹興五年以前稅賦,並蠲之。七月,詔:新復州軍請佃官田,輸租外免輸正稅。



 己田謂之稅,佃田謂之租,舊不並納,劉豫嘗並取之,至是,乃從舊法。



 九年,蠲新復州軍稅租及土貢、大禮銀絹三年,差徭五年。初,劉豫之僭,凡民間蔬圃皆令三季輸稅。宣諭官方庭實言其不便,起居舍人程克俊言:「河南父老苦豫煩苛
 久矣,賦斂及於絮縷,割剝至於果蔬。」於是詔新復州縣,取劉豫重斂之法焚之通衢。



 十三年,淮東宣撫使韓世忠請以賜田及私產自昔未輸之稅並歸之官,詔獎諭而可之。初,神武右軍統制張俊乞蠲所置產凡和買、科敷,詔特從之。後,三省言:「國家兵革未息,用度至廣,陛下哀憫元元,俾士大夫及勛戚之家與編戶等敷,蓋欲寬民力,均有無。今俊獨得免,則當均在餘戶,是使民為俊代輸也。方今大將不止俊一人,使各援例求免,何以拒
 之?望收還前詔。」詔從之。越數年間,俊復乞免歲輸和買絹,三省擬歲賜俊絹五千匹,庶免起例。上以示俊,因諭之曰:「朕固不惜,但恐公議不可。」俊惶悚,力辭賜絹。



 十五年,戶部議:「準法,輸官物用四鈔,曰戶鈔,付民執憑;曰縣鈔,關縣司銷簿;曰監鈔,納官掌之;曰住鈔,倉庫藏之。所以防偽冒、備毀失也。



 毀失縣鈔者,以監、住鈔銷鑿;若輒取戶鈔,或追驗於人戶者,科杖。」



 二十三年,知池州黃子游言:「青陽縣苗七八倍於諸縣,因南唐嘗以縣為宋齊丘食邑,畝輸三斗,後遂為額。」詔減苗稅二分有半,
 租米二分。是時,兩浙州縣合輸綿、綢、稅絹、茶絹、雜錢、米六色,皆以市價折錢,卻別科米麥,有畝輸四五斗者。京西括田,租加於舊。湖南有土戶錢、折絁錢、醋息錢、曲引錢,名色不一。荊南戶口十萬,寇亂以來,幾無人跡。議者希朝廷意,謂流民已復,可使歲輸十二,頻歲復增,積逋至二十餘萬緡。曹泳為戶部侍郎,責償甚急。蓋自檜再相,密諭諸路暗增民稅七八,故民力重困,餓死者眾,皆檜之為也。



 二十六年,先是,承議郎魯沖上書論郡邑之
 弊:「以臣前任宜興一縣言之,漕計合收窠名,有丁鹽、坊場課利錢,租地錢,租絲租紵錢,歲入不過一萬五千餘緡。其發納之數,有大軍錢、上供錢、糴本錢、造船錢、軍器物料錢、天申節銀絹錢之類,歲支不啻三萬四千餘緡。又有見任、寄居官請奉、過往官兵批券、與非泛州郡督索拖欠,略無虛日。今之為令者,茍以寬恤為意,而拙於催科,旋踵以不職罷;能迎合上司,慘刻聚斂,則以稱職聞。是使為令者惴惴惟財賦是念,朝不謀夕,亦何暇為
 陛下奉行寬恤詔書、承流宣化者哉?」吏部侍郎許興古議:「今銓曹有知縣、令二百餘闕,無願就者,正緣財賦督迫被罪,所以畏避如此。若罷獻羨餘,蠲民積欠,謹擇守臣,戒飭監司,則吏稱民安矣。」乃詔行之。



 二十九年,上聞江西盜賊,謂輔臣曰:「輕徭薄賦,所以息盜。歲之水旱,所不能免,儻不寬恤而惟務科督,豈使民不為盜之意哉?」於是詔諸路州縣,紹興二十七年以前積欠官錢三百九十七萬餘緡及四等以下官欠,悉除之。九月,詔:兩浙、
 江東西水,浙東、江東西螟,其租稅盡蠲之。自是水旱、經兵,時有蠲減,不盡書也。



 三十二年六月戊寅,孝宗受禪赦:「凡官司債負、房賃、租賦、和買、役錢及坊場、河渡等錢,自紹興三十年以前並除之。諸路或假貢奉為名,漁奪民利,使所在居民以土物為苦,太上皇帝已嘗降詔禁約。自今州軍條上土貢之物,當議參酌天地、祖宗陵寢薦獻及德壽宮甘旨之奉,止許長吏修貢,其餘並罷。州縣因緣多取,以違制坐之。」七月,諸縣受民已輸稅租等
 鈔,不即銷簿者,當職官吏並科罪;民繼戶鈔不為使,而抑令重輸者,以違制論,不以赦免,著為令。八月,詔:「州縣受納秋苗,官吏多收加耗,肆為奸欺。方時艱虞,用度未足,欲減常賦而未能,豈忍使貪贓之徒重為民蠹?自今違犯官吏,並置重典,仍沒其家。」



 此孝宗初詔也。



 先是,常州宜興縣無稅產百姓,丁輸鹽錢二百文。下戶有墓地者,謂之墓戶,經界之時均紐正稅,又令帶輸丁鹽絹作折帛錢。至隆興元年,始用知縣姜詔言,令與晉陵、武進、無錫三
 縣一例隨產均輸。二年四月,知贛州趙公稱以寬剩錢十萬緡為民代輸夏稅,是後守臣時有代輸者。五月,詔:「溫、臺、處、徽不通水路,其二稅物帛,許依折法以銀折輸,數外妄有科折,計贓定罪。」



 乾道元年,蠲興化軍「猶剩米」之半。



 以知軍張允蹈言「自建炎三年,本軍秋稅,歲餘軍儲外,猶剩米二萬四千四百餘石,供給福州,謂之『猶剩米』。四十年間,水旱相仍,不復減損」,故有是命。至八年,乃並其半蠲之。



 三年六月,減臨安府新城縣進際稅賦之半。以知縣耿秉言,曩錢氏以進際為名,虛額太重故也。十有一月,蠲臨安府屬縣欠乾
 道元年三稅、坊場課利、折帛、免丁等錢。七年,敕令所修《輸苗乞取法》,受納官比犯人減一等,州縣長官不覺察與同罪。



 暨上三等及形勢戶逋賦,雖遇赦不除。八年,蠲紹興府增起苗米四萬九千餘石。



 淳熙三年,臣僚言:「湖北百姓廣占官田,量輸常賦,似為過優,比議者欲從實起稅而開陳首之門。殊不思朝廷往年經界,獨兩淮、京西、湖北依舊。蓋以四路被邊,土廣人稀,誘之使耕,猶懼不至,若履畝而稅,孰肯遠徙力耕,以供公上之賦哉?今湖北惟鼎、澧地接湖南,
 墾田稍多,自荊南、安、復、岳、鄂、漢、沔污萊彌望,戶口稀少,且皆江南狹鄉百姓,扶老攜幼,遠來請佃,以田畝寬而稅賦輕也。若從議者之言,恐於公家無一毫之益,而良民有無窮之擾矣。如臣所見,且當誘以開耕,不宜恐以增稅。使田疇盡闢,歲收滋廣,一遇豐稔,平糴以實邊,則所省漕運亦博。望其依紹興十六年詔旨,以十分為率,年增輸一分,不願開墾者,即許退田別佃。期限稍寬,取之有漸,遠民安業,一路幸甚。」詔戶部議之。



 四年,臣僚言:「
 屢赦蠲積欠,以蘇疲民,州縣不能仰承德意,至變易名色以取之。宜下漕司,如合除者毋更取之於州,州毋取之於縣,縣銷民欠籍,書其名數,諭民通知。」詔可。五年八月,詔曰:「比年以來,五穀屢登,蠶絲盈箱,嘉與海內共享阜康之樂,尚念耕夫蠶婦終歲勤動,價賤不足以償其勞。郡邑兩稅,除折帛、折變自有常制,當輸正色者,毋以重價強之折錢。若有故違,重置於法。臨安府刻石,遍賜諸路。」六年,以諫議大夫謝廓然言:「州縣違法科斂,侵漁
 日甚,其咎雖在縣令,而督迫實由郡守。縣令按劾,而郡守自如。」詔:「自今凡有過需橫取,監司悉行按劾,無詳於小而略於大。」



 七年夏,大旱。知南康軍朱熹應詔上封事言:「今民間二稅之入,朝廷盡取以供軍,州縣無復贏餘,於是別立名色巧取。今民貧賦重,惟有核兵籍,廣屯田,練民兵,可以漸省列屯坐食之兵,稍損州郡供軍之數。使州縣之力浸紓,然後禁其苛斂,責其寬恤,庶幾窮困之民得保生業,無流移漂蕩之患。」八年,詔監司、太守察
 所部催科不擾者薦之,煩擾害民者劾之。十一年,戶部奏:「諸路州軍檢放旱傷米數近六十萬石。上諭王淮曰:「若盡令核實,恐他年郡縣懷疑,不復檢放。惟寧國數最多,可令漕司核實而蠲之。」



 紹熙元年,臣僚言:「古者賦租出於民之所有,不強其所無。今之為絹者,一倍折而為錢,再倍折而為銀。銀愈貴,錢愈艱得,穀愈不可售,使民賤糶而貴折,則大熟之歲反為民害。願詔州郡:凡多取而多折者,重置於罰;民有糶不售者,令常平就糴,異時
 歲歉,平價以糶。庶於民無傷,於國有補。」詔從之。



 秘書監楊萬里奏:「民輸粟於官謂之苗,舊以一斛輸一斛,今以二斛輸一斛矣。輸帛於官謂之稅,舊以正絹為稅絹,今正絹外有和買矣。舊和買官給其直,或以錢,或以鹽,今皆無之,又以絹估直而倍折其錢矣。舊稅畝一錢輸免役一錢,今歲增其額,不知所止矣。既一倍其粟,數倍其帛,又數倍其錢,而又有月樁錢、版帳錢、不知幾倍於祖宗之舊,又幾倍於漢、唐之制乎。此猶東南之賦可知也,
 至於蜀賦之額外無名者,不可得而知也。陛下欲薄賦斂,當節用度。用節而後財可積,財積而後國可足,國足而後賦可減,賦減而後民可富,民富而後邦可寧。不然,日復日,歲復歲,臣未知其所終也。」



 時金主璟新立,萬里迓使客於淮,聞其蠲民間房園地基錢,罷鄉村官酒坊,減鹽價,除田租,使虛譽達於吾境,故因轉對而有是言也。



 二年,詔曰:「朕惟為政之道,莫先於養民。故自即位以來,蠲除甚賦,頒宣寬條,嘉與四方臻於安富。郡守、縣令,最近民者也。誠能拊循惠愛,以承休德,庶幾政平訟理之效。今採
 之人言,乃聞科斂先期,競務辦集,而民之虛實不問;追呼相繼,敢為椎剝,而民之安否不恤。財計之外,治理蔑聞,甚不稱朕委屬之意。國用有常,固在經理,而非掊克督趣以為能也。知本末先後之誼,此朕所貴於守令者。繼自今以軫恤為心,以牧養為務,俾民安業,時予汝嘉。」



 慶元二年,詔浙江東、西夏稅、和買綢絹並依紹興十六年詔旨折納。紹興十六年詔旨:絹三分折錢,七分本色;綢八分折錢,二分本色。



 嘉熙二年臣僚言:「陛下自登大寶以來,蠲賦之詔無歲無之,而
 百姓未沾實惠。蓋民輸率先期歸於吏胥、攬戶,及遇詔下,則所放者吏胥之物,所倚閣者攬戶之錢,是以寬恤之詔雖頒,愁嘆之聲如故。嘗覺漢史恤民之詔,多減明年田租。今宜仿漢故事,如遇朝廷行大惠,則以今年下詔,明年減租,示民先知減數,則吏難為欺,民拜實賜矣。」從之。



 淳祐八年,監察御史兼崇政殿說書陳求魯奏:「本朝仁政有餘,而王制未備。今之兩稅,本大歷之弊法也。常賦之入尚為病,況預借乎?預借一歲未已也,至於再,
 至於三;預借三歲未已也,至於四,至於五。竊聞今之州縣,有借淳祐十四年者矣。以百畝之家計之,罄其永業,豈足支數年之借乎?操縱出於權宜,官吏得以簸弄,上下為奸,公私俱困。臣愚謂今日救弊之策,其大端有四焉:宜採夏侯太初並省州郡之議,俾縣令得以直達於朝廷;用宋元嘉六年為斷之法,俾縣令得以究心於撫字;法藝祖出朝紳為令之典,以重其權;遵光武擢卓茂為三公之意,以激其氣。然後為之正其經界,明其版籍,
 約其妄費,裁其橫斂,則預借可革,民瘼有瘳矣。」



 咸淳十年,侍御史陳堅、殿中侍御史陳過等奏:「今東南之民力竭矣,西北之邊患棘矣,諸葛亮所謂危急存亡之時也。而邸第戚畹、御前寺觀,田連阡陌,亡慮數千萬計,皆巧立名色,盡蠲二稅。州縣乏興,鞭撻黎庶,鬻妻買子,而鐘鳴鼎食之家,蒼頭廬兒,漿酒藿肉;琳宮梵宇之流,安居暇食,優游死生。安平無事之時尤且不可,而況艱難多事之際乎?今欲寬邊患,當紓民力;欲紓民力,當紓州縣,
 則邸第、寺觀之常賦,不可姑息而不加厘正也。望與二三大臣亟議行之。」詔可。



 建炎二年,初復鈔旁定帖錢,命諸路提刑司掌之。紹興二年,詔偽造券旁者並依軍法。五年三月,詔諸州勘合錢貫收十文足。勘合錢,即所謂鈔旁定帖錢也。初令諸州通判印賣田宅契紙,自今民間爭田,執白契者勿用。十有一月,以調度不足,詔諸路州縣出賣戶帖,令民具田宅之數而輸其直。既而以苛擾稽緩,乃立價:凡坊郭鄉村出等戶皆三十千,鄉村五
 等、坊郭九等戶皆一千,凡六等,惟閩、廣下戶差減;期三月足輸送行在,旱傷及四分以上者聽旨。



 三十一年,先是,諸州人戶典賣田宅契稅錢所收窠名,七分隸經、總制,三分屬系省。至是,總領四川財賦王之望言,請從本所措置拘收,以供軍用,詔從之。凡嫁資、遺囑及民間葬地,皆令投契納稅,一歲中得錢四百六十七萬餘引,而極邊所捐八郡及盧、夔等未輸者十九郡不與焉。乾道五年,戶部尚書曾懷言:「四川立限拘錢數百萬緡,婺州
 亦得錢三十餘萬緡,他路恬不加意。」詔:「百姓白契,期三月自陳,再期百日輸稅,通判拘入總制帳。輸送及十一萬緡者,知、通推賞;違期不首,及輸錢違期者,許人告,論如律。」淳熙六年,敕令所進《重修淳熙法》,有收舟、驢、駝、馬契書之稅,帝命刪之,曰:「恐後世有算及舟車之言。」



 建炎三年,張浚節制川、陜,承制以同主管川、秦茶馬趙開為隨軍轉運使,總領四川財賦。自蜀有西師,益、利諸司已用便宜截三路上供錢。



 川峽布絹之給陜西、河東、京西者。



 四年秋,遂盡
 起元豐以來諸路常平司坊場錢,元豐以來封樁者。



 次科激賞絹,是年初科三十三萬匹,俟邊事寧即罷。紹興十六年,減利、夔三萬匹,惟東、西川三十萬匹至今不減。



 次奇零絹估錢,即上三路綱也,歲三十萬匹。西川匹理十一引,東川十引。自紹興二十五年至慶元初,兩川並減至六引。



 次布估錢,成都崇慶府、彭漢邛州、永康六郡,自天聖間,官以三百錢市布一匹,民甚便之,後不復予錢。至是,宣撫司又令民匹輸估錢三引,歲七十餘萬匹,為錢二百餘萬引。慶元初,累減至一百三十餘萬引。



 次常平司積年本息,此熙、豐以來所謂青苗錢者。建炎元年,遣駕部員外郎喻汝礪括得八百餘萬緡,至是,取以贍軍矣。



 次對糴米,謂如戶當輸稅百石,則又科糴百石,故謂之對糴。及他名色錢。如酒、鹽等。



 大抵於先朝常賦外,歲增錢二
 千六十八萬緡,而茶不預焉。自是軍儲稍充,而蜀民始困矣。



 紹興五年,浚召拜尚書右僕射,以席益為四川安撫制置大使,趙開為四川都轉運使。益頗侵用軍期錢,開訴於朝,又數增錢引,而軍計猶不給。六年,以龍圖閣直學士李迨代開為都轉運使。都官員外郎馮康國言:「四川地狹民貧,祖宗時,正稅重者折科稍輕,正稅輕者折科稍重,二者平準,所以無偏重偏輕之患。百有餘年,民甚安之。近年,漕、總二司輒更舊法,反復紐折,取數
 務多,致民棄業逃移。望並罷之,一遵舊制。」詔如所請,令憲臣察其不如法者。



 七年三月,迨以贍軍錢糧令四路漕臣分認,而榷茶錢不用,蜀人不以為是。九月,浚罷,趙鼎為尚書左僕射。十有一月,以直秘閣張深主管四川茶馬,迨請祠。八年二月,命深及宣撫司參議官陳遠猷並兼四川轉運副使。席益以憂去,樞密直學士胡世將代之。十月,鼎罷,秦檜獨相。九年,和議成。簽書樞密院事樓照宣諭陜西還,以金四千兩、銀二十萬兩輸激賞庫,
 皆取諸蜀者。會吳玠卒,以世將為宣撫副使,以吏部尚書張燾知成都府兼本路安撫使。上諭輔臣曰:「燾可付以便宜。如四川前日橫斂,宜令減以紓民。」成都帥行民事,自燾始。世將奏以宣撫司參議官井度兼四川轉運副使。



 十一年正月,趙開卒。自金人犯陜、蜀,開職饋餉者十年,軍用無乏,一時賴之。其後計臣屢易,於開經畫無敢變更。然茶、鹽、榷酤、奇零絹布之徵,自是為蜀之常賦,雖屢經蠲減而害不去,議者不能無咎開之作俑焉。



 十
 月,鄭剛中為川、陜宣諭使。十二年,世將卒,改宣撫使。十三年,剛中獻黃金萬兩。十五年正月,剛中奏減成都路對糴米三之一。四月,省四川都轉運使,以其事歸宣撫司。剛中尋以事忤秦檜,於是置四川總領所錢糧官,以太府少卿趙不棄為之。又改命不棄總領四川宣撫司錢糧。十六年,剛中奏減兩川米腳錢三十二萬緡,激賞絹二萬匹,免創增酒錢三萬四千緡。以四川總制錢五十萬緡充邊費。十七年,以戶部員外郎符行中總
 領四川宣撫司錢糧,召剛中赴行在,不棄權工部侍郎,知成都府李璆權四川宣撫司事。



 先是,剛中奏:「本司舊貯備邊歲入錢引五百八十一萬五千道,如撥供歲計,即可對減增添,寬省民力。」詔李璆、符行中參酌減放。於是減四川科敷虛額錢歲二百八十五萬緡,兩川布估錢三十六萬五千緡,夔路鹽錢七萬六千緡,坊場、河度凈利抽貫稅錢四萬六千餘緡,又減兩川米腳錢四十二萬緡。時宣撫司降賜庫貯米一百萬石,乃命行中酌度
 對糴分數均減。



 十八年,罷四川宣撫司,以璆為四川安撫制置使兼知成都府,太府少卿汪召嗣總領四川財賦軍馬錢糧。宣撫司降賜庫錢,除制置司取撥二十萬緡,餘令總領所貯之。二十二年,總領所奏蠲諸路欠紹興十七年以前折估糴本等錢一百二十九萬餘緡,米九萬八千七百餘石,綾、絹一萬四千餘匹。先是,自講和後,歲減錢四百六十二萬緡有奇,朝廷猶以為重。二十四年,遣戶部員外郎鐘世明同四川制、總兩司措置裕
 民。二十五年,以符行中等言,減兩川絹估錢二十八萬緡,潼川府秋稅腳錢四萬緡,利路科斛腳錢十二萬緡,兩川米腳錢四十萬緡,鹽酒重額錢七十四萬緡,激賞絹九千餘匹,合一百六十餘萬緡;蠲州縣紹興十九年至二十三年折估糴本等逋欠二百九十二萬緡。



 是時,朝廷雖蠲民舊逋,而符行中督責猶峻,蜀人怨之。於是以蕭振為四川安撫制置使兼知成都府,行中提舉江州太平興國宮。二十六年,上以蜀民久困供億,詔制置
 蕭振、總領湯允恭、主管茶馬李澗、成都轉運判官許尹、潼川轉運判官王之望措置寬恤,於是之望奏減四川上供之半。二十七年,用蕭振等言,減三川對糴米十六萬九千餘石,夔路激賞絹五萬匹,兩川絹估錢二十八萬緡有奇,潼川、成都奇零折帛匹一千;又減韓球所增茶額四百六十二萬餘斤,茶司引息虛額錢歲九十五萬餘緡。



 初,利州舊宣撫司有積緡二百萬,守者密獻之朝,下制置司取撥。振曰:「此所以備水旱軍旅也,一旦
 有急,又將取諸民乎?請留其半。」是歲振卒,李文會代之。二十八年,文會卒,中書舍人王剛中代之。二十九年,蠲四川折估糴本積欠錢三百四十萬緡。



 乾道二年,蠲奇欠白稅契錢三十七萬餘緡。三年,蠲川、秦茶馬兩司紹興十九年至三十二年州縣侵用及民積欠六十六萬四千九百餘緡。四年,又詔:四川諸州欠紹興三十一年至隆興二年瞻軍諸窠名錢物,暨退剝虧分之數,及漏底折欠等錢,並蠲之。蠲成都人戶理運對糴米腳錢三十
 五萬緡。淳熙十六年詔:「四川歲發湖、廣總領所綱運百三十五萬六千餘貫,自明年始,與免三年。當議對減鹽酒之額,制置、總領同諸路轉運、提刑司條上。其湖、廣歲計,朝廷當自給之。」



 紹熙三年,蠲潼川府去年被水州縣租稅,資普榮敘州、富順監凡夏輸亦如之。尋又詔:「本路旱傷州縣租稅,官為代輸及民已輸者,悉理今年之數。」四年,蠲紹熙三年成都、潼川兩路奇零絹估錢引四十七萬一千四百五十餘道,潼川府激賞絹一十六萬六
 千九百七十五匹。又詔:四川州縣鹽、酒課額,自明年更放三年。



 嘉定七年,再蠲四川州縣鹽、酒課額三年,其合輸湖、廣總領所綱運亦免三年。十一年,蠲天水軍今年租役差科,西和州蠲十之七,成州蠲十之六,將利、河池兩縣各蠲十之五,以經兵也。



\end{pinyinscope}