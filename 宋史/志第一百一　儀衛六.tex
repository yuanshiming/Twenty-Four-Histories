\article{志第一百一 儀衛六}

\begin{pinyinscope}

 鹵簿
 儀服。



 鹵簿儀服。自漢鹵簿,像最在前。晉平吳後,南越獻馴象,作大車駕之,以載黃門鼓吹數十人,使越人騎之以試橋梁。宋鹵簿,以象居先,設木蓮花坐,金蕉盤,紫羅繡幨
 絡腦,當胸、後秋並設銅鈴杏葉,紅犛牛尾拂,跋塵。每象,南越軍一人跨其上,四人引,並花腳帕頭、緋繡窄衣、銀帶。太宗太平興國六年,兩莊養象所奏,詔以象十於南郊引駕,開寶九年南郊時,其象止在六引前排列。詔鹵簿使領其事。



 旗,皆錯採為之,漆竿、金□首、纛頭、錦帶腰、火焰腳。白澤、攝提、金鸞、金鳳、師子、苣文、天下太平、君王萬歲、仙童、螣蛇、神龜,及在步甲前後隊、後馬隊三隊、六軍儀仗內,並以
 赤。日、月及合璧、連珠、風、雨、雷、電、五星、二十八宿、祥雲,並以青。北斗以黑。五岳、四瀆、五方、四神、十二辰、五龍、五鳳、龍虎君,並以方色。天王以赤、黃二色。排攔以黃、紫、赤三色。



 元豐三年,詳定郊廟奉祀禮文所言:「鹵簿,前用二十八宿、五星、攝提旗,有司乃取方士之說,繪為人形,於禮無據。伏請改制,各著其象,以則天文。從之。元祐七年,太常寺言:「二十八宿旗,五星、攝提旗,按《鹵簿圖》畫人形及牛虎頭、婦人、小兒之類,於禮無據。元豐三年,禮文所上
 言乞改制,各著其象,以則天文。後有司循舊儀,未曾改正,今欲改造。」從之。



 元符二年,徽宗即位,兵部侍郎黃裳言:「南郊大駕諸旗名物,除用典故制號外,餘因時事取名。伏見近者璽授元符,茅山之上日有重輪,太上老君眉間發紅光,武夷君廟有仙鶴,臣請制為旗號,曰寶符,曰重輪,曰祥光,曰瑞鶴。」從之。



 政和四年,禮制局言:「鹵簿,大黃龍負圖旗畫八卦,乞改畫九、一、三、七、二、四、六、八、五之數。仙童、網子、大神三旗無所經見,乞除去。」從之。初,大
 觀三年,西京穎陽縣大慶觀聖祖殿東,有嘉禾、芝草並生。其嘉禾一本四穗,芝草葉圓而重起。至是,詔制芝禾並秀旗。又以是年二月,日上生青、赤、黃戴氣;後,日下生青、赤、黃承氣,詔制日有戴承旗。又以元符二年武夷君廟有仙鶴迎詔,政和二年延福宮宴輔臣,有群鶴自西北來,盤旋於睿謨殿上,及奏大晟樂而翔鶴屢至,詔制瑞鶴旗。



 八年,禮部侍郎張邦昌奏:「太祖時,甘露降於江陵者十日,瑞麥秀於濮陽者六歧,獲金鸚鵡於隴坻,得
 三玉兔於鄆封,馴象至而五嶺平,瓊管族而白鹿出,皆命制為旗章陳之。望詔有司取自崇、觀至今,凡中外所上瑞應,悉掇其尤殊者,增制旗物,上以丕承天貺,下以聳動民瞻。」從之。



 初,宋制旗物尢盛,中興後惟務簡約,雖參用舊制,然亦不無因革。其太常,青質夾羅,惟繡日、月、星而無龍,下有網須謂之茀,而竿頭為龍首,銜青結綬,垂青旄緌十二,謂之旒。蓋幅下無斿,而竿首垂旒,抑又取古者「注旄及羽於竿首」之遺制。竿用椆木,護以剖竹,
 膠以髹,飾以藻,玉輅建之。大旗,黃質九幅,每幅繡升龍一,側幅二,下垂黃絲網緌九,金輅建之。太赤,朱質七幅,每幅繡鳥隼二,側幅如之,下垂朱絲網緌七,像輅建之。大白,素質五幅,每幅繡熊一、虎一,側幅如之,下垂淺黃絲網緌五,革輅建之。大麾,皂質四幅,每幅繡五採龜蛇一,側幅繡龜二,下垂皂絲網緌四,木輅建之。



 其黃龍負圖旗,建隆初創為大制。有架,旗力重,以百九十人維之,今用七十人。其君王萬歲、天下太平、日月、五星、北斗、招
 搖、青龍、朱雀、白虎、玄武等十旗,皆以十七人維之。其祥瑞旗八,紹興二十五年所制也。是歲,適當郊祀,而太廟生靈芝九莖,贛州進太平瑞木,道州連理木,遂寧府嘉禾,鎮江府瑞瓜,南安軍雙蓮花,嚴州兜率寺、信州玉山芝草,黎州甘露,禮部侍郎王鈱等請繪之華旗,以紀盛美焉。



 五牛旗,依方色,皆小輿上刻木為牛,背插旗。錯採為牛,旗竿上有小盤,盤衣及輿衣,亦並繡牛形。輿士各四人,
 服繡五色牛衣。自太祖時詔用之。神宗熙寧七年,太常寺言:「大駕鹵簿羊車,本前代宮中所乘;五牛旗,蓋古之五時副車也,以木牛載旗,用人輿之,失其本制,宜省去。」從之。



 牙門旗,古者,天子出建大牙。今制,赤質,錯採為神人像,中道前後各一門,左右道五門,門二旗,蓋取周制「樹旗表門」及「天子五門」之制。



 駕頭,一名寶床,正衙法坐也。香木為之,四足□彖山,以龍
 卷之。坐面用藤織雲龍,四圍錯採,繪走龍形,微曲。上加緋羅繡褥,裹以緋羅繡帕。每車駕出幸,則使老內臣馬上擁之,為前驅焉。不設,則以朱匣韜之。



 幡,本幟也,貌幡幡然。有告止、傳教、信幡,皆絳帛,錯採為字,上有朱綠小蓋,四角垂羅文佩,系龍頭竿上。其錯採字下,告止為雙鳳,傳教為雙白虎,信幡為雙龍。又有絳引幡,制頗同此,作五色間暈,無字,兩角垂佩。中興為六角蓋,垂珠佩,下有橫木板,作碾玉文。三幡,亦以錯採篆
 書「告止」、「傳教」、「信幡」。



 幢,制如節而五層,韜以袋,繡四神,隨方色,朱漆柄。取《曲禮》「行前朱雀而後玄武,左青龍而右白虎」之義。王公所給幢,黑漆柄,紫綾袋。中興,用生色袋。



 皂纛,本後魏纛頭之制。唐衛尉器用,纛居其一,蓋旄頭之遺像。制同旗,無文採,去金□首六腳。《後志》云:「今制,皂邊皂斿,斿為火焰之形。」金吾仗主之,每纛一人持,一人拓之。乘輿行,則陳於鹵簿,左右各六。



 絳麾,如幢,止三層,紫羅囊蒙之。王公麾,以紫綾袋。



 黃麾,古有黃、朱、纁三色,所以指麾也。漢鹵簿有前黃麾護駕御史。宋制,絳帛為之,如幡,錯採成「黃麾』字,下繡交龍;朱漆竿,金龍首,上垂朱綠小蓋。神宗元豐二年,詳定朝會御殿儀注所言:



 按《周禮》「木輅建大麾,以田」,鄭氏曰:「大麾不在九旗之中。以正色言之,則黑,夏后氏所建。」《禮記》曰:「有虞氏之旗,夏后氏緌。」鄭氏曰:「緌,謂注旄牛尾於杠首。所謂大麾,《書》曰『王右秉白旄以麾』。」孔穎達曰:「虞世
 但注旄,夏世始加旒縿。」《西京雜記》,漢大駕有前黃麾。崔豹《古今注》:「麾,所以指麾,乘輿以黃,諸公以朱,刺史二千石以纁。」《開元禮義纂》曰:「唐太宗法夏后之前制,取中方之正色,故制大麾,色黃。」



 今禮有黃麾,其制十二幅。《開寶通禮義纂》曰:「黃,中央之色。此仗最近車輅,故以應像,取其居中,導達四方,含光大也。」今鹵簿黃麾,以夏制言之,則狀不類旗;以漢制言之,則色又不黃。伏請制大麾一:注麾於竿首,則法夏后氏之制;其色正黃,則用漢制;以
 十二幅為旗,則取唐制;以一旒為之,則取今龍墀旗之制。當元會陳仗衛,建大黃麾一於當御廂之前,以為表識。其當御廂之後,則建黃麾幡二。



 並上大黃麾、黃麾幡制度。神宗批曰:「黃麾制度,考詳前志,終是可疑。今鑿而為之,植於大庭中外共瞻之地,或為博聞多識者所譏。宜且闕之,更俟討求,黃麾幡仍舊。」



 氅,本緝鳥毛為之。唐有六色、孔雀、大小鵝毛、雞毛之制。《後志》云:「今制有青、緋、皂、白、黃五色,上有朱蓋,下垂帶,
 帶繡禽羽,末綴金鈴。青則繡以孔雀,五角蓋;緋則繡以鳳,六角蓋;皂則繡以鵝,六角蓋;白亦以鵝,四角蓋;黃則以雞,四角蓋。每角綴垂佩,揭以朱竿,上如戟,加橫木龍首以系之。」



 金節,隋制也。黑漆竿,上施圓盤,周綴紅絲拂八層,黃繡龍袋籠之。王公以下皆有節,制同金節,韜以碧油。



 傘,古張帛避雨之制。今有方傘、大傘,皆赤質,紫表朱裏,四角銅螭首。六引內者,其制差小。哲宗元祐七年,太常寺言:《
 開元禮》大駕八角紫傘,王公已下四角青傘。今《鹵簿圖》但引紫傘,而無青傘之文。詔改用。紹興十三年將郊,詔傘、扇如舊制,拂扇等不以珠飾。



 蓋,本黃帝時有雲氣為花FM之象,因而作也。宋有花蓋、導蓋,皆赤質,如傘而圓,瀝水繡花龍。又有曲蓋,差小,惟乘輿用之。人臣則親王或賜之,而以青繒繡瑞草焉。



 睥睨,如華蓋而小。



 扇筤,緋羅繡扇二,緋羅繡曲蓋一,並內臣馬上執之。駕
 頭在細仗前,扇筤在乘輿後。大駕、法駕、鸞駕,常出並用之。扇圓,徑四尺二寸,柄長八尺三寸,黃茸繡團龍,仍用金塗銅飾。扇有朱團及雉尾四等。朱團繡雲鳳或雜花,黑漆柄,金銅飾。雉尾皆方,繡雉尾之狀,有三等:大雉扇長五尺二寸,闊三尺七寸;中扇、小扇遞減二寸。下方上殺,以緋羅繡雉尾之狀,中有雙孔雀雜花,下施黑漆橫木長柄,以金塗銅飾。乘輿出入,必以前持鄣蔽。凡朔望朝賀、行冊禮,皇帝升御坐,必合扇,坐定去扇,禮畢駕退,
 又索扇如初。蓋謂天子升降俯仰,眾人皆得見之,非肅穆之容,故必合扇以鄣焉。



 罕、畢,像「畢、昴為天階」,故為前引,皆赤質,金銅飾,朱藤結網,金獸面。罕方,上有二螭首銜紅絲拂;畢圓,如扇。



 香鐙,唐制也。朱漆案,緋繡花龍衣,上設金塗香爐、燭臺。長竿二,輿士八人。金塗銀火鐐、香匙副之。



 大角,黑漆畫龍,紫繡龍袋。



 長鳴、次鳴、大小橫吹,五色衣幡,緋掌畫交龍。《樂令》,三
 品已上,緋掌畫蹲豹。



 犦槊。犦,擊聲也。一云象犦牛,善鬥,字從牛。唐金吾將軍執之。宋制,如節有袋,上加碧油。常置朝堂,車駕鹵簿出,則八枚前導;又四枚夾大將軍者,名衛司犦槊。



 槊,長矛也。木刃,黑質,畫雲氣。又有細槊,制同而差小。



 戟,有枝兵也。木為刃,赤質,畫雲氣,上垂交龍掌、五色帶,帶末綴銅鈴。又鈒戟,無掌,而有小橫木;鈒,插也,制本插車旁。又小戟與鈒戟同。



 殳、叉,戟之類。殳,無刃而短,黑飾兩末。叉,青飾兩末,並中白,畫雲氣,各綴朱絲拂。



 槍,槊也。唐羽林所執,制同槊而鐵刃,上綴朱絲拂。



 儀鍠,鉞屬也,秦、漢有之。唐用為儀仗,刻木如斧,塗以青,柄以黃,上綴小錦幡、五色帶。



 班劍,本漢朝服帶劍。晉以木代之,亦曰「象劍」,取裝飾斑斕之義。鞘以黃質,紫斑文,金銅飾,紫絲絳□錔。



 御刀,晉、宋以來有之。黑鞘,金花銀飾,靶軛,紫絲絳□錔。
 又儀刀,制同此,悉以銀飾,王公亦給之。



 刀盾。刀,本容刀也;盾,旁排也。一人分持。刀以木為之,無鞘,有環,紫絲絳□錔。盾,赤質,畫異獸。又朱藤絡盾,制悉同,唯綠藤綠質,皆持執之。



 幰弩,漢京尹、司隸前驅,持弓以射窺者。宋制,每弩加箭二,有□義,畫雲氣,仗內弩皆同。



 弓箭,每弓加箭二,有□義,同幰弩。



 車輻,棒也,形如車輪輻。宋制,朱漆八棱白干。



 柯舒,黑漆棒也,制同車輻,以金銅釘飾。



 鐙杖,黑漆弩柄也。以金銅為鐙及飾,其末紫絲絳系之。



 鳴鞭,唐及五代有之。《周官》條狼氏執鞭趨闢之遺法也。內侍二人執之,鞭鞘用紅絲而漬以蠟。行幸,則前騎而鳴之,大祀禮畢還宮,亦用焉;視朝、宴會,則用於殿庭。



 誕馬,散馬也。加金塗銀鬧裝鞍勒。乘輿以紅繡韉,六鞘,王公以下用紫繡及剜花韉。哲宗元祐七年,太常寺言:「誕馬,按《鹵簿圖》曰:舊並施鞍韉。景祐五年去之。昨納後,
 誕馬猶施鞍韉,今欲乞除去,仍依《鹵簿圖》。用纓、轡、緋屜。」



 御馬鞍勒之制,有金、玉、水晶、金塗四等鬧裝,□占鞢促結為坐龍,碾鈒鏤塵沙面、平面、窪面、方團、寸節、卷荷校具,皆垂六鞘,金銀裹鞍橋、銜鐙,朱黃絲絳轡秋,緋黃織繡或素園韉,衣蓋補用金銀線織或緋黃絁,鞭用紫竹,紅黃絲鞘,纓以紅、黃犛牛尾,金為鈌。每日,馬五匹供奉,鞍用玉及金塗,衣蓋補皆素。行幸則十四匹,加真金、水晶之飾。太宗至道二年詔:「先是,御馬以織成帊覆鞍勒,今後以
 廣絹代之。」



 馬珂之制,銅面,雕翎鼻拂,攀胸,上綴銅杏葉、紅絲拂。又胸前及腹下,皆有攀,綴銅鈴;後有跋塵、錦包尾。獨鹵簿中金吾衛將軍導駕者,皆有之。



 甲騎具裝,甲,人鎧也;具裝,馬鎧也。甲以布為里,黃絁表之,青綠畫為甲文,紅錦褖,青絁為下裙,絳韋為絡,金銅鈌,長短至膝。前膺為人面二,自背連膺,纏以錦螣蛇。具裝,如常馬甲,加珂拂於前膺及後秋。



 球杖,金塗銀裹,以供奉官騎執之,分左右前導。大禮,用百人,花腳帕頭、紫繡衣癸袍襖。常出,三十人,公服,皆騎導。



 雞竿,附竿為雞形,金飾,首銜絳幡,承以彩盤,維以絳索,揭以長竿。募衛士先登,爭得雞者,官給以纈襖子;或取絳幡而已。大禮畢,麗正門肆赦則設之。其義則雞為巽神,巽主號令,故宣號令則象之。陽用事則雞鳴,故布宣陽澤則象之。一曰「天雞星動為有赦」,故王者以天雞為度。金雞事,六朝已有之,或謂起於西京。南渡後,則自紹
 興十三年始也。



 大駕鹵簿巾服之制:金吾上將軍、將軍、六統軍、千牛、中郎將,服花腳帕頭、抹額、紫繡袍,佩牙刀,珂馬。諸衛大將軍、將軍、中郎將、折沖、果毅、散手翊衛,服平巾幘、紫繡袍、大口褲、錦螣蛇、銀帶,佩橫刀,執弓箭。千牛將軍,服平巾幘、紫繡袍、大口褲、銀帶、靴靿,佩橫刀,執弓箭,珂馬。千牛,服花腳帕頭、緋繡袍、抹額、大口褲、銀帶、靴靿。前馬隊內折沖及執槊者,服錦帽、緋繡袍、銀帶。監門校尉、六軍押仗,服帕頭、紫繡裲襠。隊正,服平巾幘、緋
 繡袍、大口褲。諸衛主率都尉,引駕騎,持鈒隊內校尉、旅帥,執衛司殳仗犦槊,金吾十六騎,班劍、儀刀隊,親勛翊衛,執大角人,並服平巾幘、緋繡裲襠、大口褲,佩橫刀,執弓箭。金吾押牙,服金鵝帽、紫繡袍、銀帶,儀刀。金吾持纛者,服烏紗帽、皂衣、褲、□奚襪。金吾押纛,服帕頭、皂繡衫、大口褲、銀帶、烏皮靴。執金吾犦槊,服錦袍帽、臂鞁、銀帶、烏皮靴。


清游隊、佽飛、執副仗槊,服甲騎具裝、錦臂鞁,佩橫刀,執弓箭,白褲。朱雀隊執旗及執牙門旗,執絳引幡、黃
 麾幡者,並服緋繡衫、抹額、大口褲、銀帶。執殳仗,前後步隊、真武隊執旗,前後部黃麾,執日月合璧等旗,青龍白虎隊、金吾細仗內執旗者,並服五色繡袍、抹額、行縢、銀帶;執白乾棒人,加銀褐捍腰。執龍旗及前馬隊內執旗人,服五色繡袍、銀帶、行縢、大口褲。執弓箭、執龍旗副竿人,服錦帽、五色繡袍、大口褲、銀帶。執弩、弓箭人,服錦帽、青繡袍、銀帶。前後步隊人,服五色鍪甲、錦臂鞁、□奚襪、褲、銀帶。朱雀隊內執弓箭、弩、槊,虞候佽飛,執長壽幢、寶輿
 法物人,並服平巾幘、緋繡袍、大口褲、銀帶。援寶,執絳麾、真武幢叉人,並服武弁、紫繡衫。持鈒隊,殿中黃麾、傘、扇、腰輿、香鐙、華蓋,指南、進賢等車駕士,相風、鐘漏等輿輿士,並服武弁、緋繡衫。駕羊車童子,服垂耳髻、青頭
 \gezhu{
  須巾}
 、青繡大袖衫、褲、勒帛、青耳履。執引駕龍墀旗、六軍旗者,服錦帽、五色繡衫、錦臂鞁、銀帶。引夾旗及執柯舒、鐙仗者,服帖金帽,餘同上。執花鳳、飛黃、吉利旗者,服銀褐繡衣、抹額、銀帶。夾轂隊,服五色質鍪鎧、錦臂鞁、白行縢、紫帶、
 □奚襪。驍衛翊衛三隊,服平巾幘、緋繡袍、大口褲、錦螣蛇。五輅、副輅、耕根車駕士,服平巾幘、青繡衫、青履襪。教馬官,服帕頭、紅繡抹額、紫繡衫、白褲、銀帶。掌輦、主輦,服武弁、黃繡衫、紫繡誕帶。攏御馬者,服帖金帽、紫繡大袖衫、銀帶。執真武幢者,服武弁、皂繡衫、紫繡誕帶。五牛旗輿士,服武弁、五色繡衫、大口褲,銀帶。掩後隊,服黑鍪甲、錦臂鞁、行縢。



 鼓吹令、丞,服綠褲褶冠、銀褐裙、金銅革帶、緋白大帶、履襪。太常寺府史、典事、司天令史,服帕頭、綠衫、
 黃半臂。太常主帥□鼓、金鉦、節鼓人,服平巾幘、緋繡袍、大口褲,抹帶、錦螣蛇;歌、拱宸管、簫、笳、笛、觱慄,無螣蛇。太常大鼓、長鳴、小鼓、中鳴,服黃雷花袍、褲、抹額、抹帶。太常鐃、大橫吹,服緋苣文袍、褲、抹額、抹帶。太常羽葆鼓、小橫吹,服青苣文袍、褲、抹額、抹帶。排列官、令史、府史,服黑介幘、緋衫、白褲、白勒帛。司辰、典事、漏刻生,服青褲褶冠、革帶。殿中少監、奉御、供奉、排列官,引駕仗內排列承直官、大將、金吾引駕、押仗、押旗,服帕頭、紫公服、烏皮靴。尚輦
 奉御、直長、乘黃令丞、千牛長史、進馬四色官,服帕頭、綠公服、白褲、金銅帶、烏皮靴。殿中職掌執傘扇人,服帕頭、碧襴、金銅帶、烏皮靴。



 舊衣黃,太平興國六年,並內侍省並改服以碧。



 凡繡文:金吾衛以闢邪,左右衛以瑞馬,驍衛以雕虎,屯衛以赤豹,武衛以瑞鷹,領軍衛以白澤,監門衛以師子,千牛衛以犀牛,六軍以孔雀,樂工以鸞,耕根車駕士以鳳銜嘉禾,進賢車以瑞麟,明遠車以對鳳,羊車以瑞羊,指南車以孔雀,記里鼓、黃鉞車以對鵝,白鷺車以翔鷺,鸞旗車以
 瑞鸞,崇德車以闢邪,皮軒車以虎,屬車以雲鶴,豹尾車以立豹,相風烏輿以烏,五牛旗以五色牛,餘皆以寶相花。



 六引內巾服之制:清道官,服武弁、緋繡衫、革帶。持幰弩、車輻棒者,服平巾赤幘、緋繡衫、赤褲、銀帶。青衣,服平巾青幘、青褲褶。持戟、傘、扇、刀盾者,服黃繡衫、抹額、行縢、銀帶。持幡蓋者,服繡衫、抹額、大口褲、銀帶。內告止幡、曲蓋以緋,傳教幡、信幡、絳引幡以黃。執誕馬轡、儀刀、麾、幢、節、夾槊、大角者,服平巾幘、緋繡衫、大口褲、銀帶。大駕鹵
 簿內,執轡,並錦絡衫帽。持弓箭、槊者,服武弁、緋繡衫、白褲。駕士,服錦帽、繡戎服大袍、銀帶。弓箭以青,槊以紫。持□鼓者,服平巾幘、緋繡對鳳袍、大口褲、白抹帶、銀螣蛇。鐃吹部內,服平巾幘、緋繡袍、白抹帶、白褲,餘悉同大駕前後部。



 其繡衣文:清道以雲鶴,幰弩以闢邪,車輻以白澤,駕士司徒以瑞馬,牧以隼,御史大夫以獬豸,兵部尚書以虎,太常卿以鳳,縣令以雉,樂工以鸞,餘悉以寶相花。



 太祖建隆四年,範質議:按《開元禮》,武官陪立大仗,加
 螣蛇裲襠,如袖無身,以覆其膊胳,蓋掖下縫也。從肩領覆臂膊,共一尺二寸。又按《釋文》、《玉篇》相傳云:其一當胸,其一當背,謂之「兩當」。今詳裲襠之制,其領連所覆膊胳,其一當左膊,其一當右膊,故謂之「起膊」。今請兼存兩說擇而用之,造裲襠,用當胸、當背之制。宣和元年,禮制局言:鼓吹令、丞冠,又名「褲褶冠。」今鹵簿既除褲褶,冠名不當仍舊,請依舊記如《三禮圖》「季貌冠」
 制。從之。



\end{pinyinscope}