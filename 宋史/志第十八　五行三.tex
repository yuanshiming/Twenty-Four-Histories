\article{志第十八 五行三}

\begin{pinyinscope}

 木



 曲直,木之性也。木失其性,則為妖祥。舊說以狂咎、木冰、恆雨、服妖、龜孽、雞禍、青眚、青祥之類,皆屬之木,今從之。



 太平興國六年正月,瑞安縣民張度解木五片,皆有「天
 下太平」字。



 至道六年,修昭應宮,有木斷之,文如點漆,貫徹上下,體若梵書。十一月,襄州民劉士家生木。有文如龍、魚、鳳、鶴之狀。七年五月,撫州修天慶觀,解木有文如墨畫雲氣、峰巒、人物、衣冠之狀。七月,彰明縣崇仙觀柱有文為道士形及北斗七星象。



 大中祥符八年,晉州慶唐觀古柏中別生槐,長丈餘。



 天聖元年二月,河陽柳二本連理。六月,河陽甗、棗各連理。五年正月,綿谷縣松柏同本異幹。九年十月,公井縣冬青木連理。



 明道元年八
 月,黃州橘木及柿木連枝。



 康定元年十月,始興縣柑兩本連理。



 慶歷三年十二月,澧州獻瑞木,有文曰「太平之道」。六年九月甲辰,登州有巨木浮海而出者三千餘。



 治平四年六月,汀州進桐木板二,有文曰「天下太平」。



 熙寧元年三月,簡州木連理。是歲,英州因雷震,一山梓樹盡枯而為龍腦,價為之賤,至京師,一兩才值錢一千四百。二年,建州民楊緯言:「元年三月,大雷雨,所居之西有黃龍見,下獲一木如龍,而形未具。七月,大雷雨,復有龍飛
 其下。及霽,木龍尾、翼、足皆具,歸合舊木,宛然一體。」圖像以進。十年八月乙巳,惠州柚木有文曰「王帝萬年,天下太平」。



 元豐元年五月,劍州木連理。三年六月己未,饒州長山雨木子數畝,狀類山芋子,味香而辛,土人以為桂子,又曰「菩提子」,明道中嘗有之。是歲大稔。十二月,泌陽縣甘棠木連理。六年五月,衛真縣洞霄宮枯槐生枝葉。



 元祐元年八月己丑,杭州民俞舉慶七世同居,家園木連理。五年四月,德州木連理。



 元符元年八月,施州李木
 連理。二年九月,眉山縣榿木二株,異根同乾,木枝相附。



 崇寧四年正月,襄城縣李、梨木連理。



 大觀元年三月,湟州欄木生葉。八月,瑞州、永興軍並木連理。二年十二月,岢嵐軍園池生瑞木。



 政和三年七月,玉華殿萬年枝木連理。南雄州楓木連理。十月,武義縣木根有「萬宋年歲」四字。四年,建州木連理。六月,沅陵縣江漲,流出楠木二十七,可為明堂梁柱,蔡京等拜表賀。九月丙申,彭城縣柏開華。十二月辛丑,元氏縣民王寘屋柱槐木再生枝
 葉,高四十餘尺。是歲,邵州海棠木連理,澤州、臺州槐木連理,荊門軍紫薇木連理。六年,坊、兗、洪、明、夔、徐、新、全、隰、太平州並木連理。梅州枯木生枝。



 宣和二年四月,永州民劉思析薪,有「天下太平」字。



 紹興十四年四月,虔州民毀欹屋析柱,木里有文曰:「天下太平」,時守臣薛弼上之,方大亂,近木妖也。二十年八月,福州沖虛觀皂莢木翠葉再實。二十一年,建德縣定林寺桑生李實,慄生桃實,占曰:「木生異實,國主殃。」二十五年十月,贛州獻太平木。
 時秦檜擅朝,喜飾太平,郡國多上草木之妖以為瑞。紹興間,漢陽軍有插榴枝於石罅,秀茂成陰,歲有花實者。初,郡獄有誣服孝婦殺姑,婦不能自明,屬行刑者插髻上華於石隙,曰:「生則可以驗吾冤。」行刑者如其言,後果生。



 淳熙十六年三月,揚州桑生瓜,櫻桃生茄,此草木互為妖也。七月,晉陵縣民析薪,中有木字曰「紹熙五年」,如是者二。是時,紹熙猶未改元,其後果止五年,此近木妖也。



 紹熙四年,富陽縣慄生來禽實。五年,行都雨木,與《唐
 志》貞元陳留雨木同占,木生於下而自上隕者,將有上下易位之象。



 嘉定六年五月己巳,嚴州淳安、遂安、桐廬三縣大木自拔,占曰:「木自拔,國將亂。」



 景定四年五月,成都太祖廟側大木僕,忽起立,生三芽。



 德祐二年正月戊辰,寶應縣民析薪,中有「天太下趙」四字,獻之,制置使李庭芝嘗以錢五千。



 咸平六年十一月庚戌,雨木冰。



 大中祥符五年正月戊寅,京師雨木冰。



 天禧五年正月戊寅,京師雨木冰。



 慶歷
 三年十二月丁巳,雪木冰,占曰:「兵象也。」



 嘉祐元年正月,雨木冰。



 治平二年十月乙巳,雨木冰。



 熙寧三年十月、八年正月、九年正月,京師雨木冰。



 元祐八年二月,京師大寒,霰、雪,雨木冰。



 宣和五年十月乙酉,雨木冰。



 靖康元年十月乙卯,雨木冰。二年正月丁酉,雨木冰。



 紹熙五年十一月辛亥,雨木冰。



 宣和六年,御樓觀燈,時開封尹設次以彈壓於西觀下,帝從六宮於其上,以觀天府之斷決者,簾幕深密,下無由知。眾中忽有人躍出,黑色布衣,若寺
 僧童行狀,以手畫簾,出指斥語。執於觀下,帝怒甚,令中使傳旨治之。棰掠亂下,又加炮烙,詢其誰何,略不一語,亦無痛楚之狀。又斷其足筋,俄施刀臠,血肉狼籍。帝大不悅,為罷一夕之歡,竟不得其何人,付獄盡之。七年八月,都城東門外鬻菜夫至宣德門下,忽若迷惘,釋荷擔向門戟手,出悖罵語。且曰:「太祖皇帝、神宗皇帝使我來道,尚宜速改也。」邏卒捕之,下開封獄,一夕方省,則不知向之所為者,乃於獄中盡之。



 建炎二年十一月,高
 宗在揚州,郊祀後數日,有狂人具衣冠,執香爐,攜絳囊,拜於行宮門外。自言:「天遣我為官家兒。」書於囊紙,刻於右臂,皆是語。鞫之不得姓名,高宗以其狂,釋不問。明年二月,金人犯維揚。三月,有明受之變。



 紹興元年四月庚辰,閬州有狂僧衰絰哭於郡譙門曰:「今日佛下世。」且言且哭,實隆祐太后上仙日雲。閬距行都萬里,逾月而遺詔至。



 淳熙十四年正月,紹興府有狂人突入恩平郡王第,升堂踐王坐曰:「我太上皇孫,來赴。」郡鞫訊,終不語,亦狂咎
 也。是冬,高宗崩。明年八月,王薨。



 紹熙二年十二月庚辰昧爽,成都府有人衰服入帳門,大呼閫帥京鏜姓名,亦狂咎也。



 建隆元年十月,蔡州大霖雨,道路行舟。



 開寶二年八月,帝駐潞州,積雨累日未止。九月,京師大雨霖。五年,京師雨,連旬不止。河南、河北諸州皆大霖雨。九年秋,大霖雨。



 太平興國二年,道州春夏霖雨不止,平地二丈餘。五年五月,京師連旬雨不止。七年六月,齊州逮捕臨邑尉王
 坦等六人。系獄未具,一夕,大風雨壞獄戶,王坦等六人並壓死。



 雍熙二年八月,京師大霖雨。



 淳化元年六月,隴城縣大雨,壞官私廬舍殆盡,溺死者百三十七人。三年九月,京師霖雨。四年七月,京師大雨,十晝夜不止,朱雀、崇明門外積水尤甚,軍營、廬舍多壞。是秋,陳、穎、宋、亳、許、蔡、徐、濮、澶、博諸州霖雨,秋稼多敗。五年秋,開封府、宋、亳、、陳、穎、泗、壽、鄧、蔡、潤諸州雨水害稼。



 咸平元年五月,昭州大霖雨,害民田,溺死者百五十七人。



 景德三年八月,青
 州大雨,壞鼓角樓門,壓死者四人。



 大中祥符二年八月,無為軍大風雨,折木,壞城門、軍營、民舍,壓溺千餘人。十月,兗州霖雨害稼。三年四月,升州霖雨。五月辛丑,京師大雨,平地數尺,壞軍營、民舍,多壓者,近畿積潦。五年九月,建安軍大霖雨,害農事。



 天禧四年七月,京師連雨彌月。甲子夜,大雨,流潦泛溢,民舍、軍營圮壞大半,多壓死者。自是頻雨,及冬方止。



 乾興元年二月,蘇、湖、秀州雨,壞民田。



 天聖四年六月戊寅,莫州大雨,壞城壁。七年,自
 春涉夏,雨不止。



 明道二年六月癸丑,京師雨,壞軍營、府庫。



 景祐三年七月庚子,大雨震電。



 慶歷,六年七月丁亥,河東大雨,壞忻、代等州城壁。



 皇祐二年八月,深州大雨,壞民廬舍。四年八月癸未,京城大風雨,民廬摧圮,至有壓死者。



 嘉祐二年八月,河北沿邊久雨,瀕河之民多流移。五月丁未,晝夜大雨。六月乙亥,雨壞太社、太稷壇。三年八月,霖雨害稼。六年七月,河北、京西、淮南、兩浙、江南東西淫雨為災。閏八月,京師久雨。是歲頻雨,及冬方止。



 治平元年,京師自夏歷秋,久雨不止,摧真宗及穆、獻、懿三后陵臺。



 熙寧元年八月,冀州大雨,壞官私廬舍、城壁。七年六月,陜州大雨,漂溺陜、平陸二縣。



 元豐四年七月,泰州海風駕大雨,漂浸州城,壞公私舍數千楹。



 元祐二年七月丁卯,以雨罷集英殿宴。



 元符二年九月,以久雨罷秋宴。三年七月,久雨,哲宗大升輿在道陷泥中。



 建中靖國元年二月,久雨,時欽聖憲肅皇后、欽慈皇后二陵方用工,詔京西祈晴。



 崇寧元年七月,久雨,壞京城廬舍,
 民多壓溺而死者。三年六月,久雨。四年五月,京師久雨。又自七月至九月,所在霖雨傷稼,十月始霽。



 靖康元年四月,京師大雨,天氣清寒。又自五月甲申至六月,暴雨傷麥,夏行秋令。



 建炎二年春,淫雨。三年二月癸亥,高宗初至杭州,久霖雨,占曰:「陰盛,下有陰謀。」時苗傅、劉正彥為亂。五月,霖雨,夏寒。



 紹興元年,行都雨,壞城三百八十丈。是歲,婺州雨,壞城。三年,雨,自正月朔至於二月。七月,四川霖雨,至於明年正月。四年六月,淫雨害稼,蘇、湖二
 州為甚。九月,久雨,時劉豫連金人入寇;十月,高宗親征而霽。五年三月,霖雨,傷蠶麥,行都雨甚。九月,雨,至於明年正月。六年五月,久雨不止。七年十月,高宗如建康,久雨。八年三月,積雨,至於四月,傷蠶麥,害稼。二十一年夏,襄陽府大雨十餘日。二十三年六月,大雨,壞軍壘、民田。三十年五月,久雨,傷蠶麥,害稼。八月,施州大風雨。三十二年六月,浙西大霖雨。



 隆興元年三月,霖雨,行都壞城三百三十餘丈。二年六月,陰雨。七月,浙西、江東大雨害
 稼。八月,風雨逾月。



 乾道元年二月,行都及越、湖、常、潤、溫、臺、明、處九郡寒,敗首種,損蠶麥。二年正月,淫雨,至於四月。夏寒。江、浙諸郡損稼,蠶麥不登。三年五月丙午,泉州大雨,晝夜不止者旬日。八月,淫雨,江浙淮閩禾、麻、菽、麥、粟多腐。四年四月,陰雨彌月。六年五月,連雨六十餘日。十一月,連雨。辛巳,郊祀,雲開於圜丘,百步外有澍雨。八年四月,四川陰雨七十餘日。六月壬寅,大雨徹晝夜,至於己酉。九年閏正月,淫雨。



 淳熙二年夏,建康府霖雨,壞
 城郭。三年五月,淮、浙積雨損禾麥。八月,浙東西、江東連雨。癸未、甲申,行都大風雨。九月,久雨。十月癸酉,孝宗出手詔決獄,援筆而風起開霽。四年九月丁酉、戊戌,紹興府餘姚、上虞二縣大風雨。五年閏六月己亥,階州暴雨,至於戊申。乙巳,興化軍、福州福清縣暴風雨夜作。六年四月,衢州霖雨。九月,連雨;己巳,將郊而霽。八年四月,雨腐禾麥。五月,久雨,敗首種。十年五月,信州霖雨,自甲戌至於辛巳。八月,福州大霖雨,自己未至於九月乙丑,吉
 州亦如之。十一年四月,淫雨。戊寅,建康府、太平州大霖雨。六月甲申,處州龍泉縣暴雨。十二年五月、六月,皆霖雨。十三年秋,利州路霖雨,敗禾稼穜稑,金、洋、階、成、岷、鳳六州亦如之。十五年五月,荊、淮郡國連雨。戊午,祁門縣霖雨。十六年四月,西和州霖雨,害禾麥。五月,浙西、湖北、福建、淮東、利西諸道霖雨。



 紹熙元年春,久陰連雨,至於三月。夏,階、成、岷、鳳四州霖雨傷麥。二年二月,贛州霖雨,連春夏不止,壞城四百九十丈,圮城樓、敵樓凡十五所。
 四月,福建路霖雨,至於五月。七月,利路久雨,傷種麥。癸亥,興州暴雨連日。八月,行都久雨。三年五月,江東、湖北路連雨。常德府大雨徹晝夜,自壬辰至於庚子。寧國府、池州、廣德軍自己亥至於六月辛丑朔,雨甚,祁門縣至於庚戌。七月壬申,天臺、仙居二縣大雨連旬。淮西路、鎮江、襄陽府皆害禾麥。八月,普州雨害稼。四年四月,霖雨,至於五月,浙東西、江東、湖北郡縣壞圩田,害蠶、麥、蔬、稑,紹興、寧國府尤甚。鎮江府大雨,自辛未至於丙子,淮西
 郡縣自丙子至於戊寅。五年八月,霖雨,畿縣、浙東西皆害稼。九月,雨,至於十月癸巳,大雨三晝夜不止,江東西、福建郡縣皆苦雨。



 慶元元年正月,霖雨。甲辰,帝蔬食露禱,丙午霽。二月,又雨,至於三月,傷麥。五月,霖雨。七月,雨,至於八月。二年六月壬申,臺州焱風暴雨連夕。八月,行都霖雨五十餘日。三年七月,雨連月。四年八月,久雨。五年五月,行都雨,壞城,夜壓附城民廬,多死者。六月,浙東、西霖雨,至於八月。六年五月庚午,嚴州霖雨,連五晝夜
 不止。



 嘉泰二年六月,福建路連雨,至於七月丁未,大風雨為災。三年八月,久雨。



 開禧元年七月,利路郡縣霖雨害稼。閏月,盱眙軍陰雨,至於九月,敗禾稑。十月,行都淫雨,至於明年春。二年春,淫雨,至於三月。



 嘉定二年五月戊戌,連州大雨連晝夜。六月,利、閬、成、西和四州霖雨。七月壬辰,臺州大風雨夜作。三年三月,陰雨六十餘日。五月,淫雨,至於六月,首種多敗,蠶麥不登。四年八月,霖雨,至於九月。五年春,淫雨,至於三月,傷蠶麥。十一月,雨雪
 積陰,至於明年春。六年春,淫雨,至於二月。丁亥,雨雪集霰。五月,陰雨經日。辛酉,嚴州霖雨。月戊子,紹興府大風雨,浙東、西雨,至於七月。七年九月,陰雨,至於十月,害禾麥。九年四月、六月,大霖雨,浙東、西郡縣尤甚。十年三月,連雨,至於四月。十月,霖雨害稼。十一年六月,霖雨,浙西郡縣尤甚。十二年六月,霖雨彌月。十五年七月,浙東、西霖雨為災。十六年五月,霖雨,浙西、湖北、江東、淮東尤甚。八月,大風雨害稼。十七年八月,霖雨。



 乾德四年二月長春節,甘露降江寧府報恩院。五年二月,甘露降江陵府玉泉寺松樹。



 開寶元年十二月,甘露降蔡州僧院柏樹。



 太平興國三年正月,甘露降壽州廨。四年五月,甘露降河東縣廨叢竹凡三日。七年四月丙戌,知漢州安守亮獻柏葉上甘露一器。九年三月丙子,甘露降西京南太一宮新城。



 雍熙三年四月庚子,甘露降後院草木。四年十二月,甘露降興化軍羅漢峰前五松。



 端拱二年二月,甘露降壽州廨園柏及資聖寺檜。



 淳
 化二年十二月,資州廨及延壽觀、德純寺甘露降松柏,凡六日。三年正月,舒州,二月,衢州;四月,舒州;四年六月,舒州:並甘露降。



 至道三年四月,蘄州;三年五月,泉州;六月蘇州,甘露降。



 咸平元年四月,甘露降平戎軍廨果樹,凡九十餘本。十一月,甘露降亳州真觀靈寶柏樹。二年五月,太平州、潯州;三年二月,泉州;十一月,潯州;四年二月,龔州;五年正月,桂州;十一月,許州,並甘露降。



 景德元年,義寧縣;二年正月,鬱林州;二月,晉州及神山縣;三年
 正月,梓州;四月,遂州;十二月,榮州、懷安軍,甘露降。



 大中祥符元年十二月,上饒縣、信陽軍;二年正月,信陽軍、陳、鄂二州;三月,陵、升、梓三州;三年二月,柳州、懷安軍;閏二月,富順監;五月,澤、耀、晉、益四州;四年正月,梓州;三月,澤州;四月,常州;五年四月,遂州;五月,無為軍;六月,梓州;七月,真定府;十一月,榮州開元寺;六年三月,梓州;六月,鄜州;八月,遂州;九月,信州;十月,亳州太清宮;十一月,潯州;十二月,榮州、南儀州;七年二月,鳳翔府天慶觀;五月,鄆
 州;十月,亳州太清宮;十一月,彭州天慶觀;八年正月,中江縣;二月,果州;十月,衢州:九年十一月,玉清昭應宮,並甘露降。



 天禧元年正月,貴州天慶觀;二月,玉清昭應宮;三月,後苑;四月,會靈觀;五月,廬州通判廳及後土祠;十二月,昭州天慶觀;二年十二月,榮州開元寺、懷安軍天慶觀;三年四月,舒州;五月,益州;四年三月,邵武軍;十二月,平泉縣;五年三月,泉州;十一月,韶州,並甘露降。



 天聖元年正月,柳州;十一月,河南府;二年五月,鳳州;十月,涇
 州;四年,榮州、懷安軍;六年,太平州;七年正月,益州;九年正月,榮州,並甘露降。



 明道元年十一月,韶州、梓州甘露降。



 景祐四年十一月,成德軍;慶歷四年正月,桂州;皇祐三年十二月,吉州;嘉祐七年三月,眉州、蓬州;九月,陵州,並甘露降。



 熙寧元年距元豐八年,甘露降凡二十餘處。



 元祐元年距元符三年,亦如之。



 大觀初,甘露降於九成宮帝鼐室。三年冬,降於尚書省及六曹,禦制七言四韻詩賜執政已下。其後內自禁中及宣和殿、延福宮、神霄
 宮,下至三學、開封府、大理寺、宰臣私第,皆有之,歲歲拜表稱賀。



 建隆初,蜀孟昶末年,婦女競治發為高髻,號「朝天髻」。未幾,昶入朝京師。江南李煜末年,有衛士秦友登壽昌堂榻,覆其鞋而坐,訊之,風狂不寤。識者云:「鞋,履也,李氏將覆於此地而為秦所有乎?『履』與『李』、『友』與『有』同音,趙與秦,同祖也。」又煜宮中盛雨水染淺碧為衣,號「天水碧」。未幾,為王師所克,士女至京師猶有服之者。天水,國之姓望
 也。



 淳化三年,京師裏巷婦人競剪黑光紙團靨,又裝鏤魚腮中骨,號「魚媚子」以飾面。黑,北方色;魚,水族,皆陰類也。面為六陽之首,陰侵於陽,將有水災。明年,京師秋冬積雨,衢路水深數尺。



 景德四年春,京城小兒裂裳為小兒旗,系竿首,相對揮颭,兵斗之象也。是歲,宜州卒陳進為亂,出師討平之。



 紹興二十一年,行都豪貴競為小青蓋,飾赤油火珠於蓋之頂,出都門外,傳呼於道。珠者,乘輿服御飾升龍用焉,臣庶以加於小蓋,近服妖,亦僭咎
 也。二十三年,士庶家競以胎鹿皮制婦人冠,山民採捕胎鹿無遺。時去宣和未遠,婦人服飾猶集翠羽為之,近服妖也。二十七年,交址貢翠羽數百,命焚之通衢,立法以禁。



 紹熙元年,里巷婦女以琉璃為首飾。《唐志》琉璃釵釧有流離之兆,亦服妖也,後連年有流徙之厄。



 理宗朝,宮妃系前後掩裙而長窣地,名「趕上裙」;梳高髻於頂,曰「不走落」;束足纖直,名「快上馬」;粉點眼角,名「淚妝」;剃削童發,必留大錢許於頂左,名「偏頂」,或留之頂前,束以彩繒,
 宛若博焦之狀,或曰「鵓角」。



 咸淳五年,都人以碾玉為首飾。有詩云:「京師禁珠翠,天下盡琉璃。」



 太平興國三年三月,鑿金明池,既掘地,有龜出,殆逾萬數。



 大中祥符二年四月,有黑龜甚眾,沿汴水而下。



 至和元年二月,信州貢綠毛龜。



 大觀元年閏十月丙戌,都水使者趙霆行河,得兩首龜以為瑞,蔡京信之,曰:「此齊小白所謂象罔見之而霸者也。」鄭居中曰:「首豈容有二,而京主之,意殆不可測。」帝命棄龜金明池。



 政和四年,瑞州
 進六目龜。五年,博州進白龜。



 紹興八年五月,汴京太康縣大雷雨,下冰龜數十里,隨大小皆龜形,具首足卦文。



 乾道五年,舒州民獻龜,駢生二首,不能伸縮。郡守張棟縱之灊山,近龜孽也。



 嘉定十四年春,楚州境上龜大小死者蔽野。



 咸平三年八月,黃州群雞夜鳴,至冬不止。



 紹興初,陳州民家雞忽人言,近雞禍也。松陽縣民家雞生三足,縣治有雞伏卵,毛生殼外,近雞禍,亦毛孽也。



 乾道六年,西安
 縣官塘有物,雞首人身,高丈餘,晝見於野。



 慶元三年,饒州軍營雞卵出蛇,近雞孽,亦蛇孽也。婺源縣張村民家雌雞化為雄,烹之,形冠距而腹卵孕。同里洪氏家雄雞伏子,中一雛三足。



 咸淳五年,常州雞羽生距。



 建隆元年夏,相、金、均、房、商五州鼠食苗。二年五月,商州鼠食苗。



 乾德五年九月,金州鼠食苗。



 太平興國七年十月,岳州鼠害稼。



 紹興十六年,清遠、翁源、真陽三縣鼠食稼,千萬為群。時廣東久旱,凡羽鱗皆化為鼠。有獲鼠
 於田者,腹猶蛇文,漁者夜設綱,旦視皆鼠。自夏徂秋,為患數月方息,歲為饑,近鼠妖也。



 乾道九年,隆興府鼠千萬為群,害稼。



 淳熙五年八月,淮東通、泰、楚、高郵黑鼠食禾既,歲大饑。時江陵府郭外,群鼠多至塞路,其色黑、白、青、黃各異,為車馬踐死者不可勝計,逾三月乃息。



 紹熙四年,饒州民家二小鼠食牛角,三徙牛牢不免,角穿肉瘠以斃,近鼠妖也。



 慶元元年六月,番陽縣民家一貓帶數十鼠,行止食息皆同,如母子相哺者,民殺貓而鼠舐其
 血。鼠象盜,貓職捕,而反相與同處,司盜廢職之象也,與唐龍朔洛州貓鼠同占。



 紹興三年八月辛亥,尚書省後樓無故自壞。



 慶元元年夏,建昌軍民居木柱有聲如牛鳴者,三日乃止。



 咸淳九年,丞相賈似道起復之日,在越上私第,方拜家廟,忽聞內有裂帛聲,眾賓愕然,密詢左右,知家廟棟裂,皆逡巡而退。



\end{pinyinscope}