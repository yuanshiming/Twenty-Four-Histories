\article{志第一百五 輿服四}

\begin{pinyinscope}

 諸
 臣
 服上



 諸臣祭服。唐制,有袞冕九旒,鷩冕八旒,毳冕七旒,絺冕六旒,玄冕五旒。宋初,省八旒、六旒冕。九旒冕:塗金銀花額,犀、玳瑁簪導,青羅衣繡山、龍、雉、火、虎蜼五章,緋羅裳
 繡藻、粉米、黼、黻四章,緋蔽膝繡山、火二章,白花羅中單,玉裝劍、佩,革帶,暈錦綬,二玉環,緋白羅大帶,緋羅襪、履,親王、中書門下奉祀則服之。其冕無額花者,玄衣纁裳,悉畫,小白綾中單,師子錦綬,二銀環,餘同上,三公奉祀則服之。七旒冕:犀角簪導,衣畫虎蜼、藻、粉米三章,裳畫黼、黻二章,銀裝佩、劍,革帶,餘同九旒冕,九卿奉祀則服之。五旒冕:青羅衣裳,無章,銅裝佩、劍,革帶,餘同七旒冕,四品、五品為獻官則服之;六品以下無劍、佩、綬;紫檀衣,朱
 裳,羅為之,皂大綾綬,銅裝劍、佩,御史、博士服之。平冕無旒,青衣纁裳,無劍、佩、綬,餘同五旒冕,太祝、奉禮服之。



 慶歷三年,太常博士餘靖言:「《周禮》司服之職,掌王之吉服,大裘而冕無旒,以祀昊天上帝,祀五帝亦如之。袞冕十有二旒,其服十有二章,以享先王。鷩冕八旒,其服七章,以享先公,亦以饗射。毳冕七旒,其服五章,以祀四望、山川。絺冕六旒,其服三章,以祭社稷、五祀。玄冕五旒,其服無章,以祭小祀。此皆天子親行祠事所服,冕服悉因
 所祀大小神鬼以為制度。今大祠、中祠所遣獻官並用上公九旒、九章冕服,以為初獻,其餘公卿亦皆七旒冕服,全無等降;小祠則公服行事,乖戾舊典。宜詳《周禮》,因所祭鬼神,以為獻官冕服之制。」詔下禮官議,奏曰:「聖朝之制,唯皇帝親祠郊廟及朝會大禮服袞冕外,餘冕皆不設。其每歲常祀,遣官行事,攝公則服一品九旒冕,攝卿則服三品七旒冕,自從品制為服,不以祠之大小為差。至於小祠獻官,舊以公服行事,則有違典禮。案《衣服
 令》,五旒冕,衣裳無章,皂綾綬,銅裝劍、佩,四品以下為獻官則服之。今小祠獻官,既不攝公、卿,則盡屬四品以下,當有祭服。請除公、卿祭服仍舊從本品外,小祠所遣獻官,並依令文祭服行事。若非時告祭,用香幣禮器行事之處,亦皆準此。」詔施行焉。



 皇祐四年,同知太常禮院邵必言:「伏見監祭使、監禮各冠五旒冕,衣裳無章,色以紫檀。案《周禮》六冕之制,凡有旒者,衣裳皆有章,惟大裘冕無旒,衣裳無章。一命大夫之冕無旒,衣裳亦無章。今監
 祭、監禮所服冕五旒,侯伯之冕也,而衣無章,深所不稱;色以紫檀,又無經據。竊詳監祭、監禮既非祠官,則御史、博士爾,而服用五等,蓋非所宜,而且有旒無章。況國家南郊大禮,太常卿止服朝服,前導皇帝,明非祠官也。今後監祭者請冠獬豸、監禮者冠進賢為稱。」詔不充。



 元豐元年,詳定禮文所言:「國家服章,視唐尤為不備。於令文,祀儀有九旒冕、七旒冕、五旒冕,今既無冕名,而有司仍不制七旒冕,乃有四旒冕,其非禮尤甚。又服之者不以
 官秩上下,故分獻四品官皆服四旒冕,博士、御史則冕五旒而衣紫檀,太祝、奉禮則服平冕而無佩玉,此因循不講之失也。且古者朝、祭異服,所以別事神與事君之禮。今皇帝冬至及正旦御殿,服通天冠、絳紗袍,則百官皆服朝服,乃禮之稱。至親祠郊、廟,皇帝嚴裘冕以事神,而侍祠之官止以朝服,豈禮之稱哉。至於景靈宮分獻官,皆服朝服,尤為失禮。伏請親祠郊、廟、景靈宮,除導駕、贊引、扶侍、宿衛之官,其侍祠及分獻者,並服祭服。如所
 考制度,修制五冕及爵弁服,各正冕弁之名。又國朝祀儀,祭社稷、朝日、夕月、風師、雨師皆服袞冕,其蠟祭、先蠶、五龍亦如之;祭司命、戶、灶、門、厲、行皆服鷩冕,壽星、靈星、司中、司寒、中溜、馬祭皆服毳冕,皆非是。今天子六服,自鷩冕而下,既不親祠,廢而不用,則諸臣攝事,自當從王所祭之服。伏請依《周禮》,凡祀四望、山川則以毳冕,祭社稷、五祀則以絺冕,朝夕日月、風師、雨師、司命、司中則以玄冕。若七祀、蠟祭百神、先蠶、五龍、靈星、壽星、司寒、馬祭,
 蓋皆群小祀之比,當服玄冕。」從之。



 哲宗元祐元年,太常寺言:「舊制,大禮行事、執事官並服祭服,餘服朝服。至元豐七年,呂升卿始有行事及陪祠官並服祭服之議。今欲令行事、執事官並服祭服,其贊引、行事、禮儀使、太常卿、太常博士、閣門使、樞密院官進接圭,殿中監止供奉皇帝,其陪位官止導駕、押宿及主管事務,並他處行事官仍服朝服。」從之。



 徽宗大觀元年,議禮局言:「太社、太學獻官祝禮,皆以法服奉祠,至郡邑則用常服,乞降祭服。」
 詔頒制度於州郡,然未明使制造。後政和間,始詔:州縣冠服,形制詭異,令禮制局造樣頒下轉運司,轉運司制以給州縣焉。



 二年,議禮局檢討官俞慄言:「玄以象道,纁以象事,故凡冕皆玄衣纁裳,今太常寺祭服,則衣色青矣。前三幅以象陽,後四幅以象陰,故裳制不相連屬,今之裳則為六幅而不殊矣。冕玄表而朱裏,今乃青羅為覆,以金銀飾之。佩用綬以貫玉,今既有玉佩矣,又有錦綬以銀、銅二環,飾之以玉。宗彞,宗廟之彞也,乃為虎蜼
 之狀,而不作虎彞、蜼彞。粉米,散利以養人也,乃分為二章,而以五色圓花為藉。其餘不合古者甚多。乞下禮局,博考古制,畫太常寺及古者祭服樣二本以進。至於損益裁成,斷自聖學。」詔令議禮局詳議。



 四年,議禮局官宇文粹中議改衣服制度曰:「凡冕皆玄衣纁裳,衣則繪而章數皆奇,裳則繡而章數皆偶,陰陽之義也。今衣用深青,非是。欲乞視冕之等,衣色用玄,裳色用纁,以應典禮。古者蔽前而已,芾存此象,以韋為之。今蔽膝自一品
 以下,並以緋羅為表緣,緋絹為里,無復上下廣狹及會、紕、純、紃之制,又有山、火、龍章。案《明堂位》:『有虞氏服韍,夏后氏山、商火、周龍章。」□者乃黻冕之黻,非赤芾之芾也。且芾在下體,與裳同用,而山、龍、火者,衣之章也。周既繢於上衣,不應又繢於芾。請改芾制,去山、龍、火章,以破諸儒之惑。又祭服有革帶,今不用皮革,而通裹以緋羅,又以銅為飾。其綬或錦或皂,環或銀或銅,尤無經據,宜依古制除去。至佩玉、中單、赤舄之制,則全取元豐中詳定官
 所議行之。」



 粹中又上所編《祭服制度》曰:



 古者,冕以木版為中,廣八寸,長尺六寸,後方前圓,後仰前低,染三十升之布,玄表朱裏。後方者不變之體,前圓者無方之用;仰而玄者,升而辨於物,俛而朱者,降而與萬物相見。後世以繒易布,故純儉。今群臣冕版長一尺二寸,闊六寸二分,非古廣長之制;以青羅為覆,以金塗銀棱為飾,非古玄表朱里之制,乞下有司改正。古者,冕之名雖有五,而繅就、旒玉則視其命數以為等差。合彩絲為繩,用以貫
 玉,謂之「繅」。以一玉為一成,結之使不相並,謂之「就」。就間相去一寸,則九玉者九寸,七玉者七寸,各以旒數長短為差。今群臣之冕,用藥玉、青珠、五色茸絲泉,非藻玉三採、二採之義;每旒之長各八寸,非旒數長短為差之義;又獻官冕服,雜以諸侯之制,而一品服袞冕,臣竊以為非宜。



 元豐中,禮官建言,請資政殿大學士以上侍祠服□冕,觀察使以上服毳冕,監察御史以上服絺冕,朝官以上服玄冕,選人以上爵弁。詔許之,而不用爵弁。供奉官
 以下至選人,盡服玄冕無旒。臣竊謂依此參定,乃合禮制。古者,三公一命袞,則三公在朝,其服當鷩冕。蓋出封則遠君而伸,在朝則近君而屈。今之攝事及侍祠皆在朝之臣也,在朝之臣乃與古之出封者同命數,非先王之意。乞下有司製鷩冕八旒、毳冕六旒、絺冕四旒、玄冕三旒,其次二旒,又其次無旒。依元豐詔旨,參酌等降,為侍祠及攝祭之服,長短之度、採色之別,皆乞依古制施行。



 又案《周禮》,諸侯爵有五等,而服則三,所謂「公之服自
 袞冕而下,侯、伯自鷩冕而下,子、男自毳冕而下」是也。古者,諸侯有君之道,故其服以五、七、九為節。今之郡守,雖曰猶古之侯、伯,其實皆王臣也。欲乞只用群臣之服,自鷩冕而下,分為三等:三都、四輔為一等,初獻鷩冕八旒;經略、安撫、鈐轄為一等,初獻毳冕六旒,亞獻並玄冕二旒,終獻無旒;節鎮、防、團、軍事為一等,初獻絺冕四旒,亞、終獻並玄冕無旒。其衣服之制,則各從其冕之等。



 又曰:「今之紘組,仍綴兩繒帶而結於頤,冕旁仍垂青纊而不
 以瑱,以犀為簪而不以玉笄、象笄,並非古制,乞下有司改正。」從之。



 政和議禮局言:「大觀中,所上群臣祭服制度,已依所奏修定,乞付有司依圖畫制造。」既又上群臣祭服之制:正一品,九旒冕,金塗銀棱,有額花,犀簪,青衣畫降龍,朱裳,蔽膝,白羅中單,大帶,革帶,玉佩,錦綬,青絲網玉環,朱襪、履。革帶以金塗銀,玉佩以金塗銀裝,綬以天下樂暈。親祠大禮使、亞獻、終獻、太宰、少宰、左丞,每歲大祠宰臣、親王、執政官、郡王充初獻服之。奏告官並依本
 品服,已下準此。從一品,九旒冕,無額花,白綾中單,紅錦綬,銀環,金塗銀佩,餘如正一品服。親祠吏部、戶部、禮部、兵部、工部尚書,太廟進受幣爵、奉幣爵宗室,每歲大祠捧俎官、大祠中祠初獻官服之。二品,七旒冕,角簪,青衣無降龍,餘如從一品服。親祠吏部侍郎、殿中監、大司樂、光祿卿、讀冊官,太廟薦俎、贊進飲福宗室,七祀、配享功臣分獻官,每歲大祀,謂用宮架者,大司樂、大祠中祠亞終獻、大祠禮官、小祠獻官,朔祭太常卿服之。三品,五旒
 冕,皂綾綬,銅環,金塗銅革帶,佩,餘如二品服。親祠舉冊官、大樂令、光祿丞、奉俎饌籩豆簠簋官、分獻官,分獻壇壝從祀。



 太廟奉瓚盤、薦香燈、安奉神主、奉毛血盤、蕭蒿篚、肝膋豆宗室,每歲祭祠大樂令、大中祠分獻官服之。無旒冕,素青衣,朱裳,蔽膝,無佩綬,餘如三品服。奉禮協律郎、郊社令、太祝太官令、親祠抬鼎官、進摶黍官、太廟供亞終獻金斝、供七祀獻官、執爵官服之。五旒冕,紫檀絁衣,餘如三品服,監察御史服之。



 州郡祭服:三都初獻,
 八旒冕;經略、安撫、鈐轄初獻,六旒冕;亞獻並二旒冕,終獻無旒;節鎮、防、團、軍事初獻四旒冕,亞、終獻並無旒冕。



 中興之後,省九旒、七旒、五旒冕,定為四等:一曰鷩冕,八旒;二曰毳冕,六旒;三曰絺冕,四旒;四曰玄冕,無旒。其義以公、卿、大夫、士皆北面為臣,又近尊者而屈,故其節以八、以六、以四,從陰數也。先是,紹興四年五月,國子監丞王普奏言:



 臣嘗考諸經傳,具得冕服之制。蓋王之三公八命,鷩冕八旒,衣裳七章,其章各八。孤卿六命,毳冕六
 旒,衣裳五章,其章各六。大夫四命,絺冕四旒,衣裳三章,其章各四。上士三命,玄冕三旒;中士再命,玄冕二旒;下士一命,玄冕無旒;衣皆無章。裳、韍視其命數,自三而下。其繅至笄、衡、紘、紞、瑱、纊、帶、佩、芾、舄、中衣,皆有等差。



 近世冕服制度,沿襲失真,多不如古。夫後方而前圓,後昂而前俛,玄表而朱裏,此冕之制也;今則方圓俛仰,幾於無辨,且以青為表,而飾以金銀矣。其衣皆玄,其裳皆纁,裳前三而後四幅,此衣裳之制也;今則衣色以青,裳色以
 緋,且以六幅而不殊矣。山以章也,今則以嶞。火以圜也,今則以銳。宗彞,宗廟虎蜼之彞也,乃畫虎蜼之狀,而不為虎蜼彞。粉米,米而粉之者也,乃分為二章,而以五色圓花為藉。佩有衡、璜、琚、瑀、沖牙而已,乃加以雙滴,而重設二衡。綬以貫佩玉而已,乃別為錦綬,而間以雙環。以至帶無紐約,芾無肩頸,舄無絇繶,中衣無連裳。



 臣伏讀《國朝會要》郊廟奉祀禮文,祖宗以來,屢嘗講究,第以舊服無有存者。欲乞因



 茲改作,是正訛繆,一從周制,以合
 先聖之言。



 尋禮部契勘,奏言:



 衣服之制,或因時王而為之損益,事雖變古,要皆一時制作,不無因革。或考之先王而有繆戾者,雖行之已久,不應承誤襲非,憚於改正。案《周官》,自上公服袞,王之三公服□,以至士服玄冕,凡五等。唐制自一品服袞冕九旒,至五品服玄冕無旒,亦五等。國家承唐之舊,初有五旒之名,其後去三公袞冕及絺冕,但存七旒鷩冕、五旒毳冕與無旒玄冕,凡三等而已。袞服非三公所服,去之可也,乃並絺冕去之,自尚
 書服毳冕,以至光祿丞亦服焉,貴賤幾無差等。此皆一時制作,不無因革。



 今合增鷩冕為八旒,增毳冕為六旒,復置絺冕為四旒,並及無旒玄冕,共四等,庶幾稍合周制。若冕之方圓低昂至於無辨,則制造之差也。以青為表,非不用玄也,為玄而不至者也。以緋為裳,非不用纁也,為纁而太過者也。山止而靜者也,今象其嶞,是得山之勢而不知其性。火圜而神者也,今象其銳,是得火之形而不得其神也。至於宗彞、粉米、佩綬、帶紐、芾屨之屬,
 皆宜改正施行。



 是時,諸臣奏請討論雖詳,然終以承襲之久,未能盡革也。



 鷩冕:八旒,每旒八玉,三採,朱、白、蒼,角笄,青纊,以三色紞垂之,紘以紫羅,屬於武。衣以青黑羅,三章,華蟲、火、虎蜼彞;裳以纁表羅里,繒七幅,繡四章,藻、粉、黼、黻。大帶,中單,佩以鈱,貫以藥珠,綬以絳錦、銀環。韍上紕下純,繪二章,山、火。革帶,緋羅表,金塗銀裝。襪、舄並如舊制。宰相、亞終獻、大禮使服之;前期,景靈宮、太廟亞終獻,明堂滌濯、進玉爵酒官亦如之。



 毳冕:六玉,三採,衣
 三章,繪虎蜼彞、藻、粉米;裳二章,繡黼、黻。佩藥珠、衡、璜等,以金塗銅帶,韍繪以山。革帶以金塗銅。餘如鷩冕。六部侍郎以上服之;前期,景靈宮、太廟進爵酒幣官、奉幣官、受爵酒幣官、薦俎官,明堂受玉爵、受玉幣、奉徹籩豆、進飲福酒、徹俎祝腥、贊引、亞終獻,禮儀使、亞終獻爵並盥洗官四員,並如之;前二日奏告初獻,社壇九宮壇分祭初獻、亞獻亦如之。



 絺冕:四玉,二採,朱、綠。衣一章,繪粉米;裳二章,繡黼、黻。綬以皂綾,銅環。餘如毳冕。光祿卿、監察
 御史、讀冊官、舉冊官、分獻官以上服之;前期,景靈宮、太廟奏奉神主官、明堂太府卿、光祿卿、沃水舉冊官、讀冊官、押樂太常卿、東朵殿三員、西朵殿二員、東廊二十八員、西廊二十五員、南廊二十七員、軷門祭獻官,前二日奏告亞獻終獻官、監察御史,並如之;社壇九宮壇分祭終獻官、監察御史、兵工部、光祿卿丞亦如之。



 玄冕:無旒,無佩綬,衣純黑,無章,裳刺繡而已,□無刺繡,餘如絺冕。光祿丞、奉禮郎、協律郎、進摶黍官、太社令、良醞令、太官
 令、奉俎饌等官、供祠執事官內侍以下服之;明堂光祿丞、奉禮郎、良醞令、太祝摶黍官、宮架協律郎、登歌協律郎、奉御官、內侍供祠執事官、武臣奉俎官,軷門祭奉禮郎、太祝令、太官令,社壇九宮壇分祭太社、太祝、太官令、奉禮郎,並如之。



 紫檀冕:四旒,服紫檀衣,博士、御史服之。



 外州軍祭服:鷩冕,八旒,三都初獻服之;毳冕,六旒,經略、安撫、鈐轄初獻服之;絺冕,四旒,經略、安撫、鈐轄亞獻服之,節鎮、防、團、軍事初獻亦如之;玄冕,無旒,節鎮、防、團、軍
 事亞終獻服之。



 朝服:一曰進賢冠,二曰貂蟬冠,三曰獬豸冠,皆朱衣朱裳。宋初之制,進賢五梁冠:塗金銀花額,犀、玳瑁簪導,立筆。緋羅袍,白花羅中單,緋羅裙,緋羅蔽膝,並皂縹□,白羅大帶,白羅方心曲領,玉劍、佩,銀革帶,暈錦綬,二玉環,白綾襪,皂皮履。一品、二品侍祠朝會則服之,中書門下則冠加籠巾貂蟬。三梁冠:犀角簪導,無中單,銀劍、佩,師子錦綬,銀環,餘同五梁冠。諸司三品、御史臺四品、兩省
 五品侍祠朝會則服之。御史大夫、中丞則冠有獬豸角,衣有中單。兩梁冠:犀角簪導,銅劍、佩,練鵲錦綬,銅環,餘同三梁冠。四品、五品侍祠朝會則服之。六品以下無中單,無劍、佩、綬。御史則冠有獬豸角,衣有中單。褲褶紫、緋、綠,各從本服色,白綾中單,白綾褲,白羅方心曲領,本品官導駕,則騎而服之。



 褲褶之制,建隆四年,範質與禮官議:「褲褶制度,先儒無說,惟《開元雜禮》有五品以上用細綾及羅,六品以下用小綾之制。注:褶衣,復衣也。又案令
 文,武弁,金飾平巾幘,簪導,紫褶白褲,玉梁珠寶鈿帶,靴,騎馬服之。金飾,即附蟬也。詳此,即是二品、三品所配弁之制也。附蟬之數,蓋一品九,二品八,三品七,四品六,五品五。又侍中、中書令、散騎加貂蟬,侍左者左珥,侍右者右珥。又《開元禮》導駕官並朱衣,冠履依本品。朱衣,今朝服也。故令文三品以上紫褶,五品以上緋褶,七品以上綠褶,九品以上碧褶,並白大口褲,起梁帶,烏皮靴。今請造褲褶如令文之制,其起梁帶形制,檢尋未是,望以革
 帶代之。」奏可。是歲,造成而未用。乾德六年,郊禋始服,而冠未造,乃取朝服進賢冠、帶、襪、履參用焉。



 康定二年,少府監言:「每大禮,法物庫定百官品位給朝服。今兩班內,有官卑品高、官高品卑者,難以裁定,願敕禮院詳其等第。」詔下禮院參酌舊制以聞。奏曰:



 準《衣服令》,五梁冠,一品、二品侍祠大朝會則服之,中書門下則加籠巾貂蟬。準《官品令》,一品:尚書令,太師,太傅,太保,太尉,司徒,司空,太子太師、太傅、太保;二品:中書令,侍中,左右僕射,太子
 少師、少傅、少保,諸州府牧,左右金吾衛上將軍。又準《閣門儀制》,以中書令、侍中、同中書門下平章事為宰臣,親王、樞密使、留守、節度使、京尹兼中書令、侍中、同中書門下平章事為使相,樞密使、知樞密院事、參知政事、樞密副使、同知樞密院事、宣徽南北院使、僉書樞密院事並在東宮三司之上。以上品位職事,宜準前法給朝服。宰臣、使相則加籠巾貂蟬,其散官勛爵不系品位,止從正官為之服。



 三梁冠,諸司三品、御史臺四品、兩省五品侍
 祠大朝會則服之。御史中丞則冠獬豸。準《官品令》,諸司三品,諸衛上將軍,六軍統軍,諸衛大將軍,神武、龍武大將軍,太常、宗正卿,秘書監,光祿、衛尉、太僕、大理、鴻臚、司農、太府卿,國子祭酒,殿中、少府、將作、司天監,諸衛將軍,神武、龍武將軍,下都督,三京府尹,五大都督府長史,親王傅;御史臺三品、四品,御史大夫、中丞;兩省三品、四品、五品,左右散騎常侍,門下、中書侍郎,諫議大夫,給事中,中書舍人;尚書省三品、四品,六尚書,左右丞,諸行侍郎;
 東宮三品、四品,賓客,詹事,左右庶子,少詹事,左右諭德。節度使,文明殿學士,資政殿大學士,三司使,翰林學士承旨,翰林學士,資政殿學士,端明殿學士,翰林侍讀、侍講學士,龍圖閣學士,樞密直學士,龍圖、天章閣直學士,次中書侍郎;節度觀察留後,次六尚書、侍郎;知制誥,龍圖、天章閣待制,觀察使,次中書舍人;內客省使,次太府卿;客省使,次將作監;引進使,防禦、團練、三司副使,次左右庶子。以上品位職事,宜準前法給朝服。



 兩梁冠,四品、
 五品侍祠大朝會則服之,六品則去劍、佩、綬,御史則冠獬豸。淮《官品令》,諸司四品,太常、宗正少卿,秘書少監,光祿等七寺少卿,國子司業,殿中、少府、將作、司天少監,三京府少尹,太子率更令、家令、僕、諸衛率府率、副率,諸軍衛中郎將,諸王府長史、司馬,大都督府左右司馬,內侍;尚書省五品,左右司諸行郎中;諸司五品,國子博士,經筵博士,太子中允、左右贊善大夫,都水使者,開封祥符、河南洛陽、宋城縣令,太子中舍、洗馬,內常侍,太常、宗正、
 秘書、殿中丞,著作郎,殿中省五尚奉御,大理正,諸王友,諸軍衛郎將,諸王府諮議參軍,司天五官正,太史令,內給事;諸升朝官六品以下起居郎,起居舍人,侍御史,尚書省諸行員外郎,殿中侍御史,左右司諫,左右正言,監察御史,太常博士,通事舍人。四方館使,次七寺少卿;諸州刺史,次太子僕;謂正任不帶使職者。



 東西上閣門使,次司天少監;客省、引進、閣門副使,次諸行員外郎。已上品位職事,據令文,但言四品、五品,亦不分班敘上下。今請自尚書
 省五品以上及諸州刺史已上,準前法給朝服。其諸司五品已上,實有官高品卑及品高官卑者,宜自諸司五品、國子博士至內給事,並依六品以下例去劍、佩、綬,御史則冠獬豸,衣有中單。其諸司使、副使以下至閣門祗候,如有攝事合請朝服者,並同六品。



 詔從所請。



 元豐二年,詳定朝會儀注所言:



 古者制禮上物,不過十二,天之數也。自上而下,降殺以兩。畿外諸侯,遠於尊者而伸,則以九、以七、以五,從陽奇之數;王朝公卿大夫,近於尊者
 而屈,則以八、以六、以四,從陰偶之數。本朝《衣服令》,通天冠二十四梁,為乘輿服,以應冕旒前後之數。若人臣之冠,則自五梁而下,與漢、唐少異矣。至於綬,則乘輿及皇太子以織成,諸臣用錦為之。一品、二品冠五梁,中書門下加籠巾貂蟬。諸司三品三梁,四品、五品二梁,御史臺四品、兩省五品亦三梁,而綬有暈錦、黃獅子、方勝、練鵲四等之殊。六品則去劍、佩、綬。



 隋、唐冠服皆以品為定,蓋其時官與品輕重相準故也。今之令式,尚或用品,雖因
 襲舊文,然以官方之,頗為舛謬。概舉一二,則太子中允、贊善大夫與御史中丞同品,太常博士品卑於諸寺丞,太子中舍品高於起居郎,內常侍才比內殿崇班,而在尚書諸司郎中之上,是品不可用也。若以差遺,則有官卑而任要劇者,有官品高而處之冗散者,有一官而兼領數局者,有徒以官奉朝請者,有分局蒞職特出於一時隨事立名者,是差遣又不可用也。以此言之,用品及差遣定冠綬之制,則未為允當。伏請以官為定,庶名實
 相副,輕重有準,仍乞分官為七等,冠綬亦如之。



 貂蟬籠巾七梁冠,天下樂暈錦綬,為第一等。蟬,舊以玳瑁為蝴蝶狀,今請改為黃金附蟬,宰相、親王、使相、三師、三公服之。七梁冠,雜花暈錦綬,為第二等,樞密使、知樞密院至太子太保服之。六梁冠,方勝宜男錦綬,為第三等,左右僕射至龍圖、天章、寶文閣直學士服之。五梁冠,翠毛錦綬,為第四等,左右散騎常侍至殿中、少府、將作監服之。四梁冠,簇四雕錦綬,為第五等,客省使至諸行郎中服
 之。三梁冠,黃獅子錦綬,為第六等,皇城以下諸司使至諸衛率府率服之。內臣自內常侍以上及入內省內侍省內東西頭供奉官、殿頭,前班、東西頭供奉官、左右侍禁、左右班殿直,京官秘書郎至諸寺、監主簿,既預朝會,亦宜朝服從事。今參酌自內常侍以上,冠服各從本等,寄資者如本官,入內、內侍省內東西頭供奉官、殿頭,三班使臣,陪位京官為第七等,皆二梁冠,方勝練鵲錦綬。高品以下服色依古者,□□、舄、履並從裳色。



 今制,朝服
 用絳衣,而錦有十九等。其七等綬,謂宜純用紅錦,以文採高下為差別。惟法官綬用青地荷蓮錦,以別諸臣。《後漢志》:「法冠一曰柱後,執法者服之,侍御史、廷尉正監平也,或謂之獬豸冠。」《南齊志》亦曰:「法冠,廷尉等諸執法者冠之。」今御史臺自中丞而下至監察御史,大理卿、少卿、丞,審刑院、刑部主判官,既正定厥官,真行執法之事,則宜冠法冠,改服青荷蓮錦綬,其梁數與佩準本品。從之。



 其後,又詔冬正朝會,諸軍所服衣冠,廂都軍都指揮使、
 都虞候、領團練使、刺史服第五等,軍都指揮使、都虞候服第六等,指揮使、副指揮使服第七等,並班於庭。副都頭以上常服,班殿門外。其朝會,執事高品以下,並服介幘,絳服,大帶,革帶,襪、履,方心曲領。



 政和議禮局更上群臣朝服之制:七梁冠,金塗銀棱,貂蟬籠巾,犀簪導,銀立筆,朱衣裳,白羅中單,並皂褾、□,蔽膝隨裳色,方心曲領,緋白羅大帶,金塗銀革帶,金塗銀裝玉佩,天下樂暈錦綬,青絲網間施三玉環,白襪,黑履;三公,左輔,右弼,三少,
 太宰,少宰,親王,開府儀同三司服之。七梁冠,無貂蟬籠巾,銀裝玉佩,雜花暈錦綬,餘同三公以下服;執政官,東宮三師服之。六梁冠,白紗中單,銀革帶,佩,方勝宜男錦綬,銀環,餘同七梁冠服;大學士,學士,直學士,東宮三少,御史大夫、中丞,六曹尚書、侍郎,殿中監,大司成,散騎常侍,特進,金紫、銀青光祿大夫,光祿大夫,太尉,節度使,左右金吾衛、左右衛上將軍服之。五梁冠,翠毛錦綬,餘同六梁冠服;太子賓客、詹事,給事中,中書舍人,諫議大夫,
 待制,九寺卿,大司樂,秘書監,殿中少監,國子祭酒,宣奉、正奉、通奉、通議、太中、中大夫,中奉、中散大夫,上將軍,節度觀察留後,觀察使,通侍大夫,樞密都承旨服之。四梁冠,簇四盤雕錦綬,餘同五梁冠服;九寺少卿,大晟典樂,秘書少監,國子、闢NU司業,少府、將作、軍器監,都水使者,起居舍人,侍御史,太子左右庶子、少詹事、諭德,尚書左右司郎中、員外,六曹諸司郎中,朝議、奉直、朝請、朝散、朝奉大夫,防禦、團練使,刺史,大將軍,正侍、中侍、中亮、中衛、
 拱衛、左武、右武大夫,駙馬都尉,帶遙郡武功大夫以下,樞密副都承旨服之。三梁冠,金塗銅革帶,佩,黃獅子錦綬,瑜石環,餘同四梁冠服;殿中侍御史,監察御史,司諫,正言,尚書六曹員外郎,外符寶郎,少府、將作、軍器少監,太子侍讀、侍講,中書舍人,親王府翊善、侍讀、侍講,九寺、秘書、殿中監,闢NU丞,大晟樂令,兩赤縣令,大理正、司直、評事,著作郎,秘書郎,著作佐郎,太常、宗學、國子、闢NU博士,太史局令、正、丞,五官正,朝請、朝散、朝奉、承議、奉議、通
 直郎,中亮、中衛、拱衛、左武、右武郎,諸衛將軍,衛率府率,武功、武德、武顯、武節、武略、武經、武義、武翼大夫郎,醫職翰林醫正以上,內符寶郎,閣門通事舍人,敦武郎,修武郎服之。二梁冠,角簪,方勝練鵲錦綬,餘同三梁冠服;在京職事官,閣門祗候,看班祗候,率府副率,升輦輅立侍內臣服之。御史大夫、中丞,刑部尚書、侍郎,大理卿、少卿,侍御史,刑部郎中,大理寺正、丞、司直、評事並冠獬豸冠,服青荷蓮綬。詔悉頒行。六年,詔導駕官朝服結佩。七年,
 詔夏祭百官朝、祭服用紗。



 中興,仍舊制。行事、執事官則服祭服,導引、陪祠官則服朝服,從紹興三年太常寺請也。祠畢駕回,若服通天、絳紗袍,乘大輦,則百官從駕服朝服,或服履袍;乘平輦,則百官從駕服常服,自隆興二年洪適請始也。



 進賢冠以漆布為之,上縷紙為額花,金塗銀銅飾,後有納言。以梁數為差,凡七等,以羅為纓結之:第一等七梁,加貂蟬籠巾、貂鼠尾、立筆;第二等無貂蟬籠巾;第三等六梁,第四等五梁,第五等四梁,第六
 等三梁,第七等二梁,並如舊制,服同。貂蟬冠一名籠巾,織藤漆之,形正方,如平巾幘。飾以銀,前有銀花,上綴玳瑁蟬,左右為三小蟬,御玉鼻,左插貂尾。三公、親王侍祠大朝會,則加於進賢冠而服之。獬豸冠即進賢冠,其梁上刻木為獬豸角,碧粉塗之,梁數從本品。立筆,古人臣簪筆之遺像。其制削竹為乾,裹以緋羅,以黃絲為毫,拓以銀縷葉,插於冠後。舊令,文官七品以上服朝服者,簪白筆,武官則否,今文武皆簪焉。



\end{pinyinscope}