\article{列傳第一百三十八}

\begin{pinyinscope}

 章
 誼韓肖冑陳公輔張觷胡松年曹勛李稙韓公裔



 章誼,字宜叟,建州浦城人。登崇寧四年進士第,補懷州司法參軍,歷漳、臺二州教授、杭州通判。建炎初,陳通寇
 錢塘,城閉,部使者檄誼聚杭州七縣弓兵,以張聲勢。會王淵討賊,誼隨淵得入城,賊平,旋加撫定,人皆德之。



 帝幸臨安,苗、劉為變,帝御樓,宰臣百執事咸在,人心洶洶。帝問群臣曰:「今日之事何如?」浙西安撫司主管機宜文字時希孟輒曰:「乞問三軍。」誼越班斥之曰:「問三軍何義?若將鼓亂邪?」希孟卻立屏息,帝嘉之。事定,竄希孟吉陽軍,誼遷二秩,擢倉部員外郎。奉使二浙,貿易祠牒以濟軍用,以稽遲罷。未幾,召為駕部員外郎,遷殿中侍御史。



 張浚宣撫陜西,誼奏:「自趙哲退敗,事任已重,處斷太專,當除副貳,使之自助。」何□贈官,誼論其「折沖無謀,守御無策,乃中國招禍之首」。乞寢免。



 邵青自太平乘舟抵平江,所至劫掠。誼請置水軍於駐蹕之地,且言:「古舟師有三等,大為陣腳,次為戰船,小為傳令,皆可為戰守之備。」詔淮南三宣撫措置。誼又獻戰守四策,謂:「金人累歲南侵,我亦累歲奔走,蓋謀國之臣誤陛下也。比者駐蹕揚州,有兵數十萬,可以一戰。斥候不明,金人奄至,逾江而
 東,此宰相黃潛善、汪伯彥過也。前年,移蹕建康,兵練將勇,據長江之險,可守矣。舟師不設,二相異意,金人未至,遵海而南,此宰相呂頤浩過也。不知今年守戰之策安所從出?執政大臣誰為陛下任此事者?臣愚謂有江海,必資舟楫戰守之具;有險阻,必資郡縣防守之力;有兵將,必駕馭撫循,不可為將帥自衛之資;有糧賦,必漕運轉輸,不可為盜賊侵據之用。四者各付能臣,分路以辦,重賞嚴罰,誰敢不用命哉!」



 詔問保民、弭盜、遏寇、生財之
 策,誼對曰:「去奸貪殘虐之吏,則民可保;用循良廉平之吏,則盜可弭;敵寇未遏,以未得折沖禦侮之臣;財賦未裕,以未得掌財心計之臣。凡此四者,任人不任法,則政治可得而治矣。」



 詔集議明堂配享,胡直儒等請合祭天地,而以太祖、太宗配。誼言:「稽之經旨則未合,參之典故則未盡,施之事帝則未為簡嚴。今國家既以太祖配天於郊,比周之後稷,則太宗宜配帝於明堂,以比周之文王。仁宗皇祐二年,始行明堂合祭天地,並配祖宗,乃一
 時變禮。至嘉祐七年,再行宗祀,已悟皇祐之非,乃罷配享,仍徹地示之位,故有去並侑煩文之詔。如嘉祐之詔,則太祖地示已不與祭;元豐正祀典之詔,則悉罷群祀。臣等謂將來明堂大饗,宜專祀昊天上帝,而以太宗配。」後不果行。



 紹興二年,除大理卿。宰相奏知平江府,帝曰:「誼儒者,賴其奏讞平恕,使民不冤,勿令補外。」尋除權吏部侍郎,乞:「詔有司編類四選通知之條,與一司專用之法,兼以前後續降指揮,自成一書。如此則銓曹有可守
 之法,奸吏無舞文之弊,書成而吏銓有所執守矣。」



 改刑部侍郎兼詳定一司敕令,誼奏:「比修紹興敕令格式,其忠厚之意,則本於祖宗;其綱條之舉,則仍於舊貫。今在有司,為日既久,州縣推行,漸見抵牾。欲承疑遵用,則眾聽惑而不孚;欲因事申明,則法屢變而難守。乞詔監司、郡守與承用官司,參考祖宗舊典,各摭新書之闕遺,條具以聞,然後命官審訂刪去,著為定法。」



 遷徽猷閣直學士、樞密都承旨,誼奏:「漢有南北兩屯,唐有南北兩衛,皆
 天子自將之兵。祖宗所置殿班親軍,處禁門之內,皆極天下之選。今日神武兵萃於五軍,多逃亡之餘,市井之人,殿班親軍,倚以侍衛者,曾無千百。願陛下酌漢、唐南北禁衛之意,修本朝遴選班直之法,選五軍及諸州各為一衛,合取萬人,分為兩衛,則禁衛增嚴,王室大競矣。」



 四年,金遣李永壽、王翊來,求還劉豫之俘,及西北人在東南者,又欲畫江以益劉豫。時議難之,欲遣大臣為報使。參政席益以母老辭,薦誼為代,加誼龍圖閣學士,充
 軍前奉表通問使,給事中孫近副之。誼至雲中,與粘罕、兀室論事,不少屈。金人諭亟還,誼曰:「萬里銜命,兼迎兩宮,必俟得請。」金人乃令蕭慶授書,並以風聞事責誼,誼詰其所自,金人以實告,乃還。至南京,劉豫留之,以計得歸。帝嘉勞之,擢刑部尚書。



 是冬,帝親征,王師大捷於淮陰,誼扈從。還臨安,遷戶部尚書,誼言:「祖宗設官理財,內則戶部,外則諸路轉運使、副,東南委輸最盛,則又置發運,以督諸路供輸之入,皆有移用補助之法,戶部仰以
 不乏者也。今川、廣、荊湖土貢歲輸,不入王府者累年矣,皆發運使失職之罪也。頃因定都汴京,故發運使置司真、泗,今駐吳會,則發運當在荊湖南、北之間。望討論發運置司之地,選能臣以充其任。」又言:「戶部左右曹之設,諸路運司則左曹之屬也,提舉則右曹之屬也。若復發運司,於諸路各置轉運使副二員,以一員檢察常平,以應右曹之選,則戶部財用無陷失矣。」



 五年,以疾請郡,除龍圖閣學士、知溫州。適歲大旱,米斗千錢,誼用劉晏招
 商之法,置場增直以糴,米商輻輳,其價自平。部使者以狀聞,詔遷官一等。六年,移守平江。時將臨幸,供億繁伙,誼處之皆當於理。召對,賜帶笏,帝曰:「此不足以償卿之勞,其勿謝。」



 明年,移蹕建康,復為戶部尚書。誼奏營田之策,謂:「京西、湖北、淮南東西失業者最多,朝廷必欲家給牛種、人給錢糧以勸耕,則財力不足。今三大將各屯一路,如各捐數縣地均給將士,收其餘以省轉輸,非小補也。」



 七年,帝還臨安,以誼為端明殿學士、江南東路安撫
 大使、知建康府兼行宮留守。未幾,提舉毫州明道宮,代還。八年卒,年六十一,謚忠恪。



 誼寬厚長者,故事臺官言事,非挾怨以快己私,即用仇家言為人報復,誼獨存大體,士論歸之。立朝論事,奏疏無慮數十百篇,皆經國濟時之策。初,席益薦誼使金,帝曰:「誼亦母老,朕當自諭之。」誼聞命,略無難色,戒其家人勿使母知。將行,告母曰:「是行不數月即歸,大似往年太學謁告時爾。」及還,母竟不知其使金也。誼卒,母年九十二。子八人:「駽、駒、駟、驔、□卒、
 駉、馳、駰。



 韓肖冑,字似夫,相州安陽人。曾祖琦,祖忠彥,再世為相。父治。肖冑以蔭補承務郎,歷開封府司錄。與府尹同對殿中,徽宗問其家世,賜同上舍出身,除衛尉少卿,賜三品服。



 尋假給事中、充賀遼國生辰使。既還,時治守相州,請祠。肖冑因乞補外侍疾,詔除直秘閣、知相州,代其父任。陛辭,帝曰:「先帝詔韓氏世官於相。卿父子相代,榮事也。」在相四年,王師傳燕,肖冑策幽薊且有變,宜陰為守
 備。已而金騎入境,野無所掠而去。



 建炎二年,知江州,入為祠部郎,遷左司。嘗言:「中原未復,所恃長江之險,淮南實為屏蔽。沃野千里,近多荒廢,若廣修農事,則轉餉可省,兵食可足。」自是置局建康,行屯田於江淮。又應詔陳五事,曰:遠斥堠,戢戍兵,防海道,援中原,修軍政。擢工部侍郎。



 時川、陜馬綱路通塞不常,肖冑請於廣西邕州置司,互市諸蕃馬,詔行之。時召侍從問戰守計,肖冑條奏千餘言,帝稱其所對事理簡當。吏部尚書席益嘆曰:「援
 古證今,切於時用,非世官不能也。」



 紹興二年,詔百官各言省費裕國、強兵息民之策,肖冑言:「天下財賦窠名,舊悉隸三司,今戶部惟有上供之目而已。問諸路窠名於戶部,戶部不能悉,問諸州窠名於漕司,漕司不能悉,失一窠名,則此項遂亡。願詔諸路漕司,括州縣出納,可罷罷之,可並並之,立為定籍。漕司總諸州,戶部總諸路,則無失陷矣。經費之大,莫過養兵。今人亡而冒請者眾,願立諸軍核實之法,重將帥冒請之罪,則兵數得實,餉給
 不虛,省費裕國,此其大者。生民常賦之外,迫以軍期,吏緣為奸,斂取百端。復為寇所迫逐,田桑失時,寇去復業,未及息肩,催科之吏已呼其門矣。願詔郡邑,招集流散,官貸之種,俟及三年,始責其賦,置籍書之,以課殿最,強兵息民,此其先者。」時多所採納。又請復天地、日月、星辰、社稷之祀,於是下有司定一歲祭禮。



 遷吏部侍郎,時條例散失,吏因為奸,肖冑立重賞,俾各省記,編為條目,以次行之,舞文之弊始革。陣亡補官,得占射差遣,而在部
 常調人,守待不能注授,且有短使重難。肖冑請陣亡惟許本家用恩例,異姓候經任收使,遂無不均,且嚴六部出入之禁,而請托不行。



 三年,拜端明殿學士、同簽書樞密院事,充通問使,以胡松年副之,肖冑慨然受命。時金酋粘罕專執政,方恃兵強,持和戰離合之策,行人皆危之。肖冑入奏曰:「大臣各循己見,致和戰未有定論。然和乃權時之宜,他日國家安強,軍聲大振,誓當雪此仇恥。今臣等行,或半年不返命,必復有謀,宜速進兵,不可因
 臣等在彼而緩之也。」將行,母文語之曰:「汝家世受國恩,當受命即行,勿以我老為念。」帝稱為賢母,封榮國夫人。



 肖冑至金國,金人知其家世,甚重之,往返才半年。自帝即位,使者凡六七年未嘗報聘,至是始遣人偕來。肖冑先北使入對,與朱勝非議不合,力求去,以舊職知溫州,提舉臨安府洞霄宮。



 五年,詔問前宰執戰守方略,肖冑言:「女真等軍皆畏服西兵勁銳善戰,今三帥所統多西人,吳玠繼有捷奏,軍聲益振,敵意必搖,攻戰之利,臣固
 知之。自荊、襄至江、淮,綿亙數千里,不若擇文武臣僚按行計度,求險阻之地,屯兵積糧,則形勢相接。今淮東、西雖命宣撫使,然將屯置司,乃在江上,所遣偏裨分守,不過資以輕兵,勢孤力弱,難以責其固志。當移二將於江北,使藩籬可固。」又言:「諸大將之兵自主庭戶,更相仇疾。若欲並遣進攻,宜先命總帥,分以精銳,自成一軍,號令既一,則諸將疇敢不聽命。畿甸、山東、關河之民怨金人入骨,當以安集流亡,招懷歸附為先,今淮南、江東西荒
 田至多,若招境上之人,授田給糧,捐其賦租,必將接跡而至。」又奏:「江之南岸,曠土甚多,沿江大將各分地而屯,軍士舊為農者十之五六,擇其非甚精銳者,使之力耕,農隙則試所習之技藝,秋成則均以所種之禾麥,或募江北流徒及江南無業願遷之人分給之,創為營屯。止則固守,出則攻討。」起知常州,召赴行在,提舉萬壽觀,尋除簽書樞密院事。



 和議已定,復命肖冑為報謝使。接伴者逆於境,謂當稱謝恩使。肖冑論難三四反,遂語塞。既
 至,金遣人就館議事,肖冑隨問隨答,眾皆聳聽。其還,給氈車及頓遞宴設,自肖冑始。



 除資政殿學士、知紹興府。尋奉祠,與其弟膺冑寓居於越幾十年。事母以孝聞,弟不至不食,所得恩澤,皆先給宗族。卒,年七十六,謚元穆。



 琦守相,作晝錦堂,治作榮歸堂,肖冑又作榮事堂,三世守鄉郡,人以為榮。



 陳公輔,字國佐,臺州臨海人。政和三年,上舍及第,調平江府教授。朱勉方嬖幸,當官者奴事之,公輔絕不與交。
 勉有兄喪,諸生欲往吊,公輔不予告。勉不悅,諷權要移公輔越州。累遷權應天府少尹,除秘書郎。



 靖康初,二府多宣和舊人,公輔言:「蔡京、王黼用事二十餘年,臺諫皆緣以進,唐重、師驥為太宰李邦彥引用,謝克家、孫覿為纂修蔡攸引用,及邦彥作相,又附麗以進。此四人者,處臺諫之任,臣知其決不能言宰相大臣之過。願擇人臣中樸茂純直,能安貧守節、不附權幸、慷慨論事者,列之臺諫,則所任得人,禮義廉恥稍稍振起,敵國聞之,豈不
 畏服哉!」時吳敏、李綱不協,公輔奏:「陛下初臨萬機,正賴其同心合謀,而二臣不和,已有其跡,願諭以聖訓,俾務一心以安國家。」



 徽宗渡江未還,人情疑懼,公輔力陳父子之義,宜遣大臣迎奉。欽宗嘉之,擢為右司諫。孟夏享景靈宮,遂幸陽德、祐神觀。公輔諫不當如平時事宴游,論:「蔡京父子懷奸誤國,終未行遣。今朝廷公卿百執事半出其門,必有庇之者。」詔謫京崇信軍節度副使,德安府安置。又奏:「朱勉罪惡,都城之民皆謂已族滅其家,乞
 勿許其子姓隨上皇入京。」



 時有指公輔為李綱之黨,鼓唱士庶伏闕者。公輔自列,因辭位,後陳三事:其一言李綱書生,不知軍旅,遣援太原,乃為大臣所陷,必敗事。其二言餘應求不當以言遠謫。其三言方復祖宗法度,馮澥不宜更論熙寧、元豐之政。語觸時宰,遂與應求、程瑀、李光俱得罪,斥監合州稅。



 高宗即位,召還,除尚書左司員外郎。明年,始達維揚。初,李綱得政,公輔自外除郎,未至而綱罷,改南劍州,尋予宮觀。



 紹興六年,召為吏部員
 外郎。疏言:「今日之禍,實由公卿大夫無氣節忠義,不能維持天下國家,平時既無忠言直道,緩急詎肯伏節死義,豈非王安石學術壞之邪?議者尚謂安石政事雖不善,學術尚可取。臣謂安石學術之不善,尤甚於政事,政事害人才,學術害人心,《三經》、《字說》詆誣聖人,破碎大道,非一端也。《春秋》正名分,定褒貶,俾亂臣賊子懼,安石使學者不治《春秋》;《史》、《漢》載成敗安危、存亡理亂,為聖君賢相、忠臣義士之龜鑒,安石使學者不讀《史》、《漢》。王莽之篡,
 揚雄不能死,又仕之,更為《劇秦美新》之文。安石乃曰:『雄之仕,合於孔子無可無不可之義。』五季之亂,馮道事四姓八君,安石乃曰:『道在五代時最善避難以存身。』使公卿大夫皆師安石之言,宜其無氣節忠義也。」復授左司諫,言:「中興之治在得天得人,以孝感天,以誠得民。」帝善其深得諫臣體,賜三品服,令尚書省寫圖進入,以便觀覽。



 公輔感帝知遇,益罄忠鯁,言:「正心在務學,治國在用人,朝廷之禍在朋黨。」仍乞增輪對官,令審計、官告、糧料、
 榷貨、監倉及茶場等官,有己見,許面對。時有詔將駐蹕建康,公輔上疏陳攻守之策,且乞選大臣鎮淮西,增兵將守要害,使西連鄂、岳,東接楚、泗,皆有掎角之形。



 徽宗訃至,公輔請宮中行三年之喪,視朝服淡黃,群臣未可純吉服,明堂未當以徽宗配,宜罷臨軒策士。又乞權罷講筵,事不行。



 遷尚書禮部侍郎。會趙鼎言進退人才乃其職分,疏稍侵公輔,因力請祠。除集英殿修撰、提舉江州太平觀,尋知處州。升徽猷閣待制,乃提舉太平觀。卒,
 年六十六,贈太中大夫。有《文集》二十卷,《奏議》十二卷,行於世。公輔論事剴切,疾惡如仇,惟不右程頤之學,士論惜之。



 張觷,字柔直,福州人。舉進士,為小官,不與世詭隨。時蔡京當國,求善訓子弟者,觷適到部,京族子應之以觷薦,觷再三辭,不獲,遂即館,京亦未暇與之接。觷嚴毅聳拔,意度凝然,異於他師,諸生已不能堪,忽謂之曰:「汝曹曾學走乎?」諸生駭而問曰:「嘗聞先生教令讀書徐行,未聞
 教以走也。」觷曰:「天下被而翁破壞至此,旦夕賊來,先至而家,汝曹惟有善走,庶可逃死爾。」諸子大驚,亟以所聞告京,曰:「先生心恙。」京矍然曰:「此非汝所知也。」即見觷深語,觷慷慨言曰:「宗廟社稷,危在旦夕。」京斂容問計,觷曰:「宜亟引耆德老成置諸左右,以開道上心。羅天下忠義之士,分布內外,為第一義爾。」京因扣其所知,遂以楊時薦,於是召時。



 觷後守南劍州,遷福建路轉運判官。未行,會範汝為陷建州,遣葉徹擁眾寇南劍。時統制官任士
 安駐軍城西,不肯力戰,觷獨率州兵與之戰,分為數隊,令城中殺羊牛豕作肉串,仍多具飯。將戰,則食第一隊人,既飽,遣之入陣,便食第二隊人,度所遣兵力將困,即遣第三隊人往代,第四至五六隊亦如之。更迭交戰,士卒飽而力不乏。徹中流矢死,眾敗走。觷知士安懼無功,即函徹首與之,州兵皆憤,觷曰:「賊必再至,非與大軍合力不能破也。」士安得之大喜,遂馳報諸司,謂已斬徹。未幾,徹二子果引眾聲言復父仇,縞素來攻。於是士安與
 州兵夾攻,大敗之,城賴以全。



 再知處州,嘗欲造大舟,幕僚不能計其直,觷教以造一小舟,量其尺寸,而十倍算之。又有欲築紹興園神廟垣,召匠計之,云費八萬緡,觷教之自築一丈長,約算之可直二萬,即以二萬與匠者。董役內官無所得,乃奏紹興空乏難濟,太后遂自出錢,費三十二萬緡。以直龍圖閣知虔州,蕩平餘寇,進秘閣修撰,卒。後廟食邵武。



 胡松年字茂老,海州懷仁人。幼孤貧,母粥機織,資給使
 學,讀書過目不忘,尤邃於《易》。政和二年,上舍釋褐,補濰州教授。八年,賜對便殿,徽宗偉其狀貌,改校書郎兼資善堂贊讀。為殿試參詳官,以沉晦第一,徽宗大悅曰:「朕久聞晦名,今乃得之。」遷中書舍人。



 時方有事燕云,松年累章謂邊釁一開,有不勝言者。咈時相意,提舉太平觀。建炎間,密奏中原利害,召赴行在,出知平江府。未入境,貪吏解印斂跡,以興利除害十七事揭於都市,百姓便之。加徽猷閣待制。奏防江利害:一曰立國無藩籬之固,
 二曰遣將無首尾之援,三曰不攻敵技之所短。



 召為中書舍人。言武昌、九江、建昌、京口、吳江、錢塘、明、越宜各屯水戰士三千以為備。唐恪追復觀文殿學士,松年繳奏曰:「靖康之禍,何□輕脫寡謀,宜為罪首。去年秦檜還朝,力稱其抗義守正,遂被褒贈,已大咈士論。今恪子琢自陳其父不獲伸迎請二帝之謀,飲藥而死。此事凜然,追蹤古人。宜詔有司詳考實狀,庶不為虛美,以示激勸。」



 除給事中。會選將帥,松年奏:「富貴者易為善,貧賤者難為
 功,在上之人識擢何如爾。願陛下親出勞軍,即行伍搜簡之,必有可為時用者。」又奏:「恢復中原,必自山東始,山東歸附,必自登、萊、密始,不特三郡民俗忠義,且有通、泰飛艘往來之便。」除兼侍講。



 王倫使金還,言金人欲再遣重臣來計議,以松年試工部尚書為韓肖冑副,充大金奉表通問使。時使命久不通,人皆疑懼,松年毅然而往。至汴京,劉豫令以臣禮見,肖冑未答,松年曰:「聖主萬壽。」豫曰:「聖意何在?」松年曰:「主上之意,必復故疆而後已。」使
 還,拜吏部尚書。



 岳飛收復襄、漢,令松年籌度守御事。松年奏:「乞飛班師,徐窺劉豫意向,若豫置不問,其情叵測,當飭將士謹疆場可也。」又條戰艦四利:一曰張朝廷深入之軍勢,二曰固山東欲歸之民心,三曰震疊強敵,使不敢窺江、浙,四曰牽制劉豫不暇營襄、漢。



 除端明殿學士、簽書樞密院事。首奏八事:立規摹以定中興之基,振紀綱以尊朝廷之勢,馭將帥使知畏,撫士卒使知勸,收予奪之柄,察毀譽之言,無以小疵棄人才,無以虛文廢
 實效。又薦張敵萬:「向在淮南誘敵深入,步騎四集,悉陷於淖,無得解者,金人至今膽落。乞令統率軍馬別為任使,庶幾外閫浙多名將,不獨仗倚三四人而已。」



 諜報劉豫於登、萊、海、密具舟楫,淮陽、順昌積芻粟,欲憑借金人侵我邊鄙。議者謂韓、劉、嶽各當一面,可保無虞。松年奏:「三人聲勢初不相屬,緩急必不相救。況海道闊遠,蘇、秀、明、越最為要沖,乞選精兵萬人,命一大臣往駐建康,親督世忠、光世守採石、馬家渡,以張兩軍之勢,仍以兵五
 千屯明州、平江,控禦江海。或無人可遣,臣願疾馳以赴其急。」詔遣松年往江上,與諸將會議進討,因覘賊情。帝決意親征,遂次平江,命松年權參知政事,專治戰艦,張浚專治軍器。松年曰:「議論既定,力行乃有效,若今日行,明日止,徒紛紛無益。」



 俄以疾提舉洞霄宮,卜居陽羨,雖居閑不忘朝廷事,屢言和糴科斂、防秋利害,帝皆嘉納。十六年,病革,呼其子曰:「大化推移,有所不免。」乃就枕,鼻息如雷,有頃卒,人謂不死也。年六十。



 松年平生不喜蓄
 財,每除官例賜金帛,以軍興費廣,一無所陳請,或勸其白於朝,曰:「弗請則已,白之是沽名也。」喜賓客,奉入不足以供費,或請節用為子孫計。松年曰:「賢而多財,則損其志,況俸廩,主上所以養老臣也。」自持囊至執政,所舉自代,皆一時聞人,所薦一以至公,權勢莫能奪。



 方秦檜秉政,天下識與不識,率以疑忌置之死地,故士大夫無不曲意阿附為自安計。松年獨鄙之,至死不通一書,臣以此高之。



 曹勛,字公顯,陽翟人。父組,宣和中,以閣門宣贊舍人為睿思殿應制,以占對開敏得幸。勛用恩補承信郎,特命赴進士廷試,賜甲科,為武吏如故。



 靖康初,為閣門宣贊舍人、勾當龍德宮,除武義大夫。從徽宗北遷,過河十餘日,謂勛曰:「不知中原之民推戴康王否?」翌日,出御衣書領中曰:「可便即真,來救父母。」並持韋賢妃、邢夫人信,命勛間行詣王。又諭勛:「見康王第言有清中原之策,悉舉行之,毋以我為念。」又言「藝祖有誓約藏之太廟,不殺大
 臣及言事官,違者不祥」



 勛自燕山遁歸。建炎元年七月,至南京,以御衣所書進入。高宗泣以示輔臣。勛建議募死士航海入金國東京,奉徽宗由海道歸,執政難之,出勛於外,凡九年不得遷秩。紹興五年,除江西兵馬副都監,勛以遠次為請,改浙東,言者論其不閑武藝,專事請求,竟奪新命。



 十一年,兀朮遣使議和,授勛成州團練使,副劉光遠報之。及淮,遇兀朮,遣還,言當遣尊官右職持節而來,蓋欲亟和也。勛還,遷忠州防禦使。金使蕭毅等
 來,命勛為接伴使。未幾,落階官為容州觀察使,充金國報謝副使,召入內殿,帝灑泣,諭以懇請親族之意。及見金主,正使何鑄伏地不能言,勛反復開諭,金主首肯許還梓宮及太后。勛歸,金遣高居安等衛送太后至臨安,命勛充接伴使。遷保信軍承宣使、樞密副都承旨。



 二十九年,拜昭信軍節度使,副王綸為稱謝使。時金主亮已定侵淮計,勛與綸還,言鄰國恭順,和好無他,人譏其妄。孝宗朝加太尉、提舉皇城司、開府儀同三司。淳熙元年
 卒,贈少保。



 李稙,字符直,泗州臨淮人。幼明敏篤學,兩舉於鄉。從父中行客蘇軾門,太史晁無咎見之曰:「此國士也。」以女妻焉。



 靖康初,高宗以康王開大元帥府。湖南向子諲轉運京畿,時群盜四起,餉道□厄絕,環視左右無足遣者。有以稙薦,遂借補迪功郎,使督四百艘,總押犒師銀百萬、糧百萬石,招募忠義二萬餘眾,自淮入徐趨濟,凡十餘戰,卒以計達。時高宗駐師鉅野,聞東南一布衣統眾而至,
 士氣十倍,首加勞問。稙占對詳敏,高宗大悅,親賜之食,曰:「得一士如獲拱璧,豈特軍餉而已。」承制授承直郎,留之幕府。



 稙三上表勸進:「願蚤正大寶,以定人心,以應天意。」三降手札獎諭。稙感激知遇,言無不盡,為汪伯彥、黃潛善所忌。高宗既即位,為東南發運司干辦公事,尋以奉議郎知潭州湘陰。縣經楊麼蕩析,稙披荊棘,立縣治,發廩粟,振困乏,專以撫摩為急。



 丞相張浚督師江上,知稙才,薦為朝奉郎、鄂州通判。大盜馬友、孔彥舟未平,稙
 請修戰艦,習水戰,分軍馬為左右翼,大破彥舟伏兵,誅馬友,二盜平。浚以破賊功上於朝,轉朝奉大夫、通判荊南府。秩滿,除尚書戶部員外郎。



 時秦檜當國,凡帥府舊僚率皆屏黜,浚亦去國。稙即丐祠奉親,寓居長沙之醴陵十有九年,杜門不仕。



 檜死,子諲以戶部尚書居邇列,語及龍飛舊事,識稙姓名,除戶部郎中。稙始入見,帝曰:「朕故人也。」方有意大用,以母老,每辭,願便養,除知桂陽軍。丁母憂,歸葬,哀毀廬墓,有白鷺朱草之祥。劉錡遺
 之書曰:「忠臣孝子,元直兼之矣。」



 服闋,參政錢端禮薦差知瓊州。陛辭,帝慨然曰:「卿老矣,瓊管遠在海外。」改知徽州。徽俗崇尚淫祠,稙首以息邪說、正人心為事,民俗為變。轉朝請大夫、直秘閣,改知鎮江府,遷江、淮、荊湘都大提點坑冶鑄錢公事。



 逾年,金人敗盟,朝廷將大舉,以稙漕運有才略,授直敷文閣、京西河北路計度轉運使。稙措畫有方,廷議倚重。乾道元年,遷提刑江西。二年,直寶文閣、江南東路轉運使兼知建康軍府兼本路安撫使,
 主管行宮留守司事。



 稙上書極言防江十策,其略曰:「保荊、襄之障,以固本根;審中軍所處,以俟大舉;搜選強壯,以重軍勢;度地險□厄,以保居民;避敵所長,擊其所短;金人降者宜加賞勸。」皆直指事宜,不為浮泛。疏上,帝嘉其言,以太府卿召赴闕,有疾不克上道,遂以中奉大夫、寶文閣學士致仕,還湘。



 時胡安國父子家南嶽下,劉錡家湘潭,相與往還講論,言及國事,必憂形於色,始終以和議為恨。年七十有六卒。有文集十卷,題曰《臨淮集》,廬陵
 胡銓為之序。謚忠襄。



 子五人,汝虞知桃源縣,汝士朝奉大夫、知黃州,汝工知昌化軍。



 韓公裔,字子扆,開封人。初以三館吏補官,掌韋賢妃閣箋奏,尋充康王府內知客。金兵犯京,王出使,公裔從行。渡河,將官劉浩、吳湛私鬥,公裔諭之乃解。次磁州,軍民戕奉使王云,隨王車入州廨,公裔復諭退之。王之將南也,與公裔謀,間道潛師夜起,遲明至相,磁人無知者,自是親愛愈篤。及兵退,張邦昌遣人同王舅韋淵來獻傳
 國璽。時淵自稱偽官,議者又謂邦昌不可信,王怒將誅淵,公裔曰:「神器自歸,天命也。」王遂受璽,命公裔掌之。公裔力救淵,釋其罪。



 元祐後詔王入承大統,府僚謂金兵尚近,宜屯彭城。公裔言:「國家肇基睢陽,王亦宜於睢陽受命。」時前軍已發,將趨彭城,會天大雷電,不能前,王異之,夜半抗聲語公裔曰:「明日如睢陽,決矣。」既即帝位,公裔累遷武功大夫、貴州防禦使。



 後以事忤黃潛善,適帝幸維揚,公裔丐去,潛善以為避事,遂降三官,送吏部。帝
 幸越,念其舊勞,召復故官、乾辦皇城司,仍帶御器械,累遷至廣州觀察使、提舉祐神觀。



 公裔給事藩邸三十餘年,恩寵優厚,每置酒慈寧宮,必召公裔。會修《玉牒》,元帥府事多放佚,秦檜以公裔帥府舊人,奏令修書官就質其事。俄除保康軍承宣使,檜疑其舍己而求於帝,銜之。右諫議大夫汪勃希檜意,劾罷公裔,遂與外祠,在外居住,而帝眷之不衰。



 檜死,即復提舉祐神觀,賜第和寧門西,帝曰:「朕與東朝欲常見卿,故以自近耳。」升華容軍節
 度使,尋致仕。後華容軍復為岳陽軍,公裔遂換岳陽軍節度使。高宗既內禪,嘗與孝宗語其忠勞,因詔所居郡善視之。乾道二年卒,年七十五,贈太尉,謚恭榮,官其親族八人。高宗賜金帛甚厚。



 公裔律身稍謹,不植勢,不市恩,又敢與黃潛善、秦檜異,斯亦足取云。



 論曰:章誼有蹇諤之節,肖冑席父祖之蔭,二人多所論建,奉使不辱,亦可取矣。陳公輔得諫臣之體,其劾蔡京、王黼之黨,論吳敏、李綱之隙,是矣。然既辨安石學術之
 害,而不尚程頤之學,何邪?張觷斥蔡京之禍,薦楊時之賢,其趣操正矣,況平寇有術,而不自以為功乎?松年鄙秦檜而不交,知命通方,固不易得。而曹勛崎嶇兵間,稍著勞效,然金人入侵之計已決,猶曰鄰國恭順無他,何其見幾之不早邪?若李稙、韓公裔早著忠藎,為天子故人,能與黃潛善、秦檜為異,閉門不出,待時而動,斯亦知所向方者哉!



\end{pinyinscope}