\article{本紀第四十}

\begin{pinyinscope}

 寧
 宗四



 十年春正月癸巳,雨土。乙未,大風。庚子,遣錢撫賀金主生辰。



 二月庚申,地震。



 夏四月丁未朔,金人犯光州中渡鎮,執榷場官盛允升殺之,遂分兵犯樊城。戊申,鄂州、江
 陵府副都統王守中引兵拒之,金人遂分兵圍棗陽、光化軍。丙辰,詔江淮制置使李玨、京湖制置使趙方措置調遣,仍聽便宜行事。丁巳,命四川制置使董居誼酌量緩急,便宜行事。辛酉,廬州鈐轄王辛敗金人於光山縣之安昌砦,殺其統軍完顏掩。壬戌,金兵遁去,隨州、光化皆以捷聞。丁卯,詔出戍官兵金給其家。



 五月辛巳,以久雨,釋大理、三衙、臨安府杖以下囚,蠲茶鹽賞錢。甲申,賜禮部進士吳潛以下五百二十有三人及第、出身。癸卯,
 趙方請下詔伐金,遂傳檄招諭中原官吏軍民。



 六月庚戌,太白晝見。戊午,詔厲將士,募京西忠義人進討。辛未,東川大水。癸酉,太白經天。



 秋七月丙子朔,日有食之。戊寅,以旱,釋諸路杖以下囚。甲申,雅州蠻寇邊,焚碉門砦,遣兵討之。丁亥,嗣濮王不儔薨。庚子,詔諸軍將佐有罪者送屯駐州鞫之,罷軍士淫刑。



 八月乙丑,詔監司、郡守各舉威勇才略可將帥者二人。



 冬十月乙巳朔,以久雨,釋大理、三衙、臨安府及兩浙諸州杖以下囚。癸酉,蠲三
 衙、江上諸軍公私逋負錢。



 十一月丁丑,大風。庚辰,太白晝見。甲申,詔浙東提舉司發米十萬石振給貧民。戊戌,太白經天。十二月戊申,以軍興,募民納粟補官。乙卯,詔武舉人毋復應文舉。癸亥,金鳳翔副統軍完顏贇以步騎萬人犯四川。戊辰,迫湫池堡。己巳,破天水軍,守臣黃炎孫遁。金人攻白環堡,破之。庚午,迫黃牛堡,統制劉雄棄大散關遁,金人據之。



 十一年春正月壬午,京東路忠義李全率眾來歸,詔以
 全為京東路總管。戊子,金人圍皂郊堡。壬辰,利州將麻仲率忠義人焚秦州永寧砦。乙未,以度僧牒千給四川軍費。丁酉,詔四川忠義人立功,賞視官軍。金人犯隔芽關,興元都統李貴遁,官軍大潰。



 二月甲辰,金人焚大散關而去。乙巳,沔州都統王大才馬蹶,死於河池。丙午,金人破皂郊,死者五萬人。丁未,金人破湫池堡。戊申,金人圍隨州、棗陽軍,游騎至漢上,均州守臣應謙之棄城走。丙辰,白虹貫日。楚州鈐轄梁昭祖焚金人糧舟於大清
 河,京東忠義副都統沉鐸遣兵助之。



 三月丁丑,金人焚湫池堡而去。戊子,利州統制王逸等率忠義人復皂郊,金副統軍完顏贇、包長壽遁去,沔州軍士郭雄追斬贇首,長壽僅以身免。己丑,沔州都統劉昌祖至皂郊。辛卯,忠義人十萬餘出攻秦州,官軍繼進,至赤谷口,王逸傳昌祖之命退師,且放散忠義人,軍大潰。癸巳,包長壽合長安、鳳翔之眾,復攻皂郊,遂趨西和州。是日,鎮江忠義統制彭惟誠等敗於泗州。丙申,劉昌祖焚西和州遁,守臣
 楊克家棄城去。戊戌,金人破西和州。



 夏四月甲辰,劉昌祖焚成州遁,守臣羅仲甲棄城去。是日,金人去西和州。戊申,命四川增印錢引五百萬以給軍費。階州守臣侯頤棄城去。是日,金人去成州。戊午,金人復犯大散關,守將王立遁。己未,金人犯黃牛堡,興元都統吳政拒退之。癸亥,政至大散關,執王立斬之。



 五月乙亥,命四川制置司招進忠義人。癸未,蚩尤旗見,其長竟天。丁亥,詔侍從、臺諫、兩省官集議平戎、御戎、和戎三策。壬辰,申嚴試法
 官七等之制。



 六月辛酉,詔湖州振恤被水貧民。



 秋七月癸酉,奪知天水軍黃炎孫三官、辰州居住。乙酉,修《孝宗寶訓》。辛卯,蠲四川關外諸州稅役。甲午,蠲光州民兵戰死之家稅役。



 九月己卯,朝獻於景靈宮。庚辰,朝鄉於太廟。辛己,合祭天地於明堂,大赦。辛卯,安定郡王伯渾薨。丙申,興元都統吳政、利州副都統張威各進三官。劉昌祖奪五官、韶州安置。冬十月丙午,羅仲甲、楊克家、侯頤並奪三官,仲甲常德府、克家道州、頤撫州居住。戊午,大
 風。壬戌,修盱眙軍城。



 十一月壬申,金人攻安豐軍之黃口灘。是月,陜西人張羽來歸。



 十二年春正月戊辰朔,召董居誼詣行在。以新利州路安撫使聶子述為四川制置使。庚辰,金人犯湫池堡,守將石宣拒退之。甲申,金人攻白環堡,守將董照拒退之。戊子,金人犯成州,沔州都統張威自西和州退守仙人原。庚寅,金人犯隨州、棗陽軍,又破信陽軍之二砦,京西諸將引兵拒之。辛卯,金人犯西和州,守臣趙彥吶設伏
 以待之,殲其眾乃還。金人犯安豐軍,建康都統許俊遣將卻之。金人焚成州,犯河池,守將張斌遁去。癸巳,金人圍安豐軍及光州,攻光化軍,破鄖山縣,進副均州。甲午,破鳳州,守臣雷雲棄城去,金人夷其城。乙未,興元都統吳政及金人戰於黃牛堡,死之。金人乘勝攻武休關。



 二月戊戌朔,金人破光山縣。太白晝見。壬寅,金人圍棗陽軍,京湖制置使趙方遣統制扈再興救之,不克進而還。癸卯,金人破武休關,興元都統李貴遁還,利州路提刑、
 權興元府事趙希昔棄城去。丁未,金人破興元府。戊申,金人攻棗陽軍。己酉,遣殿前司軍八千人防捍江面。庚戌,以曾從龍同知樞密院事兼江、淮宣撫使,權吏部尚書任希夷簽書樞密院事。辛亥,金人破大安軍,守臣李文子棄城去。金人犯洋州,守臣蔡晉卿遣兵拒之,不克,洋州破。壬子,四川制置使董居誼自利州遁。沔州都統張威遣統制石宣等邀擊金人於大安軍,大破之,獲其將巴土魯安,金人遂去興元府。丙辰,金人去洋州。丁巳,
 京湖制置使趙方遣統制扈再興等引兵三萬餘人出攻唐、鄧二州,隨州忠義統領劉世興等引兵攻唐州。甲子,金人去棗陽軍。乙丑,夏人復以書來四川,議夾攻金人,利州路安撫丁煜許之。



 三月己巳,以鄭昭先知樞密院事,曾從龍參知政事。癸酉,金人復入洋州,焚其城而去。乙亥,興元軍士權興等作亂,犯巴州,守臣秦季□□棄城去。鄂州統制劉世榮會兵攻唐州。丁亥,太白晝見。權興等降。癸巳,雨土。甲午,金人自盱眙退師。



 閏月己未,追
 雷雲三官、梅州安置。辛酉,贈吳政為右武大夫、忠州刺史。壬戌,詔撫諭四川官軍、忠義人。癸亥,興元軍士張福、莫簡等作亂,以紅巾為號。是春,金人圍安豐軍、滁、濠、光三州。江、淮制置使李玨命池州都統武師道、忠義軍統制陳孝忠救之,皆不克進。金人遂分兵自光州犯黃州之麻城,自濠州犯和州之石磧,自盱眙軍犯滁州之全椒、來安及揚州之天長、真州之六合。淮南流民渡江避亂,諸城悉閉。金人游騎數百至東採石、楊林渡,建康大
 震。京東總管李全自楚州、忠義總轄季先自漣水軍各引兵來援,金人乃解去。全追擊,敗之於曹家莊,獲其貴將。



 夏四月庚午,張福入利州,四川制置使聶子述遁,殺總領財賦楊九鼎。丁丑,張福掠閬州,丁亥,掠果州。癸巳,曾從龍罷。以鄭昭先兼參知政事,崇信軍節度使、開府儀同三司、萬壽觀使安丙為四川宣撫使。董居誼落職,奪三官。



 五月乙未朔,召聶子述詣行在。張福薄遂寧府,潼川府路轉運判官、權府事程遇孫棄城遁。丁酉,減兩
 淮、荊襄、湖北、利州路沿邊諸州雜犯死罪囚,釋流以下,仍蠲今年租稅。己亥,太學生何處恬等伏闕上書,以工部尚書胡矩欲和金人,請誅之以謝天下。張福入遂寧府,焚其城。甲寅,四川宣撫司命沔州都統張威引兵捕福。戊午,福入普州,守臣張已之棄城遁。癸亥,詔侍從、兩省、臺諫各舉文武可用之才二三人。



 六月戊辰,張福屯普州之茗山。庚午,張威引兵至。丙子,太白晝見。辛巳,西川地震。太白晝見。癸未,張福請降,乙酉,張威執之,歸於
 宣撫司。丁亥,嗣濮王不嫖薨。金國招諭李全等,不聽。辛卯,太白經天。癸巳,丁煜復以書約夏國攻金人。



 秋七月丙申,張福伏誅。復奪董居誼二官、永州居住。庚子,張威捕賊眾一千三百餘人誅之,莫簡自殺,紅巾賊悉平。癸亥,李全引兵至齊州,知州王贇以城降。



 八月戊辰,復合利州東、西路為一。



 九月丙午,罷江、淮制置司,置沿江、淮東西制置司。以寶文閣待制李大東為沿江制置使,淮南轉運判官趙善湘為主管淮西制置司公事,淮東提
 刑賈涉為主管淮東制置司公事兼節制京東、河北路軍馬。



 十一月辛亥,進封楊次山為會稽郡王。十二月壬申,京東節制司言復京東、河北二府九州四十縣。乙亥,築興元府城。丁丑,雅州蠻入盧山縣。己卯,四川宣撫司遣兵取洮州,召諸將議出師,招諭中原豪傑。辛巳,蠻焚碉門砦,邊丁大敗。乙酉,金人犯鳳州之長橋。丁亥,四川宣撫司命罷洮州之師。己丑,京湖置司遣統制扈再興等引兵六萬人,分二道出境。庚寅,賞茗山捕賊功。



 十三年春正月丁酉,扈再興引兵攻鄧州,鄂州都統許國攻唐州,不克而還。金人追之,遂攻樊城,趙方督諸將拒退之。己亥,雅州蠻復掠盧山縣,遣兵討之。己酉,命不凌為嗣濮王。戊午,夏人復以書來四川,議夾攻金人。



 三月辛卯朔,雨土。丁巳,黎州土丁叛,遣兵討之。



 夏四月庚申朔,淮東制置賈涉招諭山東、兩河豪傑。



 五月庚寅朔,雅州蠻降。戊戌,史彌遠等上《玉牒》及《三祖下第七世宗藩慶系錄》。



 六月癸酉,賜禮部進士劉渭以下四百七十
 有五人及第、出身。加安丙少保。丙子,以李全為左武衛大將軍。壬午,以季先為果州團練使、漣水軍忠義副都統,命赴樞密院議事,未至,殺之。



 秋七月戊戌,以京東、河北諸州守臣空名官告付京東、河北節制司,以待豪傑之來歸者。丙午,以任希夷兼參知政事。丙辰,四川宣撫司招黎人土丁,降之。



 八月癸亥,皇太子詢薨,謚曰景獻。壬申,安丙遺夏人書,定議夾攻金人。癸未,四川宣撫司命利州統制王仕信引兵赴熙、鞏州會夏人,遂傳檄招
 諭陜西五路官吏軍民。甲申,復海州,以將作監丞徐晞稷知州事。盱眙將石珪叛入漣水軍,詔以珪為漣水忠義軍統轄。



 九月辛卯,夏人引兵圍鞏州,且來趣師。甲午,太白晝見。王仕信引兵發宕昌。乙未,四川宣撫司統制質俊、李寔引兵發下城。戊戌,四川宣撫司命諸將分道進兵,沔州都統張威出天水,利州副都統程信出長道,興元副都統陳立出大散關,興元統制田胃為宣撫司帳前都統出子午谷,金州副都統陳昱出上津。己亥,張
 威下令所部諸將毋得擅進兵。庚子,質俊等克來遠鎮。辛丑,王仕信克鹽川鎮。壬寅,質俊等自來遠鎮進攻定邊城,金人來救,俊等擊破之。乙巳,程信、王仕信引兵與夏人會於鞏州城下。丁未,攻城不克。庚戌,金人犯皂郊堡,沔州統制董照等與戰,大敗。壬子,程信及夏人攻鞏州不克,信引兵趨秦州。丙辰,夏人自安遠砦退師。



 冬十月丁巳朔,程信邀夏人共攻秦州,夏人不從,信遂自伏羌城引軍還,諸將皆罷兵。戊寅,程信以四川宣撫司之
 命,斬王仕信於西和州。四川宣撫司以張威不進兵,罷其軍職。



 十一月庚戌,大風。壬子,臨安府火。十二月戊午,大風。壬申,漣水忠義軍統轄石珪叛。癸未,鎮江副都統翟朝宗以「皇帝恭膺天命之寶」來獻。



 十四年春正月丙戌朔,以雪寒,釋大理、三衙、臨安、兩浙諸州杖以下囚。乙未,地震。以李全還自山東,賜緡錢六萬。庚子,立四川運米賞格。



 二月戊辰,金人圍光州。己巳,金人犯五關。壬申,金人治舟於團風,弗克濟,遂圍黃州,
 分兵破諸縣,又遣別將犯漢陽軍。丁丑,李全棄泗州遁,還。甲申,詔淮東、京湖諸路應援淮西,沿江制置司防守江面,權殿前司職事馮榯將兵駐鄂州,京東忠義都統李全將兵救蘄、黃,榯不果行。



 三月丙戌朔,鄂州副都統扈再興引兵攻唐州。丁亥,金人破黃州,淮西提刑、知州事何大節棄城遁死。庚寅,長星見。李全自楚州引兵援淮西。癸巳,扈再興引所部趨蘄州。甲午,太白晝見。乙未,詔京湖制置司趣援蘄、黃。己亥,金人陷蘄州,知州事李
 誠之及其家人、官屬皆死之。癸丑,金人退師,扈再興邀擊,敗之於天長鎮,甲寅晦,又敗之。



 夏四月乙卯,復置諸王宮大、小學教授。乙丑,命任子簾試於御史臺。戊辰,金人渡淮而北,李全遣兵追擊,敗之。



 五月甲申朔,日有食之。壬辰,史彌遠等上《孝宗寶訓》、《皇帝會要》。丙申,西川地震。乙巳,頒《慶元寬恤詔令》。



 六月甲寅朔,初置沿江制置副使司於鄂州。丙寅,詔以侄福州觀察使貴和為皇子,更名竑,進封祁國公。丁卯,以立皇子告於天地、宗廟、社
 稷。乙亥,以太祖十世孫與莒補秉義郎。丙子,減京畿囚罪一等,釋杖以下。辛巳,大風。



 秋七月辛丑,以趙方為京湖制置大使,賈涉為淮東制置使兼京東、河北路節制使。丁未,修《光宗寶訓》。



 八月乙卯,賜史彌遠家廟。任希夷罷。壬戌,以兵部尚書宣繒同知樞密院事,給事中俞應符簽書樞密院事。甲子,以秉義郎與莒為右監門衛大將軍,賜名貴誠。乙丑,追封史浩為越王,改謚忠定,配享孝宗廟庭。戊寅,以侄右監門衛大將軍貴誠為果州團
 練使。



 九月癸未,立貴誠為沂靖惠王後。己丑,朝獻於景靈宮。庚寅,朝饗於太廟。辛卯,合祭天地於明堂,大赦。



 冬十月癸丑,京東、河北節制司言復滄州,詔以趙澤為河北東路鈐轄、知州事。甲寅,復以齊州為濟南府,兗州為襲慶府。丙寅,夏人復以書來四川趣會兵。庚午,雷。



 十一月己亥,安丙薨。是月,京東安撫張林叛。十二月庚申,鄭昭先罷。



 閏月辛巳朔,以宣繒兼參知政事,俞應符兼權參知政事。戊申,以殿前司同正將華岳等謀為變,殺之。
 是歲,浙東、江西、福建諸路旱,沔、成、階、利四州水,振之。



 十五年春正月庚戌朔,御大慶殿,受恭膺天命之寶。癸丑,立李誠之廟於蘄州。甲寅,褒贈蘄州死事官吏,錄其子孫有差。丁巳,詔撫諭山東河北軍民、將帥、官吏。己未,以受寶,大赦,文武官各進秩一級,大犒諸軍。



 二月庚子,罷御史臺簾試任子法。



 三月丁巳,詔江西提舉司振恤旱傷州縣。



 夏四月壬午,詔蠲蘄州今年租賦。



 五月庚戌,太白晝見。甲寅,詔監司慮囚,察州縣匿囚者劾之。丁巳,
 進封子祁國公竑為濟國公。己未,以侄果州團練使貴誠為邵州防禦使。壬戌,知濟南府種斌等攻張林於青州,林遁去。己巳,修《孝宗經武要略》。



 六月辛卯,俞應符薨。



 秋七月甲子,詔江淮、荊襄、四川制置監司條畫營田來上。



 八月己卯,命戶部詳議義役。辛卯,詔文武官毋得歸宗,著為令。甲午,有彗星出於氐。



 九月辛亥,以宣繒參知政事,給事中程卓同知樞密院事,吏部尚書薛極賜出身,簽書樞密院事。癸丑,雷,大雨雹。丁巳,復以隨州三關
 隸德安府,置關使。壬戌,彗星沒。辛未,太白晝見。



 冬十月丙子,以收復京東州軍,犒賞忠義有差。



 十一月戊午,赦京東、河北路。十二月乙亥朔,發米振給臨安府貧民。丙子,以雪寒,釋京畿及兩浙諸州杖以下囚。丁亥,以李全為保寧軍節度使、右金吾衛上將軍、京東路鎮撫副使。



 十六年春正月戊申,詔命官犯贓毋免約法。己酉,子坻生,辛酉,命淮東制置司振給山東流民。



 二月戊子,雨土。己丑,嗣秀王師禹薨,追封和王。戊戌,子坻薨,追封邳王,
 謚沖美。



 三月戊申,張林所部邢德來歸,詔進二官,復以為京東東路副總管。丁卯,以道州民饑,詔發米振之。夏五月甲辰,詔右選試注官如左選之制。戊申,賜禮部進士蔣重珍以下五百四十有九人及第、出身。戊辰,詔復潭州稅酒法。



 六月丁酉,程卓薨。秋八月辛巳,詔州縣經界毋增紹興稅額。癸未,申嚴泊船銅錢之禁。



 九月庚子朔,日有食之。乙巳,詔江、淮諸司振恤被水貧民。乙卯,雷。冬十一月辛亥,以太平州大水,詔振恤之。



 十二月辛巳,
 命淮東、西總領及沿江被水州募江西、湖南民入米補官。癸未,嗣濮王不凌薨。壬辰,雷。



 十七年春正月戊戌朔,詔補先聖裔孔元用為通直郎,錄程頤後。癸亥,命淮東西、湖北路轉運司提督營屯田。



 二月癸巳,蠲臺州逋賦十萬餘緡。甲午,命臨安府振糶貧民。



 三月癸丑,雪。是月,金人迫西和州,尋引兵還。



 夏四月辛卯,詔廬州振糶饑民。乙未,賜李全、彭義斌錢三十萬緡為犒賞戰士費。



 五月戊戌,詔核實兩淮、京湖、四川、
 江上諸軍之數。



 六月丁卯朔,太白經天,晝見。癸酉,知西和州尚震午坐金兵至謀遁,奪三官、岳州居住。壬辰,大名府蘇椿等舉城來歸,詔悉補官,即以其州授之。



 秋七月丁酉朔,命福建路監司振恤被水貧民。辛亥,命師巖嗣秀王。



 八月乙亥,罷通州天賜鹽場。丙戌,帝不豫。閏八月乙未朔,申嚴兩浙諸州輸苗過取之禁。丁酉,皇帝崩於福寧殿,年五十七。史彌遠傳遺詔,立侄貴誠為皇子,更名昀,即皇帝位。尊皇后為皇太后,垂簾聽政。進封皇子
 竑為濟陽郡王,出居湖州。寶慶元年正月己丑,謚曰仁文哲武恭孝皇帝,廟號寧宗。三月癸酉,葬於會稽之永茂陵。三年九月,加謚法天備道純德茂功仁文哲武聖睿恭孝皇帝。



 贊曰:宋世內禪者四,寧宗之禪,獨當事勢之難,能不失禮節焉,斯可謂善處矣。初年以舊學輔導之功,召用宿儒,引拔善類,一時守文繼體之政,燁然可觀。中更侂冑用事,內蓄群奸,至指正人為邪,正學為偽,外挑強鄰,
 流毒淮甸。頻歲兵敗,乃函侂冑之首,行成於金,國體虧矣。既而彌遠擅權,幸帝耄荒,竊弄威福。至於皇儲國統,乘機伺間,亦得遂其廢立之私,他可知也。雖然,宋東都至於仁宗,四傳而享國百年,邵雍稱為前代所無,南渡至寧宗,亦四傳而享國九十有八年,是亦豈偶然哉。惜乎神器授受之際,寧、理之視仁、英,其跡雖同,其情相去遠矣。



\end{pinyinscope}