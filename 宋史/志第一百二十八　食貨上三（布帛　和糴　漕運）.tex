\article{志第一百二十八 食貨上三(布帛 和糴 漕運)}

\begin{pinyinscope}

 布帛宋承前代之制,調絹、綢、絹、布、絲、綿以供軍須,又就所產折科、和市。其纖麗之物,則在京有綾錦院,西京、真定、青益梓州場院主織錦綺、鹿胎、透背,江寧府、潤州有織
 羅務,梓州有綾綺場,亳州市縐紗,大名府織縐縠,青、齊、鄆、濮、淄、濰、沂、密、登、萊、衡、永、全州市平絁。東京榷貨務歲入中平羅、小綾各萬匹,以供服用及歲時賜與。諸州折科、和市,皆無常數,唯內庫所須,則有司下其數供足。自周顯德中,受公私織造並須幅廣二尺五分,民所輸絹匹重十二兩,疏薄短狹、塗粉入藥者禁之;河北諸州軍重十兩,各長四十二尺。宋因其舊。



 開寶三年,令天下諸州凡絲、綿、綢、絹麻布等物,所在約支二年之用,不得廣
 科市以煩民。初,蓬州請以租絲配民織綾,給其工直,太祖不許。太宗太平興國中,停湖州織綾務,女工五十八人悉縱之。詔川峽市買場、織造院,自今非供軍布帛,其錦綺、鹿胎、透背、六銖、欹正、龜殼等段匹,不須買織,民間有織賣者勿禁。馬元方為三司判官,建言:「方春乏絕時,預給庫錢貸民,至夏秋令輸絹於官。」大中祥符三年,河北轉運使李士衡又言:「本路歲給諸軍帛七十萬,民間罕有緡錢,常預假於豪民,出倍稱之息,至期則輸賦之
 外,先償逋欠,以是工機之利愈薄。請預給帛錢,俾及時輸送,則民獲利而官亦足用。」詔優予其直。自是諸路亦如之。或蠶事不登,許以大小麥折納,仍免倉耗及頭子錢。



 天聖中,詔減兩蜀歲輸錦綺、鹿胎、透背、欹正之半,罷作綾花紗。明道中,又減兩蜀歲輸錦綺、綾羅、透背、花紗三之二,命改織綢、絹以助軍。景祐初,遂詔罷輸錦背、繡背、遍地密花透背段,自掖庭以及閭巷皆禁用。其後歲輒增益梓路紅錦、鹿胎,慶歷四年復減半。既而又減梓
 路歲輸絹三之一,紅錦、鹿胎半之。先是,咸平初,廣南西路轉運使陳堯叟言:「準詔課植桑棗,嶺外唯產苧麻,許令折數,仍聽織布赴官場博市,匹為錢百五十至二百。」



 至是,三司請以布償芻直,登、萊端布為錢千三百六十,沂布千一百,仁宗以取直過厚,命差減其數。自西邊用兵,軍須綢絹,多出益、梓、利三路,歲增所輸之數;兵罷,其費乃減。嘉祐三年,始詔寬三路所輸數。治平中,歲織十五萬五千五百餘匹。



 神宗即位,京師米有餘蓄,命發運
 司損和糴數五十萬石,市金帛上京,儲之榷貨務,備三路軍須。京東轉運司請以錢三十萬二千二百貫給貸於民,令次年輸絹,匹為錢千,隨夏稅初限督之。詔運其錢於河北,聽商人入中。



 熙寧三年,御史程顥言:「京東轉運司和買綢絹,增數抑配,率千錢課絹一匹,其後和買並稅絹,匹皆輸錢千五百。」時王廣淵為轉運使,謂和買如舊,無抑配。顥言其迎合朝廷意。王安石謂廣淵在京東盡力以赴事功,不宜罪以迎合。乃詔所給內帑別額
 綢絹錢五十萬緡,收其本儲之北京,息歸之內帑。右正言李常亦言:「廣淵以陳汝羲所進羨餘錢五十萬緡,隨和買絹錢分配,於常稅折科放買外,更取二十五萬緡,請以顥言付有司。」定州安撫司又言:「轉運司配綢、絹、綿、布於州鎮軍砦等坊郭戶,易錢數多,乞憫其災傷,又居極邊,特蠲損之。」詔提刑司別估,民不願市,令官自賣,已給而抑配者正之。自王安石秉政,專以取息為富國之務,故當時言利小人如王廣淵輩,假和買綢絹之名,配
 以錢而取其五分之息,其刻又甚於青苗。然安石右廣淵,顥、常言卒不行。二月,詔移巴蜀羨財,市布帛儲於陜西以備邊,省蜀人輸送及中都漕挽之費。



 七年,兩浙察訪沉括言:「本路歲上供帛九十八萬,民苦備償,而發運司復以移用財貨為名,增預買綢絹十二萬。」詔罷其所增之數。八年,韓琦奏倚閣預買綢絹等,雖稍豐稔,猶當五七歲帶輸。安石以為不然,言於神宗曰:「預買綢絹,祖宗以來未嘗倚閣,往歲李稷有請,因從之。近方鎮監司
 爭以寬恤為事,不計有無,異日國用闕,當復刻剝於民爾。」



 元豐以來,諸路預買綢絹,許假封樁錢或坊場錢,少者數萬緡,多者至數十萬緡。其假提舉司寬剩錢者,又或令以絹帛入常平庫,俟轉運司以價錢易取。三年,京東轉運司請增預買數三十萬,即本路移易,從之。四年,遣李元輔變運川陜四路司農物帛。中書言:物帛至陜西,擇省樣不合者貿易,糴糧儲於邊,期以一年畢。五年,戶部上其數凡八百十六萬一千七百八十匹兩,三百
 四十六萬二千緡有奇。



 紹聖元年,兩浙絲蠶薄收,和買並稅綢絹,令四等下戶輸錢,易左帑綢絹;又令轉運司以所輸錢市金銀,遇蠶絲多,兼市紗、羅、綢、絹上供。元符元年,雄州榷場輸布不如樣,監司、通判貶秩、展磨勘年有差;令損其直,後似此者勿受。



 尚書省言:「民多願請預買錢,宜視歲例增給,來歲市綢絹計綱赴京。」左司員外郎陳瓘言:「預買之息,重於常平數倍,人皆以為苦,何謂願請?今復創增,雖名濟乏,實聚斂之術。」提點京東刑獄
 程堂亦言:「京東、河北災民流未復,今轉運司東西路歲額無慮二百萬匹兩,又於例外增買,請罷之。」乃詔諸路提舉司勿更給錢,俟蠶麥多,選官置場。崇寧中,諸路預買,令所產州縣鄉民及城郭戶並準貲力高下差等均給。川陜路取元豐數最多一年為額,舊不給者如故。江西和買綢絹歲五十萬匹,舊以錢、鹽三七分預給。自鹽鈔法行,不復給鹽,令轉運司盡給以錢,而卒無有,逮今五年,循以為常,民重傷困。大觀初,詔假本路諸司封樁
 錢及鄰路所掌封樁鹽各十萬緡給之。其後提舉常平張根復言:「本路和買,未嘗給錢,請盡給一歲蠶鹽,許轉運司移運或民戶至場自請。」而江西十郡和買數多,法一匹給鹽二十斤,比錢九百,歲預於十二月前給之。轉運司得鹽不足,更下發運司會積歲所負給償。



 尚書省言大觀庫物帛不足,令兩浙、京東、淮南、江東西、成都、梓州、福建路市羅、綾、紗一千至三萬匹各有差。二年,又令京東、淮南、兩浙市絹帛五萬及三萬匹,並輸大觀庫;又
 四川各二萬,輸元豐庫。江東西如四川之數,輸崇寧庫。而州縣和買,有以鹽一席折錢六千,令民至期輸綢絹六匹,又前期督促,致多逃徙,詔遞加其罪。坊郭戶預買有加至四五百匹,興仁府萬延嗣戶業錢十四萬二千緡,歲均千餘匹,乃令減半均之。



 兩浙和買並稅綢絹布帛,頭子錢外,又收市例錢四十,例外約增數萬緡,以分給人吏。政和初,詔罷市例錢。諸路綢絹布帛比價高數倍,而給直猶用舊法,言者請稍增之,度支以元豐例定,
 沮抑不行,令如期給散而已。江東和買,弊如江西,比而才給二百,轉運司又以重十三兩為則,不及則準絲價補納以錢,兩準二百有餘。宣和三年,詔提刑司厘正以聞。先是,成都、河北預買,官戶許減半,四年,令舊嘗全科者如舊。即又以兩浙多官戶,令預買通敷。七年冬,郊祀,河北、京東和買科取物帛絲綿等數並免,以供奉物給降,其所蠲貸,幾數百萬。



 初,預買綢絹務優直以利民,然猶未免煩民,後或令民折輸錢,或物重而價輕,民力浸
 困,其終也,官不給直,而賦取益甚矣。十二月,詔令轉運司各會一路之數,分下州縣經畫,不以錢以他物、不以正月以他月給者,並論以違制。然有司鮮能承順焉。靖康元年,命轉運司以常平錢前一季預備,如正月之期給之,毋貸以他物而損其數。京東州縣勿以遷移戶舊數科著業人,仍先除其數,俟流民歸業均敷。餘路亦如之。



 建炎三年春,高宗初至杭州,朱勝非為相。兩浙轉運副使王琮言:「本路上供、和買、夏稅綢絹,歲為匹一百一
 十七萬七千八百,每匹折輸錢二千以助用。」詔許之。東南折帛錢自此始。五月,詔每歲預買綿絹,令登時給其直。又詔江、浙和預買絹減四分之一,仍給見錢,違者置之法。紹興元年,初賦鼎州和買折帛錢六萬緡,以贍蔡兵。以兩浙夏稅及和買綢絹一百六十餘萬匹,半令輸錢,匹二千。二年,以諸路上供絲、帛並半折錢如兩浙例,江、淮、閩、廣、荊湖折帛錢自此始。時江、浙、湖北、夔路歲額綢三十九萬匹,江南、川、廣、湖南、兩浙絹二百七十三萬
 匹,東川、湖南綾羅絁七萬匹,西川、廣西布七十七萬匹,成都錦綺千八百餘匹,皆有奇。



 三年三月,以兩浙和買物帛,下戶艱於得錢,聽以七分輸正色,三分折見緡。初,洪州和買,八分輸正色,二分折省錢,匹三千。四年,帥臣胡世將請以三分匹折六千省。又言絹直踴貴,請匹增為五千匹。戶部定為六千匹。殿中侍御史張致遠言:「江西殘破之餘,和預買絹請折輸錢,朝廷從之,是欲少寬民力。匹輸錢五千省,比舊直已增其半,較之兩浙時直,
 匹多一千五百,戶部又令折六貫文足,是欲乘民之急而倍其斂也。物不常貴,則絹有時而易辦;錢額既定,則價無時而可減。」於是詔江西和買絹匹折輸錢六十省,願輸正色者聽。是冬,初令江、浙民戶悉輸折帛錢。當是時,行都月費錢百餘萬緡,重以增戍之費,令民輸綢者全折,輸絹者半折,匹五千二百省。折帛錢由此愈重。



 九年正月,復河南,減折帛錢匹一千,未幾又增之。十七年,減折帛錢:江南匹為六千,兩浙七千,和買六千五百;綿,
 江南兩為三百,兩浙四百。二十年,詔:「廣西折布錢因張浚增至兩倍以上,今減作一貫文折輸。」二十九年,中書省奏:江、浙四路所起折帛錢,地里遙遠,宜就近儲之。詔除徽、處、廣德舊折輕貨,餘州當折銀者輸錢,願輸銀者聽,浙西提刑司、三總領所主之。先是,江、浙路折帛錢歲為錢五百七十三萬餘緡,並輸行都,至是,始外儲之以備軍用。



 乾道四年,減兩浙、乾道五年夏稅、和買折帛錢之半。六年,知徽州郟升卿代還,奏:「州自五代時陶雅守
 郡,妄增民賦,至今二百餘年,比鄰境諸縣之稅獨重數倍,而雜錢之稅科折尤重,請賜蠲免。」九年,詔徽州額外創科雜錢一萬二千一百八十餘緡,及元認江東、兩浙運司諸處絹一萬六千六百餘匹,並蠲之。



 紹熙五年,詔兩浙、江東西和買綢絹折帛錢太重,可自來年匹減錢一貫五百文,三年後別聽旨。所減之錢,令內藏、封樁兩庫撥還。



 慶元元年,戶部侍郎袁說友言臨安、餘杭二縣和買科取之弊:「乞將餘杭縣經界元科之額配以絹數,
 不分等則,以二十四貫定敷一匹,袞科而下,足額而止,捐其餘以惠末產之民。如此則吏不得而制民,民無資於詭戶,救弊之良策也。」說友又奏:「貫頭均科之法行,則縣邑無由多取,鄉司無所走弄,而詭挾者不能以幸免,是以奸民頑吏立為異論以搖之。」詔令集議。二年,吏部尚書葉翥等議請如帥漕所奏推行之,詔可。



 建炎元年,知越州翟汝文奏:「浙東和預買絹歲九十七萬六千匹,而越乃六十萬五百匹,以一路計之,當十之三。望將三
 等以上戶減半,四等以下戶權罷。」尋以杭之和買絹編重,均十二萬匹於兩浙。乾道九年,秘書郎趙粹中言:「兩浙和買,莫重於紹興,而會稽為最重。緣田薄稅重,詭名隱寄,多分子戶。自經界後至乾道五年,累經推排,減落物力,走失愈重,民力困竭。若據畝均輸,可絕詭戶之弊。」淳熙八年,詔兩淮漕臣吳琚與帥臣張子顏措置。子顏等言:「勢家豪民分析版籍以自托於下戶,是不可不抑。然弊必有原,謂如浙東七州,和買凡二十八萬一千七
 百三十有八;溫州本無科額,合臺、明、衢、處、婺之數,不滿一十三萬;而紹興一郡獨當一十四萬六千九百三十有八,則是以一郡視五郡之輸而又贏一萬有奇,此重額之弊也。又如賃牛物力,以其有資民用,不忍科配;酒坊、鹽亭戶,以其嘗趁官課,難令再敷;至於坍江落海之田,壞地漂沒;僧道寺觀之產,或奉詔蠲免;而省額未除,不免陰配民戶,此暗科之弊也。二弊相乘,民不堪命,於是規避之心生,而詭戶之患起。舊例物力三十八貫五
 百為第四等,降一文以下即為第五等,為詭戶者志於規避,往往止就二三十貫之間立為砧基。今若自有產有丁系真五等依舊不科,其有產無丁之戶,將實管田產錢一十五貫以上並科和買,其一十五貫以下則存而不敷,庶幾偽五等不可逃,真五等不受困。」於是詔:「紹興府攢宮田園、諸寺觀、延祥莊並租牛耕牛合蠲和買,並於省額除之;坊場、鹽亭戶見敷和買物力,及坍江田、放生池合減租稅物力,並核實取旨。」



 十一年,臣僚言兩
 浙、江東西四路和買不均之弊,送戶部、給舍等官詳議。鄭丙、丘崇議,畝頭均科之說至公至平,詔施行之。十六年,知紹興府王希呂言:「均敷和買,曩者亟於集事,不暇核實,一切以為詭戶而科之,於是物力自百文以上皆不免於和買,貧民始不勝其困。乞將創科和買二萬五十七匹有奇盡放,則民被實惠矣。於是詔下戶和買二萬五十餘匹住催一年,又減元額四萬四千匹有奇;均敷一節,令知紹興府洪邁從長施行。紹熙元年,邁定其
 法上之,詔依所措置推行,於是紹興貧民下戶稍寬矣。



 和糴宋歲漕以廣軍儲、實京邑。河北、河東、陜西三路及內郡,又自糴買,以息邊民飛挽之勞,其名不一。建隆初,河北連歲大稔,命使置場增價市糴,自是率以為常。咸平中,嘗出內府綾、羅、錦、綺計直緡錢百八十萬、銀三十萬兩,付河北轉運使糴粟實邊。繼而詔:凡邊州積穀可給三歲則止。大中祥符初,三路歲豐,仍令增糴廣蓄,靡限常數。後又時出內庫緡錢,或數十萬,或百萬,別遣官
 經畫市糴,中等戶以下免之。



 初,河東既下,減其租賦。有司言其地沃民勤,頗多積穀,請每歲和市,隨常賦輸送,其直多折色給之。京東西、陜西、河北闕兵食,州縣括民家所積糧市之,謂之推置;取上戶版籍,酌所輸租而均糴之,謂之對糴,皆非常制。麟、府州以轉餉道遠,遣常參官就置場和糴。河北又募商人輸芻粟於邊,以要券取鹽及緡錢、香藥、寶貨於京師或東南州軍,陜西則受鹽於兩池,謂之入中。陜西糴穀,又歲預給青苗錢,天聖以
 來,罷不復給,然發內藏金帛以助糴者,前後不可勝數。寶元中,出內庫珠直緡錢三十萬,付三司售之,取其直以助邊費。歐陽修奉使河東還,言:「河東禁並邊地不許人耕,而私糴北界粟麥為兵儲最為大患。」遂詔岢嵐、火山軍閑田並邊壕十里外者聽人耕,然竟無益邊備,歲糴如故。大抵入中利厚而商賈趨之,罷三路入中,悉以見錢和糴,縣官之費省矣。



 熙寧五年,詔以銀絹各二十萬賜河東經略安撫司,聽人賒買,收本息封樁備邊。自
 是三路封樁,所給甚廣,或取之三司,或取之市易務,或取之他路轉運司,或賜常平錢,或鬻爵、給度牒,而出內藏錢帛不與焉。



 七年,以岷州入中者寡,令三司具東南及西鹽鈔法經久通行利病以聞。知熙州王韶建議:「依沿邊和糴例,以一分見緡、九分西鈔,別約價,募入中者。凡邊部入中有闕,則多出京鈔或饒益誘之,以紓用度。」是歲,河東並邊大稔,詔都轉運使李師中與劉庠廣糴,積五年之蓄。復命輔臣議,更與陜西並塞芻糧之法,令
 轉運司增舊糴三分,以所糴虧羨為賞罰,仍遣吏按視。而陜西和糴,或以錢、茶、銀、綢、絹糴於弓箭手。



 八年,河東察訪使李承之言:「太原路二稅外有和糴糧草,官雖量予錢、布,而所得細微,民無所濟,遇歲兇不蠲,最為弊法。」繼而知太原韓絳復請和糴於元數省三分,罷支錢、布,乞精選才臣講求利害。詔委陳安石。元豐元年,安石奏:「河東十三州一稅,以石計凡三十九萬二千有餘,而和糴數八十三萬四千有餘,所以歲兇仍輸者,以稅輕、軍
 儲不可闕故也。舊支錢、布相半,數既奇零,以鈔貿易,略不收半,公家實費,百姓乃得虛名。欲自今罷支糴錢,歲以其錢令並邊州郡和市封樁,即歲災以填所蠲數,年豐則三歲一免其輸。」朝廷以為然,始詔河東歲給和糴錢八萬餘緡並罷,以其錢付漕司,如安石議。因用安石為河東轉運使。其後經略使呂惠卿復請別議立法,除河外三州理為邊郡宜免,餘十一州可概均糴。下有司議,以歲和糴見數十分之,裁其二,用八分為額,隨戶色
 高下裁定,毋更給錢;歲災同秋稅蠲放,以轉運司應給錢補之,災不及五分,聽以久例支移。遂易和糴之名為助軍糧草。



 元豐四年,以度支副使蹇周輔兼措置河北糴便司。明年,詔以開封府界、諸路闕額禁軍及淮、浙、福建等路剩鹽息錢,並輸糴便司為本。令瀛、定、澶等州各置倉,凡封樁,三司毋關預,委周輔專其任,司農寺市易、淤田、水利等司所計置封樁糧草並歸之。六年,詔提點河北西路王子淵兼同措置。未幾,手詔周輔:今河朔豐
 成,宜廣收糴。是歲,大名東、西濟勝二倉,定州衍積、寶盈二倉與瀛之州倉皆成,周輔召拜戶部侍郎,以左司郎中吳雍代之。明年,雍言河北倉廩皆充實,見儲糧料總千一百七十六萬石。詔賜同措置王子淵三品服。宣和中,罷畿內和糴。



 自熙寧以來,和糴、入中之外,又有坐倉、博糴、結糴、兌糴、俵糴、寄糴、括糴、勸糴、均糴等名。其曰坐倉:熙寧二年,令諸軍餘糧願糴入官者,計價支錢,復儲其米於倉。王珪奏曰:「外郡用錢四十可致斗米於京師,
 今京師乏錢,反用錢百坐倉糴斗米,此極非計。」司馬光曰:「坐倉之法,蓋因小郡乏米而庫有餘錢,故反就軍人糴米以給次月之糧,出於一時急計耳。今京師有七年之儲,而府庫無錢,更糴軍人之米,使積久陳腐,其為利害非臣所知。」呂惠卿曰:「今坐倉得米百萬石,則減東南歲漕百萬石,轉易為錢以供京師,何患無錢?」光曰:「臣聞江、淮之南,民間乏錢,謂之錢荒。而土宜粳稻,彼人食之不盡。若官不糴取以供京師,則無所發洩,必甚賤傷農
 矣。且民有米而官不用米,民無錢而官必使之出錢,豈通財利民之道乎?」不從。明年,又慮元價賤,神、龍衛及諸司每石等第增錢收糴,仍聽行於河北、河東、陜西諸路。元符以後,有低價抑糴之弊,詔禁止之。



 其曰博糴:熙寧七年,詔河北轉運、提舉司置場,以常平及省倉歲用餘糧,減直聽民以絲、綿、綾、絹增價博買,俟秋成博糴。崇寧五年,又詔陜西錢重物輕,委轉運司措置,以銀、絹、絲、綢之類博糴觔斗,以平物價。



 其曰結糴:熙寧八年,劉佐體
 量川茶,因便結糴熙河路軍儲,得七萬餘石,詔運給焉。未幾,商人王震言:結糴多散官或浮浪之人,有經年方輸者。詔措置熙河財用孫迥究治以聞。迥奏總管王君萬負熙、河兩川結糴錢十四萬六百三十餘緡、銀三百餘兩。乃遣蔡確馳往本路劾之,君萬及高遵裕皆坐借結糴違法市易,降黜有差。崇寧初,蔡京行於陜西,盡括民財以充數。五年,以星變講修闕政,罷陜西、河東結糴、對糴。



 其曰俵糴:熙寧八年,令中書計運米百萬石費約
 三十七萬緡,帝怪其多。王安石因言:「俵糴非特省六七十萬緡歲漕之費,且河北入中之價,權之在我,遇斗斛貴住糴,即百姓米無所糶,自然價損,非惟實邊,亦免傷農力。」乃詔歲以末鹽錢鈔、在京粳米六十萬貫石,付都提舉市易司貿易。度民田入多寡,豫給錢物,秋成於澶州、北京及緣邊入米麥粟封樁。即物價踴,權止入中,聽糴便司兌用,須歲豐補償。紹聖三年,呂大忠之言,召農民相保,豫貸官錢之半,循稅限催科,餘錢至夏秋用時
 價隨所輸貼納。崇寧中,蔡京令坊郭、鄉村以等第給錢,俟收,以時價入粟,邊郡弓箭手、青唐蕃部皆然。用俵多寡為官吏賞罰。



 其曰兌糴:熙寧九年,詔淮南常平司於麥熟州郡及時兌糴。元祐二年,嘗以麥熟下諸路廣糴,詔後價若與本相當,即許變轉兌糴。



 其曰寄糴:元豐二年,糴便糧草王子淵論綱舟利害,因言:「商人入中,歲小不登,必邀厚價,故設內郡寄糴之法,以權輕重。」七年,詔河北瀛、定二州所糴數以鉅萬,而散於諸郡寄糴,恐緩
 急不相及,不若致商人自運。李南公、王子淵俱言:「寄糴法行已久,且近都倉,緩急運致非難。」於是寄糴卒不罷。



 其曰括糴:元符元年,涇原經略使章楶請並邊糴買;豫榜諭民,毋得與公家爭糴,即官儲有之,括索贏糧之家,量存其所用,盡糴入官。



 其曰勸糴、均糴:政和元年,童貫宣撫陜西議行之。鄜延經略使錢即言:「勸糴非可以久行。均糴先入其觔斗乃給其直,於有觔斗之家未有害也。坊郭之人,素無觔斗,必須外糴,轉有煩費。」疏奏,坐貶。
 時又詔河北、河東仿陜西均糴,知定州王漢之坐沮格奪職罷。未幾,遂立均糴法。三年,以歲稔,諸路推行均糴。五年,言者謂:「均糴法嚴,然已糴而不償其直,或不度州縣之力,敷數過多,有一戶而糴數百石者。」乃詔諸路毋輒均糴。既而州縣以和糴為名,低裁其價,轉運司程督愈峻,科率倍於均糴,詔約止之。宣和三年,方臘平,兩浙亦量官戶輕重均糴。明年,荊湖南、北均糴,以家業為差。勸糴之法,其後浸及於新邊,鄯廓州、積石軍蕃部患之。



 自熙寧以來,王韶開熙河,章惇營溪洞,沉起、劉彞啟交址之隙,韓存寶、林廣窮乞第之役,費用科調益繁。陜西宿兵既多,元豐四年,六路大舉西討,軍費最甚於他路。帝先慮科役擾民,令趙離廉問,頗得其事。又以糧餉粗惡,欲械斬河東、涇原漕臣,以勵其餘,卒以師興役眾,鮮克辦給。又李稷為鄜延漕臣督運,詔許斬知州以下乏軍興者,民苦折運,多散走,所殺至數千人,道斃者不在焉。於是文彥博奏言:「關陜人戶,昨經調發,不遺餘力,死
 亡之餘,疲瘵已甚。為今之計,正當勞來將士,安撫百姓,全其瘡痍,使得蘇息。」明年,優詔嘉答。初,西師無功,議者慮朝廷再舉,自是,帝大感悟,申飭邊臣固境息兵,關中以蘇。



 哲宗即位,諸老大臣維持初政,益務綏靜,邊郡類無調發,第令諸路廣糴以備蓄積,及詔陜西、麟府州計五歲之糧而已。紹聖初,乃詔河北鎮、定、瀛州糴十年之儲,餘州七年。其後陜西諸路又連歲興師,及進築鄯、湟等州,費資糧不可勝計。元符二年,涇原經略使章楶諫
 曰:「伏見興師以來,陜西府庫倉廩儲蓄,內外一空,前後資貸內藏金帛,不知其幾千萬數。即今所在糧草盡乏,漕臣計無所出,文移指空而已。今者,正休兵息民、清心省事之時,唯深察臣言,裁決斯事。若更詢主議大臣,竊恐專務興師,上誤聖聽。」主議大臣,指章惇也。時內藏空乏,陜西諸路以軍賞銀絹數寡,請給於內藏庫,詔以絹五十萬匹予之。帝謂近臣曰:「內庫絹才百萬,已輟其半矣。」



 蔡京用事,復務拓土,勸徽宗招納青唐,用王厚經置,
 費錢億萬,用大兵凡再,始克之,而湟州戍兵歲費錢一千二十四萬九千餘緡。五年,熙河蘭湟運使洪中孚言:「本道青稞畝收五石,粒當大麥之三。異時人糧給精米,馬料給青稞,率皆八折,不惟人馬之食自足,而價亦相當。今邊臣不燭事情,精米、青稞與糙米、大麥一例抵鬥給散,即公有一分之耗,私有一分之贏。會計一路歲費觔斗一百八十萬、雜色五十萬外,青稞一百三十萬,抵鬥歲費二十六萬石,石三十緡,計七百八十萬。」帝慮其
 米仍粗,將士或有饑色,乃命九折。明年,復令計斗給散,竟罷九折。又於陜西建四都倉:平夏城曰裕財,鎮戎軍曰裕國,通峽砦曰裕民,西安州曰裕邊。自夏人叛命,諸路皆謀進築,陜以西保甲皆運糧。後童貫又自將兵築靖夏、制戎、伏羌等城,窮討深入,凡六七年。至宣和末,饋餉空乏,鄜延至不能支旬月。時邊臣爭務開邊,夔、峽、嶺南不毛之地,草創郡邑,調取於民,費出於縣官,不可勝計。最後有燕山之役,雄、霸等州倉廩皆竭,兵士饑忿,有
 擲瓦石擊守貳、刃將官者。燕山郭藥師所將常勝一軍,計口給錢廩,月費米三十萬石、錢一百萬緡。河北之民力不能給,於是免夫之議興。



 初,黃河歲調夫修築埽岸,其不即役者輸免夫錢。熙、豐間,淮南科黃河夫,夫錢十千,富戶有及六十夫者,劉誼蓋嘗論之。及元祐中,呂大防等主回河之議,力役既大,因配夫出錢。大觀中,修滑州魚池埽,始盡令輸錢。帝謂事易集而民不煩,乃詔凡河堤合調春夫,盡輸免夫之直,定為永法。及是,王黼建
 議,乃下詔曰:「大兵之後,非假諸路民力,其克有濟?諭民國事所當竭力,天下並輸免夫錢,夫二十千,淮、浙、江、湖、嶺、蜀夫三十千。」凡得一千七百餘萬緡,河北群盜因是大起。



 南渡,三邊饋餉,糴事所不容已。紹興間,於江、浙、湖南博糴,多者給官告,少者給度牒,或以鈔引,類多不售,而吏緣為奸,人情大擾。於是減其價以誘積粟之家,初不拘於官、編之戶。凡降金銀錢帛而州縣阻節不即還者,官吏並徒二年。廣東轉運判官周綱糴米十五萬石,
 無擾及無陳腐,撫州守臣劉汝翼餉兵不匱,及勸誘賑糶流離,皆轉一官。七年,以饒州之糴石取耗四斗,罪其郡守。自是和糴者計剩科罪。十三年,荊湖歲稔,米斗六七錢,乃就糴以寬江、浙之民。十八年,免和糴,命三總領所置場糴之。舊制:兩浙、江、湖歲當發米四百六十九萬斛,兩浙一百五十萬,江東九十三萬,江西百二十六萬,湖南六十五萬,湖北三十五萬。



 至是,欠百萬斛有奇。乃詔臨安、平江府及淮東西、湖廣三計司,歲糴米百二十萬斛:,淮西十六萬五千,湖廣、淮東皆十五
 萬。二十八年,除二浙以三十五萬斛折錢,諸路綱米及糴場歲收四百五十二萬斛。二十九年,糴二百三十萬石以備振貸,石降錢二千,以關子、茶引及銀充其數。



 孝宗乾道三年秋,江、浙、淮、閩淫雨,詔州縣以本錢坐倉收糴,毋強配於民。四年,糴本結會子及錢銀,石錢二貫五百文。淳熙三年,詔廣西運司,糴錢以歲豐歉市直高下增減給之。



 寶慶三年,監察御史汪剛中言:「和糴之弊,其來非一日矣,欲得其要而革之,非禁科抑不可。夫禁科
 抑,莫如增米價,此已試而有驗者,望飭所司奉行。」有旨從之。紹定元年,錫銀、會、度牒於湖廣總所,令和糴米七十萬石餉軍。五年,臣僚言:「若將民間合輸緡錢使輸觔斗,免令賤糶輸錢,在農人亦甚有利,此廣糴之良法也。」從之。開慶元年,沿江制置司招糴米五十萬石,湖南安撫司糴米五十萬石,兩浙轉運司五十萬石,淮、浙發運司二百萬石,江東提舉司三十萬石,江西轉運司五十萬石,湖南轉運司二十萬石,太平州一十萬石,淮安州
 三十萬石,高郵軍五十萬石,漣水軍一十萬石,廬州一十萬石,並視時以一色會子發下收糴,以供軍餉。



 咸淳六年,都省言:「咸淳五年和糴米,除浙西永遠住糴及四川制司就糴二十萬石樁充軍餉外,京湖制司、湖南、江西、廣西共糴一百四十八萬石,凡遇和糴年分皆然。」



 漕運宋都大梁,有四河以通漕運:曰汴河,曰黃河,曰惠民河,曰廣濟河,而汴河所漕為多。太祖起兵間,有天下,懲唐季五代藩鎮之禍,蓄兵京師,以成強幹弱支之勢,
 故於兵食為重。建隆以來,首浚三河,令自今諸州歲受稅租及筦榷貨利、上供物帛,悉官給舟車,輸送京師,毋役民妨農。開寶五年,率汴、蔡兩河公私船,運江、淮米數十萬石以給兵食。是時京師歲費有限,漕事尚簡。至太平興國初,兩浙既獻地,歲運米四百萬石。所在雇民挽舟,吏並緣為奸,運舟或附載錢帛、雜物輸京師,又回綱轉輸外州,主藏吏給納邀滯,於是擅貿易官物者有之。八年,乃擇干強之臣,在京分掌水陸路發運事。凡一綱
 計其舟車役人之直,給付主綱吏雇募,舟車到發、財貨出納,並關報而催督之,自是調發邀滯之弊遂革。



 初,荊湖、江、浙、淮南諸州,擇部民高貲者部送上供物,民多質魯,不能檢御舟人,舟人侵盜官物,民破產不能償。乃詔牙吏部送,勿復擾民。大通監輸鐵尚方鑄兵器,鍛練用之,十裁得四五;廣南貢藤,去其粗者,斤僅得三兩。遂令鐵就冶郎淬治之,藤取堪用者,無使負重致遠,以勞民力。汴河挽舟卒多饑凍,太宗令中黃門求得百許人,藍
 縷枯瘠,詢其故,乃主糧吏率取其口食。帝怒,捕鞫得實,斷腕殉河上三日而後斬之,押運者杖配商州。雍熙四年,並水陸路發運為一司。主綱吏卒盜用官物,及用水土雜糅官米,故毀敗舟船致沉溺者,棄市,募告者厚賞之;山河、平河實因灘磧風水所敗,以收救分數差定其罪。端拱元年,罷京城水陸發運,以其事分隸排岸司及下卸司。先是,四河所運未有定制,太平興國六年,汴河歲運江、淮米三百萬石,菽一百萬石;黃河粟五十萬石,
 菽三十萬石;惠民河粟四十萬石,菽二十萬石;廣濟河粟十二萬石:凡五百五十萬石。非水旱蠲放民租,未嘗不及其數。至道初,汴河運米五百八十萬石。大中祥符初,至七百萬石。



 江南、淮南、兩浙、荊湖路租糴,於真、揚、楚、泗州置倉受納,分調舟船溯流入沛,以達京師,置發運使領之。諸州錢帛、雜物、軍器上供亦如之。陜西諸州菽粟,自黃河三門沿流入沛,以達京師,亦置發運司領之。粟帛自廣濟河而至京師者,京東之十七州;由石塘、惠
 民河而至京師者,陳、穎、許、蔡、光、壽六州,皆有京朝官廷臣督之。河北衛州東北有御河達乾寧軍,其運物亦廷臣主之。廣南金銀、香藥、犀象、百貨,陸運至虔州而後水運。川益諸州金帛及租、市之布,自劍門列傳置,分輦負擔至嘉州,水連達荊南,自荊南遣綱吏運送京師。咸平中,定歲運六十六萬匹,分為十綱。天禧末,水陸運上供金帛、緡錢二十三萬一千餘貫、兩、端、匹,珠寶、香藥二十七萬五千餘斤。諸州歲造運船,至道末三千二百三十
 七艘,天禧末減四百二十一。先是,諸河漕數歲久益增,景德四年,定汴河歲額六百萬石。天聖四年,荊湖、江、淮州縣和糴上供,小民闕食,自五年後權減五十萬石。慶歷中,又減廣濟河二十萬石。後黃河歲漕益減耗,才運菽三十萬石,歲創漕船,市材木,役牙前,勞費甚廣;嘉祐四年,罷所運菽,減漕船三百艘。自是歲漕三河而已。



 江、湖上供米,舊轉運使以本路綱輸真、楚、泗州轉般倉,載鹽以歸,舟還其郡,卒還其家。汴舟詣轉般倉運米輸京
 師,歲折運者四。河冬涸,舟卒亦還營,至春復集,名曰放凍。卒得番休,逃亡者少;汴船不涉江路,無風波沉溺之患。後發運使權益重,六路上供米團綱發船,不復委本路,獨專其任。文移坌並,事目繁伙,不能檢察。操舟者賕諸吏,得詣富饒郡市賤貿貴,以趨京師。自是江、汴之舟,混轉無辨,挽舟卒有終身不還其家、老死河路者。籍多空名,漕事大弊。



 皇祐中,發運使許元奏:「近歲諸路因循,糧綱法壞,遂令汴綱至冬出江,為他路轉漕,兵不得息。
 宜敕諸路增船,載米輸轉般倉充歲計如故事。」於是牟利者多以元說為然,詔如元奏。久之,諸路綱不集。嘉祐三年,下詔切責有司以格詔不行,及發運使不能總綱條,轉運使不能斡歲入。預敕江、淮、兩浙轉運司,期以期年,各造船補卒,團本路綱,自嘉祐五年汴船不得復出江。至期,諸路船猶不足。汴船既不至江外,江外船不得至京師,失商販之利;而汴船工卒訖冬坐食,恆苦不足,皆盜毀船材,易錢自給,船愈壞而漕額愈不及矣。論者
 初欲漕卒得歸息,而近歲汴船多傭丁夫,每船卒不過一二人,至冬當留守船,實無得歸息者。時元罷已久,後至者數奏請出汴船,執政不許。治平三年,始詔出汴船七十綱,未幾,皆出江復故。



 治平二年,漕粟至京師,汴河五百七十五萬五千石,惠民河二十六萬七千石,廣濟河七十四萬石。又漕金帛緡錢入左藏、內藏庫者,總其數一千一百七十三萬,而諸路轉移相給者不預焉。繇京西、陜西、河東運薪炭至京師,薪以斤計一千七百一
 十三萬,炭以秤計一百萬。是歲,諸路創漕船二千五百四十艘。治平四年,京師粳米支五歲餘。是時,漕運吏卒,上下共為侵盜貿易,甚則托風水沉沒以滅跡。官物陷折,歲不減二十萬斛。熙寧二年,薛向為江、淮等路發運使,始募客舟與官舟分運,互相檢察,舊弊乃去。歲漕常數既足,募商舟運至京師者又二十六萬餘石而未已,請充明年歲計之數。



 三司使吳充言:「宜自明年減江、淮漕米二百萬石,令發運司易輕貨二百萬緡,計五年所
 得,無慮緡錢千萬,轉儲三路平糴備邊。」王安石謂:「驟變米二百萬石,米必陡賤;驟致輕貨二百萬貫,貨必陡貴。當令發運司度米貴州郡,折錢變為輕貨,儲之河東、陜西要便州軍,用常平法糶糴為便。」詔如安石議。七年,京東路察訪鄧潤甫等言:「山東沿海州郡地廣,豐歲則穀賤,募人為海運,山東之粟可轉之河朔,以助軍食。」詔京東、河北路轉運司相度,卒不果行。是歲,江、淮上供穀至京師者三分不及一,令督發運使張頡亟辦來歲漕計。



 宣徽南院使張方平言:「今之京師,古所謂陳留,天下四沖八達之地,利漕運而贍師旅。國初,浚河渠三道以通漕運,立上供平額,汴河六百萬石,廣濟河六十二萬石,惠民河六十萬石。廣濟河所運,止給太康、咸平、尉氏等縣軍糧,唯汴河運米麥,乃太倉蓄積之實。近罷廣濟河,而惠民河觔斗不入太倉,大眾所賴者汴河。議者屢作改更,必致汴河日失其舊。」十二月,詔浚廣濟河,增置漕舟。其後河成,歲漕京東谷六十萬石。東南諸路上供雜
 物舊陸運者,增舟水運。押汴河江南、荊湖綱運,七分差三班使臣,三分軍大將、殿侍。又令真、楚、泗州各造淺底舟百艘,分為十綱入汴。



 元豐五年,罷廣濟河輦運司及京北排岸司,移上供物於淮陽計置入汴,以清河輦運司為名。御史言廣濟安流而上,與清河溯流入汴,遠近險易不同。詔轉運、提點刑獄比較利害以聞。江、淮等路發運副使蔣之奇、都水監丞陳祐甫開龜山運河,漕運往來,免風濤百年沉溺之患。詔各遷兩官,餘官減年循
 資有差。八年,罷歲運百萬石赴西京。先是,道洛入汴,運東南粟實洛下,至是,戶部奏罷之。是年,立汴河糧綱賞罰,歲終檢察。紹聖二年,置汴綱,通作二百綱。在部進納官銓試不中者,注押上供糧斛,不用衙前、土人、軍將。未幾,復募土人押諸路綱如故。



 政和七年,立東南六路州軍知州、通判裝發上供糧斛任滿賞格,自一萬石至四十萬石升名次減年有差。張根為江南西路轉運副使,歲漕米百二十萬石給中都。江南州郡僻遠,官吏艱於
 督趣,根常存三十萬石為轉運之本,以寬諸郡,時甚稱之。宣和二年,詔:「六路米麥綱運依法募官,先募未到部小使臣及非泛補授校尉以上未許參部人並進納人管押;淮南以五運,兩浙及江東二千里內以四運,江東二千里外及江西三運,湖南、北二運,各欠不及五厘,依格推賞外,仍許在外指射合入差遣一次。召募土人並罷。」七年,詔結絕應奉司江淮諸局、所及罷花石綱,令逐路漕臣速拘舟船裝發綱運備邊。靖康初,汴河決口有
 至百步者,塞之,工夫未訖,乾涸月餘,綱運不通,南京及京師皆乏糧。責都水使者陳求道等,命提舉京師所陳良弼同措置。越兩旬,水復舊,綱運沓至,兩京糧乃足。



 河北、河東、陜西三路租稅薄,不足以供兵費,屯田、營田歲入無幾,糴買入中之外,歲出內藏庫金帛及上京榷貨務緡錢,皆不翅數百萬。選使臣、軍大將,河北船運至乾寧軍,河東、陜西船運至河陽,措置陸運,或用鋪兵廂軍,或發義勇保甲,或差雇夫力,車載馱行,隨道路所宜。河
 北地裡差近,西路回遠,又涉磧險,運致甚艱。熙寧六年,詔鄜延路經略司支封樁錢於河東買橐駝三百,運沿邊糧草。



 元豐四年,河東轉運司調夫萬一千人隨軍,坊郭上戶有差夫四百人者,其次一二百人。願出驢者三驢當五夫,五驢別差一夫驅喝。一夫雇直約三十千以上,一驢約八千,加之期會迫趣,民力不能勝。軍須調發煩擾,又多不急之務,如絳州運棗千石往麟、府,每石止直四百,而雇直乃約費三十緡。涇原路轉運判官張大
 寧言:「饋運之策,莫若車便。自熙寧砦至磨□移口皆大川,通車無礙,自磨□移至兜嶺下道路亦然。嶺以北即山險少水,車乘難行。可就嶺南相地利建一城砦,使大車自鎮戎軍載糧草至彼,隨軍馬所在,以軍前夫畜往來短運。更於中路量度遠近,以遣回空夫築立小堡應接,如此則省民力之半。」神宗嘉之。京西轉運司調均、鄧州夫三萬,每五百人差一官部押,赴鄜延饋運。其本路程塗日支錢米外,轉運司計自入陜西界至延州程數,日支米
 錢三十、柴菜錢十文,並先並給。陜西都轉運司於諸州差雇車乘人夫,所過州交替,人日支米二升、錢五十,至沿邊止。運糧出界,止差廂軍。六年,詔熙河蘭會經略制置司,計置蘭州人萬馬二千般運糧草,於次路州軍鏟刮官私橐駝二千與經制司,自熙、河折運。事力不足,發義勇保甲。給河東、陜西邊用非機速者,並作小綱數排日遞送。



 大觀二年,京畿都轉運使吳擇仁言:「西輔軍糧,發運司歲撥八萬石貼助,於滎澤下卸,至州尚四、五十
 里,擺置車三鋪,每鋪七十人,月可運八千四百石。所運漸多,據數增添鋪兵。」靖康元年十月,詔曰:「一方用師,數路調發,軍功未成,民力先困。京西運糧,每名六斗,用錢四十貫;陜西運糧,民間倍費百餘萬緡,聞之駭異。今歲四方豐稔,粒米狼戾,但可逐處增價收糴,不得輕般運,以稱恤民之意。若般綱水運及諸州支移之類仍舊。」三路陸運以給兵費,大略如此,其它州縣運送或軍興調發以給一時之用,此皆不著。



 轉般,自熙寧以來,其法始
 變,歲運六百萬石給京師外,諸倉常有餘蓄。州郡告歉,則折收上價,謂之額斛。計本州歲額,以倉儲代輸京師,謂之代發。復於豐熟以中價收糴,穀賤則官糴,不至傷農;饑歉則納錢,民以為便。本錢歲增,兵食有餘。崇寧初,蔡京為相,始求羨財以供侈用,費所親胡師文為發運使,以糴本數百萬緡充貢,入為戶部侍郎。來者效尤,時有進獻,而本錢竭矣;本錢既竭,不能增糴,而儲積空矣;儲積既空,無可代發,而轉般之法壞矣。



 崇寧三年,戶部
 尚書曾孝廣言:「往年,南自真州江岸,北至楚州淮堤,以堰瀦水,不通重船,般剝勞費。遂於堰旁置轉般倉,受逐州所輸,更用運河船載之入汴,以達京師。雖免推舟過堰之勞,然侵盜之弊由此而起。天聖中,發運使方仲荀奏請度真、楚州堰為水閘,自是東南金帛、茶布之類直至京師,惟六路上供觔斗,猶循用轉般法,吏卒糜費與在路折閱,動以萬數。欲將六路上供觔斗,並依東南雜運直至京師或南京府界卸納,庶免侵盜乞貸之弊。」自
 是六路郡縣各認歲額,雖湖南、北至遠處,亦直抵京師,號直達綱,豐不加糴,歉不代發。方綱米之來,立法峻甚,船有損壞,所至修整,不得逾時。州縣欲其速過,但令供狀,以錢給之,沿流鄉保悉致騷擾,公私橫費百出。又鹽法已壞,回舟無所得,舟人逃散,船亦隨壞,本法盡廢。



 大觀三年,詔直達綱自來年並依舊法復令轉般,令發運司督修倉廒,荊湖北路提舉常平王□措置諸路運糧舟船。



 政和二年,復行直達綱,毀拆轉般諸倉。譚稹上言:「
 祖宗建立真、楚、泗州轉般倉,一以備中都緩急,二以防漕渠阻節,三則綱船裝發,資次運行,更無虛日。自其法廢,河道日益淺澀,遂致中都糧儲不繼,淮南三轉般倉不可不復。乞自泗州為始,次及真、楚,既有瓦木,順流而下,不甚勞費。俟歲豐計置儲蓄,立法轉般。」淮南路轉運判官向子諲奏:「轉般之法,寓平糴之意。江、湖有米,可糴於真;兩浙有米,可糴於揚;宿、亳有麥,可糴於泗。坐視六路豐歉,有不登處,則以錢折斛,發運司得以斡旋之,不
 獨無歲額不足之憂,因可以寬民力。運渠旱乾,則有汴口倉。今所患者,向來糴本歲五百萬緡,支移殆盡。」



 宣和五年,乃降度牒及香、鹽鈔各一百萬貫,令呂淙、盧宗原均糴觔斗,專備轉般。江西轉運判官蕭序辰言:「轉般道里不加遠,而人力不勞卸納,年豐可以廣糴厚積,以待中都之用。自行直達,道裏既遠,情弊尤多,如大江東西、荊湖南北有終歲不能行一運者,有押米萬石欠七八千石,有拋失舟船、兵梢逃散、十不存一二者。折欠之弊
 生於稽留,而沿路官司多端阻節,至有一路漕司不自置舟船,截留他路回綱,尤為不便。」詔發運司措置。六年,以無額上供錢物並六路舊欠發觔斗錢,貯為糴本,別降三百萬貫付盧宗原,將湖南所起年額,並隨正額預起拋欠觔斗於轉般倉下卸,卻將已卸均糴斗斛轉運上京,所有直達,候轉般觔斗有次第日罷之。靖康元年,令東南六路上供額斛,除淮南、兩浙依舊直達外,江、湖四路並措置轉般。



 高宗建炎元年,詔諸路綱米以三分
 之一輸送行在,餘輸京師。二年,詔二廣、湖南北、江東西綱運輸送平江府,京畿、淮南、京東西、河北、陜西及三綱輸送行在。又詔二廣、湖南北綱運如過兩浙,許輸送平江府;福建綱運過江東、西,亦許輸送江寧府。三年,又詔諸路綱運見錢並糧輸送建康府戶部,其金銀、絹帛並輸送行在。紹興初,因地之宜,以兩浙之粟供行在,以江東之粟餉淮東,以江西之粟餉淮西,荊湖之粟餉鄂、岳、荊南。量所用之數,責漕臣將輸,而歸其餘於行在,錢帛
 亦然。雇舟差夫,不勝其弊,民間有自毀其舟、自廢其田者。



 紹興四年,川、陜宣撫吳玠調兩川夫運米一十五萬斛至利州,率四十餘千致一斛,饑病相仍,道死者眾,蜀人病之。漕臣趙開聽民以粟輸內郡,募舟挽之,人以為便。總領所遣官就糴於沿流諸郡,復就興、利、閬州置場,聽商人入中。然猶慮民之勞且憊也,又減成都水運對糴米。紹興十六年。



 三十年,科撥諸路上供米:鄂兵歲用米四十五萬餘石,於全、永、郴、邵、道、衡、潭、鄂、鼎科撥;荊南
 兵歲用米九萬六千石,於德安、荊南、澧、純、潭、復、荊門、漢陽科撥;池州兵歲用米十四萬四千石,於吉、信、南安科撥;建康兵歲用米五十五萬石,於洪、江、池、宣、太平、臨江、興國、南康、廣德科撥;行在合用米一百十二萬石,就用兩浙米外,於建康、太平、宣科撥;其宣州見屯殿前司牧馬歲用米,並折輸馬料三萬石,於本州科撥;並諸路轉運司樁發。時內外諸軍歲費米三百萬斛,而四川不預焉。



 嘉定兵興,揚、楚間轉輸不絕,濠、廬、安豐舟楫之通亦
 便矣,而浮光之屯,仰饋於齊安、舒、蘄之民;遠者千里,近者亦數百里。至於京西之儲,襄、郢猶可徑達,獨棗陽陸運,夫皆調於湖北鼎、澧等處,道路遼邈,夫運不過八斗,而資糧屝屨與夫所在邀求,費常十倍。中產之家雇替一夫,為錢四五十千;單弱之人一夫受役,則一家離散,至有斃於道路者。



 至於部送綱運,並差見任官,闕則選募得替待闕及寄居官有材幹者,其責繁難,人以為憚。故自紹興以來優立賞格,其有欠者亦多方而憫之。乾
 道初,蠲欠五十石以下者;三年,蠲欠百石以下者。九年,初,綱運欠及一分者送有司究弊。至是,臣僚申明綱運欠及一分者亦許其補足。淳熙元年,詔:「不以所欠多寡,並無除放。其有因綱欠追降官資者,如本非侵盜,且補輸已足,許敘復。」自是綱運欠失雖責償於官吏,然以其山川逾遠,非一人所能究,亦時寓於蠲放焉。



\end{pinyinscope}