\article{志第五十一 禮一}

\begin{pinyinscope}

 吉
 禮一



 五代之衰亂甚矣,其禮文儀注往往多草創,不能備一代之典。宋太祖興兵間,受周禪,收攬權綱,一以法度振起故弊。即位之明年,因太常博士聶崇義上《重集三
 禮圖》,詔太子詹事尹拙集儒學之士詳定之。開寶中,四方漸平,民稍休息,乃命御史中丞劉溫叟、中書舍人李昉、兵部員外郎、知制誥盧多遜、左司員外郎、知制誥扈蒙、太子詹事楊昭儉、左補闕賈黃中、司勛員外郎和峴、太子中舍陳鄂撰《開寶通禮》二百卷,本唐《開元禮》而損益之。既又定《通禮義纂》一百卷。



 太宗尚儒雅,勤於治政,修明典章,大抵曠廢舉矣。真宗承重熙之後,契丹既通好,天下無事,於是封泰山,祀汾陰,天書、聖祖崇奉迭興,專
 置詳定所,命執政、翰林、禮官參領之。尋改禮儀院,仍歲增修,纖微委曲,緣情稱宜,蓋一時彌文之制也。



 自《通禮》之後,其制度儀注傳於有司者,殆數百篇。先是,天禧中,陳寬編次禮院所承新舊詔敕,不就。天聖初,王皞始類成書,盡乾興,為《禮閣新編》,大率禮文,無著述體,而本末完具,有司便之。景祐四年,賈昌朝撰《太常新禮》及《祀儀》,止於慶歷三年。皇祐中,文彥博又撰《大享明堂記》二十卷。至嘉祐中,歐陽修纂集散失,命官設局,主《通禮》而
 記其變,及《新禮》以類相從,為一百卷,賜名《太常因革禮》,異於舊者蓋十三四焉。



 熙寧十年,禮院取慶歷以後奉祀制度,別定《祀儀》,其一留中,其二付有司。知諫院黃履言:「郊祀禮樂,未合古制,請命有司考正群祀。」詔履與禮官講求以聞。元豐元年,始命太常寺置局,以樞密直學士陳襄等為詳定官,太常博士楊完等為檢討官。襄等言:「國朝大率皆循唐故,至於壇壝神位、法駕輿輦、仗衛儀物,亦兼用歷代之制。其間情文訛舛,多戾於古。蓋有
 規摹茍略,因仍既久,而重於改作者;有出於一時之儀,而不足以為法者。請先條奏,候訓敕以為禮式。」



 未幾,又命龍圖直學士宋敏求同御史臺、閣門、禮院詳定《朝會儀注》,總四十六卷:曰《閣門儀》,曰《朝會禮文》,曰《儀注》,曰《徽號寶冊儀》。《祭祀》總百九十一卷:曰《祀儀》,曰《南郊式》,曰《大禮式》,曰《郊廟奉祀禮文》,曰《明堂袷享令式》,曰《天興殿儀》,曰《四孟朝獻儀》,曰《景靈宮供奉敕令格式》,曰《儀禮敕令格式》。《祈禳》總四十卷:曰《祀賽式》,曰《齋醮式》,曰《金菉儀》。《蕃
 國》總七十一卷:曰《大遼令式》,曰《高麗入貢儀》,曰《女真排辦儀》,曰《諸蕃進貢令式》。《喪葬》總百六十三卷:曰《葬式》,曰《宗室外臣葬敕令格式》,曰《孝贈式》。其損益之制,視前多矣。



 紹聖而後,累詔續編,起治平,訖政和,凡五十一年,為書三百卷,今皆不傳。而大觀初,置議禮局於尚書省,命詳議、檢討官具禮制本末,議定請旨。三年書成,為《吉禮》二百三十一卷、《祭服制度》十六卷,頒焉。議禮局請分秩五禮,詔依《開寶通禮》之序。政和元年,續修成四百七十
 七卷,且命仿是修定儀注。三年,《五禮新儀》成,凡二百二十卷,增置禮直官,許士庶就問新儀,而詔開封尹王革編類通行者,刊本給天下,使悉知禮意,其不奉行者論罪。宣和初,有言其煩擾者,遂罷之。



 初,議禮局之置也,詔求天下古器,更制尊、爵、鼎、彞之屬。其後,又置禮制局於編類御筆所。於是郊廟禋祀之器,多更其舊。既有詔討論冠服,遂廢靴用履,其它無所改議,而禮制局亦罷。



 大抵累朝典禮,講議最詳。祀禮修於元豐,而成於元祐,至
 崇寧復有所增損。其存於有司者,惟《元豐郊廟禮文》及《政和五禮新儀》而已。乃若圜丘之罷合祭天地;明堂專以英宗配帝,悉罷從祀群神;大蠟分四郊;壽星改祀老人;禧祖已祧而復,遂為始祖;即景靈宮建諸神御殿,以四孟薦享;虛禘祭;去牙盤食;卻尊號;罷入閣儀並常朝及正衙橫行。此熙寧、元豐變禮之最大者也。



 元祐冊后,政和冠皇子,元符創景靈西宮,崇寧親祀方澤、作明堂、立九廟、鑄九鼎、祀熒惑,大觀受八寶、大祀皆前期十日
 而戒。凡此蓋治平以前所未嘗行者。



 欽宗即位,嘗詔春秋釋奠改從《元豐儀》,罷《新儀》不用而未暇也。靖康之厄,蕩析無餘。


南渡中興,銳意修復,高宗嘗謂輔臣曰:「晉武平吳之後,上下不知有禮,旋致禍亂。周禮不秉,其何能國?」孝宗繼志,典章文物,有可稱述。治平日久,經學大明,諸儒如王普、董
 \gezhu{
  分廾}
 等多以禮名家。當時嘗續編《太常因革禮》矣,淳熙復有編輯之旨。其後朱熹講明詳備,嘗欲取《儀禮》、《周官》、《二戴記》為本,編次朝廷公卿大夫士民之
 禮,盡取漢、晉而下及唐諸儒之說,考訂辨正,以為當代之典,未及成書而沒。



 理宗四十年間,屢有意乎禮文之事,雖曰崇尚理學,所謂「禮云禮云,玉帛云乎哉」,蓋可三嘆。咸淳以降,無足言者。



 今因前史之舊,芟其繁亂,匯為五禮,以備一代之制,使後之觀者有足徵焉。



 五禮之序,以吉禮為首,主邦國神祇祭祀之事。凡祀典皆領於太常。歲之大祀三十:正月上辛祈穀,孟夏雩祀,季秋大享明堂,冬至圜丘祭昊天上帝,正月上辛又祀
 感生帝,四立及土王日祀五方帝,春分朝日,秋分夕月,東西太一,臘日大蠟祭百神,夏至祭皇地祇,孟冬祭神州地祇,四孟、季冬薦享太廟、後廟,春秋二仲及臘日祭太社、太稷,二仲九宮貴神。中祀九:仲春祭五龍,立春後丑日祀風師、亥日享先農,季春巳日享先蠶,立夏後申日祀雨師,春秋二仲上丁釋奠文宣王、上戊釋奠武成王。小祀九:仲春祀馬祖,仲夏享先牧,仲秋祭馬社,仲冬祭馬步,季夏土王日祀中溜,立秋後辰日祀靈星,秋分
 享壽星,立冬後亥日祠司中、司命、司人、司祿,孟冬祭司寒。



 其諸州奉祀,則五郊迎氣日祭嶽、鎮、海、瀆,春秋二仲享先代帝王及周六廟,並如中祀。州縣祭社稷,奠文宣王,祀風雨,並如小祀。凡有大赦,則令諸州祭嶽、瀆、名山、大川在境內者,及歷代帝王、忠臣、烈士載祀典者,仍禁近祠廟咸加祭。有不克定時日者,太卜署預擇一季祠祭之日,謂之「畫日」。凡壇壝、牲器、玉帛、饌具、齋戒之制,皆具《通禮》。後復有高禖、大小酺神之屬,增大祀為四十二
 焉。



 其後,神宗詔改定大祀:太一,東以春,西以秋,中以夏冬;增大蠟為四,東西蠟主日配月;太廟月祭朔。而中祀:四望,南北蠟。小祀:以四立祭司命、戶、灶、中溜、門、厲、行,以藏冰、出冰祭司寒,及月薦新太廟。歲通舊祀凡九十二,惟五享后廟焉。政和中,定《五禮新儀》,以熒惑、陽德觀、帝鼐、坊州朝獻聖祖、應天府祀大火為大祀;雷神、歷代帝王、寶鼎、牡鼎、蒼鼎、岡鼎、彤鼎、阜鼎、皛鼎、魁鼎、會應廟、慶厲軍祭后土為中祀;山林川澤之屬,州縣祭社稷、祀風
 伯雨師雷神為小祀。餘悉如故。



 建炎四年十一月,權工部尚書韓肖冑言:「祖宗以來,每歲大、中、小祀百有餘所,罔敢廢闕。自車駕巡幸,惟存宗廟之祭,至天地諸神之祀,則廢而不舉。今國步尚艱,天未悔禍,正宜齋明恭肅,通於神明,而忽大事、棄重禮,恐非所以消弭天災,導迎景貺。雖小祀未可遍舉,如天地、五帝、日月星辰、社稷,欲詔有司以時舉行。所有器服並牲牢禮料,恐國用未充,難如舊制,乞下太常寺相度裁定,省繁就簡,庶幾神不乏
 祀,仰副陛下昭事懷柔、為民求福之意。」尋命禮部太常裁定:每歲以立春上辛祀穀,孟夏雩祀,季秋及冬至日四祀天,夏至日一祀地,立春上辛日祀感生帝,立冬後祀神州地祇,春秋二社及臘前一日祭太社、太稷。免牲、玉,權用酒酺,仍依方色奠幣。以輔臣為初獻,禮官為亞、終獻。



 紹興三年,復大火祀,配以閼伯,以辰、戌出納之月祀之。二十七年,禮部太常寺言:「每歲大祀三十六,除天地、宗廟、社稷、感生帝、九宮貴神、高禖、文宣王等已行外,
 其餘並乞寓祠齋宮。」自紹興以來,大祀所行二十有三而已,至是乃悉復之。



 舊制,郊廟祝文稱嗣皇帝,諸祭稱皇帝。著作局準《開元禮》全稱帝號。真宗以兼秘書監李至請,改從舊制。又諸祭祝辭皆臨事撰進,多違典禮,乃命至增撰舊辭八十四首,為《正辭錄》三卷。既復命知制誥李宗諤、楊億、直史館陳彭年詳定之,以為永式。祝版當進署者,並命秘閣吏書,上親署訖,御寶封給之。凡先代帝王,祝文止稱廟
 號。凡親行大祀,則皇子弟為亞獻、終獻。



 五代以來,宰相為大禮使,太常卿為禮儀使,御史中丞為儀仗使,兵部尚書為鹵簿使,京府尹為橋道頓遞使。至是,大禮使或用親王,禮儀使專命翰林學士,儀仗、鹵簿使亦或以他官。太平興國九年,始鑄五使印。太宗將封泰山,以儀仗使兼判橋道頓遞事。大中祥符後,凡有大禮,以中書、樞密分為五使,仍特鑄印。



 景祐二年,詔有司:「皇地祇、神州,舊常參官攝事,非所以尊神,自今命兩省。歲九大祠,宰
 臣攝事者,參知政事、尚書丞郎、學士奉祠。」於是參知政事盛度,享太廟已受誓戒,除知樞密院,乃不奉祠。又故事,三歲一親郊,不郊輒代以他禮,慶賞與郊同,而五使皆以輔臣,不以官之高下。天聖中,乃以朝林學士領儀仗,御史中丞領鹵簿,始用官次。又每歲大祀,皆遣臺省近臣攝太尉,其後或委他官,太中祥符始復舊制。又國朝沿唐制,以太尉掌誓戒。今議太尉三公,非其所任,請以吏部尚書掌誓戒。詔用左僕射,闕則用右僕射、刑部
 尚書一員蒞之。



 熙寧四年,參知政事王珪言:「南郊,乘輿所過,必勘箭然後出入,此師行之法,不可施於郊祀。」禮院亦言。於是,凡車駕出入門皆罷之。六年,以詳定所請,又罷太廟及宣德、朱雀、南熏諸門勘契。又皇帝自大次至版位,內臣二人執翟羽前導,號曰「拂翟」,失禮尤甚,請除之。



 凡郊壇,值雨雪,即齋宮門望祭殿望拜,祭日不設登歌,祀官以公服行事,中祀以上皆給明衣。



 開寶元年十一月郊,以燎壇稍遠,不聞告燎之聲,始用爟火,令光明遠照,通於祀所。



 又太廟初獻,依開寶例,以玉斝、玉瓚,亞獻以金斝,終獻以瓢斝。外壇器亦如之。慶歷中,太常請皇帝獻天地、配帝以匏爵,亞獻以木爵。親祠太廟,酌以玉斝,亞獻以金斝。郊廟飲福,皇帝皆以玉斝。詔飲福,唯用金斝。亞、終獻,酌以銀斝。至飲福,尚食奉御酌上尊酒,投溫器以進。



 凡常祀,天地宗廟,皆內降御封香,仍制漆匱,付光祿、司
 農寺。每祠祭,命判寺官緘署禮料送祀所。凡祈告,亦內出香。遂為定制。嘉祐中,裴煜請:「大祠悉降御封香,中、小祠供太府香。中祠減大祠之半,小祠減中祠之半。東、西太一宮系大祠,歲太府供香,非時祈請,降御封香準大祠例。及皇地祇、五方帝、百神、文宣、武成從配神位,牲牢寡薄。」呂公著亦論廟牲未備,悉加其數。元符元年,左司員外郎曾旼言:「周人以氣臭事神,近世易之以香。按何佟之議,以為南郊、明堂用沉香,本天之質,陽所宜也;北
 效用上和香,以地於人親,宜加雜馥。今令文北極天皇而下皆用濕香,至於眾星之位,香不復設,恐於義未盡。」於是每陛各設香。又言:「先儒以為實柴所祀者無玉,□□燎所祀者無幣。今太常令式,眾星皆不用幣,蓋出於此。然考《典瑞》、《玉人》之官,皆曰『圭璧以祀日月星辰』。則實柴所祀非無玉矣。□□燎無幣,恐或未然。」至是遂命眾星隨其方色用幣。



 慶歷三年,禮官餘靖言:「祈穀、祀感生帝同日,其禮當異,不可皆用四圭有邸,色尚赤。」乃定祈穀、明
 堂蒼璧尺二寸,感生帝四圭有邸,朝日日圭、夕月月圭皆五寸,從祀神州無玉,報社稷兩圭有邸,祈不用玉。明年,《祀儀》成,比《通禮》多所更定云。嘉祐中,集賢校理江休復言:「《六典》大祀養牲,在滌三月,袷享日近,已逾其期,而牲牢未供。乞依漢、唐置廩犧局。」下禮院議:歲大小祀幾百數,而牲盛之事,儲養無素,宜如休復言。乃置廩犧局,設牢預養,籍田舊地,種植粢盛,納於神倉,以待祭祀之用。



 元豐六年,詳定禮文所言:「本朝昊天上帝、皇地祇、太
 祖位各設三牲,非尚質貴誠之義。請親祠圜丘、方澤正配位皆用犢,不設羊豕俎及鼎匕,有司攝事亦如之。又簠、簋、尊、豆皆非陶器,及用龍杓。請改用陶,以樿為杓。又請南北郊先行升煙瘞血之禮,至薦奠畢,即如舊儀,於壇坎燔瘞牲幣。又北郊皇地祇及神州地祇,當為坎瘞,今乃建壇燔燎,非是。請今祭地祝版、牲幣並瘞於坎。又《祀儀》:惟昊天上帝、皇地祇、高禖燔瘞犢首,自感生帝、神州地祇而下皆不燔瘞牲體,殊不應典禮。請自今昊天
 上帝、感生帝皆燔牲首以報陽;皇地祇、神州、太社、太稷,凡地之祭,皆瘞牲之左髀以報陰。薦享太廟亦皆升首於室。」



 又言:「古者祭祀用牲,有豚解,有體解,薦腥則解為十一體。今親祠南郊,正配位之俎,不殊左右胖,不分貴賤,無豚解、體解之別。請郊廟薦腥,解其牲兩髀、兩肩、兩肋並脊為七體,左右胖俱用。其載於俎,以兩髀在端,兩肩、兩肋次之,脊居中,皆進末。至薦熟,沉肉於湯,止用右胖。髀不升俎,前後肱骨離為三,曰肩、臂、臑。後髀股骨去
 體離為二,曰肫、胳。前脊謂之正脊,次直謂之脡脊,闊於脡脊謂之橫脊,皆二骨。肋骨最後二為短肋,旁中二為正肋,最前二為代肋。若升俎,則肩、臂臑在上端,膊、胳在下端,脊、肋在中央。其俎之序,則肩、臂、臑、正脊、脡脊、橫脊、代肋、長肋、短肋、膊、胳凡十一體,而骨體升俎,進神坐前如少牢禮,皆進下。其牲體各預以半為腥俎,半為熟俎,腸胃膚俎亦然。」



 又請:「親祠飲福酒訖,仿《儀禮》『佐食摶黍』之說,命太官令取忝於簋,摶以授祝,祝受以豆,以嘏乎皇帝而無
 嘏辭。又本朝親祠南郊,習儀於壇所,明堂習儀於大慶殿,皆近於瀆。伏請南郊習儀於青城,明堂習儀於尚書省,以遠神為恭。又賜胙:三師,三公,侍中,中書令,門下、中書侍郎,尚書左、右丞,知樞密、同知院事,禮儀、儀仗、鹵簿、頓遞使,牛羊豕肩、臂、臑各五;太子三師、三少,特進,觀文大學士、學士,御史大夫,六尚書,金紫、銀青光祿大夫,節度使,資政殿大學士,觀文翰林資政端明龍圖天章寶文承旨、侍講、侍讀,學士,左右散騎常侍,尚書列曹侍郎,
 龍圖、天章、寶文直學士,光祿、正議、通議大夫,御史中丞,太子賓客、詹事,給事中,中書舍人,節度觀察留後,左右諫議,龍圖、天章、寶文待制,太中、中大夫,秘書、殿中丞,太常、宗正卿,牛豕肩、臂、臑各三;入內內侍省押班、副都知,光祿卿,監禮官,博士,牛羊脊、肋各三;太祝,奉禮,司奠彞,郊社、太廟、宮闈令,監牲牢、供應祠事內官,羊髀、膊、胳三;應執事、職掌、樂工、門干、宰手、馭馬、馭車人,並均給脾、肫、胳、觳及腸、胃、膚之類。」



 慶歷元年,判太常寺呂公綽言:「舊禮,郊廟尊罍數皆準古,而不實三酒、五齊、明水、明酒,有司相承,名為『看器』。郊廟配位惟用祠祭酒,分大、中祠位二升,小祠位一升,止一尊酌獻、一尊飲福。宜詔酒官依法制齊、酒,分實之壇殿上下尊罍,有司毋設空器;並如唐制以井水代明水、明酒;正配位酌獻、飲福酒,用酒二升者各增二升,從祀神位用舊升數。」



\end{pinyinscope}