\article{列傳第一百一十}

\begin{pinyinscope}

 趙挺之張商英兄唐英劉正夫何執中鄭居中張康國朱諤劉逵林攄管師仁侯蒙



 趙挺之,字正夫,密州諸城人。進士上第。熙寧建學,選教授登、棣二州,通判德州。哲宗即位,賜士卒緡錢,郡守貪耄不時給,卒怒噪,持白梃突入府。守趨避,左右盡走。挺之坐堂上,呼問狀,立發庫錢,而治其為首者,眾即定。魏境河屢決,議者欲徙宗城縣。轉運使檄挺之往視,挺之云:「縣距高原千歲矣,水未嘗犯。今所遷不如舊,必為民害。」使者卒徙之,財二年,河果壞新城,漂居民略盡。



 召試館職,為秘閣校理,遷監察御史。初,挺之在德州,希意行
 市易法。黃庭堅監德安鎮,謂鎮小民貧,不堪誅求。及召試,蘇軾曰:「挺之聚斂小人,學行無取,豈堪此選。」至是,劾奏軾草麻有云「民亦勞止」,以為誹謗先帝。既而坐不論蔡確,通判徐州,俄知楚州。



 入為國子司業,歷太常少卿,權吏部侍郎,除中書舍人、給事中。使遼,遼主嘗有疾,不親宴,使近臣即館享客。比歲享乃在客省,與諸國等,挺之始爭正其禮。



 徽宗立,為禮部侍郎。哲宗祔廟,議遷宣祖,挺之言:「上於哲宗兄弟,同一世;宣祖未當遷。」從之。拜
 御史中丞,為欽聖後陵儀仗使。曾布以使事聯職,知禁中密指,諭使建議紹述,於是挺之排擊元祐諸人不遺力。由吏部尚書拜右丞,進左丞、中書門下侍郎。時蔡京獨相,帝謀置右輔,京力薦挺之,遂拜尚書右僕射。



 既相,與京爭雄,屢陳其奸惡,且請去位避之。以觀文殿大學士、中太一宮使留京師。乞歸青州,將入辭,會彗星見,帝默思咎徵,盡除京諸蠹法,罷京,召見挺之曰:「京所為,一如卿言。」加挺之特進,仍為右僕射。京在崇寧初,首興邊
 事,用兵連年不息。帝臨朝,語大臣曰:「朝廷不可與四夷生隙,隙一開,禍拿不解,兵民肝腦塗地,豈人主愛民恤物意哉!」挺之退謂同列曰:「上志在息兵,吾曹所宜將順。」已而京復相,挺之仍以大學士使祐神觀。未幾卒,年六十八。贈司徒,謚曰清憲。



 張商英,字天覺,蜀州新津人。長身偉然,姿採如峙玉。負氣俶儻,豪視一世。調通川主簿。渝州蠻叛,說降其酋。闢知南川縣。章惇經制夔夷,狎侮郡縣吏,無敢與共語。部
 使者念獨商英足抗之,檄至夔。惇詢人才,使者以商英告,即呼入同食。商英著道士服,長揖就坐。惇肆意大言,商英隨機折之,落落出其上。惇大喜,延為上客。歸,薦諸王安石,因召對,以檢正中書禮房擢監察御史。



 臺獄失出劫盜,樞密檢詳官劉奉世駁之,詔糾察司劾治。商英奏:「此出大臣私忿,願收還主柄,使耳目之官無為近臣所脅。」神宗為置不治。商英遂言奉世庇博州失入囚,因摭院吏徇私十二事,語侵樞臣,於是文彥博等上印求
 去。詔責商英監荊南稅,更十年,乃得館閣校勘、檢正刑房。商英嘗薦舒但可用,至是,但知諫院,商英以婿王溈之所業示之,但繳奏,以為事涉干請,責監赤岸鹽稅。



 哲宗初,為開封府推官,屢詣執政求進。朝廷稍更新法之不便於民者,商英上書言:「『三年無改於父之道,可謂孝矣。』今先帝陵土未幹,即議變更,得為孝乎?」且移書蘇軾求入臺,其廋詞有「老僧欲住烏寺,呵佛罵祖」之語。呂公著聞之,不悅。出提點河東刑獄,連使河北、江西、淮南。



 哲
 宗親政,召為右正言、左司諫。商英積憾元祐大臣不用己,極力攻之,上疏曰:「先帝盛德大業,跨絕今古,而司馬光、呂公著、劉摯、呂大防援引朋儔,敢行譏議。凡詳定局之所建明,中書之所勘當,戶部之所行遣,百官之所論列,詞臣之所作命,無非指擿抉揚,鄙薄嗤笑,翦除陛下羽翼於內,擊逐股肱於外,天下之勢,岌岌殆矣。今天青日明,誅賞未正,願下禁省檢索前後章牘,付臣等看詳,簽揭以上,陛下與大臣斟酌而可否焉。」遂論內侍陳衍
 以搖宣仁,至比之呂、武;乞追奪光、公著贈謚,僕碑毀塚;言文彥博背負國恩,及蘇軾、範祖禹、孫升、韓川諸人,皆相繼受譴。又言:「願陛下無忘元祐時,章惇無忘汝州時,安燾無忘許昌時,李清臣、曾布無忘河陽時。」其觀望捭闔,以險語激怒當世,概類此。



 惇、燾交惡,商英欲助惇,求所以傾燾者。陽翟民蓋氏養子漸,先為祖母所逐,以家資屬其女,經元豐訴理不得直。商英論其冤,導漸使遮執政,及詣御史府訐燾姻家與蓋女為道地。哲宗不直
 商英,徙左司員外郎。既,與漸交關事皆露,責監江寧酒。起知洪州,為江、淮發運副使,入權工部侍郎,遷中書舍人。謝表歷詆元祐諸賢,眾益畏其口。徽宗出為河北都轉運使,降知隨州。



 崇寧初,為吏部、刑部侍郎,翰林學士。蔡京拜相,商英雅與之善,適當制,過為褒美。尋拜尚書右丞,轉左丞。復與京議政不合,數詆京「身為輔相,志在逢君。」御史以為非所宜言,且取商英所作《元祐嘉禾頌》及《司馬光祭文》,斥其反復。罷知亳州,入元祐黨籍。



 京罷
 相,削籍知鄂州。京復相,以散官安置歸、峽兩州。大觀四年,京再逐,起知杭州。過闕賜對,奏曰:「神宗修建法度,務以去大害、興大利,今誠一一舉行,則盡紹述之美。法若有弊,不可不變,但不失其意足矣。」留為資政殿學士、中太一宮使。頃之,除中書侍郎,遂拜尚書右僕射。京久盜國柄,中外怨疾,見商英能立同異,更稱為賢,徽宗因人望相之。時久旱,彗星中天,是夕,彗不見,明日,雨。徽宗喜,大書「商霖」二字賜之。



 商英為政持平,謂京雖明紹述,但
 藉以劫制人主,禁錮士大夫爾。於是大革弊事,改當十錢以平泉貨,復轉般倉以罷直達,行鈔法以通商旅,蠲橫斂以寬民力。勸徽宗節華侈,息土木,抑僥幸。帝頗嚴憚之,嘗葺升平樓,戒主者遇張丞相導騎至,必匿匠樓下,過則如初。楊戩除節度使,商英曰:「祖宗之法,內侍無至團練使。有勛勞當陟,則別立昭宣、宣政諸使以寵之,未聞建旄鉞也。」訖持不下,論者益稱之。



 然意廣才疏,凡所當為,先於公坐誦言,故不便者得預為計。何執中、鄭
 居中日夜釀織其短,先使言者論其門下客唐庚,竄之惠州。有郭天信者,以方技隸太史,徽宗潛邸時,嘗言當履天位,自是稍眷寵之。商英因僧德洪、客彭幾與語言往來,事覺,鞫於開封府。御史中丞張克公疏擊之,以觀文殿大學士知河南府,旋貶崇信軍節度副使,衡州安置。天信亦斥死。京遂復用。



 未幾,太學諸生誦商英之冤,京懼,乃乞令自便。繼復還故官職。宣和三年卒,年七十九。贈少保。



 商英作相,適承蔡京之後,小變其政,譬饑者
 易為食,故蒙忠直之名。靖康褒表司馬光、範仲淹,而商英亦贈太保。紹興中,又賜謚文忠,天下皆不謂然。兄唐英。



 唐英字次功。少攻苦讀書,至經歲不知肉味。及進士第,翰林學士孫抃得其《正議》五十篇,以為馬周、魏元忠不足多。薦試賢良方正,不就。調穀城令。縣圃歲畦姜,貸種與民,還其陳,復配賣取息,銓曹指為富縣。唐英至,空其圃,植千株柳,作柳亭其中,聞者咨羨。



 英宗繼大統,唐英
 上《謹始書》云:「為人後者為子,懼他日必有引漢定陶故事以惑宸聽者,願杜其漸。」既而濮議果起。帝不豫,皇太后垂簾,又上書請立穎王為皇太子。神宗即位,知其人,擢殿中侍御史。入對,帝問何尚衣綠,對曰:「前者固得之,回授臣父。」帝嘉其孝,賜五品服。



 帝方厲精圖治,急於用人,唐英言:「知江寧府王安石經術道德,宜在陛下左右。」又論宗室祿多費鉅,宜以服為差殺;天下苦差役不均,盍思所以寬民力、代民勞者。其後略施行。帝方欲用之,
 以父憂去,未幾卒。



 唐英有史材,嘗著《仁宗政要》、《宋名臣傳》、《蜀檮杌》,行於世。



 劉正夫,字德初,衢州西安人。未冠入太學,有聲,與範致虛、吳材、江嶼號「四俊」。元豐八年,南省奏名在優選,而犯高魯王諱,凡五人皆當黜。宣仁後曰:「外家俬諱頒未久,不可以妨寒士。」命置末級。久之,為太學錄、太常博士。母服闋,御史中丞石豫薦之,召赴闕,道除左司諫。



 時方究蔡邸獄,正夫入對,徽宗語及之,徐引淮南「尺布、斗粟」之
 謠以對。帝感動,解散其獄,待蔡王如初。他日,謂正夫曰:「兄弟之間,人所難言,卿獨能及此,後必為公輔。」又言:「元祐、紹聖所修《神宗史》,互有得失,當折中其說,傳信萬世。」遂詔刊定,而以起居舍人為編修官。不閱月,遷中書舍人,進給事中、禮部侍郎。



 蔡京據相位,正夫欲附翼之,奏言:「近命官纂錄紹述先志及施行政事,願得陳力其間。」詔俾閱詳焉。京罷,正夫又與鄭居中陰援京。京憾劉逵次骨,而逵善正夫,京雖賴其助,亦惡之。因章綖鑄錢獄
 辭及正夫,時使遼還,京諷有司追逮之。帝知其情,第貶兩秩。京又出之成都,入辭,留為翰林學士。京愈不能平,謀中以事。作春宴樂語,有「紫宸朝罷袞衣閑」之句,京黨張康國密白帝曰:「袞衣豈可閑?」竟改龍圖閣直學士、知河南府。



 召為工部尚書,拜右丞,進中書侍郎。太學諸生習樂成,京欲官之。正夫曰:「朝廷長育人材,規為時用,而使與伶官齒,策名以是,得無為士子羞乎?」東封儀物已具,正夫請間,力陳不可,帝皆為之止,益喜其不與京同。



 政和六年,擢拜特進、少宰。才半歲,屬疾,三上章告老,除安化軍節度使、開府儀同三司致仕。病小愈,丐東歸,詔肩輿至內殿,長子皂民掖入坐。從容及燕云事,曰:「臣起書生,軍旅之事未之學,然兩朝信誓之久,四海生靈之眾,願深留聖思。」明日,徙節安靜軍,起充中太一宮使,封康國公。將行,賜之詩及硯筆、圖畫、藥餌、香茶之屬甚厚。正夫獻詩謝,帝又屬和以榮其歸。至盱眙,病亟,命子弟作遺牘,自書「留神根本,深戒持盈」八字,遂卒,年五十六。
 贈太保,謚文憲,再贈太傅。



 正夫由博士入都,馴致宰相,能迎時上下,持祿養權。性吝嗇,惟恐不足於財。晚年,築第杭州萬松嶺,以建閣奉御書為名,悉取其旁軍營民舍,議者譏之。帝眷念不衰,以阜民為兵部侍郎;少子阜民,徽猷閣待制。



 何執中,字伯通,處州龍泉人。進士高第,調臺、亳二州判官。亳數易守,政不治。曾鞏至,頗欲振起之,顧諸僚無可仗信者,執中一見合意,事無纖鉅,悉委以剸決。有妖獄
 久不竟,株連浸寢多。執中訊諸囚,聽其相與語,謂牛羊之角皆曰:「股」,扣其故,閉不肯言,而相視色變。執中曰:「是必為師張角諱耳。」即扣頭引伏。蔣之奇使淮甸,號強明,官吏望風震懾,見執中喜曰:「一州六邑,賴有君爾。」知海鹽縣,為政識後先,邑人紀其十異。



 入為太學博士,以母憂去,寓蘇州。比鄰夜半火,執中方索居,遑遑不能去,拊柩號慟,誓與俱焚。觀者悲其孝而危其難,有頃火卻,柩得存。紹聖中,五王就傅,選為記室,轉侍講。端王即位,是為
 徽宗,超拜寶文閣待制,遷中書舍人、兵部侍郎、工部、吏部尚書兼侍讀。四選案籍,吏多藏於家,以舞文取賄。執中請置庫架閣,命官蒞之,是後六曹皆仿其法。



 蔡京籍上書人為邪等,初無朝覲及入都之禁,執中申言之,且請任在京職秩者皆罷遣。闢雍成,執中請開學殿,使都人士女縱觀,大為士論所貶。



 崇寧四年,拜尚書右丞。大觀初,進中書、門下侍郎,積官金紫光祿大夫。一意謹事京,三年,遂代為尚書左丞,加特進。制下,太學諸生陳
 朝老詣闕上書曰:「陛下知蔡京奸,解其相印,天下之人鼓舞,有若更生。及相執中,中外默然失望。執中雖不敢肆為非法若京之蠹國害民,然碌碌庸質,初無過人。天下敗壞至此,如人一身,臟府受沴已深,豈庸庸之醫所能起乎?執中夤緣攀附,致位二府,亦已大幸,遽俾之經體贊元,是猶以蚊負山,多見其不勝任也。」疏奏不省,而眷注益異。初,賜第信陵坊,以為淺隘,更徙金順坊甲第。建嘉會成功閣,帝親書鉅額以示寵。



 執中與蔡京並相,
 凡營立皆預議,略無所建明。及張商英任事,執中惡其出己上,與鄭居中合擠之。陳瓘在臺州,執中起遷人石悈知州事,使脅取《尊堯集》,謀必死瓘,瓘不死,執中怒罷悈。



 政和二年,大長公主喪,罷上元端門觀燈,執中言:「不宜以長主故閼眾情,願特為徙日,以昭與民同樂之意。」帝重逆其請,為申五日期。用提舉修《哲宗史紀》恩,加少保。入宴太清樓,錫白玉帶。會正宰相官名,轉少傅,為太宰;又遷少師,封榮國公。



 執中輔政一紀,年益高。五年,臥
 疾甚,賜寬告。他日造朝,命止赴六參起居,退治省事。明年,乃以太傅就第,許朝朔望,儀物廩稍,一切如居位時。入見,帝曰:「自相位致為臣,數十年無此矣。」對曰:「昔張士遜亦以舊學際遇,用太傅致仕,與臣適同。」帝曰:「當時恩禮,恐未必爾。」執中頓首謝。其在政府,嘗戒邊吏勿生事,重改作,惜人材,寬民力。雖居富貴,未嘗忘貧賤時。斥緡錢萬置義莊,以贍宗族。性復謹畏,至於迎順主意,贊飾太平,則始終一致,不能自克。



 卒,年七十四。帝即幸其家,
 以不及視其病為恨,輟視朝三日,贈太師,追封清源郡王,謚曰正獻。



 鄭居中,字達夫,開封人。登進士第。崇寧中,為都官禮部員外郎,起居舍人,至中書舍人、直學士院。初,居中自言為貴妃從兄弟,妃從蕃邸進,家世微,亦倚居中為重,由是連進擢。會妃父紳客祝安中者,上書涉謗訕,言者並及居中,罷知和州,徙穎州。明年,歸故官,遷給事中、翰林學士。大觀元年,同知樞密院。時妃寵冠後宮,於居中無
 所賴,乃用宦官黃經臣策,以外戚秉政辭。改資政學士、中太一宮使兼侍讀。



 蔡京以星文變免,趙挺之相,與劉逵謀盡改京所為政。未幾,徽宗頗悔更張之暴,外莫有知者。居中往來紳所,知之,即入見言:「陛下建學校、興禮樂,以藻飾太平;置居養、安濟院,以周拯窮困,何所逆天而致威譴乎?」帝大悟。居中退語禮部侍郎劉正夫,正夫繼請對,語同。帝意乃復向京。京再得政,兩人之助為多。



 居中厚責報,京為言樞密本兵之地,與三省殊,無嫌於
 用親。經臣方恃權,力抗前說,京言不效。居中疑不己援,始怨之,乃與張康國比而間京。都水使者趙霖得龜兩首於黃河,獻以為瑞。京曰:「此齊小白所謂『象罔』,見之而霸者也。」居中曰:「首豈宜有二?人皆駭異,而京獨主之,殆不可測。」帝命棄龜金明池,謂「居中愛我」,遂申前命,進知院事。四年,京又罷。居中自許必得相,而帝覺之,不用。妃正位中宮,復以嫌,罷為觀文殿學士。



 政和中,再知樞密院,官累特進。時京總治三省,益變亂法度。居中每為帝
 言,帝亦惡京專,尋拜居中少保、太宰,使伺察之。居中存紀綱,守格令,抑僥幸,振淹滯,士論翕然望治。丁母憂,旋詔起復。逾年,加少傅,得請終喪。服除,以威武軍節度使使祐神觀。還領樞密院,加少師。連封崇、宿、燕三國公。



 朝廷遣使與金約夾攻契丹,復燕云,蔡京、童貫主之。居中力陳不可,謂京曰:「公為大臣。國之元老,不能守兩國盟約,輒造事端,誠非妙算。」京曰:「上厭歲幣五十萬,故爾。」居中曰:「公獨不思漢世和戎用兵之費乎?使百萬生靈肝
 腦塗地,公實為之。」由是議稍寢。其後金人數攻,契丹日蹙,王黼、童貫復議舉兵,居中又言:「不宜幸災而動,待其自斃可也。」不聽。燕山平,進位太保,自陳無功,不拜。



 入朝,暴遇疾歸舍,數日卒,年六十五。贈太師、華原郡王,謚文正。帝親表其隧曰:「政和寅亮醇儒宰臣文正鄭居中之墓。」



 居中始仕,蔡京即薦其有廊廟器。既不合,遂因蔡渭理其父確功狀,追治王珪。居中,珪婿也,故借是撼之,然卒不能害。



 子修年、億年,皆至侍從。億年遭靖康之難,沒
 入於金。後遣事劉豫,晚得南歸,秦檜以婦氏親擢為資政殿大學士,位視執正。檜死,亦竄死撫州。



 時又有安堯臣者,亦嘗上書論燕雲之事,其言曰:



 宦寺專命,倡為大謀,燕雲之役興,則邊釁遂開;宦寺之權重,則皇綱不振。



 昔秦始皇築長城,漢武帝通西域,隋煬帝遼左之師,唐明皇幽薊之寇,其失如彼。周宣王伐玁狁,漢文帝備北邊,元帝納賈捐之之議,光武斥臧宮、馬武之謀,其得如此。藝祖撥亂反正,躬擐甲冑,當時將相大臣,皆所與取
 天下者,豈勇略智力,不能下幽燕哉?蓋以區區之地,契丹所必爭,忍使吾民重困鋒鏑!章聖澶淵之役,與之戰而勝,乃聽其和,亦欲固本而息民也。



 今童貫深結蔡京,同納趙良嗣以為謀主,故建平燕之議。臣恐異時唇亡齒寒,邊境有可乘之釁,狼子蓄銳,伺隙以逞其欲,此臣所以日夜寒心。伏望思祖宗積累之艱難,鑒歷代君臣之得失,杜塞邊隙,務守舊好,無使外夷乘間窺中國,上以安宗廟,下以慰生靈。



 徽宗然之,命堯臣以官;後竟為
 奸謀所奪。堯臣嘗舉進士不第,蓋惇之族子也。



 論曰:君子小人,猶冰炭不可一日而處者也。趙挺之為小官,薄有才具,熙寧新法之行,迎合用事,元祐更化,宜為諸賢鄙棄。至於紹聖,首倡紹述之謀,抵排正人,靡所不至。其論蔡京,不過為攘奪權寵之計而已,所謂「楚固為失,齊亦未為得」也。徽宗知京不可顓任,乃以張商英、鄭居中輩敢與京為異者參而用之。殊不知二人者,向背離合,視利所在,亦何有於公議哉?商英以傾詖之行,
 竊忠直之名,沒齒猶見褒稱,其欺世如此!何執中夤緣舊學,致位兩府,無所建明,惟務媢嫉,至用石悈脅陳瓘取《尊堯集》,欲因以殺瓘,何為者耶?宣、政命相,得若而人,尚望治乎?劉正夫生平所為,睒眒出沒正邪之間,商英之徒也。唐英有清才而寡失德,獨薦王安石為可咎;然安石未相,正人端士孰不與之,又何責乎唐英!



 張康國,字賓老,揚州人。第進士,知雍丘縣。紹聖中,戶部尚書蔡京整治役法,薦以參詳利害,使提舉兩浙常平
 推行之,豪猾望風斂服。發倉救荒,江南就食者活數萬口。徙福建轉運判官。崇寧元年,入為吏部、左司員外郎,起居郎。二年,為中書舍人。徽宗知其能詞章,不試而命。遷翰林學士。三年,進承旨,拜尚書左丞,而以其兄康伯代為學士。尋知樞密院事。康國自外官為郎,不三歲至此。



 始因蔡京進,京定元祐黨籍,看詳講議司,編匯章牘,皆預密議,故汲汲引援之,帝亦器重焉。及得志,浸為崖異。帝惡京專愎,陰令沮其奸,嘗許以相。是時,西北邊帥
 多取部內好官自闢置,以力不以才。康國曰:「並塞當擇人以紓憂顧,奈何欲私所善乎?」乃隨闕選用,定為格。



 京使御史中丞吳執中擊康國,康國先知之。旦奏事,留白帝曰:「執中今日入對,必為京論臣,臣願闢位。」既而執中對,果陳其事,帝叱去之。他日,康國因朝退,趨殿廬,暴得疾,仰天吐舌,舁至待漏院卒,或疑中毒云。年五十四。贈開府儀同三司,謚曰文簡。康伯,仕終吏部尚書。



 朱諤,字聖與,秀州華亭人,初名紱。進士第二,調忠正軍
 推官。崇寧初,由太常丞擢殿中侍御史,遷侍御史、給事中。以同黨籍人姓名,故改名。進御史中丞,入謝,徽宗曰:「今朝廷肅清,上下無事,宜審重以稱朕意。」對曰:「前此中執法類不知職守,言事多妄,至過天津橋,見汴堤一角墊陷,乞修葺。如許細故,何足論哉?」帝曰:「然。比石豫、許敦仁妄發,皆如是。」諤遂奏:「願如神宗故事,聽政之餘,開內閣,延群臣,從容論道。」



 又言:「陛下手詔屢下,惻怛願治。然吏奉行者多安於茍簡,或懷二三,柅置不行,使德音善
 教,無由下達。願分命使者刺舉諸道,有受令而不行及行令而不盡者,論如古留令、虧令之罪,則令出而朝廷尊矣。元祐紛更,凡得罪於熙寧、元豐者,不問是否,輒陳冤訴,自歸無過之地,彰先朝之失刑,希合奸臣,規求進用。門下侍郎許將頃下御史獄,抗章云:『絲毫自知其無事,父子相系而為囚,追屬吏十有六人,系病者百有三日,終無可坐之罪,遂加不實之刑。』夫以追屬吏如是之多,系病者如是之久,卒之於無可坐,則先帝所用之刑
 為何哉?將於哲廟表,泛為平詞;至宣仁太后之前,則銜冤負痛。其辭如此,於陛下紹述成功,得無少損乎?」詔出將河南。



 六察官彈治稽違,近歲察事多者輒推賞,有僥求之敝。諤乞罷賞,使各安職分,從之。俄兼侍讀,徙兵、禮、吏三部尚書。大觀元年,拜右丞。居三月卒,年四十。贈光祿大夫,謚忠靖。



 諤出蔡京門,善附合,不能有所建白。既死,京為志其墓。



 劉逵,字公路,隨州隨縣人。進士高第,調越州觀察判官。
 入為太學、太常博士,禮部、考功員外郎,國子司業。崇寧中,連擢秘書少監、太常少卿、中書舍人、給事中、戶部侍郎,使高麗,遷尚書。繇兵部同知樞密院,拜中書侍郎。



 逵無他才能,初以附蔡京故躐進。京以彗星見去相,而逵貳中書,首勸徽宗碎《元祐黨碑》,寬上書邪籍之禁;凡京所行悖理虐民事,稍稍澄正。逵與趙挺之同心;然挺之多智,慮後患,每建白,務開其端,而使逵終其說。逵欲自以為功,直情不顧。未滿歲,帝疑逵擅政,而鄭居中、劉正
 夫之策售矣。



 帝意既移,於是御史餘深、石公弼論逵專恣反復,乘間抵巘,盡廢紹述良法;愚視丞相,陵蔑同列;凡所啟用,多取為邪黨學術者及邪籍中子弟;庇其婦兄章綖,使之盜鑄。罷知亳州。



 京復相,再責鎮江節度副使,安州居住。京再以星變去,稍起知杭州,加資政殿學士。以醴泉觀使召,及都而卒,年五十。贈光祿大夫。



 林攄,字彥振,福州人,徙蘇。父邵,顯謨閣直學士。攄用蔭至敕令檢討官。蔡京講明熙寧、元豐故事,引以為屬,遷
 屯田、右司員外郎。



 時遣朝士察諸道,攄使河北。入辭,言大府宜擇帥,邊州宜擇守,西山木不宜採伐,保甲有藝者宜貢諸朝,驕兵宜使更戍,錢貨、文書闌出疆外者宜遏絕。徽宗喜曰:「卿所陳,已盡河朔利害,毋庸行。」賜進士第,擢起居舍人,進中書舍人。俄直學士院,禁林官不乏,帝特命,遂為翰林學士。



 初,朝廷數取西夏地,夏求援於遼,遼為請命。攄報聘,京密使激怒之以啟釁。入境,盛氣以待迓者,小不如儀,輒辨詰。及見遼主,始跪授書,即抗
 言數夏人之罪,謂北朝不能加責而反為之請。禮出不意,遼之君臣不知所答。及辭,遼使攄附奏,求還進築夏人城柵。攄答語復不巽,遼人大怒,空客館水漿,絕煙火,至舍外積潦亦污以矢溲,使饑渴無所得。如是三日,乃遣還,凡饔餼、祖犒皆廢。歸復命,議者以為怒鄰生事,猶除禮部尚書。既而遼人以失禮言,出知穎州。



 尋召為開封尹。大駔負賈錢久不償,一日,盡輦當十錢來,賈疑不納,駔訟之。攄馳詣蔡京,問曰:「錢法變乎?」京色動曰:「方議
 之,未決也。」攄曰:「令未布而賈人先知,必有與為表里者。」退鞫之,得省吏主名,置於法。



 張懷素妖事覺,攄與御史中丞餘深及內侍雜治,得民士交關書疏數百,攄請悉焚蕩,以安反側,眾稱為長者,而京與懷素游最密,攄實為京地也。京深德之,用鞫獄明允,加秩二等。改兵部尚書,進同知樞密院、尚書左丞、中書侍郎。自大觀元年春至二年五月,繇朝散大夫九遷至右光祿大夫。



 集英臚唱貢士,攄當傳姓名。不識「甄盎」字,帝笑曰:「卿誤邪?」攄不
 謝,而語詆同列。御史論其寡學,倨傲不恭,失人臣禮,黜知滁州。言者不厭,罷,提舉洞霄宮。起為越州、永興軍,皆以親年高辭。拜端明殿學士,久之,知揚州,政以察察聞,鋤大俠,繩污吏,下不敢欺。有行商寓逆旅,晨出不反,館人以告,攄曰:「此當不遠,或利其貨殺之耳。」指蹤物色,得尸溝中,果城民張所為也。



 徙大名府。道過闕,為帝言:「頃使遼,見其國中攜貳,若兼而有之,勢無不可。」攄蓋以曩辱,故修怨焉。其後北伐,蓋兆於此。加觀文殿學士,拜慶
 遠軍節度使。言者復論罷之。還姑蘇,瘍生於首而卒,年五十九。帝念其奉使之勤,申贈開府儀同三司,錄子偉直秘閣,數月偉死,嗣遂絕。靖康元年,以京死黨,追貶節度副使。



 管師仁,字符善,處州龍泉人。中進士第,為廣親、睦親宅教授。通判澧州,知建昌軍,有善政。擢右正言、左司諫。論蘇軾、蘇轍深毀熙寧之政,其門下士吏部員外郎晁補之輩不宜在朝廷,逐去之。河北濱、棣諸州歲被水患,民
 流未復,租賦故在,師仁請悉蠲減,以綏彳來之,一方賴其賜。遷起居郎、中書舍人、給事中、工部侍郎。選曹吏多撓法為過,師仁暫攝領,發其奸,抵數人於罪,士論稱之。改吏部,進刑部尚書,以樞密直學士知鄧州,未行,改揚州,又徙定州。



 時承平百餘年,邊備不整,而遼橫使再至,為西人請侵疆。朝廷詔師仁設備,至則下令增陴浚湟,繕葺甲冑。僚吏懼,不知所裁。師仁預為計度,一日而舉眾十萬,轉盼迄成,外間無知者。於是日與賓客燕集,以示
 閑暇,使敵不疑。帝手書詔獎激。召為吏部尚書,俄同知樞密院。才兩月,病。拜資政殿學士、祐神觀使。卒,年六十五。贈正奉大夫。



 侯蒙,字符功,密州高密人。未冠,有俊聲,急義好施,或一日揮千金。進士及第,調寶雞尉,知柏鄉縣。民訟皆決於庭,受罰者不怨。轉運使黃湜聞其名,將推轂之,召詣行臺白事,蒙以越境不肯往。湜怒,他日行縣,閱理文書,欲翻致其罪;既而無一疵可指,始以賓禮見,曰:「君真能吏
 也。」率諸使者合薦之。徙知襄邑縣,擢監察御史,進殿中侍御史。



 崇寧星變求言,蒙疏十事,曰去冗官,容諫臣,明嫡庶,別賢否,絕幸冀,戒濫恩,寬疲民,節妄費,戚里毋預事,閹寺毋假權。徽宗聽納,有大用意。遷侍御史。



 西將高永年死於羌,帝怒,親書五路將帥劉仲武等十八人姓名,敕蒙往秦州逮治。既行,拜給事中。至秦,仲武等囚服聽命,蒙曉之曰:「君輩皆侯伯,無庸以獄吏辱君,第以實對。」案未上,又拜御史中丞。蒙奏言:「漢武帝殺王恢,不如
 秦繆公赦孟明;子玉縊而晉侯喜,孔明亡而蜀國輕。今羌殺吾一都護,而使十八將繇之而死,是自艾其支體也。欲身不病,得乎?」帝悟,釋不問。



 遷刑部尚書,改戶部。比歲郊祭先期告辦,尚書輒執政。至是,帝密諭之。對曰:「以財利要君而進,非臣所敢。」母喪,服除,歸故官,遂同知樞密院。進尚書左丞、中書侍郎。先是,御史中丞蔡薿詆張商英私事甚力,有旨令廷辨。蒙曰:「商英雖有罪,宰相也;蔡薿雖言官,從臣也。使之廷辨,豈不傷國體乎?」帝以為
 然。一日,帝從容問:「蔡京何如人?」對曰:「使京能正其心術,雖古賢相何以加。」帝頷首,且使密伺京所為。京聞而銜之。



 大錢法敝,朝廷議改十為三,主藏吏來告曰:「諸府悉輦大錢市物於肆,皆疑法當變。」蒙曰:「吾府之積若干?」曰:「八千緡。」蒙叱曰:「安有更革而吾不知!」明日,制下。又嘗有幾事蒙獨受旨,京不知也;京偵得之,白於帝,帝曰:「侯蒙亦如是邪?」罷知亳州。旋加資政殿學士。



 宋江寇京東,蒙上書言:「江以三十六人橫行齊、魏,官軍數萬無敢抗者,
 其才必過人。今青溪盜起,不若赦江,使討方臘以自贖。」帝曰:「蒙居外不忘君,忠臣也。」命知東平府,未赴而卒,年六十八。贈開府儀同三司,謚文穆。



 論曰:崇寧、宣和之間,政在蔡京,罷不旋踵輒起,奸黨日蕃。一時貪得患失之小人,度徽宗終不能去之,莫不趨走其門。若張康國、朱諤、劉逵、林攄者,皆是也。康國、逵中雖異京,然其材智皆非京敵,卒為京黨所擊。攄奉京奸謀,激怒鄰國,渝約啟釁,罪莫大焉。《易》曰:「開國承家,小人
 勿用。」其謂是歟!管師仁執政僅兩月,引疾求去,斯可尚己。侯蒙逮治五路將帥,力為申理,十八人者繇之而免,其仁人利溥之言乎?



\end{pinyinscope}