\article{列傳第六十八}

\begin{pinyinscope}

 王延德常延信程德玄王延德魏震張質楊允恭秦羲謝德權閻日新靳懷德



 王延德,開封東明人。曾祖芝,濮陽令。祖璋,相州錄事參
 軍。父溫。晉末契丹內寇,溫率鄉豪捍蔽境內,里人德之。宣祖掌畿甸兵,與溫厚善,延德方總角,宣祖愛其謹重,召置左右。太宗尹京,署為親校,專主庖膳,尤被倚信。



 太平興國初,授御廚副使,數月,遷正使。從征太原,未幾,加尚食使,賜浚儀縣壽昌坊宅一區。俄領薊州刺史,兼掌武德司,改皇城使,掌御輦院、左藏庫。延德所領凡五印,因對懇讓,遂罷左藏、御廚。八年,兼充親王諸宮使。延德素謹慎,以舊恩,每延訪外事。端拱初,領本州團練使。淳
 化中,當進秩,延德與王繼恩、杜彥鈞使額已極,特置昭宣使,以延德等為之。至道二年,加領平州防禦使。



 真宗嗣位,改領懷州。永熙復土,提點緣路供頓。咸平初,出知華州,占謝日,面請罷昭宣使,從之。實以禦侮正秩,奉給優厚故也。上幸大名,為東京舊城都巡檢使。明年,以風痺請告,遣還本郡,是冬卒,年六十四。贈邕州觀察使。



 延德所至,好撰集近事。掌御廚則為《司膳錄》,掌皇城司則為《皇城紀事錄》,從郊祀為行宮使則為《南郊錄》,奉詔修
 內則為《版築記》,從靈駕則為《永熙皇堂錄》、《山陵提轄諸司記》,及治郡則為《下車奏報錄》。先是,詔史官修太祖、太宗《實錄》,多以國初事訪延德,又上《太宗南宮事跡》三卷。子應昌,莊宅使、端州團練使。



 常延信,並州平晉人。祖思,仕周歷昭義、歸德、平盧三鎮節度,延信皆補牙職,領和州刺史。思卒,入為六宅使,領郡如故。



 建隆初,改領平州,坐與妻族相訟,左授右監門衛副率,領護滑州黃河堤。開寶中,為京新城外汴河南
 巡檢,出為潼關監軍。延信以關路巖險,奏易道路及填禁坑,役工四十餘萬。又監通許鎮兵,改梓、遂十二州都巡檢使,賜袍帶、錢百萬。太平興國初,秩滿,留再任,賜錢四十萬。時亡命卒多以山林為寇,延信嘗領徒捕殺三百餘人。又為唐、鄧都巡檢使,代還,繼改右清道、右司御二副率。



 雍熙三年,命督鎮州以北至軍前芻糧。是冬,為全、邵六州都巡檢使,令疾置之任。就充羊狀六砦都鈐轄,遷右衛副率。會誠州蠻歸款,命延信馳入溪洞,索其
 要領。又逐蠻直趣古鎮,過西延、大木諸洞,蠻人懾伏。



 淳化中,歷襄、鄧、宋、曹等州都巡檢使,改左監門衛將軍,屢部徒修護河防,改左領軍、左屯衛二將軍,充西京水南都巡檢使。有盜掠彭婆鎮及甲馬營,延信馳以往,悉擒之。咸平中,歷太康、鞏縣二監軍。景德二年,卒,年六十四。



 程德玄,字禹錫,鄭州滎澤人。善醫術。太宗尹京邑,召置左右,署押衙,頗親信用事。太祖大漸之夕,德玄宿信陵坊,夜有扣關疾呼趣赴宮邸者。德玄遽起,不暇盥櫛,詣
 府,府門尚關。方三鼓,德玄不自悟,盤桓久之。俄頃,見內侍王繼恩馳至,稱遺詔迎太宗即位。德玄因從以入,拜翰林使。



 太平興國二年,陳洪進來朝,命德玄迎勞之。船艦度淮,暴風起,眾恐,皆請勿進。德玄曰:「吾將君命,豈避險?」以酒祝而行,風浪遽止。三年,遷東上閣門使,兼翰林司事。是秋,領代州刺史。從征太原,為行宮使,師還,以功改判四方館事。俄遷領本州團練使,又加領本州防禦使。



 五年,坐市秦、隴竹木聯筏入京師,所過矯制免算,又
 高其估以入官,為王仁贍所發,責授東上閣門使,領本州刺史。陜府西南轉運使、左拾遺韋務升,京西轉運使、起居舍人程能,判官、右贊善大夫時載,坐縱德玄等於部下私販鬻,務升洎能並責授右贊善大夫,載將作監丞。是冬,車駕幸魏府,命總御營四面巡檢,掌給諸軍資糧。



 德玄攀附至近列,上頗信其言,繇是趨附者甚眾。或言其交游太盛,遂出為崇信軍節度行軍司馬。逾年,復拜慈州刺史,移知環州。時西鄙酋豪相繼內附,詔以空
 名告敕百道付德玄,得便宜補授。頃之,以疾求致仕,優詔不許。淳化三年,改本州團練使、知邠州。未半歲,復典環州。李順之寇西蜀,移知鳳州,兼領鳳、成、階、文等州駐泊兵馬事,徙慶州。咸平中,入朝,真宗命坐撫勞,訪以邊事。俄出知並州兼並代副都部署,移鎮州,受代歸闕。景德初,卒,年六十五。大中祥符中,其子繼宗上章,懇祈贈典,上憫之,特贈鄭州防禦使。



 兄德元同仕王府,至內酒坊副使。繼宗,東頭供奉官、閣門祗候,次子繼忠,內殿崇
 班。德元子賁,大中祥符五年舉進士,累遷太常博士。



 王延德,大名人。少給事晉邸。太平興國初,補殿前承旨,再遷供奉官。六年,會高昌國遣使朝貢,太宗以遠人輸誠,遣延德與殿前承旨白勛使焉。自夏州渡河,經沙磧,歷伊州,望北庭萬五千里。雍熙二年,使還,撰《西州程記》以獻,授崇儀副使,掌御廚。明年,拜正使,出知慶州。



 淳化三年,代還,監折博倉。延德與張齊賢善,因國子博士朱貽業通言齊賢,求免掌庾,希進用。齊賢為言之,上怒曰:「
 延德願掌倉以自效,未逾月,又禱宰相求免,何也?」因召延德詰責,自言未嘗遣貽業詣相府有所求請。上疑齊賢不實,召貽業至,貽業又諱之,齊賢恥自辨,因頓首稱罪。上怒,即以延德領懿州刺史以寵之。五年,提點三司衙司、磨勘憑由司。未幾,拜左屯衛大將軍、樞密都承旨,俄授度支使。



 真宗即位,轉左千牛衛上將軍,充使如故。延德前使西域,冒寒不汗,得風痺疾,艱於步履。咸平初,出為舒州團練使、知鄆州,徙青州,坐市物有剩利,降授
 左武衛將軍。久病落籍,遣家人代詣登聞鼓院求休致,上以其久事先帝,復授左千牛衛上將軍致仕。景德三年,卒,年六十八。



 延德以攀附得官,傾險好進,時人惡之。兄延之,乾德六年進士,至屯田郎中致仕。



 魏震,不知何許人。祖浩,贍國軍榷鹽制置使。父鉞,蒲臺令。震初用祖蔭,當補廷職,自以習詞業,不屑就。姚恕嘗與鉞蒲臺交代,及為皇子教授,太宗在藩邸,恕嘗稱震之材,因召寘邸中。即位,補殿直、廬壽八州巡檢。從征河
 東,掌行在左藏庫,改供奉官。雍熙初,溫州進瑞木成文,震作詩賦以獻,拜崇儀副使,賜白金二千兩,掌內弓箭庫。出知保州,會諸將北伐,為幽州西北路鈐轄。下飛狐、蔚州,以功就遷崇儀使、知蔚州。復知保州,移秦州鈐轄。端拱中,召拜西上閣門使,俄知廬州,徙澶州。淳化二年,進東上閣門使、知鳳州,坐事免。至道初,起為洛苑使、知洪州。二年,復為東上閣門使,知定、代二州並兼行營鈐轄。咸平元年卒。子致恭,殿中丞。



 張質,字守樸,博州高唐人。少孤,養於兄贊。贊為樞密院典謁,質因得隸兵房,頗為趙普、曹彬所知。太宗征河東,還駐鎮陽,彬方典樞務。一夕,議調發屯兵,時,軍載簿領,阻留在道。質潛計兵數,部分軍馬,及得兵籍較之,悉無差謬。淳化中,累遷本房副都承旨。



 咸平初,授左監門衛將軍、樞密副都承旨。先是,樞密吏皆以年勞次補,有至主事而懵其職者。景德三年夏,內出公事三條,令主事以下詳決之,命質與禮房副承旨尹德潤宿禦書院考
 第。翌日,上親臨閱視,凡選補四十餘人,不中式除崇班、供奉官、奉職者十餘人。以質為左屯衛大將軍,加給月奉,歷右神武軍、右衛二大將軍。



 大中祥符七年,轉都承旨。在樞要僅五十日,練習事程,精敏端愨,未嘗有過。舊,本院吏罕有遷至都承旨者,上素知質廉謹,故以授之。嘗召問五代以降洎國初軍籍更易之制,且命條具利害,質纂為三篇,目曰《兵要》以進,上覽而稱善。



 好養生之術,老而不衰,以是多接隱人方士,然語不及公家事。每
 大祀巡幸,質多為行宮使,或領巡檢提點供頓之務。天禧元年九月,方候對承明殿,暴中風眩,輿歸卒,年七十四。錄其子大理評事純為衛尉寺丞,孫思道為三班奉職。



 楊允恭,漢州綿竹人。家世豪富,允恭少倜儻任俠。乾德中,王師平蜀,群盜竊發,允恭裁弱冠,率鄉里子弟砦於清泉鄉,為賊所獲,將殺之。允恭曰:「茍活我,當助爾。」賊素聞其豪宗,乃釋之。陰結賊帥子,日與飲博,陽不勝,償以
 貲,使伺賊。賊將害允恭,其子以告,因遁去。內客省使丁德裕討賊至州,允恭以策干之,署綿、漢招收巡檢,賊平,補殿前承旨。



 太平興國中,以殿直掌廣州市舶。自南漢之後,海賊子孫相襲,大者及數百人,州縣苦之。允恭因部運入奏其事,太宗即命為廣、連都巡檢使。又以海鹽盜入嶺北,民犯者眾,請建大庾縣為軍,官輦鹽市之。詔建為南安軍,自是冒禁者少。賊有葉氏者,眾五百餘,往來海上。允恭集水軍,造輕舠,掩襲其首,斬之。餘黨棄船走,
 伏匿山谷,允恭伐木開道,悉殲焉。賊寇每遇風濤,則遁止洲島間。允恭領眾涉海,捕之殆盡,賊皆望風奔潰。又抵漳、泉賊所止處,盡奪先所劫男女六十餘口還其家。詔書嘉獎,賜錢十萬,轉供奉官。詔歸,改內殿崇班。



 時緣江多賊,命督江南水運,因捕寇黨。行及臨江軍,擇驍卒拏輕舟伺下江賊所止,夜發軍城,三鼓,遇賊百餘,拒敵久之,悉梟其首。又趣通州境上躡海賊,賊系眾舟。張幕,發勁弩、短炮。允恭兵刃所向,多為幕所縈,炮中允恭左
 肩,流血及袖,容色彌壯。徐遣善泅者以繩連鐵鉤散擲之,壞其幕,士卒爭進,賊赴水死者大半,擒數百人。自是江路無剽掠之患。以功轉洛苑副使,江、淮、兩浙都大發運、擘畫茶鹽捕賊事;賜紫袍、金帶、錢五十萬。先是,三路轉運使各領其職,或廩庾多積,而軍士舟楫不給,雖以官錢雇丁男挽舟,而土人憚其役,以是歲上供米,不過三百萬。允恭盡籍三路舟卒與所運物數,令諸州擇牙吏,悉集,允恭乃辨數授之。江、浙所運,止於淮、泗,由淮、泗
 輸京師,行之一歲,上供者六百萬。



 淳化五年,轉西京作坊使。初,產茶之地,民輸賦者悉計其直,官售之,精粗不校,咸輸榷務。商人弗肯售,久即焚之。允恭曰:「竭民利而取之,積腐而棄之,非善計也。」至道初,劉式建議請廢緣江榷務,許商人過江,聽私貨鬻。允恭以為諸州新陳相糅,兩河諸州風土,各有所宜,非雜以數品,即商人少利。請依舊江北置務,均色號,以年次給之。事下三司,鹽鐵使陳恕等以允恭議為是,詔從之。即命允恭為發運使,
 始改「擘畫」為「制置」,以西京作坊副使李廷遂、著作佐郎王子輿並為同發運使。



 巢、廬江二縣舊隸廬州,道遠多寇,民輸勞費。允恭請以二縣建軍,詔許之,以無為為額。淮南十八州軍,其九禁鹽地,則上下其直,民利商鹽之賤,故販者益眾,至有持兵器往來為盜者。允恭以為行法宜一,即奏請悉禁,而官遣吏主之。事下三司,三司言其不可,允恭再三為請,太宗始從之。是歲,收利巨萬。允恭與王子輿、秦羲同主茶鹽之任,多作條制,遂變新法。



 真宗即位,改西京左藏庫使。又言川峽鐵錢之弊,曰:「凡民田之稅,昔輸銅錢之一,今輸鐵錢亦一;而吏卒奉舊給銅錢之一,今給鐵錢五;及行用交易,則鐵錢之十,為銅錢之一。且民入田稅,以一為十,官失其九矣;吏卒奉給,增一為五,官又失其四矣;吏卒得五用十,復失其半矣。臣在先朝,嘗陳其事,願變法以革其弊,先帝方議行之,會賊順叛擾而止。今陛下繼成先烈,可遂建其法,使民不失所。且饒、信之銅,積數千萬,若遣運於荊,達於蜀,
 蜀素多銅,俾夔、益、遂各置監鼓鑄,歲用均給,不十年,悉用銅錢矣。」議雖未用,然自是吏卒奉給,始改用十鐵錢易銅錢之一。



 俄知通利軍,兼黃、御河發運使。會議減西鄙屯兵,以息轉餉,召允恭與崇儀副使竇神寶、閣門祗候李允則馳往經度,圖上郡縣山川之形勝。允恭因建議曰:「自環州入積石、抵靈武七日程。芻粟之運,其策有三。然以人以驢,其費頗煩,而所載數鮮。莫若用諸葛亮木牛之制,以小車發卒分鋪運之。每一車四人挽之,旁
 設兵衛,加戈刃於其上,寇至則聚車於中,合士卒之力,御寇於外。」尋為議者所沮而止。復遣之任,又議,江、淮鹽鐵使陳恕力爭,詔從允恭之議。加領康州刺史。



 咸平初,以北邊賣馬,未有定直,命允恭主平其估,乃置估馬司,鑄印以為常制。王均之亂,上慮南方有聚寇,命允恭為荊湖、江、浙都巡檢使,內殿崇班楊守遵副之,賜與甚厚。二年夏,以疾聞,遣其子大理評事可乘傳侍疾。七月,卒於升州,年五十六,賜其次子告同學究出身,賻錢二十
 萬、絹百匹。又以錢五萬、帛五十匹給其家。命揚州官造第一區賜之。



 允恭有膽幹,能以方略捕賊。王小波之亂也,李順之兄自榮據綿竹,土人多被脅從。允恭兄允升、弟允元,率鄉里子弟並力破之;又為王師鄉導,執自榮詣劍門以獻。王繼恩表其事,詔賜允升學究出身,授本縣令,允元什邡令。明年,召赴闕,授允升右贊善大夫,允元大理評事。



 可,咸平元年進士,喜屬文,有吏乾,累召試,歷戶部、鹽鐵判官,知洪、宣、潤、壽、潭州,至都官員外郎。告,
 虞部員外郎。



 秦羲,字致堯,江寧人。世仕江左。曾祖本,岳州刺史。祖進遠,寧國軍節度副使。父承裕,建州監軍使、知州事。李煜之歸朝也,承裕遣羲詣闕上符印,太祖召見,悅其趨對詳謹,補殿直,令督廣濟漕船。太平興國中,有南唐軍校馬光璉等亡命荊楚,結徒為盜。羲受詔,縛光璉以獻,太宗壯之。積勞改西頭供奉官,決獄於淮南諸州。



 淳化中,又督洛南採銅。雷有終稱其有心計,遣監興國軍茶務。
 會楊允恭改茶鹽法,薦羲掌真州榷務,尋提點淮南西路茶鹽,得羨餘十餘萬,遂與允恭同為江、淮制置,擢授閣門祗候,兼制置礬稅。



 咸平初,入奏,真宗面加慰勞。淮南榷鹽,二歲增錢八十三萬餘貫,以勞改內殿崇班,又兼制置荊湖路。江南群盜久為民患,羲討捕皆盡。四年,領發運使事,改供備庫副使,獻議增榷酤歲十八萬緡,所增既多,尤為刻下。會歲旱,詔罷之。景德初,遷供備庫使、知江陵府。坐舉官不如狀,削秩。



 大中祥符初,起授供
 備庫副使、宿州監軍,稍遷東染院副使。明年,廣州言澄海兵嘗捕宜賊,頗希恩桀驁,軍中不能制,部送闕下。上以遠方大鎮,宜得材乾之臣鎮撫之。宰相歷言數人,皆不稱旨。上曰:「秦羲可當此任。」復授供備庫使,充廣州鈐轄。歷東染院使、知蘇州,改崇儀使、提舉在京諸司庫務。因對,求典藩郡,遷內園使、知泉州。天禧四年,代還。道病卒,年六十四。



 羲知書,好為詩,喜賓客,頗有士風。歷財貨之任,凡十餘年,精勤練習,號為稱職。



 謝德權,字士衡,福州人。父文節,初仕王氏,為侯官令。後入南唐,為忠烈都虞侯、饒州團練使,以驍勇聞。周世宗南征,文節獨擐甲度大江,潛覘敵壘,吳人號為「鐵龍」。後守鄂州,拒宋師,戰沒。



 德權初以父死事,李煜署莊宅副使。歸宋,詣登聞檢院自薦,補殿前承旨,遷殿直、陜西巡檢,以勞就改右侍禁。咸陽浮橋壞,轉運使宋太初命德權規畫,乃築土實岸,聚石為倉,用河中鐵牛之制,纜以竹索,繇是無患。



 咸平二年,宜州溪蠻叛,命陳堯叟往經
 度之,德權預其行,以單騎入蠻境,諭以朝旨,眾咸聽命。堯叟以聞,加閣門祗候,廣、韶、英、雄、連、賀六州都巡檢使。代歸,提點京城倉草場。先是,BW積多患地下濕,德權累甓為臺以藉之,遂無敗腐。



 京城衢巷狹隘,命德權廣之。既受詔,則先撤貴要邸舍,群議紛然。有詔止之,德權面請曰:「臣己受命,不可中止。今沮事者皆權豪輩,吝屋室僦資耳,非有他也。」上從之。因條上衢巷廣袤及禁鼓昏曉之制。



 會有兇人劉曄、僧澄雅訟執政與許州民陰構
 西夏為叛者,詔溫仲舒、謝泌鞫問,令德權監之。既而按驗無狀,翌日,對便殿,具奏其妄。泌獨曰:「追攝大臣,獄狀乃具。」德權曰:「泌欲陷大臣耶!若使大臣無罪受辱,則人君何以使臣,臣下何以事君?」仲舒曰:「德權所奏甚善。」上乃可之。



 六年,命城新樂縣,遷供奉官。又命浚北平砦濠,葺蒲陰城。一日,遽乘傳詣闕求對,且言:「邊民多挈族入城居止。前歲契丹入塞,傅潛閉壘自固,康保裔被擒,王師未有勝捷。臣以為今歲契丹必寇內地,令邊兵聚屯
 一處,尤非便利,願速分戍鎮、定、高陽三路。天雄城壘闊遠,請急詔蹙之,仍葺澶州城,北治德清軍城塹,以為豫備。臣實慮蒲陰工作未訖,寇必暴至。」上慰遣之,既而契丹果圍蒲陰。及聞有詔修河北行宮,德權又驛奏,請車駕毋渡河,及至澶州,德權單馬間道赴行在。



 未幾,遷內殿崇班、提轄三司衙司。德權為設條制,均其差使。有大將隸內侍主藏,內侍為奏留,規免煩重之役。德權攜奏白上,極言僥幸,上稱其有守。又命提總京城四排岸,領
 護汴河兼督輦運。前是,歲役浚河夫三十萬,而主者因循,堤防不固,但挑沙擁岸址,或河流泛濫,即中流復填淤矣。德權須以沙盡至土為垠,棄沙堤外,遣三班使者分地以主其役。又為大錐以試築堤之虛實,或引錐可入者,即坐所轄官吏,多被譴免者。植樹數十萬以固岸。建議廢京師鑄錢監,徙西窯務於河陰,大省勞費。改崇儀副使,兼領東西八作司。先時,每營造患工少,至終歲不成。德權按其役,皆克日而就。



 大中祥符元年,議東封,
 命與劉承珪、戚綸同計度發運,遷供備庫使。預修玉清昭應宮。時,累徙民舍以廣宮地。劉承珪議掘地及丈,加築以壯基址。德權患其勞役過甚,日與忿爭,不能奪,遂求罷,復領京城倉草場。導金水河,自皇城西環太廟,凡十餘里。三年,出知泗州,占謝日,自陳:「臣久領京務,頗慮中外觀聽,謂臣負譴外遷,願稍進其秩。」詔改西染院使遣之。至任,逾月卒,年五十八。以其子平為定遠主簿,給奉終喪。



 德權清苦干事,好興功利,多所經畫。見官吏徇
 私者,必面斥之,所至整肅。然喜採察纖微,以聞於上,朝論惡之。



 閻日新,宿州臨渙人。少為本州牙職,補三司使役吏。淳化中,選隸壽王府,主邸中記簿。真宗即位,擢為供奉官,提點雄、霸、靜戎軍榷場。咸平元年,遷內殿崇班、永興軍駐泊都監,徙劍門關兼知劍門縣,就加供備庫副使、慶州都監。景德初,命管勾邠、寧、環州駐泊兵馬。時,部署張凝屢入邊界焚族帳,日新皆提兵應援。俄知涇州,未幾,
 移慶州。上言:「野溪、三門等族恃嶮隘,桀黠難制,請開古川道,東至樂業鎮,西出府城。」從之。就轉供備庫使、知環州兼邠寧環慶路鈐轄、緣邊都巡檢使、安撫都監。俄換涇原儀渭路。二年,遷如京使,領萬州刺史。上朝陵、東封,皆命為行宮使。



 大中祥符初,改文思使。日新起胥史,好雲為以進取,嘗上言:「群臣子弟以蔭得官,往往未童齔以受奉,望自今年二十以上,乃給廩。又京城百官早朝,而學士、丞、郎、舍人以上,導從呵止太盛,難於趨避,望令
 裁減。」又屢請對,多所建白。且自陳筋力尚壯,願正授刺郡,守邊城以效用。



 俄真拜坊州刺史、知渭州兼涇原路駐泊鈐轄。將祀汾陰,故改知同州事,儼信頓即日新所部,車駕至,迎謁獻方物。勞問久之,遂從祀睢上,賜以襲衣、金帶。還過新市鎮,又設彩樓樂伎以迎駕。明年,徙知徐州。代還,以足疾,改右領軍衛大將軍、昭州團練使、知單州。疾益甚,許還京師。天禧初,卒,年六十八。



 靳懷德,博州高唐人。祖昌範,殿中丞。父隱,禹城令。懷德
 太平興國中明法,解褐廣安軍判官。秩滿,授鴻臚寺丞,歷著作佐郎、太子左贊善大夫、通判相州,改殿中丞、通判廣州,遷國子博士、通判滄州。歷虞部、比部員外郎,又通判莫州,知德州。



 咸平中,契丹入寇,懷德固守城壁,又轉運使劉通言其善政,連有詔褒之。徙知密州,會留後孔守正之鎮,代還。鹽鐵使陳恕、判官王濟薦其武干,換如京使、知邛州。懷德本名湘,素游寇準之門,準父名湘,景德中,準方為相,懷德乃改名焉。俄知滄州。大中祥符
 初,召還,復遣之任,吏民詣轉運使李士衡借留懷德,士衡以聞。未幾,遷文思使。三年秋,以江左旱歉,命為洪、虔十州安撫都監。未至任,改知曹州。



 明年春,選為益州鈐轄,加領長州刺史。懷德歷官以強乾稱,然酗酒多失,將行,別詔戒勖。真宗又面諭之,就遷北作坊使。在劍外,軍民甚畏愛之。復以善職入拜西上閣門使,改領昭州刺史、知澶州。是州居水陸之要,懷德悉心撫治,頗著政績,使車往復,多稱譽焉。又知陜州,逾年,歸闕而卒,時天禧
 元年,年七十三。



 論曰:世乏全材,則各錄其所長而用焉,亦皆可以集事功。允恭有心計,好言事,是時摘山煮海,方舟之漕,規制未備,故因其建白而從之,利甚博焉。羲亦精心敏職,士大夫許其醞藉。德權清廉強忮,矯名好威,然其斥謝泌以大臣非可受辱,識堂陛之分,長者之言哉。延德而下,遘會進陟,迭居事任,其指使治跡,各有可取者焉。



\end{pinyinscope}