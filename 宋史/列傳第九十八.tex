\article{列傳第九十八}

\begin{pinyinscope}

 蘇轍族孫元老



 蘇轍,字子由,年十九,與兄軾同登進士科,又同策制舉。仁宗春秋高,轍慮或倦於勤,因極言得失,而於禁廷之事,尤為切至。曰:



 陛下即位三十餘年矣,平居靜慮,亦嘗
 有憂於此乎,無憂於此乎?臣伏讀制策,陛下既有憂懼之言矣。然臣愚不敏,竊意陛下有其言耳,未有其實也。往者寶元、慶歷之間,西夏作難,陛下晝不安坐,夜不安席,天下皆謂陛下憂懼小心如周文王。然自西方解兵,陛下棄置憂懼之心,二十年矣。古之聖人,無事則深憂,有事則不懼。夫無事而深憂者,所以為有事之不懼也。今陛下無事則不憂,有事則大懼,臣以為憂樂之節易矣。臣疏遠小臣,聞之道路,不知信否?



 近歲以來,宮中貴
 姬至以千數,歌舞飲酒,優笑無度,坐朝不聞咨謨,便殿無所顧問。三代之衰,漢、唐之季,女寵之害,陛下亦知之矣。久而不止,百蠹將由之而出。內則蠱惑之所污,以傷和伐性;外則私謁之所亂,以敗政害事。陛下無謂好色於內,不害外事也。今海內窮困,生民愁苦,而宮中好賜不為限極,所欲則給,不問有無。司會不敢爭,大臣不敢諫,執契持敕,迅若兵火。國家內有養士、養兵之費,外有契丹、西夏之奉,陛下又自為一阱以耗其遺餘,臣恐陛
 下以此得謗,而民心不歸也。



 策入,轍自謂必見黜。考官司馬光第以三等,範鎮難之。蔡襄曰:「吾三司使也,司會之言,吾愧之而不敢怨。」惟考官胡宿以為不遜,請黜之。仁宗曰:「以直言召人,而以直言棄之,天下其謂我何?」宰相不得已,置之下等,授商州軍事推官。時父洵被命修《禮書》,兄軾簽書鳳翔判官。轍乞養親京師。三年,軾還,轍為大名推官。逾年,丁父憂。服除,神宗立已二年,轍上書言事,召對延和殿。



 時王安石以執政與陳升之領三
 司條例,命轍為之屬。呂惠卿附安石,轍與論多相牾。安石出《青苗書》使轍熟議,曰:「有不便,以告勿疑。」轍曰:「以錢貸民,使出息二分,本以救民,非為利也。然出納之際,吏緣為奸,雖有法不能禁,錢入民手,雖良民不免妄用;及其納錢,雖富民不免逾限。如此,則恐鞭棰必用,州縣之事不勝煩矣。唐劉晏掌國計,未嘗有所假貸。有尤之者,晏曰:『使民僥幸得錢,非國之福;使吏倚法督責,非民之便。吾雖未嘗假貸,而四方豐兇貴賤,知之未嘗逾時。有賤
 必糴,有貴必糶,以此四方無甚貴、甚賤之病,安用貸為?』晏之所言,則常平法耳。今此法見在而患不修,公誠能有意於民,舉而行之,則晏之功可立俟也。」安石曰:「君言誠有理,當徐思之。」自此逾月不言青苗。



 會河北轉運判官王廣廉奏乞度僧牒數千為本錢,於陜西漕司私行青苗法,春散秋斂,與安石意合,於是青苗法遂行。安石因遣八使之四方,訪求遺利。中外知其必迎合生事,皆莫敢言。轍往見陳升之曰:「昔嘉祐末,遣使寬恤諸路,各
 務生事,還奏多不可行,為天下笑。今何以異此?」又以書抵安石,力陳其不可。安石怒,將加以罪,升之止之,以為河南推官。會張方平知陳州,闢為教授。三年,授齊州掌書記。又三年,改著作佐郎。復從方平簽書南京判官。居二年,坐兄軾以詩得罪,謫監筠州鹽酒稅,五年不得調。移知績溪縣。



 哲宗立,以秘書省校書郎召。元祐元年,為右司諫。宣仁後臨朝,用司馬光、呂公著,欲革弊事,而舊相蔡確、韓縝、樞密使章惇皆在位,窺伺得失,轍皆論去
 之。呂惠卿始諂事王安石,倡行虐政以害天下。及勢鈞力敵,則傾陷安石,甚於仇讎,世尤惡之,至是,自知不免,乞宮觀以避貶竄。轍具疏其奸,以散官安置建州。



 司馬光以王安石雇役之害,欲復差役,不知其害相半於雇役。轍言:「自罷差役僅二十年,吏民皆未習慣。況役法關涉眾事,根芽盤錯,行之徐緩,乃得審詳。若不窮究首尾,忽遽便行,恐既行之後,別生諸弊。今州縣役錢,例有積年寬剩,大約足支數年,且依舊雇役,盡今年而止。催督
 有司審議差役,趁今冬成法,來年役使鄉戶。但使既行之後,無復人言,則進退皆便。」光又以安石私設《詩》、《書新義》考試天下士,欲改科舉,別為新格。轍言:「進士來年秋試,日月無幾,而議不時決。詩賦雖小技,比次聲律,用功不淺。至於治經,誦讀講解,尤不輕易。要之,來年皆未可施行。乞來年科場,一切如舊,惟經義兼取注疏及諸家論議,或出己見,不專用王氏學。仍罷律義,令舉人知有定論,一意為學,以待選試,然後徐議元祐五年以後科
 舉格式,未為晚也。」光皆不能從。



 初,神宗以夏國內亂,用兵攻討,乃於熙河增蘭州,於延安增安疆、米脂等五砦。二年,夏遣使賀登位,使還,未出境,又遣使入境。朝廷知其有請蘭州、五砦地意,大臣議棄守未決。轍言曰:「頃者西人雖至,疆場之事,初不自言。度其狡心,蓋知朝廷厭兵,確然不請,欲使此議發自朝廷,得以為重。朝廷深覺其意,忍而不予,情得勢窮,始來請命,一失此機,必為後悔。彼若點集兵馬,屯聚境上,許之則畏兵而予,不復為
 恩;不予則邊釁一開,禍難無已。間不容發,正在此時,不可失也。況今日之事,主上妙年,母後聽斷,將帥吏士,恩情未接,兵交之日,誰使效命?若其羽書沓至,勝負紛然,臨機決斷,誰任其責?惟乞聖心以此反復思慮,早賜裁斷,無使西人別致猖狂。」於是朝廷許還五砦,夏人遂服。遷起居郎、中書舍人。



 朝廷議回河故道,轍為公著言:「河決而北,自先帝不能回。今不因其舊而修其未至,乃欲取而回之,其為力也難,而為責也重,是謂智勇勢力過
 先帝也。」公著悟,竟未能用。進戶部侍郎。轍因轉對,言曰:「財賦之原,出於四方,而委於中都。故善為國者,藏之於民,其次藏之州郡。州郡有餘,則轉運司常足;轉運司既足,則戶部不困。唐制,天下賦稅,其一上供,其一送使,其一留州。比之於今,上供之數可謂少矣。然每有緩急,王命一出,舟車相銜,大事以濟。祖宗以來,法制雖殊,而諸道蓄藏之計,猶極豐厚。是以斂散及時,縱舍由己,利柄所在,所為必成。自熙寧以來,言利之臣,不知本末之術,
 欲求富國,而先困轉運司。轉運司既困,則上供不繼;上供不繼,而戶部亦憊矣。兩司既困,故內帑別藏,雖積如丘山,而委為朽壤,無益於算也。」尋又言:



 臣以祖宗故事考之,今日本部所行,體例不同,利害相遠,宜隨事措置,以塞弊原。謹具三弊以聞:其一曰分河渠案以為都水監,其二曰分冑案以為軍器監,其三曰分修造案以為將作監。三監皆隸工部,則本部所專,其餘無幾,出納損益,制在他司。頃者,司馬光秉政,知其為害,嘗使本部
 收攬諸司利權。當時所收,不得其要,至今三案猶為他司所擅,深可惜也。



 蓋國之有財,猶人之有飲食。飲食之道,當使口司出納,而腹制多寡。然後分布氣血,以養百骸,耳目賴之以為聰明,手足賴之以為力。若不專任口腹,而使手足、耳目得分治之,則雖欲求一飽不可得矣,而況於安且壽乎!今戶部之在朝廷,猶口腹也,而使他司分治其事,何以異此?自數十年以來,群臣每因一事不舉,輒入建他司。利權一分,用財無藝。他司以辦事為效,
 則不恤財之有無;戶部以給財為功,則不問事之當否。彼此各營一職,其勢不復相知,雖使戶部得材智之臣,終亦無益,能否同病,府庫卒空。今不早救,後患必甚。



 昔嘉祐中,京師頻歲大水,大臣始取河渠案置都水監。置監以來,比之舊案,所補何事?而大不便者,河北有外監丞,侵奪轉運司職事。轉運司之領河事也,郡之諸埽,埽之吏兵、儲蓄,無事則分,有事則合。水之所向,諸埽趨之,吏兵得以並功,儲蓄得以並用。故事作之日,無暴斂傷
 財之患,事定之後,徐補其闕,兩無所妨。自有監丞,據法責成,緩急之際,諸埽不相為用,而轉運司不勝其弊矣。此工部都水監為戶部之害,一也。



 先帝一新官制,並建六曹,隨曹付事,故三司故事多隸工曹,名雖近正而實非利。昔冑案所掌,今內為軍器監而上隸工部,外為都作院而上隸提刑司,欲有興作,戶部不得與議。訪聞河北道近歲為羊渾脫,動以千計。渾脫之用,必軍行乏水,過渡無船,然後須之。而其為物,稍經歲月,必至蠹敗。朝
 廷無出兵之計,而有司營戢,不顧利害,至使公私應副,虧財害物。若專在轉運司,必不至此。此工部都作院為戶部之害,二也。



 昔修造案掌百工之事,事有緩急,物有利害,皆得專之。今工部以辦職為事,則緩急利害,誰當議之?朝廷近以箔場竹箔,積久損爛,創令出賣,上下皆以為當。指揮未幾,復以諸處營造,歲有科制,遂令般運堆積,以破出賣之計。臣不知將作見工幾何,一歲所用幾何?取此積彼,未用之間,有無損敗,而遂為此計。本部
 雖知不便,而以工部之事,不敢復言。此工部將作監為戶部之害,三也。



 凡事之類此者多矣,臣不能遍舉也。故願明詔有司,罷外水監丞,舉河北河事及諸路都作院皆歸轉運司,至於都水、軍器、將作三監,皆兼隸戶部,使定其事之可否,裁其費之多少,而工部任其功之良苦,程其作之遲速。茍可否、多少在戶部,則傷財害民,戶部無所逃其責矣。茍良苦、遲速在工部,則敗事乏用,工部無所辭其譴矣。制出於一,而後天下貧富,可責之戶部
 矣。



 哲宗從之,惟都水仍舊。



 朝廷以吏部元豐所定吏額,比舊額數倍,命轍量事裁減。吏有白中孚曰:「吏額不難定也。昔之流內銓,今侍郎左選也,事之煩劇,莫過此矣。昔銓吏止十數,而今左選吏至數十,事不加舊而用吏至數倍,何也?昔無重法、重祿,吏通賕賂,則不欲人多以分所得。今行重法,給重祿,賕賂比舊為少,則不忌人多而幸於少事。此吏額多少之大情也。舊法,日生事以難易分七等,重者至一分,輕者至一厘以下,積若干分而
 為一人。今若取逐司兩月事定其分數,則吏額多少之限,無所逃矣。」轍曰:「此群吏身計所系也。若以分數為人數,必大有所損,將大致紛訴,雖朝廷亦不能守。」乃具以白宰執,請據實立額,俟吏之年滿轉出,或事故死亡者勿補,及額而止。不過十年,羨額當盡。功雖稍緩,而見吏知非身患,不復怨矣。呂大防命諸司吏任永壽與省吏數人典之,遂背轍議以立額,日裁損吏員,復以好惡改易諸局次。永壽復以贓刺配,大防略依轍議行之。代軾
 為翰林學士,尋權吏部尚書。使契丹,館客者侍讀學士王師儒能誦洵、軾之文及轍《茯苓賦》,恨不得見全集。使還,為御史中丞。



 自元祐初,一新庶政,至是五年矣。人心已定,惟元豐舊黨分布中外,多起邪說以搖撼在位,呂大防、劉摯患之,欲稍引用,以平夙怨,謂之「調停」。宣仁後疑不決,轍面斥其非,復上疏曰:



 臣近面論,君子小人不可並處,聖意似不以臣言為非者。然天威咫尺,言詞迫遽,有所不盡,臣而不言,誰當救其失者!親君子,遠小人,
 則主尊國安;疏君子,任小人,則主憂國殆。此理之必然。未聞以小人在外,憂其不悅而引之於內,以自遺患也。故臣謂小人雖不可任以腹心,至於牧守四方,奔走庶務,無所偏廢可也。若遂引之於內,是猶患盜賊之欲得財,而導之於寢室,知虎豹之欲食肉,而開之以坰牧,無是理也。且君子小人,勢同冰炭,同處必爭。一爭之後,小人必勝,君子必敗。何者?小人貪利忍恥,擊之則難去,君子潔身重義,沮之則引退。古語曰:「一熏一蕕,十年尚猶
 有臭。」蓋謂此矣。



 先帝聰明聖智,疾頹靡之俗,將以綱紀四方,比隆三代。而臣下不能將順,造作諸法,上逆天意,下失民心。二聖因民所願,取而更之,上下忻慰。則前者用事之臣,今朝廷雖不加斥逐,其勢亦不能復留矣。尚賴二聖慈仁,宥之於外,蓋已厚矣。而議者惑於說,乃欲招而納之,與之共事,謂之「調停」。非輩若返,豈肯但已哉?必將戕害正人,漸復舊事,以快私忿。人臣被禍,蓋不足言,臣所惜者,祖宗朝廷也。惟陛下斷自聖心,勿為流言
 所惑,勿使小人一進,後有噬臍之悔,則天下幸甚。



 疏入,宣仁後命宰執讀於簾前,曰:「轍疑吾君臣兼用邪正,其言極中理。」諸臣從而和之,「調停」之說遂已。



 轍又奏曰:



 竊見方今天下雖未大治,而祖宗綱紀具在,州郡民物粗安。若大臣正己平心,無生事要功之意,因弊修法,為安民靖國之術,則人心自定,雖有異黨,誰不歸心?向者異同反復之心,蓋亦不足慮矣。但患朝廷舉事,類不審詳,曩者,黃河北流,正得水性,而水官穿鑿,欲導之使東,移
 下就高,汩五行之理。及陛下遣使按視,知不可為,猶或固執不從。經今累歲,回河雖罷,減水尚存,遂使河朔生靈,財力俱困。今者西夏、青唐,外皆臣順,朝廷招來之厚,惟恐失之。而熙河將吏創築二堡,以侵其膏腴,議納醇忠,以奪其節鉞,功未可覬,爭已先形。朝廷雖知其非,終不明白處置,若遂養成邊釁,關陜豈復安居?如此二事,則臣所謂宜正己平心,無生事要功者也。



 昔嘉祐以前,鄉差衙前,民間常有破產之患。熙寧以後,出賣坊場以
 雇衙前,民間不復知有衙前之苦。及元祐之初,務於復舊,一例復差。官收坊場之錢,民出衙前之費,四方驚顧,眾議沸騰。尋知不可,旋又復雇。去年之秋,又復差法。又熙寧雇役之法,三等人戶,並出役錢,上戶以家產高強,出錢無藝,下戶昔不充役,亦遣出錢。故此二等人戶,不免咨怨。至於中等,昔既已自差役,今又出錢不多,雇法之行,最為其便。罷行雇法,上下二等,欣躍可知,唯是中等則反為害。且如畿縣中等之家,例出役錢三貫,若經
 十年,為錢三十貫而已。今差役既行,諸縣手力,最為輕役;農民在官,日使百錢,最為輕費。然一歲之用,已為三十六貫,二年役滿,為費七十餘貫。罷役而歸,寬鄉得閑三年,狹鄉不及一歲。以此較之,則差役五年之費,倍於雇役十年。賦役所出,多在中等。如此條目,不便非一,故天下皆思雇役而厭差役,今五年矣。如此二事,則臣所謂宜因弊修法,為安民靖國之術者也。



 臣以聞見淺狹,不能盡知當今得失。然四事不去,如臣等輩猶知其非,
 而況於心懷異同,志在反復,幸國之失,有以借口者乎?臣恐如此四事,彼已默識於心,多造謗議,待時而發,以搖撼眾聽矣。伏乞宣諭宰執,事有失當,改之勿疑,法或未完,修之無倦。茍民心既得,則異議自消。陛下端拱以享承平,大臣逡巡以安富貴,海內蒙福,上下攸同,豈不休哉!



 大臣恥過,終莫肯改。



 六年,拜尚書右丞,進門下侍郎。初,夏人來賀登極,相繼求和,且議地界。朝廷許約,地界已定,付以歲賜。久之,議不決。明年,夏人以兵襲涇原。
 殺掠弓箭手數千人,朝廷忍之不問,遣使往賜策命。夏人受禮倨慢,以地界為辭,不復入謝,再犯涇原。四年,來賀坤成節,且議地界。朝廷先以歲賜予之,地界又未決。夏人乃於疆事多方侵求,熙河將佐範育、種誼等,遂背約侵築買孤、勝如二堡,夏人即平蕩之。育等又欲以兵納趙醇忠,及擅招其部人千餘,朝廷卻而不受,西邊騷然。轍乞罷育、誼,別擇老將以守熙河。宣仁後以為然,大臣竟主育、誼,不從。轍又面奏:「人君與人臣,事體不同。人
 臣雖明見是非,而力所不加,須至且止;人君於事,不知則已,知而不能行,則事權去矣。臣今言此,蓋欲陛下收攬威柄,以正君臣之分而已。若專聽所謂,不以漸制之,及其太甚,必加之罪,不免逐去。事至如此,豈朝廷美事?故臣欲保全大臣,非欲害之也。」



 六年,熙河奏:「夏人十萬騎壓通遠軍境,挑掘所爭崖巉,殺人三日而退。乞因其退,急移近裏堡砦於界,乘利而往,不須復守誠信。」下大臣會議。轍曰:「當先定議欲用兵耶,不用耶?」呂大防曰:「如
 合用兵,亦不得不用。」轍曰:「凡用兵,先論理之曲直。我若不直,兵決不當用。朝廷須與夏人議地界,欲用慶歷舊例,以彼此見今住處當中為直,此理最簡直。夏人不從,朝廷遂不固執。蓋朝廷臨事,常患先易後難,此所謂先易者也。既而許於非所賜城砦,依綏州例,以二十里為界,十里為堡鋪,十里為草地。要約才定,朝廷又要兩砦界首侵夏地,一抹取直,夏人見從。又要夏界更留草地十里,夏人亦許。凡此所謂後難者也。今欲於定西城與
 隴諾堡一抹取直,所侵夏地凡百數十里。隴諾祖宗舊疆,豈所謂非所賜城砦耶?此則不直,致寇之大者也。」劉摯曰:「不用兵雖美,然事有須用兵者,亦不可不用也。」轍奏曰:「夏兵十萬壓熙河境上,不於他處,專於所爭處殺人、掘崖巉,此意可見,此非西人之罪,皆朝廷不直之故。熙河輒敢生事,不守誠信,臣欲詰責帥臣耳。」後屢因邊兵深入夏地,宣仁後遂從轍議。



 時三省除李清臣吏部尚書,給事中範祖禹封還詔書,且言姚勉亦言之。三省
 復除蒲宗孟兵部尚書。轍奏:「前除清臣,給諫紛然,爭之未定。今又用宗孟,恐不便。」宣仁後曰:「奈闕官何?」轍曰:「尚書闕官已數年,何嘗闕事?今日用此二人,正與去年用鄧溫伯無異。此三人者,非有大惡,但昔與王珪、蔡確輩並進,意思與今日聖政不合。見今尚書共闕四人,若並用似此四人,使黨類互進,恐朝廷自是不安靜矣。」議遂止。



 紹聖初,哲宗起李清臣為中書舍人,鄧潤甫為尚書左丞。二人久在外,不得志,稍復言熙、豐事以激怒哲宗
 意。會廷試進士,清臣撰策題,即為邪說。轍諫曰:



 伏見御試策題,歷詆近歲行事,有紹復熙寧、元豐之意。臣謂先帝以天縱之才,行大有為之志,其所設施,度越前古,蓋有百世不可改者。在位近二十年,而終身不受尊號。裁損宗室,恩止袒免,減朝廷無窮之費。出賣坊場,顧募衙前,免民間破家之患。黜罷諸科誦數之學,訓練諸將慵惰之兵。置寄祿之官,復六曹之舊,嚴重祿之法,禁交謁之私。行淺攻之策以制西夏,收六色之錢以寬雜役。凡
 如此類,皆先帝之睿算,有利無害,而元祐以來,上下奉行,未嘗失墜也。至於其它,事有失當,何世無之。父作之於前,子救之於後,前後相濟,此則聖人之孝也。



 漢武帝外事四征,內興宮室,財用匱竭,於是修鹽鐵、榷酤、均輸之政,民不堪命,幾至大亂。昭帝委任霍光,罷去煩苛,漢室乃定。光武、顯宗以察為明,以讖決事,上下恐懼,人懷不安。章帝即位,深鑒其失,代之以寬厚、愷悌之政,後世稱焉。本朝真宗右文偃武,號稱太平,而群臣因其極盛,
 為天書之說。章獻臨御,攬大臣之議,藏書梓宮,以泯其跡;及仁宗聽政,絕口不言。英宗自藩邸入繼,大臣創濮廟之議。及先帝嗣位,或請復舉其事,寢而不答,遂以安靜。夫以漢昭、章之賢,與吾仁宗、神宗之聖,豈其薄於孝敬而輕事變易也哉?臣不勝區區,願陛下反復臣言,慎勿輕事改易。若輕變九年已行之事,擢任累歲不用之人,人懷私忿,而以先帝為辭,大事去矣。



 哲宗覽奏,以為引漢武方先朝,不悅。落職知汝州。居數月,元豐諸臣皆
 會於朝,再責知袁州。未至,降朝議大夫、試少府監,分司南京,筠州居住。三年,又責化州別駕,雷州安置,移循州。徽宗即位,徙永州、岳州,已而復太中大夫,提舉鳳翔上清太平宮。崇寧中,蔡京當國,又降朝請大夫,罷祠,居許州,再復太中大夫致仕。築室於許,號穎濱遺老,自作傳萬餘言,不復與人相見。終日默坐,如是者幾十年。政和二年,卒,年七十四。追復端明殿學士。淳熙中,謚文定。



 轍性沉靜簡潔,為文汪洋澹泊,似其為人,不願人知之,而
 秀傑之氣終不可掩,其高處殆與兄軾相迫。所著《詩傳》、《春秋傳》、《古史》、《老子解》、《欒城文集》並行於世。三子:遲、適、遜。族孫元老。



 元老字子廷。幼孤力學,長於《春秋》,善屬文。軾謫居海上,數以書往來。軾喜其為學有功,轍亦愛獎之。黃庭堅見而奇之,曰:「此蘇氏之秀也。」舉進士,調廣都簿,歷漢州教授、西京國子博士、通判彭州。



 政和間,宰相喜開邊西南,帥臣多啖誘近界諸族使納土,分置郡縣以為功,致茂
 州蠻叛,帥司遽下令招降。元老嘆曰:「威不足以服,則恩不足以懷。」乃移書成都帥周燾曰:「此蠻跳梁山谷間,伺間竊發。彼之所長,我之所短,惟施、黔兩州兵可與為敵。若檄數千人,使倍道往赴,賢於官軍十萬也。其次以為夔、陜兵大集,先以夔兵誘其前,陜兵從其後,不十日,賊必破。彼降而我受焉,則威懷之道得。今不討賊,既招而還,必復叛,不免重用兵矣。」燾得書,即召與計事。元老又策:「茂有兩道,正道自濕山趨長平,絕嶺而上,其路險以
 高;間道自青崖關趨刁溪,循江而行,其路夷以徑。當使正兵陣濕山,而陰出奇兵搗刁溪,與石泉並力合攻,賊腹背受敵,擒之必矣。」燾皆不能用,竟得罪。後帥至,如元老策,蠻勢蹙,乃降。



 除國子博士,歷秘書正字、將作少監、比部考功員外郎,尋除成都路轉運副使,為軍器監,司農、衛尉、太常少卿。



 元老外和內勁,不妄與人交。梁師成方用事,自言為軾外子,因緣欲見之,且求其文,拒不答。言者遂論元老蘇軾從孫,且為元祐邪說,其學術議
 論,頗仿軾、轍,不宜在中朝。罷為提點明道宮。元老嘆曰:「昔顏子附驥尾而名顯,吾今以家世坐累,榮矣。」未幾卒,年四十七。有詩文行於時。



 論曰:蘇轍論事精確,修辭簡嚴,未必劣於其兄。王安石初議青苗,轍數語柅之,安石自是不復及此,後非王廣廉傅會,則此議息矣。轍寡言鮮欲,素有以得安石之敬心,故能爾也。若是者,軾宜若不及,然至論軾英邁之氣,閎肆之文,轍為軾弟,可謂難矣。元祐秉政,力斥章、蔡,不
 主調停;及議回河、雇役,與文彥博、司馬光異同;西邊之謀,又與呂大防、劉摯不合。君子不黨,於轍見之。轍與兄進退出處,無不相同,患難之中,友愛彌篤,無少怨尤,近古罕見。獨其齒爵皆優於兄,意者造物之所賦與,亦有乘除於其間哉!



\end{pinyinscope}