\article{列傳第六十}

\begin{pinyinscope}

 邊肅梅詢馬元方薛田寇瑊楊日嚴李行簡章頻陳琰李宥張秉張擇行鄭向郭稹趙賀高覿袁抗徐起張旨齊廓
 鄭驤



 邊肅,字安國,應天府楚丘人。進士及第,除大理評事、知於潛縣,累遷太常博士。三司使魏羽薦為戶部判官,祀南郊,超薦尚書度支員外郎。帝以三司鉤取無法,至道初,置行帳司,以會財用之數,命肅主之。帳成,遷工部郎中。



 真宗幸大名府,命肅經度行在糧草。改判開拆司,出知曹州,徙邢州。會契丹大入,先是地屢震,城堞摧圮,無守備,帝在澶州,密詔肅:「若州不可守,聽便宜南保他城。」
 肅匿詔不發,督丁壯乘城而闢諸門,悉所部兵陣以代之。騎傅城下,肅與戰小勝,契丹莫測也,居三日,引去。時鎮、魏、深、趙、磁、洺六州閉壁不出,老幼趨城者,肅悉開門納之。



 擢樞密直學士,徙宣州。車駕朝陵,徙河南府。還,勾當三班院。出知天雄軍,徙真定府,累遷給事中。以王嗣宗代肅。嗣宗與肅有舊隙,諷通判東方慶訟肅前在州,私以公錢貿易規利,遣吏強市民羊,買女口自入。嗣宗上其事,帝以肅近臣,不欲屬吏,遣劉綜、任中正以章示
 之,肅引伏。以守城功,止奪三官,貶岳州團練副使。久之,徙武昌、安遠軍節度副使,起知光州,以泰寧軍節度副使徙泗州,又徙泰州,卒。



 子調,終尚書兵部員外郎、福建路轉運使。



 梅詢,字昌言,宣州宣城人。少好學,有辭辨。進士及第,為利豐監判官。後以秘書省著作佐郎、御史臺推勘官,預考進士於崇政殿,真宗過殿廬,奇其占對詳敏,召試中書,除集賢院。



 李繼遷攻靈州急,吳淑上書請遣使諭秦、
 隴以西諸戎,使攻繼遷。詢亦請以朔方授潘羅支,使自攻取。帝問誰可使羅支者,詢請行,未至而靈州陷。還,為三司戶部判官。詢自以為遇主知,屢上書陳論西北事。時契丹數侵河北,詢請遣大臣臨邊督戰,募游手擊賊。又論曹瑋、馬知節才可用,傅潛、楊瓊敗當誅,田紹斌、王榮等可責其效以贖過,凡數十事,其言甚壯。



 帝欲命知制誥,李沆力言其險薄望輕,不可用。後斷田訟失實,降通判杭州,知蘇州,就徙兩浙轉運副使,判三司開拆司。
 坐議天書,出知濠州。為湖北轉運使,擅假驛馬與邵曄子省親疾而馬死,奪官一級,降通判襄州。知鄂州,徙蘇州,為陜西轉運使。坐薦舉朱能,貶懷州團練副使。又以善寇準,徙池州。起知廣德軍,歷楚、壽、陜州。復直集賢院,改直昭文館、知荊南,擢龍圖閣待制,糾察在京刑獄。歷龍圖閣直學士、樞密直學士,知通進銀臺司,判流內銓,為翰林侍讀學士、群牧使。累遷給事中、知審官院。



 仁宗御邇英閣,讀《正說養民篇》,覽歷代戶口登耗之數,顧謂
 侍臣曰:「今天下民籍幾何?」詢對曰:「先帝所作,蓋述前代帝王恭儉有節,則戶口充羨;賦斂無藝,則版圖衰減。炳然在目,作鑒後王。自五代之季,生齒凋耗,太祖受命,而太宗、真宗休養百姓,天下戶口之數,蓋倍於前矣。」因詔三司及編修院檢閱以聞。病足,出知許州,卒。故事,侍讀學士無出外者。天禧中,張知白罷參知政事,領此職,始出知大名府。非歷二府而出者自詢始。



 詢性卞急好進,而侈於奉養,至老不衰。然數為朝廷言兵。在濠州,夢人
 告曰:「呂丞相至矣。」既而呂夷簡通判州事,故待之甚厚。其後,援詢於廢斥中,以至貴顯,夷簡力也。



 馬元方,字景山,濮州鄄城人。父應圖,嘗知頓丘縣,太宗攻幽州,應圖部芻糧,沒虜中。元方去發為浮屠,間行求父尸,不得,訴於朝。上哀之,為官其兄元吉。



 元方,淳化三年進士及第,為韋城縣主簿,改大理寺評事、知萬年縣。諸將討李繼遷,關輔轉餉逾瀚海,多失亡,獨元方所部全十九。以勞,遷本寺丞,為御史臺推勘官,遷殿中丞。戶
 部使陳恕奏為判官,元方言:「方春民貧,請預貸庫錢,至夏秋,令以絹輸官。」行之,公私果便,因下其法諸路。



 知徐州,改太常博士、梓州路轉運使。後知鄆州,量括牧地數千頃。為京東轉運副使,遷轉運使。按部至濮州,被酒毆知州蔣信,降知宿州,下詔切責之。徙滑州,為京西轉運使,知應天府,累遷太常少卿。擢右諫議大夫、權三司使公事,眾論不以為允。真宗謂宰臣曰:「元方在三司,何多謗也?」王旦曰:「元方盡心營職,然其性卞急,且不納僚屬
 議,而醜言詆之,所以賈怨。」帝曰:「僚屬顧不有賢俊邪?」歲餘,以煩苛罷。進給事中、權知開封府。以樞密直學士知並州,留再任,賜白金五百兩,詔中書諭以委屬之意。官至兵部侍郎,卒。



 薛田,字希稷,河中河東人。少師事種放,與魏野友善。進士,起家丹州推官。李允正知延州,闢為從事,向敏中至,亦薦其材。改著作佐郎、知中江縣。真宗祀汾陰,田時居父喪,經度制置使陳堯叟奏起通判陜州。還,拜監察御
 史,以母憂去。會祀太清宮,又用丁謂奏,起通判亳州。遷殿中侍御史、權三司度支判官,改侍御史、益州路轉運使。民間以鐵錢重,私為券以便交易,謂之「交子」,而富家專之,數致爭訟。田請置交子務,以榷其出入,未報。及寇瑊守益州,卒奏用其議,蜀人便之。



 就除陜西轉運使,進直昭文館、知河南府,復入度支為副使。使契丹還,擢龍圖閣待制、知天雄軍。未幾,擢知開封府,以樞密直學士知益州,累遷左司郎中。代還,知審刑院。羌人內寇,特遷
 右諫議大夫、知延州。久之,以疾徙同州,又徙永興軍,辭不行,卒。



 田性頗和厚,初以幹敏數為大臣所稱,後屢更任使,所治無赫赫名。



 寇瑊,字次公,汝州臨汝人。初,母夢神人授珠,吞之而娠,生而眉目美秀。擢進士,授蓬州軍事推官。李順餘黨謝才盛等復起為盜,瑊設方略,擒送京師。



 徙開封推官。會施州蠻叛,轉運使移瑊權領施州。先是,戍兵仰他州饋糧,瑊至,請募人入米,償以鹽,軍食遂足,而民力紓。復招
 諭高州刺史田彥伊子承寶入朝,得給印紙為高州官族。未幾,溪南蠻復內寇,瑊率眾擒其酋領戮之,以白芀子弟數百人築柵,守其險要。



 就除大理寺丞、知開州,遷殿中丞、通判河南府。坐解送諸料失實,降監晉州稅。以太常博士通判並州,改監察御史。真宗祀汾陰,王嗣宗知永興,闢權通判,專領祠事。遷殿中侍御史,為開封府判官。嘗奏事,帝詢施州備御之術,因諭之曰:「東川控蠻夷,爾功已試,其為朕鎮撫之。」命為梓州路轉運使。



 晏州
 多剛縣酋斗望劫瀘州,燒淯井監,殺官吏。瑊趨富順監,命部兵多張旗幟,逾山西北趨戎州,盡取公私舟載糧甲,具音樂,合兩路兵至江安,誘納溪、藍、順史個松,南廣移、悅等州刺史及八姓烏蠻首領,使斷賊徑。用夷法,植竹為誓門,橫竹系貓、犬、雞各一於其上,老夷人執刀劍,謂之「打誓」,呼曰:「誓與漢家同心擊賊。」即刺牲血和酒而飲。瑊給鹽及酒食、針梳、衣服等,付以大榜,約大軍至,揭榜以別逆順,「不殺汝老少,不燒汝欄柵。」夷人大喜。帝遣內
 殿崇班王懷信議攻討招輯之宜,瑊奏:「夷人嘗於二年春燒淯井監,殺吏民。既赦貸其罪,復來寇邊,聲言朝廷且招安,得酒食衣服矣。若不討除,則戎、瀘、資、榮、富順監諸夷競起為邊害矣。」詔發陜西兵,益以白芀子弟合六千三百人,緣淯井溪轉鬥,凡十一陣,破之。夷人相率來附,納牛羊、銅鼓、器械甚眾,而斗望猶旅拒不從。瑊命懷信分兵拔其柵,與都巡檢使符承順進戰思晏江口,斗望等始驚遽,勢稍卻,明日,復分三道來拒王師,懷信等
 格戰,瑊乘其後,大破之。斗望眾萬餘,囂不能軍,溺死者眾,遂降。因籍軍之勇悍千人,分五都以隸禁軍,為寧遠指揮,使守淯井監,更建砦柵,浚三壕以環之。就加侍御史,召為三司鹽鐵判官,逾月,出為河北轉運使。



 天禧中,河決澶淵。瑊視役河上,堤墊數里,眾皆奔潰,而瑊獨留自若。須臾,水為折去,眾頗異之。遷工部郎中,上言:「契丹約和以來,河北減戍卒之半,而復刺土兵,其實益三分之一,而塞下軍儲不給。請行入中、鑿頭、便糴三說之法。」
 入為三司度支副使。未幾,以右諫議大夫、集賢院學士知益州。



 仁宗即位,遷給事中。瑊與丁謂厚善,帝謂輔臣曰:「瑊有吏乾,毋深譴也。」徙鄧州,坐失舉,降少府監、知金州,復右諫議大夫。會河決,徙知滑州,總領修河。既而以歲饑罷役,瑊言:「病民者特楗芻耳,幸調率已集,若積之經年,則朽腐為棄物,後復興工斂之,是重困也。」乃再詔塞河。河平,擢樞密直學士。



 明年,復給事中、知秦州,又坐失舉奪一官。召權三司使,復其官如故。時有議茶法者,
 帝訪以利害,瑊曰:「議者未知其要爾。河北入中兵食,皆仰給於商旅。若官盡其利,則商旅不行,而邊民困於饋運,茶法豈可以數更?」帝然之。權知開封府,戚里有毆妻至死,更赦事發者。太后怒曰:「夫婦齊體,奈何毆致死邪?」瑊對曰:「傷居限外,事在赦前,有司不敢亂天下法。」卒免死。天聖末,再使契丹,未行而卒。



 瑊少孤,鞠於祖母王氏,及登朝,以妻封邑回授之,朝臣得回封祖母自瑊始。性頗疏財,通音律,知術數。初附丁謂,故少達,及謂敗左遷,
 鬱鬱不自得,秘書丞彭齊賦《喪家狗》以刺之。



 楊日嚴,字垂訓,河南人。進士及第,試秘書省校書郎、知安丘縣。三司闢為檢法官,遷大理寺丞,又為本寺檢法官,監都進奏院,通判亳、陳二州,判吏部南曹兼登聞鼓院。出知襄州,徙廬、鄲二州,入為開封府判官。



 使契丹還,為兩浙轉運副使。未行,會青、徐饑,改京東轉運使。因請江、淮、陜西轉粟五十萬,以賑貧民;又開清河八十里抵暖水河,並堤起倉廩,以便漕運。加直史館,徙益州轉運
 使,又徙江、淮制置發運使。還,歷三司戶部、度支、鹽鐵副使。累遷太常少卿,以右諫議大夫、集賢院學士知河中府,加樞密直學士、知益州。



 時用兵伐元昊,三司急財用,有詔析戶版為十等,第賦役;民以歲租占佃官田廬者,高其估,募輸錢就市為己業,人苦其擾。又陜西奏收市益、梓、利路溪洞馬,而不知其實無馬也。日嚴皆奏罷之。遷勾當三班院、知通進銀臺司。聞後為守者,其政不便蜀人,因進對,猶從容言:「遠方所宜撫安之,無容變法以
 生事。」遷給事中,以龍圖閣學士知澶州。召權知開封府,吏械囚不謹,囚自殺,坐是罷府事。判太常、司農寺,同知審官院,卒。



 日嚴初為益州轉運使,無他治能,及知益州,頗為蜀人所信愛。兄日華,歷官至太常少卿、三司副使。



 李行簡,字易從,同州馮翊人。家貧,刻志於學,讀《六經》每至夜分,寒暑不易。又聚木葉學書,筆法遒勸。與里中富人楊士元同學,既而同時中進士第,士元資遺行簡,謝不取。起家隴州司理參軍,徙彭州軍事推官。



 陵州富民
 陳子美父死,繼母詐為父書逐出之,累訴不得直,轉運使檄行簡劾正其獄。改秘書省著作郎,再遷太常博士、知坊州。御史中丞王嗣宗薦為監察御史,王旦數稱其才,真宗雅亦知之,再遷侍御史。



 陜西旱蝗,命往安撫,發倉粟救乏絕,又蠲耀州積年逋租。還,擢龍圖閣待制,歷尚書刑部郎中。帝數幸龍圖閣,命講《周易》,間訪大臣能否,行簡所對無怨暱,各道其所長,人以為長者。久之,拜右諫議大夫、集賢院學士。乾興初,改給事中,以足疾請
 外,得知河中府,徙虢州,卒。



 章頻,字簡之,建州浦城人。與弟□皆以進士試禮部預選,會詔兄弟毋並舉,頻即推其弟,棄去。後六年,乃擢第。自試秘書省校書郎、知南昌縣,改大理寺丞、知九隴縣,遷殿中丞。



 眉州大姓孫延世偽為券奪族人田,久不能辨,轉運使使按治之。頻視券墨浮朱上,曰:「是必先盜印然後書。」既引伏,獄未上,而其家人復訴於轉運使,更命知華陽縣黃夢松覆按,無所異。夢松用此入為監察御
 史,頻坐不時具獄,降監慶州酒,徙知長洲縣。



 天禧初,增置諫官、御史十二人,頻以選得召對,稱旨,擢監察御史。陳、亳間民訛言兵起,老幼皆奔,命安撫京西。還,為三司度支判官。青州麻士瑤殺從子溫裕,並其財,遣往按治,士瑤伏誅。又詔鞫邛州牙校訟鹽井事。皇城使劉美依倚後家受賕,使人市其獄,頻請捕系,真宗以後故不問。忤旨,出知宣州,改殿中侍御史,遷侍御史。



 頻雅善丁謂,謂貶,左遷尚書比部員外郎、監饒州酒。起知信州,進刑
 部員外郎、知福州。王氏時,賦民官田,歲輸租稅而已。至是,或謂鬻之可得緡錢二十餘萬,頻疏以為不可。徙知潭州。改廣西轉運使,擿宜州守貪暴不法,既罷去,反訟頻子許嘗被刑,而冒奏為秘書省校書郎,頻坐謫知饒州。復入為度支判官,累遷刑部郎中。



 使契丹,至紫蒙館卒。契丹遣內侍就館奠祭,命接伴副使吳克荷護其喪,以錦車駕橐駝載至中京,斂以銀飾棺,又具鼓吹羽葆,吏士持甲兵衛送至白溝。詔遣其子訪乘傳扈其柩以
 歸。訪官三班奉職,即許也。



 陳琰,字伯玉,澶州臨河人。進士及第,歷溧陽、欒城縣主簿,遷大理寺丞、監真定府稅,知金堂、夏津二縣。再遷太常博士。轉運使盧士倫,曹利用婿也,怙勢聽獄不以直,訟者不已,付琰評決,琰直之。御史知雜韓億聞其事,奏為監察御史。丁父喪,哀毀,墳木連理。憂除,遷殿中侍御史。



 天聖五年祀南郊,中外以為丁謂復還,琰上疏曰:「亂常肆逆,將而必誅,陰懷奸惡,有殺無赦。丁謂因緣險佞,
 據竊公臺。賄賂包苴,盈於私室;哇。琰始奏選官監視,謂之「定計斗面」。積遷至尚書工部郎中,卒。



 李宥字仲嚴,唐之後裔,自吳徙青,遂為青人。祖成,五代末,以詩酒游公卿間,善摹寫山水,至得意處,疑非筆墨所成。人欲求者,先為置酒,酒酣落筆,煙景萬狀,世傳以為寶。赦。」帝然之。



 為三司度支判官,遷侍御史。歷京西、河東、河北轉運副使,三司戶部、度支、鹽鐵副使。汴倉納糧綱,概量不實,操舟者坐亡失所載,或杖背徒重役。琰始奏選官監視,謂之「定計斗面」。積遷至尚書
 工部郎中,卒。



 李宥,字仲嚴,唐之後裔,自吳徙青,遂為青人。祖成,五代末,以詩酒游公卿間,善摹寫山水,至得意處,疑非筆墨所成。人欲求者,先為置酒,酒酣落筆,煙景萬狀,世傳以為寶。父覺,見《儒林傳》。



 宥幼孤,不好弄,長讀書屬文,不雜交游。舉進士,調火山軍判官。入館校勘書籍,遷集賢校理,遂直院。知蘄州,歲兇人散,委嬰孩而去者相屬於道。宥令吏收取,計口給穀,俾營婦均養之,每旬閱視,所活
 甚眾。或殺人,以米十石給傭者,使就獄,曰:「我重賄吏,爾必不死。」宥得其情,論如法。



 提點荊湖刑獄,權戶部判官,利州轉運使,判戶部勾院,知制誥,糾察在京刑獄,同判太常寺。舊宗廟五饗,輔臣攝事,中廢且久,止差從官。宥因對力言,遂復故事。以諫議大夫知江寧府。民有告人殺其子者,曰:「吾子去家時,巾若巾,今巾是矣。」民自誣服。宥疑,召問,卒伸其枉。府舍火,宥畏兵亂,闔門不救,降秘書監致仕。起分司南京,改太子賓客,判留司御史臺,卒。



 宥性清介,然與物無忤,好獎拔士人。外族甚貧,宥有別業,以券畀之。既死,家無餘財,官賜錢十萬。



 張秉,字孟節,歙州新安人。父諤,字昌言,南唐秘書丞、通判鄂州。宋師南伐,與州將許昌裔葉議歸款,太祖召見,勞賜良厚,授右贊善大夫。蜀平,選知閬州。太平興國中,即除西川轉運副使。先是,土人罕習舟楫,取峽江中競渡者給漕運役,覆溺常十四五。諤建議置威棹軍分隸管勾,自是無覆舟之患。累遷荊湖、江、浙等道制置茶鹽
 副使,卒。



 秉舉進士,儀狀豐麗,屬詞敏速,善書翰,太宗喜之,擢置甲科。解褐將作監丞、通判宣州。遷監察御史,深為宰相趙普所器,以弟之子妻之。會有薦其才,得知鄭州。召還,直昭文館,遷右司諫。會以趙昌言為制置茶鹽使,秉與薛映副之。入為右計司河南西道判官,俄換鹽鐵判官、度支員外郎、知制誥、判吏部銓、知審官院。唐朝故事,南省首曹罕兼掌誥,多退為行內諸曹郎。至是,用此制,其後進改,多優遷首曹,遂隳舊制矣。遷工部郎中,
 依前知制誥。



 真宗嗣位,進秩兵部郎中、判昭文館。時草敕用官制,有「頃因微累,謫於遐荒」之語,上覽之曰:「若此,則是先朝失刑矣。」遂除秉左諫議大夫,連知穎、襄二州。徙鳳翔府,訴以母老貧窶,詔給裝錢,未行,改江陵。丁母憂,起復,知河南府。景德初,徙河陽,換澶州。車駕將幸河上,又徙知滑州。道出韋城,秉迎謁境上,俾預從官侍食;遣與齊州馬應昌、濮州張晟往來河上,部丁夫鑿凌,以防契丹南渡。



 召歸闕,復判吏部銓,拜工部侍郎、同知審
 官院、通進銀臺司,糾察在京刑獄。復與周起同試東封路服勤辭學、經明行修舉人。出知永興軍府,會祀汾陰,為東京留守判官,轉禮部侍郎,加樞密直學士,復知並州。將行,懇求御詩為餞,上為作五言賜之。徙相州。九年,復糾察在京刑獄,暴疾卒。



 秉典藩府,無顯赫譽,及再至太原,臨事少斷,多與賓佐博弈。雖久踐中外,然無儀檢,好諧戲,人不以宿素稱之。好飭衣服,潔饌具,每公宴及朋友家集會,多自挈肴膳而往。家甚貧,常質衣以給費
 焉。



 張擇行,字行先,青州益都人。進士起家,歷北海、臨沂主簿,自宣州觀察推官為大理寺丞。初,石亭縣掾檄將陵塞決河,眾欲登舟以濟,擇行獨以為不可,皆笑其怯。既而舟果覆,擇行坐堤上董役,埽卒不潰。



 除監察御史、殿中侍御史,改言事御史、右司諫。與唐介、包拯共論張堯佐除節度、宣徽兩使不當,語甚切。又論河北兵多、財不足,願分兵就食內地,不報。遷侍御史知雜事,擢天章閣
 待制、知諫院,累遷吏部員外郎。御史皆言宰相陳執中嬖妾笞小婢,死外舍。擇行以為主命妾笞婢,於律不當坐,御史固迫之,因中風不能語。除戶部郎中、集賢殿修撰,提舉兗州仙源縣景靈宮,逾年而卒。



 鄭向,字公明,開封陳留人。舉進士中甲科,為大理評事、通判蔡州,累遷尚書屯田員外郎、知濠州,徙蔡州。召試集賢院,未幾,除三司戶部判官,修起居注。遷度支員外郎,為鹽鐵判官。出為兩浙轉運副使,疏潤州蒜山漕河
 抵於江,人以為便。復為鹽鐵判官,擢知制誥、同勾當三班院。使契丹,再遷兵部郎中、提點諸司庫務,以龍圖閣直學士知杭州,卒。



 五代亂亡,史冊多漏失,向著《開皇紀》三十卷,摭拾遺事,頗有補焉。



 郭稹,字仲微,開封祥符人。世寓鄭州,舉進士中甲科,為河南縣主簿。除國子監直講,議者以其資淺,罷還河南。時孫奭、馮元判監事,因奏稹學問通博,他選莫能及,乃得留。居二歲,陳堯咨知大名,闢簽書府判官事,改大理
 寺丞。奭等復薦為直講。奭出知兗州,又薦稹與賈昌朝赴中書試講說,而稹固辭。召試學士院,為集賢校理。馮元知河陽,闢為通判,徙通判河南府。入為三司度支、戶部判官,累遷尚書刑部員外郎,同修起居注。



 康定元年使契丹,告用兵西鄙。契丹厚禮之,與同出觀獵,延稹射。稹一發中走兔,眾皆愕視,契丹主遺以所乘馬及他物甚厚。既還,轉兵部,知制誥,判吏部流內銓,擢龍圖閣直學士、權知開封府。暴感風眩卒。



 稹性和易,文思敏贍,尤
 刻意於賦,好用經語對,頗近於諧。聚古書畫,不計其貲購求之。婦張悍嫉,無子。初,稹幼孤,母邊更嫁王氏,既而母亡,稹解官服喪。知禮院宋祁言稹服喪為過禮,詔下有司博議,用馮元等奏,聽解官申心喪,語在《禮志》。



 論曰:肅之守邢,以贏兵卻敵,開門納避難之民,功在王府。元方為並州,有勤留之命,其宜民可知。宥在蘄,則活饑氓;在江寧,則直冤獄。吏之良者歟!然皆不能無小累也。日嚴、行簡臨政,視秉、擇行、向、稹雖無瑕可指,亦皆無
 赫赫名。詢以厚呂夷簡,復致貴顯;瑊、頻坐善丁謂,並遭斥謫,固無足議者。琰言謂奸邪,不當用南郊恩牽復,與唐袁高論執盧杞正相類,識者韙之。



 趙賀,字餘慶,開封封邱人。少時,嘗喪明,久之,遇異醫輒愈。喜飲酒,至終日不亂。事繼母至孝。舉《毛詩》及第,補臨朐縣主簿。賀有幹力,知州寇準且知賀。淳化中,調丁壯塞澶州決河,眾多逸去,獨賀全所部而歸。臨朐父老張樂迎賀,準使由譙門過,曰:「旌賀之能也。」改大理評事。鹽
 池吏欺緡錢,選賀往解州鉤校出入,賀悉得其奸。



 契丹入寇,真宗決策澶淵,遣使八人省州縣,賀以太子中舍安撫京東。改殿中丞,歷通判明州、宿州。徙知漢州,蜀吏喜弄法,而賀精明,吏不敢欺,事更賀所,多被究詰,人目為「趙家關」,謂如關梁不可越也。



 召權三司戶部判官,真補度支判官,出為京東轉運副使,徙京西。又徙益州路轉運使,尋糾察在京刑獄,累遷尚書工部郎中、提舉諸司庫務,為江、淮制置發運使。發運司占隸三司軍將,分
 部漕船,舊皆由主吏白遣,受賕不平,或數得詣富饒郡,因以商販,貧者至不能堪其役。賀乃籍諸州物產厚薄,分劇易為三等,視其功過自裁定,由是吏巧不得施,歲漕米溢常數一百七十萬。



 蘇州太湖塘岸壞,及並海支渠多湮廢,水侵民田。詔賀與兩浙轉運使徐奭兼領其事,伐石築堤,浚積潦,自吳江東赴海。流民歸占者二萬六千戶,歲出苗租三十萬。遷刑部郎中,歷三司戶部、度支、鹽鐵副使,知延、同、秦三州、江陵府,累遷光祿卿,入
 判大理寺,以右諫議大夫知永興軍,徙鄧州,歲餘,判宗正寺,出知越州。坐失舉,降知濠州,改廬州。遷給事中,復判宗正寺,知鄭、蔡、壽三州,卒。



 在臨朐時,用轉運使李中庸薦改官。中庸沒,無子,賀為主葬,圖其象,歲時祠於家。子宗道,終集賢校理。



 高覿,字會之,宿州蘄人。進士起家,為嘉興縣主簿。後以孫奭薦,改秘書省著作佐郎,累遷尚書屯田員外郎、通判泗州。詔定淮南場茶法,覿陳說利害,不報。擢提點利
 州路刑獄,召為三司戶部判官,安撫河北。還,為京西轉運使。徙益州。彭州廣磧、麗水二峽地出金,宦者挾富人請置場,募人夫採取之。覿曰:「聚眾山谷間,與夷獠雜處,非遠方所宜,且得不償失。」奏罷之。王蒙正恃章獻太后親,多占田嘉州,詔勿收賦,覿又極論其不可。坐失察嘉州守張約受賕,貶通判杭州,徙知福州。入為三司鹽鐵判官,歷陜西、河北轉運使,累遷兵部郎中,復入戶部、鹽鐵為副使,遷右諫議大夫、河東都轉運使,加集賢院學
 士,判尚書刑部,進給事中、知單州,卒。



 子秉常,為梓州路轉運使。



 袁抗,字立之,洪州南昌人。舉進士,得同學究出身,調陽朔縣主簿,薦補桂州司法參軍。撫水蠻寇融州,轉運使俞獻可檄抗權融州推官,督兵糧與謀軍事。蠻治舟且至,抗即楊梅、石門兩隘建水柵二,據其沖,賊不得入,後因置戍不廢。事平,特遷衡州推官,改大理寺丞,累遷國子博士、知南安軍,擢提點廣南東路刑獄。浙東叛卒鄂
 鄰鈔閩、越,轉南海,與廣州兵逆戰海中。值大風,有告鄰溺死者,抗獨曰:「是日風勢趣占城,鄰未必死。」後果得鄰於占城。



 還為度支三司判官,以尚書金部員外郎為梓州路轉運使,徙益州路。時三司歲市上供綾錦、鹿胎萬二千匹,抗言:「蜀民困憊,願少紓其力,以備秦中他日之用。」是年郊祀,蠲其數之半。黎州歲售蠻馬,詔擇不任戰者卻之。抗奏:「朝廷與蠻夷互市,非所以取利也。今山前後五部落仰此為衣食,一旦失利侵侮,不知費直幾馬
 也。臣念蜀久安,不敢奉詔。」尋如舊制。除江、淮發運使,召為三司鹽鐵副使。時抗老矣,為御史所劾,罷知宣州。累遷光祿少卿,分司南京。明堂覃恩,改少府監,卒。



 抗喜藏書,至萬卷,江西士大夫家鮮及也。抗子陟,少刻厲好學,善為詩,終殿中丞。



 徐起,字豫之,濮州鄄城人。舉進士,試秘書省校書郎、知隰川縣,積官尚書都官員外、知楚州。樞密直學士張宗象薦之,擢提點廣南西路刑獄。入判三司開拆司,歷開
 封、三司度支判官。館伴契丹使,還奏:「所過州縣,使者既去,官吏將校皆出郊旅賀,燕飲久之,城邑為之空。」乃下約束禁止之。出為荊湖北路轉運使,部有戍卒殺人系獄,其徒欲劫之。起聞,亟往按誅之,分其徒隸他州。



 徙江西,知徐州,就為轉運使。募富室得米十七萬斛,賑餓殍,又移粟以贍河北、京西者,凡三百萬。與安撫使劉夔不相能,徙京西。又徙江東,起請開長淮舊浦,以便漕運。知洪州,徙兗州。有都巡檢虐所部,而部兵百餘人持兵至
 庭下。州人大恐,起不為動,以禍福開諭之,眾感泣聽命。因按致其首,奏罷都巡檢。復為度支判官,累遷秘書監、知湖州,卒。



 張旨,字仲微,懷州河內人。父延嘉,頗讀書,不願仕,州上其行,賜號嵩山處士。旨進保定軍司法參軍,上書轉運使鐘離瑾,願補一縣尉,捕劇賊以自效。瑾壯其請,為奏徙安平尉,前後捕盜二百餘人。嘗與賊鬥,流矢中臂,不顧,猶手殺數十人。擢試秘書省校書郎、知遂城縣,遷著
 作佐郎。



 明道中,淮南饑,自詣宰相陳救御之策。命知安豐縣,大募富民輸粟,以給餓者。既而浚卑河三十里,疏洩支流注芍陂,為斗門,溉田數萬頃,外築堤以備水患。再遷太常博士、知尉氏縣,徙通判忻州。



 元昊反,特遷尚書屯田員外郎、通判府州。州依山無外城,旨將築之,州將曰:「吾州據險,敵必不來。」旨不聽。城垂就,寇大至,乃聯巨木補其罅,守以強弩。中外不相聞者累日,人心震恐。庫有雜彩數千段,旨矯詔賜守城卒,卒皆東望呼萬歲,
 賊疑以救至也。州無井,民取河水以飲,賊斷其路。旨夜開門,率兵擊賊小卻,以官軍壁兩旁,使民出汲。復以渠泥覆積草,賊望見,以為水有餘。督居民乘城力戰,賊死傷者眾,隨解去。以功遷都官員外郎,徙知萊州。



 葉清臣舉材堪將帥,召對,改知邢州,擢提點河東路刑獄。範仲淹、歐陽修復言其鷙武有謀略,除閣門使,固辭。進工部郎中、知鳳翔府,加直史館、知梓州,以直龍圖閣知荊南。入判尚書刑部,累遷光祿卿,知潞、晉二州。以老疾,權判
 西京御史臺,尋卒。



 齊廓,字公闢,越州會稽人。舉進士第,自梧州推官累遷太常博士、知審刑詳議官,知通、泰州。提點荊湖南路刑獄。潭州鞫系囚七人為強盜,當論死。廓訊得其狀非強,付州使劾正,乃悉免死。平陽縣自馬氏時稅民丁錢,歲輸銀二萬八千兩,民生子,至壯不敢束發,廓奏蠲除之。歷三司度支、開封府判官,出為江西、淮南轉運使。時初兼按察,同時奉使者,競為苛刻邀聲名,獨廓奉法如平
 時,人以為長厚。入判鹽鐵勾院,加史館、知荊南府,徙明、舒、湖三州,積官光祿卿、直秘閣,以疾分司南京,改秘書監,卒。



 廓寬柔恭謹,人犯之不校。弟唐,為吉州司理參軍,博覽強記,嘗舉賢良方正,對策入等。越州蔣堂奏廓及唐父母垂老,窮居鄉里,二子委而之官,唐復久不歸省,於是罷唐,令歸侍養。廓方使湖南,雖置不問,然士論薄之。



 鄭驤,字士龍,河南人。登進士第,更慶、汝、鄭、秦州推官,改
 秘書省著作郎、知垣曲縣。康繼英闢簽書衛州判官事,劉從德代繼英,又表驤有善狀,進一官。尋監左藏庫,遷太常博士、知幹州,提點益州路刑獄,為三司度支判官。建言:「蜀人引江水溉田,率有禁,歲旱利不均,宜弛其禁。」又言:「京西旱,舊禁粟無出國門,可且勿禁。」



 慶歷中,與魚周詢刺陜西民兵十餘萬。除陜西轉運、按察使兼三門發運使,加直史館、河北轉運使,入為度支副使。河決德州,入王紀口,議欲徙州,詔驤往視之,還言州不當徙,已
 而州果無患。又為河北轉運使。王則反,討平之。除天章閣待制、知鳳翔府。先是,皇甫泌、夏安期皆為轉運使,泌先謫去,安期後至,不及賞,驤因辭不受,願命推功與二人。復為河北都轉運使,累遷尚書工部郎中,以疾知華州,卒。



 論曰:歷觀數子,風跡雖不同,其為政愛民,謙己利物,有古道焉。若旨浚卑河,覿罷採金,抗論互市,起賑窮戢暴,驤推功與人,皆無所愧矣。趙賀不忘李中庸,而齊廓兄
 弟棄親以徇榮,用心何其不同哉!



\end{pinyinscope}