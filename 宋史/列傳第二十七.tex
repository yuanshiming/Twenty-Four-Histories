\article{列傳第二十七}

\begin{pinyinscope}

 柴禹錫張遜楊守一趙鎔周瑩王繼英王顯



 柴禹錫,字玄圭,大名人。少時,有客見之曰:「子質不凡,若輔以經術,必致將相。」禹錫由是留心問學。時太宗居晉
 邸,以善應對,獲給事焉。太平興國初,授供奉官。三年,改翰林副使,遷如京使,仍掌翰林司。每夜直,上以藩府舊僚,多召訪外事。遷宣徽北院使,賜第寶積坊。告秦王廷美陰謀,擢樞密副使。逾年,轉南院使。服勞既久,益加勤敏。



 雍熙中,議廣宮城。禹錫有別業在表識中,請以易官邸,上因是薄之。又與宰相宋琪厚善。會廣州徐休復密奏轉運王延範不軌狀,且言倚附大臣,無敢動搖者。上因訪琪及禹錫曰:「延範何如人?」延範與琪妻為疏屬,甚
 言其忠勤,禹錫亦傍贊之。上意其交通,滋不悅。禹錫又為琪請盧多遜故第,上益惡其朋比。坐琪以詼諧罷相,不欲顯言之也。下詔切責禹錫,以驍衛大將軍出知滄州。在任勤於政治,部民詣濱州列狀以聞。改涪州觀察使,徙澶、鎮二州駐泊部署,俄知潞州,州民乞留三載,詔獎之。徙知永興軍府,再召為宣徽北院使、知樞密院事。



 至道初,制受鎮寧軍節度、知涇州。入謝日,上謂曰:「由宣徽罷者不過防禦使爾,今委卿旌節,兼之重鎮,可謂優
 異矣。」禹錫流涕哽咽而已。咸平中,移知貝州。是歲,契丹兵奄至城下,禹錫內嚴備御,寇尋引去。明年,徙陜州。



 景德初,子宗慶選尚,召禹錫歸闕,令公主就第謁見,行舅姑禮,固辭不許。頃之,還鎮。未幾,卒,年六十二,贈太尉。子宗亮,太子中允;宗慶,永清軍節度。



 張遜,博州高唐人。數歲喪父,養於叔父職方員外郎干,後隨母歸魏仁浦家,駙馬都尉咸信,其異父弟也。太宗在晉邸,召隸帳下。太平興國初,補左班殿直。從征太原
 還,遷文思副使,再遷香藥庫使。嶺南平後,交□止歲入貢,通關市。並海商人遂浮舶販易外國物,甗婆、三佛齊、渤泥、占城諸國亦歲至朝貢,由是犀象、香藥、珍異充溢府庫。遜請於京置榷易署,稍增其價,聽商入金帛市之,恣其販鬻,歲可獲錢五十萬緡,以濟經費。太宗允之,一歲中果得三十萬緡。自是歲有增羨,至五十萬。



 雍熙二年,錄其勞,遷領媯州刺史。三年,與安忠並命為東上閣門使。數月,會許仲宣罷判度支,即以遜為度支使。端拱初,
 遷鹽鐵使。二年,授宣徽北院使、簽署樞密院事。未幾,兼樞密副使、知院事。與同列寇準不協,每奏事,頗相矛盾。



 一日,遜等晚歸私第,準與溫仲舒並轡,有狂民迎馬首拜呼萬歲。街使王賓舊與遜同事晉邸,遜又嘗舉賓,雅相厚善,因奏民迎準拜呼萬歲。準自辯:「實與仲舒同行,蓋遜令賓獨奏斥臣。」辭意俱厲,因互發其私。太宗惡之,下詔切責,遜左降右領軍衛將軍,準亦罷職。會判右金吾街仗蔡玉冒奏富人子為州大校,黜官,命遜代掌
 其事。



 西蜀李順為亂,詔發兵水陸進討,以荊渚居其要害,命遜為右驍衛大將軍、知江陵府,賜錢二百萬,白金三千兩。遜既至,會峽路諸漕卒數千人聚江陵,有告其謀變以應蜀寇,府中議欲盡誅之。遜止捕首惡楊承進等二十一人斬於市,餘黨親加尉撫,飛奏以聞。太宗嘉之,詔以其卒分配州郡。數月,遜卒,年五十六,時至道元年也。贈桂州觀察使,歸葬京師。遜小心謹慎,徒以攀附至貴顯,其籲謀獻替無聞焉。



 子敏中,初補供奉官。遜在宣
 徽,表言嘗業文,願改秩,即換大理寺丞,累至比部郎中。次子虛中,娶宗室申國公女,至供奉官、閣門祗候。敏中子先,進士及第。



 楊守一,字象先,其先河南洛陽人。唐末避亂,徙家宋、鄭間。守一稍通《周易》及《左氏春秋》,事太宗於晉邸。太宗即位,補右班殿直。太平興國中,出護登州兵。召還,監儀鸞司。累遷西頭供奉官,其下多貴族子弟,頗豪縱徼幸。始置三班院,令守一專其事,考核授任,漸有條制。歲餘,改
 翰林學士。守一初名守素,至是詔改之。



 七年,與趙鎔、柴禹錫、相里勛等告秦王廷美陰謀事,擢東上閣門使兼樞密都承旨。八年,改判四方館事。雍熙中,詔護遷雲、朔歸附安慶兵屯於潞州。三年,轉內客省使,仍兼都承旨。端拱元年,授宣徽北院使、簽署樞密院事。是秋,卒,年六十四。贈太尉,中使護葬。



 守一性質直勤謹,無他材術,徒以肇自王府,久事左右,適會時機,故歷職通顯,飾終之禮,率加常數焉。



 子安期歷國子博士,坐事貶卒。安期子
 夢得,進士及第。



 趙鎔,字化鈞,滄州樂陵人。以刀筆事太宗於藩邸,即位,補東頭供奉官。因使吳越賜國信,及錢俶納土,遣檢校帑廩,轉內酒坊副使。以告秦王廷美陰事,遷六宅使,領羅州刺史。掌翰林司,擢東上閣門使。



 郭贄參知政事,鎔以同府之舊,嘗有所請托,贄不從。鎔摭堂吏過失以聞,贄見上,白鎔私謁,即召鎔廷辯。詞屈,出為梓、遂州都巡檢使,改左驍衛大將軍,領郡如故。代還,知滄州兼兵馬
 部署。鎔在郡完城塹,嚴戰具。寇嘗數百騎至境上,聞有備,引去。遷左神武大將軍。會崔翰知州,改鎔為本州鈐轄。



 又知廬州,因對,自陳願留,不許。逾年,召為樞密都承旨,同掌三班,俄拜宣徽北院使、同知樞密院事,與柴禹錫並掌機務。嘗遣吏卒變服,散之京城察事。卒乘醉與賣書人韓玉鬥毆,不勝,因誣玉言涉指斥。禹錫等遽以聞,玉坐抵法。太宗尋知其冤,自後廉事不復聽。禹錫出鎮,鎔加知院事。真宗即位,改南院使、檢校太傅,以心疾
 求解。是秋,授壽州觀察使。咸平元年三月,卒,年五十五。贈忠正軍節度,錄其三子官。



 鎔少涉獵文史,美書翰,委質晉邸,以勤謹被眷。本名容,太宗改為鎔,曰:「陶鎔所以成器也。」鎔性好佛,多蓄古書畫。三子:忠輔,西京左藏庫副使;忠願,虞部員外郎;忠厚,內殿崇班。



 周瑩,瀛州景城人。右領軍衛上將軍景之子也。景家富財,好交結,歷事唐、漢、周。習水利,嘗浚汴口,導鄭州郭西水入中牟渠,修滑州河堤,累遷至是官。



 太宗潛邸時,瑩
 得給事左右。即位,補殿直,領武騎卒巡警泉、福州。卒才數百,捕劇賊千餘,遷供奉官。天雄軍節度孫永祐、轉運使楊緘稱薦之,又使綏、銀州按邊事,還奏稱旨,擢鞍轡庫副使。



 雍熙二年,為杭、睦五州都巡檢使兼杭州都監。會妖僧紹倫為變,瑩擒蕕之,逮捕就戮者三百餘人,人以為酷濫。代還,改崇儀使、滄州都監。召拜西上閣門使,領鎮、定、高陽關都監,加判四方館事。與郝守浚護塞宋州決河,俄改三路排陣鈐轄,歷知天雄軍、真定二府,就
 遷引進使。



 至道二年,代還。會李繼隆討西夏,詔瑩詣軍前,授以機事,還拜客省使,簽書樞密院諸房公事,俄兼提點宣徽諸房、鼓司、登聞院,與劉承珪並任。



 真宗嗣位,承珪分使河北告諭,加領富州刺史。上聞其母老病,閔之,特封武功郡太夫人。秋,拜宣徽北院使。先是,宣徽著位在樞密副使上,瑩表請居下,從之。咸平二年,大閱,命為隨駕部署。從征河朔,又為駕前馬步都部署。



 三年,遷南院使、知樞密院事。會蜀平,部送脅從者數十百人至
 闕下。西川轉運使馬亮因入奏,請赦其罪遣還。瑩以為當盡誅之。令瑩、亮廷議,上是亮議,悉原其罪。



 五年,高陽關都部署闕,蕃侯無足領之者,宰相請輟宣徽使以居其任。時王繼英任北院,上以瑩練達軍事,乃拜永清軍節度,兼領其任,為三路排陣使。瑩隸人有錢仁度者,頗有軍功,與虎翼小校劉斌相競,為殿直閻渥所發。以瑩故,詔勿問,止徙斌隸他軍。契丹入寇,詔步兵赴寧邊軍為援。瑩至,則寇兵已去,即日還屯所。上聞曰:「瑩何不持
 重少留,示以不測。輕於舉措,非將帥體也。」



 景德初,丁內艱,起復,代王顯為天雄軍都部署兼知軍府事。嘗召洺州騎士千五百人赴大名,道與寇直,力戰,有死傷者,瑩猶謂其玩寇,將悉誅之。詔賜金帛,諭瑩勿治其罪。車駕北巡,為駕前東西貝冀路都部署。明年,改知陜州,俄徙永興軍府,又移邠州,兼環慶路都部署。時夏州內屬,詔省戍兵還營,以減饋餉之費。仍手詔諭瑩,瑩遽奏乞留,以張邊威。上謂瑩庸懦不智,以曹瑋代之,徙知澶州。



 大
 中祥符初,改天平軍節度。明年,為鎮定都部署兼知定州。轉運使奏其曠弛,徙知澶州,境內屢有寇盜,宰相以瑩任居將帥,不能以威望鎮靖,請徙他郡。上曰:「處之閑僻,適使其自偷爾。」遂下詔督責,令其擒捕。時發卒修河防,而軍中所給糗糧,多腐敗不可食;又役使不均,瑩不加恤,以故亡命者眾。



 七年,入朝,復遣還鎮。又以澶淵當契丹之沖,藉其廩給之厚,復命知澶州。九年,被疾,求還京師。卒,年六十六,贈侍中。初謚忠穆,後改元惠。錄其
 二子供奉官普、顯為內殿崇班,二孫永昌、永吉為殿直。



 瑩居樞近,無他謀略,及蒞軍旅,歷藩鎮,功業無大過人者。故事,大禮覃慶,外藩無賜物例。東封歲,瑩鎮澶淵,車駕所經,故特有襲衣、金帶、器帛之賜。祀汾陰,瑩知定州,乃預上言:「禮成,所賜望於治所支給。」人咸笑之。普後為崇儀副使,顯至內殿承制。



 王繼英,開封祥符人。少從趙普給筆札,普自罷河陽,為少保,從者皆去,繼英趨事逾謹。普再入相,繼英隸名中
 書五房、院。



 時真宗在藩邸,選為導吏兼內知客事。太宗召見,謂曰:「汝昔事趙普,朕所備知。今奉親賢,尤宜盡節。」及建儲,授左清道率府副率兼左春坊謁者。謁者本宦職,副率品秩頗崇,非趨走左右者所宜為,俾兼領之,執政之誤也。



 真宗即位,擢為引進使。咸平初,領恩州刺史兼掌閣門使,遷左神武大將軍、樞密都承旨,改客省使。契丹入寇,繼英密請車駕北巡,上從之,即命繼英馳傳詣鎮、定、高陽關閱視行宮儲頓,宣諭將士。俄充澶州鈐
 轄。會大將傅潛逗撓得罪,令繼英即軍中召還屬吏。



 尋掌三班,拜宣徽北院使,與周瑩同知樞密院事。瑩出鎮,繼英遂冠樞宥,小心慎靖,以勤敏稱,上倚賴之。



 景德初,授樞密使。舊制,樞密院使祖母及母止封郡太夫人,有詔特加國封。嘗因進補軍校,白上曰:「疏外之人急於攀附者,謂臣蒙蔽不為薦引。」上曰:「此輩雖有夤緣,亦須因事立功,方許擢用,不可過求僥幸,卿勿復言也。」



 從幸澶州,契丹請和,諏訪經略,繼英預焉。明年郊祀,加特進、檢
 校太傅。三年,卒,年六十一。上臨哭之,賜白金五千兩,贈太尉、侍中,謚恭懿。且為葬其祖父,贈其妻賈長樂郡太夫人,錄其子婿、門下親吏數十人。



 初,繼英幼孤,寄育外氏。既貴,外王父、諸舅有族殯者,時方奏遣其子營葬,會卒,特詔有司給辦焉。



 子遵式、遵誨、遵度、遵範,皆至顯宦。



 王顯,字德明,開封人。初為殿前司小吏,太宗居藩,嘗給事左右。性謹介,不好狎,未嘗踐市肆。即位,補殿直,稍遷供奉官。



 太平興國三年,授軍器庫副使,遷尚食使。逾年,
 與郭昭敏並為東上閣門使。八年春,拜宣徽南院使兼樞密副使。是夏,制授樞密使。上謂之曰:「卿世家本儒,少遭亂失學,今典朕機務,無暇博覽群書,能熟《軍戒》三篇,亦可免於面墻矣。」因取是書及道德坊宅一區賜之。



 其後居位既久,機務益繁,副使趙昌言、寇準鋒氣皆銳,慢顯,顯或失誤,護短終不肯改,上每面戒之。淳化三年八月,詔加切責,黜授隨州刺史,充崇信軍節度、觀察等使,遣之任。



 俄知永興軍,徙延州。時夏臺、益部寇擾,顯上疏
 曰:「間歲以來,戎事未息,李繼遷負恩於靈、夏,王小波干紀於巴、邛,河右坤維並興師旅。而繼遷翻然向化,遣子入覲,願修職貢。陛下曲加容納,許其內附,示以德信,伸以恩錫,所以綏懷之者至矣。然而狼子野心,未可深信。所宜謹屯戍,固城壘,積芻糧,然後遴選才勇,付以邊任,縱有緩急,則備御有素,彼又奚能為患哉?至若蜀寇未平,神人共憤,謂宜申飭將帥,速期蕩平,既免老師以費財,且防事久則生變。又況邛、蜀物產殷富,其間士卒
 驕怠,遲留顧戀,實兼有之。莫若勿憚往來,潛為更代,既可均其勞逸,抑可免於遷延。至於河北關防所當加謹者,誠以國家事西南,密謀興舉,若分中朝之勢力,則長外寇之奸謀矣。」



 時制,沿邊糧斛不許過河西,河西青鹽不得過界販鬻,犯者不以多少,處斬。顯請犯多者依法,自餘別為科斷,以差其罪。章上未報,移知秦州。



 初,溫仲舒知州日,開拓山林,諷藩部獻其地。後朝廷雖嘗給還,而採伐如故。轉運使盧知翰請量給蕃部茶彩,以酬所
 獻,詔遣張從式與顯同往規度。顯言:「乃者朝命以趙保吉修貢,邊城務使安靜,若今動眾開斥疆境,非便。」議遂罷。



 咸平初,入朝,改橫海軍節度,出知鎮州。二年,曹彬卒,復拜樞密使。郊祀,加檢校太師。真宗幸大名,內樞惟顯與副使宋湜從,言者多謂顯專司兵要,謀略非長。會湜卒,乃以參知政事向敏中權同知樞密院事。三年春,改授山南東道節度、同中書門下平章事、定州路行營都部署、河北都轉運使兼知定州。秋,吏民詣駐泊都部署
 孔守正言顯治狀,願借留。守正以聞。明年秋,加鎮、定、高陽關三路都部署,許便宜從事。十月,契丹入寇,前軍過威虜軍。比時方積雨,契丹以皮為弦,濕緩不堪用,顯因大破之,梟獲名王、貴將十五人及羽林印二鈕,斬首二萬級。顯上言:「先奉詔令於近邊布陣,及應援北平控扼之路。無何,敵騎已越亭障,顯之前陣雖有捷克,終違詔命。」上章請罪。上降手札,以慰其憂悸。



 明年,求致仕,不許,改河陽三城節度。將之鎮,時議親征契丹,顯言:「盛寒在
 序,敵未犯塞,鑾輿輕舉,直抵窮邊,寇若不逢,師乃先老。況今繼遷未滅,西鄙不寧,儻北邊部落,與之結援,則中國之患,未可量也。議者乃於此時請復幽薊,非計之得也。凡建議大事,上下協力,舉必成功。今公卿士大夫以至庶人,尚有異同,未可謂為萬全之舉。若能選擇將帥,訓練士卒,堅城壘而繕甲兵,亦足以待敵矣。必欲復燕、薊舊地,則必修文德、養勇銳,同時之利,以奉行天罰而後可。」



 景德初,徙知天雄軍府。又言:「祖宗以來,多命近臣
 統領軍旅。今後宣徽使,宜於文武群臣中擇曉達邊事者為之。蓋位高則威名著,識遠則勛勞立故也。武臣以罪黜者,宜加容貸,不以一眚遂廢,茍用之有恩,必得其死力,故曰使功不如使過也。至若臨敵命將,則貴專任,出師應敵,則約束將校,使相應援。全是數者,則軍威倍壯,人心增勇矣。」既而上表請赴行在,從之。是年秋,遣還鎮。



 契丹入寇,上議親征。顯復陳三策,謂:「大軍方在鎮定,契丹必不南侵,車駕止駐澶淵,詔鎮定出兵,會河南
 軍,合擊之可也。若契丹母子虛張聲勢,以抗我師,潛遣銳兵南攻駕前諸軍,則令鎮定之師直沖戎帳,攻其營砦,則沿河游兵不戰而自屈矣。否則遣騎兵千、步兵三千於濮州渡河,橫掠澶州,繼以大軍追北掩擊,亦可出其不意也。」已而契丹請盟,趙德明遣使修貢稱藩,朝廷加賞錫,且許通青鹽以濟邊民,從顯之請也。



 三年冬,被病,詔中使偕尚醫療視。明年正月,許還京師。時車駕上陵,顯謂賓佐曰:「餘年位偕極,今天子道出虎牢,不得一
 拜屬車之塵,是遺恨也。」言訖涕下,至京,信宿卒,年七十六。車駕至鄭州,聞之,遣宮苑使鄧永遷馳還護喪,贈中書令,謚忠肅。錄其二子。



 顯自三班不數年正樞任,獎擢之速,時無擬之者。顯吏軍司時,張永德以滑州節制為殿前都點檢。及顯自樞密鎮孟津兼相帥,永德由太子太師為相帥,同日宣制,永德兼大夫反在顯下,時人訝之。顯居中執政,矯情以厚胥吏,齪齪自固而已。在藩鎮,頗縱部曲擾下,論者非之。



 子希逸字仲莊,以蔭補供奉
 官。好學,尤熟唐史,聚書萬餘卷。換秩授朝奉大夫、太子中允。咸平初,改殿中丞、直史館,預修《冊府元龜》,加祠部員外郎,卒。希範至如京副使。



 論曰:自柴禹錫而下,率因給事藩邸,以攀附致通顯者凡七人。若守一之質直,趙鎔之勤謹,服勞雖久而益修乃職,則其被眷遇也宜矣。張遜優於理財而未免於媢嫉,周瑩練習軍旅而頗傷於酷濫,禹錫素稱勤敏而不能不涉於朋比,王顯雖謹介自將而昧於學識,故莫逃
 於齪齪之譏。若以勤謹被信任,耆德冠樞宥,而善終如始者,其惟繼英乎。《易》曰:「君子有終,吉。」此之謂也



\end{pinyinscope}