\article{列傳第一百二十八}

\begin{pinyinscope}

 張俊從子子蓋張宗顏
 劉光世王淵解元曲端



 張俊,字伯英,鳳翔府成紀人。好騎射,負才氣。起於諸盜,年十六,為三陽弓箭手。政和七年,從討南蠻,轉都指揮
 使。宣和初,從攻夏人仁多泉,始授承信郎。平鄆州賊李太及河朔、山東武胡群寇,功最,進武德郎。



 靖康元年,以守東明縣功,轉武功大夫。金人攻太原,城守,命制置副使種師中往援,屯榆次。金人以數萬騎壓之。俊時為隊將,進擊,殺傷甚眾,獲馬千匹,請乘勝要戰。師中以日不利,急令退保。金人諜俊計不行,悉兵合圍,攻益急。榆次破,師中死之。俊與所部數百人突圍而出,且行且戰,至烏河川,再與敵遇,斬五百級。



 金人圍汴京,高宗時為兵馬
 大元帥,俊勒兵從信德守臣梁揚祖勤王。高宗見俊英偉,擢元帥府後軍統制,累功轉榮州刺史。建炎元年正月,從高宗至東平府。時劇賊李昱據兗州,命俊為都統制討之。與數騎突圍撓戰,諸軍爭奮,賊遂殲。進桂州團練使,尋加貴州防禦使。



 中書舍人張澄,自汴京繼蠟詔,命高宗以兵付副帥還京,高宗問大計,俊曰:「此金人詐謀爾。今大王居外,此天授,豈可徒往?」因請進兵,高宗許之,遂如濟州。



 開啟乾龍節,迫夜,有告高宗,欲俟元帥謁
 香劫以叛。群議集諸軍屯備,俊曰:「元帥不出,奸謀自破。」遂徙州治。賊術窮,黎明,引軍北遁,俊勒兵追殺之。進徐州觀察使。



 高宗以俊忠勞日積,遷拱衛大夫。既而汴京破,二帝北遷,人心皇皇,俊懇辭勸進,高宗涕泣不許。俊曰:「大王皇帝親弟,人心所歸,當天下洶洶,不早正大位,無以稱人望。」且白耿南仲奏之,表三上。高宗發濟州,俊便道扈行。至應天府,高宗始即位。初置御營司,以俊為御營前軍統制,遣還京迎隆祐太后。權秦鳳兵馬鈐轄。
 尋奉太后及六宮以歸,除帶御器械。



 時江、淮群盜蜂起,俊討杜用於淮寧,趙萬、郭青於鎮江,陳通於杭州,蔣和尚等於蘭溪,皆平之。落階官,除正任觀察使。二年,升秦鳳路馬步軍副總管,尋破秀州賊數萬,縛徐明斬之。進武寧軍承宣使。



 帝如揚州,召諸將議恢復,俊曰:「今敵勢方張,宜且南渡,據江為險;練兵政,安人心,俟國勢定,大舉未晚。」俊又請移左藏庫於鎮江。既而敵掩至,已逼近甸,俊亟奏飭甲乘,從帝如臨安。



 苗傅、劉正彥反,俊時屯
 兵吳江縣。傅等矯詔加俊捧日、天武四廂都指揮使,以三百人赴秦鳳,命他將領餘兵。俊知其偽,拒不受。三軍洶洶,俊諭之曰:「當詣張侍郎求決。」即引所部八千人至平江。張浚語俊以傅等欲危社稷,泣數行下,俊大慟。浚諭以決策起兵問罪,俊泣拜,且曰:「此須侍郎濟以機術,毋驚動乘輿。呂頤浩至,俊見之,亦涕泣曰:「今日惟以一死報國。」劉光世以所部至,俊釋舊憾。韓世忠來自海上,俊借一軍與之俱。世忠為前軍,俊以精兵翼之,光世次
 之。戰於臨平,傅等兵敗,開城以出。世忠、俊、光世入城,見於內殿,帝嘉勞久之,拜鎮西軍節度使、御前右軍都統制,尋為浙東制置使。



 金人分兵深入,渡江攻浙,杜充棄建康,韓世忠自鎮江退保江陰。帝如明州,俊自越州引兵至。兀朮攻臨安,帝御樓船如溫州,留俊於明州以拒敵。帝賜親札曰:「朕非卿,則倡義誰先;卿舍朕,則前功俱廢。宜戮力共捍敵兵,一戰成功,當封王爵。」癸卯除夕,金兵至城下,俊使統制劉寶與戰,兵少卻,其將黨用、丘橫
 死之,於是統制楊沂中、田師中、統領趙密皆殊死戰。沂中舍舟登岸力戰,殿帥李質以班直來助,守臣劉洪道率州兵射其旁,大破之,殺數千人。金呼人至砦計事,俊令小校往。金人與語,欲如越州請降,俊拒之。戒將士毋驕惰,慮敵必再至,下令清野,多以輕舟伏弩,閉關自守。



 四年正旦,忽西風起,金人乘之,果復攻明州。俊與劉洪道坐城樓上,遣兵掩擊,殺傷大當。金人奔北,死於江者無數,夜拔砦去,屯餘姚,且請濟師於兀朮。後七日,敵再
 至,俊引兵趨入臺州,明州居民去者十七八。



 未幾,江浙群盜蜂起,授俊兩浙西路、江南東路制置使,以所部招收群盜,命後軍統制陳思恭隸之,且令兩浙宣撫使周望以兵屬俊,劉光世、韓世忠之外,諸將皆受節度。六月,改御前五軍為神武軍,俊即本軍為神武右軍都統制,除檢校少保、定江昭慶軍節度使。十月,浙西群盜悉平,改江南招討使。



 紹興元年,帝至會稽。時金人殘亂之餘,孔彥舟據武陵,張用據襄漢;李成尤悍,強據江、淮、湖湘
 十餘州,連兵數萬,有席卷東南意,多造符讖蠱惑中外,圍江州久未解,時方患之。範宗尹請遣將致討,俊慨然請行,遂改江、淮路招討使。



 成黨馬進在筠州。豫章介江、筠之間,俊聞命就道,急趨豫章,且曰:「我已得洪州,破賊決矣。」乃斂兵,若無人者,金鼓不動,令將士登城者斬。居月餘,進以大書牒來索戰,俊以細書狀報之,賊以俊為怯。俊諜知賊怠,乃議戰。岳飛為先鋒,楊沂中由上流徑絕生米渡,出賊不意,追奔七十里,至筠州。賊背筠河而
 陣,俊用楊沂中計,親以步兵當其前,精騎數千授沂中及陳思恭,俾從山後夾擊,以午為期。俊與賊鏖戰至午,精騎自山馳下,賊駭亂退走,大敗。



 既復筠州、臨江軍,捷奏,帝賜御筆,謂:「宜乘賊勢已衰,當官軍已振,驅除剿戮,速收全功。」俊未拜親詔,已追至北奉新樓子莊。賊黨商元據草山,挾險設伏,俊遣步兵從間道直趨山椒,殺伏奪險,乘勝追至江州。成勢迫,絕江而遁,號俊為「張鐵山」。復江州。已而興國軍等處群盜聞俊兵至,皆遁去。俊引
 兵渡江至黃梅縣,親與成戰。成懲奉新失險之敗,據石矢坡,憑山以木石投人。俊先遣游卒進退,若爭險狀以誑賊,俊親冒矢石,帥眾攻險,賊眾數萬俱潰,馬進為追兵所殺,成北走降劉豫,諸郡悉平。拜太尉。



 四年十月,金人與劉豫分道入侵。先是諜至,舉朝震恐,或請他幸。俊謂趙鼎曰:「避將何之?惟向前進一步,庶可脫。當聚天下兵守平江,徐為計。」鼎曰:「公言避非策,是也;以天下兵守一州,非也。公但堅前議足矣。」遂以俊為兩浙西路、江南
 東路宣撫使,屯建康。既而改淮西宣撫使。瀕江相距逾月,敵不得入。俊遣張宗顏潛渡至六合,出其背。敵將引去,俊繼遣王進曰:「敵既無留心,必徑渡淮去,可速及其未濟擊之。」進往,敵果北渡,遂薄諸淮,大敗之,獲其酋程師回、張延壽以獻。



 五年,劉麟入寇,俊與楊沂中合兵拒於泗州。六年,改崇信、奉寧軍節度使。劉麟兵十餘萬犯濠、壽,詔並以淮西屬俊,楊存中亦聽節制,與俊合兵拒敵。俊分遣存中與張宗顏、王瑋、田師中等,自定遠軍次
 越家坊,遇劉猊左右軍,擊走之。俊率大軍鼓行而前,至李家灣遇猊大兵,與戰,殺獲略盡,降者萬餘人,猊僅以身免。拜少保,加鎮洮、崇信、奉寧軍節度使。帝曰:「卿議論持重,深達敵情;兼聞挽強之士數萬,報國如此,朕復何慮。」又曰:「群臣謂朕待卿獨厚,其仰體眷懷,益思勉勵。」



 七年,改淮南西路宣撫使,置司盱眙。俊與韓世忠入見,議移屯。秦檜奏:「臣嘗語世忠、俊,陛下倚此二大將,譬如兩虎,固當各守藩籬,使寇不敢近。」帝曰:「正如左右手,豈可
 一手不盡力邪?」命俊自盱眙屯廬州。八年,金人請寢兵,許之。賜俊「安民靖難功臣」,拜少傅。



 九年冬,金復渝盟,再破河南,圖順昌府,命俊策應劉錡。俊督軍渡江,金人引退。繼而金人三路都統自東、南兩京分道來侵,抵亳州北渡河,俊收宿、亳諸軍擊之,盡復衛真、鹿邑等地,師還。十年,酈瓊在亳州,俊以大軍至城父,都統制王德下符離,乘勝趨亳與俊合。俊引軍入城,金人棄城遁,父老列香花迎俊,遂復亳州,留統制宋超守之。俊引軍還壽春,
 進少師,封濟國公。



 十一年二月,兀朮入合肥,漸攻歷陽,江東制置大使葉夢得見俊,請速出軍。俊遣兵渡江,諭諸將曰:「先得和州者勝。」王德願為諸軍先,士鼓噪而行。敵已據之,德率眾渡採石先登,俊宿中流。德抵城下,金人退屯昭關。後三日,覆敗金將韓常於含山。命關師古復巢縣,遂復昭關。使左軍統制趙密偃兵篁竹,出六丈河以分金勢。張守忠以五百騎敗金人於全椒。未幾,敵斷石梁以拒俊,俊疾作,力疾引眾涉流登岸,追擊之。王
 德與楊存中、劉錡會兵,敗金人於柘皋。拜樞密使。俊知朝廷欲罷兵,首請納所統兵。議賞宿、亳功,俊部將王德、田師中、劉寶、李橫、馬立、張澥六人同日首受上賞。



 俊力贊和議,與秦檜意合,言無不從。薦士大夫監司、郡守者甚眾,雖劉子羽自謫籍起家,亦俊力也。加太傅,封廣國公,尋進益國公。十二年十一月,以殿中侍御史江邈論之,罷為鎮洮、寧武、奉寧軍節度使,充醴泉觀使。初,檜以俊助和議,德之,故盡罷諸將,以兵權付俊。歲餘,俊無去
 意,故檜使邈攻之。尋進封清河郡王,奉朝請。



 十三年,敕修甲第,遣中使就第賜宴,侑以教坊樂部。十六年,改鎮靜江、寧武、靜海軍。二十一年冬,帝幸其第,拜太師,以其侄清海軍承宣使子蓋為安德軍節度使,其它子弟遷秩者十三人。



 南渡後,俊握兵最早,屢立戰功,與韓世忠、劉錡、岳飛並為名將,世稱張、韓、劉、岳。然濠、壽之役,俊與錡有隙,獨以楊沂中為腹心,故有濠梁之劫。岳飛冤獄,韓世忠救之,俊獨助檜成其事,心術之殊也,遠哉!帝於
 諸將中眷俊特厚,然警敕之者不絕口。自淮西入見,則教其讀《郭子儀傳》;召入禁中,戒以毋與民爭利,毋興土木。



 二十四年六月薨,年六十九。輟視朝三日,斂以一品服,帝臨奠哭之慟。追封循王。子五人:子琦、子厚、子顏、子正、正子仁。



 子蓋字德高。父宏,應募從俊軍河上。金人破開德府,宏戰死。子蓋初從韓世忠討苗傅,補承信郎,累功遷武功郎。



 紹興六年,劉猊大舉入寇,過定遠縣,將趨宣化窺淮,
 詔遣俊會劉光世軍剿之。子蓋從俊擊猊於藕塘,授閣門宣贊舍人。明年,改昌州刺史、江南東路馬步軍都總管。十年,金人再取河南,以興復宿、亳功,遷登州防禦使兼宣撫司衙兵副統制。



 十一年二月,兀朮入廬州,攻含山縣,漸攻歷陽。俊遣兵渡江,子蓋從王德馳入和州,金人退屯昭關。會劉錡自東關引兵出清溪邀擊金人,俊遣子蓋與錡會,大戰於柘皋,敗之,軍勢赫張。兀朮復攻濠州,子蓋又敗之於周梁橋,除興寧軍承宣使。
 和議成,改建康府駐札御前諸軍都統制。十三年,授龍神衛四廂都指揮使、兩浙西路馬步軍都總管。帝幸俊第,授子蓋安德軍節度使。



 三十二年春,金人攻海州急,以子蓋為鎮江府都統往援之,即日渡江,馳至楚州。淮東漕臣龔濤謂之曰:「敵眾十倍,兵力不支,宜張虛聲攻淮陽,使之必救,則海州可解。」子蓋曰:「彼若不救,將如之何?」乃亟趨漣水,取便道以進。次石湫堰,金人陳萬騎於河東,子蓋率精銳數千騎擊之,謂麾下曰:「彼眾我寡,利在速戰。」
 遣統制張□略陣,□中流矢,子蓋曰:「事急矣!」奮臂大呼,馳入陣,諸將繼之殊死戰。賊大敗,擁溺石湫河死者半,圍遂解。金人復整軍來戰,子蓋再率精銳擊之,獲其車馬、鎧仗萬計,退屯泗州。



 孝宗即位,召對,賜鞍馬、鎧甲、束帶,且令招集勇敢,相時而動。子蓋受命還,招金大將蕭鷓巴、耶律造哩將其眾來降。尋以疾還鎮江,授檢校少保、淮東招撫使,未上,卒,年五十一。贈太尉,謚恭壯。



 子蓋從俊征討藕塘、柘皋,雖多奏功,未能出諸將右,惟海州
 一捷可稱云。



 張宗顏,字希賢,延安人。父吉,為涇原將,解宣威城圍,死之。宗顏以父恩補三班借職,監閿鄉酒稅,積官至涇原副將、權殿前司統轄。御營軍統制張俊選為統領,從俊討浙西寇。秀州軍校徐明以城叛,宗顏夜襲其城,明遁。轉忠州刺史,遷御前中軍統制。



 金人攻明州,宗顏破其前軍。盜楊勍破松溪,命宗顏及李捧、陳思恭討之。宗顏次浦城不進,勍又掠建州。宗顏趨南劍州,與勍遇,遂歸。
 盜猶未平,謬言已擊退。侍御史沈與求劾宗顏三將並出,不能平數千之潰卒,何以示敵。貶二秩。從俊討李成,與成將馬進戰玉隆觀,敗之。遷環慶路馬步軍副總管、神武右軍統制,改麟州觀察使。



 偽齊挾金人攻宣化鎮,俊遣宗顏潛渡江,出其後襲之,不勝。俊庇之,以捷聞,遂加沂州防禦使。繼以兵襲擊淮北,復遷崇信軍承宣使、宣撫司前軍統制。偽齊入寇,詔張俊解淮西急。督府張浚遣楊沂中與俊合,檄宗顏自泗州為後繼。與猊遇於
 李家灣,大破之,橫尸滿野,猊僅以身遁。擢龍神衛四廂都指揮使、武信軍承宣使。



 八年,知廬州,總帥事。敵數百騎抵城下,宗顏以騎百餘御之,敵退。有至自淮北者,傳金人言曰:「此張鐵山弟也。」紹興九年卒,年四十四。贈保靜軍節度使,謚壯敏。



 劉光世,字平叔,保安軍人,延慶次子。初以蔭補三班奉職,累升鄜延路兵馬都監、蘄州防禦使。方臘反,延慶為宣撫司都統,遣光世自將一軍趨衢、婺,出其不意破之。
 賊平,授耀州觀察使,升鄜延路兵馬鈐轄。



 時有事燕洑,光世從延慶取易州,授奉國軍承宣使。金將郭藥師降,除威武、奉寧軍承宣使。延慶遣諸將搗虛趨燕,以光世為後繼。光世不至,諸將失援而潰,降三官。



 河北賊張迪掠浚州境,詔光世討之。光世曰:「賊烏合,非有紀律,佯北以邀之,其亂可取也。」即麾騎退。賊競進,光世引騎貫其中,賊大潰。復承宣使,充鄜延路馬步軍副總管。



 靖康元年,金兵攻汴京,夏人乘間寇杏子堡。堡有兩山對峙,地
 險厄,光世據之,敵至敗去。擢侍衛馬軍都虞候。金再攻汴京,光世入援,聞範致虛傳檄諸路,議引兵會之。會有詔止勤王兵,光世以為宜速進,不可以詔示眾。既而潰兵至,具言京城事。眾懼,光世矯以蕃官來自汴京,謂二帝決圍南去,眾稍安,進屯陜府。致虛欲合五路兵進與金戰,光世難之,別道趨虢,遂至濟州謁康王,命為五軍都提舉。



 王即皇帝位,命為省視陵寢使,尋為提舉御營使司一行事務、行在都巡檢使。斬山東賊李昱,遷奉國
 軍節度使。平鎮江叛兵,改滁濠太平州、無為軍、江寧府制置使。討張遇於池州,遇望其陣曰:「官軍不整,可破也。」時湖水涸,賊越湖出官軍後,官軍亂,光世幾被執,王德救之得免。遇循江而上,光世整兵追至江州,斷其後軍破之。遇復東下,又追擊於江寧。



 二年,以功加檢校少保,命討李成。光世以王德為先鋒,與成遇於上蔡驛口橋,敗之。成收散卒再戰,光世以儒服臨軍,成遙見白袍青蓋,並兵圍之,德潰圍拔光世以出。下令得成者以其官
 爵與之。士爭奮,再戰皆捷,成遁,執其謀主陶子思。加檢校少傅。



 帝在揚州,金騎掩至天長,光世迎敵,未至而軍潰。帝倉卒渡江,命光世為行在五軍制置使,屯鎮江府,控扼江口。尋加檢校太保、殿前都指揮使。



 苗、劉為亂,素憚光世,遷光世為太尉、淮南制置使。張浚在平江,馳書諭以勤王,光世不從;呂頤浩遣使至鎮江說之,乃引兵會於丹陽。兵進,光世以選卒為游擊,仍分軍殿後,遇苗翊、馬柔吉軍於臨平,與韓世忠等破之。至行在,遷太尉、
 御營副使。光世遣王德助喬仲福追傅至崇安縣,盡降其眾,傅僅以身免。逆將範瓊被執,張浚使光世撫定其眾,又招賊靳賽降之。命光世為江東宣撫使,守太平及池州,受杜充節制。光世言受充節制有不可者六,帝怒,詔毋入光世殿門,光世始受命。



 隆祐太后在南昌,議者謂金人自蘄、黃渡江,陸行二百里可至,命光世移屯江州為屏蔽。光世既至,日置酒高會。金人自黃州渡江,凡三日,無知之者。比金人至,遂遁,太后退保虔州。馮楫貽
 書光世,言:「賊深入,最兵家之忌。進則距山,退則背江,百無一利,而敢如此橫行者,以前無抗拒,後無襲逐也。太尉儻選精兵自將來洪,而開一路令歸,伏兵掩之,可使匹馬不還。」光世不能用,自信州引兵至南康。酈瓊圍固始縣,光世遣人招降之,又遣王德擒妖賊王念經於信州。



 時光世部曲無所隸,號「太尉兵」,侍御史沈與求論其非宜。會御營司廢,乃以「巡衛」名其軍,命充御前巡衛軍都統制。召赴行在,授浙西安撫大使、知鎮江府。光世言:「
 安撫控制一路,若但守鎮江,則他郡有警,不可離任。望別除守臣,光世專充安撫使,從便置司。」時光世慮金人必過江,故預擇便地,帝覺之,止許增闢通判。右諫議大夫黎確疏其擇便求佚,中外所憤,帝釋不問,加寧武軍節度使、開府儀同三司以遣之。光世乞便宜行事,不許。時韓世忠、張俊兼領浙西制置使,光世復言本路兵火之餘,不任三處需求,遂罷世忠、俊兼領。



 時金兵留淮東,光世頗畏其鋒,楚州被圍已百日,帝手札趣光世援楚
 者五,竟不行;但遣王德、酈瓊將輕兵以出,時奏殺獲而已。楚州破,命光世節制諸鎮,力守通、泰。完顏昌屯承、楚,光世知其眾思歸,欲攜貳之。乃鑄金銀銅錢,文曰「招納信寶」。獲敵不殺,令持錢文示其徒,有欲歸者,扣江執錢為信。歸者不絕,因創「奇兵」、「赤心」兩軍,昌遂拔砦去。



 紹興元年,金人渡淮,真、揚州皆闕守,命光世兼淮南、京東路宣撫使,置司揚州,措置屯田,迄不行。張俊討李成,又命光世分兵往舒、蘄搗其巢穴,光世以江北盜未平為辭。
 命兼淮南宣撫使,領真揚通承楚州、漣水軍。郭仲威謀據淮南以通劉豫,光世遣王德擒之,並其眾。範宗尹言:「光世軍多冗費,請汰其罷軟者。」帝曰:「俟作手書與之,如家人禮,庶幾不疑。」



 光世以枯秸生穗為瑞,聞於朝。帝曰:「歲豐人不乏食,朝得賢輔佐,軍有十萬鐵騎,乃可為瑞,此外不足信。」淮北人多歸附者,命光世兼海、泗宣撫使以安輯之。五湖捕魚人夏寧聚眾千餘,掠人為食,郭仲威餘黨出沒淮南,邵青據通州,光世皆招降之。光世請
 鑄淮東宣撫使印,給錢糧,增將吏,皆從其請。仍給鎮江府、常州、江陰軍苗米三十七萬斛,為軍中一歲費。



 二年,復命移屯揚州,時至鎮江視師。光世不奉詔,入朝言:鄰寇有疑,或致生事,願仍領浙西為根本計。右司諫方孟卿劾之,乞召宰執與議,使之必往,光世猶以乏糧為辭。光世之來,以繒帛、方物為獻,帝命分賜六宮,中丞沉與求以為不可,命還之。



 呂頤浩與光世有故怨,頤浩將出視師,首言光世兵冗不練,乞移其軍還闕。帝曰:「光世軍
 糧不足,若驟移,必潰,先犒軍而後料簡可也。」頤浩至鎮江,光世軍果告乏,頤浩奏光世軍月費二千萬緡,乞差官考核。詔御史江躋、度支胡蒙至軍點校,終不得實。帝方倚其成功,尋詔兩漕臣措置鎮江酒稅務,助其軍費;又罷織御服羅,省七百萬緡以助之。加寧武、寧國軍節度使。光世奏部將喬仲福、靳賽防江有勞,詔進一官,許回授。



 光世固乞轉行,給事中程瑀持不可,又言光世兵未渡江,金人或渡淮,江、浙必震。光世方遣人按行宜興
 湖洑之間,以備退保。詔以章示之,光世遷延如故。



 三年,命光世與韓世忠易鎮,同召赴闕,授檢校太傅、江東宣撫使。世忠既至鎮江城下,奸人入城焚府庫,光世擒之,皆云世忠所遣。世忠屯登雲門,光世引兵出,懼其扼己,改途趨白鷺店。世忠遣兵襲其後,光世以聞。帝遣使和解,仍書《賈復》、《寇恂傳》賜之。命為江東、淮西宣撫使,置司池州,賜錢十萬緡。



 劉豫將王彥先揚兵淮上,有渡江意。光世扼馬家渡,遣酈瓊屯無為軍,為濠、廬援,賊乃退。光
 世奏鄜延李佾充閣門祗候,言者論其涉私,罷之。金人、劉豫入侵,時光世、張俊、韓世忠權相敵,且持私隙,帝遣侍御史魏矼至軍中,諭以滅怨報國。光世乃移書二帥,二帥皆復書致情。光世始移軍太平州以援世忠。金兵退,光世入覲,遷少保。帝曰:「卿與世忠以少嫌不釋,然烈士當以氣義相許,先國家而後私仇。」復諭以光武分寇恂、賈復之事。光世泣謝,請以所置淮東田易淮西田,給事中晏敦復言其擾民而止;又請並封其三妾為孺人,
 南渡後,諸大將封妾自此始。會改神武軍為行營護軍,以光世所部稱左護軍。劉豫築劉龍城以窺淮西,光世遣王師晟破之,加保靜軍節度使,遂領三鎮。



 張浚撫淮上諸屯,劉豫挾金人分道入侵,命光世屯廬州以招北軍,與韓世忠、張俊鼎立,楊沂中將精卒為後距。劉猊驅鄉民偽為金兵,布淮境。光世奏廬難守,密幹趙鼎,欲還太平州。浚命呂祉馳往軍中督師,光世已舍廬州退,浚遣人厲其眾曰:「若有一人渡江,即斬以徇。」光世不得已,駐
 兵與沂中相應,遣王德、酈瓊領兵自安豐出謝步,遇金將三戰,皆敗之。張浚入對,言光世驕惰不戰,不可為大將,請罷之。帝命與趙鼎議,鼎曰:「光世將家子孫,將卒多出其門,罷之恐拂人心。」遂遷護國、鎮安、保靜軍節度使。



 右司諫陳公輔劾其不守廬州,張浚言其沉酣酒色,不恤國事,語以恢復,意氣怫然,乞賜罷斥。光世引疾請罷軍政,又獻所餘金穀於朝。拜少師,充萬壽觀使,奉朝請,封榮國公,賜甲第一區,以兵歸都督府。公輔又言光世
 雖罷,而遷少師,賞罰不明;中書舍人勾龍如淵又繳還賜第之命。帝曰:「光世罷兵柄,若恩禮稍加,則諸將知有後福,皆效力矣。」卒賜之。初,光世麾下多降盜,素無紀律;至是,督府命呂祉節制其軍。酈瓊殺祉,驅諸軍降劉豫。



 九年,用講和恩,賜號「和眾輔國功臣」,進封雍國公、陜西宣撫使。弟光遠疏其短於言路,如淵時為中丞,再論光世不可遣而止。十年,金人圍順昌,拜太保,為三京招撫處置使,以援劉錡。光世請李顯忠為前軍都統,又請王
 德自隸。德不願受其節制;顯忠行至宿、泗,軍多潰。進至和州,秦檜主罷兵,召還。光世入見,為萬壽觀使,改封楊國公。疾革,乞免其家科役,中書舍人張廣格不下。卒,年五十四。贈太師,官其子孫、甥侄十四人,謚武僖。乾道八年,追封安城郡王。開禧元年,追封鄜王。



 光世在諸將中最先進。律身不嚴,馭軍無法,不肯為國任事,逋寇自資,見詆公論。嘗入對,言:「願竭力報國,他日史官書臣功第一。」帝曰:「卿不可徒為空言,當見之行事。」建炎初,結內侍
 康履以自固。又蚤解兵柄,與時浮沉,不為秦檜所忌,故能竊寵榮以終其身,方之韓、岳遠矣。



 王淵,字幾道,熙州人,後徙環州。善騎射。應募擊夏國,屢有功,累遷熙河蘭湟路第三將部將、權知鞏州寧遠砦。諸羌入寇,經略司討之,表淵總領岷山蕃兵將,興師城澤州。羌悉眾來爭,淵奮擊,大破之,追至邈川城。移同總領湟州蕃兵將兼知臨宗砦,坐法免。



 宣和三年,劉延慶討方臘,以淵為先鋒。賊將據錢塘,勢張甚。淵諭小校韓
 世忠曰:「賊謂我遠來,必易我。明日爾逆戰而偽遁,我以強弩伏數百步外,必可得志。」世忠如其言,賊果追之,伏弩捽發,應弦而倒。逐北至淳安,賊據幫源峒,遂圍而平之。授閣門宣贊舍人、權京畿提舉保甲兼權提點刑獄公事。



 繼從延慶攻契丹。重兵壁盧溝南,遣淵等數千人護餉道,戰敗為敵所獲。已而逃歸,猶以出塞遷武功大夫、果州團練使。又從楊惟忠、辛興宗破群盜高托山等,遷拱衛大夫、寧州觀察使。



 靖康元年,為真定府總管,就
 遷都統制。吳湛據趙州叛,淵討平之。金人攻汴京,河東、北宣撫使範訥統勤王兵屯雍丘,以淵為先鋒。尋以所部歸康王府。



 明年,張邦昌僭立,康王如濟州,命淵以三千人入衛宗廟。淵至汴都,以朝服見邦昌,納謁曰:「參塚宰相公。」邦昌始易紫袍延之政事堂,淵慟哭宣教。康王即皇帝位,淵與楊惟忠、韓世忠以河北兵,劉光世以陜西兵,張俊、苗傅等以帥府及降群盜兵,皆在行朝,不相統一。始置御營司,以淵為都統制,扈從累月不釋甲。帝
 如揚州,授龍、神衛四廂都指揮使,尋改捧日、天武四廂都指揮使,進保大軍承宣使。



 時群盜蜂起,以淵為制置使平杭賊,提兵四出,所向皆捷。平軍賊趙萬於鎮江,誅杭賊陳通於杭州,降張遇於楊子橋;期年,群盜略盡。遷響德軍節度使。惟趙萬、陳通等已招其降,而復盡誅之。



 建炎三年二月,金人攻揚州,帝倉卒渡江,淵與內侍康履從至鎮江。奉國軍節度使劉光世見帝泣告:「淵專管江上海船,每言緩急決不誤事。今臣所部數萬,二千餘
 騎,皆不能濟。」淵忿其言,斬江北都巡檢皇甫佐以自解。中書侍郎朱勝非馳見淵督之,乃始經畫,已無所及,自是淵失諸將心。



 帝欲如鎮江以援江北,群臣亦固請。淵獨言:「鎮江止可捍一面,若金人自通州渡,先據姑蘇,將若之何?不如錢塘有重江之險。」議遂決。命淵守姑蘇,言戎器全缺,兵匠甚少,乞括民匠營繕。尋自平江赴行在,拜簽書樞密院事,仍兼都統制。命下,諸將籍籍。帝聞之,乃命免奏事簽書,仍解都統制,以慰眾心。



 先是,統制官
 苗傅自負世將,以淵驟用,頗觖望;劉正彥嘗招巨盜丁進,亦以賞薄怨淵。而內侍康履頗用事,及淵入樞府,傅、正彥以其由宦官薦,愈不平。俟淵入朝,伏兵殺之,並殺康履,遂成明受之變。淵時年五十三。



 淵為將輕財好義,家無宿儲,每言:「朝廷官人以爵祿足代耕,若事錐刀,我何愛爵祿,曷若為富商大賈邪?」初,帝在南京,聞淵疾,遣中使曾澤問疾。澤還,言其帷幔茵褥皆不具,帝輟所御紫茸茵以賜。然其平群盜多殺降,與康履深交,故及於
 禍。贈開府儀同三司,累加少保,官其子孫八人。紹興四年,又官二人。乾道六年,謚襄愍。子倚。



 解元,字善長,保安軍德清砦人。疏眉俊目,猿臂,善騎射。起行伍,為清澗都虞候。建炎三年,隸大將韓世忠麾下,擢偏將。世忠出下邳,聞金兵大至,士皆駭愕。元領二十騎擒其生口,知敵動息。俄逢騎數百,身自陷陣,橫刺酋長墜馬,餘皆遁去。授閣門宣贊舍人。苗傅、劉正彥之變,從世忠追至臨平與戰,賊勢既衰,擒於浦城。



 四年三月,
 金人攻浙西,世忠治兵京口,邀其歸路,以海艦橫截大江。金人出小舟數十,以長鉤扳艦。元在別舸躍入敵舟,以短兵擊殺數十人,擒其千戶。授忠州團練使,統制前軍。繼從討閩寇範汝為,轉討湖外諸盜。時劉忠據白面山,憑險築壘。世忠討之,距賊營三十里而陣。元獨跨馬涉水薄賊砦,四顧周覽。賊因山設望樓,從高瞰下,以兵守之,屯壯銳於四山,視其指呼而出戰。元既得其形勢,歸告世忠曰:「易與爾,若奪據其望樓,則技窮矣。」世忠然
 之,遣元率兵五百,長戟居中,翼以弓矢,自下趨高,賊眾莫支。乃據望樓,立赤幟,四面並進,賊遂平。改相州觀察使。



 紹興四年,金人、偽齊合兵入侵。世忠自鎮江趨揚州,命元屯承州。金人至近郊,元度翌日必至城下,遣百人伏要路,百人伏嶽廟,自以四百人伏路隅。令曰:「俟金人過,我當先出掩之。伏要路者,視我麾旗,則立幟以待,金人必自嶽廟走,伏者背出。」又決河岸遏其歸路。金人果走城下,伏發,金人進退無路,乃走嶽廟,元追之,獲百四
 十八人,止遣二人。時城中兵不滿三千,金萬戶黑頭虎直造城下約降。元匿其兵,以微服出,偽若降者。金人稍懈,俄伏發,擒黑頭虎。未幾,金兵四集,元戰卻之,追北數十里,金人赴水死者甚眾。改同州觀察使。六年,從世忠出下邳,以數百騎破敵伏兵,授保順軍承宣使。



 十年,略地淮陽,至劉冷莊,騎才三百,當敵騎數千。元揮戈大呼,眾爭奮,敵披靡。俄而救至,後部疑懼,元回顧曰:「我在此,若等無慮。」眾乃安。轉戰自辰至午,敵退,成列而還。加龍、
 神衛四廂都指揮使。



 明年,世忠罷兵柄為樞密使,以元為鎮江府駐札御前諸軍都統制,以統其眾。又明年,進侍衛親軍馬步軍都虞候,尋授保信軍節度使。卒,年五十四。贈檢校少保。



 曲端字正甫,鎮戎人。父渙,任左班殿直,戰死。端三歲,授三班借職。警敏知書,善屬文,長於兵略,歷秦鳳路隊將、涇原路通安砦兵馬臨押,權涇原路第三將。



 夏人入寇涇原,帥司調統制李庠捍禦,端在遣中。庠駐兵柏林堡,
 斥堠不謹,為夏人所薄,兵大潰,端力戰敗之,整軍還。夏人再入寇,西安州、懷德軍相繼陷沒。鎮戎當敵要沖,無守將,經略使席貢疾柏林功,奏端知鎮戎軍兼經略司統制官。



 建炎元年十二月,婁宿攻陜西。二年正月,入長安、鳳翔,關、隴大震。二月,義兵起,金人自鞏東還。端時治兵涇原,招流民潰卒,所過人供糧秸,道不拾遺。金游騎入境,端遣副將吳玠據清溪嶺與戰,大破之。端乘其退,遂下兵秦州,而義兵已復長安、鳳翔。統領官劉希亮自
 鳳翔歸,端斬之。六月,以集英殿修撰知延安府。



 王庶為龍圖閣待制,節制陜西六路軍馬。遂授端吉州團練使,充節制司都統制,端雅不欲屬庶。九月,金人攻陜西,庶召端會雍、耀間,端辭以未受命。庶以鄜延兵先至龍坊,端又稱已奏乞回避,席貢別遣統制官龐世才將步騎萬人來會。庶無如之何,則檄貢勒端還舊任,遣陜西節制司將官賀師範趨耀,別將王宗尹趨白水,且令原、慶出師為援,二帥各遣偏將劉仕忠、寇鯇來與師範會。
 庶欲往耀督戰,已行,會龐世才兵至邠,端中悔,以狀白庶,言已赴軍前,庶乃止。師範輕敵不戒,卒遇敵於八公原,戰死,二將各引去,端遂得涇原兵柄。



 十一月,金諜知端、庶不協,並兵攻鄜延。時端盡統涇原精兵,駐淳化。庶日移文趣其進,又遣使臣、進士十數輩往說端,端不聽。庶知事急,又遣屬官魚濤督師,端陽許而實無行意。權轉運判官張彬為端隨軍應副,問以師期。端笑謂彬曰:「公視端所部,孰與李綱救太原兵乎?」彬曰:「不及也。」端曰:「綱
 召天下兵,不度而往,以取敗。今端兵不滿萬,不幸而敗,則金騎長驅,無陜西矣。端計全陜西與鄜延一路孰輕重,是以未敢即行,不如蕩賊巢穴,攻其必救。」乃遣吳玠攻華州,拔之。端自分蒲城而不攻,引兵趨耀之同官,復迂路由邠之三水與玠會襄樂。



 金攻延安急,庶收散亡往援。溫州觀察使、知鳳翔府王□燮將所部發興元,比庶至甘泉,而延安已陷。庶無所歸,以軍付□燮,自將百騎與官屬馳赴襄樂勞軍。庶猶以節制望端,欲倚以自副,端
 彌不平。端號令素嚴,入壁者,雖貴不敢馳。庶至,端令每門減其從騎之半,及帳下,僅數騎而已。端猶虛中軍以居庶,庶坐帳中,端先以戎服趨於庭,即而與張彬及走馬承受公事高中立同見帳中。良久,端聲色俱厲,問庶延安失守狀,曰:「節制固知愛身,不知愛天子城乎?」庶曰:「吾數令不從,誰其愛身者?」端怒曰:「在耀州屢陳軍事,不一見聽,何也?」因起歸帳。庶留端軍,終夕不自安。



 端欲即軍中殺庶,奪其兵。夜走寧州,見陜西撫諭使謝亮,說之
 曰:「延安五路襟喉,今已失之,《春秋》大夫出疆得以專之,請誅庶歸報。」亮曰:「使事有指,今以人臣擅誅於外是跋扈也,公為則自為。」端意阻,復歸軍。明日,庶見端,為言已自劾待罪。端拘縻其官屬,奪其節制使印,庶乃得去。



 王□燮將兩軍在慶陽,端召之,□燮不應。會有告□燮過邠軍士劫掠者,端怒,命統制官張中孚率兵召□燮,謂中孚曰:「□燮不聽,則斬以來。」中孚至慶陽,□燮已去,遽遣兵要之,不及而止。



 初,叛賊史斌圍興元不克,引兵還關中。義兵統領
 張宗諤誘斌如長安而散其眾,欲徐圖之。端遣吳玠襲斌擒之,端自襲宗諤殺之。



 三年九月,遷康州防禦使、涇原路經略安撫使。時延安新破,端不欲去涇原,乃以知涇州郭浩權鄜延經略司公事。自謝亮歸,朝廷聞端欲斬王庶,疑有叛意,以御營司提舉召端,端疑不行。議者喧言端反,端無以自明。會張浚宣撫川、陜,入辨,以百口明端不反。浚自收攬英傑,以端在陜西屢與敵角,欲仗其威聲。承制築壇,拜端為威武大將軍、宣州觀察使、宣
 撫處置使司都統制、知渭州。端登壇受禮,軍士歡聲如雷。



 浚雖欲用端,然未測端意,遣張彬以招填禁軍為名,詣渭州察之。彬見端問曰:「公常患諸路兵不合,財不足;今兵已合,財已備,婁宿以孤軍深入吾境,我合諸路攻之不難。萬一粘罕並兵而來,何以待之?」端曰:「不然,兵法先較彼己,今敵可勝,止婁宿孤軍一事;然將士精銳,不減前日。我不可勝,亦止合五路兵一事;然將士無以大異於前。況金人因糧於我,我常為客,彼常為主。今當反
 之,按兵據險,時出偏師以擾其耕獲。彼不得耕,必取糧河東,則我為主,彼為客,不一二年必自困斃,可一舉而滅也。萬一輕舉,後憂方大。」彬以端言復命,浚不主端說。



 四年春,金人攻環慶,端遣吳玠等拒於彭原店,端自將屯宜祿,玠先勝。既而金軍復振,玠小卻,端退屯涇州,金乘勝焚邠州而去。玠怨端不為援,端謂玠前軍已敗,不得不據險以防沖突,乃劾玠違節制。



 是秋,兀朮窺江、淮,浚議出師以撓其勢。端曰:「平原廣野,賊便於沖突,而我
 軍未嘗習水戰。金人新造之勢,難與爭鋒,宜訓兵秣馬保疆而已,俟十年乃可。」端既與浚異,浚積前疑,竟以彭原事罷端兵柄,與祠,再責海州團練副使、萬州安置。



 是年,浚為富平之役,軍敗,誅趙哲,貶劉錫。浚欲慰人望,下令以富平之役,涇原軍馬出力最多,既卻退之後,先自聚集,皆緣前帥曲端訓練有方。敘端左武大夫,興州居住。



 紹興元年正月,敘正任榮州刺史,提舉江州太平觀,徙閬州。於是浚自興州移司閬州,欲復用端。玠與端有
 憾,言曲端再起,必不利於張公;王庶又從而間之。浚入其說,亦畏端難制。端嘗作詩題柱曰:「不向關中興事業,卻來江上泛漁舟。」庶告浚,謂其指斥乘輿,於是送端恭州獄。



 武臣康隨者嘗忤端,鞭其背,隨恨端入骨。浚以隨提點夔路刑獄,端聞之曰:「吾其死矣!」呼「天」者數聲;端有馬名「鐵象」,日馳四百里,至是連呼「鐵象可惜」者又數聲,乃赴逮。既至,隨令獄吏縶維之,糊其口,之以火。端乾渴求飲,予之酒,九竅流血而死,年四十一。陜西士大夫
 莫不惜之,軍民亦皆悵悵,有叛去者。浚尋得罪,追復端宣州觀察使,謚壯愍。



 端有將略,使展盡其才,要未可量。然剛愎,恃才凌物,此其所以取禍云。



 論曰:南渡諸將以張、韓、劉、嶽並稱,而俊為之冠。然夷考其行事,則有不然者。俊受心膂爪牙之寄,其平苗、劉,雖有勤王之績,然既不能守越,又棄四明,負亦不少。矧其附檜主和,謀殺岳飛,保全富貴,取媚人主,其負戾又如何哉?光世自恃宿將,選沮卻畏,不用上命,師律不嚴,卒
 致酈瓊之叛。迎合檜意,首納軍權,雖得善終牖下,君子不貴也。二人方之韓、岳益遠矣。然子蓋、宗顏號俊子弟,著海之功,泗上之捷,亦足稱焉。王淵以總率扈從有勞,遂至驕盈,失將士心,自取覆敗。況結托康履與光世一轍,烏足道哉。解元始由韓世忠進,其攻城野戰,未嘗敗衄,有可稱者,不幸早世,惜哉!曲端剛愎自用,輕視其上,勞效未著,動違節制,張浚殺之雖冤,蓋亦自取焉爾。



\end{pinyinscope}