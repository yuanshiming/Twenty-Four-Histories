\article{列傳第十八}

\begin{pinyinscope}

 張美郭守文尹崇珂劉廷讓袁繼忠崔彥進張廷翰皇甫繼明張瓊



 張美,字玄圭,貝州清河人。少善書計,初為左藏小吏,以
 強幹聞。三司薦奏,特補本庫專知,出為澶州糧料使。周世宗鎮澶淵,每有求取,美必曲為供給。周祖聞之怒,將譴責之,而恐傷世宗意,徙美為濮州馬步軍都虞候。



 世宗即位,召為樞密承旨。時宰相景範判三司,被疾,世宗命美為右領軍衛大將軍,權判三司。世宗征淮南,留美為大內部署。一日,方假寐,忽覺心動,遽驚起行視宮城中。少頃,內醞署火起,既有備,即撲滅之。俄真授三司使。



 四年,世宗再幸淮上,皆為大內都點檢。北征,又為大內
 都部署。師還,為左監門衛上將軍,充宣徽北院使,判三司。美強力有心計,周知其利病,每有所條奏厘革,上多可之,常以幹敏稱。世宗連歲征討,糧饋不乏,深委賴焉。然以澶淵有所求假,頗薄之,美亦自愧。恭帝嗣位,加檢校太傅。



 宋初,加檢校太尉。初,李筠鎮上黨,募亡命,多為不法,漸倔強難制。美度筠必叛,陰積粟於懷、孟間。後筠果叛,太祖親討之,大軍十萬出太行,經費無闕,美有力焉。拜定國軍節度。縣官市木關中,同州歲出緡錢數十
 萬以假民,長吏十取其一,謂之率分錢,歲至數百萬,美獨不取。未幾,他郡有詣闕訴長吏受率分錢者,皆命償之。



 乾德五年,移鎮滄州。太平興國初來朝,改左驍衛上將軍。美獻都城西河曲灣果園二、蔬圃六、亭舍六十餘區。八年,請老,以本官致仕。雍熙二年,卒,年六十八。淳化初,謚恭惠。子守瑛,至供備庫使。孫士宗,至內殿承制。士宗卒,士禹為崇班,士安至閣門祗候,士宣為禮賓副使。



 郭守文,並州太原人。父暉,仕漢為護聖軍使,從周祖征
 河中,戰死。守文年十四,居喪哀毀,周祖憐之,召隸帳下。廣順初,補左班殿直,再遷東第二班副都知。



 宋初,遷西頭供奉官。蜀平,遷知簡州。時劍外多寇,守文悉招來集附。從潘美征嶺南,會擒劉鋹,遣守文馳傳告捷,遷翰林副使。從曹彬等平金陵,護送李煜歸闕下。時煜以拒命頗自歉,不欲生見太祖。守文察知之,因謂煜曰:「國家止務恢復疆土,以致太平,豈復有後至之責耶?」煜心遂安。改西京作坊使、領翰林司事。俄從黨進破並寇於團柏
 谷。



 太平興國初,秦州內附,蕃部騷動,命守文乘傳撫諭,西夏悅伏。三年,遷西上閣門使。是夏,汴水決於寧陵,發宋、亳丁壯四千五百塞之,命守文董其役。是冬,又與閣門副使王侁、西入作副使石全振護塞靈河縣決河。



 及徵太原,守文與判四方館事梁迥分護行營馬步軍。會劉繼元降,其弟繼文據代州,依遼人之援以拒命,遣守文討平之。俄受詔護定州屯兵,大破遼人於蒲城。以功遷東上閣門使、領澶州刺史。召還,擢拜內客省使。八年,
 滑州房村河決,發卒塞之,命守文董其役。遼人擾雄州,命守文率禁兵數萬人赴援,既至,遼人遁去。



 雍熙二年,詔守文率兵屯三交,俄加領武州團練使。屬夏人擾攘,命守文帥師討之,破夏州鹽城鎮岌羅膩等十四族,斬首數千級,俘獲生畜萬計。又破咩嵬族,殲焉。諸部畏懼,相率來降,凡銀、麟、夏三州歸附者百二十五族、萬六千餘戶,西鄙遂寧。五年春,大舉北伐,為幽州道行營前軍步軍水陸都監。卒與遼人遇,為流矢所中,氣色不撓,督
 戰益急,軍中服其量。會大軍不利,坐違詔逗遛退軍,左遷右屯衛大將軍。事具《曹彬傳》。



 明年復舊職,裁三月,拜宣徽北院使。又與田欽祚並為北面排陣使,屯鎮州。端拱初,改南院使、鎮州路都部署。又為北面行營都部署兼鎮定、高陽關兩路排陣使。是冬,遼騎南侵,大破之唐河。端拱三年十月,卒,年五十五。太宗悼惜之,贈侍中。謚忠武,追封譙王,遣中使護喪,歸葬京師。



 守文沈厚有謀略,頗知書,每朝退,習書百行,出言溫雅,未嘗忤人意。先
 是,將臣戍邊者多致寇以邀戰功,河朔諸州殆無寧歲,既敗岐溝關,乃命守文以內職總兵鎮常山以經略之。



 守文既喪月餘,中使自北邊來言:「守文死,軍士皆流涕。」帝問:「何以得此?」對曰:「守文得奉祿賜賚悉犒勞士卒,死之日,家無餘財。」帝嗟嘆久之,賜其家錢五百萬,為真宗納其女為夫人,即章穆皇后也。



 子崇德至太子中舍。崇信至西京左藏庫使、同知皇城司,贈福州觀察使。崇儼至崇儀使、全州刺史,贈潤州觀察使。諸司使無廢朝、贈
 官之例,崇信、崇儼咸以後兄故,特示優禮。崇德子承壽,至虞部員外郎。天禧五年,錄承壽子若水為太常寺奉禮郎,崇仁為解州團練使。



 尹崇珂,秦州天水人,後徙居大名。父延勛,歷磁、同、滁三州刺史。崇珂初事周世宗於藩邸,以謹厚稱。及即位,補東西班都知。從戰高平,有勞績,遷本班副點檢。從征淮南,遷都虞候,轉都指揮使,改前殿都指揮使。



 宋初,出為淄州刺史。有善政,民詣闕請刻石頌德,太祖命殿中侍
 御史李穆撰文賜之。討湖南,為行營前軍馬軍都指揮使。荊湘平,授朗州團練使。又與潘美、丁德裕克郴州。



 乾德中,徵嶺表,以崇珂為行營馬步軍副部署。克廣州,擒劉鋹,即日詔與潘美同知廣州兼市舶轉運等使,錄功遷保信軍節度。未幾,南漢開府樂範、容州都指揮使鄧存忠、韶州賊帥周思瓊、春恩道都指揮使麥漢瓊等據五州之地以叛。崇珂討之,太祖遣中使李神祐督戰,數月,盡平其黨,還治所。



 六年,卒,年四十二。贈侍中。遣中使
 護其喪,歸葬洛陽。以其子昭吉、弟崇珪並為西京作坊使,昭吉領會州刺史,崇珪領歙州刺史。



 初,太宗在周朝娶崇珂妹,追謚淑德皇后。昭吉至洛苑使。次子昭輯,至供奉官、閣門祗候。



 劉廷讓,字光乂,其先涿州範陽人。曾祖仁恭,唐盧龍軍節度。祖守文,襲滄州盧彥威,遂據其城,昭宗授以節鉞。後其弟守光囚父仁恭,守文舉兵討之,軍敗,為守光所殺。廷讓與其父延進避難南奔。少有膂力,周祖鎮鄴,以
 隸帳下。廣順初,補內殿直押班,累遷龍捷都校。從世宗征淮南,以功領雷州刺史。再遷涪州團練使、領鐵騎右廂。



 宋初,轉江州防禦使、領龍捷右廂。從征李筠,為行營先鋒使。建隆二年,改侍衛馬軍都指揮使、領江寧軍節度。乾德二年春,詔領兵赴潞州,以備並寇。冬,興師伐蜀,為四川行營前軍兵馬副都部署,率禁兵步騎萬人、諸州兵萬人,由歸州進討。入其境,連破松木、三會、巫山等砦,獲蜀將南光海等五千餘人,擒戰棹都指揮使袁德
 宏等千二百人,奪戰艦二百餘艘。又獲水軍三千人,因度南岸,斬三千餘級。



 初,夔州有鎖江為浮梁,上設敵棚三重,夾江列炮具。廷讓等將行,太祖以地圖標之,指鎖江曰:「我軍至此,溯流而上,慎勿以舟師爭勝,當先以步騎陸行,出其不意擊之,俟其勢卻,即以戰棹夾攻,取之必矣。」及師至,距鎖江三十里,舍舟步進,先奪其橋,復牽舟而上,破州城,守將高彥儔自焚,悉如太祖計。遂進克萬、施、開、忠四州,峽中郡縣悉下。



 明年正月,次遂州,州將
 陳俞率吏民來降。盡出府庫金帛以給將士。初出師也,太祖命之曰:「所得郡縣,當傾帑藏,為朕賞戰士,國家所收唯土疆爾。」故人皆效命,所至成功。蜀平,王全斌等皆坐縱部下掠奪子女玉帛及納賄賂左降,惟廷讓秋毫無犯。及全師雄等作亂,郡縣相應,寇盜蜂起。廷讓又與曹彬破之,以功改領鎮安軍節度,從征太原。開寶六年,出為鎮寧軍節度。太平興國二年,入為右驍衛上將軍。



 雍熙三年,曹彬敗於岐溝關,諸將失律,多坐黜免。既而
 契丹擾邊,時議遣將,無愜上意者。時廷讓與宋偓、張永德並罷節鎮在環列,帝欲令擊契丹自效,乃遣分守邊郡,以廷讓知雄州,又徙瀛州兵馬都部署。是冬,契丹數萬騎來侵,廷讓與戰君子館。時天大寒,兵士弓弩皆不能彀,契丹圍廷讓數重。廷讓先分精兵屬李繼隆為後殿,緩急為援。至是,繼隆退保樂壽,廷讓一軍皆沒,死者數萬人,僅以數騎獲免。先鋒將賀令圖、楊重進皆陷於契丹。自是河朔戍兵無鬥志,又科鄉民為兵以守城,
 皆未習戰鬥。契丹遂長驅而入,陷深、祁、德數州,殺官吏,俘士民,所在輦金帛而去。博、魏之北,民尤苦焉。太宗聞之,下哀痛之詔。



 初,廷讓詣闕待罪,太宗知為李繼隆所誤,不之責。四年,復命代張永德知雄州兼兵馬部署。是秋以疾聞,帝遣內醫診視,因上言求歸京師,不俟報,乃離屯所。帝怒,下御史按問,獄具。下詔曰:「右驍衛上將軍劉廷讓,朕以其宿舊,曾董軍政,擢自環尹,付之成師,俾控邊關,式防寇鈔。而乃以病為解,不俟報命,委棄戎重,俶
 裝上道。矧萬旅所集,實制於中權,列燧相望,或虞於外侮。事機一失,咎責安歸。有司議刑,當在不赦。錄其素效,特從寬典,可削奪在身官爵,配隸商州。」又黜其子如京使永德為濠州團練副使,崇儀副使永和為唐州刺使史。廷讓既黜,怏怏不食,行至華州卒,年五十九。帝錄其舊勛,贈太師。



 子永德至內殿崇班,永恭至西京作坊副使,永和為內殿承制,永錫至崇班,永保、永昌、永規並至閣門祗候,永崇為崇班,永寧及孫允忠並為閣門祗候。



 袁繼忠,其先振武人,後徙並州。父進,仕周為階州防禦使。繼忠以父任補右班殿直。太祖平澤、潞,討並、汾,悉預攻戰。乾德中征蜀,隸大將劉廷讓麾下。既克蜀,知雲安軍,歷嘉、蜀二州監軍。開寶中伐廣南,為先鋒壕砦。廣南平,以功遷供奉官,護隰州白壁關屯兵。時河東拒命,繼忠累入其境,破三砦,擒將校二人,得生口、馬牛羊、鎧仗逾萬計。近戍主將懼無功受譴,以誠告繼忠,繼忠以所獲分與之,遂與都巡檢郭進略地忻、代州,改天平軍巡
 檢。



 太宗即位,以為閣門祗候,令擊梅山洞賊,破之。又巡遏邊部於唐龍鎮。太宗征太原,繼忠預破鷹揚軍,先登陷陣。契丹入代境,繼忠率兵擊走之。以功遷通事舍人,護高陽關屯兵。與崔彥進破契丹長城口,殺獲數萬眾,璽書褒美。時有勸繼忠自論其功者,繼忠不答。會趙保忠來朝獻其地,綏州刺史李克憲偃蹇不奉詔,遣繼忠諭旨,竟率克憲入朝。遷西上閣門副使。詔與田仁朗率兵定河西諸州,大破西人於葭蘆州,遷引進副使,護定
 州屯兵。



 雍熙二年,遷西上閣門使。三年,大將田重進徵契丹,命繼忠為定州路行營馬步軍都監。領師取飛狐,下靈丘,平蔚州,擒其帥大鵬翼以獻,事見重進傳。師還,繼忠為後殿,行列甚整。至定州,重進欲斬降卒後期至者,繼忠諭以殺降不祥,皆救免之。遷判四方館事、領播州刺史,護屯兵如故。大將李繼隆以易州靜塞騎兵尤驍果,取隸麾下,畜其妻子城中。繼忠言於繼隆曰:「此精卒,止可守城,萬一敵至,城中誰與悍者?」繼隆不從。既而
 契丹入寇,城陷,卒妻子皆為所俘。繼隆疑此卒怨己,欲分隸諸軍。繼忠曰:「不可,但奏升其軍額,優以廩給,使之盡節可也。」從之,眾皆感悅。繼忠因自請以隸麾下。



 會契丹騎大至,駐唐河北,諸將欲堅壁待之。繼忠曰:「今強敵在近,城中屯重兵不能剪滅,令長驅深入,侵略他郡,雖欲謀自安之計,豈折沖禦侮之用乎?我將身先士卒,死於寇矣!」辭氣慷慨,眾壯之。靜塞軍摧鋒先入,契丹兵大潰。太宗聞之,降璽書獎諭,賜予甚厚。淳化初,遷引進使,
 護鎮定、高陽關兩路屯兵。三年,被病,召赴闕,卒,年五十五。



 繼忠長厚忠謹,士大夫多與游,前後賜賚鉅萬計,悉以犒賞士卒。身死之日,家無餘財,搢紳稱之。子用成,雍熙初登進士第,至太常博士。



 崔彥進,大名人。純質有膽略,善騎射。漢乾祐中,隸周祖帳下。廣順初,補衛士。世宗鎮澶淵,令領禁兵以從。顯德初,為控鶴指揮使。從征淮南,以功遷散員都虞候。從平瓦橋關,改東西班指揮使、領昭州刺史。



 宋初,改控鶴右
 廂指揮使、領果州團練使。征李筠,為先鋒部署,以功遷常州防禦使。從平李重進,改虎捷右廂。建隆二年,遷侍衛步軍都指揮使、領武信軍節度。大舉伐蜀,為鳳州路行營前軍副都部署。蜀平,坐縱部下略玉帛、子女及諸不法事,左遷昭化軍節度觀察留後。太祖郊祀西洛,彥進來朝,授彰信軍節度。



 太平興國二年,移鎮河陽。四年正月,遣將征太原,分命攻城,以彥進與郢州防禦使尹勛攻其東,彰德軍節度李漢瓊、冀州刺史牛思進攻其
 南,桂州觀察使曹翰、翰林使杜彥圭攻其西,毾信軍節度劉遇、光州刺史史珪攻其北。彥進督戰甚急,太祖嘉之。晉陽平,從征幽州,又與內供奉官江守鈞率兵攻城之西北。及班師,詔彥進與西上閣門副使薛繼興、閣門祗候李守斌領兵屯關南,以功加檢校太尉。是秋,契丹侵遂城,彥進與劉廷翰、崔翰等擊破之,斬首萬級。五年,車駕北巡,以彥進為關南都部署,敗契丹於唐興口。



 雍熙三年正月,命將北伐,分兵三路,詔彥進為幽州道行
 營馬步軍水陸副都部署,與曹彬、米信出雄州。大軍失利,彥進坐違彬節制,別道回軍,為敵所敗,召還,貶右武衛上將軍,事具彬傳。四年春,授保靜軍節度。端拱元年,被病,召歸闕,卒,年六十七。贈侍中。



 彥進頻立戰功,然好聚財貨,所至無善政。沒後,諸子爭家財,有司攝治。太宗召見,為決之,謂左右曰:「此細務,朕不宜親臨,但以彥進嘗任節制,不欲令其子辱於父耳。」



 子懷遵至內殿崇班,懷清至崇儀副使。懷遵子上賢,娶鎮王女崇安縣主。懷
 清子從湜,娶岐王女永壽縣主,為西京左藏庫副使,後坐事除名。



 張廷翰,澤州陵川人。初為漢祖親校。漢祖入汴,補內殿直,遷東西班軍使。周初,改護聖指揮使。從世宗平淮甸,以功遷鐵騎右第二軍都虞候。顯德末,改殿前散都頭都虞候。宋初,權為鐵騎左第二軍都校、領開州刺史。從平揚州,又以功遷控鶴左廂都指揮使、領果州團練使。未幾,轉龍捷左廂都指揮使、領春州團練使。乾德中,興
 師伐蜀,以廷翰為歸州路行營馬軍都指揮使,隨劉廷讓由歸州路進討。師次夔州,廷讓頓兵白帝廟西,俄而夔州監軍武守謙率所部來拒戰,廷翰引兵逆擊,敗之於豬頭鋪,乘勝拔其城。蜀平,授侍衛馬步軍都虞候、領彰國軍節度。開寶二年,寢疾,太祖親臨問,未幾卒,年五十三。贈侍中。



 皇甫繼明,冀州蓨人。父濟,汾川令。繼明身長七尺,善騎射,以膂力聞郡中。刺史張廷翰以隸左右,薦於太祖,補
 殿前指揮使,歷左右番押班都知。



 太宗即位,累遷至捧日軍都指揮使、領檀州刺史。太平興國七年,坐秦王廷美事,出為汝州馬步軍都指揮使。雍熙三年,召入為馬步軍副都軍頭。四年,復為捧日右廂第三軍都指揮使、領澶州刺史。田重進北征,繼明為前鋒,以功加馬步軍都軍頭。端拱二年,轉龍、神衛四廂都指揮使、領羅州防禦使。即日命副高瓊為並代部署。淳化二年,又副範廷召為平虜橋砦兵馬都部署,改高陽關部署。



 至道元年,
 改領洋州觀察使,充環慶路馬步軍都部署。繼明謹願,御下嚴肅,士卒頗畏憚之。二年,受詔護送輜重赴靈州,繼明已先約靈州部署田紹斌率軍迎援,適被病,裨將白守榮謂繼明曰:「君疾甚,不可行,恐失期會,守榮當率兵先往。」繼明宿將,慮守榮等輕佻,與戎人接戰,因謂之曰:「我疾少間。」遂矍鑠被甲上馬,強行至清遠軍,卒,年六十三,詔贈彰武軍節度。遷其子懷信為供奉官。



 張瓊,大名館陶人。世為牙中軍。瓊少有勇力,善射,隸太
 祖帳下。周顯德中,太祖從世宗南征,擊十八里灘砦,為戰艦所圍,一人甲盾鼓噪而前,眾莫敢當,太祖命瓊射之,一發而踣,淮人遂卻。



 及攻壽春,太祖乘皮船入城壕。城上車弩遽發,矢大如椽,瓊亟以身蔽太祖,矢中瓊股,死而復蘇。鏃著髀骨,堅不可拔。瓊索杯酒滿飲,破骨出之,血流數升,神色自若。太祖壯之。及即位,擢典禁軍,累遷內外馬步軍都軍頭、領愛州刺史。數日,太宗自殿前都虞候尹開封。太祖曰:「殿前衛士如狼虎者不啻萬人,
 非瓊不能統制。」即命瓊代為都虞候,遷嘉州防禦使。



 瓊性暴無機,多所凌轢。時史珪、石漢卿方用事,瓊輕侮之,目為巫媼。二人銜之切齒,發瓊擅乘官馬,納李筠隸僕,畜部曲百餘人,恣作威福,禁軍皆懼;又誣毀太宗為殿前都虞候時事。建隆四年秋,郊禋制下,方欲肅靜京師,乃召訊瓊。瓊不伏,太祖怒,令擊之。漢卿即奮鐵撾亂下,氣垂絕,曳出,遂下御史案鞫之。瓊知不免,行至明德門,解所系帶以遺母。獄具,賜死於城西井亭。太祖旋聞家
 無餘財,止有僕三人,甚悔之。因責漢卿曰:「汝言瓊有僕百人,今何在?」漢卿曰:「瓊所養者一敵百耳。」太祖遂優恤其家。以其子尚幼,乃擢其兄進為龍捷副指揮使。



 論曰:崔彥進與王全斌征蜀,黷貨殺降,以致蜀亂,惟劉廷讓一軍秋毫無犯,紀律嚴否於斯別矣。尹崇珂斤斤謹厚,臨淄攻守之績,嶺嶠廓清之勞,至於瘁事。皇甫繼明力疾以護軍行,純誠勇節,皆足嘉尚。張廷翰西征,未睹奇效。張美雖稱幹敏,而初有自愧之行。郭守文敦詩
 閱禮,輕財好施,慎保封疆,士卒樂用,終以勛舊蒙眷,聯姻戚里。宋初諸將,要終而論,臧否異趣,何昭昭若是哉



\end{pinyinscope}