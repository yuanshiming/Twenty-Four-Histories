\article{列傳第二百文苑三}

\begin{pinyinscope}

 陳充吳淑舒雅黃夷簡盧稹謝炎許洞附徐鉉句中正曾致堯刁衎姚鉉李建中洪湛路振
 崔遵度陳越



 陳充,字若虛,益州成都人。家素豪盛,少以聲酒自娛,不樂從宦。邑人敦迫赴舉,至京師,有名場屋間。雍熙中,天府、禮部奏名皆為進士之冠,廷試擢甲科,釋褐孟州觀察推官,就改掌書記。會寇準薦其文學,得召試,授殿中丞,出知明州。入為太常博士、直昭文館,遷工部、刑部員外郎。久病告滿,除籍,真宗憐其貧病,令致仕,給半奉。未幾病間,守本官,仍充職。以久次,遷兵部員外郎。景德中,
 與趙安仁同知貢舉,改工部、刑部郎中。



 大中祥符六年,以足疾不任朝謁,出權西京留守御史臺,旋以本官分司,卒,年七十。



 充詞學典贍,唐牛僧孺著《善惡無餘論》,言堯舜之善、伯鯀之惡,俱不能慶殃及其子,充因作論以反之,文多不載。



 性曠達,善談謔,澹於榮利,自號「中庸子」。上頗熟其名,以疾故不登詞職。臨終自為墓志。有集二十卷。



 吳淑,字正儀,潤州丹陽人。父文正,事吳,至太子中允。好
 學,多自繕寫書。淑幼俊爽,屬文敏速。韓熙載、潘祐以文章著名江左,一見淑,深加器重。自是每有滯義,難於措詞者,必命淑賦述。以校書郎直內史。



 江南平,歸朝,久不得調,甚窮窘。俄以近臣延薦,試學士院,授大理評事,預修《太平御覽》、《太平廣記》、《文苑英華》。一日,召對便殿,出古碑一編,令淑與呂文仲、杜鎬讀之。歷太府寺丞、著作佐郎。始置秘閣,以本官充校理。嘗獻《九弦琴五弦阮頌》,太宗賞其學問優博。又作《事類賦》百篇以獻,詔令注釋,淑
 分注成三十卷上之,遷水部員外郎。至道二年,兼掌起居舍人事,預修《太宗實錄》,再遷職方員外郎。



 時諸路所上《閏年圖》,皆儀鸞司掌之,淑上言曰:「天下山川險要,皆王室之秘奧,國家之急務,故《周禮》職方氏掌天下圖籍。漢祖入關,蕭何收秦籍,由是周知險要,請以今閏年所納圖上職方。又州郡地里,犬牙相入,向者獨畫一州地形,則何以傅合他郡?望令諸路轉運使,每十年各畫本路圖一上職方。所冀天下險要,不窺牖而可知;九州輪
 廣,如指掌而斯在。」從之。會詔詢御戎之策,淑抗疏請用古車戰法,上覽之,頗嘉其博學。咸平五年卒,年五十六。



 淑性純靜好古,詞學典雅。初,王師圍建業,城中乏食。里閈有與淑同宗者,舉家皆死,惟存二女孩,淑即收養如所生,及長,嫁之。時論多其義。有集十卷。善筆札,好篆籀,取《說文》有字義者千八百餘條,撰《說文五義》三卷。又著《江淮異人錄》三卷、《秘閣閑談》五卷。



 子安節、讓夷、遵路皆進士及第。遵路官至祠部員外郎、秘閣校理。



 舒雅字子正,久仕李氏。江左平,為將作監丞,後充秘閣校理。好學。善屬文,與吳淑齊名。累遷職方員外郎,求出,得知舒州,仍賜金紫。恬於榮宦,州之潛山靈仙觀有神仙勝跡,郡秩滿,即請掌觀事。東封,就加主客郎中,改直昭文館,轉刑部。在觀累年,優游山水,吟詠自樂,時人美之。卒年七十餘。弟雄,端拱二年進士。



 黃夷簡,字明舉,福州人。父廷樞,為王審知從事,甚被親遇。嗣王延鈞以女妻之。錢氏取福州,署光祿卿。夷簡少
 孤,好學,有名於江東,為錢惟治明州判官。太平興國初,隨錢俶來朝,授檢校秘書少監、元帥府掌書記,賜以襲衣、器幣、鞍勒馬。八年,俶讓元帥,改授夷簡淮海國王府判官。雍熙四年,俶改封許王,出鎮南陽,加夷簡倉部員外郎,充許王府判官。



 俶薨,歸朝,為考功員外郎。累遷都官郎中,掌名表,人頗稱其得體。至道二年,上言浙右人無預館閣之職者,因自陳嘗勸錢俶入朝,詞甚懇激,太宗憐之,命直秘閣,俄判吏部南曹。咸平中,召試翰林,遷
 光祿少卿。



 初,宰相張齊賢欲引夷簡與曾致堯並知制誥,有急制,值舍人出院,即封除目命夷簡草之,物議以為不可,故但進秩而已。景德中,夷簡被病,告滿二百日,御史臺言當除籍,真宗以其吳越舊僚,有詞學,且年老母在,特命續其月廩。大中祥符初,遷秘書少監。三年,丁內艱,上遣中使存問,賻贈有加,因請護母喪歸浙右,許之。且欲不絕其奉給,特授檢校秘書監、平江軍節度副使。逾年卒,年七十七。



 夷簡喜談論,善屬文,尤工詩詠,老
 而不輟。嘗攝鴻臚卿,護許國長公主葬,在道,駙馬都尉魏咸信禮接甚薄,夷簡銜之,言於上云:「發引之日,以錢三十千遺臣治裝,不重王人,若有輕國命之意,臣拒不納。」上遣中使詰咸信,咸信言:「夷簡始受命,屢有求丐,又獻挽詞以希賂遺,臣皆不敢受,以是為慊。」既而夷簡又貢歌詩一編,大率譏咸信吝嗇,且形於怨詛。復言所未受三十千錢,意欲索取。真宗甚鄙之,且不欲其歌詩流布於外,命中書召夷簡對焚之。士大夫以是薄其為人。



 浙右士之秀者,又有盧稹、謝炎、許洞。



 盧稹字淑微,杭州人。幼穎悟,七歲能詩,十二學屬文。及長,曉《五經》大義,酷嗜《周易》、《孟子》。端拱初,游京師,時徐鉉以宿儒為士子所宗,覽稹文,甚奇之,為延譽於朝。是年登進士第,調補真定束鹿主簿。至府,值契丹圍城,未及赴官,卒,年二十七。嘗著《五帝皇極志》、《孺子問》、《翼聖書》數十篇。



 謝炎字化南,蘇州嘉興人。父崇禮,泰寧軍掌書記。炎慕
 韓、柳為文,與盧稹齊名,時謂之「盧、謝」。稹心巽懦,炎勁急,反相厚善。端拱初,舉進士,調補昭應主簿,徙伊闕,連知華容、公安二縣,卒,年三十四。有集二十卷。



 許洞字洞天,蘇州吳縣人。父仲容,太子洗馬致仕。洞性疏雋,幼時習弓矢擊刺之伎,及長,折節勵學,尤精《左氏傳》。咸平三年進士,解褐雄武軍推官。嘗詣府白事,有卒踞坐不起,即杖之。時馬知節知州,洞又移書責知節,知節怒其狂狷不遜,會洞輒用公錢,奏除名。



 歸吳中數年,
 日以酣飲為事。嘗從民坊貰酒,一日,大署壁作《酒歌》數百言,鄉人爭往觀,其酤數倍,乃盡捐洞所負。景德二年,獻所撰《虎鈐經》二十卷。應洞識韜略、運籌決勝科,以負譴報罷,就除均州參軍。大中祥符四年,祀汾陰,獻《三盛禮賦》,召試中書,改烏江縣主簿,卒,年四十二。有集一百卷。又著《春秋釋幽》五卷、《演玄》十卷。



 徐弦,字鼎臣,揚州廣陵人。十歲能屬文,不妄游處,與韓熙載齊名,江東謂之「韓、徐」。仕吳為校書郎,又仕南唐李
 昪父子,試知制誥,與宰相宋齊丘不協。時有得軍中書檄者,鉉及弟鍇評其援引不當。檄乃湯悅所作,悅與齊丘誣鉉、鍇洩機事,鉉坐貶泰州司戶掾,鍇貶為烏江尉,俄復舊官。



 時景命內臣車延規、傅宏營屯田於常、楚州,處事苛細,人不堪命,致盜賊群起。命鉉乘傳巡撫。鉉至楚州,奏罷屯田,延規等懼,逃罪,鉉捕之急,權近側目。及捕得賊首,即斬之不俟報,坐專殺流舒州。周世宗南征,景徙鉉饒州,俄召為太子右諭德,復知制誥,遷中書舍人。
 景死,事其子煜為禮部侍郎,通署中書省事,歷尚書在丞、兵部侍郎、翰林學士、御史大夫、吏部尚書。



 宋師圍金陵,煜遣鉉求緩兵。時煜將朱令贇將兵十餘萬自上江來援,煜以鉉既行,欲止令贇勿令東下。鉉曰:「此行未保必能濟難,江南所恃者援兵爾,奈何止之!」煜曰:「方求和解而復決戰,豈利於汝乎?」鉉曰「要以社稷為計,豈顧一介之使,置之度外可也。」煜泣而遣之。及至,雖不能緩兵,而入見辭歸,禮遇皆與常時同。及隨煜入覲,太祖責之,
 聲甚厲。鉉對曰:「臣為江南大臣,國亡罪當死,不當問其他。」太祖嘆曰:「忠臣也!事我當如李氏。」命為太子率更令。



 太平興國初,李昉獨直翰林,鉉直學士院。從征太原,軍中書詔填委,鉉援筆無滯,辭理精當,時論能之。師還,加給事中。八年,出為右散騎常侍,遷左常侍。淳化二年,廬州女僧道安誣鉉奸私事,道安坐不實抵罪,鉉亦貶靜難行軍司馬。



 初,鉉至京師,見被毛褐者輒哂之,邠州苦寒,終不御毛褐,致冷疾。一日晨起方冠帶,遽索筆手
 疏,約束後事,又別署曰:「道者,天地之母。」書訖而卒,年七十六。鉉無子,門人鄭文寶護其喪至汴,胡仲容歸其葬於南昌之西山。



 鉉性簡淡寡欲,質直無矯飾,不喜釋氏而好神怪,有以此獻者,所求必如其請。鉉精小學,好李斯小篆,臻其妙,隸書亦工。嘗受詔與句中正、葛湍、王惟恭等同校《說文》,《序》曰:



 許慎《說文》十四篇,並《序目》一篇,凡萬六百餘字,聖人之旨蓋云備矣。夫八卦既畫,萬象既分,則文字為之大輅,載籍為之六轡,先王教化所以行
 於百代,及物之功與造化均不可忽也。雖王帝之後改易殊體,六國之世文字異形,然猶存篆籀之跡,不失形類之本。及暴秦苛政,散隸聿興,便於末俗,人競師法。古文既變,巧偽日滋。至漢宣帝時,始命諸儒修倉頡之法,亦不能復。至光武時,馬援上疏論文字之訛謬,其言詳矣。及和帝時,申命賈逵修理舊文,於是許慎採史籀、李斯、揚雄之書,博訪通人,考之於逵,作《說文解字》,至安帝十五年始奏上之。而隸書之行已久,加以行、草、八分紛
 然間出,反以篆籀為奇怪之跡,不復輕心。



 至於六籍舊文,相承傳寫,多求便俗,漸失本原。《爾雅》所載草、木、魚、鳥之名,肆志增益,不可觀矣。諸儒傳釋,亦非精究小學之徒,莫能矯正。



 唐大歷中,李陽冰篆跡殊絕,獨冠古今,於是刊定《說文》,修正筆法,學者師慕,篆籀中興。然頗排斥許氏,自為臆說。夫以師心之獨見,破先儒之祖述,豈聖人之意乎?今之為字學者,亦多陽冰之新義,所謂貴耳而賤目也。



 自唐末喪亂,經籍道息。有宋膺運,人文國典,
 粲然復興,以為文字者六藝之本,當由古法,乃詔取許慎《說文解字》,精加詳校,垂憲百代。臣等敢竭愚陋,備加詳考。



 有許慎注義、序例中所載而諸部不見者,審知漏落,悉從補錄。復有經典相承傳,寫及時俗要用而《說文》不載者,皆附益之,以廣篆籀之路。亦皆形聲相從、不違六書之義者。



 其間《說文》具有正體而時俗論變者,則具於注中。其有義理乘舛、違戾六書者,並列序於後,俾夫學者無或致疑。大抵此書務援古以正今,不徇今而違
 古。若乃高文大冊,則宜以篆籀著之金石,至於常行簡牘,則草隸足矣。



 又許慎注解,詞簡義奧,不可周知。陽冰之後,諸儒箋述有可取者,亦從附益;猶有未盡,則臣等粗為訓釋,以成一家之書。



 《說文》之時,未有反切,後人附益,互有異同。孫愐《唐韻》行之已久,今並以孫愐音切為定,庶幾學者有所適從焉。



 鍇亦善小學,嘗以許慎《說文》依四聲譜次為十卷,目曰《說文解字韻譜》。鉉序之曰:



 昔伏羲畫八卦而文字之端見矣,蒼頡模鳥跡而文字之
 形立矣。史籀作大篆以潤色之,李斯變小篆以簡易之,其美至矣。及程邈作隸而人競趣省,古法一變,字義浸訛。先儒許慎患其若此,故集《倉》、《雅》之學,研六書之旨,博訪通識,考於賈逵,作《說文解字》十五篇,凡萬六百字。字書精博,莫過於是。篆籀之體,極於斯焉。



 其後賈魴以《三蒼》之書皆為隸字,隸字始廣而篆籀轉微。後漢及今千有餘歲,凡善書者皆草隸焉。又隸書之法有冊繁補闕之論,則其訛偽斷可知矣。故今字書之數累倍於前。



 夫
 聖人創制皆有依據,不知而作,君子慎之,及史闕文,格言斯在。若草、木、魚、鳥,形聲相從,觸類長之,良無窮極,茍不折之以古義,何足以觀?故叔重之後,《玉篇》、《切韻》所載,習俗雖久,要不可施之於篆文。往者,李陽冰天縱其能,中興斯學。贊明許氏,奐焉英發。然古法背俗,易為堙微。



 方今許、李之書僅存於世,學者殊寡,舊章罕存。秉筆操觚,要資檢閱,而偏傍奧密,不可意知,尋求一字,往往終卷,力省功倍,思得其宜。舍弟鍇特善小學,因命取叔重
 所記,以《切韻》次之,聲韻區分,開卷可睹。鍇又集《通釋》四十篇,考先賢之微言,暢許氏之玄旨,正陽冰之新義,折流俗之異端,文字之學,善矣盡矣。今此書止欲便於檢討,無恤其他,故聊存詁訓,以為別識。其餘敷演,有《通釋五音》凡十卷,貽諸同志云。



 鉉親為之篆,鏤板以行於世。



 鍇字楚金,四歲而孤,母方教鉉,未暇及鍇,能自知書。李景見其文,以為秘書省正字,累官內史舍人,因鉉奉使入宋,憂懼而卒,年五十五。李穆使江南,見其兄弟文章,
 嘆曰:「二陸不能及也!」



 鉉有文集三十卷,《質疑論》若干卷。所著《稽神錄》,多出於其客蒯亮。鍇所著則有文集、家傳、《方輿記》、《古今國典》、《賦苑》、《歲時廣記》云。



 句中正,字坦然,益州華陽人。孟昶時,館於其相毋昭裔之第,昭裔奏授崇文館校書郎,復舉進士及第,累為昭裔從事。歸朝,補曹州錄事參軍、汜水令,又為潞州錄事參軍。



 中正精於字學,古文、篆、隸、行、草無不工。太平興國二年,獻八體書。太宗素聞其名,召入,授著作佐郎、直史
 館,被詔詳定《篇》、《韻》。



 四年,命副張洎為高麗加恩使,還,遷左贊善大夫,改著作郎,與徐鉉重校定《說文》,模印頒行。太宗覽之嘉賞,因問中正,凡有聲無字有幾何?中正退,條為一卷以獻。上曰:「朕亦得二十一字,可並錄之也。」時又命中正與著作佐郎吳鉉、大理寺丞楊文舉同撰定《雍熙廣韻》。中正先以門類上進,面賜緋魚,俄加太常博士。《廣韻》成,凡一百卷,特拜虞部員外郎。



 淳化元年,改直昭文館,三遷屯田郎中,杜門守道,以文翰為樂。太宗神
 主及謚寶篆文,皆詔中正書之。嘗以大小篆、八分三體書《孝經》摹石,咸平三年表上之。真宗召見便殿,賜坐,問所書幾許時,中正曰:「臣寫此書,十五年方成。」上嘉嘆良久,賜金紫,命藏於秘閣。時乾州獻古銅鼎,狀方而四足,上有古文二十一字,人莫能曉,命中正與杜鎬詳驗以聞,援據甚悉。五年,卒,年七十四。



 中正喜藏書,家無餘財。子希古、希仲並進士及第,希仲太常博士。



 蜀人又有孫逢吉、林罕。逢吉嘗為蜀國子《毛詩》博士、檢校刊刻石經。罕
 亦善文字之學,嘗著《說文》二十篇,目曰《林氏小說》,刻石蜀中。



 曾致堯字正臣,撫州南豐人。太平興國八年進士,解褐符離主簿、梁州錄事參軍,三遷著作佐郎、直史館,改秘書丞,出為兩浙轉運使。嘗上言:「去歲所部秋租,惟湖州一郡督納及期,而蘇、常、潤三州悉有逋負,請各按賞罰。」太宗以江、淮頻年水災,蘇、常特甚,所言刻薄不可行,詔戒致堯毋擾。俄徙知壽州,轉太常博士。



 致堯性剛率,好
 言事,前後屢上章奏,辭多激訐。真宗即位,遷主客員外郎、判鹽鐵勾院。張齊賢薦其材,任詞職,命翰林試制誥,既而以輿議未允而罷。



 李繼遷擾西鄙,靈武危急,命張齊賢為涇、原、邠、寧、環、慶等州經略使,選致堯為判官,仍遷戶部員外郎。既受命,因抗疏自陳,願不受章紱之賜,詞旨狂躁。詔御史府鞫其罪,黜為黃州副使,奪金紫。未幾,復舊官,改吏部員外郎,歷知泰、泉、蘇、揚、鄂五州。大中祥符初,遷禮部郎中,坐知揚州日冒請一月奉,降掌升
 州榷酤,轉戶部郎中。五年,卒,年六十六。



 致堯頗好纂錄,所著有《仙鳧習翼》三十卷、《廣中臺志》八十卷、《清邊前要》三十卷、《西陲要紀》十卷、《為臣要紀》一十五篇。子易從、易占皆登進士第。



 刁衎,字元賓,升州人。父彥能,仕南唐為昭武軍節度。衎用蔭為秘書郎、集賢校理,衣五品服,以文翰入侍,甚被親暱。李煜嘗令直清輝殿,閱中外章奏。



 金陵平,從煜歸宋,太祖賜緋魚,授太常寺太祝。稱疾,假滿,屏居輦下者
 數歲。太平興國初,李昉、扈蒙在翰林,勉其出仕,因撰《聖德頌》獻之。詔復本官,出知睦州桐廬縣。



 會詔群臣言事,衎上《諫刑書》,謂:



 淫刑酷法非律文所載者,望詔天下悉禁止之。巡檢使臣捕得盜賊、亡卒,並送本部法官訊鞫,無得擅加酷虐。古者投奸兇於四裔,今遠方囚人盡歸京闕,以配務役,最非其宜。且神皋勝地,天子所居,豈使流囚於此聚役。自今外處罪人,望勿許解送上京,亦不留於諸務充役。



 又《禮》曰:「刑人於市,與眾棄之。」則知黃屋
 紫宸之中,非用刑行法之處。望自今御前不行決罰之刑,殿前引見司鉗黥法具,並赴御史臺、廷尉之獄;敕杖不以大小,皆引赴御史、廷尉。京府或出中使,或命法官,具禮監科,以重聖皇明刑慎法之意。



 或有犯劫盜亡命,罪重者刖足釘身,國門布令。此乃小民昧於刑憲,逼於衣食,偶然為惡,義不及他,被其慘毒,實傷風化,亦望減除其法。如此則人情不駭,各固其生;和氣無傷,必臻上瑞。



 再遷大理寺丞,獻文四十篇。召試,授殿中丞、通判湖
 州,上疏請定天下酒稅額、修郡縣城隍、條約牧宰、除兩浙丁身錢、禁汴水流尸,凡五事。俄知婺州,遷國子博士。會考校百官殿最,衎被召,以無過,得知光州,就改虞部員外郎。轉運使狀其政績,優詔嘉獎,徙知廬州。



 真宗即位,遷比部員外郎。嘗上疏曰:



 臣聞天下,大器也;群生,眾畜也。治大器者執一以正其度,保眾畜者齊化以臻其原。故至人謂莫神於天,莫富於地,莫大於帝王。又曰:帝王乘地而總萬物,以用人也。則知萬乘之尊,一人之位,
 等天地之覆燾,若日月之照臨,可不慎思慮以安民,系慘舒而被物!所以堯、舜篤善道以垂化,而民謂之所天;桀、紂懷兇德以害世,而民謂之獨夫。則君之於民,善惡有如是之驗;民之於君,毀譽有如是之異。



 陛下纂圖茲始,布政惟新,所宜上順天心,下從人欲,進善以去惡,避毀而求譽。遵唐、虞之治,斥辛、癸之亂,私賞無及於小人,私罰無施於君子,任賢勿貳,去邪勿疑。開諫諍之門,塞讒佞之口,愛而知其惡,憎而知其善,無以春秋鼎盛而
 耽於逸游,無以血氣方剛而惑於聲色。若太祖之勤儉,若太宗之惠慈,答天地敷錫之意,保祖宗艱難之業,則周成、漢文二宗之美,不可同年而議擬也。



 代還,獻所著《本說》十卷,得以本官充秘閣校理,出知潁州。入為比部員外郎,改直秘閣,充崇文院檢討。時杜鎬、陳彭年並預檢討、衎言此二人可專其任,詔許解職,判三司開拆司,預修《冊府元龜》,加主客郎中。求領外任,得知湖州,轉刑部郎中。歲滿,復預編修。大中祥符六年,書成,授兵部郎
 中。入朝,暴中風眩,真宗遣使馳賜金丹,已不救,年六十九。



 衎始仕李氏,權勢甚盛。父為藩帥,家富於財,被服飲膳,極於侈靡。歸宋,以純澹夷雅知名於時,恬於祿位,善談笑,喜棋弈,交道敦篤,士大夫多推重之。



 子湛、湜、渭,皆登進士第。湛,刑部郎中;湜,屯田員外;渭,太常博士。湛子繹、約,天聖中並進士及第。



 姚鉉,字寶之,廬州合肥人。太平興國八年進士甲科,解褐大理評事,知潭州湘鄉縣,三遷殿中丞,通判簡、宣、升
 三州。淳化五年,直史館,侍宴內苑,應制賦《賞花釣魚詩》,特被嘉賞,翌日,命中使就第賜白金以獎之。



 至道初,遷太常丞,充京西轉運使,歷右正言、右司諫、河東轉運使。俄上言曰:「伏見諸路官吏,或強明蒞事、惠愛及民者,則必立教條,除其煩擾。然狡胥之輩,非其所便,俟其罷官,悉藏記籍,害公蠹政,莫甚於此。《禮》云:『其人存則其政舉,其人亡則其政息。』又《語》曰:『舊令尹之政必告新令尹。』斯實聖人之格言,國家之急務也。欲望所在官吏,有經畫
 利濟事可長久者,歲終書歷,受代日錄付新官,俾之遵守。若事有灼然匪便,聽上聞,俟報改正。」詔從之。



 咸平三年,河決鄆州王陵埽,東南注鉅野,入淮、泗,城中積水壞廬舍,以鉉知州事,徙州於汶陽鄉之高原,委以營度,許便宜從事。工畢,加起居舍人、京東轉運使,徙兩浙路。



 鉉雋爽,頗尚氣。薛映知杭州,與之不協,事多矛盾。映摭鉉罪狀數條,密以聞,詔使劾之,當奪一官,特除名,貶連州文學。吉州之萬安抵虔,江有贛石,舟行其中,湍險萬狀,
 鉉過,感而賦之以自況。大中祥符五年,會赦,移嶽州,又移舒州,俄授本州團練副使。天禧四年卒,年五十三。



 鉉文辭敏麗,善筆札,藏書至多,頗有異本,兩浙課吏寫書,亦薛映所掎之一事。雖被竄斥,猶傭夫荷擔以自隨。有集二十卷。又採唐人文章纂為百卷,目曰《文粹》。卒後,子嗣復以其書上獻,詔藏內府,授嗣復永城主簿。幼子稱,俊穎美秀,頗善屬辭,裁十歲卒。鉉紀其事為《聰悟錄》,人多傳之。



 李建中,字得中,其先京兆人。曾祖逢,唐左衛兵曹參軍。祖稠,梁商州刺史,避地入蜀。會王建僭據,稠預佐命功臣,左衛將軍。建中幼好學,十四丁外艱。會蜀平,侍母居洛陽,聚學以自給。攜文游京師,為王祐所延譽,館於石熙載之第,熙載厚待之。



 太平興國八年進士甲科,解褐大理評事,知嶽州錄事參軍。轉運使李惟清薦其能,再遷著作佐郎、監潭州茶場,改殿中丞,歷通判道、郢二州。柴成務領漕運,再表稱薦,轉太常博士。時言事者多以
 權利進,建中表陳時政利害,序王霸之略,太宗嘉賞,因引對便殿,賜以緋魚。會考課京朝官,建中舊坐公累罰金,漏其事,坐降授殿中丞,監在京榷易院。蘇易簡方被恩顧,多得對,嘗言蜀中文士,因及建中,太宗亦素知之,命直昭文館。建中父名昭文,懇辭,改集賢院。數月,出為兩浙轉運副使,再遷主客員外郎,歷通判河南府,知曹、解、潁、蔡四州。景德中,以久次,進金部員外郎。



 建中性簡靜,風神雅秀,恬於榮利,前後三求掌西京留司御史臺,
 尤愛洛中風土,就構園池,號曰「靜居」。好吟詠,每游山水,多留題,自稱巖夫民伯。加司封員外郎、工部郎中。建中善修養之術,會命官校定《道藏》,建中預焉。又判太府寺。大中祥符五年冬,命使泗州,奉御制《汴水發願文》,就致設醮。使還得疾,明年卒,年六十九。



 建中善書札,行筆尤工,多構新體,草、隸、篆、籀、八分亦妙,人多摹習,爭取以為楷法。嘗手寫郭忠恕《汗簡集》以獻,皆科斗文字,有詔嘉獎。好古勤學,多藏古器名畫。有集三十卷。



 子周道、周士
 並進士及第。周士歷侍御史、江東、陜西轉運、三司鹽鐵判官,賜金紫,終工部郎中。周民,太子中舍。



 洪湛,字惟清,升州上元人。曾祖勛,南唐崇文館直學士。祖壽,桐城令。父慶元,獻書李煜,授奉禮郎,補新喻令。歸宋,至冤句令。湛幼好學,五歲能為詩,未冠,錄所著十卷為《齠年集》。舉進士,有聲。雍熙二年,廷試已落,復試,擢置高等,解褐歸德軍節度推官。召還,授右拾遺、直史館。



 端拱初,通判壽、許二州。歸宋,與左正言尹黃裳、馮拯、右正
 言王世則、宋沆伏閣請立許王元僖為儲貳,詞意狂率,太宗怒。時沆坐呂蒙正親黨,已出為宜州團練副使。上因語近臣曰:「儲副,邦國之本,朕豈不知。但近世淺薄,若立太子,即東宮僚屬皆須稱臣,官職聯次與上臺無異,人情深所不安。此事朕自有時爾。」湛坐削職,出知容州,黃裳知邕州,拯知端州,沆知靖州,世則知蒙州。容之戍卒謀竊發者,湛偵知,亟斬之。再遷比部員外郎,知郴、舒二州。



 咸平二年召還,命試舍人院,復直史館。是秋,命與
 閣門祗候韓紹輝使荊湖按視民事,條奏利病甚眾。還,判三司都磨勘司。又與王欽若同知貢舉,未幾,同修起居注。時議城綏州,邊臣互言利害,遣湛與閣門祗候程順奇同往按視,湛言城之利有七而害有二,遂詔營葺,終以勞人罷之。



 湛美風儀,俊辯有材幹,凡五使西北議邊要。真宗有意擢任,顧遇甚厚。曲宴苑中,賦賞花詩,不移晷以獻,深被褒賞。



 五年春,有河陰民常德方訟臨津尉任懿納賄登第,事下御史臺,鞫得懿款云:「咸平二年,補
 太學生,寓僧仁雅舍,因仁雅求院之主僧惠秦為道地,署紙許銀七鋌,仁雅、惠秦隱其二,易為五鋌。惠秦素識王欽若已在貢院,乃因館客寧文德、僕夫徐興納署紙於欽若妻李,李密召家僕祁睿書懿名於左臂,並口傳許賂之數,入省告欽若。及懿過五場,睿復持湯飲至省,欽若遣睿語李,令取其銀,懿未即與。既而懿預奏名授官,未行,丁內艱,還鄉里。仁雅馳書索銀,形於詛罵。」德方者,賣卜縣市,獲其書,以告中丞趙昌言,具其事奏白,請
 逮欽若屬吏。



 先是,欽若為亳州判官,睿其廳乾,及代歸,以睿從行而未除州之役籍。及貢舉事畢,會州人張續還鄉行服,托為睿去籍名。至是,欽若訴云:「睿休役之後,始傭於家,而惠秦未嘗及門。」欽若方被寵顧,乃詔翰林侍讀學士邢昺、內侍副都知閻承翰並驛召知曹州邊肅、知許州毋賓古就太常寺別鞫,懿易款云:「有妻兄張駕舉進士,識湛,懿亦與駕同造湛門,嘗以石榴二百枚、木炭百秤饋之。懿之輸銀也,但憑二僧達一主司,
 實不知誰何?」乃以為湛納其銀。湛適使陜西,中途召還,時張駕已死,寧文德、徐興悉遁去,欽若近參機務,門下僕使多新募至,不識惠秦,故無與左證。又固執知舉時未有祁睿,遂以湛受銀,法當死,特詔削籍、流儋州。懿杖脊、配隸忠靖軍。惠秦坐受簡札及隱銀未入已,以年七十餘,當贖銅八斤,特杖一百,黥面配商州坑冶。仁雅杖脊,配隸郢州牢城,而不窮用銀之端。



 初,王旦與欽若知舉,出拜樞密副使,以湛代領其事。湛之入貢院,懿已試第三
 場畢,及官收湛贓,家實無物。湛素與梁顥善,或假顥白金器,乃取以輸官。六年,會赦移惠州,至化州調馬驛卒,年四十一。



 湛時一子偕行,甚幼,州以聞,特詔賜錢二萬,官為護喪還揚州。因詔命官配流嶺外而沒者,悉給緡錢,聽其歸葬,如親屬幼稚者,所在遣牙校部送之。湛有集十卷。



 子鼎,大中祥符四年進士,至度支員外郎、直史館、鹽鐵判官。



 路振,字子發,永州祁陽人,唐相巖之四世孫。巖貶死嶺
 外,其子琛避地湖湘間,遂居焉。振父洵美事馬希杲,署連州從事,謝病終於家。振幼穎悟,五歲誦《孝經》、《論語》。十歲聽講《陰符》,裁百言而止,洵美責之,俾終其業。振曰:「百言演道足矣,餘何必學?」洵美大奇之。十二丁外艱,母氏慮其廢業,日加誨激,雖隆冬盛署,未始有懈。



 淳化中舉進士,太宗以詞場之弊,多事輕淺,不能該貫古道,因試《卮言日出賦》,觀其學術。時就試者凡數百人,咸𥈭眙忘其所出,雖當時馳聲場屋者亦有難色。振寒素,游京師
 人罕知者,所作賦尤為典贍,太宗甚嘉之。擢置甲科,釋褐大理評事,通判邠州,徙徐州。召還,直史館,復遣之任,遷太子中允、知濱州。一日,契丹至城下,兵少,民相恐,眾謂振文吏,無戰御方略,環聚而泣。振乃親加撫諭,且以敵盛不可與爭鋒,宜堅壁自守。數日,契丹引去。轉運使劉綜稱其能,詔書褒美。



 常作《祭戰馬文》曰:



 咸平中,契丹犯高陽關,執大將康保裔,略河朔而去。天子幸魏,特遣將王榮以五千騎追之。榮無將材,但能走馬,以馳射
 為事,受命恇怯,數日不敢行,伺賊渡河而後發。有剽淄、齊者數千騎尚屯泥沽,榮不欲見敵,遂以其騎略河南岸而還。晝夜急騎,馬不秣而道斃者十有四五,天子憫之,遣使收瘞焉。因作祭文曰:



 房駟之精,降為驪騂。飲泉呀風,流沙激霆。虎脊孤聳,龍媒鷙獰。丹髦曉霞,的顙秋星。茀方著乾,宜乘旋膺。巉臚角起,方背珠明。



 爾其絕塞草荒,八月隕霜。毛縮蹄堅,筋舒脈張。獸惡恐噬,虯獰欲驤。噴沙散沫,千里飛雪。圉人負紖,武士索鐵。前遮後突,
 雷動地裂。急挽一而制百,終伏撾而受紲。牧官劬劬,歲入券書。蹄鉅累累,通乎鬼區。名駒大𩡺,銜尾入塞。勞其酋長,節以駔儈。蜀錦吳繒,積如丘陵。馬歸於我也重,幣入於彼也輕。



 於是絡黃金之羈,浴天池之波。鼓鬣雲衢,弄影星河。或踶而嚙,或嗅而吪。原蠶申禁,駔駿何多。帝念神物,來經遠道。閱之於內殿,養之於外皂,飲以玉池,秣之瑤草。



 窮冬邊塵,入我河漘。羽書宵飛,龍馭北巡。選仗下之名馬,屬閫外之武臣。琱戈電燭,禁旅星陳。授以
 長策,帥以全軍。壯士怒兮山可擘,猛馬哮兮虎可咋。何嚄唶之無勇,反遷延而避敵。



 冰霜淒淒,介甲而馳。不飲不秣,載渴載饑。駿馬餒死,行人嗟咨。委天骨於衢路,反星精於雲霧。報主恩之無及,齊戎力而何誤。生芻致祭,弊帷成禮。瘞於崇岡,全爾具體。馬如有神,知帝之仁。嗚呼!



 又以西兵未弭,入判大理寺,改太常丞、知河中府,徙知鄧州。代還,判吏部南曹三司催欠憑由司。景德中使福建巡撫,俄判鼓司登聞院。會修《兩朝國史》,以振為編
 修官。大中祥符初,使契丹,撰《乘軺錄》以獻。改太常博士、左司諫,擢知制誥。



 振文詞溫麗,屢奏賦頌,為名輩所稱,尤長詩詠,多警句。及居文翰之職,深愜物議,自是彌加精厲。從祀譙、亳,時同職分局掌事,振獨直行在,專典綸翰,箋奏填委,應用無滯,時推其敏贍。七年,同修起居注,張復、崔遵度以書事誤失降秩,擇振與夏竦代之。嗜酒得疾,其冬卒,年五十八。錄其子綸為太常寺奉禮郎。



 振純厚無城府,恂恂如也,時人惜其登用之晚。有集二十
 卷。又嘗採五代末九國君臣行事作世家、列傳,書未成而卒。



 崔遵度,字堅白,本江陵人,後徙淄州之淄川。純介好學,始七歲,授經於叔父憲,嘗以《春秋》編年、《史》、《漢》紀傳之例問於憲,憲曰:「此兒他日成令名矣。」太平興國八年舉進士,解褐和州主簿,換臨汾。饋芻糧,三抵綏州,涉無定河,河沙與水混流無定跡,陷溺相繼,遵度憫之,著銘以紀焉。端拱初,轉運副使夏侯濤上其勤狀,召歸,對便坐,因
 獻文自薦。時新建秘閣,命中書試作頌一首,擢著作佐郎。



 淳化中,吏部侍郎李至薦之,遷殿中丞,出知忠州。李順之亂,賊遣其黨張餘來攻,遵度領甲士百餘背城而戰,賊逾堞以入,遵度投江中,賴州兵援之,得免。坐失城池,貶崇陽令,移鹿邑。咸平初,復為太子中允。景德初,內出遵度名,引對崇政殿,詔索所著文,召試舍人院,改太常丞、直史館。會修《兩朝國史》,與路振並為編修官。大中祥符元年,命同修起居注。東封,進博士,祀汾陰,是歲,真
 宗以兩省官絕少,故因覃慶選補之,命為左司諫。



 遵度與物無競,口不言是非,淳澹清素,於勢利泊如也。掌右史十餘歲,立殿墀上,常退匿楹間,慮上之見。善鼓琴,得其深趣。所僦舍甚湫隘,有小閣,手植竹數本,朝退,默坐其上,彈琴獨酌,翛然自適。嘗著《琴箋》云:



 世之言琴者,必曰長三尺六寸象期之日,十三徽象期之月,居中者象閏,前世未有辨者。至唐協律郎劉貺以樂器配諸節候,而謂琴為夏至之音。至於泛聲,卒無述者,愚嘗病之。因張
 弓附案,泛其弦而十三徽聲具焉,況琴瑟之弦乎!是知非所謂象者,蓋天地自然之節耳,又豈止夏至之音而已。



 夫《易》有太極,是生兩儀。兩儀者,太極之節也;四時者,兩儀之節也;律呂者,四時之節也;晝夜者,律呂之節也;刻漏者,晝夜之節也。節節相交,自細至大而歲成焉。既不可使之節,亦不可使之不節,氣之自然者也。氣既節矣,聲同則應,既不可使之應,亦不可使之不應,數之自然者也。既節且應,則天地之文成矣。文之義也,或任形
 而著,或假物而彰。日星文乎上,山川理乎下,動物植物,花者節者,五色具矣。斯任形者也。至於人常有五性而不著,以事觀之然後著;日常有五色而不見,以水觀之然後見;氣常有五音而不聞,以弦考之然後聞。斯假物者也。



 是故聖人不能作《易》而能知自然之數,不能作琴而能知自然之節。何則?數本於一而成於三,因而重之,故《易》六畫而成卦。及其應也,一必於四,二必於五,三必於六焉。氣氣相召,其應也必矣。卦既畫矣,故畫琴焉。始
 以一弦泛桐,當其節則清然而號,不當其節則泯然無聲,豈人力也哉!且徽有十三,而居中者為一。自中而左泛有三焉,又右泛有三焉,其聲殺而已,弦盡則聲減。及其應也,一必於四,二必於五,三必於六焉,節節相召,其應也必矣。



 《易》之書也,偶三為六,三才之配具焉,萬物由之而出。雖曰六畫,及其數也,止三而已矣。琴之畫也,偶六而根於一,一鐘者,道之所生也。在數為一,在律為黃,在音為宮,在木為根,在四體為心,眾徽由之而生。雖曰
 十三,及其節也,止三而已矣。卦之德方,經也;蓍之德圓,緯也;故萬物不能逃其象。徽三其節,經也;弦五其音,緯也;故眾音不能勝其文。先儒謂八音以絲為君,絲以琴為君。愚謂琴以中徽為君,盡矣。夫徽十三者,蓋盡昭昭可聞者也。茍盡弦而考之,乃總有二十三徽焉,是一氣也。丈弦具之,尺弦亦具之,豈有長短大小之限哉!



 是則萬物本於天地,天地本於太極,太極之外以至於萬物,聖人本於道,道本於自然,自然之外以至於無為,樂本
 於琴,琴本於中徽,中徽之外以至於無聲。是知作《易》者,考天地之象也;作琴者,考天地之聲也。往者藏音而未談,來者專聲而忘理。《琴箋》之作也,庶乎近之。茍其闕也,請俟君子。



 世稱其知言。



 七年,東郊,建壇恭謝。壇上設正坐奉天地,配坐奉二聖。遵度時與張復同典記注,書昊天為天皇,又增聖祖配位,坐謬誤,降為右正言,復亦責為工部郎中。逾歲,並復其秩。



 九年,仁宗以壽春郡王開府,詔宰相擇耆德方正有學術之士,咸曰遵度力學,有
 士行,時稱長者,遂命與張士遜並為王友。改戶部員外郎,賜服金紫,又賚襲衣、犀帶、緡錢。上作七言詩寵之。因謂左右曰:「翊善、記室,皆府屬也,故王皆受拜,今賓友之禮,當令答拜。」府中文翰皆遵度所作。王讀《孝經》徹章,復以御詩賜之。國史成,拜吏部員外郎,升邸進封,改禮部郎中,充諮議參軍。儲宮建,又加吏部兼左諭德。未幾,命使契丹,判司農寺。



 遵度性寡合,喜讀《易》,嘗云:「意有疑,則彈琴辨其數,筮《易》觀其象,無不究也。」



 天禧四年八月卒,
 年六十七。其子拜官者二人。仁宗即位,特詔贈工部侍郎,又授其二孫官,有集二十卷。



 陳越,字損之,開封尉氏人。祖守危,興道令。父夏,虞部員外郎。越少好學,尤精歷代史。善屬文,辭氣俊拔。咸平中,詔舉賢良,刑部侍郎郭贄薦之,策入第四等,解褐將作監丞、通判舒州,徙知端州,又徙袁州。未幾召還,遷著作佐郎、直史館,掌鼓司登聞院。預修《冊府元龜》,與陳從易、劉筠尤為勤職。真宗以其奉薄,並命月增錢五千。車駕
 朝陵,掌留司名表,時稱為工。自是兩府箋奏多命草之,勛貴家以銘志為請者甚眾。遷太常丞、群牧判官。祀汾陰,擢為左正言。



 越耿概任氣,喜箴切朋友,放曠杯酒間,家徒壁立,不以屑意。然嗜酒過差,每食必先引數升,罕有醒日,亦用是遘疾。大中祥符五年卒,年四十。無子,母老,人皆傷之。



 越兄咸,嘗舉進士未第,楊億、杜鎬、陳彭年列奏為言,真宗憫之。及《冊府元龜》奏御,特賜咸同《三傳》出身。



 故事,中書章表皆舍人為之,東封後,朝廷多慶禮,
 舍人或以他務所嬰,乃擇館閣官,得盛度、路振、劉筠、夏竦、宋綬洎越分撰表奏,宰相嘗以名聞,其後皆相次掌外制,唯越不及登擢,時論惜之。



\end{pinyinscope}