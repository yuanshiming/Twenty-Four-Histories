\article{列傳第六十一}

\begin{pinyinscope}

 王
 臻魚周詢賈黯李京吳鼎臣附呂景初馬遵附吳及範師道李絢何中立沉邈



 王臻,字及之,穎州汝陰人。始就學,能文辭。曾致堯知壽州,有時名,臻以文數十篇往見,致堯覽之,嘆曰:「穎、汝固多奇士。」舉進士第,為大理評事,歷知舒城、會昌縣,通判徐、定二州,以殿中丞知兗州,特遷監察御史。



 中使就營景靈宮、太極觀,臻佐助工費有勞,遷殿中侍御史,擢淮南轉運副使。時發運司建議浚淮南漕渠,廢諸堰,臻言:「揚州召伯堰,實謝安為之,人思其功,以比召伯,不可廢也。浚渠亦無所益。」召為三司度支判官,而發運司卒
 浚渠以通漕,臻坐前異議,降監察御史、知睦州。道復官,徙福州。閩人欲報仇,或先食野葛,而後趨仇家求斗,即死其處,以誣仇人。臻辨察格鬥狀,被誣者往往釋去,俗為之少變。又民間數以火訛相驚,悉捕首惡杖之,流海上,民乃定。



 仁宗即位,遷提舉在京諸司庫務,歷三司戶部、度支副使,擢龍圖閣待制、權知開封府,累遷尚書工部郎中。奸人偽為皇城司刺事卒,嚇民以取賕,臻購得其主名,黥竄三十餘人,都下肅然。以右諫議大夫權
 御史中丞,建言:「三司、開封府諸曹參軍及赤縣丞尉,率用貴游子弟,驕惰不習事。請易以孤寒登第、更仕宦書考無過者為之。」又言:「在京百司吏人入官,請如《長定格》,歸司三年。」皆可其奏。未幾,卒。臻剛嚴善決事,所至有風跡。



 魚周詢,字裕之,開封雍丘人。早孤,好學。舉進士中第,為大理評事,歷知南華、分宜、靜海三縣,遷太常博士、通判漢州。城中夜有火,部眾救之,植劍於前曰:「攘一物者斬!」
 火止,民無所失亡。以尚書屯田員外郎知真州,徙提點荊湖南路刑獄。求便郡,知安州,徙蔡州,召為侍御史。陜西用兵,科斂煩數,命安撫京西路,還賜緋衣銀魚。為開封府判官,又使陜西刺民兵,判三司理欠、憑由司。進起居舍人、知諫院,固辭,乃以尚書戶部員外郎兼侍御史知雜事,為三司鹽鐵副使。時渭州城水洛,尹洙、鄭戩爭未決,詔周詢與都轉運使程戡相利害。周詢是戩議,遂城之。遷吏部員外郎,擢天章閣待制、知成德軍,徙河北
 都轉運使,拜右諫議大夫、權御史中丞。



 慶歷八年,手詔近臣訪天下之務。周詢對曰:



 陛下患西陲御備,天下繹騷,趣募兵士,急調軍食,雖常賦有增,而經用不足。臣以謂唐季及五代,強臣專地,中國所制,疆域非廣。及祖宗有天下,俘吳、楚、蜀、晉,北捍獯鬻,西服羌戎,所用甲兵,所入租賦,比之於今,其數尚寡。然而摧堅震敵,庫府無空虛之弊,縣官無煩費之勞,蓋賞信罰必,將選兵精之效也。近元昊背惠,西方宿師。朝廷用空疏闒茸者為偏裨,
 以游惰怯懦者備行伍,故大舉即大敗,小戰輒小奔。徒日費千金,度支不給,賣官鬻爵,淆雜仕流,以鐵為錢,隳壞國法。而又官立鹽禁,驅民繼輦,蕩析恆產,怨咨盈路。去秋水旱繼作,今春饑饉相屬,生靈重困,於茲為劇。今元昊幼子新立,乃朝廷寬財用、惜民力之時也,速宜經度,以紓匱乏。願委安撫使與本路守邊、掌計臣僚同議,裁減冗兵,節抑浮費,禁止橫斂,廩假貧民,去武臣之庸懦,出守宰之貪殘。仍冀特發宸衷,出內帑錢助關陜費,
 使通鹽商之利,改錢幣之法,宣布德澤,與民休息。然後勸勉農桑,隱括稅籍,收遺利,抑兼並,則公有羨財,私有餘力矣。



 陛下患承平浸久,仕進多門,人污政濫,員多闕少,滋長奔競,靡費廩祿。臣以謂國家於制舉、進士、明經之外,復有任子、流外之補,負瑕釁、服輿臺者,亦置班列。歷年既久,紛猥塞路,求人任事,適用者鮮,而又亟更數易,交錯道塗,額置有常,詔除無限,凡守一闕,動逾再期。預閫籍、服武弁者,坐費水衡之給,虛計歲考之期;赴銓
 調、守選格者,居多困乏之嘆,行寡廉恥之風。官冗之弊,一至於此!願陛下特詔,進士先取策論,諸科兼通經義,中第解褐,無令過多。其文武班奏薦並流外出官者,權停五七年,自然名器不濫,奔競衰息矣。



 陛下患牧守之職,罕聞奏最。臣聞漢宣帝勉厲二千石,其有治效者,增秩賜金,或爵至關內侯,公卿缺,則以次用之,故良吏為盛。國家鑒諸侯專地之患,一切用郡守治之。而班行浸冗,序遷者眾。乃有地處藩宣,秩為卿監,而未歷省府提
 轉,則為沉抑。內重外輕,何以求治?改弦易轍,正在此時。願詔兩府大臣,選委兩制、臺諫官參舉,如兩任通判可充知州軍京朝官,依次除補。若治狀尤異,即升省府提轉。其常例入知州者,一切停罷,則進擢得人,牧守重矣。



 陛下患將帥之任,艱於稱職。臣聞晏子薦司馬穰苴曰:「文能附眾,武能威敵。」是知將帥之材,非文武兼備,則不可為。我朝自二邊款附,久不用兵。近歲有西北之警,補授帥臣,出於遽猝,非自卒伍,即恩澤侯。無信義以結士
 心,無莊嚴以正師律,退則奔北,進則被擒,虧損威靈,取侮夷狄,命將之失,未有若今之甚也。願擇名臣,選舉深博有謀、知兵練武之士,不限資級,試以邊任,臨軒敦遣,假以威權,如祖宗朝任郭進、李漢超輩,閫外之事,俾得專之,無以謗讒輕有遷徙,使其足以取重,則安有不稱職之憂乎?



 陛下患西北多故,邊情罔測,獻奇譎空言者多,陳悠久實效者少,備豫不虞,理當先物。臣聞國家和約北戎,爵命西夏,偃革止戈,逾四十載。而守邊多任庸
 人,不嚴武備,因循姑息,為敵所窺,致元昊悖逆,耶律張皇。未免屈己為民,息兵講好,皆用茍安之謀,而無經遠之策。此班固所謂「不選武略之臣,恃吾所以待寇而行貨賂,割剝百姓以奉寇仇」者也。願陛下特議減三路兵馬之駑冗者,以紓經費,以息科斂。然後選將帥,擇偏裨,使戢肅驕兵,飭利戎器,識山川形勝,用兵奇正。河朔曠平,可施車陣,亦宜講求其法。雖二邊異時侵軼,恃吾有以待之,庶幾無患矣。



 時執政及近臣所對多疏闊,仁宗
 頗嘉周詢詳敏。知恩州張得一誅,坐失舉,出知永興軍;數日,改知成德軍,未行,卒。帝嗟悼之,特贈尚書工部侍郎。



 周詢性和易,聞見該洽,明吏事。在安州時,園吏見大蛇垂闌楯,即視之,乃周詢醉而假寐,世傳其異。



 賈黯,字直孺,鄧州穰人。擢進士第一,起家將作臨丞、通判襄州。還為秘書省著作佐郎、直集賢院,遷左正言、判三司開拆司。



 黯自以年少遭遇,備位諫官,果於言事。首論韓琦、富弼、範仲淹可大用。杜樞覆張彥方獄,將駁正,
 忤執政意,執政以他罪絀樞。黯言:「樞無罪,且旨從中出,不因臣下彈奏。恐自此貴幸近習,言一得入,則將陰肆讒毀,害及善良,不可不察。」時言者或論事亡狀,輒戒勵窮詰。黯奏:「諫官、御史,跡既疏遠,未嘗預聞時政,不免採於傳聞,一有失言,而詰難沮辱隨之,非所以開廣言路。請如唐太宗用王珪、魏徵故事,每執政奏事,聽諫官一人隨入。」執政又患言事官旅進,論議上前不肯止。乃詔:「凡欲合班上殿者,皆稟中書俟旨。」黯論以為:「今得進見
 言事者,獨諫官、御史,若然,言路將壅,陛下不得聞外事矣。請如故便。」皆弗許。



 儂智高反,餘靖知桂州,楊畋安撫廣南東、西路,皆許便宜行事。黯言:「二人臨事,指蹤不一,則下將無所適從。又靖專節制西路,若賊東向,則非靖所統,無以使眾,不若並付靖經制兩路。」從之。皇祐四年,同修起居注,徙判鹽鐵勾院,遷左司諫。建言天下復置義倉,下其說諸路,而論者不一,黯亦反復辨析,卒不果行。宰相劉沆請中外薦舉陳乞,一切以詔令從事,毋用
 例。論者以為非便,黯奏罷之。狄青除樞密副使,黯言:「國初武臣宿將,扶建大業,平定列國,有忠勛者,不可勝數。然未有以卒伍登帷幄者。」不報。會靈觀災,又言:「天意所欲廢,當罷營繕,赦守衛者罪,以示儆懼修省之意。」擢知制誥。



 初,仁宗視事退,御邇英閣,召侍臣講讀,而修起居注官獨先出。黯言:「君臣訪對,動關政體,而史臣不得預聞,請並召侍經筵。」許之。初,邇英、延義二閣,講讀官自有記注。至是,乃罷焉。直龍圖閣錢延年擢天章閣待制,黯
 當命辭,即詆延年不才,不宜污侍從,封詞目還中書,命遂寢。



 判吏部流內銓。益州推官桑澤父留鄉里,死三年矣。澤為弗知者而調京師,既覺而去。黯奏劾,廢終身。福州推官劉抃挾數術,言人禍福,多游公卿門,黯奏以為靈臺郎。



 時詔兩制、兩省官唯公事許至中書、樞密院見執政,群臣心知其非,而嫌於自言。後黯知許州,乃言:「他官皆得見執政,而侍從近臣,反疏斥疑間如此。嘗聞先朝用王禹偁請,百官候謁宰相,並於政事堂,樞密使亦
 須聚坐接見,以防請托。令下,左正言謝泌上書,以謂非人主推赤心待大臣,大臣展四體報人主之誼。」實時追寢前詔。



 徙襄州,迎父之官,而父有故人在部中,遣直廳卒致問。黯輒笞卒,父恚,一夕歸鄉里。他日,疾且亟,黯內懷不自安,請徙郡及解官就養。不報,乃棄官去。而御史吳中復等劾黯輒委州印,撓朝廷法,絀知郢州。未及行,父死。服除,勾當三班院,為翰林學士。唐介等坐言陳升之不當柄用,皆外補。黯奏介等敢言,請寬之。以疾請郡,
 改侍讀學士、知鄧州。未行,疾愈,復以為翰林學士、知審官院。



 時官吏有以祖父嫌名,援律為請授他官。黯言:「禮不諱嫌名,二名不偏諱,律:『府號、官稱犯祖父名而冒榮居之,又上書若奏事犯祖廟諱,罪皆有差。』又曰:『若嫌名及二名偏犯者,不坐。』今官吏許避嫌名,則或有如此而不自言者,可坐以冒榮之律乎?國朝雍熙中,嘗詔:『除官犯私諱者,三省御史臺五品、文班四品以上,許用式奏改,餘不在此制。』請約雍熙詔書,自某品而上,以禮律從
 事。」詔非嫌名及二名,不以品秩高下皆聽避。



 累遷尚書左司郎中、權知開封府。兩軍獄囚歲瘐死者眾,而吏不任其責。黯言:「吏或怠於視囚,饑渴疾病,因以致死,請歲計死者多少而賞罰之。」府吏額七百人,以罪廢復敘者,皆數外補之,黯請敘者須有闕乃補。然所斷治,或出己見,人不以為允。御史中丞王疇與其屬陳經、呂誨、傅堯俞,諫官司馬光、龔鼎臣、王陶,皆言黯剛愎自任,赦書下府,罪應釋者反重行之。罷為同提舉在京諸司庫務。



 英
 宗即位,遷中書舍人。受詔撰《仁宗實錄》,權知審刑院,為群牧使。時封拜皇子,並除檢校太傅。黯言:「太師、太傅、太保,是為三師,天子之所師法。子為父師,於義不可,蓋前世因循弗思之過。請自今皇子及宗室屬卑者,皆毋兼師傅官,隨其遷序,改授三公。」下兩制議,請如黯奏。而中書亦謂:「自唐以來,親王無兼師傅者。國朝以三師、三公皆虛名,故因而授之,宜正其失。」詔可。



 遷給事中、權御史中丞。未幾,以呂誨知雜事,誨嘗彈治黯,逡巡引避。黯言
 嘗薦誨為御史,知其方正謹厚,一時公言,非有嫌怨,願終與共事,誨乃就職。時帝初即位,王廣淵、周孟陽以藩邸之舊,數召對。黯言:「俊乂滿朝,未有一被召者,獨親近一二舊人,示天下以不廣。請如太宗故事,召侍從館閣之臣,以備顧問。」帝嘗從容謂黯曰:「朕欲用人,少可任者。」黯對:「天下未嘗乏人,顧所用如何爾。」退而上五事:一、知人之明,二、養育以漸,三、材不求備,四、以類薦舉,五、擇取自代。



 後與兩制合議,請以濮王為皇伯,執政弗從,數詣
 中書爭論。會大雨水,時黯已被疾,疏言:「簡宗廟,逆天時,則水不潤下。今二三執政,知陛下為先帝後,乃阿諛容說,違背經義,建兩統貳父之說,故七廟神靈震怒,天降雨水,流殺人民。」既病,求出,以翰林侍讀學士知陳州。未行,卒,年四十四。口占遺奏數百言,猶以濮王議為請。贈尚書禮部侍郎。



 初,黯母陳歸宗,繼母史在堂,後迎陳歸,二母不相善,黯能安以事之。黯修潔自喜,在朝數言事,或從或否,人稱其介直。然卞急,初通判襄州,疑優人戲
 己,以人SX啖之。在開封,為罪人所詈,又啖以人SX,言者亦以是詆之。



 李京,字伯升,趙州人。進士中第,歷平定軍判官、冀州推官,改大理寺丞、知魏縣。奉法嚴正,吏不便,欲以苛中京,遂相率遁去。監司果議以苛刻斥京,知府任布曰:「如此,適墮吏計中。」京賴以免。徙永昌縣,通判趙州。王拱辰薦為監察御史裏行,遷監察御史。



 時太史言日當食不食,群臣皆賀。京上疏曰:「陛下因天之戒,恐懼修省,避正殿,
 減常膳,故精意感格,日當食而陰雲蔽虧。雖宋景公之熒惑退舍,商大戊之桑穀並枯,無以異也。然臣區區竊有所疑者,自寶元初,定襄地震,壞城郭,覆廬舍,壓死者以數萬人。殆今十年,震動不已,豈非西、北二邊,有窺中國之意乎?二月雷發聲,在《易》為《豫》,言萬物出地,皆悅豫也。八月收聲,在《易》為《歸妹》,言雷聲入地,避群陰之害也。今孟夏雷未發聲,豈非號令不信乎?願陛下飭邊臣備夷狄,戒輔臣慎出命,以厭禍於未形。又尚美人棄外館
 多年,比聞復召入,臣慮假媚道以為蠱惑,宜亟絕之。苗繼宗嬪御子弟,乃緣恩私,為府界提點。宜割帷薄之愛,重名器之分,庶幾不累聖政。」仁宗嘉納,授右正言、直集賢院、同管勾國子監,加史館修撰。



 數上書論事,宰相賈昌朝不悅。京嘗屬侍御史吳鼎臣薦推直官李實,鼎臣希昌朝意,以告中丞高若訥。若訥為鼎臣上京簡,謫京太常博士、監鄂州稅。既至,引令狐峘、錢徽事言:「臣為御史諫官,首尾五年,凡六上章、四親對,自陳疾故,懇求外
 補。臣之出處,粗有本末。向者在臺,見《入閣圖》,三院御史立班各異。聞元日將入閣,而御史王贄、何郯皆謁告歸。會推直官李實歲將滿,因簡鼎臣宜留實補御史,鼎臣亦謂議協公望,不意逾兩月,乃誣臣與實為朋黨。臣初被黜,閱諸橐中,鼎臣所遺私書別紙故在,臣令男諶亟悉焚毀。臣與實僚友,鼎臣鄉曲之舊,鼎臣為御史,臣延譽推引,實有力焉。待之不疑,因以誠告,豈謂傾險包藏,甘為鷹犬,惟陛下察之。」未幾,卒官。詔錄諶為郊社齋郎。



 鼎臣,棣州人。既逐京,會昌朝罷,夏竦自北京召為相。鼎臣先論竦在並州杖殺私僕,復與諫官、御史言竦論議與陳執中異,不可共事。竦既罷,遂以刑部員外郎知諫院。上言:「朝廷方與契丹保誓約,而楊懷敏增廣塘水,輒生事,民或怨叛,雖斬懷敏,無及矣。」遂為河北體量安撫,令經度塘水利害,而鼎臣更顧望,依違不能決。昌朝與都轉運使施昌言議河事不合,鼎臣自度支副使拜天章閣待制,代昌言,數月卒。



 呂景初,字沖之,開封酸棗人。以父蔭試秘書省校書郎,舉進士,歷汝州推官,改著作佐郎、知夏陽縣,僉書河南府判官,通判並州。高若訥薦為殿中侍御史。



 張貴妃薨,有司請依荊王故事,輟視朝五日,或欲更增日,聽上裁,乃增至七日。景初言:「妃一品當輟朝三日,禮官希旨,使恩禮過荊王,不可以示天下。」妃既追冊為皇后,又詔立忌,景初力爭,乃罷。



 時兵冗,用度乏,景初奏疏曰:「聖人在上,不能無災,而有救災之術。今百姓困窮,國用虛竭,利
 源已盡,惟有減用度爾。用度之廣,無如養兵。比年招置太多,未加揀汰。若兵皆勇健,能捍寇敵,竭民膏血以啖之,猶為不可,況羸疾老怯者,又常過半,徒費粟帛,戰則先奔,致勇者亦相牽以敗。當祖宗時,四方割據,中國才百餘州,民力未完,耕植未廣,然用度充足者,兵少故也,而所徵皆克。自數十年來,用數倍之兵,所向必敗。以此,知兵在精,不在眾也。議者屢以為言,陛下不即更者,由大臣偷安避怨,論事之臣,又復緘默,則此弊何時而息。
 望詔中書、樞密院,議罷招補,而汰冗濫。」



 又言:「坐而論道者,三公也。今輔臣奏事,非留身求罷免,未嘗從容獨見,以評講治道。雖願治如堯、舜,得賢如稷、契,而未至於治者,抑由此也。願陛下於輔臣、侍從、臺諫之列,擇其忠信通治道者,屢詔而數訪之,幸甚!」又與言事御史馬遵、吳中復奏彈梁適與劉宗孟連姻,而宗孟與冀州富人共商販。下開封府劾治,所言不實,皆坐謫,景初通判江寧府。徙知衡州,復召還臺。



 嘉祐初,大雨水,景初曰:「此陰
 盛陽微之誡也。」乃上疏稱:「商、周之盛,並建同姓;兩漢皇子,多封大國;有唐宗室,出為刺史;國朝二宗,相繼尹京。是欲本支盛強,有盤石之安,則奸雄不敢內窺,而天下有所倚望矣。願擇宗子之賢者,使得問安侍膳於宮中,以消奸萌,或尹京典郡,為夾輔之勢。」時狄青為樞密使,得士卒心,議者憂其為變。景初奏疏曰:「天象謫見,妖人訛言,權臣有虛聲,為兵眾所附,中外為之恟□。此機會之際,間不容發,蓋以未立皇子,社稷有此大憂。惟陛下蚤
 為之計,則人心不搖,國本固矣。」數詣中書白執政,請出青。文彥博以青忠謹有素,外言皆小人為之,不足置意。景初曰:「青雖忠,如眾心何,蓋為小人無識,則或以致變。大臣宜為朝廷慮,毋牽閭裏恩也。」知制誥劉敞亦論之甚力,卒出青知陳州。



 李仲昌以河事敗,內遣中人置獄。景初意賈昌朝為之,即言:「事無根原,不出政府,恐陰邪用此,以中傷善良。」乃更遣御史同訊。遷右司諫,安撫河北。還,奏比部員外郎鄭平占籍真定,有由七百餘頃,因
 請均其徭役,著限田令。以戶部員外郎兼侍御史知雜事,判都水監,改度支副使,遷吏部員外郎,擢天章閣待制、知諫院,以病,未入謝而卒。



 馬遵者字仲塗,饒州樂平人。嘗以監察御史為江、淮發運判官,就遷殿中侍御史為副使。入為言事御史,謫知宣州,後復為右司諫,以禮部員外郎兼侍御史知雜事,改吏部,直龍圖閣,卒。性樂易,善議論,其言事不為激訐,故多見推行,杜衍、範仲淹皆稱道之。



 吳及,字幾道,通州靜海人。年十七,以進士起家,為侯官尉。閩俗多自毒死以誣仇家,官司莫能辨,及悉為讞正,前後活五十三人,提點刑獄移其法於一路。闢大理寺檢法官,徙審刑院詳議,累遷太常博士。



 是時,仁宗春秋既高,無子,及因推言閹寺,以及繼嗣事。至和元年,上疏曰:



 臣聞「官師相規,工執藝事以諫。」臣幸得待罪法吏,輒原刑法之本,以效愚忠。切惟前世肉刑之設,斷支體,刻肌膚,使終身不息。漢文感緹縈之言,易之鞭棰,然已死
 而笞未止,外有輕刑之意,其實殺人。祖宗鑒既往之弊,蠲除煩苛,始用折杖之法,新天下耳目,茲蓋曠古聖賢,思所未至,陛下深惻民隱,親覽庶獄。歷世用刑,無如本朝之平恕,宜乎天降之祥。而方當隆盛之時,未享繼嗣之慶,臣竊惑焉。



 或者宦官太多,而陛下未悟也。何則?肉刑之五,一曰宮,古人除之,重絕人之世。今則宦官之家,競求他子,剿絕人理,希求爵命。童幼何罪,隱於刀鋸,因而夭死者,未易悉數。夫有疾而夭,治世所羞,況無疾乎?
 有罪而宮,前王不忍,況無罪乎?臣聞漢永平之際,中常侍四員,小黃門十人爾。唐太宗定制,無得逾百員。且以祖宗近事較之,祖宗時宦官凡幾何人,今凡幾何人?臣愚以謂胎卵傷而鳳凰不至,宦官多而繼嗣未育也。伏望順陽春生育之令,浚發德音,詳為條禁。進獻宦官,一切權罷,擅宮童幼,置以重法。若然,則天心必應,聖嗣必廣,召福祥、安宗廟之策,無先於此。



 書奏,帝異其言,欲用為諫官,而及以父憂去。



 嘉祐三年,始擢秘閣校理,逾月,
 改右正言。復上疏曰:「帝王之治,必敦骨肉之愛,而以至親夾輔王室。《詩》曰:『懷德惟寧,宗子惟城。』故同姓者,國家之屏翰;儲副者,天下之根本。陛下以海宇之廣,宗廟之重,而根本未立,四方無所系心,上下之憂,無大於此。謂宜發自聖斷,擇宗室子以備儲副。以服屬議之,則莫如親;以人望言之,則莫如賢。既兼親賢,然後優封爵以寵異之,選重厚樸茂之臣以教導之,聽入侍禁中,示欲為後,使中外之人悚然瞻望,曰:『宮中有子矣。』陛下他日有
 嫡嗣,則異其恩禮,復令歸邸,於理無嫌,於義為順,弭覬覦之心,屬天下之望,宗廟長久之策也。」既而又言:「開寶詔書:『內侍臣年三十無養父者,聽養一子為嗣,並以名上宣徽院,違者抵死。』比年此禁益弛,夭絕人理,陰累聖嗣。願詔大臣明示舊制,上順天意,以綏福祐。」明年,遂權罷內臣進養子。



 管勾登聞檢院。又上書論政事,謂:「倉廩空虛,內外匱乏,其弊在於官多兵冗。請汰冗兵,省冗官,然後除民之疾苦。」因條上十餘事,多施用之。建請擇館
 職,分校館閣書,並求遺書於天下,語在《藝文志》。



 明年,日食三朝,及言:「日食者,陰侵陽之戒。在人事,則臣陵君,妻乘夫,四夷侵中國。今大臣無姑息之政,非所謂臣陵君,失在陛下淵默臨朝,使陰邪未盡屏也。後妃無權橫之家,非所謂妻乘夫,失在左右親幸,驕縱亡節也。疆埸無虞,非所謂四夷侵中國,失在將帥非其人,為敵所輕也。」因言孫沔在並州,苛暴不法,燕飲無度;龐藉前在並州,輕動寡謀,輒興堡砦,屈野之衄,為國深恥。沔繇此坐廢。



 又言:「春秋有告糴,陛下恩施動植,視人如傷。然州郡官司各專其民,擅造閉糴之令,一路饑,則鄰路為之閉糴;一郡饑,則鄰郡為之閉糴。夫二千石以上,所宜同國休戚,而坐視流離,豈聖朝子育兆民之意哉!」遂詔:「鄰州、鄰路災傷而輒閉糴,論如違制律。」



 久之,遷右司諫、管勾國子監。在職數年,以勁正稱,遇事無小大輒言。嘗請毋納群臣上尊號,出後宮私身及非執事人,毋以御寶白札子賜近幸家人冠帔及比丘尼紫衣;並責執政大臣因
 循茍簡,畏避怨謗,宜用唐李吉甫故事,選拔賢俊,約杜預遺法,旌擢守令;復置將作監官屬,專領營造;論入內都知任守忠陵轢駙馬都尉李瑋及干求內降。



 會諫官陳升之建請裁節班行補授,下兩制、臺諫官集議。主鐵冶者,舊得補班行。至是,議罷之。既定稿,及與御史沈起輒增注興國軍磁湖鐵冶如舊制。主磁湖冶者,大姓程叔良也。翰林學士胡宿等即劾及與起職在臺諫,而為程氏經營占錮恩例,請詔問狀,皆引伏。及出為工部員
 外郎、知廬州,進戶部、直昭文館、知桂州。卒,錄其弟齊為太廟齋郎。



 及當官有守,初為檢法官,三司請重鑄鐵錢法至死。下有司議,及爭不可,主者恚曰:「立天下法,當由一檢法邪?」及曰:「義理為先,安有高下?」卒不為詘。



 範師道,字貫之,蘇州長洲人。進士及第,為撫州判官,後知廣德縣。縣有張王廟,民歲祠神,殺牛數千,師道禁絕之。通判許州,累遷都官員外郎,吳育舉為御史。奏請罷內降推恩,擇宰相久其任,選宗室賢者養宮中備儲貳。



 初,皇祐中,賈昌朝上議置五輔郡,設京畿轉運使、提點刑獄,號為「拱輔京師」,而論者謂宦官謀廣親事親從兵,欲取京畿財賦贍之,因以收事柄。師道力奏非便,遂復舊制。又以四年貢舉,士苦淹久,請易為三年。宰相劉沆護葬溫成皇后,禮官議稱「陵」,師道以為非典制,數以爭,沆惡之,引著令「臺官滿二年當補外」,出知常州。臺諫官共言師道不當去,不報。徙廣南東路轉運使。舊補攝官皆委吏胥,無先後遠近之差,師道為置籍次第之。召為
 鹽鐵判官,道改兩浙轉運使,遷起居舍人、同知諫院,管勾國子監。



 後宮周氏、董氏生公主,諸閣女御多遷擢。師道上疏曰:「禮以制情,義以奪愛,常人之所難,惟聰明睿哲之主然後能之。近以宮人數多而出之,此盛德事也。然而事有系風化治亂之大,而未以留意,臣敢為陛下言之。竊聞諸閣女御,以周、董育公主,御寶白札並為才人,不自中書出誥。而掖庭覬覦遷拜者甚多,周、董之遷可矣,女御何名而遷乎?才人品秩既高,古有定員,唐制
 止七人而已。祖宗朝宮闈給侍不過二三百,居五品之列者無幾,若使諸閣皆遷,則不復更有員數矣。外人不能詳知,止謂陛下於寵幸太過,恩澤不節耳。夫婦人女子,與小人之性同,寵幸太過,則瀆慢之心生,恩澤不節,則無厭之怨起,御之不可不以其道也。且用度太煩,須索太廣,一才人之奉,月直中戶百家之賦,歲時賜予不在焉。況誥命之出,不自有司,豈盛時之事耶?恐斜封、墨敕,復見於今日矣。」



 時大星隕東南,有聲如雷。又上疏曰:「《
 漢》、《晉天文志》:『天狗所下,為破軍殺將,伏尸流血。』《甘氏圖》:『天狗移,大賊起。』今朝廷非無為之時也,而備邊防盜,未見其至。雖有將帥,不老則愚,士卒雖多,勁勇者少。小人思亂,伺隙乃作,必有包藏險心,投隙而動者。宜揀拔將帥,訓練卒伍,詔天下預為備御。」仁宗晚年尤恭儉,而四方無事,師道言雖過,每優容之。遷兵部員外郎,兼侍御史知雜事、判都水監。與諫官、御史數奏樞密副使陳升之不當用,升之罷,師道亦出知福州。頃之,以工部郎中
 入為三司鹽鐵副使。感風眩,遷戶部,直龍圖閣、知明州,卒。



 師道厲風操,前後在言責,有聞即言,或獨爭,或列奏。如陳執中家人殺婢,卒坐免;奪王拱辰宣徽使、李淑翰林學士;及王德用、程戡領樞密,宦官石全彬、閻士良升進,皆嘗奏數其罪焉。



 李絢,字公素,邛州依政人。少放蕩亡檢,兄綯教之書,嚴其課業而出,絢遨自若,比暮綯歸,絢徐取書視之,一過輒誦數千言,綯奇之。稍長,能屬文,尤工歌詩。嘗以事被
 系,既而逸去。



 擢進士第,再授大理評事、通判邠州。元昊犯延州,並邊皆恐。邠城陴不完,絢方攝守,即發民治城,僚吏皆謂當言上逮報,絢不聽。帝聞之喜,因詔他州悉治守備。還為太子中允、直集賢院,歷開封府推官、三司度支判官,為京西轉運使。是時,範雍知河南,王舉正知許州,任中師知陳州,任布知河陽,並二府舊臣,絢皆以不才奏之。



 未幾,召修起居注,糾察在京刑獄。時宰相杜衍各拔知名士置臺省,惡衍者指絢為其黨。絢嘗舉陸
 經,經坐贓貶;而任布又言絢在京西苛察,出知潤州。改太常丞,徙洪州。時五溪蠻寇湖南,擇轉運使,帝曰:「有館職善飲酒者為誰,今安在?」輔臣未諭,帝曰:「是往歲城邠州者,其人才可用。」輔臣以絢對,遂除湖南轉運使。絢乘驛至邵州,戒諸部按兵毋得動,使人諭蠻以禍福,蠻罷兵受約束。



 復修起居注,權判三司鹽鐵勾院,復糾察在京刑獄。以右正言、知制誥奉使契丹,知審官院,遷龍圖閣直學士、起居舍人、權知開封府,治有能名。絢夜醉,晨
 奏事酒未解,帝曰:「開封府事劇,豈可沉湎於酒邪?」改提舉在京諸司庫務,權判吏部流內銓。初,慈孝寺亡章獻太后神御物,盜得,而絢誤釋之,詘知蘇州,未行,卒。



 絢疏明樂易,少周游四方,頗練世務。數上書言便宜。仁宗春秋高,未有繼嗣,絢因祀高禖還獻賦,大指言宜遠嬖寵,近賢良,則神降之福,子孫繁衍,帝嘉納之。性嗜酒,終以疾死。



 何中立,字公南,許州長社人。幼警邁,與狄遵度游,遵度
 曰:「美才也!」其父棐遂以女妻之。進士及第,授大理評事,歷僉書鎮安、武勝二鎮節度判官,遷殿中丞,召試學士院,為集賢校理。改太常博士、修起居注,遷祠部員外郎、知制誥,權發遣開封府事。



 初,有盜慈孝寺章獻皇太后神御服器者,既就縶,李絢以屬吏,考掠不得其情,輒釋去。中立至,人復執以來,中立曰:「此真盜也。」窮治之,卒伏罪。遷兵部員外郎,糾察在京刑獄。除龍圖閣直學士、知秦州。言者以為非治邊才,改慶州。奏曰:「臣不堪於秦,則
 不堪於慶矣,願守汝。」不報。戍卒有告大校受贓者,中立曰:「是必挾他怨也。」鞭卒竄之。或曰:「貸奸可乎?」中立曰:「部曲得持短長以制其上,則人不自安矣。」還判太常寺,遷刑部郎中,進樞密直學士、知許州,改陳州。訛言大水至,居人皆恐,中立捕誅之。又徙杭州,暴中風卒。



 中立頗以文詞自喜,然嗜酒無行。慶歷中,集賢校理蘇舜欽監進奏院,為賽神會,預者皆一時知名士,中立亦在召中。已而辭不往,後舜欽等得罪,中立有力焉。



 沉邈,字子山,信州弋陽人。進士及第,起家補大理評事、知侯官縣,通判廣州,累遷都官員外郎,歷知真州、福州。慶歷初,為侍御史。時呂夷簡罷相,輔臣皆進官,邈言:「爵祿所以勸臣下,非功而授則為濫。今邊鄙屢警,未聞廟堂之謀有以折外侮,無名進秩,臣下何勸焉。」又論:「夏竦除樞密使,而竦陰交內侍劉從願。使從願內濟狡譎,竦外專機務,奸黨得計,人主之權去矣。」其言甚切。權鹽鐵判官,轉兵部員外郎。時選諸路轉運加按察使,邈與張
 溫之、王素首被選。邈加直史館,使京東。歲餘,入為侍御史知雜事。未幾,擢天章閣待制、知澶州,徙河北都轉運使,又徙陜西,歲中,加刑部郎中、知延州,卒。



 邈疏爽有治才,然性少檢。在廣州時,歲游劉王山,會賓友縱酒,而與閭里婦女,笑言無間。



 論曰:慶歷以來,任諫官、御史,名有風採,見推於時者,繇臻、京之輩,凡數十人,觀其所陳,蓋不虛得。及之論閹宦,真仁人之言,其最優乎!絢、中立、邈亦有美才,致位通顯,
 然皆以酒失自累,故不能無貶焉。



\end{pinyinscope}