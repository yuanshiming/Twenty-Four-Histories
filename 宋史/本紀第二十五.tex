\article{本紀第二十五}

\begin{pinyinscope}

 高宗二


二年
 春正月丙戌朔,帝在揚州。丁亥,錄兩河流亡吏士。沿河給流民官田、牛、種。戊子,金人陷鄧州,安撫劉汲死之。辛卯,置行在榷貨務。壬辰,金人犯東京,宗澤遣將擊
 卻之。癸巳,復明法新科。甲午,詣壽寧寺,謁祖宗神主。乙未,金人破永興軍,前河東經制副使傅亮以兵降,經略使唐重、副總管楊宗閔、提舉軍馬陳迪、轉運副使桑景詢、判官曾謂、提點刑獄郭忠孝、經略司主管機宜文字王尚及其子建中俱死之。東平府兵馬鈐轄孔彥舟叛,渡淮犯黃州,守臣趙令
 \gezhu{
  山成}
 拒之。丙申,詔:「自今犯枉法自盜贓者,中書籍其姓名,罪至徒者,永不錄用。金人陷均州,守臣楊彥明遁去。丁酉,金人陷房州。己亥,張遇焚真
 州。秘閣修撰孫昭遠為亂兵所害。庚子,遣主客員外郎謝亮為陜西撫諭使兼宣諭使,持詔賜夏國。張遇陷鎮江府,守臣錢伯言棄城走。辛丑,內侍邵成章坐輒言人臣除名、南雄州編管。金人陷鄭州,通判趙伯振死之。癸卯,金帥窩裡嗢陷濰州,又陷青州,尋棄去。丁未,詔諭流民、潰兵之為盜賊者,釋其罪。己酉,禁諸將引潰兵入蜀,置大散關使以審驗之。庚戌,遣考功員外郎傅雱為淮東京東西撫諭使。辛亥,王淵招降張遇,以所部萬人隸
 韓世忠。改授顯謨閣直學士孟忠厚為常德軍承宣使。詔凡後族毋任侍從官,著為令。金人焚鄧州。是月,以中奉大夫劉豫知濟南府。金人陷穎昌府,守臣孫默為所殺。經制司僚屬王擇仁復永興軍。金人陷秦州,經略使李復降;又犯熙河,經略使張深遣兵馬都監劉惟輔與戰於新店,敗之,斬其帥黑鋒。



 二月丙辰,金人再犯東京,宗澤遣統制閻中立等拒之,中立戰死。戊午,移耿南仲於臨江軍。金人陷唐州。壬戌,安化軍節度副使宇文虛
 中應詔使絕域,復中大夫,召赴行在。癸亥,罷市易務。甲子,金人犯滑州,宗澤遣張摠救之,戰死。乙丑,澤遣判官範世延等表請帝還闕。河北賊楊進等詣澤降。丁卯,復延康、述古殿直學士為端明、樞密直學士。辛未,詔自今犯枉法自盜贓罪至死者,籍其貲。壬申,赦福州叛卒張員等。癸酉,金人陷蔡州,執守臣閻孝忠。丙子,金人陷淮寧府,守臣向子韶死之。丁丑,遣王貺等充金國軍前通問使。戊寅,責降知鎮江府趙子崧為單州團練副使、南
 雄州安置。己卯,奪秘書正字胡珵官,送梧州編管。朝奉大夫劉正彥應詔使絕域,授武德大夫、威州刺史,尋為御營右軍副統制。庚申,以王淵為向德軍節度使。辛巳,武功大夫、和州防禦使馬擴奔真定五馬山砦聚兵,得皇弟信王榛於民間,奉之總制諸砦。壬午,詔京畿、京東西、河北、淮南路置振華軍八萬人。是月,成都守臣盧法原修城成。



 三月辛卯,金人陷中山府。壬辰,詔諸路安撫使許便宜節制官吏。丁酉,初立《大小使臣呈試弓馬出
 官格》,先閱試,然後奏補。粘罕焚西京去。庚子,河南統制官翟進復西京,宗澤奏進為京西北路安撫制置使。丙午,遙授尚書右僕射何□為觀文殿大學士,中書侍郎陳過庭為資政殿大學士,同知樞密院事聶昌為資政殿大學士,並主管宮觀。時□已卒於金,昌為人所殺,朝遷未之知。過庭亦在金軍中。丁未,罷內外權局官之不應法者。遣楊應誠為大金、高麗國信使。己酉,張員等復作亂,擁眾突城出,命本路提點刑獄李芘討捕之。辛亥,
 以範瓊權同主管侍衛步軍司公事,屯真州。是月,金人陷鳳翔府,守臣劉清臣棄城去;又犯涇原,經略使統制官曲端遣將拒戰,敗之,金兵走同、華。石壕尉李彥仙舉兵復陜州。



 夏四月丙辰,詔文臣從官至牧守、武臣管軍至遙郡,各舉所知二人。戊午,……



 五月乙酉,許景衡罷。孫琦犯德安府。丙戌,命參酌元祐科舉條制,立詩賦、經義分試法。戊子,以翰林學士朱勝非為尚書右丞。辛卯,以金兵渡河,遣韓世忠、宗澤等逆戰。甲午,曲赦河北、陜西、京東路。福建轉運判官謝如意執張員
 等六人,誅之。丙申,復命宇文虛中為資政殿大學士,充金國祈請使。賊靳賽寇光山縣。戊戌,河北制置使王彥部兵渡河,屯滑州之沙店。癸卯,張愨薨。甲辰,金帥婁宿陷絳州。丁未,復置兩浙、福建提舉市舶司。己酉,秀州卒徐明等作亂,執守臣朱芾,迎前守趙叔近復領州事。命御營中軍統制張俊討之。癸丑,罷借諸路職田。



 六月乙卯,權罷邛州鑄錢,增印錢引。癸亥,建州卒葉濃等作亂,寇福州。甲子,親慮囚。乙丑,張俊至秀州,殺趙叔近,執徐明斬
 之。甲戌,葉濃陷福州。丁丑,詔江、浙沿流州軍練水軍,造戰艦。京畿、淮甸蝗。是月,以知延安府王庶節制陜西六路軍馬,涇原經略使統制官曲端為節制司都統制。永興軍經略使郭琰逐王擇仁,擇仁奔興元。



 秋七月甲申,葉濃入寧德縣,復還建州,命張俊同兩浙提點刑獄趙哲率兵討之。丙戌,詔吏部審量京官,非政和以後進書頌及直赴殿試人,乃聽參選。宗澤薨。丁亥,詔百官坐祭京、王黼擬授而廢者,許自新復用。戊子,禁軍中抉目刳心之刑。
 壬辰,選江、浙州軍正兵、土兵六之一赴行在。乙未,以郭仲荀為京城副留守。戊戌,錄內外諸軍將士功。辛丑,以春霖、夏旱蝗,詔監司、郡守條上闕政,州郡災甚者蠲田賦。甲辰,以降授北京留守杜充復樞密直學士,為開封尹、東京留守。



 八月甲寅,初鑄御寶三。甲戌,御集英殿策試禮部進士。罷殿中侍御史馬伸,尋責濮州。河北、京東捉殺使李成叛。辛巳,犯宿州。是月,二帝徙居韓州。



 九月甲申,丁進叛,復寇淮西。庚寅,賜禮部進士李易以下四
 百五十一人及第、出身,特奏名進士皆許調官。壬辰,召侍從所舉褚宗諤等二十一人驛赴行在。癸巳,金人陷冀州,將官李政死之。甲午,金人再犯永興軍,經略使郭琰棄城,退保義穀。辛丑,陜西節制司兵官賀師範及金人戰於八公原,敗績,死之。丙午,復所減京官奉。丁未,東京留守統制官薛廣及金人戰於相州,敗死。己酉,郭三益薨。是秋,窩裡嗢、撻懶破五馬山砦,信王榛不知所終。馬擴軍敗於北京之清平。



 冬十月甲寅,命揚州浚隍修
 城。閱江、淮州郡水軍。楊應誠還自高麗。戊午,遣劉光世討李成。壬戌,禁江、浙閉糴。癸亥,粘罕圍濮州,遣韓世忠、範瓊領兵至東平、開德府,分道拒戰,又命馬擴援之。甲子,命孟忠厚奉隆祐太后幸杭州。楊進復叛,攻汝、洛,命翟進擊於鳴皋山,翟進戰死。丙子,罷吏部審量崇寧、大觀以來濫賞,止令自陳。是月,劉正彥擊丁進,降之。



 十一月辛巳朔,提舉嵩山崇福宮李綱責授單州團練副使、萬安軍安置。劉光世及李成戰於新息縣,成敗走。高麗
 國王王楷遣其臣尹彥頤入見。金人圍陜州,守臣李彥仙拒戰,卻之。壬辰,金人陷延安府,權知府劉選、總管馬忠皆遁,通判府事魏彥明死之。癸巳,趙哲大破葉濃於建州城下,濃遁而降,復謀為變,張俊禽斬之。乙未,金人陷濮州,執守臣楊粹中,又陷開德府,守臣王棣死之。以魏行可充金國軍前通問使。庚子,詣壽寧寺朝饗祖宗神主。壬寅,冬至,祀昊天上帝於圜丘,以太祖配,大赦。金人陷相州,守臣趙不試死之。甲辰,陷德州,兵馬都監趙
 叔皎死之。庚戌,立士庶子弟習射補官法。是月,節制陜西軍馬王庶為都統制曲端所拘,奪其印。四川茶馬趙開罷官買賣茶,給引通商如政和法。金人犯晉寧軍,守臣徐徽言拒卻之,知府州折可求以城降。金人陷淄州。涇原兵馬都監吳玠襲斬史斌。濱州賊蓋進陷棣州,守臣姜剛之死之。京東賊李民詣行在請降,王淵殲其眾,留民為將。十二月乙卯,太后至杭州,扈從統制苗傅以其軍八千人駐奉國寺。庚申,金人犯東平府,京東西路制
 置使權邦彥棄城去,又犯濟南府,守臣劉豫以城降。甲子,金人陷大名府,提點刑獄郭永罵敵不屈,死之,轉運判官裴億降。又陷襲慶府。乙丑,陷虢州。丙寅,初命修國史。己巳,以黃潛善為尚書左僕射兼門下侍郎,汪伯彥右僕射兼中書侍郎,顏岐門下侍郎,朱勝非中書侍郎,兵部尚書盧益同知樞密院事。辛未,金人犯青州。丁丑,特進致仕餘深、金紫光祿大夫致仕薛昂並分司,進昌軍、徽州居住。耿南仲再責單州別駕,唐恪追落觀文殿
 大學士。戊寅,以禮部侍郎張浚兼御營參贊軍事,教習長兵。是冬,杜充決黃河,自泗入淮,以阻金兵。



 三年春正月庚辰朔,帝在揚州。以京西北路兵馬鈐轄翟興為河南尹、京西北路安撫制置兼招討使。京西賊貴仲正陷岳州。甲申,以資政殿學士路允迪簽書樞密院事。丁亥,金人再陷青州,又陷濰州,焚城而去。京東安撫劉洪道入青州守之。己丑,奉安西京會聖宮累朝御容於壽寧寺。占城國入貢。趣大金通問使李鄴、周望、宋
 彥通、吳德休等往軍前。辛卯,陜州都統邵興及金人戰於潼關,敗之。復虢州。乙未,杜充遣岳飛、桑仲討其叛將張用於城南,其徒王善救之,官軍敗績。庚子,張用、王善寇淮寧府,守臣馮長寧卻之。詔:「百官聞警遣家屬避兵,致物情動搖者,流。」丙午,粘罕陷徐州,守臣王復及子倚死之,軍校趙立結鄉兵為興復計。御營平寇左將軍韓世忠軍潰于沭陽,其將張遇死,世忠奔鹽城。金兵執淮陽守臣李寬,殺轉運副使李跋,以騎兵三千取彭城,間
 道趣淮甸。戊申,至泗州。



 二月庚戌朔,始聽士民從便避兵。命劉正彥部兵衛皇子、六宮如杭州。江、淮制置使劉光世阻淮拒金人,敵未至,自潰。金人犯楚州,守臣朱琳降。辛亥,金人陷天長軍。壬子,內侍鄺詢報金兵至,帝被甲馳幸鎮江府。是日,金兵過楊子橋。癸丑,游騎至瓜洲,太常少卿季陵奉太廟神主行,金兵追之,失太祖神主。王淵請幸杭州。命留朱勝非守鎮江,以吏部尚書呂頤浩為資政殿大學士、江淮制置使,都巡檢使劉光世為
 殿前都指揮使,充行在五軍制置使,駐鎮江府,控扼江口。主管馬軍司楊惟忠節制江東軍馬,駐江寧府。是夕,發鎮江,次呂城鎮。金人入真州。甲寅,次常州。御營統制王亦謀據江寧,不克而遁。御營平寇前將軍範瓊自東平引兵至壽春,其部兵殺守臣鄧紹密。丙辰,次平江府。丁巳,金人犯泰州,守臣曾班以城降。丁進縱兵剽掠,王淵誘誅之。戊午,次吳江縣,命朱勝非節制平江府、秀州控扼軍馬,禮部侍郎張浚副之。又命勝非兼御營副使。
 留王淵守平江。以忠訓郎劉俊民為閣門祗候,繼書使金軍。詔錄用張邦昌親屬,仍命俊民持邦昌貽金人約和書稿以行。金人陷滄州,守臣劉錫棄城走。己未,次秀州。命呂頤浩往來經制長江,以龍圖閣待制、知江州陳彥文為沿江措置使。庚申,次崇德縣。呂頤浩從行,即拜同簽書樞密院事、江淮兩浙制置使,以兵二千還屯京口。又命御營中軍統制張俊以兵八千守吳江,吏部員外郎鄭資之為沿江防托,監察御史林之平為沿海防
 托,募海舟守隘。壬戌,駐蹕杭州。金人陷晉寧軍,守臣徐徽言死之。癸亥,下詔罪己,求直言。令有司具舟常、潤,迎濟衣冠、軍民家屬。省儀物、膳羞,出宮人之無職掌者。乙丑,降德音;赦雜犯死罪以下囚,放還士大夫被竄斥者,惟李綱罪在不赦,更不放還。蓋用黃潛善計,罪綱以謝金人。置江寧府榷貨務都茶場。丁卯,百官入見,應迪功郎以上並赴朝參。戊辰,出米十萬斛,即杭、秀、常、湖州、平江府損直以糶,濟東北流寓之人。金人焚揚州。己巳,用
 御史中丞張澄言,罷黃潛善、汪伯彥,以戶部尚書葉夢得為尚書左丞,澄為右丞。庚午,詔平江、鎮江府、常、湖、杭、越州,具寓居京朝官已上姓名以備簡拔。分命浙西監司等官,募土豪守千秋、垂腳、襄陽諸嶺,以扼宣、常諸州險要。金人去揚州。辛未,詔御營使司唯掌行在五軍,凡邊防經制並歸三省、樞密。金人過高郵軍,守臣趙士瑗棄城走。潰兵宋進犯泰州,守臣曾班遁。壬申,罷軍期司掊斂民財者。呂頤浩遣將陳彥渡江襲金餘兵,復揚州。
 癸酉,靳賽犯通州。韓世忠小校李在叛據高郵。甲戌,黃潛善、汪伯彥並落職。乙亥,召朱勝非赴行在,留張浚駐平江。贈陳東、歐陽澈承事郎,官有服親一人,恤其家。召馬伸赴行在,卒,贈直龍圖閣。丙子,詔士民直言時政得失。是月,以王庶為陜西節制使、知京兆府,節制司都統制曲端為鄜延經略使、知延安府。張用據確山,號「張莽蕩」。



 三月己卯朔,日中有黑子。庚辰,以朱勝非為尚書右僕射兼中書侍郎。辛巳,葉夢得罷,以盧益為尚書左丞,
 未拜,復罷為資政殿學士。御營都統制王淵同簽書樞密院事,呂頤浩為江南東路安撫制置使、知江寧府。壬午,詔王淵免進呈書押本院文字。扈從統制苗傅忿王淵驟得君,劉正彥怨招降劇盜而賞薄。帝在揚州,閹宦用事恣橫,諸將多疾之。癸未,傅、正彥等叛,勒兵向闕,殺王淵及內侍康履以下百餘人。帝登樓,以傅為慶遠軍承宣使、御營使司增統制,正彥渭州觀察使、副都統制。傅等迫帝遜位於皇子魏國公,請隆祐太后垂簾同聽
 政。是夕,帝移御顯寧寺。甲申,尊帝為睿聖仁孝皇帝,以顯寧寺為睿聖宮,大赦。以張澄兼中書侍郎,韓世忠為御營使司提舉一行事務,前軍統制張俊為秦鳳副總管,分其眾隸諸軍。丁亥,以東京留守杜充為資政殿大學士、節制京東西路。殿前副都指揮使、東京副留守郭仲荀進昭化軍節度使。分竄內侍藍珪、高邈、張去為、張旦、曾擇、陳永錫於嶺南諸州。擇已行,傅追還,殺之。呂頤浩至江寧。戊子,以端明殿學士王孝迪為中書侍郎、盧
 益為尚書左丞。張俊部眾八千至平江,張浚諭以決策起兵問罪,約呂頤浩、劉光世招韓世忠來會。己丑,改元明受。張浚奏乞睿聖皇帝親總要務。庚寅,百官始朝睿聖宮,以苗傅為武當軍節度使,劉正彥為武成軍節度使,劉光世為太尉、淮南制置使,範瓊為慶遠軍節度、湖北制置使,楊惟忠加少保,張浚為禮部尚書,及呂頤浩並赴行在。傅等以御營中軍統制吳湛主管步軍司;黃潛善、汪伯彥並分司,衡、永州居住;王孝迪、盧益為大金
 國信使;進士黃大本、吳時敏為先期告請使。置行在都茶場。呂頤浩奏請睿聖皇帝復大位。金人陷鄜州。癸巳,張浚命節制司參議官辛道宗措置海舶,遣布衣馮□番持書說傅、正彥。甲午,有司請尊太后為太皇太后,不許。呂頤浩率勤王兵萬人發江寧。乙未,再貶黃潛善鎮東軍節度副使、英州安置。劉光世部兵會呂頤浩於丹陽。丙申,韓世忠自鹽城收散卒至平江,張俊假兵二千。戊戌,赴行在。辛丑,傅等以世忠為定國軍節度使張俊為
 武寧軍節度使、知鳳翔府,張浚責黃州團練副使、郴州安置。俊等皆不受。傅等遣軍駐臨平,拒勤王兵。壬寅,日中黑子沒。盧益罷。呂頤浩至平江。水賊邵青入泗州。癸卯,太后詔:睿聖皇帝宜稱皇太弟、天下兵馬大元帥、康王,皇帝稱皇太侄、監國。賜傅、正彥鐵券。呂頤浩、張浚傳檄中外討傅、正彥,執黃大本下獄。乙巳,太后降旨睿聖皇帝處分兵馬重事。張俊率兵發平江,劉光世繼之。丙午,張浚同知樞密院事,翰林學士李邴、御史中丞鄭□
 並同簽書樞密院事。呂頤浩、張浚發平江。丁未,次吳江,奏乞建炎皇帝還即尊位。朱勝非召傅、正彥至都堂議復闢,傅等遂朝睿聖宮。金人陷京東諸郡,劉洪道棄青州去。撻懶以劉豫知東平府、節制河南州郡。趙立復徐州。



 夏四月戊申朔,太后下詔還政,皇帝復大位。帝還宮,與太后御前殿垂簾,詔尊太后為隆祐皇太后。己酉,詔訪求太祖神主。以苗傅為淮西制置使,劉正彥副之。庚戌,復紀年建炎。命張浚知樞密院事,苗傅、劉正彥並檢
 校少保。呂頤浩、張浚軍次臨平,苗翊、馬柔吉拒戰不勝,傅、正彥引兵二千夜遁。辛亥,皇太后撤簾。呂頤浩等入見。傅犯富陽、新城二縣,遣統制王德、喬仲福追擊之。癸未,朱勝非、顏岐、王孝迪、張澄、路允迪俱罷。以呂頤浩為尚書右僕射兼中書侍郎,李邴尚書右丞,鄭□簽書樞密院事。甲寅,以劉光世為太尉、御營副使,韓世忠為武勝軍節度使、御前左軍都統制,張俊為鎮西軍節度使、御前右軍都統制,勤王所僚屬將佐進官有差。主管殿
 前司王元、左言並責官,英、賀州安置。樞密都承旨馬瑗停官、永州居住。吏部員外郎範仲熊、浙西安撫司主管機宜文字時希孟並除名,柳州、吉陽軍編管。斬中軍統制吳湛、工部侍郎王世修於市。贈王淵開府儀同三司。乙卯,大赦。舉行仁宗法度,應嘉祐條制與今不同者,自官制役法外,賞格從重,條約從寬。罷上供不急之物。元祐石刻黨人官職、恩數追復未盡者,令其家自陳。許中外直言。丁巳,禁內侍交通主兵官及饋遺假貸、借役禁
 兵、乾預朝政。庚申,詔尚書左右僕射並帶同中書門下平章事,改門下、中書侍郎為參知政事,省尚書左、右丞。以李邴參知政事。詔行在職事官各舉所知,並省館學、寺監等官。苗傅犯衢州。癸亥,以給事中周望為江、浙制置使。丁卯,帝發杭州,留鄭□衛皇太后,以韓世忠為江、浙制置使,及劉光世追討傅、正彥。己巳,詔:傅、正彥、苗瑀、苗翊、張逵不赦,餘黨並原。壬申,立子魏國公敷為皇太子。赦傅黨王鈞甫、馬柔吉罪,許其自歸。丙子,範瓊自光、
 蘄引兵屯洪州。是月,劉文舜寇濠州。西北賊薛慶襲據高郵軍。



 五月戊寅朔,帝次常州,以張浚為宣撫處置使,以川、陜、京西、湖南北路隸之,聽便宜黜陟。庚辰,苗傅統領官張翼斬王鈞甫、馬柔吉降。辛巳,次鎮江府,遣祭張愨、陳東墓,詔恤其家。癸未,以翰林學士滕康同簽書樞密院事。乙酉,至江寧府,駐蹕神霄宮,改府名建康。起復朝散郎洪皓為大金通問使。丁亥,以徽猷閣直學士陳彥文提領水軍,措置江、浙防托。召藍珪等速還朝。己丑,
 韓世忠追討傅、正彥於浦城縣,獲正彥,傅遁走。張浚撫諭薛慶於高郵,為慶所留。乙未,浚罷。以御營前軍統制王□燮為淮南招撫使。己亥,復置中書門下省檢正官,省左、右司郎中二員。苗傅裨將江池殺苗翊,降於周望。傅走建陽縣,土豪詹標執之以獻。辛丑,張浚還自高郵。復命知樞密院事。是月,翟興擊殺楊進餘黨,復推其徒劉可拒官軍。



 六月戊申朔,以東京留守杜充引兵赴行在,命兼宣撫處置副使,節制淮南、京東西路。己酉,以久雨,
 召郎官已上言闕政,呂頤浩請令實封以聞。遂用司勛員外郎趙鼎言,罷王安石配享神宗廟庭,以司馬光配。王善攻淮寧府不克,轉寇宿州,統領王冠戰敗之。甲寅,罷賞功司。乙卯,命恤死事者家,且錄其後。升浙西安撫使康允之為制置使。丙辰,劉光世招安苗傅將韓雋。戊午,命江、浙、淮南引塘濼、開畎澮,以阻金兵。庚申,皇太后至建康府。辛酉,以久陰,下詔以四失罪己:一曰昧經邦之大略,二曰昧戡難之遠圖,三曰無綏人之德,四曰失
 馭臣之柄。仍榜朝堂,遍諭天下,使知朕悔過之意。以帶御器械李質權同主管殿前司。乙丑,以建康府路安撫使連南夫兼建康府、宣、徽、太平等州制置使。丁卯,右司諫袁植請誅黃潛善及失守者權邦彥等九人。詔:「朕方念咎責己,豈可盡以過失歸於臣下?」遂罷植知池州,以趙鼎為右司諫。癸酉,置樞密院檢詳官。以右司郎中劉寧止為沿江措置副使。甲戌,移御行宮。乙亥,詔諭中外:「以迫近防秋,請太后率宗室迎奉神主如江表,百司庶
 府非軍旅之事者,並令從行。朕與輔臣宿將備禦寇敵,應接中原。官吏民士家屬南去者,有司毋禁。」金人陷磁州。是夏,賊貴仲正降。



 秋七月戊寅,贈王復為資政殿學士。己卯,親慮囚。辛巳,苗傅、劉正彥伏誅。癸未,進韓世忠檢校少保、武勝昭慶軍節度使、御營使司都統制。範瓊自洪州入朝,以瓊為御營使司提舉一行事務,後軍統制辛企宗為都統制。命學士院草夏國書、大金國表本付張浚。甲申,詔以苗、劉之變,當軸大臣不能身衛社稷,
 朱勝非、顏岐、路允迪並落職,張澄衡州居住。以知廬州胡舜陟為淮西制置使,知江州權邦彥兼本路制置使。金人犯山東,安撫使劉洪道棄濰州遁,萊州守將張成舉城降。丁亥,以範瓊跋扈無狀,收下大理獄,分其兵隸神武五軍。皇太子薨,謚元懿。戊子,鄭□薨。己丑,以資政殿大學士王綯參知政事,兵部尚書周望同簽書樞密院事。庚寅,仙井監鄉貢進士李時雨上書,乞選立宗子系屬人心,帝怒,斥還鄉里。辛卯,升杭州為臨安府。壬辰,
 言者又論範瓊逼遷徽宗及迎立張邦昌,瓊辭伏,賜死,子弟皆流嶺南。劉洪道復青州,執金守向大猷。乙未,遣謝亮使夏國。丁酉,遣崔縱使金軍前。庚子,張浚發行在。辛丑,王□燮與靳賽遇,合戰,敗績。壬寅,命李邴、滕康權知三省、樞密院事,扈從太后如洪州,楊惟忠將兵萬人以衛。以杜充同知樞密院事兼宣撫處置副使。乙巳,詔江西、閩、廣、荊湖諸路團教峒丁、槍杖手。山東賊郭仲威陷淮陽軍。翟興引兵入汝州,與賊王俊戰,敗之。



 八月己酉,
 移浙西安撫司於鎮江府。庚戌,李邴罷。壬子,以吏部尚書劉玨為端明殿學士、權同知三省、樞密院事。甲寅,王庶罷。以徽猷閣直學士、知慶陽府王似為陜西節制使。劉文舜入舒州。己未,太后發建康。丁卯,遣杜時亮使金軍前。



 閏八月丁丑朔,以胡舜陟為沿江都制置使,集英殿修撰王羲叔副之。丁亥,輔逵掠漣水軍,殺軍使郝璘,率眾降於王□燮。己丑,以呂頤浩守尚書左僕射,杜充守右僕射,並同中書門下平章事。庚寅,起居郎胡寅上書
 言二十事,呂頤浩不悅,罷之。辛卯,命杜充兼江、淮宣撫使、守建康,前軍統制王□燮隸之,韓世忠為漸西置使守鎮江,劉光世為江東宣撫使守太平、池州,並受充節制。丁酉,太后至洪州。己亥,減福建、廣南歲上供銀三之一。詔制置使唯用兵聽便宜,餘事悉禁。壬寅,帝發建康,復還浙西,張俊、辛企宗以其軍從。甲辰,次鎮江府。賜陳東家金。張浚次襄陽,招官軍、義兵分屯襄、郢、唐、鄧,以程千秋、李允文節制。是月,知濟南府宮儀及金人數戰於
 密州,兵潰,儀及劉洪道俱奔淮南,守將李逵以密州降金。靳賽詣劉光世降。



 九月丙午朔,日有食之。諜報金人治舟師,將由海道窺江、浙,遣韓世忠控守圌山、福山。辛亥,次平江府。壬子,金人陷單州、興仁府,遂陷南京,執守臣凌唐佐,降之。癸丑,以周望為兩浙、荊湖等路宣撫使,總兵守平江。翰林學士張守同簽書樞密院事。命劉光世移屯江州。丙辰,遣張邵等充金國軍前通問使。金人陷沂州。卻高麗入貢使。張浚承制罷知潭州辛炳,起復
 直龍圖閣向子諲代之。丁巳,蠲諸路青苗積欠錢。辛酉,知鼎州邢倞坐結耶律餘睹,再責汝州團練副使、英州安置。癸亥,賜宿、泗州都大提舉使李成軍絹二萬匹,成尋復叛。己巳,以胡舜陟為兩浙宣撫司參謀官,知鎮江府陳邦光為沿江都制置使。庚午,以工部侍郎湯東野知平江府兼浙西制置使。辛未,追復鄒浩龍圖閣待制。壬申夜,潭州禁卒作亂,謀竄不果,向子諲隨招安之。甲戌,金帥婁宿犯長安,經略使郭琰棄城遁,河北賊酈瓊圍光
 州。


冬十月丙子朔,詔按察官歲上所發擿贓吏姓名以為殿最。庚辰,禁諸軍擅入川、陜。癸未,帝至杭州,復如浙東。庚寅,渡浙江。郭仲威詣周望降,望以仲威為本司統制。辛卯,李成陷滁州,殺守臣向子伋。壬辰,帝至越州。癸巳,命提舉廣西峒丁李棫市馬,邕州置牧養務。戊戌,初命東南八路歲收經制五項錢輸行在。張浚治兵於興元府。金人陷壽春府。庚子,陷黃州,守臣趙令
 \gezhu{
  山成}
 死之。辛丑,張浚以同主管川、陜茶馬趙開為隨軍轉運使,專總
 四川財賦。金人自黃州濟江,劉光世引軍遁,知江州韓梠棄城去。金人自大冶縣趨洪州。是月,京西賊劉滿陷信陽軍,殺守臣趙士負。盜入宿州,殺通判盛修已。



 十一月乙巳朔,金人犯廬州,守臣李會以城降。王善叛降金,金人執之。丁未,詔降雜犯死罪,釋流以下囚,聽李綱自便,追復宋齊愈官。貴仲正犯荊南,兵馬鈐轄渠成與戰,斬之。戊申,金帥兀□犯和州,守臣李儔以城降,通判唐璟死之。己酉,張浚出行關、陜。兀□陷無為軍,守臣李知
 幾棄城走。壬子,太后退保虔州。江西制置使王子獻棄洪州走。丁巳,金人陷臨江軍,守臣吳將之遁。戊午,遣孫悟等充金國軍前致書使。金人陷洪州,權知州李積中以城降。撫、袁二州守臣王仲山、王仲嶷皆降。淮賊劉忠犯蘄州,韓世清逆戰,破之。忠入舒州,殺通判孫知微。庚申,金人陷真州,守臣向子忞棄城去。辛酉,太后至吉州。壬戌,金人犯建康府,陷溧水,縣尉潘振死之。癸亥,金人陷太平州。主管步軍司閭勍自西京奉累朝御容至
 行在,詔奉安於天慶觀,尋命勍節制淮西軍馬,以拒金人。甲子,杜充遣都統制陳淬、岳飛等及金人戰於馬家渡,王□燮以軍先遁,淬敗績,死之。乙丑,以檢正諸房公事傅崧卿為浙東防遏使。太后發吉州,次太和縣。護衛統制杜彥及後軍楊世雄率眾叛,犯永豐縣,知縣事趙訓之死之。金人至太和縣,太后自萬安陸行如虔州。丁卯,十詔回浙西迎敵。金人犯吉州,守臣楊淵棄城走,又陷六安軍。己巳,帝發越州,次錢清鎮。庚午,復還越州。以周
 望同知樞密院事,仍兼兩浙宣撫使守平江,殿前都指揮使郭仲荀為副使守越州,右軍都統制張俊為浙東制置使從行。御史中丞範宗尹參知政事。辛未,兀□入建康府,守臣陳邦光、戶部尚書李梲迎拜,通判楊邦乂拒之。癸酉,帝如明州。金人犯建昌軍,兵馬監押蔡延世擊卻之。甲戌,兀□殺楊邦乂。韓世忠自鎮江引兵之江陰軍。江、淮宣撫司潰卒李選攻陷鎮江。淮丁兵馬都監王宗望以濠州降於金。是月,張浚至秦州。桑仲自唐州
 犯襄陽,京西制置使程千秋敗走,仲遂據襄陽。



 十二月乙亥朔,張浚承制廢積石軍。丙子,帝至明州。丁丑,江、淮西撫司準備將戚方擁眾叛,犯鎮江府,殺守臣胡唐老。辛巳,金人陷常州,守臣周杞遣赤心隊官劉晏擊走之。金人陷廣德軍。殺守臣周烈。劉光世引兵趨南康軍。壬午,定議航海避兵,禁卒張寶等憚行,謀亂,命呂頤浩等伏兵,執寶等十七人斬之。甲申,張浚承制拜涇原經略使曲端為威武大將軍、宣撫處置使司都統制。乙酉,兀□犯
 臨安府,守臣康允之棄城走,錢塘縣令朱蹕死之。己丑,帝乘樓船次定海縣,給行在諸軍雪寒錢。辛卯,留範宗尹、趙鼎於明州以候金使。癸巳,帝次昌國縣。乙未,杜彥犯潭州,殺通判孟彥卿、趙民彥。金人屠洪州。戊戌,金人犯越州,安撫使李鄴以城降,衛士唐琦袖巨石要擊金帥琶八不克,死之。郭仲荀棄軍奔溫州。庚子,移幸溫、臺。癸卯,黃潛善卒於英州。李成自滁州引兵之淮西。



\end{pinyinscope}