\article{志第七十九 樂一}

\begin{pinyinscope}

 有宋之樂,自建隆訖崇寧,凡六改作。始,太祖以雅樂聲高,不合中和,乃詔和峴以王樸律準較洛陽銅望臬石尺為新度,以定律呂,故建隆以來有和峴樂。仁宗留意
 音律,判太常燕蕭言器久不諧,復以樸準考正。時李照以知音聞,謂樸準高五律,與古制殊,請依神瞽法鑄編鐘。既成,遂請改定雅樂,乃下三律,煉白石為磬,範中金為鐘,圖三辰、五靈為器之飾,故景祐中有李照樂。未幾,諫官、御史交論其非,竟復舊制。其後詔侍從、禮官參定聲律,阮逸、胡瑗實預其事,更造鐘磬,止下一律,樂名《大安》。乃試考擊,鐘聲弇鬱震掉,不和滋甚,遂獨用之常祀、朝會焉,故皇祐中有阮逸樂。神宗御歷,嗣守成憲,未遑
 制作,間從言者緒正一二。知禮院楊傑條上舊樂之失,召範鎮、劉幾與傑參議。幾、傑請遵祖訓,一切下王樸樂二律,用仁宗時所制編鐘,追考成周分樂之序,辨正二舞容節;而鎮欲求一稃二米真黍,以律生尺,改修鐘量,廢四清聲。詔悉從幾、傑議。樂成,奏之郊廟,故元豐中有楊傑、劉幾樂。範鎮言其聲雜鄭、衛,請太府銅制律造樂。哲宗嗣位,以樂來上,按試於庭,比李照樂下一律,故元祐中有範鎮樂。楊傑復議其失,謂出於鎮一家之學,卒
 置不用。徽宗銳意制作,以文太平,於是蔡京主魏漢津之說,破先儒累黍之非,用夏禹以身為度之文,以帝指為律度,鑄帝鼐、景鐘。樂成,賜名《大晟》,謂之雅樂,頒之天下,播之教坊,故崇寧以來有魏漢津樂。



 夫《韶》、《濩》之音,下逮戰國,歷千數百年,猶能使人感嘆作興。當是時,桑間、濮上之音已作,而古帝王之樂猶存,豈不以其制作有一定之器,而授受繼承亦代有其人歟?由是論之,鄭衛、《風》《雅》,不異器也。知此道也,則雖百世不易可也。禮樂道
 喪久矣,故宋之樂屢變,而卒無一定不易之論。考諸家之說,累黍既各執異論,而身為度之說尤為荒唐。方古制作,欲垂萬世,難矣!觀其高二律、下一律之說,雖賢者有所未知,直曰樂聲高下於歌聲,則童子可知矣;八音克諧之說,智者有所未諭,直以歌聲齊簫聲,以簫聲定十六聲而齊八器,則愚者可諭矣。審乎此道,以之制作,器定聲應,自不奪倫,移宮換羽,特餘事耳。去沾滯靡曼而歸之和平澹泊,大雅之音,不是過也。



 南渡之後,大抵
 皆用先朝之舊,未嘗有所改作。其後諸儒朱熹、蔡元定輩出,乃相與講明古今制作之本原,以究其歸極,著為成書,理明義析,具有條制,粲然使人知禮樂之不難行也。惜乎宋祚告終,天下未一,徒亦空言而已。



 今集累朝制作損益因革、議論是非,悉著於編,俾來者有考焉。為《樂志》。



 王者致治,有四達之道,其二曰樂,所以和民心而化天下也。歷代相因,咸有制作。唐定樂令,惟著器服之名。後
 唐莊宗起於朔野,所好不過北鄙鄭、衛而已,先王雅樂,殆將掃地。晉天福中,始詔定朝會樂章、二舞、鼓吹十二案。周世宗嘗觀樂縣,問宮人,不能答。由是患雅樂凌替,思得審音之士以考正之,乃詔翰林學士竇儼兼判太常寺,與樞密使王樸同詳定,樸作律準,編古今樂事為《正樂》。



 宋初,命儼仍兼太常。建隆元年二月,儼上言曰:「三、五之興,禮樂不相沿襲。洪惟聖宋,肇建皇極,一代之樂,宜乎立名。樂章固當易以新詞,式遵舊典。」從之,因詔儼
 專其事。儼乃改周樂文舞《崇德》之舞為《文德》之舞,武舞《象成》之舞為《武功》之舞,改樂章十二「順」為十二「安」,蓋取「治世之音安以樂」之義。祭天為《高安》,祭地為《靜安》,宗廟為《理安》,天地、宗廟登歌為《嘉安》,皇帝臨軒為《隆安》,王公出入為《正安》,皇帝食飲為《和安》,皇帝受朝、皇后入宮為《順安》皇太子軒縣出入為《良安》,正冬朝會為《永安》,郊廟俎豆入為《豐安》,祭享、酌獻、飲福、受胙為《禧安》,祭文宣王、武成王同用《永安》,籍田、先農用《靜安》。



 五月,有司上言:「僖
 祖文獻皇帝室奏《大善》之舞,順祖惠元皇帝室奏《大寧》之舞,翼祖簡恭皇帝室奏《大順》之舞,宣祖昭武皇帝室奏《大慶》之舞。」從之。



 乾德元年,翰林學士承旨陶穀等奉詔撰定祀感生帝之樂章、曲名,降神用《大安》,太尉行用《保安》,奠玉幣用《慶安》,司徒奉俎用《咸安》,酌獻用《崇安》,飲福用《廣安》,亞獻、終獻用《文安》,送神用《普安》。五代以來,樂工未具,是歲秋,行郊享之禮,詔選開封府樂工八百三十人,權隸太常習鼓吹。



 四年春,遣拾遺孫吉取成都孟昶偽
 宮縣至京師,太常官屬閱視,考其樂器,不協音律,命毀棄之。六月,判太常寺和峴言:「大樂署舊制,宮縣三十六虡設於庭,登歌兩架設於殿上。望詔有司別造,仍令徐州求泗濱石以充磬材。」許之。先是,晉開運末,禮樂之器淪陷,至是,始令有司復二舞、十二案之制。二舞郎及引舞一百五十人,按視教坊、開封樂籍,選樂工子弟以備其列,冠服準舊制。鼓吹十二案,其制:設氈床十二,為熊羆騰倚之狀,以承其下;每案設大鼓、羽葆鼓、金錞各一,
 歌、簫、笳各二,凡九人,其冠服同引舞之制。



 十月,峴又言:「樂器中有叉手笛,樂工考驗,皆與雅音相應。按唐呂才歌《白雪》之琴,馬滔進《太一》之樂,當時得與宮縣之籍。況此笛足以協十二旋相之宮,亦可通八十四調,其制如雅笛而小,長九寸,與黃鐘管等。其竅有六,左四右二,樂人執持,兩手相交,有拱揖之狀,請名之曰『拱宸管』。望於十二案、十二編磬並登歌兩架各設其一,編於令式。」詔可。



 太祖每謂雅樂聲高,近於哀思,不合中和。又念王樸、
 竇儼素名知樂,皆已淪沒,因詔峴討論其理。峴言:「以樸所定律呂之尺較西京銅望臬古制石尺短四分,樂聲之高,良由於此。」乃詔依古法別創新尺,以定律呂。自此雅音和暢,事具《律歷志》。



 自國初已來,御正殿受朝賀,用宮縣;次御別殿,群臣上壽,舉教坊樂。是歲冬至,上禦乾元殿受賀畢,群臣詣大明殿行上壽禮,始用雅樂、登歌、二舞。是月,和峴又上言:



 郊廟殿庭通用《文德》、《武功》之舞,然其綴兆未稱《武功》、《文德》之形容。又依古義,以揖讓得
 天下者,先奏文舞;以征伐得天下者,先奏武舞。陛下以推讓受禪,宜先奏文舞。按《尚書》,舜受堯禪,玄德升聞,乃命以位。請改殿宇所用文舞為《玄德升聞》之舞。其舞人,約唐太宗舞圖,用一百二十八人,以倍八佾之數,分為八行,行十六人,皆著履,執拂,服褲褶,冠進賢冠。引舞二人,各執五採纛,其舞狀、文容、變量,聊更增改。又陛下以神武平一宇內,即當次奏武舞。按《尚書》,周武王一戎衣而天下大定,請改為《天下大定》之舞,其舞人數、行列悉
 同文,其人皆被金甲、持戟。引舞二人,各執五採旗。其舞六變:一變象六師初舉,二變象上黨克平,三變象維揚底定,四變象荊湖歸復,五變象邛蜀納款,六變象兵還振旅。乃別撰舞典、樂章。其鐃、鐸、雅、相、金錞、□鼓並引二舞等工人冠服,即依樂令,而《文德》、《武功》之舞,請於郊廟仍舊通用。



 又按,唐貞觀十四年,景雲見,河水清,張文收採古《朱雁》、《天馬》之義,作《景雲河清歌》,名燕樂,元會第二奏者是也。伏見今年荊南進甘露,京兆、果州進嘉禾,
 黃州進紫芝,和州進綠毛龜,黃州進白兔。欲依月律,撰《神龜》、《甘露》、《紫芝》、《嘉禾》、《玉兔》五瑞各一曲,每朝會登歌,首奏之。



 有詔:「二舞人數衣冠悉仍舊制,樂章如所請。」



 六年,峴又言:「漢朝獲天馬、赤雁、神鼎、白麟之瑞,並為郊歌。國朝,合州進瑞木成文,馴象由遠方自至,秦州獲白烏,黃州獲白雀,並合播在管弦,薦於郊廟。」詔峴作《瑞文》、《馴象》、《玉烏》、《皓雀》四瑞樂章,以備登歌。未幾,峴復言:「按《開元禮》,郊祀,車駕還宮入喜德門,奏《採茨》之樂;入太極門,奏《太
 和》之樂。今郊祀禮畢,登樓肆赦,然後還宮,宮縣但用《隆安》,不用《採茨》。其《隆安》樂章本是御殿之辭,伏詳《禮》意,《隆安》之樂自內而出,《採茨》之樂自外而入,若不並用,有失舊典。今太樂署丞王光裕誦得唐日《採茨曲》,望依月律別撰其辭,每郊祀畢,車駕初入奏之。御樓禮畢還宮,即奏《隆安》之樂。」並從之。太常寺又言:「準令,宗廟殿庭宮縣三十虡,郊社二十虡,殿庭加鼓吹十二案。開寶四年,郊祀誤用宗廟之數,今歲親郊,欲用舊禮。」有詔,圜丘增十
 六虡,餘依前制。



 太宗太平興國二年,冬至上壽,復用教坊樂。九年,嵐州獻祥麟;雍熙中,蘇州貢白龜;端拱初,澶州河清,廣州鳳凰集;諸州麥兩穗、三穗者,連歲來上。有司請以此五瑞為《祥麟》、《丹鳳》、《河清》、《白龜》、《瑞麥》之曲,薦於朝會,從之。



 淳化二年,太子中允、直集賢院和□蒙上言:「兄峴嘗於乾德中約《唐志》故事,請改殿庭二舞之名,舞有六變之象,每變各有樂章,歌詠太祖功業。今睹來歲正會之儀,登歌五
 瑞之曲已從改制,則文武二舞亦當定其名。《周易》有『化成天下』之辭,謂文德也;漢史有『威加海內』之歌,謂武功也。望改殿庭舊用《玄德升聞》之舞為《化成天下》之舞,《天下大定》之舞為《威加海內》之舞。其舞六變:一變象登臺講武,二變象漳、泉奉土,三變象杭、越來朝,四變象克殄並、汾,五變象肅清銀、夏,六變象兵還振旅。每變樂章各一首。」詔可。



 三年,元日朝賀畢,再御朝元殿,群臣上壽,復用宮縣、二舞,登歌五瑞曲,自此遂為定制。□蒙又請取今
 朝祥瑞之殊尤者作為四瑞樂章,備郊廟奠獻,以代舊曲,詔從之。有司雖承詔,不能奉行,故今闕其曲。



 太宗嘗謂舜作五弦之琴以歌《南風》,後王因之,復加文武二弦。至道元年,乃增作九弦琴、五弦阮,別造新譜三十七卷。凡造九弦琴宮調、鳳吟商調、角調、徵調、羽調、龍仙羽調、側蜀調、黃鐘調、無射商調、瑟調變弦法各一。制宮調《鶴唳天弄》、鳳吟商調《鳳來儀弄》、龍仙羽調《八仙操》,凡三曲。又以新聲被舊曲者,宮調四十三曲,商調十三曲,角調
 二十三曲,徵調十四曲,羽調二十六曲,側蜀調四曲,黃鐘調十九曲,無射商調七曲,瑟調七曲。造五弦阮宮調、商調、鳳吟商調、角調、徵調、羽調黃鐘調、無射商調、瑟調、碧玉調、慢角調、金羽調變弦法各一。制宮調《鶴唳天弄》、鳳吟商調《鳳來儀弄》凡二曲。又以新聲被舊曲者,宮調四十四曲、商調十三曲、角調十一曲、徵調十曲、羽調十曲、黃鐘調十九曲、無射商調七曲、瑟調七曲、碧玉調十四曲、慢角調十曲、金羽調三曲。阮成,以示中書門下,因謂曰:「雅樂與
 鄭、衛不同,鄭聲淫,非中和之道。朕常思雅正之音可以治心,原古聖之旨,尚存遺美。琴七弦,朕今增之為九,其名曰君、臣、文、武、禮、樂、正、民、心,則九奏克諧而不亂矣。阮四弦,增之為五,其名曰:水、火、金、木、土,則五材並用而不悖矣。」因命待詔朱文濟、蔡裔繼琴、阮詣中書彈新聲,詔宰相及近侍咸聽焉。由是中外獻賦頌者數十人。二年,太常音律官田琮以九弦琴、五弦阮均配十二律,旋相為宮,隔八相生,並協律呂,冠於雅樂,仍具圖以獻。上覽
 而嘉之,遷其職以賞焉。自是遂廢拱宸管。



 真宗咸平四年,太常寺言:「樂工習藝匪精,每祭享郊廟,止奏黃鐘宮一調,未嘗隨月轉律,望示條約。」乃命翰林侍讀學士夏侯嶠、判寺郭贄同按試,擇其曉習月律者,悉增月奉,自餘權停廩給,再俾學習,以獎勵之。雖頗振綱紀,然亦未能精備。蓋樂工止以年勞次補,而不以藝進,至有抱其器而不能振作者,故難於驟變。



 景德二年八月,監察御史艾仲孺上言,請修飾樂器,調正音律,乃
 詔翰林學士李宗諤權判太常寺,及令內臣監修樂器。後復以龍圖閣待制戚綸同判寺事,乃命太樂、鼓吹兩署工校其優劣,黜去濫吹者五十餘人。宗諤因編次律呂法度、樂物名數,目曰《樂纂》,又裁定兩署工人試補條式及肄習程課。



 明年八月,上御崇政殿張宮縣閱試,召宰執、親王臨觀,宗諤執樂譜立侍。先以鐘磬按律準,次令登歌,鐘、磬、塤、篪、琴、阮、笙、簫各二色合奏,箏、瑟、築三色合奏,迭為一曲,復擊鎛鐘為六變、九變。又為朝會上
 壽之樂及文武二舞、鼓吹、導引、警夜之曲,頗為精習。上甚悅。舊制,巢笙、和笙每變宮之際,必換義管,然難於遽易,樂工單仲辛遂改為一定之制,不復旋易,與諸宮調皆協。又令仲辛誕唱八十四調曲,遂詔補副樂正,賜袍笏、銀帶,自餘皆賜衣帶、緡錢,又賜宗諤等器幣有差。自是,樂府制度頗有倫理。



 先是,惟天地、感生帝、宗廟用樂,親祀用宮縣,有司攝事,止用登歌,自餘大祀,未暇備樂。時既罷兵,垂意典禮,至是詔曰:「致恭明神,邦國之重事;
 升薦備樂,方冊之彞章。矧在尊神,固當嚴奉。舉行舊典,用格明靈。自今諸大祠並宜用樂,皆同感生帝,六變、八變如《通禮》所載。」



 大中祥符元年四月,詳定所言:「東封道路稍遠,欲依故事,山上園臺及山下封祀壇前俱設登歌兩架,壇下設二十架並二舞,其朝覲壇前亦設二十架,更不設熊羆十二案。」從之。



 九月,都官員外郎、判太常禮院孫奭上言:「按禮文,饗太廟終獻降階之後,武舞止,太祝徹豆,《豐安》之樂作,一成止,然後《理安》之樂作,是謂
 送神。《論語》曰:『三家者以《雍》徹。』又《周禮》樂師職曰:『及徹,帥學士而歌徹。』鄭玄曰:『謂歌《雍》也。』《郊祀錄》載登歌徹豆一章,奏無射羽。然則宗廟之樂,禮有登歌徹豆,今於終獻降階之後即作《理安之樂》,誠恐闕失,望依舊禮增用。」詔判太常寺李宗諤與檢討詳議以聞。宗諤等言:「國初撰樂章,有徹豆《豐安》曲辭,樂署因循不作,望如奭所奏。」從之。時以將行封禪,詔改酌獻昊天上帝《禧安》之樂為《封安》,皇地祇《禧安》之樂為《禪安》,飲福《禧安》之樂為《祺安》,別
 制天書樂章《瑞安》、《靈文》二曲,每親行禮用之。又作《醴泉》、《神芝》、《慶雲》、《靈鶴》、《瑞木》五曲,施於朝會、宴享,以紀瑞應。



 十月,真宗親習封禪儀於崇德殿,睹亞獻、終獻皆不作樂,因令檢討故事以聞。有司按《開寶通禮》,親郊,壇上設登歌,皇帝升降、尊獻、飲福則作樂;壇下設宮縣,降神、迎俎、退文舞、引武舞、迎送皇帝則作。亞獻、終獻、升降在退文舞引武舞之間。有司攝事,不設宮架、二舞,故三獻、升降並用登歌。今山上設登歌,山下設宮縣、二舞,其山上圜
 臺亞獻、終獻準親祠例,無用樂之文。於時特詔亞、終獻並用登歌。



 五年,聖祖降,有司言:「按唐太清宮樂章,皆明皇親制,其崇奉玉皇、聖祖及祖宗配位樂章,並望聖制。」詔可之。聖制薦獻聖祖文舞曰《發祥流慶》之舞,武舞曰《降真觀德》之舞。自是,玉清昭應宮、景靈宮親薦皆備樂,用三十六虡。景靈宮以庭狹,止用二十虡。上又取太宗所撰《萬國朝天曲》曰《同和》之舞,《平晉曲》曰《定功》之舞,親作樂辭,奏於郊廟。自時厥後,仁宗以《大明》之曲尊真宗,
 英宗以《大仁》之曲尊仁宗,神宗以《大英》之曲尊英宗。



 仁宗天聖五年十月,翰林侍講學士孫奭言:「郊廟二舞失序,願下有司考議。」於是翰林學士承旨劉筠等議曰:「周人奏《清廟》以祀文王,《執竟》以祀武王,漢高帝、文帝亦各有舞。至唐有事太廟,每室樂歌異名。蓋帝王功德既殊,舞亦隨變。屬者,有司不詳舊制,奠獻止登歌而樂舞不作,其失明甚。請如舊制,宗廟酌獻復用文舞,皇帝還版位,文舞退,武舞入。亞獻酌醴已,武舞作,至三獻已奠還
 位則正。蓋廟室各頌功德,故文舞迎神後各奏逐室之舞。郊祀則降神奏《高安》之曲,文舞已作及皇帝酌獻,惟登歌奏《禧安》之樂,而縣樂舞綴不作,亞獻、終獻仍用武舞。」詔從之。是時,仁宗始大朝會,群臣上壽,作《甘露》、《瑞木》、《嘉禾》之曲。



 明道初,章獻皇太后御前殿,見群臣,作《玉芝》、《壽星》、《奇木連理》之曲,《厚德無疆》、《四海會同》之舞。明年,太后躬謝宗廟,帝耕籍田、享先農,率有樂歌。其後親祀南郊、享太廟、奉慈廟、大享明堂、祫享,帝皆親制降神、送神、
 奠幣、瓚稞、酌獻樂章,餘詔諸臣為之。至於常祀、郊廟、社稷諸祠,亦多親制。



 景祐元年八月,判太常寺燕肅等上言:「大樂制器歲久,金石不調,願以周王樸所造律準考按修治,並閱樂工,罷其不能者。」乃命直史館宋祁、內侍李隨同肅等典其事,又命集賢校理李照預焉。於是,帝御觀文殿取律準閱視,親篆之,以屬太常。明年二月,肅等上考定樂器並見工人,帝御延福宮臨閱,奏郊廟五十一曲,因問照樂音高,命詳陳之。照言:「樸準視古樂高
 五律,視教坊樂高二律。蓋五代之亂,雅樂廢壞,樸創意造準,不合古法,用之本朝,卒無福應。又編鐘、鎛、磬無大小、輕重、厚薄、氣短之差,銅錫不精,聲韻失美,大者陵,小者抑,,非中度之器也。昔軒轅氏命伶倫截竹為律,後令神瞽協其中聲,然後聲應鳳鳴,而管之參差亦如鳳翅。其樂傳之亙古,不刊之法也。願聽臣依神瞽律法,試鑄編鐘一虡,可使度、量、權、衡協和。」乃詔於錫慶院鑄之。既成,奏御。



 照遂建議請改制大樂,取京縣秬黍累尺成律,
 鑄鐘審之,其聲猶高。更用太府布帛尺為法,乃下太常制四律。別詔潞州取羊頭山秬黍上送於官,照乃自為律管之法,以九十黍之量為四百二十星,率一星占九秒,一黍之量得四星六秒,九十黍得四百二十星,以為十二管定法。乃詔內侍鄧保信監視群工。照並引集賢校理聶冠卿為檢討雅樂制度故實官,入內都知閻文應董其事,中書門下總領焉。凡所改制,皆關中書門下詳定以聞。別詔翰林侍讀學士馮元同祁、冠卿、照討論
 樂理,為一代之典。又詔天下有深達鐘律者,在所亟以名聞。於是,杭州鄭向言阮逸、蘇州範仲淹言胡瑗皆通知古樂,詔遣詣闕。其它以樂書獻者,悉上有司。



 五月,照言:「既改制金石,則絲、笙、匏、土、革、木亦當更制,以備獻享。」奏可。照乃鑄銅為龠、合、升、斗四物,以興鐘、鎛、聲量之法,龠之率六百三十黍為黃鐘之容,合三倍於龠,升十二倍於合,斗十倍於升。乃改造諸器,以定其法。俄又以鎛之容受差大,更增六龠為合,十合為升,十升為斗,銘曰「
 樂鬥」。後數月,潞州上秬黍,照等擇大黍縱累之,檢考長短,尺成,與太府尺合,法乃定。



 先時,太常鐘磬每十六枚為虡,而四清聲相承不擊,照因上言:「十二律聲已備,餘四清聲乃鄭、衛之樂,請於編縣止留十二中聲,去四清聲,則哀思邪僻之聲無由而起也。」元等駁之曰:「前聖制樂,取法非一,故有十三管之和,十九管之巢,三十六簧之竽,十十五弦之瑟,十三弦之箏,九弦、七弦之琴,十六枚之鐘磬,各自取義,寧有一之於律呂專為十二數者?
 且鐘磬,八音之首,絲笙以下受之於均,故聖人尤所用心焉。《春秋》號樂總言金奏;《詩·頌》稱美,實依磬聲。此二器非可輕改。今照欲損為十二,不得其法,稽諸古制,臣等以為不可,且聖人既以十二律各配一鐘,又設黃鐘至夾鐘四清聲以附正聲之次,原四清之意,蓋為夷則至應鐘四宮而設也。夫五音:宮為君,商為臣,角為民,徵為事,羽為物。不相凌謂之正,迭相凌謂之慢,百王所不易也。聲重濁者為尊,輕清者為卑,卑者不可加於尊,古今
 之所同也。故列聲之尊卑者,事與物不與焉。何則?事為君治,物為君用,不能尊於君故也。惟君、臣、民三者則自有上下之分,不得相越。故四清聲之設,正謂臣民相避以為尊卑也。今若止用十二鐘旋相考擊,至夷則以下四管為宮之時,臣民相越,上下交戾,則凌犯之音作矣。此甚不可者也。其鐘、磬十六,皆本周、漢諸儒之說及唐家典法所載,欲損為十二,惟照獨見,臣以為且如舊制便。」帝令權用十二枚為一格,且詔曰:「俟有知者,能考四
 鐘協調清濁,有司別議以聞。」鐘舊飾旋蟲,改為龍。乃遣使採泗濱浮石千餘段以為縣磬。



 先是,宋祁上言:「縣設建鼓,初不考擊,又無三□,且舊用諸鼓率多陋敝。」於是敕元等詳求典故而言曰:「建鼓四,今皆具而不擊,別設四散鼓於縣間擊之,以代建鼓。乾德四年,秘書監尹拙上言:『散鼓不詳所置之由,且於古無文,去之便。』時雖奏可,而散鼓於今仍在。又雷鼓、靈鼓、路鼓雖擊之,皆不成聲,故常賴散鼓以為樂節,而雷□、靈□、路□闕而未制。今既
 修正雅樂,謂宜申敕大匠改作諸鼓,使擊考有聲。及創為三□,如古之制,使先播之,以通三鼓。罷四散鼓,如乾德詔書。」奏可。



 時有上言,以為雷鼓八面,前世用以迎神,不載考擊之法,而大樂所制,以柱貫中,故擊之無聲。更令改造,山趺上出雲以承鼓,刻龍以飾柱,面各一工擊鼓,一工左執□以先引。凡圓丘降神六變,初八面皆三擊,椎而左旋,三步則止。三者,取陽數也。又載擊以為節,率以此法至六成。靈鼓、路鼓亦如之。植建鼓於四隅,皆
 有左鞞、右應。乾隅,左鞞應鐘,亥之位也;中鼓黃鐘,子之位也;右應大呂,醜之位也。艮隅,左鞞太簇,寅之位也;中鼓夾鐘,卯之位也;右應姑洗,辰之位也。巽隅,右應仲呂,巳之位也;中鼓蕤賓,午之位也;左鞞林鐘,未之位也。坤隅,右應夷則,申之位也;中鼓南呂,酉之位也;左鞞無射,戌之位也。宜隨月建,依律呂之均擊之。後照等復以殿庭備奏,四隅既隨月協均,顧無以節樂,而《周官·鼓人》「以晉鼓鼓金奏」,應以施用。詔依《周官》舊法制焉。於是縣內
 始有晉鼓矣。



 古者,鎛鐘擊為節檢,而無合曲之義,大射有二鎛,皆亂擊焉。後周以十二鎛相生擊之。景德中,李宗諤領太常,總考十二鎛鐘,而樂工相承,殿庭習用三調六曲。三調者,黃鐘、大簇、蕤賓也;六曲者,調別有《隆安》、《正安》二曲。郊廟之縣則環而擊之。宗諤上言曰:金部之中,鎛鐘為難和,一聲不及,則宮商失序,使十二鎛工皆精習,則遲速有倫,隨月用律,諸曲無不通矣。」真宗因詔黃鐘、太簇二宮更增文舞、武舞、福酒三曲。至是,詔元等
 詢考擊之法,元等奏言:「後周嘗以相生之法擊之,音韻克諧,國朝亦用隨均合曲,然但施殿庭,未及郊廟。謂宜使十二鐘依辰列位,隨均為節,便於合樂,仍得並施郊廟。若軒縣以下則不用此制,所以重備樂尊王制也。」詔從焉。



 隋制,內宮縣二十虡,以大磬代鎛鐘而去建鼓。唐武后稱制,改用鐘,因而莫革。及是,乃詔訪元等曰:「大磬應何法考擊,何禮應用?」元等具言:「古者,特磬以代鎛鐘,本施內宮,遂及柔祀,隋、唐之代,繼有因改。先皇帝東禪
 梁甫,西瘞汾陰,並仍舊章,陳於縣奏。若其所用,吉禮則中宮之縣,祀禮則皇地祇、神州地祇、先蠶、今之奉慈廟、後廟,皆應陳設。宮縣則三十六虡,去四隅建鼓,如古便。若考擊之法,謂宜同於鎛鐘。比緣詔旨,不俾循環互擊,而立依均合曲之制,則特磬固應不出本均,與編磬相應,為樂之節也。」詔可。



 九月,翰林學士承旨章得像等言:「宋祁所上《大樂圖義》,其論武舞所執九器,經、禮但舉其凡而不著言其用後先,故旅進輩作而無終始之別。且
 □者,所謂導舞也;鐸者,所謂通鼓也;錞者,所謂和鼓也;鐃者,所謂止鼓也;相者,所謂輔樂也;雅者,所謂陔步也。寧有導舞方始而參以止鼓,止鼓既搖而亂以通鐸?臣謂當舞入之時,左執干,右執戚,離為八列,別使工人執旌最前,□、鐸以發之,錞以和之,左執相以輔之,右執雅以節之。及舞之將成也,則鳴鐃以退行列,築雅以陔步武,□、鐸、錞、相皆止而不作。如此則庶協舞儀,請如祁所論。」其冬,帝躬款奉慈廟,樂縣罷建鼓,始以賣代鎛鐘。



 禮
 官又言:「《春秋·隱公五年》:『考仲子之宮,初獻六羽。』何休、範寧等咸謂,不言佾者,明佾則干舞在其中,婦人無武事,獨奏文樂也。江左宋建平、王宏皆據以為說,故章皇后廟獨用文舞。至唐垂拱以來,中宮之縣既用鎛鐘,其後相承,故儀坤等廟獻武舞,備鐘石之樂,尤為失禮。前詔議奉慈之樂,有司援舊典,已用特磬代鎛鐘,取陰教尚柔,以靜為體。今樂去大鐘而舞進乾盾,頗戾經旨,請止用《文德》之舞。」奏可。



 大樂塤,舊以漆飾,敕令黃其色,以本
 土音。或奏言:「柷舊以方畫木為之,外圖以時卉則可矣,而中設一色,非稱也。先儒之說曰:『有柄,連底挏之。』鄭康成以為設椎其中撞之。今當創法垂久,用明制作之意有所本焉。柷之中,東方圖以青,隱而為青龍;南方圖以赤,隱而為丹鳳;西方圖以白,隱而為騶虞;北方圖以黑,隱而為靈龜;中央圖以黃,隱而為神蚓。撞擊之法,宜用康成之說。」從之。又詔以新制雙鳳管付大樂局,其制,合二管以足律聲,管端刻飾雙鳳,施兩簧焉。照因自造葦
 鑰、清管、簫管、清笛、雅笛、大笙、大竽、宮琴、宮瑟、大阮、大嵇,凡十一種,求備雅器。詔許以大竽、大笙二種下大樂用之。



 時又出兩儀琴及十二弦琴二種,以備雅樂。兩儀琴者,施兩弦、十二柱;十二弦琴者,如常琴之制而增其弦,皆以象律呂之數。又敕更造七弦、九弦琴,皆令圓其首者以祀天,方其首者以祀地。



 帝乃親制樂曲,以夾鐘之宮、黃鐘之角、太簇之徵、姑洗之羽,作《景安》之曲,以祀昊天。更以《高安》祀五帝、日月,作《太安》以享景靈宮,罷舊《真安》之曲。以黃
 鐘之宮、大呂之角、太簇之徵、應鐘之羽作《興安》,以獻宗廟,罷舊《理安》之曲。《景安》、《興安》惟乘輿親行則用之。以姑洗之角、林鐘之徵、黃鐘之宮、太簇之角、南呂之羽作《祐安》之曲,以酌獻五帝。以林鐘之宮、太簇之角、姑洗之徵、南呂之羽作《寧安》之曲,以祭地及太社、太稷,罷舊《靖安》之曲。



 於時制詔有司,以太祖、太宗、真宗三聖並侑,乃以黃鐘之宮作《廣安》之曲以奠幣、《彰安》之曲以酌獻。又詔,躬謁奉慈廟章獻皇后之室,作《達安》之曲以奠瓚、《厚安》
 以酌獻;章懿皇后之室,作《報安》之曲以奠瓚、《衍安》以酌獻。皇帝入出作《乾安》,罷舊《降安》之曲。常祀:至日祀圜丘,太祖配,以黃鐘之宮作《定安》以奠幣隆《英安》以酌獻;孟春祀感生帝,宣祖配,以太簇之宮作《皇安》以奠幣、《肅安》以酌獻;祈穀祀昊天,太宗配,作《仁安》以奠幣、《紹安》以酌獻;孟夏雩上帝地祇太祖配,以仲呂之宮作《獻安》以奠幣、《感安》以酌獻;夏至祭皇地祇,太祖配,以蕤賓之宮作《恭安》以奠幣、《英安》以酌獻;季秋大饗明堂,真宗配,以無射之宮
 作《誠安》以奠幣、《德安》以酌獻;孟冬祭神州地祇,太宗配,以應鐘之宮作《化安》以奠幣、《韶安》以酌獻。又造《沖安》之曲,以七均演之為八十四,皆作聲譜以授有司,《沖安》之曲獨未施行。親制郊廟樂章二十一曲,財成頌體,告於神明,詔宰臣呂夷簡等分造樂章,參施群祀。



 又為《景祐樂髓新經》,凡六篇:第一,釋十二均;第二,明所主事;第三,辨音聲;第四,圖律呂相生,並祭天地、宗廟用律及陰陽數配;第五,十二管長短;第六,歷代度、量、衡。皆本之於陰陽,
 配之於四時,建之於日辰,通之於鞮笁,演之於壬式遁甲之法,以授樂府,以考正聲,以賜群臣焉。



 初,照等改造金石所用員程凡七百十四:攻金之工百五十三,攻木之工二百十六,攻皮之工四十九,刮摩之工九十一,搏埴之工十六,設色之工百八十九。起五月,止九月,成金石具七縣。至於鼓吹及十二案,悉修飾之。令冠卿等纂《景祐大樂圖》二十篇,以載熔金鑢石之法、歷世八音諸器異同之狀、新舊律管之差。是月,與新樂並獻於崇政
 殿,詔中書、門下、樞密院大臣預觀焉。自董監而下至工徒凡七百餘人,進秩賞、賜各有差。其年十一月,有事南郊,悉以新樂並聖制及諸臣樂章用之。



 先是,左司諫姚仲孫言:「照所制樂多詭異,至如煉白石以為磬,範中金以作鐘,又欲以三辰、五靈為樂器之飾。臣愚,竊有所疑。自祖宗考正大樂,薦之郊廟,垂七十年,一旦黜廢而用新器,臣竊以為不可。」御史曹修睦亦為言。帝既許照制器,且欲究其術之是非,故不聽焉。



\end{pinyinscope}