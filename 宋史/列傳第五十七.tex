\article{列傳第五十七}

\begin{pinyinscope}

 彭
 乘嵇穎梅摯司馬池子旦從子裏曾孫樸李及燕肅子度孫瑛蔣堂劉夔馬亮陳希亮



 彭乘,字利建,益州華陽人。少以好學稱州里,進士及第。
 嘗與同年生登相國寺閣,皆瞻顧鄉關,有從宦之樂,乘獨西望,悵然曰:「親長矣,安敢舍晨昏之奉,而圖一身之榮乎!」翌日,奏乞侍養。居數日,授漢陽軍判官,遂得請以歸。久之,有薦其文行者,召試,為館閣校勘。固辭還家,後復除鳳州團練推官。



 天禧初,用寇準薦,為館閣校勘,改天平軍節度推官。預校正南、北《史》、《隋書》,改秘書省著作佐郎,遷本省丞、集賢校理。懇求便親,得知普州,蜀人得守鄉郡自乘始。普人鮮知學,乘為興學,召其子弟為生
 員教育之。乘父卒,既葬,有甘露降於墓柏,人以為孝感。服除,知荊門軍,改太常博士。召還,同判尚書刑部,出知安州,徙提點京西刑獄,改夔州路轉運使。會土賊田忠霸誘下溪州蠻將內寇,乘適按郡至境,大集邊吏,勒兵下山以備賊,賊遁去。因遣人間之,其黨斬忠霸,夷其家。召修起居注,擢知制誥,累遷工部郎中,入翰林為學士,領吏部流內銓、三班院,為群牧使。既病,仁宗敕太醫診視,賜以禁中珍劑。卒,賜白金三百兩。御史知雜何郯論
 請贈官,不許,詔一子給奉終喪。



 初,修起居注缺中書舍人,而乘在選中,帝指乘曰:「此老儒也,雅有恬退名,無以易之。」及召見,諭曰:「卿先朝舊臣,久補外,而未嘗自言。」對曰:「臣生孤遠,自量其分,安敢過有所望。」帝頗嘉之。乘質重寡言,性純孝,不喜事生業。聚書萬餘卷,皆手自刊校,蜀中所傳書,多出於乘。晚歲,歷典贊命,而文辭少工云。



 嵇穎,字公實,應天宋城人。父適,嘗為石首主簿。民有父子坐重系,府檄適按之,抵其父於法,而子獲免;父死,假
 人言曰:「主簿,仁人也,行且生賢子,後必大。」明年穎生。



 天聖中,進士及第,授蔡州團練判官。王曾知青州、徙天雄軍,皆闢為從事。後用曾薦,遷太子中允,為集賢校理。歷開封府推官、三司度支判官、同修起居注,擢知制誥,累遷尚書兵部員外郎。召入翰林為學士,未及謝,卒。詔以告敕、襲衣、金帶、鞍勒馬賜其家。



 穎舉進士,時王曾、張知白相繼為南京留守,見穎謹厚篤學,謂其子弟曰:「若曹師表也。」張堯封嘗從穎學,所為文,多留穎家。其後堯封
 女入禁中,為修媛,甚被寵幸,令其弟化基詣穎,求編次其父稿,為序以獻之。穎不答,亦不以獻。



 梅摯,字公儀,成都新繁人。進士,起家大理評事、知藍田上元縣,徙知昭州,通判蘇州。二浙饑,官貸種食,已而督償頗急,摯言借貸本以行惠,乃重困民,詔緩輸期。



 慶歷中,擢殿中侍御史。時數有災異,引《洪範》上《變戒》曰:「『王省惟歲』,謂王總群吏如歲,四時有不順,則省其職。今日食於春,地震於夏,雨水於秋。一歲而變及三時,此天意以
 陛下省職未至,而丁寧戒告也。伊、洛暴漲漂廬舍,海水入臺州殺人民,浙江潰防,黃河溢埽,所謂『水不潤下』。陛下宜躬責修德,以回上帝之眷祐。陰不勝陽,則災異衰止,而盛德日起矣。」



 徙開封府推官,遷判官。僧常瑩以簡札達宮人,輦官鄭玉醉呼,歐徼巡卒,皆釋不問,摯請悉杖配之。改度支判官,進侍御史。論石元孫「不死行陳,系縲以還,國之辱也,不斬無以厲邊臣。」再奏不報。李用和除宣徽使,加同中書門下平章事。摯言:「國初,杜審瓊
 亦帝舅也,官止大將軍;李繼隆累有戰功,晚年始拜使相。祖宗慎名器如此,今不宜亟授無功。」以戶部員外郎兼侍御史知雜事、權判大理寺。言:「權陜西轉運使張堯佐非才,繇宮掖以進,恐上累聖德。」及奏減資政殿學士員,召待制官同議政,復百官轉對。帝謂大臣曰:「梅摯言事有體。」以為戶部副使。



 會宴契丹使紫宸殿,三司副使當坐殿東廡下。同列有謂曲宴例坐殿上,而大宴當止殿門外爾。因不即坐,與劉湜、陳洎趨出。降知海州,徙蘇州,
 人為度支副使。初,河北歲饑,三司益漕江、淮米餉河北。後江、淮饑,有司尚責其數,摯奏減之。



 擢天章閣待制、陜西都轉運使。還判吏部流內銓,進龍圖閣學士、知滑州。州歲備河,調丁壯伐灘葦,摯以疲民,奏用州兵代之。河大漲,將決,夜率官屬督工徒完堤,水不為患,詔獎其勞。勾當三班院、同知貢舉。請知杭州,帝賜詩寵行。累遷右諫議大夫,徙江寧府,又徙河中。卒。



 摯性淳靜,不為矯厲之行,政跡如其為人。平居未嘗問生業,喜為詩,多警句。
 有奏議四十餘篇。



 司馬池,字和中,自言晉安平獻王孚後,征東大將軍陽葬安邑瀾洄曲,後魏析安邑置夏縣,遂為縣人。池少喪父,家貲數十萬,悉推諸父,而自力讀書。時議者以蒲板、竇津、大陽路官運鹽回遠聞,乃開□口道,自聞喜逾山而抵垣曲,咸以為便。池謂人曰:「昔人何為舍徑而就迂,殆必有未便者。」眾不以為然。未幾,山水暴至,鹽車人牛盡沒入河,眾乃服。



 舉進士,當試殿庭而報母亡,友匿其
 書。池心動,夜不能寐,曰:「吾母素多疾,家豈無有異乎?」行至宮城門,徘徊不能入。因語其友,而友止以母疾告,遂號慟而歸。後中第,授永寧主簿。出入乘驢。與令相惡,池以公事謁令,令南向踞坐不起,池挽令西向偶坐論事,不為少屈。歷建德、郫縣尉。蜀人妄言戍兵叛,蠻將入寇,富人爭瘞金銀逃山谷間。令閭丘夢松假他事上府,主簿稱疾不出,池攝縣事。會上元張燈,乃縱民游觀,凡三夕,民心遂安。



 調鄭州防禦判官、知光山縣。禁中營造,詔
 諸州調竹木,州符期三日畢輸。池以土不產大竹,轉市蘄、黃,非三日可致,乃更與民自為期,約過不輸者罪之,既而輸竹先諸縣。



 盛度薦於朝,改秘書省著作佐郎、監安豐酒稅,徙知小溪縣。劉燁知河南府,闢知司錄參軍事,歲餘,通判留守司。樞密使曹利用奏為群牧判官,辭不就,朝廷固授之。利用嘗委括大臣所負進馬價,池曰:「令之不行,由上犯之。公所負尚多,不先輸,何以趣他人。」利用驚曰:「吏紿我已輸矣。」亟命送官,數日而諸負者皆
 入。利用貶,其黨畏罪,徒而毀短者甚眾,池獨揚言於朝,稱利用枉,朝廷卒不問。



 會詔百官轉對,池言:「唐制門下省,詔書之出,有不便者得以封還。今門下雖有封駁之名,而詔書一切自中書以下,非所以防過舉也。」內侍皇甫繼明給事章獻太後閣,兼領估馬司,自言估馬有羨利,乞遷官。事下群牧司,閱無羨利。繼明方用事,自制置使以下皆欲附會為奏,池獨不可。除開封府推官,敕至閣門,為繼明黨所沮,罷知耀州。擢利州路轉運使、知鳳
 翔府。



 召知諫院,上表懇辭。仁宗謂宰相曰:「人皆嗜進,而池獨嗜退,亦難能也。」加直史館,復知鳳翔。有疑獄上讞,大理輒復下,掾屬惶遽引咎。池曰:「長吏者政事所繇,非諸君過。」乃獨承其罪,有詔勿劾。岐陽鎮巡檢夜飲富民家,所部卒執之,俾為約,不敢復督士卒,而後釋其縛;池捕首惡誅之,巡檢亦坐廢。



 累遷尚書兵部員外郎,遂兼侍御史知雜事。嘗言:「陜西用兵無宿將,劉平好自用而少智謀,必誤大事。」後平果敗。更戶部度支、鹽鐵副使。歲
 滿,中書進名,帝曰:「是固辭諫官者。」擢天章閣待制、知河中府,徙同州,又徙杭州。



 池性質易,不飾廚傳,剸劇非所長,又不知吳俗,以是謗譏聞朝廷。轉運使江鈞、張從革劾池決事不當十餘條,及稽留德音,降知虢州。初,轉運使既奏池,會吏有盜官銀器,械州獄,自陳為鈞掌私廚,出所賣過半;又越州通判載私物盜稅,乃從革之姻,遣人私請。或謂池可舉劾以報仇,池曰:「吾不為也。」人稱其長者。徙知晉州,卒。子旦、光,光自有傳。從子里。



 旦字伯康。清直敏強,雖小事必審思,度不中不釋。以父任,為秘書省校書郎,歷鄭縣主簿。鄭有婦藺訟奪人田者,家多金錢,市黨買吏,合為奸謾,十年不決。旦取案一閱,情偽立見,黜吏十數輩,冤者以直。又井元慶豪欺鄉里,莫敢誰何,旦擒致於法。時旦年尚少,上下易之,自是驚服。吏捕蝗,因緣搔民。旦言:「蝗,民之仇,宜聽自捕,輸之官。」後著為令。丁內外艱,服除,監饒州永平鑄錢監。知祁縣,天大旱,人乏食,群盜剽敓,富家巨室至以兵自備。旦
 召富者開以禍福,於是爭出粟,減直以糶,猶不失其贏,饑者獲濟,盜患亦弭。



 舉監在京百萬倉,時祁隸太原,以太原留,不召。通判乾州,未行,舉監在京雜物庫。知宜興縣,其民囂訟,旦每獄必窮根株,痛繩之,校系縣門,民稍以詆冒為恥。市貫大溪,賈昌朝所作長橋,壞廢歲久,旦勸民葺復,不勞而成。



 時王安石守常州,開運河,調夫諸縣。旦言:「役大而亟,民有不勝,則其患非徒不可就而已。請令諸縣歲遞一役,雖緩必成。」安石不聽。秋,大霖雨,民
 苦之,多自經死,役竟罷。歷知梁山軍、安州。旦治郡有大體,所施設,取於適理便事。再監鳳翔太平宮,以熙寧八年致仕。歷官十七,遷至太中大夫。元祐二年,卒,年八十二。



 旦澹薄無欲,奉養茍完,人不見其貴。與弟光尤友愛終始,人無間言。光居洛,旦居夏縣,皆有園沼勝概。光歲一往省旦,旦亦間至洛視光。凡光平時所與論天下事,旦有助焉。及光被門下侍郎召,固辭不拜。旦引大義語之曰:「生平誦堯、舜之道,思致其君,今時可而違,非進退
 之正也。」光幡然就位。方是時,天下懼光之終不出,及聞此,皆欣然稱旦曰:「長者之言也。」



 英宗即位,例以親屬入賀得官,時旦在梁山,諸孫未仕者皆不遣,惟遣其從兄子SU。旦與人交以信義,喜周其急。嘗有以罪免官貧不能存者,月分俸濟之,其人無以報,願以女為妾。旦驚謝之,亟出妻奩中物使嫁之。旦生於丙午,與文彥博、程公珣、席汝言為同年會,賦詩繪像,世以為盛事,比唐九老。三子:良,試將作監主簿;富永,承議郎、陜州通判;宏,陳留
 令。宏子樸。



 裏字昭遠。進士釋褐,授威勝軍判官,改大理寺丞。龐籍為鄜延經略使,奏通判鄜州。州將武人,不法,里平居與之歡甚,臨事正色力爭,不少假借。性廉靜質直,所至有惠政。每罷官,至京師,未嘗有所謁視。審官榜久闕,人所不取者,乃受之而去。後知幹州,為太常少卿而卒。



 樸字文季,少育於外祖范純仁。紹聖黨事起,父宏上書論辨得罪。純仁責永州,疾失明,客至,必令樸導以見。時
 方七歲,進揖應對如成人,客皆驚嘆。以純仁遺恩為官。宏死,徒跣負柩還。調晉寧軍士曹參軍。通判不法,轉運使王似諷樸伺其過,樸不可,曰:「下吏而陷長官,不唯亂常,人且不食吾餘矣,死不敢奉教。」似賢而薦之。



 靖康初,入為虞部、右司員外郎。金人次汴郊,命樸使之。二酋問樸家世,具以告。喜曰:「賢者之後也。」待之加禮,乃吐腹心,諭以亟求講解。樸復命,任事者疑不決。都城陷,宗思樸之言,以為兵部侍郎。二帝將北遷,又貽書請存立
 趙氏,金人憚之,挾以北去,且悉取其孥。開封儀曹趙鼎,為匿其長子倬於蜀,故得免。



 建炎登極,赦至燕,樸私令繼詣徽宗,為人所告。金主憐其忠,釋之。徽宗崩,樸與奉使朱弁在燕共議制服,弁欲先請,樸曰:「為臣子聞君父喪,當致其哀,尚何請。設請而不許,奈何?」遂服斬衰,朝夕哭。金人亦義而不問。又遣朱松年間行,以金人情實歸報。宋因王倫出使,持黃金賜樸。倫還,言金命樸為行臺左丞,樸辭而止,益重之。後卒於真定。訃聞,詔稱其忠節顯
 著,贈兵部尚書,謚曰忠潔。



 李及,字幼幾,其先範陽人,後徙鄭州。父覃,左拾遺。及舉進士,再調升州觀察推官。寇準薦其才,擢大理寺丞、知興化軍。以殿中丞通判曹州。州民趙諫者,素無賴,持郡短長,縱為奸利。及受命,諫在京師,乃謁及,及不之見,慢罵而去,投匿名書誣及,因以毀朝政。會上封者發諫事,命轉運使與及察其狀。及條上諫前後所為不道,詔御史劾得其實,斬於都市,及由是知名。擢知隴州。



 初,置提
 點刑獄,內出及與陳綱二人名付中書。明日,以綱使河北,及使陜西,特遷一官。還判三司磨勘司,出知鳳翔府,徙延州,除三司戶部副使,為淮南轉運使,累遷太常少卿、知秦州。議者以及謹厚,非守邊才。及至秦州,州將吏亦頗易之。會有禁卒白晝攫婦人金釵於市,束執以來。及方坐觀書,召之使前,略加詰問,其人服罪。及亟命斬之,觀書如故,於是將士皆驚服。改左司郎中、樞密直學士,以右諫議大夫召還,勾當三班院,再遷尚書工部侍
 郎,歷知杭州、鄆州、應天、河南府,召拜御史中丞。卒,年七十。特贈禮部尚書,謚恭惠。



 及資質清介,所治簡嚴,喜慰薦下吏,而樂道人之善。在杭州,惡其風俗輕靡,不事宴游。一日,冒雪出郊,眾謂當置酒召客,乃獨造林逋清談,至暮而歸。居官數年,未嘗市吳中物。比去,唯市《白樂天集》。在河南,杜衍為提點刑獄,間與衍會,而具甚疏薄。他日,中貴人用事者至,亦無加品,衍嘆其清德。娶張氏,性嫉悍。及嘗生子,鞠之外舍,張固請歸保養之,乃會親屬,
 以子擊堂柱,碎其首。及遂無子,以弟之子為後。



 燕肅,字穆之,青州益都人。父峻,慷慨任俠,楊光遠反時,率其屬迎符彥卿,遂家曹州。肅少孤貧,游學。舉進士,補鳳翔府觀察推官。寇準知府事,薦改秘書省著作佐郎、知臨邛縣。縣民嘗苦吏追擾,肅削木為牘,民訟有連逮者,書其姓名,使自召之,皆如期至。知考城縣,通判河南府。召為監察御史,準方知河南,奏留之。



 遷殿中侍御史、提點廣南西路刑獄,遷侍御史,徙廣南東路。還,為丁謂
 所惡,出知越州。徙明州,俗輕悍喜鬥,肅下令獨罪先毆者,於是鬥者為息。直昭文館,為定王府記室參軍,判尚書刑部。建言:「京師大闢一覆奏,而州郡之獄有疑及情可憫者上請,多為法司所駁,乃得不應奏之罪。願如京師,死許覆奏。」遂詔疑獄及情可憫者上請,語在《刑法志》。其後大闢上請者多得貸,議自肅始。



 擢龍圖閣待制、權知審刑院、知梓州,還,同糾察在京刑獄,再判刑部,累遷左諫議大夫、知亳州,徙青州。屬歲歉,命兼京東安撫使。入
 判太常寺兼大理寺,復知審刑。肅言:「舊太常鐘磬皆設色,每三歲親祠,則重飾之。歲既久,所塗積厚,聲益不協。」乃詔與李照、宋祁同按王樸律,即劃滌考擊,合以律準,試於後苑,聲皆協。又詔與章得像、馮元詳刻漏。進龍圖閣直學士、知穎州,徙鄧州。官至禮部侍郎致仕,卒。



 肅喜為詩,其多至數千篇。性精巧,能畫,入妙品,圖山水罨布濃淡,意象微遠,尤善為古木折竹。嘗造指南、記里鼓二車及欹器以獻,又上《蓮花漏法》。詔司天臺考於鐘鼓樓
 下,云不與《崇天歷》合。然肅所至,皆刻石以記其法,州郡用之以候昏曉,世推其精密。在明州,為《海潮圖》,著《海潮論》二篇。子度,孫瑛。



 度字唐卿。登進士第,知陳留縣。京東蝗,年饑盜發,度勸邑豪出粟六萬以濟民,又行保伍法以察盜,善狀日聞。通判永興軍。三司使王堯臣舉為戶部判官,以伐閱淺,始命權發遣,遂為故事。



 出知滑。滑與黎陽對境,河埽下臨魏都,霖潦暴至,薪芻不屬。度曰:「魏實為河朔根本,不
 可坐視成敗。」悉以所儲茭楗御之,埽賴以不潰。復為戶部判官。歲皇祐甲午,益州言:「歲在甲午,蜀再亂,今又值之,民為戚戚。」乃命度出使備不虞,還奏無足慮。權河北轉運副使,六塔河決,坐貶秩知蔡州,徙福州。閩故多盜,度請假事權制攝一道,遂加兵馬鈐轄。入為戶部副使,以右諫議大夫知潭州。卒,年七十。



 度有心計,凡六佐大農。慶歷中,三司請榷河北鹽。度言:「川峽不榷酒,河北不禁鹽,此祖宗順民俗,不易之制也,榷之非是。」會張方平
 亦論之,議遂寢。



 瑛字仁叔,以蔭為瑕丘尉。縣人習為盜,瑛榜諭曰:「今平民或呼以盜,必怒見詞色,顧乃舍耕稼本業,為人所不肯為者。及陷於罪,則終身不齒於鄉閭,尉不忍以是待汝。」盜感悟,為稍弭。累遷太府丞、開封少尹。歷廣東轉運判官,進副使,加進秘閣。時方尚老氏教,瑛言:「守臣任滿考課,乞以興崇教法、拯葺道宮為善最。」從之。連進直龍圖閣。



 時瑛在嶺嶠七年,括南海犀珠、香藥,奉宰相內侍,
 人目之為「香燕」。遂以徽猷閣待制提舉醴泉觀,拜戶部侍郎。徽宗賜書「仁人義士之家」以表之,蓋取王安石頌其曾大父肅詩語也。轉開封尹,賜進士出身,兼侍讀,且將大用。後以御史言瑛不能撥煩戢奸吏,致賊殺不辜,罷為龍圖閣直學士。未數月,為戶部尚書。



 靖康初,以龍圖閣學士知河陽。金兵入寇,三城當兵沖,瑛至,未及備,而兵騎大集,乘銳攻城,瑛不能御,將出奔,為亂兵所害,年五十。建炎初,賜端明殿學士。



 蔣堂,字希魯,常州宜興人。擢進士第,為楚州團練推官。滿歲,吏部引對,真宗覽所試判,善之,特授大理寺丞、知臨川縣。縣富人李甲多為不法,前令莫能制,堂戒諭不悛,白州以兵索其家,得僭乘輿物,置於死。



 歷通判眉、許、吉、楚州,以太常博士知泗州,召為監察御史。禁中火,有司請究所起,多引宮人屬吏。堂言:「火起無跡,安知非天意也,陛下宜修德應變。有司乃欲歸咎宮人,以之屬吏,何求不可,而遂賜之死,是重天譴也。」詔原之。論奏郭皇
 后不當廢,坐贖。再遷侍御史、判三司度支勾院,出為江南東路轉運使,徙淮南,兼江、淮發運事。



 時廢發運使,上封者屢以為非便。堂言:「唐裴耀卿、劉晏、第五琦、李巽、裴休,皆嘗為江淮、河南轉運使,不聞別置使名。國朝卞袞、王嗣宗、劉師道,亦止為轉運兼領發運司事,而歲輸京師常足。」時雖用其議,後卒復。在江、淮,歲薦部史二百人。或謂曰:「一有謬舉,且得罪,何以多為?」堂曰:「十得二三,亦足報國。」坐失按蘄州王蒙正故入部吏死罪,降知越州。
 州之鑒湖,馬臻所為,溉田八千頃,食利者萬家,前守建言聽民自占,多為豪右所侵,堂奏復之。



 徙蘇州,入判刑部,徙戶部勾院,歷戶部、度支、鹽鐵副使,安撫梓夔路,擢天章閣待制、江淮制置發運使。先是,發運使上計,造大舟數十,載江、湖物入遺京師權貴,堂曰:「吾豈為此,歲入自可附驛奉也。」前後五年,未嘗一至京師。就除河東路都轉運使,未行,知洪州。改應天府,累遷左司郎中、知杭州,以樞密直學士知益州。



 慶歷初,詔天下建學。漢文翁
 石室在孔子廟中,堂因廣其舍為學宮,選屬官以教諸生,士人翕然稱之。楊日嚴在蜀,有能名,堂素不樂之。於是節游宴,減廚傳,專尚寬縱,頗變日嚴之政。又建銅壺閣,其制宏敞,而材不預具,功既半,乃伐喬木於蜀先主惠陵、江瀆祠,又毀後土及劉禪祠,蜀人浸不悅,獄訟滋多。久之,或以為私官妓,徙河中府,又徙杭州、蘇州。以尚書禮部侍郎致仕,卒,特贈吏部侍郎。



 堂為人清修純飭,遇事毅然不屈,貧而樂施。好學,工文辭,延譽晚進,至老
 不倦,尤嗜作詩,有《吳門集》二十卷。



 劉夔,字道元,建州崇安人。進士中第,補廣德軍判官,累遷尚書屯田員外郎,權侍御史。李照改制大樂鐘磬,夔以為:「樂之大本,與政化通,不當輕易其器。願擇博學之士以補卿、丞,凡四方妄獻說以要進者,請一切罷之。」帝善其言。



 歷三司戶部判官,判度支勾院,江西、兩浙、淮南轉運使,加直史館、知陜州,改太常少卿、知廣州。所至有廉名。權三司度支副使。桂陽監蠻唐和寇邊,以右諫議
 大夫、龍圖閣直學士知潭州,兼湖南安撫使。初至,遣人諭蠻酋使降;不從,乃舉兵擊敗和於銀江源,進破其巢穴,蠻逃遁遠去。前將以帛購蠻首,至是有持首取購者,按問,乃輒殺平民,誅之而罷購,州境獲安。還,權判吏部流內銓、知審刑院。



 河北大水,民流入京東為盜,詔增京東守備。帝問誰可守鄆者,宰相以夔對,進給事中、樞密直學士以往。至鄆,發廩振饑,民賴全活者甚眾,盜賊衰息,賜書褒諭。大臣議欲修復河故道,夔極言其不可,遂
 罷。遷工部侍郎、知福州。請解官入武夷山為道士,弗許。知建州,尋告老,遂以戶部侍郎致仕。英宗即位,遷吏部。卒,年八十三。



 夔嘗過江東,見二囚系累年矣。問之,曰:「前此殺吉州掾徐咸,疑二人者。」夔為言於朝,釋之,後果得真盜。嘗遇隱者,得養生術,遂蔬食及獨居,退處一閣,家人罕見其面。至老,手足耳目強明,如少壯時。不治財產,所收私田有餘穀,則以振鄉里貧人。前死數日,自作遺表,以祿賜所餘分親族。告其家人曰:「某日,吾死矣。」如期
 而死。無子。



 馬亮,字叔明,廬州合肥人。舉進士,為大理評事、知蕪湖縣,再遷殿中丞、通判常州。吏民有因緣亡失官錢,籍其貲猶不足以償,妻子連逮者至數百人。亮縱去,緩與之期,不逾月,盡輸所負。羅處約使江東,以亮治行聞,擢知濮州。



 會諸路轉運司置糾察刑獄官,以福建路命亮,覆訊冤獄,全活者數十人。遷太常博士、知福州。蘇易簡薦亮才任繁劇,召還,同提點三司都勾院、磨勘憑由司。久
 之,出知饒州。州豪白氏多執吏短長,嘗殺人,以赦免,愈驁橫,為閭里患,亮發其奸,誅之,部中畏懾。州有鑄錢監,匠多而銅錫不給,亮請分其工之半,別置監於池州,歲增鑄緡錢十萬。遷殿中侍御史。



 真宗即位,上書言:「陛下初政,軍賞宜速,而所在不時給,請遣使分督之。又赦書蠲除州縣逋負,而有司趣責愈急,宜如赦推恩以寬民。故事,以親王尹開封,地尊勢重,嫌隙易生,願鑒其繇,以示保全親愛之道。契丹仍歲南侵,河朔蕭然,請修好以
 息邊民。」帝善其言,以亮為可用。



 王均反,以為西川轉運副使。賊平,主將邀功,誅殺不已,亮全活千餘人。城中米斗千錢,亮出廩米裁其價,人賴以濟。召問蜀事,會械送賊詿誤者八十九人至闕下,執政欲盡誅之。亮曰:「愚民脅從,此特百之一二,餘竄伏山林者眾。今不貸之,反側之人,聞風疑懼,一唱再起,是滅一均、生一均也。」帝悟,悉宥之。加直史館,復遣還部。



 時諸州鹽井,歲久泉涸,而官督所負課,系捕者州數百人。亮盡釋系者,而奏廢其井,
 又除屬部舊逋官物二百餘萬。還知潭州,屬縣有亡命卒剽攻,為鄉閭患,人共謀殺之。事覺,法當死者四人,亮咸貸之,曰:「為民去害,而反坐以死罪,非法意也。」徙升州。行次江州,屬歲旱民饑,湖湘漕米數十舟適至,亮移文守將,發以振貧民。因奏:「瀕江諸郡皆大歉,而吏不之救,願罷官糴,令民轉粟以相賙。」



 以右諫議大夫知廣州。時宜州陳進初平,而澄海兵從進反者家屬二百餘人,法當配隸,亮悉置不問。鹽戶逋課,質其妻子於富室,悉取
 以還其家。海舶久不至,使招來之,明年,至者倍其初,珍貨大集,朝廷遣中使賜宴以勞之。是歲東封,亮敦諭大食陀婆離、蒲含沙貢方物泰山下。



 歷知虔洪二州、江陵府,再遷尚書工部侍郎,復知升州,徙杭州,加集賢院學士。先是,江濤大溢,調兵築堤而工未就,詔問所以捍江之策。亮褒詔禱伍員祠下,明日,潮為之卻,出橫沙數里,堤遂成。人為御史中丞。建言:「士民父祖未葬而析居,請自今未葬者,毋得輒析。」明年,改兵部侍郎、知廬州,徙江
 陵,又徙江寧府。仁宗初,拜尚書右丞,復知廬州,召判尚書都省兼知審刑院,遷工部尚書、知亳州,又遷江寧府,以太子少保致仕,卒,贈尚書右僕射。



 亮有智略,敏於政事,然其所至無廉稱。呂夷簡少時,從其父蒙亨為縣福州,亮見而奇之,妻以女。妻劉恚曰:「嫁女當與縣令兒邪?」亮曰:「非爾所知也。」陳執中、梁適為京官,田況、宋庠及其弟祁為童子時,亮皆厚遇之,曰:「是後必大顯。」世以亮為知人。亮卒,時夷簡在相位,有司謚曰忠肅,人不以為是
 也。子



 仲甫,為天章閣待制。



 ……



 陳希亮,字公弼,其先京兆人。唐廣明中,違難遷眉謅青神之東山。希亮幼孤好學,年十六,將從師,其兄難之,使治錢息三十餘萬。希亮悉召取錢者,焚其券而去。業成,乃召兄子庸、諭使學,遂俱中天聖八年進士第,里人表其閭曰「三俊」。



 初為大理評事、知長沙縣。有僧海印國師,出入章獻皇后家,與諸貴人交通,恃勢據民地,人莫敢正視,希亮捕治置諸法,一縣大聳。郴州竹場有偽為券
 給輸戶送官者,事覺,輸戶當死,希亮察其非辜,出之,已而果得其造偽者。再遷殿中丞,徙知鄠縣。老吏曹腆侮法,以希亮年少,易之。希亮視事,首得其罪。腆叩頭出血,願自新,希亮戒而舍之,卒為善吏。巫覡歲斂民財祭鬼,謂之春齋,否則有火災;民訛言有緋衣三老人行火。希亮禁之,民不敢犯,火亦不作。毀淫祠數百區,勒巫為農者七十餘家。及罷去,父老送之出境,泣曰:「公去我,緋衣老人復出矣。」遷太常博士。有言郴獄活人死罪,賜五品
 服。



 初,蜀人官蜀,不得通判州事。希亮以母老,願折資為縣侍親,於是知臨津縣。母終,服除,為開封府司錄司事。福勝塔火,官欲更造,度用錢三萬,希亮言:「陜西用兵,願以此饋軍。」詔罷之。青州民趙禹上書,言趙元昊必反,宰相以禹狂言,徙建州,元昊果反。禹訟所部,不受,亡至京自理,宰相怒,下開封獄。希亮言禹可賞不可罪,爭不已。上釋禹,賞為徐州推官,且欲以希亮為御史。會外戚沈元吉以奸盜殺人,希亮一問得實,自驚僕死,沉氏訴之,
 詔御史劾希亮及諸掾吏。希亮曰:「殺此賊者獨我耳。」遂引罪坐廢。



 期年,盜起京西,殺守令,富弼薦希亮可用,起知房州。州素無兵備,民凜凜欲亡去,希亮以牢城卒雜山河戶,得數百人。日夜部勒,聲振山南,民恃以安。殿侍雷甲以兵百餘人逐盜竹山,甲不能戢,所至為暴。或疑為盜,告希亮盜入境,且及門。希亮即勒兵阻水拒之,命持滿無得發,士皆植立如偶人。甲射之,不動,乃下馬拜請死,曰:「初不知公官軍也。」吏士皆欲斬甲以徇,希亮獨
 治為暴者十餘人,使甲以捕盜自贖。



 時劇賊黨軍子方張,轉運使使供奉官崔德贇捕之。德贇既失黨軍子,遂圍竹山民賊所嘗舍者曰向氏,殺父子三人,梟首南陽市。曰:「此黨軍子也。」希亮察其冤,下德贇獄,未服。黨軍子獲於商州,詔賜向氏帛,復其家,流德贇通州。或言華陰人張元走夏州,為元昊謀臣。詔徙其族百餘口於房,幾察出入,饑寒且死。希亮曰:「元事虛實不可知,使誠有之,為國者終不顧家,徒堅其為賊耳。此又皆其疏屬,無
 罪。」乃密以聞,詔釋之。老幼哭希亮庭下曰:「今當還故鄉,然奈何去父母乎?」遂畫希亮像祠焉。



 代還,執政欲以為大理少卿,希亮曰:「法吏守文,非所願,願得一郡以自效。」乃以為宿州。州跨汴為橋,水與橋爭,常壞舟。希亮始作飛橋,無柱,以便往來。詔賜縑以褒之,仍下其法,自畿邑至於泗州,皆為飛橋。



 皇祐元年,移滑州。奏事殿上,仁宗勞之曰:「知卿疾惡,無懲沉氏子事。」未行,詔提舉河北便糴。都轉運使魏瓘劾希亮擅增損物價。已而瓘除龍圖閣
 學士、知開封府,希亮乞廷辨。既對,仁宗直希亮,奪瓘職知越州,且欲用希亮。希亮言:「臣與轉運使不和,不得為無罪。」力請還滑。會河溢魚池埽,且決,希亮悉召河上使者,發禁兵捍之。廬於所當決,吏民涕泣更諫,希亮堅臥不動,水亦去,人比之王尊。



 是歲,盜起宛句,晝劫張郭鎮,執濮州通判井淵。仁宗以為憂,問執政可用者。未及對,仁宗曰:「朕得之矣。」乃以希亮為曹州。不逾月,悉擒其黨。



 淮南饑,安撫、轉運使皆言壽春守王正民不任職,正民
 坐免,詔希亮乘傳代之。轉運使調裏胥米而蠲其役,凡十三萬石,謂之拆役米。米翔貴,民益饑。希亮至,除之,且表其事,旁郡皆得除。又言正民無罪,職事辦治。詔復以正民為鄂州。



 久之,徙知廬州。虎翼軍士屯壽春者,以謀反誅,遷其餘不反者數百人於廬,皆自疑不安。一日,有竊入府舍將為不利者。希亮笑曰:「此必醉耳。」貸而流之,盡以其餘給左右使令,且以守倉庫。人為之懼,希亮益加親信,皆感德,指心誓為希亮死。改提點刑獄江東,遷
 度支郎中,徙河北。



 嘉祐二年,入為開封府判官,改判三司戶部勾院。朝廷以三司事冗,簿書留滯,乃命希亮又兼開拆司。榮州鬻鹽凡十八井,歲久澹竭,有司責課如初,民破產籍沒者三百餘家。希亮為言,還其所籍,歲蠲三十餘萬斤。三司簿書滯留者,自天禧以來,末帳六百有四,明道以來,生事二百一十二萬,希亮日夜課吏,凡九月,去其三之二。度支吏不時勾,希亮杖之。副使以希亮擅決罰,由是事復滯。



 會接伴契丹使還,自請補外,乃
 以為京西轉運使,賜三品服。石塘河役兵叛,其首周元自稱周大王,震動汝、洛間。希亮聞之,即日輕騎出按,吏請以兵從,希亮不許。其賊二十四人道遇希亮,以希亮輕出,意色閑和,不能測,遂相與列訴道周。希亮徐問其所苦,命一老兵押之,曰:「以是付葉縣,聽吾命。」既至,令曰:「汝以自首,皆無罪,然必有首謀者。」眾不敢隱,乃斬元以徇,流軍校一人,餘悉遣赴役如初。



 遷京東轉運使。濰州參軍王康赴官,道博平,大猾有號「截道虎」者,毆康及其
 女幾死,吏不敢問。希亮移捕甚急,卒流海島;又劾吏故縱,坐免者數人。除州守暴苛,以細過籍民產數十家,獲小盜,使必自誣抵死。希亮言其狀,卒以廢去。



 數上章請老,不允,移知鳳翔。倉粟支十二年,主者以腐敗為憂,歲饑,希亮發十二萬石貸民。有司懼為擅發,希亮身任之。是秋大熟,以新易舊,官民皆便。於闐使者入朝,過秦州,經略使以客禮享之。使者驕甚,留月餘,壞傳舍什器,縱其徒入市掠飲食,民戶皆晝閉。希亮聞之曰:「吾嘗主契
 丹使,得其情。使者初不敢暴橫,皆譯者教之,吾痛繩以法,譯者懼,其使不敢動矣。況此小國乎?」乃使教練使持符告譯者曰:「入吾境,有秋毫不如法,吾且斬若。」取軍令狀以還。使者至,羅拜庭下,希亮命坐兩廊飲食之,護出其境,無一人嘩者。



 英宗即位,遷太常少卿。獄有盜,法當死,僚官持不可。久之,盜殺守吏遁去。希亮以前議讞於朝,而希亮之議是。僚官懼,欲以事中希亮,希亮自顧無有其事。始,州郡以酒相餉,例皆私有之,而法不可。希亮
 以遺游士之貧者,既而曰:「此亦私也。」以家財償之。遂借此上書自劾,求去不已,坐是分司西京。未幾致仕,卒,年六十四。希亮嘗夢異人按圖而告之年,至是果然。贈工部侍郎。



 希亮為人清勁寡欲,不假人以色,自王公貴人,皆嚴憚之。見義勇發,不計禍福。所至,奸民猾吏,易心改行,不改者必誅。然出於仁恕,故嚴而不殘。少與蜀人宋輔游,輔卒於京,母老,子端平幼,希亮養其母終身,以女妻端平,使同諸子學,卒登進士第。



 四子。忱,度支郎中。恪,
 滑州推官。恂,大理寺丞。慥字季常,少時使酒好劍,用財如糞土,慕朱家、郭解為人,閭里之俠皆宗之。在岐下,嘗從兩騎挾二矢與蘇軾游西山。鵲起於前,使騎逐而射之,不獲,乃怒馬獨出,一發得之。因與軾馬上論用兵及古今成敗,自謂一世豪士。稍壯,折節讀書,欲以此馳騁當世,然終不遇。洛陽園宅壯麗與公侯等,河北有田歲得帛千匹,晚年皆棄不取。遁於光、黃間,曰岐亭。庵居蔬食,徒步往來山中,妻子奴婢皆有自得之意,不與世相
 聞,人莫識也。見其所著帽方屋而高,曰:「此豈古方山冠之遣像乎?」因謂之「方山子。」及蘇軾謫黃,過岐亭,識之,人始知為慥云。



 論曰:乘雅恬退,穎不阿貴戚,有儒者之風。摯淳靜而不矯,池質易而長厚,肅議法平恕,及、堂、夔清修自守,蓋侍從之選也。希亮為政嚴而不殘,其良吏與。馬亮饒才智而寡廉稱,士論以此惜之



\end{pinyinscope}