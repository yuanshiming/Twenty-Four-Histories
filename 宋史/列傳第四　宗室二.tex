\article{列傳第四 宗室二}

\begin{pinyinscope}

 漢王元佐昭成太子元僖商王元份越王元傑鎮王元偓楚王元戴周王元儼悼獻太子濮王允讓



 太宗九子:長楚王元佐,次昭成太子元僖,次真宗,次商恭靖王元份,次越文惠王元傑,次鎮恭懿王元偓,次楚恭惠王元戴,次周恭肅王元儼,次崇王元億。



 漢恭憲王元佐字惟吉,初名德崇,母元德皇后。少聰警,貌類太宗,帝鐘愛之。年十三,從獵近郊,兔走乘輿前,太宗使元佐射,一發而中,契丹使在側,驚異之。從征太原、幽薊。太平興國中,出居內東門別第,拜檢校太傅、同中書門下平章事,封衛王,赴上中書。後徙居東宮,改賜
 今名,加檢校太尉,進封楚王。



 初,秦王廷美遷涪陵,元佐獨申救之。廷美死,元佐遂發狂,至以小過操挺刃傷侍人。雍熙二年,疾少間,帝喜,為赦天下。重陽日內宴,元佐疾新愈不與,諸王宴歸,暮過元佐第。曰:「若等侍上宴,我獨不與,是棄我也。」遂發忿,被酒,夜縱火焚宮。詔遣御史捕元佐,詣中書劾問,廢為庶人,均州安置。宰相宋琪率百官三上表,請留元佐京師。行至黃山,召還,廢居南宮,使者守護。諮議趙齊王遹、翊善戴元頓首請罪,帝
 赦之曰:「是子朕教之猶不悛,汝等安能輔導耶?」



 真宗即位,起為左金吾衛上將軍,復封楚王,聽養疾不朝,再加檢校太師、右衛上將軍。元佐生日,真宗賜以寶帶。平居不接人事,而事或預知。帝嘗遣術士管歸真為醮禳,左右未及白,元佐遽曰:「管歸真至矣。」帝聞之曰:「豈非為物所憑乎?」封泰山,真拜太傅;祀汾陰,遷太尉兼中書令。又加太師、尚書令兼中書令,遂拜天策上將軍、興元牧,賜劍履上殿,詔書不名。時禁中火,元佐表停奉稟助完宮
 闕,不許。加兼雍州牧。仁宗為皇太子,兼興元牧。仁宗即位,兼江陵牧。薨,年六十二,贈河中、鳳翔牧,追封齊王,謚恭憲。宗室子弟特給假七日,以鹵簿鼓吹導至永安,陪葬永熙陵。明導二年,改封潞王。又改魏王。子三人:允升、允言、允成。



 仁宗封王後,以允言子宗說恭憲王長孫,嗣封祁國公。皇祐中,坐帷薄不修除名,又坐坑殺女僕,鎖閉宮室外宅。其子仲旻,官右武衛大將軍、道州刺史,後因朝,叩頭殿下泣訴云:「父老且病,願納身官以贖。」神宗
 亦愍之,而未俞其請。出就馬,氣塞不能言,及家而卒。贈同州觀察使、馮翊侯。宗說幽死。



 熙寧三年,以允升子宗惠襲封魏國公。中書言宗惠不應封,以恭憲庶長孫允言子宗立嗣。



 宗立從張揆學《春秋》。太清樓侍宴,預坐悉賦裸玉詩,宗立詩先成,仁宗稱善。屢賜飛白書,旌其文雅。至是襲封,終武寧軍節度觀察留後,贈昭信軍節度使、同中書門下平章事、南康郡王。子仲來嗣,終金州刺史。子不儻嗣。徽宗立,改封魏王為漢王。不儻卒,子彥清
 氣襲父爵,奉漢王祀,詔從之。



 允升字吉先,初免乳,養明德太后宮,太后親撫視之。元佐有疾,允升始出第。真宗賜名元中,授右監門衛將軍,更賜今名。累遷澶州觀察使,封延安郡公,進武寧軍節度觀察留後,歷安德、建雄、安國軍節度使。景祐二年卒,贈太尉、平陽郡王,謚懿恭。子十三人,宗禮、宗旦、宗悌、宗惠知名。



 宗禮嘗侍宴太清樓,仁宗賦詩,命屬和,侍射苑中,復獻詩。終虔州觀察使、成國公,贈安遠軍節度使、同中書門下平章事、韓國公。
 子仲翹、仲髦。



 宗旦字子文,七歲如成人,選為仁宗伴讀。帝即位,獲超選,為群從所詆,上書言狀,帝曰:「宗旦陪朕幼學,勤勞居多,此出朕意,豈應訴以常格?」所生母死,請別擇葬域,歲時奠祀,後遂著為法。治平中,同知大示正事。神宗即位,拜崇信軍節度使、同中書門下平章事,為大宗正,賜方團金帶,非朝會得乘肩輿。元豐三年,封華陰郡王,加開府儀同三司。長屬籍十六年,宗子有過,優游誨導,一善必以聞。異時赴朝請者,率以私丁給侍,宗
 旦建請,始得從官給。薨,贈太尉、滕王,謚恭孝,聽旗節印綬從葬。



 宗悌字符發,輕財好施。故相王氏子持父所服帶求質錢,宗悌惻然曰:「宰相子亦至是乎!」歸帶而與之錢。所親用詐取藏鏹,得其狀,曰:「吾不以小故傷骨肉恩。」竟不問。所生母早世,宗悌不識也,聞父婢語平生,輒掩泣。繼得其肖貌,繪而奉之如生。終明州觀察使,贈保寧軍節度使、同中書門下平章事、東陽郡王,謚曰孝憲。



 宗惠,封魏國公,尋以旁支黜。終武昌軍節度觀察留後、江
 夏郡王,贈郯王。



 允言,累官左屯衛將軍。嘗托疾不朝,降太子左衛率府率,歲中復官。又坐笞侍婢,而兄允升勸止,悖慢無禮,貶副率,絕朝謁,出之別第。以祀汾陰恩,復率府率,還宮,久之,復朝謁,歷左監門衛大將軍、黃州刺史。天聖七年卒,贈明州觀察使、奉化侯。明道二年,贈安遠軍節度使,追封密國公。子宗說、宗立事並見上。宗育終右屯衛將軍,贈穎州防禦使、汝陰侯。



 允成,終右神武將軍、濮州防禦使,贈安化軍節度使、郇國公。明道二年,
 加贈鎮江軍節度使兼侍中。子宗顏、宗訥、宗鼎、宗嚴、宗魯、宗儒、宗奭,皆為環衛、刺史。



 昭成太子元僖,初名德明。太平興國七年出閣,授檢校太保、同平章事,封廣平郡王,與兄衛王德崇同日受封。八年,進封陳王,改名元祐。詔自今宰相班宜在親王上,宰相宋琪、李昉清遵舊制,不允。宋琪等懇請久之,上早「宰相之任,實總百揆,與群司禮絕;藩邸之設,止奉朝請而已。元佐等尚幼,欲其知謙損之道,卿等無固讓也。」



 雍
 熙二年,元佐被疾,以元僖為開封尹兼侍中,改今名,進封許王,加中書令。上為娶隰州團練使李謙溥女為夫人,因渭宰相曰:「朕嘗語諸子,今姻偶皆將相大臣之家,六禮具備,得不自重乎?」淳化元年,宰相呂蒙正復上言,乞班諸王下,詔不允。三年十一月己亥,元僖早入朝,方坐殿廬中,覺體中不佳,徑歸府。車駕遽臨視,疾已亟,上呼之猶能應,少頃遂薨。上哭之慟,廢朝五日,贈皇太子,謚恭孝。



 元僖姿貌雄毅,沉靜寡言,尹京五年,政事無失。
 及薨,上追念不已,悲泣達旦不寐,作《思亡子詩》示近臣。



 未幾,人有言元僖為嬖張氏妾所惑,張頗專恣,捶婢僕有至死者,而元僖不知。張又於都城西佛寺招魂葬其父母,僭差逾制。上怒,遣昭宣使王繼恩驗問,張縊死。左右親吏悉決杖停免,毀張氏父母塚墓,親屬皆配流。開封府判官、右諫議大夫呂端,推官、職方員外郎陳載,並坐裨贊有失,端黜為衛尉少卿,載為殿中侍御史。許王府諮議、工部郎中趙令圖,侍講、庫部員外郎閻象,並坐
 輔道無狀,削兩任免。詔停冊禮,以一品鹵簿葬。真宗即位,始詔中外稱太子之號焉。乾興初,改謚。無子,仁宗時,詔以允成子宗保出後昭成太子為孫。



 宗保生二歲,母抱以入見章獻後,後留與處。宗保七歲,授左侍禁,帝親為巾其首。久之,歸本宮,詔朔望出入禁省。累官代州防禦使,襲封燕國公。性仁恕,主藏吏盜米至千斛,貰不問。嘗書「忍」字於座右以為戒。熙寧七年卒。神宗臨奠,其子仲鞠泣曰:「先臣幼養宮中,終身不自言。」帝感悼,遂優贈
 靜難軍節度使、新平郡王,謚恭靜。仲鞠亦好學能詩,事親居喪以孝聞。



 宗保卒,子仲恕嗣,官至忠州團練使,謚純僖。子士盉嗣。



 商恭靖王元份,初名德嚴。太平興國八年出閣,改名元俊,拜同平章事,封冀王。雍熙三年,改今名,加兼侍中、威武軍節度使,進封越王。淳化中,兼領建寧軍,改鎮寧海、鎮東。真宗即位,加中書令,徙鎮永興、鳳翔,改王雍。永熙復土,為山陵使,拜太傅。真宗北征,為東京留守。薨年三
 十七,贈太師、尚書令、鄆王。改陳王,又改潤王。治平中,封魯王。



 元份寬厚,言動中禮,標望偉如,娶崇儀使李漢斌之女。李悍妒慘酷,宮中女婢小不如意,必加鞭杖,或致死。上每思恩賜,詔令均給,李盡取之。及元份臥病,上親臨問,見左右無侍者,因輟宮人為主湯劑。初,太宗崩,戚里皆赴禁中,朝晡臨,李多稱疾不至。元份生日,李以衣服器用為壽,皆飾以龍鳳。居元份喪,無戚容,而有謗上之語。上盡知其所為,以元份故優容之。及是,復不欲顯
 究其罪狀,止削國封,置之別所。元份子三人:長允寧,次允懷,改允中,早卒;次則濮王允讓也。



 允讓薨,以允寧子宗諤襲虢國公。至熙寧三年,以宗肅嗣封魯國公。宗肅,亦允寧子也。子仲先嗣。徽宗即位,改封魯王為商王,詔曰:「宗室諸王追封大國,其世襲子孫尚仍舊國,甚未稱正名之意。如魯王改封商王,其子尚襲魯國之類。基令大宗正司改正。」制以寧遠軍節度使、魯國公仲先改封商國公。



 允寧字德之,性至孝,因父感疾,恍惚失常。既而
 嗜學,尤喜讀唐史,通知近朝典故,工虞世南楷法,真宗賜詩激賞之。又善射,嘗侍身後苑,屢破的,賜金帶器幣。初授右千牛衛將軍,四遷右武衛,歷唐州團練、穎州防物、同州觀察使,進彰信軍節度觀察留後、武定軍節度使。景祐元年卒,贈太尉、信安郡王,謚僖簡。子宗諤、宗敏、宗孟、宗肅。



 宗諤封虢國公,官累集慶軍節度使、同中書門下平章事,進封豫章郡王。乞比外使相給奉,仁宗以非兼侍中,令詰主吏,宗諤上章自陳,於是御史張商英
 劾其招權立威等罪,坐落平章事。英宗即位,還所奪。元豐五年薨,贈太尉、韓王。太常謚榮孝,上省集議駁之,改榮恭,僕射王珪復駁之,遂謚榮思。



 宗肅封魯國公。兄宗諤嘗亡寶器,意宗肅家人子竊之,宗肅曰:「吾廉,不足取信兄弟如此乎?」立償其直。宗諤愧不取,乃施諸僧。久之器得,宗肅不復言。元豐五年,終安化軍留後,以嘗從英宗入慶寧,優贈鎮海軍節度使、開府儀同三司、北海郡王。



 宗敏終右千牛衛大將軍、文州刺史,贈趙州觀察使、
 會稽侯。頗涉書傳。緣郊恩建請封所生母範氏,宗室子得封所生母,自宗敏始。



 越文惠王元傑字明哲,初名德和。太平興國八年出閣,改名。授檢校太保、同平章事,封益王。端拱初,加兼侍中、成都尹、劍南東西川節度。淳化中,徙封吳王,領揚潤大都督府長史、淮南鎮江軍節度使。至道二年,改揚州大都督、淮南忠正軍節度。真宗即位,授檢校太尉兼中書令、徐州大都督、武寧泰寧等軍節度使,改封袞王。咸平
 中,再郊祀,皆為終獻,加守太保。六年七月暴薨,年三十二。



 元傑穎悟好學,善屬詞,工草、隸、飛白,建樓貯書二萬卷,及為亭榭游息之所。嘗作假山,既成,置酒召僚屬觀之。翊善姚坦獨俯首不視,元傑強之,坦曰:「坦見血山,安得假山。」言州縣鞭撻微民,以取租稅,假山實租稅所為耳。語見《姚坦傳》中。



 及薨,真宗聞之震悼,不俟旦,步及中禁門,乃乘輦臨視,哀動左右,廢朝五日。贈太尉、尚書令,追封安王,謚文惠,後改邢王,後改陳王。無子。仁宗以恭
 憲王之孫、允言子宗望為之後。



 宗望字子國,終右武衛大將軍、舒州防禦使,贈安化軍節度使觀察留後、高密郡公。仁宗嘗御延和殿試宗子書,以宗望為第一;又常獻所為文,賜國子監書,及以塗金紋羅御書「好學樂善」四字賜之。即所居建御書閣,帝為題其榜。



 子仲邠嗣。熙寧三年,與商恭靖王孫宗肅等同日封陳國公。官至陳州觀察使。卒,謚良僖。



 子士關嗣。父卒,徒行護喪數百里,路人嗟惻。卒,贈陳州觀察使。徽宗即位,改封陳王為越
 王。



 鎮恭懿王元偓字希道。端拱元年出閣,授檢校太保、左衛上將軍,封徐國公。至道二年,拜洪州都督、鎮南軍節度使。真宗即位,加同平章事,封彭城郡王。俄加檢校太傅,改鎮靜難、彰化,進封寧王。郊祀、東封,悉為亞獻,禮成,授檢校太尉兼侍中、護國鎮國等軍節度。



 三年,文武官詣闕請祠后土,元偓以領節帥亦奏章以請,詔許之。將行,命為河、華管內橋道頓遞使。明年,車駕入境,元偓奏
 方物、酒餼、金帛、茗藥為貢,儀物甚盛。至河中,與判府陳堯叟分導乘輿度蒲津橋。上登鄭丘亭,目元偓曰:「橋道頓置嚴謹,爾之力也。」元偓頓首謝。及還,加中書令,領成德、安國等軍節度,改封相王。五年,加守太傅。



 真宗自即位以來,屢以學術勖宗子。元偓首冠藩戚,益自修勵,上每制篇什,必令屬和。一日,謂宰相曰:「朕每戒宗子作詩習射,如聞頗精習,將臨觀焉。」因幸元官偓邸第,宴從官,宮僚畢會,賦七言詩。元偓奉觴上壽,賜襲衣、金帶、器幣、緡
 錢,又與宗室射於西南亭,日晡,從官退,上獨以中官從,幸元戴、元儼宮,復宴元偓宮,如家人禮,夜二鼓而罷。六年,進位太尉。



 八年七月,以榮王宮火,徙元偓宮於景龍門外,車駕臨幸。是冬,加兼尚書令。天禧元年二月,換成德、鎮寧二鎮,進封徐王。二年春,宮邸遺燼,燔舍數區,元偓驚悸,暴中風眩薨,年四十二。帝臨哭,廢朝五日,贈太師、尚書令、鄧王,贈謚恭懿。



 元偓姿表偉異,厚重寡言,曉音律。後改封密王,又改王蘇。治平中,追封韓王。



 子允弼,
 八歲召入禁中,令皇子致拜,允弼不敢當。御樓觀酺,得與王子並坐。皇子即位,是為仁宗。允弼累遷武寧軍節度使兼侍中,判大宗正事,封北海郡王。」英宗時,拜中書令,徙王東平。神宗即位,拜太保、鳳翔雄武軍節度使,朝朔望。熙寧二年,丁母憂,悲痛不勝喪,固辭起復。母葬有日而允弼病篤,顧諸子以不得終大事為恨。薨,帝臨哭之慟,輟朝三日,贈太師、尚書令兼中書令,追封相王,謚孝定。



 允弼性端重,時然後言。諸宮增學官員,允弼已貴,
 猶日至講席,延伴讀官讀《孟子》一節。領宗正三十年,與濮安懿王共事,相友愛,為宗屬推敬。



 子宗繢,襲祖恭懿王封為韓國公。卒,贈南康郡王,謚良孝。宗繢弟宗景,以相州觀察使同知大宗正事。神宗以其父允弼司宗久,故復選用之。宗景事母孝,居喪如不能勝。居第火冒,急赴家廟,不恤其它,火亦不為害。元祐中,累遷彰德軍節度、開府儀同三司、檢校司空,封濟陰郡王。宗景喪其夫人,將以妾繼室,先出之於外,而托為良家女且納焉。坐
 奪開府,既而還之。紹聖四年薨,年六十六,贈太師、循王,謚曰思。



 宗繢既卒,子仲歷嗣,自平川節度使徙劍南西川。徽宗改封韓王為鎮王。



 楚恭惠王元戴字令聞,七歲授檢校太保、右衛上將軍、涇國公。久之,領鄂州都督、武昌軍節度使。真宗即位,加同平章事、安定郡王,進檢校太傅。景德二年,郊祀,遷宣德、保寧兩鎮,進封舒王。大中祥符初,封泰山,加檢校太尉兼侍中,移平江、鎮江軍。從祀汾陰,加兼中書令,改鎮
 南、寧國軍節度使。五年,拜太保。自景德後,每有大事,皆為終獻。



 元戴體素羸多病,上幸真源,時已被疾,懇求扈從。至鹿邑疾甚,肩輿先歸。車駕還,臨問數四。七年,薨,年三十四。廢朝五日,贈太尉、尚書令,追封曹王,謚恭惠。後改封華王、蔡王。有集三卷、筆札一卷,上為制序,藏之秘閣。子允則,官至右千牛衛大將軍卒。



 先是,諸王子授官,即為諸衛將軍,餘以父官及族屬親疏差等。天禧元年,令宗正卿趙安仁議為定制。安仁請以宣祖、太祖、太宗
 孫初蔭授將軍,曾孫授右侍禁,玄孫授右班殿直,內父爵高者聽從高蔭,其事緣特旨者不以為例。詔中書、門下、樞密院參定行之。



 允則無子,以平陽懿恭王之子宗達為後。熙寧三年,襲封蔡國公。鄰家失火,盜因為奸,竊宗達所服帶,既而得之,且知其主名,貸不問。浚井得鏹,復投之。官累武信軍留後。薨,贈安化軍節度使、開府儀同三司、高密郡王。子仲約嗣。徽宗即位,改封蔡王為楚王。



 周恭肅王元儼,少奇穎,太宗特愛之。每朝會宴集,多侍左右。帝不欲元儼早出宮,期以年二十始就封,故宮中稱為「二十八太保」,蓋元儼于兄弟中行第八也。



 真宗即位,授檢校太保、左衛上將軍,封曹國公。明年,為平海軍節度使,拜同中書門下平章事,加檢校太傅,封廣陵郡王。封泰山,改昭武、安德軍節度使,進封榮王;祀汾陰,加兼侍中,改鎮安靜、武信,加檢校太尉;祠太清宮,加兼中書令。坐侍婢縱火,延燔禁中,奪武信節,降封端王,出居
 故駙馬都尉石保吉第。每見帝,痛自引過,帝憫憐之。尋加鎮海、安化軍節度使,封彭王,進太保。仁宗為皇子,加太傅。歷橫海永清保平定國節度、陜州大都督,改通王、涇王。仁宗即位,拜太尉、尚書令兼中書令,徙節鎮安、忠武,封定王,賜贊拜不名,又賜詔書不名。天聖七年,封鎮王,又賜劍履上殿。明道初,拜太師,換河陽三城、武成節度,封孟王,改永興鳳翔、京兆尹,封荊王,遷雍州、鳳翔牧。景祐二年大封拜宗室,授荊南、淮南節度大使,行荊州、
 揚州牧,仍賜入朝不趨。



 元儼廣顙豐頤,嚴毅不可犯,天下崇憚之,名聞外夷。事母王德妃孝,妃每有疾,躬侍藥,晨夕盥潔焚香以禱,至憂念不食。母喪,哀戚過人。平生寡嗜欲,惟喜聚書,好為文詞,頗善二王書,工飛白。



 仁宗沖年即位,章獻皇后臨朝,自以屬尊望重,恐為太后所忌,深自沉晦。因闔門卻絕人事,故謬語陽狂,不復預朝謁。及太后崩,仁宗親政,益加尊寵,凡有請報可,必手書謝牘。方陜西用兵,上所給公用錢歲五十萬以助邊費,
 帝不欲拒之,聽入其半。嘗問翊善王渙曰:「元昊平未?」對曰:「未也。」曰:「如此,安用宰相為。」聞者畏其言。



 慶歷三年冬,大雨雪,木冰,陳、楚之地尤甚。占者曰:「憂在大臣。」既而元儼病甚。上憂形於色,親至臥內,手調藥,屏人與語久之,所對多忠言。賜白金五千兩,固辭不受,曰:「臣羸憊且死,將重費家國矣。」帝為嗟泣。明年正月薨,贈天策上將軍、徐袞二州牧、燕王,謚恭肅。比葬,三臨其喪。詔以元儼墨跡及所為詩分賜宰臣,餘藏秘閣。



 子十三人:允熙、允良、
 允迪、允初,餘皆早卒。熙寧中,以允良子宗絳嗣封吳國公。徽宗改封吳王為周王。



 允熙終右監門衛將軍、滁州刺史,贈博州防禦使、博平侯。



 允良歷五節度,領寧海、平江兩軍,封華原郡王,改襄陽,由同中書門下平章事、兼侍中,至太保、中書令。好酣寢,以日為夜,由是一宮之人皆晝睡夕興。薨,贈定王,有司以其反易晦明,謚曰榮易。



 允迪累官耀州觀察使。居父喪不哀,又嘗宮中為優戲,為妻昭國夫人錢氏所告。制降右監門衛大將軍,絕朝
 謁,錢氏亦度為洞真道士。



 允初,初名允宗,勤於朝會,雖風雨不廢。未嘗問財物厚薄,惟誦佛書,人以為不慧。累遷寧國軍節度使、同中書門下平章事。治平元年卒,贈中書令、博平郡王。無子。英宗臨奠,以允初後事屬其兄允良,乃以允成孫仲連為之後。



 崇王元億,早亡,追賜名,封代國公。治平中,封安定郡王。徽宗即位,加封崇王。



 真宗六子:長溫王禔,次悼獻太子祐,次昌王祗,次信王
 祉,次欽王祈,次仁宗。禔、祗、祈皆蚤亡,徽宗賜名追封。



 悼獻太子祐,母曰章穆皇后。咸平初,封信國公。生九年而薨,追封周王,賜謚悼獻。仁宗即位,贈太尉、中書令。明道二年,追冊皇太子。



 仁宗三子:長楊王昉,次雍王昕,次荊王曦,皆早亡。徽宗時改封。



 濮安懿王允讓字益之,商王元份子也。天資渾厚,外莊內寬,喜慍不見於色。始為右千牛衛將軍。周王祐薨,真
 宗以綠車旄節迎養於禁中。仁宗生,用簫韶部樂送還邸。官衛州刺史。仁宗即位,授汝州防禦使,累拜寧江軍節度使。上建睦親宅,命知大宗正寺。宗子有好學,勉進之以善,若不率教,則勸戒之,至不變,始正其罪,故人莫不畏服焉。慶歷四年,封汝南郡王,拜同平章事,改判大宗正司。嘉祐四年薨,年六十五,贈太尉、中書令,追封濮王,謚安懿。仁宗在位久無子,乃以王第十三子宗實為皇子。仁宗崩,皇子即位,是為英宗。



 治平元年,宰相韓琦
 等奏:請下有司議濮安懿王及譙國夫人王氏、襄國夫人韓氏、仙游縣君任氏合行典禮。詔須大祥後議之。



 二年,乃詔禮官與待制以上議。翰林學士王珪等奏曰:



 謹按《儀禮喪服》:「為人後者」《傳》曰:「何以三年也?受重者必以尊服服之。」「為所後者之祖父母妻,妻之父母昆弟,昆弟之子若子。」謂皆如親子也。又「為人後者為其父母」《傳》曰::「何以期?不二斬,持重於大宗,降其小宗也。」為人後者為其昆弟」《傳》曰:「何以大功?為人後者降其昆弟也。」



 先王制
 禮,尊無二上,若恭愛之心分於彼,則不得專於此故也。是以秦、漢以來,帝王有自旁支入承大統者,或推尊其父母以為帝後,皆見非當時,取議後世,臣等不敢引以為聖朝法。



 況前代入繼者,多宮車晏駕之後,援立之策或出臣下,非如仁宗皇帝年齡未衰,深惟宗廟之重,祗承天地之意,於宗室眾多之中,簡推聖明,授以大業。陛下親為先帝之子,然後繼體承祧,光有天下。



 濮安懿王雖於陛下有天性之親,顧復之思,然陛下所以負扆端
 冕,富有四海,子子孫孫萬世相承,皆先帝德也。臣等竊以為濮王宜準先朝封贈期親尊屬故事,尊以高官大國,譙國、襄國、仙游並封太夫人,考之古今為宜稱。



 於是中書奏:王珪等所議,未見詳定濮王當稱何親,名與不名?珪等議:「濮安於仁宗為兄,於皇帝宜稱皇伯而不名,如楚王、涇王故事。」



 中書又奏:「《禮》與《令》及《五服年月敕》:出繼之子於所繼、所生皆稱父母。又漢宣帝、光武皆稱父為皇考。今珪等議稱濮王為皇伯,於典禮未有明據,請
 下尚書省,集三省、御史臺議奏。」



 方議而皇太后手詔詰責執政,於是詔曰「如聞集議不一,權宜罷議,令有司博求典故以聞。」禮官範鎮等又奏:「漢之稱皇考、稱帝、稱皇,立寢廟,序昭穆,皆非陛下聖明之所法,宜如前議為便。」自是御史呂誨等彈奏歐陽修首建邪議,韓琦、曾公亮、趙概附會不正之罪,固請如王珪等議。



 既而內出皇太后手詔曰:「吾聞群臣議請皇帝封崇濮安懿王,至今未見施行。吾載閱前史,乃知自有故事。湫安懿王、譙國夫
 人王氏、襄國夫人韓氏仙游縣君任氏,可令皇帝稱親,濮安懿王稱皇,王氏、韓氏、任氏並稱後。」



 事方施行,而英宗即日手詔曰:「稱親之禮,謹遵慈訓;追崇之典,豈易克當。且欲以塋為園,即園立廟,俾王子孫主奉祠事。」



 翌日,誨等以所論列彈奏不見聽用,繳納御史敕告,家居待罪。誨等所列,大抵以為前詔稱「權罷集議」,後詔又稱「且欲以塋為園」,即追崇之意未已。英宗命閣門以告還之。誨等力辭臺職。誨等既出,而濮議亦寢。至神宗元豐二
 年,詔以濮安懿王三夫人可並稱王夫人云。



 王二十八子。長宗懿,英宗時為宿州團練使,封和國公。神宗以宗懿濮安懿王元子,追封舒王。子仲鸞,常州防禦使。父薨,諸子皆進官,獨不忍受。喜翰墨,樂施與,九族稱賢。卒,贈武康軍節度使、洋國公,謚曰良。仲鸞弟仲汾,幼喜書史,一讀成誦。居父喪,鄰於毀瘠。卒官萊州防禦使,贈昭化軍節度使、榮國公。



 次宗樸,為隴州防禦使,封岐國公。宗樸與英宗友愛。初,詔英宗入居慶寧宮,固辭,真樸率近
 屬敦勸,乃入。治平中,建濮王園廟,宗樸遂拜彰德軍節度使,封濮國公,奉王後。神宗即位,加同平章事兼侍中,進封濮陽郡王。薨,贈太師、中書令,追封定王,謚僖穆。子仲佺,父歿,不食者數日。母葬時,天大雪,步泥中扶翼,道路嘆惻。以潤州觀察使卒,贈開府儀同三司。



 宗樸既薨,宗誼襲封。官至昭化軍節度使、同中書門下平章事。薨,贈太師、中書令、慶陵郡王,謚莊孝。



 宗暉,元豐中,以淮康軍節度使襲濮國公。安懿王及三夫人改祔,命為志並
 題神主,加同中書門下平章事、開府儀同三司,進嗣濮王。哲宗立,改鎮南節度使、檢校司徒。紹聖元年薨,年六十七,贈太師,追封懷王,謚榮穆。子仲璲。先是,濮國嗣王四孟詣洛享園廟,以河南府縣官充亞、終獻。宗暉之襲封也,神宗始命以其子為之,仲璲遂以終獻侍祠,凡十餘年。父喪,哀痛不能勝,才服除而卒。官右監門衛大將軍、合州刺史。



 宗晟,紹聖元年六月,以武安軍節度使判大宗正事,加檢校司徒,嗣濮王。明年三月薨,年六十五,
 贈太師、昌王,謚端孝。宗晟好古學,藏書數萬卷,仁宗嘉之,益以國子監書。治平將郊而雨,或議改祫享,英宗訪諸宗晟,對曰:「陛下初郊見上帝,盛禮也,豈宜改卜。至誠感神,在陛下精意而已。」帝嘉納。及郊,雨霽。帝數被疾,密請早建儲貳,以系天下之望,世稱其忠。



 宗晟薨,哲宗紹聖二年四月,宗愈以鎮安節度使、開府儀同三司、檢校司徒嗣封。故事嗣王以四時詣祠所,宗愈方屬疾,或曰不可以暑行,曰:「吾身主祀而不往,非禮也。」強輿以行,疾
 遂亟。是年八月薨,年六十五,贈太師,追封襄王,謚恭憲。



 宗綽嗣,官至河陽三城節度使、檢校司徒。紹聖三年二月薨,年六十二,贈太師,追封榮王,謚孝靖。



 宗楚,累拜武勝軍節度使、開府儀同三司,封南陽郡王。紹聖三年三月,以檢校司徒改武昌節度使,嗣濮王。既嗣爵,當詣園薦獻,會疾,以弟宗漢代行,嘆曰:「不能親奉籩豆,饗我先王,而浮食厚祿,安乎!」請以爵授弟,不許。四年六月薨,贈太師、惠王,謚僖節。



 宗祐克己自約,肅然若寒士,好讀書,
 尤喜學《易》。嘉祐中,從父允初未立嗣,咸推其賢,詔以宗祐為後,泣曰:「臣不幸幼失怙恃,將終身悲慕,忍為人後乎!敢以死請。」仁宗憐而從之。累遷清海軍節度使、開府儀同三司,封乘城郡王。紹聖四年八月,加檢校司待,嗣濮王。時已病,當祠園廟,不肯移疾,自秋涉冬連往來。元符元年春,又亟往,遂薨於祠下。贈太師,追封欽王,謚穆恪。



 宗漢,英宗幼弟也。累拜保寧軍留後、鄴國公、東陽安康郡王。元符初,以彰德軍節度使、開府儀同三司、檢校
 司空嗣濮王。徽宗即位,徙寧江、保平、泰寧三鎮,判大宗正事,加檢校司徒、太保、太尉。帝幸濮邸,遷其子孫官。時安懿王諸子獨宗漢在,恩禮隆腆。大觀三年八月薨,贈太師。追封景王,謚孝簡。宗漢善畫,當作《八雁圖》,人稱其工。仲增嗣。



 仲增,濮王孫,於屬為長,故封。官至彰德軍節度使、開府儀同三司。政和五年九月薨,贈少師,追封簡王,謚穆孝。



 仲御,自幼不群,通經史,多識朝廷典故。居父宗晟喪,哲宗起知宗正,力辭,詔虛位以須終制。累遷鎮
 寧、保寧、昭信、武安節度使,封汝南、華原郡王。政和中,以檢校少傅、泰寧軍節度使、開府儀同三司嗣封。天寧節遼使在廷,宰相適謁告,仲御攝事,率百僚上壽,若素習者。帝每見必加優禮,稱為嗣王。宣和四年五月薨,年七十一,贈太傅,追封郇王,謚康孝。



 仲爰嗣。徽宗即位,拜建武節度使,為大宗正,加開府儀同三司,封江夏郡王,徙節泰寧定武,檢校少保、少傅。宣和五年六月薨,年七十,贈太保,追封恭王。



 仲理嗣。靖康初,為安國軍節度使,加
 檢校少保、開府儀同三司。



 嗣濮王者,英宗本生父後也。治平三年,立濮王園廟。元豐七年,封王子宗暉為嗣濮王,世世不絕封。高宗南遷,奉濮王神主於紹興府光孝寺。



 仲湜字巨源,楚榮王宗輔之子,安懿王孫也,初名仲泹。熙寧十年,授右內率府副率。累遷密州觀察使、知西外宗正事、保大軍承宣使。欽宗嗣位,授靖海節度使,更今名。召知大宗正事,未行,汴京失守。康王即帝位於南京,仲湜由漢上率眾徑謁時嗣濮王仲理北遷,乃詔仲
 湜襲封,加開府儀同三司,歷檢校少保、少傅。紹興元年,充明堂亞獻。七年,薨,帝為輟朝,賜其家銀帛,追封儀王,謚恭孝。仲湜事母以孝聞,喜親圖史。性酷嗜珊瑚,每把玩不去手,大者一株至以數百千售之。高宗嘗問墜地則何如,仲湜對曰:「碎矣。」帝曰:「以民膏血易無用之物,朕所不忍。」仲湜慚不能對。



 子士從、士街、士籛、士街、士歆。士從,靖康末,為洺州防禦使。建炎二年,同知西外宗正事,主管高郵軍宗子。士從招潰卒置屯,奏假江、淮制置使,
 許之。賊李在犯楚州,士從遣部將乘虛掩襲,狃於小勝,軍無紀律,敗績。士從移司衡、溫二州。臣僚以其弟士籛撓州縣,士從不能制,遂罷。紹興四年,遷涇、洪二州觀察使,權知濮王園令。士從乞擇利便地奉安神位,從之。六年,士街授象州防禦使,遷華州觀察使、同知大宗正事、安慶軍承宣使,主奉濮王祠事。初,以軍興,南班宗子權罷歲賜,至有身歿而不能殮者,士街言於朝,詔復舊制。三十年,拜安德軍節度使。典宗司凡十四年。士籛官至
 安慶軍節度使、同知大宗正事。隆興元年,上言:「宗司文移視官敘高下,令詪,臣兄也,位反居臣下,失尊卑敘,乞易置之。」詔可其奏。士俴,官至崇慶軍節度使、知西外宗正事。右諫議何溥論士衎強市海舟,罷官。已而詔歸南班,奉朝請。隆興中,以邊事未寧,與士籛奏減奉給恩賞之半以助軍興。詔加獎諭。



 仲儡,景王宗漢子也。初授右內率府副率,轉右監門衛大將軍。建炎末,授武功大夫、忠州防禦使。紹興中,遷濟州,知南外宗正事。八年,加檢
 校少保、向德軍節度使,襲封嗣濮王。仲儡生而不慧,以次得封。入見榻前慟哭,帝驚問故,答語狂謬,帝優容之。九年,薨,上輟朝三日,追封瓊王,謚恭惠。



 士俴,安懿王曾孫也。紹興二十五年十一月襲封,除崇慶軍節度使。初,仲儡薨,秦檜專政,罷襲,檜死,始封士俴。逾年薨,贈少師,追封思王,謚溫靖。



 士輵,士俴弟也。紹興二十八年,由建州觀察使襲封,授昭化軍節度使。初,懿王神貌奉安報恩寺西挾,屋居隘陋,士輵請別營祠堂,許之。久之,加檢
 校少保,累加開府儀同三司,賜嗣濮王居為世業。除知大宗正事,累加三少,充醴泉觀使。淳熙七年薨,贈太傅,追封安王。



 士歆,仲湜第十一子也。由保康軍節度使襲封,加開府儀同三司,累升三少。慶元二年薨,贈太傅,追封韶王。



 不□去,安懿王玄孫也。年七十六,累轉武功郎。士歆既薨,不□去年最高,得襲封,除福州觀察使。由庶官襲封自不□去始。慶元五年,轉武安軍承宣使。俄薨,贈開府儀同三司,追封蔣國公。



 不□去,由武經大夫授利州觀察
 使,襲封。開禧初,遷寧遠軍承宣使。薨,贈開府儀同三司,追封安國公。



 不儔,開禧二年,由安遠軍承宣使襲封,除昭慶軍節度使,遷檢校少保。嘉定十年薨,贈少師,追封高平郡王。



 不嫖,由武翼大夫襲封,授福州觀察使,時嘉定十一年也。逾年而薨,贈開府儀同三司,追封惠國公。



 臣僚上言:「嗣濮王元降指揮,雖有擇高年行遵之文,然高宗朝儀王仲湜以德望俱隆,越仲而選拜;武德郎,次當襲封,以官卑,乃命士俴權奉祠事,越十六年始
 正士俴之封,是亦不拘定制也。乞自今應封者,命大宗司銓量,都堂審察,閣門引見,然後奏取進止。」寧宗然之。



 不凌,父士□芻。不嫖既薨,不凌由右千牛衛將軍授福州觀察使,襲封。嘉定十五年,遷奉國軍承宣使。十七年薨,贈開府儀同三司,追封惠國公



\end{pinyinscope}