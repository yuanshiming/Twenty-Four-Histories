\article{列傳第二百二十九佞幸}

\begin{pinyinscope}

 ○弭德超侯莫陳利用趙贊王黼朱勔王繼先曾覿龍大淵附張說王抃姜特立譙熙載附



 人君生長深宮之中,法家、拂士接耳目之時少,宦官、女子共啟處之日多,二者,佞幸之梯媒也。剛明之主亦有佞幸焉,剛好專任,明好偏察,彼佞幸者一投其機,為患深矣。他日敗闕,雖能殄除,隳城以求狐,灌社以索鼠,亦曰殆哉!宋世中材之君,朝有佞幸,所不免也。太宗有弭德超,趙贊,孝宗有曾覿、龍大淵,二君固不可謂非剛明之主也。作《佞幸傳》。



 弭德超,滄州清池人。李符、李琪薦之,給事太宗晉邸。太
 宗即位,補供奉官。太平興國三年,遷酒坊使、杭州兵馬都監,又為鎮州駐泊都監。



 初,太宗念邊戍勞苦,月賜士卒銀,謂之月頭銀。德超乘間以急變聞於太宗曰:「樞密使曹彬秉政歲久,得士眾心;臣從塞上來,聞士卒言:『月頭銀曹公所致,微曹公我輩餒死矣。』」又巧誣彬他事。上頗疑之,出彬為天平軍節度。以王顯為宣徽南院使,德超為宣徽北院使,並兼樞密副使。



 德超譖曹彬事成,期得樞密使,乃為副使;又柴禹錫與德超官同,先授,班在
 其上。故德超視事月餘,稱病請告,居常怏怏。一日詬顯及禹錫曰:「我言國家大事,有安社稷功,止得線許大官。汝等何人,反在我上,更令我效汝輩所為,我實恥之。」又大罵曰:「:汝輩當斷頭,我度上無守執,為汝輩所眩惑。」顯告之,太宗怒,命膳部郎中、知雜滕中正就第鞫德超,具伏,下詔奪官職,與其家配隸瓊州禁錮,未幾死。



 侯莫陳利用,益州成都人,幼得變幻之術。太平興國初,賣藥京師,言黃白事以惑人。樞密承旨陳從信白於太
 宗,即日召見,試其術頗驗,即授殿直,累遷崇儀副使。雍熙二年,改右監門衛將軍,領應州刺史。三年,諸將北征,以利用與王侁並為並州駐泊都監,擢單州刺史。四年,遷鄭州團練使。前後賜與甚渥,依附者頗獲進用,遂橫恣無復畏憚。其居處服玩皆僭乘輿,人畏之不敢言。



 會趙普再入中書,廉知殺人及諸不法,盡奏之。太宗遣近臣案得奸狀,欲貸其死,普固請曰:「陛下不誅,是亂天下法。法可惜,此何足惜哉!」遂下詔除名,配商州禁錮。初籍
 其家,俄詔還之。



 趙普恐其復用,因殿中丞竇諲嘗監鄭州榷酤,知利用每獨南向坐以接京使,犀玉帶用紅黃羅袋;澶州黃河清,鄭州用為詩題試舉人,利用判試官狀,言甚不遜。召諲至中書詰實,令上疏告之。又京西轉運副使宋沆籍利用家,得書數紙,言皆指斥切害,悉以進上。太宗怒,令中使臠殺之,已而復遣使貸其死,乘疾置至新安,馬旋濘而踣,出濘換馬,比追及之,已為前使誅矣。



 趙贊,並州人,性險詖辯給,好言利害。初為軍小吏,與都校不協,因誣營中謀叛,劉繼元屠之無遺類,稍署贊右職。太原平,隸三司為走吏,又許本司補殿直,太宗頗任之。遷供奉官、閤門祗候,提舉京西、陜西數州錢帛,發摘甚眾。又自乞捕盜,至永興,得兵士盜錢二百,欲磔諸市,知府張齊賢奪而釋之。太宗命御史臺按問,停贊官數月。復令專鉤校三司簿,令贊自選吏十數人為耳目,專伺中書、樞密及三司事,乘間白之。太宗以為忠無他腸,
 中外益畏其口。會改三司官屬,以贊為西京作坊副使、度支都監。



 時又有鄭昌嗣者,宣州人,亦起三司役吏,稍遷侍禁。奉使西川,回奏在官不治者數十人,太宗嘉其直。會市物吏因緣為奸,列肆屢謁開封訴之,乃置雜買務,使昌嗣監之。昌嗣乞著籍便殿門,許非時入奏,與贊親比相表裏,累遷至西上閣門副使、鹽鐵都監。二人既得聯事,由是益橫恣,所為皆不法。太宗頗知之,以問左右,皆畏二人,無敢言其惡。



 至道元年上元節,京城張燈,
 太宗以上清宮成,臨幸。贊與昌嗣邀其黨數人,攜妓樂登宮中玉皇閤,飲宴至夜分;掌舍宦者不能止,以其事聞。太宗大怒,並摭諸事,下詔奪贊官,許攜家配隸房州禁錮,即日驛遣之。昌嗣黜唐州團練副使,不署事。既數日,並賜死於路。



 太宗謂侍臣曰:「君子小人如芝蘭荊棘,不能絕其類,在人甄別耳。茍盡君子,則何用刑罰焉?」參知政事寇準對曰:「帝堯之時,四兇在庭,則三代之前,世質民淳,已有小人矣。今之衣儒服、居清列者,亦頗朋附
 小人,為自安計。如贊、昌嗣之類奔走賤吏,不足言也。」



 王黼字將明,開封祥符人。初名甫,後以同東漢宦官,賜名黼。為人美風姿,目睛如金,有口辯,才疏雋而寡學術,然多智善佞。中崇寧進士第,調相州司理參軍,編修《九域圖志》,何志同領局,喜其人,為父執中言之,薦擢校書郎,遷符寶郎、左司諫。張商英在相位,浸失帝意,遣使以玉環賜蔡京於杭;黼覘知之,數條奏京所行政事,並擊商英。京復相,德其助己,除左諫議大夫、給事中、御史中
 丞,自校書至是財兩歲。



 黼因執中進,乃欲去執中,使京顓國,遂疏其二十罪,不聽。俄兼侍讀,進翰林學士。京與鄭居中不合,黼復內交居中,京怒,徙為戶部尚書,大農方乏,將以邦用不給為之罪。既而諸班禁旅賚犒不如期,詣左藏鼓噪,黼聞之,即諸軍揭大榜,期以某月某日,眾讀榜皆散,京計不行。還為學士,進承旨。



 遭父憂,閱五月,起復宣和殿學士,賜第昭德坊。故門下侍郎許將宅在左,黼父事梁師成,稱為恩府先生,倚其聲焰,逼許氏
 奪之,白晝逐將家,道路憤嘆。復為承旨,拜尚書左丞、中書侍郎。宣和元年,拜特進、少宰。由通議大夫超八階,宋朝命相未有前比也。別賜城西甲第,徙居之日,導以教坊樂,供張什器,悉取於官,寵傾一時。



 蔡京致仕,黼陽順人心,悉反其所為,罷方田,毀闢雍、醫、算學,並會要、六典諸局,汰省吏,減遙郡使、橫班官奉入之半,茶鹽鈔法不復比較,富戶科抑一切蠲除之,四方翕然稱賢相。



 既得位,乘高為邪,多畜子女玉帛自奉,僭擬禁省。誘奪徽猷
 閣待制鄧之綱妾,反以罪竄之綱嶺南。加少保、太宰。請置應奉局,自兼提領,中外名錢皆許擅用,竭天下財力以供費。官吏承望風旨,凡四方水土珍異之物,悉苛取於民,進帝所者不能什一,餘皆入其家。御史陳過庭乞盡罷以御前使喚為名冗官,京西轉運使張汝霖請罷進西路花果,帝既納,黼復露章劾之,兩人皆徙遠郡。



 睦寇方臘起,黼方文太平,不以告,蔓延彌月,遂攻破六郡。帝遣童貫督秦甲十萬始平之。猶以功轉少傅,又進少
 師。貫之行也,帝全付以東南一事,謂之曰:「如有急,即以御筆行之。」貫至吳,見民困花石之擾,眾言:「賊不亟平,坐此耳。」貫即命其僚董耘作手詔,若罪已然,且有罷應奉局之令,吳民大悅。貫平賊歸,黼言於帝曰:「臘之起由茶鹽法也,而貫入奸言,歸過陛下。」帝怒。貫謀起蔡京以間黼,黼懼。



 是時朝廷已納趙良嗣之計,結女真共圖燕,大臣多不以為可。黼曰:「南北雖通好百年,然自累朝以來,彼之慢我者多矣。兼弱攻昧,武之善經也。今弗取,
 女真必強,中原故地將不復為我有。」帝雖向其言,然以兵屬貫,命以保民觀釁為上策。黼復折簡通誠於貫曰:「太師若北行,願盡死力。」時帝方以睦寇故悔其事,及黼一言,遂復治兵。



 黼於三省置經撫房,專治邊事,不關之樞密。括天下丁夫,計口出算,得錢六千二百萬緡,竟買空城五六而奏凱。率百僚稱賀,帝解玉帶以賜,優進太傅,封楚國公,許服紫花袍,騶從儀物幾與親王等。黼議上尊號,帝曰:「此神宗皇帝所不敢受者也。」卻弗許。



 始,遼使至,
 率迂其驛程,燕犒不示以華侈。及黼務於欲速,令女真使以七日自燕至都,每張宴其居,輒陳尚方錦繡、金玉、瑰寶,以誇富盛,由是女真益生心。身為三公,位元宰,至陪扈曲宴,親為俳優鄙賤之役,以獻笑取悅。



 欽宗在東宮,惡其所為。鄆王楷有寵,黼為陰畫奪宗之策。皇孫諶為節度使、崇國公,黼謂但當得觀察使,召宮臣耿南仲諭指,使草代東宮辭諶官奏,竟奪之,蓋欲以是撼搖東宮。



 帝待遇之厚,名其所居閤曰「得賢治定」,為書亭、堂榜
 九。有玉芝產堂柱,乘輿臨觀之。梁師成與連墻,穿便門往來,帝始悟其交結狀。還宮,黼眷頓熄,尋命致仕。



 欽宗受禪,黼惶駭入賀,閤門以上旨不納。金兵入汴,不俟命,載其孥以東。詔貶為崇信軍節度副使,籍其家。吳敏、李綱請誅黼,事下開封尹聶山,山方挾宿怨,遣武士躡及於雍丘南輔固村,戕之,民家取其首以獻。帝以初即位,難於誅大臣,托言為盜所殺。議者不以誅黼為過,而以天討不正為失刑矣。



 朱勔,蘇州人。父沖,狡獪有智數。家本賤微,庸於人,梗悍不馴,抵罪鞭背。去之旁邑乞貸,遇異人,得金及方書歸,設肆賣藥,病人服之輒效,遠近輻湊,家遂富。因修蒔園圃,結游客,致往來稱譽。



 始,蔡京居錢塘,過蘇,欲建僧寺閣,會費鉅萬,僧言必欲集此緣,非朱沖不可。京以屬郡守,郡守呼沖見京,京語故,沖願獨任。居數日,請京詣寺度地,至則大木數千章積庭下,京大驚,陰器其能。明年召還,挾勔與俱,以其父子姓名屬童貫竄置軍籍中,皆
 得官。



 徽宗頗垂意花石,京諷勔語其父,密取浙中珍異以進。初致黃楊三本,帝嘉之。後歲歲增加,然歲率不過再三貢,貢物裁五七品。至政和中始極盛,舳艫相銜於淮、汴,號「花石綱」,置應奉局於蘇,指取內帑如囊中物,每取以數十百萬計。延福宮、艮岳成,奇卉異植充牣其中。勔擢至防禦使,東南部刺史、郡守多出其門。



 徐鑄、應安道、王仲閎等濟其惡,竭縣官經常以為奉。所貢物,豪奪漁取於民,毛發不少償。士民家一石一木稍堪玩,即領
 健卒直入其家,用黃封表識,未即取,使護視之,微不謹,即被以大不恭罪。及發行,必徹屋抉墻以出。人不幸有一物小異,共指為不祥,唯恐芟夷之不速。民預是役者,中家悉破產,或鬻賣子女以供其須。斫山輦石,程督峭慘,雖在江湖不測之淵,百計取之,必出乃止。



 嘗得太湖石,高四丈,載以巨艦,役夫數千人,所經州縣,有拆水門、橋梁,鑿城垣以過者。既至,賜名「神運昭功石」。截諸道糧餉綱,旁羅商船,揭所貢暴其上,篙工、柁師倚勢貪橫,陵
 轢州縣,道路相視以目。廣濟卒四指揮盡給挽士猶不足。京始患之,從容言於帝,願抑其太甚者。帝亦病其擾,乃禁用糧綱船,戒伐塚藏、毀室廬,毋得加黃封帕蒙人園囿花石,凡十餘事。聽勔與蔡攸等六人入貢,餘進奉悉罷。自是勔小戢。



 既而勔甚。所居直蘇市中孫老橋,忽稱詔,凡橋東西四至壤地室廬悉買賜予己,合數百家,期五日盡徙,郡吏逼逐,民嗟哭於路。遂建神霄殿,奉青華帝君像其中,監司、都邑吏朔望皆拜庭下,命士至,輒
 朝謁,然後通刺詣勔。主趙霖建三十六浦閘,興必不可成之功,天方大寒,役死者相枕藉。霖志在媚勔,益加苛虐,吳、越不勝其苦。徽州盧宗原竭庫錢遺之,引為發運使,公肆掊克。園池擬禁籞,服飾器用上僭乘輿。又托挽舟募兵數千人,擁以自衛。子汝賢等召呼鄉州官寮,頤指目攝,皆奔走聽命,流毒州郡者二十年。



 方臘起,以誅勔為名。童貫出師,承上旨盡罷去花木進奉,帝又黜勔父子弟侄在職者,民大悅。然寇平,勔復得志,聲焰熏灼。
 邪人穢夫,候門奴事,自直秘閣至殿學士,如欲可得,不附者旋踵罷去,時謂東南小朝廷。帝末年益親任之,居中白事,傳達上旨,大略如內侍,進見不避宮嬪。歷隨州觀察使、慶遠軍承宣使。燕山奏功,進拜寧遠軍節度使、醴泉觀使。一門盡為顯官,騶僕亦至金紫,天下為之扼腕。



 靖康之難,欲為自全計,倉卒擁上皇南巡,且欲邀至其第。欽宗用御史言,放歸田里,凡由勔得官者皆罷。籍其貲財,田至三十萬畝。言者不已,羈之衡州,徙韶州、循
 州,遣使即所至斬之。



 王繼先,開封人。奸黠善佞。建炎初以醫得幸,其後浸貴寵,世號王醫師。至和安大夫、開州團練使致仕。尋以覃恩,改授武功大夫,落致仕。給事中富直柔奏:「繼先以雜流易前班,則自此轉行無礙,深恐將帥解體。」帝曰:「朕頃冒海氣,繼先診視有奇效。可特書讀。」直柔再駁,命乃寢。既而特授榮州防禦使。



 太后有疾,繼先診視有勞,特補其子悅道為閣門祗候。尋命繼先主管翰林醫官局,力
 辭。是時,繼先用事,中外切齒,乃陽乞致仕,以避人言。詔遷秩二等,許回授。俄除右武大夫、華州觀察使,詔餘人毋得援例。吳貴妃進封,推恩遷奉寧軍承宣使,特封其妻郭氏為郡夫人。



 繼先遭遇冠絕人臣,諸大帥承順下風,莫敢少忤,其權勢與秦檜埒。檜使其夫人詣之,敘拜兄弟,表裏引援。遷昭慶軍承宣使,又欲得節鉞,使其徒張孝直等校《本草》以獻,給事中楊椿沮之,計不行。繼先富埒王室,子弟通朝籍,總戎寄,姻戚黨與盤據要途,數
 十年間,無能搖之者。



 金兵將至,劉錡請為戰備,繼先乃言:「新進主兵官,好作弗靖,若斬一二人,和好復固。」帝不懌曰:「是欲我斬劉錡乎?」



 侍御史杜莘老劾其十罪,大略謂:「繼先廣造第宅,占民居數百家,都人謂之『快樂仙宮』;奪良家婦女為侍妾,鎮江有娼妙於歌舞,矯御前索之;淵聖成喪,舉家燕飲,令妓女舞而不歌,謂之『啞樂』;自金使來,日輦重寶之吳興,為避走計;陰養惡少,私置兵甲;受富民金,薦為閤職;州縣大獄,以賂解免;誣姊奸淫,加
 之黥隸;又於諸處佛寺建立生祠,凡名山大剎所有,大半入其家。此特舉其大者,其餘擢發未足數也。」



 奏入,詔繼先福州居住。其子安道,武泰軍承宣使;守道,朝議大夫、直徽猷閣;悅道朝奉郎、直秘閣;孫錡,承議郎、直秘閣,並勒停。放還良家子為奴婢者凡百餘人。籍其貲以千萬計,鬻其田園及金銀,並隸御前激賞庫。其海舟付李寶,天下稱快。



 方繼先之怙寵奸法,帝亦知之,故晚年以公議廢之,遂不復起。孝宗即位,詔任便居住,毋至行在。
 淳熙八年,卒。



 曾覿,字純甫,其先汴人也。用父任補官。紹興三十年,以寄班祗候與龍大淵同為建王內知客。孝宗受禪,大淵自左武大夫除樞密副都承旨,而覿自武翼郎除帶御器械、乾辦皇城司。諫議大夫劉度入對,首言二人潛邸舊人,待之不可無節度;又因進故事,論京房、石顯事。大淵遂除知閣門事,而覿除權知閣門事。度言:「臣欲退之,而陛下進之,何面目尚為諫官?乞賜貶黜。」中書舍人張
 震繳其命至再,出知紹興府。殿中侍御史胡沂亦論二人市權,既而給舍金安節、周必大再封還錄黃。時張燾新拜參政,亦欲以大淵、覿決去就,力言之,帝不納。燾辭去,遂以內祠兼侍讀。劉度奪言職,權工部侍郎,而二人仍知閣門事。必大格除目不下,尋與祠,二人除命亦寢。未幾,卒以大淵為宜州觀察使、知閣門事;覿,文州刺史、權知閣門:皆兼皇城司。不數月間,除命四變。劉度出知建寧府,尋放罷。



 群臣既以言二人得罪去,侍御史周操
 章十五上,不報。自是覿與大淵勢張甚,士大夫之寡恥者潛附麗之。帝嘗令大淵撫慰兩淮將士,侍御史王十朋言大淵銜命撫師,非出朝廷論選之公,有輕國體。時又有內侍押班梁珂者,三人表裏用事。及珂以罪出,右正言龔茂良入對,首論:「二人害政甚珂百倍,陛下罷行一政事,進退一人才,必掠美自歸,謂為己力。或時有少過,昌言於外,謂嘗爭之而不見聽。群臣章疏留中未出,間得窺見,出以語人。有司條陳利害,示以副封,公然可
 否。若夫交通賄賂,干求差遣,特其小者耳。願特出威斷,並行罷去。」



 先是,江、浙大水,詔侍從、臺諫陳闕政。著作郎劉夙上封事曰:「陛下與覿、大淵輩觴詠唱酬,字而不名。罷宰相,易大將,待其言而後決。嚴法守,裁僥幸,當自宮掖近侍始。」茂良時為監察御史,亦言:「水至陰,其占為女寵,為嬖佞,為小人,蓋專指左右近習也。」帝諭以二人皆潛邸舊人,非近習比;且俱有文學,敢諫諍,杜門不出,不預外事,宜退而訪問。茂良再上疏言:「德宗不知盧杞之
 奸邪,此其所以奸邪也。大淵、覿所為,行道之人能言之,特陛下未之覺耳。」疏入不報。茂良待罪,除太常少卿,五辭不拜,出知建寧府。



 一日,右史洪邁過參政陳俊卿曰:「聞將除右史,邁遷西掖,信乎?」俊卿曰:「何自得之?」邁以二人告。俊卿即以語宰相葉顒、魏杞,而己獨奏之,且以邁語質之帝前,帝怒,即出二人於外。於是遷大淵為江東總管,覿為淮西副總管,中外快之。尋改大淵浙東、覿福建。乾道四年,大淵死,覿尚在福建。帝憐,欲召之,樞密劉
 珙奏曰:「此曹奴隸爾,厚賜之可也。引以自近而待以賓友,使得與聞政事,非所以增聖德、整朝綱也。」帝納珙言,命遂寢。



 既而覿垂滿,俊卿恐其入,預請以浙東總管處之。臺臣上疏論之,不報。太學錄魏掞之亟上封事論列,且見俊卿切責之,掞之得臺州教官以出。覿至龍山已久,伺掞之去,然後入國門。會虞允文使蜀還,與俊卿同奏覿不可留。帝曰:「然,留則累朕。」卒除浙東副總管。未幾,以墨詔進覿一官為觀察使,中書舍人繳還,不因事除
 拜,必有人言。帝不聽。俊卿曰:「不爾,亦須有名。」會汪大猷為賀金正旦使,俾覿副之。比還,遷一秩,而竟申浙東之命,且戒閣門吏趣朝辭,覿由是怏怏而去。



 六年夏,俊卿罷政。十月,覿以京祠召。七年,立皇太子,覿以伴讀勞,升承宣使。八年,姚憲為賀金國尊號使,覿副之。歸,除武泰軍節度使,提舉萬壽觀。淳熙元年,除開府儀同三司。四年,覿欲以文資官其子孫,帝遣中使至省中具使相奏補法,龔茂良時以參政行丞相事,遽以文武官各隨本色
 蔭補法繳進,覿大怒。茂良退朝,覿從騎不避,茂良執而撻之,待罪乞出,不許。戶部員外郎謝廓然忽賜出身,除侍御史。廓然首論茂良,以資政殿學士知鎮江;章再上,鐫罷;言之不已,貶英州,皆覿所使也。覿前雖預事,未敢肆,至是責逐大臣,士始側目重足矣。廓然既以擅權罪茂良,從班有韓彥古者,覿之姻,廓然之黨,遂獻議助之,使人主疑大臣而信近習,至是益甚。



 六年二月,帝幸祐聖觀,召宰臣史浩及覿同賜酒。是歲,加覿少保、醴泉觀
 使。時周必大當草制,人謂其必不肯從,及制出,乃有「敬故在尊賢之上」之語,士論惜之。



 覿始與龍大淵相朋,及大淵死,則與王抃、甘昪相蟠結,文武要職多出三人之門。葉衡自小官十年至宰相。徐本中由小使臣積階至刺史、知閣門事,換文資為右文殿修撰、樞密都承旨、賜三品服,俄為浙西提刑,尋以集英殿修撰奉內祠。是二人者,皆覿所進也。



 著作郎胡晉臣因轉對,極論近習怙權之害,遂出知漢州。南康守朱熹應詔上書,其言尤力,
 有曰:「一二近習之人,蠱惑陛下心志,所謂宰相、師傅、賓友、諫諍之臣,或反出入其門墻,承望其風旨。」疏入,帝怒,諭令分析,丞相趙雄言之,事遂止。陳俊卿守金陵,過闕入見,首言曾覿、王抃招權納賂,薦進人才,皆以中批行之。帝曰:「瑣細差遣,或勉循之。至於近上之除,此輩何敢預。」俊卿入辭,又曰:「向來士大夫奔覿、抃之門,十才一二,尚畏人知;今則公然趨附,十已八九,大非朝廷美事也。」帝感悟。覿用事二十年,權震中外,至於譖逐大臣,貶死
 嶺外。自是浸覺其奸,嘗謂左右曰:「曾覿誤我不少。」遂稍疏覿。



 覿憂恚,疽發於背。七年三月,侍帝宴於翠寒堂,退為記以進。十二月,卒。於是凡前論覿得罪者皆錄贈,胡晉臣起至執政,魏掞之贈直秘閣,龔茂良悉還其職名恩數云。



 張說,開封人。父公裕,省吏也,為和州防禦使,建炎初有軍功。說受父任為右職,娶壽聖皇后女弟,由是累遷知閣門事。隆興初,兼樞密副都承旨。乾道初,為都承旨,加
 明州觀察使。



 七年三月,除簽書樞密院事。時起復劉珙同知樞密院,珙恥與之同命,力辭不拜,命既下,朝論嘩然不平,莫敢頌言於朝者。惟左司員外郎張栻在經筵力言之,中書舍人範成大不草詞。尋除說安遠軍節度使,奉祠歸第。不數月,出栻知袁州。說既奉祠,語人曰:「張左司平時不相樂,固也。範致能亦胡為見攻?」指所坐亭材植曰:「是皆致能所惠也。」



 八年二月,復自安遠軍節度使提舉萬壽觀,簽書樞密院事。侍御史李衡、右正言王希
 呂交章論之,起居郎莫濟不書錄黃,直院周必大不草答詔,於是命權給事中姚憲書讀行下,命翰林學士王曮草答詔,未幾,曮升學士承旨,憲贈出身,為諫議大夫。詔希呂合黨邀名,持論反覆,責遠小監當。衡素與說厚,所言亦婉,止罷言職,遷左史,而濟、必大皆與在外宮觀,日下出國門。國子司業劉焞移書責宰相,言說不當用,即為言者所論,出為江西轉運判官。於是說勢赫然,無敢攖之者。九年春,說露章薦濟、必大,於是二人皆予郡,
 必大卒不出。



 淳熙元年,帝廉知說欺罔數事,命侍御史範仲芑究之,遂罷為太尉,提舉玉隆宮。諫官湯邦彥又劾其奸贓,乃降為明州觀察使、責居撫州。三年,許自便。七年,卒於湖州。帝猶念之,詔復承宣使,給事中陳峴繳之,乃止。其子薦,文州刺史;嶷,明州觀察使。說敗,薦亦貶郴州。



 先是,南丹州莫延葚表乞就宜州市馬,比橫山省三十程,說在樞筦以聞,樞屬有論其不便,說不聽。說既貶,遂罷其議。說又嘗建議欲郎官、卿監通差武臣,中書
 舍人留正以為不可,遂止。與右相梁克家議使事不合,克家罷去而說留,其竊政權、傾大臣類如此。



 王抃,初為國信所小吏。金人求海、泗、唐、鄧、商、秦地,議久不決。金兵至,遣抃往使,許以地,易歲貢為歲幣而還。乾道中,積官至知閣門事,帝親信之。金使至,議國書禮,不合,抃以宰執虞允文命,紿其使曰:「兩朝通好自有常禮,使人何得妄生事,已牒知對境。」翌日,金使乃進書。帝以為可任,遣詣荊襄點閱軍馬。



 淳熙中,兼樞密都承旨,建
 議以殿、步二司軍多虛籍,請各募三千人。已而殿司輒捕市人充軍,號呼滿道,軍士乘隙掠取民財。帝專以罪殿前指揮使王友直,而命抃權殿前司事。



 時抃與曾覿、甘昪相結,恃恩專恣,其門如市。著作郎胡晉臣嘗論近習怙權,帝令執政趙雄詢其人,雄憚抃等,乃令晉臣舍抃等,指其位卑者數人以對,晉臣竟外補。校書郎鄭鑒、宗正丞袁樞因轉對,數為帝言之,帝猶未之覺也。吏部侍郎趙汝愚力疏抃罪,言:「陛下即位之初,宰相如葉顒
 等皆懼陛下左右侵其權,日夜與之為敵。陛下察數年已來,大臣還有與陛下左右角是非者否?蓋其勢積至此也。今將帥之權盡歸王抃矣。」



 先是,抃紿金使取國書,及使歸,金主誅之。嗣歲,金使至,帝以德壽宮之命,為離席受國書,尋悔之。淳熙八年,金賀正旦使至,復要帝起立如舊儀,帝遽入內,抃擅許金使用舊儀見。翌日,汝愚侍殿上,帝不懌數日。汝愚因亟攻抃,帝遂出抃外祠,不復召。淳熙十一年,以福州觀察使卒。



 姜特立字邦傑,麗水人。以父綬恩,補承信郎。



 淳熙中,累遷福建路兵馬副都監。海賊姜大獠寇泉南,特立以一舟先進,擒之。帥臣趙汝愚薦於朝,召見,獻所為詩百篇,除閣門舍人,命充太子宮左右春坊兼皇孫平陽王伴讀,由是得幸於太子。太子即位,除知閣門事,與譙熙載皆以春坊舊人用事,恃恩無所忌憚,時人謂曾、龍再出。



 留正為右相,執政尚闕人,特立一日語正曰:「帝以承相在位久,欲遷左揆,就二尚書中擇一人執政,孰可者?」明
 日,正論其招權納賄之狀,遂奪職與外祠。帝念之,復除浙東馬步軍副總管,詔賜錢二千緡為行裝。正引唐憲宗召吐突承璀事,乞罷相,不許。正復言:「臣與特立勢難兩立。」帝答曰:「成命已班,朕無反汗,卿宜自處。」正待罪國門外,帝不復召,而特立亦不至。寧宗受禪,特立遷和州防禦使,再奉祠,俄拜慶遠軍節度使,卒。



 熙載亦為平陽邸伴讀,累官至忠州防禦使、知閣門事。紹熙中卒,較之特立頗廉勤。



 熙載子令雍,以恩補承信郎、平陽郡王府
 乾辦,尋充王府內知客,小有才。王嘗與論《春秋》褒貶齊宣王易牛、秦穆公悔過事,令雍即為三詩以獻,王甚愛重之。及即位,除知閣門事,累遷至揚州承宣使。謝事,拜保成軍節度使。初賜居第,帝親書「依光」二字賜之。至是,復書「得閑知止」四字以名其堂。寶璽歸,覃恩進檢校少保,仍轉太尉致仕。卒,贈開府儀同三司。



\end{pinyinscope}