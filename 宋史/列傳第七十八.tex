\article{列傳第七十八}

\begin{pinyinscope}

 歐陽脩子發棐劉敞弟分文子奉世曾鞏弟肇



 歐陽脩,字永叔,廬陵人。四歲而孤,母鄭,守節自誓,親誨之學,家貧,至以荻畫地學書。幼敏悟過人,讀書輒成誦。及冠,嶷然有聲。



 宋興且百年,而文章體裁,猶仍五季餘
 習。鎪刻駢偶,淟涊弗振,士因陋守舊,論卑氣弱。蘇舜元、舜欽、柳開、穆修輩,咸有意作而張之,而力不足。修游隨,得唐韓愈遺稿於廢書簏中,讀而心慕焉。苦志探賾,至忘寢食,必欲並轡絕馳而追與之並。



 舉進士,試南宮第一,擢甲科,調西京推官。始從尹洙游,為古文,議論當世事,迭相師友,與梅堯臣游,為歌詩相倡和,遂以文章名冠天下。入朝,為館閣校勘。



 範仲淹以言事貶,在廷多論救,司諫高若訥獨以為當黜。修貽書責之,謂其不復知
 人間有羞恥事。若訥上其書,坐貶夷陵令,稍徙乾德令、武成節度判官。仲淹使陜西,闢掌書記。修笑而辭曰:「昔者之舉,豈以為己利哉?同其退不同其進可也。」久之,復校勘,進集賢校理。慶歷三年,知諫院。時仁宗更用大臣,杜衍、富弼、韓琦、範仲淹皆在位,增諫官員,用天下名士,修首在選中。每進見,帝延問執政,咨所宜行。既多所張弛,小人翕翕不便。修慮善人必不勝,數為帝分別言之。初,範仲淹之貶饒州也,修與尹洙、餘靖皆以直仲淹見
 逐,目之曰「黨人」。自是,朋黨之論起,修乃為《朋黨論》以進。其略曰:「君子以同道為朋,小人以同利為朋,此自然之理也。臣謂小人無朋,惟君子則有之。小人所好者利祿,所貪者財貨,當其同利之時,暫相黨引以為朋者,偽也。及其見利而爭先,或利盡而反相賊害,雖兄弟親戚,不能相保,故曰小人無朋。君子則不然,所守者道義,所行者忠信,所惜者名節。以之修身,則同道而相益,以之事國,則同心而共濟,終始如一,故曰惟君子則有朋。紂有
 臣億萬,惟億萬心,可謂無朋矣,而紂用以亡。武王有臣三千,惟一心,可謂大朋矣,而周用以興。蓋君子之朋,雖多而不厭故也。故為君但當退小人之偽朋,用君子之真朋,則天下治矣。」



 修論事切直,人視之如仇,帝獨獎其敢言,面賜立品服。顧侍臣曰:「如歐陽修者,何處得來?」同修起居注,遂知制誥。故事,必試而後命,帝知修,詔特除之。



 奉使河東。自西方用兵,議者欲廢麟州以省饋餉。修曰:「麟州,天險,不可廢;廢之,則河內郡縣,民皆不安居矣。
 不若分其兵,駐並河內諸堡,緩急得以應援,而平時可省轉輸,於策為便。」由是州得存。又言:「忻、代、岢嵐多禁地廢田,願令民得耕之,不然,將為敵有。」朝廷下其議,久乃行,歲得粟數百萬斛。凡河東賦斂過重民所不堪者,奏罷十數事。使還,會保州兵亂,以為龍圖閣直學士、河北都轉運使。陛辭,帝曰:「勿為久留計,有所欲言,言之。」對曰:「臣在諫職得論事,今越職而言,罪也。」帝曰:「第言之,毋以中外為間。」賊平,大將李昭亮、通判馮博文私納婦女,修
 捕博文系獄,昭亮懼,立出所納婦。兵之始亂也,招以不死,既而皆殺之,脅從二千人,分隸諸郡。富弼為宣撫使,恐後生變,將使同日誅之,與修遇於內黃,夜半,屏人告之故。修曰:「禍莫大於殺已降,況脅從乎?既非朝命,脫一郡不從,為變不細。」弼悟而止。



 方是時,杜衍等相繼以黨議罷去,修慨然上疏曰:「杜衍、韓琦、範仲淹、富弼,天下皆知其有可用之賢,而不聞其有可罷之罪,自古小人讒害忠賢,其說不遠。欲廣陷良善,不過指為朋黨,欲動搖
 大臣,必須誣以顓權,其故何也?去一善人,而眾善人尚在,則未為小人之利;欲盡去之,則善人少過,難為一一求瑕,唯指以為黨,則可一時盡逐,至如自古大臣,已被主知而蒙信任,則難以他事動搖,唯有顓權是上之所惡,必須此說,方可傾之。正士在朝,群邪所忌,謀臣不用,敵國之福也。今此四人一旦罷去,而使群邪相賀於內,四夷相賀於外,臣為朝廷惜之。」於是邪黨益忌修,因其孤甥張氏獄傅致以罪,左遷知制誥、知滁州。居二年,徙
 揚州、穎州。復學士,留守南京,以母憂去。服除,召判流內銓,時在外十二年矣。帝見其發白,問勞甚至。小人畏修復用,有詐為修奏,乞澄汰內侍為奸利者。其群皆怨怒,譖之,出知同州,帝納吳充言而止。遷翰林學士,俾修《唐書》。奉使契丹,其主命貴臣四人押宴,曰:「此非常制,以卿名重故爾。」



 知嘉祐二年貢舉。時士子尚為險怪奇澀之文,號「太學體」,修痛排抑之,凡如是者輒黜。畢事,向之囂薄者伺修出,聚噪於馬首,街邏不能制;然場屋之習,從
 是遂變。



 加龍圖閣學士、知開封府,承包拯威嚴之後,簡易循理,不求赫赫名,京師亦治。旬月,改群牧使。《唐書》成,拜禮部侍郎兼翰林侍讀學士。修在翰林八年,知無不言。河決商胡,北京留守賈昌朝欲開橫□故道,回河使東流。有李仲昌者,欲導入六塔河,議者莫知所從。修以為:「河水重濁,理無不淤,下流既淤,上流必決。以近事驗之,決河非不能力塞,故道非不能力復,但勢不能久耳。橫□功大難成,雖成將復決。六塔狹小,而以全河注之,
 濱、棣、德、博必被其害。不若因水所趨,增堤峻防,疏其下流,縱使入海,此數十年之利也。」宰相陳執中主昌朝,文彥博主仲昌,竟為河北患。



 臺諫論執中過惡,而執中猶遷延固位。修上疏,以為「陛下拒忠言,庇愚相,為聖德之累」。未幾,執中罷。狄青為樞密使,有威名,帝不豫,訛言籍籍,修請出之於外,以保其終,遂罷知陳州。修嘗因水災上疏曰:「陛下臨御三紀,而儲宮未建。昔漢文帝初即位,以群臣之言,即立太子,而享國長久,為漢太宗。唐明宗
 惡人言儲嗣事,不肯早定,致秦王之亂,宗社遂覆。陛下何疑而久不定乎?」其後建立英宗,蓋原於此。



 五年,拜樞密副使。六年,參知政事。修在兵府,與曾公亮考天下兵數及三路屯戍多少、地理遠近,更為圖籍。凡邊防久缺屯戍者,必加搜補。其在政府,與韓琦同心輔政。凡兵民、官吏、財利之要,中書所當知者,集為總目,遇事不復求之有司。時東宮猶未定,與韓琦等協議大議,語在《琦傳》。英宗以疾未親政,皇太后垂簾,左右交構,幾成嫌隙。韓
 琦奏事,太后泣語之故。琦以帝疾為解,太后意不釋,修進曰:「太后事仁宗數十年,仁德著於天下。昔溫成之寵,太后處之裕如;今母子之間,反不能容邪?」太后意稍和,修復曰:「仁宗在位久,德澤在人。故一日晏駕,天下奉戴嗣君,無一人敢異同者。今太后一婦人,臣等五六書生耳,非仁宗遺意,天下誰肯聽從。」太后默然,久之而罷。



 修平生與人盡言無所隱。及執政,士大夫有所干請,輒面諭可否,雖臺諫官論事,亦必以是非詰之,以是怨誹益
 眾。帝將追崇濮王,命有司議,皆謂當稱皇伯,改封大國。修引《喪服記》,以為:「『為人後者,為其父母服。』降三年為期,而不沒父母之名,以見服可降而名不可沒也。若本生之親,改稱皇伯,歷考前世,皆無典據。進封大國,則又禮無加爵之道。故中書之議,不與眾同。」太后出手書,許帝稱親,尊王為皇,王夫人為後。帝不敢當。於是御史呂誨等詆修主此議,爭論不已,皆被逐。惟蔣之奇之說合修意,修薦為御史,眾目為奸邪。之奇患之,則思所以自解。
 修婦弟薛宗孺有憾於修,造帷薄不根之謗摧辱之,展轉達於中丞彭思永,思永以告之奇,之奇即上章劾修。神宗初即位,欲深護修。訪故宮臣孫思恭,思恭為辨釋,修杜門請推治。帝使詰思永、之奇,問所從來,辭窮,皆坐黜。修亦力求退,罷為觀文殿學士、刑部尚書、知亳州。明年,遷兵部尚書、知青州,改宣徽南院使、判太原府。辭不拜,徙蔡州。



 修以風節自持,既數被污蔑,年六十,即連乞謝事,帝輒優詔弗許。及守青州,又以請止散青苗錢,為
 安石所詆,故求歸愈切。熙寧四年,以太子少師致仕。五年,卒,贈太子太師,謚曰文忠。



 修始在滁州,號醉翁,晚更號六一居士。天資剛勁,見義勇為,雖機阱在前,觸發之不顧。放逐流離,至於再三,志氣自若也。方貶夷陵時,無以自遣,因取舊案反復觀之,見其枉直乖錯不可勝數,於是仰天嘆曰:「以荒遠小邑,且如此,天下固可知。」自爾,遇事不敢忽也。學者求見,所與言,未嘗及文章,惟談吏事,謂文章止於潤身,政事可以及物。凡歷數郡,不見治
 跡,不求聲譽,寬簡而不擾,故所至民便之。或問:「為政寬簡,而事不弛廢,何也?」曰:「以縱為寬,以略為簡,則政事弛廢,而民受其弊。吾所謂寬者,不為苛急;簡者,不為繁碎耳。修幼失父,母嘗謂曰:「汝父為吏,常夜燭治官書,屢廢而嘆。吾問之,則曰:『死獄也,我求其生,不得爾。』吾曰:『生可求乎?』曰:『求其生而不得,則死者與我皆無恨。夫常求其生,猶失之死,而世常求其死也。』其平居教他子弟,常用此語,吾耳熟焉。」修聞而服之終身。



 為文天才自然,豐約
 中度。其言簡而明,信而通,引物連類,折之於至理,以服人心。超然獨騖,眾莫能及,故天下翕然師尊之。獎引後進,如恐不及,賞識之下,率為聞人。曾鞏、王安石、蘇洵、洵子軾、轍,布衣屏處,未為人知,修即游其聲譽,謂必顯於世。篤於朋友,生則振掖之,死則調護其家。



 好古嗜學,凡周、漢以降金石遺文、斷編殘簡,一切掇拾,研稽異同,立說於左,的的可表證,謂之《集古錄》。奉詔修《唐書》紀、志、表,自撰《五代史記》,法嚴詞約,多取《春秋》遺旨。蘇軾敘其文
 曰:「論大道似韓愈,論事似陸贄,記事似司馬遷,詩賦似李白。」識者以為知言。



 子發字伯和,少好學,師事安定胡瑗,得古樂鐘律之說,不治科舉文詞,獨探古始立論議。自書契以來,君臣世系,制度文物,旁及天文、地理,靡不悉究。以父恩,補將作監主薄,賜進士出身,累遷殿中丞。卒,年四十六。蘇軾哭之,以謂發得文忠公之學,漢伯喈、晉茂先之流也。



 中子棐字叔弼,廣覽強記,能文辭,年十三時,見修著《鳴
 蟬賦》,侍側不去。修撫之曰:「兒異日能為吾此賦否?」因書以遺之。用蔭,為秘書省正字,登進士乙科,調陳州判官,以親老不仕。修卒,代草遺表,神宗讀而愛之,意修自作也。服除,始為審官主簿,累遷職方員外郎、知襄州。曾布執政,其婦兄魏泰倚聲勢來居襄,規占公私田園,強市民貨,郡縣莫敢誰何。至是,指州門東偏官邸廢址為天荒,請之。吏具成牘至,棐曰:「孰謂州門之東偏而有天荒乎?」卻之。眾共白曰:「泰橫於漢南久,今求地而緩與之,且
 不可,而又可卻邪?」棐竟持不與。泰怒,譖於布,徙知潞州,旋又罷去。元符末,還朝。歷吏部、右司二郎中,以直秘閣知蔡州。蔡地薄賦重,轉運使又為覆折之令,多取於民,民不堪命。會有詔禁止,而佐吏憚使者,不敢以詔旨從事。棐曰:「州郡之於民,詔令茍有未便,猶將建請。今天子詔意深厚,知覆折之病民,手詔止之。若有憚而不行,何以為長吏?」命即日行之。未幾,坐黨籍廢,十餘年卒。



 論曰:「三代而降,薄乎秦、漢,文章雖與時盛衰,而藹如其
 言,曄如其光,皦如其音,蓋均有先王之遺烈。涉晉、魏而弊,至唐韓愈氏振起之。唐之文,涉五季而弊,至宋歐陽修又振起之。挽百川之頹波,息千古之邪說,使斯文之正氣,可以羽翼大道,扶持人心,此兩人之力也。愈不獲用,修用矣,亦弗克究其所為,可為世道惜也哉!



 劉敞,字原父,臨江新喻人。舉慶歷進士,廷試第一。編排官王堯臣,其內兄也,以親嫌自列,乃以為第二。通判蔡州,直集賢院,判尚書考功。



 夏竦薨,賜謚文正。敞言:「謚者,
 有司之事,竦行不應法。今百司各得守其職,而陛下侵臣官。」疏三上,改謚文莊。方議定大樂,使中貴人參其間。敞諫曰:「王事莫重於樂。今儒學滿朝,辨論有餘,而使若趙談者參之,臣懼為袁盎笑也。」權度支判官,徙三司使。



 秦州與羌人爭古渭地。仁宗問敞:「棄守孰便?」敞曰:「若新城可以蔽秦州,長無羌人之虞,傾國守焉可也。或地形險利,賊乘之以擾我邊鄙,傾國爭焉可也。今何所重輕,而殫財困民,捐士卒之命以規小利,使曲在中國,非計
 也。」議者多不同,秦州自是多事矣。



 溫成後追冊,有佞人獻議,求立忌。敞曰:「豈可以私暱之故,變古越禮乎?」乃止。吳充以典禮得罪,馮京救之,亦罷近職。敞因對極論之。帝曰:「充能官,京亦亡它,中書惡其太直,不兼容耳。」敞曰:「陛下寬仁好諫,而中書乃排逐言者,是蔽君之明,止君之善也。臣恐感動陰陽,有日食、地震、風霾之異。」已而果然。因勸帝收攬威權,無使聰明蔽塞,以消災咎。帝深納之,以同修起居注。未一月,擢知制誥。宰相陳執中惡其
 斥己,沮止之,帝不聽。宦者石全彬領觀察使,意不愜,有慍言,居三日為真,敞封還除書,不草制。



 奉使契丹,素習知山川道徑,契丹導之行,自古北口至柳河,回屈殆千里,欲誇示險遠。敞質譯人曰:「自松亭趨柳河,甚徑且易,不數日可抵中京,何為故道此?」譯相顧駭愧曰:「實然。但通好以來,置驛如是,不敢變也。」順州山中有異獸,如馬而食虎豹,契丹不能識,問敞。敞曰:「此所謂駁也。」為說其音聲形狀,且誦《山海經》、《管子》書曉之,契丹益嘆服。使還,
 求知揚州。



 狄青起行伍為樞密使,每出入,小民輒聚觀,至相與推誦其拳勇,至壅馬足不得行。帝不豫,人心動搖,青益不自安。敞辭赴郡,為帝言曰:「陛下幸愛青,不如出之,以全其終。」帝頷之,使出諭中書,青乃去位。



 揚之雷塘,漢雷陂也,舊為民田。其後官取瀦水而不償以它田,主皆失業。然塘亦破決不可漕,州復用為田。敞據唐舊券,悉用還民,發運使爭之,敞卒以予民。天長縣鞫王甲殺人,既具獄,敞見而察其冤,甲畏吏,不敢自直。敞以委戶
 曹杜誘,誘不能有所平反,而傅致益牢。將論囚,敞曰:「冤也。」親按問之。甲知能為己直,乃敢告,蓋殺人者,富人陳氏也。相傳以為神明。徙鄆州,鄆比易守,政不治,市邑攘敓公行。敞決獄訟,明賞罰,境內肅然。客行壽張道中,遺一囊錢,人莫敢取,以告里長,里長為守視,客還,取得之。又有暮遺物市中者,旦往訪之,故在。先是,久旱,地多蝗。敞至而雨,蝗出境。召糾察在京刑獄。營卒桑達等醉斗,指斥乘輿。皇城使捕送開封,棄達市。敞移府,問何以不
 經審訊。府報曰:「近例,凡聖旨及中書、樞密所鞫獄,皆不慮問。」敞奏請一準近格,樞密院不肯行,敞力爭之,詔以其章下府,著為令。



 嘉祐祫享,群臣上尊號,宰相請撰表。敞說止不得,乃上疏曰:「陛下不受徽號且二十年。今復加數字,不足盡聖德,而前美並棄,誠可惜也。今歲以來,頗有災異,正當寅畏天命,深自抑損,豈可於此時乃以虛名為累。」帝覽奏,顧侍臣曰:「我意本謂當爾。」遂不受。



 蜀人龍昌期著書傳經,以詭僻惑眾。文彥博薦諸朝,賜五品
 服。敞與歐陽修俱曰:「昌期違古畔道,學非而博,王制之所必誅,未使即少正卯之刑,已幸矣,又何賞焉。乞追還詔書,毋使有識之士,窺朝廷深淺。」昌期聞之,懼不敢受賜。



 敞以識論與眾忤,求知永興軍,拜翰林侍讀學士。大姓範偉為奸利,冒同姓戶籍五十年,持府縣短長,數犯法。敞窮治其事,偉伏罪,長安中言雚喜。未及受刑,敞召還,判三班院,偉即變前獄,至於四五,卒之付御史決。



 敞侍英宗講讀,每指事據經,因以諷諫。時兩宮方有小人間
 言,諫者或訐而過直。敞進讀《史記》,至堯授舜以天下,拱而言曰:「舜至側微也,堯禪之以位,天地享之,百姓戴之,非有他道,惟孝友之德,光於上下耳。」帝竦體改容,知其以義理諷也。皇太后聞之,亦大喜。



 積苦眩瞀,屢予告。帝固重其才,每燕見他學士,必問敞安否;帝食新橙,命賜之。疾少間,復求外,以為汝州,旋改集賢院學士、判南京御史臺。熙寧元年,卒,年五十。



 敞學問淵博,自佛老、卜筮、天文、方藥、山經、地志,皆究知大略。嘗夜視鎮星,謂人曰:「
 此於法當得土,不然,則生女。」後數月,兩公主生。又曰:「歲星往來虛、危間,色甚明盛,當有興於齊者。」歲餘而英宗以齊州防禦使入承大統。嘗得先秦彞鼎數十,銘識奇奧,皆案而讀之,因以考知三代制度,尤珍惜之。每曰:「我死,子孫以此蒸嘗我。」朝廷每有禮樂之事,必就其家以取決焉。為文尤贍敏。掌外制時,將下直,會追封王、主九人,立馬卻坐,頃之,九制成。歐陽修每於書有疑,折簡來問,對其使揮筆,答之不停手,修服其博。長於《春秋》,為書
 四十卷,行於時。弟分文,子奉世。



 分文字貢父,與敞同登科,仕州縣二十年,始為國子監直講。歐陽修、趙概薦試館職,御史中丞王陶有夙憾,率侍御史蘇寀共排之,分文官已員外郎,才得館閣校勘。熙寧中,判尚書考功、同知太常禮院。



 詔封太祖諸孫行尊者為王,奉太祖後。分文言:「禮,諸侯不得祖天子,當自奉其國之祖。宜崇德昭、德芳之後,世世勿降爵,宗廟祭祀,使之在位,則所以褒揚藝祖者著矣。」後二王紹封,如分文議。



 方
 更學校貢舉法,分文曰:「本朝選士之制,行之百年,累代將相名卿,皆由此出,而以為未嘗得人,不亦誣哉。願因舊貫,毋輕議改法。夫士修於家,足以成德,亦何待於學官程課督趣之哉。」



 王安石在經筵,乞講者坐。分文曰:「侍臣講論於前,不可安坐,避席立語,乃古今常禮。君使之坐,所以示人主尊德樂道也;若不命而請,則異矣。」禮官皆同其議,至今仍之。考試開封舉人,與同院王介爭詈,為監察御史所劾罷。禮院廷試始用策,初,考官呂惠卿列阿
 時者在高等,訐直者反居下。分文覆考,悉反之。又嘗詒安石書,論新法不便。安石怒摭前過,斥通判泰州,以集賢校理、判登聞檢院、戶部判官知曹州。曹為盜區,重法不能止。分文曰:「民不畏死,奈何以死懼之。」至,則治尚寬平,盜亦衰息。為開封府判官,復出為京東轉運使。部吏罷軟不逮者,務全安之。徙知兗、亳二州。吳居厚代為轉運使,能奉行法令,致財賦,乃追坐分文廢弛,黜監衡州鹽倉。



 哲宗初,起知襄州。入為秘書少監,以疾求去,加直龍圖閣、
 知蔡州。於是給事中孫覺、胡宗愈、中書舍人蘇軾、範百祿言:「分文博記能文章,政事侔古循吏,身兼數器,守道不回,宜優賜之告,使留京師。」至蔡數月,召拜中書舍人。請復舊制,建紫微閣於西省。竟以疾不起,年六十七。



 分文所著書百卷,尤邃史學。作《東漢刊誤》,為人所稱。預司馬光修《資治通鑒》,專職漢史。為人疏俊,不修威儀,喜諧謔,數用以招怨悔,終不能改。



 奉世字仲馮,天資簡重,有法度。中進士第。熙寧三年,初
 置樞密院諸房檢詳文字,以太子中允居吏房。



 先是,進奏院每五日具定本報狀,上樞密院,然後傳之四方。而邸吏輒先期報下,或矯為家書,以入郵置。奉世乞革定本,去實封,但以通函騰報。從之。神宗稱其奉職不茍,加集賢校理、檢正中書戶房公事,改刑房,進直史館、國史院編修官。大理治相州獄,詳斷官竇革以白奉世,奉世曰:「君自以法從事,毋庸白。」後蔡確以是文致奉世罪,謫降蔡州糧料院。久之,為吏部員外郎。



 元祐初,歷度支左
 司郎中、起居郎、天章閣待制、樞密都承旨、戶部吏部侍郎、權戶部尚書。七年,拜樞密直學士,簽書院事。哲宗親政,用二內侍為押班,中書舍人呂希純封還之。帝謂有近例,奉世曰:「雖有近例,奈人不可戶曉,顧以率先施行為非耳。」帝為反命。既而章惇當國,奉世乞免去。



 紹聖元年,以端明殿學士知成德軍,改定州。逾年,知成都府。過都入覲,欲述朋黨傾邪之狀。帝將聽其來,曾布曰:「元祐變先朝法,無一當者,奉世有力焉,最為漏網,恐不足
 見。」遂不許。明年,責光祿少卿,分司南京,居郴州。御史中丞邢恕劾奉世合劉摯傾害大臣,附呂大防、蘇轍,遂登政府,再貶隰州團練副使。



 徽宗立,盡還其官職,知定州、大名府、鄆州。崇寧初,再奪職,責居沂、袞,以赦得歸。政和三年,復端明殿學士。薨,年七十三。



 奉世優於吏治,尚安靜,文詞雅贍,最精《漢書》學。常云:「家世唯知事君,內省不愧怍士大夫公論而已。得喪,常理也,譬如寒暑加人,雖善攝生者不能無病,正須安以處之。」



 曾鞏,字子固,建昌南豐人。生而警敏,讀書數百言,脫口輒誦。年十二,試作《六論》,援筆而成,辭甚偉。甫冠,名聞四方。歐陽修見其文,奇之。



 中嘉祐二年進士第。調太平州司法參軍,召編校史館書籍,遷館閣校勘、集賢校理,為實錄檢討官。出通判越州,州舊取酒場錢給募牙前,錢不足,賦諸鄉戶,期七年止;期盡,募者志於多入,猶責賦如初。鞏訪得其狀,立罷之。歲饑,度常平不足贍,而田野之民,不能皆至城邑。諭告屬縣,諷富人自實粟,總十五
 萬石,視常平價稍增以予民。民得從便受粟,不出田里,而食有餘。又貸之種糧,使隨秋賦以償,農事不乏。



 知齊州,其治以疾奸急盜為本。曲堤周氏擁貲雄里中,子高橫縱,賊良民,污婦女,服器上僭,力能動權豪,州縣吏莫敢詰,鞏取置於法。章邱民聚黨村落間,號「霸王社」,椎剽奪囚,無不如志。鞏配三十一人,又屬民為保伍,使幾察其出入,有盜則鳴鼓相援,每發輒得盜。有葛友者,名在捕中,一日,自出首。鞏飲食冠裳之,假以騎從,輦所購金
 帛隨之,誇徇四境。盜聞,多出自首。鞏外視章顯,實欲攜貳其徒,使之不能復合也。自是外戶不閉。河北發民浚河,調及它路,齊當給夫二萬。縣初按籍三丁出夫一,鞏括其隱漏,至於九而取一,省費數倍。又弛無名渡錢,為橋以濟往來。徙傳舍,自長清抵博州,以達於魏,凡省六驛,人皆以為利。徙襄州、洪州。會江西歲大疫,鞏命縣鎮亭傳,悉儲藥待求,軍民不能自養者,來食息官舍,資其食飲衣衾之具,分醫視診,書其全失、多寡為殿最。師征
 安南,所過州為萬人備。他吏暴誅亟斂,民不堪。鞏先期區處猝集,師去,市里不知。加直龍圖閣、知福州。南劍將樂盜廖恩既赦罪出降,餘眾潰復合,陰相結附,旁連數州,尤桀者呼之不至,居人懾恐。鞏以計羅致之,繼自歸者二百輩。福多佛寺,僧利其富饒,爭欲為主守,賕請公行。鞏俾其徒相推擇,識諸籍,以次補之。授帖於府庭,卻其私謝,以絕左右徼求之弊。福州無職田,歲鬻園蔬收其直,自入常三四十萬。鞏曰:「太守與民爭利,可乎?」罷之。後
 至者亦不復取也。



 徙明、亳、滄三州。鞏負才名,久外徒,世頗謂偃蹇不偶。一時後生輩鋒出,鞏視之泊如也。過闕,神宗召見,勞問甚寵,遂留判三班院。上疏議經費,帝曰:「鞏以節用為理財之要,世之言理財者,未有及此。」帝以《三朝》、《兩朝國史》各自為書,將合而為一,加鞏史館修撰,專典之,不以大臣監總,既而不克成。會官制行,拜中書舍人。時自三省百職事,選授一新,除書日至十數,人人舉其職,於訓辭典約而盡。尋掌延安郡王牒奏。故事命
 翰林學士,至是特屬之。甫數月,丁母艱去。又數月而卒,年六十五。



 鞏性孝友,父亡,奉繼母益至,撫四弟、九妹於委廢單弱之中,宦學昏嫁,一出其力。為文章,上下馳騁,愈出而愈工,本原《六經》,斟酌於司馬遷、韓愈,一時工作文詞者,鮮能過也。少與王安石游,安石聲譽未振,鞏導之於歐陽修,及安石得志,遂與之異。神宗嘗問:「安石何如人?」對曰:「安石文學行義,不減揚雄,以吝故不及。」帝曰:「安石輕富貴,何吝也?」曰:「臣所謂吝者,謂其勇於有為,吝
 於改過耳。」帝然之。呂公著嘗告神宗,以鞏為人行義不如政事,政事不如文章,以是不大用云。弟布,自有傳,幼弟肇。



 肇字子開,舉進士,調黃巖簿,用薦為鄭州教授,擢崇文校書、館閣校勘兼國子監直講、同知太常禮院。太常自秦以來,禮文殘缺,先儒各以臆說,無所稽據。肇在職,多所厘正。親祠皇地祗於北郊,蓋自肇發之,異論莫能奪其議。



 兄布以論市易事被責,亦奪肇主判。滯於館下,又
 多希旨窺伺者,眾皆危之,肇恬然無慍。曾公亮薨,肇狀其行,神宗覽而嘉之。遷國史編修官,進吏部郎中,遷右司,為《神宗實錄》檢討。元祐初,擢起居舍人。未幾,為中書舍人。論葉康直知秦州不當,執政訝不先白,御史因攻之。肇求去,范純仁語於朝曰:「若善人不見容,吾輩不可居此矣。」力為之言,乃得釋。



 門下侍郎韓維奏範百祿事,太皇太后以為讒毀,出守鄧。肇言:「維為朝廷辨邪正是非,不可以疑似逐。」不草制。諫議大夫王覿,以論胡宗愈,
 出守潤,肇言:「陛下寄腹心於大臣,寄耳目於臺諫,二者相須,闕一不可。今覿論執政即去之,是愛腹心而塗耳目也。」帝悟,加覿直龍圖閣。



 太皇受冊,詔遵章獻故事,御文德殿。肇言:「天聖初,兩制定議受冊崇政,仁宗特改焉,此蓋一時之制。今帝述仁宗故事,以極崇奉孝敬之誠,可謂至矣。臣竊謂太皇當於此時特下詔揚帝孝敬之誠,而固執謙德,屈從天聖兩制之議,止於崇政,則帝孝愈顯,太皇之德愈尊矣。」坤成節上壽,議令百官班崇政。
 肇又言:「天聖三年,近臣班殿廷,百官止請內東門拜表。至九年,始御會慶。今太皇盛德,不肯自同章獻,宜如三年之制。」並從之。



 四年,春旱,有司猶講春宴。肇同彭汝礪上疏曰:「天菑方作,正君臣側身畏懼之時。乃相與飲食燕樂,恐無以消復天變。」翼日,有旨罷宴。蔡確貶新州,肇先與汝礪相約極論。會除給事中,汝礪獨封還制書,言者謂肇賣友,略不自辨。以寶文閣待制知穎州,徙鄧、齊、陳州、應天府。



 七年,入為吏部侍郎。肇在禮院時,啟親祠
 北郊之議。是歲當郊,肇堅抗前說,既而合祭天地,乃自劾,改刑部。請不已,出知徐州,徙江寧府。帝親政,更用舊臣,數稱肇議禮,趣入對。肇言:「人主雖有自然之聖質,必賴左右前後得人,以為立政之本。宜於此時選忠信端良之士,置諸近班,以參謀議,備顧問。與夫深處法宮,親近NJ御,其損益相去萬萬矣。」貴近惡其語,出知瀛州,與兄布易地。時方治實錄譏訕罪,降為滁州。稍復集賢殿修撰。歷泰州、海州。徽宗即位,復召為中書舍人。



 日食四
 月朔,當降詔求言。肇具述帝旨,詔下,投匭者如織。章惇惡之,欲因事去肇,帝不聽。元祐臣僚被譴者,咸以赦恩甄敘。肇請人並錄死者,作訓詞,哀厚惻怛,讀者為之感愴。遷翰林學士兼侍讀。諫官陳瓘、給事中龔原以言得罪,無敢救,肇極力論解。時論者謂元祐、紹聖,均為有失,兄布傳帝命,使肇作詔諭天下。肇見帝言:「陛下思建皇極,以消弭朋黨,須先分別君子小人,賞善罰惡,不可偏廢。」開說備至。已而詔從中出。布之拜相,肇適當制,國朝學
 士弟草兄制,唯韓維與肇,為衣冠榮。建中靖國元年,太史奏日又當食四月。肇請對言:「比歲日食正陽,咎異章著。陛下簡儉清凈之化,或衰於前;聲色服玩之好,或萌於心;忠邪賢不肖,或有未辨;賞慶刑威,或有未當。左右阿諛,壅蔽矯舉,民冤失職,鬱不得伸。此宜反復循省,痛自克責,以塞天變。」言發涕下,帝悚然順納。



 兄布在相位,引故事避禁職,拜龍圖閣學士、提舉中太一宮。未幾,出知陳州,歷太原、應天府、揚定二州。崇寧初,落職,謫知和
 州,徙岳州,繼貶濮州團練副使,安置汀州。四年,歸潤而卒,年六十一。



 自熙寧以來四十年,大臣更用事,邪正相軋,黨論屢起,肇身更其間,數不合。兄布與韓忠彥並相,日夕傾危之。肇既居外,移書告之曰:「兄方得君,當引用善人,翊正道,以杜惇、卞復起之萌。而數月以來,所謂端人吉士,繼跡去朝,所進以為輔佐、侍從、臺諫,往往皆前日事惇、卞者。一旦勢異今日,必首引之以為固位計,思之可為慟哭。比來主意已移,小人道長。進則必論元祐
 人於帝前,退則盡排元祐者於要路。異時惇、卞縱未至,一蔡京足以兼二人,可不深慮。」布不能從。未幾,京得政,布與肇俱不免。



 肇天資仁厚,而容貌端嚴。自少力學,博覽經傳,為文溫潤有法。更十一州,類多善政。紹興初,謚曰文昭。子統,至左諫議大夫。



 論曰:劉敞博學雄文,鄰於邃古,其為考功,仁宗賜夏竦謚,上疏爭之,以為人主不可侵臣下之官;及奉詔定樂,中貴預列,又諫曰:「臣懼為袁盎所笑。」此豈事君為容悅
 者哉。分文雖疏雋,文埒於敞。奉世克肖,世稱「三劉」。曾鞏立言於歐陽修、王安石間,紆徐而不煩,簡奧而不晦,卓然自成一家,可謂難矣。肇以儒者而有能吏之才。宋之中葉,文學法理,咸精其能,若劉氏、曾氏之家學,蓋有兩漢之風焉。



\end{pinyinscope}