\article{列傳第一百一十九}

\begin{pinyinscope}

 宗澤趙鼎



 宗澤。字汝霖,婺州義烏人。母劉,夢天大雷電,光燭其身,翌日而澤生。澤自幼豪爽有大志,登元祐六年進士第。廷對極陳時弊,考官惡直,置末甲。



 調大名館陶尉。呂惠
 卿帥鄜延,檄澤與邑令視河埽,檄至,澤適喪長子,奉檄遽行。惠卿聞之,曰:「可謂國爾忘家者。」適朝廷大開御河,時方隆冬,役夫殭僕於道,中使督之急。澤曰浚河細事,乃上書其帥曰:「時方凝寒,徒苦民而功未易集,少需之,至初春可不擾而辦。」卒用其言上聞,從之。惠卿闢為屬,辭。



 調衢州龍游令。民未知學,澤為建庠序,設師儒,講論經術,風俗一變,自此擢科者相繼。調晉州趙城令。下車,請升縣為軍,書聞,不盡如所請。澤曰:「承平時固無慮,它
 日有警,當知吾言矣。」知萊州掖縣。部使者得旨市牛黃,澤報曰:「方時疫癘,牛飲其毒則結為黃。今和氣橫流,牛安得黃?」使者怒,欲劾邑官。澤曰:「此澤意也。」獨銜以聞。通判登州。境內官田數百頃,皆不毛之地,歲輸萬餘緡,率橫取於民,澤奏免之。



 朝廷遣使由登州結女真,盟海上,謀夾攻契丹,澤語所親曰:「天下自是多事矣。」退居東陽,結廬山谷間。靖康元年,中丞陳過庭等列薦,假宗正少卿,充和議使。澤曰:「是行不生還矣。」或問之,澤曰:「敵能悔
 過退師固善,否則安能屈節北庭以辱君命乎。」議者謂澤剛方不屈,恐害和議,上不遣,命知磁州。



 時太原失守,官兩河者率托故不行。澤曰:「食祿而避難,不可也。」即日單騎就道,從嬴卒十餘人。磁經敵騎蹂躪之餘,人民逃徙,帑廩枵然。澤至,繕城壁,浚湟池,治器械,募義勇,始為固守不移之計。上言:「邢、洺、磁、趙、相五州各蓄精兵二萬人,敵攻一郡則四郡皆應,是一郡之兵常有十萬人。」上嘉之,除河北義兵都總管。金人破真定,引兵南取慶源,
 自李固渡渡河,恐澤兵躡其後,遣數千騎直扣磁州城。澤擐甲登城,令壯士以神臂弓射走之,開門縱擊,斬首數百級。所獲羊馬金帛,悉以賞軍士。



 康王再使金,行至磁,澤迎謁曰:「肅王一去不反,今敵又詭辭以致大王,願勿行。」王遂回相州。有詔以澤為副元帥,從王起兵入援。澤言宜急會兵李固渡,斷敵歸路,眾不從,乃自將兵趨渡,道遇北兵,遣秦光弼、張德夾擊,大破之。金人既敗,乃留兵分屯。澤遣壯士夜搗其軍,破三十餘砦。



 時康王開
 大元帥府,檄兵會大名。澤履冰渡河見王,謂京城受圍日久,入援不可緩。會簽書樞密院事曹輔繼蠟封欽宗手詔,至自京師,言和議可成。澤曰:「金人狡譎,是欲款我師爾。君父之望入援,何啻饑渴,宜急引軍直趨澶淵,次第進壘,以解京城之圍。萬一敵有異謀,則吾兵已在城下。」汪伯彥等難之,勸王遣澤先行,自是澤不得預府中謀議矣。



 二年正月,澤至開德,十三戰皆捷,以書勸王檄諸道兵會京城。又移書北道總管趙野、河東北路宣撫
 範訥、知興仁府曾楙合兵入援。三人皆以澤為狂,不答。澤以孤軍進,都統陳淬言敵方熾,未可輕舉。澤怒,欲斬之,諸將乞貸淬,使得效死。澤命淬進兵,遇金人,敗之。金人攻開德,澤遣孔彥威與戰,又敗之。澤度金人必犯濮,先遣三千騎往援,金人果至,敗之。金人復向開德,權邦彥、孔彥威合兵夾擊,又大敗之。



 澤兵進至衛南,度將孤兵寡,不深入不能成功。先驅雲前有敵營,澤揮眾直前與戰,敗之。轉戰而東,敵益生兵至,王孝忠戰死,前後皆
 敵壘。澤下令曰:「今日進退等死,不可不從死中求生。」士卒知必死,無不一當百,斬首數千級。金人大敗,退卻數十餘里。澤計敵眾十倍於我,今一戰而卻,勢必復來,使悉其鐵騎夜襲吾軍,則危矣。乃暮徙其軍。金人夜至,得空營,大驚,自是憚澤,不敢復出兵。澤出其不意,遣兵過大河襲擊,敗之。王承制以澤為徽猷閣待制。



 時金人逼二帝北行,澤聞,即提軍趨滑,走黎陽,至大名,欲徑渡河,據金人歸路邀還二帝,而勤王之兵卒無一至者。又聞
 張邦昌僭位,欲先行誅討。會得大元帥府書,約移師近都,按甲觀變。澤復書於王曰:「人臣豈有服赭袍、張紅蓋、御正殿者乎?自古奸臣皆外為恭順而中藏禍心,未有竊據寶位、改元肆赦、惡狀昭著若邦昌者。今二聖、諸王悉渡河而北,惟大王在濟,天意可知,宜亟行天討,興復社稷。」且言:「邦昌偽赦,或啟奸雄之意,望遣使分諭諸路,以定民心。」又上書言:「今天下所屬望者在於大王,大王行之得其道,則有心慰天下之心。所謂道者,近剛正而
 遠柔邪,納諫諍而拒諛佞,尚恭儉而抑驕侈,體憂勤而忘逸樂,進公實而退私偽。」因累表勸進。王即帝位於南京,澤入見,涕泗交頤,陳興復大計。時與李綱同入對,相見論國事,慷慨流涕,綱奇之。上欲留澤,潛善等沮之。除龍圖閣學士、知襄陽府。



 時金人有割地之議,澤上疏曰:「天下者,太祖、太宗之天下,陛下當兢兢業業,思傳之萬世,奈何遽議割河之東、西,又議割陜之蒲、解乎。自金人再至,朝廷未嘗命一將、出一師,但聞奸邪之臣,朝進一
 言以告和,幕入一說以乞盟,終致二聖北遷,宗社蒙恥。臣意陛下赫然震怒,大明黜陟,以再造王室。今即位四十日矣,未聞有大號令,但見刑部指揮云『不得□謄播赦文於河之東、西,陜之蒲、解』者,是褫天下忠義之氣,而自絕其民也。臣雖駑怯,當躬冒矢石為諸將先,得捐軀報國恩足矣。」上覽其言壯之。改知青州,時年六十九矣。



 開封尹闕,李綱言綏復舊都,非澤不可。尋徙知開封府。時敵騎留屯河上,金鼓之聲,日夕相聞,而京城樓櫓盡廢,
 兵民雜居,盜賊縱橫,人情忷□。澤威望素著,既至,首捕誅舍賊者數人。下令曰:「為盜者,贓無輕重,並從軍法。」由是盜賊屏息,民賴以安。



 王善者,河東巨寇也。擁眾七十萬、車萬乘,欲據京城。澤單騎馳至善營,泣謂之曰:「朝廷當危難之時,使有如公一二輩,豈復有敵患乎。今日乃汝立功之秋,不可失也。」善感泣曰:「敢不效力。」遂解甲降。時楊進號沒角牛,兵三十萬,王再興、李貴、王大郎等各擁眾數萬,往來京西、淮南、河南、北,侵掠為患。澤遣人諭
 以禍福,悉招降之。上疏請上還京。俄有詔:荊、襄、江、淮悉備巡幸。澤上疏言:「開封物價市肆,漸同平時。將士、農民、商旅、士大夫之懷忠義者,莫不願陛下亟歸京師,以慰人心。其唱為異議者,非為陛下忠謀,不過如張邦昌輩,陰與金人為地爾。」除延康殿學士、京城留守、兼開封尹。



 時金遣人以使偽楚為名,至開封府,澤曰:「此名為使,而實覘我也。」拘其人,乞斬之。有詔所拘金使延置別館,澤曰:「國家承平二百年,不識兵革,以敵國誕謾為可憑信,
 恬不置疑。不惟不嚴攻討之計,其有實欲賈勇思敵所愾之人,士大夫不以為狂,則以為妄,致有前日之禍。張邦昌、耿南仲輩所為,陛下所親見也。今金人假使偽楚,來覘虛實,臣愚乞斬之,以破其奸。而陛下惑於人言,令遷置別館,優加待遇,臣愚不敢奉詔,以彰國弱。」上乃親札諭澤,竟縱遣之。言者附潛善意,皆以澤拘留金使為非。尚書左丞許景衡抗疏力辨,且謂:「澤之為尹,威名政績,卓然過人,今之縉紳,未見其比。乞厚加任使,以成禦
 敵治民之功。」



 真定、懷、衛間,敵兵甚盛,方密修戰具為入攻之計,而將相恬不為慮,不修武備,澤以為憂。乃渡河約諸將共議事宜,以圖收復,而於京城四壁,各置使以領招集之兵。又據形勢立堅壁二十四所於城外,沿河鱗次為連珠砦,連結河東、河北山水砦忠義民兵,於是陜西、京東西諸路人馬咸願聽澤節制。有詔如淮甸。澤上表諫,不報。



 秉義郎岳飛犯法將刑,澤一見奇之,曰:「此將材也。」會金人攻汜水,澤以五百騎授飛,使立功贖罪。
 飛大敗金人而還,遂升飛為統制,飛由是知名。



 澤視師河北還,上疏言:「陛下尚留南都,道路籍籍,咸以為陛下舍宗廟朝廷,使社稷無依,生靈失所仰戴。陛下宜亟回汴京,以慰元元之心。」不報。復抗疏言:「國家結好金人,欲以息民,卒之劫掠侵欺,靡所不至,是守和議果不足以息民也。當時固有阿意順旨以叨富貴者,亦有不相詭隨以獲罪戾者。陛下觀之,昔富貴者為是乎?獲罪戾者為是乎?今之言遷幸者,猶前之言和議為可行者也;今
 之言不可遷者,猶前日之言和議不可行者也。惟陛下熟思而審用之。且京師二百年積累之基業,陛下奈何輕棄以遺敵國乎。」



 詔遣官迎奉六宮往金陵,澤上疏曰:「京師,天下腹心也。兩河雖未敉寧,特一手臂之不信爾。今遽欲去之,非惟一臂之弗廖,且並與腹心而棄之矣。昔景德間,契丹寇澶淵,王欽若江南人,即勸幸金陵,陳堯叟蜀人,即勸幸成都,惟寇準毅然請親征,卒用成功。臣何敢望寇準,然不敢不以章聖望陛下。」又條上五事,
 其一言黃潛善、汪伯彥贊南幸之非。澤前後建議,經從三省、樞密院,輒為潛善等所抑,每見澤奏疏,皆笑以為狂。



 金將兀朮渡河,謀攻汴京。諸將請先斷河梁,嚴兵自固,澤笑曰:「去冬,金騎直來,正坐斷河梁耳。」乃命部將劉衍趨滑、劉達趨鄭,以分敵勢,戒諸將極力保護河梁,以俟大兵之集。金人聞之,夜斷河梁遁去。二年,金人自鄭抵白沙,去汴京密邇,都人震恐。僚屬入問計,澤方對客圍棋,笑曰:「何事張皇,劉衍等在外必能禦敵。」乃選精銳
 數千,使繞出敵後,伏其歸路。金人方與衍戰,伏兵起,前後夾擊之,金人果敗。



 金將黏罕據西京,與澤相持。澤遣部將李景良、閻中立、郭俊民領兵趨鄭,遇敵大戰,中立死之,俊民降,景良遁去。澤捕得景良,謂曰:「不勝,罪可恕;私自逃,是無主將也。」斬其首以徇。既而俊民與金將史姓者及燕人何仲祖等持書來招澤,澤數俊民曰:「汝失利死,尚為忠義鬼,今反為金人持書相誘,何面目見我乎。」斬之,謂史曰:「我受此土,有死而已。汝為人將,不能以
 死敵我,乃欲以兒女子語誘我乎。」亦斬之。謂仲祖脅從,貸之。劉衍還,金人復入滑,部將張摠請往救,澤選兵五千付之,戒毋輕戰以需援。摠至滑迎戰,敵騎十倍,諸將請少避其鋒,摠曰:「避而偷生,何面目見宗公。」力戰死之。澤聞摠急,遣王宣領騎五千救之。摠死二日,宣始至,與金人大戰,破走之。澤迎摠喪歸,恤其家,以宣權知滑州,金人自是不復犯東京。



 山東盜起,執政謂其多以義師為名,請下令止勤王。澤疏曰:「自敵圍京城,忠義之士憤
 懣爭奮,廣之東西、湖之南北、福建、江、淮,越數千里,爭先勤王。當時大臣無遠識大略,不能撫而用之,使之饑餓困窮,弱者填溝壑,強者為盜賊。此非勤王者之罪,乃一時措置乖謬所致耳。今河東、西不從敵國而保山砦者,不知其幾;諸處節義之夫,自黥其面而爭先救駕者,復不知其幾。此詔一出,臣恐草澤之士一旦解體,倉卒有急,誰復有願忠效義之心哉。」



 王策者,本遼酋,為金將,往來河上。澤擒之,解其縛坐堂上,為言:「契丹本宋兄弟之
 國,今女真辱吾主,又滅而國,義當協謀雪恥。」策感泣,願效死。澤因問敵國虛實,盡得其詳,遂決大舉之計,召諸將謂曰:「汝等有忠義心,當協謀剿敵,期還二聖,以立大功。」言訖泣下,諸將皆泣聽命。金人戰不利,悉引兵去。



 澤疏諫南幸,言:「臣為陛下保護京城,自去年秋冬至於今春,又三月矣。陛下不早回京城,則天下之民何所依戴。」除資政殿學士。又遣子穎詣行闕上疏曰:「天下之事,見幾而為,待時而動,則事無不成。今收復伊、洛而金酋渡
 河,捍蔽滑臺而敵國屢敗,河東、河北山砦義民,引領舉踵,日望官兵之至。以幾以時而言之,中興之兆可見,而金人滅亡之期可必,在陛下見幾乘時而已。」又言:「昔楚人城郢,史氏鄙之。今聞有旨於儀真教習水戰,是規規為偏霸之謀,非可鄙之甚者乎?傳聞四方,必謂中原不守,遂為江寧控扼之計耳。」



 先是,澤去磁,以州事付兵馬鈐轄李侃,統制趙世隆殺之。至是,世隆及弟與興以兵三萬來歸,眾懼其變,澤曰:「世隆本吾一校爾,何能為。」世
 隆至,責之曰:「河北陷沒,吾宋法令與上下之分亦陷沒邪?」命斬之。時世興佩刀侍側,眾兵露刃庭下,澤徐謂世興曰:「汝兄誅,汝能奮志立功,足以雪恥。」世興感泣。金人攻滑州,澤遣世興往救,世興至,掩其不備,敗之。



 澤威聲日著,北方聞其名,常尊憚之,對南人言,必曰宗爺爺。



 澤疏言:「丁進數十萬眾願守護京城,李成願扈從還闕,即渡河剿敵,楊進等兵百萬,亦願渡河,同致死力。臣聞『多助之至,天下順之』。陛下及此時還京,則眾心翕然,何敵
 國之足憂乎?」又奏言:「聖人愛其親以及人之親,所以教人孝;敬其兄以及人之兄,所以教人弟。陛下當與忠臣義士合謀肆討,迎復二聖。今上皇所御龍德宮儼然如舊,惟淵聖皇帝未有宮室。望改修寶菉宮以為迎奉之所,使天下知孝於父、弟於兄,是以身教也。」上乃降詔擇日還京。



 澤前後請上還京二十餘奏,每為潛善等所抑,憂憤成疾,疽發於背。諸將入問疾,澤矍然曰:「吾以二帝蒙塵,積憤至此。汝等能殲敵,則我死無恨。」眾皆流涕曰:「
 敢不盡力!」諸將出,澤嘆曰:「『出師未捷身先死,長使英雄淚滿襟。』」翌日,風雨晝晦。澤無一語及家事,但連呼「過河」者三而薨。都人號慟。遺表猶贊上還京。贈觀文殿學士、通議大夫,謚忠簡。



 澤質直好義,親故貧者多依以為活,而自奉甚薄。常曰:「君父側身嘗膽,臣子乃安居美食邪!」始,澤詔集群盜,聚兵儲糧,結諸路義兵,連燕、趙豪傑,自謂渡河克復可指日冀。有志弗就,識者恨之。



 子穎,居戎幕,素得士心。澤薨數日,將士去者十五,都人請以穎繼父
 任。會朝廷已命杜充留守,乃以穎為判官。充反澤所為,頗失人心,穎屢爭之,不從,乃請持服歸。自是豪傑不為用,群聚城下者復去為盜,而中原不守矣。穎官終兵部郎中。



 趙鼎,字符鎮,解州聞喜人。生四歲而孤,母樊教之,通經史百家之書。登崇寧五年進士第,對策斥章惇誤國。累官為河南洛陽令,宰相吳敏和其能,擢為開封士曹。



 金人陷太原,朝廷議割三鎮地,鼎曰:「祖宗之地不可以與
 人,何庸議?」已而京師失守,二帝北行。金人議立張邦昌,鼎與胡寅、張浚逃太學中,不書議狀。



 高宗即位,除權戶部員外郎。知樞密院張浚薦之,除司勛郎官。上幸建康,詔條具防秋事宜,鼎言:「宜以六宮所止為行宮,車駕所止為行在,擇精兵以備儀衛,其餘兵將分布江、淮,使敵莫測巡幸之定所。」上納之。



 久雨,詔求闕政。鼎言:「自熙寧間王安石用事,變祖宗之法,而民始病。假闢國之謀,造生邊患;興理財之政,窮困民力;設虛無之學,敗壞人才。
 至崇寧初,蔡京托紹述之名,盡祖安石之政。凡今日之患始於安石,成於蔡京。今安石猶配享廟廷,而京之黨未除,時政之闕無大於此。」上為罷安石配享。擢右司諫,旋遷殿中侍御史。



 劉光世部將王德擅殺韓世忠之將,而世忠亦率部曲奪建康守府廨。鼎言:「德總兵在外,專殺無忌,此而不治,孰不可為?」命鼎鞫德。鼎又請下詔切責世忠,而指取其將吏付有司治罪,諸將肅然。上曰:「肅宗興靈武得一李勉,朝廷始尊。今朕得卿,無愧昔人矣。」
 中丞範宗尹言,故事無自司諫遷殿中者,上曰:「鼎在言路極舉職,所言四十事,已施行三十有六。」遂遷侍御史。



 北兵至江上,上幸會稽,召臺諫議去留,鼎陳戰、守、避三策,拜御史中丞。請督王□燮進軍宣州,周望分軍出廣德,劉光世渡江駐蘄、黃,為邀擊之計。又言:「經營中原當自關中始,經營關中當自蜀始,欲幸蜀當自荊、襄始。吳、越介在一隅,非進取中原之地。荊、襄左顧川、陜,右控湖湘,而下瞰京、洛,三國所必爭,宜以公安為行闕,而屯重兵於
 襄陽,運江、浙之粟以資川、陜之兵,經營大業,計無出此。」



 韓世忠敗金人於黃天蕩,宰相呂頤浩請上幸浙西,下詔親征,鼎以為不可輕舉。頤浩惡其異己,改鼎翰林學士,鼎不拜,改吏部尚書,又不拜,言:「陛下有聽納之誠,而宰相陳拒諫之說;陛下有眷待臺臣之意,而宰相挾挫沮言官之威。」堅臥不出,疏頤浩過失凡千言。上罷頤浩,詔鼎復為中丞,謂鼎曰:「朕每聞前朝忠諫之臣,恨不之識,今於卿見之。」除端明殿學士、簽書樞密院事。



 金人攻
 楚州,鼎奏遣張俊往援之。俊不行,山陽遂陷,金人留淮上,範宗尹奏敵未必能再渡,鼎曰:「勿恃其不來,恃吾有以待之。三省常以敵退為陛下援人才、修政事,密院常虞敵至為陛下申軍律、治甲兵,即兩得之。」上曰:「卿等如此,朕復何憂。」鼎以楚州之失,上章丐去。會辛企宗除節度使,鼎言企宗非軍功,忤旨,出奉祠,除知平江府,尋改知建康,又移知洪州。



 京西招撫使李橫欲用兵復東京,鼎言:「橫烏合之眾,不能當敵,恐遂失襄陽。」已而橫戰不
 利走,襄陽竟陷。召拜參知政事。宰相朱勝非言:「襄陽國之上流,不可不急取。」上問:「岳飛可使否?」鼎曰:「知上流利害無如飛者。」簽樞徐俯不以為然。飛出師竟復襄陽。



 鼎乞令韓世忠屯泗上,劉光世出陳、蔡。光世請入奏,俯欲許之,鼎不可。偽齊宿遷令來歸,俯欲斬送劉豫,鼎復爭之。俯積不能平,乃求去。朱勝非兼知樞密院,言者謂當國者不知兵,乞令參政通知。由是為勝非所忌。除鼎知樞密院、川陜宣撫使,鼎辭以非才。上曰:「四川全盛半天
 下之地,盡以付卿,黜陟專之可也。」時吳玠為宣撫副使,鼎奏言:「臣與玠同事,或節制之耶?」上乃改鼎都督川、陜諸軍事。



 鼎所條奏,勝非多沮抑之。鼎上疏言:「頃張浚出使川、陜,國勢百倍於今。浚有補天浴日之功,陛下有礪山帶河之誓,君臣相信,古今無二,而終致物議,以被竄逐。今臣無浚之功而當其任,遠去朝廷,其能免於紛紛乎?」又言:「臣所請兵不滿數千,半皆老弱,所繼金帛至微,薦舉之人除命甫下,彈墨已行。臣日侍宸衷,所陳已艱
 難,況在萬里之外乎?」時人士皆惜其去,臺諫有留行者。會邊報沓至,鼎每陳用兵大計,及朝辭,上曰:「卿豈可遠去,當遂相卿。」九月,拜尚書右僕射、同中書門下平章事兼知樞密院事。制下,朝士相慶。



 時劉豫子麟與金人合兵大入,舉朝震恐。鼎論戰御之計,諸將各異議,獨張俊以為當進討,鼎是其言。有勸上他幸者,鼎曰:「戰而不捷,去未晚也。」上亦曰:「朕當親總六師,臨江決戰。」鼎喜曰:「累年退怯,敵志益驕,今聖斷親征,成功可必。」於是詔張俊
 以所部援韓世忠,而命劉光世移軍建康,且促世忠進兵。世忠至揚州,大破金人於大儀鎮。方警報交馳,劉光世遣人諷鼎曰:「相公自入蜀,何事為他人任患。」世忠亦謂人曰:「趙丞相真敢為者。」鼎聞之,恐上意中變,乘間言:「陛下養兵十年,用之正在今日。若少加退沮,即人心渙散,長江之險不可復恃矣。」及捷音日至,車駕至平江,下詔聲逆豫之罪,欲自將渡江決戰。鼎曰:「敵之遠來,利於速戰,遽與爭鋒,非策也。且豫猶遣其子,豈可煩至尊耶?」
 帝為止不行。未幾,簽書樞密院事胡松年自江上還,雲北兵大集,然後知鼎之有先見也。



 張浚久廢,鼎言浚可大任,乃召除知樞密院,命浚往江上視師。時敵兵久駐淮南,知南兵有備,漸謀北歸。鼎曰:「金人無能為矣。」命諸將邀諸淮,連敗之,金人遁去。上謂鼎曰:「近將士致勇爭先,諸路守臣亦翕然自效,乃朕用卿之力也。」鼎謝曰:「皆出聖斷,臣何力之有焉。」或問鼎曰:「金人傾國來攻,眾皆忷懼,公獨言不足畏,何耶?」鼎曰:「敵眾雖盛,然以豫邀而
 來,非其本心,戰必不力,以是知其不足畏也。」上嘗語張浚曰:「趙鼎真宰相,天使佐朕中興,可謂宗社之幸也。」鼎奏金人遁歸,尤當博採群言,為善後之計。於是詔呂頤浩等議攻戰備御、措置綏懷之方。



 五年,上還臨安,制以鼎守左僕射知樞密院事、張浚守右僕射兼知樞密院事,都督諸路軍馬。鼎以政事先後及人才所當召用者,條而置之座右,次第奏行之。制以貴州防禦使瑗為保慶軍節度使,封建國公,於行宮門外建資善堂。鼎薦範
 沖為翊善、朱震為贊讀,朝論謂二人極天下之選。



 建炎初,嘗下詔以奸臣誣蔑宣仁保祐之功,命史院刊修,未及行,朱勝非為相,上諭之曰:「神宗、哲宗兩朝史事多失實,非所以傳信後世,宜召範沖刊定。」勝非言:「《神宗史》增多王安石《日錄》,《哲宗史》經京、卞之手,議論多不正,命官刪修,誠足以彰二帝盛美。」會勝非去位,鼎以宰相監修二史,是非各得其正。上親書「忠正德文」四字賜鼎,又以御書《尚書》一帙賜之,曰:「《書》所載君臣相戒飭之言,所以
 賜卿,欲共由斯道。」鼎上疏謝。



 劉豫遣子麟、猊分路入寇,時張浚屯盱眙,楊沂中屯泗,韓世忠屯楚,岳飛駐鄂,劉光世駐廬,沿江上下無兵,上與鼎以為憂。鼎移書浚,欲令俊與沂中合兵剿敵。光世乞舍廬還太平,又乞退保採石,鼎奏曰:「豫逆賊也,官軍與豫戰而不能勝,或更退守,何以立國?今賊已渡淮,當亟遣張俊合光世之軍盡掃淮南之寇,然後議去留。」上善其策,詔二將進兵。俊軍至藕塘與猊戰,大破之。鼎命沂中趨合肥以會光世,光
 世已棄廬回江北。浚以書告鼎,鼎白上詔浚:有不用命者,聽以軍法從事。光世大駭,復進至肥河與麟戰,破之。麟、猊拔柵遁去。



 浚在江上,嘗遣其屬呂祉入奏事,所言誇大,鼎每抑之。上謂鼎曰:「他日張浚與卿不和,必呂祉也。」後浚因論事,語意微侵鼎,鼎言:「臣初與浚如兄弟,因呂祉離間,遂爾睽異。今浚成功,當使展盡底蘊,浚當留,臣當去。」上曰:「俟浚歸議之。」浚嘗奏乞幸建康,而鼎與折彥質請回蹕臨安。暨浚還,乞乘勝攻河南,且罷劉光世
 軍政。鼎言:「擒豫固易耳,然得河南,能保金人不內侵乎?光世累世為將,無故而罷之,恐人心不安。」浚滋不悅。鼎以觀文殿大學士知紹興府。



 七年,上幸建康,罷劉光世,以王德為都統制,酈瓊副之,並聽參謀、兵部尚書呂祉節度制。瓊與德有宿怨,訴於祉,不得直,執祉以全軍降偽齊。浚引咎去位,乃以萬壽觀使兼侍讀召鼎,入對,拜尚書左僕射、同中書門下平章事兼樞密使,進四官。上言:「淮西之報初至,執政奏事皆失措,惟朕不為動。」鼎曰:「今
 見諸將,尤須靜以待之,不然益增其驕蹇之心。」臺諫交論淮西無備,鼎曰:「行朝擁兵十萬,敵騎直來,自足抗之,設有他虞,鼎身任其責。」淮西迄無驚。



 鼎嘗乞降詔安撫淮西,上曰:「俟行遣張浚,朕當下罪己之詔。」鼎言:「浚已落職。」上曰:「浚罪當遠竄。」鼎奏:「浚母老,且有勤王功。」上曰:「功過自不相掩。」已而內批出,浚謫置嶺南,鼎留不下。詰旦,經同列救解,上怒殊未釋,鼎力懇曰:「浚罪不過片策耳。凡人計慮,豈不欲萬全,儻因一失,便置之死地,後有奇
 謀秘計,誰復敢言者。此事自關朝廷,非獨私浚也。」上意乃解,遂以散官分司,居永州。



 鼎既再相,或議其無所施設,鼎聞之曰:「今日之事如人患羸,當靜以養之。若復加攻砭,必傷元氣矣。」金人廢劉豫,鼎遣間招河南守將,壽、亳、陳、蔡之間,往往舉城或率部曲來歸,得精兵萬餘,馬數千。知廬州劉錡亦奏言:「淮北歸正者不絕,度今歲可得四五萬。」上喜曰:「朕常慮江、池數百里備御空虛,今得此軍可無患矣。」



 金人遣使議和,朝論以為不可信,上怒。鼎
 曰:「陛下於金人有不共戴天之讎,今屈己請和,不憚為之者,以梓宮及母後耳。群臣憤懣之辭,出於愛君,不可以為罪。陛下宜諭之曰:『講和非吾意,以親故,不得已為之。但得梓宮及母後還,敵雖渝盟,吾無憾焉。』」上從其言,群議遂息。



 潘良貴以向子諲奏事久,叱之退。上欲抵良貴罪,常同為之辨,欲並逐同。鼎奏:「子諲雖無罪,而同與良貴不宜逐。」二人竟出。給事中張致遠謂不應以一子諲出二佳士,不書黃,上怒,顧鼎曰:「固知致遠必繳駁。」鼎
 問:「何也?」上曰:「與諸人善。」蓋已有先入之言,由是不樂於鼎矣。秦檜繼留身奏事,既出,鼎問:「帝何言?」檜曰:「上無他,恐丞相不樂耳。」御筆和州防禦使璩除節鉞,封國公。鼎奏:「建國雖未正名,天下皆知陛下有子,社謖大計也。在今禮數不得不異,所以系人心不使之二三而惑也。」上曰:「姑徐之。」檜後留身,不知所云。



 鼎嘗闢和議,與檜意不合,及鼎以爭璩封國事拂上意,檜乘間擠鼎,又薦蕭振為侍御史。振本鼎所引,及入臺,劾參知政事劉大中罷
 之。鼎曰:「振意不在大中也。」振亦謂人曰:「趙丞相不待論,當自為去就。」會殿中侍御史張戒論給事中勾濤,濤言:「戒之擊臣,乃趙鼎意。」因詆鼎結臺諫及諸將。上聞益疑,鼎引疾求免,言:「大中持正論,為章惇、蔡京之黨所嫉。臣議論出處與大中同,大中去,臣何可留?」乃以忠武節度使出知紹興府,尋加檢校少傅,改奉國軍節度使。檜率執政往餞其行,鼎不為禮,一揖而去,檜益憾之。



 鼎既去,王庶入對,上謂庶曰:「趙鼎兩為相,於國有大功,再贊親
 征皆能決勝,又鎮撫建康,回鑾無患,他人所不及也。」先是,王倫使金,從鼎受使指。問禮數,則答以君臣之分已定;問地界,則答以大河為界。二者從事之大者,或不從則已。倫受命而行。至是,倫與金使俱來,以撫諭江南為名,上嘆息謂庶曰:「使五日前得此報,趙鼎豈可去耶?」



 初,車駕還臨安,內侍移竹栽入內,鼎見,責之曰:「艮岳花石之擾,皆出汝曹,今欲蹈前轍耶?」因奏其事,上改容謝之。有戶部官進錢入宮者,鼎召至相府切責之。翌日,問上
 曰:「某人獻錢耶?」上曰:「朕求之也。」鼎奏:「某人不當獻,陛下不當求。」遂出其人與郡。



 鼎嘗薦胡寅、魏矼、晏敦復、潘良貴、呂本中、張致遠等數十人分布朝列。暨再相,奏曰:「今清議所與,如劉大中、胡寅、呂本中、常同、林季仲之流,陛下能用之乎?妒賢長惡,如趙霈、胡世將、周秘、陳公輔之徒,陛下能去之乎?」上為徙世將,而公輔等尋補外。上嘗中批二人付廟堂升擢。鼎奏:「疏遠小臣,陛下何由得其姓名?」上謂:「常同實稱之。」鼎曰:「同知其賢,何不露章薦引?」



 始,浚薦秦檜可與共大事,鼎再相亦以為言。然檜機阱深險,外和而中異。浚初求去,有旨召鼎。鼎至越丐祠,檜惡其逼己,徙知泉州,又諷謝祖信論鼎嘗受張邦昌偽命,遂奪節。御史中丞王次翁論鼎治郡廢馳,命提舉洞霄宮。鼎自泉州歸,復上書言時政,檜忌其復用,諷次翁又論其嘗受偽命,乾沒都督府錢十七萬緡,謫官居興化軍。論者猶不已,移漳州,又責清遠軍節度副使,潮州安置。



 在潮五年,杜門謝客,時事不掛口,有問者,但引咎
 而已。中丞詹大方誣其受賄,屬潮守放編置人移吉陽軍,鼎謝表曰:「白首何歸,悵餘生之無幾,丹心未泯,誓九死以不移。」檜見之曰:「此老倔強猶昔。」



 在吉陽三年,潛居深處,門人故吏皆不敢通問,惟廣西帥張宗元時饋醪米。檜知之,令本軍月具存亡申。鼎遣人語其子汾曰:「檜必欲殺我。我死,汝曹無患;不爾,禍及一家矣。」先得疾,自書墓中石,記鄉里及除拜歲月。至是,書銘旌云:「身騎箕尾歸天上,氣作山河壯本朝。」遺言屬其子乞歸葬,遂不
 食而死,時紹興十七年也,天下聞而悲之。明年,得旨歸葬。孝宗即位,謚忠簡,贈太傅,追封豐國公。高宗祔廟,以鼎配享廟庭,擢用其孫十有二人。



 鼎為文渾然天成,凡高宗處分軍國機事,多其視草,有擬奏表疏、雜詩文二百餘篇,號《得全集》,行於世。論中興賢相,以鼎為稱首云。



 論曰:夫謀國用兵之道,有及時乘銳而可以立功者,有養威持重而後能有為者,二者之設施不同,其為忠一而已。方金人逼二帝北行,宗社失主,宗澤一呼,而河北
 義旅數十萬眾若響之赴聲,實由澤之忠忱義氣有以風動之,抑斯民目睹君父之陷於塗淖,孰無憤激之心哉。使當其時澤得勇往直前,無或齟齬牽制之,則反二帝,復舊都,特一指顧間耳。黃潛善、汪伯彥嫉能而惎功,使澤不得信其志,發憤而薨,豈不悲哉!



 及趙鼎為相,則南北之勢成矣。兩敵之相持,非有灼然可乘之釁,則養吾力以俟時,否則,徒取危困之辱。故鼎之為國,專以固本為先,根本固而後敵可圖、讎可復,此鼎之心也。惜乎
 一見忌於秦檜,斥逐遠徙,卒繼其志而亡,君子所尤痛心也。



 竊嘗論澤、鼎之終而益有感焉。澤之易簀也,猶連呼「渡河」者三;而鼎自題其銘旌,有「氣作山河壯本朝」之語。何二臣之愛君憂國,雖處死生禍變之際,而猶不渝若是!而高宗惑於憸邪之口,乍任乍黜,所謂「善善而不能用」,千載而下,忠臣義士猶為之撫卷扼腕,國之不競,有以哉!



\end{pinyinscope}