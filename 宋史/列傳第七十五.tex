\article{列傳第七十五}

\begin{pinyinscope}

 包拯吳奎趙抃子屼唐介子淑問義問孫恕



 包拯,字希仁,廬州合肥人也。始舉進士,除大理評事,出知建昌縣。以父母皆老,辭不就。得監和州稅,父母又不欲行,拯即解官歸養。後數年,親繼亡,拯廬墓終喪,猶裴
 徊不忍去,里中父老數來勸勉。久之,赴調,知天長縣。有盜割人牛舌者,主來訴。拯曰:「第歸,殺而鬻之。」尋復有來告私殺牛者,拯曰:「何為割牛舌而又告之?」盜驚服。徙知端州,遷殿中丞。端土產硯,前守緣貢,率取數十倍以遺權貴。拯命制者才足貢數,歲滿不持一硯歸。



 尋拜監察御史裏行,改監察御史。時張堯佐除節度、宣徽兩使,右司諫張擇行、唐介與拯共論之,語甚切。又嘗建言曰:「國家歲賂契丹,非御戎之策。宜練兵選將,務實邊備。」又請
 重門下封駁之制,及廢錮贓吏,選守宰,行考試補蔭弟子之法。當時諸道轉運加按察使,其奏劾官吏多摭細故,務苛察相高尚,吏不自安,拯於是請罷按察使。



 去使契丹,契丹令典客謂拯曰:「雄州新開便門,乃欲誘我叛人,以刺疆事耶?」拯曰:涿州亦嘗開門矣,刺疆事何必開便門哉?」其人遂無以對。



 歷三司戶部判官,出為京東轉運使,改尚書工部員外郎、直集賢院,徙陜西,又徙河北,入為三司戶部副使。秦隴斜谷務造船材木,率課取於
 民;又七州出賦河橋竹索,恆數十萬,拯皆奏罷之。契丹聚兵近塞,邊郡稍警,命拯往河北調發軍食。拯曰:「漳河沃壤,人不得耕,刑、洺、趙三州民田萬五千頃,率用牧馬,請悉以賦民。」從之。解州鹽法率病民,拯往經度之,請一切通商販。除天章閣待制、知諫院。數論斥權幸大臣,請罷一切內除曲恩。又列上唐魏鄭公三疏,願置之坐右,以為龜鑒。又上言天子當明聽納,辨朋黨,惜人才,不主先入之說,凡七事;請去刻薄,抑僥幸,正刑明禁,戒興作,
 禁妖妄。朝廷多施行之。除龍圖閣直學士、河北都轉運使。嘗建議無事時徙兵內地,不報。至是,請:「罷河北屯兵,分之河南兗、鄆、齊、濮、曹、濟諸郡,設有警,無後期之憂。借曰戍兵不可遽減,請訓練義勇,少給餱糧,每歲之費,不當屯兵一月之用,一州之賦,則所給者多矣。」不報。徙知瀛州,諸州以公錢貿易,積歲所負十餘萬,悉奏除之。以喪子乞便郡,知揚州,徙廬州,遷刑部郎中。坐失保任,左授兵部員外郎、知池州。復官,徙江寧府,召權知開封府,
 遷右司郎中。



 拯立朝剛毅,貴戚宦官為之斂手,聞者皆憚之。人以包拯笑比黃河清,童稚婦女,亦知其名,呼曰「包待制」。京師為之語曰:「關節不到,有閻羅包老。」舊制,凡訟訴不得徑造庭下。拯開正門,使得至前陳曲直,吏不敢欺。中官勢族築園榭,侵惠民河,以故河塞不通,適京師大水,拯乃悉毀去。或持地券自言有偽增步數者,皆審驗劾奏之。



 遷諫議大夫、權御史中丞。奏曰:「東宮虛位日久,天下以為憂,陛下持久不決,何也?」仁宗曰:「卿欲誰
 立?」拯曰:「臣不才備位,乞豫建太子者,為宗廟萬世計也。陛下問臣欲誰立,是疑臣也。臣年七十,且無子,非邀福者。」帝喜曰:「徐當議之。」請裁抑內侍,減節冗費,條責諸路監司,御史府得自舉屬官,減一歲休暇日,事皆施行。



 張方平為三司使,坐買豪民產,拯劾奏罷之;而宋祁代方平,拯又論之;祁罷,而拯以樞密直學士權三司使。歐陽修言:「拯所謂牽牛蹊田而奪之牛,罰已重矣,又貪其富,不亦甚乎!」拯因家居避命,久之乃出。其在三司,凡諸管
 庫供上物,舊皆科率外郡,積以困民。拯特為置場和市,民得無擾。吏負錢帛多縲系,間輒逃去,並械其妻子者,類皆釋之。遷給事中,為三司使。數日,拜樞密副使。頃之,遷禮部侍郎,辭不受,尋以疾卒,年六十四。贈禮部尚書,謚孝肅。



 拯性峭直,惡吏苛刻,務敦厚,雖甚嫉惡,而未嘗不推以忠恕也。與人不茍合,不偽辭色悅人,平居無私書,故人、親黨皆絕之。雖貴,衣服、器用、飲食如布衣時。嘗曰:「後世子孫仕宦,有犯贓者,不得放歸本家,死不得葬
 大塋中。不從吾志,非吾子若孫也。」初,有子名繶,娶崔氏,通判潭州,卒。崔守死,不更嫁。拯嘗出其媵,在父母家生子,崔密撫其母,使謹視之。繶死後,取媵子歸,名曰綖。有奏議十五卷。



 吳奎,字長文,濰州北海人。性強記,於書無所不讀。舉《五經》,至大理丞,監京東排岸。慶歷宿衛之變,奎上疏曰:「涉春以來,連陰不解,《洪範》所謂『皇之不極,時則有下伐上』者。今衛士之變,起於肘腋,流傳四方,驚駭群聽。聞皇城
 司官六人,其五已受責,獨楊懷敏尚留。人謂陛下私近幸而屈公法,且獲賊之際,傳令勿殺,而左右輒屠之。此必其黨欲以滅口,不然,何以不奉詔?」遂乞召對面論,仁宗深器之。再遷殿中丞,策賢良方正入等,擢太常博士、通判陳州。



 入為右司諫,改起居舍人,同知諫院。每進言,惟勸帝禁束左右奸幸。內東門闌得賂遺物,下吏研治,而開封用內降釋之。奎劾尹魏瓘,出瓘越州。彭思永論事,詔詰所從受。奎言:「御史法許風聞,若窮核主名,則後
 誰敢來告以事?是自塗其耳目也。」上為罷不問。郭承祐、張堯佐為宣徽使,奎連疏其不當,承祐罷使,出堯佐河中。



 皇祐中,頗多災異,奎極言其徵曰:「今冬令反燠,春候反寒,太陽虧明,五星失度,水旱作沴,饑饉薦臻,此天道之不順也。自東徂西,地震為患,大河橫流,堆阜或出,此地道之不順也。邪曲害政,陰柔蔽明,群小紛爭,眾情壅塞,西、北貳敵,求欲無厭,此人事之不和也。夫帝王之美,莫大於進賢退不肖。今天下皆謂之賢,陛下知之而不
 能進;天下皆謂之不肖,陛下知之而不能退。內寵驕恣,近習回撓,陰盛如此,寧不致大異乎?又十數年來下令及所行事,或有名而無實,或始是而終非,或橫議所移,或奸謀所破,故群臣百姓,多不甚信,以謂陛下言之雖切而不能行,行之雖銳而不能久。臣願謹守前詔,堅如金石,或敢私撓,必加之罪,毋為人所測度,而取輕於天下。」



 唐介論文彥博,指奎為黨,出知密州。加直集賢院,徙兩浙轉運使。入判登聞檢院、同修起居注、知制誥。奉使
 契丹,會其主加稱號,要入賀。奎以使事有職,不為往。歸遇契丹使於塗,契丹以金冠為重,紗冠次之。故事,使者相見,其衣服重輕必相當。至是,使者服紗冠,而要奎盛服。奎殺其儀以見,坐是出知壽州。



 至和三年,大水,詔中外言得失。奎上疏曰:「陛下在位三十四年,而儲嗣未立。在禮,大宗無嗣,則擇支子之賢者。以昭穆言,則太祖、太宗之曾孫,所宜建立,以系四海之望。俟有皇子則退之,而優其禮於宗室,誰曰不然?陛下勿聽奸人邪謀,以誤
 大事。若倉卒之際,柄有所歸,書之史冊,為萬世嘆憤。臣不願以聖明之資,當危亡之比。此事不宜優游,願蚤裁定。定之不速,致宗祀無本,鬱結群望,推之咎罰,無大於此。」帝感其言,拜翰林學士,權開封府。



 奎達於從政,應事敏捷,吏不敢欺。富人孫氏辜榷財利,負其息者,至評取物產及婦女。奎發孫宿惡,徙其兄弟於淮、閩,豪猾畏斂。居三月,治聲赫然。除端明殿學士、知成都府,以親辭,改鄆州,復還翰林,拜樞密副使。治平中,丁父憂,居喪毀瘠,
 廬於墓側,歲時潔嚴祭祀,不為浮屠事。



 神宗初立,奎適終制,以故職還朝。逾月,參知政事。時已召王安石,辭不至,帝顧輔臣曰:「安石歷先帝朝,召不赴,頗以為不恭。今又不至,果病耶,有所要耶?曾公亮曰:「安石文學器業,不敢為欺。」奎曰:「臣嘗與安石同領群牧,見其護短自用,所為迂闊。萬一用之,必紊亂綱紀。」乃命知江寧。



 奎嘗進言:「陛下在推誠應天,天意無他,合人心而已。若以至誠格物,物莫不以至誠應,則和氣之感,自然而致。今民力困
 極,國用窘乏,必俟順成,乃可及他事。帝王所職,惟在於判正邪,使君子常居要近,小人不得以害之,則自治矣。」帝因言:「堯時,四兇猶在朝。」奎曰:「四兇雖在,不能惑堯之聰明。聖人以天下為度,未有顯過,固宜包容,但不可使居要近地爾。」帝然之。御史中丞王陶,以論文德不押班事詆韓琦,奎狀其過。詔除陶翰林學士,奎執不可。陶又疏奎阿附。陶既出,奎亦以資政殿大學士知青州。司馬光諫曰:「奎名望清重,今為陶絀奎,恐大臣皆不自安,各
 求引去。陛下新即位,於四方觀聽非宜。」帝乃召奎歸中書。及琦罷相,竟出知青州。明年薨,年五十八。贈兵部尚書,謚曰文肅。



 奎喜獎廉善,有所知輒言之,言之不從,不止也。少時甚貧,,既通貴,買田為義莊,以賙族黨朋友。沒之日,家無餘資,諸子至無屋以居,當時稱之。



 趙抃,字閱道,衢州西安人。進士及第,為武安軍節度推官。人有赦前偽造印,更赦而用者,法吏當以死。抃曰:「赦前不用,赦後不造,不當死。」讞而生之。知崇安、海陵、江原
 三縣,通判泗州。濠守給士卒廩賜不如法,聲欲變,守懼,日未入,輒閉門不出。轉運使檄抃攝治之,抃至,從容如平時,州以無事。



 翰林學士曾公亮未之識,薦為殿中侍御史,彈劾不避權幸,聲稱凜然,京師目為「鐵面御史。」其言務欲朝廷別白君子小人,以謂:「小人雖小過,當力遏而絕之;君子不幸詿誤,當保全愛惜,以成就其德。」溫成皇后之喪,劉沆以參知政事監護,及為相,領事如初。抃論其當罷,以全國體。又言宰相陳執中不學無術,且多
 過失;宣徽使王拱辰平生所為及奉使不法;樞密使王德用、翰林學士李淑不稱職;皆罷去。吳充、鞠真卿、刁約以治禮院吏,馬遵、呂景初、吳中復以論梁適,相繼被逐。



 抃言其故,悉召還。呂溱、蔡襄、吳奎、韓絳既出守,歐陽修、賈黯復求郡。抃言:「近日正人端士紛紛引去,侍從之賢如修輩無幾,今皆欲去者,以正色立朝,不能諂事權要,傷之者眾耳。」修、黯由是得留,一時名臣,賴以安焉。



 請知睦州,移梓州路轉運使,改益州。蜀地遠民弱,吏肆為不
 法,州郡公相饋餉。抃以身帥之,蜀風為變。窮城小邑,民或生而不識使者,抃行部無不至,父老喜相慰,奸吏竦服。召為右司諫。內侍鄧保信引退兵董吉燒煉禁中,抃引文成、五利、鄭注為比,力論之。陳升之副樞密,抃與唐介、呂海、範師道言升之奸邪,交結宦,進不以道。章二十餘上,升之去位。抃與言者亦罷,出知虔州。虔素難治,抃御之嚴而不苛,召戒諸縣令,使人自為治。令皆喜,爭盡力,獄以屢空。嶺外仕者死,多無以為歸,抃造舟百艘,
 移告諸郡曰:「仕宦之家,有不能歸者,皆於我乎出。」於是至者相繼,悉授以舟,並給其道里費。召為侍御史知雜事,改度支副使,進天章閣待制、河北都轉運使。時賈昌朝以故相守魏,抃將按視府庫,昌朝使來告曰:「前此,監司未有按視吾藏者,恐事無比,若何?抃曰:「舍是,則他郡不服。」竟往焉。昌朝不悅。初,有詔募義勇,過期不能辦,官吏當坐者八百餘人。抃被旨督之,奏言:「河朔頻歲豐,故應募者少,請寬其罪,以俟農隙。」從之。坐者獲免,而募亦
 隨足。昌朝始愧服。加龍圖閣直學士、知成都,以寬為治。抃向使蜀日,有聚為妖祀者,治以峻法。及是,復有此獄,皆謂不免。抃察其亡他,曰:「是特酒食過耳。」刑首惡而釋餘人,蜀民大悅。會榮諲除轉運使,英宗諭諲曰:「趙抃為成都,中和之政也。」



 神宗立,召知諫院。故事,近臣還自成都者,將大用,必更省府,不為諫官。大臣以為疑,帝曰:「吾賴其言耳,茍欲用之,無傷也。」及謝,帝曰:「聞卿匹馬入蜀,以一琴一鶴自隨,為政簡易,亦稱是乎?」未幾,擢參知政
 事。抃感顧知遇,朝政有未協者,必密啟聞,帝手詔褒答。



 王安石用事,抃屢斥其不便。韓琦上疏極論青苗法,帝語執政,令罷之。時安石家居求去,抃曰:「新法皆安石所建,不若俟其出。」既出,安石持之愈堅。抃大悔恨,即上言:「制置條例司建使者四十輩,騷動天下。安石強辯自用,詆天下公論以為流俗,違眾罔民,順非文過。近者臺諫侍從,多以言不聽而去;司馬光除樞密,不肯拜。且事有輕重,體有大小。財利於事為輕,而民心得失為重;青
 苗使者於體為小,而禁近耳目之臣用舍為大。今去重而取輕,失大而得小,懼非宗廟社稷之福也。」奏入,懇乞去位,拜資政殿學士、知杭州,改青州,時京東旱蝗,青獨多麥,蝗來及境,遇風退飛,盡墮水死。



 成都以戍卒為憂,遂以大學士復知成都。召見,勞之曰:「前此,未有自政府往者,能為朕行乎?」對曰:「陛下有言,即法也,奚例之問?」因乞以便宜從事。既至蜀,治益尚寬。有卒長立堂下,呼諭之曰:「吾與汝年相若,吾以一身入蜀,為天子撫一方。汝亦
 宜清謹畏戢以率眾,比戍還,得餘貲持歸,為室家計可也。」人喜轉相告,莫敢為惡,蜀郡晏然。劍州民私作僧度牒,或以為謀逆告,抃不師畀獄吏,以意決之,悉從輕比。謗者謂其縱逆黨,朝廷取具獄閱之,皆與法合。茂州夷剽境上,懼討乞降,乃縛奴將殺之,取血以受盟。抃使易用牲,皆歡呼聽命。



 乞歸,越州。吳越大饑疫,死者過半。抃盡救荒之術,療病,埋死,而生者以全。下令修城,使得食其力。復徙杭,以太子少保致仕,而官其子屼提舉兩浙
 常平以便養。屼奉抃遍游諸名山,吳人以為榮。元豐七年,薨,年七十七。贈太子少師,謚曰清獻。



 抃長厚清修,人不見其喜慍。平生不治貲業,不畜聲伎,嫁兄弟之女十數、他孤女二十餘人,施德煢貧,蓋不可勝數。日所為事,入夜必衣冠露香以告於天,不可告,則不敢為也。其為政,善因俗施設,猛寬不同,在虔與成都,尤為世所稱道。神宗每詔二郡守,必以抃為言。要之,以惠利為本。晚學道有得,將終,與屼訣,詞氣不亂,安坐而沒。宰相韓琦嘗
 稱抃真世人標表,蓋以為不可及雲。



 屼字景仁。由蔭登第,通判江州,改溫州,代還,得見。時抃已謝事,神宗命為太僕丞,擢監察御史。以父老請外,提舉兩浙常平。元祐中,復為御史。上疏言:「治平以前,大臣不敢援置親黨於要塗,子弟多處管庫,甚者不使應科舉,與寒士爭進。自王安石柄國,持內舉不避親之說,始以子雱列侍從,由是循習為常。資望淺者,或居事權繁重之地;無出身者,或預文字清切之職,今宜杜絕其源。」
 又言:「臺諫之臣,或稍遷其位,而陰奪言責;或略行其言,而退與善地;或兩全並立,茍從講解;或置而不問,外示包容。使忠鯁之士,蒙羞難退,皆朝廷所宜深察也。」傅堯俞、王巖叟、梁燾、孫升以事去,屼言:「諸人才能學術,為世推稱;忠言嘉謨,見於已試,宜悉召還朝。」所言皆切時務。



 避執政親嫌,改都官員外郎,出提點京東刑獄。元符中,歷鴻臚、太僕少卿。曾布知樞密院,將白為都承旨,蔡卞摭其救傅堯俞事,遂不用。未幾卒。



 初,抃廬母墓三年,縣
 榜其里曰「孝弟」。處士孫侔為作《孝子傳》。及□兀執父喪,而甘露降墓木。屼卒,子雲又以毀死,人稱其世孝。



 唐介,字子方,江陵人。父拱,卒漳州,州人知其貧,合錢以賻,介年尚幼,謝不取。擢第,為武陵尉,調平江令。民李氏貲而吝,吏有求不厭,誣為殺人祭鬼。岳守捕其家,無少長楚掠,不肯承。更屬介訊之,無他驗。守怒白於朝,遣御史方偕徙獄別鞫之,其究與介同。守以下得罪,偕受賞,介未嘗自言。



 知莫州任丘縣,當遼使往來道,驛吏以誅
 索破家為苦。介坐驛門,令曰:「非法所應給,一切勿與。稍毀吾什器者,必執之。」皆帖伏以去。沿邊塘水歲溢,害民田,中人楊懷敏主之,欲割邑西十一村地豬漲潦,介築提蘭之,民以為利。通判德州,轉運使崔嶧取庫絹配民而重其估。介留牒不下,且移安撫司責數之。嶧怒,數馳檄按詰,介不為動。既而果不能行。



 入為監察御史裏行,轉殿中侍御史。啟聖院造龍鳳車,內出珠玉為之飾。介言:「此太宗神御所在,不可喧瀆;後宮奇靡之器,不宜過
 制。」詔亟毀去。張堯佐驟除宣徽、節度、景靈、群牧四使,介與包拯、吳奎等力爭之,又請中丞王舉正留百官班庭論,奪其二使。無何,復除宣徽使、知河陽。介謂同列曰:「是欲與宣徽,而假河陽為名耳,不可但已也。」而同列依違,介獨抗言之。仁宗謂曰:「除擬本出中書。」介遂劾宰相文彥博守蜀日造間金奇錦,緣閹侍通宮掖,以得執政;今顯用堯佐,益自固結,請罷之而相富弼。又言諫官吳奎表裏觀望,語甚切直。帝怒,卻其奏不視,且言將遠竄。介
 徐讀畢,曰:「臣忠憤所激,鼎鑊不避,何辭於謫?」帝急召執政示之曰:「介論事是其職。至謂彥博由妃嬪致宰相,此何言也?進用塚司,豈應得預?」時彥博在前,介責之曰:「彥博宜自省,即有之,不可隱。」彥博拜謝不已,帝怒益甚。梁適□七介使下殿,修起居注蔡襄趨進救之。貶春州別駕,王舉正言以為太重,帝旋悟,明日取其疏入,改置英州,而罷彥博相,吳奎亦出。又慮介或道死,有殺直臣名,命中使護之。梅堯臣、李師中皆賦詩激美,由是直聲動天
 下,士大夫稱真御史,必曰唐子方而不敢名。



 數月,起監郴州稅,通判潭州,知復州,召為殿中侍御史。遣使賜告。趣詣闕下。入對,帝勞之曰:「卯遷謫以來,未嘗以私書至京師,可謂不易所守矣。」介頓首謝,言事益無所顧。他日請曰:「臣既任言責,言之不行將固爭,爭之重以累陛下,願得解職。」換工部員外郎、直集賢院,為開封府判官,出知揚州,徙江東轉運使。御史吳中復言,介不宜久居外。文彥博再當國,奏:「介向所言,誠中臣病,願如中復言。」然
 但徒河東。



 久之,入為度支副使,進天章閣待制,復知諫院。帝自至和後,臨朝淵默。介言:「君臣如天地,以交泰為理。願時延群下,發德音,可否萬幾,以幸天下。」又論:宮禁乾丐恩澤,出命不由中書,宜有以抑絕;賜予嬪御之費,多先朝時十數倍,日加無窮,宜有所朘損;監司薦舉,多得文法小吏,請令精擇端良敦樸之士,毋使與憸薄者同進;諸路走馬承受凌擾郡縣,可罷勿遣,以權歸監司;兗國公主夜開禁門,宜劾宿衛主吏,以嚴宮省。帝悉開
 納之。



 御史中丞韓絳劾宰相富弼,弼家居求罷,絳亦待罪。介與王陶論絳以危法中傷大臣,絳罷。介嫌於右宰相,請外,以知荊南。敕過門下,知銀臺司何郯封還之,留權開封府。旋以論罷陳升之,亦出知洪州。加龍圖閣直學士、河北都轉運使,樞密直學士、知瀛州。



 治平元年,召為御史中丞。英宗謂曰:「卿在先朝有直聲,故用卿,非繇左右言也。」介曰:「臣無狀,陛下過聽,願獻愚忠。自古欲治之主,亦非求絕世驚俗之術,要在順人情而已。祖宗遺德
 餘烈,在人未遠,願覽已成之業以為監,則天下蒙福矣。明年,以龍圖閣學士知太原府。帝曰:「朕視河東,不在中執法下,暫煩卿往耳。」夏人數擾代州邊,多築堡境上。介遣兵悉撤之,移諭以利害,遂不敢動。



 神宗立,以三司使召。熙寧元年,拜參知政事。先時,宰相省閱所進文書於待漏舍,同列不得聞。介謂曾公亮曰:「身在政府而文書弗與知,上或有所問,何辭以對?」乃與同視,後遂為常。帝欲用王安石,公亮因薦之,介言其難大任。帝曰:「文學不
 可任耶?吏事不可任耶?經術不可任耶?」對曰:「安石好學而泥古,故論議迂闊,若使為政,必多所變更。」退謂公亮曰:「安石果用,天下必困擾,諸公當自知之。」中書嘗進除目,數日不決,帝曰:「當問王安石。」介曰:「陛下以安石可大用,即用之,豈可使中書政事決於翰林學士?臣近每聞宣諭某事問安石,可即行之,不可不行,如此則執政何所用,恐非信任大臣之體也。必以臣為不才,願先罷免。」



 安石既執政,奏言:「中書處分札子,皆稱聖旨,不中理者
 十八九,宜止令中書出牒。」帝愕然。介曰:「昔寇準用札子遷馮拯官不當,拯訴之,太宗謂:『前代中書用堂牒,乃權臣假此為威福。太祖時以堂帖重於敕命,遂削去之。今復用札子,何異堂帖?』張洎因言:『廢札子,則中書行事,別無公式。」太宗曰:『大事則降敕,其當用扎子,亦須奏裁。』此所以稱聖旨也。如安石言,則是政不自天子出,使輔臣皆忠賢,猶為擅命,茍非其人,豈不害國?」帝以為然,乃止。介自是數與安石爭論。安石強辯,而帝主其說。介不勝
 憤,疽發於背,薨,年六十。



 介為人簡伉,以敢言見憚。每言官缺,眾皆望介處之,觀其風採。神宗謂其先朝遺直,故大用之。然居政府,遭時有為,而扼於安石,少所建明,聲名減於諫官、御史時。比疾亟,帝臨問流涕,復幸其第吊哭,以畫像不類,命取禁中舊藏本賜其家。贈禮部尚書,謚曰質肅。子淑問、義問,孫恕。



 淑問字士憲。第進士,至殿中丞。神宗以其家世,擢監察御史裏行,諭以謹家法、務大體。淑問見帝初即位,銳於
 治,因言:「中旨數下,一出特斷,當謹出納、別枉直,使命令必行。今詔書求直言,而久無所施用,必欲屈群策以起治道,願行其言。」初,詔侍臣講讀。淑問言:「王者之學,不必分章句、飾文辭。稽古聖人治天下之道,歷代致興亡之由,延登正人,博訪世務,以求合先王,則天下幸甚。」河北饑,流人就食京師,官振廩給食,來者不止。淑問曰:「出粟不繼,是誘之失業而就死地也。」條三策上之。



 滕甫為中丞,淑問力數其短,帝以為邀名,乃詔避其父三司使,出
 通判復州。久之,知真州,提點湖北刑獄,言新法不便,乞解使事,黜知信陽軍,以病免。數年,起知宣州,徙湖州,入為吏部員外郎。又引疾求外,帝以為避事,降監撫州酒稅。哲宗立,司馬光薦其行己有恥,難進,召為左司諫,以病致仕,數月卒。



 義問字士宣。善文辭,鎖廳試禮部,用舉者召試秘閣,父介引嫌罷之。熙寧中,闢京西轉運司管勾文字。神宗覽本道章奏,知義問所為。以其名訪輔臣,因黃好謙領使
 事,諭之曰:「唐義問風力強敏,行且用矣,可面詔之。」尋以為司農管當公事。方行手實法,所在騷然。義問言:「今造簿甫二歲,民不堪命,不宜復改為。」從曾孝寬使河東,還奏事,記利害綱目於笏,帝取而熟視之,歷舉以問,應析如流。帝喜曰:「欲見卿,非今日也。」擢湖南轉運判官。一路敷免役錢,又分戶五等,儲其羨為別賦,號「家力錢」,義問奏除之。移使京西,文彥博守西都,義問求罷去。彥博告以再入相時,嘗薦其父,晚同為執政,相得甚歡,故義問
 乃止。時陜西大舉兵,多亡卒,所至成聚。義問請令詣官自陳,給券續食,人以為便。會有不悅之者,免歸。



 元祐中,起知齊州,提點京東刑獄、河北轉運副使。屬邑尉因捕盜誤遺火,盜逸去,民家被焚,訟尉故縱火。郡守執尉,抑使服,義問辨出之,方旱而雨。用彥博薦,加集賢修撰,帥荊南,請廢渠陽諸砦。蠻楊晟秀斷之以叛,即拜湖北轉運使,討降之,復砦為州。進直龍圖閣,以集賢殿修撰知廣州。章惇秉政,治棄渠陽罪,貶舒州團練副使。後七年,
 復故官,知穎昌府,卒。



 恕,崇寧初,為華陽令,以不能奉行茶法,忤使者,謝病免歸。其弟意方為南陵令,亦以病自免,兄弟杜門躬耕。恕尋以宣教郎致仕。靖康元年,御史中丞許翰言其高行,詔起為監察御史。意亦以宰相吳敏薦,召對,而貧不能行,竟餓死江陵山中。



 論曰:拯為開封,其政嚴明,人到於今稱之。而不尚苛刻,推本忠厚,非孔子所謂剛者乎:奎博學清重,君子人也。
 抃所至善治,民思不忘,猶古遺愛。介敢言,聲動天下,斯古遺直也。夫聽諫者,明君所難,以唐文皇猶弗終於魏徵,觀四臣面諍,鯁吭逆心,或不能堪,而仁宗容之無咈,誠盛德之主哉!屼世孝,淑問難進,義問強敏,恕高行不隕家聲,有足美云。



\end{pinyinscope}