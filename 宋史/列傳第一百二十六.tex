\article{列傳第一百二十六}

\begin{pinyinscope}

 李顯忠楊存中郭浩楊政



 李顯忠,綏德軍青澗人也。初名世輔,南歸,賜名顯忠。由唐以來,世襲蘇尾九族巡檢。初,其母當產,數日不能免,有僧過門曰:「所孕乃奇男子,當以劍、矢置母旁,即生。」已
 而果生顯忠,立於蓐,咸異之。



 年十七,投效用,隨父永奇出入行陣。金人犯鄜延,經略王庶命永奇募間者,得張琦;更求一人,顯忠請行。永奇曰:「汝未涉歷,行必累琦。」顯忠曰:「顯忠年小,膽氣不小,必不累琦,當與琦俱。」有敵人夜宿陶穴,顯忠縋穴中,得十七人,皆殺之,取首二級,馬二匹,余馬悉折其足。庶大奇之,補承信郎,充隊將,由是始知名。轉武翼郎,充副將。



 金人陷延安,授顯忠父子官。永奇聚泣曰:「我宋臣也,世襲國恩,乃為彼用邪!」會劉豫
 令顯忠帥馬軍赴東京,永奇密戒之曰:「汝若得乘機,即歸本朝,無以我故貳其志。事成,我亦不朽矣。」顯忠至東京,劉麟喜之,授南路鈐轄,乃密遣其客雷燦以蠟書赴行在。已而豫廢,兀朮以萬騎馳獵淮上,與顯忠獨立馬圍場間。顯忠戒吳俊往探淮水可度馬處,欲執兀朮歸朝。俊還,顯忠馳問之,為竹刺傷馬而止。兀朮授顯忠承宣使、知同州。



 顯忠至鄜省侍,永奇教顯忠曰:「同州入南山,乃金人往來驛路,汝可於此擒其酋,渡洛、渭,由商、虢
 歸朝。第報我知,我當以兵取延安而歸。」顯忠赴同州,即遣黃士成等持書由蜀至吳,報歸朝事。元帥撒里曷來同州,顯忠以計執之,馳出城。至洛河,舟船後期不得渡,與追騎屢戰,皆勝。顯忠憩高原,望追騎益多,乃與撒里曷折箭為誓,不得殺同州人,不得害我骨肉,皆許之,遂推之下山崖,追兵爭救得免。顯忠攜老幼長驅而北,至鄜城縣,急遣人告永奇。永奇即挈家出城,至馬TG谷口,為金人所及,家屬二百口皆遇害。是日,天昏大雪,延
 安人聞之皆泣下。



 顯忠僅以二十六人奔夏國。夏人問故。顯忠泣,具言父母妻子之亡,切齒疾首,恨不即死,願得二十萬人生擒撒里曷,取陜西五路歸於夏,顯忠亦得報不共戴天之仇。夏主曰:「爾能為立功,則不靳借兵。」時有酋豪號「青面夜叉」者,久為夏國患,乃令顯忠圖之。請三千騎,晝夜疾馳,奄至其帳,擒之以歸。夏主大悅,即出二十萬騎,以文臣王樞、武臣



 ……為陜西招撫使,顯忠為延安招撫使,時紹興九年二月十四日也。



 顯忠
 引兵至延安,總管趙惟清大呼曰:「鄜延路今復歸宋矣,已有赦書。」顯忠與官吏觀赦書列拜,顯忠大哭,眾皆哭,百姓哭聲不絕。乃以舊部八百餘騎往見王樞、□移訛,諭之曰:「顯忠已得延安府,見講和赦書,招撫可以本部軍歸國。」□移訛不從,曰:「初,經略乞兵來取陜西。今既到此,乃令我歸耶?」顯忠知勢不可,乃出刀斫□移訛,不及,擒王樞縛之。夏人以鐵鷂子軍來。顯忠以所部拒之,馳揮雙刀,所向披靡,夏兵大潰,殺死蹂踐無慮萬人,獲馬四萬匹。
 顯忠揭榜招兵,以「紹興九年」為文書。每得一人,予馬一匹,旬日間得萬人,皆驍勇少壯。又擒害其父母弟侄者,皆斬於東城之內。行至鄜州,已有馬步軍四萬餘。撒里曷在耀州,聞顯忠來,一夕遁去。



 四川宣撫吳玠遣張振來撫諭云:「兩國見議和好,不可生事,可量引軍赴行在。」遂至河池縣見玠,玠撫之曰:「忠義歸朝,惟君第一。」從行使臣崔皋等六百餘人列拜庭下,玠又撫之,犒以銀絹,詣行府受告敕、金帶,除指揮使、承宣使。至行在,高宗撫
 勞再三,賜名加賚,又賜田鎮江,以崔皋輩充將佐。



 兀朮犯河南,命顯忠為招撫司前軍都統制,與李貴同破靈壁縣。兀朮犯合肥,手詔以軍與張俊會。顯忠至孔城鎮,與敵戰,敗之。兀朮謂韓常曰:「李世輔歸宋,不曾立功,此人敢勇,宜且避之。」乃焚廬江而走。顯忠欲追之與死戰,俊以奉旨監護,慮失顯忠,遂各以軍還。



 太后至臨安,顯忠入覲,加保信軍節度使、浙東副總管。顯忠熟西邊山川險易,因上恢復策,忤秦檜意。金使言顯忠私遣人過
 界,遂降官奉祠,臺州居住。復寧國軍節度使,升都統制。



 二十九年,金渝盟,詔顯忠以本部捍禦。遣統制官韋永壽等以二百騎至安豐軍,與金將小韓將軍兵五千人戰於大人洲,敗之。俄又增兵萬餘來,顯忠率騎軍出,自旦至午,氣百倍,以大刀斫敵陣,敵不能支,殺獲甚眾,掩入淮者不可計。



 金主亮犯淮西,朝廷命王權拒於合肥。權退保和州,又棄軍渡江,和州失守。金主親統細軍駐和之雞籠山,將濟採石。朝廷詔以顯忠代權,命虞允文
 趣顯忠交軍,軍中大喜,於是有採石之捷,語在《允文傳》。顯忠退軍沙上,得楊存中報:「車駕至平江,可速進兵。」顯忠選銳士萬人渡江,盡復淮西州郡。軍至橫山澗,與金射鵰軍戰,統制頓遇重傷,韋永壽死之,敵兵敗走。金主亮切責諸將不用命,諸將弒之而還。



 是役也,顯忠所將一萬九千八百六人行賞有差,張振功為最。詔賜顯忠五子金帶。授顯忠淮西制置使、京畿等處招討使,擢太尉、寧國軍節度使、主管侍衛馬軍司公事,赴行在。



 孝宗
 即位,賜田百頃,兼權池州駐札御前諸軍都統制,節制軍馬。隆興元年,兼淮西招撫使。時金主褒新立,山東、河北豪傑蜂起,耶律諸種兵數十萬據數郡之地,太行山忠義耿京、王世隆輩皆欲挈地還於朝。金懼,亟請和。顯忠陰結金統軍蕭琦為內應,請出師自宿、亳趨汴,由汴京以通關陜;關陜既通,則鄜延一路熟知顯忠威名,必皆響應,且欲起其舊部曲,可得數萬人,以取河東。



 時張浚開都督府,四月,命顯忠渡江督戰。乃自濠梁渡淮,至
 陡溝,琦背約,用拐子馬來拒,與戰,敗之。琦復背城列陣,顯忠躬率將士鏖戰,琦敗走,遂復靈壁,入城,宣布德意,不戮一人,中原歸附者踵接。時邵宏淵圍虹縣未下,顯忠遣靈壁降卒開諭禍福,金貴戚大周仁及蒲察徙穆皆出降。宏淵恥功不自己出;又有降千戶訴宏淵之卒奪其佩刀,顯忠立斬之,由是二將益不相能。



 六月,兵傅宿州城,金人來拒,顯忠敗之,斬其左翼都統及首虜數千人,追奔二十餘里。宏淵至,謂顯忠曰:「招撫真關西將
 軍也。」顯忠閉營休士,為攻城計,宏淵等不從。顯忠引麾下楊椿上城,開北門,不逾時拔其城。宏淵等殿後,趣之,乃始渡濠登城。城中巷戰,又斬首虜數千人,擒八十餘人,遂復宿州。舉寄居官劉時攝州事。捷聞,授顯忠開府儀同三司、殿前都指揮使,妻周氏封國夫人。宏淵欲發倉庫犒士卒,顯忠不可,移軍出城,止以見錢犒士,士皆不悅。



 金帥孛撒自南京率步騎十萬來,晨薄城,列大陣。顯忠親帥軍遇於城南,戰數十合,孛撒大敗,遂退走。統
 制李福、統領李保各以所部退避,皆斬以徇。翼日,敵益兵至。顯忠謂宏淵並力夾擊,宏淵按兵不動,顯忠獨與所部力戰百餘合,殺左翼都統及千戶、萬戶,斬首虜五千餘人。俄增兵復來逼城,顯忠用克敵弓射卻之。



 宏淵顧眾曰:「當此盛夏,搖扇於清涼猶不堪,況烈日中被甲苦戰乎?」人心遂搖,無鬥志。至夜,中軍統制周宏鳴鼓大噪,陽謂敵兵至,與邵世雍、劉人先各以所部兵遁;繼而統制左士淵、統領李彥孚亦遁。顯忠移軍入城,殿司前軍
 統制張訓通、馬司統制張師顏、池州統制荔澤、建康統制張淵各遁去。



 金人乘虛復來攻城,顯忠竭力捍禦,斬首虜二千餘人,積尸與羊馬墻平。城東北角敵兵二十餘人已上百餘步,顯忠取軍所執斧斫之,敵始退卻。顯忠曰:「若使諸軍相與掎角,自城外掩擊,則敵兵可盡,金帥可擒,河南之地指日可復矣。」宏淵又言:「金添生兵二十萬來,儻我軍不返,恐不測生變。」顯忠知宏淵無固志,勢不可孤立,嘆吒曰:「天未欲平中原耶?何沮撓若此!」是
 舉,所喪軍資器械殆盡,幸而金不復南。顯忠以軍還,見浚,納印待罪。責授果州團練副使,潭州安置。後朝廷知其故,移撫州。



 乾道改元,乃還會稽,復防禦使,觀察使、浙東副總管,賜銀三萬兩,絹三萬匹,綿一萬兩。提舉臺州崇道觀。召除威武軍節度使、左金吾衛上將軍,賜第京師。上奇其狀貌魁傑,命繪像閣下。復太尉。乞祠,提舉興國宮,紹興府居住,歲賜米二千石。



 淳熙四年,召赴行在,提舉萬壽觀,奉朝請。入見,給真奉,賜內庫金,再葺前所
 賜第賜之,七月卒,年六十九。贈開府儀同三司,謚忠襄。



 楊存中,本名沂中,字正甫,紹興間賜名存中,代州崞縣人。祖宗閔,永興軍路總管,與唐重同守永興,金人陷城,迎戰死之。父震,知麟州建寧砦,金人來攻,亦死於難。



 存中魁梧沉鷙,少警敏,誦書數百言,力能絕人。慨然語人曰:「大丈夫當以武功取富貴,焉用俯首為腐儒哉!」於是學孫、吳法,善射騎。宣和末,山東、河北群盜四起,存中應募擊賊,積功至忠翊郎。



 靖康元年,金人再圍汴京,諸道
 兵勤王,存中與張俊、田師中從信德府守臣梁揚祖以萬兵入援,後隸張俊部曲。上問將于俊,俊以存中對。召見,賜袍帶。時元帥府草創,存中晝夜扈衛寢幄,不頃刻去側。帝知其忠謹,親信之。劇賊李昱據任城,久不克,存中以數騎入,擊殺數百人。帝乘高望見,介冑盡赤,意其被重創。召視之,皆污賊血,壯之,飲以酒,曰:「酌此血漢。」存中請復往,帝止之。存中曰:「此賊膽碎,即成擒矣。」遂大破之,復任城,遷閣門祗候。



 建炎二年,討賊徐明於嘉興,先
 登。主帥將屠城,存中力諫止之,戮其渠魁而已,郡賴以全。遷榮州刺史。高宗南渡,以勝捷軍從張俊守吳門;苗、劉之變,又從俊赴難。遷貴州團練使,尋為御前右軍統領。金人攻明州,又從俊與田師中、趙密殊死戰,破之。以奇功遷文州防禦使、御前中軍統制。



 紹興元年,從俊討李成。諸將議,多欲分道進,存中曰:「賊勢如此,兵分則力弱,又諸將位均勢敵,非招討督之,必不相為用。」俊然之。整軍至豫章,存中率兵數千,首破賊於玉隆觀,追至筠
 州。賊驍將以眾十萬來援,夾河而營。存中謂俊曰:「彼眾我寡,擊之當用奇,願以騎見屬,公以步兵居前。」俊從之。存中夜銜枚渡筠河,出西山,馳下擊賊,俊以步兵夾攻,俘八千人。諸將夜見存中曰:「戰未休,降卒多,忽有變,奈何?非盡殲之不可。」存中曰:「殺降吾不忍。」諸將轉告俊,竟夜坑之。乘勝追至九江,成遂遁去。遷宣州觀察使。



 二年春,進神武中軍統制,宰相呂頤浩袖敕以授存中。俊奏留存中軍中,上曰:「宿衛乏帥,朕所選,為不可易也。」存中
 亦固辭,且謂:「神武諸帥如韓世忠、張俊,皆貴擁旄鉞,名望至重,如臣麼麼,一旦位與之抗,實不自安。」不許,遣中使宣押,乃視事。兼提舉宿衛親兵。時中軍卒不滿五千,疲癃者居半。存中請拘神武卒借出於外者歸軍中,由是軍政浸修。



 三年,嚴州妖賊繆羅據白馬源,殺王官,存中討平之。除帶御器械,加保信軍承宣使、權發遣鄜延路馬步軍副總管。



 六年,為龍神衛四廂都指揮使、密州觀察使。先是,張浚視師,謀渡淮以圖劉豫,倚韓世忠為
 用。世忠圍淮陽,從浚乞張俊將趙密為助,俊拒之。趙鼎語浚曰:「世忠所欲者趙密爾,存中武勇,不減於密,盍令存中助之。」浚請於朝,故有是命。於是存中以八隊萬人,趨督府助世忠。



 十月,存中與劉猊戰於藕塘,大破之。猊之初入也,淮西宣撫使劉光世欲棄廬州,退保太平。賊眾十萬已次濠、壽間,浚命張俊拒之,使存中往泗州與俊合。及至泗,則光世已舍廬去。浚遣人諭之曰:「一人渡江,即斬以徇。」光世不得已還廬駐兵,與存中相應。賊先
 犯定遠縣,存中以兵二千襲敗於越家坊。既而與猊兵遇藕塘,賊據山列陣,矢下如雨。存中急擊之,且使統制吳錫以勁騎五千突其陣。陣亂,存中鼓大軍乘之,自以精騎沖其肋,大呼曰:「破賊矣!」賊錯愕駭視。前軍統制張宗顏自泗來,乘背擊之,賊大敗。猊以首抵謀主李愕曰:「適見髯將軍,銳不可當,果楊殿前也。」即以數騎遁去。餘黨萬人殭立失措,存中躍馬叱之,皆怖而降。麟在順昌,孔彥舟方圍光州,聞之皆拔砦遁去,北方大恐。所得賊
 舟數百艘,車數千兩。



 捷聞,帝遣中使勞賜,謂宰執曰:「卿輩始知朕得人也。」除保成軍節度使、殿前都虞候尋兼領馬步帥。存中奏:「祖宗置三衙,鼎列相制,今令臣獨總,非故事也。」不允。七年,為淮南西路制置使,將以撫定酈瓊諸軍,不果行,語在《王德傳》。九年,遷殿前副都指揮使。



 十年,金人叛盟取河南,命存中為淮北宣撫副使,引兵至宿州,以步軍退屯於泗。金人詭令來告敵騎數百屯柳子鎮。存中欲即擊之,或以為不可,存中不聽。留王滋、
 蕭保以千騎守宿,自將五百騎夜襲柳子鎮,黎明,不見敵而還。金人以精兵伏歸路,存中知之,遂橫奔而潰。參議官曹勛不知存中存亡,以聞,朝廷震恐,於是有權宜退保之命。既而存中自壽春渡淮歸泗,人心始安。冬,引兵還行在。



 十一年,兀朮恥順昌之敗,復謀來侵。詔大合兵於淮西以待之。於是存中以殿司兵三萬卒戍淮,與金人戰於柘皋,敗之。時張俊為宣撫使,存中為副使,劉錡為判官,王德為都統制,田師中、張子蓋為統制官。金
 人以拐子馬翼進,存中曰:「敵恃弓矢,吾有以屈之。」使萬人操長斧,如墻而進,諸軍鼓噪奮擊,金人大敗,退屯紫金山。是役也,失將士九百人,金人死者以萬計,而濠圍猶未解。



 俊與存中、錡先議班師。會有雲濠路已通者,俊謂錡曰:「吾欲與楊太尉耀兵淮上,安撫濠梁之民,取宣化歸金陵,楊太尉則渡瓜洲還臨安。」明日,命二帥行。諜報金攻濠甚急,倉皇復回,邀錡會於黃連埠,距濠六十里,聞城陷矣,召存中、錡謀之。錡謂存中:「何以處此?」存中
 曰:「戰爾,相公與太尉在後,存中當居前。」錡曰:「本來救濠,濠既已失,進無所依,人懷歸心,勝氣已索,此危道也。不若退師據險,俟其去,為後圖。」諸將皆曰:「善。」鼎足而營,遣人俟敵,曰:「已去矣。」俊自以為功,謂錡毋往,命存中與德偕至濠。列陣未定,煙起城中,金人伏騎萬餘分兩翼出。存中顧德曰:「何如?」德曰:「德小將,焉敢預事?」存中以策麾軍曰:「那回!」諸軍以為令其走也,遂散亂南奔,無復紀律,金人追殺甚眾。後一日,韓世忠大軍至,已無及矣。存中
 乃自宣化渡江歸行在。加檢校少保、開府儀同三司兼領殿前都指揮使,蓋錄柘皋之功而掩濠梁之敗也。



 十二年,徽宗梓宮攢永固陵,命存中都護。竣事,拜少傅,以保傅為管軍自存中始。十四年,存中請詣太學謁先聖,帝曰:「學校既興,武人亦知崇尚,如漢羽林士皆通《孝經》,況其它乎?」二十年,封恭國公。二十八年,拜少師,恩數視樞密使。存中以凡重地皆有統制官,獨荊、襄無之,請於朝,於是荊南、襄陽初置諸統制。



 存中在殿嚴凡二十五
 載,權寵日盛,太常寺主簿李浩、敕令所刪定官陸游、司封員外郎王十朋、殿中侍御史陳俊卿相繼以為言。三十一年,罷為太傅、醴泉觀使,進封同安郡王,賜玉帶,朝朔望。



 時金主亮有南侵意,存中上備敵十策。步帥趙密謀奪存中權,因指為喜功生事。存中聞之,上章乞免,密竟代之。未幾,邊聲日急,九月,詔存中為御營宿衛使。劉汜戰敗於瓜洲,命存中往京口,為守江計。虞允文自採石來會,存中與之協力拒敵。敵不能濟。金主亮死,與允
 文輕舟渡江以伺敵。及金人請和,存中奏俟彼得新主之命,無遽許之。



 帝如建康,詔存中扈蹕,因語宰相曰:「楊存中唯命東西,忠無與二,朕之郭子儀也。」金使復請和,存中請拘之江口,移書審問,若能歸我族屬,還舊壤,損歲幣,復白溝之界,以通兄弟之好,如是則和議可從;不然,請斬其使,亟圖恢復。會駕還,以存中為江、淮、荊、襄路宣撫使,給、舍不書黃,命遂寢。未幾,仍奉祠。



 隆興元年,王師潰於符離,復起存中為御營使。二年,金人再入關,議
 割蜀之和尚原以畀之。存中入對,曰:「和尚原,隴右之藩要也。敵得之,則可以睥睨漢川;我得之,則可以下兵秦雍。曩議予金人,吳璘力爭不從。今璘在遠,不及知。臣若不言,非特負陛下,亦有愧於璘。近者,王師盡銳而後得,願毋棄。」



 未幾,金人復攻淮甸,詔存中同都督江、淮事。湯思退罷,升都督,陛辭,賜坐,賜玉鞍勒。時諸軍各守分地,不相統一,存中集諸將調護之。於是始更相為援。帝親札賜之曰:「諸帥協和,互相策應,卿之力也。」會金兵已深
 入,朝議欲舍淮保江,存中持不可,乃已。金兵在揚州,或勸存中擊之。存中不敢渡,獨臨江固壘以老之。



 金人尋請盟。乾道元年班師,加昭慶軍節度使,復奉祠。時興屯田,存中獻私田在楚州者三萬九千畝。二年,卒,年六十五。以太師致仕,追封和王,謚武恭。高宗追念舊臣,為之出涕,賻錢十萬。高宗假借諸將,眷存中尤深,嘗曰:「朕於存中,撫綏之過於子弟。」濠、廬之役,親筆戒之曰:「若不便進,當行軍法。」趙密代領殿帥,則舉唐崔祐甫奪王駕鶴
 兵權事,豫戒大臣。及竣事,又曰:「楊存中之罷,朕不安寢者三夕。」



 存中天資忠孝敢勇,大小二百餘戰,身被五十餘創。宿衛出入四十年,最寡過。孝宗以為舊臣,尤禮異之,常呼郡王而不名。父、祖及母皆死難,存中既顯,請於朝,宗閔謚忠介,震謚忠毅,賜廟曰顯忠,曰報忠。又以家廟、祭器為請,遂許祭五世,前所無也。祖母劉流落蜀、隴,存中日夜禱祠訪問,間關數千里,卒迎以歸。御軍寬而有紀,所用將士,專以才勇選,不私部曲之舊。李顯忠
 以罪斥,存中奏為統制官,後為名將。嘗以克敵弓雖勁而蹶張難,遂以意創馬皇弩,思巧制工,發易中遠,人服其精。嘗營居鳳山,十年而就,極山川之勝,後獻於朝廷,更築室焉。又葺園亭於湖山之間,高宗為書「水月」二字。所居建閣以藏御書,孝宗題曰「風雲慶會之閣」。



 子,偰工部侍郎;倓簽書樞密院事、昭慶軍節度使。



 郭浩,字充道,德順軍隴干人。父任三班奉職。徽宗時,充環慶路第五將部將,嘗率百騎抵靈州城下,夏人以千
 騎追之,浩手斬二騎,以首還。充渭州兵馬都監。從種師道進築葺平砦,敵據塞水源,以渴我師,浩率精騎數百奪之。敵攻石尖山,浩冒陣而前,流矢中左肋,怒不拔,奮力大呼,得賊乃已;諸軍從之,敵遁去,由是知名。累遷中州刺史。



 欽宗即位,進安州團練使。以種師道薦,召對,奏言:「金人暴露,日久思歸。乞給輕兵間道馳滑臺,時其半度,可擊也。」會和戰異議,不能用。帝問西事,浩曰:「臣在任已聞警,慮夏人必乘間盜邊,願選將設備。」已而果攻涇
 原路,取西安州、懷德軍。紹聖開拓之地,復盡失之。種師中制置河東,闢以自隨。



 建炎元年,知原州。二年,金人取長安,涇州守臣夏大節棄城遁,郡人亦降。浩適夜半至郡,所將財二百人,得金人不殺,使之還,曰:「為語汝將曰,我郭浩也,欲戰即來決戰。」金人遂引去。升本路兵馬鈐轄、知涇州、權主管鄜延路經略安撫。



 時二敵交侵,鄜延之東皆金人,西北即夏境,其屬朝廷者惟保安一軍、德靜一砦。浩間道之德靜,置司招收散亡,與敵對壘,一
 年,敵不能犯。再除涇原路兵馬鈐轄、知涇州。浩去,夏人復來,權帥耿友諒僅以身免,一路盡陷。



 張浚為宣撫處置使,以浩為秦鳳路提點刑獄、權經略使、知秦州。時浚經略陜西,有言敵可討者,浚意向之。諸帥恥於不武,莫敢出言。浚檄五路帥悉所部兵會於富平,浩獨謂敵鋒方銳,且當分守其地,掎角相援,俟釁而動。浚不聽,師出果敗,五路俱陷,帥府皆徙置他所。浚復以浩舊官移知鳳翔府,寓治寶雞縣,又退保和尚原。金人抵原下,浩與吳
 玠隨方捍禦,蜀以安全。第功,遷正任防禦使。



 紹興元年,金人破饒風嶺,盜梁、洋,入鳳州,攻和尚原。浩與吳璘往援,斬獲萬計。遷邠州觀察使,徙知興元府。饑民相聚米倉山為亂,浩討平之。徙知利州。金人以步騎十餘萬破和尚原,進窺川口,抵殺金平,浩與吳玠大破之。遷彰武軍承宣使。玠按本路提點刑獄宋萬年陰與敵境通,利所鞫不同,由是與浩意不協,朝廷乃徙浩知金州兼永興軍路經略使。



 金州殘弊特甚,戶口無幾,浩招輯流亡,
 開營田,以其規置頒示諸路。他軍以匱急仰給朝廷,浩獨積贏錢十萬緡以助戶部,朝廷嘉之,凡有奏請,得以直達。九年,改金、洋、房州節制。



 金人還河南地,以浩為龍、神衛四廂都指揮使,充陜西宣諭使、知金州。樓照行關中,闢浩樞密院都統制、節制陜西軍馬。十年,拜奉國軍節度使。五路陷,徙知夔州,未行,移知金州,仍永興路經略安撫使、節制陜西河東兼措置河東路忠義軍馬。十一年,金人內侵,宣撫使胡世將召浩及吳璘、楊政會仙
 人原,授以攻取之策。浩遺裨將設伏破之。



 十四年,召見,拜檢校少保,還鎮,賜以御府金器、繡鞍,仍官一子文資,賜田五十頃。浩辭曰:「臣父子起身行陣,不敢忘本,願還文資。」帝嘉其意,別與一子閣職。是歲,分利州為東西兩路,以浩為金房開達州經略安撫使兼知金州、樞密院都統制,屯金州,仍建帥府。十五年,卒,年五十九。贈檢校少帥,謚恭毅。淳熙元年,賜立廟金州。



 楊政,字直夫,原州臨涇人。崇寧三年,夏人舉國大入,父
 忠戰歿,政甫七歲,哀號如成人。其母奇之,曰:「孝於親者必忠於君,此兒其大吾門乎?」宣和末,應募為弓箭手。靖康初,因拒夏人,稍知名。建炎間,從吳玠擊金人,九戰九捷。累功至武顯郎。



 紹興元年春,金人趨和尚原,又攻箭筈關,政引兵大破之,斬千戶一、酋長二。遷右武大夫。十月,金兵大集,號十萬,自寶雞列柵至原下。吳玠與相持累日,以政統領將兵迎敵,日數十合,士卒無不一當百。復出奇兵斷其糧道,敵少卻,遮擊之,獲萬戶及首領三
 百餘人、甲士八百六十人。拜恭州刺史。時有嫉政者,以母妻尚留北境,不宜屬以兵權,玠不聽,政益感奮。



 二年,金合步騎數千柵魚龍川口,政帥精兵劫破之。升隴州團練使,移知方山原,軍儲芻穀在其中。三月,金大軍來攻,城且下,政擊敗之。選知鳳州。三年,金攻饒風關,政從玠戰關下,凡六日。改明州觀察使。



 四年,撒離喝裒精兵十萬,欲道仙人關入蜀,至上奢田。玠築壘於關外,政曰:「此地為蜀厄塞,當堅守,時出奇擊之。」玠用其言。金人變
 態多端,政隨機應之,連日百餘戰。敵帥督戰益急,政命卒以神臂弓射之;又選甲士千餘出山谷,斷其兵,使不得進退;又出敵不意,夜斫其營。敵遂遁去,追至河池而還。授龍神衛四廂都指揮使、環慶路經略安撫使。



 五年,金人攻淮,玠命政帥師乘機牽制,至秦州,一戰而拔,撫定居民,秋毫無犯。改經略安撫涇原兼帥環慶、利路。三鎮事叢集,剖決無滯。母留敵境,間遣人省視之,母惟勉以忠義。九年春,和議成,始得迎母及兄弟歸。乞祠以便
 養,不許。詔封其母感義郡夫人,以政為熙河蘭鞏路經略安撫使、知熙州,進武康軍承宣使。



 十年,徙利州,又徙興元。會金人渝盟,政建迎敵之策,兼川、陜宣撫副使司都統制。政偕統制楊從義劫金人於鳳翔府城南砦,敗之,獲戰馬數百。母卒,起復,遂帥師趣寶雞渭水上,以拒敵沖,凡大戰七,斬獲甚多。川、陜宣撫副使胡世將奏:「鳳翔之捷,政奮不顧身,功效顯著。」拜武當軍節度使。



 十一年秋,金將胡盞、習不祝合軍五萬來攻,政與吳璘、郭浩
 會於仙人原。世將授以攻取之策,政出和尚原,浩出商州以為援,璘駐秦州。政引兵夜入隴州界,遂趨吳山,與金人對壘,又敗金萬戶通檢於寶雞。時通檢居渭北,政欲攻拔其城,通檢將精甲萬眾出,政帥勇士鏖戰,遣裨將突出陣後,登山執幟。金軍見之,大呼曰:「伏發矣!」乃驚潰。政乘勝掩殺,通檢走至城門而橋已絕,遂擒之。



 和議成,帝召政還,軍民詣部使者借留。及入見,條奏詳明,帝善之。十三年,還鎮,加檢校少保,賜田五十頃。十四年,分
 利州為東西兩路,政屯興元府。久之,拜太尉。二十七年,卒,年六十。贈開府儀同三司,謚襄毅。



 政守漢中十八年,六堰久壞,失灌溉之利,政為修復。漢江水決為害,政築長堤捍之。凡利於民者不敢以軍旅廢。休兵十餘年,未嘗升遷將士,上下安之。政故為吳璘裨將,及與璘分道建帥,執門下之禮益恭,世頗賢之。



 論曰:李顯忠生而神奇,立功異域,父子破家徇國,志復中原,中罹讒構,屢遭廢黜,傷哉!楊存中出入淮甸,無大
 勝負,典兵最久,貴寵獨隆,然頗能知幾,不阽禍敗,其亦有天幸者歟?郭浩、楊政克左右,玠、璘兄弟保全川蜀。數君子皆人所屬倚以成功者,奈何撓於和議,頻失事機,人心沮喪,不得如吉甫、方叔,受祉振旅以成中興之業,惜哉!



\end{pinyinscope}