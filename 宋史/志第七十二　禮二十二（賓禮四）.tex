\article{志第七十二 禮二十二(賓禮四)}

\begin{pinyinscope}

 錄周後錄先聖後群臣朝使宴餞朝臣時節饋廩外國君長來朝契丹夏國使副見辭儀高麗附金國使副見辭儀諸國朝貢



 昔周滅殷,封微子為殷後,俾修其禮物,作賓於王家,與
 國咸休。宋以柴周之後為二恪,又錄孔子之後,亦先王崇德象賢之意也,故皆為賓禮。其餘則有朝使之宴餞、歲時之廩饋及外國之使聘、遠方之朝貢,著其迓餞宴賚之式,登降揖遜之儀,備一代之制焉。



 太祖建隆元年正月四日,詔曰:「封二王之後,備三恪之賓,所以示子傳孫,興滅繼絕。夏、商之居杞、宋,周、隋之啟介、酅,古先哲王,實用茲道。矧予水京德,歷試前朝,雖周德下衰,勉從於禪讓;而虞賓在位,豈忘於烝嘗?其封周帝
 為鄭王,以奉同嗣,正朔服色,一如舊制。」又詔曰:「矧惟眇躬,逮事周室。謳歌獄訟,雖歸新造之邦;廟貌園陵,豈忘舊君之禮?其周朝嵩、慶二陵及六廟,宜令有司以時差官朝拜祭饗,永為定式。仍命周宗正卿郭□行禮。」乾德六年八月,詔於周太祖、世宗陵寢側各設廟宇塑像,命右贊善大夫王碩管勾修蓋。開寶六年三月,周鄭王殂,詔輟朝十日。帝素服發哀於便殿。十月四日,葬周恭帝於順陵,詔特輟四日、五日朝參。



 仁宗天聖六年,錄故虢
 州防禦使柴貴子肅為三班奉職。七年,錄故太子少傅柴守禮孫詠為三班奉職。其後,又錄柴氏之後曰熙、曰愈、曰若拙、曰上善並為三班奉職,曰餘慶、曰誠為州長史、助教,曰貽廓等十一人復其身,仍各賜錢一萬。又錄世宗曾孫揆、柔及貴曾孫日宣、守禮曾孫若訥皆為三班奉職。



 嘉祐四年,著作郎何鬲言:「昔舜受堯、禹受舜之天下,而封丹朱、商均以為國賓。周、漢以降,以及於唐,莫不崇奉先代,延及苗裔。本朝受周天下,而近代之盛莫
 如唐,自梁以下,皆不足以崇襲。臣願考求唐、周之裔,以備二王之後,授以爵命,封縣立廟,世世承襲,永為國賓。」事下太常議,曰:「古者立二王後,不惟繼絕,兼取其明德可法。五代草創,載祀不永,文章制度,一無可考。上取唐室,世數已遠,於經不合。惟周則我受禪之所自,義不可廢。宜訪求子孫,如孔子後衍聖公,授一京官,爵以公號,使專奉廟饗,歲時存問,賜之粟帛、牲器、祭服。每遇時祀,並從官給,其廟宇亦加嚴飾。如此,則上不失繼絕之義,
 度之於今,亦簡而易行。」從之。四月,詔曰:「先王推紹天之序,尚尊賢之義,褒其後嗣,賓以殊禮,豈非聖人稽古報功之大典哉?國家受命之元,繼周而王,雖民靈欣戴,歷數允集,而虞賓將遜,德美丕顯。頃者推命本始,褒及支庶,每遇南郊,許奏白身一名充班行,恩則厚矣,而義未稱。將上採姚、姒之舊,略循周、漢之典,詳其世嫡,優以公爵,異其仕進之路,申以土田之錫,俾廟寢有奉,饗祀不輟,庶幾乎《春秋》通三統、厚先代之制矣。宜令有司取柴
 氏譜系,於諸房中推最長一人,令歲時親奉周室祀事。如白身,即與京主簿,如為班行者,即比類換文資,仍封崇義公,與河南府、鄭州合入差遣,給公田十頃,專管勾陵廟。應緣祭饗禮料所須,皆從官給。如至知州資序,即別與差遣,卻取以次近親,令襲爵授官,永為定式。」八月,太常禮院定到內殿崇班、相州兵馬都監柴詠於柴氏諸族最長,詔換殿中丞,封崇義公,簽書奉寧軍節度判官事,以奉周祀。又以六廟在西京,而歲時祭饗無器服
 之數,令有司以三品服一、四品服二及所當用祭器給之。



 熙寧四年,西京留司御史臺司馬光言:「崇義公柴詠祭祀不以儀式。周本郭姓,世宗後侄,為郭氏後。今存周後,則宜封郭氏子孫以奉周祀。」帝閱奏,問王安石,安石曰:「宋受天下於世宗,柴氏也。」帝曰:「為人後者為之子。」安石曰:「為人後於異姓,非禮也。雖受天下於郭氏,豈可以天下之故而易其姓氏所出?」帝然之。五年正月,柴詠致仕。詠長子早亡,嫡孫夷簡當襲。太常禮院言夷簡有過,
 合以次子西頭供奉官若訥承襲。詔以若訥為衛尉寺丞,襲封崇義公,簽書河南府判官廳公事。



 政和八年,徽宗詔曰:「昔我藝祖受禪於周,嘉祐中擇柴氏旁支一名封崇義公。議者謂不當封周。然禪國者周,而三恪之封不及,禮蓋未盡。除崇義公依舊外,擇柴氏最長見在者以其祖父為周恭帝後,以其孫世世為宣義郎,監周陵廟,與知縣請給,以示繼絕之仁,為國二恪,永為定制。」



 紹興五年,詔周世宗玄孫柴叔夏為右承奉郎,襲封崇義
 公,奉周後。二十六年,叔夏升知州資序,別與差遣。以子國器襲封,令居衢州。朝廷有大禮,則入侍祠如故事。其柴大有、柴安宅亦各補官。



 淳祐九年,又以世宗八世孫柴彥穎特補承務郎,襲封崇義公。



 時又求隋、唐及朱氏、李氏、石氏、劉氏、郭氏之後,及吳越、荊南、湖南、蜀漢諸國之子孫,皆命以官,使守其祀。具見《本紀》、《世家》。



 錄先聖後,仁宗景祐二年,詔以孔子四十六世孫北海尉宗願為國子監主簿,襲封文宣公。皇祐三年七月,詔
 曰:「國朝以來,世以孔氏子孫知仙源縣,使奉承廟祀。近歲廢而不行,非所以尊先聖也。宜以孔氏子孫知仙源縣事。」



 至和初,太常博士祖無擇言:「按前史,孔子後襲封者,在漢、魏曰褒成、褒尊、宗聖,在晉、宋曰奉聖,後魏曰崇聖,北齊曰恭聖,後周、隋並封鄒國,唐初曰褒聖,開元中,始追謚孔子為文宣王。又以其後為文宣公,不可以祖謚而加後嗣。」遂詔有司定封宗願衍聖公,令世襲焉。



 治平初,用京東提點王綱言,自今勿以孔氏子弟知仙源
 縣,其襲封人如無親屬在鄉里,令常任近便官,不得去家廟。



 熙寧中,以四十八代孫若蒙為沂州新泰縣主簿,襲封。



 元祐初,朝議大夫孔宗翰辭司農少卿,請依家世例知兗州以奉祀。又言:「孔子後襲封疏爵,本為侍祠,今乃兼領他官,不在故郡。請自今襲封者無兼他職,終身使在鄉里。」朝議依所請,命官以司其用度,立學以訓其子孫,襲封者專主祠事,增賜田百頃,供祭祀之餘許均贍族人。其差墓戶並如舊法。賜書,置教授一員,教諭其
 家子弟,鄉鄰或願學者聽。改衍聖公為奉聖公,及刪定家祭冕服等制度頒賜之。其後,通直郎孔宗壽等舉若蒙弟若虛襲封,仍請自今眾議擇承襲之人,不必子繼,庶幾留意祖廟,惇睦族人。



 宣和三年,詔宣議郎孔端友襲封衍聖公,為通直郎、直秘閣,仍許就任關升,以示崇獎。端友言:詔敕文宣王後與親屬一人判司簿尉,今孔若採當承繼推恩。詔補迪功郎。



 高宗紹興二年,以四十九代孫孔玠襲封衍聖公。其後,以搢、以文遠、以萬春、以
 洙,終宋世,皆襲封主祀事。



 群臣朝覲出使宴餞之儀。太祖、太宗朝,藩鎮牧伯,沿五代舊制,入覲及被召、使回,客省繼簽賜酒食。節度使十日,留後七日,觀察使五日。代還,節度使五日,留後三日,觀察一日,防禦使、團練使、刺史並賜生料。節度使以私故到闕下,及步軍都虞候以上出使回者,亦賜酒食、熟羊。群臣出使回朝,見日,面賜酒食,中書、樞密、宣徽使、使相並樞密使伴;三司使、學士、東宮三師、僕射、御史大夫、
 節度使並宣徽使伴。兩省五品以上、侍御史、中丞、三司副使、東宮三少、尚書丞郎、卿監、上將軍、留後、觀察防禦團練使、剌史、宣慶宣政昭宣使並客省使伴;少卿監、大將軍、諸司使以下任發運轉運提點刑獄、知軍州、通判、都監、巡檢回者即賜,並通事舍人伴;客省、引進、四方館、閣門使並本廳就食。群臣賀,賜衣;奉慰,並特賜茶酒,或賜食外任遣人進奉,亦賜酒食,或生料。自十月一日後盡正月,每五日起居,百官皆賜茶酒,諸軍分校三日一
 賜。冬至、二社、重陽、寒食,樞密近臣、禁軍大校或賜宴其第及府署中,率以為常。



 大中祥符五年,詔自今兩省五品、尚書省四品、諸司三品以上官,同列出使,並許醵錢餞飲,仍休假一日。餘官有親屬僚友出行,任以休務日餞送。故事,樞密、節度使、使相還朝,咸賜宴於外苑。見辭日,長春殿賜酒五行,仍設食,當直翰林龍圖閣學士以上、皇親、觀察使預坐。八年四月,侍衛步軍副都指揮使王能自鎮定來朝,宴於長春殿。閣門言:「舊制,節度使掌
 兵,無此禮例。既赴坐,則殿前馬軍都校當侍立,於品秩非便。」遂令皆預位。



 中興,仍舊制。凡宰相、樞密、執政、使相、節度、外國使見辭及來朝,皆賜宴內殿或都亭驛,或賜茶酒,並如儀。



 時節饋廩。大中祥符五年十一月,以宰相王旦生日,詔賜羊三十口、酒五十壺、米面各二十斛,令諸司供帳,京府具衙前樂,許宴其親友。旦遂會近列及丞郎、給諫、修史屬官。俄又賜樞密使副、參知政事羊三十口,酒三十
 壺、米面各三十斛。其後,以廢務非便,奏罷會,而賜如故。又制:僕射、御史大夫、中丞、節度、留後、觀察、內客省使、權知開封府,正、至、寒食,並客省繼簽賜羊、酒、米、面;立春賜春盤;寒食神餤、餳粥;端午粽子;伏日蜜沙冰;重陽糕,並有酒;三伏日,又五日一賜冰。四廂及廂都指揮使,中書舍人,統軍,防禦、團練使,刺史,客省使,樞密都承旨,知銀臺司、審刑院,三司三司勾院,諸司使,禁軍校、忠佐,海外諸蕃進奉領刺史以上,至寒食,並賜節料;立春,奉
 內朝者皆賜幡勝。



 元祐二年十一月冬至,詔賜御筵於呂公著私第,遣中使賜上尊酒、香藥、果實、縷金花等,以御飲器勸酒,遣教坊樂工,給內帑錢賜之。及暮賜燭,傳宣令繼燭,皆異恩也。



 紹興十三年十二月二十三日,高宗賜宰臣秦檜詔曰:「省所奏辭免生日賜宴。朕聞賢聖之興必五百歲,君臣之遇蓋亦千載。夫以不世之英,值難逢之會,則其始生之日,可不為天下慶乎!式燕樂衎,所以示慶也。非喬嶽之神無以生申、甫,非宣王之能任賢無
 以致中興。今日之事,不亦臣主俱榮哉?宜服異恩,毋守沖節。所請宜不允。」



 宋朝之制,凡外國使至,及其君長來朝,皆宴於內殿,近臣及刺史、正郎、都虞候以上皆預。



 太祖建隆元年八月三日,宴近臣於廣政殿,江南、吳越朝貢使皆預。乾德三年五月十六日,宴近臣及孟昶於大明殿。開寶四年五月七日,宴近臣及劉鋹於崇德殿。十一月五日,江南李煜、吳越錢俶各遣子弟來朝,宴於崇德殿。八年三月
 晦,宴契丹使於長春殿。



 太平興國二年二月十一日,宴兩浙進奉使、契丹國信使及李煜、劉鋹、禁軍都指揮使以上於崇德殿,不舉樂,酒七行而罷。契丹遣使賀登極也。五月十一日,再宴契丹使於崇德殿,酒九行而罷,以其貢助山陵也。三年正月十六日,宴劉鋹、李煜、契丹使、諸國蕃客於崇德殿,以契丹使來賀正故也。三月二十五日,吳越錢俶來朝,宴於長春殿,親王、宰相、節度使、劉鋹、李煜皆預。十月十六日,宴宰相、親王以下及契丹使、
 高麗使、諸州進奉使於崇德殿,以乾明節罷大宴故也。是後,宴外國使為常。



 其君長來朝,先遣使迎勞於候館,使者朝服稱制曰「奉制勞某主」,國主迎於門外,與使者俱入升階,使者執束帛,稱有制,國主北面再拜稽首受幣,又再拜稽首,以土物儐,使者再拜受。國主送使者出,鴻臚引詣朝堂,所司奏聞,通事舍人承敕宣勞,再拜就館。翌日,遣使戒見日如儀。又次日,奉見於乾元殿,設黃麾仗及宮縣大樂。典儀設國主位於縣南道西北向,又
 設其國諸官之位於其後。所司迎引,國主服其國服,至明德門外,通事舍人引就位。侍中奏中嚴,皇帝服通天冠、絳紗袍,出自西房,即御位。典儀贊拜,國主再拜稽首。侍中承制降勞,皆再拜稽首,敕升坐,又再拜稽首,至坐,俯伏避席。侍中承制曰「無下拜」,國主復位。次引其國諸官以次入,就位,再拜並如上儀。侍中又承制勞還館,通事舍人引國主降,復位,再拜稽首,出。其國諸官皆再拜,以次出。侍中奏禮畢,皇帝降坐。其錫宴與受諸國使表
 及幣皆有儀,具載《開寶通禮》。



 契丹國使入聘見辭儀。自景德澶淵會盟之後,始有契丹國信使副元正、聖節朝見。大中祥符九年,有司遂定儀注。



 前一日,習儀於驛。見日,皇帝御崇德殿。宰臣、樞密使以下大班起居訖,至員僚起居後,館伴使副一班入就位,東面立。次接書匣閣門使升殿立。次通事入,不通,喝拜,兩拜,奏聖躬萬福,又喝兩拜,隨呼萬歲,喝祗候,赴東西接引使副位。舍人引契丹使副自外捧書匣入,當
 殿前立。天武官抬禮物分東西向入,列於殿下,以東為上。舍人喝天武官起居,兩拜,隨呼萬歲,奏聖躬萬福,喝各祗候。閣門從東階降,至契丹使位北。舍人揖使跪進書匣,閣門側身搢笏、跪接,舍人受之。契丹使立,閣門執笏捧書匣升殿,當御前進呈訖,授內侍都知,都知拆書以授宰臣,宰臣、樞密進呈訖,遂抬禮物出。舍人與館伴使副引契丹使副至東階下,閣門使下殿揖引同升,立御前。至國信大使傳國主問聖體,通事傳譯,舍人當御前鞠躬
 傳奏訖,揖起北使。皇帝宣閣門回問國主,北使跪奏,舍人當御前鞠躬奏訖,遂揖北使起,卻引降階至辭見位,面西揖躬。舍人當殿通北朝國信使某官某祗候見,應喏絕,引當殿,喝拜,大起居其拜舞並依本國禮,出班謝面天顏,歸位,喝拜舞蹈訖,又出班謝沿路驛館禦筵茶藥及傳宣撫問,復歸位,喝拜舞蹈訖,舍人宣有敕賜窄衣一對、金蹀躞子一、金塗銀冠一、靴一兩、衣著三百匹、銀二百兩、鞍轡馬一每句應喏,跪受,起,拜舞蹈訖,喝祗候,應喏西出凡傳
 語並奏聖躬萬福、致辭,並通事傳譯,舍人當殿鞠躬奏聞,後同。次通北朝國信副使某官某祗候見,其拜舞、謝賜、致詞並如上儀,西出其敕賜衣一對,金腰帶一,帕頭、靴、笏、衣著二百匹,銀器一百兩,鞍轡馬一。次通事及舍人引舍利已下分班入,不通,便引合班,贊喝大起居,拜舞如儀。舍人喝有敕賜衣服、束帶、衣著、銀器分物,應喏跪受,抬擔床絕,起,舞蹈拜訖,喝各祗候分班引出。次引差來通事以下從人分班入,不通,便引合班,喝兩拜,奏聖躬萬福,又拜,隨呼萬歲,喝有敕各賜衣服、腰帶、衣著、銀器分物,應
 喏跪受,起,喝兩拜,隨拜萬歲,喝各祗候唱喏分班引出。次行門、殿直入,起居訖,殿上侍立。文明殿樞密直學士、三司使、內客省使下殿。舍人合班奏報閣門無事,唱喏訖,卷班西出。客省、閣門使以下東出,其排立,供奉官已下橫行合班。宣徽使殿上喝供奉官已下各祗候分班出,並如常儀。皇帝降坐還內。



 宴日,契丹使副以下服所賜,承受引赴長春殿門外,並侍宴臣僚宰執、親王、樞密使以下祗候。俟長春殿諸司排當有備,閣門使附入內
 都知奏班齊,皇帝坐,鳴鞭,宰臣、親王以下並宰執分班,舍人引入。其契丹使副綴親王班入。舍人通某甲以下,唱喏,班首奏聖躬萬福,喝各就坐、兩拜,隨呼萬歲,喝就坐,分班引上殿。或皇帝撫問契丹使副,舍人便引下殿,喝兩拜,隨拜萬歲,喝各就坐。次舍人、通事分引舍利以下東西分班,喝兩拜,喝就坐,分引赴兩廊下。次舍人引差來通事、從人東西分班入,合班,喝兩拜,隨拜萬歲,喝就坐,分引赴兩廊。次喝教坊已下兩拜,班首奏聖躬萬福,
 又喝拜,兩拜,隨拜萬歲,喝各祗候。次引看盞二人稍近前,喝拜,兩拜,隨拜萬歲,喝上殿祗候,分東西上殿立。有司進茶床,內侍酹酒,訖,閣門使殿上御前鞠躬奏某甲已下進酒,餘如常儀。宴起,宰臣已下降階,舍人喝兩拜,搢笏,舞蹈,喝各祗候,分班出。次舍利合班,喝兩拜,舞蹈,三拜,拜謝訖,喝各祗候分引出。次通事、從人合班,喝兩拜,隨拜萬歲,喝各祗候,分班引出。次喝教坊使已下兩拜,隨拜萬歲,喝各祗候。如傳宣賜茶酒,又喝謝茶酒拜,
 兩拜,隨拜萬歲,喝各祗候,出。閣門使殿上近前側奏無事,皇帝降坐,鳴鞭還內。



 辭日,皇帝坐,內殿起居班欲絕,諸司排當有備,催合侍宴臣僚東西相向,班立崇德殿庭。俟奏班齊,舍人喝拜,東西班殿侍兩拜,奏聖躬萬福,喝各祗候。次舍人通館伴使副某甲以下常起居,次通契丹使某甲常起居,次通副使某甲常起居,俱引赴西面立。次通宰臣以下橫行,通某甲以下,應喏,奏聖躬萬福,喝各就坐,應喏,兩拜呼萬歲,分升殿東西向立。次通
 事、舍人引契丹舍利以下,次差來通事、從人俱分班入,當殿兩拜,奏聖躬萬福,喝各就坐,兩拜,呼萬歲,分引赴兩廊立。次通教坊使、看盞。及進茶床、酹酒並閣門奏進酒,並如長春宴日之儀。酒五巡,起。宰臣以下降階班立,兩拜、搢笏、舞蹈,三拜,喝各祗候。宰臣以下並三司使、文明殿學士、樞密直學士升殿侍立,其餘臣僚並契丹使並出。次引舍利及差來從人,俱兩拜萬歲訖,分班引出。如傳宣賜茶酒,更喝謝拜如前儀。已上班絕,舍人再引
 契丹使入,西面揖躬。舍人當殿通北朝國信使某祗候辭,通訖,引當殿兩拜,出班致辭,歸位,又兩拜訖,宣有敕賜,跪受拜舞訖,喝好去,遂引出。次引副使致詞、受賜、拜舞如前儀,亦出。次引舍利已下,次引差來通事、從人,俱分班入,舍人喝有敕賜衣服、衣著、銀器分物,各應喏跪受,候抬擔床絕,就拜,起,又兩拜萬歲,喝好去,分班引出。其使副各服所賜,再引入,當殿兩拜萬歲訖,喝祗候,引升殿,當御前立。皇帝宣閣門使授旨傳語國主,舍人揖
 國信使跪,閣門使傳旨通譯訖,揖國信使起立,閣門使御前搢笏,於內侍都知處奉授書匣,舍人揖國信使跪,閣門使跪分付訖,揖起下殿,西出。



 政和詳定五禮,有《紫宸殿大遼使朝見儀》、《紫宸殿正旦宴大遼使儀》、《紫宸殿大遼使朝辭儀》、《崇政殿假日大遼使朝見儀》、《崇政殿假日大遼使朝辭儀》。其紫宸殿赴宴,遼使副位御坐西,諸衛上將軍之南。夏使副在東朵殿,並西向北上。高麗、交址使副在西朵殿,並東向北上,遼使舍利、從人各在其
 南。夏使從人在東廊舍利之南,諸蕃使副首領、高麗交址從人、溪峒衙內指揮使在西廊舍利之南。又至各就位,有分引兩廊班首詣御坐進酒,樂作,贊各賜酒,群官俱再拜就坐。酒五行,皆作樂賜華,皇帝再坐,赴宴官行謝華之禮。



 夏國進奉使見辭儀。夏國歲以正旦、聖節入貢。元豐八年,使來。詔夏國見辭儀制依嘉祐八年,見於皇儀殿門外,朝辭詣垂拱殿。



 政和新儀:夏使見日,俟見班絕、謝班
 前,使奉奉表,引入殿庭,副使隨入,西向立,舍人揖躬。舍人當殿躬奏夏國進奉使姓名以下祗候見,引當殿前跪進表函,舍人受之,副入內侍省官進呈。使者起,歸位,四拜起居。舍人宣有敕賜某物,兼賜酒饌。跪受,箱過,俯伏興,再拜。舍人曰各祗候,揖西出。次從人入,不奏,即引當殿四拜起居。舍人宣賜分物,兼賜酒食。跪受,箱過,俯伏興,再拜。舍人曰各祗候,揖西出。辭日,引使副入殿庭,西向立,舍人揖躬。舍人當殿躬奏夏國進奉使姓名以
 下祗候辭,引當殿四拜。宣賜某物酒饌,再拜如見儀。凡蕃使見辭,同日者,先夏國,次高麗,次交址,次海外蕃客,次諸蠻。



 高麗進奉使見辭儀。見日,使捧表函,引入殿庭,副使隨入,西向立,舍人鞠躬,當殿前通高麗國進奉使姓名以下祗候見,引當殿,使稍前跪進表函,俯伏興訖,歸位大起居。班首出班躬謝起居,歸位,再拜,又出班謝面天顏、沿路館券、都城門外茶酒,歸位,再拜,搢笏,舞蹈,俯伏興,
 再拜。舍人宣有敕賜某物兼賜酒食,搢笏,跪受,箱過,俯伏興,再拜。舍人曰各祗候,揖西出。次押物以下入,不通,即引當殿四拜起居。宣有敕賜某物兼賜酒食,跪受,箱過,俯伏興,再拜起居。舍人曰各祗候,揖西出。



 辭日,引使副入殿庭,西向立,舍人揖躬。舍人當殿躬通高麗進奉使姓名以下祗候辭,引當殿四拜起居。班首出、班致詞,歸位,再拜。舍人宣有敕賜某物兼賜酒食,搢笏,跪受,箱過,俯伏興,再拜。舍人曰好去,揖西出。次從人入辭,如見。



 政和元年,詔高麗在西北二國之間,自今可依熙寧十年指揮隸樞密院。明年入貢,詔復用熙寧例,以文臣充接伴使副,仍往還許上殿。七年,賜以籩豆各十二,簠簋各四,登一,鉶二,鼎二,罍洗一,尊二。銘曰:「惟爾令德孝恭,世稱東蕃,有來顯相,予一人嘉之。用錫爾寶尊,以寧爾祖考。子子孫孫,其永保之!」紹興二年,高麗遣使副來貢,並賜酒食於同文館。



 金國聘使見辭儀。宣和元年,金使李善慶等來,遣直秘
 閣趙有開偕善慶等報聘。已而金使復至,用新羅使人禮,引見宣政殿,徽宗臨軒受使者書。自後屢遣使來,帝待之甚厚,時引上殿奏事,賜予不貲,禮遇並用契丹故事。



 紹興三年十二月,宰臣進呈金使李永壽等正旦入見。故事,百官俱入。上曰:「全盛之時,神京會同,朝廷之尊,百官之富,所以誇示。今暫駐於此。事從簡便。舊日禮數,豈可盡行?無庸俱入。」使人見辭,並賜食於殿門外。八年,金國遣使副來,就驛議和。詔王倫就驛賜宴。十一年十
 一月,金國遣審議使來。入見,時殿陛之儀議猶未決。議者謂「兵衛單弱,則非所以隆國體;欲設仗衛,恐駭虜情。」乃設黃麾仗千五百人於殿廊,蔽以帟幕,班定徹帷。十二年,扈從徽宗梓宮、皇太后使副來。十三年十一月,有司言:「賀正旦使初至,於盱眙軍賜宴。未審回程合與不合筵待?」詔內侍省差使臣二員沿路賜御筵,一員於平江府,一員於鎮江府,一員於盱眙軍。尋詔:金國賀正旦人使到闕赴宴等坐次,令與宰臣相對稍南。使副上下
 馬於執政官上下馬處。三節人從並於宮門外上下馬。立班則於西班,與宰臣相對立。仍權移西班使相在東壁宰臣之東。十四年正月一日,宴金國人使於紫宸殿。文臣權侍郎已上、武臣刺史已上赴坐。自後正旦賜宴仿此。五月,金國始遣賀天申節使來。有司言合照舊例:北使賀生辰聖節使副隨宰臣紫宸殿上壽,進壽酒畢,皇帝、宰臣以下同使副酒三行,教坊作樂,三節人從不赴。既而三節人從有請,乞隨班上壽,詔許之,仍賜酒食。
 遇賀正,人使朝辭在上辛祠官致齋之內,仍用樂。二十九年,以皇太后崩,其賀正使副止就驛賜宴。見辭日,賜茶酒,並不舉樂。



 大率北使至闕,先遣伴使賜御筵於班荊館在赤岸,去府五十里,酒七行。翌日登舟,至北郭稅亭,茶酒畢,上馬入餘杭門,至都亭驛,賜褥被、金沙鑼等。明日,臨安府書送酒食,閣門官入位,具朝見儀,投朝見榜子。又明日,入見。伴使至南宮門外下馬,北使至隔門內下馬。皇帝御紫宸殿,六參官起居。北使見畢,退赴客省茶酒,遂宴
 垂拱殿,酒五行,惟從官已上預坐。是日,賜茶器名果。又明日,賜生餼。見之二日,與伴使偕往天竺燒香,上賜沉香、乳糖、齋筵、酒果。次至冷泉亭、呼猿洞而歸。翌日,賜內中酒果、風藥、花餳,赴守歲夜筵,酒五行,用傀儡。正月朔旦,朝賀禮畢,上遣大臣就驛賜御筵。中使傳旨宣勸,酒九行。三日,客省簽賜酒食,內中賜酒果。遂赴浙江亭觀潮,酒七行。四日,赴玉津園燕射,命諸校善射者假管軍觀察使伴之,上賜弓矢。酒行樂作,伴射官與大使並射
 弓,館伴、副使並射弩。酒九行,退。五日,大宴集英殿,尚書郎、監察御史已上皆預,學士撰致語。六日,朝辭退,賜襲衣、金帶、大銀器。臨安府書送贐儀。復遣執政官就驛賜宴。晚赴解換夜筵,伴使與北使皆親勸酬,且以衣物為侑。次日,加賜龍鳳茶、金鍍合。乘馬出北闕門登舟,宿赤岸。又次日,復遣近臣押賜御筵。



 自到闕朝見、燕射、朝辭,共賜大使金千四百兩,副使金八百八十兩,衣各三襲,金帶各三條。都管上節各賜銀四十兩,中下節各三十
 兩,衣一襲、塗金帶一條。使人到闕筵宴,凡用樂人三百人,百戲軍七十人,築球軍三十二人,起立球門行人三十二人,旗鼓四十人,並下臨安府差;相撲一十五人,於御前等子內差,並前期教習之。



 諸國朝貢,其交州、宜州、黎州諸國見辭,並如上儀。惟迓勞宴賚之數,則有殺焉。其授書皆令有司付之。又有西蕃唃氏、西南諸蕃占城、回鶻、大食、于闐、三佛齊、邛部川蠻及溪峒之屬,或比間數歲入貢。層檀、日本、大理、注輦、
 蒲甘、龜茲、佛泥、拂菻、真臘、羅殿、渤泥、邈黎、闍婆、甘眉流諸國入貢,或一再,或三四,不常至。注輦、三佛齊使者至,以真珠、龍腦、金蓮花等登陛跪散之,謂之「撒殿」。



 元祐二年,知穎昌府韓縝言:「交址小國,其使人將及境,臣嘗近弼,難以抗禮。按元豐中迓以兵官,餞以通判,使副詣府,其犒設令兵官主之。請如故事。」仍詔所過郡,凡前宰相、執政官知判者亦如之。又詔立回賜於闐國信分物法。歲遣貢使雖多,止一加賜。又命於闐國使以表章至,則
 間歲聽一入貢,餘令於熙、秦州貿易。



 禮部言:「元豐著令,西南五姓蕃,每五年許一貢。今西南蕃泰平軍入貢,期限未及。」詔特許之。學士院言:「諸蕃初入貢者,請令安撫、鈐轄、轉運等司體問其國所在遠近大小,與見今入貢何國為比,保明聞奏,庶待遇之禮不致失當。」宣和詔蕃國入貢,令本路驗實保明。如涉詐偽,以上書詐不實論。



 建炎三年,占城國王遣使進貢,適遇大禮,遂加恩,特授檢校少傅,加食邑。自後明堂郊祀,並仿此。紹興二年,占城
 國王遣使貢沉香、犀、象、玳瑁等,答以綾錦銀絹。



 建炎四年,南平王薨,差廣南西路轉運副使尹東□充吊祭使,賜絹布各五百匹,羊、酒、寓錢、寓彩、寓金銀等,就欽州授其國迎接人,制贈侍中,進封南越王。封其子為交址郡王,遇大禮,並加恩如占城國王。淳熙元年,賜「安南國王」印,銅鑄,塗以金。



 紹興七年,三佛齊國乞進章奏赴闕朝見,詔許之。令廣東經略司斟量,只許四十人到闕,進貢南珠、象齒、龍涎、珊瑚、琉璃、香藥。詔補保順慕化大將軍、
 三佛齊國王,給賜鞍馬、衣帶、銀器。賜使人宴於懷遠驛。淳熙五年,再入貢。計其直二萬五千緡,回賜綾錦羅絹等物、銀二千五百兩。



 紹興三十一年正月,安南獻馴象。帝曰:「蠻夷貢方物乃其職,但朕不欲以異獸勞遠人。其令帥臣告諭,自今不必以馴象入貢。」三十二年,孝宗登極,詔曰:「比年以來,累有外國入貢,太上皇帝沖謙弗受,況朕水京菲,又何以堪!自今諸國有欲朝貢者,令所在州軍以理諭遣,毋得以聞。」淳祐三年,安南國主陳日煚來
 貢,加賜功臣號。十一年,再來貢。景定三年六月,日煚上表貢獻,乞授其位於其子陳威晃。咸淳元年二月,加安南大國王陳日煚功臣,增「安善」二字;安南國王陳威晃功臣,增「守義」二字,各賜金帶、鞍馬、衣服。二年,復上表進貢禮物,賜金五百兩,賜帛一百匹,降詔嘉獎。



\end{pinyinscope}