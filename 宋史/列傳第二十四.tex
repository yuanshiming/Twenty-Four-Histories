\article{列傳第二十四}

\begin{pinyinscope}

 李昉
 子宗訥宗諤孫昭述等呂蒙正張齊賢子宗誨賈黃中



 李昉,字明遠,深州饒陽人。父超,晉工部郎中、集賢殿直學士。從大父右資善大夫沼無子,以昉為後,蔭補齋郎,
 選授太子校書。漢乾祐舉進士,為秘書郎。宰相馮道引之,與呂端同直弘文館,改右拾遺、集賢殿修撰。



 周顯德二年,宰相李谷征淮南,昉為記室。世宗覽軍中章奏,愛其辭理明白,已知為昉所作,及見《相國寺文英院集》,乃昉與扈蒙、崔頌、劉袞、竇儼、趙逢及昉弟載所題,益善昉詩而稱賞之曰:「吾久知有此人矣。」師還,擢為主客員外郎、知制誥、集賢殿直學士。四年,加史館修撰、判館事。是年冬,世宗南征,從至高郵,會陶穀出使,內署書詔填委,
 乃命為屯田郎中、翰林學士。六年春,丁內艱。恭帝嗣位,賜金紫。



 宋初,加中書舍人。建隆三年,罷為給事中。四年,平湖湘,受詔祀南嶽,就命知衡州,逾年代歸。陶穀誣奏昉為所親求京畿令,上怒,召吏部尚書張昭面質其事。昭老儒,氣直,免冠上前,抗聲云:「翽罔上。」上疑之不釋,出昉為彰武軍行軍司馬,居延州為生業以老。三歲當內徙,昉不願。宰相薦其可大用,開寶二年,召還,復拜中書舍人。未幾,直學士院。



 三年,知貢舉。五年,復知貢舉。
 秋,預宴大明殿,上見昉坐盧多遜下,因問宰相,對曰:「多遜學士,昉直殿爾。」即令真拜學士,令居多遜上。昉之知貢舉也,其鄉人武濟川預選,既而奏對失次,昉坐左遷太常少卿,俄判國子監。明年五月,復拜中書舍人、翰林學士。冬,判吏部銓。時趙普為多遜所構,數以其短聞於上,上詢於昉,對曰:「臣職司書詔,普之所為,非臣所知。」普尋出鎮,多遜遂參知政事。



 太宗即位,加昉戶部侍郎,受詔與扈蒙、李穆、郭贄、宋白同修《太祖實錄》。從政太原,車
 駕次常山,常山即昉之故里,因賜羊酒,俾召公侯相與宴飲盡歡,里中父老及嘗與游從者咸預焉。七日而罷,人以為榮。師還,以勞拜工部尚書兼承旨。太平興國中,改文明殿學士。時趙普、宋琪居相位久,求其能繼之者,宿舊無逾於昉,遂命參知政事。十一月,普出鎮,昉與琪俱拜平章事。未幾,加監修國史,復時政記先進御而後付有司,自昉議始也。



 雍熙元年郊祀,命昉與琪並為左右僕射,昉固辭,乃加中書侍郎。王師討幽薊不利,遣使
 分詣河南、東,籍民為兵,凡八丁取一。昉等相率奏曰:「近者分遣使籍河南、東四十餘郡之民以為邊備,非得已也。然河南之民素習農桑,罔知戰鬥,一旦括集,必致動搖,若因而嘯聚,更須剪除。如此,則河北閭閻既困於戎馬,河南生聚復擾於萑蒲,矧當春和,有妨農作。陛下若以明詔既頒,難於反汗,則當續遣使臣,嚴加戒飭,所至點募,人情若有不安,即須少緩,密奏取裁,庶免後患。」上嘉納之。



 端拱初,布衣翟馬周擊登聞鼓,訟昉居宰相位,
 當北方有事之時,不為邊備,徒知賦詩宴樂。屬籍田禮方畢,乃詔學士賈黃中草制,罷昉為右僕射,且加切責。黃中言:「僕射,百僚師長,實宰相之任,今自工部尚書而遷是職,非黜責也。若曰文昌務簡,以均勞逸為辭,斯為得體。」上然之。會邊警益急,詔文武群臣各進策備御,昉又引漢、唐故事,深以屈己修好、弭兵息民為言,時論稱之。



 淳化二年,復以本官兼中書侍郎、平章事,監修國史。三年夏,旱蝗,既雨。時昉與張齊賢、賈黃中、李沆同居宰
 輔,以燮理非材,上表待罪,上不之罪。四年,昉以私門連遭憂戚,求解機務,詔不允,遣齊賢等諭旨,復起視事。後數月,罷為右僕射。先是,上召張洎草制,授昉左僕射,罷相,洎言:「昉居燮理之任,而陰陽乖戾,不能決意引退,俾居百僚師長之任,何以示勸?」上覽奏,乃令罷守本官。



 晉侍中崧者,與昉同宗且同里,時人謂崧為東李家,昉為西李家。漢末,崧被誅。至是,其子璨自蘇州常熟縣令赴調,昉為訟其父冤,且言:「周太祖已為昭雪贈官,還其田
 宅,錄璨而官之。然璨年幾五十,尚淹州縣之職,臣昔與之同難,豈宜叨遇聖明。儻推一視之仁,澤及衰微之祚,則已往之冤獲伸於下,而繼絕之恩永光簡冊矣。」詔授璨著作佐郎,後官至右贊善大夫。



 明年,昉年七十,以特進、司空致事,朝會宴饗,令綴宰相班,歲時賜予,益加厚焉。至道元年正月望,上觀燈乾元樓,召昉賜坐於側,酌御樽酒飲之,自取果餌以賜。上觀京師繁盛,指前朝坊巷省署以諭近臣,令拓為通衢長廊,因論:「晉、漢君臣昏
 暗猜貳,枉陷善良,時人不聊生,雖欲營繕,其暇及乎?」昉謂:「晉、漢之事,臣所備經,何可與聖朝同日而語。若今日四海清晏,民物阜康,皆陛下恭勤所致也。」上曰:「勤政憂民,帝王常事。朕不以繁華為樂,蓋以民安為樂爾。」因顧侍臣曰:「李昉事朕,兩人中書,未嘗有傷人害物之事,宜其今日所享如此,可謂善人君子矣。」



 二年,陪祀南郊,禮畢入賀,因拜舞僕地,臺吏掖之以出,臥疾數日薨,年七十二。贈司徒,謚文正。



 昉和厚多恕,不念舊惡,在位小心
 循謹,無赫赫稱。為文章慕白居易,尤淺近易曉。好接賓客,江南平,士大夫歸朝者多從之游。雅厚張洎而薄張佖,及昉罷相,洎草制深攻詆之,而佖朔望必詣昉。或謂佖曰:「李公待君素不厚,何數詣之?」佖曰:「我為廷尉日,李公方秉政,未嘗一有請求,此吾所以重之也。」



 昉所居有園亭別墅之勝,多召故人親友宴樂其中。既致政,欲尋洛中九老故事,時吏部尚書宋琪年七十九,左諫議大夫楊徽之年七十五,郢州刺史魏丕年七十六,太常少
 卿致仕李運年八十,水部郎中朱昂年七十一,廬州節度副使武允成年七十九,太子中允致仕張好問年八十五,吳僧贊寧年七十八,議將集,會蜀寇而罷。



 昉素與盧多遜善,待之不疑,多遜屢譖昉於上,或以告昉,不之信。及入相,太宗言及多遜事,昉頗為解釋。帝曰:「多遜居常毀卿一錢不直。」昉始信之。上由此益重昉。



 昉居中書日,有求進用者,雖知其材可取,必正色拒絕之,已而擢用;或不足用,必和顏溫語待之。子弟問其故,曰:「用賢,人
 主之事;若受其請,是市私恩也,故峻絕之,使恩歸於上。若不用者,既失所望,又無善辭,取怨之道也。」



 初,超未有子,昉母謝方娠,指腹謂叔母張曰:「生男當與叔母為子。」故昉出繼於超。昉再相,因表其事,求贈所生父母官。詔贈其祖溫太子太傅,祖母權氏莒國太夫人,超太子太師,謝氏鄭國太夫人。



 昉素病心悸,數歲一發,發必彌年而後愈,蓋典誥命三十餘年,勞役思慮所致。及居相位,益加憂畏。有文集五十卷。子四人:宗訥、宗誨、宗諤、宗諒。
 宗誨,右贊善大夫。宗諒,主賓客員外郎。



 宗訥字大辨,以蔭補太廟齋郎,遷第四室長。代謁吏部銓,邊光範意其年少,未能屬辭,語之曰:「茍援筆成六韻詩,雖不試書判,可入等矣。」宗訥易之,光範試詩賦,立就。明日,遂擬授秘書省正字;又明日,上命擢國子監丞。蓋上居藩邸時,每有篇詠,令昉屬和,前後數百章,皆宗訥繕寫,上愛其楷麗,問知為宗訥所書,故有是命。太平興國初,詔賈黃中集《神醫普救方》,宗訥暨劉錫、吳淑、呂文
 仲、杜鎬、舒雅皆預焉。雍熙初,昉在相位,上欲命宗訥為尚書郎,昉懇辭,以為非承平故事,止改秘書丞,歷太常博士。



 宗訥頗習典禮。淳化中,呂端掌禮院,引宗訥同判,累遷比部郎中。咸平六年,卒,年五十五。子昭回,大中祥符五年獻文,召試賜進士第,後為屯田員外郎。昭遜,太子中舍。



 宗諤字昌武,七歲能屬文,恥以父任得官,獨由鄉舉,第進士,授校書郎。明年,獻文自薦,遷秘書郎、集賢校理、同
 修起居注。先是,後苑陪宴,校理官不與,京官乘馬不得入禁門。至是,皆因宗諤之請復之,遂為故事。



 真宗即位,拜起居舍人,預重修《太祖實錄》。從幸大名,上疏曰:「國家馭邊之術,制勝之謀,將帥之短長,兵衛之眾寡,宸算廟謨,盡在吾術中矣。今之言事者,不過請陛下益兵貯糧,分道掩殺,言之甚易,行之則難。始受命則無不以攻堅陷陣為壯圖,及遇敵則惟以閉壘塞關為上計,孤君父之重委,致生靈之重困,興言及此,誠可嘆息。自古行軍
 出師,無不首擇將帥。夫將帥隨材任使,守一郡,控一城,分領驍勇,爭據要害,又豈直三路主帥之名,然後能制六師生死之命乎?今陛下選任非不至也,權位非不重也,告戒非不丁寧也,處置非不專一也;而外敵犯塞,車駕親征,曾不聞出一人一騎為之救助,不知深溝高壘,秣馬厲兵,欲安用哉?臣以為臨軍易帥,拔卒為將,在此時也。有功者拔於朝,不用者戮於市,亦此時也。惟陛下圖之。然後下哀痛之詔,行蠲復之恩。回鸞上都,垂衣當
 寧,豈不盛哉。」



 遷知制誥、判集賢院,纂《西垣集制》,刻石記名氏。嘗牒御史臺不平空,中丞呂文仲移文詰之,往復再三。宗諤執言兩省故事與臺司不相統攝者凡八。事聞,卒如宗諤議。



 景德二年,召為翰林學士。是秋,將郊,命判太常大樂、鼓吹二署。先是,樂工率以年勞遷補,至有抱其器而不知聲者。宗諤素曉音律,遂加審定,奏斥謬濫者五十人。因修完器具,更署職名,條上利病二十事,帝省閱而賞嘆之。事具《樂志》。又著《樂纂》以獻,命付史
 館,自是月再肄習焉。



 時諸神祠壇多闕外壝之制,因深塹列樹以表之,營葺齋室,舊典因以振起。屬契丹遣使來賀承天節,詔宗諤為館伴使,自郊勞至飲餞,皆刊定其儀。



 大中祥符初,從封泰山,改工部郎中。二年,始建昭應宮,命副丁謂為同修宮使。三年,知審官院。屬祀汾陰后土,命為經度制置副使,同權河中府事。禮成,優拜右諫議大夫。



 嘗侍宴玉宸殿,上謂曰:「聞卿至孝,宗族頗多,長幼雍睦。朕嗣守二聖基業,亦如卿之保守門戶也。」又曰:「
 翰林,清華之地,前賢揚歷,多有故事,卿父子為之,必周知也。」宗諤嘗著《翰林雜記》,以紀國朝制度,明日上之。



 宗諤究心典禮,凡創制損益,靡不與聞。修定皇親故事、武舉武選入官資敘、閣門儀制、臣僚導從、貢院條貫,餘多裁正。



 五年,迎真州聖像,副丁謂為迎奉使。五月,以疾卒,年四十九。帝甚悼之,謂宰相曰:「國朝將相家能以聲名自立,不墜門閥,唯昉與曹彬家爾。宗諤方期大用,不幸短命,深可惜也。」既厚賻其家,以白金賜其繼母,又錄其
 子若弟以官焉。



 初,昉居三館、兩制之職,宗諤不數年,皆踐其地。風流儒雅,藏書萬卷。內行淳至,事繼母符氏以孝聞。二兄早世,奉嫂字孤,恩禮兼盡。與弟宗諒友愛尤至,覃恩所及,必先群從,及歿而己子有未仕者。程宿早卒,有弟無所依,宗諤為表請於朝而官之。勤接士類,無賢不肖,恂恂盡禮,獎拔後進,唯恐不及,以是士人皆歸仰之。



 宗諤工隸書。有文集六十卷,《內外制》三十卷。嘗預修《續通典》、《大中祥符封禪汾陰記》、《諸路圖經》,又作《家傳》、《
 談錄》,並行於世。子昭遹、昭述、昭適。



 昭述字仲祖,以父蔭為秘書省校書郎。召試學士院,賜進士出身,為刑部詳覆官,累遷秘書丞。群牧制置使曹利用薦為判官,鄆州牧地侵於民者凡數千頃,昭述悉復之。以太常博士知開封縣,特遷尚書屯田員外郎、開封推官。坐嘗被曹利用薦,出知常州,遷為三司度支判官,改河北轉運使。江陵屯兵喧言倉粟陳腐,欲以動眾。昭述取以為奉,且以飯其僚屬,眾遂定。



 徙湖南潭州,戍
 卒憤監軍酷暴,欲構亂,或指昭述謂曰:「如李公長者,何可負?」其謀遂寢。昭述聞之,以戒監軍。監軍自是不復為暴。比去,眾遮道羅拜,指妻子曰:「向非公,無□類矣。」



 徙淮南轉運使兼發運使,加直史館。徙陜西轉運使,糾察在京刑獄,為三司戶部副使,累遷刑部郎中。陜西用兵,提點陜西計置糧草,還授度支、鹽鐵副使,以右諫議大夫為河北都轉運使。



 河決澶淵,久未塞。會契丹遣劉六符來,乃命昭述城澶州,以治堤為名,調兵農八萬,逾旬而
 就。初,六符過之,真以為堤也,及還而城具,甚駭愕。初置義勇軍,人情洶洶,昭述乘疾置日行數舍,開諭父老,眾始安。宣撫使表其能,除龍圖閣直學士、知澶州,又為樞密直學士、陜西都轉運使。



 河北始置四路,以為真定府路安撫使、知成德軍。大水,民多流亡,籍僧舍積粟為粥糜,活饑民數萬計。改龍圖閣學士、知秦州。諫官、御史言昭述庸懦,不可負重鎮,留真定府。居四年,入領三班院,以翰林侍讀學士知鄭州。未幾,知通進銀臺司,判太常
 寺,復領三班,累遷尚書右丞。從祫享致齋於朝堂,得暴疾卒。贈禮部尚書,謚恪。



 李氏居京城北崇慶裏,凡七世不異爨,至昭述稍自豐殖,為族人所望,然家法亦不隳。



 昭遘字逢吉,宗諤從子也,以蔭為將作監主簿。



 幼時,楊億嘗過其家,出拜,億命為賦,既成,億曰:「桂林之下無雜木,非虛言也。」其後薦之,召試,授館閣校勘,改集賢院校理。坐失誤落秩。未幾,復為鹽鐵判官。



 初,議罷天下職田及公使錢,昭遘以為不可。三司使姚仲孫惡其異己,請
 詰所以興利之實,昭遘爭不屈,遂罷判官,為白波發運使。因入奏事,仁宗謂曰:「前所論罷職田等事,卿言是也。」遷直史館、知陜州。諫官歐陽修言:「陜府,關中要地,昭遘無治劇材,不宜遣。」改判三司理欠司,徙度支判官。



 使契丹還,道除陜西轉運使。坐家僮盜遼人銀酒杯,降知澤州。陽城冶鑄鐵錢,民冒山險輸礦炭,苦其役,為奏罷鑄錢。又言:「河東鐵錢真偽淆雜,不可不革。」



 後復直史館、知陜州。城中舊無井,唐武德中,刺史長孫操始疏廣濟渠
 水入城,眾賴其利。昭遘至,立廟祠之。歸為三司戶部判官,糾察在京刑獄,進直龍圖閣,改集賢殿修撰,累遷尚書工部郎中。歷知鳳翔、河中府、晉州,遷管勾登聞檢院。擢天章閣待制、知滄州,用諫官吳及言,復改知陜州,徙鄭州卒。昭遘性和易,不忤物,能守家法。



 呂蒙正字聖功,河南人。祖夢奇,戶部侍郎。父龜圖,起居郎。蒙正,太平興國二年擢進士第一,授將作監丞,通判升州。陛辭,有旨,民事有不便者,許騎置以聞,賜錢二十
 萬。代還,會征太原,召見行在,授著作郎、直史館,加左拾遺。五年,親拜左補闕、知制誥。



 初,龜圖多內寵,與妻劉氏不睦,並蒙正出之,頗淪躓窘乏,劉誓不復嫁。及蒙正登仕,迎二親,同堂異室,奉養備至。龜圖旋卒,詔起復。未幾,遷都官郎中,入為翰林學士,擢左諫議大夫、參知政事,賜第麗景門。上謂之曰:「凡士未達,見當世之務戾於理者,則怏怏於心;及列於位,得以獻可替否,當盡其所蘊,雖言未必盡中,亦當僉議而更之,俾協於道。朕固不以
 崇高自恃,使人不敢言也。」蒙正初入朝堂,有朝士指之曰:「此子亦參政耶?」蒙正陽為不聞而過之。同列不能平,詰其姓名,蒙正遽止之曰:「若一知其姓名,則終身不能忘,不若毋知之為愈也。」時皆服其量。



 李昉罷相,蒙正拜中書侍郎兼戶部尚書、平章事,監修國史。蒙正質厚寬簡,有重望,以正道自持。遇事敢言,每論時政,有未允者,必固稱不可,上嘉其無隱。趙普開國元老,蒙正後進,歷官一紀,遂同相位,普甚推許之。俄丁內艱,起復。



 先是,盧
 多遜為相,其子雍起家即授水部員外郎,後遂以為常。至是,蒙正奏曰:「臣忝甲科及第,釋褐止授九品京官。況天下才能,老於巖穴,不沾寸祿者多矣。今臣男始離襁褓,膺此寵命,恐罹陰譴,乞以臣釋褐時官補之。」自是宰相子止授九品京官,遂為定制。



 朝士有藏古鏡者,自言能照二百里,欲獻之蒙正以求知。蒙正笑曰:「吾面不過楪子大,安用照二百里哉?」聞者嘆服。



 淳化中,右正言宋抗上疏忤旨,抗,蒙正妻族,坐是罷為吏部尚書,復相
 李昉。四年,昉罷,蒙正復以本官入相。因對,論及征伐,上曰:「朕比來征討,蓋為民除暴,茍好功黷武,則天下之人熸亡盡矣。」蒙正對曰:「隋、唐數十年中,四征遼碣,人不堪命。煬帝全軍陷沒,太宗自運土木攻城,如此卒無所濟。且治國之要,在內修政事,則遠人來歸,自致安靜。」上韙之。



 嘗燈夕設宴,蒙正侍,上語之曰:「五代之際,生靈凋喪,周太祖自鄴南歸,士庶皆罹剽掠,下則火災,上則彗孛,觀者恐懼,當時謂無復太平之日矣。朕躬覽庶政,萬事粗
 理,每念上天之貺,致此繁盛,乃知理亂在人。」蒙正避席曰:「乘輿所在,士庶走集,故繁盛如此。臣嘗見都城外不數里,饑寒而死者甚眾,不必盡然。願陛下視近以及遠,蒼生之幸也。」上變色不言。蒙正侃然復位,同列多其直諒。



 上嘗欲遣人使朔方,諭中書選才而可責以事者,蒙正退以名上,上不許。他日,三問,三以其人對。上曰:「卿何執耶?」蒙正曰:「臣非執,蓋陛下未諒爾。」固稱:「其人可使,餘人不及。臣不欲用媚道妄隨人主意,以害國事。」同列悚
 息不敢動。上退謂左右曰:「蒙正氣量,我不如。」既而卒用蒙正所薦,果稱職。



 至道初,以右僕射出判河南府兼西京留守。蒙正至洛,多引親舊歡宴,政尚寬靜,委任僚屬,事多總裁而已。



 真宗即位,進左僕射。會營奉熙陵,蒙正追感先朝不次之遇,奉家財三百餘萬以助用。葬日,伏哭盡哀,人以為得大臣體。咸平四年,以本官同平章事、昭文館大學士。國朝以來三入相者,惟趙普與蒙正焉。郊祀禮成,加司空兼門下侍郎。六年,授太子太師,封蔡
 國公,改封隨,又封許。



 景德二年春,表請歸洛。陛辭日,肩輿至東園門,命二子掖以升殿,因言:「遠人請和,弭兵省財,古今上策,惟願陛下以百姓為念。」上嘉納之,因遷從簡太子洗馬,知簡奉禮郎。蒙正至洛,有園亭花木,日與親舊宴會,子孫環列,迭奉壽觴,怡然自得。大中祥符而後,上朝永熙陵,封泰山,祠后土,過洛,兩幸其第,錫賚有加。上謂蒙正曰:「卿諸子孰可用?」對曰:「諸子皆不足用。有侄夷簡,任穎州推官,宰相才也。」夷簡由是見知於上。



 富
 言者,蒙正客也。一日白曰:「兒子十許歲,欲令入書院,事廷評、太祝。」蒙正許之。及見,驚曰:「此兒他日名位與吾相似,而勛業遠過於吾。」令與諸子同學,供給甚厚。言之子,即弼也。後弼兩入相,亦以司徒致仕。其知人類如此。



 許國之命甫下而卒,年六十八。贈中書令,謚曰文穆。



 蒙正初為相時,張紳知蔡州,坐贓免。或言於上曰:「紳家富,不至此,特蒙正貧時勾索不如意,今報之爾。」上命即復紳官,蒙正不辨。後考課院得紳實狀,復黜為絳州團練副
 使。及蒙正再入相,太宗謂曰:「張紳果有贓。」蒙正不辨亦不謝。在西京日,上數遣中貴人將命至,蒙正待之如在相位時,不少貶,時人重焉。



 子從簡,再為國子博士;惟簡,太子中舍;承簡,司門員外郎;行簡,比部員外郎;務簡,亦國子博士;居簡,殿中丞;知簡,太子右贊善大夫。



 蒙正弟蒙休,咸平進士,至殿中丞。



 龜圖弟龜祥,殿中丞、知壽州。子蒙亨,舉進士高等,既廷試,以蒙正居中書,故報罷。後歷下蔡、武平主簿。至道初,考課州縣官,蒙亨引對,文學、
 政事俱優,命為光祿寺丞,改大理寺丞,卒。次子蒙巽,虞部員外郎;蒙周,淳化進士及第。蒙亨子即夷簡也。次子宗簡,亦進士及第。



 慶歷中,居簡提點京東刑獄,時夏竦有憾於石介,介死,竦言於上曰:「介未嘗死,北走鄰國矣。」乃遣中使發棺驗之。居簡謂曰:「萬一介果死,則朝廷為無故發人之墓,奈何?」中使曰:「於君何如?」居簡曰:「介死,當時必有內外親族及門生會葬,問之可也。」中使乃令結狀保證以聞,介事乃白。居簡長者,其行事多類此。



 徐州
 妖人孔直溫挾左道誘軍士為變,或詣轉運使告,不受詞。居簡令易其牒,盡捕究黨與,貸詿誤者,請於朝,斬直溫等。濮州復叛,都民驚潰,居簡馳往,獲首惡誅之。因大閱兵亨勞,奸不得發。用二事,遷秩鹽鐵判官,拜集賢院學士、知梓州、應天府,徙荊南,進龍圖閣直學士、知廣州,陶甓甃城,人以為便。以兵部侍郎判西京御史臺,卒,年七十二。



 張齊賢,曹州冤句人。生三歲,值晉亂,徙家洛陽。孤貧力
 學,有遠志,慕唐李大亮之為人,故字師亮。太祖幸西都,齊賢以布衣獻策馬前,召至行宮,齊賢以手畫地,條陳十事:曰下並、汾,曰富民,曰封建,曰敦孝,曰舉賢,曰太學,曰籍田,曰選良吏,曰慎刑,曰懲奸。內四說稱旨,齊賢堅執以為皆善,上怒,令武士拽出之。及還,語太宗曰:「我幸西都,唯得一張齊賢爾。我不欲爵之以官,異進可使帗汝為相也。」



 太宗擢進士,欲置齊賢高第,有司偶失掄選,上不悅,一榜盡與京官,於是齊賢以大理評事通判衡
 州。時州鞫劫盜,論皆死,齊賢至,活其失入者五人。自荊渚至桂州,水遞鋪夫數千戶,困於郵役,衣食多不給,論奏減其半。四年,代還,會親征晉陽,齊賢上謁,遷秘書丞。忻州新下,命知州事。明年召還,改著作佐郎,直史館,改左拾遺。冬,車駕北征,議者皆言宜速取幽薊,齊賢上疏曰:



 方今海內一家,朝野無事。關聖慮者,豈不以河東新平,屯兵尚眾,幽燕未下,輦運為勞?臣愚以為此不足慮也。自河東初下,臣知忻州,捕得契丹納米典吏,皆云自
 山後轉般以授河東。以臣料,契丹能自備軍食,則於太原非不盡力,然終為我有者,力不足也。河東初平,人心未固,嵐、憲、忻、代未有軍砦,入寇則田牧頓失,擾邊則守備可虞。及國家守要害,增壁壘,左控右扼,疆事甚嚴,恩信已行,民心已定,乃於雁門陽武谷來爭小利,此其智力可料而知也。聖人舉事,動在萬全,百戰百勝,不若不戰而勝,若重之慎之,則契丹不足吞,燕薊不足取。



 自古疆埸之難,非盡由敵國,亦多邊吏擾而致之。若緣邊諸
 砦撫御得人,但使峻壘深溝,畜力養銳,以逸自處,寧我致人,此李牧所以用趙也。所謂擇卒不如擇將,任力不如任人。如是則邊鄙寧,邊鄙寧則輦運減,輦運減則河北之民獲休息矣。民獲休息,則田業增而蠶績廣,務農積穀,以實邊用。且敵人之心固亦擇利避害,安肯投諸死地而為寇哉?



 臣聞家六合者以天下為心,豈止爭尺寸之事,角強弱之勢而已乎?是故聖人先本而後末,安內以養外。人民,本也,疆土,末也。五帝三王,未有不先根
 本者也。堯、舜之道無他,在乎安民而利之爾。民既安利,則遠人斂衽而至矣。陛下愛民人、利天下之心,真堯、舜也。臣慮群臣多以纖微之利,克下之術,侵苦窮民,以為功能。至於生民疾苦,見之如不見,聞之如不聞,斂怨速尤,無大於此。伏望慎擇通儒,分路採訪兩浙、江南、荊湖、西川、嶺南、河東,凡前日賦斂苛重者,改而正之,因而利之,使賦稅課利通濟,可經久而行,為聖朝定法;除去舊弊,天下諸州有不便於民者,委長吏以聞。敢循故常者,
 重置之法。使天下耳目皆知陛下之心,戴陛下之惠,以德懷遠,以惠利民,則遠人之歸,可立而待也。



 六年,為江南西路轉運副使,冬,改右補闕,加正使。齊賢至官,詢知饒、信、虔州土產銅、鐵、鉛、錫之所,推求前代鑄法,取饒州永平監所鑄以為定式,歲鑄五十萬貫,凡用銅八十五萬斤,鉛三十六萬斤,錫十六萬斤,詣闕面陳其事,敷奏詳確,議者不能奪。



 先是,諸州罪人多錮送闕下,路死者十常五六。齊賢道逢南劍、建昌、虔州所送,索牒視之,率
 非首犯,悉伸其冤抑。因力言於朝,後凡送囚至京,請委強明吏慮問,不實,則罪及原問官屬。自是江南送罪人者為減太半。



 先是,江南諸州小民,居官地者有地房錢,吉州緣江地雖淪沒,猶納勾欄地錢,編木而浮居者名水場錢,皆前代弊政,齊賢悉論免之。



 初,李氏據有江南,民戶稅錢三千已上者戶出丁一人,黥面,自備器甲輸官庫,出即給之,日支糧二升,名為義軍。既內附,皆放歸農。至是,言者以為此輩久在行伍,不樂耕農,乞遣使選
 充軍伍,並其家屬送闕下。齊賢上言:「江南義軍,例皆良民,橫遭黥配,無所逃避。克復之後,便放歸農,久被皇風,並皆樂業。若逐戶搜索,不無驚擾。法貴有常,政尚清凈,前敕既放營農,不若且仍舊貫。」齊賢居使職,勤究民弊,務行寬大,江左人思之不忘。召還,拜樞密直學士,擢右諫議大夫、簽書樞密院事。



 雍熙初,遷左諫議大夫。三年,大舉北伐,代州楊業戰沒。上訪近臣以策,齊賢請行,即授給事中、知代州,與部署潘美同領緣邊兵馬。是時遼
 兵自湖谷入寇,薄城下,神衛都校馬正以所部列南門外,眾寡不敵,副部署盧漢贇畏懦,保壁自固。齊賢選廂軍二千,出正之右,誓眾慷慨,一以當百,遼兵遂卻。



 先是,約潘美以並師來會戰,無何,間使為遼人所得。齊賢以師期既漏,且虞美眾為遼所乘。既而美使至,雲師出並州,至北井,得密詔,東師敗績於君子館,並之全軍不許出戰,已還州矣。於時遼兵塞川,齊賢曰:「賊知美之來,而不知美之退。」乃閉其使密室,中夜發兵二百,人持一幟,
 負一束芻,距州城西南三十里,列幟然芻。遼兵遙見火光中有旗幟,意謂並師至矣,駭而北走。齊賢先伏步兵二千於土磴砦,掩擊大敗之,擒其北大王之子一人,帳前舍利一人,斬數百級,獲馬二千、器甲甚眾。捷奏,且歸功漢贇。



 端拱元年冬,拜工部侍郎。遼人又自大石路南侵,齊賢預簡廂兵千人為二部,分屯繁畤、崞縣。下令曰:「代西有寇,則崞縣之師應之;代東有寇,則繁畤之師應之。比接戰,則郡兵集矣。」至是,果為繁畤兵所敗。



 二年,置
 屯田,領河東制置言方田都部署,入拜刑部侍郎、樞密副使。淳化二年夏,參知政事,數月,拜吏部侍郎、同中書門下平章事。齊賢母孫氏年八十餘,封晉國太夫人,每人謁禁中,上嘆其福壽、有令子,多手詔存問,加賜與,搢紳榮之。



 初,王延德與朱貽業同掌京庾,欲求補外,貽業與參政李沆有姻婭,托之以請於沆,沆為請於齊賢,齊賢以聞。太宗以延德嘗事晉邸。怒其不自陳而干祈執政,召見詰責。延德、貽業皆諱不以實對,齊賢不欲累沆,獨
 任其責。四年六月,罷為尚書左丞。十月,命知定州,以母老不願往,未幾,丁內艱,水漿不入口者七日,自是日啖粥一器,終喪不食酒肉蔬果。尋復轉禮部尚書、知河南府。時獄有大闢將決,齊賢至,立辨而釋之。三日,徙知永興軍。時閣門祗候趙贊以言事得幸,提點關中芻糧,所為多豪橫。齊賢論列其罪,卒抵於法。俄徙襄州,移荊南,又徙安州。逾年,加刑部尚書。



 真宗即位,召拜兵部尚書、同中書門下平章事。嘗從容為上言皇王之道,而推本
 其所以然,且言:「臣受陛下非常恩,故以非常為報。」上曰:「朕以為皇王之道非有跡,但庶事適治道則近之矣。」時戚里有分財不均者更相訟,又入宮自訴。齊賢曰:「是非臺府所能決,臣請自治。」上俞之。齊賢坐相府,召訟者問曰:「汝非以彼所分財多、汝所分少乎?」曰:「然。」命具款。乃召兩吏,令甲家入乙舍,乙家入甲舍,貨財無得動,分書則交易之。明日奏聞,上大悅曰:「朕固知非君莫能定者。」郊祀,加門下侍郎。與李沆同事,不相得。坐冬至朝會被酒
 失儀,免相。



 四年,李繼遷陷清遠軍,命為涇、原等州軍安撫經略使,以右司諫梁顥為之副。齊賢上言謂:「清遠軍陷沒以來,青岡砦燒棄之後,靈武一郡,援隔勢孤,此繼遷之所覬覦而必至者也。以事勢言之,加討則不足,防遏則有餘。其計無他,蕃部大族首領素與繼遷有隙者,若能啖以官爵,誘以貨利,結之以恩信,而激之以利害,則山西之蕃部族帳,靡不傾心朝廷矣。臣所領十二州軍,見二萬餘人,若緣邊料柬本城等軍,更得五萬餘人,
 招致蕃部,其數又逾十數萬。但彼出則我歸,東備則西擊,使之奔走不暇,何能為我患哉?今靈武軍民不翅六七萬,陷於危亡之地,若繼遷來春於我兵未舉之前,發兵救援靈武,盡驅其眾,並力攻圍,則靈州孤城必難固守。萬一失陷,賊勢益增,縱多聚甲兵,廣積財貨,亦難保必勝矣。臣所以乞封潘羅支為六谷王而厚以金帛者,恐繼遷旦暮用兵斷彼賣馬之路也。茍朝廷信使得達潘羅支,則泥埋等族、西南遠蕃,不難招集。西南既廩命,
 而緣邊之勢張,則鄜、延、環、慶之淺蕃,原、渭、震戎之熟戶,自然歸化。然後使之與對替甲兵及駐泊軍馬互為聲援,則萬山聞之,必不敢於靈州、河西頓兵矣。萬山既退,則賀蘭蕃部亦稍稍叛繼遷矣。若曰名器不可以假人,爵賞不可以濫及,此乃聖人為治之常道,非隨時變易之義也。」



 齊賢又請調江淮、荊湘丁壯八萬以益防禦,朝議以為動搖,兼澤國人民,遠戍西鄙亦非便,計遂寢。



 齊賢又言:「靈州斗絕一隅,當城鎮完全、磧路未梗之時,中
 外已言合棄,自繼遷為患已來,危困彌甚。南去鎮戎約五百餘里,東去環州僅六七日程,如此畏途,不須攻奪,則城中之民何由而出,城中之兵何由而歸?欲全軍民,理須應接。為今之計,若能增益精兵,以合西邊屯駐、對替之兵,從以原、渭、鎮戎之師,率山西熟戶從東界而入,嚴約師期,兩路交進。設若繼遷分兵以應敵,我則乘勢而易攻。且奔命途道,首尾難衛,千里趨利,不敗則禽。臣謂兵鋒未交,而靈州之圍自解。然後取靈州軍民,而置
 砦於蕭關、武延川險要處以僑寓之,如此則蕃漢士人之心有所依賴。裁候平寧,卻歸舊貫,然後縱蕃漢之兵,乘時以為進退,則成功不難矣。」時不能用。未幾,靈武果陷。



 閏十二月,拜右僕射、判汾州,不行,改判永興軍兼馬步軍部署。時薛居正子惟吉妻柴氏無子早寡,盡畜其貨產及書籍論告,欲改適齊賢。惟吉子安上訴其事,上不欲置於理,命司門員外郎張正倫就訊,柴氏所對與安上狀異。下其事於御史,乃齊賢子太子中舍宗誨教
 柴氏為詞。齊賢坐責太常卿、分司西京,宗誨貶海州別駕。



 景德初,起為兵部尚書、知青州。上幸澶淵,命兼青、淄、濰州安撫使。二年,改吏部尚書。上疏言曰:「臣在先朝,常憂靈、夏兩鎮終為繼遷並吞,言事者以臣所慮為太過,略舉既往之事以明本末。當時臣下皆以繼遷只是懷戀父祖舊地,別無他心,先帝與以銀州廉察,庶滿其意。爾後攻劫不已,直至降麟、府州界八部族蕃酋,又脅制賀蘭山下帳族,言事者猶謂封獎未厚。洎陛下賜以銀、
 夏土壤,寵以節旄,自此奸威愈滋,逆志尤暴。屢斷靈州糧路,復撓緣邊城池,數年之間,靈州終為吞噬。當靈池、清遠軍垂欲陷沒,臣方受經略之命。臣思繼遷須是得一兩處強大蕃族與之為敵,此乃以蠻夷攻蠻夷,古今之上策也。遂請以六谷名目封潘羅支,俾其展效。其時近臣所見,全與臣謀不同,多為沮撓。及繼遷為潘羅支射殺,邊患謂可少息。今其子德明依前攻劫,析逋游龍缽等盡在部下,其志又似不小。臣慮德明乘大駕東幸
 之際,去攻六谷,則瓜、沙、甘、肅、于闐諸處漸為控制矣。向使潘羅支尚在,則德明未足為虞;今潘羅支已亡,廝鐸督恐非其敵。望委大臣經制其事。」



 從東封還,復拜右僕射。時建玉清昭應宮,齊賢言繪畫符瑞,有損謙德,又違奉天之意,屢請罷其役。



 三年,出判河陽,從祀汾陰還,進左僕射。五年,代還,請老,以司空致仕。入辭便坐,方拜而僕,上遽止之,許二子扶掖升殿,命益坐茵為三。



 歸洛,得裴度午橋莊,有池榭松竹之盛,日與親舊觴詠其間,意
 甚曠適。七年夏,薨,年七十二。贈司徙,謚文定。



 齊賢姿儀豐碩,議論慷慨,有大略,以致君自負。留心刑獄,多所全活。喜提獎寒雋。少時家貧,父死無以為葬,河南縣吏為辦其事,齊賢深德之,事以兄禮,雖貴不替也。仲兄昭度嘗授齊賢經,及卒,表贈光祿寺丞。又嘗依太子少師李肅家,肅死,為營葬事,歲時祭之。趙普嘗薦齊賢於太宗,未用,普即具前列事,以謂:「陛下若進齊賢,則齊賢他日感恩,更過於此。」上大悅,遂大用。種放之起,齊賢所薦也。
 齊賢四踐兩府,九居八座,以三公就第,康寧福壽,時罕其比。居相日,數起大獄,又與寇準相傾,人或以此少之。



 齊賢諸子皆能有立:宗信,內殿崇班;宗理,大理寺丞;宗諒,殿中丞;宗簡,閣門祗候;宗訥,太子中舍;宗禮最賢,雖累資登朝,而畏羈束,故多居田里。



 宗誨字習之,齊賢第二子也。少喜學兵法,陰陽、象緯之書無不通究。以父任為秘書省正字,遷至太子中舍,貶海州別駕。嘗通判河陽,徙知富順監。會夷人鬥郎春叛,
 群獠皆騷動,宗誨將郡兵攻破之。擢開封府判官、三司度支勾院。宗誨在開封日,御史王沿劾其嗜酒廢事,及為河北轉運使,乃發沿居喪假官舟賈販,朝論惡之。



 會以調發擾民,徙知徐州。累遷太常少卿,後為永興軍兵馬鈐轄,又徙鄜延路兼知鄜州。元昊寇延安,劉平、石元孫敗沒,鈐轄黃德和遁還,延州不納,又走鄜州。宗誨曰:「軍奔將無所歸,激之則為亂矣。」乃納之,拘德和以聞。是時鄜城不完,且無備,傳言寇兵至,人心不安。宗誨乃嚴
 斥候,籍入而禁出,使老幼並力守禦之,敵亦自引去。領興州防禦使,復徙永興鈐轄兼知邠州,以秘書監致仕。



 嘗事幹謁,其子曰:「昔賀秘監以道士服東歸會稽,明皇賜以鑒湖,以為休老之地。今洛下雖無鑒湖,而嵩、少、伊、瀍天下佳處,雖非朝廷所賜,皆閑逸之人所有爾。大人盍衣羽服以優游,何必更事請謁乎?」宗誨曰:「吾作白頭老監秘書而眠,何以賀老流沙之服為哉?」時以為名言。



 初,齊賢守代州,宗誨嘗預計畫,其保任親族不問疏近,
 以年為先後。然性貪,雖謝事,猶事貨殖,以至於卒。



 子二人。子皋字叔謨,少有才名而不自負,人樂與之游。最善尹洙,洙曰:「吾交天下士多矣,不以通否易意者,子皋也。」舉進士,試秘書郎、知新鄭縣。以齊賢相,遷校書郎,館閣獻頌,擢著作佐郎,進直史館,累官至尚書司封員外郎。



 子憲字彥章,以蔭將作監主簿,以獻文賜同進士出身,累遷尚書刑部郎中、知光化軍。戍卒逐其帥韓綱,餘黨作亂,子憲招降之。征稅重,人多逋負,子憲奏除之。歷
 太常少卿、三司鹽鐵判官、直史館、知洪州。遷右諫議大夫、知桂州,不赴,御史劾之,降秘書監。復為光祿卿,加直秘閣、知廬州,遷秘書監,累職徙揚州,卒。



 賈黃中,字媧民,滄州南皮人,唐相耽四世孫。父玭字仲寶,晉天福三年進士,解褐。宋初,為刑部郎中,終水部員外郎、知浚儀縣,年七十卒。玭嚴毅,善教子,士大夫子弟來謁,必諄諄誨誘之。初,通判鎮州,葬鄉黨群從之未葬者十五喪,孤貧不自給者,咸教育而婚嫁之。



 黃中幼聰
 悟,方五歲,玭每旦令正立,展書卷比之,謂之「等身書」,課其誦讀。六歲舉童子科,七歲能屬文,觸類賦詠。父常令蔬食,曰:「俟業成,乃得食肉。」十五舉進士,授校書郎、集賢校理,遷著作佐郎、直史館。



 建隆三年,遷左拾遺,歷左補闕。開寶八年,通判定州,判太常禮院。黃中多識典故,每詳定禮文,損益得中,號為稱職。



 嶺南平,以黃中為採訪使,廉直平恕,遠人便之。還奏利害數十事,皆稱旨。會克江表,選知宣州。歲饑,民多為盜,黃中出己奉造糜粥,賴
 全活者以千數,仍設法弭盜,因悉解去。



 太宗即位,遷禮部員外郎。太平興國二年,知升州。時金陵初附,黃中為政簡易,部內甚治。一日,案行府署中,見一室扃鑰甚固,命發視之,得金寶數十匱,計直數百萬,乃李氏宮閣中遺物也,即表上之。上覽表謂侍臣曰:「非黃中廉恪,則亡國之寶,將污法而害人矣。」賜錢三十萬。丁父憂,起復視事。五年,召歸闕。



 有薦黃中文學高第,召試中書,拜駕部員外郎、知制誥。八年,與宋白、呂蒙正等同知貢舉,遷司
 封郎中,充翰林學士。雍熙二年,又知貢舉,俄掌吏部選。端拱初,加中書舍人。二年,兼史館修撰。凡再典貢部,多柬拔寒俊,除擬官吏,品藻精當。淳化二年秋,與李沆並拜給事中、參知政事。太宗召見其母王氏,命坐,謂曰:「教子如是,真孟母矣。」作詩以賜之,頒賜甚厚。



 黃中素重呂端為人,屬端出鎮襄陽,黃中力薦於上,因留為樞密直學士,遂參知政事。當世文行之士,多黃中所薦引,而未嘗言,人莫之知也。然畏慎過甚,中書政事頗留不決。



 四
 年冬,與沆並罷守本官。明年,知襄州,上言母老乞留京,改知澶州。辭日,上戒之曰:「夫小心翼翼,君臣皆當然;若太過,則失大臣之體。」黃中頓首謝。上因謂侍臣曰:「朕嘗念其母有賢德,七十餘年未覺老,每與之語,甚明敏。黃中終日憂畏,必先其母老矣。」因目參知政事蘇易簡曰:「易簡之母亦如之。自古賢母不可多得。」易簡前謝曰:「陛下以孝治天下,獎及人親,臣實何人,膺茲榮遇。」



 至道初,黃中遘疾,詔令歸闕。會建儲宮,擇大臣有德望者為賓
 友,黃中在選中。以久疾,改命李至、李沆兼賓客,黃中亦特拜禮部侍郎,代至兼秘書監。黃中素嗜文籍,既居內閣,甚以為慰。



 二年,以疾卒,年五十六,其母尚無恙,卒如上言。贈禮部尚書。上聞其素貧,別賜錢三十萬。既葬,其母入謝,又賜白金三百兩。上謂之曰:「勿以諸孫為念,朕當不忘也。」



 黃中端謹,能守家法,廉白無私。多知臺閣故事,談論亹癖,聽者忘倦焉。在翰林日,太宗召見,訪以時政得失,黃中但言:「臣職典書詔,思不出位,軍國政事,非
 臣所知。」上益重之,以為謹厚。及知政事,卒無所建明,時論不之許。有文集三十卷。



 子守謙,雍熙二年進士;守正,獻文召試,賜進士第,後為虞部員外郎;守約,國子博士;守文,殿中丞;守訥,右贊善大夫。



 論曰:《詩》云:「允也天子,降予卿士,實為阿衡,實左右商王。」言有是君則有是臣,有是臣則足以相是君也。太宗勵精庶政,注意輔相,以昉舊德,亟加進用,繼擢蒙正、齊賢,迭居相位;復進黃中,俾參大政。而四臣者將順德美,修
 明庶政,以致承平之治,可謂君臣各盡其道者矣。君子謂李昉為多遜所毀而不校,蒙正為張紳所污而不辨,齊賢為同列所累而不言,黃中多所薦引而不有其功,此固人之所難也。而況四臣者皆賢宰輔,又能進退有禮,皆以善終,非盛德君子,其孰能與於斯?



\end{pinyinscope}