\article{列傳第一百九十六儒林七}

\begin{pinyinscope}

 ○程迥劉清之真德秀魏了翁廖德明



 程迥,字可久,應天府寧陵人。家於沙隨,靖康之亂,徙紹
 興之餘姚。年十五,丁內外艱,孤貧飄泊,無以自振。二十餘,始知讀書,時亂甫定,西北士大夫多在錢塘,迥得以考德問業焉。



 登隆興元年進士第,歷揚州泰興尉。訓武郎楊大烈有田十頃,死而妻女存。俄有訟其妻非正室者,官沒其貲,且追十年所入租。部使者以諉迥,迥曰:「大烈死,貲產當歸其女。女死,當歸所生母可也。」



 調饒州德興丞。盜入縣民齊匊家,平素所不快者,皆罥絓逮獄。州屬迥決禁囚,辨其冤者縱遣之。匊訟不已。會獲盜寧國,
 匊猶訟還所縱之人,迥曰:「盜既獲矣,再令追捕,或死於道路,使其骨肉何依,豈審冤之道哉!」唐肅宗時,縣有程氏女,其父兄為盜所殺,因掠女去,隱忍十餘年,手刃盡誅其黨,刳其肝心以祭其父兄。迥取《春秋》復仇之義,頌之曰:「大而得其正者也。」表之曰「英孝程烈女」。



 改知隆興府進賢縣。省符下,知平江府王佐決陳長年輒私賣田,其從子訴有司十有八年,母魚氏年七十坐獄。廷辨按法追正,令候母死服闋日,理為己分,令天下郡縣視此
 為法。迥為議曰:「天下之人孰無母慈?子若孫宜定省溫凊,不宜有私財也。在律,別籍者有禁,異財者有禁。當報牒之初,縣令杖而遣之,使聽命於其母可矣,何稽滯遍訴有司,而達於登聞院乎?《春秋穀梁傳》注曰:『臣無訟君之道』,為衛侯鄭與元咺發論也。夫諸侯之於命大夫猶若此,子孫之於母乃使坐獄以對吏,愛其親者聞之,不覺泣涕之橫集也。按令文:分財產,謂祖父母、父母服闋已前所有者。然則母在,子孫不得有私財。借使其母一
 朝盡費,其子孫亦不得違教令也。既使歸於其母,其日前所費,乃卑幼輒用尊長物,法須五年尊長告乃為理。何至豫期母死,又開他日爭訟之端也?抑亦安知不令之子孫不死於母之前乎?守令者,民之師帥,政教之所由出。誠宜正守令不職之愆與子孫不孝之罪,以敬天下之為人母者。」



 民饑,府檄有訴閉糴及糶與商賈者,迥即論報之曰:「力田之人,細米每鬥才九十五文,逼於稅賦,是以出糶,非上戶也。縣境不出貨寶,茍不與外人交
 易,輸官之錢何由而得?今強者群聚,脅持取錢,毆傷人者甚眾,民不敢入市,坐致缺食。」申論再三,見從乃已。



 縣大水,亡稻麥,郡蠲租稅至薄,迥白於府曰:「是驅民流徙耳!賦不可得,徒存欠籍。」乃悉蠲之。郡僚猶曰:「度江後來,未嘗全放,恐戶部不從。」迥力論之曰:「唐人損七,則租、庸、調俱免。今損十矣。夏稅、役錢不免,是猶用其二也,不可謂寬。」議乃息。



 境內有婦人傭身紡績舂簸,以養其姑。姑感婦孝,每受食,即以手加額仰天而祝之。其子為人
 牧牛,亦干飯以餉祖母。迥廉得之,為紀其事,白於郡,郡給以錢粟。



 調信州上饒縣。歲納租數萬石,舊法加倍,又取斛面米。迥力止絕之,嘗曰:「令與吏服食者,皆此邦之民膏血也。曾不是思,而橫斂虐民,鬼神其無知乎!」州郡督索經總錢甚急,迥曰:「斯錢古之除陌之類,今其類乃三倍正賦,民何以堪?。反復言之當路。



 奉祠,寓居番陽之蕭寺。程祥者,從伯父待制昌禹來居番陽,昌禹死,遂失所依。祥繼亡,祥妻度氏猶質賣奩具以撫育孤子,久之
 罄竭,瀕死,鄰家皆莫識其面。有欲醮之者,度曰:「吾兒幼,若事他人,使母不得撫其子,豈不負良人乎?」終辭焉。或為迥言其事,迥走告於郡守,月給之錢粟。



 迥居官臨之以莊,政寬而明,令簡而信,綏強撫弱,導以恩義。積年讎訟,一語解去。猾吏奸民,皆以感激,久而悛悔,欺詐以革。暇則賓禮賢士,從容盡歡,進其子弟之秀者與之均禮,為之陳說《詩》、《書》。質疑問難者,不問蚤暮。勢位不得以交私,祠廟非典祀不謁。隱德潛善,無問幽明,皆表而出之,
 以勵風俗。或周其窮厄,俾全節行。聽決獄訟,期於明允。凡上官所未悉者,必再三抗辨,不為茍止。貴溪民偽作吳漸名,誣訴縣令石邦彥,迥言匿名書不當受,轉運使不謂然,遂興大獄,瘐死者十有四人。及聞省寺,訖報如迥言。



 迥嘗授經學於昆山王葆、嘉禾聞人茂德、嚴陵喻樗。所著有《古易考》、《古易章句》、《古占法》、《易傳外編》、《春秋傳顯微例目》、《論語傳》、《孟子章句》《文史評》、《經史說諸論辨》、《太玄補贊》、《戶口田制貢賦書》、《乾道振濟錄》、《醫經正本書》、《條
 具乾道新書》、《度量權三器圖義》、《四聲韻》、《淳熙雜志》、《南齋小集》。卒官。



 朝奉郎朱熹以書告迥子絢曰:「敬惟先德,博聞至行,追配古人,釋經訂史,開悟後學,當世之務又所通該,非獨章句之儒而已。曾不得一試,而奄棄盛時,此有志之士所為悼嘆咨嗟而不能已者。然著書滿家,足以傳世,是亦足以不朽。」絢以致仕恩調巴陵尉,攝邑事,能理冤獄。孫仲熊,亦有名。



 劉清之,字子澄,臨江人,受業於兄靖之,甘貧力學,博極
 書傳。登紹興二十七年進士第。調袁州宜春縣主簿,未上,丁父憂,服除,改建德縣主簿。請於州,俾民自實其戶。由是賦役平,爭訟息。



 調萬安縣丞。時江右大侵,郡檄視旱,徒步阡陌,親與民接,凡所蠲除,具得其實。州議減常平米直,清之曰:「此惠不過三十里內耳,外鄉遠民勢豈能來?老幼疾患之人必有餒死者。今有粟之家閉不肯糶,實窺伺攘奪者眾也。在我有政,則大家得錢,細民得米,兩適其便。」乃請均境內之地為八,俾有粟者分振其
 鄉,官為主之。規畫防閑,民甚賴之。帥龔茂良以救荒實跡聞於朝,又偕諸公薦之。



 發運使史正志按部至筠,俾清之拘集州縣畸零之賦,清之不可。清之有同年生在幕中,謂曰:「侍郎因子言,謂子愛民特立,將薦子矣,其以閥閱來。」清之貽之以書曰:「所謂贏資者,皆州縣侵刻於民,法所當禁。縱有贏資,是所謂羨餘也,獻之自下而詔止之,今則止而求之,乃自上焉。不奪不饜,其弊有不可勝言者。願侍郎自請於朝,姑歸貳卿之班,主大農經費,
 以佐國家。如此,則士孰不願出侍郎之門?不然,某誠不敢玷侍郎知人之鑒。」以薦者兩有審察之命,清之竟不見丞相,詣吏部銓,得知宜黃縣。



 茂良入為參知政事,與丞相周必大薦清之於孝宗。召入對,首論:「民困兵驕,大臣退托,小臣茍偷。願陛下廣覽兼聽,並謀合智,清明安定,提要挈綱而力行之。古今未有俗不可變、弊不可革者,變而通之,亦在陛下方寸之間耳。」又言用人四事:「一曰辨賢否。謂道義之臣,大者可當經綸,小者可為儀刑。
 功名之士,大者可使臨政,小者可使立事。至於專謀富貴利達而已者下也。二曰正名實。今百有司職守不明,非曠其官,則失之侵逼。願詔史官考究設官之本意,各指其合主何事,制旨親定,載之命書,依開寶中差諸州通判故事,使人人曉然知之而行賞罰焉。三曰使材能。謂軍旅必武臣,錢穀必能吏,必臨之以忠信不欺之士,使兩人者皆得以效其所長。四曰聽換授。謂文武之官不可用違其才,然不當許之自列,宜令文武臣四品以
 上,各以性行材略及文武藝,每歲互舉堪充左右選者一人,於合入資格外,稍與優獎。」



 改太常寺主簿。丁內艱,服除,通判鄂州。鄂大軍所駐,兵籍多偽,清之白郡及諸司,請自通判廳始,俾偽者以實自言而正之。州有民妻張以節死,嘉祐中,詔封旌德縣君,表其墓曰「烈女」,中更兵火,至是無知其墓者,清之與郡守羅願訪而祠之。鄂俗計利而尚鬼,家貧子壯則出贅,習為當然,而尤謹奉大洪山之祠,病者不藥而聽於巫,死則不葬而畀諸火,
 清之皆諭止之。



 差權發遣常州,改衡州。衡自建炎軍興,有所請大軍月樁過湖錢者,歲送漕司,無慮七八萬緡,以四邑所入曲引錢及郡計畸零苗米折納充之。舊法,民有吉兇聚會,許買引為酒曲,謂之曲引錢,其後直以等第敷納。衡有五邑,獨敷其四。取民之辭不正,良民遍受其害,而黠民往往侮易其上,乃並與常賦不輸。雖得曲引錢四五萬緡,而常賦之失,不啻數萬緡矣。清之請於朝,願與總領所酌損補移,漸圖蠲減。不報。遂戒諸邑:
 董常賦,緩雜征,閣舊逋,戒預折,新簿籍,謹推收,督勾銷,明逋負,防帶鈔,治頑梗,柅吏奸,擾戶長,費用有節,滲漏有防,稽考有政,補置有漸。



 先是,郡飾廚傳以事常平、刑獄二使者,月一會集,互致折饋。清之嘆曰:「此何時也?與其取諸民,孰若裁諸公。吾之所以事上官者,惟究心於所職,無負於吾民足矣。豈以酒食貨財為勤哉?」清之自常祿外,悉歸之公帑,以佐經用。至之日,兵無糧,官無奉,上供送使無可備。已而郡計漸裕。民力稍蘇。或有報白,
 手自書之,吏不與焉。



 嘗作《諭民書》一編,首言畏天積善,勤力務本,農工商賈莫不有勸,教以事親睦族,教子祀先,謹身節用,利物濟人,婚姻以時,喪葬以禮。詞意質直,簡而易從。邦人家有其書,非理之訟日為衰息。



 念士風未振,每因月講,復具酒肴以燕諸生,相與輸情論學,設為疑問,以觀其所向,然後從容示以先後本末之序。來者日眾,則增築臨蒸精舍居之。其所講,先正經,次訓詁音釋,次疏先儒議論,次述今所紬繹之說,然後各指其
 所宜用,人君治天下,諸侯治一國,學者治心治身治家治人,確然皆有可舉而措之之實。



 為閱武場。凡禁軍役於他所,隱於百工者,悉按軍籍俾詣訓閱。作朱陵道院,祠張九齡、韓愈、寇準、周敦頤、胡安國於左,祠晉死節太守劉翼、宋死節內史王應之於右。雅儒吉士日相周旋其間,而參佐謀論多在焉。劉孝昌者,摯之孫也,貧不自立,清之買田以給之。部使者以清之不能媚己,惡之,貽書所厚臺臣,誣以勞民用財,論罷,主管雲臺觀。



 歸,築槐
 陰精舍以處來學者。胡晉臣、鄭僑、尤袤、羅點皆力薦清之於上。光宗即位,起知袁州,而清之疾作,猶貽書執政論國事。諸生往候疾,不廢講論,語及天下,孜孜嘆息,若任其責者。病且革,為書以別向浯、彭龜年,賦二詩以別朱熹、楊萬里。取高氏《送終禮》以授二子曰:「自斂至葬,視此從事。」周必大來視疾,謂曰:「子澄其澄慮。」清之氣息已微,云:「無慮可澄。」遂卒。



 初,清之既舉進士,欲應博學宏詞科。及見朱熹,盡取所習焚之,慨然志於義理之學。呂伯
 恭、張栻皆神交心契,汪應辰、李燾亦敬慕之。母不逮養,每展閱手澤,涕泗交頤。從兄肅流落新吳,族父曄寓丹陽、艾寓臨川,皆迎養之。從祖子僑為邵州錄事參軍,死吳錫之亂,清之遣其孫晉之致書邵守,得其遺骨歸葬焉。族人自遠來,館留之,不忍使之遽去。嘗序範仲淹《義莊規矩》,勸大家族眾者隨力行之。本之家法,參取先儒禮書,定為祭禮行之。高安李好古以族人有以財為訟,見清之豫章,清之為說《訟》、《家人》二卦,好古惕然,遽舍所
 訟,市程氏《易》以歸,卒為善士。



 所著有《曾子內外雜篇》、《訓蒙新書外書》、《戒子通錄》、《墨莊總錄》、《祭儀》、《時令書》、《續說苑》、文集、《農書》。



 真德秀,字景元,後更為希元,建之浦城人。四歲受書,過目成誦。十五而孤,母吳氏力貧教之。同郡楊圭見而異之,使歸共諸子學,卒妻以女。



 登慶元五年進士第,授南劍州判官。繼試,中博學宏詞科,入閩帥幕,召為太學正,嘉定元年遷博士。時韓侂胄已誅,入對,首言:「權臣開邊,
 南北塗炭,今茲繼好,豈非天下之福?然日者以行人之遣,金人欲多歲幣之數,而吾亦曰可增;金人欲得奸臣之首,而吾亦曰可與。往來之稱謂,犒軍之金帛,根括歸明流徙之民,皆承之唯謹,得無滋嫚我乎?抑善謀國者不觀敵情,觀吾政事。今號為更化,而無以使敵情之畏服,正恐彼資吾歲賂以厚其力,乘吾不備以長其謀,一旦挑爭端而吾無以應,此有識所為寒心。」又言:「侂胄自知不為清議所貸,至誠憂國之士則名以好異,於是忠
 良之士斥,而正論不聞,正心誠意之學則誣以好名,於是偽學之論興,而正道不行。今日改弦更張,正當褒崇名節,明示好尚。



 召試學士院,改秘書省正字兼檢討玉牒。二年,遷校書郎。又對,言暴風、雨雹、熒惑、蝻蝗之變,皆贓吏所致。尋兼沂王府教授、學士院權直。三年,遷秘書郎。入對,乞開公道,窒旁蹊,以抑小人道長之漸;選良牧,勵戰士,以扼群盜方張之銳。四年,選著作佐郎。同列相惎讒之,德秀恬不與較。宰相將用德秀,會言官牴之,德
 秀力辭。兼禮部郎官,上疏言:「金有必亡之勢,亦可為中國憂。蓋金亡則上恬下嬉,憂不在敵而在我,多事之端恐自此始。」五年,遷軍器少監,升權直。



 六年,遷起居舍人,奏:「權奸擅政十有四年,朱熹、彭龜年以抗論逐,呂祖儉、周端朝以上書斥,當時近臣猶有爭之者。其後呂祖泰之貶,非惟近臣莫敢言,而臺諫且出力以擠之,則嘉泰之失已深於慶元矣。更化之初,群賢皆得自奮。未幾,傅伯成以諫官論事去,蔡幼學以詞臣論事去,鄒應龍、許
 奕又繼以封駁論事去。是數人者,非能大有所矯拂,已皆不容於朝。故人務自全,一辭不措。設有大安危、大利害,群臣喑嘿如此,豈不殆哉!今欲與陛下言,勤訪問、廣謀議、明黜陟三者而已。」時鈔法楮令行,告訐繁興,抵罪者眾,莫敢以上聞。德秀奏:「或一夫坐罪,而並籍昆弟之財;或虧陌四錢,而沒入百萬之貲。至於科富室之錢,拘鹽商之舟,視產高下,配民藏楮,鬻田宅以收券者,雖大家不能免,尚得名便民之策?」自此籍沒之產以漸給還。



 兼
 太常少卿。又言金人必亡,君臣上下皆當以祈天永命為心。充金國賀登位使,及盱眙,聞金人內變而返。言於上曰:「臣自揚之楚,自楚之盱眙,沃壤無際,陂湖相連,民皆堅悍強忍,此天賜吾國以屏障大江,使強兵足食為進取資。顧田疇不闢,溝洫不治,險要不扼,丁壯不練,豪傑武勇不收拾,一旦有警,則徒以長江為恃。豈如及今大修墾田之政,專為一司以領之,數年之後,積儲充實,邊民父子爭欲自保,因其什伍,勒以兵法,不待糧食尚,皆
 為精兵。」又言邊防要事。



 時史彌遠方以爵祿縻天下士,德秀慨然謂劉爚曰:「吾徒須急引去,使廟堂知世亦有不肯為從官之人。」遂力請去,出為秘閣修撰、江東轉運副使。山東盜起,朝廷猶與金通聘,德秀朝辭,奏:「國恥不可忘,鄰盜不可輕,幸安之謀不可恃,導諛之言不可聽,至公之論不可忽。」寧宗曰:「卿力有餘,到江東日為朕撙節財計,以助邊用。」



 江東旱蝗,廣德、太平為甚,德秀遂與留守、憲司分所部九郡大講荒政,而自領廣德、太平。親
 至廣德,與太守魏峴同以便宜發廩,使教授林庠振給,竣事而還。百姓數千人送之郊外,指道傍叢塚泣曰:「此皆往歲餓死者。微公,我輩已相隨入此矣。」索毀太平州私創之大斛。新徽州守林琰無廉聲,寧國守張忠恕私匿振濟米,皆劾之,而以李道傳攝徽。先是,都司胡貙、薛拯每誚德秀迂儒,試以事必敗,至是政譽日聞,因倡言旱傷本輕,監司好名,振贍太過,使峴劾庠以撼德秀。德秀上章自明,朝廷悟,與峴祠,授庠幹官,而道傳尋亦召
 還。



 德秀以右文殿修撰知泉州。番舶畏苛征,至者歲不三四,德秀首寬之,至者驟增至三十六艘。輸租令民自概,聽訟惟揭示姓名,人自詣州。泉多大家,為閭里患,痛繩之。有訟田者,至焚其券不敢爭。海賊作亂,將逼城,官軍敗衄,德秀祭兵死者,乃親授方略,禽之。復遍行海濱,審視形勢,增屯要害處,以備不虞。



 十二年,以集英殿修撰知隆興府。承寬弛之後,乃稍濟以嚴。尤留意軍政,欲分鄂州軍屯武昌,及通廣鹽於贛與南安,以弭汀、贛鹽
 寇。未及行,以母喪歸。明年,蘄、黃失守,盜起南安,討之數載始平,人服德秀先見。



 十五年,以寶謨閣待制、湖南安撫使知潭州。以「廉仁功勤」四字勵僚屬,以周惇頤、胡安國、朱熹、張栻學術源流勉其士。罷榷酤,除斛面米,申免和糴,以蘇其民。民艱食,既極力振贍之,復立惠民倉五萬石,使歲出糶。又易穀九萬五千石,分十二縣置社倉,以遍及鄉落。別立慈幼倉立義阡。惠政畢舉。月試諸軍射,捐其回易之利及官田租。凡營中病者、死未葬者、孕
 者、嫁娶者,贍給有差。朝廷從壽昌朱TY請,以飛虎軍戍壽昌,並致其家口,力爭止之。江華縣賊蘇師入境殺劫,檄廣西共討平之。司馬遵守武岡,激軍變,劾遵而誅其亂者。



 理宗即位,召為中書舍人,尋擢禮部侍郎、直學士院。入見,奏:「三綱五常,扶持宇宙之棟幹,奠安生民之柱石。晉廢三綱而劉、石之變興,唐廢三綱而安祿山之難作。我朝立國,先正名分。陛下不幸處人倫之變,流聞四方,所損非淺。霅川之變,非濟王本志,前有避匿之跡,後
 聞討捕之謀,情狀本末,灼然可考。願討論雍熙追封秦王舍罪恤孤故事,濟王未有子息,亦惟陛下興滅繼絕。」上曰:「朝廷待濟王亦至矣。」德秀曰:「若謂此事處置盡善,臣未敢以為然。觀舜所以處象,則陛下不及舜明甚。人主但當以二帝、三王為師。」上曰:「一時倉猝耳。」德秀曰:「此已往之咎,惟願陛下知有此失而益講學進德。」次言:「霅川之獄,未聞參聽於公朝,淮、蜀二閫乃出於僉論所期之外。天下之事非一家之私,何惜不與眾共之?」且言:「乾
 道、淳熙間,有位於朝者以饋及門為恥,受任於外者以包苴入都為羞。今饋賂公行,薰染成風,恬不知怪。」又疏言:「朝廷之上,敏銳之士多於老成,雖嘗以耆艾褒傅伯成、楊簡,以儒學褒柴中行,以恬退用趙蕃、劉宰,至忠亮敢言如陳宓、徐僑,皆未蒙錄用。」上問謙吏,德秀以知袁州趙䈣夫對,親擢䈣夫直秘閣、為監司。具手札入謝,因言崔與之帥蜀,楊長儒帥閩,皆有廉聲,乞廣加咨訪。



 上初御清暑殿,德秀因經筵侍上,進曰:「此高、孝二祖儲神
 燕閑之地,仰瞻楹桷,當如二祖實臨其上。陛下所居處密邇東朝,未敢遽當人主之奉。今宮閣之義浸備,以一心而受眾攻,未有不浸淫而蠹蝕者,惟學可以明此心,惟敬可以存此心,惟親君子可以維持此心。」因極陳古者居喪之法與先帝視朝之勤。



 寧宗小祥,詔群臣服純吉,德秀爭之曰:「自漢文帝率情變古,惟我孝宗方衰服三年,朝衣朝冠皆以大布,惜當時不並定臣下執喪之禮,此千載無窮之憾。孝宗崩,從臣羅點等議,令群臣易
 月之後,未釋衰服,惟朝會治事權用黑帶公服,時序仍臨慰,至大祥始除。侂胄枋政,始以小祥從吉。且帶不以金,鞓不以紅,佩不以魚,鞍轎不以文繡。此於群臣何損?朝儀何傷?」議遂格。



 德秀屢進鯁言,上皆虛心開納,而彌遠益嚴憚之,乃謀所以相撼,畏公議,未敢發。給事中王塈、盛章始駁德秀所主濟王贈典,繼而殿中侍御史莫澤劾之,遂以煥章閣待制提舉玉隆宮。諫議大夫朱端常又劾之,落職罷祠。監察御史梁成大又劾之,請加竄
 殛。上曰:「仲尼不為已甚。」乃止。



 既歸,修《讀書記》,語門人曰:「此人君為治之門,如有用我者,執此以往。」汀寇起,德秀薦陳韡有文武才幹,常平使者史彌忠言於朝,遂起韡討平之。紹定四年,改職與祠。



 五年,進徽猷閣、知泉州。迎者塞路,深村百歲老人亦扶杖而出,城中歡聲動地。諸邑二稅法預借至六七年,德秀入境,首禁預借。諸邑有累月不解一錢者,郡計赤立不可為。或咎寬恤太驟,德秀謂民困如此,寧身代其苦。決訟自卯至申未已。或勸
 嗇養精神,德秀謂郡弊無力惠民,僅有政平、訟理事當勉。建炎初置南外宗政司於泉,公族僅三百人,漕司與本州給之,而朝廷歲助度牒。已而不復給,而增至二千三百餘人,郡坐是愈不可為。德秀請於朝,詔給度牒百道。



 彌遠薨,上親政,以顯謨閣待制知福州。戒所部無濫刑橫斂,無徇私黷貨,罷市令司,曰:「物同則價同,寧有公私之異?」閩縣里正苦督賦,革之。屬縣苦貴糴,便宜發常平賑之。海寇縱橫,次第禽殄之。未幾,聞金滅,京湖帥奉
 露布圖上八陵,而江、淮有進取潼關、黃河之議。德秀以為憂,上封事曰:「移江、淮甲兵以守無用之空城,運江、淮金穀以治不耕之廢壤,富庶之效未期,根本之弊立見。惟陛下審之重之。」



 召為戶部尚書,入見,上迎謂曰:「卿去國十年,每切思賢。」乃以《大學衍義》進,復陳祁天永命之說,謂「敬者德之聚。儀狄之酒,南威之色,盤游弋射之娛,禽獸狗馬之玩,有一於茲,皆足害敬」。上欣然嘉納,改翰林學士、知制誥,時政多所論建。逾年,知貢舉,已得疾,拜
 參知政事,同編修敕令、《經武要略》。三乞祠祿,上不得已,進資政殿學士、提舉萬壽觀兼侍讀,辭。疾亟,冠帶起坐,迄謝事,猶神爽不亂。遺表聞,上震悼,輟視朝,贈銀青光祿大夫。



 德秀長身廣額,容貌如玉,望之者無不以公輔期之。立朝不滿十年,奏疏無慮數十萬言,皆切當世要務,直聲震朝廷。四方人士誦其文,想見其風採。及宦游所至,惠政深洽,不愧其言,由是中外交頌。都城人時驚傳傾洞,奔擁出關曰:「真直院至矣!」果至,則又填塞聚觀
 不置。時相益以此忌之,輒擯不用,而聲愈彰。及歸朝,適鄭清之挑敵,兵民死者數十萬,中外大耗,尤世道升降治亂之機,而德秀則既衰矣。杜範方攻清之誤國,且謂其貪黷更甚於前,而德秀乃奏言:「此皆前權臣玩心妻之罪,今日措置之失,譬如和、扁繼庸醫之後,一藥之誤,代為庸醫受責。」其議論與範不同如此。然自侂胄立偽學之名以錮善類,凡近世大儒之書,皆顯禁以絕之。德秀晚出,獨慨然以斯文自任,講習而服行之。黨禁既開,而
 正學遂明於天下後世,多其力也。



 所著《西山甲乙稿》、《對越甲乙集》、《經筵講義》、《端平廟議》、《翰林詞草四六》、《獻忠集》、《江東救荒錄》、《清源雜志》、《星沙集志》。既薨,上思之不置,謚曰文忠。



 魏了翁,字華父,邛州蒲江人。年數歲,從諸兄入學,儼如成人。少長,英悟絕出,日誦千餘言,過目不再覽,鄉里稱為神童。年十五,著《韓愈論》,抑揚頓挫,有作者風。



 慶元五年,登進士第。時方諱言道學,了翁策及之。授僉書劍南
 西川節度判官廳公事,盡心職業。嘉泰二年,召為國子正。明年,改武學博士。開禧元年,召試學士院。韓侂胄用事,謀開邊以自固,遍國中憂駭而不敢言。了翁乃言:「國家紀綱不立,國是不定,風俗茍偷,邊備廢弛,財用凋耗,人才衰弱,而道路籍籍,皆謂將有此北伐之舉,人情恟恟,憂疑錯出。金地廣勢強,未可卒圖,求其在我,未見可以勝人之實。盍亦急於內修,姑逭外攘。不然,舉天下而試於一擲,宗社存亡系焉,不可忽也。」策出,眾大驚。改秘書
 省正字。御史徐柟即劾了翁對策狂妄,獨侂胄持不可而止。



 明年,遷校書郎,以親老乞補外,乃知嘉定府。行次江陵,蜀大將吳曦以四川叛,了翁策其必敗。又明年,曦誅,蜀平,了翁奉親還里。侂胄亦以誤國誅。朝廷收召諸賢,了翁預焉。會史彌遠入相專國事,了翁察其所為,力辭召命。丁生父憂,解官心喪,築室白鶴山下,以所聞於輔廣、李燔者開門授徒,士爭負笈從之。由是蜀人盡知義理之學。



 差知漢州。漢號為繁劇,了翁以化善俗為治。
 首蠲積逋二十餘萬,除科抑賣酒之弊,嚴戶婚交訐之禁,復為文諭以厚倫止訟,其民敬奉條教不敢犯。會境內橋壞,民有壓死者,部使者以聞,詔降官一秩、主管建寧府武夷山沖祐觀。未數月,復元官、知眉州。眉雖為文物之邦,然其俗習法令,持吏短長,故號難治。聞了翁至,爭試以事。乃尊禮耆耇,簡拔俊秀,朔望詣學宮,親為講說,誘掖指授,行鄉飲酒禮以示教化,增貢士員以振文風。復蟆頤堰,築江鄉館,利民之事,知無不為。士論大服,
 俗為之變,治行彰聞。



 嘉定四年,擢潼川路提點刑獄公事。八年,兼提舉常平等事,遷轉運判官。戢吏奸,詢民瘼,舉刺不避權右,風採肅然。上疏乞與周惇頤、張載、程顥、程頤錫爵定謚,示學者趣向,朝論韙之,如其請。遂寧闕守,了翁行郡事。即具奏乞修城郭備不虞,廷議靳其費,了翁增埤浚隍,如待敵至者。後一年,潰卒攻掠郡縣,知其有備不敢逞,人始服豫防之意。十年,遷直秘閣、知瀘州、主管潼川路安撫司公事。丁母憂,免喪,差知潼川府。
 約己裕民,厥績大著。若游似、吳泳、牟子才,皆蜀名士,造門受業。



 十五年,被召入對,疏二千餘言。首論人與天地一本,必與天地相似而後可以無曠天位,並及人才、風俗五事,明白切暢。又論郡邑強幹弱枝之弊,所宜變通。蓋自了翁去國十有七年矣,至是上迎勞優渥,嘉納其言。進兵部郎中,俄改司封郎中兼國史院編修官。轉對,論江、淮、襄、蜀當分為四重鎮,擇人以任,虛心以聽,假以事權,資以才用,為聯絡守禦之計。次論蜀邊墾田及實
 錄闕文等事,皆下其章中書。十六年,為省試參詳官,遷太常少卿兼侍立修注官。



 十七年,遷秘書監,尋以起居舍人再辭而後就列。入奏,極言事變倚伏、人心向背、疆埸安危、鄰寇動靜,其幾有五,謂:「宜察時幾而共天命,尊道揆而嚴法守,集思廣益,汲汲圖之,不猶愈於坐觀事會,而聽其勢之所趨乎?」又論士大夫風俗之弊,謂:「君臣上下同心一德,而後平居有所補益,緩急有所倚仗。如人自為謀,則天下之患有不可終窮者。今則面從而腹
 誹,習諛而踵陋,臣實懼焉。盍亦察人心之邪正,推世變之倚伏,開拓規模,收拾人物,庶幾臨事無乏人之嘆。」其言剴切,無所忌避,而時相始不樂矣。



 寧宗崩,理宗自宗室入即位,時事忽異,了翁積憂成疾,三疏求閑不得請,遷起居郎。明年,改元寶慶,雷發非時,上有「朕心終夕不安」之語,了翁入對,即論:「人主之心,義理所安,是之謂天,非此心之外,別有所謂天地神明也。陛下盍即不安而求之,對天地,事太母,見群臣,親講讀,皆隨事反求,則大
 本立而無事不可為矣。」又論:「講學不明,風俗浮淺,立朝無犯顏敢諫之忠,臨難無仗節死義之勇。願敷求碩儒,丕闡正學,圖為久安長治之計。」又請申命大臣,於除授之際,公聽並觀,然後實意所孚,善類皆出矣。



 屬濟王黜削以死,有司顧望,治葬弗虔。了翁每見上,請厚倫紀,以弭人言。應詔言事者十餘人,朝士惟了翁與洪咨夔、胡夢昱、張忠恕所言能引義劘上,最為切至,而了翁亦以疾求去。右正言李知孝劾夢昱竄嶺南,了翁出關餞別,
 遂指了翁首倡異論,將擊之,彌遠猶外示優容。俄權尚書工部侍郎,了翁力以疾辭,乃以集英殿修撰知常德府。越二日,諫議大夫朱端常遂劾了翁欺世盜名,朋邪謗國,詔降三官、靖州居住。初,了翁再入朝,彌遠欲引以自助,了翁正色不撓,未嘗私謁。故三年之間,循格序遷,未嘗處以要地。了翁至靖,湖、湘、江、浙之士,不遠千里負書從學。乃著《九經要義》百卷,訂定精密,先儒所未有。



 紹定四年復職,主管建寧府武夷山沖祐觀。五年,改差提
 舉江州太平興國宮,尋知遂寧府,辭不拜。進寶章閣待制、潼川路安撫使、知瀘州。瀘大藩,控制邊面二千里,而武備不修,城郭不治。了翁乃奏葺其城樓櫓雉堞,增置器械,教習牌手,申嚴軍律,興學校,蠲宿負,復社倉,創義塚,建養濟院。居數月,百廢具舉。彌遠薨,上親庶政,進華文閣待制,賜金帶,因其任。



 了翁念國家權臣相繼,內擅國柄,外變風俗,綱常淪斁,法度墮弛,貪濁在位,舉事弊蠹,不可滌濯。遂應詔上章論十弊,乞復舊典以彰新化:「
 一曰復三省之典以重六卿,二曰復二府之典以集眾議,三曰復都堂之典以重省府,四曰復侍從之典以來忠告,五曰復經筵之典以熙聖學,六曰復臺諫之典以公黜陟,七曰復制誥之典以謹命令,八曰復聽言之典以通下情,九曰復三衙之典以強主威,十曰復制閫之典以黜私意。疏列萬言,先引故實,次陳時弊,分別利害,粲若白黑。上讀之感動,即於經筵舉之成誦。其後,舊典皆復其初。



 臣庶封章多乞召還了翁及真德秀,上因民
 望而並招之,用了翁權禮部尚書兼直學士院。入對,首乞明君子小人之辨,以為進退人物之本,以杜奸邪窺伺之端。次論故相十失猶存,又及修身、齊家、選宗賢、建內小學等,皆切於上躬者。他如和議不可信,北軍不可保,軍實財用不可恃,凡十餘端。復口奏利害,晝漏下四十刻而退。兼同修國史兼侍讀,俄兼吏部尚書。經幃進讀,上必改容以聽,詢察政事,訪問人才。復條十事以獻,皆苦心空臆,直述事情,言人所難。上悉嘉納,且手詔獎
 諭。又奏乞收還保全彌遠家御筆,乞定趙汝愚配享寧廟,乞趣崔與之參預政事,乞定履畝之令以寬民力,乞詔從臣集議以救楮弊,乞儲閫才以備緩急。又因進故事:如儲人才、凝國論,如力圖自治之策,如下罪己之詔,如分別襄、黃二帥是非,如究見黃陂叛卒利害,如分任諸帥區處降附。



 還朝六閱月,前後二十餘奏,皆當時急務。上將引以共政,而忌者相與合謀排擯,而不能安於朝矣。執政遂謂近臣惟了翁知兵體國,乃以端明殿學
 士、同僉書樞密院事督視京湖軍馬。會江、淮督府曾從龍以憂畏卒,並以江、淮付了翁。朝論大駭,以為不可,三學亦上書爭之。適邊警沓至,上心焦勞,了翁嫌於避事,既五辭弗獲,遂受命開府,宣押同二府奏事,上勉勞尤至。尋兼提舉編修《武經要略》,恩數同執政,進封臨邛郡開國侯,又賜便宜詔書如張浚故事。朝辭,面賜御書唐人嚴武詩及「鶴山書院」四大字,仍賜金帶鞍馬,詔宰臣飲餞於關外。乃酌上下流之中,開幕府江州,申儆將帥,
 調遣援師,褒死事之臣,黜退懦之將,奏邊防十事。甫二旬,召為僉書樞密院事。赴闕奏事,時以疾力辭不拜。蓋在朝諸人始謀假此命以出了翁,既出,則復以建督為非,雖恩禮赫奕,而督府奏陳動相牽制,故遽召還,前後皆非上意也。



 尋改資政殿學士、湖南安撫使、知潭州,復力辭,詔提舉臨安府洞霄宮。未幾,改知紹興府、浙東安撫使。嘉熙元年,改知福州、福建安撫使。累章乞骸骨,詔不允。疾革,復上疏。門人問疾者,猶衣冠相與酬答,且曰:「
 吾平生處己,澹然無營。」復語蜀兵亂事,蹙額久之,口授遺奏,少焉拱手而逝。後十日,詔以資政殿大學士、通奉大夫致仕。



 遺表聞,上震悼,輟視朝,嘆惜有用才不盡之恨。詔贈太師,謚文靖,賜第宅蘇州,累贈秦國公。



 所著有《鶴山集》、《九經要義》、《周易集義》、《易舉隅》、《周禮井田圖說》、《古今考》、《經史雜抄》、《師友雅言》。



 廖德明,字子晦,南劍人。少學釋氏,及得龜山楊時書,讀之大悟,遂受業朱熹。登乾道中進士第。知莆田縣。民有
 奉淫祠者,罪之,沉像於江。會有顯者欲取邑地廣其居,德明不可,守會僚屬諭之,德明曰:「太守,天子守土之臣,未聞以土地與人者。」守乃慚服。



 累官知潯州,有聲。諸司且交薦之,德明曰:「今老矣,況以道徇人乎?」固辭不受。選廣東提舉刑獄,彈劾不避權要。歲當薦士,朝貴多以書托之,德明曰:「此國家公器也。」悉不啟封還之。有鄉人為主簿,德明聞其能,薦之。會德明行縣,簿感其知己,置酒延之,悉假富人觴豆甚盛。德明怒曰:「一主簿乃若是侈
 耶?必貪也。」於是追還薦章,其公嚴類此。



 時盜陷桂陽,迫韶,韶人懼,德明燕笑自如,遣將弛擊,而親持小麾督戰,大敗之。乃分戍守,遠斥堠,明審賞罰,宣布威信,韶晏然如平時。徙知廣州,遷吏部左選郎官,奉祠,卒。



 德明初為潯州教授,為學者講明聖賢心學之要,手植三柏於學,潯士愛敬之如甘棠。在南粵時,立師悟堂,刻朱熹《家禮》及程氏諸書。公餘,延僚屬及諸生親為講說,遠近化之。嘗語人以仕學之要曰:「德明自始仕,以至為郡,惟用三
 代直道而行一句而已。」有《槎溪集》行於世。



\end{pinyinscope}