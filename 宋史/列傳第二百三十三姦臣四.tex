\article{列傳第二百三十三姦臣四}

\begin{pinyinscope}

 万俟禼,字元忠,開封陽武縣人。登政和二年上舍第。調相州、潁昌府教授,歷太學錄、樞密院編修官、尚書比部
 員外郎。紹興初,盜曹成掠荊湖間,禼時避亂沅、湘,帥臣程昌寓以便宜檄禼權沅州事。成奄至城下,禼召土豪、集丁壯以守,成食盡乃退。除湖北轉運判官,改提點湖北刑獄。岳飛宣撫荊湖,遇禼不以禮,禼憾之。禼入覲,調湖南轉運判官,陛辭,希秦檜意,譖飛於朝。留為監察御史,擢右正言。



 時檜謀收諸將兵權,禼力助之,言諸大將起行伍,知利不知義,畏死不畏法,高官大職,子女玉帛,已極其欲,盍示以逗遛之罰,敗亡之誅,不用命之戮,使
 知所懼。



 張俊歸自楚州,與檜合謀擠飛,令禼劾飛對將佐言山陽不可守。命中丞何鑄治飛獄,鑄明其無辜。檜怒,以禼代治,遂誣飛與其子雲致書張憲,令虛申警報以動朝廷,及令憲措置使還飛軍;獄不成,又誣以淮西逗遛之事。飛父子與憲俱死,天下冤之。大理卿薛仁輔、寺丞李若樸、何彥猷言飛無罪,禼劾之;知宗正寺士㒟請以百口保飛,禼又劾之,士㒟竄死建州。劉洪道與飛有舊,禼劾其足恭媚飛,聞飛罷宣撫,抵掌流涕。於是洪
 道抵罪,終身不復。參政范同為檜所引,或自奏事,檜忌之,禼劾罷,再論同罪,謫居筠州。又為檜劾李光鼓倡,孫近朋比,二人皆被竄謫。



 和議成,禼請詔戶部會計用兵之時與通和之後所費各幾何,若減於前日,乞以羨財別貯御前激賞庫,不許他用,蓄積稍實,可備緩急。梓宮還,以禼為攢宮按行使,內侍省副都知宋唐卿副之,禼請與唐卿同班上殿奏事,其無恥如此。張浚寓居長沙,禼妄劾浚卜宅踰制,至擬五鳳樓。會吳秉信自長沙還
 朝,奏浚宅不過眾人,常產可辦,浚乃得免。



 除參知政事,充金國報謝使。使還,檜假金人譽己數千言,囑禼以聞,禼難之。他日奏事退,檜坐殿廬中批上旨,輒除所厚者官,吏鈐紙尾進,禼曰:「不聞聖語。」卻不視。檜大怒,自是不交一語。言官李文會、詹大方交章劾禼,禼遂求去。帝命出守,檜愈怒。給事中楊愿封還詞頭,遂罷去,尋謫居歸州。遇赦,量移沅州。



 二十五年,召還,除參知政事,尋拜尚書右僕射、同中書門下平章事。纂次太后回鑾事實,上
 之。張浚以禼與沈該居相位,不厭天下望,上書言其專欲受命于金。禼見書大怒,以為金人未有釁,而浚所奏乃若禍在年歲間,浚坐竄謫。禼提舉刊修《貢舉敕令格式》五十卷、《看詳法意》四百八十七卷,書進,授金紫光祿大夫,致仕。卒,年七十五,諡「忠靖」。



 禼始附檜為言官,所言多出檜意;及登政府,不能受鉗制,遂忤檜去。檜死,帝親政,將反檜所為,首召禼還。禼主和固位,無異於檜,士論益薄之。



 韓侂胄,字節夫,魏忠獻王琦曾孫也。父誠,娶高宗憲聖慈烈皇后女弟,仕至寶寧軍承宣使。侂胄以父任入官,歷閤門祗候、宣贊舍人、帶御器械。淳熙末,以汝州防禦使知閤門事。



 孝宗崩,光宗以疾不能執喪,中外洶洶,趙汝愚議定策立皇子嘉王。時憲聖太后居慈福宮,而侂胄雅善慈福內侍張宗尹,汝愚乃使侂胄介宗尹以其議密啟太后。侂胄兩至宮門,不獲命,彷徨欲退,遇重華宮提舉關禮問故,入白憲聖,言甚懇切,憲聖可其議。禮
 以告侂胄,侂胄馳白汝愚。日已向夕,汝愚亟命殿帥郭杲以所部兵夜分衛南北內。翌日,憲聖太后即喪次垂簾,宰臣傳旨,命嘉王即皇帝位。



 寧宗既立,侂胄欲推定策恩,汝愚曰:「吾宗臣也,汝外戚也,何可以言功?惟爪牙之臣,則當推賞。」乃加郭杲節鉞,而侂胄但遷宜州觀察使兼樞密都承旨。侂胄始觖望,然以傳導詔旨,浸見親幸,時時乘間竊弄威福。朱熹白汝愚當用厚賞酬其勞而疏遠之,汝愚不以為意。右正言黃度欲劾侂胄,謀泄,
 斥去。朱熹奏其姦,侂胄怒,使優人峩冠闊袖象大儒,戲于上前,熹遂去。彭龜年請留熹而逐侂胄。未幾,龜年與郡;侂胄進保寧軍承宣使,提舉祐神觀。自是,胄益用事,而以抑賞故,怨汝愚日深。



 霅川劉㢸者,曩與侂胄同知閤門事,頗以知書自負。方議內禪時,汝愚獨與侂胄計議,㢸弗得與聞,內懷不平,至是,謂侂胄曰:「趙相欲專大功,君豈惟不得節度,將恐不免嶺海之行矣。」侂胄愕然,因問計,㢸曰:「惟有用臺諫爾。」侂胄問:「若何而可?」㢸曰:「
 御筆批出是也。」侂胄悟,即以內批除所知劉德秀為監察御史,楊大法為殿中侍御史;罷吳獵監察御史,而用劉三傑代之。於是言路皆侂胄之黨,汝愚之跡始危。



 侂胄欲逐汝愚而難其名,謀於京鏜,鏜曰:「彼宗姓,誣以謀危社稷可也。」慶元元年,侂胄引李沐為右正言。沐嘗有求於汝愚不獲,即奏汝愚以同姓居相位,將不利於社稷,汝愚罷相。始,侂胄之見汝愚,徐誼實薦之,汝愚既斥,遂併逐誼。朱熹、彭龜年、黃度、李祥、楊簡、呂祖儉等以攻
 侂胄得罪,太學生楊宏中、張衟、徐範、蔣傅、林仲麟、周端朝等又以上書論侂胄編置,朝士以言侂胄遭責者數十人。



 已而侂胄拜保寧軍節度使、提舉祐神觀。又設偽學之目,以網括汝愚、朱熹門下知名之士。用何澹、胡紘為言官。澹言偽學宜加風厲,或指汝愚為偽學罪首。紘條奏汝愚有十不遜,且及徐誼。汝愚謫永州,誼謫南安軍。慮他日汝愚復用,密諭衡守錢鍪圖之,汝愚抵衡暴薨。留正舊在都堂眾辱侂胄,至是,劉德秀論正引用偽
 黨,正坐罷斥。吏部尚書葉翥要侍郎倪思列疏論偽學,思不從,侂胄乃擢翥執政而免思官。侂胄加開府儀同三司。時臺諫迎合侂胄意,以攻偽學為言,然憚清議,不欲顯斥熹。侂胄意未快,以陳賈嘗攻熹,召除賈兵部侍郎。未至,亟除沈繼祖臺察。繼祖誣熹十罪,落職罷祠。三年,劉三傑入對,言前日偽黨,今變而為逆黨。侂胄大喜,即日除三傑為右正言,而坐偽學逆黨得罪者五十有九人。王沇獻言令省部籍記偽學姓名,姚愈請降詔嚴
 偽學之禁,二人皆得遷官。施康年、陳讜、鄧友龍、林采皆以攻偽學久居言路,而張釜、張巖、程松率由此秉政。



 四年,侂胄拜少傅,封豫國公。有蔡璉者嘗得罪,汝愚執而黥之。五年,侂胄使璉告汝愚定策時有異謀,具其賓客所言七十紙。侂胄欲逮彭龜年、曾三聘、徐誼、沈有開下大理鞫之,張仲藝力爭乃止。其年遷少師,封平原郡王。六年,進太傅。婺州布衣呂祖泰上書言道學不可禁,請誅侂胄,以周必大為相。侂胄大怒,決杖流欽州。言者希
 侂胄意,劾必大首植偽黨。降為少保。一時善類悉罹黨禍,雖本侂胄意,而謀實始京鏜。逮鏜死,侂胄亦稍厭前事,張孝伯以為不弛黨禁,後恐不免報復之禍。侂胄以為然,追復汝愚、朱熹職名,留正、周必大亦復秩還政,徐誼等皆先後復官。偽黨之禁寖解。



 三年,拜太師。監惠民局夏允中上書,請侂胄平章國政,侂胄繆為辭謝,乞致其仕,詔不許,允中放罷。時侂胄以勢利蠱士大夫之心,薛叔似、辛棄疾、陳謙皆起廢顯用,當時固有困於久斥,
 損晚節以規榮進者矣。若陳自強則以侂胄童子師,自選人不數年致位宰相,而蘇師旦、周筠又侂胄廝役也,亦皆預聞國政,超取顯仕。群小阿附,勢焰熏灼。侂胄凡所欲為,宰執惕息不敢為異,自強至印空名敕劄授之,惟所欲用,三省不預知也。言路厄塞,每月舉論二三常事而已,謂之月課。



 或勸侂胄立蓋世功名以自固者,於是恢復之議興。以殿前都指揮使吳曦為興州都統,識者多言曦不可,主西師必叛,侂胄不省。安豐守厲仲方
 言淮北流民願歸附,會辛棄疾入見,言敵國必亂必亡,願屬元老大臣預為應變計,鄭挺、鄧友龍等又附和其言。開禧改元,進士毛自知廷對,言當乘機以定中原,侂胄大悅。詔中外諸將密為行軍之計。先是,楊輔、傅伯成言兵不可動,抵罪。至是,武學生華岳叩閽乞斬侂胄、蘇師旦、周筠以謝天下,諫議大夫李大異亦論止開邊。岳下大理劾罪編置,大異斥去。



 陳自強援故事乞命侂胄兼領平章,臺諫鄧友龍等繼以為請,侂胄除平章軍國
 事。蕭逵、李壁時在太常,論定典禮,三日一朝,因至都堂,序班丞相之上,三省印並納其第。侂胄昵蘇師旦為腹心,除師旦安遠軍節度使。自置機速房於私第,甚者假作御筆,陞黜將帥,事關機要,未嘗奏稟,人莫敢言。



 二年,以薛叔似為京湖宣諭使;鄧友龍為兩淮宣諭使;程松為四川宣撫使,吳曦副之。徐邦憲自處州召見,以弭兵為言,忤侂胄意,削二秩。於是左司諫易袚、大理少卿陳景俊、太學博士錢廷玉皆起而言恢復之計矣。詔侂胄
 日一朝。友龍、叔似並升宣撫使。吳曦兼陝西、河東招撫使,皇甫斌副之。時鎮江武鋒軍統制陳孝廣復泗州及虹縣,江州統制許進復新息縣,光州孫成復褒信縣。捷書聞,侂胄乃議降詔趣諸將進兵。



 未幾,皇甫斌兵敗于唐州;秦世輔至城固軍潰;郭倬、李汝翼敗於宿州,敵追圍倬,倬執統制田俊邁以遺敵,乃獲免。事聞,鄧友龍罷,以丘崈代為宣撫使。侂胄既喪師,始覺為師旦所誤。侂胄招李壁飲酒,酒酣,語及師旦,壁微摘其過,侂胄以為
 然。壁乃悉數其罪,贊侂胄斥去之。翌日,師旦謫韶州,斬郭倬於京口,流李汝翼、王大節、李爽於嶺南。



 已而金人渡淮,攻廬、和、真、揚,取安豐、濠,又攻襄陽,至棗陽,乃以丘崈僉書樞密院事,督視江、淮軍馬。侂胄輸家財二十萬以助軍,而諭丘崈募人持書幣赴敵營,謂用兵乃蘇師旦、鄧友龍、皇甫斌所為,非朝廷意。金人答書辭甚倨,且多所要索,謂「侂胄無意用兵,師旦等安得專?」崈又遣書許還淮北流民及今年歲幣,金人乃有許意。



 會招撫使
 郭倪與金人戰,敗於六合;金人攻蜀,吳曦叛,受金命稱蜀王。崈乞移書敵營伸前議,且謂金人指太師平章為首謀,宜免繫銜。侂胄忿,崈坐罷。曦反狀聞,舉朝震駭。侂胄亟遺曦書,許以茅土之封,書未達而安丙、楊巨源已率義士誅曦矣。侂胄連遣方信孺使北請和,以林拱辰為通謝使。金人欲責正隆以前禮賂,以侵疆為界,且索犒軍銀凡數千萬,而縛送首議用兵之臣。信孺歸,白事朝堂,不敢斥言,侂胄窮其說,乃微及之。侂胄大怒,和議
 遂輟。起辛棄疾為樞密都承旨。會棄疾死,乃以殿前副都指揮使趙淳為江淮制置使,復銳意用兵。



 自兵興以來,蜀口、漢、淮之民死於兵戈者,不可勝計,公私之力大屈,而侂胄意猶未已,中外憂懼。禮部侍郎史彌遠,時兼資善堂翊善,謀誅侂胄,議甚秘,皇子榮王入奏,楊皇后亦從中力請,乃得密旨。彌遠以告參知政事錢象祖、李壁。御筆云:「韓侂胄久任國柄,輕啟兵端,使南北生靈枉罹凶害,可罷平章軍國事,與在外宮觀。陳自強阿附充位,
 不恤國事,可罷右丞相。日下出國門。」仍令權主管殿前司公事夏震以兵三百防護。象祖欲奏審,壁謂事留恐泄,不可。翌日,侂胄入朝,震呵止於途,擁至玉津園側殛殺之。



 先一日,周筠謂侂胄,事將不善,侂胄與自強謀用林行可為諫議大夫,盡擊謀侂胄者。是日,行可方請對,自強坐待漏院,語同列曰:「今日大成上殿。」俄侂胄先驅至,象祖色變。尋報侂胄已押出,象祖乃入奏。有詔斬蘇師旦於廣東。嘉定元年,金人求函侂胄首,乃命臨安府
 斵侂胄棺,取其首遺之。



 侂胄用事十四年,威行宮省,權震寓內。嘗鑿山為園,下瞰宗廟。出入宮闈無度。孝宗疇昔思政之所,偃然居之,老宮人見之往往垂涕。顏棫草制,言其得聖之清。易袚撰答詔,以元聖褒之。四方投書獻頌者,謂伊、霍、旦、奭不足以似其勳,有稱為「我王」者。余𡕇請加九錫,趙師𢍰乞置平原郡王府官屬。侂胄皆當之不辭。所嬖妾張、譚、王、陳皆封郡國夫人,號「四夫人」,每內宴,與妃嬪雜坐,恃勢驕倨,掖庭皆惡之;其下受封者
 尤眾。至是,論四夫人罪,或杖或徒,餘數十人縱遣之。有司籍其家,多乘輿服御之飾,其僭紊極矣。



 始,侂胄以導達中外之言,遂見寵任。朱熹、彭龜年既以論侂胄去,貴戚吳琚語人曰:「帝初無固留侂胄意,使有一人繼言之,去之易爾。而一時臺諫及執政大臣多其黨與,故稔其惡以底大僇。」開禧用兵,帝意弗善也。侂胄死,寧宗諭大臣曰:「恢復豈非美事,但不量力爾。」



 侂胄娶憲聖吳皇后姪女,無子,取魯𡬐子為後,名㣉,既誅侂胄,削籍流沙門
 島云。



 丁大全,字子萬,鎮江人。面藍色。嘉熙二年舉進士,調蕭山尉。上謁帥閫,安撫使史巖之俟眾賓退,獨留大全,款曲甚至,期以他日必大用。大全為戚里婢婿,寅緣以取寵位。事內侍盧允昇、董宋臣。累官為大理司直、添差通判饒州。入為太府寺簿,調尚書茶鹽所檢閱江州分司,復兼樞密院編修官。拜右正言兼侍講,辭。改右司諫,拜殿中侍御史。



 升侍御史兼侍讀。劾奏丞相董槐,章未下,
 大全夜半調隅兵百餘人,露刃圍槐第,以臺牒驅迫之出,紿令輿槐至大理寺,欲以此恐之。須臾,出北關,棄槐,嘂呼而散。槐徐步入接待寺,罷相之命下矣。自是志氣驕傲,道路以目。



 尋為右諫議大夫,進端明殿學士、僉書樞密院事,封丹陽郡侯,進同知樞密院事兼權參知政事。寶祐六年,拜參知政事。四月,拜右丞相兼樞密使,進封公。初,大全以袁玠為九江制置副使,玠貪且刻,逮繫漁湖土豪,督促輸錢甚急。土豪怒,盡以漁舟濟北來之
 兵。太學生陳宗、劉黼、黃鏞、曾唯、陳宜中、林則祖等六人,伏闕上書訟大全。臺臣翁應弼、吳衍為大全鷹犬,鈐制學校,貶逐宗等。



 開慶元年九月,罷相,以觀文殿大學士判鎮江府。中書舍人洪芹繳言:「大全鬼蜮之資,穿窬之行,引用兇惡,陷害忠良,遏塞言路,濁亂朝綱。乞追官遠竄,以伸國法,以謝天下。」侍御史沈炎、右正言曹永年相繼論罷。監察御史朱貔孫復論:「大全姦回險狡,狠毒貪殘,假陛下之刑威以箝天下之口,挾陛下之爵祿以籠天下
 之財。」監察御史饒虎臣又論大全四罪:絕言路,壞人才,竭民力,誤邊防。再削其官。景定元年,詔守中奉大夫致仕。臣僚言「乞遠竄使不失刑」,詔送南康軍居住。臺臣復以為言,追三官,移送南安軍居住。



 明年,監察御史劉應龍請加竄,追削兩官,移竄貴州團練使。與州守游翁明失色杯酒間,翁明訴大全陰造弓矢,將通蠻為不軌。朱禩孫以聞于朝。又明年,移置新州。太常少卿兼權直舍人院劉震孫繳奏,乞移徙海島。四年正月,將官畢遷護
 送,舟過藤州,擠之于水而死。



 大全知淮西,總領鄭羽富甲吳門,始欲結婣,羽不從。遂令臺臣卓夢卿彈之,籍其家。為子壽翁聘婦,見其豔,自取,為世所醜。



 賈似道,字師憲,台州人,制置使涉之子也。少落魄,為遊博,不事操行。以父蔭補嘉興司倉。會其姊入宮,有寵于理宗,為貴妃,遂詔赴廷對,妃於內中奉湯藥以給之。擢太常丞、軍器監。益恃寵不檢,日縱游諸妓家,至夜即燕遊湖上不反。理宗嘗夜憑高,望西湖中燈火異常時,語
 左右曰:「此必似道也。」明日詢之果然,使京尹史巖之戒敕之。巖之曰:「似道雖有少年氣習,然其材可大用也。」尋出知澧州。



 ,改湖廣總領。三年,加戶部侍郎。五年,以寶章閣直學士為沿江制置副使、知江州兼江西路安撫使。一歲中,再遷京湖制置使兼知江陵府,調度賞罰,得以便宜施行。九年,加寶文閣學士、京湖安撫制置大使。十年,以端明殿學士移鎮兩淮,年始三十餘。寶祐二年,加同知樞密院事、臨海郡開國公,威權日盛。臺
 諫嘗論其二部將,即毅然求去。孫子秀新除淮東總領,外人忽傳似道已密奏不可矣,丞相董槐懼,留身請之,帝以為無有,槐終不敢遣子秀,以似道所善陸壑代之,其見憚已如此。四年,加參知政事。五年,加知樞密院事。六年,改兩淮宣撫大使。



 自端平初,孟珙帥師會大元兵共滅金,約以陳、蔡為界。師未還而用趙范謀,發兵據殽、函,絕河津,取中原地,大元兵擊敗之,范僅以數千人遁歸。追兵至,問曰:「何為而敗盟也?」遂縱攻淮、漢,自是兵端
 大啟。



 開慶初,憲宗皇帝自將征蜀,世祖皇帝時以皇弟攻鄂州,元帥兀良哈台由雲南入交阯,自邕州蹂廣西,破湖南,傳檄數宋背盟之罪。理宗大懼,乃以趙葵軍信州,禦廣兵;以似道軍漢陽,援鄂,即軍中拜右丞相。十月,鄂東南陬破,宋人再築,再破之,賴高達率諸將力戰。似道時自漢陽入督師。十一月,攻城急,城中死傷者至萬三千人。似道乃密遣宋京詣軍中請稱臣,輸歲幣,不從。會憲宗皇帝晏駕于釣魚山,合州守王堅使阮思聰踔
 急流走報鄂,似道再遣京議歲幣,遂許之。大元兵拔砦而北,留張傑、閻旺以偏師候湖南兵。明年正月,兵至,傑作浮梁新生磯,濟師北歸。似道用劉整計,攻斷浮梁,殺殿兵百七十,遂上表以肅清聞。帝以其有再造功,以少傅、右丞相召入朝,百官郊勞如文彥博故事。



 初,似道在漢陽,時丞相吳潛用監察御史饒應子言,移之黃州,而分曹世雄等兵以屬江閫。黃雖下流,實兵衝。似道以為潛欲殺己,銜之。且聞潛事急時,每事先發後奏;帝欲立榮
 王子孟啟為太子,潛又不可。帝已積怒潛,似道遂陳建儲之策,令沈炎劾潛措置無方,致全、衡、永、桂皆破,大稱旨。乃議立孟啟,貶潛循州,盡逐其黨人。高達在圍中,恃其武勇,殊易似道,每見其督戰,即戲之曰:「巍巾者何能為哉!」每戰,必須勞始出,否即使兵士嘩于其門。呂文德諂似道,即使人呵曰:「宣撫在,何敢爾邪!」曹世雄、向士璧在軍中,事皆不關白似道,故似道皆恨之。以覈諸兵費,世雄、士璧皆坐侵盜官錢貶遠州。每言于帝欲誅達,帝
 知其有功,不從。尋論功,以文德為第一,而達居其次。



 明年,大元世祖皇帝登極,遣翰林侍讀學士、國信使郝經等持書申好息兵,且徵歲幣。似道方使廖瑩中輩撰《福華編》稱頌鄂功,通國皆不知所謂和也。似道乃密令淮東制置司拘經等於真州忠勇軍營。



 時理宗在位久,內侍董宋臣、盧允昇為之聚斂以媚之。引薦奔競之士,交通賄賂,置諸通顯。又用外戚子弟為監司、郡守。作芙蓉閣、香蘭亭宮中,進倡優傀儡,以奉帝為游燕。竊弄權柄。
 臺臣有言之者,帝宣諭去之,謂之「節貼」。



 似道入,逐盧、董所薦林光世等,悉罷之,勒外戚不得為監司、郡守,子弟門客斂跡,不敢干朝政。由是權傾中外,進用群小。取先朝舊法,率意紛更,增吏部七司法。買公田以罷和糴,浙西田畝有直千緡者,似道均以四十緡買之。數稍多,予銀絹;又多,予度牒告身。吏又恣為操切,浙中大擾。有奉行不至者,提領劉良貴劾之。有司爭相迎合,務以買田多為功,皆繆以七八斗為石。其後,田少與磽瘠、虧租與
 佃人負租而逃者,率取償田主。六郡之民,破家者多。包恢知平江,督買田,至以肉刑從事。復以楮賤,作銀關,以一準十八界會之三,自製其印文如「賈」字狀行之,十七界廢不用。銀關行,物價益踴,楮益賤。秋七月,彗出柳,光燭天,長數十丈,自四更見東方,日高始滅。臺諫、布韋皆上書,言此公田不便,民間愁怨所致。似道上書力辯之,且乞罷政。帝勉留之曰;「公田不可行,卿建議之始,朕已沮之矣。今公私兼裕,一歲軍餉,皆仰於此。使因人言而
 罷之,雖足以快一時之議,如國計何!」有太學生蕭規、葉李等上書,言似道專政。命京尹劉良貴捃摭以罪,悉黥配之。後又行推排法。江南之地,尺寸皆有稅,而民力弊矣。



 理宗崩,度宗又其所立,每朝必答拜,稱之曰「師臣」而不名,朝臣皆稱為「周公」。甫葬理宗,即棄官去,使呂文德報北兵攻下沱急,朝中大駭,帝與太后手為詔起之。似道至,欲以經筵拜太師,以典故須建節,授鎮東軍節度使,似道怒曰:「節度使粗人之極致爾!」遂命出節,都人聚
 觀。節已出,復曰:「時日不利。」亟命返之。宋制:節出,有撤關壞屋,無倒節理,以示不屈。至是,人皆駭歎。然下沱之報實無兵也。三年,又乞歸養。大臣、侍從傳旨留之者日四五至,中使加賜賚者日十數至,夜即交臥第外以守之。除太師、平章軍國重事,一月三赴經筵,三日一朝,赴中書堂治事。賜第葛嶺,使迎養其中。吏抱文書就第署,大小朝政,一切決于館客廖瑩中、堂吏翁應龍,宰執充位署紙尾而已。



 似道雖深居,凡臺諫彈劾、諸司薦辟及京
 尹、畿漕一切事,不關白不敢行,李芾、文天祥、陳文龍、陸達、杜淵、張仲微、謝章輩,小忤意輒斥,重則屏棄之,終身不錄。一時正人端士,為似道破壞殆盡。吏爭納賂求美職,其求為帥閫、監司、郡守者,貢獻不可勝計。趙溍輩爭獻寶玉,陳奕至以兄事似道之玉工陳振民以求進,一時貪風大肆。五年,復稱疾求去。帝泣涕留之,不從。令六日一朝,一月兩赴經筵。六年,命入朝不拜。朝退,帝必起避席,目送之出殿廷始坐。繼又令十日一入朝。



 時襄陽
 圍已急,似道日坐葛嶺,起樓閣亭榭,取宮人娼尼有美色者為妾,日淫樂其中。惟故博徒日至縱博,人無敢窺其第者。其妾有兄來,立府門,若將入者,似道見之,縛投火中。嘗與群妾踞地鬥蟋蟀,所狎客入,戲之曰:「此軍國重事邪?」酷嗜寶玩,建多寶閣,日一登玩。聞余玠有玉帶,求之,已殉葬矣,發其塚取之。人有物,求不予,輒得罪。自是,或累月不朝,帝如景靈宮亦不從駕。八年,明堂禮成,祀景靈宮。天大雨,似道期帝雨止升輅。胡貴嬪之父顯
 祖為帶御器械,請如開禧故事,卻輅,乘逍遙輦還宮,帝曰平章云云,顯祖紿曰:「平章已允乘逍遙輦矣。」帝遂歸。似道大怒曰:「臣為大禮使,陛下舉動不得預聞,乞罷政。」即日出嘉會門,帝留之不得,乃罷顯祖,涕泣出貴嬪為尼,始還。



 似道既專恣日甚,畏人議己,務以權術駕馭,不愛官爵,牢籠一時名士,又加太學餐錢,寬科場恩例,以小利啖之。由是言路斷絕,威福肆行。



 自圍襄陽以來,每上書請行邊,而陰使臺諫上章留己。呂文煥以急告,似
 道復申請之,事下公卿雜議。監察御史陳堅等以為師臣出,顧襄未必能及淮,顧淮未必能及襄,不若居中以運天下為得。乃就中書置機速房以調邊事。時物議多言高達可援襄陽者,監察御史李旺率朝士入言於似道。似道曰:「吾用達,如呂氏何?」旺等出,歎曰:「呂氏安則趙氏危矣。」文煥在襄,聞達且入援,亦不樂,以語其客。客曰:「易耳,今朝廷以襄陽急,故遣達援之,吾以捷聞,則達必不成遣矣。」文煥大以為然。時襄兵出,獲哨騎數人,即繆
 以大捷奏,然不知朝中實無援襄事也。襄陽降,似道曰:「臣始屢請行邊,先帝皆不之許,向使早聽臣出,當不至此爾。」



 十月,其母胡氏薨,詔以天子鹵簿葬之,起墳擬山陵,百官奉襄事,立大雨中,終日無敢易位。尋起復入朝。



 度宗崩。大兵破鄂,太學諸生亦群言非師臣親出不可。似道不得已,始開都督府臨安,然憚劉整,不行。明年正月,整死,似道欣然曰:「吾得天助也。」乃上表出師,抽諸路精兵以行,金帛輜重之舟,舳臚相銜百餘里。至安吉,似
 道所乘舟膠堰中,劉師勇以千人入水曳之不能動,乃易他舟而去。至蕪湖,遣還軍中所俘曾安撫,以荔子、黃甘遺丞相伯顏,俾宋京如軍中,請輸歲幣稱臣如開慶約,不從。夏貴自合肥以師來會,袖中出編書示似道曰:「宋曆三百二十年。」似道俛首而已。時一軍七萬餘人,盡屬孫虎臣,軍丁家洲。似道與夏貴以少軍軍魯港。二月庚申夜,虎臣以失利報,似道倉皇出呼曰:「虎臣敗矣!」命召貴與計事。頃之,虎臣至,撫膺而泣曰:「吾兵無一人用
 命也。」貴微笑曰:「吾嘗血戰當之矣。」似道曰:「計將安出?」貴曰:「諸軍已膽落,吾何以戰?公惟入揚州,招潰兵,迎駕海上,吾特以死守淮西爾。」遂解舟去。似道亦與虎臣以單舸奔揚州。明日,敗兵蔽江而下,似道使人登岸揚旗招之,皆不至,有為惡語慢罵之者。乃檄列郡如海上迎駕,上書請遷都,列郡守於是皆遁,遂入揚州。



 陳宜中請誅似道,謝太后曰:「似道勤勞三朝,安忍以一朝之罪,失待大臣之禮。」止罷平章、都督,予祠官。三月,除似道諸不恤
 民之政,放還諸竄謫人,復吳潛、向士璧等官,誅其幕官翁應龍,廖瑩中、王庭皆自殺。潘文卿、季可、陳堅、徐卿孫皆似道鷹犬,至是交章劾之。四月,高斯得乞誅似道,不從。而似道亦自上表乞保全,乃命削三官,然尚居揚不歸。五月,王爚論似道既不死忠,又不死孝,太皇太后乃詔似道歸終喪。七月,黃鏞、王應麟請移似道鄰州,不從。王爚入見太后曰:「本朝權臣稔禍,未有如似道之烈者。縉紳草茅不知幾疏,陛下皆抑而不行,非惟付人言
 於不恤,何以謝天下!」始徙似道婺州。婺人聞似道將至,率眾為露布逐之。監察御史孫嶸叟等皆以為罰輕,言之不已。又徙建寧府。翁合奏言:「建寧乃名儒朱熹故里,雖三尺童子粗知向方,聞似道來嘔惡,況見其人!」時國子司業方應發權直舍人院,封還錄黃,乞竄似道廣南;中書舍人王應麟、給事中黃鏞亦言之,皆不從。侍御史陳文龍乞俯從眾言,陳景行、徐直方、孫嶸叟及監察御史俞浙並上疏,於是始謫似道為高州團練使、循州安置,
 籍其家。



 福王與芮素恨似道,募有能殺似道者使送之貶所,有縣尉鄭虎臣欣然請行。似道行時,侍妾尚數十人,虎臣悉屏去,奪其寶玉,撤轎蓋,𣌈行秋日中,令舁轎夫唱杭州歌謔之,每名斥似道,辱之備至。似道至古寺中,壁有吳潛南行所題字,虎臣呼似道曰:「賈團練,吳丞相何以至此?」似道慚不能對。嶸叟、應麟奏似道家畜乘輿服御物,有反狀,乞斬之。詔遣鞫問,未至。八月,似道至漳州木綿庵,虎臣屢諷之自殺,不聽,曰:「太皇許我不死,
 有詔即死。」虎臣曰:「吾為天下殺似道,雖死何憾?」拉殺之。



\end{pinyinscope}