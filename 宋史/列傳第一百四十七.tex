\article{列傳第一百四十七}

\begin{pinyinscope}

 周執羔王希呂陳良祐李浩陳橐胡沂唐文若李燾



 周執羔字表卿,信州弋陽人。宣和六年舉進士,廷試,徽宗擢為第二。授湖州司士曹事,俄除太學博士。



 建炎初,
 乘輿南渡,自京師奔詣揚州,不及,遂從隆祐太后於江西,還覲會稽。尋以繼母劉疾,乞歸就養,調撫州宜黃縣丞。時四境俶擾,潰卒相挻為變,令大恐,不知所為,執羔諭以禍福,皆斂手聽命。既又訹其黨,執首謀者斬以徇。邑人德之,至繪像立祠。



 紹興五年,改秩,通判湖州。丁母憂,服闋,通判平江府。召為將作監丞。明年春,遷太常丞。會始議建明堂,大樂久廢不修,詔奉常習肄之,訪輯舊聞,庀閱工器,制作始備。累遷右司員外郎。



 八月,擢權禮
 部侍郎,充賀金生辰使。往歲奉使官得自闢其屬,賞典既厚,願行者多納金以請,執羔始拒絕之。使還,兼權吏部侍郎。請賜新進士聞喜宴於禮部,從之。軍興廢此禮,至是乃復。同知貢舉。舊例,進士試禮部下,歷十八年得免舉,又四試禮部下,始特奏名推恩。秦檜既以科第私其子,士論喧嘩,為減三年以悅眾。執羔言祖宗法不可亂,繇此忤檜,御史劾罷之。



 又六年,起知眉州,徙閬州,又改夔州,兼夔路安撫使。夔部地接蠻獠,易以生事。或告
 溱、播夷叛,其豪帥請遣兵致討,執羔謂曰:「朝廷用爾為長,今一方繹騷,責將焉往,能盡力則貰爾,一兵不可得也。」豪懼,斬叛者以獻,夷人自是皆惕息。三十年,知饒州,尋除敷文閣待制。



 乾道初,守婺州,召還,提舉祐神觀兼侍講。首進二說,以為王道在正心誠意,立國在節用愛人。二年四月,復為禮部侍郎。孝宗患人才難知,執羔曰:「今一介干進,亦蒙賜召,口舌相高,殆成風俗,豈可使之得志哉!」上曰:「卿言是也。」一日侍經筵,自言「學《易》知數,臣
 事陛下之日短」,已乃垂涕,上惻然。即拜本部尚書,升侍讀,固辭,不許。



 方士劉孝榮言《統元歷》差,命執羔厘正之。執羔用劉義叟法,推日月交食,考五緯贏縮,以紀氣朔寒溫之候,撰《歷議》、《歷書》、《五星測驗》各一卷上之。



 上嘗問豐財之術,執羔以為:「蠹民之本,莫甚於兵。古者興師十萬,日費千金。今尺籍之數,十倍於此,罷癃老弱者幾半,不汰之其弊益深。」論:「和糴本以給軍興,豫兇災。蓋國家一切之政,不得已而為之。若邊境無事,妨於民食而務
 為聚斂,可乎?舊糴有常數,比年每郡增至一二十萬石。今諸路枯旱之餘,蟲螟大起,無以供常稅,況數外取之乎?宜視一路一郡一縣豐兇之數,輕重行之,災甚者蠲之可也。」上矍然曰:「災異如此,乃無一人為朕言者!」即詔從之。



 充安恭皇后菆宮按行使,日與閹人接,卒事未嘗交一談,閹亦服其長者,不怨也。拜疏求去,上謂輔臣曰:「朕惜其老成,宜以經筵留之。」除寶文閣學士,提舉祐神觀。上曰:「遂除龍圖可也。」經筵二年,每勸上以辨忠邪、納
 諫爭,上深知其忠。



 明年三月,告老,上諭曰:「祖宗時,近臣有年逾八十尚留者,卿之齒未也。」命卻其章。閏月,復申前請。上度不可奪,詔提舉江州太平興國宮,賜茶、藥、御書,恩禮尤渥,公卿祖帳都門外,搢紳榮之。時閩、粵、江西歲饑盜起,執羔陛辭以為言,詔遣太府丞馬希言使諸路振救之。乾道六年卒,年七十七。



 執羔有雅度,立朝無朋比。治郡廉恕,有循吏風。手不釋卷,尤通於《易》。



 王希呂字仲行,宿州人。渡江後自北歸南,既仕,寓居嘉
 興府。乾道五年,登進士科。孝宗獎用西北之士,六年,召試,授秘書省正字。除右正言。時張說以攀援戚屬擢用,再除簽書樞密院事,希呂與侍御史李衡交章劾之。上疑其合黨邀名,責遠小監當,既而悔之,改授宮觀。方說之見用,氣勢顯赫,後省不書黃,學士院不草詔,皆相繼斥逐,而希呂復以身任怨,去國之日,屏徒御,躡履以行,恬不為悔。由是直聲聞於遠邇,雖以此黜,亦以此見知。出知廬州。



 淳熙二年,除吏部員外郎,尋除起居郎兼中
 書舍人。淮右擇帥,上以希呂已試有功,令知廬州兼安撫使。修葺城守,安集流散,兵民賴之。加直寶文閣、江西轉運副使。



 五年,召為起居郎,除中書舍人、給事中,轉兵部尚書,改吏部尚書,求去,乃除端明殿學士、知紹興府。尋以言者落職,處之晏如。



 治郡百廢俱興,尤敬禮文學端方之士。天性剛勁,遇利害無回護意,惟是之從。嘗論近習用事,語極切至,上變色欲起,希呂換御衣曰:「非但臣能言之,侍從、臺諫皆有文字來矣。」佐漕江西,嘗作《拳
 石記》以示僚屬,一幕官舉筆塗數字,舉坐駭愕,希呂覽之,喜其不阿,薦之。



 居官廉潔,至無屋可廬,由紹興歸,有終焉之意,然猶寓僧寺。上聞之,賜錢造第。後以疾卒於家。



 陳良祐,字天與,婺州金華人。年十九,預鄉薦,間歲入太學。紹興二十四年,擢進士第。調興國軍司戶,未上,有薦於朝者,召除太學錄、樞密院編修官。中丞汪澈薦除監察御史,累遷軍器監兼鄧王府直講。隆興元年,出為福
 建路轉運副使。丁父憂,服闋,乾道三年,除起居舍人兼權中書舍人,遷起居郎。尋除左司諫。



 首言會子之弊,願捐內帑以紓細民之急。上曰:「朕積財何用,能散可也。」慨然發內府白金數萬兩收換會子,收銅版勿造,軍民翕然。未幾,戶部得請,改造五百萬。又奏:「陛下號令在前,不能持半歲久,以此令民,誰能信之?豈有不印交子五百萬,遂不可為國乎?」既而又欲造會子二千萬,屢爭之不得,遂請以五百萬換舊會,俟通行漸收之,常使不越千
 萬之數。



 上銳意圖治,以唐太宗自比,良祐言:「太宗《政要》願賜省覽,擇善而從,知非而戒,使臣為良臣,勿為忠臣。」上曰:「卿亦當以魏徵自勉。」



 又言:「陛下躬行節儉,弗殖貨利。或者托肺腑之親,為市井之行,以公侯之貴,牟商賈之利。占田疇,擅山澤,甚者發舶舟,招蕃賈,貿易寶貨,麋費金錢。或假德壽,或托椒房,犯法冒禁,專利無厭,非所以維持紀綱,保全戚畹。願嚴戒敕,茍能改過,富貴可保,如其不悛,以義斷恩。」



 時左相丁外艱,詔起復,良祐言:「起
 復非正禮,今無疆場之事,宜使之終喪。」遂寢。遷右諫議大夫兼侍講,同知貢舉,除給事中,兼直學士院,遷吏部侍郎。尋除尚書。



 時議遣泛使請地,良祐奏:「陛下恢復之志未嘗忘懷,然詞莫貴於僉同,不可不察;博訪歸於獨斷,不可不審。固有以用眾而興,亦有以用眾而亡;固有以獨斷而成,亦有以獨斷而敗。今遣使乃啟釁之端,萬一敵騎犯邊,則民力困於供輸,州郡疲於調發,兵拏禍結,未有息期。將帥庸鄙,類乏遠謀,對君父則言效死,臨
 戰陣則各求生。有如符離之役,不戰自潰,瓜洲之遇,望敵驚奔,孰可仗者?此臣所以未敢保其萬全。且今之求地,欲得河南,曩歲嘗歸版圖,不旋踵而又失,如其不許,徒費往來,若其許我,必邀重幣。經理未定,根本內虛,又將隨而取之矣。向之四郡得之亦勤,尚不能有,今又無故而求侵地,陛下度可以虛聲下之乎?況止求陵寢,地在其中,曩亦議此,觀其答書,幾於相戲。凡此二端,皆是求釁。必須遣使,則祈請欽宗梓宮,猶為有辭。內視不足,
 何暇事外?邇者未懷,豈能綏遠?」



 奏入,忤旨,貶瑞州居住,尋移信州。九年,許令自便。淳熙四年,起知徽州,尋除敷文閣待制、知建寧府,卒。



 李浩,字德遠,其先居建昌,遷臨川。浩早有文稱。紹興十二年,擢進士第。時秦熹挾宰相子以魁多士,同年皆見之,或拉浩行,毅然不往。調饒州司戶參軍、襄陽府觀察推官,連丁內外艱,繼調金州教授,改太常寺主簿,尋兼光祿寺丞。



 輪對,首陳《無逸》之戒,且言:「宿衛大將楊存中
 恩寵特異,待之過,非其福。」上悟,旋令就第。自秦檜用事,塞言路,及上總攬權綱,激厲忠讜,此習尚存,朝士多務慎默。至是命百官轉對,浩與王十朋、馮方、查鑰、胡憲始相繼言事,聞者興起。



 浩不安於朝,請祠,主管臺州崇道觀以歸。孝宗即位,以太常丞召。時張浚督師江、淮,宰相多抑之,浩引仁宗用韓琦、範仲淹詔章得像故事,乞戒諭令同心協濟。兼權吏部郎官。浩雅為湯思退所厚,御史尹穡欲引之以共擠浚,因薦浩。及對,乃明示不同之
 意,二人皆不樂。逾年,始除員外郎兼皇子恭王府直講。



 在王府多所裨益,且因事以及時政,書之於冊,幸上或見之,王亦素所愛重。他日外補,累年以歸,王喜曰:「李直講來矣。」未幾,宰相召為郎者四人,將進用之,尤屬意浩。浩嘿然無一辭,同舍皆遷,浩獨如故。



 逾年,浙河水災,詔郎官、館職以上條時政闕失,浩謂上憂勞如此,今何可不言,即奏疏指論近臣,並及宰執惟奉行,臺諫多迎合,百執事顧忌畏縮。反復數千言,傾倒罄竭,見者悚慄。上
 不以為忤,執事者深忌之。



 乞外,得臺州。州有揀中禁軍五百人,訓練官貪殘失眾心,不逞者因謀作亂,忽露刃於庭,浩謂之曰:「汝等欲為亂乎?請先殺我。」眾駭曰:「不敢。」乃徐推其為首者四人黥徙之,迄無事。除直秘閣。並海有宿寇,久不獲,浩募其徒,自縛贖罪,即得其魁。



 里豪民鄭憲以貲給事權貴人門,囊橐為奸,事覺,械系之,死獄中,盡籍其家,徙其妻孥。權貴人教其家訟冤,且誣浩以買妾事,言者用是擠之。疏方上,權參政劉珙越次奏曰:「
 李浩為郡,獲罪豪民,為其所誣,臣考其本末甚白。」上顧曰:「守臣不畏強御,豈易得邪?」且門章安在,珙袖出之,遂留中不下。大理觀望,猶欲還其所沒貲,上批其後曰:「臺州所斷至甚允當,鄭憲家資,永不給還,流徙如故。」浩始得安。



 明年,除司農少卿。時朝廷和糴米八萬,董其事者賤糴濕惡,隱克官錢,戶部不敢詰。浩白發其奸,下有司窮竟。戶部欲就支稽見數,大理附會之,浩爭曰:「非但惠奸,且虧軍食。」上是其言。會大理奏結他獄,上顧輔臣曰:「
 棘寺官得剛正如李浩者為之。」已而卿缺,又曰:「無以易浩。」遂除大理卿。



 時上英明,有大有為之志,廷臣不能奉行,誕慢茍且,依違避事。浩前在司農,嘗因面對,陳經理兩淮之策,至是為金使接伴還,奏曰:「臣親見兩淮可耕之田,盡為廢地,心嘗痛之。條畫營屯,以為恢復根本。」又言:「比日措置邊事甚張皇,願戒將吏嚴備御,無規微利近功。日與大臣修治具,結人心,持重安靜,以俟敵釁。」上悉嘉納。



 宰相議遣泛使,浩與辨其不可,至以官職訹之,
 浩怒,以語觸之,且力求外。以直寶文閣知靜江府兼廣西安撫。有尚書郎入對,論及擇帥事,上曰:「如廣西,朕已得李浩矣。」又諭大臣曰:「李浩營田議甚可行。」大臣莫有應者。



 浩至郡,舊有靈渠通漕運及灌溉,歲久不治,命疏而通之,民賴其利。邕管所隸安平州,其酋恃險,謀聚兵為邊患,浩遣單使諭以禍福,且許其引赦自新,即日叩頭謝過,焚徹水柵,聽太府約束。



 治廣二年,召還,入對,論俗不美者八,其言曰:「陛下所求者規諫,而臣下專務迎
 合,所貴者執守,而臣下專務順從;所惜者名器,而僥幸之路未塞;所重者廉恥,而趣附之門尚開;儒術可行,而有險詖之徒;下情當盡,而有壅蔽之患;期以氣節,而偷惰者得以茍容;責以實效,而誕慢者得以自售。」上問誕慢謂誰,浩具以實對。翌日,謂宰相曰:「李浩直諒。」遂除權吏部侍郎。時政府有怙寵竊權者,黨與非一,自浩之入,已相側目,且欲以甘言誘之,浩中立不倚,拒弗納。於是相與謀嗾諫議大夫姚憲論浩以強狠之資,挾奸諛之
 志,置之近列,變亂黑白。未及正謝而罷。



 乾道九年,提舉太平興國宮。明年夏,夔路闕帥,命浩以秘閣修撰寵其行。夔有羈縻州曰思州,世襲為守則田氏,與其猶子不協,將起兵相攻,浩草檄遣官為勸解,二人感悟,歃血盟,盡釋前憾,邊得以寧。逾年,以疾請祠,提舉玉隆萬壽宮,命未至,以淳熙三年九月卒,年六十一。諸司奏浩盡瘁其職以死,詔特贈集英殿修撰。



 浩天資質直,涵養渾厚,不以利害動其心。少力學為文辭,及壯益沉潛理義。立
 朝慨然以時事為己任,忠憤激烈,言切時弊,以此見忌於眾。平居未嘗假人以辭色,不知者以為傲,或譖於上前,上謂:「斯人無他,在朕前亦如此,非為傲者。」小人憚之,誘以祿利,正色不回,謀害之者無所不至,獨賴上察其衷,始終全之。為郡尤潔己,自海右歸,不載南海一物。平生奉養如布衣時,風裁素高,人不敢干以私云。



 陳橐,字德應,紹興餘姚人。入太學有聲,登政和上舍第,教授寧州。以母老改臺州士曹,治獄平允。更攝天臺、臨
 海、黃巖三邑,易越州新昌令,皆以愷悌稱。



 呂頤浩欲援為御史,約先一見,橐曰:「宰相用人,乃使之呈身耶?」謝不往。趙鼎、李光交薦其才。紹興二年五月,召對,改秩。六月,除監察御史,論事不合。八月,詔以宰邑有治行,除江西運判。瑞昌令倚勢受賂,橐首劾罷之。期年,所按以十數,至有望風解印綬者。



 以母年高,乞歸養,詔橐善撫字,移知臺州。臺有五邑,嘗攝其三,民懷惠愛,越境歡迎,不數月稱治。母喪,邦人巷哭,相率走行在所者千餘人,請起
 橐。詔橐清謹不擾,治狀著聞,其敕所在州賜錢三十萬。橐力辭,上謂近臣曰:「陳橐有古循吏風。」終喪,以司勛郎中召。



 累遷權刑部侍郎。時秦檜力主和議,橐疏謂:「金人多詐,和不可信。且二聖遠狩沙漠,百姓肝腦塗地,天下痛心疾首。今天意既回,兵勢漸集,宜乘時掃清,以雪國恥;否亦當按兵嚴備,審勢而動。舍此不為,乃遽講和,何以系中原之望。」



 既而金厚有所邀,議久不決,將再遣使,橐復言:「金每挾講和以售其奸謀。論者因其廢劉豫又
 還河南地,遂謂其有意於和,臣以為不然。且金之立豫,蓋欲自為捍蔽,使之南窺。豫每犯順,率皆敗北,金知不足恃,從而廢之,豈為我哉?河南之地欲付之他人,則必以豫為戒,故捐以歸我。往歲金書嘗謂歲帑多寡聽我所裁,曾未淹歲,反復如此。且割地通和,則彼此各守封疆可也,而同州之橋,至今存焉。蓋金非可以義交而信結,恐其假和好之說,騁謬悠之辭,包藏禍心,變出不測。願深鑒前轍,亦嚴戰守之備,使人人激厲,常若寇至。茍
 彼通和,則吾之振飭武備不害為立國之常。如其不然,決意恢復之圖,勿循私曲之說,天意允協,人心響應,一舉以成大勛,則梓宮、太后可還,祖宗疆土可復矣。」檜憾之。橐因力請去。未幾,金果渝盟。



 除徽猷閣待制、知穎昌府。時河南新疆初復,無敢往者,橐即日就道。次壽春則穎已不守。改處州,又改廣州。兵興後,廣東盜賊無寧歲,十年九易牧守。橐盡革弊政,以恩先之。留鎮三年,民夷悅服。



 初,朝廷移韓京一軍屯循州,會郴寇駱科犯廣西,
 詔遣京討之。橐奏:「廣東累年困於寇賊,自京移屯,敵稍知畏。今悉軍赴廣西,則廣東危矣。」檜以橐為京地,坐稽留機事,降秩。屢上章告老,改婺州,請不已,遂致仕。又十二年,以疾卒於家,年六十六。



 橐博學剛介,不事產業,先世田廬,悉推予兄弟。在廣積年,四方聘幣一不入私室。既謝事歸剡中,僑寓僧寺,日糴以食,處之泰然。王十朋為《風士賦》,論近世會稽人物,曰:「杜祁公之後有陳德應」云。



 胡沂,字周伯,紹興餘姚人。父宗伋,號醇儒,能守所學,不逐時好。沂穎異,六歲誦《五經》皆畢,不忘一字。紹興五年進士甲科,陸沉州縣幾三十載,至二十八年,始入為正字。遷校書郎兼實錄院檢討官,吏部員外郎。轉右司,以憂去,終喪還朝。孝宗受禪,除國子司業、鄧王府直講,尋擢殿中侍御史。



 有旨侍從、臺諫條具方今時務,沂言:「守禦之利,莫若令沿邊屯田。前歲淮民逃移,未復舊業,中原歸附,未知所處。俾之就耕,可贍給,省餉饋。東作方興,
 且慮敵人乘時驚擾,宜聚兵險隘防守。」詔行其言。



 御史中丞辛次膺論殿帥成閔黷貨不恤士卒之罪,詔罷殿前司職事,與祠。沂再言其二十罪,遂落太尉,婺州居住。



 沂又言:「將臣定十等之目,令其舉薦,施之擇將之頃則可,施之養士有素則未也。夫設武舉,立武學,試之以弓馬,又試之以韜略之文、兵機之策,蓋將有所用也。除高等一二名,餘皆吏部授以榷酤、征商,所養非所用,所用非所養,願詔大臣詳議,中舉者定品格,分差邊將下準
 備差遣,則人人思奮,應上之求矣。」從之。



 時龍大淵、曾覿以藩邸舊恩除知閣門事,張震、劉珙、周必大相繼繳回詞命。沂論其市權招士,請屏遠之,未聽,而諫官劉度坐抗論左遷。沂累章,益懇切,曰:「大淵、覿不屏去,安知無柳宗元、劉禹錫輩撓節以從之者。」好進者嫉其言,共排之,沂亦以言不行請去,遂以直顯謨閣主管臺州崇道觀。



 乾道元年冬,召為宗正少卿兼皇子慶王府贊讀,尋兼侍講,進中書舍人、給事中。進對,論命令當謹之於造命
 之初,上曰:「三代盛時如此。卿職在繳駁,事有當然,勿謂拂君相不言。」除吏部侍郎兼權尚書。



 沂奏:「七司法自紹興十三年纂修成書,歲且一紀,歷月閱時,不無抵牾。望令敕令所官討論章旨,此法可行不可行,此條當革不當革,將見行之法與當革之條輯為一書,頒之中外,庶可戢吏胥之奸。」詔行之。尋以目疾丐祠。



 六年,出為徽猷閣待制、知處州。復引疾奉祠,提舉江州太平興國宮。八年,以待制除太子詹事,尋復拜給事中,進禮部尚書並
 兼領詹事,又改侍讀。上顧沂厚,有大用意,而沂資性恬退,無所依附,數請去。



 虞允文當國,希旨建策復中原,沂極論金無畔,而我諸將未見可任此事者,數梗其議。遂以龍圖閣學士仍提舉興國宮。



 淳熙元年卒,年六十八。方疾革,整容素冠不少惰,蓋其為學所得者如此。謚獻肅。



 唐文若,字立夫,眉山人。父庚在《文苑傳》。文若少英邁不群,為文豪健。登進士第,分教潼川府。給事中勾濤薦自
 代,詔赴行在所,既至,而勾濤出,不得見。文若奏書闕下,略曰:「昔漢高慢士,四皓去之,而西鄙少廉恥之人;光武禮賢,嚴光友之,而東都多節義之士。陛下屈萬乘之尊,駐蹕東南,兩宮將歸,五路初復,正宜市朽骨,式怒蛙,以來豪傑,與之共治,寧遽惜此數刻之對耶?」書奏,翌日召對便殿,高宗大悅,特旨改合入官,通判洋州。洋西鄉縣產茶,亙陵谷八百餘里,山窮險,賦不盡括。使者韓球將增賦以市寵,園戶避苛斂轉徙,饑饉相藉,文若力爭之,
 賦迄不增。



 再通判遂寧府。會大水,民多漂死,文若至城上,發庫錢募游者,振活甚眾。又力請於朝,除田租二萬一千頃,免場務稅二十餘所,築長堤以捍水勢,自是無水患。



 秦檜死,上訪蜀士於魏良臣,以文若對。二十六年,以光祿丞召,改秘書郎,為《文思箴》以獻,其略曰:「於赫我皇,兵既休矣。兵休如何?莫若治兵。居安思危,邦乃攸寧。爰整其旅,文王以興。載舞干羽,舜仁用成。向戍弭兵,《春秋》所懲。蕭俯去兵,禍亂乃萌。師則多矣,軍則強矣。縱弛
 不繩,猶曰無人。兵非以殘,以兵休兵。」凡千五百餘言。自檜主和,朝論諱言兵,故文若以此風焉。



 遷起居郎。勸上收用西北人材以固根本,上深納之。將命以掌制,時有為宣和執政請恩,為司諫凌哲所彈,文若喜其直,作《禾黍詩》以美之。侍御史周方崇以為譏己,劾文若狂誕,出知邵州。上屢為近臣言唐文若無罪,可改近郡。



 知饒州,興學宮,減田租奇耗二萬石,又請歲糴常平義倉之儲什三與民平市,農末俱利,而粟不腐,遂以著令。餘干嘗
 有劇盜,巡尉不能制,文若遣牙兵捕而戮之。加直敷文閣,移知溫州。三十一年,召為宗正少卿。



 金人犯邊,文若求對,首建大臣節制江上之議。上諭大臣以文若與虞允文、杜莘老、馬驥才皆可用,復除起居郎。時諸將北出,捷書日聞,上下有狃志,獨文若憂之,圖上元嘉北伐故事。上諭文若以創業所歷艱苦及敵情反復甚悉,文若對曰:「願陛下深察大勢,趨策之長而避其短,無循前代軌轍,則大善。」



 未幾,諸軍退守,金主自將,圍大將王權於
 歷陽,權遁,淮南盡沒。詔百官廷議,文若畫三策,一請上親征,二乞遣大臣勞軍,三乞起張浚。工部侍郎許尹是其言,眾遂列奏上之,不報。



 文若尋面對,上問曰:「今計安出,卿熟張浚否?」文若曰:「浚守道篤學,天下屬望,今四十年,天不死浚嶺海,正為今日。」上矍然曰:「援浚者多,非卿無以發此。」數日,遣楊存中護江上軍,緩親征之期,起浚知平江府,蓋上以浚雖忠愨,喜功,將士多不附。文若復言浚本以孤忠得眾,尋改浚鎮建康府,將以為江、淮宣
 撫使,中沮之而止。



 乘輿幸江表,以起居郎兼給事中,直學士院,同群司居守。駕還,遷中書舍人。上將內禪,前數日手詔追崇皇太子所生父,文若既書黃,因過周必大誦聖德,而疑名稱未安,歸白宰相,請更黃,堂吏不可,文若執不已,宰相以聞。詔改稱本生親,尋又改宗室子偁,其後詔稱皇兄。



 孝宗嗣位,張浚以右府都督江、淮軍事,文若時以疾請外,除敷文閣待制,知漢州,尋改都督府參贊軍事。浚使行邊按守備,多所罷行者。未還,除知鼎
 州,改江州。



 明年,浚入相,都督府罷。其冬,金復大入,官軍悉戍淮。文若謂上流當嚴兵備,以定民志,奏籍鄉丁五萬,訓練有法,人倚以固。解嚴,和糴大起,郡之數八萬,文若以民勞,堅請得減什三。旋請祠,章三上未報。



 乾道元年卒,年六十。贈左通奉大夫。



 李燾,字仁甫,眉州丹棱人,唐宗室曹王之後也。父中登第,知仙井監。燾甫冠,憤金仇未報,著《反正議》十四篇,皆救時大務。紹興八年,擢進士第。調華陽簿,再調雅州推
 官。改秩,知雙流縣。仕族張氏子居喪而爭產,燾曰:「若忍墜先訓乎?盍歸思之。」三日復來,迄悔艾無訟。又有不白其母而鬻產者,燾置之理,豪強斂跡。於是以餘暇力學。



 燾恥讀王氏書,獨博極載籍,搜羅百氏,慨然以史自任,本朝典故尤悉力研核。仿司馬光《資治通鑒》例,斷自建隆,迄於靖康,為編年一書,名曰《長編》,浩大未畢,仍效光體為《百官公卿表》。史官以聞,詔給札來上。制置王剛中闢乾辦公事。



 知榮州。榮因溪為隍,夏秋率苦水潦,燾築
 防捍之。除潼川府路轉運判官,入境,劾守令不職者四人。縣多聚斂,燾括一路財賦額,通有無,酌三年中數,定為科約,上之朝,頒之州縣。



 乾道三年,召對,首舉藝祖治身、治家、治官、治吏典故,以為恢復之法,乞增置諫官,許六察言事,請練兵,毋增兵,杜諸將私獻,核軍中虛籍。



 除兵部員外郎兼禮部郎中。會慶節上壽,在郊禮散齊內,議權作樂,燾言:「漢、唐祀天地,散齋四日,致齋三日,建隆初郊亦然。自崇寧、大觀法《周禮》祭天地,故前十日受誓
 戒。今既合祭,宜復漢、唐及建隆舊制,庶幾兩得。」詔垂拱上壽止樂,正殿為北使權用。正除禮部郎中,言中興祭禮未備,請以《開寶通禮》、《嘉祐因革禮》、《政和新儀》令太常寺參校同異,修成祭法。



 四年,上《續通鑒長編》,自建隆至治平,凡一百八卷。時《乾道新歷》成,燾言:「歷不差不改,不驗不用。未差無以知其失,未驗無以知其是。舊歷多差,不容不改,而新歷亦未有大驗,乞申飭歷官討論。」五年,遷秘書少監兼權起居舍人,尋兼實錄院檢討官。



 子垕
 試賢良方正直言極諫科。燾素謂唐三百年不愧此科者惟劉去華,心慕之,嘗以所著《通論》五十篇見蜀帥張燾,欲應詔,不偶而止。其友晁公溯以書勉之,燾答以當修此學,必不從此舉。既不克躬試,於是命二子垕、塾習焉。至是,吏部尚書汪應辰薦垕文行可應詔,故有是命。



 左相陳俊卿出知福州,右相虞允文任恢復事,更張舊典。宰相以燾數言事,不樂,燾遂請去。除直顯謨閣、湖北轉運副使,陛辭,以欲速變古為戒。



 又奏:「《禹貢》九州,荊田
 第八,賦乃在三,人功既修,遂超五等。今田多荒蕪,賦虧十八。」上命之條畫。既至,奏:「京湖之民結茅而廬,築土而坊,傭牛而犁,糴種而殖,穀苗未立,睥睨已多,有橫加科斂者。今宜寬侵冒之禁,依乾德詔書止輸舊稅,廣收募之術,如咸平、元豐故事,勸課有勞者推恩。」詔從之。總餉呂游問入奏燾攝其事。



 歲饑,發鄂州大軍倉振之,僚屬爭執不可,燾曰:「吾自任,不以累諸君。」尋如數償之。游問返,果劾燾專,上止令具析,不之罪也。



 八年,直寶文閣,帥
 潼川兼知瀘州,首葺石門堡以扼夷人,奏乞戒茶馬司市敘州羈縻馬毋溢額,戒官民毋於夷、漢禁山伐木造舟,奏移鎖水於開邊舊池,皆報可。



 淳熙改元,被召,適城中火,上章自劾。提刑何熙志奏焚數不實,且言《長編》記魏王食肥彘,語涉誣謗,上曰:「憲臣按奏火數失實,職也,何預國史?」命成都提刑李蘩究火事,詔熙志貶二秩罷,燾止貶一秩。



 燾及都門,乞祠,除江西運副,且許臨遣。或勸以方被讒,無及時事,燾曰:「聖主全度如此,竭忠所以
 為報。」遂奏:「日食、地震皆陰盛,主敵國小人,不可不慮。」且申「無變古、無欲速」兩言,又上《快箴》,引太祖罷朝悔乘快決事以諫,上曰:「朕當揭之座右。」進秘閣修撰、權同修國史、權實錄院同修撰。



 燾為左史時,嘗乞復行明堂禮,謂「南郊、明堂初無隆殺,合視圜壇,特免出郊浮費。」至是申言之,詔集議,嬖幸沮止。其後周必大為禮部尚書,申其說,始克行。權禮部侍郎。



 七月壬戌,雷震太祖廟柱,壞鴟尾,有司旋加修繕。燾奏非所以畏天變,當應以實。上諭
 大臣:「燾愛朕,屢進讜言。」賜金紫。嘗請正太祖東向之位。



 四年,駕幸太學,以執經特轉一官。燾論兩學釋奠:從祀孔子,當升範仲淹、歐陽修、司馬光、蘇軾,黜王安石父子;從祀武成王,當黜李績。眾議不葉,止黜王雱而已。真拜侍郎,仍兼工部。



 《徽宗實錄》置院已久,趣上奏篇,燾薦呂祖謙學識之明,召為秘書郎兼檢討官。夜直宣引,奏:「近者蒙氣蔽日,厥占不肖者祿,股肱耳目宜謹厥與。」賜坐。欲起,又留賜飲、賜茶。尋詔監視太史測驗天文。



 九月丁
 酉,日當夜食,燾為社壇祭告官,伐鼓禮廢,特舉行。垕既中制科,為秘書省正字,尋遷著作郎兼國史實錄院編修檢討官。父子同主史事,搢紳榮之。



 燾感上知遇,論事益切,每集議,眾莫敢發言,獨條陳可否無所避。近臣復舉其次子塾應制科,以閣試不中程黜。垕偶考上舍試卷,發策問制科,為御史所劾,語連及燾,垕罷,燾亦知常德府。



 初,政和末,澧、辰、沅、靖四州置營田刀弩手,募人開邊,範世雄等附會擾民,建炎罷之。乾道間,有建請復置
 者,燾為轉運使,嘗奏不當復,已而提刑尹機迫郡縣行之,田不能給。燾至是又申言之,請度田立額,且約帥臣張栻列奏,詔從之。境多茶園,異時禁切商賈,率至交兵,燾曰:「官捕茶賊,豈禁茶商?」聽其自如,訖無警。



 累表乞閑,提舉興國宮。秋,明堂大禮成,以其首議,復除敷文閣待制。頃之,垕垕塾繼亡,上欲以吏事紓燾憂,起知遂寧府。



 七年,《長編》全書成,上之,詔藏秘閣。燾自謂此書寧失之繁,無失之略,故一祖八宗之事凡九百七十八卷,卷第總
 目五卷。依熙寧修《三經》例,損益修換四千四百餘事,上謂其書無愧司馬遷。燾嘗舉漢石渠、白虎故事,請上稱制臨決,又請冠序,上許之,竟不克就。



 又奏:「陛下即位二十餘年,志在富強,而兵弱財匱,與『教民七年可以即戎者』異矣。」一日,召對延和殿,講臣方讀《陸贄奏議》,燾因言:「贄雖相德宗,其實不遇。今遇陛下,可謂千載一時。」遂舉贄所言切於今可舉而行者數十事,勸上力行之。上有功業不足之嘆,燾曰:「功業見乎變通,人事既修,天應乃
 至。」進敷文閣直學士,提舉祐神觀兼侍講、同修國史。薦尤袤、劉清之十人為史官。



 十年七月,久旱,進祖宗避殿減膳求言故事,上亟施行。丁丑雨。一日宣對,燾言:「外議陛下多服藥,罕御殿,宮嬪無時進見,浮費頗多。」上曰:「卿可謂忠愛,顧朕老矣,安得此聲。近惟葬李婕妤用三萬緡,他無費也。」遂因轉對,乞用祖宗故事召宰執赴經筵。



 太史言十一月朔,日當食心八分。燾復條上古今日食是月者三十四事,因奏之曰:「心,天王位,其分為宋。十一月
 於卦為復,方潛陽時,陰氣乘之,故比他食為重,非小人害政,即敵人窺中國。」明日對延和殿,又及晉何曾譏武帝無經國遠圖。



 十一年春,乞致仕,優詔不允。上數問其疾增損,給事中宇文價傳上旨,燾曰:「臣子戀闕,非老病,忍乞骸骨。」因叩價時事,勉以忠藎。又聞四川乞減酒課額,猶手札贊廟堂行之。



 病革,除敷文閣學士,致仕。命下,喜曰:「事了矣。」口占遺表云:「臣年七十,死不為夭,所恨報國缺然。願陛下經遠以藝祖為師,用人以昭陵為則。」辭
 氣舒徐,乃卒,年七十。



 上聞嗟悼,贈光祿大夫。他日謂宇文價曰:「朕嘗許燾大書『續資治通鑒長編』七字,且用神宗賜司馬光故事,為序冠篇,不謂其止此。」



 燾性剛大,特立獨行。早著書,檜尚當路,檜死始聞於朝。暨在從列,每正色以訂國論。張栻嘗曰:「李仁甫如霜松雪柏。無嗜好,無姬侍,不殖產。平生生死文字間。」《長編》一書用力四十年,葉適以為《春秋》以後才有此書。



 有《易學》五卷,《春秋學》十卷,《五學傳授》、《尚書百篇圖》、《大傳雜說》、《七十二
 子名籍》各一卷,《文集》五十卷,《奏議》三十卷,《四朝史稿》五十卷,《通論》十一卷,《南北攻守錄》三十卷,《七十二候圖》、《陶潛新傳》並《詩譜》各三卷,《歷代宰相年表》、《唐宰相譜》、《江左方鎮年表》、《晉司馬氏本支》、《齊梁本支》、《王謝世表》、《五代將帥年表》合為四十一卷。



 謚文簡,累贈太師、溫國公。子垕、TJ、塾、壁、□。垕著作郎,TJ夔州路提點刑獄,壁、□皆執政,別有傳。



 論曰:執羔宿德雅度,在經筵,忠忱啟沃,以口舌相高為
 戒。希呂剛直懇切,有古引裾風。良祐力止泛使,懼開釁端,忤旨竄斥而甘心焉。李浩獨不造秦熹,陳橐以呈身為恥,文若譏休兵,胡沂斥閹宦,其清風苦節,終始弗渝。高、孝之世,李燾恥讀王氏書,掇拾禮文殘缺之餘,粲然有則,《長編》之作,咸稱史才,然所掇拾,或出野史,《春秋》傳疑傳信之法然歟!



\end{pinyinscope}