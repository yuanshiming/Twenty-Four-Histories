\article{本紀第三十三}

\begin{pinyinscope}

 孝
 宗一



 孝宗紹統同道冠德昭功哲文神武明聖成孝皇帝,諱慎,字符永,太祖七世孫也。初,太祖少子秦王德芳生英國公惟憲,惟憲生新興侯從鬱,從鬱生華陰侯世將,世
 將生慶國公令譮,令譮生子偁,是為秀王。王夫人張氏夢人擁一羊遺之曰:「以此為識。」已而有娠,以建炎元年十月戊寅生帝於秀州青杉閘之官舍,紅光滿室,如日正中。少長,命名伯琮。



 及元懿太子薨,高宗未有後,而昭慈聖獻皇后亦自江西還行在,後嘗感異夢,密為高宗言之,高宗大寤。會右僕射範宗尹亦造膝以請,高宗曰:「太祖以神武定天下,子孫不得享之,遭時多艱,零落可憫。朕若不法仁宗,為天下計,何以慰在天之靈!」於是詔
 選太祖之後。同知樞密院事李回曰:「藝祖不以大位私其子,發於至誠。陛下為天下遠慮,合於藝祖,可以昭格天命。」參知政事張守曰:「藝祖諸子,不聞失德,而傳位太宗,過堯、舜遠甚。」高宗曰:「此事不難行,朕於『伯』字行中選擇,庶幾昭穆順序。」而上虞丞婁寅亮亦上書言:「昌陵之後,寂寥無聞,僅同民庶。藝祖在上,莫肯顧歆,此金人所以未悔禍也。望陛下於『伯』字行內選太祖諸孫有賢德者。」高宗讀之,大感嘆。紹興二年五月,選帝育於禁中。三
 年二月,除和州防禦使,賜名瑗。壬寅,改貴州。五年五月,用左僕射趙鼎議,立書院宮中教之,既成,遂以為資善堂。帝讀書強記,天資特異。己亥,制授保慶軍節度使,封建國公。六月己酉,聽讀資善堂,以徽猷閣待制範沖兼詡善,起居郎朱震兼贊讀,高宗命帝見沖、震皆拜。十二年正月丁酉,加檢校少保,封普安郡王。



 三月壬寅,出閣就外第。十三年九月,秀王歿於秀州。十四年正月庚辰,用廷臣議,聽解官行服。十六年四月乙巳,免喪,還舊官。
 十七年六月戊午,改常德軍節度使。二十四年,衢州盜起,秦檜遣殿前司將官辛立將千人捕之,不以聞。帝入侍言之,高宗大驚。明日,以問檜,檜謂不足煩聖慮,故不敢聞,俟朝夕盜平則奏矣。檜退,知為帝言,忌之。及檜疾篤,其家秘不以聞,謀以子熹代相,帝又密啟高宗破其奸。三十年二月癸酉,立為皇子,更名瑋。甲戌,詔下。丙子,制授寧國軍節度使、開府儀同三司,進封建王。制出,中外大悅。四月,賜字符瑰。三十一年十月壬子,以明堂恩,
 改鎮南軍節度使。先是,金人犯邊,高宗下詔親征,而兩淮失守,朝臣多陳退避之計,帝不勝其憤,請率師為前驅。直講史浩以疾在告,聞之亟入,為帝言,太子不宜將兵,乃為草奏,因中宮以進,請衛從以共子職。高宗因亦欲帝遍識諸將,十二月,遂扈蹕如金陵。三十二年五月甲子,立為皇太子,改名慎。初,高宗久有禪位之意,嘗以諭帝,帝流涕固辭,會有邊事不果。及歸自金陵,陳康伯求去,高宗復以倦勤諭之。中書舍人唐文若聞而請對,
 言不宜急遽,故先下建儲之詔,賜名燁。監察御史周必大密與康伯言,與唐昭宗名同音,不可。詔別擬進,乃定今名。既又命學士承旨洪遵為太子擇字,遵擬四字以進,皆不稱旨。



 六月甲戌,御筆賜字符永。



 乙亥,內降御札:「皇太子可即皇帝位。朕稱太上皇帝,退處德壽宮;皇后稱太上皇后。」丙子,遣中使召帝入禁中,面諭之,帝又推遜不受,即趨側殿門,欲還東宮,高宗勉諭再三,乃止。於是高宗出御紫宸殿,輔臣奏事畢,高宗還宮。百官移班殿門
 外,拜詔畢,復入班殿庭。頃之,內侍掖帝至御榻前,側立不坐,內侍扶掖至七八,乃略就坐。宰相率百僚稱賀,帝遽興。輔臣升殿固請,帝愀然曰:「君父之命,出於獨斷。然此大位,懼不克當。」班退,太上皇帝即駕之德壽宮,帝服袍履,步出祥曦殿門,冒雨掖輦以行,及宮門弗止。上皇麾謝再三,且令左右扶掖以還,顧曰:「吾付托得人,吾無憾矣。」左右皆呼萬歲。是日,詔有司議太上皇帝、太上皇后尊號以聞,在內諸司日輪官吏應奉德壽宮,增置,朝德
 壽宮提點、乾辦等官,德壽宮宿衛依皇城及宮門法。戊寅,大赦。詔宰相率百官月兩朝德壽宮。己卯,以即位告於天地、宗廟、社稷。庚辰,詔五日一朝德壽宮。以左武大夫龍大淵為樞密副都承旨,武翼郎曾覿帶御器械。癸未,始御後殿。甲申,詔中外士庶陳時政闕失。丙戌,詔進宰執官二等。丁亥,詔以太上皇不許五日一朝,自今月四朝。復除名勒停人胡銓官、知饒州。己丑,詔有司月奉德壽宮緡錢十萬。辛卯,詔罷四川市馬。
 壬辰,詔百官日一人入對。癸巳,蝗。甲午,上太上皇帝尊號曰光堯壽聖太上皇帝,太上皇后曰壽聖太上皇后。乙未晦,金人屠原州。



 秋七月戊戌,興州中軍統制吳挺復鞏州。庚子,判建康府張浚入見。以雨水、飛蝗,令侍從、臺諫條上民間利害。壬寅,詔戒飭諸郡守臣。癸卯,以張浚為少傅、江淮宣撫使,封魏國公。甲辰,以參知政事汪澈視師湖北、京西。遣劉珙等使金告即位。戊申,以四川宣撫使吳璘兼陜西河東路宣撫、招討使。追復岳飛元
 官,以禮改葬。是夜,地震,大風拔木。己酉,有事於太廟、別廟。癸丑,馬軍司中軍統制趙撙、忠義軍統領皇甫倜復光州。甲寅,朝獻景靈宮。詔淮南諸州存恤淮北來歸之民,權免稅役。丙辰,以少保、保康軍節度使吳益為少傅,太尉、寧武軍節度使吳蓋為開府儀同三司。丁巳,罷李寶措置海道。戊午,恩平郡王璩入見。庚申,以御前軍器所仍隸工部。辛酉,詔後省看詳中外上書,有可採者以聞。壬戌,以黃祖舜兼權參知政事。罷諸路聖節進奉。詔
 李顯忠軍馬聽張浚節制。癸亥,增將士戰傷死者推恩格。詔蠲四川積年逋負。



 八月乙丑朔,四川馬軍統制高師中與金人戰於摧沙,敗死。丙寅,吳璘與金人戰於德順軍。己巳,以翰林學士史浩為參知政事。戊寅,率群臣詣德壽宮,奉上太上皇帝、太上皇后尊號冊寶。於亥,班寬恤事十八條。起居舍人洪邁、知閣門事張掄坐奉使辱命罷。甲申,吳璘敗金人於北山。戊子,追復李光資政殿學土,趙鼎、範沖並還合得恩數。庚寅,以生日為會慶
 節。追冊故妃郭氏為皇后。



 九月甲午,以子□為少保、永興軍節度使,進封鄧王;愷為雄武軍節度使、開府儀同三司,進封慶王;惇為鎮洮軍節度使、開府儀同三司,進封恭王。甲午,金人攻德順軍東山堡,中軍將李庠戰死。丁酉,詔開講日召輔臣觀講。川、陜宣諭使虞允文以論邊事不合罷。己亥,詔侍從、臺諫舉知四川利害可為都轉運使者。庚子,以金人來索舊禮,詔宰執、侍從、臺諫各陳應敵定論以聞。辛丑,詔吳璘審度措置,保全川蜀。乙
 巳,詔纂錄勛臣名次。丙午,轉補朱震、範沖子孫官。庚戌,謚皇后郭氏曰恭懷。辛亥,振淮東義兵及歸正人。以總領四川財賦軍馬錢糧王之望為戶部侍郎、川陜宣諭使,仍命將調兵同防守興州川口。乙卯,詔虞允文赴吳璘軍議事。辛酉,以吳璘為少師。



 冬十月丙寅,詔朝臣舉堪監司、郡守者,戊辰,以岳陽軍節度使居廣開府儀同三司,史浩兼權知樞密院事。己巳,葉義問罷。詔登聞鼓院毋沮抑進狀。庚午,以恩平郡王璩為少保。詔會慶節
 權免上壽。戊寅,詔張浚、陳俊卿覆實諸將所陳功賞。改謚皇后郭氏曰安穆。壬午,官岳飛孫六人。甲申,契丹招討蕭鷓巴來奔。金人攻德順城,吳璘擊走之,復遣兵追襲,遂為所敗。乙酉,升建州為建寧府。戊子,以資政殿學士張燾同知樞密院事。己丑,安南都護、南平王同李天祚、闍婆國王悉里地茶蘭固野、占城國王鄒時巴蘭並加食邑實封。



 十一月庚子,以蕭鷓巴為忠州團練使。乙巳,金人攻水洛城。丙午,賜忠義軍統制皇甫倜軍帛五千
 匹、綿萬兩。戊申,詔改明年為隆興元年。辛亥,免楊存中所獻酒坊逋負錢四十萬緡。甲寅,定內侍官額。辛酉,史浩免權知樞密院事。裁定文武臣宮觀、嶽廟員數。立措置京西營田司。十二月乙丑,詔宰臣復兼樞密使。金人攻隴城縣,官軍拒卻之,丙寅,詔帥臣、監司具部內知州治行臧否以聞。詔棄德順城,徙兵民於秦州以里屯住。丁卯,以陳康伯兼樞密使。令江、淮宣撫司增招武勇效用軍。戊辰,詔侍從、臺諫集議當今弊事,仍命盡率其屬,
 使極言無隱。辛未,劉珙、張說還自盱眙。戊寅,蠲四川登極赦前帶白契稅錢。丙戌,詔觀察使已上各舉所知三人,三省、樞密院詳議立格以聞。庚寅,罷建康、鎮江營田官兵。辛卯,廣西賊王宣破藤州,守臣廖顒棄城遁。是歲,諸路斷大闢四十一人。



 隆興元年春正月壬辰朔,群臣朝於文德殿。帝朝德壽宮。立武臣薦舉格。甲午,四川宣撫司奉詔班師。庚子,以史浩為尚書右僕射、同中書門下平章事兼樞密使,張
 浚進樞密使、都督江淮東西路軍馬。丙午,誅殿前司後軍謀變者。戊申,詔禮部貢院試額增一百人。丁巳,詔吳璘軍進退可從便宜。璘已棄德順,道為金人所邀,將士死者數萬計。



 二月壬戌朔,用史浩策,以布衣李信為兵部員外郎,繼蠟書間道往中原,招豪傑之據有州郡者,許以封王世襲。安慶軍節度使士籛乞減奉賜之半,以助軍用。自是,諸宗室有請,悉從之。戊辰,宰執陳康伯等乞再減奉,止存舊格之半,許之。己卯,振兩淮流民及山
 東歸正忠義軍。癸未,黃祖舜罷。庚寅,逐秦檜黨人,仍禁輒至行在。



 三月壬辰朔,金左副元帥紇石烈志寧以書取侵地。癸巳,以張燾為參知政事,御史中丞辛次膺同知樞密院事,葉義問落端明殿學士、饒州居住。丙申,雨雹。丁酉,詔戶部置局,議節浮費。己亥,楊存中等乞減半奉如宰執例,許之。庚子,以龍大淵知閣門事,曾覿同知閣門事。壬寅,陳康伯上欽宗陵名曰永獻。乙巳,詔求遺逸。丁未,詔修《太上皇帝聖政》。罷龍大淵,別與差遣。曾覿
 復帶御器械。召張浚。己酉,張燾罷。立選人減舉主法。甲寅,復龍大淵知合門事。曾覿同閣門事,給事中、中書舍人留黃不行。乙卯,詔飭郡縣吏。庚申,以久雨,命有司振災傷,察刑禁。



 夏四月乙丑,定選人改官歲額。戊辰,張浚入見,議出師渡淮,三省、樞密院不預聞。壬申,賜禮部進士木待問以下五百三十八人及第、出身。乙亥,王之望罷。壬午,詔戶部、臺諫議節浮費。癸未,詔以白金二十五萬兩給江、淮都督府軍費。戊子,張浚命邵宏淵帥
 師次盱眙。己丑,又命李顯忠帥師次定遠。是月,金人拔環州,守臣強霓及其弟震死之。



 五月壬辰,申嚴鋪翠銷金及神祠僭擬之禁。丁酉,李顯忠復靈壁縣。邵宏淵次虹縣,金人拒之。戊戌,顯忠東趨虹縣。庚子,復虹縣,金知泗州蒲察徒穆及同知泗州大周仁降。辛丑,命左右史日更立前殿。壬寅,張浚渡江視師。癸卯,金右翼軍都統蕭琦降於李顯忠。甲辰,顯忠及宏淵敗金人於宿州。乙己,史浩罷。追復司馬康右諫議大夫。丙午,復宿州,戮金
 兵數千人。建康前軍統領官王珙巷戰,死之。丁未,以辛次膺為參知政事,翰林學士承旨洪遵同知樞密院事。督諸路開營田。辛亥,詣德壽宮賀天申節。金紇石烈志寧自睢陽引兵至宿州,李顯忠擊卻之。壬子,欽宗大祥,帝服衰服詣幾筵,易祥服行祥祭禮。顯忠與金人戰於宿州,邵宏淵不援,顯忠失利。是夜,建康中軍統制周宏及邵宏淵之子世雄、殿前司統制官左士淵逃歸。癸丑,進李顯忠開府儀同三司、淮南京畿京東河北招討使,
 邵宏淵檢校少保、寧遠軍節度使、招討副使。金人攻宿州城,顯忠大敗之。殿前司統制官張訓通等七人、統領官十二人,以二將不葉而遁。甲寅,李顯忠、邵宏淵軍大潰於符離。乙卯,下詔親征。丙辰,召汪澈。以張浚兼都督荊、襄軍馬。李顯忠、邵宏淵至濠州。張浚以劉寶為鎮江諸軍都統制。丁巳,以蒲察徒穆、大周仁、蕭琦並為節度使,徒穆大同軍、周仁彰國軍、琦威塞軍。遣御前忠勇軍赴都督府。是月,成都地震三。



 六月庚申朔,日有食之。遣
 內侍趣上淮東將士功賞。癸亥,汪澈罷。張浚乞致仕,且請通好,皆不許。丁卯,以觀文殿大學士湯思退為醴泉觀使兼侍讀。戊辰,召虞允文。以兵部侍郎周葵為參知政事。汪澈落資政殿學士、臺州居住。庚午,張浚自盱眙還揚州。辛未,李顯忠罷軍職。壬申,以太傅、同安郡王楊存中為御營使、節制殿前司軍馬。癸酉,下詔罪己。張浚降授特進,仍前樞密使、江淮東西路宣撫使,官屬各奪二官。邵宏淵降武義大夫,職仍舊。詔楊存中先詣建康
 措置營砦,檢視沿江守備。戊寅,詔展巡幸之期。辛次膺罷。己卯,李顯忠責授清遠軍節度副使、筠州安置。辛巳,命浙西副都總管李寶兼御營統制官、措置浙西海道。甲申,右諫議大夫王大寶入封,論移蹕。以敷文閣學士虞允文為兵部尚書兼湖北、京西宣諭使。戊子,放宮人三十人。以蕭琦為檢校少保、河北招撫使。



 秋七月庚寅朔,以虞允文為湖北、京西制置使。癸巳,以湯思退為尚書右僕射、同中書門下平章事兼樞密使。李顯忠再責
 授果州團練副使、潭州安置。乙未,詔宿州棄軍將佐奪官、貶竄有差。丙申,太白晝見,經天。罷江、淮宣撫司便宜行事。乙巳,以旱蝗、星變,詔侍從、臺諫、兩省官條上時政闕失。丁未,詔徵李顯忠侵欺官錢金銀,免籍其家。乙卯,裁減省、部、寺、監官吏。戊午,給還岳飛田宅。



 八月丙寅,張浚復都督江、淮軍馬。庚午,以劉寶兼淮東招撫使。丙子,以飛蝗、風水為災,避殿減膳。罷借諸路職田之令。戊寅,金紇石烈志寧又以書求海、泗、唐、鄧四州地及歲幣。癸
 未,復以龍大淵知閣門事,曾覿同知閣門事。丙戌,遣淮西安撫司乾辦公事盧仲賢等繼書至金帥府,戒勿許四州,差減歲幣。仍命諸將毋遣兵人出境。



 九月己酉,楊存中罷。



 冬十月戊午朔,大臣奏金帥書言四事,帝曰:「四州地、歲幣可與,名分、歸正人不可從。」辛酉,御殿復膳。己巳,遣護聖軍戍江南。丙子,詔慶上皇后教旨改稱聖旨。立賢妃夏氏為皇后。丁丑,地震。辛巳,升洪州為隆興府。詔:「江、淮軍馬調發應援,從都督府取旨,餘事悉以聞。」十
 一月己丑,盧仲賢自宿州以金都元帥僕散忠義遺三省、樞密院書來。庚子,遣王子望等為金國通問使。辛丑,詔侍從、臺諫於後省集議講和、遣使、禮數、土貢四事,仍各薦可備小使者。丙午,盧仲賢擅許四州,下大理寺,奪三官。召張浚。癸丑,以胡昉、楊由義為使金通問國信所審議官。



 十二月己未,陳康伯罷。乙丑,張浚入見。丁丑,以湯思退為尚書左僕射,張浚為右僕射,並同中書門下平章事兼樞密使。浚仍都督江、淮東西路軍馬。壬午,西
 南方有白氣。是歲,以兩浙大水、旱蝗,江東大水,悉蠲其租。



 二年春正月辛卯,詔增德壽宮車輦儀衛。壬辰,御文德殿,冊皇后。癸巳,修三省法。乙未,及皇后朝德壽宮。丙申,命虞允文調兵討廣西諸盜。庚子,罷諸州招軍。丙午,金僕散忠義復以書來。庚戌,申嚴卿監、郎官更出迭入之制。壬子,振歸正人。甲寅,白氣亙天。是月,福建諸州地震。



 二月辛未,蠲秀州貧民逋租。壬申,容州妖賊李雲作亂。
 癸酉,復王權武義大夫,命權廣西路都鈐轄,專一措置盜賊。丙子,詔飭將帥減文武官及百司吏郊賜之半。罷兩浙、福建、江西、湖南、夔州路參議官。丁丑,雨雹及雪。獲李云,其黨悉平。乙酉,胡昉自宿州還。初,金帥以昉等不許四郡,械系之,昉等不屈,金主命歸之。



 三月丙戌朔,詔張浚視師於淮。又詔王之望等以幣還。丁亥,詔荊襄、川陜帥臣嚴邊備,毋先事妄舉。盧仲賢除名,械送郴州編管。壬寅。詔知光州皇甫倜毋招納歸正人。丙午,王宣等
 降。詔三衙戍兵歸司,建康、鎮江大軍更番歸砦。庚戌,芝生德壽宮。以戶部侍郎錢端禮為淮東宣諭使,吏部侍郎王之望為淮西宣諭使。詔撫諭兩淮軍民。壬子,以廣西賊平,詔減高、藤、雷、容四州雜犯死罪囚,釋杖以下,蠲夏秋稅賦。以忠勇軍隸步軍司,神勁右軍隸鎮江都統司。癸丑,以王彥為建康諸軍都統制兼淮西招撫使。



 夏四月庚申,召張浚還朝。甲子,以李顯忠侵欺官錢給還諸軍。丁卯,以建康歸正人為忠毅軍,鎮江為忠順軍,命
 蕭琦、蕭鷓巴分領之。戊辰,罷江、淮都督府。高麗入貢。丁丑,張浚罷。癸未,言者論宰相、執政徇欺之弊,命書寘政事堂。



 五月壬辰,復置環衛官。丙申,詔吳璘毋招納歸正人。辛丑,詔劉寶量度泗州輕重取舍事宜以聞。江西總管邵宏淵責授靖州團練副使、南安軍安置,仍徵其盜用庫錢。乙巳,率群臣詣德壽宮賀天申節,始用樂。丁未,蝗。詔內外贓私不法官吏,尚書省置籍檢勘。庚戌,罷招神勁效用軍。辛亥,鬻兩淮所招戶馬。



 六月甲寅朔,日有
 食之。辛酉,以淫雨,詔州縣理滯囚。戊辰,太白晝見。壬申,命虞允文棄唐、鄧,允文不奉詔。丁丑,振江東、兩淮被水貧民。



 秋七月乙酉,召虞允文。以戶部尚書韓仲通為湖北、京西制置使。丁亥,洪遵罷。己丑,以周葵兼權知樞密院事。遣主管馬軍司公事張守忠以兵詣淮西,措置邊備。庚子,太白經天。詔內外文武官年七十不請致仕者,遇郊毋得蔭補。乙巳。命海、泗州徹戍。丁未,雨雹。戊申,蠲淮東內庫坊場錢一年。庚戌,洪遵落端明殿學士。癸丑,
 以江東、浙西大水,詔侍從、臺諫、卿監、郎官、館職陳闕失及當今急務。是月,罷內侍押班梁珂為在外宮觀。移廣西提刑司於容州。



 八月甲寅朔,以災異,避殿減膳。戊午,南丹州莫延廩為諸蠻所逐來歸,詔補修武郎。命江東、浙西守臣措置開決圍田。甲子,秦國大長公主薨。以久雨,決系囚。庚辰,以資政殿大學士賀允中為知樞密院事兼參知政事。辛巳,詔振淮東被水州縣。張浚薨。壬午,遣魏杞等為金國通問使。



 九月甲申。罷內侍李珂賜謚。
 甲午,詔江東、浙西監司、守臣講明措置田事。乙未,交址入貢。丁酉,嚴臟吏法。辛丑,以王之望為參知政事,權刑部侍郎吳芾為給事中兼淮西宣諭使。金人犯邊。以久雨,出內庫白金四十萬兩,糴米賑貧民。壬寅,王彥帥師濟江,軍昭關。癸卯。命湯思退都督江、淮東西路軍馬,辭不行。乙巳,復命楊存中為同都督,錢端禮、吳芾並為都督府參贊軍事。罷宣諭司。仍易國書以付魏杞。少保、崇信軍節度使趙密落致仕,權領殿前司職事。



 冬十月甲
 寅,魏杞至盱眙,金帥以國書未如式,弗受,欲得商、秦地及俘獲人,且邀歲幣三十萬,杞未得進。丁卯,賀允中罷為資政殿大學士致仕。己巳,以周葵兼權知樞密院事,王之望兼同知樞密院事。庚午,詔輔臣夕對便殿。丙子,大風。庚辰,蠲京西、湖北運糧所經州縣秋稅之半。以靖海軍節度使李寶為沿海駐扎御前水軍都統制。辛巳,金人分道渡淮,劉寶棄楚州遁。



 十一月乙酉,知楚州魏勝與金人戰,死之,州遂陷,濠州亦陷。王彥棄昭關遁,滁
 州又陷,丙戌,詔諭沿邊將士。丁亥,詔魏祀等以所繼禮幣犒軍。杞弗從,命留鎮江侯旨。復命王之望督視江、淮軍馬。戊子,以金人侵擾,詔郊杞改用明年。又詔諭歸正官民軍士。命王之望同都督江、淮軍馬。湯思退罷都督。召陳康伯。巳丑,王之望罷同都督。庚寅,命楊存中都督江、淮軍馬。辛卯,湯思退罷,尋以尹穡、晁公武論之,落觀文殿大學士、永州居住,未至而卒。甲午,以黃榜禁太學生伏闕。是日,太學生張觀等七十二人上書,請斬湯思
 退、王之望、尹穡,竄其黨洪適、晁公武而用陳康伯、胡銓等,以濟大計。丙申,遣國信所大通事王抃持周葵書如金帥府,請正皇帝號,為叔侄之國;易歲貢為歲幣,減十萬;割商、秦地;歸被俘人,惟叛亡者不與;誓目大略與紹興同。以金人犯淮南,詔避殿減膳。丁酉,詔擇日視師。戊戌,以少保、觀文殿大學士陳康伯為尚書左僕射、同中書門下平章事兼樞密使。庚子,遣兵部侍郎胡銓、右諫議大夫尹穡分詣兩浙措置海道。贈魏勝寧國軍節度
 使,謚忠壯。辛丑,兵部尚書錢端禮賜出身,簽書樞密院事兼提領德壽宮。壬寅,詔侍從、兩省官日一至都堂議事,有關臺諫者亦聽會議。以顯謨閣學士虞允文同簽書樞密院事。癸卯,遣王之望勞師江上。甲辰,金人犯六合縣,步軍司統制崔皋擊卻之。乙巳,以錢端禮兼權參知政事。丁未,以顯謨閣直學士沉介為沿江制置使。命沿江諸州調保甲分守渡口。己酉,劉寶落節鉞,為武泰軍承宣使,王彥落龍神衛四廂都指揮使。



 閏月甲寅,陳
 康伯入見,詔康伯間日一朝,肩輿至殿門,給扶升殿。丙辰,周葵罷。王抃見金二帥,皆得其報書以歸。戊午,蕭琦卒。壬戌。詔罷胡銓、尹穡。丙寅,召韓仲通。以沈介為兵部尚書、湖北京西制置使。戊辰,以金人且退,詔督府擇利擊之,王之望執不可。乙亥。之望罷。丙子,以王抃為奉使金國通問國信所參議官,持陳康伯報書以行。丁丑。金遣張恭愈來迓使者。詔臺諫、侍從、兩省官舉楚、廬、滁、濠四州守臣。十二月甲申,罷陜西路轉運司。戊子,魏杞始
 渡淮。詔郊祀大禮遵至道典故,改用來年正月一日上辛。辛卯,以錢端禮為參知政事兼知樞密院事,虞允文同知樞密院事兼權參知政事,禮部尚書王剛中簽書樞密院事。丙申,制曰:「比遣王抃,遠抵穎濱,得其要約。尋澶淵盟誓之信,仿大遼書題之儀,正皇帝之稱,為叔侄之國,歲幣減十萬之數,地界如紹興之時。憐彼此之無辜,約叛亡之不遣,可使歸正之士咸起寧居之心。重念數州之民,罹此一時之難,老稚有蕩析之災,丁壯有系
 累之苦,宜推蕩滌之宥,少慰凋殘之情。應沿邊被兵州軍,除逃遁官吏不赦外,雜犯死罪情輕者減一等,餘並放遣。」遣洪適等賀金主生辰。詔吳挺市馬赴行在。己亥,雨雹。壬寅,罷三衙、江上、荊襄諸軍招軍。甲辰,遣沿海水軍還屯。己酉,朝獻景靈宮。庚戌,朝饗太廟。



 乾道元年春正月辛亥朔,合祀天地於圜丘,大赦,改元。丁巳,淮西安撫韓璡勒停、賀州編管。庚申,以錢端禮兼德壽宮使。辛酉,召楊存中。通問使魏杞至燕山。丁卯,以
 王抃使金有勞,進五官。庚午,西北方有白氣。詔館職更迭補外。辛未,立兩淮守令勸民種桑賞。壬申,詔兩浙振流民。以紹興流民多死,罷守臣徐哲及兩縣令。癸酉,蠲沿邊殘破州軍官賦一年。甲戌,劉寶責果州團練副使、瓊州安置。乙亥,罷兩淮招撫司及陜西、河東宣撫、招討司。丙子,淮西守將孔福以遇敵棄城伏誅,頓遇奪官,刺面配吉陽軍牢城。



 二月庚辰朔,朝德壽宮,從太上皇、太上皇后幸四聖觀。乙酉,罷江、淮都督府。遣官檢察兩淮
 州縣,振濟饑民。庚寅,雨雹。癸巳,移濠州戍兵於藕塘。庚子,以楊存中為寧遠、昭慶軍節度使。甲辰,以久雨,避殿減膳,蠲兩淮災傷州縣身丁錢絹,決系囚。丁未,陳康伯薨,謚文恭。



 三月甲寅,太白晝見。己未,御殿復膳。庚申,以虞允文為參知政事兼同知樞密院事,王剛中同知樞密院事。命淮西、湖北、荊襄帥臣措置屯田,復置榷場。癸亥,黃祖舜薨。戊辰,白氣亙天。己巳,罷諸軍額外制領將佐。乙亥,太白經天。是春,湖南盜起,入廣東焚掠州縣,官
 軍討平之。夏四月庚子,金報問使完顏仲等入見。乙巳,吳璘入見。



 五月庚戌,以璘為太傅,封新安郡王。丙辰,詔有司治皇后家廟。壬戌,詔監司、帥守講究弊事以聞。合廣南東、西路監事為一司。癸亥,詔總領、帥、漕臣、諸軍都統制並兼提領措置屯田,沿邊守臣兼管屯田事。丁卯,詔吳璘措置馬綱、水路。壬申,蠲四川州縣虛額錢。吳璘改判興元府。乙亥,詔未銓試人毋得堂除。丙子,遣李若川等使金賀上尊號。增置諸路鈐轄、都監。郴州盜李金
 等復作亂,遣兵討捕之。



 六月癸未,王剛中薨。乙酉,詔恭王府直講王淮傾邪不正,有違禮經,可與外任。丙戌,以翰林學士洪適簽書樞密院事。戊子,步軍司統制官崔皋坐奏功冒濫,奪所遷觀察使,止進橫行三官,令本軍自效。辛卯,以武經郎令德為安定郡王。壬辰,以淮南轉運判官姚岳言境內飛蝗自死,奪一官罷之。丙申,以兩淮守令勞徠安集無效,下詔戒飭之,仍以詔置守令治所。壬寅,蠲廣東殘破郡縣稅賦。甲辰,罷湖北、京西制置
 司。



 秋七月辛亥,詔知州年七十以上者與宮觀。癸丑。輔臣晚對選德殿,御坐後有大屏,記注諸道監司、郡守姓名,因令都堂視此書之。甲寅,借職田租二年,以裨經費。己未,鑄當二錢。己巳,蠲關外四州民今年租稅及湖南賊蹂郡縣夏稅。



 八月己卯,以永豐圩田賜建康都統司。癸未,獲李金。乙酉,詔立子□為皇太子。丁亥,虞允文罷。戊子,大赦。己丑,以洪適為參知政事兼權知樞密院事,吏部侍郎葉顒簽書樞密院事兼權參知政事。庚寅,立
 知州軍、諸路總管鈐轄都監辭見法。癸巳,錢端禮以避東宮親嫌,罷為資政殿大學士、提舉萬壽觀。戊戌,吏部侍郎章服以論虞允文阿附罷,謫居汀州。



 九月乙卯,立廣國夫人錢氏為皇太子妃。丁巳,申嚴百司官出入局之制。丁卯,升鼎州為常德府。甲戌,以端明殿學士汪澈知樞密院事,洪適兼同知樞密院事。乙亥,置沿淮諸州都巡檢。



 冬十月己卯,遣方滋等使金賀正旦。戊子,增頭子錢。歸正人右通直郎劉蘊古坐以軍器法式送北境,
 伏誅。壬辰,御大慶殿,冊皇太子。癸巳,詣德壽宮稱謝。乙未,詔侍從各舉所知宗室一二人。丁酉,金遣高衎等來賀會慶節。乙巳,淮北紅巾賊逾淮劫掠,立賞討捕之,已而知楚州胡明遣巡尉擊殺其首蕭榮。



 十一月辛亥,招收兩淮流散忠義人。丙寅,白氣亙天。辛未,遣龍大淵撫諭兩淮,措置屯田,督捕盜賊。十二月戊寅,以洪適為尚書右僕射、同中書門下平章事兼樞密使,汪澈為樞密使。命廣東提刑司招安李金餘黨。癸未,遣王□嚴等賀金
 主生辰。康寅,以葉顒為參知政事兼同知樞密院事。辛卯,詔侍從、臺諫、兩省舉堪監司、郡守者各一人,三衙、知閣舉材武可守邊者一人。庚子,罷兩淮諸州權攝官。壬寅,金遣烏古論忠弼等來賀明年正旦。癸卯。詔樞密院文書依三省式,經中書門下畫黃書讀至正。



 二年春正月辛酉,省六合戍兵,以所墾田給還復業之民。辛未,命湖南監司存恤寇盜殘破郡縣。



 二月丁丑,罷盱眙屯田,振兩浙、江東饑。戊寅,幸玉津園宴射,遂幸龍
 井。



 三月乙巳,禁京西、利州路科役保勝義士。壬子,詔戒飭刑獄官。戊午,殿中侍御史王伯庠請裁定奏薦,詔三省、臺諫集議,具條式以聞。詔縣令非兩任,毋除監察御史;非任守臣,毋除郎官。著為令。丁卯,賜禮部進土蕭國梁以下四百九十有三人及第、出身。戊辰,再增諸州軍離軍添差員闕。辛未,罷洪適右僕射。癸酉,以給事中、權吏部尚書魏杞同知樞密院事兼權參知政事。丁丑,罷和糴。



 夏四月戊寅,以久雨,命侍從、臺諫議刑政所宜以
 聞。減大理、三衙、臨安府及浙西州縣雜犯死罪以下囚一等,釋杖以下。庚辰,詔兩浙漕臣王炎開平江、湖、秀圍田。辛巳,避殿減膳。甲申,太白晝見。癸巳,御殿復膳。乙未,汪澈罷。丁酉,以知荊南府李道憑恃戚里妄作,罷之。



 五月戊申,張燾薨。己酉,罷權借職田。庚戌,葉顒罷。以魏□巳為參知政事,右諫議大夫林安宅同知樞密院事兼權參知政事,中書舍人蔣芾簽書樞密院事。癸丑,太白晝見,經天。禁浙西修築圍田。罷修建康行宮。丁卯,命監司、
 守臣預備水旱。



 六月甲戌,罷兩浙路提舉市舶司。詔諸路監司、帥臣各察守令臧否以聞。丙子,刑部上《乾道新編特旨斷例》。戊寅,詔制科權罷注疏出題,守臣、監司亦許解送。庚辰,封孫挺為福州觀察使、榮國公,攄為左千牛衛大將軍。癸未,詔使相毋奏補文資,七色補官人毋任子,堂吏遷朝議大夫以五員為額。乙酉,申嚴內外牒式法,裁其額。丙戌,廢永豐圩。戊戌,詔改官人實歷知縣一任,方許關升。著為定式。秋七月己酉,調泉州左翼軍
 二千人屯許浦鎮。甲寅,以鎮江都統制戚方為武當軍節度使。



 八月辛未朔,詔兩淮行鐵錢,銅錢毋過江北。癸酉,以武鋒軍隸步軍司。甲戌。罷任子年三十得免試參選之令。丁丑,蠲淮南放歸萬弩手差役二年。壬午,詔諸州守臣兼訓練禁軍。癸未,降會子、交子於鎮江、建康務場,令江、淮之人對換。丙戌,林安宅劾葉顒之子受金失實,罷之。丁亥,詔安宅筠州居住,溫州大水。戊子,以魏杞兼同知樞密院事,蔣芾權參知政事。召葉顒。庚寅,少保、
 新興郡王吳蓋薨。甲午,立中興以來十三處戰功格目。乙未,詔吳璘復判興州。丙申,升宣州為寧國府。罷戶部諸路歲糴一年。



 九月甲辰,知上元縣李允升犯臟貸死,杖脊刺面,配惠州牢城,籍其貲。丙午,建康守臣王佐坐縱允升去官,奪三官勒停、建昌軍居住。餘失按官吏及薦舉官奪官有差。辛亥,遣官按視溫州水災,振貧民,決系囚。乙卯,詔改造大歷。辛酉,追封子恪為邵王,謚曰悼肅。甲子,詔監司各舉部內知縣、縣令二三人,守臣各舉
 屬縣一二人。己巳,魏□巳等上神宗、哲宗、徽宗三朝《帝紀》、《太上皇聖政》。太白晝見。是月,詔舉將帥,置章奏簿。



 冬十月癸酉,上《太上皇聖政》於德壽宮。乙亥,遣薛良朋等使金賀正旦。己卯,減饒州歲貢金三之一,蠲諸路灑坊逋賦。戊子,知峽州呂令問坐縱臟吏知夷陵縣韓贄冑去官,奪二官、鄂州居住。辛卯,雨雹。金遣魏子平等來賀會慶節。十一月丙午,楊存中薨。己酉,盡出內藏及南庫銀以易會子,官司並以錢銀支遣,民間從便。兩淮總領所
 許自造會子。鬻諸路營田。壬子,詔修祥曦殿記注。乙卯,密詔四川制置使汪應辰:如吳璘不起,收其宣撫使牌印,權行主管職事。甲子,大閱。戊辰,築郢州城。是月,詔汰冗兵。十二月庚午朔,白氣亙天。癸酉,詔三省、侍從、臺諫、兩淮漕臣、郡守,條具兩淮鐵錢、交子利害以聞。乙亥,遣梁克家等賀金主生辰。己卯,以資政殿學士葉顒知樞密院事。辛巳,詔免進呈《欽宗日歷》,送國史院修纂實錄。壬午,追封楊存中為和王。甲申,以葉顒為尚書左僕射,
 魏杞右僕射,並同中書門下平章事兼樞密使。蔣芾參知政事,吏部尚書陳俊卿同知樞密院事兼權參知政事。庚寅,詔宰相領兼制國用使,參知政事同知國用事。癸巳,詔監司、守臣舉廉吏。丙申,金遣烏古論元忠等來賀明年正旦。以江東兵馬鈐轄王抃為帶御器械。是歲,裁定內外軍額。



\end{pinyinscope}