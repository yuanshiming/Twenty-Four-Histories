\article{列傳第一百六}

\begin{pinyinscope}

 孫鼛吳
 時李昭玘吳師禮王漢之弟渙之黃廉朱服張舜民盛陶章衡顏復孫升韓川龔鼎臣鄭穆
 席旦喬執中



 孫鼛,字叔靜,錢塘人。父直官,徙揚之江都。鼛年十五,游太學,蘇洵、滕甫稱之。用父任,調武平尉,捕獲名盜數十,謝賞不受。再調越州司法參軍,守趙抃薦其材。知偃師縣,蒲中優人詭僧服隱民間,以不語惑眾,相傳有異法,奔湊其門。鼛收按奸狀,立伏辜。韓縝鎮長安,闢入府;縝去,後來者仍挽之使留,居五年,簽書西川判官。或薦於朝,召對,擢提舉廣東常平。徽宗初,徙兩浙。由福建轉運
 判官召為屯田員外。



 鼛微時與蔡京善,常曰:「蔡子,貴人也;然才不勝德,恐貽天下憂。」至是,京還朝,遇諸塗。既見,京逆謂曰:「我若用於天子,願助我。」鼛曰:「公誠能謹守祖宗之法,以正論輔人主,示節儉以先百吏,而絕口不言兵,天下幸甚。鼛何為者。」京默然。既相,出提點江東刑獄。



 未幾,入為少府少監、戶部郎中。縣官用度無藝,鼛與尚書曾孝廣、侍郎許幾謀曰:「日增一日,歲增一歲,天下之財豈能給哉?」共疏論之。當國者不樂,孝廣、幾由是罷,徙
 鼛開封。遷太僕卿、殿中少監。



 四輔建,以顯謨閣待制知曹州。論經始規畫之勞,轉太中大夫,徙鄆州。邑人子為「草祭」之謠,指切蔡京。鼛以聞,京怒,使言者誣以它謗,提舉鴻慶宮。起知單州,遂致仕。靖康二年卒,年八十六。贈銀青光祿大夫,謚曰通靖。



 鼛篤於行義,在廣東時,蘇軾謫居惠州,極意與周旋。二子娶晁補之、黃庭堅女,黨事起,家人危懼,鼛一無所顧。時人稱之。



 吳時,字伸道,邛州人。初舉進士,得學究出身;再試,中甲
 科。知華州鄭縣,轉運使檄州餫米五萬輸長安,鄭獨當三萬。時貽書使者曰:「會三萬斛之費,以車則千五百乘,以卒則五萬夫,縣民可役者才二百五十八戶耳。古者用師則贏糧以養兵,無事則移兵以就食,誠能移兵於華,則前費可免。華、雍相去百六十里,一旦欲用,朝發而夕至矣。」使者從其言。



 陸師閔干秦、蜀茶馬,闢為屬。章楶欲以御史薦,力辭之,徽宗求言,遠臣上章,封識多不能如式,有司悉卻之,時建言,乃得達。為睦親宅教授,提舉
 永興軍路學事。華州諸生有觸忌諱者,教授欲上之,曰:「是間言語,皆臣子所不忍聞。」時即火其書,曰:「臣子不忍聞,而令君父聞乎?」



 召為工部員外郎,改禮部,兼闢雍司業。大觀興算學,議以黃帝為先師。時言:「今祠祀聖祖,祝板書臣名,而釋奠孔子,但列中祀。數學,六藝之一耳,當以何禮事之?」乃止。遷太僕少卿。



 張商英罷相,言者指時為黨,出知耀州,又降通判鼎州;未赴,提舉河東常平。歲饑,發公粟以振民。童貫經略北方,每訪以邊事,輒不答。
 還為大晟典樂,擢中書舍人、給事中。內侍何欣謫監衡州酒,猶領節度使,時奏奪之。



 又因進對及取燕事,曰:「祖宗盟血未幹,渝之必速亂。」蔡攸聞之,以告王黼,黼怒,斥為腐儒。時求去,以徽猷閣待制兼侍讀,俄提舉上請太平宮。西歸,遇其里人趙雍,為言:「取燕必召禍。吾老,得不遭其變,幸矣。」累歲而卒,年七十八。



 時敏於為文,未嘗屬稿,落筆已就,兩學目之曰:「立地書廚。」



 李昭玘,字成季,濟南人。少與晁補之齊名,為蘇軾所知。
 擢進士第,徐州教授。守孫覺深禮之,每從容講學及古人行己處世之要,相得歡甚。用李清臣薦,為秘書省正字、校書郎,加秘閣校理。



 通判潞州,潞民死多不葬,昭□斥官地,畫兆□,具棺衾,作文風曉之,俗為一變。入為秘書丞、開封推官,俄提點永興、京西、京東路刑獄,坐元府黨奪官。



 徽宗立,召為右司員外郎,遷太常少卿。韓忠彥欲用為起居舍人,曾布持之,布使山陵,命始下。為陳次升所論,出知滄州。崇寧初,詔以昭□嘗傾搖先烈,每改
 元豐敕條,倡從寬之邪說,罷主管鴻慶宮,遂入黨籍中。居閑十五年,自號樂靜先生。寓意法書、圖畫,貯於十囊,命曰:「燕游十友」,為之序,以為:「與今之人友,或趨附而陷於禍,吾寧與十者友,久益有味也。」



 初,昭□校試高密,得侯蒙。蒙執政,思顧舊恩,使人致己意,昭□唯求秘閣法帖而已。使陜西時,延安小將車吉者被誣為盜,昭□察知無它。吉後立戰功,至皇城使,遇昭□京師,拜於前曰:「感公生存之恩,願以名馬為獻。」笑卻之。



 晚知歙州,辭不
 行。靖康初,復以起居舍人召,而已卒。紹興初,追復直徽猷閣。



 吳師禮,字安仲,杭州錢塘人。太學上舍賜第,調涇縣主簿,知天長縣。召太學博士、秘書省正字,預餞鄒浩,免。徽宗初,為開封府推官。蔡王似宮吏有不順語,下之府,師禮主治。獄成,不使一詞及王;吏雖有死者,亦不被以指斥罪。擢右司諫,改右司員外郎。



 師禮工翰墨,帝嘗訪以字學,對曰:「陛下御極之初,當志其大者,臣不敢以末伎
 對。」聞者獎其得體。以直秘閣知宿州,卒。



 師禮游太學時,兄師仁為正,守《春秋》學。它學官有惡之者,條其疑問諸生,師禮悉以兄說對。學官怒,鳴鼓坐堂上,眾質之,師禮引據《三傳》,意氣自如。江公望時在旁,心竊喜。後相遇於泌陽,公望謂曰:「子異日得志,當如何?」曰:「但為人作豐年耳。」遂定交。



 師仁字坦求。篤學厲志,不事科舉。喪親,廬墓下,日倩旁寺僧造飯一缽以充饑,不復置庖爨及蓄僮僕。郡守陳襄、鄧潤甫、蒲宗孟皆以遺逸薦於朝。元祐初,
 召為太學正,遷博士,十年無它除。後為穎川、吳王宮教授,卒。



 王漢之,字彥昭,衢州常山人。父介,舉制科,以直聞,至秘閣校理。漢之進士甲科,調秀州司戶參軍,知金華、澠池二縣,為鴻臚丞,知真州。時詔諸道經畫財用上諸朝,漢之言:「所在無都籍,是以不能周知而校其登耗以待用。願令郡縣先置籍,總之諸道,則天下如指諸掌矣。」從之。入為開封府推官,歷工、吏、禮三部員外郎,太常少卿。



 蔡
 京置講議司。漢之,其客也,引為參詳官。擢禮部侍郎,轉戶部,以顯謨閣待制知瀛州。言:「自何承矩規塘濼之地屯田,東達於海。其後又修保塞五州為堤道,備種所宜木至三百萬本,此中國萬世之利也。今浸失其道,願講行之。」雄州歸信、容城災,兩輸戶請蠲稅,吏不聽。漢之言:「雄州規小利,失大體,萬一契丹蠲之,為朝廷羞。」



 徙江寧、河南府,不至,而為蘇、潭、洪三州。召拜兵部侍郎,復以顯謨閣直學士知成都,又不至,連徙五州,入為工部侍郎。
 奉使契丹,還,言其主不恤民政,而掊克荒淫,亡可跂而待也。徽宗悅,以知定州。久之,徙江寧。



 方臘之亂,錄奏報御捕功,加龍圖閣直學士,又進延康殿學士。卒,年七十。弟渙之。



 渙之字彥舟。未冠,擢上第,有司疑年未及銓格,特補武勝軍節度推官。方新置學官,以為杭州教授,知穎上縣。元祐中,為太學博士,校對黃本秘書。通判衛州,入編修《兩朝魯衛信錄》。



 徽宗立,以日食求言。渙之用大臣交薦
 召對,因言:「求言非難,聽之難;聽之非難,察而用之難。今國家每下求言之詔,而下之報上,乃或不然,以指陳闕失為訕上,以阿諛佞諂為尊君,以論議趨時為國是,以可否相濟為邪說。志士仁人知言之無益也,不復有言,而小人肆為詭譎可駭之論,茍容偷合。願陛下虛心公聽,言無逆遜,唯是之從;事無今昔,唯當為貴;人無同異,唯正是用。則人心說,治道成,天意得矣。」帝欣然延納,欲任以諫官、御史。辭曰:「臣由大臣薦,不可以居是官。」乃拜
 吏部員外郎,遷左司員外郎、起居舍人,擢中書舍人。趨省之日,詞頭三十三,下筆即就。



 崇寧初,進給事中、吏部侍郎,以寶文閣待制知廣州。言者論渙之當元祐之末,與陳瓘、龔夬、張庭堅游,既棄於紹聖,而今復之,有害初政。解職知舒州,入黨籍。尋知福州,未至,復徙廣州。蕃客殺奴,市舶使據舊比,止送其長杖笞,渙之不可,論如法。



 召詣闕,言者復拾故語以阻之,罷為洪州。改滁州,歷潭、杭、揚三州。張商英相,為給事中、吏部侍郎。商英去,亦出
 守。越八年,知中山府,加寶文閣直學士。朝廷議北伐,渙之以疾提舉明道宮。又四年卒,年四十五。



 渙之性淡泊,恬於仕進,每云:「乘車常以顛墜處之,乘舟常以覆溺處之,仕宦常以不遇處之,則無事矣。」其歸趣如此。



 黃廉,字夷仲,洪州分寧人。第進士,歷州縣。熙寧初,或薦之王安石。安石與之言,問免役事,廉據舊法以對,甚悉。安石曰:「是必能辦新法。」白神宗,召訪時務,對曰:「陛下意在便民,法非不良也,而吏非其人。朝廷立法之意則一,
 而四方推奉紛然不同,所以法行而民病,陛下不盡察也。河朔被水,河南、齊、晉旱,淮、浙飛蝗,江南疫癘,陛下不盡知也。」帝即命廉體量振濟東道,除司農丞。還報合旨,擢利州路轉運判官,復丞司農。



 為監察御史裏行,建言:「成天下之務,莫急於人才,願令兩制近臣及轉運使各得舉士。」詔各薦一人。繼言:「寒遠下僚,既得名聞於上,願令中書審其能而表用,則急才之詔,不虛行於天下矣。」又言:「比年水旱,民蒙支貸倚閣之恩,今幸歲豐,有司悉
 當舉催。久饑初稔,累給並償,是使民遇豐年而思歉歲也,請令諸道以漸督取之。」



 論俞充結王中正致宰屬,並言中正任使太重。帝曰:「人才蓋無類,顧駕御之何如耳。」對曰:「雖然,臣慮漸不可長也。」



 河決曹村,壞田三十萬頃、民廬舍三十八萬家。受詔安撫京東,發廩振饑,遠不能至者,分遣吏移給,擇高地作舍以居民,流民過所毋徵算,轉行者賦糧,質私牛而與之錢,養男女棄於道者,丁壯則役其力,凡所活二十五萬。



 相州獄起,鄧溫伯、上官
 均論其冤,得譴去,詔廉詰之,竟不能正。未幾獄成,始悔之。加集質校理,提點河東刑獄。



 遼人求代北地,廉言:「分水畫境,失中國險固,啟豺狼心。」其後契丹果包取兩不耕地,下臨雁門,父老以為恨。王中正發西兵,用一而調二,轉運使又附益之,廉曰:「民朘剝至骨,斟酌不乏興,足矣!忍自竭根本邪?」即奏云:「師必無功,盍有以善其後?」既,大軍潰歸,中正嫁罪於轉餉。廉指上黨對理,坐貶秩。



 元祐元年,召為戶部郎中。陸師閔茶法為川、陜害,遣廉使
 蜀按察,至則奏罷其太甚者。且言:「前所為誠病民,若悉以予之,則邊計不集,蜀貨不通,園甿將受其敝。請榷熙、秦茶勿罷,而許東路通商;禁南茶毋入陜西,以利蜀貨。定博馬歲額為萬八千匹。」朝廷可其議,使以直秘閣提舉。



 明年,進為左司郎中,遷起居郎、集賢殿修撰、樞密都承旨。上官均論其往附蔡確為獄,改陜西都轉運使。拜給事中,卒,年五十九。



 朱服,字行中,湖州烏程人。熙寧進士甲科,以淮南節度
 推官充修撰、經義局檢討,歷國子直講、秘閣校理。元豐中,擢監察御史裏行。參知政事章惇遣所善袁默、周之道見服,道薦引意以市恩,服舉劾之。惇補郡,免默、之道官。



 受詔治朱明之獄。故事,制獄許上殿,非本章所云者皆取旨。服論其非是,罷之。俄知諫院,遷國子司業、起居舍人,以直龍圖閣知潤州,徙泉、婺、寧、廬、壽五州。廬人饑,守便宜振護,全活十餘萬口。明年大疫,又課醫持善藥分拯之,賴以安省甚眾。



 當元祐時,未嘗一日在朝廷,不
 能無少望。值紹聖初政,因表賀,乃力詆變亂法度之故。召為中書舍人。使遼,未反而母死,詔以其家貧,賜帛二百。喪除,拜禮部侍郎。湖州守馬城言其居喪疏幾筵而獨處它室,謫知萊州。



 徽宗即位,加集賢殿修撰,再為廬州;越兩月,徙廣州。哲宗既祥,服賦詩有「孤臣正泣龍髯草」之語,為部使者所上,黜知袁州。又坐與蘇軾游,貶海州團練副使,蘄州安置。改興國軍,卒。



 張舜民,字蕓叟,邠州人。中進士第,為襄樂令。王安石倡
 新法,舜民上書言:「便民所以窮民,強內所以弱內,闢國所以蹙國。以堂堂之天下,而與小民爭利,可恥也。」時人壯之。元豐中,朝廷討西夏,陳留縣五路出兵,環慶帥高遵裕闢掌機密文字。王師無功,舜民在靈武詩有「白骨似沙沙似雪」,及官軍「斫受降城柳為薪」之句,坐謫監邕州鹽米倉;又追赴鄜延詔獄,改監郴州酒稅。



 會赦北還,司馬光薦其才氣秀異,剛直敢言,以館閣校勘為監察御史。上疏論西夏強臣爭權,不宜加以爵命,當興師問
 罪,因及文彥博,左遷監登聞鼓院。臺諫交章乞還職,不聽。通判虢州,提點秦鳳刑獄。召拜殿中侍御史,固辭,改金部員外郎。進秘書少監,使遼,加直秘閣、陜西轉運使,知陜、潭、青三州。元符中,罷職付東銓,以為坊州、鳳翔,皆不赴。



 徽宗立,擢右諫議大夫,居職才七日,所上事已六十章。陳陜西之弊曰:「以庸將而御老師,役饑民而爭曠土。」極論河朔之困,言多剴峭。徙吏部侍郎,旋以龍圖閣待制知定州,改同州。坐元祐黨,謫楚州團練副使,商州
 安置。復集賢殿修撰,卒。



 舜民慷慨喜論事,善為文,自號浮休居士。其使遼也,見其太孫禧好音樂、美姝、名茶、古畫,以為他日必有如唐張義潮挈十三州來歸者,不四十年當見之。後如其言。紹興中,追贈寶文閣直學士。



 盛陶,字仲叔,鄭州人。第進士。熙寧中,為監察御史。神宗問河北事,對曰:「朝廷以便民省役,議廢郡縣,誠便。然沿邊地相屬,如北平至海不過五百里,其間列城十五,祖宗之意固有所在,願仍舊貫。」慶州李復圭輕敵敗國,程
 昉開河無功,籍水政以擾州縣,皆疏其過。二人實王安石所主,陶不少屈,出簽書隨州判官。



 久之,入為太常博士、考功員外郎、工部右司郎中,至侍御史。陳官冗之敝,謂恩澤舉人,宜取嘉祐、治平之制;選人改官,宜準熙寧、元豐之法。諫官劉安世等攻蔡確為謗詩,陶曰:「確以弟碩有罪,但坐罷職,不應懷恨。注釋詩語,近於捃摭,不可以長告訐之風。」安世疏言:「陶居風憲地,目睹無禮於君親之人,而附會觀望,紀綱何賴。」出知汝州,徙晉州,召為
 太常少卿。



 議合祭天地,請從先帝北郊之旨;既而合祭,陶即奉行,亦不復辨執也。進權禮部侍郎、中書舍人,以龍圖閣待制知應天府、順昌府、瀛州。元符中,例奪職,卒,年六十七。



 論曰:王氏、章、蔡之當國也,士大夫知拂之必斥,附之必進也,而孫鼛正言蔡京,不肯為之助;吳時卻童貫,忤王黼,乃幸於罷歸;昭□辭侯蒙之延致;朱服發章惇之薦引,舜民詆新法;而盛陶不屈於安石,其大節皆可取。獨
 漢之為京客,黃廉附蔡確獄,有愧鼛等多矣。《易》曰:「介於石,不終日,貞吉。」故君子貴乎知幾。



 章衡,字子平,浦城人。嘉祐二年,進士第一。通判湖州,直集賢院,改鹽鐵判官,同修起居注。物有掛空籍者,奏請蠲之。又言:「三司經費,取領而無多寡,率不預知。急則斂於民,倉卒趣迫,故苦其難供。願敕三部判官,簿正其數,即有所賦,先期下之,使公私皆濟。」三司使忌其能,出知汝州、穎州。



 熙寧初,還判太常寺。建言:「自唐開元纂修禮
 書,以『國恤』一章為豫兇事,刪而去之。故不幸遇事,則捃摭墜殘,茫無所據。今宜為《厚陵集禮》,以貽萬世。」從之。



 出知鄭州,奏罷原武監,馳牧地四千二百頃以予民。復判太常,知審官西院。使遼,燕射運發破的,遼以為文武兼備,待之異於他使。歸復命,言遼境無備,因此時可復山後八州。不聽。



 衡患學者不知古今,纂歷代帝系,名曰《編年通載》,神宗覽而善之,謂可冠冕諸史;且念其嘗先多士,進用獨後,面賜三品服。判吏部流內銓,嘗有員闕,既
 擬注,而三班院輒用之,反訟吏部。宰相主其說,衡連奏疏與之辨。或曰宰相之勢,恐不可深校,衡不為止,至訴於御前。神宗命內侍偕至中書,宰相見之怒,衡曰:「衡為朝廷法耳。」以狀上請而視之,相悟曰:「若爾,吏部是矣。」乃罪三班。



 未幾,知通進銀臺司、直舍人院,拜寶文閣待制、知澶州。神宗曰:「卿為仁宗朝魁甲,寶文藏御集之處,未始除人,今以之處卿。」衡拜謝。至郡,會官立法禁民販鹽,衡言:「民恃鹽以生,生之所在,雖犯法不顧。空令犴獄日
 繁,請如故便。」徙知成德軍,坐事免。



 元祐中,歷秀、襄、河陽、曹、蘇州,加集賢院學士,復以待制知揚、廬、宣、穎州,卒,年七十五。



 顏復,字長道,魯人,顏子四十八世孫也。父太初,以名儒為國子監直講,出為臨晉簿。嘉祐中,詔郡國敦訪遺逸,京東以復言。凡試於中書者二十有二人,考官歐陽修奏復第一。賜進士,為校書郎,知永寧縣。熙寧中,為國子直講。王安石更學法,取士率以己意,使常秩等校諸直
 講所出題及所考卷,定其優劣,復等五人皆罷。



 元祐初,召為太常博士。建言:「士民禮制不立,下無矜式。請令禮官會萃古今典範為五禮書。又請考正祀典,凡幹讖緯曲學、污條陋制、道流醮謝、術家厭勝之法,一切芟去。俾大小群祀盡合聖人之經,為後世法。」遷禮部員外郎。孔宗翰請尊奉孔子祠,復因上五議,欲專其祠饗,優其田祿,蠲其廟幹,司其法則,訓其子孫。朝廷多從之。



 兼崇政殿說書,進起居舍人兼侍講,轉起居郎。請擇經行之儒,
 補諸縣教官;凡學者考其志業,不由教官薦,不得與貢舉、升太學。拜中書舍人兼國子監祭酒。言:「太學諸生,有誘進之法,獨教官未嘗旌別,似非嚴師勸士之道。」未逾年,以疾改天章閣待制,未拜而卒,年五十七。王巖叟等言復學行超特,宜加優賻,詔賜錢五十萬。子岐,建炎中為門下侍郎。



 孫升,字君孚,高郵人。第進士,簽書泰州判官。哲宗立,為監察御史。朝廷更法度,逐奸邪,升多所建明。嘗上疏曰:「
 自二聖臨御,登用正人,天下所謂忠信端良之士,豪傑俊偉之材,俱收並用,近世得賢之盛,未有如今日者。君子日進而小人日退,正道日長而邪慝日消,在廷濟濟有成周之風,此首開言路之效也。願於耳目之臣,論議之際,置黨附之疑,杜小人之隙;疑間一開,則言者不安其職矣。言者不安其職,則循默之風熾,而壅蔽之患生,非朝廷之福也。」遷殿中待御史。



 梁燾責張問,升從而擊之,執政指為附和,出知濟州。逾年,提點京西刑獄,召為
 金部員外郎,復拜殿中侍御史,進侍御史。時翰林承旨鄧溫伯為臺臣所攻,升與賈易論之尤力。謂草蔡確制,稱其定策功比漢周勃,欺天負國,豈宜親承密命?不報。由起居郎擢中書舍人,直學士院,以天章閣待制知應天府。董敦逸、黃廷基摭升過,改集賢院學士。



 紹聖初,翟思、張商英又劾之,削職,知房州、歸州;貶水部員外郎,分司;又貶果州團練副使,汀州安置。卒,年六十二。



 升在元祐初,嘗言:「王安石擅名世之學,為一代文宗。及進居大
 位,出其私智,以蓋天下之聰明,遂為大害。今蘇軾文章學問,中外所服,然德業器識,有所不足。為翰林學士,已極其任矣;若使輔佐經綸,願以安石為戒。」世譏其失言。



 韓川,字符伯,陜人。進士上第,歷開封府推官。元祐初,用劉摯薦,為監察御史。極論市易之害,以為:「雖曰平均物直,而其實不免貨交以取利,就使有獲,尚不可為,況所獲不如所亡,果何事也?願量留官吏,與之期,使趣罷此法。」從之。



 遷殿中侍御史。疏言:「朝廷於人才,常欲推至公
 以博採,及其弊也,則幾於利權勢而抑孤寒;常欲收勤績以赴用,要其終也,則莫不收虛名而廢實效。近制太中大夫以上歲舉守臣,遇大州闕,則選諸所表;他雖考課上等,皆莫得預。推原旨意,固欲得人。然所謂太中大夫以上,率在京師,諸馳騖請求者,得之為易;至於淹歷郡縣治狀應法者,顧出其下,則是謹身修潔之人,不若營求一章之速化也。」於是詔吏部更立法。



 張舜民論西夏事,乞停封冊,朝廷以為開邊隙,罷其御史。梁燾等為舜
 民爭之。川與呂陶、上官均謂舜民之言,實不可行。燾等去,川亦改太常少卿,不拜,加集賢校理、知穎州。還為侍御史、樞密都承旨,進中書舍人、吏、禮二部侍郎,以龍圖閣待制復守穎,徙虢州。與孫升同受責,由坊州、郢州貶屯田員外郎,分司,岷州團練副使,道州安置。徽宗立,得故官,知青、襄二州,卒。



 龔鼎臣,字輔之,鄆之須城人。父誘衷,武陵令。鼎臣幼孤自立,景祐元年第進士,為平陰主簿,疏洩瀦水,得良田
 數百千頃。調孟州司法參軍,以薦,為泰寧軍節度掌書記。



 徂徠石介死,讒者謂介北走遼,詔袞州劾狀。郡守杜衍會問,掾屬莫對,鼎臣獨曰:「介寧有是,願以闔門證其死。」衍探懷出奏稿示之,曰:「吾既保介矣,君年少見義如是,未可量也。」舉為秘書省著作佐郎、知萊蕪縣。大臣薦試館職,坐與石介善,不召。徙知蒙陽縣,轉秘書丞。丁母憂,服除,知安丘縣。以賢良方正召試秘閣,轉太常博士,賜五品服,知渠州。渠故僻陋無學者,鼎臣請於朝,建廟
 學,選邑子為生,日講說,立課肄法,人大勸,始有登科者。郡人繪像事之。



 召入編校史館書籍,轉都官,擢起居舍人、同知諫院。歲冬旱,將錫春宴,鼎臣曰:「旱甾太甚,非君臣同樂之時,請罷宴以答天戒。」日當食,陰雲不見,鼎臣曰:「陽精既虧,四方必見,為異益大,願精思力行,進賢遠佞,以應皇極。」又論內侍都知鄧保信罪狀,不應出入禁中;蘇安靜年未五十,不應超押班;妃嬪贈三代,僭後禮;董淑妃賜謚,非是;凡大禮赦,請準太平興國詔書,前期
 下禁約,後有犯不原,以杜指赦為奸者,宜著為令;開封三司於法外斷獄,朝廷多曲徇其請,願先付中書審畫。仁宗悉從之。



 尋兼管勾國子監,判登聞檢院,詳定寬恤民力奏議。淮南災,以鼎臣體量安撫,蠲逋振貸,全活甚眾。為遼正旦使,鼎臣奏:「景德中,遼犯淄、青,臣祖母、兄、姊皆見略,義不忍往。」許之,仍詔後子孫並免行焉。



 俄拜戶部員外郎兼侍御史知雜事,賜三品服。轉吏、禮二部郎中。論宗室宜歲試補外官,請汰濫官冗兵,蕃財用,禁奢
 靡。連劾薛向奸暴,鬻鹽、市馬皆罔上。英宗登位,屢乞延訪臣下,親決國事。上疏勸皇太后早還政;及卷簾而御璽未復,又極論。謂昭陵宜儉葬,景靈神御殿不宜增侈,以彰先帝恭德。鼎臣在言路累歲,闊略細故,至大事,無所顧忌。然其言優游和平,不為峻激,使人主易聽,退亦未嘗語人,故其事多施行。



 改集賢殿修撰、知應天府,徙江寧。召還,判太常寺兼禮儀事。神宗即位,判吏部流內銓、太常寺。選人得官,待班謝辭,率皆留滯。鼎臣奏易為
 門謝辭,甚便之。明堂議侑帝,或云以真宗,或云以仁宗。鼎臣曰:「嚴父莫大於配天,未聞以祖也。」乃奉英宗配。王安石侍講,欲賜坐。事下禮官,鼎臣言不可,安石不悅。求補外,知袞州。



 是時,諸道方田使者希功賞,概取稅虛額及嘗所蠲者,加舊籍以病民。鼎臣獨按籍差次為十等,一無所增,袞人德之。改吏部,提舉西京崇福宮。復判太常寺,留守南京。陛辭,神宗顧語移晷,喜曰:「人言卿老不任事,精明乃爾,行且用卿矣。」



 時河決曹村,流殍滿野,鼎
 臣勞來振拊,歸者不勝計。拜諫議大夫、京東東路安撫使、知青州,改太中大夫,請老,提舉亳州太清宮。尋以正議大夫致仕,年七十七,元祐元年卒。



 鄭穆,字閎中,福州侯官人。性醇謹好學,讀書至忘櫛沐,進退容止必以禮。門人千數,與陳襄、陳烈、周希孟友,號「四先生」。舉進士,四冠鄉書,遂登第,為壽安主簿。召為國子監直講,除編校集賢院書籍。歲滿,為館閣校勘,積官太常博士。乞納一秩,先南郊追封考妣,從之。改集賢校
 理,求外補,通判汾州。



 熙寧三年,召為岐王侍講。嘉王出閣,改諸王侍講。府僚闕員,御史陳襄請擇人,神宗曰:「如鄭穆德行,乃宜左右王者。」凡居館閣三十年,而在王邸一紀,非公事不及執政之門。講說有法,可為勸戒者,必反復擿誦,岐、喜二王咸敬禮焉。



 元豐三年,出知越州,加朝散大夫。先是,鑒湖旱乾,民因田其中,延袤百里,官籍而稅之。既而連年水溢,民逋官租積萬緡,穆奏免之。未滿告老,管勾杭州洞霄宮。



 元祐初,召拜國子祭酒。每講
 益,無問寒暑,雖童子必朝服廷接,以禮送迎。諸生皆尊其經術,服其教訓。故人張景晟者死,遣白金五百兩,托其孤,穆曰:「恤孤,吾事也,金於何有?」反金而收其子,長之。三年,揚王、荊王請為侍講,罷祭酒,除直集賢院,復入王府。荊王薨,為揚王翊善。太學生乞為師,復除祭酒,兼徐王翊善。四年,拜給事中兼祭酒;五年,除寶文閣待制,仍祭酒。



 六年,請老,提舉洞霄宮。敕過門下,給事中範祖禹言:「穆雖年出七十,精力尚強。古者大夫七十而致仕,有
 不得謝,則賜之幾杖。祭酒居師資之地,正宜處老成,願毋輕聽其去。」不報。太學之士數千人,以狀詣司業,又詣宰相請留,亦不從。於是公卿大夫各為詩贈其行。空學出祖汴東門外,都人觀者如堵,嘆未嘗見。明年卒,年七十五。子璆,軍事推官。



 席旦,字晉仲,河南人。七歲能詩,嘗登沉黎嶺,得句警拔,觀者驚異。元豐中,舉進士,禮部不奏名。時方求邊功,旦詣闕上書言:「戰勝易,守勝難,知所以得之,必知所以守
 之。」神宗嘉納,令廷試賜第。歷齊州司法參軍、鄭州河陽教授、敕令所刪定官。



 徽宗召對,擢右正言,遷右司諫。御史中丞錢遹率同列請廢元祐皇后而冊劉氏為太后,旦面質為不可。遹劾旦陰佐元祐之政,左轉吏部員外郎。改太常少卿,遷中書舍人、給事中。新建殿中省,命為監,俄拜御史中丞兼侍講。



 內侍郝隨驕橫,旦劾罷之,都人誦其直。帝以其章有「媚惑先帝」之語,嫌為指斥,旋改吏部侍郎,以顯謨閣待制知宣州。召為戶部侍郎,還吏
 部。郝隨復入侍,乃以顯謨閣直學士知成都府。



 自趙諗以狂謀誅後,蜀數有妖言,議者遂言蜀土習亂。或導旦治以峻猛,旦政和平,徙鄭州。入見,言:「蜀人性善柔,自古稱兵背叛,皆非其土俗,願勿為慮。」遂言:「蜀用鐵錢,以其艱於轉移,故權以楮券,而有司冀贏羨,為之益多,使民不敢信。」帝曰:「朕為卿損數百萬虛券,而別給緡錢與本業,可乎?」對曰:「陛下幸加惠遠民,不愛重費以救敝法,此古聖王用心也。」自是錢引稍仍故。



 坐進對淹留,黜知滁
 州。久之,帝思其治蜀功,復知成都。朝廷開西南夷,黎州守詣幕府白事,言雲南大理國求入朝獻,旦引唐南詔為蜀患,拒卻之。已而威州守焦才叔言,欲誘保、霸二州內附。旦上章劾才叔為奸利斂困諸蕃之狀,宰相不悅,代以龐恭孫,而徙旦永興。恭孫俄罪去,加旦述古殿直學士,復知成都。時郅永壽、湯延俊納土,樞密院用以訹旦,旦曰:「吾以為朝廷悔開疆之禍,今猶自若邪?」力辭之。卒於長安,年六十二,贈太中大夫。



 旦立朝無所附徇,第
 為中丞時,蔡王似方以疑就第,旦糾其私出府,請推治官吏,議者哂之。子益,字大光,紹興初,參知政事。



 喬執中,字希聖,高郵人。入太學,補《五經》講書,五年不謁告。王安石為群牧判官,見而器之,命子弟與之游。擢進士,調須城主簿。時河役大興,部役者不得人。一夕,噪而潰,因致大獄。執中往代,終帖然。富民賂吏,將創橋所居以罔市利,執中疏其害,使者入吏言使成之,執中曰:「官可去,橋不可創也。」卒不能奪。



 王安石為政,引執中編修《
 熙寧條例》,選提舉湖南常平。章惇討五溪,檄執中取大田、離子二峒。峒路險絕,期迫,執中但走一校諭其酋,即相率歸命。錄功當遷秩,辭以及父母。



 就徙轉運判官,召為司農丞、提點開封縣鎮。諸縣牧地,民耕歲久,議者將取之,當夷丘墓,伐桑柘,萬家相聚而泣。執中請於朝,神宗詔復予民。改提點京西北路刑獄。時河決廣武,埽危甚,相聚莫敢登。執中不顧,立其上,眾隨之如蟻附,不日埽成。



 元祐初,為吏部郎中,請選人由縣令、錄事參軍致
 仕者,升朝籍,得封其親。兼徐王府侍講、翊善,遷起居舍人、起居郎,權給事中。有司以天下讞獄失出入者同坐,執中駁之曰:「先王重入而輕出,恤刑之至也。今一旦均之,恐自是法吏不復肯與生比,非好生洽民之意也。」進中書舍人。邢恕遇赦甄復,執中言:「恕深結蔡確,鼓唱扇搖,今復其官,懼疑中外。」遷給事中、刑部侍郎。



 紹聖初,上官均摭執中為呂大防所用,以寶文閣待制知鄆州。執中寬厚有仁心,屢典刑獄,雪活以百數。明年,夢神人畀
 以騎都尉,詰旦為客言之,少焉,談笑而逝,年六十三。



 論曰:宋之人才,自祖宗涵養,至於中葉,盛矣。顏復、鄭穆醇然儒者,宜居師表。龔鼎臣、喬執中始終不渝厥守,豈易得哉。章衡欲復山後八州,為國啟釁;孫升以蘇軾比王安石為人;韓川詆張舜民之言不可行;席旦以蔡王見疑,因而擠之。然瑕不掩瑜,它善蓋亦有可稱者。古稱「才難不其然」者,其斯之謂歟?



\end{pinyinscope}