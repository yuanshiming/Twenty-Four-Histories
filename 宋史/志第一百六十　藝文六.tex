\article{志第一百六十 藝文六}

\begin{pinyinscope}

 苗銳《新刪定廣聖歷》二卷



 僧一行《開元大衍歷議》十三卷



 啟玄子《天元玉冊》十卷



 甄鸞《五曹算術》二卷



 《海島算術》一卷



 趙君卿《周髀算經》二卷



 張丘建《算經》三卷



 夏侯
 陽《算經》三卷



 王孝通《緝古算經》一卷



 謝察微《算經》三卷



 李籍《九章算經音義》一卷



 又《周髀算經音義》一卷



 李紹穀《求一指蒙算術玄要》一卷



 郭獻之《唐寶應五紀歷》三卷



 徐承嗣《唐建中貞元歷》三卷



 邊岡《唐景福崇玄歷》十三卷



 《大唐長歷》一卷



 馬重績《晉天福調元歷》二十三卷



 王處訥《周廣順明元歷》一卷



 又《建隆應天歷》六卷



 王樸《周顯德欽天歷》十五卷



 《蜀武成永昌歷》三卷



 《唐保大齊政歷》三卷



 苗訓《太平乾元歷》九卷



 《太平興國七年新修
 歷經》三卷



 史序《儀天歷》十六卷



 曹士□《七曜符天歷》二卷



 《七曜符天人元歷》三卷



 楊緯一作「繹」



 《符天歷》一卷



 王公佐《中正歷》三卷



 《正像歷》一卷



 李思議《重注曹士□小歷》一卷



 《七曜符天歷》一卷



 《大衍通玄鑒新歷》三卷



 沉括《熙寧奉元歷》一部卷亡



 《熙寧奉元歷經》三卷



 《立成》十四卷



 《備草》六卷



 《比較交蝕》六卷



 衛樸《七曜細行》一卷



 《新歷正經》三卷



 《義略》二卷



 《立成》十五卷



 《隨經備草》五卷



 《七曜細行》一卷



 《長歷》三十卷



 並孫思恭注



 《大衍歷經》二卷



 《大衍歷立成》
 十一卷



 《大衍歷議略》一卷



 《大衍議》十卷



 《宣明歷經》二卷



 《宣明歷立成》八卷



 《宣明歷要略》一卷



 《大衍歷經》二卷



 《正元歷立成》八卷



 《崇元歷經》三卷



 《崇元歷立成》七卷



 《調元歷經》二卷



 《調元歷立成》十二卷



 《調元歷草》八卷



 《欽天歷經》二卷



 《欽天歷立成》六卷



 《欽天歷草》三卷



 《應天歷經》二卷



 《應天歷立成》一卷



 《乾元歷經》二卷



 《乾元歷立成》二卷



 《儀天歷經》二卷



 《儀天歷立成》十三卷



 《崇天歷經》二卷



 《崇天歷立成》四卷



 《明天歷經》三卷



 《明天歷立成》十五卷



 《明
 天歷略》二卷



 《符天歷》三卷



 姚舜輔《蝕神隱耀歷》三卷



 丘浚《霸國環周立成歷》一卷



 《陰陽集正歷》三卷



 《歷日纂聖精要》一卷



 《歷樞》二卷



 《難逃論》一卷



 《符天行宮》一卷



 《轉天圖》一卷



 《萬歲日出入晝夜立成歷》一卷



 《五星長歷》一卷



 《正像歷》一卷



 胡秀林《正像歷經》一卷



 章浦《符天九曜通元立成法》二卷



 《氣神經》三卷



 《氣神鈐歷》一卷



 《氣神隨日用局圖》一卷



 莊守德《七曜氣神歌訣》一卷



 呂才《刻漏經》一卷



 錢明逸《西國七曜歷》一卷



 關子明注《安修睦都利
 聿斯訣》一卷



 《聿斯隱經》一卷



 《聿斯妙利要旨》一卷



 李淳風注釋《九章經要略》一卷



 又注釋《孫子算經》三卷



 注王孝通《五經算法》一卷



 注甄鸞《五曹算法》二卷



 劉微一作「徽」



 《九章算田草》九卷



 王孝通《緝古算經》一卷



 程柔《五曹算經求一法》三卷



 魯靖《五曹時要算術》三卷



 《五曹乘除見一捷例算法》一卷



 夏翰一作「翱」



 《新重演議海島算經》一卷



 甄鸞注徐岳《大衍算術法》一卷



 謝察微《發蒙算經》三卷



 僧一行《心機算術括一作「格」》一卷僧□妻巖注



 徐仁美《增成玄一算
 經》三卷



 陳從運《得一算經》七卷



 《三問田算術》一卷



 龍受益《算法》二卷



 又《求一算術化零歌》一卷



 《新易一法算範九例要訣》一卷



 徐岳《術數記遺》一卷



 《合元萬分歷》三卷作者名術,不知姓



 《注九章算經》九卷魏劉徽、唐李淳風注



 《孫子算經》三卷不知名



 《五曹算經》五卷李淳風等注



 《長慶宣明大歷》二卷



 《萬年歷》十二卷



 《青蘿妙度真經大歷》一卷



 《行漏法》一卷



 《太始天元玉冊截法》六卷



 《求一算法》一卷



 《細歷書》一卷



 《玉歷通政經》三卷



 並不知作者



 燕肅《蓮花漏法》一卷



 錢明逸《刻漏規
 矩》一卷



 王普《小漏款職》一卷



 《官歷刻漏圖》一卷



 衛樸《奉元歷經》一卷



 《觀天歷經》一卷紹聖、元符頒行



 姚舜輔《紀元歷經》一卷



 裴伯壽、陳得一《統元歷經》七卷



 又《統元歷五星立成》二卷



 《統元歷盈縮朏朒立成》一卷



 《統元歷日出入氣刻立成》一卷



 《統元歷義》二卷



 《統元七曜細行歷》二卷



 《統元歷氣朔八行草》一卷



 《統元歷考古日食》一卷



 《三歷會同集》十卷紹興初撰,不知名



 張祚注《法算三平化零歌》一卷龍受益法



 王守忠《求一術歌》一卷



 《算範要
 訣》二卷



 《明算指掌》三卷



 江本《一位算法》二卷



 任弘濟《一位算法問答》一卷



 楊鍇明《微算經》一卷



 《法算機要賦》一卷



 《法算口訣》一卷



 《算法秘訣》一卷



 《算術玄要》一卷



 劉孝榮《新歷考古春秋日食》一卷



 《新歷考漢魏周隋日月交食》一卷



 《新歷考唐交食》一卷



 《新歷氣朔八行》一卷



 《強弱月格法數》一卷



 賈憲《黃帝九章算經細草》九卷



 張宋圖《史記律歷志訛辨》一卷



 《儀象法要》一卷紹聖中編



 《細行歷書》二十卷起慶元庚申,至嘉定己卯,太史局進



 右歷算類一百六十五部,五百九十八卷。



 《風後握機》一卷晉馬隆略序



 《六韜》六卷不知作者



 《司馬兵法》三卷齊司馬穰苴撰



 孫武《孫子》三卷



 吳起《吳子》三卷



 《黃帝秘珠三略》三卷



 《陰符二十四機》一卷



 《握機圖》一卷



 《決勝孤虛集》一卷



 《太公兵書要訣》四卷



 朱服校定《六韜》六卷



 又校定《孫子》三卷



 校定《司馬法》三卷



 校定《吳子》二卷



 校定《三略》三卷



 魏武帝注《孫子》三卷



 蕭吉注或題曹、蕭注



 《孫子》一卷



 賈林注《孫子》一卷



 陳皞注《孫子》一卷



 宋奇《孫子解》並《武經簡
 要》二卷



 吳章注《司馬穰苴兵法》三卷



 吳起《玉帳陰符》三卷



 白起《陣書一作「圖」》一卷



 又《神妙行軍法》三卷



 《戰國策》三十三卷



 黃石公《神光輔星秘訣》一卷



 又《兵法》一卷



 《三鑒圖》一卷



 《兵書統要》三卷



 《三略秘要》三卷



 成氏注《三略》三卷



 諸葛亮《行兵法》五卷



 又《用兵法》一卷



 《行軍指掌》二卷



 《占風雲氣圖》一卷



 《兵書》七卷



 陶侃《六軍鑒要》一卷



 李靖《韜鈐秘術》一卷



 又《總要》三卷



 《六十甲子厭勝法》一卷



 《兵書》三卷



 《占風雪一作「雲」氣》三卷



 《風云論》三卷



 《三軍水鑒》三卷



 《用兵手訣》七卷



 《
 出軍占風氣候》十卷



 《衛國公手記》一卷



 李世績《六十甲子內外行兵法》一卷



 李淳風《諸家秘要》三卷



 又《行軍明時秘訣》一卷



 《太白華蓋法》二卷



 《雲氣營寨占》一卷



 《行軍歷》一卷



 李筌《通幽鬼訣》二卷



 又《軍旅一作「放」指歸》三卷



 《北帝武威經》三卷



 《青囊托守勝敗歌》並《營野戰》一卷



 李光弼《將律》一卷



 又《武記》一卷



 《九天察氣訣》三卷



 《玄女厭陳法》一卷



 又《兵要式》一卷



 《行兵法》二卷



 《兵法》一卷



 《雜占法》一卷



 《六甲陰符兵法》一卷



 《軍謀前鑒》十卷



 《兵家正書》
 十卷



 《閫外紀一作「記」事》十卷



 李氏《秘要兵書》二卷



 又《秘要兵術》四卷



 《對敵權變逆順法》一卷



 《佐國玄機》一卷



 《炮經》一卷



 《總戎志》二卷



 李鼎祚《兵鈐手歷》一卷



 《許子兵勝苑》十卷



 《統軍玉鑒錄》一卷



 張守一《鑿門詩》一卷



 《秘寶興軍集》一卷



 胡萬頃《軍鑒式》二卷



 王適《行軍立成七十二局》一卷



 《安營臨陣觀災氣》一卷



 《決戰勝負圖》一卷



 《風雲氣象備急占》一卷



 《秘寶風雲歌》一卷



 《九宮軍要秘術》一卷



 《倚馬立成鑒圖》一卷



 《出軍占怪歷》三卷



 羅子岊一作「岊」



 《神機
 武略歌》三卷



 易靜《神機武略歌》一卷



 《行軍占風氣》一卷



 《軍占要略》二卷



 鄭先忠《軍機討略策》三卷



 《古今兵略》十卷



 《論天鏡一作「鑒」



 出戰要訣》一卷



 《將兵秘要法》一卷



 《武師左領記》二卷



 牛洪道《玄機立成法》一卷



 《孤虛明堂圖》一卷



 《軍用立成》一卷



 何延錫《辨解序》一卷



 紀勛《軍國要制》五卷



 《將軍總錄》三卷



 李遠《武孝經》一卷



 王巒《行軍廣要算經》三卷



 《金符經》三卷



 《十二月立成陣圖法》一卷



 《行軍走馬立成法》一卷



 《立成掌中法》一卷



 《行軍要略分野星
 圖法》一卷



 《黃道法》一卷



 徐漢卿《制勝略》三卷



 牟知白《專征小格略》一卷



 《圖南兵略》三卷



 《從征錄》五卷



 《出軍別錄》一卷



 《兵書總要》四卷



 《兵策秘訣》三卷



 《萬勝訣》二卷



 《戰鬥秘訣》一卷



 《英雄龜鑒》一卷



 《兵訣》一卷



 《隨軍要訣》一卷



 《軍謀要術》一卷



 《韜鈐秘要》一卷



 《軍旅要術》一卷



 《軍秘禳厭術》一卷



 《占軍機勝負龜訣》一卷



 《訓將勝術》二卷



 《兵書手鑒》二卷



 《尉繚子》五卷



 戰國時人



 《常禳經》一卷



 黃石公《三略》三卷



 又《素書》一卷張良所傳



 諸葛亮《將苑》一卷



 《兵
 書手訣》一卷



 《文武奇編》一卷



 《武侯八陣圖》一卷



 《鬼谷天甲兵書常禳術》三卷梁昭明太子撰



 陶弘景《真人水照》十三卷



 李靖《六軍鏡》三卷



 《六十甲子禳敵克應決勝術》一卷



 《玉帳經》一卷



 《李靖兵鈐新書》一卷



 並不知作者



 《九天玄女孤虛法》一卷



 李淳風《懸鏡經》十卷



 《郭代公安邊策》三卷唐郭震撰



 李筌《太白陰經》十卷



 《占五行星度吉兇訣》一卷



 注《孫子》一卷



 《閫外春秋》十卷



 李光弼《統軍靈轄秘策》一卷



 《五家注孫子》三卷魏武帝、杜牧、陳皞、賈隱林、孟氏



 杜牧《孫子注》三卷



 裴緒《新令》
 二卷



 曹、杜注《孫子》三卷曹操、杜牧



 劉玄之《行軍月令》一卷



 李大著《江東經略》十卷



 《綦先生兵書》一十六卷



 並不知名



 許洞《虎鈐兵經》二十卷



 樂產《太一王佐秘珠》五卷



 盧元《韜珠秘訣》一卷



 《黃帝太公兵法》三卷虞彥行進



 趙善譽《南北攻守類考》六十三卷



 柴叔達《浮光戰守錄》一卷



 《沖晦郭氏兵學》七卷郭雍述



 《論五府形勝萬言書》一卷



 《閫外策鈐》五卷



 《經武略》二百九十卷



 《治亂貫怪記》三卷



 《三賢安邊策》十一卷



 《邊防備衛策》一卷



 《出軍占候歌》一卷



 《通玄玉鑒》一
 卷



 《握鏡訣怪祥歌》一卷



 《玄女遁甲經》三卷



 《李僕射馬前訣》一卷



 《防城動用》一卷



 《彭門玉帳訣錄》一卷



 《遁甲專征賦》一卷



 《帝王中樞賦》二卷



 《長世論》十卷



 《武備圖》一卷



 《兵鑒》五卷



 《陰符握機運宜要》五卷



 並不知作者



 仁宗《攻守圖術》三卷



 曾公亮《武經總要》四十卷



 蔡挺《裕陵邊機處分》一卷



 符彥卿《人事軍律》三卷



 曾致堯《清邊前要》十卷



 王洙《三朝經武聖略》十卷



 《清邊武略》十五卷



 《風角占》一卷康定間司天臺集



 任鎮《康定論兵》一卷



 趙珣《聚米圖經》五卷



 《慶歷軍
 錄》一卷不知作者。



 曾公奭《軍政備覽》一卷



 耿恭《平戎議》三卷



 《邊臣要略》二十卷



 趙瑜《安邊致勝策》三卷



 呂夏卿《兵志》三卷



 丘浚《征蠻議》一卷



 阮逸《野言》一卷



 劉水扈《備邊機要》一卷



 薛向《陜西建明》一卷



 吉天保《十家孫子會注》十五卷



 王韶《熙河陣法》一卷



 韓縝《元豐清野備敵》一卷



 何去非《三備略講義》六卷



 《備論》十四卷



 戴溪《歷代將鑒博議》十卷



 張文伯《百將新書》十二卷



 劉溫潤《西夏須知》一卷



 王維清《武昌要訣》一卷



 徐矸《司命兵機秘略》二十八卷



 徐確《總夫要錄》一卷



 張預《集注百將傳》一百卷



 餘壹《兵籌類要》十五卷



 葉上達《神武秘略論》十卷



 夏休《兵法》三卷



 汪槔《進復府兵議》一卷



 《古今屯田總議》七卷



 游師雄《元祐分疆錄》二卷



 《崇寧邊略》三卷不知作者



 劉荀《建炎德安守禦錄》三卷



 度濟《兵錄》八十卷



 《西齋兵議》三卷文覺兄弟問答兵機



 章穎《四將傳》三卷



 《神機靈秘圖》三卷



 《軍鑒圖》二卷



 《紀重政軍機決勝立成圖》三卷



 《兵書氣候旗勢圖》一卷



 《諸家兵法秘訣》四卷



 《行師類要》七卷



 《古今兵書》十卷



 《五行
 陣決》一卷



 《會稽兵術》三卷



 《六十甲子出軍秘訣一作「略」》一卷



 《玄珠要訣》一卷



 《軍墊兵鈐》三卷



 《韜鈐秘錄》五卷



 《將略兵機論》十二卷



 《三軍指要》五卷



 《纂下六甲營圖》一卷



 《五十七陣出軍甲子》一卷



 《行軍玄機百術法》一卷



 《兵書出軍雜占》五卷



 《兵法機要》三卷



 《神兵要術》三卷



 《神兵機要》三卷



 《總戎策》二卷



 《兵書精訣》二卷



 《權經對》三卷



 《行軍用兵玄機》三卷



 《兵機要論》五卷



 《行軍備歷》六十卷



 《兵機簡要》十卷



 《兵談》三卷



 韓霸《水陸陣圖》三卷



 《強弩備術》三卷



 《
 九九陣圖》一卷



 《軍林要覽》三卷



 《制勝權略》三卷



 《兵書精妙玄術》十卷



 《兵籍要樞》三卷



 《太一行軍秘術詩》三卷



 《戎機》二卷



 《通神機要》三卷



 劉玄一作「定」之《兵家月令》一卷



 又《軍令備急》一卷



 湯渭《天一兵機舉要歌》一卷



 王洪暉《行軍月令》四卷



 裴守一《軍誡》三卷



 《兵家正史》九卷



 《行軍周易占》一卷



 張從實《將律》一卷



 焦大憲《兵易歌神兵苑》三卷



 《星度用》一卷



 《將術》一卷



 《行兵攻具術》一卷



 《行兵攻具圖》一卷



 《兵家秘寶》一卷



 《秘寶書》一卷



 《軍律》三卷



 張昭《制
 旨兵法》十卷



 王洙《青囊括》一卷



 杜希全《兵書要訣》三卷



 釋利正《長慶人事軍律》三卷



 董承祖《至德元寶玉函經》十卷



 王公亮《行師類要》七卷



 劉可久《契神經》一卷



 李洿《靈關訣》二卷一名《靈關集益智》



 《兵機法》一卷



 《太一厭禳法》一卷



 《五行陣圖》一卷



 《兵論》十卷



 《六十甲子行軍法》一卷



 《會稽兵家術日月占》一卷



 《統戎式令》一卷



 《六甲五神用軍法》一卷



 《要訣兵法立成歌》一卷



 《六甲攻城破敵法》一卷



 《馬前秘訣兵書》一卷



 石普《軍儀條目》三卷



 仁宗《神武秘略》
 十卷



 又《行軍環珠》一部卷亡



 又《四路獸守約束》一部卷亡



 《軍誡》一卷



 《武記》一卷



 《定遠安邊策》三卷



 《新集兵書要訣》三卷



 《兵書要略》一卷



 《揀將要略》十卷



 《兵論》十卷



 符彥卿《五行陣圖》一卷



 《新集行軍月令》四卷



 《雲氣圖》十二卷



 《統戎式鏡》二卷



 《行軍氣候秘法》三卷



 《天子氣章云氣圖》十二卷



 《預知歌》三卷



 《從軍占》三卷



 《兵書論語》三卷



 《彭門玉帳歌》三卷



 《太一行軍六十甲子禳厭秘術詩》三卷



 《兵機興要陽謂歌》一卷



 郯子《新修六壬大玉帳歌》十卷



 郭固《軍
 機決勝立成圖》一卷



 又《兵法攻守圖術》三卷



 王存《樞密院諸房例冊》一百四十二卷



 蔡挺《教閱陣圖》一卷



 林廣《陣法》一卷



 王拱辰《平蠻雜議》十卷



 《敵樓馬面法式及申明條約並修城女墻法》二卷



 楊伋《兵法圖議》一卷



 韓縝《樞密院五房宣式》一卷



 又《論五府形勝萬言書》一卷



 方坰《重演握奇》三卷



 又《握奇陣圖》一卷



 梁燾《安南獻議文字並目錄》五卷



 《愈見御戎》十冊



 韓絳《宣撫經制錄》三卷



 王革《政和營繕軍補錄序》一卷



 餘臺《兵籌類要》十五卷



 《溱
 播州勝兵法》二部



 任諒《兵書》十卷



 右兵書類三百四十七部,一千九百五十六卷。



 李廣《射評要錄》一卷



 梁冀《彈棋經》一卷



 梁元帝《畫山水松石格》一卷



 姚最《續畫品》一卷



 李嗣真《畫後品》一卷



 竇蒙《畫錄拾遺》一卷



 張又新《畫總載》一卷



 裴孝源《貞觀公私畫錄》一卷



 李淳風《歷監天元主物簿》三卷



 皇甫松《醉鄉日月》三卷



 張彥遠《歷代名畫記》十卷



 韋蘊《九鏡射經》一卷



 《唐畫斷》一卷



 王琚《射經》一卷



 王堅道《射訣》一卷



 荊
 浩《筆法記》一卷



 《李氏墨經》一卷



 《張學士棋經》一卷



 宋景真《唐賢名畫錄》一卷



 《墨圖》一卷



 《釣鰲圖》一卷



 《端硯圖》一卷



 《畫總錄》五卷



 《嘯真》一卷



 《樗蒲圖》一卷



 並不知作者



 蘇易簡《文房四譜》五卷



 李永德《點頭文》一卷



 李畋《益州名畫錄》三卷



 唐績《硯圖譜》一卷



 紀但《廣弓經》一卷



 王德用《神射式》一卷



 劉懷德《射法》一卷



 趙景《小酒令》一卷



 趙明遠《皇宋進士彩選》一卷



 蔡襄《墨譜》一卷



 卜恕《投壺新律》一卷



 劉敞《漢官儀》三卷亦投子選也



 唐詢《硯錄》二卷



 竇譝《飲戲助歡》
 三卷



 郭若虛《圖畫見聞志》六卷



 司馬光《投壺新格》一卷



 王趯《投壺禮格》二卷



 劉道醇《新編五代名畫記》一卷



 《宋朝畫評》四卷



 李誡《新集木書》一卷



 米芾《畫史》一卷



 任權《弓箭啟蒙》一卷



 張仲商《射訓》一卷



 馬思永《射訣》一卷



 王越石《射議》一卷



 李孝美《墨苑》三卷



 李薦《德隅堂畫品》一卷



 溫子融《畫鑒》三卷



 王慎修《宣和彩選》一卷



 陳日華《金淵利術》八卷



 黃鑄《玉簽詩》一卷



 李洪《續文房四譜》五卷



 何珪《射義提要》一卷



 《射經》三卷



 張仲素《射經》三卷



 田逸《
 射經》四卷



 王琚《射經》二卷



 《九鑒射圖》一卷



 徐鍇《射書》十五卷



 韋蘊《射訣》一卷



 李章《射訣》三卷



 張子霄《神射訣》一卷



 李靖《弓訣》一卷



 《法射指訣》一卷



 黃損《射法》一卷



 張守忠《射記》一卷



 《弓訣》一卷



 呂惠卿《弓試》一部卷亡



 上官儀《投壺經》一卷



 鐘離景伯《草書洪範無逸中庸韻譜》十卷



 唐績《棋圖》五卷



 《金穀園九局譜》一卷



 王積薪等《棋訣》三卷



 《棋勢論並圖》一卷



 徐鉉《棋圖義例》一卷



 《棋勢》三卷



 楊希□蔡一作「璨」



 《四聲角圖》一卷



 又《雙泉圖》一卷



 《玉溪圖》一卷



 蔣
 元吉等《棋勢》三卷



 太宗《棋圖》一卷



 《局譜》一卷



 韋□延《棋圖》一卷



 《弈棋經》一卷



 《棋經要略》一卷



 王子京《彈棋圖》一卷



 《樗蒲經》一卷



 《雙陸格》一卷



 李合《骰子彩選格》三卷



 劉蒙叟《彩選格》一卷



 《尋仙彩選》七卷



 《葉子格》三卷



 李煜妻周氏《系蒙小葉子格》一卷



 《偏金葉子格》一卷



 《小葉子例》一卷



 謝赫《古今畫品》一卷



 徐浩《畫品》一卷



 曹仲連《畫評》一卷



 李嗣真《畫後品》一卷



 胡嶠《廣梁朝畫目》三卷



 王叡《不絕筆畫圖》一卷



 郭若虛《圖畫見聞言志》
 六卷



 朱遵度《漆經》三卷



 《馬經》一卷



 《辨馬圖》一卷



 《馬口齒訣》一卷



 《醫馬經》一卷



 《明堂灸馬經》二卷



 《論駝經》一卷



 《療駝經》一卷



 《醫駝方》一卷



 右雜藝術類一百十六部,二百二十七卷。



 陸機《會要》一卷



 朱澹遠《語麗》十卷



 杜公瞻《編珠》四卷



 祖孝徵《修文殿御覽》三百六十卷



 歐陽詢《藝文類聚》一百卷



 虞世南《北堂書鈔》一百六十卷



 高士廉、房玄齡《文思博要》一卷



 徐堅《初學記》三十卷



 《燕公事對》十卷



 張鷟《龍
 筋鳳髓判》十卷



 杜祐《通典》二百卷



 陸贄《備舉文言》三十卷



 張仲素《詞圃》十卷



 白居易《白氏六帖》三十卷



 《前後六帖》三十卷前白居易撰,後宋孔傳撰



 李翰《蒙求》三卷



 白廷翰《唐蒙求》三卷



 劉綺莊《集類》一百卷



 李商隱《金鑰》二卷



 崔鉉《弘文館續會要》四十卷



 李途《記室新書》三卷



 顏休《文飛應韶》十五卷



 高測《韻對》十卷



 劉揚名《戚苑纂要》十卷



 又《戚苑英華》十卷



 孟詵《錦帶書》八卷



 喬舜封《古今語要》十二卷



 蘇冕《古今國典》一百卷



 又《會要》四十卷



 章得像《國朝會
 要》一百五十卷宋初至慶歷四年



 大孝一作「存」僚《御覽要略》十二卷



 《冊府元龜音義》一卷



 王欽若《彤管懿範》七十卷《目》十卷



 《彤管懿範音義》一卷



 歐陽詢《麟角》一百二十卷



 《白氏家傳記》二十卷



 薛高立《集類》三十卷



 《邊崖類聚》三十二卷



 《類事》十卷



 徐叔暘《羊頭山記》十卷



 於立政《類林》十卷



 杜光庭《歷代忠諫書》五卷



 《諫書》八十卷



 《唐諫諍論》十卷



 王昭遠《禁垣備對》十卷



 魏玄成《勵忠節》四卷



 王伯璵《勵忠節抄》十卷



 《書判幽燭》四十卷



 《軺車事類》三卷



 周祐之《
 五經資政》二十卷



 《經典政要》三卷



 尹弘遠《經史要覽》三十卷



 《章句纂類》十四卷



 李知實一作「寶」



 《檢志》三卷



 李慎微一作「征」



 《理樞》七卷



 鄒順《廣蒙書》十卷



 劉漸《群書系蒙》三卷



 《九經對語》十卷



 錢承志《九經簡要》十卷



 《經史事對》三十卷



 《子史語類拾遺》十卷



 韋稔《筆語類對》十卷



 又《應用類對》十卷一名《筆語類對》



 黃彬《經語協韻》二十卷



 朱澹《語類》五卷



 楊名《廣一作「唐」略新書》三卷



 《十議典錄》三卷



 李德孫《學堂要記一作「紀」》十卷



 裴說《修文異名錄》十一卷



 《搢紳要錄》二
 卷



 段景《文場纂要》二卷



 《文場秀句》一卷



 王云《文房纂要》十卷



 《雕玉集類》二十卷



 《雕金集》三卷



 劉國潤《廣雕金類集》十卷



 庾肩吾《彩璧》五卷



 《金鑾秀蕊》二十卷



 陸贄《青囊書》十卷



 《蔣氏寶車一作「庫」》十卷



 《瓊林摘實》三卷



 溫庭筠《學海》三十卷



 鄭□昺一作「嵎」《雙金》五卷



 孫翰《錦繡穀》五卷



 齊逸人《玉府新書》三卷



 《叢髓》三卷



 盧重華《文髓》一卷



 《勁弩子》三卷



 《玉苑麗文》五卷



 段景《疊辭》二卷



 《玉英》二卷



 《玉屑》二卷



 《金匱》二卷



 《常修半臂》十卷



 《紫香囊》二十卷



 陸羽《警
 年》十卷



 《窮神記》十卷



 李齊莊《事解》七卷



 《王氏千門》四十卷



 郭道規《事鑒》五十卷



 沈寥子《文鑒》四十卷



 李大華《康國集》四卷



 姚勖《起予集》四十卷



 李貴臣《家藏龜鑒錄》四卷



 徐德言《分史衡鑒》十卷



 薛洪《古今精義》十五卷



 《筆藏論》三卷



 蘇源《治亂集》三卷



 《治道要言》十卷



 馬幼昌《穿楊集》四卷



 李欽玄《累玉集》十卷



 支遷喬一作「奇」



 《京國記》二卷



 郭微《屬文寶海》一百卷



 樂黃目《學海搜奇錄》六十卷



 《皇覽總論》十卷



 張陟《唐年經略志》十卷



 楊九齡《名苑》五十
 卷



 晁光乂《十九書語類》十卷



 雍公叡注張楚金《翰苑》十一卷



 劉濟《九經類議一作「義」》二十卷



 黎翹《廣略》六卷



 王博古《修文海》十七卷



 郭翔《春秋義鑒》三十卷



 曹化《兩漢史海》十卷



 楊知惲《名字族》十卷



 馮洪敏《寶鑒絲綸》二十卷



 胡旦《將帥要略》二十卷



 劉顏《輔弼名對》四十卷



 景泰《邊臣要略》二十卷



 石待問《諫史》一百卷



 王純臣《青宮懿典》十五卷



 李虛一《溉漕新書》四十卷



 《童子洽聞》一卷



 《麟角抄》十二卷



 雷壽之《古文類纂》十卷



 《漢臣蒙求》二十卷



 李
 伉《系蒙求》十卷



 丘光庭《同姓名錄》一卷



 王殷範《續蒙求》三卷



 《王先生十七史蒙求》十六卷



 黃簡《文選韻粹》三十五卷



 白氏《玉連環》七卷



 白氏《隨求》一卷不知名



 《重廣會史》一百卷



 《資談》六十卷



 《聖賢事跡》三十卷



 《引證事類備用》三十卷



 《門類解題》十卷



 《瓊林會要》三十卷



 《青雲梯籍》二十卷



 《南史類要》二十卷



 《粹籍》十五卷



 《六朝摘要》十卷



 《十史事類》十卷



 《十史事語》十二卷



 《三傳分門事類》十二卷



 《嘉祐新編二經集粹》十卷



 《鹿革事類》二十卷



 《職官事對》
 九卷



 《掞天集》六卷



 《文章叢說》十卷



 《新編經史子集名卷》六卷



 《碎玉四淵海集》百九十五卷



 《書林》四卷



 《寶龜》三卷



 《離辭筆苑》二卷



 《詩句類》二卷



 《南北事偶》三卷



 《五色線》一卷



 《珠浦》一卷



 《重廣策府沿革》一卷



 《鴻都編》一卷



 《文章庫》一卷



 《十三代史選》三十卷



 《左傳類要》五卷



 《唐朝事類》十卷



 《群玉雜俎》三卷



 《增廣群玉雜俎》四卷



 《分聲類說》三十二卷



 《文選雙字類要》四十卷



 《書林事類》一百卷



 並不知作者



 鄭氏《歷代蒙求》一卷



 孫應符《初學須知》五卷



 王敦詩《書林
 韻會》二十八卷



 曾恬《孝類書》二卷



 邵笥《賡韻孝悌蒙求》二卷



 譙令憲《古今異偶》一百卷



 《宋六朝會要》三百卷章得像編,王珪續



 《續會要》三百卷虞允文等撰



 《中興會要》二百卷梁克家等撰



 《孝宗會要》二百卷楊濟、鐘必萬總修



 《光宗會要》一百卷



 《寧宗會要》一百五十卷秘書省進



 《國朝會要》五百八十八卷張從祖纂輯



 王溥《續唐會要》一百卷



 《五代會要》三十卷



 李安上《十史類要》十卷



 李昉《太平御覽》一千卷



 王倬《班史名物編》十卷



 蘇易簡《文選菁英》二十四卷



 宋白、李宗諤《續通典》
 二百卷



 皮文粲《鹿門家鈔籍詠》五十卷



 曾致堯《仙鳧羽翼》三十卷



 僧守能《典類》一百卷



 王欽若《冊府元龜》一千卷



 葉適《名臣事纂》九卷



 方龜年《群書新語》十一卷



 晏殊《天和殿御覽》四十卷



 《類要》七十七卷



 宋癢《雞跖集》二十卷



 過勖《至孝通神集》三十卷



 鄧至《群書故事》十五卷



 《故事類要》三十卷



 宋並《登瀛秘錄》八卷



 範鎮《本朝蒙求》二卷



 馬共《元祐學海》三十卷



 任廣《書敘指南》二十卷



 朱繪《事原》三十卷



 陳彥禧《黌堂要覽》十卷



 陳紹《重廣六帖
 學林》三十卷



 吳淑《事類賦》三十卷



 王資深《摭史》四卷



 馬永易《實賓錄》三十卷



 《異號錄》三十卷



 陳貽範《千題適變錄》十六卷



 楊言恣《古今名賢歌詩押韻》二十四卷



 江少虞《皇朝事實類苑》二十六卷



 葉庭珪《海錄碎事》二十三卷



 陳天麟《前漢六帖》十二卷



 蕭之美《十子奇對》三卷



 《莊子寓言類要》一卷



 《三傳合璧要覽》二卷



 《三子合璧要覽》二卷



 《四子合璧要覽》二卷



 劉玨《兩漢蒙求》十卷



 吳逢道《六言蒙求》六卷



 徐子光《補注蒙求》四卷



 又《補注蒙求》八卷



 《
 群書治要》十卷秘閣所錄



 蔡攸《政和會要》一百一十卷



 晏袤數《會要》一百卷



 謝諤《孝史》五十卷



 度濟《諫錄》二十卷



 葉才老《和李翰蒙求》三卷



 林越《漢雋》十卷



 倪遇《漢書家範》十卷



 李宗序《隆平政斷》二十卷



 鄭大中《漢規》四卷



 張磁《仕學規範》四十卷



 歐陽邦基《勸戒別錄》三卷



 閻一德《古今政事錄》二十一卷



 僧道蒙《仕途經史類對》十二卷



 呂祖謙《觀史類編》六卷



 《讀書記》四卷



 洪邁《經子法語》二十四卷



 《春秋左氏傳法語》六卷



 《史記法語》八卷



 《前漢法
 語》二十卷



 《後漢精語》十六卷



 《三國志精語》六卷



 《晉書精語》五卷



 《南史精語》六卷



 《唐書精語》一卷



 程大昌《演蕃露》十四卷



 又《續演蕃露》六卷



 《考古編》十卷



 《續考古編》十卷



 程俱《班左誨蒙》三卷



 唐仲友《帝王經世圖譜》十卷



 錢端禮《諸史提要》十五卷



 陳傅良《漢兵制》一卷



 《備邊十策》九卷



 徐天麟《西漢會要》七十卷



 《漢兵本末》一卷



 曾綎《類說》五十卷



 錢文子《補漢兵志》一卷



 錢諷《史韻》四十二卷



 鄒應龍《務學須知》二卷



 高似孫《緯略》十二卷



 《子略》四卷



 吳
 曾《南北分門事類》十二卷



 魏彥惇《名臣四科事實》十四卷



 王掄《群玉義府》五十四卷



 範師道《垂拱元龜會要詳節》四十卷



 《國朝類要》十二卷



 俞鼎、俞經《儒學警悟》四十卷



 鄭厚《通鑒分六類要》四十卷



 柳正夫《西漢蒙求》一卷



 李孝美《文房監古》三卷



 竇蘋《載籍討源》一卷



 王仲閎《語本》二十五卷



 毛友《左傳類對賦》六卷



 俞觀能《孝經類鑒》七卷



 胡宏《敘古蒙求》一卷



 《玉山題府》二十卷



 《熙寧題髓》十五卷



 《帝王事實》十卷



 《聖賢事實》十卷



 《漢唐事實》十五
 卷



 《國朝韻對》八卷



 《引證事類》三十卷



 《魯史分門屬類賦》一卷



 《古今通編》八卷



 《諸子談論》三卷



 並不知作者



 右類事類三百七部,一萬一千三百九十三卷。



 《黃帝內經素問》二十四卷唐王冰注



 《素問》八卷隋全元起注



 《黃帝靈樞經》九卷



 《黃帝針經》九卷



 《黃帝灸經明堂》三卷



 《黃帝九虛內經》五卷



 揚玄操《素問釋音一作「言」》一卷



 《素問醫療訣》一卷



 秦越人《難經疏》十三卷



 《黃帝脈經》一卷



 又《脈訣》一卷



 張仲景《脈經》一卷



 又《五藏榮衛論》一卷



 《耆婆脈經》
 三卷



 徐氏《脈經》三卷



 王叔和《脈訣一作「經」》一卷



 《孩子脈論》一卷



 李績《脈經》一卷



 張及《脈經手訣》一卷王善注



 徐裔《脈訣》二卷



 《韓氏脈訣》一卷



 《脈經》一卷



 《百會要訣脈經》一卷



 《碎金脈訣》一卷



 《元門脈訣》一卷



 《身經要集》一卷



 《太醫秘訣診候生死部》一卷



 《倉公決死生秘要》一卷



 《神農五藏論》一卷



 《黃帝五藏論》一卷



 《黃庭五藏經》一卷



 《黃庭五藏六府圖》一卷



 趙業《黃庭五藏論》七卷



 張向容《大五藏論》一卷



 又《小五藏論》一卷



 《五藏金鑒論》一卷



 段元一作「允」



 亮《五藏鑒元一作「原」》四卷



 孫思邈《五藏旁通明鑒圖》一卷



 又《針經》一卷



 張文懿《藏府通玄賦》一卷



 《五藏攝養明鑒圖》一卷



 吳兢《五藏論應像》一卷



 裴王庭《五色旁通五藏圖》一卷



 《五藏要訣》一卷



 《太元心論》一卷



 岐伯《針經》一卷



 扁鵲《針傳》一卷



 玄悟《四神針經》一卷



 甄權《針經抄》三卷



 王處明《玄秘會要針經》五卷



 呂博《金縢玉匱針經》三卷



 《黃帝問岐伯灸經》一卷



 顏齊《灸經》十卷



 《明堂灸法》三卷



 皇甫謐《黃帝三部針灸經》十二卷即《甲乙經》



 岐伯《論針灸
 要訣》一卷



 《山眺一作「兆」



 針灸經》一卷



 公孫克《針灸經》一卷



 吳復珪《小兒明堂針灸經》一卷



 王惟一《明堂經》三卷



 《明堂玄真經訣》一卷



 朱遂《明堂論》一卷



 《金鑒集歌》一卷



 《黃帝太素經》三卷楊上善注



 《刺法》一卷



 《太上天寶金鏡靈樞神景內編》九卷



 《扁鵲注黃帝八十一難經》二卷秦越人撰



 扁鵲《脈經》一卷



 張仲景《傷寒論》十卷



 《五藏論》一卷



 王叔和《脈經》十卷



 《脈訣機要》三卷



 巢元方《巢氏諸病源候論》五十卷



 崔知悌《灸勞法》一卷



 王冰《素問六脈玄珠密語》一卷



 褚澄《褚氏遺書》一卷



 華佗《藥方》一卷



 《金匱要略方》三卷張仲景撰,王叔和集



 葛洪《肘後備急百一方》三卷



 劉涓子《神仙遺論》十卷東蜀李頓錄



 宇文士及《□莊臺記》六卷



 師巫《顱經》二卷



 孫思邈《千金方》三十卷



 《千金髓方》二十卷



 《千金翼方》三十卷



 《玉函方》三卷



 王起《仙人水鏡》一卷



 王燾《外臺秘方》四十卷



 陳藏器《本草拾遺》十卷



 孔志約《唐本草》二十卷



 李昉《開寶本草》二十卷《目》一卷



 盧多遜《詳定本草》二十卷《目錄》一卷



 《補注本草》二十卷《目錄》一卷



 李含光《本
 草音義》五卷



 蕭炳《四聲本草》四卷



 《本草韻略》五卷



 楊損之《刪繁本草》五卷



 杜善方《本草性類》一卷



 陳士良《食性本草》十卷



 龐安時《難經解義》一卷



 宋庭臣《黃帝八十一難經注釋》一卷



 張仲景《療黃經》一卷



 又《口齒論》一卷



 《金匱玉函》八卷王叔和集



 扁鵲《療黃經》三卷



 又《枕中秘訣》三卷



 青鳥子《風經》一卷



 吳希言《風論山兆一作「眺」經》一卷



 支義方《通玄經》十卷



 呂廣《金韜玉鑒經》三卷



 《雷一作「靈」



 公仙人養性治一作「理」



 身經》三卷



 《醫源兆經》一卷



 林億《黃帝三部
 針灸經》十二卷



 楊曄《膳夫經手錄》四卷



 《延年秘錄》十一卷



 《混俗頤生錄》二卷



 《千金纂錄》二卷



 《金匱錄》五卷



 司空輿《發焰錄》一卷



 梅崇獻《醫門秘錄》五卷



 《治風經心錄》五卷



 郭仁普《拾遺候用深靈玄錄》五卷



 《養性要錄》一卷



 黨求平《摭醫新說》三卷



 代榮《醫鑒》一卷



 衛嵩《金寶鑒》三卷



 段元亮《病源手鑒》二卷



 田誼卿《傷寒手鑒》三卷



 《千金手鑒》二十卷



 王勃《醫語纂要》一卷



 華顒《醫門簡要》十卷



 蘇越《群方秘要一作「會」》三卷



 古詵《三教保光纂要》三卷



 《醫明
 要略》一卷



 張叔和《新集病總要略》一卷



 《外臺要略》十卷



 司馬光《醫問》七卷



 《耆婆六十四問》一卷



 伏氏《醫苑》一卷



 《神農食忌》一卷



 吳群《意醫紀歷》一卷



 孔周南《靈方志》一卷



 穆修靖《靈芝記》五卷羅公遠注



 張隱居《金石靈臺記》一卷



 《菖蒲傳》一卷



 李翱《何首烏傳》一卷



 張尚容《延齡至寶抄》一卷



 《醫家要抄》五卷



 《黃帝問答疾狀》一卷



 陶隱居《靈奇秘奧》一卷



 《南海藥譜》一卷



 《家寶義囊》一卷



 《小兒藥證》一卷



 《神仙玉芝圖》二卷



 《經食草木法》一卷



 孫思邈《芝草圖》
 三十卷



 又《太常分藥格》一卷



 《神枕方》一卷



 《崔氏產鑒圖》一卷



 《攝生月令圖》一卷



 《六氣導引圖》一卷



 《侍膳圖》一卷



 徐玉《藥對》二卷



 宗令祺《廣藥對》三卷



 《方書藥類》三卷



 江承宗《刪繁藥脈》三卷



 蔣淮《療黃歌》一卷



 晏封《草石論》六卷



 《藥性論》四卷



 張果《傷寒論》一卷



 陳昌祚《明時政要傷寒論》三卷



 李涉《傷寒方論》二十卷



 《青烏子論》一卷



 石昌璉《明醫顯微論》一卷



 清溪子《消渴論》一卷



 《龍樹眼論》一卷



 邢一作「邾」



 元樸《癰疽論》一卷



 《癰疽論》三卷



 李言少《嬰
 孺病論》一卷



 楊全迪《崔氏小兒論》一卷



 《療小兒疳病論》一卷



 劉豹子《眼論》一卷



 蘇巉一作「游」



 《玄感論》一卷



 李暄《嶺南腳氣論》二卷



 《發背論》二卷



 邵英俊《口齒論》一卷



 蕭一作「蘭」



 宗簡《水氣論》三卷



 《骨蒸論》一卷



 唐一作「廣」陵正師《口齒論》一卷



 《風疾論》一卷



 楊太業《三十六種風論》一卷



 喻義《瘡腫論》一卷



 又《療癰疽要訣》一卷



 蘇游《鐵粉論》一卷



 又《玄感傳尸方》一卷



 褚知義《鐘乳論》一卷



 李昭明《嵩臺論》三卷



 《玉鑒論》五卷



 王守愚《產前產後論》一卷



 《小兒眼論》
 一卷



 《普濟方》五卷



 《應驗方》三卷



 《應病神通方》三卷



 張文仲《法象論》一卷



 《小兒五疳二十四候論》一卷



 劉涓子《鬼論》一卷



 僧智宣《發背論》一卷



 沉泰之《癰疽論》二卷



 蘇敬徐玉唐侍中《三家腳氣論》一卷



 吳升宋處《新修鐘乳論》一卷



 白岑《發背論》一卷



 西京巢氏《水氣論》一卷



 李越一作「鉞」



 《新修榮衛養生用藥補瀉論》十卷



 楊大鄴《嬰兒論》二卷



 《摘藥論》一卷



 《制藥論法》一卷



 《連方五藏論》一卷



 《五勞論》一卷



 《夭壽性術論》一卷



 《咽喉口齒方論》五卷



 《產後十
 九論》一卷



 《小兒方術論》一卷



 李溫《萬病拾遺》三卷



 張機《金石制藥法》一卷



 王氏《食法》五卷



 嚴龜《食法》十卷



 《養身食法》三卷



 《太清服食藥法》七卷



 《按摩法》一卷



 《攝養禁忌法》一卷



 王道中《石藥異名要訣》一卷



 譚延鎬《脈色要訣》一卷



 吳復圭《金匱指微訣》一卷



 葉傳古《醫門指要訣》一卷



 華子顒《相色經妙訣》一卷



 《制藥總訣》一卷



 《修玉粉丹口訣》一卷



 《服雲母粉訣》一卷



 《伏火丹砂訣序》一卷



 陳玄《北京要術》一卷



 蕭家《法饌》三卷



 《饌林》四卷



 《藥林》一卷



 王
 氏《醫門集》二十卷



 李崇慶《燕臺集》五卷



 《穿玉集》一卷



 劉翰《今體治世集》三十卷



 雷繼暉《神聖集》三卷



 《華氏集》十卷



 《楊氏□莊臺寶鑒集》三卷南陽公主



 《傷寒證辨集》一卷



 賈黃中《神醫普救方》一千卷《目》十卷



 楊歸一作「師」



 厚《產乳集驗方》三卷



 安文恢《萬全一作「金」方》三卷



 孫廉《金鑒方》三卷



 《金匱方》三卷



 韋宙《獨行方》十二卷



 又《玉壺備急方》一卷



 鄭氏《惠民方》三卷



 鄭氏《圃田通玄方》三卷



 又《惠心方》三卷



 《纂要秘要方》三卷



 《溥濟安眾方》三卷



 支觀《通玄方》十卷



 劉氏《五藏旁通遵一作「導」養方》一卷



 白仁敘《集驗方》五卷



 《服食導養方》三卷



 孟氏《補養方》三卷



 崔元亮《海上集驗方》十卷



 崔氏《骨蒸方》三卷



 元希聲《行要備急方》二卷



 劉禹錫《傳信方》二卷



 王顏《續傳信方》十卷



 《嬰孩方》十卷



 黃漢忠《秘要合煉方》五卷



 《針眼一作「眼針」



 鉤方》一卷



 穆昌緒一作「叔」



 《療眼諸方》一卷



 《孩孺一作「嬰孩」



 雜病方》五卷



 朱傅《孩孺明珠變蒸七疳方》一卷



 《小兒秘錄集要方》一卷



 《延齡秘寶方集》五卷



 《錄古今服食導養方》三卷



 《
 服食神秘方》一卷



 姚和《眾童延齡至寶方》十卷



 又《保童方》一卷



 許詠一作「泳」



 《六十四問秘要方》一卷



 《塞上方》三卷



 《晨昏寧待方》二卷



 王道《外臺秘要乳石方》二卷



 《耆婆要用方》一卷



 崔行功《纂要方》十卷



 《千金秘要備急方》一卷



 華宗壽《升天一作「元」廣濟方》三卷



 段詠一作「泳」



 《走馬備急方》一卷



 《天寶神驗藥方》一卷



 《貞元集要廣利方》五卷



 《大和濟安方》一卷



 羅普宣《靈寶方》一百卷



 悟玄子《安神養性方》一卷



 《篋中方》一卷



 蕭存禮《百一問答方》三卷



 包會《應驗方》三卷



 《雜用藥
 方》五十五卷



 《神仙雲母粉方》一卷



 《服術方》一卷



 《慶歷善救方》一卷



 《胡道洽方》一卷



 賈耽《備急單方》一卷



 《李八百方》一卷



 波駝波利譯《吞字貼腫方》一卷



 李繼皋《南行方》三卷



 杜氏《集驗方》一卷



 韓待詔《肘後方》一卷



 王氏《秘方》五卷



 徒都子《膜外氣方》一卷



 潛真子《神仙金匱服食方》二卷



 楊太僕《醫方》一卷



 沈承澤《集妙方》三卷



 章秀言《草木諸藥單方》一卷



 吳希言《醫門括源方》一卷



 王朝昌《新集方》一卷



 《老子服食方》一卷



 《葛仙公杏仁煎方》一卷



 《刪
 繁要略方》一卷



 《集諸要妙方》一卷



 《備急簡要方》一卷



 《纂驗方》一卷



 《養性益壽備急方》一卷



 《奏聞單方》一卷



 《反魂丹方》一卷



 《玄明粉方》一卷



 《瘰□方》一卷



 《婆羅門僧服仙茅方》一卷



 高福《攝生要錄》三卷



 李絳《兵部手集方》三卷



 孟詵《食療本草》六卷



 沈知言《通玄秘術》三卷



 昝殷《產寶》三卷



 《食醫心鑒》二卷



 甘伯宗《歷代名醫錄》七卷



 鄭景岫《廣南四時攝生論》一卷



 葉長文《啟玄子元和紀用經》一卷



 張文懿《本草括要詩》三卷



 雷學《炮灸方》三卷



 宋徽宗《
 聖濟經》十卷



 通真子《續注脈賦》一卷



 《脈要新括》二卷



 李大參《家傷寒指南論》一卷



 嚴器之《傷寒明理論》四卷



 王維一《新鑄銅人腧穴灸圖經》三卷



 高若訥《素問誤文闕義》一卷



 《傷寒類要》四卷



 徐夢符《外科灸法論粹新書》一卷



 趙從古《六甲天元運氣鈐》二卷



 丁德用《醫傷寒慈濟集》三卷



 馬昌運《黃帝素問入試秘寶》七卷



 王宗正《難經疏義》二卷



 楊介存《四時傷寒總病論》六卷



 僧文宥《必效方》三卷



 陳師文《校正太平惠民和劑局方》五卷



 陳氏《
 經驗方》五卷不知名



 唐慎微《大觀經史證類備急本草》三十二卷



 王寔《傷寒證治》三卷



 又《局方續添傷寒證治》一卷



 郭稽中《婦人產育保慶集》一卷



 裴宗元《藥詮總辨》三卷



 孫用和《傳家秘寶方》五卷



 錢乙《小兒藥證直訣》八卷



 洪氏《集驗方》五卷不知名



 李石《司牧安驥集》三卷



 又《司牧安驥方》一卷



 張渙《小兒醫方妙選》三卷



 王俁《編類本草單方》三十五卷



 趙鑄《癉虐備急方》一卷



 李璆、張致遠《瘴論》二卷



 鄭樵《鶴頂方》二十四卷



 《本草外類》五卷



 《食鑒》四
 卷



 張傑《子母秘錄》十卷



 王蘧《經效癰疽方》一卷



 張銳《雞峰備急方》一卷



 王世臣《傷寒救俗方》一卷



 胡權《治癰疽膿毒方》一卷



 錢竿《海上名方》一卷



 何偁《經驗藥方》二卷



 劉元寶《神巧萬全方》十二卷



 黨永年《摭醫新說》三卷



 史源《治背瘡方》一卷



 王貺《濟世全生指迷方》三卷



 王袞《王氏博濟方》三卷



 王伯順《小兒方》三卷



 漢東王先生《小兒形證方》三卷



 胡愔《補瀉內景方》三卷



 棲真子《嬰孩寶鑒方》十卷



 蔣淮《藥證病源歌》五卷



 成無已《傷寒論》一卷



 朱旦《傷寒論方》一卷



 沈虞卿《衛生產科方》一卷



 沉柄《產乳十八論》卷亡



 《溫舍人方》一卷不知名



 黨禹錫《嘉祐本草》二十卷



 劉方明《幼幼新書》四十卷



 吳得夫《集驗方》七卷



 馬延之《馬氏錄驗方》一卷



 李朝正《備急總效方》四十卷



 陳言《三因病源方》六卷



 陳抃《手集備急經效方》一卷



 張允蹈《外科保安要用方》五卷



 《史載之方》二卷



 夏德懋《衛生十全方》十三卷



 陸游《陸氏續集驗方》二卷



 卓伯融《妙濟方》一卷



 胡元質《總效方》十卷



 王璆《百一選方》二十八卷



 朱
 端章《衛生家寶方》六卷



 又《衛生家寶產科方》八卷



 《衛生家寶小兒方》二卷



 《衛生家寶湯方》三卷



 楊倓《楊氏家藏方》二十卷



 許叔微《普濟本事方》十二卷



 胡氏《經驗方》五卷不著名



 《備用方》二卷岳州守臣編,不著名氏



 丘哲《備急效驗方》三卷



 宋霖《丹毒備急方》三卷



 黃環《備問方》二卷



 王磧《易簡方》一卷



 方導《方氏集要方》二卷



 王世明《濟世萬全方》一卷



 張松《究源方》五卷



 董大英《活幼悟神集》二十卷



 《安慶集》十卷



 曾孚先《保生護命集》一卷



 戴衍《尊生要
 訣》一卷



 定齋居士《五痔方》一卷



 李氏《癰疽方》一卷不知名



 《集效方》一卷



 《中興備急方》二卷



 《灸經背面相》二卷



 《神應針經要訣》一卷



 《伯樂針經》一卷



 《傷寒要法》一卷



 《蘭室寶鑒》二十卷



 《小兒秘要論》一卷



 《紹聖重集醫馬方》一卷



 《傳信適用方》一卷



 《治未病方》一卷



 《用藥須知》一卷



 《治發背惡瘡內補方》一卷



 《博濟嬰孩寶書》二十卷



 《川玉集》一卷



 《產後論》一卷



 沖和先生《口齒論》一卷



 《腳氣論》一卷



 《靈苑方》二十卷



 《秘寶方》二卷



 《古今秘傳必驗方》一卷



 《太醫西局濟世
 方》八卷



 《產科經真環中圖》一卷



 陳幵《醫鑒後傳》一卷



 陳蓬《天元秘演》十卷



 龐安時《難經解》一卷



 朱肱《內外二景圖》三卷



 《南陽活人書》二十卷



 席延賞《黃帝針經音義》一卷



 莊綽《膏肓腧穴灸法》一卷



 《華氏中藏經》一卷靈寶洞主探微真人撰



 劉溫舒《內經素問論奧》四卷



 劉清海《五藏類合賦》一卷



 《耆婆五藏論》一卷



 劉皓《眼論審的歌》一卷



 徐氏《黃帝脈經指下秘訣》一卷



 平堯卿《傷寒玉鑒新書》一卷



 《傷寒證類要略》二卷



 董常《南來保生回車論》一卷



 黃維《聖
 濟經解義》十卷



 東軒居士《衛濟寶書》一卷



 李檉《傷寒要旨》一卷



 《醫家妙語》一卷



 《小兒保生要方》三卷



 湯民望《嬰孩妙訣論》三卷



 伍起予《外科新書》一卷



 《癰疽方》一卷



 董汲《腳氣治法總要》一卷



 程迥《醫經正本書》一卷



 婁居中《食治通說》一卷



 蘇頌《校本草圖經》二十卷



 王懷隱《太平聖惠方》一百卷



 姚和《眾童子秘要論》三卷



 錢聞禮《錢氏傷寒百問方》一卷



 閻孝忠《重廣保生信效方》一卷



 劉甫《十全博救方》一卷



 周應《簡要濟眾方》五卷



 王素《經驗方》
 三卷



 張田《幼幼方》一卷



 劉彞《贛州正俗方》二卷



 李端願《簡驗方》一卷



 崔源《本草辨誤》一卷



 晏傅正《明效方》五卷



 葛懷敏《神效備急單方》一卷



 沈括《良方》十卷



 《蘇沈良方》十五卷沉括、蘇軾所著



 陳直《奉親養老書》一卷



 文彥博《藥準》一卷



 董汲《旅舍備要方》一卷



 初虞世《古今錄驗養生必用方》三卷



 龐安《驗方書》一卷



 《勝金方》一卷



 《王趙選秘方》二卷



 右醫書類五百九部,三千三百二十七卷。



 凡子類三千九百九十九部,二萬八千二百九十卷。
 □二
 十
 五史宋史·志表



\end{pinyinscope}