\article{志第五十六 禮六吉禮六}

\begin{pinyinscope}

 朝日夕月九宮貴神高禖大火壽星靈星風伯雨師司寒蠟七祀馬祖酺神



 朝日,夕月。慶歷,用羊、豕各二,籩豆十二,簠、簋、俎二。天禧
 初,太常禮院以監察御史王博文言,詳定:「準禮,春分朝日於東郊,秋分夕月於西郊。《國語》:『太採朝日,少採夕月。』又曰:『春朝朝日,秋夕夕月。』唐柳宗元論云:『夕之名者,朝拜之偶也。古者旦見曰朝,暮見曰夕。』按禮,秋分夕月。蓋其時晝夜平分,太陽當午而陰魄已生,遂行夕拜之祭以祀月。未前十刻,太官令率宰人割牲,未後三刻行禮,蓋是古禮以夕行朝祭之儀。又按禮云:從子至巳為陽,從午至亥為陰。參詳典禮,合於未後三刻行禮。」皇祐五
 年,定朝日壇,舊高七尺,東西六步一尺五寸;增為八尺,廣四丈,如唐《郊祀錄》。夕月壇與隋、唐制度不合,從舊則壇小,如唐則坎深。今定坎深三尺,廣四丈。壇高一尺,廣二丈。四方為陛,降入坎深,然後升壇。壇皆兩壝,壝皆二十五步。增大明、夜明壇山罍二,籩豆十二。禮生引司天監官分獻,上香,奠幣、爵,再拜。嘉祐加羊、豕各五。《五禮新儀》定二壇高廣、坎深如皇祐,無所改。中興同。



 太一九宮神位,在國門之東郊。壇之制,四陛外,西南又
 為一陛,曰坤道,俾行事者升降由之。其九宮神壇再成,第一成東西南北各百二十尺,再成東西南北各一百尺,俱高三尺。壇上置小壇九,每壇高一尺五寸,縱廣八尺,各相去一丈六尺。初用中祀,咸平中改為大祀,壇增兩壝,玉用兩圭有邸,藉用稿秸加褥如幣色,其御書祝禮如社稷。尋以封禪,別建九宮壇泰山下行宮之東,壇二成,成一尺,面各長五丈二尺,四陛及坤道各廣五丈。上九小壇,相去各八尺,四隅各留五尺。壇下兩壝,依大
 祠禮。及祀汾陰,亦遣使祀焉。自後親郊恭謝,皆遣官於本壇別祭。



 景祐二年,學士章得像等定司天監生於淵、役人單訓所請祀九宮太一依逐年飛移位次之法:「案卻良遇《九宮法》,有《飛棋立成圖》,每歲一移,推九州所主災福事。又唐術士蘇嘉慶始置九宮神壇,一成,高三尺,四陛。上依位次置九小壇:東南曰招搖,正東曰軒轅,東北曰太陰,正南曰天一,中央曰天符,北曰太一,西南曰攝提,正西曰咸池,西北曰青龍。五數為中,戴九履一,
 左三右七,二四為上,六八為下,符於遁甲,此則九宮定位。歲祭以四孟,隨歲改位行棋,謂之飛位。自乾元以後,止依本位祭之,遂不飛易,仍減冬、夏二祭。國朝因之。今於淵等所請,合天寶初祭之理,又合良遇《飛棋之圖》。然其法本術家,時祭之文經禮不載。議者或謂不必飛宮,若日月星辰躔次周流而祭有常所,此則定位之祀所當從也。若其推數於回復,候神於恍忽,因方弭沴,隨氣考祥,則飛位之文固可遵用。請依唐禮,遇祭九宮之時,遣司
 天監一員詣祠所,隨每年貴神飛棋之方,旋定祭位。仍自天聖己巳入歷,太一在一宮,歲進一位,飛棋巡行,周而復始。」詔可。慶歷儀,每坐籩、豆十二,簠、簋、俎二。皇祐增壇三成。又禮官言:「歲雩祀外,水旱稍久,皆遣官告天地、宗廟、社稷及諸寺觀、宮廟,九宮貴神今列大祀,亦宜準此。」



 熙寧四年,司天中官正周琮言:「《太一經》推算,七年甲寅歲,太一陽九、百六之數,復元之初。故《經》言:『太歲有陽九之災,太一有百六之厄,皆在入元之初終。』今陽九、百
 六當癸丑、甲寅歲,為災厄之會。然五福太一移入中都,可以消異為祥。竊詳五福太一,自國朝雍熙元年甲申歲,入東南巽宮時,修東太一宮。天聖七年己巳歲,五福太一入西南坤位,修西太一宮。請稽詳故事,崇建祠宇,迎之京師。」詔建中太一宮於集禧觀。十太一神,並用通天冠、絳紗袍。元豐中,太常博士何洵直言:「熙寧祀儀,九宮貴神祝文稱『嗣天子臣某』,以禮秩論之,當與社稷為比,請依祀儀為大祀。其祝版即依會昌故事及《開寶通
 禮》,書御名不稱臣。又近制,諸祠祭牲數,正配以全體解割,各用一牢,貴神九位悉是正坐,異壇別祝,尊為大祝,而共享二少牢,於腥熟之俎,骨體不備。謂宜每位一牢,凡九少牢。」詔下太常,修入祀儀。



 元祐七年,監察御史安鼎言:「按漢武帝始祠太一一位,唐天寶初兼祀八宮,謂之九宮貴神。漢祀太一,日用一犢,凡七日而止。唐祀類於天地。今春秋祀九宮太一,用羊、豕,其四立祭太一宮十神,皆無牲,以素饌加酒焉。再詳《星經》:太一一星在紫
 宮門右,天一之南,號曰天之貴神。其佐曰五帝,飛行諸方,躡三能以上下,以天極星其一明者為常居。主使十六神,知風雨、水旱、兵革、饑饉、疫疾、災害之事。《唐書》曰:『九宮貴神,實司水旱。太一掌十六神之法度,以輔人極。』《國朝會要》亦云:『天之尊神及十精、十六度,並主風雨。』由是觀之,十神太一、九宮太一與漢所祀太一共是一神。今十神皆用素饌,而九宮並薦羊豕,似非禮意。」詔禮官詳定:十神、九宮太一各有所主,即非一神,故自唐迄今皆
 用牲牢,別無祠壇用素食禮。遂依舊制。



 崇寧三年,太常博士羅畸言:「九宮諸神位,無禮神玉,惟有燔玉。竊謂宜用禮神玉,少仿其幣之色薦於神坐。」議禮局言:「先王制禮,用圭璧以祀日月星辰,所謂圭璧者,圭,其邸為璧,以取殺於上帝也。今九宮神皆星名,而其玉用兩圭有邸。夫兩圭有邸,祀地之玉,以祀星辰,非周禮也。乞改用圭璧以應古制。」



 《政和新儀》:「立春日祀東太一宮;立夏、季夏土王日祀中太一宮;立秋日祀西太一宮;立冬日祀中
 太一宮,宮之真室殿,五福太一在中,君基太一在東,太游太一在西,俱南向。延休殿,四神太一。承厘殿,臣基太一在東,西向,北上。凝祐殿,直符太一。臻福殿,民基太一在西,東向,北上。膺慶殿,小游太一在中,天一太一在東,地一太一在西。靈貺殿,太歲在中,太陰在西,俱南向。三皇、五方帝、日月、五星、二十八宿、十日、十二辰、天地水三官、五行、九宮、八卦、五岳、四海、四瀆、十二山神等,並為從祀。東、西太一宮準此。東太一宮大殿,五福太一在東,君
 基太一在西,俱南向。太游太一殿在大殿之北,南向。臣基太一殿在南,北向。小游太一、直符太一、四神太一殿在大殿之東,西向,北上。天一太一、民基太一、地一太一在大殿之西,東向,北上。西太一宮黃庭殿,五福在中,君基在東,太游在西;均福殿,小游在中,俱南向。延貺殿,天一在中,四神在南,臣基在北,俱西向。資祐殿,地一在中,四神在南,臣基在北,俱西向。資祐殿,地一在中,民基在南,直符在東北,俱東向。」九宮貴神壇三成,一成縱廣十四
 丈,再成縱廣十二丈,三成縱廣十丈,各高三尺。上依方位置小壇九,各高一尺五寸,縱廣八尺。四陛、坤道,兩壝,每壝二十五步,如舊制。



 紹興十一年,太常丞朱輅言:「九宮貴神所主風、雨、霜、雪、雹、疫,所系甚重,請舉行祀典。」太常寺主簿林大鼐亦言:「十神太一,九宮太一,皆天之貴神,國朝分為二,並為大祀。比一新太一宮,而九宮貴神尚寓屋而不壇。」乃詔臨安府於國城之東,建築九宮壇壝,其儀如祀上帝。其太一宮,初議者請即行宮之北隅
 建祠,後命禮官考典故,擇地建宮。十八年,宮成,御書其榜。十太一位於殿上,南面,西上。從祀,東廡九十有八,西廡九十有七,皆北上。孝宗受禪,又建本命殿,名曰崇禧。光宗又遷介福殿像於挾室,而名新殿曰崇福。



 高禖。初,仁宗未有嗣,景祐四年二月,以殿中侍御史張奎言,詔有司詳定。禮官以為:「《月令》雖可據,然《周官》闕其文,《漢志》郊祀不及禖祠,獨《枚皋傳》言『皇子禖祝』而已。後漢至江左概見其事,而儀典委曲,不可周知。惟高齊禖
 祀最顯,妃嬪參享,黷而不蠲,恐不足為後世法。唐明皇因舊《月令》,特存其事。開元定禮,已復不著。朝廷必欲行之,當築壇於南郊,春分之日以祀青帝,本《詩》『克禋以祓』之義也。配以伏羲、帝嚳,伏羲本始,嚳著祥也。以禖從祀,報古為禖之先也。以石為主,牲用太牢,樂以升歌,儀視先蠶,有司攝事,祝版所載,具言天子求嗣之意。乃以弓矢、弓韣致神前,祀已,與胙酒進內,以禮所御,使齋戒受之。仍歲令有司申請俟旨,命曰特祀。」即用其年春分,遣
 官致祭。為圜壇高九尺,廣二丈六尺,四陛,三壝,陛廣五尺,壝各二十五步。主用青石,長三尺八寸,用木生成之數,形準廟社主,植壇上稍北,露其首三寸。青玉、青幣,牲用牛一、羊一、豕一,如盧植之說。樂章、祀儀並準青帝,尊器、神坐如勾芒,唯受福不飲,回授中人為異。祀前一日,內侍請皇后宿齋於別寢,內臣引近侍宮嬪從。是日,量地設香案、褥位各二,重行,南向,於所齋之庭以望禖壇。又設褥位於香案北,重行。皇后服禕衣,褥位以緋。宮嬪
 服朝賀衣服,褥位以紫。祀日,有司行禮,以福酒、胙肉、弓矢、弓韣授內臣,奉至齋所,置弓矢等於箱,在香案東;福酒於坫,胙肉於俎,在香案西。內臣引宮嬪詣褥位,東上南向。乃請皇后行禮,導至褥位,皆再拜。導皇后詣香案位,上香三,請帶弓韣,受弓矢,轉授內臣置於箱,又再拜。內臣進胙,皇后受訖,轉授內臣。次進福酒,內臣曰:「請飲福。」飲訖,請再拜。乃解弓韣,內臣跪受,置於箱。導皇后歸東向褥位。又引宮嬪最高一人詣香案,上香二,帶弓韣,
 受弓矢,轉授左右,及飲福,解弓韣,如皇后儀,唯不進胙。又引以次宮嬪行禮,亦然。俟俱復位,內侍請皇后詣南向褥位,皆再拜退。是歲,宮中又置赤帝像以祈皇嗣。



 寶元二年,皇子生,遣參知政事王鬷以太牢報祠,準春分儀,惟不設弓矢、弓韣,著為常祀,遣兩制官攝事。慶歷三年,太常博士餘靖言:「皇帝嗣續未廣,不設弓矢、弓韣,非是。」詔仍如景祐之制。



 熙寧二年,皇子生,以太牢報祀高禖,惟不設弓矢,弓韣。既又從禮官言:「按祀儀,青帝壇廣
 四丈,高八尺。今祠高禖既以青帝為主,其壇高廣,請如青帝之制。又祀天以高禖配,今郊禖壇祀青帝於南郊,以伏羲、高辛配,復於壇下設高禖位,殊為爽誤。請準古郊禖,改祀上帝,以高禖配,改伏羲、高辛位為高禖,而徹壇下位。」詔:「高禖典禮仍舊,壇制如所議,改犢為角握牛,高禖祝版與配位並進書焉。」又言:「伏羲、高辛配,祝文並云『作主配神』。神無二主,伏羲既為主,其高辛祝文,請改云『配食於神』。」



 元祐三年,太常寺言:「祀儀,高禖壇上正位
 設青帝席,配位設伏羲、高辛氏席,壇下東南設高禖,從祀席正配位各六俎,實以羊、豕腥熟,高禖位四俎,實以牛腥熟。祀日,兵部、工部郎中奉羊、豕俎升壇,詣正配位。高禖位俎,則執事人奉焉。竊以青帝為所祀之主,而牲用羊豕;禖神因其嘉祥從祀,而牲反用牛,又牛俎執事者陳之,而羊、豕俎皆奉以郎官,輕重失當。請以三牲通行解割,正、配、眾祀位並用,皆以六曹郎官奉俎。今羊俎以兵部,豕俎以工部,牛俎請以戶部郎官。」



 《政和新儀》:春
 分祀高禖,以簡狄、姜嫄從祀,皇帝親祠,並如祈穀祀上帝儀。惟配位作《承安》之樂,而增簡狄、姜嫄位牛、羊、豕各一。紹興元年,太常少卿趙子畫言:「自車駕南巡,雖多故之餘,禮文難備,至於祓無子,祝多男,所以系萬方之心,蓋不可闕。乞自來歲之春,復行高禖之祀。」十七年,車駕親祀高禖,如政和之儀。



 大火之祀。康定初,南京鴻慶宮災,集賢校理胡宿請修其祀,而以閼伯配焉。禮官議:「閼伯為高辛火正,實居商
 丘,主祀大火。後世因之,祀為貴神,配火侑食,如周棄配稷、後士配社之比,下歷千載,遂為重祀。祖宗以來,郊祀上帝,而大辰已在從祀,閼伯之廟,每因赦文及春秋,委京司長吏致奠,咸秩之典,未始雲闕。然國家有天下之號實本於宋,五運之次,又感火德,宜因興王之地,商丘之舊,為壇兆祀大火,以閼伯配。建辰、建戌出內之月,內降祝版,留司長吏奉祭行事。」乃上壇制:高五尺,廣二丈,四陛,陛廣五尺,一壝,四面距壇各二十五步。位牌以黑
 漆朱書曰大火位,配位曰閼伯位。牲用羊、豕一,器準中祠。歲以三月、九月擇日,令南京長吏以下分三獻,州、縣官攝太祝、奉禮。慶歷,獻官有祭服。



 建中靖國元年又建陽德觀以祀熒惑。因翰林學士張康國言,天下崇寧觀並建火德真君殿,仍詔正殿以離明為名。太常博士羅畸請宜仿太一宮,遣官薦獻,或立壇於南郊,如祀靈星、壽星之儀。有司請以閼伯從祀離明殿,又請增閼伯位。按《春秋傳》曰:五行之官封為上公,祀為貴神。祝融,高辛氏之
 火正也;閼伯,陶唐氏之火正也。祝融既為上公,則閼伯亦當服上公袞冕九章之服。既又建熒惑壇於南郊赤帝壇壝外,令有司以時致祭,增用圭璧,火德、熒惑以閼伯配,俱南向。五方火精、神等為從祀。壇廣四丈,高七尺,四陛,兩壝,壝二十五步,從《新儀》所定。



 紹興三年,詔祀大火。太常寺言:「應天府祀大火,今道路未通,宜於行在春秋設位。」乾道五年,太常少卿林慄等言:「本寺已擇九月十四日,依旨設位,望祭應天府大火,以商丘宣明王配。
 二十一日內火,祀大辰,以閼伯配。大辰即大火,閼伯即商丘宣明王也。緣國朝以宋建號,以火紀德,推原發祥之所自,崇建商丘之祠,府曰應天,廟曰光德,加封王爵,錫謚宣明,所以追嚴者備矣。今有司旬日之間舉行二祭,一稱其號,一斥其名,義所未安。乞自今祀熒惑、大辰,其配位稱閼伯,祝文、位板並依應天府大火禮例,改稱宣明王,以稱國家崇奉火正之意。」



 諸星祠,有壽星、周伯、靈星之祭。大中祥符二年,翰林天
 文邢中和言:「景德中,周伯星出亢宿下。按《天文志》,角、亢為太山之根,果符上封之應。望於親郊日特置周伯星位於亢、宿間。」詔禮官與司天監定議,且言:「周伯星出氐三度,然亢、氐相去不遠,並鄭分。兗州,壽星之次,宜如中和奏,設位氐宿之間,以為永式。」景德三年,詔定壽星之祀。太常禮院言:「按《月令》:『八月,命有司享壽星於南郊。』《注》云:『秋分日,祭壽星於南郊。壽星,南極老人星也。』《爾雅》云:『壽星,角、亢也。』《注》云:『數起角、亢,列宿之長,故云壽星。』唐開
 元中,特置壽星壇,常以千秋節日祭老人星及角、亢七宿。請用祀靈星小祠禮,其壇亦如靈星壇制,築於南郊,以秋分日祭之。」



 元豐中,禮文所言:「時令秋分,享壽星於南郊。熙寧祀儀:於壇上設壽星一位,南向。又於壇下卯陛之南設角、亢、氐、房、心、尾、箕七位,東向。按《爾雅》所謂『壽星角、亢』,非此所謂秋分所享壽星也。今於壇下設角、亢位,以氐、房、心、尾、箕同祀,尤為無名。又按晉《天文志》:『老人一星在弧南,一日南極,常以秋分之旦見於丙,春分之
 夕沒於丁,見則治平,主壽昌,常以秋分候之南郊。」後漢於國都南郊立老人星廟,常以仲秋祀之,則壽星謂老人矣。請依後漢,於壇上設壽星一位,南向,祀老人星。其壇下七宿位不宜復設。」



 慶歷以立秋後辰日祀靈星,其壇東西丈三尺,南北丈二尺,壽星壇方丈八尺。皇祐定如唐制,二壇皆周八步四尺。其享禮,籩八,豆八,在神位前左右,重三行。俎二,在籩、豆外,簠、簋一,在二俎間。像尊二,在壇上東南隅,北向西上。七宿位各設籩一,豆一,在
 神位前左右。俎一,在籩、豆外,中設簠一、簋一,在俎左右。爵一,在神位正前。壺尊二,在神位右。光祿實以法酒。



 《政和新儀》改定:壇高三尺,東西袤丈三尺,南北袤丈二尺,四出陛,一壝,二十五步。初,乾興祀靈星,值屠牲有禁,乃屠於城外。至是,敕有司:「凡祭祀牲牢,無避禁日,著為令。」南渡後,靈星、壽星、風師、雨師、雷師及七祀、司寒、馬祖,並仍舊制。



 風伯、雨師,諸州亦致祭。大中祥符初,詔惟邊地要劇者,
 令通判致祭,餘皆長吏親享。未幾,澤州請立風伯、雨師廟,乃令禮官考儀式頒之。有司言:「唐制,諸郡置風伯壇社壇之東,雨師壇於西,各稍北數十步,卑下於社壇。祠用羊一,籩、豆各八,簠、簋各二。」元豐詳定局言:「《周禮》:『小宗伯之職,兆五帝於四郊,四類亦如之。」鄭氏曰:『兆為壇之營域。四類,日、月、星、辰,運行無常,以氣類為之位,兆日於東郊,兆月與風師於西郊,兆司中、司命於南郊,兆雨師於北郊。』各以氣類祭之,謂之四類。漢儀,縣邑常以丙戌
 日祠風伯於戌地,以己丑日祀雨師於丑地,亦從其類故也。熙寧祀儀:兆日東郊,兆月西郊,是以氣類為之位。至於兆風師於國城東北,兆雨師於國城西北,司中、司命於國城西北亥地,則是各從其星位,而不以氣類也。請稽舊禮,兆風師於西郊,祠以立春後丑日;兆雨師於北郊,祠以立夏後申日;兆司中、司命、司祿於南郊,祠以立冬後亥日。其壇兆則從其氣類,其祭辰則從其星位,仍依熙寧儀,以雷師從雨師之位,以司民從司中、司命、
 司祿之位。」



 舊制,風師壇高四尺,東西四步三尺,南北減一尺。皇祐定高三尺,周三十三步;雨師壇、雷師壇高三尺,方一丈九尺。皇祐定周六步。政和之制,風壇廣二十三步,雨、雷壇廣十五步,皆高三尺,四陛,並一壝,二十五步。其雨師、雷師二壇同壝。司中、司命、司民、司祿為四壇,各廣二十五步,同壝。



 又言:「《周禮》:『大宗伯以□□燎祀司中、司命、風師、雨師。』所謂周人尚臭,升煙以報陽也。今天神之祀皆燔牲首,風師、雨師請用柏柴升煙,以為歆神之始。」又
 言:「《周禮》樂師之職曰:『凡國之小事用樂者,令奏鐘鼓。』說者曰:『小祀也。』小師職《注》:『小祭祀謂司中、司命、風師。』是也。既已有鐘鼓,則是有樂明矣。請有司祀司中、司命、風師、雨師用樂,仍制樂章以為降神之節。」又言:「《周禮》小司徒之職:『凡小祭祀奉牛牲羞其肆。』又《肆師》云:『小祭祀用牲。』所謂小祭祀,即司中、司命、司民、司祿、宮中七祀之類是也。後世以有司攝事,難於純用太牢,猶宜下同大夫禮,用羊、豕可也。今祀儀,馬祖、先牧、司中、司命、司民、司祿、司
 寒,歲用羊、豕一。《祠令》:小祠,牲入滌一月,所以備潔養之法。今每位肉以豕,又取諸市,與令文相戾。請諸小祠祭以少牢,仍用體解。」又言:「社稷五祀,先薦爓,次薦熟;至於群小祀,薦熟而已。請四方百物、宮中七祠、司中、司命、風師、雨師止薦熟。」並從之。



 司寒之祭,常以四月,命官率太祝,用牲、幣及黑牡、秬黍祭玄冥之神,乃開冰以薦太廟。建隆二年,置藏冰署而修其祀焉。秘書監李至言:「案《詩·豳·七月》曰:『四之日獻羔
 祭韭。』蓋謂周以十一月為正,其四月即今之二月也。《春秋傳》曰:『日在北陸而藏冰。』謂夏十二月,日在危也。『獻羔而啟之』,謂二月春分,獻羔祭韭,始開冰室也。『火出而畢賦』,火星昏見,謂四月中也。又案《月令》:『天子獻羔開冰,先薦寢廟。』詳其開冰之祭,當在春分,乃有司之失也。」帝覽奏,曰:「今四月,韭可苫屋矣,何謂薦新?」遂正其禮。天聖新令:「春分陰冰,祭司寒於冰井務,卜日薦冰於太廟。季冬藏冰,設祭亦如之。」



 元豐,詳定所言:「熙寧祀儀,孟冬選吉
 日祀司寒。按古享司寒,惟以藏冰啟冰之日,孟冬非有事於冰,則不應祭享。今請惟季冬藏冰則享司寒,牲用黑牡羊,穀用黑秬黍。仲春開冰,則但用羔。孔穎達注《月令》曰:『藏冰則用牡黍,啟唯告而已。』祭禮大、告禮小故也。且開冰將以御至尊,當有桃弧、棘矢以禳除兇邪。設於神坐,則非禮也。當從孔氏說,出冰之時,置弓矢於凌室之戶。」



 大觀,禮局言:「《春秋左氏傳》,以少昊有四叔,其二為玄冥。杜預、鄭玄皆以玄冥為水官,故歷代祀為司寒,則
 玄冥非天神矣。今儀注,禮畢有司取祝幣瘞坎,贊者贊幣燔燎,是以祀天神之禮享人鬼也。請罷燔燎而埋祝幣。」詔從其請。



 大蠟之禮,自魏以來始定議。王者各隨其行,社以其盛,臘以其終。建隆初,以有司言:「周木德,木生火,宜以火德王,色尚赤。」遂以戌日為臘。三年,戊戌臘,有司畫日,以七日辛卯。和峴奏議曰:「按蠟始於伊耆,後歷三代及漢,其名雖改,而其實一也。漢火行,用戌臘,臘者接也,新故相
 接,畋獵禽獸以享百神,報終成之功也。王者因之,上享宗廟,旁及五祀,展其孝心,盡物示恭也。魏、晉以降,悉沿其制。唐乘土德,貞觀之際,以前寅日蠟百神,卯日祭社宮,辰日享宗廟。開元定禮,三祭皆於臘辰,以應土德。今以戌日為臘,而以前七日辛卯行蠟禮,恐未為宜。況宗廟、社稷並遵臘享,獨蠟不以臘,請下禮官議。」議如峴言,今後蠟百神、祀社稷、享宗廟皆用戌臘一日。天聖三年,同知禮院陳詁言:「蠟祭一百九十二位,祝文內載一百
 八十二位,唯五方田畯、五方郵表□綴一十位不載祝文。又《郊祀錄》、《正辭錄》、《司天監神位圖》皆以虎為於菟,乃避唐諱,請仍為虎。五方祝文,眾族之下增入田畯、郵表畷云。」



 元豐,詳定所言:「《記》曰:『八蠟以祀四方,年不順成,八蠟不通。』歷代蠟祭,獨在南郊為一壇,惟周、隋四郊之兆,乃合禮意。又《禮記·月令》以蠟與息民為二祭,故隋、唐息民祭在蠟之後日。請蠟祭,四郊各為一壇,以祀其方之神,有不順成之方則不修報。其息民祭仍在蠟祭之後。」先
 是,太常寺言:「四郊蠟祭,宜依百神制度築壇,其東西有不順成之方,即祭日月。其神農以下,更不設祭。又舊儀,神農、后稷並設位壇下,當移壇上。按《禮記正義》:伊耆氏,神農也。今壇下更設伊耆氏位,合除去之。」



 《政和新儀》:臘前一日蠟百神。四方蠟壇廣四丈,高八尺,四出陛,兩壝,每壝二十五步。東方設大明位,西方設夜明位,以神農氏、后稷氏配,配位以北為上。南北壇設神農位,以後稷配,五星、二十八宿、十二辰、五官、五岳、五鎮、四海、四瀆及
 五方山林、川澤、丘陵、墳衍、原隰、井泉、田畯,倉龍、朱鳥、麒麟、白虎、玄武,五水庸、五坊、五虎、五鱗、五羽、五介、五毛、五郵表畷、五臝、五貓、五昆蟲從祀,各依其方設位。中方鎮星、后土、田畯設於南方蠟壇酉階之西,中方岳鎮以下設於南方蠟壇午階之西。伊耆設於北方蠟壇卯階之南,其位次於辰星。



 紹興十九年,有司檢會《五禮新儀》,臘前一日蠟東方、西方為大祀,蠟南方、北方為中祀,並用牲牢。乾道四年,太常少卿王瀹又請於四郊各為一
 壇,以祀其方之神,東西以日月為主,各以神農、后稷配;南北皆以神農為主,以後稷配。自五帝、星辰、岳鎮、海瀆以至貓虎、昆蟲,各隨其方,分為從祀。其後南蠟仍於圓壇望祭殿,北蠟於餘杭門外精進寺行禮。



 太廟司命、戶、灶、中溜、門、厲、行七祀,熙寧八年,始置位版。太常禮院請禘享遍祭七祀。詳定所言:「《周禮》:天子六服,自鷩冕而下,各隨所祭而服。今既不親祀,則諸臣攝事日,當從王所祭之服,其攝事之臣不系其官。」又言:「《禮·祭
 法》曰:『王自為立七祀:曰司命,曰中溜,曰國行,曰泰厲,曰門,曰戶,曰灶。』孟春祀戶,祭先脾;孟夏祀灶。祭先肺;中央土祀中溜,祭先心;孟秋祀門,祭先肝;孟冬祀行,祭先腎。又《傳》曰:『春祀司命,秋祠厲。』此所祀之位,所祀之時,所用之俎也。《周禮》:『司服掌王之吉服,祭群小祀則服玄冕。』《注》謂宮中七祀之屬。《禮記》曰:『一獻熟。』《注》謂宮中群小神七祀之等。《周禮·大宗伯》:『若王不與祭祀則攝位。』此所祀之服,所獻之禮,所攝之官也。近世因禘袷則遍祭七祀,其
 四時則隨時享分祭,攝事以廟卿行禮而服七旒之冕,分太廟牲以為俎,一獻而不薦熟,皆非禮制。請以立春祭戶於廟室戶外之西,祭司命於廟門之西,制脾於俎;立夏祭灶於廟門之東,制肺於俎;季夏土王日祭中溜于廟庭之中,制心於俎;立秋祭門及厲於廟門外之西,制肝於俎;立冬祭司命及行於廟門外之西,制腎於俎,皆用特牲,更不隨時享分祭。有司攝事,以太廟令攝禮官,服必玄冕,獻必薦熟。親祀及臘享,即依舊禮遍祭之。」《
 政和新儀》定太廟七祀,四時分祭,如元豐儀,臘享袷享則遍祭,設位於殿下橫街之北、道西,東向,北上。



 馬祖。《祀典》:仲春祀馬祖,仲夏享先牧,仲秋祭馬社,仲冬祭馬步,並擇日。壇壝之制,三壇各廣九步,高三尺,四陛,一壝。



 又有酺神之祀。慶歷中,上封事者言:「螟蝗為害,乞內外並修祭酺。」禮院言:「按《周禮》:『族師,春秋祭酺。』酺為人物災害之神。鄭玄云:『校人職有冬祭馬步。則未知此酺者,蝝螟之酺歟,人鬼之步歟?蓋亦為壇位如雩禜云。』然
 則校人職有冬步,是與馬為害者,此酺蓋人物之害也。漢有蝝螟之酺神,又有人鬼之步神。歷代書史,悉無祭酺儀式。欲準祭馬步儀,壇在國城西北,差官就馬壇致祭,稱為酺神。



 若外州者,即略依禜禮。其儀注,先擇便方除地,設營纘為位,營纘謂立表施繩以代壇。其致齋、行禮、器物,並如小祠。先祭一日致齋,祭日,設神坐內向,用尊及籩一、豆一,實以酒酺,設於神坐左。又設罍洗及篚於酒尊之左,俱內向。執事者位於其後,皆以近神為上。
 薦神用白幣一丈八尺在篚。將祭,贊祀官拜,就盥洗訖,進至神坐前,上香、奠幣。退詣罍盥洗,實以酒,再詣神坐前奠爵,讀祝,再拜,退而瘞幣。其酺神祝文曰:「維年歲次月朔某日,州縣具官某,敢昭告於酺神:蝗蝝薦生,害於嘉穀,惟神降祐,應時消殄。請以清酒、制幣嘉薦,昭告於神,尚享。」



 紹興祀令:蟲蝗為害,則祭酺神。嘉定八年六月,以飛蝗入臨安界,詔差官祭告。又詔兩浙、淮東西路州縣,遇有蝗入境,守臣祭告酺神。



\end{pinyinscope}