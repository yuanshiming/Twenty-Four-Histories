\article{志第一百四十二 兵三(廂兵)}

\begin{pinyinscope}

 廂
 兵者,諸州之鎮兵也。內總於侍衛司。一軍之額,有分隸數州者,或一州之管兼屯數州者,在京諸司之額五,隸宣徽院,以分給畜牧繕修之役,而諸州則各以其事
 屬焉。建隆初,選諸州募兵之壯勇者部送京師,以備禁衛,餘留本城,雖無戍更,然罕教閱,類多給役而已。



 景德四年七月,如京使何士宗言:「詔條禁軍將士依等級並行伏事之理,違者按軍令。其廂軍將士等未立條制,欲望約前詔減一等定令。」帝曰:「禁衛兵士無他役使,且廩給優厚,欲其整肅,有所凜畏,故設此條禁。今以廂軍約此施行,恐難經久。況尊卑相犯,自有條律,不行可也。」十二月,詔廂軍及諸州本城犯,所部決杖訖,並移隸他軍,
 內情理重及緣邊隨軍者奏裁。



 大中祥符元年,詔應諸道州、府、軍、監廂軍及本城指揮,自都指揮使已下至長行,對本轄人員有犯階級者,並於禁軍斬罪上減等,從流三千里上定斷;副兵馬使已上,勘罪具案聞奏。廂軍軍頭已下至長行,準敕犯流免配役,並徒三年上定斷,只委逐處決訖,節級已上配別指揮長行上名,長行決訖,配別指揮下名收管。如本處別無軍分指揮,即配鄰近州、府、軍、監指揮收管。內別犯重者,自從重法。其諸司庫
 務人員兵士有犯上件罪名者,並依前項廂軍條例施行。



 五年二月,上諭王欽若等:「累議老病之兵漸多,在京者令逐司將臣,外處者散差諸司使副揀選。可指揮所揀殿前、侍衛馬步軍司,令先逐指揮自指揮使已下,據見管兵士除堪任披帶征役外,其自來懦弱、教閱不出之人及老病不堪者,籍其名,供申次第,管轄處各就逐營看詳定奪,然後繳申逐司,與差去使臣同共揀選。如有協情不當,即具始末以聞。其廂軍都指揮使已下並
 當嚴斷,外處揀就糧兵士亦如之。」又宣示:「外處就糧諸軍,有捧日、天武第七第九第十軍軍額,皆是自上軍經兩三度揀選,以其久處禁衛,不欲便揀落,特設上件軍額處之。朕深慮揀兵臣僚、軍頭等同諸軍例,更揀配下軍,可遍諭之。老病者便放歸農,內契丹、渤海、日本外國人恐無依倚,特與收充本軍剩員。」又:「所差臣僚、軍頭赴外處揀人。緣軍分指揮及出入次第名目體例甚多,令樞密院具合行條約及施行事件,並畫一處分,令遵守
 施行。」



 又:「殿前、侍衛馬步軍司自來揀下披帶禁軍,量減衣賜月糧充剩員,並無定額,散在逐營拘系,不獲營生,官中所給歲計不少,可乘此時一例揀選。除老病者放歸農外,據諸軍見管人數額定充看營剩員,餘並撥並一處收管,以備令赴諸處祗應。既有定額,必不敢多揀充剩員。」又詔:「承前遣使取內外軍中疲老者,咸給奉糧之半,以隸剩員,今可簡閱使歸農。其合留者,亦據逐營給役數外別為營舍處之。內契丹、渤海、日本外國人慮
 無所歸,且依舊。仍令所至州郡並與總管、鈐轄閱驗,連書其狀,具當去留之數,及引視軍校之不任職者,附驛以聞。其當從隸軍額,即就配近便州郡;緣邊者徙於內地,並與本州官吏移牒轉送;當停者給與公驗,止許居本州,歲申上其籍,並給次月奉糧、裝錢、日食遣之。所簡馬,但筋齒弱、老病不療者,件析以聞。在京殿前、馬步軍司有所升退,實時具名籍申樞密院,未當者悉改正之;當徙者給裝錢,在道只給糧;當停者給一月奉糧,勿復
 奏裁。外州軍士當降以次軍分者,所隸州郡聽自擇。」



 又詔:「廣南東西、荊湖南北、福建、江南、京西等七路諸州、府、軍、監見管雜犯配隸軍人等,各差使臣一人,馳驛往逐處與轉運使、副或提點臣僚、知州、通判、鈐轄、都監、監押同共簡選,就近體量人數,分配側近州軍本城收管。如年老病患,委實久遠不任醫治充役者,放令逐便;其少壯者即差人管押赴闕引見,當議選配近上軍分。如不願量移及赴闕者,亦聽其便,仍於軍份量與遷改。如地
 遠勾抽遲延,即馳驛分路簡訖,具析以聞。」



 七年,詔:「今後軍回在京者且未編排,依例引見。內有老疾合配外處軍分,及看倉庫、草場神衛剩員並看營剩員等,與限歇泊半月後,編排引見訖,限五日般移。其外處軍回經過兵士並依此例,仍見訖與假十日,令移隸所配處。」



 八年,詔:「諸路轉運司、殿前侍衛馬步軍軍頭司、三司、宣徽院、開封府、諸司庫務等處人員兵士等,如內有殺賊得功及諸般使喚得力者,或因官中取索之時具詣實結罪
 供,申所轄去處,保明申奏。」



 天禧元年,詔選天下廂兵遷隸禁軍者凡五千餘人。二年,詔:「河北禁軍疲老不任力役者,委本路提點刑獄臣僚簡閱,不得庇匿,以費廩糧。」



 慶歷中,招收廣南巡海水軍、忠敢、澄海,雖曰廂軍,皆予旗鼓訓練,備戰守之役。皇祐中,河北水災,農民流入京東三十餘萬,安撫使富弼募以為兵,拔其尤壯者得九指揮,教以武技。雖廩以廂兵,而得禁兵之用,且無驕橫難制之患。詔以其騎兵為教閱騎射、威邊,步兵為教閱
 壯武、威勇,分置青、萊、淄、徐、沂、密、淮陽七州軍,征役同禁軍。嘉祐四年,復詔西路於鄆、濮、齊、兗、濟、單州置步兵指揮六,如東路法。於是東南州軍多置教閱廂軍,皆以威勇、忠果、壯武為號,訓肄如禁軍、免其它役。



 治平初,遣使分募河北、河東、陜西、京東民為本城,遇就糧禁軍闕,即遣補。又陜西州軍悉置壯城如河北,以備繕完城壘之役。景祐中,本城四十三萬八千,逮治平三年,乃五十萬。



 熙寧三年五月,詔以禁軍分五部法檢治廂軍,其後禁軍
 或降剩員,或升補,皆以備廂軍諸路力役之事。間詔募增,而京西轉運司所募多至三萬人;陜西減額五千人,亦至三萬人。河朔流民寓京東者如舊制招募教閱,以為忠果二十指揮,分隸河北總管司,以除盜恤饑。而河北及熙河路修城壘,河北所募兵五千人,熙河亦三千人。修京城,以廢馬監兵置廣固、保忠凡十指揮,亦五千人,湖南猺人平,戎、瀘軍興,洮、河轉漕,又皆增置焉。



 初,樞密院言:「京師役兵不足,歲取於諸路,而江、淮兵每饑凍,
 道斃相屬。略計歲所用外軍七千人,調發增給不貲。請募東西八作司壯役指揮,諸司雜犯罪人情輕者並配隸,以次補雜役、效役,代諸路役兵,」從之。又言:「諸路廂軍名額猥多,自騎射至牢城,其名凡二百二十三。其間因事募人,團立新額,或因工作、榷酤、水陸運送、信道、山險、橋梁、郵傳、馬牧、提防、堰埭,若此者事在而名未可廢。及剩員直、牢城皆待有罪配隸之人;壯城專治城隍,不給他役,別為一軍;而教閱廂軍亦自為額。請以諸路不教
 閱廂軍並為一額,餘從省廢,其移並如禁軍法。」奏可。遂下諸路轉運司,以州大小高下為序,始自某州為第一指揮,差次至某州,凡為若干指揮,每指揮毋過五百人。河北曰崇勝,河東曰雄猛,陜西曰保寧,京東曰奉化,京西曰勁武,淮南曰寧淮,兩浙曰崇節,江南曰效勇,荊湖曰宣節,福建曰保節,廣南曰清化,川四路曰克寧。八年,詔忻、代州諸砦,以禁軍代廂軍。



 元豐四年,詔升南京、青、鄆、鄧、曹、濟、濮州有馬教閱廂軍,及真定府北砦勁勇、環州下
 蕃落未排定指揮,並為禁軍。五年三月,以西邊用兵,詔諸處役兵並罷,令諸路轉運司鏟刷京東西、河東北、淮南廂軍,又令都水監刷河清及客軍共三萬餘人赴陜西團結。十月,詔諸路教閱廂軍,於下禁軍內增入指揮名額,排連並同禁軍。於是,馬步排定有馬廂軍二十二指揮,無馬廂軍二百二十九指揮。元豐之末,總天下廂兵馬步指揮凡八百四十,其為兵幾二十二萬七千六百二十七人,而府界及諸司或因事募兵之額不
 與焉。



 哲宗元祐二年,太師文彥博言:「廂軍舊隸樞密院,新制改隸兵部,且本兵之府豈可無籍?」樞密院亦以為言。乃詔本部自今進冊,以其副上樞密院。三年,詔京西路廂軍以三萬五百人為額,又詔天下州郡以地理置壯城兵。



 元符元年,詔罪人應配五百里以上,皆配陜西、河東充廂軍,諸路經略司各二千人止。三年,詔撥陜西保寧指揮入諸路廂軍額。



 崇寧四年,詔諸路廂軍不以等樣選少壯人招刺。又詔:廂軍工匠除上京修造外,其
 餘路所差,並放還休息之。



 政和五年,廣固四指揮各增五百人,以備京城之役。六年三月,增置通濟兵士二千人,備御前牽挽綱運。於是工役日興,增募益廣矣。



 建炎而後,兵制靡定,逮乾道中,四川廂軍二萬九百七十二人,禁軍二萬七千九百九十二人。厥後廢置損益,隨時不同,摭其可考者以附見焉。



 其將校則有馬步軍都指揮使,有副都指揮使、都虞候。馬軍有都指揮使、副都指揮使、都虞候,步軍亦如之。馬步軍諸指揮有指揮使、
 副指揮使。每都有軍使、副兵馬使、都頭、副都頭、十將、將虞候、承局、押官。



 凡諸州騎兵、步兵、禁廂兵之類,敘列如左。其不同者,分為建隆以來及熙寧以後之制云。



 建隆以來之制



 馬軍



 騎射京東路:南京、青、兗、鄆、曹、徐、齊、濰、淮陽。京西路:西京、河陽、陳、許、鄭、穎、滑。河北路:北京、真定、滄、澶、相、恩、冀。陜西路:永興、鳳翔、河中、陜、華、秦、涇、鄜。荊湖路:江陵、潭、鄂、岳、復、安、澧、鼎、永、道、郴、邵、桂陽。內青、淮陽系教閱。



 威邊京東路:南京、青、鄆、密、徐、曹、齊、濮、濟、淄、登、萊、沂、單。內登系教閱。京西路:西京、河陽、鄭、蔡、襄、鄧、滑、穎、汝、郢、均、商、惰、唐、信陽、光化。河北路:瀛、相、邢、祁、濱、霸、磁、衛、趙、莫、洺、乾寧、廣信、通利。河東路:澤、遼。
 陜西路:永興、鳳翔、河中、陜、同、華、耀、乾、解、虢。淮南路:亳、廬、宿。荊湖路:安。



 昭武南京、河中。



 肅戎曹。



 單勇單。



 安武鄆、齊。



 必敵鄆、陜、邠。



 決勝濟。



 飛勇棣。



 靜山兗、宜。



 勇敢沂、密。



 定邊蔡、徐、涇。



 馬鬥永興、宿。



 安東登、萊。



 突陣延、定、乾、懷。



 廳直濟、滄、莫、保、雄、霸、定、華。



 保勝鄜、光、嵐。



 歸恩鳳翔。



 定戎涇。



 安塞環、慶。



 游奕許。



 衙隊曹、陳、德、永靜、永、隴、儀、峽。



 武勝濠、泗。



 保忠滑。



 輕騎邢。



 順節真定。



 敢勝深。



 飛塞環。



 保節陜西路州軍。



 本城馬軍廣。



 必勝慶。



 定塞河北路州軍。



 勁勇真定北砦,系教閱。



 下蕃落環外砦,系教閱。



 武清晉。自此至招收,凡十一軍,《兩朝志》無。



 飛騎麟。



 振邊儀、環。



 威遠府。



 本城廳子定。



 克戎並。



 清邊陜西。



 忠烈河北
 鄉兵,名神銳,後改是軍。舊制,老病者聽召人承補歸農,承補者逃亡,復取歸農者充役。大中祥符四年,詔罷之。



 無敵保、安肅、廣信軍、北平砦。



 忠銳廣濟。



 招收河北、河東。舊又有定州揀中廳子、易州靜塞、並州咸聖,後並廢。



 飛將北京、亳。自此至揀中騎射凡三軍,《三朝志》無。



 保靜恩。



 揀中騎射淮南路:揚、廬、壽、宿、泗、真、蘄、黃、濠、光、海、和、通、舒、滁、漣水、高郵、無為。江南路:宣、撫、江、吉、筠、袁、歙、太平、池、饒、信、廣德、南康、南安、建昌、臨江、興國。



 步軍



 武和開封。



 武肅開封。



 忠靜開封。



 威勇定、真定、冀、滄、雄、博、深、乾寧。內青、鄆、淄、密、濟、沂、淮陽系教閱。



 左衙南京、鄆、晉、耀、陜、通、安。



 平難亳、濠。



 奉化京西路:鄭、許、陳、蔡、滑。河北路:懷。陜西路:鳳翔。淮南路:揚、亳、廬、壽、宿、濠、和、通、泰、楚、舒、真、泗、滁、無為、漣水、高郵。



 衙隊曹、峽。



 開武曹。



 保寧濟、衛。



 開遠揚、楚、泗、齊、利、劍。



 安平齊。



 靜邊棣。



 六
 奇楚。



 開山西京、秦。



 武勇濰、泰。



 懷安秦。



 建安解、府。



 靜海徐、淮陽、通。



 隨身宿、隨、唐、商。



 崇順青、階。



 忠略淄。



 安海登。



 水軍京東路:登。河東路:潞、保德。陜西路:秦、陜。淮南路:揚、廬、壽、光、海、和、泰、楚、舒、蘄、黃、泗、漣水、高郵、無為。江南路:江寧、洪、袁、虔、宣、歙、饒、信、太平、池、江、吉、筠、撫、興國、臨江、南康、廣德。兩浙路州軍。荊湖路:江陵、潭、衡、永、郴、邵、鄂、岳、復、安、澧、峽、鼎、歸、漢陽、桂陽。福建路:福、建、漳、泉、邵武。利州路:興。廣南路:廣、英、賀、封、連、康、南雄、春、廉、白、邕。



 寧濟萊。



 永安西京。



 耀武河陽、鄧、楚、秦、寧、華。



 橋道河陽、澶、壽、興。



 開道鄭。



 雄猛絳。



 定安河中。



 開河河中。



 定遠鳳、復。



 定邊涇。



 壯武京東路:青、徐、曹、兗、密、濰、濟、濮、登、萊、淮陽。京西路:西京、陳、蔡、鄧、襄、穎、汝、光化。陜西路:鳳翔、河中、同、耀、華、乾、解、陜、保安。淮南路:揚、廬、壽、黃、光、海、和、通、蘄、楚、泰、舒、滁、高郵。荊湖路、漳、
 岳、安、復。內兗、徐、濟、萊系教閱。



 寧淮穎、壽、澶。



 忠順穎、壽。



 崇寧汝、岳。



 澄海韶、循、潮、連、梅、南雄、英、賀、封、端、南恩、春、惠、桂、容、邕、象、昭、龔、蒙、潯、貴、橫、融、化、雷、竇、南儀、白、欽、鬱林、廉、崖、儋。內廣、廉、高、藤、梧、英、賀、新、蒙、龔、儋系教閱。



 保定均、信陽。



 懷寧定、真定、祁、房。



 宣節荊湖南路諸州軍監,北路:岳、澧、鼎、郢、荊門、諸監。



 步捷金。



 崇化光。



 廣平虢。



 勇勝永興。



 清邊永興、延、渭、鄜、慶、涇、儀、隴、保安。



 開廣原、同。



 建武密、鄜、環。



 永清丹。



 昭勝坊。



 永寧潞。



 永霸澤。



 弓箭秦、晉。



 順安慈。



 順霸隰。



 崇勇成。



 肅清幹。



 懷節澶。



 崇武懷。



 廣霸北京。



 興安北京。



 制戎冀。



 雄銳真定。



 定虜深。



 招收汾、遼、澤、石、潞、慈、晉、絳、代、忻、威勝、平定。



 定和定。



 保順滄。



 清遠雄、霸。



 克勝瀛、滄、黃、保定。



 寧邊乾寧。



 開邊平
 定。



 靜勝揚。



 寧順廬。



 旌勇壽。



 備邊泗。



 三捷滁。



 寧化舒。



 保勝光。



 懷仁蘄、黃。



 武雄江陵。



 步驛襄、江陵、荊門、循、賀、封、梅、康、南雄、潮、韶。



 克寧成都路:成都、蜀、漢、雅、邛、嘉、綿、陵、彭、眉。梓州路:戎、榮、普、資、梓、合、瀘、遂、渠、昌、果、懷安、廣安。利州路:興元、洋、利、龍、劍、蓬、璧、文、興、安德、三泉。夔、渝、涪、萬、達、開、施、忠、雲安、大寧。



 威棹荊湖路:江陵、歸、峽。成都路:成都、嘉、眉、簡。梓州路:諸州軍。利州路:劍、安德。夔州路:渝、涪、萬、雲安。



 懷遠興元。



 保節河北路:定、真定、滄、瀛、相、邢、洺、冀、祁、德、濱、保、雄、磁、博、趙、深、懷、衛、順安、通利、信安、保定、安肅、永定、永靜。河東路:太原、晉、絳、汾、遼、澤、石、潞、慈、麟、府、憲、代、忻、隰、威勝、岢嵐、火山、保德、平定。陜西路:永興、秦、邠、寧、鄜、延、環、慶、涇、儀、丹、隴、坊、鎮戎、德順。淮南路:舒。江南路:洪、虔、江、池、饒、信、太平、吉、筠、袁、撫、興國。福建路:汀、南劍。荊湖路:鄂。利州路:龍、利。



 懷信利。



 廣塞興元、三泉。



 順
 化興。



 效勇江寧、廣德。



 里運江寧。



 貢運饒。



 水運潭、泰、臨江。



 廣濟京城上下金巢、陳、壽、揚、宿、高郵、漣水、通利。



 崇節兩浙路:杭、越、蘇、湖、溫、臺、衢、婺、處、睦、秀。福建路:福、漳、泉、興化。陜西路:成。



 寧塞太原、汾、遼、石、代、忻。



 牢城河北、河東、陜西、淮南、京東西、江南、荊湖、廣南、益、梓、利、夔路諸軍州,惟汝、處、昭、保安不置。



 羅城成都。



 水軍奉化京畿諸縣、泰、泗。



 船坊洺、潭、鼎。



 渡船都潭。



 橋閣龍、劍、文、三泉。



 採斫處、衢、溫。



 梢工都洪、楚、真。



 防河成都。



 捍江都杭。



 船務杭、婺。



 巡海水軍廣。



 雜作都壽。



 本城曹、秀、常、火山、南安,梁山、梅。



 勁勇邢、太原、嵐、汾、遼、澤、潞、晉、憲、代、忻、隰、岢嵐、平定、寧化、威勝。內真定北砦系教閱。



 裝發真、泗、楚、通利。



 寧海瓊。



 西懷化許。



 新招靜江邕。



 南懷化許。



 防城泗、均。



 水軍
 橋道泗。



 剩員直亳城。



 清化桂、容、邕、象、昭、梧、藤、龔、蒙、貴、柳、宜、實、融、化、竇、高、南儀、白、欽、鬱林、廉、瓊、儋。



 江橋院明。



 肅寧城寧。



 崇勝真定。



 碇手明。



 揀中宣節潭、澶、鼎。



 採造西京、秦、明。



 堰軍長安、京口、昌城、杉青。



 裝卸南京、亳。



 中軍將潭、汀。



 宣武大名、真定、懷、衛。



 順節磁。此下至新立本城凡三十八軍,天聖後無。



 七擒單。



 安化濱。



 武順懷。



 平海登。



 英武鄜。



 長劍滑。



 長寧衛。



 德勝相。



 保安博。



 興化洺。



 定勇深。



 安勝通利。



 霜水夔。



 興造衡、潭。



 水路都江陵。



 山場斫軍溫、婺、睦。



 本城廣軍廣。



 河東定、真。



 本城剩員諸州並有。



 蕃落慶。



 都竇水軍容。



 新水軍全。



 武定陜西、晉、絳、慈、隰。



 定塞河北。



 舊水軍荊湖、江南、兩浙、
 淮南。



 剩員澧、復。



 下浮橋西京。



 東南道巡海水軍、教閱澄海。



 槔手常。



 慶成慶成軍。



 梅山洞剩員丹。



 捉生延。



 河清河陰、汴口。



 宣勇河北、河東。本鄉兵,舊名忠勇。



 保毅秦鄉軍。



 新立本城曹。



 奉先會聖宮、永熙陵。此下至酒務雜役凡六十軍,天聖以後置。



 六軍京師。



 御營喝探京師。



 揀中窯務京師。



 看船廣德京師。



 揀中剩員雍丘、陳留、襄邑、咸平。



 右衙南京、徐、鄆、曹、廣濟、晉、陜。



 靜海徐、淮陽、通。



 歸定河陽。



 驍勇邠。



 感順慶。



 拓邊環。



 宣猛威勝。



 靜江京西路:陳、蔡、郢。江南路:南安。荊湖路:江陵、潭、岳、鼎、衡、永、郴、全。廣南路、廣、韶、循、潮、連、梅、南雄、英、賀、封、端、新、康、春、惠、桂、容、邕、象、昭、梧、藤、龔、蒙、潯、貴、柳、融、宜、賓、橫、化、竇、高、雷、欽、鬱林、廉、瓊。利州路:利。



 三略陳、鼎。



 靜虜
 深。



 克勝瀛、滄、黃、保定。



 武捷鳳翔、秦、鳳、鄜、延、涇、原、儀、滑、邠、寧、階、坊、丹、晉、絳、隰、慈。



 車軍真、楚、常。



 會通橋道西京。



 司牧永興、秦、階、原、德順。



 鹽車泰、真。



 新招梢工真、泗。



 拔頭水軍泗。



 造船軍匠吉。



 樓店務杭。



 造船場廣。



 駕綱水軍廣。



 建安解。



 省作院邠。



 雄勇火山。



 屯田保。



 清務杭、蘇、婺、溫、潭。



 靜淮蔡。



 捍海通、泰。



 船坊鐵作潭。



 揀中曹。



 壯城京東路:青、密、濰、登、沂、濮、萊、淄。京西路:西京、蔡、汝。河北路:諸州軍。河東路:太原、遼、澤、晉、絳、潞、汾、石、慈、麟、府、憲、代、忻、隰、嵐、寧化、保德、火山、威勝、岢嵐。陜西路:永興、河中、涇、原、儀、渭、鄜、慶、陜、耀、坊、華、丹、同、隰、解、鎮戎、德順。江南路:洪。



 強勇瀛、滄、懷、冀、晉、絳、潞、汾、遼、石、慈、代、忻、澤。



 馬監北京大名、相安陽、洺廣平、衛淇水、鄆東平、許單鎮、西京洛陽、同沙苑、鄭原武。



 城
 面廣、端、惠、循、英、春、賀、梅、連、康、新、封、白、潮。



 戰棹欽、廉。



 遞角場留。



 安遠桂。



 作院丹、儀。



 色役環。



 雜攢代。



 作院工匠太原。



 咸平橋道永興。



 運錫循。



 水磨鄭。



 東西八作西京。



 窯務西京。



 鼓角將潤、荊門。



 錢監江。



 鐵木匠營池。



 酒務營池。



 竹匠營池。



 酒務雜役江寧。



 諸司庫務、河清、馬遞鋪等役卒:



 東西八作司、廣備、雜役、效役、壯役。



 牛羊司、御輦院、軍器庫、後苑造作所、後苑工匠、文思院、內弓箭庫、南作坊、北作坊、弓弩院、法酒庫、西染院、綾錦院、裁造院、修內司、翰林司、儀鸞司、事材場、四園苑、玉津園、養象、廣德、金
 明池雜役、鞍轡庫、醴泉觀、萬壽觀、集禧觀、禮賓院、駝坊、內酒坊,右宣徽院轉補,分隸三司、提舉司。



 河清、街道司,隸都水監。



 後苑御弓箭庫、作坊物料庫、後苑東門藥庫、內茶紙庫、御廚、御膳廚、供庖務、內物庫、外物料庫、油庫、醋庫、都監院物料庫、西水磨務、東水磨務、、大通門水磨、磁器庫、都茶庫、內衣庫、朝服法物庫、祗候庫、榷貨務、內藏庫、左藏庫、布庫、奉宸庫、尚衣庫、內香藥庫、退材場、東西窯務、竹
 木務、左右廂店宅務、修造。



 諸倉、修造。



 下卸司、東西裝卸。



 排岸司、廣濟。



 左右街司、左右金吾仗司、西太一宮、鑄瀉務、開封府步驛、致遠務、車營務、諸門並府界馬遞鋪,分隸三司、提舉司、開封府。



 熙寧以後之制。



 河北路騎軍之額,自騎射而下十有二;步軍之額,自奉化而下二十有六,並改號曰崇勝。凡一百一十二指揮,二萬九千二百七十人。



 橋道澶。



 壯城、牢城諸州。



 馬監北京大名、
 相州安陽、洺州廣平、衛州淇水。



 騎射北京、真定、滄、澶、相、恩、冀、棣。



 威邊瀛、相、邢、祁、濱、磁、衛、趙、莫、洺、乾寧、廣信、通利。



 飛將北京。



 飛勇棣。



 突陣懷。



 廳直瀛、滄、雄、霸、莫、保定。



 衙隊德、永靜。



 保靜恩。



 輕騎邢。



 順節真定。



 敢勝深。



 定塞定、真定、冀、滄、雄、博、深、乾寧。



 奉化懷。熙寧七年,京東、河北置揀中廂軍,懷、衛、濮各二,德、博、棣、齊各一。



 靜邊棣。



 耀武定。



 懷節澶。



 廣霸北京。



 制戎冀。



 雄銳真定。



 定虜深。



 靜虜趙。



 定和定。



 保順滄。



 清遠雄、霸。



 克勝瀛、滄、莫、保定。



 保節定、真定、滄、瀛、相、邢、洺、冀、祁、德、濱、保、雄、磁、博、趙、深、懷、衛、順安、通利、信安、保定、安肅、永定、永靜。



 懷寧定、真定、祁。



 勁勇邢。元豐四年,升為真定府北砦勁勇,為禁軍。



 宣武大名、真定、懷、衛。元祐二年,在京師置第十三至第十五三指揮。



 威勇滄。



 崇勝真定。熙寧七年,
 京東、河北置揀中廂軍,懷、濮各一,德、博、棣、齊各二。



 肅寧肅城。



 廣濟通利。熙寧八年,詔以六分為額,罷所差客軍。



 屯田保。



 寧邊乾寧。



 強壯邢。



 宣勇瀛、滄、懷、冀。



 廣威元符元年,詔河北路大名府等二十二州軍創置馬步軍五十六指揮,馬軍以廣威為名。



 河東路騎軍之額,自威邊而下二;步軍之額,自左衙而下十有八,並改號曰雄猛。凡五十二指揮,一萬二千四百一十人。



 本城火山。



 牢城諸州。



 壯城太原、遼、澤、晉、絳、潞、汾、石、慈、麟、府、憲、代、忻、隰、嵐、寧化、保德、火山、威勝、岢嵐。



 雜攢代。



 作院工匠太原。



 威邊澤、遼。



 保勝嵐。



 左衙、右衙晉。



 水軍潞、保德。



 雄猛絳。



 永寧潞。



 永霸澤。



 弓箭晉。



 順安
 慈。



 順霸隰。



 宣猛威勝。



 招收汾、遼、澤、石、潞、慈、晉、絳、代、忻、威勝、平定。



 開邊平定。



 保節太原、晉、絳、汾、遼、澤、石、潞、慈、麟、府、憲、代、忻、隰、威勝、岢嵐、火山、保德、平定。



 勁勇太原、嵐、汾、遼、澤、潞、晉、憲、代、忻、隰、岢嵐、平定、寧化、威勝。



 武捷晉、絳、隰、慈。



 寧塞太原、汾、遼、石、代、忻。



 廣濟壽陽。熙寧八年,以六分為額,減諸路所差防河客兵。



 宣勇晉、絳、潞、汾、遼、石、慈、代、忻、澤、威勝、平定。



 陜西路騎軍之額,自騎射而下有六;步軍之額,自左衙而下二十有九,並改號曰保寧。凡一百一十一指揮,二萬五百六十三人。



 開山秦。



 關河河中。



 司牧永興、秦、階、原、德順。



 省作院邠。



 壯城永興、河中、涇、原、儀、謂、鄜、慶、陜、耀、坊、華、丹、同、隴、乾、解、鎮戎、德順。



 牢城諸州。



 馬監
 同州沙苑。



 作院丹、儀。



 色役環。



 咸陽橋道永興。



 騎射永興、鳳翔、河中、陜、華、秦、涇、鄜。



 安邊永興、鳳翔、河中、同、華、耀、乾、解、虢。



 昭武河中。



 必敵陜、邠。



 定邊涇。



 馬鬥永興。



 突陣延、同、乾。



 廳直華。



 保勝鄜。



 歸恩鳳翔。



 定戎涇。



 安塞環、慶。



 衙隊隴、儀。



 飛砦環。



 必勝慶。



 保節永興、秦、邠、寧、鄜、延、環、慶、涇、原、儀、渭、丹、隴、坊、鎮戎、德順。



 左衙耀、陜。



 右衙陜。



 保寧渭。熙寧七年,詔系役廂禁軍自今權免役,專肄習武藝,置鳳翔府簡中保寧六指揮三千人,專備熙河修城砦。元豐五年,蘭州置二。紹興三年,熙河增置四,又於涇原創置十。元符三年十月,詔撥陜西路保寧指揮入廂軍額,從知渭州章楶請也。



 隨身商。



 崇順階。



 水軍秦、陜。熙寧五年,鎮洮置一,崇寧三年,鄯州及龍支城名置二。



 耀武寧、華。



 定安河中。



 奉化鳳翔。



 廣平虢。



 勇勝永興。



 清遠永興、延、渭、鄜、慶、涇、儀、保安。



 開廣原、同。



 建武鄜、環。



 昭勝坊。



 弓箭秦。



 崇勇成。



 肅清幹。



 寧遠鳳。



 壯武鳳翔、河中、同、耀、華、乾、解、陜、保安。



 驍勇邠。



 感順慶。



 拓邊環。



 崇節成。



 武捷鳳翔、秦、鳳、鄜、延、涇、原、儀、渭、邠、寧、階、坊、丹。



 威勇河中。



 採造秦。元豐四年,通遠軍增置一。



 建安解。



 京東路騎軍之額,自騎射而下有三;步軍之額,自左衙而下十有七,並改號曰奉化。凡五十四指揮,一萬四千七百五十人。



 壯城青、密、濰、登、沂、濮、萊、淄。



 馬監鄆州東平。



 裝卸南京。



 牢城諸州。



 騎射南京、青、兗、鄆、曹、徐、齊、濰。



 威邊南京、青、鄆、密、徐、曹、齊、濮、濟、淄、萊、沂、單。



 昭武南京。



 肅戎曹。



 單勇單。



 安武鄆、齊。



 必敵鄆。



 決勝濟。



 靜山兗。



 勇敢密、沂。元符二年,環慶增置二百人。



 定邊徐。



 安東登、萊。



 衙隊曹。



 左衙南京、鄆。



 右衙南京、徐、鄆、曹、廣濟。



 開武、懷化曹。



 保寧、開遠濟。



 安平齊。



 武勇濰。



 靜海徐、濰、揚。



 崇順青。



 忠略淄。



 安海、水軍登。



 寧濟萊。



 建武密。



 壯武青、徐、曹、兗、密、濰、齊、濮、登、萊、淮陽。



 崇武濮。崇寧三年,詔於京西東、河東北、開封府界創置馬步軍五萬人,步軍以崇武為名。大觀四年,詔四輔州闕額,於崇武等軍內撥填。



 本城曹。



 京西路騎軍之額,自騎射而下六;步軍之額,自奉化而下二十有五,並改號曰勁武。凡四十五指揮,一萬五千
 一百五十人。



 橋道河陽。



 開道鄭。



 步驛襄。



 會通橋道西京。



 採造西京。



 牢城諸州。



 壯城西京、蔡、汝。



 馬監許州單鎮、鄭州原武、西京洛陽。



 三水磨鄭。



 東西八作西京。



 騎射西京、河陽、陳、許、鄭、穎、滑。



 威邊西京、河陽、鄭、蔡、襄、鄧、滑、穎、汝、郢、均、商、隨、唐、信陽、光化。



 定邊蔡。



 游奕許。



 衙隊陳。



 保忠滑。



 奉化鄭、許、陳、蔡、滑、穎。



 懷化許、穎。



 開山西京。



 隨身隨、唐。



 永安西京。



 耀武河陽、鄧。



 歸定河陽。



 壯武西京、陳、蔡、鄧、襄、穎、汝、光化。



 靜江陳、蔡、郢。



 三略陳。



 寧淮、忠順穎。



 崇寧汝。



 澄海襄。



 保定均、信陽。



 懷寧房。



 宣節郢。



 崇化光化。



 長劍滑。



 西懷化許。



 防城均。



 威勇西。



 廣濟陳。



 靜
 淮蔡。



 淮南路騎軍之額,自威邊而下六;步軍之額,自左衙而下二十有七,並改號曰寧淮。凡一百二指揮,四萬一千二百八十五人。



 橋道壽。



 水運泰。



 梢工都楚、真。



 雜作都壽。



 裝發真、泗、楚、通、和。



 水軍橋道泗。



 車軍真、楚。



 鹽車泰、真。



 新招梢工真、泗。



 拔頭水軍泗。



 牢城諸州。



 裝卸亳。



 剩員直亳永城。



 威邊亳、廬、宿。



 飛將亳。



 馬鬥宿。



 保勝光。



 武勝泗、濠。



 揀中騎射揚、廬、壽、毫、宿、泗、真、蘄、黃、濠、光、海、和、通、舒、滁、漣水、高郵、無為。



 左衙通。



 平難亳、濠。



 奉化揚、亳、廬、壽、宿、濠、和、通、泰、楚、舒、真、泗、滁、無為、漣水、高郵。



 開遠揚、楚、泗。



 六奇楚。



 武勇泰。



 懷安泰。



 靜海通。



 隨身宿。



 水
 軍揚、廬、壽、光、海、和、泰、楚、舒、真、蘄、黃、泗、漣水、高郵、無為。



 耀武泰。



 壯武揚、廬、壽、黃、光、海、和、通、蘄、楚、泰、舒、滁、高郵。



 寧淮、忠順、旌勇壽。



 靜勝揚。



 寧順廬。



 備邊泗。



 三捷滁。



 寧化舒。



 保勝光。



 懷仁蘄、黃。



 保節舒。



 廣濟宿、海、泰、通、泗、高郵、漣水。熙寧八年以六分為額。



 水軍奉化泰、泗。



 捍海通、泰。



 兩浙路步軍之額,自捍江而下三,並改號曰崇節。凡五十一指揮,一萬九千人。



 水軍諸州軍。



 船坊明。



 船務杭、婺。



 車軍常。



 採造明。



 樓店務杭。



 江橋院明。



 堰軍長安、京口、呂城、杉青。



 清務杭、蘇、婺、溫。



 捍江杭三。



 本城秀、常。



 鼓角將潤。



 江南路騎軍之額,揀中騎射一;步軍之額,自效勇而下五,並改號曰效勇。凡五十三指揮,一萬六千六百五十人。



 水軍江寧、洪、虔、宣、歙、饒、信、太平、池、江、吉、筠、撫、興國、臨江、南康、廣德。



 里運江寧。



 貢運饒。



 水軍臨江。



 梢工都洪。



 造船軍匠吉。



 步驛江寧。



 牢城諸州軍。



 壯城洪。



 下卸錢監江。



 鐵木匠營、酒務營、竹匠營池。



 酒務雜役江寧。



 揀中騎射宣、撫、江、吉、筠、袁、歙、太平、池、饒、信、廣信、南康、南安、建昌、臨江、興國。



 效勇江寧、廣德。



 本城南安。



 靜江南安。崇寧二年七月召募。



 武威江寧。



 保節洪、虔、江、池、饒、信、太平、吉、筠、袁、撫、興國。



 荊湖路騎軍之額,自騎射而下三;步軍之額,自左衙而下二十,並改號曰宣節。凡四十四指揮,一萬一千三百人。



 步驛荊門。



 水運潭。



 船坊潭、鼎。



 渡船都潭。



 清務、船坊鐵作潭。



 騎射江陵、潭、鄂、岳、安、澧、復、鼎、永、道、郴、邵、桂陽。



 威邊安。



 衙隊峽。



 左衙安。



 水軍江陵、潭、衡、永、郴、邵、鄂、岳、復、安、澧、峽、鼎、歸、漢陽、桂陽。



 寧遠復。



 壯城潭、岳、安、復。



 靜江江陵、潭、岳、鼎、衡、永、郴、全。



 三略鼎。



 寧淮澧。



 崇寧嶽。



 澄江辰。



 宣節南路諸州、軍、監。北路:岳、澧、鼎、荊門、諸監。熙寧七年九月,沅置一。大觀元年,靖置一。



 步捷全。



 威棹江陵、歸、峽。



 保節鄂。



 崇節潭。



 威勇安。



 牢城諸州軍。



 中軍將潭、江。



 揀中澧。



 揀中
 宣節潭、澧、鼎。



 鼓角將荊門。



 福建路步軍之額,自水軍而下三,並改號曰保節。凡三十三指揮,一萬一千一百五十人。



 水軍福、建、漳、泉、邵武。



 保節建、汀、南劍。



 崇節福、泉、興化。



 廣南路騎軍之額,自靜山而下二;步軍之額,自水軍而下十,並改號曰清化。凡八十二指揮,一萬二千七百人。



 步驛循、賀、封、梅、康、南雄、潮、韶。



 造船場廣。



 駕綱水軍廣。



 城面廣、端、惠、循、英、春、賀、梅、連、康、新、封、白、潮。



 遞角場雷。



 運錫循。



 牢城諸州。



 靜山宜。



 本城馬軍
 廣。



 水軍廣、惠、英、賀、封、連、康、南雄、春、廉、白、邕。



 靜江廣、韶、循、潮、連、梅、南雄、英、賀、封、端、新、康、春、惠、桂、容、邕、象、昭、梧、藤、龔、蒙、潯、貴、柳、宜、賓、橫、融、化、竇、高、雷、欽、鬱林、廉、瓊。



 澄海韶、循、潮、連、梅、南雄、英、賀、封、端、南恩、春、惠、桂、容、邕、象、昭、龔、蒙、潯、貴、柳、賓、橫、融、化、雷、竇、南儀、白、欽、鬱林、廉、崖、儋。並於配隸中選少壯者。



 巡海水軍廣。



 本城梅。



 寧海瓊。崇寧四年,廣南西路經略司請置刀牌手三千人,於桂州置營,候教閱習熟,分戍諸州。



 新招靜江邕。



 清化桂、容、邕、象、昭、梧、藤、蒙、龔、潯、貴、柳、宜、賓、橫、融、化、竇、高、南儀、雷、白、欽、鬱林、廉、瓊、儋。



 戰棹欽、廉。



 安遠桂。崇寧元年十月,詔川、陜招揀足額。



 四川路步軍之額,自開遠而下十,並改號曰克寧。凡一百一十一指揮,二萬三千四百人。



 自河北路至此,凡改號、指揮人數,並因元
 豐以前,其後增改,各隨軍額。



 橋道興。



 橋閣龍、劍、文、三泉。



 防河、羅城成都。



 牢城益、梓、利、夔。



 開遠利、劍。



 水軍興。



 靜江利。



 懷遠興元。



 廣塞興元、三泉。



 克寧成都、蜀、漢、雅、邛、嘉、綿、陵、彭、眉、簡、戎、榮、普、資、梓、合、瀘、遂、渠、昌、果、懷安、廣安、興元、洋、利、龍、劍、蓬、璧、文、興、安德、三泉、夔、渝、涪、萬、達、開、施、忠、雲安、大寧。



 威棹成都、嘉、眉、簡。梓州路諸州軍。劍、安德,夔、渝、涪、萬、雲安。



 懷信利。



 順化興。



 本城梁山。



 武寧元豐七年,詔成都府減廢武寧第八指揮,置馬軍騎射一。



 侍衛步軍司宣效、揀中宣效、揀中六軍、武嚴、左右龍武軍、左右羽林軍、左右神武軍、左右武肅、武和、忠靖、神衛剩員。軍頭司備軍。諸司庫務、河清、馬遞鋪等役卒。朝服
 法物庫、籍田司,隸太常寺。



 東西作坊、作坊物料庫、東西廣備、皮角庫,隸軍器監。



 車營、致遠務、養象所、左右騏驥院、左右天駟監、牧養上下監、鞍轡庫、駝坊、皮剝所、御輦院,隸太僕寺。



 文思院、綾錦院、西染裁造院,隸少府監。



 軍器衣甲庫、儀鸞司、左右金吾仗司、左右街司、六軍儀仗司、軍器什物庫,隸衛尉寺。



 河清、街道司,隸都水監。



 修內司、東西八作司、竹木務、東西退材場、事材場、東西窯務、作坊物料庫,隸將作監。



 御廚、翰林司、牛羊司、法酒庫、內酒坊、外物料庫、醋庫、油庫,隸光祿寺。



 左藏庫、布庫、香藥庫、都茶庫、左右廂店宅務、修造。



 榷貨務、祗候庫,隸太府寺。



 修倉司、四園苑、都水磨、排岸司、裝卸、金池明雜役,隸司農寺。



 醴泉觀、萬壽觀、集禧觀、西太一宮、禮賓院,隸鴻臚寺。



 廣固,隸修治京城所。



 孳生監,隸樞密院。



 府界諸門馬遞鋪,隸尚書駕部。



 已上並元豐以前所隸,後皆因之。



 建炎後禁廂兵



 威果安吉、嘉興、杭、平江、常、嚴、鎮江、紹興、慶元、溫、臺、婺、處、隆興、江、寧國、南康、潭、永、衢、道、邵武、建寧、南劍、全、福、興化、漳、汀。



 全捷中興立。杭、婺、安吉、平江、泉、鎮江、紹興、慶元、寧國、寶慶、福。



 雄節杭、安吉、嘉興、平江、常、嚴、溫、鎮江、紹興、江陰、慶元、臺、婺、處。



 武衛鎮江、紹興、溫、婺、潭。



 威
 捷杭、溫、鎮江、紹興、婺、潭。



 雄捷中興立。紹興。



 威勝中興立。寶慶、慶元。



 翼虎中興立。隆興。



 雄略中興立。吉、潭、永、衢、隆興、全、福、廣、容。



 忠節中興立。隆興、撫、臨江、寧國、江、建昌、興國、南康。



 武雄撫、隆興、江、建昌、吉、興國、南安、袁、臨江、寧國、南康。



 靖安中興立。全、寶慶。



 靜江桂陽、郴、衡、道、全。



 廣節中興立。邵武、福、漳、建寧、南劍、興化、汀。



 廣二、廣三指揮中興立。泉。



 親效中興立。廣。



 澄海廣、循、連、南雄、封、英德、南恩、惠、潮、藤、容、賀、德、慶、昭、高、欽、雷。



 建炎後廂兵



 武嚴、宣效、壯役中興立。



 備軍中興立。



 神衛剩員隸侍衛步軍,中興隸廂軍。



 廣豐倉剩員中興立。



 廣效中興有揀中廣效,在廣效立。



 御營喝探中興,在京師。



 武和開封一指揮。中興,左右二指揮,在京。



 武肅中興,在京師。



 忠靖一指揮,開封,屬步軍。



 奉化屬步軍,三指揮。中興有揀中奉化,在奉化上。



 勁武在京。



 崇勝一指揮。中興有揀中崇勝,在崇勝上。



 雄猛一指揮。



 保寧中興有揀中保寧,在保寧上。



 寧淮中興,在淮南。



 捍江杭。



 宣節中興,在寶慶、潭、永、武岡、郴、衡、全、桂陽、靖、道、沅。



 效勇中興,江東、西。



 保節中興,五指揮。



 克寧中興,四川。



 寧江中興立。一指揮。



 清化中興,廣西。



 牢城諸州,以待有罪配隸人。



 崇節中興,杭、安吉、平江、江陰、常、嚴、鎮江、溫、慶元、臺、婺、江東西。



 開江中興,平江。



 橫江中興,平江、杭。



 寧節中興,臺、福、寧國、建寧、靖。



 清務中興,婺。



 山場中興,婺。



 效勇中興,隆興、撫、贛、建昌、興國、南安、袁、吉、臨江、寧國、南康。



 靖安中興立。潭、永、常德。



 靜江二指揮。



 威果見禁軍。



 雄略中興,四指揮。



 澄海中興,武岡、全。



 豐國監中興立。建
 寧。



 駕綱中興立。



 長運中興立。



 修江中興,杭。



 都作院中興,杭。



 小作院中興立。杭。



 清湖閘中興立。杭。



 開湖司中興立。杭。



 北城堰中光立。杭。



 西河廣濟中興立。杭。



 樓店務中興,杭。



 長安堰閘中興立。杭。



 秤斗務中興立。杭。



 壯城帥府望郡立之。



 鼓角匠、船務中興,杭。



\end{pinyinscope}