\article{本紀第十九}

\begin{pinyinscope}

 徽宗一



 徽宗體神合道駿烈遜功聖文仁德憲慈顯孝皇帝,諱佶,神宗第十一子也,母曰欽慈皇后陳氏。元豐五年十月丁巳生於宮中。明年正月賜名,十月授鎮寧軍節度
 使、封寧國公。哲宗即位,封遂寧郡王。紹聖三年,以平江、鎮江軍節度使封端王,出就傅。五年,加司空,改昭德、彰信軍節度。元符三年正月己卯,哲宗崩,皇太后垂簾,哭謂宰臣曰:「國家不幸,大行皇帝無子,天下事須早定。」章惇又曰:「在禮律當立母弟簡王。」皇太后曰:「神宗諸子,申王長而有目疾,次則端王當立。」惇厲聲對曰:「以年則申王長,以禮律則同母之弟簡王當立。」皇太后曰:「皆神宗子,莫難如此分別,於次端王當立。」知樞密院曾布曰:「章
 惇未嘗與臣等商議,如皇太后聖諭極當。」尚書左丞蔡卞、中書門下侍郎許將相繼曰:「合依聖旨。」皇太后又曰:「先帝嘗言,端王有福壽,且仁孝,不同諸王。」於是惇為之默然。乃召端王入,即皇帝位,皇太后權同處分軍國事。



 庚辰,赦天下常赦所不原者,百官進秩一等,賞諸軍。遣宋淵告哀於遼。辛巳,尊先帝後為元符皇后。癸未,追尊母貴儀陳氏為皇太妃。甲申,命章惇為山陵使。乙酉,出先帝遺留物賜近臣。丙戌,以申王佖為太傅,進封陳王,
 賜贊拜不名。丁亥,進仁宗淑妃周氏、神宗淑妃邢氏並為貴妃,賢妃宋氏為德妃。戊子,以章惇為特進,封申國公。己丑,進封莘王俁為衛王,守太保;簡王似為蔡王,睦王偲為定王,並守司徒。罷增八廂邏卒。



 二月己亥,始聽政。尊先帝妃朱氏為聖瑞皇太妃。壬寅,以南平王李乾德為檢校太師。丁未,立順國夫人王氏為皇后。庚戌,向宗回、宗良遷節度使,太后弟侄未仕者俱授以官。癸示,初御紫宸殿。庚申,以吏部尚書韓忠彥為門下侍郎,資
 政殿大學士黃履為尚書右丞。辛酉,名懿親宅潛邸曰龍德宮。甲子,毀承極殿。丙寅,遣吳安憲、朱孝孫以遺留物遺遼國主。三月戊辰朔,詔宰臣、執政、侍從官各舉可任臺諫者。庚午,遣韓治、曹譜告即位於遼。辛未,詔追封祖宗諸子光濟等三十三人為王,女四十八人為公主。甲申,以西蕃王隴拶為河西軍節度使,尋賜姓名曰趙懷德,邈川首領瞎徵為懷遠軍節度使。己丑,以日當食,降德音於四京:減囚罪一等,流以下釋之。庚寅,錄趙普
 後。辛卯,詔求直言。癸巳,以寧遠軍節度觀察留後世雄為崇信軍節度使,封安定郡王。乙未,卻永興民王懷所進玉器。



 夏四月丁酉朔,日有食之。己亥,令監司分部決獄。甲辰,以韓忠彥為尚書右僕射兼中書侍郎,禮部尚書李清臣為門下侍郎,翰林學士蔣之奇同知樞密院事。乙巳,錄曹佾後。丁未,以帝生日為天寧節。己酉,長子但生。辛亥,大赦天下,應元符二年已前系官逋負悉蠲之。癸丑,鹿敏求等以應詔上書遷秩。乙卯,請大行皇帝
 謚於南郊。丁巳,詔范純仁等復官、宮觀,蘇軾等徙內郡居住。癸亥,罷編類臣僚章疏局。乙丑,賜禮部奏名進士及第、出身五百十八人。



 五月丁卯朔,罷理官失出之罰。丙子,詔復廢後孟氏為元祐皇后。乙酉,蔡卞罷。己丑,詔追覆文彥博、王珪、司馬光、呂公著、呂大防、劉摯等三十三人官。辛卯,還司馬光等致仕遺表恩。癸巳,河北、河東、陜西饑,詔帥臣計度振恤。



 六月丙申朔,遼主遣蕭進忠、蕭安世等來吊祭。



 秋七月丙寅朔,奉皇太后詔,罷同聽
 政。丁卯,告哲宗欽文睿武昭孝皇帝謚於天地、宗廟、社稷。戊辰,上寶冊於福寧殿。癸酉,以皇太后還政,減天下囚罪一等,流以下釋之。癸未,遣陸佃、李嗣徽報謝於遼。罷管勾陜西、京、川路坑冶及江西、廣東、湖北、夔、梓、成都路管勾措置鹽事官。辛卯,封子但為韓國公。



 八月戊戌,詔諸路遇民有疾,委官監醫往視疾給藥。庚子,作景靈西宮,奉安神宗神御,建哲宗神御殿於其西。辛丑,出內庫金帛二百萬糴陜西軍儲。壬寅,葬哲宗皇帝於永泰陵。
 丙午,遣董敦逸賀遼主生辰,呂仲甫賀正旦。戊申,高麗王王熙遣使奉表來慰。庚戌,詔以仁宗、神宗廟永世不祧。戊午,以蔡王似為太保。癸亥,祔哲宗神主於太廟,廟樂曰《大成之舞》。



 九月甲子,詔修《哲宗實錄》。丙寅,遼遣蕭穆來賀即位。丁卯,減兩京、河陽、鄭州囚罪一等,民緣山陵役者蠲其賦。己巳,幸龍德宮。辛未,章惇罷。丙子,以陳王佖為太尉。丁丑,詔修《神宗史》。己丑,復均給職田。



 十月乙未,夏國入貢。丙申,蔡京出知永興軍,貶章惇為武昌軍
 節度副使。丁酉,以韓忠彥為尚書左僕射兼門下侍郎。壬寅,以曾布為尚書右僕射兼中書侍郎。乙卯,升端州為興慶軍。己未,詔禁曲學偏見、妄意改作以害國事者。辛酉,罷平準務。



 十一月丁卯,詔修《六朝寶訓》。降德音於端州:減囚罪一等,徒以下釋之。庚午,詔改明年元。戊寅,以觀文殿學士安燾知樞密院事。庚辰,黃履罷。己丑,置《春秋》博士。辛卯,令陜西兼行銅、鐵錢。以禮部尚書范純禮為尚書右丞。十二月甲午,以皇太后不豫,禱於宮觀、
 祠廟、岳瀆。戊戌,出廩粟減價以濟民。辛丑,慮囚。甲辰,詔修《國朝會要》。戊申,降德音於諸路:減囚罪一等,流以下釋之。戊午,遼人來賀正旦。是歲,出宮女六十九人。



 建中靖國元年春正月壬戌朔,有赤氣起東北,亙西南,中函白氣。將散,復有黑昆在旁。癸亥,有星自西南入尾,其光燭地。癸酉,范純仁薨。甲戌,皇太后崩,遺詔追尊皇太妃陳氏為皇太后。丁丑,易大行皇太后園陵為山陵,命曾布為山陵使。己卯,令河、陜募人入粟,免試注官。



 二
 月丙申,雨雹。己亥,汰秦鳳路土兵。甲辰,始聽政。乙巳,出內庫及諸路常平錢各百萬,備河北邊儲。丁巳,貶章惇為雷州司戶參軍。



 三月甲子,始御紫宸殿。乙丑,遼使蕭恭來告其主洪基殂,遣謝瓘、上官均等往吊祭,黃寔賀其孫延禧立。丁丑,詔以河西軍節度使趙懷德知湟州。壬午,以日當食,避殿減膳,降天下囚罪一等,流以下釋之。



 夏四月辛卯朔,日食不見。甲午,上大行皇太后謚曰欽聖憲肅。乙未,上追尊皇太后謚曰欽慈。丁酉,御殿
 復膳。壬寅,詔諸路疑獄當奏而不奏者科罪,不當奏而輒奏者勿坐,著為令。



 五月辛酉朔,大雨雹。詔三省減吏員、節冗費。丙寅,葬欽聖憲肅皇后、欽慈皇后於永裕陵。庚辰,蘇頌薨。丙戌,祔欽聖憲肅皇后、欽慈皇后神主於太廟。戊子,減兩京、河陽、鄭州囚罪一等,民緣山陵役者蠲其賦。



 六月庚寅朔,以韓國公但為開府儀同三司,封京兆郡王。戊申,封向宗回為永陽郡王,向宗良為永嘉郡王。甲寅,封吳王顥子孝騫為廣陵郡王,頵子孝參為
 信都郡王。戊午,范純禮罷。己未,詔班《鬥殺情理輕重格》。



 秋七月辛巳,內郡置添差宗室闕。丙戌,安燾罷。丁亥,以蔣之奇知樞密院事,吏部尚書陸佃為尚書右丞,端明殿學士章楶同知樞密院事。



 九月己巳,詔諸路轉運、提舉司及諸州軍,有遺利可以講求及冗員浮費當裁損者,詳議以聞。丙戌,子示聖薨。



 冬十月乙未,李清臣罷。丁酉,天寧節,群臣及遼使初上壽於垂拱殿。



 十一月庚申,以陸佃為尚書左丞,吏部尚書溫益為尚書右丞。壬戌,以
 西蕃賒羅撒為西平軍節度使、邈川首領。辛未,出禦制南郊親祀樂章。戊寅,朝獻景靈宮。己卯,饗太廟。庚辰,祀天地於圜丘,赦天下。改彰信軍為興仁軍,昭德軍為隆德軍。改明年元。十二月壬辰,賜陳王佖詔書不名。癸卯,進神宗昭儀武氏為賢妃。丙午,奉安神宗神御於景靈西宮大明殿。丁未,詣宮行禮。己酉,降德音於四京,減囚罪一等,徒以下釋之。是歲,遼人來獻遺留物。河東地震,京畿蝗,江、淮、兩浙、湖南、福建旱。



 崇寧元年春正月丁丑,太原等十一郡地震,詔死者家賜錢有差。



 二月丙戌朔,以聖瑞皇太妃疾,慮囚。甲午,子但改名烜。以蔡確配饗哲宗廟庭。戊戌,詔:「士有懷抱道德、久沉下僚及學行兼備、可厲風俗者,待制以上各舉所知二人。」奉議郎趙諗謀反,伏誅。庚子,封子煥為魏國公。辛丑,聖瑞皇太妃薨,追尊為皇太后。庚戌,追封孔鯉為泗水侯,孔伋為沂水侯。



 三月丁巳,奉安哲宗神御於景靈西宮寶慶殿。戊午,詣宮行禮。壬戌,以定王偲為太
 保。壬申,幸定王第。



 夏四月己亥,上皇太后謚曰欽成。



 五月丁巳,熒惑入斗。庚申,韓忠彥罷。己巳,瞎征卒。庚午,降復太子太保司馬光為正議大夫,太師文彥博為太子太保,餘各以差奪官。辛未,詔待制以上舉能吏各二人。乙亥,黜後苑內侍請以箔金飾宮殿者。丙子,詔元祐諸臣各已削秩,自今無所復問,言者亦勿輒言。戊寅,葬欽成皇后於永裕陵。己卯,陸佃罷。庚辰,以許將為門下侍郎,溫益為中書侍郎,翰林學士承旨蔡京為尚書左丞,
 吏部尚書趙挺之為尚書右丞。



 六月己丑,祔欽成皇后神主於太廟。壬辰,減西京、河陽、鄭州囚罪一等,民緣山陵役者蠲其賦。癸卯,詔六曹尚書有事奏陳,許獨員上殿。己酉,太白晝見。壬子,改渝州為恭州。癸丑,詔仿《唐六典》修神宗所定官制。封伯夷為清惠侯,叔齊為仁惠侯。



 閏月甲寅朔,更名哲宗神御殿曰重光。辛酉,慮囚。壬戌,曾布罷。甲子,詔諸路州縣官有治績最著者,許監司、帥臣各舉一人。壬午,追貶李清臣為武安軍節度副使。癸
 未,詔監司、帥臣於本路小使臣以上及親民官內,有智謀勇果可備將帥者,各舉一人。



 秋七月甲申朔,建長生宮以祠熒惑。丙戌,詔省、臺、寺、監及監司、郡守,並以三年成任。戊子,以蔡京為尚書右僕射兼中書侍郎。己丑,焚元祐法。甲午,詔於都省置講議司。詔杭州、明州置市舶司。庚子,章楶罷。甲辰,以雨水壞民廬舍,詔開封府振恤壓溺者。辛亥,罷《春秋》博士。



 八月乙卯,子烜改名桓,煥改名楷。乙丑,罷權侍郎官。辛未,置安濟坊,養民之貧病者,
 仍令諸郡縣並置。甲戌,詔天下興學貢士,建外學於國南。丙子,詔司馬光等二十一人子弟毋得官京師。己卯,以趙挺之為尚書左丞,翰林學士張商英為尚書右丞。



 九月戊子,京師置居養院,以處鰥寡孤獨,仍以戶絕財產給養。乙未,詔中書籍元符三年臣僚章疏姓名為正上、正中、正下三等,邪上、邪中、邪下三等。丁酉,治臣僚議復元祐皇后及謀廢元符皇后者罪,降韓忠彥、曾布官,追貶李清臣為雷州司戶參軍,黃履為祁州團練副使,
 竄曾肇以下十七人。己亥,籍元祐及元符末宰相文彥博等、侍從蘇軾等、餘官秦觀等、內臣張士良等、武臣王獻可等凡百有二十人,御書刻石端禮門。庚子,以元符末上書人鐘世美以下四十一人為正等,悉加旌擢;範柔中以下五百餘人為邪等,降責有差。時世美已卒,詔贈官,仍官其子一人。壬寅,貶曾布為武泰軍節度副使。甲辰,詔:「元符三年、建中靖國元年責降臣僚已經牽復者,其元責告命並繳納尚書省。」冬十月癸亥,蔣之奇罷。戊
 辰,詔責降宮觀人不得同一州居住。甲戌,以御史錢遹、石豫、左膚及輔臣蔡京、許將、溫益、趙挺之、張商英等言,罷元祐皇后之號,復居瑤華宮。丙子,劉奉世等二十七人坐元符末黨與變法,並罷祠祿。戊寅,以資政殿學士蔡卞知樞密院事。



 十一月乙酉,邵州言知溪洞徽州楊光銜內附。戊子,以婉儀鄭氏為賢妃。辛卯,置河北安濟坊。癸巳,置西、南兩京宗正司及敦宗院。戊戌,置顯謨閣學士、待制官。戊申,子楷為開府儀同三司,封高密郡王。
 己酉,立卿、監、郎官三歲黜陟法。十二月癸丑,論棄湟州罪,貶韓忠彥為崇信軍節度副使,曾布為賀州別駕,安燾為寧國軍節度副使,范純禮分司南京。庚申,鑄當五錢。辛酉,贈哲宗子鄧王茂為皇太子,謚獻愍。丁丑,詔:「諸邪說詖行非先聖賢之書,及元祐學術政事,並勿施用。」是歲,京畿、京東、河北、淮南蝗。江、浙、熙河、漳、泉、潭、衡、郴州、興化軍旱。辰、沅州徭入寇。出宮女七十六人。



 二年春正月辛巳朔。乙酉,竄任伯雨、陳瓘、龔居□、鄒浩於
 嶺南,馬涓等九人分貶諸州。知荊南舒但平辰、沅州猺賊,復誠、徽州,改誠州為靖州,徽州為蒔竹縣。壬辰,溫益卒。乙巳,以復荊湖疆土,曲赦兩路。丙午,以冱寒,令監司分部決獄。丁未,以蔡京為尚書左僕射兼門下侍郎。



 二月辛亥,安化蠻入寇,廣西經略使程節敗之。壬子,遣官相度湖南、北猺地,取其材植入供在京營造。甲寅,進元符皇后為太后,宮名崇恩。辛酉,置殿中監。癸亥,奉安哲宗御容於西京會聖宮及應天院。丙子,置諸路茶場。



 三
 月壬午,進仁宗充儀張氏為賢妃。乙酉,減西京囚罪一等。詔黨人子弟毋得擅到闕下,其應緣趨附黨人、罷任在外、指射差遣及得罪停替臣僚亦如之。丁亥,御集英殿策進士。癸卯,賜禮部奏名進士及第、出身五百三十八人,其嘗上書在正等者升甲,邪等者黜之。



 夏四月甲寅,詔侍從官各舉所知二人。乙卯,於闐入貢。丁卯,詔毀呂公著、司馬光、呂大防、范純仁、劉摯、範百祿、梁燾、王巖叟景靈西宮繪像。己巳,以初謁景靈宮,赦天下。乙亥,詔
 毀刊行《唐鑒》並三蘇、秦、黃等文集。戊寅,以趙挺之為中書侍郎,張商英為尚書左丞,戶部尚書吳居厚為尚書右丞,兵部尚書安惇同知樞密院事。奪王珪贈謚,追毀程頤出身文字,其所著書令監司覺察。



 五月辛巳,以賢妃鄭氏為淑妃。癸未,以陳王佖為太師。丙戌,貶曾布為廉州司戶參軍。己亥,封子楫為楚國公。丙午,冊元符皇後劉氏為太后。六月壬子,冊王氏為皇后。庚申,詔:「元符末上書進士,類多詆訕,令州郡遣入新學,依大學自訟
 齋法,候及一年,能革心自新者許將來應舉,其不變者當屏之遠方。」壬戌,慮囚。是月,中太一宮火。復湟州。



 秋七月己卯,學士院火。辛巳,以復湟州,進蔡京官三等,蔡卞以下二等。壬午,白虹貫日。甲申,降德音於熙河蘭會路:減囚罪一等,流以下釋之。庚寅,曾肇責授濮州團練副使。辛卯,詔上書進士見充三舍生者罷歸。丁酉,詔自今戚里宗屬勿復為執政官,著為令。乙巳,詔責降人子弟毋得任在京及府界差遣。



 八月丁未朔,再論棄湟州罪,
 貶韓忠彥為磁州團練副使,安燾為祁州團練副使,範純禮為靜江軍節度副使,削蔣之奇秩三等。戊申,張商英罷。辛酉,詔張商英入元祐黨籍。



 九月辛巳,詔宗室不得與元祐奸黨子孫為婚姻。庚寅,封子樞為吳國公。詔:「上書邪等人,知縣以上資序並與外祠,選人不得改官及為縣令。」壬辰,置醫學。癸巳,令天下郡皆建崇寧寺。辛丑,改吏部選人自承直郎至將仕郎七階。令天下監司長吏廳各立《元祐奸黨碑》。甲辰,詔郡縣謹祀社稷。冬十
 一月庚辰,以元祐學術政事聚徒傳授者,委監司舉察,必罰無赦。



 十二月癸亥,祧宣祖皇帝、昭憲皇后。丙寅,詔六曹長貳歲考郎官治狀,分三等以聞。是歲,諸路蝗。纂府蠻楊晟銅、融州楊晟天、邵州黃聰內附。



 三年春正月己卯,安化蠻降。辛巳,詔上書邪等人毋得至京師。戊子,鑄當十大錢。壬辰,增縣學弟子員。甲午,賜蔡京子攸進士出身。癸卯,太白晝見。甲辰,鑄九鼎。



 二月丙午,以淑妃鄭氏為貴妃。以刊定元豐役法不當,黜錢
 遹以下九人。丁未,置漏澤園。己酉,詔王珪、章惇別為一籍,如元祐黨。詔自今御後殿,許起居郎、舍人侍立。壬子,以楚國公楫為開府儀同三司,封南陽郡王。庚申,令天下坑冶金銀復盡輸內藏。辛未,雨雹。



 三月辛巳,置文繡院。丁亥,作圜土,以居強盜貸死者。甲午,躋欽成皇后神主於欽慈皇后之上。辛丑,大內災。



 夏四月乙巳,以火災降德音於四京:減囚罪一等,流以下原之。乙卯,復鄯州,建為隴右都護府。辛酉,徙封楫為樂安郡王。復廓州。乙
 丑,罷講議司。己巳,曲赦陜西。壬申,楫薨。



 五月戊寅,罷開封權知府,置牧、尹、少尹。改定六曹,以吏、戶、儀、兵、刑、工為序,增其員數,仿《唐六典》易胥吏之稱。己卯,以復鄯、廓,蔡京為守司空,封嘉國公。庚辰,許將、趙挺之、吳居厚、安惇、蔡卞各轉三官。甲申,改鄯州為西寧州,仍為隴右節度。辛丑,詔黜守臣進金助修宮庭者。



 六月壬寅朔,圖熙寧、元豐功臣於顯謨閣。癸酉,以王安石配饗孔子廟。丙午,增諸州學未立者。壬子,置書、畫、算學。占城入貢。戊午,詔
 復位祐、元符黨人及上書邪等者合為一籍,通三百九人,刻石朝堂,餘並出籍,自今毋得復彈奏。辛酉,復置太醫局。癸亥,慮囚。乙丑,詔內外官毋得越職論事,僥幸奔競,違者御史臺彈奏。



 秋七月癸酉,以婉儀王氏為德妃。庚辰,詔自今大禮不受尊號,群臣毋上表。辛卯,行方田法。



 八月庚子,詔諸路知州、通判增入「主管學事」四字。壬寅,大雨,壞民廬舍,令收瘞死者。甲辰,蔡京上《神宗史》。丙午,許將罷。



 九月乙亥,以趙挺之為門下侍郎,吳居厚為
 中書侍郎,翰林學士承旨張康國為尚書左丞,刑部尚書鄧洵武為尚書右丞。壬辰,詔諸路州學別置齋舍,以養材武之士。



 冬十月辛居朔,大雨雹。丁未,賢妃張氏薨。丙辰,命官編類六朝勛臣。戊午,夏人入涇原,圍平夏城,寇鎮戎軍。庚申,熙河蘭會路經略安撫使王厚言,河西軍節度使趙懷德等出降。己巳,立九廟,復翼祖、宣祖。庚午,貴妃邢氏薨。



 十一月甲戌,幸太學,官論定之士十六人,遂幸闢雍,賜國子司業吳絪、蔣靜四品服,學官推恩
 有差。丙戌,封子杞為冀國公。丁亥,詔取士並繇學校,罷發解及省試法,科場如故事。癸巳,更上神宗謚曰體元顯道帝德王功英文烈武欽仁聖孝皇帝,加上哲宗謚曰憲元繼道顯德定功欽文睿武齊聖昭孝皇帝。甲午,朝獻景靈宮。乙未,饗太廟。丙申,祀昊天上帝於圜丘,赦天下。升興仁、隆德軍為府,還彰信、昭德舊節。十二月乙巳,升通遠軍為鞏州。戊午,賜陳王佖入朝不趨。是歲,諸路蝗。出宮女六十二人。廣西黎洞楊晟免等內附。



\end{pinyinscope}