\article{志第一百十 選舉三(學校試 律學等試附)}

\begin{pinyinscope}

 凡學皆隸國子監。國子生,以京朝七品以上子孫為之,初無定員,後以二百人為額。太學生,以八
 品以下子弟若庶人之俊異者為之。及三舍法行,則太學始定置外舍生二千人,內舍生三百人,上舍生百人。始入學,驗所隸州公據,試補外舍,齋長、諭月書其行藝於籍。行謂率教不戾規矩,藝謂治經程文。季終考於學諭,次學錄,次正,次博士,後考於長貳。歲終會其高下,書於籍,以俟復試,參驗而序進之。凡私試,孟月經義,仲月論,季月策。凡公試,初場經義,次場論策。試上舍,如省試法。凡內舍,行藝與所試之業俱優,為上舍上等,取旨授官;一優一平為中等,以俟殿試;俱平若一優一否為下等,以俟省試。



 元祐間,置廣文館生
 二千四百人,以待四方游士試京師者。律學生無定
 員,他雜學廢置無常。崇寧建闢雍於郊,以處貢士,而三舍考選
 法乃
 遍天下。於是由州郡貢之闢雍,由闢雍升之太學,而學校之制益詳。凡國子以奏蔭恩廣,故
 學校
 不預考
 選,其得入官賜出身者,多由銓試。



 初,國子監因周舊制,頗增學舍,以應蔭子孫隸學受業。開寶八年,國子監上言:「生徒舊數七十人,奉詔分習《五經》,然系籍者或久不至,而在京進士、諸科,常赴講
 席肄業,請以補監生之闕。」詔從之。



 景德間,許文武升朝官嫡親附國學取解,而遠鄉久寓京師,其文藝可稱,有本鄉命官保任,監官驗之,亦聽附學充貢。



 仁宗時,士之服儒術者不可勝數。即位初,賜兗州學田,已而命藩輔皆得立學。慶歷四年,詔曰:「儒者通天、地、人之理,明古今治亂之原,可謂博矣。然學者不得騁其說,而有司務先聲病章句以拘牽之,則吾豪雋奇偉之士,何以奮焉?士有純明樸茂之美,而無學學養成之法,使與不肖並進,
 則夫懿德敏行,何以見焉?此取士之甚敝,而學者自以為患。夫遇人以簿者,不可責其厚也。今朕建學興善,以尊子大夫之行;更制革敝,以盡學者之才。有司其務嚴訓導、精察舉,以稱朕意。學者其進德修業,無失其時。其令州若縣皆立學,本道使者選部屬官為教授,員不足,取於鄉里宿學有道業者。」由是州郡奉詔興學,而士有所勸矣。



 天章閣侍講王洙言:「國子監每科場詔下,許品官子投狀試藝,給牒充廣文、太學、律學三館學生,多致
 千餘。就試試已,則生徒散歸,講官倚席,但為游寓之所,殊無肄習之法。居常聽講者,一二十人爾。」乃限在學滿五百日,舊已嘗充貢者止百日。本授官會其實,京朝官保任,始預秋試,每十人與解三人。凡入學授業,月旦即親書到歷。如遇私故或疾告、歸寧,皆給假,違程及期月不來參者,去其籍。後諫官餘靖極言非便,遂罷聽讀日限。



 初立四門學,自八品至庶人子弟充學生,歲一試補。差學官鎖宿、彌封校其藝,疏名上聞而後給牒,不中式者仍
 聽讀,若三試不中,則出之。未幾,學廢。



 時太學之法寬簡,而上之人必求天下賢士,使專教導規矩之事。安定胡瑗設教蘇、湖間二十餘年,世方尚詞賦,湖學獨立經義治事齋,以敦實學。皇祐末,召瑗為國子監直講,數年,進天章閣侍講,猶兼學正。其初人未信服,謗議蜂起,瑗強力不倦,卒以有立。每公私試罷,掌儀率諸生會於首善,雅樂歌詩,乙夜乃散。士或不遠數千里來就師之,皆中心悅服。有司請下湖學,取其法以教太學。



 神宗尤垂意
 儒學,自京師至郡縣,既皆有學。歲時月各有試,程其藝能,以差次升舍,其最優者為上舍,免發解及禮部試而特賜之第。遂專以此取士。



 太學生員,慶歷嘗置內舍生二百人。熙寧初,又增百人,尋詔通額為九百人。四年,盡以錫慶院及朝集院西廡建講書堂四,諸生齋舍、掌事者直廬始僅足用。自主判官外,增置直講為十員,率二員共講一經,令中書遴選,或主判官奏舉。生員厘為三等:始入學為外舍,初不限員,後定額七百人;外舍升內
 舍,員二百;內舍升上舍,員百。各執一經,從所講官受學,月考試其業,優等上之中書。其正、錄、學諭,以上舍生為之,經各二員;學行卓異者,主判、直講復薦之中書,奏除官。始命諸州置學官,率給田十頃贍士。初置小學教授。帝嘗謂王安石曰:「今談經者人人殊,何以一道德?卿所著經,其以頒行,使學者歸一。」八年,頒王安石《書》、《詩》、《周禮義》於學官,是名《三經新義》。



 元豐二年,頒《學令》:太學置八十齋,齋各五楹,容三十人。外舍生二千人,內舍生三百
 人,上舍生百人。月一私試,歲一公試,補內舍生;間歲一舍試,補上舍生,彌封、謄錄如貢舉法;而上舍試則學官不預考校。公試,外舍生入第一、第二等,升內舍;內舍生試入優、平二等,升上舍:皆參考所書行藝乃升。上舍分三等。學正增為五人,學錄增為十人,學錄參以學生為之。歲賜緡錢至二萬五千,又取郡縣田租、屋課、息錢之類,增為學費。初,以國子名監,而實未嘗教養國子。詔許清要官親戚入監聽讀,額二百人,仍盡以開封府解額歸太學,
 其國子生解額,以太學分數取之,毋過四十人。



 哲宗時,初置在京小學,曰:「就傅」、「初筮」,凡兩齋。復取太學額百人還開封府。先是,開封解額稍優,四方士子多冒畿縣戶,又隸太學不及一年不該解試者,亦往往冒戶。禮部按舊制,凡試國子監者,先補中廣文館生,乃投牒求試。元祐七年,遂依仿其法,立廣文館生。惟開封府元解百人許自試,其嘗取諸科二百、國子額四十者,皆以為本館解額。遇貢舉年試補館生,中者執牒詣國子監驗試,凡試
 者十人取一,開封考取亦如之。紹興元年,罷廣文館,其額悉復還之開封府、國子監。



 元祐新令,罷推恩之制。紹聖初,監察御史郭知章言:「先帝立三舍法,以歲月稽其行實,故入上舍而中上等者,得不經禮部試,特命以官。責備而持久,故其得也難,誘掖激勸,莫善於此。宜復元豐法,以廣樂育之德。」又請三學補外舍生,依元豐令一歲四試。於是詔:「太學生悉用元豐制推恩,上等即注官者,歲毋過二人;免禮部試者,每舉五人而止;免解試者
 二十人而止。仍計數對除省試發解額,其元祐法勿用。諸三舍升補等法,悉推行舊制。」



 三年,三省言:「元祐試補太學生不嚴,茍務多取,後試者無闕可撥,宜遵元豐初制,雖在籍生亦重試。」乃詔在籍生再試,許取三分,創求補者半之;惟上舍生及是年充貢員內舍、外舍先自元豐補入者免再試,餘非再試而中者皆降舍。蔡京上所修《內外學制》,始頒諸天下。



 元符元年,詔許命官補國子生,毋過四十人。凡太學試,令優取《二禮》,許占全額之半,
 而以其半及他經。復置《春秋》博士。二年,初令諸州行三舍法,考選、升補,悉如太學。州許補上舍一人,內舍二人,歲貢之。其上舍附太學外舍,試中補內舍生,三試不升舍,遣還其州。其內舍免試,至則補為外舍生。諸路選監司一員提舉學校,守貳董干其事。遇補試上、內舍生,選有出身官一人,同教授考選,須彌封、謄錄。三年,太學試補外舍改用四季,學官自考,不謄錄,仍添試論一場。



 崇寧元年,宰臣請:「天下州縣並置學,州置教授二員,縣亦
 置小學。縣學生選考升諸州學,州學生每三年貢太學。至則附試,別立號。考分三等:入上等補上舍,入中等補下等上舍,入下等補內舍,餘居外舍。諸州軍解額,各以三分之一充貢士。開封府留五十五額,解士人之不入學者,餘盡均給諸州,以為貢額。外官子弟親戚,許入學一年,給牒至太學,用國子生額解試。州給常平或系省田宅充養士費,縣用地利所出及非系省錢。」三年,始定諸路增養縣學弟子員,大縣五十人,中縣四十人,小縣
 三十人。凡州縣學生曾經公、私試者復其身,內舍免戶役,上舍仍免借借如官戶法。



 命將作少監李誡,即城南門外相地營建外學,是為闢雍。蔡京又奏:「古者國內外皆有學,周成均蓋在邦中,而黨庠、遂序則在國外。臣親承聖詔,天下皆興學貢士,即國南郊建外學以受之,俟其行藝中率,然後升諸太學。凡此聖意,悉與古合。今上其所當行者:太學專處上舍、內舍生,而外學則處外舍生。今貢士盛集,欲增太學上舍至三百人,內舍六百人,
 外舍三千人。外學為四講堂、百齋,齋列五楹,一齋可容三十人。士初貢至,皆入外學,經試補入上舍、內舍,始得進處太學。太學外舍,亦令出居外學。其敕、令、格、式,悉用太學見制。國子祭酒總治學事,外學官屬,司業、丞各一人,稍減太學博士、正、錄員歸外學,仍增博士為十員,正、錄為五員,學生充學諭者十人,直學二人。」三舍生皆由升貢,遂罷國子監補試。



 又置諸王宮大、小學教授,立考選法,凡奉祠及仕而解官或需次者,悉許入內、外學。任
 子不系州土,隨所寓入學,仍別齋居處,別號試考。曾升補三舍生,後從獻助得官,其入學視任子法。凡任子,不問文武,須隸學滿一年,始得求試。乃詔取士悉由學校升貢,其州郡發解及試禮部並罷。自是,歲試上舍,悉差知舉,如禮部試。



 五年,著令:



 凡縣學生隸學已及三月,不犯上二等罰,聽次年試補州學外舍,是名「歲升」。開封祥符生員,即闢雍別為齋,教養、升進如縣學法。願入鄰縣學者聽。惟赤縣校試,主以博士。每歲正月,州以公試上
 舍及歲升員,一院鎖宿,分為三試。其公試,上舍率十取其六為中格;中格已,以其名第自上而下參考察之籍;既在籍,又中選,即六人之中取其四,以差升舍。其歲升中選者,得補外舍生。開封屬縣附闢雍別試,中者入闢雍充外舍。隸學三年,經兩試不預升貢,即除其籍,法涉太嚴。今令三年內三經公試不預選,兩經補內舍、貢上舍不及格,且曾犯三等以上罰,若外舍,即除籍罷歸縣,內舍降外舍,已嘗降而私試不入等,若曾犯罰,亦除籍,
 再赴歲升試。



 凡州學上舍生升舍,以其秋即貢入闢雍,長吏集闔郡官及提學官,具宴設以禮敦遣,限歲終悉集闕下。自川、廣、福建入貢者,給借職券,過二千里給大將券,續其路食,皆以學錢給之。如有孝弟、睦姻、任恤、忠和,若行能尤異為鄉里所推,縣上之州,免試入學。州守貳若教授詢審無謬,即保任入貢,具實以聞,不實者坐罪有差。



 太學試上舍生,本慮與科舉相並,試以間歲。今既罷科舉,又諸州歲貢士,其改用歲試。每春季,太學、闢
 雍生悉公試,同院混取,總三百七十四人。以四十七人為上等,即推恩釋褐;一百四十人為中等,遇親策士許入試;一百八十七人為下等,補內舍生。凡上等上舍生暨特舉孝弟行能之士,不待廷試推恩者,許即引見釋褐。上舍仍先以試文卷進入,得可乃引賜。若上舍已該釋褐恩,而貢入在廷試前一年者,須在學又及半年,不犯上二等罰,乃得注官。



 凡貢士入闢雍外舍,三經試不與升補,兩經試不入等,仍犯上三等罰者,削籍再赴本
 州歲升試,是名「退送」。即內舍已降舍,而又一試不與,或兩犯上四等罰者,亦如外舍法退送。太學外舍生巳預考察者,許再經一試,以中否為留遣,餘升降、退送悉如闢雍法。



 凡有官人不入學而願試貢士者,不以文、武、雜出身,悉許之,惟贓私罪廢人則否。應試者,隨內外附貢士公試,皆別考,率以七人取一人。即預貢者,與闢雍春試貢士通考。中選入上等者,升差遣兩等,賜上舍出身;文行優者,奏聞而殊擢之。中等俟殿試,下等補內舍,不
 隸學,需再試。已仕在官而願試者,悉準此制。



 凡在外官同居小功以上親,及其親姊妹女之夫,皆得為隨行親,免試入所任鄰州郡學。其有官人願學於本州者,亦免試,升補悉如諸生法,混試同考,惟升舍不侵諸生額,自用七人取一。若中者多,即以溢額名次理為考察。若所親移替,願改籍他州學者聽。



 太學上、內舍既由闢雍升入,又已罷科舉,則國子監解額無所用,盡均撥諸府、諸州解額,三分之,以為三歲貢額,並令有司均定以聞。太
 學舊制,止分立優、平二等,自今欲令闢雍、太學試上舍中程者,皆參用察考,以差升補。其考察試格,悉分上、中、下三等。貢士則以本州升貢等第,太學內舍則以校定等第。每上舍試考已定,知舉及學官以中試之等參驗於籍,通定升絀高下,兩上為上,一上一中及兩中為中,一上一下及一中下、兩下為下。若兩格名次等第適皆齊同,即以試等壓考察之格,餘率以是為差,仍推其法達之諸州。凡內外私試,始改用仲月,並試三場,試論日
 仍添律義。凡考察悉準在學人數,每內舍十人取五,外舍十人取六,自上而下分為三等籍,以俟上舍考定而參用之。



 是歲,貢士至闢雍不如令者,凡三十有八人,皆罷歸,而提學官皆罰金。建州浦城縣學生,隸籍者至千餘人,為一路最,縣丞徐秉哲特遷一官。



 初立八行科,詔曰:「學以善風俗,明人倫,而人材所自出也。今法制未立,殆無以厲天下。成周以六行賓興萬民,否則威之以不孝、不弟之刑。近因稽周法,立八行、八刑,頒之學校,兼行
 懲勸,庶幾於古。士有善父母為孝,善兄弟為悌,善內親為睦,善外親為姻,信於朋友為任,仁於州里為恤,知君臣之義為忠,達義利之分為和。凡有八行實狀,鄉上之縣,縣延入學,審考無偽,上其名於州。州第其等,孝、悌、忠、和為上,睦、姻為中,任、恤為下。茍備八行,不俟終歲,即奏貢入太學,免試補為上舍。司成以下審考不誣,申省釋褐,優命之官;不能全備者,為州學上等上舍,餘有差。」八刑則反八行而麗於罪,各以其罪名之。縣上其名於州,
 州稽於學,毋得補弟子員。然品目既立,有司必求其跡以應令,遂有牽合瑣細者。自元祐創經明行修科,主德行而略辭藝,間取禮部試黜之士,附寘恩科,當時固已咎其無所甄別。及八行科立,則三舍皆不試而補,往往設為形跡,求與名格相應。於是兩科相望幾數十年,乃無一人卓然能自著見者,而八行又有甚敝。蓋後世欲追古制,而不知風俗教化之所從出,其難固如此夫。



 開封始建府學,立貢士額凡五十,而士子不及三百,盡額
 而取,則涉太優,欲稍裁之。詔:「王畿立學,若不優誘使進,何以首善?其常解五十勿闕。」



 大觀元年,詔願兼他經者,量立升進之法。大抵用本經決去取,而兼經所中等第特為升貢。每歲附公試院而別異其號,每十五人取一人,分上、中、下等,別榜示之,唱名日,甄別奏聞,與升甲,皆優於專經者。異時內外學官闕,皆得在選。縣學生三不赴歲升試及三赴歲升試而不能升州學者,皆除其籍。諸路賓興會試闢雍,獨常州中選者多,州守若教授俱
 遷一官。



 政和四年,小學生近一千人,分十齋以處之,自八歲至十二歲,率以誦經書字多少差次補內舍。若能文,從博士試本經、小經義各一道,稍通補內舍,優補上舍。又詔:「學校教養額少,則野有遺士,應諸路學校及百人以上者,三分增一。」七年,試高麗進士權適等四人,皆賜上舍及第,遣歸其國。時宰臣留意學校,因事究敝,有司考閱防閑益密。先是,禮部上《雜修御試貢士敕令格式》,又取舊制凡關學政者,分敕、令、格、式,成書以上。用給
 事中毛友言,初試補入縣學生,並簾試以別偽冒。徽宗崇尚老氏之學,知兗州王純乞於《御注道德經》注中出論題,範致虛亦乞用《聖濟經》出題。



 宣和元年,帝親取貢士卷考定,能深通《內經》者,升之以為第一。三年,詔:「罷天下州縣學三舍法,惟太學用之課試。開封府及諸路,並以科舉取士。太學官吏及州縣嘗置學官,凡元豐舊制所有者皆如故,其闢雍官屬及宗學並諸路提舉學事官屬並罷,內外學悉遵元豐成憲。」七年,詔:「政和中嘗命
 學校分治黃、老、莊、列之書,實失專經之旨,其《內經》等書並罷治。」



 崇寧以來,士子各徇其黨,習經義則詆元祐之非,尚詞賦則誚新經之失,互相排斥,群論紛紛。欽宗即位,臣僚言:「科舉取士,要當質以史學,詢以時政。今之策問,虛無不根,古今治亂,悉所不曉。詩賦設科,所得名臣,不可勝紀,專試經義亦已五紀。救之之術,莫若遵用祖宗成憲。王安石解經,有不背聖人旨意,亦許採用。至於老、莊之書及《字說》,並應禁止。」詔禮部詳議。諫議大夫兼
 祭酒楊時言:「王安石著為邪說,以塗學者耳目,使蔡京之徒,得以輕費妄用,極侈靡以奉上,幾危社稷。乞奪安石配饗,使邪說不能為學者惑。」御史中丞陳過庭言:「《五經》義微,諸家異見,以所是者為正,所否者為邪,此一偏之大失也。頃者指蘇軾為邪學,而加禁甚切;今已弛其禁,許採其長,實為通論。而祭酒楊時矯枉太過,復詆王氏以為邪說,此又非也。」諸生習用王學,聞時之言,群起而詆詈之,時引避不出,齋生始散。詔罷時祭酒。而諫議
 大夫馮澥、崔鶠等復更相辨論,會國事危,而貢舉不及行矣。



 建炎初,即行在置國子監,立博士二員,以隨幸之士三十六人為監生。紹興八年,葉綝上書請建學,而廷臣皆以兵興食貴運為辭。十三年,兵事稍寧,始建太學,置祭酒、司業各一員,博士三員,正、錄各一員,養士七百人:上舍生三十員,內舍生百員,外舍生五百七十員。凡諸道住本州學滿一年,三試中選,不犯第三等以上罰,或不住學而曾兩預釋奠及齒於鄉飲酒者,聽充弟子員。
 每歲春秋兩試之,旋命一歲一補,於是多士雲集,至分場試之。俄又詔三年一試,增至千員,中選者皆給綾紙贊詞以寵之。每科場四取其一。



 自外舍有月校,而公試入等曰內舍;自內舍有月校,而舍試入等曰上舍;凡升上舍者,皆直赴廷對。二十七年,立定制:春季放補,遇省試年改用孟夏。



 舊,太學遇覃恩無免解法,孝宗始創行之。在朝清要官,許牒期親子弟作待補國子,別號考校。如太學生遇有期親任清要官,更為國子生,不預校定、
 升補及差職事,惟得赴公、私試,科舉則混試焉。



 淳熙中,命諸生暇日習射,以鬥力為等差,比類公、私試,別理分數。自中興以來,四力之士,有本貫在學公據,皆得就補。帝始加限節,命諸路州軍以解試終場人數為準,其薦貢不盡者,令百取六人赴太學,謂之「待補生」;其住本學及游學之類,一切禁止。元豐舊制,內舍生校定,分優、平二等。優等再赴舍試,又入優,則謂之兩優釋褐,中選者即命以京秩,除學官。至是,始令先注職官,代還,注職事
 官,恩例視進士第二人。舊校定歲額五六分為優選者,增為十分矣。



 光宗初,公試始令附省場別院。紹熙三年,禮部侍郎倪思請復混補法,命兩省、臺諫雜議可否。於是吏部尚書趙汝愚等合奏曰:「國家恢儒右文,京師、郡縣皆有學,慶歷以後,文物彬彬。中興以來,建太學於行都,行貢舉於諸郡,然奔竟之風勝,而忠信之俗微。亦惟榮辱升沉,不由學校;德行道藝,取決糊名;工雕篆之文,無進修之志;視庠序如傳舍,目師儒如路人;季考月書,
 盡成文具。今請重教官之選,假守貳之權;仿舍法以育材,因大比以取士;考終場之數,定所貢之員;期以次年,試於太學。其諸州教養、課試、升貢之法,下有司條上。」思議遂寢。四年,詔國子監試中、上等小學生,比類諸州待補中選之額,放補一次。



 寧宗慶元、嘉定中,始兩行混補。於是增外舍生為千四百員,內舍校定,不系上舍試年分,以八分為優等。又以國子生員多偽濫,命行在職事官期親、厘務官子孫乃得試補。嘉定十四年,詔自今待
 補百人取三人。舊法,自外舍升內舍,雖有校試,必公試合格,乃許升補。蓋私試皆學官自考,而公試則降敕差官。至是,歲終許取外舍生校最優者一人升內舍。



 理宗復百取六人之制。紹定二年,以待補生自外方來參齋者,間有鬻帖偽冒之弊。遂命中選之人,召升朝保官二員批書印紙,仍命州郡守倅結罪保明,比照字跡無偽,方許簾引注籍;犯者治罪,罰及保官。五年,以省試下第及待補生之群試於有司者,有請托賄求之弊,學官考
 文,有親故交通之私,命今後兩學補試,並從廟堂臨時選差,即令入院;凡用度,則用國子監供給學官事例。未幾,監察御史何處久又言:「宜遵舊制,以武學、宗學補試,並就兩學於大院排日引試,有親嫌人依避房法。且士子試卷頗多,考官頗少,期日既迫,費用不敷。」乃增給用度,仍添差考官五員。寶祐元年,復命分路取放補試員數,以免遠方士子道路往來之費及都城壅並之患。三年,復試於京師。



 度宗咸淳二年正月,幸太學,謁先聖,禮
 成,推恩三學:前廊與免省試,內舍、上舍及已免省試者與升甲;起居學生與泛免一次,內該曾經兩幸人與補上州文學,如願在學者聽。其在籍諸生,地遠不及趁赴起居者,三學申請乞並行泛免一次,命特從之。凡諸生升舍在幸學之前者,方許陳乞恩例。七年正月,以壽和聖福皇太后兩上尊號,推恩三學,在齋生員並特與免解赴省一次。九年,外舍生晏泰亨以七分三厘乞理為第三優,朝命不許,遂申嚴學法,今後及八分者方許歲
 校三名,如八分者止有一人,而援次優、三優之例者,亦須止少三、二厘,方可陳乞特放,庶不盡廢學法,當亦不過一人而止。



 律學。國初置博士,掌授法律。熙寧六年,始即國子監設學,置教授四員。凡命官、舉人皆得入學,各處一齋。舉人須得命官二人保任,先入學聽讀而後試補。習斷按,則試按一道,每道敘列刑名五事或七事;習律令,則試大義五道,中格乃得給食。各以所習,月一公試、三私試,略
 如補試法。凡朝廷有新頒條令,刑部即送學。其犯降舍殿試者,薄罰金以示辱,餘用太學規矩,而命官聽出宿。尋又置學正一員,有明法應格而守選者,特免試注官,使兼之,月奉視所授官。後以教授一員兼管干本學規矩,仍從太學例給晚食。元豐六年,用國子司業朱服言,命官在學,如公試律義、斷案俱優,準吏部試法授官;太學生能兼習律學,中公試第一,比私試第二等。



 政和間,詔博士、學正依大理寺官除授,不許用無出身人及以
 恩例陳請。生徒犯罰者,依學規;仍犯不改,書其印歷或補牒,參選則理為闕失。



 建炎三年,復明法新科,進士預薦者聽試。紹興元年,復刑法科。凡問題,號為假案,其合格分數,以五十五通分作十分,以所通定分數,以分數定等級:五分以上入第二等下,四分半以上入第三等上,四分以上入第三等中。以曾經試法人為考官。五年,以李洪嘗中刑法入第二等,命與改秩,中書駁之。趙鼎謂:「古者以刑弼教,所宜崇獎。」高宗曰:「刑名之學久廢,不
 有以優之,則其學絕矣。」卒如前詔。後議者謂得解人取應,更不兼經,白身得官,反易於有官試法。乃命所試斷案、刑名,全通及粗通以十分為率,斷及五分、《刑統》義文理全通為合格,及雖全通而斷案不及分數者勿取。仍自後舉兼經。十五年,罷明法科,以其額歸進士,惟刑法科如舊。二十五年,四川類省始附試刑法。



 淳熙七年,秘書郎李巘言:「漢世儀、律、令同藏於理官,而決疑獄者必傅以古義。本朝命學究兼習律令,而廢明法科;後復明
 法,而以三小經附。蓋欲使經生明法,法吏通經。今所試止於斷案、律義,斷案稍通、律義雖不成文,亦得中選,故法官罕能知書。宜令習大法者兼習經義,參考優劣。」帝曰:「古之儒者,以儒術決獄,若用俗吏,必流於刻。」乃從其奏,詔自今第一、第二、第三場試斷案,每場各三道,第四場大經義一道,小經義二道,第五場《刑統》律義五道。明年,命斷案三場,每場止試一道,每道刑名十件,與經義通取,四十分以上為合格,經義定去留,律義定高下。



 寧
 宗慶元三年,以議臣言罷經義,五年又復。嘉定二年,臣僚上言:「試法設科,本以六場引試,後始增經義一場,而止試五場,律義又居其一,斷案止三場而已,殊失設科之初意。且考試類多文士,輕視法家,惟以經義定去留,其弊一也。法科欲明憲章,習法令,察舉明比附之精微,識比折出入之錯綜,酌情法於數字之內,決是非於片言之間。比年案題字多,專尚困人,一日之內,僅能謄寫題目,豈暇深究法意,其弊二也。刑法考官不過曾中法
 科丞、評數人,由是請托之風盛,換易之弊興,其弊三也。今請罷去經義,仍分六場,以五場斷案,一場律義為定。問題稍減字數,而求精於法律者為試官,各供五六題,納監試或主文臨時點定。如是,讞議得人矣。」從之。六年,以議者言法科止試《刑統》,是盡廢理義而專事法律,遂命復用經義一場,以《尚書》、《語》、《孟》題各一篇及《刑統》大義,通為五場。所出經題,不必拘刑名倫類,以防預備,以斷案定去留,經義為高下,仍禁雜流入貲人收試。八年,罷
 四川類試刑法科。



 初,凡試法科者,皆取撰成見義挾入試場。理宗淳祐三年,令刑部措置關防,其考試則選差大理丞、正歷任中外有聲望者,不許止用新科評事未經作縣之人。逮其試中,又當仿省試、中書覆試之法,質以疑獄,觀其讞筆明允,始與差除。時所立等第,文法俱通者為上,徑除評事;文法粗通者為次,與檢法;不通者駁放。



 度宗咸淳元年,申嚴選試之法,凡引試刑法官,命題一如《紹興式》。八年,以試法科者少,特命考試命題,務
 在簡嚴,毋用長語。有過而願試者,照見行條法,除私罪應徒、或入己贓、失入死罪並停替外,作犯輕罪者,與放行收試。或已經三試終場之人,已歷三考,赴部參注,命本部考核元試,果有所批分數,不須舉狀,與注外郡刑法獄官差使一次,庶可激厲誘掖。格法,試法科者,批及八分,方在取放之數。咸淳末,有僅及二分以上者,亦特取一名,授提刑司檢法官,寬以勸之也。



 初,宗學廢置無常。凡諸王屬尊者,立小學於其宮。其子
 孫,自八歲至十四歲皆入學,日誦二十字。其已授環衛官、有學藝得召試遷轉者每有之,然非有司常試,乃特恩也。熙寧十年,始立《宗子試法》。凡祖宗袒免親已受命者,附鎖廳試;自袒免以外,得試於國子監。禮部別異其卷而校之,十取其五,舉者雖多,解毋過五十人。廷試亦不與進士同考。年及四十、嘗累舉不中,疏其名以聞而錄用之。其官於外而不願附各路鎖試,許謁告試國子監。



 崇寧初,疏屬年二十五,以經義、律義試禮部合格,分
 二等附進士榜,與三班奉職,文優者奏裁。其不能試及試而黜者,讀律於禮部,推恩與三班借職,勿著為令。及兩京皆置敦宗院,院皆置大、小學教授,立考選法,如《熙寧格》出官,所蒞長貳或監司有二人任之,乃注授。後又許見在任者,於本任附貢士試。大觀三年,宗子釋褐者十二人。宗學官,須宗子中上舍第且有行者,方始為之。四年,詔:「宗子之升上舍,不經殿試,遽命之官,熙寧法不如是。其依貢士法,俟殿試補入上、中等者,唱名日取裁。」
 後又定上等賜上舍及第,中等賜出身,授官有差。凡隸學,有篤疾若親老無兼侍者,大宗正察其實,罷歸。宣和二年,詔罷量試出官之法。



 紹興二年,帝初策士及宗子於集英殿。五年,初復南省試。十四年,始建宗學於臨安,生員額百人:大學生五十人,小學生四十人,職事各五人。置諸王宮大、小學教授一員。在學者皆南宮、北宅子孫,若親賢宅近屬,則別選館職教授。初,行在宗室試國子監者,有官鎖廳,七取其三;無官應舉,七取其四;無官
 袒免親取應,文理通為合格,不限其數;而外任主宮觀、嶽廟試乾轉運司者,取放之額同進士。十五年,命諸路宗室願赴行在試者,依熙寧舊制,並國子監請解;不願者,依崇寧通用貢舉法,所以優國族也。



 孝宗登極,凡宗子不以服屬遠近、人數多寡,其曾獲文解兩次者,並直赴廷試,略通文墨者,量試推恩。習經人本經義二道,習賦人詩賦各一首,試論人論一首,仍限二十五歲以上,合格,第一名承節郎,餘並承信郎。曾經下省人,免量試,
 推恩。四川則附試於安撫制置司。於是入仕者驟逾千人。隆興元年,詔量試不中、年四十以上補承信郎,展三年出官,餘並於後舉再試。四月,御射殿引見取應省試第一人,賜同進士出身,第二、第三人補保義郎,餘四十人承節郎,七人承信郎。凡宗室鎖廳得出身者,京官進一秩,選人比類循資;無官應舉得出身者,補修職郎;濮、秀二王下子孫中進士舉者,更特轉一秩。



 乾道五年,命宗室職事隨侍子弟許赴國子監補。六年,臣僚上言:「神
 宗朝,始立教養、選舉宗子之法。保義至秉義,鎖試則與京秩,在末科則升甲,取應不過量試注官,所以寵異同姓,不與寒畯等也。然曩時向學者少,比年雋異者多,或冠多士,或登詞科,幾與寒士齊驅;而入仕浸煩,未知裁抑,非所以示至公也。」於是禮部請鎖廳登第者,舊於元官上轉行兩官,自今止依元資改授,餘準舊制。十二年,右正言胡衛請:「自今宗室監試,無官應舉,照鎖廳例七取其二;省試則三舉所放人數如取應例,立為定額。」從
 之。



 寧宗嘉定四年,詔鎖廳應舉,省試第一名,殿試唱名授官日,於應得恩例外,更遷一秩。九年,以宮學並歸宗庠,教授改為博士、宗諭。十四年,命前隸宮學近屬,令附宗學公、私試,中選者與正補宗學生,近屬子孫年十五以下者,許試小學生。復置諸王宮大、小學教授一員。宗學解試依太學例取放,每舉附國子監發解所,異題別考。



 理宗寶慶二年,以鎖廳宗子第一名若搢學深《春秋》,秀出譜籍,與補保義郎,特賜同進士出身,仍換修職郎。
 端平元年,命宗子鎖廳應舉解試,凡在外州軍,或寄居,或見任隨侍,及見寓行在就試者,各召知識官委保正身,國子監取其宗子出身、訓名、生長左驗,以憑保收試,仍於試卷家狀內具保官職位、姓名,以防欺詐。淳祐二年,建內小學,置教授二員,選宗子就學。寶祐元年五月,特、正奏名進士宗子必晃等二人特授保義郎,若瑰等二十九人承節郎,敕略曰:「必晃等取應及選,咸補右階,蓋欲誘之進學,而教以入仕也。其毋以是自畫焉。」



 度宗
 咸淳元年,以鎖廳應舉宗子兩請,舉人遇即位赦恩,並赴類試。其曾經覆試文理通者,照例升等;文理不通及未經覆試者則否;第五等人特與免銓出官。九年,凡無官宗子應舉,初生則用乳名給據,既長則用訓名。其赴諸路漕司之試,有一人前後用兩據、印二卷者。至是,命漕司並索乳名、訓名各項公據,方許收試,以杜奸弊。



 武舉、武選。咸平時,令兩制、館閣詳定入官資序故事,而未及行。仁宗時,法置武學,既而中輟。天聖八年,親試武舉
 十二人,先閱其騎射而試之,以策為去留,弓馬為高下。



 神宗熙寧五年,樞密請建武學於武成王廟,以尚書兵部郎中韓縝判學,內藏庫副使郭固同判,賜食本錢萬緡。生員以百人為額,選文武官知兵者為教授。使臣未參班與門蔭、草澤人召京官保任,人材弓馬應格,聽入學,習諸家兵法。教授纂次歷代用兵成敗、前世忠義之節足以訓者,講釋之。願試陣隊者,量給兵伍。在學三年,具藝業考試等第推恩,未及格者,逾年再試。凡試中,三
 班使臣與三路巡檢、砦主,未有官人與經略司教隊、差使,三年無過,則升至大使臣,有兩省、待制或本路鈐轄以上三人保舉堪將領者,並兼諸衛將軍,外任回,歸環衛班。



 科場前一年,武臣路分都監、文官轉運判官以上各奏舉一人,聽免試人學。生員及應舉者不過二百人。春秋各一試,步射以一石三斗,馬射以八斗,矢五發中的;或習武伎,副之策略,雖弓力不及,學業卓然:並為優等,補上舍生,毋過三十人。試馬射以六斗,步射以九斗,
 策一道,《孫》、《吳》、《六韜》義十道,五通補內舍生。馬步射、馬戰應格,對策精通、士行可稱者,上樞密院審察試用;雖不應格而曉術數、知陣法、智略可用,或累試策優等,悉取旨補上舍;武藝、策略累居下等,復降外舍。



 先是,樞密院修《武舉試法》,不能答策者,答兵書墨義。王安石奏曰:「三路義勇藝入三等以上,皆有旨錄用,陛下又欲推府界保甲法於三路,則武力之人已多。近以學究一科,從誦書不曉理廢之,而武舉復試墨義,則亦學究之流,無補
 於事。先王收勇力之士,皆屬於車右者,欲以備禦侮之用,則記誦何所施?」於是悉從中書所定。凡武舉,始試義、策於秘閣,武藝則試於殿前司,及殿試,則又試騎射及策於庭。策、武藝俱優為右班殿直,武藝次優為三班奉職,又次借職,末等三班差使、減磨勘年。策入平等而武藝優者除奉職,次優借職,又次三班差使、減磨勘年,武藝末等者三班差使。八年,詔武舉與文舉進士,同時鎖試於貢院,以防進士之被黜而改習者,遂罷秘閣試。又
 以《六韜》本非全書,止以《孫》、《吳》書為題。



 元豐元年,立《大小使臣試弓馬藝業出官法》:第一等,步射一石,矢十發三中,馬射七斗,馬上武藝五種,《孫》、《吳》義十通七,時務邊防策五道文理優長,律令義十通七,中五事以上免短使、減一任監當,三事以上免短使、升半年名次,兩事升半年,一事升一季;第二等,步射八斗,矢十發二中,馬射六斗,馬上武藝三種,《孫》、《吳》義十通五,策三道成文理,律令義十通五,中五事免短使、升半年,三事升半年,兩事升一
 季,一事與出官;第三等,步射六斗,矢十發一中,馬射五斗,馬上武藝兩種,《孫》、《吳》義十通三,策三道成文理,律令義十通三,計算錢穀文書五通三,中五事升半年,三事升一季,兩事與出官。其步射並發兩矢,馬射發三矢,皆著為格。四年,罷試律義。七年,止試《孫》、《吳》書大義一場,第一等取四通、次二等三通、三等二通為中格。元祐四年,詔解試、省試增策一道。



 崇寧間,諸州置武學。立《考選升貢法》,仿儒學制,其武藝絕倫、文又優特者,用文士上舍
 上等法,歲貢釋褐;中等仍隸學俟殿試。凡試出官使臣,仍赴殿前司呈試。諸州武士試補,不得文士同一場。馬射三上垛,九斗為五分,八斗為四分,七斗為三分。九斗、八斗、七斗再上垛及一上垛,視此為差,理為分數。馬射一中帖當兩上垛,一中的當兩中帖。



 舊制,武舉三年一試,命官不過三十餘人,後增額,以每貢者三人即取一以升上舍,積迭增展,遂至百人入流,比文額太優。大觀四年,詔自今貢試上舍者,取十人入上等,四十人入中等,五
 十人入下等,皆補充武學內舍,人材不足聽闕之,餘不入等者,處之外舍。大抵以弓馬程文兩上一上、兩中一中、兩下一下相參以為第。凡州教諭,須州都監乃得兼,吏部取武舉、武士上舍出身者。



 政和三年,以隸學者眾,凡經三歲校試而不得一與者,除其籍。宣和二年,尚書省言:「州縣武學既罷,有願隸京城武學者,請用元豐法補試。舊制,不入學而從保舉以試者,附試武學外舍,通取一百人。偕上舍生發解。今既罷科舉,請依元豐法奏
 舉,歲終集闕下,免試補外舍生,赴次年公試。其春選升補推恩,依大觀法。」



 靖康元年,詔諸路有習武藝、知兵書者,州長貳以禮遣送詣闕,毋限數,將親策而用之。



 建炎三年,詔武舉人先經兵部驗視弓馬於殿前司,仍權就淮南轉運司別場附試《七書》義五道,兵機策二首。紹興五年,帝御集英殿策武舉進士,翌日閱試騎射,策入優等與保義、承節郎,平等承信郎,其武藝不合格者,與進義校尉。川、陜宣撫司類省試武藝合格人並補官。十二
 年,御試,正奏名,策入優等承節郎,平等承信郎、進義校尉;特奏名,平等進義校尉,各展磨勘有差。十六年,始建武學。兵部上《武士弓馬及選試去留格》,凡初補入學,步射弓一石,若公、私試步騎射不中,即不許試程文,其射格自一石五斗以下至九斗,凡五等。



 二十六年,帝見武學頹弊,因諭輔臣曰:「文武一道也,今太學就緒,而武學幾廢,恐有遺才。」詔兵部討論典故,參立新制。凡武學生習《七書》兵法、步騎射,分上、內、外三舍,學生額百人,置博
 士一員,以文臣有出身或武舉高選人為之;學諭一員,以武舉補官人為之。凡補外舍,先類聚五人以上附私試,先試步射一石弓,不合格不得試程文,中格者依文士例試《七書》義一道。其內舍生私試,程文三在優等,弓馬兩在次優,公試入等,具名奏補。試上舍者,以就試人三取其一,以十分為率,上等一分,中等二分,下等七分,仍以三年與發解同試。凡內舍補上舍,以上舍試合格入等與行藝相參,兩上者為上等,一上一中或兩中及
 一上一下為中等,一中一下或兩下、一上一否為下等,仍不犯第三等罰、士行可稱者,具名奏補。二十七年,御試第一名趙應熊武藝絕倫,又省試第一,特與保義郎、閣門祗候。二十九年,修立武舉入官資格;命武舉人自今依府監年數免解。



 孝宗隆興元年禦試,得正奏名三十七人。殿中侍御史胡沂言:「唐郭子儀以武舉異等,初補右衛長史,歷振遠、橫塞、天德軍使。國初,試中武藝人並赴陜西任使。又武舉中選者,或除京東捉賊,或三路
 沿邊,試其效用,或經略司教押軍隊、準備差使,今率授以榷酤之事,是所取非所用,所用非所學也。請取近歲中選人數,量其材品、考任,授以軍職,使之習練邊事,諳曉軍旅,實選用之初意也。」



 乾道二年,中書舍人蔣芾亦以為言,請以武舉登第者悉處之軍中。帝以問洪適,適對曰:「武舉人以文墨進,雜於卒伍非便也。」帝曰:「累經任者,可以將佐處之。」是歲,以登極推恩,武舉進士比文科正奏名例,第一名升一秩為成忠郎,第二、第三名依第
 一名恩例。



 五年,兵部請外舍有校定人,參考榜上等者,候滿一年,私試四入等及不犯三等以上罰,或有校定而參考在中下等,候再試參考入中等,聽升補外舍生,赴公試。舊,除射親許試五等弓外,步射、馬射止許試第三等以下弓,程文雖優而參考弓馬分數難以對入優等;自今許比上舍法,不以馬、步、射親,並通試五等。



 吏部言:「武舉比試、發解、省試三場,依條以策義考定等第,具字號,會封彌所,以武藝並策義參考。今比試自依舊法,
 其解、省兩場,請依文士例,考定字號,先具奏聞,拆號放榜。」從之。初命武學生該遇登極覃恩,曾升補內舍或在學及五年曾經公、私試中人,並令赴省。是歲廷試,始依文科給黃牒,榜首賜武舉及第,餘並賜武舉出身。其年,頒武舉之法。令四川帥臣、憲、漕、知州軍監及寄居侍從以上各舉武士一員,興元府、利閬金洋階成西和鳳州各三員,拔其尤者送四川安撫司,解試類省,並如文科。



 淳熙元年,議者請:「武學外舍生有校定公試合格,令試
 五等弓馬,與程文五等相參,入中上等者,據闕升補,餘俟再試入等升補。」從之。帝御幄殿,引見正奏名,呈試武藝。二年,以武科授官與文士不類,詔自今第一人補秉義郎,堂除諸司計議官,序位在機宜之上;第二、第三人保義郎,諸路帥司準備將領,代還,轉忠翊郎;第四、第五人承節郎,諸路兵馬監押,代還,轉保義郎:皆仿進士甲科恩例。



 四年,以文科狀元代還,例除館職,亦召武舉榜首為閣門舍人。五年,始立武學國子額,收補武臣親屬;
 其文臣親屬,願附補者亦聽。七年,初立《武舉絕倫並從軍法》:凡願從軍者,殿試第一人與同正將,第二、第三名同副將,五名以上、省試第一名、六名以下並同準備將;從軍以後,立軍功及人材出眾者,特旨擢用。帝曰:「武舉本求將帥之材,今前名皆從軍,以七年為限,則久在軍中,諳練軍政,他日可備委任。」八年,命特奏名補官,展減磨勘有差。九年,議者以為從軍之人,率多養望,不屑軍旅。詔自今職事勤恪者,從主帥保奏升差,懈惰者按劾。



 光宗紹熙元年,武臣試換文資,南渡以前許從官三人薦舉,紹興令敦武郎以下聽召保官二人,以經義、詩賦求試,其後太學諸生久不第者,多去從武舉,已乃鎖廳應進士第。凡以秉義或忠翊皆換京秩,恩數與第一人等。後以林穎秀言:「武士舍棄弓矢,更習程文,褒衣大袖,專做舉子。夫科以武名,不得雄健喜功之士,徒啟其僥幸名爵之心。」於是詔罷鎖廳換試。



 寧宗即位,復其制,慶元五年,命兩淮、京西、湖北諸郡仿兵部及四川法,於本
 道安撫司試武士,合格者,赴行在解試,別立字號,分項考校,撥十名為解額,五名省額。



 理宗紹定元年,命武舉進士避親及所舉之人止避本廳,令無妨嫌官引試,若合格,則朝廷別遣官復試。淳祐九年,以北兵屢至,命極邊、次邊一體收試,仍量增解額五名、省額二名。是歲,武舉正奏名王時發已系從軍之人,充殿前司左軍統領,既登第,換授,特命就本職上與帶「同」字,以示優厚勸獎。



 度宗咸淳六年,命禮部貢院於武舉進士平等每百人
 內,取放待補十人,絕倫每百人內,取待補十三人。



 算學。崇寧三年始建學,生員以二百一十人為額,許命官及庶人為之。其業以《九章》、《周髀》及假設疑數為算問,仍兼《海島》、《孫子》、《五曹》、張丘建夏侯陽算法並歷算、三式、天文書為本科。本科外,人占一小經,願占大經者聽。公私試、三舍法略如太學。上舍三等推恩,以通仕、登仕、將仕郎為次。大觀四年,以算學生歸之太史局,並書學生入翰林書藝局,畫學生入翰林圖畫局,醫學生入太醫局。



 紹興初,命太史局試補,並募草澤人。淳熙元年春,聚局生子弟試歷算《崇天》、《宣明》、《大衍歷》三經,取其通習者。五年,以《紀元歷》試。九年,以《統元歷》試。十四年,用《崇天》、《紀元》、《統元歷》三歲一試。紹熙二年,命今歲春銓太史局試,應三全通、一粗通,合格者並特收取,時局生多闕故也。嘉定四年,命局生必俟試中,方許轉補。



 理宗淳祐十二年,秘書省言:「舊典以太史局隸秘省,今引試局生不經秘書,非也。稽之於令,諸局官應試歷算、天文、三式官,每歲
 附試,通等則以精熟為上,精熟等則以習他書多為上,習書等則以占事有驗為上。諸局生補及二年以上者,並許就試。一年試歷算一科,一年試天文、三式兩科,每科取一人。諸同知算造官闕有試,翰林天文官闕有試,諸靈臺郎有應試補直長者,諸正名學生有試問《景祐新書》者,諸判局闕而合差,諸秤漏官五年而轉資者,無不屬於秘書;而局官等人各置腳色,遇有差遣、改補、功過之類,並申秘書。今乃一切自行陳請,殊乖初意。自今
 有違令補差,及不經秘書公試補中者,中書執奏改正,仍從舊制,申嚴試法。」從之。



 書學生,習篆、隸、草三體,明《說文》、《字說》、《爾雅》、《博雅》、《方言》,兼通《論語》、《孟子》義,願占大經者聽。篆以古文、大小二篆為法,隸以二王、歐、虞、顏、柳真行為法,草以章草、張芝九體為法。考書之等,以方圓肥瘦適中,鋒藏畫勁,氣清韻古,老而不俗為上;方而有圓筆,圓而有方意,瘦而不枯,肥而不濁,各得一體者為中;方而不能圓,肥而不能瘦,模
 仿古人筆畫不得其意,而均齊可觀為下。其三舍補試升降略同算學法,惟推恩降一等。自初置及並罷年數,悉同算學。



 畫學之業,曰佛道,曰人物,曰山水,曰鳥獸,曰花竹,曰屋木,以《說文》、《爾雅》、《方言》、《釋名》教授。《說文》則令書篆字,著音訓,餘書皆設問答,以所解義觀其能通畫意與否。仍分士流、雜流,別其齋以居之。士流兼習一大經或一小經,雜流則誦小經或讀律。考畫之等,以不仿前人而物之
 情態形色俱若自然,筆韻高簡寫工。三舍試補、升降以及推恩如前法。惟雜流授官,止自三班借職以下三等。



 醫學,初隸太常寺,神宗時始置提舉判局官及教授一人,學生三百人。設三科以教之,曰方脈科、針科、瘍科。凡方脈以《素問》、《難經》、《脈經》為大經,以《巢氏病源》、《龍樹論》、《千金翼方》為小經,針、瘍科則去《脈經》而增《三部針灸經》。常以春試,三學生願與者聽。崇寧間,改隸國子監,置博士、正、錄各四員,分科教導,糾行規矩。立上舍四十人,內舍
 六十,外舍二百,齋各置長諭一人。其考試:第一場問三經大義五道;次場方脈試脈證、運氣大義各二道,針、瘍試小經大義三道,運氣大義二道;三場假令治病法三道。中格高等,為尚藥局醫師以下職,餘各以等補官,為本學博士、正、錄及外州醫學教授。



 紹興中,復置醫學,以醫師主之。翰林局醫生並奏試人,並試經義一十二道,取六通為合格。乾道三年,罷局而存御醫諸科,後更不置局而存留醫學科,令每舉附省闈別試所解發,太常
 寺掌行其事。淳熙十五年,命內外白身醫士,經禮部先附銓闈,試脈義一場三道,取其二通者赴次年省試,經義三場一十二道,以五通為合格,五取其一補醫生,俟再赴省試升補,八通翰林醫學,六通祗候,其特補、薦補並停。紹熙二年,復置太醫局,銓試依舊格。其省試三場,以第一場定去留,墨義、大義等題仿此。



 補道職,舊無試,元豐三年始差官考試,以《道德經》、《靈寶度人經》、《南華真經》等命題,仍試齋醮科儀祝讀。政和間,
 即州、縣學別置齋授道徒。蔡攸上《諸州選試道職法》,其業以《黃帝內經》、《道德經》為大經,《莊子》、《列子》為小經。提學司訪求精通道經者,不問已命、未仕,皆審驗以聞。其業儒而能慕從道教者聽。每路於見任官內,選有學術者二人為乾官,分詣諸州檢察教習。《內經》、《道德經》置博士,《聖濟經》兼講。道徒升貢,悉如文士。初入官,補志士道職,賜褐服,藝能高出其徒者,得推恩。道徒術業精退,州守貳有考課殿最罪法。陳州學生慕從道教,逾月而道徒
 換籍,殆與儒生相半。有宋瑀者,願改道徒內舍,獻《神霄玉清萬壽宮雅》一篇,特換志士,俟殿試。由是長倅以下受賞有差,其誘勸之重如此。宣和二年,學罷。



\end{pinyinscope}