\article{列傳第三十五}

\begin{pinyinscope}

 劉保勛滕中正劉蟠孔承恭宋璫袁廓樊知古郭載附臧丙徐休復張觀陳從信張平子從吉王繼升子昭遠
 尹憲王賓安忠



 劉保勛,字修業,河南人。父處讓,仕後唐,入晉拜樞密使,出為彰德軍節度。保勛少好騎射。後唐清泰中,裁十許歲,攝潞州左司馬,隨父署彰德軍衙內都校。父卒,補供奉官。習刑名之學,頗工詩。因獻詩,宰相桑維翰奇之,奏擢為太常丞。歷漢為秘書丞。周廣順初,有薦其詳練法律,兼大理正,遷工部員外郎。歷掌鄆、宋、楚三州鹽、曲、商稅。



 宋初,拜戶部。遭母喪,起復,出掌蘄口榷茶。徙雲安監
 鹽制置使,歲滿,出羨餘百萬,轉運使欲以狀聞,保勛曰:「貪官物為己功,可乎?」乃止。開寶初,遷司封員外郎、監左藏庫。六年,知宋州。太平興國初,遷祠部郎中、通判晉州。二年,選為江南西路轉運使,賜錢百萬。三年,徙兩浙東北路。太宗征晉陽,改戶部郎中,為隨軍轉運使兼勾當北面轉運事。又與侯陟同勾當軍前諸事。會陜西北路轉運使雷德驤調發沁州軍糧後期,詔劾德驤,以保勛代之。太原平,命知並州。逾年召入,判大理寺,出知升州。
 是冬,召歸,點檢三司開拆司,會鹽鐵使闕,又命權領其事。遷兵部郎中兼判三司勾院。八年,拜右諫議大夫,俄知開封府。寡婦劉詣府訴夫主前妻子元吉置堇食中,毒己將死。按驗獄成,元吉妻撾登聞鼓訴冤,事下御史臺。其實劉有奸狀,元吉知之,劉慚悸成疾,故誣告之。保勛坐奪奉三月,俄以辛仲甫代之。未幾,復判大理寺。



 雍熙二年,權御史中丞兼勾當差遣院。是秋,罷權中丞。三年春,命曹彬等征幽州,保勛以本官知幽州行府事。子
 利涉以開封府兵曹督芻粟隨軍,常從其父。會王師不利,濟拒馬河,更相蹂躪,多死。保勛馬陷淖中,利涉自後掀出之,力不勝,人馬相擠壓,遂俱死。時年六十二。上命恤其後。保勛三子:二子先保勛死,季子隨沒。以其孫巨川為嗣,授秘書正字。端拱初,特召贈工部侍郎。



 保勛性純謹,少寐,未嘗忤物,精於吏事,不憚繁劇。嘗語人曰:「吾受君命未嘗辭避,接同僚未嘗失意,居家積貲未嘗至千錢。」及死,聞者皆痛惜之。至道三年,又錄其次孫世長
 為正字。咸平初,保勛妻卒,詔賜錢十萬。巨川,累為比部郎中。



 滕中正,字普光,青州北海人。曾祖瑤,高郵令。祖煦,即墨令。父保裔,興平令。中正弱冠,舉進士不第。周顯德中,滑帥向拱奏闢為掌書記。拱移鎮彭門,會中正丁外艱,復表奪情,仍署舊職,加朝散大夫。拱鎮襄陽,以中正為襄、均、房、復觀察判官。及留守西洛,又奏署河南府判官、檢校戶部員外郎。



 乾德五年,度支員外郎侯陟表中正有
 材幹,入為殿中侍御史。兩川平,選知興元府,判西京留臺,俄通判河南府留守司事。太祖雩祀西洛,以祗事之勤,轉倉部員外郎。



 太宗即位,遷考功員外郎,授四川東路轉運使。太平興國五年,召為膳部郎中兼侍御史知雜事。六年,命與中書舍人郭贄、戶部郎中雷德驤同知京朝官考課。中正嘗薦舉監察御史張白知蔡州,假貸官錢二百貫糴粟麥以射利,坐棄市。中正降為本曹員外郎,依舊知雜。未幾,又擢拜右諫議大夫,權御史中丞。



 雍熙元年春,大宴,上歡甚,以虛盞示群臣。宰相言飲酒過度,恐有失儀之責。上顧謂中正曰:「今君臣相遇,有失者勿彈劾也。」因是伶官盛言宴會之樂。上曰:「朕樂在時平民安。」是冬乾明節,群臣上壽酒,既三行,上目中正曰:「三爵之飲,實惟常禮,朕欲與群臣更舉一卮,可乎?」中正曰:「陛下聖恩甚厚,臣敢不奉詔。」殿上皆稱萬歲。



 二年,以年老辭,出知河南府。未幾,被病罷,分司西京。淳化初,判留司御史臺,命其子元錫權河南司錄以便養。二年,卒,
 年八十四。



 中正性峻刻,連鞫大獄,時議以為深文。權中丞日,振舉綱憲,人以稱職許之。二子並舉進士,元錫至刑部郎中,元晏後名世寧,至工部郎中。



 劉蟠,字士龍,濱州渤海人。漢乾祐二年舉進士,解褐益都主簿。宋初,歷安遠軍及河陽節度推官、保義軍掌書記。乾德五年,召拜監察御史,典染院事。初,蘇曉掌京城市征,頗乾集,及卒,選蟠代之。冬,命為太宗生辰使。開寶七年,與殿中丞劉德言同知淮南諸州轉運事。太平興
 國初,就遷倉部員外郎,改轉運使,歲漕江東米四百萬斛以給京師,頗為稱職。秩滿,部內僧道乞留,詔許再任,賜金紫,改駕部員外郎。八年,丁內艱,時以諸州綱運留滯,起復,知京城陸路發運司事。會河決韓村,大發丁夫塞之,命蟠調給其餉,未幾河塞。朝廷方議封禪,以蟠為東封水陸計度轉運使,會詔罷其禮。俄遷工部郎中,充河北水路轉運使。改刑部郎中,就充水陸轉運使,入判本部事。籍田畢,遷左諫議大夫。淳化初,兼同考京朝官
 差遣。二年,暴中風眩,上遣太醫視之,賜以金丹。卒,年七十三。賜錢十萬,給其喪事。



 蟠性清介寡合,能攻苦食淡,專事苛刻,好設奇詐,以售知人主。典染作日,太祖多臨視之,蟠偵車駕至,輒衣短後衣,芒屩持梃以督役,頭蓬不治,遽出迎謁。太祖以為勤事,賜錢二十萬。嘗受詔巡茶淮南,部民私販者眾。蟠乘羸馬,偽稱商人,抵民家求市茶,民家不疑,出與之,即擒置於法。



 子鍇,初以父蔭為大理評事,咸平二年,擢進士第。嘗獻《幸太學頌》。真宗中
 夜觀書,得鍇頌,頗嘉賞之,出以示輔臣,且言鍇幼孤,能自立,召試,命直史館。累遷至戶部郎中、鹽鐵副使。



 孔承恭,字光祖,京兆萬年人。唐昭宗東遷,舉族隨之,遂占籍河南。五世祖戡,《唐書》有傳。戡孫迥,萊州刺史。迥子昌庶,虞部郎中。昌庶子莊,仕晉為右諫議大夫。由戡至莊,皆登進士第。承恭,莊之子也。以門蔭授秘書省正字,歷溫、安豐二縣主簿。時王審琦節制壽春,以承恭名家子,奏攝節度推官。府罷,調補鄭州錄事參軍,入為大理
 寺丞。獻宮詞,托意求進。太祖怒其引喻非宜,免所居官,放歸田里。



 太宗即位,以赦復授舊官。時初榷酒,以承恭監西京酒曲,歲增課六千萬。遷大理正,議獄平允,擢庫部員外郎,判大理少卿事。遷屯田、兵部二郎中,同考校京朝官課第。端拱三年,下詔曰:「九寺三監,國之羽儀,制度聲名,往往而在。各有副貳,率其司存,品秩素高,職任尤重。郎吏遷授,斯為舊章。比聞縉紳之流,頗以臺閣自許,目為散地,甚無謂焉。朕將振之,自我而始。其以兵部
 郎中孔承恭為太常少卿,魏羽為秘書少監,戶部郎中柴成務為光祿少卿,魏庠為衛尉少卿,張洎為太僕少卿,呂端為大理少卿,臧丙為司農少卿,袁廓為鴻臚少卿,工部郎中張雍為太府少卿。」又以屯田郎中雷有終為少府少監,虞部郎中索湘為將作少監。時裴祚、慎從吉、宋雍先為少卿,皆改授東宮官。又詔承恭與左散騎常侍徐鉉刊正道書,俄以疾求解官,且言早游嵩、少間,樂其風土,願卜居焉。上召見,哀其羸瘠,出御藥賜之,授
 將作監致仕。以其子玢同學究出身,為登封縣尉,俾就祿養。未果行而卒,年六十二。



 承恭少疏縱,及長,能折節自勵。嘗上疏請令州縣長吏詢訪耆老,求知民間疾苦、吏治得失,及舉令文「賤避貴,少避長,輕避重,去避來」,請詔京兆並諸州於要害處設木牌刻其字,違者論如律。上皆為行之。尤奉佛,多蔬食,所得奉祿,大半以飯僧。嘗勸上不殺人,又請於征戰地修寺及普度僧尼,人多言其迂闊雲。



 宋璫,字寶臣,華州渭南人。父鸞,監察御史。璫,乾德中進士及第,拔萃登科,解褐青城主簿。好寫書,秩滿,載數千卷以歸。吳廷祚鎮永興,闢掌書奏。廷祚卒,復調下邽主簿,擢著作佐郎、知綿州。太宗即位,改右贊善大夫,為峽路轉運副使。代還,召對,賜緋魚。復出知秦州,有善政,就拜監察御史,充陜西轉運使,以韋但代知秦州,璫去州未百日,但坐事系獄。上以璫前有治績,賜錢五十萬,再命知秦州,安集諸戎,部內清肅。



 雍熙初,轉比部員外郎。
 在任凡六年,召歸,面賜金紫,授度支判官。俄遷屯田郎中、知益州,屬歲饑多盜,璫始至,以方略擒捕招輯,盜皆首伏屏息,下詔嘉獎。端拱初,就拜右諫議大夫。時兩川轉運使副皆坐事免,以璫為西川轉運使,加左諫議大夫,改知陜州。



 淳化中,三吳歲饑、疾病,民多死,擇長吏養治之,命璫知蘇州。璫體豐碩,素病足,至州,地卑濕,疾益甚。人或勸其謝疾北歸,璫曰:「天子以民病俾我綏撫,我以身病而辭焉,非臣子之義也。」既而太白犯南斗,曰:「斗
 為吳分,民方饑,天象如此,長吏得無咎乎!」四年,卒,年六十一。上聞之嗟悼,錄其子明遠為蒲城主簿,俾護其喪歸葬焉。



 璫性清簡,歷官三十年,未嘗問家事,唯聚書以貽子孫。且曰:「使不忘本也。」明遠,淳化三年進士,後為都官員外郎。次子柔遠,亦舉進士及第。垂遠,閣門祗候。



 袁廓,劍州梓潼人。在蜀舉進士及第。入宋,補雙流縣主簿。又為西平縣主簿,勾稽漏籍,得民丁萬餘,州將薦其勤職,就遷上蔡令,又以課最,擢太子右贊善大夫,令於
 御史府分領推事,掌榷貨務。廓性誇誕,敢大言,好詆訐,太祖以奇士待之。



 太宗即位,遷殿中丞,出知楚州。歸,掌京師市征,歲中增課數萬緡,上嘉之,賜緋魚,繼錢百萬。會錢俶盡籍土宇以獻,命廓按籍浙中,諸州軍倉庫之物悉輸京師,得以便宜從事。仍詔每公宴別席而坐,以寵異之。復命知鄆州,會河決,溢入城,浸居人廬舍,至冬月結為冰。廓大發民鑿取,以竹輿舁出城,散積之。使者至,謂其有略,致水不入城,乃以狀聞,拜監察御史。至春
 凍解,州地下,流澌溢入為民患。



 會秦王廷美遷置房州,以崇儀副使閻彥進知州事,廓通判州事,並賜白金三百兩。廓俄轉殿中侍御史,召為戶部判官,命與陳恕、李惟清專計度芻糧事。改戶部員外郎,又為度支判官。籍田,轉本曹郎中,判戶部勾院。



 廓強項好爭,數與判使等較曲直於上前,聲氣俱厲,上每優容之。然勾稽精密,由是部領擁積,為郡吏所訴,詔御史辨問,廓謁見宰相趙普自理。屬鄭州團練使侯莫陳利用得罪,廓嘗與利用
 書札往還稔暱。普謂之曰:「職司常事,此不足雲,與利用交結款密,於理可乎?」廓驚慚泣下,不能對。數日,出知溫州。就遷鴻臚少卿。



 同郡袁仁甫掌州之關征,素以宗盟之分,頗相親善,一旦不協,互有論奏。上遣光祿寺丞牛韶往按驗,韶至,並攝系獄置對。上疑廓被誣,驛召赴闕。廓性剛褊,被詰治峻急,詔書未至,以憤死。上聞,甚追悼之。復驗仁甫所訴,多無實狀,免韶官,貶仁甫商州長史,贈廓右諫議大夫。錄其子丘賀為奉禮郎,始十歲。上猶
 念廓不已,又詔削仁甫名籍,配隸商州。



 樊知古,字仲師,其先京兆長安人。曾祖偁,濮州司戶參軍。祖知諭,事吳為金壇令。父潛,事李景,任漢陽、石埭二縣令,因家池州。知古嘗舉進士不第,遂謀北歸,乃漁釣採石江上數月,乘小舟載絲繩,維南岸,疾棹抵北岸,以度江之廣狹。開寶三年,詣闕上書,言江南可取狀,以求進用。太祖令送學士院試,賜本科及第,解褐舒州軍事推官。嘗啟於上,言老母親屬數十口在江南,恐為李煜
 所害,願迎至治所。即詔煜令遣之。煜方聞命,即厚給繼裝,護送至境上。



 七年,召拜太子右贊善大夫。會王師征江表,知古為鄉導,下池州。八年,以知古領州事。先是,州民保險為寇,知古擊之,連拔三砦,擒其魁以獻,餘皆潰散。方議南征,命高品石全振往湖南造黃黑龍船,以大艦載巨竹緪,自荊南而下,遣八作使郝守浚浚等率丁匠營之。議者以謂江濤險壯,恐不能就,乃於石碑口試造之,移至採石,三日橋成,不差尺寸,從知古之請也。



 金陵
 平,擢拜侍御史,令乘傳按行江南諸州,詢訪利民,復命知江南東路轉運事。數日,改授江南轉運使,賜錢一百萬。先是,江南諸州官市茶十分之八,復徵其餘分,然後給符聽其所往,商人苦之。知古請蠲其稅,仍差增所市之直,以便於民。江南舊用鐵錢,十當銅錢之一,物價翔踴,民不便,知古亦奏罷之。先是,李煜用兵,權宜調斂,知古悉奏為常額。豫章洪氏嘗掌升州榷酤,逋鐵錢數百萬。至是,知古挾微時嘗辱於洪氏,責償銅錢以快意。



 太
 宗即位,授庫部員外郎。召歸,換金紫,賜錢百萬,命為京西北路轉運使。太平興國六年,加虞部郎中,就改知邠州,移鳳翔府。入為鹽鐵判官,出領荊湖轉運使。雍熙初,遷比部郎中。會河朔用兵,分諸郡為兩路,以給漕挽。遷知古為東路轉運使,遷駕部郎中,賜錢五十萬。知古本名若水,字叔清,因召見,上問之曰:「卿名出何書?」對曰:「唐尚書右丞倪若水亮直,臣竊慕之。」上笑曰:「可改名『知古』」。知古頓首奉詔。倪若水實名「若冰」,知古學淺,妄引以對,
 人皆笑之。



 端拱初,遷右諫議大夫、河北東西路都轉運使,賜白金千兩。兩路各置轉運副使,都轉運使之名自知古始。二年,詔加河北西路招置營田使。奏請修城木五百餘萬、牛革三百萬。上曰:「萬里長城豈在於此?自古匈奴、黃河,互為中國之患。朕自即位以來,或疆埸無事,則有修築圩堤之役。近者邊烽稍警,則黃河安流無害,此蓋天意更迭垂戒,常令惕勵。然而預備不虞,古之善教,深溝高壘,亦王公設險之義也。所請過當,不亦重困吾
 民乎?」乃詔有司量以官物給之。



 會度支使李惟清上言河北軍儲無備,請發河南十七軍州轉粟以赴。太宗謂:「農事方殷,豈可更興此役?」惟清固以為請,上遣左正言馮拯乘傳與知古計之。知古即言:「河北軍儲可以均濟足,俟農隙令民轉餉。」拯復命,太宗曰:「不細籌之,則民果受弊矣。」未幾,入朝奏事稱旨,拜給事中。俄為戶部使。



 知古有才力,累任轉運使,甚得時譽。及在戶部,頻以職事不治,詔書切責,名益減。素與陳恕親善,恕時參知政事,
 太宗言及計司事有乖違者,恕具以告。後因奏事,知古遂自解。上問:「從何得知?」曰:「陳恕告臣。」上怒恕洩禁中語,且嫉知古輕佻,故兩罷之。出知古知梓州,未至,改西川轉運使。



 知古自以嘗任三司使,一旦掌漕運劍外,鬱鬱不得志,常稱足疾,未嘗按行郡縣。蜀中富饒,羅紈錦綺等物甲天下,言事者競商榷功利。又土狹民稠,耕種不足給,由是兼並者益糴賤販貴以規利。



 淳化中,青城縣民王小波聚眾為亂,謂其眾曰:「吾疾貧富不均,今為汝
 輩均之。」附者益眾,遂攻陷青城縣,掠彭山,殺其令齊元振。巡檢使張□與鬥於江源縣,射小波,中其額,旋病創死,□亦被殺。眾遂推小波妻弟李順為帥。初,小波黨與裁百人,州縣失於備御,故所在蜂起,至萬餘人。攻蜀州,殺監軍王亮及官吏十餘人。陷邛州,害知州桑保紳、通判王從式及諸僚吏,逐都巡檢使郭允能。允能率麾下與戰新津江口,為賊所殺,同巡檢、殿直毛儼徒步以身免。賊勢益張,眾至數萬人,陷永康軍、雙流、新津、溫江、郫
 縣,縱火大掠,留其黨守之。往攻成都,燒西郭門,不利,引去。陷漢州、彭州,旋陷成都。



 時已詔知梓州、右諫議大夫張雍代知古為轉運使。雍未至,知古與知府郭載及屬官走東川。詔復令掌兩川漕運。知古具伏擅離所部,制置無狀,上特宥之,以本官出知均州。視事旬日,憂悸卒,年五十二。上猶嗟憫,賜其子漢公同學究出身。



 知古明俊有吏乾,辭辨捷給,及任西川,不能弭盜而逃,雖獲宥,終以慚死雲。



 郭載,字咸熙,開封浚儀人。父暉,右監門衛將軍、義州刺史。載蔭為右班殿直,累遷供奉官、閣門祗候。雍熙初,提舉西川兵馬捕盜事,太宗賜鞍馬、器械、銀錢以遣之。四年,以積勞加崇儀副使。召還,上言:「川、峽富人俗多贅婿,死則與其子均分其財,故貧者多。」詔禁之。端拱二年,擢引進副使、知天雄軍,入同勾當三班,出知秦州兼沿邊都巡檢使。先是,巡邊者多領兵騎以威戎人,所至頗煩苦之。載悉減去,戎人感悅。遷西上閣門使,改知成都府。



 載在天雄軍,屢奏市糴朝臣段獻可、馮侃等所市粗惡,軍人皆曰:「此物安可充食?」太宗頗疑,使覆驗之,及報,與戴奏同。獻可等皆坐削官,仍令填償。及載受代,獻可等所市皆支畢,復有羨數。三司判勾馮拯以聞,太宗召度支使魏羽詰之。羽曰:「獻可等所市不至粗惡,亦無欠數。臣與侃親舊,是以未敢白。」太宗曰:「此公事爾,何用畏避?」因詔宰相謂曰:「此乃郭載力奏,朕累與卿等議,皆云有實。今支畢,頗有羨餘,軍士復無詞訴。郭載,朕向以純誠
 待之,何為矯誣及此?然已委西川,俟還日別當詰責。」於是獻可等悉復官。



 載行至梓州,時李順已構亂,有日者潛告載曰:「益州必陷,公往當受禍,少留數日可免。」載怒曰:「吾受詔領方面,阽危之際,豈敢遷延邪?」即日入成都。順兵攻城益急,不能拒守,乃與樊知古率僚屬斬關出,以餘眾由梓州趨劍門,隨招安使王繼恩統兵討順,平之,復入成都。月餘,憂患成病,卒,年四十。



 載前在蜀,頗能為民除害,故蜀民悅之。再至成都,即值兵亂,及隨繼恩
 平賊,亦有所全濟。故其死也,成都人多嘆惜之。



 臧丙,字夢壽,大名人。弱冠好學。太平興國初舉進士,解褐大理評事,通判大寧監,官課民煮井為鹽,丙職兼總其事。先是,官給錢市薪,吏多侵牟,至歲課不充,坐械系者常數十百人。丙至,召井戶面付以錢,既而市薪山積,歲鹽致有羨數。



 太宗平晉陽,以丙為右贊善大夫、知遼州。丙素剛果,有吏乾。會同年生馮汝士以秘書丞知石州,與監軍不協,一夕剚刃於腹而死,事可疑。丙上疏言,
 汝士死非自殺,乞按治。上覽奏驚駭,即遣使鞫之,召丙問狀。丙曰:「汝士居牧守之任,不聞有私罪,而言自殺,若使冤死不明,不加宿直者以罪,今後書生不能治邊郡矣。」上嘉其直,改著作郎,俄遷右拾遺、直史館。加工部員外郎,充河東轉運使,俄兼本路營田使。代歸,授戶部郎中、同知審官院。



 朝廷方以九寺亞列為重,改司農少卿。淳化二年,拜右諫議大夫,出知江陵府。歲餘,疾。上聞之,遣中使及尚醫馳往視之,逾月卒,年五十三。上軫悼之,
 以其子待用為四門助教。



 丙舊名愚,字仲回。既孤,常夢其父召丙偶立於庭,向空指曰:「老人星見矣。」丙仰視之,黃明潤大,因望而拜。既寤,私喜曰:「吉祥也。」以壽星出丙入丁,乃改名焉,至是無驗。丙於禮不當更名,古人戒數占夢,無妄喜也。



 待用歷金部郎中、東染院使、賀州刺史。次子列進士及第,至太常丞。



 徐休復,字廣初,濮州鄄城人。太平興國初舉進士,解褐大理評事、通判。轉運使薦其材,代歸,授太子右贊善大
 夫,改著作郎、直史館,賜緋魚,遷左拾遺。六年,加右補闕,充兩浙東北路轉運副使,移知明州。七年秋,被召赴闕,明年,授庫部員外郎、知制誥。九年,出知廣州,是歲,加水部郎中。雍熙二年,就遷比部郎中,充樞密直學士,賜金紫,依舊知州事。



 休復與轉運使王延範不協,乃奏延範私養術士,厚待過客,撫部下吏有恩,發書與故人韋務升作隱語,偵朝廷事,反狀已具。詔遣內侍閻承翰與休復同按劾之,遂抵於法。



 端拱初,加左諫議大夫,召為戶
 部使。淳化元年,罷使,遷給事中,連知青、潞二州。休復先上言,以父母蒿葬青社,願得領州事,因營丘壟。至青州逾年,但聚財殖貨,終不言葬事。至潞州數月,瘍生於腦。既而疾甚,若見王延範,休復但號呼稱死罪,後數日卒,年五十三。



 休復無他能,掌誥命甚不稱職,履行不見稱於搢紳云。



 張觀,字仲賓,常州毗陵人。在江南登進士第。歸宋,為彭原主簿。太平興國初,移興元府掾,復舉進士不第,調雞
 澤主簿。再求試,特授忠武掌書記,就改觀察判官。上請復刺史及不遣武德卒詣外州偵事,頗稱旨,召拜監察御史,充桂陽監使。獻所業文,賜進士及第。



 會三司言劍外賦稅輕,詔觀乘傳按行諸州,因令稍增之。觀上疏言:「遠民不宜輕動撓,因而撫之,猶慮其失所,況增賦以擾之乎?設使積粟流衍,用輸京師,愈煩漕挽,固不可也。或以分兵就食,亦非安存之策,徒斂怨於民,未見國家之利。」太宗深以為然,因留不遣。



 其後,復上疏曰:



 臣憑
 借光寵,備位風憲,每遇百官起居日,分立於庭,司察不如儀者舉之。因見陛下天慈優容,多與近臣論政,德音往復,頗亦煩勞。至於有司職官,承意將順,簿書叢脞,咸以上聞,豈徒褻黷至尊,實亦輕紊國體。況帝王之道,言則左史書之,動則右史書之,列於緗素,垂為軌範,不可不慎也。若夫方今之急者,遠人未服,邊鄙不寧。陰陽未序,倉廩猶虛。淳樸未還,奢風尚熾。縣道未治,逋逃尚多。刑法未措,禁令猶密。墜典未復,封祀猶闕。凡此數者,皆
 朝廷之急務也。誠願陛下聽斷之暇,宴息之餘,體貌大臣,以之揚榷,使沃心造膝,極意論思,則治體化源,何所不至?



 臣又嘗讀唐史,見貞觀初始置崇文館,命學士、耆儒更直互進,聽朝之際,則入內殿講論文義,商榷時政。或日旰忘倦,或宵分始罷,書諸信史,垂為不朽。況陛下左右前後,皆端士偉人,伏望釋循常之務,養浩然之氣,深詔近臣,闡揚玄風,上為祖宗播無疆之休,下為子孫建不拔之業。與夫較量金谷,剖析毫厘,以有限之光陰,
 役無涯之細務者,安可同年而語哉!



 上覽而稱之,召賜緋魚,以為度支判官。



 歲餘,遷左司,改鹽鐵判官。嘗因奏事白上曰:「陛下務敦淳化,殿宇採飾,皆徹去之,惟尚樸素,天下幸甚。然於服御器用,臣願亦從純儉。」上曰:「朕庶事簡約,至於所服,多用絁絹,皆經浣濯爾,卿言甚善。」觀頓首謝。觀數在省署及長春殿次中,諮事於其使李惟清,辨說抵牾,失禮容,惟清不能甘,因奏解其任。觀抗章論列,上亦察其無失,故未幾復授舊職。又諫罷治佛寺,
 不報。俄出為諸路茶鹽制置副使,上疏言:更茶鹽之制,於理非便。不合旨,改知黃州,遷揚州,皆有善政。



 會三司改舊貫,均州縣之籍以分其職,召為三司河東道判官。有詔計司官屬不得越局言他事,觀自以任諫官,乃上書指陳拾遺補闕之職,言事固當然,不奉詔。上怒,謂宰相曰:「朕俾警三司僚屬各率其職,非令諫官不言時務,觀乃妄有援引,以諷刺朕,姑為容忍,不欲深責。」乃令出知道州,移廣南西路轉運使。坐奏交州黎桓為亂兵所
 殺、丁浚復位事不實,被劾。獄未具,卒於桂州,年五十三。



 觀廣覽《漢》、《史》,雅好論事,辭理切直,有古人之風焉。



 論曰:保勛從其子以死事,宋榼忘其身以恤民,臧丙信友誼以明枉,其所履歷,皆有足觀。中正粗振風紀而峻深寡恕,袁廓剛狷誇誕以徼寵任,承恭平恕知止而好佞佛,固皆未盡於善。知古首獻征南之謀,遂階試用,而其攬轡舊都,猶尋宿怨,與昔人所謂不以私怨惡廢鄉黨之好者異矣。郭載肆為矯誣,而懷恚以死;休復虧慎
 終之孝,而樂致人於禍,庸何議焉?若觀之獻納忠讜,識達體要,則又可嘉者也。



 陳從信,字思齊,亳州永城人。恭謹強力,心計精敏。太宗在晉邸,令典財用,王宮事無大小悉委焉。累官右知客押衙。開寶三年秋,三司言:倉儲月給止及明年二月,請分屯諸軍盡率民船,以資江、淮漕運。太祖大怒,責之曰:「國無九年之蓄曰不足,爾不素計而使倉儲垂盡,乃請屯兵括民船以運,是可卒致乎?今設汝安用?茍有所闕,
 當罪汝以謝眾!」三司使楚昭輔懼,詣太宗求寬釋,使得盡力。



 太宗既許,召從信問之,對曰:「從信嘗游楚、泗,知糧運之患。良以舟人之食,日歷郡縣勘給,是以凝滯。若自發舟計日往復並支,可以責其程限。又楚、泗運米於舟,至京復輦入倉,宜宿備運卒,令實時出納,如此,每運可減數十日。楚、泗至京千里,舊八十日一運,一歲三運。若去淹留之虛日,則歲可增一運焉。今三司欲籍民舟,若不許,則無以責辦,許之,則冬中京師薪炭殆絕矣。不若
 募舟之堅者漕糧,其損敗者任載薪炭,則公私俱濟。今市米騰貴,官價鬥錢七十,賈者失利,無敢致於京師,雖居商厚儲亦匿而不糶,是以米益貴,民將餓殍。若聽民自便,即四方奔湊,米多而價自賤矣。」太宗明日具奏,太祖可之,其事果集焉。



 太宗即位,遷東上閣門使,充樞密都承旨。會八作副使綦廷珪,因疾假滿不落籍,愈日不朝參,即入班中,宣徽使潘美、王仁贍並坐奪奉一季,從信與閣門使商鳳責授閑廄使、閣門祗候,餘抵罪有差。
 太平興國三年,改左衛將軍,復為樞密都承旨。太宗征並、汾,以為大內副部署。七年,坐秦王廷美事,以本官罷。明年,分使三部,以從信為度支使,賜第於浚儀寶積坊,加右衛大將軍。九年,卒,年七十三,贈太尉。



 從信好方術,有李八百者,自言八百歲,從信事之甚謹,冀傳其術,竟無所得。又侯莫陳利用者,所為多不法,始因從信推薦,人以是少之。



 張平,青州臨朐人。弱冠寓單州,依刺史羅金山。金山移
 滁州,署平馬步都虞候。太宗尹京兆,置其邸。及秦王廷美領貴州,復署為親吏。後數年,有訴平匿府中錢物,秦王白太宗鞫之,無狀,秦王益不喜,遂遣去。太宗憐其非罪,以屬徐帥高繼沖,繼沖署為鎮將。平嘆曰:「吾命雖蹇,後未必不為福也。」



 太宗即位,召補右班殿直,監市木秦、隴,平悉更新制,建都務,計水陸之費,以春秋二時聯巨筏,自渭達河,歷砥柱以集於京。期歲之間,良材山積。太宗嘉其功,遷供奉官、監陽平都木務兼造船場。舊官造
 舟既成,以河流湍悍,備其漂失,凡一舟調三戶守之,歲役戶數千。平遂穿池引水,系舟其中,不復調民。有寇陽拔華者,往來關輔間,為患積年。朝廷命內侍督數州兵討之,不克。平以好辭遣人說之,遂來歸。改崇儀副使,仍領其務。凡九年,計省官錢八十萬緡。



 雍熙初,召還,同知三班事,遷如京使。三年,改西上閣門使。才三月,又改客省使。四年,代王明為鹽鐵使。平掌陽平署積年,是秋,聞陜西轉運使李安發其舊為陽平奸利,憂恚成疾而卒,
 年六十三。廢朝,贈右千牛衛上將軍,官給葬具。



 平好史傳,微時遇異書,盡日耽玩,或解衣易之。及貴,聚書數千卷。在彭門日,郡吏有侮平者數輩,後悉被罪配京窯務。平子從式適董其役,見之,以語平。平召至第,為設酒饌勞之,曰:「公等不幸,偶罹斯患,慎勿以前為念。」給以緡錢,且戒從式善視之。未幾,遇赦得原,時人稱其寬厚。



 從式事太宗藩邸,累官文思使。次子從吉,以蔭補殿直,轉供奉官、知宜州,屢破溪蠻。轉運使堯叟上其狀,累遷內殿
 崇班、閣門祗候。在任凡八年,代還,為如京副使。咸平中,知環州,嘗與宋沆率兵襲西夏,小衄,部署張凝表其專,責授內殿崇班。俄知澧州,復舊秩。景德四年,宜州軍校陳進叛,命副曹利用為廣南東、西路安撫使,將兵討之。次象州大鳥砦,與賊戰,進為先鋒郭志言所刺,遂入城,斬首六十級。以平賊功,改莊宅副使。未還,卒,年四十九。



 王繼升,冀州阜城人。性純質謹願。事太宗於藩邸,太宗信任之。即位,補供奉官,累遷軍器庫副使。陳洪進來獻
 漳、泉之地,以繼升為泉州兵馬都監。會游洋洞民萬餘叛,攻泉,繼升潛率精騎二百夜擊破之,擒其魁,械送闕下,餘黨悉平。召還,遷軍器庫使,領順州刺史,知諸道陸路發運事。



 雍熙四年,以諸道水陸發運並為一司,命繼升與刑部員外郎董儼同掌其事,號為稱職。俄遷右神武軍將軍。端拱初,改領本州團練使,三月,卒,年六十四。太宗頗嗟悼,贈洋州觀察使,葬事官給。子昭遠。



 昭遠,形質魁偉,色黑,繼升名之「鐵山」。有膂力,善騎射。少
 時入山捕鷹鶻,值澗水暴漲十餘丈,昭遠升大樹,經宿得免。嘗涉河,冰陷,二公傍共援出之,昭遠神色自若。喜與里中惡少游處,一日,眾祀里神,昭遠適至,有以博投授之,謂曰:「汝他日儻有節鉞,試擲以卜之。」昭遠一擲,六齒皆赤。



 南游京師,事太宗於晉邸,特被親遇,常呼其小字。及即位,補殿前指揮使,稍遷都知。從征太原,先登,為流矢所中,血漬甲縷,戰益急。會劉繼元降,命守城門,籍兵仗。又從征範陽,多所擒獲,超散員指揮使。



 涪王之遷
 房陵也,禁衛諸校楊均、王榮等以依附被譴,獨昭遠無所預,太宗以為忠。再遷東西班都虞候,轉殿前班都指揮使,領寰州刺史。改馬步軍都軍頭,命乘傳鎮、定、高陽關,募兵以備契丹。又為冀州駐泊都監,俄授澤州團練使、洺州都部署。太宗屢稱其能,可備急使。



 端拱初,召為殿前都虞候,領勤州防禦使。命有司治綾錦院為公署,掘地得鐵若山形,或言此地即鐵山故營,又與昭遠幼名合,聞者異之。太宗嘗草書紈扇,作古詩賜諸將,意
 多比諷,其賜昭遠,尤加賞遇。二年,領沙州觀察使,再為並、代副都部署。至道中,李繼遷擾西鄙,絕靈武糧道,命昭遠為靈州路都部署,護二十五州芻粟,竟達靈武,繼遷不敢犯。



 真宗即位,徙定州行營都部署。未幾,拜保靜軍節度使,充天雄軍都部署、知府事。咸平二年,移知河陽,數月卒,年五十六。時車駕在大名,為廢朝。贈太尉,謚惠和,中使護葬。



 昭遠頗知書,性吝嗇,所至無善政。母弟昭懿亦事晉邸,至捧日都虞候。弟昭遜,西京作坊使。初,
 祖母郭氏嘗對昭遠母指昭遠曰:「此兒有貴相,他日必至公侯。」指昭懿曰:「此兒奉錢過二萬,不能勝矣。」果皆如其言。



 昭遠子懷普,九歲事太宗左右,至西京左藏庫使、平州刺史。懷一,供備庫副使。懷正,內殿承制。懷英,內殿崇班。



 尹憲,並州晉陽人。開寶中,事太宗於藩邸。太宗即位,擢為殿直,充延州保安軍使,改供奉官。太平興國四年,護府州屯兵,與鄜州三族會攻嵐州,破敵千餘眾,擒偽知
 嵐州事馬延忠,拔緣河諸砦。以功轉西京作坊副使。入朔州界,破寧武軍,殺其軍使,獲人馬、器甲甚眾。改護夏州兵,轉供備庫使。殺戮三水義、醜奴莊、岌伽羅膩葉十四族,及誘其渠帥。屢降詔書褒美。雍熙初,詔就知夏州,攻破李繼遷之眾於地斤澤,繼遷遁走,俘獲四百餘帳。奏請於所部抽移諸帳,別置騎兵,號曰平砦,以備其用,詔從之。俄殺蘆關及南山野貍數族,諸族遂擾。代還,為洪州巡檢。未幾,命護莫州屯兵。



 三年,詔知瀛州兼兵馬鈐
 轄,領富州刺史,遷東上閣門使。端拱二年,知滄州,移邢州,皆兼鈐轄。淳化初,與王文寶並命為四方館使,連護鎮、定州屯兵。改知貝州,移高陽關兵馬鈐轄。五年,知定州,與兵馬部署王榮不協。榮素粗暴,因忿毆憲僕地,憲怏怏致疾,數日卒,年六十三。



 王賓,許州許田人。小心謹願。年十餘,事宣祖左右,及長,善騎射。太宗領兗海節制,太祖以署府中右職。太平興國初,補東頭供奉官、亳州監軍。賓妻妒悍,賓不能制,時
 監軍不許挈家至任所,妻擅至亳,賓具白上。太宗召其妻,俾衛士捽之,杖百,以妻忠靖卒,一夕死。遷賓儀鸞副使,領內酒坊。



 從征太原,又從征範陽,與彰信節度劉遇攻城東面。五年,車駕北巡,副王仁贍為大內都部署。七年,改洛苑使。會汴漕壅滯,軍食不給,詔別置水陸發運兩司,以賓有心計會,領演州刺史,與儒州刺史許昌裔同掌其事。凡四年,儲積增羨,號為稱職,俄改右神武將軍。



 黎陽當舟車交會,禁兵常屯萬餘,以度支使張遜薦,
 命賓護黎陽軍,兼領黃、御兩河發運事,俄領本州團練使。以賓請黎陽建通利軍,命就知軍事。賓規起公署、郵館,供帳之器咸具。加本軍大將軍,歲別給錢二百萬,俄兼河北水陸路轉運使。



 貝州兵屯無壁壘,分寓邸肆,賓選隙地築舍千二百餘以處之,優詔褒美。召為右羽林大將軍、判左金吾兼六軍諸衛儀仗司事。淳化四年,出知揚州兼淮南發運使,徙為通許鎮都監。至道元年,卒,年七十三,賻贈加等。



 賓事宣祖、太祖、太宗殆六十年,
 最為勤舊,故恩寵尤異,前後賜賚數千萬,俱奉釋氏。在黎陽日,按見古寺基,即以奉錢修之,掘地丈餘,得數石佛及石碣,有賓姓名,賓異其事以聞。詔名寺為淳化,賜新印經一藏、錢三百萬以助之。



 安忠,河南洛陽人。祖叔千,仕晉累任方鎮,以太子太師致仕。父延韜,左清道率府率。忠形質魁岸,不知書,才通姓名而已。事太宗藩邸殆二十年,太宗即位,授東頭供奉官,掌弓箭庫。遷內弓箭庫副使、西京作坊使,掌翰林
 司、內衣庫,提點醫官院,掌屯兵於雄州。



 會曹彬敗於拒馬河,忠分砦兵布列緣邊,以備游騎,又鑿河葺城壁。俄徙威虜軍,又隸鎮定路大陣之左廂,就擢東上閣門使。與大將李繼隆、田重進、崔翰追契丹兵祁州北,詔書獎飭。端拱元年,移護高陽關屯兵。契丹侵鎮、定,又與崔翰拒之。傅潛陣於瀛州,忠當城之西面。二年,徙知壽州,逾月,移貝州。有劇賊十二人久為民患,忠捕之,悉獲。



 淳化四年,判左金吾街仗。王賓出知揚州,以忠代為左龍武
 軍大將軍。忠泣請:「諸衛將軍列在朝外,不得迎左右,願復舊職。」上笑曰:「環列之官,古官也。大將軍三品,汝終不知朝廷表著之位。」因從其請。俄復東上閣門使,充淮南諸州兵馬鈐轄。至道三年,以病求歸,至泗州卒,年六十四。天禧元年,錄其孫惟慶為殿直。



 論曰:太宗居潛,左右必求忠厚強干之士。及即位,修舊邸之功,陳從信、張平、王繼升、尹憲、王賓、安忠六人者,咸備任使,又皆畀以兵食之重寄,而各振舉其職焉,有足
 稱者矣。然平不修舊怨,庶幾進於士夫之度。從信所進邪佞以術蠱惑上心,猶不免於近侍之常態歟!



\end{pinyinscope}