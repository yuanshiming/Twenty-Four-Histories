\article{列傳第一百三}

\begin{pinyinscope}

 孫覺弟覽李常孔文仲弟武仲平仲李周鮮于侁顧臨李之純從弟之儀王覿子俊義馬默



 孫覺,字莘老,高郵人。甫冠,從胡瑗受學。瑗之弟子千數,
 別其老成者為經社,覺年最少,儼然居其間,眾皆推服。登進士第,調合肥主簿。歲旱,州課民捕蝗輸之官,覺言:「民方艱食,難督以威。若以米易之,必盡力,是為除害而享利也。」守悅,推其說下之他縣。嘉祐中,擇名士編校昭文書籍,覺首預選,進館閣校勘。神宗即位,直集賢院,為昌王記室,王問終身之戒,為陳諸侯之孝,作《富貴二箴》。擢右正言。



 神宗將大革積弊,覺言:「弊政固不可不革,革而當,其悔乃亡。」神宗稱其知理。嘗從容語及知人之難,
 覺曰:「堯以知人為難,終享其易。蓋知人之要,在於知言。人主用臣之道,任賢使能而已。賢能之分既殊,任使之方亦異。至於所知有限量,所能有彼此,是功用之士也,可以處外而不可以處內,可以責之事而不可責之言。陛下欲興太平之治,而所擢數十人者,多有口才,而無實行。臣恐日浸月長,匯征墻進,充滿朝廷之上,則賢人日遠,其為患禍,尚可以一二言之哉。願觀《詩》、《書》之所任使,無速於小利近功,則王道可成矣。」



 邵亢在樞府,無所
 建明,神宗語覺,欲出之,用陳升之以代。覺退,即奏疏如所言。神宗以為希旨,奪官兩級。執政曰:「諫官有出外,無降官之理。」神宗曰:「但降官,自不能住。」覺連章丐去云:「去歲有罰金御史,今茲有貶秩諫官,未聞罰金貶秩,而猶可居位者。」乃通判越州,復右正言,徙知通州。熙寧二年,詔知諫院,同修起居注,知審官院。



 王安石早與覺善,驟引用之,將援以為助。時呂惠卿用事,神宗詢于覺,對曰:「惠卿即辯而有才,過於人數等,特以為利之故,屈身於
 安石,安石不悟,臣竊以為憂。」神宗曰:「朕亦疑之。」其後王、呂果交惡。



 青苗法行,首議者謂:「《周官》泉府,民之貸者,至輸息二十而五,國事之財用取具焉。」覺奏條其妄,曰:「成周賒貸,特以備民之緩急,不可徒與也,故以國服為之息。然國服之息;說者不明。鄭康成釋經,乃引王莽計羸受息,無過歲什一為據,不應周公取息,重於莽時。況載師所任地,漆林之徵特重,所以抑末作也。今以農民乏絕,將補耕助斂,顧比末作而征之,可乎?國事取具,蓋謂
 泉府所領,若市之不售,貨之滯於民用,有買有予,並賒貸之法而舉之。倘專取具於泉府,則塚宰九賦,將安用邪?聖世宜講求先王之法,不當取疑文虛說以圖治。今老臣疏外而不見聽,輔臣遷延而不就職,門下執正而不行,諫官請罪而求去。臣誠恐奸邪之人,結黨連伍,乘眾情之洶洶,動搖朝廷,釣直幹譽,非國家之福也。」安石覽之,怒,覺適以事詣中書,安石以語動之曰:「不意學士亦如此!」始有逐覺意。會曾公亮言畿縣散常平錢,有追
 呼抑配之擾,安石因請遣覺行視虛實。覺既受命,復奏疏辭行,且言:「如陳留一縣,前後曉示,情願請錢,卒無一人至者,故陳留不散一錢。以此見民實不願與官中相交。所有體量,望賜寢罷。」遂以覺為反復,出知廣德軍,徙湖州。



 松江堤沒,水為民患。覺易以石,高丈餘,長百里,堤下化為良田。徙廬州,改右司諫。以祖母喪求解官,下太常議,不可。詔知潤州,覺已持喪矣。服除,知蘇州,徙福州。閩俗厚於婚喪,其費無藝。覺裁為中法,使資裝無得過
 百千。令下,嫁娶以百數,葬埋之費亦率減什伍。連徙亳、揚、徐州。徐多盜,捕得殺人者五,其一僅勝衣,疑而訊之,曰:「我耕於野,與甲遇,強以梃與我,半夜挾我東,使候諸門,不知其它也。」問吏:「法何如?」曰:「死。」覺止誅其首,後遂為例。



 知應天府,入為太常少卿,易秘書少監。哲宗即位,兼侍講,遷右諫議大夫。時諫官、御史論事有限,毋得越職。覺請申《唐六典》及天禧詔書,凡發令造事之未便,皆得奏陳。論宰相蔡確、韓縝進不以德,確自訟有功無罪,覺
 隨所言折之,確竟去。縝白遷覺給事中,辭曰:「間者,執政畏人議己,則遷官以餌之,願與縝俱罷。」逾月,縝去。



 進吏部侍郎,領右選,在選萬五千員,闕才五之二,至有三年不得調者。覺請自軍功、保甲進者補指使,宗室袒免從員外置,一日得闕數千。改主左選,請磨勘歲以百人為限。擢御史中丞,數月,以疾請罷,除龍圖閣學士兼侍講,提舉醴泉觀,求舒州靈仙觀以歸。哲宗遣使存勞,賜白金五百兩。卒,年六十三。



 覺有德量,為王安石所逐。安石
 退居鐘山,覺枉駕道舊,為從容累夕;迨其死,又作文以誄,談者稱之。紹聖中,以覺為元祐黨,奪職追兩官。徽宗即位,復官職。有《文集》、《奏議》六十卷,《春秋傳》十五卷。弟覽。



 覽字傅師。擢第,知尉氏縣。有屯將遇下虐,士卒謀因大閱殺之以叛。覽聞之,馳往,士猶群語不顧,覽呼諭之曰:「將誠無狀,然天子何負汝輩,乃欲致族滅邪?」皆感謝去就列。屯將徐至,覽命吏趣具奏,眾意遂安。神宗壯其材,以為司農主簿。舒但判寺且兼諫院,欲引覽自助,覽拒
 不答。但怒,用帳籍違事劾之。出提舉利州、湖南常平,改京西轉運判官,入為右司員外郎。荊湖開疆,命往相其便。覽言:「沅州所招溪洞百三十,宜從本郡隨事要束,勿建官置戍以為民困。自誠州至融江口,可通西廣鹽,以省北道餉饋。」悉從之。



 使還,為河東、河北轉運副使,加直龍圖閣,歷知河中應天府、江淮發運使。進寶文閣待制,由桂徙廣,又改渭州。夏人入邊,檄大將苗履御之,履稱疾移告,立按正其罪,竄諸房陵,轅門肅然。召知開封府,
 至則拜戶部侍郎。與蔡京論役法不合,以龍圖閣直學士知太原。夏人據橫山,並河為寨,秦、晉之路皆塞。覽謀復取葭蘆戍,阻險不得前。夏人數萬屯境上,覽下令吾兵少,須滿五萬。及西夏人聞而濟師,覽不為動,相持益久,忽令具糗糧,嚴兵械,曰:「敵至矣!」居數日,果大入,覽奮擊敗之,遂城葭蘆而還。策勛,加樞密直學士。



 覽雖立邊功,議論多觸執政,屢遭絀削,歷知河南、永興,徙成都。辭不行,降為寶文閣待制。卒,年五十九。



 李常,字公擇,南康建昌人。少讀書廬山白石僧舍。既擢第,留所抄書九千卷,名舍曰李氏山房。調江州判官、宣州觀察推官。發運使楊佐將薦改秩,常推其友劉琦,佐曰:「世無此風久矣。」並薦之。



 熙寧初,為秘閣校理。王安石與之善,以為三司條例檢詳官,改右正言、知諫院。安石立新法,常預議,不欲青苗收息。至是,疏言:「條例司始建,已致中外之議。至於均輸、青苗,斂散取息,傅會經義,人且大駭,何異王莽猥析《周官》片言,以流毒天下!」安石見
 之,遣所親密諭意,常不為止。又言:「州縣散常平錢,實不出本,勒民出息。」神宗詰安石,安石請令常具官吏主名,常以非諫官體,落校理,通判滑州。歲餘復職,知鄂州,徙湖、齊二州。齊多盜,論報無虛日。常得黠盜,刺為兵,使在麾下,盡知囊括處,悉發屋破柱,拔其根株,半歲間,誅七百人,奸無所匿。徙淮南西路提點刑獄。元豐六年,召為太常少卿,遷禮部侍郎。



 哲宗立,改吏部,進戶部尚書。或疑其少幹局,慮不勝任,質於司馬光。光曰:「用常主邦計,
 則人知朝廷不急於征利,聚斂少息矣。」常轉對,上七事,曰崇廉恥,存鄉舉,別守宰,廢貪贓,審疑獄,擇儒師,修役法。時役法差、免二科未定,常謂:「法無新陳,便民者良;論無彼己,可久者確。今使民俱出貲則貧者難辦,俱出力則富者難堪,各從其願,則可久爾。」乃折衷條上之。赦恩,蠲市易逋負不滿二百緡者,常請息過其數亦勿取。



 拜御史中丞,兼侍讀,加龍圖閣直學士。論取士,請分詩賦、經義為兩科,以盡所長。初,河決小吳,議者欲自孫村口
 導還故處,及是,役興,常言:「京東、河北饑困,不宜導河。」詔罷之。諫官劉安世以吳處厚繳蔡確詩為謗訕,因力攻確。常上疏論以詩罪確,非所以厚風俗。安世並劾常,徙兵部尚書,辭不拜,出知鄧州。徙成都,行次陜,暴卒,年六十四。有文集、奏議六十卷,《詩傳》十卷,《元祐會計錄》三十卷。



 常長孫覺一歲,始與覺齊名,俱受知於呂公著。其論議趣舍,大略多同;所終官職又同;其死,先後一夕雲。



 孔文仲,字經父,臨江新喻人。性狷直,寡言笑,少刻苦問
 學,號博洽。舉進士,南省考宮呂夏卿,稱其詞賦贍麗,策論深博,文勢似荀卿、楊雄,白主司,擢第一。調餘杭尉。恬介自守,不事請謁。轉運使在杭,召與議事,事已,馳歸,不詣府。人問之,曰:「吾於府無事也。」再轉臺州推官。



 熙寧初,翰林學士範鎮以制舉薦,對策九千餘言,力論王安石所建理財、訓兵之法為非是,宋敏求第為異等。安石怒,啟神宗,御批罷歸故官。齊恢、孫固封還御批,韓維、陳薦、孫永皆力言文仲不當黜,五上章,不聽。範鎮又言:「文仲
 草茅疏遠,不識忌諱。且以直言求之,而又罪之,恐為聖明之累。」亦不聽。蘇頌嘆曰:「方朝廷求賢如饑渴,有如此人而不見錄,豈其論太高而難合邪,言太激而取怨邪?」



 吳充為相,欲置之館閣,又有忌之者,僅得國子直講。學者方用王氏經義進取,文仲不習其書,換為三班主簿,出通判保德軍。時征西夏,眾數十萬皆道境上,久不解,邊人厭苦。文仲陳三不便,曰:「大兵未出,而丁夫預集;河東顧夫,勞民而損費;諸路出兵,首尾不相應。虞、夏、商、周
 之盛,未嘗無外侮,然懷柔制御之要,不在彼而在此也。」



 元祐初,哲宗召為秘書省校書郎,進禮部員外郎。有言:「皇族唯楊、荊二王得稱皇叔,餘宜各系其祖,若唐人稱諸王孫之比。」文仲曰:「上新即位,宜廣敦睦之義,不應疏間骨肉。」議遂寢。遷起居舍人,擢左諫議大夫。日食七月朔,上疏條五事,曰邪說亂正道,小人乘君子,遠服侮中國,斜封奪公論,人臣輕國命,宜察此以消厭兆祥。論青苗、免役,首困天下,保甲、保馬、茶鹽之法,為遣螫留蠹。改
 中書舍人。



 三年,同知貢舉。文仲先有寒疾,及是,晝夜不廢職。同院以其形瘵,勸之先出,或居別寢。謝曰:「居官則任其責,敢以疾自便乎!」於是疾益甚,還家而卒,年五十一。士大夫哭之皆失聲。蘇軾拊其柩曰:「世方嘉軟熟而惡崢嶸,求勁直如吾經父者,今無有矣!」詔厚恤其家,命弟平仲為江東轉運判官,視其葬。



 初,文仲與弟武仲、平仲皆以文聲起江西,時號「三孔」。後追貶梅州別駕。元符末,復其官。有文集五十卷。



 武仲字常父。幼力學,舉進士,中甲科。調穀城主簿,選教授齊州,為國子直講。喪二親,毀瘠特甚,右肱為不舉。元祐初,歷秘書省正字、校書,集賢校理,著作郎,國子司業。嘗論科舉之弊,詆王氏學,請復詩賦取士。又欲罷大義,而益以諸經策,御試仍用三題。進起居郎兼侍講邇英殿,除起居舍人,數月,拜中書舍人,直學士院。



 初,罷侍從轉對,專責以論思。武仲言:「茍不持之以法,則言與不言,將各從其意。願輪二人次對。」時議祠北郊,久不決。武仲
 建用純陰之月親祠,如神州地祗。擢給事中,遷禮部侍郎,以寶文閣待制知洪州。請:「從臣為州者,杖以下公坐止劾官屬,俟獄成,聽大理約法,庶幾刑不逮貴近,又全朝廷體貌之意。」遂著為令。



 徙宣州,坐元祐黨奪職,居池州。卒,年五十七。元符末,追復之。所著《詩書論語說》、《金華講義》、《內外制》、《雜文》共百餘卷。



 平仲字義甫。登進士第,又應制科。用呂公著薦,為秘書丞、集賢校理。文仲卒,歸葬南康。詔以平仲為江東轉運
 判官護葬事,提點江浙鑄錢、京西刑獄。紹聖中,言者詆其元祐時附會當路,譏毀先烈,削校理,知衡州。提舉董必劾其不推行常平法,陷失官米之直六十萬,置獄潭州。平仲疏言:「米貯倉五年半,陳不堪食,若非乘民闕食,隨宜洩之,將成棄物矣。倘以為非,臣不敢逃罪。」乃徙韶州。又坐前上書之故,責惠州別駕,安置英州。徽宗立,復朝散大夫,召為戶部、金部郎中,出提舉永興路刑獄,帥鄜延、環慶。黨論再起,罷,主管袞州景靈官,卒。平仲長史
 學,工文詞,著《續世說》、《繹解稗》、《詩戲》諸書傳於世。



 李周,字純之,馮翊人。登進士第,調長安尉。歲饑,官為粥以食餓者,民坌集不可禁,縣以屬周,周設梐枑,間老少男女,無一亂者。都巡檢趙瑜詰盜南山,諸尉皆屬焉,瑜悍急,多行無禮,獨於周不敢肆。



 轉洪洞令。民有世絕而官錄其產者,其族晚得遺券,周取以還之。郡吏咎周,周曰:「利民,所以利國也。」縣之南有澗,支流湓入,歲賦菑楗,調徒遏之。周始築新堤,民不告病。改知雲安縣,蠲鹽井
 之征且百萬。通判施州。州介群獠,不習服牛之利,為闢田數千畝,選謫戍知田者,市牛使耕,軍食賴以足。



 司馬光將薦為御史,欲使來見,周曰:「司馬公之賢,吾固願見,但聞薦而往,所謂『呈身御史』也。」卒不往。神宗詔近臣舉士,孫固以周聞。神宗召對,謂曰:「知卿不游權門,識今執政乎?」對曰:「不識也。」「識司馬光乎?」曰:「不識也。」訪御邊之術,曰:「四邊,手足爾。若疲中國以勤遠略,致百姓窮困,聚為盜賊,懼成腹心之憂。」神宗頷之,翼日,語固曰:「李周,樸忠
 之士也。朕且以為御史。」執政意其異己,請試以事。除提點京西刑獄。



 時方興水利,或請釃湍河為六渠,以益鉗廬陂水,度用工八十萬。周曰:「湍河原高委下,捍以堤,猶患決溢,若又導之,必致為害。」乃疏言:「渠成未可必,而費已不貲。盍姑鑿其一而試之,倘可以足用,行之。」渠卒無功。明年,河溢,鄧城幾沒,始思其議。竟以直道罷,判西京國子監。慈聖後復士,庀職陵下,中貴人至者旁午,次舍帟幕,競為華靡。周曰:「臣子執喪,不能寢苫枕塊,奈何又
 從而侈乎?」訖役,山陵使第功載,人人自言,周獨否。



 哲宗立,召為職方郎中。朝廷議和西夏,畀以侵地,至欲棄蘭州。周曰:「隴右故為唃氏所有,常為吾藩籬。今唃氏破滅,若棄之,必歸夏人。彼以區區河南,百年為勍敵,茍益以河湟,是盡得吐蕃之地,非秦、蜀之利也。」遂不果棄。遷太常少卿、秘書少監,以直龍圖閣為陜西轉運使,復入為太常少卿,進權工部侍郎,旋以集賢院學士知邠州,恩禮如待制。徙鳳翔府、河中府、陜州,提舉崇福宮,改集賢
 殿修撰。卒年八十。紹聖中,追貶賀州別駕,後復舊職。



 周自為小官,沉晦自匿,未嘗私謁執政,有公事,公詣中書白之。薛向使三司,欲闢為屬,及相見,卒不敢言,退而嘆曰:「若人未易屈也。」以是不偶於世。



 鮮于侁,字子駿,閬州人。唐劍南節度使叔明裔孫也。性莊重,力學。舉進士,為江陵右司理參軍。慶歷中,天下旱,詔求言。侁推災變所由興,又條當世之失有四,其語剴切。唐介與同鄉里,稱其名於上官,交章論薦。侁盛言
 左參軍李景陽、枝江令高汝士之美,乞移與之,介益以為賢。調黟令,攝治婺源。奸民汪氏富而狠,橫里中,因事抵法,群吏羅拜曰:「汪族敗前令不少,今不舍,後當詒患。」侁怒,立杖之,惡類屏跡。



 通判綿州。綿處蜀左,吏狃貪成風,至課卒伍供薪炭、芻豆,鬻果蔬多取贏直。侁一切弗取,郡守以下效之。趙抃使蜀,薦於朝,未及用。從何郯闢,簽書永興軍判官。萬年令不任職,系囚累百,府使往治,數日,空其獄。神宗詔求直言,侁為蔡河撥發,應詔陳十六
 事,神宗愛其文。詔近臣舉所知,範鎮以人先應選,除利州路轉運判官。



 初,王安石居金陵,有重名,士大夫期以為相。侁惡其沽激要君,語人曰:「是人若用,必壞亂天下。」至是,乃上書論時政,曰:「可為憂患者一,可為太息者二,其它逆治體而召民怨者,不可概舉。」其意專指安石。安石怒,毀短之。神宗曰:「侁有文學,可用。」安石曰:「陛下何以知之?」神宗曰:「有章奏在。」安石乃不敢言。初,助役法行,詔諸路各定所役緡錢。利州轉運使李瑜定四十萬,侁爭之
 曰:「利州民貧地瘠,半此可矣。」瑜不從,各以其事聞。時諸路役書皆未就,神宗是侁議,諭司農曾布使頒以為式。因黜瑜,而升侁副使,仍兼提舉常平。部民不請青苗錢,安石遣吏廉按,且詰侁不散之故。侁曰:「青苗之法,願取則與,民自不願,豈能強之哉!」



 左藏庫使周永懿守利州,貪虐不法,前使者畏其兇,莫敢問。侁捕械於獄,流之衡湘,因請更以文臣為守,並易班行領縣事。凡居部九年,治所去閬中近,姻戚旁午,待之無所私,各得其歡心。蘇
 軾稱侁上不害法,中不廢親,下不傷民,以為「三難」。二稅輸絹綿,侁奏聽民以畸零納直。其後有李元輔者,輒變而多取之,父老流涕曰:「老運使之法,何可改?」蓋侁之侄師中亦居是職,故稱「老」以別之。



 徙京東西路。河決澶淵,議欲勿塞,侁言:「東州匯澤惟兩濼,夏秋雨淫,猶溢而害,若縱大河注其中,民為魚矣。」作《議河書》上之,神宗嘉納。後兩路合為一,以侁為轉運使。



 時王安石、呂惠卿當路,正人多不容。侁曰:「吾有薦舉之權,而所列非賢,恥也。」故
 凡所薦如劉摯、李常、蘇軾、蘇轍、劉分文、範祖禹,皆守道背時之士。元豐二年召對,命知揚州。神宗曰:「廣陵重鎮,久不得人,今朕自選卿往,宜善治之。」蘇軾自湖州赴獄,親朋皆絕交。道揚,侁往見,臺吏不許通。或曰:「公與軾相知久,其所往來書文,宜焚之勿留,不然,且獲罪。」人先曰:「欺君負友,吾不忍為,以忠義分譴,則所願也。」為舉吏所累,罷主管西京御史臺。



 哲宗立,念東國困於役,吳居厚掊斂虐害,竄之,復以侁使京東。司馬光言於朝曰:「以侁之賢,
 不宜使居外。顧齊魯之區,凋敝已甚,須侁往救之,安得如侁百輩,布列天下乎?」士民聞其重臨,如見慈父母。召為太常少卿。侍從議神宗廟配享,有欲用王安石、吳充者,侁曰:「先朝宰相之賢,誰出富弼右?」乃用弼。拜左諫議大夫。



 侁見哲宗幼沖,首言君子小人消長之理甚備。又言:「制舉,誠取士之要,國朝尤為得人。王安石用事,諱人詆訾新政,遂廢其科。今方搜羅俊賢,廓通言路,宜復六科之舊。」又乞罷大理獄,許兩省、諫官相往來,減特奏名
 舉人,嚴出官之法,京東鹽得通商,復三路義勇以寬保甲,罷戎、瀘保甲以寬民力,事多施行。在職三月,以疾求去。除集賢殿修撰、知陳州。詔滿歲進待制。居無何,卒,年六十九。



 侁刻意經術,著《詩傳》、《易斷》,為範鎮、孫甫推許。孫復與論《春秋》,謂今學者不能如之。作詩平澹淵粹,尤長於《楚辭》,蘇軾讀《九誦》,謂近屈原、宋玉,自以為不可及也。



 顧臨,字子敦,會稽人。通經學,長於訓詁。皇祐中,舉說書科,為國子監直講,遷館閣校勘、同知禮院。熙寧初,神宗
 以臨喜論兵,詔編《武經要略》。初命都副承旨提舉,神宗謂臨館職,改提舉曰館幹。且召臨問兵,對曰:「兵以仁義為本,動靜之機,安危所系,不可輕也。」因條十事以獻。出權湖南轉運判官,提舉常平。議事戾執政意,罷歸。改同判武學,進集賢校理、開封府推官,請知穎州。入為吏部郎中、秘書少監,以直龍圖閣為河東轉運使。



 元祐二年,擢給事中。朝廷方事回河,拜臨天章閣待制、河北都轉運使。於是,翰林學士蘇軾與李常、王古、鄧溫伯、孫覺、胡
 宗愈言:「臨資性方正,學有根本,慷慨中立,無所回撓。自處東省,封駁論議,凜然有古人之風。僥幸之流,側目畏憚。忽去朝廷,眾所嗟惜,宜留置左右,以補闕遺,別選深知河事者往使河北。」諫議大夫梁燾亦言:「都漕之職,在外豈無其人,在朝求如臨者,恐不易得。」皆不報。臨至部,請因河勢回使東流。復以給事中召還。歷刑、兵、吏三部侍郎兼侍讀,為翰林學士。



 紹聖初,以龍圖閣學士知定州,徙應天、河南府。中人梁惟簡坐嘗事宣仁太后得罪,
 過洛,轉運使郭茂恂徇時宰意,劾臨與之宴集,奪職知歙州,又以附會黨人,斥饒州居住。卒,年七十二。徽宗立,追復之。



 李之純,字端伯,滄州無棣人。登進士第。熙寧中,為度支判官、江西轉運副使。御史周尹劾廣西提點刑獄許彥先受邕吏金,命之純往究其端,乃起於出婢之口。之純以為蕪俚之言,不治,彥先得免。



 徙成都路轉運使。成都歲發官米六千石,損直與民,言者謂惠民損上,詔下其
 議。之純曰:「蜀郡人恃此為生百年,奈何一旦奪之。」事遂已。秩滿復留,凡數歲,始還朝。神宗勞之曰:「遐方不欲數易大吏,使劍外安靖,年穀屢豐,以彰朝廷綏遠之意,汝知之乎?」以為右司郎中,轉太僕卿。



 元祐初,加直龍圖閣、知滄州,召為戶部侍郎。未至,改集賢殿修撰、河北都轉運使,進寶文閣待制、知瀛州。俄以直學士知成都府,還為戶部,三遷御史中丞。建言:「朝廷事下六部,但隨省吏視其前後批,以制緩急之序,是為胥吏顓處命令也。若
 大臣不暇省,宜令列曹長貳隨其所承,當行即行,當止即止,必稟而後決,毋拘於文,則吏不得舞權,而下情達矣。」又言:「眾賢和於朝,則萬物和於野。燮理陰陽,輔相之職。間者,國論稍虧雍睦,語言播傳,動系觀望,不可以不謹。」



 董敦逸、黃慶基論蘇軾托詞命以毀先帝,蘇轍以名器私所親,皆以臨司罷,之純疏其誣罔,乃更黜之。以疾,改工部尚書。紹聖中,劉拯劾其阿附轍,出知單州。卒,年七十五。從弟之儀。



 之儀字端叔。登第幾三十年,乃從蘇軾於定州幕府。歷樞密院編修官,通判原州。元符中,監內香藥庫。御史石豫言其嘗從蘇軾闢,不可以任京官,詔勒停。徽宗初,提舉河東常平。坐為范純仁遺表,作行狀,編管太平,遂居姑熟,久之,徙唐州,終朝請大夫。



 之儀能為文,尤工尺牘,軾謂入刀筆三昧。



 王覿,字明叟,泰州如皋人。第進士。熙寧中,為編修三司令式刪定官。不樂久居職,求潤州推官。二浙旱,郡遣吏
 視苗傷,承監司風旨,不敢多除稅。覿受檄覆按,嘆曰:「旱勢如是,民食已絕,倒廩贍之,猶懼不克濟,尚可責以賦邪?」行數日,盡除之。監司怒,捃摭百出。會朝廷遣使振貸,覿請見,為言民間利病。使者喜,歸薦之,除司農寺主簿,轉為丞。司農時為要官,進用者多由此選。覿拜命一日,即求外,韓絳高其節,留檢詳三司會計。絳出穎昌,闢簽書判官。坐在潤公闕免,屏居累年,起為太僕丞,徙太常。



 哲宗立,呂公著、范純仁薦其可大任,擢右正言,進司諫。
 上疏言:「國家安危治亂,系於大臣。今執政八人,而奸邪居半,使一二元老,何以行其志哉?」因極論蔡確、章惇、韓縝、張璪朋邪害正。章數十上,相繼斥去。又劾竄呂惠卿。朝論以大奸既黜,慮人情不安,將下詔慰釋之,且戒止言者。覿言:「誠出於此,恐海內有識之士,得以輕議朝廷。舜罪四兇而天下服,孔子誅少正卯而魯國治。當是之時,不聞人情不安,亦不聞出命令以悅其黨也。蓋人君之所以御下者,黜陟二柄而已。陟一善而天下之為善
 者勸,黜一惡而天下之為惡者懼。豈以為惡者懼而朝廷亦為之懼哉?誠為陛下惜之。」覿言雖切,然不能止也。



 夏主新立,有輕中國心。覿曰:「小羌窺我厭兵,故桀驁若是。然所當憂者,不在今秋而在異日,所當謹者,不在邊備而在廟謨。翕張取予之權,必持重而後可。」洮東擒鬼章,檻至闕下,覿曰:「老羌雖就擒,其子統眾如故,疆土種落未減於前,安可遽戮以賈怨。宜處之洮、岷、秦、雍間,以示含容好生之德,離其石交而壞其死黨。」又言:「今民力
 凋瘵,邊費亡極,不可不深為之計。」於是疏將帥非其人者請易之,茶鹽之害民者請革之,至逋債、振瞻、賦斂、科須,皆指陳其故。



 差役法復行,覿以為:「朝廷意在便民,而議者遂謂免役法無一事可用。夫法無新舊,惟善之從。」因採掇數十事於差法有助可以通行者上之。遂論青苗之害,乞盡罷新令,而復常平舊法,曰:「聚斂之臣,惟知罔利自媒,不顧後害。以國家之尊,而與民爭錐刀之利,何以示天下?」又言:「刑罰世輕世重。熙寧大臣,謂刑罰不
 重,則人無所憚。今法令已行,可以適輕之時,願擇質厚通練之士,載加芟正。」於是置局編匯,俾覿預焉。大抵皆用中典,《元祐敕》是也。



 神宗復唐制,諫官分列兩省。至是,大臣議徙之外門,而以其直舍為制敕院,名防漏洩,實不欲使與給舍相通。覿爭之曰:「制敕院,吏舍也。奪諫省以廣吏舍,信胥吏而疑諍臣,何示不廣也。」乃不果徙。



 覿在言路,欲深破朋黨之說。朱光庭訐蘇軾試館職策問,呂陶辯其不然,遂起洛、蜀二黨之說。覿言:「軾之辭,不過失
 輕重之體爾。若悉考同異,深究嫌疑,則兩岐遂分,黨論滋熾。夫學士命詞失指,其事尚小;使士大夫有朋黨之名,大患也。」帝深然之,置不問。



 尋改右司員外郎,未幾,拜侍御史、右諫議大夫。坐論尚書右丞胡宗愈,出知潤州,加直龍圖閣、知蘇州。州有狡吏,善刺守將意以撓權,前守用是得譏議。覿窮其奸狀,置於法,一郡肅然。民歌詠其政,有「吏行水上,人在鏡心」之語。徙江、淮發運使,入拜刑、戶二部侍郎,與豐稷偕使遼,為遼人禮重。紹聖初,以
 寶文閣直學士知成都府。蜀地膏腴,畝千金,無閑田以葬,覿索侵耕官地,表為墓田。江水貫城中為渠,歲久湮塞,積苦霖潦而多水災,覿疏治復故,民德之,號「王公渠」。徙河陽,貶少府少監,分司南京,又貶鼎州團練副使。



 徽宗即位,還故職,知永興軍。過闕,留為工部侍郎,遷御史中丞。改元詔下,覿言:「『建中』之名,雖取皇極。然重襲前代紀號,非是,宜以德宗為戒。」時任事者多乖異不同,覿言:「堯、舜、禹相授一道,堯不去四兇而舜去之,堯不舉元凱
 而舜舉之,事未必盡同;文王作邑於豐而武王治鎬,文王關市不征,澤梁無禁,周公征而禁之,不害其為善繼、善述。神宗作法於前,子孫當守於後。至於時異事殊,須損益者損益之,於理固未為有失也。」當國者忿其言,遂改為翰林學士。



 日食四月朔,帝下詔責躬,覿當制,有「惟德弗類,未足以當天心」之語,宰相去之,乃力請外。以龍圖閣學士知潤州,徙海州,罷主管太平觀,遂安置臨江軍。



 覿清修簡澹,人莫見其喜慍。持正論始終,再罹譴逐,
 不少變。無疾而卒,年六十八。紹興初,追復龍圖閣學士。從子俊義。



 俊義字堯明。游學京師,資用乏,或薦之童貫,欲厚聘之,拒不答。林靈素設講席寶菉宮,詔兩學選士問道。車駕將臨視推恩,司成以俊義及曹偉應詔,俊義辭焉。人曰:「此顯仕捷徑也,不可失。」俊義曰:「使辭不獲命,至彼亦不拜。倘見困辱,則以死繼之。」逮至講所,去御幄跬步,內侍呼姓名至再,俊義但望幄致敬,不肯出;次呼曹偉,偉回
 首,俊義目之,亦不出。既罷,皆為之懼,俊義處之恬然。



 以太學上舍選,奏名列其下,徽宗親程其文,擢為第一。及賜第,望見容貌甚偉,大說,顧侍臣曰:「此朕所親擢也,真所謂『俊義』矣。自古未有人主自為主司者,宜即超用。」蔡京邀使來見,曰:「一見我,左右史可立得。」俊義不往,僅拜國子博士。居二年,乃得改太學博士。



 鄆王謁先聖,有司議諸生門迎。俊義曰:「此豈可施於人臣哉?禮如見宰相足矣。」乃序立敦化堂下,及王至,猶辭不敢當。進吏部員
 外郎。嘗入對,帝問:「卿知前所以親擢乎?蓋主司之意不一,是以天子自提文衡也。衛膚敏、吳安國今安在?」具以對,即召為館職,而遷俊義右司員外郎。為王黼所惡,以直秘閣知嶽州。卒,年四十七。



 俊義與李祁友善,首建正論於宣和間。當是時,諸公卿稍知分別善惡邪正,兩人力也。祁字肅遠,亦知名士,官不顯。



 馬默,字處厚,單州成武人。家貧,徒步詣徂徠從石介學。諸生時以百數,一旦出其上。既而將歸,介語諸生曰:「馬
 君他日必為名臣,宜送之山下。」



 登進士第,調臨濮尉,知須城縣。縣為鄆治所,鄆吏犯法不可捕,默趨府,取而杖之客次,闔府皆驚。曹佾守鄆,心不善也,默亦不為屈。後守張方平素貴,掾屬來前,多閉目不與語。見默白事,忽開目熟視久之,盡行其言,自是諉以事。治平中,方平還翰林,薦為監察御史裹行,遇事輒言無顧。方平間遣所親儆之曰:「言太直,得無累舉者乎?」默謝曰:「辱知之深,不敢為身謀,所以報也。」



 時議尊崇濮安懿王,臺諫呂誨等
 力爭以為不可,悉出補外。默請還之,不報。遂上言:「濮王生育聖躬,人誰不知。若稱之為親,義無可據,名之不正,失莫大焉。願蔽自宸心,明詔寢罷,以感召和氣,安七廟之神靈,是一舉而眾善隨之也。」又言:「致治之要,求賢為本。仁宗以官人之權,盡委輔相,數十年間,賢而公者無幾。官之進也,不由實績,不自實聲,但趨權門,必得顯仕。今待制以上,數倍祖宗之時,至謀一帥臣,則協於公議者十無三四。庶僚之眾,不知幾人,一有難事,則曰無人
 可使。豈非不才者在上,而賢不肖混淆乎?願陛下明目達聰,務既其實,歷試而超升之,以幸天下。」



 刑部郎中張師顏提舉諸司庫務,繩治不法,眾吏懼搖,飛語讒去之。默力陳其故,以為:「惡直醜正,實繁有徒。今將去積年之弊,以興太平,必先官舉其職。宜崇獎師顏,厲以忠勤,則尸素括囊之徒,知所勸矣。」



 西京會聖宮將創仁宗神御殿,默言:「事不師古,前典所戒。漢以諸帝所幸郡國立廟,知禮者非之。況先帝未嘗幸洛,而創建廟祀,實乖典則。
 願以禮為之節,義為之制,亟止此役,以章清靜奉先之意。」會地震河東、陜西郡,默以為陰盛,慮為邊患,宜備之。後數月,西夏果來侵。



 神宗即位,以論歐陽修事,通判懷州。上疏陳十事:一曰攬威權,二曰察奸佞,三曰近正人,四曰明功罪,五曰息大費,六曰備兇年,七曰崇儉素,八曰久任使,九曰擇守宰,十曰御邊患。攬威權,則天子勢重,而大臣安矣;察奸佞,則忠臣用,而小人不能幸進矣;近正人,則諫諍日聞,而聖性開明矣;明功罪,則朝廷無
 私,而天下服矣;息大費,則公私富,而軍旅有積矣;備兇年,則大恩常施,而禍亂不起矣;崇儉素,則自上化下,而民樸素矣;久任使,則官不虛授,而職事舉矣;擇守宰,則庶績有成,而民受賜矣;御邊患,則四遠畏服,而中國強矣。



 除知登州。沙門島囚眾,官給糧者才三百人,每益數,則投諸海。砦主李慶以二年殺七百人,默責之曰:「人命至重,恩既貸其生,又從而殺之,不若實時死鄉里也。汝胡不以乏糧告,而顓殺之如此?」欲按其罪,慶懼,自縊死。
 默為奏請,更定《配島法》凡二十條,溢數而年深無過者移登州,自是多全活者。其後蘇軾知登州,父老迎於路曰:「公為政愛民,得如馬使君乎?」



 徙知曹州,召為三司鹽鐵判官。以默與富弼善,且論新法不便,出知濟、袞二州。還,提舉三司帳司。為神宗言用兵形勢,及指畫河北山川道里,應對如流。神宗喜,將用之,大臣滋不悅,以提點京東刑獄。



 默性剛嚴疾惡,部吏有望風投檄去者。金鄉令以賄著,其父方執政,詒書曰:「馬公素剛,汝有過,將不
 免。」令懼,悉取不義之物焚撤之。改廣西轉運使,會安化等蠻歲饑內寇,默上平蠻方略,以為「勝負不在兵而在將。富良宵遁,郭逵怯懦;邕城陷沒,蘇緘老謬;歸仁鋪覆軍,陳曙先走;昆侖關喪師,張守節不戰,儂智高破亡,因狄青之智勇;歐希範之誅滅,乃杜杞之方略,此足驗矣。」



 以疾求歸,知徐州。屬城利國監苦吳居厚之虐,默皆革之。召為司農少卿。司馬光為相,欲盡修祖宗法,問默以復鄉差衙前法如何?默曰:「不可。如常平,自漢為良法,豈
 宜盡廢?去其害民者可也。」其後役人立為一州一縣法,常平提舉官省歸提刑司,頗自默發之。除河東轉運使。時議棄葭蘆、吳堡二砦,默奏控扼險阻,敵不可攻,棄之不便。由是二砦得不棄。移袞州,請褒錄石介後,詔官其孫。東州薦饑,流民大集,所振活數萬計。入拜衛尉卿,權工部侍郎,轉戶部。告老,以寶文閣待制復知徐州,改河北都轉運使。



 初,元豐間,河決小吳,因不復塞,縱之北流。元祐議臣以為東流便,水官遂與之合。默與同時監司
 上議,以北流為便。御史郭知章復請從東流,於是作東西馬頭,約水復故道,為長堤壅河之北流者,勞費甚大。明年,復決而北,竟不能使之東。



 久之,告老,提舉鴻慶宮。紹聖時,坐附司馬光,落待制致仕。元符三年,復之。卒,年八十。紹興中,以其子純請,贈開府儀同三司,加贈太保。



 論曰:《詩》云:「時靡有爭,王心載寧。」王安石之為相,可謂致天下之爭,而君心不寧矣。孫覺、李常力諍新法,寧失故人之意,毅然去之而無悔,賢哉。孔文仲之策制科,以微
 官慷慨論事,言雖不聽,而名徹上聰。安石既斥其人,又廢其科,何遷怒之甚耶!鮮于侁早識安石敗事,與呂誨同見幾先。馬默用張方平薦為御史,至於盡言而不諱,方平止之而不聽,斯為不負知己矣。李周之耿介,顧臨之用兵,李之純、王覿再黜而不改其正,亦足以見一時之多賢焉。



\end{pinyinscope}