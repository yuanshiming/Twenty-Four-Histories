\article{志第三十二 律歷十二}

\begin{pinyinscope}

 紀元歷



 崇寧《紀元歷》



 演紀上元上章執徐之歲,距元符三年庚辰,歲積二千八百六十一萬三千四百六十算;至崇寧五年丙戌,歲積二千八百六十一萬三千四百六十六
 算。



 步氣朔第一



 日法:七千二百九十。



 期實:二百六十六萬二千六百二十六。



 朔實:二十一萬五千二百七十八。



 歲周:三百六十五日、餘一千七百七十六。



 氣策:一十五、餘一千五百九十二太。



 朔策:二十九、餘三千八百六十八。



 望策:一十四、餘五千五百七十九。



 弦策:七、餘二千七百八十九半。



 中盈分:三千一百八十五半。



 朔虛分:三千四百二十二。



 沒限:五千六百九十七少。



 旬周:四十三萬七千四百。



 紀法:六十。



 求天正冬至:置上元距所求積年,以期實乘之,為天正
 冬至氣積分;滿旬周去之,不滿,如日法而一為大餘,不盡為小餘。其大餘命己卯,算外,即所求年天正冬至日辰及餘。



 求次氣:置天正冬至大、小餘,以氣策加之,四分之一為少,之二為半,之三為太。如滿秒母,收從小餘,小餘滿日法從大餘,大餘盈紀法乃去之。



 去命如前,即次氣日辰及餘。



 求天正經朔:置天正冬至氣積分,以朔實去之,不盡,為天正閏餘;用減氣積分,餘為天正十一月經朔加時積
 分。滿旬周去之,不滿,如日法而一為大餘,不盡為小餘。其大餘命己卯,算外,即所求年天正十一月經朔日辰及餘。



 求弦望及次朔經日:置天正經朔大、小餘,以弦策累加之,去命如前,即各得弦、望及次朔經日辰及餘。



 求沒日:置有沒常氣小餘,凡常氣小餘在沒限已上者,為有沒之氣。



 六十乘之,用減四十四萬三千七百七十一,餘滿六千三百七十一而一為日,不滿為餘。命日起其氣初日辰,算外,即
 為氣內沒日辰。



 求滅日:置有滅經朔小餘,凡經朔小餘不滿朔虛分者,為有滅之朔。



 三十乘之,滿朔虛分而一為日,不滿為餘。命日起其月經朔日辰,算外,即為月內滅日辰。



 步發斂



 候策:五、餘五百三十、秒五十五。



 卦策:六、餘六百三十七、秒六。



 土王策:三、餘三百一十八、秒三十三。



 歲閏:七萬九千二百九十。



 月閏:六千六百七半。



 閏限:二十萬八千六百七十半。



 辰法:一千二百一十五。



 半辰法:六百七半。



 刻法:七百二十九。



 秒法:六十。



 求七十二候:各置中節大、小餘命之,為初候;以候策加
 之為次候;又加之為末候。各命己卯,算外,即得所求日辰。



 求六十四卦:各置中氣大、小餘命之,為公卦用事日;以卦策加之,得闢卦用事日;又加之,得諸侯內卦用事日;以土王策加之,得十有二節之初諸侯外卦用事日;又加之,得大夫卦用事日;復以卦策加之,得卿卦用事日。各命己卯,算外,即得所求日辰。



 求五行用事:各因四立之節大、小餘命之,即春木、夏火、
 秋金、冬水首用事日。以土王策減四季中氣大、小餘,即其季土始用事之日。各命己卯,算外,即得所求日辰。



 七十二候及卦目與前歷同。



 求中氣去經朔:置天正閏餘,以月閏累加之,滿日法為閏日,不滿為餘,即其月中氣去經朔日算。因求卦候者,各以卦、候策依次累加減之,中氣前減,中氣後加。



 各得其月卦、候去經朔日算。



 求發斂加時:置所求小餘,倍之,如辰法而一為辰數,不
 滿,五因之,如刻法而一為刻,不盡為分。命辰數起子正,算外,即各得加時所在辰、刻及分。如半辰數,即命起子初。



 步日躔



 周天分:二億一千三百一萬八千一十七。



 歲差:七千九百三十七。



 周天度:三百六十五、約分二十五、秒七十二。



 象限:九十一、約分三十一、秒九。



 乘法:一百一十九。



 除法:一千八百一十一。



 秒法:一百。



 求每日盈縮分先後數:
 置所求盈縮分,以乘法乘之,如除法而一,為其氣中平率;與後氣中平率相減,為合差;半合差,加減其氣中平率,為初、末泛率。至後加為初、減為末,分後減為初、加為末。



 又以乘法乘合差,如除法而一,為日差;半日差,加減初、末泛率,為初、末定率。至後減初加末,分後加初減末。



 以日差累加
 減其氣初定率,為每日盈縮分;至後減,分後加。



 各以每日盈縮分加減氣下先後數。冬至後,積盈為先,在縮減之;夏至後,積縮為後,在盈減之。其分、至前一氣,無後氣相減,皆因前氣合差為其氣合差。餘依前術,求朏朒仿此。



 求經朔弦望入氣:置天正閏日及餘,如氣策以下者,以減氣策,為入大雪氣;以上者去之,餘以減氣策,為入小雪氣:即天正十一月經朔入氣日及餘。求弦、望及後朔入氣,以弦策累加之,滿氣策去之,即各得弦、望及次朔入氣日及餘。



 求經朔弦望入氣朏朒定數:各以所入氣小餘乘其日
 損益率,如日法而一,所得,以損益其日下朏朒積,各為定數。



 赤道宿度



 斗:二十五



 牛:七少



 女:十一少



 虛:九少秒七十二



 危:十五半



 室:十七



 壁:八太。



 北方七宿九十四度秒七十二。



 奎:十六半



 婁:十二



 胃:十五



 昴:十一少



 畢:十七少。



 觜:半。



 參:十半。



 西方七宿八十三度。



 井:三十三少



 鬼:二半



 柳:十三太。



 星:六太



 張:十七少



 翼:十八太



 軫:十七



 南方七宿一百九度少。



 角:十二



 亢:九少



 氐:十六



 房:五太



 心:六少



 尾:十九少



 箕:十半



 東方七宿七十九度。



 按諸歷赤道宿次,就立全度,頗失真數。今依宋朝渾儀
 校測距度,分定太、半、少,用為常數,校之天道,最為密近。如考唐,用唐所測;考古,用古所測:即各得當時宿度。



 求冬至赤道日度:以歲差乘所求積年,滿周天分去之,不滿,覆減周天分,餘如五千八百三十二而一為分,不盡,退除為秒。其分,滿百為度,命起赤道虛宿七度外去之,至不滿宿,即所求年天正冬至加時日躔赤道宿度及分秒。



 求春分、夏至、秋分赤道日度:置天正冬至加時赤道日
 度,累加象限,滿赤道宿次去之,即各得春分、夏至、秋分加時日在宿度及分秒。



 求四正後赤道宿積度:置四正赤道宿全度,以四正赤道日度及分減之,餘為距後度;以赤道宿度累加之,各得四正後赤道宿積度及分。



 求赤道宿積度入初末限:視四正後赤道宿積度及分,在四十五度六十五分、秒五十四半已下為入初限;已上,用減象限,餘為入末限。



 求二十八宿黃道度:以四正後赤道宿入初、末限度及分,減一百一度,餘以初、末限度及分乘之,進位,滿百為分,分滿百為度,至後以減、分後以加赤道宿積度,為其宿黃道積度;以前宿黃道積度減之,其四正之宿,先加象限,然後以前宿減之。



 為其宿黃道度分。其分就近約為太、半、少。



 黃道宿度



 斗:二十三



 牛:七



 女:十一



 虛:九少秒七十二



 危:十六。



 室:十八。



 壁:九半。



 北方七宿九十三度太秒七十二。



 奎:十八



 婁:十二太



 胃:十五半



 昴:十一



 畢:十六半



 觜;半



 參:九太



 西方七宿八十四度。



 井:三十半



 鬼:二半



 柳:十三少



 星:六太



 張:十七太



 翼:二十



 軫:十八半



 南方七宿一百九度。



 角:十二太



 亢:九太



 氐:十六少



 房:五太



 心:六



 尾:十八少



 箕:九半



 東方七宿七十八度少。



 前黃道宿度,依今歷歲差所在算定。如上考往古,下驗將來,當據歲差,每移一度,依術推變當時宿度,然後可步七曜,知其所在。如徑求七曜所在,置所在積度,以前黃道宿積度減之,為所在黃道宿度及分。



 求天正冬至加時黃道日度:以冬至加時赤道日度及分秒,減一百一度,餘以冬至加時赤道日度及分秒乘
 之,進位,滿百為分,分滿百為度,命曰黃赤道差;用減冬至赤道日度及分秒,即所求年天正冬至加時黃道日度及分秒。



 求二十四氣加時黃道日度:置所求年冬至日躔黃赤道差,以次年黃赤道差減之,餘以所求氣數乘之,二十四而一,所得以加其氣中積及約分,又以其氣初日先後數先加後減之,用加冬至加時黃道日度,依宿次命之,即各得其氣加時黃道日躔宿度及分秒。如其年冬至加時赤道宿
 度空,分秒在歲差已下者,即加前宿全度。然求黃赤道差,餘依術算。



 求二十四氣晨前夜半黃道日度:置日法,以其氣小餘減之,餘副置之;以其氣初日盈縮分乘之,如萬約之,所得,盈加縮減其副,滿日法為度,不滿,退除為分秒,以加其氣加時黃道日度,即各得其氣一日晨前夜半黃道日度及分秒;每日加一度,以百約每日盈縮分為分秒,盈加縮減之,滿黃道宿次去之,即每日晨前夜半黃道日躔宿度及分秒。其二十四氣初日晨前夜半黃道日度,系屬前氣,自前氣攤算,即各得所
 求。



 求每日午中黃道日度:置一萬分,以所入氣日盈縮分盈加縮減而半之,滿百為分,不滿為秒,以加其日晨前夜半黃道日度,即其日午中日躔黃道宿度及分。



 求夏至加時黃道日度:置天正冬至加時黃道日度及分秒,以二至限及分秒加之,滿黃道宿次去之,不滿,為夏至加時黃道日度及分秒。



 求每日午中黃道積度:以二至加時黃道日度距至所
 求日午中黃道日度,為入二至後黃道積度及分。



 求每日午中黃道入初末限:視二至後黃道積度,在四十三度一十二分、秒八十七以下為初限;以上,用減象限,餘為入末限。其積度滿象限去之,為二分後黃道積度,在四十八度一十八分、秒二十二以下為初限;以上,用減象限,餘為入末限。



 求每日午中赤道日度:以所求日午中黃道積度,入至後初限、分後末限度及分秒,進三位,加二十萬二千五
 十少,開平方除之,所得,減去四百四十九半,餘在初限者,直以二至赤道日度加而命之;在末限者,以減象限,餘以二分赤道日度加而命之:即每日午中赤道日度。以所求日午中黃道積度,入至後末限、分後初限度及分秒,進三位,用減三十萬三千五十少,開平方除之,所得,以減五百五十半,餘在初限者,直以二分赤道日度加而命之;在末限者,以減象限,餘以二至赤道日度加而命之:即每日午中赤道日度。



 求太陽入宮日時刻及分:各置入宮宿度及分秒,以其日晨前夜半日度減之,餘以二十四乘,為時實;以其日太陽行度及分秒為法實,如法而一,為半時數;不滿,進二位,為刻實;以二十四乘,前法除之為刻,不滿,退除為分。其半時命起子正,算外,即得太陽入宮初正時、刻及分。其逐刻日、時及分,舊歷均其日數,從其簡略,未盡其詳。今但依入宮正術求之,即允協天道。



 步晷漏



 二至限:一百八十二、分六十二、秒一十八。



 象限:九十一、分二十一、秒九。



 一象度:九十一、分二十一、秒四十三。



 冬至後初限夏至後末限:六十二日、分二十。



 夏至後初限冬至後末限:一百二十日、分四十二。



 已上分秒母各同一百。



 冬至嶽臺晷影常數:一丈二尺八寸三分。



 夏至嶽臺晷影常數:一尺五寸六分。



 昏明分:一百八十二少。



 昏明刻:二分三百六十四半。



 辰刻:八分二百四十三。



 半辰刻:四分一百二十一半。



 刻法:七百二十九。



 求午中入氣:置所求日大餘及半法,以所入氣大、小餘減之,為其日午中入氣日及餘。



 求午中中積:置其氣中積,以午中入氣日及餘加之,其餘以日法退除為分秒。



 為所求日午中中積及分秒。



 求午中入二至後初末限;置午中中積及分,為入冬至後;滿二至限去之,為入夏至後。其二至後,如在初限已下為入初限;已上,覆減二至限,餘為入末限。



 求嶽臺晷影午中定數:冬至後初限、夏至後末限,以百通日,內分,自相乘為實,置之;以七百二十五除之,所得,加一十萬六百一十七,並入限分,折半為法,實如法而一為分,不滿,退除為小分,其分滿十為寸,寸滿十為尺,用減冬至嶽臺晷影常數,即得所求午中晷影定數。夏
 至後初限、冬至後末限,以百通日,內分,自相乘,為實,乃置入限分,九因,再折,加一十九萬八千七十五為法,其夏至前後,日如在半限以上者,減去半限,餘置於上,列半限於下,以上減下,餘以乘上,進二位,七十七除之,所得加法為定法,然後除之。



 實如法而一為分,不滿,退除為小分,其分滿十為寸,寸滿十為尺,以加夏至嶽臺晷影常數,即得所求日午中晷影定數。



 求每日日行積度:以午中入氣餘乘其日盈縮分,日法而一,冬至後盈加縮減、夏至後縮加盈減先後數,以先
 加後減中積日及分秒,滿與不足,進退其日,為所求日行積度及分秒。



 求每日赤道內外度:置所求日午中日行積度及分,如不滿二至限,在象限已下為冬至後度;象限已上,用減二至限,為夏至前度。如滿二至限去之,餘在象限以下為夏至後度;象限以上,用減二至限,為冬至前度。並置之於上,列象限於下,以上減下,餘以乘上,冬至前後五百一十七而一,夏至前後四百而一為度,不滿,退除為
 分,以加二至前後度,所得,用減象限,餘置於上,列二至限於下,以上減下,餘以乘上,其度、分、秒皆以百通,然後乘之。



 退一位,如三十四萬八千八百五十六而一為秒,滿百為分,分滿百為度,即所求日黃道去赤道內外度及分。冬至前後為外,夏至前後為內。



 求每日午中太陽去極度;以每日午中黃道去赤道內、外度及分,內減外加一象度及分,為每日午中太陽去極度及分。



 求每日日出入分晨昏分半晝分:置所求日黃道去赤道內外度及分,以三百六十三乘之,進一位,如二百三十九而一,所得,以加減一千八百二十二半,赤道內以減,赤道外以加。



 為所求日日出分;用減日法,為日入分。以昏明分減日出分,為晨分;加日入分,為昏分;以日出分減半法,為半晝分。



 求每日晝夜刻日出入辰刻:置日出分,倍之,進一位,滿刻法為刻,不滿為分,即所求日夜刻;以減百刻,餘為晝
 刻;半夜刻,滿辰刻為辰數;命子正,算外,即日出辰刻;以半辰刻加之,即命起時初。



 以晝刻加之,滿辰刻為辰數;命日出,算外,即日入辰刻及分。



 求每更點差刻及逐更點辰刻:置夜刻,減去十五刻,五而一,為更差;又五而一,為點差。以昏明刻加日入辰刻,即初更辰刻;以更點差刻累加之,滿辰刻及分去之,各得更點所入辰刻及分。



 求每日距中度及每更差度:置所求日黃道去赤道內、
 外度及分,以四千四百三十五乘之,如五千八百一十二而一為度,不滿,退除為分,以內加外減一百度七十二分、秒七為距中度。用減一百六十四度八十一分、秒五十七,餘四因,退一位,為每更差度。



 求昏曉五更及攢點中星:置距中度,以其日午中赤道日度加而命之,即昏中星所格宿次,命為初更中星;以每更差度加而命之,即二更中星;以每更差度累加之,滿赤道宿度去之,即逐更及攢點中星;加三十六度六
 十二分、秒五十七,滿赤道宿度去之,即曉中星。



 求九服晷景:各於所在測冬夏二至晷數,乃相減之,餘為二至差數。如地在嶽臺南測夏至晷景在表南者,並冬夏二至晷數為二至差數。其所求日在冬至後初限、夏至後末限者,置嶽臺冬至晷景常數,以所求日嶽臺午中晷景定數減之,餘以其處二至差數乘之,如嶽臺二至差數一丈一尺二寸七分而一,所得,以減其處冬至晷數,即其地其日中晷定數。所求日在夏至後初限、
 冬至後末限者,置所求日嶽臺午中晷景定數,以嶽臺夏至晷景常數減之,餘以其處二至差數乘之,如嶽臺二至差數而一,所得,以加其處夏至晷數,即其地其日中晷定數。如其處夏至景在表南者,以所得之數減其處夏至晷數,餘為其地其日中晷定數,亦在表南也。其所得之數多於其處夏至晷數,即減去夏至晷數,餘為其地其日中晷定數,在表北也。



 求九服所在晝夜漏刻:各於所在下水漏,以定其處冬
 夏二至夜刻,但得一至可矣,不必須要冬夏二至。



 乃與五十刻相減,餘為至差刻。置所求日黃道去赤道內外度及分,以至差刻乘之,進一位,如二百三十九而一為刻,不盡,以刻法乘之,復八而一為分,內減外加五十刻,即所求日夜刻;減百刻,餘為晝刻。其日日出入辰刻及更點差刻、每更點辰刻,並依嶽臺術求之。



 步月離



 轉周分:二十萬八百七十三、秒九百九十。



 轉周日:二十七、餘四千四十三、秒九百九十。



 朔差日:一、餘七千一百一十四、秒九千一十。



 望策:一十四、餘五千五百七十九。



 弦策:七、餘二千七百八十九半。



 已上秒母一萬。



 七日:初數六千四百七十八,初約分八十九;末數八百一十二,末約分一十一。



 十四日:初數五千六百六十六,初約分七十八;末數一千六百二十四,末約分二十二。



 二十一日:初數四千八百五十四,初約分六十七;末數二千四百三十六,末約分三十三。



 二十八日:初數四千四十三,初約分五十五。



 上弦:九十一度、分三十一、秒四十三。



 望:一百八十二度、分六十二、秒八十六。



 下弦:二百七十三度、分九十四、秒二十九。



 月平行:十三度、分三十六、秒八十七太。



 已上分、秒母皆同一百。



 求天正十一月經朔入轉:置天正十一月經朔加時積分,以轉周分及秒去之,不盡,滿日法除之為日,不滿為餘秒,命日,算外,即所求年天正十一月經朔加時入轉
 日及餘秒。若以朔差日及餘秒加之,滿轉周日及餘秒去之,即次朔加時入轉日。



 求弦望入轉:各因其月經朔加時入轉日及餘秒,以弦策累加之,去命如前,即上弦、望及下弦經日加時入轉日及餘秒。



 求朔弦望入轉朏朒定數:置入轉餘,以其日算外損益率乘之,如日法而一,所得,以損益其下朏朒積為定數。其四七日下餘如初數已下者,初率乘之,初數而一,以損益朏朒為定數。如初數已上者,以初數減之,餘乘末率,末數而一,用減初
 率,餘加朏朒為定數。其十四日下餘如初數已上者,初數減之,餘乘末率,末數而一,為朏朒定數。



 求朔弦
 望定日:
 各置經朔、弦、望小餘,以入氣、入轉朏朒定數朏減朒加之,滿與不足,進退大餘,命己卯,算外,各得定日日辰及餘。定朔幹名與後朔幹名同者月大,不同者月小,其月內無中氣者為閏月。凡注歷,觀定朔小餘,秋分後在日法四分之三已上者,進一日;春分後定朔日出分差如春分之日者,三約之,用減四分之三;定朔小餘及此數已上者,亦進一日;或當交虧初在日入已前者,其朔不進。弦、望定小餘不滿日出分者,退一日;望若有食虧初在日出已前者,定望小餘進滿日出分,亦進一日。又月行九道遲疾,有三大二小;日行盈縮累增損之,則有四大三小,理數然也。若俯循常儀,當察加時早晚,隨其所近而進退之,使不過三大二小。



 求定朔弦望加時日所在度:置定朔、弦、望約餘,副之,以乘其日盈縮分,萬約之,所得,盈加縮減其副,滿百為分,分滿百為度,以加其日夜半日度,命之,各得其日加時日躔黃道宿次。



 求平交日辰:置交終日及餘秒,以其月經朔加時入交泛日及餘秒減之,餘為平交入其月經朔加時後日算及餘秒,以加減其月經朔大、小餘,其大餘命己卯,算外,即平交日辰及餘秒。求次交者,以交終日及餘秒加之,大餘滿紀法去之,命如前,即次平
 交日辰及餘秒。



 求平交入轉朏朒定數:置平交小餘,加其日夜半入轉餘,以乘其日損益率,日法而一,所得,以損益其下朏朒積為定數。



 求正交日辰:置平交小餘,以平交入轉朏朒定數朏減朒加之,滿與不足,進退日辰,即正交日辰及餘秒;與定朔日辰相距,即所在月日。



 求經朔加時中積:各以其月經朔加時入氣日及餘,加
 其氣中積及餘,其日命為度,其餘以日法退除為分秒,即其月經朔加時中積度及分秒。



 求正交加時黃道月度:置平交入經朔加時後日算及約餘秒,以日法通日,內餘,進一位,如五千四百五十三而一為度,不滿,退除為分秒,以加其月經朔加時中積,然後以冬至加時黃道日度加而命之,即得其月正加時月離黃道宿度及分秒。如求次交者,以交終度及分秒加而命之,即得所求。



 求黃道宿積度:置正交加時黃道宿全度,以正交加時月離黃道宿度及分秒減之,餘為距後度及分秒,以黃道宿度累加之,即各得正交後黃道宿積度及分秒。



 求黃道宿積度入初末限:各置黃道宿積度及分秒,滿交象度及分去之,在半交像已下為初限;已上者,以減交象度,餘為入末限。入交積度、交象度並在交會術中。



 求月行九道宿度:凡月行所交,冬入陰歷,夏入陽歷,月行青道。冬至、夏至後,青道半交在春分之宿,當黃道東;立冬、立夏後,青道半交在立春之宿,當黃道東
 南:至所沖之宿亦如之。



 冬入陽歷,夏入陰歷,月行白道。冬至、夏至後,白道半交在秋分之宿,當黃道西;立冬、立夏後,白道半交在立秋之宿,當黃道西北:至所沖之宿亦如之。



 春入陽歷,秋入陰歷,月行朱道。春分、秋分後,朱道半交在夏至之宿,當黃道南;立春、立秋後,朱道半交在立夏之宿,當黃道西南:至所沖之宿亦如之。



 春入陰歷,秋入陽歷,月行黑道。春分、秋分後,黑道半交在冬至之宿,當黃道北;立春、立秋後,黑道半交在立冬之宿,當黃道東北:至所沖之宿亦如之。



 四序離為八節,至陰陽之所交,皆與黃道相會,故月行有九道。各以所入初、末限度及分減一百一度,餘以所入初、末限度及分乘之,半而退位為分,分
 滿百為度,命為月道與黃道泛差。凡日以赤道內為陰,外為陽;月以黃道內為陰、外為陽。故月行正交,入夏至後宿度內為同名,入冬至後宿度內為異名。其在同名者,置月行與黃道泛差,九因八約之,為定差。半交後、正交前以差減,正交後、半交前以差加。此加減出入六度,正如黃、赤道相交同名之差。若較之漸異,則隨交所在,遷變不常。



 仍以正交度距秋分度數乘定差,如象限而一,所得,為月道與赤道定差,前加者為減,減者為加。其在異名者,置月行與黃道泛差,七因八約
 之,為定差;半交後、正交前以差加,正交後、半交前以差減。此加減出入六度,異如黃赤道相交異名之差,若較之漸同,則隨交所在,遷變不常。



 仍以正交度距春分度數乘定差,如象限而一,所得,為月行與赤道定差,前加者為減,減者為加;皆加減黃道宿積度,為九道宿積度;以前宿九道積度減之,為其宿九道度及分。其分就近約為太、半、少。論春、夏、秋、冬,以四時日所在宿度為正。



 求正交加時月離九道宿度:以正交加時黃道日度及分減一百一度,餘以正交度及分乘之,半而退位為分,
 分滿百為度,命為月道與黃道泛差。其在同名者,置月行與黃道泛差,九因八約之,為定差,以加;仍以正交度距秋分度數乘定差,如象限而一,所得,為月道與赤道定差,以減。其在異名者,置月行與黃道泛差,七因八約之,為定差,以減;仍以正交度距春分度數乘定差,如象限而一,所得,為月道與赤道定差,以加。置正交加時黃道月度及分,以二差加減之,即正交加時月離九道宿度及分。



 求定朔弦望加時月所在度:置定朔加時日躔黃道宿次,凡合朔加時,月行潛在日下,與太陽同度,是為加時月離宿次;各以弦、望度及分秒加其所當弦、望加時日躔黃道宿度,滿宿次去之,命如前,各得定朔、弦、望加時月所在黃道宿度及分秒。



 求定朔弦望加時九道月度:各以定朔、弦、望加時月離黃道宿度及分秒,加前宿正交後黃道積度,為定朔、弦、望加時正交後黃道積度。如前求九道積度,以前宿九
 道積度減之,餘為定朔、弦、望加時九道月離宿度及分秒。其合朔加時若非正交,則日在黃道、月在九道。所入宿度雖多少不同,考其兩極,若應繩準,故云月行潛在日下,與太陽同度。



 求定朔午中入轉:以經朔小餘與半法相減,餘以加減經朔加時入轉,經朔小餘少,如半法加之;多,如半法減之。



 為經朔午中入轉。若定朔大餘有進退,亦加減轉日,否則因經為定,命日,算外,即得所求。次月仿此求之。



 求每日午中入轉:因定朔午中入轉日及餘秒,每日累
 加一日,滿轉周日及餘秒去之,命如前,即得每日午中入轉日及餘秒。



 求晨昏月度:置其日晨分,乘其日算外轉定分,日法而一,為晨轉分;用減轉定分,餘為昏轉分;又以朔、弦、望定小餘乘轉定分,日法而一,為加時分;以減晨昏轉分,為前;不足,覆減之,餘為後;乃前加後減加時月度,即晨、昏月所在宿度及分秒。



 求朔弦望晨昏定程:各以其朔昏定月減上弦昏定月,
 餘為朔後昏定程;以上弦昏定月減望昏定月,餘為上弦後昏定程;以望晨定月減下弦晨定月,餘為望後晨定程;以下弦晨定月減後朔晨定月,餘為下弦後晨定程。



 求每日轉定度:累計每程相距日轉定分,與晨昏定程相減,餘以相距日數除之,為日差;定程多為加,定程少為減。



 以加減每日轉定分,為每日轉定度及分秒。



 求每日晨昏月:因朔、弦、望晨昏月,加每日轉定度及分
 秒,滿宿次去之,為每日晨昏月。凡注歷,目朔日注昏月,望後次日注晨月。



 已前月度以究算術之精微,如求其速要,即依後術徑求。



 求經朔加時平行月:各以其月經朔入氣日及餘秒,其餘以日法退除為分秒。加其氣中積日及約分,命日為度,即為經朔加時平行月積度及分秒。



 求所求日加時平行月:置所求日大餘及加時小餘,以其月經朔大、小餘減之,餘為入經朔加時後日數及餘;以其日乘月平行度及分秒,列於上位,又以其餘乘月
 平行度及分秒,滿日法除之為度,不滿,退除為分秒,並上位,用加經朔加時平行月,滿周天度及分秒去之,即得所求日加時平行月積度及分秒。



 求所求日加時入轉:以所求日加時入經朔加時後日數及餘,加經朔加時入轉日及餘秒,滿轉周日及餘秒去之,命日,算外,即得所求。其餘先以日法退除為分秒。



 求所求日加時定月:置所求日加時入轉分,以其日算外加減差乘之,百約為分,分滿百為度,加減其下遲疾
 度,為遲疾定度;乃以遲減疾加所求日加時平行月,為定月。各以天正冬至加時黃道日度加而命之,即得所求日加時月離黃道宿度及分秒。其入轉若在四、七日者,如求朏朒術入之。



\end{pinyinscope}