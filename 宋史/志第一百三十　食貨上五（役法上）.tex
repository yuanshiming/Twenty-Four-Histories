\article{志第一百三十 食貨上五(役法上)}

\begin{pinyinscope}

 役法役出於民,州縣皆有常數。宋因前代之制,以衙前主官物,以里正、戶長、鄉書手課督賦稅,以耆長、弓手、壯丁逐捕盜賊,以承符、人力、手力、散從官給使令;縣曹司
 至押、錄,州曹司至孔目官,下至雜職、虞候、揀、掏等人,各以鄉戶等第定差。京百司補吏,須不礙役乃聽。



 建隆中,詔文武官、內諸司、臺省、寺監、諸軍、諸使,不得占州縣課役戶,州縣不得役道路居民為遞夫。後又詔諸州職官不得私占役戶供課。京西轉運使程能請定諸州戶為九等,著於籍,上四等量輕重給役,餘五等免之,後有貧富,隨時升降。詔加裁定。淳化五年,始令諸縣以第一等戶為里正,第二等戶為戶長,勿冒名以給役。自餘眾役,
 多調廂軍。大中祥符五年,提點刑獄府界段惟幾發中牟縣夫二百修馬監倉。群牧制置使代以廄卒,因下詔禁之。惟詔令有大興作而後調丁夫。然役有輕重勞佚之不齊,人有貧富強弱之不一,承平既久,奸偽滋生。命官、形勢占田無限,皆得復役,衙前將吏得免里正、戶長;而應役之戶,困於繁數,偽為券售田於形勢之家,假佃戶之名,以避徭役。乾興初,始立限田法,形勢敢挾他戶田者聽人告,子所挾田三之一。



 時州縣既廣,徭役益
 眾,太常博士範諷知廣濟軍,因言:「軍地方四十里,戶口不及一縣,而徭差與諸郡等,願復為縣。」轉運司執不可,因詔裁捐役人。自是數下詔書,督州縣長吏與轉運使議蠲冗役,以寬民力。又令州縣錄丁產及所產役使,前期揭示,不實者民得自言。役之重者,自里正、鄉戶為衙前,主典府庫或輦運官物,往往破產。景祐中,稍欲寬其法,乃命募人充役。初,官八品以下死者,子孫役同編戶;至是,詔特蠲之。民避役者,或竄名浮圖籍,號為出家,趙州
 至千餘人,詔出家者須落發為僧,乃聽免役。禁諸縣非捕盜毋擅役壯丁。慶歷中,令京東西、河北、陜西、河東裁捐役人,即給使不足,益以廂兵。既而詔諸路轉運司條析州縣差徭賦斂之數,委二府大臣裁減,科役不均,以鄉村、坊郭戶均差。時範仲淹執政,謂天下縣多,故役蕃而民瘠,首廢河南諸縣,欲以次及他州。當時以為非,未幾悉復。王逵為荊湖轉運使,率民輸錢免役,得緡錢三十萬,進為羨餘,蒙詔獎。繇是他路競為掊克以市恩。皇
 祐中,詔州縣里正、押司、錄事既代而令輸錢免役者,論如違制律。又禁役鄉戶為長名衙前。



 初,知並州韓琦上疏曰:「州縣生民之苦,無重於里正衙前。有孀母改嫁,親族分居;或棄田與人,以免上等;或非命求死,以就單丁。規圖百端,茍免溝壑之患。每鄉被差疏密,與貲力高下不均。假有一縣甲乙二鄉,甲鄉第一等戶十五戶,計貲為錢三百萬,乙鄉第一等戶五戶,計貲為錢五十萬;番休遞役,即甲鄉十五年一周,乙鄉五年一周。富者休息
 有餘,貧者敗亡相繼,豈朝廷為民父母意乎?請罷里正衙前,命轉運司以州軍見役人數為額,令、佐視五等簿,通一縣計之,籍皆在第一等,選貲最高者一戶為鄉戶衙前,後差人放此。即甲縣戶少而役蕃,聽差乙縣戶多石役簡者。簿書未盡實,聽換取他戶。里正主督租賦,請以戶長代之,二年一易。」下其議京畿、河北、河東、陜西、京東西轉運司度利害,皆以為便。而知制誥韓絳、蔡襄極論江南、福建里正衙前之弊,絳請行鄉戶五則之法,襄
 請以產錢多少定役重輕。至和中,命絳、襄與三司置司參定,繼遣尚書都官員外郎吳幾復趨江東,殿中丞蔡稟趨江西,與長吏、轉運使議可否。因請行五則法,凡差鄉戶衙前,視貲產多寡置籍,分為五則,又第其役輕重放此。假有第一等重役十,當役十人,列第一等戶百;第二等重役五,當役五人,列第二等戶五十,以備十番役使。藏其籍通判治所,遇差人,長吏以下同按視之,轉運使、提點刑獄察其違慢。遂更著淮南、江南、兩浙、荊湖、福
 建之法,下三司頒焉。



 自罷里正衙前,民稍休息。又詔諸路轉運司、開封府界訪衙前之役有重為害者條奏之;能件悉便利、大去勞弊者議賞。置寬恤民力司,遣使四出。自是州縣力役多所裁損,凡二萬三千六百二十二人。



 治平四年,詔曰:「農,天下之大本也。間因水旱,頗致流離,殆州郡差役之法甚煩,其詔中外臣庶條陳利害以聞。」先是,三司使韓絳言:「聞京東民有父子二丁將為衙前役者,其父告其子曰『吾當求死,使汝曹免於凍餒』,遂
 自縊而死。又聞江南有嫁其祖母及與母析居以避役者,又有鬻田減其戶等者。田歸官戶不役之家,而役並於同等見存之戶。望博訪利害,集議裁定,使力役無偏重之寄。」役法更議始此。



 熙寧元年,知諫院吳充言:「今鄉役之中,衙前為重。民間規避重役,土地不敢多耕,而避戶等;骨肉不敢義聚,而憚人丁。故近年上戶浸少,中下戶浸多,役使頻仍,生資不給,則轉為工商,不得已而為盜賊。宜早定鄉役利害,以時施行。」後帝閱內藏庫奏,有
 衙前越千里輸金七錢,庫吏邀乞,逾年不得還者。帝重傷之,乃詔制置條例司講立役法。二年,遣劉彞、謝卿材、侯叔獻、程顥、盧秉、王汝翼、曾伉、王廣廉八人行諸路,相度農田水利、稅賦科率、徭役利害。



 條例司檢詳文字蘇轍言:「役人之不可不用鄉戶,猶官吏之不可不用士人也。今遂欲兩稅之外別立一科,謂之庸錢,以備官雇,不問戶之高低,例使出錢,上戶則便,下戶實難。」轍以議不合罷。



 條例司言:「使民出錢雇役,即先王致民財以祿庶
 人在官者之意,願以條目遣官分行天下,博盡眾議。」於是條諭諸路曰:「衙前既用重難分數,凡買撲酒稅坊場,舊以酬衙前者,從官自賣,以其錢同役錢隨分數給之。其廂鎮場務之類,舊酬獎衙前、不可令民買占者,即用舊定分數為投名衙前酬獎。如部水陸運及領倉驛、場務、公使庫之類,其舊煩擾且使陪備者,今當省使毋費。承符、散從官等舊若重役償欠者,今當改法除弊,庶使無困。凡有產業物力而舊無役者,今當出錢以助役。」久
 之,司農寺言:「今立役條,所寬優者,皆村鄉樸蠢不能自達之窮氓;所裁取者,乃仕宦兼並能致人言之豪右。若經制一定,則衙司縣吏無以施誅求巧舞之奸,故新法之行尤所不便。欲先自一兩州為始,候其成就,即令諸州軍仿視施行,若實便百姓,當特獎之。」詔可。



 於是提點府界公事趙子幾奏上府界所在條目,下之司農,詔判寺鄧綰、曾布更議之。綰、布言:「畿內鄉戶,計產業若家資之貧富,上下分為五等。歲以夏秋隨等輸錢,鄉戶自四
 等、坊郭自六等以下勿輸。兩縣有產業者,上等各隨縣,中等並一縣輸。析居者隨所析而定、降其等。若官戶、女戶、寺觀、未成丁,減半輸。皆用其錢募三等以上稅戶代役,隨役重輕制祿。開封縣戶二萬二千六百有奇,歲輸錢萬二千九百緡。以萬二百為祿,贏其二千七百,以備兇荒欠閣,他縣仿此。」然輸錢計等高下,而戶等著籍,昔緣巧避失實。乃詔責郡縣,坊郭三年,鄉村五年,農隙集眾,稽其物產,考其貧富,察其詐偽,為之升降;若故為高
 下者,以違制論。



 募法:三人相任,衙前仍供物產為抵;弓手試武藝,典吏試書計;以三年或二年乃更。為法既具,揭示一月,民無異辭,著為令。令下,募者執役,被差者得散去。開封一府罷衙前八百三十人,畿縣鄉役數千,遂頒其法於天下。



 天下土俗不同,役重輕不一,民貧富不等,從所便為法。凡當役人戶,以等第出錢,名免役錢。其坊郭等第戶及未成丁、單丁、女戶、寺觀、品官之家,舊無色役而出錢者,名助役錢。凡敷錢,先視州若縣應用雇
 直多少,隨戶等均取;雇直既已用足,又率其數增取二分,以備水旱欠閣,雖增毋得過二分,謂之免役寬剩錢。



 三年,命集賢校理呂惠卿同判司農寺,已而林旦、曾布相繼典主其事。四年,罷許州衙前幹公使庫,以軍校主之,月給食錢三千。後行於諸路,人皆便之。



 兩浙提點刑獄王庭光、提舉常平張靚率民助役錢至七十萬。薛向為帝言,帝問王安石,安石曰:「提舉官據數取之,朝廷以恩惠科減,於體為順。」御史中丞楊繪亦言:「靚等科配民
 輸錢,多者一戶至三百千,乞少裁損,以安民心。」五月,東明縣民數百詣開封府訴超升等第,不受,遂突入王安石私第,安石諭以相府不知;訴之御史臺,臺不受訴,諭令散去。楊繪又言:「司農寺不用舊則,自據戶數創立助役錢等第,下縣令著之籍,如酸棗縣升戶等皆失實。」帝乃命提點司究所從升降,仍嚴升降之法,畿民不願輸錢免役,縣按所當供役歲月,如期役之,與免輸錢。先是,帝既知東明事,及聞繪言,兩降手敕問王安石曰:「酸棗
 既有自下戶升入上戶,則四等有免輸役錢之名,而無其實。」安石力言嘗取諸縣新舊籍對覆升降,聞外間扇搖役法者,謂輸多必有贏餘,若群訴必可免,彼既聚眾僥幸,茍受其訴,與免輸錢,當仍役之。帝乃盡用其言。



 中書孫迪、張景溫體量不願出錢之民,欲困以重役,楊繪復論之。而監察御史劉摯謂:「昨者團結保甲,民方驚擾,又作法使人均出緡錢,非時升降戶等,期會急迫,人情惶駭。」因陳新法十害,其要曰:「上戶常少,中下戶常多,故
 舊法上戶之役類皆數而重,下戶之役率常簡而輕;今不問上下戶,概視物力以差出錢,故上戶以為幸,而下戶苦之。歲有豐兇,而役人有定數,助錢歲不可闕,則是賦稅有時減閣,而助錢更無蠲損也。役人必用鄉戶,為其有常產則自重,今既招雇,恐止得浮浪奸偽之人,則帑庾、場務、綱運不惟不能典幹,竊恐不勝其盜用而冒法者眾;至於弓手、耆、壯、承符、散從、手力、胥史之類,恐遇寇則有縱逸,因事輒為搔擾也。司農新法,衙前不差鄉
 戶,其舊嘗願為長名者,聽仍其舊,卻用官自召賣酒稅坊場並州縣坊郭人戶助役錢數,酬其重難,惟此一法,有若可行;然坊郭十等戶,緩急科率,郡縣賴之,難更使之均出助錢。乞詔有司,若坊場錢可足衙前雇直,則詳究條目,徐行而觀之。」帝因安石進呈役錢文字,謂之曰:「民供稅斂已重,坊郭及官戶等不須減,稅戶升等事更與少裁之。」安石曰:「朝廷制法,當斷以義,豈須規規恤淺近之人議論耶?」



 於是提點趙子幾怒知東明縣賈蕃不
 能禁遏縣民來訟,雜摭他事致蕃於理。又使子幾自鞫之。楊繪謂是希安石意指,而致縣令於罪也。即疏辨之曰:「子幾若劾蕃五月十日前事,臣固無言;若所劾後乎此日,是以威脅令佐使民不得赴訴,得為便乎?」又言:「助役之利一,而難行有五。請先言其利:假如民田有一家而百頃者,亦有戶才三頃者,其等乃俱在第一,以百頃而較三頃,則已三十倍矣,而受役月日,均齊無異;況如官戶,則除耆長外皆應無役,今例使均出雇錢,則百頃
 所輸必三十倍於三頃者,而又永無決射之訟,此其利也。然難行之說亦有五:民惟種田,而責其輸錢,錢非田之所出,一也。近邊州軍,就募者非土著,奸細難防,二也。逐處田稅,多少不同,三也。耆長雇人,則盜賊難止,四也。衙前雇人,則失陷官物,五也。乞先議防此五害,然後著為定制,仍先戒農寺無欲速就以祈恩賞,提舉司無得多取於民以自為功,如此則誰復妄議。」



 劉摯亦言:「趙子幾以他事捃摭賈蕃為過,且變更役法,意欲便民,民茍
 以為有利害也,安可禁其所欲言!今因畿民有訴,而刻薄之人,反怒縣官不能禁遏。臣恐四遠人情,必疑朝廷欲鉗天下之口,而職在主民者,必皆視蕃為戒,則天下休戚,陛下何由知之?子幾挾情之罪,伏請付吏部施行。」



 於是同判司農寺曾布摭繪、摯所言而條奏辨詰之,其略曰:



 畿內上等戶盡罷昔日衙前之役,故今所輸錢比舊受役時,其費十減四五;中等人戶舊充弓手、手力、承符、戶長之類,今使上等及坊郭、寺觀、單丁、官戶皆出錢
 以助之,故其費十減六七;下等人戶盡除前日冗役,而專充壯丁,且不輸一錢,故其費十減八九。大抵上戶所減之費少,下戶所減之費多。言者謂優上戶而虐下戶,得聚斂之謗,臣所未喻也。



 提舉司以諸縣等第不實,故首立品量升降之法,開封府、司農寺方奏議時,蓋不知已嘗增減舊數。然舊敕每三年一造簿書,等第嘗有升降,則今品量增減亦未為非;又況方曉諭民戶,茍有未便,皆與厘正,則凡所增減,實未嘗行。言者則以謂品量
 立等者,蓋欲多斂雇錢,升補上等以足配錢之數。至於祥符等縣,以上等人戶數多減充下等,乃獨掩而不言,此臣所未諭也。



 凡州縣之役,無不可募人之理。今投名衙前半天下,未嘗不典主倉庫、場務、綱運;而承符、手力之類,舊法皆許雇人,行之久矣;惟耆長、壯丁,以今所措置最為輕役,故但輪差鄉戶,不復募人。言者則以謂衙前雇人,則失陷官物;耆長雇人,則盜賊難止;又以謂近邊奸細之人應募,則焚燒倉庫,或守把城門,則恐潛通
 外境,此臣所未諭也。



 免役或輸見錢,或納觔斗,皆從民便,為法至此,亦已周矣。言者則謂直使輸錢,則絲帛粟麥必賤;若用他物準直為錢,則又退揀乞索,且為民害。如此則當如何而可?此臣所未諭也。



 昔之徭役皆百姓所為,雖兇荒饑饉,未嘗罷役;今役錢必欲稍有餘羨,乃所以為兇年蠲減之備,其餘又專以興田利、增吏祿。言者則以謂助錢非如稅賦有倚閣減放之期,臣不知昔之衙前、弓手、承符、手力之類,亦嘗倚閣減放否?此臣所
 未諭也。



 兩浙一路,戶一百四十餘萬,所輸緡錢七十萬爾;而畿內戶十六萬,率緡錢亦十六萬。是兩浙所輸才半畿內,然畿內用以募役,所餘亦自無幾。言者則以謂吏緣法意,廣收大計,如兩浙欲以羨錢徼幸,司農欲以出剩為功,此臣所未諭也。



 賈蕃為令,不受民訴,使趨京師喧嘩,其意必有謂也。誠令用心無他,亦可謂不職矣。蕃之不職不法,其狀甚眾,皆趙子幾所不得不問;御史之言,欲舍蕃而治子幾,是不顧陛下之法、陛下之民,宜
 莫如蕃與御史也。



 於是下其疏於繪、摯,使各言狀。



 繪錄前後四奏以自辨。摯言:「助役斂錢之法,有大臣及御史主之於內,有大臣親黨為監司、提舉官而行之於諸路,其勢順易矣;然曠日彌年,終未有定論,為不順乎民心而已。陛下以司農為是耶,則事盡前奏,可以覆視;以臣言為非耶,則貶黜而已。雖復使臣言之,亦不過所謂十害者,而風憲之官,豈當與有司較是非勝負耶?」詔繪知鄭州;摯落館閣校勘、監察御史裏行,監衡州鹽倉。



 遣察
 訪使遍行諸路,促成役書,改助役為免役,不願就募而強之者論如律。初,詔監司各定所部助役錢數,利路轉運使李瑜欲定四十萬,判官鮮于侁曰:「利路民貧,二十萬足矣。」議不合,遂各為奏。帝是侁議。侍御史鄧綰言利路役歲須緡錢九萬餘,而李瑜率取至三十三萬有奇,提點刑獄周約亦占名無異辭。詔責瑜、約,而擢侁為副使。



 諸路役書既上之司農,乃頒募役法於天下,用免役錢祿內外胥吏,有祿而贓者,用倉法重其坐。初,京師賦
 吏祿,歲僅四千緡。至八年,計緡錢三十八萬有奇,京師吏舊有祿及外路吏祿又不在是焉。時知長葛縣樂京稱助役之法不可久行,常平司詢其故,不答,遂罷。京西使者召知湖陽縣劉蒙會議,蒙不肯議,退而條上利害,即投劾去。而權江西提刑提舉金君卿首募受代官部錢帛綱趨京,不差鄉戶衙前,而費減十五六。賜詔獎諭,仍落權為真。



 免役剩錢,詔州縣用常平法給散休息,添給吏人餐錢,仍立為法。京東免役錢以秋料起催,若雇
 直多少、役使重輕有未究者,命監司詳具來上,仍須熙寧七年乃行。永興、秦鳳比之他路,民貧役重,詔提舉司並省冗役,次第蠲減,當留二分寬剩,以為水旱閣放之備。



 七年,詔:「役錢千別納頭子五錢,凡修官舍,作什器,夫力輦運之類,皆許取以供費;不給,以情輕贖銅錢足之。諸路公人如弓箭手法,給田募人為之。凡逃、絕、監牧之田籍於轉運司者,不許射買請佃。提刑司以其田給應募者,而核其所直,準一年雇役為錢幾何,而歸其直於
 轉運司。」衢州西安縣用緡錢十二萬買田,始足募一縣之役。司農寺言,不獨兩浙如此,他路宜亦如之。費多難贍,乃欲改法。遂詔自今用寬剩錢買募役田,須先參會餘錢可以枝梧災傷,方許給買。若田價翔貴之地,則已之。



 時免役出錢或未均,參知政事呂惠卿及其弟曲陽縣尉和卿皆請行手實法。其法:官為定立田產中價,使民各以田畝多少高下,隨價自占;仍並屋宅分有無蕃息立等,凡居錢五當蕃息之錢一。非用買田谷而輒隱
 落者許告,有實,以三分之一充賞。將造簿,預具式示民,令依式為狀,縣受而籍之。以其價列定高下,分為五等。既該見一縣之民物產錢數,乃參會通縣役錢本額而定所當輸,明書其數,示眾兩月,使悉知之。詔從其請。



 司農寺乞廢戶長、坊正,令州縣坊郭擇相鄰戶三二十家,排比成甲,迭為甲頭,督輸稅賦苗役,一稅一替。其後,諸路皆言甲頭催稅未便,遂詔耆戶長、壯丁仍舊募充,其保正、甲頭、承帖法並罷。



 王安石言給田募役,有害十餘。
 八年,罷給田募役法,已就募人如舊,闕者弗補。官戶輸役錢免其半,所免雖多,各無過二十千。兩縣以上有物產者通計之,兩州兩縣以上有物產者隨所輸錢,等第不及者從一多處並之。



 初,手實法行,言者多論其長告訐,增煩擾。至是,惠卿罷政,御史中丞鄧綰言其法不便,罷之,委司農寺再詳定以聞。



 九年,以荊湖兩路敷役錢太重,較一歲入出,寬剩錢數多,詔權減二年。尋詔自今寬剩役錢及買撲坊場錢,更不以給役人,歲具羨數上
 之司農,餘物凡籍之常平司者,常留一半。侍御史周尹言:「募役錢數外留寬剩一分,聞州縣希提舉司風旨,廣敷民錢,省役額,損雇直,而民間輸數一切如舊,寬剩數多。募直輕而倉法重,役人多不願就募。天下皆謂朝廷設法聚斂,不無疑怨。乞募耆長、戶長及役人不可過減者悉復舊額,約募錢足用,其寬剩止留二分。」



 是歲,諸路上司農寺歲收免役錢一千四十一萬四千五百五十三貫、石、匹、兩:金銀錢斛匹帛一千四十一萬四千三百
 五十二貫、石、匹、兩,絲綿二百一兩;支金銀錢斛六百四十八萬七千六百八十八兩、貫、石、匹;應在銀錢斛匹帛二百六十九萬三千二十貫、匹、石、兩,見在八十七萬九千二百六十七貫、石、匹、兩。



 十年,知彭州呂陶奏:「朝廷欲寬力役,立法召募,初無過斂民財之意,有司奉行過當,增添科出,謂之寬剩。自熙寧六年施行役法,至今四年,臣本州四縣,已有寬剩錢四萬八千七百餘貫,今歲又須科納一萬餘貫。以成都一路計之,無慮五六十萬,推
 之天下,見今約有六七百萬貫寬剩在官。歲歲如此,泉幣絕乏,貨法不通,商旅農夫,最受其弊。臣恐朝廷不知免役錢外有此寬剩數目,乞契勘見今約支幾歲不至闕乏,霈發德音,特免數年;或逐年限定,不得過十分之一。所貴民不重困。」不報。



 王安石去位,吳充為相,沉括獻議莫若稍變役法,雜以差徭為便。御史知雜蔡確言括反復,貶括知宣州。



 役錢立額,浙東多以田稅錢數為則,浙西多用物力。至是,詔令通物力、稅錢互紐為數,從便
 輸納。淮東路估定物產,如其實直,以均敷取。初,許兩浙坊郭戶家產不及二百千,鄉村戶不及五十千,毋輸役錢,已而鄉戶不及五十千亦不免輸。元豐二年,提舉司言坊郭戶免輸法太優,乃詔如鄉戶法裁定所敷錢數。提舉廣西常平劉誼言:「廣西一路戶口二十萬,而民出役錢至十九萬緡,先用稅錢敷出;稅數不足,又敷之田米;田米不足,復算於身丁。夫廣西之民,身之有丁,既稅以錢,又算以米,是一身而輸二稅,殆前世弊法。今既未
 能蠲除,而又益以役錢,甚可憫也。至於廣東西監司、提舉司吏一月之給,上同令錄,下倍攝官,乞裁損其數,則兩路身丁田米亦可少寬。」遂詔吏輩月給錢遞減二千,歲遂減役錢一千二百餘緡。三年,司農寺丞吳雍言:「議定淮、浙役書,減冗占千三百餘人,裁省緡錢近二十九萬,會定歲用,寬剩錢一百四萬餘緡,諸路役書多若此類。乞先自近京三兩路修定,下之諸路。」從之。



 七年,天下免役緡錢歲計一千八百七十二萬九千三百,場務錢
 五百五萬九千,穀帛石匹九十七萬六千六百五十七,役錢較熙寧所入多三之一。



 帝之力主免役也,知民間通苦差役,而衙役之任重行遠者尤甚,特創免役。雖均敷雇直,不能不取之民;然民得一意田畝,實解前日困弊。故群議雜起,意不為變。顧其間採王安石策,不正用雇直為額,而展敷二分以備吏祿、水旱之用。群臣每以為言,屢疑屢詰,而安石持之益堅。此其為法既不究終防弊,而聚斂小人又乘此增取,帝雖數詔禁戒,而不能
 盡止。至是,雇役不加多,而歲入比前增廣,則安石不能將順德意,其流弊已見矣。



 哲宗立,宣仁後垂簾同聽政,門下侍郎司馬光言:



 「按因差役破產者,惟鄉戶衙前。蓋山野愚戇之人,不能幹事,或因水火損敗官物,或為上下侵欺乞取,是致欠拆,備償不足,有破產者。至於長名衙前,在公精熟,每經重難,別得優輕場務酬獎,往往致富,何破產之有?又曰向者役人皆上等戶為之,其下等、單丁、女戶及品官、僧道,本來無役,今使之一概輸錢,則
 是賦斂愈重。自行免役法以來,富室差得自寬,貧者困窮日甚,監司、守令之不仁者,於雇役人之外多取羨餘,或一縣至數萬貫,以冀恩賞。又青苗、免役,賦斂多責見錢。錢非私家所鑄,要須貿易,豐歲追限,尚失半價,若值兇年,無穀可糶,賣田不售,遂致殺牛賣肉,伐桑鬻薪,來年生計,不暇復顧,此農民所以重困也。



 臣愚以為宜悉罷免役錢,諸色役人,並如舊制定差,見雇役人皆罷遣之。衙前先募人投充長名,召募不足,然後差鄉村人
 戶,每經歷重難差遣,依舊以優輕場務充酬獎。所有見在役錢,撥充州縣常平本錢,以戶口為率,存三年之蓄,有餘則歸轉運司。凡免役之法,縱富強應役之人,徵貧弱不役之戶,利於富不利於貧。及今耳目相接,猶可復舊名,若更年深,富者安之,民不可復差役矣。」



 於是始詔修定役書,凡役錢,惟元定額及額外寬剩二分已下許著為準,餘並除之。若寬剩元不及二分者,自如舊則。尋詔耆戶長、壯丁皆仍舊募人供役,保正、甲頭、承帖人並
 罷。



 元祐元年,侍御史劉摯言:「率戶賦錢,有從來不預差役而概被斂取者,有一戶而輸數百以至千緡者。昔惟衙前一役,有至破產者爾。今天下坊場,官收而官賣之,歲計緡錢無慮數百萬,自可足衙前雇募支酬之直,則役之重者已無所事於農民矣。外惟散從、承符、弓手、手力、耆戶長、壯丁之類,無大勞費,宜並用祖宗差法,自第一等而下通任之。」監察御史王巖叟請於衙前大役立本等相助法,以盡變通之利。借如一邑之中當應大役
 者百家,而歲取十人,則九十家出力為助,明年易十戶,復如之,則大役無偏重之弊;其於百色無名之差占,一切非理之資陪,悉用熙寧新法禁之,雖不助猶可為也。



 殿中侍御史劉次莊言:「近制許雇耆戶長須三等已上戶。不知三等已上戶不願受雇,既無願者,則郡縣必陽循雇名,陰用差法,不若立法明差之為便。」戶部言:「詔凡耆戶長、壯丁並募人供役,竊慮戶長雇錢數少,無應募者。兼四等以下戶舊不敷役錢,惟輸差壯丁,今悉雇募,
 用錢額廣,提舉司必從人戶增敷。蓋舊法役不盡雇,亦有輪差輪募之處,欲且如本法。」



 中書舍人蘇軾言:「先帝初行役法,取寬剩錢不得過二分,以備災傷。有司奉行過當,行之幾十六七年,積而不用,至三千餘萬貫石。熙寧中,行給田募役法,大略如邊郡弓箭手。臣知密州,先募弓手,民甚便之,曾未半年,此法復罷。」因列其五利。王巖叟言:「蘇軾乞買田募役,其五利難信,而有十弊。」大指謂:「官市民田,慮不當價;民受田就募,既非永業,則鹵莽
 其耕,又將轉而他之。」而其六弊特詳,曰:「弓箭手雖名應募,實與家居農民無異,雖或番上及緩急不免點集,實不廢田業,非如州縣色役長在官寺,則弓箭手之擾可知矣。然猶聞闕額常難補招,已就招者又時時竄去,引以為比,不切事情。」其七弊曰:「戶及三等以上,皆能自足,必不肯佃田供役。今立法須二等以上方得供弓手,三等以上方得供散從官以下色役,乃是用給田募役之名,行揭簿定差之實。既云百姓樂於應募,何以戶降四
 等必須上二等戶保任?任之而逃,則勒保者就供田役,此豈得雲樂應也耶?」上官均亦陳五不可行,軾議遂格。



 司馬光復奏:



 「今免役之法,其害有五:上戶舊充役,固有陪備,而得番休,今出錢比舊費特多,年年無休息。下戶元不充役,今例使出錢。舊日所差皆土著良民,今皆浮浪之人應募,無顧藉,受賕,侵陷官物。又農民出錢難於出力,若遇兇年,則賣莊田、牛具、桑柘,以錢納官。提舉常平倉司惟務多斂役錢,廣積寬剩。此五害也。



 今莫若直
 降敕命,盡罷天下免役錢,其諸色役人,並依熙寧元年以前舊法人數,委本縣令佐揭簿定差。其人不願身自供役,許擇可任者雇代,有逋逃失陷,雇者任之。惟衙前一役,最號重難,固有因而破產者,為此始作助役法。自後色色優假,禁止陪備,別募命官將校部押遠綱,遂不聞更有破產之人;若今衙前仍行差法,陪備既少,當不至破家。若猶矜其力難獨任,即乞如舊法,於官戶、寺觀、單丁、女戶有屋產月收僦直可及十五千、莊田中熟所
 收及百石以上者,並隨貧富以差出助役錢,自餘物產,約此為準。每州樁收,候有重難役使,即以支給。



 尚慮役人利害,四方不能齊同。乞許監司、守令審其可否,可則亟行,如未究盡,縣許五日具措畫上之州,州一月上轉運司,轉運司季以聞。朝廷委執政審定,隨一路一州各為之敕,務要曲盡。然免役行之近二十年,富戶習於優利,一旦變更,不能不懷異同。又差役復行,州縣不能不有小擾,提舉官專以多斂役錢為功,必競言免役錢不
 可罷。當此之際,願弗以人言輕壞良法。」



 知樞密院章惇取光所奏疏略未盡者駁奏之。尚書左丞呂公著言惇專欲求勝,不顧命令大體,望選差近臣詳定。右正言王覿奏:「光議初上,惇嘗同奏,待既施行,方列光短,其實小人,不當置腹心地。」於是詔以資政殿大學士韓維、給事中范純仁等專切詳定以聞。



 王覿又言:「近制改募為差,用舊法人數為則,而熙寧元年以後,募數屢經裁減,則舊數不可復用,請悉準見額定差。」先是,差法既復,知開
 封府蔡京如敕五日內盡用開封、祥符兩縣舊役人數,差一千餘人以足舊額。右司諫蘇轍言:「開封府亟用舊額盡差,如壇子之類,近例率用剩員,今悉改差民戶,故為煩擾以搖成法,乞正其罪。」



 司馬光之始議差役,中書舍人範百祿言於光曰:「熙寧免役法行,百祿為咸平縣,開封罷遣衙前數百人,民皆欣幸。其後有司求羨餘,務刻剝,乃以法為病。今第減助免錢額以寬民力可也。」光雖不從,及議州縣吏因差役受賕從重法加等配流,百
 祿押刑房,固執不可曰:「鄉民因徭為吏,今日執事而受賕,明日罷役,復以財遺人,若盡以重法繩之,將見黥面赭衣充塞道路矣。」光曰:「微公言,幾為民害。」遂已之。



 蘇轍又言:



 「差役復行,應議者有五:其一曰舊差鄉戶為衙前,破敗人家,甚如兵火。自新法行,天下不復知有衙前之患;然而天下反以為苦者,農家歲出役錢為難,及許人添鏟見賣坊場,遂有輸納京給者爾。向使止用官賣坊場課入以雇衙前,自可足辦,而他色役人止如舊法,則
 為利較然矣。初疑衙前多是浮浪投雇,不如鄉差稅戶可托。然行之十餘年,投雇者亦無大敗闕,不足以易鄉差衙前之害。今略計天下坊場錢,一歲可得四百二十餘萬貫,若立定中價,不許添鏟,三分減一,尚有二百八十餘萬貫。而衙前支費及召募非泛綱運,一歲共不過一百五十餘萬緡,則是坊場之直,自可了辨衙前百費,何用更差鄉戶?今制盡復差役,知衙前若無陪備,故以鄉戶為之;至於坊場,元無明降處分,不知官自出賣耶,
 抑仍用以酬獎衙前也?若仍用以酬獎,即召募部綱以何錢應用?若不與之錢,即舊名重難,鄉戶衙前仍前自備,為害不小。



 其二,坊郭人戶舊苦科配,新法令與鄉戶並出役錢,而免科配,其法甚便。但敷錢大重,未為經久之法。乞取坊郭、官戶、寺觀、單丁、女戶,酌今役錢減定中數,與坊場錢用以支雇衙前及召募非泛綱運外,卻令樁備募雇諸色役人之用。



 其三,乞用見今在役人數定差,熙寧未減定前,其數實冗,不可遵用。



 其四,熙寧以前,
 散從、弓手、手力諸役人常苦迎送,自新法以來,官吏皆請雇錢,役人既便,官亦不至闕事,乞仍用雇法。



 其五,州縣胥吏並量支雇錢募充,仍罷重法,亦許以坊場、坊郭錢為用;不足用,方差鄉戶,鄉戶所出雇錢,不得過官雇本數。」



 詔送看詳役法所詳定,擇其要者先奏以行。



 於是役人悉用見數為額,惟衙前用坊場、河渡錢雇募,不足,方許揭簿定差。其餘役人,惟該募者得募,餘悉定差。遂罷官戶、寺觀、單丁、女戶出助役法,其今夏役錢即免輸。
 尋以衙前不皆有雇直,遂改雇募為招募。凡熙、豐嘗立法禁以衙前及役人非理役使及令陪備圓融之類,悉申行之,耆壯依保正長法。坊場河渡錢、量添酒錢之類,名色不一,惟於法許用者支用外,並樁備招募衙前、支酬重難及應緣役事之用。如一州錢不供用,許移別州錢用之,一路不足,許從戶部通他路移用;其或有餘,毋得妄用,其或不足,毋得減募增置。衙前最為重役,若已招募足額,上一等戶有虛閑不差者,令供次等色役。鄉
 差役人,在職官如敢抑令別雇承符、散從承代其役者,轉運司劾奏重責。時提舉常平司已罷置,凡役事改隸提刑司。



 殿中侍御史呂陶言:「天下版籍不齊,或以稅錢貫百,或以田地頃畝,或以家之積財,或以田之受種。雖皆別為五等,然有稅賦錢一貫、占田一頃、積財千緡、受種十石而入之一等。一等之上,無等可加,遂至稅緡、田頃、積財、受種十倍於此,亦不過同在一等。憑此差役,必不均平。雖無今日納錢之勞,反有昔時偏頗陪費之害。
 莫若裁量新舊,著為條約:如稅錢一貫為第一等,合於本等中差一役,稅錢兩倍於一役者並差二役,又倍即差三役;雖稅錢更多,不過三役,並聽雇人。或本縣戶多役少,則上戶之役不須並差,但可次敘休役年月遠近而均其勞逸。假令甲充役後可閑五年,乙稅錢兩倍於甲,可閑三年,丙又倍於乙,可閑一年。以其田土頃畝之類為等並其餘同等多少不侔者,並仿此。又成、梓兩路差役,舊專以戶稅為差等,熙寧初,別定坊郭戶營運錢
 以助免役。乃在稅產之外,州縣抑認成額,至今不減,至有停閑居業移避鄉村,猶不得免。今方議法,坊郭等第固不可偏廢,然須參究虛實,別行排定,以寬民力。」並送詳定所。



 蘇轍又言:「雇募衙前改為招募,既非明以錢雇,必無肯就招者,勢須差撥,不知歲收坊場、河渡緡錢四百二十餘萬,欲於何地用之?熙寧以前,諸路衙前多雇長名當役,如西川全是長名,淮南、兩浙長名太半以上,餘路亦不減半。今坊場官既自賣,必無願充長名,則衙
 前並是鄉戶。雖號招募,而上戶利於免役,方肯占名,與差無異。上戶既免衙前重役,則凡役皆當均及以次人戶,如此則下戶充役,多如熙寧前矣。」



\end{pinyinscope}