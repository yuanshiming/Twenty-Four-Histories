\article{列傳第一百五十六}

\begin{pinyinscope}

 ○徐
 誼吳獵項安世薛叔似劉甲楊輔劉光祖



 徐誼,字子宜,一字宏父,溫州人。乾道八年進士,累官太常丞。孝宗臨禦久,事皆上決,執政惟奉旨而行,群下多
 恐懼顧望。誼諫曰:「若是則人主日聖,人臣日愚,陛下誰與共功名乎?」及論樂制,誼對以「宮亂則荒,其君驕;商亂則陂,其官壞。」上遽改容曰:「卿可謂不以官自惰矣。」



 知徽州,陛辭,屬光宗初受禪,誼奏:「三代聖王,有至誠而無權術,至誠不息,則可以達天德矣。」至郡,歙縣有妻殺夫系獄,以五歲女為證,誼疑曰:「婦人能一掌致人死乎?」緩之未覆也。會郡究實稅於庭,死者父母及弟在焉,乃言:「我子欠租久系,饑而大叫,役者批之,墮水死矣。」然後冤者
 得釋,吏皆坐罪,闔郡以為神。移提舉浙西常平,守右司郎中,遷左司。



 孝宗疾浸棘,上久稽定省,誼入諫,退告宰相曰:「上慰納從容,然目瞪不瞬,意思恍惚,真疾也。宜禱祠郊廟,進皇子嘉王參決。」丞相留正不克用。



 孝宗崩,上不能喪,祭奠有祝,有司不敢攝,百官皆未成服。誼與少保吳琚議請太皇太后臨朝,扶嘉王代祭。及將禫,正憂懼,僕於殿庭而去。誼以書譙趙汝愚曰:「自古人臣為忠則忠,為奸則奸,忠奸雜而能濟者,未之有也。公內雖心
 惕,外欲坐觀,非雜之謂歟?國家安危,在此一舉。」汝愚問策安出,誼曰:「此大事,非憲聖太后命不可。而知閣門事韓侂胄,憲聖之戚也,同里蔡必勝與侂胄同在閣門,可因必勝招之。」侂胄至,汝愚以內禪議遣侂胄請於憲聖,侂胄因內侍張宗尹、關禮達汝愚意,憲聖許之。



 寧宗即位,誼遷檢正中書門下諸房公事兼權刑部侍郎,進權工部侍郎、知臨安府。侂胄恃功,以賞薄浸觖望。誼告汝愚曰:「異時必為國患,宜飽其欲而遠之。」不聽。



 汝愚雅器
 誼,除授建明多咨訪,誼隨事裨助,不避形跡,怨者始眾。嘗勸汝愚早退,汝愚亦自請:「名在屬籍,不宜久司揆事,願因阜陵訖事以去。」寧宗已許之。侂胄出入禁中無度,誼密啟汝愚,無計防之,乃直面諷侂胄。侂胄疑將排己,首謁誼,退束裝,冀誼還謁,留之通殷勤,誼不往。



 吏部侍郎彭龜年論侂胄罪狀,侂胄疑汝愚、誼知其情,益怨恨。以御史劉德秀、胡紘疏誼,責惠州團練副使、南安軍安置,移袁州,又移婺州。久之,許自便。復官,提舉崇道觀,起
 守江州,加集英殿修撰,升寶謨閣待制,移知建康府,兼江、淮制置使。初,金攻廬、楚不下,留兵綴濠州以待和,時時鈔掠,與宋師遇,殺傷相當,淮人大驚,復迸流江南,在建康者以數十萬計。誼晝夜拊循,益嚴備御,請專捍敵,勿從中御。朝廷懼生事,移知隆興府以卒。



 誼嘗與紹興老將接,於行陣之法,分數奇正,皆有指授,自為圖式。後謚忠文。



 吳獵,字德夫,潭州醴陵人。登進士第,初主潯州平南簿。
 時張栻經略廣西,檄攝靜江府教授。劉焞代栻,栻以獵薦,闢本司準備差遣。



 盜李接起,陷容、雷、高、化、貴、鬱林等州,獵請賞勞誅罪,焞於是錄鬱林功,誅南流縣尉、鬱林巡檢,人人驚厲,爭死鬥,不逾時,盜悉就擒。尉,宰相王淮甥也,獵坐降官。久之,知常州無錫縣。用陳傅良薦,召試,守正字。



 光宗以疾久不覲重華宮,獵上疏曰:「今慈福有八十之大母,重華有垂白之二親,陛下宜於此時問安上壽,恪共子職。」辭甚切。又白宰相留正,乞召朱熹、楊萬
 里。時陳傅良以言過宮事不行求去,獵責之曰:「今安危之機,判然可見,未聞有牽裾折檻之士。公不於此時有所奮發,為士大夫倡,第潔身而去,於國奚益!」傅良為改容謝之。



 寧宗即位,遷校書郎,除監察御史。上趣修大內,將移御,獵言:「壽皇破漢、魏以來之薄俗,服高宗三年之喪,陛下萬一輕去喪次,將無以慰在天之靈。」又言:「陛下即位,未見上皇,宜篤厲精誠,以俟上皇和豫而祗見焉。」會偽學禁興,獵言:「陛下臨御未數月,今日出一紙去宰
 相,明日出一紙去諫臣,昨又聞侍講朱熹遽以御札畀祠,中外惶駭,謂事不出於中書,是謂亂政。」獵既駁史浩謚,又請以張浚配享阜陵曰:「艱難以來,首倡大義,不以成敗利鈍異其心,精忠茂烈,貫日月、動天地,未有過於張浚也。孝宗皇帝規恢之志,一飯不忘。歷考相臣,終始此念,足以上配孝宗在天之意,亦惟浚一人耳。」議皆不合。出為江西轉運判官,尋劾罷。



 久之,黨禁馳,起為廣西轉運判官,除戶部員外郎、總領湖廣江西京西財賦。韓
 侂胄議開邊,獵貽書當路,請號召義士以保邊場,刺子弟以補軍實,增棗陽、信陽之戍以備沖突,分屯陽羅五關以捍武昌,杜越境誘竊以謹邊隙,選試良家子以衛府庫。且謂:「金人懲紹興末年之敗,今其來必出荊、襄逾湖。」乃輸湖南米於襄陽,凡五十萬石;又以湖北漕司和糴米三十萬石分輸荊、郢、安、信四郡;蓄銀帛百萬計以備進討;拔董逵、孟宗政、柴發等分列要郡,厥後皆為名將。



 召除秘書少監,首陳邊事,乞增光、鄂、江、黃四郡戍。屬
 江陵告饑,除秘閣修撰、主管荊湖北路安撫司公事、知江陵府。陛辭,請出大農十萬緡以振饑者。道武昌,遣人招商分糴;至郡,減價發糶,米價為平。



 獵計金攻襄陽,則荊為重鎮,乃修成「高氏三海」,築金鸞、內湖、通濟、保安四匱,達於上海而注之中海;拱辰、長林、藥山、棗林四匱,達於下海;分高沙、東獎之流,由寸金堤外歷南紀、楚望諸門,東匯沙市為南海。又於赤湖城西南遏走馬湖、熨斗陂之水,西北置李公匱,水勢四合,可限戎馬。



 金人圍襄
 陽、德安,游騎迫竟陵,朝廷命獵節制本路兵馬。獵遣張榮將兵援竟陵,又招神馬陂潰卒得萬人,分援襄陽、德安。加寶謨閣待制、京湖宣撫使。



 時金人再犯竟陵,張榮死之,襄陽、德安俱急。吳曦俄反於蜀,警報至,獵請魏了翁攝參議官,訪以西事,募死士入竟陵,命其將王宗廉死守,調大軍及忠義、保捷分道夾擊,金人遂去。又督董逵等援德安,董世雄、孟宗政等解襄陽之圍。



 西事方殷,獵為討叛計,請於朝,以王大才、彭輅任西事,仍分兵抗
 均、房諸險,漕粟歸、峽以待王師。及曦誅,除刑部侍郎,充四川宣諭使。朝廷命旌別淑慝。以敷文閣學士、四川安撫制置使兼知成都府。嘉定六年召還,卒,家無餘資。蜀人思其政,畫像祠之。



 獵初從張栻學,乾道初,朱熹會栻於潭,獵又親炙,湖湘之學一出於正,獵實表率之。有《畏齋文集》、奏議六十卷。謚文定。



 項安世,字平父,其先括蒼人,後家江陵。淳熙二年進士,召試,除秘書正字。光宗以疾不過重華宮,安世上書言:「
 陛下仁足以覆天下,而不能施愛於庭闈之間;量足以容群臣,而不能忍於父子之際。以一身寄於六軍、萬姓之上,有父子然後有君臣。願陛下自入思慮,父子之情,終無可斷之理;愛敬之念,必有油然之時。聖心一回,何用擇日,早往則謂之省,暮往則謂之定。即日就駕,旋乾轉坤,在返掌間爾。」疏入不報。安世遺宰相留正書求去,尋遷校書郎。



 寧宗即位,詔求言,安世應詔言:



 管夷吾治齊,諸葛亮治蜀,立國之本,不過曰量地以制賦,量賦以
 制用而已。陛下試披輿地圖,今郡縣之數,比祖宗時孰為多少?比秦、漢、隋、唐時孰為多少?陛下必自知其狹且少矣。試命版曹具一歲賦入之數,祖宗盛時,東南之賦入幾何?建炎、紹興以來至乾道、淳熙,其所增取幾何?陛下試命內外群臣有司具一歲之用,人主供奉、好賜之費幾何?御前工役、器械之費幾何?嬪嬙、宦寺廩給之費幾何?戶部、四總領養兵之費幾何?州縣公使、迎送、請給之費幾何?陛下必自知其為侈且濫矣!用不量賦而至
 於侈且濫,內外上下之積不得而不空,天地山川之藏不得而不竭,非忍痛耐謗,一舉而更張之,未知其所以終也。



 今天下之費最重而當省者,兵也。能用土兵則兵可省,能用屯田則兵可省。其次莫如宮掖。兵以待敵國,常畏而不敢省,故省兵難。宮掖以私一身,常愛而不忍省,故省宮掖難。不敢省者,事在他人;不忍省者,在陛下。宮中之嬙嬪、宦寺,陛下事也,宮中之器械、工役,陛下事也,陛下肯省則省之。宮中既省,則外廷之官吏,四方之
 州縣,從風而省,奔走不暇,簡樸成風,民志堅定,民生日厚,雖有水旱蟲蝗之災,可活也;國力日壯,雖有夷狄盜賊之變,可為也。復祖宗之業,雪人神之憤,惟吾所為,無不可者。



 時朱熹召至闕,未幾予祠,安世率館職上書留之,言:「御筆除熹宮祠,不經宰執,不由給舍,徑使快行,直送熹家。竊揣聖意,必明知熹賢不當使去,宰相見之必執奏,給舍見之必繳駁,是以為此駭異變常之舉也。夫人主患不知賢爾,明知其賢而明去之,是示天下以不
 復用賢也。人主患不聞公議爾,明知公議之不可而明犯之,是示天下以不復顧公議也。且朱熹本一庶官,在二千里外,陛下即位未數日,即加號召,畀以從官,俾侍經幄,天下皆以為初政之美。供職甫四十日,即以內批逐之,舉朝驚愕,不知所措。臣願陛下謹守紀綱,毋忽公議,復留朱熹,使輔聖學,則人主無失,公議尚存。」不報。俄為言者劾去,通判重慶府,未拜,以偽黨罷。



 安世素善吳獵,二人坐學禁久廢。開禧用兵,獵起帥荊渚,安世方丁
 內艱。起復,知鄂州。俄淮、漢師潰,薛叔似以怯懦為侂胄所惡,安世因貽侂胄書,其末曰:「偶送客至江頭,飲竹光酒,半醉,書不成字。」侂胄大喜曰:「項平父乃爾閑暇。」遂除戶部員外郎、湖廣總領。



 會叔似罷,金圍德安益急,諸將無所屬。安世不俟朝命,徑遣兵解圍。高悅等與金人力戰,馬雄獲萬戶,周勝獲千戶,安世第其功以聞。獵代叔似為宣撫使,尋以宣諭使入蜀。朝命安世權宣撫使,又升太府卿。



 有宣撫幕官王度者,吳獵客也。獵與安世素
 相友,及安世招軍,名項家軍,多不逞,好虜掠,獵斬其為首者,安世憾之,至是斬度於大別寺。獵聞於朝,安世坐免。後以直龍圖閣為湖南轉運判官,未上,用臺章奪職而罷。嘉定元年,卒。所著《易玩辭》、他書,多行於世。



 薛叔似,字象先,其先河東人,後徙永嘉。游太學,解褐國子錄。初登對,論:「祖宗立國之初,除二稅外,取民甚輕。自熙寧以來,賦日增而民困滋甚。」孝宗嘉納,因曰:「朕在宮中如一僧。」叔似曰「此非所望於陛下,當論功業如何。正
 使海內富庶如文、景,不過江左之文、景;法度修明如明、章,不過江左之明、章。陛下即位二十餘年,國勢未張,未免牽於茍安無事之說。」上默然。



 復數日,宰執進擬朝士,上出寸紙書叔似及應孟明姓名,嘉其奏對也。遷太常博士,尋除樞密院編修官。時仿唐制,置補闕、拾遺,宰臣啟,擬令侍從、臺諫薦人,上自除叔似左補闕。叔似論事,遂劾首相王淮去位。



 屬金主殂,太孫景立,叔似奏:「規模果定,則乘五單于爭立之機;規模不存,則恐成五胡迭
 起之勢。」光宗受禪,時傳金使入界使名未正,叔似奏:「自壽皇一正匹敵之禮,金人常有南顧之虞,使名未正而遽受之,祗以重其玩侮。」翼日復奏:「謀國者畏敵太過。」上奮然開納。



 除將作監,出為江東轉運判官。俄以諫臣論罷,主管沖祐觀,尋除湖北運判,加直秘閣,移福建,召為太常少卿兼實錄院檢討官、守秘書監、權戶部侍郎。初,丞相周必大請擇侍從、臺諫忠直者提舉太史局,蓋用神宗朝司馬光與王安禮故事,躔度少差,豫圖銷弭,遂
 命叔似提舉。尋兼樞密都承旨,以劉德秀疏罷,提舉興國宮。起知贛州,移隆興府、廬州,召除在京宮觀兼侍讀,進權兵部侍郎兼同修國史兼國用司參議官。兩浙民有身丁錢,叔似請於朝,遂蠲之。



 試吏部侍郎兼侍讀,充京、湖宣諭使。時韓侂胄開邊,除兵部尚書、宣撫使。叔似方乞給降官會,分撥綱運,募兵鬻馬,闢致僚佐,而皇甫斌唐州之師已敗矣。遂劾斌,南安軍安置。叔似料敵必侵光、黃,委總領陳謙按行五關,發鄂卒守三關。金果入
 寇,謙駐漢陽為江左節制。



 尋除叔似端明殿學士兼侍讀。時宣司兵戍襄陽,都統趙淳、副統制魏友諒與統制呂渭孫不相下,渭孫死之,叔似遂自劾委任失當。叔似夙以功業自期,逮臨事,絕無可稱。以御史王益祥論,奪職罷祠。侂胄誅,諫官葉時再論,降兩官,謫福州,以兵端之開,叔似迎合故也。久之,許自便。嘉定十四年卒,贈銀青光祿大夫,謚恭翼。



 叔似雅慕朱熹,窮道德性命之旨,談天文、地理、鐘律、象數之學,有稿二十卷。



 劉甲,字師文,其先永靜軍東光人,元祐宰相摯之後也。父著,為成都漕幕,葬龍游,因家焉。甲,淳熙二年進士,累官至度支郎中,遷樞密院檢詳兼國史院編修官、實錄院檢討官。



 使金,至燕山,伴宴完顏者,名犯仁廟嫌諱,甲力辭,完顏更名修。自紹興後,凡出疆遇忌,俱辭設宴,皆不得免,秦檜所定也。九月三日,金宴甲,以宣仁聖烈後忌,辭。還除司農少卿,進太常,擢權工部侍郎,升同修撰,除寶謨閣待制,知江陵府,湖北安撫使。甲謂:「荊州
 為吳、蜀脊,高保融分江流,瀦之以為北海,太祖常令決去之,蓋保江陵之要害也。」即因遺址浚築,亙四十里。移知廬州。



 程松為四川宣撫使,吳曦副之,以甲知興元府、利東安撫使。時蜀口出師敗衄,金陷西和、成州,曦焚河池縣。先是,曦已遣姚淮源獻四州於金,金鑄印立曦為蜀王。甲時在漢嘉,未至鎮也。金人破大散關,興元都統制毋思以重兵守關,而曦陰徹驀關之戍,金自板岔穀繞出關後,思挺身免。



 甲告急於朝,乞下兩宣撫司協力
 捍禦。松謀遁,甲固留不可,遽以便宜檄甲兼沿邊制置。曦遣後軍統制王鉞、準備將趙觀以書致甲,甲援大義拒之,因臥疾。曦又遣其弟旼邀甲相見,甲叱而去之。乃援顏真卿河北故事,欲自拔歸朝,先募二兵持帛書遣參知政事李壁告變,且曰:「若遣吳總以右職入川,即日可瓦解矣。」



 曦僭王位,甲遂去官。朝廷久乃微聞曦反狀,韓侂胄猶不之信,甲奏至,舉朝震駭。壁袖帛書進,上覽之,稱「忠臣」者再。召甲赴行在,命吳總以雜學士知鄂州,
 多賜告身、金錢,使招諭諸軍為入蜀計。復命以帛書賜甲曰:「所乞致仕,實難允從,已降指揮,召赴行在。今朝廷已遣使與金通和,襄、漢近日大捷,北兵悉已渡江而去。恐蜀遠未知,更在審度事宜,從長區處。」二兵皆補官。



 甲舟行至重慶,聞安丙等誅曦,復還漢中,上奏待罪。詔趣還任。甲奏叛臣子孫族屬及附偽罪狀,公論快之。會宣撫副使安丙以楊巨源自負倡義之功,陰欲除之,語在《巨源傳》。臣源既死,軍情叵測,除甲宣撫使。楊輔亦以為
 請,當國者疑輔避事,李壁曰:「昔吳璘屬疾,孝宗嘗密詔汪應辰權宣撫司事,既而璘果死,應辰即日領印,軍情遂安,此的例也。」乃以密札命甲,甲鐍藏之。未幾,金自鶻嶺關札金崖,進屯八里山,甲分兵進守諸關,截潼川戍兵駐饒風以待之。金人知有備,引去。



 侂胄誅,上念甲精忠,拜寶謨閣學士,賜衣帶、鞍馬。是歲,和議成,朝廷聞彭輅與丙不協,以書問甲,又俾諭丙減汰諸軍勿過甚,及訪蜀人才之可用者。蓋自楊輔召歸,西邊諸事,朝論多
 於甲取決,人無知者。



 紹興中,蜀軍無見糧,創為科糴。孝宗聞其病民,命總領李蘩以本所錢招糴,懼不給,又命勸糴其半,「勸糴」之名自此始。久之,李昌圖總計,復奏令金、梁守倅任責收糴,而勸糴遂罷。及是,宣、總司令金洋、興元三郡勸糴小麥三十萬石,甲乞下總所照李蘩成法措置,從之。



 明年,罷宣撫司,合利東、西為一帥,治興元,移甲知潼川府。安丙既同知樞密院事,董居誼為制置使,甲進寶謨閣學士、知興元府、利路安撫使,節制本路
 屯駐軍馬。朝廷計居誼猶在道,命甲權四川制置司事。



 先是,大臣撫蜀者,諸將事之,有所謂互送禮,實賄賂也。甲下令首罷之,凡丙所立茶鹽柴邸悉廢之。又乞以皂郊博易鋪場還隸沔戎司,復通吳氏莊,歲收租四萬斛有奇,錢十三萬,以裨總計。從之。丙增多田稅,甲命屬吏討論,由一府言之,歲減凡百六十萬緡、米麥萬七千石,邊民感泣。嘉定七年,卒於官,年七十三。



 甲幼孤多難,母病,刲股以進。生平常謂:「吾無他長,惟足履實地。」晝所為,
 夜必書之,名曰「自監」。為文平澹,有奏議十卷。理宗詔謚清惠。



 楊輔,字嗣勛,遂寧人。乾道二年進士甲科,召試館職,除秘書省正字,遷校書郎。出知眉州,累遷戶部郎中、總領四川財賦,升太府少卿、利西安撫使。



 吳挺病,輔以吳氏世帥武興,久恐生變,密白二府,早擇人望以鎮方面。又貽書四川制置丘崈言:「統制官李奭乃吳氏腹心,緩急不可令權軍。」崈然之。挺卒,崈檄輔權帥事,輔謂:「職為王
 人,若輕往,第疑軍心。」遂索印即益昌領事。復數月,奏以權興州事楊虞仲兼權。



 召守秘書監、禮部侍郎,以顯謨閣待制知江陵府,移襄陽,又移潼川。召還,除顯謨閣直學士,奉外祠,尋以敷文閣直學士知成都府、兼本路安撫使。韓侂胄決意用兵,以吳曦為四川宣撫副使,假以節制財利之權。輔知曦有異志,貽書大臣言:「自昔兵帥與計臣不相統攝,故總領有報發覺察之權。今所在皆受節制,內憂不輕。」因托言他事,遣人以礬書告於朝。朔日,
 率官屬東望拜表如常儀。上意輔能誅曦,密詔授寶謨閣學士、四川制置使,許以便宜從事。時人望輔倡義,劉光祖、李道傳皆勉之。輔自以不習兵事,且內郡無兵可用,遷延兩月,但為去計。曦移輔知遂寧府,輔遂以印授通判韓植而去。



 安丙、楊巨源密謀誅曦,以輔有人望,謂密詔自輔所來,聞者皆信。曦既誅,丙趣輔還成都,除四川宣撫使。奏言:「臣以衰病軟懦,而居建元功者之上,徒恐牽制敗事。安丙才力強濟,賞罰明果,乞以事任付丙。」
 又論:「蜀中三帥,惟武興事權特重,故致今日之變。乞並置兩帥,分其營屯、隸屬。」



 安丙奏乞兩宣撫分司,朝廷察丙與輔異,召輔赴闕。議者謂蜀亂初平,如輔未宜去,乃復以為制置使兼知成都府。再被召,逾年財抵建康,復引咎不進。上召輔益堅,乃之鎮江俟命。著作佐郎楊簡言輔嘗棄成都,不當召,乃除兵部尚書兼侍讀,以龍圖閣學士知建康府兼江、淮制置使。卒於官,謚曰莊惠。



 劉光祖,字德修,簡州陽安人。幼出於外祖賈暉,後以暉
 遣澤補官。登進士第,廷對,言:「陛下睿察太精,宸斷太嚴,求治太速,喜功太甚。」又言:「陛下躬擐甲胄,間馭球馬,一旦有警,豈能親董六師以督戰乎?夫人主自將,危道也。臣恐球馬之事,敵人聞之,適以貽笑,不足以示武。」除劍南東川節度推官,闢潼川提刑司檢法。



 淳熙五年,召對,論恢復事,請以太祖用人為法,且曰:「人臣獻言,不可不察:其一,不量可否,勸陛下輕出驟進,則是即日誤國;其一,不思振立,茍且偷安,則是久遠誤國。」除太學正。召試,
 守正字,兼吳、益王府教授,遷校書郎,除右正言、知果州。以趙汝愚薦,召入。



 光宗即位,除軍器少監兼權侍左郎官,又兼禮部。時殿中侍御史闕,上方嚴其選,謂宰相留正曰:「卿監、郎官中有其人。」正沈思久之,曰:「得非劉光祖乎?」上曰:「是久在朕心矣。」



 光祖入謝,因論:



 近世是非不明,則邪正互攻;公論不立,則私情交起。此固道之消長,時之否泰,而實為國家之禍福,社稷之存亡,甚可畏也。本朝士大夫學術議論,最為近古,初非有強國之術,而國
 勢尊安,根本深厚。咸平、景德之間,道臻皇極,治保太和,至於慶歷、嘉祐盛矣。不幸而壞於熙、豐之邪說,疏棄正士,招來小人,幸而元祐君子起而救之,末流大分,事故反覆。紹聖、元符之際,群兇得志,絕滅綱常,其論既勝,其勢既成,崇、觀而下,尚復何言。



 臣始至時,聞有譏貶道學之說,而實未睹朋黨之分。中更外艱,去國六載,已憂兩議之各甚,而恐一旦之交攻也。逮臣復來,其事果見。因惡道學,乃生朋黨,因生朋黨,乃罪忠諫。嗟乎,以忠諫為
 罪,其去紹聖幾何!陛下履位之初,端拱而治,凡所進退,率用人言,初無好惡之私,豈以黨偏為主。而一歲之內,逐者紛紛,中間好人固亦不少,反以人臣之私意,微累天日之清明。往往推忠之言,謂為沽名之舉;至於潔身以退,亦曰憤懟而然。欲激怒於至尊,必加之以訐訕。事勢至此,循默乃宜,循默成風,國家安賴?



 臣欲熄將來之禍,故不憚反復以陳。伏幾聖心豁然,永為皇極之主,使是非由此而定,邪正由此而別,公論由此而明,私情由
 此而熄,道學之譏由此而消,朋黨之跡由此而泯,和平之福由此而集,國家之事由此而理,則生靈之幸,社稷之福也。不然,相激相勝,展轉反復,為禍無窮,臣實未知稅駕之所。



 章既下,讀之有流涕者。劾罷戶部尚書葉翥、太府卿兼中書舍人沈揆結近習,圖進用,言:「比年以來,士大夫不慕廉靖而慕奔競,不尊名節而尊爵位,不樂公正而樂軟美,不敬君子而敬庸人,既安習以成風,謂茍得為至計。良由前輩老成,零落殆盡,後生晚進,議論
 無所據依,學術無所宗主,正論益衰,士風不競。幸詔大臣,妙求人物,必朝野所共屬、賢愚所同敬者一二十人,參錯立朝,國勢自壯。臣雖終歲無所奏糾,固亦未至曠官。今日之患,在於不封殖人才,臺諫但有摧殘,廟堂初無長養。臣處當言之地,豈以排擊為能哉?」徙太府少卿。求去不已,除直秘閣、潼川運判。改江西提刑,又改夔州。



 時孝宗不豫,上久不過宮,光祖致書留正、趙汝愚曰:「宜與群賢並心一力,若上未過宮,宰執不可歸安私第。林、
 陳二閹,自以獲罪重華,日夜交諜其間。宜用韓魏公逐任守忠故事,以釋兩宮疑謗。大臣亦當收兵柄,密布腹心,俾緩急有可仗者。」聞孝宗崩,又貽書汝愚,勉以安國家、定社稷之事。



 寧宗即位,除侍御史,改司農少卿。入對,獻《謹始》五箴。又論:「人主有六易:天命易恃,天位易樂,無事易安,意欲易奢,政令易怠,歲時易玩。又有六難:君子難進,小人難退,苦言難入,巧佞難遠,是非難明,取舍難決。暗主之所易,明主之所難;暗主之所難,明主之所易。」
 又言:「陛下以隆慈之命,踐祚於素幄,蓋有甚不得已者,宜躬自貶損,盡禮於上皇,使聖意歡然知釋位之樂,然後足以昭陛下之大孝。」上悚然嘉納。



 進起居舍人。論:「政令當出中書,陛下審而行之,人主操柄,無要於此。」知閣門事韓侂胄浸擅威福,故首及之。遷起居郎。集議卜孝宗山陵,與朱熹皆謂會稽山陵,土薄水淺,乞議改卜。既而熹與祠,光祖言:「漢武帝之於汲黯,唐太宗之於魏徵,仁宗之於唐介,皆暫怒旋悔。熹明先聖之道,為今宿儒,
 又非三臣比。陛下初膺大寶,招徠耆儒,比初政之最善者。今一旦無故去之,可乎?」且曰:「臣非助熹,助陛下者也。」再疏,不聽。



 劉德秀劾光祖,出為湖南運判,不就,主管玉局觀。趙汝愚既罷相,侂胄擅朝,遂目士大夫為偽學逆黨,禁錮之。光祖撰《涪州學記》,謂:「學之大者,明聖人之道以修其身,而世方以道為偽;小者治文章以達其志,而時方以文為病。好惡出於一時,是非定於萬世。」諫官張釜指為謗訕,比之楊惲,奪職,謫居房州。久之,許自便。起
 知眉州,復職,將漕利路,以不習邊事辭。進直寶謨閣,主管沖祐觀。



 吳曦叛,光祖白郡守,焚其榜通衢,且馳告帥守、監司之所素知者,仗大義,連衡以抗賊。俄聞曦誅,則以書屬宣撫使楊輔,講行營田,前日利歸吳氏者,悉收之公上,以省餉軍費;獎名節、旌死事以激忠烈之心。除潼川路提刑、權知瀘州。侂胄誅,召除右文殿修撰、知襄陽府,進寶謨閣待制、知遂寧府,改京、湖制置使,以寶謨閣直學士知潼川府。



 詔以閔雨求言,光祖奏:「女直乃吾
 不共戴天之仇,天亡此仇,送死於汴。陛下為天之子,不知所以圖之,天與不取,是謂棄天,未有棄天而天不我怒也。青、鄆、藺、會求通弗納,陛下為中國衣冠之主,人歸而我絕之,是謂棄人,未有棄人而人不我怨也。且金人舍其巢穴,污我汴京,尚可使吾使人拜之於祖宗昔日朝會之廷乎?」



 又請改正憲聖慈烈皇后諱日。先是,後崩以慶元三年十一月二日,郊禋期迫,或謂侂胄曰:「上親郊,不可不成禮。且有司所費既夥,奈何已之?」侂胄入其
 言,五日祀圜丘,六日始宣遺誥。於是光祖言:「憲聖,陛下之曾祖母,克相高宗,再造大業。侂胄敢視之如卑喪,遷就若此。賊臣就戮,盍告謝祖宗,改從本日?」從之。



 升顯謨閣直學士、提舉玉隆萬壽宮。引年不許,提舉西京嵩山崇福宮。嘉定十五年卒,進華文閣學士,謚文節。



 趙汝愚稱光祖論諫激烈似蘇軾,懇惻似範祖禹,世以為名言。所著《後溪集》十卷。子:端之、靖之、翊之、竑之。



 論曰:徐誼竄逐於小人之手,身之否,道之亨也。吳獵之
 以學為政,項安世之通經博古,皆一時之英才,今更定舊史,公論其少伸歟!薛叔似通儒也,不幸以開邊事累之。劉甲、楊輔蔚乎有用之才。劉光祖盛名與《涪州學記》並傳穹壤,世之人何憚而不為君子也!



\end{pinyinscope}