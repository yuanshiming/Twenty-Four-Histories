\article{列傳第七十九}

\begin{pinyinscope}

 蔡
 襄呂溱王素從子靖從孫震余靖彭思永張存



 蔡襄,字君謨,興化仙游人。舉進士,為西京留守推官、館閣校勘。範仲淹以言事去國,餘靖論救之,尹洙請與同
 貶,歐陽修移書責司諫高若訥,由是三人者皆坐譴。襄作《四賢一不肖詩》,都人士爭相傳寫,鬻書者市之,得厚利。契丹使適至,買以歸,張於幽州館。



 慶歷三年,仁宗更用輔相,親擢靖、修及王素為諫官,襄又以詩賀,三人列薦之,帝亦命襄知諫院。襄喜言路開,而慮正人難久立也。乃上疏曰:「朝廷增用諫臣,修、靖、素一日並命,朝野相慶。然任諫非難,聽諫為難;聽諫非難,用諫為難。三人忠誠剛正,必能盡言。臣恐邪人不利,必造為御之之說。其
 御之之說不過有三,臣請為陛下辨之。一曰好名。夫忠臣引君當道,論事唯恐不至,若避好名之嫌無所陳,則土木之人,皆可為矣。二曰好進。前世諫者之難,激於忠憤,遭世昏亂,死猶不辭,何好進之有?近世獎拔太速,但久而勿遷,雖死是官,猶無悔也。三曰彰君過。諫爭之臣,蓋以司過舉耳,人主聽而行之,足以致從諫之譽,何過之能彰。至於巧者亦然,事難言則喑而不言,擇其無所忤者,時一發焉,猶或不行,則退而曰吾嘗論某事矣,此
 之謂好名。默默容容,無所愧恥,躡資累級,以挹顯仕,此之謂好進。君有過失,不救之於未然,傳之天下後世,其事愈不可掩,此之謂彰君過。願陛下察之,毋使有好諫之名而無其實。」



 時有旱蝗、日食、地震之變,襄以為:「災害之來,皆由人事。數年以來,天戒屢至。原其所以致之,由君臣上下皆闕失也。不顓聽斷,不攬威權,使號令不信於人,恩澤不及於下,此陛下之失也。持天下之柄,司生民之命,無嘉謀異畫以矯時弊,不盡忠竭節以副任使,
 此大臣之失也。朝有敝政而不能正,民有疾苦而不能去,陛下寬仁少斷而不能規,大臣循默避事而不能斥,此臣等之罪也。陛下既有引過之言,達於天地神祇矣,願思其實以應之。」疏出,聞者皆悚然。



 進直史館,兼修起居注,襄益任職論事,無所回撓。開寶浮圖災,下有舊瘞佛舍利,詔取以入,宮人多灼臂落發者。方議復營之,襄諫曰:「非理之福,不可徼幸。今生民困苦,四夷驕慢,陛下當修人事,奈何專信佛法?或以舍利有光,推為神異,彼
 其所居尚不能護,何有於威靈。天之降災,以示儆戒,顧大興功役,是將以人力排天意也。」



 呂夷簡平章國事,宰相以下就其第議政事,襄奏請罷之。元昊納款,始自稱「兀卒」,既又譯為「吾祖」。襄言:「『吾祖』猶云『我翁』,慢侮甚矣。使朝廷賜之詔,而亦曰『吾祖』,是何等語邪?」



 夏竦罷樞密使,韓琦、範仲淹在位,襄言:「陛下罷竦而用琦、仲淹,士大夫賀於朝,庶民歌於路,至飲酒叫號以為歡。且退一邪,進一賢,豈遂能關天下輕重哉?蓋一邪退則其類退,一賢
 進則其類進。眾邪並退,眾賢並進,海內有不泰乎!雖然,臣切憂之。天下之勢,譬猶病者,陛下既得良醫矣,信任不疑,非徒愈病,而又壽民。醫雖良術。不得盡用,則病且日深,雖有和、扁,難責效矣。」



 保州卒作亂,推懦兵十餘輩為首惡,殺之以求招撫。襄曰:「天下兵百萬,茍無誅殺決行之令,必開驕慢暴亂之源。今州兵戕官吏、閉城門,不能討,從而招之,豈不為四方笑。乞將兵入城,盡誅之。」詔從其議。



 以母老,求知福州,改福建路轉運使,開古五塘
 溉民田,奏減五代時丁口稅之半。復修起居注。唐介擊宰相,觸盛怒,襄趨進曰:「介誠狂愚,然出於進忠,必望全貸。」既貶春州,又上疏以為此必死之謫,得改英州。溫成後追冊,請勿立忌,而罷監護園陵官。



 進知制誥,三御史論梁適解職,襄不草制。後每除授非當職,輒封還之。帝遇之益厚,賜其母冠帔以示寵,又親書「君謨」兩字,遣使持詔予之。遷龍圖閣直學士、知開封府。襄精吏事,談笑剖決,破奸發隱,吏不能欺。以樞密直學不士再知福州。郡
 士周希孟、陳烈、陳襄、鄭穆以行義著,襄備禮招延,誨諸生以經學。俗重兇儀,親亡或秘不舉,至破產飯僧,下令禁止之。徙知泉州,距州二十里萬安渡,絕海而濟,往來畏其險。襄立石為梁,其長三百六十丈,種蠣於礎以為固,至今賴焉。又植松七百里以庇道路,閩人刻碑紀德。



 召為翰林學士、三司使,較天下盈虛出入,量力以制用。鏟剔蠹敝,簿書紀綱纖悉皆可法。



 英宗不豫,皇太后聽政,為輔臣言:「先帝既立皇子,宦妾更加熒惑,而近臣知
 名者亦然,幾敗大事,近已焚其章矣。」已而外人遂云襄有論議,帝聞而疑之。會襄數謁告,因命擇人代襄。襄乞為杭州,拜端明殿學士以往。治平三年,丁母憂。明年卒,年五十六。贈吏部侍郎。



 襄工於書,為當時第一,仁宗尤愛之,制《元舅隴西王碑文》命書之。及令書《溫成後父碑》,則曰:「此待詔職耳。」不奉詔。於朋友尚信義,聞其喪,則不御酒肉,為位而哭。嘗飲會靈東園,坐客誤射矢傷人,遽指襄。他日帝問之,再拜愧謝,終不自辨。



 蔡京與同郡而
 晚出,欲附名閥,自謂為族弟。政和初,襄孫佃廷試唱名,居舉首,京侍殿上,以族孫引嫌,降為第二,佃終身恨之。乾道中,賜襄謚曰忠惠。



 呂溱,字濟叔,揚州人。進士第一。通判亳州,直集賢院,同修起居注。坐預進奏院宴飲,出知蘄、楚、舒三州。復修起居注。



 儂智高寇嶺南,詔奏邸毋得輒報。溱言:「一方有警,使諸道聞之,共得為備。今欲人不知,此何意也。」進知制誥,又出知杭州,入為翰林學士。疏論宰相陳執中奸邪,
 仁宗還其疏。溱曰:「以口舌論人,是陰中大臣也。願出以示執中,使得自辨。」未幾,執中去,溱亦以侍讀學士知徐州,賜宴資善堂,遣使諭曰:「此特為卿設,宜盡醉也。」詔自今由經筵出者視為例。



 徙成德軍,時方開六塔河,宰相主其議。會地震,溱請罷之,以答天戒。溱豪侈自放,簡忽於事。與都轉運使李參不相能,還,判流內銓,參劾其借官曲作酒,以私貨往河東貿易,及違式受饋贐,事下大理議。溱乃未嘗受,而外廷紛然謂溱有死罪。帝知其過
 輕,但貶秩,知和州。御史以為未抵罪,分司南京。起知池州、江寧府,復集賢院學士,加龍圖閣直學士、知開封府。



 時為京尹者比不稱職,溱精識過人,辨訟立斷,豪惡斂跡。嘗以職事對,神宗察其有疾色,勉以近醫藥,已而果病。改樞密直學士、提舉醴泉觀,遂卒,年五十五。贈禮部侍郎。帝悼念之,詔中書曰:「溱立朝最孤,知事君之節,絕跡權貴,故中廢十餘年,人無言者。方擢領要劇,而奄忽淪亡,家貧子幼,遭此大禍,必至狼狽。宜優給賻禮,官庀
 其葬,以厲臣節。」敕其婦兄護喪歸。



 溱開敏,善議論,一時名輩皆推許。然自貴重,在杭州接賓客,不過數語,時目為「七字舍人」云。



 王素,字仲儀,太尉旦季子也。賜進士出身,至屯田員外郎。御史中丞孔道輔薦為侍御史。道輔貶,出知鄂州。仁宗思其賢,擢知諫院。素方壯年,遇事感發。嘗言:「今中外無名之費,倍蓰於前,請省其非急者。」適皇子生,將進百僚以官,惠諸軍以賞。素爭曰:「今西夏畔渙,契丹要求,縣
 官之須,且日急矣。宜留爵秩以賞戰功,儲金繒以佐邊費。」議遂已。



 京師旱,素請帝禱於郊,帝曰:「太史言月二日當雨,今將以旦日出禱。」素曰:「臣非太史,然度是日必不雨。帝問故,曰:「陛下知其且雨而禱之,應天不以誠,故臣知不雨。」帝曰:「然則明日詣醴泉觀。」素曰:「醴泉之近,猶外朝耳,豈憚暑不遠出邪?」帝悚然。更詔詣西太一宮,諫官故不在屬車間,乃命素扈從。日甚熾,埃氛翳空,比輿駕還,未薄城,天大雷電而雨。



 王德用進二女子,素論之,帝
 曰:「朕真宗皇帝之子,卿王旦之子,有世舊,非他人比也。德用實進女,然已事朕左右,奈何?」素曰:「臣之憂正恐在左右爾。」帝動容,立命遣二女出。賜素銀緋,擢天章閣待制、淮南都轉運按察使。時新置按察,類多以苛為明。素獨不擿細故,即有貪刻,必繩治窮竟,以故下吏愛而畏之。改知渭州,坐市木河東有擾民狀,降華州,又奪職徙汝。俄悉還其故,遷龍圖閣直學士。



 初,原州蔣偕建議築大蟲巉堡,宣撫使聽之。役未具,敵伺間要擊,不得成。偕
 懼來歸死。素曰:「若罪偕,乃是墮敵計。」責偕使畢力自效。總管狄青曰:「偕往益敗,不可遣。」素曰:「偕敗則總管行,總管敗,素即行矣。」青不敢復言,偕卒城而還。以樞密直學士知開封府。至和秋,大雨,蔡河裂,水入城。詔軍吏障朱雀門,素曰:「皇上不豫,兵民廬舍多覆壓,眾心怦怦然,奈何更塞門以動眾。」違詔止其役,水亦不害。



 出知定州、成都府。先是,牙校歲輸酒坊錢以供廚傳,日加厚,輸者轉困。素一切裁約之。鐵錢布滿兩蜀,而鼓鑄不止,幣益輕,
 商賈不行,命罷鑄十年,以權物價。凡為政,務合人情,蜀人紀其目,號曰「王公異斷」。復知開封。素以三公子少知名,出入侍從將帥,久頗鞅鞅,厭倦劇煩,事多鹵莽不治,盜賊數發。御史糾其過,出知許州。



 治平初,夏人寇靜邊砦。召拜端明殿學士,復知渭州,於是三鎮、涇原蕃夷故老皆歡賀,比至,敵解去。拓渭西南城,浚隍三周,積粟支十年。屬羌奉土地來獻,悉增募弓箭手。行陳出入之法,身自督訓。其居舊穿土為室,寇至,老幼多焚死,為築八
 堡使居之。其從領於兩巡檢,人莫得自便。素曰:「是豈募民兵意邪?」聽散耕田里,有警則聚,故士氣感奮,精悍他道莫及。嘗宴堂上,邊民傳寇至,驚入城。諸將曰:「使奸人亦從而入,將必為內應,合拒勿內。」素曰:「若拒之東去,關中必搖。吾在此,敵必不敢犯我,此當有奸言。」乃下令:「敢稱寇至者斬。」有頃,候騎從西來,人傳果妄,諸將皆服其明。



 換澶州觀察使、知成德軍,改青州觀察使。熙寧初。還,以學士知太原府。汾河大溢,素曰:「若壞平晉,遂灌州城
 矣。」亟命具舟楫,築堤以捍之。一夕,水驟至,人賴以安。入知通進、銀臺司,轉工部尚書,仍故職致仕。故事,雖三公致仕,亦不帶職。朝廷方新法制,素首以學士就第。卒,年六十七,謚曰懿敏。子鞏,從子靖,從孫震。



 鞏有雋才,長於詩,從蘇軾游。軾守徐州,鞏往訪之,與客游泗水,登魋山,吹笛飲酒,乘月而歸。軾待之於黃樓上,謂鞏曰:「李太白死,世無此樂三百年矣。」軾得罪,鞏亦竄賓州。數歲得還,豪氣不少挫。後歷宗正丞,以跌蕩傲世,每除官,輒為言
 者所議,故終不顯。



 靖字詹叔,蚤孤,自力於學,好講切天下利害。以祖蔭歷通判閬州、知滁州,主管北京御史臺。契丹數遣橫使來,靖疏言:「彼利中國賜遺,挾虛聲以濟其欲,漸不可長,宜有以折之。」又請復明經科,加試貢士以策,觀其所學,稍變聲律之習。



 擢利州路轉運判官,提點陜西刑獄。鄉戶役於州縣者,優則願久留,勞則欲亟去,吏得權其遲速。靖一以歲月遣代,遂為令。徙河東長子縣。賊殺人,捕治
 十數輩,不得實,皆釋去。靖閱其牘曰:「此真盜也。」教吏曲折訊囚,果服罪。為開封府推官。曹、濮盜害,官吏久不獲,靖受詔督捕,成擒者十八九。因言盜之不戢,由大姓為囊橐,請並坐之,著為令。



 徙廣南轉運使。熙寧初,廣人訛言交址且至,老幼入保。事聞,中外以為憂。神宗曰:「王靖在彼,可無念。」即拜太常少卿、直昭文館、知廣州。居二年,入為度支副使,卒。



 子古,字敏仲,第進士。熙寧中,為司農主簿,使行淮、浙振旱菑,究張若濟獄,劾轉運使王廷老、
 張靚失職,皆罷之。連提舉四路常平,王安禮欲用為太常丞,神宗謂古好異論,止以為博士。加上仁宗、英宗謚,因升祔四後,初議不發冊,古言:「發冊之禮,雖為祔廟節文,而升祔之重,乃由冊而後顯。今既行升祔,則禮不可廢。」乃詔用竹冊。又定諸神祠封額、爵號之序。



 出為湖南轉運判官,提點淮東刑獄,歷工部、吏部、右司員外郎,太府少卿。奉使契丹,異時北使所過,凡供張悉貸於民,古請出公錢為之,民得不擾。紹聖初,遷戶部侍郎,詳定役
 法,與尚書蔡京多不合。京言:「臣欲用元豐人額雇直,而古乃用司馬光法。」詔徙古兵部,尋以集賢殿修撰為江、淮發運使,進寶文閣待制、知廣州。言者論其常指平歲為兇年,妄散邦財,奪職知袁州。



 徽宗立,復拜戶部侍郎,遷尚書。與御史中丞趙挺之偕領放欠,挺之言:「古蠲除太多,欲盡傾天下之財,不可用。」遂改刑部。攻不已,以寶文閣直學士知成都。墮崇寧黨籍,責衡州別駕,安置溫州。復朝散郎,尋卒。



 震字子發,以父任試銓優等,賜及第。上諸路學制,神宗稱其才。以習學中書刑房公事,遂為檢正。預修條例,加館閣校勘,檢正孔目吏房。



 元豐官制行,震與吳雍從輔臣執筆入記上語,面授尚書右司員外郎,使自書除目,舉朝榮之。兼修《市易敕》,帝諭之曰:「朝廷造法,皆本先王之制,推行非人,故不能善後。且以錢貸民,有不能償,輒籍其家,豈善政也。宜計其負幾何,悉捐之。」震頓首奉詔。



 進起居舍人,使行西邊,還為中書舍人。元祐初,遷給事
 中,御史王巖叟劾之,以龍圖閣待制知蔡州,歷五郡。紹聖初,復為給事中,權吏部尚書,拜龍圖閣直學士、知開封府。



 震與章惇皆呂惠卿所薦,而素不相能。府奏獄空,哲宗疑不實。震謂惇抑已,於是穎昌蓋漸有訟,許賂惇子弟,震捕漸掠治,頗得蹤跡。惇懼,以獄付大理,而徙震為樞密都承旨,遂坐折獄滋蔓、傾搖大臣,奪職知嶽州,卒。



 餘靖,字安道,韶州曲江人。少不事羈檢,以文學稱鄉里。
 舉進士起家,為贛縣尉,試書判拔萃,改將作監丞、知新建縣,遷秘書丞。數上書論事,建言班固《漢書》舛謬,命與王洙並校司馬遷、範曄二史。書奏,擢集賢校理。



 範仲淹貶饒州,諫官御史莫敢言。靖言:「仲淹以刺譏大臣重加譴謫,倘其言未合聖慮,在陛下聽與不聽耳,安可以為罪乎?汲黯在廷,以平津為多詐;張昭論將,以魯肅為粗疏。漢皇、吳主熟聞訾毀,兩用無猜,豈損令德。陛下自親政以來,屢逐言事者,恐鉗天下口,不可。」疏入,落職監筠
 州酒稅。尹洙、歐陽修亦以仲淹故,相繼貶逐,靖繇是益知名。徙監泰州稅,知英州,遷太常博士,復為校理、同知禮院。



 慶歷中,仁宗銳意欲更天下敝事,增諫官員,使論得失,以靖為右正言。時四方盜賊竊發,州郡不能制。靖言:「朝廷威制天下在賞罰,今官吏弛事,群盜蜂起,大臣齷齪守常,不立法禁,可為國家憂也。請嚴捕賊賞罰,及定為賊劫質、亡失器甲除名追官之法。」



 司天言太白犯歲星,又犯執法。靖上疏請責躬修德,以謝天變。使契丹,
 辭日,以所奏事書笏,各舉一字為目,凡數十事。帝顧見之,命悉條奏,日幾昃,乃罷。進修進居注。開寶寺靈感塔災,復上疏言:「五行之占,本是災變,朝廷所宜誡懼,以答天意。聞嘗詔取舊瘞舍利入禁中閱視,道路傳言,舍利在內廷有光怪,竊恐巧佞之人,推為靈異,惑亂視聽,再圖營造。臣聞帝王之道,能勤儉厥德,感動人心,則雖有危難,後必安濟。今自西垂用兵,國帑虛竭,民亡儲蓄,十室九空。陛下若勤勞罪己,憂人之憂,則四民安居,海內
 蒙福。如不恤民病,廣事浮費,奉佛求福,非天下所望也。若以舍利經火不壞,遽為神異,即本在土中,火所不及。若言舍利皆能出光怪,必有神靈憑之,此妄言也。且一塔不能自衛,為火所毀,況藉其福以庇民哉?」



 靖在職數言事,嘗論夏竦奸邪,不可為樞密使;王舉正不才,不宜在政府;狄青武人,使之獨守渭州,恐敗邊事;張堯佐以修媛故,除提點府界公事,非政事之美,且郭后之禍,起於楊、尚,不可不監。太常博士王翼西京治獄還,賜五品
 服,靖曰:「治獄而錫服,外人不知,必以為翼深文重法,能希陛下意,以取此寵,所損非細事也。嘗有工部郎中呂覺以治獄賜對,祈易章綬,陛下諭之曰:『朕不欲因鞫囚與人恩澤。』覺退以告臣,臣嘗書之起居注。陛下前日諭覺是,則今日賜翼非矣。是非與奪之間,貴乎一體。小人望風希進,無所不至,幸陛下每於事端,抑其奔競。」其說多見納用。



 會西鄙厭兵,元昊請和,議增歲賜。靖言:「景德中,契丹舉國興師,直抵澶淵,先帝北征渡河,止捐金繒
 三十萬與之。今元昊戰雖累勝,皆由將帥輕敵易動之故。數年選將練兵,始知守戰之備,而銳意解仇,所予至二十六萬。且戎事有機,國力有限,失之於始,雖悔何追。夫以景德之患,近在封域之內,而歲賜如彼;今日之警,遠在邊鄙之外,而歲賜如此。若元昊使還,益有所許,契丹聞之,寧不生心?無厭之求,自此始矣。儻移西而備北,為禍更深。但思和與不和,皆有後患,則不必曲意俯徇,以貽國羞。」擢知制誥。



 元昊既歸款,朝廷欲加封冊,而契
 丹以兵臨西境,遣使言:「為中國討賊,請止毋和。」朝議難之。會靖數言契丹挾詐,不可輕許,即遣靖往報,而留夏國封策不發。靖至契丹,卒屈其議而還。朝廷遂發夏冊,臣元昊。西師既解嚴,北邊亦無事。靖三使契丹,亦習外國語,嘗為番語詩,御史王平等劾靖失使者體,出知吉州。靖為諫官時,嘗劾奏太常博士茹孝標不孝,匿母喪,坐廢。靖既失勢,孝標詣闕言靖少游廣州,犯法受榜。靖聞之不自得,求侍養去。改將作少監,分司南京,居曲江。
 已而授左神武軍大將軍、雅州刺史、壽州兵馬鈐轄,辭不就。再遷衛尉卿、知虔州,丁父憂去。



 儂智高反邕州,乘勝掠九郡,以兵圍廣州。朝廷方顧南事,就喪次起靖為秘書監、知潭州,改桂州,詔以廣南西路委靖經制。智高西走邕州,靖策其必結援交□止,而脅諸峒以自固,乃約李德政會兵擊賊於邕州,備萬人糧以待之;而詔亦給緡錢二萬助德政興師,且約賊平更賞以緡錢二萬。又募儂、黃諸姓酋長,皆縻以職,使不與智高合。既而朝廷
 遣狄青、孫沔將兵共討賊。青卻交址,援兵不用,賊平。就遷靖給事中。御史梁茜言賞薄,又遷尚書工部侍郎。初,青兵未至前,戒部將勿戰。靖迫鈐轄陳曙出鬥,敗走。青至,按軍法斬曙及指使袁用等於坐,靖瞿然起拜。及諸將班師,獨留靖廣西,遣人入特磨道擒智高母子弟三人,生致之闕下。加集賢院學士,徙知潭州,又徙青州。



 交址蠻申紹泰寇邕州,殺五巡檢。以靖安撫廣西,至則召交址用事臣費嘉祐詰問之,嘉祐至,紿以近邊種落相
 侵報,誤犯官軍,願悉推治,還所掠及械罪人以自贖。靖信之,厚謝遣去,嘉祐遂歸,不復出。



 知廣州,官至工部尚書,代歸,卒。三司使蔡襄為靖言,特贈刑部尚書,謚曰襄。靖嘗夢神人告以所終官而死秦亭,故靖常畏西行。及卒,則江寧府秦淮亭也。



 彭思永,字季長,廬陵人。第進士,知南海、分寧縣,通判睦州。臺州大水敗城,人多溺,往攝治焉。盡葬死者,作文祭之;民貧不能葺居,為伐木以助之,數月,公私之舍皆具,
 城築高於前,而堅亦如之。



 知潮州、常州。入為侍御史,論內降授官賞之弊,謂斜封非盛世所當有,仁宗深然之。皇祐祀明堂前一日,有傳百官皆進秩者。思永言不宜濫恩,以益僥幸。時張堯佐已貴而猶覬執政,王守忠已受寵而求旄節。思永率同列言之,或曰:「俟命出,未晚也。」思永曰:「先事而言,第得罪爾;命一出,不可止矣。」遂獨抗疏曰:「陛下覃此謬恩,豈為天下孤寒哉。不過為堯佐、守忠取悅眾人耳。外戚秉政,宦侍用權,非社稷之福也。」帝
 怒,中丞郭勸、諫官吳奎為之請,乃以泛恩轉司封員外郎而解臺職,為湖北轉運使。



 下溪蠻彭仕羲作亂,先移書激罵辰州守。守將討之,思永按部適至,仕羲懼,遣使迎謝,寢其謀。



 加直史館,為益州路轉運使。成都府吏盜公錢,付獄已三歲,出入自如。思永攝府事甫一日,即具獄。民以楮券為市,藏衣帶中,盜置刃於爪,捷取之,鮮敗者。思永得一人詰之,悉黥其黨隸兵間。中使歲祠峨眉,率留成都掊珍玩,價直數百萬錢,悉出於民。思永朘其
 三之一,使怒去,而不能有所中傷也。



 尋為戶部副使,擢天章閣待制、河北都轉運使、知瀛州。北俗以桑麻為產籍,民懼賦不敢藝,日益貧,思永始奏更之。徒知江寧府。



 治平中,召為御史中丞。濮王有稱親之議,言事者爭之,皆斥去。思永更上疏極論曰:「濮王生陛下,而仁宗以陛下為嗣,是仁宗為皇考,而濮王於屬為伯,此天地大義,生人大倫。如乾坤定位,不可得而變也。陛下為仁廟子,曰考曰親,乃仁廟也;若更施於濮王,是有二親矣。使王
 與諸父夷等,無有殊別,則於大孝之心亦為難安。臣以為當尊為濮國大王,祭告之辭,則曰『侄嗣皇帝書名昭告於皇伯父』。在王則極尊崇之道,而於仁廟亦無所嫌矣,此萬世之法也。」疏入,英宗感其切至,垂欲施行,而中書持之甚力,卒不果。



 神宗即位,御史蔣之奇糾歐陽修陰事,挽思永自助。思永以為帷薄之私,非外人所知,但其首建濮議,違典禮以犯眾怒,不宜更在政府。詔問語所從來,思永不肯對,而極陳大臣專恣朋黨。乃出知黃
 州,改太平州。熙寧三年,以戶部侍郎致仕,卒,年七十一。



 思永仁厚廉恕。為兒時,旦起就學,得金釵於門外,默坐其處。須臾亡釵者來物色,審之良是,即付之。其人欲謝以錢,思永笑曰:「使我欲之,則匿金矣。」始就舉,持數釧為資。同舉者過之,出而玩,或墜其一於袖間,眾相為求索。思永曰:「數止此耳。」客去,舉手揖,釧墜於地,眾皆服其量。居母喪,窶甚,鄉人饋之,無所受。子衛,亦孝謹,以父老,棄官家居十餘年,族里稱之。



 張存,字誠之,冀州人。舉進士,為安肅軍判官。天禧中,詔銓司以身言書判取士,才得二人,存預其選。改著作佐郎,知大名府朝城縣。寇準為守,異待之。御史中丞王曙,屢薦為殿中侍御史,遷侍御史。



 仁宗初親政,罷百官轉對,存請復之。又言:「前者曹修古輩同忤旨廢黜,布衣林獻可因上封事竄惡地,恐自今忠直之言,與夫理亂安危之機,蔽而不達。」因歷引周昌、朱雲、辛慶忌、辛毗事,以開帝意。歷京東陜西、河北、轉運使、戶部度支副使。西邊
 動兵,以天章閣待制為陜西都轉運使。



 黃德和之誣劉平也,存奏言:「平與敵接戰,自旦至暮,殺傷相當,因德和引卻,以致潰敗。方賊勢甚張,非平搏戰,其勢必不沮;延州孤壘,非平解圍,其城必不守。身既陷沒,而不幸又為讒狡所困,邊臣自此無復死節矣。」朝廷採其說,始遣文彥博按治,由是平得直,而德和誅。



 元昊求款附,議者猶執攻討之策。存建言:「兵役不息,生民疲弊。敵既有悛心,雖名號未正,頗羈縻之。」遷龍圖閣直學士,知延州。以母
 老憚行,徙澤州,還為待制。逾年,知成德軍,復學士。



 契丹與元昊結昏,陰謀相首尾,聚兵塞上而求關南。存言:「河北城久不治,宜留意。」乃以為都運使,盡城諸州。入知開封府,復使河北。王則反,坐失察,降知汀州。



 存婿李易文之弟李教,因醉為妖言,事覺自縊死。或言教不死,在貝州,父母私屬以存故得免。御史案驗無狀,猶奪職知池州,又徙郴。久之,乃復職,以吏部侍郎致仕,凡十五年,積遷禮部尚書。



 存性孝友,嘗為蜀郡,得奇繒文錦以歸,悉布
 之堂上,恣兄弟擇取。常曰:「兄弟,手足也;妻妾,外舍人耳。奈何先外人而後手足乎?」收恤宗屬,嫁聘窮嫠,不使一人失所。家居矜莊,子孫非正衣冠不見。與賓友燕接,垂足危坐終日,未嘗傾倚。棗強河決,勢逼冀城,或勸使他徒,曰:「吾家,眾所望也,茍輕舉動,使一州吏民何以自安。」訖不徙。卒,年八十八,謚恭安。



 論曰:蔡襄、王素、餘靖,皆昭陵賢御史也。襄數論治體,推韓琦、範仲淹之賢。素請罷不急之賞,論仁宗納二女子
 為非。靖黜夏竦、王舉正為不可用。蓋仁宗銳於求治,數君子提綱振紀而扶持之,卒成慶歷之治,良有以也。夫襄精於民事,吏不敢欺;靖用兵蠻徼,卒收功名;素在西邊多惠政,其尹開封,雖頗厭煩劇,再為渭州,邊民老幼,至相率稱賀,其惠之在民者,深矣哉。若呂溱論陳執中,則不欲以口舌中人。彭思永名士,能識程頤之賢,而不能容歐陽修之剛;蔣之奇之誣,竟坐是黜,士論憾之。劉平之死,眾莫敢言,張存獨處而明之。使忠義之氣,死而
 復生,較之諸人,亦無忝焉。



\end{pinyinscope}