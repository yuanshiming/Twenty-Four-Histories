\article{列傳第九十五}

\begin{pinyinscope}

 司馬光子康呂公著子希哲希純



 司馬光,字君實,陜州夏縣人也。父池,天章閣待制。光生七歲,凜然如成人,聞講《左氏春秋》,愛之,退為家人講,即了其大指。自是手不釋書,至不知饑渴寒暑。群兒戲於
 庭,一兒登甕,足跌沒水中,眾皆棄去,光持石擊甕破之,水迸,兒得活。其後京、洛間畫以為圖。仁宗寶元初,中進士甲科。年甫冠,性不喜華靡,聞喜宴獨不戴花,同列語之曰:「君賜不可違。」乃簪一枝。



 除奉禮郎,時池在杭,求簽蘇州判官事以便親,許之。丁內外艱,執喪累年,毀瘠如禮。服除,簽書武成軍判官事,改大理評事,補國子直講。樞密副使龐籍薦為館閣校勘,同知禮院。中官麥允言死,給鹵簿。光言:「繁纓以朝,孔子且猶不可。允言近習之
 臣,非有元勛大勞而贈以三公官,給一品鹵簿,其視繁纓,不亦大乎。」夏竦賜謚文正,光言:「此謚之至美者,竦何人,可以當之?」改文莊。加集賢校理。



 從龐籍闢,通判並州。麟州屈野河西多良田,夏人蠶食其地,為河東患。籍命光按視,光建:「築二堡以制夏人,募民耕之,耕者眾則糴賤,亦可漸紓河東貴糴遠輸之憂。」籍從其策;而麟將郭恩勇且狂,引兵夜渡河,不設備,沒於敵,籍得罪去。光三上書自引咎,不報。籍沒,光升堂拜其妻如母,撫其子如
 昆弟,時人賢之。



 改直秘閣、開封府推官。交趾貢異獸,謂之麟,光言:「真偽不可知,使其真,非自至不足為瑞,願還其獻。」又奏賦以風。修起居注,判禮部。有司奏日當食,故事食不滿分,或京師不見,皆表賀。光言:「四方見、京師不見,此人君為陰邪所蔽;天下皆知而朝廷獨不知,其為災當益甚,不當賀。」從之。



 同知諫院。蘇轍答制策切直,考官胡宿將黜之,光言:「轍有愛君憂國之心,不宜黜。」詔置末級。



 仁宗始不豫,國嗣未立,天下寒心而莫敢言。諫官
 範鎮首發其議,光在並州聞而繼之,且貽書勸鎮以死爭。至是,復面言:「臣昔通判並州,所上三章,願陛下果斷力行。」帝沉思久之,曰:「得非欲選宗室為繼嗣者乎?此忠臣之言,但人不敢及耳。」光曰:「臣言此,自謂必死,不意陛下開納。」帝曰:「此何害,古今皆有之。」光退未聞命,復上疏曰:「臣向者進說,意謂即行,今寂無所聞,此必有小人言陛下春秋鼎盛,何遽為不祥之事。小人無遠慮,特欲倉卒之際,援立其所厚善者耳。『定策國老』、『門生天子』之禍,
 可勝言哉?」帝大感動曰:「送中書。」光見韓琦等曰:「諸公不及今定議,異日禁中夜半出寸紙,以某人為嗣,則天下莫敢違。」琦等拱手曰:「敢不盡力。」未幾,詔英宗判宗正,辭不就,遂立為皇子,又稱疾不入。光言:「皇子辭不貲之富,至於旬月,其賢於人遠矣。然父召無諾,君命召不俟駕,願以臣子大義責皇子,宜必入。」英宗遂受命。



 袞國公主嫁李瑋,不相能,詔出瑋衛州,母楊歸其兄璋,主入居禁中。光言:「陛下追念章懿太后,故使瑋尚主。今乃母子離
 析,家事流落,獨無雨露之感乎?瑋既黜,主安得無罪?」帝悟,降主沂國,待李氏恩不衰。進知制誥,固辭,改天章閣待制兼侍講、知諫院。時朝政頗姑息,胥史喧嘩則逐中執法,輦官悖慢則退宰相,衛士兇逆而獄不窮治,軍卒詈三司使而以為非犯階級。光言皆陵遲之漸,不可以不正。充媛董氏薨,贈淑妃,輟朝成服,百官奉慰,定謚,行冊禮,葬給鹵簿。光言:「董氏秩本微,病革方拜充媛。古者婦人無謚,近制惟皇后有之。鹵簿本以賞軍功,未嘗施
 於婦人。唐平陽公主有舉兵佐高祖定天下功,乃得給。至韋庶人始令妃主葬日皆給鼓吹,非令典,不足法。」時有司定後宮封贈法,後與妃俱贈三代,光論:「妃不當與後同,袁盎引卻慎夫人席,正為此耳。天聖親郊,太妃止贈二代,而況妃乎?」



 英宗立,遇疾,慈聖光獻後同聽政。光上疏曰:「昔章獻明肅有保祐先帝之功,特以親用外戚小人,負謗海內。今攝政之際,大臣忠厚如王曾,清純如張知白,剛正如魯宗道,質直如薛奎者,當信用之;猥鄙
 如馬季良,讒諂如羅崇勛者,當疏遠之,則天下服。」帝疾愈,光料必有追隆本生事,即奏言:「漢宣帝為孝昭後,終不追尊衛太子、史皇孫;光武上繼元帝,亦不追尊鉅鹿、南頓君,此萬世法也。」後詔兩制集議濮王典禮,學士王珪等相視莫敢先,光獨奮筆書曰:「為人後者為之子,不得顧私親。王宜準封贈期親尊屬故事,稱為皇伯,高官大國,極其尊榮。」議成,珪即命吏以其手稿為按。既上與大臣意殊,御史六人爭之力,皆斥去。光乞留之,不可,遂
 請與俱貶。



 初,西夏遣使致祭,延州指使高宜押伴,傲其使者,侮其國主,使者訴於朝。光與呂誨乞加宜罪,不從。明年,夏人犯邊,殺略吏士。趙滋為雄州,專以猛悍治邊,光論其不可。至是,契丹之民捕魚界河,伐柳白溝之南,朝廷以知雄州李中祐為不材,將代之。光謂:「國家當戎夷附順時,好與之計較末節,及其桀驁,又從而姑息之。近者西禍生於高宜,北禍起於趙滋;時方賢此二人,故邊臣皆以生事為能,漸不可長。宜敕邊吏,疆場細故輒
 以矢刃相加者,罪之。」



 仁宗遺賜直百餘萬,光率同列三上章,謂:「國有大憂,中外窘乏,不可專用乾興故事。若遺賜不可辭,宜許侍從上進金錢佐山陵。」不許。光乃以所得珠為諫院公使錢,金以遣舅氏,義不藏於家。後還政,有司立式,凡後有所取用,當覆奏乃供。光云:「當移所屬使立供已,乃具數白後,以防矯偽。」



 曹佾無功除使相,兩府皆遷官。光言:「陛下欲以慰母心,而遷除無名,則宿衛將帥、內侍小臣,必有覬望。」已而遷都知任守忠等官,光
 復爭之,因論:「守忠大奸,陛下為皇子,非守忠意,沮壞大策,離間百端,賴先帝不聽;及陛下嗣位,反復交構,國之大賊。乞斬於都市,以謝天下。」責守忠為節度副使,蘄州安置,天下快之。



 詔刺陜西義勇二十萬,民情驚撓,而紀律疏略不可用。光抗言其非,持白韓琦。琦曰:「兵貴先聲,諒祚方桀驁,使驟聞益兵二十萬,豈不震懾?」光曰:「兵之貴先聲,為無其實也,獨可欺之於一日之間耳。今吾雖益兵,實不可用,不過十日,彼將知其詳,尚何懼?」琦曰:「君
 但見慶歷間鄉兵刺為保捷,憂今復然,已降敕榜與民約,永不充軍戍邊矣。」光曰:「朝廷嘗失信,民未敢以為然,雖光亦不能不疑也。」琦曰:「吾在此,君無憂。」光曰:「公長在此地,可也;異日他人當位,因公見兵,用之運糧戍邊,反掌間事耳。」琦嘿然,而訖不為止。不十年,皆如光慮。



 王廣淵除直集賢院,光論其奸邪不可近:「昔漢景帝重衛綰,周世宗薄張美。廣淵當仁宗之世,私自結於陛下,豈忠臣哉?宜黜之以厲天下。」進龍圖閣直學士。



 神宗即位,擢
 為翰林學士,光力辭。帝曰:「古之君子,或學而不文,或文而不學,惟董仲舒、揚雄兼之。卿有文學,何辭為?」對曰:「臣不能為四六。」帝曰:「如兩漢制詔可也;且卿能進士取高第,而云不能四六,何邪?」竟不獲辭。



 御史中丞王陶以論宰相不押班罷,光代之,光言:「陶由論宰相罷,則中丞不可復為。臣願俟既押班,然後就職。」許之。遂上疏論修心之要三:曰仁,曰明,曰武;治國之要三:曰官人,曰信賞,曰必罰。其說甚備。且曰:「臣獲事三朝,皆以此六言獻,平生
 力學所得,盡在是矣。」御藥院內臣,國朝常用供奉官以下,至內殿崇班則出;近歲暗理官資,非祖宗本意。因論高居簡奸邪,乞加遠竄。章五上,帝為出居簡,盡罷寄資者。既而復留二人,光又力爭之。張方平參知政事,光論其不葉物望,帝不從。還光翰林兼侍讀學士。



 光常患歷代史繁,人主不能遍鑒,遂為《通志》八卷以獻。英宗悅之,命置局秘閣,續其書。至是,神宗名之曰《資治通鑒》,自制《序》授之,俾日進讀。



 詔錄穎邸直省官四人為閣門祗候,
 光曰:「國初草創,天步尚艱,故御極之初,必以左右舊人為腹心耳目,謂之隨龍,非平日法也。閣門祗候在文臣為館職,豈可使廝役為之。」



 西戎部將嵬名山欲以橫山之眾,取諒祚以降,詔邊臣招納其眾。光上疏極論,以為:「名山之眾,未必能制諒祚。幸而勝之,滅一諒祚,生一諒祚,何利之有;若其不勝,必引眾歸我,不知何以待之。臣恐朝廷不獨失信諒祚,又將失信於名山矣。若名山餘眾尚多,還北不可,入南不受,窮無所歸,必將突據邊城
 以救其命。陛下不見侯景之事乎?」上不聽,遣將種諤發兵迎之,取綏州,費六十萬,西方用兵,蓋自此始矣。



 百官上尊號,光當答詔,言:「先帝親郊,不受尊號。末年有獻議者,謂國家與契丹往來通信,彼有尊號我獨無,於是復以非時奉冊。昔匈奴冒頓自稱『天地所生日月所置匈奴大單于』,不聞漢文帝復為大名以加之也。願追述先帝本意,不受此名。」帝大悅,手詔獎光,使善為答辭,以示中外。



 執政以河朔旱傷,國用不足,乞南郊勿賜金帛。詔
 學士議,光與王珪、王安石同見,光曰:「救災節用,宜自貴近始,可聽也。」安石曰:「常袞辭堂饌,時以為袞自知不能,當辭位不當辭祿。且國用不足,非當世急務,所以不足者,以未得善理財者故也。」光曰:「善理財者,不過頭會箕斂爾。」安石曰:「不然,善理財者,不加賦而國用足。」光曰:「天下安有此理?天地所生財貨百物,不在民,則在官,彼設法奪民,其害乃甚於加賦。此蓋桑羊欺武帝之言,太史公書之以見其不明耳。」爭議不已。帝曰:「朕意與光同,然
 姑以不允答之。」會安石草詔,引常袞事責兩府,兩府不敢復辭。



 安石得政,行新法,光逆疏其利害。邇英進讀,至曹參代蕭何事,帝曰:「漢常守蕭何之法不變,可乎?」對曰:「寧獨漢也,使三代之君常守禹、湯、文、武之法,雖至今存可也。漢武取高帝約束紛更,盜賊半天下;元帝改孝宣之政,漢業遂衰。由此言之,祖宗之法不可變也。」呂惠卿言:「先王之法,有一年一變者,『正月始和,布法象魏』是也;有五年一變者,巡守考制度是也;有三十年一變者,『刑
 罰世輕世重』是也。光言非是,其意以風朝廷耳。」帝問光,光曰:「布法象魏,布舊法也。諸侯變禮易樂者,王巡守則誅之,不自變也。刑新國用輕典,亂國用重典,是為世輕世重,非變也。且治天下譬如居室,敝則修之,非大壞不更造也。公卿侍從皆在此,願陛下問之。三司使掌天下財,不才而黜可也,不可使執政侵其事。今為制置三司條例司,何也?宰相以道佐人主,安用例?茍用例,則胥吏矣。今為看詳中書條例司,何也?」惠卿不能對,則以他語
 詆光。帝曰:「相與論是非耳,何至是。」光曰:「平民舉錢出息,尚能蠶食下戶,況懸官督責之威乎!」惠卿曰:「青苗法,願取則與之,不願不強也。」光曰:「愚民知取債之利,不知還債之害,非獨縣官不強,富民亦不強也。昔太宗平河東,立糴法,時米斗十錢,民樂與官為市。其後物貴而和糴不解,遂為河東世世患。臣恐異日之青苗,亦猶是也。」帝曰:「坐倉糴米何如?」坐者皆起,光曰:「不便。」惠卿曰:「糴米百萬斛,則省東南之漕,以其錢供京師。」光曰:「東南錢荒而
 粒米狼戾,今不糴米而漕錢,棄其有餘,取其所無,農末皆病矣!」侍講吳申起曰:「光言,至論也。」



 它日留對,帝曰:「今天下洶洶者,孫叔敖所謂『國之有是,眾之所惡』也。」光曰:「然。陛下當論其是非。今條例司所為,獨安石、韓絳、惠卿以為是耳,陛下豈能獨與此三人共為天下邪?」帝欲用光,訪之安石。安石曰:「光外托劘上之名,內懷附之下實。所言盡害政之事,所與盡害政之人,而欲置之左右,使與國論,此消長之大機也。光才豈能害政,但在高位,則
 異論之人倚以為重。韓信立漢赤幟,趙卒氣奪,今用光,是與異論者立赤幟也。」



 安石以韓琦上疏,臥家求退。帝乃拜光樞密副使,光辭之曰:「陛下所以用臣,蓋察其狂直,庶有補於國家。若徒以祿位榮之,而不取其言,是以天官私非其人也。臣徒以祿位自榮,而不能救生民之患,是盜竊名器以私其身也。陛下誠能罷制置條例司,追還提舉官,不行青苗、助役等法,雖不用臣,臣受賜多矣。今言青苗之害者,不過謂使者騷動州縣,為今日之
 患耳。而臣之所憂,乃在十年之外,非今日也。夫民之貧富,由勤惰不同,惰者常乏,故必資於人。今出錢貸民而斂其息,富者不願取,使者以多散為功,一切抑配。恐其逋負,必令貧富相保,貧者無可償,則散而之四方;富者不能去,必責使代償數家之負。春算秋計,展轉日滋,貧者既盡,富者亦貧。十年之外,百姓無復存者矣。又盡散常平錢穀,專行青苗,它日若思復之,將何所取?富室既盡,常平已廢,加之以師旅,因之以饑饉,民之羸者必委
 死溝壑,壯者必聚而為盜賊,此事之必至者也。」抗章至七八,帝使謂曰:「樞密,兵事也,官各有職,不當以他事為辭。」對曰:「臣未受命,則猶侍從也,於事無不可言者。」安石起視事,光乃得請,遂求去。



 以端明殿學士知永興軍。宣撫使下令分義勇戍邊,選諸軍驍勇士,募市井惡少年為奇兵;調民造乾□,悉修城池樓櫓,關輔騷然。光極言:「公私困敝,不可舉事,而京兆一路皆內郡,繕治非急。宣撫之令,皆未敢從,若乏軍興,臣當任其責。」於是一路獨
 得免。徙知許州,趣入覲,不赴;請判西京御史臺歸洛,自是絕口不論事。而求言詔下,光讀之感泣,欲嘿不忍,乃復陳六事,又移書責宰相吳充,事見《充傳》。



 葵天申為察訪,妄作威福,河南尹、轉運使敬事之如上官;嘗朝謁應天院神御殿,府獨為設一班,示不敢與抗。光顧謂臺吏曰:「引蔡寺丞歸本班。」吏即引天申立監竹木務官富贊善之下。天申窘沮,即日行。



 元豐五年,忽得語澀疾,疑且死,豫作遺表置臥內,即有緩急,當以畀所善者上之。官
 制行,帝指御史大夫曰:「非司馬光不可。」又將以為東宮師傅。蔡確曰:「國是方定,願少遲之。」《資治通鑒》未就,帝尤重之,以為賢於荀悅《漢紀》,數促使終篇,賜以穎邸舊書二千四百卷。及書成,加資政殿學士。凡居洛陽十五年,天下以為真宰相,田夫野老皆號為司馬相公,婦人孺子亦知其為君實也。



 帝崩,赴闕臨,衛士望見,皆以手加額曰:「此司馬相公也。」所至,民遮道聚觀,馬至不得行,曰:「公無歸洛,留相天子,活百姓。」哲宗幼沖,太皇太后臨政,
 遣使問所當先,光謂:「開言路。」詔榜朝堂。而大臣有不悅者,設六語云:「若陰有所懷;犯非其分;或扇搖機事之重;或迎合已行之令;上以徼幸希進;下以眩惑流俗。若此者,罰無赦。」後復命示光,光曰:「此非求諫,乃拒諫也。人臣惟不言,言則入六事矣。」乃具論其情,改詔行之,於是上封者以千數。



 起光知陳州,過闕,留為門下侍郎。蘇軾自登州召還,緣道人相聚號呼曰:「寄謝司馬相公,毋去朝廷,厚自愛以活我。」是時天下之民,引領拭目以觀新政,
 而議者猶謂「三年無改於父之道」,但毛舉細事,稍塞人言。光曰:「先帝之法,其善者雖百世不可變也。若安石、惠卿所建,為天下害者,改之當如救焚拯溺。況太皇太后以母改子,非子改父。」眾議甫定。遂罷保甲團教,不復置保馬;廢市易法,所儲物皆鬻之,不取息,除民所欠錢;京東鐵錢及茶鹽之法,皆復其舊。或謂光曰:「熙、豐舊臣,多憸巧小人,他日有以父子義間上,則禍作矣。」光正色曰:「天若祚宗社,必無此事。」於是天下釋然,曰:「此先帝本意
 也。」



 元祐元年復得疾,詔朝會再拜,勿舞蹈。時青苗、免役、將官之法猶在,而西戎之議未決。光嘆曰:「四患未除,吾死不瞑目矣。」折簡與呂公著云:「光以身付醫,以家事付愚子,惟國事未有所托,今以屬公。」乃論免役五害,乞直降敕罷之。諸將兵皆隸州縣,軍政委守令通決。廢提舉常平司,以其事歸之轉運、提點刑獄。邊計以和戎為便。謂監司多新進少年,務為刻急,令近臣於郡守中選舉,而於通判中舉轉運判官。又立十科薦士法。皆從之。



 拜
 尚書左僕射兼門下侍郎,免朝覲,許乘肩輿,三日一入省。光不敢當,曰:「不見君,不可以視事。」詔令子康扶入對,且曰:「毋拜。」遂罷青苗錢,復常平糶糴法。兩宮虛己以聽。遼、夏使至,必問光起居,敕其邊吏曰:「中國相司馬矣,毋輕生事、開邊隙。」光自見言行計從,欲以身徇社稷,躬親庶務,不舍晝夜。賓客見其體羸,舉諸葛亮食少事煩以為戒,光曰:「死生,命也。」為之益力。病革,不復自覺,諄諄如夢中語,然皆朝廷天下事也。



 是年九月薨,年六十八。太
 皇太后聞之慟,與帝即臨其喪,明堂禮成不賀,贈太師、溫國公,襚以一品禮服,賻銀絹七千。詔戶部侍郎趙瞻、內侍省押班馮宗道護其喪,歸葬陜州。謚曰文正,賜碑曰「忠清粹德」。京師人罷市往吊,鬻衣以致奠,巷哭以過車。及葬,哭者如哭其私親。嶺南封州父老,亦相率具祭,都中及四方皆畫像以祀,飲食必祝。



 光孝友忠信,恭儉正直,居處有法,動作有禮。在洛時,每往夏縣展墓,必過其兄旦,旦年將八十,奉之如嚴父,保之如嬰兒。自少至
 老,語未嘗妄,自言:「吾無過人者,但平生所為,未嘗有不可對人言者耳。」誠心自然,天下敬信,陜、洛間皆化其德,有不善,曰:「君實得無知之乎?」



 光於物澹然無所好,於學無所不通,惟不喜釋、老,曰:「其微言不能出吾書,其誕吾不信也。」洛中有田三頃,喪妻,賣田以葬,惡衣菲食以終其身。



 紹聖初,御史周秩首論光誣謗先帝,盡廢其法。章惇、蔡卞請發塚斫棺,帝不許,乃令奪贈謚,僕所立碑。而惇言不已,追貶清遠軍節度副使,又貶崖州司戶參軍。
 徽宗立,復太子太保。蔡京擅政,復降正議大夫,京撰《奸黨碑》,令郡國皆刻石。長安石工安民當鐫字,辭曰:「民愚人,固不知立碑之意。但如司馬相公者,海內稱其正直,今謂之奸邪,民不忍刻也。」府官怒,欲加罪,泣曰:「被役不敢辭,乞免鐫安民二字於石末,恐得罪於後世。」聞者愧之。



 靖康元年,還贈謚。建炎中,配饗哲宗廟庭。



 康字公休,幼端謹,不妄言笑,事父母至孝。敏學過人,博通群書,以明經上第。光修《資治通鑒》,奏檢閱文字。丁母
 憂,勺飲不入口三日,毀幾滅性。光居洛,士之從學者退與康語,未嘗不有得。塗之人見其容止,雖不識,皆知其為司馬氏子也。以韓絳薦,為秘書,由正字遷校書郎。光薨,治喪皆用《禮經》家法,不為世俗事。得遺恩,悉以與族人。服除,召為著作佐郎兼侍講。



 上疏言:「比年以來,旱□為虐,民多艱食。若復一不稔,則公私困竭,盜賊可乘。自古聖賢之君,非無水旱,惟有以待之,則不為甚害。願及今秋熟,令州縣廣糴,民食所餘,悉歸於官。今冬來春,令
 流民就食,候鄉里豐穰,乃還本土。凡為國者,一絲一毫皆當愛惜,惟於濟民則不宜吝。誠能捐數十萬金帛,以為天下大本,則天下幸甚。」拜右正言,以親嫌未就職。



 為哲宗言前世治少亂多,祖宗創業之艱難,積累之勤勞,勸帝及時向學,守天下大器,且勸太皇太后每於禁中訓迪,其言切至。邇英進講,又言:「《孟子》於書最醇正,陳王道尤明白,所宜觀覽。」帝曰:「方讀其書」。尋詔講官節以進。



 康自居父喪,居廬疏食,寢於地,遂得腹疾,至是不能朝
 謁。賜優告。疾且殆,猶具疏所當言者以待,曰:「得一見天子極言而死無恨。」使召醫李積於兗。積老矣,鄉民聞之,往告曰:「百姓受司馬公恩深,今其子病,願速往也。」來者日夜不絕,積遂行;至,則不可為矣。年四十一而卒。公卿嗟痛於朝,士大夫相吊於家,市井之人,無不哀之。詔贈右諫議大夫。



 康為人廉潔,口不言財。初,光立神道碑,帝遣使賜白金二千兩,康以費皆官給,辭不受。不聽。遣家吏如京師納之,乃止。



 論曰:熙寧新法病民,海內騷動,忠言讜論,沮抑不行;正人端士,擯棄不用。聚斂之臣日進,民被其虐者將二十年。方是時,光退居於洛,若將終身焉。而世之賢人君子,以及庸夫愚婦,日夕引領望其為相,至或號呼道路,願其毋去朝廷,是豈以區區材智所能得此於人人哉?德之盛而誠之著也。



 一旦起而為政,毅然以天下自任,開言路,進賢才。凡新法之為民害者,次第取而更張之,不數月之間,鏟革略盡。海內之民,如寒極而春,旱極而雨,
 如解倒懸,如脫桎梏,如出之水火之中也。相與咨嗟嘆息,歡欣鼓舞,甚若更生,一變而為嘉祐、治平之治。君子稱其有旋乾轉坤之功,而光於是亦老且病矣。天若祚宋,ME遺一老,則奸邪之勢未遽張,紹述之說未遽行,元祐之臣固無恙也。人眾能勝天,靖康之變,或者其可少緩乎?借曰有之,當不至如是其酷也。《詩》曰:「哲人云亡,邦國殄瘁。」嗚呼悲夫!



 康濟美象賢,不幸短命而死,世尤惜之。然康不死,亦將不免於紹聖之禍矣。



 呂公著,字晦叔,幼嗜學,至忘寢食。父夷簡器異之,曰:「他日必為公輔。」恩補奉禮郎,登進士第,召試館職,不就。通判穎州,郡守歐陽修與為講學之友。後修使契丹,契丹主問中國學行之士,首以公著對。判吏部南曹,仁宗獎其恬退,賜五品服。除崇文院檢討、同判太常寺。壽星觀營真宗神御殿,公著言:「先帝已有三種御,而建立不已,殆非祀無豐暱之義。」進知制誥,三辭不拜。改天章閣待制兼侍讀。



 英宗親政,加龍圖閣直學士。方議追崇濮王,
 或欲稱皇伯考,公著曰:「此真宗所以稱太祖,豈可施於王。」及下詔稱親,且班諱,又言:「稱親則有二父之嫌,王諱但可避於上前,不應與七廟同諱。」呂誨等坐論濮王去,公著言:「陛下即位以來,納諫之風未彰,而屢絀言者,何以風示天下?」不聽。遂乞補外,帝曰:「學士朕所重,其可以去朝廷?」請不已,出知蔡州。



 神宗立,召為翰林學士、知通進銀臺司。司馬光以論事罷中丞,還經幄。公著封還其命曰:「光以舉職賜罷,是為有言責者不得盡其言也。」詔
 以告直付閣門。公著又言:「制命不由門下,則封駁之職,因臣而廢。願理臣之罪,以正紀綱。」帝諭之曰:「所以徙光者,賴其勸學耳,非以言事故也。」公著請不已,竟解銀臺司。



 熙寧初,知開封府。時夏秋淫雨,京師地震。公著上疏曰:「自昔人君遇災者,或恐懼以致福,或簡誣以致禍。上以至誠待下,則下思盡誠以應之,上下至誠而變異不消者,未之有也。惟君人者去偏聽獨任之弊,而不主先入之語,則不為邪說所亂。顏淵問為邦,孔子以遠佞人
 為戒。蓋佞人惟恐不合於君,則其勢易親;正人惟恐不合於義,則其勢易疏。惟先格王正厥事,未有事正而世不治者也。」禮官用唐故事,請以五月御大慶殿受朝,因上尊號。公著曰:「陛下方度越漢、唐,追復三代,何必於陰長之日,為非禮之會,受無益之名?」從之。



 二年,為御史中丞。時王安石方行青苗法,公著極言曰:「自古有為之君,未有失人心而能圖治,亦未有能脅之以威、勝之以辯而能得人心者也。昔日之所謂賢者,今皆以此舉為非,
 而生議者一切祗為流俗浮論,豈昔皆賢而今皆不肖乎?」安石怒其深切。帝使舉呂惠卿為御史,公著曰:「惠卿固有才,然奸邪不可用。」帝以語安石,安石益怒,誣以惡語,出知穎州。



 八年,彗星見,詔求直言。公著上疏曰:「陛下臨朝願治,為日已久,而左右前後,莫敢正言。使陛下有欲治之心,而無致治之實,此任事之臣負陛下也。夫士之邪正、賢不肖,既素定矣。今則不然,前日所舉,以為天下之至賢;而後日逐之,以為天下至不肖。其於人材既
 反復不常,則於政事亦乖戾不審矣。古之為政,初不信於民者有之,若子產治鄭,一年而人怨之,三年而人歌之。陛下垂拱仰成,七年於此,然輿人之誦,亦未有異於前日,陛下獨不察乎?」



 起知河陽,召還,提舉中太一宮,遷翰林學士承旨,改端明殿學士、知審官院。帝從容與論治道,遂及釋、老,公著問曰:「堯、舜知此道乎?」帝曰:「堯、舜豈不知?」公著曰:「堯、舜雖如此,而惟以知人安民為難,所以為堯、舜也。」帝又言唐太宗能以權智御臣下。對曰:「太宗
 之德,以能屈己從諫爾。」帝善其言。



 未幾,同知樞密院事。有欲復肉刑者,議取死囚試劓、刖,公著曰:「試之不死,則肉刑遂行矣。」乃止。夏人幽其主,將大舉討之。公著曰:「問罪之師,當先擇帥,茍未得人,不如勿舉。」及兵興,秦、晉民力大困,大臣不敢言,公著數白其害。



 元豐五年,以疾丐去位,除資政殿學士、定州安撫使。俄永樂城陷,帝臨朝嘆曰:「邊民疲弊如此,獨呂公著為朕言之耳。」徙揚州,加大學士。將立太子,帝謂輔臣,當以呂公著、司馬光為師
 傅。



 哲宗即位,以侍讀還朝。太皇太后遣使迎,問所欲言,公著曰:「先帝本意,以寬省民力為先。而建議者以變法侵民為務,與己異者一切斥去,故日久而弊愈深,法行而民愈困。誠得中正之士,講求天下利病,協力而為之,宜不難矣。」至則上言曰:「人君初即位,當正始以示天下,修德以安百姓。修德之要,莫先於學。學有緝熙於光明,則日新以底至治者,學之力也。謹昧死陳十事,曰畏天、愛民、修身、講學、任賢、納諫、薄斂、省刑、去奢、無逸。」又乞備
 置諫員,以開言路。拜尚書左丞、門下侍郎。



 元祐元年,拜尚書右僕射兼中書侍郎。三省並建,中書獨為取旨之地。乃請事於三省者,與執政同進呈,取旨而各行之。又執政官率數日一聚政事堂,事多決於其長,同列莫得預。至是,始命日集,遂為定制。與司馬光同心輔政,推本先帝之志,凡欲革而未暇與革而未定者,一一舉行之。民歡呼鼓舞,咸以為便。光薨,獨當國,除吏皆一時之選。時科舉罷詞賦,專用王安石經義,且雜以釋氏之說。凡士
 子自一語上,非新義不得用,學者至不誦正經,唯竊安石之書以干進,精熟者轉上第,故科舉益弊。公著始令禁主司不得出題老、莊書,舉子不得以申、韓、佛書為學,經義參用古今諸儒說,毋得專取王氏。復賢良方正科。



 右司諫賈易以言事訐直詆大臣,將峻責,公著以為言,止罷知懷州。退謂同列曰:「諫官所論,得失未足言。顧主上春秋方盛,慮異明有進諛說惑亂者,正賴左右爭臣耳,不可豫使人主輕厭言者也。」眾莫不嘆服。



 吐蕃首領
 鬼章青宜結久為洮、河患,聞朝廷弭兵省戍,陰與夏人合謀復取熙、岷。公著白遣軍器丞游師雄以便宜諭諸將,不逾月,生致於闕下。



 帝宴近臣於資善堂,出所書唐人詩分賜。公著乃集所講書要語明白、切於治道者,凡百篇進之,以備游意翰墨,為聖學之助。



 三年四月,懇辭位,拜司空、同平章軍國事。宋興以來,宰相以三公平章重事者四人,而公著與父居其二,士艷其榮。詔建第於東府之南,啟北扉,以便執政會議。凡三省、樞密院之職,
 皆得總理。間日一朝,因至都堂,其出不以時,蓋異禮也。



 明年二月薨,年七十二。太皇太后見輔臣泣曰:「邦國不幸,司馬相公既亡,呂司空復逝。」痛閔久之。帝亦悲感,即詣其家臨奠,賜金帛萬。贈太師、申國公,謚曰正獻,御筆碑首曰「純誠厚德」。



 公著自少講學,即以治心養性為本,平居無疾言遽色,於聲利紛華,泊然無所好。暑不揮扇,寒不親火,簡重清靜,蓋天稟然。其識慮深敏,量閎而學粹,遇事善決,茍便於國,不以私利害動其心。與人交,出
 於至誠,好德樂善,見士大夫以人物為意者,必問其所知與其所聞,參互考實,以達於上。每議政事,博取眾善以為善,至所當守,則毅然不回奪。神宗嘗言其於人材不欺,如權衡之稱物。尤能避遠聲跡,不以知人自處。



 始與王安石善,安石兄事之,安石博辯騁辭,人莫敢與亢,公著獨以精識約言服之。安石嘗曰:「疵吝每不自勝,一詣長者,即廢然而反,所謂使人之意消者,於晦叔見之。」又謂人曰:「晦叔為相,吾輩可以言仕矣。」後安石得志,意
 其必助己,而數用公議,列其過失,以故交情不終。於講說尤精,語約而理盡。司馬光曰:「每聞晦叔講,便覺己語為煩。」其為名流所敬如此。



 紹聖元年,章惇為相,以翟思、張商英、周秩居言路,論公著更熙、豐法度,削贈謚,毀所賜碑,再貶建武軍節度副使、昌化軍司戶參軍。徽宗立,追復太子太保。蔡京擅政,復降左光祿大夫,入黨籍,尋復銀青光祿大夫。紹興初,悉還贈謚。子希哲、希純。



 希哲字原明,少從焦千之、孫復、石介、胡瑗學,復從程顥、
 程頤、張載游,聞見由是益廣。以蔭入官,父友王安石勸其勿事科舉,以僥幸利祿,遂絕意進取。安石為政,將置其子雱於講官,以希哲有賢名,欲先用之。希哲辭曰:「辱公相知久,萬一從仕,將不免異同,則疇昔相與之意盡矣。」安石乃止。



 公著作相,二弟已官省寺,希哲獨滯管庫,久乃判登聞鼓院,力辭。公著嘆曰:「當世善士,吾收拾略盡,爾獨以吾故置不試,命也夫!」希哲母賢明有法度,聞公著言,笑曰:「是亦未知其子矣。」



 終公著喪,始為兵部員
 外郎。範祖禹,其妹婿也,言於哲宗曰:「希哲經術操行,宜備勸講,其父常稱為不欺暗室。臣以婦兄之故,不敢稱薦,今方將引去,竊謂無嫌。」詔以為崇政殿說書。其勸導人主以修身為本,修身以正心誠意為主。其言曰:「心正意誠,則身修而天下化。若身不能修,雖左右之人且不能諭,況天下乎?」



 擢右司諫,辭,未聽,私語祖禹曰:「若不得請,當以楊畏、來之邵為首。」既而不拜。會紹聖黨論起,御史劉拯論其進不由科第,以秘閣校理知懷州。中書舍
 人林希又言:「呂大防由公著援引,故進希哲以酬私恩。凡大防輩欺君賣國,皆公著為之倡;而公著之惡,則希哲導成之,豈宜污華職。」於是但守本秩,俄分司南京,居和州。



 徽宗初,召為秘書少監,或以為太峻,改光祿少卿。希哲力請外,以直秘閣知曹州。旋遭崇寧黨禍,奪職知相州,徙邢州。罷為宮祠。羈寓淮、泗間,十餘年卒。



 希哲樂易簡儉,有至行,晚年名益重,遠近皆師尊之。子好問,有傳。



 希純字子進,登第,為太常博士。元祐祀明堂,將用皇祐故事,並饗天地百神,皆以祖宗配。希純言:「皇祐之禮,事不經見,嘉祐既已厘正。至元豐中,但以英宗配上帝,悉罷從祀群神,得嚴父之義,請循其式。」從之。



 歷宗正、太常、秘書丞。哲宗議納後,希純請考三代昏禮,參祖宗之制,博訪令族,參求德配。凡世俗所謂勘婚之書,淺陋不經,且一切屏絕,以防附會。遷著作郎,以父諱不拜。擢起居舍人,權太常少卿。



 宣仁太后崩,希純慮奸人乘間進說
 搖主聽,即上疏曰:「自元祐初年,太皇聽斷,所用之人皆宿有時望,所行之事皆人所願行。唯是過惡得罪之徒,日伺變故,捭闔規利,今必以更改神宗法度為說。臣以為先帝之功烈,萬世莫掩。間有數事,為小人所誤,勢雖頗有損益,在於聖德,固無所虧。且英宗、神宗何嘗不改真宗、仁宗之政,亦豈盡用慶祖、太宗之法乎?小人既誤先帝,復欲誤陛下,不可不察。」未幾,拜中書舍人、同修國史。



 內侍梁從政、劉惟簡除內省押班,希純以親政之始,
 首錄二人,無以示天下,持不行。由是閹寺側目,或於庭中指以相示曰:「此繳還二押班詞頭者也。」



 章惇既相,出為寶文閣待制、知亳州。諫官張商英憾希純,攻之力。又以外親嫌,連徙睦州、歸州。自京東而之浙西,自浙西而上三峽,名為易地,實困之也。公著追貶,希純亦以屯田員外郎分司南京,居金州。又責舒州團練副使,道州安置。建中靖國元年,還為待制、知瀛州。徽宗聞其名,數稱之。曾布忌希純,因其請覲,未及見,亟以邊,遽趣遣之。俄
 改穎州,入崇寧黨籍。卒,年六十。



 論曰:公著父子俱位至宰相,俱以司空平章軍國事,雖漢之韋、平,唐之蘇、李,榮盛孰加焉。夷簡多智數,公著則一切持正,以應天下之務,嗚呼賢哉。其論人才,如權衡之稱物,故一時賢士,收拾略盡。司馬光疾甚,諄諄焉以國事為托,當時廷臣,莫公著若也審矣。追考其平生事業,蓋守成之良相也。然知子之賢而不能薦,殆猶未免於避嫌,而有愧於從祖云。希哲、希純世濟其美,然皆隱
 於崇寧黨禍,何君子之不幸歟!



\end{pinyinscope}