\article{列傳第一百四十一}

\begin{pinyinscope}

 張燾黃中孫道夫曾幾兄開勾濤李彌遜弟彌大



 張燾,字子公,饒之德興人,秘閣修撰根之子也。政和八年進士第三人,嘗為闢雍錄、秘書省正字。靖康元年,李
 綱為親征行營使,闢燾入幕。綱貶,親知坐累者十七人,燾亦貶。



 建炎初,起通判湖州。明受之變,賊矯詔俾燾撫諭江、浙,燾不受。上既復闢,詔求言。燾上書略曰:「人主戡定禍亂,未有不本於至誠而能有濟者。陛下踐祚以來,號令之發未足以感人心,政事之施未足以慰人望,豈非在我之誠有未修乎?天下治亂,在君子小人用舍而已。小人之黨日勝,則君子之類日退,將何以弭亂而圖治?」又言措置江防非計,徒費民財、損官賦,不適於用。又
 言:「侍從、臺諫觀望意指,毛舉細務,至國家大事,坐視不言。」又言:「巡幸所至,營繕困民,越棲會稽,似不如是。」



 紹興二年,呂頤浩薦,除司勛員外郎,遷起居舍人。言:「自古未有不知敵人之情而能勝者,願詔大臣、諸將,厚爵賞,募可任用者往伺敵動靜。既審知之,則戰守進退,在我皆備,彼尚安得出不意犯吾行闕。」詔以付都督府及沿邊諸帥。遷中書舍人。



 呂祉之撫諭淮西也,燾謂張浚曰:「祉書生,不更軍旅,何可輕付。」浚不從,遂致酈瓊之變。七年,
 張滉特賜進士出身。滉,浚兄也,將母至行在,上引對而命之。燾言:「宣和以來,奸臣子弟濫得儒科。陛下方與浚圖回大業,當以公道革前弊。今首賜滉第,何以塞公議?」上念浚功,欲慰其母心,乃命起居郎樓照行下,照又封還。著作郎兼起居舍人何掄曰:「賢良之子,宰相之兄,賜科第不為過。」乃與書行。燾不自安,與照皆求去,不許,言者論之,以集英殿修撰提舉江州太平觀。



 明年,以兵部侍郎召,詔引對,上曰:「卿去止緣張滉。」燾曰:「臣茍有所見,
 不敢不言。如內侍王鑒,陛下所親信,臣尚論列,豈有宰相親兄自賜出身,公論不與。臣若不言,豈惟負陛下,亦負張浚。」上因問:「朕圖治一紀,收效蔑然,其弊安在?」燾曰:「自昔有為之君,未有不先定規模而能收效者,臣紹興初首以是為言,今七年。往者進臨大江,退守吳會,未期月而或進或卻,豈不為敵所窺乎?今陛下相與斷國論者,二三大臣而已。一紀之間,十四命相,執政遞遷無慮二十餘。日月逝矣,大計不容復誤,願以先定規模為急。」



 尋權吏部尚書。徽猷閣待制黎確卒,詔贈官推恩,燾言:「確素號正人,一旦臨變,失臣節,北面邦昌之庭,且為將命止勤王之師。今曲加贈恤,何以示天下?」詔追奪職名。



 時金使至境,詔欲屈己就和,令侍從、臺諫條上。燾言:「金使之來,欲議和好,將歸我梓宮,歸我淵聖,歸我母後,歸我宗社,歸我土地人民,其意甚美,其言甚甘,廟堂以為信然,而群臣、國人未敢以為信然也。蓋事關國體,臣請推原天意為陛下陳之。《傳》曰:『天將興之,誰能廢之?』臣考
 人事以驗天意,陛下飛龍濟州,天所命也。敵騎屢犯行闕,不能為虞。甲寅一戰敗敵師,丙辰再戰卻劉豫,丁巳酈瓊雖叛,實為偽齊廢滅之資,皆天所贊也。是蓋陛下躬履艱難,側身修行,布德立正,上副天意,而天祐之之所致也。臣以是知上天悔禍有期,中興不遠矣。願益自修自強,以享天心,以俟天時。時之既至,吉無不利,則何戰不勝,何功不立。今此和議,姑為聽之,而必無信之可恃也。彼使已及境,勢難固拒。使其果願和好,如前所陳,
 是天誘其衷,必不復強我以難行之禮。如其初無此心,二三其說,責我以必不可行之禮,要我以必不可從之事,其包藏何所不有,便當以大義絕之。謹邊防,厲將士,相時而動。願斷自淵衷,毋取必於彼而取必於天而已。乃若略國家之大恥,置宗社之深仇,躬率臣民,屈膝於金而臣事之,而凱和議之必成,非臣所敢知也。」上覽奏,愀然變色曰:「卿言可謂忠,然朕必不至為彼所紿,方且熟議,必非詐偽而後可從,不然,當再使審虛實,拘其使
 人。」燾頓首謝。



 金使張通古、蕭哲至行在,朝議欲上拜金詔。燾曰:「陛下信王倫之虛詐,發自聖斷,不復謀議,便欲行禮,群臣震懼罔措。必已得梓宮,已得母後,已得宗族,始可議通好經久之禮。今彼特以通好為說,意謂割地講和而已,陛下之所願欲而切於聖心者,無一言及之,其情可見,奈何遽欲屈而聽之。一屈之後,不可復伸,廷臣莫能正救,曾魯仲連之不如,豈不獲罪於天下萬世。」



 既而監察御史施廷臣抗章力贊和議,擢為侍御史。司
 農寺丞莫將忽賜第,擢為起居郎。朝論大駭。燾率吏部侍郎晏敦復上疏曰:「仰惟陛下痛梓宮未還,兩宮未復,不憚屈己與敵議和,特以眾論未同,故未敢輕屈爾。幸小大之臣,無復異議,從容獻納,庶幾天聽為回,卒不敢屈,此宗社之福也。彼施廷臣乃務迎合,輒敢抗章,力贊此議,姑為一身進用之資,不恤君父屈辱之恥,罪不容誅,乃由察官超擢柱史。夫御史府朝廷紀綱之地,而陛下耳目之司,前日勾龍如淵以附會而得中丞,眾論固
 已喧鄙之矣。今廷臣又以此而躋橫榻,一臺之中,長貳皆然,既同鄉曲,又同心腹,惟相朋附,變亂是非,豈不紊紀綱而蔽陛下之耳目乎?眾論沸騰,方且切齒,而莫將者又以此議由寺丞擢右史。如淵、廷臣庸人也,初無所長,但知觀望,而將則奸人也,考其平昔無所不為,此輩烏可與之斷國論乎?望加斥逐,庶幾少杜群枉之門。至於和議,則王倫實為謀主,彼往來敵中至再四矣,陛下恃以為心腹,信之如蓍龜,今其為言自己二三,事之端
 倪,蓋亦可見。更望仰念祖宗付托之重,俯念億兆愛戴之誠,貴重此身,無輕於屈。但務雪恥以思復仇,加禮其使,厚資遣發,諭以必得事實之意,告以國人皆曰不可之狀。使彼悔禍,果出誠心,惟我所欲,盡歸於我,然後徐議報之之禮,亦未晚也。如其變詐,誘我以虛詞,則包藏終不可測,便當厲將士,保疆埸,自治自強,以俟天時,何為不成?伏願陛下少忍而已。自朝廷有屈己之議,上下解體,儻遂成屈己之事,則上下必至離心,人心既離,何
 以立國?伏願戒之重之。」於是將、廷臣皆不敢拜。燾又面折如淵曰:「達觀其所舉,君薦七人,皆北面張邦昌,今囁嚅附會,墮敵計,他日必背君親矣。」



 燾既力詆拜詔之議,秦檜患之,燾亦自知得罪,托疾在告。檜使樓照諭之曰:「北扉闕人,欲以公為直院。」燾大駭曰:「果有此言,愈不敢出矣。」檜不能奪,乃止。



 和議成,範如圭請遣使朝八陵,遂命判大宗正士人褭與燾偕行,且命修奉,令荊湖帥臣岳飛濟其役。燾與士人褭道武昌,出蔡、穎,河南百姓歡迎夾
 道,以喜以泣曰:「久隔王化,不圖今日復為宋民。」九年五月,至永安諸陵,朝謁如禮。陵前石澗水久涸,二使垂至忽湧溢,父老驚嘆,以為中興之兆。



 燾等入柏城,披鉏荊棘,隨所葺治,留二日而還,自鄭州歷汴、宋、宿、泗、淮南以歸。即奏疏曰:「金人之禍,上及山陵,雖殄滅之,未足以雪此恥、復此仇也。陛下聖孝天至,豈勝痛憤,顧以梓宮、兩宮之故,方且與和,未可遽言兵也。祖宗在天之靈,震怒既久,豈容但已,異時恭行天罰,得無望於陛下乎?自古
 戡定禍亂,非武不可,狼子野心不可保恃久矣;伏望修武備,俟釁隙起而應之,電掃風驅,盡俘醜類以告諸陵。夫如是然後盡天子之孝,而為人子孫之責塞矣。」上問諸陵寢如何?燾不對,唯言「萬世不可忘此賊。」上黯然。



 燾因請永固陵不用金玉,大略謂:「金玉珍寶,聚而藏之,固足以動人耳目,又其為物,自當流布於世,理必發露,無足怪者。」上覽疏,謂秦檜曰:「前世厚葬之禍,如循一軌。朕斷不用金玉,庶先帝神靈有萬世之安。」燾又言:「頃劉豫
 初廢,人情恟恟,我斥候不明,坐失機會。今又聞敵於淮陽作筏、造繩索,不知安用?諸將朝廷戒勿得遣間探,遂不復遣,我之動息,敵無不知,敵之情狀,我則不聞。又見黃河船盡拘北岸,悉為敵用,往來自若,無一人敢北渡者。願飭邊吏廣耳目,先事而防。」又言:「酈瓊部伍皆西陲勁兵,今在河南,尚可收用。新疆租賦已蠲,而使命絡繹,推恩費用猶循兵興時例,願加裁損,非甚不得已勿遣使,以寬民力。」又論:「陜西諸帥不相下,動輒喧爭,請置一
 大帥統之,庶首尾相應,緩急可恃。」燾所言皆切中時病,秦檜方主和,惟恐少忤敵意,悉置不問。



 成都謀帥,上諭檜曰:「張燾可,第道遠,恐其憚行。」檜以諭燾,燾曰:「君命也,焉敢辭。」十月,以寶文閣學士知成都府兼本路安撫使,付以便宜,雖安撫一路,而四川賦斂無藝者,悉得蠲減。陛辭,奏曰:「蜀民困矣,官吏從而誅剝之,去朝廷遠,無所赴愬。俟臣至所部,首宣德意,但一路咸沾惠澤。」上曰:「豈惟一路,四川恤民事悉委卿。」燾因言官吏害民者,請先
 罷後劾,上許之。又言:「軍興十餘年,日不暇給。今和議甫定,願汲汲以政刑為先務。」上曰:「當書之座右。」十年三月,至成都。



 在蜀四年,戢貪吏,薄租賦;撫雅州蕃部,西邊不驚;歲旱則發粟,民得不饑;暇則修學校,與諸生講論。會有詔令宣撫司納契丹降人,燾為宣撫使胡世將言:「蜀地狹不能容,前朝常勝軍可為戒。」世將奏寢其事。



 燾乞祠,以李璆代之。燾自蜀歸,臥家凡十有三年。二十五年冬,檜死,舊人在者皆起,燾除知建康府兼行宮留守。金
 陵積歲負內庫錢帛鉅萬,悉為奏免。池有義子與父爭訟,守昏謬,系父,連年不決,燾移大理,斥其守。居二年,進端明殿學士。二十九年,提舉萬壽觀兼侍讀,以衰疾力辭,不許。除吏部尚書。



 初,上知普安郡王賢,欲建為嗣,顯仁皇后意未欲,遲回久之。顯仁崩,上問燾方今大計,燾曰:「儲貳者,國之本也,天下大計,無逾於此。」上曰:「朕懷此久矣,卿言契朕心,開春當議典禮。」又勸上省賜予,罷土木,減冗吏,止北貨。上嘉獎之。



 金使施宜生來,燾奉詔館
 客。宜生本閩人,素聞燾名,一見顧副使曰:「是南朝不拜詔者。」燾以「首丘桑梓」動之,宜生於是漏敵情,燾密奏早為備。



 先是,御前置甲庫,凡乘輿所需圖畫什物,有司不能供者悉聚焉。日費不貲。禁中既有內酒庫,釀殊勝,酤賣其餘,頗侵大農。燾因對,言甲庫萃工巧以蕩上心,酒庫酤良醞以奪官課。且乞罷減教坊樂工人數。上曰:「卿言可謂責難於君。」明日悉詔罷之。



 屢以衰疾乞骸。三十年,以資政殿學士致仕,尋遷太中大夫,給真奉。三十一
 年八月,落致仕,復知建康府。時金人窺江,建業民驚徙過半,聞燾至,人情稍安。尋詔沿江帥臣條上恢復事宜,燾首陳十事,大率欲預備不虞,持重養威,觀釁而動,期於必勝。



 孝宗受禪,除同知樞密院,遣子埏入辭。詔肩輿至宮,給扶上殿,首問為治之要,言內治乃可外攘。又乞命百執條弊事,詔從之,令侍從、臺諫集都堂給札以聞。隆興元年,遷參知政事,以老病不拜,臺諫交章留之,除資政殿大學士、提舉萬壽觀兼侍讀。謁告將理,許之。及
 家,固求致仕。後二年卒,年七十五,謚忠定。



 燾外和內剛,帥蜀有惠政,民祠之不忘。始論和議,歸之於天,士論歉然。洎繳駁施廷臣之奏,朝野復一辭歸重焉。



 黃中,字通老,邵武人。幼受書,一再輒成誦。初以族祖蔭補官。紹興五年廷試,言孝弟動上心,擢進士第二人,授保寧軍節度推官。二十餘年,秦檜死,乃召為校書郎,歷遷普安、恩平府教授。中在王府時,龍大淵已親幸,中未嘗與之狎,見則揖而退,後他教授多蒙其力,中獨不徙
 官。



 遷司封員外郎兼國子司業。芝草生武成廟,官吏請以聞,中不答,官吏陰畫圖以獻。宰相謂祭酒周綰與中曰:「治世之瑞,抑而不奏,何耶?」綰未對,中曰:「治世何用此為?」綰退,謂人曰:「黃司業之言精切簡當,惜不為諫官。」



 充賀金生辰使,還,為秘書少監,尋除起居郎,累遷權禮部侍郎。中使金回,言其治汴宮,必徙居見迫,宜早為計。上矍然。宰相顧謂中曰:「沉介歸,殊不聞此,何耶?」居數日,中白宰相,請以妄言待罪。湯思退怒,語侵中。已乃除介吏
 部侍郎,徙中以補其處。中猶以備邊為言,又不聽,遂請補外,上不許,曰:「黃中恬退有守。」除左史,且錫鞍馬。



 金使賀天申節,遽以欽宗訃聞,朝論俟使去發喪,中馳白宰相:「此國家大事,臣子至痛,一有失禮,謂天下後世何!」竟得如禮。中自使還,每進;見輒言邊事,又獨陳御備方略,高宗稱善。不數月,金亮已擁眾渡淮。中因入謝,論淮西將士不用命,請擇大臣督師。既而以殿帥楊存中為御營使,中率同列力論不可遣。敵既臨江,朝臣爭遣家逃
 匿,中獨晏然。比敵退,唯中與陳康伯家屬在城中,眾慚服。



 天申節上壽,議者以欽宗服除當舉樂。中言:「《春秋》君弒賊不討,雖葬不書,以明臣子之罪,況欽宗實未葬而可遽作樂乎?」事竟寢。兼給事中。內侍遷官不應法,諫官劉度坐論近習龍大淵忤旨補郡,已復罷之,中皆不書讀。群小相與媒薛,中罷去。尹穡希意詆中為張浚黨。



 乾道改元,中年適七十,即告老,以集英殿修撰致仕,進敷文閣待制。居六年,上御講筵,顧侍臣曰:「黃中老儒,今居
 何許?年幾許?筋力或未衰耶?」召引對內殿,問勞甚渥,以為兵部尚書兼侍讀。



 中前在禮部,嘗諫止作樂事,中去,卒用之。至是又將錫宴,遂奏申前說。詔遣範成大使金以山陵為請。中言:「陛下聖孝及此,天下幸甚,然欽廟梓宮置不問,有所未盡。」上善其言,不能用。



 未滿歲,有歸志,乃陳十要道:以為用人而不自用;以公議進退人才;察邪正;廣言路;核事實;節用度;擇監司;懲貪吏;陳方略;考兵籍。上亟稱善。中力求去,除顯謨閣、提舉江州太平興
 國宮,賜犀帶、香茗。



 除龍圖閣學士,致仕。凡邑里後生上謁,必訓以孝弟忠信。朱熹裁書以見,有曰:「今日之來,將再拜堂下,惟公坐而受之,俾進於門弟子之列,則某之志也。」其為人敬慕如此。其後,上手書遣使訪朝政闕失,進職端明殿學士。屬疾,手草遺表,猶以山陵、欽宗梓宮為言,深以人主之職不可假之左右為戒。淳熙七年八月庚寅卒,年八十有五。九月,詔贈正議大夫。中有奏議十卷。謚簡肅。



 孫道夫,字太沖,眉州丹棱人。年十八貢闢雍。時禁元祐學,坐收蘇氏文除籍。再貢,入優等。張浚薦於高宗,召對,道夫奏:「願修德以回天意,定都以系人心,任賢材、圖興復以雪國恥。」



 上在越,浚遣道夫奏事,賜出身,改左承奉郎。再詔對,言:「漢中前瞰三秦,後蔽巴蜀,孔明、蔣琬出圖關輔,未有不屯漢中者。今欲進兵陜右,當先經營漢中。荊南東連吳會,北通漢沔,號用武之國,晉、宋以來,嘗倚為重鎮。武帝亦以荊南居上流,故以諸子居之。今守江
 當先措置荊南,時至則蜀漢師出秦關,荊楚師出宛洛,陛下親御六軍,由淮甸與諸將會咸陽,孰能御之?」上嘉納,召試館職。上諭宰相:「自渡江以來,文氣未有如道夫者,涵養一二年,當命為詞臣。」



 除秘書正字、權禮部郎官。徽宗兇問禮儀,多所草定。尋權左司員外郎。上問蜀中水運陸運孰便?道夫奏:「水運遲而省費,陸運速而勞民。宣撫司初由水運,率石費錢十千,後以為緩,從陸起丁夫十數萬,率石費五十餘千。」上曰:「水運便,行之。」



 遷校書
 郎。出知懷安軍,乞罷都運司以寬民力,罷戍兵以弭亂階,罷泛使以省浮費。知資州,宣撫鄭剛中薦其治行第一。移知蜀州,盜不敢入境。州產綾,先是,守以軍匠置機買絲虧直,民病之,道夫斷其機。遇事明了,人目為「水晶燈籠。」九年不遷,蓋非秦檜所樂也。



 以吏部郎中入對,言蜀民二稅監酒茶額之弊,上納其言。除太常少卿,假禮部侍郎充賀金正旦使。金將敗盟,詰秦檜存亡,及關、陜買馬非約,道夫隨事折之。使還,擢權禮部侍郎。上曰:「卿
 自小官已為朕知,第趙鼎與張浚相失後,蜀士仕於朝者,皆為沮抑。繼自今有所見,可數求對。」



 兼侍講,奏敵有窺江、淮意。上曰:「朝廷待之甚厚,彼以何名為兵端?」道夫曰:「彼金人身弒其父兄而奪其位,興兵豈問有名,臣願預為之圖。」宰相沈該不以為慮,道夫每進對,輒言武事,該疑其引用張浚,忌之。道夫不自安,請出,除知綿州,致仕,卒,年六十六。



 道夫居官,一意為民,不可干以私。仕宦三十年,奉給多置書籍。然性剛直,喜面折,不容人之短,
 或以此少之云。



 曾幾,字吉甫,其先贛州人,徙河南府。幼有識度,事親孝,母死,蔬食十五年。入太學有聲。兄弼,提舉京西南路學事,按部溺死,無後,特命幾將仕郎。試吏部,考官異其文,置優等,賜上舍出身,擢國子正兼欽慈皇后宅教授。遷闢雍博士,除校書郎。



 林靈素得幸,作符書號《神霄錄》,朝士爭趨之,幾與李綱、傅崧卿皆稱疾不往視。久之,為應天少尹,庭無留訟。閹人得旨取金而無文書,府尹徐處
 仁與之,幾力爭不得。



 靖康初,提舉淮東茶監。高宗即位,改提舉湖北,徙廣西運判、江西提刑,又改浙西。會兄開為禮部侍郎,與秦檜力爭和議,檜怒,開去,幾亦罷。逾月,除廣西轉運副使,徙荊南路。盜駱科起郴之宜章,郴、桂皆澒洞,宣撫司調兵未至,謾以捷聞。幾疏其實,朝廷遣他將平之。請間,得崇道觀。復為廣西運判,固辭,僑居上饒七年。



 檜死,起為浙西提刑、知臺州,治尚清凈,民安之。黃巖令受賄為兩吏所持,令械吏置獄,一夕皆死,幾詰
 其罪。或曰:「令,丞相沈該客也。」治之益急。



 賀允中薦,召對,以疾辭,除直秘閣,歸故治。未幾,復召對,幾言:「士氣久不振,陛下欲起之於一朝,矯枉者必過直,雖有折檻斷鞅、牽裾還笏、若賣直幹譽者,願加優容。」時帝懲檜擅權之弊,方開言路,應詔者眾,幾懼有獲戾者,先事陳之。帝大悅,授秘書少監。



 幾承平已為館職,去三十八年而復至,須鬢皓白,衣冠偉然。每會同舍,多談前輩言行、臺閣典章,薦紳推重焉。詔修《神宗寶訓》,書成,奏薦,帝稱善。
 權禮部侍郎。兄楙、開皆嘗貳春官,幾復為之,人以為榮。



 吳、越大水、地震,幾舉唐貞元故事反復論奏,帝韙其言。他日謂幾曰:「前所進陸贄事甚切,已遣漕臣振濟矣。」引年請謝,上曰:「卿氣貌不類老人,姑為朕留。」謝曰:「臣無補萬一,惟進退有禮,尚不負陛下拔擢。」上閔勞以事,提舉玉隆觀,紹興二十七年也。除集英殿修撰,又三年,升敷文閣待制。



 金犯塞,中外大震,帝召楊存中偕宰執對便殿,諭以將散百官,浮海避之。左僕射陳康伯持不可。存
 中言:「敵空國遠來,已闖淮甸,此正賢智馳騖不足之時。臣願率先將士,北首死敵。」帝喜,遂定議親征,下詔進討。有欲遣使詣敵求緩師者,幾疏言:「增幣請和,無小益,有大害,為朝廷計,正當嘗膽枕戈,專務節儉,經武外一切置之,如是雖北取中原可也。且前日詔諸將傳檄數金君臣,如叱奴隸,何辭可與之和耶?」帝壯之。



 孝宗受禪,幾又上疏數千言。將召,屢請老,乃遷通奉大夫,致仕,擢其子逮為浙西提刑以便養。乾道二年卒,年八十二,謚文
 清。



 幾三仕嶺表,家無長物,人稱其廉。早從舅氏孔文仲、武仲講學。初佐應天時,諫官劉安世亡恙,黨禁方厲,無敢窺其門者,幾獨從之,談經論事,與之合。避地衡岳,又從胡安國游,其學益粹。為文純正雅健,詩尤工。有《經說》二十卷、文集三十卷。



 二子:逢仕至司農卿,逮亦終敷文閣待制,而逢最以學稱。



 開字天游。少好學,善屬文。崇寧間登進士第,調真州司戶,累遷國子司業,擢起居舍人,權中書舍人。掖垣草制,
 多所論駁,忤時相意,左遷太常少卿,責監大寧監鹽井,匹馬之官,不以自卑。召還,時相復用事,監杭州市易務。除直秘閣,知和州,徙知恩州。請祠,得鴻慶宮,判南京國子監。復為中書舍人,罷。提舉洞霄宮。



 欽宗即位,除顯謨閣待制、提舉萬壽觀、知穎昌府,兼京西安撫使。奪職,奉祠。建炎初,復職,知潭州、湖南安撫使。逾年求去,復得鴻慶宮,起知平江府、廣東經略安撫使。奉詔駐潮陽招捕處寇,訖事,乃之鎮。居二年,盡平群盜。提舉太平觀。



 復以
 中書舍人召,首論:「自古興衰撥亂之主,必有一定之論,然後能成功。願講明大計,使議論一定,斷而必行,則功烈可與周宣侔矣。」又論:「車駕撫巡東南,重兵所聚,限以大江,敵未易遽犯,其所窺伺者全蜀也。一失其防,陛下不得高枕而臥矣。願擇重臣與吳玠協力固護全蜀。」屢請去,進寶文閣待制,知鎮江府兼沿江安撫使。



 召為刑部侍郎。言:「太祖懲五季尾大不掉之患,畿甸屯營,倍於天下,周廬宿衛,領以三衙。今禁旅單弱,願參舊制增補
 之。」帝悉嘉納。



 遷禮部侍郎兼直學士院。時秦檜專主和議,開當草國書,辨視體制非是,論之,不聽,遂請罷,改兼侍讀。檜嘗招開慰以溫言,且曰:「主上虛執政以待。」開曰:「儒者所爭在義,茍為非義,高爵厚祿弗顧也。願聞所以事敵之禮。」檜曰:「若高麗之於本朝耳。」開曰:「主上以聖德登大位,臣民之所推戴,列聖之所聽聞,公當強兵富國,尊主庇民,奈可自卑辱至此,非開所敢聞也。」又引古誼以折之。檜大怒曰:「侍郎知故事,檜獨不知耶?」他日,開又
 至政事堂,問「計果安出?」檜曰:「聖意已定,尚何言!公自取大名而去,如檜,第欲濟國事耳。」然猶以梓宮未還,母后、欽宗未復,詔侍從、臺諫集議以聞。開上疏略曰:「但當修德立政,嚴於為備,以我之仁敵彼之不仁,以我之義敵彼之不義,以我之戒懼敵彼之驕泰,真積力久,如元氣固而病自消,大陽升而陰自散,不待屈己,陛下之志成矣。不然,恐非在天之靈與太后、淵聖所望於陛下者也。」檜曰:「此事大系安危。」開曰:「今日不當說安危,只當論存
 亡。」檜矍然。



 會樞密編修胡銓上封事,痛詆檜,極稱開,由是罷,以寶文閣待制知婺州。開言:「議論妄發,實緣國事。」力請歸。檜議奪職,同列以為不可,提舉太平觀、知徽州。以病免,居閑十餘年。黃達如請籍和議同異為士大夫升黜,即擢達如監察御史,首劾開,褫職。引年請還政,僅復秘閣修撰,卒,年七十一。檜死,始復待制,盡還致仕遺表恩數。



 開孝友厚族,信於朋友。其守歷陽也,從游酢學,日讀《論語》,求諸言而不得,則反求諸心,每有會意,欣然
 忘食。其留南京,劉安世一見如舊,定交終身。故立朝遇事,臨大節而不可奪,師友淵源,固有所自云。



 勾濤,字景山,成都新繁人。登崇寧二年進士第,調嘉州法掾、川陜鑄錢司屬官。建炎初,通判黔州。田祐恭兵道境上,濤白守,燕勞之,祐恭感恩厲下,郡得以無犯。湖湘賊王闢破秭歸,桑仲、郭守忠攻茶務箭窠砦,將犯夔門。夔兵素單弱,宣司檄祐恭捍禦,濤帥黔兵佐之,賊潰去。宣撫張浚奏濤知巴州,不赴。



 翰林侍讀學士範仲薦,召
 見,論五事,除兵部郎中。七年,遷右司郎官兼校正。日食,上言。八月,遷起居舍人,以足疾,命閣門賜墩待班。九月,兼權中書舍人。



 時沿邊久宿兵,江、浙罷於饋餉,荊、襄、淮、楚多曠土,濤因進羊祜屯田故事,事下諸大將,於是邊方議行屯田。淮西都統制劉光世乞罷,丞相張浚欲以呂祉代之,濤謂:「祉疏庸淺謀,必敗事,莫若就擇將士素所推服者用之,否則劉錡可。」浚不納,祉至,果以輕易失士心,未幾,酈瓊叛,祉死於亂。浚聞之,夜半召濤愧謝。



 時
 帝駐蹕建康,欲亟還臨安。濤入見曰:「今江、淮列戍十餘萬,茍付托得人,可無憂顧。適此危疑,詎宜輕退,以啟敵心。」因薦劉錡。帝即命以其眾鎮合肥。川、陜宣撫使吳玠言都轉運使李迨朘刻賞格,迨亦奏玠苛費,帝以問濤。濤曰:「玠忠在西蜀,縱費,寧可核?第移迨他路可爾。」帝然之。



 會金人廢劉豫,金、房鎮撫使郭浩遣其弟沔奏事。濤察沔警敏可仗,乞詔諭陜右諸叛將乘機南歸,帝命濤草詔,沔持以往,聞者流涕。十二月,除中書舍人。



 八年,除
 史館修撰。重修《哲宗實錄》,帝諭之曰:「昭慈聖獻皇后病革,朕流涕問所欲言,後愴然謂朕曰:『吾逮事宣仁聖烈皇后,見其任賢使能,約己便民,憂勤宗社,疏遠外家,古今母後無與為比。不幸奸邪罔上,史官蔡卞等同惡相濟,造謗史以損聖德,誰不切齒!在天之靈亦或介介。其以筆屬正臣,亟從刪削,以信來世。』朕痛念遺訓,未嘗一日輒忘,今以命卿。」濤奏:「數十年來,宰相不學無術,邪正貿亂,所以奸臣子孫得逞其私智,幾亂裕陵成書。非賴
 陛下聖明,則任申必先有過嶺之謫,臣亦恐復蹈媒薛之禍。」帝慰勉之。六月,《實錄》成,進一秩,就館賜宴。復修《徽宗實錄》,以中書舍人呂本中為薦,丞相趙鼎諭旨宜婉辭紀載。濤曰:「崇寧、大觀大臣誤國,以稔今禍,藉有隱諱,如天下野史何?」



 七月,除給事中。求去,以徽猷閣待制知池州,改提舉江州太平觀。俄除荊湖北路安撫使、知潭州。秦檜嘗令人諭意,欲與共政,濤以書謝之。檜諷言劾之,不報。



 濤上書論時事之害政者:「大臣密諭王倫變易
 地界,一也;蔡攸之妻近居臨平,咫尺行都,略不畏避,二也;小大之臣,凡在謫籍,皆已甄敘,惡如京、黼,尚蒙寬宥,今待從之臣,初無大過,理宜牽復,三也;河南故地復歸中國,新附之民,延頸德澤,承流之寄,當加精選,四也;臺諫為耳目之司,今宰相引援,皆同舍之舊,倚為鷹犬,五也。」帝嘆其忠直,賜以繒彩、茶藥,且令事有大於此者,悉以聞。秩滿,提舉太平觀。



 十一年,帝謂秦檜曰:「勾濤久閑,性喜泉石,可進職與一山水近郡。」檜對:「永嘉有天臺、雁
 蕩之勝。」帝曰:「永嘉太遠,其以湖州命之。」俄以疾卒,年五十九。遺表聞,帝震悼,顧近臣曰:「勾濤死矣,惜哉!」贈左太中大夫。



 濤身長七尺,風貌偉然,頗以忠亮自許。國有大議,帝必委心延訪,往復酬詰,率漏下數刻始罷。料邊情如在目前,知名之士多所薦進。有文集十卷,《西掖制書》十卷,奏議十卷。



 李彌遜,字似之,蘇州吳縣人。弱冠,以上舍登大觀三年第,調單州司戶,再調陽穀簿。政和四年,除國朝會要所
 檢閱文字。引見,特遷校書郎,充編修六典校閱,累官起成郎。以封事剴切,貶知盧山縣,改奉嵩山祠。廢斥隱居者八載。



 宣和末,知冀州。金人犯河朔,諸郡皆驚備,彌遜損金帛,致勇士,修城堞,決河護塹,邀擊其游騎,斬首甚眾。兀朮北還,戒師毋犯其城。



 靖康元年,召為衛尉少卿,出知瑞州。二年,建康府牙校周德叛,執帥宇文粹中,殺官吏,嬰城自守,勢猖獗。彌遜以江東判運領郡事,單騎扣賊圍,以蠟書射城中招降。賊通款,開關迎之,彌遜諭
 以禍福,勉使勤王。時李綱行次建康,共謀誅首惡五十人,撫其餘黨,一郡帖然。



 改淮南運副。後奉興國宮祠,知饒州,召對,首奏「當堅定規模,排斥奸言」。又謂:「朝廷一日無事,幸一日之安,一月無事,幸一月之安,欲求終歲之安,已不可得,況能定天下大計乎?」帝嘉其讜直。輔臣有不悅者,以直寶文閣知吉州。陛辭,帝曰:「朕欲留卿,大臣欲重試卿民事,行召卿矣。」



 七年秋,遷起居郎。彌遜自政和末以上封事得貶,垂二十年,及復居是職,直前論事,
 鯁切如初。冬,試中書舍人,奏六事曰:「固蕃維以禦外侮,嚴禁衛以尊朝廷,練兵以壯國勢,節用以備軍食,收民心以固根本,擇守帥以責實效。」時駐蹕未定,有旨料舟給卒以濟宮人。彌遜繳奏曰:「六飛雷動,百司豫嚴,時方孔艱,宜以宗社為心,不宜於內幸細故,更勤聖慮,事雖至微,懼傷大體。」帝嘉納之。試戶部侍郎。



 秦檜再相,惟彌遜與吏部侍郎晏敦復有憂色。八年,彌遜上疏乞外甚力,詔不允。趙鼎罷相,檜專國,贊帝決策通和。金國遣烏
 陵思謀等入界,索禮甚悖,軍民皆不平,人言紛紛。檜於御榻前求去,欲要決意屈己從和。樞密院編修官胡銓上疏乞斬檜,校書郎範如圭以書責檜曲學背師,忘仇辱國,禮部侍郎曾開抗聲引古誼以折檜,相繼貶逐。



 彌遜請對,言金使之請和,欲行君臣之禮,有大不可。帝以為然,詔廷臣大議,即日入奏。彌遜手疏力言:「陛下受金人空言,未有一毫之得,乃欲輕祖宗之付托,屈身委命,自同下國而尊奉之,倒持太阿,授人以柄,危國之道,而
 謂之和可乎?借使金人姑從吾欲,假以目前之安,異時一有無厭之求,意外之欲,從之則害吾社稷之計,不從則釁端復開,是今日徒有屈身之辱,而後患未已。」又言:「陛下率國人以事仇,將何以責天下忠臣義士之氣?」力陳不可者三。



 檜嘗邀彌遜至私第,曰:「政府方虛員,茍和好無異議,當以兩地相浼。」答曰:「彌遜受國恩深厚,何敢見利忘義。顧今日之事,國人皆不以為然,獨有一去可報相公。」檜默然。次日,彌遜再上疏,言愈切直,又言:「送伴
 使揣摩迎合,不恤社稷,乞別選忠信之人,協濟國事。」檜大怒。彌遜引疾,帝諭大臣留之。時和議已決,附會其說者,至謂「向使明州時,主上雖百拜亦不問」,議論靡然。賴彌遜廷爭,檜雖不從,亦憚公論。再與金使者計,議和不受封冊,如宰相就館見金使,受其書納入禁中,多所降殺,惟君臣之禮不得盡爭。



 九年春,再上疏乞歸田,以徽猷閣直學士知端州,改知漳州。十年,歸隱連江西山。是歲,兀朮分四道入侵,明年,又侵淮西,取壽春,竟如彌遜
 言。



 十二年,檜乘金兵既敗,收諸路兵,復通和好,追仇向者盡言之臣,嗾言者論彌遜與趙鼎、王庶、曾開四人同沮和議。於是彌遜落職,十餘年間不通時相書,不請磨勘,不乞任子,不序封爵,以終其身,常憂國,無怨懟意。二十三年,卒。朝廷思其忠節,詔復敷文閣待制。有奏議三卷,外制二卷,《議古》三卷,詩十卷。弟彌大。



 彌大字似矩,登崇寧三年進士第。以大臣薦召對,除校書郎,遷監察御史。假太常少卿充契丹賀正旦使。時傳
 聞燕民欲歸漢,徽宗遣彌大覘之。使還,奏所聞有二:「或謂彼主淫刑滅親,種類畔離,女真侵迫,國勢危殆為可取;或謂下詔罪己,擢用耆舊,招赦盜賊,國尚有人未可取;莫若聽其自相攻並。」遷起居郎,試中書舍人,同修國史。



 童貫宣撫永興,走馬承受白鍔恃貫不報師期,朝廷止從薄責。彌大繳奏,以為邊報不至,非朝廷福。鍔坐除名,彌大亦出知光州。移知鄂州。召為給事中兼校正御前文籍詳定官,拜禮部侍郎。



 金人大舉入侵,李綱定城
 守之策,命彌大為參議,與綱不合,罷。未幾,除刑部尚書。初,朝廷許割三鎮畀金人,既而遣種師道、師中援河北,姚古援河東,彌大上疏乞起河東西境麟、府諸郡及陜西兵以濟古之師,起河東路及京東近郡兵以濟師道、師中之師,為腹背攻劫之圖。遂除彌大河東宣撫副使。張師正領勝捷軍敗於河東,潰歸,彌大誅之。復遣餘卒援真定,餘卒叛。



 宣撫罷,命彌大知陜州。河東破,小將李彥先來謁,言軍事,彌大壯之,留為將,戍崤、澠間以遏敵。
 詔遣使召援,彌大未敢進。會永興帥範致虛糾兵勤王,檄彌大充諸道計議。行至方城,道阻,乃率眾赴大元帥府。



 建炎元年,除知淮寧府。到郡未幾,杜用等夜叛,彌大縋城出,賊散乃還,坐貶秩。尋召為吏部侍郎。帝如杭州,命權紹興府,試戶部尚書兼侍讀。呂頤浩視師,以彌大為參謀官。彌大奏:「王導、謝安為都督,未嘗離朝廷,今邊圉幸無他,頤浩不宜輕動。」又言:「已為天子從官,非宰相可闢。乞於諸軍悉置軍正,如漢朝故事,以察官、郎官為
 之。陛下必欲留臣,當別為一司,伺察頤浩過失。」忤旨,出知平江府。



 中丞沉與求劾彌大謀間君臣,妄自尊大,奪職歸。起知靜江府,奏廣西邊防利害。入為工部尚書。未幾,罷去。廣西提刑韓璜劾其在靜江日斷強盜死罪,引絞入斬,貶兩秩。紹興十年卒,年六十一。



 論曰:宋既南渡,日以徽宗梓宮及韋後為念。秦檜主和,甘心屈己。張燾連章論列,謀深慮遠,其言取必於天,豈忘宗社之仇哉,亦曰相時而動耳!惜其利澤專於蜀也。
 黃中不黨不阿,明察料敵,立朝忠實,退不忘君。道夫受知張浚,憂國而不為身謀。曾幾積學潔行,風節凜凜,陳嘗膽枕戈之言,以贊親征,亦壯矣哉!勾濤直節正論,不受檜私,潔身歸老。彌遜、曾開同沮和議,廢絀以沒,無怨懟心,所謂臨大節而不可奪者歟!



\end{pinyinscope}