\article{列傳第一百八十六道學一}

\begin{pinyinscope}

 ○周敦頤程顥程頤張載弟戩邵雍



 「道學」之名,古無是也。三代盛時,天子以是道為政教,大臣百官有司以是道為職業,黨、庠、術、序師弟子以是道
 為講習,四方百姓日用是道而不知。是故盈覆載之間,無一民一物不被是道之澤,以遂其性。於斯時也,道學之名,何自而立哉。



 文王、周公既沒,孔子有德無位,既不能使是道之用漸被斯世,退而與其徒定禮樂,明憲章,刪《詩》,修《春秋》,贊《易象》,討論《墳》、《典》,期使五三聖人之道昭明於無窮。故曰:「夫子賢於堯、舜遠矣。」孔子沒,曾子獨得其傳,傳之子思,以及孟子,孟子沒而無傳。兩漢而下,儒者之論大道,察焉而弗精,語焉而弗詳,異端邪說起而
 乘之,幾至大壞。



 千有餘載,至宋中葉,周敦頤出於舂陵,乃得聖賢不傳之學,作《太極圖說》、《通書》,推明陰陽五行之理,命於天而性於人者,了若指掌。張載作《西銘》,又極言理一分殊之旨,然後道之大原出於天者,灼然而無疑焉。仁宗明道初年,程顥及弟頤實生,及長,受業周氏,已乃擴大其所聞,表章《大學》、《中庸》二篇,與《語》、《孟》並行,於是上自帝王傅心之奧,下至初學入德之門。融會貫通,無復餘蘊。



 迄宋南渡,新安朱熹得程氏正傳,其學加親
 切焉。大抵以格物致知為先,明善誠身為要,凡《詩》、《書》,六藝之文,與夫孔、孟之遺言,顛錯於秦火,支離於漢儒,幽沉於魏、晉六朝者,至是皆煥然而大明,秩然而各得其所。此宋儒之學所以度越諸子,而上接孟氏者歟。其於世代之污隆,氣化之榮悴,有所關系也甚大。道學盛於宋,宋弗究於用,甚至有厲禁焉。後之時君世主,欲復天德王道之治,必來此取法矣。



 邵雍高明英悟,程氏實推重之,舊史列之隱逸,未當,今置張載後。張栻之學,亦出
 程氏,既見朱熹,相與博約又大進焉。其他程、朱門人,考其源委,各以類從,作《道學傳》。



 周敦頤,字茂叔,道州營道人。元名敦實,避英宗舊諱改焉。以舅龍圖閣學士鄭向任,為分寧主簿。有獄久不決,敦頤至,一訊立辨。邑人驚曰:「老吏不如也。」部使者薦之,調南安軍司理參軍。有囚法不當死,轉運使王逵欲深治之。逵,酷悍吏也,眾莫敢爭,敦頤獨與之辨,不聽,乃委手版歸,將棄官去,曰:「如此尚可仕乎!殺人以媚人,吾不
 為也。」逵悟,囚得免。



 移郴之桂陽令,治績尤著。郡守李初平賢之,語之曰:「吾欲讀書,何如?」敦頤曰:「公老無及矣,請為公言之。」二年果有得。徙知南昌,南昌人皆曰:「是能辨分寧獄者,吾屬得所訴矣。」富家大姓、黠吏惡少,惴惴焉不獨以得罪於令為憂,而又以污穢善政為恥。歷合州判官,事不經手,吏不敢決。雖下之,民不肯從。部使者趙抃惑於譖口,臨之甚威,敦頤處之超然。通判虔州,抃守虔,熟視其所為,乃大悟,執其手曰:「吾幾失君矣,今而後
 乃知周茂叔也。」



 熙寧初,知郴州。用抃及呂公著薦,為廣東轉運判官,提點刑獄,以洗冤澤物為己任。行部不憚勞苦,雖瘴癘險遠,亦緩視徐按。以疾求知南康軍。因家廬山蓮花峰下。前有溪,合於溢江,取營道所居濂溪以名之。抃再鎮蜀,將奏用之,未及而卒,年五十七。



 黃庭堅稱其「人品甚高,胸懷灑落,如光風霽月。廉於取名而銳於求志,薄於徼福而厚於得民,菲於奉身而燕及煢嫠,陋於希世而尚友千古。」



 博學行力,著《太極圖》,明天理之
 根源,究萬物之終始。其說曰:



 無極而太極。太極動而生陽,動極而靜,靜而生陰,靜極復動,一動一靜,互為其根,分陰分陽,兩儀立焉。陽變陰合,而生水、火、木、金、土,五氣順布,四時行焉。五行一陰陽也,陰陽一太極也。太極本無極也。五行之生也,各一其性。無極之真,二五之精,妙合而凝,乾道成男,坤道成女。二氣交感,化生萬物,萬物生生,而變化無窮焉。



 惟人也得其秀而最靈,形既生矣,神發知矣,五性感動而善惡分,萬事出矣。聖人定之以
 中正仁義而主靜,立人極焉。故聖人與天地合其德,日月合其明,四時合其序,鬼神合其吉兇。君子修之吉,小人悖之兇。故曰:「立天之道,曰陰與陽。立地之道,曰柔與剛。立人之道,曰仁與義。」又曰:「原始反終,故知死生之說。」大哉《易》也,斯其至矣。



 又著《通書》四十篇,發明太極之蘊。序者謂「其言約而道大,文質而義精,得孔、孟之本源,大有功於學者也。」



 掾南安時,程珦通判軍事,視其氣貌非常人,與語,知其為學知道,因與為友,使二子顥、頤往受
 業焉。敦頤每令尋孔、顏樂處,所樂何事,二程之學源流乎此矣。故顥之言曰:「自再見周茂叔後,吟風弄月以歸,有『吾與點也』之意。」侯師聖學於程頤,未悟,訪敦頤,敦頤曰:「吾老矣,說不可不詳。」留對榻夜談,越三日乃還。頤驚異之,曰:「非從周茂叔來耶?」其善開發人類此。



 嘉定十三年,賜謚曰元公,淳祐元年,封汝南伯,從祀孔子廟庭。



 二子壽、燾,燾官至寶文閣待制。



 程顥,字伯淳,世居中山,後從開封徙河南。高祖羽,太宗
 朝三司使。父珦,仁宗錄舊臣後,以為黃陂尉。久之,知龔州。時宜獠區希範既誅,鄉人忽傳其神降,言「當為我南海立祠」,於是迎其神以往,至龔,珦使詰之,曰:「比過潯,潯守以為妖,投祠具江中,逆流而上,守懼,乃更致禮。」珦使復投之,順流去,其妄乃息。徙知磁州,又徙漢州。嘗宴客開元僧舍,酒方行,人歡言佛光見,觀者相騰踐,不可禁,珦安坐不動,頃之遂定。熙寧法行,為守令者奉命唯恐後,珦獨抗議,指其未便。使者李元瑜怒,即移病歸,旋致
 仕,累轉太中大夫。元祐五年,卒,年八十五。



 珦慈恕而剛斷,平居與幼賤處,唯恐有傷其意,至於犯義理,則不假也。左右使令之人,無日不察其饑飽寒燠。前後五得任子,以均諸父之子孫。嫁遣孤女,必盡其力。所得奉祿,分贍親戚之貧者。伯母寡居,奉養甚至。從女兄既適人而喪其夫,珦迎以歸,教養其子,均於子侄。時官小祿薄,克己為義,人以為難。文彥博、蘇頌等九人表其清節,詔賜帛二百,官給其葬。



 顥舉進士,調鄮、上元主簿。鄮民有借
 兄宅居者,發地得瘞錢,兄之子訴曰:「父所藏。」顥問:「幾何年?」曰:「四十年。」彼借居幾時?」曰:「二十年矣。」遣吏取十千視之,謂訴者曰:「今官所鑄錢,不五六年即遍天下,此皆未藏前數十年所鑄,何也?」其人不能答。茅山有池,產龍如蜥蜴而五色。祥符中嘗取二龍入都,半塗失其一,中使雲飛空而逝。民俗嚴奉不懈,顥捕而脯之。



 為晉城令,富人張氏父死,旦有老叟踵門曰:「我,汝父也。」子驚疑莫測,相與詣縣。叟曰:「身為醫,遠出治疾,而妻生子,貧不能養,
 以與張。」顥質其驗。取懷中一書進,其所記曰:「某年月日,抱兒與張三翁家。」顥問:「張是時才四十,安得有翁稱?」叟駭謝。



 民稅粟多移近邊,載往則道遠,就糴則價高。顥擇富而可任者,預使貯粟以待,費大省。民以事至縣者,必告以孝弟忠信,入所以事其父兄,出所以事其長上。度鄉村遠近為伍保,使之力役相助,患難相恤,而奸偽無所容。凡孤煢殘廢者,責之親戚鄉黨,使無失所。行旅出於其途者,疾病皆有所養。鄉必有校,暇時親至,召父老
 與之語。兒童所讀書,親為正句讀,教者不善,則為易置。擇子弟之秀者,聚而教之。鄉民為社會,為立科條,旌別善惡,使有勸有恥。在縣三歲,民愛之如父母。



 熙寧初,用呂公著薦,為太子中允、監察御史裏行。神宗素知其名,數召見,每退,必曰:「頻求對,欲常常見卿。」一日,從容咨訪,報正午,始趨出,庭中人曰:「御史不知上未食乎?」前後進說甚多,大要以正心窒欲、求賢育材為言,務以誠意感悟主上。嘗勸帝防未萌之欲,及勿輕天下士,帝俯躬曰:「
 當為卿戒之。」



 王安石執政,議更法令,中外皆不以為便,言者攻之甚力。顥被旨赴中堂議事,安石方怒言者,厲色待之。顥徐曰:「天下事非一家私議,願平氣以聽。」安石為之愧屈。自安石用事,顥未嘗一語及於功利。居職八九月,數論時政,最後言曰:「智者若禹之行水,行其所無事也;舍而之險阻,不足以言智。自古興治立事,未有中外人情交謂不可而能有成者,況於排斥忠良,沮廢公議,用賤陵貴,以邪干正者乎?正使徼幸有小成,而興利
 之臣日進,尚德之風浸衰,尤非朝廷之福。」遂乞去言職。安石本與之善,及是雖不合,猶敬其忠信,不深怒,但出提點京西刑獄。顥固辭,改簽書鎮寧軍判官。司馬光在長安,上疏求退,稱顥公直,以為己所不如。



 程昉治河,取澶卒八百而虐用之,眾逃歸。群僚畏昉,欲勿納。顥曰:「彼逃死自歸,弗納必亂。若昉怒,吾自任之。」即親往啟門拊勞,約少休三日復役,眾歡踴而入。具以事上,得不遣。昉後過州,揚言曰:「澶卒之潰,蓋程中允誘之,吾且訴於上。」
 顥聞之,曰:「彼方憚我,何能為。」果不敢言。



 曹村埽決,顥謂郡守劉渙曰:「曹村決,京師可虞。臣子之分,身可塞亦所當為,盍盡遣廂卒見付。」渙以鎮印付顥,立走決所,激諭士卒。議者以為勢不可塞,徒勞人爾。顥命善泅者度決口,引巨索濟眾,兩岸並進,數日而合。



 求監洛河竹木務,歷年不敘伐閱,特遷太常丞。帝又欲使修《三經義》,執政不可,命知扶溝縣。廣濟、蔡河在縣境,瀕河惡子無生理,專脅取行舟財貨,歲必焚舟十數以立威。顥捕得一
 人,使引其類,貰宿惡,分地處之,令以挽繂為業,且察為奸者,自是境無焚剽患。內侍王中正按閱保甲,權焰章震,諸邑競侈供張悅之,主吏來請,顥曰:「吾邑貧,安能效他邑。取於民,法所禁也,獨有令故青帳可用爾。」除判武學,李定劾其新法之初首為異論,罷歸故官。又坐獄逸囚,責監汝州鹽稅。哲宗立,召為宗正丞,未行而卒,年五十四。



 顥資性過人,充養有道,和粹之氣,盎於面背,門人交友從之數十年,亦未嘗見其忿厲之容。遇事優為,雖當
 倉卒,不動聲色。自十五六時,與弟頤聞汝南周敦頤論學,遂厭科舉之習,慨然有求道之志。泛濫於諸家,出入於老、釋者幾十年,返求諸《六經》而後得之。秦、漢以來,未有臻斯理者。



 教人自致知至於知止,誠意至於平天下,灑掃應對至於窮理盡性,循循有序。病學者厭卑近而鶩高遠,卒無成焉,故其言曰:「道之不明,異端害之也。昔之害近而易知,今之害深而難辨。昔之惑人也乘其迷暗,今之惑人也因其高明。自謂之窮神知化,而不足以
 開物成務,言為無不周遍,實則外於倫理,窮深極微,而不可以入堯、舜之道。天下之學,非淺陋固滯,則必入於此。自道之不明也,邪誕妖妄之說競起,塗生民之耳目,溺天下於污濁,雖高才明智,膠於見聞,醉生夢死,不自覺也。是皆正路之蓁蕪,聖門之蔽塞,闢之而後可以入道。」



 顥之死,士大夫識與不識,莫不哀傷焉。文彥博採眾論,題其墓曰明道先生。其弟頤序之曰:「周公沒,聖人之道不行;孟軻死,聖人之學不傳。道不行,百世無善治;學
 不傳,千載無真儒。無善治,士猶得以明夫善治之道,以淑諸人,以傳諸後;無真儒,則貿貿焉莫知所之,人欲肆而天理滅矣。先生生於千四百年之後,得不傳之學於遺經,以興起斯文為己任,辨異喘,闢邪說,使聖人之道煥然復明於世,蓋自孟子之後,一人而已。然學者於道不知所向,則孰知斯人之為功;不知所至,則孰知斯名之稱情也哉。」



 嘉定十三年,賜謚曰純公。淳祐元年封河南伯,從祀孔子廟庭。



 程頤,字正叔。年十八,上書闕下,欲天子黜世俗之論,以王道為心。游太學,見胡瑗問諸生以顏子所好何學,頤因答曰:



 學以至聖人之道也。聖人可學而至歟?曰:然。學之道如何?曰:天地儲精,得五行之秀者為人,其本也真而靜,其未發也。五性具焉,曰仁、義、禮、智、信。形既生矣,外物觸其形而動其中矣,其中動而七情出焉,曰喜、怒、哀、樂、愛、惡、欲。情既熾而益蕩,其性鑿矣。是故覺者約其情使合於中,正其心,養其性;愚者則不知制之,縱其情而
 至於邪僻,梏其性而亡之。



 然學之道,必先明諸心,知所養;然後力行以求至,所謂「自明而誠」也。誠之之道,在乎信道篤,信道篤則行之果,行之果則守之固,仁義忠信不離乎心,造次必於是,顛沛必於是,出處語默必於是,久而弗失,則居之安,動容周旋中禮,而邪僻之心無自生矣。



 故顏子所事,則曰:「非禮勿視,非禮勿聽,非禮勿言,非禮勿動。」仲尼稱之,則曰:「得一善則拳拳服膺而弗失之矣。」又曰:「不遷怒,不貳過。」「有不善未嘗不知,知之未嘗
 復行。」此其好之篤,學之得其道也。然聖人則不思而得,不勉而中;顏子則必思而後得,必勉而後中。其與聖人相去一息,所未至者守之也,非化之也。以其好學之心,假之以年,則不日而化矣。



 後人不達,以謂聖本生知,非學可至,而為學之道遂失。不求諸己,而求諸外,以博聞強記、巧文麗辭為工,榮華其言,鮮有至於道者。則今之學,與顏子所好異矣。



 瑗得其文,大驚異之,即延見,處以學職。呂希哲首以師禮事頤。



 治平、元豐間,大臣屢薦,皆
 不起。哲宗初,司馬光、呂公著共疏其行義曰:「伏見河南府處士程頤,力學好古,安貧守節,言必忠信,動遵禮法。年逾五十,不求仕進,真儒者之高蹈,聖世之逸民。望擢以不次,使士類有所矜式。」詔以為西京國子監教授,力辭。



 尋召為秘書省校書郎,既入見,擢崇政殿說書。即上疏言:「習與智長,化與心成。今夫人民善教其子弟者,亦必延名德之士,使與之處,以薰陶成性。況陛下春秋之富,雖睿聖得於天資,而輔養之道不可不至。大率一日
 之中,接賢士大夫之時多,親寺人宮女之時少,則氣質變化,自然而成。願選名儒入侍勸講,講罷留之分直,以備訪問,或有小失,隨事獻規,歲月積久,必能養成聖德。」頤每進講,色甚莊,繼以諷諫。聞帝在宮中盥而避蟻,問:「有是乎?」曰:「然,誠恐傷之爾。」頤曰:「推此心以及四海,帝王之要道也。」



 神宗喪未除,冬至,百官表賀,頤言:「節序變遷,時思方切,乞改賀為慰。」既除喪,有司請開樂置宴,頤又言:「除喪而用吉禮,尚當因事張樂,今特設宴,是喜之也。」
 皆從之。帝嘗以瘡疹不御邇英累日,頤詣宰相問安否,且曰:「上不御殿,太后不當獨坐。且人主有疾,大臣可不知乎?」翌日,宰相以下始奏請問疾。



 蘇軾不悅於頤,頤門人賈易、朱光庭不能平,合攻軾。胡宗愈、顧臨詆頤不宜用,孔文仲極論之,遂出管勾西京國子監。久之,加直秘閣,再上表辭。董敦逸復摭其有怨望語,去官。紹聖中,削籍竄涪州。李清臣尹洛,即日迫遣之,欲入別叔母亦不許,明日贐以銀百兩,頤不受。徽宗即位,徙峽州,俄復其
 官,又奪於崇寧。卒年七十五。



 頤於書無所不讀。其學本於誠,以《大學》、《語》、《孟》、《中庸》為標指,而達於《六經》。動止語默,一以聖人為師,其不至乎聖人不止也。張載稱其兄弟從十四五時,便脫然欲學聖人,故卒得孔、孟不傳之學,以為諸儒倡。其言之旨,若布帛菽粟然,知德者尤尊崇之。嘗言:「今農夫祁寒暑雨,深耕易耨,播種五穀,吾得而食之;百工技藝,作為器物,吾得而用之;介胄之士,被堅執銳,以守土宇,吾得而安之。無功澤及人,而浪度歲月,
 晏然為天地間一蠹,唯綴緝聖人遺書,庶幾有補爾。」於是著《易》、《春秋傳》以傳於世。《易傳序》曰:



 《易》,變易也,隨時變易以從道也。其為書也,廣大悉備,將以順性命之理,通幽明之故,盡事物之情,而示開物成務之道也。聖人之憂患後世,可謂至矣。去古雖遠,遺經尚存,然而前儒失意以傳言,後學誦言而忘味,自秦而下,蓋無傳矣。予生千載之後,悼斯文之湮晦,將俾後人沿流而求源,此《傳》所以作也。



 「《易》有聖人之道四焉:以言者尚其辭,以動者
 尚其變,以制器者尚其象,以卜筮者尚其占」。吉兇消長之理、進退存亡之道備於辭,推辭考卦可以知變,象與占在其中矣。「君子居則觀其象而玩其辭,動則觀其變而玩其占」,得於辭不達其意者有矣,未有不得於辭而能通其意者也。至微者理也,至著者象也。體用一源,顯微無間,觀會通以行其典禮,則辭無所不備。故善學者,求言必自近,易於近者,非知言者也。予所傳者辭也,由辭以得意,則在乎人焉。



 《春秋傳序》曰:



 天之生民,必有出
 類之才起而君長之,治之而爭奪息,導之而生養遂,教之而倫理明,然後人道立,天道成,地道平。二帝而上,聖賢世出,隨時有作,順乎風氣之宜,不先天以開人,各因時而立政。暨乎三王迭興,三重既備,子、丑、寅之建正,忠、質、文之更尚,人道備矣,天運周矣。聖王既不復作,有天下者雖欲仿古之跡,亦私意妄為而已。事之繆,秦至以建亥為正;道之悖,漢專以智力持世,豈復知先王之道也。



 夫子當周之末,以聖人不復作也,順天應時之治不
 復有也,於是作《春秋》,為百王不易之大法。所謂「考諸三王而不繆,建諸天地而不悖,質諸鬼神而無疑,百世以俟聖人而不惑」者也。先儒之傳,游、夏不能贊一辭,辭不待贊者也,言不能與於斯爾。斯道也,唯顏子嘗聞之矣。「行夏之時,乘殷之輅,服周之冕,樂則《韶舞》,此其準的也。後史以吏視《春秋》,謂褒善貶惡而已,至於經世之大法,則不知也。



 《春秋》大義數十,其義雖大,炳如日星,乃易見也。惟其微辭隱義、時措從宜者,為難知也。或抑或縱,或
 予或奪,或進或退,或微或顯,而得乎義理之安,文質之中,寬猛之宜,是非之公,乃制事之權衡,揆道之模範也。夫觀百物然後識化工之神,聚眾材然後知作室之用,於一事一義而欲窺聖人之用心,非上智不能也。故學《春秋》者,必優游涵泳,默識心通,然後能造其微也。後王知《春秋》之義,則雖德非禹、湯,尚可以法三代之治。



 自秦而下,其學不傳,予悼夫聖人之志不明於後世也,故作《傳》以明之,俾後之人通其文而求其義,得其意而法其
 用,則三代可復也。是《傳》也,雖未能極聖人之蘊奧,庶幾學者得其門而入矣。



 平生誨人不倦,故學者出其門最多,淵源所漸,皆為名士。涪人祠頤於北巖,世稱為伊川先生。嘉定十三年,賜謚曰正公。淳祐元年,封伊陽伯,從祀孔子廟庭。



 門人劉絢、李籲、謝良佐、游酢、張繹、蘇昞皆班班可書,附於左。呂大鈞、大臨見《大防傳》。



 張載,字子厚,長安人。少喜談兵。至欲結客取洮西之地。年二十一,以書謁範仲淹,一見知其遠器,乃警之曰:「儒
 者自有名教可樂,何事於兵。」因勸讀《中庸》。載讀其書,猶以為未足,又訪諸釋、老,累年究極其說,知無所得,反而求之《六經》。嘗坐虎皮講《易》京師,聽從者甚眾。一夕,二程至,與論《易》,次日語人曰:「比見二程,深明《易》道,吾所弗及,汝輩可師之。」撤坐輟講。與二程語道學之要,渙然自信曰:「吾道自足,何事旁求。」於是盡棄異學,淳如也。



 舉進士,為祈州司法參軍,雲巖令。政事以敦本善俗為先,每月吉,具酒食,召鄉人高年會縣庭,親為勸酬。使人知養老
 事長之義,因問民疾苦,及告所以訓戒子弟之意。



 熙寧初,御史中丞呂公著言其有古學,神宗方一新百度,思得才哲士謀之,召見問治道,對曰:「為政不法三代者,終茍道也。」帝悅,以為崇文院校書。他日見王安石,安石問以新政,載曰:「公與人為善,則人以善歸公;如教玉人琢玉,則宜有不受命者矣。」明州苗振獄起,往治之,末殺其罪。



 還朝,即移疾屏居南山下,終日危坐一室,左右簡編,俯而讀,仰而思,有得則識之,或中夜起坐,取燭以書。其
 志道精思,未始須臾息,亦未嘗須臾忘也。敝衣蔬食,與諸生講學,每告以知禮成性、變化氣質之道,學必如聖人而後已。以為知人而不知天,求為賢人而不求為聖人,此秦、漢以來學者大蔽也。故其學尊禮貴德、樂天安命,以《易》為宗,以《中庸》為體,以《孔》、《孟》為法,黜怪妄,辨鬼神。其家昏喪葬祭,率用先王之意,而傅以今禮。又論定井田、宅里、發斂、學校之法,皆欲條理成書,使可舉而措諸事業。



 呂大防薦之曰:「載之始終,善發明聖人之遺旨,其
 論政治略可復古。宜還其舊職,以備諮訪。」乃詔知太常禮院。與有司議禮不合,復以疾歸,中道疾甚,沐浴更衣而寢,旦而卒。貧無以斂,門人共買棺奉其喪還。翰林學士許將等言其恬於進取,乞加贈恤,詔賜館職半賻。



 載學古力行,為關中士人宗師,世稱為橫渠先生。著書號《正蒙》,又作《西銘》曰:



 乾稱父而坤母,予茲藐焉,乃混然中處。故天地之塞吾其體,天地之帥吾其性,民吾同胞,物吾與也。



 大君者,吾父母宗子;其大臣,宗子之家相也。尊
 高年所以長其長,慈孤幼所以幼其幼,聖其合德,賢其秀也。凡天下疲癃殘疾、恂獨鰥寡,皆吾兄弟之顛連而無告者也。「於時保之」,子之翼也。「樂且不憂」,純乎孝者也。違曰悖德,害仁曰賊,濟惡者不才,其踐形惟肖者也。



 知化則善述其事,窮神則善繼其志,不愧屋漏為無忝,存心養性為匪懈。惡旨酒,崇伯之子顧養;育英材,潁封人之錫類。不弛勞而底豫,舜其功也;無所逃而待烹,申生其恭也。體其受而歸全者,參乎;勇於從而順令者,伯奇
 也。富貴福澤,將厚吾之生也;貧賤憂戚,庸玉女於成也。存,吾順事;歿,吾寧也。



 程頤嘗言:「《西銘》明理一而分殊,擴前聖所未發,與孟子性善養氣之論同功,自孟子後蓋未之見。」學者至今尊其書。



 嘉定十三年,賜謚曰明公。淳祐元年封郿伯,從祀孔子廟庭。弟戩。



 戩,字天祺。起進士,調閿鄉主簿,知金堂縣。誠心愛人,養老恤窮,間召父老使教督子弟。民有小善,皆籍記之。以奉錢為酒食,月吉,召老者飲勞,使其子孫侍,勸以孝弟。
 民化其德,所至獄訟日少。



 熙寧初,為監察御史裏行。累章論王安石亂法,乞罷條例司及追還常平使者。劾曾公亮、陳升之、趙抃依違不能救正,韓絳左右徇從,與為死黨,李定以邪諂竊臺諫。且安石擅國,輔以絳之詭隨,臺臣又用定輩,繼續而來,芽蘗漸盛。呂惠卿劾薄辯給,假經術以文奸言,豈宜勸講君側。書數十上,又詣中書爭之,安石舉扇掩面而笑,戩曰:「戩之狂直宜為公笑,然天下之笑公者不少矣。」趙抃從旁解之,戩曰:「公亦不得
 為無罪。」抃有愧色。遂稱病待罪。



 出知公安縣,徙監司竹監,至舉家不食筍。常愛用一卒,及將代,自見其人盜筍籜,治之無少貸;罪已正,待之復如初,略不介意,其德量如此。卒於官,年四十七。



 邵雍字堯夫。其先範陽人,父古徙衡漳,又徙共城。雍年三十,游河南,葬其親伊水上,遂為河南人。



 雍少時,自雄其才,慷慨欲樹功名。於書無所不讀,始為學,即堅苦刻厲,寒不爐,暑不扇,夜不就席者數年。已而嘆曰:「昔人尚
 友於古,而吾獨未及四方。」於是逾河、汾,涉淮、漢,周流齊、魯、宋、鄭之墟,久之,幡然來歸,曰:「道在是矣。」遂不復出。



 北海李之才攝共城令,聞雍好學,嘗造其廬,謂曰:「子亦聞物理性命之學乎?」雍對曰:「幸受教。」乃事之才,受《河圖》、《洛書》、《宓義》八卦六十四卦圖像。之才之傳,遠有端緒,而雍探賾索隱,妙悟神契,洞徹蘊奧,汪洋浩博,多其所自得者。及其學益老,德益邵,玩心高明,以觀夫天地之運化,陰陽之消長,遠而古今世變,微而走飛草木之性情,深
 造曲暢,庶幾所謂不惑,而非依仿象類、億則屢中者。遂衍宓羲先天之旨,著書十餘萬言行於世,然世之知其道者鮮矣。



 初至洛,蓬蓽環堵,不芘風雨,躬樵爨以事父母,雖平居屢空,而怡然有所甚樂,人莫能窺也。及執親喪,哀毀盡禮。富弼、司馬光、呂公著諸賢退居洛中,雅敬雍,恆相從游,為市園宅。雍歲時耕稼,僅給衣食。名其居曰「安樂窩」,因自號安樂先生。旦則焚香燕坐,晡時酌酒三四甌,微醺即止,常不及醉也,興至輒哦詩自詠。春秋
 時出游城中,風雨常不出,出則乘小車,一人挽之,惟意所適。士大夫家識其車音,爭相迎候,童孺廝隸皆歡相謂曰:「吾家先生至也。」不復稱其姓字。或留信宿乃去。好事者別作屋如雍所居,以候其至,名曰「行窩」。



 司馬光兄事雍,而二人純德尤鄉里所慕向,父子昆弟每相飭曰:「毋為不善,恐司馬端明、邵先生知。」士之道洛者,有不之公府,必之雍。雍德氣粹然,望之知其賢,然不事表襮,不設防畛,群居燕笑終日,不為甚異。與人言,樂道其善而
 隱其惡。有就問學則答之,未嘗強以語人。人無貴賤少長,一接以誠,故賢者悅其德,不賢者服其化。一時洛中人才特盛,而忠厚之風聞天下。



 熙寧行新法,吏牽迫不可為,或投劾去。雍門生故友居州縣者,皆貽書訪雍,雍曰:「此賢者所當盡力之時,新法固嚴,能寬一分,則民受一分賜矣。投劾何益耶?」



 嘉祐詔求遺逸,留守王拱辰以雍應詔,授將作監主簿,復舉逸士,補潁州團練推官,皆固辭乃受命,竟稱疾不之官。熙寧十年,卒,年六十七,贈
 秘書省著作郎。元祐中賜謚康節。



 雍高明英邁,迥出千古,而坦夷渾厚,不見圭角,是以清而不激,和而不流,人與交久,益尊信之。河南程顥初侍其父識雍,論議終日,退而嘆曰:「堯夫,內聖外王之學也。」



 雍知慮絕人,遇事能前知。程頤嘗曰:「其心虛明,自能知之。」當時學者因雍超詣之識,務高雍所為,至謂雍有玩世之意;又因雍之前知,謂雍於凡物聲氣之所感觸,輒以其動而推其變焉。於是摭世事之已然者,皆以雍言先之,雍蓋未必然也。



 雍疾病,司馬光、張載、程顥、程頤晨夕候之,將終,共議喪葬事外庭,雍皆能聞眾人所言,召子伯溫謂曰:「諸君欲葬我近城地,當從先塋爾。」既葬,顥為銘墓,稱雍之道純一不雜,就其所至,可謂安且成矣。所著書曰《皇極經世》、《觀物內外篇》、《漁樵問對》,詩曰《伊川擊壤集》。



 子伯溫,別有
 傳。



\end{pinyinscope}