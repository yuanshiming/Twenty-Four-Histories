\article{志第一百三十九 食貨下八}

\begin{pinyinscope}

 商稅市易均輸互市舶法



 商稅凡州縣皆置務,關鎮亦或有之;大則專置官監臨,小則令、佐兼領;諸州仍令都監、監押同掌。行者繼貨,謂之「過稅」,每千錢算二十;居者市鬻,謂之「住稅」,每千錢算
 三十,大約如此。然無定制,其名物各隨地宜而不一焉。行旅繼裝,非有貨幣當算者,無得發篋搜索。凡販夫販婦細碎交易,嶺南商賈繼生藥及民間所織縑帛,非鬻於市者皆勿算。常稅名物,令有司件析頒行天下,揭於版,置官署屋壁,俾其遵守。應算物貨而輒藏匿,為官司所捕獲,沒其三分之一,以半畀捕者。販鬻而不由官路者罪之。有官須者十取其一,謂之「抽稅」。



 自唐室藩鎮多便宜從事,擅其征利,以及五季,諸國益務掊聚財貨以
 自贍,故征算尤繁。宋興,所下之國,必詔蠲省,屢敕官吏毋事煩苛、規羨餘以徼恩寵。大中祥符六年,始免諸路州軍農器之稅。



 諸州津渡舊皆有算,或水涸改置橋梁,有司猶責主者備償。建隆初,詔除滄、德、棣、淄、齊、鄆干渡三十九處算錢,水漲聽民置渡,勿收其算。自是,有類此者多因恩宥蠲除。其餘橘園、魚池、水磑、社酒、蓮藕、鵝鴨、螺蚌、柴薪、地鋪、枯牛骨、溉田水利等名,皆因諸國舊制,前後屢詔廢省。緣河州縣民船載粟亦輸算,三年,始罷。



 陳州私置蔡河鎖,民船勝百斛者取百錢,有所載倍其征,太平興國三年,乃悉除之。至道元年詔:「江南溪渡,多公吏豪民典其事,量輸官課而厚算行旅。州縣宜加嚴禁,所輸年額錢五千以下者並免,不系色役近便人戶掌船濟渡,毋得擾人。」至道中,歲入稅課錢四百萬貫;天禧末,增八百四萬貫。



 天聖以來,國用浸廣,有請算緡錢以助經費者。仁宗曰:「貨泉之利,欲流天下通有無,何可算也?」一日,內出蜀羅一端,為印朱所漬者數重,因詔天
 下稅務,毋輒污壞商人物帛。康定元年,西邊兵費不給,州縣或增所算名物,朝廷知之,悉命蠲去。既而下詔敕勵,且戒毋搜索行者家屬,歲儉則免算耕牛,水鄉又或弛蒲、魚、果、蓏之稅,民流而渡河者亦為之免算。應算而匿不自言者,雖聽人捕告,抵罪如舊法,然須物皆見在乃聽,以防誣罔。至於歲課贏縮,屢詔有司裁定,前後以詔蠲放者,不可勝數。



 皇祐中,歲課緡錢七百八十六萬三千九百。嘉祐以後,弛茶禁,所歷州縣收算錢。至治平
 中,歲課增六十餘萬,而茶稅錢居四十九萬八千六百。



 熙寧以來,河北、河東、陜西三路支移,民以租賦繼貨至邊貿易以輸官者,勿稅;河北流民復業者所過免算。後以歲稔,慮逸稅課,復舊。五年,以在京商稅院隸提舉市易務。七年,減國門之稅數十種,錢不滿三十者蠲之。其先,外城二十門皆責以課息,近令隨閑、要分等,以檢捕獲失之數為賞罰;既而以歲旱,復有是命。



 元豐元年,濱、棣、滄州竹木、魚果、炭箔稅不及百錢者蠲之。二年,熙河
 路制置邊防財用李憲擅榷本路商貨,令漕臣蔣之奇劾其罪。導洛通汴司請置堆垛場於泗州,賈物至者,先入官場,官以船運至京,稍輸船算。明年,詔:近京以通津水門外順成倉為場。非導洛司船而載商稅入汴者,許糾告,雖自請稅,猶如私載法。惟日用物非販易,若發箔、柴草、竹木之類勿禁。瓊管奏:「海南收稅,較船之丈尺,謂之『格納』。其法分三等,有所較無幾,而輸錢多寡十倍。賈物自泉、福、兩浙、湖、廣至者,皆金銀物帛,直或至萬餘緡;
 自高、化至者,唯米包、瓦器、牛畜之類,直才百一,而概收以丈尺。故高、化商人不至,海南遂乏牛米。請自今用物貴賤多寡計稅,官給文憑,聽鬻於部內,否則許糾告,以船貨給賞。」詔如所奏。六年,京東漕臣吳居厚言:「商人負正稅七萬六千餘緡,倍稅十五萬二千餘緡。」詔蠲其倍稅,納正稅,百千以下期以三年,百千以上五年。



 元祐元年,戶部請令在京商稅院,酌取元豐八年錢五十五萬二千二百六十一緡有奇,以為新額,自明年始。三年,又
 以天聖歲課為額,蓋戶部用五年並增之法,立額既重,歲課不登,故言者論而更之。七年,罷諸路承買土產稅場。初,罷江南路承買,而河東轉運司以為較元祐六年官鹽額增三萬餘緡,遂行之諸路。



 八年,權蠲商人載米入京糶賣力勝之稅。先是,熙寧六年,蘇、湖歲稔,穀價比淮南十五,而商船以力勝稅不至,嘗命權蠲。惠止一方,未為定法。及汴泗垛場法行,穀船毋得增置,而力勝之稅益三之一。至是,蘇軾言:「法不稅五穀,請削去力勝錢
 之條,而行天聖免稅之制。」既而尚書省亦言在京穀貴,欲平其直,復權蠲之。後徽宗宣和中,以州縣災傷並贍給都下,亦一再免,旋復如舊;惟兩浙並東北鹽,以鹽事司之請,遂不復徵。



 自哲宗即位,罷導洛物貨場。紹聖四年,藍從熙提舉京城所,欲復其事,令泗州及京師洛口各置垛場,並請復面市、牛羊圈。詔下尚書省,久之遂寢。至是,提舉汴河堤岸王憲復言之,且請假溫、明州運船給用。命太府少卿鄭僅同詳度。明年,竟詔勿行。五年,令
 戶部取天下稅務五年所收之數,酌多寡為中制,頒諸路揭版示之,率十年一易;其增名額及多稅者,並論以違制。



 大觀元年,凡典買牛畜、舟車之類未印契者,更期以百日,免倍稅。二年,詔在京諸門,凡民衣屨、穀菽、雞魚、蔬果、柴炭、瓷瓦器之類,並蠲其稅;歲終計所蠲數,令大觀庫給償。宣和二年,宮觀、寺院、臣僚之家商販,令關津搜閱,如元豐法輸稅,歲終以次數報轉運司取旨。初,元符令,品官供家服用物免稅。至建中靖國初,馬、牛、駝、驢、
 騾已不入服用例,而比年臣僚營私牟利者眾,宮觀寺院多有專降免稅之旨,皆以船艘賈販,故有是詔。漕臣劉既濟起應奉物,兩浙、淮南等路稅例外,增一分以供費;三年,詔罷之。凡以蠶織農具、耕牛至兩浙、江東者,給文憑蠲稅一年。四年,令諸路近歲所增稅錢,悉歸應奉司。七年,以歲歉之後,用物少而民艱食,在京及畿內油、炭、面、布、絮稅並力勝錢並權免。提舉京東常平楊連奏:「本路牛價貴,田多荒萊,請令販牛至本路者,仍給文憑
 蠲稅,俟二年足如舊。」從之。



 靖康元年詔:「都城物價未平,凡稅物,權更蠲稅一年。」臣僚上言:「祖宗舊制並政和新令,場務立額之法,並以五年增虧數較之,並增者取中數,並虧者取最高數,以為新額,故課息易給而商旅可通。近諸路轉運司不循其法,有益無損,致物價騰踴,官課愈負。請令諸路提刑下諸郡,準舊法厘正立額。」詔依所奏。



 高宗建炎元年詔,販貨上京者免稅。明年又詔,販糧草入京抑稅者罪之;凡殘破州縣免竹木、磚瓦稅,北
 來歸正人及兩淮復業者亦免路稅。紹興三年,臨安火,免竹木稅。然當時都邑未奠,兵革未息,四方之稅,間有增置,及於江灣浦口量收海船稅,凡官司回易亦並收稅;而寬弛之令亦錯見焉,如諸路增置之稅場,山間迂僻之縣鎮,經理未定之州郡,悉罷而免之。又以稅網太密,減並者一百三十四,罷者九,免過稅者五,至於牛、米、薪、面民間日用者並罷。



 孝宗繼志,凡高宗省罷之未盡者,悉推行之;又以臨安府物價未平,免淳熙七年稅一
 半。光、寧以降,亦屢與放免商稅,或一年,或五月,或三月。凡遇火,放免竹木之稅亦然。光、寧嗣服,諸郡稅額皆累有放免。然當是時,雖寬大之旨屢頒,關市之徵迭放,而貪吏並緣,苛取百出。私立稅場,算及緡錢、斗米、束薪、菜茹之屬,擅用稽察措置,添置專欄收檢。虛市有稅,空舟有稅,以食米為酒米,以衣服為布帛,皆有稅。遇士夫行李則搜囊發篋,目以興販。甚者貧民貿易瑣細於村落,指為漏稅,輒加以罪。空身行旅,亦白取百金,方紆路避
 之,則欄截叫呼;或有貨物,則抽分給賞,斷罪倍輸,倒囊而歸矣。聞者咨嗟,指為大小法場,與斯民相刃相劘,不啻仇敵,而其弊有不可勝言矣。



 市易之設,本漢平準,將以制物之低昂而均通之。其弊也,以官府作賈區,公取牙儈之利,而民不勝其煩矣。



 熙寧三年,保平軍節度推官王韶倡而緣邊市易之說,丐假官錢為本。詔秦鳳路經略司以川交子易物貨給之,因命韶為本路帥司乾當兼領市易事。時欲移司於古
 渭城,李若愚等以為多聚貨以啟戎心,又妨秦州小馬、大馬私貿易,不可。文彥博、曾公亮、馮京皆韙之,韓絳亦以去秦州為非,唯王安石曰:「古渭置市易利害,臣雖不敢斷,然如若愚奏,必無可慮。」七月,詔轉運司詳度,復問陳升之。升之謂古渭極邊,恐啟群羌窺覬心。安石乃言:「今蕃戶富者,往往蓄緡錢二三十萬,彼尚不畏劫奪,豈朝廷威靈,乃至衰弱如此?今欲連生羌,則形勢欲張,應接欲近。古渭邊砦,便於應接,商旅並集,居者愈多,因建
 為軍,增兵馬,擇人守之,則形勢張矣。且蕃部得與官市,邊民無復逋負,足以懷來其心,因收其贏以助軍費,更闢荒土,異日可以聚兵。」時王安石為政,汲汲焉以財利兵革為先,其市易之說,已見於熙寧二年建議立均輸平準法之時,故王韶首迎合其意,而安石力主之,雖以李若愚、陳升之、韓絳諸人之議,而卒不可回。五年,遂詔出內帑錢帛,置市易務於京師。



 先是,有魏繼宗者,自稱草澤,上言:「京師百貨無常價,貴賤相傾,富能奪,貧能與,
 乃可以為天下。今富人大姓,乘民之亟,牟利數倍,財既偏聚,國用亦屈。請假榷貨務錢,置常平市易司,擇通財之官任其責,求良賈為之轉易。使審知市物之價,賤則增價市之,貴則損價鬻之,因收餘息,以給公上。」於是中書奏在京置市易務官。凡貨之可市及滯於民而不售者,平其價市之,願以易官物者聽。若欲市於官,則度其抵而貸之錢,責期使償,半歲輸息十一,及歲倍之。凡諸司配率,並仰給焉。以呂嘉問為提舉,賜內庫錢百萬緡、
 京東路錢八十七萬緡為本。三司請立市易條,有「兼並之家,較固取利,有害新法,本務覺察,三司按治」之文,帝削去之。



 七月,以榷貨務為市易西務下界,市易務為東務上界,以在京商稅院、雜買務、雜賣場隸焉。又賜錢帛五十萬,於鎮洮軍置司。市易極苛細,道路怨謗者籍籍。上以諭安石,請宣示事實,帝以鬻冰、市梳樸等數事語之,安石皆辯解。後帝復言:「市易鬻果太煩碎,罷之如何?」安石謂:「立法當論有害於人與否,不當以煩碎廢也。」自
 是諸州上供藨席、黃蘆之類六十色,悉令計直,從民願鬻者市之以給用。



 六年,詔在京市易乾當公事孫迪同兩浙、淮東轉運司,議置杭州市易務利病以聞。其後以市易上界所償內帑錢二十萬緡假之為本。又賜夔州路轉運司度僧牒五百,置市易於黔州,選本路在任已替官監之,仍以知州或通判提舉。令在京市易務及開封府司錄同詳度諸行利病,於是詳定所請:「約諸行利入薄厚,輸免行錢以祿吏,蠲其供官之物。禁中所須,並
 下雜賣場、雜買務。置市司估物價低昂,凡內外官司欲占物價,悉於是乎取決。」從之。改提舉在京市易務為都提舉市易司,諸州市易務皆隸焉。又詔三司乾當公事李□巳等同詳度成都置市易務。



 七年,帝與輔臣論及成都市易事。馮京曰:「曩因榷市物,致王小波之亂,今頗以市易為言。」安石曰:「彼以饑民眾,官不之恤,相聚為盜耳。」帝問:「李□巳行邪?」安石曰:「未也。然保市易必不能致亂。」帝猶慮蜀人駭擾,安石謂:「已遣使乃遽罷,豈不為四方笑?」
 乃已。然其後竟罷□巳等詳度。



 三月,詔權三司使曾布、翰林學士呂惠卿同究詰市易事。先是,帝出手詔付布,謂市易司市物,頗害小民之業,眾言喧嘩。布乃引監市易務魏繼宗之言,以為呂嘉問多取息以干賞,商旅所有者盡收,市肆所無者必索,率賤市貴鬻,廣裒贏餘,是挾官府為兼並也。王安石具奏,明其不然。乃更令惠卿偕布究詰之。帝尋復以手札賜布,令求對,布即上行人所訴,並疏惠卿奸欺狀,且言:「臣自立朝以來,每聞德音,未
 嘗不欲以王道治天下,今市易之為虐,凜凜乎間架、除陌之事矣。嘉問奏:『近遣官往湖南販茶,陜西販鹽,兩浙販紗,皆未敢計息。』臣以謂如此政事,書之簡牘,不獨唐、虞、三代所無,歷觀秦、漢以來衰亂之世,恐未之有也。」四月,布復陳薛向罪茶儈不當,帝惻然咨嗟;及言三司決責商人多濫,時帝猶必欲按治。而安石主用惠卿不可去,蓋謀變其事也。帝疑焉,故仍以屬布。



 既而中書奏事已,帝論及市易,且曰:「朝廷設此,本欲為平準之法以便
 民,今正爾相反,使中下之民失業若此,宜修補其法。」令元詳定呂嘉問、吳安持同韓維、孫永問行人輸錢免行利病。參知政事馮京曰:「開封祥符縣給民錢,有出息抵當銀絹米麥、緩急喪葬之目七八種。其初給錢,往往願請,積數既多,實艱輸送。」帝曰:「如此,吾民安得泰然也?」時布與惠卿方究市易事,率數日一對。帝初是布言,已而從惠卿之請,拘魏繼宗於開封府。既而布與惠卿即東府再詰行人,所訴狀如前不變。而安石懇求去位,引惠
 卿執政。



 提舉楚州市易蔣之奇奏:「監務王景彰榷市商人物非法,及虛作中糴入務,立詭名糴之,白輸息錢,謂之『乾息』;又抑賈販毋得至他郡,名為留難。」帝謂輔臣曰:「景彰違法害人,宜即治其罪。」時呂惠卿已參朝政,而究詰市易未竟,詔促之,惠卿請令中書悉取按牘異同以奏。後二日,布對延和殿,條祈先後所陳,並較治平、熙寧出入錢物數以聞。帝方慮歲費浸廣,令布送中書。五月,乃詔章惇、曾孝寬即軍器監鞫布所究市易事,又令戶
 房會財賦數,與布所陳異;而呂嘉問亦以雜買務多入月息不覺,皆從公坐有差。未幾,布褫職,與嘉問俱出守郡,魏繼宗仍奪秩勒停。初,市易之建,布實預之。後揣上意有疑,遂急治嘉問,而惠卿與布有夙怨,故卒擠之,而市易如故。



 三司使章惇請假內藏錢五百萬緡,令市易司有幹局者,分四路入中,計見鹽引及乘賤糴買。詔假二百萬緡。八年,復呂嘉問提舉市易。二月,鳳翔、大名、真定府、永興、安肅軍,秦、瀛、定、越、真州,並置市易司。以惠州
 阜民監錢十萬緡給廣州市易務,司農寺坊場錢三十萬緡給鄆州市易。九年,又以在京市易司物貨十五萬緡給熙河市易司。九月,中書言:「市易息錢並市例錢,總收百三十三萬二千緡有奇。詔嘉問、安持等推恩有差。自後凡二年一較。十年,定上界本錢以七百萬緡為額,不足,以歲所收息益之;其貸內帑錢,歲償以息二十萬緡。



 元豐元年,以都提舉王居卿請,令貸市易錢貨者,許用金帛等為抵,收息毋過一分二厘,不及年者月計之,
 願皆得錢或欲以物貨兼給者聽。市易司請遣官以物貨至諸路貿易,十萬緡以上期以二年,二十萬緡以上三年,斂及三分者比遞年推恩,八分者理為任,期盡不及者勿賞,官吏廩給並罷。



 二年,經制熙河路邊防財用李憲言:蕃賈與牙儈私市,其貨皆由他路避稅入秦州。乃令秦熙河岷州、通遠軍五市易務,募牙儈引蕃貨赴市易務中賈,私市者許糾告,賞倍所告之數。以田宅抵市易錢久不償者,估實直,如賣坊場、河渡法;若未輸錢
 者,官收其租息,在京市易務亦如之。



 三年,詔免行月納錢不及百者皆免,凡除八千六百五十四人。九月,王居卿又言:「市易法有三:結保貸請,一也;契要金銀為抵,二也;貿遷物貨,三也。三者惟保貸法行之久,負失益多,往歲罷貸錢而物貨如故。請自今所貸歲約毋過二百萬緡,聽舊戶貸請以相濟續,非舊戶惟用抵當、貿遷之法。」詔中書立法以聞。於是中書奏:「在京物貨,許舊戶貸請,斂而復散,通所負毋過三百萬緡,諸路毋過四之一。」詔
 如所奏。是歲,經制熙河邊防財用司會其置司以來所收息:元豐初四十一萬四千六百二十六緡、石,次年六十八萬四千九十九緡、石。四年,從都提舉賈青請,於新舊城外內置四抵當,遣官掌之,罷市易上界等處抵當以便民。



 五年,詔外內市易務所負錢,寬以三歲,均月限以輸,限內罰息並除之。先是,王安禮在開封日,有負市易錢者,累訴於庭。安禮既執政,言於帝曰:「市易法行,取息滋多,而輸官不時者有罰息,民至窮困。願詔蠲之。」帝
 曰:「群臣未有為朕言者,其令民以限輸,免其罰息。」安禮退,批詔加「內外」字。蔡確曰:「方帝有旨,無外內字,公欲增詔邪?」安禮曰:「亦不止言內字。」卒加之。八月,置饒州景德鎮瓷窯博易務。



 六年,蘭州增置市易務,以通蕃漢貿易。七年,改市易下界為榷貨務。令諸州旬估物價既定,報提舉司,提舉司下所部州,州下所屬,募民出抵或錢以市,收息毋過二分。詔諸路常平司錢留其半,以二分為市易抵當。蓋自五年賈青以平準物價與金銀之類,行
 抵當於畿縣,次年行之諸路,以常平、市易賒貸及寬剩錢為本,五路各十萬緡,餘路五萬緡。至是,復有是詔。若無抵當而物貨宜易者,亦聽變鬻。八年,罷諸鎮砦市易抵當。八月,詔諸郡抵當,有取息薄、可濟民乏者存之,其餘抵當並州縣市易並罷。



 元祐元年,內外監督市易及坊場凈利錢,許以所入息並罰錢比計,若及官本者,並釋之。紹聖四年,三省言熙寧興置市易,元祐一切罷去,不原立法之意。詔戶部、太府寺詳度,復置市易務,惟以
 錢交市,收息毋過二分,勿令貸請。元符三年,改市易務為平準務,戶部、太府寺市易案改為平準案。尚書省言:「平準務官吏等給費多,並遣官市物,搔動於外,近官鬻石炭,市直遽增,皆不便民。」詔罷平準務及官鬻石炭,其在官物貨,令有司轉易錢鈔,償元給之所。



 崇寧元年,戶部奏:平準務錢物毋得他司移用。二年,以平準為南北兩務,如舊分置官吏。歲終考察能否,行勸沮法。五年,郡縣應置市易者,凡歲收息,官吏用度之餘,及千緡以上
 置官監,五百緡以上令場務兼領,餘並罷。先是,嘗詔府界萬戶縣及路在沖要,市易抵當已設官置局;其不及萬戶、非沖要,並諸鎮有官監而商販所會,並如元豐令監當官兼領。至是,戶部復詳度以聞,遂行其議。建炎二年,言者以為得不償費,遂罷之,而以其錢輸左藏庫,惟抵當庫仍舊。



 紹興元年,罷諸州軍免行錢及行戶供應,見任官買賣並依時,違者以盜論。四年,兩浙轉運司檄婺州市禦爐炭,須胡桃紋、鵓鳩色,守臣王居正以為言。
 上曰:「隆冬附火,取溫暖而已,豈問炭之紋色乎?」命罷之,諸類此者並禁止焉。十三年,蠲雷、化、高、融、宜、廉、邕、欽、賀、貴免行錢。十四年,以開州兩縣在夔部尤為僻遠,減免行錢之半。十五年,以知漢陽軍韓昕言,諸路收免行錢,定數外多取一文以上,以擅增稅賦法罪之。十七年,蠲百姓見輸免行錢三分之一。十九年,南郊赦,盡蠲百姓免行錢欠。是後凡赦皆然。二十五年,罷見輸免行錢,禁下行買物,以害及小商、敷於鄉村故也。



 淳熙元年,罷市
 令司。詔臨安府及屬縣交易儈保錢減十之五。七年,諸路州縣交易儈保錢,亦以十分為率,與減五分。



 嘉定二年,以臣僚言,輦轂之下,買物於鋪戶,無從得錢。凡臨安府未支物價,令即日盡數給還,是後買物須給見錢,違許陳訴於臺。



 嘉熙三年,臣僚言:「今官司以官價買物,行鋪以時直計之,什不得二三。重以遷延歲月而不償,胥卒並緣之無藝,積日既久,類成白著,至有遷居以避其擾、改業以逃其害者。甚而蔬菜魚肉,日用所需瑣瑣之
 物,販夫販婦所資錐刀以營鬥升者,亦皆以官價強取之。終日營營,而錢本俱成乾沒。商旅不行,衣食路絕。望特降睿旨,凡諸路州縣官司買物,並以時直;不許輒用官價,違者以贓定罪。」從之。



 均輸之法,所以通天下之貨,制為輕重斂散之術,使輸者既便,而有無得以懋遷焉。



 熙寧二年,制置三司條例司言:「天下財用無餘,典領之官拘於弊法,內外不相知,盈虛不相補。諸路上供,歲有常數。豐年便道,可以多致
 而不能贏;年儉物貴,難於供億而不敢不足。遠方有倍蓰之輸,中都有半價之鬻,徒使富商大賈乘公私之急,以擅輕重斂散之權。今發運使實總六路賦入,其職以制置茶、鹽、礬、酒稅為事,軍儲國用,多所仰給。宜假以錢貨,資其用度,周知六路財賦之有無而移用之。凡糴買稅斂上供之物,皆得徙貴就賤,用近易遠。令預知中都帑藏年支見在之定數,所當供辦者,得以從便變易蓄買,以待上令。稍收輕重斂散之權歸之公上,而制其有
 無,以便轉輸,省勞費,去重斂,寬農民。庶幾國用可足,民財不匱。」詔本司具條例以聞,而以發運使薛向領均輸平準事,賜內藏錢五百萬緡、上供米三百萬石。時議慮其為擾,多以為非。向既董其事,乃請設置官屬,神宗使自擇之。向於是闢劉忱、衛琪、孫珪、張穆之、陳倩為屬,又請有司具六路歲當上供數、中都歲用及見儲度可支歲月,凡當計置歲何,皆預降有司。從之。



 八月,侍御史劉琦、侍御史裏行錢顗等言:「向小人,假以貨泉,任其變易,
 縱有所入,不免奪商賈之利。」琦、顗皆坐貶。條例司檢詳文字蘇轍言:「昔漢武外事四夷,內興宮室,財用匱竭,力不能支,用賈人桑弘羊之說,買賤賣貴,謂之均輸。雖曰民不加賦而國用饒足,然法術不正,吏緣為奸,掊克日深,民受其病。孝昭既立,學者爭排其說,霍光順民所欲,從而予之,天下歸心,遂以無事。今此論復興,眾口紛然,皆謂其患必甚於漢。何者?方今聚斂之臣,材智方略,未見有桑弘羊比;而朝廷破壞規矩,解縱繩墨,使得馳騁
 自有,唯利是嗜,其害必有不可勝言者矣。」轍亦坐去官。



 於是知諫院范純仁言:「向憸巧刻薄,不可為發運使。人主當務農桑、節用,不當言利。」自後,罷純仁諫職,而諫官李常復論均輸不便,權開封府推官蘇軾亦言:「均輸徙貴就賤,用近易遠。然廣置官屬,多出緡錢,豪商大賈皆疑而不敢動,以為雖不明言販賣,既已許之變易,變易既行,而不與商賈爭利,未之聞也。夫商賈之事,曲折難行,其買也先期而予錢,其賣也後期而取直,多方相濟,
 委曲相通,倍稱之息,由此而得。今先設官置吏,簿書廩祿,為費已厚,非良不售,非賄不行。是官買之價比民必貴,及其賣也,弊復如前,商賈之利,何緣而得?朝廷不知慮此,乃捐五百萬緡以予之。此錢一出,恐不可復。縱使其間薄有所獲,而征商之額所損必多矣。」



 帝方惑於安石之說,言皆不行。乃以向為天章閣待制,遣太常少卿羅拯為使,手詔賜向曰:「政事之先,理財為急。朕托卿以東南賦入,皆得消息盈虛、翕張斂散之。而卿忠誠內固,
 能倡舉職業,導揚朕意,底於成績,朕甚嘉之。覽奏慮流言致惑,朕心匪石,豈易轉也?卿其濟之以強,終之以不倦,以稱朕意。」然均輸後迄不能成。



 互市舶法自漢初與南越通關市,而互市之制行焉。後漢通交易於烏桓、北單于、鮮卑,北魏立互市於南陲,隋、唐通貿易於西北。開元定令,載其條目,後唐亦然。而高麗、回鶻、黑水諸國,又各以風土所產與中國交易。



 宋初,循周制,與江南通市。乾德二年,禁商旅毋得渡江,於建
 安、漢陽、蘄口置三榷署,通其交易;內外群臣輒遣人往江、浙販易者,沒入其貨。緣江百姓及煎鹽亭戶,恣其樵漁,所造屨席之類,榷署給券,聽渡江販易。開寶三年,徙建安榷署於揚州。江南平,榷署雖存,止掌茶貨。四年,置市舶司於廣州,後又於杭、明州置司。凡大食、古邏、闍婆、占城、勃泥、麻逸、三佛齋諸蕃並通貨易,以金銀、緡錢、鉛錫、雜色帛、瓷器,市香藥、犀象、珊瑚、琥珀、珠琲、鑌鐵、PZ皮、玳瑁、瑪瑙、車渠、水精、蕃布、烏樠、蘇木等物。



 太宗時,置榷
 署於京師,詔諸蕃香藥寶貨至廣州、交址、兩浙、泉州,非出官庫者,無得私相貿易。其後乃詔:「自今惟珠貝、玳瑁、犀象、鑌鐵、PZ皮、珊瑚、瑪瑙、乳香禁榷外,他藥官市之餘,聽市於民。」



 雍熙中,遣內侍八人繼敕書金帛,分四路招致海南諸蕃。商人出海外蕃國販易者,令並詣兩浙司市舶司請給官券,違者沒入其寶貨。淳熙二年,詔廣州市舶,除榷貨外,他貨之良者止市其半。大抵海船至,十先徵其一,價直酌蕃貨輕重而差給之,歲約獲五十餘
 萬斤、條、株、顆。太平興國初,私與蕃國人貿易者,計直滿百錢以上論罪,十五貫以上黥面流海島,過此送闕下。淳化五年申其禁,至四貫以上徒一年,稍加至二十貫以上,黥面配本州為役兵。



 天聖以來,像犀、珠玉、香藥、寶貨充牣府庫,嘗斥其餘以易金帛、芻粟,縣官用度實有助焉。而官市貨數,視淳化則微有所損。皇祐中,總歲入象犀、珠玉、香藥之類,其數五十三萬有餘。至治平中,又增十萬。



 熙寧五年,詔發運使薛向曰:「東南之利,舶商居
 其一。比言者請置司泉州,其創法講求之。」七年,令泊船遇風至諸州界,亟報所隸,送近地舶司榷賦分買;泉、福瀕海舟船未經賦買者,仍赴司勘驗。時廣州市舶虧歲課二十萬緡,或以為市易司擾之,故海商不至,令提舉司究詰以聞。既而市易務呂邈入舶司闌取蕃商物,詔提舉司劾之。九年,集賢殿修撰程師孟請罷杭、明州市舶,諸舶皆隸廣州一司。令師孟與三司詳議之。是年,杭、明、廣三司市舶,收錢、糧、銀、香、藥等五十四萬一百七十
 三緡、匹、斤、兩、段、條、個、顆、臍、只、粒,支二十三萬八千五十六緡、匹、斤、兩、段、條、個、顆、臍、只、粒。



 元豐二年,賈人入高麗,貲及五千緡者,明州籍其名,歲責保給引發船,無引者如盜販法。先是,禁人私販,然不能絕;至是,復通中國,故明立是法。



 三年,中書言,廣州市舶已修定條約,宜選官推行。詔廣東以轉運使孫迥,廣西以陳倩,兩浙以副使周直孺,福建以判官王子京,罷廣東帥臣兼領。五年,廣西漕臣吳潛言:「雷、化州與瓊島對境,而發船請引於廣
 州舶司,約五千里。乞令廣西瀕海郡縣,土著商人載米穀、牛酒、黃魚及非舶司賦取之物,免至廣州請引。」詔孫迥詳度行之。



 知密州範鍔言:「板橋瀕海,東則二廣、福建、淮、浙,西則京東、河北、河東三路,商賈所聚,海舶之利顓於富家大姓。宜即本州置市舶司,板橋鎮置抽解務。」六年,詔都轉運使吳居厚條析以聞。



 元祐三年,鍔等復言:「廣南、福建、淮、浙賈人,航海販物至京東、河北、河東等路,運載錢帛絲綿貿易,而象犀、乳香珍異之物,雖嘗禁榷,
 未免欺隱。若板橋市舶法行,則海外諸物積於府庫者,必倍於杭、明二州。使商舶通行,無冒禁罹刑之患,而上供之物,免道路風水之虞。」乃置密州板橋市舶司。而前一年,亦增置市舶司於泉州。



 賈人由海道往外蕃,令以物貨名數並所詣之地,報所在州召保,毋得參帶兵器或可造兵器及違禁之物,官給以券。擅乘船由海入界河及往高麗、新羅、登萊州境者,罪以徒,往北界者加等。



 崇寧元年,復置杭、明市舶司,官吏如舊額。三年,令蕃商
 欲往他郡者,從舶司給券,毋雜禁物、奸人。初,廣南舶司言,海外蕃商至廣州貿易,聽其往還居止,而大食諸國商亦丐通入他州及京東販易,故有是詔。凡海舶欲至福建、兩浙販易者,廣南舶司給防船兵仗,如詣諸國法。廣南舶司鬻所市物貨,取息毋過二分。政和三年,詔如至道之法,凡知州、通判、官吏並舶司、使臣等,毋得市蕃商香藥、禁物。



 宣和元年,秀州開修青龍江浦,泊船輻輳,請復置監官。先是,政和中,置務設官於華亭縣,後江浦
 湮塞,蕃舶鮮至,止令縣官兼掌。至是,復設官專領焉。四年,蕃國進奉物,如元豐法,令舶司即其地鬻之,毋發至京師,違者論罪。



 契丹在太祖時,雖聽緣邊市易,而未有官署。太平興國二年,始令鎮、易、雄、霸、滄州各置榷務,輦香藥、犀象及茶與交易。後有範陽之師,罷不與通。雍熙三年,禁河北商民與之貿易。時累年興師,千里饋糧,居民疲乏,太宗亦頗有厭兵之意。端拱元年,詔曰:「朕受命上穹,居尊中土,惟思禁暴,豈欲窮兵?至於幽薊之民,皆
 吾赤子,宜許邊疆互相市易。自今緣邊戍兵,不得輒恣侵略。」未幾復禁,違者抵死,北界商旅輒入內地販易,所在捕斬之。淳化二年,令雄、霸州、靜戎軍、代州雁門砦置榷署如舊制,所鬻物增蘇木,尋復罷。



 咸平五年,契丹求復置署,朝議以其翻覆,不許。知雄州何承矩繼請,乃聽置於雄州;六年,罷。景德初,復通好,請商賈即新城貿易。詔北商繼物貨至境上則許之。二年,令雄、霸州、安肅軍置三榷場,北商趨他路者,勿與為市。遣都官員外郎孔
 揆等乘傳詣三榷場,與轉運使劉綜並所在長吏平互市物價,稍優其直予之。又於廣信軍置場,皆廷臣專掌,通判兼領焉。三年,詔民以書籍赴沿邊榷場博易者,非《九經》書疏悉禁之。凡官鬻物如舊,而增繒帛、漆器、粳糯,所入者有銀錢、布、羊馬、橐駝,歲獲四十餘萬。



 天聖中,知雄州張昭遠請歲會入中金錢,仁宗曰:「先朝置互市以通有無,非以計利。」不許。終仁宗、英宗之世,契丹固守盟好,互市不絕。



 熙寧八年,市易司請假奉宸庫象、犀、珠直
 總二十萬緡,於榷場貿易,明年終償之。詔許。九年,立與化外人私貿易罪賞法。河北四榷場,自治平四年,其貨物專掌於三司之催轄司,而度支賞給案判官置簿督計之。至是,以私販者眾,故有是命。未幾,又禁私市硫黃、焰硝及以盧甘石入他界者,河東亦如之。元豐元年,復申賣書北界告捕之法。



 西夏自景德四年,於保安軍置榷場,以繒帛、羅綺易駝馬、牛羊、玉、氈毯、甘草,以香藥、瓷漆器、姜桂等物易蜜蠟、麝臍、毛褐、羱羚角、□岡砂、柴胡、蓯
 蓉、紅花、翎毛,非官市者聽與民交易,入貢至京者縱其為市。



 天聖中,陜西榷場二、並代路亦請置場和市,許之。及元昊反,即詔陜西、河東絕其互市,廢保安軍榷場;後又禁陜西並邊主兵官與屬羌交易。久之,元昊請臣,數遣使求復互市。慶歷六年,復為置場於保安、鎮戎二軍。繼言驅馬羊至,無放牧之地,為徙保安軍榷場於順寧砦。既而蕃商卒無至者。嘉祐初,西人侵耕屈野河地,知並州龐籍謂:「非絕其互市,則內侵不已。且聞出兀臧訛
 龐之謀,若互市不通,其國必歸罪訛龐,年歲間,然後可與計議。」從之。初,第禁陜西四路私與西人貿易,未幾,乃悉絕之。



 治平四年,河東經略司言,西界乞通和市。自夏人攻慶州大順城,詔罷歲賜,嚴禁邊民無得私相貿易。至是,上章謝罪,乃復許之。後二年,令涇原熟戶及河東、陜西邊民勿與通市。又二年,因回使議立和市,而私販不能止,遂申詔諸路禁絕。既而河東轉運司請罷吳堡,於寧星和市如舊。而麟州復奏夏人之請,乃令鬻銅、錫
 以市馬,而纖縞與急須之物皆禁。西北歲入馬,事具《兵志》。



 楚、蜀、南粵之地,與蠻獠溪峒相接者,以及西州沿邊羌戎,皆聽與民通市。熙寧三年,王韶置市易司於秦鳳路古渭砦,六年,增置市易於蘭州。自後,於熙、河、蘭、湟、慶、渭、延等州,又各置折博務。湖北路及沅、錦、黔江口,蜀之黎、雅州皆置博易場。重和元年,燕瑛言交人服順久,毋令阻其貿易。初,廣西帥曾布請即欽、廉州各創驛,令交人就驛博買。至是,即用瑛兼廣西轉運副使,同王蕃計
 畫焉。



 建炎四年三月,宣撫使張浚奏,大食國遣人進珠玉寶貝。上曰:「大觀、宣和間,川茶不以博馬,惟市珠玉,故武備不修,遂致危弱如此。今復捐數十萬緡易無用之物,曷若惜財以養戰士乎?」諭張浚勿受,量賜予以答之。六月,罷宜州歲市朱砂二萬兩。



 紹興三年,邕州守臣言大理請入貢。上諭大臣,止令賣馬,不許其進貢。四年,詔川、陜即永興軍、威茂州置博易場;移廣西買馬司於邕管,歲捐金帛,倍酬其直。然言語不通,一聽譯者高下其
 手,吏得因緣為奸。六年,大理國獻象及馬五百匹,詔償其馬直,卻像勿受,而賜書勞遣之。十二年,盱眙軍置榷場官監,與北商博易,淮西、京西、陜西榷場亦如之。十九年,罷國信所博易。二十六年,罷廉州貢珠,散蜑丁。蓋珠池之在廉州凡十餘,按交址者水深百尺,而大珠生焉。蜑往採之,多為交人所取,又為大魚所害。至是,罷之。二十九年,存盱眙軍榷場,餘並罷。



 乾道元年,襄陽鄧城鎮、壽春花靨鎮、光州光山縣中渡市皆置榷場,以守臣措
 置,通判提轄。五年,省提轄官。淳熙二年,臣僚言:溪峒緣邊州縣置博易場,官主之。七年,塞外諸戎販珠玉入黎州,官常邀市之。臣僚言其黷貨啟釁,非便,止合聽商賈、百姓收買。詔從之。



 建炎元年,詔:「市舶多以無用之物費國用,自今有博買篤耨香環、瑪瑙、貓兒眼睛之類,皆置於法;惟宣賜臣僚象笏、犀帶,選可者輸送。」胡人謂三百斤為一婆蘭,凡舶舟最大者曰獨檣,載一千婆蘭。次者曰牛頭,比獨檣得三之一。又次曰木舶,曰料河,遞得三
 之一。



 隆興二年,臣僚言:「熙寧初,立市舶以通物貨。舊法抽解有定數,而取之不苛,輸稅寬其期,而使之待價,懷遠之意實寓焉。邇來抽解既多,又迫使之輸,致貨滯而價減。擇其良者,如犀角、象齒十分抽二,又博買四分;珠十分抽一,又博買六分。舶戶懼抽買數多,止買粗色雜貨。若象齒、珠犀比他貨至重,乞十分抽一,更不博買。」



 乾道二年,罷兩浙路提舉,以守倅及知縣、監官共事,轉運司提督之。三年,詔廣南、兩浙市舶司所發舟還,因風水
 不便、船破檣壞者,即不得抽解。七年,詔見任官以錢附綱首商旅過蕃買物者有罰,舶至除抽解和買,違法抑買者,許蕃商越訴,計贓罪之。



 舊法,細色綱龍腦、珠之類,每一綱五千兩,其餘犀象、紫礦、乳檀香之類,為粗色,每綱一萬斤。凡起一綱,遣衙前一名部送,支腳乘贍家錢一百餘緡。大觀以後,張大其數,像犀、紫礦皆作細色起發,以舊日一綱分為三十二綱,多費腳乘贍家錢三千餘貫。至於乾道七年,詔廣南起發粗色香藥物貨,每綱
 二萬斤,加耗六百斤,依舊支破水腳錢一千六百六十二貫有奇。淳熙二年,戶部言:「福建、廣南市舶司粗細物貨,並以五萬斤為一全綱。」



 南渡,三路舶司歲入固不少,然金銀銅鐵,海舶飛運,所失良多,而銅錢之洩尤甚。法禁雖嚴,奸巧愈密,商人貪利而貿遷,黠吏受賕而縱釋,其弊卒不可禁。



\end{pinyinscope}