\article{列傳第一百七十}

\begin{pinyinscope}

 ○湯璹
 蔣重珍牟子才朱貔孫歐陽守道



 湯璹,字君寶,瀏陽人。淳熙十四年進士,調德安府學教授,轉三省樞密院架閣,遷國子博士。時召朱熹為侍講,
 未幾辭歸,朝廷從其請,予祠。璹上疏言:「熹以正學為講官,四方顒望其有啟沃之益。曾未逾時,輒聽其去,必駭物論。宜追召熹還,仍授講職。」疏上,不報。由是浸惡權相意,而璹之直聲亦大聞於時。歷禮部、駕部二郎官,出知常州,入為大理少卿,進直徽猷閣,卒。



 璹負直概,與韓侂胄、陳自強不合,故屢嗾言者中傷。璹生平奉祠閑居之日,多於揚歷,其在禮曹,例掌三省奏記。臨安大火,寧宗遇災避正殿,中書三表請復,不許。璹屬辭務持大體,不
 為阿曲,言者摭其語涉訕上,而朝廷實知其無他,故起復制詞有「清風峻節」之語。璹嘗擇婿得蔣重珍,後舉進士第一。



 蔣重珍,字良貴,無錫人。嘉定十六年進士第一,簽判建康軍,丁母憂,改昭慶軍,尋以公事與部使者異議,請祠,易簽判奉國軍。紹定二年,召入對,首以「自天子至於庶人所當先知者本心外物二者之界限」為言:「界限明,則知有天下治亂而已,何樂其尊;知有生民休戚而已,何
 樂其奉。」且論:「苞苴有昔所未有之物,故吾民罹昔所未有之害;苞苴有不可勝窮之費,故吾民有不可勝窮之憂。」遷秘書省正字,屢乞祠,以伯父喪予告,遷校書郎,辭,不可。明年,待命霅川,移文閣門,請對,當路憚之,添差通判鎮江府,辭。會行都火,應詔曰:「



 臣頃進本心外物界限之說,蓋欲陛下親攬大柄,不退托於人,盡破恩私,求無愧於己。儻以富貴之私視之,一言一動,不忘其私,則是以天下生靈、社稷宗廟之事為輕,而以一身富貴之所
 從來為重,不惟上負天命,以先帝聖母至於公卿百執事之所以望陛下者,亦不如此也。昔周勃今日握璽授文帝,是夜即以宋昌領南北軍;霍光今年定策立宣帝,而明年稽首歸政。今臨御八年,未聞有所作為。進退人才,興廢政事,天下皆曰此丞相意,一時恩怨,雖歸廟堂,異日治亂,實在陛下。焉有為天之子,為人之主,而自朝廷達於天下,皆言相而不言君哉?天之所以火宗廟、火都城者殆以此。



 臣所以痛心者,九廟至重,事如生存,而
 徹小塗大,不防於火之未至;宰相之居,華屋廣袤,而焦頭爛額,獨全於火之未然,亦足以見人心陷溺,知有權勢,不知有君父矣。他有變故,何所倚仗,陛下自視,不亦孤乎?昔史浩兩入相,才五月或九月即罷,孝宗之報功,寧有窮已,顧如此其亟,何哉?保全功臣之道,可厚以富貴,不可久以權也。



 上讀之感動,授寶章閣,主管雲臺觀,則告吏部,不受貼職祿,不願貼職恩。



 它日星變求言,復申前說。又慮柄臣或果去位,君心易縱,大權旁落,則進《
 為君難》六箴。召為秘書郎兼莊文府教授。端平初入對,上五事,且曰:「隱蔽君德,昔咎故相,故臣得以專詆權臣;昭明君德,今在陛下,故臣以責難君父。」乞召真德秀、魏了翁用之,帝謂之曰:「人主之職無它,惟辨君子小人。」重珍對曰:「小人亦指君子為小人,此為難辨。人主當精擇人望,處之要津,正論日聞,則必知君子姓名、小人情狀矣。」兼崇政殿說書,戒家事勿以白,務積精誠以寤上意。每草奏,齋心盛服,有密啟則手書削稿,帝稱其平實。遷
 著作佐郎。



 邊帥以《八陵圖》來上,詔百官集議,重珍言史嵩之既失相位,危於幕巢,猶欲邀功,自固其位,請擇賢帥如漢用充國,使之親至邊境,審度事勢,條上便宜。丞相主出師關、洛,重珍力爭。會邊帥義和戰不一,復召集議,重珍奏:「曩乞專意備守,不得已則用應兵,今不敢變前說。」不聽,遂自劾以密勿清光,乃不能遏兵端,乞免說書職。遷著作郎兼權司封郎官、起居舍人,言:「近者當侍講席,旋命止之,或曰是日道流生朝。夫輟講偶以它
 故,則當知聖躬舉措之難;或所傳果得其實,則當知聖心持守之難。」帝曰:「非卿不聞此言。」關、洛師大衄,復進兵,重珍言:「若恥敗而欲勝之,則心不平而成忿,氣不平而成怒,生靈之命,豈可以忿怒用哉!」又言:「邇來用臺諫,頗主不必矯激之說,似畏剛方大過之士。竊窺選用之意,正謂其平易而省事耳。然數月之間,一失於某,再失於某,借曰慎重臺綱而憂其激,亦當以平正者居之。」又論禁旅貧弱,教習頻嚴,輒不能堪,不稍變通,非消變之道。



 兼國史院編修官、實錄院檢討官,言:「更化以來,舊敝未去者五:徇私、調停、覆護、姑息、依違是也。今又益之以輕易。」遷起居郎,以疾求去。以集英殿修撰知安吉州,權刑部侍郎,三辭不許,自劾其不能取信朝廷之罪,乞鐫斥置閑散,促覲愈力而疾不可起。詔守刑部侍郎致仕,贈朝請大夫,謚忠文。



 牟子才,字薦叟,井研人。八世祖允良生期歲,淳化間盜起,舉家殲焉,惟一姑未笄,以甕覆之,得免。子才少從其
 父客陳咸,咸張樂大宴,子才閉戶讀書若不聞,見者咸異之。學於魏了翁、楊子謨、虞剛簡,又從李方子,方子,朱熹門人也。嘉定十六年舉進士,對策詆丞相史彌遠,調嘉定府洪雅縣尉,監成都府榷茶司賣引所,闢四川提舉茶馬司準備差遣,使者魏泌眾人遇之,子才拂衣竟去,泌以書幣謝,不受。改闢總領四川財賦所乾辦公事。



 詔李心傳即成都修《四朝會要》,闢兼檢閱文字。制置司遣之文州,視王宣軍餉,鄧艾縋兵處也。道遇宣曰:「敵且
 壓境,宣已退矣,君毋庸往。」子才不可,遂至州視軍慶而還。甫出境,文州陷。闢知成都府溫江縣事,未上,連丁內外艱。時成都已破,遂盡室東下。免喪,心傳方修《中興四朝國史》,請子才自助,擢史館檢閱。



 入對,首言大臣不公不和六事,次陳備邊三策。理宗顧問甚悉,將下殿,復召與語。翼日,帝諭宰相曰:「人才如此,可峻擢之。」左丞相李宗勉擬秘書郎,右丞相史嵩之怨子才言己,遽曰:「姑遷校勘。」俄宗勉卒,嵩之獨相,亟請外,通判吉州,轉通判衢
 州。日食,詔求言,上封事萬言,極陳時政得失,且乞蚤定立太子。入為國子監主簿兼史館校勘,逾年,遷太常博士。



 鄭清之再相,子才兩上封事,言今日有徽、欽時十證,又請為濟王立後,以回天怒。校書郎徐霖言諫議大夫鄭寀、臨安府尹趙與TP,不報,出關。子才言:「陛下行霖言則霖留,不然則不留也。二人之中,寀尤無恥,請先罷之。」寀去。至若嵩之謀復相,清之誤引嵩之之黨別之傑共政,皆歷歷為上言之。作書以孔光、張禹切責清之,清之
 復書愧謝。謁告還安吉州寓舍,遷秘書郎,屢辭,主管崇道觀。逾年,遷著作佐郎,又辭。清之卒之明日,詔子才還朝,遷著作郎;左丞相謝方叔、右丞相吳潛交書道上意,趣行益急,乃至。兼崇政殿說書,子才隨事奏陳,舉朝誦子才奏疏,皆曰:「有德之言也。」兼國史院編修官、實錄院檢討官兼權禮部郎官。時修《四朝史》,乃復兼史館檢討。



 信州守徐謂禮奉行經界苛急,又以脊杖比校催科,饑民嘯聚為亂。子才言於上,立罷經界,謫謂禮。浙東、福建
 九郡同日大水,子才言:「今日納私謁,溺近習,勞土木,庇小人,失人心,五者皆蹈宣和之失。茍不恐懼修省,臣恐宣和京城之水將至矣。燮理陰陽,大臣之事,宜諭大臣息乖爭以召和氣,除壅蔽以通下情。今遣使訪問水災,德至渥也,願出內帑振之。」又言:「君子難聚而易散,今聚者將散,其幾有十。」又言:「謚以勸懲,當出自朝廷,毋待其家自請。」



 左司徐霖言諫議大夫葉大有,帝大怒,逐霖,給事中趙汝騰繳之,徙它官。汝騰即出關,子才上疏留之,
 大有遂劾汝騰。子才上疏訟汝騰誣及大有之欺,未幾,罷大有言職。故事,早講講讀官皆在,晚講惟說書一員,宰相懼子才言己,並晚講於早,自是不得獨對矣。遷軍器少監。御史蕭泰來劾高斯得、徐霖,右司李伯玉言泰來所劾不當,上切責伯玉,降兩官,罷。子才言:「陛下更化,召用諸賢,今汝騰、斯得、霖相繼劾去,伯玉又重獲罪,善人盡矣。」除兼侍立修注官,力辭。



 行都大火,子才應詔上封事,言甚切直,兼直舍人院。會泰來亦遷起居郎,恥與
 泰來同列,七疏力辭,上為出泰來,而子才亦請去不已,曰:「泰來既去,臣豈得獨留。」上不允。又言:「蜀當以嘉、渝、夔三城為要,欲保夔則巴、蓬之間不可無屯以控扼之,欲保渝則利、閬之間不可無屯以遏截之,欲守嘉則潼、遂之間不可無屯以掎角之,屯必萬人而後可。」升兼侍講。御史徐經孫劾府尹厲文翁,不報,出關,子才奏留之。文翁改知紹興府,又繳其命。伯玉降官已逾年,舍人院不敢行詞,子才曰:「故事,文書行不過百刻。」即為書行,以為
 敘復地。帝曰:「謫詞皆褒語,可更之。」子才不奉詔,丞相又道帝意,子才曰:「腕可斷,詞不可改。丞相欲改則自改之。」乃已。



 淮東制置使賈似道以海州之捷,子才草獎諭詔,第述軍容之盛,不言其功,且語多戒敕,似道不樂。又言:「全蜀盛時,官軍七八萬人,通忠義為十四萬,今官軍不過五萬而已,宜招新軍三萬,並撫慰田、楊二家,使歲以兵來助。如此則蜀猶可保,不則不出三年,蜀必亡矣。」湯漢、黃蛻召試學士院,子才發策,蛻譽嵩之,罷蛻正字去。
 遷起居郎,言:「外郡以進奉易富貴,左右以土木蠱上心,小人以嘩競朋比陷君子,此天災所以數見也。」



 明堂禮成,帝將幸西太乙宮款謝,實欲游西湖爾,子才力諫止。皇子冠,面諭作樂章,禮部言:「古者適子一醮無樂,庶子三醮有樂,用樂非是。」子才言:「嫡庶之分,特以所立之地不同,非適專用醴,庶專用醮也。樂章乃學士院故事,況面諭臣,不敢不作。」詔從之。又言:「首蜀尾吳,幾二萬里。今兩淮惟賈似道、荊蜀惟李曾伯二人而已,可為寒心。」謂:「
 宜於合肥別立淮西制置司、江淮別立荊湖制置司,且於漣、楚、光、黃、均、房、巴、閬、綿、劍要害之郡,或築城、或增戍以守之。」似道聞之,怒曰:「是欲削吾地也。」正月望,召妓入禁中,子才言:「此皆董宋臣輩壞陛下素履。」權兵部侍郎,屢辭,帝不允。升同修國史、實錄院同修撰。



 御史洪天錫劾宋臣、文翁及謝堂等,不報,出關。子才請行其言,文翁別與州郡,堂自請外補,宋臣自請解內轄職,而宋臣錄黃竟不至院,蓋子才復有言也。吳子聰之姑知古為
 女冠得幸,子聰因之以進,得知閣門事。子才繳之曰:「子聰依憑城社,勢焰熏灼,以官爵為市,搢紳之無恥者輻湊其門,公論素所切齒,不可用。」帝曰:「子聰之除,將一月矣,乃始繳駁,何也?可即為書行。」子才曰:「文書不過百刻,此舊制也。今子聰錄黃二十餘日乃至後省,蓋欲俟其供職,使臣不得繳之耳。給、舍紀綱之地,豈容此輩得以行私於其間。」於是子聰改知澧州,待次。子才力辭去,帝遣檢正姚希得挽留之,不可。



 以集英殿修撰知太平州,
 前是例兼提領江、淮茶鹽,子才以不諳財懇免。至郡,首教民孝弟,以前人《慈竹》、《義木》二詩刻而頒之,間詣學為諸生講說經義。修採石戰艦百餘艘,造兵仗以千計。前政負上供綱及總所綱七十萬緡,悉為補之。蠲黃、池酒息六十餘萬貫,三縣秋苗畸零萬五千餘石,夏稅畸零綢帛四千五百餘匹、絲七百餘兩、綿一萬三千餘兩、麥二千餘石。郡有平糴倉,以米五千石益之,又以緡錢二十六萬創抵庫,歲收其息以助糴本。召入對,權工部侍
 郎。



 時丁大全與宋臣表裏濁亂朝政,子才累疏辭歸。初,子才在太平建李白祠,自為記曰:「白之斥,實由高力士激怒妃子,以報脫靴之憾也。力士方貴倨,豈甘以奴隸自處者。白非直以氣陵亢而已,蓋以為掃除之職固當爾,所以反其極重之勢也。彼昏不知,顧為逐其所忌,力士聲勢益張,宦官之盛,遂自是始。其後分提禁旅,蹀血宮庭,雖天子且不得奴隸之矣。」又寫力士脫靴之狀,為之贊而刻諸石。屬有拓本遺宋臣,宋臣大怒,持二碑泣
 訴於帝,乃與大全合謀,嗾御史交章誣劾子才在郡公燕及饋遺過客為入己,降兩官,猶未已。帝疑之,密以槧問安吉守吳子明,子明奏曰:「臣嘗至子才家,四壁蕭然,人咸知其清貧,陛下毋信讒言。」帝語經筵宮曰:「牟子才之事,吳子明乃謂無之,何也?」眾莫敢對,戴慶炣曰:「臣憶子才嘗繳子明之兄子聰。」帝曰:「然。」事遂解。蓋公論所在,雖仇讎不可廢也。未幾,大全敗,宋臣斥,誣劾子才者悉竄嶺海外,乃復子才官職,提舉玉隆萬壽宮。



 帝即欲召
 子才。會似道入相,素憚子才,又憾草詔事,僅進寶章閣待制、知溫州;又嗾御史造飛語目子才為潛黨,將中以危禍。上意不可奪,遂以禮部侍郎召,屢辭,不許。乃賜御筆曰:「朕久思見卿,故有是命,卿其勿疑,為我強起。」故事,近臣自外召者,必先見帝乃供職;子才至北關,請內引奏事,宦者在旁沮之,帝特令見,大說,慰諭久之。



 時似道自謂有再造功,四方無虞皆其力,故肆意逸樂,惡聞讜言。子才言:「開慶之時,天下岌岌殆矣,今幸復安。不知天
 將去疾,遂無復憂耶?抑順適吾意,而基異時不可測之禍也。奈何懷宴安以鴆毒,而不明閑暇之政刑乎!忠厚者,我朝之家法也。乃者小人枋國,始用一切以戕其脈,今當反其所為,奈何愈益甚乎!」謂「宜悉取祖宗所以待士愛民、祈天永命者循而行之」言:「議者國之元氣也。今言及乘輿,尚見優假,事關廊廟,忿怒斯形,朝政之闕失,臣下之蔽蒙,何由上達乎?」帝曰:「非卿不聞此言。」宣坐賜茶,問外事甚悉,子才具以田里疾苦對,帝顰蹙久之,
 即兼侍讀,尋兼同修國史、實錄院同修撰。



 宋臣有內侍省押班之命,舉朝爭之不能得。子才入疏,詰朝,帝出其疏示輔臣,皆曰:「子才有憂君愛國之真,無要譽沽名之巧。」擢權禮部尚書。祀明堂,子才為執綏官,帝問漢、唐文物,占對詳贍。時士大夫小迕權臣,輒竄流,子才請重者量移,輕者放還。兼直學士院,前是儤直多以疾免,子才始復舊制,帝賜詩褒賞。每直,輒召對內殿,語至夜分,或就賜酒果。



 兼給事中,彗星見,應詔上封事,請罷公田,更七
 司法。正為尚書,力辭,不許。升修國史、實錄院修撰。徐敏子以星赦量移,似道惡其為潛所用,諷後省繳之,子才不可。葉李、呂宙之等上書攻似道,似道怒,欲殺之,以它事下天府獄。子才請宥之,又遺書似道,似道復書辭甚忿,徑從天府斷遣,不復以聞,蓋懼子才再有所論駁也。



 度宗在東宮,雅敬子才,言必稱先生。即位,授翰林學士、知制誥,力辭不拜,請去不已。進端明殿學士,以資政殿學士致仕,卒,贈四官,官其後二人。



 子才事親甚孝。弟子
 方客死公安,挾其柩葬安吉。女弟在眉山,拔其家於兵火,致之安吉。在吉州,文天祥以童子見,即期以遠大。所薦士若李芾、趙卯發、劉黻、家鉉翁,後皆為忠義士。平江守吳淵籍富民田以千餘畝遺子才,皆即之。身後家無餘貲,賣金帶乃克葬。有《存齋集》、內制外制、《四朝史稿》、奏議、經筵講義口義、《故事四尚》、《易編》、《春秋輪輻》。子巘,大理少卿。



 朱貔孫,字興甫,浮梁人。淳祐四年進士,授臨江軍學教
 授。丞相史嵩之聞貔孫名,欲致之館下,以祿未及親辭。喪父,服除,授福州學教授,差充江東安撫司干辦公事。制置使王野、丘岳、馬光祖、趙與陋皆薦之。丁大全在臺,勢焰熏灼,天久陰雨,貔孫貽書政府,言回積陰之道,去奸邪,罷手實,蠲米稅。奸邪,指大全也。丞相董槐得書嘉嘆。主管尚書刑、工部架閣文字。



 宦者董宋臣寵幸用事,貔孫發策試胄子,極論宦寺專權之患,宋臣諷言者論罷之。光祖闢添差江東安撫司機宜文字,擢史館校勘。
 時大全執政,使其黨許以驟用,貔孫力拒之,且謁告歸省。遷太學博士,屬帝親擢監察御史兼崇政殿說書,首疏論大全權奸誤國之罪,倡言學校六士之冤。又以翕聚人才,凝固人心,精擇人言;增禁旅以壯帝畿,擇良守以牧內郡,選全才以守江面,嚴舟師以防海道;因地募兵,以應突至之敵,並力合勢,以援必守之地。時有建議遷都四明者,貔孫亟上疏言:「鑾輿若動,則三邊之將士瓦解,而四方之盜賊蜂起,必不可。」遂止。貔孫在講筵,言
 及宋臣撓政事忤旨。遷大理少卿,又遷司農少卿兼太子右諭德,詔許乘馬赴講。貔孫諭導得體,衍說經義,有關於君道者必委曲敷暢,陰寓警戒,太子每為之改容。兼國史院編修官、實錄院檢討官兼權直舍人院。



 時大禮成,封命叢委,吏持詞頭下,每夕無慮數十,貔孫運筆如飛,夜未中已就,皆溫潤典雅。遷宗正少卿。丁母憂,服除,授秘書監兼太子左諭德。改監察御史兼崇政殿說書,姓名已付外矣,尋復改命浙西行公田。吏並緣為奸,
 貔孫疏其敝。推《春秋》尊王絀霸之旨,勸帝崇仁政,用吉士,行正論,賜齎甚渥。擢殿中侍御史兼侍講,請嚴京師淫聲奇服之禁。他所論苗耗役害及經理川蜀,皆當世急務。



 宋臣覆出,朝論紛然,貔孫因對,力斥其奸,卒奪祠。升侍御史兼侍講。長星出東方,貔孫力詆外戚內臣及進奉羨餘失人心者,且曰:「回天心自回人心始。」辭旨懇切,帝為之感動,升侍讀。貔孫之再入臺,屬疆場多事,屢陳備御之策。理宗春秋高,倚成賈似道,似道擅命,貔孫
 隨事進諫,不肯阿附,至若行公田之政,屢於經筵密以告帝,似道自是深忌之。貔孫累疏求去。



 理宗崩,度宗即位,擢右諫議大夫,賜紫金魚袋兼賜章服犀帶,以疾乞辭言職,遷吏部尚書,不拜。帝以舊學故雅欲留貔孫,使者旁午於道,而貔孫辭益力,以華文閣學士知寧國府,似道諷言者論罷。久之,提舉太平興國宮,復華文閣學士、知袁州。至郡,宣布德意,以戢暴禁貪為先務。郡倉受租,舊倚斛面取贏,吏加漁取。貔孫知其敝,悉榜除之,許
 民自概量。宿敝頓革,田里歡聲。興學校以勸士。升敷文閣學士,知福州、福建安撫使。未幾,卒於袁之郡治。贈四官,與恩澤二,令所在給喪事。有文集、奏議行世。



 歐陽守道,字公權,一字迂父,吉州人。初名巽,自以更名應舉非是,當祭必稱巽。少孤貧,無師,自力於學。里人聘為子弟師,主人瞷其每食舍肉,密歸遺母,為設二器馳送,乃肯肉食,鄰媼兒無不嘆息感動。年未三十,翕然以德行為鄉郡儒宗。江萬里守吉州,守道適貢於鄉,萬里
 獨異視之。



 淳祐元年舉進士,廷對,言:「國事成敗在宰相,人才消長在臺諫。昔者當國惡箴規,言者疑觸迕,及其去位,共謂非才。或有迎合時宰,自效殷勤,亦有疾惡乖方,茍求玼纇,以致忠邪不辨,黜陟無章。」唱名,徐儼夫為第一,儼夫握守道起曰:「吾愧出君上矣,君文未嘗不在我上也。」授雩都主簿。



 丁母憂,服除,調贛州司戶,其次在十年,後萬里作白鷺洲書院,首致守道為諸生講說。湖南轉運副使吳子良聘守道為嶽麓書院副山長。守道初
 升講,發明孟氏正人心、承三聖之說,學者悅服。宗人新及子必泰先寓居長沙,聞守道至,往訪之。初猶未識也,晤語相契,守道即請於子良,禮新為嶽麓書院講書。新講《禮記》「天降時雨、山川出雲」一章,守道起曰:「長沙自有仲齊,吾何為至此。」仲齊,新之字也。逾年,新卒,守道哭之慟,自銘其墓,又薦其子必泰於當道。子良代,守道復還吉州。



 里有張某喪其父,小祥,而舅氏訟以事,系之獄,使不得祭,邀其售己地以葬。守道聞之,嘆曰:「吾惟痛斯子
 之不得一哭其父也,且其痛奈何?」明日告之邑令曰:「此非人心,濱祭而薄之,撓葬而奪之,舅如此,是自食其肉也。請任斯子出,祭而復獄。」令亟出之。其舅醜誣守道,守道亦不自辨。轉運使包恢為請祠於朝。萬里入為國子祭酒,薦為史館檢閱,召試館職,授秘書省正字。



 安南國王陳日照傳位其子,求封太上國王,下省官議。守道謂:「太上者,漢高帝以尊其父,累朝未之有改,若賜詔書稱太上國王,非便。南越尉佗嘗自稱『蠻夷大長老』,正南夷
 事也。《禮》,方伯自稱曰『天子之老』,大夫致仕曰『老』,自稱亦曰『老』。自蠻夷言之則有尉佗之故事;自中國言之,亦方伯致仕者之常稱。漢亦有老上單于之號,易『太』以『老』無損。或去『上』字存其『太』字,太王則有古公,三太、三少,太宰、少宰,『太』所以別於『少』也。謂父為太,則子為少矣。太以尊言,則太后、太妃、太子、太孫;以卑言,則太史、太卜、太祝、樂太、師太,固上下所通用也。」時病足,不及與議。



 遷校書郎兼景憲府教授,遷秘書郎,轉對,言:「欲家給人足,必使中
 外臣庶無復前日言利之風而後可。風化惟反諸身。化之以儉,而彼不為儉,吾惟有卑宮室、菲飲食;化之以廉,而彼不興廉,吾惟有不貴難得之貨、不厚無益之藏。」以言罷。守道徒步出錢塘門,唯書兩篋而已。理宗遺詔聞,守道與其徒相向哭踴,僮奴孺子各為悲哀。咸淳三年,特旨與祠。詔大臣舉賢才,少傅呂文德舉九十六人,守道預焉。添差通判建昌軍,以書謝廟堂曰:「史贊大將軍不薦士,今大將軍薦士矣,而某何以得此於大將軍哉。
 幸嘗蒙召,擢備數三館,異時或者謂其放廢無聊,托身諸貴人,虧傷國體,則寧得而解,願仍賦祠祿足矣。」遷著作佐郎兼崇政殿說書兼權都官郎官。經筵所進,皆切於當世務,上為動色。遷著作郎,卒,家無一錢。



 守道之兄之妻蚤喪,其子演五歲餘,且多病,浚生甫數月,守道三十未有室,顧無能乳哺者,日夜抱二子泣,里巷憐之。演既長,出莫知所之,守道哭而求諸野,終不能得,三年不食肉,憔悴不釋者終身。吉有賢守而大家怨之厚誣以
 贓者,下其事常平使者。會旱甚,禱雲騰,守道曰:「無以禱也,雲騰之神,唐郡守吳侯也。冤莫甚於前守,冤不直而吳侯於禱,侯有辭矣。匹婦藏冤,旱或三年,冤在民牧,害豈其小。」反覆千餘言,或迂笑之,守道不改,告來者不倦,守卒以得直。所著有《易故》、文集。



 論曰:湯璹立朝蹇諤。蔣重珍自擢巍科,既居盛名之下,而能樹立於當世,可謂難矣。牟子才、朱貔孫,直聲著於中外。歐陽守道,廬陵之醇儒也。



\end{pinyinscope}