\article{列傳第一百四十六}

\begin{pinyinscope}

 黃洽汪應辰王十朋吳芾陳良翰杜莘老



 黃洽,字德潤,福州候官人。隆興元年,以太學生試春官第二,詔循故事,未臨軒,賜第二人及第。授紹興府觀察
 判官。秩滿,就銓選,不用前名例謁廟堂。宰相陳俊卿白於上,改宣義郎,除國子博士。



 適有旨職事官無待次,改差浙東安撫司主管機宜文字。繼為太學國子博士,樞密院編修官,通判福州。奉祠,召為太常丞。請外,孝宗方厲精求治,曰:「黃洽厚德,方任以事。」不許。當對,奏三事:備事莫若儲才,士卒當練其心,軍政必預為謀。上矍然,洽徐奏:「願戒飭州郡,毋煩擾以致寇,毋輕易以玩寇。寇擾而後定,傷根本多矣。」繇秘書郎遷著作郎。上諭詞臣:「秘
 閣儲英俊為異時公卿用,行黃洽詞,可及之。」



 除右正言,首奏:「諫臣非具員,職在諫爭,朝政有闕,所當盡言。」上亦以為端士,許其盡言無隱。除侍御史。會水旱頻仍,因祠祭上言:「此事全在一念,陛下夙興默想,專精在民,身雖法宮,心則壇壝,洋洋左右,理非漠然。幾歲荒歉之由,必有未盡契神示之心者。」一日特詔:「諸路奉行荒政不虔,差官按視安集。」洽亟奏:「使者一出,官吏必須知畏。其常平一司,所職何事?淮、浙、江東見有使,以五使分五路,尚
 慮不周知。今遣一人兼二三路,不過閱圖帳戶口多寡,地里遼邈,安能遍歷乎?若專責常平,名正而職舉,事分而察精。」又奏:「藝祖懲藩鎮偏重之失,不欲兵民之權聚於一夫之手。今使主兵官兼郡寄,是合兵民權為一,且屬邊徼,偏重尤甚。」上皆嘉納。洽所論列,未嘗捃摭細故他慝以累其終身。



 除右諫議大夫。上方銳志肄武,洽因風諫,言:「《頤》之大象:『君子以慎言語,節飲食。』言語飲食猶謹節之,況其它乎?凡筋力喘息之間,一有過差,皆非所
 以養其身也。」上曰:「卿言無非仁義忠孝,可為萬世臣子之法,朕常念之。」洽在經筵,言:「宰相代天理物,要在為國得人。人主之命相,任則勿疑。宰相重則朝廷尊,朝廷尊則廟社安。宰相掄才任職,當盡公心。君子進則庶職舉,庶職舉則天下治。」上首肯再三,乃曰:「卿如良金美玉,渾厚無瑕,天其以卿為朕弼耶?」



 除御史中丞,奏:「薦舉請托,必競於宰執、臺諫之門,若宰執、臺諫不為人覓舉,使士大夫咸自率厲,以公道得之,豈不甚善。或果知其人,露
 章以薦,亦何不可。」潭州奏強盜罪不至死應配者坐加役流,有旨具議。洽曰:「強盜異他盜,以其故為也。若止髡役,三年之後,圈檻一弛,豨突四出,善良受害,可勝數耶?況役時必去防閑之具,走逸結合,患尤甚焉。」上深然之。



 除參知政事。上曰:「卿每告朕用人,今卿居用人之地,不可不勉。」上因商榷除目,洽罄謁無所顧避,上大喜曰:「五十年無此差除。」除知樞密院事。洽累章求去,許之,除資政殿大學士、知隆興府。



 光宗受禪,特詔言事,洽奏:「用人
 為萬世不易之論,臣前以此納忠壽皇,今復告於陛下。」屢乞歸田,尋畀提舉洞霄宮。方未得請也,人勸之治第,洽曰:「吾書生,蒙拔擢至此,未有以報國,而先營私乎?使吾一旦罪去,猶有先人敝廬可庇風雨,夫復何憂。」慶元二年致仕。



 洽常言:「居家不欺親,仕不欺君,仰不欺天,俯不欺人,幽不欺鬼神,何用求福報哉!」六年七月,薨,年七十九。贈金紫光祿大夫。洽質直端重,有大臣體,兩朝推為名臣。有文集、奏議八十五卷。



 汪應辰,字聖錫,信州玉山人。幼凝重異常童,五歲知讀書,屬對應聲語驚人,多識奇字。家貧無膏油,每拾薪蘇以繼晷。從人借書,一經目不忘。十歲能詩,游鄉校,郡博士戲之曰:「韓愈十三而能文,今子奚若?」應辰答曰:「仲尼三千而論道,惟公其然。」



 未冠,首貢鄉舉,試禮部,居高選。時趙鼎為相,延之館塾,奇之。紹興五年,進士第一人,年甫十八。御策以吏道、民力、兵勢為問,應辰答以為治之要,以至誠為本,在人主反求而已。上覽其對,意其為老
 成之士,及唱第,乃年少子,引見者掖而前,上甚異之。鼎出班特謝。舊進士第一人賜以御詩,及是,特書《中庸篇》以賜。初名洋,與姓字若有語病,特改賜應辰。上欲即除館職,趙鼎言:「且令歷外任,養成其材。」乃授鎮東軍簽判。故事,殿試第一人無待次者,至是,取一年半闕以歸。舍人胡寅行詞曰:「屬者延見多士,問以治道,爾年未及冠,而能推明帝王躬行之本,無曲學阿世之態。」



 應辰少受知於喻樗,既擢第,知張九成賢,問之於樗,往從之游,所
 學益進。初任,趙鼎為帥,幕府事悉諮焉。歲小旱,命應辰禱雨名山即應,越人語之曰:「此相公雨。」鼎曰:「不然,乃狀元雨也。」



 召為秘書省正字。時秦檜力主和議,王倫使還,金人欲以河南地歸我。應辰上疏,謂:「和議不諧非所患,和議諧矣,而因循無備之可畏。異議不息非所患,異議息矣,而上下相蒙之可畏。金雖通和,疆埸之上宜各戒嚴,以備他盜。今方且肆赦中外,褒寵將帥,以為休兵息民自此而始。縱忘積年之恥,獨不思異時意外之患乎?
 此因循無備之所以可畏也。方朝廷力排群議之初,大則竄逐,小則罷黜,至有一言迎合,則不次擢用。是以小人窺見間隙,輕躁者阿諛以希寵,畏懦者循默以備位,而忠臣正士乃無以自立於群小之間,此上下相蒙之所以可畏也。臣願勿以和好之可無虞,而思患預防,常若敵人之至。」疏奏,秦檜大不悅,出通判建州,遂請祠以歸。寓居常山之永年院,蓬蒿滿徑,一室蕭然,饘粥不繼,人不堪其憂,處之裕如也,益以修身講學為事。自是凡
 三主管崇道觀,在隱約時,胸中浩然之氣凜然不可屈。



 張九成謫邵州,交游皆絕,應辰時通問。及其喪父,言者猶攻之,而應辰不遠千里往吊,人皆危之。通判袁州,凡所予奪,人無異詞。始至,或以其書生易之,已乃知吏師所不能及。丞相趙鼎死朱崖,扶喪過郡,應辰為文祭之曰:「惟公兩登上宰,皆直艱危之時;一斥南荒,遂為死生之別。事已定於蓋棺,恩特容於歸骨。」吏付之火。其子借三兵以歸,道出衢州,章傑為守,希檜意,指應辰為阿附,
 為死黨,符移訊鞫,偏搜行橐,求祭文不可得。時胡寅遺檜書,謂此事不足竟,事乃寢。



 通判靜江府,逾期不得代,乃沿檄歸省其母。繼差通判廣州。時檜所深忌者趙鼎、張浚,鼎既死而浚獨存,未快其意。江西運判張常先箋注前帥張宗元與浚詩,言於朝,其詞連逮者數十家,將誣以不軌而盡去之。獄既具,檜死,應辰幸而免。



 明年,召為吏部郎官,遷右司。母老乞外,丞相苦留之曰:「方進用,未應爾。」應辰曰:「親老矣,不可緩。」乃出知婺州。郡積欠上
 供十三萬緡,朝廷命憲漕究治,應辰謂急則擾民,乃與諸邑蠲宿逋,去苛斂,定期會,窒滲漏,悉為補發。尋丁內艱去,廬於墓側。



 服闋,除秘書少監,遷權吏部尚書。李顯忠冒具安豐軍功賞五千餘人,應辰奏駁之。權戶部侍郎兼侍講。應辰獨員當劇務,節冗費,常奏:「班直轉官三日,而堂吏增給食錢萬餘緡;工匠洗澤器皿僅給百餘千,而堂吏食錢六百千;塑顯仁神御,半年功未及半,而堂吏食錢已支三萬、銀絹六百匹兩。他皆類此。」上驚其
 費冗,命吏部裁之。



 金渝盟,詔求足食足兵之策,應辰奏曰:「陸贄有云:『將非其人,兵雖多不足恃;操失其柄,將雖才不為用』。臣之所憂,不在兵之不足,在乎軍政之不修。自講和以來,將士驕惰,兵不閱習,敵未至則望風逃遁,敵既退則謾列戰功,不惟佚罰,且或受賞。方時無事,詔令有所不行,一旦有急,誰能聽命以赴國家之難。望發英斷,賞善罰惡,使人人洗心易慮,以聽上命,然後號令必行矣。」



 三十二年建儲,以孝宗名與唐廬江王、晉楚王
 同,詔改為「曄」,應辰以為與唐昭宗同,白左相陳康伯,遂改今名。集議秀王封爵,應辰定其稱曰「太子本生之親」。議入,內降曰:「皇太子所生父,可封秀王。」暨內禪,擬於傳位日降赦,應辰言:「唐太宗受禪於高祖,明年正月始改元。」乃從其說。又議改元「重熙」,應辰謂契丹嘗以紀年,遂改隆興。一朝大典禮,多應辰所定。



 議太上尊號,李燾、陳康伯密議以「光堯壽聖」為稱。及集議,或謂:「尊號始自開元,罷於元豐,今不當復,況太上視天下如棄敝屣,豈復
 顧此?」應辰主之尤力。或又言:「主上奉親,烏得援元豐自卻為比?」於是議狀書者半,不書者半。明日,應辰復與金安節等十二人各陳所見,大概謂「光堯」近乎「神堯」,「壽聖」乃英宗誕節,嘗以名寺。御史周必大亦以為問,應辰答以「堯」豈可「光」。是語有聞之德壽者,高宗因上過宮,云:「汪應辰素不樂吾。」於是有詔:尊號之議,已嘗奏知,不容但已。安節等遂奉詔。



 應辰連乞補外,遂知福州。未幾,升敷文閣待制,舉朱熹自代。在鎮二年,會朝廷謀蜀帥,乃以
 敷文閣直學士為四川制置使、知成都府。陛辭,特降詔撫諭。入境,以書與宣撫使吳璘,令以撫諭詔申嚴號令。既至,免利路民餉運,徙沿邊戍兵就糧內郡,縱保勝義士復業,存左藏所解白契二百萬以備不虞,悉奏行之。有謂蜀中綱馬驛程由梁、洋、金、房,山路峻險,宜浮江而下,詔吳璘措置。執政、大將皆主其說,應辰與夔帥王十朋力言其不便,遂得中止。二稅勘合,每貫取二十錢,乾道詔旨嘗減三之一,有欲增之者,應辰與兩漕臣列奏,
 言:「勘合不以鈔計,而以貫石匹兩計,是陽為減而陰實增之也。以成都一路計之,歲入三十萬,今以所增為六十萬,計以四路,不知幾倍。雖非興利者所便,而民受其賜多矣。」



 璘時駐蜀口武興,精兵為天下冠,既老且病,應辰密奏以關陜大將系國安危,所當預圖。於是執政傳旨,若璘不起,令制司暫領其任。暨璘死,應辰遂攝宣撫之職,蜀道晏然。



 虞允文尋以知樞密院事宣撫四川,應辰援張浚例,乞罷制司,不許。總所牒委官核四川匿契
 稅,應辰奏:「其不便者四,曰妨農廢業,曰縱吏擾民,曰違法害教,曰長奸起訟。比戶部已令人自首,州縣收並已不少,其未盡者,有見行法令,不宜為此煩擾。」上曰:「論極有理,速罷止之。」



 蜀大旱,詔問救荒之策,應辰奏:「利、閬、綿、梓軍馬糧料,隨民力均敷,官雖支糴錢,民不得半價,若選官就歲熟處糴之,可以寬民力,第無錢束手,乞給度牒。」上曰:「汪應辰治蜀甚有聲,且留意民事如此。」給度牒四百,永為糴本振濟,遂移書諸路漕臣,亟救荒,且以綿、
 劍和糴告之,而全蜀蒙惠。



 劉珙拜同知樞密院事,進言曰:「汪應辰、陳良翰、張栻學行才能,臣所不及。」已,得旨召還。邛之安仁年饑,挻起為盜,害及旁郡,即具奏,且檄茶馬使招捕。旬月間,誅其渠魁,餘悉撫定。或白之虞允文曰:「汪帥得無掩盜事不上聞乎?」宣司乃密奏,使人紿應辰曰:「邛寇事未敢奏,不審制司如何?」應辰以奏檢報之,允文內愧。將行,代納成都一府激賞絹估三萬三千九百八十四匹。



 冬,入覲,陛對,以畏天愛民為言。上曰:「卿久
 在蜀,寬朕西顧憂,軍政民事革弊殆盡,蜀中除虛額,民間當被實惠。」應辰奏:「虛額去則州縣寬,尚有兩事,曰預借,曰對糴。預借乃州縣累歲相仍,對糴則以補州縣闕乏,民輸米一石,即就糴一石,或半價,或不支,且多取贏。陛下近捐百萬除預借之弊,對糴患止數州,願並除之,則弊革無餘矣。」



 除吏部尚書,尋兼翰林學士並侍讀。論愛民六事,廟堂議不合,不悅者眾。一日,陳良祐登對,上告以「汪應辰言卿在蜀多誕謾。」良祐奏:「臣與應辰昨同
 從班,應辰請外,得衢州,臣惜其去,同奏留之。時邊奏方急,臣不知應辰將為便私計也。奏既上,應辰以此大憾,乃為是說以中臣耳。」上曰:「乃爾邪!」



 應辰在朝多革弊事,中貴人皆側目。德壽宮方甃石池,以水銀浮金鳧魚於上,上過之,高宗指示曰:「水銀正乏,此買之汪尚書家。」上怒曰:「汪應辰力言朕置房廓與民爭利,乃自販水銀邪?」應辰知之,力求去。會復出發運均輸之旨,嘆曰:「吾不可留矣,但力辨群枉,則補外之請自得。」乃力論其事有害
 無利,遂以端明殿學士知平江府。



 韓玉被旨揀馬,過郡,應辰簡其禮。玉歸,譖之於上曰:「臣所過州縣,未有若平江之不治者。」上怪之。平江米綱至,有折閱,事上,連貶秩。力疾請祠,自是臥家不起矣,以淳熙三年二月卒於家。



 應辰接物溫遜,遇事特立不回,流落嶺嶠十有七年。檜死,始還朝,剛方正直,敢言不避。少從呂居仁、胡安國游,張栻、呂祖謙深器許之,告以造道之方。嘗釋克己之私如用兵克敵,《易》懲忿窒欲,《書》剛制於酒,懲窒、剛制皆克
 勝義,可不常省察乎?其義理之精如此。好賢樂善,出於天性,尤篤友愛,嘗以先疇遜其兄衢,雖無屋可居不顧也。子達,繼登進士第,仕至吏部尚書、端明殿學士。



 王十朋,字龜齡,溫州樂清人。資穎悟,日誦數千言。及長,有文行,聚徒梅溪,受業者以百數。入太學,主司異其文。



 秦檜死,上親政,策士,諭考官曰:「對策中有陳朝政切直者,並置上列。」十朋以「權」為對,大略曰:「攬權者,非欲衡石程書如秦皇,傳餐聽政如隋文,強明自任、不任宰相如
 唐德宗,精於吏事、以察為明如唐宣宗,蓋欲陛下懲既往而戒未然,威福一出於上而已。嘗有鋪翠之禁,而以翠羽為首飾者自若,是豈法令不可禁乎?抑宮中服浣濯之化,衣不曳地之風未形於外乎?法之至公者莫如選士,名器之至重者莫如科第。往歲權臣子孫、門客類竊巍科,有司以國家名器為媚權臣之具,而欲得人可乎?願陛下正身以為本,任賢以為助,博採兼聽以收其效。」幾萬餘言。上嘉其經學淹通,議論醇正,遂擢為第一。
 學者爭傳誦其策,以擬古晁、董。



 上用其言,嚴銷金鋪翠之令,取交址所貢翠物焚之。詔:「十朋乃朕親擢。」授紹興府簽判。既至,或以書生易之,十朋裁決如神,吏奸不行。時以四科求士,帥王師心謂十朋身兼四者,獨以應詔。召為秘書郎兼建王府小學教授。先是,教授入講堂居賓位,十朋不可,皇孫特加禮而位教授中坐。



 金將渝盟,十朋輪對,言:「自建炎至今,金未嘗不內相殘賊,然一主斃,一主生,曷嘗為中國利?要在自備如何。禦敵莫急於
 用人,今有天資忠義、材兼文武可為將相者,有長於用兵、士卒樂為之用可為大帥者,或投閑置散,或老於藩郡,願起而用之,以寢敵謀,以圖恢復。」蓋指張浚、劉錡也。又言:「今權雖歸於陛下,政復出於多門,是一檜死百檜生也。楊存中以三衙而交結北司,以盜大權。漢之禍起於恭、顯,王氏之相為終始;唐之禍起於北軍,藩鎮之相為表裏。今以管軍位三公,利源皆入其門,陰結諸將,相為黨援。樞密本兵之地,立班甘居其後。子弟親戚,布滿
 清要。臺諫論列,委曲庇護,風憲獨不行於管軍之門,何以為國!至若清資加於噲五;高爵濫於醫門;諸軍承受,威福自恣,甚於唐之監軍;皇城邏卒,旁午察事,甚於周之監謗;將帥剝下賂上,結怨三軍;道路捕人為卒,結怨百姓;皆非治世事。」上嘉納,戢邏卒,罷諸軍承受,更定樞密、管軍班次,解楊存中兵權,其言大略施行。秦檜久塞言路,至是十朋與馮方、胡憲、查鑰、李浩相繼論事,太學生為《五賢詩》述其事。除著作郎。



 三十一年正月,風雷雨
 雪交作,十朋以為陽不勝陰之驗,遣陳康伯書,冀以《春秋》災異之說力陳於上,崇陽抑陰,以弭天變。遷大宗正丞,亟請祠歸。金犯邊,起劉錡為江、淮、浙西制置,張浚帥金陵,悉如其言。



 孝宗受禪,起知嚴州。召對,首言:「太皇非倦勤時,而以大器付陛下,賢於堯、舜,陛下當思以副太上者。今社稷之安危,生民之休戚,人才之進退,朝廷之刑賞,宜若舜之協堯,斷然行之,以盡繼述之道。」拜司封郎中,累遷國子司業。言:「今居位者往往職之不舉,宜有
 以革之。人主有大職三,任賢、納諫、賞罰是也。」上嘉之。除起居舍人,升侍講。時左右史失職久,十朋除起居郎,胡銓奏四事,語在《胡銓傳》。除侍御史,上謂胡銓曰:「比除臺官,外議如何?」銓曰:「皆謂得人。」上曰:「卿與十朋皆朕親擢。」



 十朋見上英銳,每見必陳恢復之計。及將北伐,上疏曰:「天子之孝莫大於光祖宗、安社稷,因前王盈成而守者,周成康、漢文景是也;承前世衰微而興者,商高宗、周宣王是也;先君有恥而雪之,漢宣帝臣單于、唐太宗俘頡
 利是也;先君有仇而復之,夏少康滅澆、漢光武誅莽是也。跡雖不同,其為孝一也。靖康之禍,亙古未有,陛下英武,慨然志在興復。竊聞每對群臣奏事,則曰:『當如創業時。』又曰:『當以馬上治之。』又曰:『某事當俟恢復後為之』。比因宣召,語及陵寢,聖容惻然,曰:『四十年矣。』陛下之心真少康、高宗、宣王、光武之心,奈何大臣不能仰副聖心?願戒在位者,去附和之私心,贊國家之大計,則中興日月可冀矣。」因論史浩八罪,曰懷奸、誤國、植黨、盜權、忌言、蔽
 賢、欺君、訕上,上為出浩知紹興府。十朋再疏,謂:「陛下雖能如舜之去邪,未能如舜之正名定罪。紹興密邇行都,浩嘗為屬吏,奸臟彰聞,亦何顏復見其吏民。」遂改與祠。



 史正志與浩族異,拜浩而父事之,十朋論正志傾險奸邪,觀時求進,宜黜正志以正典刑。林安宅出入史浩、龍大淵門,盜弄威福,至是詐病求致仕,十朋並疏其罪。皆罷去。



 張浚出師復靈壁、虹縣,歸附者萬計,又復宿州。十朋奏:「王師以吊民為主,先之以招納,不獲已而戰伐隨
 之,乞以此指戒浚。金將既降,宜速加爵賞,以勸來者。」上皆嘉納。



 會李顯忠、邵宏淵不協,王師失律,張浚上表自劾,主和者乘此唱異議。十朋上疏言:「臣素不識浚,聞其誓不與敵俱生,心實慕之。前因輪對,言金必敗盟,乞用浚。陛下嗣位,命督師江、淮,今浚遣將取二縣,一月三捷,皆服陛下任浚之難。及王師一不利,橫議蜂起。臣謂今日之師,為祖宗陵寢,為二帝復仇,為二百年境土,為中原吊民伐罪,非前代好大生事者比。益當內修,俟時而
 動。陛下恢復志立,固不以一衄為群議所搖,然異論紛紛,浚既待罪,臣其可尚居風憲之職!乞賜竄殛。」因言:「臣聞近日欲遣龍大淵撫諭淮南,信否?」上曰:「無之。」又言:「聞欲以楊存中充御營使。」上嘿然。



 改除吏部侍郎,力辭,出知饒州。饒並湖,盜出沒其間,聞十朋至,一夕遁去。丞相洪適請故學基益其圃,十朋曰:「先聖所居,十朋何敢予人。」移知夔州,饒民走諸司乞留不得,至斷其橋,乃以車從間道去,眾葺斷橋,以「王公」名之。



 移知湖州,召對,劉珙
 請留之,上曰:「朕豈不知王十朋,顧湖州被水,非十朋莫能鎮撫。」至郡,戶部責虛逋三十四萬,命吏持券往辨,不聽,即請祠去。起知泉州,十朋前在湖割奉錢創貢闈,又為泉建之,尤宏壯。



 凡歷四郡,布上恩,恤民隱,士之賢者詣門,以禮致之。朔望會諸生學宮,講經詢政,僚屬間有不善,反復告戒,俾之自新。民輸租俾自概量,聞者相告,宿逋亦願償。訟至庭,溫詞曉以理義,多退聽者。所至人繪而祠之,去之日,老稚攀留涕泣,越境以送,思之如父
 母。饒久旱,入境雨至;湖積霖,入境即霽。凡禱必應,其至誠不獨感人,而亦動天地鬼神。



 東宮建,除太子詹事,力辭,詔州郡禮致,遂力疾造朝,以足疾不能趨,詔給扶減拜。謁東宮,太子以其舊學,待遇有加。又詔免朝參,遣中使以告及襲衣、金帶就其家賜之。疾革,累章告老,以龍圖閣學士致仕,命下而卒,年六十。紹熙三年,謚曰忠文。



 十朋事親孝,終喪不處內,友愛二弟,郊恩先奏其名,沒而二子猶布衣。書室扁曰「不欺」,每以諸葛亮、顏真卿、寇
 準、範仲淹、韓琦、唐介自比,朱熹、張栻雅敬之。



 子聞詩、聞禮,皆篤學自立。聞詩知光州、提點江東刑獄;聞禮知常州、江東轉運判官,為治能守家法,人亦思慕之。



 吳芾,字明可,臺州仙居人。舉進士第,遷秘書正字。與秦檜舊故,至是檜已專政,芾退然如未嘗識。公坐旅進,揖而退,檜疑之,風言者論罷。通判處、婺、越三郡。知處州。處舊苦丁絹重,芾損之,以新丁補其額。



 何溥薦芾材中御史,除監察御史。時金將敗盟,芾勸高宗:「專務修德,痛自
 悔咎,延見群臣,俾陳闕失,求合乎天地,無愧乎祖宗,則人心悅服,天亦助順矣。」上韙其言。遷殿中侍御史。



 兩淮戰不利,廷臣爭陳退避計,芾言:「今日之事,有進無退,進為上策,退為無策。」既而金主亮斃,上疏勸親征。車駕至建康,芾請遂駐蹕,以系中原之望,高宗納其說。會有密啟還東者,下侍從、臺諫議,芾言:「今欲控帶襄、漢,引輸湖、廣,則臨安不如建康便;經理淮甸,應接梁、宋,則臨安不如建康近。議者徒悅一時扈從思歸之人,非為國計。臣
 恐回鑿之後,西師之聲援不接,北土之謳吟絕望矣。」又言:「去歲兩淮諸城望風奔潰,無一城能拒守者,此秦檜壅塞言路、挫折士氣之餘毒也。能反其道,則士氣日振,而見危授命者有人矣。」



 知婺州。孝宗初即位,陛辭,陳裴□對唐憲宗「為治先正其心」,以為臨御之初,出治大原,無越於此。上嘉納。至郡,勸民義役。金華長仙鄉民十有一家,自以甲乙第其產,相次執役,幾二十年。芾輿致十一人者,與合宴,更其鄉曰「循理」,里曰「信義」,以褒異之。



 知
 紹興府。會稽賦重而折色尤甚,芾以攢宮在,奏免支移折變。鑒湖久廢,會歲大饑,出常平米募饑民浚治。芾去,大姓利於田,湖復廢。



 權刑部侍郎,遷給事中,改吏部侍郎。以敷文閣直學士知臨安府。內侍家僮毆傷酒家保,芾捕治之,徇於市,權豪側目。執政議以芾使金,復除吏部侍郎,且議以龍大淵為副,芾曰:「是可與言行事者邪?」語聞,得罷不行。下遷禮部侍郎,力求去,提舉太平興國宮。



 時芾與陳俊卿俱以剛直見忌,未幾,俊卿亦引去。中
 書舍人閻安中為孝宗言二臣之去,非國之福。起知太平州。造舟以梁姑溪。歷陽築者久役潰歸,聲言欲趨郡境,芾呼至城下,厚犒遣之,而密捕倡亂者系獄以聞,詔褒諭。知隆興府。



 芾前後守六郡,各因其俗為寬猛,吏莫容奸,民懷惠利。再奉太平祠,屢告老,以龍圖閣直學士致仕。後十年卒,年八十。嘗曰:「視官物當如己物,視公事當如私事。與其得罪於百姓,寧得罪於上官。」立朝不偶,晚退閑者十有四年,自號湖山居士。為文豪健俊整,有
 表奏五卷、詩文三十卷。



 陳良翰,字邦彥,臺州臨海人。蚤孤,事母孝。資莊重,為文恢博有氣。中紹興五年進士第。知溫州瑞安縣。俗號強梗,吏治尚嚴,良翰獨撫以寬,催租不下文符,但揭示名物,民競樂輸,聽訟咸得其情。或問何術,良翰曰:「無術,第公此心如虛堂懸鏡耳。」殿中侍御史吳芾薦為檢法官,遷監察御史。



 孝宗初元,金主褒新立,求和,而中原舊人多求歸,詔問何以處此,良翰言:「議和,復納降,皆非是。必
 定計自治,而和不和,任之乃可。」張浚軍淮、泗以規進取,而議者爭獻防江策,良翰言:「當固藩籬,專委任。今舍淮防江,卻地奪便,朝廷過聽,使督府不得專閫外事,誤矣。」除右正言。



 金再移書求故疆,良翰言:「中原皆吾故土,況唐、鄧、海、泗又金渝盟後以兵取之,安得以故疆為言而歸之?」湯思退主遣小使盧仲賢、李栻,良翰言:「仲賢輕儇無恥,栻自北來難信。」又言:「廟堂督府論議不同,邊奏上聞,皆陽唯諾而陰沮敗之。萬一失事機,督府安得獨任
 其責?」上矍然稱善。



 朝廷遣史正志至建康,與張浚議事乖牾,良翰劾之,上曰:「正志亦無罪。」良翰言:「陛下使浚守淮,則任浚為重,一郎官為輕,且正志居中,浚必為去就。」上悟,出正志為福建漕運。楊存中為御營使,總殿前軍,良翰言:「存中久擅兵柄,太上皇罷就第,奈何復假使名?宜慎履霜之戒。」疏三上,存中竟罷。



 李栻不敢涉淮,良翰奏奪其官。仲賢至汴,輒許金人以疆土、歲幣而還,上大怒,下仲賢吏,欲誅之,宰相叩頭懇請得免。復遣王之望、
 龍大淵,良翰言:「前遣使已辱命,大臣不悔前失,不謂秦檜復見今日!且金要我罷四郡屯兵以歸之,是不折一兵,而坐收四千里要害之地,決不可許。若歲幣,則俟得陵寢然後與,庶猶有名。今議未決而之望遂行,恐其辱國不止於仲賢,願先馳一介往,俟議決,行未晚也。」詔侍從、臺諫議,多是良翰,遂以胡昉、楊由義為審議官,與敵議四郡不合,困辱而歸。



 思退尚執前論,正言尹穡附思退以撼督府。良翰為左司諫,疏論:「思退奸邪誤國,宜早
 罷黜,張浚精忠老謀,不宜以小人言搖之。」孝宗曰:「思退前議固失,然朕愛其警敏,冀可效,卿其置之。若魏公則今日孰出其右,朕豈容有此意?縱有之,亦豈不謀卿等?此殆言者有異意,卿為朕諭之。」良翰頓首謝曰:「陛下言及此,天下幸甚。宰相縱無全才,寧取樸實,緩急猶可倚賴。思退庸狡,小黠大癡,將誤國,且『警敏』二字,恐非明主卜相之法。」既退,以上語諭同列,穡勃然變色,明日亦請對,遂罷良翰言職。



 兩淮既撤備,金大入,孝宗始深悔。太
 學生數百人伏闕,乞召用良翰、胡銓、王十朋而斬思退等,思退由是始敗。



 良翰在諫省,成恭皇后受冊,官內外親屬二十五人,良翰論其冗,詔減七人。知建寧府、福建轉運副使,提點江東刑獄,移浙西,召為宗正少卿、兵部侍郎,除右諫議大夫。良翰言:「以蜀漢之師下關陜,以荊、襄、韓、魏,江、淮搗青、徐,此今日大計。四川既命大臣,而荊、淮未有任責者,亦當擇重臣臨之。」上稱善。



 進給事中。大將成閔冒請真奉,有司坐獲譴,閣門王抃矯詔遣妄
 人謝顯出境,顯既抵罪,置閔與抃不問,良翰皆駁議,請正典刑。遂改禮部侍郎,不拜,以敷文閣待制提舉江州太平興國宮。



 召為太子詹事,既見,上屬以調護之責。一日,召對選德殿,出手書唐太宗與魏徵論仁德功利之說,俾極陳今日所未至者。良翰退,上疏,略曰:「仁德治之本,功利治之效,務本而效自至。今承天意,結民心,任賢能,退小人,擇將帥,收軍情,擇監司,吏久任,,皆行之有未至,誠能革此八弊,則仁德無累,功利自致矣。」上為之嘉
 嘆,詔兼侍講。



 未幾,以疾告老,除敷文閣直學士、提舉太平宮。卒,年六十五。光宗立,特謚獻肅。



 杜莘老,字起莘,眉州青神人,唐工部甫十三世孫也。幼歲時,方禁蘇氏文,獨喜誦習。紀興間,第進士,以親老不赴廷對,賜同進士出身。授梁山軍教授,從游者眾。



 秦檜死,魏良臣參大政,莘老疏天下利害以聞。良臣薦之,主管禮、兵部架閣文字。彗星見東方,高宗下詔求言,莘老上書,論:「彗,盩氣所生,多為兵兆。國家為民息兵,而將驕
 卒惰,軍政不肅。今因天戒以修人事,思患預防,莫大於此。」因陳時弊十事。時應詔者眾,上命擇其議論切當推恩以勸之,後省以莘老為首,進一階,遷敕令刪定官、太常寺主簿,升博士。輪對,論:「金將敗盟,宜飭邊備,勿恃其不來,恃吾有以待之。」上稱善再三。



 南渡後,典秩散失,多有司所記省,至兇禮又諱不錄。顯仁皇后崩,議禮有疑,吏皆拱手,莘老以古義裁定。大斂前一日,宰相傳旨問含玉之制,莘老曰:「禮院故實所不載,請以《周禮》典瑞鄭
 玄《注》制之,其可。」國立具奏,上覽之曰:「真禮官也。」及虞祭,或謂上哀勞,欲以宰相行事。莘老曰:「古今無是。」卒正之。



 遷秘書丞,論江、淮守備,上曰:「卿言及此,憂國深矣。」擢監察御史。遷殿中侍御史,入對,上曰:「知卿不畏強御,故有此授,自是用卿矣。」陳俊卿既解言職,力求去,莘老因奏事,從容曰:「多事之際,令俊卿輩在論思之地,必有補益。」上以為然,俊卿乃復留。



 金遣使致嫚書,傳欽宗兇問,請淮、漢地,指索大臣。上決策親征,莘老疏奏贊上,且謂:「敵
 欺天背盟,當待以不懼,勿以小利鈍為異議所搖,諛言所惰,則人心有恃而士氣振矣。宜不限早暮,延見大臣、侍從,謀議國事;申敕侍從、臺諫、監司、守臣,亟舉可用之才。」又言:「親征有期,而禁衛才五千餘,羸老居半,至不能介冑者,願亟留聖慮」事皆施行。



 帶御器械劉炎筦禁中市易,通北賈,大為奸利。一日,見莘老,輒及朝政,語狂悖,莘老以聞,斥監嘉州稅。知樞密院事周麟之初請使金,及嫚書至,聞金將盛兵犯邊,乃大恐,建言不必遣使。莘
 老劾麟之:「挾奸罔上,避事辭難,恐懼至於掩泣,眾有『哭殺富鄭公』之誚。」尋與宮觀。疏再上,乃責瑞州。



 幸醫承宣使王繼先怙寵干法,富浮公室,子弟直延閣,居第僭擬,別業、外帑遍畿甸,數十年無敢搖之者,聞邊警,亟輦重寶歸吳興為避敵計。莘老疏其十罪,上曰:「初以太后鉺其藥,稍假恩寵,不謂小人驕橫乃爾。」莘老曰:「繼先罪擢發不足數,臣所奏,其大概耳。」上作而曰:「有恩無威,有賞無罰,雖堯舜不能治天下。」詔繼先福州居住,子孫皆勒
 停。籍其貲以千萬計,詔鬻錢入御前激賞庫,專以賞將士,天下稱快。



 內侍張去為取御馬院西兵二百髡其頂,都人異之,口語籍籍。莘老彈治,上疑其未審,不樂。莘老執奏不已,竟罷去為御馬院,致仕,而莘老亦以直顯謨閣知遂寧府。給事中金安節、中書舍人劉珙封還制書,改司農少卿,尋請外,仍與遂寧。



 始莘老自蜀造朝,不以家行。高宗聞其清修獨處,甚重之,一日因對,褒諭曰:「聞卿出蜀,即蒲團、紙帳如僧然,難及也。」未幾,遂擢用。莘老
 官中都久,知公論所予奪,奸蠹者皆得其根本脈絡,嘗嘆曰:「臺諫當論天下第一事,若有所畏,姑言其次,是欺其心不敬其君者也。」及任言責,極言無隱,取眾所指目者悉擊去,聲振一時,都人稱骨鯁敢言者必曰杜殿院云。治郡,課績為諸州最。



 孝宗受禪,莘老進三議,曰定國是、修內政、養根本。尋卒,年五十八。



 論曰:黃洽渾厚有守,應辰學術精醇,尤稱骨鯁。十朋、吳芾、良翰、莘老相繼在臺府,歷詆奸幸,直言無隱,皆事上
 忠而自信篤,足以當大任者,惜不盡其用焉。



\end{pinyinscope}