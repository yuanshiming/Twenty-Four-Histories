\article{志第一百六 輿服五}

\begin{pinyinscope}

 諸臣服下士庶人服



 公服。凡朝服謂之具服,公服從省,今謂之常服。宋因唐制,三品以上服紫,五品以上服朱,七品以上服綠,九品以上服青。其制,曲領大袖,下施橫襴,束以革帶,帕頭,烏
 皮靴。自王公至一命之士,通服之。



 太宗太平興國二年,詔朝官出知節鎮及轉運使、副,衣緋、綠者並借紫。知防禦、團練、刺史州,衣綠者借緋,衣緋者借紫;其為通判、知軍監,止借緋。其後,江淮發運使同轉運,提點刑獄同知刺史州。雍熙初,郊祀慶成,始許升朝官服緋、綠二十年者,敘賜緋、紫。



 真宗登極,京朝官亦聽敘,及東封、西祀赦書,京朝官並以十五年為限。後每帝登極,亦如例。景德三年,詔內諸司使以下出入內庭,不得服皂衣,違者論
 其罪;內職亦許服窄袍。



 仁宗景祐元年,詔軍使曾任通判者借緋,曾任知州者借紫。慶歷元年,龍圖閣直學士任布言:「欲望自今贈官至正郎者,其畫像許服緋,至卿監許服紫。」從之。嘉祐三年,詔三路轉運使朝辭上殿日,與賜章服;諸路轉運使候及十年,即與賜章服。



 神宗熙寧元年,中書門下奏:「六品以上犯贓濫或私罪徒重者,不得因本品改章服。」從之。元豐元年,去青不用,階官至四品服紫,至六品服緋,皆象笏、佩魚,九品以上則服綠,
 笏以木。武臣、內侍皆服紫,不佩魚。假版官及伎術若公人之人入品者,並聽服綠。官應品而服色未易,與品未及而已易者,或以年格,或以特恩。五年,詔六曹尚書依翰林學士例,六曹侍郎、給事中依直學士例,朝謝日不以行、守、試並賜服佩魚;罷職除他官日,不帶行。



 徽宗重和元年,詔禮制局自冠服討論以聞,其見服靴,先改用履。禮制局奏:「履有絇、繶、純、綦,古者舄履各隨裳之色,有赤舄、白舄、黑舄。今履欲用黑革為之,其絇、繶、純、綦並隨
 服色用之,以仿古隨裳色之意。」詔以明年正旦改用。禮制局又言:「履隨其服色。武臣服色一等,當議差別。」詔文武官大夫以上具四飾,朝請郎、武功郎以下去繶,並稱履;從義郎、宣教郎以下至將校、伎術官去繶、純,並稱履。當時議者以靴不當用之中國,實廢釋氏之漸雲。



 中興,仍元豐之制,四品以上紫,六品以上緋,九品以上綠。服緋、紫者必佩魚,謂之章服。非官至本品,不以假人。若官卑而職高,則特許者有三:自庶官遷六部侍郎,自庶官
 為待制,或出奉使者是也。又有以年勞而賜者,有品未及而借者。升朝官服綠,大夫以上服緋,蒞事至今日以前及二十年歷任無過者,許磨勘改授章服,此賜者也。或為通判者,許借緋;為知州、監司者,許借紫;任滿還朝,仍服本品,此借者也。又有出於恩賜者焉。紹興十二年九月,以皇太后回鑾,詔承務郎以上服緋、綠,蒞事至今日以前十七年者,並改轉服色。



 三十二年六月,孝宗即位,詔承務郎以上服緋、綠及十五年者,並許改轉服色。
 然計年之法,亦不輕許。無出身人自年二十出官服綠日起理,服緋人亦自年二十服緋日起理,有出身人自賜出身日起理;內並除豁丁憂年、月、日不理外,歷任無過者方許焉。先是,殿中侍御史張震奏:「今日之弊,在於人有僥幸。能革其俗,然後天下可治。且改轉服色,常赦自升朝官以上服綠,大夫以上服緋,蒞事及二十年,方得改賜。今赦自承務郎以上服緋、綠及十五年,便與改轉。比之常赦,不惟年限已減,而又官品相絕,蓋已為異
 恩矣。今竊聞省、部欲自補官日便理歲月,即是嬰孩授命,年才十五者今遂服緋;而貴近之子,或初年賜緋,年才及冠者今遂賜紫。朱、紫紛紛,不亦濫乎?況靖康、建炎恩赦,亦不曾以補官日為始。若始於出官之日,頗為折衷,蓋比之蒞事所減已多,而比之初補粗為有節。」帝從其言,故有是命。



 又有出於特賜者,旌直臣則賜之,勸循吏則賜之,廣孝治則賜之,優老臣則賜之,此皆非常制焉。內品未至而賜服及借者,並於銜內帶賜及借。



 帕頭。一名折上巾,起自後周,然止以軟帛垂腳,隋始以桐木為之,唐始以羅代繒。惟帝服則腳上曲,人臣下垂。五代漸變平直。國朝之制,君臣通服平腳,乘輿或服上曲焉。其初以藤織草巾子為里,紗為表,而塗以漆。後惟以漆為堅,去其藤裏,前為一折,平施兩腳,以鐵為之。



 帶。古惟用革,自曹魏而下,始有金、銀、銅之飾。宋制尤詳,有玉、有金、有銀、有犀,其下銅、鐵、角、石、墨玉之類,各有等差。玉帶不許施於公服。犀非品官、通犀非特旨皆禁。銅、
 鐵、角、石、墨玉之類,民庶及郡縣吏、伎術等人,皆得服之。



 其制有金球路、荔支、師蠻、海捷、寶藏,方團二十五兩;荔支自二十五兩至七兩,有四等;師蠻二十五兩;海捷十五兩;寶藏三十兩。惟球路方團胯,餘悉方胯。荔支或為御仙花,束帶亦同。



 金塗天王、八仙、犀牛、寶瓶、荔支、師蠻、海捷、雙鹿、行虎、窪面。



 天王、八仙二十五兩;犀牛、寶瓶自二十五兩至十五兩,有二等;荔支自二十兩至十兩,有三等;師蠻自二十兩至十八兩,有二等;海捷自十五兩至十兩,有三等;雙鹿自二十兩至四兩,有九等;行虎七兩;雱面自十五兩至十二兩,有二等。



 束帶則有金荔支、師蠻、戲童、海捷、犀牛、胡荽、鳳子、寶相花,荔支自二十五兩至十五兩,有三等;師蠻、戲童二十五兩;海捷自二十兩至十兩,有二等;
 犀牛二十兩;鳳子、寶相花十五兩。



 金塗犀牛、雙鹿、野馬、胡荽。犀牛、野馬十五兩;雙鹿自二十兩,有三等;胡荽自十五兩至十兩,有三等。



 犀有上等、次等,以牯牸為別。出黔南者,在南海之下。



 太宗太平興國七年正月,翰林學士承旨李昉等奏曰:「奉詔詳定車服制度,請從三品以上服玉帶,四品以上服金帶,以下升朝官、雖未升朝已賜紫緋、內職諸軍將校,並服紅□金塗銀排方。雖升朝著綠者,公服上不得系銀帶,餘官服黑銀方團胯及犀角帶。貢士及胥吏、工商、庶人服鐵角帶,恩賜者不用此制。荔支
 帶本是內出以賜將相,在於庶僚,豈合僭服?望非恩賜者,官至三品乃得服之。」景德三年,詔通犀、金、玉帶,除官品合服及恩賜外,餘人不得服用。大中祥符五年,詔曰:「方團金帶,優寵輔臣,今文武庶官及伎術之流,率以金銀仿效,甚紊彞制。自今除恩賜外,悉禁之。」端拱中,詔作瑞草地球路文方團胯帶,副以金魚,賜中書、樞密院文臣。



 仁宗慶歷八年,彰信軍節度使兼侍中李用和言:「伏見張耆授兼侍中日,特賜笏頭金帶以為榮異,欲望正
 謝日,準例特賜。」詔如耆例。



 神宗熙寧六年,熙河路奏捷,宰臣王安石率群臣賀紫宸殿,神宗解所服白玉帶賜之。八年,岐王顥、嘉王頵言:「蒙賜方團玉帶,著為朝儀,乞寶藏於家,不敢服用。」神宗不許,命工別琢玉帶以賜之。顥等固辭,不聽;請加佩金魚以別嫌,詔以玉魚賜之。親王佩玉魚自此始。宗旦、宗諤皆以使相遇郊恩告謝,特賜球文方團金帶、佩魚,自是宗室節度帶同平章事者,著為例。宣徽使張方平、郭逵、王拱辰皆嘗特賜。元豐五
 年,詔:「三師、三公、宰相、執政官、開府儀同三司、節度使嘗任宰相者、觀文殿大學士已上,金球文方團帶,佩魚。觀文殿學士至寶文閣直學士、節度使、御史大夫、中丞、六曹尚書、侍郎、散騎常侍禦仙花帶,內御史大夫、六曹尚書、翰林學士以上及資政殿學士特班翰林學士上者,仍佩魚。」六年,詔:「北使經過處,守臣曾借朝議大夫者,令權服紫,不系金帶。其押賜御筵官仍互借,先借朝議大夫者,即借中散大夫,並許系金帶,不佩魚。」哲宗元祐五
 年,詔:臣僚曾賜金帶後至不該系者,在外許系。



 徽宗崇寧二年,詔:六尚局奉御,今後許服金帶。四年,中書省檢會哲宗《元符儀制令》:「諸帶,三師、三公、宰相、執政官、使相、節度使、觀文殿大學士球文,佩魚。節度使非曾任宰相即御仙花,佩魚。觀文殿學士至寶文閣直學士、御史大夫、中丞、六曹尚書、侍郎、散騎常侍並御仙花,權侍郎不同;內御史大夫、六曹尚書、觀文殿學士至翰林學士仍佩魚,資政殿學士特旨班在翰林學士上者同,權尚書
 不同。其官職未至而特賜者,不拘此令。因任職事官經賜金帶者,雖後任不該賜,亦許服。」看詳:若稱因任六曹侍郎經賜帶,後除知開封府之類,既非職事官,又非在外,皆不許系,似非元立法之意。蓋立文該舉未盡,其特賜者既不緣官職,自無時不許系外;因任職事官賜金帶,後任不該者亦許服,即在外與在京非職事官,皆可用。詔申明行下。大觀二年,詔中書舍人、諫議大夫、待制、殿中少監許系紅□犀帶,不佩魚。



 中興仍之,其等亦有
 玉、有金、有銀、有金塗銀、有犀、有通犀、有角。其制,球文者四方五團,御仙花者排方。凡金帶:三公、左右丞相、三少、使相、執政官、觀文殿大學士、節度使球文,佩魚;觀文殿學士至華文閣直學士、御史大夫、中丞、六曹尚書、侍郎、散騎常侍、開封尹、給事中並御仙花,內御史大夫、六曹尚書、觀文殿學士至翰林學士仍佩魚;中書舍人、左右諫議大夫、龍圖天章寶文顯謨徽猷敷文煥章華文閣待制、權侍郎服紅□排方黑犀帶,仍佩魚;權侍郎以上
 罷任不帶職者,亦許服之。



 魚袋。其制自唐始,蓋以為符契也。其始曰魚符,左一,右一。左者進內,左者隨身,刻官姓名,出入合之。因盛以袋,故曰魚袋。宋因之,其制以金銀飾為魚形,公服則系於帶而垂於後,以明貴賤,非復如唐之符契也。



 太宗雍熙元年,南郊後,內出以賜近臣,由是內外升朝文武官皆佩魚。凡服紫者,飾以金;服緋者,飾以銀。庭賜紫,則給金塗銀者;賜緋,亦有特給者。京官、幕職州縣官賜緋紫者,
 亦佩。親王武官、內職將校皆不佩。真宗大中祥符六年,詔伎術官未升朝賜緋、紫者,不得佩魚。



 仁宗天聖二年,翰林待詔、太子中舍同正王文度因勒碑賜紫章服,以舊佩銀魚,請佩金魚。仁宗曰:「先朝不許伎術人輒佩魚,以別士類,不令混淆,宜卻其請。」景祐三年,詔殿中省尚藥奉御賜紫徐安仁,特許佩魚。至和元年,詔:中書提點五房公事,自今雖無出身,亦聽佩魚。舊制,自選人入為堂後官,轉至五房提點,始得佩魚。提點五房呂惟和
 非選人入,援司天監五官正例求佩魚,特許之。



 神宗元豐二年,蒲宗孟除翰林學士,神宗曰:「學士職清地近,非它官比,而官儀未寵,自今宜加佩魚。」遂著為令。三年,詔:自今中書堂後官,並帶賜緋魚袋,餘依舊例。徽宗政和元年,尚書兵部侍郎王詔奏:「今監司、守、倅等,並許借服色而不許佩魚,即是有服而無章,殆與吏無別。乞今後應借緋、紫臣僚,並許隨服色佩魚,仍各許入銜,候回日依舊服色。」從之。中興,並仍舊制。



 笏。唐制五品以上用象。上圓下方;六品以下用竹、木,上挫下方。宋文散五品以上用象,九品以上用木。武臣、內職並用象,千牛衣綠亦用象,廷賜緋、綠者給之。中興同。



 靴。宋初沿舊制,朝履用靴。政和更定禮制,改靴用履。中興仍之。乾道七年,復改用靴,以黑革為之,大抵參用履制,惟加靿焉。其飾亦有絇、繶、純、綦,大夫以上具四飾,朝請、武功郎以下去繶,從義、宣教郎以下至將校、伎術官並去純。底用麻再重,革一重。里用素衲氈,高八寸。諸文
 武官通服之,惟以四飾為別。服綠者飾以綠,服緋、紫者飾亦如之,仿古隨裳色之意。



 簪戴。帕頭簪花,謂之簪戴。中興,郊祀、明堂禮畢回鑾,臣僚及扈從並簪花,恭謝日亦如之。大羅花以紅、黃、銀紅三色,欒枝以雜色羅,大絹花以紅、銀紅二色。羅花以賜百官,欒枝,卿監以上有之;絹花以賜將校以下。太上兩宮上壽畢,及聖節、及錫宴、及賜新進士聞喜宴,並如之。



 重戴。唐士人多尚之,蓋古大裁帽之遺制,本野夫巖叟
 之服。以皂羅為之,方而垂簷,紫裏,兩紫絲組為纓,垂而結之頷下。所謂重戴者,蓋折上巾又加以帽焉。宋初,御史臺皆重戴,餘官或戴或否。後新進士亦戴,至釋褐則止。太宗淳化二年,御史臺言:「舊儀,三院御史在臺及出使,並重戴,事已久廢。其御史出臺為省職及在京厘務者,請依舊儀,違者罰俸一月。」從之。又詔兩省及尚書省五品以上皆重戴,樞密三司使、副則不。中興後,御史、兩制、知貢舉官、新進士上三人,許服之。



 時服。宋初因五代舊制,每歲諸臣皆賜時服,然止賜將相、學士、禁軍大校。建隆三年,太祖謂侍臣曰:「百官不賜,甚無謂也。」乃遍賜之。歲遇端午、十月一日,文武群臣將校皆給焉。是歲十月,近臣、軍校增給錦襯袍,中書門下、樞密、宣徽院、節度使及侍衛步軍都虞候以上,皇親大將軍以上,天下樂暈錦;三司使、學士、中丞、內客省使、駙馬、留後、觀察使,皇親將軍、諸司使、廂主以上,簇四盤雕細錦;三司副使、宮觀判官,黃師子大錦;防禦團練使、刺
 史、皇親諸司副使,翠毛細錦;權中丞、知開封府、銀臺司、審刑院及待制以上,知檢院鼓院、同三司副使、六統軍、金吾大將軍,紅錦。諸班及諸軍將校,亦賜窄錦袍。有翠毛、宜男、雲雁細錦,師子、練鵲、寶照大錦,寶照中錦,凡七等。



 應給錦袍者,皆五事;公服、錦寬袍,綾汗衫、褲,勒帛,丞郎、給舍、大卿監以上不給錦袍者,加以黃綾繡抱肚。



 大將軍、少卿監、郎中以上,樞密諸房副承旨以上,諸司使,皇親承制、崇班,皆四事;無錦袍。



 將軍至副率、知雜御史至大理正、入內都知、內侍都知、皇親殿直以
 上,皆三事;無褲。



 通事舍人、承制、崇班、入內押班、內侍副都知押班、內常侍、六尚奉御以下,京官充館閣、宗正寺、刑法官者,皆二事;無勒帛,內職汗衫以綾,文臣以絹。



 閣門祗候、內供奉官至殿直,京官編修、校勘,止給公服。端午,亦給。應給錦袍者,汗衫以黃縠,別加繡抱肚、小扇。誕聖節所給,如時服。



 京師禁廂軍校、衛士、內諸司胥史、工巧人,並給服有差。



 朝官、京官、內職出為外任通判、監押、巡檢以上者,大藩府監務者,亦或給之。



 每歲十月時服,開寶中,皆賜窄錦袍。太平興國以後,文官知制誥、武官上
 將軍、內職諸司使以上,皆賜錦。



 藩鎮觀察使以上,天下樂暈錦;尚書及步軍都虞候以上及知益州、並州,次暈錦,皆五事。學士、丞郎,簇四盤雕錦;刺史以上及知廣州,翠毛錦,皆三件。待制以上、橫班諸司使,翠毛錦;知代州,御仙花錦;諸司使領郡,宜男錦;諸司使,雲雁錦。駙馬,錦如丞郎,增至四事。益州鈐轄,錦從本官,增綾褲。



 朝官供奉官以上,皆賜紫地皂花欹正。京官殿直以下,皆賜紫大綾。在外禁軍將校,亦賜窄錦袍,次賜紫綾色絹。景德元年,始詔河北、河東、陜西三路轉運使、副,並給方勝練鵲錦。校獵從官兼賜紫羅錦、旋襴、暖靴。



 雍熙四年,令節度使給皂地金線盤雲鳳鹿胎旋
 襴,侍衛步軍都虞候以上給皂地金線盤花鴛鴦。



 親王、宰相、使相生日,並賜衣五事,錦彩百匹,金花銀器百兩,馬二匹,金塗銀鞍勒一。宰相、樞密使、參知政事、樞密副使、宣徽使初拜、加恩中謝日,並賜衣五事,金帶一,舊荔支帶,淳化後,宰相、參知政事、文臣任樞密副使,改賜方團胯球路金帶,加以金魚。



 塗金銀鞍勒馬一。三司使、學士、御史中丞初拜中謝日,賜衣五事,荔支金帶一,塗金銀鞍勒馬一。文明學士以下,初賜金裝犀帶,後改賜金帶。



 中書舍人,賜襲衣、犀帶。宰相以下對御抬賜;樞密直學士、中
 書舍人謝訖,中使押賜,再入謝於別殿。中書舍人或告謝日已改賜章服,則罷中使押賜。



 郊禋禮畢,親王、宰相至龍圖閣直學士、禁軍將校,各賜襲衣、金帶,親王、中書門下、樞密、宣徽、三司使、四廂都指揮使以上,加鞍勒馬一。其後宮觀副使、天書扶侍使,並同學士。



 同中謝日。雍熙元年,兩省五品以上,御史臺、尚書省四品以上,各賜襲衣、犀帶、魚袋。其為五使,則皆賜金帶,仍各加器幣。



 文武行事官,各賜金帛。牧伯在外者,遇大禮,不賜。大中祥符元年,詔節度、觀察、防禦、團練使,刺史,因東封為諸州部署鈐轄者,並特賜焉。



 使相、節度使自鎮來朝入見日,賜衣五事,
 金帶,鞍馬;朝辭日,賜窄衣六事,金束帶,鞍勒馬一,散馬二;節度使減散馬。



 為都部署者,別賜帶甲鞍勒馬一。觀察使為都部署、副都部署赴本任、知州,賜窄衣三事,金束帶,鞍勒馬。防禦團練使、刺史為部署、鈐轄,賜窄衣三事,金束帶;赴本任,賜窄衣三事,塗金銀腰帶;為知州、都監,賜窄衣三事,絹三十匹。諸司為鈐轄者,賜窄衣、金束帶。文武官內職出為知州軍、通判、發運、轉運使副、提點刑獄、都監、巡檢、砦主、軍使及任使繁要者,僕射賜窄衣三事,絹
 五十匹;尚書、丞郎、學士、諫舍、待制、大卿監及統軍、上將軍、諸司使,減絹二十匹;少卿監至五官正、大將軍至副率、諸司副使,減絹一十匹;中郎將、京官內殿承制至借職、內常侍,減衣二事,又減絹一十匹。窄衣,起二月給紫羅衫;起十月給紫欹正錦襖。



 給公服者,單夾亦然。



 諸道衙內指揮使、都虞候入貢辭日,賜紫羅窄衫,金塗銀帶。



 士庶人車服之制。太宗太平興國七年,詔曰:「士庶之間,車服之制,至於喪葬,各有等差。近年以來,頗成逾僭。宣
 令翰林學士承旨李昉詳定以聞。」昉奏:「今後富商大賈乘馬,漆素鞍者勿禁。近年品官綠袍及舉子白襴下皆服紫色,亦請禁之。其私第便服,許紫皂衣、白袍。舊制,庶人服白,今請流外官及貢舉人、庶人通許服皂。工商、庶人家乘簷子,或用四人、八人,請禁斷,聽乘車;兜子,舁不得過二人。」並從之。端拱二年,詔縣鎮場務諸色公人並庶人、商賈、伎術、不系官伶人,只許服皂、白衣,鐵、角帶,不得服紫。文武升朝官及諸司副使、禁軍指揮使、廂軍都
 虞候之家子弟,不拘此限。帕頭巾子,自今高不過二寸五分。婦人假髻並宜禁斷,仍不得作高髻及高冠。其銷金、泥金、真珠裝綴衣服,除命婦許服外,餘人並禁。至道元年,復許庶人服紫。



 真宗咸平四年,禁民間造銀鞍瓦、金線、盤蹙金線。大中祥符元年,三司言:「竊惟山澤之寶,所得至難,儻縱銷釋,實為虛費。今約天下所用,歲不下十萬兩,俾上幣棄於下民。自今金銀箔線,貼金、銷金、泥金、蹙金線裝貼什器土木玩用之物,並請禁斷,非命婦
 不得以為首飾。冶工所用器,悉送官。諸州寺觀有以金箔飾尊像者,據申三司,聽自繼金銀工價,就文思院換給。」從之。二年,詔申禁熔金以飾器服。又太常博士知溫州李邈言:「兩浙僧求丐金銀、珠玉,錯末和泥以為塔像,有高袤丈者。毀碎珠寶,浸以成俗,望嚴行禁絕,違者重論。」從之。



 七年,禁民間服銷金及鈸遮那纈。八年,詔:「內庭自中宮以下,並不得銷金、貼金、間金、戭金、圈金、解金、剔金、陷金、明金、泥金、楞金、背影金、盤金、織金、金線捻絲,裝
 著衣服,並不得以金為飾。其外庭臣庶家,悉皆禁斷。臣民舊有者,限一月許回易。為真像前供養物,應寺觀裝功德用金箔,須具殿位真像顯合增修創造數,經官司陳狀勘會,詣實聞奏,方給公憑,詣三司收買。其明金裝假果、花枝、樂身之類,應金為裝彩物,降詔前已有者,更不毀壞,自餘悉禁。違者,犯人及工匠皆坐。」是年,又禁民間服皂班纈衣。



 仁宗天聖三年,詔:「在京士庶不得衣黑褐地白花衣服並藍、黃、紫地撮暈花樣,婦女不得將白
 色、褐色毛段並淡褐色匹帛制造衣服,令開封府限十日斷絕;婦女出入乘騎,在路披毛褐以御風塵者,不在禁限。」七年,詔士庶、僧道無得以朱漆飾床榻。九年,禁京城造朱紅器皿。


景祐元年,詔禁錦背、繡背、遍地密花透背採段,其稀花團窠、斜窠雜花不相連者非。二年,詔:市肆造作縷金為婦人首飾等物者禁。三年,「臣庶之家,毋得採捕鹿胎制造冠子。又屋宇非邸店、樓閣臨街市之處,毋得為四鋪作鬧斗八;非品官毋得起門屋;非宮室、
 寺觀毋得彩繪棟宇及朱黝漆梁柱窗牖、雕鏤柱礎。凡器用毋得表裏朱漆、金漆,下毋得襯朱。非三品以上官及宗室、戚里之家,毋得用金棱器,其用銀者毋得塗金。玳瑁酒食器,非宮禁毋得用。純金器若經賜者,聽用之。凡命婦許以金為首飾,及為小兒鈐鐲、釵篸、金川纏、珥環之屬;仍毋得為牙魚、飛魚、奇巧飛動若龍形者。非命婦之家,毋得以真珠裝綴首飾、衣服,及項珠、纓絡、耳墜、頭
 \gezhu{
  須巾}
 、抹子之類。凡帳幔、繳壁、承塵、柱衣、額道、項帕、覆旌、床
 裙,毋得用純錦遍繡。宗室戚裏茶簷、食合,毋得以緋紅蓋覆。豪貴之族所乘坐車,毋得用朱漆及五彩裝繪,若用黝而間以五彩者聽。民間毋得乘簷子,及以銀骨朵、水罐引喝隨行。」



 慶歷八年,詔禁士庶效契丹服及乘騎鞍轡、婦人衣銅綠兔褐之類。皇祐元年,詔婦人冠高毋得逾四寸,廣毋得逾尺,梳長毋得逾四寸,仍禁以角為之。先是,宮中尚白角冠梳,人爭仿之,至謂之內樣。冠名曰垂肩等肩,至有長三尺者;梳長亦逾尺。議者以為服妖,
 遂禁止之。嘉祐七年,初,皇親與內臣所衣紫,皆再入為黝色。後士庶浸相效,言者以為奇邪之服,於是禁天下衣黑紫服者。



 神宗熙寧九年,禁朝服紫色近黑者;民庶止令乘犢車,聽以黑飾,間五彩為飾,不許呵引及前列儀物。哲宗紹聖二年,侍御史翟思言:「京城士人與豪右大姓,出入率以轎自載,四人舁之,甚者飾以棕蓋,徹去簾蔽,翼其左右,旁午於通衢,甚為僭擬,乞行止絕。」從之。



 徽宗大觀元年,郭天信乞中外並罷翡翠裝飾,帝曰:「先王之
 政,仁及草木禽獸,今取其羽毛,用於不急,傷生害性,非先王惠養萬物之意。宜令有司立法禁之。」政和二年,詔後苑造纈帛。蓋自元豐初,置為行軍之號,又為衛士之衣,以辨奸詐,遂禁止民間打造。令開封府申嚴其禁,客旅不許興販纈板。



 七年,臣僚上言:「輦轂之下,奔競侈靡,有未革者。居室服用以壯麗相誇,珠璣金玉以奇巧相勝,不獨貴近,比比紛紛,日益滋甚。臣嘗考之,申令法禁雖具,其罰尚輕,有司玩習,以至於此。如民庶之家不得
 乘轎,今京城內暖轎,非命官至富民、娼優、下賤,遂以為常。竊見近日有赴內禁乘以至皇城門者,奉祀乘至宮廟者,坦然無所畏避。臣妄以為僭禮犯分,禁亦不可以緩。」於是詔,非品官不得乘暖轎。先是,權發遣提舉淮南東路學事丁瓘言:「衣服之制,尤不可緩。今閭閻之卑,倡優之賤,男子服帶犀玉,婦人塗飾金珠,尚多僭侈,未合古制。臣恐禮官所議,止正大典,未遑及此。伏願明詔有司,嚴立法度,酌古便今,以義起禮。俾閭閻之卑,不得與
 尊者同榮;倡優之賤,不得與貴者並麗。此法一正,名分自明,革澆偷以歸忠厚,豈曰小補之哉。」是歲,又詔敢為契丹服若氈笠、釣墩之類者,以違御筆論。釣墩,今亦謂之襪褲,婦人之服也。



 中興,士大夫之服,大抵因東都之舊,而其後稍變焉。一曰深衣,二曰紫衫,三曰涼衫,四曰帽衫,五曰襴衫。淳熙中,朱熹又定祭祀、冠婚之服,特頒行之。凡士大夫家祭祀、冠婚,則具盛服。有官者帕頭、帶、靴、笏,進士則帕頭、襴
 衫、帶,處士則帕頭、皂衫、帶,無官者通用帽子、衫、帶;又不能具,則或深衣,或涼衫。有官者亦通用帽子以下,但不為盛服。婦人則假髻、大衣、長裙。女子在室者冠子、背子。眾妾則假紒、背子。



 冠禮,三加冠服,初加,緇布冠、深衣、大帶、納履;再加,帽子、皂衫、革帶、系鞋;三加,帕頭、公服、革帶、納靴。其品官嫡庶子初加,折上巾、公服;再加,二梁冠、朝服;三加,平冕服,若以巾帽、折上巾為三加者,聽之。深衣用白細布,度用指尺,衣全四幅,其長過脅,下屬於裳。裳
 交解十二幅,上屬於衣,其長及踝。圓袂方領,曲裾黑緣。大帶、緇冠、幅巾、黑履。士大夫家冠昏、祭祀、宴居、交際服之。



 紫衫。本軍校服。中興,士大夫服之,以便戎事。紹興九年,詔公卿、長吏服用冠帶,然迄不行。二十六年,再申嚴禁,毋得以戎服臨民,自是紫衫遂廢。士大夫皆服涼衫,以為便服矣。



 涼衫。其制如紫衫,亦曰白衫。乾道初,禮部侍郎王□嚴奏:「
 竊見近日士大夫皆服涼衫,甚非美觀,而以交際、居官、臨民,純素可憎,有似兇服。陛下方奉兩宮,所宜革之。且紫衫之設以從戎,故為之禁,而人情趨簡便,靡而至此。文武並用。本不偏廢,朝章之外,宜有便衣,仍存紫衫,未害大體。」於是禁服白衫,除乘馬道塗許服外,餘不得服。若便服,許用紫衫。自後,涼衫祗用為兇服矣。



 帽衫。帽以烏紗、衫以皂羅為之,角帶,系鞋。東都時,士大夫交際常服之。南渡後,一變為紫衫,再變為涼衫,自是
 服帽衫少矣。惟士大夫家冠昏、祭祀猶服焉。若國子生,常服之。



 襴衫。以白細布為之,圓領大袖,下施橫襴為裳,腰間有闢積。進士及國子生、州縣生服之。



 紹興五年,高宗謂輔臣曰:「金翠為婦人服飾,不惟靡貨害物,而侈靡之習,實關風化。已戒中外,及下令不許入宮門,今無一人犯者。尚恐士民之家未能盡革,宜申嚴禁,仍定銷金及採捕金翠罪賞格。」淳熙二年,孝宗宣示中宮苧衣曰:「珠玉就
 用禁中舊物,所費不及五萬,革弊當自宮禁始。」因問風俗,龔茂良奏:「由貴近之家,放效宮禁,以致流傳民間。粥簪珥者,必言內樣。彼若知上崇尚淳樸,必觀感而化矣。臣又聞中宮服浣濯之衣,數年不易。請宣示中外,仍敕有司嚴戢奢僭。」寧宗嘉泰初,以風俗侈靡,詔官民營建室屋,一遵制度,務從簡樸。又以宮中金翠,燔之通衢,貴近之家,犯者必罰。



\end{pinyinscope}