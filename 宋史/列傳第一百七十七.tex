\article{列傳第一百七十七}

\begin{pinyinscope}

 ○吳潛程元鳳江萬里王爚章鑒陳宜中文天祥



 吳潛,字毅夫,宣州寧國人。秘閣修撰柔勝之季子。嘉定十年進士第一,授承事郎、簽鎮東軍節度判官。改簽廣
 德軍判官。丁父憂,服除,授秘書省正字,遷校書郎,添差通判嘉興府,權發遣嘉興府事。轉朝散郎、尚書金部員外郎。



 紹定四年,遷尚右郎官。都城大火,潛上疏論致災之由:「願陛下齋戒修省,恐懼對越,菲衣惡食,必使國人信之,毋徒減膳而已。疏損聲色,必使天下孚之,毋徒徹樂而已。閹官之竊弄威福者勿親,女寵之根萌禍患者勿暱。以暗室屋漏為尊嚴之區,而必敬必戒,以恆舞酣歌為亂亡之宅,而不淫不泆。使皇天后土知陛下有畏
 之之心,使三軍百姓知陛下有憂之之心。然後,明詔二三大臣,和衷竭慮,力改弦轍,收召賢哲,選用忠良。貪殘者屏,回邪者斥,懷奸黨賊者誅,賈怨誤國者黜。毋並進君子、小人以為包荒,毋兼容邪說、正論以為皇極,以培國家一線之脈,以救生民一旦之命。庶幾天意可回,天災可息,弭災為祥,易亂為治。」



 又言:「重地要區,當豫畜人才以備患。論大順之理,貫通天人,當以此為致治之本。」又貽書丞相史彌遠論事:一曰格君心,二曰節奉給,三
 曰振恤都民,四曰用老成廉潔之人,五曰用良將以禦外患,六曰革吏弊以新治道。授直寶章閣、浙東提舉常平,辭不赴。改吏部員外郎兼國史編修、實錄檢討,遷太府少卿、淮西總領。



 又告執政,論用兵復河南不可輕易,以為:「金人既滅,與北為鄰,法當以和為形,以守為實,以戰為應。自荊襄首納空城,合兵攻蔡,兵事一開,調度浸廣,百姓狼狽,死者枕藉,使生靈肝腦塗地,得城不過荊榛之區,獲俘不過暖昧之骨,而吾之內地荼毒如此,邊
 臣誤國之罪,不待言矣。聞有進恢復之畫者,其算可謂俊傑,然取之若易,守之實難。征行之具,何所取資,民窮不堪,激而為變,內郡率為盜賊矣。今日之事,豈容輕議。」自後,興師入洛,潰敗失亡不貲,潛之言率驗。遷太府卿兼權沿江制置、知建康府、江東安撫留守。上疏論保蜀之方,護襄之策,防江之算,備海之宜,進取有甚難者三事。



 端平元年,詔求直言,潛所陳九事:一曰顧天命以新立國之意,二曰植國本以廣傳家之慶,三曰篤人倫以
 為綱常之宗主,四曰正學術以還斯文之氣脈,五曰廣畜人才以待乏絕,六曰實恤民力以致寬舒,七曰邊事當鑒前轍以圖新功,八曰楮幣當權新制以解後憂,九曰盜賊當探禍端而圖長策。以直論忤時相,罷奉千秋鴻禧祠。改秘閣修撰、權江西轉運副使兼知隆興府,主管江西安撫司。擢太常少卿,奏造斛斗輸諸郡租,寬恤人戶,培植根本,凡十五事。



 進右文殿修撰、集英殿修撰、樞密都承旨、督府參謀官兼知太平州,五辭不允。又言
 和戰成敗大計,宜急救襄陽等事。貽書執政,論京西既失,當招收京淮丁壯為精兵,以保江西。權工部侍郎、知江州,辭不赴。請養宗子以系國本,以鎮人心。改權兵部侍郎兼檢正。論士大夫私意之敝,以為:「襄、漢潰決,興、沔破亡,兩淮俶擾,三川陷沒。欲望陛下念大業將傾,士習已壞,以靜專察群情,以剛明消眾慝,警於有位,各勵至公。毋以術數相高,而以事功相勉;毋以陰謀相訐,而以識見相先。協謀並智,戮力一心,則危者尚可安,而衰證
 尚可起也。」又請分路取士,以收淮、襄之人物。



 試工部侍郎、知慶元府兼沿海制置使,改知平江府,條具財計凋敝本末,以寬郡民,與轉運使王野爭論利害。授寶謨閣待制,提舉太平興國宮,改玉隆萬壽宮。試戶部侍郎、淮東總領兼知鎮江府。言邊儲防禦等十有五事。改寶謨閣直學士,兼浙西都大提點坑冶,權兵部尚書、浙西制置使。申論防拓江海,團結措置等事。



 進工部尚書,改吏部尚書兼知臨安府,乃論艱屯蹇困之時,非反身修德,
 無以求亨通之理。乞遴選近族以系人望,而俟太子之生。帝嘉納。兼侍讀經筵,以臺臣徐榮叟論列,授寶謨閣學士、知紹興府、浙東安撫使,辭,提舉南京鴻慶宮。遂請致仕,授華文閣學士知建寧府,辭。



 丁母憂,服除,轉中大夫、試兵部尚書兼侍讀,轉翰林學士、知制誥兼侍讀,改端明殿學士、簽書樞密院事,進封金陵郡侯。以亢旱乞罷,免,改資政殿學士、提舉洞霄宮,改知福州兼本路安撫使。徙知紹興府、浙東安撫使。



 召同知樞密院兼參知
 政事。入對,言:「國家之不能無敝,猶人之不能無病。今日之病,不但倉、扁望之而驚,庸醫亦望而驚矣。願陛下篤任元老,以為醫師,博採眾益,以為醫工。使臣輩得以效牛溲馬勃之助,以不辱陛下知人之明。」



 淳祐十一年,入為參知政事,拜右丞相兼樞密使。明年,以水災乞解機政。以觀文殿大學士、提舉洞霄宮。又四年,授沿海制置大使,判慶元府。至官,條具軍民久遠之計,告於政府,奏皆行之。又積錢百四十七萬三千八百有奇,代民輸帛,
 前後所蠲五百四十九萬一千七百有奇。以久任丐祠,且累章乞歸田里,進封崇國公,判寧國府。還家,以醴泉觀使兼侍讀,召入對,論畏天命,結民心,進賢才,通下情。帝嘉納。拜特進、左丞相,進封慶國公。奏:「乞令在朝之臣各陳所見,以決處置之宜。」改封許國公。



 大元兵渡江攻鄂州,別將由大理下交址,破廣西、湖南諸郡。潛奏:「今鄂渚被兵,湖南擾動,推原禍根,良由近年奸臣憸士設為虛議,迷國誤軍,其禍一二年而愈酷。附和逢迎,壬MF諂
 媚,積至於大不靖。臣年將七十,捐軀致命,所不敢辭。所深痛者,臣交任之日,上流之兵已逾黃、漢,廣右之兵已蹈賓、柳,謂臣壞天下之事,亦可哀已。」



 又論國家安危治亂之原:「蓋自近年公道晦蝕,私意橫流,仁賢空虛,名節喪敗,忠嘉絕響,諛佞成風,天怒而陛下不知,人怨而陛下不察,稔成兵戈之禍,積為宗社之憂。章鑒、高鑄嘗與丁大全同官,傾心附麗,躐躋要途。蕭泰來等群小噂沓,國事日非,浸淫至於今日。陛下稍垂日月之明,毋使小
 人翕聚,以貽善類之禍。沈炎實趙與TP之腹心爪牙,而任臺臣,甘為之搏擊。奸黨盤據,血脈貫穿,以欺陛下。致危亂者,皆此等小人為之。」又乞令大全致仕,炎等與祠,高鑄羈管州軍。不報。



 屬將立度宗為太子,潛密奏云:「臣無彌遠之材,忠王無陛下之福。」帝怒潛,卒以炎論劾落職。命下,中書舍人洪芹繳還詞頭,不報,謫建昌軍,尋徙潮州,責授化州團練使、循州安置。潛預知死日,語人曰:「吾將逝矣,夜必雷風大作。」已而果然,四鼓開霽,撰遺表,
 作詩頌,端坐而逝。時景定三年五月也。循人聞之,咨嗟悲慟。德祐元年,追復元官,仍還執政恩數。明年,以太府卿柳岳請贈謚,特贈少師。



 程元鳳,字申甫,徽州人。紹定元年進士,調江陵府教授。端平元年,差江西轉運司干辦公事。丁母憂。淳祐元年,遷禮、兵二部架閣,以父老不忍去側,遷太學正,以祖諱辭,改國子錄。父憂,服闋,遷太學博士,改宗學博士。以《詩》、《禮》講榮王府。旁諷曲諭,隨事規正,多所裨益,王亦傾心
 敬聽。輪對,極論世運剝復之機及人主所當法天者。理宗覽之曰:「有古遺直風。」



 六年,進秘書丞兼權刑部郎官。七年,兼權右司郎官,遷著作郎,仍權右司郎官。輪對,指陳時病尤激切,當國者以為厲己。丐外,知饒州。郡初罹水災,元鳳訪民疾苦,夙夜究心,修城堞,置義阡,寬誅求,察誣證。進江、淮、荊、浙、福建、廣南都大提點坑冶,仍兼知饒州冶司,歲有冬夏帳銀,悉舉以補郡積年諸稅斂之不足者。芝生治所,眾以治行之致,元鳳曰:「五穀熟則民
 蒙惠,此不足異也。」



 召奏事,辭,不允,遷右曹郎官。疏言實學、實政、國本、人才、吏治、生民、財計、兵威八事。尋兼右司郎官,拜監察御史兼崇政殿說書。丞相鄭清之久專國柄,老不任事,臺官潘凱、吳燧合章論列,清之不悅,改遷之,二人不拜命去。元鳳上疏斥清之罪,其言明白正大,凱、燧得召還。有事於明堂,元鳳疏言「祈天以實不以文」。又言邊備,謂「當申儆軍實,以起積玩之勢。」及言濫刑之敝。十二年,拜右正言兼侍講,以祖諱辭。詔權以右補闕
 系銜。上疏論格心之學,謂「革士大夫之風俗,當革士大夫之心術。」至於文敝、邊儲、人才、民心、儲將帥、救災異,莫不盡言。



 餘晦以從父天錫恃恩妄作,三學諸生伏闕上書白其罪狀,司業蔡抗又力言之,元鳳數其罪劾之。奏上,以晦為大理少卿,抗為宗正少卿。元鳳又上疏留抗而黜晦,以安士心。乃命抗仍兼司業,晦予郡。



 升殿中侍御史,仍兼侍講。京城災,疏言:「輟土木無益之役,以濟暴露之民;移緇流泛濫之恩,以給顛沛之眾。務行寬大之
 政,固結億兆之心。旁招俊乂,而私暱無濫及之恩;屏去奸私,而貪黷無覆出之患。謹便嬖之防,而不使之弄權;抑恩澤之請,而不至於無節。」言多剴切。



 寶祐元年,兼侍讀,遷侍御史,言法孝宗八事。薦名士二十餘人,進尚書吏部侍郎兼中書舍人,兼同修國史、實錄院同修撰,仍兼侍讀。亟辭,出關,不允。有事於南郊,元鳳為執綏官,答問多所開陳。帝因欲幸西太乙宮,力諫止之。三年,遷權工部尚書,力求補外,特授端明殿學士、同簽書樞密院
 事。



 蜀境與沅、靖交急,朝廷欲擇重臣出鎮上流,用徐敏子易蜀帥及用向士璧為鎮撫。元鳳請下荊南,調兵援蜀,移呂文德上沅、靖。進依前職,簽書樞密院事兼權參知政事,進參知政事,尋進拜右丞相兼樞密使,進封新安郡公。力辭,御筆勉諭,猶周回累日而後治事。疏奏正心、待臣、進賢、愛民、備邊、守法、謹微、審令八事。高、孝、光、寧四朝國史未就,奏轉任尤焴領其事,纂修成之。會丁大全謀奪相位,元鳳力辭,授觀文殿大學士判福州、福建
 安撫使。又力辭,依前職,提舉洞霄宮。



 開慶兵興,上手疏收人心、重賞罰、團結民兵數事。俄起判平江府兼淮、浙發運使。四上章乞免。三年,御筆趣行,奏免修明局米五萬石。拜特進,依前職。充醴泉觀兼侍讀。度宗即位,進少保。三年,拜少傅、右丞相兼樞密使,進封吉國公,以言罷,依舊少保、觀文殿大學士、醴泉觀使。乞致仕,不許。四年,罷觀使,以守少保、觀文殿大學士致仕。卒,遺表聞,帝震悼輟朝,特贈少師。



 元鳳之在政府也,一契家子求貳令,
 元鳳謝之曰:「除授須由資。」其人累請不許,乃以先世為言。元鳳曰:「先公疇昔相薦者,以某粗知恬退故也。今子所求躐次,豈先大夫意哉?矧以國家官爵報私恩,某所不敢。」有嘗遭元鳳論列者,其後見其可用,更薦拔之,每曰:「前日之彈劾,成其才也;今日之擢用,盡其才也。」所著《訥齋文集》若干卷。



 江萬里,字子遠,都昌人。自其父燁始業儒。大父璘,鄉稱善人,其鄰史知縣者誇其能杖嘩健士,璘俯首不答,歸
 語燁曰:「史祖父故寒士,今居官以杖士人自憙,於我心有不釋然。審爾,史氏且不昌,汝其戒之。」是夕燁妻陳夢一貴人入其家,曰:「以汝家長有善言,故來。」已而有娠,生萬里。少神雋,有鋒穎,連舉於鄉。入太學,有文聲。理宗在潛邸,嘗書其姓名幾研間。以舍選出身,歷池州教授沿江制置司準備差遣,兩浙安撫司干辦公事。召試館職,累遷著作佐郎、權尚左郎官兼樞密院檢詳文字。知吉州,創白鷺洲書院,兼提舉江西常平茶鹽。召為屯田郎
 官,未行,遷直秘閣、江西轉運判官兼權知隆興府。創宗濂書院。遷考功郎官,命旋寢。久之,以駕部郎官召,遷尚右兼侍講。



 史嵩之罷相,拜監察御史,仍兼侍講。未幾,遷右正言、殿中侍御史,又遷侍御史,未及拜。萬里器望清峻,論議風採傾動一時,帝眷注尤厚。嘗丐祠、省母疾,不許。屬弟萬頃奉母歸南康,旋以母病聞,萬里不俟報馳歸,至祁門得訃。而議者謂萬里母死,秘不奔喪,反挾妾媵自隨,於是側目萬里者,相與騰謗。萬里無以自解,坐
 是閑廢者十有二年。後陸德輿嘗辨其非辜於帝前。



 賈似道宣撫兩浙,闢參謀官。及似道同知樞密院,為京湖宣撫大使,以萬裏帶行寶章閣待制,為參謀官。大元兵圍鄂,似道以右丞兼樞密使移軍漢陽,萬里遷刑部侍郎。似道入相,萬里兼國子祭酒、侍讀。入對,遷權吏部尚書,又拜端明殿學士、同簽書樞密院事兼太子賓客。隨以言者去官。後以原職知建寧府兼權福建轉運使。已而,加資政殿學士,依舊職,知福州兼福建安撫使。



 度宗
 即位,召同知樞密院事,又兼權參知政事,遷參知政事。萬里始雖俯仰容默,為似道用,然性峭直,臨事不能無言。似道常惡其輕發,故每人不能久在位。似道以去要君,帝初即位,呼為師相,至涕泣拜留之。萬里以身掖帝云:「自古無此君臣禮,陛下不可拜,似道不可復言去。」似道不知所為,下殿舉笏謝萬里曰:「微公,似道幾為千古罪人。」然以此益忌之。



 帝在講筵,每問經史疑義及古人姓名,似道不能對,萬里常從旁代對。時王夫人頗知書,帝語夫人以
 為笑。似道聞之,積慚怒,謀逐之。萬里四丐祠,不候報出關。加資政殿大學士、知慶元府兼沿海制置使,不拜,予祠。後二年,知太平州兼提領江淮茶鹽兼江東轉運使,召拜參知政事,進封南康郡公,既至,拜左丞相兼樞密使。丐祠,加觀文殿大學士知福州,辭,依舊職,提舉洞霄宮。又授知潭州、湖南安撫大使,加特進,尋予祠。時咸淳九年,萬里年七十有六矣。



 明年,大元兵渡江,萬裏隱草野間,為游騎所執,大詬,欲自戕,既而脫歸。先是,萬里聞
 襄樊失守,鑿池芝山後圃,扁其亭曰「止水」,人莫諭其意,及聞警,執門人陳偉器手,曰:「大勢不可支,餘雖不在位,當與國為存亡。」及饒州城破,軍士執萬頃,索金銀不得,支解之。萬里竟赴止水死。左右及子鎬相繼投沼中,積尸如疊。翼日,萬里尸獨浮出水上,從者草斂之。萬里無子,以蜀人王橚子為後,即鎬也。事聞,贈太傅、益國公,後加贈太師,謚文忠。萬頃歷守大郡,為提舉江西常平茶鹽,官至正郎。城破時,郴州守趙崇榞寓居城中,亦死之。



 王爚,字仲潛,一字伯晦,紹興新昌人。登嘉定十三年進士第,知常熟縣。紹定四年,江淮制置司闢通判泰州。五年,差知滁州。端平元年,知瑞州。嘉熙元年,提轄左藏東西庫兼提轄封樁下庫。二年,遷籍田令兼督視乾辦公事。淳祐二年,改監三省樞密院門,乞免所居官,詔從之。四年,再任。五年,遷太府寺丞、秘書丞,戶部郎官、淮西總領,主管右曹。六年,為尚書左司員外郎。賜對,乞祠,不許。七年,遷秘書少監,以侍御史周坦言,罷為福建提點刑
 獄,差知溫州。十年,差知寧國府,遷太府卿。



 寶祐元年,兼國史編修、實錄檢討兼權兵部侍郎,試司農卿兼中書門下省檢正諸房公事。疏奏:「願詔大臣相與憂亂而思治,懼危而圖安,哀恫警省,修德行政,摧抑群陰之氣焰,保護微陽之根本。批札畢杜於私蹊,官賞宏闢於正路。使內治明如天日,外治勁如風霆。則精神運動,陽匯昭蘇,世道昌明,物情熙洽。上以迓續天命於譴告之餘,下以固結人心於解紐之際。其孰能御之。」以右文殿修撰
 提舉太平興國宮。五年,京湖宣撫大使趙葵闢為判官。



 開慶元年,召赴行在,授集英殿修撰、樞密都承旨、權吏部侍郎。景定元年,兼同修國史、實錄院同修撰兼侍讀,為真侍郎兼太子左庶子。極言正論,太子聽而說之,帝聞之甚喜。二年,遷禮部尚書,權吏部尚書,加龍圖閣學士、知平江府、淮浙發運使。五年,召赴行在,進端明殿學士,提舉祐神觀兼侍讀。召赴行在。



 咸淳元年二月,拜簽書樞密院事;閏月,同知樞密院事兼權參知政事。二年,
 以疾乞祠,不許。乞放歸田里。帝遣尚醫視之,且賜食,復兩乞歸,皆不許。二年,拜參知政事。三年,知樞密院事兼參知政事。立皇太子,加食邑,三辭免官,不許。乞奉祠、休假,皆不許。最後乞祠祿,乃授資政殿學士知慶元府兼沿海制置使。四辭免,不許。七年,臺州言:「乞差爚充上蔡書院山主,」詔從之。八年,加觀文殿學士提舉萬壽宮兼侍讀,詔遣刑部郎官董樸起之,四上疏辭免,始從之。十年,乞致仕,不許。十一月,以爚為左丞相,章鑒為右丞相,
 並兼樞密使。尋授爚特進,加食邑。乞致仕,兩乞辭免,皆不許。



 德祐元年,兩乞改命經筵庶可優閑,再乞以舊職奉京祠侍讀,皆不許。右丞相章鑒、參知政事陳宜中奏「諭留爚以鎮人心,以康世道」。從之。爚兩請毋暑省院公櫝,不許;又奏:「乞將臣先賜罷斥,臣本志誓死報國,願假臣以宣撫招討等職,臣當招募忠義,共圖興復。」鑒、宜中又奏「爚單車絕江,已至蕭山,乞遣中使趣還治事」。乃授觀文殿大學士、浙西江東路宣撫招討大使,置司在京,
 以備咨訪。乞解大使職名,不許。進少保、左丞相兼樞密使,尋加都督諸路軍馬。累辭,皆不許。



 奏言:「今天下所以大壞至此者,正以一私蟠塞,賞罰無章故也。救之之策,在反其所以壞之之由。大明賞罰,動合乎天,庶幾人心興起,天下事尚可為也。」因言賈似道誤國喪師之罪,於是始降詔切責似道不忠不孝。六月庚子朔,日食,爚奏:「日食不盡僅一分,白晝晦冥者數刻。陰盛陽微,災異未有大於此者。臣待罪首相,上佐天子理陰陽,下遂萬物,
 外鎮諸侯,皆其職也。氛祲充塞而未能消,生民塗炭而未能拯,反復思之,咎實在臣,乞罷黜以答天譴。」答詔不許,第降授金紫光祿大夫而已。辭降官,乞罷斥,又不許。



 尋進平章軍國重事,辭,不許。或請:「出宜中或夢炎出督吳門,否則臣雖老無能為,若效死封疆,亦不敢辭。」詔三省集議。乞罷平章事,不許。京學生上書詆宜中,宜中亦上疏乞骸骨。初,宜中在相位,政事多不關白爚,或謂京學之論,實爚嗾之。



 七月壬辰,詔:「給、舍之奏三入,爚與宜
 中必難共處,兼爚近奏乞免平章侍經筵,辭氣不平,誠有如人言者矣。」遂罷爚平章,依前少保、特授觀文殿大學士充醴泉觀使。爚為人清修剛勁,似道歸天臺葬母,過新昌,爚獨不見之。後以元老入相位,值國勢危亡之際,天下所屬望也,而卒與宜中不協而去云。



 章鑒,字公秉,分寧人。以別院省試及第,累官中書舍人、侍左郎官、崇政殿說書,進簽書樞密院事兼權參知政事,遷同知樞密院事。咸淳十年,王爚拜左丞相,鑒拜右
 丞相,並兼樞密使。明年,大元兵逼臨安,鑒託故徑去。遣使亟召還朝,既至,罷相予祠。殿帥韓震之死,鑒與曾淵子明震無他。至是,御史王應麟繳其錄黃,謂震有逆謀,鑒與淵子曲芘之。坐是削一官,放歸田里。



 後有告鑒家匿寶璽者,霜晨,鑒方擁敗衾臥,兵士至,大索其室,惟敝篋貯一玉杯,餘無一物,人頗嘆其清約。鑒在朝日,號寬厚,然與人多許可,士大夫目為「滿朝歡」云。



 陳宜中,字與權,永嘉人也。少甚貧,而性特俊拔。有賈人
 推其生時,以為當大貴,以女妻之。既入太學,有文譽。寶祐中,丁大全以戚里婢婿事權幸盧允升、董宋臣,困得寵於理宗,擢為殿中侍御史,在臺橫甚。宜中與黃鏞、劉黻、林測祖、陳宗、曾唯六人上書攻之。大全怒,使監察御史吳衍劾宜中,削其籍,拘管他州。司業率十二齋生,冠帶送之橋門之外,大全益怒,立碑學中,戒諸生亡妄議國政,且令自後有上書者,前廊生看詳以牒報檢院。由是,士論翕然稱之,號為「六君子」。宜中謫建昌軍。



 大全既
 竄,丞相吳潛奏還之。賈似道入相,復為之請,有詔六人皆免省試令赴。景定三年,廷試,而宜中中第二人。六人之中,宜中尤達時務。由紹興府推官、戶部架閣、秘書省正字、校書郎,數年遷監察御史。



 程元鳳再相,似道恐其侵權,欲去之。宜中首劾元鳳縱丁大全肆惡,基宗社之禍。命格,除太府卿。宜中亦自請外,為江東提舉茶鹽常平公事。四年,改浙西提刑。五年,召為崇政殿說書,累遷禮部侍郎兼中書舍人。七年,閩闕帥,以敷文閣待制、知
 福州。在官得民心,歲餘入為刑部尚書。十年,拜簽書樞密院事兼權參知政事。



 德祐元年,升同知樞密院事。二月,似道喪師蕪湖,乃以宜中知樞密院兼參知政事。已而翁應龍自軍中歸,宜中問似道所在,應龍以不知對。宜中以為似道已死,即上疏乞正似道誤國之罪。似道行時,以所親信韓震總禁兵,人有言震欲以兵劫遷者,宜中召震計事,伏壯士袖鐵椎擊殺之,以示不黨於似道。



 時右丞相章鑒宵遁,曾淵子等請命宜中攝丞相事。
 詔以王爚為左丞相,拜宜中特進、右丞相。四月,爚還朝論事,即與宜中不合。臺臣孫嶸叟請竄籍潛說友、吳益、李玨,宜中以為「簿錄非盛世事,祖宗忠厚,未嘗輕用之。玨方召入朝,遽加重刑,恐後無以示信」。爚力爭,以為當如嶸叟議。會留夢炎自湖南入朝爚與宜中俱乞罷政,請以夢炎為相。太皇太后乃以宜中為左丞相,夢炎為右丞相,爚進平章軍國重事。爚拜命,即日僦民居,以丞相府讓宜中,宜中上疏,以為「一辭一受,何以解天下之
 譏」,亦去。遣使數輩遮留之,始至。



 時命張世傑等四道進師,二丞相都督軍馬而不出督。爚請以一丞相建閫吳門,以護諸將;不然,則已請行。宜中愧,始與夢炎上疏乞行邊。事下公卿議不決。七月,世傑等兵果敗於焦山。爚奏言:「事無重於兵,今二相並建都督,廟算指授,臣不得而知。比者,六月出師,諸將無統。臣豈不知吳門距京不遠,而必為此請者,蓋大敵在境,非陛下自將則大臣開督。今世傑以諸將心力不一而敗,不知國家尚堪幾敗
 邪?臣既不得其職,又不得其言,乞罷免。」不允。



 爚子囗乃嗾京學生伏闕上書,數宜中過失數十事,其略以為:「趙溍、趙與鑒皆棄城遁,宜中乃借使過之說,以報私恩。令狐概、潛說友皆以城降,乃受其包苴而為之羽翼。文天詳率兵勤王,信讒而沮撓之。似道喪師誤國,陽請致罰而陰祐之。大兵薄國門,勤王之師乃留之京城而不遣。宰相當出督,而畏縮猶豫,第令集議而不行。呂師夔狼子野心,而使之通好乞盟。張世傑步兵而用之於水,劉
 師勇水兵而用之於步,指授失宜,因以敗事。臣恐誤國將不止於一似道也。」



 臨S臨安府捕逮京學生。召之亦不至。太皇太后自為書遺其母楊,使勉諭之,宜中始乞以祠官入侍,乃拜醴泉觀使。十月壬寅,始造朝,尋為右丞相,然事已去矣。宜中倉皇發京城民為兵,民年十五以上者皆籍之,人皆以為笑。十一月,遣張全合尹玉、麻士龍兵援常州,玉與士龍皆戰死,全不發一矢,奔還。文天祥請誅全,宜
 中釋不問。已而,常州破,兵薄獨松關,鄰邑望風皆遁。



 宜中遣使如軍中請和不得,即率群臣入宮請遷都,太皇太后不可。宜中痛哭請之,太皇太后乃命裝俟升車,給百官路費銀。及暮,宜中不入,太皇太后怒曰:「吾初不欲遷,而大臣數以為請,顧欺我邪?」脫簪珥擲之地,遂閉閣,群臣求內引,皆不納。蓋宜中實以明日遷,倉卒奏陳失審耳。



 宜中初與大元丞相伯顏期會軍中,既而悔之,不果往。伯顏將兵至皋亭山,宜中宵遁,陸秀夫奉二王入
 溫州,遣人召宜中。宜中至溫州,而其母死。張世傑舁其棺舟中,遂與俱入閩中。益王立,復以為左丞相。井澳之敗,宜中欲奉王走占城,乃先如占城諭意,度事不可為,遂不反。二王累使召之,終不至。至元十九年,大軍伐占城,宜中走暹,後沒於暹。



 宜中為人多術數,少為縣學生,其父為吏受贓當黥,宜中上書溫守魏克愚請貸之。克愚以為黠吏,卒置之法。其後宜中為浙西提刑,克愚郊迎,宜中報禮不書銜,亦云「部下民陳某』,克愚皇恐不敢
 受,袖而謝之。宜中陽禮之,而陰摭其過,無所得。其後,克愚發賈德生冒借官木事,忤似道,廢罷家居。宜中入,乃極言克愚居鄉不法事,似道令章鑒劾之,貶嚴州。克愚之死,宜中擠之為多。



 論曰:「孔子曰:「才難,不其然乎?」理宗在位長久,命相實多其人,若吳潛之忠亮剛直,財數人焉。潛論事雖近於訐,度宗之立,謀議及之,潛以正對,人臣懷顧望為子孫地者能為斯言哉?程元鳳謹飭而有餘而乏風節,尚為賈似
 道所諅。江萬里問學德望優於諸臣,不免為似道籠絡,晚年微露鋒,穎輒見擯斥。士大夫不幸與權奸同朝,自處難矣。似道督視江上之師,以國事付王爚、章鑒、陳宜中,蓋取其平時素與己者。爚、宜中於其既出,稍欲自異,及聞其敗,乘勢蹙之。既而,二人自為矛盾,宋事至此,危急存亡之秋也。當國者交歡戮力,猶懼不逮,所為若是,何望其能匡濟乎。似道誅,爚死,鑒遁,宜中走海島,宋亡。



 文天祥,字宋瑞,又字履善,吉之吉水人也。體貌豐偉,美
 皙如玉,秀眉而長目,顧盼燁然。自為童子時,見學宮所祠鄉先生歐陽修、楊邦乂、胡銓像,皆謚「忠」,即欣然慕之。曰:「沒不俎豆其間,非夫也。」年二十舉進士,對策集英殿。時理宗在位久,政理浸怠,天祥以法天不息為對,其言萬餘,不為稿,一揮而成。帝親拔為第一。考官王應麟奏曰:「是卷古誼若龜鑒,忠肝如鐵石,臣敢為得人賀。」尋丁父憂,歸。



 開慶初,大元兵伐宋,宦官董宋臣說上遷都,人莫敢議其非者。天祥時入為寧海軍節度判官,上書「乞
 斬宋臣,以一人心」。不報,即自免歸。後稍遷至刑部郎官。宋臣復入為都知,天祥又上書極言其罪,亦不報。出守瑞州,改江西提刑,遷尚書左司郎官,累為臺臣論罷。除軍器監兼權直學士院。賈似道稱病,乞致仕,以要君,有詔不允。天祥當制,語皆諷似道。時內制相承皆呈稿,天祥不呈稿,似道不樂,使臺臣張志立劾罷之。天祥既數斥,援錢若水例致仕,時年三十七。



 咸淳九年,起為湖南提刑,因見故相江萬里。萬里素奇天祥志節,語及國事,
 愀然曰:「吾老矣,觀天時人事當有變,吾閱人多矣,世道之責,其在君乎?君其勉之。」十年,改知贛州。



 德祐初,江上報急,詔天下勤王。天祥捧詔涕泣,使陳繼周發郡中豪傑,並結溪峒蠻,使方興召吉州兵,諸豪傑皆應,有眾萬人。事聞,以江西提刑安撫使召入衛。其友止之,曰:「今大兵三道鼓行,破郊畿,薄內地,君以烏合萬餘赴之,是何異驅群羊而搏猛虎。」天祥曰:「吾亦知其然也。第國家養育臣庶三百餘年,一旦有急,徵天下兵,無一人一騎入
 關者,吾深恨於此,故不自量力,而以身徇之,庶天下忠臣義士將有聞風而起者。義勝者謀立,人眾者功濟,如此則社稷猶可保也。」



 天祥性豪華,平生自奉甚厚,聲伎滿前。至是,痛自貶損,盡以家貲為軍費。每與賓佐語及時事,輒流涕,撫幾言曰:「樂人之樂者憂人之憂,食人之食者死人之事。」八月,天祥提兵至臨安,除知平江府。時以丞相宜中未還朝,不遣。十月,宜中至,始遣之。朝議方擢呂師孟為兵部尚書,封呂文德和義郡王,欲賴以求
 好。師孟益偃蹇自肆。



 天祥陛辭,上疏言:「朝廷姑息牽制之意多,奮發剛斷之義少,乞斬師孟釁鼓,以作將士之氣。」且言:「宋懲五季之亂,削藩鎮,建郡邑,一時雖足以矯尾大之弊,然國亦以浸弱。故敵至一州則破一州,至一縣則破一縣,中原陸沈,痛悔何及。今宜分天下為四鎮,建都督統御於其中。以廣西益湖南而建閫於長沙;以廣東益江西而建閫於隆興;以福建益江東而建閫於番陽;以淮西益淮東而建閫於揚州。責長沙取鄂,隆興
 取蘄、黃,番陽取江東,揚州取兩淮,使其地大力眾,足以抗敵。約日齊奮,有進無退,日夜以圖之,彼備多力分,疲於奔命,而吾民之豪傑者又伺間出於其中,如此則敵不難卻也。」時議以天祥論闊遠,書奏不報。



 十月,天祥入平江,大元兵已發金陵入常州矣。天祥遣其將朱華、尹玉、麻士龍與張全援常,至虞橋,士龍戰死,朱華以廣軍戰五牧,敗績,玉軍亦敗,爭渡水,挽全軍舟,全軍斷其指,皆溺死,玉以殘兵五百人夜戰,比旦皆沒。全不發一矢,
 走歸。大元兵破常州,入獨松關。宜中、夢炎召天祥,棄平江,守餘杭。



 明年正月,除知臨安府。未幾,宋降,宜中、世傑皆去。仍除天祥樞密使。尋除右丞相兼樞密使,使如軍中請和,與大元丞相伯顏抗論皋亭山。丞相怒拘之,偕左丞相吳堅、右丞相賈餘慶、知樞密院事謝堂、簽書樞密院事家鉉翁、同簽書樞密院事劉岊,北至鎮江。天祥與其客杜滸十二人,夜亡入真州。苗再成出迎,喜且泣曰:「兩淮兵足以興復,特二閫小隙,不能合從耳。」天祥問:「計
 將安出?」再成曰:「今先約淮西兵趨建康,彼必悉力以捍吾西兵。指揮東諸將,以通、泰兵攻灣頭,以高郵、寶應、淮安兵攻楊子橋,以揚兵攻瓜步,吾以舟師直搗鎮江,同日大舉。灣頭、楊子橋皆沿江脆兵,且日夜望我師之至,攻之即下。合攻瓜步之三面,吾自江中一面薄之,雖有智者不能為之謀矣。瓜步既舉,以東兵入京口,西兵入金陵,要浙歸路,其大帥可坐致也。」天祥大稱善,即以書遺二制置,遣使四出約結。



 天祥未至時,揚有脫歸兵言:「
 密遣一丞相入真州說降矣。」庭芝信之,以為天祥來說降也。使再成亟殺之。再成不忍,紿天祥出相城壘,以制司文示之,閉之門外。久之,復遣二路分覘天祥,果說降者即殺之。二路分與天祥語,見其忠義,亦不忍殺,以兵二十人道之揚,四鼓抵城下,聞候門者談,制置司下令備文丞相甚急,眾相顧吐舌,乃東入海道,遇兵,伏環堵中得免。然亦饑莫能起,從樵者乞得餘糝羹。行入板橋,兵又至,眾走伏叢筱中,兵入索之,執杜滸、金應而去。虞
 候張慶矢中目,身被二創,天祥偶不見獲。滸、應解所懷金與卒,獲免,募二樵者以蕢荷天祥至高郵,泛海至溫州。



 聞益王未立,乃上表勸進,以觀文殿學士、侍讀召至福,拜右丞相。尋與宜中等議不合。七月,乃以同都督出江西,遂行,收兵入汀州。十月,遣參謀趙時賞、諮議趙孟溁將一軍取寧都,參贊吳浚將一軍取雩都,劉洙、蕭明哲、陳子敬皆自江西起兵來會。鄒洬以招諭副使聚兵寧都,大元兵攻之,洬兵敗,同起事者劉欽、鞠華叔、顏斯
 立、顏起巖皆死。武岡教授羅開禮,起兵復永豐縣,已而兵敗被執,死於獄。天祥聞開禮死,制服哭之哀。



 至元十四年正月,大元兵入汀州,天祥遂移漳州,乞入衛。時賞、孟溁亦提兵歸,獨浚兵不至。未幾,浚降,來說天祥。天祥縛浚,縊殺之。四月,入梅州,都統王福、錢漢英跋扈,斬以徇。五月,出江西,入會昌。六月,入興國縣。七月,遣參謀張汴、監軍趙時賞、趙孟溁等盛兵薄贛城,鄒洬以贛諸縣兵搗永豐,其副黎貴達以吉諸縣兵攻泰和。吉八縣復
 其半,惟贛不下。臨洪諸郡,皆送款。潭趙璠、張虎、張唐、熊桂、劉斗元、吳希奭、陳子全、王夢應起兵邵、永間,復數縣,撫州何時等皆起兵應天祥。分寧、武寧、建昌三縣豪傑,皆遣人如軍中受約束。



 江西宣慰使李恆遣兵援贛州,而自將兵攻天祥於興國。天祥不意恆兵猝至,乃引兵走,即鄒洬於永豐。洬兵先潰,恆窮追天祥方石嶺。鞏信拒戰,箭被體,死之。至空坑,軍士皆潰,天祥妻妾子女皆見執。時賞坐肩輿,後兵問謂誰,時賞曰「我姓文」,眾以為
 天祥,禽之而歸,天祥以此得逸去。



 孫
 
  
   
  
 
 、彭震龍、張汴死於兵,繆朝宗自縊死。吳文炳、林棟、劉洙皆被執歸隆興。時賞奮罵不屈,有系累至者,輒麾去,云:「小小簽廳官耳,執此何為?」由是得脫者甚眾。臨刑,洙頗自辯,時賞𠮟曰:「死耳,何必然?」於是棟、文炳、蕭敬夫、蕭燾夫皆不免。



 天祥收殘兵奔循州,駐南嶺。黎貴達潛謀降,執而殺之。至元十五年三月,進屯麗江浦。六月,入船澳。益王殂,衛王繼立。天祥上表自劾,乞入朝,不許。八月,加天祥少保、信國
 公。軍中疫且起,兵士死者數百人。天祥惟一子,與其母皆死。十一月,進屯潮陽縣。潮州盜陳懿、劉興數叛附,為潮人害。天祥攻走懿,執興誅之。十二月,趨南嶺,鄒洬、劉子俊又自江西起兵來,再攻懿黨,懿乃潛道元帥張弘範兵濟潮陽。天祥方飯五坡嶺,張弘範兵突至,眾不及戰,皆頓首伏草莽。天祥倉皇出走,千戶王惟義前執之。天祥吞腦子,不死。鄒洬自頸,眾扶入南嶺死。官屬士卒得脫空坑者,至是劉子俊、陳龍復、蕭明哲、蕭資皆死,杜
 滸被執,以憂死。惟趙孟溁遁,張唐、熊桂、吳希奭、陳子全兵敗被獲,俱死焉。唐,廣漢張栻後也。



 天祥至潮陽,見弘範,左右命之拜,不拜,弘範遂以客禮見之,與俱入厓山,使為書招張世傑。天祥曰:「吾不能捍父母,乃教人叛父母,可乎?」索之固,乃書所過《零丁洋詩》與之。其末有云:「人生自古誰無死,留取丹心照汗青。」弘範笑而置之。厓山破,軍中置酒大會,弘範曰:「國亡,丞相忠孝盡矣,能改心以事宋者事皇上,將不失為宰相也。」天祥泫然出涕,曰:「
 國亡不能救,為人臣者死有餘罪,況敢逃其死而二其心乎。」弘範義之,遣使護送天祥至京師。



 天祥在道,不食八日,不死,即復食。至燕,館人供張甚盛,天祥不寢處,坐達旦。遂移兵馬司,設卒以守之。時世祖皇帝多求才南官,王積翁言:「南人無如天祥者。」遂遣積翁諭旨,天祥曰:「國亡,吾分一死矣。儻緣寬假,得以黃冠歸故鄉,他日以方外備顧問,可也。若遽官之,非直亡國之大夫不可與圖存,舉其平生而盡棄之,將焉用我?」積翁欲合宋官謝
 昌元等十人請釋天祥為道士,留夢炎不可,曰「天祥出,復號召江南,置吾十人於何地!」事遂已。天祥在燕凡三年,上知天祥終不屈也,與宰相議釋之,有以天祥起兵江西事為言者,不果釋。



 至元十九年,有閩僧言土星犯帝坐,疑有變。未幾,中山有狂人自稱「宋主」,有兵千人,欲取文丞相。京城亦有匿名書,言某日燒蓑城葦,率兩翼兵為亂,丞相可無憂者。時盜新殺左丞相阿合馬,命撤城葦,遷瀛國公及宋宗室開平,疑丞相者天祥也。召入
 諭之曰:「汝何願?」天祥對曰:「天祥受宋恩,為宰相,安事二姓?願賜之一死足矣。」然猶不忍,遽麾之退。言者力贊從天祥之請,從之。俄有詔使止之,天祥死矣。天祥臨刑殊從容,謂吏卒曰:「吾事畢矣。」南鄉拜而死。數日,其妻歐陽氏收其尸,面如生,年四十七。其衣帶中有贊曰:「孔曰成仁,孟曰取義,惟其義盡,所以仁至。讀聖賢書,所學何事,而今而後,庶幾無愧。」



 論曰:自古志士,欲信大義於天下者,不以成敗利鈍動
 其心,君子命之曰「仁」,以其合天理之正,即人心之安爾。商之衰,周有代德,盟津之師不期而會者八百國。伯夷、叔齊以兩男子欲扣馬而止之,三尺童子知其不可。他日,孔子賢之,則曰:「求仁而得仁。」宋至德祐亡矣,文天祥往來兵間,初欲以口舌存之,事既無成,奉兩孱王崎嶇嶺海,以圖興復,兵敗身執。我世祖皇帝以天地有容之量,既壯其節,又惜其才,留之數年,如虎兕在柙,百計馴之,終不可得。觀其從容伏質,就死如歸,是其所欲有甚
 於生者,可不謂之「仁」哉。宋三百餘年,取士之科,莫盛於進士,進士莫盛於倫魁。自天祥死,世之好為高論者,謂科目不足以得偉人,豈其然乎!



\end{pinyinscope}