\article{志第一百四十五 兵六(鄉兵三)}

\begin{pinyinscope}

 保甲建炎後鄉兵建炎後砦兵



 保甲熙寧初,王安石變募兵而行保甲,帝從其議。三年,
 始聯比其民以相保任。及詔畿內之民十家為一
 保,選主戶有幹力者一人為保長。五十家為一大保,選一人為大保長。十大保為一都保,選為眾所服者為都保正,又以一人為之副。應主客戶兩丁以上,選
 一
 人為保丁。附保。兩丁以上有餘丁而壯勇者亦附之。內家貲最厚、材勇過人者亦充保丁,兵器非禁者聽習。每一大保夜輪五人警盜。凡告捕所獲,以賞格從事。同保犯強盜、殺人、放火、強奸、略人、傳習妖教、造畜蠱毒,知而不告,依律伍保法。餘事非干己,又非敕律所聽糾,皆毋得告,雖知
 情亦不坐。若於法鄰保合坐罪者乃坐之。其居停強盜三人,經三日,保鄰雖不知情,科失覺罪。逃移、死絕、同保不及五家,並他保。有自外入保者,收為同保,戶數足則附之,俟及十家,則別為保,置牌以書其戶數姓名。既行之畿甸,遂推之五路,以達於天下。時則以捕盜賊相保任,而未肄以武事也。



 四年,始詔畿內保丁肄習武事。歲農隙,所隸官期日於要便鄉村都試騎步射,並以射中親疏遠近為等。騎射校其用馬,有餘藝而願試者聽。第
 一等保明以聞,天子親閱試之,命以官使。第二等免當年春夫一月,馬蒿四十,役錢二千。本戶無可免,或所免不及,聽移免他戶而受其直。第三、第四等視此有差。藝未精願候閱試,或附甲單丁願就閱試者,並聽。都副保正武藝雖不及等,而能整齊保戶無擾,勸誘丁壯習藝及等,捕盜比他保最多,弭盜比他保最少,所隸官以聞,其恩視第一等焉。都副保正有闕,選大保長充。都副保正雖勸誘丁壯習藝,而輒強率妨務者,禁之。吏因保甲
 事受賕、斂掠,加乞取監臨三等,仗、徒、編管、配隸,告者次第賞之,命官犯者除名。時雖使之習武技而未番上也。



 五年,右正言、知制誥、判司農寺曾布言:「近日保戶數以狀詣縣,願分番隸巡檢司習武技,提點司以聞朝廷及司農寺,未敢輒議,願下提點司送中書詳審,付司農具為令。」於是詔:「主戶保丁願上番於巡檢司,十日一更,疾故者次番代之。月給口糧、薪菜錢,分番巡警,每五十人輪大保長二、都副保正一統領之。都副保正月各給錢
 七千,大保長三千。當番者毋得輒離本所。捕逐劇盜,雖下番人亦聽追集,給其錢斛,事訖遣還,毋過上番人數,仍折除其上番日。巡檢司量留廂界給使,餘兵悉罷。應番保丁武技及第三等已上,並記於籍。遇歲兇,五分已上者第振之,自十五石至三石為差。」十一月,又詔尉司上番保丁如巡檢司法。



 六年,詔開封府畿以都保置木契,左留司農寺,右付其縣,凡追胥、閱試、肄習則出契。是月,又詔行於永興、秦鳳、河北東西、河東五路,唯毋上番。
 餘路止相保任,毋習武藝,內荊湖、川、廣並邊者可肄武事,令監司度之。後惟全、邵土丁、邕、欽洞丁、廣東槍手改為保甲者則肄焉。十二月,乃罷河北西路強壯、緣邊弓箭社系籍番上巡守者。



 初,開封府畿、五路保甲及五萬人,二年一解發,詣京師閱試命官,開封府畿十人,五路七人。八年,詔開封府畿及一萬人、五路及一萬五千人,各許解發一人。



 九年,樞密院請自今都副保正、義勇軍校二年一比選,縣考其訓習武藝及等最多、捕察而盜
 賊最少者上於州,州上所轄官司,同比較以聞。或中選人多,則擇武藝最優者。額外尚有可解發者,則第其次為之旌勸。第一次,州縣籍記姓名,犯杖以下聽贖;第二次,以等第賜杖子、紫衫、銀帶,犯徒罪情輕奏裁;累及三次者,降宣補之,給馬及芻菽。五路義勇軍校二千,解發毋得過三人。保甲都副保正之解發者亦以二年,府界六人,河北、河東各四人,永興、秦鳳等路七人。都保正、指揮使與下班殿侍,副保正、副指揮使與三司軍將,正副
 都頭與守闕軍將,並賜衣及銀帶、銀裹頭杖,給馬有差。



 初,保甲隸司農,熙寧八年,改隸兵部,增同判一、主簿二、乾當公事官十,分按諸州,其政令則聽於樞密院。十年,樞密院副都承旨張誠一上《五路義勇保甲敕》。元豐元年,翰林學士、權判尚書兵部許將修《開封府界保甲敕》,成書上之,詔皆頒焉。



 二年十一月,始立《府界集教大保長法》,以昭宣使入內內侍省副都知王中正、東上閣門使狄諮兼提舉府界教保甲大保長,總二十二縣為教
 場十一所,大保長凡二千八百二十五人,每十人一色事藝,置教頭一。凡禁軍教頭二百七十,都教頭三十,使臣十。弓以八斗、九斗、一石為三等,弩以二石四斗、二石七斗、三石為三等,馬射九斗、八斗為二等,其材力超拔者為出等。當教時,月給錢三千,日給食,官予戎械、戰袍,又具銀楪、酒醪以為賞犒。



 三年,大保長藝成,乃立團教法,以大保長為教頭,教保丁焉。凡一都保相近者分為五團,即本團都副保正所居空地聚教之。以大保長藝
 成者十人袞教,五日一周之。五分其丁,以其一為騎,二為弓,三為弩。府界法成,乃推之三路,各置文武官一人提舉,河北則狄諮、劉定,陜西則張山甫,河東則黃廉、王崇拯,以封樁養贍義勇保甲錢糧給其費。是歲,引府界保甲武藝成,帝親閱,錄作能者,余賜金帛。



 四年,改五路義勇為保甲。狄諮、劉定部領澶州集教大保長四百八十二人見於崇政殿,召執政賜坐閱試,補三班借職、差使、借差凡三十六人,余賜金帛有差。遷諮四方館使,定
 集賢校理。又詔曰:「三路見訓民兵非久,什長藝成,須便行府界團教之,錢糧、官吏並如畿縣,未知及期能辦與不。若更稽延日月,必致有誤措置大法,可令承旨取索會校之。」其年,府界、河北、河東、陜西路會校保甲,都保凡三千二百六十六,正長、壯丁凡六十九萬一千九百四十五,歲省舊費緡錢一百六十六萬一千四百八十三,歲費緡錢三十一萬三千一百六十六,而團教之賞為錢一百萬有奇不與焉。凡集教、團教成,歲遣使則謂之
 提舉按閱,率以近臣挾內侍往給賞錢,按格令從事。諸路皆以番次藝成者為序,率五六歲一遍,獨河東以金帛不足,乃至十一歲。上以晉人勇悍,介遼、夏間,講勸宜不可後,詔賜緡錢十五萬。時系籍義勇、保甲及民兵凡七百一十八萬二千二十八人云。



 熙寧九年之數。



 保甲立法之初,故老大臣皆以為不便,而安石主議甚力,帝卒從之。今悉著其論難,使來者考焉。



 帝嘗論租庸調法而善之,安石對曰:「此法近井田,後世立事粗得先王遺意,則無
 不善。今亦無不可為。顧難速成爾。」及帝再問,則曰:「人主誠能知天下利害,以其所謂害者制法,而加於兼並之人,則人自不敢保過限之田;以其所謂利者制法,而加於力耕之人,則人自勸於力耕,而授田不能過限。然此須漸乃能成法。使人主誠知利害之權,因以好惡加之,則所好何患人之不從,所惡何患人之不避?若人主無道以揆之,則多為異議所奪,雖有善法,何由立哉?」



 帝謂府兵與租庸調法相須,安石則曰:「今義勇、土軍上番供
 役,既有廩給,則無貧富皆可以入衛出戍,雖無租庸調法,亦自可為。第義勇皆良民,當以禮義獎養。今皆倒置者,以涅其手背也,教閱而縻費也,使之運糧也。三者皆人所不樂,若更毆之就敵,使被殺戮,尤人所憚也。」



 馮京曰:「義勇亦有以挽強得試推恩者。」安石曰:「挽強而力有不足,則絕於進取,是朝廷有推恩之濫。初非勸獎使人趨武用也。今欲措置義勇皆當反此,使害在於不為義勇,而利在於為義勇,則俗可變而眾技可成。臣願擇鄉
 閭豪傑以為將校,稍加獎拔,則人自悅服。矧今募兵為宿衛,及有積官至刺吏以上者。移此與彼,固無不可,況不至如此費官祿,已足使人樂為哉!陛下誠能審擇,近臣皆有政事之材,則異時可使分將此等軍矣。今募兵出於無賴之人,尚可為軍、廂主,則近臣以上豈不及此輩,此乃先王成法,社稷之長計也。」帝以為然。



 時有欲以義勇代正兵者,曾公亮以為置義勇、弓手,漸可以省正兵。安石曰:「誠然,第今江、淮置新弓手,適足以傷農。」富弼
 亦論京西弓手非便。安石曰:「揆文教,奮武衛,先王所以待遠邇者固不同。今處置江、淮與三邊,事當有異。」



 帝又言節財用,安石對以減兵最急。帝曰:「比慶歷數已甚減矣。」因舉河北、陜西兵數,慮募兵太少,又訓擇不精,緩急或闕事。安石則曰:「精訓練募兵而鼓舞三路之民習兵,則兵可省。臣屢言河北舊為武人割據,內抗朝廷,外敵四鄰,亦有御奚、契丹者,兵儲不外求而足。今河北戶口蕃息,又舉天下財物奉之,常若不足。以當一面之敵,其
 施設乃不如武人割據時。則三路事有當講畫者,在專用其民而已。」帝又言:「邊兵不足以守,徒費衣廩。然固邊圉又不可悉減。」安石曰:「今更減兵,即誠無以待急緩;不減,則費財困國無已時。臣以謂儻不能理兵,稍復古制,則中國無富強之理。」



 帝曰:「唐都長安,府兵多在關中,則為強本。今都關東而府兵盛。則京師反不足待四方。」安石曰:「府兵在處可為,又可令入衛,則不患本不強。」韓絳、呂公弼皆以入衛為難。文彥博曰:「如曹、濮人專為盜賊,
 豈宜使入衛?」安石曰:「曹、濮人豈無應募?皆暴猾無賴之人,尚不以為虞;義勇皆良民,又以物力戶為將校,豈當復以為可虞也?」



 陳升之欲令義勇以漸戍近州。安石曰:「陛下若欲去數百年募兵之敝,則宜果斷,詳立法制,令本末備具。不然,無補也。」帝曰:「制而用之,在法當預立條制,以漸推行。」彥博等又以為土兵難使千里出戍。安石曰:「前代征流求,討黨項,豈非府兵乎?」帝曰:「募兵專於戰守,故可恃;至民兵,則兵農之業相半,可恃以戰守乎?」安
 石曰:「唐以前未有黥兵,然亦可以戰守。臣以謂募兵與民兵無異,顧所用將帥如何爾。將帥非難求,但人主能察見群臣情偽,善駕御之,則人材出而為用,不患無將帥。有將帥,則不患民兵不為用矣。」



 帝曰:「經遠之策,必至什伍其民,費省而兵眾,且與募兵相為用矣。」安石對曰:「欲公私財用不匱,為宗社長久計,募兵水法誠當變革。」帝曰:「密院以為必有建中之變。」安石對曰:「陛下躬行德義,憂勤政事,上下不蔽,必無此理。建中所以致變,德宗
 用盧杞之徒而疏陸贄,其不亡者幸也。」



 時開封鞫保戶有質衣而買弓箭者,帝恐其貧乏,難於出備。安石曰:「民貧宜有之,抑民使置弓箭,則法所弗去也。往者冬閱及巡檢番上,唯就用在官弓矢,不知百姓何故至於質衣也。然自生民以來。兵農為一,耒耜以養生,弓矢以免死,皆凡民所宜自具,未有造耒耜、弓矢以給百姓者也。然則雖使百姓置弓矢,亦不為過。第陛下優恤百姓甚至,故今立法,一聽民便爾。且府界素多群盜,攻劫殺掠,
 一歲之間至二百火,逐火皆有賞錢,備賞之人即今保丁也。方其備賞之時,豈無賣易衣服以納官賞者?然人皆以謂賞錢宜出於百姓。夫出錢之多不足以止盜,而保甲之能止盜,其效已見,則雖令民出少錢以置器械,未有損也。」帝曰:「賞錢人所習慣,則安之如自然;不習慣,則不能無怨。如何決壞民產,民不怨;決河以壞民產,則怨矣。」



 帝嘗批:「陳留縣所行保甲,每十人一小保,中三人或五人須要弓箭,縣吏督責,無者有刑。百姓買一弓至千
 五百,十箭至六七百,當青黃不接之際,貧下客丁安能出辦?又每一小保用民力築射垛,又自辦錢糧起鋪屋。每保置鼓,遇賊聲擊,民居遠近不一,甲家遭賊,鼓在乙家,則無緣聲擊。如此,須人置一鼓,費錢不少。可速指揮令止如元議,團保覺察盜賊,餘無得施行。鄉民既憂無錢買弓箭,加以傳惑徙之戍邊,是以父子聚首號泣者非虛也。」安石進呈不行。



 帝謂安石:「保甲誠有斬指者,此事宜緩而密。」安石曰:「日力可惜。」帝曰:「然亦不可遽,恐卻
 沮事。」安石曰:「此事自不敢不密。」權知開封府韓維等言:「諸縣團結保甲,鄉民驚擾。祥符等縣已畢,其餘縣乞候農閑排定。」時府界諸縣鄉民,或自殘傷以避團結。安石辨說甚力。時曾孝寬為府界提點,榜募告捕扇惑保甲者雖甚嚴,有匿名書封丘郭門者,於是詔重賞捕之。



 安石曰:「乃者保甲,人得其願上番狀,然後使之,宜於人情無所驚疑。且今居藏盜賊及為盜賊之人,固不便新法。陛下觀長社一縣,捕獲府界劇賊為保甲迫逐出外者
 至三十人。此曹既不容京畿,又見捕于輔郡,其計無聊,專務扇惑。比聞為首扇惑者已就捕,然至京師亦止有二十許人。以十七縣十數萬家,而被扇惑者才二十許人,不可謂多。自古作事,未有不以勢率眾而能令上下如一者。今聯十數萬人為保甲,又待其應募乃使之番上,比乃以陛下矜恤之至。令保甲番上捕盜,若任其自去來,即孰肯聽命?若以法驅之,又非人所願。且為天下者,如止欲任民情所願而已,則何必立君而為之張官
 置吏也?今輔郡保甲,宜先遣官諭上旨,後以法推行之。」帝曰:「然。」



 一日,帝謂安石曰:「曾孝寬言,民有斬指訴保甲者。」安石曰:「此事得於蔡駰。趙子幾使駰驗問,乃民因斫木誤斬指,參證者數人。大抵保甲法,上自執政大臣,中則兩制,下則盜賊及停藏之人,皆所不欲。然臣召鄉人問之,皆以為便。則雖有斬指以避丁者,不皆然也。況保甲非特除盜,固可漸習為兵。既人皆能射,又為旗鼓變其耳目,且約以免稅上番代巡檢兵;又自正、長而上,能
 捕賊者獎之以官,則人競勸。然後使與募兵相參,則可以銷募兵驕志,且省財費,此宗社長久之計。」



 帝謂什伍百姓如保甲,恐難成,不如便團結成指揮,以使臣管轄。安石曰:「陛下誠能果斷,不恤人言,即便團結指揮,亦無所妨。然指揮是虛名,五百人為一保,緩急可喚集,雖不名為指揮,與指揮使無異,乃是實事。幸不至大急,即免令人駭擾而事集為上策。」帝遂變三路義勇如府畿保甲法。



 馮京曰:「義勇已有指揮使,指揮使即其鄉里豪傑。
 今復作保甲,令何人為大保長?」安石曰:「古者民居則為鄉,伍家為比,比有長,及用兵,即五人為伍,伍有伍司馬。二十五家為閭,閭有閭胥,二十五人為兩,兩有兩司馬。兩司馬即閭胥,伍司馬即比長,第隨事異名而已。此乃三代六鄉六軍之遺法。其法見於書,自夏以來,至周不改。秦雖決裂阡陌,然什伍之尚如古制,此所以兵眾而強也。征伐唯府兵為近之。今舍已然之成憲,而乃守五代亂亡之餘法,其不足以致安強無疑。然人皆恬然不
 以因循為可憂者,所見淺近也。」



 安石又奏:「義勇須三丁以上,請如府界,兩丁以上盡籍之。三丁即出戍,誘以厚利;而兩丁即止令於巡檢上番,如府界法。大略不過如此。當遣人與經略、轉運司及諸州長吏議之,及訪本路民情所苦所欲,因以寓法。」帝曰:「河東修義勇強壯法,又令團集保甲,如何?」安石對曰:「義勇須隱括丁數,若因團集保甲,即一動而兩業就。今既遣官隱括義勇,又別遣官團結保甲,即分為兩事,恐民不能無擾。」或曰:「保甲不
 可代正軍上番否?」安石曰:「俟其習熟,然後上番。然東兵技藝亦弗能優於義勇、保甲,臣觀廣勇、虎翼兵固然。今為募兵者,大抵皆偷惰頑猾不能自振之人。為農者,皆樸力一心聽令之人,則緩急莫如民兵可用。」馮京曰:「太祖征伐天下,豈用農兵?」安石曰:「太祖時接五代,百姓困極,豪傑多以從軍為利。今百姓安業樂生,而軍中不復有如向時拔起為公侯者,即豪傑不復在軍,而應募者大抵皆偷惰不能自振之人爾。」帝曰:「兵之強弱在人。五
 代兵弱,至世宗而強。」安石曰:「世宗所收,亦皆天下亡命強梁之人。」文彥博曰:「以道佐人主者不以兵強天下。」安石曰:「以兵強天下者非道也,然有道者固能柔能剛,能弱能強。方其能剛強,必不至柔弱。張皇六師,固先王之所尚也,但不當專務兵強爾。」帝卒從安石議。



 帝曰:「保甲、義勇芻糧之費,當預為之計。」安石曰:「當減募兵之費以供之。所供保甲之費,才養兵十之一二。」帝曰:「畿內募兵之數已減於舊。強本之勢,未可悉減。」安石曰:「既有保甲
 代其役,即不須募兵。今京師募兵,逃死停放,一季乃數千,但勿招填,即為可減。然今廂軍既少,禁兵亦不多,臣願早訓練民兵。民兵成,則募兵當減矣。」又為上言:「今河北義勇雖十八萬,然所可獎慰者不過酋豪百數十人而已。此府兵之遺意也。」帝以為然,令議其法。



 樞密院傳上旨,以府界保甲十日一番,慮大促無以精武事,其以一月為一番。安石奏曰:「今保甲十日一番,計一年餘八月當番,若須一月,即番愈疏。又昨與百姓約十日一番,
 今遽改命,恐愈為人扇惑。宜俟其習熟,徐議其更番。且今保甲閱藝八等,勸獎至優,人競私習,不必上番然後就學。臣愚,願以數年,其藝非特勝義勇,當必勝正兵。正兵技藝取應官法而已,非若保甲人人有勸心也。」



 元豐八年,哲宗嗣位,知陳州司馬光上疏乞罷保甲,曰:



 兵出民間,雖云古法,然古者八百家才出甲士三人、步卒七十二人,閑民甚多,三時務農,一時講武,不妨稼穡。自兩司馬以上,皆選賢士大夫為之,無侵漁之患,故卒乘輯
 睦,動則有功。今籍鄉村之民,二丁取一以為保甲,授以弓弩,教之戰陣,是農民半為兵也。三四年來,又令河北、河東、陜西置都教場,無問四時,每五日一教。特置使者比監司,專切提舉,州縣不得關預。每一丁教閱,一丁供送,雖云五日,而保正,長以泥堋除草為名,聚之教場,得賂則縱,否則留之,是三路耕耘收獲稼穡之業幾盡廢也。



 自唐開元以來,民兵法壞,戍守戰攻,盡募長征兵士,民間何嘗習兵?國家承平百有餘年,戴白之老不識兵
 革,一旦畎畝之人皆戎服執兵,奔驅滿野,耆舊嘆息,以為不祥。事既草創,調發無法,比戶騷擾,不遺一家。又巡檢、指使按行鄉村,往來如織;保正、保長,依倚弄權,坐索供給,多責賂遺,小不副意,妄加鞭撻,蠶食行伍,不知紀極。中下之民,罄家所有,侵肌削骨,無以供億,愁苦困弊,靡所投訴,流移四方,襁屬盈路。又朝廷時遣使者,遍行按閱,所至犒設賞賚,糜費金帛,以巨萬計。此皆鞭撻平民銖兩丈尺而斂之,一旦用之如糞土。而鄉村之民,但
 苦勞役,不感恩澤。農民之勞既如披,國家之費又如此,終何所用哉?若使之捕盜賊,衛鄉里,則何必如此之多?使之戍邊境,事征伐,則彼遠方之民,以騎射為業,以攻戰為俗,自幼及長,更無他務。中國之民,大半服田力穡,雖復授以兵械,教之擊刺,在教場之中坐作進退,有似嚴整,必若使之與敵人相遇,填然鼓之,鳴鏑始交,其奔北潰敗可以前料,決無疑也,豈不誤國事乎?又悉罷三路巡檢下兵士及諸縣弓手,皆易以保甲。主簿兼縣尉,
 但主草市以里;其鄉村盜賊,悉委巡檢,而巡檢兼掌巡按保甲教閱,朝夕奔走,猶恐不辦,何暇逐捕盜賊哉?又保甲中往往有自為盜者,亦有乘保馬行劫者。然則設保甲、保馬本以除盜,乃更資盜也。



 自教閱保甲以來,河東、陜西、京西盜賊已多,至敢白晝公行,入縣鎮,殺官吏。官軍追討,經歷歲月,終不能制。況三路未至大饑,而盜賊猖熾已如此,萬一遇數千里之蝗旱,而失業饑寒、武藝成就之人,所在蜂起以應之,其為國家之患,可勝言哉!此非
 小事,不可以忽。夫奪其衣食,使無以為生,是驅民為盜也;使比屋習戰,勸以官賞,是教民為盜也;又撤去捕盜之人,是縱民為盜也。謀國如此,果為利乎?害乎?



 且向者干進之士,說先帝以征伐開拓之策,故立保甲、戶馬、保馬等法。近者登極赦書有云:「應緣邊州軍,仰逐處長吏並巡檢、使臣、鈐轄、兵士及邊上人戶不得侵擾外界,務要靜守疆埸,勿令騷擾。」此蓋聖意欲惠綏殊才,休息生民,中外之人孰不歸戴?然則保甲、戶馬復何所用?或今
 雖罷戶馬,寬保馬,而保甲猶存者,蓋未有以其利害之詳奏聞者也。



 臣愚以為悉罷保甲使歸農,召提舉官還朝,量逐縣戶口,每五十戶置弓手一人,略依緣邊弓箭手法,許蔭本戶田二頃,悉免其稅役。除出賊地分,更不立三限科校,但令捕賊給賞。若獲賊數多及能獲強惡賊人者,各隨功大小遷補職級,或補班行,務在優假弓手,使人勸募。然後募本縣鄉村戶有勇力武藝者投充,計即今保甲中有勇力武藝者必多願應募。若一人缺
 額,有二人以上爭投者,即委本縣令、尉選武藝高強者充。或武藝衰退者,許他人指名與之比較,若武藝勝於舊者,即令沖替,其被替者,更不得蔭田。如此,則不必教閱,武藝自然精熟。一縣之中,其壯勇者既為弓手,其羸弱者雖使為盜,亦不能為患。仍委本州及提點刑獄常按察,令佐有取舍不公者,嚴行典憲。若召募不足,且即於鄉村戶上依舊條權差,候有投名者即令充替。其餘巡檢兵士、縣尉弓手、耆老、壯丁逐捕盜賊,並乞依祖宗
 舊法。



 五月,以光為門下侍郎。光欲復申前說,以為教閱保甲公私勞費而無所用。是時,資政殿學士韓維、侍讀呂公著欲復上前奏,先是進呈,乞罷團教。詔府界、三路保甲自來年正月以後並罷團教,仍依舊每歲農隙赴縣教閱一月,其差官置場,排備軍器,教閱法式番次,按賞費用,令樞密院、三省同立法。後六日,光再上奏,極其懇切,蔡確等執奏不行。詔保甲依樞密院已得指揮,保馬別議立法。



 九月,監察御史王巖叟言:「保甲之害,三路
 之民如在湯火,未必皆法之弊,蓋由提舉一司上下官吏逼之使然。而近日指揮雖令冬教,然尚存官司,則所以為保甲之害者,十分之六七猶在,陛下所不知也。此皆奸邪遂非飾過,而巧辭強辨以欺惑聖聽,將至深之病略示更張,以應副陛下聖意而已,非至誠為國家去大害、復大利,以便百姓,為太平長久之計者也。此忠義之良心所以猶抑,奸邪之素計所以尚存。天下之識者,皆言陛下不絕害源,百姓無由樂生;不屏群邪,太平終
 是難致。臣願陛下奮然獨斷,如聽政之初行數事,則天下之大體無虧,陛下高枕而臥矣。」十月,詔提舉府界、三路保甲官並罷,令逐路提刑及府界提點司兼領所有保甲,止冬教三月。又詔逐縣監教官並罷,委令佐監教。



 十一月,巖叟言:



 保甲行之累年,朝廷固已知人情之所共苦,而前日下詔蠲疾病,汰小弱,釋第五等之田不及二十畝者,省一月之六教而為三月之並教,甚大惠也。然其司尚存,其患終在。今以臣之所見者為陛下言,不
 敢隱其實以欺朝廷,亦不敢飾其事以罔成法。



 夫朝廷知教民以為兵,而不知教之太苛而民不能堪;知別為一司以總之,而不知擾之太煩而民以生怨。教之欲以為用也,而使之至於怨,則恐一日用之,有不能如吾意者,不可不思也。



 民之言曰,教法之難不足以為苦,而羈縻之虐有甚焉;羈縻不足以為苦,而鞭笞之酷有甚焉;鞭笞不足以為苦,而誅求之無已有甚焉。方耕方耘而罷,方幹方營而去,此羈縻之所以為苦也。其教也,保長
 得笞之,保正又笞之,巡檢之指使與巡檢者又交撻之,提舉司之指使與提舉司之乾當公事者又互鞭之,提舉之官長又鞭之,一有逃避,縣令又鞭之。人無聊生,恨不得死,此鞭笞之所以為苦也。創袍、市巾、買弓、絳箭、添弦、換包指、治鞍轡、蓋涼棚、畫像法、造隊牌、緝架、僦椅卓、圍典紙墨、看定人雇直、均菜緡、納秸粒之類,其名百出,不可勝數。故父老之諺曰:「兒曹空手,不可以入教場。」非虛語也。都副兩保正、大小兩保長,平居於家,婚姻喪葬之問
 遺,秋成夏熟,絲麻穀麥之要求,遇於城市,飲食之責望。此迫於勢而不敢不致者也。一不如意,即以藝不如法為名,而捶辱之無所不至。又所謂巡檢、指使者,多由此徒以出,貪而冒法,不顧後禍,有逾於保正、保長者,此誅求之所以為甚苦也。



 又有逐養子、出贅婿、再嫁其母、兄弟析居以求免者,有毒其目、斷其指、炙其肌膚以自殘廢而求免者,有盡室以逃而不歸者,有委老弱於家而保丁自逃者。保丁者逃,則法當督其家出賞錢十千以
 募之。使其家有所出,當未至於逃;至於逃,則其困窮可知,而督取十千,何可以得?故每縣常有數十百家老弱嗟咨於道路,哀訴於公庭。如臣之愚,且知不忍,使陛下仁聖知之,當如何也?



 又保丁之外,平民凡有一馬,皆令借供。逐場教騎,終日馳驟,往往饑羸以至於斃,誰復敢言?其或主家倘因他出,一誤借供,遂有追呼笞責之害。或因官逋督迫,不得已而易之,則有抑令還取之苦,故人人以有馬為禍。此皆提舉官吏倚法以生事,重為百
 姓之擾者也。



 竊惟古者未嘗不教民以戰,而不聞其有此者,因人之情以為法也。夫緣情以推法,則愈久而愈行;倚威以行令,則愈嚴而愈悖。此自然之理也。獸窮則搏,人窮則詐,自古及今,未有窮其下而能無危者也。臣觀保甲一司,上下官吏,無豪發愛百姓意,故百姓視其官司不啻虎狼,積憤銜怨,人人所同。比者保丁執指使,逐巡檢,攻提舉司乾當官,大獄相繼,今猶未已。雖民之愚,顧豈忘父母妻子之愛,而喜為犯上之惡以取禍哉?
 蓋激之至於此極爾!激之至深,安知其發有不甚於此者?情狀如此,不可不先事而慮,以保大體而圖安靜。



 夫三時務農,一時講武,先王之通制也。一月之間並教三日,不若一歲之中並教一月。農事既畢,無他用心,人自安於講武而無憾。遂可罷提舉司,廢巡教官,一以隸州縣,而俾逐路安撫司總之。每俟冬教於城下,一邑分兩番,當一月。起教則與正長論階級,罷教則與正長不相誰何。庶使百姓得以優游治生,無終年遁逃之苦,無侵
 漁苛虐之患,無爭陵犯上之惡矣。且武事不廢,威聲亦全,豈不易而有功哉?惟陛下深計遠慮,斷在必行,以省多事,以為生靈安樂之惠,以為國家安靜之福。



 又乞罷三路提舉保甲錢糧司及罷提舉教閱,及每歲分保甲為兩番,於十一、十二兩月上教,不必分作四番,且不必自京師遣官視教,止令安撫司差那使臣為便。並從之。



 元祐元年正月,樞密院言:「府界、三路保甲已罷團教,其教閱器械悉上送官,仍立禁約。」閏二月,詔河北東西
 路、永興、秦鳳等路提點刑獄兼提舉保甲,並依提刑司例各為一司。三月,王巖叟劾狄諮、劉定奸贓狀。御史孫升亦言:「劉定上挾章惇之奸黨,下附狄諮之庸材,大肆憑陵,公行恐喝,故真定獲鹿之變起於後,澶、滑之盜作於前,願早正其罪。」於是諮、定皆罷,與在外宮觀。十一月,詔府界、三路保甲人戶五等已下、地土不及二十畝者,雖三丁以上,並免教。從殿中侍御史呂陶之請也。



 紹聖二年七月,帝問義勇、保甲數,宰臣章惇曰:「義勇,自祖宗
 以來舊法。治平中,韓琦請遣使詣陜西再括丁數添刺。熙寧中,先帝始行保甲法,府界、三路得七十餘萬丁。設官教閱始於府界,眾議沸騰。教藝既成,更勝正兵。元豐中,始遣使遍教三路。先帝留神按閱,藝精者厚賞,或擢以差使、軍將名目,而一時賞賚率取諸封樁或禁軍闕額,未嘗費戶部一錢。元祐馳廢,深可惜也。」



 元符二年九月,御史中丞安惇奏乞教習保甲月分,差官按試。曾布言:「保甲固當教習,然陜西、河東連年進築城砦,調發未
 已,河北連年水災,流民未復,以此未可督責訓練。」帝曰:「府界豈不可先行?」布曰:「熙寧中教保甲,臣在司農。是時諸縣引見保甲,事藝精熟。」章惇即曰:「多得班行。」布曰:「止是得殿侍、軍將,然俱更差充巡檢司指揮。以此,仕宦及有力之家子弟,皆欣然趨赴。及引對,所乘皆良馬,鞍韉華楚,馬上事藝往往勝諸軍。知縣、巡檢又皆得轉官或減年。以此,上下皆踴躍自效。然是時司農官親任其事,督責檢察極精密,縣令有抑令保甲置衣裝非理騷擾
 者,亦皆沖替,故人莫敢不奉法。其後乃令上番。」帝曰:「且與先是府界檢舉施行。」蔡卞曰:「於先朝法中稍加裁損,無不可之理。」布以為甚便,容檢尋文字進呈。



 十一月,蔡卞勸上復行畿內保甲教閱法,帝屢以督曾布。是日,布進呈畿內保丁總二十六萬,熙寧中教事藝者凡七萬,因言:「此事固當講求,然廢罷已十五年,一旦復行,與事初無異,當以漸行,則人不至於驚擾。」帝曰:「固當以漸行之。」布曰:「聖諭如此,盡之矣。若便以元豐成法一切舉行,
 當時保丁存者無幾,以未教習之人,便令上番及集教,則人情洶洶,未易安也。熙寧中,施行亦有漸。容臣講求施行次第。」退以語卞,卞殊以為不快,乃云:「熙寧初,人未知保甲之法。今耳目已習熟,自不同矣。」布不答。



 徽宗崇寧四年,樞密院言:「比者京畿保甲投八百七十一牒乞免教閱,又二百三十餘牒遮樞密張康國馬首訴焉。」是月,詔京畿、三路保甲並於農隙時教閱,其月教指揮勿行。



 五年,詔河北東西、河東、永興、秦鳳路各武臣一員充
 提舉保甲並兼提刑,其見專提舉保甲文臣並罷。是月,詔京畿差武臣一員充提舉保甲兼提刑,仍差文臣提刑兼提舉保甲。



 政和三年四月,樞密院言:「神考制保甲之法,京畿、三路聚教,每番雖號五十日,其間有能勤習弓弩該賞者首先拍放。一歲之中,在場閱教,遠者不過二十七日,近者止於十八日而已。若秋稼災傷,則免當年聚教。如武藝稍能精熟,則有激賞之法。鬥力出等,則免戶下春夫、科配;最高強者,則解發引見,試藝命官。行
 之累年,人皆樂從。惟京東、西雖有團成保甲之名,未嘗訓以武事,慮其間亦有人材甚眾,能習武藝,可以命官任使之人。今欲依三路保甲編修點擇條約。」從之。八月,樞密院言:「諸路團成保甲者六十一萬餘人,悉皆樂從無擾。其京東、西路提舉官任諒已轉一官,直秘閣。其朝議大夫已上與轉行,武臣武功大夫特與轉遙郡刺史,餘官減磨勘年有差。」



 宣和元年,詔提舉保甲督察州縣都保不如令者,限一月改正,每歲以改正多寡為殿最。
 二年,詔諸路保甲法並遵依元豐舊制,京東、京西路並罷。



 三年,詔:「先帝若稽成周制保伍之法,自五家相比,推而達之,二十五家為一大保,二百五十家為一都保。保各有長,都各有正,正各有副,使之相保相愛,以察奸慝。故有所行,諸自外來者,同保互告,使各相知;行止不明者,聽送所屬。保內盜賊,畫時集捕,知而不糾,又論如律。所以糾禁幾察,纖悉具備,奇邪寇盜,何所容跡?訪聞法行既久,州縣玩習弛廢,保丁開收既不以實,保長役使
 又不以時。如修鼓鋪、飾粉壁、守敗船、治道路、給夫役、催稅賦之類,科率騷擾不一,遂使寇賊奇邪無復糾察,良法美意浸成虛文。可令尚書省於諸路提點刑獄或提舉常平官內,每路選委一員,令專一督責逐縣令佐,將系籍人丁開收取實;選擇保正長,各更替如法,使鈐束保丁,遞相覺察,毋得舍亡賴作過等人,遇有盜賊,畫時追捕,若有過致藏匿者,許諸人告首,仍具條揭示。」



 欽宗靖康元年三月,以尚書戶部侍郎錢蓋為龍圖閣學士、
 陜西五路制置使,專一措置京兆府路保甲。六月,御史胡舜陟奏:「秦元學兵法三十年,陛下拔之下僚,為京畿提刑,訓練保甲,聞者莫不慰悅。乞罷武臣提刑,以保甲屬元,庶得專一。」從之。十一月,京畿提舉秦元集保甲三萬,先請出屯,自當一面。不從。金兵薄城,又乞行訓練,乘間出戰。守禦使劉韐奏取保甲自益,元謀遂塞雲。



 建炎後鄉兵



 巡社建炎元年,詔諸路州軍巡社並以忠義巡社為名,隸宣撫司,後募鄉民為之。每十人為一甲,有甲長,有隊長;四隊為一部,有部長;五部為一社,有社長;五社為一都,有都正。於鄉井便處
 駐扎。紹興初,罷之。



 槍杖手建炎二年,令福建招五千人。



 土豪建炎四年,詔諸州守臣募土豪、民兵,聽州縣守令節制。後存留強壯,餘並放散。



 義兵紹興十年團集,諸州名數不等。後皆以縣令為軍正。



 義士紹興元年,籍興元良家子弟,兩古取一,四丁取二,每二十人為一隊,號曰義士。



 民兵建炎二年,每五十人為一隊,有長、副。一戶取一丁,五丁取二丁。淳熙十四年,三丁取一,五丁取二,十丁取三。



 弓箭手建炎初,應諸路漢蕃弓箭手限百日自陳承襲,紹興間,以京城外閑地,依陜西沿邊例,招弓箭手蒔種。



 土丁紹興中,詔依嘉祐措置,三時務農,一時講武,諸縣逐鄉置教場,自十一月起教,至次年正月罷教。



 把截將紹興二十七年,詔恭州、雁門控扼之地置土丁二百人。



 峒丁建炎三年,命江西、福建諸處總領官籍定槍杖手、峒丁人數,以備調遣。紹興中,罷之。



 保勝紹興六年,詔金、均、房三州保甲分為五軍,
 以保勝為名。



 勇敢紹興二年,詔池州就招土人充,二千為額。



 保丁二廣保丁,每戶一名,土丁父子兄弟皆在其數。乾道中,以拘留擾民,罷之。



 山水砦詳見砦兵。



 萬弩手初,熙寧間,以鼎、澧、辰、沅、靖五郡弓弩手萬三千人散居邊境訓練,無事耕作,有警調發。紹興以後,增損靡定。



 壯丁民社乾道四年,楚州置。



 良家子紹興四年,招兩淮、關陜流寓及陣亡主兵將子弟驍武不能存立者充,月給比強弓手,五十人為一隊。



 義勇湖北諸郡皆有義勇,惟澧州石門、慈利不置籍。其法取於主戶之雙丁。每十戶為一甲,五甲為團。甲皆有長,擇邑豪為總首。農隙教武藝,食從官司給。



 湖北土丁刀弩手政和七年,募土丁充,授以閑山,散居邊境,教以武藝。紹興因之。淳熙中,李燾力言其不便,罷之。



 湖南鄉社舊制,以鄉豪領之,大者統數百家,小者亦二三百家。後言者以其不便,淳熙中,擇其首領,使大者不
 過五十家,小者減半。



 忠勇關外西和、階、成、鳳四州所聚民兵,謂之忠勇。



 鎮淮初,淮南募邊民號鎮淮軍,數至十萬,月給視效勇,惟不黥涅。久之,廩不足,肆劫掠。嘉定初,選汰歸農,僅存八千餘人,以充效用,餘補鎮江大軍。淮西選二萬六千餘充御前定武軍,分為六軍,軍設統制。



 忠義民兵福州諸縣舊有忠義社,屯結邑民,擇豪右為長,量授器甲,盜由是息,人甚賴之。後有司煩擾,失初意。開禧用兵,淮、襄民兵有籍於官者,至用百六十緡以養一兵。後又放令歸業,而無所歸,多散為盜。乃令每郡擇豪酋一人,授以官民鎮之。



 建炎後砦兵。



 兩浙西路



 臨安府十三砦外沙、海內、管界、茶槽、南蕩、東梓、上管、赭山、黃灣、硤石、奉口、許村、下塘。



 安吉州七砦管界、安吉、秀塞、呂小幽嶺、下塘、北豪、皋塘。



 平江府八砦吳江、吳長、許浦、福山、白茅、江灣、楊林、角頭。



 常州五砦管界、小河、馬跡、香蘭、分界。



 江陰軍二砦申港、石牌。



 嚴州五砦威平、港口、鳳林、茶山、管界。



 兩浙東路



 慶元府十砦浙東、結埼、三姑、管界、大嵩、海內、白峰、岱山、鳴鶴、公塘。



 溫州十三砦城下、管界、館頭、青奧、梅奧、鹿西、浦門、南監、東北、三尖、北監、小鹿、大荊。



 臺州六砦管界、亭場、吳都、白塔、松門、臨門。



 處州二砦管界、梓亭。



 江南東路



 南康軍五砦大孤山、水陸、四望山、河湖、左望。



 江南西路



 隆興府七砦都巡、鄔子、松門、港口、定江、杉甫、管界。



 撫州七砦城南、曾田、樂安、鎮馬、旗步、招攜、湖平。



 江州六砦管界、江內、茭石、馬當、城子頭、孤山。



 興國二砦池口、磁湖。



 袁州四砦都巡、四縣、管界、白斜。



 臨江軍三砦本軍、水陸、管界。



 吉州十六砦富田、走馬塍、永和鎮、觀山、明德、沙溪、西平山、楊宅、慄傳、禾山、勝鄉、造口、秀洲、新砦、北鄉、黃茅峽。



 荊
 湖南路



 永州三砦都巡、同巡、衡永界。



 寶慶三砦黃茅、西縣、盧溪。



 郴州五砦管界、安福、青要、赤石、上猶。



 武岡軍十砦三門、石查、真良、岳溪、臨口、關硤、黃石、新寧、綏寧、永和。



 道州四砦營道、寧遠、江華、永明。



 全州四砦上軍、角口、吉寧、平塘。



 福建路



 邵武軍十砦同巡檢、大寺、水口、永安、明溪、仁壽、西安、永平、軍口、梅口。



 建寧府七砦黃琦、籌嶺、盆亭、麻沙、水吉、苦竹、仁壽。



 南劍州八砦滄峽、洛陽、浮流、巖前、同巡、仁壽、萬安、黃土。



 泉州五砦都巡、同巡、石井、小兜、三縣。



 福州四砦辜嶺、甘蔗、五縣、水口。



 興化軍二砦同巡、巡鹽。



 漳州二砦同巡、虎嶺。



 廣西路



 賀州二砦臨賀、富川。



 昭州四砦昭平、雲峒、西嶺、直山。



 欽州二砦西縣、管界。



\end{pinyinscope}