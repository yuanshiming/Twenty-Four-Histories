\article{列傳第四十五}

\begin{pinyinscope}

 魯
 宗道薛奎王曙子益柔蔡齊從子延慶



 魯宗道,字貫之,亳州譙人。少孤,鞠於外家。諸舅皆武人,頗易宗道,宗道益自奮厲讀書。袖所著文謁戚綸,綸器
 重之。舉進士,為濠州定遠尉,再調海鹽令。縣東南舊有港,導海水至邑下,歲久湮塞,宗道發鄉丁疏治之,人號「魯公浦」。改歙州軍事判官,再遷秘書丞。陳堯叟闢通判河陽。



 天禧元年,始詔兩省置諫官六員,考所言為殿最,首擢宗道與劉燁為右正言。諫章由閣門始得進而不賜對,宗道請面論事而上奏通進司,遂為故事。嘗言:「守宰去民近,而無以區別能否。今除一守令,雖資材低下,而考任應格,則左司無擯斥,故天下親民者黷貨害政,
 十常二三,欲裕民而美化,不可得矣。漢宣帝除刺史守相,必親見而考察之。今守佐雖未暇親見,宜令大臣延之中書,詢考以言,察其應對,設之以事,觀其施為才不肖,皆得進退之。吏部之擇縣令放此,庶得良守宰宣助聖化矣。」真宗納之。宗道風聞,多所論列,帝意頗厭其數。後因對,自訟曰:「陛下用臣,豈欲徒事納諫之虛名邪?臣竊恥尸祿,請得罷去。」帝撫諭良久,他日書殿壁曰:「魯直」,蓋思念之也。尋除戶部員外郎兼右諭德。逾年,遷左諭
 德、直龍圖閣。



 仁宗即位,遷戶部郎中、龍圖閣直學士兼侍講、判吏部流內銓。宗道在選調久,患銓格煩密,及知吏所以為奸狀,多厘正之,悉揭科條廡下,人便之。雷允恭擅易山陵,詔與呂夷簡等按視。還,拜右諫議大夫、參知政事。



 章獻太后臨朝,問宗道曰:「唐武后何如主?」對曰:「唐之罪人也,幾危社稷。」後默然。時有請立劉氏七廟者,太后問輔臣,眾不敢對。宗道不可,曰:「若立劉氏七廟,如嗣君何?」帝、太后將同幸慈孝寺,欲以大安輦先帝行,宗
 道曰:「夫死從子,婦人之道也。」太后遽命輦後乘輿。時執政多任子於館閣讀書,宗道曰:「館閣育天下英才,豈紈褲子弟得以恩澤處邪?」樞密使曹利用恃權驕橫,宗道屢於帝前折之。自貴戚用事者皆憚之,目為「魚頭參政」,因其姓,且言骨鯁如魚頭也。再遷尚書禮部侍郎、祥源觀使。在政府七年,務抑僥幸,不以名器私人。疾劇,帝臨問,賜白金三千兩。既卒,皇太后臨奠之,贈兵部尚書。



 宗道為人剛正,疾惡少容,遇事敢言,不為小謹。為諭德時,
 居近酒肆,嘗微行就飲肆中,偶真宗亟召,使者及門久之,宗道方自酒肆來。使者先入,約曰:「即上怪公來遲,何以為對?」宗道曰:「第以實言之。」使者曰:「然則公當得罪。」曰:「飲酒,人之常情;欺君,臣子之大罪也。」真宗果問,使者具以宗道所言對。帝詰之,宗道謝曰:「有故人自鄉里來,臣家貧無杯盤,故就酒家飲。」帝以為忠實可大用,嘗以語太后,太后臨朝,遂大用之。初,太常議謚曰剛簡,復改為肅簡。議者以為「肅」不若「剛」為得其實云。



 薛奎,字宿藝,絳州正平人。父化光,善數術,嘗以平晉策干太宗行在,召見不用,罷歸。適奎始生,撫其首曰:「是子必至公輔。」奎舉進士,為州第一,乃推與里人王嚴,而處嚴下。進士及第,為隰州軍事推官。州民常聚博僧舍,一日,盜殺寺奴取財去,博者適至,血偶涴衣,邏卒捕送州,考訊誣伏。奎獨疑之,白州緩其獄,後果得殺人者。徙儀州推官,嘗部丁夫運糧至鹽州,會久雨,粟麥漬腐,奎白轉運盧之翰,請縱民還州而償所失。之翰怒,欲劾奏之。
 奎徐曰:「用兵久,人疲轉餉,今幸兵食有餘,安用此陳腐以困民哉!」之翰意解,凡民所失,悉奏除之。改大理寺丞、知莆田縣。請蠲南閩時稅咸魚、蒲草錢。



 遷殿中丞、知長水縣,徙知永州。州有錢監,歲調兵三百人採鐵,而歲入不償費。奎奏聽民自採,而所輸輒倍之。遷太常博士。向敏中薦為殿中侍御史,出為陜西轉運使。趙德明言延州蕃落侵其地黑林平,下詔按驗。奎閱郡籍,德明嘗假道黑林平,移文錄示之,德明遂伏。未幾,坐失舉免。數月,
 起通判陜州,改尚書戶部員外郎、淮南轉運副使,遷江、淮制置發運使。疏漕河、廢三堰以便餉運,進吏部員外郎。父喪,奪哀,擢三司戶部副使。與使李士衡爭論事,改戶部郎中、直昭文館、知延州。



 趙元昊每遣吏至京師請奉予,吏因市禁物,隱關算為奸利,奎廉得狀,請留蜀道縑帛於關中,轉致給之。遷吏部,擢龍圖閣待制、權知開封府。為政嚴敏,擊斷無所貸,帝益加重。使契丹,還,遷右諫議大夫、權御史中丞。上疏論擇人、求治、崇節儉、屏聲
 色,凡十數事。章獻太后稱制,契丹使蕭從順請見太后,且言南使至契丹者皆見太后,而契丹使來乃不得見。奎時館伴,折之曰:「皇太后垂簾聽政,雖本朝群臣,亦未嘗見也。」從順乃已。或讒云奎漏禁中語,改授集賢院學士、知並州,改秦州。州宿重兵,經費常不足,奎務為儉約,教民水耕,謹商算。歲中積粟三百萬,徵算餘三千萬,核民隱田數千頃,得芻粟十餘萬。加樞密直學士、知益州。秦民與夷落數千人列奎治狀,請留,璽書褒諭,不許。成
 都民婦訟其子不孝,詰之,乃曰:「貧無以為養。」奎出俸錢與之,戒曰:「若復失養,吾不貸汝矣!」其母子遂如初。嘗夜燕,有戍卒殺人,人皆奔走,奎密遣捕殺之,坐客莫有知者。臨事持重明決,多此類也。



 召為龍圖閣學士、權三司使,遂參知政事。帝諭曰:「先帝嘗以為卿可任,今用卿,先帝意也。」俄遷給事中。帝嘗謂輔臣曰:「臣事君鮮有克終者。」奎曰:「保終之道,匪獨臣不然也。」歷數唐開元、天寶時事以對,帝然之。遷尚書禮部侍郎。太后謁太廟,欲被服
 天子袞冕,奎曰:「必御此,若何為拜?」力陳其不可,終不見聽。及太后崩,帝見左右泣曰:「太后疾不能言,猶數引其衣若有所屬,何也?」奎曰:「其在袞冕也。服之豈可見先帝於地下!」帝悟,卒以後服斂。因上言請逐內侍羅崇勛等。時二府大臣多罷去,奎得喘疾,數辭位,罷為戶部侍郎、資政殿學士、判尚書都省。帝手書禁方賜之,小間,入見。疾尋作,卒,贈兵部尚書,謚簡肅。



 奎性剛不茍合,遇事敢言,真宗時數宴大臣,至有沾醉者。奎諫曰:「陛下即位之
 初,勵精萬幾而簡宴幸。今天下誠無事,而宴樂無度,大臣數被酒無威儀,非所以重朝廷也。」真宗善其言。及參政事,謀議無所避。能知人,範仲淹、龐籍、明鎬自為吏部選人,皆以公輔許之。無子,以從子為嗣。



 王曙,字晦叔,隋東皋子績之後。世居河汾,後為河南人。中進士第,再調定國軍節度推官。咸平中,舉賢良方正科,策入等,遷秘書省著作佐郎、知定海縣。還,為群牧判官,考集古今馬政,為《群牧故事》六卷,上之。遷太常丞、判
 三司憑由理欠司。坐舉進士失實,降監盧州茶稅,再遷尚書工部員外郎、龍圖閣待制。以右諫議大夫為河北轉運使,坐部吏受賕,降知壽州。徙淮南轉運使,勾當三班院,權知開封府。以樞密直學士知益州。繩盜以峻法,多致之死。有卒夜告其軍將亂,立辨其偽,斬之。蜀人比之張詠,號「前張後王」。入為給事中。仁宗為皇太子,與李迪同選兼賓客,復坐貢舉失實,黜官。復為給事中兼群牧使。其妻,寇準女也。準罷相且貶,曙亦降知汝州。準再
 貶,曙亦貶郢州團練副使。起為光祿卿、知襄州,又徙汝州。復給事中、知潞州。州有殺人者,獄已具,曙獨疑之。既而提點刑獄杜衍至,事果辨。曙為作《辨獄記》以戒官吏。



 徙河南府、永興軍,召為御史中丞兼理檢使,理檢置使自此始。玉清昭應宮災,系守衛者御史獄。曙恐朝廷議修復,上言:「昔魯桓、僖宮災,孔子以為桓、僖親盡當毀者也。遼東高廟及高園便殿災,董仲舒以為高廟不當居陵旁,故災。魏崇華殿災,高堂隆以壹榭宮室為戒,宜罷
 之勿治,文帝不聽,明年,復災。今所建宮非應經義,災變之來若有警者。願除其地,罷諸禱祠,以應天變。」仁宗與太后感悟,遂減守衛者罪。已而詔以不復繕修諭天下。又請三品以上立家廟,復唐舊制。以尚書工部侍郎參知政事。以疾請罷,改戶部侍郎、資政殿學士、知陜州,徙河陽。再知河南府,遷吏部。召為樞密使,拜同中書門下平章事。逾月,首發疽,卒。贈太保、中書令,謚文康。



 曙方嚴簡重,有大臣體,居官深自抑損。喜浮圖法,齊居蔬食,泊
 如也。初,錢惟演留守西京,歐陽修、尹洙為官屬。修等頗游宴,曙後至,嘗厲色戒修等曰:「諸君縱酒過度,獨不知寇萊公晚年之禍邪!」修起對曰:「以修聞之,萊公正坐老而不知止爾!」曙默然,終不怒。及為樞密使,首薦修等,置之館閣。有集四十卷,《周書音訓》十二卷,《唐書備問》三卷,《莊子旨歸》三篇,《列子旨歸》一篇,《戴斗奉使錄》二卷,集《兩漢詔議》四十卷。



 子益恭、益柔。益恭字達夫,以蔭為衛尉寺丞。性恬淡,慕唐王龜之為人,數解官就養。曙參知政
 事,治第西京,益恭勸曙引年謝事,曙不果去。終父喪,遂以尚書司門員外郎致仕,間與浮圖、隱者出游,洛陽名園山水,無不至也。以子登朝,累遷司農少卿,卒。



 益柔字勝之。為人伉直尚氣,喜論天下事。用蔭至殿中丞。元昊叛,上備邊選將之策。杜衍、丁度宣撫河東,益柔寓書言:「河外兵餉無法,非易帥臣、轉運使不可。」因條其可任者。衍、度使還,以學術政事薦,知介丘縣。慶歷更用執政,異意者指為朋黨,仁宗下詔戒敕,益柔上書論辨,
 言尤切直。尹洙與劉滬爭城水洛事,自涇原貶慶州。益柔訟之曰:「水洛一障耳,不足以拒賊。滬裨將,洙為將軍,以天子命呼之不至,戮之不為過;顧不敢專執之以聽命,是洙不伸將軍之職而上尊朝廷,未見其有罪也。」不聽。範仲淹未識面,以館閣薦之,除集賢校理。預蘇舜欽奏邸會,醉作《傲歌》。時諸人欲遂傾正黨,宰相章得像、晏殊不可否,參政賈昌朝陰主之,張方平、宋祁、王拱辰攻排不遺力,至列狀言益柔罪當誅。韓琦為帝言:「益柔狂
 語何足深計。方平等皆陛下近臣,今西陲用兵,大事何限,一不為陛下論列,而同狀攻一王益柔,此其意可見矣。」帝感悟,但黜監復州酒。久之,為開封府推官、鹽鐵判官。凡中旨所需不應法式,有司迎合以求進者,悉論之不置。出為兩浙、京東西轉運使。上言:「今考課法區別長吏能否,必明有顯狀,顯狀必取其更置興作大利。夫小政小善,積而不已,然後能成其大。取其大而遺其細,將競利圖功,恐事之不舉者日多,而虛名無實之風日起。
 願參以唐四善,兼取行實,列為三等。」不行。



 熙寧元年,入判度支審院。詔百官轉對,益柔言:「人君之難,莫大於辨邪正;邪正之辨,莫大於置相。相之忠邪,百官之賢否也。若唐高宗之李義甫,明皇之李林甫,德宗之盧杞,憲宗之皇甫鎛,帝王之鑒也。高宗、德宗之昏蒙,固無足論;明皇、憲宗之聰明,乃蔽於二人如此。以二人之庸,猶足以致禍,況誦六藝、挾才智以文致其奸說者哉!」意蓋指王安石也。判吏部流內銓。舊制,選人當改京官,滿十人乃
 引見。由是士多困滯,且遇舉者有故,輒不用。益柔請才二人即引見,眾論翕然稱之。直舍人院、知制誥兼直學士院。董氈遇明堂恩,中書熟狀加光祿大夫,而舊階已特進,益柔以聞。帝謂中書曰:「非翰林,幾何不為羌夷所笑。」宰相怒其不申堂,用他事罷其兼直。遷龍圖閣直學士、秘書監,知蔡揚亳州、江寧應天府。卒,年七十二。



 益柔少力學,通群書,為文日數千言。尹洙見之曰:「贍而不流,制而不窘,語淳而厲,氣壯而長,未可量也。」時方以詩賦
 取士,益柔去不為。範仲淹薦試館職,以其不善詞賦,乞試以策論,特聽之。司馬光嘗語人曰:「自吾為《資治通鑒》,人多欲求觀讀,未終一紙,已欠伸思睡。能閱之終篇者,惟王勝之耳。」其好學類此。



 蔡齊,字子思,其先洛陽人也。曾祖綰,為萊州膠水令,因家焉。齊少孤,依外家劉氏。舉進士第一。儀狀俊偉,舉止端重,真宗見之,顧宰相寇準曰:「得人矣。」詔金吾給七騶,傳呼以寵之。狀元給騶,自齊始也。除將作監丞、通判袞
 州,徙濰州。以秘書省著作郎直集賢院。



 仁宗初,為司諫、修起居注,改尚書禮部員外郎兼侍御史知雜事。錢惟演守河陽,請曲賜鎮兵錢,章獻太后將許之。齊曰:「上新即位,惟演外戚,請偏賞以示私恩,不可許。」遂劾奏惟演。以起居舍人知制誥,入為翰林學士,加侍讀學士。太后大出金帛修景德寺,遣內侍羅崇勛主之,命齊為文記之。崇勛陰使人誘齊曰:「趣為記,當得參知政事矣。」齊久之不上,崇勛讒之,罷為龍圖閣學士、知河南府。參知政
 事魯宗道固爭留之,不能得。以親老,改密州,徙應天府,召為右諫議大夫、御史中丞。



 太后崩,遺詔以楊太妃為皇太后,同裁制軍國事。閣門趣百官賀,齊使臺吏毋追班,乃入白執政曰:「上春秋富,習知天下情偽,今始親政事,豈宜使女後相踵稱制乎!」遂罷預政。復為龍圖閣學士、權三司使。有飛語傳荊王元儼為天下兵馬都元帥者,捕得系獄,連逮甚眾。帝怒,使齊按問之。齊曰:「此小人無知,不足治,且無以安荊王。」帝悟,遽釋之。拜樞密副使。
 交址虐其部人,款宜州自歸者八百餘人,議者謂不可內。齊曰:「蠻人去暴而歸有德,卻之不祥,請給荊湖閑田使自營;若縱去,當不復還舊部,必聚而為盜賊矣。」不從。後數年,蠻果為亂。蜀大姓王齊雄坐殺人除名。齊雄,太后姻家,未更赦,復官。齊曰:「果如此,法撓矣!」明日,入奏事曰:「齊雄恃勢殺人,不死,又亟授以官,是以恩廢法也。」帝曰:「降一等與官可乎?」齊曰:「以恩廢法,如朝廷何!」帝勉從之,乃抵齊雄罪。錢惟演附丁謂,樞密題名,輒削去寇準
 姓氏,云「逆準不書」。齊言於仁宗曰:「寇準忠義聞天下,社稷之臣也,豈可為奸黨所誣哉!」仁宗遽令磨去。



 郭皇后廢,將立富人陳氏女為後,齊極論之。拜禮部侍郎、參知政事。契丹祭天於幽州,以兵屯境上。輔臣欲調兵備邊,與齊迭議帝前,齊畫三策,料契丹必不叛盟。王曾與齊善,曾與夷簡不相能,曾罷相,齊亦以戶部侍郎歸班。尋出知穎州,卒,年五十二,贈兵部尚書,謚曰文忠。穎人見其故吏朱採會喪,猶號泣思之。



 齊方重有風採,性謙退,
 不妄言。有善未嘗自伐。丁謂秉政,欲齊附己,齊終不往。少與徐人劉顏善,顏罪廢,齊上其書數十萬言,得復官。顏卒,又以女妻其子庠。所薦龐籍、楊偕、劉隨、段少連,後率為名臣。始,齊無子,以從子延慶為後。既歿,有遺腹子曰延嗣。



 延慶字仲遠,中進士第,通判明州。歷福建路轉運判官,提點京東、陜西刑獄。神宗初,以集賢校理歷開封府推官。有衛士告黃衣老卒筒火入直,延慶察卒色辭,疑焉,
 詢之,果為所誣,即反坐告者。事聞,帝重之,加直史館、知河中府。明年,同修起居注,直舍人院、判流內銓,拜天章閣待制、秦鳳等路都轉運使,以應辦熙河軍須功,進龍圖閣直學士。



 王韶進師河州,羌斷其歸路。延慶曰:「兵事非吾所宜預,然主帥在難,不急援之,恐敗國事。」遂檄兵赴救,羌解去,韶得全師還。轉運判官蔡曚劾其擅興,朝廷問知狀,易曚他道。韶入朝,延慶攝熙帥。元夕張燈,羌乘隙伏兵北關下,遣其種二十九人偽請來屬,將舉火
 內應。延慶覘知,悉斬以徇,伏者宵潰。蕃官詐稱木徵欲降,邀大將景思立來迎。延慶命毋輒出,即違節制,雖有功亦誅,思立不從,卒敗死。



 徙知成都府兼兵馬都鈐轄。本道舊不置都鈐轄,至是特命之。茂州羈縻州蠻族九,自推一人為將統其眾,將常在州聽要束。州居群蠻中,無城塹,惟樹鹿角為固。蠻屢夜入剽人畜,徼貨來贖。民患苦,詣郡守李琪請築城。琪上於朝,詔延慶度其利便,延慶下其事,琪已去。後守範百常以為利,築之。蠻酋訴
 謂侵其土地,乞罷築,不許。蠻數百奄至,拒卻之。明日,又大至,盡焚鹿角及民盧舍,引梯沖攻牙城,百常捍禦,殺二蠻酋,乃退。然游騎猶繞四山,南北路皆為所據,城中不敢出。百常募人間道告急於成都。延慶命與之和,奏乞遣近上內臣共經蠻事。詔押班王中正往,中正受旨,凡軍事皆令與都鈐轄議。將行,言茂去成都遠,一一與議,慮失事機,請得專決。於是事無鉅細皆自處,延慶不復預。監司附中正,奏延慶區理失宜,致生邊患。徙知渭
 州,仍降為天章閣待制。



 夏人禹臧苑麻疑邊境有謀,使人入塞賣馬,吏執以告。延慶曰:「彼疑,故來覘。執之,是成其疑。」約馬直授之使去。疆吏入敵境攘羊馬,得而戮諸境上,且告之曰:「兩境不相侵,則相保以安,故戮以戒。若有之,亦當爾也。」夏人悅服。



 嘗得《安南行軍法》讀之,仿其制,部分正兵弓箭手人馬,團為九將,合百隊,分左右前後四部。隊有駐戰、拓戰之別,步騎器械,每將皆同。以蕃兵人馬為別隊,各隨所近分隸焉。諸將之數,不及正兵
 之半,乃所以制之。處老弱於城砦,較其遠近而為區別。使蕃、漢無得相雜,以防其變。具為書上之。時鄜延呂惠卿亦分畫兵,延慶條其不便,神宗善其議。召知開封府,拜翰林學士。以言者罷知滁州,歷瀛、洪州,復龍圖閣待制,帥高陽。閱歲,復直學士,移定武。元祐中,入為工部、吏部侍郎。卒,年六十二,賜錢三十萬,官庀其葬。



 延慶有學問,平居簡嘿,遇事能別白是非,所至有惠政。既為伯父齊後,齊晚得子,乃歸其宗,籍家所有付之,無一毫自予,
 萊人義焉。



 論曰:「章獻太后稱制時,群臣多希合用事,魯宗道、薛奎、蔡齊參預其間,正色孤立,無所回撓。宗道能沮劉氏七廟之議,奎正母後袞冕為非禮,齊從容一言絕女後相踵稱制之患,真所謂以道事君者歟!曙辨奸斷獄,為時良吏,在位又多薦拔名臣,若請群臣立家廟以復古禮,皆知為政之本焉



\end{pinyinscope}