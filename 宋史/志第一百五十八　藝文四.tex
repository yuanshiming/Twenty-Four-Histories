\article{志第一百五十八 藝文四}

\begin{pinyinscope}

 子類十七:一曰儒家類,二曰道家類釋氏及神仙附,三曰法家類,四曰名家類,五曰墨家類,六曰縱橫家類,七曰農家類,八曰雜家類,九曰小說家類,十曰天文類,十一曰五
 行類,十二曰蓍龜類,十三曰歷算類,十四曰兵書類,十五曰雜藝術類,十六曰類事類,十七曰醫書類。



 《晏子春秋》十二卷



 《曾子》二卷



 《子思子》七卷



 《孟子》十四卷



 陸善經《孟子注》七卷



 王雱注《孟子》十四卷



 蔣之奇《孟子解》六卷



 《荀卿子》二十卷戰國趙人荀況書



 楊倞注《荀子》二十卷



 黎錞《校勘荀子》二十卷



 《魯仲連子》五卷戰國齊人



 《董子》一卷董無心撰



 《尸子》一卷尸佼撰



 《子華子》十卷自言程氏名本,字子華,晉國人。《中興書目》曰:「近世依托。」朱熹曰:「偽書也。」



 《孔叢子》七卷漢孔鮒撰。朱熹曰:「偽書也。」



 桓寬《監鐵論》
 十卷



 揚雄《太玄經》十卷



 又《揚子法言》十三卷



 張齊《太玄正義統論》一卷



 又《太玄釋文玄說》二卷



 宋惟乾《太玄經注》十卷



 王涯注《太玄經》六卷



 柳宗元注《揚子法言》十三卷宋咸補注



 馬融《忠經》一卷



 《玄測》一卷漢宋衷解,吳陸績釋之



 王符《潛夫論》十卷



 關朗《洞極元經傳》五卷



 《四注孟子》十四卷揚雄、韓愈、李翱、熙時子四家注



 王通《文中子》十卷宋阮逸注



 太宗《帝範》二卷



 顏師古《糾繆正俗》八卷



 王涯《說玄》一卷



 林慎思《續孟子》二卷



 韓熙載《格言》五卷



 真宗《正說》十卷



 徐鉉《質論》一卷



 許
 洞《演玄》十卷



 刁衎《本說》十卷



 王敏《太平書》十卷



 賈岡《山東野錄》七卷



 宋咸《過文中子》十卷



 又《太玄音》一卷



 章察《太玄圖》一卷



 又《太玄經發隱》一卷



 聱隅子《歔欷巢微論》一卷黃晞撰



 邵亢《體論》十卷



 周惇頤《太極通書》一卷



 司馬光《潛虛》一卷



 又《文中子傳》一卷



 《集注四家揚子》十三卷



 《集注太玄經》六卷



 並司馬光集



 《家範》十卷



 師望《元鑒》十卷



 範鎮《正書》一卷



 張載《正蒙書》十卷



 又《雜述》一卷



 《程頤遺書》二十五卷



 《語錄》二卷程頤與弟子問答



 《孟子解》四卷程頤門人記



 徐
 積《節孝語》一卷江端禮錄



 呂大臨《孟子講義》十四卷



 蘇轍《孟子解》一卷



 王令《孟子講義》五卷



 龔原《孟子解》十卷



 陳暘《孟子解義》十四卷



 張鎰《孟子音義》三卷



 丁公著《孟子手音》一卷



 孫奭《孟子音義》二卷



 劉安世《語錄》二卷



 王開祖《儒志》一卷



 游酢《孟子解義》十四卷



 又《雜解》一卷



 謝良佐《語錄》一卷



 陳禾《孟子傳》十四卷



 晁說之《易玄星紀譜》二卷



 陳漸《演玄》七卷



 許允成《孟子新義》十四卷



 範沖《要語》一卷



 張九成《孟子拾遺》一卷



 《語錄》十四卷



 張憲武《勸學
 錄》六卷



 劉子翬《十論》一卷



 張行成《潛虛衍義》十六卷



 又《皇極經世索隱》一卷



 《觀物外篇衍義》九卷



 《翼玄》十二卷



 鄭樵《刊繆正俗跋正》八卷



 文軫《信書》三卷



 《宋衷解太玄經義訣》十卷李沂集



 馮休《刪孟子》一卷



 陳之方《致君堯舜論》一卷



 又《削荀子疵》一卷



 徐庸《注太玄經》十二卷



 又《玄頤》一卷



 僧全瑩《太玄略例》一卷



 王紹珪《古今孝悌錄》二十四卷



 尹焞《孟子解》十四卷



 《語錄》四卷尹焞門人馮忠恕、祁寬、呂堅中記



 鄒浩《孟子解》十四卷



 朱熹《孟子集注》十四卷



 又《孟子
 集義》十四卷



 《或問》十四卷



 《延平師弟子問答》一卷



 《語錄》四十三卷朱熹門人所記



 張栻《孟子詳說》十七卷



 又《孟子解》七卷



 蔡沆《至書》一卷



 張氏《孟子傳》三十六卷



 錢文子《孟子傳贊》十四卷



 王汝猷《孟子辨疑》十四卷



 《諸儒鳴道集》七十二卷濂溪、涑水、橫渠等書



 程迥《諸論辨》一卷



 《近思錄》十四卷朱熹、呂祖謙編類周敦頤、程頤、程顥、張載等書



 《外書》十二卷程顥、程頤講學



 邵雍《漁樵問對》一卷



 祝禹圭《東西銘解》一卷



 蘇籀《遺言》一卷



 曾發《泮林討古》二卷



 張九成《語錄》十四卷



 胡宏《知言》一卷



 《麗澤
 論說集》十卷呂祖謙門人記



 周揆《聖傳錄》一卷



 吳仁傑《鹽石論》丙丁二卷



 陳舜申《審是集》一卷



 塗近正《明倫》二卷



 彭龜年《止堂訓蒙》二卷



 《呂氏鄉約儀》一卷呂大鈞撰



 《李公省心雜言》一卷不知名



 董與幾《學政發縱》一卷



 高登《修學門庭》一卷



 劉敞《弟子記》一卷



 《石月至言》一卷餘應求刊其父之言



 戴溪《石鼓孟子答問》三卷



 陳師道《後山理究》一卷



 《北山家訓》一卷



 《伊洛淵源》十三卷



 《聞見善善錄》一卷



 《質疑請益》一卷



 並不知作者



 楊浚《韋子內篇》三卷



 又《聖典》三卷



 王向《忠經》三
 卷



 劉貺《續說苑》十卷



 《法聖要言》十卷



 李琪《皇天大政論》十卷



 高舉《帝道書》十卷



 魯大公《公侯正術》十卷



 蕭佚《牧宰政術》二卷



 趙瑩《君臣政論》二十五卷



 《興政論》三卷



 丘光庭《康教論》一卷



 張弧《素履子》一卷



 張陟《里訓》十卷



 趙澡《中庸論》一卷



 趙鄰幾《鯫子》一卷



 朱昂《理論》三卷



 何涉《治道中說》三十篇卷亡



 龔鼎臣《中說解》十卷



 範祖禹《帝學》八卷



 章懷太子《修身要覽》十卷



 太宗《文明政化》十卷



 真宗《承華要略》二十卷



 《名墨縱橫家無所增益答邇英
 聖問》一卷



 仁宗書三十五事,丁度等答



 顏之推《家訓》七卷



 狄仁傑《家範》一卷



 《先賢誡子書》二卷



 《開元禦集誡子書》一卷



 《古今家戒》四卷



 黃訥《家戒》一卷



 柳玢《誡子拾遺》十卷



 孫奕《示兒編》一部



 右儒家類一百六十九部,一千二百三十四卷、篇。



 河上公《老子道德經注》一卷



 嚴遵《老子指歸》十三卷



 王弼《老子注》二卷



 又《道德略歸》一卷



 陸修靜《老子道德經雜說》一卷



 傅奕《道德經音義》二卷



 唐玄宗注《老子道德
 經》二卷有序



 唐玄宗《道德經音疏》六卷



 成玄英《道德經開題序訣義疏》七卷



 杜光庭《道德經廣聖義疏》三十卷



 僧文儻《道德經疏義》十卷



 趙至堅《道德經疏》三卷



 張惠超《道德經志玄疏》三卷



 陸氏《道德經傳》四卷



 扶少明《道德經譜》二卷



 《穀神子注經諸家道德經疏》二卷河上公、葛仙公、鄭思遠、睿宗、玄宗疏



 李若愚《道德經注》一卷



 喬諷《道德經疏義節解》二卷



 《道德經小解》一卷



 陳景元《道德注》二卷



 蔣之奇《老子解》二卷



 又《老子系辭解》二卷



 張湛《列子音義》一卷



 張
 昭《補注莊子》十卷



 張烜《莊子通真論》三卷



 《南華真經篇目義》三卷



 李暹《訓文子注》十二卷



 朱棄《文子注》十二卷



 墨布一作「希」



 子《文子注》十二卷



 王源《亢倉子注》三卷



 《亢倉子音義》一卷



 範乾元一作「九」



 《四子樞要》二卷



 衛偕一作「稽」



 《白朮子》三卷



 太公等《陰符經注》一卷



 張果《陰符經注》一卷



 又《陰符經辨命論》一卷



 袁淑真《陰符經注》一卷



 又《陰符經疏》三卷



 《陰符集解》五卷



 韋洪《陰符經疏訣》一卷



 蔡珪《陰符經注》一卷



 又《陰符經要義》一卷



 《陰符經小解》一卷



 張魯《陰符經元義》一卷



 李靖《陰符機》一卷



 房山長注《大丹黃帝陰符經》一卷



 梁丘子注《黃庭內景玉經》一卷



 《黃庭外景經》一卷



 《黃庭外景玉經注訣》一卷



 《黃庭五藏論圖》一卷



 《老子黃庭內視圖》一卷



 胡愔《黃庭內景圖》一卷



 《黃庭外景圖》一卷



 魏伯陽《周易參同契》三卷



 《參同大易言志》三卷



 徐從事注《周易參同契》三卷



 《參同契合金丹行狀十六變通真訣》一卷



 鄭遠之《參同契心鑒》一卷



 張處《參同契大易圖》一卷



 晁公武《老子通述》二卷



 《老子道德
 經三十家注》六卷唐道士張君相集解



 葛玄《老子道德經節解》二卷



 《道德經內解》二卷不知作者



 《老子道德經內節解》二卷題尹先生注



 王顧《老子道德經疏》四卷



 李榮《老子道德經注》二卷



 李約《老子道德經注》四卷



 碧雲子《老子道德經藏室纂微》二卷不知名



 《老子道德經義》二卷



 《老子指例略》一卷



 並不知作者



 張湛《列子注》八卷



 郭象注《莊子》十卷



 成玄英《莊子疏》十卷



 文如海《莊子正義》十卷



 又《莊子邈》一卷



 《黃帝陰符經》一卷舊目云驪山老母注、李筌撰



 《集注老子》二卷明皇、河上公、王弼、王
 雱等注



 呂知常《老子講義》十二卷



 李筌《陰符經疏》一卷



 《陰符玄譚》一卷不知作者



 《文子》十二卷舊書目云周文子撰



 《鶡冠子》三卷不知姓名。《漢志》云:「楚人,居深山,以鶡羽為冠,因號云。」



 《亢倉子》三卷一名庚桑子。戰國時人,老子弟子



 《抱樸子別旨》二卷不知作者



 司馬子微《坐忘論》一卷



 《天機經》一卷



 《道體論》一卷



 《無能子》一卷



 並不知作者



 吳筠《玄綱》一卷



 劉向《關尹子》九卷



 劉驥《老子通論語》二卷



 徽宗《老子解》二卷



 《列子解》八卷



 呂惠卿《莊子解》十卷



 司馬光《老子道德經注》二卷



 蘇轍《老子道德經義》二卷



 趙令穆《老子
 道德經解》二卷



 李士表《莊子十論》一卷



 沈該《陰符經注》一卷



 朱熹《周易參同契》一卷



 朱安國《陰符元機》一卷



 程大昌《易老通言》十卷



 右道家類一百二部,三百九十九卷



 鳩摩羅什譯《金剛般若波羅蜜經》一卷



 沙門曇景譯《佛說未曾有因緣經》二卷



 玄奘譯《波般若波羅蜜多心經》一卷



 般刺密帝彌伽釋迦譯《首楞嚴經》十卷



 《佛說一乘究竟佛心戒經》一卷



 《佛說三亭廚法經》二卷



 《佛說法句
 經》一卷



 《佛說涅盤略說教戒經》一卷四經失譯



 馬鳴大師《摩訶衍論》五卷



 《起信論》二卷



 僧肇《寶藏論》三卷



 彥琮《福田論》一卷



 道信《大乘入道坐禪次第要論》一卷



 法琳《辨正論》八卷陳子良注



 彗海大師《入道要門論》一卷



 凈本和尚《語論》一卷



 惠能《仰山辨宗論》一卷



 《勸修破迷論》一卷



 《金沙論》一卷



 《明道宗論》一卷



 《偈宗秘論》一卷



 四論不知撰人



 法藏《心經》一卷



 惟愨《首楞嚴經疏》六卷



 宗密《圓覺經疏》六卷



 《圓覺道場修證儀》十八卷



 《起信論鈔》三卷



 傅大士、寶志《金剛
 經贊》一卷



 惠能《金剛經口訣義》一卷



 《金剛經大義訣》二卷



 大白和尚《金剛經訣》一卷



 法深《起信論疏》二卷



 忠師《百法明門論疏》二卷



 蕭子良《統略凈住行法門》一卷



 元康《中觀論三十六門勢疏》一卷



 《華嚴法界觀門》一卷宗密注



 傅大士《心王傳語》一卷



 《行道難歌》一卷



 竺道生《十四科元贊義記》一卷



 灌頂《國清道場百錄》一卷



 楞伽山主《小參錄》一卷



 道宣《通感決疑錄》一卷



 《大唐國師小錄法要集》一卷



 紹修《漳州羅漢和尚法要》三卷持琛



 白居易《八漸
 通真議》一卷



 張云《元中語寶》三卷



 大閬和尚《顯宗集》一卷



 《大雲和尚要法》一卷惠海



 元覺《一宿覺傳》一卷



 魏靜《永嘉一宿覺禪宗集》一卷



 《達摩血脈》一卷



 本先《竹林集》一卷



 寶覺禪師《見道頌》一卷寓言居士注



 道瑾《禪宗理偈》一卷



 《石頭和尚參同契》一卷宗美注



 《惠忠國師語》一卷冉氏



 《東平大師默論》一卷



 義榮《天臺國師百會語要》一卷



 齊寶《神要》三卷



 懷和《百丈廣語》一卷



 統休《無性和尚說法記》一卷



 惠明《□妻賢法雋》一卷



 《龍濟和尚語要》一卷



 《荷澤禪師
 微訣》一卷



 楊士達《禪關八問》一卷宗美



 句令《禪門法印傳》五卷



 《凈惠禪師偈頌》一卷



 義凈《求法高僧傳》二卷



 飛錫《往生凈土傳》五卷



 法海《六祖法寶記》一卷



 《壇經》一卷



 辛崇《僧伽行狀》一卷



 靈湍《攝山□妻霞寺記》一卷



 師哲《前代國王修行記》一卷



 盧求《金剛經報應記》三卷



 賢首《華嚴經纂靈記》五卷



 元偉《真門聖冑集》五卷



 《雲居和尚示化實錄》一卷



 覺旻《高僧纂要》五卷



 智月《僧美》三卷



 裴休《拾遺問》一卷



 神澈《七科義狀》一卷



 夢微《內典編要》十卷



 《紫陵語》一
 卷



 《大藏經音》四卷



 《真覺傳》一卷



 《渾混子》三卷解《寶藏論》



 《遺聖集》一卷



 《菩提心記》一卷



 《積元集》一卷



 《相傳雜語要》一卷



 《德山集》一卷仰山、溈山語



 《會昌破胡集》一卷



 《妙香丸子法》一卷



 《潤文官錄》一卷唐人。



 《迦葉祖裔記》一卷



 《釋門要錄》五卷



 《紫陵語》以下不知撰人



 十朋《請禱集》一卷



 《瑞象歷年記》一卷



 《惟勁禪師贊頌》一卷



 《釋華嚴漩澓偈》一卷



 馬裔孫《看經贊》一卷



 《法喜集》二卷



 文益《法眼禪師集》一卷



 《法眼禪師集真贊》一卷



 高越《舍利塔記》一卷



 可洪《藏經音義隨函》三十卷



 建隆《雍熙禪頌》三卷



 魏德謨《無上秘密小錄》五卷



 程讜《釋氏蒙求》五卷



 延壽《感通賦》一卷



 李遵《天聖廣燈錄》三十卷



 呂夷簡《景祐寶錄》二十一卷



 僧肇《寶藏論》一卷



 又《般若無知論》一卷



 《涅盤無名論》一卷



 僧慧皎《高僧傳》十四卷



 僧祐《弘明集》十四卷



 僧寶唱《比丘尼傳》五卷



 僧祐《釋迦譜》五卷



 甄鸞《笑道論》三卷



 僧慧可《達摩血脈論》一卷



 費長房《開皇歷代三寶記》十四卷



 又《開皇三寶錄總目》一卷



 《國清道場百錄》五卷僧灌頂纂,僧智顗修



 僧法琳《破邪論》
 三卷



 又《辨正論》八卷



 僧彥琮《釋法琳別傳》三卷



 僧慧能注《金壇經》一卷



 又撰《金剛經口訣》一卷



 僧慧昕注《壇經》二卷



 僧辨機《唐西域志》十二卷



 僧道宣《續高僧傳》三卷



 又《佛道論衡》三卷



 《三寶感應錄》三卷



 《釋迦氏譜》一卷



 《廣弘明集》三十卷



 僧政覺《金沙論》一卷



 僧神會《荷澤顯宗記》一卷



 《華嚴法界觀門》一卷僧法順集,僧宗密注



 僧宗密《禪源諸詮》二卷



 又《原人論》一卷



 《大乘起信論》一卷



 魏靜《永嘉一宿覺禪師集》一卷



 僧道世《法苑珠林》一百卷



 僧慧忠《十答問
 語錄》一卷



 《無住和尚說法》二卷僧鈍林集



 僧普願《語要》一卷



 《龐蘊語錄》一卷唐於□編



 僧神清《北山參元語錄》十卷



 僧慧海《屯悟入道要門論》一卷



 僧義凈《求法高僧傳》三卷



 僧元應《唐一切經音義》一十五卷



 僧澄觀《華嚴經疏》十卷



 僧紹修《語要》一卷



 裴休《傳心法要》一卷



 《唐六譯金剛經贊》一卷鄭覃等撰



 僧慧祥《古清涼傳》二卷



 《釋迦方志》一卷唐終南大一山僧撰



 僧應之《四注金剛經》一卷



 僧延壽《宗鏡錄》一百卷



 僧贊寧《僧史略》三卷



 僧道原《景德傳燈錄》三十卷



 晁
 迥《法藏碎金》十卷



 《道院集要》三卷不知作者



 僧延昭《眾吼集》一卷



 僧重顯《瀑布集》一卷



 又《語錄》八卷



 僧世沖《釋氏詠史詩》三卷



 僧居本《廣法門名義》一卷



 僧慧皎《僧史》二卷



 僧契嵩《輔教編》三卷



 僧省常《錢塘西湖凈社錄》三卷



 僧道誠《釋氏須知》三卷



 僧道誠《釋氏要覽》三卷



 王安石注《維摩詰經》三卷



 朱士挺《伏虎行狀》一卷



 《僧自嚴行狀》一卷陳嘉謨撰



 李之純《成都大悲寺集》二卷



 又《成都大慈寺記》二卷



 僧惟白《續燈錄》三十卷



 僧宗頤《勸孝文》二卷



 又《禪
 苑清規》十卷



 蹇序辰《諸經譯梵》三卷



 王敏中《勸善錄》六卷



 楊諤《水陸儀》二卷



 僧智達《祖門悟宗集》二卷



 樓穎《傳翕小錄要集》一卷



 僧宗永《宗門統要》十卷



 僧智圓《閑居編》五十一卷



 僧懷深注《般若波羅密多心經》一卷



 僧原白注《證道歌》一卷



 《僧宗杲語錄》五卷黃文昌撰



 僧慧達《夾科肇論》二卷



 僧應乾《楞嚴經標指要義》二卷



 僧靈操《釋氏蒙求》一卷



 僧馬鳴《釋摩訶衍論》十卷



 僧闍那多迦譯《羅漢頌》一卷



 僧菩提達磨《存想法》一卷



 又菩提達磨《胎息
 訣》一卷



 《頌證道歌》一卷篇首題正覺禪師撰



 《凈慧禪師語錄》一卷



 《蓮社十八賢行狀》一卷



 《法顯傳》一卷



 《諸經提要》二卷



 《五公符》一卷



 《寶林傳錄》一卷



 並不知作者



 李通玄《華嚴合論》一卷



 張戒注《楞伽集注》八卷



 佛陀多羅譯《圓覺經》二卷



 般刺密諦譯《楞嚴經》十卷



 《法寶標目》十卷王右編



 僧肇譯《維摩經》十卷



 晁迥《耄智餘書》三卷



 《八方珠玉集》四卷大圓、塗毒二僧集諸家禪語



 王日休《金剛經解》四十二卷



 《凈土文》十一卷王日休撰



 《語錄》二卷松源和尚講解答問



 《普燈錄》三十卷僧正受集



 《諸天
 傳》二卷僧行霆述



 《奏對錄》一卷佛照禪師淳熙間奏對之語



 《崇正辨》三卷胡寅撰



 右釋氏類二百二十二部,九百四十九卷



 劉向《列仙傳》三卷



 王褒《桐柏真人王君外傳》一卷



 周季通《玄洲上卿蘇君記》一卷



 葛洪《神仙傳》十卷



 《馬陰二君內傳》一卷



 《上真眾仙記》一卷



 《隱論雜訣》一卷



 《金木萬靈訣》一卷



 《抱樸子養生論》一卷



 《太清玉碑子》一卷葛洪與鄭惠遠問答



 《二女真詩》一卷紫微夫人及東華中侯王夫人作



 施真人《銘真論》一卷



 旌陽令許遜《靈劍子》一卷



 《黃帝內傳》一卷籛鏗得於石室



 東方朔《十洲三島記》一卷



 淮南
 王劉安《太陽真粹論》一卷



 黃玄鐘《蓬萊山西鰲還丹歌》一卷



 婁敬《草衣子還丹訣》一卷



 魏伯陽《還丹訣》一卷



 《周易門戶參同契》一卷



 《大丹九轉歌》一卷



 華佗《老子五禽六氣訣》一卷



 陸修靜《老子道德經雜說》一卷



 《五牙導引元精經》一卷



 《黃庭經》一卷其文初為五言四章,後皆七言,論人身扶養修治之理



 李千乘《黃庭中景經注》一卷



 尹喜《黃庭外景經注》一卷



 襄楷《太平經》一百七十卷



 李堅《東極謝真人傳》一卷



 王禹錫《海陵三仙傳》一卷



 施肩吾《真仙傳道集》二卷



 《三
 住銘》一卷



 《西山群仙會真記》一卷



 長孫滋《崔氏守一詩傳》一卷



 吳筠《神仙可學論》一卷



 又《形神可固論》一卷



 《著生論》一卷



 《明真辨偽論》一卷



 《心目論》一卷



 《玄門論》一卷



 《元綱論》一卷



 《諸家論優劣事》一卷



 《辨方正惑論》一卷



 杜光庭《二十四化詩》一卷



 又《二十四化圖》一卷



 《神仙感遇傳》十卷



 《墉城集仙錄》十卷



 《應現圖》三卷



 《仙傳拾遺》四十卷



 《歷代帝王崇道記》一卷



 《道教靈驗記》二十卷



 《道經降傳世授年載圖》一卷



 謝良嗣《中嶽吳天師內傳》一卷



 李渤《
 李天師傳》一卷



 《真系傳》一卷



 張隱居《演龍虎上經》二卷



 盧潘《侯真人傳》一卷



 沉汾《續仙傳》三卷



 尹文操《樓觀先師本行內傳》一卷



 《玄元聖記經》十卷



 刁琰《廣仙錄》一卷



 見素子《洞仙傳》十卷



 傅元鎮《應緣道傳》十一卷



 晞暘子《寶仙傳》三卷



 《南嶽夫人清虛玉君內傳》一卷



 範邈《南嶽魏夫人內傳》一卷



 李遵《三茅君內傳》一卷



 梁日廣《釋仙論》一卷



 赤松子《中誡篇》一卷



 《金石論》一卷



 《門天老歷》一卷



 冷然子《學神仙法》一卷



 賈嵩《陶先生傳序》三卷



 吳先
 主孫氏《太極左仙公神仙本起內傳》一卷



 華嶠《真人周君內傳》一卷



 《劉海蟾詩》一卷



 《太乙真君固命歌》一卷晉葛洪譯



 張融《三破論》一卷



 陶弘景《養性延命錄》二卷



 《導引養生圖》一卷



 《神仙玉芝瑞草圖》二卷



 《上清握中訣》三卷



 《登真隱訣》三十五卷



 《真誥》十卷



 華陽道士韋處玄注《老子西升經》二卷



 魏曇巒法師《服氣要訣》一卷



 陳處士同洪讓書《老子道經》一卷



 李淳風《正一五真圖》一卷



 孫思邈《退居志》一卷



 《真氣銘》一卷



 《九幽福壽論》一卷



 《龍虎亂日
 篇》一卷



 李用德《晉州羊角山慶歷觀記》一卷



 王元正《清虛子龍虎丹》一卷



 《驪山母黃帝陰符大丹經解》一卷房山長集



 吳兢《保聖長生纂要坐隅障》二卷



 僧一行《天真皇人九仙經》一卷



 尹愔《老子五廚經注》一卷



 周蒞《穎陽書》一卷



 昝商《導養方》三卷



 李廣《中指真訣》一卷



 僧遵化《養生胎息秘訣》一卷



 高駢《性箴金液頌》一卷



 黃仲山《玄珠龜鏡》三卷



 裴鉉《延壽赤書》一卷



 張果《紫靈丹砂表》一卷



 《內真妙用訣》一卷



 《休糧服氣法》一卷



 《大易志圖參同經》
 一卷玄宗與葉靜能、一行答問語



 王紳《太清宮簡要記》一卷



 康真人《氣訣》一卷



 盧遵元《太上肘後玉經方》一卷



 楊知玄《淮南王練聖法》一卷



 《老子元道經》一卷南統孟謫仙傳授



 李延章《中元論》一卷



 胡微《玉景內篇》二卷



 《黃庭內景五藏六腑圖》一卷大白山見素女子胡愔撰



 王懸河《三洞珠囊》三十卷



 王貞範《洞天集》二卷



 捷神子《唐元指玄篇》一卷



 《中央黃老君洞房內經》一卷



 《黃老中道君洞房內經》一卷



 《黃老神臨藥經》一卷



 《太清真人絡命訣》一卷



 《太上老君血脈論》一卷



 《靈寶服
 食五芝精》一卷



 《黃帝內經靈樞略》一卷



 《黃帝九鼎神丹經訣》十卷



 《黃帝內丹訣》一卷



 《太極真人風鳴爐火經》一卷



 《紫微帝君王經寶訣》一卷



 《太上老君服氣胎息訣》一卷



 《老子中經》二卷



 《老子神仙歷藏經》一卷



 《王母太上還童摘華法》一卷



 《紫微帝君紫庭秘訣》一卷



 《茅真君靜中吟》一卷



 《王茅君雜記》一卷



 《陰真君還丹歌》一卷



 《金液還丹歌》一卷



 《元君付道傳心法門》一卷



 《徐真君丹訣》一卷



 《張真君靈芝集》一卷



 《彭君訣黃白五元神丹經》一卷



 《太
 乙真元丹訣》一卷



 陳大素《九天飛步內訣真經》一卷



 河間真人劉演《金碧潛通秘訣》一卷



 大白山李真人《調元妙經》一卷



 陳少微《大洞煉真寶經》一卷



 申天師《服氣要訣》一卷



 張天師《石金記》一卷



 玄元先生《日月混元經》一卷



 鄭先生《不傳氣經》一卷



 建平然先生《少來苦樂傳》一卷



 赤城隱士《服藥經》三卷



 臥龍隱者《少玄胎息歌》一卷



 蜀郡處士《胎息訣》一卷



 成都李道士《太上洞玄靈寶修真論》一卷



 務元子《混成經》一卷



 務成子注《太上黃庭
 內景經》一卷



 含光子《契真刊謬玉鑰匙》一卷



 鄧云子《清虛真人裴君內傳》一卷



 廣成子《靈仙秘錄陰丹經》一卷



 《紫陽金碧經》一卷



 《升玄養生論》



 青霞子《旨道篇》一卷



 又《龍虎金液還丹通玄論》一卷



 《寶藏論》一卷



 易元子《勸道詩》一卷



 逍遙子《內指通玄訣》三卷



 《攝生秘旨》一卷



 升玄子《造化伏汞圖》一卷



 穎陽子《神仙修真秘訣》十二卷



 元陽子《金石還丹訣》一卷



 真一子《金鑰匙》一卷



 《九真中經》一卷赤松子傳



 暢元子《雜錄經訣尊用要事》一卷



 狐剛子《粉
 團》五卷



 左掌子《證道歌》一卷



 中皇子《服氣要訣》一卷



 桑榆子《新舊氣經》一卷



 玄明子柳沖用《巨勝歌》一卷



 葉真卿《玄中經》一卷



 丁少微《真一服元氣法》一卷



 洞元子通元子《通玄指真訣》一卷



 真常子《服食還丹證驗法》一卷



 煙蘿子《內真通玄歌》一卷



 獨孤滔《丹房鏡源文》三卷



 天臺白雲《服氣精義論》一卷



 徐懷遇《學道登真論》一卷



 曹聖圖《鉛汞五行圖》一卷



 張素居《金石靈臺記》一卷



 高先《大道金丹歌》一卷



 陳君舉《朝元子玉芝書》三卷



 呂洞賓《
 九真玉書》一卷



 陶植《蓬壺集》三卷


《修仙要訣》一卷
 \gezhu{
  華子期授於角里先生}



 《上相青童太上八術知慧滅魔神虎隱文》一卷



 碧巖張道者《中山玉櫃服神氣經》一卷



 《司世抱陽劍術》一卷



 金明七真人《三洞奉道科誡》三卷



 楊歸年《修真延秘集》三卷



 陰長生《三皇經》一卷



 馬明生《赤龍金虎中鉛煉七返還丹訣》卷亡



 上官翼《養生經》一卷



 王弁《新舊服氣法》一卷



 傅士安《還丹訣》一卷



 徐道邈注《老子西升經》二卷



 劉仁會注《西升經》一卷



 張隨《解參同契》一卷



 李審《頤
 神論》二卷



 處士劉詞《混俗頤生錄》一卷



 閭丘方遠《太上經秘旨》一卷



 道士張乾森《自然券立成儀》一卷



 張承先《度靈寶經表具事》一卷



 《玉晨奔日月圖》一卷



 《真秘訣》一卷寶冠授達磨



 僧玄玄《疑甄正論》三卷



 王長生《紫微內庭秘訣》三卷



 《傳授五法立成儀》一卷



 寒山子《大還心鑒》一卷



 守文居絲茲《長生纂要》一卷



 《莊周氣訣》一卷



 《朗然子詩》一卷



 山居道士《佩服經符儀》一卷不知名



 蘇登《天老神光經》一卷



 《內外丹訣》二卷集王元正、李黃中等撰



 《崔公入藥鏡》三卷



 《混
 元內外觀》十卷



 張君房《雲笈七簽》百二十卷



 樂史《總仙秘錄》一百三十卷



 餘卞《十二真君傳》二卷



 李信之《雲臺異境集》一卷



 賈善翔《高道傳》十卷



 《猶龍傳》三卷



 張隱龍《三茅山記》一卷



 王松年《仙苑編珠》一卷



 李昌齡《感應篇》一卷



 朱宋卿《徐神翁語錄》一卷



 《太宗真宗三朝傳授贊詠儀》二卷



 真宗《汴水發願文》一卷



 徽宗《天真示現記》三卷



 陳摶《九室指玄篇》一卷



 王欽若《七元圖》一卷



 《先天紀》三十六卷



 《翊聖保德傳》三卷



 丁謂《降聖記》三十卷



 耿肱《
 養生真訣》一卷



 青霞子《丹臺新錄》九卷



 李思聰《道門三界詠》三卷



 張端《金液還丹悟真篇》一卷



 彭曉《周易參同契分章通真儀》三卷



 《參同契明鑒訣》一卷



 姚稱《攝生月令圖》一卷



 錢景衎《南嶽勝概編》一卷



 謝修通《玉笥山祖記實錄》一卷



 張無夢《還元篇》一卷



 《純陽集》一卷



 《上清五牙真秘訣》一卷



 《二仙傳》一卷



 《成仙君傳》一卷



 《劉真人傳》一卷



 《平都山仙都觀記》二卷



 《師譜》一卷



 《十真記》一卷



 《仙班朝會圖》五卷



 《賴卿記》一卷



 《大還丹照鑒登仙集》一卷



 《
 斷穀要法》一卷



 《裴君傳行事訣》一卷



 《太上墨子枕中記》二卷



 《太上太素玉錄》一卷



 《太上倉元上錄》一卷



 《學仙辨真訣》一卷



 《洞真金元八景玉錄》一卷



 《五岳真形圖》一卷



 《祭六丁神法》一卷



 《神仙雜歌詩》一卷



 《玄門大論》一卷



 《九轉丹歌》一卷



 《太和樓觀內紀本草記》一卷



 《老君出塞記》一卷



 《五岳真形論》一卷



 《黃帝三陽經五明乾羸坤巴訣》一卷



 《正一肘後修用訣》一卷



 《正一法文目》一卷



 《正一論》一卷



 《正一上元九星圖》一卷



 《正一修行指要》三卷



 《正一
 法十菉召儀》一卷



 《正一奏章儀》一卷



 《正一醮江海龍王神儀都功版儀》一卷



 《太上符鏡》一卷



 《穀神賦》一卷



 《黃書過度儀》一卷



 《太上八道命籍》二卷



 《靈寶聖真品位》一卷



 《靈寶飛雲天篆》一卷



 《上清佩文訣》五卷



 《上清佩文黑券訣》一卷



 《福地記》一卷



 《曲素憂樂慧辭》一卷



 《皇人三一圖》一卷



 《西升記》一卷



 《胎精記解結行事訣》一卷



 《高上金真元菉》一卷



 《長睡法》一卷



 《大洞玄保真養生論》一卷



 《曲素訣辭》一卷



 《太上丹字紫書》一卷



 《絕玄金章》一卷



 《紫鳳赤
 書》一卷



 《靈寶步虛詞》一卷



 《金紐太清陰陽戒文》一卷



 《太上紫書錄傳》一卷



 《度太一玉傳儀》一卷



 《奔日月二景隱文》一卷



 《司命楊君傳記》一卷



 《回耀太真隱書》一卷



 《思道誡》一卷



 《潘尊師傳》一卷



 《三尸經》一卷



 《金簡集》三卷



 《無名道者歌》一卷



 《大丹會明論》一卷



 《太清真人九丹神秘經》一卷



 《金鏡九真玉書》卷《八公紫府河車歌》一卷



 《大還秘經》一卷



 《神仙肘後三宮訣》二卷



 《太極紫微元君補命秘錄》一卷



 《老君八純玄鼎經》一卷



 《海蟾子還金篇》一卷



 《
 太清篇火式》一卷



 《太一真人五行重玄論》一卷



 《龍虎大還丹秘訣》一卷



 《煉五神丹法》一卷



 《太清丹經經》一卷



 《神仙庚辛經》一卷



 《紫白金丹訣》一卷



 《仙公藥要訣》一卷



 《三十六水法》一卷



 《金虎赤龍經》一卷



 《玉清內書》一卷



 《太上老子服氣口訣》一卷



 《燒煉雜訣法》一卷



 《太清金液神丹經》三卷



 《休糧諸方》一卷



 《胎息根旨要訣》一卷



 《修真內煉秘訣》一卷



 《上清修行訣》一卷



 《大道感應論》一卷



 《太上習仙經契錄》一卷



 《回耀飛光日月精氣上經》一卷



 《攝生增
 益錄》一卷



 《神氣養形論》一卷



 《服餌仙方》一卷



 《鉛汞指真訣》一卷



 《服氣日月皇華訣》一卷



 《神仙藥名隱訣》一卷



 《煉花露仙醹訣》一卷



 《繕生集》一卷



 《道術旨歸》一卷



 《按摩要法》一卷



 《醮人神法》一卷



 《上清大洞真經玉訣》一卷



 《草金丹法》一卷



 《十二月五藏導引》一卷



 《大易二十四篇》一卷



 《服氣煉神秘訣》一卷



 《老君金書內序》一卷



 《尹真人本行記》一卷



 《陶陸問答》一卷



 《諸家修行纂要》一卷



 《穀神秘訣》三卷



 《太清導引調氣經》一卷



 《大玄部道興論》二十七卷



 《
 富貴日用篇》一卷



 《入室思赤子經》一卷



 《餌芝草黃精經》一卷



 《治身服氣訣》一卷



 《玉皇聖臺神用訣》一卷



 《燒金石藥法》一卷



 《神仙服食經》一卷



 《三天君烈紀》一卷



 《養生要錄》三卷



 《神仙九化經》一卷



 《調元氣法》一卷



 《太上保真養生論》一卷



 《神仙秘訣三論》三卷



 《元君肘後術》三卷



 《山水穴竇圖》一卷



 《養生諸神仙方》一卷



 《五經題迷》一卷



 右神仙類三百九十四部,一千二百十六卷



 右道家附釋氏神仙類凡七百十七部,二千五百二十
 四卷。



 《管子》二十四卷齊管夷吾撰。



 《商子》五卷衛公孫鞅撰



 《慎子》一卷慎到撰



 《韓子》二十卷韓非撰



 尹知章注《管子》十九卷



 杜祐《管氏指略》二卷



 丁度《管子要略》五篇卷亡



 董仲舒《春秋決事一作「獄」》十卷丁氏平,黃氏正



 李文博《治道集》十卷



 張去華《大政要錄》三卷



 右法家類十部,九十九卷。



 《公孫龍子》一卷趙人



 《尹文子》一卷齊人



 《鄧析子》二卷鄭人



 劉邵《人物志》二卷



 杜周士《廣人物志》二卷



 右名家類五部,八卷。



 《墨子》十五卷宋墨翟撰



 右墨家類一部,十五卷。



 《鬼谷子》三卷



 高誘注《戰國策》三十三卷



 鮑彪注《國策》十卷



 右縱橫家類三部,四十六卷。



 《夏小正戴氏傳》四卷傅崧卿注



 蔡邕《月令章句》一卷



 杜臺卿《玉燭寶典》十二卷



 唐玄宗《刪定禮記月令》一卷



 李林甫《
 批注月令》一卷



 韓鄂《歲華紀麗》四卷



 韋行規《月錄》一卷



 李綽《秦中歲時記》一卷一名《咸鎬記》



 李邕《金穀園記》一卷



 徐鍇《歲時廣記》一百二十卷內八卷闕



 賈昌朝《國朝時令集解》十二卷



 宋綬《歲時雜詠》二十卷



 劉安靖《時鏡新書》五卷



 孫厔《備閱注時令》一卷



 《歲中記》一卷



 《十二月纂要》一卷



 《保生月錄》二卷



 《四時錄》四卷



 並不知作者



 張方《夏時志別錄》一卷



 又《夏時考異》一卷



 《許狀元節序故事》十二卷許尚編



 真宗《授時要錄》十二卷



 孫思邈《齊人月令》三卷



 宗懍《荊
 楚歲時記》一卷



 李綽《輦下歲時記》一卷



 劉靖《時鑒雜一作「新」



 書》四卷



 岑賁《月壁》一卷



 孫翰《月鑒》二卷



 嵇含《南方草木狀》三卷



 賈思勰《齊民要術》十卷



 則天皇后《兆人本業》三卷



 陸羽《茶經》三卷



 又《茶記》一卷



 溫庭筠《採茶錄》一卷



 《茶苑雜錄》一卷不知作者



 張又新《煎茶水記》一卷



 韓鄂《四時纂要》十卷



 賈耽《醫牛經》卷亡



 淮南王《養蠶經》一卷



 孫光憲《蠶書》三卷



 秦處度《蠶書》一卷



 毛文錫《茶譜》一卷



 史正志《菊譜》一卷



 任□《彭門花譜》一卷



 周序《洛陽花木記》一卷



 陶朱公《養魚經》一卷



 熊寅亮《農子》一卷



 賈樸《牛書》一卷



 王旻《山居要術》三卷



 又《在居雜要》三卷



 《山居種蒔要術》一卷



 戴凱之《竹譜》三卷



 無求子《酒經》一卷不知姓名



 大隱翁《酒經》一卷



 《是齋售用》一卷



 李淳風《四民福祿論》二卷



 《牛皇經》一卷



 《辨五音牛欄法》一卷



 《農家切要》一卷



 《荔枝故事》一卷



 並不知作者



 封演《錢譜》一卷



 張臺《錢錄》一卷



 於公甫《古今泉貨圖》一卷



 侯氏《萱堂香譜》一卷



 範如圭《田夫書》一卷



 賈元道《大農孝經》一卷



 陳靖《勸農奏議》三十篇



 林
 勛《本政書》十卷



 又《本政書比校》二卷



 《治地旁通》一卷



 王章《水利編》三卷



 僧贊寧《筍譜》一卷



 僧仲休《花品記》一卷



 丁謂《北苑茶錄》三卷



 又《天香傳》一卷



 歐陽修《牡丹譜》一卷



 蔡襄《茶錄》一卷



 沉立《香譜》一卷



 又《錦譜》一卷



 《茶法易覽》十卷



 丁度《土牛經》一卷



 孔武仲《芍藥譜》一卷



 張峋《花譜》一卷



 沉括《志懷錄》三卷



 竇蘋《酒譜》一卷



 馮安世《林泉備》五卷



 呂惠卿《建安茶用記》二卷



 劉分文《芍藥譜》一卷



 王觀《芍藥譜》一卷



 洪芻《香譜》五卷



 章炳文《壑源茶錄》一卷



 吳良輔《竹譜》二卷



 葛澧《酒譜》一卷



 高伸《食禁經》三卷



 劉異《北苑拾遺》一卷



 宋子安《東溪茶錄》一卷



 陳翥《桐譜》一卷



 張宗誨《花木錄》七卷



 周絳《補茶經》一卷



 葉庭珪《南蕃香錄》一卷



 樓□《耕織圖》一卷



 曾安止《禾譜》五卷



 曾之謹《農器譜》三卷



 陳敷《農書》三卷



 熊蕃《宣和北苑貢茶錄》一卷



 韓彥直《永嘉橘錄》三卷



 王居安《經界弓量法》一卷



 右農家類一百七部,四百二十三卷、篇。



 《鬻熊子》一卷



 呂不韋《呂氏春秋》二十六卷高誘注



 陸賈《新
 語》二卷



 賈誼《新書》十卷



 《淮南子鴻烈解》二十一卷淮南王安撰



 許慎注《淮南子》二十一卷



 高誘注《淮南子》十三卷



 劉向《新序》十卷



 又《說苑》二十卷



 仲長統《昌言》二卷



 王充《論衡》三十卷



 邊誼《續論衡》二十卷



 應劭《風俗通義》十卷



 徐乾《中論》十卷



 《蔣子萬機論》十卷魏蔣濟撰



 諸葛亮《武侯十六條》一卷



 沉顏《聱書》十卷



 《傅子》五卷傅玄撰



 陸機《正訓》十卷



 崔豹《古今注》三卷



 周蒙《續古今注》三卷



 張華《博物志》十卷



 葛洪《抱樸子內篇》二十卷



 又《抱樸子外篇》五十卷



 《劉
 子》三卷劉晝撰



 奚克讓《劉子音釋》三卷



 又《音義》三卷



 湘東王繹《金樓子》十卷



 庾仲容《子鈔》三十卷



 顧野王《符瑞圖》二卷



 《孫綽子》十卷



 範泰《古今善言》三十卷



 沉約《袖中記》三卷



 《尹子五機論》三卷



 商子逸《商子新書》三卷



 鄭瑋《道言錄》三卷



 杜正倫《百行章》一卷



 李文博《治道集》十卷



 虞世南《帝王略論》五卷



 劉嚴《芻蕘論》三卷



 李賢《修書要覽》十卷



 羅隱《兩同書》二卷



 李直方《正性論》一卷



 韓熙載《格言》五卷



 又《格言後述》三卷



 黃晞《聱隅書》十卷



 李淳風《感
 應經》三卷



 魏征《時務策》一卷



 又《祥瑞錄》十卷



 朱敬則《十代興亡論》十卷



 張說《才命論》一卷



 楊相如《君臣政要論》三卷



 趙自勉《造化權輿》六卷



 《元子》十卷元結撰



 杜祐《理道要訣》十卷



 皇甫選注《何亮本書》三卷



 邵元《體論》十卷



 馬總《意林》三卷



 又《意樞》二十卷



 林慎思《伸蒙子》三卷



 丘光庭《規書》一卷



 又《兼明書》十二卷



 牛希濟《理源》二卷



 又《治書》十卷



 朱樸《致理書》十卷



 盧藏用《子書要略》三卷



 臧嘉猷《史玄機論》十卷



 歐陽浚《周紀聖賢故實》十卷



 徐融《帝
 王指要》三卷



 張輔《宰輔明鑒》十卷



 趙湘《補政忠言》十篇卷亡



 徐氏《忠列圖》一卷



 《孝義圖》一卷



 趙彥衛《雲麓漫鈔》二十卷



 又《雲麓續鈔》二卷



 南唐後主李煜《雜說》二卷



 《劉子法語》二十卷劉鶚撰



 又《通論》五卷



 宋齊丘《化書》六卷



 又《理訓》十卷



 葛澧《經史摭微》四卷



 劉賡《稽瑞》一卷



 趙蕤《長短要術》九卷



 吳筠《兩同書》二卷



 馬縞《中華古今注》三卷



 蘇鶚《演義》十卷



 樂朋龜《五書》一卷



 征微子《服飾變古》一卷



 狐剛子《感應類從譜》一卷



 通幽子《靈臺隱秘寶符》一
 卷扶風隱者



 李恂《前言往行錄》三卷



 《尹子》五卷



 鄭至道《諭俗編》一卷



 彭仲剛《諭俗續編》一卷



 黃嚴《虙犧範圍圖傳》二卷



 張時舉《弟子職女誡鄉約家儀鄉儀》一卷



 李宗思《尊幼儀訓》一卷



 呂本中《官箴》一卷



 何薳《春渚記聞》十三卷



 王普《答問難疑》一卷



 徐度《崇道卻掃編》十三卷



 吳曾《漫錄》十三卷



 魏泰《書可記》一卷



 又《續東軒雜錄》一卷



 馮忠恕《涪陵記》一卷



 洪興祖《聖賢眼目》一卷



 又《語林》五卷



 姚寬《叢語》上下二卷



 唐稷《硯岡筆志》一卷



 吳箕《常譚》二卷



 袁採《世範》三卷



 又《欷歔子》一卷



 葉適《習學記言》四十五卷



 項安世《項氏家記》十卷



 徐彭年《涉世錄》二十五卷



 又《涉世後錄》二十五卷



 《坐忘論》一卷



 呂祖謙《紫微語錄》一卷



 葉模《石林過庭錄》三十七卷



 李石《樂善錄》十卷



 劉鵬《縣務綱目》二十卷



 周樸《三教辨道論》一卷



 僧贊寧《物類相感志》十卷



 又《要言》二卷



 柳寀《藪記》十卷



 王錡《動書》一卷



 宋祁《筆錄》一卷



 龍昌期《天保正名論》八卷



 胥餘慶《瑞應雜錄》十卷



 刁衎《治道中術》三卷



 朱景先《默書》三卷



 鄧綰《
 馭臣鑒古論》二十卷



 王韶《敷陽子》七卷



 《天鬻子》一卷不知姓名



 吳宏《群公典刑》二十卷



 高承《事物紀原》十卷



 陳瓘《中說》一卷



 孔平仲《良史事證》一卷



 李新《塾訓》十三卷



 又《欲書》五卷



 李格非《史傳辨志》五卷



 晁說之《客語》一卷



 方行可《治本書》一卷



 王揚英《黼扆誡》一卷



 何伯熊《機密利害一》卷



 李皜《審理書》一卷



 張大楫《翠微洞隱》百八十卷



 李易《要論》一卷



 何亮《本書》三卷



 劉長源《治本論》一卷



 鄭樵《十說》二卷



 潘植《忘筌書》二卷



 洪氏《雜家》五卷不知名



 《瑞錄》
 十卷



 《冗錄》一卷



 《治獄須和》一卷



 《之官申戒》一卷



 《瑞應圖》十卷



 《玉泉子》一卷



 《中興書》一卷



 《汲世論》一卷



 並不知作者



 《東筦子》十卷



 《李子正辨》十卷



 劉潛《群書集》三卷



 成嵩《韻史》一卷



 陳鄂《十經韻對》二十卷



 又《四庫韻對》九十九卷



 魏玄成《祥應一作「瑞」圖》十卷



 劉振《通籍錄異》二十卷



 趙志忠《大遼事跡》十卷



 右雜家類一百六十八部,一千五百二十三卷、篇



\end{pinyinscope}