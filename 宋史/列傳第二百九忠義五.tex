\article{列傳第二百九忠義五}

\begin{pinyinscope}

 陳元桂張順張貴範天順牛富邊居誼陳炤王安節尹玉李芾尹谷楊霆
 趙卯發唐震趙與BT趙孟錦方洪趙淮



 陳元桂,撫州人。淳祐四年進士。累官知臨江軍。時聞警報,築城備御,以焦心勞思致疾。開慶元年春,北兵至臨江,時制置使徐敏子在隆興,頓兵不進。元桂力疾登城,坐北門亭上督戰,矢石如雨,力不能敵。吏卒勸之避去,不從。有以門廊鼓翼蔽之者,麾之使去。有欲抱而走者,元桂曰:「死不可去此。」左右走遁。師至,元桂瞠目叱罵,遂
 死之。縣其首於敵樓,越四日方斂,體色如生。



 初,親戚有勸其移治者,元桂曰:「子亦為浮議所搖耶?時事如此,與其死於饑饉,死於疾病,死於盜賊,孰若死於守土之為光明俊偉哉?」家人或請登舟,不許,且戒之曰:「守臣家屬豈可先動,以搖民心。」敏子以聞,贈寶章閣待制,賜緡錢十萬,與一子京官、一子選人恩澤,立廟北門,謚曰正節。



 張順,民兵部將也。襄陽受圍五年,宋闖知其西北一水曰清泥河,源於均、房,即其地造輕舟百艘,以三舟聯為
 一舫,中一舟裝載,左右舟則虛其底而掩覆之。出重賞募死士,得三千。求將,得順與張貴,俗呼順曰「矮張」,貴曰「竹園張」,俱智勇,素為諸將所服,俾為都統。出令曰:「此行有死而已,汝輩或非本心,宜亟去,毋敗吾事。」人人感奮。



 漢水方生,發舟百艘,稍進團山下。越二日,進高頭港口,結方陳,各船置火槍、火炮、熾炭、巨斧、勁弩。夜漏下三刻,起矴出江,以紅燈為識。貴先登,順殿之,乘風破浪,徑犯重圍。至磨洪灘以上,北軍舟師布滿江面,無隙可入。眾乘
 銳凡斷鐵絙攢杙數百,轉戰百二十里,黎明抵襄城下。城中久絕援,聞救至,踴躍氣百倍。及收軍,獨失順。越數日,有浮尸溯流而上,被介胄,執弓矢,直抵浮梁,視之順也,身中四槍六箭,怒氣勃勃如生。諸軍驚以為神,結塚斂葬,立廟祀之。



 張貴既抵襄,襄帥呂文煥力留共守。貴恃其驍勇,欲還郢,乃募二士能伏水中數日不食,使持蠟書赴郢求援。北兵增守益密,水路連鎖數十里,列撒星樁,雖魚蝦不
 得度。二人遇樁即鋸斷之,竟達郢,還報,許發兵五千駐龍尾洲以助夾擊。



 刻日既定,乃別文煥東下,點視所部軍,洎登舟,帳前一人亡去,乃有過被撻者。貴驚曰:「吾事洩矣,亟行,彼或未及知。」復不能銜枚隱跡,乃舉炮鼓噪發舟,乘夜順流斷絙破圍冒進,眾皆闢易。既出險地,夜半天黑,至小新城,大兵邀擊,以死拒戰。沿岸束荻列炬,火光燭天如白晝。至勾林灘,漸近龍尾洲,遙望軍船旗幟紛披,貴軍喜躍,舉流星火示之。軍船見火即前迎,及
 勢近欲合,則來舟皆北兵也。蓋郢兵前二日以風水驚疑,退屯三十里,而大兵得逃卒之報,據龍尾洲以逸待勞。貴戰已困,出於不意,殺傷殆盡,身被數十槍,力不支見執,卒不屈,死之。乃命降卒四人舁尸至襄,令於城下曰:「識矮張乎?此是也。」守陴者皆哭,城中喪氣。文煥斬四卒,以貴祔葬順塚,立雙廟祀之。



 範天順,荊湖都統也。襄陽受圍,天順日夕守戰尤力。及呂文煥出降,天順仰天嘆曰:「生為宋臣,死當為宋鬼。」即
 所守處縊死。贈定江軍承宣使,制曰:「賀蘭擁兵,坐視睢陽之失;李陵失節,重為隴士之羞。今有人焉,得其死所,可無褒恤,以示寵綏?範天順功烈雖卑,忠義莫奪,自均、房泛舟之役克濟于艱,而襄、樊坐甲之師益堅所守。俄州刺史為降將軍,爾乃不屈自經,可謂見危致命。」封其妻宜人,官其二子,仍賜白金五百兩,田五百畝。



 牛富,霍丘人。制置司游擊砦兵籍。勇而知義。為侍衛馬軍司統制。戍襄陽五年,移守樊城,累戰不為衄,且數射
 書襄陽城中遺呂文煥,相與固守為唇齒。兩城凡六年不拔,富力居多。城破,富率死士百人巷戰,死傷不可計,渴飲血水。轉戰前,遇民居燒絕街道,身被重傷,以頭觸柱赴火死。贈靜江軍節度使,謚忠烈,賜廟建康。



 裨將王福見富死,嘆曰:「將軍死國事,吾豈宜獨生!」亦赴火死。



 邊居誼,隨人也。初事李庭芝,積戰功至都統制。咸淳十年,以京湖制置帳前都統守新城。居誼善御下,得士心,凡戰守之具,治之皆有法。



 大兵至沙陽,守將王大用不
 降,麾兵攻城,破之,執大用。呂文煥至新城,意其小壘可不攻而破,居誼率舟師拒之,文煥列沙陽所斬首招降,不從。明日,縛大用至壁下,使呼曰:「邊都統急降,不然禍即至矣。」居誼不答。又射榜檄入壁中,居誼曰:「吾欲與呂參政語耳。」文煥聞之,以為居誼降己也,馳馬至,伏弩亂發,中文煥者三,並中其馬,馬僕,幾鉤得之,眾挾文煥以他馬奔走。越二日,總制黃順挾一人開東門走出降。明日,使順來招之,居誼曰:「若欲得新城邪?吾誓以死守此,何可得
 也。」順又呼其部曲,部曲欲縋城出,居誼悉驅以入,當門斬之。文煥乃麾兵攻城,以火具卻之,旋蟻附而上。居誼乃取其家金盡散將士,往來督戰。會暮,破侵漢樓,樓火延毀民居,居誼度力不支,走還第,拔劍自殺,不殊,赴火死。丞相伯顏壯其勇,購得其尸燼中,觀之。事聞,贈利州觀察使,立廟死所。



 陳炤,字光伯,常州人。少工詞賦,登第,為丹徒縣尉,歷兩淮制置司參議官、大軍倉曹壽春府教授,復入帥幕,改
 知朐山縣,仍兼主管機宜文字。尋丁母憂歸。



 北兵至常,常守趙與鑒走匿,郡人錢訔以城降。淮民王通居常州,陰以書約劉師勇,許為內應。朝議乃以姚希得子訔知常州。師通肋常州,走錢訔,執安撫戴之泰等,遂迎訔以入。訔以照久任邊知兵,闢為通判。或謂照曰:「今闢難有辭矣。」照曰:「鄉邦淪沒,何可坐視,與其偷生而茍全,不若死之愈也。」遂墨衰而出。凡可以備御者,無不為之。



 訔入常甫十餘日,大軍攻常,照等率義兵戰御,自夏徂冬不能下。
 以功加帶行提轄文思院。常將張彥攻呂城,兵敗而降,因盡言常城中虛實,遂急攻之。照等晝夜城守,招之不下。丞相伯顏自將圍其城,照與訔持以忠義,協力固守。再加訔太府寺丞,照乾辦諸軍糧料院,常將士皆轉五官。城益急,常兵阻壕水為陳,矢盡亦不降。城破,訔死之,照猶斂兵巷戰,家人請曰:「城東北門圍未合,可走常熟入臨安也。」照曰:「去此一步,非死所矣。」日中兵至,死焉。事上,追贈訔龍圖閣待制,希得贈太師,照直寶章閣,並官
 其子。



 王安節,節度使堅之子也。少從其父守合州有功,安節等兄弟五人皆受官。堅為賈似道所忌,出知和州,鬱鬱而死。



 安節至咸淳末為東南第七副將。德祐初,似道潰師蕪湖,列城皆降,不降者亦棄城遁。時安節駐兵江陵,即走臨安,上疏乞募兵為捍禦,授閣門祗候、浙西添差兵馬副都監。收兵入平江,合張世傑兵戰鳳皇港,有功,轉三官。



 劉師勇復常州,攻走王良臣,師勇還平江,以安
 節與張詹守常。已而良臣導大兵攻常,常城素惡,安節等築柵以守,相拒兩月不下。大元丞相伯顏自將攻之,屢遣使招降,亦不下。丞相怒,麾兵破其南門,安節揮雙刀率死士巷戰,臂傷被執。有求其姓名者,安節呼曰:「我王堅子安節也。」降之不得,乃殺之。



 尹玉,寧都人。以捕盜功為贛州三砦巡檢。秩滿城居,從文天祥勤王。及天祥至平江,調玉同淮將張全、廣將朱華拒大兵,戰於伍牧,全等軍敗,以淮、廣軍先遁,曾全、胡
 遇、謝榮、曾玉以贛州四指揮軍亦遁,唯玉殘軍五百殊死戰。玉手殺數十人,箭集於胄如蝟毛,援絕力屈,遂被執。大軍橫四槍於其項,以梃擊之死。餘兵猶夜戰,殺人馬蔽田間,無一降者。質明,生還者四人。贈玉濠州圍練使,官其二子,賜田二頃,以恤其家。



 李芾,字叔章,其先廣平人,中徙汴。高祖升起進士,為吏有廉名。靖康中,金人破汴,以刃迫其父,升前捍之,與父俱死。曾祖椿徙家衡州,遂為衡人。



 芾生而聰警,少自樹
 立,名其齋曰無暴棄。魏了翁一見禮之,謂有祖風,易其名曰肯齋。初以蔭補南安司戶,闢祁陽尉,出振荒,即有聲。攝祁陽縣,縣大治,闢湖南安撫司幕官。時盜起永州,招之,歲餘不下。芾與參議鄧坰提千三百人破其巢,禽賊魁蔣時選父子以歸,餘黨遂平。攝湘潭縣,縣多大家,前令束手不敢犯。芾稽籍出賦,不避貴勢,賦役大均。



 入朝,差知德清縣。屬浙西饑,芾置保伍振民,活數萬計。遷主管酒庫所。德清有妖人扇民為亂,民蜂起附之,至數
 萬人。遣芾討之,盜聞其來,眾立散歸。除司農寺丞,歷知永州,有惠政,永人祠之。以浙東提刑知溫州。州瀕海多盜,芾至盜息,遂以前官移浙西。時浙西亦多盜,群穴太湖中,芾跡得其出沒按捕之,盜亦駭散。作虎丘書院以祠尹焞,置學官,親為學規以教之,學者甚盛。



 咸淳元年,入知臨安府。時賈似道當國,前尹事無巨細先關白始行,芾獨無所問。福王府有迫人死者,似道力為營救,芾以書往復辨論,竟置諸法。嘗出閱火具,民有不為具者,
 問之,曰:「似道家人也。」立杖之。似道大怒,使臺臣黃萬石誣以贓罪,罷之。



 大軍取鄂州,始起為湖南提刑。時郡縣盜擾,民多奔竄,芾令所部發民兵自衛,縣予一皂幟,令曰:「作亂者斬幟下。」民始帖然。乃號召發兵,擇壯士三千人,使土豪尹奮忠將之勤王,別召民兵集衡為守備。未幾,似道兵潰蕪湖,乃復芾官,知潭州兼湖南安撫使。時湖北州郡皆已歸附,其友勸芾勿行,曰:「無已,即以身行可也。」芾泣曰:「吾豈昧於謀身哉?第以世受國恩,雖廢棄
 中猶思所以報者,今幸用我,我以家許國矣。」時其所愛女死,一慟而行。



 德祐元年七月,至潭,潭兵調且盡,游騎已入湘陰、益陽諸縣。倉卒召募不滿三千人,乃結溪峒蠻為聲援,繕器械,峙芻糧,柵江修壁,命劉孝忠統諸軍。吳繼明自湖北至,陳義、陳元自戍蜀歸,芾奏請留之戍潭,推誠任之,皆得其死力。



 大元右丞阿里海牙既下江陵,分軍戍常德遏諸蠻,而以大兵入潭。芾遣其將於興帥兵御之於湘陰,興戰死。九月,再調繼明出御,兵不及
 出,而大軍已圍城。芾慷慨登陴,與諸將分地而守,民老弱亦皆出,結保伍助之,不令而集。十月,兵攻西壁,孝忠輩奮戰,芾親冒矢石以督之。城中矢盡,有故矢皆羽敗,芾命括民間羽扇,羽立具。又苦食無鹽,芾取庫中積鹽席,焚取鹽給之。有中傷者,躬自撫勞,日以忠義勉其將士。死傷相藉,人猶飲血乘城殊死戰。有來招降者,芾殺之以徇。



 十二月,城圍益急,孝忠中炮,風不能起,諸將泣請曰:「事急矣,吾屬為國死可也,如民何?」芾罵曰:「國家平
 時所以厚養汝者,為今日也。汝第死守,有後言者吾先戮汝。」除夕,大兵登城,戰少卻,旋蟻附而登,衡守尹谷及其家人自焚,芾命酒酹之。因留賓佐會飲,夜傳令,猶手書「盡忠」字為號。飲達旦,諸賓佐出,參議楊震赴園池死。芾坐熊湘閣召帳下沈忠遺之金曰:「吾力竭,分當死,吾家人亦不可辱於俘,汝盡殺之,而後殺我。」忠伏地扣頭,辭以不能,芾固命之,忠泣而諾,取酒飲其家人盡醉,乃遍刃之。芾亦引頸受刃。忠縱火焚其居,還家殺其妻子,復
 至火所,大慟,舉身投地,乃自刎。幕屬茶陵顧應焱、安仁陳億孫皆死。潭民聞之,多舉家自盡,城無虛井,縊林木者累累相比。繼明等以城降,陳毅潰圍,將奔閩,中道戰死。事聞,贈端明殿大學士,謚忠節。芾初至潭,遣其子裕孫出,曰:「存汝以奉祀也。」其孫輔叔時亦親迎於溫,皆得不死。二王悉詔入閩官之。



 芾為人剛介,不畏強御,臨事精敏,奸猾不能欺。且強力過人,自旦治事至暮無倦色,夜率至三鼓始休,五鼓復起視事。望之凜然猶神明,而
 好賢禮士,即之溫然,雖一藝小善亦惓惓獎薦之。平生居官廉,及擯斥,家無餘貲。



 尹谷,字耕叟,潭州長沙人。性剛直莊厲,初處郡學,士友皆嚴憚之。



 宋以詞賦取士,季年,惟閩、浙賦擅四方。穀與同郡刑天榮、董景舒、歐陽逢泰諸人為賦,體裁務為典雅,每一篇出,士爭學之,由是湘賦與閩、浙頡頏。中年登進士第。調常德推官,知崇陽縣,所至廉正有聲。



 丁內艱,居家教授,不改儒素。日未出,授諸生經及朱氏《四書》,士
 雖有才思而不謹飭者擯不齒。諸生隆暑必盛服,端居終日,夜滅燭始免巾幘,早作必冠而後出帷。行市中,市人見其舉動有禮,相謂曰:「是必尹先生門人也。」詰人果然。



 晚入李庭芝制幕,用薦擢知衡州,需次於家。潭城受兵,帥臣李芾禮以為參謀,共畫備御策。時城中壯士皆入衛臨安,所餘軍僅四百五十人,老弱太半。芾糾率民丁,獎勵以義,人殊死戰,三月城不下。大軍斷絕險要,援兵不至,穀知城危,與妻子訣曰:「吾以寒儒受國恩,典方
 州,誼不可屈,若輩必當從吾已耳。」召弟岳秀使出,以存尹氏祀,嶽秀泣而許之死。乃積薪扃戶,朝服望闕拜已,先取歷官告身焚之,即縱火自焚。鄰家救之,火熾不可前,但於烈焰中遙見穀正冠端笏危坐,闔門無少長皆死焉。芾聞之,命酒酹穀曰:「尹務實,男子也,先我就義矣。」務實,穀號也。



 初,潭士以居學肄業為重,州學生月試積分高等,升湘西嶽麓書院生,又積分高等,升嶽麓精舍生,潭人號為「三學生」。兵興時,三學生聚居州學,猶不廢業。
 穀死,諸生數百人往哭之,城破,多感激死義者。



 楊霆,字震仲。少有志節。以世澤奏補將仕郎,銓試第一,授修職郎、桂嶺主簿,有能聲。又五中漕舉,改鄂州教授,遷復州司理參軍,轉常、澧觀察推官,擢知監利縣。縣有疑獄,歷年不決,霆未上,微服廉得其實,立決之,人稱神明。



 闢荊湖制置司干官。呂文德為帥,素慢侮士,常試以難事,霆倉卒立辦,皆合其意。一日謂曰:「朝廷有密旨,出師策應淮東,誰可往者?」即對曰某將可。又曰:「兵器糧草若
 何?」即對曰某營兵馬、某庫器甲、某處矢石、某處芻糧,口占授吏,頃刻案成。文德大驚曰:「吾平生輕文人,以其不事事也。公材幹如此,何官不可為,吾何敢不敬。」密薦諸朝,除通判江陵府。



 江陵大府,雄據上流,表裏襄、漢,西控巴蜀,南扼湖、廣。兵民雜處,庶務叢集,霆隨事裁決,處之泰然。暇日詣郡庠,與諸生講學,又取隸官閑田,增益廩稍。選民之強壯,當農隙訓練之,時付以器械,雜兵行肄習,親閱試行賞以激勸之。未幾,有能擐甲騎射者,遂皆
 獲其用,而兵不復擾民。



 丁內艱,德祐初,起復奉議郎、湖南安撫司參議,與安撫使李芾協力戰守。霆有心計,善出奇應變,帥府機務,芾一以委之。城初被圍,日夜守御,數日西北隅破,霆麾兵巷戰,抵暮增築月城,比旦城復完,策厲將士,以死守之。城既破,霆赴水死,妻妾奔救無及,遂皆死。



 趙卯發,字漢卿,昌州人。淳祐十年,以上舍登第,為遂寧府司戶、潼川簽判、宣城宰。素以節行稱。中被論罷。咸淳
 七年,起為彭澤令。十年,權通判池州。



 大兵渡江,池守王起宗棄官去,卯發攝州事,繕壁聚糧,為守禦計。夏貴兵敗歸,所過縱掠,卯發捕斬十餘人,兵乃戢。明年正月,大兵至李王河,都統張林屢諷之降,卯發忿氣填膺,瞠目視林不能言。有問以禔身之道者,卯發曰:「忠義所以禔身也,此外非臣子所得言。」林以兵出巡江,陰降,歸而陽助卯發為守,守兵五百餘,柄皆歸林。卯發知不可守,乃置酒會親友,與飲訣,謂其妻雍氏曰:「城將破,吾守臣不
 當去,汝先出走。」雍氏曰:「君為命官,我為命婦,君為忠臣,我獨不能為忠臣婦乎?」卯發笑曰:「此豈婦人女子之所能也。」雍氏曰:「吾請先君死。」卯發笑止之。明日乃散其家資與其弟侄,僕婢悉遣之。



 二月,兵薄池,卯發晨起書幾上曰:「君不可叛,城不可降,夫妻同死,節義成雙。」又為詩別其兄弟,與雍盛服同縊從容堂死。卯發始為此堂,名「可以從容」,及兵遽,領客堂中,指所題扁曰:「吾必死於是。」客問其故,曰:「古人謂『慷慨殺身易,從容就義難」,此殆其
 兆也。」卯發死,林開門降。大元丞相伯顏入,問太守何在,左右以死對。即如堂中觀之,皆嘆息。為具棺衾合葬於池上,祭其墓而去。事聞,贈華文閣待制,謚文節,雍氏贈順義夫人,錄二子為京官。



 唐震,字景實,會稽人。少居鄉,介然不茍交,有言其過者輒喜。既登第為小官,有權貴以牒薦之者,震內牒篋中,已而干政,震取牒還之,封題未啟,其人大愧。後為他官,所至以公廉稱。楊棟、葉夢鼎居政府,交薦其賢。



 咸淳中,
 由大理司直通判臨安府。時潛說友尹京,恃賈似道勢,甚驕蹇,政事一切無所顧讓。會府有具獄將置闢,震力辨其非,說友爭之不得,上其事刑部,卒是震議。



 六年,江東大旱,擢知信州。震奏減綱運米,蠲其租賦,令坊置一吏,籍其戶,勸富人分粟,使坊吏主給之。吏有勞者,輒為具奏復其身,吏感其誠,事為盡力,所活無算。州有民庸童牧牛,童逸而牧舍火,其父訟庸者殺其子投火中,民不勝掠,自誣服。震視牘疑之,密物色之,得童傍郡,以詰
 其父,對如初,震出其子示之,獄遂直。擢浙西提刑。過闕陛辭,似道以類田屬震,震謝不能行,至部,又以疏力爭之。趙氏有守阡僧甚暴橫,震遣吏捕治,似道以書營救,震不省,卒按以法。似道怒,使侍御史陳堅劾去之。



 咸淳十年,起震知饒州。時興國、南康、江州諸郡皆已歸附,大兵略饒。饒兵止千八百人,震發州民城守,昧爽出治兵,至夜中始寐,上書求援,不報。大兵使人入饒取降款,通判萬道同陰使於所部斂白金、牛酒備降禮,饒寓士皆
 從之。道同風震降,震叱之曰:「我忍偷生負國邪?」城中少年感震言,殺使者。民有李希聖者謀出降,械置獄中。明年二月,兵大至,都大提舉鄧益遁去,震盡出府中金錢,書官資揭於城,募有能出戰者賞之。眾懼不能戰,北兵登陴,眾遂潰。震入府中玉芝堂,其僕前請曰:「事急矣,番江門兵未合,亟出猶可免。」震罵曰:「城中民命皆系於我,我若從爾言得不死,城中民死,我何面目生邪?」左右不復敢言,皆出。有頃,兵入,執牘鋪案上,使震署降,震擲筆
 於地,不屈,遂死之。兄椿與家人俱死。張世傑尋復饒州,判官鄔宗節求震尸葬之。贈華文閣待制,謚忠介,廟號褒忠,官其二子。



 震客馮驥、何新之,驥後守獨松關,新之守閩之新壘,皆戰死。



 趙與BT,為嗣秀王。德祐二年,為浙、閩、廣察訪使。益王之立,舅楊亮節居中秉權,與BT自以國家親賢,多所諫止,遂犯忌嫉,諸將俱憚之。未幾,北兵逼浙東,乃命與BT出瑞安,與守臣方洪共任備御。朝臣言與BT有劉更生之
 忠,曹王皋之孝,宜留輔以隆國本。譖者益急,卒遣之。瑞安受圍,城中危急,與洪誓以死守。小校李雄夜開門納外兵,與BT、洪率眾巷戰,兵敗被縶,董文炳問之曰:「汝為秀王耶?今能降乎?」與BT厲聲曰:「我國家近親,今力屈而死,分也,尚何問為?」遂殺之。洪亦伏節而死。



 又有趙孟錦者,少不羈,游淮以軍功為將佐。北兵攻真州,每戰輒為士卒先,守苗再成倚之為重。北兵重艦駐江上,孟錦乘大霧來襲,俄霧解,日已高,北兵見其兵少,逐之,登舟失
 足墮水,身荷重甲,溺焉。



 趙淮,丞相葵之從子也。李全之叛,屢立戰功,累官至淮東轉運使。德祐中,戍銀樹埧,兵敗,與其妾俱被執至瓜州,元帥阿術使淮招李庭芝,許以大官。淮陽許諾,至揚城下,乃大呼曰:「李庭芝!男子死耳,毋降也!」元帥怒,殺之,棄尸江濱。



\end{pinyinscope}