\article{志第四十八 河渠五}

\begin{pinyinscope}

 漳河滹沱河御河塘濼緣邊諸水河北諸水岷江



 漳河源於西山,由磁、洺州南入冀州新河鎮,與胡盧河合流,其後變徙,入於大河。



 神宗熙寧三年,詔程昉同河北提點刑獄王廣廉相視。四年,開修,役兵萬人,袤一百
 六十里。帝因與大臣論財用,文彥博曰:「足財用在乎安百姓,安百姓在乎省力役。且河久不開,不出於東,則出於西,利害一也。今發夫開治,徙東從西,何利之有?」王安石曰:「使漳河不由地中行,則或東或西,為害一也。治之使行地中,則有利而無害。勞民,先王所謹,然以佚道使民,雖勞不可不勉。」會京東、河北大風,三月,詔曰:「風變異常,當安靜以應天災。漳河之役妨農,來歲為之未晚。」中書格詔不下。尋有旨權令罷役,程昉憤恚,遂請休退。朝
 廷令以都水丞領淤田事於河上。



 五月,御史劉摯言:「昉等開修漳河,凡用九萬夫。物料本不預備,官私應急,勞費百倍。逼人夫夜役,踐蹂田苗,發掘墳墓,殘壞桑柘,不知其數。愁怨之聲,流播道路,而昉等妄奏民間樂於工役。河北廂軍,鏟刷都盡,而昉等仍乞於洺州調急夫,又欲令役兵不分番次,其急切擾攘,至於如此。乞重行貶竄,以謝疲民。」中丞楊繪亦以為言。王安石為昉辨說甚力,後卒開之。五年,工畢,昉與大理寺丞李宜之、知洺州
 黃秉推恩有差。



 七年六月,知冀州王慶民言:「州有小漳河,向為黃河北流所壅,今河已東,乞開浚。」詔外都水監相度而已。



 滹沱河源於西山,由真定、深州、乾寧,與御河合流。



 神宗熙寧元年,河水漲溢,詔都水監、河北轉運司疏治。六年,深州、祁州、永寧軍修新河。八年正月,發夫五千人,並胡盧河增治之。



 元豐四年正月,北外都水丞陳祐甫言:「滹沱自熙寧八年以後,泛濫深州諸邑,為患甚大。諸司累
 相度不決,謂其下流舊入邊吳、宜子澱,最為便順,而屯田司懼填淤塘濼,煩文往復,無所適從。昨差官計之,若障入胡盧河,約用工千六百萬,若治程昉新河,約用工六百萬,若依舊入邊吳等澱,約用工二十九萬,其工費固已相遠。乞嚴立期會,定歸一策。」詔河北屯田轉運司同北外都水丞司相視。



 五年八月癸酉,前河北轉運副使周革言:「熙寧中,程昉於真定府中渡創系浮梁,增費數倍。既非形勢控扼,請歲八九月易以版橋,至四五月
 防河即拆去,權用船渡。」從之。



 御河源出衛州共城縣百門泉,自通利、乾寧入界河,達於海。



 神宗熙寧二年九月,劉彞、程昉言:「二股河北流今已閉塞,然御河水由冀州下流,尚當疏導,以絕河患。」先是,議者欲於恩州武城縣開御河約二十里,入黃河北流故道,下五股河,故命彞、昉相度。而通判冀州王庠謂,第開見行流處,下接胡盧河,尤便近。彞等又奏:「如庠言,雖於河流為順,然其間漫淺沮洳,費工猶多,不若開烏
 欄堤東北至大、小流港,橫截黃河,入五股河,復故道,尤便。」遂命河北提舉糴便糧草皮公弼、提舉常平王廣廉按視,二人議協,詔調鎮、趙、邢、洺、磁、相州兵夫六萬浚之,以寒食後入役。



 三年正月,韓琦言:「河朔累經災傷,雖得去年夏秋一稔,瘡痍未復。而六州之人,奔走河役,遠者十一二程,近者不下七八程,比常歲勞費過倍。兼鎮、趙兩州,舊以次邊,未嘗差夫,一旦調發,人心不安。又於寒食後入役,比滿一月,正妨農務。」詔河北都轉運使劉庠
 相度,如可就寒食前入役,即亟興工,仍相度最遠州縣,量減差夫,而輟修塘堤兵千人代其役。二月,琦又奏:「御河漕運通流,不宜減大河夫役。」於是止令樞密院調兵三千,並都水監卒二千。三月,又益發壯城兵三千,仍詔提舉官程昉等促迫功限。六月,河成,詔昉赴闕,遷宮苑副使。四年,命昉為都大提舉黃、御等河。



 八年,昉與劉璯言:「衛州沙河湮沒,宜自王供埽開浚,引大河水注之御河,以通江、淮漕運。仍置斗門,以時啟閉。其利有五:王供
 危急,免河勢變移而別開口地,一也。漕舟出汴,橫絕沙河,免大河風濤之患,二也。沙河引水入於御河,大河漲溢,沙河自有限節,三也。御河漲溢,有斗門啟閉,無沖注淤塞之弊,四也。德、博舟運,免數百里大河之險,五也。一舉而五利附焉。請發卒萬人,一月可成。」從之。



 九年秋,昉奏畢功。中書欲論賞,帝令河北監司案視保明,大名安撫使文彥博覆實。十月,彥博言:



 去秋開舊沙河,取黃河行運,欲通江、淮舟楫,徹於河北極邊。自今春開口放水,
 後來漲落不定,所行舟□伐皆輕載,有害無利,枉費功料極多。今御河上源,止是百門泉水,其勢壯猛,至衛州以下,可勝三四百斛之舟,四時行運,未嘗阻滯。堤防不至高厚,亦無水患。今乃取黃河水以益之,大即不能吞納,必致決溢;小則緩漫淺澀,必致淤澱。凡上下千餘里,必難歲歲開浚。況此河穿北京城中,利害易睹。今始初冬,已見阻滯,恐年歲間,反壞久來行運。儻謂通江、淮之漕,即尤不然。自江、浙、淮、汴入黃河,順流而下,又合於御河,
 大約歲不過一百萬斛。若自汴順流徑入黃河,達於北京,自北京和雇車乘,陸行入倉,約用錢五六千緡,卻於御河裝載赴邊城,其省工役、物料及河清衣糧之費,不可勝計。



 又去冬,外監丞欲於北京黃河新堤開置水口,以通行運,其策尤疏。此乃熙寧四年秋黃河下注御河之處,當時朝廷選差近臣,督役修塞,所費不貲。大名、恩冀之人,至今瘡痍未平,今奈何反欲開口導水耶?都水監雖令所屬相視,而官吏恐忤建謀之官,止作遷延,回
 報謂俟修固禦河堤防,方議開置河口,況御河堤道,僅如蔡河之類,若欲吞納河水,須如汴岸增修,猶恐不能制蓄。乞別委清強官相視利害,並議可否。



 又言:「今之水官,尤為不職,容易建言,僥幸恩賞。朝廷便為主張,中外莫敢異議,事若不效,都無譴罰。臣謂更當選擇其人,不宜令狂妄輩橫費生民膏血。」



 已而都水監言,運河乞置雙閘,例放舟船實便,與彥博所言不同。十二月,命知制誥熊本與都水監、河北轉運司官相視。本奏:



 河北州軍
 賞給茶貨,以至應接沿邊榷場要用之物,並自黃河運至黎陽出卸,轉入御河,費用止於客軍數百人添支而已。向者,朝廷曾賜米河北,亦於黎陽或馬陵道口下卸,倒裝轉致,費亦不多。昨因程昉等擘畫,於衛州西南,循沙河故跡決口置閘,鑿堤引河,以通江、淮舟楫,而實邊郡倉稟。自興役至畢,凡用錢米、功料二百萬有奇。今後每歲用物料一百一十六萬,廂軍一千七百餘人,約費錢五萬七千餘緡。開河行水,才百餘日,所過船□伐六百
 二十五,而衛州界御河淤淺,已及三萬八千餘步;沙河左右民田,渰浸者幾千頃,所免租稅二千貫石有餘。有費無利,誠如議者所論。



 然尚有大者,衛州居御河上游,而西南當王供向著之會,所以捍黃河之患者,一堤而已。今穴堤引河,而置閘之地,才及堤身之半。詢之土人云,自慶歷八年後,大水七至,方其盛時,游波有平堤者。今河流安順三年矣,設復礬水暴漲,則河身乃在閘口之上。以湍悍之勢而無堤防之阻,泛濫沖溢,下合御河,
 臣恐墊溺之禍,不特在乎衛州,而瀕御河郡縣,皆罹其患矣。



 夫此河之興,一歲所濟船□伐,其數止此,而萌每歲不測之患,積無窮不貲之費,豈陛下所以垂世裕民之意哉!臣博採眾論,究極利病,咸以謂葺故堤,堰新口,存新閘而勿治,庶可以銷淤澱決溢之患,而省無窮之費。萬一他日欲由此河轉粟塞下,則暫開亟止,或可紓飛挽之勞。



 未幾,河果決衛州。



 元豐五年,提舉河北黃河堤防司言:「御河狹隘,堤防不固,不足容大河分水,乞令綱
 運轉入大河,而閉截徐曲。」既從之矣。明年,戶部侍郎蹇周輔復請開撥,以通漕運,及令商旅舟船至邊。是時,每有一議,朝廷輒下水官相度,或作或輟,迄莫能定。大抵自小吳埽決,大河北流,御河數為漲水所冒,亦或湮沒。哲宗紹聖三年四月,河北都轉運使吳安持始奏,大河東流,御河復出。詔委前都水丞李仲提舉開導。



 徽宗崇寧元年冬,詔侯臨同北外都水丞司開臨清縣壩子口,增修御河西堤,高三尺,並計度西堤開置斗門,決北京、
 恩、冀、滄州、永靜軍積水入御河枯源。明年秋,黃河漲入御河,行流浸大名府館陶縣,敗廬舍,復用夫七千,役二十一萬餘工修西堤,三月始畢,漲水復壞之。



 政和五年閏正月,詔於恩州北增修御河東堤,為治水堤防,令京西路差借來年分溝河夫千人赴役。於是都水使者孟揆移撥十八埽官兵,分地步修築,又取棗強上埽水口以下舊堤所管榆柳為樁木。



 塘濼,緣邊諸水所聚,因以限遼。河北屯田司、緣邊安撫
 司皆掌之,而以河北轉運使兼都大制置。凡水之淺深,屯田司季申工部。其水東起滄州界,拒海岸黑龍港,西至乾寧軍,沿永濟河合破船澱、灰澱、方澱為一水,衡廣一百一十里,縱九十里至一百三十里,其深五尺。東起乾寧軍、西信安軍永濟渠為一水,西合鵝巢澱、陳人澱、燕丹澱、大光澱、孟宗澱為一水,衡廣一百二十里,縱三十里,或五十里,其深丈餘或六尺。東起信安軍永濟渠,西至霸州莫金口,合水汶澱、得勝澱、下光澱、小蘭澱、李
 子澱、大蘭澱為一水,衡廣七十里,或十五里或六里,其深六尺或七尺。東北起霸州莫金口,西南保定軍父母砦,合糧料澱、回澱為一水,衡廣二十七里,縱八里,其深六尺。霸州至保定軍並塘岸水最淺,故咸平、景德中,契丹南牧,以霸州、信安軍為歸路。東南起保安軍,西北雄州,合百水澱、黑羊澱、小蓮花澱為一水,衡廣六十里,縱二十五里或十里,其深八尺或九尺。東起雄州,西至順安軍,合大蓮花澱、洛陽澱、牛橫澱、康池澱、疇澱、白羊澱
 為一水,衡廣七十里,縱三十里或四十五里,其深一丈或六尺或七尺。東起順安軍,西邊吳澱至保州,合齊女澱、勞澱為一水,衡廣三十餘里,縱百五十里,其深一丈三尺或一丈。起安肅、廣信軍之南,保州西北,畜沈苑河為塘,衡廣二十里,縱十里,其深五尺,淺或三尺,曰沈苑泊。自保州西合雞距泉、尚泉為稻田、方田,衡廣十里,其深五尺至三尺,曰西塘泊。自何承矩以黃懋為判官,始開置屯田,築堤儲水為阻固,其後益增廣之。凡並邊諸
 河,若滹沱、胡盧、永濟等河,皆匯於塘。



 天聖以後,相循而不廢,仍領於沿邊屯田司。而當職之吏,各從其所見,或曰:「有兵將在,契丹來,雲無所事塘。自邊吳澱西望長城口,尚百餘里,皆山阜高仰,水不能至,契丹騎馳突,得此路足矣,塘雖距海,亦無所用。夫以無用之塘,而廢可耕之田,則邊穀貴,自困之道也。不如勿廣,以息民為根本。」或者則曰:「河朔幅員二千里,地平夷無險阻。契丹從西方入,放兵大掠,由東方而歸,我嬰城之不暇,其何以御
 之?自邊吳澱至泥姑海口,綿亙七州軍,屈曲九百里,深不可以舟行,淺不可以徒涉,雖有勁兵,不能度也。東有所阻,則甲兵之備,可以專力於其西矣。孰謂無益?」論者自是分為兩歧,而廷朝以契丹出沒無常,阻固終不可以廢也。



 仁宗明道二年,劉平自雄州徙知成德軍,奏曰:「臣向為沿邊安撫使,與安撫都監劉志嘗陳備邊之略。臣今徙真定路,由順安、安肅、保定州界,自邊吳澱望趙曠川、長城口,乃契丹出入要害之地,東西不及一百五
 十里。臣竊恨聖朝七十餘年,守邊之臣,何可勝數,皆不能為朝廷預設深溝高壘,以為阨塞。臣聞太宗朝,嘗有建請置方田者。今契丹國多事,兵荒相繼,我乘此以引水植稻為名,開方田,隨田塍四面穿溝渠,縱廣一丈,深二丈,鱗次交錯,兩溝間屈曲為徑路,才令通步兵。引曹河、鮑河、徐河、雞距泉分注溝中,地高則用水車汲引,灌溉甚便。願以劉志知廣信軍,與楊懷敏共主其事,數年之後,必有成績。」帝遂密敕平與懷敏建方田。侍禁劉
 宗言又奏請種木於西山之麓,以法榆塞,云可以限契丹也。後劉平去真定,懷敏猶領屯田司。塘泊益廣,至吞沒民田,蕩溺丘墓,百姓始告病,乃有盜決以免水患者,懷敏奏立法依盜決堤防律。



 景祐二年,懷敏知雄州,又請立木為水則,以限盈縮。寶元元年十一月己未,河北屯田司言:「欲於石塚口導永濟河水,以注緣邊塘泊,請免所經民田稅。」從之。時歲旱,塘水涸,懷敏慮契丹使至,測知其廣深,乃壅界河水注之,塘復如故。



 慶歷二年三
 月己巳,契丹遣使致書,求關南十縣。且曰:「營築長堤,填塞隘路,開決塘水,添置邊軍,既潛稔於猜嫌,慮難敦於信睦。」四月庚辰,復書曰:「營築堤埭,開決陂塘,昨緣霖潦之餘,大為衍溢之患,既非疏導,當稍繕防,豈蘊猜嫌,以虧信睦。」遼使劉六符嘗謂賈昌朝曰:「南朝塘濼何為者哉?一葦可杭,投棰可平。不然,決其堤,十萬土囊,遂可逾矣。」時議者亦請涸其地以養兵。帝問王拱辰,對曰:「兵事尚詭,彼誠有謀,不應以語敵,此六符誇言爾。設險守國,
 先王不廢,且祖宗所以限遼騎也。」帝深然之。



 七月,契丹復議和好,約兩界河澱已前開畎者並依舊外,自今已後,各不添展。其見堤堰水口,逐時決洩壅塞,量差兵夫,取便修疊疏導。非時霖潦,別至大段漲溢,並不在關報之限。是歲,劉宗言知順安軍,上言:「屯田司浚塘水,漂招賢鄉六千戶。」



 五年七月,初與契丹約,罷廣兩界塘澱。約既定,朝廷重生事,自是每邊臣言利害,雖聽許,必戒之以毋張皇,使契丹有詞。而楊懷敏獨治塘益急,是月,懷
 敏密奏曰:「前轉運使沉邈開七汲口洩塘水,臣已亟塞之。知順安軍劉宗言閉五門帕頭港、下赤大渦柳林口漳河水,不使入塘,臣已復通之,令注白羊澱矣。邈、宗言朋黨沮事如此,不譴誅無以懲後。」詔從懷敏奏,自今有妄乞改水口者,重責之。



 嘉祐中,御史中丞韓絳言:「宣祖已上,本籍保州,懷敏廣塘水,侵皇朝遠祖墳。近聞詔旨以錢二百千賜本宗使易葬,此虧薄國體尤甚,物論駭嘆,願請州縣屏水患而已。」知雄州趙滋言:「屯田司當徐
 河間築堤斷水,塘堤具存,可覆視也。宜開水竇六十尺,修石限以節之。」咸可其奏。八年,河北提點刑獄張問言:「視八州軍塘,出土為堤,以畜西山之水,涉夏河溢,而民田無患。」亦施行焉。



 神宗熙寧元年正月,復汾州西河濼。濼舊在城東,圍四十里,歲旱以溉民田,雨以瀦水,又有蒲魚、茭芡之利,可給貧民。前轉運使王沿廢為田,人不以為便。至是,知雜御史劉述請復之。是歲,又遣程昉諭邊臣營治諸濼,以備守御。



 五年,
 東頭供奉官趙忠政言:「界河以南至滄州凡三百里,夏秋可徒涉,遇冬則冰合,無異平地。請自滄州東接海,西抵西山,植榆柳、桑棗,數年之間,可限契丹。然後施力耕種,益出租賦,以助邊儲。」詔程昉察視利害以聞。



 六年五月,帝與王安石論王公設險守國,安石曰:「《周官》亦有掌固之官,但多侵民田,恃以為國,亦非計也。太祖時未有塘泊,然契丹莫敢侵軼。」他日,樞密院官言:「程昉放滹沱水,大懼填淤塘濼,失險固之利。」安石謂:「滹沱舊入邊吳澱,新入洪城澱,均塘濼
 也。何昔不言而今言乎?」蓋安石方主昉等,故其論如此。



 六年十二月癸酉,命河北同提點制置屯田使閻士良專興修樸樁口,增灌東塘澱濼。先是,滄州北三堂等塘濼,為黃河所注,其後河改而濼塞。程昉嘗請開琵琶灣引河水,而功不成。至是,士良請堰水絕禦河,引西塘水灌之,故有是命。



 七年六月丁丑,河北沿邊安撫司上《制置沿邊浚陂塘築堤道條式圖》,請付邊郡屯田司。又言於沿邊軍城植柳蒔麻,以備邊用。並從之。



 九年六月,高
 陽關言:「信安、乾寧塘濼,昨因不收獨流決口,至今乾涸。」於是命河北東、西路分遣監司,視廣狹淺深,具圖本上。十年正月甲子,詔:「比修築河北破缺塘堤,收匱水勢。其信安軍等處因塘水減涸,退出田土,己召入耕佃者復取之。」



 元豐三年,詔諭邊臣曰:「比者契丹出沒不常,不可全恃信約以為萬世之安。況河朔地勢坦平,略無險阻,殆非前世之比。惟是塘水實為礙塞,卿等當體朕意,協力增修,自非地勢高仰,人力所不可施者,皆在滋廣,用
 謹邊防。蓋功利近在目前而不為,良可惜也。」六年十二月,定州路安撫使韓絳言:「定州界西自山麓,東接塘澱,綿地百餘里,可瀦水設險。」詔以引水灌田陂為名。哲宗元祐中,大臣欲回河東流者,皆以北流壞塘濼為言,事見前篇。



 徽宗大觀二年十二月,詔曰:「瀦水為塘,以備泛濫,留屯營田,以實塞下,國家設官置吏,專總其事。州縣習玩,歲久隳壞。其令屯田司循祖宗以來塘堤故跡修治之,毋得增益生事。」大抵河北塘濼,東距海,西抵廣信、
 安肅,深不可涉,淺不可舟,故指為險固之地。其後淤澱乾涸,不復開浚,官司利於稻田,往往洩去積水,自是堤防壞矣。



 河北諸水,有通轉餉者,有為方田限遼人者。太宗太平興國六年正月,遣八作使郝守浚分行河道,抵於遼境者,皆疏導之。又於清苑界開徐河、雞距河五十里入白河。自是關南之漕,悉通濟焉。端拱二年,以左諫議大夫陳恕為河北東路招置營田使,魏羽為副使;右諫議大
 夫樊知古為河北西路招置營田使,索湘為副使,欲大興營田也。



 先是,自雄州東際於海,多積水,契丹患之,未嘗敢由此路入,每歲,數擾順安軍。議者以為宜度地形高下,因水陸之便,建阡陌,浚溝洫,益樹五稼,所以實邊廩而限契丹。雍熙後,數用兵,岐溝、君子館敗衄之後,河朔之民,農桑失業,多閑田,且戍兵增倍,故遣恕等經營之。恕密奏:「戍卒皆墮游,仰食縣官,一旦使冬被甲兵,春執耒耜,恐變生不測。」乃詔止令葺營堡,營田之議遂寢。



 淳化二年,從河北轉運使請,自深州新砦鎮開新河,導胡盧河,分為一派,凡二百里抵常山,以通漕運。胡盧河源於西山,始自冀州新河鎮入深州武強縣,與滹沱河合流,其後變徙,入大河。至神宗熙寧中,內侍程昉請開決引水入新河故道,詔本路遣官按視。永靜軍判官林伸、東光縣令張言舉言:「新河地形高仰,恐害民田」昉言:「地勢最順,宜無不便。」乃復遣劉璯、李直躬考實,而□會等卒如昉言,伸等坐貶官。



 四年春,詔六宅使何承矩等督
 戍兵萬八千人,自霸州界引滹沱水灌稻為屯田,用實軍廩,且為備御焉。初,臨津令黃懋上封事,盛稱水田之利,乃以承矩洎內供奉官閻承翰、殿直張從古同制置河北緣邊屯田事,仍以懋為大理寺丞,充屯田判官,其所經畫,悉如懋奏。



 真宗咸平四年,知靜戎軍王能請自姜女廟東決鮑河水,北入閻臺澱,又自靜戎之東,引北注三臺、小李村,其水溢入長城口而南,又壅使北流而東入於雄州。五年,順安軍兵馬都監馬濟復請自靜戎
 軍東,擁鮑河開渠入順安軍,又自順安軍之西引入威虜軍,置水陸營田於渠側。濟等言:「役成,可以達糧漕,隔遼騎。」帝許之,獨鹽臺澱稍高,恐決引非便,不從其議。因詔莫州部署石普並護其役。逾年功畢。帝曰:「普引軍壁馬村以西,開鑿深廣,足以張大軍勢。若邊城壕溝悉如此,則遼人倉卒難馳突而易追襲矣。」其年,河北轉運使耿望開鎮州常山鎮南河水入洨河至趙州,有詔褒之。三月,西京左藏庫使舒知白請於泥姑海口、章口復置
 海作務造舟,令民入海捕魚,因偵平州機事。異日王師征討,亦可由此進兵,以分敵勢。先是,置船務,以近海之民與遼人往還,遼人嘗泛舟直入千乘縣,亦疑有鄉導之者,故廢務。至是,令轉運使條上利害。既而以為非便,罷之。



 景德元年,北面都鈐轄閻承翰自嘉山東引唐河三十二里至定州,釃而為渠,直蒲陰縣東六十二里會沙河,徑邊吳泊,遂入於界河,以達方舟之漕。又引保州趙彬堰徐河水入雞距泉,以息挽舟之役,自是朔方之
 民,灌溉饒益,大蒙其利矣。八月,詔滄州、乾寧軍謹視斗門水口,壅潮水入御河東塘堰,以廣溉蔭。四年五月,知雄州李允則決渠為水田,帝以渠接界河,罷之。因下詔曰:「頃修國好,聽其盟約,不欲生事,姑務息民。自今邊城止可修葺城壕,其餘河道,不得輒有浚治。」



 大中祥符七年四月,涇原都鈐轄曹瑋言:「渭北有古池,連帶山麓,今浚為渠,令民導以溉田。」六月,知永興軍陳堯咨導龍首渠入城,民庶便之。並詔嘉獎。天禧末,諸州屯田總四千
 二百餘頃,而河北屯田歲收二萬九千四百餘石,保州最多,逾其半焉。江、淮、兩浙承偽制,皆有屯田,克復後,多賦與民輸租,第存其名。在河北者雖有其實,而歲入無幾,利在畜水以限遼騎而已。



 仁宗天聖四年閏五月,陜西轉運使王博文等言:「準敕相度開治解州安邑縣至白家場永豐渠,行舟運鹽,經久不至勞民。按此渠自後魏正始二年,都水校尉元清引平坑水西入黃河以運鹽,故號永豐渠。周、齊之間,渠遂廢絕。隋大業中,都水監
 姚暹決堰浚渠,自陜郊西入解縣,民賴其利。及唐末至五代亂離,迄今湮沒,水甚淺涸,舟楫不行。」詔三司相度以聞。



 神宗即位,志在富國,故以劭農為先。熙寧元年六月,詔諸路監司:「比歲所在陂塘堙沒,瀕江圩堤浸壞,沃壤不得耕,宜訪其可興者,勸民興之,具所增田畝稅賦以聞。」二年十月,權三司使吳充言:「前宜城令朱紘,治平間修復木渠,不費公家束薪斗粟,而民樂趨之。渠成,溉田六千餘頃,數邑蒙其利。」詔遷紘大理寺丞,知比陽縣。
 或云紘之木渠,繞工度溪以行水,數勤民而終無功。



 十一月,制置三司條例司具《農田利害條約》,詔頒諸路:「凡有能知土地所宜種植之法,及修復陂湖河港,或元無陂塘、圩堤、堤堰、溝洫而可以創修,或水利可及眾而為人所擅有,或田去河港不遠,為地界所隔,可以均濟流通者;縣有廢田曠土,可糾合興修,大川溝瀆淺塞荒穢,合行浚導,及陂塘堰埭可以取水灌溉,若廢壞可興治者,各述所見,編為圖籍,上之有司。其土田迫大川,數經
 水害,或地勢污下,雨潦所鐘,要在修築圩堤、堤防之類,以障水澇,或疏導溝洫、畎澮,以洩積水。縣不能辦,州為遣官,事關數州,具奏取旨。民修水利,許貸常平錢穀給用。」初,條例司奏遣劉彞等八人行天下,相鋧農田水利,又下諸路轉運司各條上利害,又詔諸路各置相度農田水利官。至是,以《條約》頒焉。



 秘書丞侯叔獻言:「汴岸沃壤千里,而夾河公私廢田,略計二萬餘頃,多用牧馬。計馬而牧,不過用地之半,則是萬有餘頃常為不耕之地。
 觀其地勢,利於行水。欲於汴河兩岸置斗門,洩其餘水,分為支渠,及引京、索河並三十六陂,以灌溉田。」詔叔獻提舉開封府界常平,使行之,而以著作佐郎楊汲同提舉。叔獻又引汴水淤田,而祥符、中牟之民大被水患,都水監或以為非。



 三年三月,帝謂王安石、韓絳曰:「都水沮壞淤田者,以侵其職事爾。」安石曰:「必欲任屬,當以楊汲為都水監。今每事稟於沉立、張鞏,何能辦集。」七月,帝聞淤田多浸民田稼、屋宇,令內侍馮宗道往視,宗道以說
 者為妄。八月,叔獻、汲並權都水監丞、提舉沿汴淤田。



 九月戊申,遣殿中丞陳世修乘驛經度陳、穎州八丈溝故跡。初,世修言:「陳州項城縣界蔡河東岸有八丈溝,或斷或續,迤邐東去,由穎及壽,綿亙三百五十餘里,乞因其故道,量加浚治。興復大江、次河、射虎、流龍、百尺等陂塘,導水行溝中,棋布灌溉,俾數百里復為稻田,則其利百倍。」繪圖來上,帝意向之。王安石曰:「世修言引水事即可試,八丈溝新河則不然。昔鄧艾不賴蔡河漕運,故能並
 水東下,大興水田。厥後既分水以注蔡河,又有新修閘以限之,與昔不同。惟無所用水,即水可並而溝可復矣。」故先命世修相度。



 四年三月,帝語侍臣:「中人視麥者,言淤田甚佳,有未淤不可耕之地,一望數百里。獨樞密院以淤田無益,謂其薄如餅。」安石曰:「就令薄,固可再淤,厚而後止。」是月,帝以慶州軍亂,召執政對資政殿。馮京曰:「府界既淤田,又行免役,作保甲,人極勞弊。」帝曰:「淤田於百姓何苦?聞土細如面。」王安石曰:「慶卒之變,陛下旰食。
 大臣宜於此時共圖消弭,乃合為浮議,歸咎淤田、保甲,了不相關,此非待至明而後察也。」十月,前知襄州光祿卿史照言:「開修古淳河一百六里,灌田六千六百餘頃,修治陂堰,民已獲利,慮州縣遽欲增稅。」詔三司應興修水利,墾開荒梗,毋增稅。



 五年二月侯叔獻等言:「民願買官淤田者七十餘戶,已分赤淤、花淤等,及定其直各有差,仍於次年起稅。若願增錢者,不以投狀先後給之。」五月,御史張商英言:「嘗聞獻議者請開鄧州穰縣永國渠,
 引湍河水灌溉民田,失邵信臣故道,鑿焦家莊,地勢偏仰,水不通流。」詔京西路覆實,遣程昉領其事。昉刳河去疏土,築為巨堰。水行再歲,會霖雨,溪谷合流大漲,堰下土疏惡,莫能御,由此廢不復治。閏七月,程昉奏引漳、洺河淤地凡二千四百餘頃,帝曰:「灌溉之利,農事大本,但陜西、河東民素不習此,茍享其利,後必樂趨。三白渠為利尤大,有舊跡,可極力修治。凡疏積水,須自下流開導,則畎澮易治。《書》所謂『浚畎澮距川』是也。」



 時人人爭言水
 利。提舉京西常平陳世修乞於唐州引淮水入東西邵渠,灌注九子等十五陂,溉田二百里。提舉陜西常平沉披乞復京兆府武功縣古跡六門堰,於石渠南二百步傍為土洞,以木為門,回改河流,溉田三百四十里。大抵迂闊少效。披坐前為兩浙提舉,開常州五瀉堰不當,法寺論之,至是降一官。十一月,陜西提舉常平楊蟠議修鄭、白渠,詔都水丞周良孺相視。乃自石門堰涇水開新渠,至三限口以合白渠。王安石請捐常平息錢助民興
 作,帝曰:「縱用內帑錢,亦何惜也。」



 六年三月,程昉言:「得共城縣舊河槽,若疏導入三渡河,可灌西垙稻田。」從之。五月,詔:「諸創置水磑碾碓妨灌溉民田者,以違制論。」命贊善大夫蔡朦修永興軍白渠。八月,程昉欲引水淤漳旁地,王安石以為長利,須及冬乃可經畫。九月丙辰,賜侯叔獻、楊汲府界淤田各十頃。十月,命叔獻理提點刑獄資序,周良孺與升一任,皆賞淤田之勞也。陽武縣民邢晏等三百六十四戶言:「田沙堿瘠薄,乞淤溉,候淤深一
 尺,計畝輸錢,以助興修。」詔與淤溉,勿輸錢。



 十二月,河北提舉常平韓宗師論程昉十六罪,盛陶亦言昉。帝以問安石,安石請令昉、宗師及京東轉運司各差官同考實以聞。還奏得良田萬頃,又淤四千餘頃。於是進呈。宗師疏至言:「昉奏百姓乞淤田,實未嘗乞。」帝曰:「此小失,何罪,但不知淤田如何爾?」安石曰:「今檢到好田萬頃,又淤田四千餘頃,陛下以為不知,臣實未喻。」帝曰:「昉修漳河,漳河歲決;修滹沱,又無下尾。」安石力為辨說。已而宗師與
 昉皆放罪。他日,帝論唐太宗能受諫,安石因言:「陛下判功罪不及太宗。如程昉開閉四河,除漳河、黃河外,尚有溉淤及退出田四萬餘頃。自秦以來,水利之功,未有及此。止轉一官,又令與韓宗師同放罪,臣恐後世有以議聖德。」安石右昉,大率類此。



 是時,原武等縣民因淤田壞廬舍墳墓,妨秋稼,相率詣闕訴。使者聞之,急責縣令追呼,將杖之。民謬云:「詣闕謝耳。」使者因為民謝表,遣二吏詣鼓院投之,安石大喜。久之,帝始知雍丘等縣淤田清
 水頗害民田,詔提舉常平官視民耕地,蠲稅一料。樞密院奏:「淤田役兵多死,每一指揮,僅存軍員數人。」下提點司密究其事,提點司言:「死事者數不及三厘。」



 七年正月,程昉言:「滄州增修西流河堤,引黃河水淤田種稻,增灌塘泊,並深州開引滹沱水淤田,及開回胡盧河,並回滹沱河下尾。」六月,金州西城縣民葛德出私財修長樂堰,引水灌溉鄉戶土田,授本州司士參軍。八月甲戌,詔司農寺具所興修農田水利次第。九月,又詔:「籍所興水利,
 自今遣使體訪,其不實不當者,案驗以聞。」從侍御史張琥請也。十一月壬寅,知諫院鄧潤甫言:「淤田司引河水淤酸棗、陽武縣田,已役夫四五十萬,後以地下難淤而止。相度官吏初不審議,妄興夫役,乞加絀罰。」詔開封劾元檢計按覆官。丁未,同知諫院範百祿言:「向者都水監丞王孝先獻議,於同州朝邑縣界畎黃河,淤安昌等處堿地。及放河水,而堿地高原不能及,乃灌注朝邑縣長豐鄉永豐等十社千九百戶秋苗田三百六十餘頃。」詔
 蠲被水戶夏稅。是歲,知耀州閻充國募流民治漆水堤。



 八年正月,程昉言:「開滹沱、胡盧河直河淤田等部役官吏勞績,別為三等,乞推恩。」從之。三月庚戌,發京東常平米,募饑民修水利。四月,管轄京東淤田李孝寬言:「礬山漲水甚濁,乞開四斗門,引以淤田,權罷漕運再旬。」從之。深州靜安令任迪乞俟來年刈麥畢,全放滹沱、胡盧兩河,又引永靜軍雙陵口河水,淤溉南北岸田二萬七千餘頃,河北安撫副使沉披請治保州東南沿邊陸地為
 水田,皆從之。閏四月丁未,提點秦鳳等路刑獄鄭民憲請於熙州南關以南開渠堰,堰引洮水並東山直北道下至北關,並自通遠軍熟羊砦導渭河至軍溉田。詔民憲經度,如可作陂,即募京西、江南陂匠以往。



 五月乙酉,右班殿直、乾當修內司楊琰言:「開封、陳留、咸平三縣種稻,乞於陳留界舊汴河下口,因新舊二堤之間修築水塘,用碎甓築虛堤五步以來,取汴河清水入塘灌溉。」從之。七月,江寧府上元縣主簿韓宗厚引水溉田二千七
 百餘頃,遷光祿寺丞。太原府草澤史守一修晉祠水利,溉田六百餘頃。八月,知河中府陸經奏,管下淤官私田約二千餘頃,下司農覆實。九月癸未,提舉出賣解鹽張景溫言:「陳留等八縣堿地,可引黃、汴河水淤溉。」詔次年差夫。十二月癸丑,侯叔獻言:「劉瑾相度淮南合興修水利,僅十萬餘頃,皆並運河,乞候開河畢工,以水利司錢募民修築圩堤。」



 九年八月,程師孟言:「河東多土山高下,旁有川谷,每春夏大雨,眾水合流,濁如黃河礬山水,俗
 謂之天河水,可以淤田。絳州正平縣南董村旁有馬璧谷水,嘗誘民置地開渠,淤瘠田五百餘頃。其餘州縣有天河水及泉源處,亦開渠築堰。凡九州二十六縣,新舊之田,皆為沃壤,嘉祐五年畢功,纘成《水利圖經》二卷,迨今十七年矣。聞南董村田畝舊直三兩千,收穀五七斗。自灌淤後,其直三倍,所收至三兩石。今臣權領都水淤田,竊見累歲淤京東、西堿鹵之地,盡成膏腴,為利極大。尚慮河東猶有荒瘠之田,可引天河淤溉者。」於是遣都
 水監丞耿琬淤河東路田。



 十年六月,師孟、琬引河水淤京東、西沿汴田九千餘頃;七月,前權提點開封府界劉淑奏淤田八千七百餘頃;三人皆減磨勘年以賞之。九月,入內內侍省都知張茂則言:「河北東、西路夏秋霖雨,諸河決溢,占壓民田。」詔委官開畎。



 元豐元年二月,都大提舉淤田司言:「京東、西淤官私瘠地五千八百餘頃,乞差使臣管幹。」許之。四月,詔:「闢廢田、興水利、建立堤防、修貼圩堤之類,民力不給者,許貸常平錢穀。」六月,京東路
 體量安撫黃廉言:「梁山張澤兩濼,十數年來淤澱,每歲泛浸近城民田,乞自張澤濼下流浚至濱州,可洩壅滯。」從之。十二月壬申,二府奏事,語及淤田之利。帝曰:「大河源深流長,皆山川膏腴滲漉,故灌溉民田,可以變斥鹵而為肥沃。朕取淤土親嘗,極為潤膩。」二年,導洛通汴。六月,罷沿汴淤田司。十二月辛酉,置提舉定州路水利司。二年,知濰州楊採開白浪河。



 哲宗元祐以後,朝廷方務邊事,水利亦浸緩矣。四年二月甲辰,詔:「瀕河州縣,積水
 占田,在任官能為民溝畎疏導,退出良田百頃至千頃以上者,遞賞之,功利大者取特旨。」四年六月乙丑,知陳州胡宗愈言:「本州地勢卑下,秋夏之間,許蔡汝鄧、西京及開封諸處大雨,則諸河之水,並由陳州沙河、蔡河同入穎河,不能容受,故境內瀦為陂澤。今沙河合入穎河處,有古八丈溝,可以開浚,分決蔡河之水,自為一支,由穎、壽界直入於淮,則沙河之水雖甚洶湧,不能壅遏。」詔可。



 徽宗建中靖國元年十一月庚辰,赦書略曰:「熙寧、元
 豐中,諸路專置提舉官,兼領農田水利,應民田堤防灌溉之利,莫不修舉。近多因循廢弛,慮歲久日更隳壞,命典者以時檢舉推行。」



 崇寧二年三月,宰臣蔡京言:「熙寧初,修水土之政,元祐例多廢弛。紹復先烈,當在今日。如荒閑可耕,瘠鹵可腴,陸可為水,水可為陸,陂塘可修,灌溉可復,積潦可洩,圩堤可興,許民具陳利害。或官為借貸,或自備工力,或從官辦集。如能興修,依格酬獎,事功顯著,優與推恩。」從之。



 三年十月,臣僚言:「元豐官制,水之
 政令,詳立法之意,非徒為穿塞開導、修舉目前而已,凡天下水利,皆在所掌。在今尤急者,如浙右積水,比連震澤,未有歸宿,此最宜講明而未之及者也。願推廣元豐修明水政,條具以聞。」從之。



 岷江水發源處古導江,今為永康軍。《漢史》所謂秦蜀守李冰始鑿離堆,闢沫水之害,是也。



 沫水出蜀西徼外,今陽山江、大皂江皆為沫水,入於西川。始,嘉、眉、蜀、益間,夏潦洋溢,必有潰暴沖決可畏之患。自鑿離堆以分其勢,
 一派南流於成都以合岷江,一派由永康至瀘州以合大江,一派入東川,而後西川沫水之害減,而耕桑之利博矣。



 皂江支流迤北曰都江口,置大堰,疏北流為三:曰外應,溉永康之導江、成都之新繁,而達於懷安之金堂;東北曰三石洞,溉導江與彭之九隴、崇寧、蒙陽,而達於漢之雒;東南曰馬騎,溉導江與彭之崇寧、成都之郫、溫江、新都、新繁、成都、華陽。三流而下,派別支分,不可悉紀,其大者十有四:自外應而分,曰保堂,曰倉門;自三石洞
 曰將軍橋,曰灌田,曰雒源;自馬騎曰石址,曰豉彘,曰道溪,曰東穴,曰投龍,曰北,曰樽下,曰玉徙。而石渠之水,則自離堆別而東,與上下馬騎、乾溪合。凡為堰九:曰李光,曰膺村,曰百丈,曰石門,曰廣濟,曰顏上,曰弱水,曰濟,曰導,皆以堤攝北流,注之東而防其決。離堆之南,實支流故道,以竹籠石為大堤,凡七壘,如象鼻狀以捍之。離堆之趾,舊鑱石為水則,則盈一尺,至十而止。水及六則,流始足用,過則從侍郎堰減水河洩而歸於江。歲作侍郎
 堰,必以竹為繩,自北引而南,準水則第四以為高下之度。江道既分,水復湍暴,沙石填委,多成灘磧。歲暮水落,築堤壅水上流,春正月則役工浚治,謂之「穿淘。」



 元祐間,差憲臣提舉,守臣提督,通判提轄。縣各置籍,凡堰高下、闊狹、淺深,以至灌溉頃畝、夫役工料及監臨官吏,皆注于籍,歲終計效,賞如格。政和四年,又因臣僚之請,檢計修作不能如式以致決壞者,罰亦如之。大觀二年七月,詔曰:「蜀江之利,置堰溉田,旱則引灌,澇則疏導,故無水
 旱。然歲計修堰之費,敷調於民,工作之人,並緣為奸,濱江之民,困於騷動。自今如敢妄有檢計,大為工費,所剩坐贓論,入己準自盜法,許人告。」



 興元府褒斜谷口,古有六堰,澆溉民田,頃畝浩瀚。每春首,隨食水戶田畝多寡,均出夫力修葺。後經靖康之亂,民力不足,夏月暴水,沖損堰身。紹興二十二年,利州東路帥臣楊庚奏謂:「若全資水戶修理,農忙之時,恐致重困。欲過夏月,於見屯將兵內差不入隊人,並力修治,庶幾便民。」從之。



 興元府山
 河堰灌溉甚廣,世傳為漢蕭何所作。嘉祐中,提舉常平史照奏上堰法,獲降敕書,刻石堰上。中興以來,戶口凋疏,堰事荒廢,累增修葺,旋即決壞。乾道七年,遂委御前諸軍統制吳拱經理,發卒萬人助役,盡修六堰,浚大小渠六十五,復見古跡,並用水工準法修定。凡溉南鄭、褒城田二十三萬餘畝,昔之瘠薄,今為膏腴。四川宣撫王炎表稱拱宣力最多,詔書褒美焉。



\end{pinyinscope}