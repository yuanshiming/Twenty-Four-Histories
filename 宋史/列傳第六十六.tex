\article{列傳第六十六}

\begin{pinyinscope}

 喬維嶽王陟附張雍董儼魏廷式盧琰宋摶凌策楊覃陳世卿李若拙子繹陳知微



 喬維嶽,字伯周,陳州南頓人。治《三傳》。周顯德初登第,授
 太湖主簿。四年,遷平輿令。開寶中,右拾遺劉稹薦其才,擢為太子中舍、知高郵軍,通判揚州,徙常州。金陵平,又移升州,改殿中丞。太平興國初,徙襄州,俄丁內艱。三年,陳洪進表納疆土,以其子文顯為泉州留後,朝廷議擇能臣關掌郡事,即起維岳為通判。會盜起仙游莆田縣、百丈鎮,眾十餘萬攻城,城中兵裁三千,勢甚危急。監軍何承矩、王文寶欲盡屠其民,燔府庫而遁。維嶽挺然抗議,以為:「朝廷寄以綏遠,今惠澤未布,盜賊連結,反欲屠
 城,豈詔意哉。」承矩等因復堅守,既而轉運使楊克讓率福州兵破賊,圍遂解,詔褒之。



 歸朝,為淮南轉運副使,遷右補闕,進為使。淮河西流三十里曰山陽灣,水勢湍悍,運舟多罹覆溺。維嶽規度開故沙河,自末口至淮陰磨盤口,凡四十里。又建安北至淮澨,總五堰,運舟所至,十經上下,其重載者皆卸糧而過,舟時壞失糧,綱卒緣此為奸,潛有侵盜。維嶽始命創二斗門於西河第三堰,二門相距逾五十步,覆以廈屋,設縣門積水,俟潮平乃洩
 之。建橫橋岸上,築土累石,以牢其址。自是弊盡革,而運舟往來無滯矣。



 嘗按部至泗州,慮獄,法掾誤斷囚至死。維嶽詰之,法掾俯伏,且泣曰:「有母年八十餘,今獲罪,則母不能活矣。」維嶽憫之,因謂曰:「他日朝制按問,第雲轉運使令處茲罪。」卒如其言,獲免;維嶽坐贖金百二十斤,罷使職,權知楚州。遷戶部員外郎。代還,為度支判官,轉本曹郎中,出為兩浙轉運使,歷知懷州、滄州。



 會考課京朝官,召還。屬真宗以壽王尹京,精擇府僚,留為開封府
 推官。或言維嶽在淮南,決獄不平允,左右有知其事者辨之,太宗特加賞異。儲闈建,兼左諭德,轉太常少卿。京府事繁,維嶽評處詳敏。有王陟為司錄,真宗亦稱其明幹。及踐祚,即命維岳與畢士安權知開封府,拜給事中、知審官院。維嶽體肥年衰,艱於拜趨,陳乞外遷小州。上嘉其靜退,特授海州刺史。



 咸平初,知蘇州。素病風,上以吳中多食魚蟹,乃徙壽州,仍命太醫馳療之。四年,卒,年七十六。贈兵部侍郎,官給其葬。大中祥符中,錄其孫世
 昌、獻之,並賜同學究出身。維嶽明習吏事,有治劇才。在懷州,王欽若始舉進士,維嶽知其貴;又善待陳彭年,自刺郡連奏為通判,皆稱薦之。



 王陟者,潞州上黨人。淳化三年舉進士,補嵐州團練推官。內侍羅懷嗣言其督運有勞,遷晉州觀察推官。至道初,度支判官李擇言薦為著作佐郎、同判大名府,留知開封府司錄參軍。前司錄閻仲卿喜云為,屢升殿奏事,真宗尹京時頗不悅。及陟代之,以謹幹聞,尤被待遇。即
 位,召賜緋魚袋,改著作郎、開封府推官,乘傳陜西,與轉運使督饋靈武芻糧。



 咸平初,遷太常博士,出為河東轉運使,賜金紫。時趙保吉納款,屢遣與內侍張崇貴裁度邊事,正其經界,又副崇貴使夏州賜告命。代歸,會溫仲舒知貢舉,命陟與刑部員外郎董龜正同考試及封印卷首。俄改工部員外郎、知棣州。



 五年,召歸,判三司鹽鐵勾院。初,上以京府之舊,頗隆眷遇,將加擢用。會有言其在貢部,舉子有納賄成名者,恃恩寵,希顯要,僦大第以
 居,事遂寢。六年,卒。上甚憫之,錄其子若拙為奉禮郎,若谷為太廟齋郎。後陟妻卒,又命給其子奉,使終喪制。若拙官國子博士。



 張雍,德州安德縣人。治《毛氏詩》。開寶六年中第,釋褐東關尉。太平興國初,有薦其材者,召歸,改將作監丞、知南雄州。遷太子右贊善大夫、知開封府司錄參軍事,俄為秘書丞,充推官。



 京城民王元吉者,母劉早寡,有奸狀,為姻族所知,憂悸成疾。又懼元吉告之,遂遣侍婢訴元吉
 寘堇食中以毒己,病將死。事下右軍巡按之,未得實;移左軍巡,推吏受劉賂掠治,元吉自誣伏。俄而劉死,府慮囚,元吉始以實對。又移付司錄,盡捕元推吏,稍見誣構之跡。且以逮捕者眾,又獄已累月未能決,府中懼其淹,列狀引見,詔免死決徒。元吉大呼曰:「府中官吏悉受我賂,反使我受刑乎?」府不敢決,元吉歷陳所受賂主名,又令妻張擊登聞鼓訴之。上召張臨軒顧問,盡得其枉狀,立遣中使捕元推官吏,付御史鞫治。時滕中正為中丞,
 雍妻父也,詔供奉官蔚進別鞫之。雍坐與知府劉保勛、判官李繼凝初慮問,元吉稱冤,徙左軍巡,雍戒吏止令鞫其毒母狀,致吏訊掠慘暴。上怒,雍及左右軍巡判官韓昭裔、宋廷煦悉坐免所居官,保勛、繼凝各奪一等奉,左右軍巡使殿直龐則、王榮並降為殿前承旨。



 雍熙初,雍復為秘書丞、御史臺推直官,改鹽鐵推官,遷右補闕,充判官。端拱初,轉工部郎中、判度支勾院。未幾,又為鹽鐵判官兼判勾院。逾年,以本官兼侍御史知雜事。月餘,
 出為淮南轉運使。淳化初,選為太府少卿。二年,加右諫議大夫,徙兩浙轉運使,入知審刑院。三年,充戶部使,出知梓州,就命為西川轉運使,俄復知梓州。



 五年,蜀州青城民王小波、李順作亂,眾至萬人。雍訓練士卒,得城中兵三千餘人,又募強勇千餘守城,輦綿州金帛以實帑藏。推官陳世卿治戎器,掌書記施謂、榷鹽院判官謝濤伐山木為竿,銷銅鐘為箭鏑,紐布為索,守械悉備。遣推官盛梁請兵於朝。未幾,益綿邛彭漢州、永康軍悉陷於
 賊。順入成都,僭號大蜀王,勢甚盛,遣其黨楊廣將十萬眾寇劍門,相里貴帥眾十萬圍梓潼。雍與監軍盧斌登堞望之,賊所出兵,皆老弱疲憊,無鎧甲,斌笑請開北門擊之,雍曰:「不可,賊或詐見老弱,設伏伺我。又城中吏民心未定,脫為伏兵所突,則墮其奸計,非良策也。」言未畢,果有卒依敵樓呼嘯,與外應和,雍亟斬以徇。賊大設梯沖火車,晝夜鼓噪,攻城益急,城中大恐,雍命發機石碎之,火箭雜下。賊稍退,復治攻具城西北隅,雍紿曰:「軍士
 趣治裝,吾將開東門擊賊。」陽遣步騎五百臨東門。賊升牛頭山瞰城內,信然,伏精兵萬餘山之東隅以待我。雍即召敢死士百輩縋而下,盡焚其攻具,自午達申殆盡,賊以為神。兇黨數乘城進戰,皆不利。一日,北風晝晦,賊乘風縱火,急攻北門。雍與盧斌等領兵據門,立矢石間,固守不動,賊為之少卻。長圍八十餘日,會王繼恩遣石知顒來援,賊始潰去。遣施謂入奏,上手詔褒美,擢雍給事中,斌西京作坊使、領成州刺史,世卿掌書記,謂節
 度判官,濤觀察推官。又以通判將作監丞趙賀為太子中舍,監軍供奉官辛規為內殿崇班。



 至道二年,改工部侍郎。明年召歸,復知永興軍,轉禮部侍郎,改刑部,充度支使。咸平四年,遷鹽鐵使。上以雍齪齪小心,三司事重,宜有裁制,乃用王嗣宗代之。又以其無過,特拜戶部侍郎,復知審刑院,出知秦州,徙鳳翔府。



 景德初,權知開封府事。上覽奏獄,京府囚二百餘人,以為淹系,遣給事中董儼、直昭文館韓國華同慮問,決遣之。三年,改兵部侍郎、
 同知審官院。明年,車駕朝陵,判留司尚書省,出知鄧州。大中祥符元年,請老,以尚書右丞致仕,誥命未至而卒,年七十。



 雍性鄙吝,蒞事勤恪,善為米鹽苛察以肅下,恃其清幹,受遇於時,益矯厲以取名譽。所至藩鎮宴犒,率皆裁節;聚公錢為羨餘,以輸官帑;集會賓佐,糲食而已。在三司置簿籍,有「桉前急」、「馬前急」、「急中急」之目,頗為時論所誚。雍姿貌魯樸,始登科,為滕中正婿,中正子錫、世寧咸笑之。中正曰:「此人異日必顯達壽考,非汝曹所及。」
 錫兄弟雖有名,然終不越郎署,亦無耆年者。子太沖,官殿中丞。



 董儼,字望之,河南洛陽人。太平興國三年進士,解褐大理評事、通判饒州,加著作佐郎。五年,授左拾遺、直史館。轉右補闕,充淮南西路轉運副使。會罷使,就命知光州。儼狂躁務進,不樂外郡,上書乞還京師。太宗怒,降為秘書丞,削史館職,徙知忠州。復為右補闕,俄復直史館。會並水陸發運為一,儼與王繼升同領其事,就轉刑部員
 外郎。



 端拱初,進郎中、三司度支副使。坐翟馬周事,左授海州團練副使,移知泰州。逾年,以戶部員外郎知泉州,召為京東轉運使。時三司改易制度,置三計使,因留拜右諫議大夫,充右計使。使罷,出知揚州,遷右諫議大夫。徙潭州,轉給事中,歷知廣岳洪三州、江陵府。



 景德中,歸朝。會開封府系囚二百餘人,朝議以其稽滯,命儼與韓國華、張雍同慮問,裁決之。俄判吏部銓,加工部侍郎。時黃觀罷西川轉運歸闕,儼與知雜御史王濟姻家,因托
 濟言於觀,求薦己知益州。未幾,觀復領陜西轉運,得對便殿,儼謂其必薦己。他日,面陳:「自以孤直不為權要所容,況黃觀庸淺無操持,恐為執政所使,妄有論薦,俾臣遠適,惟陛下察之。」真宗不之詰。數日,王濟得對,因述儼嘗有私托,且言:「儼性本矯詐,臣語觀不可許之。」真宗不欲暴其事,乃出儼知青州。儼復請對,言為權臣所擯,上慰遣之,久而不去,乃謂之曰:「爾自告黃觀求知益州,復有何人排斥乎?」儼即矍然,且言:「觀、濟嘗議益州須得臣
 往彈壓之。」上以其詞不類,因令條析以聞,復遣使陜西質問黃觀。觀具述儼托王濟求薦之事,且言儼素待臣非厚。初,淳化中,儼為計使,觀為判官。儼知觀不飲酒,一日聚食,親酌以勸觀,觀為強飲之。有頃,都監趙贊召觀議事,觀即往。贊曰:「飲酒耶?」觀以實對。翌日,儼與贊密奏觀嗜酒廢職,故觀因是及之。乃詔樞密直學士劉綜與御史雜治之,儼方引伏,坐責授山南東道節度行軍司馬,不署州事。



 大中祥符初,會赦,起知郢州,病疽卒,年五
 十四。儼俊辯有材幹,不學無操行,所至厚納貨賂。嘗令引贊吏改制朱衣,每夕納儼第,而潛以輕帛制衣易之。在銓司,命胥吏市物,及請其直,則呵責之,其鄙屑如此。又廣畜姬媵,頗事豪侈。用傾狡圖位,終以是敗,士大夫醜之。東封恩,復其官。子仲容、仲宗,並為太子中舍。兄偉至殿中丞致仕。



 魏廷式,字君憲,大名宗城人。少明法學。嘗客游趙州,舍於監軍魏咸美之廨,廨有西堂,素兇,咸美知廷式有膽
 氣,命居之,卒無恙。來京師,咸美弟咸信延置館舍,以同宗善待之。太平興國五年中第,釋褐朗州法曹掾。轉運使李惟清以其吏材奏,知桃源縣,遷將作監丞。端拱初,改著作佐郎、通判穎州。



 淳化二年,始命李昌齡判審刑院,以廷式明練刑章,奏為詳議官。屢進對,太宗悅其明辨,遷太子左贊善大夫。時初較廷臣殿最,命廷式與樞密都承旨趙鎔、李著同勾當三班,多所規制。越王生日,令持禮物賜之,超拜主客員外郎、判三司都勾院,換河
 南東道判官,改戶部員外郎、知利州。



 李順為盜,就命充陜西至益州路轉運使。後入奏事,太宗謂曰:「有事當白中書。」廷式曰:「臣三千七百里外乘驛而至,以機事上聞,願取斷宸衷,非為宰相來也。」即不時召對,問方略稱旨,賜錢五十萬,令還任。賊平,知寧州,未至,召入判大理寺。



 至道初,乘傳河朔決獄,復出知宋、潭二州。湖南地土衍沃,民喜訟產,有根柢巧偽難辨者,廷式立裁之,吏民咸服。轉吏部員外郎、知桂州,歷工部郎中。真宗即位,改刑
 部。會王繼恩有罪下吏,命廷式同按之,逾宿而獄具。俄知審官院、通進銀臺封駁司,拜右諫議大夫、知審刑院,出知涇州。咸平二年卒,年四十九。錄其子攝太常寺太祝舜卿為太祝,禹卿同學究出身。



 廷式所至,以嚴明稱,剛果敢言,為人主厚遇,然性傾險,喜中傷人,士君子憚其口而鄙其行。



 盧琰,字錫珪,淄州淄川人。父浚,右諫議大夫。琰太平興國八年進士舉,解褐歷城主簿。歷大理評事、知安吉縣。
 三遷太常丞、通判並州。至道中,就加太常博士。咸平二年,選為開封府判官,與推官李防並命。真宗謂宰相曰:「人之有材,難得盡知,但歷試而後可見。」占謝日,特升殿,諭以天府事繁慎選之意,仍賜緡錢。會獄空,有詔獎之。遷工部員外郎,為河北轉運副使。



 時北鄙未寧,調發軍儲,糧道不絕。以職務修舉,召入,遷秩刑部,賜金紫,復遣之任。會城祁州,命專董其役。契丹入邊,車駕幸澶州,琰自定州隨軍至大名,即單騎赴行在。召對,勞問久之。其
 子士宗時為隰州推官,特遷大理寺丞。契丹請和,琰上言領職六年,求歸闕,許之。以使勞,優拜吏部員外郎、判三司三勾院。會宋摶使契丹,命權戶部副使。時議東封,又權京東轉運使,往營頓置。加戶部郎中,復判三勾院。



 大中祥符二年,以本官兼侍御史知雜事。數月,授三司度支副使。祀汾陰歲,命與鮑中和同判留守司三司,加吏部郎中,俄拜右諫議大夫、知永興軍府。五年,再為河北轉運使。琰勤於吏職,所至以干集聞。頗知命,嘗語親舊
 曰:「官五品,服三品,天不與者壽爾。」明年被疾,詔遣中使將太醫診視。六年,卒,年五十九。時琰母八十餘,無恙,上憫之,以士宗為太常博士,特命知懷州;又以次子秘書丞士倫為太常博士,給祿終喪。士倫至工部郎中、度支副使,士宗自有傳。



 宋摶,字鵬舉,萊州掖人。治《毛氏詩》。開寶八年,宋準典貢部,得第,調補遂寧尉。歷濰州司理參軍,改白龍令。膳部員外郎鞠礪薦其能,遷右贊善大夫、知利豐監,徙知藤
 州。改殿中丞、通判洪州。復有薦者,召還,命提點河北西路刑獄,未行,改監左藏庫。遷國子博士、通判西京留守司,得對便坐,賜錢三十萬。久之,徙江南轉運使,就遷度支員外郎。



 真宗嗣位,遷司封員外郎、河東轉運使。上言:「大通監冶鐵盈積,可備諸州軍數十年鼓鑄,願權罷採以紓民。」又請科諸州丁壯為兵,以增戎備。在任凡十一年。河東接西北境,時邊事未息,屯師甚廣,摶經制漕運,以乾治稱。連他徙,州郡輒乞留,有詔褒飭。兩至夏州界
 部發居民,數詣闕奏事稱旨。屢以秩滿請代,朝議以摶善職,就加祠部郎中,賜金紫。嘗薦代州承受使臣王白,上以本置此職,止於視軍政、察邊事,摶不應保奏。因詔諸路,自今勿得舉承受使臣。



 景德四年,入判三司勾院,逾月,為戶部副使。大中祥符初,進秩刑部郎中,俄使契丹,會疾,契丹主以車迎之。二年,卒,年六十六。子可法至太子中舍,舜元登進士第。摶卒,舜元自筠州判官改著作佐郎。又賜其孫出身。



 凌策,字子奇,宣州涇人。世給事州縣。策幼孤,獨厲志好學,宗族初不加禮,因決意渡江,與姚鉉同學於廬州。雍熙二年舉進士,起家廣安軍判官。改西川節度推官,以強幹聞。淳化三年,就命為光祿寺丞,簽書兩使判官。代還,拜左贊善大夫、通判定州,賜朱衣、銀章、御書歷,給以實奉。李順之亂,川陜選官多憚行,策自陳三蒞蜀境,諳其民俗,即命知蜀州。又以巴西當益之餫道,徙綿州,加太常博士。



 還朝,會命為廣南西路轉運使,進屯田員外
 郎。入為戶部判官,遷都官。先是,嶺南輸香藥,以郵置卒萬人,分鋪二百,負簷抵京師,且以煩役為患。詔策規制之,策請陸運至南安,泛舟而北,止役卒八百,大省轉送之費。盧之翰任廣州,無廉稱,以策有干名,拜職方員外郎、直史館,命代之,賜金紫。廣、英路自吉河趣板步二百里,當盛夏時瘴起,行旅死者十八九。策請由英州大源洞伐山開道,直抵曲江,人以為便。代還,知青州。東封,以供億之勤,超拜都官郎中,入判三司三勾院,出知揚州。
 屬江、淮歲儉,頗有盜賊,以策領淮南東路安撫使。駕旋,使停,進秩司封。時洪州水,知州李玄病,上與宰相歷選朝士,將徙策代之。上曰:「南昌水潦艱殆,長吏當便宜從事,不必稟於外計也。」王旦言:「策蒞事和平,可寄方面,望即以江南轉運使授之,仍詔諭差選之意。」饒州產金,嘗禁商市鬻,或有論告,逮系滿獄。策請縱民販市,官責其算,人甚便之。五年,召拜右諫議大夫、集賢殿學士、知益州。初,策登第,夢人以六印加劍上遺之,其後往劍外凡
 六任,時以為異。策勤吏職,處事精審,所至有治跡。



 九年,自蜀代還,上頗有意擢用,會已病,命知通進、銀臺司兼門下封駁事,糾察在京刑獄。真宗嘗對王旦言:「策有才用,治蜀敏而有斷。」旦曰:「策性淳質和,臨事強濟。」上深然之。是秋,拜給事中、權御史中丞。時榷茶之法弊甚,詔與翰林學士李迪、知雜御史呂夷簡同議經制,稍寬其舊。



 明年,疾甚,不能朝謁,累遣中使挾醫存問,賜名藥。復表求典益,尋遷工部侍郎,從其請。天禧二年三月卒,年六
 十二。錄其子將作監主簿瓘、琬並為奉禮郎,續給其奉。策兄簡,官國子博士,分司南京。



 楊覃,字申錫,漢太尉震之後。唐有京兆尹憑居履道坊,僕射於陵居新昌坊,刑部尚書汝士居靖恭坊,時稱「三楊」,皆為盛門,而靖恭尤著。汝士弟虞卿、漢公、魯士皆顯名。虞卿至工部侍郎、京兆尹,生堪,為太子少師。堪生承休,昭宗朝,以兵部員外郎使吳越,會楊行密據淮甸,絕其歸路,因留浙中。承休生巖,即覃祖也,署為鎮海軍節
 度副使,奏領春州刺史。巖生鬱,早卒。



 覃少獻書於嗣王俶,俶私署著作佐郎,從俶歸朝,為禹城尉。太平興國八年,舉進士擢第,授徐州觀察推官,改著作佐郎、知戎州。再遷太常博士,使陜西,蠲逋負。覃本名蟫,至是,太宗為改焉。淳化中,轉屯田員外郎、同判壽州。巡撫使潘慎修上其政績,有詔嘉獎,就命知州事。數月,召還,未上道,會丁內艱,州民列狀乞留,轉運使以聞,有詔奪情。



 時田重進為永興節度,選覃與林特同判軍府事,賜覃緋魚,仍
 賜御書歷,給以實奉。重進不法,覃事多抗執,重進頗不悅,形於辭色。覃表求徙任,不許,就轉都官員外郎。時討李繼遷,調發芻糧,覃、特皆以苛急促辦為務。覃令鉗手,特令即械頸,雖衣冠舊族不免,人用怨嗟。改職方員外郎。



 咸平初,遷屯田郎中、三門發運使。呂蒙正在河南,薦其材,詔入判三司磨勘、憑由、理欠司。四年春,旱,覃上言:「古之用刑,皆避三統之月,漢舊章斷獄報重,盡三冬之月。又唐太宗凡斷重刑日,敕減膳徹樂。今春物方盛,時雨
 尚愆,輦觳之下,獄系甚繁。望詔有司,死罪未得論決,俟雨降,乃復常典。仍望自今凡決重刑日,依唐故事,以彰至仁之德。」嘗獻《時務策》五篇:一曰御戎,二曰用兵,三曰為政,四曰選賢,五曰刑罰。文多不載。



 明年,權同知貢舉,出為陜西轉運使,賜金紫。會邊臣言繼遷死,願乘此時深入致討。覃建議:「伐喪非禮,且其子尚在,當為之備。請詔邊臣謹守疆候,毋得輕舉,俟其眾叛親離,則亡無日矣。」時西鄙屯兵,調役甚繁,副使朱臺符務有為,而覃務
 循舊,且言邊事不宜更張。初,寇準知青州,臺符為通判。至是,準作相,覃意臺符憑恃僚舊,密以上聞。坐不協,徙知隨州。王超節制漢東,覃移唐州。



 景德二年,召歸。屬河北兵革之後,命覃詣澶、濱、棣、德、博州巡撫振給之。出知潭州,王師討宜賊,軍須多出長沙,曹利用以聞,詔書褒勞,加刑部郎中。大中祥符二年,代馮亮為淮南、江、浙、荊湖制置發運使。月餘,改太常少卿、直昭文館、知廣州。



 覃勤於吏事,所至以幹濟稱。南海有蕃舶之利,前後牧守
 或致謗議,惟覃以廉著,遠人便之。加右諫議大夫。四年,卒,年五十四。遣其長子奉禮郎文友乘傳赴喪,詔本州護柩還其家,官給所費。錄其次子文敏為揚州司士參軍。覃從弟蛻及從子侃、傅,並登進士第。蛻官司封員外郎,侃後名大雅,自有傳。



 陳世卿字光遠,南劍人。雍熙二年,登進士第,解褐衡州推官。再調東川節度推官。會李順寇兩川,知州張雍以州兵馬為數部,使官分領。世卿素善射,當城一面,親射
 中數百人。賊寢盛,同幕皆謀圖全計。世卿正色曰:「食君祿,當委身報國,奈何欲避難為他圖耶?」亟出白雍曰:「此徒皆懦儒,存之適足惑眾,不若遣出求援。」雍從之。賊既引去,世卿適丁外艱,雍表其材,詔追出視事,就改掌書記。凡七年,歸朝,為秘書郎,遷太常丞、知新安縣。或薦其堪任憲臺,即召歸,會張金監出知廣州,表為通判。將行,召見,賜緋,加太常博士。



 景德初,徙知建州。真宗知其材幹,逾月,授福轉建運使,規畫南劍州安仁等銀場,歲增課
 羨,詔獎之。俄代姚鉉為兩浙路轉運使,歷祠部員外郎,判三司三勾院。大中祥符四年,改度支員外郎,出為荊湖北路轉運使。屬澧州慈利縣下溪等四州蠻人侵縣境地四百餘里,朝命世卿與閣門祗候史方、知澧州劉仁霸同領兵討之,遂還所侵地,標正經界,取其要領,又令納所掠漢口千餘,復置澧川、武口等砦以控制之,自是平定,有詔嘉獎。還朝,屢述溪洞利害。召對,真宗器其材,復自言願效用於煩劇。會邵曄知廣州,被疾,乃授世
 卿秘書少監代之,加賜金紫。郡有計口買鹽之制,人多不便,至,即奏除之。九年,卒,年六十四。錄其子南安主簿儼為太祝。



 李若拙,字藏用,京兆萬年人。父光贊,貝、冀觀察判官。若拙初以蔭補太廟齋郎,復舉拔萃,授大名府戶曹參軍。時符彥卿在鎮,光贊居幕下,若拙得以就養。俄又舉進士,王祐典貢舉,擢上第,授密州防禦推官。登賢良方正直言極諫科,太祖嘉其敏贍,改著作佐郎。故事,制策中
 選者除拾遺、補闕。若拙以恩例不及,上書自陳,執政惡之,出監商州坑冶。遷太子左贊善大夫,以官稱與父名同,辭,不許。太平興國二年,知幹州,會李飛雄詐乘驛稱詔使,事敗伏法。太宗以若拙與飛雄父若愚連名,疑其昆弟,命殿直盧令珣即捕系州獄,乃與若愚同宗,通家非親,不知其謀,猶坐削籍流海島。歲餘,起授衛尉寺丞、知隴州。



 四年,復舊官。以政聞,超授監察御史、通判泰州。同帥宋偓年老政弛,又徙若拙通判焉。未幾,御史中丞
 滕中正薦之,召歸臺。頃之,改右補闕。時諸王出閣,若拙獻頌稱旨,召見,賜緋魚,同勾當河東轉運兼雲、應等八州事。嘗詣闕言邊事,太宗嘉之。又同掌水陸發運司。



 雍熙三年,假秘書監使交州。先是,黎桓制度逾僭。若拙既入境,即遣左右戒以臣禮,繇是桓聽命,拜詔盡恭。燕饗日,以奇貨異物列於前,若拙一不留眄。取先陷蠻使鄧君辯以歸,禮幣外,不受其私覿。使還,上謂其不辱命。遷起居舍人,充鹽鐵判官。



 淳化二年,出為兩浙轉運使。契
 丹寇邊,改職方員外郎,徙河北路,賜金紫。五年,直昭文館,遷主客郎中、江南轉運使。若拙質狀魁偉,尚氣有幹才,然臨事太緩。宰相以為言,罷使知涇州。至道二年,黎桓復侵南鄙,又詔若拙充使,至,則桓復稟命。使還,真宗嗣位,召見慰問,進秩金部郎中。召試學士院,改兵部郎中,充史館修撰,俄知制誥。咸平初,同知貢舉,被疾,改右諫議大夫。車駕北巡,判留司御史臺。明年,使河朔按邊事,知升、貝二州。四年,卒,年五十八。子繹。



 繹字縱之,幼謹願自修。初,以父使交址有勞,補太廟齋郎,改太常寺太祝。舉進士中第,除將作監丞。累遷尚書屯田員外郎、知華州。蒲城民李蘊訴人盜其從子亡去,繹問曰:「若有仇耶?」曰:無有。」曰:「有失亡邪?」曰:「無有。」繹揮蘊去,因密刺蘊。蘊有陰罪,侄覺之,懼事暴,殺之以滅口。遂收蘊致法。擢提點河北刑獄,權知貝州。歲旱,繹為酒務,市民薪草溢常數,餓者皆以樵採自給,得不死,官入亦數倍。邊民歲輸防城火牛草十餘萬,委積久,輒腐敗,繹
 奏罷之。三遷本曹郎中,為利州路轉運使。



 河北經費不支,仁宗問誰可任者,參知政事薛奎薦繹,遂徙河北。進刑部郎中、直史館、知延州,改兵部,為江、淮制置發運使。內出絹五十萬匹,責貿於東南。繹曰:「百姓饑,不宜重擾。」輒奏罷之。甫半年,漕課視常歲增五之一。遷太常少卿,再知延州。繹所至頗稱治,自以久宦在外,意不自得,作《五知先生傳》,謂知時、知難、知命、知退、知足也。嘗兩知鳳翔府,至是,又徙鳳翔。尋為右諫議大夫,卒。



 陳知微,字希顏,高郵人。咸平五年,進士甲科,解褐將作監丞、通判歙州。擢為著作佐郎、直史館,俄充三司戶部判官。奉使契丹,遷太常博士、判三司都磨勘司,再為戶部判官,出為京東轉運副使,奏還東平監所侵民田六百八十家。又決古廣濟河通運路,罷夾黃河,歲減夫役數萬計。



 遷右司諫,徙荊湖南路轉運使。召還,拜比部員外郎、知制誥。淮南饑,遣知微巡撫,所至按視儲糧,察諸官吏能否。使還,判吏部銓,兼刑部。知微祠藻雖無奇
 採,而平雅適用。一日,進改群官,除目紛委,適當知微次直,思亦敏速。又判司農寺,糾察在京刑獄。天禧二年,加玉清昭應宮判官,俄以疾聞,真宗遣中貴挾太醫往視之。卒,年五十。錄其子舜卿為太常奉禮郎,給奉終喪,又假官船載其柩還鄉里。



 知微儀狀甚偉,沉厚有材幹,不務皦察,時人許其處劇,惜其母老不克終養。有集三十卷。子堯卿,大中祥符五年,進士及第。



 論曰:維嶽明習吏事,才足以治劇,而能曲全法掾,其仁
 恕藹然。雍雖素稱鄙吝,而勤恪清幹,觀其捍守,亦可見矣。儼務進黷貨,廷式傾險忌刻,自不容於清議。若琰、摶經制漕運有方,策之處事精詳,治跡昭著,覃之律身廉潔,兼勤吏事,世卿之安遠,若拙之專對,皆為時論所許。繹以謹願,克世其家,知微敦實有材幹,不辱其職,亦可尚也。至若王陟以謹乾稱,而取士以謗致污,惜哉!



\end{pinyinscope}