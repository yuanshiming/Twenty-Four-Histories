\article{本紀第一}

\begin{pinyinscope}

 太祖一



 太祖啟運立極英武睿文神德聖功至明大孝皇帝諱匡胤,姓趙氏,涿郡人也。高祖朓,是為僖祖,仕唐歷永清、文安、幽都令。朓生珽,是為順祖,歷藩鎮從事,累官兼御
 史中丞。珽生敬,是為翼祖,歷營、薊、涿三州刺史。敬生弘殷,是為宣祖。周顯德中,宣祖貴,贈敬左驍騎衛上將軍。宣祖少驍勇,善騎射,事趙王王鎔,為鎔將五百騎援唐莊宗於河上,有功。莊宗愛其勇,留典禁軍。漢乾祐中,討王景於鳳翔,會蜀兵來援,戰於陳倉。始合,矢集左目,氣彌盛,奮擊大敗之,以功遷護聖都指揮使。周廣順末,改鐵騎第一軍都指揮使,轉右廂都指揮,領岳州防禦使。從征淮南,前軍卻,吳人來乘,宣祖邀擊,敗之。顯德三年,
 督軍平揚州,與世宗會壽春。壽春賣餅家餅薄小,世宗怒,執十餘輩將誅之,宣祖固諫得釋。累官檢校司徒、天水縣男。與太祖分典禁兵,一時榮之。卒,贈武清軍節度使、太尉。



 太祖,宣祖仲子也,母杜氏。後唐天成二年,生於洛陽夾馬營,赤光繞室,異香經宿不散。體有金色,三日不變。既長,容貌雄偉,器度豁如,識者知其非常人。學騎射,輒出人上。嘗試惡馬,不施銜勒,馬逸上城斜道,額觸門楣墜地,人以為首必碎,太祖徐起,更追馬騰上,一無
 所傷。又嘗與韓令坤博土室中,雀鬥戶外,因競起掩雀,而室隨壞。漢初,漫游無所遇,舍襄陽僧寺。有老僧善術數,顧曰:「吾厚贐汝,北往則有遇矣。」會周祖以樞密使徵李守真,應募居帳下。廣順初,補東西班行首,拜滑州副指揮。世宗尹京,轉開封府馬直軍使。世宗即位,復典禁兵。北漢來寇,世宗率師御之,戰於高平。將合,指揮樊愛能等先遁,軍危。太祖麾同列馳馬沖其鋒,漢兵大潰。乘勝攻河東城,焚其門。左臂中流矢,世宗止之。還,拜殿前
 都虞候,領嚴州刺史。



 三年春,從征淮南,首敗萬眾於渦口,斬兵馬都監何延錫等。南唐節度皇甫暉、姚鳳眾號十五萬,塞清流關,擊走之。追至城下,暉曰:「人各為其主,願成列以決勝負。」太祖笑而許之。暉整陣出,太祖擁馬項直入,手刃暉中腦,並姚鳳禽之。宣祖率兵夜半至城下,傳呼開門,太祖曰:「父子固親,啟閉,王事也。」詰旦,乃得入。韓令坤平揚州,南唐來援,令坤議退,世宗命太祖率兵二千趨六合。太祖下令曰:「揚州兵敢有過六合者,斷其足!」
 令坤始固守。太祖尋敗齊王景達於六合東,斬首萬餘級。還,拜殿前都指揮使,尋拜定國軍節度使。



 四年春,從征壽春,拔連珠砦,遂下壽州。還,拜義成軍節度、檢校太保,仍殿前都指揮使。冬,從征濠、泗,為前鋒。時南唐砦於十八里灘,世宗方議以橐駝濟師,而太祖獨躍馬截流先渡,麾下騎隨之,遂破其砦。因其戰艦乘勝攻泗州,下之。南唐屯清口,太祖從世宗翼淮東下,夜追至山陽,俘唐節度使陳承詔以獻,遂拔楚州。進破唐人於水鑾江口,
 直抵南岸,焚其營柵,又破之於瓜步,淮南平。唐主畏太祖威名,用間於世宗,遣使遺太祖書,饋白金三千兩,太祖悉輸之內府,間乃不行。五年,改忠武軍節度使。



 六年,世宗北征,為水陸都部署。及莫州,先至瓦橋關,降其守將姚內斌,戰卻數千騎,關南平。世宗在道,閱四方文書,得韋囊,中有木三尺餘,題云「點檢作天子」,異之。時張德為點檢,世宗不豫,還京師,拜太祖檢校太傅、殿前都點檢,以代永德。恭帝即位,改歸德軍節度、檢校太尉。



 七
 年春,北漢結契丹入寇,命出師御之。次陳橋驛,軍中知星者苗訓引門吏楚昭輔視日下復有一日,黑光摩蕩者久之。夜五鼓,軍士集驛門,宣言策點檢為天子,或止之,眾不聽。遲明,逼寢所,太宗入白,太祖起。諸校露刃列於庭,曰:「諸軍無主,願策太尉為天子。」未及對,有以黃衣加太祖身,眾皆羅拜,呼萬歲,即掖太祖乘馬。太祖攬轡謂諸將曰:「我有號令,爾能從乎?」皆下馬曰:「唯命。」太祖曰:「太后、主上,吾皆北面事之,汝輩不得驚犯;大臣皆我比
 肩,不得侵凌;朝廷府庫、士庶之家,不得侵掠。用令有重賞,違即孥戮汝。」諸將皆載拜,肅隊以入。副都指揮使韓通謀禦之,王彥升遽殺通於其第。太祖進登明德門,令甲士歸營,乃退居公署。有頃,諸將擁宰相範質等至,太祖見之,嗚咽流涕曰:「違負天地,今至於此!」質等未及對,列校羅彥環按劍厲聲謂質等曰:「我輩無主,今日須得天子。」質等相顧,計無從出,乃降階列拜。召文武百僚,至晡,班定。翰林承旨陶穀出周恭帝禪位制書於袖中,宣
 徽使引太祖就庭,北面拜受已,乃掖太祖升崇元殿,服袞冕,即皇帝位。遷恭帝及符後於西宮,易其帝號曰鄭王,而尊符後為周太后。



 建隆元年春正月乙巳,大赦,改元,定有天下之號曰宋。賜內外百官軍士爵賞,貶降者敘復,流配者釋放,父母該恩者封贈。遣使遍告郡國。丙午,詔諭諸鎮將帥。戊申,賜書南唐。贈韓通中書令,命以禮收葬。己酉,遣官告祭天地社稷。復安州、華州、兗州為節度。辛亥,論翊戴功,以
 周義成軍節度使、殿前都指揮使石守信為歸德軍節度使、侍衛親軍馬步軍副都指揮使,江寧軍節度使、侍衛親軍馬軍都指揮使高懷德為義成軍節度使、殿前副都點檢,武信軍節度使、侍衛親軍步軍都指揮使張令鐸為鎮安軍節度使、侍衛親軍馬步軍都虞候,殿前都虞候王審琦為泰寧軍節度使、殿前都指揮使,虎捷右廂都虞候張光翰為江寧軍節度使、侍衛親軍馬軍都指揮使,龍捷右廂都指揮使趙彥徽為武信軍節度
 使,其餘領軍者並進爵。壬子,賜宰相、樞密、諸軍校襲衣、犀玉帶、鞍馬有差。癸丑,放南唐降將周成等歸國。乙卯,遣使分振諸州。丁巳,命周宗正郭□祀周陵廟,仍以時祭享。己未,宰相表請以二月十六日為長春節。癸亥,以周天雄軍節度使、魏王符彥卿守太師,雄武軍節度使王景守太保、太原郡王,定難軍節度使、守太傅、西平王李彞殷守太尉,荊南節度使高保融守太傅,餘領節鎮者並進爵。甲子,賜皇弟殿前都虞候匡義名光義。己巳,
 立太廟。鎮州郭崇報契丹與北漢軍皆遁。二月乙亥,尊母南陽郡夫人杜氏為皇太后。以周宰相範質依前守司徒、兼侍中,王溥守司空、兼門下侍郎、同中書門下平章事,魏仁浦為尚書右僕射、兼中書侍郎、同中書門下平章事,樞密使吳廷祚同中書門下二品。丙戌,長春節,賜群臣衣各一襲。三月乙巳,改天下郡縣之犯御名、廟諱者。丙辰,南唐主李景、吳越王錢俶遣使以御服、錦綺、金帛來賀。宿州火,遣使恤災。壬戌,定國運以火德王,色
 尚赤,臘用戌。癸亥,命武勝軍節度使宋延渥等率舟師巡江徼。是春,均、房、商、洛鼠食苗。夏四月癸酉,竇儼上二舞十二樂曲名、樂章。乙酉,幸玉津園。遣使分詣京城門,賜饑民粥。丙戌,浚蔡河。癸巳,昭義軍節度使李筠叛,遣歸德軍節度使石守信討之。五月己亥朔,日有食之。庚子,遣昭化軍節度使慕容延釗、彰德軍節度使王全斌將兵出東道,與守信會討李筠。壬寅,竇儼上太廟舞曲名。癸卯,石守信敗李筠於長平。甲辰,命諸道進討。丙午,
 幸魏仁浦第視疾。己酉,西京作周六廟成,遣官奉遷。丁巳,詔親征,以樞密使吳廷祚留守上都,都虞候光義為大內都點檢,命天平軍節度使韓令坤屯兵河陽。己未,發京師。丁卯,石守信、高懷德破筠眾於澤州,禽偽節度範守圖,殺北漢援兵之降者數千人,筠遁入澤州。戊辰,王師圍之。六月癸酉,有星赤色,出心。辛未,拔澤州,筠赴火死,命埋胔骼。釋河東相衛融,禁剽掠。甲申,免澤州今年租。有星赤色,出太微垣,歷上相。乙酉,伐上黨。丁亥,筠
 子守節以城降,赦之。上如潞。辛卯,大赦,減死罪,免附潞三十里今年租,錄陣歿將校子孫,丁夫給復三年。甲午,永安軍節度使折德扆破北漢沙谷砦。秋七月戊申,上至自潞。壬子,幸範質第視疾。甲子,遣工部侍郎艾穎拜嵩、慶陵。乙丑,南唐進白金,賀平澤、潞。丁卯,南唐進乘輿御服物。八月戊辰朔,御崇元殿,行入閣儀。辛未,遣郭□饗周廟。壬申,復貝州為永清軍節度。甲戌,命宰相禱雨。辛巳,以周武勝軍節度使侯章為太子太師。壬午,以光
 義領泰寧軍節度,依前殿前都虞候。甲申,立瑯琊郡夫人王氏為皇后。戊子,南唐進賀平澤潞金銀器、羅綺以千計。九月壬寅,昭義軍節度使李繼勛焚北漢平遙縣。癸卯,三佛齊國遣使貢方物。丙午,奉玉冊謚高祖曰文獻皇帝,廟號僖祖,高祖妣崔氏曰文懿皇后;曾祖曰惠元皇帝,廟號順祖,曾祖妣桑氏曰惠明皇后;祖曰簡恭皇帝,廟號翼祖,祖妣劉氏曰簡穆皇后;皇考曰武昭皇帝,廟號宣祖。己酉,幸宜春苑。中書舍人趙逢坐從征
 避難,貶房州司戶參軍。己未,淮南節度李重進以揚州叛,遣石守信等討之。甲子,歸太原俘。冬十月丁卯朔,賜內外文武官冬衣有差。壬申,定縣為望、緊、上、中、下,令三年一注。壬午,河決厭次。乙酉,晉州兵馬鈐轄荊罕儒襲北漢汾州,死之。龍捷指揮石進二十九人坐不救棄市。丁亥,詔親征揚州,以都虞候光義為大內都部署,樞密使吳廷祚權上都留守。戊子,詔諸道長貳有異政、眾舉留請立碑者,委參軍驗實以聞。庚寅,發京師。十一月丁
 未,師傅揚州城,拔之,重進盡室自焚。戊申,誅重進黨,揚州平。命諸軍習戰艦於迎鑾,南唐主懼甚,其臣杜著、薛良因詭跡來奔,帝疾其不忠,斬著下蜀市,配良廬州牙校。己酉,振揚州城中民人米一斛,十歲以下者半之。脅隸為軍者,賜衣屨遣還。庚戌,給攻城役夫死者人絹三匹,復三年。乙卯,南唐主遣使來犒師。庚申,遣其子從鎰來朝。



 十二月己巳,駕還。丁亥,上至自揚。辛卯,泉州節度使留從效稱藩。



 二年春正月丙申朔,上詣太后宮門稱慶。庚子,占城國王遣使來朝。壬寅,幸造船務觀習水戰。戊申,以揚州行宮為建隆寺。太僕少卿王承哲坐舉官失實,責授殿中丞。壬子,商州鼠食苗,詔免賦。謂宰臣曰:「比命使度田,多邀功弊民,當慎其選,以見朕意。」丁巳,導蔡水入穎。己未,遣郭□饗周廟。靈武節度使馮繼業獻馬五百、橐駝百、野馬十。甲子,澤州刺史張崇詁坐黨李重進棄市。二月丙寅,幸飛山營,閱炮車。壬申,疏五丈河。癸酉,有司奏進
 士合格者十一人。荊南高保勖進黃金什器。甲戌,幸城南,觀修水匱。丁丑,南唐進長春節御衣、金帶及金銀器。己卯,賜天雄軍節度符彥卿粟。禁春夏捕魚射鳥。己丑,定竊盜律。三月丙申,內酒坊火,酒工死者三十餘人,乘火為盜者五十人,擒斬三十八人,餘以宰臣諫獲免。酒坊使左承規、副使田處巖以酒工為盜,坐棄市。閏月己巳,幸玉津園。謂侍臣曰:「沉湎非令儀,朕宴偶醉,恆悔之。」壬辰,南唐進謝賜生辰金器、羅綺。丁丑,金、商、房三州饑,
 振之。癸未,幸迎春苑宴射。夏四月癸巳朔,日有食之。壬寅,詔郡國置前代帝王、賢臣陵塚戶。己酉,無棣男子趙遇詐稱皇弟,伏誅。己未,商河縣令李瑤坐贓杖死,左贊善大夫申文緯坐失覺察除籍。庚申,班私煉貨易鹽及貨造酒曲律。五月癸亥朔,以皇太后疾,赦雜犯死罪已下。乙丑,天狗墮西南。丙寅,三佛齊國來獻方物。丁丑,以安邑、解兩池鹽給徐、宿、鄆、濟。庚寅,供奉官李繼昭坐盜賣官船棄市。詔諸道郵傳以軍卒遞。六月甲午,皇太后
 崩於滋德殿。己亥,群臣請聽政,從之。庚子,以太后喪,權停時享。辛丑,見百官於紫宸殿門。壬子,祈雨。庚申,釋服。秋七月壬戌,以皇太后殯,不受朝。辛未,晉州神山縣穀水泛出鐵,方圓二丈三尺,重七千斤。壬申,以光義為開封府尹,光美行興元尹。己卯,隴州進黃鸚鵡。八月壬辰朔,不視朝。壬寅,詔諸大闢送所屬州軍決判。甲辰,南唐主李景死,子煜嗣,遣使請追尊帝號,從之。己酉,執易定節度使、同平章事孫行友,削官勒歸私第。辛亥,幸崇夏
 寺,觀修三門。女直國遣使來朝獻。大名府永濟主簿郭顗坐贓棄市。庚申,《周世宗實錄》成。九月壬戌朔,不御殿。南唐遣使來進金銀、繒彩。甲子,契丹解利來降。荊南節度使高保勖遣其弟保寅來朝。戊子,遣使南唐賻祭。冬十月癸巳,南唐遣其臣韓熙載、田霖來會皇太后葬。丙申,遣樞密承旨王仁贍賜南唐禮物。戊戌,禁邊民盜塞外馬。辛丑,丹州大雨、雹。丙午,葬明憲皇太后於安陵。十一月辛酉朔,不視朝。甲子,太后祔廟。己巳,幸相國寺,遂
 幸國子監。癸酉,沙州節度使來曹元忠、瓜州團練使曹延繼等遣使獻玉鞍勒馬。十二月壬申,回鶻可汗景瓊遣使獻方物。乙未,李繼勛敗北漢軍,俘遼州刺史傅廷彥、弟勛來獻。辛丑,幸新修河倉。庚戌,畋於近郊。癸丑,遣使賜南唐、吳越馬、羊、橐駝有差。



 三年春正月庚申朔,以喪不受朝賀。己已,淮南饑,振之。庚午,幸迎春苑宴射。甲戌,廣皇城。詔郡國長吏勸民播種。丙子,瓜沙歸義節度使曹元忠獻馬。庚辰,女直國遣使
 只骨來獻。詔郡國不得役道路居民。癸未,幸國子監。二月丙辰,復幸國子監,遂如迎春苑宴從官。庚寅,詔文班官舉堪為賓佐、令錄者各一人,不當者比事連坐。甲午,詔自今百官朝對,須陳時政利病,無以觸諱為懼。乙未,滑州節度使張建豐坐失火免官。己亥,更定竊盜律。壬午,上謂侍臣曰:「朕欲武臣盡讀書以通治道,何如?」左右不知所對。甲寅,北漢寇潞、晉,守將擊走之。三月戊午朔,厭次霣霜殺桑。壬戌,三佛齊國遣使來獻。癸亥,禱雨。
 丁卯,幸太清觀,遂幸開封尹後園宴射。己巳,大雨。詔申律文諭郡國,犯大闢者刑部審覆。乙亥,遣使賜南唐主生辰禮物。丁丑,女直國遣使來獻。丁亥,命徙北漢降人於邢、洺。夏四月乙未,延州大雨雪,趙、衛二州旱。丙申,寧州大雨雪,溝洫冰。戊戌,幸太清觀。庚子,回鶻阿督等來獻方物。壬寅,丹州雪二尺。乙巳,贈兄光濟為邕王,弟光贊為夔王,追冊夫人賀氏為皇后。五月甲子,幸相國寺禱雨,遂幸迎春苑宴射。乙亥,海州火。開太行運路。癸未,
 命使檢諸州旱。甲申,詔均戶役,敢蔽占者有罪。復幸相國寺禱雨。乙酉,廣大內。齊、博、德、相、霸五州自春不雨,以旱,減膳徹樂。六月辛卯,振宿州饑。癸巳,吳廷祚以雄武軍節度使罷。乙未,賜酒國子監。丁酉,幸太清觀。己亥,減京畿、河北死罪以下。壬寅,京師雨。壬子,蕃部尚波於等爭採造務,以兵犯渭北,知秦州高防擊走之。乙卯,幸迎春苑宴射。黃陂縣有象自南來食稼。秋七月庚申,南唐遣其臣翟如璧謝賜生辰禮,貢金銀、錦綺千萬。壬戌,放
 南唐降卒弱者數千人歸國。乙丑,免舒州菰蒲新稅。丁卯,潞州大雨、雹。索內外軍不律者配沙門島。己卯,北漢捉生指揮使路貴等來降。辛巳,遣從臣十人檢河北旱。癸未,兗、濟、德、磁、洺五州蝝。



 八月癸巳,蔡河務綱官王訓等四人坐以糠土雜軍糧,磔於市。乙未,用知制誥高錫言,諸行賂獲薦者許告訐,奴婢鄰親能告者賞。詔注諸道司法參軍皆以律疏試判。詔尚書吏部舉書判拔萃科。九月庚午,吐蕃尚波於等歸伏羌縣地。壬申,修武成
 王廟。丙子,占城國來獻。禁伐桑、棗。冬十月乙酉朔,賜百官冬服有差。丙戌,幸太清觀,遂幸造船務,觀習水戰。己亥,幸嶽臺,命諸軍習騎射,復幸玉津園。辛丑,以樞密副使趙普為樞密使。辛亥,畋近郊。十一月癸亥,禁奉使請托。縣令考課以戶口增減為黜陟。丙寅,南唐遣其臣顧彞來朝。丙子,三佛齊國遣使李麗林等來獻,高麗國遣李興祐等來朝。己卯,畋於近郊。壬午,賜南唐建隆四年歷。十二月丙戌,詔縣置尉一員,理盜訟。置弓手,視縣戶
 為差。戊戌,蒲、晉、慈、隰、相、衛六州饑,振之。庚子,班捕盜令。甲辰,衡州刺史張文表叛。是歲,周鄭王出居房州。



 乾德元年春正月甲寅朔,不御殿。乙卯,發關西鄉兵赴慶州。丁巳,修畿內河堤。己未,遣使賜南唐、吳越馬、橐駝、羊有差。庚申,遣山南東道節度使慕容延釗率十州兵以討張文表。乙丑,幸造船務,觀造戰船。甲戌,詔荊南發水卒三千應延釗於潭。己卯,女直國遣使來獻。



 二月壬辰,周保權將楊師璠梟文表於朗陵市。甲午,慕容延釗
 入荊南,高繼沖請歸朝,得州三、縣十七。乙未,克潭州。辛亥,澶、滑、衛、魏、晉、絳、蒲、孟八州饑,命發廩振之。三月辛未,幸金鳳園習射,七發皆中。符彥卿等進馬稱賀,乃遍賜從臣名馬、銀器有差。壬申,高繼沖籍其錢帛芻粟來上。癸酉,班新定律。戊寅,慕容延釗破三江口,下嶽州,克復朗州,湖南平。得州十四、監一、縣六十六。夏四月,旱。甲申,遍禱京城祠廟,夕雨。減荊南朗州、潭州管內死罪一等,鹵掠者給主。乙酉,遣使祭南嶽。丁亥,幸國子監,遂幸武
 成王廟,宴射玉津園。庚寅,出內錢募諸軍子弟鑿習戰池。辛卯,《建隆應天歷》成,禦制序。壬辰,賞湖南立功將士。癸巳,幸玉津園。丙申,兵部郎中曹匪躬棄市,海陵鹽城屯田副使張藹除名,並坐不法。庚子,荊南節度使高繼沖進助宴金銀、羅紈、柱衣、屏風等物。癸卯,辰、錦、敘等州歸順。甲辰,詔疏鑿三門。禁涇、原、邠、慶等州補蕃人為邊鎮將。夏西平王李彞興獻犛牛一。乙巳,幸玉津園,閱諸軍騎射。丙午,免湖南茶稅,禁陜州鹽井。辛亥,貸澶州民
 種食。五月壬子朔,禱雨京城。甲寅,遣使禱雨岳瀆。乙丑,廣大內。庚午,給荊南管內符印。癸酉,幸玉津園。六月乙酉,免潭州諸縣無名配斂。壬辰,暑,罷營造,賜工匠衫履。乙未,詔荊南兵願歸農者聽。丙申,詔歷代帝王三年一饗,立漢光武、唐太宗廟。己亥,澶、濮、曹、絳蝗,命以牢祭。庚子,百官三上表請舉樂,從之。減左右仗千牛員。丙午,雨。詔蠟祀,廟、社皆用戌臘一日。己酉,命習水戰於新池。秋七月辛亥朔,定州縣所置雜職、承符、廳子等名數。甲寅,
 以湖湘歿王事靳彥朗男承勛等三十人補殿直。丙辰,幸新池,賜役夫錢,遂幸玉津園。丁巳,安國軍節度使王全斌等率兵入太原境,以俘來獻,給錢米以釋之。己未,詔民有疾而親屬遺去者罪之。癸亥,湖南疫,賜行營將校藥。丁卯,幸武成王廟,遂幸新池,觀習水戰。己巳,朗州賊將汪端寇州城,都監尹重睿擊走之。詔免荊南管內夏稅之半。甲戌,釋周保權罪。乙亥,詔繕朗州城,免其管內夏稅。丁丑,分命近臣禱雨。己卯,班《復位刑統》等書。八
 月壬午,殿前都虞候張瓊以陵侮軍校史珪、石漢卿等,為所誣譖,下吏,瓊自殺。丙戌,遣給事中劉載朝拜安陵。丁亥,王全斌攻北漢樂平縣,降之。辛卯,以樂平縣為平晉軍,降卒千八百人為效順軍人,賜錢帛。壬辰,詔九經舉人下第者再試。癸巳,女直國遣使獻名馬。蠲登州沙門島民稅,令專治船渡馬。丙申,北漢靜陽十八砦首領來降。泉州陳洪進遣使來朝貢。齊州河決。京師雨。己亥,契丹幽州岐溝關使柴廷翰等來降。癸卯,宰相質率百
 官上尊號,不允。九月甲寅,三上表請,從之。丙寅,宴廣政殿,始用樂。丁卯,責宣徽南院使兼樞密副使李處耘為淄州刺史。戊辰,女直國遣使獻海東青名鷹。丙子,禁朝臣公薦貢舉人。賜南唐羊萬口。磔汪端於朗州。戊寅,北漢引契丹兵攻平晉,遣洺州防禦使郭進等救之。冬十月庚辰,詔州縣徵科置簿籍。己亥,畋近郊。丁未,吳越國王進郊祀禮金銀、珠器、犀象、香藥皆萬計。十一月乙卯,荊南節度使高繼沖進郊祀銀萬兩。甲子,有事南郊,大
 赦,改元乾德。百官奉玉冊上尊號曰應天廣運仁聖文武至德皇帝。丙寅,南唐進賀南郊尊號、銀絹萬計。丁卯,賜近臣襲衣、金帶、器幣、鞍馬有差。乙亥,畋近郊。十二月庚辰,殿前祗候李璘以父仇殺員僚陳友,璘自首,義而釋之。辛巳,開封府尹光義、興元尹光美各益食邑,賜功臣號;宰相質、溥、仁浦並特進,易封,益食邑;樞密使普加光祿大夫,易功臣號;文武臣僚各進階、勛、爵、邑。甲申,皇后王氏崩。辛卯,罷登州都督。己亥,泉州陳洪進遣使貢
 白金千兩,乳香、茶藥皆萬計。己巳,南唐主上表乞呼名,詔不允。閏月己酉朔,校醫官,黜其藝不精者二十二人。甲寅,命近臣祈雪。丁卯,覆試拔萃科,田可封、宋白、譚利用等稱旨,賜與有差。辛未,卜安陵於鞏縣。乙亥,折德扆敗北漢軍於府州城下,禽其將楊璘。以太常議,奉赤帝為感生帝。



 二年春正月辛巳,諭郡國長吏勸農耕作。有象入南陽,虞人殺之,以齒、革來獻。京師雨雪、雷。癸未,幸迎春苑宴
 射。甲申,詔著四時聽選式。回鶻遣使獻方物。戊子,質以太子太傅、溥以太子太保、仁浦仍尚書左僕射罷。庚寅,以趙普為門下侍郎、同中書門下平章事,李崇矩樞密使。壬辰,詔親試制舉三科,不限官庶,許直詣閣門進狀。甲辰,詔諸道獄詞令大理、刑部檢詳,或淹留差失致中書門下改正者,重其罪。乙已,幸玉津園宴射。丁未,詔縣令、簿、尉非公事毋至村落。令、錄、簿、尉諸職官有耄耋篤疾者舉劾之。二月戊申朔,北漢遼州刺史杜延韜以城
 來降。癸丑,遣使振陜州饑。導潩水入京。丁巳,治安陵,隧壞,役兵壓死者二百人,命有司瘞恤。庚午,府州俘北漢衛州刺史楊璘來獻。甲戌,南唐進改葬安陵銀綾絹各萬計。浚汴河。三月辛巳,幸教船池,賜水軍將士衣有差,還,幸玉津園宴射。乙未,北漢耀州團練使周審玉等來降。丁酉,遣使祈雨於五嶽。禁臣僚往來假官軍部送。辛丑,遣攝太尉光義奉冊寶上明憲皇太后謚曰昭憲,皇后賀氏謚曰孝惠,王氏謚曰孝明。夏四月丁未朔,策
 賢良方正直言極諫科,博州判官穎贄中第。戊申,振河中饑。己酉,免諸道今年夏稅之無苗者。乙卯,葬昭憲皇太后、孝明皇后於安陵。乙丑,始置參知政事,以兵部侍郎薛居正、呂餘慶為之。己已,靈武饑,轉涇粟以餉。壬申,祔二后於別廟。徙永州諸縣民之畜蠱者三百二十六家於縣之僻處,不得復齒於鄉。五月己卯,知制誥高錫坐受藩鎮賂,貶萊州司馬。辛巳,宗正卿趙礪坐贓杖、除籍。癸未,幸玉津園宴射。六月己酉,以光義為中書令,光
 美同中書門下平章事,子德昭貴州防禦使。庚申,幸相國寺,遂幸教船池、玉津園。辛未,河南、北及秦諸州蝗,惟趙州不食稼。秋七月乙亥,春州暴水溺民。庚辰,合陽雨雹。辛巳,幸玉津園。還,幸新池,觀習水戰。辛卯,詔翰林學士陶谷、竇儀舉堪為藩郡通判者各一人,不當者連坐。九月甲戌朔,《周易》博士奚嶼責乾州司戶,庫部員外王貽孫責左贊善大夫,並坐試任子不公。戊子,延州雨雹。乙未,幸北郊觀稼。辛丑,太子太傅質薨。壬寅,潘美等
 克郴州。冬十月戊申,周紀王熙謹薨。輟視朝。十一月甲戌,命忠武軍節度使王全斌為西川行營前軍兵馬都部署,武信軍節度崔彥進副之,將步騎三萬出鳳州道;江寧軍節度使劉光義為西川行營前軍兵馬副都部署,樞密承旨曹彬副之,將步騎二萬出歸州道以伐蜀。乙亥,宴西川行營將校於崇德殿,示川峽地圖,授攻取方略,賜金玉帶、衣物各有差。壬辰,畋近郊。十二月乙巳,釋廣南郴州都監陳琄等二百人。戊申,劉光義拔夔州,蜀
 節度高彥儔自焚。丁巳,蠲歸、峽秋稅。辛酉,王全斌克萬仞、燕子二砦,下興州,連拔石圌等二十餘砦。甲子,光義拔巫山等砦,斬蜀將南光海等八千級,禽其戰□翟都指揮袁德宏等千二百人。全斌先鋒史進德敗蜀人於三泉砦,禽其節度使韓保正、李進等。南唐進銀二萬兩、金銀器皿數百事。庚午,詔招復山林聚匿。辛未,畋北郊。



\end{pinyinscope}