\article{列傳第五十二}

\begin{pinyinscope}

 田錫王禹偁張詠



 田錫,字表聖,嘉州洪雅人。幼聰悟,好讀書屬文。楊徽之宰峨眉,宋白宰玉津,皆厚遇之,為之延譽,繇是聲稱翕然。太平興國三年,進士高等,釋褐將作監丞、通判宣州。
 遷著作郎、京西北路轉運判官。改左拾遺、直史館,賜緋魚。錫好言時務,既居諫官,即上疏獻軍國要機者一、朝廷大體者四。其略曰:



 頃歲王師平太原,未賞軍功,迄今二載。幽燕竊據,固當用兵,雖稟宸謀,必資武力。願陛下因郊禋、耕籍之禮,議平戩之功,則駕馭戎臣,莫茲為重,此要機也。



 今交州未下,戰士無功,《春秋》所謂「老師費財」者是也。臣聞聖人不務廣疆士,惟務廣德業,聲教遠被,自當來賓。周成王時,越裳九譯來貢,且曰:「天無迅風疾
 雨、海不揚波三年矣。意者中國其有聖人乎?盍往朝之。」交州瘴海,得之如獲石田,臣願陛下務修德以來遠,無鈍兵以挫銳,又何必以蕞爾蠻夷,上勞震怒乎?此大體之一也。



 今諫官不聞廷爭,給事中不聞封駁,左右史不聞升陛軒、記言動,豈聖朝美事乎?又御史不敢彈奏,中書舍人未嘗訪以政事,集賢院雖有書籍而無職官,秘書省雖有職官而無圖籍。臣願陛下擇才任人,使各司其局,茍職業修舉,則威儀自嚴。此大體之二也。



 爾者寓
 縣平寧,京師富庶。軍營馬監,靡不恢崇;佛寺道宮,悉皆輪奐。加又闢西苑,廣御池,雖周之靈囿,漢之昆明,未足為比。而尚書省湫隘尤甚,郎曹無本局,尚書無聽事。九寺三監,寓天街之兩廊,貢院就武成王廟,是豈太平之制度邪?臣願陛下別修省寺,用列職官。此大體之三也。



 案獄官令,枷杻有短長,鉗鎖有輕重,尺寸斤兩,並載刑書,未聞以鐵為枷者也。昔唐太宗觀《明堂圖》,見人之五藏皆麗於背,遂減徒刑。況隆平之時,將措刑不用,於法
 所無,去之可矣。此大體之四也。



 疏奏,優詔褒答,賜錢五十萬。僚友謂錫曰:「今日之事鮮矣,宜少晦以遠讒忌。」錫曰:「事君之誠,惟恐不竭,矧天植其性,豈為一賞奪邪?」時趙普為相,令有司受群臣章奏,必先白錫。錫貽書於普,以為失至公之體,普引咎謝之。



 六年,為河北轉運副使,驛書言邊事曰:



 臣聞動靜之機,不可妄舉;安危之理,不可輕言。利害相生,變易不定;取舍無惑,思慮必精。夫動靜之機,不可妄舉者,動謂用兵,靜謂持重。應動而靜,則
 養寇以生奸;應靜而動,則失時以敗事。動靜中節,乃得其宜。今北鄙繹騷,蓋亦有以居邊任者,規羊馬細利為捷,矜捕斬小勝為功,賈怨結仇,興戎致寇,職此之由。前歲邊陲俶擾,親迂革輅,戎騎既退,萬乘方歸。是皆失我機先,落其術內,勞煩耗斁,可勝言哉。伏願申飭將帥,慎固封守,勿尚小功。許通互市,俘獲蕃口,撫而還之。如此不出五載,河朔之民,得務農業,亭障之地,可積軍諸。然後待其亂而取之則克,乘其衰而兵之則降,既心服而
 忘歸,則力省而功倍。



 誠願考古道,務遠圖,示綏懷萬國之心,用駕馭四夷之策,事戒輒發,理貴深謀,所謂安危之理,不可輕言者。國家務大體,求至治則安;舍近謀遠,勞而無功則危。為君有常道,為臣有常職,是務大體也。上不拒諫,下不隱情,是求至治也。漢武帝躬秉武節,登單于之臺;唐太宗手結雨衣,伐遼東之國:則是舍近謀遠也。沙漠窮荒,得之無用,則是勞而無功也。在位之臣,敢言者少,言而見聽,未必蒙福,言而不從,方且虞禍,欲
 下不隱情得乎?惡在其務大體而求至治也。



 臣又謂利害相生,變易不定者,《兵書》曰:「不能盡知用兵之害者,則不能盡知用兵之利。」蓋事有可進而退,則害成之事至焉;可退而進,則利用之事去焉。可速而緩,則利必從之而失;可緩而速,則害必由之而致。可誅而赦,則奸宄之心,或有時而生害;可赦而誅,則患勇之人,或無心於利國。可賞而罰,則有以害勤勞之功;可罰而賞,則有以利僭逾之幸。能審利害,則為聰明。以天下之耳聽之則聰,
 以天下之目視之則明。故《書》曰「明四目、達四聰」,此之謂也。臣又謂取舍不可以有惑者,故曰「孟賁之狐疑,不如童子之必至」。思慮不可以不精者,故曰「差若毫厘,繆以千里」。自國家圖燕以來,連兵未解,財用不得不耗,人心不得不憂,願陛下精思慮,決取舍,無使曠日持久,窮兵極武焉。



 書奏,上嘉之。七年,徙知相州,改右補闕。復上章論事。



 明年,移睦州。睦州人舊阻禮教,錫建孔子廟,表請以經籍給諸生,詔賜《九經》,自是人知向學。會文明殿災,
 又拜章極言時政,上嘉納焉。轉起居舍人,還判登聞鼓院,上書請封禪。以本官知制誥,尋加兵部員外郎。



 端拱二年,京畿大旱,錫上章,有「調變倒置」語,忤宰相,罷為戶部郎中,出知陳州。坐稽留殺人獄,責授海州團練副使,後徙單州。召為工部員外郎,復論時政闕失,俄詔直集賢院。至道中,復舊官。



 真宗嗣位,遷吏部。出使秦、隴,還,連上章言,陜西數十州苦於靈、夏之役,生民重困,上為之戚然。同知審官院兼通進、銀臺、封駁司,賜金紫;與魏廷
 式聯職,以議論不協求罷,出知泰州。會彗星見,拜疏請責躬以答天戒,再召見便殿。及行,降中使撫諭,仍加優賜。



 咸平三年,詔近臣舉賢良方正,翰林學士承旨宋白以錫應詔。還朝,屢召對言事。錫嘗奏曰:「陛下即位以來,治天下何道?臣願以皇王之道治之。舊有《御覽》,但記分門事類。臣請鈔略四部,別為《御覽》三百六十卷,萬幾之暇,日覽一卷,經歲而畢。又採經史要切之言。為《御屏風》十卷,置扆座之側,則治亂興亡之鑒,常在目矣。」真宗
 善其言,詔史館以群書借之,每成書數卷,即先進內。錫乃先上《御覽》三十卷、《御屏風》五卷。



 《御覽序》曰:「聖人之道,布在方冊。《六經》則言高旨遠,非講求討論,不可測其淵深。諸史則跡異事殊,非參會異同,豈易記其繁雜。子書則異端之說勝,文集則宗經之辭寡。非獵精義以為鑒戒,舉綱要以觀會通,為日覽之書,資日新之德,則雖白首,未能窮經,矧王者乎?臣每讀書,思以所得上補聖聰,可以銘於座隅者,書於御屏;可以用於常道者,錄為御覽。
 冀以涓埃之微,上裨天地之德,俾功業與堯、舜比崇,而生靈亦躋仁壽之域矣。」



 《御屏風序》曰:「古之帝王,盤盂皆銘,幾杖有戒,蓋起居必睹,而夙夜不忘也。湯之《盤銘》曰:『茍日新,日日新,又日新。』武王銘於幾杖曰:『安不忘危,存不忘亡,熟惟二者,後必無兇。』唐黃門侍郎趙智為高宗講《孝經》,舉其要切者言之曰:『天子有爭臣七人,雖無道不失其天下。』憲宗採《史》、《漢》、《三國》已來經濟之要,號《前代君臣事跡》,書於屏間。臣每覽經、史、子、集,因取其語要,輒
 用進獻,題之御屏,置之座右,日夕觀省,則聖德日新,與湯、武比隆矣。」



 五年,再掌銀臺,覽天下奏章,有言民饑盜起及詔敕不便者,悉條奏其事。上對宰相稱錫「得爭臣之體」,即日以本官兼侍御史知雜事,擢右諫議大夫、史館修撰。連上八疏,皆直言時政得失。六年冬,病卒,年六十四。遺表勸上以慈儉守位,以清凈化人,居安思危,在治思亂。上覽之惻然,謂宰相李沆曰:「田錫,直臣也。朝廷少有闕失,方在思慮,錫之章奏已至矣。若此諫官,亦不
 可得。」嗟惜久之,特贈工部侍郎。錄其二子,並為大理評事,給奉終喪。



 錫耿介寡合,未嘗趨權貴之門,居公庭,危坐終日,無懈容。慕魏徵、李絳之為人,以盡規獻替為己任。嘗曰:「吾立朝以來,章疏五十有二,皆諫臣任職之常言。茍獲從,幸也,豈可藏副示後,謗時賣直邪?」悉命焚之。然性凝執,治郡無稱。所著有《咸平集》五十卷。



 王禹偁,字符之,濟州鉅野人。世為農家,九歲能文,畢士安見而器之。太平興國八年擢進士,授成武主簿。徙知
 長洲縣,就改大事評事。同年生羅處約時宰吳縣,日相與賦詠,人多傳誦。端拱初,太宗聞其名,召試,擢右拾遺、直史館,賜緋。故事,賜緋者給塗金銀帶,上特命以文犀帶寵之。即日獻《端拱箴》以寓規諷。



 時北庭未寧,訪群臣以邊事。禹偁獻《御戎十策》,大略假漢事以明之:「漢十二君,言賢明者,文、景也;言昏亂者,哀、平也。然而文、景之世,軍臣單于最為強盛,肆行侵掠,候騎至雍,火照甘泉。哀、平之時,呼韓邪單于每歲來朝,委質稱臣,邊烽罷警。何
 邪?蓋漢文當軍臣強盛之時,而外任人、內修政,使不能為深患者,由乎德也。哀、平當呼韓衰弱之際,雖外無良將,內無賢臣,而致其來朝者,系於時也。今國家之廣大,不下漢朝,陛下之聖明,豈讓文帝。契丹之強盛,不及軍臣單于,至如撓邊侵塞,豈有候騎至雍,而火照甘泉之患乎?亦在乎外任人、內修德爾。臣愚以為:外則合兵勢而重將權,罷小臣詗邏邊事,行間諜離其黨,遣趙保忠、折御卿率所部以掎角。下詔感勵邊人,使知取燕薊舊
 疆,非貪其土地;內則省官以寬經費,抑文士以激武夫,信用大臣以資其謀,不貴虛名以戒無益,禁游惰以厚民力。」帝深嘉之。又與夏侯嘉正、羅處約、杜鎬表請同校《三史書》,多所厘正。



 二年,親試貢士,召禹偁,賦詩立就。上悅曰:「此不逾月遍天下矣。」即拜左司諫、知制誥。是冬,京城旱,禹偁疏云:「一穀不收謂之饉,五穀不收謂之饑。饉則大夫以下,皆損其祿;饑則盡無祿,廩食而已。今旱雲未沾,宿麥未茁,既無積蓄,民饑可憂。望下詔直云:『君臣
 之間,政教有闕,自乘輿服御,下至百官奉料,非宿衛軍士、邊庭將帥,悉第減之,上答天譴,下厭人心,俟雨足復故。』臣朝行中家最貧,奉最薄,亦願首減奉,以贖耗蠹之咎。外則停歲市之物;內則罷工巧之伎。近城掘土,侵塚墓者瘞之;外州配隸之眾,非贓盜者釋之。然後以古者猛虎渡河、飛蝗越境之事,戒敕州縣官吏。其餘軍民刑政之弊,非臣所知者,望委宰臣裁議頒行,但感人心,必召和氣。」



 未幾,判大理寺,廬州妖尼道安誣訟徐鉉,道安
 當反坐,有詔勿治。禹偁抗疏雪鉉,請論道安罪,坐貶商州團練副使,歲餘移解州。四年,召拜左正言,上以其性剛直不容物,命宰相戒之。直弘文館,求補郡以便奉養,得知單州,賜錢三十萬。至郡十五日,召為禮部員外郎,再知制誥。屢獻討李繼遷便宜,以為繼遷不必勞力而誅,自可用計而取。謂宜明數繼遷罪惡,曉諭蕃漢,垂立賞賜,高與官資,則繼遷身首,不梟即擒矣。其後潘羅支射死繼遷,夏人款附,卒如禹偁言。



 至道元年,召入翰林
 為學士,知審官院兼通進、銀臺、封駁司。詔命有不便者,多所論奏。孝章皇后崩,遷梓宮於故燕國長公主第,群臣不成服。禹偁與客言,後嘗母信儀天下,當遵用舊禮。坐謗訕,罷為工部郎中、知滁州。初,禹偁嘗草《李繼遷制》,送馬五十匹為潤筆,禹偁卻之。及出滁,閩人鄭褒徒步來謁,禹偁愛其儒雅,為買一馬。或言買馬虧價者,太宗曰:「彼能卻繼遷五十馬,顧肯虧一馬價哉?」移知揚州。真宗即位,遷秩刑部,會詔求直言,禹偁上疏言五事:



 一曰謹
 邊防,通盟好,使輦運之民有所休息。方今北有契丹,西有繼遷。契丹雖不侵邊,戍兵豈能減削?繼遷既未歸命,饋餉固難寢停。關輔之民,倒懸尤甚。臣愚以為宜敕封疆之吏,致書遼臣,俾達其主,請尋舊好。下詔赦繼遷罪,復與夏臺。彼必感恩內附,且使天下知陛下屈己而為民也。



 二曰減冗兵,並冗吏,使山澤之饒,稍流於下。當乾德、開寶之時,土地未廣,財賦未豐,然而擊河東,備北鄙,國用未足,兵威亦強,其義安在?由所蓄之兵銳而不眾,
 所用之將專而不疑故也。自後盡取東南數國,又平河東,土地財賦,可謂廣且豐矣,而兵威不振,國用轉急,其義安在?由所蓄之兵冗而不盡銳,所用之將眾而不自專故也。臣愚以為宜經制兵賦,如開寶中,則可高枕而治矣。且開寶中設官至少。臣本魯人,占籍濟上,未及第時,一州止有刺史一人、司戶一人,當時未嘗闕事。自後有團練推官一人,太平興國中,增置通判、副使、判官、推官,而監酒、榷稅算又增四員。曹官之外,更益司理。問其
 租稅,減於曩日也;問其人民,逃於昔時也。一州既爾,天下可知。冗吏耗於上,冗兵耗於下,此所以盡取山澤之利,而不能足也。夫山澤之利,與民共之。自漢以來,取為國用,不可棄也;然亦不可盡也。只如茶法從古無稅,唐元和中,以用兵齊、蔡,始稅茶。唐史稱是歲得錢四十萬貫,今則數百萬矣,民何以堪?臣故曰減冗兵,並冗吏,使山澤之饒,稍流於下者此也。



 三曰艱難選舉,使入官不濫。古者鄉舉里選,為官擇人,士君子學行修於家,然後
 薦之朝廷,歷代雖有沿革,未嘗遠去其道。隋、唐始有科試,太祖之世,每歲進士不過三十人,經學五十人。重以諸侯不得奏闢,士大夫罕有資蔭,故有終身不獲一第,沒齒不獲一官者。太宗毓德王藩,睹其如此。臨御之後,不求備以取人,舍短用長,拔十得五。在位將逾二紀,登第殆近萬人,雖有俊傑之才,亦有容易而得。臣愚以為數百年之艱難,故先帝濟之以泛取,二十載之霈澤,陛下宜糾之以舊章,望以舉場還有司,如故事。至於吏部
 銓官,亦非帝王躬親之事,自來五品已下,謂之旨授官,今幕職、州縣而已,京官雖有選限,多不施行。臣愚以為宜以吏部還有司,依格敕注擬可也。



 四曰沙汰僧尼,使疲民無耗。夫古者惟有四民,兵不在其數。蓋古者井田之法,農即兵也。自秦以來,戰士不服農業,是四民之外,又生一民,故農益困。然執干戈衛社稷,理不可去。漢明之後,佛法流入中國,度人修寺,歷代增加。不蠶而衣,不耕而食,是五民之外,又益一而為六矣。假使天下有萬
 僧,日食米一升,歲用絹一匹,是至儉也,猶月費三千斛,歲用萬縑,何況五七萬輩哉。不曰民蠹得乎?臣愚以為國家度人眾矣,造寺多矣,計其費耗,何啻億萬。先朝不豫,舍施又多,佛若有靈,豈不蒙福?事佛無效,斷可知矣。願陛下深鑒治本,亟行沙汰,如以嗣位之初,未欲驚駭此輩,且可以二十載,不度人修寺,使自銷鑠,亦救弊之一端也」



 五曰親大臣,遠小人,使忠良蹇諤之士,知進而不疑,奸憸傾巧之徒,知退而有懼。夫君為元首,臣為股
 肱,言同體也。得其人則勿疑,非其人則不用。凡議帝王之盛者,豈不曰堯、舜之時,契作司徒,咎繇作士,伯夷典禮,後夔典樂,禹平水土,益作虞官。委任責成,而堯有知人任賢之德。雖然,堯之道遠矣,臣請以近事言之。唐元和中,憲宗嘗命裴□銓品庶官,□曰:「天子擇宰相,宰相擇諸司長官,長官自擇僚屬,則上下不疑,而政成矣。」識者以□為知言。願陛下遠取帝堯,近鑒唐室,既得宰相,用而不疑。使宰相擇諸司長官,長官自取僚屬,則垂拱
 而治矣。古者刑人不在君側,《語》曰:「放鄭聲,遠佞人。」是以周文王左右,無可結襪者,言皆賢也。夫小人巧言令色,先意希旨,事必害正,心惟忌賢,非聖明不能深察。舊制,南班三品,尚書方得升殿;比來三班奉職,或因遣使,亦許升殿,惑亂天聽,無甚於此。願陛下振舉紀綱,尊嚴視聽,在此時矣。



 臣愚又以為今之所急,在先議兵,使眾寡得其宜,措置得其道。然後議吏,使清濁殊塗,品流不雜,然後艱選舉以塞其源,禁僧尼以去其耗,自然國用足
 而王道行矣。



 疏奏,召還,復知制誥。咸平初,預修《太祖實錄》,直書其事。時宰相張齊賢、李沆不協,意禹偁議論輕重其間。出知黃州,嘗作《三黜賦》以見志。其卒章云:「屈於0身而不屈於道兮,雖百謫而何虧!」三年,濮州盜夜入城,略知州王守信、監軍王昭度,禹偁聞而奏疏,略曰:



 伏以體國經野,王者保邦之制也。《易》曰「王公設險,以守其國」。自五季亂離,各據城壘,豆分瓜剖,七十餘年。太祖、太宗,削平僭偽,天下一家。當時議者,乃令江淮諸郡毀城隍、
 收兵甲、徹武備者,二十餘年。書生領州,大郡給二十人,小郡減五人,以充常從。號曰長吏,實同旅人;名為郡城,蕩若平地。雖則尊京師而抑郡縣,為強幹弱枝之術,亦匪得其中道也。臣比在滁州,值發兵挽漕,關城無人守御,止以白直代主開閉,城池頹圮,鎧仗不完。及徙維揚,稱為重鎮,乃與滁州無異。嘗出鎧甲三十副,與巡警使臣,彀弩張弓,十損四五,蓋不敢擅有修治,上下因循,遂至於此。今黃州城雉器甲,復不及滁、揚。萬一水旱為
 災,盜賊竊發,雖思御備,何以枝梧。蓋太祖削諸侯跋扈之勢,太宗杜僭偽覬望之心,不得不爾。其如設法救世,久則弊生,救弊之道,在乎從宜。疾若轉規,固不可膠柱而鼓瑟也。今江、淮諸州,大患有三:城池墮圮,一也;兵仗不完,二也;軍不服習,三也;濮賊之興,慢防可見。望陛下特紆宸斷,許江、淮諸郡,酌民戶眾寡,城池大小,並置守捉。軍士多不過五百人,閱習弓劍,然後漸葺城壁,繕完甲冑,則郡國有禦侮之備,長吏免剽略之虞矣。



 疏奏,上嘉
 納之。



 四年,州境二虎鬥,其一死,食之殆半。群雞夜鳴,經月不止。冬雷暴作。禹偁手疏引《洪範傳》陳戒,且自劾;上遣內侍乘馹勞問,醮禳之,詢日官,云:「守土者當其咎。」上惜禹偁才,是日,命徙蘄州。禹偁上表謝,有「宣室鬼神之問,不望生還;茂陵封禪之書,止期身後」之語。上異之,果至郡未逾月而卒,年四十八。訃聞,甚悼之,厚賻其家。賜一子出身。



 禹偁詞學敏贍,遇事敢言,喜臧否人物,以直躬行道為己任。嘗云:「吾若生元和時,從事於李絳、崔群
 間,斯無愧矣。」其為文著書,多涉規諷,以是頗為流俗所不容,故屢見擯斥。所與游必儒雅,後進有詞藝者,極意稱揚之。如孫何、丁謂輩,多游其門。有《小畜集》二十卷、《承明集》十卷、《集議》十卷、詩三卷。子嘉祐、嘉言俱知名。



 嘉祐為館職,寇準曰:「吾尹京,外議云何?」對曰:「人言丈人且入相。」準曰:「於吾子意何如?」嘉祐曰:「以愚觀之,不若不為相之善也,相則譽望損矣。自古賢相,所以能建功業、澤生民者,其君臣相得,如魚之有水,故言聽計從,而臣主俱
 榮。今丈人負天下重望,中外有太平之責焉,丈人於明主,能若魚之有水乎?」準大喜,執其手曰:「元之雖文章冠天下,至於深識遠慮,或不逮吾子也。」嘉祐官不顯。



 嘉言以進士第為江都簿,真宗嘗觀禹偁奏章,嗟美切直,因訪其後,宰相以嘉言聞。即召對,擢大理評事,至殿中侍御史。



 曾孫汾舉進士甲科,仕至工部侍郎,入元祐黨籍。



 張詠,字復之,濮州鄄城人。少任氣,不拘小節,雖貧賤客游,未嘗下人。太平興國五年,郡舉進士,議以詠首薦。有
 夙儒張覃者未第,詠與寇準致書郡將,薦覃為首,眾許其能讓。是歲,詠登進士乙科,大理評事、知鄂州崇陽縣。再遷著作佐郎。以蘇易簡薦,入為太子中允,遷秘書丞、通判麟相二州,乞掌濮州市征以便養。俄召還,賜緋魚,知浚儀縣。會李沆、宋湜、寇準連薦其才,以為荊湖北路轉運使,奏罷歸、峽二州水遞夫,就轉太常博士。



 太宗聞其強幹,召還,超拜虞部郎中,賜金紫。旬日,與向敏中並擢為樞密直學士、同知銀臺通進封駁司兼掌三班院。
 張永德為並代部署,有小校犯法,笞之至死,詔案其罪。詠封還詔書,且言:「陛下方委永德邊任,若以一部校故,推辱主帥,臣恐下有輕上之心。」太宗不從。未幾,果有營兵脅訴軍校者,詠引前事為言,太宗改容勞之。



 出知益州,時李順構亂,王繼恩、上官正總兵攻討,緩師不進。詠以言激正,勉其親行,仍盛為供帳餞之。酒酣,舉爵屬軍校曰:「汝曹蒙國厚恩,無以塞責,此行當直抵寇壘,平蕩醜類。若老師曠日,即此地還為爾死所矣。」正由是決行
 深入,大致克捷。繼恩帳下卒縋城夜遁,吏執以告。詠不欲與繼恩失歡,即命縶投眢井,人無知者。時寇略之際,民多脅從,詠移文諭以朝廷恩信,使各歸田里。且曰:「前日李順脅民為賊,今日吾化賊為民,不亦可乎?」時民間訛言,有白頭翁午後食人兒女,一郡囂然。至暮,路無行人,既而得造訛者戮之,民遂帖息。詠曰:「妖訛之興,沴氣乘之,妖則有形,訛則有聲,止訛之術,在乎識斷,不在乎厭勝也。」



 初,蜀士知向學,而不樂仕宦。詠察郡人張及、李
 畋、張逵者皆有學行,為鄉里所稱;遂敦勉就舉,而三人者悉登科,士由是知勸。民有諜訴者,詠灼見情偽,立為判決,人皆厭服。好事者編集其辭,鏤板傳布。詠嘗曰:「詢君子得君子,詢小人得小人,各就其黨詢之,則無不審矣。」其為政,恩威並用,蜀民畏而愛之。丁外艱,起復,改兵部郎中。會詔川、陜諸州參用銅鐵錢,每銅錢一當鐵錢十。詠上言:「昨經利州,以銅錢一換鐵錢五,綿州銅錢一換鐵錢六,益州銅錢一換鐵錢八。若一其法,公私非便。
 望依旬估折納銅錢。」



 真宗即位,加左諫議大夫。咸平初,入拜給事中、戶部使,改御史中丞。承天節齊會,丞相大僚有酒失者,詠奏彈之。二年,同知貢舉。是夏,以工部侍郎出知杭州。屬歲歉,民多私鬻鹽以自給,捕獲犯者數百人,詠悉寬其罰而遣之。官屬請曰:「不痛繩之,恐無以禁。」詠曰:「錢塘十萬家,饑者八九,茍不以鹽自活,一旦蜂聚為盜,則為患深矣。俟秋成,當仍舊法。」有民家子與姊婿訟家財。婿言妻父臨終,此子裁三歲,故見命掌貲產;
 且有遺書,令異日以十之三與子,餘七與婿。詠覽之,索酒酹地,曰:「汝妻父,智人也,以子幼故托汝。茍以七與子,則子死汝手矣。」亟命以七給其子,餘三給婿,人皆服其明斷。知永興軍府。



 五年,馬知節自益徙延州,朝議擇可代者。真宗以詠前在蜀治行優異,復命知益州,仍加刑部侍郎、樞密直學士,就遷吏部侍郎。轉運使黃觀上其治狀,有詔褒美。會遣謝濤巡撫西蜀,上因令傳諭詠曰:「得卿在蜀,朕無西顧之憂矣。」歸朝,復掌三班,領登聞檢
 院。



 詠剛中歲瘍生腦,頗妨巾櫛,求知穎州。真宗以其公直,有時望,再任益部,皆以政績聞,不當蒞小郡。令中書召問,將委以青社或真定,令其自擇。詠辭不就,遂命知升州。大中祥符初,加左丞。三年春,州民以詠秩滿借留,就轉工部尚書,令再任。是秋,以江左旱歉,命充升、宣等十州安扶使,進禮部。上聞詠腦瘍甚,憫之,令薛映馳驛代還。以疾未見,恨不得面陳所蘊,乃抗論言:「近年虛國帑藏,竭生民膏血,以奉無用之土木,皆賊臣丁謂、王欽若
 啟上侈心之為也。不誅死,無以謝天下。」章三上,出知陳州。



 初,詠與青州傅霖少同學。霖隱不仕。詠既顯,求霖者三十年不可得,至是來謁。閽吏白傅霖請見,詠責之曰:「傅先生天下賢士,吾尚不得為友,汝何人,敢名之!」霖笑曰:「別子一世尚爾邪,是豈知世間有傅霖者乎?」詠問:「昔何隱,今何出?」霖曰:「子將去矣,來報子爾。」詠曰:「詠亦自知之。」霖曰:「知復何言。」翌日別去。後一月而詠卒,年七十。贈左僕射,謚忠定。



 詠剛方自任,為治尚嚴猛,嘗有小吏忤
 詠,詠械其頸。吏恚曰:「非斬某,此枷終不脫。」詠怒其悖,即斬之。少學擊劍,慷慨好大言,樂為奇節。有士人游宦遠郡,為僕夫所持,且欲得其女為妻,士人者不能制。詠遇於傳舍,知其事,即陽假此僕為馭,單騎出近郊,至林麓中,斬之而還。嘗謂其友人曰:「張詠幸生明時,讀典墳以自律,不爾,則為何人邪?」故其言曰:「事君者廉不言貧,勤不言苦,忠不言己效,公不言己能,斯可以事君矣。」性躁果卞急,病創甚,飲食則痛楚增劇,御下益峻,尤不喜人
 拜跪,命典客預戒止。有違者,詠即連拜不止,或倨坐罵之。真守嘗稱其材任將帥,以疾不盡其用。自號乖崖,以為「乖」則違眾,「崖」不利物。有集十卷。弟詵,為虞部員外郎。



 論曰:《傳》云:「邦有道,危言危行。」三人者,躬骨鯁蹇諤之節,蔚為名臣,所遇之時然也。禹偁制戎之策,厥後果符其言,而醇文奧學,為世宗仰。錫身沒之後,特降褒命,以賁直操,與夫容容嘿嘿,以持祿固位者異矣。詠所至以政績聞。天子嘗曰:「詠在蜀,吾無西顧之憂。」其被獎與如
 此。然皆骯臟自信,道不諧偶,故不極於用雲



\end{pinyinscope}