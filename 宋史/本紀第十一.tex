\article{本紀第十一}

\begin{pinyinscope}

 仁宗三



 慶歷元年春正月辛亥朔,御大慶殿受朝。己未,加唃廝囉河西節度使。壬申,詔歲以春分祠高禖。是月,元昊請和。



 二月己亥,壽國公昕薨。辛亥,罷大宴。京東西、淮、浙、江
 南、荊湖置宣毅軍。甲辰,詔臣僚受外任者,毋得因臨遣祈恩。丙午,京師雨藥。是月,元昊寇渭州,環慶路馬步軍副總管任福敗於好水川,福及將佐軍士死者六千餘人。



 三月庚戌朔,修金堤。乙卯,詔止郡國舉人,勿以邊機為名希求恩澤。



 夏四月甲申,以資政殿學士陳執中同陜西馬步軍都總管兼經略安撫沿邊招討等使、知永興軍。詔夏竦仍判永興軍。乙巳,下德音:降陜西囚死罪一等,流以下釋之。特支軍士緡錢,振撫邊民被鈔略者
 親屬。



 五月丁巳,錄系囚。甲子,出內藏緡錢一百萬助軍費。乙丑,追封皇長子為褒王,賜名昉。丁卯,罷陜西經略安撫沿邊招討都監。辛未,宋庠、鄭戩罷,以王舉正參知政事,任中師、任布為樞密副使。詔夏竦屯軍鄜州,陳執中屯軍涇州。癸酉,閱試衛士。



 六月壬辰,詔陜西諸路總管司嚴邊備,毋輒入賊界,賊至則御之。乙巳,詔近臣舉河北、陜西、河東知州、通判、縣令。



 秋七月丙辰,月掩心後星。戊午,月掩南斗。壬戌,置萬勝軍凡二十指揮。是月,元
 昊寇麟、府州。



 八月戊寅,詔鄜延部署以兵援麟、府。甲申,河北置場括市戰馬,緣邊七州軍免括。乙未,毀潼關新置樓櫓。庚子,月掩歲星。乙巳,募民間材勇者補神捷指揮。是月,元昊寇金明砦,破寧遠砦,砦主王世但、兵馬監押王顯死之。陷豐州,知州王餘慶、兵馬監押孫吉死之。



 九月壬子,命河東鑄大鐵錢。乙亥,復置義倉。



 冬十月甲午,詔罷陜西都部署,分四路置使。置陜西營田務。己亥,罷銅符、木契。是月,修河北城池。



 十一月壬子,置涇原路強壯
 弓箭手。丙辰,發廩粟,減價以濟京城民。甲子,朝饗景靈宮。乙丑,饗太廟、奉慈廟。丙寅,祀天地於圜丘,大赦,改元。蠲陜西來年夏稅十之二,及麟、府民二年賦租。臣僚許立家廟。功臣不限品數賜戟。增天下解額。弛京東八州鹽禁。是月,令江、饒、池三州鑄鐵錢。十二月丙子,加恩百官。丁丑,司天監上《崇天萬年歷》。戊寅,詔陜西四路總管及轉運使兼營田。甲午,置陜西護塞軍。是歲,湖南洞蠻知徽州楊通漢貢方物。



 二年春正月丁巳,復京師榷鹽法。壬戌,詔以京西閑田處內附蕃族無親屬者。遣使河北募兵,及萬人者賞之。癸亥,詔磨勘院考提點刑獄功罪為三等,以待黜陟。



 二月乙未,詔河北強壯刺手背為義勇軍。



 三月甲辰朔,詔殿前指揮使、兩省都知舉武臣才堪為將者。丁巳,杜衍宣撫河東。辛酉,晁宗愨罷。己巳,契丹遣蕭英、劉六符來致書求割地。是月,賜禮部奏名進士、諸科及第出身八百三十九人。



 夏四月戊寅,命御史中丞、諫官同較三司
 用度,罷其不急者。庚辰,知制誥富弼報使契丹。五月辛亥,錄系囚。壬子,減皇后及宗室婦郊賜之半。甲寅,詔三館臣僚上封事及聽請對。丙辰,詔醫官毋得換右職。戊午,建大名府為北京。降河北州軍系囚罪一等,杖、笞以下釋之。乙酉,罷左藏庫月進錢。戊辰,禁銷金為服飾。是月,契丹集兵幽州,聲言來侵,河北、京東皆為邊備。



 六月甲戌,出內藏銀、紬、絹三百萬助邊費。癸未,以特奏名武藝人補三班。丙戌,置北平軍。丙申,閱蕃落將士騎射。戊
 戌,詔減省南郊臣僚賜與。



 秋七月丙午,任布罷。丁未,詔軍校戰沒無子孫者,賜其家緡錢。戊午,大雨雹。以呂夷簡兼判樞密院事,章得像兼樞密使,晏殊加平章事。癸亥,富弼再使契丹。詔京官告病者,一年方聽朝參。



 八月丁丑,策制舉人。戊寅,策武舉人試騎射。甲申,白氣貫北斗。戊子,出內藏庫緡錢十萬修北京行宮。己亥,遣使安撫京東,督捕盜賊。



 九月丙午,呂夷簡改兼樞密使。乙丑,契丹遣耶律仁先、劉六符持誓書來。



 閏月戊戌,罷河北
 民間科徭。是月,元昊寇定川砦,涇原路馬步軍副都總管葛懷敏戰沒,諸將死者十四人,元昊大掠渭州而去。



 冬十月庚戌,刺陜西保捷軍。甲寅,遣使安撫涇原路。丙辰,知制誥梁適報使契丹。戊午,發定州禁軍二萬二千人屯涇原。庚申,詔恤將校陣亡,其妻女無依者養之宮中。丙寅,契丹遣使來再致誓書,報徹兵。



 十一月壬申,黑氣貫北斗柄。辛巳,復都部署兼招討等使,命韓琦、範仲淹、龐籍分領之。甲申,以泰山處士孫復為國子監直講。
 是歲,占城獻馴像三。



 三年春正月庚午朔,封皇子曦為鄂王。辛未,曦薨。丙子,減陜西歲市木三之一。辛巳,詔輔臣議蠲減天下賦役。戊子,詔錄將校死王事而無子孫者親屬。辛卯,置德順軍。壬辰,錄唐狄仁傑後。癸巳,元昊自名曩霄,遣人來納款,稱夏國。



 二月丙午,賜陜西招討韓琦、範仲淹、龐籍錢各百萬。辛酉,立四門學。



 三月壬申,閱衛士武技。戊子,呂夷簡罷為司徒、監修國史,與議軍國大事。以章得像為
 昭文館大學士,晏殊為集賢殿大學士並兼樞密使,夏竦為樞密使,賈昌朝參知政事。



 夏四月戊戌朔,幸瓊林苑閱騎士。癸卯,遣保安軍判官邵良佐使元昊,許封冊為夏國主,歲賜絹十萬匹、茶三萬斤。甲辰,以韓琦、範仲淹為樞密副使。乙巳,詔夏竦還本鎮,以杜衍為樞密使。丙辰,以春夏不雨,遣使祠禱於岳瀆。甲子,呂夷簡罷議軍國大事。



 五月丁卯朔,日有食之。庚午,錄系囚。戊寅,詔諸路轉運使並兼按察使,歲具官吏能否以聞。庚辰,祈
 雨於相國寺、會靈觀。癸未,置御史六員,罷推直官。丁亥,置武學。戊子,雨。己丑,謝雨。辛卯,築欽天壇於禁中。乙未,近臣薦試方略者六人,授官有差。是月,忻州地大震。虎翼卒王倫叛於沂州。



 六月甲辰,詔諸路漕臣令所部官吏條茶、鹽、礬及坑冶利害以聞。



 秋七月辛未,詔許二府不限奏事常制,得敷陳留對。丙子,王舉正罷。壬午,罷陜西管內營田。甲申,命任中師宣撫河東,範仲淹宣撫陜西。乙酉,獲王倫。



 八月乙未朔,命官詳定編敕。戊戌,詔諫
 官日赴內朝。丁未,以範仲淹參知政事,富弼為樞密副使。壬子,白氣貫北斗魁。癸丑,韓琦代範仲淹宣撫陜西。甲寅,太白晝見。戊午,罷武學。



 九月丁卯,詔輔臣對天章閣。戊辰,呂夷簡以太尉致仕。乙亥,任中師罷。丁丑,詔執政大臣非假休不許私第受謁。是月,桂陽洞蠻寇邊,湖南提刑募兵討平之。



 冬十月丙午,詔中書、樞密同選諸路轉運使。丁未,詔縣令佐能根括編戶隱偽以增賦入者,量其數賞之。戊申,詔二府同選諸路提刑。甲寅,復諸
 路轉運判官。乙卯,詔修兵書。壬戌,詔二府頒新定磨勘式。甲子,築水洛城。是月,光化軍亂,討平之。



 十一月丙寅,上清宮火。癸未,詔館職有闕,以兩府、兩省保舉,然後召試補用。丁亥,更蔭補法。壬辰,限職田。是月,五星皆在東方。十二月乙巳,桂陽監徭賊復寇邊。丁巳,大雨雪,木冰。河北雨赤雪。交址獻馴象五。安化州蠻來貢。



 四年春正月庚午,京城雪寒,詔三司減價出薪米以濟之。壬申,西蕃磨氈角入貢。乙亥,荊王元儼薨。辛卯,太常
 禮儀院上新修《禮書》及《慶歷祀儀》。



 二月丙申,出奉宸庫銀三萬兩振陜西饑民。己酉,白虹貫日。甲寅,罷陜西四路馬步軍都總管、經略安撫招討使,復置隨路都總管、經略安撫招討使。



 三月癸亥朔,以旱遣內侍祈雨。辛未,省廣濟河歲漕軍儲二十萬石。乙亥,詔天下州縣立學,更定科舉法,語在《選舉志》。己卯,出御書治道三十五事賜講讀官。庚辰,錄唐郭子儀後。甲申,免衡、道州、桂陽監民經徭賊劫略者賦役一年。



 夏四月丙申,詔湖南民誤
 為征徭軍所殺者,賜帛存撫其家。丁酉,宜州蠻區希範叛,詔廣西轉運鈐轄司發兵討捕。壬子,以錫慶院為太學。



 五月庚午,錄系囚。壬申,幸國子監謁孔子,有司言舊儀止肅揖,帝特再拜。賜直講孫復五品服。遂幸武成王廟,又幸玉津園觀種稻。乙亥,撫州獻生金山。丙子,詔西川知州軍監,罷任未出界而卒者,錄其子孫一人。戊寅,詔募人納粟振淮南饑。乙酉,忻州言地震,有聲如雷。丙戌,曩霄遣人來,復稱臣。



 六月壬子,降天下系囚流、徒罪
 一等,杖、笞釋之。範仲淹宣撫陜西、河東。癸丑,詔諸軍因戰傷廢停不能自存及死事之家孤老,月給米,人三斗。



 秋七月戊寅,封宗室德文等十人為郡王、國公。壬午,月犯熒惑。癸未,契丹遣使來告伐夏國。甲申,夷人寇三江砦,淯井監官兵擊走之。丙戌,詔諸路轉運、提刑察舉守令有治狀者。



 八月辛卯,命賈昌朝領天下農田,範仲淹領刑法事。甲午,富弼宣撫河北。戊戌,命右正言餘靖報使契丹。保州雲翼軍殺官吏、據城叛。庚子,命右正言田
 況度視保州,仍聽便宜行事。丙午,進宗室官有差。戊午,詔輔臣所薦官毋以為諫官、御史。



 九月辛酉,保州平。壬戌,詔保州官吏死亂兵而無親屬者,官為殯斂,兵官被害及戰沒,並優賜其家。民田遭蹂踐者,蠲其租。癸亥,以真宗賢妃沉氏為德妃,婉儀杜氏為賢妃。戊辰,呂夷簡薨。庚午,晏殊罷。乙亥,遣使安撫湖南。甲申,以杜衍同中書門下平章事兼樞密使、集賢殿大學士,賈昌朝為樞密使,陳執中參知政事。丁亥,宴宗室太清樓,射於苑中。



 冬十月庚寅,賜曩霄誓詔,歲賜銀、絹、茶、彩凡二十五萬五千。陳堯佐薨。丙申,命範仲淹提舉三館秘閣繕校書籍。癸丑,桂陽蠻降,授蠻酋三人奉職。



 十一月壬戌,以西界內附香布為團練使。己巳,詔戒朋黨相訐,及按察恣為苛刻、文人肆言行怪者。己卯,改上莊穆皇后謚曰章穆,莊獻明肅皇太后曰章獻明肅,莊懿皇太后曰章懿,莊懷皇后曰章懷,莊惠皇太后曰章惠。庚辰,朝饗景靈宮。辛巳,饗太廟、奉慈廟。壬午,冬至,祀天地於圜丘,大赦。
 十二月壬辰,加恩百官。乙未,封曩霄為夏國主。丁酉,詔州縣以先帝所賜七條相誨敕。辛亥,置保安、鎮戎軍榷場。是歲,黎州邛部川山前、山後百蠻都鬼主牟黑來貢。



 五年春正月甲戌,罷河東、陜西諸路招討使。乙亥,復置言事御史。丙子,契丹遣使來告伐夏國還。庚辰,命知制誥餘靖報使契丹。癸未,詔京朝官因被彈奏,雖不曾責罰,但有改移差遣,並四周年磨勘。乙酉,範仲淹、富弼罷。丙戌,杜衍罷,以賈昌朝同中書門下平章事兼樞密使、
 集賢殿大學士,王貽永為樞密使,宋庠參知政事,吳育、龐籍並為樞密副使。



 二月辛卯,詔罷京朝官用保任敘遷法,又罷蔭補限年法。壬辰,曩霄初遣人來賀正旦。癸卯,以久旱,詔州縣毋得淹系刑獄。辛亥,祈雨於相國天清寺、會靈祥源觀。癸丑,桂陽監言唐和等復內寇。乙卯,謝雨。



 三月己未,詔大宗正勵諸宗子授經務學。辛酉,韓琦罷。癸亥,詔禮部貢舉。甲子,宜州蠻賊區希範平。庚午,東方有黃氣如虹貫月。甲戌,詔監司按察屬吏,毋得差
 官體量。甲申,詔陜西以曩霄稱臣,降系囚罪一等,笞釋之,邊兵第賜緡錢。民去年逋負皆勿責,蠲其租稅之半,麟、府州嘗為羌所掠,除逋負租稅如之。丙戌,罷入粟補官。



 夏四月丁亥朔,司天言日當食,陰晦不見。錄系囚,遣官錄三京囚。辛卯,曩霄初遣人來賀乾元節。戊申,章得像罷,以賈昌朝為昭文館大學士,陳執中同中書門下平章事、集賢殿大學士兼樞密使。庚戌,以吳育參知政事,丁度為樞密副使。



 五月己巳,罷諸路轉運判官。



 閏月
 丙午,曩霄遣人來謝冊命。



 六月丁卯,減益、梓州上供絹歲三之一,紅錦、鹿胎半之。



 秋七月戊申,以廣州地震。



 八月庚午,荊南府、岳州地震。



 九月庚寅,詔文武官己致仕而舉官犯罪當連坐者,除之。辛卯,以重陽,曲宴近臣、宗室於太清樓,遂射苑中。



 冬十月乙卯,契丹遣使來獻九龍車及所獲夏國羊馬。辛酉,祔章獻明肅皇后、章懿皇后神主於太廟,大赦。罷轉運使兼按察。庚午,幸瓊林苑,遂畋楊村,遣使以所獲馳薦太廟,召父老,賜以飲食、茶
 帛。辛未,頒歷於夏國。庚辰,罷宰臣兼樞密使。



 十一月丁亥,冬至,宴宗室於崇政殿。己酉,詔河北長吏舉殿直、供奉官有武才者。是歲,施州溪洞蠻、西南夷龍以特來貢。



 六年春正月戊申,徙廣南戍兵善地,以避瘴毒。



 二月戊寅,青州地震。詔陜西經略安撫及轉運司議裁節諸費及所置官員無用者以聞。



 三月辛巳朔,日有食之。錄系囚。庚寅,登州地震,岠嵎山摧,自是屢震,輒海底有聲如雷。甲午,月犯歲星。是月,賜禮部奏名進士、諸科及第出
 身八百五十三人。



 夏四月甲寅,遣使賜湖南戍兵方藥。



 五月甲申,京師雨雹,地震。丙戌,錄系囚。戊子,減邛州鹽井歲課緡錢一百萬。丙申,詔陜西市蕃部馬。丁酉,京東人劉巹、劉沔、胡信謀反,伏誅。



 六月庚戌朔,詔夏竦與河北監司察帥臣、長吏之不職者。丁巳,有流星出營室南,其光燭地,隱然有聲。丙寅,以久旱,民多暍死,命京城增鑿井三百九十。丁丑,詔制科隨禮部貢舉。



 秋七月丁亥,月犯南斗。庚寅,河東經略司言雨壞忻、代等州城壁。



 八
 月癸亥,策試賢良方正能直言極諫,並試武舉人。癸酉,以吳育為樞密副使,丁度參知政事。



 九月甲辰,登州言有巨木三千餘浮海而出。



 冬十月辛未,詔發兵討湖南猺賊。



 十一月己卯,遣官議夏國公封界。癸未,湖南猺賊寇英、韶州界。辛丑,畋東韓村,乘輿所過及圍內田,蠲其租一年。是歲,邈川首領唃廝囉、西蕃瞎氈、磨氈角、安化州蠻蒙光速等來貢。交址獻馴象十。道州部瀧酋李石壁等降。



 七年春正月丙子朔,御大慶殿受朝。丁亥,詔河北所括馬死者,限二年償之。己亥,頒《慶歷編敕》。壬寅,詔減連州民被猺害者來年夏租。



 二月己酉,詔取益州交子三十萬,於秦州募人入中糧。丙辰,令內侍二人提舉月給軍糧。



 三月壬午,錄系囚。癸未,詔天下有能言寬恤民力之事者,有司驛置以聞,以其副上之轉運司,詳其可行者輒行之。毀後苑龍船。丁亥,以旱,罷大宴。癸巳,詔避正殿,減常膳。許中外臣僚實封條上三事。乙未,賈昌朝罷,以
 陳執中為昭文館大學士,夏竦同中書門下平章事、集賢殿大學士,吳育為給事中歸班,文彥博為樞密副使。罷出獵。丁酉,以夏竦為樞密使,文彥博參知政事,高若訥為樞密副使。辛丑,祈雨於西太一宮,及還,遂雨。壬寅,陳執中、宋庠、丁度以旱,降官一等。



 夏四月丁未,謝雨。己酉,詔:「前京東轉運使薛紳專任文吏伺察郡縣細過,江東轉運使楊弦、判官王綽、提點刑獄王鼎苛刻相尚,並削職知州,自今毋復用為部使者。」壬子,御正殿,復常膳。
 乙卯,復執中、庠、度官。己巳,詔諫官非公事毋得私謁。



 五月戊寅,詔武臣非歷知州、軍無過者,毋授同提點刑獄。己丑,補降猺唐和等為峒主。己亥,命翰林學士楊察蠲放天下逋負。辛丑,詔西北二邊有大事,二府與兩制以上雜議之。



 六月乙巳,詔禁畜猛獸害人者。辛酉,詔天下知縣非鞫獄毋得差遣。壬戌,詔臣僚朝見者,留京毋過十日。



 秋七月癸未,奉安太祖、太宗、真宗御容於南京鴻慶宮。甲申,德音:降南京畿內囚罪一等,徒以下釋之。賜
 民夏稅之半。除災傷倚閣稅及欠折官物非侵盜者。辛丑,禁貢餘物饋近臣。



 八月乙丑,析河北為四路,各置都總管。



 九月丁酉,詔刪定《一州一縣敕》。



 冬十月壬子,李迪薨。甲子,幸廣親宅,謁太祖、太宗神御殿,宴宗室,賜器幣有差。乙丑,河陽、許州地震。



 十一月乙未,加上真宗謚。丙申,朝饗景靈宮。丁酉,饗太廟、奉慈廟。戊戌,冬至,祀天地於圜丘,大赦。貝州宣毅卒王則據城反。十二月戊申,加恩百官。庚戌,樞密直學士明鎬體量安撫河北。癸丑,詔
 貝州有能引致官兵獲賊者,授諸衛上將軍。甲寅,遣內侍以敕榜招安貝賊。是歲,西蕃磨氈角來貢。



 八年春正月丁丑,文彥博宣撫河北,明鎬副之。壬午,江寧府火。乙未,日赤無光。



 閏月辛丑,貝州平。甲辰,曲赦河北,賜平貝州將士緡錢,戰沒者官為葬祭,兵所踐民田蠲其稅,改貝州為恩州。戊申,文彥博同中書門下平章事、集賢殿大學士,官吏將士有功者遷擢有差。辛酉,親從官顏秀等四人夜入禁中謀為變,宿衛兵捕殺之。丙
 寅,磔王則於都市。丁卯,知貝州張得一坐降賊伏誅。



 二月癸酉,頒《慶歷善救方》。夏國來告曩霄卒。己卯,賜瀛、莫、恩、冀州緡錢二萬,贖還饑民鬻子。丁酉,奉安宣祖、太祖、太宗御容於睦親宅。



 三月甲辰,詔禮部貢舉。辛亥,遣使體量安撫陜西。甲寅,幸龍圖、天章閣,詔輔臣曰:「西陲備御,兵冗賞濫,罔知所從,卿等各以所見條奏。」又詔翰林學士、三司使、知開封府、御史中丞曰:「朕躬闕失,左右朋邪,中外險詐,州郡暴虐,法令有不便於民者,朕欲聞之,
 其悉以陳。」壬戌,以霖雨,錄系囚。癸亥,以朝政得失、兵農要務、邊防備豫、將帥能否、財賦利害、錢法是非與夫讒人害政、奸盜亂俗及防微杜漸之策,召知制誥、諫官、御史等諭之,使悉對於篇。



 夏四月己巳朔,封曩霄子諒祚為夏國主。壬申,丁度罷,明鎬參知政事。



 五月辛酉,夏竦罷,宋庠為樞密使,龐籍參知政事。



 六月戊辰朔,詔近臣舉文武官材堪將帥者。丙子,河決澶州商胡埽。壬辰,以久雨齋禱。甲午,明鎬卒。乙未,詔館閣官須親民一任,方
 許入省、府及轉運、提點刑獄差遣。丙申,章得像薨。



 秋七月戊戌,以河北水,令州縣募饑民為軍。辛丑,罷鑄鐵錢。



 八月己丑,以河北、京東西水災,罷秋宴。



 九月戊午,詔三司以今年江、淮漕米轉給河北州軍。冬十一月己亥,作「皇帝欽崇國祀之寶」。壬戌,出廩米,減價以濟畿內貧民。



 十二月乙丑朔,以霖雨為災,頒德音,改明年元,減天下囚罪一等,徒以下釋之。出內藏錢帛賜三司,貿粟以濟河北,流民所過,官為舍止之,所繼物毋收算。丁卯,冊美
 人張氏為貴妃。戊子,遣使體量安撫利州路。是歲,廬州合肥縣稻再實。交州來貢。



 皇祐元年春正月甲戌朔,日有食之。以河北水災,罷上元張燈,停作樂。庚戌,張士遜薨。己未,詔以緡錢二十萬市穀種,分給河北貧民。辛酉,詔臺諫非朝廷得失、民間利病,毋風聞彈奏。



 二月戊辰,以河北疫,遣使頒藥。辛未,發禁軍十指揮赴京東、西路備盜。



 三月丁巳,錄系囚。己未,契丹遣使來告伐夏國。庚申,翰林院學士錢明逸報
 使契丹。是月,賜禮部奏名進士、諸科及第出身千三百九人。



 四月癸未,梓州轉運司言淯井監夷人平。



 六月甲子,蠲河北復業民租賦二年。甲戌,始置觀文殿大學士。戊寅,詔中書、樞密非聚議毋通賓客。戊子,詔轉運使、提點刑獄,所部官吏受贓失覺察者,降黜。



 秋七月丁酉,詔臣僚毋得保薦要近內臣。己未,詔諸州歲市藥以療民疾。



 八月壬戌,陳執中罷。以文彥博為昭文館大學士,宋庠同中書門下平章事、集賢殿大學士,龐籍為樞密使,
 高若訥參知政事,梁適為樞密副使。甲申,策制舉、武舉人。



 九月乙巳,廣源州蠻儂智高寇邕州,詔江南、福建等路發兵以備。戊午,太白犯南斗。己未,罷武舉。冬十一月丙申,詔河北被災民八十以上及篤疾不能自存者,人賜米一石、酒一斗。辛丑,詔民有冤、貧不能詣闕者,聽訴於監司以聞。



 十二月甲子,遣入內供奉高懷政督捕邕州盜賊。是歲,大留國來貢。



\end{pinyinscope}