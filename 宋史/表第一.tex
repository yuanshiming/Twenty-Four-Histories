\article{表第一}

\begin{pinyinscope}

 宰輔一



 宋宰輔年表,前九朝始建隆庚申,終靖康丙午,凡一百六十七年,居相位者七十二人,位執政者二百三十八人。後七朝始建炎丁未,終德祐丙子,凡一百四十九年,
 居相位者六十一人,位執政者二百四十四人。



 敘古曰:古之史法主於編年,至司馬遷作《史記》始易以新意,然國家世祚,人事歲月,散於紀、傳、世家,先後始終,遽難考見,此表之不可無,而編年不容於盡變也。厥後班固《漢史》乃曰《百官公卿表》,先敘官名、職秩、印授等,然後書年以表其姓名。歐陽修《唐史》又專以宰相名篇,意必有所在矣。



 宋自太祖至欽宗,舊史雖以《三朝》、《兩朝》、《四朝》各自為編,而年表未有成書。神宗時常命陳繹檢閱二府除罷
 官職事,因為《拜罷錄》。元豐間,司馬光嘗敘宋興以來百官公卿沿革除拜,作年表上之史館。自時而後,曾鞏、譚世績、蔡幼學、李燾諸人皆嘗續為之。然表文簡嚴,世罕知好,故多淪落無傳。



 今纂修《宋史》,故……採紀、傳以為是表。其間所書宰輔官、職、勛……間有不同者,官制沿革有時而異也。然中書位次既止於參知政事,而樞府職序自同知、副使而下雖簽書、同簽書亦與焉者,皆執政也,故不得而略焉。



 夫大臣之
 用舍,關於世道之隆污,千載而下,將使覽者即表之年觀紀及傳之事,此登載之不容於不謹也。表之所書,雖無褒貶是非於其間,然歲月昭於上,姓名著於下,則不惟其人之賢佞邪正可指而議,而當時任用之專否,政治之得失,皆可得而見矣。後之覽者,其必有所勸也夫,其亦有所戒也夫!



 表略



\end{pinyinscope}