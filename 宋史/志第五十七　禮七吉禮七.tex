\article{志第五十七 禮七吉禮七}

\begin{pinyinscope}

 封禪汾陰后土朝謁太清宮天書九鼎



 封禪。太宗即位之八年,泰山父老千餘人詣闕,請東封。帝謙讓未遑,厚賜以遣之。明年,宰臣宋琪率文武官、僧
 道、耆壽三上表以請,乃詔以十一月二十一日有事於泰山,命翰林學士扈蒙等詳定儀注。既而乾元、文明二殿災,詔停封禪,而以是日有事於南郊。



 真宗大中祥符元年,兗州父老呂良等千二百八十七人及諸道貢舉之士八百四十六人詣闕陳請,而宰臣王旦等又率百官、諸軍將校、州縣官吏、蕃夷、僧道、父老二萬四千三百七十人五上表請,始詔今年十月有事於泰山。遣官告天地、宗廟、社稷、太一宮及在京祠廟、岳瀆,命翰林、太常
 禮院詳定儀注,知樞密院王欽若、參知政事趙安仁為封禪經度制置使並判兗州,三司使丁謂計度糧草,引進使曹利用、宣政使李神福修行宮道路,皇城使劉承珪等計度發運。詔禁緣路採捕及車騎蹂踐田稼,以行宮側官舍、佛寺為百官宿頓之所,調兗、鄆兵充山下丁役。行宮除前後殿外,並張幕為屋,覆以油帊。仍增自京至泰山驛馬,令三司沿汴、蔡、御河入廣濟河運儀仗什物赴兗州,發上供木,由黃河浮筏至鄆州,給置頓費用,
 省輦送之役。以王旦為大禮使,王欽若為禮儀使,參知政事馮拯為儀仗使,知樞密院陳堯叟為鹵簿使,趙安仁為橋道頓遞使,仍鑄五使印及經度制置使印給之。遣使詣嶽州,採三脊茅三十束,有老人黃皓識之,補州助教,賜以粟帛。



 初,太平興國中,有得唐玄宗社首玉冊、蒼璧,至是令瘞於舊所。其前代封禪壇址摧圯者,命修完之。山上置圜臺,徑五丈,高九尺,四陛,上飾以青,四面如其方色;一壝,廣一丈,圍以青繩三周。燎壇在其東南,
 高丈二尺,方一丈,開上南出戶,方六尺。山下封祀壇,四成,十二陛,如圜丘制,上飾以玄,四面如方色;外為三壝,燎壇如山上壇制。社首壇,八角;三成,每等高四尺,上闊十六步;八陛,上等廣八尺,中等廣一丈,下等廣一丈二尺;三壝四門:如方丘制。又為瘞埳於壬地外壝之內。以玉為五牒,牒各長尺二寸,廣五寸,厚一寸,刻字而填以金,聯以金繩,緘以玉匱,置石□感中。金脆難用,以金塗繩代之。



 正坐、配坐,用玉冊六副,每簡長一尺二寸,廣一寸二分,厚三分,
 簡數量文多少。匱長一尺三寸。檢長如匱,厚二寸,闊五寸,纏金繩五周,當纏繩處刻為五道,而封以金泥,泥和金粉、乳香為之。



 印以受命寶。封匱當寶處,刻深二分,用石□感藏之。其□感用用石再累,各方五尺,厚一尺,鑿中廣深,令容玉匱。□感旁施檢處,皆刻深七寸,闊一尺,南北各三,東西各二,去隅皆七寸,纏繩處皆刻三道,廣一寸五分,深三分。為石檢十以擫□感,皆長三尺,闊一尺,厚七寸,刻三道,廣深如纏繩。其當封處,刻深二寸,取足容寶,皆有小石蓋,與
 封刻相應。其檢立□感旁,當刻處又為金繩三以纏□感,皆五周,徑三分,為石泥封□感。泥用石末和方色土為之。



 用金鑄寶,曰「天下同文」,如御前寶,以封□感際。距石十二分,距四隅皆闊二尺,厚一尺,長一丈,斜刻其道,與□感隅相應,皆再累,為五色土圜封□感,上徑一丈二尺,下徑三丈九尺。命直史館劉鍇、內侍張承素領徒封圜臺石□感,直集賢院宋皋、內侍郝昭信封社首石□感,並先往規度之。



 詳定所言:「朝覲壇在行宮南,方九丈六尺,高九尺,四陛。陛,南面兩陛,
 餘三面各一陛。一壝,二分在南,一分在北。又按唐封禪,備法駕。準故事,乘輿出京,並用法駕,所過州縣不備儀仗。其圜臺上設登歌、鐘、磬各一虡,封祀壇宮架二十虡,四隅立建鼓、二舞。社道壇設登歌如圜臺,壇下宮架、二舞如封祀壇。朝覲壇宮架二十虡,不用熊羆十二案。又按《六典》,南郊合祀天地,服袞冕,垂白珠十有二,黝衣纁裳十二章。欲望封禪日依南郊例。洎禮畢,御朝覲壇。諸州所貢方物,陳列如元正儀。令尚書戶部告示,並先集
 泰山下。」仍詔出京日,具小駕儀仗:太常寺三百二十五人,兵部五百六十六人,殿中省九十一人,太僕寺二百九十九人,六軍諸衛四百六十八人,左右金吾仗各一百七十六人,司天監三十七人。



 有司言:「南郊惟昊天、皇地祇、配帝、日月、五方、神州各用幣,內官而下別設六十六段分充。按《開寶通禮》,岳鎮、海瀆幣從方色,即明皆有制幣。今請封祀壇內官至外官三百一十八位,社首壇岳鎮以下一十八位,並用方色幣。又南郊牲,正坐、配坐
 用犢,五方帝、日月、神州共享羊、豕二十二,從祀七百三十七位,仍以前數分充。今請神州而上十二位用犢,其舊供羊、豕,改充從祀牲。又景德中,升天皇、北極在第一等,今請亦於從祀牲內體薦。」



 舊制,郊祀正坐、配坐褥以黃,皇帝拜褥以緋。至是,詔配坐以緋,拜褥以紫。又以靈山清潔,命祀官差減其數,或令兼攝,有期喪未滿、餘服未卒哭者不得預祭,內侍諸司官,除掌事宿衛外,從升者裁二十四人,諸司職掌九十三人。其文武官升山者,
 皆公服。



 詳定所言:「《漢書》八神與歷代封禪帝王及所禪山,並於前祀七日遣官致祭,以太牢祀泰山,少牢祀社首。」九月,詔審刑院、開封府毋奏大闢案。帝習儀於崇德殿。初,禮官言無帝王親習之文,帝曰:「朕以達寅恭之意,豈憚勞也。」既畢,帝見禮文有未便,諭宰臣與禮官再議。於是詳定所言:「按《開寶禮》,則燔燎畢封冊;開元故事,則封□感後燔燎。今如不對神封冊,則未稱寅恭,或封□感後送神,則並為喧瀆。欲望俟終獻畢,皇帝升壇,封玉匱,置
 □感中,泥印訖,復位,飲福,送神,樂止,舉燎火。次天書降,次金匱降。禮儀使奏禮畢,皇帝還大次,俟封□感畢,皇帝再升壇省視。緣祀禮已畢,更不舉樂。省訖,降壇。」仍詔山上亞獻、終獻,登歌作樂。



 十月戊子朔,禁天下屠殺一月。帝自告廟,即屏葷蔬食,自進發至行禮前,並禁音樂。有司請登封日圜臺立黃麾仗,至山下壇設權火。將行禮,然炬相屬,又出朱字漆牌,遣執仗者傳付山下。牌至,公卿就位,皇帝就望燎位,山上傳呼萬歲,下即舉燎。皇帝還
 大次,解嚴,又傳呼而下,祀官始退。社首瘞坎,亦設權火三為準。遣司天設漏壺山之上下,命中官覆校日景,復於壇側擊板相應。自太平頂、天門、黃峴嶺、岱嶽觀,各豎長竿,揭籠燈下照,以相參候。



 辛卯,發京師,以玉輅載天書先行。次日如之。至鄆州,令從官、衛士蔬食。丁未,次奉高宮。戊申,齋於穆清殿,諸升山者官給衣,令祀日沐浴服之。庚戌,帝服通天冠、絳紗袍,乘金輅,備法駕,至山門幄次,改服靴袍,乘步輦登山,鹵簿、儀衛列山下,天書仗
 不上山,與法駕仗間立。知制誥朱巽奉玉冊牒及圜臺行事官先升,且以回馬嶺至天門路峻絕,人給橫板二,長三尺許,系彩兩端,施於背,膺選從卒,推引而上。衛士皆給釘鞋,供奉馬止於中路。自山趾盤道至太平頂,凡兩步一人,彩繒相間,樹當道者不伐,止縈以繒。帝每經陜險,必降輦徒步。亞獻寧王元偓,終獻舒王元偁,鹵簿使陳堯叟從。祀官、點饌習儀於圜壇。是夕,山下罷警場。



 辛亥,設昊天上帝位於圜臺,奉天書於坐左,太祖、太宗
 並配西北側向,帝服袞冕,升臺奠獻,悉去侍衛,拂翟止於壝門,籠燭前導亦徹之。玉冊文曰:「嗣天子臣某,敢昭告於昊天上帝:臣嗣膺景命,昭事上穹。昔太祖揖讓開基,太宗憂勤致治,廓清寰宇,混一車書,固抑升中,以延積慶。元符錫祚,眾寶效祥,異域咸懷,豐年屢應。虔修封祀,祈福黎元。謹以玉帛、犧牲、粢盛、庶品,備茲禋燎,式薦至誠。皇伯考太祖皇帝、皇考太宗皇帝配神作主。尚饗。」玉牒文曰:「有宋嗣天子臣某,敢昭告於昊天上帝:啟運
 大同,惟宋受命,太祖肇基,功成治定;太宗膺圖,重熙累盛。粵惟沖人,丕承列聖,寅恭奉天,憂勤聽政。一紀於茲,四隩來暨,丕貺殊尤,元符章示,儲慶發祥,清凈可致,時和年豐,群生咸遂。仰荷顧懷,敢忘繼志,僉議大封,聿申昭事。躬陟喬嶽,對越上天,率禮祗肅,備物吉蠲,以仁守位,以孝奉先。祈福逮下,侑神昭德,惠綏黎元,懋建皇極,天祿無疆,靈休允迪,萬葉其昌,永保純錫。」命群官享五方帝及諸神於山下封祀壇。上飲福酒,攝中書令王旦
 跪稱曰:「天賜皇帝太一神策,周而復始,永綏兆人。」三獻畢,封金、玉匱。王旦奉玉匱,置於石□感,攝太尉馮拯奉金匱以降,將作監領徒封□感。帝登圜臺閱視訖,還御幄。宰臣率從官稱賀,山下傳呼萬歲,聲動山谷。即日仗還奉高宮,百官奉迎於谷口。帝復齋於穆清殿。



 壬子,禪祭皇地祇於社首山,奉天書升壇,以祖宗配。玉冊文曰:「嗣天子臣某,敢昭告於皇地祇:無私垂祐,有宋肇基,命惟天啟,慶賴坤儀。太祖神武,歲震萬宇;太宗聖文,德綏九土。
 臣恭膺寶命,纂承丕緒,穹昊降祥,靈符下付,景祚延鴻,秘文昭著。八表以寧,五兵不試,九穀豐穰,百姓親比,方輿所資,涼德是愧。溥率同詞,縉紳協議,因以時巡,亦既肆類。躬陳典禮,祗事厚載,致孝祖宗,潔誠嚴配。以伸大報,聿修明祀,本支百世,黎元受祉。謹以玉帛、犧牲、粢盛、庶品,備茲禋瘞,式薦至誠。皇伯考太祖皇帝、皇考太宗皇帝配神作主。尚饗。」帝至山下,服靴袍,步出大次。



 癸丑,有司設仗衛、宮縣於壇下,帝服袞冕,御封禪壇上之壽
 昌殿受朝賀,大赦天下,文武遞進官勛,減免賦稅、工役各有差,改乾封縣曰奉符縣,宴百官卿監以上於穆清殿、泰山父老於殿門。甲寅,發奉符,始進常膳。



 帝之巡祭也,往還四十七日,未嘗遇雨雪,嚴冬之候,景氣恬和,祥應紛委。前祀之夕,陰雺風勁,不可以燭,及行事,風頓止,天宇澄霽,燭焰凝然,封□感訖,紫氣蒙壇,黃光如帛,繞天書匣。悉縱四方所獻珍鳥異獸山下。法駕還奉高宮,日重輪,五色雲見。鼓吹振作,觀者塞路,歡呼動天地。改奉
 高宮曰會真宮。九天司命上卿加號保生天尊,青帝加號廣生帝君,天齊王加號仁聖,各遣使祭告。詔王旦撰《封祀壇頌》,王欽若撰《社首壇頌》,陳堯叟撰《朝覲壇頌》。圜臺奉祀官並於山上刻名,封祀、九宮、社首壇奉祀官並於《社首頌》碑陰刻名,扈從升朝官及內殿崇班、軍校領刺史以上與蕃夷酋長並於《朝覲頌》碑陰刻名。



 明年二月,詔知兗州李迪、京東轉運使馬元方等同修圜封,以呂良首請,命攝兗州助教。



 政和三年,兗、鄆耆壽、道釋等
 及知開德府張為等五十二人表請東封,優詔不允。六年,知兗州宋康年請下秘閣檢尋祥符東封典故付臣經畫。時蔡京當國,將講封禪以文太平,預具金繩、玉檢及他物甚備,造舟四千艘,雨具亦千萬計,迄不能行。



 汾陰后土。真宗東封之又明年,河中府言:「進士薛南及父老、僧道千二百人列狀乞赴闕,請親祠后土。」詔不允。已而南又請,河南尹寧王元偓亦表請,文武百僚詣東上閣門三表以請。詔明年春有事於汾陰后土,命知樞
 密院陳堯叟為祀汾陰經度制置使,翰林學士李宗諤副之,樞密直學士戚綸、昭宣使劉承珪計度發運,河北轉運使李士衡、鹽鐵副使林特計度糧草,龍圖閣待制王曙、西京左藏庫使張景宗、供備庫使藍繼宗修治行宮、道路,宰臣王旦為大禮使,知樞密院王欽若為禮儀使,參知政事馮拯為儀仗使,趙安仁為鹵簿使,陳堯叟為橋道頓遞使。又以旦為天書儀衛使,欽若、安仁副之,丁謂為扶侍使,藍繼宗為扶侍都監,內侍周懷政、皇甫
 繼明為夾侍。發陜西、河東兵五千人赴汾陰給役,出廄馬,增傳置,命翰林、禮院詳定儀注,造玉冊、祭器。先令堯叟詣后土祠祭告,分遣常參官告天地、廟社、岳鎮、海瀆。



 詳定所言:「祀汾陰后土,請如封禪,以太祖、太宗並配。其方丘之制,八角,三成,每等高四尺,上闊十六步。八陛,上陛廣八尺,中廣一丈,下廣一丈二尺。三重壝,四面開門。為瘞坎於壇之壬地外壝之內,方深取足容物。其後土壇別無方色。正坐玉冊,玉匱一副;配坐玉冊,金匱二副;
 金泥,金繩。所用石匱並蓋三層,方廣五尺,下層高二尺,上開牙縫一周,闊四寸,深五寸,中容玉匱,其闊一尺,長一尺六寸。匱刻金繩道三周,各相去五寸,每纏繩處,闊一寸,深五分。上層厚一尺,仍於上四角更刻牙縫,長八寸,深四寸。每纏金繩處深四寸,方三寸五分,取容封寶。先即廟庭規地為坎,深五尺,闊容石匱及封固者。先以金繩三道南北絡石匱,候祀畢封匱訖,中書侍郎奉匱至廟,與太尉同置石匱中,將作監加蓋,系金繩畢,各填
 以石泥,印以『天下同文之寶』,如社首封□感制。帝省視後,將作監率執事更加盝頂石蓋,然後封固如法。上為小壇,如方丘狀,廣厚皆五尺。」



 經度制置使詣脽上築壇如方丘,廟北古雙柏旁有堆阜,即其地為之。有司請祭前七日遣祀河中府境內伏羲、神農、帝舜、成湯、周文武、漢文帝、周公廟及於脽下祭漢、唐六帝。



 四年正月,帝習儀於崇德殿。丁酉,法駕發京師。二月丙辰,至寶鼎縣奉祇宮。戊午,致齋。己未,遣入內都知鄧永遷詣祠上衣服、供
 具。庚申,百官宿祀所。是夜一鼓,扶侍使奉天書升玉輅,先至脽上。二鼓,帝乘金輅,法駕詣壇,夾路設燎火,盤道回曲,周以黃麾仗。初,路出廟南,帝以未修謁,不欲乘輿輦過其前,令鑿路由廟後至壇次。翼日,帝服袞冕登壇,祀后土地祇,備三獻,奉天書於神坐之左次,以太祖、太宗配侑。



 冊文曰:「維大中祥符四年,歲次辛亥,二月乙巳朔,十七日辛酉,嗣天子臣某,敢昭告於後土地祇:恭惟位配穹旻,化敷品匯。瞻言分壤,是宅景靈。備禮親祠,抑
 惟令典。肇啟皇宋,混一方輿,祖檷紹隆,承平茲久。眇躬纘嗣,勵翼靡遑,厚德資生,綿區允穆,清寧孚祐,戴履蒙休。申錫寶符,震以珍物,虔遵時邁,已建天封。明察禮均,有所未答,櫛沐祇事,用致其恭。夷夏駿奔,瑄牲以薦,肅然鄈上,對越坤元。式祈年豐,楙昭政本,兆民樂育,百福蕃滋,介祉無疆,敢忘祇畏。恭以琮幣、犧牲、粢盛、庶品,備茲瘞禮。皇伯考太祖皇帝、皇考太宗皇帝侑神作主。尚享。」親封玉冊,正坐於玉匱,配坐於金匱,攝太尉奉之以
 降,置於石匱,將作監封固之。



 帝還次,改服通天冠、絳紗袍,乘輦謁後土廟,設登歌奠獻,遣官分奠諸神。至庭中,視所封石匱。還奉祇宮,鈞容樂、太常鼓吹始振作。是日,詔改奉祇曰太寧宮。壬戌,御朝覲壇受朝賀,肆赦,宴群臣於穆清殿、父老於宮門。穆清殿,奉祇宮之前殿也。詔五使、從臣刻名碑陰。謁西嶽廟,從官皆刻名廟中,仗衛儀物大略如東封之制。命薛南試將作監主簿,以首請祠汾陰故也。



 太清宮。大中祥符六年,亳州父老、道釋、舉人三千三百十六人詣闕,請車駕朝謁太清宮,宰臣帥百官表請。詔以明年春親行朝謁禮。命參知政事丁謂為奉祀經度制置使、判亳州,翰林學士陳彭年副之,權三司使林特計度糧草。禮儀院言:「按唐太清宮令,奠獻用碧幣,同人靈,故不用玉。今詳太上老君,宜同天神用玉。胙薦獻聖祖大帝用四圭有邸。」詔用蒼璧,太清宮用竹冊一副。丁謂言:「太清宮封藏太上老君寶冊,請用玉匱各一副,長
 廣一尺,高如之,檢厚一寸二分,長廣如匱。刻金繩道五,封處深二分,方取容受命寶。石匱三層,各長五尺三寸。闊四尺二寸,下層高二尺,中容玉匱,鑿深尺二寸,長二尺五寸,闊尺三寸。中層高一尺,南北刻金繩道三,相距各五寸,闊一寸,深五分。系金繩處各深四分,方取容『天下同文』寶,上層為盝頂蓋。」以王旦為奉祀大禮使,向敏中為儀仗使,王欽若為禮儀使,陳堯叟為鹵簿使,丁謂為橋道頓遞使。又以王旦為天書儀衛使,王欽若同儀
 衛使,丁謂副之,兵部侍郎趙安仁為扶侍使,入內副都知張繼能為扶侍都監。帝朝謁玉清昭應宮,賜亳州真源縣行宮名曰奉元,殿曰迎禧。



 七年正月十五日,發京師。十九日,至奉元宮,齋於迎禧殿。二十一日,帝服通天冠、絳紗袍,奉上太上老君混元上德皇帝加號冊寶。夜漏上五刻,天書扶侍使奉天書赴太清宮。二鼓,帝乘玉輅,駐大次。三鼓,奉天書升殿,改服袞冕,行朝謁之禮,相王元偓為亞獻,榮王元儼為終獻。帝還大次,太尉奉冊
 寶於玉匱,纏以金繩,封以金泥,印以受命之寶,納於醮壇石匱,將作監加石蓋其上。群臣稱賀於大次。分命輔臣薦獻諸殿,改奉元宮曰明道宮,奉安玉皇大帝像,改真源曰衛真縣。車駕次亳州城西,詣新立聖祖殿朝拜。至應天府朝拜聖祖殿,詔號曰鴻慶宮,仍奉安太祖、太宗像。駕至自亳州,百官迎對於太一宮西之幄殿,有司以衛真靈芝二百輿洎白鹿前導天書而入。帝服靴袍,乘大輦,備儀衛還宮。



 先是,大中祥符元年正月乙丑,帝
 謂輔臣曰:「朕去年十一月二十七日夜將半,方就寢,忽室中光曜,見神人星冠、絳衣,告曰:『來月三日,宜於正殿建黃菉道場一月,將降天書《大中祥符》三篇。』朕竦然起對,已復無見,命筆識之。自十二月朔,即齋戒於朝元殿,建道場以佇神貺。適皇城司奏,左承天門屋南角有黃帛曳鴟尾上,帛長二丈許,緘物如書卷,纏以青縷三道,封處有字隱隱,蓋神人所謂天降之書也。」王旦等皆再拜稱賀。帝即步至承天門,瞻望再拜,遣二內臣升屋,奉
 之下。旦跪奉而進,帝再拜受之,親奉安輿,導至道場,付陳堯叟啟封。帛上有文曰:「趙受命,興於宋,付於慎。居其器,守於正。世七百,九九定。」緘書甚密,抉以利刀方起。帝跪受,復授堯叟讀之。其書黃字三幅,詞類《書·洪範》、老子《道德經》,始言帝能以至孝至道紹世,次諭以清凈簡儉,終述世祚延永之意。讀訖,帝復跪奉,蘊以所緘帛,盛以金匱。旦等稱賀於殿之北廡。丙寅,群臣入賀,於崇政殿賜宴,帝與輔臣皆蔬食。遣官奏告天地、宗廟、社稷及京
 城祠廟。丁卯,有司設大次朝元殿之西廊,黃麾仗,宮縣、登歌,文武官陪列,帝服靴袍升殿,酌獻三清天書。禮畢,步導入內。戊辰,大赦,改元,百官並加恩,改左承天門為左承天祥符。



 四月辛卯朔,天書再降內中功德閣。六月八日,封祀制置使王欽若言:「泰山西南垂刀山上,有紅紫雲氣,漸成華蓋,至地而散。其日,木工董祚於靈液亭北,見黃素書曳林木之上,有字不能識,言於皇城使王居正,居正睹上有御名,馳告欽若,遂迎至官舍,授中
 使捧詣闕。」帝御崇正殿,趣召輔臣曰:「朕五月丙子夜,復夢鄉者神人言:『來月上旬,當賜天書於泰山,宜齋戒祇受。』朕雖荷降告,未敢宣露,惟密諭王欽若等,凡有祥異即上聞。朕今得其奏,果與夢協。上天眷祐,惟懼不稱。」王旦等曰:「陛下至德動天,感應昭著,臣等不勝大慶。」再拜稱賀。己亥,迎導天書,安於含芳園之正殿。辛丑,帝致齋。翼曰,備法駕詣殿再拜受,授陳堯叟啟封。其文曰:「汝崇孝奉吾,育民廣福。錫爾嘉瑞,黎庶咸知。秘守斯言,善解吾
 意。國祚延永,壽歷遐歲。」讀訖,復奉以升殿。



 九月甲子,告太廟,奉安天書朝元殿,建道場。扶侍使上香,庭中奏法曲,將行禮,詣幄殿酌獻訖,奉以玉輅,中設幾褥,夾侍立旁,周以黃麾仗,前後部鼓吹,道門威儀。扶侍使以下前導,封禪日皆奉以升壇,置正位之東。自是凡舉大禮,皆如此制。於是制行殿供物,定儀仗千六百人。每歲元日,召宰臣、宗室至禁中朝拜。前一日,卻去葷茹。帝自制誓文,刻石,置玉清昭應宮寶符閣下,摹刻天書奉安昭應
 宮刻玉殿,行酌獻禮,令刻玉使日赴殿行香,副使已下,日蒞事焉。



 天禧元年正月,詔以十五日行宣讀天書之禮。前二日,齋於長春殿,以王欽若為宣讀天書禮儀使。有司設次天安殿,中位玉皇像,置錄本天書於東,聖祖板位於西,建金菉道場三晝夜。其日三鼓,帝服通天冠、絳紗袍詣道場,焚香再拜,西向立,百官朝服升殿。攝中書令任中正跪奏:「嗣天子臣某,謹與宰臣等宣讀天書,講求聖意,虔思睿訓,撫育生民。」儀衛使王旦跪取左承
 天門天書置案上,攝殿中監張景宗、張繼能捧案,攝司徒王曾、攝司空張知白跪展天書,攝太尉向敏中宣讀,每句已,即詳繹其旨,言上天訓諭之意,攝中書令王欽若錄之。宣讀畢,攝侍中張旻跪奏:「嗣天子臣某,敢不虔遵天命。」儀衛使受天書,跪納匣中。又取功德閣天書、泰山天書宣讀如上儀。王欽若跪進所錄天書,帝跪受之,登歌酌獻。禮畢,奉天書還內。帝自作《欽承寶訓述》以示中外。是月之朔,又奉天書升太初殿,恭上玉皇大天帝
 聖號寶冊、袞服焉。



 帝於大中祥符五年十月語輔臣曰:「朕夢先降神人傳玉皇之命云:『先令汝祖趙某授汝天書,令再見汝,如唐朝恭奉玄元皇帝。』翼日,復夢神人傳天尊言:『吾坐西,斜設六位以候。』是日,即於延恩殿設道場。五鼓一籌,先聞異香,頃之,黃光滿殿,蔽燈燭,睹靈仙儀衛天尊至,朕再拜殿下。俄黃霧起,須臾霧散,由西陛升,見侍從在東陛。天尊就坐,有六人揖天尊而後坐。朕欲拜六人,天尊止令揖,命
 朕前,曰:『吾人皇九人中一人也,是趙之始祖,再降,乃軒轅皇帝,凡世所知少典之子,非也。母感電夢天人,生於壽丘。後唐時,奉玉帝命,七月一日下降,總治下方,主趙氏之族,今已百年。皇帝善為撫育蒼生,無怠前志。』即離坐,乘雲而去。」王旦等皆再拜稱賀。即召旦等至延恩殿,歷觀臨降之所,並布告天下,命參知政事丁謂、翰林學士李宗諤、龍圖閣待制陳彭年與禮官修崇奉儀注。閏十月,制九天司命保生天尊號曰聖祖上靈高道九天
 司命保生天尊大帝,聖祖母號曰元天大聖後,遣官就南郊設昊天及四位告之。



 七年九月,即滋福殿設玉皇像,奉聖號匣,安於朝元殿後天書刻玉幄次。詔以來年正月上玉帝聖號,帝親撰文,及天書下,亦以此日奏告,仍定儀式班之。以王旦為奏告大禮使,向敏中為儀仗使,寇準為鹵簿使,丁謂為禮儀使,王嗣宗為橋道頓遞使。



 八年正月朔,駕詣玉清昭應宮奉表奏告,上玉皇大帝聖號曰太上開天執符御歷含真體道玉皇大天帝,
 奉刻玉天書安於寶符閣,以帝御容侍立於側,升閣酌獻。復朝拜明慶二聖殿。禮畢還宮,易常服,御崇德殿,百官稱賀。



 九年,詔以來年正月朔詣玉清昭應宮上玉皇聖號寶冊,二日詣景靈宮上聖祖天尊大帝徽號。十二月己亥,奉寶冊、仙衣安於文德殿,乃齋於天安殿後室。四鼓,帝詣天安殿酌獻天書畢,大駕赴玉清昭應宮,袞冕升太初殿,奉冊訖,奠玉幣,薦饌三獻,飲福,登歌,二舞,望燎,如祀昊天上帝儀。畢,詣二聖殿,奉上絳紗袍,奉幣
 進酒,分遣攝殿中監上紫微大帝絳紗袍、七元輔弼真君紅綃衣、翊聖保德真君皂袍。帝改服靴袍,詣紫微殿、寶符閣焚香,群臣詣集禧殿門表賀。是日,天書赴景靈宮,大駕次至,齋於明福殿。二日,帝服袞冕,詣天興殿奉上聖祖天尊大帝冊寶、仙衣,薦獻如上儀。乃改服詣保寧閣焚香,還宮,群臣入賀於崇德殿。命諸州設羅天大醮,先建道場二十七日。命王旦為兗州太極觀奉上寶冊使,趙安仁副之,遣官攝中書侍郎、殿中監,押當冊寶、
 仙衣。二月丁亥,帝齋於長春殿。翼日,有司設聖母板位文德殿,行酌獻禮,拜授冊寶於王旦、仙衣於趙安仁,以升金輅,具鹵簿儀衛,所過禁屠宰。三月乙巳,旦等詣觀奉冊上懿號曰聖祖母元天大聖後。其日,帝不視朝。禮畢,群臣入賀,賜飲崇德殿。



 徽宗政和六年九月朔,復奉玉冊、玉寶,上玉帝尊號曰太上開天執符御歷含真體道昊天玉皇上帝,蓋以論者析玉皇大天帝、昊天上帝言之,不能致一故也。又詔以王者父天母地,乃者只率萬
 邦黎庶,強為之名,以玉冊、玉寶昭告上帝,而地祇未有稱謂,謹上徽號曰承天效法厚德光大後土皇地祇。



 明年五月,詣玉清和陽宮奉上寶冊,所用之禮,以瘞坎易燎柴,設望瘞位,玉以黃琮及兩珪有邸,幣以黃,舞以八成,其餘並如奉上玉皇尊號之儀。徽宗崇尚道教,制郊祀大禮,以方士百人執威儀前引,分列兩序,立於壇下。



 政和三年十一月五日,恭上神宗、哲宗徽號於太廟。翌日,祀昊天上帝於圜丘。太師蔡京奏:「天神降格,實為大
 慶,乞付史館。」帝出手詔,播告天下。群臣詣東上閣門拜表稱賀,禦制《天真示現記》,尋以天神降日為天應節,即其地建迎真宮。明年夏至,躬祀方丘,又制《神應記》,略云:「羽衛多士,奉輦武夫,與陪祝官,顧瞻中天,有形有象,若人若鬼,持矛執戟,列於空際,見者駭愕。」仍遣使奏告陵廟,詔天下。



 又用方士魏漢津之說,備百物之象,鑄鼎九,於中太一宮南為殿奉安之,各周以垣,上施埤堄,墁如方色,外築垣環之,曰九成宮。中央曰帝鼐,其色黃,祭以
 土王日,為大祠,幣用黃,樂用宮架。北方曰寶鼎,其色黑,祭以冬至,幣用皂。東北方曰牡鼎,其色青,祭以立春,幣用皂。東方曰蒼鼎,其色碧,祭以春分,幣用青。東南曰岡鼎,其色綠,祭以立夏,幣用緋。南方曰彤鼎,其色紫,祭以夏至,幣用緋。西南曰阜鼎,其色黑,祭以立秋,幣用白。西方曰皛鼎,其色赤,祭以秋分,幣用白。西北曰魁鼎,其色白,祭以立冬,幣用皂。八鼎皆為中祠,樂用登歌,享用素饌,復於帝鼐之宮立大角鼎星祠。



 崇寧四年八月,奉安
 九鼎,以蔡京為定鼎禮儀使。帝幸九成宮酌獻。九月朔,百官稱賀於大慶殿,如大朝會儀。鄭居中言:「亳州太清宮道士王與之進《黃帝崇天祀鼎儀訣》,皆本於天元玉冊、九宮太一,合於漢津所授上帝錫夏禹隱文。同修為《祭鼎儀範》,修成《鼎書》十七卷、《祭鼎儀範》六卷。先是,詔曰:「九鼎以奠九州,以御神奸,其用有法,後失其傳。閱王與之所上《祀儀》,推鼎之意,施於有用,蓋非今人所能作。去古綿邈,文字雜糅,可擇其當理合經,修為定制,班付有
 司。」至是書成,並以每歲祀鼎常典,付有司行之。



 又詔以鑄鼎之地作寶成宮,總屋七十一區,中置殿曰神靈,以祠黃帝;東廡殿曰成功,祀夏后氏;西廡殿曰持盈,祠周成王及周公、召公;後置堂曰昭應,祀唐李良及隱士嘉成侯魏漢津。太常禮部言:「每歲欲於大樂告成崇政殿元進樂日,秋八月二十七日舉祀事,祀黃帝依感生帝、神州地祇為大祠,幣用黃,樂用宮架,祝文依祀聖祖稱嗣皇帝臣名。其成功、持盈二殿,禮用中祀,幣各用白。昭
 應堂禮用小祀,並以素饌。」從之。



 政和六年,用方士王仔昔議,定鼎閣於天章閣,自九成宮徙九鼎奉安之。又詔改帝鼐為隆鼐,正南彤鼎為明鼎,西南阜鼎為順鼎,正西皛鼎為蘊鼎,西北魁鼎為健鼎,正北寶鼎如舊,東北牡鼎為和鼎,正東蒼鼎為育鼎,東南岡鼎為潔鼎,鼎閣為圜象徽調之閣。閣上神像,左周鼎星君,中帝席星君,右大角星君;閣下鼎鼐神像,各守逐鼎布列,亦用仔昔議也。駕詣鼎閣奉安神像,明日復詣閣行香,百僚陪位。
 其後,又詔九鼎新名乃狂人妄改,皆無依據,宜復舊名,惟圜象徽調閣仍舊。



 八年,用方士言,鑄神霄九鼎成,曰太極飛雲洞劫之鼐、蒼壺祀天貯醇酒之鼎、山嶽五神之鼎、精明洞淵之鼎、天地陰陽之鼎、混沌之鼎、浮光洞天之鼎、靈光晃耀煉神之鼎、蒼龜火蛇蟲魚金輪之鼎,奉安於上清寶菉宮神霄殿,與魏漢津所鑄,凡十八鼎焉。



\end{pinyinscope}