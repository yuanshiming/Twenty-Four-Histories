\article{志第七十五 禮二十五(兇禮一)}

\begin{pinyinscope}

 山陵



 山陵、謚祔、服紀、葬儀與士庶之喪制為兇禮。其上陵忌日,漢儀如吉祭。宋制,是日禁屠殺,設素饌,輟樂舉哭,素
 服行事,因以類附焉。



 太祖建國,號僖祖曰欽陵,順祖曰康陵,翼祖曰定陵,宣祖曰安陵。



 安陵在京城東南隅,乾德初,改卜河南府鞏縣西南四十里訾鄉鄧封村。以司徒範質為改卜安陵使,學士竇儀禮儀使,中丞劉溫叟儀仗使,樞密直學士薛居正鹵簿使,太宗時尹開封,為橋道頓遞使。質尋免相,以太宗兼轄五使事,修奉新陵。皇堂下深五十七尺,高三十九尺,陵臺三層,正方,下層每面長九十尺。南神門至乳臺、乳臺至鵲臺皆九十五
 步。乳臺高二十五尺,鵲臺增四尺。神墻高九尺五寸,環四百六十步,各置神門、角闕。



 有司言:「改卜陵寢,宣祖合用哀冊及文班官各撰歌辭二首。吉仗用大駕鹵簿。兇仗用大升輿、龍輴、鵝茸纛、魂車、香輿、銘旌、哀謚冊寶車、方相、買道車、白幰弩、素信幡、錢山輿、黃白紙帳、暖帳、夏帳、千味臺盤、衣輿、拂纛、明器輿、漆梓宮、夷衾、儀槨、素翣、包牲、倉瓶、五穀輿、瓷甒、闢惡車。進玄宮有鐵帳覆梓宮,藉以棕櫚褥,鐵盆、鐵山用然漆燈。宣祖袞冕,昭憲
 皇后花釵、翬衣,贈玉。十二神、當壙、當野、祖明、祖思、地軸及留陵刻漏等,並制如儀。」



 有司又言:「按《儀禮》『改葬緦』注云:『臣為君,子為父,妻為夫也,必服緦者,親見尸柩,不可以無服,緦三月而除之。』又《五禮精義》云:『改葬無祖奠,蓋祖奠設於柩車之前以為行始,至於改葬,告遷而已。』今請皇帝服緦,皇親及文武官護送靈駕者亦服緦,既葬而除。不設祖奠,止於陵所行一虞之祭。宣祖謚冊、謚寶舊藏廟室,合遷置陵內。改葬之禮,與始葬同,幾筵宜新,
 明器壞者改作。凡斂衣、斂物並易之。其皇堂贈玉、鎮圭、劍佩、旒冕、玉寶,並以鈱玉、藥玉,綬以青錦。安陵中玉圭、劍佩、玉寶等皆用於闐玉。孝明、孝惠陵內用鈱玉、藥玉。啟故安陵,奉安宣祖、昭憲孝惠二后梓官於幄殿。靈駕發引,所過州府縣鎮,長吏令佐素服出城奉迎並辭,皆哭。自發引至揜皇堂,皆廢朝,禁京城音樂。」



 順祖、翼祖皆葬幽州,至真宗始命營奉二陵,遂以一品禮葬河南縣。制度比安陵減五分之一,石作減三分之一,尋改上定
 陵名曰靖陵。



 開寶九年十月二十日,太祖崩,遺詔:「以日易月,皇帝三日而聽政,十三日小祥,二十七日大祥。諸道節度防禦團練使、刺史、知州等不得輒離任赴闕。諸州軍府臨三日釋服。」群臣敘班殿庭,宰臣宣制發哀畢,太宗即位,號哭見群臣。群臣稱賀,復奉慰盡哀而退。



 禮官言:「群臣當服布斜巾、四腳,直領布衣蘭,腰絰。命婦布帕首、裙、帔。皇弟、皇子、文武二品以上,加布冠、斜巾、帽,首絰,大袖、裙、褲,竹
 杖。士民縞素,婦人素縵。諸軍就屯營三日哭。」群臣屢請聽政,始御長春殿。群臣喪服就列,帝去杖、絰,服斜巾、垂帽,卷簾視事。小祥,改服布四腳、直領布衣蘭,腰絰,布褲,二品以上官亦如之。大祥,帝服素紗軟腳折上巾、淺黃衫、緅皮□黑銀帶。群臣及軍校以上,皆本色慘服、鐵帶,靴、笏。諸王入內服衰,出則服慘。又成服後,群臣朝晡臨三日。大小祥、禫除、朔望,皆入臨奉慰。內出遺留物頒賜諸臣親王,遣使賚賜方鎮。二十七日,命宰臣撰陵名、哀冊
 文。



 明年三月十七日,群臣奉謚號冊寶告於南郊,明日,讀於靈坐前。四月十日,啟攢宮,帝與群臣皆服如初喪。群臣朝晡臨殿中,退,易常服出宮城。十三日,發引,帝衰服,啟奠哭,群臣入臨,升梓宮於龍輴。祖奠徹,設次明德門外,行遣奠禮,讀哀冊,帝哭盡哀,再拜辭,釋衰還宮,百官辭於都城外。二十五日,掩皇堂。二十九日,虞主至,奉安於大明殿。五月十九日,祔廟之第五室,以孝明皇后王氏升配。禮畢,群臣奉慰。其吉兇仗如安陵,惟增轀輬車、神
 帛肩輿,鹵簿三千五百三十九人。陵在鞏縣,祔宣祖,曰永昌。



 至道三年三月二十九日,太宗崩於萬歲殿。真宗散發號擗,奉遺詔即位於殿之東楹。制永熙陵,皇堂深百尺,方廣八十尺,陵臺方二百五十尺。大駕鹵簿,用玉輅一、革車五外,凡用九千四百六十八人。有司定散發之禮,皇帝、皇后、諸王、公主、縣主、諸王夫人、六宮內人並左被發,皇太后全被發。帝服布斜巾、四腳、大袖、裙、褲、帽,竹杖,腰絰、首絰,直領布衣蘭衫、白綾襯服。諸王皇親以下
 如之,加布頭冠、絹示親服。皇太后、皇后、內外命婦布裙、衫、帔、帕頭,首絰,絹示親服。宮人無帔。文武二品以上布斜巾、四腳、頭冠、大袖、衣蘭衫、裙、褲,腰絰,竹杖,絹襯服。自餘百官並布帕頭、衣蘭衫,腰絰。兩省五品、御史臺尚書省四品、諸司三品以上,見任前任防禦、團練、刺史,內客省、閣門、入內都知、押班等,布頭冠、帕頭、大袖、衣蘭,衫、裙、褲,腰絰。諸軍、庶民白衫紙帽,婦人素縵不花釵,三日哭而止。山陵前,朔望不視事。



 六月,詔翰林寫先帝常服及絳紗袍、通天冠
 御容二,奉帳坐,列於大升輿之前,仍以太宗玩好、弓劍、筆硯、琴棋之屬,蒙組繡置輿中,陳於仗內。十月三日,靈駕發引,其兇仗法物擎舁牽駕兵士力士,凡用萬二千一百九十三人。挽郎服白練寬衫、練裙,勒帛絹幘。餘並如昌陵制。十一月二日,有司奉神主至太廟,近臣題謚號,祔於第六室,以懿德皇后符氏升配。置衛十五百人於陵所,作殿以安御容,朝暮上食,四時致祭焉。



 乾興元年二月十九日,真宗崩,仁宗即位。二十日,禮儀
 院言:「準禮例,差官奏告天地、社稷、太廟、諸陵,應祠祭惟天地、社稷、五方帝諸大祠,宗廟及諸中小祠並權停,俟祔廟禮畢,仍舊。」是日,命閣門使薛貽廓告哀於契丹。宣慶使韓守英為大內都巡檢,內侍分領宮殿門,衛士屯護。閣門使王遵度為皇城四面巡檢,新舊城巡檢各權添差,益以禁兵器仗,城門亦設器甲,以辨奸詐。



 二十一日,群臣入臨,見帝於東序。閣門使宣口敕曰:「先皇帝奄棄萬國,凡在臣僚,畢同號慕,及中外將校,並加存撫。」群
 臣拜舞稱萬歲,復哭盡哀,退。是日上表請聽政,凡三上,始允。二十三日,陳先帝服玩及珠襦、玉匣、含、襚應入梓宮之物於延慶殿,召輔臣通觀。明日,大斂成服。二十五日,有司設御坐,垂簾崇政殿之西廡,簾幕皆縞素,群臣敘班殿門外。帝衰服,去杖、絰,侍臣扶升坐。通事舍人引群臣入殿庭,西向合班。俟簾卷,群臣再拜,班首奏聖躬萬福,隨班三呼萬歲,退。宰臣升殿奏事如儀。三月一日,小祥,帝行奠,釋衰服,群臣入臨,退,赴內東門,進名奉慰。自
 是每七日皆臨,至四十九日止。十三日,大祥,帝釋服,服慘。



 十四日,司天監言:「山陵斬草,用四月一日丙時吉。」十六日,山陵按行使藍繼宗言:「據司天監定永安縣東北六里曰臥龍岡,堪充山陵。」詔雷允恭覆按以聞。皇堂之制,深八十一尺,方百四十尺。制陵名曰永定。九月十一日,召輔臣赴會慶殿,觀入皇堂物,皆生平服御玩好之具。帝與輔臣議及天書,皆先帝尊道膺受靈貺,殊尤之瑞屬於元聖,不可留於人間,宜於永定陵奉安。二十三
 日,奉導天書至長春殿,帝上香再拜奉辭。二十四日,天書先發,帝啟奠梓宮,讀哀冊,禮畢,具吉兇儀仗。百官素服赴順天門外,至板橋立班奉辭。還,詣西上閣門,進名奉慰。十月十三日,掩皇堂。十八日,虞主至京。十九日,群臣詣會慶殿行九虞祭。二十三日,祔太廟第七室。



 嘉祐八年三月晦日,仁宗崩,英宗立。喪服制度及修奉永昭陵,並用定陵故事,發諸路卒四萬六千七百人治之。宣慶使石全彬提舉制梓宮,畫樣以進,命務堅完,毋
 過華飾。三司請內藏錢百五十萬貫、紬絹二百五十萬匹、銀五十萬兩,助山陵及賞賚。遣使告哀遼、夏及賜遺留物,又遣使告諭諸路。又以聽政奠告大行,近臣告升遐於天地、社稷、宗廟、宮觀,又告嗣位。賜兩府、宗室、近臣遺留物。



 五月,翰林學士王珪言:「天子之謚,當集中書門下御史臺五品以上、尚書省四品以上、諸司三品以上,於南郊告天,議定,然後連奏。近制唯詞臣撰議,即降詔命,庶僚不得參聞,頗違稱天之義。臣擬上先帝尊謚,望
 詔有司稽詳舊典,先之郊,而後下臣之議。」七月,宰臣以下宿尚書省,宗室團練使以上宿都亭驛,請謚於南郊。八月,告於福寧殿、天地、宗社、宮觀。



 九月二十八日,啟菆宮,以初喪服日一臨,易常服出。十月六日,靈駕發引,天子啟奠,梓宮升龍輴。祖奠徹,與皇太后步出宣德門,群臣辭於板橋。十五日,奉安梓宮陵側。十七日,開皇堂。十一月二日,虞主至,皇太后奠於瓊林苑,天子步出集英殿門奉迎,奠於幄。七日,祭虞主。二十九日,祔太廟。主如漢
 制,不題謚號,及終虞,而行卒哭之祭。



 禮院言:「故事,大祥變除服制,以三月二十九日大祥,至五月二十九日示覃,六月二十九日禫除,至七月一日從吉,已蒙降敕。謹按禮學,王肅以二十五月為畢喪,而鄭康成以二十七月,《通典》用其說,又加至二十七月終,則是二十八月畢喪,而二十九月始吉,蓋失之也。天聖中,《更定五服年月敕》斷以二十七月,今士庶所同遵用。夫三年之喪,自天子達,不宜有異。請以三月二十九日為大祥,五月擇日而為
 禫,六月一日而從吉。」於是大祥日不御前後殿,開封府停決大闢及禁屠至四月五日,待制、觀察使以上及宗室管軍官日一奠,二十八日而群臣俱入奠,二十九日禫除,群臣皆奉慰焉。



 治平四年正月八日,英宗崩,神宗即位。十一日,大斂。二月三日,殯。四月三日,請謚。十八日,奏告及讀謚冊於福寧殿。七月二十五日,啟菆。八月八日,靈駕發引。二十七日,葬永厚陵。



 禮院準禮:群臣成服後,乘布裹鞍韉。小祥
 臨訖,除頭冠、方裙、大袖。大祥臨訖,裹素紗軟腳帕頭,慘公服,乘皂鞍韉。禫除訖,素紗帕頭、常服、黑帶。二日,改吉服,去佩魚。虞主至自掩壙,五虞皆在途,四虞於集英殿。曲赦兩京、畿內、鄭、孟等州如故事。



 元豐八年三月五日,神宗崩。十三日,大斂,帝成服。十七日,小祥。四月一日,禫除。七月五日,請謚於南郊。九月八日,讀謚寶冊於福寧殿。二十三日,啟菆。十月一日,靈駕發引。二十一日,葬永裕陵。二十九日,虞主至。十一月一
 日,虞祭於集英殿。自復土,六虞在途,太常卿攝事,三虞行禮於殿。四日,卒哭。五日,祔廟。



 秘書正字範祖禹言:「先王制禮,以君服同於父,皆斬衰三年,蓋恐為人臣者不以父事其君,此所以管乎人情也。自漢以來,不惟人臣無服,而人君遂亦不為三年之喪。唯國朝自祖宗以來,外廷雖用易月之制,而宮中實行三年之喪。且易月之制,前世所以難改者,以人君自不為服也。今群臣易月,而人主實行三年之喪,故十二日而小祥,期而又小祥,
 二十四日大祥,再期而又大祥。夫練、祥不可以有二也,既以日為之,又以月為之,此禮之無據者。再期而大祥,中月而禫,禫者祭之名,非服之色也。今乃為之慘服三日然後禫,此禮之不經者也。既除服,至葬而又服之,蓋不可以無服也。祔廟而後即吉,財八月矣,而遽純吉,無所不佩,此又禮之無漸也。易月之制,因襲已久,既不可追,宜令群臣朝服,止如今日而未除衰,至期而服之,漸除其重者,再期而又服之,乃釋衰,其餘則君服斯服可
 也。至於禫,不必為之服,惟未純吉以至於祥,然後無所不佩,則三年之制略如古矣。」詔禮官詳議。



 禮部尚書韓忠彥等議:「朝廷典禮,時世異宜,不必循古。若先王之制,不可盡用,則當以祖宗故事為法。今言者欲令群臣服喪三年,民間禁樂如之,雖過山陵,不去衰服,庶協古制。緣先王恤典節文甚多,必欲循古,又非特如所言而已。今既不能盡用,則當循祖宗故事及先帝遺制。」詔從其議。



 神主祔廟,是月冬至,百官表賀。崇政殿說書程頤言:「
 神宗喪未除,節序變遷,時思方切,恐失居喪之禮,無以風化天下。乞改賀為慰。」不從。



 紹聖四年,太史請遷去永裕陵禁山民塚一千三百餘,以便國音。帝曰:「遷墓得無擾乎?若無所害,則令勿遷,果不便國音,當給官錢,以資葬費。」



 元符三年正月十二日,哲宗崩,徽宗即位。詔山陵制度並如元豐。七月十一日,啟菆。二十日,靈駕發引。八月八日,葬永泰陵。九月九日,以升祔畢,群臣吉服如故事。



 太常寺言:「太宗皇帝上繼太祖,兄弟相及,雖行易月之制,實斬衰三年,以重君臣之義。公除已後,庶事相稱,具載國史。今皇帝嗣位哲宗,實承神考之世,已用開寶故事,為哲宗服衰重。今神主已祔,百官之服並用純吉,皇帝服御宜如太平興國二年故事。」



 禮部言:「太平興國中,宰臣薛居正表稱:『公除以來,庶事相稱,獨命徹樂,誠未得宜。』即是公除後,除不舉樂外,釋衰從吉,事理甚明。今皇帝當御常服、素紗展腳帕頭、淡黃衫、黑犀帶,請下有司
 裁制。」宰臣請從禮官議,乃詔候周期服吉。



 時詔不由門下,徑付有司。給事中龔原言:「喪制乃朝廷大事,今行不由門下,是廢法也。臣為君服斬衰三年,古未嘗改。且陛下前此議服,禮官持兩可之論,陛下既察見其奸,其服遂正。今乃不得已從之,臣竊為陛下惜。開寶時,並、汾未下,兵革未弭,祖宗櫛風沐雨之不暇,其服制權宜一時,非故事也。」原坐黜知南康軍。於是詔依元降服喪三年之制,其元符三年九月「自小祥從吉」指揮,改正。



 紹興五年四月甲子,徽宗崩於五國城。七年正月,問安使何蘚等還以聞,宰執入見,帝號慟擗踴,終日不食。宰臣張浚等力請,始進麋粥。成服於幾筵殿,文武百僚朝晡臨於行宮。自聞喪至小祥,百官朝晡臨;自小祥至禫祭,朝一臨。太常等言:「舊制,沿邊州軍,不許舉哀。緣諸大帥皆國家腹心爪牙之臣,休戚一體,至於將佐,皆懷忠憤,宜就所屯,自副將而上成服,日朝晡臨,故校哭於本營。」命徽猷閣待制王倫等為奉迎梓宮使。



 時知邵州胡
 寅上疏,略曰:「三年之喪,自天子至於庶人,一也。及漢孝文自執謙德,用日易月,至今行之。子以便身忘其親,臣以便身忘其君,心知其非而不肯改,自常禮言之,猶且不可,況變故特異如今日者,又當如何?恭惟大行太上皇帝、大行寧德皇后,蒙塵北狩,永訣不復,實由粘罕,是有不共戴天之仇。考之於禮,仇不復則服不除,寢苫枕戈,無時而終。所以然者,天下雖大,萬事雖眾,皆無以加於父子之恩,君臣之義也。伏睹某月某日聖旨,緣國朝
 故典,以日易月,臣切以為非矣。自常禮言之,猶須大行有遺詔,然後遵承。今也大行詔旨不聞,而陛下降旨行之,是以日易月,出陛下意也。大行幽厄之中,服御飲食,人所不堪,疾病粥藥,必無供億,崩殂之後,衣衾斂藏,豈得周備?正棺卜兆,知在何所?茫茫沙漠,瞻守為誰?伏惟陛下一念及此,荼毒摧割,備難堪忍,縱未能遵《春秋》復仇之義,俟仇殄而後除服,猶當革漢景之薄,喪紀以三年為斷。不然,以終身不可除之服,二十七日而除之,是
 薄之中又加薄焉,必非聖人之所安也。」



 又曰:「雖宅憂三祀,而軍旅之事,皆當決於聖裁,則諒冱之典,有不可舉。蓋非枕塊無聞之日,是乃枕戈有事之辰,故魯侯有周公之喪,而徐夷並興,東郊不開,則是墨衰即戎,孔子取其誓命。今六師戒嚴,方將北討,萬幾之眾,孰非軍務。陛下聽斷平決,得禮之變,卒哭之後,以墨衰臨朝,合於孔子所取,其可行無疑也。如合聖意,便乞直降詔旨云『恭惟太上皇帝、寧德皇后,誕育眇躬,大恩難報,欲酬罔極,
 百未一伸。鑾輿遠征,遂至大故,訃音所至,痛貫五情。想慕慈顏,杳不復見,怨仇有在,朕敢忘之。雖軍國多虞,難以諒闇,然衰麻枕戈,非異人任。以日易月,情所不安,興自朕躬,致喪三年。即戎衣墨,況有權制,布告中外,昭示至懷。其合行典禮,令有司集議來上。如敢沮格,是使朕為人子而忘孝之道,當以大不恭論其罪。』陛下親御翰墨,自中降出,一新四方耳目,以化天下,天地神明,亦必有以祐助。臣不勝大願。」



 六月,張浚請謚於南郊。戶部尚
 書章誼等言:「梓宮未還,久廢謚冊之禮,請依景德元年明德皇后故事,行埋重、虞祭、祔廟之禮,及依嘉祐八年、治平四年虞祭畢而後卒哭,卒哭而後祔廟,仍於小祥前卜日行之。異時梓宮之至,宜遵用安陵故事,行改葬之禮,更不立虞主。」從之。九月甲子,上廟號曰徽宗。九年正月,太常寺言:「徽宗及顯肅皇后將及大祥,雖皇堂未置,若不先建陵名,則春秋二仲,有妨薦獻。請先上陵名。」宰臣秦檜等請上陵名曰永固。



 徽宗與顯肅初葬五國城,
 十二年,金人以梓宮來還。將至,帝服黃袍乘輦,詣臨平奉迎,登舟易緦服,百官皆如之。既至行在,安奉於龍德別宮,帝後異殿。禮官請用安陵故事,梓宮入境,即承之以槨;有司預備袞冕、翬衣以往,至則納之槨中,不復改斂。秦檜白令侍從、臺諫、禮官集議,靈駕既還,當崇奉陵寢,或稱攢宮。禮部員外郎程敦厚希檜意,獨上奏言:「仍攢宮之舊稱,則莫能示通和之大信,而用因山之正典,則若亡存本之後圖。臣以為宜勿褟虛名,當示大信。」於
 是議者工部尚書莫將等乃言:「太史稱歲中不利大葬,請用明德皇后故事,權攢。」從之。以八月奉迎,九月發引,十月掩攢,在昭慈攢宮西北五十步,用地二百五十畝。十三年,改陵名曰永祐。



 紹興三十一年五月,金國使至,以欽宗訃聞。詔:「朕當持斬衰三年之服,以申哀慕。」是日,文武百僚並常服、黑帶,去魚,詣天章閣南空地立班,聽詔旨,舉哭畢,次赴後殿門外進名奉慰,次詣幾筵殿焚香舉哭。六月,權禮部侍
 郎金安節等請依典故,以日易月,自五月二十二日立重,安奉幾筵,至六月十七日大祥,所有衰服,權留以待梓宮之還。從之。七月,宰臣陳康伯等率百官詣南郊請謚,廟號欽宗,遙上陵名曰永獻。其餘並如徽宗典禮。



 淳熙十四年十月八日,高宗崩,孝宗號慟擗踴,逾二日不進膳。尋諭宰執王淮,欲不用易月之制,如晉武、魏孝文實行三年之喪,自不妨聽政。淮等奏:「《通鑒》載晉武帝雖有此意,後來只是宮中深衣、練冠。」帝曰:「當時群臣不
 能將順其美,司馬光所以譏之。後來武帝竟欲行之。」淮曰:「記得亦不能行。」帝曰:「自我作古何害?」淮曰:「御殿之時,人主衰絰,群臣吉服,可乎?」帝曰:「自有等降。」乃出內批:「朕當衰絰三年,群臣自行易月之令。其合行儀制,令有司討論。」詔百官於以日易月之內,衰服治事。



 二十日丁亥,小祥,帝未改服,王淮等乞俯從禮制。上流涕曰:「大恩難報,情所未忍。」二十一日,車駕還內,帝衰絰御輦,設素仗,軍民見者,往往感泣。詔自今五日一詣梓宮前焚香。帝
 欲衰服素幄,引輔臣及班次,而禮官奏謂:「苴麻三年,難行於外庭。」奏入,不出。十一月戊戌朔,禮官顏師魯、尤袤等奏:「乞禮畢改服小祥之服,去杖、絰。禫祭禮畢,改服素紗軟腳折上巾、淡黃袍、黑銀帶。神主祔廟畢,改服皂帕頭、黑□犀帶。遇過宮燒香,則於宮中衰絰行禮,二十五月而除。」帝批:「淡黃袍改服白袍。」二日己亥,大祥。四日辛丑,禫祭禮畢。五日壬寅,百官請聽政,不允。八日,百官三上表,引《康誥》「被冕服出應門」等語以證。九日,詔可。



 十五年
 正月十八日甲寅,百日,帝過宮行焚香禮。二十一日丁巳,諭輔臣曰:「昨內引洪邁,見朕已過百日,猶服衰粗因奏事應以漸,今宜服如古人墨衰之義,而巾則用繒或羅。朕以羅絹非是,若用細布則可。」王淮等言:「尋常士大夫丁憂過百日,巾衫皆用細布,出而見客,則以黲布。今陛下舉曠古不能行之禮,足為萬世法。」帝又曰:「晚間引宿直官之類如何?」淮曰:「布巾、布背子便是常服。」上不以為然。自是每御延和殿,止服白布折上巾、布衫,過宮則
 衰絰而杖。



 三月壬子,啟攢,帝服初喪之服。甲寅,發引。丙寅,掩攢。甲戍,親行第七虞祭。大臣言:「虞祭乃吉禮,合用靴袍。」上曰:「只用布折上巾、黑帶、布袍可也。」



 二十日丙戍,神主祔廟。是日詔曰:「朕昨降指揮,欲衰絰三年,緣群臣屢請御殿易服,故以布素視事內殿。雖有俟過祔廟勉從所請之詔,稽諸典禮,心實未安,行之終制。乃為近古。宜體至意,勿復有請。」於是大臣乃不敢言。蓋三年之制,斷自帝心,執政近臣皆主易月之說。諫官謝鍔、禮官尤
 袤心知其不可,而不敢盡言。惟敕令所刪定官沉清臣再上書:「願堅『主聽大事於內殿』之旨,將來祔廟畢日,預降御筆,截然示以終喪之志,杜絕輔臣方來之章,勿令再有奏請,力全聖孝,以示百官,以刑四海。」帝納用焉。仍詔:「攢宮遵遺誥務從儉約,凡修營百費,並從內庫,毋侵有司經常之費。諸路監司、州軍府監止進慰表,其餘禮並免,不得以進奉攢宮為名,有所貢獻。」上陵名曰永思。



 紹熙五年六月九日,孝宗崩。太皇太后有旨,皇帝以疾
 聽在內成服,太皇太后代皇帝行禮。



 慶元二年六月九日,大祥。八月十六日,禫祭。時光宗不能執喪,寧宗嗣服,欲大祥畢更服兩月,曰:「但欲禮制全盡,不較此兩月。」於是監察御史胡紘言:「孫為祖服,已過期矣。議者欲更持禫兩月,不知用何典禮?若曰嫡孫承重,則太上聖躬亦已康復,於宮中自行二十七月之重服,而陛下又行之,是喪有二孤也。自古孫為祖服,何嘗有此禮?詔侍從、臺諫、給舍集議。吏部尚書葉翥等言:「孝宗升遐之初,太上
 聖體違豫,就宮中行三年之喪。皇帝受禪,正宜仿古方喪之服以為服,昨來有司失於討論。今胡紘所奏,引古據經,別嫌明微,委為允當。欲從所請,參以典故:六月六日,大祥禮畢,皇帝及百官並純吉服;七月一日,皇帝御正殿,饗祖廟。將來禫祭,令禮官檢照累朝禮例施行。」四月庚戌,詔:「群臣所議雖合禮經,然於朕追慕之意,有所未安,早來奏知太皇太后,面奉聖旨,以太上皇帝雖未康愈,宮中亦行三年之制,宜從所議。朕躬奉慈訓,敢不
 遵依。」



 初,高宗之喪,孝宗為三年服。及孝宗之喪,有司請於易月之外,用漆紗淺黃之制,蓋循紹興以前之舊。朱熹初至,不以為然,奏言:「今已往之失,不及追改,惟有將來啟攢發引,禮當復用初喪之服,則其變除之節,尚有可議。望明詔禮官稽考禮律,豫行指定。其官吏軍民方喪之服,亦宜稍為之制,勿使肆為華靡。」其後,詔中外百官,皆以涼衫視事,蓋用此也。方朱熹上議時,門人有疑者,未有以折之。後讀《禮記正義·喪服小記》「為祖後者」條,
 因自識於本議之末,其略云:「準《五服年月格》,斬衰三年,嫡孫為祖謂承重者,法意甚明,而《禮經》無文,但《傳》云:『父沒而為祖後者服斬。』然而不見本經,未詳何據。但《小記》云:『祖父沒而為祖母後者三年。』可以傍照。至『為祖後者』條下疏中所引《鄭志》,乃有『諸侯父有廢疾不任國政,不任喪事』之問,而鄭答以『天子、諸侯之服皆斬』之文,方見父在而承國於祖服。向來上此奏時,無文字可檢,又無朋友可問,故大約且以禮律言之。亦有疑父在不當承重者,
 時無明白證驗,但以禮律人情大意答之,心常不安。歸來稽考,始見此說,方得無疑。乃知學之不講,其害如此。而《禮經》之文,誠有闕略,不無待於後人。向使無鄭康成,則此事終未有所斷決,不可直謂古經定制,一字不可增損也。」已而詔於永思陵下宮之西,修奉攢宮,上陵名曰永阜。



 慶元六年,光宗崩,上陵名曰永崇。



 嘉定十七年,寧宗崩,上陵名曰永茂。



 景定五年,理宗崩,上陵名曰永穆。



 咸淳十年,度宗崩,上陵名曰永紹。



 自孝宗以降,外庭雖用易月之制,而宮中實行三年之喪云。



\end{pinyinscope}