\article{列傳第二百三十二姦臣三}

\begin{pinyinscope}

 黃
 潛善,字茂和,邵武人。擢進士第,宣和初,為左司郎。陝西、河東地大震,陵谷易處,徽宗命潛善察訪陝西,因往
 視。潛善歸,不以實聞,但言震而已。擢戶部侍郎,坐事謫亳州,以徽猷閣待制知河間府。



 靖康初,金人入攻,康王開大元帥府,檄潛善將兵入援。張邦昌僭位,潛善趨白于帥府,王承制拜潛善為副元帥。



 二年,高宗即位,拜中書侍郎。時上從人望,擢李綱為右相,綱將奏逐潛善及汪伯彥,右丞呂好問止之。未幾,潛善拜右僕射兼中書侍郎,綱遂罷。御史張所言潛善姦邪,恐害新政,左遷所尚書郎,尋謫江州。太學生陳東論李綱不可去,潛善、伯
 彥不可任,潛善恚。會歐陽澈上書詆時事,語侵宮掖,帝謂其言不實,潛善乘間啟殺澈並東誅之,識與不識皆為之垂涕,帝悔焉。



 明年,金人攻陝西,京東、山東盜起,潛善、伯彥匿不以聞。張遇焚真州,距行在六十里,內侍邵成章疏潛善、伯彥誤國,成章坐除名。御史馬伸亦以劾潛善、伯彥得罪,謫監濮州酒稅,道卒。



 潛善進左僕射兼門下侍郎。鄆、濮相繼陷沒,宿、泗屢警,右丞許景衡以扈衛單弱,請帝避其鋒,潛善以為不足慮,率同列聽浮屠
 克勤說法。俄泗州奏金人且至,帝大驚,決策南渡。御舟已戒,潛善、伯彥方共食,堂吏大呼曰:「駕行矣!」乃相視蒼黃鞭馬南馳。都人爭門而出,死者相枕藉,人無不怨憤。會司農卿黃鍔至江上,軍士聞其姓以為潛善也,爭數其罪,揮刃而前,鍔方辯其非是,而首已斷矣。



 帝渡瓜州,幸鎮江,敵兵已躡其後。潛善、伯彥聯疏言艱難之時,不敢具文求退。中丞張澄劾之,乃罷潛善為觀文殿大學士、知江寧府,落職居衡州。鄭瑴又論潛善、伯彥均于誤
 國,而潛善之惡居多,王庭秀繼以為言,責置英州。諫官袁植乞斬之都市,帝不許。尋卒于梅州。



 潛善猥持國柄,嫉害忠良。李綱既逐,張愨、宗澤、許景衡輩相繼貶死,憲諫一言,隨陷其禍,中外為之切齒。高宗末年有旨,潛善、余深、薛昂皆復官錄後。諫官淩哲言深、昂朋附蔡京,潛善專恣誤國,今盡復三人恩數,恐政刑失平,忠義解體。詔以潛善嘗任副元帥,特復元官,錄一子。



 汪伯彥,字廷俊,徽之祁門人。登進士第,積官為虞部郎
 官。靖康改元,召見,獻河北邊防十策,直龍圖閣、知相州。是冬,金人陷真定,詔徙真定帥司於相,俾伯彥領之。



 高宗以康王使金至磁,時金騎充斥,嘗有甲馬數百至城下,蹤跡王所在。伯彥亟以帛書請王還相,躬服橐鞬,部兵逆王於河上。王勞之曰:「他日見上,當首以京兆薦公。」其受知自此始矣。未幾,王奉蠟書,開天下兵馬大元帥府,以伯彥為副將。王引兵渡河,謀所向,言人人殊,伯彥獨曰:「非出北門濟子城不可。」王喜曰:「廷俊言是也。」既濟,
 由大名歷鄆、濟達于南京,奏為集英殿修撰。



 北兵薄京城,欽宗詔:金人見議通和,康王將兵,毋得輕動。伯彥以為然。宗澤曰:「女真狂譎,是欲款我師爾。如即信之,後悔何及乎!宜亟進兵。」伯彥等難之。及城破,金人逼二帝北行,張邦昌僭立,王聞之涕泣。明年春,王承制除伯彥顯謨閣待制,升元帥,進直學士。高宗即位,擢知樞密院事。未幾,拜右僕射。



 方高宗初政,天下望治。伯彥、潛善逾年在相位,專權自恣,不能有所經畫。御史諫官,下至韋布內
 侍,皆劾奏之。罷伯彥為觀文殿大學士、知洪州,改提舉崇福宮,尋落職居永州。紹興初,復職,知池州、江東安撫大使。言者弗置,乃詔以舊職奉祠,尋知廣州。四年,帝追贈陳東、歐陽澈。舍人王居正論伯彥、潛善不已,復褫前職。



 七年,帝謂輔臣曰:「元帥舊僚,往往淪謝,惟汪伯彥實同艱難。朕之故人,所存無幾,宜與牽復。」秦檜、張浚曰:「臣等已議曰郊恩取旨,更得天筆明其舊勞,庶幾內外孚信。」始伯彥之未第也,受館于王氏,檜嘗從之學,而浚亦
 伯彥所引,故共贊焉。九年,知宣州,過闕,帝謂檜曰:「伯彥便令之官,庶免紛紜。」又曰:「伯彥潛藩舊僚,去國七年。漢之高、光不忘豐沛、南陽故舊,皆人情之常。」伯彥上所著《中興日曆》五卷,拜檢校少傅、保信軍節度使。十年,請祠,從之。明年五月,卒,贈少師,諡「忠定」。



 初,伯彥既去相州,金人執其子軍器監丞似,使割地以至相州,守臣趙不試固守不下,遂拘而北,久之乃還。或云似之得歸,伯彥實使人贖之。似後更名召嗣。



 秦檜,字會之,江寧人。登政和五年第,補密州教授。繼中詞學兼茂科,歷太學學正。靖康元年,金兵攻汴京,遣使求三鎮,檜上兵機四事:一言金人要請無厭,乞止許燕山一路;二言金人狙詐,守禦不可緩;三乞集百官詳議,擇其當者載之誓書;四乞館金使於外,不可令入門及引上殿。不報。除職方員外郎。尋屬張邦昌為幹當公事,檜言:「是行專為割地,與臣初議矛盾,失臣本心。」三上章辭,許之。



 時議割三鎮以弭兵,命檜借禮部侍郎與程瑀
 為割地使,奉肅王以往。金師退,檜、瑀至燕而還。御史中丞李回、翰林承旨吳幵共薦檜,拜殿中侍御史,遷左司諫。王雲、李若水見金二酋歸,言金堅欲得地,不然,進兵取汴京。十一月,集百官議於延和殿,范宗尹等七十人請與之,檜等三十六人持不可。未幾,除御史中丞。



 閏十一月,汴京失守,二帝幸金營。二年二月,莫儔、吳幵自金營來,傳金帥命推立異姓。留守王時雍等召百官軍民共議立張邦昌,皆失色不敢答,監察御史馬伸言於眾
 曰:「吾曹職為爭臣,豈容坐視不吐一辭?當共入議狀,乞存趙氏。」時檜為臺長,聞伸言以為然,即進狀曰:



 :「檜荷國厚恩,甚愧無報。今金人擁重兵,臨已拔之城,操生殺之柄,必欲易姓,檜盡死以辨,非特忠於主也,且明兩國之利害爾。趙氏自祖宗以至嗣君,百七十餘載。頃緣姦臣敗盟,結怨鄰國,謀臣失計,誤主喪師,遂致生靈被禍,京都失守,主上出郊,求和軍前。兩元帥既允其議,布聞中外矣,且空竭帑藏,追取服御所用,割兩河地,恭為臣子,
 今乃變易前議,人臣安忍畏死不論哉?



 :宋於中國,號令一統,綿地萬里,德澤加于百姓,前古未有。雖興亡之命在天有數,焉可以一城決廢立哉?昔西漢絕於新室,光武以興;東漢絕于曹氏,劉備帝蜀;唐為朱溫篡奪,李克用猶推其世序而繼之。蓋基廣則難傾,根深則難拔。



 :張邦昌在上皇時,附會權幸,共為蠹國之政。社稷傾危,生民塗炭,固非一人所致,亦邦昌為之也。天下方疾之如仇讎,若付以土地,使主人民,四方豪傑必共起而誅之,終
 不足為大金屏翰。必立邦昌,則京師之民可服,天下之民不可服;京師之宗子可滅,天下之宗子不可滅。檜不顧斧鉞之誅,言兩朝之利害,願復嗣君位以安四方,非特大宋蒙福,亦大金萬世利也。」



 金人尋取檜詣軍前。三月,金人立邦昌為偽楚。邦昌遺金書請還孫傅、張叔夜及檜,不許。初,二帝北遷,檜與傅、叔夜、何㮚,司馬朴從至燕山,又徙韓州。上皇聞康王即位,作書貽粘罕,與約和議,俾檜潤色之。檜以厚賂達粘罕。會金主吳乞買以檜
 賜其弟撻懶為任用,撻懶攻山陽,建炎四年十月甲辰,檜與妻王氏及婢僕一家,自軍中取漣水軍水砦航海歸行在。丙午,檜入見。丁未,拜禮部尚書,賜以銀帛。



 檜之歸也,自言殺金人監己者奔舟而來。朝士多謂檜與㮚、傅、朴同拘,而檜獨歸;又自燕至楚二千八百里,逾河越海,豈無譏訶之者,安得殺監而南?就令從軍撻懶,金人縱之,必質妻屬,安得與王氏偕?惟宰相范宗尹、同知樞密院李回與檜善,盡破群疑,力薦其忠。未對前一日,帝
 命先見宰執。檜首言「如欲天下無事,南自南,北自北」,及首奏所草與撻懶求和書。帝曰:「檜樸忠過人,朕得之喜而不寐。蓋聞二帝、母后消息,又得一佳士也。」宗尹欲處之經筵,帝曰:「且與一事簡尚書。」故有禮部之命。從行王安道、馮由義、水砦丁不異及參議官並改京秩,舟人孫靖亦補承信郎。始,朝廷雖數遣使,但且守且和,而專與金人解仇議和,實自檜始。蓋檜在金庭首唱和議,故撻懶縱之使歸也。



 紹興元年二月,除參知政事。七月,宗尹罷。
 先是,范宗尹建議討論崇寧、大觀以來濫賞,檜力贊其議,見帝意堅,反以此擠之。宗尹既去,相位久虛。檜揚言曰:「我有二策,可聳動天下。」或問何以不言,檜曰:「今無相,不可行也。」八月,拜右僕射、同中書門下平章事兼知樞密院事。九月,呂頤浩再相,檜同秉政,謀奪其柄,風其黨建言:「周宣王內修外攘,故能中興,今二相宜分任內外。」頤浩遂建都督府于鎮江。帝曰:「頤浩專治軍旅,檜專理庶務,如種、蠡之分職可也。」



 二年,檜奏置修政局,自為提
 舉,參知政事翟汝文同領之。未幾,檜面劾汝文擅治堂吏,汝文求去;諫官方孟卿一再論之,汝文竟罷。監察御史劉一止,檜黨也,言:「宣王內修,修其所謂外攘之政而已。今簿書獄訟、官吏差除、土木營繕俱非所當急者。」屯田郎曾統亦謂檜曰:「宰相事無不統,何以局為?」檜皆不聽。既而有議廢局以搖檜者,一止及檢討官林待聘皆上疏言不可廢。七月,一止出臺,除起居郎,蓋自叛其說,識者笑之。



 頤浩自江上還,謀逐檜,有教以引朱勝非為
 助者。詔以勝非同都督。給事中胡安國言勝非不可用,勝非遂以醴泉觀使兼侍讀。安國求去,檜三上章留之,不報。頤浩尋以黃龜年為殿中侍御史,劉棐為右司諫,蓋將逐檜。於是江躋、吳表臣、程瑀、張燾、胡世將、劉一止、林待聘、樓炤並落職予祠,臺省一空,皆檜黨也。檜初欲傾頤浩,引一時名賢如安國、燾、瑀輩布列清要。頤浩問去檜之術于席益,益曰:「目為黨可也。今黨魁胡安國在瑣闥,宜先去之。」蓋安國嘗問人材于遊酢,酢以檜為言,
 且比之荀文若。故安國力言檜賢于張浚諸人,檜亦力引安國。至是,安國等去,檜亦尋去。檜再相誤國,安國已死矣。黃龜年始劾檜專主和議,沮止恢復,植黨專權,漸不可長,至比檜為莽、卓。八月,檜罷,乃為觀文殿學士、提舉江州太平觀。



 前一日,上召直學士院綦崈禮入對,示以檜所陳二策,欲以河北人還金國,中原人還劉豫。帝曰:「檜言『南人歸南,北人歸北』。朕北人,將安歸?檜又言『為相數月,可聳動天下』,今無聞。」崈禮即以上意載訓辭,播
 告中外,人始知檜之姦。龜年等論檜不已,詔落職,榜朝堂,示不復用。三年,韓肖胄等使還,洎金使李永壽、王翊偕來,求盡還北俘,與檜前議吻合。識者益知檜與金人共謀,國家之辱未已也。



 五年,金主既死,撻懶主議,卒成其和。二月,復資政殿學士,仍舊宮祠。六月,除觀文殿學士、知溫州。六年七月,改知紹興府。尋除醴泉觀使兼侍讀,充行宮留守;孟庾同留守,並權赴尚書、樞密院參決庶事。時已降詔將行幸,檜乞扈從,不許。帝駐蹕平江,召
 檜赴行在,用右相張浚薦也。十二月,檜以醴泉觀兼侍讀赴講筵。七年正月,何蘚使金還,得徽宗及寧德后訃,帝號慟發喪,即日授檜樞密使,恩數視宰臣。四月,命王倫使金國迎奉梓宮。



 九月,浚求去,帝問:「誰可代卿?」浚不對。帝曰:「秦檜何如?」浚曰:「與之共事,始知其暗。」帝曰:「然則用趙鼎。」鼎於是復相。臺諫交章論浚,安置嶺表。鼎約同列救解。與張守面奏,各數千百言,檜獨無一語。浚遂謫永州。始,浚、鼎相得甚,浚先達,力引鼎。嘗共論人才,浚劇
 談檜善,鼎曰:「此人得志,吾人無所措足矣!」浚不以為然,故引檜,共政方知其暗,不復再薦也。檜因此憾浚,反謂鼎曰:「上欲召公,而張相遲留。」蓋怒鼎使擠浚也。檜在樞府惟聽鼎,鼎素惡檜,由是反深信之,卒為所傾。鼎與浚晚遇於閩,言及此,始知皆為檜所賣。



 十一月,奉使朱弁以書報粘罕死,帝曰:「金人暴虐,不亡何待?」檜曰:「陛下但積德,中興固有時。」帝曰:「此固有時,然亦須有所施為,然後可以得志。」



 八年三月,拜右僕射、同中書門下平章事
 兼樞密使。吏部侍郎晏敦復有憂色,曰:「姦人相矣。」五月,金遣烏陵思謀等來議和,與王倫偕至。思謀即宣和始通好海上者。議以吏部侍郎魏矼館伴,矼辭曰:「頃任御史,嘗言和議之非,今不可專對。」檜問矼所以不主和,矼備言敵情。檜曰:「公以智料敵,檜以誠待敵。」矼曰:「第恐敵不以誠待相公爾。」檜乃改命。六月,思謀等入見。帝愀然謂宰相曰:「先帝梓宮,果有還期,雖待二三年尚庶幾。惟是太后春秋高,朕旦夕思念,欲早相見,此所以不憚屈
 己,冀和議之速成也。」檜曰:「屈己議和,此人主之孝也。見主卑屈,懷憤不平,此人臣之忠也。」帝曰:「雖然,有備無患,使和議可成,邊備亦不可弛。」



 十月,宰執入見,檜獨留身,言:「臣僚畏首尾,多持兩端,此不足與斷大事。若陛下決欲講和,乞顓與臣議,勿許群臣預。」帝曰:「朕獨委卿。」檜曰:「臣亦恐未便,望陛下更思三日,容臣別奏。」又三日,檜復留身奏事,帝意欲和甚堅,檜猶以為未也,曰:「臣恐別有未便,欲望陛下更思三日,容臣別奏。」帝曰:「然。」又三日。檜
 復留身奏事如初,知上意確不移,乃出文字乞決和議,勿許群臣預。



 鼎力求去位,以少傅出知紹興府。初,帝無子。建炎末,范宗尹造膝有請,遂命宗室令懬擇藝祖後,得伯琮、伯玖入宮,皆藝祖七世孫。伯琮改名瑗,伯玖改名璩。瑗先建節,封建國公。帝諭鼎專任其事。又請建資善堂,鼎罷,言者攻鼎,必以資善為口實。及鼎、檜再相,帝出御劄,除璩節度使,封吳國公。執政聚議,樞密副使王庶見之,大呼曰:「並后匹嫡,此不可行。」鼎以問檜,不答。檜
 更問鼎,鼎曰:「自丙辰罷相,議者專以此藉口,今當避嫌。」約同奏面納御筆,及至帝前,檜無一語。鼎曰:「今建國在上,名雖未正,天下之人知陛下有子矣。今日禮數不得不異。」帝乃留御筆俟議。明日,檜留身奏事。後數日,參知政事劉大中參告,亦以此為言。故鼎與大中俱罷。明年,璩卒授保大軍節度使,封崇國公。故鼎入辭,勸帝曰:「臣去後,必有以孝弟之說脅制陛下者。」出見檜,一揖而去,檜亦憾之。



 鼎既去,檜獨專國,決意議和。中朝賢士,以議
 論不合,相繼而去。於是,中書舍人呂本中、禮部侍郎張九成皆不附和議,檜諭之使優遊委曲,九成曰:「未有枉己而能正人者。」檜深憾之。殿中侍御史張戒上疏乞留趙鼎,又陳十三事論和議之非,忤檜。王庶與檜尤不合,自淮西入樞庭,始終言和議非是,疏凡七上,且謂檜曰:「而忘東都欲存趙氏時,何遺此敵邪?」檜方挾金人自重,尤恨庶言,故出之。



 樞密院編修官胡銓上疏,願斬檜與王倫以謝天下。於是上下洶洶。檜謬為解救,卒械送銓
 貶昭州。陳剛中以啟賀銓,檜大怒,送剛中吏部,差知贛州安遠縣。贛有十二邑,安遠濱嶺,地惡瘴深,諺曰:「龍南、安遠,一去不轉。」言必死也。剛中果死。尋以銓事戒諭中外。既而校書郎許忻、樞密院編修官趙雍同日上疏,猶祖銓意,力排和議。雍又欲正南北兄弟之名,檜亦不能罪。曾開見檜,言今日當論存亡,不當論安危。檜駭愕,遂出之。司勳員外郎朱松、館職胡珵、張擴、淩景夏、常明、范如圭同上一疏言:「金人以和之一字得志於我者十有
 二年,以覆我王室,以弛我邊備,以竭我國力,以懈緩我不共戴天之仇,以絕望我中國謳吟思漢之赤子,以詔諭江南為名,要陛下以稽首之禮。自公卿大夫至六軍萬姓,莫不扼腕憤怒,豈肯聽陛下北面為仇敵之臣哉!天下將有仗大義,問相公之罪者。」後數日,權吏部尚書張燾、吏部侍郎晏敦復、魏矼、戶部侍郎李彌遜、梁汝嘉、給事中樓炤、中書舍人蘇符、工部侍郎蕭振、起居舍人薛徽言同班入奏,極言屈己之禮非是。新除禮部侍郎
 尹焞獨上疏,且移書切責檜,檜始大怒,焞于是固辭新命不拜。奉禮郎馮時行召對,言和議不可信,至引漢高祖分羹事為喻。帝曰:「朕不忍聞。」顰蹙而起。檜乃謫時行知萬州,尋亦抵罪。中書舍人勾龍如淵抗言於檜曰:「邪說橫起,胡不擇臺官擊去之。」檜遂奏如淵為御史中丞,首劾銓。



 金使張通古、蕭哲以詔諭江南為名,檜猶恐物論咎己,與哲等議,改江南為宋,詔諭為國信。京淮宣撫處置使韓世忠凡四上疏力諫,有「金以劉豫相待」之語,
 且言兵勢重處,願以身當之,不許。哲等既至泗州,要所過州縣迎以臣禮,至臨安日,欲帝待以客禮,世忠益憤,再疏言:「金以詔諭為名,暗致陛下歸順之義,此主辱臣死之時,願效死戰以決勝敗。若其不克,委曲從之未晚。」亦不許。哲等既入境,接伴使范同再拜問金主起居,軍民見者,往往流涕。過平江,守臣向子諲不拜,乞致仕。哲等至淮安,言先歸河南地,且冊上為帝,徐議餘事。



 檜至是欲上行屈己之禮,帝曰:「朕嗣守太祖、太宗基業,豈可
 受金人封冊。」會三衙帥楊沂中、解潛、韓世良相率見檜曰:「軍民洶洶,若之何?」退,又白之臺諫。於是勾龍如淵、李誼數見檜議國書事,如淵謂得其書納之禁中,則禮不行而事定。給事中樓炤亦舉「諒陰三年不言」事以告檜,於是定檜攝塚宰受書之議。帝亦切責王倫,倫諭金使,金使亦懼而從。帝命檜即館中見哲等受其書。金使欲百官備禮,檜使省吏朝服導從,以書納禁中。先一日,詔金使來,將盡割河南、陝西故地,又許還梓宮及母兄親
 族,初無需索。以參知政事李光素有時望,俾押和議榜以鎮浮言。又降御劄賜三大將。



 九年,金人歸河南、陝西故地,以王倫簽書樞密院事,充迎奉梓宮、奉還兩宮、交割地界使,藍公佐副之。判大宗正事士㒟、兵部侍郎張燾朝八陵。帝謂宰執曰:「河南新復,宜命守臣專撫遺民,勸農桑,各因其地以食,因其人以守,不可移東南之財,虛內以事外。」帝雖聽檜和而實疑金詐,未嘗弛備也。



 時張浚在永州,馳奏,力言以石晉、劉豫為戒,復遺書孫近,
 以「帝秦之禍,發遲而大」。徐俯守上饒,連南夫帥廣東,岳飛宣撫淮西,皆因賀表寓諷。俯曰:「禍福倚伏,情偽多端。」南夫曰:「不信亦信,其然豈然?雖虞舜之十二州,皆歸王化;然商於之六百里,當念爾欺!」飛曰:「救暫急而解倒懸,猶之可也;欲長慮而尊中國,豈其然乎?」他如秘書省正字汪應辰、樊光遠、澧州推官韓紃、臨安府司戶參軍毛叔慶,皆言金人叵測;迪功郎張行成獻《詢蕘書》二十篇,大意言自古講和,未有終不變者,條具者皆豫備之策。
 檜悉加黜責,紃貶循州。



 七月,兀朮殺其領三省事宗磐及左副元帥撻懶,拘王倫於中山府。蓋兀朮以歸地為二人所主,將有他謀也。倫嘗密奏於朝,檜不之備,但趣倫進。時韓世忠有乘懈掩擊之請,檜言《春秋》不伐喪,與帝意合,遂已。



 十年,金人果敗盟,分四道入侵。兀朮入東京,葛王褎取南京,李成取西京,撒離喝趨永興軍。河南諸郡相繼陷沒。帝始大怪,下詔罪狀兀朮。御史中丞王次翁奏曰:「前日國是,初無主議。事有小變,則更用他相,
 後來者未必賢,而排黜異黨,紛紛累月不能定,願陛下以為至戒。」帝深然之。檜力排群言,始終以和議自任,而次翁謂無主議者,專為檜地也。於是檜位復安,據之凡十八年,公論不能撼搖矣。



 六月,檜奏曰:「德無常師,主善為師。臣昨見撻懶有割地講和之議,故贊陛下取河南故疆。今兀朮戕其叔撻懶,藍公佐歸,和議已變,故贊陛下定吊伐之計。願至江上諭諸帥同力招討。」卒不行。閏六月,貶趙鼎興化軍,以王次翁受檜旨,言其規圖復用
 也。言者不已,尋竄潮州。



 時張俊克亳州,王勝克海州,岳飛克郾城,幾獲兀朮。張浚戰勝于長安,韓世忠勝於泇口鎮,諸將所向皆奏捷,而檜力主班師。九月,詔飛還行在,沂中還鎮江,光世還池州,錡還太平。飛軍聞詔,旗靡轍亂,飛口呿不能合。於是淮寧、蔡、鄭復為金人有。以明堂恩封檜莘國公。十一年,兀朮再舉,取壽春,入廬州,諸將邵隆、王德、關師古等連戰皆捷。楊沂中戰拓皋,又破之。檜忽諭沂中及張俊遽班師。韓世忠聞之,止濠州不
 進;劉錡聞之,棄壽春而歸。自是不復出兵。



 四月,檜欲盡收諸將兵權,給事中范同獻策,檜納之。密奏召三大將論功行賞,韓世忠、張俊並為樞密使,岳飛為副使,以宣撫司軍隸樞密院。六月,拜左僕射、同中書門下平章事兼樞密使,進封慶國公。《徽宗實錄》成,遷少保,加封冀國公。先是,莫將、韓恕使金,拘於涿州。至是,兀朮有求和意,縱之歸。檜復奏遣劉光遠、曹勳使金,又以魏良臣為通問使。未幾,良臣偕金使蕭毅等來,議以淮水為界,求割
 唐、鄧二州。尋遣何鑄報聘,許之。



 十月,興岳飛之獄。檜使諫官万俟禼論其罪,張俊又誣飛舊將張憲謀反,於是飛及子雲俱送大理寺,命御史中丞何鑄、大理卿周三畏鞫之。十一月,貶李光藤州,范同罷參知政事。同雖附和議,以自奏事,檜忌之也。十二月,殺岳飛。檜以飛屢言和議失計,且嘗奏請定國本,俱與檜大異,必欲殺之。鑄、三畏初鞫,久不伏;禼入臺,獄遂上。誣飛嘗自言「己與太祖皆三十歲建節」為指斥乘輿,受詔不救淮西罪,賜死
 獄中。子雲及張憲殺於都市。天下冤之,聞者流涕。飛之死,張俊有力焉,語在《宋史/卷365
 
 '''飛傳'''》。



 十二年,胡銓再編管新州。八月,徽宗及顯肅、懿節二梓宮至行在。太后還慈寧宮。九月,加太師,進封魏國公。十月,進封秦、魏兩國公。檜以封兩國與蔡京、童貫同,請改封母為秦、魏國夫人。子熺舉進士,館客何溥赴南省,皆為第一。熺本王唤孽子,檜妻喚妹,無子,喚妻貴而妒,檜在金國,齣熺為檜後。檜還,其家以熺見,檜喜甚。檜幸和議復成,益咎前日之異己者。
 先是,趙鼎貶潮州,王庶貶道州,胡銓再貶新州。至是,皆遇赦永不檢舉。曾開、李彌遜並落職。張俊本助和議,居位歲餘無去意,檜諷江邈論罷之。



 十三年,賀瑞雪,賀雪自檜始。賀日食不見,是後日食多書不見。彗星常見,選人康倬上書言彗星不足畏,檜大喜,特改京秩。楚州奏鹽城縣海清,檜請賀,帝不許。知虔州薛弼言木內有文曰「天下太平年」,詔付史館。於是修飾彌文,以粉飾治具,如鄉飲、耕籍之類節節備舉,為苟安餘杭之計,自此不
 復巡幸江上,而祥瑞之奏日聞矣。



 洪皓歸自金國,名節獨著,以致金酋室撚語,直翰苑不一月逐去。室撚者,粘罕之左右也。初,粘罕行軍至淮上,檜嘗為之草檄,為室撚所見,故因皓歸寄聲。檜意士大夫莫有知者,聞皓語,深以為憾,遂令李文會論之。胡舜陟以非笑朝政下獄死,張九成以鼓唱浮言貶,累及僧宗杲,編配,皆以語忤檜也。張邵亦坐與檜言金人有歸欽宗及諸王后妃意,斥為外祠。十四年,貶黃龜年,以前嘗論檜也。閩、浙大水,
 右武大夫白鍔有「燮理乖謬」語,刺配萬安軍。太學生張伯麟嘗題壁曰:「夫差,爾忘越王殺而父乎?」杖脊刺配吉陽軍。故將解潛罷官閒居,辛永宗總戎外郡,亦坐不附和議,潛竄南安死,永宗編置肇慶死。趙鼎、李光皆再竄過海。皓之罪由白鍔延譽,光以在藤州唱和有諷刺及檜者,為守臣所告也。



 先是,議建國公出閣,吏部尚書吳表臣、禮部尚書蘇符等七人論禮與檜意異,於是表臣等以討論不祥、懷姦附鼎皆罷。始,檜為上言:「趙鼎欲立
 皇太子,是待陛下終無子也,宜俟親子乃立。」遂嗾御史中丞詹大方言鼎邪謀密計,深不可測,與范冲等咸懷異意,以徼無妄之福。沖嘗為資善翊善,故大方誣之。其後監察御史王鎡言帝未有嗣,宜祠高禖,詔築壇於圜丘東,皆檜意也。



 台州曾惇獻檜詩稱「聖相」。凡投獻者以皋、夔、稷、契為不足,必曰「元聖」。檜乞禁野史。又命子熺以秘書少監、領國史,進建炎元年至紹興十二年《日曆》五百九十卷。熺因太后北還,自頌檜功德凡二千餘言,使
 著作郎王揚英、周執羔上之,皆遷秩。自檜再相,凡前罷相以來詔書章疏稍及檜者,率更易焚棄,日曆、時政亡失已多,是後記錄皆熺筆,無復有公是非矣。冬十月,右正言何若指程頤、張載遺書為專門曲學,力加禁絕,人無敢以為非。



 十五年,熺除翰林學士兼侍讀。四月,賜檜甲第,命教坊樂導之入,賜緡錢金綿有差。六月,帝幸檜第,檜妻婦子孫皆加恩。檜先禁私史,七月,又對帝言私史害正道。時司馬伋遂言《涑水記聞》非其曾祖光論著
 之書,其後李光家亦舉光所藏書萬卷焚之。十月,帝親書「一德格天」扁其閣。十六年正月,檜立家廟。三月,賜祭器,將相賜祭器自檜始。



 先是,帝以彗星見求言。張浚上疏,言:「今事勢如養大疽于頭目心腹之間,不決不止,願謀為豫備。不然,異時以國與敵者,反歸罪正議。」檜久憾浚,至是大怒,即落浚節鉞,貶連州,尋移永州。



 十七年,改封檜益國公。五月,移貶洪皓于英州。八月,趙鼎死于吉陽軍。是夏,先有趙鼎遇赦永不檢舉之旨,又令月申存
 亡,鼎知之,不食而卒。自鼎之謫,門人故吏皆被羅織,雖聞其死而歎息者亦加以罪。又竄呂頤浩子摭於藤州。十二月,進士施鍔上《中興頌》、《行都賦》及《紹興雅》十篇,永免文解。自此頌詠導諛愈多。賜百官喜雪御筵於檜第。



 十八年,熺除知樞密院事,檜問胡寅曰:「外議如何?」寅曰:「以為公相必不襲蔡京之跡。」五月,李顯忠上恢復策,落軍職,與祠。六月,迪功郎王廷珪編管辰州,以作詩送胡銓也。閏八月,福州言民采竹實萬斛以濟饑。十一月,胡銓
 自新州移貶吉陽軍,以作頌謗訕也。



 十九年,帝命繪檜像,自為贊。是歲,湖、廣、江西、建康府皆言甘露降,諸郡奏獄空。帝嘗語檜曰:「自今有奏獄空者,當令監司驗實。果妄誕,即按治,仍命御史臺察之。苟不懲戒,則奏甘露瑞芝之類,崇虛飾誕,無所不至。」帝雖眷檜,而不可蔽欺也如此。十二月,禁私作野史,許人告。



 二十年正月,檜趨朝,殿司小校施全刺檜不中,磔於市。自是每出,列五十兵持長梃以自衛。是月,曹泳告李光子孟堅省記光所作私史,
 獄成,光竄已久,詔永不檢舉;孟堅編置峽州;朝士連坐者八人,皆落職貶秩;胡寅竄新州。泳由是驟用。五月,秘書少監湯思退奏以檜存趙氏本末付史館。六月,熺加少保。鄭煒告其鄉人福建安撫司機宜吳元美作《夏二子傳》,指蚊、蠅也;家有潛光亭、商隱堂,以亭號潛光,有心于黨李,堂名商隱,無意于事秦。故檜尤惡之。編管右迪功郎安誠、布衣汪大圭,斬有蔭人惠俊、進義副尉劉允中,黥徑山僧清言,皆以訕謗也。時檜疾愈,朝參許肩輿,二
 孫扶掖,仍免拜。二十一年,朝散郎王揚英上書薦熺為相,檜奏揚英知泰州。



 二十二年,又興王庶二子之奇、之荀、葉三省、楊煒、袁敏求四大獄,皆坐謗訕。煒又以嘗登李光、蕭振之門,言時事也。於是光永不檢舉,振貶池州。二十三年,檜請下台州于謝伋家取綦崈禮所受御筆繳進。檜初罷相,上有責檜語,欲泯其跡焉。是歲,進士黃友龍坐謗訕,黥配嶺南;內侍裴詠坐指斥,編管瓊州。二十四年二月,楊炬以弟煒舊累死賓州,炬編管邕州。何
 兌訟其師馬伸發端上金人書乞存趙氏,為分檜功,兌編管英州。三月,檜孫敷文閣待制塤試進士舉,省殿試皆為第一,檜從子焞、焴、姻黨周夤,沈興傑皆登上第,士論為之不平。考官則魏師遜、湯思退、鄭仲熊、沈虛中、董德元也。師遜等初知貢舉,即語人曰:「吾曹可以富貴矣。」及廷試,檜又奏思退為編排,師遜為詳定。塤與第二人曹冠策皆攻專門之學,張孝祥策則主一德元老且及存趙事。帝讀塤策,皆檜、熺語,於是擢孝祥為第一,降
 塤第三。未幾,塤修撰實錄院,宰相子孫同領史職,前所無也。



 六月,以王循友前知建康嘗罪檜族黨,循友安置藤州。八月,王趯為李光求內徙,趯編管辰州。鄭玘、賈子展以會中有嘲謔講和之語,玘竄容州,子展竄德慶府。方疇以與胡銓通書,編置永州。十二月,魏安行、洪興祖以廣傳程瑀《論語解》,安行編置欽州,興祖編置昭州。又竄程緯,以其慢上無禮也。



 帝嘗諭檜曰:「近輪對者,多謁告避免。百官輪對,正欲聞所未聞,可令檢舉約束。」檜擅政
 以來,屏塞人言,蔽上耳目,凡一時獻言者,非誦檜功德,則訐人語言以中傷善類。欲有言者恐觸忌諱,畏言國事,僅論銷金鋪翠、乞禁鹿胎冠子之類,以塞責而已。故帝及之,蓋亦防檜之壅蔽也。



 衢州嘗有盜起,檜遣殿前司將官辛立將千人捕之,不以聞。晉安郡王因入侍言之,帝大驚,問檜,檜曰:「不足上煩聖慮,故不敢聞,盜平即奏矣。」退而求其故,知晉安言之,遂奏晉安居秀王喪不當給俸,月損二百緡,帝為出內帑給之。



 二十五年二月,
 以沈長卿舊與李光啟譏和議,又與芮燁共賦《牡丹詩》,有「寧令漢社稷,變作莽乾坤」之句,為鄰人所告,長卿編置化州,燁武岡軍。靜江有驛名秦城,知府呂愿中率賓僚共賦《秦城王氣詩》以媚檜,不賦者劉芮、李燮、羅博文三人而已。愿中由此得召。又張扶請檜乘金根車,又有乞置益國官屬及議九錫者,檜聞之安然。十月,申禁專門之學。以太廟靈芝繪為華旗,凡郡國所奏瑞木、嘉禾、瑞瓜、雙蓮悉繪之。



 趙令衿觀檜《家廟記》,口誦「君子之澤,
 五世而斬」,為汪召錫所告。御史徐𡕇又論趙鼎子汾與令衿飲別厚贐,必有姦謀,詔送大理,拘令衿南外宗正司。檜於一德格天閣書趙鼎、李光、胡銓姓名,必欲殺之而後已。鼎已死而憾之不置,遂欲孥戮汾。檜忌張浚尤甚,故令衿之獄,張宗元之罷,皆波及浚。浚在永州,檜又使其死黨張柄知潭州,與郡丞汪召錫共伺察之。至是,使汾自誣與浚及李光、胡寅謀大逆,凡一時賢士五十三人皆與焉。獄成,而檜病不能書。



 是月乙未,帝幸檜第
 問疾,檜無一語,惟流涕而已。熺奏請代居相位者,帝曰:「此事卿不當與。」帝遂命權直學士院沈虛中草檜父子致仕制。熺猶遣其子塤與林一飛、鄭柟夜見臺諫徐喜、張扶謀奏請己為相。丙申,詔檜加封建康郡王,熺進少師,皆致仕,塤、堪並提舉江州太平興國宮。是夜,檜卒,年六十六。後贈申王,諡「忠獻」。



 檜兩據相位者,凡十九年,劫制君父,包藏禍心,倡和誤國,忘仇斁倫。一時忠臣良將,誅鋤略盡。其頑鈍無恥者,率為檜用,爭以誣陷善類為功。其
 矯誣也,無罪可狀,不過曰謗訕,曰指斥,曰怨望,曰立黨沽名,甚則曰有無君心。凡論人章疏,皆檜自操以授言者,識之者曰:「此老秦筆也。」察事之卒,佈滿京城,小涉譏議,即捕治,中以深文。又陰結內侍及醫師王繼先,伺上動靜。郡國事惟申省,無一至上前者。檜死,帝方與人言之。



 檜立久任之說,士淹滯失職,有十年不解者。附己者立與擢用。自其獨相,至死之日,易執政二十八人,皆世無一譽。柔佞易制者,如孫近、韓肖胄、樓炤、王次翁、范同、
 万俟禼、程克俊、李文會、楊愿、李若谷、何若、段拂、汪勃、詹大方、余堯弼、巫伋、章夏、宋朴、史才、魏師遜、施鉅、鄭仲熊之徒,率拔之冗散,遽躋政地。既共政,則拱默而已。又多自言官聽檜彈擊,輒以政府報之,由中丞、諫議而升者凡十有二人,然甫入即出,或一閱月,或半年即罷去。惟王次翁閱四年,以金人敗盟之初持不易相之論,檜德之深也。開門受賂,富敵于國,外國珍寶,死猶及門。人謂熺自檜秉政無日不鍛酒具,治書畫,特其細爾。



 檜陰
 險如崖阱,深阻竟叵測。同列論事上前,未嘗力辨,但以一二語傾擠之。李光嘗與檜爭論,言頗侵檜,檜不答。及光言畢,檜徐曰:「李光無人臣禮。」帝始怒之。凡陷忠良,率用此術。晚年殘忍尤甚,數興大獄,而又喜諛佞,不避形跡。



 然檜死熺廢,其黨祖述餘說,力持和議,以竊據相位者尚數人,至孝宗始蕩滌無餘。開禧二年四月,追奪王爵,改諡「謬醜」。嘉定元年,史彌遠奏復王爵、贈諡。



\end{pinyinscope}