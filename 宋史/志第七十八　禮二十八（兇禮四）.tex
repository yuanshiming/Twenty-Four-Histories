\article{志第七十八 禮二十八(兇禮四)}

\begin{pinyinscope}

 士庶人喪禮服紀



 士庶人喪禮。開寶三年十月,詔開封府:禁喪葬之家不得用道、釋威儀及裝束異色人物前引。太平興國七年
 正月,命翰林學士李昉等復位士庶喪葬制度。昉等奏議曰:「唐大歷七年,詔喪葬之家送葬祭盤,只得於喪家及塋所置祭,不得於街衢張設。又長慶三年,令百姓喪葬祭奠不得以金銀、錦繡為飾及陳設音樂,葬物稍涉僭越,並勒毀除。臣等參詳子孫之葬父祖,卑幼之葬尊親,全尚樸素即有傷孝道。其所用錦繡,伏請不加禁斷。其用音樂及欄街設祭,身無官而葬用方相者,望嚴禁之。其詔葬設祭者不在此限。又準後唐長興二年詔:五
 品、六品常參官,喪輿舁者二十人,挽歌八人,明器三十事,共置八床;七品常參官舁者十六人,挽歌六人,明器二十事,置六床;六品以下京官及檢校、試官等,舁者十二人,挽歌四人,明器十五事,置五床,並許設紗籠二。庶人,舁者八人,明器十二事,置兩床。悉用香輿、魂車。其品官葬祖父母、父母,品卑者聽以子品,葬妻子者遞降一等,其四品以上依令式施行。望令御史臺、街司頒行,限百日率從新制;限滿違者,以違禁之物給巡司為賞。喪
 家輒舉樂者,譴伶人。他不如制者,但罪下裏工作。」從之。



 九年,詔曰:「訪聞喪葬之家,有舉樂及令章者。蓋聞鄰里之內,喪不相舂,苴麻之旁,食未嘗飽,此聖王教化之道,治世不刊之言。何乃匪人,親罹釁酷,或則舉奠之際歌吹為娛,靈柩之前令章為戲,甚傷風教,實紊人倫。今後有犯此者,並以不孝論,預坐人等第科斷。所在官吏,常加覺察,如不用心,並當連坐。」



 景德二年,開封府言:「文武官亡歿,諸寺擊鐘未有定制。欲望自今大卿監、大將軍、
 觀察使、命婦郡夫人已上,即據狀聞奏,許於天清、開寶二寺擊鐘,其聲數旋俟進止,自餘悉禁。」從之。



 紹興二十七年,監登聞鼓院範同言:「今民俗有所謂火化者,生則奉養之具唯恐不至,死則燔爇而棄捐之,何獨厚於生而薄於死乎?甚者焚而置之水中,識者見之動心。國朝著令,貧無葬地者,許以系官之地安葬。河東地狹人眾,雖至親之喪,悉皆焚棄。韓琦鎮並州,以官錢市田數頃,給民安葬,至今為美談。然則承流宣化,使民不畔於禮
 法,正守臣之職也。方今火葬之慘,日益熾甚,事關風化,理宜禁止。仍飭守臣措置荒閑之地,使貧民得以收葬,少裨風化之美。」從之。二十八年,戶部侍郎榮薿言:「比因臣僚陳請禁火葬,令州郡置荒閑之地,使貧民得以收葬,誠為善政。臣聞吳越之俗,葬送費廣,必積累而後辦。至於貧下之家,送終之具,唯務從簡,是以從來率以火化為便,相習成風,勢難遽革。況州縣休息之久,生聚日繁,所用之地,必須寬廣。乃附郭近便處,官司以艱得之
 故,有未行摽撥者。既葬埋未有處所,而行火化之禁,恐非人情所安。欲乞除豪富士族申嚴禁止外,貧下之民並客旅遠方之人,若有死亡,姑從其便,候將來州縣摽撥到荒閑之地,別行取旨。」詔依,仍令諸州依已降指揮,措置摽撥。



 服紀。宋天子及諸臣服制,前史皆散記諸禮中,未嘗特錄之也,後史則表而出之。高宗於外廷以日易月,於內廷則行三年之禮,御朝則淺素、淺黃。孝宗又力持三年
 之制。皇帝未成服,則素紗軟腳帕頭、白羅袍、黑銀帶、絲鞋。成服日,布梁冠朱熹云:當用十二梁、首絰、直領布大袖衫朱熹云:不當用衣蘭,蓋下已有裙、布裙、褲、腰絰、竹杖、白綾襯衫,或斜巾、帽子。視事日,去杖、首絰。小祥日,改服布帕頭、衣蘭衫、腰絰、布褲。大祥畢,服素紗軟腳帕頭、白羅袍、素履、黑銀帶。禫祭畢,素紗軟腳帕頭、淺色黃羅袍、黑銀帶。祔廟日,服履、黃袍、紅帶。御正殿視事,則皂帕頭、淡黃袍、黑□犀帶、素絲鞋。此中興後制也。



 孝宗居憂,再定三年之制。其服:布冠、直領
 大袖衫、布裙、首絰、腰絰、竹杖。小祥不易服。大祥禮畢,始去杖、去絰。禫祭畢,始服素紗軟腳帕頭、白袍、黑銀帶。祔廟畢,服皂帕頭、黑□犀帶。每遇過宮廟謁,則衰絰行禮,二十五月而除。三年之內,禁中常服布巾、布衫、布背子。視事則御內殿,服白布帕頭、白布袍、黑銀帶,殿設素幄。每五日一次過宮,則衰絰而杖。虞祭則布折上巾、黑帶、布袍。受金使吊則衰絰,御德壽殿東廊之素幄。受賀節使,則御垂拱殿東楹之素幄。是時,宰執、近臣皆不肯行,
 惟斷自上心,堅不可奪,大臣乃不敢言。贊其決者,惟敕局下僚沉清臣一人而已。



 臣為君服,宋制有三等:中書門下、樞密使副、尚書、翰林學士、節度使、金吾上將軍、文武二品以上,布梁冠、直領大袖衫、布裙、褲、腰絰、竹杖,或布帕頭、襴衫、布斜巾、絹襯服。文武五品以上並職事官監察御史以上、內客省、宣政、昭宣、知閣門事、前殿都知、押班,布梁冠、直領大袖衫、裙、褲、腰絰,或帕頭、衣蘭衫。自餘文武百官,布帕頭、衣蘭衫、腰
 絰而已。入局治事,並不易服。宰執奏事去杖,小祥去冠,餘官奏事如之。大祥,素紗軟腳折上巾、黲公服、白□錫帶。禫除畢,去黲服,常服仍黑帶、皂鞍韉。祔廟畢,始純吉服。宗室出則常服,居則衰麻以終制。



 光宗居孝宗之憂,趙汝愚當國,始令群臣服白涼衫、皂帶治事,逮終制乃止。寧宗居光宗之憂,復令百官以日易月,禫除畢,服紫衫、皂帶以治事,從禮部侍郎陳宗召請也。諸路監司、州軍縣鎮長吏以下,服布四腳、直領布衣蘭衫、麻腰絰,朝晡
 臨,三日除之。內外命婦當入臨者,布裙、初、帔、首絰、絹襯衫帕首。士庶於本家素服,三日而除。婚嫁,服除外不禁。文武臣僚之家,至山陵祔畢,乃許嫁娶,仍不用花彩及樂。



 淳熙十四年十月,以將作監韋璞充金國告哀使,閣門舍人姜特立副之。禮部、太堂寺言:「告哀使、副並三節人,從禮例,如在大祥內,合服布帕頭、示闌衫、布褲、腰絰,布涼傘,鞍韉;在禫服內,合服素紗軟腳帕頭、黲色公服、黑□呈犀帶,青傘,皂鞍韉;俟禫除,即從吉服,仍系黑帶,去魚,
 涼傘、韉並從禫制,並去犬戎座。三節人衣紫衫、黑帶,並不聽樂,不射弓弩,候過界,聽使、副審度,隨宜改易服用。」從之。或遣留遺信物使,同上服。



 喪服雜議。慶歷七年,侍御史吳鼎臣言:「武班及諸職司人吏,曾因親喪出入禁門,甚有裹素紗帕頭者,殊失肅下尊上之禮。欲乞武兩班,除以官品起復許裹素紗外,其餘臣僚並諸職司人吏,雖有親喪服未除,並須光紗加首,不得更裹素紗。」詔送太常禮院。禮官言:「準令文,
 兇服不入公門。其遭喪被起,在朝參處,常服各依品服,惟色以淺,無金玉飾;在家,依其服制。其被起者,及期喪以下居式假者,衣冠朝集,皆聽不預。今鼎臣所奏,有礙令文。」詔依所定,如遇筵宴,其服淺色素紗人,更不令祗應。



 丁父母憂。淳化五年八月,詔曰:「孝為百行之本,喪有三年之制,著於典禮,以厚人倫。中外文武官子弟,或父兄之淪亡,蒙朝廷之齒敘,未及卒哭,已聞蒞官,遽忘哀戚,
 頗玷風教。自今文武官子弟,有因父亡兄歿特被敘用,未經百日,不得趣赴公參。御史臺專加糾察;並有冒哀求仕、釋服從吉者,並以名聞。」



 咸平元年,詔任三司、館閣職事者丁憂,並令持服。又詔:「川峽、廣南、福建路官,丁憂不得離任,既受代而喪制未畢者,許其終制。」尋令川峽官,除州軍長吏奏裁,餘並許解官。



 大中祥符九年,殿中侍御史張廓言:「京朝官丁父母憂者,多因陳乞,與免持服。且忠教恩義,士所執守,一悖於禮,其何能立?今執事
 盈庭,各務簡易,況無金革之事,中外之官不闕,不可習以為例。望自後並依典禮,三年服滿,得赴朝請。」



 天禧四年,御史臺言:「文武官並丁憂者,相承服五十四月,別無條例。」下太常,禮官議曰:「按《禮·喪服小記》云:『父母之喪偕,先葬者不虞、祔,待後事,其葬服斬衰。』《注》:『謂同月若同日死也。先葬者母也,其葬服斬衰者,喪之隆哀宜從重也。假令父死在前月而同月葬,猶服斬衰,不葬不變服也。言其葬服斬衰,則虞、祔各以其服矣。及練、祥皆然。卒事,
 反服重。』《雜記》云:『有父之喪,如未沒喪而母死,其除父之喪也,服其除服,卒事,反喪服。』《注》云:『沒,猶終也。除服謂祥祭之服,卒事既祭,反喪服,服後死者之服。』又杜預云:『若父母同日卒,其葬先母后父,皆服斬衰,其虞、祔先父後母,各服其服,卒事,反服父服。若父已葬而母卒,則服母之服,虞訖,反服父之服。既除練,則服母之服。喪可除,則服父之服以除之,訖則服母之服。』賀循云:『父之喪未終,又遭母喪,當父服應終之月,皆服祥祭之服,如除喪之
 禮。卒事,反母之服。』臣等參考典故,則是隨其先後而除之,無通服五十四月之文。請依舊禮改正。」



 慶歷三年,太常禮院議:《禮記》:『父母之喪,無貴賤,一也。』又曰:『三年之喪,人道之至大也。』請不以文武品秩高下,並聽終喪。」時以武臣入流者雜,難盡解官。詔:「自今三司副使已上,非領邊寄,並聽終制,仍續月奉。武臣非在邊而願解官者聽。」



 凡奪情之制,文臣諫舍以上,牧伯刺史以上,皆卒哭後恩制起復;其在切要者,不候卒哭。內職遭喪,但給假而
 已,願終喪者亦聽。惟京朝、幕職、州縣官皆解官行服,亦有特追出者。



 凡公除與祭。景祐二年,禮儀使言:天聖五年,太常禮院言:自來宗廟祠祭,皆宰臣、參知政事行事,每有服制,旋復改差,多致妨闕。檢會《唐會要》,貞元六年詔,百官有私喪公除者,聽赴宗廟之祭。監祭御史以《禮》有「緦麻已上喪不得饗廟」,移牒吏部詰之。吏部奏:準《禮》,「諸侯絕周、大夫絕緦」者,所以殺旁親,不敢廢大宗之祭事,則緦不祭者,謂同宮未葬,欲人吉兇不相黷也。魏、晉
 已降,變而從權,緦已上喪服,假滿即吉,謂之公除。凡既葬公除,則無事不可,故於祭無妨。乞今凡有慘服既葬公除,及聞哀假滿,許吉服赴祭。同宮未葬,雖公除依前禁之。詔從。又王涇《郊祀錄》:「緦麻已上喪,不行宗廟之祭者,以明吉兇不相干也。貞元,吏部奏請,得許權改吉服,以從宗廟之祭,此一時之事,非舊典也。」今本院看詳,律稱:「如有緦麻已上喪遣充掌事者,笞五十。」此唐初所定。吏部起請,皆援引典故。奉詔,百官有私喪公除者,聽赴
 宗廟之祭。後雖王涇著《郊祀錄》稱是一時之事,非舊典也。又別無詔敕改更,是以歷代止依貞元詔命施行。至大中祥符中,詳定官請依《郊祀錄》,緦麻以上喪,不預宗廟之祭。今詳貞元起請,證據分明,王涇所說,別無典故。望自今後有私喪公除者,聽赴宗廟之祭,免致廢闕。



 慶歷七年,禮官邵必言:「古之臣子,未有居父母喪而輒與國家大祭者。今但不許入宗廟,至於南郊壇、景靈宮,皆許行事。按唐吏部所請慘服既葬公除者,謂周以下也,前
 後相承,誤以為三年之喪,得吉服從祭,失之甚也。又據律文:『諸廟享,有緦麻以上喪,不許執事,祭天地、社稷不禁。』此唐之定律者不詳經典意也。《王制》曰:『喪三年不祭,惟天地、社稷為越紼而行事。』《注》云:『不敢以卑廢尊」也。是指王者不敢以私親之喪,廢天地、社稷之祭,非謂臣下有父母喪,而得從天子祭天地、社稷也。兼律文所以不禁者,亦止謂緦麻以上、周以下故也。南郊、太廟,俱為吉祀,奉承之意,無容異禮。今居父母喪不得入太廟,至南
 郊則為愈重。朝廷每因大禮,侍祠之官普有沾賚,使居喪之人得預祠事,是不欲慶澤之行,有所不被,奈何以小惠而傷大禮?近歲兩制以上,並許終喪,惟於武臣尚仍舊制,是亦取古之墨縗從事,金革無避之義也。然於郊祀吉禮則為不可。」下禮院,議曰:「郊祀大禮,國之重事,百司聯職,僅取齊集。若居喪被起之官悉不與事,則或有妨闕。但不以慘粗之容接於祭次,則亦可行。請依《太常新禮》,宗室及文武官有遭喪被起及卒哭赴朝參者,
 遇大朝會,聽不入;若緣郊廟大禮,惟不入宗廟,其郊壇、景靈宮得權從吉服陪位,或差攝行事。」詔可。



 天聖五年,侍講學士孫奭言:「伏見禮院及刑法司外州執守服制,詞旨俚淺,如外祖卑於舅姨,大功加於嫂叔,顛倒謬妄,難可遽言。臣於《開寶正禮》錄出五服年月,並見行喪服制度,編祔《假寧令》,請下兩制、禮院詳定。」翰林學士承旨劉筠等言:「奭所上五服制度,皆應禮經。然其義簡奧,世俗不能盡通,今解之以就平易。若『兩相為服,無所降殺』,
 舊皆言『服』者,具載所為服之人;其言『周』者,本避唐諱,合復為『期』。又節取《假寧令》附《五服敕》後,以便有司;仍板印頒行,而喪服親疏隆殺之紀,始有定制矣。」



 子為嫁母。景祐二年,禮官宋祁言:「前祠部員外郎、集賢校理郭稹幼孤,母邊更嫁,有子。稹無伯叔兄弟,獨承郭氏之祭。今邊不幸,而稹解宮行服。按《五服制度敕》齊衰杖期降服之條曰:『父卒母嫁及出妻之子為母。』其左方注:『謂不為父後者。若為父後者,則為嫁母無服。』詔議之。
 侍御史劉夔曰:



 按天聖六年敕,《開元五服制度》、《開寶正禮》並載齊衰降服條例,雖與祁言不異,然《假寧令》:「諸喪,斬、齊三年,並解官;齊衰杖期及為人後者為其父母,若庶子為後為其母,亦解官,申心喪;母出及嫁,為父後者雖不服,亦申心喪。」《注》云:「皆為生己者。」《律疏》云:「心喪者,為妾子及出妻之子合降其服,二十五月內為心喪。」再詳格令:「子為嫁母,雖為父後者不服,亦當申心喪。」又稱:「居心喪者,釋服從吉及忘哀作樂、冒哀求仕者,並同父母
 正服。」今龍圖閣學士王博文、御史中丞杜衍嘗為出嫁母解官行喪。若使生為母子,沒為路人,則必虧損名教,上玷孝治。



 且杖期降服之制,本出《開元禮》文,逮乎天寶降敕,俾終三年,然則當時已悟失禮。晉袁準謂:「為人後,猶服嫁母。據外祖異族,猶廢祭行服,知父後應服嫁母。」劉智釋云:「雖為父後,猶為嫁母齊衰。」譙周云:「非父所絕,為之服周可也。」昔孔鯉之妻為子思之母,鯉卒而嫁於衛,故《檀弓》曰:「子思之母死,柳若謂子思曰:『子聖人之後
 也,四方於子乎觀禮,子盍慎諸!』子思曰:『吾何慎哉!』」喪之禮,如子。云「子聖人之後」,即父後也。石苞問淳于睿:「為父後者,不為出母服。嫁母猶出母也,或者以為嫁與出不異,不達禮意,雖執從重之義,而以廢祭見譏。君為詳正。」睿引子思之義為答,且言:「聖人之後服嫁母,明矣。」稹之行服,是不為過。



 詔兩制、御史臺、禮院再議,曰:「按《儀禮》:『父卒繼母嫁,為之服期。』謂非生己者,故父卒改嫁,降不為己母。唐上元元年敕,父在為母尚許服三年。今母嫁既
 是父終,得申本服。唐紹議曰:『為父後者為嫁母杖周,不為父後者請不降服。』至天寶六載敕,五服之紀,所宜企及,三年之數,以報免懷。其嫁母亡,宜終三年。又唐八坐議吉兇加減禮云『凡父卒,親母嫁,齊衰杖期,為父後者亦不服,不以私親廢祭祀,惟素服居堊室,心喪三年,免役解官。母亦心服之,母子無絕道也。』按《通禮五服制度》:父卒母嫁,及出妻之子為母,及為祖後,祖在為祖母,雖周除,仍心喪三年。」



 侍講學士馮元言:「《儀禮》、《禮記正義》,古
 之正禮;《開寶通禮五服年月敕》,國朝見行典制,為父後者,為出母無服。惟《通禮義纂》引唐天寶六年制:『出母、嫁母並終服三年。』又引劉智《釋議》:『雖為父後,猶為出母、嫁母齊衰,卒哭乃除。』蓋天寶之制,言諸子為出母,嫁母,故云『並終服三年』;劉智言為父後者為出母、嫁母,故云『猶為齊衰,卒哭乃除』,各有所謂,固無疑也。況《天聖五服年月敕》:『父卒母嫁及出妻之子為母降杖期。』則天寶之制已不可行。又但言母出及嫁,為父後者雖不服,亦申心
 喪,即不言解官。若專用禮經,則是全無服式;若俯同諸子杖期,又於條制相戾。請凡子為父後,無人可奉祭祀者,依《通禮義纂》、劉智《釋議》,服齊衰,卒哭乃除,逾月乃祭,仍申心喪,則與《儀禮》、《禮記正義》、《通典》、《通禮》、《五服年月敕》『為父後,為出母、嫁母無服。』之言不遠。如諸子非為父後者,為出母、嫁母,依《五服年月敕》,降服齊衰杖期,亦解官申心喪,則與《通禮五服制度》言『雖周除,仍心喪三年』,及《刑統》言『出妻之子合降其服,皆二十五月內為心喪』,其
 義一也。郭稹應得子為父後之條,緣其解官行服已過期年,難於追改,後當依此施行。」



 詔自今並聽解官,以申心喪。



 子為生母。大中祥符八年,樞密使王欽若言:「編修《冊府元龜》官太常博士、秘合校理聶震丁所生母憂,嫡母尚在,望特免持服。」禮官言:「按周制,庶子在父之室,則為其母不禫。晉解遂問蔡謨曰:『庶子喪所生,嫡母尚存,不知制服輕重。』答云:『士之妾子服其母,與凡人喪母同。』鐘陵
 胡澹所生母喪,自有嫡兄承統,而嫡母存,疑不得三年,問範宣,答曰:『為慈母且猶三年,況親所生乎?嫡母雖尊,然厭降之制,父所不及。婦人無專制之事,豈得引父為比而屈降支子也?』南齊褚淵遭庶母郭氏喪,葬畢,起為中軍將軍。後嫡母吳郡公主薨,葬畢,令攝職。則震當解官行服,心喪三年;若特有奪情之命,望不以追出為名。自今顯官有類此者,亦請不稱起復,第遣厘職。」



 熙寧三年,詔御史臺審決秀州軍事判官李定追服所生母喪。
 御史臺言:「在法,庶子為父後,如嫡母存,為所生母服緦三月,仍解官申心喪;若不為父後,為所生母持齊衰三年,正服而禫。今定所生仇氏亡日,定未嘗請解官持心喪,止以父老乞還侍養。宜依禮制追服緦麻,而解官心喪三年。」時王安石庇定,擢為太子中允,而言者俱罷免。



 婦為舅姑。乾德三年,判大理寺尹拙言:「按律及《儀禮喪服傳》、《開元禮儀纂》、《五禮精義》、《三禮圖》等書,所載婦為舅姑服周;近代時俗多為重服,劉岳《書儀》有奏請之文。《禮
 圖》、《刑統》乃邦家之典,豈可守《書儀》小說而為國章邪?」判少卿事薛允中等言:「《戶婚律》:『居父母及夫喪而嫁娶者,徒三年,各離之。若居周喪而嫁娶者,杖一百。』又《書儀》:『舅姑之服斬衰三年。』亦準敕行。用律敕有差,望加裁定。」



 右僕射魏仁浦等二十一人奏議曰:「謹按《禮·內則》云:『婦事舅姑,如事父母。』則舅姑與父母一也。而古禮有期年之說,至於後唐始定三年之喪,在理為當。況五服制度,前代增益甚多。按《唐會要》,嫂叔無服,太宗令服小功。曾祖
 父母舊服三月,增為五月。嫡子婦大功,增為期。眾子婦小功,增為大功。父在為母服期,高宗增為三年。婦為夫之姨舅無服,玄宗令從夫服,又增姨舅同服緦麻及堂姨舅袒免。至今遵行。況三年之內,幾筵尚存,豈可夫處苫塊之中,婦被綺紈之飾?夫婦齊體,哀樂不同,求之人情,實傷理本。況婦為夫有三年之服,於舅姑止服期年,乃是尊夫而卑舅姑也。況孝明皇后為昭憲太后服喪三年,足以為萬世法。欲望自今婦為舅姑服,並如後唐
 之制,其三年齊、斬,一從其夫。」



 嫡孫承重。天聖四年,大理評事杜杞言:「祖母穎川郡君鐘歿,並無服重子婦,餘孤孫七人,臣最居長,今己服斬衰,即未審解官以否?」禮院言:「按《禮·喪服小記》曰:『祖父卒,而後,為祖母後者三年。』《正義》曰:『此論適孫承重之服。祖父卒者,適謂孫無父而為祖後。祖父已卒,今遭祖母喪,故云為祖母後也。若父卒為母,故三年。若祖父卒時,父已先亡,亦為祖父三年。若祖卒時父在,己雖為祖期,今父
 歿,祖母亡時,己亦為祖母三年也。』又按令文:『為祖後者,祖卒為祖母,祖父歿,嫡孫為祖母承重者,齊衰三年,並解官。』合依《禮》、令。」



 寶元二年,度支判官、集賢校理薛紳言:「祖母萬壽縣太君王氏卒,是先臣所生母,服紀之制,罔知所適,乞降條制,庶知遵守。」詔送太常禮院詳定。禮官言:「《五服年月敕》:『齊衰三年,為祖後者,祖卒則為祖母。』又曰:『齊衰不杖期,為祖父母。』《注》云:『父之所生庶母亦同,惟為祖後者不服。』又按《通禮義纂》:『為祖後者,父所生庶母亡,
 合三年否?』《記》云:『為祖母也,為後三年。不言嫡庶。然奉宗廟,當以貴賤為差,庶祖母不祔於皇姑,已受重於祖,當為祭主,不得申於私恩;若受重於父代而養,為後可也。』又曰:『庶祖母合從何服?禮無服庶祖母之文,有為祖庶母後者之服。晉王暠議曰:受命為後,則服之無嫌。婦人無子,托後族人,猶為之服,況其子孫乎?人莫敢卑其祖也。且妾子,父歿為母得申三年。孫無由獨屈,當服之也。』看詳《五服年月敕》,不載持重之文,於《義纂》即有所據。今
 薛紳不為祖後,受重於父,合申三年之制。」



 史館檢討、同知太常禮院王洙言:「《五服年月敕》與新定令文及《通禮》正文內五服制度,皆聖朝典法,此三處並無為父所生庶母服三年之文。唯《義纂》者是唐世蕭嵩、王仲丘等撰集,非創修之書,未可據以決事。且所引兩條,皆近世諸儒之說,不出於《六經》,臣已別狀奏駁。今薛紳為映之孫,耀卿為別子始祖,紳繼別之後為大宗,所守至重,非如次庶了等承傳其重者也。不可輒服父所生庶母三年
 之喪,以廢始祖之祭也。臣謹按《禮經》所謂重者,皆承後之文。據《義纂》稱重於父,亦有二說:一者,嫡長子自為正體,受重可知;二者,或嫡長亡,取嫡或庶次承傳父重,亦名為受重也。若繼別子之後,自為大宗,所承至重,不得更遠系庶祖母為之服三年,惟其父以生己之故,為之三年可也。詳《義纂》所謂『受重於父者』,指嫡長子亡、次子承傳父重者也,但其文不同耳。」



 詔太常禮院與御史臺詳定聞奏。眾官參詳:「耀卿,王氏子;紳,王氏孫,尤親於慈
 母、庶母,祖母、庶祖母也,耀卿既亡,紳受重代養,當服之也。又薛紳頃因籍田覃恩,乞將敘封母氏恩澤,回授與故父所生母王氏,其薛紳官爵未合敘封祖母,蓋朝廷以耀卿已亡,紳是長孫,敦以教道,特許封邑,豈可王氏生則輒邀國恩,歿則不受重服?況紳被王氏鞠育之恩,體尊義重,合令解官持齊衰三年之服。」詔從之。



 皇祐元年,大理評事石祖仁奏:「叔從簡為祖父中立服後四十日亡,乞下禮院定承祖父重服。」禮官宋敏求議曰:「自《開
 元禮》以前,嫡孫卒則次孫承重,況從簡為中子已卒,而祖仁為嫡孫乎?古者重嫡,正貴所傳,其為後者皆服三年,以主虞、練、祥、禫之祭。且三年之喪,必以日月之久而服之者有變也。今中立未及卒哭,從簡已卒,是日月未久而服未經變也。或謂已服期,不當改服斬,而更為重制。按《儀禮》:『子嫁,反在父之室,為父三年。』鄭氏注:『謂遭喪而出者,始服齊衰期,出而虞則以三年之喪。』是服可再制明矣。今祖仁宜解官,因其葬而制斬衰三年。後有如
 其類而已葬者,用再喪制服。」遂著為定式。



 熙寧八年,禮院請為祖承重者依《封爵令》立嫡孫,以次立嫡子同母弟,無母弟立庶子,無庶子立嫡孫同母弟;如又無之,即立庶長孫,行斬衰服。於是禮房詳定:「古者封建國邑而立宗子,故周禮適子死,雖有諸子,猶令嫡孫傳重,所以一本統、明尊尊之義也。至於商禮,則嫡子死立眾子,然後立孫。今既不立宗子,又未嘗封建國邑,則嫡孫喪祖,不宜純用周禮。若嫡子死無眾子,然後嫡孫承重,即嫡
 孫傳襲封爵者,雖有眾子猶承重。」時知廬州孫覺以嫡孫解官持祖母服,覺叔父在,有司以新令,乃改知潤州。



 元豐三年,太常丞劉次莊祖母亡,有嫡曾孫,次莊為嫡孫同母弟,在法未有庶孫承重之文。詔下禮官立法:「自今承重者,嫡子死無諸子,即嫡孫承重;無嫡孫,嫡孫同母弟承重;無母弟,庶孫長者承重;曾孫以下準此。其傳襲封爵,自依禮、令。」



 雜議。大中祥符八年,廣平公德彞聘王顯孫女,將大歸
 而德彞卒,疑其禮制。禮官言:「按《禮》:『曾子問曰:娶女有吉日而女死,如之何?孔子曰:婿齊衰而吊,既葬而除之。夫死亦如之。』《注》云:『謂無期三年之恩也,女服斬衰。』又《刑統》云:『依禮,有三月廟見、有未廟見就婚等三種之文,妻並同夫法,其有克吉日及定婚夫等,惟不得違約改嫁,自餘相犯,並同凡人。』今詳女合服斬衰於室,既葬而除;或未葬,但出攢即除之。」



 天聖七年,興化軍進士陳可言:「臣昨與本軍進士黃價同保,臣預解送之後,本軍言黃價
 昨赴舉時,有叔為僧,喪服未滿,臣例當駁放。竊思出家制服,禮律俱無明文,況僧犯大罪,並無緣坐;犯事還俗,準敕不得均分父母田園。又釋門儀式,見父母不拜,居父母喪不絰,死則法門弟子為之制服,其於本族並無服式。望下禮官詳議,許其赴試。」太常禮院言:「檢會敕文,期周尊長服,不得取應。又禮為叔父齊衰期,外繼者降服大功九月。其黃價為叔僧,合比外繼,降服大功。」



 皇祐四年,吉州司理參軍祝紳幼孤,鞠於兄嫂。已嘗為嫂持
 服,兄喪,又請解官持喪。有司以為言。仁宗曰:「近世蓋有匿親喪而干進者。紳雖所服非禮,然不忘鞠養恩,亦可勸也。候服闋日與幕職、知縣。」



 繼絕。熙寧二年,同修起居注、直史館蔡延慶父褒,故太尉齊之弟也。齊初無子,子延慶。後齊有子,而褒絕,請復本宗。禮官以請,許之。紹聖元年,尚書省言:「元祐南郊赦文,戶絕之家,近親不為立繼者,官為施行。今戶絕家許近親尊長命繼,已有著令,即不當官為施行。」四年,右武
 衛大將軍克務,乞故登州防禦使東牟侯克端子叔博為嗣,請赴期朝參起居,而不為克端服。大宗正司以聞。下禮官議,宜終喪三年。遂詔宗室居父母喪者,毋得乞為繼嗣。



 大觀四年,詔曰:「孔子謂興滅繼絕,天下之民歸心。王安石子雱無嗣,有族子棣,已嘗用安石孫恩例官,可以棣為雱後,以稱朕善善之意。」先是,元豐國子博士孟開,請以侄孫宗顏為孫,據晉侍中荀顗無子,以兄之孫為孫;其後王彥林請以弟彥通為叔母宋繼絕孫,詔
 皆如所請。淳熙四年十月二十七日,戶部言:「知蜀州吳擴申明:乞自今養同宗昭穆相當之子,夫死之後,不許其妻非理遣還。若所養子破蕩家產,不能侍養,實有顯過,即聽所養母訴官,近親尊長證驗得實,依條遣還,仍公共繼嗣。」



\end{pinyinscope}