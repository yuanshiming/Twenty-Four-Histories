\article{列傳第二十二}

\begin{pinyinscope}

 張
 昭竇儀弟儼偁呂餘慶劉熙古子蒙正蒙叟石熙載子中立李穆弟肅



 張昭,字潛夫,本名昭遠,避漢祖諱,止稱昭。自言漢常山王耳之後,世居濮州範縣。祖楚平,壽張令。楚平生直,即
 昭父也。初,楚平赴調長安,值巢寇亂,不知所終。直幼避地河朔,既寇,以父失所在,時盜賊蜂起,道路榛梗,乃自秦抵蜀,徒行丐食,求父所在,積十年不能得。乃發哀行服,躬耕海濱。青州王師範開學館,延置儒士,再以書幣招直,署賓職。師範降梁,直脫難北歸,以《周易》、《春秋》教授,學者自遠而至,時號逍遙先生。



 昭始十歲,能誦古樂府、詠史詩百餘篇;未冠,遍讀《九經》,盡通其義。處儕類中,緩步闊視,以為馬、鄭不己若也。後至贊皇,遇程生者,專史
 學,以為專究經旨,不通今古,率多拘滯,繁而寡要;若極談王霸,經緯治亂,非史不可。因出班、範《漢書》十餘義商榷,乃授昭《荀紀》、《國志》等,後又盡得十三史,五七年間,能馳騁上下數千百年事。又注《十代興亡論》。處亂世,躬耕負米以養親。



 後唐莊宗入魏,河朔游士,多自效軍門,昭因至魏,攜文數十軸謁興唐尹張憲。憲家富文籍,每與昭燕語,講論經史要事,恨相見之晚,即署府推官。同光初,奏授真秩,加監察御史裏行。憲為北京留守,昭亦從
 至晉陽。莊宗及難,聞鄴中兵士推戴明宗,憲部將符彥超合戍將應之。昭謂憲曰:「得無奉表勸進為自安之計乎?」憲曰:「我本書生,見知主上,位至保厘,乃布衣之極。茍靦顏求生,何面目見主於地下?」昭曰:「此古人之志也,公能行之,死且不朽矣。」相泣而去,憲遂死之,時論重昭能成憲之節。



 時有害昭者,昭曰:「明誠所至,期不再生,主辱臣亡,死而無悔。」眾執以送彥超,彥超曰:「推官正人,無得害之。」又逼昭為榜安撫軍民。事寧,以昭為北京留守
 推官,加殿中侍御史、內供奉官,賜緋。天成三年,改安義軍節度掌書記。



 時以武皇、莊宗實錄未修,詔正國軍節度盧質、西川節度副使何瓚、秘書監韓彥輝纘錄事跡。瓚上言:「昭有史材,嘗私撰《同光實錄》十二卷,又聞其欲撰《三祖志》,並藏昭宗朝賜武皇制詔九十餘篇,請以昭所撰送史館。」拜昭為左補闕、史館修撰,委之撰錄。昭以懿祖、獻祖、太祖並不踐帝位,仍補為《紀年錄》二十卷,又撰《莊宗實錄》三十卷上之。優詔褒美,遷都官員外郎。



 時
 皇子競尚奢侈,昭疏諫曰:



 帝王之子,長於深宮,安於逸樂,紛華之玩,絲竹之音,日接於耳目,不與驕期而驕自至。儻非天資英敏,識本清明,以此蕩心,焉能無惑。茍不豫為教道,何以置之盤牙?臣見先帝時,皇子、皇弟盡喜無稽玩物之言,厭聞致治經邦之論,入則務飾姬姜,出則廣增僕馬;親賓滿坐,食客盈門,箴規者少,諧謔者多。以此而欲托以主治,不亦難乎?臣請諸皇子各置師傅,陛下令皇子屈身師事之,講論道德。使一日之中,止記
 一事,一歲之內,所記漸多。每月終,令師傅具錄聞奏。或皇子上謁之時,陛下更令侍臣面問,十中得五,為益良多,博識安危之理,深知成敗之由。



 臣又聞古之人君,即位而封太子、拜諸王,究其所由,蓋有深旨。使庶不亂嫡,疏不間親,禮秩有常,邪慝不作。近代人君,失於此道,以至邦家構患,釁隙萌生。昔隋祖聰明,煬帝亦傾楊勇;太宗齊聖,魏王終覆承乾。臣每讀古書,深悲其事。願於聖代,杜此厲階。其於卜貳封宗,在臣未敢輕議。臣請諸皇
 子於恩澤賜與之間,婚姻省侍之際,依嫡庶而為禮秩,據親疏而定節文,示以等威,絕其徼幸,保宗之道,莫大於斯。



 明宗覽疏而不用。



 四年,上《武王以來功臣列傳》三十卷,以本官知制誥。明宗好畋獵,昭疏諫曰:



 太祖初鎮太原,每年打鹿於北鄙;先帝在位,暇日射雁於近郊。此蓋軍務之餘,畋游自適。自先帝因圖啟祚,向明御宇,則宜易彼諸侯之事,肅乎萬乘之儀。而猶因習舊風,失其威重,驅逐原獸,殆無虛日。



 臣愚以為事有可畏者四焉。
 洛都舊制,宮城與禁苑相連,人君宴游,不離苑囿,御馬來往,輦路坦夷,不涉荒郊,何憂蹶失。今則驅馳驂服,涉歷榛蕪,此後節氣嚴凝,徑途凍滑,萬一有銜橛之變,陛下縱自輕,奈宗廟社稷何?所可畏者一也。又陛下新有四海,宜以德服萬邦。今則江、嶺未平,淮夷尚梗,彼初聞陛下革先朝之失政,還太古之淳風,御物以慈,節財以儉,有典有則,不矜不驕,彼必有三苗率服之心,七旬來格之意。如聞陛下暫游近甸,彼即以為復好畋游。所可
 畏者二也。臣又聞「作法於涼,其弊猶貪,作法於貪,弊將如何?」且打鹿射雁之事新,敗軌傾輈之轍在,常宜取鑒,不可因循。所可畏者三也。臣又聞「作事可法,貽厥孫謀。」若陛下以齊聖廣淵之機,聰明神武之量,其可以宴游搜狩之事,少累聖明,所謂「城中好廣眉,城外加半額」,為法之弊,靡不由茲。所可畏者四也。



 伏望陛下居高慮遠,慎始圖終,思創業之艱難,知守成之不易,念老氏馳騁之戒,樹文王忠厚之基,約三驅之舊章,定四時之游幸。
 始出有節,後不敢違。



 疏奏,明宗嘉納之。



 長興二年,丁內艱,賻絹布五十匹,米麥五十石。昭性至孝,明宗聞其居喪哀毀,復賜以錢幣。服除,改職方員外郎、知制誥,充史館修撰。上言乞復本朝故事,置觀察使察民疾苦,御史彈事,諫官月給諫紙。並從之。又奏請勸農耕及置常平倉等數事。



 明宗方務聽納,昭復上疏曰:「臣聞『安不忘危,治不忘亂』者,先儒之丕訓;『靡不有初,鮮克有終』者,前經之至戒。究觀列闢,莫不以驕矜怠惰,有虧盛德。恭惟太
 宗貞觀之初,玄宗開元之際,焦勞庶政,以致太平。及國富兵消,年高志逸,乃忽守約之道,或貽執簡之譏。陛下以慈儉化天下,以禮法檢臣鄰,絀奸邪之黨,延正直之論,務遵純儉,以節浮費,信賞必罰,至公無私。其創業垂統之基,如貞觀、開元之始,然陛下有始有終,無荒無怠。臣又伏念保邦之道,有八審焉,願為陛下陳之:夫委任審於材器,聽受審於忠邪,出令審於煩苛,興師審於德力,賞罰審於喜怒,毀譽審於愛憎,議論審於賢愚,嬖寵
 審於奸佞。推是八審,以決萬機,庶可以臻至治。」明宗覽之稱善。



 清泰初,改駕部郎中、知制誥,撰皇后冊文,遷中書舍人,賜金紫。二年,加判史館兼點閱三館書籍,校正添補。預修《明宗實錄》,成三十卷以獻。三年,遷禮部侍郎,改御史中丞。



 晉天福初,從幸汴州。昭請創宮闕名額及振舉朝綱、條疏百司廨舍。二年,改戶部侍郎,宰相桑維翰薦為翰林學士。內署故事,以先後入為次,不系官序。特詔昭立位次承旨崔梲。晉祖嘗幸內署,與昭語及並、
 魏舊事,甚重之,錫賚頗厚。直以昭故,授著作佐郎致仕,至是卒。歸西洛,賻賜加等。五年,服闋,召為戶部侍郎。以唐史未成,詔與呂琦、崔梲等續成之,別置史院,命昭兼判院事。昭又撰《唐朝君臣正論》二十五卷上之。改兵部侍郎。八年,遷吏部,判東銓,兼史館修撰、判館事。開運二年秋,《唐書》成二百卷,加金紫階,進爵邑。三年,拜尚書右丞,判流內銓,權知貢舉。



 漢初,復為吏部侍郎。時追尊六廟,定謚號、樂章、舞曲,命昭權判太常卿事,月餘即真。乾
 祐二年,加檢校禮部尚書。少帝年十九,猶有童心,暱比群小。昭上言請聽政之暇,數召儒臣講論經義。



 周廣順初,拜戶部尚書。子秉陽,為陽翟主簿,抵罪,昭自以失教,奉表引咎,左遷太子賓客。歲餘,復舊官。嘗奏請興制舉,設賢良方正能直言極諫、經學優深可為師法、詳閑吏治達於教化三科,職官、士流、黃衣、草澤並許應詔。諸州依貢舉體式,量試策論三道,共以三千字以上為準,考其文理俱優,解送尚書吏部,其登朝之官亦聽自舉。從
 之。



 顯德元年,遷兵部尚書。世宗以昭舊德,甚重焉。二年,表求致仕,優詔不允,促其入謁。嘗詔撰《制旨兵法》十卷,又撰《周祖實錄》三十卷,及梁郢王均帝、後唐閔帝廢帝、漢隱帝五朝實錄;梁二主年祀浸遠,事皆遺失,遂不克修,餘三帝實錄,皆藏史閣。



 世宗好拔奇俊,有自布衣及下位上書言事者,多不次進用。昭疏諫曰:「昔唐初,劉洎、馬周起於徒步,太宗擢用為相;其後,柳璨、朱樸方居下僚,昭宗亦加大用。此四士者,受知於明主;然太宗用之
 而國興,昭宗用之而國亡,士之難知如此。臣願陛下存舊法而用人,當以此四士為鑒戒。」世宗善之。詔令詳定《經典釋文》、《九經文字》、《制科條式》,及問六璽所出,並議《三禮圖》祭玉及鼎釜等。昭援引經據,時稱其該博。恭帝即位,封舒國公。



 宋初,拜吏部尚書。乾德元年郊祀,昭為鹵簿使,奏復宮闕、廟門、郊壇夜警晨嚴之制。禮畢,進封鄭國公,與翰林承旨陶穀同掌選。穀嘗誣奏事,引昭為證,昭免冠抗論。太祖不說,遂三拜章告老,以本官致仕,改
 封陳國公。開寶五年,卒,年七十九。



 昭博通學術,書無不覽,兼善天文、風角、太一、卜相、兵法、釋老之說,藏書數萬卷。尤好纂述,自唐、晉至宋,專筆削典章之任。嶺南平,擒劉鋹,將獻俘,莫能知其禮。時昭已致政,太祖遣近臣就其家問之,昭方臥病,口占以授使者。著《嘉善集》五十卷、《名臣事跡》五卷。



 子秉圖進士及第,秉謙至尚書郎。



 竇儀,字可像。薊州漁陽人。曾祖遜,玉田令。祖思恭,媯州司馬。父禹鈞,與兄禹錫皆以詞學名。禹鈞,唐天祐末起
 家幽州掾,歷沂、鄧、安、同、鄭、華、宋、澶州支使判官。周初,為戶部郎中,賜金紫。顯德中,遷太常少卿、右諫議大夫致仕。



 儀十五能屬文,晉天福中舉進士。侍衛軍帥景延廣領夔州節度,表為記室。延廣後歷滑、陜、孟、鄆四鎮,儀並為從事。



 開運中,楊光遠以青州叛,時契丹南侵,博州刺史周儒以城降,光遠與儒遣人引契丹輕騎於馬家渡渡河。時延廣掌衛兵,顏衎知州事,即遣儀入奏。儀謂執政曰:「昨與衎論事勢,有所預慮,所以乘驛晝夜不息而
 來。國家若不以良將重兵控博州渡,必恐儒引契丹逾東岸與光遠兵合,則河南危矣。」俄而儒果導契丹渡河,增置壘柵。少帝軍河上,即遣李守貞等率兵萬人,水陸並進,守汶陽,據要害。契丹果大至,擊走之。漢初,召為左補闕、禮部員外郎。



 周廣順初,改倉部員外郎、知制誥。未幾,召為翰林學士。周祖幸南御莊宴射,坐中賜金紫。歷駕部郎中、給事中,並充職。



 劉溫叟知貢舉,所取士有覆落者,加儀禮部侍郎,權知貢舉。儀上言:「請依晉天福五
 年制,廢明經、童子科。進士省卷,令納五軸以上,不得有神道碑志之類;帖經對義,有三通為合格;卻復盡試。其落第者,分為五等:以詞理紕繆之甚者為第五等,殿五舉;其次為第四等,殿三舉;以次稍可者為第三、第二、第一等,並許次年赴舉。其學究,請並《周易》、《尚書》為一科,各對墨義三十道;《毛詩》依舊為一科,亦對墨義六十道。及第後,並減為七選集。諸科舉人,第一場十否,殿五舉;第二、第三場十否,殿三舉;二場內有九否,殿一舉。解試之官坐
 其罪。進士請解,加試論一首,以五百言以上為準。」奏可。



 俄以父病,上表解官。世宗親加慰撫,手封金丹,俾賜其父。父卒,歸葬洛陽。詔賜錢三十萬,米麥三百斛。終喪,召拜端明殿學士。從征淮南,判行在三司,世宗以其餉饋不繼,將罪之,宰相範質救解得免。淮南平,判河南府兼知西京留守事。恭帝即位,遷兵部侍郎,充職。俄使南唐,既至,將宣詔,會雨雪,李景請於廡下拜受,儀曰:「儀獲將國命,不敢失舊禮。儻以沾服失容,請俟他日。」景即拜命
 於庭。



 建隆元年秋,遷工部尚書,罷學士,兼判大理寺。奉詔復位《刑統》,為三十卷。會翰林學士王著以酒失貶官,太祖謂宰相曰:「深嚴之地,當得宿儒處之。」範質等對曰:」竇儀清介重厚,然已自翰林遷端明矣。」太祖曰:「非斯人不可處禁中,卿當諭以朕意,勉令就職。」即日再入翰林為學士。



 乾德二年,範質等三相並罷。越三日,始命趙普平章事。制書既下,太祖問翰林學士曰:「質等已罷,普敕何官當署?」承旨陶穀時任尚書,乃建議相位不可以久
 虛,今尚書乃南省六官之長,可以署敕。儀曰:「穀所陳非承平之制,皇弟開封尹、同平章事,即宰相之任。」太祖曰:「儀言是也。」即命太宗署敕賜之。俄加禮部尚書。



 時御史臺議,欲以左右僕射合為表首,太常禮院以東宮三師為表首。儀援典故,以僕射合為表首者六,而謂三師無所據。朝議是之。四年秋,知貢舉。是冬卒,年五十三,贈右僕射。



 儀學問優博,風度峻整。弟儼、侃、偁、僖,皆相繼登科。馮道與禹鈞有舊,嘗贈詩,有「靈椿一株老,丹桂五枝芳」
 之句,縉紳多諷誦之,當時號為竇氏五龍。



 初,周祖平兗州,議將盡誅脅從者。儀白馮道、範質,同請於周祖,皆得全活。顯德中,太祖克滁州,世宗遣儀籍其府庫。太祖復令親吏取藏中絹給麾下,儀曰:「太尉初下城,雖傾藏以給軍士,誰敢言者。今既著籍,乃公帑物也,非詔不可取。」後太祖屢對大臣稱儀有執守,欲相之。趙普忌儀剛直,乃引薛居正參知政事。及儀卒,太祖憫然謂左右曰:「天何奪我竇儀之速耶!」蓋惜其未大用也。



 侃,漢乾祐初及
 第,至起居郎。僖,周廣順初及第,至左補闕。



 子絺、鷿、誥,俱登進士第,絺至都官員外郎,鷿至秘書丞。



 儼字望之,幼能屬文。既冠,舉晉天福六年進士,闢滑州從事。府罷,授著作佐郎、集賢校理,出為天平軍掌書記,以母憂去職。服除,拜左拾遺。開運中,諸鎮恣用酷刑,儼上疏曰:「案名例律,死刑二,絞、斬之謂也。絞者筋骨相連,斬者頭頸異處,大闢之目,不出兩端。淫刑之興,近聞數等,蓋緣外地不守通規,或以長釘貫人手足,或以短刀
 臠人肌膚,遷延信宿,不令就死。冤聲上達,和氣有傷,望加禁止。」上從之。



 儼仕漢為史館修撰。周廣順初,遷右補闕,與賈緯、王伸同修晉高祖少帝、漢祖三朝實錄。改主客員外郎、知制誥。時儀自閣下入翰林,兄弟同日拜命,分居兩制,時人榮之。俄加金部郎中,拜中書舍人。



 顯德元年,加集賢殿學士,判院事。父憂去職,服闋,復舊官。時世宗方切於治道,儼上疏曰:「歷代致理,六綱為首:一曰明禮,禮不明則彞倫不敘。二曰崇樂,樂不崇則二儀不
 和。三曰熙政,政不熙則群務不整。四曰正刑,刑不正則巨奸不懾。五曰勸農,農不勸則資澤不流。六曰經武,武不經則軍功不盛。故禮有紀,若人之衣冠;樂有章,若人之喉舌;政有統,若人之情性;刑有制,若人之呼吸;農為本,若人之飲食;武為用,若人之手足。斯六者,不可斯須而去身也。陛下思服帝猷,寤寐獻納,亟下方正之詔,廓開藝能之路。士有一技,必得自效。故小臣不揆,輒陳禮、樂、刑、政、勸農、經武之言。」世宗多見聽納。



 南征還。詔儼考
 正雅樂,俄權知貢舉。未幾,拜翰林學士,判太常寺。儼校鐘磬筦龠之數,辨清濁上下之節,復舉律呂旋相為宮之法,迄今遵用。



 會詔中外臣僚,有所聞見,並許上章論議。儼疏曰:「設官分職,授政任功,欲為政之有倫,在位官之無曠。今朝廷多士,省寺華資,無事有員,十乃六七,止於計月待奉,計年待遷。其中廉幹之人,不無愧恥之意。如非歷試,何展公才。請改兩畿諸縣令及外州府五千戶以上縣令為縣大夫,升為從五品下。畿大夫見府尹
 如赤令之儀,其諸州府縣大夫見本部長如賓從之禮。郎中、員外郎、起居、補闕、拾遺、侍御史、殿中侍御史、監察御史、光祿少卿以下四品,太常丞以下五品等,並得衣朱紫。滿日,準在朝一任,約舊官遷二等。自拾遺、監察除授回日,即為起居、侍御史、中行員外郎。若前官不是三署,即罷後一年方得求仕。如此,則士大夫足以陳力,賢不肖無以駕肩,各系否臧,明行黜陟,利民益國,斯實良規。」又以為:「家國之方,守穀帛而已,二者不出國而出於
 民。其道在天,其利在地,得其理者蕃阜,失其理者耗嗇。民之顓蒙,宜有勸教。請於《齊民要術》及《四時纂要》、《韋氏月錄》中,採其關於田蠶園囿之事,集為一卷,鏤板頒行,使之流布。」疏奏不報。



 宋初,就轉禮部侍郎,代儀知貢舉。當是時,祀事樂章、宗廟謚號多儼撰定,議者服其該博。車駕征澤、潞,以疾不從。卒,年四十二。



 儼性夷曠,好賢樂善,優游策府凡十餘年。所撰《周正樂》成一百二十卷,詔藏於史閣;其《通禮》未及編纂而卒。有文集七十卷。儼與
 儀尤為才俊,對景覽古,皆形諷詠,更迭倡和至三百篇,多以道義相敦勵,並著集。



 儼顯德中奉使荊南。荊南自唐季以來,高氏據有其地,雖名藩臣,車服多僭侈逾制,以至司賓賤隸、候館小胥,皆盛服彯纓,與王人亢禮。儼諷以天子在上,諸侯當各守法度,悉令去之,然後宣達君命。



 尤善推步星歷,逆知吉兇。盧多遜、楊徽之同任諫官,儼嘗謂之曰:「丁卯歲五星聚奎,自此天下太平,二拾遺見之,儼不與也。」又曰:「儼家昆弟五人,皆登進士第,可
 謂盛矣,然無及相輔者,唯偁稍近之,亦不久居其位。」卒如其言。儼有子早卒,以侄說為嗣。



 偁字日章,漢乾祐二年舉進士。周廣順初,補單州軍事判官,遷秘書郎,出為絳州防禦判官。宋初,歷武寧軍掌書記西京留守判官、天雄歸德軍節度判官。開寶六年,拜右補闕、知宋州。嘗作《遂命賦》以自悼。太宗領開封尹,選偁判官。時賈琰為推官,偁不樂其為人。太宗嘗宴諸王,偁、琰與會,琰言矯誕,偁叱之曰:「巧言令色,心不獨
 愧乎。」上愕然,因罷會,出偁為彰義軍節度判官。



 太平興國五年,車駕幸大名府,召至行在所,拜比部郎中。時議北征,偁請休兵牧馬,以徐圖之,上從其言。歸,以偁為樞密直學士,賜第一區。六年,遷左諫議大夫,充職。



 七年,參知政事。上謂偁曰:「汝何能臻此?」偁曰:「陛下不忘舊臣。」太宗曰:「非也,卿能以公正責賈琰,朕旌直臣爾。」是秋卒,年五十八。車駕臨哭,贈工部尚書。



 初,偁在涇州,與丁顥同官,顥子謂方幼,偁見之曰:「此兒必遠到。」以女妻之。後為
 宰相、三公。太祖嘗謂宰相曰:「近朝卿士,竇儀質重嚴整,有家法,閨門敦睦,人無讕語,諸弟不能及。僖亦中人材爾,偁有操尚,可嘉也。



 呂餘慶,幽州安次人,本名胤,犯太祖偏諱,因以字行。祖兗,橫海軍節度判官。父琦,晉兵部侍郎。餘慶以蔭補千牛備身,歷開封府參軍,遷戶曹掾。晉少帝弟重睿領忠武軍節度,以餘慶為推官。仕漢歷周,遷濮州錄事參軍。太祖領同州節制,聞餘慶有材,奏為從事。世宗問曰:「得
 非嘗為濮州糾曹者乎?」即以為定國軍掌書記。世宗嘗鎮澶淵,濮為屬郡,故知其為人也。



 太祖歷滑、許、宋三鎮,餘慶並為賓佐。及即位,自宋、亳觀察判官召拜給事中,充端明殿學士。清泰中,琦亦居是職,官秩皆同,時人榮之。未幾,知開封府。太祖征潞及揚,並領上都副留守。建隆三年,遷戶部侍郎。丁母憂。荊湖平,出知潭州,改襄州,遷兵部侍郎、知江陵府。召還,以本官參知政事。



 蜀平,命知成都府。時盜賊四起,軍士恃功驕恣,大將王全斌等
 不能戢下。一日,藥市始集,街吏馳報有軍校被酒持刃奪賈人物。餘慶立捕斬之以徇,軍中畏伏,民用按堵。就加吏部侍郎。歸朝,兼劍南、荊南等道都提舉、三司水陸發運等使。開寶六年,與宰相更知政事印,旋以疾上表求解機務,拜尚書左丞。九年,卒,年五十。贈鎮南軍節度。



 餘慶重厚簡易,自太祖繼領藩鎮,餘慶為元僚。及受禪,趙普、李處耘皆先進用,餘慶恬不為意。未幾,處耘黜守淄州,餘慶自江陵還,太祖委曲問處耘事,餘慶以理辨釋,
 上以為實,遂命參知政事。會趙普忤旨,左右爭傾普,餘慶獨辨明之,太祖意稍解,時稱其長者。至道中,以弟端為宰相,特詔贈侍中。



 劉熙古,字義淳,宋州寧陵人,唐左僕射仁軌十一世孫。祖實進,嘗為汝陰令。熙古年十五,通《易》、《詩》、《書》;十九,通《春秋》、子、史。避祖諱,不舉進士。後唐長興中,以《三傳》舉。時翰林學士和凝掌貢舉,熙古獻《春秋極論》二篇、《演論》三篇,凝甚加賞,召與進士試,擢第,遂館於門下。



 清泰中,驍將
 孫鐸以戰功授金州防禦使,表熙古為從事。晉天福初,鐸移汝州,又闢以隨。熙古善騎射,一日,有抃集戟門槐樹,高百尺,鐸惡之,投以瓦石不去,熙古引弓一發,貫抃於樹。鐸喜,令勿拔矢,以旌其能。後二歲,鐸卒,調補下邑令。俄為三司戶部出使巡官,領永興、渭橋、華州諸倉制置發運。仕漢,為盧氏令。周廣順中,改亳州防禦推官,歷澶州支使。秦、鳳平,以為秦州觀察判官。



 太祖領宋州,為節度判官。即位,召為左諫議大夫,知青州。車駕征惟
 揚,追赴行在。建隆二年,受詔制置晉州榷礬,增課八十餘萬緡。乾德初,遷刑部侍郎、知鳳翔府。未幾,移秦州。州境所接多寇患,熙古至,諭以朝廷恩信,取蕃部酋豪子弟為質,邊鄙以寧。轉兵部侍郎,徙知成都府。六年,就拜端明殿學士。丁母憂。開寶五年,詔以本官參知政事,選名馬、銀鞍以賜。歲餘,以足疾求解,拜戶部尚書致仕。九年,卒,年七十四。贈右僕射。



 熙古兼通陰陽象緯之術,作《續聿斯歌》一卷、《六壬釋卦序例》一卷。性淳謹,雖顯貴不
 改寒素。歷官十八,登朝三十餘年,未嘗有過。嘗集古今事跡為《歷代紀要》十五卷。頗精小學,作《切韻拾玉》二篇,摹刻以獻,詔付國子監頒行之。子蒙正、蒙叟。



 蒙正字頤正,善騎射。乾德中,以蔭補殿直,遷供奉官。王師征江南,命乘傳軍中承奉事。盧絳以舟師來援潤州,蒙正白部署丁德裕,請分精甲百人,出與絳戰,矢中左肋,戰愈力。及下潤州,獲知州劉澄、監軍崔亮,部送闕下。



 嶺南陸運香藥入京,詔蒙正往規畫。蒙正請自廣、韶江
 溯流至南雄;由大庾嶺步運至南安軍,凡三鋪,鋪給卒三十人;復由水路輸送。



 又掌朝服法物庫,會重制繡衣、鹵簿,多其規式。太平興國四年,轉內藏庫副使,進崇儀使。自創內藏庫,即詔蒙正典領,凡二十餘年。



 真宗初,改如京使,出知滄、冀、磁三州。戎人犯境,蒙正調丁男乘城固守,有勞。未幾,以擅乘驛馬,責授亳州團練副使。咸平四年,卒,年七十二。



 蒙叟字道民,乾德中,進士甲科。歷岳、宿二州推官,以所
 知論薦,授太子中允、知乾興,拜監察御史,徙知濟州。俄以秦王子德恭判州事,就命為通判,郡事皆決於蒙叟。遷右補闕,轉起居舍人、戶部鹽鐵判官。再遷屯田郎中,歷知廬、濠、滁、汝四州,遷都官。



 咸平中,上疏曰:「陛下已周諒闇,方勤萬務,望崇儉德、遵守前規,無自矜能,無作奢縱,厚三軍之賜,輕萬姓之徭,使化育被於生靈,聲教加於中外。且萬國已觀其始,惟陛下慎守其終,思鮮克之言,戒性習之漸,則天下幸甚。」上嘉之,以本官直史館。



 車
 駕北巡,令知中宮名。表獻《宋都賦》,述國家受命建號之地,宜建都,立宗廟。時雖未遑,後卒從之。會詔直史館各獻舊文,以蒙叟所著為嘉,改職方郎中。景德中,以足疾,拜太常少卿致仕。卒,年七十三。



 蒙叟好學,善屬辭,著《五運甲子編年歷》三卷。



 子宗儒,太子中書;宗弼、宗誨,並進士及第。



 石熙載,字凝績,河南洛陽人。周顯德中,進士登第。疏俊有量,居家嚴謹,有禮法。宋初,太宗以殿前都虞候領泰
 寧軍節制,闢為掌書記。及尹京邑,表為開封府推官。授右拾遺,遷左補闕。丁外艱,將起復,以讒出為忠武、崇義二軍掌書記。太宗即位,復以左補闕召,同知貢舉。時梅山洞蠻屢為寇,以熙載知潭州。召還,擢為兵部員外郎,領樞密直學士。未幾,簽書樞密院事,詔賜官第一區。



 太平興國四年,親征河東,以給事中充樞密副使從行,還,遷刑部侍郎。五年,拜戶部尚書、樞密使,以病足在告,寢疾久之未愈。八年,上表求解職,詔加慰撫,授尚書右僕
 射。九年,卒,年五十七。贈侍中,謚元懿。上為悲嘆累日,且謂其事君之心,純正無他,適當委用,而奄忽至此,深為可惜。國朝大臣謝事而卒,車駕臨視者,唯熙載焉。



 熙載性忠實,遇事盡言,是非好惡,無所顧避。人有善,即推薦之,時論稱其長者。初,微時,為養負米。嘗行嵩陽道中,遇一叟,熟視熙載曰:「真人將興,子當居輔弼之位。」言訖不見。及居太宗幕下,頗盡誠節。典樞務日,上眷注甚篤,方將倚以為相,俄遘疾不起。



 熙載事繼母牛氏以孝聞。弟
 熙導,牛氏前夫子,隨母歸石氏。以熙載故,奏補殿直。從弟熙古、幼弟熙政,皆登進士第,熙載撫之如一。熙載卒時,子中孚、中立皆幼,熙政患熙導以異姓居己上,乃詐傳上旨,令己籍熙導家財,由是交訟。有司歸罪熙導,上召問中孚、中立,令有司再鞫得實。熙導還本姓,中孚亦養子勿問,熙政坐除名。上素知熙載以母故育熙導甚厚,雖令還宗,而不奪其官,復以財產量給之。



 咸平二年八月,熙載配饗太宗廟庭。熙政後至供備庫副使。中孚
 至尚書虞部員外郎,子行簡,大中祥符進士。



 中立字表臣,年十三而孤。性疏曠,好諧謔,人不以為怒。初補西頭供奉官,後五年,改光祿寺丞。家財悉推與諸父,無所愛。擢直集賢院,與李宗諤、楊億、劉筠、陳越相厚善。校讎秘書,凡更中立者,人皆傳之。判三司理欠、憑由司。



 帝幸亳,命修所過國經。為鹽鐵判官,累遷尚書禮部侍郎,判吏部南曹。注釋御集,為檢閱官。改判戶部勾院,遷戶部郎中、史館修撰,糾察在京刑獄。以吏部郎中、知
 制誥領審官院。又同知禮部貢舉,判集賢院。坐舉官不當,落史館修撰,罷審官院。頃之,復糾察刑獄,領三班院。歷右諫議大夫、給事中,入為翰林學士,判秘閣。知制誥並知貢舉,詔中立與張觀兼行外制,遷尚書禮部侍郎,為學士承旨兼龍圖閣學士。景祐四年,拜參知政事。明年,災異數見,諫官韓琦言:「中立在位,喜詼笑,非大臣體。」與王隨、陳堯佐、韓億皆罷,以戶部侍郎為資政殿學士,領通進、銀臺司,判尚書都省,進大學士。遷吏部侍郎、提
 舉祥源觀,以太子少傅致仕,遷少師。卒,贈太子太傅,謚文定。



 中立練習臺閣故事,不汲汲近名。喜賓客,客至必與飲酒,醉乃得去。初,家產歲入百萬錢,末年費幾盡。帝聞其病,賜白金三百兩。既死,其家至不能辦喪。子居簡,至太子中允、集賢校理。



 李穆,字孟雍,開封府陽武人。父咸秩,陜西大都督府司馬。穆幼能屬文,有至行。行路得遺物,必訪主歸之。從酸棗王昭素受《易》及《莊》、《老》書,盡究其義。昭素謂曰:「子所
 得皆精理,往往出吾意表。」且語人曰:「李生異日必為廊廟器。」以所著《易論》三十三篇授之。



 周顯德初,以進士為郢、汝二州從事,遷右拾遺。宋初,以殿中侍御史選為洋州通判。既至,剖決滯訟,無留獄焉。移陜州通判,有司調郡租輸河南,穆以本鎮軍食闕,不即應命,坐免。又坐舉官,削前資。時弟肅為博州從事,穆將母就肅居,雖貧甚,兄弟相與講學,意泊如也。



 開寶五年,以太子中允召。明年,拜左拾遺、知制誥。五代以還,詞令尚華靡,至穆而獨用
 雅正,悉矯其弊。穆與盧多遜為同門生,太祖嘗謂多遜:「李穆性仁善,辭學之外無所豫。」對曰:「穆操行端直,臨事不以生死易節,仁而有勇者也。」上曰:「誠如是,吾當用之。」時將有事江南,已部分諸將,而未有發兵之端。乃先召李煜入朝,以穆為使。穆至諭旨,煜辭以疾,且言「事大朝以望全濟,今若此,有死而已。」穆曰:「朝與否,國主自處之。然朝廷甲兵精銳,物力雄富,恐不易當其鋒,宜熟思之,無自貽後悔。」使還,具言狀,上以為所諭要切。江南亦謂
 其言誠實。



 太平興國初,轉左補闕。三年冬,加史館修撰、判館事,面賜金紫。四年,從征太原還,拜中書舍人。預修《太祖實錄》,賜衣帶、銀器、繒彩。七年,以與盧多遜款狎,又為秦王廷美草朝辭笏記,為言者所劾,責授司封員外郎。



 八年春,與宋白等同知貢舉,及侍上御崇政殿親試進士,上憫其顏貌懼瘁,即日復拜中書舍人、史館修撰、判館事。五月,召為翰林學士。六月,知開封府,剖判精敏,奸猾無所假貸,由是豪右屏跡,權貴無敢干以私,上益
 知其才。十一月,擢拜左諫議大夫、參知政事。月餘,丁母憂,未幾,起復本官。穆三上表乞終制,詔強起之,穆益哀毀盡禮。九年正月,晨起將朝,風眩暴卒,年五十七。



 穆自責授員外郎,復中書舍人,入翰林,參知政事,以至於卒,不及周歲。上聞其死,哭謂近臣曰:「穆國之良臣,朕方倚用,遽茲淪沒,非斯人之不幸,乃朕之不幸也。」贈工部尚書。



 穆性至孝,母嘗臥疾,每動止轉側,皆親自扶掖,乃稱母意。初,穆坐秦王事屬吏,其子惟簡紿祖母以穆奉詔
 鞫獄臺中。及責授為省郎,還家,亦不以白母。每隔日,陽為入直,即訪親友,或游僧寺。免歸,暨於牽復,母終弗之知。及居喪,思慕以至滅性。



 穆善篆隸,又工畫,常晦其事。質厚忠恪,謹言慎行,所為純至,無有矯飾。深信釋典,善談名理,接引後進,多所薦達。尤寬厚,家人未嘗見其喜慍。所著文章,隨即毀之,多不留稿。



 子惟簡,以父任將作監丞,多才藝,性沖澹,不樂仕進。去官家居三十餘年,人多稱之。真宗素聞其有履行,景德三年,詔授惟簡子郯
 將作監主簿。大中祥符七年冬,召惟簡入對,特拜太子中允致仕,後加太常丞。天禧四年,卒,賜其家錢十萬,仍給郯月奉終制。郯後為太子中舍。



 肅字季雍,七歲誦書知大義,十歲為詩,往往有警語。舉進士,登甲科。性嗜酒。歷濮、博二州從事,遷保靜軍節度推官。詔方下,一夕與親友會飲,酣寢而卒,年三十三。嘗作《大宋樂章》九首,取九成、九夏之義,以頌國家盛德,其文甚工。又作《代周顒答北山移文》、《吊幽憂子文》、《病雞賦》,
 意皆有所規焉。



 論曰:張昭居五季之末,專以典章撰述為事,博洽文史,旁通治亂,君違必諫,時君雖嘉尚之而不能從。宋興,敦獎碩儒,多所詢訪,庶幾獲稽古之效矣。竇氏弟昆以儒學進,並馳時望。儀之剛方清介,有應務之才,將試大用而遽淪亡。儼優游文藝,修起禮樂。太宗尹京,偁實元僚,沖淡回翔,晚著忠讜。若其門族宦業之盛,世或以為陰德之報,其亦義方之效也。餘慶當太祖居潛,歷任幕府,名亞
 趙普、李處耘;及二人登用,一不介意,其後相繼為眾所傾,乃能為之辯釋。熙古居大任,自處如寒素。熙載立朝,言無顧避,喜薦善人。穆以文學孝行見稱於時。數賢雖當創業之始,而進退之際,藹然承平多士之風焉,宜宋治之日進於盛也



\end{pinyinscope}