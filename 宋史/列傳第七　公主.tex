\article{列傳第七 公主}

\begin{pinyinscope}

 秦國大長公主太祖六女太宗七女真宗二女仁宗十三女英宗四女
 神宗十女哲宗四女徽宗三十四女孝宗二女光宗三女魏惠獻王一女寧宗一女理宗一女



 秦國大長公主,太祖同母妹也。初適米福德,福德卒。太祖即位,建隆元年,封燕國長公主,再適忠武軍節度使高懷德,賜第興寧坊。開寶六年十月薨,太祖臨哭,廢朝五日,賜謚恭懿。真宗追封大長公主。元符三年,改秦國。
 政和四年,改封恭懿大長帝姬。



 有姊一人,未笄而夭。建隆三年,追封陳國長公主。元符改封荊國大長公主。政和改封恭獻大長帝姬。



 太祖六女。申國、成國、永國三公主,皆早亡。



 魏國大長公主,開寶三年,封昭慶公主,下嫁左衛將軍王承衍,賜第景龍門外。太宗即位,進封鄭國。淳化元年,改封秦國。真宗至道三年,進長公主。大中祥符元年薨,賜謚賢肅。元符改封魏國大長公主。政和改賢肅大長
 帝姬。



 魯國大長公主,開寶五年,封延慶公主,下嫁左衛將軍石保吉。太宗即位,進封許國。淳化元年,改晉國。真宗初,進長公主。大中祥符二年,進大長公主。薨,賜謚賢靖。元符改封魯國。政和改賢肅大長帝姬。



 陳國大長公主,開寶五年,封永慶公主,下嫁右衛將軍魏咸信。太宗即位,進封虢國。淳化元年,改齊國。真宗初,進許國長公主。咸平二年薨,謚貞惠,後改恭惠。景祐三
 年,追封大長公主。元符改封陳國。政和改賢惠大長帝姬。



 太宗七女。長滕國公主,早亡。



 徐國大長公主,太平興國九年,封蔡國,下嫁左衛將軍吳元扆。淳化元年,改魏國。薨,謚英惠。至道三年,追封燕國長公主。景祐三年,進大長公主。元符改徐國。政和改英惠大長帝姬。



 邠國大長公主,太平興國七年為尼,號員明大師。八年
 卒。至道三年,追封曹國長公主。景祐三年,進大長公主。元符改邠國。



 揚國大長公主,至道三年,封宣慈長公主。咸平五年,進魯國,下嫁左衛將軍柴宗慶,賜第普寧坊。宗慶,禹錫之孫,帝命主以婦禮謁禹錫第。歷徙韓、魏、徐、福四國。仁宗立,進鄧國大長公主。明道二年薨,追封晉曙,謚力靖。元符封揚國。政和改和靖大長帝姬。主性妒,宗慶無子,以兄子為後。



 雍國大長公主,至道三年,封賢懿長公主。咸平六年,下嫁右衛將軍王貽永,進封鄭國,賜第。景德元年薨,謚懿順。景祐三年,追封大長公主。皇祐三年,改韓國。徽宗改封雍國。政和改懿順大長帝姬。



 衛國大長公主,至道三年,封壽昌長公主。大中祥符二年,進封陳國,改吳國,號報慈正覺大師。改楚國,又改邠國。天禧二年,改建國。乾興元年,封申國大長公主。天聖二年薨,賜謚慈明。徽宗改衛國。政和改慈明大長帝姬。



 荊國大長公主,幼不好弄,未嘗出房闥。太宗嘗發寶藏,令諸女擇取之,欲以觀其志,主獨無所取。真宗即位,封萬壽長公主,改隨國,下嫁附馬都尉郴遵勖。舊制選尚者降其父為兄弟行,時遵勖父繼昌亡恙,主因繼昌生日以舅禮謁之。帝聞,密以兼衣、寶帶、器模式幣助其為壽。遵勖賓客皆一昌賢士,每燕集,主必親視饔饎。嘗有盜入主第,帝命有司訊捕。主請出所逮系人,以私錢募告者,果得真盜,法當死,復請貰之。歷封越、宿、鄂、冀四國。明道
 元年,進魏國。



 初,遵勖出守許州,暴得疾,主亟欲馳視之,左右白:須奏得報乃可行,主不待報而往,從者裁五六人。帝聞,遽命內侍督諸縣邏兵以衛主車。其後居夫喪,衰麻未嘗去身,服除,不復御華麗。嘗燕禁中,帝親為簪花,辭曰:「自誓不復為此久矣。」嘗因浴僕地,傷右肱,帝遣內侍責侍者,主曰:「早衰力弱,不任步趨,非左右之過。」由是悉得免。



 主善筆札,喜圖史,能為歌詩,尤善女工之事。嘗誡諸子以「忠義自守,無恃吾以速悔尤」,視他子與己
 出均。及病目,帝挾醫診視,自後妃以下皆至第候問。帝親舐其目,左右皆感泣,帝亦悲慟曰:「先帝伯仲之籍十有四人,今獨存大主,奈何嬰斯疾!」復顧問子孫所欲,主曰:「豈可以母病邀賞邪?」繼白金三千兩,辭不受。帝因謂從臣曰:「大主之疾,倘可移於朕,亦所不避也。」主雖喪明,平居隱幾,沖淡自若。誡諸子曰:「汝父遺令:柩中無藏金玉,時衣數襲而已。吾歿後當亦如是。」



 皇祐三年薨,年六十四。帝臨奠,輟視朝五日。追封齊國大長公主,謚獻穆。
 徽宗改封荊國。政和改獻穆大長帝姬。



 真宗二女。長惠國公主,早亡。



 升國大長公主,初入道。明道二年,封衛國長公主,號清虛靈照大師。慶歷七年,追封魯國,謚昭懷。徽宗改封升國大長公主。政和改昭大長帝姬。



 仁宗十三女。徐國、鄧國、鎮國、楚國、商國、魯國、唐國、陳國、豫國九公主,皆早亡。



 周、陳國大長公主,帝長女也。寶元二年,封福康。嘉祐二
 年,進封袞國。主幼警慧,性純孝。帝嘗不豫,主侍左右,徒跣籲天,乞以身代。帝隆愛之。



 帝念章懿太后不及享天下養,故擇其兄子李瑋使尚主。瑋樸陋,與主積不相能。主中夜扣皇城門入訴,瑋皇懼自劾。諫官王陶論宮門夜開,乞繩治護衛,御史又共論主第內臣多不謹,帝為黜都監梁懷一輩十餘人。後數年不復協,詔出瑋於外,主降封沂,屏居內廷。久之,復召瑋,使為附馬都尉如初。英宗立,進越國長公主。神宗治平四年,進楚國大長公
 主。



 熙寧三年薨,年三十三。以瑋奉主無狀,貶陳州。輔臣議謚,帝以主事仁祖孝,命曰莊孝,追封秦國。徽宗加周、陳國。政和改封莊孝明懿大長帝姬。



 秦、魯國賢穆明懿大長公主,仁宗皇帝第十女也。母曰周貴妃。嘉祐五年,封慶壽,進惠國。治平四年,進許國大長公主。下嫁吳越忠懿王之曾孫、右領軍衛大將軍錢景臻。改韓、周燕國。徽宗朝,進秦、魏兩國。政和三年,更封令德景行大長帝姬。



 靖康二年,諸帝姬北徙,姬以先朝
 女,金人不知,留於汴。建炎初,復公主號,改封秦魯國。避地南渡,賊張遇掠其家,中子愕被害。公主至揚州朝謁,復避地之閩。



 紹興三年,自閩至會稽,請入見,因留居焉。後徙臺州。上以公主行尊年高,甚敬之,每入內,見必先揖。靖康中,戚里例納節,至是,公主為其子忱請還舊官,上以忱為滬川節度使,仍詔戚里不得援例。久之,又為忱請優賜推恩,上重違之,加忱開府儀同三司。時主有三子,愐、愷非己所出,故獨厚於忱。上戒之曰:「長主壽考
 如此,乃仁宗皇帝四十二年深仁厚澤,是以鐘慶於長主。長主待遇諸子,宜法仁宗用心之均一。」主感服。



 薨,年八十六。上輟朝五日,幸其第臨奠,詔子孫皆進官一等。謚曰賢穆。二十九年,加謚明懿。



 袞國大長公主,帝第十一女也。嘉祐六年,封永壽。進榮國長公主。治平四年,進邠國大長公主。熙寧九年,改魯國。下嫁左領軍衛大將軍曹詩。主性儉節,於池臺苑囿一無所增飭。十年夏,旱,曹族以
 主生日將盛具為壽,主曰:「上方損膳徹樂,吾何心能安。」悉屏之。



 元豐六年薨,年二十四,追封荊國,謚賢懿。遷其二子曄、旼皆領團練使。徽宗追封袞國,又改賢懿恭穆大長帝姬。



 燕、舒國大長公主,帝第十二女也。嘉祐六年,封寶壽。八年,進順國長公主。治平四年,進冀國大長公主。元豐五年,改魏國,下嫁開州團練使郭獻卿。上,進楚國。徽宗改吳國,進吳、越國,改秦、袞國。政和二年薨,追封燕、舒國,謚懿穆,復改懿物大長帝姬。



 英宗四女。舒國公主,早亡。



 魏、楚國大長公主,帝長女。嘉祐八年,封德寧。治平三年,進封徐國,下嫁左衛將軍王師約。四年,進陳國長公主。元豐八年薨,追封燕國大長公主,謚惠和。元祐四年,追封秦國。徽宗追封魏國,加韓、魏國,又改惠和大長帝姬。



 魏國大長公主,帝第二女,母曰宣仁聖烈皇后。嘉祐八年,封寶安公主。神宗立,進舒國長公主,改蜀國,下嫁左
 衛將軍王詵。詵母盧寡居,主處之近舍,日致膳羞。盧病,自和湯劑以進。帝厚於姊妹,故主第池SP服玩極其華縟。主以不得日侍宣仁於寶慈宮,居常悒然。間遇旱□,帝降損以禱,主亦如之,曰:「我奉賜皆出公上,固應同其僳戚。」帝居慈聖光獻皇后喪,毀甚,主曰:「吾與上同體,視此亦復保聊!」立散遣歌舞三十輩。



 元豐三年,病篤。主性不妒忌,王詵以是自恣,嘗貶官。至是,帝命還詵官,以慰主意。太后臨問,已不省,後慟哭,久稍能言,自訴必不起,
 相持而泣。帝繼至,自為診脈,親持粥食之,主強為帝盡食。賜金帛六千,且問所須,但謝復詵官而已。明日薨,年三十。帝未上食即駕往,望第門而哭,輟朝五日。追封越國,謚賢惠。後進封大長公主,累改秦、荊、魏三國。



 主好讀古文,喜筆札,賙恤族黨,中外稱賢。詵不矜細行,至與妾奸主旁,妾數抵戾主。薨後,乳母訴之,帝命窮治,杖八妝以配兵。既葬,謫詵均州。子彥弼,生三歲卒。



 韓、魏國大長公主,帝第三女,與魏國同生。始封壽康公
 主,改祁國、衛國,下嫁張敦禮。進冀國大長公主,改秦、越、楚國,加今封。政和三年,改賢德懿行大長帝姬。宣和五年薨。



 神宗十女。楚國、鄆國、潞國、刑國、邠國、袞國六公主,皆早薨。



 周國長公主,帝長女也。母曰欽聖憲肅後後。封延禧公主。生而警悟,自羈丱習嗜宛如成人。年十二卒,帝後皆變服哀送。追贈燕國。元符末,改封周國。



 唐國長公主,帝第三女也。始封淑壽公主。初,帝念韓琦
 功德,欲與為婚姻,故哲宗緣先帝,以主降琦之子嘉彥。歷封溫、曹、冀、雍、越、燕六國。政和元年薨,追封唐國長公主。



 潭國賢孝長公主,帝第四女也。母曰宋貴妃。始封康國。紹聖四年,下嫁王遇。歷韓、魯、陳、鄆四國。大觀二年薨,追加封謚。



 徐國長公主,帝幼女也。母曰欽成皇后。始封慶國,進益、冀、蜀、徐四國。年及笄,猶處聖瑞宮。侍母疾,晝夜不暫去,
 藥餌非經手弗以進。迨疾革,號慟屢絕,左右不忍視。



 崇寧三年,下嫁鄭王潘美之曾孫意。事姑修婦道。潘故大族,夫黨數千百人,賓接皆盡禮,無裏外言。志向沖淡,服玩不為紛華,歲時簡嬉游,十年間惟一適西池而已。再生子,不成而死,滕妾得女,拊視如己出。政和三年,改稱柔惠帝姬。五年薨,年三十一,追封賢靜長帝姬。



 哲宗四女。鄧國、揚國二公主,早亡。



 陳國公主,始封德康公主,進瀛國、榮國。大觀四年,下嫁
 石端禮,徙陳國。改淑和帝姬。政和七年薨。



 秦國康懿長公主,帝第三女也。始封康懿,進嘉國、慶國。政和二年,改韓國公主,出降潘正夫。改淑慎帝姬。靖康末,與賢德懿行大長公主俱以先朝女留於汴。建炎初,復公主號,改封吳國。覲上於越,以玉管筆、小玉山、奇畫為獻,上溫辭卻之。避地至婺州。



 紹興四年入見,其子堯卿等五人各進官一等。主奏言:「祖宗以來,駙馬都尉石保吉、魏咸信、柴宗慶皆除使相。今正夫歷事四朝,在汴
 京曾建議迎陛下,至杭州又言禁衛未集,預宜防變,乞除開府。」上不許。八年再入見。留宮中三日。時極暑,上每正衣冠對之飲食,又為正夫求恩數,上曰:「官爵豈可私與人,況今日多事,未暇及此。」時趙鼎當國,方論群臣紹述之奸,頗抑正夫。鼎去位,正夫始得開府之命。給事中劉一正言其非舊制,恐援例者多,乃詔:「哲宗惟正夫為近親,餘人毋得援。」顯仁太后歸,主同秦、魯國大長公主迎於道。十九年,又入朝。子長卿、粹卿、端卿皆自團練
 使升觀察使,從所請也。孝宗即位,進封秦國大長主。隆興二年薨,謚康懿。



 主在日,正夫官至少傅,封和國公;溫卿寧國軍承宣使,長卿寧江軍承宣使,端卿昭信軍承宣使,清卿容州觀察使,墨卿、才卿並帶團練使,其盛如此。正夫薨於紹興二十二年,贈太傅。



 徽宗三十四女。政和三年,改公主號為帝姬,國號易以美名,二字。



 嘉德帝姬,建中靖國元年六月,封德慶公主。改封嘉福,
 尋改號帝姬,再封嘉德。下嫁左衛將軍曾夤。



 榮德帝姬,初封永慶公主,改封榮福。尋改號帝姬,再封榮德。下嫁左衛將軍曹晟。



 順淑帝姬,初封順慶公主。薨,追封益國。及改帝姬號,追封順淑。



 安德帝姬,初封淑慶公主,改封安福。尋改號帝姬,再封安德。下嫁左衛將軍邦光。



 茂德帝姬,初封延慶公主,改封康福。尋改號帝姬,再封
 茂德。下嫁宣和殿待制蔡鞗。



 壽淑帝姬,初封壽慶公主。薨,追封豫國。及改帝姬號,追封壽淑。



 安淑帝姬,初封安慶公主,改封隆福。薨,追封蜀國。及改帝姬號,追封安淑。



 崇德帝姬,初封和慶公主,改封崇福。尋改帝姬號。下嫁
 左衛將軍曹湜。再封崇德。宣和二年薨。



 □柔福在五國城,適徐還而薨。靜善遂伏誅。柔福薨在紹興十一年,從梓宮來者以其骨至,葬之,追封和國長公主。



 孝宗二女:長嘉國公主,紹興二十四年,封碩人,進永嘉郡主,三十二年卒。詔以醫官李師克等屬吏,孝宗時居東宮,奏:「臣女幼而多疾,不宜罪醫。」遂寢。乾道二年,贈嘉國公主。次女生五月而夭,未及封。



 文安郡主,光宗長女也;次女封和政郡主;季女封齊安郡主。皆
 早卒。紹熙元年,並追贈公主。



 安康郡主,魏惠獻王女也。初封永寧郡主,改封通義。以父遺表,遂升安康。歸殿前司前軍統領羅忠信子良臣。詔王府主管鄧從義諭旨:「皇女孫郡主宜執婦道,務成肅雍之德,毋敢或違。」賜申第居之。良臣以恩轉秉義郎,除閣門祗候官。開禧元年,郡主薨,年三十九。



 祁國
 公主,寧宗女也。生六月而薨,追封祁國。



 周、漢國公主,理宗女也。母賈貴妃,早薨。帝無子,公主生而甚鐘愛。初封瑞國公主,改升國。開慶初,公主年及笄,詔議選尚。宰臣請用唐太宗降士人故事,欲以進士第一人尚主,遂取周震炎。廷謝日,公主適從屏內窺見,意頗不懌,帝微知之。



 景定二年四月,帝以楊太后擁立功,乃選太后侄孫鎮尚主。擢鎮右領軍衛將軍、駙馬都統,進封公主為周國公主。帝俗俚時見之,乃為主起第嘉會門,飛樓閣道,密邇宮苑,帝常御小輦從宮人過公主第。特賜董役官減三年磨勘,工匠犒賞有差。明年,進封周、漢國公主,拜鎮慶遠軍承宣使。鎮宗族娣姒皆推官加封,寵異甚渥。



 七月,主病。有鳥九首大如箕,集主家搗衣石上,是夕薨,年二十二。無子,帝哭之甚哀,謚端孝。鎮官節度使雲



\end{pinyinscope}