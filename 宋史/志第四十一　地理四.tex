\article{志第四十一 地理四}

\begin{pinyinscope}

 兩浙淮南東路淮南西路江南東路江南西路荊湖南路荊湖北路



 兩浙路。熙寧七年,分為兩路,尋合為一;九年,復分;十年,復合。府二:平江,鎮江。州十二:杭,越,湖,婺,明,常,溫,臺,處,衢,
 嚴,秀。縣七十九。南渡後,復分臨安、平江、鎮江、嘉興四府、安吉、常、嚴三州、江陰一軍為西路;紹興、慶元瑞安三府,婺、臺、衢、處四州為東路。紹興三十二年,戶二百二十四萬三千五百四十八,口四百三十二萬七千三百二十二。



 臨安府,大都督府,本杭州,餘杭郡。淳化五年,改寧海軍節度。大觀元年,升為帥府。舊領兩浙西路兵馬鈐轄。建炎元年,帶本路安撫使,領杭、湖、嚴、秀四州。三年,升為府,
 帶兵馬鈐轄。紹興五年,兼浙西安撫使。崇寧戶二十萬三千五百七十四,口二十九萬六千六百一十五。貢綾、藤紙。縣九:錢塘,望。有鹽監。



 仁和,望。梁錢江縣。太平興國四年改。紹興中,與錢塘並升赤。



 餘杭,望。



 臨安,望。錢鏐奏改衣錦軍。太平興國四年,改順化軍,縣復舊名。五年,軍廢。



 富陽,緊。於潛,緊。



 新城,上。梁改新登。太平興國四年復。淳化五年,升南新場為縣;熙寧五年,省南新縣為鎮入焉。



 鹽官,上。



 昌化。中。唐唐山縣。太平興國四年改。有紫溪鹽場。



 紹興中,七縣並升畿。



 紹興府,本越州,大都督府,會稽郡,鎮東軍節度。大觀元
 年,升為帥府。舊領兩浙東路兵馬鈐轄。紹興元年,升為府。崇寧戶二十七萬九千三百六,口三十六萬七千三百九十。貢越綾、輕庸紗、紙。縣八:會稽,望。



 山陰,望。



 嵊,望。舊剡縣,宣和三年改。



 諸暨,望。有龍泉一銀坑。餘姚,望。



 上虞望。



 蕭山緊。



 新昌。緊,乾道八年,以楓橋鎮置義安縣,淳熙元年省。



 平江府,望,吳郡。太平興國三年,改平江軍節度。本蘇州,政和三年,升為府。紹興初,節制許浦軍。崇寧戶一十五萬二千八百二十一,口四十四萬八千三百一十二。貢
 葛、蛇床子、白石脂、花席。縣六:吳,望。



 長洲,望。



 昆山,望。



 常熟,望。



 吳江,緊。



 嘉定。上。嘉定十五年,析昆山縣置,以年為名。



 鎮江府,望,丹陽郡,鎮江軍節度,開寶八年改。本潤州,政和三年升為府,建炎三年置帥。四年,加大使兼沿江安撫,以浙西安撫復還臨安。崇寧戶六萬三千六百五十七,口一十六萬四千五百六十六。貢羅、綾。縣三:丹徒,緊。有圌山砦。



 丹陽,緊。熙寧五年,省延陵縣為鎮入焉。



 金壇。緊。



 湖州,上,吳興郡,景祐元年,升昭慶軍節度。寶慶元年,改
 安吉州。崇寧戶一十六萬二千三百三十五,口三十六萬一千六百九十八。貢白絲寧、漆器。縣六:烏程,望。



 歸安,望。太平興國七年,析烏程地置縣。



 安吉,望。



 長興,望。



 德清,緊。



 武康。上。太平興國三年,自杭州來隸。



 婺州,上,東陽郡,淳化元年,改保寧軍節度。崇寧戶一十三萬四千八十,口二十六萬一千六百七十八。貢綿、藤紙。縣七:金華,望。



 義烏,望。



 永康,緊。



 武義,上。



 浦江,上。唐浦陽縣,梁錢寘奏改。



 蘭溪,望。



 東陽。望。


慶元府,本明州,奉化郡,建隆元年,升奉國軍節度。本上州,大觀元年,升為望。紹興初,置沿海制置使。八年,以浙東安撫使兼制司;十一年,罷;隆興元年,復置。淳熙元年,魏惠憲王自宣州移鎮,置長史、司馬。紹熙五年,以寧宗潛邸,升為府。崇寧戶一十一萬六千一百四十,口二十二萬一十七。貢綾、乾山蕷、烏
 \gezhu{
  則蟲}
 魚骨。縣六:鄞,望。



 奉化,望。



 慈溪,上。



 定海,上。象山,下。



 昌國。下。熙寧六年,析鄞縣地置,有鹽監。紹興間,升望。



 常州,望。毗陵郡,軍事。崇寧戶一十六萬五千一百一十
 六,口二十四萬六千九百九。貢白紵、紗、席。縣四:晉陵,望。



 武進,望。



 宜興,望。唐義興縣。太平興國初改。



 無錫。望。



 江陰軍,同下州。熙寧四年,廢江陰軍為縣,隸常州。建炎初,以江陰縣復置軍;紹興二十七年廢,三十一年,復置。縣一:江陰。下。



 瑞安府,本溫州,永嘉郡,太平興國三年,降為軍事。政和七年,升應道軍節度。建炎三年,罷軍額。咸淳元年,以度宗潛邸,升府。崇寧戶一十一萬九千六百四十,口二十六
 萬二千七百一十。貢鮫魚皮、蠲糨紙。縣四:永嘉,緊。有永嘉鹽場。



 平陽,望。有天富鹽場。



 瑞安,緊。有雙穗鹽場。



 樂清。上。唐樂成縣,梁錢鏐改。



 臺州,上,臨海郡,軍事。崇寧戶一十五萬六千七百九十二,口三十五萬一千九百五十五。貢甲香、金漆、鮫魚皮。縣五:臨海,望。



 黃巖,望。有於浦、杜瀆二鹽場。



 寧海,緊。



 天臺,上。



 仙居。上。唐樂安縣,梁錢鏐改永安。景德四年改今名。



 處州,上,縉雲郡,軍事。崇寧戶一十萬八千五百二十三,口一十六萬五百三十六。貢綿、黃連。縣六:麗水,望。龍泉,
 望。宣和三年,改為劍川縣。紹興元年復故。有高亭一銀場。



 松陽,上。梁錢鏐奏改長松,錢元瓘奏改白龍。咸平二年復故。



 遂昌,上。有永豐銀場。



 縉雲,上。



 青田。中。



 南渡後,增縣一:慶元。中。慶元三年,分龍泉松源鄉置縣,因以年紀名。



 衢州,上,信安郡,軍事。崇寧戶一十萬七千九百三,口二十八萬八千八百五十八。貢綿、藤紙。縣五:西安,望。



 禮賢,緊。本江山縣,南渡後改。



 龍游,上。唐龍丘縣。宣和三年,改為盈川縣。紹興初復改。信安,中。本常山縣,咸淳三年改。



 開化。中。太平興國六年,升開化場為縣。



 建德府,本嚴州,新定郡,遂安軍節度。本睦州,軍事。宣和
 元年,升建德軍節度;三年,改州名、軍額。咸淳元年,升府。崇寧戶八萬二千三百四十一,口一十萬七千五百二十一。貢白絲寧、簟。縣六:建德,望。



 淳安,望。舊青溪縣。宣和初,改淳化,南渡改今名。



 桐廬,上。太平興國三年,自杭州來隸。



 分水,中。



 遂安,中。



 壽昌。中。



 監一:神泉。熙寧七年置,鑄銅錢,尋罷。慶元三年復。



 嘉興府,本秀州,軍事。政和七年,賜郡名曰嘉禾。慶元元年,以孝宗所生之地,升府。嘉定元年,升嘉興軍節度。崇寧戶一十二萬二千八百一十三,口二十二萬八千六
 百七十六。貢綾。縣四:嘉興,望。



 華亭,緊。海鹽。上。有鹽監,沙腰、蘆瀝二鹽場。



 崇德。中。



 兩浙路,蓋《禹貢》揚州之域,當南斗、須女之分。東南際海,西控震澤,北又濱於海。有魚鹽、布帛、粳稻之產。人性柔慧,尚浮屠之教。俗奢靡而無積聚,厚於滋味。善進取,急圖利,而奇技之巧出焉。餘杭、四明,通蕃互市,珠貝外國之物,頗充於中藏雲。



 淮南路。舊為一路,熙寧五年,分為東、西兩路。



 東路。州十:揚,亳,宿,楚,海,泰,泗,滁,真,通。軍二:高郵,漣水。縣三十八。南渡後,州九:揚、楚、海、泰、泗、滁、淮安、真、通,軍四:高郵、招信、淮安、清河,為淮東路,宿、亳不與焉。紹興三十二年,戶一十一萬八百九十七,口二十七萬八千九百五十四。



 揚州,大都督府,廣陵郡,淮南節度。熙寧五年,廢高郵軍,並以縣隸州。元祐元年,復高郵軍。舊領淮南東路兵馬鈐轄。建炎元年,升帥府。二年,高宗駐蹕。四年,為真、揚鎮
 撫使,尋罷。嘉定中,淮東制置開幕府於楚州,仍兼安撫。崇寧戶五萬六千四百八十五,口十萬七千五百七十九。貢白苧布、莞席、銅鏡。縣一:江都。緊。熙寧五年,省廣陵縣入焉。



 南渡後,增縣二:廣陵,緊。泰興。中。舊隸泰州,紹興五年來屬。十年,又屬泰州。十二年,又來隸,以柴墟鎮延冷村隸海陵。二十九年,盡仍舊。



 亳州,望。譙郡,本防禦。大中祥符七年,建為集慶軍節度。南渡後,沒於金。崇寧戶一十三萬一百一十九,口一十八萬三千五百八十一。貢縐紗、絹。縣七:譙,望。



 城父,望。



 酇,
 望。



 永城,望。



 衛真,望。唐真源縣。大中祥符七年改。



 鹿邑,緊。



 蒙城。望。



 宿州,上,符離郡,建隆元年,升防禦。開寶元年,建為保靜軍節度。元領五縣,紹興中,割虹縣隸楚州,後沒於金。崇寧戶九萬一千四百八十三,口一十六萬七千三百七十九。貢絹。縣四:符離,望。



 蘄,望。



 臨渙,緊。大中祥符七年,割隸亳州,天禧元年來隸。



 靈壁。元祐元年,以虹之零壁鎮為縣,七月,復為鎮。七年二月,零壁復為縣。政和七年,改零壁為靈壁。



 楚州,緊,山陽郡,團練。乾德初,以盱眙屬泗州。開寶七年,以鹽城還隸。太平興國二年,又以鹽城監來隸。熙寧五
 年,廢漣水軍,以漣水縣隸州;元祐二年,復為漣水軍。建炎四年,置楚泗承州漣水軍鎮撫使、淮東安撫制置使、京東河北鎮撫大使。紹興五年,權廢承州兩縣,和、廬、濠、黃、滁、楚州各一縣,置鎮官。三十二年,漣水復來屬。嘉定初,節制本路沿邊軍馬。十年,制置安撫司公事。寶慶三年,升寶應縣為州。紹定元年,升山陽縣為淮安軍。端平元年,改軍為淮安州。崇寧戶七萬八千五百四十九,口二十萬七千二百二。貢苧布。縣四:山陽,望。建炎間入於金,紹興元年收
 復。紹定元年,升淮安軍,改縣為淮安。



 鹽城,上。有九鹽場。建炎間入於金,紹興元年隸漣水,三年,又來屬。



 淮陰,中。紹興五年,廢為鎮,六年,復。嘉定七年,徙治八里莊。



 寶應。緊。寶慶三年,升為寶應州,而縣如故。



 海州,上,東海郡,團練。建炎間,入於金,紹興七年復。隆興初,割以畀金,隸山東路,以漣水縣來屬。嘉定十二年復。寶慶末,李全據之。紹定四年,全死,又復。端平二年,徙治東海縣。淳祐十二年,全子□又據之,治朐山。景定二年,□降,置西海州。崇寧戶五萬四千八百三十,口九萬九千七百五十。貢絹、獐皮、鹿皮。縣四:朐山,緊。



 懷仁,中。



 沐陽,
 中



 東海。中。



 泰州,上,海陵郡。本團練,乾德五年,降為軍事。建炎三年,入於金,尋復。四年,置通、泰鎮撫使。紹興十年,移治泰興沙上,時泰興隸海陵,復舊治。元領四縣,紹興十二年,割泰興隸揚州。建炎四年,又以興化隸高郵軍。崇寧戶五萬六千九百七十二,口一十一萬七千二百七十四。貢隔織。縣二:海陵,望。



 如皋。中下。開寶七年,以海陵監移治。



 泗州,上,臨淮郡。建隆二年,廢徐城縣。乾德元年,以楚州
 之盱眙、濠州之招信來屬。建炎四年,復屬濠州。紹興十二年入金,後復。崇寧戶六萬三千六百三十二,口一十五萬七千三百五十一。貢絹。縣三:臨淮,上。



 虹,中。紹興九年,自宿州來隸。



 淮平。上。紹興二十一年,地入於金,析臨淮地置今縣。南渡後,有淮平無盱眙,蓋盱眙縣即招信軍也。



 滁州,上,永陽郡,軍事。建炎間,置滁、濠鎮撫使,尋廢。嘉熙中,移治王家沙。景定五年,復舊治。崇寧戶四萬二十六,口九萬七千八十九。貢絹。縣三:清流,望。



 全椒,緊。



 來安。望。唐永陽縣,南唐改。紹興五年,廢入清流。十八年,復。乾道九年廢為鎮。淳熙二年復。



 真州,望,軍事。本上州。乾德三年,升為建安軍。至道二年,以揚州之六合來屬。大中祥符六年,為真州。大觀元年,升為望。政和七年,賜郡名曰儀真。建炎三年,入於金,尋復。崇寧戶二萬四千二百四十二,口八萬二千四十三。貢麻紙。縣二:揚子,中。本揚州永正縣之白沙鎮,南唐改為迎鑾鎮。建炎元年升軍,四年,廢為縣。紹興十一年復升軍,十二年,復為縣。



 六合。望。



 通州,中,軍事。政和七年,賜郡名曰靜海。建炎四年,入於金,尋復。崇寧戶二萬七千五百二十七,口四萬三千一
 百八十九。貢獐皮、鹿皮、鰾膠。縣二:靜海,望周屬揚州,析其地為縣,與海門同來隸。



 海門。望。



 監一:利豐。掌煎鹽。太平興國八年,移治於州西南四里。



 高郵軍,同下州,高沙郡,軍事。開寶四年,以揚州高郵縣為軍。熙寧五年,廢為縣,隸揚州。元祐元年,復為軍。建炎四年,升承州,割泰州興化縣來屬;置鎮撫使。紹興五年,廢為縣,復隸揚州,以知縣兼軍使。三十一年,復為軍,仍以興化來屬。崇寧戶二萬八百一十三,口三萬八千七百五十一。縣一;今縣二:高郵,望。



 興化。緊。舊隸揚州,改隸泰州。建炎四年來
 隸。紹興五年廢為鎮,十九年,復縣,隸泰州。乾道二年還隸,尋又隸泰州,淳熙四年復舊。



 安東州,本漣水軍。太平興國三年,以泗州漣水縣置軍。熙寧五年,廢為縣,隸楚州。元祐二年,復為軍。紹興五年,廢為縣;三十二年,復為軍。紹定元年,屬寶應州。端平元年,復為軍。景定初,升安東州。崇寧戶一萬九千五百七十九,口四萬七百八十五。縣一:漣水。望。



 招信軍,本泗州盱眙縣,建炎三年,升軍,四年為縣,隸濠州。紹興二年,復隸泗州。七年,仍舊隸京東。十一年,隸天
 長軍。十二年,復升軍,以天長來屬。寶慶三年,入於金。紹定四年復,仍為招信軍。縣二:天長,望。舊天長軍。至道二年軍廢,復為縣,隸揚州。建炎元年升軍,紹興元年為縣。十一年,復升軍;十三年,復為縣,隸。招信。建炎四年,隸濠州。紹興四年復;十一年,隸天長軍;十二年,復來隸。



 淮安軍,本泗州五河口。端平二年,金亡,遺民來歸,置隘使屯田。咸淳七年六月,置軍。縣一:五河。咸淳七年置,有潧、涇、沱、崇、淮五河,故名。



 清河軍,咸淳九年置。縣一:清河。



 西路。府:壽春。州七:廬,蘄,和,舒,濠,光,黃。軍二:六安,無為。縣三十三。南渡後,府二:安慶、壽春,州六:廬、蘄、和、濠、光、黃,軍四:安豐、鎮巢、懷遠、六安。為淮西路。



 壽春府,壽春郡,緊,忠正軍節度。本壽州。開寶中,廢霍山、盛唐二縣。政和六年,升為府。八年,以府之六安縣為六安軍。紹興十二年,升安豐為軍,以六安、霍丘、壽春三縣來隸。三十二年,升壽春為府,以安豐軍隸焉。隆興二年,軍使兼知安豐縣事。乾道三年,罷壽春,復為安豐軍。崇
 寧戶一十二萬六千三百八十三,口二十四萬六千三百八十一。貢葛布、石斛。縣四:下蔡,緊。



 安豐,望。



 霍丘,望。



 壽春。緊。紹興初,隸安豐,三十二年為府,乾道三年為倚郭。



 六安軍,政和八年,升縣為軍。紹興十三年,廢為縣。景定五年,復為軍。端平元年,又為縣,後復為軍。縣一:六安。中。



 廬州,望,保信軍節度。大觀二年,升為望。舊領淮南西路兵馬鈐轄。建炎二年,兼本路安撫使。紹興初,寄治巢縣。乾道二年,置司於和州。五年,復舊。崇寧戶八萬三千五
 十六,口一十七萬八千三百五十九。貢紗、絹、蠟、石斛。縣三:合肥,上。



 舒城,下。



 梁。中。本慎縣。紹興三十二年,避孝宗諱,改今名。



 蘄州,望,蘄春郡,防禦。建炎初,為盜所據,紹興五年收復。景定元年,移治龍磯。崇寧戶一十一萬四千九十七,口一十九萬三千一百一十六。貢苧布、簟。縣五:蘄春,望。嘉熙元年治宿,景定二年,隨州治泰和門外。



 蘄水,望。



 廣濟,望。



 黃梅,上。



 羅田。元祐八年,以蘄水縣石橋為羅田縣。



 和州,上,歷陽郡,防禦。南渡後,為姑熟、金陵藩蔽也。淳熙
 二年,兼管內安撫。崇寧戶三萬四千一百四,口六萬六千三百七十一。貢苧布、練布。縣三:歷陽,緊。有梁山、柵江二砦。



 含山,中。有東關砦。



 烏江。中。紹興五年廢為鎮,七年,復。



 安慶府,本舒州,同安郡,德慶軍節度。本團練州。建隆元年,升為防禦。政和五年,賜軍額。建炎間,置舒、蘄鎮撫使。紹興三年,舒、黃、蘄三州仍聽江南西路安撫司節制。十七年,改安慶軍。慶元元年,以寧宗潛邸,升為府。端平三年,移治羅剎洲,又移楊槎洲。景定元年,改築宜城。舊屬
 沿江制置使司。崇寧戶一十二萬八千三百五十,口三十四萬一千八百六十六。貢白朮。縣五:懷寧,上。



 桐城,上。



 宿松,上。



 望江,上。



 太湖上



 監一:同安。熙寧八年置,鑄銅錢。



 濠州,上,鐘離郡,團練。乾道初,移戍藕塘,嘉定四年,始城定遠縣,復舊。崇寧戶六萬四千五百七十,口一十五萬三千四百五十七。貢絹、糟魚。縣二:鐘離,望。



 定遠。望。



 光州,上,弋陽郡,光山軍節度。本軍事州。宣和元年,賜軍額。紹興二十八年,避金太子光瑛諱,改蔣州。嘉熙元年,
 兵亂,徙治金剛臺,尋復故。崇寧戶一萬二千二百六十八,口一十五萬六千四百六十。貢石斛、葛布。縣四:定城,上。



 固始,望。



 光山,中下。同上避諱,改期思,尋復故。



 仙居。中下。南渡無。



 黃州,下,齊安郡,軍事。建炎隸沿江制置副使司。崇寧戶八萬六千九百五十三,口一十三萬五千九百一十六。貢苧布、連翹。縣三:黃岡,望。



 黃陂,上。端平三年,寓治青山磯。麻城。中。端平三年,治什子山。



 無為軍,同下州。太平興國三年,以廬州巢縣無為鎮建
 為軍,以巢、廬江二縣來屬。建炎二年,入於金,尋復。景定三年,升巢縣為鎮巢軍。崇寧戶六萬一百三十八,口一十一萬二千一百九十九。貢絹。縣三:無為,望。熙寧三年,析巢、廬江二縣地置縣。



 巢,望。至道二年,移治郭下。紹興五年廢,六年,復。十一年,隸廬州,十二年,復來屬。景定三年升軍,屬沿江制置使司。



 廬江。望。有昆山礬場。



 懷遠軍,寶祐五年五月置。縣一:荊山。



 淮南東、西路,本淮南路,蓋《禹貢》荊、徐、揚、豫四州之域,而揚州為多。當南斗、須女之分。東至於海,西抵濉、渙,南濱
 大江,北界清、淮。土壤膏沃,有茶、鹽、絲、帛之利。人性輕揚,善商賈,廛里饒富,多高貲之家。揚、壽皆為巨鎮,而真州當運路之要,符離、譙、亳、臨淮、朐山皆便水運,而隸淮服。其俗與京東、西略同。



 江南東、西路。建炎元年,以江寧府、洪州並升帥府,四年,合江東、西為江南路,以鄂、岳來屬。又置三帥:鄂州路,統鄂、岳、筠、袁、虔、吉州、南安軍;江西路,統江、洪、撫、信州、興國、南康、臨江、建昌軍;建康府路,統建康府、池、饒、宣、徽、太平
 州、廣德軍。紹興初,復分東西,以建康府、池、饒、徽、宣、信、撫、太平州、廣德建昌軍為江南東路;以江、洪、筠、袁、虔、吉州、興國、南康、臨江、南安軍為江南西路。尋以撫州、建昌軍還隸西路,南康軍還隸東路。置帥於池、江二州。未幾,以二州地僻隘,復還建康府、洪州。



 東路。府一:江寧。州七:宣,徽,江,池,饒,信,太平。軍二:南康,廣德。縣四十三。南渡後,府二:建康,寧國。州五:徽,池,饒,信,太平。軍二:南康,廣德,為東路。紹興三十二年,戶九十六萬
 六千四百二十八,口一百七十二萬四千一百三十七。



 江寧府,上,開寶八年,平江南,復為升州節度。天禧二年,升為建康軍節度。舊領江南東路兵馬鈐轄。建炎元年,為帥府。三年,復為建康府,統太平、宣、徽、廣德。五月,高宗即府治建行宮。紹興八年,置主管行宮留守司公事;三十一年,為行宮留守。乾道三年,兼沿江軍,尋省。崇寧戶一十二萬七百一十三,口二十萬二百七十六。貢筆。縣五:上元,次赤。



 江寧,次赤。



 句容,次畿。天禧四年,改名常寧。



 溧水,次畿。



 溧陽。次畿。



 寧國府,本宣州,宣城郡,寧國軍節度。乾道二年,以孝宗潛邸,升為府。七年,魏惠憲王出鎮,置長史、司馬。崇寧戶十四萬七千四十,口四十七萬七百四十九。貢絲寧布、黃連筆。縣六:宣城,望。



 南陵,望。



 寧國。緊。



 旌德,緊。



 太平,中。



 涇。緊。



 徽州,上,新安郡,軍事。宣和三年,改歙州為徽州。崇寧戶一十萬八千三百一十六,口一十六萬七千八百九十六。貢白苧、紙。縣六:歙,望。



 休寧,望。



 祁門,望。



 婺源,望。



 績溪,望。



 黟。緊。



 池
 州,上,池陽郡,軍事,建炎四年,分江東、西置安撫使,領建康、太平、宣、微、饒、廣德。後以建康路安撫使兼知池州。崇寧戶一十三萬五千五十九,口二十萬六千九百三十二。貢紙、紅白姜。縣六:貴池,望。



 青陽,上。開寶末,自升州與銅陵並來隸。銅陵,上。



 建德,上。唐至德縣,吳改。



 石埭,上。東流。中下。太平興國三年,自江州來隸。



 監一:永豐。鑄銅錢。



 饒州,上,鄱陽郡,軍事。崇寧戶一十八萬一千三百,口三十三萬六千八百四十五。貢麩金、竹簟。縣六:鄱陽,望。



 餘
 乾,望。



 浮梁,望



 樂平,望。



 德興,緊。



 安仁。中。開寶八年,以餘乾縣地置安仁場,端拱元年,升為縣。



 監一:永平。鑄銅錢。



 信州,上,上饒郡,軍事。崇寧戶一十五萬四千三百六十四,口三十三萬四千九十七。貢蜜、葛粉、水晶器。縣六:上饒,望。



 玉山,望。



 弋陽,望。淳化五年,升弋陽之寶豐場為縣;景德元年,廢寶豐縣為鎮,康定中復,慶歷三年又廢。



 貴溪,望。



 鉛山,中。開寶八年平江南,以鉛山直屬京,後還隸。



 永豐。中。舊永豐鎮,隸上饒,熙寧七年為縣。



 太平州,上,軍事。開寶八年,改南平軍。太平興國二年,升
 為州。崇寧戶五萬三千二百六十一,口八萬一百三十七。貢紗。縣三:當塗,上。



 蕪湖,中。開寶末,自建康軍與繁昌同隸宣州。太平興國三年,與繁昌復來隸。



 繁昌。中



 南康軍,同下州。太平興國七年,以江州星子縣建為軍。本隸西路,紹興初,來屬。崇寧戶七萬六百一十五,口一十一萬二千三百四十三。貢茶芽。縣三:星子,上。太平興國三年,升星子鎮為縣。七年,與都昌同來隸。



 建昌,望。太平興國七年,自洪州來隸。



 都昌。上。以縣有都村,南接南昌,西望建昌,故名。紹興七年,自江州來隸。



 廣德軍,同下州。太平興國四年,以宣州廣德縣為軍。崇寧戶四萬一千五百,口一十萬七百二十二。貢茶芽。縣二:廣德,望。開寶末,自江寧府隸宣州。建平。望。端拱元年,以郎步鎮為縣,來隸。



 西路。州六:洪,虔,吉,袁,撫,筠。軍四:興國,南安,臨江,建昌。縣四十九。南渡後,府一:隆興。州六:江,贛,吉,袁,撫,筠。軍四:興國,建昌,臨江,南安,為西路。紹興三十二年,戶一百八十九萬一千三百九十二,口三百二十二萬一千五百三十八。



 隆興府,本洪州,都督府,豫章郡,鎮南軍節度。舊領江南西路兵馬鈐轄。紹興三年,以淮西屯兵聽江西節制,兼宣撫舒、蘄、光、黃、安、復州,尋罷。四年,止稱安撫、制置使。八年,復兼安撫、制置大使。隆興三年,以孝宗潛藩,升為府。崇寧戶二十六萬一千一百五,口五十三萬二千四百四十六。貢葛。縣八:南昌,望。



 新建,望。太平興國六年置縣。



 奉新,望。唐新吳縣,南唐改。



 豐城,望。



 分寧,望。建炎間,升義寧軍,尋復。



 武寧,緊。



 靖安,中。南唐改。



 進賢。崇寧二年,以南昌縣進賢鎮升為縣。



 江州,上,潯陽郡,開寶八年,降為軍事。大觀元年,升為望郡。舊隸江南東路。建炎元年,升定江軍節度。二年,置安撫、制置使,以江、池、饒、信為江州路。紹興元年,復為二路,本路置安撫大使。嘉熙四年,為制置副使司治所。咸淳四年,移制置司黃州;十年,還舊治。崇寧戶八萬四千五百六十九,口一十三萬八千五百九十。貢云母、石斛。縣五:德化,望。唐潯陽縣,南唐改。



 德安,緊。



 瑞昌,中。



 湖口,中。



 彭澤。中。



 監一:廣寧。鑄銅錢。



 贛
 州,上。本虔州,南康郡,昭信軍節度。大觀元年,升為望郡。建炎間,置管內安撫使,紹興十五年罷,復置江西兵馬鈐轄,兼提舉南安軍、南雄州兵甲司公事。二十三年,改今名。崇寧戶二十七萬二千四百三十二,口七十萬二千一百二十七。貢白絲寧。縣十:贛,望。有蛤湖銀場。



 虔化,望。紹興二十三年,改寧都。有寶積鉛場。



 興國,望。太平興國中,析贛縣之七鄉置。



 信豐,望。



 雩都,望。



 會昌,望。太平興國中,析雩都六鄉於九州鎮置。有銀場。瑞金,望。有九龍銀場。



 石城,緊。



 安遠,上。



 龍南。中。南唐縣,本名龍南。宣和三年,改虔南。紹興二十三年,改龍南,取百丈龍灘之南為義。



 吉州,上,廬陵郡,軍事。崇寧戶三十三萬五千七百一十,口九十五萬七千二百五十六。貢絲寧布、葛。縣八:廬陵,望。



 吉水,望。雍熙元年,析廬陵地置縣。



 安福,望。



 太和,望。



 龍泉,望。宣和三年,改泉江,紹興復舊。



 永新,望。至和元年,徙吉水縣地置永新縣。



 永豐,望。



 萬安。望。熙寧四年,以龍泉縣萬安鎮置。



 袁州,上,宜春郡,軍事。崇寧戶一十三萬二千二百九十九,口三十二萬四千三百五十三。貢絲寧布。縣四:宜春,望。



 分宜,望。雍熙元年置。有貴山鐵務。



 萍鄉,望。



 萬載。緊。開寶末,自筠州來屬。宣和三年,改名建
 城。紹興元年,復今名。



 撫州,上,臨川郡,軍事。建炎四年,隸江南東路。紹興四年,復來隸。崇寧戶一十六萬一千四百八十,口三十七萬三千六百五十二。貢葛。縣五:臨川,望。紹興十九年,析惠安、穎秀二鄉入崇仁。



 崇仁,望。



 宜黃,望。開寶三年,升宜黃場為縣。



 金溪,緊。開寶五年,升金溪場為縣。



 樂安。紹興十九年置,割崇仁、吉水四鄉隸之。二十四年,以雲蓋鄉還隸永豐。



 瑞州,上,本筠州,軍事。紹興十三年,改高安郡。寶慶元年,避理宗諱,改今名。崇寧戶一十一萬一千四百二十一,
 口二十萬四千五百六十四。貢絲寧。縣三:高安,望。



 上高,望。



 新昌。望。太平興國六年,析高安地置縣。



 興國軍,同下州。太平興國二年,以鄂州永興縣置永興軍。三年,改興國。崇寧戶六萬三千四百二十二,口一十萬五千三百五十六。貢絲寧。縣三:永興,望。



 大冶,緊。南唐縣,自鄂州與通山並來隸。有富民錢監及銅場、磁湖鐵務。通山。中。太平興國二年,升羊山鎮為縣。紹興四年,又為鎮,五年復。



 南安軍,同下州。淳化元年,以虔州大庾縣建為軍。崇寧
 戶三萬七千七百二十一,口五萬五千五百八十二。貢絲寧。縣三:南康,望。《元豐九域志》南安軍領縣二,《崇寧地理》不載南康縣。據《元豐志》,南康系望縣,有瑞陽錫務,不知並於何時。



 大庾,中。淳化元年,自虔州與上猶、南康並來隸。



 上猶。上。有上田鐵務。嘉定四年,改南安。



 臨江軍,同下州。淳化三年,以筠州之清江建軍。崇寧戶九萬一千六百九十九,口二十萬二千六百五十六。貢絹。縣三:清江,望。



 新淦,望。淳化三年,自吉州來隸。



 新喻。望。淳化三年,自袁州來隸。



 建昌軍,同下州。舊建武軍,太平興國四年改。崇寧戶一
 十一萬二千八百八十七,口一十八萬五千三十六。貢絹。縣二:南城,望。淳化二年,自撫州來隸。有太平等四銀場。



 南豐。望。



 南渡後增縣二:新城,紹興八年,析南城五鄉置。



 廣昌。紹興八年,析南豐南境三鄉置。



 江南東、西路,蓋《禹貢》揚州之域,當牽牛、須女之分。東限七閩,西略夏口,南抵大庾,北際大江。川澤沃衍,有水物之饒。永嘉東遷,衣冠多所萃止,其後文物頗盛。而茗荈、冶鑄、金帛、粳稻之利,歲給縣官用度,蓋半天下之入焉。其俗性悍而急,喪葬或不中禮,尤好爭訟,其氣尚使然
 也。



 荊湖南、北路。紹興元年,以鄂、岳、潭、衡、永、郴、道州、桂陽軍為東路、鄂州置安撫司;鼎、澧、辰、沅、靖、邵、全州、武岡軍為西路,鼎州置安撫司。二年,罷東、西路,仍分南、北路安撫司,南路治潭州;北路治鄂,尋治江陵。



 北路。府二:江陵,德安,州十:鄂,復,鼎,澧,峽,岳,歸,辰,沅,靖。軍二:荊門,漢陽。縣五十六。南渡後,府三:江陵,常德,德安。州九:鄂,岳,歸,峽,復,澧,辰,沅,靖。軍三:漢陽,荊門,壽昌。紹興三
 十二年,戶二十五萬四千一百一,口四十四萬五千八百四十四。



 江陵府,次府,江陵郡,荊南節度。舊領荊湖北路兵馬鈐轄,兼提舉本路及施、夔州兵馬巡檢事。建炎二年,升帥府。四年,置荊南府、歸、峽州、荊門、公安軍鎮撫使,紹興五年罷。始制安撫使兼營田使,六年,為經略安撫使;七年,罷經略,止除安撫使。淳熙元年,還為荊南府。未幾,復為江陵府制置使。景定元年,移治於鄂。咸淳十年,荊湖、四
 川宣撫使兼江陵府事。崇寧戶八萬五千八百一,口二十二萬三千二百八十四。貢綾、絲寧、碧澗茶芽、柑桔。縣八:江陵,次赤。



 公安,次畿。



 潛江,次畿。乾德三年,升白伏巡為縣。



 監利,次畿。至道三年,以玉沙隸復州。熙寧六年,廢復州,以玉沙縣入監利縣,尋復其舊。



 松滋,次畿。



 石首,次畿。



 枝江,次畿。熙寧六年,省入松滋,元祐元年復。建炎四年,江陵寄治,紹興五年還舊。嘉熙元年,移澌、涅州。咸淳六年,移江南白水鎮下沱市。



 建寧。次畿。乾德三年,升白舊巡為縣,並置萬庾縣,萬庾尋廢。熙寧六年,省建寧入石首。元祐元年復。南渡後,省。



 鄂州,緊,江夏郡,武昌軍節度。初為武清軍,至道二年,始
 改。建炎二年,兼鄂、岳制置使。四年,兼江南鄂州路安撫,尋改鄂州路安撫。紹興二年,改兼荊湖北路安撫。六年,管內安撫;十一年,罷。嘉定十一年,置沿江制置副使。淳祐五年,兼荊湖北路安撫使。九年,罷。景定元年,改荊湖制置使。咸淳七年,罷。崇寧戶九萬六千七百六十九,口二十四萬七百六十七。貢銀。縣七:江夏,緊。



 崇陽,望。唐縣。開寶八年,又改今名。武昌,上。



 蒲圻,中。



 咸寧,中。



 通城,中。熙寧五年,升崇陽縣通城鎮為縣。紹興五年,廢為鎮。十七年,復。



 嘉魚。下。熙寧六年,析復州地入焉。



 監一:寶泉。熙寧七年置,鑄
 銅錢。



 南渡後,升武昌縣為壽昌軍。



 德安府,中,安陸郡,安遠軍節度。本安州。天聖元年,隸京西路,慶歷元年還本路。宣和元年,升為府。開寶中,廢吉陽縣。建炎四年,為安陸、漢陽鎮撫使。紹興三年,復來屬。咸淳中,徙治漢陽城頭山。崇寧戶五萬九千一百八十六,口一十四萬三千八百九十二。貢青絲寧。縣五:安陸,中。熙寧二年,省雲夢縣為鎮入焉,元祐元年復。



 應城,中。



 孝感,中。建炎間,移治紫資砦。



 應山,中下。



 雲夢。中。紹興七年,移治仵落市,十八年復舊。



 南渡後,無應山。



 復州,上,景陵郡,防禦。建炎四年,置德安、復州、漢陽軍鎮撫使。紹興三年,置荊湖北路安撫使。端平三年,移治沔陽鎮。貢闕。縣二:景陵,緊。晉縣。熙寧六年廢州,以景陵屬安州。元祐元年復。



 玉沙。下。至道三年,自江陵來隸。寶元二年,廢沔陽入焉。熙寧六年,又隸江陵府。元祐元年,與景陵皆復。



 常德府,本鼎州,武陵郡,常德軍節度。乾德二年,降為團練。本朗州。大中祥符五年,改今名。熙寧七年,廢桃源、湯口、白崖三砦。元豐三年,廢白磚、黃石二砦。政和七年,升為軍。建炎四年,升鼎、澧州鎮撫使。紹興元年,置荊湖北
 路安撫使,治鼎州,領鼎、澧、辰、沅、靖州;三十二年,罷。乾道元年,以孝宗潛藩,升府。八年,依舊提舉五州。崇寧戶五萬八千二百九十七,口一十三萬八百六十五。貢絲寧、布、練布。縣三:武陵,望。



 桃源,望。乾德中,析武陵地置縣。



 龍陽。中。大觀中,改辰陽。紹興元年復舊。五年,升軍使,移治黃城砦。三十年,復縣。



 南渡後,增縣一:沅江。中下。自岳州來隸。乾道中,割隸岳州,今復來隸。



 澧州,上,澧陽郡,軍事。建炎四年,寓治陶家市山砦,隨復舊。崇寧戶八萬一千六百七十三,口二十三萬六千九
 百二十一。貢綾、竹簟。縣四:澧陽,望。



 安鄉,中下。



 石門中下。有臺宜砦。



 慈利。下。有索口、安福、西牛、武口、澧州五砦。



 峽州,中,「峽」字舊從「硤」,今從「山」。



 夷陵郡,軍事。建炎中,移治石鼻山。紹興五年,復舊。端平元年,徙治於江南縣。崇寧戶四萬九百八十,口一十一萬六千四百。貢五加皮、芒硝、杜若。縣四:夷陵,中。有漢流、巴山、麻溪、魚陽、長樂、梅子六砦,及鉛錫場。



 宜都,中。



 長楊,中下。有漢流、飛魚二鹽井。元豐五年,廢新安、長楊二砦。



 遠安。中下。



 岳州,下,巴陵郡,岳陽軍節度。本軍事州。宣和元年賜軍
 額。建炎間,岳、鄂二州各帶沿江管內安撫司公事。紹興二十五年,改州曰純,改軍曰華容;三十一年,復舊。崇寧戶九萬七千七百九十一,口一十二萬八千四百五十。貢絲寧。縣四:巴陵,上。



 華容,望。有古樓砦。



 平江,上。



 臨湘。淳化元年,升王朝場為縣,尋改。



 歸州,下,巴東郡,軍事。建炎四年,隸夔路;紹興五年,復。三十一年,又隸夔;淳熙十四年,復。明年,又隸夔。端平三年,徙郡治於南浦。崇寧戶二萬一千五十八,口五萬二
 千一百四十七。貢絲寧。縣三:秭歸,下。熙寧五年,省興山縣為鎮入焉;元祐元年復。有筏禮砦、青林鹽井。



 巴東,下。有折疊砦。



 興山。下。開寶元年,移治昭君院。端拱二年,又徙香溪北。



 辰州,下,盧溪郡,軍事。太平興國七年,置招諭縣。熙寧七年,以麻陽、招諭二縣隸沅州;廢慢水砦、龍門、水浦、銅安、龔溪木砦。九年,廢明溪、豐溪、畬溪、新興、鳳伊、鐵爐、竹平、木樓、烏速、騾子、酉溪砦堡。崇寧戶一萬七百三十,口二萬三千三百五十。貢朱砂、水銀。縣四:沅陵,中。



 漵浦,中下。有懸鼓砦。元豐二年,置龍潭堡。



 辰溪,下。有龍門、銅安二砦。



 盧溪。下。



 城一:會溪。熙寧八年十二
 月置。



 砦三:池蓬,鎮溪,黔安。嘉祐三年,置池蓬,熙寧三年,置鎮溪。八年,置黔安。



 沅州,下,潭陽郡,軍事。本懿州。熙寧七年收復,以潭陽縣地置盧陽縣,以辰州麻陽、招諭二縣隸州。八年,並錦州砦人戶及廢招諭縣入麻陽,為一縣。元豐三年,並鎮江砦人戶入黔江城,為黔陽縣,尋廢鎮江砦為鋪。五年,升舊渠陽砦為縣,元祐六年,省為砦,崇寧二年,復為縣。崇寧戶九千六百五十九,口一萬九千一百五十七。貢朱砂、水銀。縣四:盧陽,下。有蔣州、西縣、八洲、長宜、回溪、鎮江、龍門、懷化八鋪。



 麻陽,下。有錦州
 砦,龍溪、龍家、竹砦、虛踵、齊天、叉溪六鋪。



 龔溪砦,熙寧六年賜名,其後為鋪,未詳。



 黔陽,下。有竹砦、煙溪、無狀、木州、洪江五鋪。



 渠陽。砦八:熙寧間,復硤中勝雲鶴繡五州、富錦圓三州。六年,以硤州新城為安江砦,富州新城為鎮江砦。七年,廢慢水砦、龍家堡,以辰州龍門、銅安二砦隸州,尋廢為鋪。宣和元年,復置銅安砦。元豐三年,置托口砦。四年,以古誠州貫保新砦為貫保砦,奉愛、豐山新堡為豐山新堡,小田、長渡村堡為小田砦。



 安江,有洪江、銅安二鋪。



 托口,有竹灘一鋪,元豐八年罷。



 貫保,元豐三年置,六年,隸誠州。元祐六年廢,崇寧二年復置。



 渠陽,元祐三年,以渠陽軍改,來隸。



 竹灘,洪江,並元祐五年置,隸黔陽縣。



 若溪,崇寧三年置。



 便溪。崇寧三年,以蔣州改。



 靖州,下,軍事。熙寧九年,收復唐溪洞誠州。元豐四年,仍
 建為誠州。五年,沅州貫保砦改為縣,總治本砦並托口、小由、豐山四堡砦戶口,以渠陽縣為名,隸州。六年,移托口、小由兩砦卻屬沅州,析邵州蒔竹縣隸州,移渠陽縣為州治。七年,沅州小由砦復隸州,尋廢小由砦、豐山堡。元祐二年,廢為渠陽軍。三年,廢軍為砦,屬沅州。元祐五年,復以渠陽砦為誠州。崇寧二年,改為靖州。大觀元年為望郡。崇寧戶一萬八千六百九十二,口闕。貢白絹。縣三:永平,下。本渠陽縣,崇寧二年,改名,紹興八年,移入州。



 會同,下。本三江縣,崇寧二年改。信道。
 下。本羅蒙縣,崇寧二年改。



 砦四:狼江,收溪,貫保,羅蒙。元豐六年,置收溪,復以沅州貫保來隸。七年,置羅蒙。元祐三年,廢收溪、羅蒙。崇寧二年,又置若水、豐山二砦。



 堡五:石家滻村,



 多星,大由,天村。元豐四年,置石家、滻村;六年,置多星;七年,置大由、天村。元祐三年,廢多星、大由、天村等堡,崇寧三年復置;又置羊鎮堡、木砦堡。大觀二年,又置飛山堡。政和三年,又置零溪堡。八年,又置通平堡。



 荊門軍,開寶五年,長林、當陽二縣自江陵來隸。熙寧六年,廢軍,縣復隸江陵府。元祐三年,復為軍。端平三年,移治當陽縣。縣二:長林,次畿。



 當陽。次畿。紹興十四年,廢入長林;十六年復。



 漢陽軍,同下州。熙寧四年,廢為縣,以漢川縣為鎮,屬鄂州。元祐元年,復置。紹興五年,又廢為縣;七年,復為軍。縣二:漢陽,緊。漢川。下。太平興國二年,自德安來隸。紹興五年廢,七年復。



 壽昌軍,下,本鄂州武昌縣。嘉定十五年,升壽昌軍使,續升軍。端平元年,以武昌縣還隸鄂州。縣一:武昌。上。以武昌山為名。孫權所都。南渡後,為江州治所,後復故。



 南路。州七:潭,衡,道,永,邵,郴,全。軍一:武岡。監一:桂陽。縣三十九。南渡後,增茶陵軍。紹興三十二年,戶九十六萬八
 千九百三十,口二百一十三萬六千七百六十七。



 潭州,上,長沙郡,武安軍節度。乾德元年,平湖南,降為防禦。端拱元年,復為軍。舊領荊湖南路安撫使。大觀元年,升為帥府。建炎元年,復為總管安撫司。紹興元年,兼東路兵馬鈐轄;二年,復為安撫司。崇寧戶四十三萬九千九百八十八,口九十六萬二千八百五十三。貢葛、茶。縣十二:長沙,望。開寶中,廢長豐縣入焉。



 衡山,望。淳化四年,以衡山、岳州湘陰並來隸。有黃竿銀場。



 安化,望。熙寧六年置,改七星砦為鎮入焉,廢首溪砦。元祐三年,置博易場。



 醴陵,緊。



 攸,上。



 湘
 鄉,中。



 湘潭,中



 益陽,中。



 瀏陽,中。有永興及舊溪銀場。



 湘陰,中。乾德二年,自鼎州隸岳州,俄而來隸。



 寧鄉,中。



 善化。元符元年,以長沙縣五鄉、湘潭縣兩鄉為善化縣。



 衡州,上,衡陽郡,軍事。崇寧戶一十六萬八千九十五,口三十萬八千二百五十三。貢麩金、犀。縣五:衡陽,緊。有熙寧錢監。



 耒陽,中。



 常寧,中下。熙寧六年,廢常寧縣獎中砦。有茭源銀場。



 安仁。中下。乾德二年,升安仁場為縣。



 南渡後,升茶陵為軍。



 道州,中,江華郡,軍事。乾德三年,廢大歷縣。熙寧六年,廢楊梅、勝岡、綿田三砦。紹興元年,隸荊湖東路;二年,復舊。
 崇寧戶四萬一千五百三十五,口八萬六千五百五十三。貢白絲寧、零陵香。縣四:營道,緊。熙寧五年,省永明縣為鎮入焉,元祐元年復。



 江華,緊。有黃富鐵場。寧遠,緊。唐延唐縣。乾德三年改。



 永明。上。



 永州,中,零陵郡,軍事。熙寧六年,廢福田、樂山二砦。八年,廢零陵砦。崇寧戶八萬九千三百八十七,口二十四萬三千三百二十二。貢葛、石燕。縣三:零陵,望。



 祁陽,中。



 東安。中。雍熙元年,升東安場為縣。有東安砦。



 郴州,中,桂陽郡,軍事。紹興初,改隸荊湖東路,二年,仍來
 屬。崇寧戶三萬九千三百九十二,口一十三萬八千五百九十九。貢絲寧。縣四:郴,緊,有新塘、浦溪二銀坑。



 桂陽,中。唐義昌縣,後唐改郴義。太平興國初,又改。有延壽銀坑。



 宜章,中。唐義章縣。太平興國初改。



 永興。中。舊高亭縣。熙寧六年改。



 南渡後,增縣二:興寧,嘉定二年,析郴縣資興、程水二鄉置資興縣,後改今名。



 桂東。本郴縣地。嘉定四年,析桂陽之零陵、宜城二鄉置今縣於上猶砦。



 寶慶府,本邵州,邵陽郡,軍事。大觀九年,升為望郡。寶慶元年,以理宗潛藩,升府。淳祐六年,升寶慶軍節度。崇寧戶九萬八千八百六十一,口二十一萬八千一百六十。
 貢犀角、銀。縣二:邵陽,望。



 新化。望。熙寧五年收復梅山,以其地置縣。有惜溪、柘溪、藤溪、深溪、雲溪五砦。



 全州,下,軍事。紹興元年,聽廣西路經略安撫司節制。崇寧戶三萬四千六百六十三,口一十萬六千四百三十二。貢葛、零陵香。縣二:清湘,望。有香煙、祿塘、長烏、羊狀、硤石、磨石、獲源七砦。



 灌陽。中。有洮水、灌水、吉寧砦。



 茶陵軍,紹興九年,升縣為軍,仍隸衡州。嘉定四年,析康樂、雲陽、常平三鄉置酃縣,亦嘗隸衡州。縣一:酃。下。因酃湖為名。



 桂陽軍,本桂陽監,同下州。紹興元年,隸荊湖東路,二年,復故。三年,升軍。崇寧戶四萬四百七十六,口一十一萬五千九百。貢銀。縣二;平陽,上。隋縣,晉廢。天禧三年置。有大富等九銀坑,熙寧七年復。



 藍山。中。景德三年,自郴州來隸。



 南渡後,增縣一:臨武。中。自石晉廢,紹興十一年復。



 武岡軍,崇寧五年,以邵州武岡縣升為軍。縣三:武岡,中。有山塘一砦。熙寧六年,廢白沙砦,置關硤、武陽、城步三砦。元祐四年,置赤木砦。紹聖元年,置神山砦。崇寧二年,置通硤。大觀元年,置峽口砦。



 綏寧,中。本邵州蒔竹縣地。熙寧九年廢,崇寧九年復。紹興十一年,移治武陽砦,二十五年,還舊。後廢臨岡來入。



 臨岡。本蒔竹縣。元豐四年,以溪洞徽州為縣,隸邵州。八年,建臨口砦。崇寧
 五年,改砦為縣,隸武岡軍。



 南渡後,廢臨岡,增新寧。下。漢夷地。紹興二十五年,於水頭江北立今縣。



 荊湖南、北路,蓋《禹貢》荊州之域。當張、翼、軫之分。東界鄂渚,西接溪洞,南抵五嶺,北連襄漢。唐末藩臣分據,宋初下之。鄂、岳本屬河南,安、復中土舊地,今以壤制而分隸焉。江陵國南巨鎮,當荊江上游,西控巴蜀。澧、鼎、辰三州,皆旁通溪洞,置兵戍守。潭州為湘、嶺要劇,鄂、岳處江、湖之都會,全、邵屯兵,以扼蠻獠。大率有材木、茗荈之饒,金
 鐵、羽毛之利。其土宜穀稻,賦入稍多。而南路有袁、吉壤接者,其民往往遷徙自占,深耕穊種,率致富饒,自是好訟者亦多矣。北路農作稍惰,多曠土,俗薄而質。歸、峽信巫鬼,重淫祀,故嘗下令禁之。



\end{pinyinscope}