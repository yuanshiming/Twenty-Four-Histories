\article{列傳第一百七十六}

\begin{pinyinscope}

 ○喬行
 簡範鐘游似趙葵兄範謝方叔



 喬行簡,字壽朋,婺州東陽人。學於呂祖謙之門。登紹熙四年進士第。歷官知通州,條上便民事。主管戶部架閣,
 召試館職,為秘書省正字兼樞密院編修官。升秘書郎,為淮西轉運判官,知嘉興府。改淮南轉運判官兼淮西提點刑獄、提舉常平。言金有必亡之形,中國宜靜以觀變。因列上備邊四事。會近臣有主戰者,師遂出,金人因破蘄、黃。移浙西提點刑獄兼知鎮江府。遷起居郎兼國子司業,兼國史編修、實錄檢討,兼侍講。尋遷宗正少卿、秘書監,權工部侍郎,皆任兼職。



 理宗即位,行簡貽書丞相史彌遠,請帝法孝宗行三年喪。應詔上疏曰:



 求賢、求
 言二詔之頒,果能確守初意,深求實益,則人才振而治本立,國威張而奸宄銷。臣竊觀近事,似或不然。夫自侍從至郎官凡幾人,自監司至郡守凡幾人,今其所舉賢能才識之士又不知其幾人也,陛下蓋嘗摭其一二欲召用之矣。凡內外小大之臣囊封來上,或直或巽,或切或泛,無所不有,陛下亦嘗摭其一二見之施行且褒賞之矣。而天下終疑陛下之為具文。



 蓋以所召者,非久無宦情決不肯來之人,則年已衰暮決不可來之人耳。彼
 風節素著、持正不阿、廉介有守、臨事不撓者,論薦雖多,固未嘗收拾而召之也。其所施行褒賞者,往往皆末節細故,無關於理亂,粗述古今,不至於抵觸,然後取之以示吾有聽受之意。其間亦豈無深憂遠識高出眾見之表,忠言至計有補聖聽之聰者,固未聞採納而用之也。



 自陛下臨御至今,班行之彥,麾節之臣,有因論列而去,有因自請而歸。其人或以職業有聞,或以言語自見,天下未知其得罪之由,徒見其置散投閑,倏來驟去,甚至廢
 罷而鐫褫,削奪而流竄,皆以為陛下黜遠善士,厭惡直言。去者遂以此而得名,朝廷乃因是而致謗,其亦何便於此。夫賢路當廣而不當狹,言路當開而不當塞,治亂安危,莫不由此。



 又言:「敬天命,伸士氣。」時帝移御清燕殿,行簡奏「願加畏謹」,且言:「群賢方集,願勿因濟王議異同,致有渙散。」升兼侍讀,兼國子祭酒、吏部侍郎,權禮部尚書。權刑部尚書,拜端明殿學士、同簽書樞密院事,進簽書樞密院事。



 太后崩,疏言:



 向者,陛下內廷舉動,皆有稟
 承。小人縱有蠱惑干求之心,猶有所忌憚而不敢發,今者,安能保小人之不萌是心?陛下又安能保聖心之不無少肆?陛下為天下君,當懋建皇極,一循大公,不私應徇小人為其所誤。



 凡為此者,皆戚畹肺肝之親,近習貴幸之臣,奔走使令之輩。外取貨財,內壞綱紀。上以罔人君之聰明,來天下之怨謗;下以撓官府之公道,亂民間之曲直。縱而不已,其勢必至於假採聽之言而傷動善類,設眾人之譽而進拔憸人,借納忠效勤之意而售其
 陰險巧佞之奸。日積月累,氣勢益張,人主之威權,將為所竊弄而不自知矣。



 陛下衰絰在身,愈當警戒,宮庭之間既無所嚴憚,嬪御之人又視昔眾多,以春秋方富之年,居聲色易縱之地,萬一於此不能自制,必於盛德大有虧損。願陛下常加警省。



 又論火災求言,乞取其切者付外行之。又論許國不當換文資,其當慮者有五;鄭損不當帥蜀。



 又言:「時青者,以官則國家之節度,以人則邊陲之大將,一旦遽為李全所戕,是必疑其終為我用,慮
 變生肘腋,故先其未發驅除之。竊意軍中必有憤激思奮之人,莫若乘勢就淮陰一軍拔其尤者以護其師,然後明指殺青者之姓名,俾之誅戮,加贈恤之典於青,則其勢自分,而吾得籍此以制之,則可折其奸心而存吾之大體。不然,跋扈者專殺而不敢誅,有功者見殺而不敢訴,彼知朝廷一用柔道而威斷不施,烏保其不遞相視效?則其所當慮者,不獨李全一人而已。」



 又言:「山陽民散財殫,非兇賊久安之地,當日夜為鴟張之計。揚州城
 堅勢壯,足以坐制全淮,此曹未必無窺伺之心,或為所入,則淮東俱非我有,不可不先為之慮也。」又請屯駐重兵海道,內為吳、越之捍蔽,外為南北之限制。



 又論:「李全攻圍泰州,剿除之兵今不可已。此賊氣貌無以逾人,未必有長算深謀,直剽捍勇決,能長雄於其黨耳,況其守泗之西城則失西城,守下邳則失下邳,守青社則失青社,既又降北,此特敗軍之將。十年之內,自白丁至三孤,功薄報豐,反背義忘恩,此天理人情之所共憤,惟決意
 行之。」後皆如行簡所料。拜參知政事兼知樞密院事。時議收復三京,行簡在告,上疏曰:



 八陵有可朝之路,中原有可復之機,以大有為之資,當有可為之會,則事之有成,固可坐而策也。臣不憂出師之無功,而憂事力之不可繼。有功而至於不可繼,則其憂始深矣。夫自古英君,必先治內而後治外。陛下視今日之內治,其已舉乎,其未舉乎?向未攬權之前,其敞凡幾?今既親政之後,其已更新者凡幾?欲用君子,則其志未盡伸;欲去小人,則其
 心未盡革。上有厲精更始之意,而士大夫之茍且不務任責者自若。朝廷有禁包苴、戒貪墨之令,而州縣之黷貨不知盈厭者自如。欲行楮令,則外郡之新券雖低價而莫售;欲平物價,則京師之百貨視舊直而不殊。紀綱法度,多頹弛而未張;賞刑號令,皆玩視而不肅。此皆陛下國內之臣子,猶令之而未從,作之而不應,乃欲闔闢乾坤,混一區宇,制奸雄而折戎狄,其能盡如吾意乎?此臣之所憂者一也。



 自古帝王,欲用其民者,必先得其心
 以為根本。數十年來,上下皆懷利以相接,而不知有所謂義。民方憾於守令,緩急豈有效死勿去之人;卒不愛其將校,臨陳豈有奮勇直前之士。蓄怨含憤,積於平日,見難則避,遇敵則奔,惟利是顧,皇恤其他。人心如此,陛下曾未有以轉移固結之,遽欲驅之北鄉,從事於鋒鏑,忠義之心何由而發?況乎境內之民,困於州縣之貪刻,厄於勢家之兼並,饑寒之氓常欲乘時而報怨,茶鹽之寇常欲伺間而竊發,蕭墻之憂凜未可保。萬一兵興於
 外,綴於強敵而不得休,潢池赤子,復有如江、閩、東浙之事,其將奈何?夫民至愚而不可忽,內郡武備單弱,民之所素易也。往時江、閩、東浙之寇,皆藉邊兵以制之。今此曹猶多竄伏山谷,窺伺田里,彼知朝廷方有事於北方,其勢不能以相及,寧不又動其奸心?此臣之所憂者二也。



 自古英君,規恢進取,必須選將練兵,豐財足食,然後舉事。今邊面遼闊,出師非止一途,陛下之將,足當一面者幾人?勇而能鬥者幾人?智而善謀者幾人?非屈指得
 二三十輩,恐不足以備驅馳。陛下之兵,能戰者幾萬?分道而趣京、洛者幾萬?留屯而守淮、襄者幾萬?非按籍得二三十萬眾,恐不足以事進取。借曰帥臣威望素著,以意氣招徠,以功賞激勸,推擇行伍即可為將,接納降附即可為兵,臣實未知錢糧之所從出也。興師十萬,日費千金,千里饋糧,士有饑色。今之饋餉,累日不已,至於累月,累月不已,至於累歲,不知累幾千金而後可以供其費也。今百姓多垂罄之室,州縣多赤立之帑,大軍一
 動,厥費多端,其將何以給之?今陛下不愛金幣以應邊臣之求,可一而不可再,可再而不可三。再三之後,兵事未已,欲中輟則廢前功,欲勉強則無事力。國既不足,民亦不堪。臣恐北方未可圖,而南方已先騷動矣。中原蹂踐之餘,所在空曠,縱使東南有米可運,然道里遼遠,寧免乏絕,由淮而進,縱有河渠可通,寧無盜賊邀取之患?由襄而進,必須負載二十鐘而致一石,亦恐未必能達。若頓師千里之外,糧道不繼,當此之時,孫、吳為謀主,韓、
 彭為兵帥,亦恐無以為策。他日運糧不繼,進退不能,必勞聖慮,此臣之所憂者三也。願陛下堅持聖意,定為國論,以絕紛紛之說。



 不果從。進知樞密院事。



 時議禦閱不果,反驟汰之,殿司軍哄,為之黜主帥,罷都司官,給黃榜撫存,軍愈呼噪。行簡以聞,戮為首者二十餘人,眾乃帖息。尋拜右丞相,言「三京撓敗之餘,事與前異,但當益修戰守之備。襄陽失守,請急收復。」或又陳進取之計,行簡奏:「今內外事勢可憂而不可恃者七。」言甚懇切,師得不
 出。



 端平三年九月,有事於明堂,大雷雨。行簡與鄭清之並策免。既去,而獨趣召行簡還京,留之,拜左丞相。援韓琦故事,乞以邊防、財用分委三執政,請修中興五朝國事。十上章請謝事。嘉熙三年,拜平章軍國重事,封肅國公。每以上游重地為念,請建節度宣撫使,提兵戍夔。邊事稍寧,復告老,章十八上。四年,加少師、保寧軍節度使、醴泉觀使,封魯國公,淳祐元年二月,薨於家,年八十六。贈太師,謚文惠。



 行簡歷練老成,識量弘遠,居官無所不
 言。好薦士,多至顯達,至於舉錢時、吳如愚,又皆當時隱逸之賢者。所著有《周禮總說》、《孔山文集》。



 範鐘,字仲和,婺州蘭溪人。嘉定二年,舉進士。歷官調武學博士,添差通判太平州,知徽州。召赴闕,遷刑部郎官,又遷尚右郎官兼崇政殿說書。進對,帝曰:「仁宗時甚多事。」鐘對曰:「仁宗始雖多事,乃以憂勤致治。徽宗始雖無事,餘患至於今日。」帝悅。尋遷吏部郎中兼說書,又遷秘書少監、國子司業兼國史編修、實祿檢討。拜起居郎兼
 祭酒,權兵部侍郎兼同修國史、實祿同修撰。遷兵部侍郎兼給事中,權兵部尚書兼侍講,尋兼侍讀。嘉熙三年,拜端明殿學士、簽書樞密院事。四年,授參知政事。淳祐元年,乞歸田里,不許。四年,知樞密院事,乞歸田里。五年,特拜左丞相兼樞密使,封東陽郡公,再乞歸田里,不許。六年,復請,許之。加觀文殿大學士、醴泉觀使兼侍讀,辭不拜,以保晚節,乃提舉洞霄宮。九年正月,薨。



 鐘為相,直清守法,重惜名器,雖無赫赫可稱,而清德雅量,與杜範、
 李宗勉齊名。贈少師,謚文肅。所著書有《禮記解》。



 游似,字景仁,利路提點刑獄仲鴻之子。嘉定十四年進士,歷官為大理司直,升大理寺丞,遷太常丞兼權兵部郎官。遷秘書丞兼權考功郎中、直秘閣、夔路轉運判官,移潼川提點刑獄兼提舉常平。請封謚田錫,從之。遷軍器監、宗正少卿兼權樞密都承旨。



 時暫兼權禮部侍郎兼侍講、權禮部侍郎。有事於明堂,似上疏言:「欲盡事天之禮,當盡敬天之心。心存則政事必適其宜,言動必當
 其理,雨蛇菴循其序,夷夏必安其生。」兼同修國史、實錄院同修撰,權禮部尚書兼侍讀。言:「軍賞冒濫,請給告之制,奏功者書填真命付之,候從軍十年,別能立功,升至統領已上,方許從所屬保明申朝廷,立名給告,則冒濫者革,功勞者勸。」



 遷禮部尚書兼給事中兼修國史、實錄院修撰,權工部侍郎,充四川宣撫司參贊軍事兼給事中。遷吏部尚書,入侍經幄。帝問:「唐太宗貞觀治效何速如是?」似對曰:「人主一念之烈,足以旋乾轉坤。或謂霸圖
 速而王道遲,不知一日歸仁,期月而可,王道曷嘗不速。一念有時間斷,則無以挽回天下之大勢。至於憂勤,既切宸念,而佐理非人,亦何以布宣九重之實。」乃摭太宗事以陳,且謂:「太宗矜心易啟,漸弗克終,僅止貞觀之治。陛下嗣服十有五年,艱危之勢滋甚,回視太宗治效敏速、相越乃爾。意者親儒而從諫,敬畏以檢身,未若貞觀之超卓乎?節用以致愛,選廉以共理,未若貞觀之切至乎?願陛下益加聖心。」



 嘉熙三年正月,拜端明殿學士、同
 簽書樞密院事,封南充縣伯。八月,拜參知政事。四年閏月,知樞密院事兼參知政事。淳祐四年,提舉萬壽觀兼侍讀,仍奉朝請,授知樞密院事兼參知政事,進爵郡公。五年,拜右丞相兼樞密使。十上章,乞歸田里,帝不許。七年,特授觀文殿大學士、醴泉觀使兼侍讀,進爵國公。十一年,轉兩官致仕,薨。特贈少師。



 趙葵,字南仲,京湖制置使方之子。初生時,或夢南嶽神降其家。方在襄陽,命葵專督飲食共養之事。與兄範俱
 有志事功,方器之,聘鄭清之、全子才為之師。又遣從南康李燔為有用之學。每聞警報,與諸將偕出,遇敵則深入死戰,諸將惟恐失制置子,盡死救之,屢以此獲捷。一日,方賞將士,恩不償勞,軍欲為變。葵時十二三,覺之,亟呼曰:「此朝廷賜也,本司別有賞齎。」軍心賴一言而定,人服其機警。



 嘉定十年,金將高琪、烏古論慶壽犯襄陽,圍棗陽。時邊烽久熄,金兵猝至,人情震懼。方帥範、葵往戰,敗走之。十三年,方遣葵及都統扈再興攻金人至高頭。高
 頭,金人必守之處也,出勁兵拒戰,葵率先鋒奮擊,再興繼進殲之。翼日,進次鄧州,金人阻沘河以拒。葵麾軍進擊,楊義諸將繼至,金兵亦大出合戰,大破之,俘斬及降者幾二萬,獲萬戶而下十數人,奪馬八百,逐北直傅城下而還。



 十四年,金人犯蘄州,葵與範攻唐、鄧。方命之曰:「不克敵,毋相見也。」三月丁亥,至唐州,薄城而陳。金大將阿海引兵出戰,葵帥精騎赴敵,再興從之,大捷,斬馘萬餘。金人閉門不出。時金人陷蘄州者至久長,數十騎出
 山椒,葵帥楊大成以十四騎逐之。金騎漸益至數百,葵力戰連破之,而金步騎大集。會範、再興軍合戰,至夜分始解。庚寅,官軍分二陣,範將左,再興將右,葵帥突騎左右策應。金人背山亦分為二以相當,而不先動。範曰:「金人必復謀夜戰以幸勝,乃預備大鼓,令軍中聞疊鼓聲始動,若彼未至五十步內而輒動者斬。未幾,金兵稍下山,再興遽沖之,果為敵所乘,遂逼範軍。範疊鼓麾軍突斗,葵繼進,殲金兵數千。敵並力向再興,葵率土豪祝文
 蔚等以精騎橫沖之,金人殭尸相屬。復相持至夜分,金人雖斂,而陣如故。範、葵急會將校,選死士數千,黎明四面奮擊,喚聲撼山谷。金人走,乘勝逐北,斬首數千級,副統軍投戈降,拔所掠子女萬餘,得輜重器械山積。補葵承務郎、知棗陽軍,範授安撫司內機。



 方卒,十五年,起復直秘閣、通判廬州,進大理司直、淮西安撫參議官。十七年,李全往青州,淮東制置使許國檄葵議兵。葵至曰:「君侯欲圖賊,而坐賊阱中,悔已無及,惟有重帳前兵,猶足
 制之爾。」國曰:「兵不能集,集不能精,奈何?」曰:「葵請視兩路之兵,別其精銳,君侯留三萬帳前,賊不敢動矣。」國曰:「不若集淮兵來閱,而君董之,既足示眾,亦可選銳。」葵曰:「有兵之郡,必當沖要,守將豈可空壁以從制使命耶?必將力爭於朝,分留自衛。一得朝命,必匿其強壯,遣老弱以備數。本欲選銳,適得其鈍,本欲示眾,適示單弱,徒啟戎心。」國不聽,卒敗。



 寶慶元年,範知揚州,乞調葵以強勇、雄邊軍五千屯寶應備賊。葵在廬州,數費私錢會諸將球
 射,與制置使曾式中不合,葵去之。言者以為擅,遂奉祠。三年,起為將作監丞。



 紹定元年,出知滁州。二年,全將入浙西告糴,實欲覘畿甸也。初,全之獻俘也,朝廷授以節鉞,葵策其必叛,乃上書丞相史彌遠曰:「此賊若止於得粟,尚不宜使輕至內地,況包藏禍心,不止告糴。若不痛抑其萌,則自此肆行無憚,所謂延盜入室,恐畿內有不可勝諱之憂。」至滁,以其地當賊沖,又與金人對境,實兩淮門戶,修城浚隍,經武不少暇。命秦喜守青平,趙必勝
 守萬山,以壯形勢。葵母疾,謁告省侍不得,刲股雜藥以寄之。母卒,葵求解官,不許,不得已,卒哭復視事。



 全造舟益急,葵復致書史彌遠曰:「李全既破鹽城,反稱陳知縣自棄城,蓋欲欺朝廷以款討罪之師,彼得一意修舟楫,造器械,窺伺城邑,或直浮海以搗腹心,此其奸謀,明若觀火。葵自聞鹽城失守,日夕延頸以俟制帥之設施,今乃聞遣王節入鹽城祈哀於逆。葵又聞遣二吏入山陽,請命於賊婦。堂堂制閫,如此舉措,豈不墮賊計,貽笑天
 下、貽笑外夷乎?又聞張國明前此出山陽,已知賊將舉鹽城之兵,今若聽國明言,更從闊略,則自此人心解體,萬事渙散,社稷之憂有不可勝諱者。葵非欲張皇生事啟釁,李全決非忠臣,非孝子。丞相茍聽葵之言,翻然改圖,發兵討叛,則豈獨可以強國勢安社稷,葵父子世受國恩,亦庶幾萬一之報。使丞相不聽葵言,不發兵討賊,則豈特不可以強國勢安社稷,而葵亦不知死所,不復可報君相之恩矣。一安一危,一治一亂,系朝廷之討叛
 與不討爾。淮東安則江南安,江南安則社稷安,社稷安則丞相安,丞相安則凡為國之臣子、為丞相之門人弟子莫不安矣。」



 又言於朝曰:「葵父子兄弟,世受國恩,每見外夷、盜賊侵侮國家,未嘗不為忠憤所激。今大逆不道,邈視朝廷,負君相卵翼之恩,無如李全。前此畔逆未彰,猶可言也,今已破蕩城邑,略無忌憚,若朝廷更從隱忍,則將何以為國?欲望特發剛斷,名其為賊,即日命將遣師,水陸並進,誅鋤此逆,以安社稷,以保生靈。葵雖不才,
 願身許朝廷;如或不然,乞將葵早賜處分,以安邊鄙,以便國事。」



 彌遠猶未欲興討,參知政事鄭清之贊決之。乃加葵直寶章閣、淮東提點刑獄兼知滁州。範刻日約葵,葵帥雄勝、寧淮、武定、強勇步騎萬四千,命王鑒、扈斌、胡顯等將之,以葵兼參議官。顯,穎之兄也,拳力絕人,方在襄陽,每出師必使顯及葵各領精銳分道赴戰,摧堅陷陣,聚散離合,前無勁敵,以功至檢校太尉。



 已而,全攻揚州東門,葵親出搏戰。賊將張友呼城門請葵出,及出,全
 在隔壕立馬相勞苦。左右欲射全,葵止之,問全來何為?全曰:「朝廷動見猜疑,今復絕我糧餉,我非背叛,索錢糧耳。」葵曰:「朝廷資汝錢糧,寵汝官職,蓋不貲矣。待汝以忠臣孝子,而乃反戈攻陷城邑,朝廷安得不絕汝錢糧。汝雲非叛,欺人乎?欺天乎?」切責之言甚多,全無以對,彎弓抽矢向葵而去。於是數戰皆捷。四年正月壬寅,遂殺全。事見《全傳》。進葵福州觀察使、左驍衛上將軍,葵辭不受。八月,召封樞密院稟議,受寶章閣待制、樞密副都承旨,
 依舊職仍落起復,尋進兵部侍郎。



 六年十一月,詔授淮東制置使兼知揚州,入對,帝曰:「卿父子兄弟,宣力甚多,卿在行陣又能率先士卒,捐身報國,此尤儒臣之所難,朕甚嘉之。」葵頓首謝曰:「臣不佞,忠孝之義,嘗奉教於君子,世受國恩,當捐軀以報陛下。」



 端平元年,朝議收復三京,葵上疏請出戰,乃授權兵部尚書、京河制置使,知應天府、南京留守兼淮東制置使。時盛暑行師,汴堤破決,水潦泛溢,糧運不繼,所復州郡,皆空城,無兵食可因。未
 幾,北兵南下,渡河,發水閘,兵多溺死,遂潰而歸。範上表劾葵,詔與全子才各降一秩,授兵部侍郎、淮東制置使,移司泗州。



 嘉熙元年,以寶章閣學士知揚州,依舊制置使。二年,以應援安豐捷,奏拜刑部尚書,進端明殿學士,特予執政恩例,復兼本路屯田使。葵前後留揚八年,墾田治兵,邊備益飭。淳祐二年,進大學士、知潭州、湖南安撫使,改福州。



 三年,葬其母,乞追服終制,不允。葵上疏曰:「移忠為孝,臣子之通誼;教孝求忠,君父之至仁。忠孝一
 原,並行不悖。故曰忠臣以事其君,孝子以事其親,其本一也。臣不佞,戒謹持循,惟恐先墜。往歲叨當事任,服在戎行,偕同氣以率先,冒萬死而不顧,捐軀戡難,效命守封,是以孝事君之充也。陛下昭示顯揚,優崇寵數,使為人子者感恩,為人親者知勸矣。臣昨於草土,被命起家,勉從權制,先國家之急而後親喪也。今釋位去官,已追服居廬,乞從彞制。」又不許。再上疏曰:「臣昔者奉詔討逆,適丁家難,閔然哀疚之中,命以驅馳之事,移孝為忠,所不
 敢辭。是臣嘗先國家之急,而效臣子之義矣。親恩未報,浸逾一紀,食稻衣錦,俯仰增愧。且臣業已追衰麻之制,伸苫塊之哀,負土成墳,倚廬待盡,喪事有進而無退,固不應數月而除也。」乃命提舉洞霄宮,不拜。



 淳祐四年,授同知樞密院事。疏奏:「今天下之事,其大者有幾?天下之才,其可用者有幾?吾從其大者而講明之,疏其可用者而任使之。有勇略者治兵,有心計者治財,寬厚者任牧養,剛正者持風憲。為官擇人,不為人而擇官。用之既當,
 任之既久,然後可以責其成效。」又乞「亟與宰臣講求規畫,凡有關於宗社安危治亂之大計者條具以聞,審其所先後緩急以圖籌策,則治功可成,外患不足畏」。又乞「創游擊軍三萬人以防江」。詔從之。十二月,拜知樞密院事兼參知政事。又特授樞密使兼參知政事、督視江、淮、京西、湖北軍馬,封長沙郡公。尋知建康府、行宮留守、江東安撫使。



 九年,特授光祿大夫、右丞相兼樞密使,封信國公。四上表力辭,言者以宰相須用讀書人,罷為觀文
 殿學士,充醴泉觀使兼侍讀,仍奉朝請。尋判潭州、湖南安撫使,加特進。寶祐二年,宣撫廣西。三年,改鎮荊湖,城荊門及郢州。改授湖南路安撫使、判潭州,再辭,依舊職醴泉觀使。五年,進少保、寧遠軍節度使,進封魏國公、醴泉觀使兼侍讀。四辭,免。開慶元年,判慶元府、沿海制置使,尋授沿江、江東宣撫使,置司建康府,任責隆興府、饒州江州徽州兩界防拓調遣,時暫兼判建康府、行宮留守,尋授江東西宣撫使,節制調遣饒、信、袁、臨江、撫、吉、隆
 興官軍民兵。訪問百姓疾苦,罷行黜陟,並許便宜從事。



 景定元年,授兩淮宣撫使、判揚州,進封魯國公,尋奉祠。咸淳元年,加少傅。二年,乞致仕,特授少師、武安軍節度使,進封冀國公。舟次小孤山,薨,年八十一。是夕,五洲星隕如箕。贈太傅,謚忠靖。



 範字武仲,少從父軍中。嘉定十三年,嘗與弟葵殲金人於高頭。十四年,出師唐、鄧,範與葵監軍。孟宗政時知棗陽,憚於供億,使人問曰:「金人在蘄、黃,而君攻唐、鄧,何也?」
 範曰:「不然,徹襄陽之備以救蘄、黃,則唐、鄧必將躡吾後。且蘄、黃之寇正銳,曷若先搗唐、鄧以示有餘,唐、鄧應我之不暇,則吾圉不守而自固,寇在蘄、黃師日以老,然後回師蹙之,可勝敵而無後患。」又敗金人於久長,與弟葵俱授制置安撫司內機,事具《葵傳》。



 十五年,丁父憂,起復直秘閣、通判揚州。十六年,為軍器監丞,以直秘閣知光州。十七年,入為知大宗正丞、刑部侍郎、試將作監兼權知鎮江府。進直徽猷閣、知揚州、淮東安撫副使。劉全、王
 文信二軍老幼留揚州,範欲修軍政,懼其徒漏洩兵機,乃時饋勞。二家既大喜,範即遺徐晞稷書,令教二人挈家歸楚,二人從之,範厚齎以遣。有孫海者,其眾亦八百。範並請抽還楚州,又請創馬軍三千,招游手之強壯者及籍牢城重役人充之。別籍民為半年兵,春夏在田,秋冬教閱。官免建砦而私不廢農。



 彭義斌使統領張士顯見範,請合謀討李全。範告於制置使趙善湘曰:「以義斌蹙全,如山壓卵;然必請而後討者,知有朝廷也。失此不
 右,而右兇徒,則權綱解紐矣。萬一義斌無朝命而成大勛,是又唐藩鎮之事,非計之得也。莫若移揚州增戍之兵往盱眙,而四總管兵各留半以備金人,餘皆起發,擇一能將統之,命葵摘淮西精銳萬人與會於楚州,出許浦海道,五十艘入淮,以斷賊歸路,密約義斌自北攻之,事無不濟。四總管權位相侔,劉琸雖能得其歡心,而不能制其死命。如用琸,須令親履行陣,指蹤四人,不可止坐籌帷幄也。」不報。



 範又曰:「國家討賊則自此中興,否則
 自此不振。若朝廷不欲張皇,則範乃提刑,職在捕盜,但令範以本路兵措置楚州鹽賊,範當調時青、張惠兩軍之半,及其船數百,徑薄楚城,以遏賊路,調夏全、範成進之半,據漣、海而守之,又移揚州之戍以戍盱眙。然得親提精銳雄勝、強勇等就時青於城外,示賊以形勢,諭賊以禍福,賊必自降。若猶拒守,則南北軍民雜處,必有內應者矣。別約義斌攻之於北,山陽下則進駐漣、海以應之,撫歸附家屬以離其黨,不出半月,此賊必亡。若是,則
 不調許浦水軍,但得趙葵三千人亦足矣。若朝廷憚費,則全有豫買軍需錢二十萬在真州,且漣、楚積聚,多自足用。」



 丞相史彌遠報範書,令諭四總管各享安靖之福。範所遣計議官聞之,曰:「但恐禍根轉深,不得安靖爾。」各揮涕而歸。會全且至,範又獻計曰:「撫機不發,事已無及。侯景困喪河南,致毒蕭氏;今逆全不得志於義斌,而復慮四總管應之,歸據舊巢,其謀必急。然蹙之於喪敗之餘者易,圖之於休息之後者難;矧四總管合謀章露,必
 難遂已。但事機既變,局面不同。若廟算果定,不欲出教令,但得密易指授,範一切伏藏不動,只約義斌,使自彼攻其所必救,則機會在我,而前日之策可用矣。」還報,戒範無出位專兵。



 範乃為書謝廟堂,且決之曰:「今上自一人,下至公卿百執事,又下至士民軍吏,無不知禍賊之必反。雖先生之心,亦自知其必反也。眾人知之則言之,先生知而獨不言,不言誠是也。內無臥薪嘗膽之志,外無戰勝攻取之備,先生隱忍不言而徐思所以制之,此
 廟謨所以為高也。然以撫定責之晞稷,而以鎮守責之範。責晞稷者函人之事也,責範者矢人之事也。既責範以惟恐不傷人之事,又禁其為傷人之痛,惡其為傷人之言,何哉?其禍賊見範為備,則必忌而不得以肆其奸,他日必將指範為首禍激變之人,劫朝廷以去範。先生始未之信也,左右曰可,卿大夫曰可,先生必將曰:『是何惜一趙範而不以紓禍哉?』必將縛範以授賊,而範遂為宋晁錯。雖然,使以範授賊而果足以紓國禍,範死何害
 哉?諺曰:『護家之狗,盜賊所惡。』故盜賊見有護家之狗,必將指斥於主人,使先去之,然後肆穿窬之奸而無所忌。然則殺犬固無益於弭盜也。欲望矜憐,別與閑慢差遣。」彌遠得書,為之動心。



 二年春,奉祠。三年,知安慶府,未行,改知池州,繼兼江東提舉常平。彌遠訪將材於葵,葵以範對。進範直敷文閣、淮東提點刑獄兼知滁州。範曰:「弟而薦兄,不順。」以母老辭。乃上書彌遠曰:「淮東之事,日異日新。然有淮則有江,無淮則長江以北,港水義蘆葦之處,
 敵人皆可潛師以濟,江面數千里,何從而防哉。今或謂巽辭厚惠可以啖賊,而不知陷彼款兵之計。或謂斂兵退屯可以緩賊,而不知成彼深入之謀。或欲行清野以嬰城,或欲聚烏合而浪戰,或以賊詞之乍順乍逆而為喜懼,或以賊兵之乍進乍退而為寬緊,皆失策也。失策則失淮,失淮則失江,而其失有不可勝諱者矣。夫有遏寇之兵,有游擊之兵,有討賊之兵。今寶應之逼山陽,天長之逼盱眙,須各增戍兵萬人,遣良將統之,賊來則堅
 壁以挫其鋒,不來則耀武以壓其境;而又觀釁伺隙,時遣偏師掩其不備,以示敢戰,使雖欲深入而畏吾之搗其虛,此遏寇之兵也。盱眙之寇,素無儲蓄,金人亦無以養之,不過分兵擄掠而食;當量出精兵,授以勇校,募土豪,出奇設伏以剿殺之,此游擊之兵也。惟揚、金陵、合肥,各聚二三萬人,人物必精,將校必勇,器械必利,教閱必熟,紀律必嚴,賞罰必公,其心術念慮必人人思親其上而死其長;信能行此,半年而可以強國,一年而可討賊
 矣。賊既不能深入,擄掠復無所獲,而又懷見討之恐,則必反而求贍於金;金無餘力及此,則必怨之怒之,吾於是可以嫁禍於金人矣。或謂揚州不可屯重兵,恐連賊禍,是不然。揚州者,國之北門,一以統淮,一以蔽江,一以守運河,豈可無備哉。善守者,敵不知所攻。今若設寶應、天長二屯以扼其沖,復重二三帥閫以張吾勢,賊將不知所攻,而敢犯我揚州哉?設使賊不知兵勢而犯揚州,是送死矣。」朝廷乃召範稟議,復令知池州。



 紹定元年,試
 將作監、知鎮江府。三年,丁母憂,求解官,不許。起復直徽猷閣、淮東安撫副使。尋轉右文殿修撰,賜章服金帶。不得已,卒哭復視事。又為書告廟堂:「請罷調停之議,一請檄沿江制置司,調王明本軍駐泰興港以扼泰州下江之捷徑;一請檄射陽湖人為兵,屯其半高郵以制賊後,屯其半瓜州以扼賊前;一請速調淮西兵合滁陽、六合諸軍圖救江面。不然,範雖死江皋無益也。」朝旨乃許範刺射陽湖兵毋過二萬人,就聽節制。



 範又遺善湘書,曰:「
 今日與宗社同休戚者,在內惟丞相,在外惟制使與範及範弟葵耳。賊若得志,此四家必無存理。」於是討賊之謀遂決,遂戮全。進範兵部侍郎、淮東安撫使兼知揚州兼江淮制置司參謀官,以次復淮東。加吏部侍郎,進工部尚書、沿江制置副使,權移司兼知黃州,尋兼淮西制置副使。未幾,為兩淮制置使、節制巡邊軍馬,仍兼沿江制置副使。



 又進端明殿學士,京河關陜宣撫使、知開封府、東京留守兼江、淮制置使。入洛之師大潰,乃授京湖
 安撫制置使兼知襄陽府。範至,則倚王旻、樊文彬、李伯淵、黃國弼數人為腹心,朝夕酣狎,了無上下之序。民訟邊防,一切廢馳。屬南北軍將交爭,範失於撫御。於是北軍王旻內叛,李伯淵繼之,焚襄陽北去;南軍大將李虎不救焚,不定變,乃因之劫掠。城中官民尚四萬七千有奇,錢糧在倉庫者無慮三十萬,弓矢器械二十有四庫,皆為敵有。蓋自岳飛收復百三十年,生聚繁庶,城高池深,甲於西陲,一旦灰燼,禍至慘也。言者劾範,降三官落
 職,依舊制置使。尋奉祠,以言罷;論者未已,再降兩官,送建寧府居住。嘉熙三年,敘復官職,與宮觀。四年,知靜江府,後卒於家。



 謝方叔,字德方,威州人。嘉定十六年進士,歷官監察御史。疏奏:「秉剛德以回上帝之心,奮威斷以回天下之勢,或者猶恐前習便嬖之人,有以私陛下之聽而悅陛下之心,則前日之畏者怠,憂者喜,慮者玩矣。左右前後之人,進憂危恐懼之言者,是納忠於上也;進燕安逸樂之
 言者,是不忠於上也。凡有水旱盜賊之奏者,必忠臣也;有諂諛蒙蔽之言者,必佞臣也。陛下享玉食珍羞之奉,當思兩淮流莩轉壑之可矜;聞管弦鐘鼓之聲,當思西蜀白骨如山之可念。」又言:「崇儉德以契天理,儲人才以供天職,恢遠略以需天討,行仁政以答天意。」帝悅。差知衡州,除宗正少卿,又除太常少卿兼國史編修、實錄檢討。



 時劉漢弼、杜範、徐元傑相繼死,方叔言:「元傑之死,陛下既為命官鞫獄,立賞捕奸,罪人未得,忠冤未伸。陛下
 茍不始終主持,將恐紀綱掃地,而國無以為國矣。」遷殿中侍御史,進對,言:「操存本於方寸,治亂系於天下。人主宅如法宮蠖濩之邃,朝夕親近者左右近習承意伺旨之徒,往往覘上之所好,不過保恩寵、希貨利而已。而冥冥之中,或有游揚之說,潛伏而莫之覺。防微杜漸,實以是心主之。」又言:「今日為兩淮謀者有五:一曰明間諜,二曰修馬政,三曰營山水砦,四曰經理近城之方田,五曰加重遏絕游騎及救奪擄掠之賞罰。」請行限田,請錄朱
 熹門人胡安定、呂燾、蔡模,詔皆從之。



 權刑部侍郎兼權給事中,升兼侍講,正授刑部侍郎,權國史編修、實錄檢討。拜端明殿學士、簽書樞密院事、參知政事。淳祐九年,拜參知政事,封永康郡侯。十一年,特授知樞密院事兼參知政事,尋拜左丞相兼樞密使,進封惠國公。勸帝以愛身育德。



 屬監察御史洪天錫論宦者盧允升、董宋臣,疏留中不下,大宗正寺丞趙崇璠移書方叔云:「閹寺驕恣特甚,宰執不聞正救,臺諫不敢誰何,一新入孤立之
 察官,乃銳意出身攻之,此豈易得哉?側耳數日,寂無所聞,公議不責備他人,而責備於宰相。不然,倉卒出御筆,某人授少卿,亦必無可遏之理矣,丞相不可謂非我責也。丞相得君最深,名位已極。儻言之勝,宗社賴之;言之不勝,則去。去則諸君必不容不爭,是勝亦勝,負亦勝,況未必去耶。」方叔得書,有赧色。



 翼日,果得御筆授天錫大理少卿,而天錫去國。於是太學生池元堅、太常寺丞趙崇潔、左史李昴英皆論擊允升、宋臣。而讒者又曰:「天錫
 之論,方叔意也。」及天錫之去,亦曰:「方叔意也。」方叔上疏自解,於是監察御史朱應元論方叔,罷相。既罷,允升、宋臣猶以為未快,厚賂太學生林自養,上書力詆天錫、方叔,且曰:「乞誅方叔,使天下明知宰相臺諫之去,出自獨斷,於內侍初無預焉。」書既上,學舍惡自養黨奸,相與鳴鼓攻之,上書以聲其罪。乃授方叔觀文殿大學士、提舉洞霄宮。復以監察御史李衢兩劾,褫職罷祠。後依舊職,與祠,起居郎召澤、中書舍人林存劾罷;監察御史章士
 元請更與降削,竄廣南。景定二年,請致仕,乃敘復官職。



 度宗即位,方叔以一琴、一鶴、金丹一粒來進。丞相賈似道恐其希望,諷權右司郎官盧越、左司諫趙順孫、給事中馮夢得、右正言黃鏞相繼請奪方叔官職封爵,制置使呂文德願以己官贖其罪。咸淳七年,詔敘復致仕。八年卒。特贈少師,方叔在相位,子弟干政,若讒餘玠之類是也。



 論曰:喬行簡弘深好賢,論事通諫。範鐘、游似同在相位,
 皆謹飭自將,而意見不侔。趙方豫計二子後當若何,而葵、範所立,皆如所言,所謂知子莫若父也。然宋自端平以來,捍禦淮、蜀兩邊者,非葵材館之士,即其偏裨之將。朝廷倚之,如長城之勢。及其筋力既老,而衛國之志不衰,亦曰壯哉!謝方叔相業無過人者,晚困於權臣,至以玩好丹劑為人主壽,坐是貶削,有愧金鏡多矣!



\end{pinyinscope}