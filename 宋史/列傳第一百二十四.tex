\article{列傳第一百二十四 岳飛子雲}

\begin{pinyinscope}

 岳飛子雲



 岳飛,字鵬舉,相州湯陰人。世力農。父和,能節食以濟饑者。有耕侵其地,割而與之;貰其財者不責償。飛生時,有大禽若鵠,飛鳴室上,因以為名。未彌月,河決內黃,水暴
 至,母姚抱飛坐甕中,沖濤及岸得免,人異之。



 少負氣節,沉厚寡言,家貧力學,尤好《左氏春秋》、孫吳兵法。生有神力,未冠,挽弓三百斤,弩八石,學射於周同,盡其術,能左右射。同死,朔望設祭於其塚。父義之,曰:「汝為時用,其徇國死義乎!」



 宣和四年,真定宣撫劉韐募敢戰士,飛應募。相有劇賊陶俊、賈進和,飛請百騎滅之。遣卒偽為商入賊境,賊掠以充部伍。飛遣百人伏山下,自領數十騎逼賊壘。賊出戰,飛陽北,賊來追之,伏兵起,先所遣卒擒俊
 及進和以歸。



 康王至相,飛因劉浩見,命招賊吉倩,倩以眾三百八十人降。補承信郎。以鐵騎三百往李固渡嘗敵,敗之。從浩解東京圍,與敵相持於滑南,領百騎習兵河上。敵猝至,飛麾其徒曰:「敵雖眾,未知吾虛實,當及其未定擊之。」乃獨馳迎敵。有梟將舞刀而前,飛斬之,敵大敗。遷秉義郎,隸留守宗澤。戰開德、曹州皆有功,澤大奇之,曰:「爾勇智才藝,古良將不能過,然好野戰,非萬全計。」因授以陣圖。飛曰:「陣而後戰,兵法之常,運用之妙,存乎
 一心。」澤是其言。



 康王即位,飛上書數千言,大略謂:「陛下已登大寶,社稷有主,已足伐敵之謀,而勤王之師日集,彼方謂吾素弱,宜乘其怠擊之。黃潛善、汪伯彥輩不能承聖意恢復,奉車駕日益南,恐不足系中原之望。臣願陛下乘敵穴未固,親率六軍北渡,則將士作氣,中原可復。」書聞,以越職奪官歸。



 詣河北招討使張所,所待以國士,借補修武郎,充中軍統領。所問曰:「汝能敵幾何?」飛曰:「勇不足恃,用兵在先定謀,欒枝曳柴以敗荊,莫敖採樵
 以致絞,皆謀定也。」所矍然曰:「君殆非行伍中人。」飛因說之曰:「國家都汴,恃河北以為固。茍馮據要沖,峙列重鎮,一城受圍,則諸城或撓或救,金人不能窺河南,而京師根本之地固矣。招撫誠能提兵壓境,飛唯命是從。」所大喜,借補武經郎。



 命從王彥渡河,至新鄉,金兵盛,彥不敢進。飛獨引所部鏖戰,奪其纛而舞,諸軍爭奮,遂拔新鄉。翌日,戰侯兆川,身被十餘創,士皆死戰,又敗之。夜屯石門山下,或傳金兵復至,一軍皆驚,飛堅臥不動,金兵卒
 不來。食盡,走彥壁乞糧,彥不許。飛引兵益北,戰於太行山,擒金將拓跋耶烏。居數日,復遇敵,飛單騎持丈八鐵槍,刺殺黑風大王,敵眾敗走。飛自知與彥有隙,復歸宗澤,為留守司統制。澤卒,杜充代之,飛居故職。



 二年,戰胙城,又戰黑龍潭,皆大捷。從閭勍保護陵寢,大戰汜水關,射殪金將,大破其眾。駐軍竹蘆渡,與敵相持,選精銳三百伏前山下,令各以薪芻交縛兩束,夜半,爇四端而舉之。金人疑援兵至,驚潰。



 三年,賊王善、曹成、孔彥舟等合
 眾五十萬,薄南熏門。飛所部僅八百,眾懼不敵,飛曰:「吾為諸君破之。」左挾弓,右運矛,橫沖其陣,賊亂,大敗之。又擒賊杜叔五、孫海於東明。借補英州刺史。王善圍陳州,飛戰於清河,擒其將孫勝、孫清,授真刺史。



 杜充將還建康,飛曰:「中原地尺寸不可棄,今一舉足,此地非我有,他日欲復取之,非數十萬眾不可。」充不聽,遂與俱歸。師次鐵路步,遇賊張用,至六合遇李成,與戰,皆敗之。成遣輕騎劫憲臣犒軍銀帛,飛進兵掩擊之,成奔江西。時命充
 守建康,金人與成合寇烏江,充閉門不出。飛泣諫請視師,充竟不出。金人遂由馬家渡渡江,充遣飛等迎戰,王□燮先遁,諸將皆潰,獨飛力戰。



 會充已降金,諸將多行剽掠,惟飛軍秋毫無所犯。兀朮趨杭州,飛要擊至廣德境中,六戰皆捷,擒其將王權,俘簽軍首領四十餘。察其可用者,結以恩遣還,令夜斫營縱火,飛乘亂縱擊,大敗之。駐軍鐘村,軍無見糧,將士忍饑,不敢擾民。金所籍兵相謂曰:「此岳爺爺軍。」爭來降附。



 四年,兀朮攻常州,宜興令
 迎飛移屯焉。盜郭吉聞飛來,遁入湖,飛遣王貴、傅慶追破之,又遣辯士馬皋、林聚盡降其眾。有張威武者不從,飛單騎入其營,斬之。避地者賴以免,圖飛像祠之。



 金人再攻常州,飛四戰皆捷;尾襲於鎮江東,又捷;戰於清水亭,又大捷,橫尸十五里。兀朮趨建康,飛設伏牛頭山待之。夜,令百人黑衣混金營中擾之,金兵驚,自相攻擊。兀朮次龍灣,飛以騎三百、步兵二千馳至新城,大破之。兀朮奔淮西,遂復建康。飛奏:「建康為要害之地,宜選兵固
 守,仍益兵守淮,拱護腹心。」帝嘉納。兀朮歸,飛邀擊於靜安,敗之。



 詔討戚方,飛以三千人營於苦嶺。方遁,俄益兵來,飛自領兵千人,戰數十合,皆捷。會張俊兵至,方遂降。範宗尹言張俊自浙西來,盛稱飛可用,遷通、泰鎮撫使兼知泰州。飛辭,乞淮南東路一重難任使,收復本路州郡,乘機漸進,使山東、河北、河東、京畿等路次第而復。



 會金攻楚急,詔張俊援之。俊辭,乃遣飛行,而命劉光世出兵援飛。飛屯三墩為楚援,尋抵承州,三戰三捷,殺高太
 保,俘酋長七十餘人。光世等皆不敢前,飛師孤力寡,楚遂陷。詔飛還守通、泰,有旨可守即守,如不可,但以沙洲保護百姓,伺便掩擊。飛以泰無險可恃,退保柴墟,戰於南霸橋,金大敗。渡百姓於沙上,飛以精騎二百殿,金兵不敢近。飛以泰州失守待罪。



 紹興元年,張俊請飛同討李成。時成將馬進犯洪州,連營西山。飛曰:「賊貪而不慮後,若以騎兵自上流絕生米渡,出其不意,破之必矣。」飛請自為先鋒,俊大喜。飛重鎧躍馬,潛出賊右,突其陣,所
 部從之。進大敗,走筠州。飛抵城東,賊出城,布陣十五里,飛設伏,以紅羅為幟,上刺「岳」字,選騎二百隨幟而前。賊易其少,薄之,伏發,賊敗走。飛使人呼曰:「不從賊者坐,吾不汝殺。」坐而降者八萬餘人。進以餘卒奔成於南康。飛夜引兵至朱家山,又斬其將趙萬。成聞進敗,自引兵十餘萬來。飛與遇於樓子莊,大破成軍,追斬進。成走蘄州,降偽齊。



 張用寇江西,用亦相人,飛以書諭之曰:「吾與汝同里,南熏門、鐵路步之戰,皆汝所悉。今吾在此,欲戰則
 出,不戰則降。」用得書曰:「果吾父也。」遂降。



 江、淮平,俊奏飛功第一,加神武右軍副統制,留洪州,彈壓盜賊,授親衛大夫、建州觀察使。建寇範汝為陷邵武,江西安撫李回檄飛分兵保建昌軍及撫州,飛遣人以「岳」字幟植城門,賊望見,相戒勿犯。賊黨姚達、饒青逼建昌,飛遣王萬、徐慶討擒之。升神武副軍都統制。



 二年,賊曹成擁眾十餘萬,由江公歷湖湘,據道、賀二州。命飛權知潭州,兼權荊湖東路安撫都總管,付金字牌、黃旗招成。成聞飛將至,
 驚曰:「岳家軍來矣。」即分道而遁。飛至茶陵,奉詔招之,成不從。飛奏:「比年多命招安,故盜力強則肆暴,力屈則就招,茍不略加剿除,蜂起之眾未可遽殄。」許之。



 飛入賀州境,得成諜者,縛之帳下。飛出帳調兵食,吏曰:「糧盡矣,奈何?」飛陽曰:「姑反茶陵。」已而顧諜若失意狀,頓足而入,陰令逸之。諜歸告成,成大喜,期翌日來追。飛命士蓐食,潛趨繞嶺,未明,已至太平場,破其砦。成據險拒飛,飛麾兵掩擊,賊大潰。成走據北藏嶺、上梧關,遣將迎戰,飛不陣
 而鼓,士爭奮,奪二隘據之。成又自桂嶺置砦至北藏嶺,連控隘道,親以眾十餘萬守蓬頭嶺。飛部才八千,一鼓登嶺,破其眾,成奔連州。飛謂張憲等曰:「成黨散去,追而殺之,則脅從者可憫,縱之則復聚為盜。今遣若等誅其酋而撫其眾,慎勿妄殺,累主上保民之仁。」於是憲自賀、連,徐慶自邵、道,王貴自郴、桂,招降者二萬,與飛會連州。進兵追成,成走宣撫司降。時以盛夏行師瘴地,撫循有方,士無一人死癘者,嶺表平。授武安軍承宣使,屯江州。
 甫入境,安撫李回檄飛捕劇賊馬友、郝通、劉忠、李通、李宗亮、張式,皆平之。



 三年春,召赴行在。江西宣諭劉大中奏:「飛兵有紀律,人恃以安,今赴行在,恐盜復起。」不果行。時虔、吉盜連兵寇掠循、梅、廣、惠、英、韶、南雄、南安、建昌、汀、邵武諸郡,帝乃專命飛平之。飛至虔州,固石洞賊彭友悉眾至雩都迎戰,躍馬馳突,飛麾兵即馬上擒之,餘酋退保固石洞。洞高峻環水,止一徑可入。飛列騎山下,令皆持滿,黎明,遣死士疾馳登山,賊眾亂,棄山而下,騎兵
 圍之。賊呼丐命,飛令勿殺,受其降。授徐慶等方略,捕諸郡餘賊,皆破降之。初,以隆祐震驚之故,密旨令飛屠虔城。飛請誅首惡而赦脅從,不許;請至三四,帝乃曲赦。人感其德,繪像祠之。餘寇高聚、張成犯袁州,飛遣王貴平之。



 秋,入見,帝手書「精忠岳飛」字,制旗以賜之。授鎮南軍承宣使、江南西路沿江制置使,又改神武後軍都統制,仍制置使,李山、吳全、吳錫、李橫、牛皋皆隸焉。



 偽齊遣李成挾金人入侵,破襄陽、唐、鄧、隨、郢諸州及信陽軍,湖寇
 楊麼亦與偽齊通,欲順流而下,李成又欲自江西陸行,趨兩浙與麼會。帝命飛為之備。



 四年,除兼荊南、鄂岳州制置使。飛奏:「襄陽等六郡為恢復中原基本,今當先取六郡,以除心膂之病。李成遠遁,然後加兵湖湘,以殄群盜。」帝以諭趙鼎,鼎曰:「知上流利害,無如飛者。」遂授黃復州、漢陽軍、德安府制置使。飛渡江中流,顧幕屬曰:「飛不擒賊,不涉此江。」抵郢州城下,偽將京超號「萬人敵」,乘城拒飛。飛鼓眾而登,超投崖死,復郢州,遣張憲、徐慶復隨
 州。飛趣襄陽,李成迎戰,左臨襄江,飛笑曰:「步兵利險阻,騎兵利平曠。成左列騎江岸,右列步平地,雖眾十萬何能為。」舉鞭指王貴曰:「爾以長槍步卒擊其騎兵。」指牛皋曰:「爾以騎兵擊其步卒。」合戰,馬應槍而斃,後騎皆擁入江,步卒死者無數,成夜遁,復襄陽。劉豫益成兵屯新野,飛與王萬夾擊之,連破其眾。



 飛奏:「金賊所愛惟子女金帛,志已驕惰;劉豫僭偽,人心終不忘宋。如以精兵二十萬,直搗中原,恢復故疆,誠易為力。襄陽、隨、郢地皆膏腴,
 茍行營田,其利為厚。臣候糧足,即過江北剿戮敵兵。」時方重深入之舉,而營田之議自是興矣。



 進兵鄧州,成與金將劉合孛堇列砦拒飛。飛遣王貴、張憲掩擊,賊眾大潰,劉合孛堇僅以身免。賊黨高仲退保鄧城,飛引兵一鼓拔之,擒高仲,復鄧州。帝聞之,喜曰:「朕素聞岳飛行軍有紀律,未知能破敵如此。」又復唐州、信陽軍。



 襄漢平,飛辭制置使,乞委重臣經畫荊襄,不許。趙鼎奏:「湖北鄂、岳最為上流要害,乞令飛屯鄂、岳,不惟江西藉其聲勢,湖、
 廣、江、浙亦獲安妥。」乃以隨、郢、唐、鄧、信陽並為襄陽府路隸飛,飛移屯鄂,授清遠軍節度使、湖北路、荊、襄、潭州制置使,封武昌縣開國子。



 兀朮、劉豫合兵圍廬州,帝手札命飛解圍,提兵趨廬,偽齊已驅甲騎五千逼城。飛張「岳」字旗與「精忠」旗,金兵一戰而潰,廬州平。飛奏:「襄陽等六郡人戶闕牛、糧,乞量給官錢,免官私逋負,州縣官以招集流亡為殿最。」



 五年,入覲,封母國夫人;授飛鎮寧、崇信軍節度使,湖北路、荊襄潭州制置使,進封武昌郡開國
 侯;又除荊湖南北、襄陽路制置使,神武後軍都統制,命招捕楊麼。飛所部皆西北人,不習水戰,飛曰:「兵何常,顧用之何如耳。」先遣使招諭之。賊黨黃佐曰:「岳節使號令如山,若與之敵,萬無生理,不如往降。節使誠信,必善遇我。」遂降。飛表授佐武義大夫,單騎按其部,拊佐背曰:「子知逆順者。果能立功,封侯豈足道?欲復遣子至湖中,視其可乘者擒之,可勸者招之,如何?」佐感泣,誓以死報。



 時張浚以都督軍事至潭,參政席益與浚語,疑飛玩寇,欲
 以聞。浚曰:「岳侯,忠孝人也,兵有深機,胡可易言?」益慚而止。黃佐襲周倫砦,殺倫,擒其統制陳貴等。飛上其功,遷武經大夫。統制任士安不稟王□燮令,軍以此無功。飛鞭士安使餌賊,曰:「三日賊不平,斬汝。」士安宣言:「岳太尉兵二十萬至矣。」賊見止士安軍,並力攻之。飛設伏,士安戰急,伏四起擊賊,賊走。



 會召浚還防秋,飛袖小圖標浚,浚欲俟來年議之。飛曰:「已有定畫,都督能少留,不八日可破賊。」浚曰:「何言之易?」飛曰:「王四廂以王師攻水寇則難,
 飛以水寇攻水寇則易。水戰我短彼長,以所短攻所長,所以難。若因敵將用敵兵,奪其手足之助,離其腹心之托,使孤立,而後以王師乘之,八日之內,當俘諸酋。」浚許之。



 飛遂如鼎州。黃佐招楊欽來降,飛喜曰:「楊欽驍悍,既降,賊腹心潰矣。」表授欽武義大夫,禮遇甚厚,乃復遣歸湖中。兩日,欽說餘端、劉詵等降,飛詭罵欽曰:「賊不盡降,何來也?」杖之,復令入湖。是夜,掩賊營,降其眾數萬。麼負固不服,方浮舟湖中,以輪激水,其行如飛,旁置撞竿,官
 舟迎之輒碎。飛伐君山木為巨筏,塞諸港水義,又以腐木亂草浮上流而下,擇水淺處,遣善罵者挑之,且行且罵。賊怒來追,則草木壅積,舟輪礙不行。飛亟遣兵擊之,賊奔港中,為筏所拒。官軍乘筏,張牛革以蔽矢石,舉巨木撞其舟,盡壞。麼投水,牛皋擒斬之。飛入賊壘,餘酋驚曰:「何神也!」俱降。飛親行諸砦慰撫之,縱老弱歸田,籍少壯為軍,果八日而賊平。浚嘆曰:「岳侯神算也。」初,賊恃其險曰:「欲犯我者,除是飛來。」至是,人以其言為讖。獲賊舟千
 餘,鄂渚水軍為沿江之冠。詔兼蘄、黃制置使,飛以目疾乞辭軍事,不許,加檢校少保,進封公。還軍鄂州,除荊湖南北、襄陽路招討使。



 六年,太行山忠義社梁興等百餘人,慕飛義率眾來歸。飛入覲,面陳:「襄陽自收復後,未置監司,州縣無以按察。」帝從之,以李若虛為京西南路提舉兼轉運、提刑,又令湖北、襄陽府路自知州、通判以下賢否,許飛得自黜陟。



 張浚至江上會諸大帥,獨稱飛與韓世忠可倚大事,命飛屯襄陽,以窺中原,曰:「此君素志
 也。」飛移軍京西,改武勝、定國軍節度使,除宣撫副使,置司襄陽。命往武昌調軍。居母憂,降制起復,飛扶櫬還廬山,連表乞終喪,不許,累詔趣起,乃就軍。又命宣撫河東,節制河北路。首遣王貴等攻虢州,下之,獲糧十五萬石,降其眾數萬。張浚曰:「飛措畫甚大,令已至伊、洛,則太行一帶山砦,必有應者。」飛遣楊再興進兵至長水縣,再戰皆捷,中原響應。又遣人焚蔡州糧。



 九月,劉豫遣子麟、侄猊分道寇淮西,劉光世欲舍廬州,張俊欲棄盱眙,同奏召
 飛以兵東下,欲使飛當其鋒,而己得退保。張浚謂:「岳飛一動,則襄漢何所制?」力沮其議。帝慮俊、光世不足任,命飛東下。飛自破曹成、平楊麼,凡六年,皆盛夏行師,致目疾,至是,甚;聞詔即日啟行,未至,麟敗。飛奏至,帝語趙鼎曰:「劉麟敗北不足喜,諸將知尊朝廷為可喜。」遂賜札,言:「敵兵已去淮,卿不須進發,其或襄、鄧、陳、蔡有機可乘,從長措置。」飛乃還軍。時偽齊屯兵窺唐州,飛遣王貴、董先等攻破之,焚其營。奏圖蔡以取中原,不許。飛召貴等還。



 七年,入見,帝從容問曰:「卿得良馬否?」飛曰:「臣有二馬,日啖芻豆數斗,飲泉一斛,然非精潔則不受。介而馳,初不甚疾,比行百里始奮迅,自午至酉,猶可二百里。褫鞍甲而不息不汗,若無事然。此其受大而不茍取,力裕而不求逞,致遠之材也。不幸相繼以死。今所乘者,日不過數升,而秣不擇粟,飲不擇泉,攬轡未安,踴踴疾驅,甫百里,力竭汗喘,殆欲斃然。此其寡取易盈,好逞易窮,駑鈍之材也。」帝稱善,曰:「卿今議論極進。」拜太尉,繼除宣撫使兼
 營田大使。從幸建康,以王德、酈瓊兵隸飛,詔諭德等曰:「聽飛號令,如朕親行。」



 飛數見帝,論恢復之略。又手疏言:「金人所以立劉豫於河南,蓋欲荼毒中原,以中國攻中國,粘罕因得休兵觀釁。臣欲陛下假臣月日,便則提兵趨京、洛,據河陽、陜府、潼關,以號召五路叛將。叛將既還,遣王師前進,彼必棄汴而走河北,京畿、陜右可以盡復。然後分兵浚、滑,經略兩河,如此則劉豫成擒,金人可滅,社稷長久之計,實在此舉。」帝答曰:「有臣如此,顧復何憂,
 進止之機,朕不中制。」又召至寢閣命之曰:「中興之事,一以委卿。」命節制光州。



 飛方圖大舉,會秦檜主和,遂不以德、瓊兵隸飛。詔詣都督府與張浚議事,浚謂飛曰:「王德淮西軍所服,浚欲以為都統,而命呂祉以督府參謀領之,如何?」飛曰:「德與瓊素不相下,一旦揠之在上,則必爭。呂尚書不習軍旅,恐不足服眾。」浚曰:「張宣撫如何?」飛曰:「暴而寡謀,尤瓊所不服。」浚曰:「然則楊沂中爾?」飛曰:「沂中視德等爾,豈能馭此軍?」浚艴然曰:「浚固知非太尉不可。」
 飛曰:「都督以正問飛,不敢不盡其愚,豈以得兵為念耶?」即日上章乞解兵柄,終喪服,以張憲攝軍事,步歸,廬母墓側。浚怒,奏以張宗元為宣撫判官,監其軍。



 帝累詔趣飛還職,飛力辭,詔幕屬造廬以死請,凡六日,飛趨朝待罪,帝尉遣之。宗元還言:「將和士銳,人懷忠孝,皆飛訓養所致。」帝大悅。飛奏:「比者寢閣之命,咸謂聖斷已堅,何至今尚未決?臣願提兵進討,順天道,固人心,以曲直為老壯,以逆順為強弱,萬全之效可必。」又奏:「錢塘僻在海隅,
 非用武地。願陛下建都上游,用漢光武故事,親率六軍,往來督戰。庶將士知聖意所向,人人用命。」未報而酈瓊叛,浚始悔。飛復奏:「願進屯淮甸,伺便擊瓊,期於破滅。」不許,詔駐師江州為淮、浙援。



 飛知劉豫結粘罕,而兀朮惡劉豫,可以間而動。會軍中得兀朮諜者,飛陽責之曰:「汝非吾軍中人張斌耶?吾向遣汝至齊,約誘至四太子,汝往不復來。吾繼遣人問,齊已許我,今冬以會合寇江為名,致四太子於清河。汝所持書竟不至,何背我耶?」諜冀
 緩死,即詭服。乃作蠟書,言與劉豫同謀誅兀朮事,因謂諜曰:「吾今貸汝。」復遣至齊,問舉兵期,刲股納書,戒勿洩。諜歸,以書示兀朮,兀朮大驚,馳白其主,遂廢豫。飛奏:「宜乘廢豫之際,搗其不備,長驅以取中原。」不報。



 八年,還軍鄂州。王庶視師江、淮,飛與庶書:「今歲若不舉兵,當納節請閑。」庶甚壯之。秋,召赴行在,命詣資善堂見皇太子。飛退而喜曰:「社稷得人矣,中興基業,其在是乎?」會金遣使將歸河南地,飛言:「金人不可信,和好不可恃,相臣謀國
 不臧,恐貽後世譏」檜銜之。



 九年,以復河南,大赦。飛表謝,寓和議不便之意,有「唾手燕雲,復仇報國」之語。授開府儀同三司,飛力辭,謂:「今日之事,可危而不可安;可憂而不可賀;可訓兵飭士,謹備不虞,而不可論功行賞,取笑敵人。」三詔不受,帝溫言獎諭,乃受。會遣士人褭謁諸陵,飛請以輕騎從灑埽,實欲觀釁以伐謀。又奏:「金人無事請和,此必有肘腋之虞,名以地歸我,實寄之也。」檜白帝止其行。



 十年,金人攻拱、亳,劉錡告急,命飛馳援,飛遣張憲、
 姚政赴之。帝賜札曰:「設施之方,一以委卿,朕不遙度。」飛乃遣王貴、牛皋、董先、楊再興、孟邦傑、李寶等,分布經略西京、汝、鄭、穎昌、陳、曹、光、蔡諸郡;又命梁興渡河,糾合忠義社,取河東、北州縣。又遣兵東援劉錡,西援郭浩,自以其軍長驅以闞中原。將發,密奏言:「先正國本以安人心,然後不常厥居,以示無忘復仇之意。」帝得奏,大褒其忠,授少保,河南府路、陜西、河東北路招討使,尋改河南、北諸路招討使。未幾,所遣諸將相繼奏捷。大軍在穎昌,諸
 將分道出戰,飛自以輕騎駐郾城,兵勢甚銳。



 兀朮大懼,會龍虎大王議,以為諸帥易與,獨飛不可當,欲誘致其師,並力一戰。中外聞之,大懼,詔飛審處自固。飛曰:「金人伎窮矣。」乃日出挑戰,且罵之。兀朮怒,合龍虎大王、蓋天大王與韓常之兵逼郾城。飛遣子雲領騎兵直貫其陣,戒之曰:「不勝,先斬汝!」鏖戰數十合,賊尸布野。



 初,兀朮有勁軍,皆重鎧,貫以韋索,三人為聯,號「拐子馬」,官軍不能當。是役也,以萬五千騎來,飛戒步卒以麻札刀入陣,勿
 仰視,第斫馬足。拐子馬相連,一馬僕,二馬不能行,官軍奮擊,遂大敗之。兀朮大慟曰:「自海上起兵,皆以此勝,今已矣!」兀朮益兵來,部將王剛以五十騎覘敵,遇之,奮斬其將。飛時出視戰地,望見黃塵蔽天,自以四十騎突戰,敗之。



 方郾城再捷,飛謂雲曰:「賊屢敗,必還攻穎昌,汝宜速援王貴。」既而兀朮果至,貴將游奕、雲將背嵬戰於城西。雲以騎兵八百挺前決戰,步軍張左右翼繼之,殺兀朮婿夏金吾、副統軍粘罕索孛堇,兀朮遁去。



 梁興會太
 行忠義及兩河豪傑等,累戰皆捷,中原大震。飛奏:「興等過河,人心願歸朝廷。金兵累敗,兀朮等皆令老少北去,正中興之機。」飛進軍朱仙鎮,距汴京四十五里,與兀朮對壘而陣,遣驍將以背嵬騎五百奮擊,大破之,兀朮遁還汴京。飛檄陵臺令行視諸陵,葺治之。



 先是,紹興五年,飛遣梁興等布德意,招結兩河豪傑,山砦韋銓、孫謀等斂兵固堡,以待王師,李通、胡清、李寶、李興、張恩、孫琪等舉眾來歸。金人動息,山川險要,一時皆得其實。盡磁、相、
 開德、澤、潞、晉、絳、汾、隰之境,皆期日興兵,與官軍會。其所揭旗以「岳」為號,父老百姓爭挽車牽牛,載糗糧以饋義軍,頂盆焚香迎候者,充滿道路。自燕以南,金號令不行,兀朮欲簽軍以抗飛,河北無一人從者。乃嘆曰:「自我起北方以來,未有如今日之挫衄。」金帥烏陵思謀素號桀黠,亦不能制其下,但諭之曰:「毋輕動,俟岳家軍來即降。」金統制王鎮、統領崔慶、將官李覬崔虎華旺等皆率所部降,以至禁衛龍虎大王下忔查千戶高勇之屬,皆密
 受飛旗榜,自北方來降。金將軍韓常欲以五萬眾內附。飛大喜,語其下曰:「直抵黃龍府,與諸君痛飲爾!」



 方指日渡河,而檜欲畫淮以北棄之,風臺臣請班師。飛奏:「金人銳氣沮喪,盡棄輜重,疾走渡河,豪傑向風,士卒用命,時不再來,機難輕失。」檜知飛志銳不可回,乃先請張俊、楊沂中等歸,而後言飛孤軍不可久留,乞令班師。一日奉十二金字牌,飛憤惋泣下,東向再拜曰:「十年之力,廢於一旦。」飛班師,民遮馬慟哭,訴曰:「我等戴香盆、運糧草以
 迎官軍,金人悉知之。相公去,我輩無□類矣。」飛亦悲泣,取詔示之曰:「吾不得擅留。」哭聲震野,飛留五日以待其徙,從而南者如市,亟奏以漢上六郡閑田處之。



 方兀朮棄汴去,有書生叩馬曰:「太子毋走,岳少保且退矣。」兀朮曰:「岳少保以五百騎破吾十萬,京城日夜望其來,何謂可守?」生曰:「自古未有權臣在內,而大將能立功於外者,岳少保且不免,況欲成功乎?」兀朮悟,遂留。飛既歸,所得州縣,旋復失之。飛力請解兵柄,不許,自廬入覲,帝問之,
 飛拜謝而已。



 十一年,諜報金分道渡淮,飛請合諸帥之兵破敵。兀朮、韓常與龍虎大王疾驅至廬,帝趣飛應援,凡十七札。飛策金人舉國南來,巢穴必虛,若長驅京、洛以搗之,彼必奔命,可坐而敝。時飛方苦寒嗽,力疾而行。又恐帝急於退敵,乃奏:「臣如搗虛,勢必得利,若以為敵方在近,未暇遠圖,欲乞親至蘄、黃,以議攻卻。」帝得奏大喜,賜札曰:「卿苦寒疾,乃為朕行,國爾忘身,誰如卿者?」師至廬州,金兵望風而遁。飛還兵於舒以俟命,帝又賜札,
 以飛小心恭謹、不專進退為得體。兀朮破濠州,張俊駐軍黃連鎮,不敢進;楊沂中遇伏而敗,帝命飛救之。金人聞飛至,又遁。



 時和議既決,檜患飛異己,乃密奏召三大將論功行賞。韓世忠、張俊已至,飛獨後,檜又用參政王次翁計,俟之六七日。既至,授樞密副使,位參知政事上,飛固請還兵柄。五月,詔同俊往楚州措置邊防,總韓世忠軍還駐鎮江。



 初,飛在諸將中年最少,以列校拔起,累立顯功,世忠、俊不能平,飛屈己下之,幕中輕銳教飛勿
 苦降意。金人攻淮西,俊分地也,俊始不敢行,師卒無功。飛聞命即行,遂解廬州圍,帝授飛兩鎮節,俊益恥。楊麼平,飛獻俊、世忠樓船各一,兵械畢備,世忠大悅,俊反忌之。淮西之役,俊以前途糧乏訹飛,飛不為止,帝賜札褒諭,有曰:「轉餉艱阻,卿不復顧。」俊疑飛漏言,還朝,反倡言飛逗遛不進,以乏餉為辭。至視世忠軍,俊知世忠忤檜,欲與飛分其背嵬軍,飛議不肯,俊大不悅。及同行楚州城,俊欲修城為備,飛曰:「當戮力以圖恢復,豈可為退保
 計?」俊變色。



 會世忠軍吏景著與總領胡紡言:「二樞密若分世忠軍,恐至生事。」紡上之朝,檜捕著下大理寺,將以扇搖誣世忠。飛馳書告以檜意,世忠見帝自明。俊於是大憾飛,遂倡言飛議棄山陽,且密以飛報世忠事告檜,檜大怒。



 初,檜逐趙鼎,飛每對客嘆息,又以恢復為己任,不肯附和議。讀檜奏,至「德無常師,主善為師」之語,惡其欺罔,恚曰:「君臣大倫,根於天性,大臣而忍面謾其主耶!」兀朮遺檜書曰:「汝朝夕以和請,而岳飛方為河北圖,必
 殺飛,始可和。」檜亦以飛不死,終梗和議,己必及禍,故力謀殺之。以諫議大夫萬俟離與飛有怨,風離劾飛,又風中丞何鑄、侍御史羅汝楫交章彈論,大率謂:「今春金人攻淮西,飛略至舒、蘄而不進,比與俊按兵淮上,又欲棄山陽而不守。」飛累章請罷樞柄,尋還兩鎮節,充萬壽觀使、奉朝請。檜志未伸也,又諭張俊令劫王貴、誘王俊誣告張憲謀還飛兵。



 檜遣使捕飛父子證張憲事,使者至,飛笑曰:「皇天后土,可表此心。」初命何鑄鞠之,飛裂裳以
 背示鑄,有「盡忠報國」四大字,深入膚理。既而閱實無左驗,鑄明其無辜。改命萬俟離。離誣:飛與憲書,令虛申探報以動朝廷,雲與憲書,令措置使飛還軍;且言其書已焚。



 飛坐系兩月,無可證者。或教離以臺章所指淮西事為言,離喜白檜,簿錄飛家,取當時御札藏之以滅跡。又逼孫革等證飛受詔逗遛,命評事元龜年取行軍時日雜定之,傅會其獄。歲暮,獄不成,檜手書小紙付獄,即報飛死,時年三十九。雲棄市。籍家貲,徙家嶺南。幕屬於鵬等
 從坐者六人。



 初,飛在獄,大理寺丞李若樸何彥猷、大理卿薛仁輔並言飛無罪,離俱劾去。宗正卿士人褭請以百口保飛,離亦劾之,竄死建州。布衣劉允升上書訟飛冤,下棘寺以死。凡傅成其獄者,皆遷轉有差。



 獄之將上也,韓世忠不平,詣檜詰其實,檜曰:「飛子雲與張憲書雖不明,其事體莫須有。」世忠曰:「『莫須有』三字,何以服天下?」時洪皓在金國中,蠟書馳奏,以為金人所畏服者惟飛,至以父呼之,諸酋聞其死,酌酒相賀。



 飛至孝,母留河北,遣
 人求訪,迎歸。母有痼疾,藥餌必親。母卒,水漿不入口者三日。家無姬侍。吳玠素服飛,願與交歡,飾名姝遺之。飛曰:「主上宵旰,豈大將安樂時?」卻不受,玠益敬服。少豪飲,帝戒之曰:「卿異時到河朔,乃可飲。」遂絕不飲。帝初為飛營第,飛辭曰:「敵未滅,何以家為?」或問天下何時太平,飛曰:「文臣不愛錢,武臣不惜死,天下太平矣。」



 師每休舍,課將士注坡跳壕,皆重鎧習之。子雲嘗習注坡,馬躓,怒而鞭之。卒有取民麻一縷以束芻者,立斬以徇。卒夜宿,民開
 門願納,無敢入者。軍號「凍死不拆屋,餓死不鹵掠。」卒有疾,躬為調藥;諸將遠戍,遣妻問勞其家;死事者哭之而育其孤,或以子婚其女。凡有頒犒,均給軍吏,秋毫不私。



 善以少擊眾。欲有所舉,盡召諸統制與謀,謀定而後戰,故有勝無敗。猝遇敵不動,故敵為之語曰:「撼山易,撼岳家軍難。」張俊嘗問用兵之術,曰:「仁、智、信、勇、嚴,闕一不可。」調軍食,必蹙額曰:「東南民力,耗敝極矣。」荊湖平,募民營田,又為屯田,歲省漕運之半。帝手書曹操、諸葛亮、羊祜
 三事賜之。飛跋其後,獨指操為奸賊而鄙之,尤檜所惡也。



 張所死,飛感舊恩,鞠其子宗本,奏以官。李寶自楚來歸,韓世忠留之,寶痛哭願歸飛,世忠以書來諗,飛復曰:「均為國家,何分彼此?」世忠嘆服。襄陽之役,詔光世為援,六郡既復,光世始至,飛奏先賞光世軍。好賢禮士,覽經史,雅歌投壺,恂恂如書生。每辭官,必曰:「將士效力,飛何功之有?」然忠憤激烈,議論持正,不挫於人,卒以此得禍。



 檜死,議復飛官。萬俟離謂金方願和,一旦錄故將,疑天
 下心,不可。及紹興末,金益猖獗,太學生程宏圖上書訟飛冤,詔飛家自便。初,檜惡岳州同飛姓,改為純州,至是仍舊。中丞汪澈宣撫荊、襄,故部曲合辭訟之,哭聲雷震。孝宗詔復飛官,以禮改葬,賜錢百萬,求其後悉官之。建廟於鄂,號忠烈。淳熙六年,謚武穆。嘉定四年,追封鄂王。



 五子:雲、雷、霖、震、霆。



 雲,飛養子。年十二,從張憲戰,多得其力,軍中呼曰「贏官人」。飛征伐,未嘗不與,數立奇功,飛輒隱之。每戰,以手握
 兩鐵椎,重八十斤,先諸軍登城。攻下隨州,又攻破鄧州,襄漢平,功在第一,飛不言。逾年,銓曹辯之,始遷武翼郎、楊麼平,功亦第一,又不上。張浚廉得其實,曰:「岳侯避寵榮,廉則廉矣,未得為公也。」奏乞推異數,飛力辭不受。嘗以特旨遷三資,飛辭曰:「士卒冒矢石立奇功,始升一級,男雲遽躐崇資,何以服眾?」累表不受。穎昌大戰,無慮十數,出入行陣,體被百餘創,甲裳為赤。以功遷忠州防禦使,飛又辭;命帶御器械,飛又力辭之。終左武大夫、提舉
 醴泉觀。死年二十三。孝宗初,與飛同復元官,以禮祔葬,贈安遠軍承宣使。



 雷,忠訓郎、閣門祗候,贈武略郎。霖,朝散大夫、敷文閣待制,贈太中大夫。初,飛下獄,檜令親黨王會搜其家,得御札數篋,束之左藏南庫,霖請於孝宗,還之。霖子珂,以淮西十五御札辯驗匯次,凡出師應援之先後皆可考。嘉定間,為《籲天辯誣集》五卷、《天定錄》二卷上之。震,朝奉大夫、提舉江南東路茶鹽公事。霆,修武郎、閣門祗候。



 論曰:西漢而下,若韓、彭、絳、灌之為將,代不乏人,求其文武全器、仁智並施如宋岳飛者,一代豈多見哉。史稱關雲長通《春秋左氏》學,然未嘗見其文章。飛北伐,軍至汴梁之朱仙鎮,有詔班師,飛自為表答詔,忠義之言,流出肺腑,真有諸葛孔明之風,而卒死於秦檜之手。蓋飛與檜勢不兩立,使飛得志,則金仇可復,宋恥可雪;檜得志,則飛有死而已。昔劉宋殺檀道濟,道濟下獄,嗔目曰:「自壞汝萬里長城!」高宗忍自棄其中原,故忍殺飛,嗚呼冤
 哉!嗚呼冤哉!



\end{pinyinscope}