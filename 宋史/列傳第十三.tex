\article{列傳第十三}

\begin{pinyinscope}

 侯益子仁矩仁寶孫延廣張從恩扈彥珂薛懷讓趙贊李繼勛藥元福趙晁子延溥



 侯益,汾州平遙人。祖父以農為業。唐光化中,李克用據
 太原,益以拳勇隸麾下。從莊宗攻大名,先登,擒軍校,擢為馬前直副兵馬使。征劉守光,先登,遷軍使。破洺州,為機石傷足,莊宗親以藥傅其瘡。及愈,改護衛指揮使。梁小將李立、李建以驍勇聞,軍中憚之。會莊宗與梁人戰河上,益挺身出鬥,擒其二將,遷馬前直指揮使。莊宗入汴,為本直副都校。從明宗討趙在禮於鄴。會諸軍推戴明宗,益脫身歸洛,莊宗撫其背出涕。



 明宗立,益面縛請罪,明宗曰:「爾盡忠節,又何罪也。」改本直左廂都校。天成
 初,朱守殷據夷門叛,益率所部斬關先入。轉左右馬前從馬直都校、領潘州刺史。王都據定州叛,益從王晏球攻討。會契丹來援,益逆擊之,破其眾唐河北,克其城,授寧州刺史。入為羽林軍五十指揮都校、領費州刺史。



 時夏帥李仁福卒,子彞超擅命自立,以邀節鋮,命益帥師討之。明宗不豫,遽追還。



 應順初,潞王舉兵鳳翔,以益為西面行營都虞侯。益知軍情必變,稱疾不奉詔,執政怒,出為商州刺史。蜀軍寇金州,益率鎮兵襲擊,大破這。詔
 賜襲衣、名馬,加西面行營都巡檢使。



 晉初,召為奉國都校、領光州防禦使。範延光反大名,張從賓據河陽為聲援。晉祖召益謂曰:「宗社危若綴旒,卿能為朕死耶?」益曰:「願假銳卒五千人,破賊必矣。」以益為西面得營副都部署,率禁兵數千人,次虎牢。從賓軍萬餘人,夾汜水而陣。益親鼓,士乘之,大敗其眾,擊殺殆盡,汜水為之不流,從賓乘馬入水溺死。築京觀,刻石紀功,晉祖大喜,拜河陽三城節度,充鄴都行營都虞候。會延光以城降,移鎮潞
 州。



 天福四年,晉祖追念虎牢之功,遷武寧軍節度、同平章事,遣中使謂益曰:「朕思卿前年七月九日大立戰功,故復以此月此日徙卿鎮彭門,領相印。」仍賜門戟,改鄉里為將相鄉勛賢里,九月,徐州大火,益出金、粟振之。



 明年,徙鎮秦州,充西面都部署。階州義軍校王君懷苦其刺史暴虐,率眾數千投蜀,請為先鋒下秦、成諸州。益聞之懼,請援於朝;又潛遺書於蜀將,以達誠意。少帝聞之,疑為邊患,議徙於內地。會蒲帥安審琦移鎮許下,以益
 為河中尹、護國軍節度。



 契丹入汴,益率僚屬歸京師,詣契丹主,自陳不預北伐之謀。契丹授以鳳翔節度。



 漢祖即位,加兼侍中。益自以嘗受契丹命,聞漢兵入洛,憂之,浚城隍為備,孟昶遣益所親掌樞密王處回繼書招益,復遣綿州刺史吳崇惲厚遺之。崇惲本秦州押衙,益故吏也。及何重建為帥遣崇惲奉表以階、秦歸蜀,授刺史,故昶遣之。益遂與其子歸蜀,昶令重建率川兵數萬出大散關以應之。漢祖知其事,遣客省使王景崇率禁軍數
 千,倍道趨岐下,召益入朝。時漢祖已不豫,召至臥內,謂之曰:「侯益貌順朝廷,心懷攜貳。爾往至彼,如益來,即置勿問;茍遲疑不決,即以便宜從事。」景崇至京兆,合岐、雍、邠、涇之師以破蜀軍。益懼,即謀入朝。



 會聞漢祖崩,景崇欲誅益,慮隱帝不知先朝密旨。從事程渥,景崇里人也。益因遣之說景崇曰:「君致位通顯,亦可少知止足,何必懷禍人之心,為已甚之事乎。況侯君親戚爪牙甚眾,事若妄發,禍亦旋踵至矣。」景崇怒曰:「子去,勿為游說,吾將
 族爾。」益知不用渥言,即率數十騎奔入朝。隱帝遣侍臣問益結連蜀軍之由,益對曰:「臣欲誘之出關,掩殺這耳。」隱帝笑之。益厚賂史弘肇輩,言景崇之橫恣。諸權貴深庇護之,乃授以開封尹兼中書令。俄封魯國公。景崇聞之,遂據城叛,益親屬在城中餘七十口悉為景崇所害。



 及周祖起兵,隱帝議出師御之,益獻計曰:「王者無敵於天下,兵不宜輕出,況大名戍卒家屬盡在京城,不如閉關以挫其銳,遣其母妻發降以招之,可不戰而定。」慕容
 彥超以為益衰老,作懦夫計,沮之。隱帝遣益與彥超及張彥超、閻晉卿,吳虔裕守澶州。至赤罔,周師奄至,戰留子陂,漢軍不利。益臨陣,見士卒無鬥志,又占候不祥,乃與焦繼勛等夜謁周祖,周祖慰勞遣還。



 廣順初,封楚國公,改太子太師,俄又改封齊國公。顯德元年冬,告老,以本官致仕歸洛。遣使賜茶藥錢帛,就撫問之。



 太祖即位,遣賜器幣,歲一來朝,及祖以耆舊厚待之。乾德初,郊祀,詔綴中書門下班,禮與丞相等。三年,卒,年八十,贈中書
 令。



 五子:仁願、仁矩、仁寶、仁遇、仁興。仁願至左金吾衛大將軍、蓬州刺史。仁遇,西京內園使。仁興,右屯衛將軍。仁願子延濟,西京作坊使、康州刺史。



 仁矩從益為商州牙校。益之討張從賓也,仁矩首犯賊鋒,以功領蓬州刺史,充河南牙職。從益歷潞、徐、秦三鎮。開運初,入為氈毯使,出為天平行軍司馬。



 漢妝,授隰州刺史,至郡決滯訟,一日釋擊囚百餘,獄為之空,民情悅服。仕周,歷左羽林將軍,出為泗州刺史,改通州,兼屯田
 鹽鐵監使。



 宋初,歷祁、雄二州刺史。治軍有方略,歷數郡,咸有善政。開寶二年,卒年五十六。太祖甚惜之,特命中使護喪。子延廣、延之、咸平二年進士及弟。



 仁寶以蔭遷太子中允,即趙普妹婿。盧多遜與普有隙,普罷相,即以仁寶知邕州。州之右江生毒藥樹,宣化縣人常採貨之。仁寶以聞,詔盡伐去。九年不代。太平興國中,上言陳取交州之策,太宗大喜,令馳驛召歸。多遜遽奏曰:「若召仁寶,其謀必洩,蠻夷增備,未易取也。不如授
 仁寶飛挽之任,且經度之,別遣偏將發荊湖士卒一二萬人,長驅而往,勢必萬全。」帝以為然。遂以仁寶為交州水陸計度轉運使。前軍發,遇賊鋒甚盛,援兵不繼,遇害死江中。太宗聞之,甚悼惜,特贈工部侍郎,錄其子延齡、延世並為齋郎。延齡至殿中丞。延世至太子中舍。



 延廣,初在襁褓中,遭王景崇之難,乳母劉氏以己子代延廣死。劉氏行丐抱持延廣至京師,還益。延廣父歷通、祁、雄三州刺史,悉以補牙職。仁矩在雄州日,方飲宴,虜
 數十騎白晝入州城,居民驚擾。延廣引親信數騎馳出衙門,射殺其酋長一人,斬首數級,悉禽其餘黨。延廣持首級以獻,仁矩喜,拊其背曰:「興吾門者必汝也。」監軍李漢超以其事聞,詔書褒美,賜錦袍銀帶。



 仁矩卒,補西頭供奉官。從黨進討太原。太平興國初,預修永昌陵,出護延州軍兼緣邊巡檢,善撫士卒,下樂為用,戎人畏服,遷閣門祗候。會西北戎入寇,邊人擾亂,求可使徼巡者。近臣言:「延廣將家子,習邊事無出其右。」延廣時被病,強起
 之,遷崇儀副使,充同、鄜、坊、延、丹緣邊都巡檢使。延廣力疾入辭,太宗賜以各藥及方,遣太醫隨侍,其疾亦尋愈。戎人聞延廣之至,不敢復為寇亂。



 叛卒劉渥嘯聚亡命數百人,寇耀州富平縣,謀入京兆,其勢甚盛。所過殺居民,奪財物,縱火而去,關右騷然。延廣率兵數百,自間道追之,會渥於富平西十五里,渥眾已千餘人,相持久之。渥素憚延廣,傳言:「我草間求活,觀死如鴻毛耳,侯公家世富貴,奈何不思保守,而與亡卒爭一旦之命於鋒鏑
 之下。」延廣怒,因擊之,挺身與渥斗大樹下,斷渥右臂,渥脫走,乘勢大破其眾。渥創甚,止谷中,後數日為追兵所獲。渥素號驍勇無敵,至是為延廣所殺,群盜喪氣,餘黨稍稍自歸,關右以定。上嘉之,擢拜崇儀使。



 淳化二年,李繼遷始擾夏臺,即命延廣領獎州刺史、知靈州,賜金帶名馬。會趙保忠陰結繼遷,朝廷命騎將李繼隆率兵問罪,以延廣護其軍。既而夏臺平,保忠就縛。手詔褒美,錫賚甚厚。師還,留為延州鈐轄。會節帥田重進老耄,郡中
 不治,以延廣同知州事兼緣邊都巡檢使。



 先是,延廣知靈州,部下嚴整,戎人悅服,李繼遷素避其鋒。監軍康贊元害其功,誣奏延廣得虜情,恐後倔強難制。遽詔還,以慕容德豐代之,部內甚不治。至道間,繼遷寇靈州,朝廷謀帥,同知樞密院事錢若水稱延廣可使,就拜寧州團練使、知靈州兼兵馬都部署。賜白金二千兩,歲增給錢二百萬。戎人塞道,郵傳饋餫皆不通,延廣獨自變量十騎之鎮,戎人素服其威名,皆相率引避。



 二年春,被病,上遣
 御醫馳驛視之。醫至,疾已亟,延廣謂中使李知信曰:「延廣自度必不起,家世受國恩,今日得死所矣,但恨未立尺寸功以報上耳。」言訖而卒,年五十。上聞之為出涕,賵賻甚厚,以其子為六品正員官。子紹隆,東染院使、帶御器械。紹隆子宗亮,右侍禁、閣門祗候。



 張從恩,並州太原人。父存信,振武軍節度。後唐明宗微時,嘗隸存信麾下。時從恩尚幼,頗無賴,明宗甚薄之,及即位,止授散秩。從恩不得志,乃退歸太原。



 晉祖鎮河東,
 為少帝娶從恩女。晉初,以外戚擢為右金吾衛將軍,未幾,改刺貝州,遷北京副留守,移授澶州防禦使。歷樞密副使、宣徽南院使、權西京留守,俄判三司。安從進叛於襄陽,以從恩為行營兵馬都監。



 少帝嗣位,襄陽平,遷檢校太尉、開封尹,充東京留守。少帝自鄴歸汴,改鄴都留守。錫賚加等,仍賜銀裝肩輿二,俾迎其家。明年,契丹擾河朔,從恩僅能完守。尋加同中書門下平章事。是歲,契丹將趙延昭據甘陵,命從恩為貝州行營都部署。從恩
 至,延昭遁去。詔與杜重威合兵三萬北伐。



 開運初,改天平軍節度。契丹復擾邊,命十五將北征,以從恩充北面行營都監。二年,移鎮晉州,又改潞州。及契丹入汴,從恩欲降,從事高防諫曰:「公晉室之親,宜盡宦節。」從恩不聽,乃棄城而去。巡檢使王守恩悉取其家財,以城歸漢祖。漢祖至汴。從恩惶懼不敢出。漢祖召賜襲衣、金帶、鞍勒馬、器幣以安慰之。尋拜右衛上將軍,奉朝請。



 周初,迂左金吾衛上將軍。周祖征兗州,從恩從行。世宗嗣位,加檢
 校太師,封褒國公。宋初,改封許國公,久之,以病免。乾德四後,卒,年六十九。



 扈彥珂,代州雁門人。幼事王建立,以謹厚稱。晉天福中,建立節制潞州,卒,遺表薦彥珂,得補河東節度左都押衙。會漢祖自太原建號,擢為宣徽南院使。未幾,授鎮國軍節度,華商等州觀察、處置等使。



 乾祐初,河中李守貞、永興趙思綰、鳳翔王景崇並據城叛,周祖為樞密使,總兵出征,道出華州。時議多以先討景崇、思綰為便,周祖
 意未決,彥珂曰:「三叛連衡,推守貞為主,宜先擊河中;河中平,則永興、鳳翔失勢矣。今舍近圖遠,若景崇、思綰逆戰於前,守貞兵其後,腹背受敵,為之奈何?」周祖從其言,及平河中,以功遷護國軍節度。時蒲人雕弊,思得良帥鎮撫。彥珂闇弱,朝議少之。



 廣順初,就加同平章事,移鎮滑州。歲餘代歸。與鳳翔趙暉俱獻緡帛,請開宴,不納,以滑州李守貞宅賜之。世宗嗣位,授左衛上將軍。顯德三年,以老疾上章求退,授開府儀同三司、太子太師致仕,
 歸西京。太祖即位,遣使就賜器幣,數月卒,年七十五。



 薛懷讓,其先戎人,徙居太原。少勇敢,喜戰鬥。後唐莊宗在鎮,得隸帳下,累歷軍職。明宗時,改神武右廂都校、領獎州刺史。東川董璋遣懷讓率本軍從晉祖討賊,賊平,遷絳州刺史。清泰初,移申州。明年,表乞罷郡赴代北軍,力陳不允。



 晉天福中,範延光叛於鄴,以懷讓為招牧使。及戰,中流矢,詔賜湯藥存問。又歷沂、遼、密、懷四州刺史,所至無善政,頗事誅斂。楊光遠反青州,召懷讓至闕,賜
 襲衣、玉帶,為行營先鋒都指揮使,以功改宿州團練使。



 會契丹南侵,少帝幸澶州,遣懷讓與李守貞、皇甫遇、梁漢璋率兵萬人緣河而下,以守汶陽。時契丹歲擾邊陲,朝廷擇驍將守要郡,命懷讓為洺州團練使。會符彥卿北討契丹,以懷讓為馬軍左廂排陣使。又從北面都招討杜重威為先鋒都指揮使。及重威降契丹於中渡橋,懷讓亦在籍中,非其志也。



 契丹主北歸,留麻答守鎮州,麻答遣步健督洺州供運。懷讓聞漢祖舉義晉陽。即殺
 步健,奉表歸漢,漢祖遣郭從義分兵萬餘,與懷讓取邢州。時偽帥劉鐸守邢臺,堅壁拒之,不克而還。麻答遣副將楊安以八百騎攻懷讓,又命剛鐵將三百騎繼之。懷讓戰不勝,退保本州,契丹大掠其封內。及麻答為鎮軍所逐,楊安亟遁,鐸又納款漢祖。懷讓乘其不虞,遣人紿鐸云:「我奉詔為邢州帥,今率眾襲契丹,請置頓於郡。」鐸無拒心,輒開門迎之,懷讓殺鐸,奪其城。漢祖即授以安國軍節度。



 隱帝即位,移鎮同州。及殺楊邠等,急召懷讓
 至闕。會北郊兵敗,懷讓降於周祖。



 周祖登位,賜襲衣、金帶、鞍勒馬,遣還任,加同平章事。劉崇入寇,懷讓表求西征,詔褒之。夏陽富人張廷徽誣告趙隱等五人為盜殺人,且厚賂懷讓子有光。懷讓知之,即諷吏掠治隱等,強伏之,遣掌書記李炳、親校賈進蒙追、判官劉震等鍛成其獄,隱等皆棄市。家人詣闕訴冤,懷讓亦自入朝,遽獻錢百萬,請開宴,不納。俄捕獲本賊,下御史臺鞫問,懷讓懼,獻馬十匹,復不納。有司請逮懷讓系獄,周祖以宿將,
 釋不問,杖流震等。俄以懷讓為左屯衛上將軍。



 世宗即位,加左武衛上將軍。顯德五年,請老,拜太子太師致仕。恭帝即位,封杞國公。建隆元年,卒,年六十九。贈侍中。



 懷讓好畜馬駝,馬有大鳥小鳥者,尤奇駿。漢隱帝使求之,吝而不獻。及罷節鎮,環衛祿薄,猶有馬百匹、橐駝三十頭,傾資以給芻粟,朝夕閱視為娛。家人屢勸鬻以供費,懷讓不聽。及死,童僕皆剺面以哭,蓋其俗也。



 趙贊字符輔。本名美,後改焉。幽州薊人。祖德鈞,後唐盧
 龍節度,封北平王。父延壽,尚明宗女興平公主,至樞密使、忠武軍節度。



 贊幼聰慧,明宗甚愛之,與諸孫、外孫石氏並育於六宅。暇日,因遍閱諸孫數十人,目贊曰:「是兒令器也。」贊七歲誦書二十七卷,應神童舉。明宗詔曰:「都尉之子,太尉之孫,幼能誦書,弱不好弄,克彰庭訓,宜錫科名,可特賜童子及弟。仍附長興三年禮部春榜。」久之,延壽出鎮宣武軍,因奏署牙內都校。



 清泰末,晉祖起並門,命延壽以樞密使將兵屯上黨,德鈞將本軍自幽州
 來會。時晉祖以契丹之援,引兵南下,德鈞父子降晉,契丹主盡錮之北去,贊獨與母公主留西洛。天福三年,晉祖命贊奉母歸薊門,契丹署為金吾將軍。數年,契丹以延壽為範陽節度,又署贊為牙內都校。開運末,契丹主將謀南侵,委政延壽。及平原陷,贊復受契丹署為河中節度。延壽從契丹北歸,贊得留鎮河中。



 未幾,漢祖起晉陽,贊奉表勸進,漢祖加檢校太尉,仍鎮河中。改京兆尹、晉昌軍節度。贊懼漢疑已,潛遣親吏趙仙奉表歸蜀。判
 官李恕者,本延壽賓佐,深所委賴,至家事亦參之。及贊出鎮,從為上介。至是,恕語贊曰:「燕王入遼,非所願也,漢方建國,必務懷柔,公若泥首歸朝,必保富貴,狠狽入蜀,理難萬全。儻復不容。後悔無及。公能聽納,請先入朝,為公申理。」贊即遣恕詣闕。漢祖見恕,問贊何以附蜀,恕曰:「贊家在燕薊,身受契丹之命,自懷憂恐,謂陛下終不能容,招引西軍,蓋圖茍免。臣意國家甫定,務安臣民,所以令臣乞哀求覲。」漢祖曰:「贊之父子亦吾人也,事契丹出
 於不幸。今聞延壽落於陷阱,吾忍不容贊耶。」恕未還,贊已離鎮入朝,即命為左繞衛上將軍,徙恕邠州判官。



 贊仕周,歷左右羽林、左龍武三統軍。世宗南征,初遣贊率師巡警壽州城外,俄命為淮南道行營左廂排陣使。世宗歸京,留贊與諸將分兵圍壽春,贊獨當東面。諸將戰多不利,贊獨持重,自秋涉冬,未嘗挫衄。及受詔移軍,尺椽片瓦,悉輦而行,城中人無敢睥睨者。會吳遣驍將魯公綰帥十餘萬眾溯淮奄至,跨山為柵,阻服水,俯瞰城
 中。時大軍已解圍,贊與大將楊承信將輕騎斷吳人饟路,又獨以所部襲破公綰軍,為流矢所中。



 世宗再征壽春,命造橋渦口,以通濠、泗。令騎帥韓令坤董其役,俾贊副之。屬霖雨淮水漲溢,濠人謀乘輕舟奄焚其橋,贊覘知之,設伏橋下。濠人果至,贊令強弩亂發,殺獲甚眾。及世宗移兵趣濠,以牛革蒙大盾攻城,贊親督役,矢集於冑,中被重傷,猶力戰,遂拔其羊馬城,刺史唐景思死焉,團練使郭廷謂以城降。世宗詔褒美之。又以所部兵巡
 撫滁、和之間,破吳人五百於石潭橋。淮南平,以戰功多,授保信軍節度。贊入視事,盡去苛政,務從寬簡,居民便之。恭帝即位,加開府階。



 宋初,加檢校太師,移忠正軍節度,預平維揚。歲餘,改鎮延州,受密旨許以便宜行事。將及州境,乃前後分置步騎,綿綿不絕,林莽之際,遠見旌旗,所部羌、渾來迎,無不懾服。



 乾德六年,移建雄軍節度。秋,命將征太原,以贊為邠州路部署。開寶二年,太祖將討晉陽,又以為河東道行營前軍馬步軍都虞候。車駕
 薄城下,分軍四面,贊扼其西偏。並人乘晦自突門潛犯贊壘,贊率眾擊之,久而方退,弩矢貫足。及祖勞問數四,賜良藥傅之。四年,議鎮鄜州。



 太宗即位,進封衛國公。及平興國二年,來朝,未見而卒,年五十五。贈侍中。



 贊頗知書,喜為詩,容止閑雅,接士大夫以禮,馭眾有方略。其為政雖無異跡,而吏民畏服,亦近代賢帥也。



 李繼勛,大名元城人。周祖領鎮,選隸帳下。廣順初,補禁軍列校,累遷至虎捷左廂都指揮使、領永州防禦使。顯
 德初,遷侍衛步軍都指揮使、領昭武軍節度。歲餘,改領曹州。



 世宗親征淮上,令繼勛領兵屯壽州城南,進洞屋、雲梯,以攻其城。繼勛怠於守御,為其所敗,死者數萬,梯、屋悉皆被焚。召歸闕,出為河陽三城節度。議者以為失責帥之義。及再幸壽春回,左授繼勛右武衛大將軍,又以其掌書記陳南金裨贊無狀,並黜之。



 顯德四年冬,復從世宗南征,及次迎鑾,即命繼勛帥黑龍船三十艘於江口灘,敗吳兵數百,獲戰船二艘,以功遷左領軍衛上
 將軍。七月,改右羽林統軍。六年春,世宗幸滄州,以繼勛為戰棹左廂都部署,前澤州刺史劉洪副之,俄權知邢州。恭帝即位,授安國軍節度,加檢校太傅。



 宋初,加檢校太尉。太祖平澤、潞,繼勛朝於行在,即以為昭義軍節度。是秋,率師入河東,燔平遙縣,俘獲甚眾。建隆二年冬,又敗並軍千餘人,斬首百餘級,獲其遼州刺史傅延彥及弟延勛來獻。



 乾德二年,詔與康延沼、尹訓率步騎萬餘攻遼州,太原將郝貴超領兵來援,戰於城下,繼勛大敗
 之。州將杜延韜危蹙,與拱衛都指揮使冀進、兵馬都監供奉官侯美籍部下兵三千送款於繼勛。即遣內供奉官都知慕容延忠入奏,詔褒之。未歲,並人誘契丹步騎六萬人來取遼州,復遣繼勛與羅彥瑰、郭進、曹彬等領六萬眾赴之,大破契丹及太原軍於城下。五年,加同平章事。



 開寶初,將征河東,以繼勛為行營前軍都部署,敗並人於渦河。二年,太祖親征河東,命繼勛為行營前軍都部署。駕至城下,分軍四面,繼勛柵其南。三年春,移鎮
 大名。太平興國初,加兼侍中。俄以疾求歸洛陽,許之,賜錢千萬、白金萬兩。是秋,上表乞骸骨,拜太子太師致仕,朝會許綴中書門下班。尋卒,年六十二,贈中書令。



 繼勛累歷藩鎮,所至無善政,然以質直稱。信奉釋氏。與太祖有舊,故特承寵遇。



 弟繼偓,亦有武勇,周顯德末,補內殿直。宋初,累歷軍職。開寶中,為步軍副都軍頭。太平興國三年,遷內外馬步軍副都軍頭。坐事改右衛率府率。六年,加本衛將軍、領獎州刺史。累至龍衛右廂都指揮使、
 領本州團練使。



 繼勛子守恩至如京使。守元至北作坊使,守徽為崇儀副使。



 藥元福,並州晉陰人。幼有膽氣,善騎射。初事邢帥王檀為廳頭軍使,以勇敢聞。事後唐,為拱衛、威和親從馬鬥軍都校,天平軍內外馬軍都指揮使。晉天福中,為深州刺史。



 開運初,契丹陷甘陵,圍魏郡,師次於河。少帝駐軍澶淵,契丹陣於城北,東西連亙,掩城兩隅,登陴望之,不見其際。元福以左千牛衛將軍領兵居陣東偏。澶民有
 馬破龍者告契丹曰:「先攻其東,即浮梁可奪。」契丹信之,盡銳來戰。元福與慕容鄴各領二百騎為一隊,躍出而鬥,元福奮鐵撾擊契丹,斃者數人,左右馳突,無不披靡,契丹兵潰。少帝登城,見元福力戰,召撫之曰:「汝奮不顧命,雖古之忠烈無以過之。」元福三馬皆中流矢,少帝擇名馬賜之。明日將戰,面授元福鄭州刺史,為權臣所沮,止刺原州,俄改泰州。



 明年,契丹復入。命元福與李守貞、符彥卿、皇甫遇、張彥澤等御之於陽城,為右廂副排陣
 使。晉師列方陣,設拒馬為行砦。契丹以奇兵出陣後,斷糧道,晉人乏水,士馬饑渴,鑿井未及泉,土輒壞塞,契丹順風揚塵,諸將皆曰:「彼勢甚銳,俟風反與戰,破之必矣。」守貞與元福謀曰:「軍中饑渴已甚,若俟風反出戰,吾屬為虜矣。彼謂我不能逆風以戰,宜出其不意以擊之,此兵家之奇也。」元福乃率麾下騎,開拒馬出戰,諸將繼至,契丹大敗,追北二十餘里,殺獲甚眾,敵帥與百餘騎遁去。以元福為威州刺史。



 會靈武節度王令溫以漢法治
 蕃部,西人苦之,共謀為亂,三族酋長拓跋彥超、石存、乜廝褒率眾攻靈州。令溫遣人間道入奏,乃以河陽節度馮暉鎮朔方,召關右兵進討,以元福將行營騎兵。元福與暉出威州土橋西,遇彥超兵七千餘,邀暉行李。元福轉戰五十里,殺千級,禽三十餘人,又遣部校援出令溫,護送洛下。



 朔方距威州七百里,無水草,號旱海,師須繼糧以行,至耀德食盡,比明,行四十里。彥超等眾數萬,布為三陣,扼要路,據水泉,以待暉軍,軍中大懼。暉遣人賂
 以金帛,求和解彥超許之。使者往復數四,至日中,列陣如故。元福曰:「彼知我軍饑渴,邀我於險,既許和解而日中未決,此豈可信哉?欲困我耳。遷延至暮,則吾黨成禽矣。」暉驚曰:「奈何?」元福曰:「彼雖眾而精兵絕少,依西山為陣者是也,餘不足患。元福請以麾下騎先擊西山兵,公但嚴陣不動,俟敵少卻,當舉黃旗為號;旗舉則合勢進擊,敗之必矣。」暉然其策,遂率眾進擊,敵眾果潰。元福即舉黃旗以招暉,暉軍繼進,彥超大敗,橫尸蔽野。是夕,入
 清邊軍。明日,至靈州。元福還郡,詔賜暉、元福衣帶繒帛銀器。



 漢乾祐中,從趙暉討王景崇於鳳翔。時兵力寡弱,不滿萬人,蜀兵數萬來援,景崇至寶雞,依山列柵。都監李彥從以數千人擊蜀軍,眾寡不敵,漢軍少卻。元福領數百騎自後驅之,下令還顧者斬,眾皆殊死戰,大敗蜀兵,追至大散關,殺三千餘人,餘皆棄甲遁去。鳳翔平,以功遷淄州刺史。



 周廣順初,王彥超討徐州叛將楊溫,以元福為行營兵馬都監。數月克之,率師還京,改陳州防
 禦使。



 未幾,劉崇引契丹擾晉州,命樞密使王峻率兵拒之,以元福為西北面都排陣使。軍過蒙坑,崇夜燒營遁。峻令元福與仇超、陳思讓追至霍邑,既行,又遣止之。元福謂思讓等曰:「劉崇召契丹擾邊,志在疲弊中國,今兵未交而遁,宜追奔深入,以挫其勢。」諸將畏懦,遂止。周祖知其事,明年,因調兵戍晉州,謂左右曰:「去年劉崇之遁,若從藥元福之言,則無邊患矣。」



 俄與曹英、向訓討慕容彥超於兗州,元福為行營馬步軍都虞候。詔元福自晉
 州率所部入朝,即遣東行,賜六銖、袍帶、鞍馬、器仗。周祖謂曰:「比用曹州防禦使鄭璋,我度彥超兇狡,多計謀,恐璋不能集事,選爾代之。已敕曹英、向訓不令以軍禮見汝。」及至軍中,英、訓皆尊禮之,當時有為宿將。築連城以圍兗,彥超晝夜出兵,元福屢擊敗之,遂閉壁不敢出。十餘日,元福營柵皆就,又穴地及築土山,百道攻其城。會周祖親征,元福以所部先入羊馬城,諸軍鼓噪角進,拔之。以功授建雄軍節度。



 世宗高平之戰,劉崇敗走太原,
 遂縱兵圍其城。以元福為同州節度,充太原四面壕砦都部署。時攻具悉備,城中危急,以糧運不繼,詔令班師。元福上言曰:「進軍甚易,退軍甚難。」世宗曰:「一以委卿。」遂部分卒伍為方陣而南,元福以麾下為後殿,崇果出兵來追,元福擊走之。師還,加檢校太尉,移鎮陜州。又歷定、廬、曹三鎮。



 宋初,加檢校太師。九月卒,年七十七,贈侍中。



 元福雖老,筋骨不衰,人或言其氣貌益壯,當復領兵,必大喜,曲致禮待,或加以贈遣,時稱驍將。



 趙晁,真定人。初事杜重威為列校。重威誅,屬周祖鎮鄴中,晁因委質麾下。周祖開國,擢為作坊副使。慕容彥超據兗州叛,以晁為行營步軍都監。兗州平,轉作坊使。晁自以逮事霸府,復有軍功,而遷拜不滿所望,居常怏怏。時樞密使王峻秉政,晁疑其軋己。一日使酒詣其第,毀峻,峻不之責。世宗嗣位,改控鶴左廂都指揮使、領賀州刺史。



 從征劉崇,轉虎捷右廂都指揮使、領本州團練使兼行營步軍都指揮使。軍至河內,世宗意在速戰,令晁
 倍道兼行。晁私語通事舍人鄭好謙曰:「賊勢方盛,未易敵也,宜持重以挫其銳。」好謙以所言入白,世宗怒曰:「汝安得此言,必他人所教。言其人,則舍爾;不言,當死!」好謙懼,遂以實對。世宗即命並晁械於州獄,軍回始赦之。



 及征淮南,改虎捷左廂、領閬州防禦使,充前軍行營步軍都指揮使,又為緣江步軍都指揮使。時李重進敗吳人於正陽,以降卒三千人付晁,晁一夕盡殺之。世宗不之罪。壽春平,拜檢校太保、河陽三城節度、孟懷等州觀察
 措置等使。恭帝即位,加檢校太傅。



 宋初,加檢校太尉。未幾,以疾歸京師,卒,年五十二。太祖甚悼之,贈太子太師,再贈侍中。



 晁身長七尺,儀貌雄偉,好聚斂,處方鎮以賄聞。以周初與宣祖分掌禁軍,有宗盟之分,故太祖常優禮之,再加贈典焉。子延溥。



 延溥,周顯德中,以父任補左班殿直。宋初,為鐵騎指揮使。開寶初,太祖親征晉陽,太宗守京邑,延溥以所部為帳下牙軍,轉殿前散員指揮使。九年,改鐵騎都虞候。



 太
 宗即位,遷散指揮都虞候、領思州刺史。太平興國二年,轉內殿直都虞候。三年,改馬步軍都虞候。從平太原,略地燕薊。六軍扈從有後期至者,帝怒,欲置於法。延溥遂進曰:「陛下巡行邊陲,以防禦外侮,今契丹未殄,而誅譴將士,若舉後圖,誰為陛下戮力乎?」帝嘉納之。師還,遷內外馬步軍都軍頭、領本州防禦使。



 五年,殿前白進超卒,即日以延溥為日騎、天武左右廂都指揮使。兼權殿前都虞候事。坐遣親吏市竹木所過關渡稱制免算,責
 授登州團練使,令赴任。是冬,帝北巡至大名,復以延溥為本州防禦使,即命為幽州東路行營壕砦都監。詔修緣邊城壘。逾年,加涼州觀察使,仍判登州。又為鎮州兵馬都部署,俄判霸州。



 雍熙二年,改蔚州觀察使,判冀州。會命曹彬等北征,又與內衣庫使張紹勍、引進副使董願為幽州西北道行營都監。師還,命知貝州,改滑州部署。四年,再知貝州,以疾求代,代未至,卒,年五十。贈天德軍節度。



 子承彬,至內殿崇班。承彬子咸一,為虞部員外
 郎,知宗正丞事。咸熙,天聖八年進十及第。



 論曰:侯益在晉、漢時,數為反復,觀其受命契丹,私交偽蜀,赤罔之戰,復夜謁周祖,宗屬長幼,遭景崇鯨鯢,殆無□類,推其心跡,豈懷貳之罰歟?薛懷讓、趙晁為將,皆忍於殺降。晁子延溥,能救後至之誅,雖父子之親,仁暴相戾有若是者。餘皆逢時奮武,致身榮顯。扈彥珂請擊河中,卒用其策,愚者之一慮云



\end{pinyinscope}