\article{本紀第三十四}

\begin{pinyinscope}

 孝宗二



 三年春正月甲辰,詔廷尉大理官毋以獄情白宰執,探刺旨意為輕重。庚戌,置三省戶房國用司。初,以國用匱乏,罷江州屯駐軍馬,至是復留之。癸亥,罷銅錢過江之
 禁。裁定利州西路諸軍額。



 二月壬申,詔國用司月上宮禁及百司官吏、三衙將士請給之數。癸酉,出龍大淵為江東總管,曾覿為淮西總管。甲戌,大淵改浙東,覿改福建。乙亥。罷成都、潼川路轉運司輪年銓試,以其事付制置司。辛巳,以端明殿學士虞允文知樞密院事。癸未,雨雹。甲申,為知陳州陳亨祖立廟於光州,賜名愍忠。丙戌,以《武經龜鑒》、《孫子》賜鎮江都統戚方、建康都統劉源。癸巳,措置淮東山水砦。丙申,從太上皇、太上皇后幸玉津
 園。戊戌,直秘閣、前廣東提刑石敦義犯臟,刺面配柳州,籍其家。



 三月甲辰,從太上皇、太上皇后幸聚景園。辛亥,詣德壽宮,恭請裁定醫官員額。丁巳,詔四川宣撫司創招千人,置司所在屯駐。壬戌,伯母秀王夫人張氏薨。



 夏四月辛未,蠲諸路州軍逋負。癸酉,為秀王王夫人成服於後苑,百官進名奉慰。丁丑,合利州東、西路為一。戊寅,以吳璘知興元府、充利州路安撫使、四川宣撫使。



 五月癸卯,葉顒等上《三祖下仙源積慶圖》及《太宗真宗玉牒》、《
 哲宗寶訓》。甲寅,吳璘薨。庚申。命四川制置使汪應辰主管宣撫司事,移司利州。修揚州城。壬戌,大減三衙官屬。



 六月己巳,命汪應辰權節制利州路屯駐御前軍馬。辛未,復分利州東、西路為二。甲戌,以虞允文為資政殿大學士、四川宣撫使。乙亥,金遣使來取被俘人。詔實俘在民間者還之,軍中人及叛亡者不預。戊寅,復以虞允文為知樞密院事,充宣撫使,帝親書九事戒之。罷淮西、江東總領所營田,募人耕佃,壯丁各還本屯,癃老存留,減
 半請給。甲申,詔鎮江都統制戚方、武鋒軍都統制陳敏各上清河口戰守之策。追封吳璘為信王。丁亥,詔後省參考理檢院典故。辛卯,皇后夏氏崩。振泉州水災。



 秋七月己亥,立薦舉改官額。壬寅,以皇太子疾,減雜犯死罪囚,釋流以下。乙巳,皇太子薨,謚曰莊文。己酉,東宮醫官杜揖除名、昭州編管,尋改瓊州。



 閏月辛未,詔諸軍復置副都統制,文字與都統制連書,軍馬調發從都統制,違者奏劾。戚方罷。癸酉,權攢安恭皇后於臨安修吉寺。丁
 亥,戚方落節鉞、信州居住。



 八月丁酉,內侍陳瑜、李宗回坐交結戚方受賂,瑜除名、決杖、黥面配循州,宗回除名、筠州編管,方責授果州團練副使、潭州安置,籍所盜庫金以犒軍。甲寅,以久雨,命臨安府決系囚。丁巳,葉顒等請罷,不許。蠲光、濠、廬三州、壽春府賦一年。戊午,遣官分決滯獄。壬戌,以知建康府史正志兼沿江水軍制置使,自鹽官至鄂州沿江南北及沿海十五州水軍悉隸之。癸亥,詔給、舍討論考課舊法。四川旱,賜制置司度牒四
 百,備振濟。九月戊子,太白晝見。冬十月乙未朔,占城入貢。丁酉,遣唐□彖等使金賀正旦。戊戌。修真州城。以嗣濮王士□曷為開府儀同三司。庚子,定內外薦舉改官人歲額。癸卯,詔歸正借補官資人充樞密院效士,於指定州軍以官庫酒息贍之者,毋罷其給。乙卯,金遣蒲察莎魯窩等來賀會慶節。



 十一月丙寅,合祀天地於圜丘,大赦。戊辰,雷。己巳,詔戒飭武臣及百官。癸酉,以郊祀雷,葉顒、魏杞並罷,命陳俊卿為參知政事,翰林學士劉珙同知
 樞密院事。甲戌,蔣芾、陳俊卿請罷,不許。丁丑,以雷發非時,詔臺諫、侍從、兩省官指陳闕失。辛巳,詔侍從、兩省、臺諫、卿監、郎官舉堪郎官、寺監丞、監司、郡守者。癸巳,罷川路馬船。十二月丙申,增修六合城。己亥,遣王瀹賀金主生辰。乙巳,置豐儲倉。增印會子。辛亥,以吳益為太傅。庚申,金遣徒單忠衛等來賀明年正旦。是歲,兩浙水,四川旱,江東西、湖南北路蝗,振之。



 四年春正月戊辰,籍荊南義勇民兵,增給衣甲,遇農隙
 日番教。壬午,奪秦塤、秦堪郊恩蔭補。癸未,雨雹。甲申,幸天竺寺,遂幸玉津園。辛卯,罷吳益郊恩蔭補。壬辰,葉顒薨。



 二月甲午朔,罷福建路賣鈔鹽,蠲轉運司歲發鈔鹽錢十五萬緡。詔四川宣撫使虞允文集四路漕臣會計財賦所入,對立兵額。丁酉,命湖北安撫司給田募辰、沅、靖三州刀弩手。戊戌,置和州鑄錢鹽。己亥,以蔣芾為尚書右僕射、同中書門下平章事兼樞密使兼制國用使,觀文殿大學士史浩為四川制置使,浩辭不行。庚子,詔
 蔣芾常朝,贊拜不名。芾辭,許之。乙巳,賜王炎出身,簽書樞密院事。癸丑,五星皆見。乙卯,雪,雨雹。



 三月庚午,以敷文閣待制晁公武為四川安撫制置使。戊寅,詔贈果州團練使韓崇岳立廟,賜名忠勇,宣州觀察使朱勇立廟,賜名忠節。己丑,四方霧下若塵。庚寅,蠲楚州壯丁、社民稅役。謚陳亨伯曰愍節。



 夏四月乙未,置漢陽軍收發馬監。詔公吏非犯公罪,毋得引用並計案問法。巳亥,置郢州轉般倉。癸卯,遣使撫邛、蜀二州饑民為亂者。己酉,追
 封韓世忠為蘄王。甲寅,蔣芾等上欽宗《帝紀》、《實錄》。丙辰,禮部員外郎李燾上所著《續通鑒長編》自建隆至治平一百八卷。丁巳,詔太史局參用新舊歷。戊午,詔販牛過淮者,論如興販軍須之罪。是月,振綿、漢等州饑。五月癸亥,出度牒千道,續減四川科調。乙丑,太白晝見。以邛州安仁縣荒旱,失於蠲放,致饑民擾亂,守、貳、縣令降罷追停有差。甲申,謚趙鼎曰忠簡。丙戌,行乾道新歷。丁亥,以饒、信二州、建寧府饑民嘯聚,遣官措置振濟。是月,西夏
 任敬德遣使至四川宣撫司,約發兵攻西番。



 六月辛卯朔,太白晝見,經天。甲午,詔罷廣西鈔鹽,復官般官賣法,歲減轉運司鈔錢十九萬緡,其秋苗毋得科折。戊戌,蠲諸路逋負乾道元年二月和市、折帛、雜色錢。辛丑,龍大淵卒,詔以為寧武軍節度使致仕。五星皆見。癸卯,詔四川宣撫司增印錢引一百萬,對償民間預借錢。蠲邛、蜀二州夏稅。丁巳,召興化軍布衣林彖赴行在。戊午,蔣芾以母喪去位。



 秋七月壬戌,以劉珙兼參知政事。召建寧
 府布衣魏掞之赴行在。申禁異服異樂。癸亥,徽州大水。己巳,罷沿江水軍制置司。辛未,衢州大水。戊寅,知衢州王悅以盛暑禱雨、蔬食減膳、尤勤致疾而死,贈直龍圖閣。丁亥,以經、總制餘剩錢二十一萬緡樁留邛、蜀州,以備振濟。己丑,以久雨,御延和殿慮囚,減臨安府、三衙死罪以下囚,釋杖以下。是月,西夏遣間使來。



 八月乙未,班祈雨雪之法於諸路。己亥,五星皆見。丁未,主管殿前司公事王琪傳旨不實,擅興工役,降三官放罷。庚戌,劉珙
 罷。辛亥,陳俊卿請罷政,不許。



 九月庚申,立內外將佐升差審察法。庚午,從太上皇幸天竺寺。限品官子孫名田。是秋,罷關外四州營田官兵,募民耕佃。



 冬十月壬辰,遣鄭聞等使金賀正旦。甲午,禁歸正人藏匿金人者。乙未,臣僚言:「天下之事,必歷而後知,試而後見。為縣令者必為丞簿,為郡守者必為通判,為監司者必為郡守,皆有等差。自今職事官及局務官,必任滿方許求外,未歷親民任使,即未得擬州郡,且授通判。」詔從之。庚子,蔣芾起
 復尚書左僕射,陳俊卿右僕射,並同中書門下平章事兼樞密使兼制國用使。甲辰,大閱。己酉,金遣移刺神獨斡等來賀會慶節。庚戌,大風。



 十一月壬戌,遣知無為軍徐子寅措置楚州官田,招集歸正忠義人以耕。甲戌,嚴盜賊法。乙亥,詔峽州布衣郭雍赴行在。壬申,兩淮歸正忠義有田產者,蠲役五年。癸未,岳陽軍節度使居廣封永陽郡王。十二月丙申,遣胡元質等賀金主生辰。甲辰,賜魏掞之同進士出身,為太學錄。蔣芾辭起復,許之。減
 兩浙、江東西路明年夏稅、和市之半。甲寅,金遣完顏仲仁等來賀明年正旦。



 五年春正月甲戌,措置兩淮屯田。



 二月己丑,申嚴太廟季點法。乙未,命楚州兵馬鈐轄羊滋專一措置沿淮、海盜賊。先是,海州人時旺聚眾數千來請命,旺尋為金人所獲,其徒渡淮而南者甚眾,故命滋彈壓之。戊戌,贈張浚太師,謚忠獻。壬寅,以給事中梁克家簽書樞密院事。癸卯,大風。甲辰,以王炎參知政事兼同知樞密院事。
 丙午,雨雹。辛亥,詔自今詔令未經兩省書讀者毋輒行,給、舍駁正毋連銜同奏。



 三月丁巳朔,詔趣修廬、和二州城。己巳,蠲成都府路民戶歲輸對糴米腳錢三十五萬緡。乙亥,以王炎為四川宣撫使,仍參知政事。召虞允文赴行在。丙子,賜禮部進土鄭僑以下三百九十有二人及第、出身。壬午,賜郭雍號沖晦處士。癸未。罷利州路諸州營田官兵,募民耕佃。詔侍從、監司、帥臣、管軍薦武舉出身人可將佐者。



 夏四月己丑,復置將作、軍器少監。壬
 辰,以梁克家兼參知政事。辛丑,詔福建路貧民生子,官給錢米。庚戌,修襄陽府城。辛亥,振恤衢、婺、饒、信四州流民。



 五月己巳,帝以射弩弦斷傷目,不視朝。金牒取俘獲人,王抃議盡遣時旺餘黨,陳俊卿持不可,帝然之。



 六月庚寅,太白晝見。戊戌,始視朝。己酉,以虞允文為樞密使。



 秋七月乙丑,召曾覿入見,陳俊卿及虞允文請罷之,不許。至行在,俊卿、允文復言其不可留,詔以覿為浙東總管。



 八月甲申朔,日有食之。己丑,以陳俊卿為尚書左
 僕射,虞允文為尚書右僕射,並同中書門下平章事兼樞密使兼制國用使,辛亥,命淮西路鑄小鐵錢。



 九月己未,罷淮東屯田官兵,募民耕佃。辛酉,詔淮東諸州農隙教閱民丁。甲子,詔侍從、臺諫集議欽宗配饗功臣。壬申,大風。命淮西安撫司參議官許子中措置淮西山水砦,招集歸正忠義人耕墾官田。



 冬十月乙酉,遣汪大猷等使金賀正旦。戊子,振溫、臺二州被水貧民,以守臣、監司失職,降責有差。戊戌,大風。己亥,命饒、信二州歲各留上
 供米三萬石,以備振糶。癸卯,金遣高德基等來賀會慶節。



 十一月癸丑朔,復置淮東萬弩手,名神勁軍。庚申,增置廣東水軍。乙丑,以孫擴為右千牛衛大將軍。以明州定海縣水軍為御前水軍。丙寅,為岳飛立廟於鄂州。己巳,太白晝見。辛未,詔侍從、臺諫、兩省官,各舉京朝官以上、才堪監司、郡守者三人。壬申,復成閔慶遠軍節度使、鎮江諸軍都統制。十二月己丑,遣司馬伋等賀金主生辰。辛卯,大風。丁酉,置應城縣馬監。復李顯忠威武軍節
 度使。乙巳,復置成都府廣惠倉。戊申,金遣完顏毅等來賀明年正旦。



 六年春正月癸丑,雅州沙平蠻寇邊,焚碉緬砦。四川制置使晁公武調兵討之,失利。乙卯,修楚州城。丁巳,復強盜舊法,其四年十一月指揮勿行。癸亥,初降金字牌下四川宣撫司,備邊奏。乙丑,增築豐儲倉。庚午,以奉國軍承宣使、知廬州郭振為武泰軍節度使。



 二月乙酉,詔戶部侍郎二人分領諸路財賦。丁亥,復置舒州同安監,鑄
 鐵錢。辛卯,王炎遣人約沙平蠻歸部,稍損邊稅與之。丙申,廣西路復行鈔鹽法,仍增收通貨錢四十萬緡,以備漕計。壬寅,詔諭大臣:均役法,嚴限田,抑游手,務農桑。己酉,置應城縣孳生監。庚戌,以曾覿為福州觀察使。遣司農寺丞許子中詣淮西,措置鐵錢。



 三月癸丑,用三省言,兩淮守帥宜久其任,二年後察其能否,以行賞罰。乙卯,裁減樞密院吏額一百十有四人。丁巳,詔步軍司權以三萬五千人為額。起復王抃僉知閣門事,專一措置三衙
 揀選官兵。贈彰國軍節度使大周仁為太尉。庚申,從太上皇、太上皇后幸聚景園。乙丑,以晁公武、王炎不協,罷四川制置司歸宣撫司。辛未,從太上皇、太上皇后幸聚景園。甲戌,裁減三省吏額七十人。戊寅,以知紹興府史浩為檢校少傅、保寧軍節度使。己卯,詔兩淮州縣官以繁簡易其任。復置江、浙、京湖、兩廣、福建等路都大發運使,以新知成都府史正志為之。



 夏四月辛巳朔,罷鑄錢司歸發運司。並淮東總領所歸淮西總領所。以敷文閣
 直學士張震知成都府,充本路安撫使,乙未,賜發運使史正志緡錢二百萬,為均輸、和糴之用。吏部尚書汪應辰三上疏論發運司。戊戌,以應辰知平江府。



 五月甲寅,裁減六部吏額百五十人,其餘百司、三衙以是為差。己未,陳俊卿、虞允文等上神宗、哲宗、徽宗、欽宗四朝《會要》、太上皇玉牒。已已,陳俊卿以議遣使不合,罷為觀文殿大學士、知福州。罷行在至鎮江征稅所比近者十有三。甲戌,詔戒飭百官。丁丑,知潮州曾造犯贓,貸命、南雄州
 編管,籍其家。戊寅,詔給舍、臺諫言事。



 閏月壬午,詔監司、帥臣舉守令臧否失實,依舉清要官法定罪。甲申,印給諸州上供綱目,季申而歲校之,以為殿最。戊子,遣範成大等使金求陵寢地,且請更定受書禮。辛卯,吏部侍郎陳良祐論祈請使不當遣,恐生邊釁。詔以良祐妄興異論,不忠不孝,放罷、送筠州居住。癸巳,增環衛官奉。以梁克家為參知政事兼同知樞密院事。壬寅,以江東漕臣黃石不親按行水災州郡,降二官。甲辰,辛次膺薨。戊申,
 復置武臣提刑。



 六月壬子,申嚴卿監、郎官更迭補外之制。壬申,增武學生為百人。癸酉,置蘄州蘄春監、黃州齊安監,鑄鐵錢。是月,榮國公挺自東宮出居外第。



 秋七月癸未,詔以沙田、蘆場歲收租稅六十餘萬緡入左藏南庫。丙戌,詔川廣監司、郡守任滿奏事訖方調。己丑,置興國軍興國監。甲午,詔除郎官並引對畢供職。辛丑,復置御前弓馬子弟所,命吳挺兼提舉。賜岳飛廟曰忠烈。



 八月庚戌,虞允文請蚤建太子。癸丑,復置詳定一司敕令
 所。丙寅,置閣門舍人十員。是月,虞允文上《乾道敕令格式》。



 九月壬辰,賜蘇軾謚曰文忠。辛丑,沅州猺人相仇殺,守臣孫叔傑出兵擊之,失利。徭人進迫州城,安撫司諭解之,叔傑尋抵罪。是月,範成大至自金,金許以遷奉及歸欽廟梓宮,而不易受書禮。



 冬十月己酉,以孫攄為左千牛衛大將軍。丙辰,詔發運使置司行在。謚司馬樸曰忠潔。辛酉,遣呂正己等使金賀正旦。丁卯,金遣耶律子敬等來賀會慶節。甲戌,起居舍人趙雄請置局議恢復,
 詔以雄為中書舍人。



 十一月丁丑朔,復置軍器監一員。壬午,合祀天地於圜丘,大赦。乙未,復置神武中軍,以吳挺為都統制。召曾覿提舉祐神觀。丁酉,加上光堯壽聖太上皇帝尊號曰光堯壽聖憲天體道太上皇帝、壽聖太上皇后尊號曰壽聖明慈太上皇后。是月,遣趙雄等賀金主生辰,別函書請更受書之禮。置左藏南上庫。十二月戊申,大閱。甲子,置江州廣寧監、臨江軍豐餘監、撫州裕國監鑄鐵錢。壬申,金遣蒲察願等來賀明年正旦。
 癸酉,罷發運司。以史正志奏課不實,責為楚州團練副使、永州安置。是歲,兩浙、江東西、福建水、旱。



 七年春正月丙子,率群臣奉上太上皇、太上皇后冊寶於德壽宮。庚辰,虞允文復請建太子,帝命允文擬詔以進。壬寅,命三省旬錄宣諭聖語及時政記同進。是月,復鑄錢司。



 二月癸丑,詔立子惇為皇太子,大赦。以慶王愷為雄武、保寧軍節度使、判寧國府,進封魏王。丁巳,增置皇太子宮講讀官。庚申,罷會子庫,仍賜戶部內藏南庫
 緡錢二百萬、銀九十萬兩,以增給官兵之奉。甲子,詔寺觀毋免稅役。丁卯,太傅、大寧郡王吳益薨。壬申,大風。



 三月乙亥朔,趙雄至金,金拒其請。詔訓習水軍。丙子,立恭王夫人李氏為皇太子妃。戊寅,徙侍衛馬軍司戍建康。己卯,起復劉珙同知樞密院事。以明州觀察使、知閣門事兼樞密都承旨張說簽書樞密院事。左司員外郎兼侍講張栻言說不宜執政。乙酉,立沿海州軍私繼銅錢下海船法。丙戌,復置將作監。殿中侍御史李處全乞遣
 張說按行邊戍,以息眾論,中書舍人範成大乞不草詞。戊子,說罷為安慶軍節度使、提舉萬壽觀。庚寅,遣使核兩淮種麥。丙申,御大慶殿冊皇太子。禮部侍郎鄭聞、工部侍郎胡銓、樞密院檢詳文字李衡、秘書丞潘慈明並罷。虞允文乞留銓,乃以為寶文閣待制兼侍講。己亥,皇太子謝於紫宸殿,宰相率百官赴東宮賀。



 夏四月戊申,以曾覿為安德軍承宣使。庚申,詔諸路增收無額錢物,並輸南上庫。壬戌,從太上皇、太上皇后幸聚景園。。甲子,
 詔皇太子判臨安府。己巳,詔侍從、臺諫、兩省官舉任刑獄、錢穀及有智略吏能者各二人。辛未,詔皇太子領臨安尹。



 五月戊寅,復置淮東總領所。丁亥,劉珙起復同知樞密院事,為荊、襄宣撫使,珙辭不拜。庚寅,金人葬欽宗於鞏原。丁酉,詔廣西帥臣措置南丹州市馬。是月,遣知閣門事王抃點閱荊、襄軍馬。



 六月丙午,復主管馬軍司公事李顯忠為太尉。己巳,賜吳璘謚曰武順。壬申,詔兩淮墾田毋創增稅賦。



 秋七月庚子,以王炎為樞密使、四
 川宣撫使。



 八月丙辰,詔兩淮民丁充民兵者,本名丁錢勿輸。辛酉,復修襄陽城。



 九月壬申朔,以江西、湖南旱,命募民為兵。甲申,從太上皇、太上皇后幸東園。戊子,安定郡王令德薨。



 冬十月丁未,罷紹興宗正行司,改恩平郡王璩判西外宗正。己酉,遣莫蒙等使金賀正旦。壬戌,金遣烏林答天錫等來賀會慶節,天錫要帝降榻問金主起居,虞允文請帝還內,命知閣門事王抃諭天錫以明日見,天錫沮退。癸亥,會慶節,金使隨班入見。



 十一月甲
 戌,御集英殿策試應賢良方正能直言極諫科李垕。戊寅,錫垕制科出身。十二月丁未,遣翟紱等賀金主生辰。庚申,詔閣門舍人依文臣館閣以次輪對。癸亥,罷太醫局。丙寅,金遣完顏宗寧等來賀明年正旦。是歲,湖南、江東西路旱,振之。



 八年春正月庚午朔,班《乾道敕令格式》。丁酉,朝獻景靈宮,遂幸天竺寺、玉津園。



 二月乙巳,詔改尚書左右僕射、同中書門下平章事為左、右丞相。丙午,詔六察分隸,事
 有違戾,許監察御史隨事具實狀糾劾以聞。戊申,遣姚憲等使金賀上尊號,附請受書之事。辛亥,以虞允文為左丞相,梁克家為右丞相,並兼樞密使。癸丑,以安慶軍節度使張說、吏部侍郎王之奇並簽書樞密院事。侍御史李衡、右正言王希呂交章論說不可為執政,不報。禮部侍郎兼直學士院周必大不草答詔,權給事中莫濟封還錄黃,詔並與在外宮觀。丙辰,詔罷王希呂與遠小監當,尋詔與宮觀。丁巳,李衡罷為起居郎。丙寅,戶部尚
 書曾懷賜出身,參知政事。三月戊子,詔省侍中、中書、尚書令員,以左、右丞充其位。



 夏四月庚子,賜禮部進士黃定以下三百八十有九人及第、出身。己酉,殿中侍御史蕭之敏劾虞允文擅權不公,允文請罷政,許之。翼日復留,出之敏提點江東刑獄。甲子,措置兩淮官田徐子寅等坐授田歸正人逃亡,奪官有差。乙丑,詔再蠲兩淮二稅一年。



 五月戊子,福建鹽行鈔法。丙申,立宗室銓試法。



 六月庚子,以武德郎令揖為金州觀察使,封安定郡
 王。壬寅,蠲兩淮歸正人撮收課子。淮東巡尉有縱逸歸正戶口過淮者,奪官有差。壬子,省監司薦舉員。



 秋七月辛巳,罷淮西屯田官兵,募歸正人耕佃。姚憲、曾覿至自金,金人拒其請。癸未,以覿為武泰軍節度使。壬辰,雨雹。



 九月戊辰,定江西四監鐵錢額。乙亥,詔王炎赴都堂治事。戊寅,以虞允文為少保、武安軍節度使、四川宣撫使,封雍國公。己丑,賜允文家廟祭器。壬辰,允文入辭,帝諭以決策親征,令允文治兵俟報。



 冬十月丁未,遣馮撙等
 使金賀正旦。丙辰,金遣夾谷清臣等來賀會慶節。罷借諸路職田。



 十一月辛未,遣官鬻江、浙、福建、二廣、湖南八路官田。辛巳,復四川諸州教授員。庚寅,進檢校少傅、知福州史浩開府儀同三司。十二月戊戌,蠲兩淮明年租賦。甲辰,詔京西招集歸正人授田如兩淮。甲寅,命四川試武舉。丙辰,追封劉光世為安成郡王。丁巳,遣韓元吉等賀金主生辰。庚申,復置鑄錢司提點官二員。辛酉,金遣曹望之等來賀明年正旦。是歲,隆興府、江、筠州、臨江、興
 國軍大旱,四川水。



 九年春正月辛未,王之奇罷為淮南安撫使,王炎罷為觀文殿大學士、提舉洞霄宮。乙亥,以張說同知樞密院事,戶部侍郎沈夏簽書樞密院事。戊寅,遣官鬻兩浙營田及沒官田,次及江東、西、四川如之。以刑部尚書鄭聞簽書樞密院事。乙酉,福建鹽復官賣法。是月,以措置兩淮、荊襄十六事敕安撫、轉運司督諸州守臣,月具所行事奏,仍審擇臧否,以議黜陟。



 閏月戊申,以久雨,命大理、
 三衙、臨安府及兩浙州縣決系囚,減雜犯死罪以下一等,釋杖以下。乙卯,修廬州城。辛酉,大風。幸天竺寺、玉津園。



 二月壬申,蠲江西旱傷五州逋負米。乙亥,青羌奴兒結寇安靜砦,黎州推官黎商老戰死。乙酉,孫榮國公挺薨,追封豫國公。丁亥,特贈蘇軾為太師。



 三月甲午,禁北界博易銀絹。戊申,從太上皇、太上皇后幸聚景園。癸丑,復以進奏院隸門下後省。丙辰,復分淮南安撫司為東、西路。



 夏四月丁丑,裁定武鋒軍軍額。己丑,皇太子解臨
 安尹事。



 五月壬辰朔,日有食之。己未,以迪功郎朱熹屢詔不起,特改宣教郎、主管臺州崇道觀。



 六月甲戌,禁兩淮、荊襄、四川諸州籍民戶馬。己丑,戒飭監司、守令勸農。



 秋七月壬寅,青羌奴兒結降。辛亥,吐蕃彌羌畜列陷安靜砦,引兵深入,黎州守臣誘邛部川蠻擊卻之。



 八月丙子,詔興修水利。癸未,合荊、鄂二軍為一,以吳挺充都統制。



 九月丙申,梁克家等上《中興會要》、太上皇及皇帝玉牒。庚子,命盱眙軍以受書禮移牒泗州,示金生辰使,金
 使不從。



 冬十月甲子,遣留正等使金賀正旦。右丞相梁克家與同知樞密院張說議使事不合,乃求去。辛未,克家罷為觀文殿大學士、知建寧府。壬申,矞雲見。甲戌,以曾懷為右丞相,張說知樞密院事,鄭聞參知政事,沈夏同知樞密院事。庚辰,金遣完顏襄等來賀會慶節。丁亥,襄等入辭,別函申議受書之禮,仍示虞允文速為邊備。



 十一月辛卯,詔樞密院除授及財賦,事關中書、門下省,其邊機軍政更不錄送。戊戌,合祀天地於圜丘,大赦,改
 明年為淳熙元年。十二月未朔,戒敕沿邊諸軍,毋輒遣間探、招納叛亡。甲子,沈夏罷。乙丑,以御史中丞姚憲簽書樞密院事。遣韓彥直等賀金主生辰。辛未,交址入貢。癸酉,罷廣西客鈔鹽,復官般官賣法。甲戌,遣使措置宜州市馬。乙亥,以嗣濮王士輵、永陽郡王居廣並為少保。乙酉,金遣完顏璋等來賀明年正旦,以議受書禮不合,詔俟改日。以太上皇有旨,姑聽仍舊。丁亥,璋等入見。是歲,浙東、江東西、湖北旱。



 淳熙元年春正月乙未,禁淮西諸關採伐林木。戊戌,罷坐倉糴米賞。庚子,罷兩淮將帥權攝官。丙午,禁兩淮耕牛出境。以交址入貢,詔賜國名安南,封南平王李天祚為安南國王。



 二月癸酉,虞允文薨。辛巳,為郭浩立廟於金州。



 三月戊子朔,詔寄祿官及選人並去左右字。丙申,以鄭聞為資政殿大學士、四川宣撫使。戊申,幸玉津園。癸丑,金遣梁肅等來計事。



 夏四月戊辰,從太上皇幸聚景園。壬申,許桂陽軍溪洞子弟入州學聽讀。乙亥,詔四
 川宣撫司教閱諸州將兵。戊寅,遣張子顏等使金報聘。己卯,以姚憲參知政事,戶部尚書葉衡簽書樞密院事。



 五月壬寅,班鄭興裔所創《檢驗格目》。



 六月丙辰朔,詔禮官討論別建四祖廟,正太祖東向位。戊午,以興州都統制吳挺為定江軍節度使。癸酉,改江陵府為荊南府。戊寅,曾懷罷。癸未,姚憲罷。甲申,落憲端明殿學士,罷宮觀。以葉衡參知政事。



 秋七月丁亥,以鄭聞參知政事。罷四川宣撫司。以成都府路安撫使薛良朋為四川安撫制
 置使。戊子,詔舉廉吏。壬辰,以曾懷為右丞相。己酉,姚憲南康軍居住。



 八月己未,張說罷為太尉,提舉隆興府玉隆觀。以徽猷閣學士楊倓為昭慶軍節度使、簽書樞密院事。



 九月乙酉朔,以曾覿開府儀同三司。壬寅,幸玉津園宴射。乙巳,罷宜州市馬。



 冬十月辛酉,立金銀出界罪賞。壬戌,遣蔡洸使金賀正旦。癸亥,以積雨,命中外決系囚。丙寅,鄭聞薨。乙亥,金遣完顏讓等來賀會慶節。戊寅,占城入貢。辛巳,再蠲臨安府民身丁錢三年。壬午,以魏
 王愷判明州。蠲郴州、桂陽軍借貸常平米。



 十一月甲申朔,日有食之。戊戌,以禮部侍郎龔茂良參知政事。楊倓罷,以葉衡兼權知樞密院事。丙午,曾懷罷。戊申,以葉衡為右丞相兼樞密使。十二月丁巳,以吏部尚書李彥穎簽書樞密院事。壬戌,遣吳琚等賀金主生辰。丙寅,罷鐵錢,改鑄銅錢。庚午,詔禮官論復魏悼王襲封。壬申,葉衡等上《真宗玉牒》。金遣劉仲誨等來賀明年正旦。以資政殿學士、知江陵府沈夏升大學士,為四川宣撫使,仍命
 升差從主帥,場務還軍中。新四川制置使範成大改管內制置使。



 二年春正月癸巳,前宰相梁克家、曾懷坐擅改堂除,克家落觀文殿學士,懷降為觀文殿學士。甲午,廢同安、蘄春監。丁未,以兩淮諸莊歸正人安業,徐子寅等行賞有差。庚戌,詔籍諸軍子弟為背嵬軍。



 三月丙申,以太上皇壽七十,詔禮官討論慶壽典禮。乙巳,詔武舉第一人補秉義郎,堂除諸軍計議官。



 夏四月乙卯,賜禮部進士詹
 騤以下四百二十有六人及第、出身。己巳,幸玉津園。是月,茶寇賴文政起湖北,轉入湖南、江西,官軍數為所敗,命江州都統皇甫倜招之。



 五月辛卯,諭宰相以朝政闕失,士民皆得獻言。庚子,命鄂州都統李川調兵捕茶寇。乙巳,詔知縣三年為任。



 六月庚戌朔,詔自今宰執、侍從以下除外任,非有功績者不除職名,外任人非有勞效,亦不除職。以沈夏同知樞密院事。辛酉,罷四川宣撫司。以倉部郎中辛棄疾為江西提刑,節制諸軍,討捕茶寇。
 丁卯,用左司諫湯邦彥言,落蔣芾、王炎觀文殿大學士,張說落節度使,芾建昌軍、炎袁州、說撫州,並居住。戊辰,振濟湖南、江西被寇州縣。是月,茶寇自湖南犯廣東。



 秋七月辛丑,有星孛於西方。



 八月丙辰,江西總管賈和仲以捕茶寇失律,除名、賀州編管。甲子,賜安南國王印。丁卯,蠲湖南、江西被寇州縣租稅。丁丑,遣左司諫湯邦彥等使金申議。



 九月乙卯朔,湯邦彥請分揚、廬州、荊南、襄陽府、金州、興元府、興州為七路,每路文臣一人充安撫
 使以治民,武臣一人充都總管以治兵,三載視其成以議誅賞。從之。乙酉,振恤淮南水旱州縣。乙未,葉衡罷。丁未,沈夏罷。贈趙鼎為太傅,還其爵邑,追封豐國公。



 閏月丁巳,以李彥穎參知政事,翰林學士王淮簽書樞密院事。甲子,詔武臣從軍毋帶內職。



 是月,辛棄疾誘賴文政殺之,茶寇平。



 冬十月戊寅朔,賞平茶寇功,湖南、江西、廣東監帥黜陟有差。庚辰,大風。壬午,詣德壽宮,加上光堯壽聖憲天體道太上皇帝尊號曰光堯壽聖憲天體道
 性仁誠德經武緯文太上皇帝,壽聖明慈太上皇后尊號曰壽聖齊明廣慈太上皇后。乙酉,遣謝廓然等使金賀正旦。戊戌,金遣完顏禧等來賀會慶節。



 十一月戊申朔,奉上太上皇、太上皇后冊寶於德壽宮。庚戌,麗正門內火。癸丑,大風。戊午,提點坑冶王揖進羨餘十萬緡,詔卻之。十二月辛巳,班《淳熙吏部七司法》。遣張宗元等賀金主生辰。甲午,朝德壽宮,行慶壽禮。大赦。文武官封父母,賞諸軍。議放天下苗稅三之一,大臣言國用不足,乃
 止。丙申,更定強盜贓法。甲辰,金遣完顏迨等來賀明年正旦。



 三年春正月甲寅,以常州旱,寬其逋負之半。刪犯贓蔭補法。振淮東饑,仍命貸貧民種。乙丑,振恤歸正人。



 二月壬午,蠲兩淮教閱民兵夏稅。癸未,以伯圭為安德軍節度使。甲申,詔四川監司、帥守,聞命之官毋候告敕。賜韓世忠謚曰忠武。是月,罷諸路鬻沒官田。



 三月丙午朔,日有食之,霧雲不見。辛亥,上《太上皇日歷》於德壽宮。己未,
 置六部編敕司。癸亥,幸報恩寺,遂幸聚景園。己巳,並左藏四庫為二。辛未,詔四川制置司歲擇梁、洋義士材武者二人,遣赴樞密院。壬申,立任子參選覆試法。



 夏四月戊寅,詔侍從、臺諫、兩省官歲舉監司、郡守各五人。辛巳,靖州人寇邊,遣兵討捕之。丁亥,雨雹。己丑,責授葉衡德軍節度副使、郴州安置。丁酉,湯邦彥、陳雷奉使無狀,除名,邦彥新州、雷永州編管。己亥,詔諸路提刑歲五月理囚。



 五月癸丑,合利州東、西路為一。安南國王李天
 祚卒。戊午,遣使吊祭。壬申,太白晝見。



 六月乙酉,減四川酒課四十七萬餘緡。甲午,以朱熹屢詔不起,特命為秘書郎,熹不就。



 秋七月乙丑,禁浙西圍田。



 八月乙亥,以王淮同知樞密院事,禮部尚書趙雄簽書樞密院事。詔六察官糾察庶務,臺綱益振,各進二官。庚辰,太上皇詔立貴妃謝氏為皇后。壬午,以久雨,命中外決系囚。戊戌,靖州猺寇平。



 九月癸亥,詔自今犯公罪至死者,其蔭補具所犯奏裁,著為令。



 冬十月甲戌,以久雨,命中外決系囚。
 丙子,御文德殿,冊皇后。丁丑,命臨安守臣嚴禁逾侈。庚辰,詔自今非歉歲不許鬻爵。癸未,遣閻蒼舒等使金賀正旦。壬辰,金遣蒲察通等來賀會慶節。



 十一月癸丑,合祀天地於圜丘,大赦。庚午,遣張子正等賀金主生辰。十二月己丑,黎州蠻寇邊,官軍失利,蠻亦遁去。甲午,詔職事官補外者,復除職如故事。追封吳玠為涪王。丁酉,定鑄錢司歲鑄額為十五萬緡。戊戌,金遣劉珫等來賀明年正旦。是歲,京西湖北諸州、興元府、金、洋州旱,紹興府、
 臺、婺州水,並振之。



 四年春正月戊申,詔自今內外諸軍歲一閱試。庚申,詔沿江諸軍歲再習水戰。丙寅,雨雹。丁卯,班《淳熙歷》。



 二月乙亥,幸太學,祗謁先聖,退御敦化堂,命國子祭酒林光朝講《中庸》。下詔,遂幸武學,謁武成王廟。監、學官進秩一等,諸生推恩、賜帛有差。己卯,詔諸軍毋以未補官人任軍職。戊子,立邊人逃入溪洞及告捕法。癸巳,立武臣授環衛官法。戊戌,以新知荊南府胡元質為四川安撫制
 置使兼知成都府。



 三月乙巳,以史浩為少保、觀文殿大學士、醴泉觀使兼侍讀,進封永國公。己酉,龔茂良等上《仁宗玉牒》、《徽宗實錄》、《皇帝玉牒》。庚戌,幸玉津園宴射。壬子,貸隨、郢二州饑民米。詔李龍[A147]襲封安南國王。甲寅、修韶州城。丙寅,幸聚景園。



 夏四月甲戌,以魏王愷為荊南、集慶軍節度使、行江陵尹、判明州如故。乙亥,參知政事龔茂良以曾覿從騎不避道,杖之。戊寅,上奏乞罷政,不許。甲午,給歸正官子孫田屋。



 五月庚子朔,幸祐聖觀。
 罷四川和糴。



 六月丁丑,龔茂良罷。己卯,以王淮參知政事。辛巳,班《幸學詔》。癸未,升蜀州為崇慶府。甲申,詔自今宰執朝殿得旨,事須覆奏乃行。



 秋七月辛丑,禁江上諸軍盜易戰馬。振襄陽饑民。壬寅。立待補太學試法。戊申,班御史臺彈奏格。乙酉,罷臨川伯王雱從祀。癸丑,龔茂良責授寧遠軍節度副使、英州安置。甲寅,申嚴四川入蕃茶禁。甲子,班《淳熙重修敕令格式》。



 八月辛巳,禁耕牛過淮。



 九月丁酉朔,日有食之。己亥,命修築海潮所壞塘
 岸。辛丑,免宰執以下會慶節進奉。庚戌,命禮官定開寶、政和祀禮。戊午,閱蹴踘於選德殿。



 冬十月丙子,以久陰,命中外決系囚。遣錢良臣等使金賀正旦。丁丑,詔監司、守臣歲舉武臣堪知縣者各二人。己卯,詔將士智勇傑出者,躐等升差。丁亥,金遣完顏忠等來賀會慶節。



 十一月丁酉,詔兩淮歸正人為強勇軍。庚子,以趙雄同知樞密院事。壬戌,太白晝見。癸亥,遣趙思等賀金主生辰。十二月丁卯,試四川所上義士二人,官而遣之。己巳,詔行
 薦舉事實格法。乙亥,大閱。辛巳,蠲太平州民貸常平錢米。壬辰,金遣完顏炳等來賀明年正旦。是歲,福州、建寧府、南劍州水,並振之。



\end{pinyinscope}