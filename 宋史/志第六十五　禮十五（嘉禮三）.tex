\article{志第六十五 禮十五(嘉禮三)}

\begin{pinyinscope}

 聖節諸慶節



 聖節。建隆元年,群臣請以二月十六日為長春節。正月十七日,於大相國寺建道場以祝壽,至日,上壽退,百僚
 詣寺行香。尋詔:「今後長春節及諸慶節,常參官、致仕官、僧道、百姓等毋得進奉。」



 太宗以十月七日為乾明節,復改為壽寧節。



 真宗以十二月二日為承天節。其儀:帝禦長春殿,諸王上壽,次樞密使副、宣徽、三司使,次使相,次管軍節度使、兩使留後、觀察使,次節度使至觀察使,次皇親任觀察使以下,各上壽,仍以金酒器、銀香合、馬、袖表為獻。既畢,咸赴崇德殿敘班,宰相率百官上壽,賜酒三行,皆用教坊樂,賜衣一襲,文武群臣、方鎮州軍皆有貢
 禮。前一月,百官、內職、牧伯各就佛寺修齋祝壽,罷日以香賜之,仍各設會,賜上尊酒及諸果,百官兼賜教坊樂。



 景德二年,始令樞密三司使副、學士復赴百官齋會,少卿、監、刺史以上及近職一子賜恩,僧道則賜紫衣、師號,禁屠,輟刑。



 仁宗以四月十四日為乾元節,正月八日皇太后為長寧節。詔定長寧節上壽儀:太后垂簾崇政殿,百官及契丹使班庭下,宰臣以下進奉上壽,閣門使於殿上簾外立侍,百官再拜,宰臣升殿,跪進酒簾外,內臣
 跪承以入。宰臣奏曰:「長寧節,臣等不勝歡抃,謹上千萬歲壽。」復降,再拜,三稱萬歲。內臣承旨宣曰:「得公等壽酒,與公等同喜。」咸再拜。宰臣升殿,內侍出簾外跪授虛盞,宰臣跪受,降,再拜舞蹈,三稱萬歲。內侍承旨宣群臣升殿,再拜,升,陳進奉物當殿庭,通事舍人稱「宰臣以下進奉」,客省使殿上喝「進奉出」。內謁者監進第二盞,賜酒三行,侍中奏禮畢,皆再拜舞蹈。太后還內,百官詣內東門拜表稱賀。其外命婦舊入內者即入內上壽,不入內者
 進表。內侍引內命婦上壽,次引外命婦,如百官儀。次日大宴。



 英宗以正月三日為壽聖節。禮官奏:「故事,聖節上壽,親王、樞密於長春殿,宰臣、百官於崇德殿,天聖諒冱皆於崇政殿。」於是紫宸上壽,群臣升殿間,飲獻一觴而退。又一日,賜宴於錫慶院。



 神宗以熙寧元年四月十日為同天節,以宅憂罷上壽,惟拜表稱賀。明年,親王、樞密使、管軍、駙馬、諸司使副詣垂拱殿,宰臣、百官、大國使詣紫宸殿上壽,命坐,賜酒三行,不舉樂。明年,以大旱,罷同
 天節上壽,群臣赴東上閣門表賀。



 中書門下言:「同天節上壽班,自今樞密使副、宣徽、三司使、殿前馬步軍副都指揮使以上共作一班,進酒一盞;親王、宗室、使相至觀察、駙馬、管軍觀察使以上,皆赴紫宸殿,依本班序立上壽,更不赴垂拱殿。」蓋以管軍觀察使以上及親王、駙馬並於垂拱殿以官序高下各班進酒畢而日晏,外朝有班者仍詣紫宸殿,議者以為近瀆,改焉。而詔袒免以上宗婦聽班賀於禁中。



 哲宗即位,詔以太皇太后七月十
 六日為坤成節。宰臣請以十二月八日為興龍節。哲宗本七日生,以避僖祖忌,故後一日。



 徽宗以十月十日為天寧節,定上壽儀:皇帝御垂拱殿,群臣通班起居畢,分班,從義郎以下醫官、待詔等先退。知引進司官一員讀奏目,知東上閣門官一員奏進壽酒,由東階升,舍人通教坊使以下贊再拜,奏聖躬萬福,又再拜,復位。次看盞人稍前,舍人贊再拜,上殿祗候,分東西兩階立,侯進酒升殿。次舍人引親王入殿庭,北向立,贊再拜,班首奏萬
 福。舍人引進奉西入,列於親王後,酒器簷床置馬前,揖天武躬奏萬福,進奉馬先出。內侍進御茶床,殿中監酹酒訖,知東上閣門官殿上躬奏:「親王某以下進壽酒。」舍人揖親王以下躬贊再拜,乃引親王二員升殿,知東上閣門官引詣御坐前,舍人東階下西向立,後準此。



 尚醞典御奉盤、盞授班首,搢笏受盤、盞,西向立,奉御啟盞,親王一員搢笏注酒,班首奉詣御坐東進訖,少退,虛跪,興,以盤授典御,退,閣門引降階。舍人引當殿北向立,東上,贊
 拜,興,搢笏跪奉表,舍人接表,一員在東,餘詣親王西,置表笏上,授引進。知引進司官殿上讀奏目,退,親王以下俯伏,興,躬,舍人贊再拜,引班首升東階,餘殿下分立。閣門引詣御坐東,北向搢笏,尚醞典御如前奉盤立,樂作,皇帝飲訖,受盞,復位,再拜如上儀。知引進司官詣折檻東,西向宣曰「進奉收」。贊拜,舞蹈,又再拜,西出。親王以下赴紫宸殿立班。引進官宣「進奉出」,天武奉進奉以出。閣門復立殿上,教坊使贊送御酒,又再拜,教坊致語訖,贊
 再拜,退。次樞密官上壽,次管軍觀察以上上壽、進奉並如儀。內侍舉御茶床,舍人贊教坊使以下謝祗應,再拜訖,閣門側奏無公事。



 皇帝赴紫宸殿後閣受群臣上壽。質明,三公以下百僚並於殿門外就次,東上閣門、御史臺、太常寺分引入詣殿庭東西立。閣門附內侍進班齊牌,皇帝出閣,禁衛諸班親從迎駕,自贊常起居。皇帝升坐,鳴鞭,禮直官、通事舍人引三公至執政官,御史臺、東上閣門分引百官,並橫行北向立,典儀贊再拜舞蹈,班
 首奏萬福,又再拜訖,分東西立。禮直官引殿中監、少監升東階,詣酒尊所稍西,南向西上立,舍人揖教坊使以下通班大起居,次看盞人謝升殿,贊再拜。內侍進御茶床,殿侍酹酒訖,禮直官、通事舍人分引三公至執政官,御史臺、東上閣門分引百僚,並橫行北向立,典儀贊再拜,贊者承傳,在位官皆再拜。禮直官、通事舍人引上公升東階,東上閣門官接引升殿,授盞、啟盞如上儀。上公詣御坐,俯伏跪奏:「文武百僚、上公具官臣某等稽首言:
 天寧令節,臣等不勝大慶,謹上千萬歲壽。」俯伏,興,退,降階,舍人接引復位,典儀贊再拜訖,禮直官引知樞密院官詣御坐前承旨,退詣折檻稍東,西向宣曰:「得公等壽酒,與公等內外同慶。」典儀贊拜如儀,百官分東西立。禮直官、通事舍人引上公升東階,東上閣門官接引詣御坐東,搢笏,殿中監授盤,上公奉進御坐東,北向,樂作,皇帝飲訖,閣門引接盞,典儀贊拜如上儀。宗室遙郡以下先退。禮直官引樞密院官詣御坐前承旨,退詣
 折檻稍東,宣曰:「宣群臣升殿。」典儀贊拜訖,禮直官、通事舍人分引三公以下升東階,親王、使相以下升西階;御史臺、東上閣門分引秘書監以下升兩朵殿,並東西廊席後立。尚醞典御以盞授殿中監,奉御啟盞,殿中監西向立,殿中少監以酒注于盞,第二、第三準此。



 奉詣御坐前,躬進訖,少退,奉盤西向立。樂作,皇帝飲訖,殿中監接盞退,授奉御,出笏復位。通事舍人分引殿上官橫行北向,舍人贊再拜,典儀曰「再拜」,贊者承傳,皆再拜。舍人贊就坐,各
 立席後,復贊就坐,群官皆坐。酒初行,先宰臣,次百官,皆作樂。尚食典御、奉御進食,太官設群官食,皇帝再舉酒,群官興,立席後,樂作,飲訖,舍人贊就坐,再行群官酒,皇帝三舉酒,並如第二之儀。酒三行,舍人曰「可起」,群官興,立席後。若宣示盞,即隨東上閣門官以下揖,稱「宣示盞」,躬,贊就坐。若宣勸,即立席後,躬,飲訖,贊再拜。內侍舉御茶床,禮直官引左輔詣御坐前北向俯伏跪奏:「左輔具官臣某言禮畢。」俯伏,興,退,復位。禮直官、通事舍人分引
 三公以下文武百僚降階橫行北向立,樞密院官在親王後。典儀贊再拜,皆舞蹈再拜退。



 靖康元年四月十三日,太宰徐處仁等表請為乾龍節。至日,皇帝帥百官詣龍德宮上壽畢,即本宮賜侍從官以上宴。



 建炎元年五月,宰臣等上言,請以五月二十一日為天申節。詔曰:「朕承祖宗遺澤,獲托士民之上,求所以扶危持顛之道,未知攸濟。念二聖鑾輿在遠,萬民失業,將士暴露,夙夜痛悼,寢食幾廢,況以眇躬之故,聞樂飲酒,以自為樂乎?非
 惟深拂朕志,實增感於朕心。所有將來天申節百官上壽常禮,可令寢罷。」至是止就佛寺啟散祝壽道場,詣閣門或後殿拜表稱賀。



 紹興十三年二月,臣僚奏:「臣聞孝理天下者,帝王之盛德,歸美報上者,臣子之至誠。是皆因性自然,發於觀感,必各盡其至,然後為稱。恭惟陛下撫艱難之運,憂勤在御,兢兢業業,圖濟中興,孝德通於神明,皇天為之悔禍,長樂還闕,適當誕節之前,陛下以天下養,獲伸宮闈上壽之儀,臣民得於觀聽,天下無不
 欣慶,所以崇大養而成孝理之功者,既以盡善盡美矣。陛下誕聖佳辰,乃臣子所願奉觴上壽,以盡歸美之意,其可不舉而行之乎?臣愚,欲望將來天申節許令有司舉行舊典,至日,百官得以奉萬年之觴,仰祝聖壽,天下幸甚。」太常、禮部討論:每遇聖節,樞密院以下先詣垂拱殿上壽畢,宰臣率百僚於紫宸殿上壽。前一月,分日啟建道場,至前一日,樞密院官滿散依例作齋筵。至日,三省官上壽立班訖,次赴滿散作齋筵。後二日,大宴於集
 英殿。時命御史臺、太常寺修立儀注。



 孝宗隆興元年,太上皇帝天申節,皇帝及宰臣、文武百僚詣德壽宮上壽。是日,以欽宗大祥,前一日,皇帝起居如宮中儀,百僚拜表稱賀。



 乾道八年,立皇太子,皇帝率皇太子及文武百僚詣德壽宮上壽。前期,儀鸞司陳設德壽宮殿門之內外,設御坐於殿上當中南向,設大次於德壽宮門內南向,小次於殿東廊西向,設皇帝褥位二:一於御坐東南,西向;一於御坐之南,北向。尚醞設御酒尊、酒器於御坐
 之東,有司又設御茶床於御坐之西,俱稍北。其日,文武百僚內不系從駕者,並先赴德壽宮門外以俟迎駕起居。質明,皇帝服靴袍出即御坐,從駕臣僚、禁衛起居如常儀。皇帝降坐,乘輦將至德壽宮,文武百僚迎駕再拜起居訖,前導官、太常卿、閣門官、太常博士、禮直官先入,詣大次前分左右立定。皇帝降輦入次,御史臺、閣門、太常寺分引皇太子並文武百僚入詣殿廷,東西相向立定,前導官導皇帝入小次,簾降。皇太子並文武百僚並
 橫行北向立。太上皇帝出宮升御坐,鳴鞭,小次簾卷。前導官導皇帝升殿東階,詣殿折檻前北向褥位,再拜,躬奏聖躬萬福,再拜,皇帝詣太上皇帝御坐之東褥位西向立,前導官於殿上隨地之宜立。次舍人揖皇太子並文武百僚躬,典儀曰「再拜」,贊者承傳,在位官皆再拜,搢笏舞蹈,又再拜,皇太子不離位,奏聖躬萬福,各再拜,直身,分東西相向立。禮直官引奉盤盞官、受盤盞官、承旨宣答官、奏禮畢官、殿中監、少監升殿。內侍進御茶床,尚醞
 典御以盤盞、酒注授殿中監、少監,次禮直官引奉盤盞官詣酒尊所北向,殿中監啟盞,殿中少監注酒,奉盤盞官奉酒詣皇帝前北向,禮直官引受盤盞官詣太上皇帝御坐前,西向立,皇太子並文武百僚橫行北向立。奉盤盞官躬進皇帝,皇帝奉酒,前導官導皇帝詣太上皇帝御坐前躬進訖,少後,以盤授受盤盞官。前導官導皇帝詣太上御坐前褥位北向俯伏跪,殿下皇太子並百僚皆躬身。皇帝奏:「臣某謹率文武百僚稽首言:天申令節,臣
 某與百僚等不勝大慶,謹上千萬歲壽。」奏訖,伏,興,再拜,在位官皆再拜。承旨宣答官宣曰:「得皇帝壽酒,與皇帝並百僚內外同慶。」皇帝再拜,在位官皆再拜訖,分東西相向立。皇帝詣御坐東,西向立,奉盤盞官以盤北向恭進,皇帝奉盤,樂作,俟太上皇帝飲酒,皇帝躬接盞訖,皇帝少後,以盤盞授受盤盞官,以授殿中監,各復位立。皇太子並文武百僚橫行北向,皇帝詣褥位北向再拜,在位官皆再拜。皇帝詣太上御坐東褥位西向立,皇太子、
 文武百僚再拜,搢笏舞蹈,又再拜訖,內侍舉茶床,奏禮畢官北向俯伏跪奏:「具官臣某言禮畢。」在位官再拜。太上皇帝駕興,皇帝從入,文武百僚以次退。



 淳熙二年十一月,詔:「太上皇帝聖壽無疆,新歲七十,以十一日冬至加上尊號冊寶,十二月十七日立春行慶壽禮。」是日早,文武百僚並簪花赴文德殿立班,聽宣慶壽赦。宣赦訖,從駕至德壽宮行慶壽禮,致詞曰:「皇帝臣某言:天祐君親,錫茲難老,維春之吉,年德加新。臣某與群臣等不勝
 大慶,謹上千萬歲壽。」餘與前上壽儀注同。禮畢,從駕官、應奉官、禁衛等並簪花從駕還內,文武百僚文德殿拜表稱賀。



 十年十二月,以太上皇后新年七十,詔以立春日行慶賀之禮。十三年春正月朔,以太上皇帝聖壽八十,帝率群臣詣德壽宮行禮,其儀注、恩赦並如淳熙二年典故。



 孝宗以十月二十二日為會慶節,光宗以九月四日為重明節,寧宗以十月十九日為天祐節、尋改為瑞慶節,理宗以正月五日為天基節,度宗以四月九日
 為乾會節,瀛國公以九月二十八日為天瑞節。其上壽稱賀之禮,大略皆如天申節儀。



 諸慶節,古無是也,真宗以後始有之。大中祥符元年,詔以正月三日天書降日為天慶節,休假五日,兩京諸路州、府、軍、監前七日建道場設醮,斷屠宰;節日,士庶特令宴樂,京師然燈。又以六月六日為天貺節,京師斷屠宰,百官行香上清宮。又以七月一日聖祖降日為先天節,十月二十四日降延恩殿日為降聖節,休假、宴樂並
 如天慶節。中書、親王、節度、樞密、三司以下至駙馬都尉,詣長春殿進金縷延壽帶、金絲續命縷,上保生壽酒。改御崇德殿,賜百官飲,如聖節儀。前一日,以金縷延壽帶、金塗銀結續命縷、緋彩羅延壽帶、彩絲續命縷分賜百官,節日戴以入。禮畢,宴百官於錫慶院。天禧初,詔以大中祥符元年四月一日天書再降內中功德閣為天禎節,一如天貺節。尋以仁宗嫌名,改為天祺節。



 政和三年十一月五日,以修祀事,天真示見,詔為天應節。又以五
 月十二日祭方丘日為寧貺節,既又以二月十五日太上混元上德皇帝降聖日為真元節,八月九日青華帝君生辰為元成節,正月四日有太祖神御之州府宮殿行香為開基節,十月二十五日為天符節,皆如天慶節,著為令。



 高宗建炎元年十一月五日,詔:「政和以來添置諸節,除開基節外,餘並依祖宗法。」



\end{pinyinscope}