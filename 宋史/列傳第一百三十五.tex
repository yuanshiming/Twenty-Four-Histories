\article{列傳第一百三十五}

\begin{pinyinscope}

 常同張致遠薛徽言陳淵魏矼潘良貴呂本中



 常同,字子正,邛州臨邛人,紹聖御史安民之子也。登政和八年進士第。靖康初,除大理司直,以敵難不赴,闢元
 帥府主管機宜文字,尋除太常博士。



 高宗南渡,闢浙帥機幕。建炎四年,詔:「故監察御史常安民、左司諫江公望,抗節剛直,觸怒權臣,擯斥至死。今其子孫不能自振,朕甚憫之。」召同至行在,至則為大宗丞。



 紹興元年,乞郡,得柳州。三年,召還,首論朋黨之禍:「自元豐新法之行,始分黨與,邪正相攻五十年。章惇唱於紹聖之初,蔡京和於崇寧之後,元祐臣僚,竄逐貶死,上下蔽蒙,豢成夷虜之禍。今國步艱難,而分朋締交、背公死黨者,固自若也。
 恩歸私門,不知朝廷之尊;重報私怨,寧復公議之顧。臣以為欲破朋黨,先明是非,欲明是非,先辨邪正,則公道開而奸邪息矣。」上曰:「朋黨亦難破。」同對:「朋黨之結,蓋緣邪正不分,但觀其言行之實,察其朋附之私,則邪正分而朋黨破矣。」上曰:「君子小人皆有黨。」同又對曰:「君子之黨,協心濟國;小人之黨,挾私害公。為黨則同,而所以為黨則異。且如元祐臣僚,中遭讒謗,竄殛流死,而後禍亂成。今在朝之士,猶謂元祐之政不可行,元祐子孫不可
 用。」上曰:「聞有此論。」同對以:「禍亂未成,元祐臣僚固不能以自明。今可謂是非定矣,尚猶如此,蓋今日士大夫猶宗京、黼等傾邪不正之論。朋黨如此,公論何自而出?願陛下始終主張善類,勿為小人所惑。」



 又奏:「自古禁旅所寄,必參錯相制。漢有南北軍,周勃用南軍入北軍以安劉氏,唐李晟亦用神策軍以復京師,是其效也。今國家所仗,惟劉光世、韓世忠、張俊三將之兵耳。陛下且無心腹禁旅,可備緩急,頃者苗、劉之變,亦可鑒矣。」除殿中侍
 御史。



 時韓世忠屯鎮江,劉光世屯建康,以私忿欲交兵。同奏:「光世等不思待遇之恩,而驕狠尚氣,無所忌憚,一旦有急,其能相為唇齒乎?望分是非,正國典。昔漢諸侯王有過,猶責師傅,今兩軍幕屬贊畫無狀,乞先黜責。」上以章示兩軍。



 呂頤浩再相,同論其十事,且曰:「陛下未欲遽罷頤浩者,豈非以其有復闢之功乎?臣謂功出眾人,非一頤浩之力。縱使有功,宰相代天理物,張九齡所謂不以賞功者也。」頤浩罷相。論知樞密院宣撫川陜張浚
 喪師失地,遂詔浚福州居住。同與辛炳在臺同好惡,上皆重之。



 金使李永壽等入見,同言:「先振國威,則和戰常在我;若一意議和,則和戰常在彼。」上因語及武備曰:「今養兵已二十萬。」同奏:「未聞二十萬兵而畏人者也。」



 偽齊宿遷令張澤以二千人自拔來歸,泗州守徐宗誠納之,韓世忠以聞。朝論令世忠卻澤等,而械宗誠赴行在。同奏:「敵雖議和,而兩界人往來未嘗有禁,偽齊尚能置歸受館,立賞以招吾民,今乃卻澤,人心自此離矣。況宗誠
 起土豪,不用縣官財賦,募兵自養,為國障捍,今因受澤而械之,以沮士氣,非策也。」詔處來歸者於淮南,釋宗誠罪。



 四年,除起居郎、中書舍人、史館修撰。先是,同嘗上疏論神、哲二史曰:「章惇、蔡京、蔡卞之徒積惡造謗,痛加誣詆,是非顛倒,循致亂危。在紹聖時,則章惇取王安石《日錄》私書改修《神宗實錄》;在崇寧後,則蔡京盡焚毀《時政記》、《日歷》,以私意修定《哲宗實錄》。其間所載,悉出一時奸人之論,不可信於後世。恭惟宣仁保祐之德,豈容異辭,
 而蔡確貪天之功,以為己力,厚誣聖後,收恩私門。陛下即位之初,嘗下詔明宣仁安社稷大功,令國史院摭實刊修,又復悠悠。望精擇史官,先修《哲宗實錄》,候書成,取《神宗朱墨史》考證修定,庶毀譽是非皆得其實。」上深嘉納。至是,命同修撰,且諭之曰:「是除以卿家世傳聞多得事實故也。」一日奏事,上愀然曰:「向昭慈嘗言,宣仁有保祐大功,哲宗自能言之,正為宮中有不得志於宣仁者,因生誣謗。欲辨白其事,須重修《實錄》,具以保立勞效,昭
 示來世,此朕選卿意也。」同乞以所得聖語宣付史館,仍記於《實錄》卷末。



 張俊乞復其田產稅役,令一卒持書瑞昌,而凌悖其令郭彥參,彥參系之獄。後訴於朝,命罷彥參,同並封還二命。俄除集英殿修撰、知衢州,以疾辭,除徽猷閣待制、提舉江州太平觀。



 七年秋,以禮部侍郎召還。未數日,除御史中丞。車駕自建康回臨安,同奏:「旋蹕之初,去淮益遠,宜遣重臣出按兩淮,詢人情利病,察官吏侵擾,縱民耕墾,勿收租稅。數年之後,田野加闢,百姓
 足而國亦足矣。」乃遣樞密使王庶視師,同乞以此奏付庶,詢究罷行。又言:「江浙困於月樁錢,民不聊生。」上為減數千緡。又言:「吳



 玠屯師興、利,而西川人力已困。玠頃年嘗講屯田,願聞其積穀幾何,減饋運幾何,趙開、李迨相繼為都漕,先後饋運各幾何,令制、漕、帥司條具以聞,然後按實講究,以紓民力。」又言:「國家養兵,不為不多,患在於偏聚而不同力,自用而不同心。今韓世忠在楚,張俊在建康,岳飛在江州,吳玠在蜀,相去隔遠,情不相通。今
 陛下遣樞臣王庶措置邊防,宜令庶會集將帥,諭以國體,協心共議禦敵,常令諸軍相接以常山蛇勢,一意國家,無分彼此,緩急應援,皆有素定之術。」詔付王庶出示諸將。



 同乞郡,除顯謨閣直學士、知湖州。復召,請祠,詔提舉江州太平觀。紹興二十年卒。



 張致遠,字子猷,南劍州沙縣人。宣和三年,中進士第。宰相範宗尹薦其才,召對,擢為樞密院計議官。建寇範汝為已降,猶懷反側,而招安官謝向、陸棠受賊賂,陰與之
 通。致遠謁告歸,知其情,還白執政,請鋤其根蘗,於是捕響、棠及制置司屬官施宜生付獄。詔參知政事孟庾為福州宣撫使討賊,韓世忠副之,闢致遠為隨軍機宜文字。賊平,除兩浙轉運判官,改廣東轉運判官。招撫劇盜曾兗等,賊眾悉降。



 紹興四年,以監察御史召。未至,除殿中侍御史。時江西帥胡世將請增和買絹折納錢,致遠上疏言:「折納絹錢本欲少寬民力,而比舊增半,是欲乘民之急而厚其斂也。」從之。



 金人與劉豫分道入寇,宰相
 趙鼎勸高宗親征,朝士尚以為疑,白鼎審處。致遠入對,獨贊其決。遷侍御史。言:「聚財養兵,皆出民力,善理財者,宜固邦本。請罷榷福建鹽,精擇三司使、副,以常平茶鹽合為一官,令計經常,量入為出,先務省節,次及經理。」詔戶部講究。



 五年,除戶部侍郎,進吏部侍郎,尋復為戶部侍郎。言:「陛下欲富國強兵,大有為於天下,願詔大臣力務省節,明禁僭侈,自宮禁始,自朝廷始。額員可減者減之,司屬可並者並之。使州縣無妄用,歸其餘於監司;監
 司無妄用,歸其餘於朝廷;朝廷無橫費,日積月聚,惟軍須是慮,中興之業可致也。」除給事中。



 尋以老母丐外,以顯謨閣待制知臺州。朝廷以海寇鄭廣未平,改知福州。六年八月,廣等降,致遠選留四百人,置營城外,餘遣還業。復遣廣討他郡諸盜,數月悉平。



 八年正月,再召為給事中。出知廣州。尋以顯謨閣待制致仕。十七年卒,年五十八。



 致遠鯁亮有學識,歷臺省、侍從,言論風旨皆卓然可觀。趙鼎嘗謂其客曰:「自鼎再相,除政府外,從官如張
 致遠、常同、胡寅、張九成、潘良貴、呂本忠、魏矼皆有士望,他日所守當不渝。」識者謂鼎為知人云。



 薛徽言,字德老,溫州人。登進士第,為樞密院計議官。紹興二年,遣使分行諸路,徽言在選中,以權監察御史宣諭湖南。時郴、道、桂陽旱饑,徽言請於朝,不待報即諭漕臣發衡、永米以振,而以經制銀市米償之,所刺舉二十人。使還,他使皆進擢,宰相呂頤浩以徽言擅易守臣,而移用經制銀,出知興國軍。入為郎、遷右司,擢起居舍人。
 時秦檜與金人議和,徽言與吏部侍郎晏敦復等七人同拜疏爭之。一日,檜於上前論和,徽言直前引義固爭,反復數刻。中寒疾而卒。高宗念之,賻絹百匹,特與遺表恩。



 陳淵,字知默,南劍州沙縣人也。紹興五年,給事中廖剛、中書舍人胡寅朱震、權戶部侍郎張致遠言:「淵乃瓘之諸孫,有文有學,自瓘在時,器重特甚,垂老流落,負材未試。」充樞密院編修官。會李綱以前宰相為江南西路安
 撫制置大使,闢為制置司機宜文字。



 七年,詔侍從舉直言極諫之士,胡安國以淵應。召對,改官,賜進士出身。九年,除監察御史,尋遷右正言。入對,論:「比年以來,恩惠太濫,賞給太厚,頒賚賜予之費太過。所用既眾,而所入實寡,此臣所甚懼也。《周官》『唯王及後、世子不會』,說者謂不得以有司之法治之,非周公作法開後世人主侈用之端也。臣謂塚宰以九式均節財用,有司雖不會,塚宰得以越式而論之。若事事以式,雖不會猶會也。臣願陛下
 凡有錫賚,法之所無而於例有疑者,三省得以共議,戶部得以執奏,則前日之弊息矣。」



 淵面對,因論程頤、王安石學術同異,上曰:「楊時之學能宗孔、孟,其《三經義辨》甚當理。」淵曰:「楊時始宗安石,後得程顥師之,乃悟其非。」上曰:「以《三經義解》觀之,具見安石穿鑿。」淵曰:「穿鑿之過尚小,至於道之大原,安石無一不差。推行其學,遂為大害。」上曰:「差者何謂?」淵曰:「聖學所傳止有《論》、《孟》、《中庸》,《論語》主仁,《中庸》主誠,《孟子》主性,安石皆暗其原。仁道至大,《論語》
 隨問隨答,惟樊遲問,始對曰:『愛人。』愛特仁之一端,而安石遂以愛為仁。其言《中庸》,則謂中庸所以接人,高明所以處己。《孟子》七篇,專發明性善,而安石取揚雄善惡混之言,至於無善無惡,又溺於佛,其失性遠矣。」



 鄭億年復資政殿學士、奉朝請,召見於內殿。淵言:「億年故相居中之子,雖為從官,而有從賊之醜,乞浸其職名。」不報。億年,右僕射秦檜之親黨也,由是檜怒之。除秘書少監兼崇政殿說書,以祖名辭。改宗正少卿,以何鑄論罷。主管臺
 州崇道觀。十五年,卒。



 魏矼,字邦達,和州歷陽人,唐丞相知古後也。少穎悟。時方尚王氏新說,矼獨守所學。宣和三年,上舍及第。建炎四年,召赴闕,詔改宣教郎,除詳定一司敕令所刪定官。



 紹興元年,遷樞密院計議官,遷考功郎。會星變,矼因轉對,言:「治平間,彗出東方,英宗問輔臣所以消弭之道,韓琦以明賞罰為對。比年以來,賞之所加,有未參選而官已升朝者,有未經任而輒為正郎者,罰之所加,有未到
 任而例被沖替者,有罪犯同而罰有輕重者。」力言大臣黜陟不公,所以致異。上識其忠,擢監察御史,遷殿中侍御史。



 臨安火,延燒數千家,獻諛者謂非災異。矼言:「《春秋》定、哀間數言火災,說者謂孔子有德而魯不能用,季孫有惡而不能去,故天降之咎。今朝廷之上有奸慝邪佞之人未逐乎?百執事之間有朋附奔競之徒未汰乎?搢紳有公忠宿望及抱道懷藝、有猷有守之士未用乎?在位之人,畏人軋己,方且蔽賢,未聞推誠盡公,旁招俊乂。
 宜鑒定、哀之失,甄別邪正,亟加進用。」



 內侍李暠飲韓世忠家,刃傷弓匠,事下廷尉。矼言:「內侍出入宮禁,而狠戾發於杯酒,乃至如此,豈得不過為之慮?建炎詔令禁內侍不得交通主兵官及預朝政,違者處以軍法。乞申嚴其禁,以謹履霜之戒。」於是廙杖脊配瓊州。遷侍御史,賜矼五品服。



 時朱勝非獨相,矼論:「勝非無所建明,惟知今日進呈一二細故,明日啟擬一二故人,而機務不決,軍政不修,除授挾私,賢士解體。」又疏其五罪,詔令勝非持
 餘服。又言:「國家命令之出,必先錄黃。其過兩省,則給舍得以封駁;其下所屬,則臺諫得以論列。此萬世良法也。竊聞近時三省、樞密院,間有不用錄黃而直降指揮者,亦有雖畫黃而不下六部者,望並依舊制。」



 劉豫挾金人入寇,宰相趙鼎決親征之議,矼請扈從,因命督江上諸軍。時劉光世、韓世忠、張俊三大將權均勢敵,又懷私隙,莫肯協心。矼首至光世軍中,諭之曰:「賊眾我寡,合力猶懼不支,況軍自為心,將何以戰?為諸公計,當思為國雪
 恥,釋去私隙,不獨有利於國,亦將有利其身。」光世許之,遂勸其貽書二帥,示以無他,二帥復書交歡。光世以書聞,由此眾戰屢捷,軍聲大振。



 上至平江,魏良臣、王繪使金回,約再遣使,且有恐迫語。矼請罷「講和」二字,飭厲諸將,力圖攻取。會金屢敗遁去,使亦不遣。遷秘書少監。



 矼在職七閱月,論事凡百二十餘章。尋乞補外,除直龍圖閣、知泉州,以親老辭,知建州。尋召還,丐祠,不允,除權吏部侍郎。



 八年,金使入境,命矼充館伴使,矼言:「頃任御史,
 嘗論和議之非,今難以專論。」秦檜召矼至都堂,問其所以不主和之意,矼具陳敵情難保,檜諭之曰:「公以智料敵,檜以誠待敵。」矼曰:「相公固以誠待敵,第恐敵人不以誠待相公耳。」檜不能屈,乃改命吳表臣。



 詔金使入境,欲屈己就和,令侍從、臺諫條奏來上。矼言:「臣素不熟敵情,不知使人所需者何禮,陛下所以屈己者何事。賊豫為金人所立,為之北面,陛下承祖宗基業,天命所歸,何藉於金國乎?傳聞奉使之歸,謂金人悉從我所欲,必無難
 行之禮,以重困我,陛下何過自取侮乎?如或不可從之事,儻輕許之,他時反為所制,號令廢置將出其手,一有不從,便生兵隙。予奪在彼,失信在我,非計之得也。雖使還我空地,如之何而可保?雖欲寢兵,如之何而可寢?雖欲息民,如之何而可息?非計之得也。陛下既欲為親少屈,更願審思天下治亂之機,酌之群情,擇其經久可行者行之,其不可從者,以國人之意拒之,庶無後悔。所謂國人者,不過萬民、三軍爾。搢紳與萬民一體,大將與三
 軍一體,今陛下詢於搢紳,民情大可見矣。欲望速召大將,各帶近上統制官數人同來,詳加訪問,以塞他日意外之憂。大將以為不可,則其氣益堅,何憂此敵。」



 未幾,丁父憂。免喪,除集英殿修撰、知宣州,不就。改提舉太平興國宮,自是奉祠,凡四任。丁內艱以卒。



 潘良貴,字子賤,婺州金華人。以上舍釋褐為闢雍博士,遷秘書郎。時宰相蔡京與其子攸方以爵祿鉤知名士,良貴屹然特立,親故數為京致願交意,良貴正色謝絕。
 除主客郎中,尋提舉淮南東路常平。



 靖康元年,召還。賜對,欽宗問孰可秉鈞軸者,良貴極言:「何□、唐恪等四人不可用,他日必誤社稷。陛下若欲扶危持顛之相,非博詢於下僚,明揚於微陋,未見其可。」語徹於外,當國者指為狂率,黜監信州汭口排岸。



 高宗即位,召為左司諫。既見,請誅偽黨,使叛命者受刃國門,即敵人不敢輕議宋鼎。又乞封宗室賢者於山東、河北,以壯國體,巡幸維揚,養兵威以圖恢復。黃潛善、汪伯彥惡其言,改除工部。良
 貴以不得其言,求去,主管明道宮。



 越數年,除提點荊湖南路刑獄,主管江州太平觀,除考功郎,遷左司。宰相呂頤浩從容謂良貴曰:「旦夕相引入兩省。」良貴正色對曰:「親老方欲乞外,兩省官非良貴可為也。」退語人曰:「宰相進退一世人才,以為賢邪,自當擢用,何可握手密語,先示私恩。若士大夫受其牢籠,又何以立朝。」即日乞補外,以直龍圖閣知嚴州。到官兩月,請祠,主管亳州明道宮。起為中書舍人。



 會戶部侍郎向子諲入見,語言煩褻,良
 貴故善子諲,是日攝起居,立殿上,徑至榻前厲聲曰:「子諲以無益之談久煩聖聽!」子諲欲退,高宗顧良貴曰:「是朕問之。」又諭子諲且款語。子諲復語,久不止,良貴叱之退者再。高宗色變,閣門並彈之,於是二人俱待罪。有旨良貴放罪,子諲無罪可待。



 良貴求去,以集英殿修撰提舉江州太平觀。起知明州。期年,除徽猷閣待制、提舉亳州明道宮。既歸,不出者十年。李光得罪,良貴坐嘗與通書,降三官。卒,年五十七。



 良貴剛介清苦,壯老一節。為博
 士時,王黼、張邦昌俱欲妻以女,拒之。晚家居貧甚,秦檜諷令求郡,良貴曰:「從臣除授合辭免,今求之於宰相,辭之於君父,良貴不敢為也。」其諫疏多焚稿,僅存雜著十五卷,新安朱熹為之序。



 呂本中字居仁,元祐宰相公著之曾孫、好問之子。幼而敏悟,公著奇愛之。公著薨,宣仁太后及哲宗臨奠,諸童稚立庭下,宣仁獨進本中,摩其頭曰:「孝於親,忠於君,兒勉焉。」



 祖希哲師程頤,本中聞見習熟。少長,從楊時、游酢、
 尹焞游,三家或有疑異,未嘗茍同。以公著遺表恩,授承務郎。紹聖間,黨事起,公著追貶,本中坐焉。



 元符中,主濟陰簿、秦州士曹掾,闢大名府帥司干官。宣和六年,除樞密院編修官。靖康改元,遷職方員外郎,以父嫌奉祠。丁父憂,服除,召為祠部員外郎,以疾告去。再直秘閣,主管崇道觀。



 紹興六年,召赴行在,特賜進士出身,擢起居舍人兼權中書舍人。內侍李琮失料歷,上以潛邸舊人,不用保任特給之。本中言:「若以異恩別給,非所謂『宮中府
 中當為一體』者。」上見繳還,甚悅,令宰臣諭之曰:「自今有所見,第言之。」



 監階州草場苗亙以贓敗,有詔從黥,本中奏:「近歲官吏犯贓,多至黥籍,然四方之遠,或有枉濫,何由盡知?異時察其非辜,雖欲抆拭,其可得乎?若祖宗以來此刑嘗用,則紹聖權臣當國之時,士大夫無遺類久矣。願酌處常罰,毋令奸臣得以借口於後世。」從之。



 七年,上幸建康,本中奏曰:「當今之計,必先為恢復事業,求人才,恤民隱,講明法度,詳審刑政,開直言之路,俾人人得
 以盡情。然後練兵謀帥,增師上流,固守淮甸,使江南先有不可動之勢,伺彼有釁,一舉可克。若徒有恢復之志,而無其策,邦本未強,恐生他患。今江南、兩浙科須日繁,閭里告病,倘有水旱乏絕,奸宄竊發,未審朝廷何以待之?近者臣庶勸興師問罪者,不可勝數,觀其辭固甚順,考其實不可行。大抵獻言之人,與朝廷利害絕不相侔,言不酬,事不濟,則脫身而去。朝廷施設失當,誰任其咎?鷙鳥將擊,必匿其形,今朝廷於進取未有秋毫之實,所
 下詔命,已傳賊境,使之得以為備,非策也。」又奏:「江左形勢如九江、鄂渚、荊南諸路,當宿重兵,臨以重臣。吳時謂西陵、建平,國之藩表,願精擇守帥,以待緩急,則江南自守之計備矣。」



 內侍鄭諶落致仕,得兵官。本中言:「陛下進臨江滸,將以有為,今賢士大夫未能顯用,巖穴幽隱未能招致,乃起諶以統兵之任,何邪?」命遂寢。引疾乞祠,直龍圖閣、知臺州,不就,主管太平觀。召為太常少卿。



 八年二月,遷中書舍人。三月,兼侍講。六月,兼權直學士院。金
 使通和,有司議行人之供,本中言:「使人之來,正當示以儉約,客館芻粟若務充悅,適啟戎心。且成敗大計,初不在此,在吾治政得失,兵財強弱,願詔有司令無乏可也。」



 初,本中與秦檜同為郎,相得甚歡。檜既相,私有引用,本中封還除目,檜勉其書行,卒不從。趙鼎素主元祐之學,謂本中公著後,又範沖所薦,故深相知。會《哲宗實錄》成,鼎遷僕射,本中草制,有曰:「合晉、楚之成,不若尊王而賤霸;散牛李之黨,未如明是以去非。」檜大怒,言於上曰:「本
 中受鼎風旨,伺和議不成,為脫身之計。」風御史蕭振劾罷之。提舉太平觀,卒。學者稱為東萊先生,賜謚文清。



 有詩二十卷得黃庭堅陳師道句法,《春秋解》一十卷、《童蒙訓》三卷、《師友淵源錄》五卷,行於世。



 論曰:《傳》有之:「不有君子,其何能國。」紹興之世,呂頤浩、秦檜在相位,雖有君子,豈得盡其忠,宋之不能圖復中原,雖曰天命,豈非人事乎?若常同、張致遠、薛徽言、陳淵、魏矼、潘良貴、呂本中,其才猷皆可以經邦,其風節皆可以
 厲世,然皆論議不合,奉祠去國,可為永慨矣。



\end{pinyinscope}