\article{列傳第一百一十三}

\begin{pinyinscope}

 沈銖弟錫路昌衡謝文瓘陸蘊黃寔姚祐樓異沉積中李伯宗汪澥何常葉祖洽時彥霍端友俞慄蔡薿



 沈銖,字子平,真州揚子人。父季長,王安石妹婿也。銖少從安石學,進士高第,至國子直講。季長領監事,改審官主簿,坐虞蕃事免歸。元祐置訴理所,被罪者爭自列,銖獨不言。



 紹聖初,起為太學博士、秘書省正字、崇政殿說書,受旨同編類元祐臣僚章疏。以進講為解,拜右司諫,辭,改起居郎、權中書舍人。吳居厚除戶部尚書,銖論其使京東時聚斂,詔具實狀,不能對,罰金。講《詩·南山有臺》,至「萬壽無期」,以為此太平之基,立而可久之應,哲宗屢
 首肯之。真拜中書舍人兼侍講,俄引疾,以龍圖閣待制知宣州卒。弟錫。



 錫字子昭,以王安禮任,為鄂州司戶參軍。崇寧初,為講議司檢討。蔡京方銓次元符上書人,欲定罪,錫曰:「遠方之士,未能知朝廷好惡,若概罪之,恐非敦世厲俗之道。」京不從。除衛尉丞,遷祠部員外郎,提點江東刑獄、知婺州。入為左司員外郎,兼定、嘉二王侍講,進太常少卿,拜兵部侍郎,以徽猷閣待制知應天府,徙江寧。



 張懷素誅,
 朝廷疑其黨有脫者,江、淮間往往以誣告興獄。錫至郡,有告者,按之,則妄也。具疏於朝,由是他郡系者皆得釋。歷知海、泰、汝、宣四州,以通議大夫致仕。卒,贈宣奉大夫。



 路昌衡,字持正,開封祥符人。起進士,至太常博士。參鞫陳世儒獄,逮治苛峻,至士大夫及命婦,皆不免。遷右司員外郎,歷江淮發運、陜西轉運副使,知廣州,徙荊南,又徙潭州,加直龍圖閣、知慶州。



 紹聖中,召為衛尉、大理卿,遷工部侍郎,俄以寶文閣待制知開封府。李清臣有狂
 婦人之訴,昌衡致之重闢。出知瀛州,徙永興軍,進直學士、知成都。



 徽宗立,應詔上書曰:「頻年以來,西方用兵,致興大役,利源害政,佞臣蔽主,四者皆陰之過盛。自陜以西,民力傷殘,人不聊生。災異之變,生於天地之不和,起於人心之怨望。故妖星出見,大河橫決,秋雨霖淫,諸路饑饉,殍死道路,妻子棄捐,破析貲儲,以應星火之令。勤勞憔悴,多不生還,人心如此,而欲其無怨,難矣。」



 俄坐清臣獄事,責司農少卿,分司,居郢州。明年,起為滁州、定州,復
 直學士、知開封府。乞嚴告捕虛妄之法,以靖訐訴。徙南京留守,又坐前上書事落職,入黨籍,卒。宣和五年,贈龍圖閣學士。



 謝文瓘,字聖藻,陳州人。進士甲科,教授大名府。元豐中,上疏言:「臣下推行新法,多失本意,而榜笞禁錮,民受其虐,掊克聚斂,不勝多門。其不急之徵,非理之取,宜罷減之。」大臣以為訕朝廷,議置之罪。神宗曰:「彼謂奉法者非其人爾,匪訕也。」



 哲宗時,御史中丞黃履薦為主簿,三年
 不詣執政府。召對,除秘書省正字,考功、右司員外郎。紹聖末,都水使者議建廣武四埽石岸,朝廷命先治岸數十步,以驗其可否。黃流湍悍,役人多死,一方甚病,功不可成,而使者申前說愈力。文瓘條別利害,罷其役。



 徽宗立,擢起居舍人、給事中。詔修《神宗寶訓》,文瓘請擇當時大政事、大黜陟,節其要旨,而為之說以進。然所論率是王安石,謂神宗能察眾多之謗,任之而不貳,於是朋黨消而威柄立,他皆放此。遼主洪基殂,使往吊之,令從
 者變服而入,貶秩二等。



 崇寧元年,出知濮州。尋治黨事,坐元豐上疏及嘗詒呂公著書,再謫邵武軍,移處州。帝披黨籍曰:「朕究知文瓘本末。」命出籍,乃以為集英殿修撰、知濟州,卒。



 子貺,宣和中,為駕部員外郎、知汝州。欽宗時,上封事十篇,論事切至。使於金,還,提點京西北路刑獄。金人犯汝州,貺自襄陽領兵往援之,戰死。



 陸蘊,字敦信,福州侯官人。少知名,登進士第,為太學《春秋》博士。經廢員省,改國朝會要所檢閱文字。



 崇寧中,提
 舉河北、兩浙學事,召對,言:「元祐異意俗學,既不為我用,近詔不以使一路,而猶得為守令,臣愚未知其可。」遂拜禮部員外郎,轉吏部,遷闢雍司業、太常少卿。議原廟不合,黜知瑞金縣。還為太常,進國子祭酒、中書舍人。請葺諸州天慶觀,立學事司考課法。遷大司成,擢御史中丞。引門下侍郎餘深親嫌自列,徽宗曰:「相避之法,防有司不能盡公爾,侍從吾所信任,豈得下同庶僚乎?」不許。



 蘊頗論事,嘗言:御筆一日數下,而前後相違,非所以重命
 令;輔相大臣,宦官戚里,賜第營築,縱撤民居,縣官市材於民,而不予直;貴游子弟以從官領閑局,奉朝請,為員猥多,無益於事;又賜予過制,中外用度多於賦入;數幸私室,乖尊卑之分,亦非臣下之福。其言皆中時病。



 以龍圖閣待制知福州,改建州。時弟藻由列曹侍郎出為泉州,過蘊,合樂燕款,閩人以為盛事。加顯謨閣直學士,引疾,提舉鴻慶宮。方二浙用兵,旁郡皆繕治守備,蘊聞命就道,使者劾為避事,奪職。稍復集英殿修撰,
 卒。



 黃寔,字師是,陳州人。登進士第,歷司農主簿,積官提舉京西、淮東常平。元豐末,議罷提舉官,命末布,寔舅章惇屬蔡確徙寔提點開封縣鎮。遷提點梓州路、兩浙刑獄,京東、河北轉運副使。



 哲宗以寔為監司久,議召用,曾布陰沮之。林希曰:「寔兩女皆嫁蘇軾子,所為不正,不宜用。」乃以知陜州,為江、淮發運副使。賀遼主登位,及境,迓者移牒來,稱為賀登寶位使。寔報以受命無「寶」字,拒不受。還除太僕卿,再擢寶文閣待制、知瀛州,徙定州。朝旨籍
 民兵旁郡,因緣擾困,寔懷檄不下,而畫利害請之,事得寢。卒於官,贈龍圖閣直學士。



 寔孝友敦睦,世稱其內行。蘇轍在陳與寔游,因結昏,其後又與軾友善。紹聖黨禍起,寔以章惇甥故獲免,然亦不得久於朝著焉。



 姚祐,字伯受,湖州長興人。元豐末,第進士。徽宗初,除夔州路轉運判官。且行,會帝幸禁苑御弓矢,祐奏《聖武臨射賦》。帝大悅,留為右正言。歷陳紹述之說,遷左司諫。建議置輔郡以拱大畿,進殿中監。六尚局官制成,凡所以
 享上率屬、察舉稽違、殿最勤惰之法,皆祐裁定。以親老請郡,授顯謨閣待制、知江寧府。時召捕張懷素,祐追獲之,復為殿中監。



 逾歲,以直學士知鄭州,改秦州。或請調熙河弓箭士徙邊,以省更戍。祐謂人情懷土重遷,丐以二年為更發之期,滿歲樂業而願留者,乃聽。且請擇熙、秦富民分丁授地,蠲役借糧,以勸耕植。益廣秦之東、西川,建城壁,嚴保障,以控熙河、涇原。皆從之。復為殿中監,改吏部侍郎,命鎮蜀,用母老辭。遷工部尚書,加龍圖閣
 學士,為大名尹,進延康殿學士,復為工部尚書,徙禮部。母喪,除知太原府。



 縣有小胥造塚逼其先墓者,祐疑為厭己,請解官持服。先是,詔許祐悉買墓旁地,遂並徙他塚,小胥不從,故祐持以為說。言者論其挾仇要君,乃止。以提舉上清寶菉宮卒,贈特進,謚曰文禧。」



 樓異,字試可,明州奉化人。進士高第,調汾州司理參軍,徙永興虞策幕府,監在京文繡院,知大宗正丞,遷度支員外郎。以養親求知泗州,復為吏部右司員外郎、左司
 郎中、太府鴻臚卿,除直秘閣、知秀州。



 政和末,知隨州,入辭,請於明州置高麗一司,創百舟,應使者之須,以遵元豐舊制。州有廣德湖,墾而為田,收其租可以給用。徽宗納其說。改知明州,賜金紫。出內帑緡錢六萬為造舟費,治湖田七百二十頃,歲得穀三萬六千。加直龍圖閣、秘閣修撰,至徽猷閣待制。郡資湖水灌溉,為利甚廣,往者為民包侵,異令盡洩之墾田。自是苦旱,鄉人怨之。



 在郡五年,既請溫之船官自隸以便役,又請越、臺之鹽以佐
 費,詔責之曰:「郡自有鹽筴不能興,而欲東取諸臺,西取諸越,斯乃以鄰國為壑也。」睦寇起,善理城戍有績,進徽猷閣直學士、知平江府,卒。



 沈積中,常州人。賜進士出身,為闢雍正、戶部員外郎,至秘閣修撰、河北轉運使,召拜戶部侍郎,進尚書,知河間、真定府。積中本王黼所引拔,黼方圖燕地,使覘邊隙。中書舍人程振語之曰:「當思異時覆族之禍。」積中感其戒,至鎮,以書謝振,盛言其不可,振宣告於朝。已而師敗於
 白溝,童貫還,罷積中提舉上清寶菉宮。既得燕山,又命以資政殿學士同知府,未行而卒,或曰為盜所殺,或曰婢殺之,終亦不能明也。貫惡其曩言,追削官職。建炎中,宰相上其書,乃悉復之。



 李伯宗,字會之,河陽人。第進士,知內丘、咸陽、太康縣。建言:「朝廷行方田均稅之法,令以豐歲推行。今州縣吏,茍簡懷異者指熟為災,而貪進幸賞者掩災為熟,望深察其違戾,而置諸罰。」括縣壯丁為兵,得千人,上其名數與
 按閱之法。知樞密院蔡卞喜而薦之,提舉京畿保甲,使行其說,增籍二萬。已而有訴者,陳牒至八百七十,左遷通判相州、提舉白波輦運,提點江、淮坑冶鑄錢,入為將作少監。



 開封民有鬻神祠故帽飾以龍者,吏以為乘輿服御,伯宗曰:「此無他,當坐不應為爾。」尹不從,具以請,如伯宗議。歷大理卿,入對言:「今情重法輕者許奏請,而情輕法重者不得焉,恐非仁聖忠恕之意。」徽宗納之。遷刑部侍郎。與王黼不相能,有胥吏微過罷,提舉崇福宮。



 明年,知同州,徙陜西都轉運使。以通奉大夫、顯謨閣待制卒,贈光祿大夫,謚曰榮。



 汪澥字仲容,宣州旌德人。少從胡瑗學《易》。又學於王安石,著《三經義傳》,澥與其議,又首傳其說。熙寧太學成,分錄學政。登進士第,調鼎州司理參軍、知黟縣,入為太學正,累遷國子祭酒,兼定、嘉二王翊善,擢中書舍人,為大司成。議學制不合,以顯謨閣待制知婺州,改穎昌,又改陳、壽二州,徙應天府。上章辭行,提舉崇福宮。卒,贈
 宣奉大夫。



 澥自布衣錄天子學,至為正,為司業、祭酒,迄於司成,官以儒名者三十年,一時人士推之。



 何常,字德固,京兆人。中進士第,為開封府兵曹。紹聖初,或言蘇軾主文柄,取士之非毀宗廟者,常預其間,出通判原州。歷將作丞、陜西轉運判官、熙河轉運副使。議者欲貸民金帛,而使入粟塞下。常曰:「車牛轉輸,民力已病,然未至於死亡者,粟自官出,而民無害也。今強以金帛,使自入粟,懼非貧弱之利。」熙帥及監軍劾之,貶秩,徙成
 都路。



 中使持御札至,令織戲龍羅二千,繡旗五百。常奏:「旗者,軍器之飾,敢不奉詔。戲龍羅唯供御服,日衣一匹,歲不過三百有奇;今乃數倍,無益也。」詔獎其言,為減四之三。



 除直龍圖閣,加集賢殿修撰,為使徙陜西,以顯謨閣待制知秦州,轉通議大夫。諜告夏人多築堡柵,朝議出兵牽制,常言:「羌人生長射獵,今困於版築,違所長,用所短,可以拱手待其弊,無煩有為也。」從之。



 鎮秦六歲,察訪方邵劾其越法貨酒,借米曲於官而毀其歷。獄具,責
 昭化軍節度副使。數月,復其官。終右文殿修撰,年七十三。



 論曰:西漢之末,士大夫阿諛銷懦,遂底於亡。東都諸賢以風節相尚,激成黨禍。宋元祐類東都,崇、宣類西漢末世,蓋忠鯁獲罪,則相習容悅而已。君驕臣諂,此邦之所繇喪也。觀沉銖諸人,徒徇時軒輊,不能為有亡,惡足以言士哉!



 葉祖洽,字敦禮,邵武人。熙寧初,策試進士,祖洽所對,專
 投合用事者,考官宋敏求、蘇軾欲黜之,呂惠卿擢為第一。簽書奉國軍判官、判登聞檢院,由國子丞知湖州,留為校書郎。



 元祐初,歷職方、兵部員外郎,加集賢校理,進禮部郎中。給事中趙君錫論其對策訕及宗朝,祖洽自辨,事下從官定議。蘇軾、劉分文言:「祖洽謂祖宗紀綱法度,因循茍簡,願朝廷與大臣合謀而新之。可以為議論乖謬,若謂之訕則不可。」於是但出提點淮西刑獄。



 紹聖中,入為左司郎中、起居郎、中書舍人、給事中。祖洽性狠
 愎,喜諛附,密言王珪於冊立時有異論。哲宗曰:「宣仁聖烈,婦人之堯、舜也。其於社稷大計,聖意素定,朕已令作告命,明述此旨。」祖洽復言:「若以珪為無跡,則黃履、劉拯相繼論之矣,願稽合群情,決之獨斷。」珪遂追貶。又言:「司馬光、呂公著獲終牖下,恩禮隆縟;蔡確受遺定策,而貶死嶺外,乞恤其孤。」其論率類此。林希薦祖洽,謂其最向正,帝言不可大用,乃已。坐舉王回出知濟州,徙洪州,以牟利黷貨聞。



 祖洽與曾布厚,人目為「小訓狐」。布用事,欲以
 吏部侍郎召,韓忠彥不可,白為寶文閣待制、知青州。未赴,布竟引為吏部。布罷,乃出知定州,且行,大言於上,至云:「當時蔡確稍失事幾,王珪果遂奸謀,則神宗遂失正統,不知今日神器孰歸。臣為朝廷宗社明確之功,正珪之罪,勸沮忠邪於千萬年,以此報神宗足矣。」徽宗怒其躁妄,降集賢殿修撰、提舉沖祐觀,自是不復用。久之,知洪州,改亳州,加徽猷閣直學士。政和末,卒。



 時彥,字邦美,開封人。舉進士第,簽
 書穎昌判官,入為秘書省正字,累至集賢校理。紹聖中,遷右司員外郎。使遼失職,坐廢,旋復校理,提點河東刑獄,蹇序辰使遼還,又坐前受賜增拜,隱不言,復停官。徽宗立,召為吏部員外郎,擢起居舍人,改太常少卿,以直龍圖閣為河東轉運使,加集賢殿修撰、知廣州。未行,拜吏部侍郎,徙戶部,為開封尹。異時都城苦多盜,捕得,則皆亡,卒吏憚於移問,往往略之。彥始請一以公憑為驗,否則拘系之以俟報,坊邑少安,獄屢空。數月,遷工部尚書,進吏部,卒。



 霍端友,字仁仲,常州武進人。徽宗即位,策進士第一,授宣義郎。不閱月,擢秘書省校書郎,遷著作佐郎、起居郎、中書舍人,服金紫。故事唯服黑角帶,帝顧見之,曰:「給事、舍人等爾,而服飾相絕如是。」始命犀帶佩魚。進給事中、大司成、禮部侍郎。端友言:「朝廷尊安,重內輕外。可令內外侍從更出迭入,以奉禁闥,殿大邦,俾天下之勢如持衡,庶無首重尾輕之患。」疏入,即請補郡,乃以顯謨閣待制知平江。改陳州,為政以寬聞,不立聲威。陳地污下,久
 雨則積潦,時疏新河八百里,而去淮尚遠,水不時洩。端友請益開二百里,徹於淮,自是水患遂去。內侍石燾傳詔索瑞香花數十本,端友不可,疏罷之。復以禮部召,轉吏部。官至通議大夫。卒,贈宣奉大夫。



 俞慄,字祗若,江寧人。崇寧四年,以上舍生賜進士第,簽書鎮南軍判官。未赴,為闢雍博士、秘書省正字、吏部員外郎、起居舍人,兼定、嘉二王記室,擢中書舍人。居三月,進給事中、殿中侍御史。毛注建議罷增石炭場,慄駁其
 非。除顯謨閣待制、知蔡州,明日復留。逾年,竟出為襄州。還,拜給事中,上言:「學校,三代之學也。然崇寧四年以前,議者以為是,五年,則非之;大觀三年以前,議者以為是,四年,則非之。豈學校固若是哉?觀望者無定說爾。必使士有成才,人無異論,事之不美者不出於學校,然後為得。」言頗見行。



 蔡京再相,憾向所用士多畔己,葉夢得言慄獨否,遂拜御史中丞。陳士風六弊,又發戶部尚書劉炳為舉子時陰事。京方倚炳為腹心,戾其意,改慄翰林
 學士。遷兵部尚書,以樞密直學士知開德府。石公弼在襄州,以論衙前事謫言者,謂慄實倡之,罷,提舉崇禧觀。竟以毀紹聖法度,貶常州團練副使,安置太平州。行未至,復述古殿直學士、知江寧府,卒。



 蔡薿,字文饒,開封人。崇寧五年,以諸生試策,揣蔡京且復用,即對曰:「熙、豐之德業,足以配天,不幸繼之以元祐;紹聖之纘述,足以永賴,不幸繼之以靖國。陛下兩下求言之詔,冀以聞至言、收實用也。而見於元符之末者,方
 且幸時變而肆奸言,乘間隙而投異意,詆誣先烈不以為疑,動搖國是不以為憚。願逆處其未至而絕其原。」於是擢為第一,以所對頒天下,甫解褐,即除秘書省正字,遷起居舍人。未幾,為中書舍人。自布衣至侍從,才九月,前所未有也。



 旋進給事中。一意附蔡京,敘族屬,尊為叔父。京命攸、修等出見,薿亟云:「向者大誤,公乃叔祖,此諸父行也。」遽列拜之。八寶赦恩,詔兩省差擇元祐黨人,情輕者出籍。薿不肯書,言者論其不能推廣上恩,使
 歲久獲罪之人得以洗濯。出知和州。明年,加顯謨閣待制、知杭州。



 始,薿未第時,以書謁陳瓘,稱其諫疏似陸贄,剛方似狄仁傑,明道似韓愈。及對策,所持論頓異,遂欲害瓘以絕口。因其子正匯告蔡京不軌,執送京師。薿復入為給事中,又與宰相何執中謀,使石悈治瓘,幾不免,事具《瓘傳》。御史毛注言:「陛下修善政以應天,斥大奸以定國,而薿巧言惑眾,造為釁端。」疏入不報。



 範柔中者,頃以上書入邪等,至是進階。薿言:「柔中嘗毀神考,哲宗有弗
 共戴天之讎。自今春黨人復官,士論駭愕,有致疑於紹述者。乞削其敘遷,昭示好惡。」從之。張商英作相,常安民與之書,激使為善。薿弟萊剽其稿示薿,即論之以搖商英。薿遷翰林學士,坐妄議政事罷,提舉洞霄宮。起知建寧府。



 方建神霄宮,薿先一路奏辦,下詔褒獎,召為學士承旨、禮部尚書。嘗陰附權幸,事覺,徽宗令入對,將面詰之。逾月不奉詔,帝怒,命黜之。御史言:「薿游太學,則挾詭計以鉗諸生;居侍從,則抉私事以脅宰輔;處門下,則借國
 法以快私忿;為郡守,則妄尊大而蔑監司。召自金陵,偃然以丞轄自處,既升宗伯,乃懷不滿之心。宜重置諸罰。」遂貶單州團練副使,房州安置。



 宣和中,復龍圖閣直學士,再知杭州。為政喜怒徇情,任刑大慘。方臘亂後,西北戍卒代歸,人得犒絹,薿禁民與為市,乃下其直,強取之。卒怒,乘薿夜飲客,縱火焚州治,須其出救,殺之。薿知事勢洶洶,逾垣走,僅得免。詔奪職罷歸。明年,以徽猷閣待制卒。



 論曰:自太宗歲設大科,致多士,居首選者躐取華要,有不十年至宰相,亦多忠亮雅厚,為時名臣。治平更三歲之制,繼以王安石改新法,士習始變。哲、徽紹述,尚王氏學,非是無以得高第。葉祖洽首迎合時相意,擢第一,自是靡然,士風大壞,得人亦衰,而上之恩秩亦薄矣。熙寧而後,訖於宣和,首選十八人,唯何慄、馬涓與此五人有傳,然時彥、端友齪齪,祖洽、俞慄、蔡薿憸邪小人。繇王氏之學不正,害人心術,橫潰爛漫,並邦家而覆之;
 如是其慘焉,此孟子所以必辯邪說、正人心也。



\end{pinyinscope}