\article{列傳第二十六}

\begin{pinyinscope}

 張
 宏趙昌言陳恕魏羽劉式附劉昌言張洎李惟清



 張宏,字臣卿,青州益都人。高祖茂昭,唐易、定節度使。曾祖元,易州刺史。祖持,蒲城令。父峭,業《春秋》,一舉不第,退
 居丘園,後唐天成中以賢帥後補協律郎,至平利令。



 宏,太平興國二年,舉進士,為將作監丞,通判宣州。改太子中允、直史館,遷著作郎,賜緋魚,預修《太平御覽》,歷左拾遺。六年,出為峽路轉運副使,就加左補闕、會省副使,知遂州,以勤幹聞,入為度支員外郎。



 雍熙中,呂蒙正、李至、張齊賢、王沔薦其文行,改主客郎中、史館修撰。數日,以本官充樞密直學士,賜金紫。太宗召對便殿,謂曰:「成都重地,卿為朕鎮之。」因厚賜以遣。至鄭州,促召歸闕,拜右
 諫議大夫、樞密副使。會太宗親試禮部不合格貢士,令樞密院給牒,因謂宏曰:「朕自御極以來,親擇群材,大者為棟梁,小者為榱桷,卿與呂蒙正皆中朕選,大臣頗有沮議。非朕獨斷,豈能及此乎?」宏頓首謝。



 時河朔用兵,宏居位無所建白,御史中丞趙昌言多言邊事,乃以昌言副樞密,宏為中丞,兩更其任。端拱初,改工部侍郎,再為樞密副使。淳化二年,以吏部侍郎罷,俄判吏部銓,權知開封府。太宗御便殿慮囚,以府獄多壅,詔劾其官屬,宏
 等頓首請罪,乃釋之。真宗尹京,宏罷奉朝請。至道初,出知潞州。二年,就轉右丞。真宗即位,加工部尚書。咸平初,還朝,知審官院、通進銀臺封駁司。二年,真宗以上封者眾,慮其稽留,命宏與王旦知登聞鼓院,再掌吏部選。四年,卒,年六十三。廢朝,贈右僕射,命中使蒞葬事。錄其子可久大理評事,可道太祝,可度奉禮郎。



 宏循謹守位,不求赫赫之譽,歷踐通顯,未嘗敗事。可久至虞部員外郎,可道國子博士,可度太子中舍。



 趙昌言,字仲謨,汾州孝義人。父叡,從事使府,太宗尹開封,選為雍丘、太康二縣令,後終安、申觀察判官。



 昌言少有大志,趙逢、高錫、寇準皆稱許之。太平興國三年,舉進士,文思甚敏,有聲於場屋,為貢部首薦。廷試日,太宗見其辭氣俊辯,又睹其父名,謂左右曰:「是嘗為東畿宰,朕之生辰,必獻詩百韻為壽,善訓其子,亦為可嘉也。」擢置甲科,為將作監丞,通判鄂州。拜右拾遺、直史館,賜緋魚。選為荊湖轉運副使,遷右補闕,會省副職,改知青州。入
 拜職方員外郎,知制誥,預修《文苑英華》。雍熙初,加屯田郎中。明年,同知貢舉,俄出知天雄軍。



 時曹彬、崔彥進、米信失律於歧溝,昌言遣觀察支使鄭蒙上疏,請誅彬等。優詔褒答,召拜御史中丞。太宗宴金明池,特召預焉。憲官從宴,自昌言始也。



 河東用兵,樞密副使張宏循默守位,昌言多條上邊事,太宗即以昌言為左諫議大夫,代宏為樞密副使,遷工部侍郎。時鹽鐵副使陳象輿與昌言善,知制誥胡旦、度支副使董儼皆昌言同年,右正言
 梁顥嘗在大名幕下。四人者,日夕會昌言之第。京師為之語曰:「陳三更,董半夜。」有傭書翟穎,性險誕,與旦狎,旦為作大言之辭,使穎上之,為穎改姓名馬周,以為唐馬周復出也。其言多毀時政,自薦為大臣,及歷舉數十人皆公輔器,期昌言為內應。陳王尹開封,廉知以聞,詔捕穎系獄,鞠之,盡得其狀。昌言坐貶崇信軍節度行軍司馬,穎仗脊黥面,流海島,禁錮終身。



 初,太宗厚遇昌言,垂欲相之。趙普以勛舊復入,晉昌言剛戾,乃相呂蒙正。裁
 數月,會有穎獄,普以昌言樹黨,再勸太宗誅之,太宗特寬焉。淳化二年,起昌言知蔡州,逾年,召拜右諫議大夫。或議馳茶鹽禁,以省轉漕。命昌言為江淮、兩浙制置茶鹽使,昌言極言非便,太宗不納,趣昌言往。昌言固執如初。即以戶部副使雷有終代之,卒以無利而罷。



 昌言復知天雄軍,賜錢二百萬。大河貫府境,豪民峙芻茭圖利,誘奸人潛穴堤防,歲仍決溢。昌言知之。一日,堤吏告急,命徑取豪家廥積以給用,自是無敢為奸利者。屬澶州
 河決,流入御河,漲溢浸府城,昌言籍府兵負土增堤,數不及千,乃索禁卒佐役,皆偃蹇不進。昌言怒曰:「府城將墊,人民且溺,汝輩食厚祿,欲坐觀耶?敢不從命者斬。」眾股心慄赴役,不浹旬城完。太宗手詔褒諭之,召拜給事中、參知政事,俾乘疾置以入,即赴中書。



 時京城連雨,昌言請出廄馬分牧外郡。或以盛秋備敵,馬不可闕。昌言曰:「塞下積水,敵必不至。」太宗從之。未幾,王小波、李順構亂於蜀,議遣大臣撫慰。昌言獨請發兵,無使滋蔓,廷論未決。
 會嘉、眉連陷,始命王繼恩等分路進討。昌言攝祭太廟,宿齋中,因召對滋福殿,復贊兵計,遂遣使督繼恩戰。繼恩御眾寡術,餘寇未殄,握兵留成都,士無鬥志,郡縣復有陷者。太宗意頗厭兵,召昌言謂曰:「西川本自一國,太祖平之,訖今三十年矣。」昌言知意,即前指畫攻取之策。太宗喜,命昌言為川峽五十二州招安行營馬步軍都部署。昌言懇辭,敦諭不許,賜精鎧、良馬、白金五千兩,別賜手札數幅,皆討賊方略。自繼恩以下,並受節度。既行,
 有奏昌言無嗣,鼻折山根,頗有反相,不宜遣握兵入蜀。從旬日,召宰相於北苑門曰:「昨令昌言入蜀,朕思之有所未便。且蜀賊小醜,昌言大臣,未易前進。且令駐鳳翔,止遣內侍衛紹欽繼手書指揮軍事,亦可濟也。」詔書追及,昌言已至鳳州,留候館百餘日。賊平,改戶部侍郎,罷政事,知鳳翔府。徙澶、涇、延三州。



 真宗即位,遷兵部侍郎、知陜州,表求還京,不許。未幾,移知永興軍。咸平三年,與呂蒙正、寇準同召,以本官兼御史中丞、知審官院。有言
 門資官不宜任親民,昌言手疏,以才不才在人,豈以寒雋世家為限,遂罷其議。加工部尚書,仍兼中丞。



 先時,多遣臺吏巡察群臣逾越法式者,昌言建議請準故事,令左右巡使分領之。會知審刑院趙安仁、判大理寺韓國華斷獄失中解職,昌言因上言:「詳斷官宜加慎擇,自今有議刑不當,嚴示懲罰,授以遠官,若有罪被問不即引伏者,許令追攝。又天下大闢斷訖,皆錄款聞奏,付刑部詳覆,用刑乖理者皆行按劾。惟開封府未嘗奏案,或斷
 獄有失,止罪元勘官吏,知府、判官、推官、檢法官皆不及責,則何以辨明枉濫,表則方夏?望自今如外州例施行。」從之。會孟州民常德方訟臨津尉任懿以賄登第,事下御史,乃知舉王欽若受之,昌言以聞。欽若自訴,詔刑昺覆按,坐昌言故入,奪官,貶安遠軍行軍司馬,移武勝軍。



 景德初,拜刑部侍郎。求兼三館職,命判尚書都省。真宗幸澶淵,以盟津居要,增屯兵,命知河陽。歷知天雄軍府。境內有小盜,昌言榜諭:「能告執者給賞,牙吏即遷職。」樞
 密使王繼英以為小盜不當擅為賞格,乃詔昌言易其榜,有勞者俟朝旨。未幾,徙知鎮州,遷戶部侍郎。大中祥符二年,卒,年六十五。贈吏部尚書,謚曰景肅。錄其子慶嗣為國子監丞,賦祿終喪。侄孫允明同學究出身。



 昌言喜推獎後進,掌漕湖外時,李沆通判潭州,昌言謂有臺輔之量,表聞於朝。王旦宰岳州平江,昌言一見,識其遠大,以女妻之,後皆為賢相。王禹偁自卑秩擢詞職,亦昌言所薦也。



 昌言強力尚氣概,當官無所顧避,所至以威
 斷立名,雖屢經擯斥,未嘗少自抑損。然剛愎縱率,對僚吏倨慢,時論以此少之。慶嗣至太子洗馬。



 陳恕,字仲言,洪州南昌人。少為縣吏,折節讀書。江南平,禮部侍郎王明知洪州,恕以儒服見,明與語,大奇之,因資送令預計偕。太平興國二年進士,解褐大理評事、通判洪州,恕以鄉里辭。改澧州。澧自唐季為節鎮兼領,吏多緣簿書乾沒為奸。恕盡擿發其弊,郡中稱為強明,以吏幹聞。



 召入,為右贊善大夫,同判三司勾院,遷左拾遺,
 充度支判官。與判使王仁贍廷爭本司事,仁贍屈伏,坐貶秩;擢恕為度支員外郎,仍舊職。



 再遷工部郎中、知大名府。時契丹內寇,受詔增浚城隍,其器用取於民者不時集,恕立擒府中大豪一人,會將吏將斬之。宗族號訴,賓佐競前請救,大豪叩頭流血,請翌日集事,違期甘死。恕令械之以徇,民皆恐心慄,無敢後期者,數日功就。



 會契丹引去,召入為戶部郎中、戶部副使,遷右諫議大夫、知澶州。驛召為河北東路營田制置使。太宗諭以農戰之
 旨,恕對曰:「古者兵出於民,無寇則耕,寇至則戰。今之戎士皆以募致,衣食仰給縣官,若使之冬持兵禦寇,春執耒服田,萬一生變,悔無及矣。」太宗曰:「卿第行,朕思之。」恕行數日,果有詔,止令修完城堡、通導溝瀆而已,營田之議遂寢。俄知代州,入判吏部選事,拜鹽鐵使。恕有心計,厘去宿弊,太宗深器之,親題殿柱曰:「真鹽鐵陳恕」。



 遷給事中、參知政事。數月,太宗言及戶部使樊知古所部不治。恕與知古聯事,情好款洽,密以語之,欲知古修舉其
 職。知古訴於太宗,太宗怒恕洩禁中語,罷守本官。旋出知江陵府,大發群吏奸臟,坐徒、流、停、廢者甚眾,郡內惕息。



 淳化四年,太宗從魏羽、段惟一之請,分三司為十道,置左右計使,以魏羽、董儼分主之;召恕為工部侍郎,充總計使,判左右計事。左右計使分判十道事,凡議論、計度並令恕等參預。恕以官司分隸,政令互出,難以經久,極言其非使。歲餘,果罷,復以恕為鹽鐵使。



 時太宗留意金谷,召三司吏李溥等二十七人對於崇政殿,詢以計
 司利害。溥等言條目煩多,不可以口占,願給筆札以對。太宗遣中黃門送詣相府,限五日悉條上之。溥等共上七十一事,詔以四十四事付有司行之,其十九事下恕等議可否。遣知雜御史張秉、中使張崇貴監議,令中書籍其事,專檢舉之,無致廢格。賜溥等白金緡錢,悉補侍禁、殿直,領其職。太宗謂宰相曰:「溥等條奏事頗有所長。朕嘗語恕等,若文章稽古,此輩固不可望;若錢谷利病,頗自幼至長寢處其中,必周知根本。卿等但假以顏色,
 引令剖陳,必有所益。恕等剛強,終不肯降意詢問。」呂端對曰:「耕當問奴,織當問婢。」寇準曰:「夫子入太廟,每事問,乃以貴下賤,先有司之義。」



 後數日,太宗又曰:「國家歲入財數倍於唐。唐中葉以降,藩鎮擅命,征賦多不入公家,下陵上替,經制隳壞。若前代為得,即已致太平,豈復煩朕心慮也。」因召恕等責以職事曠廢。恕等對曰:「今土宇至廣,庶務至繁,國用軍須,所費浩瀚,又遇諸州凡有災沴,必盡蠲其租。臣等每舉榷利,朝廷必以侵民為慮,皆
 尼而不行。縱使耿受昌、桑弘羊復生,亦所不逮。臣等駑力,惟盡心簿領,終不足上裨聖治。」太宗曰:「卿等清而不通,專守繩墨,終不能為國家度長絜大,剖煩析滯。只如京城倉庫,主吏當改職者,簿領中壹處節目未備,即至十年五年不決,以致貧無資給,轉徙溝壑。此卿等之過,豈不傷和氣哉?」恕等頓首謝。五年,賜三司錢百萬,募吏有能言本司不便者,令恕等量事大小,以錢賞之,錢盡更給。



 至道二年,欲並三司,命官總判。其勾院、磨勘、理欠、
 憑由、支收、行帳、提點等司,令恕條列其事以聞。恕奏曰:「伏以封域浸廣,財穀繁多,三司之中,簿牒填委,朝廷設法,督責尤嚴,官吏救過不暇。若為三部各設主司,擇才非難,辦事亦易。事辦過鮮,不撓上心,此亦一時之良策也。其勾院、磨勘兩司,出於舊制,關防之要,莫加於此。理欠、憑由二司,雖非舊設,自理欠失序,憑由散落,故設二司專令典掌。綱目咸具,制置有倫,逋欠無失理之名,憑由鮮流散之弊,實亦要切,不可廢除。若兩司並委一官,
 方及判官一員之事。其主轄支收司,先因從京支度財貨,轉輸外地,此除彼附,照驗稽滯,若京城得賢主史,使居此司,專行檢轄,凡支撥官物,便給除破文憑,卻於所司置簿記錄,催到收附文記,即乃勾銷簿書取捷之門,亦為允當。其行帳司近日權置,了絕舊帳,帳目告盡,司額自除。提點司是中旨特置,提振三司廢怠之事,固非有司敢得擬議也。」詔三司都憑由、理欠司宜令為一處,命官兼判。應諸道逋負官物,令三司逐部理約,理欠司
 但總其所逋之數糾督之。餘悉從恕奏。



 恕將立茶法,召茶商數十人,俾各條利害,恕閱之第為三等,語副使宋大初曰:「吾觀下等固滅裂無取。上等取利太深,此可行於商賈,不可行於朝廷。惟中等公私皆濟,吾裁損之,可以經久。」於是始為三法行之,貨財流通。



 峽路諸州,承孟氏舊政,賦稅輕重不均,閬州稅錢千八百為一絹,果州六百為一絹。民前後擊登聞鼓陳訴,歷二十年,詔下本道官吏,因循不理。轉運副使張曄年少氣銳,會受詔按
 覆,即便宜行之。恕奏曄擅改法,計果州一歲虧上供絹萬餘,曄坐削一任免。



 恕每便殿奏事,太宗或未深察,必形誚讓。恕斂板踧縮,退至殿壁負立,若無所容。俟意稍解復進,愨執前奏,終不改易,如是或至三四。太宗以其忠,多從之。遷禮部侍郎。真宗即位,加戶部,命條具中外錢穀以聞。恕久不進,屢趣之,恕曰:「陛下富於春秋,若知府庫充實,恐生侈心,臣是以不敢進。」真宗嘉之。



 咸平二年,帝北巡,充行在轉運使。俄以母老求解,拜吏部侍郎,
 知通進銀臺封駁司、審官院。上言:「封駁之任,實給事中之職,隸於左曹。雖別建官局,不可失其故號。請以門下封駁事隸銀臺司。」從之。五年,知貢舉。恕自以洪人避嫌,凡江南貢士悉被黜退。又援貢舉非其人之條,故所取甚少,而所取以王曾為首,及廷試糊名考校,曾復得甲科,時議稱之。恕每自嘆曰:「吾得曾,名世才也,不愧於知人矣。」



 恕事母孝,母亡,哀慕過甚,不食葷茹,遂至羸瘠。起復視事,遷尚書左丞、權知開封府。恕已病,猶勉強親職,
 數月增劇,表求館殿之職,獲奉以濟其貧。真宗曰:「卿求一人可代者,聽卿去。」是時寇準罷樞密使,恕即薦以自代,遂以準為三司使,恕為集賢學士、判院事。準即檢尋恕前後改革興立之事,類以為冊,及以所出榜,別用新板,躬至恕第請判押。恕亦不讓,一一押之,自是計使無不循其舊貴。至李諮為三司使,始改茶法,恕之規模漸革矣。



 帝重恕,詔太醫診療。百日,有司請停奉,不許,未幾,卒,年五十九。恕將卒,口占遺奏及約束後事,送終之具,
 無不周悉。真宗悼惜,廢朝,贈吏部尚書。錄其子執中為太常寺太祝,執古為奉禮郎。



 恕頗涉史傳,多識典故,精於吏理,深刻少恩,人不敢干以私。前後掌利柄十餘年,強力干事,胥吏畏服,有稱職之譽。善談論,聽者忘倦。素不喜釋氏,嘗請廢譯經院,辭甚激切。真宗曰:「三教之興,其來已久,前代毀之者多矣,但存而不論可也。」



 恕性吝,怒子淳私用錢。及寢疾,上言淳不率教導,多與非類游,常習武藝,願出為外州軍校。真宗曰:「戎校管鎮兵,非丞
 郎家子弟所蒞也。」以為滁州司馬。恕卒,召復舊官,後竟以賄敗。執中至同中書門下平章事,別有傳;執古至虞部員外郎;執方、執禮,並太子中舍。



 魏羽者,字垂天,歙州婺源人。少能屬文,上書李煜,署弘文館校書郎。時建當塗縣為雄遠軍,以羽為判官。宋師渡江出其境,羽以城降,太祖擢為太子中舍,仍舊職。金陵平,入朝,出知興州。



 太平興國初,知棣州,改京兆府。六年,受詔詣瀛州覆軍市租,得隱漏數萬計。因上言:「本州
 錄事參軍郭震十年未代;河間令崔能前任即墨,未滿歲遷秩。有司調選失平,疏遠何由聞達,請罪典司,以肅欺弊。」上賜詔褒諭。復命,遷太常博士、知宋州,又徙閬州,就改膳部員外郎。丁外艱,起復蒞事,入判大理寺。歷度支、戶部二判官,召拜本曹郎中。因上疏言三司職官頗眾,願省其半,可以責成,仍條列利病凡二十事。詔下有司詳議,皆以為便。改鹽鐵判官。時北邊多警,朝議耕戰之術,以羽為河北東路營田副使,改兩浙轉運使,遷兵
 部郎中。



 淳化初,選為秘書少監,逾月,遷左諫議大夫,俄拜度支使,改鹽鐵使。四年,並三部為一司,以羽判三司。先是,三司簿領堆積,吏緣為奸,雖嘗更立新制,未為適中。是冬,羽上言:「依唐制天下郡縣為十道,兩京為左右計,各署判官領之。」制三司使二員,以羽為左計使,董儼為右計使,中分諸道以隸焉。未久,以非便罷,守本官,出知滑州。丁內艱,起復,加給事中,徙潭州,遣使諭旨。真宗即位,遷工部侍郎,連徙杭、揚二州,召權知開封府。車駕
 北巡,判留司三司,再為戶部度支使。



 咸平四年,以疾解職,拜禮部侍郎。謝日,召升便殿,從容問諭,勉以醫藥。月餘卒,年五十八。



 羽涉獵史傳,好言事。淳化中,許王暴薨,或有以宮府舊事上聞者。太宗怒,追捕僚吏,將窮究之。羽乘間上言曰:「漢戾太子竊弄父兵,當時言者以其罪當笞耳。今許王之過,未甚於是。」太宗嘉納之,繇是被劾者皆獲輕典。嘗建議有唐以來,凡制詔皆經門下省審,有非便者許其封駁,請遵故事,擇名臣專領其職,迄今
 不廢。



 羽強力有吏乾,尤小心謹事。太宗嘗謂左右曰:「羽有心計,亦明吏道,但無執守,與物推移耳。歷劇職十年,始逾四十,須鬢盡白,亦可憐也。」羽出入計司凡十八年,習知金穀之事,然頗傷煩急,不達大體。



 景德二年,長子玠卒,其妻自陳家貧無祿,上憫之。次子校書郎瓘為奉禮郎,後為殿中丞;琰為太子中舍。孫平仲,天禧三年同進士出身。



 羽同時有劉式者,亦久居計司,創端拱中三年磨勘之法,首以式主之。



 式字叔度,袁州人也。李煜時,舉《三傳》中第。歸宋,歷遷大理寺丞、贊善大夫、監通州豐利監及主三司都磨勘司,仍賜緋。式又建議置主轄支收司,以謹財賦出納,時以為當。遷秘書丞,與陳靖使高麗。至道中,並三勾院為一,命式領之。再轉工部員外郎,賜金紫。遷刑部。式深究簿領之弊,江、淮間舊有橫賦,逋積至多,式奏免之,人以為便。然多所條奏,檢校過峻,為下吏所訟,免官,卒。



 真宗追錄前效,賜其子立本學究出身。次子立之,後為國子博
 士。立德、立禮,並進士及第,立禮為殿中丞。



 劉昌言,字禹謨,泉州南安人。少篤學,文詞靡麗。本道節度陳洪進闢功曹參軍,掌箋奏。洪進遣子文顯入貢,令昌言偕行,太祖親勞之。



 太平興國二年,洪進歸朝,改鎮徐州,又闢推官。五年,舉進士入格,太宗初惜科第,止授歸德軍掌書記。八年,復舉得第,遷保信、武信二鎮判官。宰相趙普鎮南陽,重昌言有吏乾。錢俶帥鄧,表薦之。移泰寧軍節度判官。入為左司諫、廣南安撫使。淳化初,趙
 普留守西京,表為通判,委以府政。普疾,屬昌言後事。普卒,昌言感普知己,經理其家事。太宗以為忠於所舉,拜起居郎,賜金紫、錢五十萬。連對三日,皆至日旰。昌言捷給詼詭,能揣人主意,無不稱旨。太宗謂宰相曰:「昌言質狀非偉,若以貌取,失之子羽矣。」遷工部郎中,逾月,守本官,充樞密直學士,與錢若水同知審官院。二十八日,遷右諫議大夫、同知樞密院事。



 昌言驟用,不為時望所伏,或短其閩語難曉,太宗曰:「惟朕能曉之。」又短其委母妻
 鄉里,十餘年不迎侍,別娶旁妻。太宗既寵之,詔令迎歸京師,本州給錢辦裝,縣次續食。時又有光祿丞何亮家果州,秘書丞陳靖家泉州,不迎其親。下詔戒諭文武官,父母在劍南、峽路、漳泉、福建、嶺南,皆令迎侍,敢有違者,御史臺糾舉以聞。



 昌言自以登擢非次,懼人傾奪。會誅兇人趙贊,昌言與贊素善,前在河南嘗保任之,心不自安。因太宗言及近侍有與贊交者,昌言蹶然出位,頓首稱死罪。太宗慰勉之,然自此惡其為人。以給事中罷,出
 知襄州。上言:「水旱民輸稅愆期。舊制六月開倉,臣令先一月許所在縣驛輸納以便民。獲盜當部送闕下,臣恐吏柔懦不能制,再亡命,配隸軍籍。此二事,臣從便宜,不如詔書,慮讒慝因而浸潤,願陛下察之。」太宗下詔責其不循舊章,斂怨於民,自今敢背棄詔條,譴責不復恕。



 至道二年,徙知荊南府。真宗即位,就拜工部侍郎。咸平二年,卒,年五十八,贈工部尚書。子有方,比部員外郎;有政,虞部員外郎。



 張洎,滁州全椒人。曾祖旼,澄城尉。祖蘊,泗上轉運巡官。父煦,滁州司法掾。洎少有俊才,博通墳典。江南舉進士,解褐上元尉。李景長子弘冀卒,有司謚武宣。洎議以為世子之禮,但當問安視膳,不宜以「武」為稱。旋命改謚,擢監察御史。洎自以論事稱旨,遂肆彈擊無所忌,大臣游簡言等嫉之。會景遷國豫章,留煜居守,即薦洎為煜記室,不得從。未幾,景卒,煜嗣。擢工部員外郎、試知制誥;滿歲,為禮部員外郎、知制誥。遷中書舍人、清輝殿學士,參
 預機密,恩寵第一。



 洎舊字師黯,改字偕仁。清輝殿在後苑中,煜寵洎,不欲離左右,授職內殿,中外之務一以諮之。每兄弟宴飲,作妓樂,洎獨得預。為建大第宮城東北隅,及賜書萬餘卷。煜嘗至其第,召見妻子,賜予甚厚。



 洎尤好建議,每上言,未即行,必稱疾,煜手札慰諭之,始復視事。及王師圍城,逾年,城危甚,洎勸煜勿降,每引符命云:「玄象無變,金湯之固,未易取也。北軍旦夕當自引退。茍一旦不虞,即臣當先死。」既而城陷,洎攜妻子及橐裝,
 自便門入止宮中,紿光政使陳喬同升閣,欲與俱死。喬自經氣絕,洎反下見煜曰:「臣與喬同掌樞務,國亡當俱死。又念主在,誰能為主白其事,不死,將有以報也。」



 歸朝,太祖召責之曰:「汝教煜不降,使至今日。」因出帛書示之,乃圍城日洎所草詔,召上江救兵蠟丸書也。洎頓首請罪曰:「實臣所為也。犬吠非其主,此其一爾,他尚多有。今得死,臣之分也。」辭色不變。上奇之,貸其死,謂曰:「卿大有膽,不加卿罪。今之事我,無替昔日之忠也。」拜太子中允,
 歲餘,判刑部。太宗即位,以其文雅,選直舍人院,考試諸州進士。未幾,使高麗,復命,改戶部員外郎。太平興國四年,出知相州。明年夏,徙貝州。是冬,又知相州。部內不治,轉運使田錫言其狀,代還。洎求見廷辯,上以其儒生,不責以吏事,詔不問。令以本官知譯經院,遷兵部員外郎、禮、戶二部郎中。雍熙二年,同知貢舉。



 端拱初,契丹寇邊,詔群臣言事。洎上奏,以練兵聚穀,分屯塞下,來則備御,去則勿追為要略。會錢俶薨,太常定謚忠懿。洎時判考
 功,為覆狀,經尚書省集議。虞部郎中張佖奏駁曰:「按考功覆狀一句云『亢龍無悔』,實非臣子宜言者。況錢俶生長島夷,夙為荒服,未嘗略居尊位,終是藩臣,故名不可稱龍,位不可為亢。其『亢龍無悔』四字,請改正。」事下中書,以詰洎。對狀曰:「竊以故秦國王明德茂勛,格於天壤,處崇高之富貴,絕纖介之譏嫌。太常禮院稽其功行,定茲嘉謚,考功詳覆之際,率遵至公,故其議狀云:『茲所謂受寵若驚,居亢無悔者也』。謹按《易·乾》之九三云:『君子幹幹,
 夕惕若厲,無咎。』王弼注云:『處下體之極,居上體之下,履重剛之險,因時而惕,不失其幾,可以無咎。處下卦之極,愈於上九之亢。』《易例》云:『初九為元士,九二為大夫,九三為諸侯。』《正義》云:『《易》之本理,以體為君臣。九三居下體之極,是人臣之體也。其免亢龍之咎者,是人臣之極,可以慎守免禍。故云免亢極之禍也。』《漢書·梁商傳贊》云:『地居亢滿,而能以謹厚自終。』楊植《許由碑》云:『錙銖九有,亢極一夫。』杜鴻漸《讓元帥表》云:『祿位亢極,過逾涯量。』盧
 杞《郭子儀碑》云:『居亢無悔,其心益降。』李翰《書霍光傳》云:『有伊、周負荷之明,無九三亢極之悔。』張說《祁國公碑》云:『一無目牛之全,一無亢龍之悔也。』況考功狀內止稱云:『受寵若驚,居亢無悔。』即本無『亢龍無悔』之語。斯蓋張佖擅改公奏,罔冒天聰。請以元狀看詳,反坐其人,以懲奸妄。」俄下詔曰:「張洎援引故實,皆有依據。張佖學識甚淺,敷陳失實,尚示矜容,免其黜降,可罰一月俸。」



 洎未幾選為太僕少卿、同知京朝官考課,拜右諫議大夫、判大理寺。
 又充史館修撰、判集賢院事。淳化中,上令史館修撰楊徽之等四人修正入閣舊圖,洎同奏詔,因討論故事,獨草奏以聞。洎又言:



 按舊史,中書、門下、御史臺為三署,謂侍從供奉之官。今起居日侍從官先入殿庭,東西立定,俟正班入,一時起居。其侍從官東西列拜,甚失北面朝謁之儀。請準舊儀,侍從官先入起居,行畢,分侍立於丹墀之下,謂之「蛾眉班」。然後宰相率正班入起居,雅合於禮。



 臣又聞古之王者,躬勤庶務,其臨朝之疏數,視政事
 之繁簡。唐初五日一朝,景雲初,始修貞觀故事。自天寶兵興之後,四方多故,肅宗而下,咸只日臨朝,雙日不坐。其只日或遇陰霪、盛暑、大寒、泥濘,亦放百官起居。雙日宰相當奏事,即特開延英召對。或夷蠻入貢,勛臣歸朝,亦特開紫宸殿引見。陛下自臨大寶,十有五年,未嘗一日不雞鳴而起,聽天下之政,雖剛健不息,固天德之常然,而游焉息焉,亦聖人之謨訓。儻君父焦勞於上,臣子緘默於下,不能引大體以爭,則忠良之心,有所不至矣。



 臣欲望陛下依前代舊規,只日視朝,雙日不坐。其只日遇大寒、盛暑、陰霪、泥濘,亦放百官起居,其雙日於崇德、崇政兩殿召對宰臣。常參官以下及非時蠻夷入貢、勛臣歸朝,亦特開上閣引見,並請準前代故事處分。



 奏入不報。



 時,上令以《儒行篇》刻於版,印賜近臣及新第舉人。洎得之,上表稱謝,上覽而嘉之。翌日,謂宰相曰:「群臣上章獻文,朕無不再三省覽。如張洎一表,援引古今,甚不可得。可召至中書,宣諭朕意。」數月,擢拜中書舍人,充翰
 林學士。上顧謂近臣曰:「學士之職,清要貴重,非他官可比,朕常恨不得為之。」故事,赴上日設燕,教坊以雜戲進,久罷其事。至是,令盡設之,仍詔樞密直學士呂端、劉昌言及知制誥柴成務等預會,時以為榮。



 俄判吏部銓。嘗引對選人,上顧之謂近臣曰:「張洎富有文藝,至今尚苦學,江東士人之冠也。」洎與錢若水同在禁林,甚被寵顧。時劉昌言驟擢樞要,人望甚輕,董儼方掌財賦,欲以計傾之。會楊徽之、錢熙嘗言洎及若水旦夕當大用。熙以
 語昌言,昌言曰:「洎必參政柄。若水後進年少,豈遽及此。」時翰林小吏諮事在側,昌言慮洎聞之,即對小吏盡述熙言,令告洎。洎方修飭邊幅以固恩寵,疑徽之遣熙以構飛語中己,遂白於上。上怒,召昌言質之,以徽之為鎮安軍行軍司馬,熙罷職,通判朗州。



 會皇子益王元傑改封吳王,行揚州、潤州大都督府長史,領淮南、鎮江兩軍節制。洎當草制,因上疏議曰:「謹按前史,皇子封王,以郡為國,置傅相及內史、中尉等,佐王為治。自漢、魏以降,所
 封之王始不之國,朝廷命卿大夫臨郡,即稱內史行郡事。東晉永和、泰元之際,有瑯邪王、會稽王、臨川王,故謝靈運、王羲之等為會稽、臨川內史,即其事也。唐有天下,以揚、益、潞、幽、荊五郡為大都督,署長史、司馬為上佐,即前代內史之類也。其大都督之號,非親王不授;其揚、益等郡,或有親王遙領,朝廷命大臣臨郡者,即皆長史、副大使知節度事也。臣請質之前代,段文昌出鎮揚州,云『淮南節度副大使知節度事、兼揚州大都督府長史』。
 李載義鎮幽州,云『盧龍軍節度副大使知節度事、兼幽州大都督府長史』,即其例也。今益王以揚、潤二郡建社為吳國王,居大都督之任,又己正領節度事,豈宜卻加長史之號,乃是國王自為上佐矣。若或朝廷且以長史拜受,其加銜內又無副大使、知節度使之目,倘或他日別命守將,俾臨本郡,即不知以何名目而授除也。臣草制之夕,便欲上陳,慮奏報往反,有妨明日宣降。茲事有關國體,況吳王未領恩命,尚可改正,乞付中書門下,商議
 施行。」宰相以制命已行,難於追改。洎又上表論列,呂蒙正言:「越王領福州長史,今吳王獨為大都督,居越王之上,非便。」上令俟異日除授,並改正之。至明年,上郊祀覃慶,遂改焉。



 俄奉詔與李至、範杲、張佖同修國史,又判史館。洎博涉經史,多知典故。每上有著述,或賜近臣詩什,洎必上表,援引經傳,以將順其意。上因賜詩褒美,有「翰長老儒臣」之句。與蘇易簡同在翰林,尤不協,及易簡參知政事,洎多攻其失。既而易簡罷,即以洎為給事中、參
 知政事,與寇準同列。



 先是,準知吏部選事,洎掌考功,為吏部官屬。準年少,新進氣銳,思欲老儒附己以自大。洎夙夜坐曹視事,每冠帶候準出入於省門,揖而退,不交一談。準益重焉,因延與語。洎捷給善持論,多為準規畫,準心伏,乃兄事之,極口談洎於上。上欲進用,又知其在江左日多讒毀良善,李煜殺潘祐,洎嘗預謀,心疑之。翰林待詔尹熙古、吳郢皆江東人,洎嘗善待之。上一夕召熙古輩侍書禁中,因問以祐得罪故。熙古言煜忿祐諫
 說太直耳,非洎謀也。自是洗然,遂加擢用,蓋準推挽之也。既同秉政,奉準愈謹,政事一決於準,無所參預。專修時政記,甘言善柔而已。後因奏事異同,準復忌之。



 至道二年五月,四方館使曹璨自河西馳騎入奏邊事,言繼遷率萬餘眾寇靈州。上詔宰相呂端、知樞密院事趙鎔等各以所見畫策,即日具奏來上。呂端相率詣長春殿見上,言曰:「臣等若各述所見,則非詢謀僉同之議,望許其為一狀,陳其利害。」洎越次奏曰:「端等備位輔弼,上有
 所詢問,反緘默不言,深失訐謨之體。」端曰:「洎欲有言,不過揣摩陛下意耳,必無鯁切之理。」上默然。翌日,洎上疏引賈捐之棄珠崖事,願棄靈武以省關西饋運。上嘗有此意,既而悔之,洎果迎合,覽奏不悅。既以疏付洎,謂之曰:「卿所陳,朕不曉一句。」洎惶恐而退。上召同知樞密院事向敏中等謂曰:「張洎上言,果為呂端所料,朕已還其疏矣。」



 洎既議事不稱旨,恐懼,欲自固權位。上已嫉準專恣,恩寵衰替。洎慮一旦同罷免,因奏事,大言寇準退後
 多誹謗。準但色變,不敢自辯。上由是大怒,準旬日罷。未幾,洎病在告,滿百日,力疾請對,方拜,踣於上前,左右掖起之。明日,上章求解職,優詔不允。後月餘,改刑部侍郎,罷知政事。奉詔嗚咽,疾遂亟,十餘日卒,年六十四。贈刑部尚書,以其二子皆為京官。



 洎風儀灑落,文採清麗,博覽道釋書,兼通禪寂虛無之理。終日清談,亹癖可聽。尤險詖,好攻人之短。李煜既歸朝,貧甚,洎猶丐索之。煜以白金摐面器與洎,洎尚未滿意。時潘慎修掌煜記室,洎
 疑慎修教煜,素與慎修善,自是亦稍疏之。煜子仲宇雅好蒱博飲宴,洎因切諫之,仲宇謝過。後數月,人有言仲宇蒱博如故,洎遂與之絕。及仲宇死郢州,葬京師,洎亦不赴吊。與張佖議事不協,遂為仇隙,始以從父禮事佖,既而不拜。尤善事內官,在翰林日,引唐故事,奏內供奉官藍敏政為學士使,內侍裴愈副之。上覽奏,謂曰:「此唐室弊政,朕安可踵此覆轍,卿言過也。」洎慚而退。性鄙吝,雖親戚無所沾,及江表故舊,亦罕登其門。素與徐鉉厚
 善,後因議事相忤,遂絕交。然手寫鉉文章,訪求其筆札,藏篋笥,甚於珍玩。洎有文集五十卷行於世。



 子安期,至國子博士;方回,後為虞部員外郎。方回子懷玉,王欽若婿,賜進士及第,大理寺丞,秘書校理。



 李惟清,字直臣,下邑人。父仲行,為章丘簿,因徙家焉。惟清,開寶中,以三史解褐涪陵尉。蜀民尚淫祀,病不療治,聽於巫覡,惟清擒大巫笞之,民以為及禍。他日又加棰焉,民知不神。然後教以醫藥,稍變風俗。時遣宦官督輸
 造船木,縱恣不法,惟清奏殺之,由是知名。秩滿,遷大理寺丞。



 太平興國三年,遷為荊湖北路轉運判官。五年,改左贊善大夫,充轉運副使,升正使,就改監察御史,兼總南路。嘗入奏事,太宗問曰:「荊湖累年豐稔,又無徭役,民間蘇否?」惟清曰:「臣見官賣鹽斤為錢六十四,民以三數斗稻價,方可買一斤。」。乃詔斤減十錢。徙京西轉運使,入為度支判官,改主客員外郎。



 雍熙三年,大舉取幽州,惟清以為兵食未豐,不可輕動。朝廷業已興師,奏入不報。
 判度支許仲宣建議通鹽法,以賣鹽歲課賦於鄉村,與戶稅均納。惟清奉詔往荊湖諸路詳定,奏言以鹽配民非便,遂罷。使還,上又問民間苦樂不均事,惟清言:「前在荊湖,民市清酒務官釀轉鬻者,鬥給耗二升,今三司給一升,民多他圖,而歲課甚減。」詔復其舊。未幾,出為京東轉運使。會募丁壯為義軍,惟清曰:「若是,天下不耕矣。」三上疏諫,繇是獨選河北,而餘路悉罷。擢屯田郎中、度支副使。



 端拱初,遷右諫議大夫,歷戶部使,改度支使。會遣
 使河朔治方田,大發兵。惟清以盛春妨農,懇求罷廢。太宗曰:「兵夫已發矣。止令完治邊城而已。」淳化三年,遷給事中,充鹽鐵使,遂以帳式奏御。太宗曰:「費用若此,民力久何以堪?如可減省,即便裁度。」惟清曰:「此開寶軍興之際,其數倍多,蓋以將帥未得其人,邊事未寧,屯兵至廣也。臣聞漢有衛青、霍去病,唐有郭子儀、李晟,西北望而畏之。如此則邊事息而支用減矣。望慎擢將帥,以有威名者俾安邊塞,庶節費用。」上言:「彼一時,此一時也。今之
 西北變詐,與古不同。選用將帥,亦須深體今之幾宜。韓、彭雖古之名將,以彼時之見,制今之敵,亦恐不能成功。今縱得人,未可便如古委之。此乃機事,卿所未知也。」



 淮南榷貨務賣嶽茶,斤為錢百五十。主吏言陳惡者二十六萬六千餘斤,惟清擅減斤五十錢,不以聞。滁、泗、濠、楚州、漣水軍亦以嶽茶陳惡,減價市之。計虧錢萬四千餘貫,為勾院吏盧守仁所發,左授衛尉少卿,黜判官李管為本曹員外郎,賜守仁錢十五萬。俄出知廣州。至道初,
 就拜右諫議大夫。太宗聞其廉平,詔獎之。二年,徙廣南東、西路都轉運使,尋召拜給事中。逾月,同知樞密院事。



 惟清倜儻自任,有鉤距。臨事峻刻,所至稱強幹。然以俗吏進,無人望。才數月,真宗即位,加刑部侍郎,復除御史中丞。既去樞要,怫鬱尤甚,肆情彈擊。咸平元年。卒,年五十六,贈戶部尚書。



 子永錫,蔭至光祿寺丞。頗涉學屬辭,尚氣少檢,喜交結。馮拯、王濟、皇甫選多與之游,日聚舉子於家,談議時政。真宗將幸河朔,永錫猶服父喪,上章
 大言,列詆近臣,自謂有致太平滅敵之術。選為戶部判官,因對,袖表以獻,又自薦揚。真宗駐蹕大名,召赴行在,試策不中,貶瀧水縣主簿。選為南劍州團練副使,俄復光祿寺丞。六年,又坐交游非類,監和州商稅,後至右贊善大夫。



 次子永德,至殿中丞。



 論曰:張宏為樞副,當用兵之際,循默備位;趙昌言為御史中丞,屢上書言兵,乃兩易之。中丞可使循默者居之乎?宋失政矣。昌言識李沆,器王旦;陳恕取士得王曾,舉
 代得寇準;皆可謂知人之明。然趙好獎拔,而頗樹黨與,終以取敗;陳典貢舉,務黜南士,以避嫌疑,皆非君子所為也。昌言尚氣敢言,恕為宋人能吏之首,庶足稱矣。劉昌言感趙普之遇,身後經理其家;然委親鄉里,十年而不迎侍,厚薄失措,又何取乎?張洎初勸李煜勿降,既而不能死之,「犬吠非主」之對,徒以辯舌,僥幸得免。厥後揣摩百端,讒毀正直,利口之士,鮮不為反復小人也。李惟清居臺端,恨失政柄,恣情鷙擊。舊史稱為俗吏,又奚責
 焉



\end{pinyinscope}