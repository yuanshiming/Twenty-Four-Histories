\article{列傳第二百四文苑七}

\begin{pinyinscope}

 ○陳與義汪藻葉夢得程俱張嵲韓駒朱敦儒葛勝仲熊克張即之趙蕃
 附



 陳與義,字去非,其先居京兆,自曾祖希亮始遷洛。故為洛人。與義天資卓偉,為兒時已能作文,致名譽,流輩斂衽,莫敢與抗。登政和三年上舍甲科,授開德府教授。累遷太學博士,擢符寶郎,尋謫監陳留酒稅。



 及金人入汴,高宗南遷,遂避亂襄漢,轉湖湘,踰嶺嶠。久之,召為兵部員外郎。紹興元年夏,至行在。遷中書舍人,兼掌內制。拜吏部侍郎,尋以徽猷閣直學士知湖州。召為給事中。駁議詳雅。又以顯謨閣直學士提舉江州太平觀。被召,會
 宰相有不樂與義者,復用為中書舍人、直學士院。六年九月,高宗如平江,十一月,拜翰林學士、知制誥。



 七年正月,參知政事,唯師用道德以輔朝廷,務尊主威而振綱紀。時丞相趙鼎言:「人多謂中原有可圖之勢,宜便進兵,恐他時咎今日之失機。」上曰:「今梓宮與太后、淵聖皆未還,若不與金議和,則無可還之理。」與義曰:「若和議成,豈不賢於用兵,萬一無成,則用兵必不免。」上曰:「然。」三月,從帝如建康。明年,扈蹕還臨安。以疾請,復以資政殿學士
 知湖州,陛辭,帝勞問甚渥,遂請閑,提舉臨安洞霄宮。十一月,卒,年四十九。



 與義容狀儼恪,不妄言笑,平居雖謙以接物,然內剛不可犯。其薦士於朝,退未嘗以語人,士以是多之。尤長於詩,體物寓興,清邃紆餘,高舉橫厲,上下陶、謝、韋、柳之間。嘗賦墨梅,徽宗嘉賞之,以是受知於上云。



 汪藻,字彥章,饒州德興人。幼穎異,入太學,中進士第。調婺州觀察推官,改宣州教授,稍遷江西提舉學事司乾
 當公事。



 徽宗親制《君臣慶會閣詩》,群臣皆賡進,惟藻和篇,眾莫能及。時胡伸亦以文名,人為之語曰:「江左二寶,胡伸、汪藻。」尋除《九域圖志》所編修官,再遷著作佐郎。時王黼與藻同舍,素不咸,出通判宣州,提點江州太平觀,投閑凡八年,終黼之世不得用。



 欽宗即位,召為屯田員外郎,再遷太常少卿、起居舍人。高宗踐祚,召試中書舍人。時次揚州,藻多論奏,宰相黃潛善惡之,遂假他事,免為集英殿修撰、提舉太平觀。明年,復召為中書舍人
 兼直學士院,擢給事中,遷兵部待郎兼侍講,拜翰林學士。帝以所御白團扇,親書「紫誥仍兼綰,黃麻似《六經》」十字以賜,縉紳艷之。



 屬時多事,詔令類出其手。嘗論諸大將擁重兵,浸成外重之勢,且陳所以待將帥者三事,後十年,卒如其策。又言:「崇、觀以來,貲結權幸,奴事閹宦,與開邊誤國,得職名自觀文殿大學士而下直秘閣、官至銀青光祿大夫者,近稍鐫褫,而建炎恩宥,又當甄復,盍依國初法,止中大夫。」



 紹興元年,除龍圖閣直學士、知湖
 州,以顏真卿盡忠唐室,嘗守是邦,乞表章之,詔賜廟忠烈。又言:「古者有國必有史,古書榻前議論之辭,則有時政記,錄柱下見聞之實,則有起居注,類而次之,謂之日歷,修而成之,謂之實錄。今逾三十年,無復日歷,何以示來世?乞即臣所領州,許臣訪尋故家文書,纂集元符庚辰以來詔旨,為日歷之備。」制可。史館既開,修撰綦崇禮言不必別設外局,乃已。郡人顏經投匭訴其敷糴軍食,遂貶秩停官。起知撫州,御史張致遠又論之,予祠。六年,
 修撰範沖言:「日歷,國之大典,比詔藻纂修,事復中止,恐遂散逸,宜令就閑復卒前業。」詔賜史館修撰餐錢,聽闢屬編類。八年,上所修書,自元符庚辰至宣和乙巳詔旨,凡六百六十五卷。藻再進官,其屬鮑延祖、孟處義咸增秩有差。藻升顯謨閣學士,遣使賜茶藥。尋知徽州,逾年,徙宣州。言者論其嘗為蔡京、王黼之客,奪職居永州,累赦不宥。二十四年,卒。



 秦檜死,復職,官其二子。二十八年,《徽宗實錄》成書,右僕射湯思退言藻嘗纂集詔旨,比修實
 錄,所取十蓋七八,深有力於斯文。詔贈端明殿學士。



 藻通顯三十年,無屋廬以居。博極群書,老不釋卷,尤喜讀《春秋左氏傳》及《西漢書》。工儷語,多著述,所為制詞,人多傳誦。子六人,恬、恪、憺、怲、懍、憘。



 葉夢得,字少蘊,蘇州吳縣人。嗜學蚤成,多識前言往行,談論亹癖不窮。紹聖四年登進士第,調丹徒尉。徽宗朝,自婺州教授召為議禮武選編修官。用蔡京薦,召對,言:「自古帝王為治,廣狹大小,規模各不同,然必自先治其
 心者始。今國勢有安危,法度有利害,人材有邪正,民情有休戚,四者,治之大也。若不先治其心,或誘之以貨利,或陷之以聲色,則所謂安危、利害、邪正、休戚者,未嘗不顛倒易位,而況求其功乎?」上異其言,特遷祠部郎官。



 大觀初,京再相,向所立法度已罷者復行,夢得言:「《周官》太宰以八柄詔王馭群臣,所謂廢置賞罰者,王之事也,太宰得以詔王而不得自專。夫事不過可不可二者而已,以為可而出於陛下,則前日不應廢,以為不可而不出
 於陛下,則今不可復,今徒以大臣進退為可否,無乃陛下有未了然於中者乎?」上喜曰:「邇來士多朋比媒進,卿言獨無觀望。」遂除起居郎。時用事者喜小有才,夢得言:「自古用人必先辨賢能。賢者,有德之稱,能者,有才之稱,故先王常使德勝才,不使才勝德。崇寧以來,在內惟取議論與朝廷同者為純正,在外惟取推行法今速成者為幹敏,未聞器業任重、識度輕遠者,特有表異。恐用才太勝,願繼今用人以有德為先。」



 二年,累遷翰林學士,極
 論士大夫朋黨之弊,專於重內輕外,且乞身先眾人補郡。蔡京初欲以童貫宣撫陜西,取青唐。夢得見京問曰:「祖宗時,宣撫使皆是見任執政,文彥博,韓絳因此即軍中拜相,未有以中人為之。元豐末,神宗命李憲,雖王珪亦能力爭,此相公所見也。昨八寶恩遽除貫節度使,天下皆知非祖宗法,此已不可救。今又付以執政之任,使得青唐,何以處之?」京有慚色,然卒用貫取青唐。



 三年,以龍圖閣直學士知汝州,尋落職,提舉洞霄宮。政和五
 年,起知蔡州,復龍圖閣直學士。移帥潁昌府,發常平粟振民,常平使者劉寄惡之。宦官楊戩用事,寄括部內,得常平錢五十萬緡,請糴粳米輸後苑以媚戩。戩委其屬持御筆來,責以米樣如蘇州。夢得上疏極論潁昌地力與東南異,願隨品色,不報。時旁郡糾民輸鏹就糴京師,怨聲載道,獨潁昌賴夢得得免。李彥括公田,以黠吏告訐,籍郟城、舞陽隱田數千頃,民詣府訴者八百戶。夢得上其事,捕吏按治之,郡人大悅。戩、彥交怒,尋提舉南京
 鴻慶宮,自是或廢或起。



 逮高宗駐蹕揚州,遷翰林學士兼侍讀,除戶部尚書。陳「待敵之計有三,曰形、曰勢、曰氣而已。形以地理山川為本,勢以城池、芻粟、器械為重,氣以將帥士卒為急。形固則可恃以守,勢強則可資以立,氣振則可作以用,如是則敵皆在吾度內矣」。因請上南巡,阻江為險,以備不虞。又請命重臣為宣總使,一居泗上,總兩淮及東方之師以待敵;一居金陵,總江、浙之路以備退保。疏入,不報。



 既而帝駐蹕杭州,遷尚書左丞,奏
 監司、州縣擅立軍期司掊斂民財者,宜罷。上諭以兵、食二事最大,當擇大臣分掌。門下侍郎顏岐、知杭州康允之皆嫉夢得,又與宰相朱勝非議論不協,會州民有上書訟夢得過失者,上以夢得深曉財賦,乃除資政殿學士、提舉中太一宮,專一提領戶部財用,充車駕巡幸頓遞使,辭不拜,歸湖州。



 紹興初,起為江東安撫大使兼知建康府,兼壽春等六州宣撫使。時建康荒殘,兵不滿三千。夢得奏移統制官韓世清軍屯建康,崔增屯採石,閻
 皋分守要害。會王才降劉豫,引兵入寇,夢得遣使臣張偉諭才降之,以其眾分隸諸軍。濠、壽叛將寇宏、陳卞雖陽受朝命,陰與劉豫通,夢得諭以福禍,皆聽命。及豫入寇,卞擊敗之,齊兵宵遁。



 八年,除江東安撫制置大使兼知建康府、行宮留守。又奏防江措畫八事:一、申飭邊備,二、分布地分,三、把截要害,四、約束舟船,五、團結鄉社,六、明審斥堠,七、措置積聚,八、責官吏死守。又言建康、太平、池州緊要隘口、江北可濟渡去處共一十九處,願聚集
 民兵,把截要害,命諸將審度敵形,並力進討。



 金都元帥宗弼犯含山縣,進逼歷陽,張俊諸軍遷延未發,夢得見俊,請速出軍,曰:「敵已過含山縣,萬一金人得和州,長江不可保矣。」俊趣諸軍進發,聲勢大振,金兵退屯昭關。明年,金復入寇,遂至柘皋,夢得團結沿江民兵數萬,分據江津,遣子模將千人守馬家渡,金兵不得渡而去。



 初,建康屯兵歲費錢八百萬緡,米八十萬斛,榷貨務所入不足以支。至是,禁旅與諸道兵咸集,夢得兼總四路漕計
 以給饋餉,軍用不乏,故諸將得悉力以戰。詔加觀文殿學士,移知福州,兼福建安撫使。



 海寇朱明猖獗,詔夢得挾御前將士便道之鎮,或招或捕,或誘之相戕,遂平寇五十餘群。然頗與監司異議,上章請老,特遷一官,提舉臨安府洞霄宮。尋拜崇信軍節度使致仕。十八年,卒湖州,贈檢校少保。



 程俱,字致道,衢州開化人。以外祖尚書左丞鄧潤甫恩,補蘇州吳江主簿,監舒州太湖茶場,坐上書論事罷歸。
 起知泗州臨淮縣,累遷將作監丞。近臣以詵述薦,遷著作佐郎。宣和二年,進頌,賜上舍出身,除禮部郎,以病告老,不俟報而歸。



 建炎中,為太常少卿、知秀州。會車駕臨幸,賜對。俱言:「陛下德日新,政日舉,賞罰施置,仰當天意,俯合人心,則趙氏安而社稷固;不然,則宗社危而天下亂,其間蓋不容發。」高宗嘉納之。金兵南渡,據臨安,遣兵破崇德、海鹽,馳檄諭降。俱率官屬棄城保華亭,留兵馬都監守城。朝廷命俱部金帛赴行在,既至,以病乞歸。



 紹
 興初,始置秘書省,召俱為少監。奏修日歷,秘書長貳得預修纂,自俱始。時庶事草創,百司文書例從省記,俱摭三館舊聞,比次為書,名曰《麟臺故事》上之。擢中書舍人兼侍講。俱論:「國家之患,在於論事者不敢盡情,當事者不敢任責,言有用否,事有成敗,理固不齊。今言不合則見排於當時,事不諧則追咎於始議。故雖有智如陳平,不敢請金以行間;勇如相如,不敢全璧以抗秦;通財如劉晏,不敢言理財以贍軍食。使人人不敢當事,不敢盡
 謀,則艱危之時,誰與圖回而恢復乎?」



 武功大夫蘇易轉橫行,俱論:「祖宗之法,文臣自將作監主簿至尚書左僕射,武臣自三班奉職至節度使,此以次遷轉之官也。武臣自閣門副使至內客省使為橫行,不系磨勘遷轉之列,其除授皆頒特旨。故元豐之制,以承務郎至特進為寄祿官,易監主簿至僕射之名;武臣獨不以寄祿官易之者,蓋有深意也。政和間,改武臣官稱為郎、大夫,遂並橫行易之為轉官等級,蓋當時有司不習典故,以開僥
 幸之門。自改使為大夫以來,常調之官,下至皂隸,轉為橫行者,不可勝數。且文臣所謂庶官者,轉不得過中大夫,而武臣乃得過皇城使,此何理也!夫官職輕重在朝廷,朝廷愛重官職,不妄與人,則官職重;反是則輕,輕則得者不以為恩,未得者常懷觖望,此安危治亂所關也。」



 徐俯為諫議大夫,俱繳還,以為:「俯雖才俊氣豪,所歷尚淺,以前任省郎,遽除諫議,自元豐更制以來,未之有也。昔唐元稹為荊南判司,忽命從中出,召為省郎,使知制
 誥,遂喧朝聽,時謂監軍崔潭峻之所引也。近聞外傳,俯與中官唱和,有『魚須』之句,號為警策。臣恐外人以此為疑,仰累聖德。陛下誠知俯,姑以所應得者命之。」不報。後二日,言者論俱前棄秀州城,罷為提舉江州太平觀。久之,除徽猷閣待制。



 俱晚病風痺,秦檜薦俱領史事,除提舉萬壽觀、實錄院修撰,使免朝參,俱力辭不至。卒,年六十七。俱在掖垣,命令下有不安於心者,必反覆言之,不少畏避。其為文典雅閎奧,為世所稱。



 張嵲,字巨山,襄陽人。宣和三年,上舍選中第。調唐州方城尉,改房州司刑曹。劉子羽薦於川、陜宣撫使張浚,闢利州路安撫司干辦公事,以母病去官。



 紹興五年,召對,嵲上疏曰:「金人去冬深涉吾地,王師屢捷,一朝宵遁,金有自敗之道,非我幸勝之也。今士氣稍振,乘其銳而用之;固無不可。然兵疲民勞,若便圖進取,似未可遽。臣竊謂為今日計,當築塢堡以守淮南之地,興屯田以為久戍之資,備舟楫以阻長江之險,以我之常,待彼之變。又
 荊、襄、壽春皆古重鎮,敵之侵軼,多出此途。願速擇良將勁兵,戍守其地,以重上流之勢。」召試,除秘書省正字。



 六年,地震。嵲奏:「比年以來,賦斂繁重,徵求百出,流移者擠溝壑,土著者失常業,地震之異,殆或為此。願深思變異之由,修政之闕,致民之安。」



 七年,遷校書郎兼史館校勘,再遷著作郎。嵲因對言:「吳、蜀,唇齒之勢也。蜀去朝廷遠,今無元帥一年矣。蜀之利害,臣粗知之。忠勇之人,使之捍外侮則可,至於撫循斯民,則非所能辦也。宜於前宰
 執中,擇其可以任川事者委任之。然川蜀系國利害,非腹心之臣不可,今早得一賢宣撫使為要。」又言:「自駐蹕吳會以來,似未嘗以襄陽、荊南為意,今宜亟選儒臣有牧御之才者為二路帥,使之招集流散,興農桑,治城壁,以為保固之資,益重上流之勢。」



 即而何掄以刊改《神宗實錄》得罪,語連嵲,出為福建路轉運判官。上疏略曰:「古之人君,其患有二,不在於拒諫,在納諫而不能用;不在於不知天下利害,在知而不以為意。陛下渡江十年矣,
 外有勍敵之國,內有驕悍之兵,下有窮困無聊之民。進言者多矣,今皆以為陳腐而別取新奇之說;任事者眾矣,今皆習是以為當然而更為迂闊之事。此近於納諫而不知用,知利害而不知恤也。為今之計,朝斯夕斯,非是二者不務,數年之後,庶其有濟!有國之所惡者,莫大於朋黨,今一宰相用,凡其所與者,不擇賢否而盡用之,一宰相去,凡其所與者,不擇賢否而盡逐之,宜其朋黨之浸成也。」



 九年,除司勛員外郎兼實錄院檢討官。金人
 叛盟,上命兩省、卿、監、郎、曹各草檄以進,獨取嵲所進者,播之四方。十年,擢中書舍人,升實錄院同修撰。論王德收復宿、亳兩郡,乃擅退軍,使岳飛勢孤,金人猖獗,授承宣防禦使,何應罰而反賞?封還詞頭,乞罷已降轉官指揮。未幾,右正言萬俟卨論嵲為侍從日,薦引非才,以酬私恩,邊報始至,托疾家居,由是罷去。頃之,起知衢州,除敷文閣待制。為政頗尚嚴酷,歲滿,得請提舉江州太平興國宮。時方修好息兵,朝廷講稽古禮文之事,嵲作《中
 興復古詩》以進。上將召用,會疽發背卒,年五十三。子昌時。



 韓駒,字子蒼,仙井監人。少有文稱。政和初,以獻頌補假將仕郎,召試舍人院,賜進士出身,除秘書省正字。尋坐為蘇氏學,請監華州蒲城縣市易務。知洪州分寧縣。召為著作郎,校正御前文籍。駒言國家祠事,歲一百十有八,用樂者六十有二,舊撰樂章,辭多牴牾。於是詔三館士分撰親祠明堂、圓壇、方澤等樂曲五十餘章,多駒所
 作。



 宣和五年,除秘書少監。六年,遷中書舍人兼修國史,入謝。上曰:「近年為制誥者,所褒必溢美,所貶必溢惡,豈王言之體。且《盤》、《誥》具在,寧若是乎?」駒對:「若止作制誥,則粗知文墨者皆可為,先帝置兩省,豈止使行文書而已。」上曰:「給事實掌封駁。」駒奏:「舍人亦許繳還詞頭。」上曰:「自今朝廷事有可論者,一切繳來。」尋兼權直學士院,制詞簡重,為時所推。未幾,復坐鄉黨曲學,以集英殿修撰提舉江州太平觀。



 高宗即位,知江州。紹興五年,卒於撫州。
 進一官致仕,贈中奉大夫,與遺澤三人。駒嘗在許下從蘇轍學,評其詩似儲光羲。其後由宦者以進用,頗為識者所薄雲。子遜、游。



 朱敦儒,字希真,河南人。父勃,紹聖諫官。敦儒志行高潔,雖為布衣,而有朝野之望。靖康中,召至京師,將處以學官,敦儒辭曰:「麋鹿之性,自樂閑曠,爵祿非所願也。」固辭還山。高宗即位,詔舉草澤才德之士,預選者命中書策試,授以官,於是淮西部使者言敦儒有文武才,召之。敦
 儒又辭。避亂客南雄州,張浚奏赴軍前計議,弗起。



 紹興二年,宣諭使明TY言敦儒深達治體,有經世才,廷臣亦多稱其靖退。詔以為右迪功郎,下肇慶府敦遣詣行在,敦儒不肯受詔。其故人勸之曰:「今天子側席幽士,翼宣中興,譙定召於蜀,蘇庠召於浙,張自牧召於長蘆,莫不聲流天京,風動郡國,君何為棲茅茹藿,白首巖谷乎!」敦儒始幡然而起。既至,命對便殿,論議明暢。上悅,賜進士出身,為秘書省正字。俄兼兵部郎官,遷兩浙東路提點刑
 獄。會右諫議大夫汪勃劾敦儒專立異論,與李光交通。高宗曰:「爵祿所以厲世,如其可與,則文臣便至侍從,武臣便至節鉞。如其不可,雖一命亦不容輕授。」郭儒遂罷。十九年,上疏請歸,許之。



 郭儒素工詩及樂府,婉麗清暢。時奏檜當國,喜獎用騷人墨客以文太平,檜子熺亦好詩,於是先用敦儒子為刪定官,復除敦儒鴻臚少卿。檜死,郭儒亦廢。談者謂敦儒老懷舐犢之愛,而畏避竄逐,故其節不終云。



 葛勝仲,字魯卿,丹陽人。登紹聖四年進士第,調杭州司理參軍。林希薦試學官及詞科,俱第一,除兗州教授,入為太學正。上幸學,多獻頌者,勝仲獨獻賦,上命中書第其優劣,勝仲為首,差提舉議歷所檢討官兼宗正丞。始,朝廷以從臣提舉議歷所,至是,代以郭天信,勝仲力請罷之。稍遷禮部員外郎。會御史中丞石公弼言:「僖祖原廟增置殿室,違元豐之舊。」詔禮官議。勝仲建言:「予而復奪,在常人猶難之,況在天之靈乎!」議者非之,責知歙州
 休寧縣,復召為禮部員外郎,權國子司業。時朝廷命諸生習雅樂,樂成,進一官,遷太常少卿。



 宋自建隆至治平所行典禮,歐陽修嘗裒集為書,凡百篇,號《太常因革禮》,詔勝仲續之,增為三百卷,詔藏太常。及建春宮,以勝仲兼諭德,勝仲為《仁》、《孝》、《學》三論獻之太子,復採春秋、戰國以來歷代太子善惡成敗之跡,日進數事。詔嘉之,徙太府少卿,除國子祭酒,尋知汝州。李彥括田,破產者眾,勝仲請蠲不當括者,彥怒,劾勝仲,上寢其奏,改湖州,尋徙鄧
 州。朱勔先求白雀之屬,勝仲不與,至是媒蘗其短,罷歸。



 建炎中,範宗尹為相,凡前日以朋附被罪遠貶者,咸赦還,復知湖州,時群盜縱橫,聲搖諸郡,勝仲修城郭,作戰艦,閱士卒,賊知有備,引去。歲大饑,發官廩振之,民賴以濟。紹興元年,丐祠歸。十四年,卒,年七十三,謚文康。子立方,官至侍從。孫邲,為右相,自有傳。



 熊克,字子復,建寧建陽人,御史大夫博之後。將生,有翠羽雀翔臥內。克幼而翹秀,既長,好學善屬文,郡博士胡
 憲器之,曰:「子學老於年,他日當以文章顯。」紹興中進士第,知紹興府諸暨縣,越帥課賦頗急,諸邑率督趣以應,克曰:「寧吾獲罪,不忍困吾民。」他日,府遣幕僚閱視有亡,時方不雨,克對之泣曰:「此催租時耶!」部使者芮輝行縣至其境,謂克曰:「曩知子文墨而已,今乃見古循吏。」為表薦之,入為提轄文思院。



 嘗以文獻曾覿,覿持白於孝宗,孝宗喜之,內出御筆,除直學士院。宰相趙雄甚異之,因奏曰:「翰院清選,熊克小臣,不由論薦而得,無以服眾論,請自
 朝廷召試,然後用之。」上曰:「善。」乃以為校書郎,累遷學士院權直,上御選德殿,召諭曰:「卿制誥甚工,且有體,自此燕閑可論治道。」



 克自以見知於上,數有論奏。嘗言:「金人雖講和,而不能保於他日,今宜以和為守,以守為攻。當和好之時,為備守之計,彼不能禁吾不為也。邊備既實,金人萬一猖獗,必不得志於我,退而乘我,曲不在我矣。且今日之守,莫重淮東。金犯淮西,負糧自隨,其勢必難。若犯淮東,清河糧船直下,易耳。然則守淮之策,以墾田、
 修堰、教民兵為先。援淮東之策,莫若即江陰建水軍,緩急可相應。然驟立一軍,慮敵生疑,當托以海道商賈之沖,多奪攘,置一巡檢警督之,自此歲增兵,不出十年,隱然一軍矣。中興之際,不患兵不可用,而患將權難收。今日之弊,不患將不可馭,而患軍情易動。往時諸大將拊士卒如家人,自罷諸將兵權,御前主帥,更徙不常,凡軍中筦榷之利,所以養士卒者,今皆轉而為包苴矣,又朘其餘以佐之,得無怨乎!宜嚴戒將帥,毋縱掊削。」帝嘉其
 有志,召草明堂赦書。克言:「二浙薦饑,蝗且起,赦文不宜飾詞。」帝嘉其識體。除起居郎兼直學士院,以言者出知臺州,奉祠。



 克博聞強記,自少至老,著述外無他嗜。尤淹習宋朝典故,有問者酬對如響。家素儉約,雖貴不改,舊所居卑陋,門不容轍,雖部使者、郡守至,必降車乃入。嘗愛臨川童子王克勤之才,將妻以女而乏資遣,會草制獲賜金,遂以歸之,人稱其清介。卒,年七十三。



 張即之,字溫夫,參知政事孝伯之子。以父恩授承務郎,
 銓中兩浙轉運司進士舉,歷監平江府糧料院。丁父憂,服除,監臨安府樓店務。丁母憂,服除,監臨安府龍山稅、寧國府城下酒曲務,簽書荊門軍判官廳公事,烏程丞,特差簽書江陰軍判官廳公事,提領戶部犒賞酒庫所乾辦公事,添差兩浙轉運司主管文字,行在檢點贍軍激賞酒庫所主管文字,監尚書六部門,淮南東路提舉常平司主管文字,添差通判揚州,改鎮江,又改嘉興,將作監簿,軍器監丞,司農寺丞,知嘉興,未赴,以言者罷,丐
 祠,主管雲臺觀,引年告老,特授直秘閣致仕。



 寶祐四年,制置使餘晦入蜀,以讒劾閬州守王惟忠。於是削惟忠五官,沒入其資,下詔獄鍛煉誣伏,坐棄市。惟忠臨刑,謂其友陳大方曰:「吾死當上訴於天。」七揮刃不殊,血逆流。即之雖閑居,移書言於淮東制置使賈似道恤其遺孤。又使從孫士倩娶惟忠孤女。未幾,似道入相,中書舍人常挺變以為言。景定元年,給還首領,以禮改葬,復金壇田,多即之倡義云。即之以能書聞天下,金人尤寶其翰
 墨。



 惟忠字肖尊,慶元之鄞人,嘉定十三年進士。



 趙蕃字昌父,其先鄭州人。建炎初,大父暘以秘書少監出提點坑冶,寓信州之玉山。蕃以暘致仕恩,補州文學。調浮梁尉、連江主簿,皆不赴。為太和主簿,受知於楊萬里。調辰州司理參軍,與郡守爭獄,罷,人以蕃為直。



 始,蕃受學於劉清之,清之守衡州,乃求監安仁贍軍酒庫,因以卒業。至衡而清之罷,蕃即丐祠,從清之歸。其後真德秀書之《國史》曰:「蕃於師友之際蓋如此,肯負國乎!」家居,
 連書祠官之考者三十有一,理宗即位,以太社令與劉宰同召,不拜,特改奉議郎、直秘閣,又辭。奉祠,得致仕,轉承議郎,依前直秘閣。卒,年八十七。



 蕃年五十,猶問學於朱熹。既耄,猶患末路之難,命所居曰難齋。蕃賦性寬平,與人樂易而剛介不可奪。丞相周必大與蕃契,屢加引薦,蕃竟不受。宰之言曰:「文獻之家,典刑之彥,巋然獨存,猶有以系學者之望者,蕃一人而已。」信州守吳旂乞錄其後,詔其子遂補上州文學,遂亦力辭。又詔以承務郎
 致仕,與一子恩澤。景定三年,秘閣修撰鄭協等請謚,乃謚文節。



\end{pinyinscope}