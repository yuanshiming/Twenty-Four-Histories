\article{志第一百 儀衛五}

\begin{pinyinscope}

 紹興鹵簿
 皇太后皇后鹵簿皇太子鹵簿妃附王公以下鹵簿



 紹興鹵簿。宋初,大駕用一萬一千二百二十二人。宣和,增用二萬六十一人。建炎初,裁定一千三百三十五人。
 紹興初,用宋初之數,十六年以後,遂用一萬五千五十人;明堂三分省一,用一萬一十五人,孝宗用六千八百八十九人,明堂用三千三百十九人。以後,並用孝宗之數。



 紹興用象六、副像一。乾道用像一,淳熙用象六而不設副,紹熙如乾道,慶元後不設。



 六引。第一引,清道二人;孝宗省之。



 幰弩一人,騎;方傘一,雜花扇二,曲蓋一;外仗青衣二人,車輻棒二,告止、傳教、信幡各二,戟十。第二引,清道二人;孝宗省之。



 幰弩一人,騎;鼓一,鉦一,大鼓十;節一,槊二,皆騎;方傘一,雜花扇四,孝宗省為二。



 曲
 蓋一,幢一,麾一,皆騎;大角四,鐃一,簫二,笳二,橫吹二,笛一,簫一,觱慄一,笳一;外仗青衣四人,孝宗省為二。



 車輻棒四,孝宗省為二。



 告止、傳教、信幡各二,儀刀十,戟二十,弓矢二十,孝宗皆省為十六。



 刀盾二十,槊二十。孝宗並省。



 第三、第四、第五、第六引,並同第二引。內花扇、大角各二,青衣二人。孝宗朝,第三、第四、第五、第六引內大角省為二,餘並同第二引已省之數。



 金吾纛槊隊。纛十二,孝宗省為六。



 押纛二人,孝宗省為一。



 押衙四人,孝宗省為二。



 上將軍四人,將軍四人,孝宗省之。



 大將軍二人,孝宗省為一。



 犦槊十二,並騎。孝宗省為八。



 朱雀隊。朱雀旗一,牛暴槊二,弩四,隊前後引、押各天武都指揮使一人,騎。龍旗隊。引旗一,風師、雨師、雷旗、電旗各一,五星旗五,攝提旗二,北斗旗一,護旗一,左右衛大將軍一人。金吾引駕騎,神勇都指揮使;次弩、弓矢、槊各四,並騎。



 太常前部鼓吹。鼓吹令二,府史四人,管轄指揮使一人,帥兵官三十六人,孝宗省作十四人。



 □鼓十二,金鉦十二,孝宗鼓、鉦並省為十。



 大鼓六十,孝宗省作二十四。



 小鼓六十,孝宗省作三十。



 節鼓一,鐃鼓六,羽葆鼓六,歌工二十四,拱宸管二十
 四,孝宗歌工、管並省為十八。



 簫、笳各三十六,孝宗朝,簫十八、笳二十四。



 長鳴六十,中鳴六十,孝宗朝,並省為十八。



 大橫吹六十,孝宗省為二十四。笛十二,孝宗增為十八。



 觱慄十二,桃皮觱慄十二。



 持鈒前隊。驍騎都指揮使一人,將軍二人,軍使四人,並騎。稱長一人,靈芝旗二,瑞瓜旗二,雙蓮花旗二,太平瑞木旗二,朱雀旗一,甘露旗二,嘉禾旗二,芝草旗二。絳引幡一,孝宗省之。



 黃麾幡一,青龍、白虎幢各一,金節十二,罕、畢各一,叉一,鈒戟五十。孝宗省為四十八。



 六軍儀仗。第一隊,軍將二,卒長二,騎。熊虎旗
 二,赤豹旗二,吏兵旗、力士旗二,戈六,矛四,戟四,鉞四,白柯槍五十。平列旗二十,在仗外分夾旗槍。第二隊,軍將二,卒長二,騎。龍君旗、虎君旗各三,黃熊旗四,赤豹旗二,吏兵旗、力士旗各一,戈六,矛四,戟四,鉞四,白柯槍四十。平列旗二十,分夾仗外。第三隊,軍將二,卒長二,騎。通直官二,吏兵旗、力士旗各一,熊虎旗二,龍君旗、虎君旗各一,天王旗四,十二辰旗各一,戈六,矛、戟、鉞各四,白柯槍三十。平列旗二十,分夾仗外。



 孝宗朝,第一隊,軍將、卒長各一,龍虎旗、赤豹旗、吏兵旗、力
 士旗各二,矛四、戈四、戟二、鉞二、白柯槍三十,平列旗十四,餘同。第二隊軍將、卒長各一,龍君、虎君、黃熊、赤豹旗同。戟六、鉞六、戈四、矛四、白柯槍二十。第三隊,軍將、卒長各一,吏兵、力士、熊虎、龍君、虎君、天王旗並同,十二辰旗十二,通直官二,白柯槍十,平列旗十二。



 龍墀旗隊。天下太平旗一,排仗大將二人夾之;五方龍旗各一,為三重。



 赤在前,黃在中,黑在後,青左、白右。次金鸞旗一,左,金鳳旗一,右;獅子旗二;君王萬歲旗一;日旗一,左,月旗一,右。御馬十匹,分左右,為五重。中道隊。左右衛大將軍一人檢校,騎。日月合璧旗一,慶雲旗二,五星連珠旗一,祥光旗、長壽幢各一。



 金吾牙門第一門。牙門旗四,次監門使臣六,分左右,騎。孝宗省旗為二,監門為三。



 金吾細仗。青龍旗一,左,白虎旗一,右;五嶽神旗五,分前、中、後、左、右,為三列;五方神旗五,陳列亦如之。五方龍旗二十五,相間為五隊,每隊赤前、黃中、黑后、青左、白右。五方鳳旗二十五,相間為五隊,陳列亦如之。五岳旗在左,五方旗在右;五龍旗在左,五鳳旗在右;四瀆旗,江、淮在左,河、濟在右;押二人,分左右,騎。



 孝宗五龍、五鳳旗止各一隊,共省四十旗,餘同。



 八寶輿。鎮國神寶左,受命之寶右;皇帝之
 寶左,天子之寶右;皇帝信寶左,天子信寶右;皇帝行寶左,天子行寶右,為四列。每寶一輿,每輿一香案,輿、案前燭罩三十二。引寶職掌八人,侍寶官一人,內外符寶郎各二人,扈衛一百人。碧襴二十人,夾扈衛之外。



 孝宗省碧襴為十二,餘同。



 殿中傘扇、輿輦。方傘二,孝宗省一。



 朱團扇四,孝宗省二。



 金吾四色官六人,孝宗省為二。



 押仗二人,騎,金甲二人,執鉞,進馬官四人,騎,千牛衛大將軍一人,孝宗省之。



 千牛衛將軍八人,孝宗省為二。



 金吾引駕官二人,導駕官四人,並騎導。大傘二,
 孝宗省一。



 鳳扇四,孝宗省二。



 夾傘而行。前同。



 腰輿一,鳳扇十六,夾輿。孝宗省為四。



 華蓋二,排列官一人,香鐙一,火燎一,小輿一,逍遙子,平輦。



 駕前諸班直。駕頭、鳴鞭、誕馬、燭罩三百三十人。孝宗省為二百一十人。前驅都下親從官一百五十人,孝宗省為四十五人。



 東西班六人,孝宗省為二十二人約攔。



 殿前指揮使四十人,東第三班長入祗候五十二人,班直主首九人,孝宗省為三人。



 茶酒新舊班一百六人,孝宗省為四十四人。



 開道旗一,纛一十二,鈞容直二百七十人,架回則作樂。孝宗乾道元年省之,乾道六年以
 後再用。



 吉利旗五,五方龍旗五,龍旗二十,孝宗省之。



 門旗六十,孝宗省為三十。



 殿前指揮使、引駕骨朵子直四十人。分左右,夾門旗外。



 駕頭,駕頭下天武官二十二人,孝宗省為十七人。



 都下親從一十六人,孝宗省為八人。



 茶酒班執從物殿侍二十二人,又都下親從二十二人,孝宗省為十七人。



 劍六人,孝宗省為三人。



 麋旗一,人員一,孝宗省之。



 殿前指揮使、行門二十二人,鳴鞭十二人。孝宗增為一十四人。



 次御龍直百二十人,孝宗省為八十六人。



 快行五十人,日、月、麟、鳳旗各一,青龍、白龍、赤龍、黑龍旗四,人員二,引駕
 千牛上將軍一人。



 玉輅奉宸隊。分左右,充禁衛,圍子八重:崇政殿親從圍子二百人,為第一重;從里數出。



 御龍直二百五十人,為第二重;崇政殿親從外圍子二百五十人,為第三重;御龍直、骨朵子直二百五十人,為第四重;御龍弓箭直二百五十人,為第五重;御龍弩直二百五十人,為第六重;禁衛天武二百五十人,為第七重;都下親從圍子三百人,為第八重。



 孝宗以上並同。



 天武約攔二百人,孝宗省作百八十八人。



 在禁衛圍子外,編排禁衛行子二十一人,
 快行五十九人,孝宗省為四十二。



 管押相視御龍四直八人,孝宗省為四人。



 照管行子御龍四直二十四人,孝宗省為八人。



 天武六人,孝宗省之。



 禁衛內攔前崇政殿親從三十二人。孝宗省作二十五人。



 駕後部。扇筤,大黃龍旗一。駕後樂:東西班三十六人,鈞容直三十一人,並騎。孝宗此下增招箭班三十四人。



 扇筤,扇筤下天武二十二人,孝宗省作一十七人。



 都下親從十六人,孝宗省作八人。



 茶酒班執從物五十人,騎。孝宗省為三十人。



 大輦。輦下應奉並人員合六百一十四人,分五番;孝宗乾道元年省之,六年以後復設。



 御馬十
 疋,為五重。



 持鈒後隊。神勇都指揮使二人,騎,重輪旗二人,大傘二,孝宗省為一。朱團扇八,孝宗省為四。



 鳳扇二,小雉扇二十二,孝宗省鳳扇,而減雉扇為六。



 華蓋二,孝宗省為一。



 俾倪十二,孝宗省為六。



 御刀六,玄武幢一,絳麾二,叉、細槊十二,孝宗省為六。



 驍騎都指揮使一人,騎,總領大角。



 大角四十。孝宗省為二十。



 太常後部鼓吹。鼓吹丞二人,典吏四人,孝宗省為三人。



 管轄指揮使一人,羽葆鼓六,歌工二十四,拱宸管十二,簫三十六,笳二十四,鐃鼓六,小橫吹六十,笛十二,觱慄十二,帥兵官十人。



 孝宗歌工十八,拱宸管十二,簫十八,笳二十四,鐃鼓六,笛十八,節鼓一,小橫吹三十,觱慄十八,桃皮觱慄十二,羽葆鼓吹六,帥兵官八人。



 黃麾幡一,中道。



 金輅、象輅、革輅、木輅各一,每輅誕馬各六在輅前,駕士各百五十四人。乾道元年省之,六年以後復用。



 掩後隊。中道。



 宣武都指揮使二人,大戟、刀盾、弓矢、槊各十五。



 金吾牙門第二門。中道。



 牙門旗四,分左右,孝宗省之。



 監門使臣六,分左右,騎。孝宗省為三。



 玄武隊。並騎。中道。



 虎翼都指揮使一人,犦槊二,玄武旗一,槊、弓矢各十,孝宗並省為五。



 弩五。外仗。分左右道,以夾中道儀仗。



 清游隊。並騎。



 白澤旗二,捧日指揮使
 二,弩四,弓矢十,槊十六。左、右金吾十六,騎。天武都頭二人,弩八,弓矢十二,槊十二。孝宗弩、弓矢、槊並省為六。



 佽飛隊。並騎。



 拱聖指揮使二,虞候佽飛二十,鐵甲佽飛十二。前隊殳仗。都頭六人,騎,殳、叉六十。後隊殳仗。都頭四人,騎,殳、叉四十。



 前部馬隊。第一隊,捧日都指揮使二人,角、斗、亢、牛旗各一,弩四,弓矢十,槊八;第二隊,捧日都指揮使二人,氐、女、房、虛旗各一,弩、弓矢、槊如第一隊;第三隊,天武都指揮使二人,心、危旗各一,弩、弓矢、槊如第二隊;第
 四隊,天武都指揮使二人,尾、室旗各一,弩、弓矢、槊如第三隊;第五隊,拱聖指揮使二人,箕、壁旗各一,弩、弓矢、槊如第四隊;第六隊,拱聖都指揮使二人,奎、井旗各一,弩、弓矢、槊如第五隊;第七隊,神勇都指揮使二人,婁、鬼旗各一,弩、弓矢、槊如第六隊;第八隊,神勇都指揮使二人,胃、柳旗各一,弩、弓矢、槊如第七隊;第九隊,驍騎都指揮使二人,昴、星旗各一,弩、弓矢、槊如第八隊;第十隊,宣武都指揮使二人,畢、張旗各一,弩、弓矢、槊如第九隊;第十一
 隊,虎翼都指揮使二人,觜、翼旗各一,弩、弓矢、槊如第十隊;第十二隊,廣勇都指揮使二人,參、軫旗各一,弩、弓矢、槊如第十一隊。



 孝宗省為七隊,二十八宿旗每隊四,弓矢、槊每隊六,餘同。



 步甲前隊。第一隊,捧日指揮使、都頭各二人,騎,下同。



 鶡雞旗二,青鍪甲、刀盾二十;孝宗刀盾省為十二,下並同。



 第二隊,捧日指揮使、都頭,貔旗,朱鍪甲、刀盾;第三隊,天武指揮使、都頭,萬年連理木旗,黃鍪甲、刀盾;第四隊,天武指揮使、都頭,芝禾並秀旗,白鍪甲、刀盾;第五隊,拱聖指揮使、都頭,祥鶴旗,黑
 鍪甲、刀盾;第六隊,拱聖指揮使、都頭,犀旗,黃鍪甲、刀盾。



 孝宗改黃鍪甲為青鍪甲,餘並同。



 金吾左右道牙門第一門。牙門旗四,分左右。監門使臣八人,並騎。孝宗旗省為二,使臣省為四人。



 步甲前隊第七隊,神武指揮使、都頭,鶡雞旗,青鍪甲、刀盾;第八隊,神武指揮使、都頭,麟旗,朱鍪甲、刀盾;第九隊,驍騎指揮使、都頭,白狼旗,黃鍪甲、刀盾;第十隊,驍騎指揮使、都頭,蒼烏旗,次白鍪甲、刀盾;第十一隊,虎翼指揮使、都頭,鸚鵡旗,黑鍪甲、刀盾;第十二隊,廣勇指揮使、都頭,太平旗,
 黃鍪甲、刀盾。自二至十二隊,人、旗、刀盾,數列如第一隊。



 孝宗內去鶡雞旗、麟旗而用慶雲旗、瑞麥旗。



 金吾左右道牙門第二門。牙門旗四,分左右,監門使臣八人,並騎。孝宗旗省為二,監門省為四人。



 前部黃麾仗。第一部,殿中侍御史二員,騎,下同。



 絳引幡二十,孝宗省為十。



 犦槊二,捧日指揮使二,都頭五,並騎,下同。



 黃氅五十,孝宗省為二十。



 鼓四,斧十,戟、弓矢二十,槊三十,孝宗省為二十。



 弩十;第二部,殿中侍御史,天武指揮使、都頭,青氅,鼓,斧,戟、弓矢,槊,弩;第三部,殿中御史,拱聖指揮使、都頭,緋氅,鼓,斧,戟、
 弓矢,槊,弩;孝宗省作三部。



 第四部,殿中御史,神勇指揮使、都頭,黃氅,鼓,斧,戟、弓矢,槊,弩;第五部,殿中御史,驍騎指揮使、都頭,白氅,鼓,斧,戟,弓矢,槊,弩;第六部,殿中御史,廣勇指揮使、都頭,黑氅,鼓,斧,戟、弓矢,槊,弩。自二至六部,數列並如初部。



 青龍白虎隊。並騎。



 青龍旗一,白虎旗一,虎翼都指揮使二,弩四,弓矢十,槊八。



 班劍、儀刀隊。並騎。



 武衛將軍二人,捧日、天武、拱聖、神勇指揮使各二人,班劍六十,儀刀六十。次驍騎、驍勝、宣武、虎翼指揮使各二人,班劍
 六十,儀刀六十。



 親勛、散手、驍衛翊衛隊。並騎。中衛郎四人,翊衛郎二人,親衛郎二人,衛兵四十,甲騎四十在衛兵外。左右驍衛、翊衛三隊。並騎。第一隊,左右驍衛大將軍二人,雙蓮花旗二,弩四,弓矢十,孝宗減弓矢為六,下同。



 槊十六;孝宗減槊為八,下同。



 第二隊,廣勇指揮使二人,吉利旗,弩、弓矢、槊數如初隊。



 金吾左右道牙門第三門。牙門旗四,分左右,監門八人,並騎。孝宗旗減為二,監門減為四人。



 捧日隊三十四隊。左右各十七隊,孝宗減為十隊,左右各五隊。



 每隊引一人,押一人,旗
 三人,槍五人,弓箭二十人。



 後部黃麾仗。凡六部,第一部至六部,並同前部黃麾仗,惟無絳引幡、犦槊。孝宗減為三部,仗數亦同前部黃麾已減之數,並去犦槊、絳引幡。



 絳引幡二十。孝宗減為十。



 金吾左右道牙門第四門。牙門旗四,監門八人,騎。孝宗旗減為二,監門減為四人。



 步甲後隊。第一隊,捧日指揮使、都頭各二人,騎,鶡旗、鶡雞旗各二,青鍪甲、刀盾二十;孝宗減刀盾為十六,逐隊並同。



 第二隊,天武指揮使、都頭,芝禾並秀旗、萬年連理木旗,朱鍪甲、刀盾;第三隊,拱聖指揮使、都頭,犀旗、鶴旗,黃鍪
 甲、刀盾;第四隊,神武指揮使、都頭,蒼烏旗、白狼旗,白鍪甲、刀盾;第五隊,驍騎指揮使、都頭,天下太平旗、鸚鵡旗,黑鍪甲、刀盾;第六隊,虎翼指揮使、都頭,鶡雞旗、昆旗,黃鍪甲、刀盾。自二至六隊,數列並如初隊。



 金吾左右道牙門第五門。牙門旗四,監門八人,騎。孝宗減旗為二,減監門為四。



 後部馬隊。第一隊,捧日都指揮使二,角端旗二,弩四,弓矢十,槊十六;孝宗弓矢減為六,槊減為八。



 第二隊,捧日都指揮使,孝宗更用天武。



 赤熊旗,弩、弓矢、槊;第三隊,天武都指揮使,孝宗更用拱聖。



 兕
 旗,弩、弓矢、槊;第四隊,天武指揮使,孝宗時更神勇。



 天下太平旗,弩、弓矢、槊;第五隊,拱聖都指揮使,犀旗,孝宗用龍馬旗。



 弩、弓矢、槊;第六隊,拱聖都指揮使,芝禾並秀旗,孝宗用金牛旗。



 弩、弓矢、槊;第七隊,神勇都指揮使,萬年連理旗,弩、弓矢、槊;第八隊,神勇都指揮使,騶牙旗,弩、弓矢、槊;第九隊,驍騎都指揮使,蒼烏旗,弩、弓矢、槊;第十隊,宣武都指揮使,白狼旗,弩、弓矢、槊;第十一隊,虎翼都指揮使,龍馬旗,弩、弓矢、槊;第十二隊,廣勇都指揮使,金牛旗,弩、弓矢、槊。自二至十
 二隊,數列並如初隊。



 皇太后、皇后鹵簿,皆如禮令。徽宗政和元年,詔皇后受冊排黃麾仗及重翟車,陳小駕鹵簿。後謙避,於是詔延福宮受冊仍舊;而小駕鹵簿、端禮門外黃麾仗、紫宸殿臣僚稱賀上禮,並罷。其景靈宮朝謁,則依近例。三年,議禮局上皇后鹵簿之制。



 清游隊。旗一。執一人,引二人,夾二人,並騎。



 金吾衛折沖都尉一員,騎,執犦槊二人夾。



 領四十騎,執槊二十人,弩四人,橫刀一十六人。次虞候佽飛二十八,騎。次內僕、
 內僕丞各一員。各書令史二人,並騎。



 次正道黃麾一。執一人,夾二人,並騎。



 次左右廂黃麾仗,廂各三行,行一百人:第一行,短戟、五色氅;第二行,戈、五色氅;第三行,儀鍠、五色幡。



 左右領軍衛、左右威衛、左右武衛、左右驍衛、左右衛等各三行,行二十人,各帥兵官六人領,內左右領軍衛帥兵官各三人,各果毅都尉一員檢校,各一人步從。



 左右領軍衛絳引旗,引前、掩後各六。



 次內謁者監四人,給事、內常侍、內侍各二人,並騎。內給使各一人,步從。



 次內給使一百二十人。次偏扇、團扇、
 方扇各二十四。次香鐙一。次執擎內給使四人。在重翟車前。



 次重翟車。駕青馬六,駕士二十四人,行障六、坐障三,夾車,並宮人執。次內寺伯二人,騎,領寺人六人,分左右夾重翟車。



 次腰輿一,輿士八人。



 團雉尾扇二,夾輿。次大傘四,大雉尾扇八,錦花蓋二,小雉尾扇、朱畫團扇各十二,錦曲蓋二十,錦六柱八扇。自腰輿以下,並內給使執。



 次宮人車。次絳麾二。各一人執。



 次正道後黃麾一。執一人,夾二人,並騎。



 次供奉宮人。次厭翟車駕赤騮,翟車駕黃騮,安車駕赤騮,各四,駕士各二十
 四人。四望車、金根車、各駕牛三,駕士各一十二人。



 次左右廂各置牙門二。每門執二人,夾四人,一在前黃麾前,一在後黃麾後。



 次左右領軍衛,每廂各一百五十人執殳,帥兵官四人檢校。次左右領軍衛折沖都尉各一員,檢校殳仗。各一人騎從。



 次後殳仗。內正道置牙門一。每門監門校尉二人,騎;每廂各巡檢校尉一員,騎,來往檢校。



 前後部鼓吹。金鉦、□鼓、大鼓、長鳴、中鳴、鐃吹、羽葆、鼓吹、節鼓、御馬,並減大駕之半。



 皇太子鹵簿。禮令,三師、詹事、率更令、家令各用本品鹵
 簿前導。太宗至道中,真宗升儲,事多謙抑,謁廟日止用東宮鹵簿,六引官,但乘車而不設儀仗。天禧二年,仁宗為皇太子,亦依此制。政和三年,議禮局上皇太子鹵簿之制。



 家令、率更令、詹事各乘輅車,太保、太傅、太師乘輅,各正道,威儀、鹵簿依本品。次清游隊旗,執一人,引二人,夾二人。



 並正道。清道率府折沖都尉一員,領二十騎,執槊一十八人,弓矢九人,弩三人,二人騎從折沖。次左、右清道率府率各一員,領清道直蕩及檢校清游隊龍旗等,執犦槊
 各二人。次外清道直蕩二十四人,騎。



 次正道龍旗各六,執一人,前二人引,後二人護。



 副竿二。執各一人,騎。



 次正道細仗引。為六重,每重二人,自龍旗後均布至細仗,槊與弓箭相間,並騎;每廂各果毅都尉一員領。次率更丞一員。



 次正道前部鼓吹。府史二人領鼓吹,並騎。□鼓、金鉦各二,執各一人,夾二人,以下準此。



 帥兵官二人;次大鼓三十六,橫行,長鳴以下準此。



 帥兵官八人;長鳴三十六,帥兵官二人;鐃吹一部,鐃鼓二,各執一人,夾二人,後部鐃節鼓準此。



 簫、笳各六,帥兵官二人;□鼓、金鉦各二,帥
 兵官二人;次小鼓三十六,帥兵官四人;中鳴三十六,帥兵官二人。以上並騎。



 次誕馬十,每匹二人控,餘準此。



 廄牧令、丞各一員。各府史二人騎從。



 次左、右翊府郎將各一員,領班劍,左右翊衛執班劍二十四人,通事舍人四人,司直二人,文學四人,洗馬、司議郎、太子舍人、中允、中舍、左右諭德各二人,左、右庶子四人,並騎。



 自通事舍人以後,各步從一人。



 次左、右衛率府副率各一員,步從,親、勛、翊衛每廂各中郎將、郎將一員,並領六行儀刀:第一行,親衛二十三人,曲折三人;第二
 行,親衛二十五人,曲折四人;第三行,勛衛二十七人,曲折五人;第四行,勛衛二十九人,曲折六人;第五行,翊衛三十一人,曲折七人;第六行,翊衛三十三人,曲折八人。



 曲折人並部後門。



 以上三衛並騎。



 次三衛一十八人,騎;中郎將二人夾輅,在六行儀刀仗內。金輅,駕馬四,僕寺僕馭,左右率府率一員,駕士二十二人。夾輅左、右衛率府率各一員。



 各步從一人。



 次左、右內率府率各一員,副率各一員,並騎。各步從一人。



 次千牛騎,執細刀、弓矢,三衛儀刀仗,後開牙門。
 次左右監門率府直長各六人,監後門。並騎。



 次左右衛率府每廂各翊衛二隊。並騎。



 次厭角隊各三十人,執旗一人。引二人,夾二人。



 執槊一十五人,弓矢七人,弩三人,每隊各郎將一員領。



 次正道傘二,雉尾扇四,夾傘。次腰輿一,輿士八人,團雉尾扇二、小方雉尾扇八夾。執各一人。



 次內直郎、令史各二人騎從檢校。次誕馬十,典乘二人,府史二人騎從。



 次左右司禦率府校尉各一人,並騎從。



 領團扇、曲蓋。次朱團扇、紫曲蓋各六。執各一人。



 次諸司供奉官人。



 次左右清道
 率府校尉各一人,並騎。



 領大角三十六。鐃鼓二,簫、笳各六,帥兵官二人;橫吹十,節鼓一,笛、簫、觱慄五,帥兵官二人。並騎。



 次管轄指揮使二人檢校。



 次副輅,駕四馬,駕士二十人。軺車,駕一馬,駕士十四人。四望車,駕一馬,駕士一十人。



 次左右廂步隊凡十六,每隊各果毅都尉一人領,並騎。



 隊三十人,執旗一人,引二人,夾二人,並帶弓矢,騎。



 步二十五人。前一隊執槊,一隊帶弓矢,以次相間。左右司禦率府、左右衛率府廂各四隊,二在前,二在後。



 次左右司禦率府副率各一員檢校,步隊各二人,
 執犦槊騎從。



 次儀仗。左右廂各六色,色九行,行六人。前第一行,戟、赤氅;第二行,弓矢;第三行,儀鋋並毦;第四行,刀盾;第五行,儀鍠、五色幡;第六行,油戟。次前仗首左右廂各六色,色三行,行六人。左右司禦率府各一員,果毅都尉各一員,帥兵官各六人領。次左右廂各六色,色三行,行六人。左、右衛率府副率各一員,果毅都尉各一員,帥兵官各六人領。次盡後鹵簿左右廂各六色,色三行,行六人,左右司禦率府副率各一員,各一人步從。果毅都尉
 各一人,帥兵官各六人領,左右司禦率府率兵官各六人護後,並騎。每廂各絳引幡十二,執各一人,引前旗六,引後旗六。



 揭鼓十二。揭鼓左右司禦率府四重,左右衛率府二重。



 次左右廂殳。各一百五十人,左右司禦率府各八十六人,左右衛率府各六十四人。



 並分前後,在步隊儀仗外、馬隊內,前接六旗,後盡鹵簿,曲折至門,每廂各司禦率府果毅都尉一員檢校,各一人從,每廂各帥兵官七人。



 並騎,左右司禦率府各四人,左右率府各三人。



 次馬隊。左右廂各十隊,每隊帥兵官以下三十一人,旗一,執一人,引二人,夾二人。



 執槊十六人,
 弓矢七人,弩三人。前第一隊,左右清道率府果毅都尉各一員領;第二、第三、第四隊,左右司禦率府果毅都尉各一員領;第五、第六、第七隊,左右衛率府果毅都尉各一員領;第八、第九、第十隊,左右司禦率府果毅都尉各一員領。次後拒隊。



 旗一,執一人,引二人,夾二人。



 清道率府果毅都尉一員領四十騎,執槊二十人,弓矢十六人,弩四人。叉二人,騎從。



 次後拒隊前當正道殳仗行內開牙門。次左右廂各開牙門三:前第一門,左右司禦率府步隊後,左
 右率府步隊前;第二門,左右衛率府步隊後,司禦率府儀仗前;第三門,左右司禦率府儀仗後,左右衛率府步隊前。



 每開牙門,執旗二人,夾四人,並騎。



 監門率府直長各二人,並騎;次左右監門率府副率各一員,騎;來往檢校諸門,各一人騎從。次左右清道率府副率各三人,仗內檢校並糾察,各一人騎從。次少師、少傅、少保,正道乘輅,威儀、鹵簿各依本品,次文武官以次陪從。



 皇太子妃鹵簿之制。政和三年,議禮局上。清道率府校
 尉六人,騎。次青衣十人。次導客舍人四人,內給使六十人,偏扇、團扇、方扇各十八,並宮人執。



 行障四,坐障二,夾車,宮人執。



 典內二人,騎,厭翟車,駕三馬,駕士十四人。次閣帥二人,領內給使十八人,夾車,六柱二扇,內給使執。次供奉內人,乘犢車。次傘一,正道。雉尾扇二,團扇四,曲蓋二。



 執傘、扇各內給使一人。次戟九十。



 宋制,臣子無鹵簿名,遇升儲則草具儀注。《政和禮》雖創具鹵簿,然未及行也。南渡後,雖嘗討論,然皇太子皆水中挹不受,朝謁宮廟及陪祀及常朝,
 皆乘馬,止以宮僚導從,有傘、扇而無圍子。用三接青羅傘一,紫羅障扇四人從,指使二人,直省官二人,客司四人,親事官二十人,輦官二十人,翰林司四人,儀鸞司四人,廚子六人,教駿四人,背印二人,步軍司宣效一十人,步司兵級七十八人,防警兵士四人。朝位在三公上,扈從在駕後方圍子內。



 皇太子妃,政和亦有鹵簿,南渡後亦省之。妃出入惟乘簷子,三接青羅傘一,黃紅羅障扇四人從。以皇太子府
 親事官充輦官,前執從物,簷子前小殿侍一人,抱塗金香球。先驅,則教駿兵士呵止。



 王公以下鹵簿。凡大駕六引,用本品鹵簿,奉冊、充使及詔葬皆給之。親王用一品之制,加告止幡、傳教幡、信幡各二,其葬日,用六引內儀仗。真宗咸平二年,王承衍出葬日,在禁樂,禮官請鹵簿鼓吹備而不作,從之。景德二年,南郊鹵簿使王欽若言:「鄆王攢日所給鹵簿,與南郊儀仗吉兇相參。望依令別制王公車輅,所有鼓吹、儀仗,
 亦請增置,以備拜官、朝會、婚葬之用。」從之。於是儀服悉以畫,其葬日在途,以革車代輅。



 徽宗政和三年,議禮局上王公鹵簿之制:中道清道六人。次幰弩一騎。次大晟府前部鼓吹。令及職掌、局長、院官各一人,□鼓、金鉦各一,大鼓、長鳴各一十八,□鼓、金鉦各一。次引樂官二人,小鼓、中鳴各一十。次麾、幢各一,節一,夾槊二,誕馬八,每匹,控馬各二人。



 革車一乘,駕赤馬四,駕士二十五人,散扇十,方傘二,朱團扇四夾方傘,曲蓋二。次大角八。次後部鼓
 吹,丞一員,錄事一人。次鐃鼓一,簫四,笳四,大橫吹六,節鼓一,夾色二,笛、簫、觱慄、笳各四。次外仗。青衣十二,車輻棒十二,戟九十,絳引幡六,刀盾、槊、弓矢各八十,儀刀十八,信幡八,告止幡、傳教幡各四,儀鋋二,儀鍠斧掛五色幡六,油戟十八,儀槊十二,細槊十二。次左右衛尉寺押當職掌一十一人,騎;部轄步兵、部轄騎兵、太僕寺部押人員各一人,教馬官一人。押當職掌四人,騎。



 公主鹵簿。惟葬日給之。秦國成聖繼明夫人葬日,亦給
 外命婦一品鹵簿,自餘未嘗用。



 一品鹵簿。命婦同。



 中道清道四人。幰弩一,騎。大晟府前部鼓吹。令一,職掌一人,局長、院官各一人。□鼓、金鉦各一,大鼓、長鳴各一十六,麾、幢、節各一,槊二,誕馬六。次革車一乘,駕赤馬四,駕士二十五人。



 命婦厭翟車,駕士二十三人,二品、三品準此。



 散扇八,二品減四,三品減六,命婦散扇五十,行障五,行於車前,二品、三品準此。



 方傘二,朱團扇四,曲蓋二,大角八。命婦屬車六,駕黃牛十八,駕士五十九人,行大角前,二品、三品準此。



 次後部鼓吹。丞一員,錄事一人,引樂官二員。鐃鼓一,簫、
 笳、大橫吹各四,節鼓一,笛、簫、觱慄、笳各四。外仗。青衣十人,車輻棒十,戟九十,刀盾、槊各八十,弓矢六十,儀刀三十,信幡八,告止幡、傳教幡、儀鍠斧掛五色幡各四。次衛尉寺排列、押當職掌一十一人,部轄人員、太僕寺部押人員、教馬官各一人。押當職掌四人。



 命婦加二人。



 二品鹵簿。命婦同。



 中道清道二人。幰弩一。大晟府前部鼓吹。令一,及職掌、局長、院官各一人。□鼓、金鉦各一,大鼓十四,麾、幢、節各一,夾槊二,誕馬四。次革車一乘,駕赤馬
 四,駕士二十五人。散扇四,方傘、朱團扇、曲蓋各二。次大角八。次後部鼓吹。丞一,錄事、引樂官各一人。鐃鼓一,簫、笳各二,大橫吹四,笛、簫、觱慄、笳各二。外仗。青衣八人,車輻棒八,戟七十,刀盾、槊、弓矢各六十,儀刀十四,信幡四,告止、傳教幡各二。次衛尉排列、押當職掌九人,部轄人員、太僕寺部押人員、教馬官各一人。押當職掌四人。



 命婦加二人。



 三品鹵簿。命婦同。



 中道清道二。幰弩
 一。麾、幢各一,節一,夾槊二,誕馬四。次革車一乘,駕赤馬四,駕士二十五人。散扇二,方傘二,曲蓋一,大角四。外仗。青衣八人,車輻棒六,戟六十,刀盾、槊、弓矢各五十,儀刀十二,信幡四,告止、傳教幡各二。次衛尉排列、押當職掌七人,部轄人員、太僕寺部押人員、教馬官各一人。押當職掌四人。



 命婦加二人。



 以上皆政和所定也。



\end{pinyinscope}