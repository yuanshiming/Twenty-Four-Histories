\article{志第一百四十六 兵七(召募之制)}

\begin{pinyinscope}

 召募之制起於府衛之廢。唐末士卒疲於征役,多亡命者,梁祖令諸軍悉黵面為字,以識軍號,是為長征之兵。方其募時,先度人材,次閱走躍,試瞻視,然後黵面,賜以
 緡錢、衣履而隸諸籍。國初因之,或募土人就所在團立,或取營伍子弟聽從本軍,或募饑民以補本城,或以有罪配隸給役。取之雖非一途,而伉健者遷禁衛,短弱者為廂軍,制以隊伍,束以法令。當其無事時,雖不無爵賞衣廩之費,一有征討,則以之力戰鬥,給漕挽,而天下獷悍失職之徒,皆為良民之衛矣。



 初,太祖揀軍中強勇者號兵樣,分送諸道,令如樣招募。後更為木梃,差以尺寸高下,謂之等長杖,委長吏、都監度人材取之。當部送闕
 者,軍頭司覆驗,引對便坐,分隸諸軍。



 真宗祥符中,復位等杖,自五尺八寸至五尺五寸為五等,諸州部送闕下,及等者隸次軍。



 仁宗天聖元年,詔京東西、河北、河東、淮南、陜西路募兵,當部送者刺「指揮」二字,家屬給口糧。兵官代還,以所募多寡為賞罰。又詔益、利、梓、夔路歲募民充軍士,及數即部送,分隸奉節、川效忠、川忠節。於是遠方健勇失業之民,悉有所歸。



 慶歷七年,諸路募廂軍及五尺七寸已上者,部送闕下,試補禁衛。



 至和元年,河北、
 河東、陜西募就糧兵,騎以四百人、步以五百人為一營。



 嘉祐二年復定等仗,自上四軍至武肅、忠靖皆五尺已上,差以寸分而視其奉錢:一千者以五尺八寸、七寸、三寸為三等。奉錢七百者,以五尺七寸、六寸、五寸為三等。奉錢五百者,以五尺六寸、五寸五分為三等。奉錢四百者,以五尺五寸、四寸五分為二等。奉錢三百者,以五尺五寸、四寸五分、四寸、三寸、二寸為六等。奉錢二百者,以五尺四寸、三寸五分、三寸、二寸為四等。不給奉錢者,以
 五尺二寸或下五寸七指、八指為等。唯武嚴、御營喝探以藝精者充,諸司筦庫執技者不設等杖。



 七年,御史唐介言:「比歲等募禁軍多小弱,不勝鎧甲,請以初創尺寸為定,敢議減縮者,論以違制。」詔:「禁軍備戰者,宜著此令。其備役雄武、宣敕六軍、搭材之類,如軍馬敕。」



 治平二年,募陜西土民、營伍子弟隸禁軍,一營填止八分。又遣使畿縣、南京、曹、濮、單、陳、許、蔡、亳州募民補虎翼、廣勇,人加賜絹、布各一。



 治平四年,詔延州募保捷五營,以備更戍。



 熙寧元年,詔諸州募饑民補廂軍。



 二年,樞密院言:「國初邊州無警則罷兵,今既講和,而屯兵至多,徒耗金帛。若於近裏糧賤處增募營兵,但令往戍極邊,甚為便計。」帝與文彥博及韓絳、陳升之、呂公弼等議之,或以為自古皆募營兵,遇事息即罷,或以為緣邊之兵不可多減。乃命彥博等詳議以聞。



 三年七月,詔京西路於有糧草州軍招廂軍,共三萬人為額。十一月,知定州滕甫乞下本路依舊制募弓箭社,以為邊備。從之。



 四年十二月,樞密
 院言:「在京系役兵士,舊額一萬八千二百五十九人,見闕六千三百九十二人,若招揀得足,即不須外路勾抽,以免不習水土、凍餒道斃之患。欲於在京及府界、京東西、河北招少壯兵,止供在京功役,不許臣僚占差,不過期年,可使充足。卻對減在外招募之數,樁管所減糧賜上供,以給有司之用。」從之。



 五年,權發遣延州趙離招到漢蕃弓箭手人騎四千九百八十四,為八指揮,遂擢吏部員外郎,加賜銀絹二百。



 七年,分遣使臣諸路選募熙
 河效用,先以名聞。河北、河東所募兵悉罷。



 八年,詔軍士祖父母、父母老疾無侍丁而應募在他處者,聽徙。



 九年,詔選補捧日、天武以下諸軍闕,馬軍三分補一,步軍十分補五。



 元豐二年二月,經制熙河路邊防財用司言:「岷州床川、荔川、閭川砦,通遠軍熟羊砦,乞置牧養十監,募兵為監牧指揮。其營田乞依官莊例,募永濟卒二百人,其永濟卒通以千人為額。」從之。七月,沿邊安撫司言:「北邊州軍主管刺事人乞給錢三千,選募使臣職員或百
 姓為之,以鉤致敵情。仍選通判及監官考其虛實,以行賞罰。」從之。是年,以兗、鄆、齊、濟、濱、棣、德、博民饑,募為兵,以補開封府界、京東西將兵之闕。



 三年,又詔:「府界諸路將下闕禁軍萬數,有司其速募之。」又詔:「河北水災,闕食民甚眾,宜寄招補軍。」



 四年,京東、西路以調發兵將,累請增戍。朝廷以兵員有數,多寢其章。然州郡實有負山帶海,奸盜所窺,亦當過為之慮,其令益廣應募者,與免貼軍及他役一年。六月,詔:「在京奉錢七百以下,選募馬步軍
 萬五千人;開封府界及本路共選募義兵保甲萬人;如涇原五千人不足,於秦鳳路選募。」



 五年五月,同提舉成都府等路茶場蒲宗閔乞自秦州至熙州量地里遠近險易,置車鋪二十八,招刺兵士。從之。八月,詔開封府界、京西招軍依式賜外,仍增錢千。十二月,詔京城四面巡檢募士於四門,取民年三十五以下者。又詔河北立額步軍,各於逐指揮額外招百人。



 五年,詔一歲內能募及百人者,加秩一等。四月,河東路經略司請以麟州飛騎、
 府州威遠子弟二十五以下剌為兵。



 七年,廣西都鈐轄司言:「本路土兵闕額數多,乞選使臣往福建、江南、廣東招簡投換兵四千人。」詔於江南、福建路委官招換。



 八年四月,河東路安撫使呂惠卿言:「河東敢勇以三百人為額,請給微薄,應募者少。臣頃在鄜延路日,奏請增三等請給,借支省馬給七分草料,置營教習,自後應募者眾。願依陜西路已得指揮。」從之。



 哲宗元祐元年三月,詔河北保甲願投軍人及得上四軍等杖事藝者,特許招填,
 合給例物外,更增錢五千,中軍以下三千。比等杖短一指,射保甲第一等弓弩,並許招刺。從右司諫蘇轍請也。六月,門下侍郎司馬光言:「諸州軍兵馬全欠,不足守禦之處,量與立額招添。」



 八年,樞密院言:「今新招兵士多是饑民,未諳教閱,乞自今住營州軍差官訓練,候半年發遣赴軍前。」



 紹聖元年,樞密院乞立招禁軍官員賞格,如不及數,罰亦隨之。



 四年,熙河蘭岷路都總管、提點熙河蘭岷等路漢蕃弓箭手司言,蘭州金城關欲招置步軍
 保捷四指揮、馬軍蕃落一指揮,從之。詔陜西路添置蕃落軍十指揮,各以五百人為額,於永興軍、河中、鳳翔、同、華州各置兩指揮,並隸住營州軍將下統制訓練,委逐路所屬都總管司選官招人。初,三省、密院欲以牧地募民牧養馬,久而未集,曾布以謂不若增騎兵為簡便。兼土兵乃勁兵,又諸路出戍者已竭,及建此議,眾翕然皆以為允,帝亦樂從之。蓋牧租見存者七百萬,歲額一百七十萬,而十指揮之費二十五萬而已,故可與募人養
 馬之法兼行也。



 徽宗崇寧元年,湖北都鈐轄舒但奉旨相度召募施、黔州土丁,致討辰、沅山猺,每州無過七百人。緣猺賊深在溪洞,險阻不通正軍故也。



 三年,京東等路招軍五萬,馬軍以崇捷、崇銳名,步軍以崇武、崇威名。



 四年七月,熙河蘭湟路轉運使洪中孚自河東入覲,帝問崇威、崇銳新兵教閱就緒否。中孚曰:「教閱易事也。臣不知藝祖取天下之兵與神考所分將兵曾無減損,若未嘗減損,似不須增。蓋兵貴簡練不貴多,今遽增二軍,
 所費至廣,臣不知獻議者於經費之外別有措置,或只仰給朝廷也。」帝愕然曰:「初議增兵,未嘗議費,可即罷去。」中孚曰:「惰游之卒不復安於南畝,今一旦罷遣,強者聚而為盜,弱者轉徙,則重為朝廷憂。不若使填諸營闕;無闕,聽於額外收管,不一二年盡矣。」帝稱善。九月,詔:「近降指揮,在京、諸路招崇捷、崇武等指揮十萬人,又招效忠、蕃落指揮及額內不足人數,慮卒難敷額,可先招崇捷、崇武十萬人。候人數稍見次第,即具申取旨。」



 五年,詔:「抑
 勒諸色人投軍者,並許自身及親屬越訴,其已剌字,仍並改正。」



 政和二年,廣西都鈐司奏:「廣西兩將額一萬三百餘人,事故逃亡,於荊湖南北、江南東西寄招,緣諸路以非本職,多不用心。今兵闕六分,欲乞本路、鄰路有犯徒並杖以下情重之人,除配沙門島、廣南遠惡並犯強盜兇惡、殺人放火、事乾化外並依法外,餘並免決刺填。」從之。



 四年,中衛大夫童師敏言:「東南州郡例闕廂軍,凡有役使,並是和雇。若令諸郡守臣並提刑司措置招填,
 庶可省費。」從之。



 宣和元年,高陽關路安撫使吳玠奉手詔招填諸路禁軍闕額,以十分為率,招及四分以下遞展磨勘年,七分以上遞減磨勘年。高陽關路河間府、滄、霸、恩州、信安軍招填數足,乞行推賞。從之。



 二年,手詔:「比聞諸路州軍招置廂軍河清、壯城等,往往怯懦幼小,不及等樣,虛費廩食,不堪驅使。今後並仰遵著令招填,如違戾,以違制論。」



 四年正月,兩浙東路鈐轄司奏:「乞將溫、處、衢、婺州元管不系將禁軍六指揮,更招置增為十指
 揮,並以五百人為額,凡五千人,庶成全將。及更於臺州招置不系將禁軍一指揮,以四百人為額。」從之。三月,臣僚言:「竊聞道路洶洶相怖,云諸軍捉人刺涅以補闕額,率數人驅一壯夫,且曳且毆,百姓叫呼,或嚙指求免。日者,金明池人大和會,忽遮門大索,但長身少年,牽之而去,云『充軍』。致賣蔬茹者不敢入城,行旅市人下逮奴隸,皆避藏恐懼,事駭見聞。今國家閑暇,必欲招填禁旅,當明示法令,賚以金帛,捐財百萬,則十萬人應募矣。捉人
 於途,實虧國體,流聞四方,傳播遠邇,殊為未便。伏望亟行禁止,以弭疑畏。」時寶菉宮道士張繼滋因往尉氏,亦被刺涅,事聞,手詔提刑司根治。四月,臣僚因言:「招刺闕額禁軍,樞密院立限太遽,諸營弗戢,人用大駭。幸不旋踵德音禁止,群情悅服。其已被刺涅而非願者,頗亦改正,尚有經官求免而未得者。輦轂若此,況其遠乎?竊聞小人假借聲勢,因緣奪攘,所在多有,若或哀鳴得脫,其家已空。今往來猶懷畏避。伏望聖明特賜戒敕,應在外
 招軍去處,毋得橫濫。」從之。



 七年,減掖庭用度,減侍從官以上月廩,罷諸兼局,有司據所得數撥充諸路糴本及募兵賞軍之用。



 欽宗即位,詔守令募州縣鄉村土豪為隊長,各自募其親識鄉里以行。及五十人以上先與進義副尉,三百人以上與承信郎,募文武官習武勇者為統領。行日,所發州軍授以器甲,人給糧半月,地里遠者,所至州縣接續批支。京畿輔郡兵馬制置使司言:「諸路召募敢勇效用,每名先給錢三千,赴本司試驗給據訖,
 支散銀絹激賞。若監司、知通、令佐並應有官人,能召到敢勇效用事藝高強及二百人以上者,乞與轉一官,每加二百人依此。或監司、郡守、州縣官以下應緣軍期事件,稍有稽緩,並依軍法。」從之。



 靖康元年春正月,臣僚言:「諸路見招募人兵,緣逐處漕計闕乏,乞於近州應奉司及延福宮西城錢帛,並許請用,庶得速辦。」從之。又詔:「龍猛、龍騎、歸遠、壯勇諸軍闕額,可行下諸路揀選配填。」又詔:「已降指揮,逐處各以召募效用敢勇武藝人數多寡
 等第推賞。」又詔:「聞希賞之人,抑勒強募。自今並取情願,敢有違戾,當議重罰。毋得將羸弱不堪出戰及已有系軍籍者一例充募。」及詔:「募武舉及第有材武方略,或有戰功、曾經戰陣,及經邊任大小使臣不以罪犯已發未敘,及武學有方略智謀,及曾充弓馬所子弟,及諸色有膽勇敢戰之人,度許赴親征行營司。」又詔:「募陜西土人為兵並使臣、效用等赴姚平仲軍使喚,其應募人修武郎已上二十貫,進義副尉以上十五貫,軍人、百姓十貫,
 並於開封府應管官錢內支。」



 四月,詔:「已降指揮發還歸朝人往大金軍前,如不願往,所在量給口券津遣;元有官守人並不厘務,支奉給之半。其願效力軍前者,許自陳。」



 五月,河北、河東路宣撫司奏:「河北諸州軍所管正兵絕少,又陜西游手惰民願充軍者亦眾,祗緣招刺闕乏例物,是致軍額常闕。今若給一色銀絹,折充例物犒設起發,召募人作義勇,止於右臂上刺字,依禁軍例物支衣糧料錢,陜西五路共可得二萬人,比之淮、浙等路所
 得將兵,實可使喚。」從之,詔遣文武官各一員前去陜西路募兵二萬人赴闕。遂命趙鼎特除開封府曹官,種湘差宣撫司準備將領,並充陜西路乾當公事,專一募兵。是月,遣戶部員外郎陳師尹往福建路募槍杖手。都水使者陳求道言:「朝廷差官往陜西招軍,適當歲豐,恐未易招填。若就委監司招募保甲,啖以例物,與免科差,以作其氣,可得勁兵五萬。」從之。



 六月,樞密都承旨折彥實奏:「西人結連女真,為日甚久,豈無覬覦關中之志?即今
 諸路人馬皆空,萬一敵人長驅,何以枝梧?言之可為寒心,朝廷似未深慮也。河東、河塑之患已形,人故憂之;陜西之患未作,人故忽之。若每路先與十萬緡,令帥臣招募土人為保護之計,責以控扼,不得放令侵入,仍須朝廷應副。漕司乘時廣行儲蓄,以為急務。」



 又開封府尹聶山奏:「招兵者,今日之急務。近緣京畿諸邑例各招刺,至於無人就募,則強捕村民及往來行人為之。遂致里氓奔駭,商旅不行,殊失朝廷愛民之意。檢準政和令,諸盜
 再犯杖以上、情理不可決放而堪充軍者,給例物刺充廂軍。今京城裏外間有盜賊,皆是豪猾,無所畏憚,雖經斷罪,頑惡弗悛,若依上條刺充廂軍,不惟得強壯之用,又且收集奸黠不復為盜。如允所請,則自內及外皆可見之施行。」從之。



 七月,陜西五路制置使錢蓋言:「都水使者陳求道請招刺保甲五萬充軍。緣比來陜右正兵數少,全籍保甲守御,及運糧諸役差使外,所餘無幾,若更招刺五萬充軍,則是正丁占使殆遍,不唯難以選擇,兼
 慮民情驚疑,別致生事。欲乞令州縣曉諭保甲,取其情願;如未有情願之人,即乞令保甲司於正丁餘數內選擇。通赴闕人共成七萬,可以足用。」從之。是月,錢蓋奏:「陜西募土人充軍,多是市井烏合,不堪臨敵。今折彥實支陜西六路銅錢各十萬緡,每名添錢十千,自可精擇少壯及等杖人,可得正軍一萬,六路共得六萬人。」從之。



 十月,樞密院奏:「召募有材武勇銳及膽勇人並射獵射生戶。」從之。又奏:「福建路有忠義武勇立功自效取仕之人,
 理宜召募,除保甲正兵外,弓手、百姓、僧行、有罪軍人並聽應募。如有武藝高強、實有膽勇、眾所推服、願應募為部領人者,依逐項名目權攝部領,各以所募人數借補官資。」從之。



 十一月,京城四壁共十萬人,黃人黃旗滿市。時應募者多庸丐,殊無鬥志。閏十一月,何□用王健募奇兵,雖操瓢行乞之人,亦皆應募,倉卒未就紀律。奇兵亂,毆王健,殺使臣數十人,內前大擾。王宗濋斬渠魁數人,乃定。及出戰,為鐵騎所沖,望風奔潰,殲焉。



 十二月,詔:「
 諸軍詐效蕃裝,焚劫財物,限十日繼贓自首,與免罪。」仍召募潰兵收管。給口食焉。



 逃亡之法,國初以來各有增損。熙寧五年詔,禁軍奉錢至五百而亡滿七日者,斬。舊制,三日者死。初,執政議更法,請滿十日。帝曰:「臨陣而亡,過十日而首,得不長奸乎。」安石曰:「臨陣而亡,法不計日,即入斬刑。今當立在軍興所亡滿三日,論如對寇賊律?」樞密使蔡挺請沿邊而亡滿三日者斬。安石曰:「沿邊有非軍興之所,不可一概坐以重刑。本立重法,以禁避寇
 賊及軍興而已。」帝曰:「然。」文彥博固言:「軍法臣等所當總領,不宜輕改,如前代銷兵乃生變。」安石曰:「前代如杜元穎等銷兵,乃其措置失當,非兵不可銷也。且當蕭俛時,天下兵至多,民力不給,安得不減?方幽州以朱克融等送京師,請毋遣克融還幽州煽眾為亂,而朝廷乃令克融等飄泊京師,久之不調,復遣歸北。克融所以復亂,亦何預銷兵事?」彥博曰:「國初,禁軍逃亡滿一日者斬。仁宗改滿三日,當時議者已慮壞軍法。」安石曰:「仁宗改法以
 來,活人命至多,然於軍人逃亡,比舊不聞加多,仁宗改法不為不善。」帝乃詔增為七日。



 元豐元年,知鄂州王韶言:「乞自今逃亡配軍為盜,聽捕斬,賞錢。」詔坐條札韶照會:「如所犯情重,罪不至死,奏裁。」



 三年六月,詔:「軍士、民兵逃亡隨軍效用,若首獲,並械送所屬,論如法。雖立戰功不賞,仍不許以功贖過。令隨軍榜諭。」



 四年,詔沉括:「奏以軍前士卒逃亡,潰散在路,本非得已,須當急且招安。卿可速具朝旨出榜,雲聞戰士止是不禁饑寒,逃歸其家,
 可各隨所在城砦權送納器甲,請給糧食,聽歸所屬。節次具招撫數以聞。」



 崇寧四年九月,樞密院言:「熙河都總管司舊無兵籍,乞令諸將各置籍,日具有無開收,旬具元額、見管及逃亡事故細目,申總管司,本司揭貼都簿,委機宜一員逐時抽摘點檢。」從之。



 十月,尚書省言:「今所在逃軍聚集,至以千數,小則驚動鄉邑,大則公為劫盜。累降指揮,許以首身,或令投換,終未革絕。昔神宗以將不知兵,兵不知將,故分兵領將。統兵官司,凡兵之事無
 所不統,則其逃亡走死,豈得不任其責?檢會將敕與見行敕令,皆未有將官與人員任責之法,致令來兵將不加存恤,勞役其身,至於逃避,而任職之人悉不加罪。近日熙河一路逃者幾四萬,將副坐視而不禁,人員將校故縱而不問,至逃亡軍人所在皆有。蓋自來立法未詳,兼軍中長行節級人員,將校,什長相統,同營相依,上下相制,豈得致其逃亡漫不省察?況招軍既立賞格,則逃走安可無禁?今參詳修立賞罰十數條。」並從之。



 五年,樞
 密院備童貫所言:「陜西等處差官招諭逃亡軍人,並許所在首身,更不會問,便支口券令歸本營。邊上軍人憚於戍守之勞,往往逃竄於內郡首身,遂得口券歸營,恐相習成風,有害軍政。乞自今應軍人首身,並須會問逃亡赦限,依今來招諭指揮:若系赦後逃亡,即乞依條施行。」從之。



 大觀三年,樞密院備臣僚言云:「自陜西路提點刑獄吳安憲始陳招誘逃亡廂禁軍之法,乃著許令投換改刺之令。自此諸弊浸生,軍律不肅。朝廷洞見其弊,
 已嚴立法,然尚有冒名一節,其弊未除。請如主兵官舊曾占使書札、作匠、雜技、手業之徒,或與統轄軍員素有嫌忌、意欲舍此而就彼,或所部逃亡數多,欲避譴責,輒將逃軍承逃亡之名便與請給。既避譴責,又冒請受,上下相蒙,莫之能革,致使軍士多懷擅去之心者,良以易得擅住之地也。若加重賞,申以嚴刑,庶革斯弊,有裨成法。」從之。



 四年,樞密院言:「諸路及京畿逃亡軍數居多,雖赦敕立限許首,終懷畏避。若諸路專委知州、通判或職
 官一員,京畿委知縣,若招誘累及三百人以上,與減一年磨勘,五百人以上一年半,千人以上取旨推恩,於理為便。」



 政和二年,臣僚言:「祖宗軍政大備,無可議者。比多逃亡者,緣所在推行未至,及主兵司官遵奉未嚴故也。其弊有六:一曰上下率斂,二曰舉放營債,三曰聚集賭博,四曰差使不均,五曰防送過遠,六曰單身無火聚。似此雖具有條禁,而犯者極多。欲乞下有司推究,除兵將官歲終立定賞罰條格外,詔諸路提刑司,每歲終將本
 路州軍不系將禁軍見管及逃亡人數,參互比較,具最多最少處各一州知、通職位姓名,申樞密院。」從之。



 三年十一月,開封少尹陳彥修言:「諸廂收到寒凍赤露共五千七百餘人,其間逃軍數多,合行措置。今欲依押送逃軍格,每二十人各差使臣一員付與系押送人,各踏逐穩便官屋安泊,依居養法關請錢米存養,候晴和,管押前去。所有沿路支破口券,並依本府押送逃軍法,請於合破口券等外,更量支盤纏。」詔:「每人支盤纏錢三百,衲
 襖一領,候二月晴暖即行發遣。」



 四年,尚書省著令:「諸禁軍差發出戍未到軍前,或已到而代去半年以上,逃亡首獲,雖會恩,配如捕獲法;上軍首身或捕獲,會恩,配依七日內法;下軍本名應配者,配千里。若本管輒停留,與同罪,雖該赦仍依配法。」從之。



 五年,立錢監兵匠逃走刺手背法。



 宣和二年,手詔:「逃卒頗多,仰宣撫司措置以聞。」童貫言:「凡逃卒,冬祀大赦已有百日首身免罪之文,緣內有元犯雖首身,於常法尚合移降移配者,即未敢赴
 官自陳。欲乞在京並京畿、京西、陜西、河東路逃軍,自今指揮到日,通未滿赦限共一百日,許令首身免罪,依舊軍分職次收管。仍免本司本營問當,及放免官逋。如本犯經冬祀赦後,猶有移降移配,特與原免。若限滿不首,則依常法科罪。凡逃軍系在京住營,依限於在京首身者,令所隸軍司當日押赴本營。若見出戍者,即破口券轉押赴本路駐泊州軍,並依前項指揮免罪,依舊收管。凡逃軍在外,依限首身者,並於所在日破米二升,其縣、
 鎮、砦並限當日解本州軍,每二十人作一番,差職員管押,仍沿路給破口食,交付前路州軍,轉送住營去處。如見出戍,即轉駐泊州軍收管。凡首身軍人,並不許投換他軍。凡所在當職官,如能於限內用心招收逃軍,措置轉送住營或出戍處收管,候滿,在外委提刑司,在京委開封府取索到營、出戍處公文,驗人數,最優者申宣撫司取旨推恩。」並從之。



 三年,詔:「江、浙軍前等處應逃竄軍兵,並特放罪,許於本將見出軍路分州縣首身,依舊給
 請,隨處權行收管。若走往他處,或於住營去處首獲,即令所在官司逐旋發遣赴本將應副使喚。仍委逐路安撫、鈐轄、提刑司覺察,如所在輒敢隱芘,或逐司不行覺察,並論違制。」



 四年,臣僚言:「中外士卒無故逃亡,所在有之。祖宗治軍紀律甚嚴,若在戍者還家,當役者避事,必有轅門之戮。今既宥其罪,且許投換,不制於什伍之長;既立赦限,又特展日,以寬其自首之期。臣恐逃亡得計,其弊益滋。乞除恩赦外不輕與限,使知限之不可為常,
 庶有畏懼。」從之。



 五年,臣僚言:「今諸軍逃亡者不以實聞。諸處冒名請給,至於揀閱差役,則巧為占破,甚不獲已,則雇募逋逃以充名數,旋即遁去,無復實用。平居難於供億,緩急無以應用。而奸人攘臂其間,坐費財賦。雖開收勘斂,法制滋詳,而共利之人,一體傅會。望賜處分,先令當職官核見實數,保明申達轉運司,期日委諸郡守貳點閱,仍關掌兵官司照會行下;不可勾押至州者,差官就閱,期以同日究見的實。稍涉欺罔,根治不赦。監司
 使者分郡覆實,具數申達於朝,以待差官分按,必行罪賞,使官無虛費,而軍有實用,則紀律可明,國用可省。」詔送樞密院條畫措置。



 七年二月,尚書省言:「開封府狀:『乞應在京犯盜配降出外之人,復走入京投換者,許人告捕,科以逃亡捕獲之罪,酌情增配。其官司及本營典首人員、曹級容庇收留,各杖一百;因致為盜者,依差使配軍入京作過法,與犯人同罪。罪止徒二年,不以去官赦原減。及在京犯罪編管出外逃亡入京之人,雖有斷罪
 增加地裡條法,緣止是募告賞格太輕,是致往往復走入京。欲乞元犯杖罪賞錢十貫,徒罪二十貫,流罪三十貫,並以犯事人家財充。』」從之。



 十二月,詔:「應諸路逃竄軍人或已該赦恩出首避免,卻歸出戍去處再行逃竄之人,令於所在去處首身,並特與免罪,於一般軍分安排,支破請給,發赴軍前使喚。」



 靖康元年三月,詔:「隨從行宮禁衛軍兵等有逃亡者,並依法施行。」五月,臣僚言:「泗州頃遣勤王之師,管押者不善統制,類多遁歸,既而畏法
 不敢出,本州遂開閣請受。在外無以給養,竊慮因聚為盜,恐他州亦多如此。乞敕應勤王兵有遁歸已經赦宥者,並令首身。」從之。



 六月,詔:「應河東潰散諸路將佐,並仰逐路帥守發遣赴河東、河北制置司,以功贖過。」河北路制置司都統制王淵言:「被旨差充招集種師道等下潰散人馬,應援太原,限滿不首,即寄禁家屬,許人收捕赴軍前,重行處置。」從之。仍自指揮到日,限以十日。河北路制使劉韐奏:「近制置使種師中領軍到於榆次,失利潰
 散,師中不知存在。奉旨,師中下應統制、將佐、使臣等,並與放罪。臣按:用兵失主將,統制、將佐並合行軍法。軍法行,則人以主將為重,緩急必須護救。若不行軍法,緩急之際爭先逃遁,視主將如路人,略不顧恤。近年以來,高永年陷歿,一行將佐及中軍將、提轄等未嘗罪以軍法,繼而劉法陷歿,今種師中又死王事。若兩軍相遇,勢力不加,血戰而敗,或失主將,亦無可言。榆次之戰,頃刻而潰,統制、將佐、使臣走者十已八九,軍士中傷十無一二,
 獨師中不出。若謂師中撫御少恩,紀律不嚴,而其受命即行,奮不顧身,初聞右軍戰卻,即遣應援,比時諸將已無在者。至賊兵犯營,師中猶未肯上馬。使師中有偷生之心,聞敗即行,亦必得出。一時將佐若能戮力相救,或可破敵。今一軍才卻,諸將不有主帥,相繼而遁。其初猶有懼色,既聞放罪,遂皆釋然。朝廷以太原之圍未解,未欲窮治。今師旅方興,深恐無所懲艾,遇敵必不用命。欲乞指揮,應種師中下統制、將佐並依聖旨處分,仍令軍
 前自效。如能用命立功。與免前罪;今後非立戰功,雖該恩赦不得敘復。仍乞優詔褒贈師中,以為忠義之勸。」詔:「種師中下統制、將佐並降五官,仍開具職位、姓名申尚書省,餘依劉韐所奏。」



 八月,河北、河東路宣撫司奏:「近據都統制王淵捉獲潰敗使臣,已管押赴宣撫副使劉韐軍前交割,依軍法施行外,訪聞尚有未曾出首將佐、使臣。」詔:「限今指揮到日更與展限十日,許令於所在州軍出首,仍依元降指揮免罪,特與支破遞馬驛券,疾速發
 赴軍前自效,候立功日優加推賞。如再限滿日更不首身,當取見職名重賞購捕,定行軍法。仍多出榜示諭。」



 二年四月,詔:「訪聞諸處潰散軍人嘯聚作過,將百姓強刺充軍,驅虜隨行使喚,遇敵使前,害枉良民。其令有司榜諭:被虜強刺之人許以自陳,給據各令歸業。願充軍者,隨等杖刺填禁、廂軍,依條支給例物。」又詔:「昨逃亡班直、諸軍,雖已降指揮撫諭,並與免罪,發歸元處。其管押兵官未有指揮,可候指揮到,許於所在官司自陳,亦與免
 罪。」



 建炎初,招募多西北之人,其後令諸路州、軍、砦或三衙招募,或選刺三衙軍中子弟,或從諸郡選刺中軍子弟解發。復詔滄、濱及江、淮沿流州軍,募善沒水經時伏藏者,以五千為額。神武右軍統制張俊言:「牙軍多招集烏合之眾,擬上等改刺勝捷,次等刺振華、振武,庶得部分歸一訓練為便。」詔兩浙、江東,除江陰軍,各募水軍二百人。



 紹興元年,廣東帥臣言:「本路將兵元五千二百,見千三百十九。今擬將官駐扎諸軍洎本路州軍,以十分
 為率,各招其半。」



 二年,累降令行在諸軍,毋互相招收,及將別軍人拘執,違者行軍法。



 四年,詔:「所招河北人充河北振武,餘人刺陜西振華指揮。沿江招置水軍,備戰艦,募東南諳水者充,每指揮以五百為額。」



 十年,詔三京路招撫處置使司招效用軍兵萬人,內招使臣二千員。



 十五年,福建安撫莫將言:「汀、漳、泉、建四州,與廣東、江西接壤。比年寇盜剽劫居民,土豪備私錢集社戶,防捍有勞,有司不為上聞推恩,破家無所依歸,勢必從賊。官軍不
 習山險,且瘴癘侵加,不能窮追,管屬良民悉轉為盜。請委四州守臣,募此游手無歸勇健之人,各收千人,仍以效用為名,足可備用,實永久利。」詔令張淵同措置。



 二十四年,殿前都指揮使楊存中言:「舊制,在京所管捧日、天武、拱聖、驍騎、驍勝、寧朔、神騎、神勇、宣武、虎翼、廣勇諸指揮禁軍內,捧日、天武依條升揀扈衛諸班直,拱聖、神勇以下升揀捧日、天武,除逃亡有故,僅千九百人。請於今年分定月內招千人。」



 二十七年,楊存中奉旨,三衙所招
 效用兵令住招。今闕六千七百二十六人,若不招填,兵數日損。詔本司來年正月為始,依舊招募。



 隆興元年,步軍司郭振言:「本司在京日軍額三萬九千五百,今行在僅千二百一十九。」詔招填千七百八十一人,以三千為額,刺充神衛,虎翼,飛山、床子弩雄武等指揮。



 乾道七年,馬軍司王友直言:「見管戰馬二千七百餘,止有傔馬六百餘人,請招傔兵千五百,並充雄威。」詔招千人,刺「步傔」二字。步軍司吳挺言:「步司五軍,額二萬五千,見闕三千
 六百。」詔令招填。



 淳熙十六年,殿前副都指揮郭鈞言:「淳熙五年住招兵,今逾十載,戰隊合用火分傔兵闕。」詔招千人。



 紹熙二年,詔步軍司招軍千人。



 慶元元年,詔楚州招到二百六十一人補弩手、效用。五年,詔給降度牒付金州都統,招填闕額並揀汰兵,照紹熙初年令,自五尺四寸至五尺六寸三等招收。



 開禧元年,興元都統秦世輔言:「本司軍多闕額,紹興之末,管二萬九千餘人。乾道三年,立額二萬七千,今二萬五千四百,差戍、官占實萬
 一百四十三人,點閱所部,堪披帶人僅六百二十七。請從本司酌紹興額招刺。」參知政事蔣芾言:「在內諸軍,每月逃亡不下四百人,若權住招一年半,俟財用稍足招強壯,不惟省費,又得兵精。且南渡以來兵籍之數,紹興十二年二十一萬四千五百餘人,二十三年二十五萬四千五百四十人,三十年三十一萬八千一百三十八人,乾道三年三十二萬三千三百一人,只比二十三年,已增六萬九千六十一人,如此何緣財用有餘?」



 寶慶二
 年,知武岡軍吳愈言:「禁衛兵所以重根本、威外夷,太祖聚天下精兵在京者十餘萬,州郡亦十餘萬。嘉定十五年,三衛馬步諸軍凡七萬餘,闕舊額三萬,若以川蜀、荊襄、兩淮屯戍較之,奚啻數倍於禁衛?宜遵舊制,擇州郡禁兵補禁衛闕,州郡闕額帥守招填。」



 紹定四年,臣僚言:「州郡有禁卒,有壯城,有廂軍,有土兵,一州之財自足以給一州之兵。比年尺籍多虛,月招歲補,悉成文具。蓋州郡吝養兵之費,所招無二三,逃亡已六七。宜申嚴帥臣,
 應郡守到罷,具兵額若干、逃故若干、招填若干、考其數而黜陟之。」



 寶祐間,州郡闕守,承攝者遣令招刺,不詢材武,務盜帑儲。



 咸淳季年,邊報日聞,召募尤急,官降錢甚優厚。強刺平民,非無法禁。所司莫能體上意,執民為兵。或甘言誑誘,或詐名賈舟,候負販者群至,輒載之去;或購航船人,全船疾趨所隸;或令軍婦冶容誘於路,盡涅刺之。由是野無耕人,途無商旅,往往聚丁壯數十,而後敢入市。民有被執而赴水火者,有自斷指臂以求免者,
 有與軍人抗而殺傷者,無賴乘機假名為擾。



 九年,賈似道疏云:「景定元年迄今,節次招軍凡二十三萬三千有奇,除填額,創招者九萬五千,近又招五萬,謂之無兵不可。」十年,汪立信書抵賈似道陳三策,一謂:「內地何用多兵,宜悉抽以過江,可行六十萬矣。蓋兵不貴多,貴乎訓練之有素。茍不堪受甲,徒取充數,將焉用之!」



 考之舊制,凡軍有闕額即招填。熙寧、元豐講求民兵之政,於是募兵浸減,而三衙多虛籍。至於靖康,禁衛弱矣。中興復用
 招募。立等杖,選勇壯,核人才,驗虛實,審刺之法雖在諸屯,而已招者兵籍悉總於樞府云。



\end{pinyinscope}