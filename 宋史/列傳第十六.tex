\article{列傳第十六}

\begin{pinyinscope}

 吳廷祚子元輔元載元扆李崇矩子繼昌王仁贍楚昭輔李處耘子繼隆繼和



 吳廷祚,字慶之,並州太原人。少頗讀書,事周祖,為親校。廣順初,授莊宅副使,遷內軍器庫使、知懷州,入為皇城
 使。會天平符彥卿移鎮大名,以廷祚權知鄆州。



 世宗即位,遷右羽林將軍,充內客省使。未幾,拜宣徽北院使。世宗征劉崇,為北面都巡檢使。師還,權判澶州。歸闕,加右監門衛大將軍。俄遷宣徽南院使、判河南府、知西京留守事。汴河決,命廷祚督丁壯數萬塞之。因增築堤防,自京城至臨淮,數旬訖工。世宗北征,權東京留守。是夏,河決鄭州原武縣,命廷祚發近縣丁壯二萬餘塞之。師還,以廷祚為左驍衛上將軍、檢校太傅,充樞密使。恭帝即
 位,加檢校太尉。



 宋初,加同中書門下三品,以其父名璋,故避之。會李筠叛,廷祚白太祖曰:「潞城巖險,且阻太行,賊據之,未易破也。筠素勇而輕,若速擊之,必離上黨來邀我戰,猶獸亡其藪,魚脫於淵,因可擒矣。」太祖遂親征,以廷祚留守東京兼判開封府。筠果領兵來,戰澤州南,其眾敗走。及討李重進,又為東京留守。



 建隆二年夏,帝謂之曰:「卿掌樞務,有年於茲,與卿秦州,以均勞逸。明日制出,恐卿以離朕左右為憂,故先告卿。」即以為雄武軍
 節度。先是,秦州夕陽鎮西北接大藪,多材植,古伏羌縣之地。高防知州日,建議就置採造務,調軍卒分番取其材以給京師。西夏酋長尚波於率眾爭奪,頗傷役卒,防捕擊其黨,以狀聞。上令廷祚代防,繼詔赦尚波於等,夏人感悅。是年秋,以伏羌地來獻。



 乾德二年來朝,改鎮京兆。開寶四年長春節來朝。俄遇疾,車駕臨問,命爇艾灸其腹,遣中使王繼恩監視之。未幾卒,年五十四。贈侍中,官給葬事。



 廷祚謹厚寡言,性至孝,居母喪,絕水漿累日。
 好學,聚書萬餘卷。治家嚴肅,尤崇奉釋氏。



 子元輔、元載、元範、元扆、元吉、元慶。元範、元慶仕皆至禮賓副使。元吉,閣門祗候。元吉子昭允,太子中舍。元慶子守仁,內殿崇班。



 元輔字正臣,頗好學,善筆札。周廣順中,以父任補供奉官。世宗嗣位,遷洛苑使。宋初,授左驍衛將軍、澶州巡檢,累官至定州鈐轄。卒,年四十八。子昭德、昭遜、昭普,並閣門祗候。



 元載,建隆初,授太子右春坊通事舍人,賜緋魚袋。廷祚出鎮秦、雍,並補衙門都校。廷祚卒,授供奉官。太平興國三年,加閣門祗候,與太祝母賓古使契丹。九年,擢為西上閣門副使,出知陜州。



 雍熙三年,徙知秦州。州民李益者,為長道縣酒務官,家饒於財,僮奴數千指,恣橫持郡吏短長,長吏而下皆畏之。民負息錢者數百家,郡為督理如公家租調,獨推官馮伉不從。益遣奴數輩伺伉按行市中,拽之下馬,因毀辱之。先是,益厚賂朝中權貴為
 庇護,故累年不敗。及伉屢表其事,又為邸吏所匿,不得達。後因市馬譯者附表以聞,譯因入見,上其表。帝大怒,詔元載逮捕之。詔書未至,京師權貴已報益,益懼,亡命。元載以聞,帝愈怒,詔州郡物色急捕之,獲於河中府民郝氏家,鞫於御史府,具得其狀,斬之,盡沒其家。益子仕衡先舉進士,任光祿寺丞,詔除籍,終身不齒。益之伏法,民皆飯僧相慶。



 端拱初,遷西上閣門使。淳化二年,加領富州刺史,俄徙知成都府。蜀俗奢侈,好游蕩,民無贏餘,
 悉市酒肉為聲妓樂,元載禁止之;吏民細罪又不少貸,人多怨咎。及王小波亂,元載不能捕滅,受代歸闕,而成都不守。



 時李仕衡通判華州,常銜元載因事殺其父,伺元載至闕,遣人閱行裝,收其關市之稅。元載拒之,仕衡抗章疏其罪,坐責郢州團練副使。移單州,以疾授左衛將軍致政。卒,年五十三。



 子昭明,為內殿崇班;昭矩,太子中舍。



 元扆字君華。太平興國八年,選尚太宗第四女蔡國公
 主,授左衛將軍、駙馬都尉。明年正月,領愛州刺史。是冬,領本州團練使。



 雍熙三年,有事北邊,元扆表求試劇郡,命知鄆州。逾年召入,尋知河陽。還朝,改鄯州觀察使。特詔朝會序班次節度使,奉祿賜予悉增之。再知河陽。



 淳化元年,以主疾召還。主薨,復遣之任。五年,秋霖河溢,奔注溝洫,城壘將壞,元扆躬涉泥滓,督工補塞。民多構木樹杪以避水,元扆命濟以舟楫,設餅餌以食。時澶、陜悉罹水災,元扆所部賴以獲安。



 真宗即位,換安州觀察使,
 俄知澶州。咸平三年,轉運使劉錫上其治狀,詔書嘉獎,遷寧國軍留後、知定州。時王超、王繼忠領兵逾唐河,與遼人戰,元扆度其必敗,乃急發州兵護河橋。既而超輩果敗,遼人乘之,至橋,見陣兵甚盛,遂引去。考滿,吏民詣闕貢馬,疏其善政十事,願借留樹碑,表其德政。詔褒之。屬歲旱,吏白召巫以土龍請雨。元扆曰:「巫本妖民,龍止獸也,安能格天?惟精誠可以動天。」乃集道人設壇,潔齋三日,百拜祈禱,澍雨沾洽。



 景德三年代歸,拜武勝軍節
 度。三年,以陵域積水,議堙掘溝澗,命為修諸陵都部署,以內侍副都知閻承翰副之。出知潞州。初,並、代、澤、潞皆分轄戍卒,後並於太原。至是,以元扆臨鎮,遂分領澤、潞、晉、絳、磁、隰、威勝七州軍戎事,委元扆專總之。東封,表求扈從,命祀青帝。禮畢,加檢校太傅、知徐州。大中祥符四年,以祀汾陰恩,改領山南東道。五月,制書下,元扆被疾卒,年五十。贈中書令,謚忠惠。子弟進秩者五人。五年,葬元扆,時上元欲觀燈,帝為移次夕。



 元扆性謹讓,在藩鎮
 有憂民心,待賓佐以禮。喜讀《春秋左氏》,聲色狗馬,一不介意。所得祿賜,即給親族孤貧者。將赴徐州,請對言:「臣族屬至多,其堪祿仕者皆為表薦,餘皆均奉贍之。」公主有乳媼,得入參宮禁,元扆慮其去後妄有請托,白上拒之。真宗深所嘉嘆,於帝婿中獨稱其賢。及歿,甚悼惜之。且以元扆得疾,本州不以聞,詔劾其官屬。



 子守禮,至六宅使、澄州刺史,以帝甥特贈和州防禦使;守嚴,至內殿崇班,天禧中,錄守嚴子承嗣、承緒並為殿直;守良為內
 殿崇班;守讓閣門祗候。



 李崇矩,字守則,潞州上黨人。幼孤貧,有至行,鄉里推服。漢祖起晉陽,次上黨,史弘肇時為先鋒都校,聞崇矩名,召署親吏。乾祐初,弘肇總禁兵兼京城巡檢,多殘殺軍民,左右懼,稍稍引去,惟崇矩事之益謹。及弘肇被誅,獨得免。



 周祖與弘肇素厚善,即位,訪求弘肇親舊,得崇矩。謂之曰:「我與史公受漢厚恩,戳力同心,共獎王室,為奸邪所構,史公卒罹大禍,我亦僅免。汝史氏家故吏也,為
 我求其近屬,吾將恤之。」崇矩上其母弟福。崇矩素主其家,盡籍財產以付福,周祖嘉之,以崇矩隸世宗帳下。顯德初,補供奉官。從征高平,以功轉供備庫副使,改作坊使。恭帝嗣位,命崇矩告哀於南唐。還判四方館事。



 宋初,李筠叛,命崇矩率龍捷、驍武左右射禁軍數千人屯河陽,以所部攻大會砦,拔之,斬首五百級。改澤、潞南面行營前軍都監,與石守信、高懷德、羅彥瑰同破筠眾於碾子穀。及平澤、潞,遣崇矩先入城,收圖籍,視府庫。因上言
 曰:「上黨,臣鄉里也。臣父尚槁葬,願護櫬歸京師。」許之,賜予甚厚。師還,會判三司張美出鎮,拜右監門衛大將軍,充三司使。從征李重進,還為宣徽北院使,仍判三司。



 乾德二年,代趙普拜樞密使。五年,加檢校太傅。時劍南初平,禁軍校呂翰聚眾構亂,軍多亡命在其黨中,言者請誅其妻子。太祖疑之,以語崇矩。崇矩曰:「叛亡之徒固當孥戮,然案籍合誅者萬餘人。」太祖曰:「朕恐有被其驅率,非本心者。」乃令盡釋之。翰眾聞之,亦稍稍自歸。未幾,翰
 敗滅。



 開寶初,從征太原。會班師,命崇矩為後殿。次常山,被病,帝遣太醫診視,命乘涼車還京師。崇矩叩頭言:「涼車乃至尊所御,是速臣死爾。」固辭得免。



 時趙普為相,崇矩以女妻普子承宗,相厚善,帝聞之不悅。有鄭伸者,客崇矩門下僅十年,性險詖無行,崇矩待之漸薄。伸銜之,因上書告崇矩陰事。崇矩不能自明。太祖釋不問,出為鎮國軍節度,賜伸同進士出身,以為酸棗主簿;仍賜器幣、襲衣、銀帶。六年,崇矩入為左衛大將軍。



 太平興國二
 年夏,河防多決,詔崇矩乘傳自陜至滄、棣,按行河堤。是秋,出為邕、貴、潯、賓、橫、欽六州都巡檢使。未幾,移瓊、崖、儋、萬四州都巡檢使,麾下軍士咸憚於行,崇矩盡出器皿金帛,凡直數百萬,悉分給之,眾乃感悅。時黎賊擾動,崇矩悉抵其洞穴撫慰,以己財遺其酋長,眾皆懷附。代還,拜右千牛衛上將軍。雍熙三年,命代宋偓判右金吾街仗兼六軍司事。端拱元年,卒,年六十五。贈太尉,謚元靖。



 崇矩性純厚寡言,尤重然諾。嘗事史弘肇,及卒,見其子
 孫,必厚禮之,振其乏絕。在嶺海四五年,恬不以炎荒嬰慮。舊涉海者多艤舟以俟便風,或旬餘,或彌月,崇矩往來皆一日而渡,未嘗留滯,士卒僮僕隨者皆無恙。信奉釋氏,飯僧至七十萬,造像建寺尤多。又喜黃白朮,自遠迎其人,館於家以師之,雖知其詐,猶以為神仙,試已終無悔恨。子繼昌。



 繼昌字世長。初,崇矩與太祖同府厚善,每太祖誕辰,必遣繼昌奉幣為壽。嘗畀弱弓輕矢,教以射法。建隆三年,
 蔭補西頭供奉官。太祖欲選尚公主,崇矩謙讓不敢當,繼昌亦自言不願。崇矩亟為繼昌聘婦,太祖聞之,頗不悅。



 開寶五年,選魏咸信為駙馬都尉,繼昌同日遷如京副使。崇矩出華州,補鎮國軍牙職。入為右班殿直、東頭供奉官,監大名府商稅,歲課增羨。會詔擇廷臣有勞者,府以名聞。丁外艱,服闋,授西京作坊副使。淳化中,齊饑多盜,命為登、萊、沂、密七州都巡檢使。



 至道二年,蜀賊平,餘黨頗嘯聚,拜西京作坊使、峽路二十五州軍捉賊招
 安都巡檢使,旋改兵馬鈐轄。賊酋喻雷燒者,久為民患,以金帶遺繼昌,繼昌偽納之,賊懈不設備,因掩殺之。進西京左藏庫使。



 咸平三年,王均亂蜀,與雷有終、上官正、石普同受詔進討,砦於城西門。賊忽開城偽遁,有終等各以所部徑入,繼昌覺,亟止之不聽,因獨還砦。賊果閉關發伏,悉陷之,有終等僅以身免。繼昌按堵如故,所部諸校聞城中戰聲,泣請引去。繼昌曰:「吾位最下,當俟主帥命。」是夕,有終馳報至,徙繼昌屯雁橋門。三月,破彌
 牟砦,斬首千級,大獲器仗,進逼魚橋門,均脫走。繼昌入城,嚴戒部下,無擾民者。獲婦女童幼置空寺中,俟事平遣還其家。繼昌急領兵追賊至資州,聞均梟首乃還。以功領獎州刺史。俄知青州,入掌軍頭引見司。



 景德二年,將幸澶州,遣先赴河上給諸軍鎧甲。遼人請和,欲近臣充使,乃令繼昌與其使姚東之偕詣遼部,俄與韓杞同至行在;及遼人聘至,又命至境首接伴。尋擢為西上閣門使。三年,又副任中正使契丹。是冬,將朝陵寢,以汝州近
 洛,衛兵所駐,命知州事兼兵馬鈐轄。駕還,召歸,出知延州兼鄜延路鈐轄。



 大中祥符元年,進秩東上閣門使。俄以目疾求歸京師。入對,勞問再三,遣尚醫診視,假滿仍給以奉。少愈,令樞密院傳旨,將真拜刺史,復任延安。繼昌以疾表求休致。未幾,改右驍衛大將軍,領郡如故。祀汾陰,留為京師新城巡檢鈐轄,改左神武軍大將軍、權判右金吾街仗。其子遵勖,尚萬壽長公主。



 天禧初,主誕日,邀繼昌過其家,迎拜為壽。帝知之,密以襲衣、金帶、器
 幣、珍果、美饌賜之。翌日,主入對,帝問繼昌強健能飲食,拜連州刺史,出知涇州。表求兩朝御書及謁拜諸陵,皆許之。二年冬,卒,年七十二。遣中使護櫬以歸。錄其子贊善大夫文晟為殿中丞,殿直文旦為侍禁。



 繼昌性謹厚,士大夫樂與之游。為治尚寬,所至民懷之。任峽路時,與上官正聯職。正殘忍好殺,嘗有縣胥護芻糧,地遠後期,正令斬之,繼昌徐為解貸焉。鄭伸者,早死,其母貧餓,嘗詣繼昌乞丐,家人兢前詬逐。繼昌召見,與白金百兩,時
 人稱之。



 遵勖初尚主,詔升為崇矩子,授昭德軍留後、駙馬都尉。



 王仁贍,唐州方城人。少倜儻,不事生產,委質刺史劉詞。詞遷永興節度,署為牙校。詞將卒,遺表薦仁贍材可用。太祖素知其名,請於世宗,以隸帳下。



 宋初,授武德使,出知秦州,改左飛龍使。建隆二年,遷右領軍衛將軍,充樞密承旨。高繼沖請命,以仁贍為荊南巡檢使。繼沖入朝,命知軍府。乾德初,遷左千牛衛大將軍。不逾月,加內客
 省使。



 二年春,召赴闕,擢為樞密副使。七月,加左衛大將軍。興師討蜀,命仁贍為鳳州路行營前軍都監。蜀平,坐沒入生口財貨、殺降兵致蜀土擾亂,責授右衛大將軍。初,劍南之役,大將王全斌等貪財,軍政廢弛,寇盜充斥。太祖知之,每使蜀來者,令陳全斌等所入賄賂、子女及發官庫分取珠金等事,盡得其狀。及全斌等歸,帝詰仁贍,仁贍歷詆諸將過失,欲自解。帝曰:「納李廷珪妓女,開豐德庫取金寶,豈全斌輩邪?」仁贍不能對。廷珪,故蜀將
 也。帝怒,令送中書鞫全斌等罪,仁贍以新立功,第行降黜而已。帝幸洛,以仁贍判留守司、三司兼知開封府事。及召沉倫赴行在,以仁贍為東京留守兼大內都部署。駕還,遂判三司,俄命權宣徽北院事。



 太平興國初,拜北院使兼判如故,加檢校太保。四年,親征太原,充大內部署,仍判留守司、三司,總轄里外巡檢司公事。師還,加檢校太傅。五年,仁贍廉得近臣戚里遣人市竹木秦、隴間,聯巨筏至京師,所過關渡,矯稱制免算;既至,厚結有司,
 悉官市之,倍收其直。仁贍密奏之,帝怒,以三司副使範旻、戶部判官杜載、開封府判官呂端屬吏。旻、載具伏罔上為市竹木入官;端為秦府親吏喬璉請托執事者。貶旻為房州司戶,載均州司戶,端商州司戶。判四方館事程德玄、武德使劉知信,翰林使杜彥圭,日騎、天武四廂都指揮使趙延溥,武德副使竇神興,左衛上將軍張永德,左領軍衛上將軍祁廷訓,駙馬都尉王承衎、石保吉、魏咸信,並坐販竹木入官,責降罰奉。是歲,車駕北巡,命
 仁贍為大內部署。



 七年春,以政事與僚屬相矛盾,爭辯帝前,仁贍辭屈,責授右衛大將軍。翌日,改唐州防禦使,月給奉錢三十萬。仁贍之獲罪也,兵部郎中、判勾院宋琪及三司判官並降秩。先是,仁贍掌計司殆十年,恣下吏為奸,怙恩寵無敢發者;前者發範旻等事,中外益畏其口。會屬吏陳恕等數人率以皦察不畏強禦自任,因議本司事有不協者。朝參日,恕獨出班持狀奏其事。帝詰之,仁贍屈伏。帝怒甚,故及於譴,而恕等悉獎擢。琪與
 恕等聯事,始合謀同奏,至帝前而宋琪猶附會仁贍,故亦左降。仁贍既失權勢,因怏怏成疾,數日卒,年六十六。



 後帝因言及三司財賦,謂宰相趙普等曰:「王仁贍領邦計積年,恣吏為奸,諸場院官皆隱沒官錢以千萬計,朕悉令罷之,命使分掌。仁贍再三言,恐虧舊數,朕拒之。未逾年,舊獲千緡者為一二萬緡,萬緡者為六七萬緡,其利數倍,用度既足,儻遇水旱,即可免民租稅。仁贍心知其非,頗亦慚悸,朕優容之。」子昭雍,為崇儀副使。



 楚昭輔,字拱辰,宋州宋城人。少事華帥劉詞。詞卒,事太祖,隸麾下,以才幹稱,甚信任之。陳橋師還,昭憲太后在城中,太祖憂之,遣昭輔問起居,昭輔具言士眾推戴之狀,太后乃安。



 宋初,為軍器庫使。太祖親討澤、潞,及征淮揚,並以昭輔為京城巡檢。建隆四年,權知揚州,使江表。還,命鉤校左藏庫金帛,數日而畢,條對稱旨。開寶四年,帝以其能心計,拜左驍衛大將軍、權判三司。六年,遷樞密副使。九年,命權宣徽南院事。



 太平興國初,拜樞密使。
 三年,加檢校太傅。從征太原,加檢校太尉。俄以足疾請告,帝親臨問。以所居湫隘,命有司廣之,昭輔慮侵民地,固讓不願治。帝嘉其意,賜白金萬兩,令別市第。昭輔被疾,家居近一歲,始以石熙載代之。昭輔不求解職,上亦不忍罷。會郊祀畢,罷為驍騎衛上將軍。逾年卒,年六十九。廢朝,贈侍中,命中使護其喪歸葬鄉里。無子,錄其兄子吉為供奉官,敏為殿直。



 昭輔性勤介,人不敢干以私,然頗吝嗇,前後賜予萬計,悉聚而畜之。嘗引賓客故舊
 至藏中縱觀,且曰:「吾無汗馬勞,徒以際會得此,吾為國家守爾,後當獻於上。」及罷機務,悉以市善田宅,時論鄙之。



 初,詞卒,昭輔來京師,問卜於瞽者劉悟。悟為筮卜,曰:「汝遇貴人,見奇表豐下者即汝主也,宜謹事之,汝當貴矣。」及見太祖,狀貌如悟言,遂委質焉。



 咸平三年,錄弟之子諒為借職。大中祥符八年,又錄從孫鼎為右班殿直。吉至內殿崇班。吉子隨,敏子咸,並進士及第,隨為太常博士,咸屯田員外郎。



 李處耘,潞州上黨人。父肇,仕後唐,歷軍校,至檢校司徒。從討王都定州,契丹來援,唐師不利,肇力戰死之。晉末,處耘尚幼,隨兄處疇至京師,遇張彥澤斬關而入,縱士卒剽略。處耘年猶未冠,獨當里門,射殺十數人,眾無敢當者。會暮夜,遂退。迨曉復鬥,又殺數人,鬥未解。有所親握兵,聞難來赴,遂得釋,里中賴之。



 漢初,折從阮帥府州,召置門下,委以軍務。從阮後歷鄧、滑、陜、邠四節度,處耘皆從之。在新平日,折氏甥詣闕誣告處耘之罪,周祖信
 之,黜為宜祿鎮將。從阮表雪其冤,詔復隸麾下。



 顯德中,從阮遺表稱處耘可用,會李繼勛鎮河陽,詔署以右職。繼勛初不為禮,因會將吏宴射,處耘連四發中的,繼勛大奇之,令升堂拜母,稍委郡務,俾掌河津。處耘白繼勛曰:「此津往來者懼有奸焉,不可不察也。」居數月,果得契丹諜者,索之,有與西川、江南蠟書,即遣處耘部送闕下。



 太祖時領殿前親軍,繼勛罷鎮,世宗以處耘隸太祖帳下,補都押衙。會太祖出征,駐軍陳橋,處耘見軍中謀欲
 推戴,遽白太宗,與王彥升謀,召馬仁瑀、李漢超等定議,始入白太祖,太祖拒之。俄而諸軍大噪,入驛門,太祖不能卻。處耘臨機決事,謀無不中,太祖嘉之,授客省使兼樞密承旨、右衛將軍。



 從平澤、潞,遷羽林大將軍、宣徽北院使。討李重進,為行營兵馬都監。賊平,以處耘知揚州。大兵之後,境內凋弊,處耘勤於綏撫,奏減城中居民屋稅,民皆悅服。建隆三年,詔歸京師,老幼遮道涕泣,累日不得去。拜宣徽南院使兼樞密副使,賜甲第一區。



 朗州
 軍亂,詔慕容延釗率師討之,以處耘為都監。入辭,帝親授方略,令會兵漢上。先是,朝廷遣內酒坊副使盧懷忠使荊南,覘勢強弱。使還,具言可取之狀,遂命處耘圖之。處耘至襄州,先遣閣門使丁德裕假道荊南,請具薪水給軍,荊人辭以民庶恐懼,願供芻餼於百里外。處耘又遣德裕諭之,乃聽命。遂令軍中曰:「入江陵城有不由路及擅入民舍者斬。」



 師次荊門,高繼沖遣其叔保寅及軍校梁延嗣奉牛酒犒師,且來覘也。處耘待之有加,諭令
 翌日先還。延嗣大喜,令報繼沖以無虞。荊門距江陵百餘里,是夕,召保寅等飲宴延釗之帳。處耘密遣輕騎數千倍道前進。繼沖但俟保寅、延嗣之還,遽聞大軍奄至,即惶怖出迎,遇處耘於江陵北十五里。處耘揖繼沖,令待延釗,遂率親兵先入登北門。比繼沖還,則兵已分據城中,荊人束手聽命。即調發江陵卒萬餘人,並其師,晨夜趨朗州。又先遣別將分麾下及江陵兵趨岳州,大破賊於三江口,獲船七百餘艘,斬首四千級。又遇賊帥張
 從富於澧江,擊敗之。逐北至敖山砦,賊棄砦走,俘獲甚眾。處耘釋所俘體肥者數十人,令左右分啖之,黥其少健者,令先入朗州。會暮,宿砦中,遲明,延釗大軍繼至。黥者先入城言,被擒者悉為大軍所啖,朗人大懼,縱火焚城而潰。會朗帥周保權年尚幼,為大將汪端劫匿於江南砦僧寺中。處耘遣麾下將田守奇帥師渡江獲之。遂入潭州,盡得荊湖之地。



 初,師至襄州,衢肆鬻餅者率減少,倍取軍人之直。處耘捕得其尤者二人送延釗,延釗
 怒不受,往復三四,處耘遂命斬於市以徇。延釗所部小校司義舍於荊州客將王氏家,使酒兇恣,王氏醞于處耘。處耘召義呵責,義又譖處耘於延釗。至白湖,處耘望見軍人入民舍,良久,舍中人大呼求救,遣捕之,即延釗圉人也,乃鞭其背,延釗怒斬之。由是大不協,更相論奏。朝議以延釗宿將貰其過,謫處耘為淄州刺史。處耘懼,不敢自明。在州數年,乾德四年卒,年四十七。廢朝,贈宣德軍節度、檢校太傅,賜地葬於洛陽偏橋村。



 處耘有度
 量,善談當世之務,居常以功名為己任。荊湖之役,處耘以近臣護軍,自以受太祖之遇,思有以報,故臨事專制,不顧群議,遂至於貶。後太祖頗追念之。及開寶中,為太宗納其次女為妃,即明德皇后也。



 子繼隆、繼和,自有傳;繼恂,官至洛苑使、順州刺史,贈左神武大將軍。繼恂子昭遜,為供備庫使。處疇,官至作坊使,子繼凝。



 繼隆字霸圖,幼養於伯父處疇。及長,以父蔭補供奉官。處耘貶淄州,繼隆亦除籍。會長春節,與其母入貢,復舊
 官。時權臣與處耘有宿憾者,忌繼隆有才,繼隆因落魄不治產,以游獵為娛。



 乾德中平蜀,選為果、閬監軍,年方弱冠,母憂其未更事,將輔以處耘左右。繼隆曰:「是行兒自有立,豈須此輩,願不以為慮。」母慰而遣之。代還,夜涉棧道,雨滑,與馬偕墜絕澗,深十餘丈,絓於大樹。騎卒馳數十里外,取火引綆以出之。



 會征江南,領雄武卒三百戍邵州,止給刀盾。蠻賊數千陣長沙南,截其道。繼隆率眾力戰,賊遁去,手足俱中毒矢,得良藥而愈,部卒死傷
 者三之一。太祖聞其勇敢而器重之。又與石曦率兵襲袁州,破桃田砦,追賊二十里,入潭富砦,焚其梯沖芻積。



 復從李符督荊湖漕運,給征南諸軍。吳人以王師不便水戰,多出舟師斷餉道,繼隆屢與鬥,糧悉善達。日馳四五百里,常令往來覘候。一日中途遇虎,射殺之。嘗獲吳將,部送赴闕,至項縣而病,斬其首以獻,太祖益嘉之。與吳人戰,流矢中額,以所冠冑堅厚,得不傷。



 太祖察其才,且追念其父,欲拔用之,謂曰:「升州平,可持捷書來,當厚
 賞汝。」時內侍使軍中者十數輩,皆伺城陷獻捷,會有機事當入奏,皆不願行,而繼隆獨請赴闕。太宗見其來,時城尚未下,甚訝之。繼隆度金陵破在旦夕,因言在途遇大風晦暝,城破之兆也。翌日,捷奏至,太祖召謂曰:「如汝所料矣。」吳將盧絳聚眾萬餘,攻掠州縣,命繼隆招來之。江南平,錄功遷莊宅副使。從幸西洛,改御營前後巡檢使。



 太平興國二年,改六宅使。嘗詔與王文寶、李神祐、劉承珪同護浚京西河,又與梁迥、竇神寶治決河。迥體肥
 碩,所乘舟弊不能濟,繼隆易以己舟。已而繼隆舟果覆,棲枯桑杪,賴他舟以度。



 從征太原,為四面提舉都監,與李漢瓊領梯沖地道攻城西面,機石過其旁,從卒僕死,繼隆督戰無怠。討幽州,與郭守文領先鋒,破契丹數千眾。及圍範陽,又與守文為先鋒,大敗其眾於湖翟河南。



 後為鎮州都監,契丹犯邊,與崔翰諸將御之。初,太宗授以陣圖,及臨陣有不便,眾以上命不可違。繼隆曰:「事有應變,安可預定,設獲違詔之罪,請獨當也。」即從宜而行,
 敗之於徐河。



 四年,遷宮苑使、領媯州刺史,護三交屯兵。與潘美出征北邊,破靈丘縣,盡略其人以歸。改定州駐泊都監。嘗領兵出土鐙砦,與賊戰,獲牛羊、車帳甚眾。詔書褒美。



 李繼遷叛,命繼隆與田仁朗、王侁率兵擊之。四月,出銀州北,破悉利諸族,追奔數十里,斬三千餘級,俘蕃漢老幼千餘,梟代州刺史折羅遇及其弟埋乞首,牛馬、鎧仗所獲尤多。又出開光谷西杏子坪,破保寺、保香族,斬其副首領埋乜已五十七人,降銀三族首領析八
 軍等三千餘眾,復破沒邵浪、悉訛諸族,及濁輪川東、兔頭川西,生擒七十八人,斬首五十九級,俘獲數千計。引師至監城,吳移、越移四族來降,惟岌伽羅膩十四族怙其眾不下,乃與尹憲襲擊之,夷其帳千餘,俘斬七千餘級。俄改領環州團練使,又護高陽關屯兵。



 從曹彬徵幽州,率兵助先鋒薛繼昭破其眾數千於固安南,下固安、新城,進克涿州,矢中左股,血流至踵,獲契丹貴臣一人。彬欲上其功,繼隆止之。俄而傅潛、米信軍敗眾潰,獨繼
 隆所部振旅而還。即命繼隆知定州,尋詔分屯諸軍,繼隆令書吏盡錄其詔。旬餘,有敗卒集城下,不知所向,繼隆按詔給券,俾各持詣所部。太宗益嘉其有謀。



 三年,遷侍衛馬軍都虞候、領武州防禦使。契丹大入邊,出為滄州都部署。劉廷讓與敵戰君子館,先約繼隆以精卒後殿,緩急為援。既而敵圍廷讓數重,繼隆引麾下兵退保樂壽,廷讓力不敵,全軍陷沒,裁以單騎遁免。上怒,追繼隆赴闕,令中書問狀,既而得釋。逾年,加領本州觀察使。



 端拱初,制授侍衛馬軍都指揮使、領保順節度。九月,出為定州都部署。初,朝議有寇至,令堅壁清野,勿與戰。一日,契丹驟至,攻蒲城,至唐河。護軍袁繼忠慷慨請出師,中黃門林延壽等五人以詔書止之。繼隆曰:「閫外之事,將帥得專。」乃與繼忠出兵,戰數合,擊走之。



 二年冬,送芻粟入威虜軍,蕃將於越率騎八萬來邀王師,繼隆所領步騎裁一萬,先命千人設伏城北十里,而與尹繼倫列陣以待。敵眾方食,繼倫出其不意,擊走之。繼隆追奔過
 徐河,俘獲甚眾。嘗有詔廢威虜軍,繼隆言:「梁門為北面保障,不可廢。」遂城守如故,訖為要地。



 淳化初,上遣使至定州,密諭繼隆:「若契丹復入寇,朕當親討。」繼隆上奏曰:「自北邊肆孽,邊邑多虞,陛下不知臣不材,任以疆事,臣敢不講求軍實,震耀戎容,奉揚天聲,以遏外侮。然臣奉辭之日,曾瀝愚衷,誠以蜂蟻之妖,必就鯨鯢之戮。臣子之分,死生以之,望不議於親巡,庶靡勞於天步,今聆聖誨,將決親征,且一人既行,百司景從,次舍驅馳,郡縣供
 饋,勞費滋甚。殄此微妖,當責將帥,臣雖駑弱,誓死為期。」是歲,契丹不入邊,議遂止。



 四年夏,召還,太宗面獎之,改領靜難軍節度,復遣還屯所。時夏州趙保忠與繼遷連謀,朝廷患之,又綏州牙校高文□不舉城效順,河外蕃漢大擾,以繼隆為河西行營都部署、尚食使尹繼隆為都監以討之。既而繼遷遁去,擒保忠以獻。初,裨將侯延廣、監軍秦翰議請誅保忠,及出兵追之,繼隆曰:「保忠杌上肉爾,當請於天子。今繼遷遁去,千里窮磧,艱於轉餉,宜
 養威持重,未易輕舉。」延廣等服其言。



 會密詔廢夏州,隳其城。繼隆命秦翰與弟繼和及高繼勛同入奏,以為朔方古鎮,賊所窺覦之地,存之可依以破賊;並請於銀、夏兩州南界山中增置保戍,以扼其沖,且為內屬蕃部之障蔽,而斷賊糧運。皆不報。



 至道二年,白守宗守榮、馬紹忠等送糧靈州,為繼遷所邀,敗於洛浦河。上聞之怒,亟命繼隆為靈、環十州都部署。是秋,五路討繼遷,以繼隆出環州,取東關鎮,由赤檉、苦井路赴之。繼隆以所出道
 回遠乏水,請由橐駝路徑趨賊之巢穴。且遣繼和入奏,太宗召詰之,知其必敗,因遣周瑩繼手詔切責,督其進軍赤檉。瑩至,繼隆以便宜發兵,不俟報,與丁罕行十餘日,果不見賊而還。諸將失期,士卒困乏。繼隆素剛,因慚憤,肆殺戮,乃奏轉運使陳絳、梁鼎軍儲不繼,並坐削秩。



 三年春,繼遷以蕃部從順者眾,遣其軍主史□□遇率兵屯橐駝口西北雙雉,以遏絕之。執倉族蕃官□□遇來告,繼隆遣劉承蘊、田敏會□□遇討之,斬首數千級,獲牛馬、
 橐駝萬計。



 先是,受詔送軍糧赴靈州,必由旱海路,自冬至春,而芻粟始集。繼隆請由古原州蔚茹河路便,眾議不一,繼隆固執論其事,太宗許焉。遂率師以進,壁古原州,令如京使胡守澄城之,是為鎮戎軍。



 真宗即位,改領鎮安軍節度、檢校太傅。逾月召還,加同中書門下平章事,解兵柄歸本鎮。咸平二年,丁內艱,起復。會秋潦暴集,蔡水壞岸,繼隆乘危督士卒補塞,自辰訖午,沖波稍息。四年,加檢校太師。王師失利於望都,繼隆累表求詣闕
 面陳邊事,因乞自效。俄召還,延見詢訪,因言:「醜類侵擾,蓋亦常事,願委將帥討伐,不煩親征。」真宗慰諭之,改山南東道節度,判許州。景德初,明德皇太后不豫,詔入省疾。九月,復許會葬。是冬,契丹大入,逾魏郡至河上。真宗幸澶淵,繼隆表求扈從,命為駕前東西排陣使,先赴澶州,陳師於北城外,毀車為營。敵數萬騎急攻,繼隆與石保吉率眾御之,追奔數里。及上至,幸北門觀兵,召問慰勞,見其所部整肅,嘆賞久之。翌日,幸營中,召從臣飲宴。
 二年春,還京,加開府儀同三司、食邑、實封。詔始下,會疾作,上親臨問。繼和時為並、代鈐轄,驛召省視。卒,年五十六。車駕臨哭之慟,為制服發哀。贈中書令,謚忠武。以其子昭慶為洛苑使,從子昭囗、昭遜,並為內殿崇班。又錄其門下二十餘人。乾興初,詔與李沆,王旦同配享真宗廟庭。



 繼隆出貴冑,善騎射,曉音律,感慨自樹,深沉有城府,嚴於御下。好讀《春秋左氏傳》,喜名譽,賓禮儒士。在太宗朝,特被親信,每征行,必委以機要。真宗以元舅之親,
 不欲煩以軍旅,優游近藩,恩禮甚篤。然多智用,能謙謹保身。明德寢疾,欲面見之,上促其往。繼隆但詣萬安宮門拜箋,終不入。又嘗命諸王詣第候謁,繼隆不設湯茗,第假王府從行茶爐烹飲焉。昭慶改名昭亮,至東上閣門使、高州刺史。



 繼和字周叔,少以蔭補供奉官,三遷洛苑使。淳化後,繼隆多在邊任,繼和常從行,友愛尤至,每令入奏機事。繼隆罷兵柄,手錄唐李績遺戒授繼和,曰:「吾門不墜者在
 爾矣。」



 初,繼隆之請城鎮戎軍也,朝廷不果於行。繼和面奏曰:「平涼舊地,山川阻險,旁扼夷落,為中華襟帶,城之為便。」太宗乃許焉。後復不守。咸平中,繼和又以為言,乃命版築,以繼和知其軍,兼原、渭、儀都巡檢使。城畢,加領平州刺史。建議募貧民及弓箭手,墾田積粟,又屢請益兵,朝議未許。上曰:「茍緩急,部署不為濟師,則或至失援矣。」命繼和兼涇、原、儀、渭鈐轄。時繼遷未弭,命張齊賢、梁顥經略,因訪繼和邊事。繼和上言:



 鎮戎軍為涇、原、儀、渭
 北面捍蔽,又為環、慶、原、渭、儀、秦熟戶所依,正當回鶻、西涼、六谷、吐蕃、咩逋、賤遇、馬臧、梁家諸族之路。自置軍已來,克張邊備,方於至道中所葺,今已數倍。誠能常用步騎五千守之,涇、原、渭州茍有緩急,會於此軍,並力戰守,則賊必不敢過此軍;而緣邊民戶不廢耕織,熟戶老幼有所歸宿。



 此軍茍廢,則過此新城,止皆廢壘。有數路來寇:若自隴山下南去,則由三百堡入儀州制勝關;自瓦亭路南去,則由彈箏峽入渭州安國鎮;自清石嶺東南
 去,則由小盧、大盧、潘穀入潘原縣;若至潘原而西則入渭州,東則入涇州;若自東石嶺東公主泉南去,則由東山砦故彭陽城西並入原州;其餘細路不可盡數。如以五千步騎,令四州各為備御,不相會合,則兵勢分而力不足禦矣。故置此城以扼要路。



 即令自靈、環、慶、鄜、延、石、隰、麟、府等州以外河曲之地,皆屬於賊,若更攻陷靈州,西取回鶻,則吐蕃震懼,皆為吞噬,西北邊民,將受驅劫。若以可惜之地,甘受賊攻,便思委棄,以為良策,是則有
 盡之地,不能供無已之求也。



 臣慮議者以調發芻糧擾民為言,則此軍所費,上出四川,地里非遙,輸送甚易。又劉琮方興屯田,屯田若成,積中有備,則四州稅物,亦不須得。



 況今繼遷強盛,有逾曩日。從靈州至原、渭、儀州界,次更取棨子山以西接環州山內及平夏,次並黃河以東以南、隴山內外接儀州界,及靈州以北河外。蕃部約數十萬帳,賊來足以鬥敵,賊遷未盛,不敢深入。今則靈州北河外,鎮戎軍、環州並北徹靈武、平夏及山外黃河
 以東族帳,悉為繼遷所吞,縱有一二十族,殘破奔迸,事力十無二三。



 自官軍瀚海失利,賊愈猖狂,群蕃震懼,絕無鬥志。兼以咸平二年棄鎮戎後,繼遷徑來侵掠軍界蕃族,南至渭州安國鎮北一二十里,西至南市界三百餘里,便於蕭關屯聚萬子、米逋、西鼠等三千,以脅原、渭、靈、環熟戶,常時族帳謀歸賊者甚多。賴聖謨深遠,不惑群議,復置此軍,一年以來,蕃部咸以安集,邊民無復愁苦。以此較之,則存廢之說,相失萬倍矣。



 又靈州遠絕,居
 常非有尺布斗粟以供王府,今關西老幼,疲苦轉餉,所以不可棄者,誠恐滋大賊勢,使繼遷西取秦、成之群蕃,北掠回鶻之健馬,長驅南牧,何以枝梧。昨朝廷訪問臣送芻糧道路,臣欲自蕭關至鎮戎城砦,西就胡盧河川運送。但恐靈州食盡,或至不守,清遠固亦難保,青岡、白馬曷足禦□,則環州便為極邊。若賊從蕭關、武延、石門路入鎮戎,縱有五七千兵,亦恐不敵,即回鶻、西涼路亦斷絕。



 伏見咸平三年詔書,緣邊不得出兵生事蕃夷,蓋
 謂賊如猛獸,不怫其心,必且不動。臣愚慮此賊他日愈熾,不若聽驍將銳旅屢入其境,彼或聚兵自固,則勿與鬥,妖黨才散,則令掩擊。如此則王師逸而賊兵勞,賊心內離,然後大舉。



 及靈州孤壘,戍守最苦,望比他州尤加存恤。且守邊之臣,內憂家屬之窘匱,外憂奸邪之憎毀。憂家則思為不廉,憂身則思為退跡,思不廉則官局不治,思退跡則庶事無心,欲其奮不顧身,令出惟行,不可得已。良由賞未厚、恩未深也。賞厚則人無顧內之憂,恩
 深則士有效死之志。古之帝王皆懸爵賞以拔英俊,卒能成大功。



 大凡君子求名,小人徇利。臣為兒童時,嘗聞齊州防禦使李漢超守關南,齊州屬州城錢七八萬貫,悉以給與,非次賞賚,動及千萬。漢超猶私販榷場,規免商算,當時有以此事達於太祖者,即詔漢超私物所在,悉免關征。故漢超居則營生,戰則誓死,貲產厚則心有所系,必死戰則動有成績。故畢太祖之世,一方為之安靜。今如漢超之材固亦不少,茍能用皇祖之遺法,選擇
 英傑,使守靈武,高官厚賞,不吝先與;往日,留半奉給其家,半奉資其用,然後可以責潔廉之節,保必勝之功也。



 又戎事內制,或失權宜,漢時渤海盜起,龔遂為太守,尚聽便宜從事。且渤海,漢之內地,盜賊,國之饑民;況靈武絕塞,西鄙強戎,又非渤海之比。茍許其專制,則無失事機,縱有營私冒利,民政不舉,亦乞不問。用將之術,異於他官,貪勇知愚,無不皆錄,但使法寬而人有所慕,則久居者安心展體,竭材盡慮,何患靈州之不可守哉?



 又朝
 廷比禁青鹽,甚為允愜。或聞議者欲開其禁。且鹽之不入中土,困賊之良策也。今若謂糧食自蕃界來,雖鹽禁不能困賊,此鬻鹽行賄者之妄談也。蕃粟不入賊境,而入於邊廩,其利甚明。況漢地不食青鹽,熟戶亦不入蕃界博易,所禁者非徒糧食也,至於兵甲皮乾之物,其名益多。以朝廷雄富,猶言摘山煮海,一年商利不入,則或闕軍須。況蕃戎所賴,止在青鹽,禁之則彼自困矣。望固守前詔為便。



 五年,繼和領兵殺衛埋族於天麻川。自是
 壟山外諸族皆恐懼內附,願於要害處置族帳砦柵,以為戍守。繼和因請移涇原部署於鎮戎,以壯軍勢,又請開道環、延為應援。真宗以其精心戎事,甚嘉之。戎人伺警巡馳備,一夕,塞長壕,越古長城抵城下。繼和與都監史重貴出兵御之,賊據險再突城隍,列陣接戰,重貴中重創,敗走之,大獲甲騎。有詔嘉獎,別出良藥、縑帛、牢酒以賜。



 繼和習武藝,好談方略,頗知書,所至乾治。然性剛忍,御下少恩,部兵終日擐甲,常如寇至;及較閱之際,杖
 罰過當,人多怨焉。真宗屢加勖勵,且為覆護之。嘗上言:「保捷軍新到屯所,多亡命者,請優賜緡錢;茍有亡逸,即按軍法。」舊制,凡賜軍中,雖緣奏請者,亦以特旨給之。上以繼和峻酷,欲軍士感其惠,特令以所奏著詔書中而加賜之。且以計情定罪,自有常制,不許其請。終以邊防之地,慮人不為用,遣張志言代還。既即路,軍中皆恐其復來。



 六年,又出為並、代鈐轄。將行請對,欲領兵去按度邊壘。上曰:「河東巖險,兵甲甚眾,賊若入寇,但邀其歸路,
 自可致勝,不必率兵而往也。」



 景德初,北邊入寇,徙北平砦。車駕駐澶淵,繼和受詔與魏能、張凝領兵赴趙州躡敵後。契丹請和,邊民猶未寧,又命副將張凝為緣邊巡檢安撫使。事平,復還並、代。時朝廷每詔書約束邊事,或有當行極斷之語,官吏不詳深意,即處大闢。繼和言其事,乃詔:「自今有雲重斷、極斷、處斬、決配之類,悉須裁奏。」先是,繼隆卒,繼和恥以遺奏得官。久之,遷西上閣門使。未幾,擢殿前都虞候、領端州防禦使。大中祥符元年卒,
 年四十六。贈鎮國軍節度,遣諸王率宗室素服赴吊。二子早卒。帝以其族盛大,諸侄皆幼,令三班選使臣為主家事。



 弟繼恂,至洛苑使、順州刺史,贈左神武大將軍。子昭遜為供備庫使。



 論曰:夫乘風雲之會,依日月之光,感慨發憤,效忠駿奔,居備要任,出握重兵,如是而令名克終,斯固可偉也。吳廷祚策李筠之破,如目睹其事,誠有將略。李崇矩秉純厚之德,感史弘肇之恩,保其叛亡之孥,然交鄭伸不知
 其傾險,坐謫炎海,固無先見之明矣;其子繼昌,忘父仇以恤伸母之貧,雖非中道,亦人所難。王仁贍征蜀,殺降附之卒,肆貪矯之行,鬱鬱而斃,自貽伊戚,尚何尤乎?楚昭輔當陳橋推戴,太祖遣之入安母後,亦必可托以事者;及為三司,善於心計,人不可干以私,然終以訐直,取寡信之名,何歟?處耘於創業之始,功參締構,克荊山,靖衡、湘,勢如拉枯,而志昧在和,勛業弗究,良可惜也;幸聯戚畹之貴,秉旄繼世,抑造物之報,嗇此而豐彼歟?



\end{pinyinscope}