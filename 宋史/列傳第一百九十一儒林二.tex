\article{列傳第一百九十一儒林二}

\begin{pinyinscope}

 ○胡旦賈同劉顏高弁孫復石介胡瑗劉羲叟林概李覯何涉王回弟向周堯卿王當陳暘



 胡旦字周父,濱州渤海人。少有雋才,博學能文辭。舉進士第一,為將作監丞、通判升州。時江南初平,汰李氏時所度僧,十減六七。旦曰:「彼無田廬可歸,將聚而為盜。」悉黥為兵。遷左拾遺、直史館,數上書言時政利病。出為淮南東路轉運副使、知海州。逾年,召歸。



 先是,盧多遜貶,趙普罷相。其夏,河決韓村,尋復塞。旦獻《河平頌》曰:「天祚我宋,以君兆民。配天成休,惟堯與鄰。粵有大水,昏墊下人。非曰聖作,孰究孰度。蔽賢者退,壅澤者罪。我防大患,河
 豈雲敗。逆遜遠投,奸普屏外。聖道如堤,崇崇海內。帝曰守文,是塞是親。調爾衛兵,程是烝民。民以盡力,臣以勤職。役云其終,河以之塞。唐堯懷山,實警神德。漢武宣防,實彰令式。我塞長河,融流惠澤。明明聖功,萬代成則。」太宗覽頌有「逆遜、奸普」之語,召宰相謂曰:「胡旦獻頌,詞意悖戾。朕自擢於甲科,歷試外任,所至無善狀。知海州日為部下所訟,猶已具,適會大赦,朕錄其材而舍其過,尚令在近列,又領史職,乃敢恣胸臆狂躁如此,其亟逐之!」
 即貶殿中丞、商州團練副使。



 上《平燕議》曰:



 今幽州在北門之外,東封非國家所急,願移其資以事北伐。且天時、地利、人事皆有可伐之意。歲之所臨,其地受福。今年春末至來年,歲在宋分,今年初秋至六年,鎮在燕分。從今年為備,至來春興師。北兵之遇春夏,則氈裘、皮履、羊弓、塞馬不為用,而中原士卒素不能寒,往北逢暄,筋力勇健。以勇健之士驅不用之敵,承福慶之時討災殃之城,成功立事,在於此矣。



 長淮以北,太行以東,河水罷災,土
 地甚沃。因其豐實,取其穀帛,減價以折納,見錢以貴糴,官府多積,兵役無虞,用兵豐財,可濟大事。



 太原克復以來,於今七載,兵甲甚利,士卒甚雄,夜寢晨興,寒裘饑粟。若以促裝之賜,發軍而用之,恩賞之貲,成功而賚之,可以齊心平敵,恢拓舊境。



 幽州平土而負敵,為勢必擇四人,分之方面,以剛斷勇毅者主之,選和平恭慎者一人部之。幽州之北,皆是山谷,通人馬者不過十處,領將士者亦擇十人,同行則共議兵機,分出則各司軍事,寇來
 則同戰以驅逐,寇歸則畫疆以捍蔽。茍塞斷山路,餘寇在燕與大軍相持,則遷延其時以度春夏,寇不能熱,有退無前。使士之剛勇才力者各為一將,多則分部捍敵攻城,兩盡其力。定其軍名,實其軍數。我寡彼多則力不勝,我實彼虛則勝有餘。力均則較其地形,地均則爭其謀略,分明勇怯,各致其用。



 以茶鹽香藥之價十分減二,從新者先賣於邊城要路、軍馬屯所。以芻粟錢帛之價十分增二,納貨以出券者詣本場以交貨,得貨者緣
 逐路以納稅。出往來四方之饒,為兩地費用之耗,自然商得其利,則買之於人,人得其資,則勤之於穡。故必民效兼倍之力,國貯九年之積,科撥不假於度支,轉般何勞於漕挽。芻粟之給,攻具之用,委輸發運,以為後繼。



 今將用二十萬之眾,役三十州之民,願陛下明降日月之信,先示雨露之澤。民知信賞則悅而忘死,士得仰給則死而力戰。如此則逆壘不足下,猾寇不足殄也。



 起為左補闕,復直史館。遷修撰,預修國史,以尚書戶部員外郎
 知制誥,遷司封員外郎。



 有傭書人翟穎者,旦嘗與之善,因為改姓名馬周,以為唐馬周復出,上書詆時政,且自薦可為大臣。又舉材任輔者十人,其辭頗壯。當時皆謂旦所為。馬周坐流海島,旦亦貶坊州團練副使。坐擅離所部謁宋白於鄜州,既被劾,特釋之。徙絳州。稍復工部員外郎、直集賢院,遷本曹郎中、知制誥、史館修撰。



 素善中官王繼恩,為繼恩草制辭過美。繼恩敗,真宗聞而惡之,貶安遠軍行軍司馬,又削籍流潯州。咸平初,移通
 州團練副使,徙徐州,以祠部員外郎分司西京,又為保信軍節度副使。久之,以司封員外郎通判襄州。封泰山,改祠部郎中,服母喪,既除,乃言父卒時嘗詔奪哀從事,請追行服三年。已而失明,以秘書省少監致仕,居襄州。再遷秘書監,卒。



 旦喜讀書,既喪明,猶令人誦經史,隱幾聽之不少輟。著《漢春秋》、《五代史略》、《將帥要略》、《演聖通論》、《唐乘》、《家傳》三百餘卷。斫大硯,方五六尺,刻而瘞之,曰「胡旦修《漢春秋》硯。」晚尤黷貨,干擾州縣,持吏短長,為時論
 所薄。既死,子孫貧甚,寓柩民間。皇祐末,知襄州王田為言於朝,得錢二十萬以葬。



 賈同字希得,青州臨淄人。五代時,楊光遠反,同祖崇率鄉里四百餘家保愚谷山,全活者二千人。同初名罔,字公疏,篤學好古,有時名,著《山東野錄》七篇。年四十餘,同進士出身,真宗命改今名。王欽若方貴盛,聞同名,欲致之,固謝不往。居八九年,始補歷城主簿。張知白薦為大理評事,通判兗州。



 天聖初,上書言:「自祥符以來,諫諍路
 塞,丁謂乘間造符瑞以欺先帝。今謂奸既白,宜明告天下,正符瑞之謬,罷宮觀崇奉,歸不急之衛兵,收無名之實費,使先帝免後世之議,國家無因循之失。」又言:「寇準忠規亮節,疾惡擯邪。自其貶黜,天下之人弗見其罪,宜還之內地,以明忠邪善惡之分。」時章獻太后臨朝,而同言如此,人以為難。



 再遷殿中丞、知棣州,卒。劉顏、李冠、王無忌及其門人謚同曰存道先生。



 劉顏字子望,彭城人。少孤,好古,學不專章句。師事高弁。
 舉進士第,以試秘書省校書郎知龍興縣,坐法免。久之,授徐州文學。居鄉里,教授數十百人。採漢、唐奏議為《輔弼名對》。馮元、劉筠、錢易、滕涉、蔡齊上其書,除任城主簿。歲饑,發大姓所積粟,活數千人。李迪知兗州、青州,皆闢為從事,卒。著《儒術通要》、《經濟樞言》復數十篇。石介見其書,嘆曰:「恨不在弟子之列。」子庠,自有傳。



 高弁字公儀,濮州雷澤人。弱冠,徒步從種放學於終南山,又學古文於柳開,與張景齊名。至道中,以文謁王
 禹偁,禹偁奇之。舉進士,累官侍御史。諫修玉清昭應宮,降知廣濟軍。尋以戶部判官試開封府進士,私發糊名,奪二官。稍復知單州、邢州、鹽鐵判官。河決澶州,請弛堤防,縱水所之,可省民力,且以扼契丹南向。議寢。知陜州,卒。



 弁性孝友。所為文章多祖《六經》及《孟子》,喜言仁義。有《帝則》三篇,為世所傳。與李迪、賈同、陸參、朱頔、伊淳相友善。石延年、劉潛皆其門人也。



 孫復,字明復,晉州平陽人。舉進士不第,退居泰山。學《春
 秋》,著《尊王發微》十二篇,大約本於陸淳,而增新意。



 石介有名山東,自介而下皆以先生事復。年四十不娶。李迪知其賢,以其弟之子妻之。復初猶豫,石介與諸弟子請曰:「公卿不下士久矣,今丞相不以先生貧賤,欲托以子,宜因以成丞相之賢名。」復乃聽。孔道輔聞復之賢,就見之,介執杖屨立侍復左右,升降拜則扶之,其往謝亦然。介既為學官,語人曰:「孫先生非隱者也。」於是範仲淹、富弼皆言復有經術,宜在朝廷。除秘書省校書郎、國子監
 直講。車駕幸太學,賜緋衣銀魚,召為邇英閣祗候說書。楊安國言其講說多異先儒,罷之。



 孔直溫敗,得所遺復詩,坐貶虔州監稅,徙泗州,又知長水縣,簽書應天府判官事。通判陵州,未行,翰林學士趙概等十餘人言復經為人師,不宜使佐州縣。留為直講,稍遷殿中丞,卒,賜錢十萬。



 復與胡瑗不合,在太學常相避。瑗治經不如復,而教養諸生過之。復既病,韓琦言於仁宗,選書吏,給紙筆,命其門人祖無擇就復家得書十五萬言,錄藏秘閣。特
 官其一子。



 石介,字守道,兗州奉符人。進士及第,歷鄆州、南京推官。篤學有志尚,樂善疾惡,喜聲名,遇事奮然敢為。御史臺闢為主簿,未至,以論赦書不當求五代及諸偽國後,罷為鎮南掌書記。代父丙遠官,為嘉州軍事判官。丁父母憂,耕徂徠山下,葬五世之未葬者七十喪。以《易》教授於家,魯人號介徂徠先生。入為國子監直講,學者從之甚眾,太學繇此益盛。



 介為文有氣,嘗患文章之弊,佛、老為
 蠹,著《怪說》、《中國論》,言去此三者,乃可以有為。又著《唐鑒》以戒奸臣、宦官、宮女,指切當時,無所諱忌。杜衍、韓琦薦,擢太子中允、直集賢院。會呂夷簡罷相,夏竦既除樞密使,復奪之,以衍代。章得象、晏殊、賈昌朝、範仲淹、富弼及琦同時執政,歐陽修、餘靖、王素、蔡襄並為諫官,介喜曰:「此盛事也,歌頌吾職,其可已乎」作《慶歷聖德詩》,曰:



 於惟慶歷三年三月,皇帝龍興,徐出闈闥。晨坐太極,晝開閶闔。躬覽英賢,手鉏奸枿。大聲渢々,震搖六合。如乾之動,
 如雷之發。昆蟲躑躅,怪妖藏滅。同明道初,天地嘉吉。



 初聞皇帝,蹙然言曰:「予祖予父,付予大業,予恐失墜,實賴輔弼。汝得象、殊,重慎微密。君相予久,予嘉君伐。君仍相予,竹鏞斯協。昌朝儒者,學問該洽。與予論政,傅以經術。汝貳二相,庶績咸秩。



 惟汝仲淹,汝誠予察。太后乘勢,湯沸火熱。汝時小臣,危言QJ摑。為予司諫,正予門闑。為予京兆,SW予讒說。賊叛予夏,往予式遏。六月酷日,大冬積雪。汝寒汝暑,同予士卒。予聞辛酸,汝不告乏。予晚得弼,
 予心弼悅。弼每見予,無有私謁。以道輔予,弼言深切。予不堯、舜,弼自笞罰。諫官一年,疏奏滿篋。侍從周歲,忠力廑竭。契丹忘義,檮杌饕餮。敢侮大國,其辭慢悖。弼將予命,不畏不怯。卒復舊好,民得食褐。沙磧萬里,死生一節。視弼之膚,霜剝風裂。觀弼之心,煉金鍛鐵。寵名大官,以酬勞渴。弼辭不受,其志莫奪。惟仲淹、弼,一夔一契。天實賚予,予其敢忽。並來弼予,民無瘥札。



 曰衍汝來,汝予黃發。事予二紀,毛禿齒豁。心如一兮,率履弗越。遂長樞府,
 兵政無蹶。予早識琦,琦有奇骨。其器魁落,豈視扂楔。其人渾樸,不施剞劂。可屬大事,敦厚如勃。琦汝副衍,知人予哲。



 惟修惟靖,立朝䡾䡾。言論磥砢,忠誠特達。祿微身賤,其志不怯。嘗詆大官,亟遭貶黜。萬里歸來,剛氣不折。屢進直言,以補予闕。素相之後,含忠履潔。昔為御史,幾叩予榻。襄雖小官,名聞予徹。亦嘗獻言,箴予之失。剛守粹愨,與修儔匹。並為諫官,正色在列。予過汝言,毋鉗汝舌。」



 皇帝聖明,忠邪辨別。舉擢俊良,掃除妖魃。眾賢之進,
 如茅斯拔。大奸之去,如距斯脫。上倚輔弼,司予調燮。下賴諫諍,維予紀法。左右正人,無有邪孽。予望太平,日不逾浹。



 皇帝嗣位,二十二年。神武不殺,其默如淵。聖人不測,其動如天。賞罰在予,不失其權。恭己南面,退奸進賢。知賢不易,非明弗得。去邪惟艱,惟斷乃克。明則不貳,斷則不惑。既明且斷,惟皇帝之德。



 群臣踧踖,重足屏息,交相教語:曰惟正直,毋作側僻,皇帝汝殛。諸侯危慄,墜玉失舄,交相告語:皇帝神明,四時朝覲,謹修臣職。四夷走
 馬,墜鐙遺策,交相告語:皇帝英武,解兵修貢,永為屬國。皇帝一舉,群臣懾焉,諸侯畏焉,四夷服焉。



 臣願皇帝,壽萬千年。



 詩所稱多一時名臣,其言大奸,蓋斥竦也。詩且出,孫復曰:「子禍始於此矣。」



 介不畜馬,借馬而乘,出入大臣之門,頗招賓客,預政事,人多指目。不自安,求出,通判濮州,未赴,卒。



 會徐狂人孔直溫謀反,搜其家,得介書。夏竦銜介甚,且欲中傷杜衍等,因言介詐死,北走契丹,請發棺以驗。詔下京東訪其存亡。衍時在兗州,以驗介事
 語官屬,眾不敢答,掌書記龔鼎臣願以闔族保介必死,衍探懷出奏稿示之,曰:「老夫已保介矣。君年少,見義必為,豈可量哉。」提點刑獄呂居簡亦曰:「發棺空,介果走北,孥戮非酷。不然,是國家無故剖人塚墓,何以示後世?且介死必有親族門生會葬及棺斂之人,茍召問無異,即令具軍令狀保之,亦足應詔。」於是眾數百保介已死,乃免斫棺。子弟羈管他州,久之得還。



 介家故貧,妻子幾凍餒,富弼、韓琦共分奉買田以贍養之。有《徂徠集》行於世。



 胡瑗,字翼之,泰州海陵人。以經術教授吳中,年四十餘。景祐初,更定雅樂,詔求知音者。範仲淹薦瑗,白衣對崇政殿。與鎮東軍節度推官阮逸同較鐘律,分造鐘磬各一虡。以一黍之廣為分,以制尺,律徑三分四厘六毫四絲,圍十分三厘九毫三絲。又以大黍累尺,小黍實龠。丁度等以為非古制,罷之,授瑗試秘書省校書郎。範仲淹經略陜西,闢丹州推官。以保寧節度推官教授湖州。瑗教人有法,科條纖悉備具,以身先之。雖盛暑,必公服坐
 堂上,嚴師弟子之禮。視諸生如其子弟,諸生亦信愛如其父兄,從之游者常數百人。慶歷中,興太學,下湖州取其法,著為令。召為諸王宮教授,辭疾不行。為太子中舍,以殿中丞致仕。



 皇祐中,更鑄太常鐘磬,驛召瑗、逸,與近臣、太常官議於秘閣,遂典作樂事。復以大理評事兼太常寺主簿,辭不就。歲餘,授光祿寺丞、國子監直講。樂成,遷大理寺丞,賜緋衣銀魚。瑗既居太學,其徒益眾,太學至不能容,取旁官舍處之。禮部所得士,瑗弟子十常居
 四五,隨材高下,喜自修飭,衣服容止,往往相類,人遇之雖不識,皆知其瑗弟子也。嘉祐初,擢太子中允、天章閣侍講,仍治太學。既而疾不能朝,以太常博士致仕,歸老於家。諸生與朝士祖餞東門外,時以為榮。既卒,詔賻其家。



 劉羲叟,字仲更,澤州晉城人。歐陽修使河東,薦其學術。試大理評事,權趙州軍事判官。精算術,兼通《大衍》諸歷。及修唐史,令專修《律歷》、《天文》、《五行志》。尋為編修官,改秘
 書省著作佐郎。以母喪去,詔令家居編修。書成,擢崇文院檢討,未入謝,疽發背卒。



 羲叟強記多識,尤長於星歷、術數。皇祐五年,日食心,時胡瑗鑄鐘弇而直,聲鬱不發。又陜西鑄大錢,羲叟曰:「此所謂害金再興,與周景王同占,上將感心腹之疾。」其後仁宗果不豫。又月入太微,曰:「後宮當有喪。」已而張貴妃薨。至和元年,日食正陽,客星出於昴,曰:「契丹宗真其死乎?」事皆驗。羲叟未病,嘗曰:「吾及秋必死。」自擇地於父塚旁,占庚穴,以語其妻,如其言
 葬之。著《十三代史志》、《劉氏輯歷》、《春秋災異》諸書。



 林概,字端父,福州福清人。父高,太常博士,有治行。概幼警悟,舉進士,以秘書省校書郎知長興縣。歲大饑,富人閉糴以邀價,概出奉粟庭下,誘士豪輸數千石以飼饑者。



 知連州。康定初,上封事曰:「古者民為兵,而今兵食民。古馬寓於民,而今不習馬。此兵與馬之大患也。請附唐府兵之法,四斂一民,部以為軍,閑耕田里,被甲皆兵。因命其家咸得畜馬,私乘休暇,官為調習。則人便干戈,馬
 識行列。又行陣無法,而出於臨時;將無素備,而取於倉卒;軍不予權,而監以宦侍:若是者,雖得古之材,使循今之法,亦必屢戰而屢敗。」又請備蠻,籍土民為兵,柵要沖,購徭人使守御。徙淮安軍。



 程琳嘗禁蜀人不得自為渠堰,概奏罷之。又言蜀饑,願罷川峽漕,發常平粟貸民租,募富人輕粟價,除商旅之禁,使通貨相資。官至太常博士、集賢校理,卒。著《史論》、《辨國語》。子希,自有傳。



 李覯,字泰伯,建昌軍南城人。俊辯能文,舉茂才異等不
 中。親老,以教授自資,學者常數十百人。皇祐初,範仲淹薦為試太學助教,上《明堂定制圖序》曰:



 《考工記》「周人明堂,度九尺之筵」,是言堂基修廣,非謂立室之數。「東西九筵,南北七筵,堂崇一筵」,是言堂上,非謂室中。東西之堂各深四筵半,南北之堂各深三筵半。「五室,凡室二筵」,是言四堂中央有方十筵之地,自東至西可營五室,自南至北可營五室。十筵中央方二筵之地,既為太室,連作餘室,則不能令十二位各直其辰,當於東南西北四面
 及四角缺處,各虛方二筵之地,周而通之,以為太廟。太室正居中,《月令》所謂「中央土」、「居太廟太室」者,言此太廟之中有太室也。太廟之外,當子、午、卯、酉四位上各畫方二筵地,以與太廟相通,為青陽、明堂、總章、元堂四太廟;當寅、申、巳、亥、辰、戌、丑、未八位上各畫方二筵地,以為左個、右個也。



 《大戴禮。盛德記》:「明堂凡九室,室四戶八牖,共三十六戶、七十二牖。」八個之室,並太室而九,室四面各有戶,戶旁夾兩牖也。



 《白虎通》:「明堂上圓下方,八窗、四闥、
 九室、十二坐。」四太廟前各為一門,出於堂上,門旁夾兩窗也。左右之個其實皆室,但以分處左右,形如夾房,故有個名。太廟之內以及太室,其實祀文王配上帝之位,謂之廟者,義當然矣。土者分王四時,於五行最尊,故天子當其時居太室,用祭天地之位以尊嚴之也。四仲之月,各得一時之中,與餘月有異。故復於子、午、卯、酉之方,取二筵地,假太廟之名以聽朔也。



 《周禮》言基而不及室,《大戴》言室而不及廟,稽之《月令》則備矣,然非《白虎通》,亦
 無以知窗闥之制也。聶崇義所謂秦人《明堂圖》者,其制有十二階,古之遺法,當亦取之。



 《禮記外傳》曰:「明堂四面各五門。」今按《明堂位》:四夷之國,四門之外。九採之國,應門之外。時天子負斧扆南向而立。南門之外者北面東上,應門之外者亦北面東上,是南門之外有應門也。既有應門,則不得不有皋、庫、雉門。明堂者,四時所居,四面如一,南面既有五門,則餘三面皆各有五門。鄭注《明堂位》則云「正門謂之應門」,其意當謂變南門之文以為應
 門。又見王宮有路門,其次乃有應門。今明堂無路門之名,而但有應門,便謂更無重門,而南門即是應門。且路寢之前則名路門,其次有應門。明堂非路寢,乃變其內門之名為東門南門,而次有應門,何害於義?四夷之君,既在四門之外,而外無重門,則是列於郊野道路之間,豈朝會之儀乎?王宮常居,猶設五門,以限中外。明堂者,效天法地,尊祖配帝,而止一門以表之,豈為稱哉!



 若其建置之所,則淳于登云「在國之陽,三里之外,七里之內,
 丙巳之地」。《玉藻》「聽朔於南門之外」,康成之注亦與是合。夫稱明也,宜在國之陽。事天神也,宜在城門之外。



 今圖以九分當九尺之筵,東西之堂共九筵,南北之堂共七筵。中央之地自東至西凡五室,自南至北凡五室,每室二筵,取於《考工記》也。一太室、八左右個,共九室,室有四戶、八牖,共三十六戶、七十二牖,協於戴德《記》也。九室四廟,共十三位,本於《月令》也。四廟之面,各為一門,門夾兩窗,是為八窗四闥,稽於《白虎通》也。十二階,採於《三禮圖》
 也。四面各五門,酌於《明堂位》、《禮記外傳》也。



 嘉祐中,用國子監奏,召為海門主簿、太學說書而卒。覯嘗著《周禮致太平論》、《平土書》、《禮論》。門人鄧潤甫,熙寧中,上其《退居類稿》、《皇祐續稿》並《後集》,請官其子參魯,詔以為郊社齋郎。



 何涉,字濟川,南充人。父,祖皆業農,涉始讀書,晝夜刻苦,泛覽博古。上自《六經》、諸子百家,旁及山經、地志、醫卜之術,無所不學,一過目不復再讀,而終身不忘。人問書傳中事,必指卷第冊葉所在,驗之果然。



 登進士第,調洛交
 主簿,改中部令。範仲淹一見奇之,闢彰武軍節度推官。用龐籍奏,遷著作佐郎、管勾鄜延等路經略安撫招討司機宜文字。時元昊擾邊,軍中經畫,涉預有力。元昊納款,籍召為樞密使,欲與之俱,涉曰:「親老矣,非人子自便之時。」拜章願得歸養,特改秘書丞、通判眉州,徙嘉州。用文彥博、龐籍薦,召還,除集賢校理。既又求歸蜀,遂得知漢州。歲滿,移合州。累官尚書司封員外郎。父喪,罷歸,卒。詔恤其家,並官其一子。



 涉長厚有操行,事親至孝,平居
 未嘗談人過惡。所至多建學館,勸誨諸生,從之游者甚眾。雖在軍中,亦嘗為諸將講《左氏春秋》,狄青之徒皆橫經以聽。有《治道中術》、《春秋本旨》、《廬江集》七十卷。



 王回,字深父,福州候官人。父平言,試御史。回敦行孝友,質直平恕,造次必稽古人所為,而不為小廉曲謹以求名譽。嘗舉進士中第,為衛真簿,有所不合,稱病自免。



 作《告友》曰:



 古之言天下達道者,曰君臣也,父子也。夫婦也,兄弟也,朋友也。五者各以其義行而人倫立,其義廢則
 人倫亦從而亡矣。



 然而父子兄弟之親,天性之自然者也;夫婦之合,以人情而然者也;君臣之從,以眾心而然者也。是雖欲自廢,而理勢持之,何能斬也。惟朋友者,舉天下之人莫不可同,亦舉天下之人莫不可異,同異在我,則義安所卒歸乎?是其漸廢之所繇也。



 君之於臣也,父之於子也,夫之於婦也,兄之於弟也,過且惡,必亂敗其國家,國家敗而皆受其難,被其名,而終身不可辭也。故其為上者不敢不誨,為下者不敢不諫。世治道行,則
 人能循義而自得;世衰道微,則人猶顧義而自全。間有不若,則亦無害於眾焉耳。此所謂理勢持之,雖百代可知也。



 親非天性也,合非人情也,從非眾心也,群而同,別而異,有善不足與榮,有惡不足與辱。大道之行,公與義者可至焉,下斯而言,其能及者鮮矣。是以聖人崇之,以列於君臣、父子、夫婦、兄弟而壹為達道也。聖人既沒,而其義益廢,於今則亡矣。



 夫人有四肢,所以成身;一體不備,則謂之廢疾。而人倫缺焉,何以為世?嗚呼,處今之時
 而望古之道,難矣。姑求其肯告吾過也,而樂聞其過者,與之友乎!



 退居潁州,久之不肯仕,在廷多薦者。治平中,以為忠武軍節度推官、知南頓縣,命下而卒。回在潁川,與處士常秩友善。熙寧中,秩上其文集,補回子汾為郊社齋郎。弟向。



 向字子直,為文長於序事,戲作《公默先生傳》曰:



 公議先生剛直任氣,好議論,取當世是非辨明。游梁、宋間,不得意。去居潁,其徒從者百人。居二年,與其徒謀,又去潁。弟
 子任意對曰:「先生無復念去也,弟子從先生久矣,亦各厭行役。先生舍潁為居廬,少有生計。主人公賢,遇先生不淺薄,今又去之,弟子未見先生止處也。先生豈薄潁邪?」



 公議先生曰:「來,吾語爾!君子貴行道信於世,不信貴容,不容貴去,古之闢世、闢地、闢色、闢言是也。吾行年三十,立節循名,被服先王,究窮《六經》。頑鈍晚成,所得無幾。張羅大綱,漏略零細。校其所見,未為完人。豈敢自忘,冀用於世?予所厭苦,正謂不容。予行世間,波混流同。予譽
 不至,予毀日隆。小人鑿空,造事形跡;侵排萬端,地隘天側。《詩》不云乎,『讒人罔極』。主人明恕,故未見疑。不幸去我,來者謂誰?讒一日效,我終顛危。智者利身,遠害全德,不如亟行,以適異國。」



 語已,任意對曰:「先生無言也。意輩弟子嘗竊論先生樂取怨憎,為人所難,不知不樂也。今定不樂,先生知所以取之乎?先生聰明才能,過人遠甚,而刺口論世事,立是立非,其間不容毫發。又以公議名,此人之怨府也。《傳》曰:『議人者不得其死』,先生憂之是也,其
 去未是。意有三事為先生計,先生幸聽意,不必行;不聽,先生雖去絕海,未見先生安也。」



 公議先生強舌不語,下視任意,目不轉移時,卒問任意,對曰:「人之肺肝,安得可視,高出重泉,險不足比。聞善於彼,陽譽陰非,反背復憎,詆笑縱橫。得其細過,聲張口播,緣飾百端,得敗行破。自然是人,賤彼善我。意策之三,此為最上者也。先生能用之乎?」公議先生曰:「不能,爾試言其次者。」對曰:「捐棄骨肉,佯狂而去,令世人不復顧忌。此策之次者,先生能用之
 乎?」公議先生曰:「不能,爾試言其又次者。」對曰:「先生之行己,視世人所不逮何等也!曾未得稱高世,而詆訶鋒起,幾不得與妄庸人伍者,良以口禍也。先生能不好議而好默,是非不及口而心存焉,何疾於不容?此策之最下者也,先生能用之乎?」公議先生喟然嘆曰:「籲,吾為爾用下策也。」



 任意乃大笑,顧其徒曰:「宜吾先生之病於世也。吾三策之,卒取其下者矣。」弟子陽思曰:「今日非任意,先生不可得留。」與其徒謝意,更因意請,去公議為公默先
 生。



 弟同,字容季。性純篤,亦善序事。皆早卒。仕止於縣主簿。



 周堯卿,字子俞,道州永明人。警悟強記,以學行知名。天聖二年舉進士,歷連、衡二州司理參軍、桂州司錄。知高安、寧化二縣,提點刑獄紘入境,有被刑而耘苗者,紘就詢其故,對曰:「貧以利故,為人直其枉,令不我欺而我欺之,我又何怨?」紘至縣,以所聞薦之。後通判饒州,積官至太常博士。範仲淹薦經行可為師表,未及用,以慶歷
 五年卒,年五十一。



 始,堯卿年十二喪父,憂戚如成人,見母則抑情忍哀,不欲傷其意。母知而異之,謂族人曰:「是兒愛我如此,多知孝養矣。」卒能如母之言。及母喪,倚廬三年,席薪枕塊,雖疾病,不飲酒食肉。既葬,慈烏百數銜土集隴上,人以為孝感所致。其於昆弟尤篤友愛。又為人簡重不校,有慢己者,必厚為禮以愧之。居官祿雖薄,必以周宗族朋友,罄而後已。



 為學不專於傳注,問辨思索,以通為期。長於毛、鄭《詩》及《左氏春秋》。其學《詩》,以孔子
 所謂「《詩》三百,一言以蔽之曰:『思無邪』」,孟子所謂「說《詩》者以意逆志,是為得之」,考經指歸,而見毛、鄭之得失。曰:「毛之傳欲簡,或寡於義理,非一言以蔽之也。鄭之箋欲詳,或遠於性情,非以意逆志也。是可以無去取乎?」其學《春秋》,由左氏記之詳,得經之所以書者,至《三傳》之異同,均有所不取。曰:「聖人之意豈二致耶?」讀莊周、孟子之書,曰:「周善言理,未至於窮理。窮理,則好惡不繆於聖人,孟軻是已。孟善言性,未至於盡己之性。能盡己之性,則能盡
 物之性,而可與天地參,其唯聖人乎。天何言哉?性與天道,子貢所以不可得而聞也。昔宰我、子貢善為說辭,冉牛、閔子、顏淵善言德行,孔子曰:『我於辭命,則不能也。』惟不言,故曰不能而已,蓋言生於不足者也。」其講解議論皆若是。



 有《詩》、《春秋說》各三十卷,文集二十卷。七子:諭,鼎州司理參軍;詵,湖州歸安主簿;謚、諷、諲、說、誼。



 王當,字子思,眉州眉山人。幼好學,博覽古今,所取惟王佐大略。嘗謂三公論道經邦,燮理陰陽,填撫四方,親附
 百姓,皆出於一道,其言之雖大,其行之甚易。嘗舉進士不中,退居田野,嘆曰:「士之居世,茍不見其用,必見其言。」遂著《春秋列國名臣傳》五十卷,人競傳之。



 元祐中,蘇轍以賢良方正薦。廷對慷慨,不避權貴,策入四等。調龍游縣尉。蔡京知成都,舉為學官,當不就。其後京相,當遂不復仕,卒,年七十二。當於經學尤邃《易》與《春秋》,皆為之傳,得聖人之旨居多。又有《經旨》三卷,《史論》十二卷,《兵書》十二篇。



 陳暘字晉之,福州人。中紹聖制科,授順昌軍節度推官。徽宗初,進《迓衡集》以勸導紹述,得太學博士、秘書省正字。禮部侍郎趙挺之言,暘所著《樂書》二十卷貫穿明備,乞援其兄祥道進《禮書》故事給札。既上,遷太常丞,進駕部員外郎,為講議司參詳禮樂官。



 魏漢津議樂,用京房二變四清。暘曰:「五聲十二律,樂之正也。二變四清,樂之蠹也。二變以變宮為君,四清以黃鐘清為君。事以時作,固可變也,而君不可變。太簇、大呂、夾鐘,或可分也,而黃
 鐘不可分。豈古人所謂尊無二上之旨哉?」時論方右漢津,絀暘議。



 進鴻臚太常少卿、禮部侍郎,以顯謨閣待制提舉醴泉觀。嘗坐事奪,已而復之。卒,年六十八。



 祥道字用之。元祐中,為太常博士,終秘書省正字。所著《禮書》一百五十卷,與暘《樂書》並行於世。



\end{pinyinscope}