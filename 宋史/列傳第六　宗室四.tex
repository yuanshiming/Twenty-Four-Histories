\article{列傳第六 宗室四}

\begin{pinyinscope}

 子淔子崧子櫟子砥子晝子潚師希言希懌士珸士人褭士□
 穹士□士群不棄不尤不│善俊善譽汝述叔近叔向彥倓彥橚彥逾



 子淔字正之,燕王五世孫。父令鑠,官至寶文閣待制。子淔以蔭補承務郎,累遷少府監主簿,改河南少尹。



 時治西內,子淔有幹才,漕使宋忭器之。或事有未便,子淔輒力爭,忭每改容謝之。除蔡河撥發綱運官。會夏旱,河水
 涸,轉餉後期,貶秩一級。提舉三門、白波輦運事,除直秘閣。丁內艱,起復。累進龍圖閣、秘閣修撰,除陜西轉運副使。



 初,蔡京鑄夾錫錢,民病壅滯,子淔請鑄小鐵錢以權之,因範格以進。徽宗大說,御書「宣和通寶」四字為錢文。既成,子淔奏令民以舊銅錢入官,易新鐵錢。旬日,易得百餘萬緡。帝手札以新錢百萬緡付五路,均糴細麥,命子淔領其事。民苦限迫,詣子淔訴者日數百人,子淔奏請寬其期,民便之。會蔡京再相,言者希京意,論子淔亂
 錢法,落職奉祠。



 靖康初,復秘閣修撰。金人侵洛,子淔奔荊南。潰兵祝靖、盛德破荊南城,子淔匿民家,靖等知之,來謁,言京城已破。子淔泣,說之曰「君輩宜亟還都城,護社稷,取功名,無貪財擾州縣也。」皆應曰:「諾。」子淔因草檄趣之。翌日,靖等遂北行。



 紹興元年,召見,復徽猷閣直學士、知西外宗正司,改江西都轉運使。時建督府,軍須浩繁,子淔運餉不絕,以功進寶文閣直學士,再知西外宗正司。三京新復,除京畿都轉運使,以疾辭。卒於家,年六
 十七。



 子淔幼警悟,蘇軾過其家,抱置膝上,謂其父曰:「此公家千里駒也。」及長,善談論,工詩。然崇寧、大觀間土木繁興,子淔每董其役,識者鄙之。



 子崧字伯山,燕懿王後五世孫。登崇寧五年進士第。宣和間,官至宗正少卿,除徽猷閣直學士、知淮寧府。



 汴京失守,起兵勤王,道阻未得進。聞張邦昌僭位,以書白康王:宜遣師邀金人河上,迎請兩宮,問罪僭逆,若議渡江,恐誤大計。遂與知穎昌府何志同等盟,傳檄中外。已而
 聞金人退,引兵襄邑,遣範塤、徐文中詣濟州,請王進兵南京,且言:「國家之制,無視王在外者,主上特付大王以元帥之權,此殆天意。亟宜承制號召四方豪傑,則中原可傳檄而定。」王命子崧充大元帥府參議官、東南道都總管。邦昌家在廬州,子崧檄通守趙令儦幾察之,且請捕誅其母子,以絕奸心。



 又言:「自圍城以來,朝命隔絕,乞下諸路,凡有事宜,並取大元帥府裁決,偽檄毋輒行。宣撫使範訥逗撓營私,所宜加罪。宜蠲被兵州縣租,經理
 淮南、荊、浙形勢之地,毋為群盜所據。」



 檄止諸路毋受邦昌偽赦,移書責邦昌曰:「人臣當見危致命,今議者籍籍,謂劫請傾危之計實由閣下,不然,金人何堅拒孫傅之請,而卒歸於閣下也。敵既遠去,宜速反正,若少遲疑,則天下共誅逆節,雖悔無及矣。」又遺書王時雍曰:「諸公相與亡人之國,方且以為佐命功臣,不知平日所學何事。」



 會邦昌遣使迎王次第白子崧,子崧即貽王書曰:「似聞謂以京師殘破,不可復入,止欲即位軍中,便圖遷徙,臣
 竊惑焉。夫欲致中興,當謹舉措,宜先謁宗廟,覲母後,明正誅賞,降霈四方。若京師果不可都,然後徐議所向。」



 遂傳檄京師,奏於隆祐太后曰:「諸路先聞二聖北遷,易姓改國,恐間有假討逆之名,以竊據州郡者。乞速下明詔,諭四方以迎立康王之意,庶幾人心慰安,奸宄自消矣。」尋以所部兵會濟州。



 康王即位,子崧請放諸路常平積欠錢,又言:「臺諫為人主耳目,近年用非其人,率取旨言事。請尊舊制,聽學士、中丞互舉。範祖禹、常安民、上官均
 先朝言事盡忠,請錄其子。」帝皆可其奏。因建三屯之議:一屯澶淵,一屯河中、陜、華,一屯青、鄆間,以張聲勢。萬一敵騎南侵,則三道並進,可成大功。



 除延康殿學士、知鎮江府、兩浙路兵馬鈐轄。上章論王時雍、徐秉哲、吳開、莫儔、範瓊、胡思、王紹、王及之、顏博文、余大均等逼遷上皇,取太子,辱六宮,捕宗室,竊禁物,都人指為國賊。伏望肆諸市朝,以為臣子之戒。時滑州兩經殘破,子崧薦傅亮可任。除亮滑州通判,黃潛善沮之,命遂寢。



 賊趙萬犯鎮
 江,子崧遣將擊萬於丹徒,調鄉兵乘城為備。頃之,官軍敗歸,鄉兵驚潰,子崧率親兵保焦山寺,賊據鎮江。



 初,昌陵復土,司天監苗昌裔謂人曰:「太祖後當再有天下。」子崧習聞其說,靖康末起兵,檄文頗涉不遜。子崧與御營統制辛道宗有隙,道示求得其文,上之。詔年史往案其獄,情得,帝震怒,不欲暴其罪,坐以前擅棄城,降單州團練副使,謫居南雄州。紹興二年赦,復集英殿修撰,而子崧已卒於貶所。



 子櫟,燕懿王後五世孫。登元祐六年進士第。靖康中,為汝州太守。金人再渝盟,破荊湖諸州,獨子櫟能保境土。李綱言於朝,遷寶文閣直學士,尋提舉萬壽觀。紹興七年卒。



 子砥,藝祖後令珦之子也。仕至鴻臚丞。北遷至燕山,久之,欲遁歸,乃遣其徒朱國賓、王孝安至中京,求得上皇宸翰,懷之以歸。建炎二年六月,至行在,帝命輔臣召問於都堂。子砥言:「金人講和以用兵,我國斂兵以待和。往
 者契存丹主和議,女真主用兵,十餘年間竟滅契丹。今復蹈其轍。譬人畏虎,啖虎以肉,食盡終必食人。若設陷阱待之,庶能制虎。」因復故官。已而賜對稱旨,命知臺州,卒。



 子晝字叔問,燕王五世孫。少警敏強記,工書翰。累官憲州通判。宣和初,充詳定《九域圖志》編修官。出知澤州,改密州。詔為刑部員外郎,以憂去。



 建炎四年,遷吏部員外郎。尋用大宗正士人褭薦,遷尚書左司員外郎,兼權貨務,歲收茶、鹽、香錢六百九萬餘緡,以功進秩一階。試太常
 少卿,集《太常因革禮》八十篇,為二十七卷。上言復春分祀高禖禮。除權禮部侍郎,遷徽猷待制、樞密都承旨。以公族為侍從,及改官制後都承旨用文臣,皆自子晝始。



 衢、嚴、信、饒之民,生子多不舉,子晝請禁絕之。累求補外,遷徽猷閣直學士、知秀州。既而奉祠以歸,寓於衢。紹興十二年卒,年五十四。



 子潚字清卿,秦康惠王後,孝靖公令奧之子也。七歲而孤,家貧力學。登宣和中進士第。調真州刑曹掾,與守爭
 獄事,解官去。改衢州推官。胡唐老奇其才,任之。屬時多故,子潚佐唐老繕完城具,苗、劉兵至城下,不能攻,以功進一秩。累官吏部郎中,求補外,遷戶部郎中,總領江、淮軍馬錢糧。諸司饋禮,月以千緡,悉歸之公帑。除直秘閣、兩浙轉運副使。朝廷遣人檢沙田蘆場,欲概增租額,子潚以承買異冒占,力止之。



 時議者言:田之並太湖者被水患,宜分道諸浦注之江。詔子潚往案視。還言:「太湖當數州巨浸,豈松江一川所能獨洩。昔人於常熟北開浦
 二十四以達大江,又開浦十於昆山東南以入海,今皆湮塞,宜加疏浚。」從之。遂浚常熟東柵至雉浦入於涇谷;又疏鑿福山塘,至尚市橋北注大江,分殺其勢,水患用息。



 明州守趙善繼治郡殘酷,子潚率諸監司劾罷之。除直敷文閣、知臨安府,吏不能欺,禁權家僦人子女為僕妾者。詔權戶部侍郎,升敷文閣待制,復知臨安府。調三衙卒修築都城,不擾而辦。金主亮渝盟,子潚獻助軍一十五萬錢,特遷一秩。帝幸建康,充行宮留守參謀官。扈
 蹕還,復知臨安府。金人來議和,子,子潚謂事情叵測,宜以軍禮待之。



 孝宗嗣位,志圖恢復,子潚練兵,習為「鵝鸛魚麗陣」,上觀於便殿,嘉之,賜金帶。擢敷文閣直學士,移知明州、沿海制置使。臺諫王十朋、王大寶抗疏留之,帝曰:「朕委以防海,行召還矣。」初,海寇以賂通郡胥吏,吏反為之用,匿其蹤跡,賊遂大熾,商舶不通。子潚以禮延土豪,俾率郡胥分道入海,告之曰:「用命者有厚賞,不則殺無貸。」胥眾震恐,爭指賊處,悉禽獲。凡豪猾為賊囊橐者,窮
 治之,海道遂平。



 升龍圖閣直學士、知福州。歲饑,告糴旁郡,米價頓平,民賴以濟,進龍圖閣學士,移知泉州。吏有掠民女為妾者,其妻妒悍,殺而磔之,貯以缶,抵其兄興化掾,安廨中。妾父每日郡訴,吏不決。子潚訪知狀,丞遣人往興化,果得缶以歸,獄遂決。其發手適概類此。乾道二年卒於官,年六十六。



 師字從善,系出燕懿王。王生彰化軍節度使惟忠,惟忠生宣城侯從謹,從謹生崇國公世恬,世恬生嘉國公
 唆令。中興初,韓世清挾令唆為變,裂黃旗被其身,固拒獲免。令唆生朝奉郎子笈,子笈生和州防禦使伯□肅。伯□肅少從高宗於康邸,以文藝侍左右。



 師,伯□肅之子也。舉進士第,除司農簿,遷金部郎中。孝宗奇其才,顧遇頗厚。師奏:左右曹、度支、倉部宜立總計,司歸並財物之數,以絕吏奸。制可。知吉州,即山煉銅,足冶欠額二十萬。進戶部郎官、淮東總領。



 光宗初,擢太府少卿、知秀州,改淮南運判。時郡鐵錢不行,鹽商弗至,師請發度牒,出
 倉粟,以收鐵錢,鹽利遂通。累遷司農卿、知臨安府。有僧號散聖者,以妖術惑眾,師𢍰捕治黥之。



 韓侂冑用事,師𢍰附之,遂得尹京。侂冑生日,百官爭貢珍異,師𢍰最後至,出小合曰:「願獻少果核侑觴。」啟之,乃粟金蒲萄小架,上綴大珠百餘,眾慚沮。侂冑有愛妾十四人,或獻北珠冠四枚於侂冑,侂冑以遺四妾,其十人亦欲之,侂冑未有以應也。師𢍰聞之,亟出錢十萬緡市北珠,制十冠以獻。妾為求遷官,得轉工部侍郎。侂冑嘗飲南園,過山莊,
 顧竹籬茅舍,謂師:「此真田舍間氣象,但欠犬吠雞鳴耳。」俄聞犬嗥從薄間,視之乃師也,侂冑大笑久之。以工部尚書知臨安府。



 侂冑將用兵,師度侂冑材疏意廣,必召禍,乃持異論,侍御史鄭友龍劾罷之。侂冑死,其黨多坐謫,以師嘗與侂冑異,故獲用。除寶謨閣直學士、知鎮江府。



 會荊湖始置制閫,以命師,給事中蔡幼學繳其命,遂罷歸。未幾,詔為兵部尚書、知臨安府。幼學時為學士,亦不草詔,留元剛草之。時楮輕糴貴,師𢍰
 尹京未數月,楮價浸昂,糴亦稍平,執政愈益賢之。會武學士柯子沖、盧宣德以事至府,師𢍰擅撻遣之,眾盡讙,文武二學之士交投牒,師𢍰乃罷免,與祠。卒於家,年七十。



 師四尹臨安,有能聲。嘗鉤致民罪,沒其家貲,謅事權貴,人以是鄙之。



 希言字若訥,惠王令元孫也。淳熙十四年登第。調衢州司戶,合郡民以計,表其坊,標其戶數,為圖獻於守,守才之。西安令不職,守檄希言攝邑。漕善令,會嚴州請
 復烏龍嶺稅場,檄希言往訪之,俾令得復職。希言力陳烏龍場不當復,漕怒曰:「衢已復孔步、章戴二場,何烏龍獨不可復?」希言謂二場當並罷去,漕不能奪,二場竟亦廢。改吉州司理,屬邑有誣人以殺人罪者,吏治之急,囚誣服。希言鞫得實,檄縣他捕,乃得真盜。



 用楊萬里、周必大薦,授臨安府司法,改淮西總所乾辦。移書約諸郡:綱必時發,至即受納,無滯留。始至,軍庫見錢不滿千緡,比去,庫錢充溢。



 知臨安仁和縣。闢學宮四百餘畝。適大旱,
 蝗集御前蘆場中,互數里。希言欲去蘆以除害,中使沮其策,希言驅卒燔之。臨平塘堤決,希言督役,親捧土投石,兵民爭奪,堤成,因築重堤,後不復決。民病和買絹折錢重,希言節公費,代其輸。



 除太社令,遷樞密院編修官兼右司。上言:「諸將但務城守,敵來不拒,去不復追,異時之憂,殆不止保江而已。宜諭諸將,一軍受圍,諸軍共守,敵不渡淮則均受賞,以戰為守,毋以守為守。」遷宗正丞,請南班得與輸對,許之。累遷秘書丞、著作郎、軍器少監,
 皆兼右司,又充密院檢詳,為宰屬、樞掾凡六年,奉祠去。嘉定十七年卒,年六十一。贈資政殿大學士,封越國公,謚忠憲。



 子與權,登進士第,再中刑法科。官至開府儀同三司。



 希懌字伯和,燕王八世孫。登淳熙十四年進士第。趙汝愚帥福建,希懌為屬吏,嘗言:治人如修身,治政如理家,愛民如處昆弟。取古今官著惠愛者緝為一編,曰:「是吾師矣。」汝愚嘉之,薦於憲辛棄疾。棄疾尚氣,僚吏不敢與
 可否,希懌獨盡言無所避。屬邑候官苦稅重,每不登額,希懌稽核公帑羨錢以足之。棄疾亦薦其能。汝愚當國,調江東運司干辦。



 同寅有坐侂冑黨者,諸司莫敢薦,希懌賢其人,請以薦己者薦之。改太平州通判。先是盜黥而逃者,捕得處死。希懌言:「強盜特貸命而輒逃者斬,今黥罪致死,非法之平也。」自是皆減死論。



 遷江西茶鹽提舉。歲饑,惡少聚劫,希懌將自臨按,幕屬力止之,不聽,曰:「希懌不出,饑民終不得食,且召亂矣。」遂行,發粟賑給,禽
 首謀者治之,其黨遂散。升本路帥兼漕事。黑風峒羅世傳寇郴陽,奸民潛通賊,陰濟以糧。希懌捕治之,賊乏食,乃去。未幾,李元礪寇郴,陳廷佐寇南安,復誘羅世傳與合,劫掠至龍泉。有何光世者,能知賊動息,希懌授光世計,俾誘世傳誅元礪以自贖。功未竟,移知平江府,其後世傳果縛元礪以獻,廷佐勢孤,亦降。



 移知太平州,希懌為倅日,習知其民利病,遂損折帛價,減榷酤額,以蘇民力。已而乞祠,遷端明殿學士,換昭信軍節度使、開府儀
 同三司,致仕。嘉定五年卒,年五十八,贈少保,封成國公。



 士珸字公美,濮字懿王曾孫也。天資警敏,兒時儼如成人。比弱冠,為右監門衛大將軍、貴州團練使。從上皇北遷,次洺州東,與諸宗室議,欲遁還據城。謀未就而金人圍合,皆散走。士珸乘驢西亡,夜半盜奪驢去,徒步疾趨,遲明,抵武安酒家,語人曰:「我皇叔也。」邑官聞之來謁,資以衣冠鞍馬。因募得少壯百餘人,從至磁州,招集義兵以解洺圍。旬日間,得勝兵五千人,歸附者數萬。



 時洺州
 守臣王麟欲叛降敵,軍民怒殺之,推統制韓一為主。士珸夜半薄城下,力戰破圍。翌日入城,部分守御。敵治壕塹,樹鹿角,示以持久。士珸礪將士死守,飛火炮碎其攻具,以計生得其首領,敵乃解圍去。以功遷權知洺州,仍兼防禦使。



 建炎二年,金人再犯洺,糧盡援絕,眾不能守,乃擁士珸出城,由白家灘抵大名府,詔赴行在。



 紹興五年,遷泉州觀察使,再遷平海軍承宣使、知南外宗正事。時泉邸新建,向學者少,士珸奏宗子善軫文藝卓絕,眾
 所推譽,乞免文解,由是人知激勸。遷節度使,未拜而卒,年四十六。贈少師,追封和義郡王。淳熙中,謚忠靖。子不流,歷臨安、紹興帥,治有聲。



 士人褭字立之,郇康孝王仲御第四子。有大志,好學,善屬文。初補右班殿直,累遷忠州防禦使、鄭州觀察使,由寧遠軍承宣使轉權同知大宗正事。時康王建大元帥府,士人褭請於孟太后,乞命帥府得承制便宜行事,又請奉王承大統,太后從之,王遂即位。



 除光山軍節度使,扈蹕
 南幸。黃潛善等用事,士人褭論其誤國,潛善斥之,出知南外宗正事。會苗傅、劉正彥作亂,士人褭易服入杭,以蠟書遺張浚,趣其勤王;復遺呂頤浩書,勉其與浚同濟國難。苗傅等怒浚,浚坐謫。復遺浚書,謂朝廷無他意,俾賊勿疑耳。事平,加檢校少保,除同知大宗正事。



 丁母憂,起復,除知大宗正事。請序位安定郡王下,從之。累乞祠,不許。以定策功,詔其子不議改文秩,不怞易環衛官。加士人褭檢校少師。尋加開府儀同三司,判大宗正事。入覲,勸帝留
 意恤民。



 金人既歸河南、陜西地,命士人褭謁陵寢,遂入柏城,披歷榛莽,隨宜葺治,禮畢而還。特封齊安郡王,以旌其勞。



 尋權主奉濮安懿王祠事。軍興,罷宗室賜予,至有喪不能斂者,士人褭以聞。詔緦麻、袒免親任環衛官而身亡者,賜錢有差。



 士人褭數言事,忤秦檜。及岳飛被誣,士人褭力辨曰:「中原未靖,禍及忠義,是忘二聖不欲復中原也。臣以百口保飛無他。」檜大怒,諷言者論士人褭交通飛,蹤跡詭秘,事切聖躬,遂奪官。中丞萬俟離復希旨連擊之。
 謫居於建,凡十二年而薨,年七十。帝哀之,贈太傅,追封循王。六子皆進官二階。



 長子不凡,方苗傅之亂,刲股納蠟書,持告張浚,以功轉兩官,易文資。從趙哲收復建州,殺葉濃,以功賜爵二級。



 士□穹字仰夫,太宗五世孫。初以蔭補官,累轉太子率府副率。建炎初,隆祐太后幸洪州,敵奄至,百司散走。士□穹至一大船中,見二帝御容,負以走。遇潰兵數百,同行至山中,眾欲聚為盜,士□穹出御容示之曰:「盜不過求食為
 朝夕計耳,孰若仰給州縣。士□穹以近屬諭之,必從。如此,則今日不饑餓,後日不失賞,是一舉而兩得也。」眾聽命。乃走謁太后虔州。



 會虔民作亂,鄉兵在外為應,與官軍相持。士□穹詣執政,謂當請太后急肆赦,人知免死,庶可安集;又宜急諭城中,城中定,則外寇可弭,譬如服藥,心腹已安,外禦風濕,乃餘事耳。赦既下,城中遂定。遷右監門衛大將軍、惠州防禦使。紹興二十一年卒,贈建寧軍承宣使,追封建安郡王。



 士□,太宗之後,商、濮王之裔也。從上皇俱北遷,乘間變姓名入僧寺中,落發,衣僧衣以行,抵會稽。扈駕循幸,以覃恩轉千牛衛將軍奉朝請而卒。



 不群字介然,太宗六世孫。宣和中,量試授承事郎。靖康初,宰濟南章丘縣。縣當山東、河北之沖,不群募效用五千人,增城浚濠,為戰守備,敵攻圍兩月不能下。



 遷維州通判,升直秘閣,通判鎮江府,闢充兩浙宣撫司主管機宜文字。高宗在越,詔改郴州。時群盜出沒湖、湘間,不群
 嚴備御,盜不能犯。進直顯謨閣,移知鼎州,充湖北兵馬副鈐轄。既而朝廷慮郴失守,復留不群於郴。坐岳飛破曹成,成遁,因犯郴,不群乘城固守,拒卻之。



 進直寶文閣,移知宣州。軍需以時辦,而民不擾。進秩二階。知廬州。酈瓊叛,擁不群北去,尋釋之以歸。帝召見,問瓊叛故,不群曰:「由劉錡除制置,瓊等以為圖己,兼撫諭后時,故叛。」帝悔之。除知荊南府,累遷兩浙路轉運副使,卒於官。



 不棄字德夫,太宗之裔。紹興中,為江東轉運判官。秦檜
 忌四川宣撫使鄭剛中,以不棄能制之,除太府少卿、四川宣撫司總領官。初,趙開總蜀賦,宣撫司文移率用申狀,不棄至官,用張憲成故事,以平牒見剛中。剛中愕然,久之始悟其不隸己,遂有隙。不棄欲盡取宣撫司所儲,剛中不與,不棄怒。剛中闢利州轉運使王陟兼本司參議,不棄劾罷之。二人愈不相能,檜並召還,剛中在蜀,服用頗逾制,不棄復文致其事。檜乃罷剛中,升不棄敷文閣待制,知臨安府。



 逾年改工部侍郎,尋除敷文閣直學
 士、知紹興府。時浙東旱,饑民多流亡。提舉秦昌時,檜兄子也,不棄言其悉心振恤,全活甚眾,昌時得遷秩。其媚檜如此。未幾卒。



 不尤,有武力。靖康之難,與王明募義兵,與金人戰,雄張河南、北。盜皆避其鋒,曰:「此小使軍也。」高宗即位,引眾歸,補武翼郎。從岳飛平湖寇。飛死,檜奪其兵,遣守橫州而卒。



 子善悉,進士登第。累官敷文閣直學士、兩浙轉運副使。



 不│字仁仲,嗣濮王宗暉曾孫也。父士圃,從上皇北遷,遙拜集慶軍節度使。不│初補保義郎,紹興二十七年登第,易左宣義郎,調婺州金華丞。治縣豪河汝翼,械請於郡,編隸他州,邑人懾服。



 除永州通判。郡歲輸米,倍收其贏,民病之,不│言於守,損其數。帥司檄不│錄靖州獄,辨出冤者數十百人,靖人德之,繪其像以祠。



 除知開州。開在巴東,俗鄙陋,不│為興學,俾民知孝義。郡有鹽井,舊長吏必遣所親監之,私其處。不│罷遣,鹽利倍入,
 郡計用饒,以羨餘代民輸夏秋兩稅及天申節銀絹。在開二年,民絕鬥爭,夜戶不閉。諸司交薦,以比古循吏。轉夔州轉運判官,開人數千遮城門,不得行。



 至夔,民病上供銀。時部使者以親故攝大寧鹽場,專其利。不│斥去,而鹽獲羨餘。乃出錢市羨鹽數十萬斤,易米得三萬餘斛,運抵湖北,市銀以歸,代諸郡納上供銀,省緡錢十五餘萬。



 改成都路轉運判官。適歲饑,不│行抵瀘南,貸官錢五萬緡,遣吏分糴。比至,下令曰:「米至矣。」富民爭發粟,
 米價遂平。只流朱氏獨閉糴,邑民群聚發其廩。不│抵朱氏法,籍其米,黥盜米者,民遂定。



 永康軍歲治都江堰,籠石蛇絕江遏水,以灌數郡田。吏盜金,減役夫,堰不固而圮,田失水,故歲屢饑。不│躬視,操板築,繩吏以法。乃出令:民業耕者田主貸之,事末作者富民振之,老幼疾患者官為粥視。全活數百萬。



 黎州青羌奴兒結反,制司調兵往戍,屬不│給餉。故事,富人出糧,而下戶以力致於邊。不│曰:「民饑,不可擾也。」以糴餘米發卒運之。已而
 朝廷命不│攝制司。初,官兵敗,前制使遣人賂奴兒結以和。不│曰:「奴兒結,吐蕃小種也,今且和,若大族何?」不聽。



 會酋豪夢束畜列率數千人入漢地二百餘里,成都大恐。不│靜以鎮之,召僚屬飲。夜遣步將領飛山軍徑赴沉黎,又徙綿州兵戍邛州為後援,戒之曰:「堅守勿動。」密檄諸蕃部:生獲吐蕃一人賞十縑,殺一人二縑。於是邛部川首領崖衣蔑合諸部落,大破吐蕃於漢源,斬夢束畜列首來獻,凡十有六日而平。嘉州虛恨蠻入寇,不│
 標吐蕃首境上,蠻懼,一夕遁去。不│乃令緣邊家出丁夫一人,分戍諸堡,復其家。不│罷歸,蜀人送者自成都至雙流,遮道不得行。



 未幾,除成都提刑,改江西路轉運判官。廷臣薦其賢,詔授右監門衛大將軍、惠州防禦使、知大宗正事。非常制也。吏白承受奏請須用中貴人,不│曰:「有司不存乎?」罷不用。中貴人或請見。輒謝出之。



 進明州觀察使,俄升昭慶軍承宣使。金人完顏烈來聘,充館伴副使。金使從者舊見館使,皆對揖,不│不為禮。宴
 玉津園,不│連射皆中,使者驚服。



 不│以文行訓勉族屬,薦其秀傑者,奏新學宮,增廣弟子員,仿大學校定法。置自訟齋,使有過者讀書其中,人人感勵。淳熙十四年卒,年六十七。贈開府儀同三司,封崇國公。



 不│性篤孝,生戒嚴歲,遭父北遷,每思慕涕泣。長力學,母曹氏止之,答曰:「君父仇未報,非敢志富貴也。」登第時已入仕,法當超兩秩,請回授其母。母封法止令人,高宗嘉其志,特封郡夫人。



 居官所至有聲,立朝好言天下事。蜀中武帥操重
 權,不│請復置安撫司,相維而治。其論王抃不宜揀選諸路軍,王友直不可為副都指揮使,尤人所難言者。遇大旱,一日九疏,勸上求直言,通下情,退而燔其稿。時布衣上書狂悖,多抵罪,不│謂太上皇帝不罪言者,此宜書之御座右。帝悚然可之。既嘉其忠諒,每宴禁中,帝飲之酒,顧謂皇太子曰:「此賢宗室也。」一日,坐待漏院,有給事中白英國公借擊球馬,不│正色曰:「上惟一皇孫,萬一馬驚墮,斬汝輩無益也。」馬竟不可得。所敬者朱熹、張
 栻,栻死為請謚,又請用熹。其好尚如此。



 善俊字俊臣,太宗七世孫。父不衰,閩路兵馬鈐轄。善俊初補承節郎。紹興二十七年登第。換左承務郎,調南城丞,改昭信軍,簽判奇之。虞允文亦薦其有邊帥才,除乾辦諸司審計司。知郴州,敷奏稱旨,留為太府寺丞。



 尋攝帥、知廬州。會歲旱,江、浙饑,民麋至。善俊括境內官田均給之,貸牛種,僦屋以居,死者為給槥,人至如歸。州城舊毀於兵,善俊葺完之,因言:「異時恃焦湖以通饋餫,今既
 堙涸,宜募鄉兵保孤、姥二山,治屋以儲粟。敵或敗盟,則吾城守有餘,餉道無乏矣。」又增築學舍,新包拯祠,春秋祀之,人感其化。



 累遷龍圖閣直學士,移知建州。建俗生子往往不舉,善俊痛繩之,給金谷,捐己奉,以助其費。



 再知廬州。首言和好不可恃,當高城浚池以為備。復芍陂、七門堰,農政用修。免責屬邑坊場、河渡羨錢,百姓德之。



 以父憂去,服闋,起知鄂州。適南市火,善俊亟往視事,弛竹木稅,發粟振民,開古溝,創火巷,以絕後患。僚屬爭言
 用度將不足,善俊曰「吾將瘠己肥人。」乃省燕游車騎鼓吹之費,郡計用饒,代輸民役錢。



 再知建州。歲饑,民群趨富家發其廩,監司議調兵掩捕,善俊曰:「是趣亂也。」諭許自新,平米價,民乃定。邑尉入盜十三人死罪,以希賞,善俊辨其冤。



 徙知隆興府,移江西轉運副使。時朝廷議減月樁錢,善俊言:「及州不及縣,則縣仍迫取於民,猶不減也。宜一路通裁其額,下之漕臣,科郡縣輕重均減之。」又奏:「和買已是白科,從而折變,益加糜費,其數反重於正
 絹,並乞蠲減。黥卒遇赦還者,刺充鋪兵,可除民害。」所言多見用。



 轉湖南帥。郴、桂地絕遠,守多非才,善俊謂宜精其選。代輸潭州經、總制錢,停醴陵淥水渡錢。加秘閣修撰,移知鎮江府。丁母憂,終喪而卒,年六十四。



 善俊風儀秀整,喜功名,尤好論事。孝宗時,日中有黑子,地屢震,每以飭邊備為戒。孝宗英武獨運,缺相者累年,善俊極言相位不可無人,尤人所難言者。



 善譽字靜之,父不倚,太宗之後也。善譽幼敏慧,力學。乾
 道五年,試禮部第一。初調昌國簿,攝邑事。勸編戶裒金買田,以助嫁娶喪葬。捕得海盜全黨,守欲上其功,善譽曰:「奈何以人命希賞。」守益賢之,薦於朝。授兩浙運乾,改知撫州臨川縣。縣嘗預借民賦,善譽閱籍發逋負,按籍征催,卒以時辦集,遂罷預借。



 改常州添差通判。史浩言其賢,詔赴部堂審察,累遷大理丞、湖北常平茶鹽提舉。會大旱,善譽通融諸郡常平,計戶振貸,嗣歲麥禾倍收,民爭負以償。奏罷稅場十餘、渡四十五,民便之。俾諸郡
 售田,委郡文學董其入,以給計偕者。



 移潼川路提刑、轉運判官。遂寧守徐詡乏廉聲,部使者以其故御史,寬假之。善譽過遂寧,詡出迎,善譽抑使循廊,詡大沮。郡人聞之,爭訟其過。善譽劾諸朝,宰相王淮善詡,寢其章。善譽徑以聞,罷詡。又以羨貲給諸郡置莊,民生子及娠者俱給米,威惠並孚。宗子寓蜀者,少業儒,善譽即郡庠立學以教之,人始感勵。引年乞祠,歸處一室,以圖書自娛。無疾而卒,年四十七,時淳熙十六年也。



 善譽早失怙恃,撫
 育諸季備至,居官廉靖自將,多所著述,郭雍、朱熹嘗取其《易說》云。



 汝述字明可,太宗八世孫。曾祖士說,從二帝北遷,臨河罵敵而死。汝述登淳熙十一年進士第。調南劍州順昌尉。嘉定六年,詔主管官告院,自是常兼宰士,累遷將作少監,權侍立修注官。八年,除起居郎兼密院都承旨,俄遷兵部侍郎。以母憂去,服闋,改邢部侍郎,遷尚書,知平江府,卒。



 汝述為尉,應詔上封事,論議懇惻。立朝薦引,多
 知名之士。然為時相所親,躐躋通顯,人亦以此少之。



 叔近,悼王元孫,榮良公克類之子也。建炎元年,為秀州守,杭卒陳通反,詔辛道宗將西兵討之。兵潰為亂,抵秀州城下,叔近乘城諭以禍福,亂兵乃去。未幾,差權兩浙提刑。叔近招通,通聽命。叔近以素隊數十人入賊城,眾猶不解甲。叔近置酒,推誠待之,遂皆感服,城中稍定。叔近奏:通初無叛心,止緣葉夢得賞不時給,遂至紛爭;今已就招,請赦其待二百餘人。帝許之。臺諫皆言不可,遂
 寢。



 叔近還秀州,已而王淵兵至杭,詐傳呼云:「趙秀州來。」通郊迎,淵遂誅之。初,淵在汴京,狎娼周氏,周氏後歸叔近,淵銜之,乃誣叔近通賊,奪職拘於州,以朱芾代之。芾肆殘虐,軍民怨憤,小卒徐明率眾囚芾,迎叔近領郡事,叔近不得辭,因撫定之,請擇守於朝。



 奏未達,朝廷命張俊致討。俊,淵部曲也,辭行,淵謂之「叔近在彼。」俊諭意。領兵至郡,叔近出迎,俊叱令置對。方操筆,群刀遽前,斷其右臂,叔近呼曰:「我宗室也。」俊曰:「汝既從賊,何云宗室!」
 語未竟,已折首於地。徐明等見叔近死,遂反戈嬰城,縱火驅掠。翌日,俊斬關入,捕明等誅之。取周氏歸於淵,紹興九年,御史言叔近之冤,贈集英殿修撰。



 叔向,魏王之系也。方汴京破時,叔向潛出,之京西。金人退,引眾屯青城,入至都堂,叱王時雍等速歸政,置救駕義兵。其後為部將於渙上變,告叔向謀為亂,詔劉光世捕誅之。



 彥倓字安卿,彭城侯叔褧曾孫也。父公廣,饒州太守。彥
 倓初調溧陽尉,邑民潘氏兄弟橫邑中,號「三虎」,畜僮僕數百,邑官莫敢誰何。彥倓白其寧治之,縛潘氏弟,正其罪。



 改揚州司戶,攝獄掾。有告主藏吏錢餘千萬,治之急,吏泣請死。彥倓察其情,屏人問,則諸共貸也,攝宜興縣。縣自中興後,預借民明年稅,民挾此得慢其令。彥倓請禁預借,邑遂易治。



 知臨安於潛縣。縣胥往往通臺省吏,得肆其奸。彥倓執其黠者,械送府。臺省吏從中救之,彥倓力
 爭,竟抵胥罪。浮橋屢以水敗,彥倓梁以石,民免溺死。臨安府通判。



 開禧初,知興國軍。歲旱蝗,而軍需急,屬邑令吳格負上供銀尤多,彥倓坐累貶秩格愧謝。彥倓曰:「屬時多艱,宜寬民力以崇根本,何謝為?」潰卒據外城為變,彥倓募能斬捕者賞之。既而各斬首以獻,散其餘黨。



 累遷湖南運判。徭人羅孟傳反,累歲不能平。彥倓謂帥臣曰:「徭人仇殺,乃其常情,況主斷不平,是游之使叛也。能遣諜者離其黨與,俾還自相仇,破之易矣。」帥從其
 計,遂降隈傳。



 尋知紹興府。楮價輕,彥倓權以法,民便之。復鹿鳴禮,置興賢莊以資其費。築捍海石塘,亦置莊以備增築。會,饑民聚陂湖中,彥倓取死囚,幕首刖足,徇於眾曰:「此劫SO藕也。」遂散其眾。乃第民高下,損其稅有差,免輸湖籍田米,舉緡錢四十萬以助荒政,民賴以濟。詔改太府少卿,遷顯謨閣、知太平州,調江西轉運使。嘉定十一年卒於官,年六十四。



 彥橚字文長,悼王七世孫。祖訓之在《忠義傳》。彥橚登乾
 道二年進士第。尉樂清,會大旱,令循故事禱雨,而責租益急。彥橚曰:「損斂已責,所以和氣,何禱為?」已而果雨。累官福建路運乾,屬邑負振鹽本錢數千萬,累歲不能償,彥橚白其長,蠲之。



 慶元初,知晉陵縣,歲饑,彥橚振恤有方,所活幾二十萬。又以羨錢為五等戶代輸。



 擢監登聞檢院。時韓侂冑方柄用,朝士悉趨其門,彥橚切嘆惋。出知汀州,州民葉姓者,嘯聚汀、贛間,彥橚遣將捕戮之。遷廣西提刑,諸郡鬻官鹽,取息之六以奉漕司,系增至
 八分。彥橚復其舊,以蘇民力,朝廷從之。



 侂冑死,詔戶部侍郎兼樞密院檢詳。士大夫人前與兵議者,坐侂冑黨,將並逐之。彥橚嘆曰:「士方以偽學廢,今又以兵端斥去,茍欲錮士,何患無名!」每見帝,必言才難。



 遷湖廣總領。舊士卒物故,大將不落其籍,而私其月請,彥橚置別籍稽核之。或傳軍中有怨言,彥橚曰:「不樂者主帥耳,何損士卒。」持之三年,掛虛籍者贏三萬,額減錢百萬緡,用度以饒。比去,餘七百萬,而諸路累積逋負猶四百萬,盡蠲之。



 知
 平江府。郡之昆山並大海,盜出沒,可蹤跡,彥橚奏分其半置嘉定縣,屯兵以守。轉寶謨閣待制。卒於官,年七十一。



 彥逾字德先,魏悼王後,崇簡國公叔寓曾孫也。紹興三十年登第。淳熙五年,知秀州。累遷太府少卿、四川總領。將入境,利西帥吳挺遣屬吏安丙來迓,彥逾見即喜其人,從容問之曰:「太尉統眾六萬,得無虛籍乎?」丙以情告。彥逾遺挺書,俾損虛籍數千,以寬四川之賦。挺不敢隱。
 改知鎮江府,郡適旱饑,彥逾節浮費,發粟振糴,民賴以濟。



 遷戶部侍郎、工部尚書。孝宗崩,光宗疾,不能持喪。樞密趙汝愚議請立嘉王為皇帝,欲倚殿帥郭杲為用,遣中郎將範任告之,杲不應。時中外洶洶,彥逾見汝愚,對泣,汝愚密告以翊戴之議。彥逾大喜,力贊其決。郭杲嘗被誣,彥逾為白於帝,杲德之,遂馳告杲曰:「彥逾與樞密第能謀之耳,太尉為國虎臣,當任其責。」杲未及對,彥逾急責之,杲許諾,遂領兵為衛。寧宗即位,汝愚謂彥逾曰:「
 我輩宗臣,不當言功。」



 會留正免相,汝愚登右揆,彥逾以端明殿學士出知建康,兼江東安撫使。未行,改撫制置使,兼知成都府。彥逾為政不擾,蜀人便安之。以定策勛,累遷資政殿大學士。嘉泰間,知明州兼沿海制置使。嘉定間,乞祠以歸,尋卒。



 彥逾始與汝愚協濟大計,冀汝愚引己共政,及外除,頗觖望,乃疏當時名臣上之,目為汝愚黨,帝由是疑汝愚。



 其兩入蜀皆有聲。然吳氏世守武興,兼利西安撫,操重權。吳挺卒,朝廷用丘宗議,
 並利西安撫於東路,以革世將之弊。而彥逾奏利西安撫,乃領以武帥。其後吳曦因之以生變,人以是咎彥逾云



\end{pinyinscope}