\article{列傳第一百四十九}

\begin{pinyinscope}

 李衡王自中家願張綱張大經蔡洸莫蒙周淙劉章沉作賓



 李衡,字彥平,江都人。高祖昭素仕至侍御史。衡幼善博誦,為文操筆立就。登進士第,授吳江主簿。有部使者怙
 勢作威,侵刻下民,衡不忍以敲撲迎合,投劾於府,拂衣而歸。後知溧陽縣,專以誠意化民,民莫不敬。夏秋二稅,以期日榜縣門,鄉無府吏跡,而輸送先他邑辦。因任歷四年,獄戶未嘗系一重囚。



 隆興二年,金犯淮堧,人相驚曰:「寇深矣!」官沿江者多送其孥,衡獨自浙右移家入縣,民心大安。盜蝟起旁境,而溧陽靖晏自如。帥汪澈、轉運使韓元吉等列上治狀,詔進一秩,尋召入為監察御史。歷司封郎中、樞密院檢詳,出知溫、婺、臺三州,惟婺嘗蒞其
 治。加直秘閣,而衡引年乞身,懇懇不休,上累卻其奏,除秘閣修撰致仕。上思其僕忠,旋召落致仕,除侍御史,以老固辭,不獲命。差同知貢舉。會外戚張說以節度使掌兵柄,衡力疏其事,謂「不當以母後肺腑為人擇官」,廷爭移時。改除起居郎,衡曰:「與其進而負於君,孰若退而合於道。」章五上,請老愈力,上知不可奪,仍以秘撰致仕。時給事中莫濟不書敕,翰林周必大不草制,右正言王希呂亦與衡相繼論奏,同時去國,士為《四賢詩》以紀之。衡
 後定居昆山,結茅別墅,杖屨徜徉,左右惟二蒼頭,聚書逾萬卷,號曰「樂庵」,卒,年七十九。



 衡自宣和間入闢雍,同舍有趙孝孫者,洛人也,其父實師程頤,家學有源,勸衡讀《論語》曰:「學非記誦辭章之謂,所以學聖賢也,不可有絲毫偽實處,方可以言學。」衡心佩其訓,雖博通群書而以《論語》為根本。臨沒,沐浴冠櫛,翛然而逝。周必大聞之曰:「世謂潛心釋氏,乃能達死生,衡非逃儒入釋者,而臨終超然如此,殆幾孔門所謂聞道者歟。」



 王自中,字道甫,溫州平陽人。少負奇氣,自立崖岸,繇是忤世。乾道四年,議遣歸正人,自中伏麗正門爭論,且言:「今內空無賢,外空無兵,當搜羅豪俊,廣募忠力,以圖中原。」坐斥徽州,放還。淳熙中,登進士第,主舒州懷寧簿。嚴州分水令。



 樞密使王藺薦,召對,帝壯其言,將改秩為籍田令,又俾舉所知,且響用矣,以諫疏罷。自中本韓彥古客,王藺既薦之,上大喜。韓彥直、彥質輩恐其為彥古報仇,力請交結於自中;而密達意近習,謂「自中受彥古賂,
 伏闕上書薦彥古為相。」上遣人物色其事,中書舍人王信恆懼自中入臺將不利於王淮,知彥直輩譖已行,亟請對,探上意;退即走白右正言蔣繼周。繼周方敢劾奏,讀至「受賂伏闕」處,上曰:「卿可謂中其膏肓。」繼周奏:「臣非不知孤蹤忤王藺,但不敢曠職。」蓋欲並中藺以媚淮,上但喜繼周善論事,不知曲折如此。



 通判郢州,道除知光化軍,改信州,丁內艱,服闋,還朝。光宗即位,迎謂曰:「朕得卿名於壽皇,留為郎可乎?」言者不置。主管沖祐觀,起知
 邵州、興化軍,命下而自中已病,慶元五年八月,卒,年六十。



 家願,字處厚,眉山人。父勤國,慶歷、嘉祐間與從兄安國、定國同從劉巨游,與蘇軾兄弟為同門友。王安石久廢《春秋》學,勤國憤之,著《春秋新義》。熙寧、元豐諸人紛更,而元祐諸賢矯枉過正,勤國憂之,為築室,作《室喻》,二蘇讀之敬嘆。



 願弱冠游京師,以廣文館進士登第,時紹聖元年也。廷策進士,中書侍郎李清臣擬進策問,力詆元祐
 之政,願答策惟以守九年之所已行者為言。時門下侍郎蘇轍嘗上疏辨策問,舉漢武帝事,觸上怒待罪,願未及知也,因見轍,誦所對,驚喜曰:「故人子道同志合,猶若是也。」楊畏覆考,專主熙寧、元豐,取畢漸為第一,願遂居下第。轍尋出守汝,而國論大變矣。



 元符三年,以日食求言,願時為普州樂至令,應詔上言,極論時政凡萬言,其大要有十:一曰謹始以正本,二曰敬德以格天,三曰謹好惡以防小人,四曰審信任以辨君子,五曰開言路以
 來直諫,六曰詳聽言以觀事實,七曰破黨議以存至公,八曰登碩德以服天下,九曰從寬厚以盡人才,十曰崇名節以厚士風。疏上不報。崇寧元年,詔籍元祐、元符上書人姓名,願以選人籍入邪下等,謫監華州西嶽廟。時當改京秩,迄不改,禁錮不調凡十年。大觀四年,孛星出,降赦,黨禁解,始改秩,調知雙流縣。通判文州。郡守鄭行純憑內侍勢自恣,罷蕃夷互市,啟邊隙。願爭之,不從,徑下令復其舊。守怒,交章互奏,俱報罷。而願以曾入黨籍,
 謫英州酒稅,量移黃州,數年始予祠。興元帥臣王庶薦自代,通判果州。靖康初,左丞馮澥薦備諫列,除開封府工曹,京城失守,不克赴。高宗南渡,擢知閬州。會張浚謀大舉,願謂浚厲兵足穀以俟機會,浚不悅,以便旨移彭州。有論邊防書,名曰《罪言》。守彭之明年,乞骸骨以歸,卒。



 方蘇轍之讀願策,謂願少年能不為進取計,異時當以直道聞,恨不及見,轍之言至是而驗。淳祐間,願曾孫大酉侍講經筵,因從容及之,上改容嘉嘆,宣取所上書,又
 親書「西社同門友,元符上書人」十大字以賜。



 願同郡楊恂,丹棱人也,字信仲。元豐五年,登進士第。元符初,知廣都縣,與願同時上書,語甚切直。越三年,亦同入黨籍邪下第五等。其書以火不存。



 張綱,字彥正,潤州丹陽人。入太學,以上舍及第。釋褐,徽宗知綱三中首選,特除太學正,遷博士,除校書郎。入對,論:「君子小人簉殽,詢言試事則邪正自別。小人得志邀功生事,禍有不可勝言者。今用事者大言罔上,風俗侈
 靡,背本趨末,日甚一日。宜以祖考躬行之教為法,天下有不難化矣。」上稱善。論事與蔡京不相合,擠之去,主管玉局觀。久之還故官,兼修《國朝會要》、校正御前文字。遷著作佐郎、屯田司勛郎。



 初,朝議遣童貫、蔡攸使朔方,綱力論不可出師狀,不報。及金渝盟犯京闕,命綱分守四壁,旋解嚴,詔登陴足月者遷。綱曰:「主憂臣辱,義當爾,顧因此受賞邪?」卒不自言。出為兩浙提刑,移江東。池將王進剽悍恣睢,曹官以小過違忤,遂釘手於門。事聞,詔綱
 乘傳窮竟。時國勢未安,諸將往往易朝廷,進擁甲騎數百突至綱前,綱叱進階下,即按問,罪立具,自是無越法者。以左司召,權監察御史。請令郡邑月具系囚存亡數,申提刑司,歲終校多寡行殿最。進起居舍人,改中書舍人。建言乞依祖宗法命大臣兼領史事,詔宰臣呂頤浩監修國史,著為令。



 試給事中。大將有以軍中田不均乞不收租,朝廷將從之,綱執不可。會推恩元祐黨籍家,有司無限制,自陳者紛至。綱建議以崇寧所刻九十八人為
 正。自軍興後,小人多乘時召亂,歷五年而怨家告訐者眾。綱謂非所以廣好生之德,乞自蔽囚,後有告勿受。宗室令懬特轉太中大夫,綱言:「庶官超轉侍從非法,且自崇寧以來官職不循資任,致綱紀大壞,今方丕變其俗,奈何以令懬故復違舊章。」詔以次官命詞,舍人王居正復執不行,命遂寢。宣撫使張俊駐師九江,遣營卒以書至瑞昌,縣令郭彥章揣知卒與獄囚通,乃械系之。俊醞於朝,彥章坐免。綱言:「近時州縣吏多獻諛當路,彥章不
 隨流俗,是能奉法守職,今不獎而黜,何以示勸?」



 除給事中。侍御史魏矼劾綱,提舉太平觀。進徽猷閣待制,引年致仕。秦檜用事久,綱臥家二十年絕不與通問。檜死,召為吏部侍郎兼侍讀。初講《詩關雎》,因後妃淑女事,歷陳文王用人,寓意規戒。上曰:「久不聞博雅之言,今日所講析理精詳,深啟朕心。」綱言:「比年監司資淺望輕,請擇七品以上清望官,或曾任郡守有治狀者為之,庶位望既重,材能已試,可舉其職。」從之。權吏部尚書。時以彗出東
 方,詔求言。綱奏:「求言易,聽察難。宜命有司詳審章奏,必究極其情,無事茍簡。」除參知政事。高宗頻諭輔臣寬恤民力,蓋懲秦檜苛政,期安黎庶。綱乃摘其切於利民八十事,標以大指,乞鏤版宣布中外,於是人皆昭知上德意。告老,以資政殿學士知婺州,尋致仕。高宗幸建康,綱朝行宮。孝宗登極,召綱陪祀南郊,以老辭不至,詔嘉之,命所在州郡恆存問,仍賜羊酒,卒,年八十四。



 綱嘗書坐右曰:「以直行己,以正立朝,以靜退高天下。」其篤守如此。
 初謚文定,吏部尚書汪應辰論駁之,孫釜再請,特賜曰章簡。釜,慶元間為諫官,力排道學諸賢,累官至簽書樞密院事。



 張大經,字彥文,建昌南城人。紹興十五年,中進士第,宰吉之龍泉,有善政。諸司列薦,賜對便殿,出知儀真。時兩淮監司、帥守多興事邀功,大經獨以平易近民,民咸德之。提舉湖南常平,提點湖北刑獄,尋移江東。他路有巨豪犯法,獄久不竟,命移屬大經。豪挾權勢求脫,大經卒
 正其罪。孝宗重風憲之選,命條上部使者十人,上獨可大經,召見,上曰:「朕十人中得卿一人,以卿風力峻整。」遂除監察御史,命下,中外聳嘆。



 大經首陳士風掊克、偷惰、誕慢、浮虛四弊。時理官間多居外,大經奏非便,乃作舍守庭。遷大理少卿,守殿中侍御史。言:「今日不治,由大臣不任責。」又言:「諸路荒政不實,飛蝗頗多。願益加恐懼,申飭大臣,俾內而百官有司輸忠讜、修厥職,外而監司守臣察貪理冤、去苛斂、寬民力。」上皆嘉納。因論近習韓俁
 薦士,上曰:「此亦無害。昔楊得意為狗監,亦嘗薦司馬相如。」大經奏:「彼何人斯,使得薦士,將恐無廉恥者望風希旨,傷毀士俗。」後數日,上謂大經曰:「卿前所論韓俁,朕思之誠是也。」又論宦者董璉暴橫,將命淮甸,所至誅求,且自號「董閻羅」。上曰:「然,人皆言之。」即依奏鐫罷,竄南康軍。除侍御史。上宣諭曰:「卿論事得體,且詳練。」大經遂言:「士風未厚,吏治未肅,民力未蘇,和氣未應,皆由人心未正。願察公正,明義利,以彰好惡,抑浮薄,去貪刻,則莫不靡
 然洗濯,一歸於正。」上稱善再三。又言:「監司治民之本,不可限以資格。」上納其言,即選四寺丞同時臨遣。試右諫議大夫兼侍講。請通漕臣之計,以補州郡之有無;拘戶絕之租,以廣常平之儲彳侍;嚴臟罪改正法,以懲貪黷;收外路闢闕歸吏部,以杜私謁而通孤寒。



 秋旱,詔求言。大經極言:「人心不和有以致之。民力竭而愁嘆多,軍士貧而怨嗟眾,二者當今大弊。州縣之間,絹帛多折其估,米粟過收其贏,關市苛征,榷酤峻禁。中外兵帥多出貴幸
 之門,營利自豐,素召眾怨,教閱滅裂,軍容不整。且近習甲第名園,越法逾制,別墅列肆,在在有之,非賂遺何以濟欲?願陛下疏斥憸腐,抑絕幸門,垂意人主之職,責成宰輔,一提其綱,則天下事必有能辦之者。」俄而池司郝政降充統制官,殿帥補外,蓋用其言也。



 除禮部尚書兼侍讀。大經屢請祠,上曰:「卿公廉必能為朕牧民。」以徽猷閣學士知建寧府。未幾,移鎮紹興,辭不拜,予祠。進龍圖閣學士,告老,以通奉大夫致仕。方主眷未衰,抗疏引去,
 人方之孔戣。壽逾八帙,紹熙五年,寧宗即位,進正議大夫,降詔撫問,賜銀奩藥茗。慶元四年七月,疾革,語諸子曰:「吾目可暝,吾愛君憂國之心不可泯。」無一語及私。卒,年八十九。訃聞,上甚悼之,贈銀青光祿大夫,謚簡肅。



 蔡洸,字子平,其先興化仙游人,端明殿學士襄之後,徙霅川。父伸,左中大夫。洸以蔭補將仕郎,中法科,除大理評事,遷寺丞,出知吉州。召為刑部郎,徙度支,以戶部郎總領淮東軍馬錢糧、知鎮江府。會西溪卒移屯建康,舳
 艫相銜。時久旱,郡民築陂TK水灌溉,漕司檄郡決之,父老泣訴。洸曰:「吾不忍獲罪百姓也。」卻之。已而大雨,漕運通,歲亦大熟。民歌之曰:「我TK我水,以灌以溉。俾我不奪,蔡公是賴。」就除司農少卿,言:「鎮江三邑稅戶客戶輸丁各異,請為一體,不得自為同異。所輸丁絹,依和買之直,計尺折納,人給一鈔,官自買絹起發,公私皆便。」上嘉納。以戶部侍郎召,試吏部尚書,移戶部。上謂侍臣曰:「朕以版曹得人為喜。」洸常言:「財無滲漏則不可勝用。」未幾求
 去,除徽猷閣學士、知寧國府。陛辭賜坐,上慰勞曰:「卿面有火色,風證也,朕有二方賜卿。」洸謝,即奉祠以歸。卒,年五十七。



 洸事親孝,曾祖襄未易名,力請於朝,賜謚忠惠。所得奉,每以振親戚之貧者,去朝之日,囊無餘資,至售所賜銀鞍韉治行,人服其清潔雲。



 莫蒙字子蒙,湖州歸安人。以祖蔭補將仕郎,兩魁法科,累官至大理評事、提舉廣南市舶。張子華以臟敗,朝廷命蒙往鞫之,蒙正其罪。又言秦熹、鄭時中受子華賂,計
 直數千緡。還朝,除大理寺正。吏部火,連坐者數百人,久不決,命蒙治之。蒙察其最可疑者留於獄,出餘人為耳目以蹤跡之,約三日復來,遂得其實,系者乃得釋。黃州卒奏親擒盜五十餘人,上命蒙窮竟,既至,咸以冤告。蒙命囚去桎梏,引卒至庭,詢竊發之由,鬥敵之所,遠近時日悉皆抵牾,折之,語塞。蒙具正犯數人奏上,餘釋之。上諭輔臣曰:「莫蒙非獨曉刑獄,可俾理金穀。」除戶部員外郎。



 朝廷遣蒙措置浙西、江淮沙田蘆場,上語之曰:「得此可
 助經費,歸日以版曹處卿。」蒙多方括責,得二百五十三萬七千餘畝。言者論其丈量失實,徵收及貧民,責監饒州景德鎮。起知光化軍。諜知金渝盟,郡乏舟,眾以為慮,蒙力為辦集,及敵犯境,民賴以濟。時餉饋急,除淮南轉運判官,蒙遷延不之任,右司諫梁仲敏劾其慢命,罷官勒停。宣諭使汪澈為言於上,復舊職,召見,上諭曰:「朕常記向措置沙田甚不易。」



 蒙謝曰:「職爾,不敢避怨。」上曰:「使任責者人人如卿,天下何事不成。」



 除湖北轉運判官。未
 幾,知鄂州,召除戶部左曹郎中,出知揚州。陛辭,上以城圮,命蒙增築。蒙至州,規度城闉,分授諸將各刻姓名甃堞間,縣重賞激勸,閱數月告成。除直寶文閣學士、大理少卿兼詳定司敕令官,兼權知臨安府。未幾,假工部尚書使金賀正旦。金庭錫宴,蒙以本朝忌日不敢簪花聽樂,金遣人趣赴,蒙堅執不從,竟不能奪。使還,除刑部侍郎,改工部侍郎兼臨安府少尹,以言者罷。起知鄂州。卒於官,年六十一,贈正奉大夫。



 周淙,字彥廣,湖州長興人。父需,以進士起家,官至左中奉大夫。淙幼警敏,力學,宣和間以父任為郎,歷官至通判建康府。紹興三十年,金渝盟,邊事方興,帥守難其選,士夫亦憚行。首命淙守滁陽,未赴,移楚州,又徙濠梁。淮、楚舊有並山水置砦自衛者,淙為立約束,結保伍。金主亮傾國犯邊,民賴以全活者不可勝計。除直秘閣,再任。孝宗受禪,王師進取虹縣,中原之民翕然來歸,扶老攜幼相屬於道。淙計口給食,行者犒以牛酒,至者處以室
 廬,人人感悅。張浚視師,駐於都梁,見淙謀,輒稱嘆,且曰:「有急,公當與我俱死。」淙亦感激,至謂「頭可斷,身不可去」。浚入朝,悉陳其狀,上嘉嘆不已,進直徽猷閣,帥維揚。



 會錢端禮以尚書宣諭淮東,復以淙薦,進直顯謨閣。時兩淮經踐蹂,民多流亡,淙極力招輯,按堵如故。勸民植桑柘,開屯田,上亦專以屬淙,屢賜親札。淙奉行益力,進直龍圖閣,除兩浙轉運副使。未幾,知臨安府,上言:「自古風化必自近始。陛下躬履節儉,以示四方,而貴近奢靡,殊
 不知革。」乃條上禁止十五事,上嘉納之,降詔獎諭,賜金帶。臨安駐蹕歲久,居民日增,河流湫隘,舟楫病之,淙請疏浚。工畢,除秘閣修撰,進右文殿修撰,提舉江州太平興國宮以歸。上念淙不忘,除敷文閣待制,起知寧國府,趣入奏,上慰撫愈渥。魏王出鎮,移守婺州。明年春,復奉祠,亟告老。十月卒,年六十,積階至右中奉大夫,封長興縣男。



 劉章,字文孺,衢州龍游人。少警異,日誦數千言,通《小戴
 禮》,四冠鄉舉。紹興十五年廷對,考官定其級在三,迨進御,上擢為第一,授鎮江軍簽判。是冬,入省為正字。明年,遷秘書郎兼普安、恩平兩王府教授,遷著作佐郎。事王邸四歲,盡忠誠,專以經誼文學啟迪掖導,受知孝宗自此始。秦檜當國,嗛不附已,風言者媒薛其罪,出倅筠州。檜死,召為司封員外郎、檢詳樞密院文字兼玉牒檢討官。擢秘書少監、起居郎。使金還,除權工部侍郎,俄兼吏部、兼侍講。郊祀畢,侍從,上《慶成詩》。



 初,章在秘省,嘗議郊
 廟禮文,當置局討論,詔行其說。正遷吏部,御史論章使胥長買絹,高宗愕然曰:「劉章必無是事。」御史執不已,罷提舉崇道觀,舉朝嗟鬱。起居郎王佐訟其冤,亦坐絀。起知信州,未久,復請祠。孝宗受禪,念舊學,命知漳州,為諫議大夫王大寶所格。尋除秘閣修撰、敷文閣待制,召提舉祐神觀兼侍讀,遂拜禮部侍郎。奏禁遏淫祀,仍於《三朝史》中刪去《道釋》、《符瑞志》,大略以為非《春秋》法。



 朝廷議經略中原,調諸郡兵,民頗擾。少卿趙彥端指言非是。或
 譖彥端曰:「陛下究心大舉,凡所圖回,但資趙彥端一笑爾。」顏端懼不測。上因夜對問章曰:「聞卿監中有笑朕者。」章不知狀,從容對曰:「聖主所為,人焉敢笑,若議論不同或有之。」上意頗解。彥端獲免,人稱章長者。詔詢唐太宗所問魏征德仁功利優劣,章上疏諄復,且言:「太宗問征在貞觀十六年,陛下宅天命十載於茲,原益加意,將越商、周紹唐、虞矣,太宗非難到也。」進權禮部尚書兼給事中。對選德殿,問章:「今年幾而容貌未衰,頗嘗學道否?」
 章拱對曰:「臣書生無他長,惟菲儉自度。晏嬰一狐裘三十年不易,人以為難,臣以為易。」上嘉嘆久之。親灑宸翰以賜,俾安職。章力告歸,以顯謨閣學士食祠祿。



 淳熙元年,子之衡由御史、檢法出守廣德軍,當陛辭,對便殿,問:「卿父學士安否?」撫勞再三,臨退復謂曰:「卿歸侍,為朕致此意。」旋遣閣門祗候蘇曦至家宣問,拜端明殿學士,賜銀絹四百匹。四年,上表告老,以資政殿學士致仕,卒,年八十,贈光祿大夫,謚曰靖文。章容狀魁碩,以周密自守,
 出入兩朝,被顧遇,未嘗洩禁中一語。



 沈作賓,字賓王,世為吳興歸安人。以父任入仕,監饒州永平監,冶鑄堅致,又承詔造雁翎刀,稱上意,連進兩資。中刑法科,歷江西提刑司檢法官,入為大理評事。改秩,通判紹興府。帥守丘崇遇僚吏剛嚴,作賓從容裨贊,每濟以寬。秩滿,知臺州,首訪民疾苦,弛鹽禁,寬租期,均徭役,更酒政,決滯獄,五十日間盡除前政之不便民者,邦人胥悅;而前守嫉其勝己,巧媒薛之,罷去。民請於朝,借
 留不遂,為立「留賢碑」。除大理正,親嫌,改太府丞,遷刑部郎。



 慶元初,歷官至淮南轉運判官,以治辦聞。直華文閣,因其任。擢太府少卿,總領淮東軍馬錢糧,繼升為卿。尋除直龍圖閣,帥浙東,知紹興府。入對,奏:「徽州、南康軍月樁不如期,朝廷科降額,比年曰『權免一次』,來年督促如初,適足啟吏奸、重民害,乞明詔示。又楚州武鋒一軍已招三千五百餘人,朝廷初欲減戍,數年未就紀律:一,主將望輕;二,郡守節制不為禮;三,訓練不盡其能。願令
 本州少假借,責之練習,期以歲月,考績用成否,上於朝而黜陟之。」上嘉納。韓侂冑方用事,族有居越者,私釀公行,作賓逮捕置於獄,而竄其奴。又論紹興府和買事,語在《食貨志》。



 除兩浙轉運副使。入對,奏:「攢宮一司,歲拔經、總制錢為緡率四萬有奇,丹雘未弊,加之塗飾,墻壁具存,從而創易,妄費固不足計,亡謂驚黷,非所以妥神靈、彰聖孝。今後有合營繕,聞於朝,下守臣稽核,畫旨而後興役。」上首肯再三,而修奉者不樂也。



 除權工部侍郎,繼
 兼戶部侍郎。奏請修紹興三十一年以前故事,復敕令所刪修官五員以待選人有才者,又乞申嚴保伍法。以言者罷歸,起知鎮江府,除集英殿修撰,改知寧國府,除寶謨閣待制,知潭州,除戶部侍郎兼詳定敕令官。奏湖北當儲粟,湖南當增兵。未幾,除龍圖閣待制,知平江府,請得節制許浦水軍,詔可。郡有使臣,故海盜也,作賓使招誘其黨,既至,慰勉之,錫衣物,又得強勇者幾千人,置將以統之,號曰「義士」;復募郡城內外惡少亦幾千人,號
 曰「壯士」。衣糧器械皆視官軍,而輕捷善鬥過之,於是海道不警,市井無嘩。尋命參贊督府,兼權鎮江府。請留戍兵千人,又欲以江、閩新軍二千人易舊軍千人,備不虞。朝廷難之,遂請祠。言者繼及之,復召為戶部侍郎。軍興之餘,國力殫耗,見存金谷,僅支旬日。作賓考逋負,柅吏奸,閱三月即有半年之儲。充館伴使,兼權工部尚書。



 會臨安闕知府事,時相欲奏用作賓,力辭。除權戶部尚書,以母憂解,服闋,授顯謨閣直學士、知建寧府。入覲,乞申
 嚴詭戶之禁。除寶謨閣學士、江西安撫兼知隆興府。奏部內南安、南康、龍泉三縣,迫近溪峒,三縣令尉及近峒之砦曰秀洲,曰北鄉,曰蓮塘,並永新縣之勝鄉砦,宜就委帥、憲兩司擇才闢置,量加賞格。又乞詔諸道監司分詣州郡,選禁軍,精練閱,改刺其懦弱者為廂軍。在郡撙錢二十餘萬緡,僚屬請獻諸朝,作賓謂平生未嘗獻羨,以半歸帥司犒師,半隸本府。除煥章閣學士、提舉隆興府玉隆萬壽宮,進顯謨閣學士致仕,卒於家,贈金紫光
 祿大夫。



 論曰:李衡進退雍容,幾於聞道。王自中、家願奇邁危言,摧折弗悔,咸有可稱。嘗考宋之立國,元氣在臺諫。崇寧、大觀而後,奸佞擅權,爵賞冒濫,馴至覆亡。高、孝重繩糾封駁之司,張綱抑令懬恩,大經劾韓俁、斥董璉,人人振揚風採,正氣稍伸矣。時則有若洸、蒙、淙、章、作賓,班班有善,同傳亦宜。



\end{pinyinscope}