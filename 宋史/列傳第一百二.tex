\article{列傳第一百二}

\begin{pinyinscope}

 元絳許將鄧潤甫林希弟旦蔣之奇陸佃吳居厚溫益



 元絳,字厚之,其先臨川危氏。唐末,曾祖仔倡聚眾保鄉里,進據信州,為楊氏所敗,奔杭州,易姓曰元。祖德昭,仕
 吳越至丞相,遂為錢塘人。絳生而敏悟,五歲能作詩,九歲謁荊南太守,試以三題,上諸朝,貧不能行。長,舉進士,以廷試誤賦韻,得學究出身。再舉登第,調江寧推官,攝上元令。



 民有號王豹子者,豪占人田,略男女為僕妾,有欲告者,則殺以滅口。絳捕置於法。甲與乙被酒相毆擊,甲歸臥,夜為盜斷足。妻稱乙,告里長,執乙詣縣,而甲已死。絳敕其妻曰:「歸治而夫喪,乙已伏矣。」陰使信謹吏跡其後,望一僧迎笑,切切私語。絳命取僧系廡下,詰妻奸
 狀,即吐實。人問其故,絳曰:「吾見妻哭不哀,且與傷者共席而襦無血污,是以知之。」



 安撫使範仲淹表其材,知永新縣。豪子龍聿誘少年周整飲博,以技勝之,計其貲折取上腴田,立券。久而整母始知之,訟於縣,縣索券為證,則母手印存,弗受。又訟於州,於使者,擊登聞鼓,皆不得直。絳至,母又來訴,絳視券,呼謂聿曰:「券年月居印上,是必得周母他牘尾印,而撰偽券續之耳。」聿駭謝,即日歸整田。



 知通州海門縣。淮民多盜販鹽,制置使建言,滿二
 十斤者皆坐徒。絳曰「海濱之人,恃鹽以為命,非群販比也。」笞而縱之。擢江西轉運判官、知臺州。州大水冒城,民廬蕩析。絳出庫錢,即其處作室數千區,命人自占,與期三歲償費,流移者皆復業。又甓其城,因門為閘,以御湍漲,後人守其法。入為度支判官。



 儂智高叛嶺南,宿軍邕州而歲漕不足。絳以直集賢院為廣東轉運使,建瀕江水砦數十,以待逋寇;繕治十五城,樓堞械器皆備,軍食有餘。以功遷工部郎中,歷兩浙、河北轉運使,召拜鹽鐵副
 使,擢天章閣待制、知福州,進龍圖閣直學士,徙廣、越、荊南,為翰林學士、知開封府,拜三司使、參知政事。數請老,神宗命其子耆寧校書崇文院,慰留之。



 會太學虞蕃訟博士受賄,事連耆寧,當下獄。絳請上還職祿,而容耆寧即於訊外,從之。於是御史至第薄責絳,絳一不自辨,罷知亳州。入辭,帝謂曰:「朕知卿,一歲即召矣。卿意欲陳訴乎?」絳謝罪,願得穎,即以為穎州。明年,加資政殿學士、知青州,過都,留提舉中太一宮,力疾入謁,曰:「臣疾憊子弱,
 倘一旦不幸死,則遺骸不得近先人丘墓。」帝惻然曰:「朕為卿辦襄,雖百子何以加。」詔毋多拜,乘輿行幸勿扈從。又明年,以太子少保致仕。



 絳所至有威名,而無特操,少儀矩。仕已顯,猶謂遲晚。在翰林,諂事王安石及其子弟,時論鄙之。然工於文辭,為流輩推許。景靈宮作神御十一殿,夜傳詔草《上梁文》,遲明,上之。雖在中書,而蕃夷書詔,猶多出其手。既得謝,帝眷眷命之曰:「卿可營居京師,朕當資幣金,且便耆寧仕進。」絳曰:「臣有田廬在吳,乞歸
 鬻之,即築室都城,得望屬車之塵,幸矣。敢冀賜邪。」既行,追繼白金千兩,敕以蚤還。絳至吳逾歲,以老病奏,恐不能奉詔。三年而薨,年七十六。贈太子少師,謚曰章簡。



 許將字沖元,福州閩人。舉進士第一。歐陽修讀其賦,謂曰:「君辭氣似沂公,未可量也。」簽書昭慶軍判官,代還,當試館職,辭曰:「起家為官,本代耕爾,願以守選餘日,讀所未見書。」宰相善其志,以通判明州。神宗召對,除集賢校理、同知禮院,編修中書條例。自太常丞當轉博士,超改
 右正言;明日,直舍人院;又明日,判流內銓:皆神宗特命,舉朝榮之。初,選人調擬,先南曹,次考功。綜核無法,吏得緣文為奸,選者又不得訴長吏。將奏罷南曹,闢公舍以待來訴者,士無留難。進知制誥,特敕不試而命之。



 契丹以兵二十萬壓代州境,遣使請代地,歲聘之使不敢行,以命將。將入對曰:「臣備位侍從,朝廷大議不容不知。萬一北人言及代州事,不有以折之,則傷國體。」遂命將詣樞密院閱文書。及至北境,居人跨屋棟聚觀,曰:「看南朝
 狀元。」及肄射,將先破的。契丹使蕭禧館客,禧果以代州為問,將隨問隨答。禧又曰:「界渠未定,顧和好體重,吾且往大國分畫矣。」將曰:「此事,申飭邊臣豈不可,何以使為?」禧慚不能對。歸報,神宗善之,以將知審官西院、直學士院、判尚書兵部。



 時河北保甲、陜西河東弓箭社、閩楚槍仗手雖有名籍,其多少與年月不均,以致閱按無法,將一切整攝之。進翰林學士、權知開封府,為同進所忌。會治太學虞蕃訟,釋諸生無罪者,蔡確、舒但因陷之,逮其
 父子入御史府,逾月得解,黜知蘄州。



 明年,以龍圖閣待制起知秦州,改揚州,又改鄆州。上元張燈,吏籍為盜者系獄,將曰:「是絕其自新之路也。」悉縱遣之,自是民無一人犯法,三圄皆空。父老嘆曰:「自王沂公後五十六年,始再見獄空耳。」鄆俗士子喜聚肆以謗官政,將雖弗禁,其俗自息。



 召為兵部侍郎。上疏言:「兵措於形勢之內,最彰而易知;隱於權用之表,最微而難能。此天下之至機也。是以治兵有制,名雖不同,從而橫之,方而圓之,使萬眾
 猶一人;車馬有數,用雖不同,合而分之,散而斂之,取四方猶跬步;制器有度,工雖不同,左而右之,近而遠之,運眾算猶掌握。非天下之至神,孰能與此?」又條奏八事,以為「兵之事有三:曰禁兵,曰廂兵,曰民兵。馬之事有三:曰養馬,曰市馬,曰牧馬。兵器之事有二:曰繕作,曰給用。」及西方用兵,神宗遣近侍問兵馬之數,將立具上之;明日,訪樞臣,不能對也。



 以龍圖閣直學士知成都府。元祐三年,再為翰林學士。四年,拜尚書右丞。將自以在先朝為
 侍從,每討熙、豐舊章以聞。中旨用王文鬱、姚兕領軍,執政復議用張利一、張守約。將始與執政同議,復密疏利一不可用。言者論其窺伺主意,炫直賣友。罷為資政殿學士、知定州,移揚州,又移大名府。



 會黃河東、北二議未決,將曰:「度今之利,謂宜因梁村之口以行東,因內黃之口以行北,而盡閉諸口,以絕大名諸州之患。俟水大至,觀故道足以受之,則內黃之口可塞;不足以受之,則梁村之口可以止;兩不能相奪,則各因其自流以待之」



 紹
 聖初,入為吏部尚書,上疏乞依元豐詔,定北郊夏至親祀。拜尚書左丞、中書侍郎。章惇為相,與蔡卞同肆羅織,貶謫元祐諸臣,奏發司馬光墓。哲宗以問將,對曰:「發人之墓,非盛德事。」方黨禍作,或舉漢、唐誅戮故事,帝復問將,對曰:「二代固有之,但祖宗以來未之有,本朝治道所以遠過漢、唐者,以未嘗輒戮大臣也。」哲宗皆納之。



 將嘗議正夏人罪,以涇原近夏而地廣,謀帥尤難,乞用章楶,楶果有功。崇寧元年,進門下侍郎,累官金紫光祿大夫,
 撫定鄯、廓州。邊臣欲舉師渡河,朝議難之。將獨謂:「外國不可以爽信,而兵機有不可失,既已戒期,願遂從之。」未幾,捷書至,將以復河、湟功轉特進,凡居政地十年。



 御史中丞朱諤取將舊謝章表,析文句以為謗,且謂:「將左顧右視,見利則回,幡然改圖,初無定論。元祐間嘗為丞轄,則盡更元豐之所守。紹聖初復秉鈞軸,則陰匿元祐之所為。逮至建中,尚此冒居,則紹聖之所為已皆非矣。強顏今日,亦復偷安,則建中之所為亦隨改焉。」遂以資政
 殿大學士知河南府。言者不已,降資政殿學士、知穎昌府,移大名,加觀文殿學士、奉國軍節度使。在大名六年,數告老,召為祐神觀使。政和初,卒,年七十五。贈開府儀同三司,謚曰文定。



 子份,龍圖閣學士。



 鄧潤甫,字溫伯,建昌人。嘗避高魯王諱,以字為名,別字聖求,後皆復之。第進士,為上饒尉、武昌令。舉賢良方正,召試不應。熙寧中,王安石以潤甫為編修中書條例、檢正中書戶房事。神宗覽其文,除集賢校理、直舍人院,改知諫院、知制誥。
 同鄧綰、張琥治鄭俠獄,深致其文,入馮京、王安國、丁諷、王堯臣罪。



 擢御史中丞。上疏曰:「向者陛下登用雋賢,更易百度,士狃於見聞,蔽於俗學,競起而萃非之,故陛下排斥異論,以圖治功。然言責之路,反為壅抑;非徒抑之,又或疑之。論恤民力,則疑其違道干譽;論補法度,則疑其同乎流俗;論斥人物,則疑其訐以為直。故敢言之氣日以折,而天下事變,有不得盡聞。曩變法之初,勢自當爾。今法度已就緒,宜有以來天下論議。至於淫辭詖行,
 有挾而發,自當屏棄。如此,則善言不伏,而致大治也。」



 李憲措置熙河邊事,潤甫率其屬周尹、蔡承禧、彭汝礪上書切諫,其略云:「自唐開元以來,用楊思勖、魚朝恩、程元振、吐突承璀為將。有功,則負勢驕恣,陵轢公卿;無功,則挫損國威,為四國笑。今陛下使憲將兵,功之成否,非臣等所能預料。然以往事監之,其有害必矣。陛下仁聖神武,駕御豪傑,雖憲百輩,顧何能為,獨不長念卻慮,為萬世之計乎?豈可使國史所書,以中人將兵自陛下始?後
 世沿襲故跡,視以為常,進用其徒握兵柄,則天下之患,將有不可勝言者矣!」不聽。



 又言:「興利之臣,議前代帝王陵寢,許民請射耕墾,而司農可之。唐之諸陵,因此悉見芟劉,昭陵喬木,翦伐無遺。熙寧著令,本禁樵採,遇郊祀則敕吏致祭,德意可謂遠矣。小人掊克,不顧大體。願絀創議之人,而一切如令。」從之。



 遷翰林學士。因論奏相州獄,為蔡確所陷,落職知撫州。移杭州,以龍圖閣直學士知成都府。召復翰林學士兼掌皇子閣箋記,一時制作,
 獨倚潤甫焉。哲宗立,惟潤甫在院,一夕草制二十有二。進承旨,修撰《神宗實錄》。以母喪去,終制,為吏部尚書。梁燾論其草蔡確制,妄稱有定策功,乃以龍圖閣學士知亳州。閱歲,復以承旨召。數月,除端明殿學士、禮部尚書。請郡,得知蔡州,移永興軍。



 元祐末,以兵部尚書召。紹聖初,哲宗親政,潤甫首陳武王能廣文王之聲,成王能嗣文、武之道,以開紹述。遂拜尚書左丞。章惇議重謫呂大防、劉摯,潤甫不以為然,曰:「俟見上,當力爭。」無何,暴卒,年
 六十八。輟視朝二日。以嘗掌均邸箋奏,優贈開府儀同三司,謚曰安惠。



 林希,字子中,福州人。舉進士,調涇縣主簿,為館閣校勘、集賢校理。神宗朝,同知太常禮院。皇后父喪,太常議服淺素,希奏:「禮,後為父降服期。今服淺素,不經。」及遣使高麗,希聞命,懼形於色,辭行。神宗怒,責監杭州樓店務。歲餘,通判秀州,復知太常禮院,遷著作佐郎、禮部郎中。元豐六年,詔修《兩朝寶訓》,上之。元祐初,歷秘書少監、起居
 舍人、起居郎,進中書舍人。言者疏其行誼浮偽,士論羞薄,不足以玷從列。以集賢殿修撰知蘇州,更宣、湖、潤、杭、亳五州,加天章閣待制。



 紹聖初,進寶文閣直學士、知成都府。道闕下,會哲宗親政,章□惇用事,嘗曰:「元祐初,司馬光作相,用蘇軾掌制,所以能鼓動四方,安得斯人而用之。」或曰:「希可。」□惇欲使希典書命,逞毒於元祐諸臣,且許以為執政。希亦以久不得志,將甘心焉,遂留行。復為中書舍人,修《神宗實錄》兼侍讀。



 哲宗問:「神宗殿曰宣光,前
 代有此名乎?」希對曰:「此石勒殿名也。」乃更為顯承。時方推明紹述,盡黜元祐群臣,希皆密豫其議。自司馬光、呂公著、大防、劉摯、蘇軾、轍等數十人之制,皆希為之,詞極其醜詆,至以「老奸擅國」之語陰斥宣仁,讀者無不憤嘆。一日,希草制罷,擲筆於地曰:「壞了名節矣。」



 遷禮部,吏部尚書、翰林學士,擢同知樞密院。始,惇疑曾布在樞府間己,使希為貳,以相伺察。希日為布所誘,且怨惇不引為執政,遂叛惇。會邢恕論希罪,惇因並去之,罷知亳州,移
 杭州,布不能救也。旋以端明殿學士知太原府。



 徽宗立,徙大名。上河東邊計三策,朝廷以其詞命醜正之罪,奪職知揚州,徙舒州。未幾卒,年六十七。追贈資政殿學士,謚曰文節。弟旦。



 旦,第進士,熙寧中,由著作佐郎主管淮南常平,擢太子中允、監察御史裏行。居臺五月,以論李定事罷守故官。久之,乾當奏院;陳繹領門下封駁,又摭其前論罷之。累年,乃簽書淮南判官。入為太常博士,工部、考功員外郎。



 元祐元年,拜殿中侍御史。甫蒞職,即上疏曰:「廣言路然後知得失,達民情然後知利病。竊見去歲五月,詔求讜言,士民爭欲自獻。及詳觀詔語,名雖求諫,實欲拒言,約束丁寧,使不得觀望迎合,犯令干譽,終之,必行黜罰以恐懼之。於是人人知戒,言將出而復止;至於冉申諭告,方達天聰。聞初詔乃蔡確、章惇造端,其詞盡出於惇。今二人既去,其餘黨常懷醜正惡直之心,願深留宸慮,以折邪謀。」遂論呂惠卿、鄧綰:「雖罷揚州,猶蒞小郡,小郡之
 民奚罪焉?乞投之散地,以謝天下。」又言:「近彈王中正、石得一等,雖已薄責,得一所任肘腋小人,如翟勍之徒,亦宜編削。」詔並降支郡營校。又論崔臺符、賈種民舞文深酷之罪,皆逐之。出為淮南轉運副使,歷右司郎中、秘書少監、太僕卿,終河東轉運使。



 子膚,坐元符上書,陷於黨籍。



 蔣之奇,字穎叔,常州宜興人。以伯父樞密直學士堂蔭得官。擢進士第,中《春秋三傳》科,至太常博士;又舉賢良
 方正,試六論中選,及對策,失書問目,報罷。英宗覽而善之,擢監察御史。



 神宗立,轉殿中侍御史,上謹始五事:一曰進忠賢,二曰退奸邪,三曰納諫諍,四曰遠近習,五曰閉女謁。神宗顧之曰:「斜封、墨敕必無有,至於近習之戒,孟子所謂『觀遠臣以其所主」者也。」之奇對曰:「陛下之言及此,天下何憂不治。」



 初,之奇為歐陽修所厚,制科既黜,乃詣修盛言濮議之善,以得御史。復懼不為眾所容,因修妻弟薛良孺得罪怨修,誣修及婦吳氏事,遂劾修。神
 宗批付中書,問狀無實,貶監道州酒稅,仍榜朝堂。至州,上表哀謝,神宗憐其有母,改監宣州稅。



 新法行,為福建轉運判官。時諸道免役推行失平,之奇約僦庸費,隨算錢高下均取之,民以為便。遷淮東轉運副使。歲惡民流,之奇募使修水利以食流者。如揚之天長三十六陂,宿之臨渙橫斜三溝,尤其大也,用工至百萬,溉田九千頃,活民八萬四千。



 歷江西、河北、陜西副使。之奇在陜西,經賦入以給用度,公私用足。比其去,庫緡八十餘萬,邊
 粟皆支二年。移淮南,擢江、淮、荊、浙發運副使。元豐六年,漕粟至京,比常歲溢六百二十萬石,錫服三品。請鑿龜山左肘至洪澤為新河,以避淮險,自是無覆溺之患。詔增二秩,加直龍圖閣,升發運使。凡六年,其所經度,皆為一司故事。



 元祐初,進天章閣待制、知潭州。御史韓川、孫升、諫官朱光庭皆言之奇小人,不足當斯選。改集賢殿修撰、知廣州。妖人岑探善幻,聚黨二千人,謀取新興,略番禺,包據嶺表,群不逞借之為虐,其勢張甚。之奇遣鈐
 轄楊從先致討,生擒之。加寶文閣待制。南海饒寶貨,為吏者多貪聲,之奇取前世牧守有清節者吳隱之、宋璟、盧奐、李勉等,繪其象,建十賢堂以祀,冀變其習。



 徙河北都轉運使、知瀛州。遼使耶律迪道死,所過郡守皆再拜致祭。之奇曰:「天子方伯,奈何為之屈膝邪!」奠而不拜。入為戶部侍郎。未幾,復出知熙州。夏人論和,請畫封境。之奇揣其非誠心,務守備,謹斥候,常若敵至。終之奇去,夏人不敢犯塞。



 紹聖中,召為中書舍人,改知開封府,進龍
 圖閣直學士,拜翰林學士兼侍讀。元符末,鄒浩以言事得罪,之奇折簡別之,責守汝州。閱月,徙慶州。



 徽宗立,復為翰林學士,拜同知樞密院。明年,知院事。沅州蠻擾邊,之奇請遣將討之,以其地為徽、靖二州。崇寧元年,除觀文殿學士、知杭州。以棄河、湟事奪職,由正議大夫降中大夫。以疾告歸,提舉靈仙觀。三年,卒,年七十四。後錄其嘗陳紹述之言,盡復官職。



 之奇為部使者十二任,六曲會府,以治辦稱。且孜孜以人物為己任,在閩薦處士陳
 烈,在淮南薦孝子徐積,每行部至,必造之。特以畔歐陽修之故,為清議所薄。



 子瑎至侍從,曾孫芾別有傳。



 陸佃,字農師,越州山陰人。居貧苦學,夜無燈,映月光讀書。躡屩從師,不遠千里。過金陵,受經於王安石。熙寧三年,應舉入京。適安石當國,首問新政,佃曰:「法非不善,但推行不能如初意,還為擾民,如青苗是也。」安石驚曰:「何為乃爾?吾與呂惠卿議之,又訪外議。」佃曰:「公樂聞善,古所未有,然外間頗以為拒諫。」安石笑曰:「吾豈拒諫者?但
 邪說營營,顧無足聽。」佃曰:「是乃所以致人言也。」明日,安石召謂之曰:「惠卿云:『私家取債,亦須一雞半豚。』已遣李承之使淮南質究矣。」既而承之還,詭言於民無不便,佃說不行。



 禮部奏名為舉首。方廷試賦,遽發策題,士皆愕然;佃從容條對,擢甲科。授蔡州推官。初置五路學,選為鄆州教授,召補國子監直講。安石以佃不附己,專付之經術,不復咨以政。安石子雱用事,好進者坌集其門,至崇以師禮,佃待之如常。



 同王子韶修定《說文》。入見,神宗
 問大裘襲袞,佃考禮以對。神宗悅,用為祥定郊廟禮文官。時同列皆侍從,佃獨以光祿丞居其間。每有所議,神宗輒曰:「自王、鄭以來,言禮未有如佃者。」加集賢校理、崇政殿說書,進講《周官》,神宗稱善,始命先一夕進稿。同修起居注。元豐定官制,擢中書舍人、給事中。哲宗立,太常請復太廟牙盤食。博士呂希純、少卿趙令鑠皆以為當復。佃言:「太廟,用先王之禮,於用俎豆為稱;景靈宮、原廟,用時王之禮,於用牙盤為稱,不可易也。」卒從佃議。



 是時,
 更先朝法度,去安石之黨,士多諱變所從。安石卒,佃率諸生供佛,哭而祭之,識者嘉其無向背。遷吏部侍郎,以修撰《神宗實錄》徙禮部。數與史官範祖禹、黃庭堅爭辨,大要多是安石,為之晦隱。庭堅曰:「如公言,蓋佞史也。」佃曰:「盡用君意,豈非謗書乎!」



 進權禮部尚書。鄭雍論其穿鑿附會,改龍圖閣待制、知穎州。佃以歐陽修守穎有遺愛,為建祠宇。《實錄》成,加直學士,又為韓川、朱光庭所議,詔止增秩,徙知鄧州。未幾,知江寧府。甫至,祭安石墓。句
 容人盜嫂害其兄,別誣三人同謀。既皆訊服,一囚父以冤訴,通判以下皆曰:「彼怖死耳,獄已成,不可變。」佃為閱實,三人皆得生。紹聖初,治《實錄》罪,坐落職,知秦州,改海州。朝論灼其情,復集賢殿修撰,移知蔡。



 徽宗即位,召為禮部侍郎。上疏曰:「人君踐祚,要在正始,正始之道,本於朝廷。近時學士大夫相傾競進,以善求事為精神,以能訐人為風採,以忠厚為重遲,以靜退為卑弱。相師成風,莫之或止,正而救之,實在今日。神宗延登真儒,立法制
 治,而元祐之際,悉肆紛更。紹聖以來,又皆稱頌。夫善續前人者,不必因所為,否者賡之,善者揚焉。元祐紛更,是知賡之而不知揚之之罪也;紹聖稱頌,是知揚之而不知賡之之過也。願咨謀人賢,詢考政事,惟其當之為貴,大中之期,亦在今日也。」徽宗遂命修《哲宗實錄》。



 遷吏部尚書,報聘於遼,歸,半道聞遼主洪基喪,送伴者赴臨而返,誚佃曰:「國哀如是,漢使殊無吊唁之儀,何也?」佃徐應曰:「始意君匍匐哭踴而相見,即行吊禮;今偃然如常時,
 尚何所吊?」伴者不能答。



 拜尚書右丞。將祀南郊,有司欲飾大裘匣,度用黃金多,佃請易以銀。徽宗曰:「匣必用飾邪?」對曰:「大裘尚質,後世加飾焉,非禮也。」徽宗曰:「然則罷之可乎?數日來,豐稷屢言之矣。」佃因贊曰:「陛下及此,盛德之舉也。」徽宗欲親祀北郊,大臣以為盛暑不可,徽宗意甚確。朝退,皆曰:「上不以為勞,當遂行之。」李清臣不以為然。佃曰:「元豐非合祭而是北郊,公之議也。今反以為不可,何耶?」清臣乃止。



 御史中丞趙挺之以論事不當,罰金。
 佃曰:「中丞不可罰,罰則不可為中丞。」諫官陳瓘上書,曾布怒其尊私史而壓宗廟。佃曰:「瓘上書雖無取,不必深怒,若不能容,是成其名也。」佃執政與曾布比,而持論多近恕。每欲參用元祐人才,尤惡奔競,嘗曰:「天下多事,須不次用人;茍安寧時,人之才無大相遠,當以資歷序進。少緩之,則士知自重矣。」又曰:「今天下之勢,如人大病向愈,當以藥餌輔養之,須其安平;茍為輕事改作,是使之騎射也。」



 轉左丞。御史論呂希純、劉安世復職太驟,請加
 鐫抑,且欲更懲元祐餘黨。佃為徽宗言不宜窮治,乃下詔申諭,揭之朝堂。讒者用是詆佃,曰:「佃名在黨籍,不欲窮治,正恐自及耳。」遂罷為中大夫、知亳州,數月卒,年六十一。追復資政殿學士。



 佃著書二百四十二卷,於禮家、名數之說尤精,如《埤雅》、《禮象》、《春秋後傳》皆傳於世。



 吳居厚,字敦老,洪州人。第嘉祐進士,熙寧初,為武安節度推官。奉行新法,盡力核閑田,以均給梅山,計勞,得大理丞,轉補司農屬。元豐間,提舉河北常平,增損役法
 五十一條,賜銀緋,為京東轉運判官,升副使。



 天子方興鹽、鐵,居厚精心計,籠絡鉤稽,收羨息錢數百萬。即萊蕪、利國二冶官自鑄錢,歲得十萬緡。詔褒揭其能。擢天章閣待制、都轉運使。前使者皆以不任職蒙譴,居厚與河北蹇周輔、李南公會境上,議鹽法,搜剔無遺。居厚起州縣凡流,無閥閱勛庸,徒以言利得幸,不數歲,至侍從,嗜進之士從風羨美。又請以鹽息買絹,資河東直;發大鐵錢二十萬貫,佐陜西軍興;且募民養保馬。當時商功利
 之臣,所在成聚,居厚最為掊克。



 劇盜王沖因民不忍,聚眾數千,欲乘其行部至徐,篡取投諸冶。居厚聞知,間道遁去。元祐治其罪,責成州團練副使,安置黃州。章惇用事,起為江、淮發運使。疏支家河通漕,楚、海之間賴其利。召拜戶部侍郎、尚書,以龍圖閣學士知開封府,為永泰陵橋道頓遞使。坐積雨留滯,罷知和州。



 崇寧初,復尹開封,拜尚書右丞,進中書門下侍郎。以老避位,為資政殿學士、東太一宮使,恩許仍服方團金球文帶。自是,前
 執政在京師者視此。出為亳州、洪州,徙太原,道都門,留使祐神觀,復還政府,遷知樞密院。政和三年,以武康軍節度使知洪州,卒,年七十九。贈開府儀同三司。



 居厚在政地久,以周謹自媚,無赫顯惡,唯一時聚斂,推為稱首。



 溫益,字禹弼,泉州人。第進士,歷大宗正丞、利州路湖南轉運判官、工部員外郎。紹聖中,由諸王府記室出知福州,徙潭州。鄒浩南遷過潭,暮投宿村寺,益即遣州都監將數卒夜出城,逼使登舟,竟凌風絕江而去。他逐臣在
 其境內,若范純仁、劉奉世、韓川、呂希純、呂陶,率為所侵困,用事者悅之。未及用,而徽宗以藩邸恩,召為太常少卿,遷給事中兼侍讀。陳瓘指言其過,謂不宜列侍從、處經帳,不報。改龍圖閣待制、知開封府,猶兼侍讀。時執政倡言,帝當為哲宗服兄弟之服。曾肇在邇英讀《史記·舜紀》,因言:「昔堯、舜同出黃帝,世數已遠,然舜為堯喪三年者,以嘗臣堯故也。」益意附執政,進曰:「《史記》世次不足信,堯、舜非同出。」遷吏部尚書。



 建中靖國元年,拜尚書右書。
 鄧洵武獻《愛莫助之圖》,帝初付曾布,布辭。改付益,益得藉手以為宜相蔡京,天下之善士,一切指為異論,時人惡之。布與京爭事帝前,辭頗厲,益叱曰:「曾布安得無禮!」帝不樂,布由是得罪,而京遂為相。進益中書侍郎。



 益仕宦從微至著,無片善可紀,至其狡譎傅合,蓋天稟然。及是,乃時有立異。京一日除監司、郡守十人,益稍不謂然。京知中書舍人鄭居中與益厚,使居中自從其所問之,居中以告。益曰:「君在西掖,每見所論事,舍人得舉職,侍
 郎顧不許耶?今丞相所擬錢和而下十人,皆其姻黨耳,欲不逆其意得乎?」京聞而頗憚焉。逾年,卒,年六十六。



 子萬石至尚書。



 論曰:王安石為政,一時士大夫之素知名者,變其所守而從之,比比皆然;元絳所蒞,咸有異政,亦諂事之,陋矣。許將嘗力止發司馬光墓,此為可稱;而言者謂其仕於元祐、紹聖以至建中,左右視利,幡然改圖,初無定論。鄧潤甫初掌箋記,盛有文名,而首贊紹述之謀,又表章蔡
 確定策之功,雖有他長,無足觀矣。林希草制,務醜詆正人,自知隳壞名節,擲筆而悔之,何晚也;弟旦反其所為,糾劾巨奸,善惡豈相掩哉!蔣之奇始慫恿濮議,晚摭飛語,擊舉主以自文,小人之魁傑者也。吳居厚奉行新法,剝下媚上,溫益阿附二蔡,物議不容。陸佃雖受經安石,而不主新法,元祐黨人之罪,請一施薄罰而已,猶差賢於眾人焉。



\end{pinyinscope}