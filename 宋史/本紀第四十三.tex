\article{本紀第四十三}

\begin{pinyinscope}

 理宗三



 四年春正月壬寅朔,詔邊將毋擅興暴掠,虐殺無辜,以慰中原遺黎之望。帝制《訓廉》、《謹刑》二銘,戒飭中外。以李鳴復參知政事,杜範同知樞密院事,劉伯正簽書樞密
 院事,餘玠華文閣待制、依舊四川安撫制置使、知重慶府兼四川總領財賦,李曾伯寶章閣直學士、依舊淮東安撫制置使、知揚州兼淮西制置使。戊午,樞密院言:「四川帥臣餘玠,大小三十六戰,多有勞效,宜第功行賞。」詔玠趣上立功將士姓名等第,即與推恩。庚申,以餘玠兼四川屯田使。



 二月癸酉,出封樁庫緡錢各十萬,命兩淮、京湖、四川制司收瘞頻年交兵遺骸,立為義□塚。夏四月丁丑,有流星大如太白,出於尾。癸未,填星守太微垣。乙
 未,祈雨。



 五月庚戌,餘玠言:「利閬城大獲山、蓬州城營山,渠州城大良平,嘉定城舊治,瀘州城神臂山,諸城工役,次第就緒。神臂山城成,知瀘州曹致大厥功可嘉,乞推賞以勵其餘。」詔致大帶行遙郡刺史。丁巳,武功大夫、雄威軍都統制楊價世守南邊,連年調戍播州,捍御勤瘁,詔價轉右武大夫、文州刺史。戊午,大元兵圍壽春府。呂文德節制水陸諸軍解圍有功,詔赴樞密院稟議,發緡錢百萬,詣兩淮制司犒師。庚申,守闕進勇副尉桂虎、進
 義副尉楚富、吐渾將虞候鄭蔡捍禦壽春,俱有勞效,詔各官資兩轉,給緡錢。乙丑,前簽書樞密院事鄒應龍薨,贈少保、監察御史。胡清獻劾淮西提刑徐敏子三罪,詔削兩秩,送江州居住。



 六月庚午朔,呂文德依舊侍衛馬軍副都指揮使兼淮西招撫使、知濠州。乙亥,賜禮部進土留夢炎以下四百二十四人及第、出身有差。壬午,詔安豐軍策應解壽春圍將士補轉官資有差。詔:壽春一軍先涉大海,搗山東膠、密諸州有功,今大元兵圍城,
 能守城不隳,其立功將士皆補轉有差。乙未,有流星大如太白,出於畢。丙申,吳潛提舉隆興府玉隆萬壽宮,任便居住。



 秋七月己亥朔,祈雨。乙卯,招收沿淮失業壯丁為武勝軍,以五千人為額。辛酉,盜發永州東安縣,飛虎軍正將吳龍、統制鄭存等討捕有功,詔補轉官資有差。甲子,詔:「故直龍圖閣項安世正學直節,先朝名儒,可特贈集英殿修撰。」八月壬辰,太白晝見。



 九月癸卯,右丞相史嵩之以父病謁告,許之,詔範鐘、劉伯正暫領相事。甲
 辰,史彌忠卒,贈少師,封鄭國公,賜謚文靖。詔史嵩之起復右丞相兼樞密使。癸丑,熒惑、填星合於軫。甲寅,京湖制司言,諸將李福等破申州、蔡州西平縣城壁及馬家等砦,詔將士各補官推賞有差。己未,將作監徐元傑上疏論史嵩之起復,宜許其舉執政自代。帝不允,遂求去。帝曰:「經筵賴卿規益,何事引去耶?」癸亥,太白犯鬥宿距星。乙丑,雷。丁卯,雷。臺臣言嚴州及紹興、蕭山等縣征商煩苛,詔亟罷之。



 冬十月甲戌,詔慶元府守臣敦諭史嵩
 之赴闕,嵩之控辭,不允。壬辰,杜範、游似提舉萬壽觀兼侍講。



 十一月辛丑,詔趣游似、杜範赴闕。戊申,雷。庚戌,詔陳韡、李性傳赴闕。十二月庚午,以範鐘為左丞相兼樞密使,杜範為右丞相兼樞密使,游似知樞密院事,劉伯正參知政事兼簽書樞密院事。詔戒飭百官。許右丞相史嵩之終喪。甲戌,以趙葵同知樞密院事。乙亥,鄭清之授少保,依舊觀文殿大學士、醴泉觀使兼侍讀,仍奉朝請,進封衛國公。



 五年春正月丁酉朔,詔更新庶政,綏撫中原遺民。丙午,杜範辭免右丞相,不允。己酉,雷。乙卯,以李性傳簽書樞密院事兼權參知政事。



 二月丙寅朔,雨土。甲戌,復五河,詔呂文德進三秩,羊洪進二秩,餘有戰功者推賞,其陣沒人,具姓名贈恤。丁丑,範鐘等上《玉牒》、《日歷》及孝宗、光宗《御集》、《經武要略》、《寧宗實錄》。壬辰,太白晝見,經天。



 三月庚子,詔嚴贓吏法,仍命有司舉行彭大雅、程以升、吳淇、徐敏子納賄之罪。準淳熙故事,戒吏貪虐、預借、抑配、重
 催、取贏。以緡錢百萬犒淮東師。



 夏四月甲申,填星犯上相星。丙戌,杜範薨,贈少傅,謚清獻。戊子,餘玠言權巴州何震之守城死於兵,詔進贈官三秩,一子與下州文學。京湖制司言:「鈐轄王云等襲鄧州鎮平縣靈山,戰順陽鐵撅峪,皆有勞效,野戰數十合,雲等六人被重創死,路鈐於江一軍力戰。」詔王云贈三秩,仍官其二子為承信郎。王寬、王立、田秀、董亮、董玉各加贈恤,於江等各轉一官資。詔李曾伯、餘玠、董槐、孟珙、王鑒職事修舉,曾伯、玠
 升閣職,槐、珙、鑒轉官,並因其任。



 五月丁酉,呂文福、夏貴上戰功,詔貴官兩轉,文福帶行閣職。丁未,詔:「沿江、湖南、江西、湖廣、兩浙制帥漕司及許浦水軍司,共造輕捷戰船千艘,置游擊軍壯士三萬人,分備捍禦。」戊申,日生赤黃背氣。辛亥,詔董槐赴闕。丁巳,淮東制置使李曾伯辭免煥章閣學士,從之。



 六月甲申,祈雨。丙戌,工部侍郎徐元傑暴卒,贈四秩。置詔獄。



 秋七月癸巳朔,日有食之。旱。辛丑,鎮江、常州亢旱,詔監司、守臣及沿江諸郡安集流
 民。甲辰,祈雨。乙卯,詔給徐元傑、劉漢弼官田五百畝、緡錢五千恤其家。丁巳,京湖制司言總制亢國用師眾戰裕州拐河,戰黑山,戰大神山,皆有勞效。詔國用官兩轉,李山等四十七人官一轉。呂文德言與大元兵戰五河隘口,又戰於濠州,大元兵還。詔文德屯駐諸軍戰守將士,推恩有差。



 八月庚辰,範鐘再乞歸田,不允。



 九月甲辰,京湖制置司言:「劉整等率精銳,以雲梯四面登鎮平縣城,入城巷戰,焚城中倉庫、糗糧、器甲,路將武勝等四人
 死之;略廣陽,焚列屯、砦柵、廬舍凡二十餘所;還抵靈山,又力戰有功。」詔整官兩轉,同行蔡貴等二百二十人各官一轉。辛亥,祀明堂,奉太祖、太宗、寧宗並侑。大赦。冬十一月乙未,鄭清之乞歸田,不允。丙申,詔師彌典伺屬籍,職事修舉,授太傅,加食邑,依前判大宗正事、嗣秀王。壬子,詔:大元兵入蜀,權成都府馮有碩、權漢州王驤、權成都縣楊兌、權資州劉永、權潼川府魏靄死於官守,其各贈官三轉,仍官其一子。癸丑,詔將領關貴、統制白傅才
 率眾復洋州,還遇大元兵交戰,將士百五十三人皆陣沒,已祔饗閔忠廟,贍恤其家。關貴、白傅才各贈承節郎,官其一子進勇副尉。



 十二月甲戌,詔壽春守臣劉雄飛等以大元兵圍城捍禦有功,雄飛及呂文福、林子崇等十一人各官三轉,劉用等補轉官資有差。己卯,以游似為右丞相兼樞密使,鄭清之為少師、奉國軍節度使,依前醴泉觀使兼侍讀,仍奉朝請,賜玉帶及賜第行在。兄與歡換授安德軍節度使、開府儀同三司、萬壽觀使,仍
 奉朝請;弟嗣沂王貴謙、嗣榮王與芮並加授少保。以趙葵知樞密院事兼參知政事,李性傳同知樞密院事,陳韡兼參知政事。壬午,太史奏來歲正旦日當食,詔以是月二十一日避殿減膳,命百司講行闕政,凡可以消弭災變者,直言毋隱。



 六年春正月辛卯朔,日有食之。置國用所,命趙與為提領官。



 二月戊辰,範鐘再乞歸田里,詔官三轉,觀文殿大學士、醴泉觀使兼侍讀。己巳,範鐘再辭,詔提舉洞霄
 宮,任便居住。庚午,以劉雄飛知壽春府、節制屯田軍馬。



 三月癸巳,日暈周匝,珥氣。



 夏四月辛酉,太白晝見。壬戌,太陰犯太白。甲戌,以丘岳兼兩淮屯田副使,賈似道兼蘄、黃屯田副使。丁丑,日暈周匝。戊寅,詔朱熹門人胡安之、呂燾、蔡模並迪功郎、本州州學教授。給札錄其著述,並條具所欲言者以聞。閏四月辛卯,李曾伯以臺諫論,詔落職予祠,尋罷祠祿。戊戌,呂文德言:「今春北兵攻兩淮,統制汪懷忠等逆戰趙家園,拔還俘獲人民;路鈐夏
 貴,知州王成、倪政等帥舟師援安豐軍,所至數戰,將士陣亡者眾。」詔倪政贈官三轉,官一子承信郎;許通、夏珪、孫才江德仙各贈官兩轉,官其一子下班祗應,給緡錢恤其家;餘立功將士恩賞有差。」辛丑,月暈五重。癸卯,餘玠言:北兵分四道入蜀,將士捍禦有功者,輒以便宜推賞,具立功等第補轉官資以聞。詔從之。



 五月庚申,詔賈似道措置淮西山砦城築。壬戌,太白犯權星。己卯,詔諸鎮募兵、造舟、置馬,帥臣其務獎激將士,以嚴邊防。



 六月
 甲午,保信軍節度使希丞薨。丙午,祈雨。壬子,以陳韡參知政事兼同知樞密院事。乙卯,臺臣言李鳴復、劉伯正進則害善類,退則蠹州里。詔鳴復落職罷宮觀,伯正削一秩。



 秋七月壬戌,泉州歲饑,其民謝應瑞非因有司勸分,自出私錢四十餘萬,糴米以振鄉井,所全活甚眾。詔補進義校尉。丁卯,太陰犯鬥。己巳,呂文德言:「北兵圍壽春城,州師至黃家穴,總管孫琦、呂文信、夏貴等戰龍堽有功。」詔文德官一轉,餘依等第轉補;其陣沒董先等二
 十二人、傷者四百三十七人,贈恤恩賞有差。癸酉,有流星出自室,大如太白。



 八月辛卯,太陰犯房。己酉,賜文士劉克莊進士出身,以為秘書少監兼國史院編修官、實錄院檢討官。壬子,太白晝見。癸丑,以劉克莊兼崇政殿說書。樞密院言:「前知普州何叔丁、簽書判官楊仁舉,淳祐元年冬北兵攻城,兩家二十餘人死於難,叔丁孫嗣祖、仁舉幼子肖翁被俘逃歸。」詔叔丁等贈官恤後有差。



 九月甲子,有流星出於斗,大如太白。戊辰,以賈似道為
 敷文閣直學士、京湖制置使、知江陵府兼夔路策應使。太白晝見。癸酉,孟珙薨,贈少師。



 冬十月己丑,少保、嗣榮王與芮之子賜名孟啟,授貴州刺史。乙未,填星、歲星、熒惑合於亢。己酉,太白入氐。



 十一月癸亥,歲星入氐。甲戌,右丞相游似五請歸田里,詔不允。辛巳,詔:「北兵入蜀,前四川制置使陳隆之闔家數百口罹害,死不易節,其特賜徽猷閣待制,官其二子,賜謚立廟。死事史季儉、楊戡子各賜官兩轉,官一子。」十二月乙未,詔史嵩之依所乞
 守金紫光祿大夫、觀文殿大學士、永國公致仕。臺諫論史嵩之無父無君,醜聲穢行,律以無將之法,罪有餘誅,乞寢宮祠,削官遠竄。



 七年春正月乙卯朔,詔:「間者絀逐非才,收召眾正,史嵩之已令致事,示不復用。咨爾二三大臣,其一乃心,務舉實政,以輯寧我邦家。若辭浮於實,玩愒歲月,朕何賴焉。」建資善堂,授孟啟宜州觀察使,就內小學。



 二月庚寅,詔:「淮安主簿周子鎔,久俘於北,數遣蠟書諜報邊事,今
 遂生還,可改朝奉郎,優與升擢。」己亥,貴妃賈氏薨。戊申,日暈周匝。壬子,詔改潛邸為龍翔宮。



 三月庚午,祈雨。



 夏四月丁亥,填星犯亢。庚子,以王伯大簽書樞密院事,吳潛同簽書樞密院事。辛丑,以鄭清之為太傅、右丞相兼樞密使,封越國公;游似罷為觀文殿大學士、醴泉觀使兼侍讀;趙葵為樞密使兼參知政事,督視江淮、京西、湖北軍馬;陳韡知樞密院事、湖南安撫大使、知潭州。甲辰,趙葵兼知建康府、行宮留守、江東安撫使,應軍行調
 度並聽便宜行事;趙希塈禮部尚書、督視行府參贊軍事。庚戌,出緡錢千萬、銀十五萬兩、祠牒千、絹萬,並戶部銀五千萬兩,付督視行府趙葵調用。



 五月甲寅,寧淮軍統制張忠戍浮山,手搏北將,俱溺水死,贈武略大夫,官一子承信郎,緡錢五千給其家。祈雨。壬申,以吳潛兼權參知政事。乙亥,御集英殿策士,詔求直言弭旱。



 六月癸巳,賜禮部進士張淵微以下五百二十七人及第、出身有差。丙申,以旱,避殿減膳。詔中外臣僚士民直陳過失,毋
 有所諱。戊申,詔:「旱勢未釋,兩淮、襄、蜀及江、閩內地,曾經兵州縣,遣骼暴露,感傷和氣,所屬有司收瘞之。」秋七月己未,太陰犯心。乙丑,吳潛罷。丁卯,以別之傑參知政事,鄭寀同簽書樞密院事。己卯,吳潛依舊端明殿學士、知福州、福建安撫使。



 八月甲申,鄭寀罷。辛卯,雨。辛丑,前彭州守臣宇文景訥死事,詔贈官、進三秩,官一子下州文學。壬寅,詔監司、守臣議荒政以振乏絕,租稅合蠲減者具實來上。甲辰,高定子薨,贈少保。丙午,蔡抗進其父沈《
 尚書解》。



 九月丙辰,有流星出於室。癸酉,雷。



 冬十月辛巳,太白晝見。己酉,臺臣言添差、抽差、攝局、須入、奏闢、改任、薦舉、借補、曠職、匿過十弊。



 十一月丁巳,詔:「茶陵知縣事黃端卿為郴寇所害,進官三秩,官一子將仕郎,立廟衡州。」十二月辛巳,李鳴復卒。壬辰,詔:「太學生程九萬自北脫身來歸,且條上邊事,賜迪功郎。」



 八年春二月丁亥,趙葵言呂文德洎諸將解泗州之圍有功,詔補轉推賞有差。戊子,太陰生黃白暈。癸巳,雨雹。
 乙未,福州福安縣民羅母年過百歲,特封孺人,復其家。敕有司歲時存問,以厚風化。辛丑,趙葵表:「招、泗斷橋,將士用命,兵退。陳奕、譚涓玉、王成等戰渦河、龜山有勞,聞其步兵多山東人,遂調史用政等襲膠州,復襲高密縣,以牽制侵淮之師。」詔趣上立功將士等第、姓名推賞。乙丑,雨雹。甲戌,詔:「先鋒軍統制田智潤泗州潮河壩之戰,父子俱死於兵,贈智潤修武郎,子承節郎,更官其一子承信郎,給緡錢五千恤其家。」夏四月庚辰,詔淮東制置司
 於泗州立廟,祠夏皋及張忠、田智潤父子,賜額以旌忠節。丁亥,贈朝奉郎程克己妻王氏同沒王事,進贈安人。



 五月癸丑,趙葵進三秩。



 六月乙酉,日生赤黃暈周匝。戊戌,以徐鹿卿為樞密使兼參知政事兼侍講。甲辰,有流星出河鼓,大如太白。



 秋七月戊申,太白入井。辛亥,以王伯大參知政事,應彳繇同知樞密院事,謝方叔簽書樞密院事,史宅之同簽書樞密院事,趙與資政殿學士,依舊知臨安府、浙西安撫使。癸酉,王伯大罷為資政殿
 學士、知建寧府。



 九月辛酉,祀明堂,大赦。雷。



 冬十月甲戌朔,別之傑三疏乞歸田里,詔以資政殿大學士知紹興府。乙亥,應彳繇、謝方叔並兼參知政事。己卯,餘玠言:「都統制張實等以戰功,承制便宜與官三轉,給刺史象符、金銀器二百兩、銀三百兩、緡錢一萬,餘將士依等第轉官,給金銀符、錢帛有差。」詔命詞、給告身付之。



 九年春正月乙巳,孟啟授慶遠軍節度使,進封益國公。庚申,詔周世宗八世孫柴彥穎補承務郎,襲封崇義
 公。辛酉,詔兩淮、京湖沿江曠土,軍民從便耕種,秋成日官不分收,制帥嚴勸諭覺察。」癸亥,詔給官田五百畝,命臨安府創慈幼局,收養道路遺棄初生嬰兒,仍置藥局療貧民疾病。乙丑,雨雹。丁卯,許應龍薨。己巳,範鐘薨,贈少保,謚文肅。辛未,詔以官田三百畝給表忠觀,旌錢氏功德,仍禁樵採。閏二月甲辰,以鄭清之為太師、左丞相兼樞密使,進封魏國公;趙葵為右丞相兼樞密使;應彳繇、謝方叔並參知政事;史宅之同知樞密院事。乙卯,鄭清之
 五辭免太師,許之。



 三月癸未,以賈似道為寶文閣學士、京湖安撫制置大使。乙酉,程元鳳江、淮等路都大提點坑冶鑄錢公事兼知饒州。丁亥,詔以四月朔日食,自二十一日避殿、減膳、徹樂。



 夏四月壬寅朔,日有食之。庚戌,趙葵四辭免右丞相兼樞密使,詔不允。



 五月己丑,趙葵乞歸田里,又不允。甲午,鄭寀薨。



 六月壬戌晝,南方有星,急流至濁沒,大如太白。丙寅,詔邊郡各立廟一,賜額曰「褒忠」,凡沒於王事忠節顯著者並祠焉,守臣春秋致祀。



 秋
 七月壬辰,詔知吉州李義山更削三秩,監贓錢銀納安邊所。癸酉,太白犯進賢星。



 八月己酉,以吳潛為資政殿學士、知紹興府、浙東安撫使。辛亥,詔趣趙葵治事,命吳淵宣諭赴闕。



 九月丙子,詔趙與提領戶部財用,置新倉,積貯百二十萬,名淳祐倉,許闢官四人。乙未,冊命婉容閻氏為貴妃。



 冬十月辛丑,太白入氐。丁卯,諫臣周坦言:知建寧府楊棟任成都制幕時,盡載激賞庫珍寶先遁,陷丁黼於死,致全蜀生靈塗炭。詔褫棟閣職,罷新任。



 十一月辛未,太白入氐。壬申,有流星出自織女星。丙子,趙與資政殿學士、提領國用、浙西安撫使。癸未,應彳繇乞歸田里,詔以資政殿學士知平江府。十二月己亥,以董槐兼侍讀。乙巳,以吳潛同知樞密院事兼參知政事,徐清叟簽書樞密院事。戊申,太白晝見。戊午,史宅之薨,贈少師。



 十年春正月甲午,應彳繇三乞歸田里,與祠祿。



 二月乙卯,雨土。



 三月癸未,趙葵辭,以為觀文殿大學士、醴泉觀使
 兼侍讀,奉朝請。庚寅,以賈似道為端明殿學士、兩淮制置大使、淮東安撫使、知揚州;餘玠龍圖閣學士,職任依舊;李曾伯徽猷閣學士、京湖安撫制置使、知江陵府。丙申,有流星夕隕。



 夏四月己酉,幸龍翔宮。



 五月丙寅朔,以福州觀察使、提舉祐神觀善珘為保康軍節度使、提舉萬壽觀、嗣濮王;吳淵資政殿學士,依舊職任,與執政恩數。癸未,賈似道言王登浚築江陵城濠有勞,詔登初官選人,減舉主三員。



 八月甲寅,臺州大水。



 九月甲子朔,賈
 似道兼淮西安撫使。己巳,賜禮部進士方夢魁以下五百一十三人及第、出身有差。甲戌,進士第一名方夢魁改賜名逢辰。戊寅,以嚴州水,復民田租。



 冬十月丁酉,詔郡邑間有水患,其被災細民,隨處發義倉振之。辛酉,詔諸主兵官今後行罰,毋杖脊以傷人命。



 十一月壬申,趙葵授特進,依舊觀文殿大學士、判潭州、湖南安撫大使。壬午,雷。癸未,以雷震非時,自二十四日避殿減膳。詔:「公卿大夫百執事各揚乃職,裨朕不逮。」參知政事謝方叔、
 吳潛、簽書樞密院事徐清叟並乞解機政,詔不允。十二月壬辰朔,鄭清之乞歸田里,詔不允。戊戌,太白、歲星合於危。丁巳,虹見。



 十一年春正月丁卯,詔孟啟改賜名孜,依前慶遠軍節度使,進封建安郡王。己丑,詔沿海沿江州郡,申嚴水軍之制。監察御史程元鳳言:資善堂宜選用重厚篤實之士。上嘉納之。



 二月乙未,左丞相鄭清之等上《玉牒》、《日歷》、《會要》及《光宗寧宗寶訓》、《寧宗經武要略》。丁酉,詔清之等
 各進秩有差。庚子,游似乞致仕,詔依舊觀文殿大學士、進二秩。甲寅,太白犯昴。乙卯,太白晝見。



 三月丁卯,少保、保寧軍節度使、嗣濮王不擅薨,贈少師,追封新興郡王。乙亥,雨土。戊寅,以謝方叔知樞密院、參知政事,吳潛參知政事,徐清叟同知樞密院事。辛巳,城寶應,詔移一軍戍守。李庭芝進一秩,將士推恩有差。俞興升成都安撫副使、知嘉定府,任責威、茂、黎、雅邊防。



 夏四月戊戌,潭州民林符三世孝行,一門義居,福州陳氏,笄年守志,壽逾
 九帙,詔皆旌表其門。丁未,進《淳祐條法事類》凡四百三十篇,鄭清之等各進二秩。



 六月甲午,四川餘玠奏進北馬五百,詔立功將士趣上姓名推恩。丙申,高達帶行遙郡刺史、權知襄陽府、管內安撫、節制屯戍軍馬。乙巳,詔求遺書並山林之士有著述者,許上進。



 秋七月癸亥,太白晝見。丙寅,太陰入氐。壬申,太白入井。丁丑,有流星出於畢,大如太白。庚辰,前簽書樞密院事陳卓薨,贈少師。



 八月己丑朔,流星夕隕。癸巳,太陰入氐。丁酉,熒惑入
 井。丁未,命呂文福廬州駐扎御前諸軍都統制。庚戌,詔以故直龍圖閣樓昉所著《中興小傳》百篇、《宋十朝綱目》並《撮要》二書,付史館□謄寫,昉追贈龍圖閣待制。辛亥,詔:「比覽林光世《易範》,明《易》推星配象演義,有司其以禮津遣赴闕。」九月辛未,祀明堂,大赦。閏十月癸丑,太白入氐。癸酉,吳潛五疏乞罷機政,不允。



 十一月丙申,京湖制司表都統高達等復襄、樊,詔立功將士三萬二千七百有二人各官一轉,以緡錢三百五十萬犒師。甲辰,鄭清之乞
 解機政,詔依前太傅、保寧軍節度使充醴泉觀使,封齊國公,仍奉朝請。己酉,詔承信郎陳思獻書籍,賜官一轉。庚戌,太師鄭清之薨,贈尚書令,追封魏郡王,謚忠定。甲寅,以謝方叔為左丞相,吳潛為右丞相。乙卯,以徐清叟參知政事兼同知樞密院事,董槐端明殿學士、簽書樞密院事。十二月戊辰,詔以八事訓飭在廷,曰肅紀綱、用正人、救楮幣、固邊陲、清吏道、淑士氣、定軍制、結人心。己卯,游似薨,贈少師,謚清獻。



 十二年春正月癸巳,武功大夫王堅以復興元功,轉遙郡團練使。辛丑,太學錄楊懋卿以孝行卓異,詔表其門,以其事宣付史館。癸丑,詔宰執議立方田,開溝澮,自近圻始。創置游擊軍,水步各半。



 二月乙卯朔,日有食之。巳未,詔陳顯伯資善堂翊善,蔡抗資善堂贊讀、翁甫資善堂直講。壬午,詔襄、郢新復州縣,賦稅復三年。大元兵數萬攻隨、郢、安、復,京西馬步軍副總管馬榮率將士戰嚴竇山。癸未,再戰銅冶坪。



 三月丁亥,又戰子陵大脊山。詔
 榮兵不滿千,能御大難,賞官兩轉,進州鈐,帶行閣門祗候,賜金帶。諸將王成、楊進各官兩轉升遷,餘推恩有差。丁未,守三水義口諸將焚北屯積蓄,斷其浮梁。



 夏四月庚申,有流星出自角、亢,大如太白。戊辰,詔襄、郢新復州郡,耕屯為急,以緡錢百萬命京閫措置,給種與牛。壬申,熒惑犯權星。乙亥,葵抗兼侍立修注官。丙子,置池州游擊水軍。



 五月甲申朔,祈雨。壬辰,詔申儆江防,每歲以葺戰艦、練舟師勤惰為殿最賞罰。乙巳,盜起信州玉山縣。罷
 諸郡經界。戊申,太陰犯畢。



 六月癸亥,發米三萬石振衢、信饑、玉山寇平。丙寅,嚴、衢、婺、臺、處、上饒、建寧、南劍、邵武大水,遣使分行振恤存問,除今年田租。



 秋七月庚寅,太白、熒惑合於軫。



 八月己未,詔來年省試仍舊用二月一日,殿試用四月十五日以前,庶免滯留遠方士子。己巳,詔以緡錢四十萬振恤在京軍民。丁丑,詔行《會天歷》。辛巳,詔改明年為寶祐元年。



 九月丁亥,少師、保康軍節度使、嗣沂王貴謙薨,贈太傅,追封申王。戊戌,太白、填星合
 於箕。丙午,太白犯鬥。



 冬十月癸丑,以徐清叟參知政事,董槐同知樞密院事。嗣濮王善珘薨,贈少師、追封咸寧郡王。戊午,濮安懿王長孫善奐福州觀察使、提舉祐神觀、嗣濮王。壬申,詔襄、樊已復,其務措置屯田,修渠堰。



 十一月庚寅,吳潛罷。丙申夜,臨安火;丁酉夜,火乃熄。戊戌,詔避殿減膳。壬寅,詔求直言。十二月乙卯,以吳潛為觀文殿大學士、提舉江州太平興國宮。己未,詔追錄彭大雅創城渝州功,復承議郎,官其子。癸亥,詔海神為大
 祀,春秋遣從臣奉命往祠,奉常其條具典禮來上。壬申,太陰入氐。丁丑,立春,雷。



 寶祐元年春正月庚寅朔,詔以藝祖嫡系十一世孫嗣榮王與芮之子建安郡王孜為皇子,改賜名祺,授崇慶軍節度使,進封永嘉郡王。制《資善堂記》賜皇子。戊戌,日生戴氣。癸卯,大元兵渡漢江,屯萬州,入西柳關。高達調將士扼河關,上山大戰,至鱉坑、石碑港而還。詔高達、程大元、李和各官兩轉,餘恩賞有差。



 二月己酉朔,日有食
 之。戊辰,陳垓貪贓不法,竄潮州。辛未,罷尚書省,創置呈白房。



 三月戊子,與芮授少師,加食邑七百戶;希邐檢少傅,加食邑五百戶;與歡授少保,加食邑七百戶;乃裕保康軍節度使,加食邑五百戶。丙申,別之傑薨,贈少師。



 夏四月丁巳,有流星大如太白。



 五月甲午,詔餘玠赴闕。乙未,詔侍從、臺諫、給舍、制司各舉帥才二人。丁酉,熒惑、歲星合在昴。己亥,賜禮部進士姚勉以下及第、出身有差。



 六月戊申朔,江、湖、閩、廣旱。庚戌,四川制司言餘玠疾
 革,詔玠資政殿學士,與執政恩數。辛亥,以賈似道為資政殿大學士,李曾伯端明殿學士、職任依舊。庚申,以餘晦為司農卿、四川宣諭使。祈雨。秋七月壬午,王伯大薨。丙戌,蔡抗兼資善堂翊善,施退翁兼資善堂直講。庚寅,溫、臺、處三郡大水,詔發豐儲倉米並各州義廩振之。癸巳,詔餘玠以興元歸附之兵分隸本路諸州都統,務撫存之,仍各給良田,制司濟以錢粟。甲午,餘玠卒,贈官五轉。庚子,以董槐兼參知政事。癸卯,詔撫諭四川官吏軍
 民。



 八月丁未朔,以馬光祖為司農卿、淮西總領財賦。甲寅,起居郎蕭泰來出知隆興府。先是,起居舍人牟子才與泰來並除,子才四疏辭,極陳泰來奸險污穢,恥與為伍,泰來不得已,請祠,遂予郡。丙辰,以餘晦權刑部侍郎、四川安撫制置使、知重慶府兼四川總領財賦。乙丑,行皇宋元寶錢。



 九月壬午,程元鳳升兼侍讀,牟子才升兼侍講。壬辰,城夔門。太陰入畢。



 冬十月丙午朔,詔出緡錢二百萬,振恤京城軍民。



 十一月丙子朔,詔獎諭襄陽
 守臣高達。己丑,賈似道獻所獲良馬,賜詔褒嘉,其將士增秩、賞賚有差。十二月乙卯,冊瑞國公主。庚申,劉伯正薨,贈五秩。



\end{pinyinscope}