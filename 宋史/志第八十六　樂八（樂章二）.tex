\article{志第八十六 樂八(樂章二)}

\begin{pinyinscope}

 明堂大饗皇地祇神州地祇朝日夕月高禖九宮貴神



 景祐大享明堂二首



 真宗配位奠幣,《誠安》



 思文聖考,對越在天。侑神作主,
 奉幣申虔。



 酌獻,《德安》



 偃革興文,封巒考瑞。威烈巍巍,允膺宗祀。



 皇祐親享明堂六首



 降神,《誠安》



 維聖享帝,維孝嚴親。肇圖世室,躬展精禋。



 鏞鼓既設,籩豆既陳。至誠攸感,保格上神。



 奠玉幣,《鎮安》



 乾亨坤慶育函生,路寢明堂致潔誠。玉帛非馨期感格,降康億載保登平。



 酌獻,《慶安》



 肅肅路寢,相維明堂。二儀鑒止,三聖侑旁。



 靈期欣合,祠節齊莊。至誠並貺,降福無疆。



 三聖配位奠幣,《信安》



 祖功宗德啟隆熙,嚴配交修太室祠。圭幣薦誠知顧享,木支錫羨固邦基。



 酌獻《孝安》



 藝祖造邦,二宗紹德。肅雍孝享,登配圜極。



 先訓有開,菲躬何力!歆馨錫羨,保民麗億。



 送神,《誠安》



 我將我享,闢公顯助。獻終豆徹,禮成樂具。



 飾駕上游,升煙高騖。神保聿歸,介茲景祚。



 嘉祐親享明堂二首



 降神,《誠安》



 燁燁房、心,下照重屋。我嚴帝親,匪配之瀆。



 西顥沆碭,夕景已肅。靈其來娭,嘉薦芳鬱。



 送神,《誠安》



 明明合宮,莫尊享帝。禮樂熙成,精與神契。



 桂尊初闌,羽駕倏逝。遺我嘉祥,於顯萬世。



 熙寧享明堂二首



 英宗奠幣,《誠安》



 於皇聖考,克配上帝。永言孝思,昭薦嘉幣。



 酌獻,《德安》



 英聲邁古,施德在民。允秩宗祀,賓延上神。



 元符親享明堂十一首



 皇帝升降,《儀安》



 嚴父配天,孝乎明堂。輿奠升階,降音以將。



 天步有節,帝容必莊。闢公憲之,禮元不臧。



 上帝位奠玉幣,《鎮安》



 聖能享帝,孝克事親。於皇宗祀,盛節此陳。



 何以薦虔?二精有煒。何以致祥?上天鑒止。



 神宗奠幣,《信安》



 合宮禮備,時維哲王。堂筵四敞,明德馨香。



 聖考來格,降福穰穰。承承繼繼,萬祀其昌。



 奉俎,《禧安》



 奕奕明堂,天子即事。奠我聖考,配於上帝。



 凡百有職,疇敢不祗!俎潔牲肥,其登有儀。



 上帝位酌獻,《慶安》



 惟禮不瀆,所以嚴親。惟孝不匱,所以教民。



 陟配文考,享天大神。重禧累福,祚裔無垠。



 配位酌獻,《德安》



 隆功駿德,兩有烈光。陟配宗祀,惠我無疆。



 退文舞、迎武舞,《穆安》



 舞以象功,樂惟崇德。文經萬邦,武靖四國。



 一張一弛,其儀不忒。神鑒孔昭,孝思維則。



 亞獻,《穆安》



 於昭盛禮,嚴父配天。盡物盡誠,莫匪吉
 蠲。



 重觴既薦,九奏相宣。神介景福,億萬斯年。



 飲福,《胙安》



 莫尊乎天,莫親乎父。既享既侑,誠申禮舉。



 戛擊堂上,八音始具。天子億齡,飲神之胙。



 徹豆,《欽安》



 穆穆在堂,肅肅在庭。於顯闢公,來相思成。



 神既歆止,有聞無聲。錫我休嘉,燕及群生。



 歸大次,《憩安》



 有奕明堂,萬方時會。宗子聖考,作帝之配。



 樂酌虞典,禮從周志。厘事即成,於皇來暨。



 大觀宗祀明堂五首



 奠玉幣,《鎮安》



 交於神明,內心為貴。外致其文,亦效精意。



 嘉玉既陳,將以量幣。肅肅邕邕。惟帝之對。



 有邦事神,享帝為尊。內心致德,外示彌文。



 嘉玉效珍,薦以量幣。恭欽伊何?惟以宗祀。



 配位奠幣,《信安》



 肇祀明堂,告成大報。顒顒祗祗,率見昭考。



 涓選休辰,齊明朝夕。於惟皇王,孝思罔極。



 酌獻,《孝安》



 若昔大猷,孝思維則。永言孝思,丕承其德。



 於昭明威,侑於上帝。繼我思成,永綏福祉。



 配位酌獻,《大明》



 於昭皇考,大明體神。憲章文思,宜民宜人。



 嚴父之道,陟配於天。躬行孝告,有孚於先。



 紹興親享明堂二十六首



 皇帝入門,《儀安》



 惟我有宋,昊天子之。三年卜祀,百世承基。



 施及沖眇,奉牲以祠。敢忘齋慄,偏舉上儀。



 升堂,《儀安》



 於赫明堂,肇稱禋祀。祖宗來游,亦侑於帝。



 九州駿奔,百闢咸事。斂時純休,錫我萬世。



 降神,《誠安》



 噫神何親?惟德是輔。玉牲具陳,誠則來顧。



 我開明堂,遵國之故。尚蒙居歆,以篤宗祜。



 盥洗,《儀安》



 肇開九筵,維古之仿。皇皇大神,來顧來享。



 庶儀交修,百避顯相。微誠自中,交際天壤。



 上帝位奠玉幣,《鎮安》



 皇皇后帝,周覽四方。眷我前烈,燕娭此堂。



 金支秀發,黼帳高張。世歆明祀,曰宋是常。



 皇地祗位奠玉幣,《嘉安》



 至哉坤元,持載萬物!繼天神聖,觀世治忽。



 頌祗之堂,薦以圭黻。孰為邦休,四海無拂?



 太祖位奠幣,《廣安》



 推尊太元,重屋為盛。誰其配之?我祖齊聖。



 開基握符,正位凝命。於萬斯年,孝孫有慶。



 太宗位奠幣,《化安》



 帝神來格,靡祀不從。侑坐而食,獨升祖宗。



 在庭祗肅,展採錯重。三獻之禮,百年之容。



 徽宗位奠幣,《泰安》



 於穆帝臨,至矣元造!克配其儀,惟我文考。



 仁恩廣覃,奕葉永保。宗祀惟初,以揚孝道。



 皇帝還位,《儀安》



 耳聽鋗玉,目瞻熉珠。樂備周奏,儀參漢圖。



 神人並況,天地同符。亦既見帝,王心則愉。



 尚書捧俎,《禧安》



 展牲登俎,《簫韶》在庭。羞陳五室,意徹三靈。



 匪物斯享,惟誠則馨。永作祭主,神其億寧。



 昊天上帝位酌獻,《慶安》



 日在東陸,維時上辛。肇開陽館,恭禮尊神。



 蒼玉輝夜,紫煙煬晨。祖宗並配,天地同禋。



 皇地祗位酌獻,《彰安》



 地祀泰折,歌同我將。黝牲純潔,絲竹發揚。



 博厚而久,含洪以光。扶持宗社,曰篤不忘。



 太祖位酌獻,《孝安》



 一德開基,百年垂統。中天禘郊,薄
 海朝貢。



 寶龜相承,器鼎加重。澤深慶綿,帝復命宋。



 太宗位酌獻,《韶安》



 紹天承業,繼世立功。帷幄屢勝,車書始同。



 武掃氛霧,文垂日虹。遺澤所及,孰知其終!



 徽宗位酌獻,《成安》



 欽惟合宮,承神至尊。祗戒專精,儼然若存。



 奠茲嘉觴,茞蘭其芬。發祉隤祥,以子以孫。



 皇帝還小次,《儀安》



 匏尊既舉,鞂席未移。有德斯顧,靡神不娭。



 物情肅穆,天宇清夷。宅中受命,永復邦基。



 文舞退、武舞進,《穆安》



 神之欻至,慶陰杳冥,風馬雲車,
 恍若有承。



 備形聲容,於昭文明。庶幾嘉虞,來享來寧。



 亞獻,《穆安》



 四阿有嚴,神既戾止。備物雖儀,潔誠惟已。



 有來振振,相我熙事。載酌陶匏,以成毖祀。



 終獻,《穆安》



 誠一為專,禮三而稱。孰陪邦祠?惟我同姓。



 金絲屢調,圭玉交映。是謂熙成,福來神聽。



 皇帝飲福,《胙安》



 孰謂天遠,至誠則通。孰謂地厚,與天則同。



 惠我純嘏,克成大功。握圖而治,如日之中。



 徹豆,《歆安》



 工祝告休,笙鏞雲闋。酒茅既除,牲俎斯徹。



 幽明罔恫,中外咸悅。禮成伊何?天地同節。



 送神,《誠安》



 奕奕宗祀,煌煌禮文。高靈下墮,精意升聞。



 熙事既畢,忽乘青雲。敢拜明貺,永清世氛。



 望燎,《儀安》



 載酌載獻,以純以精。歌傳夜誦,物備秋成。



 報本斯極,聽卑則明。願儲景貺,福我群生。



 望瘞,《儀安》



 禮協豐融,誠交徬佛。闢公受膰,宗祀臨瘞。



 貽我來牟,以興嗣歲。山川出雲,天地同氣。



 還大次,《憩安》



 應天以實,已事而竣。氈案朝帝,竹宮拜
 神。



 靈光下燭,協氣斯陳。福祿時萬,基圖日新。



 紹興、淳熙分命館職定撰十七首



 降神,《景安》



 圜鐘為宮



 上直房、心,時惟明堂。配天享親,宗祀有常。



 盛德在金,日吉辰良。享我克誠,來格來康。



 黃鐘為角



 合宮盛禮,金商令時。備成熙事,搜揚上儀。



 駿奔在庭,精意肅祗。來享嘉薦,神靈燕娭。



 太簇為征



 休德孔昭,靈承上帝。孝極尊親,嚴配於位。



 嘉薦芬芳,禮無不備。神其格思,享茲誠至。



 姑洗為羽



 霜露既降,孝思奉先。陟降上帝,禮隆九筵。



 有馨黍稷,有肥牲牷。神來燕娭,想象肅然。



 盥洗,《正安》



 禮經之重,祭典為宗。上公攝事,進退彌恭。



 庶品豐潔,令儀肅雍。百祥萃止,惟吉之從。



 升殿,《正安》



 皇祖配帝,歲祀明堂。冕服陟降,玉佩瑲瑲。



 疾徐有節,進止克莊。維時右享,日靖四方。



 上帝位奠玉幣,《嘉安》



 大享季秋,百執揚厲。明明太宗,赫赫上帝。



 祗薦忱誠,式嚴圭幣。祚我明德,錫茲來
 裔。



 太宗位奠幣,《宗安》



 穆穆皇祖,丕昭聖功。聲律身度,樂備禮隆。



 祗薦量幣,祀於合宮。玉帛萬國,驩心載同。



 捧俎,《豐安》



 備物昭陳,工祝告具。維羊維牛,孔碩孔庶。



 有嘉維馨,加食宜飫。斂時五福,永膺豐胙。



 上帝位酌獻,《嘉安》



 燁彼房、心,明明有融。維聖享帝,禮行合宮。



 祀事時止,粢盛潔豐。昭受申命,萬福攸同。



 太宗位酌獻,《德安》



 受命溥將,勛高百王。寰宇大定,聖治平康。



 有嚴陟配,宗祀明堂。神保是格,申錫無疆。



 文舞退、武舞進,《正安》



 溫厚嚴凝,於皇上帝。文德武功,列聖並配。



 舞綴象成,肅雍進退。秉翟踆□,總乾蹈厲。



 亞、終獻,《文安》



 總章靈承,維國之常。禮樂宣鬯,降升齊莊。



 竭誠盡志,薦茲累觴。於昭在上,申錫無疆。



 徹豆,《肅安》



 於皇上帝,肅然來臨。恭薦芳俎,以達高明。



 烹飪既事,享於克誠。以介景福,惟德之馨。



 送神,《景安》



 帝在合宮,鑒觀盛禮。黍稷惟馨,神心則喜。



 禮備樂成,亦既歸止。億萬斯年,以貺多祉。



 高宗位奠幣,《宗安》



 赫赫高廟,於堯有光。覆被萬祀,冠冕百王。



 有量斯幣,蠲潔是將。在帝左右,維時降康。



 酌獻,《德安》



 炎運中興,蒼生載寧。九秩燕豫,三紀豐凝。



 精祀上帝,陟配威靈。錫羨胙祉,萬世承承。



 孝宗親享明堂樂曲並同,惟天地位奠幣、酌獻及太祖酌獻、皇帝入小次、還大次、亞獻、送神等篇,各有刪潤。又以太祖奠幣曲改名《廣安》,酌獻改名《恭安》,太宗奠幣改名《化安》,酌獻改名《英安》。



 景德祀皇地祇三首



 降神,《靜安》



 至哉厚德,陟配天長!沈潛剛克,廣大無疆。



 資生萬物,神化含章。同和八變,神靈效祥。



 奠玉幣,酌獻,《嘉安》



 於昭祀典,致享坤儀。備物咸秩,柔祗格思。



 功宣敏樹,日益鴻禧。持載品匯,率土攸宜。



 送神,《靜安》



 妙用無方,倏來忽逝。蠲潔寅恭,式終禋瘞。



 景祐夏至祀皇地祇二首仁宗御制



 太祖奠幣,《恭安》



 赫矣淳耀,俶載帝基!一戎以定,萬國
 來儀。



 寅恭潔祀,博厚皇祗。威靈攸在,福祿如茨。



 酌獻,《英安》



 丕命惟皇,萬物咸睹。卜年邁周,崇功冠禹。



 有燁炎精,大昌聖祚。酌鬯祈年,永錫繁祜。



 熙寧祀皇地祇十二首



 迎神,《導安》



 昭靈積厚,混混坤輿。配天作極,陰慘陽舒。



 齊明薦享,百福其儲。庶幾來止,風馬雲車。



 升降,《靖安》



 有來穆穆,臨此方丘。其行風動,其止霆收。



 躬事匪懈,豐盛潔羞。百昌咸殖,允矣神休!



 奠幣《厘安》



 純誠昭融,芳美嘉薦。肅將二精,以享以奠。



 休光四充,靈祗來燕。其祥伊何?永世錫羨。



 太祖,《肇安》



 於皇烈祖,維帝所興。光輝宗祀,如日之升。



 告靈作配,孝享烝烝。錫茲祉福,百世其承。



 司徒奉俎,《承安》



 我修祀事,於何致誠?罔敢怠佚,視茲碩牲。



 納烹薦俎,侑以和聲。格哉休應,世濟皇明。



 酌獻,《和安》



 猗嗟富媼,博厚含弘。發榮敷秀,動植茲豐。



 爰酌茲酒,肸蠁交通。眾祥萃止,垂祐無窮。



 太祖,《祐安》



 光大含弘,坤元之力。海宇咸寧,烈祖之德。



 作配方壇,不僭不忒。子孫其承,毋替厥則。



 飲福,《禔安》



 載登壇阼,載酌尊彞。牲酒嘉旨,福祿純熙。



 其福維何?萬物咸宜。其祿維何?永承神禧。



 退文舞、迎武舞,《威安》



 雍雍肅肅,建我採旄。舞以玉戚,不吳不敖。



 其將其肆,脾臄嘉肴。何以侑樂?鐘鼓管簫。



 亞、終獻,《儀安》



 折俎在籩,胾羹在豆。何以酌之?酒醴是侑。



 何以錫之?貽爾眉壽。何以格之?永爾康阜。



 徹豆,《豐安》



 曳我黼黻,履舄接武。鏘我珩璜,降升圉圉。



 其將肆兮,既曰不侮。其終徹兮,恭欽惟主。



 送神,《阜安》



 神兮來下,享此苾芬。酌獻雍雍,執事孔勤。



 神之還矣,忽乘飛雲。遺我祺祥,物象忻忻。



 常祀皇地祇五首



 迎神,《寧安》八變



 坤元之德,光大無疆。一氣交感,百物阜昌。



 吉蠲致享,精明是將。介茲景福,鼎祚靈長。



 升降,《正安》



 禮經之重,祭典為宗。上公攝事,登降彌恭。



 庶品豐潔,令儀肅雍。百祥萃止,維吉之從。



 奉俎,《豐安》



 禮崇禋祀,神鑒孔明。牲牷博腯,以炰以烹。



 馨香蠲潔,品物惟精。錫以純嘏,享茲至誠。



 退文舞、迎武舞,《威安》



 進旅退旅,載揚乾揚。不愆於儀,容服有章。



 式綏式侑,神休是聽。鼓之舞之,神永安寧。



 送神,《寧安》



 物備百嘉,樂周八變。克誠是享,明德斯薦。



 神鑒孔昭,蕃禧錫羨。回馭飄然,邈不可見。



 紹興祀皇地祇十五首



 迎神,《寧安》



 函鐘為宮



 至哉厚德,物生是資!直方維則,翕闢攸宜。



 於昭祀典,致享坤儀。禮罔不答,神之格思。



 太簇為角



 蕆事方丘,舊典時式。至誠感神,馨非黍稷。



 肸蠁來臨,鑒茲明德。永錫坤珍,時萬時億。



 姑洗為征



 至哉坤元,乃順承天。厚德載物,含洪八埏。



 日北多暑,祀儀吉蠲。式昭無事,敢告恭虔。



 南呂為羽



 蕆事方丘,情文孔時。名山大澤,侑祭無遺。



 牲陳黝犢,樂備《咸池》。柔祗皆出,介我繁禧。



 盥洗,《正安》



 於穆盛禮,肅肅在宮。蕆事有初,直於東榮。



 滌濯是謹,惟寅惟清。祗薦柔嘉,享茲克誠。



 升殿,《正安》



 景風應時,聿嚴毖祀。用事方丘,鏘鏘濟濟。



 登降有節,三獻成禮。神其格思,錫我繁祉。



 正位奠玉幣,《嘉安》



 坤元博厚,對越天明。展事方澤,但惟顧歆。



 嘉玉量幣,祗薦純精。錫我繁祉,燕及函生。



 太祖位奠幣,《定安》



 毖祀泰折,柔祗是承。於赫藝祖,道格三靈。



 式嚴配侑,厚德惟寧。爰昭薦幣,享於克誠。



 捧俎,《豐安》



 丕答靈貺,歲事方丘,豆登在列,鼎俎斯儔。



 牲牷告具,寅畏彌周。柔祗昭格,飆至雲流。



 正位酌獻,《光安》



 祗事坤元,飭躬敢憚!爰潔粢盛,載嚴圭瓚。



 清明內融,嘉旨外粲。介我繁厘,時億時萬。



 太祖位酌獻,《英安》



 皇矣藝祖,九圍是式!至哉柔祗,萬匯允殖。



 保茲嘉邦,介我黍稷。酌鬯告虔,作配無極。



 文舞退、武舞進,《正安》



 於穆媼神,媲德彼天。我修毖祀,以莫不虔。



 肆陳時夏,干羽相宣。靈其來游,降福綿綿。



 亞、終獻,《文安》



 禮有祈報,國惟典常。籩豆豐潔,升降齊莊。



 備物致志,式薦累觴。昭格來享,自天降康。



 徹豆,《娛安》



 承天效法,其道貴誠。牲羞黃犢,薦德之馨。



 芳俎告畢,禮備樂盈。既靜既安,庶物沾生。



 送神,《寧安》



 至厚至深,其動也剛。精誠默通,或出其藏。



 神之言歸,化斯有光。相我炎圖,萬世無疆。



 宋初祀神州地祇三首



 降神,《靜安》



 膴□郊原,茫茫宇縣。畫野分疆,禹功疏奠。



 靈祇是臻,豆籩祗薦。幽贊皇圖,視之不見。



 奠玉幣,酌獻,《嘉安》



 肸蠁儲靈,肅恭用幣。鏘洋導和,洪休允契。



 嘉氣雲蒸,浹於華裔。式薦坤珍,聿符明世。



 送神,《靜安》



 獻奠雲畢,純嘏祁祁。威靈藏用,邈矣何之?



 景祐孟冬祭神州地祇二首



 太宗位奠幣,《化安》



 削平偽邦,嗣興鴻業。禮樂交修,仁德該洽。



 柔祗薦享,量幣攸攝。侑坐延靈,神休允答。



 酌獻,《韶安》



 有煒彌文,克隆宏構。貽此燕謀,具膺多祐。



 嶰律吹莩,彞尊奠酒。祐乃沉潛,永祈豐楙。



 元符祭神州地祇二首



 迎神,《寧安》八變



 膴□浚邦,皇天是宅。必有幽贊,聰明正直。



 布列籩豆,考擊金石。中外謐寧,繄神之力。



 送神,《寧安》



 都邑浩穰,民物富盛。主以靈祗,昭乃丕應。



 玉帛牲牷,鼓鐘筦磬。祗薦攸歆,歸於至靜。



 紹興祀神州地祇十六首



 迎神,《寧安》



 函鐘為宮



 芒芒下土,恢恢方儀。富媼統攝,
 潛運八維。



 爰稱元祀,告備吉時。揭茲虔恭,人愛其格思。



 太簇為角



 洪惟坤元,道著品物。上配紫旻,後載其德。



 良月肇蕆,祭器布列。必先皇祗,以迓景福。



 姑洗為征



 坱圠無垠,磅礡罔測。山盈川沖,自生自殖。



 其報惟何?率禮靡忒。億萬斯年,功被無極。



 南呂為羽



 翕闢以時,協氣陶蒸。播之金石,鏘厥和聲。



 冥冥眑□,孔享純誠。是聽是娭,邦基永寧。



 盥洗,《正安》



 晨煬致煙,浡然四施。飄飄風馬,徬佛來斯。



 祀事維清,沃之盥之。載涓載肅,罔有愧辭。



 升殿,《正安》



 崇崇其壇,屹矣層級。佩約步趨,降登中節。



 左瞻右睨,祥風藹集。斿旆羽紛,昭鑒翊翊。



 神州地祇位奠玉幣,《嘉安》



 璇璣諧序,籍斂薦嘉。昭答柔祗,迭奏雅歌。



 幣琮以侑,儀腆氣和。靈其溥臨,容與燕嘉。



 太示位奠幣,《嘉安》



 穆穆令聞,溥博有容。澤被萬宇,靡不率從。



 恭陳量幣,明薦其衷。禮亦宜之,享德攸同。



 奉俎,《豐安》



 肅肅嘉承,唯德其物。工祝以告,繄民之力。



 神哉廣生,孔蕃且碩。奠於嘉壇,吐之則弗。



 神州地祇位酌獻,《嘉安》



 恭承明祀,嘉薦令芳。亦有桂酒,誠愨是將。



 瑟瓚以酌,效歡厥觴。庶乎燕享,永懷不忘。



 太宗位酌獻,《化安》



 宗德含洪,方祗可儗。闢土開疆,八埏同軌。



 是用作配,有永無紀。稞獻以享,茂格蕃祉。



 文舞退、武舞進,《文安》



 奕奕綴兆,《咸池》孔彰。丕闡文德,
 靡忘發揚。



 進退有節,乃容之常。樂備爾奏,燁燁榮光。



 亞、終獻,《文安》



 縮酌以裸,既旨且多。三獻有序,情文愈加。



 黃祗臨享,錫以休嘉。廣茲靈祲,覃及邇遐。



 徹豆,《成安》



 展牲告全,乃登於俎。竣事而徹,侑以樂語。



 奉厘宣室,祚我神主。斂敷庶民,並受其祜。



 送神,《寧安》



 雲馭洋洋,既歆既顧。悠然聿歸,曷求厥路。



 欽想頌堂,跛立以慕。繼我肸蠁,莫不懌豫。



 望瘞,《正安》



 神罔怨恫,燕其有喜。蕆事告成,爰修瘞禮。



 樂闋儀備,休氣四起。尚謹不愆,念終如始。



 景德朝日三首



 降神,《高安》六變



 陽德之母,羲御寅賓。得天久照,首茲三辰。



 正辭備物,肅肅振振。淪精降監,克享明禋。



 奠玉幣酌獻,《嘉安》



 醴齊良潔,有牲斯純。大採玄冕,乃昭其文。



 王宮定位,粢盛苾芬。民事以敘,盛德升聞。



 送神,《高安》



 縣象著明,照臨下土。降福穰穰,德施周普。



 夕月三首



 降神,《高安》六變



 凝陰稟粹,照臨八埏。麗天垂象,繼日代明。



 一氣資始,四時運行。靈祗昭格,備物薦誠。



 奠玉幣、酌獻,《嘉安》



 夕耀乘秋,功存宇縣。金奏在縣,以時致薦。



 祀事孔寅,明靈降眷。潔粢豐盛,倉箱流衍。



 送神,《高安》



 夙陳籩豆,潔誠致祈。垂休保祐,景祚巍巍。



 大觀秋分夕月四首



 降神,《高安》



 至陰之精,虧而復盈。輪高仙桂,階應祥蓂。



 玉兔影孤,金莖露溢。其駕星車,顧於茲夕。



 奠玉幣



 玉鉤初彎,冰盤乍圓。扇掩秋後,烏飛枝籩。



 精凝蟾蜍,輝光嬋娟。歆於明祀,弭芳節焉。



 酌獻



 名稽《漢儀》,歌參唐宗。往於卿少,乘秋氣中。



 周天而行,如姊之崇。可飛霞佩,下琉璃宮。



 送神



 四扉大開,五雲車立。霓裾娣從,風罘童執。



 搖曳胥來,鏘洋爰集。歆我嚴禋,西面以揖。



 紹興朝日十首



 降神,《高安》



 圜鐘為宮



 玄鳥即至,序屬春分。朝於太陽,
 厥典備存。



 載嚴大採,示民有尊。揚光下燭,煜□龠東門。



 黃鐘為角



 升暉麗天,陽德之母。率無頗偏,兼燭下土。



 恭事崇壇,禮樂具舉。頓御六龍,裴回容與。



 太簇為征



 周祀及闇,漢制中營。肸蠁是屆,禮神以兄。



 我潔斯璧,我肥斯牲。神兮燕享,鑒觀孔明。



 姑洗為羽



 屹爾王宮,泛臨翊翊。惠此萬方,豈惟五色。



 以修陽政,以習地德。雲景杳冥,施祥無極。



 酌獻升殿,《正安》



 天宇四霽,嘉壇聿崇。肅祗嚴祀,登降
 有容。



 仰瞻曜靈,位居其中。既安既妥,沛哉豐融!



 奠玉幣,《嘉安》



 物之備矣,以交於神。時惟炎精,不忘顧歆。



 經緯之文,璆琳之質。燦然相輝,其儀秩秩。



 奉俎,《豐安》



 扶桑朝暾,和氣肸飭。奉此牲牢,為俎孔碩。



 芬馨進聞,介我黍稷。所將以誠,茲用享德。



 酌獻,《嘉安》



 匏爵斯陳,百味旨酒。勺以獻之,再拜稽首。



 鐘鼓在列,靈方安留。眷然加薦,惟時之休。



 亞、終獻,《文安》



 禮罄沃盥,誠意肅將。包茅是縮,冀畢重
 觴。



 煥矣情文,既具醉止。熙事備誠,靈其有喜。



 送神,《禮安》



 羲和駕兮,其容杲杲。將安之兮?言歸黃道。



 光赫萬物,無古無今。人君之表,咸仰照臨。



 夕月十首



 降神,《高安》



 圜鐘為宮



 金行告遒,玉律分秋。禮蕆西郊,毖祀聿修。



 精意潛達,永孚於休。神之聽之,爰格飆斿。



 黃鐘為角



 時維秋仲,夜寂天清。實嚴姊事,用答陰靈。



 壇壝斯設,黍稷惟馨。雲車來下,庶歆厥誠。



 太簇為征



 溯日著明,麗天作配。潔誠以祠,禮行肅拜。



 光凝冕服,氣肅環佩。庶幾昭格,祗而不懈。



 姑洗為羽



 穆穆流輝,太陰之精。盈虧靡忒,寒暑以均。



 克禋克祀,揆日涓辰。牲碩酒旨,來燕來寧。



 升殿,《正安》



 猗歟崇基,右平左戚。祗率典常,屆茲秋夕。



 陟降惟寅,威儀抑抑。神其鑒觀,穰簡是集。



 奠玉幣,《嘉安》



 少採陳儀,實曰坎祭。禮備樂舉,嚴恭將事。



 於以奠之,嘉玉量幣。神兮昭受,陰騭萬匯。



 奉俎,《豐安》



 穀旦其差,有牲在滌。工祝致告,為俎孔碩。



 肸蠁是期,祚我明德。備茲孝欽,式和民則。



 酌獻,《嘉安》



 白藏在序,享惟其時。躬即明壇,禮惟載祗。



 斟以瑤爵,神靈燕娭。歆馨顧德,錫我蕃厘。



 亞、終獻,《文安》



 肅雍嚴祀,聖治昭彰。清酒既載,或肆或將。



 禮匝三獻,終然允臧。神具醉止,其樂且康。



 送神,《理安》



 歌奏雲闋,式禮莫愆。以我齊明,罄其吉蠲。



 神保聿歸,降康自天。蘿圖永固,億萬斯年。



 熙寧以後祀高禖六首



 降神,《高安》六變



 容臺講禮,禖宮立祠。司分屆後,帶韣陳儀。



 嘉祥萃止,靈馭來思。皇支蕃衍,永固邦基。



 升降,《正安》



 郊禖之應,肇自生商。誕膺寶命,浚發其祥。



 天材蕃衍,德稱君王。本支萬世,與天無疆。



 奠玉幣,《嘉安》



 昔帝高辛,先禖肇祀。爰揆仲陽,式祈嘉祉。



 陳之犧牲,授以弓矢。敷祐皇宗,施於孫子。



 酌獻,《祐安》



 昭薦精衷,靈承端命。青帝顧懷,神禖儲慶。



 祚以蕃昌,協於熙盛。螽斯眾多,流於雅詠。



 亞、終獻,《文安》



 赫赫高禖,萬世所祀。其德不回,錫茲福祉。



 蕃衍椒聊,和平芣苡。傳類降康,世濟其美。



 送神,《理安》



 禮奠蠲衷,祭儀竣事。丕擁靈休,蕃衍皇嗣。



 紹興祀高禖十首



 降神,《高安》



 圜鐘為宮



 聿分春氣,施生在時。禖宮肇啟,精意以祠。



 禮儀告備,神其格思!厥靈有赫,錫我繁厘。



 黃鐘為角



 眷此尊祀,實惟仲春。青圭束帛,克祀克
 禋。



 庶蒙嘉惠,嗣續詵詵。神之降鑒,雲車來臻。



 太簇為征



 猗歟禖宮,祀典所貴。粵自艱難,禮或弗備。



 以迄於今,始建壇壝。願戒云車,歆此誠意。



 姑洗為羽



 春氣肇分,萬類滋榮。惟此祀事,皆象發生。



 求神以類,式昭至誠。庶幾來格,子孫繩繩。



 升壇,《正安》



 有奕禖宮,在國之南。壇壝既設,威儀孔嚴。



 登祀濟濟,神兮顧瞻。佐我皇祚,宜百斯男。



 奠玉幣,《嘉安》



 青律載陽,有鳦頡頏。祈我繁祉,立子生
 商。



 三牲既薦,玉帛是將。克禋克祀,有嘉其祥。



 奉俎,《豐安》



 祗祓禖壇,潔蠲羊豕。博碩肥腯,爰具牲醴。



 執事駿奔,肅將俎幾。神其顧歆,永錫多子。



 青帝位酌獻,《祐安》伏羲、高辛酌獻並同



 瑞鳦至止,祀事孔時。酌以清酒,稞獻載祗。



 神具醉止,介我蕃禧。乃占吉夢。維熊維羆。



 亞、終獻,《文安》



 中春涓吉,蕆事禖祠。禮備樂作,籩豆孔時。



 貳觴畢舉,薦獻無違。庶幾神惠,祥啟熊羆。



 送神,《理安》



 嘉薦令芳,有嚴禋祀。神來燕娭,亦即醉止。



 風馭言還,慄然欻起。以祓以除,錫我蕃祉。



 景德祀九宮貴神三首



 降神,《高安》



 倬彼垂象,照臨下土。躔次運行,功德周普。



 九宮即位,惟德是輔。神之至上,皇皇斯睹。



 奠玉幣,酌獻,《嘉安》



 靈禋既肅,明神既秩。在國之東,協日之吉。



 升歌有儀,六變中律。懷和萬靈,降茲陰騭。



 送神,《高安》



 祗薦有常,惟神無方。回飆整馭,垂休降祥。



 元祐祀九宮貴神二首



 降神,《景安》六變



 上天貴神,九宮設位。功德及物,乃秩明祀。



 望拜紫壇,赫然靈氣。奠玉薦幣,歆之無愧。



 送神,《景安》



 天之貴神,推移九宮。厥位靡常,降康則同。



 來集於壇,顧歆恪恭。歌以送之,飆靜旋穹。



 紹興祀九宮貴神十首



 降神,《景安》



 圜鐘為宮



 紫闕幽宏,惟神靈尊。輔成泰元,贊役乃坤。



 曰雨曰暘,縕豫調紛。享薦隕光,蒙祉如屯。



 黃鐘為角



 載陽衍德,農祥孔昭。繼茲元嘏,穰穰黍苗。



 象輿眇冥,金奏遠姚。無閼厥靈,丹衷匪恌。



 太簇為征



 於赫九宮,天神之貴。煌煌彪列,下土是蒞。



 幽贊高穹,陰騭萬類。肅若舊典,有嚴祗事。



 姑洗為羽



 練時吉良,聿崇明祀。粢盛潔豐,牲碩酒旨。



 肅唱和聲,來燕來止。嘉承天休,繼及含齒。



 初獻升壇,《正安》



 於昭毖祀,周旋有容。歷階將事,趨進鞠躬。



 改步如初,沒階彌恭。左戚右平,陟降雍雍。



 太一位奠玉幣,《嘉安》



 煌煌九宮,照臨下土。陰騭庶類,功施周普。



 恪修祀典,禮備樂舉。嘉玉量幣,馨非稷黍攝提、權星、招搖、天符、青龍、咸池、太陰、天乙位樂曲並同。



 奉俎,《豐安》



 靈鑒匪遠,誠心肅祗。是烝是享,俎實孔時。



 禮行樂奏,肸蠁是期。雲車風馬,神其燕娭。



 太一位酌獻,《嘉安》



 惟天丕冒,彪列九神。財成元化,陰騭下民。



 有酒斯旨,登薦苾芬。昭哉降鑒,茀祿來臻九位並同。



 亞、終獻,《文安》



 均調大化,陰騭下民。駿功有赫,誕舉明禋。



 嘉觴中貳,執事惟寅。清明鬯矣,福祿攸臻。



 送神,《景安》



 薦獻有序,降登無違。禮樂備舉,昭格燕娭。



 雲車縹緲,神曰還歸。報以景貺,翊我昌期。



\end{pinyinscope}