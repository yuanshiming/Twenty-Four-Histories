\article{列傳第二十五}

\begin{pinyinscope}

 錢若水從弟若沖蘇易簡郭贄李至辛仲甫王沔溫仲舒王化基子舉正舉元孫詔



 錢若水,字澹成,一字長卿,河南新安人。父文敏,漢青州
 帥劉銖闢為錄事參軍,歷長水禜都尉、扶風令、相州錄事參軍。先是,府帥多以筆牘私取官庫錢,韓重贇領節制,頗仍其弊。文敏不從,重贇假他事廷責之,文敏不為屈。太祖嘉其有守,授右贊善大夫、知瀘州,召見講武殿,謂曰:「瀘州近蠻境,尤宜綏撫。聞知州郭思齊、監軍郭重遷掊斂不法,恃其荒遠,謂朝廷不知爾。至,為朕鞫之,茍一毫有侵於民,朕必不赦。」至郡,有政跡,夷人詣闕借留。詔改殿中丞,許再任。三遷司封員外郎,又知洺州、建昌
 軍。卒,年七十二。



 若水幼聰悟,十歲能屬文。華山陳摶見之,謂曰:「子神清,可以學道;不然,當富貴,但忌太速爾。」雍熙中,舉進士,釋褐同州觀察推官,聽決明允,郡治賴之。淳化初,寇準掌選,薦若水洎王扶、程肅、陳充、錢熙五人文學高第,召試翰林,若水最優,擢秘書丞、直史館。歲餘。遷右正言、知制誥。會置理檢院於乾元門外,命若水領之。俄同知貢舉,加屯田員外郎。詔詣原、鹽等州制置邊事,還奏合旨,翌日改職方員外郎、翰林學士,與張洎並
 命。俄知審官院、銀臺通進封駁司。嘗草賜趙保忠詔,有云:「不斬繼遷,開狡兔之三穴,潛疑光嗣,持首鼠之兩端。」太宗大以為當。



 至道初,以右諫議大夫同知樞密院事。真宗即位,加工部侍郎。數月,以母老上章,求解機務,詔不許。若水請益堅,遂以本官充集賢院學士、判院事。俄詔修《太宗實錄》,若水引柴成務、宗度、吳淑、楊億同修,成八十卷。真宗覽書流涕,錫賚有差。



 初,太宗有畜犬甚馴,常在乘輿左右。及崩,嗚號不食,因送永熙陵寢。李至嘗
 詠其事,欲若水書之以戒浮俗,若水不從。呂端雖為監修,以不蒞局不得署名,至抉其事以為專美。若水稱詔旨及唐朝故事以折之,時議不能奪。既又重修《太祖實錄》,參以王禹偁、李宗諤、梁顥、趙安仁,未周歲畢。安仁時為宗正卿,上言夔王於太宗屬當為兄,《實錄》所紀繆誤。若水援國初詔令,廷諍數四乃定。



 俄判吏部流內銓。從幸大名,若水陳禦敵安邊之策,有曰:



 孫武著書,以伐謀為主;漢高將將,以用法為先。伐謀者,以將帥能料敵制
 勝也;用法者,以朝廷能賞罰不私也。今傅潛領雄師數萬,閉門不出,坐視邊寇俘掠生民,上孤委注之恩,下挫銳師之氣,蓋潛輩不能制勝,朝廷未能用法使然也。軍法,臨陣不用命者斬。今若斬潛以徇,然後擢如楊延朗、楊嗣者五七人,增其爵秩,分授兵柄,使將萬人,間以強弩,分路討除,孰敢不用命哉?敵人聞我將帥不用命,退則有死,豈獨思遁,抑亦來歲不敢犯邊矣。如此則可以坐清邊塞,然後鑾輅還京,天威懾於四海矣。



 臣嘗讀前
 史,周世宗即位之始,劉崇結敵入寇,敵遣其將楊袞領騎兵數萬,隨崇至高平。當時懦將樊愛能、何徽等臨敵不戰,世宗大陳宴會,斬愛能等,拔偏將十餘人,分兵擊太原。劉崇聞之,股心慄不敢出,即日遁去。自是兵威大振。其後收淮甸,下秦、鳳,平關南,特席卷爾。以陛下之神武,豈讓世宗乎?此今日禦敵之奇策也。



 若將來安邊之術,請以近事言之,太祖朝制置最得其宜。止以郭進在邢州,李漢超在關南,何繼筠在鎮定,賀惟忠在易州,李謙
 溥在隰州,姚內斌在慶州,董遵誨在通遠軍,王彥升在原州,但授緣邊巡檢之名,不加行營部署之號,率皆十餘年不易其任。立邊功者厚加賞賚,其位皆不至觀察使。蓋位不高則朝廷易制,任不易則邊事盡知。然後授以聖謀,來則掩殺,去則勿追,所以十七年中,北邊、西蕃不敢犯塞,以至屢使乞和,此皆陛下之所知也。茍能遵太祖故事,慎擇名臣,分理邊郡;罷部署之號,使不相統轄;置巡檢之名,俾遞相救應。如此則出必擊寇,入則守
 城,不數年間,可致邊烽罷警矣。



 俄知開封府。時北邊未寧,內出手札訪若水以策。若水陳備邊之要有五:



 一曰擇郡守,二曰募鄉兵,三曰積芻粟,四曰革將帥,五曰明賞罰。



 何謂擇郡守?今之所患,患在戰守不同心。望陛下選沉厚有謀諳邊事者,任為邊郡刺史,令兼緣邊巡檢,許召勇敢之士為隨身部曲。廩贍不充則官為支給。然後嚴亭障,明斥候,每得事宜,密相報示。寇來則互為救應,齊出討除;寇去則不令遠追,各務安靜。茍無大過,勿
 為替移;儻立微功,就加爵賞。如此則戰守必能同心,敵人不敢近塞矣。



 何謂募鄉兵?今之所患,患在不知敵情。望詔逐州沿邊民為招收軍,給與糧賜,蠲其賦租。彼兩地之中,各有親族,使其懷惠,來布腹心。彼若舉兵,此必預知,茍能預知,則百戰百勝矣。



 何謂積芻粟?今之所患,患在困民力。望陛下令緣邊各廣營田,以州郡長官兼其使額,每歲秋夏,較其課程,立鼓旗以齊之,行賞罰以勸之。仍縱商人入粟緣邊。儻鎮戍有三年之備,則敵人
 不敢動矣。



 何謂革將帥?今之所患,患在重兵居外,輕兵居內。去歲傅潛以八萬騎屯中山,魏、博之間鎮兵全少,非鑾輅親征,則城邑危矣。望陛下慎選將臣任河北近鎮,仍依舊事節制邊兵,未能削部署之名,望且減行營之號;有警則暫巡邊徼,無事則卻復舊藩。豈惟不啟戎心,況復待勞以逸。如此則不失備邊之要,又無舉兵之名,且使重兵不屯一處,進退動靜,無施不可矣。



 何謂明賞罰?今之所患,患在戎卒驕惰。臣自知府以來,見侍衛、
 殿前兩司送到邊上亡命軍卒,人數甚多。臣試訊之,皆以思親為言,此蓋令之不嚴也。平時尚敢如此,況臨大敵乎?望陛下以此言示將帥,俾申嚴號令,以警其下。古人云:「賞不勸謂之止善,罰不懲謂之縱惡。」又曰:「法不可移,令不可違。」臣嘗聞郭進出鎮西山,太祖每遣戍卒,必諭之曰:「汝等謹奉法。我猶赦汝,郭進殺汝矣。」其假借如此,故郭進所至,未嘗少衄。陛下能鑒前日之事,即今日之元龜也。



 若水又言:「邊部用兵,唯視太白與月為進退
 者,誠以太白者將軍也,星辰者廷尉也。合則有戰,不合則無戰;合於東則主勝,合於西則客勝。陛下能用臣言以謹邊備,則邊部不召而自來矣。太祖臨御十七年間,未嘗生事疆埸,而敵人往往遣使乞和者,以其任用得人而備御有方也。陛下茍思兵者兇器,戰者危事,而不倒持太阿,授人以柄,則守在四夷,而常獲靜勝,此備御之上策也。」



 未幾,出知天雄軍兼兵馬部署。時言事者請城綏州,屯兵積穀以備黨項。邊城互言利害,前後遣使
 數輩按視,不能決。時已大發丁夫,將興其役,詔若水自大名馳往視之。若水上言:「綏州頃為內地,民賦登集,尚須旁郡轉餉。自賜地趙保忠以來,人戶凋殘,若復城之,即須增戍。芻糧之給,全仰河東。其地隔越黃河、鐵碣二山,無定河在其城下,緩急用兵,輸送艱阻。且其地險,若未葺未完,邊寇奔沖,難於固守。況城邑焚毀,片瓦不存,所過山林,材木匱乏。城之甚勞,未見其利。」復詣闕面陳其事,上嘉納之,遂罷役。初,若水率眾過河,分布軍伍,咸
 有節制,深為戍將推服。上謂左右曰:「若水,儒臣中知兵者也。」是秋,又遣巡撫陜西緣邊諸郡,令便宜制置邊事。還拜鄧州觀察使、並代經略使、知並州事。



 六年春,因疾灸兩足,創潰出血數斗,自是體貌羸鼛,手詔慰勞之,俾歸京師。數月,始赴朝謁,因與僚友會食僧舍,假寢而卒,年四十四。贈戶部尚書,賜其母白金五百兩。子延年甫七歲,錄為太常奉禮郎。



 若水美風神,有器識,能斷大事,事繼母以孝聞。雅善談論,尤輕財好施。所至推誠待物,
 委任僚佐,總其綱領,無不稱治。汲引後進,推賢重士,襟度豁如也。精術數,知年壽不永,故懇避權位。其死也,士君子尤惜之。有集二十卷。



 兄若愚,比部員外郎。從弟若沖,大中祥符中,調河陽令。有僕酗酒,杖之百數。僕挾刀夜潛室中,斷其臂,若沖大呼;又害其幼子。詔磔僕於其門。真宗念若水母老,遣使存問,賜緡、綿、羊、酒;且賜若沖帛三十端,補孟州別駕。延年後以獻文賜進士出身,歷太常博士、集賢校理。



 蘇易簡,字太簡,梓州銅山人。父協舉蜀進士,歸宋,累任州縣,以易簡居翰林,任開封縣兵曹參軍,俄遷光祿寺丞,卒,特贈秘書丞。



 易簡少聰悟好學,風度奇秀,才思敏贍。太平興國五年,年逾弱冠,舉進士。太宗方留心儒術,貢士皆臨軒覆試。易簡所試三千餘言立就,奏上,覽之稱賞,擢冠甲科。解褐將作監丞,通判升州,遷左贊善大夫。八年,以右拾遺知制誥。雍熙初,以郊祀恩進秩祠部員外郎。二年,與賈黃中同知貢舉。有詔,凡親屬就舉者,
 籍名別試。易簡妻弟崔範,匿父喪充貢,奏名在上第;又王千里者,水部員外郎孚之子,協為孚門生,千里預薦。上聞,坐範及千里罪。易簡緣是罷知制誥,以本官奉朝請。未幾,復知制誥。三年,充翰林學士。初,易簡充貢,宋白掌貢部,至是裁七年。易簡幼時隨父河南,賈黃中來使,嘗教之屬辭;及是,悉為同列。易簡連知貢舉,陳堯叟、孫何並甲廷試。



 淳化元年,丁外艱。二年,同知京朝官考課,遷中書舍人,充承旨。先是,曲宴將相,翰林學士皆預坐,
 梁迥啟太祖罷之;又皇帝御丹鳳樓,翰林承旨侍從升樓西南隅,禮亦廢。至是,易簡請之,皆復舊制。易簡續唐李肇《翰林志》二卷以獻,帝賜詩以嘉之。帝嘗以輕綃飛白大書「玉堂之署」四字,令易簡榜於廳額。易簡會韓伾、畢士安、李至等往觀。上聞,遣中使賜宴甚盛,至等各賦詩紀其事,宰相李昉等亦作詩頌美之。他日,易簡直禁中,以水試欹器。上密聞之,因晚朝,問曰:「卿所玩得非欹器耶?」易簡曰:「然,江南徐邈所作也。」命取試之。易簡奏曰:「
 臣聞日中則昃,月滿則虧,器盈則覆,物盛則衰。願陛下持盈守成,慎終如始,以固丕基,則天下幸甚。」



 會郊祀,充禮儀使。先是,扈蒙建議以宣祖升配。易簡引唐故事,請以宣祖、太祖同配。從之。知審官院,言初任京朝官,未嘗歷州縣,不得擬知州、通判。詔可。改知審刑院,俄掌吏部選,遷給事中、參知政事。時趙昌言亦參知政事,與易簡不協,至忿爭上前,上皆優容之。未幾,昌言出使劍南,中路命改知鳳翔府。明年,易簡亦以禮部侍郎出知鄧州,
 移陳州。至道二年,卒,年三十九,贈禮部尚書。



 易簡外雖坦率,中有城府。由知制誥入為學士,年未滿三十。屬文初不達體要,及掌誥命,頗自刻勵。在翰林八年,眷遇□絕倫等。李沆後入,在易簡下,先參知政事,故以易簡為承旨,錫賚均焉。太宗遵舊制,且欲稔其名望而後正臺輔,易簡以親老急於進用,因亟言時政闕失,遂參大政。



 蜀人何光逢,易簡之執友也,嘗任縣令,坐賂削籍,流寓京師。會易簡典貢部,光逢代人充試以取貲,易簡於稠
 人中屏出之。光逢遂造謗書,斥言朝廷事,且譏易簡。易簡得其書以聞,逮捕光逢,獄具,坐棄市。易簡以殺光逢非其意,居常怏怏。母薛氏以殺父執切責之,易簡泣曰:「不渭及此易簡罪也。」及易簡參知政事,召薛氏入禁中,賜冠帔,命坐,問曰:「何以教子成此令器?」對曰:「幼則束以禮讓,長則教以詩書。」上顧左右曰:「真孟母也。」



 易簡性嗜酒,初入翰林,謝日飲已微醉,餘日多沉湎。上嘗戒約深切,且草書《勸酒》二章以賜,令對其母讀之。自是每入
 直,不敢飲。及卒,上曰:「易簡果以酒死,可惜也。」易簡常居雅善筆札,尤善談笑,旁通釋典,所著《文房四譜》、《續翰林志》及《文集》二十卷,藏於秘閣。三子,曰宿、曰壽、曰耆,大中祥符間,皆祿之以官云。



 郭贄,字仲儀,開封襄邑人。乾德中,舉進士,中首薦。太宗尹京,因事藩邸。太平興國初,擢為著作佐郎、右贊善大夫。俄兼皇子侍講,賜緋魚。太宗至東宮,出《戒子篇》命贄批注,且令委曲講說,以喻諸王。三年,與劉兼、張洎、王克
 正同知貢舉,遷右補闕,與宋白並拜中書舍人,賜金紫。五年,復與程羽、侯陟、宋白同知貢舉。置京朝官差遣院,凡將命出入、受代歸闕官,悉考校勞績,銓量才品,命贄、洎、滕中正、雷德驤領之。



 七年,以本官參知政事。曹彬為弭德超所誣,贄極言救解,深為宰相趙普所重。嘗因論事奏曰:「臣受不次之遇,誓以愚直上報。」太宗曰:「愚直何益於事?」贄言:「雖然,猶勝奸邪。」



 無何,以入對宿酲未解,左遷秘書少監、知荊南府。府俗尚淫祀,屬久旱,盛陳
 禱雨之具。贄始至,命悉撤去,投之江,不數日大雨。就加左諫議大夫,入為鹽鐵使。時諸路積逋欠犯人,雖死猶系其子孫。贄條陳其事,多所蠲貸。籍田,超拜工部侍郎。淳化中,知澶州,坐河決免所居官。久之,起為給事中,復工部侍郎,知審官院、通進銀臺封駁司。



 真宗即位,拜刑部,出知天雄軍。翌日,贄入對,懇辭。上曰:「全魏之地,所寄尤重,卿宜亟去。」入判太常寺、吏部流內銓,加集賢院學士、判院事。知河南府,歸朝,獻詩自陳,進秩吏部,俄兼秘書監。



 初,真宗未出閣,贄已授經,上嘗至其家;後楊可法繼其任,上以為輔導不及贄,嘗稱贄純厚長者。至是,在秘府,屢賜對,詢訪舊事。且愍其已老,特拜工部尚書、翰林侍讀學士,作詩賜之,有「啟發沖言曉典常」語。東封,遷禮部尚書。太宗在晉邸時,凡制篇詠,多令屬和。真宗嘗訪其賜本,贄集為四卷以獻,詔獎之。大中祥符三年,卒,年七十六。上以舊學之故,特親臨哭之,贈左僕射,謚文懿。錄其子昭度為大理寺丞,昭升、昭用並大理評事,昭允左
 贊善大夫。



 贄屬文敏速而不雕刻,昭度集為三十卷上之,賜名《文懿集》。性溫和,頗能延譽時雋。宋白以文學沉下位,贄薦引之,遂同掌誥命。趙昌言兒時,一見器之,及掌貢部,以為奏名之首,後卒貴顯。贄初充賦有聲,邑人同在籍中者忌之,潛加構毀,自是連上不中選。洎贄再知貢舉,邑人子以明經充薦,詔下日,悔泣而去。贄聞之,命其所親召還,慰諭俾就舉,遂預薦中第。然吝嗇,切於治生,晚節不事事,人頗以是少之。



 李至,字言幾,真定人。母張氏,嘗夢八仙人自天降,授字圖使吞之,及寤,猶若有物在胸中,未幾,生至。七歲而孤,鞠於飛龍使李知審家。幼沉靜好學,能屬文。及長,辭華典贍。舉進士,釋褐將作監丞,通判鄂州。旋擢著作郎、直史館。會征太原,命督澤、潞芻糧,累遷右補闕、知制誥。太平興國八年,轉比部郎中,為翰林學士。冬,拜右諫議大夫、參知政事。



 雍熙初,加給事中。時議親征範陽,至上疏以為:「兵者兇器,戰者危事,用之之道,必務萬全。幽州為
 敵右臂,王師所向,彼必拒張,攻城數萬,兵食倍之。今日邊庾未充,況範陽之傍,坦無陵阜,去山既遠,取石尤難。金湯之堅,必資機石,儻有未備,願且繕完。畜威養銳,觀釁以伐謀,更縱彌年,亦未為晚。必若聖心獨斷,在於必行,則京師天下之本,陛下恭守宗廟,不離京國,示敵人以閑暇,慰億兆之仰望,策之上也。大名,河朔之咽喉,或暫駐鑾輅,揚言自將,以壯軍威,策之中也。若乃遠提師旅,親抵邊陲,北有契丹之虞,南有中原之慮,則曳裾之
 懇切,斷鞅之狂愚,臣雖不肖,恥在二賢後也。」至以目疾累表求解機政,授禮部侍郎,進秩吏部。



 會建秘閣,命兼秘書監,選三館書置閣中,俾至總之。至每與李昉、王化基等觀書閣下,上必遣使賜宴,且命三館學士皆與焉。至是升秘閣,次於三館,從至請也。上嘗臨幸秘閣,出草書《千字文》為賜,至勒石,上曰:「《千文》乃梁武得破碑鐘繇書,命周興嗣次韻而成,理無足取。若有資於教化,莫《孝經》若也。」乃書以賜至。薦潘慎修、舒雅、杜鎬、吳淑等入
 充直館校理。請購亡書,間以新書奏御,必便坐延見,恩禮甚厚。淳化五年,兼判國子監。至上言:「《五經》書疏已板行,惟二《傳》、二《禮》、《孝經》、《論語》、《爾雅》七經疏未備,豈副仁君垂訓之意。今直講崔頤正、孫奭、崔偓佺皆勵精強學,博通經義,望令重加讎校,以備刊刻。」從之。後又引吳淑、舒雅、杜鎬檢正訛謬,至與李沆總領而裁處之。



 至道初,真宗初正儲位,以至與李沆並兼賓客,詔太子事以師傅禮。真宗每見必先拜,至等上表,不敢當禮。詔答曰:「朕旁稽
 古訓,肇建承華,用選端良,資於輔導。藉卿宿望,委以護調,蓋將勖以謙沖,故乃異其禮數。勿飾當仁之讓,副予知子之心。」至等相率謝。太宗謂曰:「太子賢明仁孝,國本固矣。卿等可盡心規誨,若動皆由禮,則宜贊助,事有未當,必須力言。至於《禮》、《樂》、《詩》、《書》義有可裨益者,皆卿等素習,不假朕之言諭也。」



 真宗即位,拜工部尚書、參知政事。一日,上訪以靈武事,至上疏曰:「河湟之地,夷夏雜居,是以先王置之度外。繼遷異類,騷動疆埸,然臍不足弭其
 患,擢發不足數其罪。然聖人之道,務屈己含垢以安億民,蓋所損者小,所益者大。望陛下以元元為念,不以巨憝介意。料彼脅從亦厭兵久矣,茍朝廷舍之不問,啖以厚利,縻以重爵,亦安肯迷而不復訖於淪胥哉?昨鄭文寶絕青鹽使不入漢界,禁粒食使不及羌夷,致彼有詞,而我無謂,此之失策,雖悔何追。今若復禁止不許通糧,恐非制敵懷遠、不戰屈人之意。昔唐代宗雖罪田承嗣而不禁魏鹽,陛下宜行此事,以安邊鄙。使其族類有無
 交易,售鹽以利之,通糧以濟之,彼雖遠夷,必然向化,互相誥諭。一旦懷恩,舍逆效順,則繼遷豎子孤而無輔,又安能為我蜂蠆哉!今靈州不可不棄,非獨臣愚以為當然,若移朔方軍額於環州,亦一時之權也。或指靈州為咽喉之地,西北要沖,安可棄之以為敵有,此不智之甚,非臣之所敢知也。」後靈武卒不能守。



 咸平元年,以目疾求解政柄,授武信軍節度,入辭節制,不允。居二年,徙知河南府。四年,以病求歸本鎮,許之。詔甫下,卒,年五十五。
 贈侍中,詔給其子惟良、惟允、惟熙等奉終制。



 至嘗師徐鉉,手寫鉉及其弟鍇集,置於幾案。又賦《五君詠》,為鉉及李昉、石熙載、王祐、李穆作也。至剛嚴簡重,人士罕登其門。性吝嗇。幼育於知審,及貴,即逐其養子以利其資。知審因至亦至右金吾衛大將軍。



 辛仲甫,字之翰,汾州孝義人。曾祖實,石州推官。祖迪,壽陽令。父藩,河東節度判官。仲甫少好學,及長,能吏事,偉姿儀,器局沉厚。周廣順中,郭崇掌親軍,領武定節制,
 置仲甫掌書記。顯德初,出鎮澶淵,仍署舊職。崇所親吏為廂虞候,部民有被劫殺者,訴陰識賊魁,即捕盜吏也,官不敢詰。仲甫請自捕逮,鞫之,吏故稽其獄,仲甫曰:「民被寇害而使自誣服,蠹政甚矣,焉用僚佐為?」請易吏以雪冤憤。崇悟,移鞫之,乃得實狀。崇移鎮真定,改深、趙、鎮觀察判官。



 太祖受命,以崇為監軍。陳思誨密奏崇有奸狀,上怒且疑,遣中使馳往驗之。未至,崇憂懣失據,謂賓佐曰:「茍主人不察,為之奈何?」皆愕相視。仲甫曰:「皇帝膺運,
 公首效節,軍民處置,率循常度,且何以加辭。第遠偵使者,率僚屬盡郊迎禮,聽彼伺察,久當自辨矣。」崇如其言。使者至,視崇無他意,還奏,上大喜,歸罪於思誨。仲甫又隨崇為平盧軍節度判官。崇卒,改鄆、齊觀察判官,累雪冤枉。



 乾德五年,入拜右補闕,出知光州。州有橫河與城直,會霖潦暴疾,水溢潰廬舍。仲甫集船數百艘,軍資民儲,皆賴以濟。六年,移知彭州。州卒誘營兵及諸屯戍,謀以長春節宴集日為亂。屬春初,仲甫出城巡視,見壕中
 草深,意可藏伏,命燒薙之。兇黨疑謀洩,有自首者。禽百餘人,盡斬之。先是州少種樹,暑無所休。仲甫課民栽柳蔭行路,郡人德之,名為「補闕柳」。太祖問群臣文武兼資者為誰,趙普以仲甫對。徙益州兵馬都監,代還,選為三司戶部判官。



 太平興國初,遷起居舍人,奉使契丹。遼主問:「黨進何如人?如進之比有幾?」仲甫曰:「國家名將輩出,如進鷹犬材耳,何足道哉!」遼主欲留之,仲甫曰:「信以成命,義不可留,有死而已。」遼主竟不能屈。使還,以刑部郎
 中知成都府。既至,奏免歲輸銅錢,罷榷酤,政尚寬簡,蜀人安之。八年,加右諫議大夫。時彭州盜賊連結為害,詔捕未獲。仲甫誘令自縛詣吏者凡百餘人,餘因散去。



 九年,入知開封府,拜御史中丞。雍熙二年,拜給事中、參知政事。端拱中,進戶部侍郎。時呂蒙正以長厚居相位,王沔任事,仲甫從容其間而已。淳化二年,以足疾罷為工部尚書,出知陳州。代歸,會蜀有寇,以仲甫素著恩信,將令輿疾招撫,以疾未行。無何,以太子少保致仕。真宗即
 位,加太子少傅。咸平三年,卒,年七十四,贈太子太保。子若沖、若虛、若蒙、若濟、若渝,皆能其官。孫有孚、有鄰,俱中進士。



 王沔,字楚望,齊州人。太平興國初,舉進士,解褐大理評事。四年,太宗親征太原,見於行在,授著作郎、直史館。遷右拾遺,出為京西轉運副使。明年,加右補闕、知懷州。八年春,與宋白、賈黃中等同知貢舉,擢膳部郎中、樞密直學士。遷右諫議大夫、同簽書樞密院事,賜第崇德坊。雍
 熙元年,加左諫議大夫、樞密副使。端拱初,改戶部侍郎,參知政事。



 淳化初,宰相趙普出守西洛。呂蒙正以寬簡自任,政事多決於沔,沔與張齊賢同掌樞務,頗不葉。齊賢出知代州,沔遂為副使,參預政事。陳恕好苛察,亦嘗與沔忤。淳化二年,齊賢洎恕參知政事,沔不自安,慮僚屬有以中書舊事告齊賢等。會左司諫王禹偁上言:「自今宰相及樞密使不得於本廳見客,許於都堂延接。」沔喜,即奏行之。直史館謝泌以為如此是疑大臣以私也,
 疏駁之。太宗追還前詔,沔暨恕因是罷守本官。翌日,蒙正亦罷。沔見上,涕泣,不願離左右。未幾,須鬢皆白。會省吏事發,連中書,因有奏毀者。上語毀者曰:「呂蒙正有大臣體,王沔甚明敏。」毀者慚而止。



 三年,上欲黜陟官吏,命沔與謝泌、王仲華同知京朝官考課。沔上言,應京朝官殿犯,乞令刑部條報,以贓及公私罪分三等以聞。立法苛察,欲因是以求再用。受命甫旬日,方視事,以暴疾卒,年四十三,贈工部尚書。



 沔聰察敏辯,有適時之用,上前
 言事,能委曲敷繹。每對御讀所試進士辭賦,音吐明暢,經讀者多中高第。性苛刻,少誠信。掌機務日,凡謁見者必啖以甘言,皆喜過望,既而進退非允,人胥怨之。



 沔弟淮,太平興國五年進士,任殿中丞。嘗掌香藥榷易院,坐臟論當棄市,以沔故,詔杖一百,降定遠主簿。沔以是頻為寇準所詆云。



 溫仲舒,字秉陽,河南人。太平興國二年,舉進士,為大理評事,通判吉州。再遷秘書丞、知汾州,坐事除名。未幾,復
 起為右贊善大夫,通判睦州。端拱初,拜右正言、直史館、判戶部憑由司。三年,拜工部郎中、樞密直學士,知三班院。秋,彗星見,召對別殿,仲舒以為「國家平太原以來,燕、代之交,城守年深,殺傷剽掠,彼此迭見。大河以北,農桑廢業,戶口減耗。凋弊之餘,極力奉邊。丁壯備徭,老弱供賦。遺廬壞堵,不亡即死。邪人媚上,猶云樂輸。加以兵卒踐更,行者辛苦,居者怨曠。願推恩宥,以綏民庶。」太宗嘉納之,遂赦河北。



 淳化二年,拜右諫議大夫、樞密副使,改
 同知樞密院事。四年,罷知秦州。先是,俗雜羌、戎,有兩馬家、朵藏、梟波等部,唐末以來,居於渭河之南,大洛、小洛門砦,多產良木,為其所據。歲調卒採伐給京師,必以貲假道於羌戶。然不免攘奪,甚至殺掠,為平民患。仲舒至,部兵歷按諸砦,諭其酋以威信,諸部獻地內屬。既而悉徙其部落於渭北,立堡砦以限之。民感其惠,為畫像祠之。會有言仲舒生事者,上謂近臣曰:「仲舒嘗總機密之職,在吾左右,當以綏懷為務。古伊、洛之間,尚有羌、渾
 雜居,況此羌部內屬,素居渭南,土著已久,一旦擅意斥逐,或至騷動,又煩吾關右之民。」乃命知鳳翔薛惟吉與仲舒對易其任。連知興元、江陵二府,加給事中。會內侍藍繼宗使秦州還,言得地甚利。乃召仲舒,拜戶部侍郎,尋參知政事。二砦後為內地,歲獲巨木之利。



 咸平初,拜禮部尚書,罷政,出知河陽。逾年,知開封府。五年,以京府務劇求罷,遂以本官兼御史中丞,尋遷刑部尚書、知天雄軍,徙河南。景德中,並州缺守,上以北門重鎮須大臣
 鎮撫,非張齊賢、溫仲舒不可,令宰相諭旨,皆不願往。未幾,復知審官院。大中祥符中,進秩戶部尚書。三年,判昭文館大學士,命下,卒,年六十七。贈左僕射,謚恭肅。



 仲舒敏於應務。少與呂蒙正契厚,又同登第。仲舒黜廢累年,蒙正居中書,極力援引,及被任用,反攻蒙正,士論薄之。自為正言至貳樞密,皆與寇準同進,時人謂之「溫寇」。子嗣宗、嗣良、嗣先、嗣立。仲舒既卒,帝憫其孤弱,並祿以官。



 王化基,字永圖,鎮定人。太平興國二年,舉進士,為大理
 評事,通判常州。遷太子右贊善大夫、知嵐州。時趙普為相,建議以驟用人無益於治,改淮南節度判官,入為著作郎,遷右拾遺,抗疏自薦。太宗覽奏曰:「化基自結人主,慷慨之士也。」召試,知制誥,以右諫議大夫權御史中丞。一日,侍便殿,問以邊事,對曰:「治天下猶植木焉,所患根本未固,固則枝幹不足憂。朝廷治,則邊鄙何患乎不安?」又嘗令薦士,即一疏數十人,王嗣宗、薛映、耿望,皆其人也。



 化基嘗慕範滂為人,獻《澄清略》,言時事有五:



 其一,
 復尚書省,曰:國家立制,動必法天。尚書省上應玄象,對臨紫垣,故六卿擬喉舌之官,郎吏應星辰之位,斯實乾文昭著,故事具明。方今省署,名實未稱。夫三司使額,乃近代權制;判官、推官、勾院、開拆、磨勘、憑由、理欠、孔目、勾押、前後行,皆州郡吏局之名。請廢三司,止於尚書省設六尚書分掌其事;廢判官、推官,設郎官分掌二十四司及左右司公事,使一人掌一司;廢孔目、勾押、前後行為都事、主事、令史;廢勾院、開拆、磨勘、憑由、理欠等司歸比部
 及左右司。如此即事益精詳,且盡去州郡吏局之名也。六卿如闕,即選名品相近、有才望者權之;郎官如闕,則於兩省三院選名乾有清望者,依資除之。其二十四司公事,若繁簡不同,望下本省府屬參酌其類,均而行之。



 其二,慎公舉,曰:朝廷頻年下詔,以類求人。但聞例得舉官,未見擇其舉主。欲望自今先責朝官有聲望者,各舉所知,其舉得官員則置籍,並舉主名姓籍之。所舉之官,實著廉能,則特旌舉主;若所舉貪冒敗事,連坐舉主。陛
 下自登寶位,十年於茲,七經選掄,得人多矣。然下僚遠官,不無沉滯。望令採訪司及州郡長吏,廉察以聞,籍以待用,則下無遺材矣。



 其三,懲貪吏,曰:貪吏之於民,其損甚大。屈法煩刑,徇私肆虐,使民之受害甚於木之受蠹。若乃用非其人而不繩以法,雖夷、齊、顏、閔不能自見。蓋中人之性,如水之在器,方員不常,顧用之者何如爾。望令諸路轉運使副兼採訪之名,責以覺察州、府、軍、監長吏得失,俟其澄清部內,則待以不次之擢,置於侍從之
 間。所貴周知物理,能備顧問,且足為外官之勸也。



 其四,省冗官,曰:古人建官,初不必備者,惟得其人也。國家封疆雖逾前世,而分設庶官實倍常數,意欲盡籠天下之利,而民物轉加凋弊。二十年前,江、淮諸郡,揚、楚最居要沖,務穰事眾,地廣民繁。然止設知州一人署領官事,其餘通判官、推官及州官等,悉皆分管榷務、倉庫。當時事無不集,兼少獄訟。其後十年,臣任揚州時,朝廷添置監臨、使臣等職,實逾本州官數。諸州冗員,似此非一。今以
 朝官、諸色使臣及縣令、簿、尉等高卑相折而計之,一人月費不啻十千,以千人約之,歲計用十餘萬千,更倍萬約之,萬又過倍。使皆廉吏,止糜公帑;設或貪夫參錯其間,則取於民者又加倍焉。望委各路轉運使副,與知州同議裁減。若縣令、簿、尉等官自前多不備置,可兼者兼之,如此則冗官汰矣。



 其五,擇遠官,曰:負罪之人,多非良善,貪殘兇暴,無所不至。若授以遠方牧民之官,其或怙惡不悛,恃遠肆毒。小民罹殃,卒莫上訴,甚非撫綏遠人
 之意也。若自今以往,西川、廣南長吏不任負罪之人,則遠人受賜矣。



 書奏,太宗嘉納之。



 初,柴禹錫任樞密,有奴受人金,而禹錫實不知也。參知政事陳恕欲因以中禹錫。太宗怒,引囚訊其事,化基為辨其誣。太宗感悟,以化基為長者。淳化中,拜中丞,俄知京朝官考課,遷工部侍郎。至道三年,超拜參知政事。咸平四年,以工部尚書罷知揚州。移知河南府,進禮部尚書。大中祥符三年,卒,年六十七。贈右僕射,謚惠獻。化基寬厚有容,喜慍不形,僚
 佐有相凌慢者,輒優容之。在中書,不以蔭補諸子官,然善教訓,故其子舉正、舉直、舉善、舉元皆有所立。



 舉正字伯仲,幼嗜學,厚重寡言。化基以為類己,器愛異諸子,以蔭補秘書省校書郎。進士及第,知伊闕、任丘縣,館閣校勘、集賢校理、《真宗實錄》院檢討、國史編修官。三遷尚書度支員外郎、直集賢院,修《三朝寶訓》,同修起居注,擢知制誥。其妻父陳堯佐為相,改龍圖閣待制、堯佐罷,以兵部郎中復知制誥,為翰林學士,拜右諫議大夫、
 參知政事。前一日,吏有馳報者,舉正方燕居齋舍,徐謂吏曰:「安得漏禁中語?」既入謝,仁宗曰:「卿恬於進取,未嘗干朝廷以私,故不次用卿。」



 時陜西用兵,呂夷簡以宰相判樞密院,舉正曰:「判名重,不可不避也。」乃改兼樞密使。遷給事中。御史臺舉李徽之為御史,舉正友婿也,格不行。徽之訟曰:「舉正妻悍不能制,如謀國何?」歐陽修等亦論舉正懦默不任事,舉正亦自求去,遂以資政殿學士、尚書禮部侍郎知許州。光化軍叛卒轉寇傍境,而州兵
 有謀起為應者,舉正潛捕首惡者斬之。徙知應天府,累遷左丞。



 皇祐初,拜御史中丞,乃奏:「張堯佐庸人,緣妃家,一日領四使,使賢士大夫無所勸。」不報,舉正因留班廷諍,乃奪宣徽、景靈二使。又曰:「先朝用人,雖守邊累年者,官止遙郡刺史。今所用未盡得人,而克期待遷,使後有功者何所勸耶?且轉運使察官吏能否,生民休戚賴焉。命甫下而數更,不終歲而再易,恩澤所以未宣,民疾所以未瘳者,職此故也。」御史唐介坐言事貶春州,舉正力
 言之,介得徙英州。居半歲,堯佐復為宣徽使。家居凡七上疏。及狄青為樞密使,又言青出兵伍不可為執政,力爭不能奪,因請解言職。帝稱其得風憲體,遣賜就第,賜白金三百兩,除觀文殿學士、禮部尚書、知河南府,入兼翰林侍讀學士。每進讀及前代治亂之際,必再三諷諭。



 以太子少傅致仕,卒,贈太子太保,謚安簡,賜黃金百兩。文章雅厚如其為人,有《平山集》、《中書制集》、《內制集》五十卷。



 舉元字懿臣,以上文章賜進士出身。知潮州,江水敗堤,盜乘間竊發,舉元夜召里豪計事,盜既獲,乃治堤。為河陰發運判官。或言大河決,將犯京師。舉元適入對,具論地形證其妄,已而果然。歷郡牧、戶部判官、京東轉運使。沙門島多流人,守吏顧貨橐,陰殺之。舉元請立監以較賞罰,自是全活者眾。徙淮南、河東。夏人來爭屈野地。舉元從數騎度河,設幕與之議,示以赤心,夏人感服。



 治平中,又徙成都。邙井鹽歲入二百五十萬,為丹棱卓個所
 侵,積不售,下令止之,鹽登於舊。召提舉在京修造,英宗勞之曰:「官廬舍害於水,僅有存者,卿究心公家,毋憚其勞。」俄進鹽鐵副使,拜天章閣待制,知滄州,改河北都轉運使,知永興軍。慶人、夏人屯境上,有窺我意。舉元使二裨將以千騎扼其要害。長安遣從事來會兵涇原,戒勿輕舉。大將竇舜卿銳意請行,不聽。舉元曰:「不過三日,虜去矣。」至期果去。神宗以細札諮攻守策,舉元請省官減戍,益備去兵,勿營亭障。輿論不合,遂引疾求解,徙陳州,
 未行而卒。官至給事中,年六十二。子詔。



 詔字景獻,用蔭補官,通判廣信軍事,知博州。魏俗尚椎剽,奸盜相囊橐,詔請開反告殺並贖罪法,以攜其黨。元祐初,朝廷起回河之議,未決,而開河之役遽興。詔言河朔秋潦,水淫為災,民人流徙,賴發廩振贍恩,稍蘇其生,謂宜安之,未可以力役傷也。從之。擢開封府推官。富民貸後絕僧牒為緡錢十三萬,逾期復責倍輸,身死貲籍,又錮其妻子,詔請免之。出為滑州。州屬縣有退灘百餘
 頃,歲調民刈草給河堤,民病其役,詔募人佃之,而收其餘。為度支郎中,使契丹。時方討西夏,迓者耶律誠欲嘗我,言曰:「河西無禮,大國能容之乎?」詔曰:「夏人侮邊,既正其罪矣,何預兩朝和好事?」入賀,故事,跪而飲,蓋有誤拜者,乃強詔。詔曰:「南北百年,所守者禮,其可紛更耶?」卒跪飲之。



 崇寧中,由大理少卿為卿,徙司農。御史論詔在滁日請蘇軾書《醉翁亭碑》,罷主崇福宮。旋知汝州,鑄錢卒罵大校,詔斬以徇,而上章待罪。除直秘閣,言者復抉滁
 州事,罷去。起知深、兗二州,徙同州,過闕,留為左司郎中,遷衛尉、太府卿、刑部侍郎,詳定敕令。舊借緋紫者不佩魚,詔言:「章服所以辨上下,今與胥吏不異。」遂皆佩魚。歷工、兵、戶三部侍郎,轉開封尹。時子□使京西,攝尹洛。父子兩京相望,人以為榮。



 進刑部尚書,拜延康殿學士,提舉上清寶菉宮,復為工部尚書。徽宗閔其老,命毋拜,詔皇恐,於是但朝朔望。俄以銀青光祿大夫致仕,卒,年七十九。



 論曰:自昔參大政、贊機務,非明敏特達之士,不能勝其任。若又飭以文雅,濟以治具,則盡善矣。若水機鑒明敏,儒而知兵;李至剛嚴簡重,好古博雅,其於柄用宜矣。王沔臨事精密,能遠私謁,而考課之議,頗傷苛刻;仲甫以吏事為時用,未免茍容之誚,瑕瑜固不相掩也。仲舒見舉於蒙正,而反攻其短;易簡不能周恤光逢,而置之死地,其不可與郭贄辨曹彬之誣、化基伸禹錫之枉同日而語也明矣。此純厚長者之稱,所以獨歸於二子歟!舉
 正繼踐臺佐,得風憲體;舉元任職邊郡,有持重稱。矧詔之父子又並尹兩京,克濟其美,何王氏子孫之多賢也!



\end{pinyinscope}