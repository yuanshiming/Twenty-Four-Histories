\article{列傳第二百二十一方技下}

\begin{pinyinscope}

 ○賀蘭棲真柴通玄甄棲真楚衍僧志言僧懷丙許希龐安時錢乙僧智緣郭天信魏漢津
 王老志王仔昔林靈素皇甫坦王克明莎衣道人孫守榮



 賀蘭棲真,不知何許人。為道士,自言百歲。善服氣,不憚寒暑,往往不食。或時縱酒,游市廛間,能啖肉至數斤。始居嵩山紫虛觀,後徙濟源奉仙觀,張齊賢與之善。景德二年,詔曰:「師棲身巖壑,抗志煙霞,觀心眾妙之門,脫屣浮雲之外。朕奉希夷而為教,法清靜以臨民,思得有道之人,訪以無為之理。久懷上士,欲覿真風,爰命使車,往
 申禮聘。師其暫別林谷,來儀闕庭,必副招延,無憚登涉。今遣入內內品李懷贇召師赴闕。」既至,真宗作二韻詩賜之,號宗玄大師,賚以紫服、白金、茶、帛、香、藥,特蠲觀之田租,度其侍者。未幾,求還舊居。大中祥符三年卒,時大雪,經三日,頂猶熱,人多異之。



 紫通玄,字又玄,陜州閿鄉人。為道士於承天觀。年百餘歲,善闢谷長嘯,唯飲酒。言唐末事,歷歷可聽。太宗召至闕下,懇求歸本觀。真宗即位,屢來京師。召對,語無文飾,
 多以修身慎行為說。祀汾陰,召至行在,命坐,問以無為之要。所居觀即唐軒游宮,有明皇詩石及所書《道德經》二碑。上作二韻詩賜之,並賚以茶、藥、束帛。詔為修道院,蠲其田租,度弟子二人。明年春,通玄作遺表,自稱羅山太一洞主,遣弟子張守元、李守一詣闕,以龜鶴為獻;又召官僚士庶言生死之要。夜分,盥濯,然香庭中,望闕而坐,遲明卒。



 時又召河中草澤劉巽、華山隱士鄭隱、敷水隱士李寧。巽年七十餘,以經傳講授,躬耕自給。授大理
 評事致仕,賜綠袍、笏、銀帶。隱以經術為業,遇道士傳闢穀煉氣之法,修習頗驗,居華山王刁巖逾二十年,冬夏裳衣皮裘。寧精於藥術,老而不衰,常以藥施人,人以金帛為報,輒拒之。景德中,萬安太后不豫,驛召寧赴闕,未至而後崩。大中祥符四年,賜號正晦先生。上並作詩為賜,加以茶、藥、繒帛。獨隱辭賜物不受。



 甄棲真,字道淵,單州單父人。博涉經傳,長於詩賦。一應進士舉,不中第,嘆曰:「勞神敝精,以追虛名,無益也。」遂棄
 其業,讀道家書以自樂。初訪道於牢山華蓋先生,久之出游京師,因入建隆觀為道士。周歷四方。以藥術濟人,不取其報。祥符中,寓居晉州,性和靜無所好惡,晉人愛之。以為紫極宮主。



 年七十有五,遇人,或以為許元陽,語之曰:「汝風神秀異,有如李筌。雖老矣,尚可仙也。」因授煉形養元之訣,且曰:「得道如反掌,第行之惟艱,汝勉之。」棲真行之二三年,漸反童顏,攀高攝危,輕若飛舉。乾興元年秋,謂其徒曰:「此歲之暮,吾當逝矣。」即宮西北隅自甃
 殯室。室成,不食一月,與平居所知敘別,以十二月二日衣紙衣臥磚塌卒。人未之奇也。及歲久,形如生,眾始驚,傳以為尸解。



 棲真自號神光子,與隱人海蟾子者以詩往還。論養生秘術,目曰《還金篇》,凡兩卷。



 楚衍,開封阼城人。少通四聲字母,里人柳曜師事衍,里中以先生目之。衍於《九章》、《緝古》、《綴術》、《海島》諸算經尤得其妙。明相法及《聿斯經》,善推步、陰陽、星歷之數,間語休咎無不中。自陳試《宣明歷》,補司天監學生,遷保章正。天
 聖初,造新歷,眾推衍明歷數,授靈臺郎,與掌歷官宋行古等九人制崇天歷。進司天監丞,入隸翰林天文。皇祐中,同造《司辰星漏歷》十二卷。久之,與周琮同管勾司天監。卒,無子,有女亦善算術。



 僧志言,自言姓許,壽春人。落發東京景德寺七俱胝院,事清璲。初,璲誦經勤苦,志言忽造璲,跪前願為弟子。璲見其相貌奇古,直視不瞬,心異之,為授具戒。然動止軒昂,語笑無度,多行市里,褰裳疾趨,舉指書空,佇立良久;
 時從屠酤游,飲啖無所擇。眾以為狂,璲獨曰:「此異人也。」



 人有欲為齋施,輒先知其至,不俟款門,指名取供。溫州人林仲方自其家以摩衲來獻,舟始及岸,遽來取去。仁宗每延入禁中,徑登坐結趺,飯畢遽出,未嘗揖也。王公士庶召即赴,然莫與交一言者。或陰卜休咎,書紙揮翰甚疾,字體遒壯,初不可曉,其後多驗。仁宗春秋漸高,嗣未立,默遣內侍至言所。言所書有「十三郎」字,人莫測何謂。後英宗以濮王第十三子入繼,眾始悟。大宗正守節
 請書,言不顧,迫之,得「潤州」字。未幾,守節薨,贈丹陽郡王。見寺童義懷,撫其背曰:「德山、臨濟。」懷既落發,住天衣,說法,大為學者所宗,其前知多類此。



 普凈院施浴,夜漏初盡,門扉未啟,方迎佛而浴室有人聲,往視,則言在焉。有具齋薦鱠者,並食之,臨流而吐,化為小鮮,群泳而去。海客遇風且沒,見僧操絙引舶而濟。客至都下遇言,忽謂之曰:「非我,汝奈何?」客記其貌,真引舟者也。與曹州士趙棠善,後棠棄官隱居番禺。人傳棠與言數以偈頌相寄,
 萬里間輒數日而達。棠死,亦盛夏身不壞。



 言將死,作頌,不可曉。已而曰:「我從古始成就,逃多國土,今南國矣。」仁宗遣內侍以真身塑像置寺中,榜曰顯化禪師。其後善厚者禮之,見額上熒然有光,就視之,得舍利。



 僧懷丙,真定人。巧思出天性,非學所能至也。真定構木為浮圖十三級,勢尤孤絕。既久而中級大柱壞,欲西北傾,他匠莫能為。懷丙度短長,別作柱,命眾工維而上。已而卻眾工,以一介自從,閉戶良久,易柱下,不聞斧鑿聲。



 趙州洨河鑿石為橋,熔鐵貫其中。自唐以來相傳數百年,大水不能壞。歲久,鄉民多盜鑿鐵,橋遂欹倒,計千夫不能正。懷丙不役眾工,以術正之,使復故。河中府浮梁用鐵牛八維之,一牛且數萬斤。後水暴漲絕梁,牽牛沒於河,募能出之者。懷丙以二大舟實土,夾牛維之,用大木為權衡狀鉤牛,徐去其土,舟浮牛出。轉運使張燾以聞,賜紫衣。尋卒。



 許希,開封人。以醫為業,補翰林醫學。景祐元年,仁宗
 不豫,侍醫數進藥,不效,人心憂恐。冀國大長公主薦希,希診曰:「針心下包絡之間,可亟愈。」左右爭以為不可,諸黃門祈以身試,試之,無所害。遂以針進,而帝疾愈。命為翰林醫官,賜緋衣、銀魚及器幣。希拜謝已,又西向拜,帝問其故,對曰:「扁鵲,臣師也。今者非臣之功,殆臣師之賜,安敢忘師乎?」乃請以所得金興扁鵲廟。帝為築廟於城西隅,封靈應侯。其後廟益完,學醫者歸趨之,因立太醫局於其旁。



 希至殿中省尚藥奉御,卒。著《神應針經要
 訣》行於世。錄其子宗道至內殿崇班。



 龐安時字安常,蘄州蘄水人。兒時能讀書,過目輒記。父,世醫也,授以脈訣。安時曰:「是不足為也。」獨取黃帝、扁鵲之脈書治之,未久,已能通其說,時出新意,辨詰不可屈,父大驚,時年猶未冠。已而病聵,乃益讀《靈樞》、《太素》、《甲乙》諸秘書,凡經傳百家之涉其道者,靡不通貫。嘗曰:「世所謂醫書,予皆見之,惟扁鵲之言深矣。蓋所謂《難經》者,扁鵲寓術於其書,而言之不祥,意者使後人自求之歟!予
 之術蓋出於此。以之視淺深,決死生,若合符節。且察脈之要,莫急於人迎、寸口。是二脈陰陽相應,如兩引繩,陰陽均,則繩之大小等,故定陰陽於喉、手,配覆溢於尺、寸,寓九候於浮沉,分四溫於傷寒。此皆扁鵲略開其端,而予參以《內經》諸書,考究而得其說。審而用之,順而治之,病不得逃矣。」又欲以術告後世,故著《難經辨》數萬言。觀草木之性與五藏之宜,秩其職任,官其寒熱,班其奇偶,以療百疾,著《主對集》一卷。古今異宜,方術脫遺,備陰陽
 之變,補仲景《論》。藥有後出,古所未知,今不能辨,嘗試有功,不可遺也。作《本草補遺》。



 為人治病,率十愈八九。踵門求診者,為闢邸舍居之,親視饘粥、藥物,必愈而後遣;其不可為者,必實告之,不復為治。活人無數。病家持金帛來謝,不盡取也。



 嘗詣舒之桐城,有民家婦孕將產,七日而子不下,百術無所效。安時之弟子李百全適在傍舍,邀安時往視之。才見,即連呼不死,令其家人以湯溫其腰腹,自為上下拊摩。孕者覺腸胃微痛,呻吟間生一男
 子。其家驚喜,而不知所以然。安時曰:「兒已出胞,而一手誤執母腸不復能脫,故非符藥所能為。吾隔腹捫兒手所在,針其虎口,既痛即縮手,所以遽生,無他術也。」取兒視之,右手虎口針痕存焉。其妙如此。



 有問以華佗之事者,曰:「術若是,非人所能為也。其史之妄乎!」年五十八而疾作,門人請自視脈,笑曰:「吾察之審矣。且出入息亦脈也,今胃氣已絕。死矣。」遂屏卻藥餌。後數日,與客坐語而卒。



 錢乙字仲陽,本吳越王俶支屬,祖從北遷,遂為鄆州人。父穎善醫,然嗜酒喜游,一旦,東之海上不反。乙方三歲,母前死,姑嫁呂氏,哀而收養之,長誨之醫,乃告以家世。即泣,請往跡尋,凡八九反。積數歲,遂迎父以歸,時已三十年矣。鄉人感慨,賦詩詠之。其事呂如事父,呂沒無嗣,為收葬行服。



 乙始以《顱
 
  
   
  
 
 方》著名,至京師視長公主女疾,授翰林醫學。皇子病瘈瘲,乙進黃土湯而愈。神宗召問黃土所以愈疾狀,對曰:「以土勝水,水得其平,則風自
 止。」帝悅,擢太醫丞,賜金紫。由是公卿宗戚家延致無虛日。



 廣親宗子病,診之曰:「此可毋藥而愈。」其幼在傍,指之曰:「是且暴疾驚人,後三日過午,可無恙。」其家恚,不答。明日,幼果發UH甚急,召乙治之,三日愈。問其故,曰:「火色直視,心與肝俱受邪。過午者,所用時當更也。」王子病嘔洩,他醫與剛劑,加喘焉。乙曰:「是本中熱,脾且傷,奈何復燥之?將不得前後溲。」與之石膏湯,王不信,謝去。信宿浸劇,竟如言而效。



 士病欬,面青而光,氣哽哽。乙曰:「肝乘肺,此
 逆候也。若秋得之,可治;今春,不可治。」其人祈哀,強予藥。明日,曰:「吾藥再瀉肝,而不少卻;三補肺,而益虛;又加唇白,法當三日死。今尚能粥,當過期。」居五日而絕。



 孕婦病,醫言胎且墜。乙曰:「娠者五藏傳養,率六旬乃更。誠能候其月,偏補之,何必墜?」已而母子皆得全。又乳婦因悸而病,既愈,目張不得瞑。乙曰:「煮鬱李酒飲之使醉,即愈。所以然者,目系內連肝膽,恐則氣結,膽衡不下。鬱李能去結,隨酒入膽,結去膽下,則目能瞑矣。」飲之,果驗。



 乙本有
 羸疾,每自以意治之,而後甚,嘆曰:「此所謂周痺也。入藏者死,吾其已夫。」既而曰:「吾能移之使在末。」因自制藥,日夜飲之。左手足忽攣不能用,喜曰:「可矣!」所親登東山,得茯苓大逾斗。以法啖之盡,由是雖偏廢,而風骨悍堅如全人。以病免歸,不復出。



 乙為方不名一師,於書無不窺,不靳靳守古法。時度越縱舍,卒與法會。尤遽《本草》諸書,辨正闕誤。或得異藥,問之,必為言生出本末、物色、名貌差別之詳,退而考之皆合。末年攣痺浸劇,知不可為,召
 親戚訣別,易衣待盡,遂卒,年八十二。



 僧智緣,隨州人,善醫。嘉祐末,召至京師,舍於相國寺。每察脈,知人貴賤、禍福、休咎,診父之脈而能道其子吉兇,所言若神,士大夫爭造之。王珪與王安石在翰林,珪疑古無此,安石曰:「昔醫和診晉侯,而知其良臣將死。夫良臣之命乃見於其君之脈,則視父知子,亦何足怪哉!」



 熙寧中,王韶謀取青唐,上言蕃族重僧,而僧結吳叱臘主部帳甚眾,請智緣與俱至邊。神宗召見,賜白金,遣乘傳
 而西,遂稱「經略大師」。智緣有辯口,徑入蕃中,說結吳叱臘歸化,而他族俞龍珂、禹藏訥令支等皆因以書款。韶頗忌惡之,言其撓邊事,召還,以為右街首坐,卒。



 郭天信字祐之,開封人。以技隸太史局。徽宗為端王,嘗退朝,天信密遮白曰:「王當有天下。」既而即帝位,因得親暱。不數年,至樞密都承旨、節度觀察留後。其子中復為閣門通事舍人,許陪進士徑試大廷,擢秘書省校書郎。未幾,天信覺已甚,乞還武爵,又從之。



 政和初,拜定武軍
 節度使、祐神觀使,頗與聞外朝政事。見蔡京亂國,每托天文以撼之,且云:「日中有黑子。」帝甚懼,言之不已,京由是黜。張商英方有時望,天信往往稱於內朝。商英亦欲借左右游談之助,陰與相結,使僧德洪輩道達語言。商英勸帝節儉,稍裁抑僧寺,帝始敬畏之,而近侍積不樂,間言浸潤,眷日衰。京黨因是告商英與天信漏洩禁中語言,天信先發端,窺伺上旨,動息必報,乃從外庭決之,無不如志。商英遂罷。御史中丞張克公復論之,詔貶天
 信昭化軍節度副使、單州安置,命宋康年守單,幾其起居。再貶行軍司馬,竄新州,又徒康年使廣東,天信至數月,死。京已再相,猶疑天信挾術多能,死未必實,令康年選吏發棺驗視焉。



 魏漢津,本蜀黥卒也。自言師事唐仙人李良號「李八百」者,授以鼎樂之法。嘗過三山龍門,聞水聲,謂人曰:「其下必有玉。」即脫衣沒水,抱石而出,果玉也。皇祐中,與房庶俱以善樂薦,時阮逸方定黍律,不獲用。崇寧初猶在,朝
 廷方協考鐘律,得召見,獻樂議,言得黃帝,夏禹聲為律、身為度之說。謂人主稟賦與眾異,請以帝指三節三寸為度,定黃鐘之律;而中指之徑圍,則度量權衡所自出也。又云:「聲有太有少。太者,清聲,陽也。天道也。少者,濁聲,陰也,地道也。中聲在其間,人道也。合三才之道,備陰陽奇偶,然後四序可得而調,萬物可得而理。」當時以為迂怪,蔡京獨神之。或言漢津本範鎮之役,稍窺見其制作,而京托之於李良云。



 於是請先鑄九鼎,次鑄帝坐大鐘
 及二十四氣鐘。四年三月鼎成,賜號沖顯處士。八月,《大晟樂》成。徽宗御大慶殿受群臣朝賀,加漢津虛和沖顯寶應先生,頒其樂書天下。而京之客劉昺主樂事,論太少之說為非,將議改作。既而以樂成久,易之恐動觀聽,遂止。漢津密為京言:「《大晟》獨得古意什三四爾,他多非古說,異日當以訪任宗堯。」宗堯學於漢津者也。



 漢津曉陰陽數術,多奇中,嘗語所知曰:「不三十年,天下亂矣。」未幾死。京遂召宗堯為典樂,復欲有所建,而為田為所奪,
 語在《樂志》。後即鑄鼎之所建寶成殿,祀黃帝、夏禹、成王、周、召而良、漢津俱配食。謚漢津為嘉晟侯。



 有馬賁者,出京之門,在大晟府十三年,方魏、劉、任、田異論時,依違其間,無所質正,擢至通議大夫、徽猷閣待制。議者咎當時名器之濫如此。



 王老志,濮州臨泉人。事親以孝聞。為轉運小吏,不受賂謝。遇異人於丐中,自言:「吾所謂鐘離先生也。」予之丹,服之而狂。遂棄妻子,結草廬田間,時為人言休咎。



 政和三
 年,太僕卿王亶以其名聞。召至京師,館於蔡京第。嘗緘書一封至帝所,徽宗啟讀,乃昔歲秋中與喬、劉二妃燕好之語也。帝由是稍信之,封為洞微先生。朝士多從求書,初若不可解,後卒應者十八九,故其門如市。京慮太甚,頗以為戒;老志亦謹畏,乃奏禁絕之。嘗獻乾坤鑒法,命鑄之。既成,謂帝與皇後他日皆有難,請時坐鑒下,思所以儆懼消變者。



 明年,見其師,責以擅處富貴,乃丐歸,未得請,病甚,始許其去。步行出,就居,病已失矣。歸濮而
 死。詔賜金以葬,贈正議大夫。



 初,王黼未達時,父為臨泉令,問黼名位所至,即書「太平宰相」四字。旋以墨塗去之,曰:「恐洩機也。」黼敗,人乃悟。



 王仔昔,洪州人。始學儒,自言遇許遜,得《大洞》、《隱書》豁落七元之法,出游嵩山,能道人未來事。政和中,徽宗召見,賜號沖隱處士。帝以旱禱雨,每遣小黃門持紙求仔昔畫,日又至,忽篆符其上,仍細書「焚符湯沃而洗之」。黃門懼不肯受,強之,乃持去。蓋帝默祝為宮妃療赤目者,用
 其說一沃,立愈。進封通妙先生,居上清寶籙宮。獻議九鼎神器不可藏於外。乃於禁中建圓象徽調閣以貯之。



 仔昔資倨傲,又少戇,帝常待以客禮,故其遇巨閹殆若童奴,又欲群道士皆宗己。及林靈素有寵,忌之,陷以事,囚之東太一宮。旋坐言語不遜,下獄死。仔昔之得罪,宦者馮浩力最多。未死時,書示其徒曰:「上蔡遇冤人。」其後浩南竄,至上蔡被誅。



 林靈素,溫州人。少從浮屠學,苦其師笞罵,去為道士。善
 妖幻,往來淮、泗間,丐食僧寺,僧寺苦之。



 政和末,王老志、王仔昔既衰,徽宗訪方士於左道錄徐知常,以靈素對。既見,大言曰:「天有九霄,而神霄為最高,其治曰府。神霄玉清王者,上帝之長子,主南方,號長生大帝君,陛下是也,既下降於世,其弟號青華帝君者,主東方,攝領之。己乃府仙卿曰褚慧,亦下降佐帝君之治。」又謂蔡京為左元仙伯,王黼為文華吏,盛章、王革為園苑寶華吏,鄭居中、童貫及諸巨閹皆為之名。貴妃劉氏方有寵,曰九華
 玉真安妃。帝心獨喜其事,賜號通真達靈先生,賞賚無算。



 建上清寶籙宮,密連禁省。天下皆建神霄萬壽宮。浸浸造為青華正晝臨壇,及火龍神劍夜降內宮之事,假帝誥、天書、雲篆,務以欺世惑眾。其說妄誕,不可究質,實無所能解。惟稍識五雷法,召呼風霆,間禱雨有小驗而已。令吏民詣宮受神霄秘錄,朝士之嗜進者,亦靡然趨之。每設大齋,輒費緡錢數萬,謂之千道會。帝設幄其側,而靈素升高正坐,問者皆再拜以請。所言無殊異,時時
 雜捷給嘲詼以資媟笑。其徒美衣玉食,幾二萬人。遂立道學,置郎、大夫十等,有諸殿侍晨、校籍、授經,以擬待制、修撰、直閣。始欲盡廢釋氏以逞前憾,既而改其名稱冠服。



 靈素益尊重,升溫州為應道軍節度,加號元妙先生、金門羽客、沖和殿侍晨,出入呵引,至與諸王爭道。都人稱曰:「道家兩府。」本與道士王允誠共為怪神,後忌其相軋,毒之死。宣和初,都城暴水,遣靈素厭勝。方率其徒步虛城上,役夫爭舉梃將擊之,走而免。帝知眾所怨,始不
 樂。



 靈素在京師四年,恣橫愈不悛,道遇皇太子弗斂避。太子入訴,帝怒,以為太虛大夫,斥還故里,命江端本通判溫州,幾察之。端本廉得其居處過制罪,詔徙置楚州而已死。遺奏至,猶以侍從禮葬焉。



 皇甫坦,蜀之夾江人。善醫術。顯仁太后苦目疾,國醫不能療,詔募他醫,臨安守臣張偁以坦聞。高宗召見,問何以治身,坦曰:「心無為則身安,人主無為則天下治。」引至慈寧殿治太后目疾,立愈。帝喜,厚賜之,一無所受。令持
 香禱青城山,還,復召問以長生久視之術,坦曰:「先禁諸欲,勿令放逸。丹經萬卷,不如守一。」帝嘆服,書「清靜」二字以名其庵,且繪其像禁中。



 荊南帥李道雅敬坦,坦歲謁道。隆興初,道入朝,高宗、孝宗問之,皆稱皇甫先生而不名。坦又善相人,嘗相道中女必為天下母,後果為光宗後。



 王克明字彥昭,其始饒州樂平人,後徙湖州烏程縣。紹興、乾道間名醫也。初生時,母乏乳,餌以粥,遂得脾胃疾,
 長益甚,醫以為不可治。克明自讀《難經》、《素問》以求其法,刻意處藥,其病乃愈。始以術行江、淮,入蘇、湖,針灸尤精。診脈有難療者,必沉思得其要,然後予之藥。病雖數證,或用一藥以除其本,本除而餘病自去。亦有不予藥者,期以某日自安。有以為非藥之過,過在某事,當隨其事治之。言無不驗。士大夫皆自屈與游。



 魏安行妻風痿十年不起,克明施針,而步履如初。胡秉妻病氣秘腹脹,號呼逾旬,克明視之。時秉家方會食,克明謂秉曰:「吾愈恭
 人病,使預會可乎?」以半硫圓碾生姜調乳香下之,俄起對食如平常。廬州守王安道風禁不語旬日,他醫莫知所為。克明令熾炭燒地,灑藥,置安道於上,須臾而蘇。金使黑鹿谷過姑蘇,病傷寒垂死,克明治之,明日愈。及從徐度聘金,黑鹿谷適為先排使,待克明厚甚。克明訝之,穀乃道其故,由是名聞北方。後再從呂正己使金,金接伴使忽被危疾,克明立起之,卻其謝。張子蓋救海州,戰士大疫,克明時在軍中,全活者幾萬人。子蓋上其功,克
 明力辭之。



 克明頗知書,好俠尚義,常數千里赴人之急。初試禮部中選,累任醫官。王炎宣撫四川,闢克明,不就。炎怒,劾克明避事,坐貶秩。後遷至額內翰林醫痊局,賜金紫。紹興五年卒,年六十七。



 莎衣道人,姓何氏,淮陽軍朐山人。祖執禮,仕至朝議大夫。道人避亂渡江,嘗舉進士不中。紹興末,來平江。一日,自外歸,倏若狂者,身衣白示間,晝丐食於市,夜止天慶觀。久之,衣益敝,以莎緝之。嘗游妙嚴寺,臨池見影,豁然大
 悟。人無貴賤,問休咎,罔不奇中。會有瘵者乞醫,命持一草去,旬日而愈。眾翕然傳莎草可以愈疾,求而不得者,或遂不起,由是遠近異之。



 孝宗一夕夢莎衣人跣哭來吊者,訊之曰:「蘇人也。」詰其故,不肯言。帝寤,以語內侍。會後及太子薨,帝哀泣,內侍進前勉釋,並道前夢。帝乃矍然,因遣使召之,不至。帝念恢復大計,累歲未有所屬,後位虛且久,乃焚香默言:「何誠能仙顧,必知朕意。」遂遣中官致贄,不言所以。道人見之掉首,吳音曰:「有中國即有
 外夷;有日即有月,不須問。」趣之去。使者歸奏,帝甚異之,遂賜號通神先生,為築庵觀中,賜衣數襲,皆不受。好事者強邀入庵,大笑而出,復於故處。眾日以珍饌餉之,每食於通衢,逮飽即去。



 帝歲命內侍即其居設十道齋,合雲水之士,施予優普。一歲,偶逾期,眾咸訝而請,道人亟起於臥,搖手瞬目而招之曰:「亟來,亟來!」是日內侍至平望,眾益服其神。光宗即位,召之,又不至。慶元六年卒。



 孫守榮,臨安富陽人。生七歲,病瞽。遇異人教以風角、鳥
 占之術,其法以音律推五數,播五行,測度萬物始終盛衰之理。凡問者,一語頃,輒知休咎。守榮既悟,異人授以鐵笛,遂去不復見。守榮因號富春子,吹笛市中,人初不異也。然其術率驗。



 寶慶間,游吳興,聞譙樓鼓角聲,驚曰:「旦夕且有變,土人當有典郡者。」見王元春,即賀之曰:「作鄉郡者,必君也。」元春初不之信。越兩月,潘丙作亂,元春以告變功,果典郡。自是富春子之名大顯,貴人爭延致之。



 淮南帥李曾伯薦諸朝。既至,謁丞相史嵩之,閽者以
 晝寢辭。守榮曰:「丞相方釣魚園池,何得云爾。」閽者驚異,入白丞相,丞相一見,頗喜之。自是數出入相府。一日,庭鵲噪,令占之,曰:「來日晡時,當有寶物至。」明日,李全果以玉柱斧為貢。嵩之又嘗得李全檄藏袖中,詢其事,守榮曰:「此李全詐假布囊二十萬爾。」剝封,果如其說。



 士大夫咸詢履歷,守榮不盡答。私謂所知曰:「吾以音推諸朝紳,互有贏縮,宋祿其殆終乎!」後為嵩之所忌,誣以他罪,貶死遠郡。



\end{pinyinscope}