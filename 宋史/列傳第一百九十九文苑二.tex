\article{列傳第一百九十九文苑二}

\begin{pinyinscope}

 ○高頔李度韓溥鞠常宋準柳開夏侯嘉正羅處約安德裕錢熙



 高頔字子奇,開封雍丘人。後唐清泰中舉進士,同輩紿之曰:「何不從裴僕射求知乎?」時裴皞以左僕射致仕,後進無至其門者。頔性純樸,信其言,以文贄於皞。明年,禮部侍郎馬裔孫知貢舉,乃皞門下生也。皞以頔語之,遂擢乙科,四遷魏博觀察支使。



 周顯德中,符彥卿奏署掌書記。時太宗親迎懿德皇后於大名,彥卿遣頔迎候,日夕陪接,尤伸款好。後隨彥卿鎮鳳翔,會詔留彥卿洛陽,頔復為天雄軍掌書記。後以病免,居於魏。



 雍熙二年,太
 宗親試貢士,頔子南金舉學究,自陳曰:「臣父年八十四,嘗佐使幕,久已罷職,家貧無以存養。願賜一第,庶獲寸祿,以及老父。」上問左右其父何人,宰相宋琪以頔對,且言其素行廉介,老而彌厲,甚為搢紳推重。上曰:「此高頔子耶!頔在大名幕中,嘗與朕游處,迨逾旬月。晨暮對案飲食,常拱手危坐,未曾少懈,其恭謹蓋天性也。惜其老矣,不欲煩以官政。」即擢南金第,拜頔左補闕致仕,賜錢十萬。後卒於家。



 頔有清節,力學強記,手寫書千餘卷。彥
 卿待之甚厚,或過致優給,頔計口受費,餘皆不納。彥卿左右多肆貪虐,民不能堪,及彥卿罷鎮,其故時將吏、賓客皆心愧,無敢復游魏者。惟頔清苦守法,魏人愛之。在魏三十年,無一人言其非者。所乘馬老,以糜飼之。僕夫年七十,待之如初,時稱其長者。



 次子鼎,舉進士,至殿中丞。



 李度,河南洛陽人。周顯德中舉進士。度工於詩,有「醉輕浮世事,老重故鄉人」之句。時翰林學士申文炳知貢舉,
 樞密使王樸移書錄其句以薦之,文炳即擢度為第三人。釋褐永寧縣主簿。



 累遷殿中丞、知歙州。坐事左遷絳州團練使,十年不調。度在歙州,嘗以所著詩刻於石,有中黃門得其石本,傳入禁中,太宗見之,謂宰相曰:「度今安在?」即令召至,對於便殿,與語甚悅,擢為虞部員外郎、直史館,賜緋。端拱初,籍田畢,交州黎桓加恩,命度借太常少卿充官告國信副使,上賜詩以寵行。未至交州,卒於太平軍傳舍,年五十七。



 度之南使,每至州府,即借圖
 經觀其勝跡,皆形篇詩,以上所賜詩有「奉使南游多好景」之句,遂題為《奉使南游集》,未成編而亡。



 弟康亦善詩,太平興國二年,登進士第,官至太子右贊善大夫。



 韓溥,京兆長安人,唐相休之裔孫。少俊敏,善屬文。周顯德初舉進士,累遷歷使府。開寶三年,自靜難軍掌書記召為監察御史,三遷至庫部員外郎、知華州,同判靈州,再轉司門郎中。淳化二年被病,表請辭職尋醫,許之。溥博學善持論,詳練臺閣故事,多知唐朝氏族,與人談
 亹癖然可聽,號為「近世肉譜」,搢紳頗推重之。尤善筆札,人多藏尺牘。



 弟洎,亦進士及第。



 鞠常,字可久,密州高密人。祖真,黃縣令。父慶孫,申州團練判官,有詩名。常少好學,善屬文。漢乾祐二年擢進士第,裁二十一,釋褐秘書省校書郎。周廣順中,宰相範質奏充集賢校理,出為鄆州觀察支使,歷永興軍節度掌書記、伊陽令。顯德四年,詣闕進策,召試,復授猗氏令,遷蔡州防禦判官,復宰介休、魏縣。開寶中,趙普為相,擢為
 著作佐郎。時任此官,惟常與楊徽之、李若拙、趙鄰幾四人,皆有名於時。常應舉時,著《四時成歲賦》萬餘言,又為《春蘭賦》,頗存興托。後為清河令。七年,卒,年四十七。



 子仲謀,字有開,雍熙中進士,有材幹,歷御史、東京留守推官、陜西轉運,至兵部員外郎。仲謀集其父所為文成二十卷。



 弟愉,周廣順中進士,與常齊名。



 宋準,字子平,開封雍丘人。祖彥升,庫部員外郎。父鵬,秘書郎。準開寶中舉進士,翰林學士李皞知貢舉,擢準甲
 科。會貢士徐士廉擊登聞鼓,訴皞用情取舍非當。太祖怒,召準覆試於便殿,見準形神偉茂,程試敏速,甚嘉之,以為宜首冠俊造,由是復擢準甲科,即授秘書省校書郎、直史館。



 八年,受詔修定諸道圖經。俄奉使契丹,復命稱旨。明年,出知南平軍,會改軍為太平州,依前知州事,就加著作佐郎。太平興國四年,遷著作郎、通判梓州,轉左拾遺。歸朝,預修諸書。八年,同知貢舉,出為河北轉運使,歲餘,以本官知制誥。雍熙中,加主客員外郎,復預知
 貢舉,俄判大理寺。四年,被病,遷金部郎中,罷知制誥。端拱二年卒,年五十二,賜錢百萬。



 準美風儀,善談論,辭採清麗,蒞官所至,皆有治聲。盧多遜之南流也,李穆坐同門生黜免,左右無敢言者。準因奏事,盛言穆長者,有檢操,常惡多遜專恣,固非其黨也。上寤,未幾,盡復穆舊官。時論以此稱之。天禧三年,錄其子大年試秘書省校書郎。



 準從弟可觀,金部郎中。族子郊、祁,並天聖二年進士甲科,別有傳。



 柳開,字仲塗,大名人。父承翰,乾德初監察御史。開幼穎異,有膽勇。周顯德末,侍父任南樂,夜與家人立庭中,有盜入室,眾恐不敢動,開裁十三,亟取劍逐之,盜逾垣出,開揮刃斷二足指。



 既就學,喜討論經義。五代文格淺弱,慕韓愈、柳宗元為文,因名肩愈,字紹先。既而改名字,以為能開聖道之塗也。著書自號東郊野夫,又號補亡先生,作二傳以見意。尚氣自任,不顧小節,所交皆一時豪雋。範杲好古學,尤重開文,世稱為「柳、範」。王祐知大名,開
 以文贄大蒙賞激。楊昭儉、盧多遜並加延獎。開寶六年舉進士,補宋州司寇參軍,以治獄稱職,遷本州錄事參軍。太平興國中,擢右贊善大夫。會征太原,督楚、泗八州運糧。選知常州,遷殿中丞,徙潤州,拜監察御史。召還,知貝州,轉殿中侍御史。雍熙二年,坐與監軍忿爭,貶上蔡令。



 會大舉北征,開部送軍糧,將至涿州,有契丹酋長領萬騎與米信戰,相持不解,俄遣使紿言求降,開謂信曰:「兵法云:『無約而請和,謀也。』彼將有謀,急攻之必勝。」信遲
 疑不決。逾二日,賊復引兵挑戰,後偵知果以矢盡,俟取於幽州也。師還,詣闕上書,願從邊軍效死,太宗憐之,復授殿中侍御史。



 雍熙中,使河北,因抗疏曰:「臣受非常恩,未有以報,年裁四十,膽力方壯。今契丹未滅,願陛下賜臣步騎數千,任以河北用兵之地,必能出生入死,為陛下復幽、薊,雖身沒戰場,臣之願也。」上以五代戰爭以來,自節鎮至刺史皆用武臣,多不曉政事,人受其弊。欲兼用文士,乃以侍御史鄭宣、戶部員外郎趙載、司門員外
 郎劉墀並為如京使,左拾遺劉慶為西京作坊使,開為崇儀使、知寧邊軍。



 徙全州。全西溪洞有粟氏,聚族五百餘人,常鈔劫民口糧畜,開為作衣帶巾帽,選牙吏勇辯者得三輩,使入,諭之曰:「爾能歸我,即有厚賞,給田為屋處之;不然,發兵深入,滅爾類矣。」粟氏懼,留二吏為質,率其酋四人與一吏偕來。開厚其犒賜,吏民爭以鼓吹飲之。居數日遣還,如期攜老幼悉至。開即賦其居業,作《時鑒》一篇,刻石戒之。遣其酋入朝,授本州上佐。賜開錢三
 十萬。



 淳化初,移知桂州。初,開在全州,有卒訟開,開即杖背黥面送闕下。有司言卒罪不及徒,召開下御史獄劾系,削二官,黜為復州團練副使,移滁州。復舊官,知環州。三年,移邠州。時調民輦送趨環、慶,己再運,民皆蕩析產業,轉運使復督後運,民數千人入州署號訴。開貽書轉運使曰:「開近離環州,知芻糧之數不增,大兵可支四年,今蠶農方作,再運半發,老幼疲弊,畜乘困竭,奈何又苦之?不罷,開即馳詣闕下,白於上前矣。」卒罷之。又知曹、邢
 二州。



 真宗即位,加如京使,歸朝,命知代州。上言曰:



 國家創業將四十年,陛下紹二聖之祚,精求至治。若守舊規,斯未盡善。能立新法,乃顯神機。



 臣以益州稍靜,望陛下選賢能以鎮之,必須望重有威,即群小畏服。又西鄙今雖歸明,他日未可必保,茍有翻覆,須得人制御,若以契丹比議,為患更深。何者?契丹則君臣久定,蕃、漢久分,縱萌南顧之心,亦須自有思慮。西鄙積恨未泯,貪心不悛,其下猖狂,竟謀兇惡,侵漁未必知足,姑息未能感恩,望
 常預備之。以良將守其要害,以厚賜足其貪婪,以撫慰來其情,以寬假息其念。多命人使西入甘、涼,厚結其心,為我聲援,如有動靜,使其掩襲,令彼有後顧之憂,乃可制其輕動。今甲兵雖眾,不及太祖之時人人練習,謀臣猛將則又縣殊,是以比年西北屢遭侵擾,養育則月費甚廣,征戰則軍捷未聞。誠願訓練禁戢,使如往日行伍必求於勇敢,指顧無縱於後先,失律者悉誅,獲功者必賞。偏裨主將,不威嚴者去之。聽斷之暇,親臨殿庭,更召
 貔虎,使其擊刺馳驟,以彰神武之盛。



 臣又以宰相、樞密,朝廷大臣,委之必無疑,用之必至當。銓總僚屬,評品職官,內則主管百司,外則分治四海。今京朝官則別置審官,供奉、殿直則別立三班,刑部不令詳斷,別立審刑,宣徽一司全同散地。大臣不獲親信,小臣乃謂至公。至如銀臺一司,舊屬樞密,近年改制,職掌甚多,加倍置人,事則依舊,別無利害,虛有變更。臣欲望停審官、三班,復委中書、樞密、宣徽院,銀臺司復歸樞密,審刑院復歸刑部,
 去其繁細,省其頭目。



 又京府大都,萬方軌則,望仍舊貫,選委親賢。今皇族宗子悉多成長,但令優逸,無以試材,宜委之外藩,擇文武忠直之士,為左右贊弼之任。



 又天下州縣官吏不均,或冗長至多,或歲年久闕。欲望縣四千戶已上選朝官知,三千戶已上選京官知。省去主簿,令縣尉兼領其事。自餘通判、監軍、巡檢、監臨使臣並酌量省減,免虛費於利祿,仍均濟於職官。



 又人情貪競,時態輕浮,雖骨肉之至親,臨勢利而多變。同僚之內,多或
 不和,伺隙則致於傾危,患難則全無相救,仁義之風蕩然不復。欲望有頒告諭,各使改更,庶厚化原,永敦政本。



 恭惟太祖神武,太宗聖文,光掩百王,威加萬國,無賢不用,無事不知。望陛下開豁聖懷,如天如海,可斷即斷,合行即行,愛惜忠直之臣,體察奸諛之黨。臣久塵著位,寢荷恩寵,辭狂理拙,唯聖明恕之!



 開至州,葺城壘戰具,諸將多沮議不協。開謂其從子曰:「吾觀昂宿有光,雲多從北來犯境上,寇將至矣。吾聞師克在和,今諸將怨我,一
 旦寇至,必危我矣。」即求換郡,徙忻州刺史。及契丹犯邊,開上書,又請車駕觀兵河朔。四年,徙滄州,道病首瘍卒,年五十四。錄其子涉為三班奉職。



 開善射,喜弈棋。有集十五卷。作《家戒》千餘言,刻石以訓諸子。性倜儻重義。在大名,嘗過酒肆飲,有士人在旁,辭貌稍異,開詢其名,則至自京師,以貧不克葬其親,聞王祐篤義,將丐之。問所費,曰:「二十萬足矣。」開即罄所有,得白金百餘兩,益錢數萬遣之。



 開兄肩吾,至御史。肩吾三子,湜、灝、沆並進士第,
 灝秘書丞。



 夏侯嘉,正字會之,江陵人,少有俊才。太平興國中舉進士,歷官至著作佐郎。使於巴陵,為《洞庭賦》曰:



 楚之南有水曰洞庭,環帶五郡,淼不知其幾百里。臣乙酉夏使岳陽,抵湖上,思構賦。明日披襟而觀之,則翼然動,促然跂,心慄然駭,愕然眙。怳若駕春雲而軾霓,浩若浮汗漫而朝躋。退若據泰山之安,進若履千仞之危。懵若無識,智若通微。跛若不倚,蹌若將馳。耳不及掩,目不暇逃,情悸心
 嬉。二三日而後,神始宅,氣始正,若此不敢以賦為事者二年,然眷眷不已。



 一日登崇丘,望大澤,有雲崪兮興,欻兮止。興止未霽,急若有遇。由是漬陽輝,沐芳澤,睹一異人於巖之際,霞為裾,云為袂,冰膚雪肌,金玦玉佩,浮丘、羨門,斯實其對。



 因言曰:「若非好辭者耶?」臣曰:「然。」「然則若智有所不通,識有所不窮,用不通不窮而循乎無端之紀,若得無殆乎?」臣又曰:「然。」「然志極則物應,思精則道來,嘉若之勤無嘩談,吾為若稱云:『太極之生,曰地曰天。中
 含五精,五精之用而水居一焉。水之疏,邇則為江兮,遠則為河;積則為瀦兮,總則為湖。若今所謂洞庭者,傑立而孤,廓然如無區,其大無徒。含陽字陰,玄神之都。曖曖昧昧,百川不敢逾。有若臣者,有若賓者,有若僕者,有若子者,有若附庸者,有若娣姒者。若禹會塗山,武巡牧野,千出百會,咸處麾下。每六合澄靜,中流回睨。莽莽蒼蒼,纖靄不翳。太陽望舒,出沒其間。萬頃咸沸,強而名之為巨澤,為長川,為水府,為大淵。縱之不逾,跼忠心之不卑。乍若
 賢人,以重自持。誘之不前,犯之愈堅。又若良將,以謀守邊。澎澎濞濞,浩爾一致。又若太始,未有仁義。沖訓漠漠,二氣交錯。又若混沌,凝然未鑿。此乃方輿之心胸,溟海之郛郭也。三代之前,其氣濩落。浩浩滔天,與物回薄。滅木襄陵,無際無廓。上帝降鑒,巨人斯作。乃命玄夷,授禹之機。隧山陻谷,滌源暢微。然後若金在熔,若木在工,流精成器,夫何不通。是澤之設,允執厥中。既巽其性,遂得其正。有升有降,有動有靜。』」



 臣應之曰:「升降動靜,可得聞
 乎?」神曰:「水之性非圓非方,非柔非剛,非直非曲,非玄非黃。劃象為《坎》,本乎羲皇。外婉而固,內健而彰。降以《姤》始,升以《復》張。其靜處陰,其動隨陽。六府之甲,萬化之綱。式觀是澤,乃知天常。若乃四序之變,九夏攸處。烘然而炎,沸然而煮。群物鴻洞,爍為隆暑。澤之作,頎然其容,若去若住,若茹若吐。靈趨怪覲,杳不可睹。蒸之為雲,散之為雨。倏急萬象,如還太古。真可嘉也。若乃秋之為神,素氣清泚。肅肅翛閹,群籟四起。澤之動,黝然其姿,若挺若倚,
 若行若止,《巽》宮離離,為之騰風。蒼梧崇崇,為之供云。四顧一色,黯然氤氳。其聲瀰瀰,若商非商,若徵非徵。東湊海門,一浪千里。又足畏也。言其狀,則石然而骨,岸然而革。氣然而榮,洚然而脈。有山而心,有洞而腹。有玉而體,有珠而目。穹鼻孤島,呀口萬穀。臂帶三吳,足跬荊、巫。或跂然而望,或翼然而趨。彭蠡、震澤,詎可云乎?」



 臣又問曰:「澤之態已聞命矣。水之族將如何居?」神曰:「大道變易,或文或質。沉潛自遂,其類非一。或被甲而邅,或曳裾而圓。
 或禿而跂,或角而蜿。或吞而呀,或呿而牙。或心以之蟹,或目以之蝦。或修臂而立,或橫騖而疾。或發於首,或髯於肘。或儼而莊,或毅而黝。彪彪玢玢,若大虛之含萬匯,名循其生而合乎群者也。」



 臣又問曰:「若神之資,其品何如也?」神曰:「清矣靜矣,麗矣至矣,邈難知矣。肇於古,古有所未達;形於今,今有所未察。非希夷合其心於自然,然後上天入地,把三根六。況水居陸處,夫何不燭。彼鞚鯉之賢,轡龍之仙,乃吾之肩也。其餘海若、天吳,陽侯、神胥,
 齪齪而游,曾不我儔。」



 臣又問曰:「《易》稱『王公設險』,是澤之險可以為固。而歷代興衰,其義安取?」神曰:「天道以順不以逆,地道以謙不以盈。故治理之世,建仁為旌,聚心為城。而弧不暇弦,矛不暇鋒,四海以之而大同。何必恃險阻,何必據要沖?若秦得百二為帝,齊得十二而王。其山為金,其水為湯。守之不義,欻然而亡。水不在大,恃之者敗。水不在微,怙之者危。若漢疲於昆明,桀困於酒池,亦其類也。故黃帝張樂而興,三苗棄義而傾。則知洞庭之
 波以仁不以亂,以道不以賊,惟賢者觀其知而後得也。」



 於是盤桓徙倚,凝精流視。罄以辭對,倏然而晦。



 徐鉉見之,曰:「是玄虛之流也。」人多傳寫。



 端拱初,太宗知其名,召試辭賦,擢為右正言、直史館兼直秘閣,賜緋魚。元夕,上禦乾元門觀燈,嘉正獻五言十韻詩,其末句云:「兩制誠堪羨,青雲侍玉輿。」上依韻和以賜之,有「狹劣終雖舉,通才列上居」之句,議者以為誡嘉正之好進也。未幾被病,詔以為益王生辰使。所獲金幣,鬻得錢輦歸家,忽一緡
 自地起立,良久而僕,聞者異之。嘉正疾遂篤,月餘卒,年三十七。



 子紓,太子中舍。



 羅處約,字思純,益州華陽人,唐酷吏希奭之裔孫。伯祖袞,唐末為諫官。父濟,仕蜀為升朝官。歸朝,至太常丞。處約嘗作《黃老先六經論》,曰:



 先儒以太史公論道德,先黃、老而後《六經》,此其所以病也。某曰:「不然,道者何?無之稱也,無不由也。混成而仙,兩儀至虛而應萬物,不可致詰。況名之曰『道』,道既名矣,降而為聖人者,為能知來藏往,
 與天地準,故黃、老、姬、孔通稱焉。其體曰道,其用曰神,無適也,無莫也,一以貫之,胡先而尊,孰後而卑。」



 「《六經》者,《易》以明人之權而本之於道;《禮》以節民之情,趣於性也;《樂》以和民之心,全天真也;《書》以敘九疇之秘,煥二帝之美;《春秋》以正君臣而敦名教;《詩》以正風雅而存規戒。是道與《六經》一也。」



 「矧仲尼祖述堯、舜,則況於帝鴻氏乎?華胥之治,太上之德,史傳詳矣。老聃世謂方外之教,然而與《六經》皆足以治國治身,清凈則得之矣。漢文之時,未遑
 學校,竇后以之而治,曹參得之而相,幾至措刑。且仲尼嘗問禮焉,俗儒或否其說。」



 餘曰:「《春秋》昭十七年,郯子來朝,仲尼從而學焉,俾後之人敦好問之旨。矧老子有道之士,周之史氏乎?余謂《六經》之教,化而不已則臻於大同,大道之行則蠟賓息嘆。黃、老之與《六經》,孰為先而孰為後乎?又何必繅藉玉帛然後為禮,筍虡鏞鼓然後為樂乎?余謂太史公之志,斯見之矣。惡可以道之跡、儒之末相戾而疾其說?病之者可以觀徼,未可以觀妙。」



 人多
 重之。



 登第,為臨渙主簿,再遷大理評事、知吳縣。王禹偁知長洲縣,日以詩什唱酬,蘇、杭間多傳誦。後並召赴闕,上自定題以試之,以禹偁為右拾遺,處約著作郎,皆直史館,賜緋魚。會下詔求讜言,處約上奏曰:



 伏睹今年春詔旨,責以諫官備員未嘗言事,雖九寺、三監之官,亦得盡其讜議。陛下虔恭勞神,厲精求理,力行王道,坐致太平。心先天而不違,德生民而未有,所以散玄黃之協氣,為動植之休祥,而猶不伐功成,屢求獻替,此真唐堯、虞
 舜之用心也。



 臣累日以來,趨朝之暇,或於卿士之內預聞時政之言,皆曰聖上以三司之中,邦計所屬,簿書既廣,綱條實繁,將求盡善之規,冀協酌中之道。竊聞省上言,欲置十二員判官兼領其職,貴各司其局,允執厥中。臣以三司之制非古也。蓋唐朝中葉之後,兵寇相仍,河朔不王,軍旅未弭,以賦調筦榷之所出,故自尚書省分三司以董之。然國用所須,朝廷急務,故僚吏之屬倚注尤深。或重其位以處之,優其祿以寵之,黽勉從事者姑
 務其因循,盡瘁事國者或生於睚眥,因循則無補於國,睚眥則不協於時。或淺近之人用指瑕於心計,深識之士以多可為身謀。蠹弊相沿,為日已久。今若如十二員判官之說,亦從權救敝之一端也。



 然而聖朝之政臻乎治平,當求稽古之規,以為垂世之法。臣嘗讀《說命》之書,以為「事不師古,匪說攸聞」又《二典》曰:「若稽古帝堯。」「若稽古帝舜。」皆謂順考古道而致治平。以臣所見,莫若復尚書都省故事,其尚書丞郎、正郎、員外郎、主事、令史之屬,
 請依六典舊儀。以今三司錢刀粟帛筦筦榷支度之事,均在二十四司,如此則各有司存,可以責其集事。今則金部、倉部安能知儲廩帑藏之盈虛,司田、司川孰能知屯役河渠之遠近?有名無實,積久生常。況此卻復都省之事,下臣猶能僉知其可,況陛下聰明濬哲乎!



 然議者以為不行已久,難於改更,若斷自宸心,下於相府,都省之制,故典存焉。上令下從,孰為不可,蓋人者可與習常,難與適變;可與樂成,難與慮始。在《周易》有之:「天地革而四
 時成。」此言能改命而創制,及小人樂成則革面以順上矣。況三司之名興於近代,堆案盈幾之籍,何嘗能省覽之乎?復就三司之中,更分置僚屬,則愈失其本原矣。今三司勾院即尚書省,比部元為勾覆之司,周知內外經費,陛下若欲復之,則制度盡在。迨及九寺、三監多為冗長之司,雖有其民,不舉其職。



 伏望陛下當治平之日,建垂久之規,不煩更差使臣,別置公署。如此則名正而言順,言順而事成,省其冗員則息其經費,故《書》曰:「唐虞稽
 古,建官惟百。夏、商官倍,亦克用乂。」伏望法天地簡易之化,建《洪範》大中之道,可以億萬斯年,垂衣裳而端拱矣。



 受詔荊湖路巡撫,欲以苛察立名,所奏劾甚眾,官吏多被黜責。淳化三年卒,年三十三。



 初,濟為開封府司錄,太宗尹京,頗嘉其強幹。太平興國中,處約與兄賁同舉進士,上臨試,知賁,濟之子,遂置之高等。八年,處約復登第。賁後至員外郎。



 處約形神豐碩,見者加重,雖有詞採而急於進用,時論亦以此薄之。卒後,蘇易簡、王禹偁集其
 文凡十卷,題曰《東觀集》。禹偁為序,易簡表上之,詔付史館。



 蜀士又有嚴儲者,太平興國中進士,後直史館,使河北督軍糧,陷於契丹。



 安德裕,字益之,一字師皋,河南人。父重榮,晉成德軍節度,《五代史》有傳。德裕生於真定,未期,重榮舉兵敗,乳母抱逃水竇中。將出,為守兵所得,執以見軍校秦習,習與重榮有舊,因匿之。習先養石守瓊為子,及年壯無嗣,以德裕付瓊養之,因姓秦氏。習世兵家,以弓矢、狗馬為事。
 德裕孩提即喜筆硯,遇文字輒為誦讀聲,諸子不之齒,習獨異之。既成童,俾就學,遂博貫文史,精於《禮》、《傳》,嗜《西漢書》。習卒,德裕行三年服,然後還本姓。習家盡以TY裝與之,凡白金萬餘兩。德裕卻之,曰:「斯秦氏之蓄,於我何有?丈夫當自樹功名,以取富貴,豈屑於他人所有耶!」聞者高之。



 開寶二年,擢進士甲科、歸州軍事推官,歷大理寺丞、著作佐郎。太平興國中,累遷秘書丞、知廣濟軍。時軍城新建,德裕作《軍記》及《圖經》三卷,優詔嘉獎。俄改太
 常博士。八年,通判秦州,就知州事。雍熙初,遷主客員外郎、通判廣州,未行,宰相李昉言其有史才,即以本官直史館。端拱初,改金部員外郎。



 淳化初,知開封縣,會備三館職,改直昭文館。三年春,廷試貢士,德裕與史館修撰梁周翰並為考官,上顧宰相曰:「此皆有聞之士而老於郎署,周翰狹中,德裕嗜酒,朕聞其能改矣。」遂並賜金紫。俄遷司勛員外郎。至道初,德裕常作《九弦琴五弦阮頌》以獻,上稱其詞採古雅。至道三年,轉金部郎中、出知睦
 州,還判太府寺。咸平五年卒,年六十三。



 德裕性介潔,以風鑒自負。王禹偁、孫何皆初游詞場,德裕力為延譽。及領考試,何又其首選。然酣飲太過,故不被獎擢。有集四十卷。



 錢熙,字太雅,泉州南安人。父居讓,陳洪進署清溪令。熙幼穎悟,及長,博貫群籍,善屬文,洪進嘉其才,以弟之子妻之。將署熙府職,辭不就,著《楚雁賦》以見志。尋復闢為巡官,專掌箋奏。



 洪進歸朝,熙不敘舊職,舉進士。雍熙初,
 攜文謁宰相李昉,昉深加賞重,為延譽於朝,令子宗諤與之游。明年,登甲科,補度州觀察推官。代還,寇準掌吏部選,上封薦錢若水、陳充、王扶洎熙皆有文,得試中書,遷殿中丞,賜緋魚。著《四夷來王賦》以獻,凡萬餘言,太宗嘉之,即以本官直史館。



 淳化中,參知政事。蘇易簡對太宗言趙鄰幾追補《唐實錄》,鄰幾卒,家睢陽,即命熙乘傳而往,盡取其書來上。熙嘗與楊徽之言及張洎、錢若水將被進用,熙與劉昌言同鄉里,相親善,又語及其事。昌
 言因以語洎,洎疑熙交構,訴之,熙坐削職、通判朗州,俄徙衡州,就改太常博士。真宗即位,遷右司諫。李宗諤、楊億素厚善熙,乃與梁顥、趙況、趙安仁同表請復熙舊職,不報。尋通判杭州,政多專達,為轉運使所奏,徙通判越州。



 熙負氣好學,善談笑,精筆札,狷躁務進。自罷職,因憤恚成疾,咸平三年卒,年四十八。嘗擬古樂府,著《雜言》十數篇及《措刑論》,為識者所許。有集十卷。



 子蒙吉,亦進士及第。



\end{pinyinscope}