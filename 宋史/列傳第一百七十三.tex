\article{列傳第一百七十三}

\begin{pinyinscope}

 ○史彌遠鄭清之史嵩之董槐葉夢鼎馬廷鸞



 史彌遠,字同叔,浩之子也。淳熙六年,補承事郎。八年,轉宣義郎,銓試第一,調建康府糧料院,改沿海制置司干
 辦公事。十四年,舉進士。紹熙元年,授大理司直。二年,遷太社令。三年,遷太常寺主簿,以親老請祠,主管沖祐觀。丁父憂。慶元二年,復為大理司直,尋改諸王宮大小學教授。輪對,乞旌廉潔之士,推舉薦之賞;浚溝洫,固堤防,實倉廩,均賦役,課農桑,禁末作,為水旱之備;葺城郭,修器械,選將帥,練士卒,儲粟谷,明烽燧,為邊鄙之防。丞相京鏜屏左右曰:「君他日功名事業過鏜遠甚,願以子孫為托。」四年,授樞密院編修官,遷太常丞,尋兼工部郎官,
 改刑部。六年,改宗正丞。丐外,知池州。嘉泰四年,提舉浙西常平。開禧元年,授司封郎官兼國史編修、實錄檢討,遷秘書少監,遷起居郎。二年,兼資善堂直講。



 韓侂胄建開邊之議,以堅寵固位,已而邊兵大衄,詔在位者言事,彌遠上疏曰:「今之議者,以為先發者制人,後發者制於人,此為將之事,施於一勝一負之間,則可以爭雄而捷出。若夫事關國體、宗廟社稷,所系甚重,詎可舉數千萬人之命輕於一擲乎?京師根本之地,今出戍既多,留衛
 者寡,萬一盜賊竊發,誰其御之?若夫沿江屯駐之兵,各當一面,皆所以拱護行都,尤當整備,繼今勿輕調發,則內外表裏俱有足恃,而無可伺之隙矣。所遣撫諭之臣,止令按歷邊陲,招集逋寇,戒飭將士,固守封圻。毋惑浮言以撓吾之規,毋貪小利以滋敵之釁,使民力愈寬,國勢愈壯,遲之歲月,以俟大舉,實宗社無疆之福。」



 奏方具,客曰:「侂胄必以奏議占人情,太夫人年高,能無貽親憂乎?」彌遠曰:「時事如此,言入而益於國,利於人,吾得罪甘
 心焉。」封鄞縣男兼權刑部侍郎。三年,改禮部兼同修國史、實錄院同修撰,仍兼刑部。



 兵端既開,敗衄相屬,累使求和,金人不聽。都城震搖,宮闈疑懼,常若禍在朝暮,然皆畏侂胄莫敢言。彌遠力陳危迫之勢,皇子詢聞之,亟具奏,乃罷侂胄並陳自強右丞相。既而臺諫、給舍交章論駁,侂胄乃就誅。召彌遠對延和殿,帝欲命為簽書樞密院事,力辭,乃遷禮部尚書兼國史實錄院修撰。



 詢立為太子,兼詹事,遣使詣金求和,金人以大散隔牙二關、
 濠州來歸,疏奏:「今兩淮、襄、漢沿邊之地,瘡痍未瘳,軍實未充。當勉厲將帥,盡吾委寄之誠;簡閱士卒,核其尺籍之闕。繕城堡,葺器械,儲糗糧。當聘使既通之後,常如干戈未定之日,推擇帥守以壯藩屏之勢,獎拔智勇以備緩急之求。」拜同知樞密院事兼太子賓客,進封伯。



 嘉定元年,遷知樞密院事,進奉化郡侯兼參知政事,拜右丞相兼樞密使兼太子少傅,進開國公。丁母憂,歸治葬,太子請賜第行在,令就第持服,以便咨訪。二年,以使者趣
 行急,乃就道,起復右丞相兼樞密使兼太子少師。四年,落起復。雪趙汝愚之冤,乞褒贈賜謚,厘正誣史,一時偽學黨人朱熹、彭龜年、楊萬里、呂祖儉雖已歿,或褒贈易名,或錄用其後,召還正人故老於外。十四年,賜家廟祭器。



 寧宗崩,擁立理宗,於是拜太師,依前右丞相兼樞密使,進封魏國公,六辭不拜,因乞解機政,歸田里,亟出關,帝從之。實慶二年,拜少師,賜玉帶。勸上傾心順承以事太后,力學修德以答皇天眷祐,以副四海歸戴。紹定元
 年,上太后尊號,拜太傅,八辭不拜。夏,得疾,累疏丐歸,不許。都城災,五疏乞罷斥,乃降封奉化郡公。五年春,復爵。六年,將拜太師,三具奏辭,乞免出命,不許。乃拜太師,依前右丞相兼樞密使、魯國公,又三具奏辭。紹定五年,上疏乞謝事,拜太傅。未幾,拜太師、左丞相兼樞密使。上疏乞解機政,依前太師特授保寧、昭信軍節度使,充醴泉觀使,進封會稽郡王。卒,遺表聞,帝震悼,輟朝三日,特贈中書令,追封衛王,謚忠獻。戶部支賻贈銀絹以千計,內
 帑特頒五千匹兩,遣使祭奠。及其喪還,遣禮官致路祭於都門外,賜纛、佩玉、黝纁。



 初,誅李全,復淮安,克盱眙,第功行賞,諸將皆望不次拔擢。或言於彌遠,彌遠曰:「御將之道,譬如養鷹,饑則依入,飽則揚去。曹彬下江南,太祖未肯以使相與之。況今邊戍未撤,警報時聞,若諸將一一遂其所求,志得意滿,猝有緩急,孰肯效死?」趙善湘以從官開閫,指授之功居多,日夜望執政。彌遠曰:「天族於國有嫌,高宗有詔止許任從官,不許為執政。紹熙末,慶
 元初,因汝愚、彥逾有定策功,是以權宜行之。某與善湘姻家,則又豈敢。」彌遠親密友周鑄、兄彌茂、甥夏周篆皆寄以腹心,人皆謂三人者必顯貴,然鑄老於布衣,彌茂以執政恩入流,周篆以捧香恩補官,俱止訓武郎而已。



 初,彌遠既誅韓侂胄,相寧宗十有七年。迨寧宗崩,廢濟王,非寧宗意。立理宗,又獨相九年,擅權用事,專任儉壬。理宗德其立己之功,不思社稷大計,雖臺諫言其奸惡,弗恤也。彌遠死,寵渥猶優其子孫,厥後為制碑銘,以「公
 忠翊運,定策元勛」題其首。濟王不得其死,識者群起而論之,而彌遠反用李知孝、梁成大等以為鷹犬,於是一時之君子貶竄斥逐,不遺餘力雲。



 鄭清之字,德源,慶元之鄞人。初名燮,字文叔。少從樓昉學,能文,樓鑰亟加稱賞。嘉泰二年,入太學。十年,登進士第,調峽州教授。帥趙方嚴重,靳許可,清之往白事,為置酒,命其子範、葵出拜,方掖清之無答拜,且曰:「他日願以二子相累。」湖北茶商群聚暴橫,清之白總領何炳曰:「此
 輩精悍,宜籍為兵,緩急可用。」炳亟下召募之令,趨者雲集,號曰「茶商軍」,後多賴其用。調湖、廣總所準備差遣、國子監書庫官。十六年,遷國子學錄。丞相史彌遠與清之謀廢濟國公,事見《皇子竑傳》。俄以清之兼魏惠憲王府教授,遷宗學諭,遷太學博士,皆仍兼教授。寧宗崩,丞相入定策,詔旨皆清之所定。



 理宗即帝位,授諸王宮大小學教授,遷宗學博士、宗正寺丞兼權工部郎、兼崇政殿說書。帝問外人因閣子庫進絲履有謗議,清之言:「禁中
 服用頗事新潔者。」帝曰:「故事,月進鞵數兩,朕非敝不易,何由致謗?」清之奏:「孝宗繼高宗,故儉德易章,陛下繼寧考,故儉德難著。寧考自奉如寒士,衣領重浣,革舄屢補,今欲儉德著聞,須過於寧考方可。」帝嘉納。



 寶慶元年,改兼兵部兼國史院編修官、實錄院檢討官,遷起居郎,仍兼史官、說書、樞密院編修官。二年,權工部侍郎,暫權給事中,進給事中,升兼同修國史、實錄院同修撰。紹定元年,遷翰林學士、知制誥兼侍讀,升兼修國史實錄院修
 撰、端明殿學士、簽書樞密院事。三年,授參知政事兼簽書樞密院事。四年,兼同知樞密院事。六年,彌遠卒,命清之為右丞相兼樞密使。



 端平元年,上既親總庶政,赫然獨斷,而清之亦慨然以天下為己任,召還真德秀、魏了翁、崔與之、李𡌴、徐僑、趙汝談、尤焴、游似、洪咨夔、王遂、李宗勉、杜範、徐清叟、袁甫、李韶,時號「小元祐。」大者相繼為宰輔,惟與之終始辭不至,遣逸如劉宰、趙蕃皆見旌異。是時金雖亡而入洛之師大潰。二年,上疏乞罷,不可,拜
 特進、左丞相兼樞密使。三年八月,霖雨大風,四疏丐去。九月,禋祀雷變,請益力。乃授觀文殿大學士、醴泉觀使兼侍讀,四疏控辭,依舊大學士、提舉洞霄宮。及聞邊警,密疏:「恐陛下憂悔太過,以汩清明之躬,累剛大之志。」嘉熙三年,封申國公。四年,遣中使賜御書「輔德明謨之閣」,賜楮十萬緡為築室,乃日與賓客門生相羊山水間。



 淳祐四年,依前觀文殿大學士、醴泉觀使兼侍讀,屢辭不允,拜少保、觀文殿大學士、醴泉觀使兼侍讀,進封衛國
 公。趣入見,有旨賜第。五年正月,上壽畢,亦疏丐歸,不允。拜少傅,依前觀文殿大學士、醴泉觀使兼侍讀,進封越國公。居無何,喪其子士昌,決意東還,又不許。拜少師、奉國軍節度使,依前醴泉觀使兼侍讀、越國公,賜玉帶,更賜第於西湖之漁莊。進讀《仁皇訓典》,謂:「仁祖之仁厚,發為英明,故能修明紀綱,而無寬弛不振之患;孝宗之英明,本於仁厚,故能涵養士氣,而無矯勵峭刻之習。蓋仁厚、英明二者相須,此仁祖、孝宗所以為盛也。」帝褒諭之。



 六年,拜太保,力辭。故事,許回授子孫,清之請追封高祖洽,帝從之,蓋異恩也。七年,拜太傅、右丞相兼樞密使、越國公。中使及門,清之方放浪湖山,寓僧剎,竟夕不歸。詰旦內引,叩頭辭免,帝勉諭有外間所不及知者。甫退,則中使接踵而至。或請更化改元,清之曰:「改元,天子之始事,更化,朝廷之大端,漢事已非古,然不因易相而為之。」



 帝以邊事為憂,詔趙葵以樞使視師,陳靴以知樞密院事帥湖、廣,二人方辭遜,會清之再相,力主之,科降闢置
 無所留難,葵、靴遂往。於是戰於泗水、渦口、木庫,皆以捷聞。九年,拜太師、左丞相兼樞密使,辭太師不拜,依前太傅。每謂天下之財困於養兵,兵費困於生券,思所以變通之,遇調戍防邊,命樞屬量遠近以便其道塗,時緩急以次其遣發。又議移歲調兵屯以戍淮面,並軍分頭目以節廩稍,先移鎮江策勝一軍屯泗水,公私便之。



 諸路虧鹽,執其事者破家以償,清之核其犯科者追理,掛誤者悉蠲之,全活甚眾。沿江算舟之賦素重,清之次第停
 罷,如池之雁水義有大法場之目,其錢分隸諸司,清之奏罷其並緣漁取者,蓋數倍公家之入,合分隸者從朝廷償之。報下,清之方與客飲,舉杯曰:「今日飲此酒殊快!」四上謝事之章。



 十年,進《十龜元吉箴》,一持敬,二典學,三崇儉,四力行,五能定,六明善,七謹微,八察言,九惜時,十務實。疏奏:「敬天之怒易,敬天之休難,天怒可憂而以為易,天休可喜而以為難,何哉?蓋憂則懼心生,懼則怒可轉而為休;喜則玩心生,玩則休或轉而為怒。」帝大喜,命史
 官書之,賜詔獎諭。十一年,十疏乞罷政,皆不許。拜太師,力辭。有事於明堂,有旨閤門給扶掖二人,再賜玉帶,令服以朝。十一月丁酉,退朝感寒疾,危甚,猶以未得雪為憂。俄大雪,起曰:「百官賀雪,上必甚喜。」命掬雪床前觀之。累奏乞罷政,不允,奏不已,拜太傅、保寧軍節度使充醴泉觀使,進封齊國公致仕。卒,遺表聞,帝震悼,輟朝三日,特贈尚書令,追封魏郡王,賜謚忠定。



 清之不好立異,湯巾嘗論事侵清之,及清之再相,巾求去,清之曰:「己欲作
 君子,使誰為小人。」力挽留之。徐清叟嘗論列清之,乃引之共政。趙葵視師年餘,乞罷,上未有以處之,清之曰:「非使作相不足以酬勞,陛下豈以臣故耶?臣必不因葵來遽引退,臣願為左,使葵居右。」上訖從之,然葵竟不果來。



 清之代言奏對,多不存稿,有《安晚集》六十卷。清之自與彌遠議廢濟王竑,立理宗,駸駸至宰輔,然端平之間召用正人,清之之力也。至再相,則年齒衰暮,政歸妻子,而閑廢之人或因緣以賄進,為世所少云。



 史嵩之,字子由,慶元府鄞人。嘉定十三年進士,調光化軍司戶參軍。十六年,差充京西、湖北路制置司準備差遣。十七年,升乾辦公事。寶慶三年,主管機宜文字,通判襄陽府。紹定元年,以經理屯田,襄陽積穀六十八萬,加其官,權知棗陽軍。二年,遷軍器監丞兼權知棗陽軍,尋兼制置司參議官。三年,棗陽屯田成,轉兩官。以明堂恩,封鄞縣男,賜食邑。以直秘閣、京西轉運判官兼提舉常平兼安撫制置司參議官。四年,遷大理少卿兼京西、湖
 北制置副使。五年,加大理卿兼權刑部侍郎,升制置使兼知襄陽府,賜便宜指揮。六年,遷刑部侍郎,仍舊職。



 端平元年,破蔡滅金,獻俘上露布,降詔獎諭,進封子,加食邑。移書廟堂,乞經理三邊,不合,丐祠歸侍,手詔勉留之。會出師,與淮閫協謀掎角,嵩之力陳非計,疏為六條上之。詔令嵩之籌畫糧餉,嵩之奏言:



 臣熟慮根本,周思利害,甘受遲鈍之譏,思出萬全之計。荊襄連年水潦螟蝗之災,饑饉流亡之患,極力振救,尚不聊生,徵調既繁,夫
 豈堪命?其勢必至於主戶棄業以逃亡,役夫中道而竄逸,無歸之民,聚而為盜,饑饉之卒,未戰先潰。當此之際,正恐重貽宵旰之慮矣。兵民,陛下之兵民也,片紙調發,東西惟命。然事關根本,願計其成,必計其敗,既慮其始,必慮其終,謹而審之,與二三大臣深計而熟圖之。



 若夫和好之與進取,決不兩立。臣受任守邊,適當事會交至之沖,議論紛紜之際。雷同和附,以致誤國,其罪當誅;確守不移之愚,上迕丁寧之旨,罪亦當誅。迕旨則止於一
 身,誤國則及天下。



 丞相鄭清之亦以書言勿為異同,嵩之力求去。



 朝陵之使未還,而諸軍數道並進,復上疏乞黜罷,權兵部尚書,不拜。乞祠,進寶章閣直學士,提舉太平宮,歸養田里。尋以華文閣直學士知隆興府兼江西安撫使。帝自師潰,始悔不用嵩之言,召見,力辭,權刑部尚書。引見,疏言結人心、作士氣、核實理財等事。且言:「今日之事,當先自治,不可專恃和議。」乞祠,以前職知平江府,以母病乞侍醫藥,不俟報可而歸。進寶章閣學士、淮
 西制置使兼沿江制置副使兼知鄂州。既內引,賜便宜指揮,兼湖、廣總領兼淮西安撫使。嘉熙元年,進華文閣學士、京西荊湖安撫制置使,依舊沿江制置副使兼節制光、黃、蘄、舒。乞免兼總領,從之。



 廬州圍解,詔獎諭之。以明堂恩,進封伯,加食邑。條奏江、淮各三事,又陳十難,又言江陵非孟珙不可守,乞勉諭之。漢陽受攻,嵩之帥師發江陵,奏誅張可大,竄盧普、李士達,以其棄城也。二年,黃州圍解,降詔獎諭,拜端明殿學士,職任依舊,恩數視
 執政,進封奉化郡侯,加食邑。詔入覲,拜參知政事,督視東西、荊湖南北、江西路軍馬,鄂州置司,兼督視淮南西路軍馬兼督視光、蘄、黃、夔、施州軍馬,加食邑。城黃州。十一月,復光州。十二月,復滁州。三年,授宣奉大夫、右丞相兼樞密、都督兩淮四川京西湖北軍馬,進封公,加食邑,兼督江西、湖南軍馬,改都督江、淮、京、湖、四川軍馬。薦士三十有二人,其後董槐、吳潛皆號賢相。



 復信陽,以督府米拯淮民之饑。六月,復襄陽,嵩之言:「襄陽雖復,未易守。」
 自是邊境多以捷聞,降詔獎諭。四年,乞祠,趣召奏事,轉三官,依前右丞相兼樞密使,眷顧特隆,賜齎無虛日。久旱,乞解機政。地震,屢疏乞罷免,皆不許。淳祐元年,進《玉斧箴》。安南入貢,不用正朔,嵩之議用範仲淹卻西夏書例,以不敢聞於朝還之。二年,進高、孝、光、寧帝《紀》,《孝宗經武要略》,《寧宗實錄》、《日歷》,《會要》、《玉牒》,進金紫光祿大夫,加食邑。是冬,封永國公,加食邑。四年,遭父喪,起復右丞相兼樞密使。累賜手詔,遣中使趣行。於是太學生黃愷伯、
 金九萬、孫翼鳳等百四十四人,武學生翁日善等六十七人,京學生劉時舉、王元野、黃道等九十四人,宗學生與寰等三十四人,建昌軍學教授盧鉞,皆上書論嵩之不當起復,不報。將作監徐元傑奏對及劉鎮上封事,帝意頗悟。



 初,嵩之從子璟卿嘗以書諫曰:



 伯父秉天下之大政,必辦天下之大事;膺天下之大任,必能成天下之大功。比所行浸不克終,用人之法,不待舉削而改官者有之,譴責未幾而旋蒙敘理者有之,丁難未幾而遽被
 起復者有之。借曰有非常之才,有不次之除,醲恩異賞,所以收拾人才,而不知斯人者果能運籌帷幄、獻六奇之策而得之乎?抑亦獻賂幕賓而得之乎?果能馳身鞍馬,效一戰之勇而得之乎?抑亦效顰奴僕而得之乎?徒聞包苴公行,政出多門,便嬖私暱,狼狽萬狀,祖宗格法,壞於今日也。



 自開督府,東南民力,困於供需,州縣倉卒,匱於應辦,輦金帛,輓芻粟,絡繹道路,曰一則督府,二則督府,不知所乾者何事,所成者何功!近聞蜀川不守,議
 者多歸退師於鄂之失。何者?分戍列屯,備邊御戎,首尾相援,如常山之蛇。維揚則有趙葵,廬江則有杜伯虎,金陵則有別之傑。為督府者,宜據鄂渚形勢之地,西可以援蜀,東可以援淮,北可以鎮荊湖。不此之圖,盡損藩籬,深入堂奧,伯父謀身自固之計則安,其如天下蒼生何!



 是以饑民叛將,乘虛搗危,侵軼於沅、湘,搖蕩於鼎、澧。為江陵之勢茍孤,則武昌之勢未易守;刑湖之路稍警,則江、浙之諸郡焉得高枕而臥?況殺降失信,則前日徹疆
 之計不可復用矣;內地失護,則前日清野之策不可復施矣。此隙一開,東南生靈特幾上之肉耳。則宋室南渡之疆土,惡能保其金甌之無闕也。盍早為之圖,上以寬九重宵旰之憂,下以慰雙親朝夕之望。不然,師老財殫,績用不成,主憂臣辱,公論不容。萬一不畏強禦之士,繩以《春秋》之法,聲其討罪不效之咎,當此之時,雖優游菽水之養,其可得乎?異日國史載之,不得齒於趙普開國勛臣之列,而乃廁於蔡京誤國亂臣之後,遺臭萬年,果
 何面目見我祖於地下乎?人謂禍起蕭墻,危如朝露,此愚所痛心疾首為伯父苦口極言。



 為今之計,莫若盡去在幕之群小,悉召在野之君子,相與改弦易轍,戮力王事,庶幾失之東隅,收之桑榆矣。如其視失而不知救,視非而不知革,薰蕕同器,駑驥同櫪,天下大勢,駸駸日趨於危亡之域矣。伯父與璟卿,親猶父子也,伯父無以少年而忽之,則吾族幸甚!天下生靈幸甚!我祖宗社稷幸甚!



 居無何,璟卿暴卒,相傳嵩之致毒云。嵩之為公論所
 不容,居閑十有三年。寶祐四年春,授觀文殿大學士,加食邑。八月癸巳卒,遺表上,帝輟朝,贈少師、安德軍範度使,進封魯國公,謚忠簡,以家諱改謚莊肅。德祐初,以右正言徐直方言奪謚。



 董槐,字庭植,濠州定遠人。少喜言兵,陰讀孫武、曹操之書,而曰:「使吾得用,將汛掃中土以還天子。」槐貌甚偉,廣顙而豐頤,又美髯,論事慷慨,自方諸葛亮、周瑜。父永,遇槐嚴,聞其自方,怒而嘻曰:「不力學,又自喜大言,此狂生
 耳,吾弗願也。」槐心愧,乃益自摧折,學於永嘉葉師雍。聞輔廣者,朱熹之門人,復往從廣,廣嘆其善學。嘉定六年,登進士第,調靖安主簿。丁父憂去官。



 十四年,起為廣德軍錄事參軍,民有誣富人李桷私鑄兵結豪傑以應李全者,郡捕系之獄,槐察其枉,以白守,守曰:「為反者解說,族矣。」槐曰:「吏明知獄有枉,而擠諸死地以傅於法:顧法豈謂諸被告者無論枉不枉,皆可殺乎?」不聽。頃之,守以憂去,槐攝通判州事,嘆曰:「桷誠枉,今不為出之,生無繇
 矣。」乃為翻其辭,明其不反,書上,卒脫桷獄。紹定二年,遷鎮江觀察推官。明年春,入為主管刑部架閣文字。秋,兼權禮兵部架閣,遷籍田令,特差權通判鎮江府。至州,會全叛,涉淮臨大江,大府急發州兵。槐即日將兵濟江而西,全遁去,乃還。五年,丁母憂。端平三年,差通判蘄州,辭。



 嘉熙元年,召赴都堂,遷宗正寺簿、出知常州。後三日,提點湖北刑獄。常德軍亂,夜縱火而噪,守尉闖不出。槐騎從數人於火所,且問亂故。亂者曰:「將軍馬彥直奪吾歲
 請,吾屬將責之償,不為亂也。」槐坐馬上,召彥直斬馬前,亂者還入伍中,明日,乃捕首亂者七人戮諸市,而賻彥直之家。差充歸、峽、岳察訪使。二年,兼權知常德府,尋兼軍器少監,依舊提點刑獄。



 三年,以直寶謨閣知江州兼都督府參謀。秋,流民渡江而來歸者十餘萬,議者皆謂:「方軍興,郡國急儲粟,不暇食民也。」槐曰:「民,吾民也,發吾粟振之,胡不可?」至者如歸焉。當是時,宋與金為鄰國,而襄、漢、揚、楚之間,豪傑皆自相結以保其族,無賴者往往
 去為群盜。浮光人翟全寓黃陂,有眾三千餘,稍出鹵掠。



 槐令客說下全,徙之陽烏洲,使雜耕蘄春間,又享賜之,用為裨將。於是曹聰、劉清之屬皆來自歸。



 四年,進直華文閣、知潭州、主管湖南安撫司公事。方三邊急於守御,督府日夜徵發,民且困,槐為畫策應之,令民不傷而軍須亦不匱。淳祐二年,遷左司郎官,進直龍圖閣、沿江制置副使兼知江州、主管江西安撫司公事。視其賦則吏侵甚,下教曰:「蒞州而吏猶為盜不自悔,吾且誅之!」吏
 乃震恐,願自新。槐因除民患害,凡利有宜,弛以利民,惟恐不盡弛。大計軍實,常若敵且至。裨將盧淵兇猾不受命,斬以徇師,軍中肅然。



 三年,進秘閣修撰。四年,召入奏事,遷權戶部侍即,賜紫,進集英殿修撰、沿江制置使、江東安撫使兼知建康府兼行宮留守。軍政弛弗治,乃為賞三等以教射,春秋教肄士卒坐作進退擊刺之技,歲餘盡為精兵。六年,召至闕,辭。出知靜江府兼廣西經略安撫使,又辭。權廣西運判兼提點刑獄。宰相移書槐曰:「
 國家方用兵,人臣不辭急難,公幸毋固辭。」槐即日就道,至邕州,上守御七策。邕州之地西通諸蠻夷,南引交址及符奴、月烏、流鱗之屬,數寇邊,槐與約無相侵,推赤心遇之,皆伏不動。又與交址約五事:一無犯邊,二歸我侵地,三還鹵掠生口,四奉正朔,五通貿易。於是遣使來獻方物、大象南方悉定。



 七年,進寶章閣待制。八年,遷工部侍郎,職事依舊,兼轉運使。九年,召赴闕,封定遠縣男。遷兵部侍郎兼權給事中兼侍讀,升給事中,上疏請抑損
 戚裏恩澤以慰天下士大夫。群臣奏事少與法違,憚槐不敢上。兼侍讀,進寶章閣直學士、知福州福建安撫使,辭。進封子。是年冬,拜端明殿學士、簽書樞密院事,進封侯。十二年,為同知樞密院事。寶祐元年,權參知政事。二年,進參知政事。四川制置使餘晦以戰敗奪官,詔荊襄制置使李曾伯往視師,曾伯辭,槐曰:「事如此,尚可坐而睨乎?」上疏請行,頓重兵夔門以固荊、蜀輔車之勢,詔報曰:「腹心之臣,所與共理天下者也,宜在朝廷,不宜在四
 方。」復上疏曰:「天下之事,不進則退,人臣無敢為岐意者,茍以臣為可任,宜少聽臣自效,即臣不足與軍旅之事,願上官爵。」不許,進封濠梁郡公。



 帝日鄉用槐,槐言事無所隱,意在於格君心之非而不為容悅。帝問糴民粟積邊,則對曰:「吳民困甚,有司急糴不復省。夫民惟邦本,願先垂意根本。」帝問修太乙祠,則對曰:「土工薦起,民罷於徵發,非所以事天也。」帝問邊事,對曰:「外有敵國,則其計先自強。自強者人畏我,我不畏人。」又言:「敵國在前,宜拔
 材能用之。士大夫有過失,為執法吏所刺劾,終身擯弗用,深為朝廷惜此。茍非奸邪,皆願為昭洗,勿廢其他善。又遷謫之臣,久墮遐方,稍稍內徙,今得生還,顧弗用可矣。」槐每奏,帝輒稱善。



 三年,拜右丞相兼樞密使。槐自以為人主所振拔,茍可以利安國家無不為,然務先大體,任人先取故舊之在疏遠者,在官者率滿歲而遷。嗜進者始不說矣。槐又言於帝曰:「臣為政而有害政者三。」帝曰:「胡為害政者三?」對曰:「戚里不奉法,一矣;執法大吏久
 於其官而擅威福,二矣;皇城司不檢士,三矣。將率不檢下故士卒橫,士卒橫則變生於無時;執法威福擅故賢不肖混淆,賢不肖混淆則奸邪肆,賢人伏而不出;親戚不奉法故法令輕,法令輕故朝廷卑。三者弗去,政且廢,願自上除之。」於是嫉之者滋甚。



 帝年浸高,操柄獨斷,群臣無當意者,漸喜狎佞人。丁大全善為佞,帝躐貴之,竊弄威權而帝弗覺悟。大全已為侍御史,遣客私自結於槐,槐曰:「吾聞人臣無私交,吾惟事上,不敢私結約,幸為
 謝丁君。」大全度槐弗善己,銜甚,乃日夜刻求槐短。槐入見,極言大全邪佞不可近。帝曰:「大全未嘗短卿,卿勿疑。」槐曰:「臣與大全何怨?顧陛下拔臣至此,臣知大全邪而噤不言,是負陛下也。且陛下謂大全忠而臣以為奸,不可與俱事陛下矣。」既罷出,即上書乞骸骨,不報。四年,策免丞相,以觀文殿大學士提舉洞霄宮。時大全亦論劾槐,書未下,自發省兵迫遣之。於是太學諸生陳宜中等上書爭之,語見《大全傳》。



 五年及景定元年,俱用祀明
 堂恩加食邑。二年,特授判福州、福建路安撫大使,固辭。進封吉國,又進封許國公。三年五月二十八日既夕,天大雨,烈風雷電,槐起衣冠而坐,麾婦人出,為諸生說《兌》、《謙》二卦,問夜如何?諸生以夜中對,遂薨。遺表上,贈太子少師,謚文清。帝使使致金六十斤、帛千匹以賻。



 葉夢鼎,字鎮之,臺之寧海人。本陳待聘之子,七歲後於母族。少從直龍圖閣鄭霖、宗正少卿趙逢龍學,以太學上舍試入優等,兩優釋褐出身,授信州軍事推官,攝教
 事,講荒政。遷太學錄。



 淳祐二年,雷變,上封事,言召人才,戒媟近。明年,輪對,言君子、直言、軍制、楮幣、任官、分閫六事。同番易湯巾召試館職,授秘書省正字。四年,升校書郎兼莊文府教授。五年,遷秘書郎,轉對,言定國本,求哲輔,專閫帥,獎用介直。雷變上言,援唐康澄「五可畏」之說,遷著作佐郎。六年,拜軍器少監兼兵部郎官,轉對,言國計、邊事、國體三事。又言:「外有窺邊之大敵,內有伺隙之巨奸;奇邪蠱媚於宮闈,熏腐依憑於城社;強藩悍將,牙
 蘗易搖,草竊奸宄,肘腋階變。」



 權知袁州,轉運司和糴米三萬斛,夢鼎言:「袁山多而田少,朝廷免和糴已百年,自今開之,百姓子孫受無窮之害,則無窮之怨從之。」民湯頎獻田學官,妻子離散,夢鼎遂還之。毀萬載旗𥮉村淫祠,塞其妖井。召赴行在。丁本生母憂。十一年,免喪,拜司封員外郎。輪對,言:「陛下惑於左右之讒說,例視言者為好名,中傷既深,膠固莫解。近歲以來,言稍犯人主之所難者,不顯罷則陰黜,不久外則設間,去者屢召而不還,
 來者一鳴而輒斥。」兼玉牒檢討官,以直秘閣、江西提舉常平兼知吉州。節制悍將,置社倉、義倉,平反李義山受贓之冤,以國子司業召。



 寶祐元年陛對,言國論主平江西義倉,不可待申省而後發。考試集英殿,授崇政殿說書,進講《尚書》。兼國史編修、實錄檢討,遷國子祭酒。二年,兼權禮部侍郎,諫幸西太乙宮。三年,權禮部侍郎,仍兼祭酒,升兼同修國史、實錄院同修撰,尋兼侍講。丁母憂。五年,以集英殿修撰差知贛州。丁大全柄國,欲挽夢鼎
 登朝,卒辭謝之。六年,改知建寧府,又改知隆興府。開慶元年,復知建寧府,作橋梁,置驛舍,建大安關,決疑獄。



 景定元年,召為太子詹事,上疏以「法天」為言。遷吏部侍郎,賜寧海縣食邑。二年,權兵部尚書兼權吏部尚書。三年,遷兵部尚書兼修國史兼實錄修撰。遷吏部尚書,五辭免,請祠,不允。拜端明殿學士、同簽書樞密院事,屢辭不許。同提舉編修《經武要略》兼太子賓客,進封寧海伯。四年,簽書樞密院事,進封臨海郡侯,以明堂恩進封臨海
 郡公。丞相賈似道欲造關子,罷十七、十八兩界會子,夢鼎以為厲民,乃止罷十七界。公田法行,夢鼎又以為厲民,故行之浙右而止。五年,三辭,不許,進同知樞密院事、權參知政事。以彗星出,夢鼎言政上下恐懼交修之日,乞解機政,又不許。奏免浙西經界。



 理宗崩,議太子即位,太后垂簾聽政,夢鼎曰:「母後垂簾,豈是美事!」進參知政事,加食邑。夢鼎力辭,似道懇留之,不可。帝勉諭再三,詔閣門封還奏疏。似道奏:「參政去則江萬里、王爚必不來。」
 理宗復土,攝少傅,竣事,引疾歸里,累詔,力辭,授資政殿學士、知慶元府、沿海制置使。肅清海寇,罪止首惡,羨餘之費,悉卻不受。建濟民倉以備饑歲,造驛舍以待賓旅。



 咸淳三年,再召為參知政事,加食邑,六辭,不許。詔著作佐郎盧鉞與臺州守項公採趣行,拜特進、右丞相兼樞密使,累辭,不許,乃與似道分任。利州轉運使王價嘗以言去官,非其罪也,四川制置司已闢參議,及死,其子訴求遺澤。至是,夢鼎明其無罪,似道以為恩不己出,罷省
 部吏數人,榜其姓名於朝。夢鼎怒曰:「我斷不為陳自強。」即求去。似道之母讓似道曰:「葉丞相安於家食,未嘗希進,汝強與以相印,今乃牽制至此,若不從吾言,吾不食矣。」似道曰:「為官不得不如此。」會太學諸生亦上書言似道專權固位,乃悔悟,屬府尹洪燾求解,而夢鼎屢上章乞閑。冬雷,引咎求去愈力。



 四年,策楊妃,宰相無拜禮,吏贊拜,夢鼎以笏揮之,趨出。明日,乞還田里,詔勉留之。詔免諸州守臣上殿奏事,夢鼎言:「祖宗謹重牧守之寄,將
 赴官,必令奏事,蓋欲察其人品,及面諭以廉律己,愛育百姓。其至郡延見吏民,具宣上意,庶幾求無負臨遣之意。今不遠數千里而來,咫尺天顏而不得見,甚非立法之本意。」又乞容受直言。進少保。五年,引杜衍致仕單車宵遁故事累辭,乃授觀文殿學士、判福州、福建安撫大使,進封信國公,不拜;充醴泉觀使,又不拜。七年,再充醴泉使。



 九年,授少傅、右丞相兼樞密使,引疾力辭,宰、掾、郎、曹沓至趣行,扶病至嵊縣,請辭不獲,乞還山林。疏奏:「願
 上厲精寡欲,規當國者收人心,固邦本,勵將帥,飭州縣,重振恤。」扁舟徑歸。使者以禍福告,夢鼎語之曰:「廉恥事大,死生事小,萬無可回之理。」似道大怒,臺臣奏從歸田之請,詔仍少保、觀文殿大學士、醴泉觀使,不請祠祿。



 瀛國公初即位,咨訪故老,夢鼎上封事,曰:「敦教道,訓廉德,厲臣節,拯民瘼,重士選,勸吏廉,懲吏奸,補軍籍。授判慶元府、沿海制置大使,力辭,依前醴泉觀使兼侍讀,不拜。二年,益王即位於閩,召為少師、太乙宮使。航海遂行,道
 梗不能進,南向慟哭失聲而還。後二年卒。子應及,太府寺丞、知建德府軍器少監、駐戍軍馬;應有,朝請郎、太社令。



 馬廷鸞,字翔仲,饒州樂平人。本灼之子,繼灼兄光後。甘貧力學,既冠,里人聘為童子師,遇有酒食饌,則念母藜藿不給,為之食不下咽。登淳祐七年進士第,調池州教授,需次六年。



 寶祐元年,召赴都堂審察,辭。至池以禮帥諸生。二年,調主管戶部架閣。三年,遷太學錄,召試館職。
 時外戚謝堂厲文翁、內侍盧允升董宋臣用事,廷鸞試策言強君德,重相權,收直臣,防近習。大與時迕,遷秘書省正字。四年,尤焴提舉史事,闢為史館校勘。



 初,丁大全令浮梁,雅慕廷鸞,彌欲鉤致之,廷鸞不為動。試策稍及大全,及廷鸞當輪對,大全私謂王持垕往𣊺焉。廷鸞素厚持垕且同館,不虞其諜也,密露大意。持垕紿曰:「君猶未改秩,姑托疾為後圖乎?」廷鸞曰:「此微臣千一之遭,其何敢不力。」持垕以告大全,及候對殿門,格不得見。翼日,
 以監察御史朱熠劾罷。宋臣遣八廂貌士索奏稿,稿雖焚,聞者浸廣,忌者愈深,而廷鸞之名重天下。開慶元年,吳潛入相,召為校書郎。



 景定元年,兼沂靖惠王府教授。時大全黨多斥,宋臣尚居中,言路無肯言者,諸學官抗疏,疏上即行。會日食,與秘書省同守局,因相與草疏。潛以書告廷鸞曰:「諸公言事紛紛,皆疑潛所嗾,聞館中又將論列,校書宜無與,以重吾過。」廷鸞對曰:「公論也,不敢避私嫌。」越數日,宋臣竟坐謫,徙安吉州。兼權樞密院編
 修官。時賈似道自江上還,位望赫奕,廷鸞未嘗親之。輪對,言:「國於東南者,楚、越霸而有餘,東晉王而不足。乞遏惡揚善以順天,舉直錯枉以服民。」遷樞密院編修官兼權倉部郎官。



 二年,進著作佐郎兼右司,遷將作少監。三年,一再乞外補,不許。廷鸞論貢舉三事:嚴鄉里之舉,重臺省之覆試,訪山林之遺逸。又言荒政,宜蠲除被災州縣租賦之不可得者。擢軍器監兼左司,兼太子右諭德,升左諭德,行國子司業,乞免兼左司。輪對,言:「集和平之
 福者自陛下之身始,養和平之德者自陛下之心始。」兼翰林權直,擢秘書少監,升權直學士院。四年,擢起居舍人兼太子右庶子兼國史院編修官、實錄院檢討官。入奏言:「太史必當謹書災異。願陛下翕受敷施,以壯人才之精神;虛心容納,以植人言之骨幹。念邦本而以公滅私,嚴邊備而思患豫防。」時再召用宋臣,廷鸞引何郯之說進,極言宋臣不可用,帝從之。薦士二十人,進中書舍人。程奎污穢詭秘,不當補將仕郎;王之淵為大全黨,不
 當通判江州;朱熠不當知慶元府及為制置使;林奭、趙必、張稱孫不當與郡:皆繳還詞頭。兼國史實錄院。五年,彗出,上疏極言天人之際。遷禮部侍郎。理宗遺詔、度宗登極詔,皆廷鸞所草。兼侍讀,辭,不許。疏列孝宗之政以告。升直學士院。



 咸淳元年,進端明殿學士、簽書樞密院事兼同提舉編修《經武要略》。丁母憂。三年,同知樞密院事兼同提舉編修《經武要略》。入奏言培命脈,植根本,崇寬大,行仁厚。又言:「恢大度以優容,虛聖心而延佇,推
 內恕以假借,忍難行而聽納,則情無不達,理無不盡,奸人破膽,直士吐氣,天下事尚可為也。」兼權參知政事。五年,進參知政事兼同知樞密院事,進右丞相兼樞密使。八年,九疏乞罷政。九年,依舊觀文殿大學士、知紹興府、浙東安撫大使。上疏辭免,依舊職提舉臨安府洞霄宮。



 度宗初年,詔詢故老,專以修攘大計叩之趙葵。葵極意指陳曰:「老臣出入兵間,備諳此事,願朝廷謹之重之。」似道作色曰:「此三京敗事者,詞臣失言。」廷鸞每見文法密,
 功賞稽遲,將校不出死力,於邊閫升闢,稍越拘攣。似道頗疑異己,黥堂吏以洩其憤。及辭相位,帝惻怛久之曰:「丞相勉為朕留。」廷鸞言:「臣死亡無日,恐不得再見君父。然國事方殷,疆圉孔棘。天下安危,人主不知;國家利害,群臣不知;軍前勝負,列閫不知。陛下與元老大臣惟懷永圖,臣死且瞑目。」頓首涕泣而退。



 瀛國公即位,召不至。自罷相歸,又十七年而薨。所著《六經集傳》、《語孟會編》、《楚辭補記》、《洙泗裔編》、《讀莊筆記》、《張氏祝氏皇極觀物外篇》
 諸書。



 論曰:史彌遠廢親立疏,諱聞直言。鄭清之墮名於再相之日。彌遠之罪既著,故當時不樂嵩之之繼也,因喪起復,群起攻之,然固將才也。董槐毋得而議之矣。葉夢鼎、馬廷鸞之所遭逢,其不幸也夫!



\end{pinyinscope}