\article{本紀第四十一}

\begin{pinyinscope}

 理宗一



 理宗建道備德大功復興烈文仁武聖明安孝皇帝,諱昀,太祖十世孫。父希□盧,追封榮王,家於紹興府山陰縣,母全氏。以開禧元年正月癸亥生於邑中虹橋里第。前
 一夕,榮王夢一紫衣金帽人來謁,比寤,夜漏未盡十刻,室中五採爛然,赤光屬天,如日正中。既誕三日,家人聞戶外車馬聲,亟出,無所睹。幼嘗晝寢,人忽見身隱隱如龍鱗。是時,寧宗弟沂靖惠王薨,無嗣,以宗室希瞿子賜名均為沂王後,尋改賜名貴和。嘉定十三年八月,景獻太子薨,寧宗以國本未立,選太祖十世孫年十五以上者教育,如高宗擇普安、恩平故事,遂以十四年六月丙寅立貴和為皇子,改賜名竑,而以帝嗣沂王。六月乙亥,
 補秉義郎。八月甲子,授右監門衛大將軍,賜名貴誠。十五年五月丁巳,以竑為檢校少保,進封濟國公。己未,以帝為邵州防禦使。帝性凝重寡言,潔修好學,每朝參待漏,或多笑語,帝獨儼然。出入殿庭,矩度有常,見者斂容。會濟國公竑與丞相史彌遠有違言,彌遠日謀媒薛其失於寧宗,屬意於帝而未遂。



 十七年八月丙戌,寧宗違豫,自是不視朝。壬辰,疾篤,彌遠稱詔以貴誠為皇子,改賜名昀,授武泰軍節度使,封成國公。



 閏月丙申,寧宗疾
 甚,丁酉,崩於福寧殿。彌遠使楊谷、楊石入白楊皇后,稱遺旨以皇子竑開府儀同三司,進封濟陽郡王、判寧國府,命子昀嗣皇帝位。大赦。尊楊皇后曰皇太后,同聽政。封竑為濟王,賜第湖州,以醴泉觀使就第。癸亥,詔宮中自服三年喪。



 九月乙亥,詔褒表老儒,以傅伯成為顯謨閣學士,楊簡寶謨閣直學士,並提舉南京鴻慶宮。柴中行敘復元職,授右文殿修撰、主管南京鴻慶宮。戊寅,詔兄濟王妻衛國夫人吳氏封許國夫人。己卯,皇太后、皇
 帝御便殿垂簾。詔以先聖四十九代孫行可為迪功郎,授判、司、簿、尉;以禮部侍郎程珌、吏部侍郎朱著、中書舍人真德秀兼侍讀;工部侍郎葛洪、起居郎喬行簡、宗正少卿陳貴誼、軍器監王塈兼侍講。壬午,葛洪權工部尚書,升兼侍讀。辛卯,祀明堂,大赦。



 冬十月戊戌,詔諸路提點刑獄以十一月按理囚徒。己亥,嗣秀王師巖薨。壬子,詔百官奉按月給。



 十一月甲子,右正言麋溧請承順東朝,繼志述事,壹以孝宗為法,而新政之切者,曰畏天、悅
 親、講學、仁民。上嘉納焉。癸未,以五月十六日為皇太后壽慶節。丁亥,詔改明年為寶慶元年。戊子,以葛洪為端明殿學士、同簽書樞密院事。己丑,詔以生日為天基節。十二月甲午,雪寒,免京城官私房賃地、門稅等錢。自是祥慶、災異、寒暑皆免。癸丑,開經筵,詔輔臣觀講。詔太后所居殿號曰慈明。辛酉,請大行皇帝謚號於南郊,謚曰仁文哲武恭孝皇帝,廟號曰寧宗。



 寶慶元年春正月壬戌朔,詔舉賢良。庚午,湖州盜潘壬、
 潘丙、潘甫謀立濟王竑,竑聞變,匿水竇中,盜得之,擁至州治,以黃袍加其身,守臣謝周卿率官屬入賀。初,壬等偽稱李全以精兵二十萬助討史彌遠擅廢立之罪,比明視之,皆太湖漁人及巡尉兵卒,竑乃遣王元春告於朝,而率州兵誅賊。彌遠奏遣殿司將彭任討之,至則盜平。又遣其客秦天錫托宣醫治竑疾,諭旨逼竑死,尋詔貶為巴陵郡公。辛未,詔保寧軍節度使師彌為檢校少保。詔以皇太后弟奉國軍節度使楊谷、保寧節度楊石
 並開府儀同三司。丙戌,濟王竑訃聞,特輟視朝。己丑,上寧宗謚冊、寶。



 二月甲午,詔故太師、武勝定國軍節度使、鄂王岳飛謚忠武。丙申,詔師彌檢校少師、嗣秀王。丙辰,楚州火。戊午,發廩振在京細民,給犒馬步軍、皇城司守衛軍有差。



 三月癸酉,葬寧宗於會稽永茂陵。



 夏四月辛卯朔,寧宗祔廟。壬辰,詔皇兄竑贈少師、保靜、鎮潼軍節度使,直舍人院王塈等繳奏命,遂寢。丁酉,皇太后手書:「多病,自今免垂簾聽政。」壬寅,帝兩請皇太后垂簾,不允。
 辛亥,發廩振在京細民。



 五月甲子,詔:「內外文武大小之臣,於國政有所見聞,封章來上,毋或有隱。」丙寅,詔不熄為保康軍承宣使、嗣濮王。



 六月辛卯,太白晝見。丁未,詔史彌遠為太師,依前右丞相兼樞密使,進封魏國公。彌遠辭免太師。



 秋七月丁丑,滁州大水,詔振恤之。乙酉,詔行大宋元寶錢。


八月壬寅,以司農丞姚子才封事切直,詔進一秩,授秘書郎。癸卯,詔知袁州趙
 \gezhu{
  □政}
 夫直秘閣、福建提點刑獄,以旌廉吏。丙午,詔侍從、給諫、卿監、郎官,並
 在外前執政、侍從、帥臣、監司,各舉廉吏三人。戊申,詔侍從、兩省、臺諫、三衙、知閣、御帶、環衛官,在外前執政、侍從,帥臣、監司、都副都統制及屯戍主將,其各舉堪充將帥三人。己酉,地震。壬子,張九成贈太師,追封崇國公,謚文忠。甲寅,以程頤四世孫源為籍田令。乙卯,莫澤言真德秀舛論綱常,簡節上語,曲為濟王地。詔德秀煥章閣待制、提舉玉隆萬壽宮。丁巳,詔戒貪吏。



 九月丙寅,著作佐郎陶崇上保業、慎獨、謹微、持久四事,帝嘉納之。



 冬十
 月癸巳,有流星大如太白。甲寅,詔會稽攢宮所在,稅賦盡免折科,山陰縣權免三年。十一月癸亥,宣繒兼同知樞密院事,薛極參知政事,葛洪簽書樞密院事。詔邵州潛藩,可升為寶慶府。筠州與御名音相近,改為瑞州。壬午,雪寒,在京諸軍給緡錢有差,出戍之家倍之。自是祥慶、災異、霪雨、雪寒咸給。甲申,朱端常言魏了翁封章謗訕,真德秀奏札誣詆。詔魏了翁落職,奪三秩、靖州居住;真德秀落職罷祠。



 十二月甲辰,詔刪修敕令。是歲,兩浙
 路戶一百九十七萬五千九百九十六,口二百八十二萬二千三十二。福建路戶一百七十萬四千一百八十六,口二百五十五萬三千七十九。



 二年春正月癸亥,詔贈沉煥、陸九齡官,煥謚端憲,九齡謚文達。錄張九成、呂祖謙、張栻、陸九淵子孫官各有差。癸酉,詔布衣李心傳赴闕。戊寅,熒惑入氐。壬午,太白、歲星、填星合於女。



 二月辛卯,臨察御史梁成大言真德秀有大惡五,僅褫職罷祠,罰輕。詔削二秩。



 三月癸酉,以久
 雨,詔大理寺、三衙、兩浙運司、臨安府諸屬縣榷酒所,凡贓賞等錢,罪已決者,一切勿征,毋錮留妻子。自是霖潦、寒暑皆免。戊寅,詔太常寺建功臣閣,以「昭勛崇德」為名。己卯,蘄州火。



 夏四月己丑,詔輔臣奉薄,其以《隆興格》為制。辛亥,有流星大如太白。



 六月丙申,御後殿,賜進士王會龍以下九百八十九人及第、出身有差。壬寅,詔以孔子五十二代孫萬春襲封衍聖公。



 秋七月戊辰,雷電、雨,晝晦,大風。遂安、休寧兩縣界山裂,洪水壞公宇、民居、田
 疇。八月乙巳,濟王竑追降巴陵縣公。辛亥,衛涇薨。



 九月庚申,雷。



 冬十月甲申,詔《寧宗御集》閣以「寶章」為名,仍置學士、待制員。辛丑,又雷。辛亥,熒惑、歲星、填星合於女,熒惑犯填星。改湖州為安吉州。



 十一月甲寅,修祚德廟,以嚴程嬰、公孫杵臼之祀。丙辰,始御紫宸殿。辛酉,熒惑犯歲星。丙子,日南至,上詣慈明殿。十二月癸卯,親享太廟。



 三年春正月辛亥朔,上壽明皇太后尊號冊、寶於慈明殿。壬子,史彌遠二秩。辛酉,以楊谷、楊石並為少傅。知
 楚州姚翀朝辭,奏淮楚忠義軍事,上曰:「南北皆吾赤子,何分彼此,卿其為朕撫定之。」己巳,詔:「朕觀朱熹集注《大學》、《論語》、《孟子》、《中庸》,發揮聖賢蘊奧,有補治道,朕勵志講學,緬懷典刑,可特贈熹太師,追封信國公。」三月庚戌朔,詔郡縣長吏勸農桑,抑末作,戒苛擾。工部侍郎朱在進對,奏人主學問之要,上曰:「先卿《中庸序》言之甚詳,朕讀之不釋手,恨不與同時。」辛亥,以皇太后尊號冊、寶禮成,侄孫楊鳳孫以下推恩有差。



 夏四月戊戌,宣引前丞相
 謝深甫孫女謝氏詣慈明殿進見。



 五月壬子,詔岳珂戶部侍郎,依前淮東總領兼制置使。



 閏月己卯朔,詔:郡縣系囚不實書歷,未經結錄,守臣輒行特判,憲司其詳覆所部獄案,歲月淹延者重置於憲。



 六月戊申朔,日有食之。



 秋七月乙酉,太陰犯心。丁酉,詔振贍被水郡縣,其竹木等稅勿征。丙午,史彌遠乞歸田里,詔不允。



 八月庚戌,詔謝氏特封通義郡夫人。癸亥,詔凡試邑兩經罷黜,更勿授知縣、縣令。甲戌,太白、熒惑合於翼。丙子,城太平州,
 詔知州綦奎進中奉大夫,餘推恩有差。



 九月癸未,故觀文殿大學士、魏國公、贈太師留正謚忠宣。丙午,追上寧宗徽號曰法天備道純德茂功仁文哲武聖睿恭孝皇帝。



 冬十月甲子,右監門衛大將軍與奭改賜名貴謙,授宜州觀察使,繼沂王後。右千牛衛將軍孟杓改賜名乃裕,授和州防禦使,繼景獻太子後。甲戌,趙範江東提刑兼知池州,節制防江水步軍、池州都統司軍馬。



 十一月戊寅,奉上寧宗徽號冊寶於太廟。辛巳,日南至,郊,大赦。
 改明年為紹定元年。十二月己酉,日旁有氣如珥。壬申,發廩振贍京城細民。大元兵破關外諸隘,四川制置鄭損棄三關。



 紹定元年春正月丙子朔,上壽明慈睿皇太后尊號冊寶於慈明殿。楊谷、楊石並升少師。



 六月壬寅朔,日有食之。己酉,流星晝隕。



 秋七月戊戌,熒惑犯南斗。



 冬十月戊申,熒惑犯壁壘陣星。丁巳,熒惑、填星合於危。甲子,熒惑犯填星。



 十一月癸酉,熒惑入羽林。庚辰,雷。丁酉,詔申嚴
 皇城司給符之制,照闌入法。十二月辛亥,以薛極知樞密院事兼參知政事,葛洪參知政事,袁韶同知樞密院事,鄭清之端明殿學士、簽書樞密院事。



 二年春正月庚辰,大理司直張衍上檢驗、推鞠四事。詔刑獄人命所關,其令有司究行之。丁亥,熒惑、歲星合於婁。



 二月庚戌,詔歲舉廉吏或犯奸贓,保任同坐,監司、守臣其申嚴覺察。



 三月辛卯,詔郡縣系囚多瘐死獄中,憲司其具獄官姓名以聞,黜罷之。



 夏四月庚申,詔郡縣官
 闕,毋令藝術人、豪民、罷吏借補權攝。



 五月,詔成都、潼川路歲旱民歉,制司、監司其亟振恤,仍察郡縣奉令勤惰以聞。辛巳,賜進士黃樸以下五百五十七人及第、出身有差。詔戶絕者許立嗣,毋妄籍沒。



 六月丁巳,詔通義郡夫人謝氏進封美人。



 九月丁卯,臺州大水。壬辰,有流星大如太白。



 冬十月壬戌,詔臺州水災,除民田租及茶、鹽、酒酤諸雜稅,郡縣抑納者監司察之。



 十一月己丑,熒惑入氐。



 三年春正月甲申,詔故皇子緝贈保信、奉國軍節度使,開府儀同三司,追封永王,謚沖安。壬辰,知棗陽軍史嵩之創置屯田,以勞賞官兩轉。



 二月丙申,日有背氣。戊戌,詔汀、贛、吉、建昌蠻獠竊發,經擾郡縣復賦稅一年。庚戌,詔趙範起復,依前知鎮江府、節制防江水步並本州在砦軍馬;趙葵起復,依前知滁州、節制本州屯戍軍馬。壬子,詔故皇子繹賜忠正、保寧軍節度使、開府儀同三司,追封昭王,謚沖純。



 閏月癸酉,逃卒穆椿夜竊入皇城,燒
 毀甲仗,衛士捕得之,詔磔於市。乙酉,太白、歲星合於畢。



 三月丁酉,雨土。戊申,奉國軍節度使不心冬薨,贈少傅,追封樂平郡王。



 夏四月己卯,漳州、連城盜起,知龍巖縣莊夢詵、尉鐘自強不能效死守土,詔各削二秩罷。



 五月甲寅,檢校少保李全授彰、化保康軍節度使,開府儀同三司、京東鎮撫使,依舊京東忠義諸軍都統制。戊午,李全左右金吾衛上將軍,職任仍舊。



 六月乙酉,歲星入井。



 秋七月丁酉,汀州寧化縣曾氏寡婦晏給軍糧御漳寇
 有功,又全活鄉民數萬人,詔封恭人,賜冠帔,官其子承信郎。



 九月辛丑,祀明堂,大赦。丙午,美人謝氏進封貴妃。冬十月己巳,熒惑、填星合於室。



 十一月丁酉,有星孛於天市垣。丁未,流星晝隕。十二月庚申,詔錄用孔子四十九代孫燦補官。李全叛。壬戌,淮東官兵王青力戰,死之,贈右武大夫、蘄州防禦使。甲子,詔:「逆賊李全,反形日著,今乃肆為不道,已敕江、淮制臣率兵進討,有能擒斬全以降者,加以不次之賞。」乙丑,詔免明年元會禮。以鄭清
 之參知政事兼簽書樞密院事,喬行簡端明殿學士、同簽書樞密院事。詔:「史彌遠敷奏精敏,氣體向安,朕未欲勞以朝謁,可十日一赴都堂治事。」丁卯,冊命貴妃謝氏為皇后。己卯,慈明殿出緡錢百五十萬犒諸軍,振贍在京細民。癸未,上壽明仁福慈睿皇太后尊號冊寶。



 四年春正月戊子,皇太后年七十有五,上詣慈明殿行慶壽禮,大赦,史彌遠以下進秩有差。賜李心傳同進士出身。壬寅,趙範、趙葵等誅李全於新塘,詔各進兩秩,餘
 推恩有差。



 二月戊午朔,詔:雄邊軍統制、總轄範勝、穀汝礪等誅逆著勞,各官五轉,將士立功者,趣具等第、姓名來上。丙子,詔起復孟珙從義郎、京西路分,棗陽軍駐扎。



 夏四月戊辰,趙範、趙葵並進中大夫、右文殿修撰,賜紫章服、金帶。丁丑,以鄭清之兼同知樞密院事;喬行簡簽書樞密院事;趙善湘兵部尚書、江淮制置大使、知建康府,依舊安撫使;趙範權兵部侍郎、淮東安撫副使、知揚州兼江淮制司參謀官;趙葵換福州觀察使、右驍衛
 大將軍、淮東提刑、知滁州兼大使司參議官。



 五月丙午,宗室司正檢校少傅、安德軍節度使、天水郡公,加食邑五百戶;貴謙承宣使;乃裕觀察使。



 六月己未,詔魏了翁、真德秀、尤煜、尤□龠並敘復元官職祠祿。



 七月己丑,日生承氣。丁酉,賈涉女侍後宮,詔封文安郡夫人。庚戌,葛洪資政殿學士、知紹興府。有流星大如太白。



 八月己未,大元兵破武休,入興元,攻仙人關。辛酉,洪咨夔敘復元官祠祿。辛未,文安郡夫人賈氏封才人。



 九月丙戌夜,臨安
 火,延及太廟,統制徐儀、統領馬振遠坐救焚不力,貶削有差。上素服視朝,減膳徹樂。庚子,建昌軍火。甲辰,流星晝隕。



 冬十月戊午,太常少卿度正、國史院編修官李心傳各疏言:宗廟之制,未合於古,茲緣災異,宜舉行之。詔兩省、侍從、臺諫集議以聞。甲子,以餘天錫為戶部侍郎兼知臨安府、浙西安撫使。癸酉,大元兵破蜀口諸郡,御前中軍統制張宣戰青野原有功,詔授沔州都統。戊寅,以李□為煥章閣直學士、四川制置使、知成都府,趙彥
 吶直龍圖閣、四川安撫制置副使、知興元府、利路安撫使,安癸仲戶部郎中、總領四川財賦。



 十一月乙酉,詔忠義總管田遂力戰而歿,贈武節大夫、忠州刺史,加封立廟。十二月乙亥,以史嵩之為大理少卿兼京湖制置副使。



 五年春正月己丑,以孟珙為京西路兵馬鈐轄、棗陽軍駐扎。庚寅,詔:李全之叛,淮東提刑司檢法吳澄等出泰州城謁賊,各追官勒停。其不出見賊者高夢月、劉賓云
 循升二資。罵賊而死者海陵簿吳哲,特贈朝奉郎,官其一子將仕郎。」壬辰,史嵩之進大理卿、權刑部侍郎、京湖安撫制置使、知襄陽府。壬寅,新作太廟成。



 二月癸丑,帝謁太廟。



 三月乙酉,詔京城內外免征商三月。丁酉,日後抱氣、承氣。



 夏四月癸亥,以寶章閣直學士桂如淵頃帥蜀日,北兵攻城,不能合謀死守而遁,致軍民罹殃,反以捷聞,詔褫職罷祠。丁卯,起魏了翁以集英殿修撰知遂寧府。



 五月己丑,詔:「昨鬱攸為災,延及太室,罪在朕躬,而
 二三執政,引咎去職。今宗廟崇成,神御妥安,薛極、鄭清之、喬行簡並復元官。」辛卯,臣僚言:「積陰霖霪,歷夏徂秋,疑必有致咎之徵。比聞蘄州進士馮傑,本儒家,都大坑冶司抑為爐戶,誅求日增,傑妻以憂死,其女繼之,弟大聲因赴訴,死於道路,傑知不免,毒其二子一妾,舉火自經而死。民冤至此,豈不上乾陰陽之和?」詔都大坑冶魏峴罷職。癸巳,太白經天,晝見。戊戌,詔今後齊民有罪,監司、守臣毋輒籍沒其家,必具聞俟命。



 六月乙丑,熒惑、填
 星合於婁,熒惑順行犯填星。丙子,詔諸獄官不理他務。



 秋七月甲申,詔:「近歲北兵再入利、閬,迫近順慶,承奉郎胡元琰攝郡事,能收散卒,定居民,諭叛將,以全闔郡,以功特轉官三資。」太白入井。丙戌,監楚州大軍倉富起宗軍變死難,詔贈宣教郎,官一子文林郎。張煥同時被創,害及其家,詔轉官一資。丁酉,以吳潛為太府少卿、總領淮西財賦,陳貴誼端明殿學士、同簽書樞密院事。



 八月乙卯,起真德秀為徽猷閣待制、知泉州。丁巳,泗州路分
 劉虎、副都統董琳焚斷盱泗橋遏金兵。己未,魏了翁以寶章閣待制、潼川安撫使知瀘州。乙丑,賜進士徐元傑等四百九十三人及第、出身有差。壬申,太白、歲星合於張。甲戌,新作玉牒殿,奉安累朝玉牒。



 九月乙巳,雨雹,雷。



 閏月己酉,有流星大如太白。庚戌,彗星出於角。戊辰,史彌遠乞歸田里,詔不允。



 冬十月戊子,以星變,大赦。金將以盱眙軍來降,赦盱眙,改為招信軍。



 十一月己巳,喬行簡累疏乞歸田,詔不允。十二月丙子朔,進封才人賈氏
 為貴妃。辛巳,皇太后不豫。壬午,大赦。皇太后崩。癸卯,群臣凡七表請聽政,從之。詔:外朝大典,不敢輕改,宮中自服三年喪。時宋與大元兵合圍汴京,金主奔歸德府,尋奔蔡州,大元再遣使議攻金,史嵩之以鄒伸之報謝。



 六年春正月己酉,以少傅、保寧軍節度使、嗣秀王師彌判大宗正事,趙善湘光祿大夫、江淮制置大使兼知建康府、行宮留守,加食邑四百戶。戊辰,史彌遠加食邑千戶。



 二月丁丑,上大行皇太后謚曰恭聖仁烈皇后。以趙
 範為工部侍郎兼中書門下省檢正公事,趙葵秘書監兼侍講,餘天錫禮部侍郎兼侍讀。癸卯,熒惑犯東井。



 三月丙辰,大雨、雹。



 夏四月壬寅,葬恭聖烈皇后於永茂陵。



 五月庚戌,太白、熒惑合於柳。鄧州移剌以城來降。



 六月丁酉,史嵩之刑部侍郎兼京湖安撫制置使兼知襄陽府。



 秋七月,敗武仙於浙江。



 八月,拔唐州。



 九月壬寅朔,日有食之。辛亥,祀明堂,大赦。辛酉,經筵官請以禦制敬天、法祖、事親、齊家四十八條及緝熙殿榜、《殿記》宣付史
 館。



 冬十月,江海領襄軍從大元兵合圍金主於蔡州。甲申,史宅之太府少卿,史宇之將作少監,並賜同進士出身。丙戌,史彌遠進太師、左丞相兼樞密使、魯國公,加食邑一千戶;鄭清之光祿大夫、右丞相兼樞密使,加食邑一千戶。丁亥,史彌遠保寧、昭信軍節度使,充醴泉觀使,進封會稽郡王,仍奉朝請,加食邑封。以薛極為樞密使,喬行簡參知政事兼同知樞密院事,陳貴誼參知政事兼簽書樞密院事。詔:「史彌遠有定策大功,勤勞王室,今
 以疾解政,宜加優禮。長子宅之權戶部侍郎兼崇政殿說書,次子宇之直華文閣、樞密院副都承旨,長孫同卿直寶章閣,次孫紹卿、良卿、會卿、晉卿並承事郎,女夫趙汝禖軍器少監,孫女夫趙崇梓官一轉。」己丑,詔崔與之、李□、鄭性之赴闕。庚寅,以顯謨閣待制、知福州真德秀兼福建安撫使。乙未,史彌遠薨,贈中書令,追封衛王,謚忠獻。詔戒貪吏。



 十一月乙巳,給事中莫澤等言,差提舉千秋鴻禧觀梁成大暴狠貪婪,茍賤無恥,詔奪成大祠
 祿。丙午,詔改明年為端平元年。己未,以魏了翁為華文閣待制、知瀘州、潼川安撫使,賜金帶。癸亥,進趙葵兵部侍郎、淮東制置使兼知揚州。甲子,臺臣劾刑部尚書莫澤貪淫忮害,罷之。丙寅,權工部尚書趙範言:「宣和海上之盟,厥初甚美,迄以取禍,其事不可不鑒。」帝嘉納之。丁卯,詔趙葵任責防禦。戊辰,禮部郎中洪咨夔進對:今日急務,進君子,退小人,如真德秀、魏了翁當聚之於朝。帝是其言,命咨夔洎王遂同為監察御史。己巳,趙葵入見,
 帝問以金事,對曰:「今國家兵力未贍,姑從和議。俟根本既壯,雪二帝之恥,以復中原。」十二月戊寅,史宅之繳納賜第,詔給賜本家,仍奉家廟。庚辰,以薛極為觀文殿大學士、知紹興府兼浙東安撫使。甲申,吳潛太府卿,仍淮西總領財賦,暫兼沿江制置、知建康府。戊申,洪咨夔言:「資政殿學士、提舉洞霄宮袁韶,仇視善類,諂附彌遠,險忮傾危。」詔袁韶奪職、罷祠祿。壬辰,臺臣言:「趙善湘、陳賅、鄭損納賂彌遠,怙勢肆奸,失江淮、荊襄、蜀漢人心,罪狀
 顯著。」詔趙善湘有討李全功,特寢免;陳賅與祠,鄭損落職與祠。



 端平元年春正月庚子朔,詔求直言。侍從、卿監、郎官,在外執政、從官,舉堪為監司、守令者各二人。三衙、統帥、知閣、御帶、環衛官,在外總管、軍帥,舉堪為將帥者各二人。鐘震、陳公益、李性傳、張虙並兼侍讀。徐清叟、黃樸、李大同、葉味道並兼崇政殿說書。辛丑,趙範依前沿江制置副使,權移司知黃州,史嵩之權京湖安撫制置使兼知
 襄陽府,陳韡華文閣待制,仍知隆興府、江西安撫使。詔德安三關使彭哲,去年十月北兵至,棄關遁,削二秩勒停。乙巳,賜故少傅、權參知政事任希夷謚宣憲。丙午,詔趙範兼淮西制置副使,任責防禦。太白、熒惑合在斗。戊申,金主完顏守緒傳位於宗室承麟。己酉,城破,守緒自經死,承麟為亂兵所殺,執其參知政事張天綱。丙寅,詔:「太師、中書令榮王已進王爵,宜封三代,曾祖子奭贈太師、吳國公,祖伯旴贈太師、益國公,父師意贈太師、越國
 公。」戊辰,以樞密院言,詔:「京西忠順統制江海、棗陽同統制郭勝,向因所部兵行劫,坐不發覺,除名、廣州拘管。遇赦還軍前自效有功,並敘復元受軍職。」史嵩之露布告金亡,謹遣郭春按循故壤,詣奉先縣汛掃祖宗諸陵。還師屯信陽。命王旻守隨州,王安國守棗陽,蔣成守光化,楊恢守均,並益兵飭備,經理唐、鄧屯田。



 二月辛未,監察御史洪咨夔言:「上親政之始,斥逐李知孝、梁成大,其諂事權奸,黨私罔上,倡淫黷貨,罪大罰輕。」詔李知孝削一
 秩,罷祠;梁成大削兩秩。壬申,以趙彥吶為四川安撫制置使兼知興元府。丁亥,詔端平元年正月以前諸命官貶竄物故者,許令歸葬。



 三月己酉,以賈涉子似道為籍田令。辛酉,詔遣太常寺主簿朱揚祖、閣門祗候林拓詣洛陽省謁八陵。



 四月辛未,詔遣朱復之詣八陵,相度修奉。丁丑,詔:「比年宗親貧窶,或致失所。甚非國家睦族之意。大宗正司、南外西外宗正司,其申嚴州郡,以時贍給,違者有刑。」監察御史王遂言:「史嵩之本不知兵,矜功自
 侈,謀身詭秘,欺君誤國,留之襄陽一日,則有一日之憂。」不報。戊寅,歲星守太微垣上相星。壬午,監察御史洪咨夔言:「今殘金雖滅,鄰國方強,益嚴守備猶恐不逮,豈可動色相賀,渙然解體,以重方來之憂?」上嘉納。甲申,日生赤暈。丙戌,以滅金獲其主完顏守緒遺骨告太廟,其玉寶、法物並俘囚張天綱、完顏好海等命有司審實以聞。庚寅,詔授孟珙帶御器械,京、襄部押官屬陳一薦、江海官兩轉,餘論功行賞。金降人夾谷奴婢改姓同名鼎,王
 聞顯、呼延實、來伯友、石大瑞、白華各授官有差。丁酉,臣僚言:「江淮、荊襄諸路都大提點坑冶吳淵,恃才貪虐,籍人家貲以數百萬計,掩為己有,其弟潛違道乾譽,任用非類。」詔吳淵落右文殿修撰,吳潛落秘閣修撰,並放罷。



 五月庚子,薛極卒,贈少師。戊申,太平州螟。己酉,太陰入氐。乙卯,詔李知孝瑞州居住,梁成大潮州居住,莫澤南康軍居住,並再降授官,尋盡追爵秩。詔魏了翁赴闕。丙辰,以趙範為兩淮制置使、節制軍馬兼沿江制置副使。壬
 戌,以崔與之為端明殿學士、提舉西京嵩山崇福宮,陳韡權工部尚書、知隆興府、江西安撫使。丙寅,詔:「黃乾、李燔、李道傳、陳宓、樓昉、徐宣、胡夢昱皆厄於權奸,而各行其志,沒齒無怨,其賜謚、復官、優贈、存恤,仍各錄用其子,以旌忠義。戴野,其復元資,以勵士風。」建陽縣盜發,眾數千人,焚劫邵武、麻沙、長平。



 六月戊辰朔,鄭清之等進奏選德殿柱有金書六字曰:「毋不敬,思無邪。」上曰:「此坐右銘也。」庚午,熒惑、填星合於胃。壬申,詔蠲漳、泉、興化三州
 丁米錢。丙子,以李鳴復為侍御史兼侍講。戊寅,以喬行簡知樞密院事,曾從龍參知政事,鄭性之簽書樞密院事,陳貴誼兼同知樞密院事。己卯,詔:「故巴陵縣公竑可盡復本身官爵,有司其檢視墓域,以時致祭。妻吳昨自請為尼,特賜慧凈法空大師,紹興府月給衣資緡錢。」詔殿司選精銳千人,命統制婁拱、統領楊辛討捕建陽縣盜。幸巳,詔故端明殿學士、開府儀同三司史彌遠贈資政殿大學士,謚忠宣。熒惑犯填星。丙戌,有流星大如太
 白。戊子,日暈不匝,生格氣。癸巳,史嵩之進兵部尚書。禁毀銅錢作器用並貿易下海。



 秋七月乙巳,詔嘉興縣王臨年百二歲,補迪功郎致仕。



 八月癸酉,詔:「河南新復郡縣,久廢播種,民甚艱食,江、淮制司其發米麥百萬石往濟歸附軍民,仍榜諭開封、應天、河南三京。」甲戌,朱揚祖、林拓朝謁八陵回,以圖進,上問諸陵相去幾何及陵前澗水新復,揚祖悉以對,上忍涕太息。乙亥,以趙範為京河關陜宣撫使、知開封府、東京留守,趙葵京河制置使、
 知應天府、南京留守,全子才關陜制置使、知河南府、西京留守。甲午,權邵武軍王野以平建陽寇有功,官兩轉,餘推賞有差。



 九月庚子,趙範依舊京西、湖北安撫制置大使、知襄陽府。辛丑,熒惑入井。壬寅,趙範言:「趙葵、全子才輕遣偏師復西京,趙楷、劉子澄參贊失計,師退無律,致後陣敗覆。」詔趙葵削一秩,措置河南、京東營田邊備;全子才削一秩,措置唐、鄧、息營田邊備;劉子澄、趙楷並削三秩放罷。又言:「楊義一軍之敗,皆由徐敏子、範用吉
 怠於赴援,致不能支。」詔範用吉降武翼郎,徐敏子削三秩放罷,楊義削四秩,勒停自效。己酉,真德秀言:權臣罔上,講筵官亦傅會其言,今承其弊,有當慮者五事,並及泉、漳寇盜、鹽法之弊。帝嘉納之。詔:進士何霆編類朱熹解注文字,有補經筵,授上文學。



 冬十月己卯,真德秀進《大學衍義》。辛卯,陳貴誼薨,贈少保。



 十一月壬子,京、湖制司創鎮北軍,詔以襄陽府駐扎御前忠衛軍為名。壬戌,太白經天。十二月己卯,大元遣王楫來。戊子,王楫辭於
 後殿。辛卯,遣鄒伸之、李復禮、喬仕安、劉溥報謝,各進二秩。



\end{pinyinscope}