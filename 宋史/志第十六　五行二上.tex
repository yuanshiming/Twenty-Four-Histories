\article{志第十六 五行二上}

\begin{pinyinscope}

 火上



 炎
 上,火之性也。火失其性,則為災眚。舊說以恆燠、草妖、羽蟲之孽,羊禍、赤眚、赤祥之類皆屬之火,今從之。



 建隆元年,宿州火,燔民舍萬餘區。二年三月,內酒坊火,燔舍
 百八十區,酒工死者三十餘。三年正月,滑州甲仗庫火。燔儀門及軍資庫一百九十區,兵器、錢帛並盡。開封府通許鎮民家火。燔廬舍三百四十餘區。二月,安州牙吏施延業家火。燔民舍並顯義軍營六百餘區。五月,京師相國寺火,燔舍數百區。海州火,燔數百家,死者十八人。



 乾德四年二月,岳州衙署、稟庫火,燔市肆、民舍殆盡,官吏逾城僅免。三月,陳州火,燔民舍數十區。潭州火,燔民舍五百餘區;逾月,民周澤家火,又燔倉稟、民舍數百區,
 死者三十六人。是春,諸州言火者甚眾。八月,衡州火,燔公署、倉庫、民舍僅千餘區。五年,京師建隆觀火。



 開寶三年八月,辰州廨火,燔軍資庫。五年七月,忠州火,倉庫殆盡。七年九月,永城縣火,燔民舍一千八百餘區。八年四月,洋州火,燔州廨、民舍千七百區。永城縣火,燔軍營、民舍千九百八十區,死者九人。



 太平興國七年八月,益州西倉災。



 雍熙元年五月丁丑,乾元、文明二殿災。初夕,陰雲雷震,火起月華門,翌日辰、巳方止。二年九月庚寅夜,
 楚王元佐宮火,燔舍數百區,王自是以疾廢於家。三年,光化軍民卻勛家火,延燔軍廨、舍、庫。



 端拱元年二月,雲安軍威棹營火。二年三月,衡州火,燔州縣官舍、倉庫、軍營三百餘區。又崇賢坊有鳥燔數十處,七日不滅。



 淳化三年十月,蔡州懷慶軍營火,燔汝河橋民居、官舍三千餘區,死者數人。十二月,建安軍城西火,燔民舍、官廨等殆盡。四年二月,永州保安津舍火,飛焰過江,燒州門及民屋三百餘家。



 咸平二年四月,池州倉火,燔米八萬七
 千斛。



 景德元年正月,平虜軍營火,焚民居廬舍甚眾。四年十一月,郢州火,燔倉庫並盡。



 大中祥符元年正月,桂州甲仗庫災。二年四月,升州火,燔軍營、民舍殆盡。四年八月,徐州草場火。十月,鎮州城樓、戰棚火。七月,雄州甲仗庫火。八年二月甲寅,宗正寺火。四月壬申夜,榮王元儼宮火,自三鼓北風甚,癸酉亭午乃止,延燔左承天祥符門、內藏庫、朝元殿、乾元門、崇文院、秘閣、天書法物內香藏庫。九年五月甲子,左天廄坊草場火。



 天禧二年
 二月戊寅,北宅蔡州團練使德雍院火,延燔數百區。三年春,京師多火。六月,永州軍營火,延民舍數百餘區。五年四月丁巳,事材場火。



 天聖三年二月丁卯,蘄州榷貨務火。五年四月壬辰,壽寧觀火。七年六月丁未,玉清昭應宮災。初,大中祥符元年,詔建宮以藏天書。七年,宮始成,凡二千六百一十楹。至是,火發夜中,大雷雨,至曉而盡。



 明道元年八月壬戌,修文德殿成。是夜,禁中火,延燔崇德、長春、滋福、會慶、延慶、崇徽、天和、承明八殿。



 景祐三年七月庚
 子,太平興國寺火起閣中,延燔開先殿及寺舍數百楹。是夕,大雨雹。十月巳酉,澶州橫龍水口西岸料物場火,焚薪芻一百九十餘萬。



 寶元二年六月丁丑,益州火,焚民廬舍二千餘區。



 康定元年六月乙未,南京鴻慶宮神御殿火。



 慶歷元年五月癸亥,慶州草場火,延燔州城樓櫓。三年十一月丙寅,上清宮火。四年三月丙戌夜,代州五臺山寺火。六月丁未,開寶寺靈感塔災。七月甲子,燕王宮火。六年七月辛丑,洪福禪院火。八年正月壬午,江
 寧府火。初,李景江南大建宮室、府寺,其制多仿帝室,至是一夕而焚,唯玉燭殿獨存。



 皇祐五年正月丁巳,會靈觀火。



 至和元年四月辛丑,祥元觀火。二年,並州太宗神御殿火。



 嘉祐三年正月,溫州火,燔屋萬四千間,死者五十人。



 治平四年十二月壬子夜,睦親宮火,焚九百餘間。甲寅,廣親宮又火。



 熙寧六年二月丙申,永昌陵上宮火,燔東城門。七年九月壬子,三司火,自巳至戌,焚屋千八百楹,案牘殆盡。十一月,洞真宮火。九年十月,魯王、濮王宮
 火。十年正月,仙韶院火。撤屋二百五十楹。三月丙子,開封府火。



 元豐元年八月,邕州火,焚官舍千三百四十六區,諸軍衣萬餘襲,穀帛軍器百五十萬。四年六月,衡州火,燒官舍、民居七千二百楹。欽州大雷震,火焚城屋。五年二月,洞真宮火。八年二月辛巳,開寶寺火。時寓禮部貢院於寺,點校試卷官翟曼、陳之方、馬希孟焚死,吏卒死者十四人。



 元祐元年三月,宗室宮院火。六年十二月,開封府火。



 紹聖三年三月七日,內尚書省火,尋撲滅。
 上逾執政:禁中屢火,方醮禳,已罷春宴,仍不御垂拱殿三日。四年七月甲子,禁中火。



 元符元年四月,宗室宮院火。



 建中靖國元年六月壬寅,集禧觀火,大雨中久而後滅。



 崇寧二年六月,中太乙宮火。三年三月辛丑,大內火。



 政和三年四月,蘇州火,延燒公私屋一百七十餘間。五月,封州火,延燒公私屋六百八十二間。五月辛丑,京師大盈倉火。是歲,成都府大慈寺、溫州絳州皆火。



 重和元年九月,掖庭大火,自甲夜達曉,大雨如傾,火益熾,凡爇五
 千餘間,後苑廣聖宮及宮人所居幾盡,焚死者甚眾。



 靖康元年十二月丙子夜,尚書省火,延燒禮、祠、工、刑、吏部,拆尚書省牌擲火中禳之乃息。二年三月戊戌,天漢橋火,焚百餘家。頃之,都亭驛又火。己酉,保康門火。



 紹興元年十月乙酉,臨安府、越州大火,民多露處。十二月辛未,越州火,焚吏部文書,乙酉,移蹕錢塘。二年正月丁巳,宣州火,燔民居幾半。五月庚辰,臨安府大火,亙六七里,燔萬數千家。十二月甲午,行都大火。燔吏刑工部、御史臺、官
 府、民居、軍壘盡,乙未旦乃熄。三年九月庚申,行都闕門外火,多燔民居。四年正月戊寅,行都火,燔數千家。六年二月,行都屢火,燔千餘家。十二月,行都大火,燔萬餘家,人有死者。時高宗親征劉豫,都民之暴露者多凍死。七年正月辛未,平江府火。二月辛丑,鎮江府、楚、真、揚、太平州火。是歲,臨安府火。八年二月丁酉,太平府大火,宣撫司及官舍、民居、帑藏、文書皆盡,死者甚眾,錄事參軍呂應中、當塗縣丞李致虛死焉。九年二月己卯,行都火。七
 月壬寅,又火。十年十月,行都火,燔民居,延及省部。十一月丁巳,溫州大火,燔州學、酤征舶等務、永嘉縣治及民居千餘。十一年七月癸亥,婺州大火,燔州獄、倉場、寺觀暨民居幾半。九月甲寅,建康府火,燔府治三十餘區,民居三千餘家。十二年二月辛巳,鎮江府火,燔倉米數萬石,芻六萬束,民居尤眾。是月,太平、池州及蕪湖縣皆火。三月丙申,行都火。四月,行都又火。十四年正月甲子,行都火。十五年,大寧監火,燔官舍、帑藏、文書。九月丙子,行
 都火,經夕,漸近太室而滅。十七年八月,建康府火。十二月辛亥,靜江府火,燔民舍甚眾。二十年正月壬午,行都火,燔吏部文書皆盡。二十五年,汴京宮室悉焚。二十六年。潭州南嶽廟火。二十九年四月,鎮江府火,焚軍壘、民居。十二月丙子,夔州大火,燔官舍、民居、寺觀,人有死者。



 乾道元年正月,泰州火,燔民舍幾盡。是年春,德安府應城縣廄驛火。二年冬,真州六合縣武鋒軍壘火。十二月,婺州火。自是火患不息,人火之也。三年五月,泉州火。五
 年十二月壬申,太室東北垣外民舍火。七年十一月丁亥,禁垣外閹人私舍火,延及民居。九年九月,臺州火,經夕,至於翌日晝漏半,燔州獄、縣治、酒務及居民七千餘家。



 淳熙元年十二月丁巳,泉州火,燔城樓及五十餘家。二年六月戊午,潭州南嶽廟火。八月,嚴州火。十一月癸亥,麗正門內東廡災。是歲,瀘州火,坐上焚民居不實,守臣貶秩。三年九月,大內射殿災,延及東宮門。四年十一月辛酉,鄂州南市火,暴風通夕,燔千餘家。五年四月庚
 寅,興州沙市火,燔三百四十餘家,有死者。十一月,和州牧營火,燔一百六十區。七年二月,江陵府沙市大火,燔數千家,延及船艦,死者甚眾。八月,溫州試士,火作於貢闈。八年正月,揚州火。九月乙亥,行都火。九年九月,合州大火。燔民居幾盡,官舍僅有存者。十一年二月辛酉,興元府義勝軍壘舍火。十二年八月,溫州火,燔城樓及四百餘家。十月,鄂州大火,燔萬餘家。江風暴作,結廬堤上、泊舟岸下者,焚溺無遺。十四年五月,大內武庫災,戎器
 不害。六月庚寅,行都寶蓮山民居火,延燒七百餘家,救焚將校有死者。五月,成都府市火,燔萬餘家。十六年九月,南劍州大火,民居存者無幾。



 紹熙元年八月壬寅,處州火,燔數百家。十二月戊申,建寧府浦城縣火。時查洞寇張海作亂,焚五百餘家。二年四月,行都傳法寺火,延及民居。言者以戚里土木為孽,火數起之應。是月,徽州大火,夜燔州治、譙樓、官舍、獄宇、錢帑庫務,凡十有九所,五百二十餘區,延燒千五百家,自庚子至於壬寅乃熄。
 五月己巳,金州火,燔州治、官舍、帑藏、保勝軍器庫、城內外民居甚眾。三年正月己巳,行都火,通夕,至於翌日,闤闠焚者半。十一月,又火,燔五百餘家。十二月甲辰,鄂州火,至於翌日,燔八百家。



 慶元二年八月己酉,永州火,燔三百家。三年閏月甲申,金州都統司中軍壘舍火,焚千三百餘區。閱六月乙酉,又火。燔二千餘區。是冬,紹興府僧寺火,延燒數百家。六年八月戊戌,徽州火。燔州獄、官舍,延及八百餘家。



 嘉定元年三月戊寅,行都大火,至於
 四月辛巳,燔御史臺、司農寺、將作軍器監、進奏文思御輦院、太史局、軍頭皇城司、法物庫、御廚、班直諸軍壘,延燒五萬八千九十七家。城內外亙十餘里,死者五十有九人,踐死者不可計。城中廬舍九毀其七,百官多僦舟以居。火作於寶蓮山御史臺胥楊浩家,諫議大夫程松請戮浩以謝都民。疏再上,始黥配萬安軍,猶免決。自是民訛言相驚,亡賴因縱火為奸利。二年六月己卯,臨安府火。三年正月丁酉,襄陽府火作而風暴,選鋒軍校於
 友直死於救焚,止延燒六十餘家。帥、漕臣上其功,贈二秩,官其子二。十一月甲午,福州火。燔四百餘家。四年三月丁卯,行都大火,燔尚書中書省、樞密院、六部、右丞相府、制敕糧料院、親兵營、修內司,延及學士院、內酒庫、內宮門廡,夜召禁旅救撲。太室撤廟廡,遷神主並冊、寶於壽慈宮。翼日戊辰旦,火及和寧門鴟吻,禁卒張隆飛梯斧之,門以不焚。火作時,分數道,燔二千七十餘家。又翌日己巳,神主還太室。時省部皆寓治驛、寺。四月丙申,臨
 安府梵天寺火。六月,盱眙軍天長縣禁軍營火,鎧械為盡。八月壬辰,鄂州外南市火,燔五百餘家。



 開禧二年二月癸丑,壽慈宮災。四月壬子,行都火,燔數百家。



 嘉定二年八月己巳,信州火,燔二百家。九月丁酉,吉州火,燔五百餘家。是歲,瀘州火,燔千餘家。十一月丁亥,建寧府政和縣火,燔百餘家。四年閏二月己卯,紹興府嵊縣浦橋火,燔百餘家。三月,滁州火,燔民居甚多。十月,撫州火。辛卯,福州一夕再火,燔城門、僧寺,延燒千餘家,死者數人。
 五年五月己未,和州火。燔二千家。八年八月辛丑,湖州火,燔寺觀,延燒三百家。九年七月甲戌,南劍州沙縣火,燔縣門、官舍及千一百餘家,民有死者。十一年二月,行都火。燔數百家。九月己巳,禁垣外萬松嶺民舍火,燔四百八十餘家。十三年二月庚寅,安豐軍故步鎮火,燔千餘家,死者五十餘人。八月庚午,慶元府火,燔官舍、第宅、寺觀、民居甚眾。十一月壬子,行都火,燔城內外數萬家、禁壘百二十區。十七年四月丁卯,西和州焚軍壘及民居
 二千餘家,人火之也。守臣尚震午誤以為金人至而遁。六月丁亥,岳州火,燔岳陽樓、州獄、帑庫,延及八十家。己丑又火,燔百餘家。



 紹定元年三月,行都火,燔六百餘家。



 嘉熙元年六月,臨安府火,燔三萬家。



 淳祐元年,徽州火。十二年十一月丙申,行都火,至丁酉夜始熄。



 景定四年,紹興火。五年,臨安府大火。



 德祐元年,玉牒所災。



 乾德二年冬,無雪。五年冬,無雪。



 開寶元年冬,京師無雪。二年冬,無雪。



 淳化二年冬,京師無冰。



 至道元年冬,無雪。
 二年冬,無雪。



 大中祥符二年,京師冬溫,無冰。



 天聖五年,夏秋大暑,毒氣中人。



 嘉祐六年冬,京師無冰。



 治平四年冬,無雪。



 元豐八年冬,無雪。



 元祐元年冬,無雪。四年冬,京師無雪。五年冬,無冰雪。



 紹興五年五月,大燠四十餘日,草木焦槁,山石灼人,□曷死者甚眾。三十一年冬,無雪。



 乾道三年,冬溫,少雪無冰。五年,冬溫,無雪。六年,冬溫,無雪冰。



 紹熙三年冬,潼川路不雨,氣燠如仲夏,日月皆赤,榮州尤甚。



 慶元元年冬,無雪。二年冬,無雪。四年冬,無雪。越
 歲,春燠而雷。六年,冬燠無雪,桃李華,蟲不蟄。



 開禧三年冬,少雪。



 嘉定元年,春燠如夏。六年冬,燠而雷,無冰,蟲不蟄。八年夏五月,大燠,草木枯槁,百泉皆竭,行都斛水百錢,江、淮杯水數十錢,□曷死者甚眾。九年冬,無雪。十三年冬,無冰雪。越歲,春暴燠,土燥泉竭。



 建隆二年九月,亳州獻芝一株。



 乾德四年閏八月,黃岡縣民段贊屋柱生紫芝一本二莖,知州鄭守忠以獻。十二月,登州獻芝五莖。



 開寶四年,成都府民羅達家生芝。
 六年正月,知梓州趙延通獻芝一本。河中府大明觀殿芝草生,節度使陳思讓以聞。七年七月,陳州節度黨進獻控鶴營卒孫洪家芝二本。八月,又獻芝一本,四十九莖。九月,麻城縣廨芝生柱上,刺史王明以獻。十月,梓州獻芝草。



 太平興國二年八月,青城縣民家竹一本,上分雙莖。三年六月,項城縣令王元正獻芝草。七月,廣州獻芝草。八月,功臣堂柱生芝二本,知州範旻畫圖以獻。四年八月,廣州獻芝草。九月,華山道士丁少微獻白芝、黝
 芝各一器。五年五月,眉山縣竹一莖十四節,上分二枝,長丈四尺。九月,真定府行宮殿梁生芝。如荷花,知府趙賢進以圖來上。十月,龍水縣華嚴寺舊截竹為筒引水。忽生枝葉,長二丈許,知州姜宣以聞。六年三月,廣州獻黃芝一本九莖。七月,新津縣趙豐村竹一莖十二節,上分兩岐,知州崔憲以聞。七年六月,知黃州裴仁鳳獻芝草。七月,知羅江縣陳覃于羅瑰山獲芝四本以獻。湘陰縣萬壽寺松根,芝草二本生,轉運副使趙昌獻之。八月,
 再生四本,昌又獻。潭州民歐陽進、夏侯敏園中芝三本。宜興縣民長孫裕家生芝,紫莖黃蓋。十月,雄州實信院竹業叢芝草,僧致仁採之復生,悉以上獻。八年二月,知福州何允昭獻芝二本。五月,漢州獻芝。十月乙酉,蜀州獻瑞竹一本十六節,上分兩枝。知連州史昭文獻芝二莖。十一月,婺源縣民王化於王陵山石上得紫芝一本,叢生五莖。金州監軍廨生芝三本。九年十月,金州獻芝三本,永康軍獻芝九莖,同日至闕下。十一月,知梓州沉
 護獲芝三莖。



 雍熙二年七月,靈州芝草生,知州侯贇刻木為其狀來獻。三年三月,殿前承旨張思能使楚、泗,獻所得芝草五本。四月,眉山縣獻《異竹圖》。八月,刑部尚書宋琪家牡丹三華。



 端拱元年五月,知襄州郝正獻芝五本。八月,廣州鳳集合歡樹下,得芝三本。二年七月,彭山縣民家生異竹。舒州芝草生,知州趙孚以獻。十月,密州獻芝草。



 淳化元年四月,永州監軍廨芝草生,知州克憲以聞。八月,黃州刺史魏丕獻芝草。二年二月,射洪縣安
 國寺竹二莖同本。六月,舒州竹連理,知州樂史以聞。十一月,陵州民趙崇家慈竹二莖,長六尺許,其上別有根柢,莖分十枝,長丈餘;又一本三莖並聳。三年十月,朗州異竹生。京師太平興國寺牡丹生華,占云:「有喪。」是月,恭孝太子薨。四年正月,知興化軍馮亮獻芝草。十月,彭門芝草生。十二月,榮州獻《異竹圖》。五年正月,密州獻芝草四本,枝葉扶疏。二月,知溫州何士宗獻芝草五本;十月,又獻十本。



 至道元年十一月,潭州監軍廨生竹一本,長
 二尺許,枝葉萬餘,尤為殊異。二年六月,虔州龍泉縣合龍院一竿分兩枝。河南縣民張知遠家芝草生,判府呂蒙正表上之。閏七月,密州獻芝二本。三年二月,廣東轉運使康戩獻紫芝。



 咸平元年十二月,宣化縣保聖山瑞竹生一本二枝。二年閏二月,宣、池、歙、杭、越、睦、衢、婺諸州箭竹生米如稻。時民饑,採之充食。九月,劍州驛廳梁上生芝草,一枝三朵,其色黃白,知州李仁衡圖以獻。四年正月,灘州獻芝草一本,如佛狀。十二月,知淮陽軍王礪
 獻芝草三本。六年五月,導江縣民潘矩田生芝,三層,黃紫色,高五寸許。九月,相州牧龍坊生芝一莖,色紫黃,長尺餘,分七枝,枝如手五指狀。其最上枝類鳳者,知州張鑒以獻。



 景德三年八月,蔡州獻芝草。四年十月,知廣安軍王奇上《芝草圖》。十二月,蓬州上《瑞竹圖》。



 大中祥符元年四月,溫州獻《瑞竹圖》。五月辛未,以東封,遣經度制置使王欽若祭文宣王廟,於孔林得芝五株,色黃紫如雲色,及人戴冠冕之狀。詔內侍楊懷玉祭謝。復得芝四本,
 輕黃,如雲氣之狀。癸未,內侍江德明於白龍潭石上得紫黃芝一本以獻。六月,瑕丘縣民宋固於堯祠前得黃紫芝九本,連理者四;又縣民蔡珍得芝一本,王欽若以獻。欽若又於岱岳及堯祠前,再得芝二十二本。連理者二,及有貫草石者。七月,欽若親獲芝十一本,又州長及民所得二十六本,有重臺連理及外白內紫之狀,且言:「泰山至日生芝草,軍民競採,請給緡帛。」從之。兗州獄空,司理參軍郭保讓掃除其間,得芝四本。八月,須城縣民
 家芝草生。乾封縣民家屋柱生芝。滋長連袤,色鮮潔如繪畫。欽若獻芝草八千一百三十九本,有貫草木、附石、連理及飾為寶山者。黃州獻異竹一本雙莖。九月,趙安仁來獻五色金玉丹紫芝八千七百十一本。鞏縣柴務牡丹華。十月,泰山芝草再生者甚眾。辛丑,車駕次鄆州,知州馬元方獻芝草五本。甲辰,欽若等又獻泰山芝草三萬八千五十本,有並五連、三連理者,五色重暈如寶蓋,下相連帶,凡草木五穀如寶山、靈禽、瑞獸之象者六
 百四十二。詔令封禪日列天書輦前,又送諸路名山勝景及賜宰相。是月,復州獻芝草,類神仙佛像。河中府酒廚梁上生芝,一本十二葉,其色如玉。安陽縣段贊家紫芝連理,長尺餘,又民李釗屋柱生芝三本。霍丘縣河亭及聖惠坊並有紫芝生。十二月,福州懷安縣龍眼樹上紫芝連理。溫州獻《靈芝圖》。二年正月,福州荔枝樹生連理芝二本。二月,饒州獻芝草四本。七月,遂州皇澤寺芝草生,凡五十本。九月,榮州廨庭中生芝二本。十月,果州
 青居山獻金暈連理芝草。十一月,華山張超谷石上生紫芝二本。嵩岳生芝草五十本。石首縣文宣王廟殿柱芝草生。又龍蓋山萬福里民宗永昌園藤上芝草生,一本雙莖。十二月,漢州芝草生。黔州芝草一莖十二枝,若山峰狀。三年正月,井研縣三惠寺生芝草十本。二月,昌州廨廳柱芝生四本。閏二月,饒州芝草生。三月,西充縣青蓮塔院、太平觀並生芝草。四月,京師竹有華,占云:「歲不豐。」六月,綿、邵、鄂州並芝草生。七月,虢州聖女觀生芝
 草三本。八月,穎縣民得田芝十二本。蜀州生芝草,一莖九葉。江陵縣民張仲家竹自根上分乾,其一乾又分三莖,九月,江陵府永泰寺竹出地七節,分為兩莖,長丈餘,知府陳堯咨以聞。華州敷水民侯元則入華陽川石罅,得芝一本,知州顧端以獻。十月,內侍任文慶詣茅山,設醮洞中,獲芝草二十八本,有如人手者。十一月,安鄉縣謝山獲芝二十二本,其七狀如珊瑚而色紫。十二月,神泉縣獲芝四本。四年正月,知華州崔端獻芝草,狀如仙
 人掌。須城縣民李道安於黃仙公洞臺上得芝草一本以獻。二月,崔端又獻芝草十本。知益州任中正獻芝草二十二本。知遂州毋賓古獻芝草。四月,古田縣僧舍竹一本,上分三莖。端昌縣民李讓家筀竹一本,去地五尺許,分為二莖,知州範應辰以聞。六月,夔州芝草生稟舍中。七月,知亳州徐泌、知江州王文震並獻芝草。知郴州袁延慶獻芝草十本。八月,邕州雲封寺柏樹生芝五本,知州劉知詰以獻。西充縣廣川王廟生芝十本,其三連
 理。八月,知信州李放獻《瑞竹圖》三本。十一月,河中府獻芝草。真源縣民王順慈、司徒捷家生芝各一本。岳州、道州並獻芝草。南嶽奉冊使薛映、副使錢惟演過荊門軍神林石上,獲芝草以獻。十二月,鉛山縣仁壽僧舍生芝草一本,雙枝,長尺八寸。五年六月,潯州六祖院法堂紫芝雙秀,知州高志寧以聞。八月,亳州獻芝草。十月,澤州廳事梁上,生白莖紫蓋芝二十四本。閏十月,常州芝草生。又蕭山縣芝生李樹上十一本。十一月,廣州獻芝草
 二百三十七本。晉原縣僧舍芝草一本。十二月,隨州芝草生。亳州獻鹿邑縣民所獲芝草四本。候官縣山上生芝草五十四本。閩縣望泉寺生芝草十本。寧德縣支提山石上生芝草十五本。六年二月,江州廬山崇聖院生芝九本,知州王文震以獻。四月,饒州承天院東山生芝四本,連葉。六月,壽丘縣獲紫莖金芝一本。景陵縣管陽山林中獲芝三十本。七月,內侍石延福登兗州壽丘,獲芝一本,貫草而生,又旁得三十本。亳州團練使高漢英
 獻芝八本。鼎州城門柱下生芝一本。八月,繼照堂生芝一本,紫莖黃蓋。奉祀經度制置使丁謂至真源縣,太清宮道士、瀨陽鄉民繼獲芝八十本以獻。乙丑,又獲二百五十本,有一本三莖,一莖如雲氣佛像者。九月,又得宋城縣民所獲芝五十本獻之。十月,丁謂來朝,獻芝草三萬七千一百八本,飾以仙人、寶禽、異獸之狀;十一月,又獻九萬五千一百本。明年,車駕至真源,民有詣行闕獻者,又一萬八千本。是冬,兗州景靈宮芝草生。慶成軍
 大寧廟聖制碑閣生金芝二本。昭州龍岳山資壽寺芝草生。潯州廳廨柱芝生一本,上分為二,其上又生二本,凡三重。無錫縣民曹詵家食案生芝,赤黃有光,長尺許。又知南安軍章得一獻芝草。七年正月,明州獻茹侯山石上芝草一本四莖。二月,知信州歐陽陟獻芝草七本,忠州獻芝五本。四月,福州獻芝草二本。五月,郪縣西上石崖生紫芝十五本。七月,華州民入華山,得白石上芝草,雙莖連蓋。八月,均州、獻州獻芝千二百二十七本。十月,
 慶成軍大寧廟石雙莖芝生,其上合乾,明、英二州芝草生。十一月,蜀州芝草二本生竹根。八年二月,青州武成王廟柱生芝一本,知州張知白以圖獻。三月,榮州應靈縣彌陀佛舍生紫芝三本,其一雙乾,上如蓋。四月,昌州有芝生石上,一本四莖,其色黃白。四月,彰明縣民家竹一根,上分二本、十三節。又開元寺桃竹一莖,上分十八節,皆相對。五月,道州舜祠旁生芝二十一本。六月,盩啡縣民家芝草三莖,共成一葉,又石芝一本。十月,晉原縣
 民柏扆家生芝三根,合為一本。九年七月,知信州董溫獻芝十二本。八月,知廬州餘獻卿獻芝二本。九月,涪城縣石壁生芝一本。十一月,武岡縣民何文化園竹生兩株同本,上分四莖。十二月,晉原縣民李彥滔家竹一本,雙莖對節,知州王世昌圖以獻。



 天禧元年三月,新津縣平蓋下玉皇案下芝草生。鄂州天慶觀聖祖殿芝草生。四月,邵陽等縣竹生穗如米,民饑,食之。又浮梁縣竹生穗如米。七月,漢陽軍太平興國寺異竹一本,生二莖,節
 皆相對。十二月庚午。內出芝草如真武像。二年正月庚子,內出真游、崇徽二殿《梁上芝草圖》示宰相。五月,兗州景靈宮昭慶殿生金芝二本。三年六月,漢陽軍芝草生一百五十餘本。七月,嵩山崇福宮獲芝草一百本,有重臺連理、貫草者,知河南府馮拯以獻。四年四月,梁山軍民王崇扆竹園生金暈紫芝五本。十一月,上饒縣民王壽園中生芝草三本,皆金暈,其二連理。



 乾興元年六月,蘇、秀二州湖田生聖米,居民取以食。興州竹有實,如大
 麥,民取以食。占曰:「大饑。」八月,洋州民李永負土成母墳,芝生墳上。



 天聖元年五月,興州竹有實如大麥,民取以食之。占曰:「竹有實,大饑。」八月甲寅,有芝生於天安殿柱,召輔臣觀之。九月戊午,城西下木場芝草生。三年七月,梓州城門生芝二本。四年正月。梓州民家生芝四本。九月,榮州芝生。



 明道元年七月,榮州、連州並芝生。



 景祐二年九月,嘉州芝草生。四年五月丙寅,有芝生於化成殿楹。



 慶歷元年二月丙午,京師雨藥。二年八月,梓州芝草
 生。五年八月,洪州章江禪院堂柱芝草生,高一尺三寸,葉二十一層,色白黃,有紫暈,旁生小芝,葉九層,上有氣如煙。



 皇祐元年七月,福州生芝一十二本。十月,湖州芝草生。三年六月丁亥,無為軍獻芝草凡三百五十本。上曰:「朕以豐年為上瑞,賢臣為寶,至於草木、魚蟲之異,焉足尚哉!」



 五年閏七月,榮州芝草生。



 嘉祐三年十一月,河南府芝草生。六年正月,清川縣漢光武祠生芝草,一本三岐。八月,施州歌羅砦生芝四本。十月,汝州新砦巡檢
 廨舍生芝五本。



 熙寧元年,益陽縣雷震山裂,出米可數十萬斛,繼至京師,信米也,但色黑如炭。八年七月,鼎州產芝三本,其一類珊瑚,枝葉摎結,鹽官縣自三月地產物如珠,可食;水產菜如菌,可為菹,饑民賴之。九年五月,流江縣產芝二十一本。



 元豐二年四月,眉州生瑞竹。六月,忠州雨豆。七月甲午,南賓縣雨豆。十一月,全州芝生十二本。三年六月,安州芝生二十九本,其一連理。臨江軍芝生四十三本。四年十月,郪縣天慶觀生瑞竹一
 本,自第九節分莖雙起。五年七月,永康縣生紫芝九本。十一月,閬中縣生紫芝六、金芝七。永康縣生紫芝九。六年八月,吉州芝生三十三本。十二月,滕縣官舍生異草,經月不腐。七年四月,景靈宮芝生六本於天元殿門。五月,開化縣芝生九本,黃白紫色。八月,永安陵下宮芝生一本。十月,青州芝生二十一本。



 元祐元年七月,武安軍言:「前秘書省正字鄭忠臣母墳前生芝草一本,紫莖黃蓋。」三年六月,臨江縣塗井鎮雨白黍;七月,又雨黑黍。四年
 九月,江津縣石上生芝草二本六莖。五年二月,晉原縣生芝草四十二本。七年十一月,滁州生芝二百餘本。



 紹聖三年九月,淄川縣生芝草,有穀十科穿芝草生二枝。十月,河南府大內地生芝草。



 元符二年正月,處州民田生瑞竹。二月,瀘州生芝草一本,同根駢乾,至蓋復合為一。又衡州郡廳生紫芝一根十六葉。



 崇寧元年八月,盤石縣芝草連理。三年十月,復州、澤州芝草生。四年正月,戎州、宿州芝草生。七月,瀘州芝草、瑞竹生。五年冬,澶州、
 安州芝草生。



 大觀元年三月,宣、鄆、湖、潤州皆芝草生。廬州雨大豆。九月,崇天臺及兗州孔林芝草生。二年,陳、兗、筠州、廣德軍芝草生。三年秋,西京、湖、海、普、渠州、南安軍芝草生。



 政和元年正月,萊州芝草生。十一月,虔州聖祖殿芝草生。二年二月戊子,河南府新安縣蟾蜍背生芝草。自是而後,祥瑞日聞。玉芝產禁中殆無虛歲,凡殿宇、園苑及妃嬪位皆有之。外則中書尚書二省、太學、醫學亦產紫芝。四年八月,建州境內竹生米數千萬石。五年
 十一月癸酉,越州承天寺瑞竹一竿七枝,乾相似,其葉圓細,生花結實。詔送秘書省,仍拜表賀。五年五月,禁中芭蕉連理。八月甲子,蘄州進芝草一萬一千六百枝,內一枝紫色,九乾。十二月己未,汝州進六萬本,其間連理、雙枝者一千八百八十,有司不勝其紀。初猶表賀,後以為常,不皆賀也。時朱勝非為京東提舉學事,行部至密州界,見縣令部數百夫入山採芝。彌漫山谷,皆芝菌也。或附木石,或出平地,有一本數十葉,層疊高大,眾色咸
 備。郡守李文仲採及三十萬本,每萬本作一綱入貢。文仲尋進職,授本道轉運使。



 建炎二年九月癸卯,權知密州杜彥獻芝草,五葉,如人指掌,色赤而澤。宰臣黃潛善奏:「色符火德,形像股肱之瑞。」高宗不啟視,卻之。



 紹興元年七月乙未,浙西安撫大使劉光世以枯秸生穗奏瑞。高宗曰:「朕在潛邸,梁間生芝草,官僚皆欲上聞,朕手碎之,不欲寶此奇怪。」乃卻之。十六年正月辛未,瀘州天雨豆,近草妖也。十六年,梅州孔子廟生芝。二十一年,紹興
 府學御書下生芝。番陽縣石門民家籬竹生重萼牡丹,又民家灶鼎生金色蓮華。房州治所彩山下生萱。萬州、虔州放生池生蓮,皆同蒂異萼。二十三年六月,汀州生蓮,同蒂異萼者十有二。二十五年五月,太室楹生芝九莖,秦檜帥百官觀之,稱賀。勾龍廉、沈中立以獻頌遷擢,周麟之請繪之鹵簿行旗。檜孫禮部侍郎塤請以黎州甘露降草木、道州連理木、鎮江府瑞瓜、南安軍瑞蓮、嚴、信州瑞芝悉圖之旗。是冬,檜薨,高宗曰:「比年四方奏瑞,
 文飾取悅,若信州林機奏秦檜父祠堂生芝,佞諛尤甚。」明年四月甲午,詔郡國無獻瑞。



 乾道元年七月,池州竹生穗,實如米,饑民採之以食,守臣魯察為《野穀生竹圖》以獻。御史劾察不以民食草木為病,坐佞免官。



 慶元五年八月,太室西北夾室楹生白芝,四葉,前史以白芝為喪祥。明年八月,國連有大喪。



 嘉泰二年十一月,秘書省右文殿楹生芝
 二莖。



\end{pinyinscope}