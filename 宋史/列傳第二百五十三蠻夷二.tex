\article{列傳第二百五十三蠻夷二}

\begin{pinyinscope}

 紹興三年,臣僚言:「武岡軍溪峒舊嚐集人戶為義保,蓋
 其風土、習俗、服食、器械悉同徭人。故可為疆場捍蔽,雖曰籍之於官,然亦未嚐遠戍。靖康間,調之以勤王,其後湖南盜起,征斂百出,義保無復舊制,困苦不勝,乃舉其世業,客依蠻峒,聽其繇役。州縣猶驗舊籍催科,胥隸及門,則挈家遠徙,官失其稅,蠻獠日強。兼武岡所屬三縣,悉為徭人所有,遠戍之實已無,而鄉戶弩手之名尚在,歲取其直,人戶谘怨。乞擇本路監司詳議以聞。」詔從之。



 四年,辰州言,歸明保靜、南渭、永順三州彭儒武等久欲
 奉表入貢。詔以道路未通,俾荊湖北帥司慰諭,免赴闕。遣人持表及方物赴行在,仍優賜以答之。九月,詔荊湖南、北路溪峒頭首土人及主管年滿人合給恩賜,俾各路帥司會計覆實以聞。



 六年,知鼎州張觷言:「鼎、澧、辰、沅、靖州與溪峒接壤,祖宗時嚐置弓弩手,得其死力,比緣多故,遂皆廢闕。萬一蠻夷生變,將誰與捍禦?今雖各出良田,募人以補其額,率皆豪強遣僮奴竄名籍中,乘時射利,無益公家,所宜汰去。則募溪峒司兵得三百人,俾
 加習練,足為守禦,給田募人開墾,以供軍儲。」詔荊湖北路帥司相度以聞。帥司言:「營田四州舊置弓弩手九千一百一十人,練習武事,散居邊境,鎮撫蠻夷,平居則事耕作,緩急以備戰守,深為利便。靖康初,調發應援河東,全軍陷沒。今辰、沅、澧、靖等州乏兵防守,竊慮蠻夷生變叵測。若將四州弓弩手減元額,定為三千五百人,辰州置千人,沅州置千五百人,澧州、靖州各置五百人,分處要害,量給土田,訓練以時,耕戰合度,庶可備禦。以所餘
 閑田募人耕作,歲收其租,其於邊防財賦,兩得其便,可為經久之計。」詔從之。



 七年六月,張觷言:「湖外自靖康以來,盜賊盤踞,鍾相、楊太山、雷德進等相繼叛,澧州所屬尤甚。獨慈利縣向思勝等五人素號溪峒歸明,誓掌防拓,卒能保境息民,使德進賊黨無所剽掠,思勝後竟殺德進。會官軍招撫劉智等,而彭永健、彭永政、彭永全、彭永勝及思勝共獻糧助官軍,招復諸山四十餘柵,宣力效忠功居多,宜加恩賞。」詔思勝等五人各轉兩資。九月,
 詔荊湖、廣南路溪峒頭首土人內有子孫應襲職名差遣,及主管年滿合給恩賜之數,俾帥司取會核實以聞。



 九年,宜章峒民駱科作亂,寇郴、道、連、桂陽諸州縣,詔發大兵往討之,獲駱科。餘黨歐幼四等復叛,據藍山,寇平陽縣,遣江西兵馬都監程師回討平之。



 十年,承信郎琴州溪峒楊進顒等率族屬歸生界五百餘戶、疆土三百餘里,獻累世所造兵器及金爐、酒杯各一,求入覲,詔本路帥司敦遣以行。十二年,詔以施州南砦路夷人向再
 健襲父思遷充銀青光祿大夫、檢校國子祭酒兼監察御史、武騎尉、知懿州事。



 十四年十月,湖南安撫使劉昉奏,武岡軍徭人有父子相殺者,宜出兵助其父,俾還省地。上以問輔臣秦檜,檜曰:「恐輕舉生事。」帝曰:「恩威不可偏廢,可懷則示之以恩,否則威之。不侵省地則已,或有所侵,奈何不舉,俾知所畏哉。」十二月,成忠郎充武岡軍綏寧縣管界都巡檢兼溪峒首領楊進京,率其族三百人,備黃金、朱砂、方物求入貢,先遣其子孝友陳請。詔本
 路帥司閱舊制以聞,給孝友錢三百貫,俾還聽進止。



 十五年,楊進顒復求入貢,以武岡軍不時敦遣為言。詔本路帥司閱實應襲人姓名來上,並促進顒入覲。四月,廣南東路提刑黃應南言:「溪峒巡檢、尉、砦官不嚴守備,縱民與徭交通,恐啟邊釁,乞詔有司申嚴法令,俾帥臣、監司常加覺察。」宰臣以為沿邊互市,恐不宜禁絕。帝曰:「往年禁西夏互市,遂至用兵,可令帥司裁決。」前知全州高楫言:「徭人今皆微弱,不敢先侵省地,砦官每縱人深入,
 略其財物,遂致乘間竊發。宜詔與溪峒接壤州郡毋侵徭人,庶使邊民安業,以廣陛下柔遠好生之德。」帝從其言,詔守臣一遵成法,務在撫綏。



 二十四年,禽楊正修及其弟正拱,送理寺獄鞫治,斬之。初,正修侍其父再興入覲,獻還省民疆土,遂命以官。建炎後,與弟正拱率九十團峒徭人出武岡軍,縱火殺掠民財為亂。紹興間,潭州帥司嚐招徠之,後復作亂,屢抗官軍,至是伏誅。二十八年七月,楊進京等復求入貢,詔以道遠慰諭之,優其賜
 與。



 隆興初,右正言尹穡言:「湖南州縣多鄰溪峒,省民往往交通徭人,擅自易田,豪猾大姓或詐匿其產徭人,以避科差。內虧國賦,外滋邊患。宜詔湖南安撫司表正經界,禁民毋質田徭人。詐匿其產徭人者論如法,仍沒入其田,以賞告奸者。田前賣入徭人,俾為別籍,毋遽奪,能還其田者,縣代給錢嚐之。」帝從其言。



 乾道元年,宜章峒賊李金陷郴州,焚桂陽軍,州將棄城遁,衡州調常寧縣兵救之,弗克。世忠峒李昂霄者,率壯丁禦賊,民恃以安。
 湖南提舉常平鄭丙請發鄂渚軍討賊,平之。昂霄以功補承節郎,管轄衡州常寧縣溪峒,及官其子當年,俾後得襲職。



 三年,靖州界徭人姚明教等作亂,詔荊、鄂駐紮明椿選將率精銳千人,會屯戍官合擊之,能立功者有厚賞。八月,詔平溪峒互市鹽米價,聽民便,毋相抑配,其徭人歲輸身丁米,務平收,無取羨餘及折輸錢,違者論罪。十一月,南郊禮成,詔以緣邊溪峒,州縣失於拊循,致懷反側,或逃竄山谷,其在赦恩以前,並加寬宥,能復業
 者,罪一切置不問,互市如故,悉聽其便,守臣常加撫問。以稱綏遠之意。



 四年二月,詔湖南北、四川、二廣州軍應有溪峒處,務先恩信綏懷,毋弛防閑,毋襲科擾,毋貪功而啟釁。委各路帥臣、監司常加覺察。是月,詔禁沿邊奸人毋越逸溪峒,誘致蠻獠侵內地,違者論如律,其不能防閑致越逸者亦罪之。湖廣總領周嗣武言邊事,如二年四月之詔,帝嘉納之。是歲,田彥古死,子忠佐襲職,授銀青光祿大夫、檢校散騎常侍、知溪峒安化州兼監察
 御史、飛龍騎尉。



 六年,盧陽西據獠楊添朝寇邊,知沅州孫叔傑調兵數千討之,敗績,死者十七八。初,徭人與省戶交爭,殺二人死,叔傑輒出兵破其十三柵,奪還所侵地,於是徭人相結為亂。諸司請調常德府城兵三百人,益官兵三千人,合擊討之。宰臣虞允文奏曰:「蠻夷為變,皆守臣貪功所致。今徭人仇視守臣,若更去叔傑,量遣官軍,示以兵威,徐與盟誓,自可平定。」帝允其奏,俾葉行代叔傑,開示恩信,諭以禍福,遂招降之,邊境悉平。前知
 武岡軍趙善穀言:「武岡與湖北、廣西鄰壤,為極邊之地,溪峒七百八十餘所,七峒隸綏寧縣,五溪峒隸臨岡縣。紹興三十年,減冗員,改縣為臨口砦。然五峒之徭俗尤獷悍,釁生毫髮,則操戈相仇,砦官不能為輕重。況本軍巡防砦柵,惟真良、三門、兵溪、香平有土軍可備守禦,餘有官無兵,其關硤、武陽等砦設巡檢二員,徒費廩祿。以臣所知,宜復臨口砦為縣,則徭蠻易於製服,汰去冗員,則官廩亦無虛費,實邊郡之利也。」



 七年,前知辰州章才
 邵上言:「辰之諸蠻與羈縻保靜、南渭、永順三州接壤,其蠻酋歲貢溪布,利於回賜,頗覺馴伏。盧溪諸蠻以靖康多故,縣無守禦,犵狑乘隙焚劫。後徙縣治於沅陵縣之江口,蠻酋田仕羅、龔誌能等遂雄據其地。沅陵之浦口,地平衍膏腴,多水田,頃為徭蠻侵掠,民皆轉徙而田野荒穢。會守倅無遠慮,乃以其田給靖州犵狑楊姓者,俾佃作而課其租,所獲甚微。楊氏專其地將二十年,其地當沅、靖二州水陸之衝,一有蠻隙,則為害不細,臣謂宜
 預為之備。靖康前,辰州每歲蒙朝廷賜錢七萬貫,、絹、布共八千一百匹,綿一萬七千兩。是時,本州廂禁軍一千四百餘人,沿邊一十六砦,土兵六百餘人,皆可贍給。其後中外多故,今歲賜止得一萬二千緡,而本州財復匱乏,無以充召募之費。禁軍止二百一十餘人,諸砦土兵止一百五人,甚至砦官有全無一兵而徒存虛名者,其於邊防豈可不為深慮?若歲增給民錢一萬,俾本州募強壯禁軍或效用二百人,分屯盧溪等處,以防諸蠻,
 庶使邊患永消,可免異時調遣之費。」書奏,詔湖北帥臣詳議以聞。是年,申嚴邊民售田之禁,守令不能奉法者除名,部刺史常加糾察。



 八年,知貴州陳乂上疏言:「臣前知靖州時,居蠻夷腹心,民不服役,田不輸賦,其地似若可棄。然為重湖、二廣保障,實南服之要區也。或控製失宜,或金穀不繼,或兵甲少振,蠻獠則乘時竊發,勤勞王師,朝廷當重守臣之選。崇寧初戍兵三千人,建炎以來,每於都統司或帥司摘兵二千人,以備屯戍。其凶悍者,
 以州郡不能製,遂慢守臣,反通徭蠻以撓編民。州郡非白主帥不敢治,比得報,已晚矣。故戍兵敢肆其惡,一旦有警,復安能為用?臣以為宜聽守臣節製為便。」帝嘉其言,復問左右曰:「靖隸湖北,今聞仰給廣西,何也?」趙雄對曰:「靖州本溪峒,神宗時創為誠州,元祐間廢,尋復為軍,徽宗朝始改靖州,與桂府為鄰,故令廣西給其金穀之費。近歲漕司匱乏,乃責辦諸州,以故不能如約。宜復舊制,俾廣西漕臣如期饋運。靖州屯戍官兵聽守臣節制,
 於事為便。」帝從之。



 十年四月,全州上言:「本州密邇溪峒,邊民本非奸惡。其始,朝廷禁法非不嚴密,監司、州郡非不奉行,特以平居失於防閑,故馴致其亂。又兼溪穀山徑非止一途,如靜江、興安之大通虛,武岡軍之新寧、盆溪及八十里山,永州之東安,皆可以徑達溪峒。其地綿亙郡邑,非一州得專約束,故遊民惡少之棄本者,商旅之避征稅者,盜賊之亡命者,往往由之以入。萃為淵藪,交相鼓扇,深為邊患。如武岡楊再興、桂陽陳峒相繼為
 亂,實原於此。為今計者,宜徙閑地巡檢兵,及分遣士卒屯諸溪穀山徑間,俾湖南北、廣西帥憲總其役,庶幾事權有歸,號令可行也。」儒林郎李大性上言:「比年徭蠻為亂,邊吏慮妨賞格,往往匿不以聞,遂致猖獗,使一方民命寄於徭人之手,誠可哀憫。近如梁牟等寇沅州,劫墟市,殺戮齊民,州縣告急於兩月之後,比調官軍討捕,俘降其賊,而人之被害已酷矣。宜戒州縣或遇徭人竊發,畫時以聞,違者論罪。仍命監司、帥臣常加覺察,庶幾先
 事備禦,俾徭人亦知畏懼,不敢侵軼,以傷吾民也。」



 十一年,詔給事中、中書舍人、戶部長貳同敕令所議,禁民毋質徭人田,以奪其業,俾能自養,以息邊釁。從知沅州王鎮之請也。沅州生界犵狑副峒官吳自由子三人,貨丹砂麻陽縣,巡檢唐人傑誣為盜,執之送獄,自由率峒官楊友祿等謀為亂。帥司調神勁軍三百人及沅州民兵屯境上,聲言進討。先遣歸明官田思忠往招撫之,以孔目官為質,世祿等既盟,自由取其三子以歸。



 嘉泰三年,
 前知潭州、湖南安撫趙彥勵上言:「湖南九郡皆接溪峒,蠻夷叛服不常,深為邊患。製馭之方,豈無其說?臣以為宜擇素有知勇為徭人所信服者,立為酋長,借補小官以鎮撫之。況其習俗嗜欲悉同徭人,利害情偽莫不習知,故可坐而製服之也。五年之間能立勞效,即與補正。彼既榮顯其身,取重鄉曲,豈不自愛,盡忠公家哉?所謂捐虛名而收實利,安邊之上策也。」帝下其議。既而諸司復上言:「往時溪峒設首領、峒主、頭角官及防遏、指揮等
 使,皆其長也。比年往往行賄得之,為害滋甚。今宜一新蠻夷耳目,如趙彥勵之請,所謂以蠻夷治蠻夷,策之上也。」帝從之。



 嘉定元年,郴州黑風峒徭人羅世傳寇邊,飛虎統製邊寧戰沒。江西、湖南驚擾,知隆興趙希懌、知潭州史彌堅共招降之。二年,李元礪、羅孟二寇江西,攻破龍泉縣。李再興戰敗,死之,江州駐紮都統製趙選亦戰死。初,吉州獲賊長七人繫獄,土豪黃從龍為賊畫策,賂吉守李絪,得縱還,賊遂無所忌。有侯押隊者,領兵戍龍
 泉境上,元礪復用從龍計,椎牛釃酒以犒官軍。賊至,官軍皆醉,狼狽散走。寇之初起甚微,賊伺知議論不一,故玩侮官軍。方江西力戰則求降湖南,湖南戰則求降江西,牽製王師,使不得相應援。其後命工部侍郎王居安知豫章,擒獲之,溪峒略平。



 五年,臣僚上言:「辰、沅、靖等州舊嚐募民為弓弩手,給地以耕,俾為世業。邊陲獲保障之安,州縣無轉輸之費。比年多故,其製浸弛,徭蠻因之為亂,沿邊諸郡悉受其害。比申朝廷調兵詔捕,曠日持
 久,蠻夷習玩,成其猖獗之勢。其如楊晟台、李金、姚明教、羅孟二、李元礪、陳廷佐之徒,皆近事之明驗也。為今計者,宜講舊制,可紓饋餉之勞而得備禦之實,其安邊息民之長策歟。」



 七年,臣僚復上言:「辰、沅、靖三州之地,多接溪峒,其居內地者謂之省民,熟戶、山徭、峒丁乃居外為捍蔽。其初,區處詳密,立法行事,悉有定制。峒丁等皆計口給田,多寡闊狹,疆畔井井,擅鬻者有禁,私易者有罰。一夫歲輸租三鬥,無他繇役,故皆樂為之用。邊陲有警,
 眾庶雲集,爭負弩矢前驅,出萬死不顧。比年防禁日弛,山徭、峒丁得私售田。田之歸於民者,常賦外復輸稅,公家因資之以為利,故謾不加省。而山徭、峒丁之常租仍虛掛版籍,責其償益急,往往不能聊生,反寄命徭人,或導其入寇,為害滋甚。宜敕湖、廣監司檄諸郡,俾循舊制毋廢,庶邊境綏靖而遠人獲安也。」



 梅山峒蠻,舊不與中國通。其地東接潭,南接邵,其西則辰,其北則鼎、澧,而梅山居其中。開寶八年,嚐寇邵之武
 岡、潭之長沙。太平興國二年,左甲首領苞漢陽、右甲首領頓漢淩寇掠邊界,朝廷累遣使招諭,不聽,命客省使翟守素調潭州兵討平之。自是,禁不得與漢民交通,其地不得耕牧。後有蘇方者居之,數侵奪舒、向二族。



 嘉祐末,知益陽縣張頡收捕其桀黠符三等,遂經營開拓。安撫使吳中復以聞,其議中格。湖南轉運副使範子奇復奏,蠻恃險為邊患,宜臣屬而郡縣之。子奇尋召還,又述前議。熙寧五年,乃詔知潭州潘夙、湖南轉運副使蔡熚、
 判官喬執中同經製章惇招納之。惇遣執中知全州,將行,而大田三砦蠻犯境。又飛山之蠻近在全州之西,執中至全州,大田諸蠻納款,於是遂檄諭開梅山,蠻徭爭辟道路,以待得其地。東起寧鄉縣司徒嶺,西抵邵陽白沙砦,北界益陽四里河,南止湘鄉佛子嶺。籍其民,得主、客萬四千八百九戶,萬九千八十九丁。田二十六萬四百三十六畝,均定其稅,使歲一輸。乃築武陽、關硤二城,詔以山地置新化縣,並二城隸邵州。自是,鼎、澧可以南
 至邵。



 誠、徽州,唐溪峒州。宋初,楊氏居之,號十峒首領,以其族姓散掌州峒。



 太平興國四年,首領楊蘊始來內附。五年,楊通寶始入貢,命為誠州刺史。淳化二年,其刺史楊政岩復來貢。是歲,政岩卒,以其子通盈繼知州事。



 熙寧八年,有楊光富者,率其族姓二十三州峒歸附,詔以光富為右班殿直,昌運五人補三班奉職,晟情等十六人補三司軍將。繼有楊昌銜者,亦願罷進奉,出租賦為漢民,
 詔補為右班殿直,子弟侄十八人補授有差。獨光僭頗負固不從命,詔湖南轉運使朱初平羈縻之,未幾亦降,乃與其子日儼請於其側建學舍,求名士教子孫。詔潭州長史樸成為徽、誠等州教授;光僭皇城使、誠州刺史致仕,官為建宅;置飛山一帶道路巡檢。光僭未及拜而卒,遂以贈之,錄其子六人。



 元豐三年,知邵州關杞請於徽、誠州融嶺鎮擇要害地築城砦,以絕邊患。詔湖南安撫謝景溫、轉運使朱初平、判官趙揚商度以聞,景溫等以
 為宜如杞言。乃議誠州以沅州貫保砦為渠陽縣隸之,以徽州為蒔竹縣隸邵州。趙揚言上江、多星、銅鼓、羊鎮、潭溪、上和、上誠、天村、大田等團並至誠州城下貿易,可漸招撫,並乞下湖南邵州蒔竹縣招諭芙蓉、萬驛諸團,從之,徒誠州治渠陽而貫保為砦如故。上江等諸團果皆納土,於是增築多星等砦,還連徽、廣西融州王口砦焉。



 元祐二年,改誠州為渠陽軍,罷兩州兵馬及守禦民丁。有楊晟台者,乘間寇文村堡,知渠陽軍胡田措置亡
 術,蠻結西融州蠻砦粟仁催,往來兩路為民患,調兵屯渠陽至萬人,湖南亦增屯兵應援,三路俱驚。朝廷方務省事,議廢堡砦,徹戍守,而以其地予蠻,乃詔湖北轉運副使李茂直招撫,又遣唐乂同措置邊事討之。後以渠陽為誠州,命光僭之子供備庫使昌達、供備庫副使楊昌等同知州事,而貫保、豐山、若水等砦皆罷戍,擇授土官,俾乂間毀樓櫓,撤官舍,護領居民入砦。崇寧初,改誠州為靖州。



 南丹州蠻,亦溪峒之別種也,地與宜州及西南夷接壤。開寶七年,酋帥莫洪[B16D]遣使陳紹規奉表求內附。九年,復來貢,求賜牌印,詔刻印以給之。太平興國五年,洪[B16D]貢銀百兩,以賀太平。



 雍熙四年,洪[B16D]族人知寶隆鎮莫淮閬牛一頭,逐水草至金城州河池縣,宜州牙校周承鑒以其牛耕作,淮閬三遣人取牛,承鑒不還,凡耕十日,始釋牛逐水草去。淮閬怒,領鄉兵六十人劫取承鑒家資財,驅縣民莫世家牛六頭以歸,誘群蠻為寇。上遣供
 奉官王承緒乘傳劾承鑒,具伏占牛,詔棄市。時知宜州、讚善大夫侯汀失於備禦,群蠻之擾,頗害及民庶,詔發諸州兵進討,兵未至,悉已遁歸,汀坐免官。詔諭宜、融、柳州百姓及蠻界人戶曰:「朕托兆庶之上,處司牧之重,照臨所暨,撫養是均,矧於遐陬,尤所軫慮。昨以知宜州事侯汀失於綏緝,恣其侵牟,致茲邊夷,起為寇鈔,侵騷閭裏,虔劉士庶。及興師而討伐,乃畏威而竄伏。朕以興戎召釁,職由於汀,爰舉國章,削其官秩。汝等所宜體予含
 垢,革乃前非,安土厚生,保境延世,嬉我至化,是為永圖。或尚恣於陸梁,當盡剿其族類。」自是不復為寇。



 淳化元年,洪[B16D]卒,其弟洪皓襲稱刺史,遣其子淮通來貢銀碗二十,銅鼓三面,銅印一鈕,旗一帖,繡真珠紅羅襦一。上降優詔,賜彩百匹,還其襦。自洪[B16D]領州十餘年,歲輸白金百兩。洪皓之襲兄位,專其地利,不修常貢。其弟洪沅忿之,挈妻子來奔宜州。洪皓怒其背己,數引兵攻洪沅。洪沅與二男並牙將一人,乘傳詣闕訴其事,請發兵致
 討。上以蠻夷之俗,羈縻而已,不欲為之興師報怨。洪沅先自稱南丹州副使,以為邵州團練使,給田十頃,下詔戒敕洪皓。



 景德二年,洪皓死,長子淮襲父任。俄為弟淮辿攻南丹州,淮帥屬來奔,詔宜州賜閑田資給之。大中詳符五年,宜州言淮辿頗集諸蠻,阻富仁監道路,上廉知淮辿無侵擾狀,遣使犒設撫勞之。九年,撫水蠻叛,詔淮辿約勒溪峒,勿從誘脅。明年,平撫水蠻,淮辿等並以勞進秩。景祐三年,有淮戟者舉族來歸,命為湖南
 州團練副使,敕州縣拊存。後淮辿老,自言願傳其子世漸。至和元年,命世漸為檢校散騎常侍,權發遣州事。明年,以淮辿為懷遠大將軍致仕,世漸為刺史、檢校工部尚書,賜袍帶,錢十萬,絹百匹。又補其親黨數十人為檢校官,如故事也。世漸死,嘉祐末,命其子公帳襲之。



 有世忍者,亦淮辿之子也。初率其屬人內附,治平初逃歸,攻殺公帳,奪其地自首,請於朝廷,願授刺史,補其親黨如故事,歲輸銀百兩。三年,遂命為刺史,皆如其請。熙寧二
 年,徭賊殺人,世忍執以獻,授檢校禮部尚書。元豐三年入貢,其印以「西南諸道武盛軍德政官家明天國主」為文,詔以南丹州印賜之,令毀其舊印。六年,大軍討安化,世忍獻弓矢,自言願世世為外臣,修貢不懈,遷檢校戶部尚書,給銅牌旗號,官其子侄九人。世忍死,子公佞襲。



 大觀元年,廣西經略使王祖道言公佞就擒。進築平、允、從州,牧文、地、蘭、那、安、外、習、南丹八州之地,並為鎮庭孚觀州、延德軍,以其弟公晟襲刺史。宣和四年,公晟乞以
 州事付其侄延豐,願與其子歸朝,詔從之,仍乘驛給券。



 紹興三年,公晟攻圍觀州,焚寶積監。朱勝非奏:「崇、觀、宣和間所開新邊,比來往往棄而不守,帥臣、監司屢言觀州為控扼之地。不宜棄。」帝曰:「前日用事之臣,貪功生事,公為欺罔,其實勞民費財,使遠俗不安也。」又用廣南經略安撫使劉彥適言,以公晟知南丹州兼溪峒都巡檢使、提舉盜賊公事,給以南丹州刺史舊印,公晟未受命。二十四年,公晟始貢馬,率諸蠻來歸。帝諭輔臣曰:「得南
 丹非為廣地也,但徭人不叛,百姓安業,為可喜耳。」遂以延沈襲公晟職,授銀青光祿大夫、檢校太子賓客、使持節南丹州諸軍事、南丹州刺史兼御史大夫、知南丹州公事、武騎尉。廣西經略安撫使呂願中諭降諸蠻三十二種,得州二十七,縣一百三十五,砦四十,峒一百七十九及一鎮、三十二團,皆為羈縻州縣。二十五年,延沈進補團練、防禦二使。三十一年,延沈恣行慘酷,為諸蠻所逐,歸死省地,眾推延廩襲職。隆興二年,延廩復為諸蠻
 所圖,攜家歸朝,經略司奏以延葚襲職。淳熙元年,南丹為永樂州所攻,使來告急,廣西帥臣遣將領陳泰權、天河縣主簿徐彌高諭和之。十四年,經略司奏以延陰襲職,詔從其請。嘉定五年,延陰之子光熙襲職,知南丹州事。



\end{pinyinscope}