\article{列傳第二百三十七世家一}

\begin{pinyinscope}

 唐自安、史之亂,藩鎮專制,百有餘年,浸成割據。及巢賊蹂躪,郡邑丘墟。降臻五季,豪傑蠭午,各挾智力,擅為封
 疆,自制位號,以爭長雄。天厭禍亂,授宋大柄。太祖命將出師,十餘年間,南平荊、楚,西取巴、蜀,劉鋹既俘,李氏納款。至於太宗,吳越請吏,漳、泉來歸,薄伐太原,遂僨北漢,而海內一矣!王偁《東都事略》用東漢隗囂、公孫述例,置孟昶、劉鋹等於列傳,舊史因之。今仿歐陽修《五代史記》,列之世家。凡諸國治亂之原,天下離合之勢,有足鑒者,悉著於篇。其子孫諸臣事業有可考者,各疏本國之下。作《列國世家》。



 =南唐李氏=



 南唐李景,本名景通,後改為璟。避周廟諱,復改為景。父昪,吳楊行密將徐溫養子,冒姓徐氏,名知誥,《五代史》有。景十餘歲,以父任駕部郎中、諸衛將軍。後唐天成二年,溫卒,遂專吳政。將出鎮,欲以國事付景,拜兵部尚書、參知政事。出鎮金陵,遷景司徒、平章事、知內外左右諸軍事。頃之,亦赴金陵,為中外諸軍副都統,受吳禪,國號大齊,改元昪元,僭帝號,居金陵。自云唐宗室建王恪之後,下令復姓李氏,國號唐。封景吳王、諸道元
 帥、錄尚書事,改封齊王。



 昪立七年卒,景襲位,改元保大,尊母宋氏為皇太后,立妻鍾氏為皇后。用宋齊丘、周宗為宰相,郊祀天地。天福末,遣其將祖思全、何洙侵福、建、漳、泉之地。漢乾祐初,李守貞以河中叛,潛遣舒元、楊訥間道求援於景。景命其將李金全、郭全義出師應之。金全以聲勢不接,初不願行,景固遣之。至沭陽,聞守貞敗,乃還。周廣順初,景又遣其將邊鎬平湖湘,尋復失之。



 顯德二年,周世宗征淮南,破景眾於正陽,遂進圍壽州。太
 祖時總禁兵,破景將何延錫於渦口,又擒皇甫暉於滁州。景大懼,遣其臣鍾謨、李德明奉表願為附庸。未幾,又遣其臣孫晟、王崇質奉表獻濠、壽、泗、楚、光、海六州之地,願罷兵,世宗未之許。



 四年春,世宗大破景軍於紫金山,降其將朱元,克壽州。冬,又克濠、泗二州。五年春,改元中興。未幾,又改元交泰。是春,周師克楚州,又進克揚州。將議濟江,景大懼,請盡割江北之地,畫江為界,稱臣於中朝,歲貢土物數十萬,世宗許之。始稟周之正朔,上表稱
 唐國主。世宗答書用唐報回鶻可汗之制,云「皇帝恭問江南國主」,臨汴水置懷信驛以待其使。景又上言世宗,請傳位於世子冀,世宗賜書勉諭之乃止。景既失淮南之地,頗躁憤,惡其大臣宋齊丘、陳覺、李徵吉,皆殺之。六年十月,冀卒,命御廚使張延範充使吊祭。



 建隆元年,太祖受命,即遣使以書諭景。初,顯德中,江南將校相繼來降,周成等三十四人皆在京師,至是遣歸。三月,景遣使貢絹二萬匹、銀萬兩,賀登極。及澤、潞平,景又貢銀五千
 兩為賀,七月還京,又貢金器五百兩、銀器三千兩、羅紈千匹、絹五千匹,又遣其禮部郎中龔慎儀貢乘輿服御物。每歲冬正、端午、長春節皆以土產珍異、金銀器用、繒帛、片茶為貢。每景及錢俶遣親屬入貢,皆御前殿曲宴以寵之。景生日,遣使賜以金幣及賜羊萬口、馬三百疋、橐駝三十,以為常制。是年,親征李重進,駐蹕廣陵,遣其左僕射嚴續來犒師。俄遣其子蔣國公從鎰朝行在所,又遣其戶部尚書馮延魯貢金買宴,並伶官五十人作
 樂上壽,又貢金銀器、金玉鞍勒、銀裝兵器及錢銀、綾絹,皆有加常數,太祖亦厚賜之。



 初,景之襲父位也,屬中原多故,盧文進、李金全、皇甫暉之徒皆奔於景。跨據江、淮三十餘州,擅魚鹽之利,即山鑄錢,物力富盛。嘗試貢士《高祖入關詩》,頗有窺覦中土之意。自世宗平淮甸,浸以衰弱。及太祖平揚州,日習馬舫戰艦於京城之南池,景懼甚。其小臣杜著頗有辭辨,偽作商人,由建安渡來歸;又彭澤令薛良坐事責授池州文學,亦挺身來奔,獻《平
 南策》,景聞之益懼。太祖命斬著於下蜀市,良配隸廬州衙校,景乃安。終以國境蹙弱,不遑寧居,遂遷於豫章。上遣通事舍人王守正持詔撫之。



 俄而景卒,其臣桂陽郡公徐邈奉遺表來上,太祖廢朝五日,遣鞍轡庫使梁義吊祭,贈賻絹三千匹。子煜又遣其臣馮謐奉表,願追尊帝號,許之。煜乃諡景為明道崇德文宣孝皇帝,廟號元宗,陵號順陵。



 煜,字重光,景第六子也,本名從嘉。少聰悟,喜讀書屬文,
 工書畫,知音律。初封安定郡公,累遷諸衛大將軍、副元帥,封鄭王。



 景始嗣位,以弟齊王景遂為元帥,居東宮,燕王景達為副元帥,就昪柩前盟約,兄弟相繼,中外庶政,並委景遂參決。景長子冀為東都留守,後又立景遂為太弟,景達為齊王、元帥,冀為燕王、副元帥。冀鎮京口,周師征淮,吳越圍常州,冀部將敗之。景達屯濠州,兵衄,遁還。及割地後,出景遂為洪州元帥,封晉王。景達撫州元帥,立冀為太子。景遂尋卒,數月冀亦卒,乃立從嘉為吳
 王。



 建隆二年,景遷洪州,立為太子監國,是秋襲位,居建康,改名煜。立母鍾氏為聖尊后,以鍾氏父名泰章故也,妻周氏為國后。遣戶部尚書馮謐來貢金器二千兩、銀器二萬兩、紗羅繒彩三萬匹。且奉表陳紹襲之意曰:



 :「臣本於諸子,實愧非才,自出膠庠,心疏利祿。被父兄之蔭育,樂日月以優遊,思追巢、許之餘塵,遠慕夷、齊之高義。繼傾懇悃,上告先君,固匪虛詞,人多知者。徒以伯仲繼沒,次第推遷,先世謂臣克習義方,既長且嫡,俾司國事,
 遽易年華。及乎暫赴豫章,留居建業,正儲副之位,分監撫之權,懼弗克堪,常深自勵。不謂掩丁艱罰,遂玷纘承,因顧肯堂,不敢滅性。然念先世君臨江表垂二十年,中間務在倦勤,將思釋負。臣亡兄文獻太子從冀將從內禪,已決宿心,而世宗敦勸既深,議言因息。及陛下顯膺帝籙,彌篤睿情,方誓子孫,仰酬臨照。則臣向於脫屣,亦匪邀名,既嗣宗枋,敢忘負荷。唯堅臣節,上奉天朝。若曰稍易初心,輒萌異志,豈獨不遵於祖禰,實當受譴於神
 明。方主一國之生靈,遐賴九天之覆燾。況陛下懷柔義廣,煦嫗仁深,必假清光,更逾曩日。遠憑帝力,下撫舊邦,克獲宴安,得從康泰。然所慮者,吳越國鄰於弊土,近似深仇,猶恐輒向封疆,或生紛擾。臣即自嚴部曲,終不先有侵漁,免結釁嫌,撓干旒扆。仍慮巧肆如簧之舌,仰成投杼之疑,曲構異端,潛行詭道。願廻鑒燭,顯諭是非,庶使遠臣得安危懇。」



 太祖詔答焉。自景畫江內附,周世宗貽書於景,至是,因煜之立,始下詔而不名。



 會昭憲太后
 葬,煜遣戶部侍郎韓熙載、太府卿田霖來貢。三年,詔煜應朝廷橫海、飛江、水鬥、懷順諸軍親屬有在江表者,悉遣令渡江。煜每聞朝廷出師克捷及嘉慶之事,必遣使犒師修貢。其大慶,即更以買宴為名,別奉珍玩為獻。吉凶大禮,皆別修貢助。煜有母妻之喪,亦遣使往吊。乾德元年,煜上表乞呼名,詔不許。二年,又詔江北,許諸州民及諸監鹽亭戶緣江采捕及過江貿易。先是,江北置榷場,禁商人渡江及百姓緣江樵采。是歲,以江南薦饑,特
 弛其禁。三年,獻銀二萬兩、金銀龍鳳茶酒器數百事。開寶四年,又以占城、闍婆、大食國所送禮物來上,又遣弟從謙奉珍寶器用金帛為貢,且買宴,其數皆倍於前。是冬,以將郊祀,又遣弟從善來貢。



 會嶺南平,煜懼,上表,遂改唐國主為江南國主,唐國印為江南國印。又上表請所賜詔呼名,許之。煜又貶損制度,下書稱教;改中書門下省為左右內史府,尚書省為司會府,御史臺為司憲府,翰林為文館,樞密院為光政院;降封諸王為國公,
 官號多所改易。五年,長春節,別貢錢三十萬,遂以為常。太祖以從善為泰寧軍節度,賜第留京師。是歲,煜又貢米麥二十萬石。雖外示畏服,修藩臣之禮,而內實繕甲募兵,潛為戰備。太祖慮其難制,令從善諭旨於煜,使來朝,煜但奉方物為貢。六年,賜米麥十萬斛,振其饑民。



 七年秋,遂詔煜赴闕,煜稱疾不奉詔。冬,乃興師致討,以宣徽南院使、義成軍節度曹彬為升州西南面行營都部署,山南東道節度潘美為都監。煜初聞大兵將舉,甚惶懼,遣
 其弟從鎰及潘慎修來買宴,貢絹二十萬匹、茶二十萬斤及金銀器用、乘輿服物等。及至,遂留於別館。王師克池州,又破其眾二萬於采石磯,擒其龍驤都虞候楊收等,獲馬三百匹。江表無戰馬,朝廷歲賜之。及是所獲,觀其印文,皆歲賜之馬也。初,將有事江表,江南進士樊若水詣闕獻策,請造浮梁以濟師。太祖遣高品石全振往荊湖造黃黑龍船數千艘,又以大艦載巨竹縆,自荊渚而下。及命曹彬等出師,乃遣八作使郝守濬等率丁匠營
 之。議者以為古未有作浮梁渡大江者,恐不能就。乃先試於石牌口,移置采石,三日而成,渡江若履平地。煜初聞朝廷作浮梁,語其臣張洎,洎對曰:「載籍已來,長江無為梁之事。」煜曰:「吾亦以為兒戲耳。」



 王師渡江,煜委兵柄於皇甫繼勳,委機事於陳喬、張洎,又以徐溫諸孫元楀等為傳詔,每軍書告急,多不時通。八年春,王師傅城下,煜猶不知。一日登城,見列柵於外,旌旗遍野,始大懼,知為近習所蔽,遂殺繼勳。召朱令贇於上江,令連巨筏
 載甲士數萬人順流而下,將斷浮梁,未至,為劉遇所破。又募勇士五千餘人謀襲官軍,皆素不習戰,以暮夜人秉一炬來攻襲北砦。宋師縱其至,擊之,殲焉。獲其將帥,悉佩印符。



 初,彬之南征也,太祖親諭之曰:「卿至彼慎勿暴略,可示以兵威,俾自歸順,不必急攻。」及彬軍圍城,又命左拾遺、知制誥李穆送從鎰還本國,諭以手詔,促其降。會潤州平,煜危迫甚,遣其臣徐鉉、周惟簡奉方物來貢,手書奏自以來,哀懇求罷兵,太祖不許。俄復遣鉉等
 入貢,仍乞緩師,又不答,但厚賜遣之。初,從鎰之還,詔諸將罷攻城,而煜終惑左右之言,猶豫不決,遂詔進兵。



 八年冬,城陷,曹彬等駐兵於宮門,煜率其近臣迎拜於門。彬等上露布,以煜並其宰相湯悅等四十五人上獻。太祖御明德樓,以煜嘗奉正朔,詔有司勿宣露布,止令煜等白衣紗帽至樓下待罪。詔並釋之,賜冠帶、器幣、鞍馬有差。下詔曰:



 :「上天之德本於好生,為君之心貴乎含垢。自亂離之云瘼,致跨據之相承,諭文告而弗賓,申吊伐
 而斯在。慶茲混一,加以寵綏。



 :江南偽主李煜,承奕世之遺基,據偏方而竊號。惟乃先父早荷朝恩,當爾襲位之初,未嘗稟命。朕方示以寬大,每為含容。雖陳內附之言,罔效駿奔之禮,聚兵峻壘,包蓄日彰。朕欲全彼始終,去其疑間,雖頒召節,亦冀來朝,庶成玉帛之儀,豈願干戈之役。蹇然弗顧,潛蓄陰謀。勞銳旅以徂征,傅孤城而問罪。洎聞危迫,累示招攜,何迷復之不悛,果覆亡之自掇。



 :昔者唐堯光宅,非無丹浦之師;夏禹泣辜,不赦防風之
 罪。稽諸古典,諒有明刑。朕以道在包荒,恩推惡殺。在昔騾車出蜀,青蓋辭吳,彼皆閏位之降君,不預中朝之正朔,及頒爵命,方列公侯。爾實為外臣,戾我恩德,比禪與皓,又非其倫。特升拱極之班,賜以列侯之號,式優待遇,盡捨尤違。可光祿大夫、檢校太傅、右千牛衛上將軍,仍封違命侯。」



 召升殿撫問。妻周氏封鄭國夫人,又以其子神武右廂都指揮使仲寓為左千牛衛大將軍,弟宣州節度使從鎰為左領軍衛大將軍,江州節度使從謙為
 右領軍衛大將軍,神武統軍從度為左監門衛大將軍,神武左廂都指揮使從信為右監門衛大將軍,侄戶部尚書仲遠為右驍衛將軍,刑部尚書仲興為右武衛將軍,禮部尚書仲偉為右屯衛將軍,宗正卿季操為左武衛將軍,殿中監仲康為右領衛將軍,殿中少監仲宣為監門衛將軍。仍賜其弟侄宅各一區。



 太宗即位,始去違命侯,加特進,封隴西郡公。太平興國二年,煜自言其貧,詔增給月奉,仍賜錢三百萬。太宗嘗幸崇文院觀書,召
 煜及劉鋹,令縱觀,謂煜曰:「聞卿在江南好讀書,此簡策多卿之舊物,歸朝來頗讀書否?」煜頓首謝。三年七月,卒,年四十二。廢朝三日,贈太師,追封吳王。



 先是,江南自後漢以來,民間有服玩侈靡者,人詢之,必對曰:「此物屬趙寶子。」又煜之妓妾嘗染碧,經夕未收,會露下,其色愈鮮明,煜愛之。自是宮中競收露水,染碧以衣之,謂之「天水碧」。及江南滅,方悟「趙」,國姓也;「寶」,年號也;「天水」,趙之望也。



 從善,字子師,偽封鄭王,累遷太尉、中書令,後降封南楚
 國公。開寶四年春,奉方物來貢,授泰寧軍節度、兗海沂等州觀察等使,留京師。時太祖平劉鋹,將召煜入朝,故授從善節制,仍賜汴陽坊甲第一區。煜手疏求遣從善歸國,優詔不許。七年,推恩將佐,以掌書記江直木為司門員外郎、同判兗州,衙內都指揮使兼左都押衙崔光習為右千牛衛將軍,衙內都虞候兼右都押衙子再興為右千牛衛中郎將,並同正。又封從善母淩氏吳國太夫人。江南平,改右神武大將軍。雍熙初,再遷右千牛衛
 上將軍,出為通許監軍。四年,卒,年四十八。



 子仲翊,大中祥符初,賜同進士出身。二年,復召試,除楚州推官,累遷殿中丞,坐事免。次子仲猷,景德中,特錄為三班借職。



 從誧,本名從謙,偽封吉王,後降封諤國公。隨煜歸朝,為右領軍衛大將軍,遷右龍武大將軍,歷知隨、復、成三州。上表改名。淳化五年,上言貧不能自給,求外任。以本官充武勝軍行軍司馬,月給奉錢三萬。



 子仲偃,大中祥符八年,舉進士。



 季操,昪從父弟偽江王逷之子也。從煜入朝,後為右神武將軍,累遷左衛大將軍,領康州刺史,出為單州都監。歷知淮陽、漣水二軍、蔡、舒二州。大中祥符四年,卒。



 仲寓,字叔章,少聰慧,能屬文,多才藝。偽封清源郡公,歸朝為千牛衛大將軍。煜卒,太宗賜仲寓積珍坊第一區、白金五千兩。仲寓宗族百餘口,猶貧不能給,上書自陳。太宗憐之,授郢州刺史。在郡迨十年,為政寬簡,部內甚治。淳化五年,卒,年三十七。



 子正言,景德三年,特補供奉
 官。早卒無嗣,唯一女孤幼,真宗愍之,賜絹百匹、錢二百萬,以備聘財,仍遣內臣主其事。



 煜有土田在常州,官為檢校。上聞其宗屬貧甚,命鬻其半,置資產以贍之。



 舒元,潁州沈丘人。少倜儻好學,與道士楊訥講習於嵩陽,通《左氏》及《公》、《穀》二傳。與訥同詣河中謁李守貞,與語奇之,俱館於門下。守貞謀叛,遣元與訥間道乞師江南。江南遣大將軍皇甫暉等率眾數萬次沭陽,為之聲援。會守貞敗,元與訥留江南。元易姓朱,楊訥更姓名為李
 平。



 元事李景,歷江寧令、駕部員外郎、文理院待詔,嘗坐事左遷。世宗征淮南,諸郡多下,元求見言兵事,景大悅,遣率兵攻舒州,復之,即以為團練使。又平歷陽,景以元為淮南北面招討使。周師圍壽春,景以其弟齊王景達為元帥,率兵來救,以陳覺為監軍,總軍政。元素與覺有隙,覺密表譖元於景,信之,立遣大將楊守忠代元。元憤怒,自以戰功高,又不忍負景,欲自殺。門下客宋洎諫曰:「大丈夫何往不取富貴,豈必為妻子死哉!」元聽之,將其
 眾歸世宗,景盡誅其妻子。世宗素知元驍果,得之甚喜,以為檢校太保、蔡州防禦使。淮南平,改濠州防禦使。



 宋初,從平李重進,改沂州防禦使。為滑州巡檢使,與節帥不協,誣奏元為同產妹婿宋玘請求。事得釋。詔元復姓舒氏。開寶五年,為白波兵馬都監。太平興國二年,卒,年五十五,特贈武泰軍節度。



 元辯捷強記,治郡日,或奏其不親獄訟,事多冤滯。太祖面詰問之,凡所詰,元必具誦款占,指述曲直,太祖甚嘉歎之。子知白、知雄、知崇。



 知白
 至作坊使。知雄初補殿直,雷有終薦授供奉官、鄜延路駐泊都監,後辭疾居嵩山。知白嘗奏事太宗,語及之,即召出,授西京作坊副使、泉福都巡檢使。真宗初,懇請入道,歸嵩陽舊隱。復為王嗣宗、李元則所薦,授供備庫使,歷知棣州、麟府鄜延鈐轄,又知虔州。復求入道,面賜紫冠服,號崇玄大師。嘗獻《字母圖》,有詔褒獎。乾興元年,卒,年八十一。知崇累歷內職,至供備庫使。嘗為廣州鈐轄、河北安撫副使,卒。



 知白子昭遠,大中祥符五年,任大理
 評事,因對自陳,改大理寺丞,賜進士第,至太常博士。



 韓熙載,字叔言,濰州北海人。後唐同光中,舉進士,名聞京、洛。父光嗣,為平盧軍節度副使。同光末,青州軍亂,逐其帥符習,推光嗣為留後。明宗即位,誅光嗣,熙載奔江南,歷偽吳滁、和、常三州從事。



 李昪僭號,為秘書郎,令事其子景於東宮。景嗣位,遷虞部員外郎、史館修撰。熙載自言:「受知遇,不得顯位,是以我屬嗣君也。」遂上章,言事切直,景嘉納之。又改吉凶儀禮不如式者十數事,大
 為宋齊丘、馮延己所忌。



 昪將葬,以熙載知禮,令兼太常博士。時江左草創,典禮多闕,議者以繼唐昭宗之後,廟號合稱宗。熙載建議,以為古者帝王己失之,己得之,謂之反正;非我失之,自我復之,謂之中興,中興之君廟號稱祖。以為興既墜之業,請號「烈祖」。景由是益加恩禮,擢知制誥。熙載性懶慢,朝直多闕,未幾罷去。



 晉開運末,中原多事,江南方盛,其臣陳覺、馮延魯建討福州,師敗而還,景釋不問罪。熙載與徐鉉同上疏,請置於法。覺、
 延魯,宋齊丘之黨也。熙載為齊丘所排,貶和州司馬,語在《徐鉉傳》。久之,召為虞部郎中、史館修撰,拜中書舍人。世宗平淮甸,景患國用不足,熙載請鑄鐵錢。及煜襲位,卒行其議,以熙載為兵部尚書,充鑄錢使。錢貨益輕,不勝其弊,熙載頗亦自悔。



 熙載善為文,江東士人、道釋載金帛以求銘誌碑記者不絕,又累獲賞賜。由是畜妓妾四十餘人,多善音樂,不加防閑,恣其出入外齋,與賓客生徒雜處。煜以其盡忠言事,垂欲相之,終以帷薄不修,
 責授右庶子,分司洪州。熙載盡斥諸妓,單車即路,煜留之,改秘書監,俄而復位。向所斥之妓稍稍而集,頃之如故。煜歎曰:「吾亦無如之何!」遷中書侍郎、光政殿學士承旨。,卒,年六十。煜痛惜之,贈左僕射、平章事,諡「文靖」,葬於梅嶺岡謝安墓側,命徐鍇集其遺文。



 熙載才氣俊逸,機用周敏,性高簡,無所卑屈,未嘗拜人。雖被遣逐,終不改節,江左號為「韓夫子」。顯德中,熙載來朝廷,歸,景問中國大臣,時太祖方典禁兵,熙載對曰:「趙點檢顧
 視不常,不可測也。」及太祖登極,景益重之。頗以文章自負,好大言。初,年,五星連珠於奎,奎主文章,又在魯分,時太宗鎮兗、海,中國太平之符也。是歲,熙載著《格言》五卷,自序其事云:「魯無其應,韓子《格言》成之。」人多笑之。



 馮謐,本名延魯,字叔文,其先彭城人,唐末南渡,家於新安。李僭號,立子景為太子,謐與兄延己俱以文學得幸。及景嗣位,累遷至中書舍人。



 晉開運末,閩越大亂,景
 遣謐與諫議大夫陳覺乘傳安撫,謐遂矯詔發數郡兵攻福州。及敗,引佩刀自刺,親吏制之,不死,長流舒州。會赦敘用,復為中書舍人,改工部侍郎。江南以揚州為東都,命謐副留守。周世宗下揚州,謐髡發為僧,匿於佛寺,為官軍所獲。世宗釋之,授太常卿,賜與甚厚。數年,拜刑部侍郎,放還,為戶部尚書。建隆三年,煜遣來貢,因表求舒州田宅,詔賜之。後改常州觀察使而卒。



 子伉歸中朝,與兄儀、價並登進士第。伉文辭清麗,嘗著《平晉頌》,時人
 稱之。累遷殿中侍御史,歷典藩郡,皆有治跡。咸平三年,知福州,卒。特賜錢十萬,錄其子玄應同學究出身。



 潘祐,南唐散騎常侍處常之子。少介僻,杜門讀書,不交人事。及長,善屬文,尤長於論議。陳喬、韓熙載、徐鉉等共薦於景,為秘書省正字、直崇文館。煜襲位,遷虞部員外郎、史館修撰。未幾,知制誥,為內史舍人。



 有李平者,本嵩山道士楊訥,依河中帥李守貞。漢乾祐中,守貞反,遣訥與舒元乞師江南。守貞敗,訥遂易姓名,江南以為員
 外郎,遷衛尉少卿、蘄州刺史、戶部侍郎。平好神仙修養之事,動作妖妄,自言常與神接。祐亦好神仙,遂相善。二家皆置淨室,圖神像,常被髮裸袒處室中,家人亦不得至。又嘗建議復井田,及依《周禮》置牛籍,薦平判司農寺以督之。事行,百姓大撓,未幾而罷。祐自以為眾所排,因憤怒,歷詆大臣與握兵者兩為朋比,將謀反叛;又言國將亡,非己為相不可救。江南政事多在尚書省,因薦平知省事,又薦星官楊熙澄為樞密使,小校侯英典禁兵,煜
 不納。祐益忿,抗疏請誅宰相湯悅等數十人,煜手書教戒之。祐不復朝謁,乃於家上書曰:「臣聞『三軍可奪帥也,匹夫不可奪志也』。近者連上表章指陳姦惡,何面目以見士人乎?」遂自縊死。



 皇甫繼勳,江州節度使暉之子。幼以父蔭為軍校,父死難於滁州,累遷將軍,池、饒二州刺史,勤於吏事。入為諸軍都虞候,遷神衛統軍都指揮使。諸老將相次皆死,而繼勳尚少,遂為大將。貲產優贍,營第舍、車服,畜妓樂,潔
 飲食,極遊宴之好。



 及宋師至,諸軍多敗衄,繼勳欲煜之速降,每眾中流言,頗道國中蹙弱。侄紹傑亦以繼勳故,為巡檢。常令紹傑入見煜,陳歸命之計。會有風雹,繼勳又密陳滅亡之兆。偏裨或有募勇士欲夜出營邀宋師者,輒鞭而拘之。又因請出煜親兵千餘守闕城,為宋師所掩。一日,煜躬自巡城,見宋師列柵城外,旌旗遍野,始驚懼,知為左右所蔽。及巡城還,繼勳從至宮。煜乃責其流言惑眾及不用命之狀,收付大理。始出,軍士悉集,臠
 割其肉,頃刻都盡。紹傑亦被誅。煜皆赦其妻子。



 周惟簡,饒州鄱陽人。隱居,好學問,明《易》義。煜召為國子博士、集賢侍講。頃之,以虞部郎中致仕。宋師圍金陵,煜求能使交兵者,張洎薦惟簡有遠略,可以談笑和解之。召為給事中,與徐鉉奉使至京師。太祖召見詰責,惟簡惶恐,反言曰:「臣本居山野,無仕進之意,李煜強遣來耳。臣素聞終南山多靈藥,事寧後,願得棲隱。」太祖許之。江南平,以惟簡為國子《周易》博士、判監事。開寶九年,上書
 述前志,求解官,蓋不得已,非其心也。改虞部郎中,致仕。以其子繕為京兆府鄠縣主簿,俾就養。



 太平興國初,惟簡自終南至闕下,求入見。有司以致仕官非有詔召無求對之制,乃還。歲餘,復上表自求用,除太常博士,遷水部員外郎,卒。繕後舉進士,至都官員外郎。



\end{pinyinscope}