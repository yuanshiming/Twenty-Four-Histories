\article{列傳第一百三十四}

\begin{pinyinscope}

 鄧
 肅李邴滕康張守富直柔馮康國



 鄧肅,字志宏,南劍沙縣人。少警敏能文,美風儀,善談論。李綱見而奇之,相倡和,為忘年交。居父喪,哀毀逾禮,芝
 產其廬。入太學,所與游皆天下名士。時東南貢花石綱,肅作詩十一章,言守令搜求擾民,用事者見之,屏出學。



 欽宗嗣位,召對便殿,補承務郎,授鴻臚寺簿。金人犯闕,肅被命詣敵營,留五十日而還。張邦昌僭位,肅義不屈,奔赴南京,擢左正言。



 先是,朝廷賜金國帛一千萬,肅在其營,密覘,均與將士之數,大約不過八萬人,至是為上言之,且言:「金人不足畏,但其信賞必罰,不假文字,故人各用命。朝廷則不然,有同時立功而功又相等者,或已
 轉數官,或尚為布衣,輕重上下,只在吏手。賞既不明,誰肯自勸?欲望專立功賞一司,使凡立功者得以自陳。若功狀已明而賞不行,或功同而賞有輕重先後者,並置之法。」上從之。



 朝臣受偽命者眾,肅請分三等定罪。上以肅在圍城中,知其姓名,令具奏。肅言:「叛臣之上者,其惡有五:諸侍從而為執政者,王時雍、徐秉哲、吳開、呂好問、莫儔、李回是也;諸庶官及宮觀而起為侍從者,胡思、朱宗、周懿文、盧襄、李擢、範宗尹是也;撰勸進文與赦書者,
 顏博文、王紹是也;朝臣之為事務官者,私結十友講冊立邦昌之儀者是也;因張邦昌改名者,何昌言改為善言、其弟昌辰改為知辰是也。乞置之嶺外。所謂叛臣之次者,其惡有三:諸執政、侍從、臺諫稱臣於偽庭,執政馮澥、曹輔是也,侍從者已行遣,獨李會尚為中書舍人,臺諫中有為金人根括而被杖,一以病得免者,其餘無不在偽楚之庭;以庶官而升擢者,不可勝數,乞委留守司按籍考之,則無有遺者;願為奉使者,黎確、李健、陳戩是
 也,乞於遠小處編管。若夫庶官在位供職不廢者,但茍祿而已,乞赦其罪而錄其名,不復用為臺諫、侍從。」上以為然。



 耿南仲得祠祿歸,其子延禧為郡守,肅劾:「南仲父子同惡,沮渡河之戰,遏勤王之兵,今日割三鎮,明日截兩河。及陛下欲進援京城,又為南仲父子所沮。誤國如此,乞正典刑。」南仲嘗薦肅於欽宗,肅言之不恤,上嘉其直,賜五品服。



 範訥留守東京,肅言:「訥出師兩河,望風先遁,今語人曰:『留守之說有四,戰、守、降、走而已。戰無卒,守
 無糧,不降則走。』且漢得人傑,乃守關中,奔軍之將,豈宜與此。」訥遂罷。內侍陳良弼肩輿至橫門外,開封買入內女童,肅連章論之。時官吏多托故而去,肅建議削其仕版,而取其祿以給禁衛,若夫先假指揮徑徙江湖者,乞追付有司以正其罪。



 因入對,言:「外夷之巧在文書簡,簡故速;中國之患在文書煩,煩故遲。」上曰:「正此討論,故並三省盡依祖宗法。」及建局討論祖宗官制,兩月不見施行,肅言:「太祖、太宗之時,法嚴而令速,事簡而官清,未嘗
 旁搜曲引以稽賞罰,故能以十萬精兵混一六合。自時厥後,群臣無可議者,今日獻一策,明日獻一言,煩冗瑣碎,惟恐不備,此文書所以益煩,而政事所以益緩也。今兵戈未息,豈可揖遜進退,尚循無事之時?欲乞限以旬日,期於必至,庶幾法嚴事簡,賞罰之權不至濡滯。」肅在諫垣,遇事感激,不三月凡抗二十疏,言皆切至,上多採納。



 會李綱罷,肅奏曰:「綱學雖正而術疏,謀雖深而機淺,固不足以副聖意。惟陛下嘗顧臣曰:『李綱真以身徇國
 者。』今日罷之,而責詞甚嚴,此臣所以有疑也。且兩河百姓無所適從,綱措置不一月間,民兵稍集,今綱既去,兩河之民將如何哉?偽楚之臣紛紛在朝,李綱先乞逐逆臣邦昌,然後叛黨稍能正罪,今綱既去,叛臣將如何哉?叛臣在朝,政事乖矣,兩河無兵,外夷驕矣,李綱於此,亦不可謂無一日之長。」執政怒,送肅吏部,罷歸居家。紹興二年,避寇福唐,以疾卒。



 李邴,字漢老,濟州任城縣人。中崇寧五年進士第,累官
 為起居舍人,試中書舍人。北方用兵,酬功第賞,日數十百,邴辭命無留難。除給事中、同修國史兼直學士院,遷翰林學士。嘗與禁中曲宴,徽宗命賦詩,高麗使入貢,邴為館伴,徽宗遣中使持示,使者請傳錄以歸。未幾,坐言者罷,提舉南京鴻慶宮。



 欽宗即位,除徽猷閣待制、知越州。久之,再落職,提舉西京嵩山崇福宮。高宗即位,復徽猷閣待制。逾歲,召為兵部侍郎兼直學士院。



 苗傅、劉正彥迫上遜位,上顧邴草詔,邴請得御札而後敢作。朱勝
 非請降詔赦,邴就都堂草之。除翰林學士。初,邴見苗傅,面諭以逆順禍福之理,且密勸殿帥王元俾以禁旅擊賊,元唯唯不能用,即詣政事堂白朱勝非,適正彥及其黨王世修在焉,又以大義責之,人為之危,邴不顧也。時御史中丞鄭□又抗疏言睿聖皇帝不當改號,於是邴、□為端明殿學士、同簽書樞密院事。邴與張守分草百官章奏,三奏三答,及太后手詔與復闢赦文,一日而具。



 四月,拜尚書右丞,未幾,改參知政事。上巡江寧,太后六
 宮往豫章,命邴為資政殿學士、權知行臺三省樞密院事。以與呂頤浩論不合,乞罷,遂以本職提舉杭州洞霄宮。未閱月,起知平江府。會兄鄴失守越州,坐累落職。明年,即引赦復之,又升資政殿學士。



 紹興五年,詔問宰執方略,邴條上戰陣、守備、措畫、綏懷各五事。



 戰陣之利五,曰出輕兵、務遠略、儲將帥、責成功、重賞格,大略謂:「關陜為進取之地,淮南為保固之地。關陜雖利於進取,然不用師於京東以牽制其勢,則彼得一力以拒我。今大將
 統兵者數人,皆所恃以為根本,萬一失利,將不可復用。偏裨中如牛皋、王進、楊珪、史康民皆京東土人,知地險易,可各配以部曲三五千人,或出淮陽,或出徐、泗,彼將奔命之不暇,此不動而分陜西重兵之一端也。關陜今雖有二宣撫,其體尚輕,非遣大臣不可。呂頤浩氣節高亮,李綱識量宏遠,威名素著,願擇其一而用之,必有以報陛下。」又言:「陛下即位之初,韓世忠、劉光世、張俊威名隱然為大將,今又有吳玠、岳飛者出矣。願詔大將,於所
 部舉智謀忠勇可以馭眾統師各兩三人,朝廷籍記。遇有事宜,使當一隊,毋隸大將,則諸人競奮才智,皆飛、玠之儔矣。大將爵位已崇,難相統一,自今用兵,第可授以成算,使自為戰而已,慎勿遣重臣臨之,以輕其權而分其功。今卻敵退師之後,必論功行賞,願因此詔有司預定賞格,謂如得城邑及近上首領之類,自一命至節度使,皆差次使足相當。」



 所謂守備之宜有五,曰固根本、習舟師、防他道、講遺策、列長戍,大略謂:「江、浙為今日根本,
 欲保守則失進取之利,欲進取則慮根本之傷。古之名將,內必屯田以自足,外必因糧於敵。誠能得以功名自任如祖逖者,舉淮南而付之,使自為進取,而不至虛內以事外。臣聞朝廷下福建造海船七百隻,必如期而辦,乞仿古制,建伏波、下瀨、樓船之官,以教習水戰,俾近上將佐領之,自成一軍,而專隸於朝廷。無事則散之緣江州郡,緩急則聚而用之。臣度敵人他年入寇,懲創今日之敗,必先以一軍來自淮甸,為築室反耕之計,以綴我
 師。然後由登、萊泛海窺吳、越,以出吾左,由武昌渡江窺江、池,以出吾右,一處不支則大事去矣。願預講左支右吾之策。夫兵之形無窮,願詔臨江守臣,凡可設奇以誤敵者,如吳人疑城之類,皆預為措畫。今長江之險,綿數千里,守備非一,茍制得其要,則用力少而見功多。願差次其最緊處,屯軍若干人,一將領之,聽其郡守節制,次緊稍緩處差降焉,有事則以大將兼統之。既久則諳熟風土,緩急可用,與旋發之師不侔矣。」



 所謂措畫之方有
 五,曰親大閱、補禁衛、講軍制、訂使事、降敕榜,大略謂:「因秋冬之交,闢廣場,會諸將,取士卒才藝絕特者而爵賞之。建炎以來,禁衛單寡,乃藉五軍以為重,臣常寒心。願擇忠實嚴重之將以為殿帥,稍補禁衛之闕,使隱然自成一軍,則其馭諸將也,若臂之使指矣。今諸郡廂禁冗占私役者,大郡二三千人,小郡亦數百人。臣願講求,除郡守兵將官自禁軍給事外,餘兼從衣糧使自僦人以役。大抵殺廂軍三分之二,而以其衣糧之數盡募禁軍。
 金人自用兵以來,未嘗不以和好為言,此決不可恃。然二聖在彼,不可遂已,姑以餘力行之耳。臣謂宜專命一官,如古所謂行人者,或止左右司領之,當遣使人,舉成法而授之,庶免臨時斟酌之勞,而朝廷得以專意治兵矣。劉豫僭叛,理必滅之,謂宜降敕榜,明著豫僭逆之罪,曉諭江北士民,此亦兵家所謂伐謀伐交者。」



 所謂綏懷之略有五,曰宣德意、先振恤,通關津、選材能、務寬貸,大略謂:「山東大姓結為山砦以自保,今雖累年,勢必有未
 下者。願募有心力之人,密往詔諭。應淮北遺民來歸者,令淮南州郡給以行由,差船津濟,量差地分人護送,毋得邀阻。有官人先次注授差遣,無官而貧乏者,令沿江州郡以官舍居之,仍量給錢米三兩月,其能自營為生乃止。內有才智可用之人,隨宜任使,勿但縻以爵秩而已。凡諸將行師入境,敢抗拒者,固在剿戮。其有善良、老弱之人,皆從寬貸,使之有更生之望。」不報。



 邴閑居十有七年,薨於泉州,年六十二,謚文敏。有《草堂集》一百卷。



 滕康,字子濟,應天府宋城人。登崇寧五年進士第,又中詞學兼茂科,除秘書省正字,遷著作佐郎、尚書工部禮部員外郎、國子司業。



 靖康二年,元帥府聞康習憲章,召至濟州。康率群臣勸進,除太常少卿,使定登極禮儀。凡告天及肆赦之文,皆康為之,辭意激切,聞者感動。除起居舍人、權給事中,進起居郎兼討論祖宗法度檢討官,試中書舍人。



 會顯謨閣學士孟忠厚乞用父任減年遷官,康言:「忠厚,隆祐太后之侄也,太宗以來,凡母后兄弟之
 子無為侍從者。」武義大夫康義用登極恩,遷遙郡刺史,康又封還詞頭,言:「恩例遷官一等,謂於階官上進一階。今康義得特旨轉一官,自武義大夫躐上遙郡刺史,名為遷一官,實升五等,紊法之甚也。自古召亂之源,非外戚撓法,則內侍干政,漢、唐可鑒。」凡再降旨,竟不肯行。



 後軍統制韓世忠以不能戢所部,坐贖金。康言:「世忠無赫赫功,祗緣捕盜微勞,遂亞節鉞。今其所部卒伍至奪御器,逼諫臣於死地,乃止罰金,何以懲後?」詔降世忠一
 官。



 知江州陳彥文用劉光世奏,錄其守城功,遷龍圖閣待制。康以光世所上彥文功狀前後抵牾,閣而未下。宰相力主彥文趣康行詞,康論不已,宰相銜之。會布衣省試卷子不合式,康以其文取之,諫官李處遁論奏,遂以集英殿修撰提舉杭州洞霄宮。



 未幾,移蹕錢塘,再除中書舍人,奏曰:「去歲郊禮前日食,而日官不以聞,廷臣不以告,使陛下所以應天者未至,故逆臣敢萌不軌者,無先事之戒也。陛下即位,行再歲矣,側怛愛民之政徒為
 空言,而百姓不被其恩;哀痛責躬之詔不著事實,四方不以為信。忠佞並馳,而多士解體;刑賞失當,而三軍沮氣。臣願陛下取建炎初元以來所下詔書,所舉政事,熟思審度,得無一二不類臣言者乎?望參稽得失而罷行之。」上再三褒諭,稱其有諫臣風。除左諫議大夫。旬日間,封章屢上,遂擢翰林學士。翌日,除端明殿學士、同簽書樞密院事。



 建炎三年,宰相呂頤浩議幸武昌為趨陜之計,既移蹕建康,又議欲盡棄中原,徙居民於東南。康力
 持不可,上悟而止。未幾,上請太后奉神主如江西,以參知政事李邴權知三省樞密院事,康為資政殿學士,同從衛以行。邴辭疾,又命康權知,以劉玨為貳。賜康褒詔,許綴宰執班奏事。



 康從衛至洪州,劉光世護江不密,金人絕而渡,康等倉卒奉太后趨虔州。殿中侍御史張延壽論康與玨無憂國之心,至使太后涉險,為敵人追迫,責授康秘書少監,分司南京,永州居住。未幾,許自便,復左朝請大夫,提舉明道宮。紹興二年九月卒,年四十八。
 八年,追復龍圖閣學士。有文集二十卷。



 張守,字子固,常州晉陵人。家貧無書,從人假借,過目輒不忘。登崇寧元年進士第,中詞學兼茂科。除詳定《九域圖志》編修官。以省員罷,改宣德郎,擢為監察御史。丁內艱去。



 建炎元年冬,召還,改官,賜五品服。上在維揚,粘罕將自東平歷泗、淮以窺行在,宰臣汪伯彥、黃潛善以為李成餘黨不足畏,上召百官各言所見。葉夢得請上南巡,阻江為守,張俊亦奏敵勢方張,宜且南渡。守獨抗疏,
 上防淮渡江利害六事,又別疏言金人犯淮甸之路有四,宜擇四路帥守繕兵儲粟以捍禦之。疏再上,又請詔大臣惟以選將治兵為急,凡不急之務,付之都司、六曹。二相滋不悅,遂建議遣守撫諭京城,守聞命即就道。



 三年正月,還,奏金人必來,願早為之圖,上惻然。除起居郎兼直學士院。金人果渡淮,上幸臨安。遷御史中丞。



 苗、劉既平,詔赦百官,表奏皆守與李邴分為之。守論宰相朱勝非不能思患預防,致賊猖獗,乞罷政,疏留中不出,既
 而勝非竟罷政。



 呂頤浩初相,舉行司馬光之言,欲並合三省,詔侍從、臺諫集議。守言光之所奏,較然可行,若更集眾,徙為紛紜。既而悉無異論,竟合三省為一。



 上幸建康,呂頤浩、張浚葉議將奉上幸武昌為趨陜之計。時方拜浚為宣撫處置使,身任陜、蜀,守與諫議大夫滕康皆持不可,曰:「東南今日根本也,陛下遠適,則奸雄生窺伺之心。況將士多陜西人,以蜀近關陜,可圖西歸,自為計耳,非為陛下與國家計也。」守又陳十害,至殿廬謂康曰:「
 幸蜀之事,吾曹當以死爭之。」上曰:「朕固以為難行。」議遂寢。



 六月,久雨恆陰,呂頤浩、張浚皆謝罪求去,詔郎官以上言闕政。初,守為副端時嘗上疏曰:「陛下處宮室之安,則思二帝、母後穹廬毳幕之居;享膳羞之奉,則思二帝、母後膻肉酪漿之味;服細暖之衣,則思二帝、母後窮邊絕塞之寒苦;操與奪之柄,則思二帝、母後語言動作受制於人;享嬪御之適,則思二帝、母後誰為之使令;對臣下之朝,則思二帝、母後誰為之尊禮。思之又思,兢兢慄
 慄,聖心不倦,而天不為之助順者,萬無是理也。」至是復申前說,曰:「今罪己之詔數下,而天未悔禍,實有所未至耳。」且曰:「天時人事至此極矣,陛下睹今日之勢與去年孰愈?而朝廷之措置施設,與前日未始異也。俟其如維揚之變而後言之,則雖斥逐大臣,無救於禍。漢制災異策免三公,今任宰相者,雖有勛勞,然其器識不足以斡旋機務。願更擇文武全材、海內所共推者,親擢而並用之。上書論事,或有切直,宜加褒擢以來言路。」



 先是,守嘗
 論呂頤浩不可獨任,張浚不可西去,與上意異,乞補外。除禮部侍郎,不拜,上命呂頤浩至政事堂,諭以正人端士不宜輕去,守始受命。殿中侍御史趙鼎入對,論守無故下遷,上曰:「以其資淺。」鼎曰:「言事官無他過,願陛下毋沮其氣。」於是遷翰林學士、知制誥。九月,拜端明殿學士、同簽書樞密院事。扈從由海道至永嘉,回至會稽。



 四年五月,除參知政事,守嘗薦汪伯彥,沈與求劾其短,以資政殿學士提舉洞霄宮。未幾,知紹興府。尋以內祠兼侍
 讀,守力辭,改知福州。時右司員外郎張宗臣請令福建築城,守奏:「福州城於晉太康三年,偽閩增廣至六千七百餘步,國初削平已久,公私困弊,請俟他年。」遂止。尋以變易度牒錢百萬餘緡輸之行在,助國用。



 時劉豫導金人寇淮,上次平江,諸將獻俘者相踵,守聞之,上疏曰:「今以獻俘誠皆金人,或借諸國,則戮之可也。至如兩河、山東之民,皆陛下赤子,驅迫以來,豈得已哉?且諭以恩信,貸之使歸,願留者亦聽,則賊兵可不戰而潰。」金人既遁,
 詔諸將渡江追擊,守復上疏,以敵情難測,願留劉光世控御諸渡。



 上既還臨安,又詔問守以攻戰之利、守備之宜、綏懷之略、措置之方,守言:



 明詔四事,臣以為莫急於措置,措置茍當,則餘不足為陛下道矣。臣請言措置之大略,其一措置軍旅,其二措置糧食。



 神武中軍當專衛行在,而以餘軍分戍三路,一軍駐於淮東,一軍駐於淮西,一軍駐鄂、岳或荊南,擇要害之處以處之。使北至關輔,西抵川、陜,血脈相通,號令相聞,有唇齒輔車之勢,則
 自江而南可奠枕而臥也。然今之大將皆握重兵,貴極富溢,前無祿利之望,退無誅罰之憂,故朝廷之勢日削,兵將之權日重。而又為大將者,萬一有稱病而賜罷,或卒然不諱,則所統之眾將安屬耶?臣謂宜拔擢麾下之將,使為統制,每將不過五千人,棋布四路,朝廷號令徑達其軍,分合使令悉由朝廷,可以有為也。



 何謂措置軍食?諸軍既分屯諸路,則所患者財穀轉輸也。祖宗以來,每歲上供六百餘萬,出於東南轉輸,未嘗以為病也。今
 宜舉兩浙之粟以餉淮東,江西之粟以餉淮西,荊湖之粟以餉鄂、岳、荊南。量所用之數,責漕臣將輸,而歸其餘於行在,錢帛亦然,恐未至於不足也。錢糧無乏絕之患,然後戒飭諸將,不得侵擾州縣,以復業之民戶口多寡,為諸將殿最,歲核實而黜陟之。如是措置既定,俟至防秋,復遣大臣為之統督,使諸路之兵首尾相應,綏懷之略亦在是矣。究其本原,則在陛下內修德而外修政耳。



 閩自範汝為之擾,公私赤立,守在鎮四年,撫綏凋瘵,且
 請於朝,蠲除福州所貸常平緡錢十五萬。累請去郡,以提舉萬壽觀兼侍讀召還,甫兩月,復引病丐去,知平江府,力丐祠以歸。



 六年十二月,召見,即日除參知政事,明日兼權樞密院事。七年,張浚罷劉光世兵柄,而欲以呂祉往淮西撫諭諸軍,守以為不可,浚不從,守曰:「必曰改圖,亦須得聞望素高、能服諸將之心者乃可。」浚不聽,遂有酈瓊之變。及臺諫父章論浚,御批安置嶺表,赴鼎不即行,守力解上曰:「浚為陛下捍兩淮,罷劉光世,正以其
 眾烏合不為用,今其驗矣,群臣從而媒薛其短,臣恐後之繼者,必以浚為鑒,誰肯為陛下任事乎?」浚謫永州,守亦引咎請去,弗許。



 八年正月,上自建康將還臨安,守言:「建康自六朝為帝王都,江流險闊,氣象雄偉,且據都會以經理中原,依險阻以捍禦強敵,可為別都以圖恢復。」鼎持不可,守力求去,以資政殿大學士知婺州,尋改洪州,兼江南西路安撫使。入對,時江西盜賊未息,上問以弭盜之策,守曰:「莫先德政,伺其不悛,然後加之以兵。」因
 請出師屯要害。既至部,揭榜郡邑,開諭禍福,約以期限,許之自新,不數月盜平。



 後徙知紹興府。會朝廷遣三使者括諸路財賦,所至以鞭撻立威,韓球在會稽,所斂五十餘萬緡。守既視事,即求入覲,為上言之,詔追還三使。時秦檜當國,不悅,守亦不自安,復奉祠。



 建康謀帥,上曰:「建康重地,用大臣有德望者,惟張守可。」至鎮數月薨。



 守嘗薦秦檜於時宰張浚,及檜為樞密使,同朝。一日,守在省閣執浚手曰:「守前者誤公矣。今同班列,與之朝夕相
 處,觀其趨向,有患失之心,公宜力陳於上。」守在江右,以郡縣供億科擾,上疏請蠲和買,罷和糴。上欲行之,時秦檜方損度支為月進,且日憂四方財用之不至,見守疏,怒曰:「張帥何損國如是?」守聞之,嘆曰:「彼謂損國,乃益國也。」卒謚文靖。孫抑,戶部侍郎。



 富直柔,字季申,宰相弼之孫也。以父任補官。少敏悟,有才名。靖康初,晁說之奇其文,薦於朝,召賜同進士出身,除秘書省正字。



 建炎二年,召近臣舉所知,禮部侍郎張
 浚以直柔應。詔授著作佐郎,尋除禮部員外郎、起居舍人,遷右諫議大夫。範致虛自謫籍中召入,直柔力言致虛不當復用,出知鼎州。



 遷給事中。醫官、團練使王繼先以覃恩轉防禦使,法當回授,得旨特與換武功大夫。直柔論:「繼先以計換授,既授之後,轉行官資,除授差遣,更無所礙。且武功大夫惟有戰功、歷邊任、負材武者乃遷,不可以輕授。」上謂宰相範宗尹曰:「此除出自朕意。今直柔抗論,朕屈意從之,以伸直言之氣。」



 四年,遷御史中丞。
 直柔請罷右司侯延慶,而以蘇遲代之,上曰:「臺諫以拾遺補過為職,不當薦某人為某官。」於是延慶改禮部員外郎,而遲為太常少卿。



 十月,除端明殿學士、簽書樞密院事。故事,簽書有以員外郎為之,而無三丞為之者。中書言非舊典,時直柔為奉議郎,乃特遷朝奉郎。自是寄祿官三丞除二府者,遷員外郎,自直柔始,遂為例。



 紹興元年,詔禮部太常寺討論隆祐太后冊禮,範宗尹曰:「太母前後廢斥,實出章惇、蔡京,人皆知非二聖之過。」直柔
 曰:「陛下推崇隆祐,天下以為當,然人亦不以為非哲廟與上皇意,願陛下勿復致疑。」乃命禮官討論典禮。既而王居正言:「太后隆名定位,已正於元符,宜用欽聖詔,奏告天地宗廟,其典禮不須討論。」議遂定。



 上虞縣丞婁寅亮上書言宗社大計,欲選太祖諸孫「伯」字行下有賢德者視秩親王,使牧九州,以待皇嗣之生,退處藩服。疏入,上大嘆悟,直柔從而薦之,召赴行在,除監察御史。於是孝宗立為普安郡王,以寅亮之言也。



 除同知樞密院事。
 侍御史沈與求論直柔附會辛道宗、永宗兄弟得進,並論其所薦右司諫韓璜。先是,直柔嘗短呂頤浩於上前,頤浩與秦檜皆忌之,由是二人俱罷,礌責監潯州酒稅,而直柔以本官提舉洞霄宮。



 六年,丁所生母憂。起復資政殿學士、知鎮江府,辭不赴。起知衢州。以失入死罪,落職奉祠。尋復端明殿學士。徜徉山澤,放意吟詠,與蘇遲、葉夢得諸人游,以壽終於家。



 馮康國,字符通,本名轓,遂寧府人。為太學生,負氣節。建
 炎中,高宗次杭州,禮部侍郎張浚以御營參贊軍事留平江。苗、劉作亂,浚外倡帥諸將合兵致討,念傅等居中,欲得辯士往說之。時轓客浚所,慷慨請行,浚遣之至杭,說傅、正彥曰:「自古宦官亂政,根株相連,若誅鋤必受禍。今二公一旦為國家去數十年之患,天下蒙福甚大。然主上春秋鼎盛,天下不聞其過,豈可遽傳位於襁褓之子?且前日名為傳位,其實廢立,二公本心為國,奈何以此負謗天下?」傅按劍大怒,轓辭氣不屈。正彥乃善諭之
 曰:「張侍郎欲復闢固善,然須用面議。」乃遣轓還,約浚至杭。



 浚復遣轓移書傅等,告以禍福使改。既又復傅書,誦言其罪。轓至,傅黨馬柔吉訹之曰:「昨張侍郎書不委曲,二公大怒,已發兵出杭矣,君尚敢來耶?」轓曰:「畏則不來,來則不畏。」王世修欲拘留轓,會浚謬為書遺轓云:「適有客自杭來,方知二公於社稷初無不利之心,甚悔前書之輕易也。」傅等見之喜,轓得免。



 俄勤王之兵大集,傅等始懼,轓知其可動,乃說宰相朱勝非,以今日之事,當以
 淵聖皇帝為主,睿聖皇帝宜復為大元帥,少主為皇太侄,太后垂簾。勝非令與傅、正彥議,皆許諾。轓又請褒傅、正彥如趙普故事,遂皆賜鐵券。詔補轓奉議郎、守兵部員外郎,賜五品服,更名康國。



 高宗反正,以張浚宣撫川、陜,浚闢康國主管機宜文字。浚至蜀,遣康國入奏事,詔進兩官,為荊湖宣諭使。康國之行也,上幸浙東,不暇降詔旨,康國以自意為之,言者劾以擅造制書,坐貶秩二等。紹興三年,浚召還,與康國俱赴行在。浚既黜,御史常
 同因論康國,罷之。起知萬州、湖北轉運判官。



 浚相,入為都官員外郎。康國言:「四川稅色,祖宗以來,正稅重者科折輕,正稅輕者科折重,科折權衡與稅平準,故無偏重。近年監司總漕悉改舊法,取數務多,失業逃亡皆由於此。盍從舊法。」詔以其言下四川憲司察不如法者。又言:「蜀苦陸運,當諭吳玠,非防秋月,分兵就糧;兼選守牧治梁、洋,招集流散,耕鑿就緒,則漕運可省。此保蜀之良策也。」



 浚去相位,康國乞補外。趙鼎言於高宗曰:「自張浚罷,
 蜀士不自安,今留者十餘人,臣恐臺諫以浚故有論列,望陛下察之。」高宗曰:「朝廷用人,止當論其才與否耳。頃臺諫好以朋黨論士大夫,如罷一宰相,則凡所薦引,不問才否一時罷黜,乃朝廷使之為朋黨,非所以愛人才、厚風俗也。」遷右司員外郎,除直顯謨閣、知夔州。丁母憂,起復,撫諭吳玠軍,除都大主管川陜茶馬,卒。



 論曰:鄧肅、李邴、滕康當危急存亡之秋,皆侃侃正色,知無不言。張守論事明遠,富直柔厄於秦檜、呂頤浩,馮康
 國說折二兇,皆有用之才也。



\end{pinyinscope}