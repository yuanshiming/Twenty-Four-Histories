\article{列傳第一百二十}

\begin{pinyinscope}

 張浚子枃



 張浚,字
 德遠,漢州綿竹人,唐宰相九齡弟九皋之後。父咸,舉進士、賢良兩科。浚四歲而孤,行直視端,無誑言,識者知為大器。入太學,中進士第。靖康初,為太常簿。張邦
 昌僭立,逃入太學中。聞高宗即位,馳赴南京,除樞密院編修官,改虞部郎,擢殿中侍御史。駕幸東南,後軍統制韓世忠所部逼逐諫臣墜水死,浚奏奪世忠觀察使,上下始知有國法。遷侍御史。



 時乘輿在揚州,浚言:「中原天下之根本,願下詔葺東京、關陜、襄鄧以待巡幸。」咈宰相意,除集英殿修撰、知興元府。未行,擢禮部侍郎,高宗召諭曰:「卿知無不言,言無不盡,朕將有為,正如欲一飛沖天而無羽翼,卿勉留輔朕。」除御營使司參贊軍事。浚度
 金人必來攻,而廟堂晏然,殊不為備,力言之宰相,黃潛善、汪伯彥皆笑其過計。



 建炎三年春,金人南侵,車駕幸錢塘,留朱勝非於吳門捍禦,以浚同節制軍馬,已而勝非召,浚獨留。時潰兵數萬,所至剽掠,浚招集甫定。會苗傅、劉正彥作亂,改元赦書至平江,浚命守臣湯東野秘不宣。未幾,傅等以檄來,浚慟哭,召東野及提點刑獄趙哲謀起兵討賊。



 時傅等以承宣使張俊為秦鳳路總管,俊將萬人還,將卸兵而西。浚知上遇俊厚,而俊純實可
 謀大事,急邀俊,握手語故,相持而泣,因告以將起兵問罪。時呂頤浩節制建業,劉光世領兵鎮江,浚遣人繼蠟書,約頤浩、光世以兵來會,而命俊分兵扼吳江。上疏請復闢。傅等謀除浚禮部尚書,命將所部詣行在,浚以大兵未集,未欲誦言討賊,乃托云張俊驟回,人情震讋,不可不少留以撫其軍。



 會韓世忠舟師抵常熟,張俊曰:「世忠來,事濟矣。」白浚以書招之。世忠至,對浚慟器曰:「世忠與俊請以身任之。」浚因大犒俊、世忠將士,呼諸將校至
 前,抗聲問曰:「今日之舉,孰順孰逆?」眾皆曰:「賊逆我順。」浚曰:「聞賊以重賞購吾首,若浚此舉違天悖人,汝等可取浚頭去;不然,一有退縮,悉以軍法從事。」眾感憾憤。於是,令世忠以兵赴闕,而戒其急趨秀州,據糧道以俟大軍之至。世忠至秀,即大治戰具。



 會傅等以書招浚,浚報云:「自古言涉不順,謂之指斥乘輿;事涉不遜,謂之震驚宮闕;廢立之事,謂之大逆不道,大逆不道者族。今建炎皇帝不聞失德,一旦遜位,豈所宜聞。」傅等得書恐,乃遣重
 兵扼臨平,亟除俊、世忠節度使,而誣浚欲危社稷,責柳州安置。俊、世忠拒不受。會呂頤浩、劉光世兵踵至,浚乃聲傅、正彥罪,傳檄中外,率諸軍繼進。



 初,浚遣客馮轓以計策往說傅等,會大軍且至,傅、正彥憂恐不知所出。轓知其可動,即以大義白宰相朱勝非,使率百官請復闢。高宗御筆除浚知樞密院事。浚進次臨平,賊兵拒不得前,世忠等搏戰,大破之,傅、正彥脫遁。浚與頤浩等入見,伏地涕泣待罪,高宗問勞再三,曰:「曩在睿聖,兩宮隔絕。
 一日啜羹,小黃門忽傳太母之命,不得已貶卿郴州。朕不覺羹覆於手,念卿被謫,此事誰任。」留浚,引入內殿,曰:「皇太后知卿忠義,欲識卿面,適垂簾,見卿過庭矣。」解所服玉帶以賜。高宗欲相浚,浚以晚進,不敢當。傅、正彥走閩中,浚命世忠追縛之以獻,與其黨皆伏誅。



 初,浚次秀州,嘗夜坐,警備甚嚴,忽有客至前,出一紙懷中曰:「此苗傅、劉正彥募賊公賞格也。」浚問欲何如,客曰:「僕河北人,粗讀書,知逆順,豈以身為賊用?特見為備不嚴,恐有後
 來者耳。」浚下執其手,問姓名,不告而去。浚翌日斬死囚徇於眾,曰:「此苗、劉刺客也。」私識其狀貌物色之,終不遇。



 巨盜薛慶嘯聚淮甸,至數萬人。浚恐其滋蔓,徑至高郵,入慶壘,喻以朝廷恩意。慶感服下拜,浚留撫其眾。或傳浚為賊所執,呂頤浩等遽罷浚樞筦。浚歸,高宗驚嘆,即日趣就職。



 浚謂中興當自關陜始,慮金人或先入陜取蜀,則東南不可保,遂慷慨請行。詔以浚為川、陜宣撫處置使,得便宜黜陟。將行,御營平寇將軍範瓊,擁眾自豫
 章至行在。先是,靖康城破,金人逼脅君、後、太子、宗室北行,多瓊之謀;又乘勢剽掠,左右張邦昌,為之從衛。至是入朝,悖傲無禮,且乞貸逆黨傅、正彥等死罪。浚奏瓊大逆不道,乞伸典憲。翌日,召瓊至都堂,數其罪切責之,送棘寺論死。分其軍隸神武軍,然後行。與沿江襄、漢守臣議儲蓄,以待臨幸。



 高宗問浚大計,浚請身任陜、蜀之事,置幕府於秦川,別遣大臣與韓世忠鎮淮東,令呂頤浩扈蹕來武昌,復以張俊、劉光世與秦川相首尾。議既定,
 浚行,未及武昌,而頤浩變初議。浚既抵興元,金人已取鄜延,驍將婁宿孛堇引大兵渡渭,攻永興,諸將莫肯相援。浚至,即出行關陜,訪問風俗,罷斥奸贓,以搜攬豪傑為先務,諸將惕息聽命。



 會諜報金人將攻東南,浚命諸將整軍向敵。已而金人大攻江、淮,浚即治軍入衛。至房州,知金人北歸,復還關陜。時金帥兀朮猶在淮西,浚懼其復擾東南,謀牽制之,遂決策治兵,合五路之師以復永興。金人大恐,急調兀朮等由京西入援,大戰於富平。
 涇原帥劉錡身率將士薄敵陳,殺獲頗眾。會環慶帥趙哲擅離所部,哲軍將校望見塵起,驚遁,諸軍皆潰。浚斬哲以徇,退保興州。命吳玠聚兵扼險於鳳翔之和尚原、大散關,以斷敵來路,關師古等聚熙河兵於岷州大潭,孫渥、賈世方等聚涇原、鳳翔兵於階、成、鳳三州,以固蜀口。浚上書待罪,帝手詔慰勉。



 紹興元年,金將烏魯攻和尚原,吳玠乘險擊之,金人大敗走。兀朮復合兵至,玠及其弟璘復邀擊,大破之,兀朮僅以身免,亟剃其須髯遁
 歸。始,粘罕病篤,語諸將曰:「自吾入中國,未嘗有敢攖吾鋒者,獨張樞密與我抗。我在,猶不能取蜀;我死,爾曹宜絕意,但務自保而已。」兀朮怒曰:「是謂我不能邪!」粘罕死,竟入攻,果敗。拜浚檢校少保、定國軍節度使。



 浚在關陜三年,訓新集之兵,當方張之敵,以劉子羽為上賓,任趙開為都轉運使,擢吳玠為大將守鳳翔。子羽慷慨有才略,開善理財,而玠每戰輒勝。西北遺民,歸附日眾。故關陜雖失,而全蜀按堵,且以形勢牽制東南,江、淮亦賴以
 安。



 將軍曲端者,建炎中,嘗迫逐帥臣王庶而奪其印。吳玠敗於彭原,訴端不整師。富平之役,端議不合,其腹心張忠彥等降敵。浚初超用端,中坐廢,猶欲再用之,後卒下端獄論死。會有言浚殺趙哲、曲端無辜,而任子羽、開、玠非是,朝廷疑之。三年,遣王似副浚。會金將撒離曷及劉豫叛黨聚兵入攻,破金州。子羽為興元帥,約吳玠同守三泉。金人至金牛,宋師掩擊之,斬馘及墮溪谷死者,以數千計。浚聞王似來,求解兵柄,且奏似不可任。宰相
 呂頤浩不悅,而朱勝非以宿憾日毀短浚,詔浚赴行在。



 四年初,辛炳知潭州,浚在陜,以檄發兵,炳不遣,浚奏劾之。至是,炳為御史中丞,率同列劾浚,以本官提舉洞霄宮,居福州。浚既去國,慮金人釋川、陜之兵,必將並力窺東南,而朝廷已議講解,乃上疏極言其狀。未幾,劉豫之子麟果引金人入攻。高宗思浚前言,策免朱勝非;而參知政事趙鼎請幸平江,乃召浚以資政殿學士提舉萬壽觀兼侍讀。入見,高宗手詔辨浚前誣,除知樞密院事。



 浚既受命,即日赴江上視師。時兀朮擁兵十萬於揚州,約日渡江決戰。浚長驅臨江,召韓世忠、張俊、劉光世議事。將士見浚,勇氣十倍。浚既部分諸將,身留鎮江節度之。世忠遣麾下王愈詣兀朮約戰,且言張樞密已在鎮江。兀朮曰:「張樞密貶嶺南,何得乃在此?」愈出浚所下文書示之。兀朮色變,夕遁。



 五年,除尚書右僕射、同中書門下平章事兼知樞密院事,都督諸路軍馬,趙鼎除左僕射。浚與鼎同志輔治,務在塞幸門,抑近習。時巨寇楊么
 據洞庭,屢攻不克,浚以建康東南都會,而洞庭據上流,恐滋蔓為害,請因盛夏乘其怠討之,具奏請行。至醴陵,釋邑囚數百,皆楊么諜者,給以文書,俾招諭諸砦,囚歡呼而往。至潭,賊眾二十餘萬相繼來降,湖寇盡平。上賜浚書,謂:「上流既定,則川陜、荊襄形勢接連,事力增倍,天其以中興之功付卿乎。」浚遂奏遣岳飛屯荊、襄以圖中原,乃自鄂、岳轉淮東,大會諸將,議防秋之宜。高宗遣使賜詔趣歸,勞問之曰:「卿暑行甚勞,湖湘群寇既就招撫,
 成朕不殺之仁,卿之功也。」召對便殿,進《中興備覽》四十一篇,高宗嘉嘆,置之坐隅。



 浚以敵勢未衰,而叛臣劉豫復據中原,六年,會諸將議事江上,榜豫僭逆之罪。命韓世忠據承、楚以圖淮陽;命劉光世屯合肥以招北軍;命張俊練兵建康,進屯盱眙;命楊沂中領精兵為後翼以佐俊;命岳飛進屯襄陽以窺中原。浚渡江,遍撫淮上諸戍。時張俊軍進屯盱眙,岳飛遣兵入至蔡州,浚入覲,力請幸建康。車駕進發,浚先往江上,諜報劉豫與侄猊挾
 金人入攻,浚奏:「金人不敢悉眾而來,此必豫兵也。」邊遽不一,俊、光世皆張大敵勢,浚謂:「賊豫以逆犯順,不剿除何以為國?今日之事,有進無退。」且命楊沂中往屯濠州。劉麟逼合肥,張俊請益兵,劉光世欲退師,趙鼎及簽書折彥質欲召岳飛兵東下。御書付浚,令俊、光世、沂中等還保江。浚奏:「俊等渡江,則無淮南,而長江之險與敵共矣。且岳飛一動,襄、漢有警,復何所恃乎?」詔書從之。沂中兵抵濠州,光世舍廬州而南,淮西洶動。浚聞,疾馳至採
 石,令其眾曰:「有一人渡江者斬!」光世復駐軍,與沂中接。劉猊攻沂中,沂中大破之,猊、麟皆拔柵遁。高宗手書嘉獎,召浚還,勞之。



 時趙鼎等議回蹕臨安,浚奏:「天下之事,不倡則不起,三歲之間,陛下一再臨江,士氣百倍。今六飛一還,人心解體。」高宗幡然從浚計。鼎出知紹興府。浚以親民之官,治道所急,條具郡守、監司、省郎、館閣出入迭補之法;又以災異奏復賢良方正科。



 七年,以浚卻敵功,制除特進。未幾,加金紫光祿大夫。問安使何蘚歸報
 徽宗皇帝、寧德皇后相繼崩殂,上號慟擗踴,哀不自勝。浚奏:「天子之孝,不與士庶同,必思所以奉宗廟社稷,今梓宮未返,天下塗炭,願陛下揮涕而起,斂發而趨,一怒以安天下之民。」上乃命浚草詔告諭中外,辭甚哀切。浚又請命諸大將率三軍發哀成服,中外感動。浚退上疏曰:「陛下思慕兩宮,憂勞百姓。臣之至愚,獲遭任用,臣每感慨自期,誓殲敵仇。十年之間,親養闕然,爰及妻孥,莫之私顧,其意亦欲遂陛下孝養之心,拯生民於塗炭。昊
 天不吊,禍變忽生,使陛下抱無窮之痛,罪將誰執。念昔陜、蜀之行,陛下命臣曰:『我有大隙於北,刷此至恥,惟爾是屬。』而臣終隳成功,使敵無憚,今日之禍,端自臣致,乞賜罷黜。」上詔浚起視事。浚再疏待罪,不許,乃請乘輿發平江,至建康。



 浚總中外之政,幾事叢委,以一身任之。每奏對,必言仇恥之大,反復再三,上未嘗不改容流涕。時天子方厲精克己,戒飭宮庭內侍,無敢越度,事無鉅細,必以咨浚,賜諸將詔,往往命浚草之。



 劉光世在淮西,軍
 無紀律,浚奏罷光世,以其兵屬督府,命參謀兵部尚書呂祉往廬州節制。而樞密院以督府握兵為嫌,乞置武帥,乃以王德為都統制,即軍中取酈瓊副之。浚奏其不當,瓊亦與德有宿怨,列狀訴御史臺,乃命張俊為宣撫使,楊沂中、劉錡為制置判官以撫之。未至,瓊等舉軍叛,執呂祉以歸劉豫。祉不行,詈瓊等,碎齒折首而死。浚引咎求去位,高宗問可代者,且曰:「秦檜何如?」浚曰:「近與共事,方知其暗。」高宗曰:「然則用趙鼎。」檜由是憾浚。浚以觀
 文殿大學士提舉江州太平興國宮。先是,浚遣人持手榜入偽地間劉豫,及酈瓊叛去,復遣間持蠟書遺瓊,金人果疑豫,尋廢之。臺諫交詆,浚落職,以秘書少監分司西京,居永州。九年,以赦復官。提舉臨安府洞霄宮。未幾,除資政殿大學士、知福州兼福建安撫大使。



 金遣使來,以詔諭為名,浚五上疏爭之。十年,金敗盟,復取河南。浚奏願因權制變,則大勛可集,因大治海舟千艘,為直指山東之計。十一年,除檢校少傅、崇信軍節度使,充萬壽
 觀使,免奉朝請。十二年,封和國公。



 十六年,彗星出西方,浚將極論時事,恐貽母憂。母訝其瘠,問故,浚以實對。母誦其父對策之語曰:「臣寧言而死於斧鉞,不能忍不言以負陛下。」浚意乃決。上疏謂:「當今事勢,譬如養成大疽於頭目心腹之間,不決不止。惟陛下謀之於心,謹察情偽,使在我有不可犯之勢,庶幾社稷安全;不然,後將噬臍。」事下三省,秦檜大怒,令臺諫論浚,以特進提舉江州太平興國宮,居連州。二十年,徙永州。浚去國幾二十載,
 天下士無賢不肖,莫不傾心慕之。武夫健將,言浚者必咨嗟太息,至兒童婦女,亦知有張都督也。金人憚浚,每使至,必問浚安在,惟恐其復用。



 當是時,秦檜怙寵固位,懼浚為正論以害己,令臺臣有所彈劾,論必及浚,反謂浚為國賊,必欲殺之。以張柄知潭州,汪召錫使湖南,使圖浚。張常先使江西,治張宗元獄,株連及浚,捕趙鼎子汾下大理,令自誣與浚謀大逆,會檜死乃免。



 二十五年,復觀文殿大學士、判洪州。浚時以母喪將歸葬。念天下
 事二十年為檜所壞,邊備蕩馳;又聞金亮篡立,必將舉兵,自以大臣,義同休戚,不敢以居喪為嫌,具奏論之。會星變求直言,浚謂金人數年間,勢決求釁用兵,而國家溺於宴安,蕩然無備,乃上疏極言。而大臣沉該、萬俟離、湯思退等見之,謂敵初無釁,笑浚為狂。臺諫湯鵬舉、凌哲論浚歸蜀,恐搖動遠方,詔復居永州。服除落職,以本官奉祠。



 三十一年春,有旨自便。浚至潭,聞欽宗崩,號慟不食,上疏請早定守戰之策。未幾,亮兵大入,中外震動,
 復浚觀文殿大學士、判潭州。



 時金騎充斥,王權兵潰,劉錡退歸鎮江,遂改命浚判建康府兼行宮留守。浚至岳陽,買舟冒風雪而行,遇東來者云:「敵兵方焚採石,煙炎漲天,慎無輕進。」浚曰:「吾赴君父之急,知直前求乘輿所在而已。」時長江無一舟敢行北岸者。浚乘小舟徑進,過池陽,聞亮死,餘眾猶二萬屯和州。李顯忠兵在沙上,浚往犒之,一軍見浚,以為從天而下。浚至建康,即牒通判劉子昂辦行宮儀物,請乘輿亟臨幸。



 三十二年,車駕幸
 建康,浚迎拜道左,衛士見浚,無不以手加額。時浚起廢復用,風採隱然,軍民皆倚以為重。車駕將還臨安,勞浚曰:「卿在此,朕無北顧憂矣。」兼節制建康、鎮江府、江州、池州、江陰軍軍馬。



 金兵十萬圍海州,浚命鎮江都統張子蓋往救,大破之。浚招集忠義,及募淮楚壯勇,以陳敏為統制。且謂敵長於騎,我長於步,衛步莫如弩,衛弩莫如車,命敏專制弩治車。



 孝宗即位,召浚入見,改容曰:「久聞公名,今朝廷所恃唯公。」賜坐降問,浚從容言:「人主之學,
 以心為本,一心合天,何事不濟?所謂天者,天下之公理而已。必兢業自持,使清明在躬,則賞罰舉措,無有不當,人心自歸,敵仇自服。」孝宗悚然曰:「當不忘公言。」除少傅、江淮東西路宣撫使,進封魏國公。翰林學士史浩議欲城瓜州、採石。浚謂不守兩淮而守江幹,是示敵以削弱,怠戰守之氣,不若先城泗州。及浩參知政事,浚所規畫,浩必沮之。浚薦陳俊卿為宣撫判官,孝宗召俊卿及浚子栻赴行在。浚附奏請上臨幸建康,以動中原之心,用
 師淮堧,進舟山東,以為吳璘聲援。孝宗見俊卿等,問浚動靜飲食顏貌,曰:「朕倚魏公如長城,不容浮言搖奪。」金人以十萬眾屯河南,聲言規兩淮,移文索海、泗、唐、鄧、商州及歲幣。浚言北敵詭詐,不當為之動,以大兵屯盱眙、濠、廬備之,卒以無事。



 隆興元年,除樞密使,都督建康、鎮江府、江州、池州、江陰軍軍馬。時金將蒲察徒穆及知泗州大周仁屯虹縣,都統蕭琦,屯靈壁,積糧修城,將為南攻計。浚欲及其未發攻之。會主管殿前司李顯忠、建康
 都統邵宏淵亦獻搗二邑之策,浚具以聞。上報可,召浚赴行在,命先圖兩城。乃遣顯忠出濠州,趨靈壁;宏淵出泗州,趨虹縣,而浚自往臨之。顯忠至靈壁,敗蕭琦;宏淵圍虹縣,降徒穆、周仁,乘勝進克宿州,中原震動。孝宗手書勞之曰:「近日邊報,中外鼓舞,十年來無此克捷。」



 浚以盛夏人疲,急召李顯忠等還師。會金帥紇石烈志寧率兵至宿州,與顯忠戰。連日南軍小不利,忽諜報敵兵大至,顯忠夜引歸。浚上疏待罪,有旨降授特進,更為江、淮
 宣撫使。



 宿師之還,士大夫主和者皆議浚之非,孝宗復賜浚書曰:「今日邊事倚卿為重,卿不可畏人言而懷猶豫。前日舉事之初,朕與卿任之,今日亦須與卿終之。」浚乃以魏勝守海州,陳敏守泗州,戚方守濠州,郭振守六合。治高郵、巢縣兩城為大勢,修滁州關山以扼敵沖,聚水軍淮陰、馬軍壽春,大飭兩淮守備。



 孝宗復召栻奏事,浚附奏云:「自古有為之君,腹心之臣相與協謀同志,以成治功。今臣以孤蹤,動輒掣肘,陛下將安用之。」因乞骸
 骨。孝宗覽奏,謂栻曰:「朕待魏公有加,不為浮議所惑。」帝眷遇浚猶至,對近臣言,必曰魏公,未嘗斥其名。每遣使來,必令視浚飲食多寡,肥瘠何如。尋詔復浚都督之號。



 金帥僕散忠義貽書三省、樞密院,索四郡及歲幣,不然,以農隙治兵。浚言:「金強則來,弱則止,不在和與不和。」時湯思退為右相。思退,秦檜黨也,急於求和,遂遣盧仲賢持書報金。浚言仲賢小人多妄,不可委信。已而仲賢果以許四郡辱命。朝廷復以王之望為通問使,龍大淵副
 之,浚爭不能得。未幾,召浚入見,復力陳和議之失。孝宗為止誓書,留之望、大淵待命,而令通書官胡昉、楊由義往,諭金以四郡不可割;若金人必欲得四郡,當追還使人,罷和議。拜浚尚書右僕射、同中書門下平章事兼樞密使,都督如故;思退為左僕射。



 胡昉等至宿,金人械系迫脅之,昉等不屈,更禮而歸之。孝宗諭浚曰:「和議之不成,天也,自此事當歸一矣。」二年,議進幸建康,詔之望等還。思退聞之大駭,陽為乞祠狀,而陰與其黨謀為陷浚
 計。



 俄詔浚行視江、淮。時浚所招徠山東、淮北忠義之士,以實建康、鎮江兩軍,凡萬二千餘人,萬弩營所招淮南壯士及江西群盜又萬餘人,陳敏統之,以守泗州。凡要害之地,皆築城堡;其可因水為險者,皆積水為匱;增置江、淮戰艦,諸軍弓矢器械悉備。時金人屯重兵於河南,為虛聲脅和,有刻日決戰之語。及聞浚來,亟徹兵歸。淮北之來歸者日不絕,山東豪傑,悉願受節度。浚以蕭琦契丹望族,沉勇有謀,欲令盡領契丹降眾,且以檄諭契
 丹,約為應援,金人益懼。思退乃令王之望盛毀守備,以為不可恃;令尹穡論罷督府參議官馮方;又論浚費國不貲,奏留張深守泗不受趙廓之代為拒命。浚亦請解督府,詔從其請。左司諫陳良翰、侍御史周操言浚忠勤,人望所屬,不當使去國。浚留平江,凡八章乞致仕,除少師、保信軍節度、判福州。浚辭,改醴泉觀使。朝廷遂決棄地求和之議。



 浚既去,猶上疏論尹穡奸邪,必誤國事,且勸上務學親賢。或勉浚勿復以時事為言,浚曰:「君臣之
 羲,無所逃於天地之間。吾荷兩朝厚恩,久尸重任,今雖去國,猶日望上心感悟,茍有所見,安忍弗言。上如欲復用浚,浚當即日就道,不敢以老病為辭。如若等言,是誠何心哉!」聞者聳然。行次餘干,得疾,手書付二子曰:「吾嘗相國,不能恢復中原,雪祖宗之恥,即死,不當葬我先人墓左,葬我衡山下足矣。」訃聞,孝宗震悼,輟視朝,贈太保,後加贈太師,謚忠獻。



 浚幼有大志,及為熙河幕官,遍行邊壘,覽觀山川形勢,時時與舊戍守將握手飲酒,問祖
 宗以來守邊舊法,及軍陳方略之宜。故一旦起自疏遠,當樞筦之任,悉能通知邊事本末。在京城中,親見二帝北行,皇族系虜,生民塗炭,誓不與敵俱存,故終身不主和議。每論定都大計,以為東南形勢,莫如建康,人主居之,可以北望中原,常懷憤惕。至如錢塘,僻在一隅,易於安肆,不足以號召北方。與趙鼎共政,多所引擢,從臣朝列,皆一時之望,人號「小元祐」。所薦虞允文、汪應辰、王十朋、劉珙等為名臣;拔吳玠、吳璘於行間,謂韓世忠忠勇,
 可倚以大事,一見劉錡奇之,付以事任,卒皆為名將,有成功,一時稱浚為知人。浚事母以孝稱,學邃於《易》,有《易解》及《雜說》十卷,《書》、《詩》、《禮》、《春秋》、《中庸》亦各有解,文集十卷,奏議二十卷。子二人、栻、枃。栻自有傳。



 枃字定叟,以父恩授承奉郎,歷廣西經略司機宜、通判嚴州。方年少,已有能稱,浙西使者薦所部吏而不及枃,孝宗特令再薦。召對,差知袁州,戢豪強,弭盜賊。尉獲盜上之州,枃察知其枉,縱去,莫不怪之,未幾,果獲真盜。改
 知衢州。



 兄栻喪,無壯子,請祠以營葬事,主管玉局觀,遷湖北提舉常平。奏事,帝大喜,諭輔臣曰:「張浚有子如此。」改浙西,督理荒政,蘇、湖二州皆闕守,命兼攝焉。有執政姻黨閉糶,枃首治之,帝獎其不畏強御,遷兩浙轉運判官。



 未幾,以直徽猷閣升副使,改知臨安府。奏除逋欠四萬緡,米八百斛,進直龍圖閣。都城浩穰,奸盜聚慝,枃畫分地以警捕,夜戶不閉。張師尹納女掖庭供給使,恃以恣橫,枃因事痛繩之,徙其家信州,其類帖伏。南郊禮成,
 賜五品服,權兵部侍郎,仍知臨安,加賜三品服。修三閘,復六井。府治火,延及民居,上疏自劾,詔削二秩。枃再疏乞罷,移知鎮江。尋改明州,辭,仍知鎮江。召為戶部侍郎,面對言事,迕時相意。高宗崩,以集英殿修撰知紹興府,董山陵事。召還,為吏部侍郎。



 光宗即位,權刑部侍郎,復兼知臨安府。紹熙元年,為刑部侍郎,仍為府尹。內侍毛伯益冒西湖茭地為亭,外戚有殺其僕者,獄具,夤緣宣諭求免,枃皆執奏論如律。孝宗觀湖,枃以彈壓伏謁道
 左,孝宗止輦問勞,賜以酒炙。



 京西謀帥,進煥章閣學士、知襄陽府,賜金二百兩,別賜金百兩,白金倍之。未幾,進徽猷閣學士、知建康府,繼復命還襄陽。寧宗嗣位,歸正人陳應祥、忠義人黨琪等謀襲均州,副都統馮湛間道疾馳以聞。枃不為動,徐部分掩捕,獄成,斬其為首者二人,盡釋黨與,反側以安。



 升寶文閣學士、知平江府,未行,改知建康府。升龍圖閣學士、知隆興府兼江西安撫使。奉新縣舊有營田,募民耕之,畝賦米斗五升,錢六十,其
 後議臣請鬻之。始,征兩稅和買,且加折變,民重為困,枃悉奏蠲之。進端明殿學士,復知建康府。以疾乞祠,卒。



 枃天分高爽,吏材敏給,遇事不凝滯,多隨宜變通,所至以治辨稱。再渡以來,論尹京者,以枃為首。子忠純、忠恕,自有傳。



 論曰:儒者之於國家,能養其正直之氣,則足以正君心,一眾志,攘兇逆,處憂患,蓋無往而不自得焉。若張浚者,可謂善養其氣者矣。觀其初逃張邦昌之議,平苗、劉之
 亂,其才識固有非偷懦之所敢望。及其攘卻勍敵,招降劇盜,能使將帥用命,所向如志。遠人伺其用舍為進退,天下占其出處為安危,豈非卓然所謂人豪者歟!群言沸騰,屢奮屢躓,而辭氣概然。嘗曰:「上如欲復用浚,當即日就道,不敢以老病辭。」其言如是,則其愛君憂國之心,為何如哉!時論以浚之忠大類漢諸葛亮,然亮能使魏延、楊儀終其身不為異同,浚以吳玠故遂殺曲端,亮能容法孝直,浚不能容李綱、趙鼎而又詆之,茲所以為不
 及歟!至於富平之潰師,淮西之兵變,則成敗利鈍,雖亮不能逆睹也。



\end{pinyinscope}