\article{列傳第二百二十七宦者三}

\begin{pinyinscope}

 ○李祥陳衍馮世寧李繼和高居簡程昉蘇利涉雷允恭閻文應任守忠童貫方臘附梁師成
 楊戩



 李祥,開封人。為入內黃門。資驍銳,善騎射,用材武中選,授涇原儀渭同巡檢。從景思立於河、湟,以功遷內殿崇班,為河州駐泊兵馬都監。從郭逵討交阯,駐富良江,賊兵大至,與涇原將姚兕力戰,敗之。遷皇城使、鎮戎軍沿邊都巡檢使。從劉昌祚徵靈武,議功加沂州團練使。或言所部兵失亡多,降簡州刺史,權熙河蘭會路都監,總岷州兵。夏人攻蘭州,祥赴援,保險待變,數日,虜徹圍去。
 復團練使,進階州防禦使。從種誼襲鬼章有功,升兵馬都鈐轄。在熙河二十餘年,以宣慶使、內侍押班卒。



 陳衍,開封人。以內侍給事殿庭,累官供備庫使。梁惟簡薦諸宣仁聖烈皇后,主管高韓王宅,領御藥院、內東門司。宣仁山陵,為按行使。俄以左藏庫使、文州刺史出為真定路都監。



 御史來之邵方力詆元祐政事,首言:「衍在垂簾日,怙寵驕肆,交結戚里,進退大臣,力引所私,俾居耳目之地。」張商英亦論:「衍交通宰相,御服為之賜珠;結
 托詞臣,儲祥為之賜膳。」蓋指呂大防、蘇軾也。衍坐貶,監郴州酒稅務。惟簡以援引,張士良、梁知新以黨附,皆得罪。已又編管白州,徙配朱崖。



 章惇起獄,誣元祐諸老、大臣,雲結衍輩以謀廢立。士良嘗與衍同在宣仁後閣,自郴州召之,使實其說。士良至,但言宣仁彌留之際,衍嘗可否二府事及用御寶付外而已。鍛煉無所得,安惇、蔡京乃奏衍疏隔兩宮,斥隨龍內侍十餘人於外,以剪除人主腹心羽翼,意在動搖,大逆不道。乃詔處死,令廣西
 轉運使程節涖其刑。



 馮世寧,字靜之,以入內黃門累遷昭宣使、忠州團練使、入內押班。揚國公主寢疾,哲宗欲夜出問訊,世寧執言不可,帝雖微忤,卒為之改容。再遷景福殿使、明州觀察使。至副都知。崇寧新官名,世寧首知入內內侍省事。禁中夜火,使宿衛士撲滅之,既定,令自他途出,蓋不欲使知宮省曲折也。徽宗賞嘆。進感德軍留後。政和初,以內客省使、彰化軍留後致仕。



 世寧出入禁闥六十年,循謹
 無過。卒,年六十七,贈開府儀同三司。謚曰恭節。



 李繼和,開封人。以父任為內侍黃門。慶歷中,為河北西路承受。保州兵叛,塞城門距守,官軍重圍之,不得入。繼和獨上南關門,密呼所結內應者,諭以禍福。眾言:「俟李昭亮至,即斬關自歸。」已而果然。賊平,遷兩秩。王則反貝州,為城下走馬承受。



 沙苑闕馬,詔秦州置場以券市之,繼和領職不數月,得馬千數,而人不擾。舊制,內侍入仕三十年始得磨勘,至是,乃令以勞進官者無拘於年。



 環
 州弓箭手歲時給酒,州將不與,眾喧訴,亟闔府門不敢出,繼和步入眾中譬曉之曰:「汝曹為一杯酒,遂喪軀命乎!」眾悟散去。事聞,擢帶御器械。累遷宣慶使、文州團練使、入內副都知,卒。子從善援例求贈官,神宗曰:「此弊事也!繼和無軍功,何必贈?」自是為定制云。



 高居簡,字仲略,世本番禺人。以父任為入內黃門。護作溫成原廟奉神物,以精辦稱,超轉殿頭,領後苑事。坐奉使梓夔路多占驛兵,降高品。歷領龍圖、天章、寶文閣、內
 東門司,乾當御藥院。



 神宗即位,御史張唐英言其資性憸巧,善迎合取容。中丞司馬光亦言其「久處近職,罪惡已多。祖宗舊制,乾當御藥院官至內殿崇班以上,即須出外。今陛下獨留四人,中外以此竊議。況居簡頃在先朝,依憑城社,物論切齒。及陛下繼統,乃復先自結納,使寵信之恩過於先帝。願明治其罪,以解天下之惑」。於是罷為供備庫使。稍遷帶御器械,進內侍押班。以文思使領忠州刺史。卒,贈耀州觀察使。



 居簡聞外廷議論,必以
 入告,省中目為「高直奏」。仁宗時,嘗使南海,遇廣州火,救者不力,居簡督眾護軍資甲仗二庫,賴以獲全。事聞,詔褒之。



 程昉,開封人。以小黃門積遷西京左藏庫副使。熙寧初,為河北屯田都監。河決棗強,釃二股河導之使東,為鋸牙,下以竹落塞決口。加帶御器械。河決商胡北流,與御河合為一。及二股東流,御河遂淺澱。昉以開浚功,遷宮苑副使。又塞漳河,作浮梁於洺州。兼外都水丞,詔相度
 興修水利。河決大名第五埽,昉議塞之,因疏塘水溉深州田。又導葫蘆河,自樂壽之東至滄州二百里。塞孟家口,開乾寧軍直河,作橋於真定之中渡。又自衛州王供埽導沙河入御河,以廣運路。累遷達州團練使,制置河北河防水利。



 御史盛陶言:「昉挾第五埽之功,專為己力。假朝廷威福,恐動州縣。所開共城河,頗廢人戶水磑,久無成功。又議開沁河,因察訪官按行,始知不便。漳河、滹沱之役,水占邢、洺、趙、深、祁五州之田,王廣廉、孔嗣宗、錢
 勰、趙子幾皆嘗論奏其奸欺之狀,則多置撻口,指決河所侵便為淤田。其事權之盛,則舉官廢吏,惟其所欲。悖慢豪橫,則受聖旨者三,受提點刑獄司牒者十二,故有違拒。小人誤當賞擢,驕暴自肆。願遣官代還,仍行究治。」神宗曰:「王安石以昉知河事,故加任使,令開漳河,用工七百萬,滹沱八九百萬,已議體量矣。」



 始,安石欲興水利,驟用昉,昉挾安石勢而慢韓琦,後安石覺其虛誕,亦疏之。以憂死,贈耀州觀察使。遂罷都大制置河防水利司。



 蘇利涉,字公濟。祖保遷,自廣州以閹人從劉鋹入朝。利涉初為入內內品。慶歷中衛士之變,以護衛有勞,賞激加等。英宗為皇子,利涉給事東宮。及即位,遷東頭供奉官,欲以為穎王府都監,力辭,乾當御藥院,遷供備庫使。帝不豫,侍醫藥最勤,言輒流涕。及帝崩,乞與醫官同貶,三上表待罪,不許。



 神宗即位,授達州刺史。歷內侍押班、副都知,轉海州團練使。仙韶院火,營救甚力,賜襲衣、金帶。卒,年六十四,贈奉國軍節度使,謚曰勤僖。



 利涉嘗乾
 當皇城司,循故事,廂卒邏報不皆以聞。後石得一代之,事無巨細悉以奏,往往有緣飛語受禍者,人始以利涉為賢。



 雷允恭,開封人。初為黃門,頗慧黠,稍遷入內殿頭,給事東宮。周懷政偽為天書,允恭豫發其事,懷政死,擢內殿崇班,遷承制。再遷西京作坊使、普州刺史、入內內侍省押班。



 章獻后初臨政,丁謂潛結允恭,凡機密事令傳達禁中,由是允恭勢橫中外。山陵事起,允恭請效力陵上,
 章獻后曰:「吾慮汝有妄動,恐為汝累也。」乃以為山陵都監。允恭馳至陵下,司天監邢中和為允恭言:「今山陵上百步,法宜子孫,類汝州秦王墳。」允恭曰:「何不就?」中和曰:「恐下有石與水爾。」允恭曰:「上無他子,若如秦王墳,何不可?」中和曰:「山陵事重,踏行覆按,動經月日,恐不及七月之期耳。」允恭曰:「第移就上穴,我走馬入見太后言之。」允恭素貴橫,人不敢違,即改穿上穴。入白其事,章獻后曰:「此大事,何輕易如此?」允恭曰:「使先帝宜子孫,何惜不可?」
 章獻後意不然,曰:「出與山陵使議可否。」時丁謂為山陵使,允恭具道所以,謂唯唯而已。允恭入奏曰:「山陵使亦無異議矣。」既而上穴果有石,石盡水出。允恭竟以是並坐盜金寶賜死,籍其家。中和流沙門島。謂尋竄海上。



 閻文應,開封人。給事掖庭,積遷至入內副都知。仁宗初親政,與宰相呂夷簡謀,以張耆、夏竦、陳堯佐、範雍、趙稹、晏殊、錢惟演皆章獻後所任用,悉罷之。退以語郭後,後曰:「夷簡獨不附太后邪?但多機巧,善應變耳。」由是並夷
 簡罷。



 夷簡素與文應相結,使為中詗。久之,乃知事由郭後,夷簡遂怨後,及再相,楊、尚二美人方寵,尚美人於仁宗前有語侵後,後不勝忿,批其頰,仁宗自起救之,誤中其頸,仁宗大怒。文應乘隙,遂與謀廢後,且勸以爪痕示執政。夷簡以怨,力主廢事,因奏仁宗出諫官,竟廢後為凈妃,以所居宮名瑤華,皆文應為夷簡內應也。



 郭后既廢,楊、尚二美人益寵專夕,仁宗體為之弊,或累日不進食,中外憂懼。楊太后亟以為言,仁宗未能去。文應早暮
 入侍,言之不已,仁宗厭其煩,強應曰:「諾。」文應即以氈車載二美人出,二美人涕泣,詞說云云不肯行。文應罵曰:「官婢尚何言?」驅使登車。翌日,以尚氏為女道士,居洞真宮;楊氏別宅安置。既而仁宗復悔廢郭後,有復後之意,文應大懼。會後有小疾,挾太醫診視數日,乃言後暴崩,實文應為之也。



 累至昭宣使、恩州團練使。時諫官劾其罪,請並其子士良出之。以文應領嘉州防禦使,為秦州鈐轄,改鄆州,士良罷御藥院,為內殿崇班。



 始楊、尚二美
 人之出宮也,左右引陳氏女入宮,父號陳子城,楊太后嘗許以為後,宋綬不可。王曾、呂夷簡、蔡齊相繼論諫。陳氏女將進御,士良聞之,遽見仁宗。仁宗披百葉擇日,士良曰:「陛下閱此,豈非欲納陳氏女為後邪?」仁宗曰:「然。」士良曰:「子城使,大臣家奴僕官名也,陛下納其女為後,無乃不可乎!」仁宗遽命出之。文應後徙相州鈐轄。卒,贈邠州觀察使。



 任守忠,字稷臣,蔭入內黃門,累轉西頭供奉官,領御藥
 院,坐事廢。久之,復故官,稍遷上御藥供奉。初,章獻後聽政,守忠與都知江德明等交通請謁,權寵過盛。仁宗親政,出為黃州都監,又謫監英州酒稅,稍遷潭州都監,徙合流鎮。西鄙用兵,又為秦鳳、涇原路駐泊都監,以功再遷東染院使、內侍押班。出為定州鈐轄,加內侍副都知。累遷宣政使、洋州觀察使,為入內都知。



 仁宗未有嗣,屬意英宗,守忠居中建議,欲援立昏弱以徼大利。及英宗即位,拜宣慶使、安靜軍留後。守忠又語言誕妄,交亂兩
 宮。於是知諫院司馬光論守忠離間之罪,為國之大賊,民之巨蠹,乞斬於都市。英宗猶未行,宰相韓琦出空頭敕一道,參政歐陽修已簽,趙概難之,修曰:「第書之,韓公必自有說。」琦遂坐政事堂,立守忠庭下,曰:「汝罪當死,貶保信軍節度副使、蘄州安置。」取空頭敕填與之,即日押行,琦意以為少緩則中變也。



 守忠久被寵幸,用事於中,人不敢言其過,及貶,中外快之。久之,起為左武衛將軍,致仕,卒,年七十九。



 童貫,少出李憲之門。性巧媚,自給事宮掖,即善策人主微指,先事順承。微宗立,置明金局於杭,貫以供奉官主之,始與蔡京游。京進,貫力也。京既相,贊策取青唐,因言貫嘗十使陜右,審五路事宜與諸將之能否為最悉,力薦之。合兵十萬,命王厚專閫寄,而貫用李憲故事監其軍。至湟川,適禁中火,帝下手札,驛止貫毋西兵。貫發視,遽納鞾中。厚問故,貫曰:「上趣成功耳。」師竟出,復四州。擢景福殿使、襄州觀察使,內侍寄資轉兩使自茲始。



 未幾,
 為熙河蘭湟、秦鳳路經略安撫制置使,累遷武康軍節度使。討溪哥臧征,復積石軍、洮州,加檢校司空。頗恃功驕恣,選置將吏,皆捷取中旨,不復關朝廷,浸咈京意。除開府儀同三司,京曰:「使相豈應授宦官?」不奉詔。



 政和元年,進檢校太尉,使契丹。或言:「以宦官為上介,國無人乎?」帝曰:「契丹聞貫破羌,故欲見之,因使覘國,策之善者也。」使還,益展奮,廟謨兵柄皆屬焉。遂請進築夏國橫山,以太尉為陜西、河東、河北宣撫使。俄開府儀同三司,簽書
 樞密院河西北兩房。不三歲,領院事。更武信、武寧、護國、河東、山南東道、劍南、東川等九鎮、太傅、涇國公。時人稱蔡京為公相,因稱貫為媼相。



 將秦、晉銳師深入河、隴,薄於蕭關古骨龍,謂可制夏人死命。遣大將劉法取朔方,法不可,貫逼之曰:「君在京師時,親授命於王所,自言必成功,今難之,何也?」法不得已出塞,遇伏而死。法,西州名將,既死,諸軍恟懼。貫隱其敗,以捷聞,百官入賀,皆切齒,然莫敢言。關右既困,夏人亦不能支,乃因遼人進誓
 表納款。使至,授以誓詔,辭不取,貫強館伴使固與之,還及境,棄諸道上。舊制,熟羌不授漢官,貫故引拔之,有至節度使者。弓箭手失其分地而使守新疆,禁卒逃亡不死而得改隸他籍,軍政盡壞。



 政和元年,副鄭允中使於遼,得燕人馬植,歸薦諸朝,遂造平燕之謀,選健將勁卒,刻日發命。會方臘起睦州,勢甚張,改江、浙、淮南宣撫使,即以所聚兵帥諸將討平之。



 方臘者,睦州青溪人也。世居縣堨村,托左道以惑眾。初,唐永徽中,睦州女子陳碩
 真反,自稱文佳皇帝,故其地相傳有天子基、萬年樓,臘益得憑籍以自信。縣境梓桐、幫源諸峒皆落山谷幽險處,民物繁夥,有漆楮、杉材之饒,富商巨賈多往來。



 時吳中困於朱勔花石之擾,比屋致怨,臘因民不忍,陰聚貧乏游手之徒。宣和二年十月,起為亂,自號聖公,建元永樂,置官吏將帥,以巾飾為別,自紅巾而上凡六等。無弓矢、介胄,唯以鬼神詭秘事相扇訹,焚室廬,掠金帛子女,誘脅良民為兵。人安於太平,不識兵革,聞金鼓聲即斂
 手聽命,不旬日聚眾至數萬,破殺將官蔡遵於息坑。十一月陷青溪,十二月陷睦、歙二州。南陷衢,殺郡守彭汝方;北掠新城、桐廬、富陽諸縣,進逼杭州。郡守棄城走,州即陷,殺制置使陳建、廉訪使趙約,縱火六日,死者不可計。凡得官吏,必斷臠支體,探其肺腸,或熬以膏油,叢鏑亂射,備盡楚毒,以償怨心。



 警奏至京師,王黼匿不以聞,於是兇焰日熾。蘭溪靈山賊朱言吳邦、剡縣仇道人、仙居呂師囊、方巖山陳十四、蘇州石生、歸安陸行兒皆合
 黨應之,東南大震。



 發運使陳亨伯請調京畿兵及鼎、澧槍牌手兼程以來,使不至滋蔓。徽宗始大驚,亟遣童貫、譚稹為宣撫制置使,率禁旅及秦、晉蕃漢兵十五萬以東,且諭貫使作詔罷應奉局。三年正月,臘將方七佛引眾六萬攻秀州,統軍王子武乘城固守,已而大軍至,合擊賊,斬首九千,築京觀五,賊還據杭。二月,貫、稹前鋒至清河堰,水陸並進,臘復焚官舍、府庫、民居,乃宵遁。諸將劉延慶、王稟、王渙、楊惟忠、辛興宗相繼至,盡復所失城。
 四月,生擒臘及妻邵、子毫二太子、偽相方肥等五十二人於梓桐石穴中,殺賊七萬。四年三月,餘黨悉平。進貫太師,徙國楚。



 臘之起,破六州五十二縣,戕平民二百萬,所掠婦女自賊峒逃出,惈而縊於林中者,由湯巖、椔嶺八十五里間,九村山谷相望。王師自出至凱旋,四百五十日。



 臘雖平,而北伐之役遂起。既而以復燕山功,詔解節鉞為真三公,加封徐、豫兩國。越兩月,命致仕,而代以譚稹。明年復起,領樞密院,宣撫河北、燕山。宣和七年,詔
 用神宗遺訓,能復全燕之境者胙本邦,疏王爵,遂封廣陽郡王。



 是年,粘罕南侵,貫在太原,遣馬擴、辛興宗往聘以嘗金,金人以納張覺為責,且遣使告興兵,貫厚禮之,謂曰:「如此大事,何不素告我?」使者勸貫速割兩河以謝,貫氣褫不能應,謀遁歸。太原守張孝純誚之曰:「金人渝盟,王當令天下兵悉力枝梧,今委之而去,是棄河東與敵也。河東入敵手,奈河北乎?」貫怒叱之曰:「貫受命宣撫,非守土也。君必欲留貫,置帥何為?」孝純拊掌嘆曰:「平生
 童太師作幾許威望,及臨事乃蓄縮畏懾,奉頭鼠竄,何面目復見天子乎?」



 貫奔入都,欽宗已受禪,下詔親征,以貫為東京留守,貫不受命而奉上皇南巡。貫在西邊募長大少年號勝捷軍,幾萬人,以為親軍,環列第舍,至是擁之自隨。上皇過浮橋,衛士攀望號慟,貫唯恐行不速,使親軍射之,中矢而踣者百餘人,道路流涕,於是諫官、御史與國人議者蜂起。初貶左衛上將軍,連謫昭化軍節度副使,竄之英州、吉陽軍。行未至,詔數其十大罪,命
 監察御史張澂跡其所至,蒞斬之,及於南雄。既誅,函首赴闕,梟於都市。



 貫握兵二十年,權傾一時,奔走期會過於制敕。嘗有論其過者,詔方劭往察,劭一動一息,貫悉偵得之,先密以白,且陷以他事,劭反得罪,逐死。貫狀魁梧,偉觀視,頤下生須十數,皮骨勁如鐵,不類閹人。有度量,能疏財。後宮自妃嬪以下皆獻餉結內,左右婦寺譽言日聞。寵煽翕赫,庭戶雜遝成市,岳牧、輔弼多出其門,廝養、僕圉官諸使者至數百輩。窮奸稔禍,流毒四海,
 雖菹醢不償責也。



 梁師成,字守道,慧黠習文法,稍知書。初隸賈詳書藝局,詳死,得領睿思殿文字外庫,主出外傳道上旨。政和間,得君貴幸,至竄名進士籍中,積遷晉州觀察使、興德軍留後。建明堂,為都監,既成,拜節度使、加中太一、神霄宮使。歷護國、鎮東、河東三節度,至檢校太傅,遂拜太尉、開府儀同三司,換節淮南。



 時中外泰寧,徽宗留意禮文符瑞之事,師成善逢迎,希恩寵。帝本以隸人畜之,命入處
 殿中,凡御書號令皆出其手,多擇善書吏習仿帝書,雜詔旨以出,外廷莫能辨。師成實不能文,而高自標榜,自言蘇軾出子。是時,天下禁誦軾文,其尺牘在人間者皆毀去,師成訴於帝曰:「先臣何罪?」自是,軾之文乃稍出。以翰墨為己任,四方俊秀名士必招致門下,往往遭點污。多置書畫卷軸於外舍,邀賓客縱觀,得其題識合意者,輒密加汲引,執政、侍從可階而升。王黼父事之,雖蔡京父子亦諂附焉,都人目為「隱相」,所領職局至數十百。



 黼
 造伐燕議,師成始猶依違,卒乃贊決,又薦譚稹為宣撫。燕山平,策勛進少保。益通賄謝,人士入錢數百萬,以獻頌上書為名,令赴廷試,唱第之日,侍於帝前,囁嚅升降。其小吏儲宏亦豫科甲,而執廝養之役如初。李彥括民田於京東、西,所至倨坐堂上,監司、郡守不敢抗禮。有言於帝,師成適在旁,抗聲曰:「王人雖微,序於諸侯之上,豈足為過?」言者懼而止。師成貌若不能言,然陰賊險鷙,遇間即發。



 家居與黼鄰,帝幸黼第,見其交通狀,已怒,朱勔
 又以應奉與黼軋,因乘隙攻之。帝罷黼相,師成由是益絀。鄆王楷寵盛,有動搖東宮意,師成能力保護。欽宗立,嬖臣多從上皇東下,師成以舊恩留京師。於是太學生陳東、布衣張炳力疏其罪。炳指之為李輔國,且言宦官表裏相應,變恐不測。東復論其有異志,攘定策功,當正典刑。帝迫於公議,猶未誦言逐之。師成疑之,寢食不離帝所,雖奏廁亦侍於外,久未有以發。會鄭望之使金營還,帝命師成及望之以宣和殿珠玉器玩復往。先令
 望之詣中書諭宰相,至則留之,始詔暴其罪,責為彰化軍節度副使。開封吏護至貶所,行次八角鎮,縊殺之,以暴死聞,籍其家。



 楊戩,少給事掖庭,主掌後苑,善測伺人主意。自崇寧後,日有寵,知入內內侍省。立明堂,鑄鼎鼐、起大晟府、龍德宮,皆為提舉。



 政和四年,拜彰化軍節度使,首建期門行幸事以固其權,勢與梁師成埒。歷鎮安、清海、鎮東三鎮,由檢校少保至太傅,遂謀撼東宮。



 有胥吏杜公才者獻
 策於戩,立法索民田契,自甲之乙,乙之丙,展轉究尋,至無可證,則度地所出,增立賦租。始於汝州,浸淫於京東西、淮西北,括廢堤、棄堰、荒山、退灘及大河淤流之處,皆勒民主佃。額一定後,雖沖蕩回復不可減,號為「西城所」。築山濼古鉅野澤,綿亙數百里,濟、鄆數州,賴其蒲魚之利,立租算船納直,犯者盜執之。一邑率於常賦外增租錢至十餘萬緡,水旱蠲稅,此不得免。擢公才為觀察使。宣和三年,戩死,贈太師、吳國公,而李彥繼其職。



 彥天資
 狠愎,密與王黼表裏,置局汝州,臨事愈劇。凡民間美田,使他人投牒告陳,皆指為天荒,雖執印券皆不省。魯山闔縣盡括為公田,焚民故券,使田主輸租佃本業,訴者輒加威刑,致死者千萬。公田既無二稅,轉運使亦不為奏除,悉均諸別州。京西提舉官及京東州縣吏劉寄、任輝彥、李士漁、王滸、毛孝立、王隨、江惇、呂坯、錢棫、宋憲皆助彥為虐,如奴事主,民不勝忿痛。前執政冠帶操笏,迎謁馬首獻媚,花朝夕造請,賓客徑趍謁舍,不敢對之上
 馬,而彥處之自如。



 發物供奉,大抵類朱勔,凡竹數竿用一大車、牛驢數十頭,其數無極,皆責辦於民,經時閱月,無休息期。農不得之田,牛不得耕墾,殫財靡芻,力竭餓死,或自縊轅軛間。如龍鱗薜荔一本,輦致之費逾百萬。喜賞怒刑,禍福轉手,因之得美官者甚眾。潁昌兵馬鈐轄範寥不為取竹,誣刊蘇軾詩文於石為十惡,朝廷察其捃摭,亦令勒停。當時謂朱勔結怨於東南,李彥結怨於西北。



 靖康初,詔追戩所贈官爵,彥削官賜死,籍其家;
 劉寄以下十人皆停廢;復範寥官。



\end{pinyinscope}