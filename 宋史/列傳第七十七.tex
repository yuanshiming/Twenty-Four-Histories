\article{列傳第七十七}

\begin{pinyinscope}

 張方平王拱辰張趙
 概胡宿
 子宗
 炎從子宗愈宗回



 張方平,字安道,南京人。少穎悟絕倫,家貧無書,從人假三史,旬日即歸之,曰:「吾已得其詳矣。」凡書皆一閱不再
 讀,宋綬、蔡齊以為天下奇才。舉茂材異等,為校書郎、知昆山縣。又中賢良方正,選遷著作佐郎、通判睦州。



 趙元昊且叛,為嫚書來,規得譴絕以激使其眾。方平請:「順適其意,使未有以發,得歲月之頃,以其間選將厲士,堅城除器,為不可勝以待之。雖終於必叛,而兵出無名,吏士不直其上,難以決勝。小國用兵三年,而不見勝負,不折則破;我以全制其後,必勝之道也。」時天下全盛,皆謂其論出姑息,決計用兵。方平上《平戎十策》,以為:「入寇當自
 延、渭,巢穴之守必虛。宜屯兵河東,卷甲而趨之,所謂攻其所必救,形格勢禁之道也。」宰相呂夷簡善其策而不果行。當召試館職,仁宗曰:「是非兩策制科者乎?何試也?」命直集賢院,俄知諫院。夏人寇邊,方平首乞合樞密之職於中書,以通謀議。帝然之,遂以宰相兼樞密使。時調諸道弓手,刺其壯者為宣毅、保捷,方平連疏爭之,弗聽。既而兩軍驕甚,合二十餘萬,皆市人不可用,如方平言。



 夏竦節制陜西並護諸將,四路以稟復失事機,且詔使
 出師,逗遛不行。及豐州陷,劉平等覆師,主帥皆坐譴,竦獨不預,方平劾罷之,而請四路帥臣,各自任戰守。西師久未解,元昊亦困弊,方平言:「陛下猶天地父母也,豈與犬豕豺狼較乎?願因郊赦,引咎示信,開其自新之路。」帝喜曰:「是吾心也。」是歲,改慶歷赦書,敕邊吏通其善意,元昊竟降。既,以修起居注使契丹。契丹主顧左右曰:「有臣如此,佳哉」騎而擊球於前,酌玉卮飲之,且贈以所乘馬。還,知制誥,權知開封府。府事叢集,前尹率書板識之,方
 平獨默記決遣,無少差忘。進翰林學士。元昊既臣,而與契丹有隙,來請絕其使,議者不可。方平曰:「得新附之小羌,失久和之強敵,非計也。宜賜元昊詔,使之審處,但嫌隙朝除,則封冊暮下。如此,於西、北為兩得矣。」時韙其謀。拜御史中丞,改三司使。



 初,王拱辰議榷河北鹽,方平見曰:「河北再榷鹽,何也?」帝曰:「始立法耳。」方平曰:「昔周世宗以鹽課均之稅中,今兩稅鹽錢是也。豈非再榷乎?」帝驚悟,方平請直降手詔罷之。河朔父老迎拜於澶州,為佛
 老會七日,以報上恩,事具《食貨志》。加端明殿學士、判太常寺。



 禁中衛卒夜變,帝旦語二府,獎張貴妃扈蹕功。夏竦即倡言:「當求所以尊異之禮。」方平聞之,謂陳執中曰:「漢馮婕妤身當猛獸,不聞有所尊異;且皇后在而尊貴妃,古無是事。果行之,天下之責,將萃於公矣。」執中瞿然而罷。



 帝以豐財省費訪群臣,方平既條對,又獨上數千言,大略以為:「祥符以來,務為姑息,漸失祖宗之舊。取士、任子、磨勘、遷補之法壞,命將養兵,皆非舊律。國用既窘,
 則政出多門;大商豪民乘隙射利,而茶鹽香礬之法亂。此治忽盛衰之本,不可以不急。」帝覽對甚悅,且大用,會判官楊儀得罪,坐與交,出知滁州。頃之,知江寧府,入判流內銓。



 以侍講學士知滑州,徙益州。未至,或扇言儂智高在南詔,將入寇,攝守亟調兵築城,日夜不得息,民大驚擾。朝廷聞之,發陜西步騎兵仗,絡繹往戍蜀。詔趣方平行,許以便宜從事,方平曰:「此必妄也。」道遇戍卒,皆遣歸,他役盡罷。適上元張燈,城門三夕不閉,得邛部川譯
 人始造此語者,梟首境上,而流其餘黨,蜀人遂安。



 復以三司使召。方西鄙用兵,兩蜀多所調發,方平為奏免橫賦四十萬,減鑄鐵錢十餘萬緡。又建言:「國家都陳留,當四通五達之道,非若雍、洛有山川足恃,特倚重兵以立國耳。兵恃食,食恃漕運,以汴為主,汴帶引淮、江,利盡南海。天聖已前,歲調民浚之,故水行地中。其後,淺妄者爭以裁減役費為功,汴日以塞,今仰而望焉,是利尺寸而喪丘山也。」乃畫上十四策。富弼讀其奏,漏盡十刻,帝稱
 善。弼曰:「此國計大本,非常奏也。」悉如其說行之。



 遷尚書左丞、知南京。未幾,以工部尚書帥秦州。諜告夏人將壓境,方平料簡士馬,聲言出塞。已而寇不至,言者論其輕舉,曾公亮曰:「兵不出塞,何名輕舉?寇之不得至,有備故也。倘罪之,後之邊臣,將不敢為先事之備矣。」方平不自安,請知南京。



 英宗立,遷禮部尚書,請知鄆州。還,為學士承旨。帝不豫,召至福寧殿,帝馮幾言,言不可辨。方平進筆請,乃書云:「明日降詔,立皇太子。」方平抗聲曰:「必穎
 王也,嫡長而賢,請書其名。」帝力疾書之,乃退草制。



 神宗即位,召見,請約山陵費,帝曰:「奉先可損乎?」對曰:「遺制固云,以先志行之,可謂孝矣。」又請差減錫賚,以乾興為準,費省什七八。方平進詔草,帝親批之,曰:「卿文章典雅,煥然有三代風,又善以豐為約,意博而辭寡,雖《書》之訓誥,殆無加也。」其見稱重如此。



 拜參知政事。御史中丞司馬光疏其不當用,不聽。光解中丞,曾公亮議用王安石,方平以為不可。數日,遭父憂,服闋,以觀文殿學士留守西
 京。入覲,留判尚書都省,力請知陳州。安石行新法,方平陛辭,極論其害,曰:「民猶水也,可以載舟,亦可以覆舟;兵猶火也,弗戢必自焚。若新法卒行,必有覆舟、自焚之禍。」帝憮然。



 韓絳主西師,慶卒亂,京西轉運使令一路各會兵於州,民大駭。方平持檄不下而奏之,帝曰:「守臣不當爾邪!」命罷諸郡兵。召為宣徽北院使,留京師。王安石深沮之,以為青州。未行,帝問祖宗御戎之要,對曰:「太祖不勤遠略,如靈夏、河西,皆因其酋豪,許之世襲;環州董遵
 誨、西山郭進、關南李漢超,皆優其祿賜,寬其文法。諸將財力豐而威令行,間諜精審,吏士用命,故能以十五萬人而獲百萬之用。及太宗謀取燕薊,又內徙李彞興、馮暉,於是朝廷始旰食矣。真宗澶淵之克,與契丹盟,至今人不識兵革。三朝之事如此。近歲疆埸之臣,乃欲試天下於一擲,事成徼利,不成詒患,不可聽也。」帝曰:「慶歷以來,卿知之乎?元昊初臣,何以待之?」對曰:「臣時為學士,誓詔封冊,皆出臣手。」帝曰:「卿時已為學士,可謂舊德矣。」



 契
 丹泛使蕭禧來議疆事,臨當辭,臥驛中不起。方平謂樞密使吳充曰:「但令主者日致饋勿問,且使邊郡檄其國可也。」充啟從之,禧即行。除中太一宮使。



 王安石弛銅禁,奸民日銷錢為器,邊關海舶不復譏錢出,錢日耗。方平極論其害,請詰安石:「舉累朝之令典,一旦削除之,其意安在?」帝頗採其言,而方平求去。進使南院,判應天府。帝曰:「朕欲卿與韓絳共事,而卿論政不同;欲置卿樞密,而卿論兵復異。卿受先帝末命,訖無以副朕意乎?」遂行。



 高
 麗使過府,長吏當送迎,方平言:「臣班視二府,不可為陪臣屈。」詔但遣少尹。王師征安南,方平言:「舉西北壯士健馬,棄之炎荒,其患有不可勝言者。若師老費財,無功而還,社稷之福也。」後皆如其言。



 新法鬻河渡坊場,司農並及祠廟,宋閼伯、微子廟皆為賈區。方平言:「宋王業所基,閼伯封於商丘,以主大火;微子為始封之君,是二祠者,亦不得免乎?」帝震怒,批牘尾曰:「慢神辱國,無甚於斯!」於是天下祠廟皆得不鬻。數請老,以太子少師致仕。官制
 行,廢宣徽使,獨命領之如故。哲宗立,加太子太保。元祐六年,薨,年八十五。贈司空。遺令毋請謚,尚書右丞蘇轍為請,乃謚曰文定。



 方平慷慨有氣節,既告老,論事益切,至於用兵、起獄,尤反復言之。且曰:「臣且死,見先帝地下,有以借口矣。」平居未嘗以言徇物、以色假人。守蜀日,得眉山蘇洵與其二子軾、轍,深器異之。嘗薦軾為諫官。軾下制獄,又抗章為請,故軾終身敬事之,敘其文,以比孔融、諸葛亮。晚,受知神宗。王安石方用事,嶷然不小屈,以
 是望高一時。守宋都日,富弼自亳移汝,過見之曰:「人固難知也。」方平曰:「謂王安石乎?亦豈難知者!方平頃知皇祐貢舉,或稱其文學,闢以考校。既入院,凡院中之事,皆欲紛更。方平惡其人,檄使出,自是未嘗與語也。」弼有愧色,蓋弼素亦善安石云。



 王拱辰,字君貺,開封咸平人。元名拱壽,年十九,舉進士第一,仁宗賜以今名。通判懷州,入直集賢院,歷鹽鐵判官、修起居注、知制誥。慶歷元年,為翰林學士。



 契丹使劉
 六符嘗謂賈昌朝曰:「塘濼何為者?一葦可杭,投棰可平;不然,決其堤,十萬土囊,即可路矣。」仁宗以問拱辰,對曰:「兵事尚詭,彼誠有謀,不應以語我,此誇言爾。設險守國,先王不廢,而祖宗所以限敵人也。」至是,又使六符來,求關南十縣,斥太宗伐燕為無名,舉朝莫知所答。拱辰曰:「王師征河東,契丹既通使,而寇石嶺關以援賊。太宗怒,遂回軍伐之,豈謂無名?」乃作報書曰:「既交石嶺之鋒,遂有薊門之役。」契丹得報,遂繼好如初。帝喜,謂輔臣曰:「非
 拱辰深練故實,殆難答也。」



 權知開封府,拜御史中丞。夏竦除樞密使,拱辰言:「竦經略西師,無功稱而歸。今置諸二府,何以厲世?」因對,極論之。帝未省,遽起,拱辰前引裾,乃納其說,竦遂罷。又言:「滕宗諒在慶州,所為不度,而但降秩守虢,懼邊臣則效,宜施重責。」未聽,即家居,求自貶。乃徙宗諒岳州,敕拱辰赴臺。入見,帝曰:「言事官第自舉職,勿以朝廷未行為沮己,而輕去以沽名。自今有當言者,宜力陳毋避。」



 僧紹宗以鑄佛像惑眾,都人競投金冶
 中,宮掖亦出貲佐之。拱辰言:「西師宿邊,而財費於不急,動士心,起民怨。」詔亟禁之。蘇舜欽會賓客於進奏院,王益柔醉作《傲歌》,拱辰風其僚魚周詢、劉元瑜舉劾之。兩人既竄廢,同席者俱逐。時杜衍、範促淹為政,多所更張,拱辰之黨不便。舜欽、益柔皆仲淹所薦,而舜欽,衍婿也,故因是傾之,由此為公議所薄。



 復以翰林學士權三司使。坐舉富民鄭旭,出知鄭州,徙澶、瀛、並三州。數歲還,為學士承旨兼侍讀。帝於邇英閣置《太玄經》、蓍草,顧曰:「朕
 每閱此。卿亦知其說乎?」拱辰具以對,且曰:「願陛下垂意《六經》,旁採史策,此不足學也。」



 至和三年,復拜三司使。聘契丹,見其主混同江,設宴垂釣,每得魚,必酌拱辰酒,親鼓琵琶以侑飲。謂其相曰:「此南朝少年狀元也,入翰林十五年,故吾厚待之。」使還,御史趙抃論其輒當非正之禮,「異時北使援此以請,將何辭拒之?」湖南轉運判官李章、知潭州任顓市死商真珠,事敗,具獄上,拱辰悉入珠掖庭。抃並劾之。除宣徽北院使,抃言:「宣徽之職,本以待
 勛勞者,唯前執政及節度使得為之,拱辰安得污此選?」乃以端明殿學士知永興軍,歷泰定二州、河南大名府,積官至吏部尚書。



 神宗登極,恩當轉僕射,歐陽修以為此宰相官,不應序進,但遷太子少保。熙寧元年,復以北院使召還。王安石參知政事,惡其異己,乘二相有故,出為應天府。八年,入朝,為中太一宮使。



 元豐初,轉南院使,賜金方團帶。再判大名,改武安軍節度使。三路籍民為保甲,日聚而教之,禁令苛急,往往去為盜,郡縣不敢以
 聞。拱辰抗言其害曰:「非止困其財力,奪其農時,是以法驅之,使陷於罪罟也。浸淫為大盜,其兆已見。縱未能盡罷,願財損下戶以紓之。」主者指拱辰為沮法,拱辰曰:「此老臣所以報國也。」上章不已。帝悟,於是第五等戶得免。



 哲宗立,徙節彰德,加檢校太師。是年薨,年七十四。贈開府儀同三司,謚懿恪。



 論曰:方平、拱辰之才,皆較然有過人者,而不免司馬光、趙抃之論。豈其英發之氣,勇於見得,一時趨鄉未能盡
 適於正與?及新法行,方平痛陳其弊,拱辰爭保甲,言尤剴切,皆諤諤不少貶,為國老成,望始重矣。若方平識王安石於闢校貢舉之時,而知其後必亂政,其先見之明,無忝呂誨云。



 張昪字杲卿,韓城人。舉進士,為楚邱主簿。南京留守王曾稱其有公輔器。累官度支員外郎。夏竦經略陜西,薦其才,換六宅使、涇原秦鳳安撫都監。未幾,以母老,求歸故官,得知絳州,改京西轉運使。知鄧州,又以母辭。或指
 為避事,範仲淹言於朝曰:「張忭豈避事者?」乃許歸養。歷戶部判官、開封府推官,至知雜御史。



 張堯佐緣恩驟用,知開封府;內侍楊懷敏夜直禁中,而衛士為變,皆極論之。忭性質樸,不善擇言,至斥張貴妃為一婦人,謂懷敏得志,將不減劉季述。仁宗讀之不懌,以語陳升之。升之曰:「此忠直之言,不激切,則聖意不可回矣。」帝乃解。以天章閣待制知慶州,改龍圖閣直學士、知秦州。



 初,青唐蕃部藺氈,世居古渭,積與夏人有隙,懼而獻其地。攝帥範
 祥無遠慮,亟城之。諸族畏其逼,舉兵叛。忭至,請棄勿城。詔戶部副使傅求審視之,以為不可棄,與忭議殊。先是,副總管劉渙討叛羌,逗撓不時進,忭命他將郭恩代之,羌乃潰去。渙黜其功,讕訟恩多殺老稚,以撼忭。朝廷命張方平守秦,徙渙涇原,亦徙忭青州。將罪忭,方平辭曰:「渙、忭有階級,今互言而兩罷帥,不可為也。」忭乃復留。



 至和二年,召兼侍讀,拜御史中丞。劉沆在相位,以御史範師道、趙抃嘗攻其惡,陰欲出之。忭曰:「天子耳目之官,奈
 何用宰相怒而斥?」上章力爭之,沆竟罷去。帝見忭指切時事無所避,謂曰:「卿孤立,乃能如是。」對曰:「臣仰托聖主,致位侍從,是為不孤。今陛下之臣,持祿養望者多,而赤心謀國者少,竊以為如陛下乃孤立爾。」帝為之感動。



 契丹主宗真遣使繼其畫像來,求帝畫像,未報而死。子洪基立,以為請,詔忭報聘,諭使更致新主像。契丹欲先得之,忭曰:「昔文成以弟為兄屈,尚先致敬,況今為伯父哉!」遂無以奪,乃復以洪基像來。嘉祐三年,擢樞密副使,遷
 參知政事、樞密使。忭愛惜官資,凡內降所與,多持不下。見帝春秋高,前後屢進言儲嗣事,卒與韓琦同決策。



 英宗立,請老,帝曰:「太尉勤勞王家,詎可遽去?」但命五日一至院,進見無蹈舞。司馬光上疏言:「近歲以來,大臣年高者皆不敢自安其位,言事者欲以為名,又從而攻之。使其人無可取,雖少壯何為?果有益於時,雖老何傷?忭為人忠謹清直,不可干以私,若使且居其位,於事亦未有曠廢也。」忭請不已,始賜告,令養疾,遂以彰信軍節度使、
 同中書門下平章事判許州,改鎮河陽三城。拜太子太師致仕。熙寧十年薨,年八十六。贈司徒兼侍中,謚曰康節。



 趙概,字叔平,南京虞城人。少篤學自力,器識宏遠,為一時名輩稱許。中進士第,通判海州,為集賢校理、開封府推官。奏事殿中,仁宗面賜銀緋。



 出知洪州,州城西南薄章江,有泛溢之虞,概作石堤二百丈,高五丈,以障其沖,水不為患。僚吏鄭陶、饒奭挾持郡事,為不法,前守莫能
 制。州之歸化卒,皆故時群盜。奭造飛語曰:「卒得廩米陳惡,有怨言,不更給善米,且生變。」概不答。卒有自容州戍逃歸而犯夜者,斬之以徇,因收陶、奭抵罪,闔府股慄。



 加直集賢院、知青州。坐失舉澠池令張誥免,久乃起,監密州酒。知滁州,山東有寇李二過境上,告人曰:「我東人也,公嘗為青州,民愛之如父母,我不忍犯。」率眾去。



 召修起居注。歐陽修後至,朝廷欲驟用之,難於越次。概聞,請郡,除天章閣待制、糾察在京刑獄,修遂知制誥。逾歲,概始
 代之。郊祀,當任子、進階爵,乞回其恩,封母郡太君。宰相謂曰:「君即為學士,擬封不久矣。」概曰:「母年八十二,願及今拜君賜以為榮。」乃許之,後遂為例。



 蘇舜欽等以群飲逐,概言:「預會者皆館閣名士,舉而棄之,觖士大夫望,非國之福也。」不報。求知蘇州,終母喪,入為翰林學士。聘契丹,契丹主會獵,請賦《信誓如山河詩》」詩成,親酌玉杯為概勸,且授侍臣劉六符素扇,寫之納袖中,其禮重如此。還,兼侍讀學士。諫官郭申錫論事忤旨,帝欲加罪,概曰:「
 陛下始面諭申錫毋面從,今黜之,何以示天下?」乃止。



 以龍圖閣學士知鄆州、應天府,代韓絳為御史中丞。絳以論張茂實不宜典宿衛罷,概至,首言之,茂實竟去。御藥院內臣有寄資至團練使者,謂之暗轉。概請明限以年,詔俟出院優遷之,毋得累寄。擢樞密使、參知政事。數以老求去。熙寧初,拜觀文殿學士、知徐州。自左丞轉吏部尚書,前此,執政遷官,未有也。以太子少師致仕,退居十五年,嘗集古今諫爭事,為《諫林》百二十卷上之。神宗賜詔
 曰:「請老而去者,類以聲問不至朝廷為高。唯卿有志愛君,雖退處山林,未嘗一日忘也。當置於坐右,時用省閱。」元豐六年薨,年八十八。贈太子太師,謚曰康靖。



 概秉心和平,與人無怨怒。雖在事如不言,然陰以利物者為不少,議者以比劉寬、婁師德。坐張誥貶六年,念之終不衰,誥死,恤其家備至。歐陽修遇概素薄,又躐知制誥,及修有獄,概獨抗章明其罪,言為仇者所中傷,不可以天下法為人報怨。修得解,始服其長者。為鄆州時,吏按前守
 馮浩侵公使錢三十萬,當以職田租償。概知其貧,為代以己奉。其平生所為類此。



 概初名禋,嘗夢神人金書名簿有「趙概」,遂更云。



 胡宿,字武平,常州晉陵人。登第,為揚子尉。縣大水,民被溺,令不能救,宿率公私船活數千人。以薦為館閣校勘,進集賢校理。通判宣州,囚有殺人者,將抵死,宿疑而訊之,囚憚棰楚不敢言。闢左右復問,久乃云:「旦將之田,縣吏縛以赴官,莫知其故。」宿取具獄翻閱,探其本辭,蓋婦
 人與所私者殺其夫,而執平民以告也。



 知湖州,前守滕宗諒大興學校,費錢數十萬。宗諒去,通判、僚吏皆疑以為欺,不肯書歷。宿誚之曰:「君輩佐滕侯久矣,茍有過,盍不早正?乃陰拱以觀,俟其去而非之,豈昔人分謗之意乎?」坐者大慚謝。其後湖學為東南最,宿之力為多。築石塘百里,捍水患,民號曰胡公塘,而學者為立生祠。



 久之,為兩浙轉運使。召修起居注、知制誥。入內都知楊懷敏坐衛士之變,斥為和州都監,未幾,召入復故職。宿封還
 詞頭,且言:「懷敏得不窮治誅死,已幸,豈宜復在左右?」命遂寢。



 慶歷六年,京東、兩河地震,登、萊尤甚。宿兼通陰陽五行災異之學,乃上疏曰:「明年丁亥,歲之刑德,皆在北宮。陰生於午,而極於亥。然陰猶強而未即伏,陽猶微而不能勝,此所以震也。是謂龍戰之會,其位在乾。若西北二邊不動,恐有內盜起於河朔。又登、萊視京師,為東北少陽之位,今二州置金坑,多聚民鑿山谷,陽氣耗洩,故陰乘而動。宜即禁止,以寧地道。時以為迂闊。明年,王則
 果以貝州叛。皇祐五年正月,會靈宮災,是歲冬至,郊,以二帝並配。明年大旱,宿言:「五行,火,禮也。去歲火而今又旱,其應在禮,此殆郊丘並配之失也。」即建言並配非古,宜用迭配如初。時議者謂士大夫言,七十當致仕,其不知止者,請令有司按籍舉行之。宿以為非優老之義,當少緩其期法:武吏察其任事與否,勿斷以年;文吏使得自陳而全其節。及言皇祐新樂與舊樂難並用;禮部間歲一貢士不便,當用三年之制。皆如其言。



 唐介貶嶺南,
 帝遣中使護以往。宿言:「事有不可測,介如不幸道死,陛下受殺直臣之名。」帝悟,追還使者。遷翰林學士,知審官、刑院。李仲昌開六塔河,民被害,詔獄薄其罪。宿請斬以謝河北,仲昌由是南竄。袞國公主下降,將行冊禮。宿諫曰:「陛下昔封兩長主,未嘗冊命,今施之愛女,殆非漢明帝所謂『我子豈得與先帝子等』之義也。」



 涇州卒以折支不時給,出惡言,且欲相扇為亂。既置於法,乃命劾三司吏。三司使包拯護弗遣。宿曰:「涇卒固悖慢,然當給之物,
 越八十五日而不與,計吏安得為無罪?拯不知自省,公拒制命,紀綱益廢矣。」拯懼,立遣吏。韓琦守並州,請復其節鎮。宿言:「參、商為仇讎之星。國家受命於商丘,而參為晉地。今欲崇晉,非國之利也。宋興削平四方,並最後服,故太宗不使列於方鎮,八十年矣,宜如故便。」議遂止。後琦秉政,卒復之。



 拜樞密副使。曾公亮任雄州,趙滋顓治界河事。宿言於英宗曰:「憂患之來,多藏於隱微,而生於所忽。自滋守邊,北人捕魚伐葦,一切禁絕,由此常與鬥
 爭。南北通好六十載,內外無患,近年邊遽來上,不過侵誣尺寸,此城砦之吏移文足以辨詰,何至於興甲兵哉?今搢紳中有恥燕薊外屬者,天時人事未至,而妄意難成之福。願守兩朝法度,以惠養元元,天下幸甚。」宿以老,數乞謝事。治平三年,罷為觀文殿學士、知杭州。明年,以太子少師致仕,未拜而薨,年七十二。贈太子太傅,謚曰文恭。



 宿為人清謹忠實,內剛外和,群居不嘩笑,與人言,必思而後對。故臨事重慎,不輒發,發亦不可回止。居母
 喪三年,不至私室。其當重任,尤顧惜大體。在審官、刑院,擇詳議官,有在選中者,嘗監征榷,以水災負課。同列謂小累不足白,宿竟白之,而薦其才足用,仁宗聽納。同列退而誚曰:「公固欲白上,倘緣是不用,奈何?」宿曰:「彼之得否,不過一詳議官。宿平生以誠事主,今白首矣,忍以毫發欺乎?為之開陳,聽吾君自擇爾。」少與一僧善,僧有秘術,能化瓦石為黃金。且死,將以授宿,使葬之。宿曰:「後事當盡力,他非吾所冀也。」僧嘆曰:「子之志,未可量也。」
 其篤行自勵,至於貴達,常如布衣時。



 子宗炎,從子宗愈、宗回。



 宗炎字彥聖,由將作監主簿鎖廳登第。為國子大宗正丞、開封府推官、考功吏部郎中。舊制,選人改京官,舉將小絓吏議,輒尼不行。宗炎請先引見,俟舉者罪即追止,從之。



 哲宗崩,遼使來吊祭,宗炎以鴻臚少卿迓境上。使者不易服,宗炎以禮折之,須其聽命,乃相見。暨還,升為卿。初,父宿使遼,遼人重之。其後宗炎婿鄧忠臣迓客,客問:「中外嘗有充使者否?」忠臣以宿告,且言:「前使鴻臚,其
 子也。」客嘆:「胡氏世不乏人。」俄以直龍圖閣知穎昌府,歷密州而卒。



 宗炎善為詩,藻思清婉。歐陽修守亳,與客游郡圃,或誦其詩,修賞味不已,以為有鮑、謝風致。其重之如此。



 宗愈字完夫,舉進士甲科,為光祿丞。宿得請杭州,英宗問:「子弟誰可繼者?」以宗愈對。召試學士院。



 神宗立,以為集賢校理。久之,兼史館檢討,遂同知諫院。修內卒盜皇城器物,宗愈言:「唐長孫無忌不解佩刀入東上閣門,校
 尉論當死。今禁卒為盜,而入內都知不能覺察,願正其罪。」殿帥直廬在長慶門內,久而自置隸圉。宗愈曰:「嚴禁旅,所以杜奸宄也。奈何令私人得為之?萬一兇黠者竄名其間,將不可悔。請易募老卒。」



 王安石用李定為御史,宗愈言:「御史當用學士及丞、雜論薦,又須官博士、員外郎。今定以幕職不因薦得之,是殆一出執政意,即大臣不法,誰復言之?」蘇頌、李大臨不草制,坐絀;宗愈又爭之,安石怒,出通判真州。歷提點河東刑獄、開封府推官、吏
 部右司郎中。



 元祐初,進起居郎、中書舍人、給事中、御史中丞。時更定役法,書成,衙校募不足者,聽差入等戶。宗愈言:「法貴均一,若持兩端,則於文有害。是乃差法,非募法也。請刪之。」



 哲宗嘗問朋黨之弊,對曰:「君子指小人為奸,則小人指君子為黨。君子,蓋義之與比者。陛下能擇中立之士而用之,則黨禍熄矣。」明日,具《君子無黨論》以進。拜尚書右丞。於是諫議大夫王覿論其不當,而劉安世、韓川、孫覺等合攻之,朝廷依違。逾年,出覿潤州,而言
 者愈力。乃罷為資政殿學士、知陳州,徙成都府,蜀人安其政。召為禮部尚書,遷吏部,卒,年六十六。贈左銀青光祿大夫。



 宗回字醇夫,用蔭登第,為編修敕令官、司農寺乾當公事、京西轉運判官、提點刑獄、京東陜西轉運使、吏部郎中。紹聖初,以直龍圖閣知桂州,進寶文閣待制。坐系平民死,降集賢殿修撰、知隨州,改秦州、慶州,復為待制。



 先是,熙河將王贍下邈川有功,帥孫路不樂贍,奪其兵與
 王愍。朝廷知之,以宗回代路,加直學士。時青唐瞎征內附,而心牟欽氈勒兵立別酋隴拶,還其地,勢復張。瞎征大懼,自髡為僧以祈免。王贍怨孫路,因言青唐不煩兵可下。至,則駐宗哥城不進。宗回怒,日夜檄趣之,且戒贍曰:「青唐兵甚弱,隴拶稚子,何能為,而怯懦逗遛,吾將以軍法從事。」又遣王愍復至邈川,聲言代贍。贍懼,乃率步騎掩青唐,據之,隴拶降。詔以青唐為鄯州,邈川為湟州。未幾,屬羌郎阿章叛,拒官軍。宗回遣將王吉、魏釗討之,
 皆敗死。又遣鈐轄種樸往。樸言:「賊鋒方銳,且盛寒,宜少緩師。」宗回不聽,督之急。樸不得已,行,亦敗死。於是轉運判官秦希甫言湟、鄯難守,以為棄之便。事下宗回,宗回持不可,希甫罷去。會徽宗棄鄯州,於是任伯雨再疏其罪,奪職知蘄州。



 還,為待制。歷慶、渭、陳、延、澶州。兄宗愈入黨籍,宗回亦罷郡。居亡何,錄其堅守湟、鄯之議,起知秦州。進樞密直學士,徙永興、鄭州、成德軍,復坐事去。大觀中卒,贈銀青光祿大夫。



 胡氏自宿始大,及宗愈仍世執
 政,其後子孫至侍從、九卿者十數,遂為晉陵名族。



 論曰:張忭清忠諒直,趙概雅量過人,胡宿學通天人之奧,考其立朝大節,皆磊落為良執政。宗愈仍居右轄,而學術視宿則有間矣。宗回非邊將材,其守河湟之議,蓋以趣種樸於死,蘄合上意,以解其責爾。若胡氏之世大也,殆脫萬人於水死,而陰德之所致與?



\end{pinyinscope}