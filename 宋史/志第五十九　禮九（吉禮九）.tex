\article{志第五十九 禮九(吉禮九)}

\begin{pinyinscope}

 宗廟之制



 宗廟之制。建隆元年,有司請立宗廟,詔下其議。兵部尚書張昭等奏:「謹案堯、舜、禹皆立五廟,蓋二昭二
 穆與其始祖也。有商建國,改立六廟,蓋昭穆之外,祀契與湯也。周立七廟,蓋親廟之外,祀太祖與文王、武王也。漢初立廟,悉不如禮。魏、晉始復七廟之制,江左相承不改。然七廟之室,隋文但立高、曾、祖、檷四廟而已。唐因立親廟,梁氏而下,不易其法。稽古之道,斯為折衷。伏請追尊高、曾四代,崇建廟室。」於是判太常寺竇儼奏上皇高祖文安府君曰文獻皇帝,廟號僖祖;皇曾祖中丞府君曰惠元皇帝,廟號順祖;皇祖驍衛府君曰簡恭皇帝,廟號翼祖;
 皇考武清府君曰昭武皇帝,廟號宣祖;皇高祖妣崔氏曰文懿皇后;皇曾祖妣桑氏曰惠明皇后;皇祖妣京兆郡太夫人劉氏曰簡穆皇后。太祖御崇元殿,備禮冊四親廟,奉安神主,行上謚之禮。二年十月,祔明憲皇后杜氏於宣祖室。



 太平興國二年,有司言:「唐制,長安太廟,凡九廟,同殿異室。其制:二十一間皆四柱,東西夾室各一,前後面各三階,東西各二側階。本朝太廟四室,室三間。今太祖升祔,共成五室,請依長安之制,東西留夾室外,
 餘十間分為五室,室二間。」從之。四月己卯,奉神主祔廟,以孝明皇后王氏配。



 至道三年十一月甲子,奉太宗神主祔廟,以懿德皇后符氏配。咸平元年,判太常禮院李宗訥等言:「僖祖稱曾高祖,太祖稱伯;文懿、惠明、簡穆、昭憲皇后並稱祖妣,孝明、孝惠、孝章皇后並稱伯妣。按《爾雅》有考妣、王父母、曾祖王父母、高祖王父母及世父之別。以此觀之,唯父母得稱考妣。今請僖祖止稱廟號,順祖而下,即依《爾雅》之文。」事下尚書省議,戶部尚書張齊
 賢等言:「《王制》『天子七廟』。謂三昭三穆與太祖之廟而七。前代或有兄弟繼及,亦移昭穆之列,是以《漢書》『為人後者為之子』,所以尊本祖而重正統也。又《禮》云:『天子絕期喪。』安得宗廟中有伯氏之稱乎?其唐及五代有所稱者,蓋禮官之失,非正典也。請自今有事於太廟,則太祖並諸祖室,稱孝孫、孝曾孫嗣皇帝;太宗室,稱孝子嗣皇帝。其《爾雅》『考妣』、『王父』之文,本不為宗廟言也。歷代既無所取,於今亦不可行。」



 詔下禮官議。議曰:「按《春秋正義》『躋魯僖
 公』云:『禮,父子異昭穆,兄弟昭、穆同。』此明兄弟繼統,同為一代。又魯隱、桓繼及,皆當穆位。又《尚書》盤庚有商及王,《史記》云陽甲至小乙兄弟四人相承,故不稱嗣子而曰及王,明不繼兄之統也。又唐中、睿皆處昭位,敬、文、武昭穆同為一世。伏請僖祖室止稱廟號,後曰祖妣,順祖室曰高祖,後曰高祖妣,翼祖室曰曾祖,後曰曾祖妣,祝文皆稱孝曾孫。宣祖室曰皇祖考,後曰皇祖妣,祝文稱孝孫。太祖室曰皇伯考妣,太宗室曰皇考妣。每大祭,太祖、
 太宗昭、穆同位,祝文並稱孝子。其別廟稱謂,亦請依此。」



 詔都省復集議,曰:「古者,祖有功,宗有德,皆先有其實而後正其名。今太祖受命開基,太宗纘承大寶,則百世不祧之廟矣。豈有祖宗之廟已分二世,昭穆之位翻為一代?如臣等議,禮『為人後者為之子』,以正父子之道,以定昭、穆之義,則無疑也。必若同為一代,則太宗不得自為世數,而何以得為宗乎?不得為宗,又何以得為百世不祧之主乎?《春秋正義》亦不言昭。穆不可異,此又不可以
 為證也。今若序為六世,以一昭一穆言之,則上無毀廟之嫌,下有善繼之美,於禮為大順,於時為合宜,何嫌而謂不可乎?」翰林學士宋湜言:「三代而下,兄弟相繼則多,昭、穆異位,未之見也。今詳都省所議,皇帝於太祖室稱孫,竊有疑焉。」



 詔令禮官再議。禮官言:「按《祭統》曰:『祭有昭、穆者,所以別父子遠近長幼親疏之序而無亂也。』《公羊傳》,公孫嬰齊為兄歸父之後,《春秋》謂之仲嬰齊。何休云:『弟無後兄之義,為亂昭穆之序,失父子之親,故不言仲
 孫,明不以子為父孫。』晉賀循議兄弟不合繼位昭穆云:『商人六廟,親廟四,並契、湯而六,比有兄弟四人相襲為君者,便當上毀四廟乎?如此,四世之親盡,無復祖檷之神矣。』溫嶠議兄弟相繼、藏主夾室之事云:『若以一帝為一世,則當不得祭於檷,乃不及庶人之祭也。』夫兄弟同世,於恩既順,於義無否。玄宗朝禘袷,皇伯考中宗、皇考睿宗同列於穆位。德宗亦以中宗為高伯祖。晉王導、荀崧議『大宗無子,則立支子』,又曰『為人後者為之子』,無兄
 弟相為之文。所以舍至親取遠屬者,蓋以兄弟一體,無父子之道故也。竊以七廟之制,百王是尊。至於祖有功,宗有德,則百世不遷之廟也;父為昭,子為穆,則千古不刊之典也。今議者引《漢書》曰:『為人後者為之子。』殊不知弟不為兄後,子不為父孫,《春秋》之深旨。父謂之昭,子謂之穆,《禮記》之明文也。又按太宗享祀太祖二十有二載,稱曰『孝弟』,此不易之制,又安可追改乎?唐玄宗謂中宗為皇伯考,德宗謂中宗為高伯祖,則伯氏之稱,復何不
 可?臣等參議:自今合祭日,太祖、太宗依典禮同位異坐,皇帝於太祖仍稱孝子,餘並遵舊制。」



 景德元年,有司詳定明德皇太后李氏升祔之禮:「按唐睿宗昭成、肅明二後,先天初,以昭成配;開元末,以肅明祔。此時儒官名臣,步武相接,宗廟重事,必有據依。推之閨門,亦可擬議。晉驃騎將軍溫嶠有三夫人,嶠薨,詔問學官陳舒。舒謂秦、漢之後,廢一娶九女之制,妻卒更娶,無復繼室,生既加禮,亡不應貶。朝旨以李氏卒於嶠之微時,不沾贈典;王、
 何二氏追加章綬。唐太子少傅鄭餘慶將立家廟,祖有二夫人。禮官韋公肅議與舒同。略稽禮文,參諸故事,二夫人並祔,於理為宜。恭惟懿德皇后久從升祔,雖先後有殊,在尊親則一,請同列太宗室,以先後次之。」詔尚書省集議,咸如禮官之請,祔神主於太廟。



 乾興元年十月,奉真宗神主祔廟,以章穆皇后郭氏配。康定元年,直秘閣趙希言奏:「太廟自來有寢無廟,因堂為室,東西十六間,內十四間為七室,兩首各一夾室。按禮,天子七廟,親廟
 五、祧廟二。據古則僖、順二神當遷。國家道觀佛寺,並建別殿,奉安神御,豈若每主為一廟一寢。或前立一廟,以今十六間為寢,更立一祧廟,逐室各題廟號。扣寶神御物,宜銷毀之。」同判太常寺宋祁言:「周制有廟有寢,以像人君前有朝後有寢也。廟藏木主,寢藏衣冠。至秦乃出寢於墓側,故陵上更稱寢殿,後世因之。今宗廟無寢,蓋本於茲。鄭康成謂周制立二昭二穆,與太祖、文、武共為七廟,此一家之說,未足援正。自荀卿、王肅等皆云天子
 七廟,諸侯五,大夫三,士一,降殺以兩。則國家七世之數,不用康成之說也。僖祖至真宗方及六世,不應便立祧廟。自周、漢每帝各立廟,晉、宋以來多同殿異室,國朝以七室代七廟,相承已久,不可輕改。《周禮》:『天府掌祖廟之守藏。』寶物世傳者皆在焉。其神御法物、寶盝、扣床,請別為庫藏之。」自是室題廟號,而建神御庫焉。



 嘉祐年,仁宗將祔廟,修奉太廟使蔡襄上八室圖,為十八間。初,禮院請增廟室,孫抃等以為:「七世之廟,據父子而言,兄弟則
 昭、穆同,不得以世數之。廟有始祖、有太祖、有太宗、有中宗。若以一君為一世,則小乙之祭不及其父。故晉之廟十一室而六世,唐之廟十一室而九世。國朝太祖之室,太宗稱孝弟,真宗稱孝子,大行稱孝孫。而《禘袷圖》:太祖、太宗同居昭位,南向;真宗居穆位,北向。蓋先朝稽用古禮,著之祀典。大行神主祔廟,請增為八室,以備天子事七世之禮。」盧士宗、司馬光以為:「太祖已上之主,雖屬尊於太祖,親盡則遷。入漢元之世,太上廟主瘞於寢園;魏
 明之世,處士廟主遷於園邑;晉武祔廟,遷征西府君;惠帝祔廟,遷豫章府君。自是以下,大抵過六世則遷。蓋太祖未正東向,故上祀三昭三穆;巳正東向,則並昭、穆為七世。唐初祀四世,太宗增祀六世。及太宗祔廟,則遷弘農府君,高宗祔廟,又遷宣帝,皆祀六世,前世成法也。玄宗立九室祀八世,事不經見。若以太祖、太宗為一世,則大行祔廟,僖祖親盡,當遷夾室,祀三昭三穆,於先王典禮及近世之制,無不符合。」抃等復議曰:「自唐至周,廟制
 不同,而皆七世。自周以上,所謂太祖,非始受命之主,特始封之君而已。今僖祖雖非始封之君,要為立廟之祖,方廟數未過七世,遂毀其廟,遷其主,考之三代,禮未有此。漢、魏及唐一時之議,恐未合先王制禮之意。」乃存僖祖室以備七室。



 治平四年,英宗將祔廟,太常禮院請以神主祔第八室,祧藏僖祖及文懿皇后神主於西夾室。自仁宗而上,以次遞遷。翰林承旨張方平等議:「同堂八室,廟制已定,僖祖當祧,合於典禮。」乃於九月奉安八室
 神主,祧僖祖及後,祔英宗,罷僖祖諱及文懿皇后忌日。



 熙寧五年,中書門下言:「僖祖以上世次,不可得而知,則僖祖有廟,與商周契、稷疑無以異。今毀其廟而藏主夾室,替祖考之尊而下祔於子孫,殆非所以順祖宗孝心、事亡如存之義。請以所奏付兩制議,取其當者。」時王安石為相,不主祧遷之說,故復有是請。



 翰林學士元絳等上議曰:「自古受命之王,既以功德享有天下,皆推其本統以尊事其祖。故商、周以契、稷有功於唐、虞之際,故謂
 之祖有功,若必以有功而為祖,則夏后氏不郊鯀矣。今太祖受命之初,立親廟,自僖祖以上世次,既不可知,則僖祖之為始祖無疑矣。儻謂僖祖不當比契、稷為始祖,是使天下之人不復知尊祖,而子孫得以有功加其祖考也。《傳》曰:『毀廟之主,陳於太祖;未毀廟之主,皆升,合食於太祖。』今遷僖祖之主,藏於太祖之室,則是四祖袷祭之日,皆降而合食也。請以僖祖之廟為太祖,則合於先王禮意。」翰林學士韓維議曰:「昔先王有天下,跡其基業
 之所起,奉以為太祖。故子夏序《詩》,稱文、武之功起於後稷。後世有天下者,特起無所因,故遂為一代太祖。太祖皇帝功德卓然,為宋太祖,無少議者。僖祖雖為高祖,然仰跡功業,未見所因,上尋世系,又不知所以始,若以所事契、稷奉之,竊恐於古無考,而於今亦所未安。今之廟室與古殊制,古者每廟異宮,今祖宗同處一室,則西夾室在順祖之右,考之尊卑之次,似亦無嫌。」



 天章閣待制孫固請:「特為僖祖立室,由太祖而上,親盡迭毀之主皆
 藏之。當禘袷時,以僖祖權居東向之位,太祖順昭穆之列而從之,取毀廟之主而合食,則僖祖之尊自有所申。以僖祖立廟為非,則周人別廟姜嫄,不可謂非禮。」秘閣校理王介請依《周官》守祧之制,創祧廟以奉僖祖,庶不下祔子孫夾室,以替遠祖之尊。



 帝以維之說近是,而安石以維言夾室在右為尊為非理,帝亦然之。又安石以尊僖祖為始祖,則郊祀當以配天,若宗祀明堂,則太祖、太宗當迭配帝。又疑明堂以英宗配天,與僖祖為非始
 祖之說。遂下禮官詳定。



 同判太常寺兼禮儀事張師顏等議:「昔商、周之興,本於契、稷,故奉之為太祖。後世受命之君,功業特起,不因先代,則親廟迭毀,身自為祖。鄭玄云『夏五廟無太祖,禹與二昭二穆而已』,張薦云『夏后以禹始封,遂為不遷之祖』是也。若始封世近,上有親廟,則擬祖上遷,而太祖不毀。魏祖武帝則處士迭毀,唐祖景帝則弘農迭毀,此前世祖其始封之君,以法契、稷之明例也。唐韓愈有言:『事異商、周,禮從而變。』晉瑯邪王德文
 曰:『七廟之義,自由德厚流光,享祀及遠,非是為太祖申尊祖之祀。』其說是也。禮,天子七廟,而太祖之遠近不可以必,但云三昭三穆與太祖之廟而七,未嘗言親廣之首,必為始祖也。國家以僖祖親盡而祧之,奉景祐之詔,以太祖為帝者之祖,是合於禮矣。張昭、任徹之徒,不能遠推隆極之制,因緣近比,請建四廟,遂使天子之禮下同諸侯。若使廟數備六,則更當上推兩世,而僖祖次在第三,亦未可謂之始祖也。謹按建隆四年,親郊崇配不
 及僖祖。開國以來,大祭虛其東向,斯乃祖宗已行之意。請略仿《周官》守祧之制,築別廟以藏僖祖神主,大祭之歲,祀於其室。太廟則一依舊制,虛東向之位。郊配之禮,則仍其舊。」



 同知太常禮院蘇梲請:「即景靈宮祔僖祖,即與唐祔獻、懿二祖於興聖、明德廟,禮意無異。」同判禮院周孟陽等言:「自僖祖而上,世次莫知,則僖祖為始祖無疑,宜以僖祖配感生帝。」章衡請:「尊僖祖為始祖,而次祧順祖,以合子為父屈之義。推僖祖侑感生之祀,而罷宣
 祖配位,以合祖以孫尊之義,餘且如舊制。」而馮京欲以太祖正東向之位,安石力主元絳初議,遂從之。帝問:「配天孰始?」安石曰:「宣祖見配感生帝,欲改以僖祖配。」帝然之。於是請奉僖祖神主為始祖,遷順祖神主夾室,以僖祖配感生帝祀。詔下太常禮院詳定儀注。安石本議以僖祖配天,帝不許,故更以配感生帝焉。



 元豐元年,詳定郊廟禮文所圖上八廟異宮之制,以始祖居中,分昭穆為左右。自北而南,僖祖為始祖;翼祖、太祖、太宗、仁宗為
 穆,在右;宣祖、真宗、英宗為昭,在左。皆南面北上。陸佃言:「太祖之廟百世不遷,三昭三穆,親盡則迭毀。如周以後稷為太祖,王季為昭,文王為穆,武王為昭,成王為穆,康王為昭,昭王為穆,其後穆王入廟,王季親盡而遷,則文王宜居昭位,武王宜居穆位,成王、昭王宜居昭位,康王、穆王宜居穆位,所謂父昭子穆是也。說者以昭常為昭,穆常為穆,則尊卑失序。」復圖上八廟昭穆之制,以翼祖、太祖、太宗、仁宗為昭,在左;宣祖、真宗、英宗為穆,在右。皆
 南面北上。



 何洵直圖上八廟異宮,引熙寧儀:僖祖正東向之位,順祖、宣祖、真宗、英宗南面為昭,翼祖、太祖、太宗、仁宗北面為穆,正得祖宗繼序、德厚流光之本意。又以晉孫毓、唐賈公彥言「始祖居中,三昭在左,南面西上;三穆在右,南面東上。」為兩圖上之。又援《祭法》,言:「翼祖、宣祖在二祧之位,猶同祖檷之廟,皆月祭之,與親廟一等,無親疏遠近之殺。順祖實去祧之主,若有四時祈禱,猶當就壇受祭。請自今二祧神主,殺於親廟,四時之祭,享嘗
 乃止,不及大烝,不薦新物。去祧神主,有禱則為壇而祭,庶合典禮。」又請建新廟於始祖之西,略如古方明壇制。有詔,俟廟制成日取旨。



 三年,禮文所言:「古者宗廟為石室以藏主,謂之宗祏。夫婦一體,同幾共牢。一室之中,有左主、右主之別,正廟之主,各藏廟室西壁之中;遷廟之主,藏於太祖太室北壁之中,其埳去地六尺一寸。今太廟藏主之室,帝後異處,遷主仍藏西夾室,求之於禮,有所未合。請新廟成,並遵古制。」從之。二月,慈聖光獻皇后
 祔廟,前二日,告天地、社稷、太廟、皇后廟如故事。至日,奉神主先詣僖祖室,次翼祖室,次宣祖室,次太祖室,次太宗室。次太宗與懿德皇后、明德皇后同一祝,次享元德皇后。慈聖光獻皇后,異饌位、異祝,行祔謁禮。次真宗室,次仁宗室,次英宗室。禮畢,奉神主歸仁宗室。



 元豐六年六月,孝惠、孝章、淑德、章懷四后升祔,準章獻明肅、章懿二後,升祔禮畢,遞享太廟,止行升祔享禮及祭七祀,權罷孟冬薦享,仍以配繼先後為序。八年,禮部太常寺言:「
 詔書定七世八室之制。今神宗皇帝崇祔,翼祖在七世之外,與簡穆皇后祧藏於西夾室,置石室中。」十一月丁酉,祔神宗神主於第八室。自英宗上至宣祖以次升遷。紹聖元年二月,祔宣仁聖烈皇后於太廟。



 元符三年,禮部太常寺言:「哲宗升祔,宜如晉成帝故事,於太廟殿增一室,候祔廟日,神主祔第九室。」詔下侍從官議,皆如所言。蔡京議:「以哲宗嗣神宗大統,父子相承,自當為世。今若不祧遠祖,不以哲宗為世,則三昭四穆與太祖之廟
 而八。宜深考載籍,遷祔如禮。」陸佃、曾肇等議:「國朝自僖祖而下始備七廟,故英宗祔廟,則遷順祖,神宗祔廟,則遷翼祖。今哲宗於神宗,父子也,如禮官議,則廟中當有八世。況唐文宗即位則遷肅宗,以敬宗為一世,故事不遠。哲宗祔廟,當以神宗為昭,上遷宣祖,以合古三昭三穆之義。」先是,李清臣為禮部尚書,首建增室之議,侍郎趙挺之等和之。會清臣為門下侍郎,論者多從其議,惟京、佃等議異。二議既上,清臣辯說甚力,帝迄從焉。



 六月,
 禮部請用太廟東夾室奉安哲宗神主。太常少卿孫傑言:「先帝神主,錯之夾室,即是不得祔於正廟,與前詔增建一室之議不同。昨用嘉祐故事,專置使修奉,請以夾室奉安神主,亦與元置使之意相違。請如太常前議,增建一室。」尚書省以廟室未備,行禮有期,權宜升祔,隨即增修,比之前代設幄行事者,不為不至。詔依初旨行之,乃祔哲宗神主於夾室。



 崇寧二年,祧宣祖與昭憲皇后神主藏西夾室,居翼祖、簡穆皇後石室之次。五年,詔曰:「
 去古既遠,諸儒之說不同。鄭氏謂:『太祖及文、武不祧之廟與親廟四,為七。』是不祧之宗,在七廟之內。王氏謂:『非太祖而不毀,不為常數。』是不祧之宗,在七廟之外。本朝今已五宗,則七廟當祧者,二宗而已。遷毀之禮,近及祖考,殆非先王尊祖之意,宜令有司復議。」禮官言:「先王之制,廟止於七,後王以義起禮,乃有增置九廟者。」禮部尚書徐鐸又言:「唐之獻祖、中宗、代宗與本朝僖祖,皆嘗祧而復。今存宣祖於當祧之際,復翼祖於已祧之後,以備
 九廟,禮無不稱。」乃命鐸為修奉使,增太廟殿為十室。四年十二月,復翼祖、宣祖廟,行奉安禮,惟不用前期誓戒及亞、終獻之樂舞焉。



 高宗建炎二年,奉太廟神主於揚州壽寧寺。三年,幸杭州,奉安於溫州。紹興五年,司封郎中林待聘言:「太廟神主宜在國都。今新邑未奠,當如古行師載主之義,遷之行闕,以彰聖孝。」於是始建太廟於臨安,奉迎安置。



\end{pinyinscope}