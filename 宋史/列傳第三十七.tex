\article{列傳第三十七}

\begin{pinyinscope}

 馬全義子知節雷德驤子有終孫孝先曾孫簡夫王超子德用



 馬全義,幽州薊人。十餘歲學擊劍,善騎射。十五,隸魏帥範延光帳下。延光叛,晉祖征之,以城降,悉籍所部來上。
 全義在籍中,因補禁軍。以不得志,遂遁去。漢乾祐中,李守貞鎮河中,召置帳下。及守貞叛,周祖討之,全義每率敢死士,夜出攻周祖壘,多所殺傷。守貞貪而無謀,性多忌克,全義屢為畫策,皆不能用。城陷,遂變姓名亡命。



 周廣順初,世宗鎮澶淵,全義往事之。從世宗入朝,周祖召見,補殿前指揮使,謂左右曰:「此人忠於所事,昔在河中,屢挫吾軍,汝等宜效之。」世宗即位,遷右番行首。從世宗戰高平,以功遷散員指揮使。從征淮南,以功遷殿前指
 揮使、右番都虞候。恭帝即位,授鐵騎左第二軍都校,領播州刺史。



 宋初,歷內殿直都知、控鶴左廂都校,領果州團練使。從征李筠,筠退保澤州,城小而固,攻之未下,太祖患之,召全義賜食御榻前問計,對曰:「筠守孤城,若並力急攻,立可殄滅,儻緩之,適足長其奸爾。」太祖曰:「此吾心也。」即麾兵急擊之。全義率敢死士數十人乘城,攀堞而上,為飛矢貫臂,流血被體。全義拔鏃臨敵,士氣益奮,遂克其城。遷虎捷左廂都校,領睦州防禦使。又從征李
 重進,領控鶴、虎捷兩軍為後殿。賊平班師,錄功居多,改龍捷左廂都校,領江州防禦使,俄被疾。太祖遣太醫診視,仍諭密旨曰:「俟疾間,當授以河陽節制。」全義疾已亟,但叩頭謝。數日卒,年三十八。特贈檢校太保、大同軍節度使。子知節。



 知節字子元,幼孤。太宗時,以蔭補供奉官,賜今名。年十八,監彭州兵,以嚴蒞眾,眾憚之如老將。又監潭州兵,時何承矩為守,頗以文雅飾吏治,知節慕之,因折節讀書。
 雍熙間,護兵博州,契丹入邊,敗我師於君子館。先是,知節完城繕甲,儲積芻粟,吏民以為生事。既而契丹果至,以有備,引去。



 徙知定遠軍。時議調河南十三州之民輸餉,河北轉運使樊知古適至軍議事,知節曰:「軍少粟多,簸其紅腐,尚當得十之六七。」知古從之,果獲粟五十萬斛,分給諸屯,遂省河南之役。時部民入保避寇,卒有盜婦女首飾者,護軍止笞遣之。知節曰:「民避外患而來,反罹內寇,此而可恕,何以肅下?」即命斬之。知深、慶二州,遷
 西京作坊使。旋知梓州。李順之叛,詔與王繼恩同討賊。繼恩恃勢自任,惡知節不附己,遣守彭州,付以羸兵三百,彭之舊卒,悉召還成都。知節累請益兵,不從。賊眾十萬攻城,知節力戰,自晨抵晡,士多死,慨然嘆曰:「死賊手,非壯夫也。」即橫槊潰圍出,遲明,援兵至,復鼓噪入,賊遂潰去。太宗聞而嘆曰:「賊眾我寡,知節不易當也。」授益州鈐轄,加益、漢九州都巡檢使,遷內園使。會韓景祐帳下劉旰脅牙兵為亂,連下州縣,眾逾二千,知節領兵三百,
 追至蜀州,與戰,旰走邛州。知節曰:「賊破邛州,必乘勝渡江薄我,既息而後戰,官軍雖倍,制之亦勞,不如乘其弊急擊之,破之必矣。」遂行。次方井鎮,與旰遇,殺之無□類。



 咸平初,領登州刺史、知秦州。州嘗質羌酋支屬餘二十人,逾二紀矣。知節曰:「羌亦人爾,豈不懷歸?」悉遣之。羌人感之,訖終,更不犯塞。時州有銀坑,歲久礦竭,課額弗除,主吏破產,償之不足。知節請蠲之,章三上,乃允。遷西上閣門使,知益州兼本路轉運使。自乾德後,歲漕蜀物,動
 逾萬計。時籍富民以部舟運,坐沉覆破產者眾。知節請代督以省校而程其漕事,自是蜀人賴以免患。



 徙知延州兼鄜延駐泊部署。邊寇將至,方上元節,遽命張燈啟關,累夕宴樂。寇不測,即引去。會鎮州程德玄政事曠弛,徙知節代之。詔發澶、魏等六州糧輸定武,時兵交境上,知節曰:「糧之來,是資盜也。」止令於舟車所至收之,寇無所得而遁。



 車駕在澶淵,時王超擁兵數十萬屯真定,逗留不進。知節移書誚讓之,超始出兵,猶以中渡無橋為
 辭。知節預命度材,一夕而具。景德中,徙知定州,未幾,拜東上閣門使、樞密都承旨,擢拜簽書樞密院事。



 當是時,契丹已盟,中國無事,大臣方言符瑞,而知節每不然之,嘗言「天下雖安,不可忘戰去兵」以為戒。自陳年齒未衰,五七年間尚可驅策,如邊方有警,願預其行,但得副都部署名及良馬數匹、輕甲一聯足矣。上以為然,因命制銅鐵鎖子甲以賜焉。進宣徽北院使,加兼樞密副使。時王欽若為樞密使,知節薄其為人,遇事敢言,未嘗少屈。
 每廷議,得其不直,輒面詆之。時欽若寵顧方隆,知節愈不為下。



 大中祥符七年,出為穎州防禦使、知潞州。天禧初,移知天雄軍,召拜宣徽南院使、知樞密院事。以疾乞罷,除彰德軍留後、知貝州兼部署。將行,真宗閔其懼瘁,止命歸鎮。時上黨、大名之民已爭來迎謁。未幾,卒,年六十五。贈侍中,謚正惠。



 知節將家子,慷慨以武力智謀自許,又能好書,賓友儒者,所與善厚,必一時豪傑,論事謇謇未嘗有所顧忌,故聞其風者,亦知其為正直云。



 雷德驤,字善行,同州合陽人。周廣順三年舉進士,解褐磁州軍事判官。召為右拾遺,充三司判官,賜緋魚。顯德中,入受詔均定隨州諸縣民田屋稅,稱為平允。



 宋初,拜殿中侍御史,改屯田員外郎、判大理寺。其官屬與堂吏附會宰相趙普,擅增刑名,因上言,欲求見太祖以白其事。未引對,直詣講武殿奏之,辭氣俱厲。太祖詰之,德驤對曰:「臣值陛下日旰未食,方震威嚴爾。」帝怒,令左右曳出,詔置極典。俄怒解,黜為商州司戶參軍。刺史知德驤舊
 為省郎,以客禮之。及奚嶼知州,希宰相旨,至則倨受庭參。德驤不能堪,出怨言,嶼銜之。適有言德驤至郡為文訕上者,嶼召德驤與語,潛遣吏紿其家人取得之,即械系德驤,具狀以聞。太祖貸其罪,削籍徙靈武。數年,其子有鄰擊登聞鼓,訴中書不法事,趙普由是出鎮河陽。召德驤為秘書丞,俄分判御史臺三院事,又兼判吏部南曹。開寶七年,同知貢舉。太祖崩,以德驤為吳越國告哀使。還,遷戶部員外郎兼御史知雜事,改職方員外郎,充
 陜西、河北轉運使。歷禮部、戶部郎中,入為度支判官。



 太平興國四年,車駕征太原,為太原西路轉運使。六年,同知京朝官考課,俄遷兵部郎中。七年,以公累降本曹員外郎、出知懷州,未幾,復舊官,又命為兩浙轉運使。其子殿中丞有終亦為淮南轉運使,父子同日受詔,搢紳榮之。俄遷右諫議大夫。



 雍熙二年,征歸朝,同知京朝官考課,初,帝謂宰相曰:「朕前日閱班籍,擇官為河北轉運使,所患不能周知群臣履行。自今令德驤錄京朝官履歷
 功過之狀引對,既得漸識群臣,擇才委任,且使有官政者樂於召對,負瑕累者恥於顧問,可以為懲勸矣。」



 端拱初,遷戶部侍郎。會趙普再入相,宣制之日,德驤方立班,不覺墜笏,遽上疏,乞歸田里。太宗召見,安諭之,賜白金三千兩,罷知考課,止以本官奉朝請。會議事尚書省,乘酒叱起居員外郎鄭構為盜,御史奏劾,下御史臺案問,具伏,帝止令罰月奉而釋之。訖趙普出守西洛,帝終保全之。



 淳化二年,為其婿如京副使衛濯訟有鄰子秘書
 省校書郎孝先內亂,帝素憐德驤,恐暴揚其醜,不以孝先屬吏,止除名配均州。德驤坐失教,責授感德軍行軍司馬。並其子少府少監有終責授衡州團練副使。德驤因慚憤成疾,三年,卒,年七十五。有終為三司鹽鐵副使,表乞追復舊官,從之。



 德驤無文採,頗以強直自任,性褊躁,多忤物,不為士大夫所與。



 有鄰,開寶中,舉進士不第。其父既竄靈武,意宰相趙普擠抑之。時堂後吏胡贊、李可度在職歲久,或稱其請托受賕,而秘書丞王洞與
 德驤同年登第,有鄰每造謁洞,洞多以家事委之。一日,洞令有鄰市白金半鋌,因曰:「此令吾子知,要與胡將軍。」蓋謂贊也。時又有詔,應攝官三任解由全者許投牒有司,即得召試錄用。有鄰素與前攝上蔡主簿劉偉交游,知偉雖嘗三攝,而一任失其解由,偉造偽印,令其兄前進士侁書寫之,因是得試送銓。遂具章告其事,並下御史府按鞫。有鄰出入贊家,故其事多實。獄具,偉坐棄市,洞等並決杖除名,贊、可度仍籍其家。有鄰授秘書省正
 字,賜公服靴笏、銀鞍勒馬、絹百匹,自是累上疏密告人陰事。俄被病,白晝見偉入室,以杖棰其背。有鄰號呼聞於外,數日而死。賜德驤錢十萬,以給喪事。



 有終字道成,幼聰敏,以蔭補漢州司戶參軍。時侯陟典選,木強難犯,選人聽署於庭,無敢嘩者。有終獨抗言,願為大郡治獄掾,陟叱之曰:「年未三十,安可任此官?」有終不為沮。署萊蕪尉。知監、左拾遺劉祺以有終年少,頗易之。有終發其奸贓,祺坐罪杖流海島,以有終代知監事。
 先是,三司補吏為冶官,率以貲進,多恣橫。至是,受署者憚有終,率多避免。太宗即位,聞其名,遣內侍伍守忠同掌監事,且察其治跡。守忠至裁周月,即還奏有終強濟之狀,亟詔為大理寺丞。會德驤任陜西轉運,奏為解州通判,特許德驤不巡察是州。有終入奏鹽池利害,改贊善大夫,令還權知軍事,省通判。太平興國六年,遷殿中丞、知密州,徙淮南轉運副使,賜緋魚,改太常博士。時德驤主漕兩浙,往往省於境上,時人榮之。



 雍熙中,王師
 北征,命為蔚州飛狐路隨軍轉運使。入為鹽鐵判官,歷戶部、度支副使,賜金紫,出知升州。淳化初,就遷少府少監、知廣州。二年,女弟婿衛濯訟其家法不謹,有終坐親累,責授衡州團練副使,奪章服。俄丁外艱,行及許田,召歸,入對,賜錢八十萬,起為都官員外郎,歷度支、鹽鐵副使,復金紫。時以江南、嶺外茶鹽價不一,細民冒禁私販,多陷重闢。詔有終領江、淮、兩浙、荊湖、福建、廣南路茶鹽制置使,就出鹽產茶之地,以便宜裁制。使還,改工部郎中、
 知大名府,不逾月,復為少府少監,徙知江陵。



 李順之亂,王師西征,命與裴莊為峽路隨軍轉運使、同知兵馬事。調發兵食,規畫戎事,皆有節制。師行至峽中,遇盜格鬥,眾渴乏,會天雨,軍士以兜牟承水飲之,且行且戰,進至廣安軍,軍壘瀕江,三面樹柵。會夜陰晦,賊眾奄至,鼓噪舉火,士伍恐懼,有終安坐櫛發自若。賊圍既合,有終引奇兵出其後擊之,賊眾驚擾,赴水死者無算。就拜右諫議大夫、知益州。次簡州,寓佛舍,度賊必至,命左右重閉,召
 土人嚴更警備,初夕,間道而出。賊圍守數重,及壞寺入,惟擊柝者在焉。俄兼同招安使。賊平,改知許州。三年,改給事中、知並州。



 真宗嗣位,加工部侍郎。咸平二年,代還,知審刑院,俄授戶部使。三年,將巡師大名,遣有終乘驛先詣澶州督納糧草。車駕還,次德清軍,會益州奏至,神衛戍卒以正旦竊發,害兵馬鈐轄符昭壽,擁都虞候王均為亂,逐知州牛冕。即日,拜有終瀘州觀察使、知益州兼川峽兩路招安捉賊事。御廚使李惠、洛苑使石普、供
 備庫副使李守倫並為招安巡檢使,給步騎八千,命往招討。又以洺州團練使上官正為東川都鈐轄,西京作坊使李繼昌為峽路都鈐轄,崇儀副使高繼勛、王阮並為益州駐泊都監,供奉官、閣門祗候孫正辭為諸州都巡檢使。



 正月三日,均率眾陷漢州,進攻綿州,旬日不能下,趣劍門。先是,知劍州、秘書丞李士衡度寇必至,城不能守,悉徙官帑保劍門,焚其倉廩,及署榜招軍卒之流逸者,得數千人。已而賊果至,士衡與劍門都監、左藏庫副
 使裴臻逆擊之。時風雪連日,均眾無所掠,唯食敗糟,臻與戰,斬首數千級。賊眾疲劇宵遁,還保益州。士衡即馳騎入奏,上嘉之,拜士衡度支員外郎,賜緋;臻崇儀使、領峰州刺史,仍舊職。知蜀州、供奉官、閣門祗候楊懷忠聞變,即調鄉丁會十一路巡檢兵,刻期進討。蜀民不從賊者相率抗禦,儕伍謂之「清壇眾」。擇「清壇」之魁七十餘人,悉補巡檢將,遣判官高本馳驛以聞。十七日,懷忠率眾入益州,焚城北門,至三井橋。時均尚留劍門,與賊將魯
 麻胡陣於江瀆廟前,自晨至晡,戰數合,懷忠兵勢不敵,退還所部。懷忠部下多李順舊黨,頗貪剽劫,故敗績焉。



 懷忠移文嘉、眉七州,調軍士丁男來會。二月,再攻益州。時均方遣逆黨趙延順攻邛、蜀,懷忠逆與之戰,賊稍退。懷忠與轉運使陳緯麾兵由子城南門直入軍資庫,與緯署其庫鑰。均眾皆銀槍繡衣,為數隊,分列子城中。賊兵出通遠門,與懷忠戰數合,會暮,懷忠復退軍筰橋,背水列陣,砦櫧木橋南,以捍邛、蜀之路。賊故不復能南略,
 自清水壩、溫江、金馬三道來攻櫧木砦,出官軍後,焚江原神祠,斷邛、蜀援路。懷忠三路分兵以抗之,斬首五百餘級,驅其餘眾入皂江,獲甲弩甚眾。乘勝逐賊至益州南十五里,砦於雞鳴原,以俟王師。均亦閉成都東門以自固。



 是月,有終等至,令石普先與綿、漢都巡檢張思鈞收復漢州,進壁升仙橋。賊出攻砦,有終擊走之。一日,均開城偽為遁狀,有終與上官正、石普率兵徑入,官軍分剽民財,部伍不肅。賊閉關發伏,布床榻於路口,官軍不
 得出,因為所殺。有終等緣堞而墜,李惠死之,退保漢州。益州城中民皆奔迸四出,復為賊黨分騎追殺,或囚縶之,支解族誅以恐眾。又脅士民僧道之少壯者為兵,先刺手背,次髡首,次黥面,給軍裝,令乘城,與舊賊黨相間。有終署榜招之,至則署其衣袂釋之,日數百人。



 三月,進攻彌牟砦,斬首千餘級,復為賊所拒。四月,賊由升仙橋分路來寇,並軍於東偏,有終率兵逆擊,大敗之,殺千餘人,奪其傘蓋、金槍等物,均單騎還城。有終遣其子奉禮
 郎孝若馳奏,上召孝若問敗賊之由,笑謂左右曰:「均鼠竊爾,雖嬰城自守,計日可擒矣。」孝若因言嘗習武藝,願改秩以效,即補供奉官。俄以刑部員外郎馬亮為轉運使,國子博士張志言副之,供備庫副使張煦為綿、漢都巡檢使。楊懷忠又分所部砦於合水尾、浣花等處,樹機石、設笓籬以拒之。



 賊自升仙之敗,徹橋塞門,官軍進至清還江,為梁而度。有終與石普屯於城北門之西,依壕為土山,分設鹿角,又得舊草場,造梯沖洞車攻具,普專
 主之。高繼勛、張煦、孫正辭攻城東,上官正、李繼昌、王阮攻城西,楊懷忠與巡檢殿直、閣門祗候馬貴攻城南,賊將趙延順盡驅兇黨以拒。既而延順中流矢死,又遣其黨丁重萬立城門上,官軍射之,殪。每攻城,輒會雨,城滑不能上,官軍及丁夫為洞屋以進,賊又鑿地道出掩之,溺壕中死者千餘,軍勢小衄。時方暑濕,軍士多疾,有終市藥他州療之。



 是月,詔洛苑使、入內副都知秦翰為兩川捉賊招安使。有終與翰葉議,於城北魚橋又築土山。
 八月,克城北羊馬城,遂設雁翅敵棚,覆洞屋以進,逼羅城。九月,城北洞屋成,賊對設敵樓以抗官軍,有終遣卒焚之,賊自是銷沮,築月城以自固。有終募敢死士間道以入,賊為藥矢,中者立死。有終令卒蒙氈秉燧以入,悉焚其望櫓機石,先遣東西南砦鼓噪攻城,有終與石普分主洞屋以進。普穴城為暗門,門成,賊攢戟於前,無敢進者。有二卒請行,許以厚賞,乃麾戈直沖之,賊鋒稍卻,遂入城。有終登城樓下瞰,賊之餘眾,猶砦天長觀前,於
 文翁坊密設炮架。高繼勛白於馬亮,請給秸稈油□凡,眾執長戟巨斧,秉炬以進,悉焚之。楊懷忠焚其砦天長觀前,追至大安門,覆敗焉。是夕二鼓,均與其黨二萬餘南出萬里橋門,突圍而遁。有終疑有伏,遣人縱火城中。詰朝,與秦翰登門樓,牙吏有受偽署官職者,捕得,立樓下,傍積薪,厝火其上,索男子魁壯者令辨之,曰某嘗受某職,即命左右捽投火中。自晨至晡,焚死者數百人,時謂冤酷。均既走,度合水尾,由廣都略陵、榮,趣富順監,所過
 斷橋塞路,焚倉庫而去。



 初,有終遣懷忠領虎翼軍追之,後二日,石普繼往,以全軍為後援。十月,均至富順,其將校以筏度江,趨戎、瀘蠻境。朝廷每歲孟冬朔,詔富順監具酒肴,犒內屬蠻酋。是日裁設具,而均黨適至,皆食焉。聞懷忠追騎將至,均心易之,謂其黨曰:「速降懷忠。」令其眾負擔以行。懷忠距富順六十里,於楊家市少憩,賊眾在後者邀戰,懷忠遣騎士登高原覘賊,且語其左右曰:「縱賊度江,後悔無及,聞石侯將至,當以奇兵取之。」乃臨
 江列陣擊之,餘黨散走,有拏舟將度江而遁者,懷忠合強弩射之,溺死甚眾。懷忠張旗鳴鼙入城,均方在監署中,其眾多醉,均窮蹙縊死。虎翼軍校魯斌斬其首詣懷忠,獲僭偽法物、旌旗、甲馬甚眾,禽其黨六千餘人,逆徒殲焉。懷忠旋軍出北門,石普之眾方至,奪均首馳歸成都,梟於北市。



 均本隸開封散從直,後補軍校。初,神衛軍之戍成都者,以均及董福分二指揮以領之。福御眾有法,部下皆優足。均縱其下飲博,軍裝亦以給費。是歲,車
 駕幸河朔,符昭壽與牛冕大閱於東郊,蜀人趨觀之,二軍衣服鮮弊不等,均眾因是慚憤。益州知州與鈐轄二廨並禁旅為牙隊,歲除,冕以酒肴犒部士,而昭壽既驕恣,復肆侵虐,冕亦寬馳無政,故詰朝合起為亂。



 神衛卒既殺昭壽,是日,成都官吏方相與賀正,聞變,皆奔竄,牛冕與轉運張適縋城而出,惟都巡檢使劉紹榮冒刃格鬥。既而眾寡不敵,叛卒尚未有主,或欲奉紹榮為帥者,紹榮攝弓罵曰:「我燕人也,比棄鄉土來歸本朝,豈能與
 汝同逆,汝亟殺我,我肯負朝廷哉!」眾未敢動。監軍王澤與均適至,乃謂均曰:「汝所部為亂,盍自往招安?」均既往,叛卒即擁之為主,紹榮自經死。均僭號大蜀,改元化順,署置官稱,設貢舉,以張鍇為謀主。



 鍇本名美,太原舊卒,後為神衛小校。狡獪,嘗歷戰陣,粗習陰陽,以熒惑同惡,故勸均為亂。均實戇心耎無謀,嘗言:「官軍若至,我當先路出迎,自陳被脅之狀。」鍇聞之,擇軍中子弟署寄班,以防守均,令不與人接見。官軍圍城,每射箭招誘,及令均子
 弟至城下,均皆不之知。得箭書,鍇悉焚之。自起至敗,所守止一城而已。均初署親軍為天降虎翼,後果為虎翼軍所殺。



 賊既平,遣承受供奉官楊崇勛乘傳告捷,賜崇勛錦袍、銀帶、器幣,有終加保信軍節度觀察留後,以秦翰為內園使、恩州刺史,石普為冀州團練使,高繼勛、王阮並為崇儀使,孫正辭為內殿崇班,李繼昌為獎州刺史,張煦為供備庫副使,楊懷忠為供備庫副使,馬貴為供奉官。是役也,懷忠之功居最,為石普所忌,朝廷微聞
 之,遣寄班安守忠按視戰所,盡得其功狀,以故懷忠復遷崇儀使,領恩州刺史。



 四年,有終代還,命為涇、原、儀、渭、鎮戎路都部署,辭不拜,改知永興軍府,徙秦州。景德初,徙為並、代副都部署,賜黃金四百兩。丁內艱,起復,契丹入寇,上幸澶淵,詔有終率所部由土門抵鎮州,與大兵會。既而王超、桑贊逗撓無功,唯有終赴援,威聲甚振,河北列城,賴其雄張。俄而契丹修好,命還屯所,就判並州,召拜宣徽北院使、檢校太保。二年七月,暴疾卒,年五十
 九,贈侍中。錄其子孝若為內殿崇班、閣門祗候,孝傑為內殿崇班,孝緒為供奉官,孝恭為侍禁,親族、門客、給事輩遷補者八人。



 有終倜儻自任,不拘小節,有幹局,沉敏善斷,不畏強御,輕財好施。歷典藩閫,能撫士卒,豐於宴犒,官用不足,則傾私帑及榷錢以給之。家無餘財,奉身甚薄,常所御者,銅鞍勒馬而已。第在崇仁里者,德驤所創。有終在蜀嘗貸備用庫錢數百萬,奏納其第償之,優詔蠲免。為宣徽使,特給廉鎮公用錢歲二千貫。身沒之
 日,宿負猶不啻千萬,官為償之。王繼英在樞密,頗忌有終進用,屢言其在蜀及守邊厚費以收士卒心,真宗不之信,卒保護焉。



 孝先字子思,有鄰子也。舉進士,試秘書省校書郎,知天長縣。以衛濯訟其內亂,除籍配均州。後復知宛丘縣,李繼隆判陳州,薦其能,加試大理評事。契丹內寇,真宗幸大名,孝先以部芻糧河北,首至行在,擢太常寺奉禮郎。



 王均反益州,隨季父有終進討,孝先率先鋒與賊戰升
 仙橋,斬首數百,得均金槍黃傘以獻,改將作監丞。



 李繼遷陷靈州,朝廷調兵,軍費多出於民,關內大擾。孝先請益募商人入粟塞下,償以茶鹽。召對稱旨,命馳驛陜西,與轉運使鄭文寶議立規畫,後多施行。累遷尚書屯田員外郎。嘗建置三司拘收司,以檢天下財利出入之數,詔如其請。



 知興元府,坐保任失實,降通判華州,徙知鄆州。宰相寇準舉,換內園使、知貝州。會慈州民張熙載詐稱黃河都總管,籍並河州郡芻糧數,至貝州。孝先覺其
 奸,捕系獄。孝先欲因此為奇功,以動朝廷,迫司理參軍紀瑛教熙載偽為契丹諜者,號景州刺史兼侍中、司空、大靈宮使,部送京師。樞密院按得孝先所教狀,謫澤州都監,利、虢三州,改環慶路兵馬鈐轄、知邠州。逾年,領昭州刺史,為益州鈐轄,再遷左藏庫使,擢西上閣門使、涇原路鈐轄兼知渭州,復知邠州,徙耀州,以為右領軍衛大將軍、昭州刺史,分司西京卒。子簡夫。



 簡夫字太簡,隱居不仕。康定中,樞密使杜衍薦之,召見,
 以秘書省校書郎簽書秦州觀察判官。公事既罷,居長安,自以處士起,不復肯隨眾調官,多為岐路求闢薦。時三白渠久廢,京兆府遂薦簡夫治渠事。先時,治渠歲役六縣民四十日,用梢木數百萬,而水不足。簡夫用三十日,梢木比舊三之一,而水有餘。知坊州,徙簡州,用張方平薦,知雅州。



 既而辰州蠻酋彭仕羲內寇,三司副使李參、侍御史朱處約安撫不能定,繼命簡夫往。至則督諸將進兵,築明溪上、下二砦,據其險要,拓取故省地石馬
 崖五百餘里。仕羲內附。擢三司鹽鐵判官,以疾,知虢、同二州,累遷尚書職方員外郎,卒。錄其子壽臣為郊社齋郎。



 簡夫始起隱者,出入乘牛,冠鐵冠,自號「山長」。關中用兵,以口舌捭闔公卿。既仕,自奉稍驕侈,騶御服飾,頓忘其舊,裏閭指笑之曰:「牛及鐵冠安在?」



 王超,趙州人,弱冠長七尺餘。太宗尹京,召置麾下。及即位,以隸御龍直。淳化二年,累遷至河西軍節度使、殿前都虞候。



 真宗嗣位,以翊戴功,加檢校太傅、領天平軍節
 度。咸平二年秋,大閱禁兵二十萬於東郊,超執五方旗以節進退,上御戎幄觀之,面賜褒獎。從幸大名,與都虞候張進並為先鋒。都大點檢傅潛逗撓得罪,以超為侍衛馬步軍都虞候、鎮州行營都部署,又帥鎮、定、高陽關三路。契丹入邊,與戰於遂城西,俘馘二萬計,斬其裨王騎將十五人,手詔褒美。



 李繼遷陷清遠軍,以超將西面行營之師御之,徙帥永興軍。宰相言超材堪將帥,遂以超帥定州路行營,王繼忠副之。尋加鎮、定、高陽關三路
 都部署,密遣中使賜以御弓矢,許便宜從事。加開府儀同三司、檢校太尉。咸平六年,遼師大入,超召鎮州桑贊、高陽關周瑩率兵會定州,瑩以非詔旨不至。遼兵圍望都,超、贊率兵赴之,陣於縣南六里。繼忠在陣東偏,契丹出其背,遮絕糧道,人馬乏困,繼忠馳前與契丹戰,超、贊遂旋師,繼忠孤軍沒焉。上即遣劉承珪、李允則馳往,察退衄之狀,且言鎮州副部署李福、拱聖軍都指揮使王升當戰先旋,福坐削籍流封州,升決杖配隸瓊州。



 景
 德初,上親巡澶淵,召超赴行在,復緩師期,契丹遂深入。會南北通好,故薄其責,止罷超三路帥,為崇信軍節度使,徙知河陽。又移鎮建雄,知青州,卒。贈侍中,再贈尚書令,追封魯國公,謚武康。



 超為將善部分,御下有恩。與高瓊同典禁旅,嘗休假他適,過營壘,軍校不時將迎,瓊即命棰罰,超以為非公行,不當加罪,人稱其恕。然臨軍寡謀,拙於戰鬥。子德用。



 德用字符輔。父超為懷州防禦使,補衙內都指揮使。至道
 二年,分五路出兵擊李繼遷,超帥兵六萬出綏、夏,德用年十七,為先鋒,將萬人戰鐵門關,斬首十三級,俘掠畜產以數萬計。進師烏白池,他將多失道不至,虜銳甚,超按兵不進,德用請乘之,得精兵五千,轉戰三日,敵勢卻。德用曰:「歸師迫險必亂。」乃領兵距夏州五十里,絕其歸路,下令曰:「亂行者斬!」一軍肅然,超亦為之按轡。繼遷躡其後,左右望見隊伍甚嚴整,莫敢近。超撫其背曰:「王氏有子矣。」



 累遷內殿崇班,以御前忠佐為馬軍都軍頭,出
 為邢、洺、磁、相巡檢。盜張洪霸相聚界上,吏不能捕。德用以氈車載勇士,詐為婦人飾,過邯鄲。賊果來邀,勇士奮出,悉禽之。徙督捕陜西東路,盜賊相戒曰:「此禽張洪霸者。」皆相率逃去。為環、慶路指揮使,尋以奏事忤旨,責授鄆州馬步軍都指揮使。歷內殿直都虞候、殿前左班都虞候、柳州刺史,遷捧日左廂都指揮使、英州團練使。



 天聖初,以博州團練使知廣信軍。城壞久不治,德用率禁軍增築之,有詔褒諭。徙冀州,歷龍神衛、捧日天武四廂
 都指揮使、康州防禦使、侍衛親軍步軍馬軍都虞候。召還,又為並、代州馬步軍副都總管,遷殿前都虞候、步軍副都指揮使。歷桂州、福州觀察使。



 章獻太后臨朝,有求內降補軍吏者,德用曰:「補吏,軍政也,不可與。」太后固欲與之,卒不奉詔,乃止。太后崩,有司請衛士坐甲,德用曰:「非故事也。」不奉詔。



 仁宗閱太後閣中,得德用前奏軍吏事,奇之,以為可大用,拜檢校太保、簽書樞密院事。德用謝曰:「臣武人,幸得以馳驅自效,賴陛下威靈,待罪行間足
 矣。且臣不學,不足以當大事。」帝遣使者趣入院,遂為副使。久之,以奉國軍節度觀察留後同知院事,遷知院。歷安德軍,加檢校太尉、定國軍節度使、宣徽南院使。趙元昊反,德用請自將討之,不許。



 德用狀貌雄毅,面黑,頸以下白晰,人皆異之。言者論德用貌類藝祖,御史中丞孔道輔繼言之,且謂德用得士心,不宜久典機密,遂罷為武寧軍節度使、徐州大都督府長史。有言德用市馬於府州者,上其券,乃市於商人者。言者猶不已,降右千牛
 衛上將軍、知隨州,州置判官,家人皆惶恐,德用舉止言色如平時,惟不接賓客而已。徙知曹州,或謂德用曰:「孔中丞害公,今死矣。」德用曰:「中丞言官,豈害我者?朝廷亡一忠臣,可惜也。」起為保靜軍節度觀察留後、知青州,改澶州。陜西用兵久無功,契丹遣劉六符來求復關南地,以兵壓境。德用見帝,流涕言:「臣前被罪,陛下赦而不誅,今不足辱命。」帝慰勞,曰:「河北方警,藉卿鎮撫之。」又賜手詔慰勉,拜保靜軍節度使。歲大熟,六符見德用拜曰:「此
 公仁政所及也。」徙真定府、定州路都總管,還奏事,復以宣徽南院使判成德軍。未行,徙定州路都總管。日訓練士卒,久之,士殊可用。



 契丹使諜者來覘,或請捕殺之,德用曰:「第舍之,彼得實以告,是服人之兵以不戰也。」明日大閱,援桴鼓之士皆踴躍,進退坐作,終日不戮一人。乃下令:「具糗糧,聽吾鼓聲,視吾旗幟所向。」覘者歸告契丹,謂漢兵將大入。既而復議和,遂徙陳州,又徙河陽。不行,入奉朝請,出判相州,拜同中書門下平章事、判澶州。徙
 鄭州,封祁國公,還,為會靈觀使。



 德用素善射,雖老不衰。侍射瑞聖園,辭曰:「臣老矣,不能勝弓矢。」帝再三諭之,持二矢未發。帝顧之,使必中,乃收弓矢謝,一發中的,再發又中。帝笑曰:「德用欲中即中爾,孰謂老且衰乎?」賜襲衣、金帶,加檢校太師,復判鄭州,徙澶州,改集慶軍節度使,封冀國公。皇祐三年,上疏乞骸骨,以太子太師致仕,大朝會綴中書門下班。



 德用將家子,習知軍中情偽,善以恩撫下,故多得士心。雖屢臨邊境,未嘗親矢石、督攻戰,
 而名聞四夷,閭閻婦女小兒,皆呼德用曰「黑王相公」。



 帝嘗遣使問邊事,德用曰:「咸平、景德中,賜諸將陣圖,人皆死守戰法,緩急不相救,以至於屢敗。誠願不以陣圖賜諸將,使得應變出奇,自立異效。」帝以為然。



 德用雖致仕,乾元節上壽,預班廷中。契丹使語譯者曰:「黑王相公乃復起耶?」帝聞之,起為河陽三城節度使、同中書門下平章事、判鄭州。至和元年,遂以為樞密使,命入謁拜。明年,富弼相,契丹使耶律防至,德用與防射玉津園。防曰:「天
 子以公典樞密而用富公為相,將相皆得人矣。」帝聞之喜,賜弓一,矢五十。後封魯國公,求去位至六七,乃以為忠武軍節度使、景靈宮使,又以為同群牧制置使。有詔五日一會朝,聽子孫一人扶掖。卒,年七十九,贈太尉、中書令,謚武恭。加賜其家黃金。



 德用諸子中,咸融最鐘愛,晚年頗縱之,多不法,後更折節自飭,官至左藏庫使、眉州防禦使。



 論曰:全義、德驤,遇知太祖、太宗,超復翊戴真宗,宜致崇
 顯,然堇堇無□俞人者,而各有子勒勛於國籍。若知節生將家,喜讀書,立朝爭事,以剛正稱天下,其邦之司直歟。有終起進士,明乾知兵,平蜀鉅賊,振聲鄰敵,可謂「肇敏戎公」矣。至於精神折沖,名聞四夷,矯矯虎臣,則德用其有焉



\end{pinyinscope}