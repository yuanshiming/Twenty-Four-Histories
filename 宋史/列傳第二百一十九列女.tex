\article{列傳第二百一十九列女}

\begin{pinyinscope}

 朱娥張氏彭列女郝節娥朱氏崔氏趙氏丁氏項氏
 王氏二婦徐氏榮氏何氏董氏譚氏劉氏張氏師氏陳堂前節婦廖氏劉當可母曾氏婦王袤妻徐端友妻詹氏女劉生妻謝泌妻謝枋得妻王貞婦趙淮妾譚氏婦吳中孚妻呂仲洙女
 林老女童氏女韓氏女王氏婦劉仝子妻毛惜惜附



 古者天子親耕,教男子力作,皇后親蠶,教女子治生。王道之本,風俗之原,固有在矣。男有塾師,女有師氏,國有其官,家有其訓,然而詩書所稱男女之賢,尚可數也。世道既降,教典非古,男子之志四方,猶可隆師親友以為善;女子生長環堵之中,能著美行垂於汗青,豈易得哉。故歷代所傳列女,何可棄也?考宋舊史得列女若干人,
 作《列女傳》。



 朱娥者,越州上虞朱回女也。母早亡,養於祖媼。娥十歲,里中朱顏與媼競,持刀欲殺媼,一家驚潰,獨娥號呼突前,擁蔽其媼,手挽顏衣,以身下墜顏刀,曰:「寧殺我,毋殺媼也。」媼以娥故得脫。娥連被數十刀,猶手挽顏衣不釋,顏忿恚,斷其喉以死。事聞,賜其家粟帛。其後,會稽令董皆為娥立像於曹娥廟,歲時配享焉。



 張氏,鄂州江夏民婦。裡惡少謝師乞過其家,持刀逼欲
 與為亂,曰:「從我則全,不從則死。」張大罵曰:「庸奴!可死,不可它也。」至以刃斷其喉,猶能走,擒師乞,以告鄰人。既死,朝廷聞之,詔封旌德縣君,表墳曰「列女之墓」,賜酒帛,令郡縣致奠。



 彭列女,生洪州分寧農家。從父泰入山伐薪,父遇虎,將不脫,女拔刀斫虎,奪其父而還。事聞,詔賜粟帛,敕州縣歲時存問。



 郝節娥,嘉州娼家女。生五歲,母娼苦貧,賣於洪雅良家
 為養女。始笄,母奪而歸,欲令世其娼,娥不樂娼,日逼之,娥曰:「少育良家,習織作組紃之事,又輒精巧,粗可以給母朝夕,欲求此身使終為良,可乎?」母益怒,且箠且罵。



 洪雅春時為蠶叢祠,娼與邑少年期,因蠶叢具酒邀娥。娼與娥徐往,娥見少年,倉皇驚走,母挽捽不使去。不得已留坐中,時時顧酒食輒唾,強飲之,則嘔噦滿地,少年卒不得侵凌。暮歸,過雞鳴渡,娥度他日必不可脫,陽渴求飲,自投於江以死。鄉人謂之「節娥」云。



 朱氏,開封民婦也。家貧,賣巾屨簪珥以給其夫。夫日與俠少飲博,不以家為事,犯法徒武昌。父母欲奪而嫁之,朱曰:「何迫我如是耶?」其夫將行,一夕自經死,且曰:「及吾夫未去,使知我不為不義屈也。」吳充時為開封府判官,作《阿朱詩》以道其事。



 崔氏,合淝包繶妻。繶,樞密副使拯之子,早亡,惟一稚兒。拯夫婦意崔不能守也,使左右嘗其心。崔蓬垢涕泣出堂下,見拯曰:「翁,天下名公也。婦得齒賤獲,執瀚滌之事
 幸矣,況敢污家乎!生為包婦,死為包鬼,誓無它也。」



 其後,稚兒亦卒。母呂自荊州來,誘崔欲嫁其族人,因謂曰:「喪夫守子,子死孰守?」崔曰:「昔之留也,非以子也,舅姑故也。今舅歿,姑老矣,將舍而去乎?」呂怒,詛罵曰:「我寧死此,決不獨歸,須爾同往也。」崔泣曰:「母遠來,義不當使母獨還。然到荊州儻以不義見迫,必絕於尺組之下,願以尸還包氏。」遂偕去。母見其誓必死,卒還包氏。



 趙氏,貝州人。父嘗舉學究。王則反,聞趙氏有殊色,使人
 劫致之,欲納為妻。趙日號哭慢罵求死,賊愛其色不殺,多使人守之。趙知不脫,乃紿曰:「必欲妻我,宜擇日以禮聘。」賊信之,使歸其家。家人懼其自殞,得禍於賊,益使人守視。賊具聘帛,盛輿從來迎。趙與家人訣曰:「吾不復歸此矣。」問其故,答曰:「豈有為賊污辱至此,而尚有生理乎!」家人曰:「汝忍不為家族計?」趙曰:「第亡患。」遂涕泣登輿而去。至州廨,舉簾視之,已自縊輿中死矣。尚書屯田員外郎張寅有《趙女詩》。



 張晉卿妻丁氏,鄭州新鄭人,參知政事度五世孫也。靖康中,與晉卿避金兵於大隗山。金兵入山,為所得,挾之鞍上。丁自投於地,戟手大罵,連呼曰:「我死即死耳,誓不受辱於爾輩。」復挾上馬,再三罵不已。卒乃忿然舉梃縱擊,遂死杖下。



 項氏,吉州吉水人。居永昌里,適同里孫氏。宣和七年,為里胥所逮,至中途欲侵凌之,項引刀自刺而死。郡以聞,詔贈孺人,旌表其廬。



 王氏二婦,汝州人。建炎初,金人至汝州,二婦為所掠,擁置舟中,遂投漢江以死。尸皆浮出不壞,人為收葬之城外江上,為雙塚以表之。



 徐氏,和州人。閎中女也,適同郡張弼。建炎三年春,金人犯惟揚,官軍望風奔潰,多肆虜掠,執徐欲污之。徐瞋目大罵曰:「朝廷蓄汝輩以備緩急,今敵犯行在,既不能赴難,又乘時為盜,我恨一女子不能引劍斷汝頭,以快眾憤,肯為汝辱以茍活耶!第速殺我。」賊慚恚,以刃刺殺之,
 投江中而去。



 榮氏,薿女弟也。自幼如成人,讀《論語》、《孝經》,能通大義,事父母孝。歸將作監主簿馬元穎。建炎二年,賊張遇寇儀真,榮與其姑及二女走惟揚,姑素羸,榮扶掖不忍舍。俄賊至,脅之不從,賊殺其女,脅之益急,榮厲聲詬罵,遂遇害。



 何氏,吳人。吳永年之妻也。建炎四年春,金兵道三吳,官兵遁去,城中人死者五十餘萬。永年與其姊及其妻何
 奉母而逃。母老,待挾持而行,卒為賊所得,將縶其姊及何,何紿謂賊曰:「諸君何不武耶!婦人東西惟命爾。」賊信之。行次水濱,謂其夫曰:「我不負君。」遂投於河,其姊繼之。



 董氏,沂州滕縣人,許適劉氏子。建炎元年,盜李昱攻剽滕縣,悅其色,欲亂之,誘諭再三,曰:「汝不我從,當銼汝萬段。」女終不屈,遂斷其首。劉氏子聞女死狀,大慟曰:「列女也。」葬之,為立祠。



 三年春,盜馬進掠臨淮縣,王宣要其妻曹氏避之,曹曰:「我聞婦人死不出閨房。」賊至,宣避之,曹
 堅臥不起。眾賊劫持之,大罵不屈,為所害。



 四年,盜祝友聚眾於滁州龔家城,掠人為糧。東安縣民丁國兵者及其妻為友所掠,妻泣曰:「丁氏族流亡已盡,乞存夫以續其祀。」賊遂釋夫而害之。



 同時,叛卒楊勍寇南劍州,道出小常村,掠一民婦,欲與亂,婦毅然誓死不受污,遂遇害,棄尸道傍。賊退,人為收瘞。尸所枕藉處,跡宛然不滅。每雨則乾,睛則濕,則削去即復見。覆以他土,其跡愈明。



 譚氏,英州真陽縣人,曲江村士人吳琪妻也。紹興五年,
 英州饑,觀音山盜起,攻剽鄉落。琪竄去,譚不能俱,與其女被執。譚有姿色,盜欲妻之,譚怒罵曰:「爾輩賊也。我良家女,豈若偶耶?」賊度無可奈何,害之。



 同時,有南雄李科妻謝氏,保昌故村人。囚於虔盜中,數日,有欲犯之,謝唾其面目:「寧萬段我,不汝徇也。」盜怒,銼之而去。



 劉氏,海州朐山人,適同里陳公緒。紹興末,金人犯山東,郡縣震響,公緒倡義來歸,偶劉歸寧,倉卒不得與偕,惟挈其子庚以行,宋授以八品官,後累功至正使。劉留北
 方,音問不通。或語之曰:「人言『貴易交,富易妻』。今陳已貴,必他娶矣,盍改適?」曰:「吾知守吾志而已,皇恤乎他?」公緒亦不他娶。子庚浸長,輒思念涕泣,傾家貲,結任俠,奔走淮甸,險阻備嘗。如是者十餘年,遂得迎母以歸。劉在北二十五年,嘗緯蕭以自給。



 張氏,羅江士人女。其母楊氏寡居。一日,親黨有婚會,母女偕往,其典庫雍乙者從行。既就坐,乙先歸。會罷,楊氏歸,則乙死於庫,莫知殺者主名。提點成都府路刑獄張
 文饒疑楊有私,懼為人知,殺乙以滅口,遂命石泉軍劾治。楊言與女同榻,實無他。遂逮其女,考掠無實。吏乃掘地為坑,縛母於其內,旁列熾火,間以水沃之,絕而復蘇者屢,辭終不服。一日,女謂獄吏曰:「我不勝苦毒,將死矣,願一見母而絕。」吏憐而許之。既見,謂母曰:「母以清潔聞,奈何受此污辱。寧死箠楚,不可自誣。女今死,死將訟冤於天。」言終而絕。於是石泉連三日地大震,有聲如雷,天雨雪,屋瓦皆落,邦人震恐。



 勘官李志寧疑其獄,夕具衣
 冠禱於天。俄假寐坐廳事,恍有猿墜前,驚寤,呼吏卒索之,不見。志寧自念夢兆:「非殺人者袁姓乎?」有門卒忽言張氏饋食之夫曰袁大,明日袁至,使吏執之,曰:「殺人者汝也。」袁色動,遽曰:「吾憐之久矣,願就死。」問之,云:「適盜庫金,會雍歸,遂殺之。」楊乃得免。時女死才數日也。獄上,郡榜其所居曰孝感坊。



 師氏,彭州永豐人。父驥,政和二年省試第一。宣和中,為右正言十餘日,凡七八疏,論權幸及廉訪使者之害而
 去。女適範世雍子孝純。建炎初,還蜀,至唐州方城縣,會賊朱顯終掠方城,孝純先被害,賊執師氏欲強之,許以不死。師罵曰:「我中朝言官女,豈可受賊辱!吾夫已死,宜速殺我。」賊知不可屈,遂害之。



 陳堂前,漢州雒縣王氏女。節操行義,為鄉人所敬,但呼曰「堂前」,猶私家尊其母也。堂前年十八,歸同郡陳安節,歲餘夫卒,僅有一子。舅姑無生事,堂前斂泣告曰:「人之有子,在奉親克家爾。今已無可奈何,婦願幹蠱,如子在
 日。」舅姑曰:「若然,吾子不亡矣。」既葬其夫,事親治家有法,舅姑安之。子日新,年稍長,延名儒訓導,既冠,入太學,年三十卒。二孫曰綱曰紱,咸篤學有聞。



 初,堂前歸陳,夫之妹尚幼,堂前教育之,及笄,以厚禮嫁遣。舅姑亡,妹求分財產,堂前盡遺室中所有,無靳色。不五年,妹所得財為夫所罄,乃歸悔。堂前為買田置屋,撫育諸甥無異己子。親屬有貧窶不能自存者,收養婚嫁至三四十人,自後宗族無慮百數。里有故家甘氏,貧而質其季女於酒家,
 堂前出金贖之,俾有所歸。子孫遵其遺訓,五世同居,並以孝友儒業著聞。乾道九年,詔旌表其門閭云。



 廖氏,臨江軍貢士歐陽希文之妻也。紹興三年春,盜起建昌,號「白氈笠」,過臨江,希文與妻共挾其母傅走山中,為賊所追。廖以身蔽姑,使希文負之逃。賊執廖氏,廖正色叱之。賊知不可屈,揮刃斷其耳與臂,廖猶謂賊曰:「爾輩叛逆至此,我即死,爾輩亦不久屠戮。」語絕而僕。鄉人義而葬之,號「廖節婦墓」。



 是年,盜彭友犯吉州龍泉,李生
 妻梁氏義不受辱,赴水而死。



 王氏,利州路提舉常平司干辦公事劉當可之母也。紹定三年,就養興元。大元兵破蜀,提刑龐授檄當可詣行司議事。當可捧檄白母,王氏毅然勉之曰:「汝食君祿,豈可辭難。」當可行,大元軍屠興元,王氏義不辱,大罵投江而死。其婦杜氏及婢僕五人,咸及於難。當可聞變,奔赴江滸,得母喪以歸。詔贈和義郡太夫人。



 曾氏婦晏,汀州寧化人。夫死,守幼子不嫁。紹定間,寇破
 寧化縣,令佐俱逃,將樂縣宰黃垺令土豪王萬全、王倫結約諸砦以拒賊,晏首助兵給糧,多所殺獲。賊忿其敗,結集愈眾,諸砦不能御,晏乃依黃牛山傍,自為一砦。



 一日,賊遣數十人來索婦女金帛,晏召其田丁諭曰:「汝曹衣食我家,賊求婦女,意實在我。汝念主母,各當用命,不勝即殺我。」因解首飾悉與田丁,田丁感激思奮。晏自捶鼓,使諸婢鳴金,以作其勇。賊復退敗。鄰鄉知其可依,挈家依黃牛山避難者甚眾。有不能自給者,晏悉以家糧
 助之。於是聚眾日廣,復與倫、萬全共措置,析黃牛山為五砦,選少壯為義丁,有急則互相應援以為犄角,賊屢攻弗克。所活老幼數萬人。



 知南劍州陳韡遣人遺以金帛,晏悉散給其下;又遺楮幣以勞五砦之義丁,且借補其子,名其砦曰萬安。事聞,詔特封晏為恭人,仍賜冠帔,其子特與補承信郎。



 王袤妻趙氏,饒州樂平人。建炎中,袤監上高酒稅,金兵犯筠,袤棄官逃去,趙從之行。遇金人,縛以去,系袤夫婦
 於劉氏門,而入剽掠劉室。趙宛轉解縛,並解袤,謂袤曰:「君速去。」俄而金人出,問袤安往,趙他指以誤之。金人追之不得,怒趙欺己,殺之。袤方伏叢薄間,望之悲痛,歸刻趙像以葬。袤後仕至孝順監鎮。



 塗端友妻陳氏,撫州臨川人。紹興九年,盜起,被驅入黃山寺,賊逼之不從,以刃加其頸,叱曰:「汝輩鼠竊,命若蜉蝣,我良家子,義豈爾辱!縱殺我,官兵即至,爾其免乎?」賊知不可屈,乃幽之屋壁。居數日,族黨有得釋者,咸齎金
 帛以贖其孥。賊引端友妻令婦。曰:「吾聞貞女不出閨閣,今吾被驅至此,何面目登塗氏堂!」復罵賊不絕,竟死之。



 詹氏女,蕪湖人。紹興初,年十七,淮寇號「一窠蜂」倏破縣,女嘆曰:「父子無俱生理,我計決矣。」頃之賊至,欲殺其父兄,女趨而前拜曰:「妾雖窶陋,願執巾帚以事將軍,贖父兄命。不然,父子並命,無益也。」賊釋父兄縛,女麾手使亟去:「無顧我,我得侍將軍,何所憾哉。」遂隨賊。行數里,過市東橋,躍身入水死。賊相顧駭嘆而去。



 劉生妻歐陽氏,吉州安福人。生居新樂鄉,以事出,惡少來欲侵凌之,歐陽不受辱而死。邑人劉寬作詩以吊之,時紹興十年也。



 同縣有朱雲孫妻劉氏,姑病,雲孫刲股肉作糜以進而愈。姑復病,劉亦刲股以進,又愈。尚書謝諤為賦《孝婦詩》。



 謝泌妻侯氏,南豐人。始笄,家貧,事姑孝謹。盜起,焚里舍殺人,遠近逃避。姑疾篤不能去,侯號泣姑側。盜逼之,侯曰:「寧死不從。」盜刃之,僕溝中。賊退,漸蘇,見一篋在側,發
 之皆金珠,族婦以為己物,侯悉歸之,婦分其一以謝,侯辭曰:「非我有,不願也。」後夫與姑俱亡,子幼,父母欲更嫁之,侯曰:「兒以賤婦人,得歸隱居賢者之門已幸矣,忍去而使謝氏無後乎?寧貧以養其子,雖餓死亦命也。」



 同縣有樂氏女,父以鬻果為業。紹定二年,盜入境,其父買舟挈家走建昌。盜掠其舟,將逼二女,俱不從,一赴水死,一見殺。



 謝枋得妻李氏,饒州安仁人也。色美而慧,通女訓諸書。
 嫁枋得,事舅姑、奉祭、待賓皆有禮。枋得起兵守安仁,兵敗逃入閩中。武萬戶以枋得豪傑,恐其扇變,購捕之,根及其家人。李氏攜二子匿貴溪山荊棘中,採草木而食。至元十四年冬,信兵蹤跡至山中,令曰:「茍不獲李氏,屠而墟!」李聞之,曰:「豈可以我故累人,吾出,事塞矣。」遂就俘。明年,徙囚建康。或指李言曰:「明當沒入矣。」李聞之,撫二子,淒然而泣。左右曰:「雖沒入,將不失為官人妻,何泣也?」李曰:「吾豈可嫁二夫耶!」顧謂二子曰:「若幸生還,善事吾
 姑,吾不得終養矣。」是夕,解裙帶自經獄中死。



 枋得母桂氏尤賢達,自枋得逋播,婦與孫幽遠方,處之泰然,無一怨語。人問之,曰:「義所當然也。」人稱為賢母云。



 王貞婦,夫家臨海人也。德祐二年冬,大元兵入浙東,婦與其舅、姑、夫皆被執。既而舅、姑與夫皆死,主將見婦皙美,欲內之,婦號慟欲自殺,為奪挽不得死。夜令俘囚婦人雜守之。婦乃陽謂主將曰:「若以吾為妻妾者,欲令終身善事主君也。吾舅、姑與夫死,而我不為之衰,是不天
 也。不天之人,若將焉用之!願請為服期,即惟命。茍不聽我,我終死耳,不能為若妻也。」主將恐其誠死,許之,然防守益嚴。



 明年春,師還,挈行至嵊青楓嶺,下臨絕壑。婦待守者少懈,嚙指出血,書字山石上,南望慟哭,自投崖下而死。後其血皆漬入石間,盡化為石。天且陰雨,即墳起如始書時。至治中,朝廷旌之曰「貞婦」,郡守立石祠嶺上,易名曰清風嶺。



 趙淮妾,長沙人也,逸其姓名。德祐中,從淮戍銀樹埧。淮
 兵敗,俱執至瓜州。元帥阿術使淮招李庭芝,淮陽諾,至揚城下,乃大呼曰:「李庭芝,男子死耳,毋降也。」元帥怒,殺之,棄其尸江濱。妾俘一軍校帳中,乃解衣中金遺其左右,且告之曰:「妾夙事趙運使,今其死不葬,妾誠不能忘情。願因公言使掩埋之,當終身事相公無憾矣。」軍校憐其言,使數兵輿如江上。妾聚薪焚淮骨置瓦缶中,自抱持,操小舟至急流,仰天慟哭,躍水而死。



 譚氏婦趙,吉州永新人。至元十四年,江南既內附,永新
 復嬰城自守。天兵破城,趙氏抱嬰兒隨其舅、姑同匿邑校中,為悍卒所獲,殺其舅、姑,執趙欲污之,不可,臨之以刃曰:「從我則生,不從則死。」趙罵曰:「吾舅死於汝,吾姑又死於汝,吾與其不義而生,寧從吾舅、姑以死耳。」遂與嬰兒同遇害。血漬於禮殿兩楹之間,入磚為婦人與嬰兒狀,久而宛然如新。或訝之,磨以沙石不滅,又煅以熾炭,其狀益顯。



 吳中孚妻,隆興之進賢人,少寡。景定元年,兵亂,攜孤女
 自沈於縣之染步,曰:「義不辱吾夫。」



 呂仲洙女,名良子,泉州晉江人。父得疾瀕殆,女焚香祝天,請以身代,刲股為粥以進。時夜中,群鵲繞屋飛噪,仰視空中,大星燁煜如月者三。越翼日,父瘳。女弟細良亦相從拜禱,良子卻之,細良恚曰:「豈姊能之,兒不能耶!」守真德秀嘉之,表其居曰「懿孝」。



 林老女,永春人,及笄未婚。紹定三年夏,寇犯邑,入山避之。猝遇寇,欲污之,不從。度不得脫,紿曰:「有金帛埋於家,
 盍同取之?」甫入門,大呼曰:「吾寧死於家,決不辱吾身。」賊怒殺之,越三日面如生。



 童八娜,鄞之通遠鄉建奧人。虎銜其大母,女手拽虎尾,祈以身代。虎為釋其大母,銜女以去。始,林慄侍親官其地,嘗目睹之。已而為守,以聞於朝,祠祀之。



 韓氏女,字希孟,巴陵人,或曰丞相琦之裔。少明慧,知讀書。開慶元年,大元兵至岳陽,女年十有八,為卒所掠,將挾以獻其主將。女知必不免,竟赴水死。越三日得其尸,
 於練裙帶有詩曰:「我質本瑚璉,宗廟供蘋蘩。一朝嬰禍難,失身戎馬間。寧當血刃死,不作衽席完。漢上有王猛,江南無謝安。長號赴洪流,激烈摧心肝。」



 王氏婦梁,臨川人。歸夫家才數月,會大元兵至,一夕,與夫約曰:「吾遇兵必死,義不受污辱。若後娶,當告我。」頃之,夫婦被掠。有軍千戶強使從己,婦紿曰:「夫在,伉儷之情有所不忍,乞歸之而後可。」千戶以所得金帛與其夫而歸之,並與一矢,以卻後兵。約行十餘里,千戶即之,婦拒
 且罵曰:「斫頭奴!吾與夫誓,天地鬼神寔臨之,此身寧死不可得也。」因奮搏之,乃被殺。有同掠脫歸者道其事。越數年,夫以無嗣謀更娶,議輒不諧,因告其故妻,夜夢妻曰:「我死後生某氏家,今十歲矣。後七年,當復為君婦。」明日遣人聘之,一言而合。詢其生,與婦死年月同雲。



 劉仝子妻林氏,福州福清人。其父公遇,知名士。仝子為福建招撫使起義兵,事見《林同傳》。仝子亡命自經死,有司執其妻具反狀,林叱曰:「林、劉二族,世為宋臣,欲以忠
 義報國,事不成,天也,何為反乎!汝知去歲有以血書壁而死者乎?是吾兄也。吾與兄,忠義之心則一也,死且求治汝於地下,可生為汝等凌辱耶!」遂遇害。



 毛惜惜者,高郵妓女也。端平二年,別將榮全率眾據城以畔,制置使遣人以武翼郎招之。全偽降,欲殺使者,方與同黨王安等宴飲,惜惜恥於供給,安斥責之,惜惜曰:「初謂太尉降,為太尉更生賀。今乃閉門不納使者,縱酒不法,乃畔逆耳。妾雖賤妓,不能事畔臣。」全怒,遂殺之。越
 三日,李虎破關,禽全斬之,並其妻子及王安以下預畔者百有餘人悉傅以法。



\end{pinyinscope}