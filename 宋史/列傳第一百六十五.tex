\article{列傳第一百六十五}

\begin{pinyinscope}

 ○崔與之洪咨夔許奕陳居仁劉漢弼



 崔與之,字正子,廣州人。父世明,試有司連黜,每曰「不為宰相則為良醫」,遂究心岐、黃之書,貧者療之不受直。與
 之少卓犖有奇節,不遠數千里游太學。紹熙四年舉進士,廣之士繇太學取科第自與之始。



 授潯州司法參軍。常平倉久弗葺,慮雨壞米,撤居廨瓦覆之。郡守欲移兌常平之積,堅不可,守敬服,更薦之。調淮西提刑司檢法官。民有窘於豪民逋負,毆死其子誣之者,其長欲流之,與之曰:「小民計出倉猝,忍使一家轉徙乎?況故殺子孫,罪止徒。」卒從之,知建昌之新城,歲適大歉,有強發民廩者,執其首,折手足以徇,盜為止,勸分有法,貧富安之。開
 禧用兵,軍旅所需,天下騷然,與之獨買以系省錢。吏告月解不登,曰:「寧罷去。」和糴令下,與之獨以時賈糴,今民自概。通判邕州,守武人,苛刻,衣賜不時給,諸卒大哄。漕司檄與之攝守,叛者貼然,乃密訪其首事一人斬之,闔郡以寧。擢發遣賓州軍事,郡政清簡。



 尋特授廣西提點刑獄,遍歷所部,至浮海巡朱崖,秋毫無擾州縣,而停車裁決,獎廉劾貪,風採凜然。朱崖地產苦䔲,民或取葉以代茗,州郡征之,歲五百緡。瓊人以吉貝織為衣衾,工作
 皆婦人,役之有至期年者,棄稚違老,民尤苦之。與之皆為榜免。其他利病,罷行甚眾。瓊之人次其事為《海上澄清錄》。嶺海去天萬里,用刑慘酷,貪吏厲民,乃疏為十事,申論而痛懲之。高惟肖嘗刻之,號《嶺海便民榜》。廣右僻縣多右選攝事者,類多貪黷,與之請援廣東循、梅諸邑,減舉員賞格,以勸選人。熙寧免役之法,獨不及海外四州,民破家相望。與之議舉行未果,以語顏戣,戣守瓊,遂行之。



 召為金部員外郎,時郎官多養資望,不省事,與之
 鉅細必親省決,吏為欺者必杖之,莫不震慄。金南遷於汴,朝議疑其進迫,特授直寶謨閣、權發遣揚州事、主管淮東安撫司公事。寧宗宣引入內,親遣之,奏選守將、集民兵為邊防第一事。既至,浚濠廣十有二丈,深二丈。西城濠勢低,因疏塘水以限戎馬。開月河,置釣橋。州城與堡砦城不相屬,舊築夾土城往來,為易以甓。因滁有山林之阻,創五砦,結忠義民兵,金人犯淮西,沿邊之民得附山自固,金人亦疑設伏,自是不敢深入。



 揚州兵久不
 練,分強勇、鎮淮兩軍,月以三、八日習馬射,令所部兵皆仿行之。淮民多畜馬善射,欲依萬弩手法創萬馬社,募民為之,宰相不果行。浙東饑,流民渡江,與之開門撫納,所活萬餘。楚州工役繁夥,士卒苦之,叛入射陽湖,亡命多從之者。與之給旗帖招之,眾聞呼皆至,首謀者獨遲疑不前,禽戮之,分其餘隸諸軍。



 山東李全以眾來歸,與之移書宰相,謂:「自昔召外兵以集事者,必有後憂。」宰相欲圖邊功,諸將皆懷僥幸,都統劉琸承密札取泗州,兵
 渡淮而後牒報。琸全軍覆沒,與之憂憤,馳書宰相,言:「與之乘鄣五年,子養士卒,今以萬人之命,壞於一夫之手,敵將乘勝襲我。」金人入境,宰相連遺與之三書,俾議和。與之答曰:「彼方得勢,而我與之和,必遭屈辱。今山砦相望,邊民米麥已盡輸藏,野無可掠,諸軍與山砦並力剿逐,勢必不能久駐。況東海、漣水已為我有,山東歸順之徒已為我用,一旦議和,則漣、海二邑若為區處?山東諸酋若為措置?望別選通才,以任和議。」與之自劉琸敗,亟
 修守戰備,遣精銳,布要害。金人深入無功,而和議亦寢。



 時議將姑闕兩淮制置,命兩淮帥臣互相為援,與之啟廟堂曰:「兩淮分任其責,而無制閫總其權,則東淮有警,西帥果能疾馳往救乎?東帥亦果能疾馳往救西淮乎?制閫俯瞰兩淮,特一水之隔,文移往來,朝發夕至,無制閫則事事稟命朝廷,必稽緩誤事矣。」議遂寢。



 召為秘書少監,軍民遮道垂涕。與之力辭召命,竟還。將度嶺,趣召不已,行次池口,聞金人至邊,乃造朝奏:「今邊聲可慮者
 非一,惟山東忠義區處要不容緩。」前後累疏數千言,每嘆養虎將自遺患。



 升秘書監兼太子侍講,權工部侍郎。未幾,成都帥董居誼以黷貨為叛卒所逐,總領楊九鼎遇害,蜀大擾。與之以選為煥章閣待制、知成都府、本路安撫使,至即帖然。時安丙握蜀重兵久,每忌蜀帥之自東南來者,至是獨推誠相與。丙卒,詔盡護四蜀之師,開誠布公,兼用吳、蜀之士,拊循將士,人心悅服。先是,軍政不立,戎帥多不協和,劉昌祖在西和,王大才在沔州,大
 才之兵屢衄,昌祖不救,遂棄皂郊。吳政屯鳳州,張威屯西和,金人自白還堡突入黑谷,威不尾襲,而迂路由七方關上青野原,金人遂得入鳳州。與之戒以同心體國之大義,於是戎帥協和,而軍政始立。



 先是,丙嘗納夏人合從之請,會師攻秦、鞏,而夏人不至,遂有皂郊之敗。與之至是飭邊將不得輕納。逾年,夏人復攻金人,遣百騎入鳳州,邀守將求援兵。與之使都統李沖來言曰:「通問當遣介持書,不當遣兵徑入。若邊民不相悉,或有相傷,
 則失兩國之好,宜斂兵退屯。」夏人知不可動,不復有言。初,金人既弊,率眾南歸者所在而有,或疑不敢納。與之優加爵賞以來之。未幾,金萬戶呼延棫等扣洋州以歸,與之察其誠,納之,籍其兵千餘人,皆精悍善戰,金人自是不敢窺興元。既復鏤榜邊關,開諭招納,金人諜得之,自是上下相疑,多所屠戮,人無固志,以至於亡。



 蜀盛時,四戎司馬萬五千有奇,開禧後,安丙裁去三之一,嘉定損耗過半,比與之至,馬僅五千。與之移檄茶馬司,許戎
 司自於關外收市如舊,嚴私商之禁,給細茶,增馬價,使無為金人所邀。總司之給料不足者,亦移檄增給之。乞移大帥於興元,雖不果行,而凡關外林木厚加封殖,以防金人突至。隔第關、盤車嶺皆極邊,號天險,因厚間探者賞,使覘之,動息悉知,邊防益密。總計告匱,首撥成都府等錢百五十萬緡助糴本。又慮關外歲糴不多,運米三十萬石積沔州倉,以備不測。初至,府庫錢僅萬餘,其後至千餘萬,金帛稱是。蜀知名士若家大酉、游似、李性
 傳、李心傳、度正之徒皆薦達之,其有名浮於實,用過其才者,亦歷歷以為言。沔帥趙彥吶方有時名,與之獨察其大言亡實,它日誤事者必此人,移書廟堂,欲因乞祠而從之,不可付以邊藩之寄,後果如其言。與之以疾丐歸,朝廷以鄭損代,既受代,金諜知之,大入,與之再為臨邊,金人乃退。召為禮部尚書,不拜,便道還廣。蜀人思之,肖其像於成都仙游閣,以配張詠、趙抃,名三賢祠。



 理宗即位,授充顯謨閣直學士、知潭州、湖南安撫使,辭,提舉
 西京嵩山崇福宮。遷煥章閣學士、知隆興府、江西安撫使,又辭,授徽猷閣學士、提舉南京鴻慶宮。端平初,帝既親政,召為吏部尚書,數以御筆起之,皆力辭。金亡,朝廷議取三京,聞之頓足浩嘆。繼而授端明殿學士、提舉嵩山崇福宮,亦辭,俄授廣東經略安撫使兼知廣州。



 先是,廣州摧鋒軍遠戍建康,留四年,也撤戍歸,未逾嶺,就留戍江西,又四年,轉戰所向皆捷,而上功幕府,不報,求撤戍,又不報,遂相率倡亂,縱火惠陽郡,長驅至廣州城,聲
 言欲得連帥洎幕屬甘心焉。與之家居,肩輿登城,叛兵望之,俯伏聽命,曉以逆順禍福,其徒皆釋甲,而首謀數人,懼事定獨受禍,遂率之遁去,入古端州以自固。至是,與之聞命亟拜,即家治事,屬提刑彭鉉討捕,潛移密運,人無知者。俄而新調諸軍畢集,賊戰敗請降,桀黠不悛者戮之,其餘分隸諸州。



 帝於是注想彌切,拜參知政事,拜右丞相,皆力辭。乃訪以政事之孰當罷行,人才之孰當用舍?與之力疾奏:「天生人才,自足以供一代之用,惟
 辨其君子小人而已。忠實而有才者,上也;才雖不高,而忠實有守者,次也。用人之道,無越於此。蓋忠實之才,謂之有德而有才者也。若以君子為無才,必欲求有才者用之,意向或差,名實無別,君子、小人消長之勢,基於此矣。陛下勵精更始,擢用老成,然以正人為迂闊而疑其難以集事,以忠言為矯激而疑其近於好名,任之不專,信之不篤。或謂世數將衰,則人才先已凋謝,如真德秀、洪咨夔、魏了翁,方此柄用,相繼而去,天意固不可曉。至
 於敢諫之臣,忠於為國,言未脫口,斥逐隨之,一去而不可復留,人才豈易得,而輕棄如此。陛下悟已往而圖方來,昨以直言去位者亟加峻擢,補外者蚤與召還,使天下明知陛下非疏遠正人,非厭惡忠言,一轉移力耳。陛下收攬大權,悉歸獨斷。謂之獨斷者,必是非利害,胸中卓然有定見,而後獨斷以行之。比聞獨斷以來,朝廷之事體愈輕,宰相進擬多沮格不行,或除命中出,而宰相不與知,立政造命之原,失其要矣。大抵獨斷當以兼聽
 為先,儻不兼聽而斷,其勢必至於偏聽,實為亂階,威令雖行於上,而權柄潛移於下矣。」



 又曰:「邊臣主和,朝廷雖知,而未嘗明有施行。憂邊之士,剴切而言,一鳴輒斥,得非朝廷亦陰主之乎?假使和而可保,亦當議而行之可也。」又曰:「比年以變故層出,盜賊跳梁,雷雹震驚,星辰乖異,皆非細故。京城之災,七年而兩見,豈數萬戶生靈皆獲罪於天者。百姓有過,在予一人,此陛下所當凜凜,惟有求直言可以裨助君德,感格天心。」又曰:「戚畹、舊僚,凡
 有絲發寅緣者,孰不乘間伺隙以求其所大欲,近習之臣,朝夕在側,易於親暱,而難於防閑。司馬光謂『內臣不可令其採訪外事,及問以群臣能否』,蓋乾預之門自此始也。若謂其所言出於無心,豈知愛惡之私,因此而入,其於聖德,寧無玷乎?」帝覽奏嘉嘆,趣召愈力,控辭至十有三疏。



 嘉熙三年,乃得致仕,以觀文殿大學士提舉洞霄宮。自領鄉郡,不受廩祿之入,凡奉餘皆以均親黨。薨時年八十有二,遺戒不得作佛事。累封至南海郡公,謚
 清獻。



 洪咨夔,字舜俞,於潛人。嘉定二年進士,授如皋主簿,尋試為饒州教授。作《大治賦》,樓鑰賞識之。授南外宗學教授,以言去。丁母憂,服除,應博學宏詞科,直院莊夏舉自代。



 崔與之帥淮東,闢置幕府,邊事纖悉為盡力。丘壽雋代與之為帥,金人犯六合,揚州閉門設守,咨夔亟詣壽雋言曰:「金人忌楚,必未至揚,乃先自示弱,不特淮左之人心動,而金人且驕必來矣。第當遠斥堠、精間探,簡士
 馬,張外郡聲援而大開城門,晏然如平時。若金人果來犯,某當身任之。」壽雋愧謝。已而金人果遁。山陽兼帥事青州張林清獻銅錢二十萬緡,咨夔謂宜以所獻就犒其軍,如唐魏博故事,使無輕量中國心。帥乃令輸其半,林亦不復來。



 與之帥成都,請於帝,授咨夔籍田令、通判成都府。與之為制置使,首檄咨夔自近,辭曰:「今當開誠心、布公道,合西南人物以濟國事,乃一未有聞而先及門生、故吏,是示人私也。」卒不受,惟以通判職事往來效
 忠,蜀人高之。尋知龍州。州歲貢麩金,率科礦戶,咨夔曰:「將奉上乃厲民乎?」出官錢市之。江油之民歲戍邊,復苦餫餉,為請於制、漕司免之。毀鄧艾祠,更祠諸葛亮,告其民曰:「毋事仇讎而忘父母。」



 還朝,為秘書郎,遷金部員外郎。會詔求直言,慨然曰:「吾可以盡言寤主矣。」其父見其疏,曰:「吾能吃茄子飯,汝無憂。」史彌遠讀至「濟王之死,非陛下本心」,大恚,擲於地。轉考功員外郎。轉對,復言李全必為國患。於是臺諫李知孝、梁成大交論,鐫二秩。讀書
 故山,七年而彌遠死,帝親政五日,即以禮部員外郎召,入見,乞養英明之氣,及論君子小人之分。帝問今日急務,對以「進君子而退小人,開誠心而布公道」。且言「在陛下一念堅凝」。又問在外人物,對以「崔與之護蜀而歸,閑居十年,終始全德之老臣,若趣其來,可為朝廷重。真德秀、魏了翁陛下所簡知,當聚之本朝。」



 翼日,與王遂並拜監察御史。咨夔感激知遇,謂遂曰:「朝無親擢臺諫久矣,要當極本窮原而先論之。」乃上疏曰:「臣歷考往古治
 亂之原,權歸人主,政出中書,天下未有不治。權不歸人主,則廉級一夷,綱常且不立,奚政之問?政不出中書,則腹心無寄,必轉而他屬,奚權之攬?此八政馭群臣,所以獨歸之王,而詔之者必天官塚宰也。陛下親政以來,威福操柄,收還掌握,揚廷出令,震撼海宇,天下始知有吾君。元首既明,股肱不容於自惰,撤副封,罷先行,坐政事堂以治事,天下始知有朝廷。此其大權、大政,亦略舉矣。然中書之敝端,其大者有四:一曰自用,二曰自專,三曰
 自私,四曰自固。願陛下於從容論道之頃,宣示臣言,俾大臣充初志而加定力,懲往轍而圖方來,以仰稱勵精更始之意。」帝嘉納之。又首乞罷樞密使薛極以厲大臣之節,章三上,卒出之。其他得罪清議者,相繼劾去,朝綱大振。



 明年,改元端平。咨夔預乞於正月朔下詔求直言,使人人得盡言無隱,又乞令內職任之穹者各舉所知,皆從之。時登進諸儒,以廣講讀、說書之選。咨夔言聖學之實,所當講明而推行者有六:一,親睦本支;二,正始閨
 門;三,警肅侍御;四,審正邪用舍;五,儲養文武之才;六,憂根本無生事邀功。又言常平義倉、鹽課及苗稅多取之敝。京湖以《八陵圖》來上,咨夔援紹興留司奉表八陵及東晉大都督親謁五陵故事,乞先詔制臣往省,俟還,別議朝祭。又復以完顏守緒骨來獻,時相侈大其事,咨夔曰:「此朽骨耳,函之以葬大理寺可也。第當以金亡告九廟,歸諸祖宗德澤,況與大敵為鄰,抱虎枕蛟,事變叵測,顧可侈因人之獲,使邊臣論功,朝臣頌德。且陛下知慕
 崇政受俘之元祐,獨不鑒端門受降之崇寧乎?」然不果悉從。



 擢殿中侍御史,會王定入臺察,力詆蔣重珍,咨夔乃按定疾視善良,乞罷之。越三日,左遷定,而擢咨夔中書舍人,尋兼權吏部侍郎,與真德秀同知貢舉,俄兼直學士院。時咨夔口瘍已深,復上疏謂當引咎悔過,且乞祠,帝曰:「卿在朝多有裨益,何輕去?」咨夔奏:「臣數備臺諫、給舍,皆不能遏六月之師,何補於朝?臣病久當去,去猶足裨風俗。」帝勉留之,遷吏部侍郎兼給事中。奏:「比徇私
 成俗,化實未更,所恃以一公鑠萬私者,獨陛下耳,而好樂營繕,親厚近屬,保護舊臣,若未能無所系累。」上在位逾一紀,國本未立,未有敢深言之者,咨夔乞擇宗室子養之,並為濟王立後。



 擢給事中,史嵩之入相,召赴闕下,進刑部尚書,拜翰林學士、知制誥。求去愈力,加端明殿學士,卒。御筆:「洪咨夔鯁亮忠愨,有助親政,與執政恩例,特贈兩官。」其遺文有《兩漢詔令攬抄》、《春秋說》、外內制、奏議、詩文行於世。



 許奕,字成子,簡州人。以父任主長江簿。丁內艱,免喪調涪城尉。慶元五年,寧宗親擢進士第一,授簽書劍南東川節度判官。未期年,持所生父心喪,召為秘書省正字,遷校書郎兼吳興郡王府教授。尋遷秘書郎、著作佐郎、著作郎,權考功郎官,非報謁問疾不出。



 遷起居舍人,韓侂胄議開邊,奕貽書曰:「今日之勢,如元氣僅屬,不足以當寒暑之寇。」又因轉對,論:「今日之急惟備邊,而朝廷晏然,百官充位如平時。京西、淮上之師敗同罰異。總領,王
 人也,而聽宣撫司節制,或為參謀。廟堂之議,外廷莫得聞,護聖之軍,半發於外,而禁衛單薄。」乞鞫勘贓吏,永廢勿用。特與放行以啟僥幸者,宜加遏絕。所言皆侂胄所不樂也。



 蜀盜既平,以起居舍人宣撫四川。奕謂:「使從中遣,必淹時乃至,既又徒雲犒師,而不以旌別淑慝為指,無以尉蜀父老之望。」執政是其言。又請:「遇朝會,起居郎、舍人分左右立如常儀。前後殿坐,侍立官御坐東南面西立,可以獲聞聖訓,傳示無極。臣僚奏事,亦不敢易。」詔
 下其疏討論之。



 遣奕使金,奕與骨肉死訣,詣執政趣受指請行,執政曰:「金人要索,議未決者尚多,今將奈何?」奕曰:「往集議時,奕嘗謂增歲幣、歸俘虜或可耳,外此其可從乎?不可行者,當死守之。」尋遷起居郎兼權給事中,以國事未濟力辭,不許。金人聞奕名久,禮迓甚恭,方清暑,離宮相距二十里,至是特為奕還內。方射,奕破的十有一,乃卒行成。還奏,帝優勞久之,奕復奏:「和不可恃,宜葺紀綱,練將卒,使屈信進退之權,復歸於我。」客有以使事
 賀者,奕憮然曰:「是豈得已者,吾深為天下愧之。」



 權禮部侍郎,條六事以獻。俄兼侍講。會諫官五居安、傅伯成以言事去職,奕上疏力爭之。其後又因災異申言曰:「比年上下以言為諱,諫官無故而去者再矣。以言名官,且不得盡,況疏遠乎。」又論:「用兵以來,資賞泛濫,僥幸捷出,宜加裁制。」夏旱,詔求言,奕言:「當以實意行實政,活民於死,不可責償於禱祠之間而已也。蝗至都城,然後下禮寺講酺祭,孰非王土,顧及境而懼,偶不至輦下,則終不以
 為災乎。」又曰:「權臣之誅也,下至閭巷,歡聲如雷。蓋更化之初,人有厚望,久而無以相遠也,此謗讟之所從生。」又曰:「內降非盛世事也,王璇進狀不實而經營以求幸免,裴伸何人,驟為帶御器械。」時應詔者甚眾,奕言最為剴切。攝兼侍讀,每進讀至古今治亂,必參言時事:「願陛下試思,設遇事若此,當何以處之。」必拱默移時,俟帝凝思,乃徐竟其說。帝曰:「如此則經筵不徒設矣。」



 遷吏部侍郎兼修玉牒官,兼權給事中,論駁十有六事,皆貴族近習
 之撓政體者。而封還劉德秀贈典、高文虎之奉祠,士論尤韙之。加楊次山少保、永陽郡王,奕上疏曰:「自古外戚恩寵太甚,鮮不禍咎,天道惡盈,理所必至。次山果辭,則宜從之,如欲更示優恩,則超轉少傅,在陛下既隆於恩,在次山知止於義,顧不休哉!」又言:「史彌遠力辭恩命,宜從之以成其美。」疏入,不報。奕遂臥家求補外,以顯謨閣待制知瀘州。彌遠問所欲言,奕曰:「比觀時事,調護之功深,扶持之意少,非朝廷之利也。」



 嘉、敘、瀘俱接夷壤,董蠻
 米在大入,俘殺兵民,四路創安邊司窮治其事。奕得夷人質之以致所掠,由是迕安邊司。夷酋王粲浮TN木萬計入賈,奕慮其蕩水陸之險,驅之。



 安撫使安丙新立大功,讒忌日聞,宰相錢象祖出謗書問奕,奕喟而言:「士不愛一死而因於眾多之口,亦可悲也。奕願以百口保之。」象祖艴然曰:「公悉安子文若此乎?」適宇文紹節宣撫荊湖還,亦曰:「僕願亦百口以信許公之言。」於是異論頓息,委寄益專。奕於丙深相知,而職事所關必反復辯數以
 求直。其後士多畔丙,奕獨以書疏候問愈數。



 移知夔州,表辭不行,改知遂寧府。捐緡錢數十萬以代民輸,復鹽策之利以養士,為浮梁作堤數百丈,民德之,畫像祠於學。進龍圖閣待制,加寶謨閣直學士,知潼川府。霖雨壞城,撤而築之,不以煩民,亦捐緡錢十二萬為十縣民代輸,於是其民亦相與祠於東山僧舍。



 會金人敗盟,蜀道震擾,奕請「速選威望大臣宣撫,信賞必罰,以獎忠義、收人心。」又言:「忠義之招,體勢倒持,兵食頓增,未知攸濟,且
 斬將之人未聞褒擢,敗軍之將未見施行,事勢不決,將有後時之悔。」御史劾奕欺罔,降一官。詔提舉玉降宮,未數月,特復元官,提舉崇福宮。



 還家,草遺表曰:「自念本非衰病,初染微痾。當湯熨可去之時,臣以疾而為諱;及針石已窮之後,醫束手而莫圖。靖言膏肓所致之由,大抵脈絡不通之故。」皆寓諷諫之意。進顯謨閣直學士致仕,贈通議大夫。初,奕之守瀘,帝顧禮部尚書章穎曰:「許奕已去乎?」起居舍人真德秀侍帝前,論人才,上以骨鯁稱
 之。



 奕天性孝友,送死恤孤,恩意備至。通籀隸書,所著有《毛詩說》、《論語尚書周禮講義》、奏議、雜文行世。



 陳居仁,字安行,興化軍人。父太府少卿膏,娶明州汪氏女,因家焉。膏初為汾州教授,佐守臣張克戩捍金人。後知惠州,單馬造曾袞壘,譬曉降之。鄞僧王法恩謀逆事覺,或請屠城,膏方為御史,力論多殺非聖世事,脅從者悉寬宥之。



 居仁年十四而孤,以蔭授鉛山尉。紹興二十一年舉進士。秦檜與膏有故,有勸以一見可得美官,居
 仁曰:「是有命焉。」終不自通。移永豐令,入鹽行在點檢贍軍激賞酒庫所糴場,詔修《高宗聖政》,妙選寮屬,與範成大並充檢討官。



 淮甸交兵,魏杞以宗正少卿使金,闢居仁幕下。時和戰未決,金兵駐淮北,人情恟懼,突騎大至,彎弓夾道,居仁上馬,猶從容舉酒屬杞:「天寒且酹此觴。」觀者壯之。乃諭金人開道入,卒成禮,減歲幣而還。因出疆賞,轉承議郎,授諸王宮大小學教授。杞秉國柄,居仁忍貧需遠次,未嘗求進。虞允文欲引以為用,不就。允文
 欲與論兵,謝不能,退而貽書謂:「有定力乃可立事,若徒為大言,終必無成,幸成亦旋敗。」允文為之色動。



 徙主軍器監簿、宗正修玉牒。轉對,言:「立國須定規模,陛下非無可致之資,而規模未立。」孝宗初頗不懌,曰:「朕未嘗不立規模。」居仁奏:「陛下銳意恢復,繼乃通和,和、戰、守三者迄今未定,孰為規模耶?」允文曰:「此正前日定力之論,某今益知此言之當也。」



 遷將作監丞,轉國子丞。九年,進秘書丞。入對,論文武並用長久之術:「陛下獎進武臣,深得持
 平救偏之道,然未必得智謀勇略之士,或多便佞輕躁之徒,將復有偏勝之患。」帝喜納。權禮部郎官。嘗言臺閣宜多用明習典故之士,帝問其人,居仁以李燾、莫濟對。甫數日,召燾。



 居仁力請外,乃知徽州。帝令陛辭,慰諭遣之。至郡,告以天子節經費以惠儉瘠,不能推廣聖德,吏則有罪。乃招三衙軍,植二表於庭,有輸納中度而遭抑退者,抱所輸立表下,親視之,人無留滯,吏不能措手,輸稅者恆裹贏以歸。鄰州有訟,多詣臺省乞決於居仁。秩
 滿,邦人挽留,由間道始得去。



 入對,帝舉新安之政獎之。請編類隆興以來寬恤詔令,有曰:「法久則易玩,事久則易怠。惟申加戒飭,有以儆其觀聽,則千萬年猶一日。」帝曰:「名言也。」又言:「歸正忠順,過於優渥,而遇戰士反輕。此曹出萬死策勛,今老矣,添差已罷,廩稍半給,至丐於市,軍士解體。乞加優恤,以終始念功之意,堅後生圖報之心。」帝覽之嘉嘆。會駕大閱白石,即命再添差兩任,衣糧全給,三軍為之呼舞。



 留為戶部右曹郎官,命未下,朝方
 推《會要》賞,帝曰:「陳居仁治行為天下第一,可因是並賞之。」特轉朝議大夫兼權度支,又兼權禮部。會樞屬闕員,方進擬,帝曰:「豈有人才如陳居仁而可久為郎乎?」即授樞密院檢詳文字,尋為右司,遷左司,又遷檢正中書門下省諸房公事,歷兼左藏諸庫。居仁親視按牘,嘗謂:「有罪幸免則冤者何告,誣枉者七人皆當敘復。」執政難之,居仁退,疏其冤狀上之。帝曰:「居仁精審,尚復何疑。」詔以旱求言,居仁乞命公卿務行寬大,御史京鏜極論從窄
 之敝,此風未革。



 假吏部尚書使金,還,遷起居郎,尋兼詳定一司敕令兼權中書舍人,泛恩濫賞,封繳無所避。因言:「恩惠不及小民,名為寬逋負,實以惠頑民耳;名為赦有罪,實以惠奸民耳。願盡放天下五等戶身丁,四等戶一半。」從之。安定王子肜乞封妾為夫人,居仁繳奏,帝喜迎,謂有補風教。又論:「君人之道,貴在執要,今陛下親細故而忽遠猷,事末節而忘大體,願舉綱要以御臣下,省思慮以頤精神。」詰旦,令清中書之務。權直學士院。帝曰:「
 內外制向委數人,今陳居仁一人當之,不見其難。」乞詔大臣博議「絕浮費,汰冗兵,計當省之數,定蠲除之目,此富民之要術也。」



 以集英殿修撰知鄂州,築長堤捍江,新安樂寮以養貧病之民,撥閑田歸之。進煥章閣待制,移建寧府。歲饑,出儲粟平其價,弛逋負以巨萬計,代輸畸零繭稅。有因告糴殺人者,會赦免,居仁曰:「此亂民也,釋之將覆出為惡。」遂誅之。觀察推官柳某死,貧不克歸,二子行丐於道,聞而憐之,予之衣食,買田以養之,擇師以
 教之。鎮江大旱,又移居仁守鎮江。請以緡錢十四萬給兵食,不報;為書以義撼丞相,然後許。發時密往覘之。間遣糴運於荊楚商人,商人曰:「是陳待制耶?」爭以粟就糴。居仁區畫有方,所存活數萬計。因饑民治古海鮮界港,為石䃮丹徒境上,蓄洩以時,以通漕運。治江陰奸僧。



 加寶文閣待制、知福州。入境,有饑民嘯聚,部分迓兵遮擊之,首惡計窮,自經死。治宗室之暴橫,申蠱毒之舊禁。有召命求間者,再進華文閣直學士,提舉太平興國宮,卒,
 贈金紫光祿大夫。



 居仁風度凝遠,處己應物,壹以誠信。臨事毅然有守,所至號稱循吏,皆立祠祀之。有奏議、制稿、詩文行世。子卓。



 卓字立道,紹熙元年進士,其後知江州,移寧國府。丞相以故欲見之,卓謝不往,丞相益器之。李全叛,褫其爵,詔書至淮,人益自勵;太廟災,降罪已詔,京師感動,皆卓所草也。為簽書樞密院事。未幾,丐祠還里。平生不營產業,以贊書所酬金築世綸堂。閑居十有六年,卒年八十有
 六。將葬,事不能具,丞相吳潛聞之,貽書制置使以助。其孫定孫力請謚於朝,乃謚清敏。



 劉漢弼,字正甫,上虞人。生二歲而孤,母謝氏撫而教之。嘉定九年舉進士,授吉州教授。歷江西安撫司干官,監南嶽廟、浙西提舉茶鹽司干官。召試館職,改秘書省正字,序遷秘書郎兼沂王府教授,改著作佐郎兼史館校勘,權考功員外郎。升著作郎、知嘉興府兼兵部員外郎,改兼考功。尋為考功員外郎兼崇政殿說書、編修國史、
 檢討實錄,擢監察御史。出知溫州。尋擢太常少卿,以左司諫召,擢侍御史兼侍講,以戶部侍郎致仕。



 漢弼學明義利之辨,為正字時,應詔言事,極論致災弭災之道。為校書郎,轉對,舉蘇軾所言結人心,厚風俗,存紀綱。又論制閫當復其舊,戎司當各還其所,邊郡守當用武臣。又論決和戰以定國論,合江、淮以壹帥權,公賞罰以勵人心,廣規撫以用人才。為著作佐郎,言兵財楮幣權不可分。又言取士之法,詞學不當去「宏博」字,混補不如復待
 補之便。為著作,為考功員外,所陳皆切於時務。及為言官,帝獎諭曰:「以卿純實不欺,故此親擢,宜悉心以告。」



 漢弼以臺綱久馳,疏三事,曰:定規撫,正體統,遠謀慮。首論給事中錢相巧於迎合,睥睨政地,直學士院吳愈不稱其職,罷去之。又劾中書舍人濮斗南、左正言葉賁,疏留中不出。賁,松陽人,為時相史嵩之腹心。有使賁互按者,明日賁有他命,而漢弼由是去國。嵩之久擅國柄,帝益患苦之,既復以左司諫召,首贊帝分別邪正以息眾疑。
 奏疏論立聖心、正君道、謹事機、伸士氣、收人才五事,帝嘉其言,並付外行之。



 及為侍御史,密奏曰:「自古未有一日無宰相之朝,今虛相位已三月,尚可狐疑而不斷乎?願奮發英斷,拔去陰邪,庶可轉危而安;否則是非不可兩立,邪正不並進,陛下雖欲收召善類,不可得矣。臣聞富弼之起復,止於五請,蔣芾之起復,止於三請,今嵩之既六請矣,願聽其終喪,亟選賢臣,早定相位。」帝覽納,遂決。乃命範鐘、杜範並相,百官舉笏相慶,漢弼之力為多。又
 累章言金淵、鄭起潛、陳一薦、謝達、韓祥、濮斗南、王德明,皆疇昔托身私門,為之腹心,盤據要路,公論之所切齒者。至論馬光祖奪情,總賦淮東,乃嵩之預為引例之地,乞勒令追服終喪,以補名教。



 帝嘗屬漢弼以進人才,退而條具以奏,皆時望所歸重。漢弼以受知特異,而奸邪未盡屏汰,論議未能堅定為慮,遂感末疾,居亡何,遂卒。特贈四官,未幾,賜官田五百畝、楮五千緡給其家,謚曰忠。漢弼之沒也,太學生蔡德潤等百七十有三人伏闕
 上書以為暴卒,而程公許著《漢弼墓銘》,亦與徐元傑並言,其旨微矣。



 論曰:唐張九齡、姜公輔,宋餘靖皆出於嶺嶠之南,而為名世公卿,造物者曷嘗擇地而生賢哉?先王立賢無方,蓋為是也。番禺崔與之晚出,屹然大臣之風,卒與三子者方駕齊驅。洪咨夔、許奕直道正言於理宗在位之日。陳居仁見稱循吏,親結主知。劉漢弼抱忠以死,哀哉!



\end{pinyinscope}