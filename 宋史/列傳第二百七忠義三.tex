\article{列傳第二百七忠義三}

\begin{pinyinscope}

 曾怘弟悟劉汲鄭驤呂由誠郭永韓浩朱庭傑王允功王薦周中周辛附歐陽珣張忠輔李彥仙邵云呂圓登宋炎附
 趙立王復鄭褒附王忠植唐琦李震陳求道



 曾怘,字仲常,中書舍人鞏之孫。補太學內舍生,以父任郊社齋郎,累官司農丞、通判溫州,須次於越。


建炎三年,金人陷越,以琶八為帥,約詰旦城中文武官並詣府,有不至及藏匿、不覺察者,皆死。怘獨不往,為鄰人糾察逮捕,見琶八,辭氣不屈。且言:「國家何負汝,乃叛盟欺天,恣為不道。我宋世臣也,恨無尺寸柄以死國,安能貪生事
 爾狗奴邪?」時金人帳中執兵者皆愕眙相視,琶八曰:「且令出。」左右盡驅其家屬四十口同日殺之越南門外,越人作窖瘞其尸。金人去,怘弟朝散郎
 \gezhu{
  旦心}
 時知杭州餘杭縣事,制大棺斂其骨,葬之天柱山。事聞,予三資恩澤,官其弟怤、子崈、兄子UD,皆將仕郎。



 方遇難時,崇甫四歲,與乳母張皆死。夜值小雨,張得蘇,顧見崇亦蘇,尚吮其乳,郡卒陳海匿崇以歸。後仕至知南安軍。怘從弟悟。



 悟字蒙伯,翰林學士肇之孫也。宣和二年進士,靖康間
 為亳州士曹。金人破亳州,悟被執,抗辭慢罵,眾刃劘之,尸體無存者,妻孥同日被害。年三十三。



 劉汲,字直夫,眉州丹棱人。紹聖四年進士。為合州司理、武信軍推官,改宣德郎、知開封府鄢陵縣。奉行神霄宮不如令,以京畿轉運使趙霆奏,徙通判隆德府。時方士林靈素用事,郡人班自改《易系辭》為妖言,以應靈素。汲攝守,下自獄。靈素薦自有道。命轉運使陳知存按驗,掾史懼,欲變獄。汲責數掾史,知存憚之,卒以實聞。



 通判河
 中府,闢開封府推官。自盛章等尹京,果於誅殺,率取特旨以快意,汲每白府奏罷之。宰相王黼初領應奉司,汲對客輒詆之,黼聞,奏謫監蓬州稅。欽宗召赴闕,汲奏願得驅馳外服,治兵食以衛京師。時置京西轉運司於鄧州,以汲添差副使。建炎元年,範致虛師至陜,汲貽書勸以一軍自蒲中越河陽,焚金人積聚,絕河橋;一軍自陜路直抵鄭、許,與諸道連衡,敵必解散。致虛以書謝汲而行。



 金人再犯京師,諸道不知朝廷動息者三月。馮延緒
 傳詔撫諭,謂車駕出郊定和議,令諸道罷兵。汲謂副總管高公純曰:「詔書未可遽信。」公純問故,汲曰:「詔下以去年十二月,鄧去京七百里,今始至州何也?安有議和以三月,而敵猶未退乎?此必金人脅朝廷以款勤王之師爾,可速進兵。」公純難之,汲請自行,公純不得已俱至南陽,不進,汲獨馳數十騎赴都城,二帝已北行,汲素服慟哭。尋代公純攝帥事,捐金帛饗士,為戰守計。詔鄧州備巡幸,汲廣城池,飾行闕,所以待乘輿之具甚備。就加直
 龍圖閣、知鄧州兼京西路安撫使。



 汲奏:「欲復兩河,當先河東,欲復河東,當用陜兵,請先從事河東,以定西河之根本。」於是金人復渡河,諜知鄧州為行在所,命其將銀朱急攻京西。汲遣副總管侯成林守南陽,金人奄至,殺成林。汲集將吏謂曰:「吾受國恩,恨未得死所,金人來必死,汝有能與吾俱死者乎?」皆流涕曰:「惟命。」民有請涉山作砦以避敵者,汲曰:「是棄城矣。然若屬俱死無益。」乃下令曰:「城中有材武願從軍者聽留,餘從便。」得敢死士四
 百人。又令曰:「凡仕於此,其聽送其家,寅出午反,違者從軍法。」眾皆感服,無一人失期。



 及南陽陷,命將戚鼎將兵三千逆戰,及命靳儀與趙宗印分西、南門犄之。汲自以牙兵四百登陴望,見宗印從間道遁,即自至鼎軍中,麾其眾陣以待,敵至皆死鬥,敵卻。俄而儀敗,金人攻之益急,矢下如雨,軍中請汲去,汲不許,曰:「使敵知安撫使在此為國家致死。」敵大至,汲死之。事聞,贈太中大夫,謚忠介。



 鄭驤,字潛翁,信之玉山人。登元符三年進士第。知溧陽縣,歲饑,民多逃亡。漕司按籍督逋賦不少貸,驤患之,盡去其籍。使者欲繩以法,驤曰:「著令約二稅為定數,今不除,則逋愈多,民愈貧,賦愈不辦。」使者不能屈。時議自建康鑿漕渠導太湖以通大江,將破數州民田,調江、浙二十五州丁夫,所費百萬計。朝廷遣官視可否,驤條析利病,力止之。



 通判岢嵐軍,改慶陽府。姚古奏為熙河蘭廓路經略司屬官。錢蓋自渭易熙,奏闢幕下。地震,秦隴金
 城六城壞,驤為蓋言六城熙河重地,宜趣繕治,因自請董兵護築益機灘新堡六百步,以控西夏。堡成,以功遷官,賜緋衣銀魚。



 唃廝羅氏舊據青唐,置西寧州,董氈入朝,其弟益麻黨征走西夏。大觀中,羌人假其名歸附,童貫奏賜姓名趙懷恭,官團練使。至是黨征自西寧求歸,貫懼事露,議者希貫意欲絕之。驤謂貫欺君,請辨其偽。貫怒,將厚誣以罪,會敗而止。擢京兆府等路提舉常平。驤按格為《常平總目》十卷,頒之所部。時陜右大稔,驤奏
 乞以所部本息乘時廣糴,得米六十萬斛。



 高宗初,以直秘閣知同州兼沿河安撫使。時謀巡近甸金陵、南陽、長安為駐蹕計,驤言:「南陽、金陵偏方,非興王地;長安四塞,天府之國,可以駐蹕。」會帝東幸揚州,復請自楚、泗、汴、洛以迄陜、華,各募精兵,首尾相應,庶敵勢不得沖決。不報。金將婁宿犯同州及韓城,驤遣兵拒險擊之,師失利,金人乘勝徑至城下,通判以下皆遁去。驤曰:「所謂太守者,守死而已。」翼日城陷,驤赴井死,贈通議大夫、樞密直學
 士,謚威愍,詔賜廟愍節。



 驤在熙河,嘗摭熙寧迄政和攻取建置之跡為《拓邊錄》十卷,兵將蕃漢雜事為《別錄》八十卷,圖畫西蕃、西夏、回鶻、盧甘諸國人物圖書為《河隴人物志》十卷,序贊普迄溪巴溫、董氈世族為《蕃譜系》十卷。



 呂由誠,字子明,御史中丞誨之季子。幼明爽有智略,範鎮、司馬光,父友也,皆器重之。以父恩補官,調鄧州酒稅,臨事精敏,老吏不能欺。會營兵竊發,聚眾閉城,守貳
 逃匿,由誠親往招諭,賊斂兵聽命。以功遷秩,尋擢提舉三門、白波輦運,言者謂其資淺,罷之。知合水縣。王中立、種諤徵靈州,由誠部運隨軍,天寒食盡,他邑役夫多潰去,唯由誠所部分無失者。尋改知乘氏縣。丞相呂大防為山陵使,闢為屬。通判成都府,知雅、嘉、溫、綿四州,復知嘉州,皆有治績。



 靖康元年,宰相唐恪薦由誠剛正有家法,宜任臺臣。召至京師,與恪議不合,且憂其蓄縮不足以濟時艱,力辭求退。差知襲慶府,未及出關,金人再入,陷
 京師,立張邦昌,以兵脅士大夫臣之,由誠微服得免。時群盜所在蜂起,由誠崎嶇至郡。城圮糧竭,於是晝夜為備,版築甫就,劇賊李昱擁十萬眾奔至城中,知其有備,陽受元帥府招安而去。康王移軍濟陽,由誠竭力饋餉,軍以不乏。遣官屬王允恭奉表勸進。



 時京東諸郡,兵驕多內訌,獨由誠拊循有方,士樂為用。前後數被攻圍,屹然自立群盜中,救援皆絕。孔彥舟以鄆兵叛,首犯郡境,攻之累旬不能下,始引去。胡選者眾尤殘暴,攻由誠示
 必取,由誠夜焚其攻具,直入帳下,賊駭散,不知所為,忽解圍去。



 一日金兵四集,由誠嚴立賞罰,厲以忠義,守兵爭奮,晝夜警備。金人百道攻城,矢石如雨,人無叛志。郡官有迎降者,執而械之。判官趙令佳同心誓守,城陷俱被執。金人欲生降之,由誠不屈,乃殺其子仍於前,由誠不顧,與令佳同遇害。子偰與其家四十口皆被執,無生還者。南北隔絕,其孫紹清留蜀,後自蜀走江、浙訪由誠生死,遇令佳之子子彞於江陰,知令佳與由誠同死被
 褒典,乃訴於朝,詔贈由誠三官,為通奉大夫,與二子恩澤。



 郭永,大名府元城人。少剛明勇決,身長七尺,須髯若神。以祖任為丹州司法參軍,守武人,為奸利無所忌,永數引法裁之。守大怒,盛威臨永,永不為動,則繆為好言薦之朝。後守欲變具獄,永力爭不能得,袖舉牒還之,拂衣去。



 調清河丞,尋知大谷縣。太原帥率用重臣,每宴饗費千金,取諸縣以給,斂諸大谷者尤亟。永以書抵幕府曰:「
 非什一而取,皆民膏血也,以資觴豆之費可乎?脫不獲命,令有投劾而歸耳。」府不敢迫。縣有潭出雲雨,歲旱,巫乘此嘩民,永杖巫,暴日中,雨立至,縣人刻石紀其異。府遣卒數輩號「警盜」,刺諸縣短長,游蠹不歸,莫敢迕,永械致之府,府為並它縣追還。於是部使者及郡文移有不便於民者,必條利病反復,或遂寢而不行。或謂永:「世方雷同,毋以此賈禍。」永曰:「吾知行吾志而已,遑恤其它。」大谷人安其政,以為自有令無永比者。既去數年,復過之,
 則老稚遮留如永始去。



 調東平府司錄參軍,府事無大小,永咸決之。吏有不能辦者,私相靳曰:「爾非郭司錄耶!」通判鄭州,燕山兵起,以永為其路轉運判官。郭藥師屯邊,怙恩暴甚,與民市不償其直,復驅之,至壞目折支乃已。安撫使王安中莫敢問。永白安中,不治且難制,請見而顯責之;不從,則取其尤者磔之市。乃見藥師曰:「朝廷負將軍乎?」藥師驚曰:「何謂也?」永曰:「前日將軍杖策歸朝廷,上推赤心置將軍腹中,客遇之禮無所不至,而將軍
 未有尺寸功報上也。今乃倚將軍為重,乃縱部曲戕民不禁,平居尚爾,如緩急何!」藥師雖謝無愧容,永謂安中曰:「它日亂邊者必此人也。」已而安中罷,永亦辭去,移河北西路提舉常平。



 會金人趨京師,所過城邑欲立取之。是時天寒,城池皆凍,金率藉冰梯城,不攻而入。永適在大名,聞之,先弛壕漁之禁,人爭出漁,冰不能合。金人至城下,睥睨久之而去。遷河東提點刑獄。



 時高宗在揚州,命宗澤守京師,澤厲兵積粟,將復兩河,以大名當沖要,
 檄永與帥杜充、漕張益謙相掎角。永即朝夕謀戰守具,因結東平權邦彥為援,不數日聲振河朔,已沒州縣皆復應官軍,金人亦畏之不敢動。



 居亡何,澤卒,充守京師,以張益謙代之,而裴億為轉運使。益謙、億齷齪小人。會範瓊脅邦彥南去,劉豫舉濟南來寇,大名孤城無援,永率士晝夜乘城,伺間則出兵狙擊。或勸益謙委城遁,永曰:「北門所以蔽遮梁、宋,彼得志則席卷而南,朝廷危矣。借力不敵,猶當死守,徐銼其鋒,待外援之至,奈何棄之?」
 因募士齎帛書夜縋城出,告急朝廷,乞先為備。攻圍益急,俘東平、濟南人,大呼城下曰:「二郡已降。降者富貴,不降者無噍類。」益謙輩相顧色動,永大言曰:「今日正吾儕報國之時。」又行城撫將士曰:「王師至矣,吾城堅完可守,汝曹努力,敵不足畏也。」眾感泣。質明,大霧四塞,豫以車發斷碑殘礎攻城,樓櫓皆壞,左右蒙盾而立,多碎首者。良久城陷,永坐城樓上,或掖之以歸,諸子環泣請去,永曰:「吾世受國恩,當以死報,然巢傾卵覆,汝輩亦何之?茲命
 也,奚懼。」



 益謙、億率眾迎降,金人曰:「城破始降,何也?」眾以永不從為辭。金人遣騎召永,永正衣冠南向再拜訖,易幅巾而入,黏罕曰:「沮降者誰?」永熟視曰:「不降者我。」金人奇永狀貌,且素聞其賢,乃自相語,欲以富貴啖永,永瞋目唾曰:「無知犬豕,恨不醢爾以報國家,何說降乎?」怒罵不絕。金人諱其言,麾之使去,永復厲聲曰:「胡不速殺我死?當率義鬼滅爾曹。」大名人在系者無不以手加額,為之出涕,金人怒斷所舉手。乃殺之,一家皆遇害。雖素不
 與永合者皆面慟,金人去,相與負其尸瘞之。



 永博通古今,得錢即買書,家藏書萬卷,為文不求人知。見古人立名節者,未嘗不慨然掩卷終日,而尤慕顏真卿為人。充之守大名,名稱甚盛,永嘗畫數策見之,它日問其目。曰:「未暇讀也。」永數之曰:「人有志而無才,好名而遺實,驕蹇自用而得名聲,以此當大任,鮮不顛沛者,公等足與為治乎?」充大慚。靖康元年冬,金人再犯京師,中外阻絕,或以兩宮北狩告永者,永號絕僕地,家人舁歸,不食者數
 日,聞大元帥府檄書至,始勉強一餐。其忠義蓋天性然。



 紹興初,贈中大夫、資政殿學士,謚勇節,官其族數人。



 韓浩,丞相琦孫。以奉直大夫守濰州。建炎二年,金人攻城,浩率眾死守,城陷力戰死。通判朱庭傑身被數箭,亦死。權北海縣丞王允功、司理參軍王薦皆全家陷沒。浩特贈三官,官其家三人。庭傑、允功、薦各官其家一人。



 朝議大夫周中世居濰州,率家人乘城拒守,中弟辛家最富,盡散其財以享戰士。城陷,中闔門百口皆死。紹興六
 年,以周聿請,贈官。



 歐陽珣字全美,吉州廬陵人。崇寧五年進士。調忠州學教授、南安軍司錄,知鹽官縣。以薦上京師,遇國難,及出使,加將作監丞。金人犯京師,朝議割河北絳、磁、深三鎮地講和。珣率其友九人上書,極言祖宗之地尺寸不可以與人。及事急,會群臣議,珣復抗論當與力戰,戰敗而失其地,它日取之直;不戰而割其地,它日取之曲。時宰怒,欲殺珣,乃遣珣奉使割深州,珣至深州城下,慟哭謂
 城上人曰:「朝廷為奸臣所誤至此,吾已辦一死來矣,汝等宜勉為忠義報國。」金人怒,執送燕,焚死之。



 張忠輔,宣和末為將,同崔中、折可與守崞縣。金人來攻,嬰城固守,率士卒以死拒敵。中度不可支,有二心。忠輔宣言於眾曰:「必欲降,請先殺我。」中設伏紿約議事,斬忠輔首擲陴外以示金人。既開城門,可與不屈見殺。可與兄可求建炎中言於朝,官可與之子五人,而忠輔不與,士論惜之。



 李彥仙,字少嚴,初名孝忠,寧州彭原人,徙鞏州。有大志,所交皆豪俠士。閑騎射。家極邊,每出必陰察山川形勢,或瞷敵人縱牧,取其善馬以歸。嘗為種師中部曲,入雲中,獲首級,補校尉。靖康元年,金人犯境,郡縣募兵勤王,遂率士應募,補承節郎。李綱宣撫兩河,上書言綱不知兵,恐誤國。書聞,下有司追捕,乃亡去,易名彥仙。以效用從河東軍,諜金人還,復補校尉。



 河東陷,彥仙拔歸,道出陜,以兵事見守臣李彌大,彌大與語,壯之,留為裨將,戍
 淆、澠間。金人再犯汴,永興帥範致虛合西兵入援,彥仙遮說曰:「淆、澠道隘難以眾進,不若分兵而前,留其半於陜,可為後圖。」致虛怒其沮眾,罷遣之。師至千秋鎮,果敗,官吏皆遁。



 時彥仙為石壕尉,堅守三觜,民爭依之。下令曰:「尉異縣人,非如汝室墓於是。今尉為汝守,若不悉力,金人將尸汝於市。」眾皆奮。金人攻三觜,彥仙戰佯北,金人追之,伏發,掩殺千計,分兵四出,下五十餘壁。



 初,金人得陜,用降者守之,使招集散亡,彥仙陰遣士廁其間,金
 人不覺。乃引兵攻其南郭,夜潛師薄東北隅,所納士內應,噪而入,復陜州。乘勝渡河,列柵中條諸山,旁郡邑皆響附,分遣邵云等下絳、解諸邑。吏行文書,請州印章,彥仙曰:「吾以尉守此,第用吾印。」事聞,上謂輔臣曰:「近知彥仙與金人戰,再三獲捷,朕喜而不寐。」即命知陜州兼安撫使,遷武節郎、閣門宣贊舍人。彥仙搜軍實,增陴浚湟,益為戰守備,盡取家屬以來,曰:「吾以家徇國,與城俱存亡。」聞者感服。邵興在神稷山,以其眾來,願受節制。彥仙
 闢興統領河北忠義軍馬,屯三門,後賴其力復虢州。



 金將烏魯撒拔再攻陜,彥仙極力禦之,金人技窮而去。三年,婁宿悉兵自蒲、解大入,彥仙伏兵中條山擊之,金兵大潰,婁宿僅以身免。授右武大夫、寧州觀察使兼同、虢州制置。彥仙度金人必並力來攻,即遣人詣宣撫使張浚求三千騎,俟金人攻陜,即空城度河北趨晉、絳、並、汾,搗其心腹,金人必自救,乃繇嵐、石西渡河,道鄜、延以歸。浚貽書勸彥仙空城清野,據險保聚,俟隙而動。彥仙不
 從。



 婁宿率叛將折可求眾號十萬來攻,分其軍為十,以正月旦為始,日輪一軍攻城,聚十軍並攻,期以三旬必拔。彥仙意氣如平常,登譙門,大作技樂,潛使人縋而出,焚其攻具,金人愕而卻。食盡,煮豆以啖其下,而取汁自飲。至是亦盡,告急於浚,浚間道以金幣使犒其軍,檄都統制曲端涇原兵來援。端素疾彥仙出己上,無出兵意。浚幕官謝升言於浚曰:「金旦暮下陜,則全據大河,且窺蜀矣。」浚乃出師至長安。道阻不得進,裨將邵隆、呂圓
 登、楊伯孫自外來援,間關傷僕,僅有至者。



 彥仙日與金人戰,將士未嘗解甲。婁宿雅奇彥仙才,嘗啖以河南兵馬元帥,彥仙斬其使。至是使人呼曰:「即降,畀前秩。」彥仙曰:「吾寧為宋鬼,安用汝富貴為!」命強弩一發斃之。設鉤索,日鉤取金人,舂斮城上。殺傷相當,守陴者傷夷日盡,金益兵急攻,城陷,彥仙率眾巷戰,矢集身如蝟,左臂中刃不斷,戰愈力。金人惜其才,以重賞募人生致之,彥仙易敝衣走渡河,曰:「吾不甘以身受敵人之刃。」既而聞金
 人縱兵屠掠,曰:「金人所以甘心此城,以我堅守不下故也,我何面目復生乎?」遂投河死,年三十六。金人害其家,惟弟夔、子毅得免。浚承制贈彥仙彰武軍節度使,建廟商州,號忠烈。官其子,給宅一區,田五頃。紹興九年,宣撫使周聿請即陜州立廟,名義烈。後以商、陜與金人,徙其廟閬州。乾道八年,易謚忠威。



 彥仙頎而長面,嚴厲不可犯,以信義治陜,犯令者雖貴不貸。與其下同甘苦,故士樂為用。有籌略,善應變。嘗略地至青澗,猝遇金人,眾愕
 眙,彥仙依山植疑幟,徐據柳林,解甲自如。金人疑有伏,引去,彥仙追襲於隘,躪死相枕。關以東皆下,陜獨存,金人必欲下陜,然後並力西向。彥仙以孤城扼其沖再逾年,大小二百戰,金人不得西。至城陷,民無貳心,雖婦女亦升屋以瓦擲金人,哭李觀察不絕。金人怒,屠其城,全陜遂沒。裨將邵云、呂圓登、宋炎、賈何、閻平、趙成皆死,並贈官錄其家。



 邵云,龍門人。金人陷蒲城,雲聚少年數百,壁山谷,時出
 撓之。會邵隆起兵,雲往從之,約為兄弟。聞胡夜義者眾強,乃舉所部聽命。李彥仙嘗假夜義官,夜義意不滿,掠南原而去,彥仙誘殺之。雲欲攻陜,彥仙遣客說以義,遂來歸。累有功,官至武翼郎、閣門宣贊舍人。城破被執,婁宿欲命以千戶長,云大罵不屈,婁宿怒,釘云五日而磔之。金人有就視者,猶咀血噴其面,至抉眼摘肝,罵不絕。



 呂圓登,夏縣人。嘗為僧,後以良家子應募,捍金人淆、澠間。彥仙保三觜,圓登歸之,功最多,為愛將。城垂破,以兵
 來援,身重創,持彥仙泣曰:「圍久,不知公安否,今得見公,且死無恨。」創身方臥,聞城陷,遽起戰死。



 宋炎,陜縣人。蹶張命中,補秉義郎。先,金人圍城,炎射死數百人。比再圍,炎以勁弩數百,發毒矢殺千餘人。城陷,金人聲言求善射者貴之,炎不應,力戰死。



 趙立,徐州張益村人。以敢勇隸兵籍。靖康初,金人大入,盜賊群起,立數有戰功,為武衛都虞候。建炎三年,金人攻徐,王復拒守,命立督戰,中六矢,戰益厲。復壯其勇,酌
 卮酒揮涕勞之。城陷,復與其家皆死,獨子佾先去。州教授鄭褒亦罵敵而死。城始破,立巷戰,奪門以出,金人擊之死,夜半得微雨而蘇,乃殺守者,入城求復尸,慟哭手瘞之。陰結鄉民為收復計。金人北還,立率殘兵邀擊,斷其歸路,奪舟船金帛以千計,軍聲復振。乃盡結鄉民為兵,遂復徐州。詔授忠翊郎、權知州事。立奏為復立廟,每遇歲時及出師,必帥眾泣禱曰:「公為朝廷死,必能陰祐其遺民也。」齊人聞之歸心焉。



 時山東諸郡莽為盜區,立
 介居其間,威名流聞。累遷右武大夫、忠州刺史。會金左將軍昌圍楚州急,通守賈敦詩欲以城降,宣撫使杜充命立將所部兵往赴之。且戰且行,連七戰勝而後能達楚。兩頰中流矢,不能言,以手指麾,既入城休士,而後拔鏃。詔以立守楚州。明年正月,金人攻城,立命撤廢屋,城下然火池,壯士持長矛以待。金人登城,鉤取投火中。金人選死士突入,又搏殺之,乃稍引退。五月,兀術北歸,築高臺六合,以輜重假道於楚,立斬其使。兀術怒,乃設南
 北兩屯,絕楚餉道,立引兵出戰,大破之。



 會朝廷分鎮,以立為徐州觀察使、泗州漣水軍鎮撫使兼知楚州。立一日擁六騎出城,呼曰:「我鎮撫也,可來接戰。」有兩騎將襲其背,立奮二矛刺之,俱墮地,奪兩馬而還。眾數十追其後,立瞋目大呼,人馬皆闢易。明日,金人列三隊邀戰,立為三陣應之,金人以鐵騎數百橫分其陣而圍之,立奮身突圍,持挺左右大呼,金人落馬者不知數。承、楚間有樊梁、新開、白馬三湖,賊張敵萬窟穴其間,立絕不與通,
 故楚糧道愈梗。始受圍,菽麥野生,澤有鳧茨可採,後皆盡,至屑榆皮食之。



 承州既陷,楚勢益孤,立遣人詣朝廷告急。簽書樞密院事趙鼎欲遣張俊救之,俊不肯行。鼎曰:「江東新造,全藉兩淮,失楚則大事去矣。若俊憚行,臣願與之偕往。」俊復力辭,乃命劉光世督淮南諸鎮救楚。東海李彥先首以兵至淮河,扼不得進;高郵薛慶至揚州,轉戰被執死;光世將王德至承州,下不用命;揚州郭仲威按兵天長,陰懷顧望;獨海陵岳飛僅能為援,而眾
 寡不敵。高宗覽立奏,嘆曰:「立堅守孤城,雖古名將無以逾之。」以書趣光世會兵者五,光世訖不行。金知外救絕,圍益急。九月,攻東城,立募壯士焚其梯,火輒反向,立嘆曰:「豈天未助順乎。」一旦風轉,焚一梯,立喜,登磴道以觀,飛炮中其首,左右馳救之,立曰:「我終不能為國殄賊矣。」言訖而絕,年三十有七。眾巷哭。以參謀官程括攝鎮撫使以守。金人疑立詐死,不敢動。越旬餘,城始陷。初,朝廷聞楚乏食,與粟萬斛,命兩浙轉運李承造自海道先致
 三千斛,未發而楚失守矣。



 立家先殘於徐,以單騎入楚。為人木強,不知書,忠義出天性。善騎射,不喜聲色財利,與士卒均廩給。每戰擐甲胄先登,有退卻者,大呼馳至,捽而斬之。初入城,合徐、楚兵不滿萬,二州眾不相能,立善撫馭,無敢私隙。仇視金人,言之必嚼齒而怒,所俘獲磔以示眾,未嘗獻馘行在也。劉豫遣立故人齎書約降,立不發書,束以油布焚市中,且曰:「吾了此賊,必滅豫乃止。」由是忠義之聲遠近皆傾下之,金人不敢斥其名。圍
 既久,眾益困,立夜焚香望東南拜,且泣曰:「誓死守,不敢負國家。」命其眾擊鼓,曰:「援兵至,聞吾鼓聲則應矣。」如是累月,終無至者。立嘗戒士卒:不幸城破,必巷戰決死。及陷,眾如其言。



 自金人犯中國,所下城率以虛聲脅降,惟太原堅守逾二年,濮州城破,殺傷大相當,皆為金人所憚。而立威名戰多,咸出其上。訃聞,輟朝,贈奉國節度使、開府儀同三司,官其子孫十人,謚忠烈。明年,金人退,得立尸譙樓下,頰骨箭穴存焉。命官給葬事,後為立祠,名
 曰顯忠。



 王復,以龍圖閣待制知徐州。建炎三年,金人自襲慶府引兵圍徐州,復與男倚同守城,率軍民力戰。外援不至,城陷,復堅坐聽事不去,謂粘罕曰:「死守者我也,監郡而次無預焉,願殺我而舍僚吏百姓。」粘罕欲降之,復慢罵求死,闔門百口皆被殺。巡檢楊彭年亦死焉。事聞,贈復資政殿學士,謚壯節,立廟楚州,號忠烈,官其家五人。



 王忠植,太行義士也。紹興九年,取石州等十一郡,授武
 功大夫、華州觀察、統制河東忠義軍馬,遂知代州。尋落階官,為建寧軍承宣使、龍神衛四廂都指揮使、河東經略安撫使。



 明年,金人圍慶陽急,帥臣宋萬年乘城拒守。會川、陜宣撫副使胡世將檄忠植以所部赴陜西會合,行次延安,叛將趙惟清執忠植使拜詔,忠植曰:「本朝詔則拜,金國詔則不拜。」惟清械詣其右副元帥撒離曷,不能屈。使甲士引詣慶陽城下,諭使降,忠植大呼曰:「我河東步佛山忠義人也,為金人所執,使來招降,願將士勿
 負朝廷,堅守城壁。忠植即死城下。」撒離曷怒詰之,忠植披襟大呼曰:「當速殺我。」遂遇害。世將上其事,贈奉國軍節度使、開府儀同三司,官其家十人。



 唐琦,本衛士。建炎間,高宗航海,琦病留越州。李鄴以城降,金人琶八守之,琦袖石伏道旁,伺其出,擊之,不中被執。琶八詰之,琦曰:「欲碎爾首,死為趙氏鬼耳。」琶八曰:「使人人如此,趙氏豈至是哉。」又問曰:「李鄴為帥尚以城降,汝何人,敢爾?」琦曰:「鄴為臣不忠,吾恨不得手刃之,尚何
 言斯人為!」乃顧鄴曰:「我月給才石五斗米,不肯背其主,爾享國厚恩乃若此,豈復齒人類哉?」詬罵不少屈,琶八趣殺之,至死不絕口。事聞,詔為立廟,賜名旌忠。



 李震,汴人也。靖康初,金人迫京師,震時為小校,率所部三百人出戰,殺人馬七百餘,已而被執。金人曰:「南朝皇帝安在?」震曰:「我官家非爾所當問。」金人怒,絣諸庭柱,臠割之,膚肉垂盡,腹有餘氣,猶罵不絕口。



 陳求道,字得之,咸寧人。登進士第。靖康間判都水監。及
 朝議二帝出郊請和,求道力爭之,不聽。欽宗知康王兵眾,求道請以元帥加之,齎蠟書者八人皆遇害,惟求道所薦劉定致書而還。金人立張邦昌,下令在京官不朝者死,求道稱疾不往,嘔血累日。開封尹親以邦昌命召之,竟不能屈。求道以二帝蒙塵,屢欲自殺,因救得免。



 先是,陳留河決,四十餘日漕輸不通,京城大恐,開封尹宗澤命求道治之,七日河盡復故道。建炎四年,命為襄、鄧、隨、郢鎮撫,以奏兵食不給,待命未行。自咸寧挈家就食
 嘉魚,值亂兵起,乃之蒲圻,寓龍堂僧寺。未久,招撫劉忠叛,一夕數千人麇至,驅求道家還嘉魚。至茗山逆旅,具酒食奉求道為主,將南走湖湘。求道正色厲辭,賊怒,殺求道妻蔡及二子符、佺,必欲從己。求道罵愈厲,賊斫其口拔出舌斷之。獨符子凱竄山谷得免。賊退,始得求道尸,瘞於興陂。



\end{pinyinscope}