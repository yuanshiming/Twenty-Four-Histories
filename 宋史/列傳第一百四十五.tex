\article{列傳第一百四十五}

\begin{pinyinscope}

 劉珙王蘭黃祖舜王大寶金安節王剛中李彥穎範成大



 劉珙,字共父,子羽長子也。生有奇質,從季父子翬學。以蔭補承務郎,登進士乙科,監紹興府都稅務。請祠歸,杜
 門力學,不急仕進。主管西外敦宗院,召除諸王宮大小學教授,遷禮部郎官。



 秦檜欲追謚其父,召禮官會問,珙不至,檜怒,風言者逐之。檜死,召為大宗正丞,遷吏部員外郎。置令式庭中,使選集者得自翻閱,與吏辨,吏無得藏其巧。兼權秘書少監,兼權中書舍人。金犯邊,王師北向,詔檄多出其手,詞氣激烈,聞者泣下。御史杜莘老劾宦者張去為,忤旨左遷,珙不草制,莘老得不去。從幸建康,兼直學士院。車駕將還,軍務未有所付,時張浚留守
 建康,眾望屬之。及詔出,以楊存中為江、淮宣撫使,珙不書錄黃,仍論其不可。上怒,謂宰相曰:「劉珙父為浚所知,此特為浚地耳!」命再下,宰相召珙諭旨,且曰:「再繳則累張公。」珙曰:「某為國家計,豈暇為張公謀。」執奏如初,存中命乃寢。真除中書舍人、直學士院。田師中死,其家請以沒入王繼先第為賜,李珂關通近習,求為督府掾,詔從中下,珙皆論罷之。出知泉州,改衢州。



 湖南旱,郴州宜章縣李金為亂,朝廷憂之,以珙知潭州、湖南安撫使。入境,
 聲言發郡縣兵討擊,而移書制使沉介,請以便宜出師,曰:「擅興之罪,吾自當之。」介即遣田寶、楊欽以兵至,珙知其暑行疲怠,發夫數程外迎之,代其負任,至則犒賜過望,軍士感奮。珙知欽可用,檄諸軍皆受節制,下令募賊徒相捕斬詣吏者,除罪受賞。欽與寶連戰破賊,追至莽山,賊黨曹彥,黃拱執李金以降。支黨竄匿者尚眾,珙諭欽等卻兵,聽其自降,賊相率納兵,給據歸田里。第上諸將功狀有差,上賜璽書曰:「近世書生但務清談,經綸實
 才蓋未之見,朕以是每有東晉之憂。今卿既誅群盜,而功狀詳實,諸將優劣,破賊先後,歷歷可觀,宜益勉副朕意。」



 除翰林學士、知制誥兼侍讀,言於上曰:「世儒多病漢高帝不悅學,輕儒生,臣以為高帝所不悅,特腐儒俗學耳。使當時有以二帝三王之學告之,知其必敬信,功烈不止此。」因陳「聖王之學所以明理正心,為萬事之綱。」上亟稱善。



 拜中大夫、同知樞密院事,辭不獲,因進言曰:「汪應辰、陳良翰、張栻學行才能,皆臣所不逮,而栻窮探聖
 微,曉暢軍務,曩幸破賊,栻謀為多,願亟召用。」上可其奏。兼參知政事。奏除福建鈔鹽歲額二萬萬,罷江西和糴及廣西折米鹽錢,及蠲諸路累年逋負金錢穀帛巨億計。上嘗以久旱齊居禱雨,一夕而應,珙進言曰:「陛下誠心感格,其應如響,天人相與之際,真不容發,隱微纖芥之失,其應豈不亦猶是乎?臣願益謹其獨。」上竦然稱善。



 龍大淵、曾覿既被逐,未幾,大淵死,上憐覿欲還之。珙言:「二人之去,天下方仰威斷。此曹奴隸耳,厚賜之可也,若
 引以自近,使與聞機事,進退人才,非所以光德業、振紀綱。」命遂止。



 殿前指揮使王琪被旨,按視兩淮城壁,還,密薦和州教授劉甄夫。上諭執政召之,珙請曰:「此人名位微,何自知之?」上以琪告。珙退坐堂上,追琪至,詰其故,授牘使對。珙恐,請後不敢,乃叱使責戒勵狀而去。會揚州奏琪檄郡增築新城,珙遂奏罷琪,語在《陳俊卿傳》。珙時爭之尤力,殿中皆驚,以故獨罷為端明殿學士,奉外祠。陳俊卿言:「珙正直有才,肯任怨,臣所不及,願留之。」詔改
 知隆興府、江西安撫使。入辭,猶以六事為獻,上曰:「卿雖去國,不忘忠言,材美非他人所及,行召卿矣。」至鎮,首蠲稅務新額,及罷苗倉大斛。屬邑奉新有復出租稅,窮民不能輸,相率逃去,反失正稅,並奏除之。



 除資政殿學士、知荊南府、湖北安撫使,以繼母憂去。起復同知樞密院事、荊襄安撫使。珙六上奏懇辭,引經據禮,詞甚切,最後言曰:「三年通喪,三代未之有改,漢儒乃有『金革無避』之說,已為先王罪人。今邊陲幸無犬吠之驚,臣乃欲冒金
 革之名,以私利祿之實,不亦又為漢儒之罪人乎?」



 服闋,再除知潭州、湖南安撫使。過闕入見,極論時事,言甚切至,上再三加勞,進資政殿大學士以行。安南貢象,所過發夫除道,毀屋廬,數十州騷然。珙奏曰:「象之用於郊祀,不見於經,驅而遠之,則有若周公之典。且使吾中國之疲民,困於遠夷之野獸,豈仁聖之所為哉!」湖北茶盜數千人入境,疆吏以告,珙曰:「此非必死之寇,緩之則散而求生,急之則聚而致死。」揭榜諭以自新,聲言兵且至,令
 屬州縣具數千人食,盜果散去,其存者無幾。珙乃遣兵,戒曰:「來毋亟戰,去毋窮追,不去者擊之耳。」盜意益緩,於是一戰敗之,盡擒以歸,誅首惡數十,餘隸軍籍。



 淳熙二年,移知建康府、江東安撫使、行宮留守。會水且旱,首奏蠲夏稅錢六十萬緡、秋苗米十六萬六千斛。禁止上流稅米遏糴,得商人米三百萬斛。貸諸司錢合三萬,遣官糴米上江,得十四萬九千斛。籍主客戶高下,給米有差。又運米村落,置場平價振糶,貸者不取償。起是年九月,
 盡明年四月,闔境數十萬人,無一人捐瘠流徙者。



 進觀文殿學士,屬疾,請致仕。孝宗遣中使以醫來,疾革,草遣奏言:「恭、顯、伾、文,近習用事之戒,今以腹心耳目寄之此曹,朝綱以紊,士氣以索,民心以離,咎皆在此。陳俊卿忠良確實,可以任重致遠,張栻學問醇正,可以拾遺補闕,願亟召用之。」既又手書訣栻與朱熹,其言皆以未能為國報雪仇恥為恨。薨,年五十七。贈光祿大夫,謚忠肅。



 珙精明果斷,居家孝,喪繼母卓氏,年已逾五十,盡哀致毀,
 內外功緦之戚,必素服以終月數。喜受盡言,事有小失,下吏言之立改。臨數鎮,民愛之若父母,聞訃,有罷市巷哭相與祠之者。



 王蘭,字謙仲、廬江人。乾道五年,擢進士第。為信州上饒簿、鄂州教授、四川宣撫司干辦公事,除武學諭。孝宗幸學,蘭迎法駕,立道周,上目而異之,命小黃門問知姓名,由是簡記。



 遷樞密院編修官,輪對,奏五事,讀未竟,上喜見顏色。明日,諭輔臣曰:「王蘭敢言,宜加獎擢。」除宗正丞,
 尋出守舒州。陛辭,奏疏數條,皆極言時事之未得其正者,上曰:「卿議論峭直。」尋出手詔:「王蘭鯁直敢言,除監察御史。」一日,上袖出幅紙賜之,曰:「比覽陸贄《奏議》,所陳深切,今日之政恐有如德宗之弊者,可思朕之闕失,條陳來上。」蘭即對曰:「德宗之失,在於自用遂非,疑天下士。」退即上疏,陳德宗之弊,並及時政闕失,上嘉納之。



 遷起居舍人,言:「朝廷除授失當,臺諫不悉舉職,給、舍始廢繳駁,內官、醫官、藥官賜予之多,遷轉之易,可不思警懼而正
 之乎?」上竦然曰:「非卿言,朕皆不聞。磊磊落落,惟卿一人。」除禮部侍郎兼吏部。嘗因手詔「謀選監司,欲得剛正如卿者,可舉數人。」即奏舉潘時、鄭矯、林大中等八人,乞擢用。會以母憂去。服除,召還為禮部尚書,進參知政事。



 光宗即位,遷知樞密院事兼參政,拜樞密使。光宗精厲初政,蘭亦不存形跡,除目或自中出,未愜人心者,輒留之,納諸御坐。或議建皇后家廟,力爭以為不可,因應詔上疏「願陛下先定聖志」,條列八事,疏入,不報。中丞何澹論
 之,以罷去。起帥閫,易鎮蜀,皆不就。後領祠,帥江陵。寧宗即位,改帥湖南。臺臣論罷,歸里奉祠。七年薨。



 蘭盡言無隱,然嫉惡太甚,同列多忌之,竟以不合去。有《奏議》傳於世。



 黃祖舜,福州福清人。登進士第,累任至軍器監丞。入對,言:「縣令付銓曹,專用資格,曷若委郡守,汰其尤無良者。」上然之。



 權守尚書屯田員外郎,徙吏部員外郎,出通判泉州。將行,言:「抱道懷德之士,不應書干祿,老於韋布。乞
 自科舉外,有學行修明、孝友純篤者,縣薦之州,州延之庠序,以表率多士;其卓行尤異者,州以名聞,是亦鄉舉里選之意。」下其奏禮部,遂留為倉部郎中,遷右司郎中、權刑部侍郎兼詳定敕令司兼侍講。進《論語講義》,上命金安節校勘,安節言其書詞義明粹,乃令國子監板行。薦李寶勇足以冠軍,智足以料敵,詔以寶為帶御器械。



 兼權給事中。張浚薨,其家奏留使臣五十餘人理資任,祖舜言:「武臣守闕者數年,今素食無代,坐進崇秩,曷以
 勸功?乞為之限制。」遂詔勛臣家兵校留五之一。戶部奏以官田授汰去使臣,祖舜言:「使臣汰者一千六百餘人,臨安官田僅為畝一千一百,計其請而給田,則不過數十人。」事不行。保義郎梁舜弼、漢弼,邦彥養孫也,並閣門祗候,祖舜言:「閣門不可以恩澤補遷。」知池州劉堯仁升右文殿修撰,知新州韓彥直升秘閣修撰,祖舜言:「修撰本以待文學,不可幸得。」故資政殿學士楊願家乞遺表恩,祖舜言:「願陰濟秦檜,中傷善類。」皆寢其命。秦熹卒,贈
 太傅,祖舜言:「熹預其父檜謀議,今不宜贈帝傅之秩。」追奪之。



 遷同知樞密院事。金主亮犯淮,劉汜敗,王權走,上將誅權以厲其餘,祖舜言:「權罪當誅,汜不容貸。劉錡有大功,聞其病已殆,權、汜誅,錡必愧忿以死,是國家一敗兵而殺三將,得無快於敵乎?」上嘉納。薨於官,謚莊定。



 王大寶,字符龜,其先繇溫陵徙潮州。政和間,貢闢雍。建炎初,廷試第二,授南雄州教授。以祿不逮養,移病而歸。閱數年,差監登聞鼓院、主管臺州崇道觀,復累年。



 趙鼎
 謫潮,大寶日從講《論語》,鼎嘆曰:「吾居此,平時所薦無一至者,君獨肯從吾游,過人遠矣。」知連州。張浚亦謫居,命其子栻與講學。時趙、張客貶斥無虛日,人為累息,大寶獨泰然。浚奉不時得,大寶以經制錢給之,浚曰:「如累君何?」大寶不為變。



 代還,言連、英、循、惠、新、恩六州,居民才數百,非懋遷之地,月輸免行錢宜蠲減。高宗謂大臣曰:「守臣上殿,令陳民事,遂得知田里疾苦,所陳五六,得一可行,其利亦不細矣。」乃命廣西諸司具減數聞。



 知袁州,進《
 詩》、《書》、《易解》,上謂執政曰:「大寶留意經術,其書甚可採,可與內除。」執政擬國子司業,上喜曰:「適合朕意。」時經筵闕官,遂除國子司業兼崇政殿說書。奏:「江南諸州有月樁錢,無定名數,吏緣為奸,刻剝民。又有折帛錢,方南渡兵興,物價翔貴,令下戶折納,務以優之,今市帛匹四千,而令輸六千。盍委監司核月樁為定制,樁減折帛惠小民。」詔戶部詳其奏。



 直敷文閣、知溫州、提點福建刑獄。道臨漳,有峻嶺曰蔡岡,藂薄蔽醫,山石犖確,盜乘間剽劫。大寶
 以囊金三十萬,募民抉藪甃道十餘里,行者便之。提點廣東刑獄。



 孝宗即位,除禮部侍郎。大寶言:「古致治之君,先明國是,而行之以果斷。自軍興以來,曰徵曰和,浮議靡定。太上傳丕基於陛下,四方日徯恢復,國論未定,眾志未孚。願陛下果斷,則無不濟。」擢右諫議大夫,首論朱倬、沉該之罪,皆行其言。汪澈督師荊、襄,大寶劾其不能節制,坐視方城之敗,疏再上,澈落職謫臺州。大寶嘗論及移蹕,上曰:「吾欲亟行。」大寶奏:「今日之勢殆未可,願少
 寬歲月。」



 張浚復起為都督,大寶力贊其議,符離失律,群言洶洶。大寶言:「危疑之際,非果斷持重,何以息橫議。」未幾,湯思退議罷督府,力請講和,大寶奏謂:「今國事莫大於恢復,莫仇於金敵,莫難於攻守,莫審於用人。宰相以財計乏,軍儲虛,符離師潰,名額不除,意在核軍籍,減月給。臣恐不惟邊鄙之憂,而患起蕭墻矣。」章三上,除兵部侍郎。



 胡銓為起居郎,奏曰:「近日王十朋、王大寶相繼引去,非國之福。」上曰:「十朋力自引去,朕留之不能得。大寶
 論湯思退太早,令為兵部侍郎,豈容復聽其去。」未幾,以敷文閣直學士提舉太平興國宮。他日,銓奏事,上復諭之曰:「大寶留之經筵,亦固求去,勢不兩立。」銓奏:「自古臺諫論宰相多矣,若謂勢不兩立,則論宰相者皆當去。」大寶尋請致仕。督府既罷,撤邊防,棄四州,金復犯邊,詔思退都督軍馬,辭不行。上震怒,竄思退,中外以大寶前言不用為恨。



 乾道元年,落致仕,召為禮部尚書。入對,言理財之道,當務本抑末。右正言程叔達奏大寶乞復免行
 錢非是,以舊職提舉太平興國宮。中書舍人閻安中欲留其行,叔達並劾之。詔大寶致仕。尋卒,年七十七。



 金安節,字彥亨,歙州休寧人。資穎悟,日記千言,博洽經史,尤精於《易》。宣和六年,由太學擢進士第,調洪州新建縣主簿。紹興初,範宗尹引為刪定官。入對,言:「司馬光以財用乏,請用宰相領總計使,宜以為法。」



 除司農丞,又遷殿中侍御史。韓世忠子彥直直秘閣,安節言:「崇、觀以來,因父兄秉政而得貼職近制,皆在討論。今彥直復因父
 任而授,是自廢法也。」不報。任申先除待制致仕,安節劾其忿戾,乞追奪。秦檜兄梓知臺州,安節劾其附麗梁師成,梓遂罷,檜銜之。未幾,丁母憂去,遂不出。



 檜死,起知嚴州,除浙西提刑。入為大理卿,首言:「治民之道,先德後刑,今守令慮不及遠,簿書期會,賦稅輸納,窮日力辦之,而無卓然以教化為務者。願申飭守令,俾無專事法律,茍可以贊教化,必力行之。」時獲偽造鹽引者,大臣欲置之死,安節力爭,以為事已十餘年,且自首無死法,因得減
 等。兩浙漕屬王悅道鞫仁和令楊績獄不實,事下大理,安節並逮悅道。悅道,幸醫王繼先子也,屢因人求免,安節不從。



 遷宗正少卿。為金使施宜生賀正,安節館伴。屬顯仁皇后喪,服黑帶,宜生曰:「使人以賀禮來,迓使安得服黑帶?」安節辭難再四,宜生屈服。遷禮部侍郎。明年,再充送伴使。至楚州,副使耶律翼奪巡檢王松馬不得,鞭笞之。安節遣人責翼,詞色俱厲,朝廷恐生事,坐削兩秩。葉義問使金,金主因言:「前日奪馬事,曲在翼,已笞二百,
 回日可詳奏。」乃復元官。



 遷禮部侍郎。將祠明堂,時已聞欽宗升遐,安節言:「宮廟行禮,皆當以大臣攝事。」從之。遷侍講、給事中。殿院杜莘老論張去為補外,安節言:「不可因內侍而去言官。」上遂留莘老。



 金主亮犯淮,從幸建康。亮死,安節陳進取、招納、備守三策,而以備守為進取、招納之本。上將還臨安,命楊存中宣撫江、淮、荊、襄,安節言:「存中頃以權太盛,人言籍籍,方解軍政,復授茲職,非所以全之。」又言:「方今正當大明賞罰,乃首用劉寶、王權刻
 剝庸懦之人,何以激勸將士。」上皆納之。



 楊存中議省江、淮州縣,安節言:「廬之合肥,和之濡須,皆昔人控扼孔道。魏明帝云:『先帝東置合肥,南守襄陽,西固祁山,賊來輒破於三城之下』。孫權築濡須塢,魏軍累攻不克,守將如甘寧等,常以寡制眾。蓋形勢之地,攻守百倍,豈有昔人得之成功,今日有之而反棄之耶?且濡須、巢湖之水,上接店步,下接江口,可通漕舟,乞擇將經理。」存中議遂格。



 孝宗嗣位,給廷臣筆札陳當世事,安節請:「嚴內降之科,
 凡內侍省、御藥院、內東門司冗費,一切罷去。堂除省歸吏部,長官聽闢僚屬,以清中書之務。文武蔭補,各有定制,毋令易文資。臣僚致仕遺表恩澤,不宜奏異姓,使得高貲為市。」上嘗對大臣稱其誠實。一日,因奏事面勞之曰:「近不見繳駁,有所見,但繳駁,朕無不聽。」



 龍大淵、曾覿以潛邸舊恩,大淵除樞密都承旨,覿帶御器械,諫議大夫劉度仍累疏論之。隆興改元,大淵、覿並除知閣門事,宰相知安節必以為言,使人諷之曰:「若書行,即坐政府
 矣。」安節拒不納,封還錄黃。時臺諫相繼論列,奏入不出,上意未回,安節與給事中周必大奏:「陛下即位,臺諫有所彈劾,雖兩府大將,欲罷則罷,欲貶則貶,獨於二臣乃為遷就諱避。臣等若奉明詔,則臣等負中外之謗;大臣若不開陳,則大臣負中外之責;陛下若不俯從,則中外紛紛未止也。」上怒,安節即自劾乞竄,上意解,命遂寢。潛邸舊人李珂擢編修官,安節又奏罷之,上諭之曰「朕知卿孤立無黨。」張浚聞之,語人曰:「金給事真金石人也。」



 拜
 兵部侍郎。金將僕散忠義遺三省、樞密院書,論和議,乃畫定四事,詔群臣議。安節謂:「世稱侄國,國號不加『大』字及用『再拜』二字,皆不可從。海、泗、唐、鄧為淮、襄屏蔽,不可與。必不得已,寧少增歲幣。欽宗梓宮當迎奉。陵寢地必不肯歸我,宜每因遣使恭謁。但講好之後,當益選將厲兵,以為後圖。」已而請祠,得請。中書舍人胡銓繳奏,謂:「安節太上之舊人,而陛下之老成也。漢張蒼、唐張柬之、國朝富弼文彥博皆年八旬尚不聽其去,安節膂力未愆,
 有憂國心,豈宜從其引去。」上遂留之。



 逾年,權吏部尚書兼侍讀。自是力請謝事,詔以敷文閣學士致仕。陛辭,上曰:「卿且暫歸,旦夕召卿矣。」去之日,縉紳相與嘆羨,以為中興以來全名高節,鮮有其比。乾道六年卒,年七十七。遺表聞,贈通奉大夫,累贈開府儀同三司、少保。



 安節至孝,居喪有禮。與兄相友愛,田業悉推與之,又以恩奏其孤子與。初筮仕,未嘗求薦於人,及貴,有舉薦不令人知。其除司農丞,或語之曰:「公是命,張侍郎致遠為中司時
 所薦,盍往謝之?」安節曰:「彼為朝廷薦人,豈私我耶!」竟不往。薦晁公武、龔茂良可臺諫,皆稱職,二人弗知也。與秦檜忤,不出者十八年,及再起,論事終不屈,人以此服之。有文集三十卷、《奏議表疏》、《周易解》。



 王剛中,字時亨,饒州樂平人。剛中博覽強記。紹興十五年,進士第二人。任某州推官,改左宣義郎。故事當召試,秦檜怒其不詣己,授洪州教授。檜死,召見,擢秘書省校書郎,遷著作佐郎。



 孝宗為普安郡王,剛中兼王府教授,
 每侍講,極陳古今治亂之故,君子小人忠佞之辨。遷中書舍人,言:「禦敵今日先務,敵強則犯邊,弱則請盟。今勿計敵人之強弱,必先自治,擇將帥,搜戰士,實邊儲,備器械,國勢富強,將良士勇,請盟則為漢文帝,犯邊則為唐太宗。」上韙其言。會西蜀謀帥,上曰:「無以逾王剛中矣。」以龍圖閣待制知成都府、制置四川。御便殿,臨遣錫金帶、象笏。進敷文閣直學士。



 時吳璘累官閥至大帥,其下姚仲、王彥等亦建節雄一方。守帥以文治則玩於柔,而號
 令不行;以武競則窒於暴,而下情不通。惟剛中檢身以法,示人以禮,不立崖塹,馭吏恩威並行,羽檄紛沓,從容裁決,皆中機會。



 敵騎度大散關,人情洶洶。剛中跨一馬,夜馳二百里,起吳璘於帳中,責之曰:「大將與國義同休戚,臨敵安得高枕而臥?」璘大驚。又以蠟書抵張正彥濟師。西師大集,金兵敗走。方議奏捷,剛中倍道馳還,謂其屬李燾曰:「將帥之功,吾何有焉。」燾唶曰:「身督戰而功成不居,過人遠矣。」已乃差擇將士,眾所推者上之朝,備統
 帥選。又疏蜀名勝士與幕府之賢,備部使者、州刺史之佐。目使頤指,內外響應。諸汰遣使臣困絕不能自存,剛中以為冒刃於少壯之年,不可斥棄於既老之後,悉召詣府,有善射者復其祿秩,以禁軍闕額糧給之,其罷癃不堪事,則給以義倉米。



 成都萬歲池廣袤十里,溉三鄉田,歲久淤澱,剛中集三鄉夫共疏之,累土為防,上植榆柳,表以石柱,州人指曰:「王公之甘棠也。」府學禮殿,東漢興平中建,後又建新學,遭時多故,日就傾圮,屬九縣繕
 完,悉復其舊。葺諸葛武侯祠、張文定公廟,夷黃巢墓,表賢癉惡以示民。有女巫蓄蛇為妖,殺蛇,黥之。



 孝宗受禪,以宮僚進左朝奉大夫,召赴闕,以足疾請祠,提舉太平興國宮。歸次番陽,營圃植竹,號竹塢。



 金犯淮,有旨趣剛中入見,陳戰守之策。除禮部尚書、直學士院兼給事中,為鹵簿使,除端明殿學士、簽書樞密院事,進同知院事。剛中曰:「戰守者實事,和議者虛名,不可恃虛名害實事。」又奏四事:開屯田、省浮費、選將帥、汰冗兵。居政府,屬疾
 卒,年六十三,贈資政殿大學士、光祿大夫,謚恭簡。



 建炎間,詔階、成、岷、鳳四州刺壯丁為兵,眾以為憂。剛中建言五害罷之,免符下,民歡呼,聲震山谷。比去,蜀父老遮道,有追送數百里者。繇布衣至公卿,無他嗜好,公退惟讀書著文為樂。有《易說》、《春秋通義》、《仙源聖紀》、《經史辨》、《漢唐史要覽》、《天人修應錄》、《東溪集》、《應齋筆錄》,凡百餘卷。



 李彥穎字秀叔,湖州德清人。少端重,強記覽。金犯浙西,父挾家人逃避,彥穎方十歲,追不及,敵已迫其後,能趨
 支徑,亂流獲濟。



 紹興十八年,擢進士第,主餘杭簿。守曹泳豪敓酒家業為官監,利其貲具,彥穎爭之。泳怒,戒吏段煉,不得毫發罪。調建德丞,改秩。時宰知其才,將處之學官,或勸使一見,彥穎恥自獻。調富陽丞。御史周操薦為御史臺主簿。



 金敗盟,張浚督師進討。上方向浚,執政堅主和,陳良翰、周操不以為然。右正言尹穡陰符執政,薦引同己者,轉言和於上前。上惑之,罷督府,良翰、操相繼黜,而穡進殿中,遷諫議大夫。一日,穡以和、戰、守叩彥
 穎,彥穎曰:「人所見固不同。公既以和議為是,曷不明陳於上前,以身任之,事成功歸於公,不成奉身而退。若欲享其利而不及其害,國事將誰倚?」穡大怒曰:「自為諫官,前後百餘奏,曷嘗及一『和』字,而臺簿有是言!」自是銜彥穎,陰排之。



 改國子博士,權吏部郎中,以父喪去。免喪,復為吏部兼皇子恭王府直講,權右史兼兵部侍郎。經筵,張栻講《葛覃》,言先王正家之道,因及時事,語激切,上意不懌。彥穎曰:「人臣事君,豈不能阿諛取容?栻所以敢直言,
 正為聖明在上,得盡愛君之誠耳。《書》曰:『有言逆於汝心,必求諸道。』」上意遽解,曰:「使臣下皆若此,人主應無過。」



 立皇太子,兼左諭德。首論建置宮僚,以為詹事於東宮內外無所不當省,事須白詹事而後行。司馬光論皇太子講讀官有奏疏,錄以進。上大喜,行之。皇太子尹臨安,兼判官兼中書舍人。張說再登樞筦,彥穎論:「說無寸長,去年驟躋宥府,物議沸騰。今此命復出,中外駭然。臣恐六軍解體,人心不服。」未幾,權禮部侍郎兼侍講,因言:「士習委
 靡,不然則矯激,宜擇篤實鯁亮者用之。」升詹事,見上,言:「皇太子尹臨安已久,雖欲更嘗民事,然非便,宜一意講學。」他日以言於上者告太子,趣草奏辭尹事,三辭乃免。



 兼吏部侍郎,權尚書兼侍讀。月食淫雨,言:「甲申歲以淫雨求言,今十年矣,中間非無水旱,而不聞求言之詔,豈以言多沽激厭之耶?比欺蔽成風,侍從、臺諫猶慎嘿,況其它乎?陰沴之興,未必不由此。」時廷臣多以中批斥去,彥穎又言:「臣下有過,宜顯逐之,使中外知獲罪之由以
 為戒。今譖毀潛行,斥命中出,在廷莫測其故,將恐陰邪得伸,善類喪氣,非盛世事也。」除吏部尚書。接送金賀正使,言兩淮兵備城築及裁減接送浮費甚悉,上嘉納焉。



 十二月,除端明殿學士、簽書樞密院事。二年閏九月,參知政事。金使至,上遣王抃諭金使稍變受書舊禮,議久不決。彥穎曰:「須於國體無損而事可濟,乃善,若如去年張子顏之行,不但無益。」時左司諫湯邦彥新進,冀僥幸集事,自許立節。彥穎言邦彥輕脫,必誤國。他日,對便
 殿,上復語及之。顏穎欲進說,上色動,宰相亟引退。遂以邦彥為申議國信使,且命福建造海船,起兩淮民兵赴合肥訓練,並詔諸軍飭戎備,中外騷然。彥穎復言:「兩淮州縣去合肥,遠者千餘里,近亦二三百里。令民戶三丁起其二,限三月而罷,事未集,民先失業矣。」上作色曰:「卿欲盡撤邊備耶?」彥穎曰:「今不得已,令三百里內,家起一丁詣合肥,三百里外,就州縣訓習,日增給錢米,限一月罷,庶不大擾。」翌日,復執奏,從之。洎邦彥辱命而還,彥穎
 論其罪,貶新州。



 彥穎在東府三歲,實攝相事,內降繳回甚多。內侍白札籍名造器械並犒師,降旨發左藏、封樁諸庫錢,動億萬計。彥穎疏歲中經費以進,因言:「虞允文建此庫以備邊,故曰『封樁』,陛下方有意恢復,茍用之不節,徒啟他日妄費,失封樁初意。」上矍然曰:「卿言是,朕失之矣。」自是絕不支。



 墜馬在告,力求去,以資政殿學士知紹興府,勤約有惠政。提舉洞霄宮,復參知政事,病羸,艱拜起,力辭,上曰:「老者不以筋力為禮,孟享禮繁,特免卿。」
 諫官論其子毆人至死,奉祠鐫秩。起知婺州,禁民屠牛,捐屬縣稅十三萬三千緡。復知紹興府,進資政殿大學士,再奉祠,進觀文殿學士。



 紹熙元年,致仕。家居凡十載,自奉澹約,食才米數合。室無姬媵,蕭然永日,與州縣了不相聞。薨,年八十一,贈少保,謚忠文。



 子沐,慶元中,與一時臺諫排趙汝愚,善類一空,公論丑之。



 範成大,字致能,吳郡人。紹興二十四年,擢進士第。授戶曹,監和劑局。隆興元年,遷正字。累遷著作佐郎,除吏部
 郎官。言者論其超躐,罷,奉祠。



 起知處州。陛對,論力之所及者三,曰日力,曰國力,曰人力,今盡以虛文耗之,上嘉納。處民以爭役囂訟,成大為創義役,隨家貧富輸金買田,助當役者,甲乙輪第至二十年,民便之。其後入奏,言及此,詔頒其法於諸路。處多山田,梁天監中,詹、南二司馬作通濟堰在松陽、遂昌之間,激溪水四十里,溉田二十萬畝。堰歲久壞,成大訪故跡,疊石築防,置堤閘四十九所,立水則,上中下溉灌有序,民食其利。



 除禮部員外
 郎兼崇政殿說書。乾道《令》以絹計臟,估價輕而論罪重,成大奏:「承平時絹匹不及千錢,而估價過倍。紹興初年遞增五分,為錢三千足。今絹實貴,當倍時直。」上驚曰:「是陷民深文。」遂增為四千,而刑輕矣。



 隆興再講和,失定受書之禮,上嘗悔之。遷成大起居郎,假資政殿大學士,充金祈請國信使。國書專求陵寢,蓋泛使也。上面諭受書事,成大乞並載書中,不從。金迎使者慕成大名,至求巾幘效之。至燕山,密草奏,具言受書式,懷之入。初進國書,
 詞氣慷慨,金君臣方傾聽,成大忽奏曰:「兩朝既為叔侄,而受書禮未稱,臣有疏。」搢笏出之。金主大駭,曰:「此豈獻書處耶?」左右以笏標起之,成大屹不動,必欲書達。既而歸館所,金主遣伴使宣旨取奏。成大之未起也,金庭紛然,太子欲殺成大,越王止之,竟得全節而歸。



 除中書舍人。初,上書崔寔《政論》賜輔臣,成大奏曰:「御書《政論》,意在飭綱紀,振積敝。而近日大理議刑,遞加一等,此非以嚴致平,乃酷也。」上稱為知言。張說除簽書樞密院事,成大
 當制,留詞頭七日不下,又上疏言之,說命竟寢。



 知靜江府。廣西窘匱,專藉鹽利,漕臣盡取之,於是屬邑有增價抑配之敝,詔復行鈔鹽,漕司拘鈔錢均給所部,而錢不時至。成大入境,曰:「利害有大於此乎?」奏疏謂:「能裁抑漕司強取之數,以寬郡縣,則科抑可禁。」上從之。數年,廣州鹽商上書,乞復令客販,宰相可其說,大出銀錢助之。人多以為非,下有司議,卒不易成大說。舊法馬以四尺三寸為限,詔加至四寸以上,成大謂互市四十年,不宜
 驟改。



 除敷文閣待制、四川制置使,疏言:「吐蕃、青羌兩犯黎州,而奴兒結、蕃列等尤桀黠,輕視中國。臣當教閱將兵,外修堡砦,仍講明教閱團結之法,使人自為戰,三者非財不可。」上賜度牒錢四十萬緡。成大謂西南諸邊,黎為要地,增戰兵五千,奏置路分都監。吐蕃入寇之路十有八,悉築柵分戍。奴兒結擾安靜砦,發飛山軍千人赴之,料其三日必遁,已而果然。白水砦將王文才私娶蠻女,常導之寇邊,成大重賞檄群蠻使相疑貳,俄禽文才
 以獻,即斬之。蜀北邊舊有義士三萬,本民兵也,監司、郡守雜役之,都統司又俾與大軍更戍,成大力言其不可,詔遵舊法。蜀知名士孫松壽年六十餘,樊漢廣甫五十九,皆掛冠不仕,表其節,詔召之,皆不起,蜀士由是歸心。凡人才可用者,悉致幕下,用所長,不拘小節,其傑然者露章薦之,往往顯於朝,位至二府。



 召對,除權吏部尚書,拜參知政事。兩月,為言者所論,奉祠。起知明州,奏罷海物之獻。除端明殿學士,尋帥金陵。會歲旱,奏移軍儲米
 二十萬振饑民,減租米五萬。水賊徐五竊發,號「靜江大將軍」,捕而戮之。以病請閑,進資政殿學士,再領洞霄宮。紹熙三年,加大學士。四年薨。



 成大素有文名,尤工於詩。上嘗命陳俊卿擇文士掌內制,俊卿以成大及張震對。自號石湖,有《石湖集》、《攬轡錄》、《桂海虞衡集》行於世。



 論曰:劉珙忠義世家,迨屬纊,以未雪仇恥為深恨。王蘭犯顏忠諫,剛腸嫉惡。方趙鼎、張浚非罪遠謫,朋交絕蹤,大寶獨從之游,逮斥權奸,了無顧忌。安節拒秦檜,排淵、
 覿,堅如金石,孤立無黨,死生禍福,曾不一動其心。當金兵犯大散關,剛中單騎星馳,夜起吳璘,一戰卻敵。成大致書北庭,幾於見殺,卒不辱命。俱有古大臣風烈,孔子所謂「歲寒然後知松柏之後凋」者歟?若祖舜奪楊願恩,褫秦熹秩,誅檜惡於既死,彥穎論事激烈,披露忠藎,直氣亦可尚
 已。



\end{pinyinscope}