\article{本紀第四十五}

\begin{pinyinscope}

 理宗五



 景定元年春正月丙子,詔獎賈似道功。庚辰,歲星、熒惑合在尾。壬辰,詔:「知涪州趙,聚糧不運餉兵士,遂為北有,已削一秩,罰輕,再削兩秩。」乙未,潼川城仙侶山。賈似
 道言:「高達守鄂州城凡三月,大元師北還。」二月丙午,詔賈似道以緡錢三千萬犒師,並示賞功之典。己酉,以高達為寧江軍承宣使、右金吾衛上將軍,賜緡錢五十萬;呂文德賜緡錢百萬、浙西良田百頃;鄂州戰守將士賜緡錢三千萬;王鑒、孫虎臣、蘇劉義等各官十轉。高達遷湖北安撫副使、知江陵府兼夔路策應使,陳奕、阮思聰並正任防禦使。江西、湖南帥司言:大元兵破瑞州、臨江軍城,興國、壽昌、洪、撫、全、永、衡諸郡民皆被兵,存者奔竄
 它所。甲寅,詔臨江守臣陳元桂死節,官五轉,贈寶章閣待制。與一子京官、一子選人恩澤。給緡錢十萬治葬,立廟死所,謚曰正節。瑞州守臣陳昌世治郡雖有善政,兵至,民擁之以逃,以棄城失守,削三秩勒停。」乙卯,詔孫虎臣和州防禦使,張世傑以下十三人各官五轉;立功將士並補兩官資,賜銀絹。庚申,雨雹。辛酉,大元遣偏師自大理由廣南抵衡州,向士璧會合劉雄飛逆戰於道,俘民獲還者甚眾。詔雄飛升保康軍承宣使,餘轉官、賜銀
 錢。賈似道賜金器千兩、幣千匹,命國子監主簿劉錫趣召赴闕。向士璧遷兵部侍郎,職任依舊。呂文德、高達、陳奕等各賜金幣有差。丙寅,大元軍過分寧、武寧二縣,河湖砦都監權巡檢張興宗死之,詔贈武翼郎,官一子承信郎,以緡錢三萬給其家。湖南諸將溫和轉左武大夫、帶行遙郡刺史,李虎官三轉、帶行閣門宣贊,鄮進帶行復州團練使,各賜銀絹,旌其守禦之功。



 三月戊辰朔,日有食之。庚午,命夏貴兼黃、壽策應使,總舟師。癸酉,以橫
 山之戰將士效節,多死行陣,總管張世雄、沉彥雄、陳喜、秦安、李孝信、鄭俊、李安國各贈十官資,賜緡錢萬恤其家。甲戌,賞夏貴鴻宿州、白鹿磯戰功,遷福州觀察使,職任仍舊。將士推賞。乙亥,詔全、岳、永、衡、柳、象、瑞、興國、南康、隆興、江州、臨江、潭州諸縣經兵,農民失業,應開慶元年以前二稅盡除之。癸未,賈似道奏蘱草坪大戰,進至黃州。乙酉,詔範文虎轉左武大夫、環衛官、黃州武定諸軍都統制,張世傑環衛官、職任依舊。鄂州統制張勝死於
 漢陽戰陣,贈官五轉,官其子煥進武校尉。丙戌,賈似道言,自鄂趨黃,與北朝回軍相遇,諸將用命捍禦。詔孫虎臣、範文虎、張世傑以下各賜金帛。



 夏四月戊戌朔,侍御史沈炎疏吳潛過失,以「忠王之立,人心所屬,潛獨不然,章汝鈞對館職策,乞為濟邸立後,潛樂聞其論,授汝鈞正字,奸謀叵測。請速詔賈似道正位鼎軸。」詔朱熠、戴慶□輪日判事,大政則共議以聞。己亥,賈似道表言夏貴等戰新生洲,進至白鹿磯,皆身自督戰有功。詔赴闕。庚
 子,以王堅為侍衛步軍司都指揮使。戊申,以劉整知瀘州兼潼川安撫副使。己酉,揚州大火。吳潛以觀文殿大學士提舉臨安府洞霄宮。癸丑,進賈似道少師,依前右丞相兼樞密使,進封衛國公;朱熠知樞密院事兼參知政事;饒虎臣參知政事;戴慶□同知樞密院事兼參知政事,皮龍榮端明殿學士、簽書樞密院事。己未,以夏貴為保康軍承宣使、左金吾衛上將軍、知淮安州兼淮東安撫副使、京東招撫使,賜金器幣、溧陽田三十頃。壬戌,進
 馬光祖資政殿大學士,職任依舊。癸亥,以呂文德兼夔路策應使。丙寅,命馬光祖兼淮西總領財賦。



 五月戊辰朔,詔趙葵依舊少保、兩淮宣撫使、判揚州,進封魯國公;徐清叟觀文殿大學士、知建寧府。饒虎臣罷。壬申,李曾伯、史巖之並落職解官,曾伯坐嶺南閉城自守,不能備御;巖之坐鄂州圍解,大元兵已渡江北還,然後出兵,又命程芾任事,以致敗績。甲戌,詔贈呂文信寧遠軍承宣使,立廟賜額,子師憲帶行閣職,更與兩子承信郎;輔周
 和州防禦使,錄其白鹿磯死事。乙亥,詔李虎馭軍無律,貸命追奪、竄鬱林州。丁丑,賜賈似道玉帶。庚辰,戴慶□卒,贈資政殿大學士。壬午,熒惑犯鬥。癸未,以皮龍榮兼權參知政事;沈炎端明殿學士、同簽書樞密院事;馬坤鄂州都統制,駐扎江陵府。甲申,祈雨。戊子,詔饒虎臣以資政殿學士提舉臨安府洞霄宮、任便居住。楊棟召赴闕。壬辰,以姚希得為敷文閣待制、知慶元府兼沿海制置使。乙未,詔李庭芝起復秘閣修撰、主管兩淮安撫制
 置司公事兼知楊州。



 六月丁酉朔,夏貴奏淮安戰功。庚子,竄丁大全於南康軍。壬寅,詔立皇子忠王祺為皇太子,賜字長源。戊申,王野卒。壬子,賜李遇龍金帶。陳奕帶御器械,依舊鎮江駐扎御前諸軍都統制,賜田三十頃。詔升巢縣為鎮巢軍。甲寅,楊棟、葉夢鼎並太子詹事。乙卯,陳韡進一秩、福建安撫使、知福州,徐清叟觀文殿學士、知泉州。



 秋七月丁卯朔,皇太子入東宮,行冊禮,大赦。壬申,貴妃閻氏薨,賜謚惠昭。東南有星如太白。丁亥,命
 皇太子昕朝侍立。戊子,上謂宰執曰:「北朝使來,事體當議。」賈似道奏:「和出彼謀,豈容一切輕徇?倘以交鄰國之道來,當令入見。」己丑,侍御史何夢然劾丁大全、吳潛欺君無君之罪。庚寅,賈似道兼太子少師,朱熠、皮龍榮、沉炎並兼賓客。辛卯,詔丁大全削三秩、謫居南安軍,吳潛奪觀文殿大學士,罷祠,削二秩、謫居建昌軍。癸巳,詔舉孝廉。



 八月壬寅,以程元鳳為淮、浙發運使、判平江府。己酉,太陰犯填星。詔皇太子受冊畢,賈似道、朱熠、皮龍榮、
 沉炎各進一秩,東宮官吏諸軍兵等官一轉,餘皆推恩。壬子,與[B170]薨,贈少師,謚忠憲。太白犯房。壬戌,李曾伯、史巖之各削二秩。甲子,饒虎臣削二秩,奪資政殿學士,罷祠。



 九月癸酉,守瀘州劉整以功來上。丁丑,知漳州、節制屯戍軍馬洪天錫言,援例創闢乾官一員,報行軍機密文字,奏可。辛巳,祀明堂,大赦。丙戌,熒惑犯壁。戊子,李松壽犯淮安。



 冬十月乙未朔,詔申嚴邊防。甲辰,詔黨丁大全、吳潛者,臺諫其嚴覺察舉劾以聞,當置於罪,以為同
 惡相濟者之戒。時似道專政,臺諫何夢然、孫附鳳、桂錫孫、劉應龍承順風指,凡為似道所惡者無賢否皆斥,帝弗悟其奸,為下是詔。戊申,李松壽修南城,詔趣淮閫調兵毀之。壬子,破李松壽兵於漣水城下,夷南城舊址。乙卯,有星自東北急流向太陰。壬戌,竄吳潛于潮州。



 十一月丙寅,詔內侍何時修削二秩,永罷不敘。洪燾知臨安府兼浙西安撫使。壬午,以中軍統制、知簡州馬千權興州都統兼知合州。戊子,熒惑與填星順行,太陰犯房。十
 二月甲午朔,詔華亭奉宸莊,其隸外廷助軍餉。包恢敘復元官職、知常州。辛丑,建陽縣嘉禾生,一本十五穗,詔改建陽為嘉禾縣。甲寅,呂文德上夔路戰功。乙卯,少師、廬陵郡王思正薨,謚簡惠。印應雷直徽猷閣、知江州、主管江西安撫司公事,節制蘄、黃、興國三郡。庚申,以監察御史桂錫孫言,追寢全子才敘復之命。



 二年春正月癸亥朔,詔:「監司率半歲具劾去贓吏之數來上,視多寡為殿最,行賞罰。守臣助監司所不及,以一
 歲為殿最,定賞罰。本路、州無所劾,而臺諫論列,則監司守臣皆以殿定罰。有治狀廉聲者,摭實以聞。」乙丑,城安慶。詔馬光祖進二秩。丁丑,命皇太子謁拜孔子於太學。己卯,福建安撫使陳韡累疏請老,詔進一秩,守觀文殿學士致仕。以董槐判福州、福建安撫使。乙酉,詔封張栻為華陽伯,呂祖謙開封伯,從祀孔子廟庭。



 二月丙申,孫虎臣戰邳州,全師而歸。癸卯,詔諸路監司申嚴偽會賞罰之令。甲寅,進封周國公主。



 三月壬戌朔,日有食之。乙
 亥,故寧遠軍承宣使張祥、都統制閻忠進,以援蜀之功,祥贈節度使,忠進贈復州團練。除恩澤外,各更官一子承信郎,賜緡錢二萬。戊寅,賈似道等上《玉牒》、《日歷》、《會要》、《經武要略》及孝宗、光宗、寧宗《實錄》,詔似道、皮龍榮、朱熠、沉炎各進二秩。



 夏四月癸巳朔,餘思忠追毀出身文字,除名勒停、竄新州。乙未,以皮龍榮參知政事,沉炎同知樞密院事兼權參知政事,何夢然簽書樞密院事,俞興保康軍承宣使、四川安撫制置使。丙申,呂文德超授太
 尉、京湖安撫制置屯田使、夔路策應使兼知鄂州,李庭芝右文殿修撰、樞密都承旨、兩淮安撫制置副使、知揚州。己亥,詔申嚴江防。壬寅,呂文德兼湖廣總領財賦。乙巳,馬天驥資政殿學士、知福州、福建安撫使,呂文福帶御器械、淮西安撫副使兼知廬州,官一轉。戊申,馬光祖進觀文殿學士,職任依舊。乙卯,竄吳潛於循州。丙辰,竄丁大全於貴州,追削二秩。丁巳,楊鎮授左領軍衛將軍、駙馬都尉,高達知廬州、淮西安撫副使。



 五月癸亥,賈似
 道請祠祿,詔不允。庚午,謝方叔敘復觀文殿大學士致仕。戊寅,以劉雄飛知夔州、夔路安撫使。乙酉,王堅遷左金吾衛上將軍、湖北安撫使兼知江陵府。



 六月乙未,詔霖雨為沴,避殿、減膳、徹樂。乙巳,詔近畿水災,安吉為甚,亟講行荒政。辛亥,以範文虎為左領軍衛大將軍,主管侍衛步軍司兼馬軍司。



 秋七月甲子,蜀帥俞興奏守瀘州劉整率所部兵北降,由興構隙致變也。至是,興移檄討整。辛未,制置使蒲擇之坐密通蠟書叛賊羅顯,詔竄
 萬安軍。太陰犯鬥。乙亥,以厲文翁為資政殿學士、沿海制置使、知慶元府。戊寅,王惟忠家訟冤,詔奪謝方叔合得恩數。丁大全責授新州團練使、貴州安置。臺臣吳燧奪職罷祠,陳大方、胡大昌皆鐫官。壬午。陳韡卒,贈少師,謚忠肅。丙戌,吳潛責授化州團練使、循州安置。



 八月壬辰,命韓宣兼常德、辰、沅、澧、靖五郡鎮撫使。呂文德兼四川宣撫使,範文虎以白鹿磯之功賞七官,以五官轉行遙郡防禦使,餘官給憑。丁酉,詔奪向士璧從官恩數,窮
 竟侵盜掩匿之罪。時以兵退,遣官會計邊費,似道忌功,欲以污篾一時閫臣,士璧及趙葵、史巖之、杜庶皆責征償。信州謝枋得以趙葵檄給錢粟募民兵守御,至是,自償萬緡。壬寅,築周國公主館於安濟橋。乙巳,以江萬里為端明殿學士、同簽書樞密院事,依執政恩數。



 九月辛酉,詔湖、秀二郡水災,守令其亟勸分,監司申嚴荒政。乙亥,李庭芝言李松壽已遁。大元使赦經久留真州,帝趣與錫賚。經之留,謀出賈似道,帝惑其言不悟。蓋似道在
 鄂時,值我世祖皇帝歸正大位撤兵,似道自詭有再造之功,諱言歲幣及講和之事,故不使經入見。



 冬十月癸巳,呂文德言已復瀘州外堡,擬即對江壘石為城,以示持久之計,從之。戊戌,雷電。甲申,詔申獎賈似道鄂州之功。丙午,以何夢然同知樞密院事兼參知政事。癸丑,程元鳳授特進、觀文殿大學士、醴泉觀使兼侍讀。甲寅,皇太子擇配,帝詔其母族全昭孫之女擇日入見。寶祐中,昭孫沒於王事,全氏見上,上曰:「爾父死可念。」對曰:「臣妾
 父固可念,淮、湖百姓尤可念。」上曰:「即此語可母天下。」迨開慶丁大全用事,以京尹顧巖女為議,大全敗,故有是命。丙辰,沈炎資政殿學士、提舉臨安府洞霄宮、任便居住。



 十一月己未朔,劉雄飛和州防禦使、樞密副都承旨、四川安撫制置副使兼知重慶府、四川總領、夔路轉運使。庚申,周國公主館成,詔董宋臣、李忠輔各官一轉。甲戌,資政殿學士致仕汝騰卒,贈官四轉,謚忠清。安南國貢像二。丁丑,馬光祖提領戶部財用兼知臨安府、浙西
 安撫使。下嫁周國公主於楊鎮。己卯,以鎮為宜州觀察使,賜玉帶,尋升慶遠軍承宣使。詔:「駙馬都尉楊鎮家合有賞典,楊蕃孫官兩轉,楊鐸、楊鑒官一轉,並直秘閣,餘轉官進封有差。」癸未,封全氏永嘉郡夫人。十二月庚寅,改竄蒲擇之於南康軍。辛卯,宰臣奏:「太子語臣等言:『近奉聖訓,夫婦之道,王化之基,男女正位,天地大義。平日所講修身齊家之道,當真履實踐,勿為口耳之學。』請宣付史館,永為世程法。」從之。甲午,以皮龍榮兼權知樞密
 院事,何夢然參知政事兼太子賓客,馬光祖同知樞密院事兼太子賓客、知臨安府。己亥,太陰犯五車。壬寅,江萬里依舊端明殿學士、提舉臨安府洞霄宮、任便居住。癸卯,冊永嘉郡夫人全氏為皇太子妃。



 三年春正月戊子朔,詔申飭百官盡言。詔量移丁大全、吳潛黨人,並永不錄用。壬戌,詔:「陳塏等耆年奉祠,宜示崇獎:陳塏端明殿學士,林彬之寶章閣待制,史季溫直華文閣,丁仁直寶謨閣,仍並予祠祿。」甲子,福建路安撫
 使馬天驥進資政殿大學士,職任依舊。乙丑,詔諭西蜀郡縣等官,已授遇闕,毋遙受虛批月日,違期不赴。丁卯,以善諮嗣濮王。戊辰,周國公主進封周、漢國公主。庚午,賜賈似道第宅於集芳園,給緡錢百萬,就建家廟。甲戌,詔權知梁山軍李鑒守城有功,帶行閣門宣贊舍人,就知梁山軍。復瀘州,改為江安軍。呂文德進開府儀同三司。



 二月丁亥朔,臨安、安吉、嘉興屬邑水,民溺死者眾,詔守臣給示彗瘞之。詔獎諭制置司,其立功參贊將士,進秩、
 升職犒給有差。乃裕授檢校少保。以皮龍榮為資政殿大學士、知潭州、湖南安撫使。乙巳,太陰入氐。戊申,詔省試中選士人覆試於御史臺,為定制。庚戌,李□以漣、海三城叛大元來歸,獻山東郡縣。詔改漣水為安東州,授□保信寧武軍節度使、督視京東河北等路軍馬、齊郡王,復其父李全官爵。□即松壽。



 三月乙丑,以孫附鳳為端明殿學士、簽書樞密院事兼太子賓客。辛未,詔升海州東海縣為東海軍。丁丑,汪立信升直華文閣、知江州、
 主管江西安撫司公事,節制蘄、黃、興國三郡軍馬。庚辰,呂文福依舊職差知濠州兼淮西招撫使。



 夏四月庚寅,太白晝見。庚子,熒惑與歲星合在危。甲辰,有流星大如杯。



 五月壬戌,熒惑犯壁壘陣。丙寅,雨雹。己巳,詔:「廣西靜江屯田,小試有效,其邕、欽、宜、融、柳、象、潯諸州守臣任責措置,經略安撫以課殿最,仍條具來上。」辛未,馬光祖以病請祠,詔知福州兼福建安撫使。丁丑,賜禮部進士方山京以下六百三十七人及第、出身。庚辰,夏貴上蘄縣
 戰功。



 六月戊子,詔李□受圍,給銀五萬兩,下益都府犒師,遣青陽夢炎率師援之。庚寅,以孫附鳳兼權參知政事,楊棟端明殿學士、同簽書樞密院事兼太子賓客。壬辰,吳潛沒於循州,詔許歸葬。己亥,董槐乞致仕,詔授特進。戊申,詔青陽夢炎援李□,不俟解圍,輒提援兵南歸,諭制置司劾之。己酉,有流星大如熒惑。庚戌,安南國王日煚上表乞世襲,詔授檢校太師、安南國王,加食邑,男威晃授靜海軍節度觀察處置使、檢校太尉兼御史大
 夫、上柱國、安南國王、效忠順化功臣,仍賜金帶、器幣、鞍馬。癸丑,詔應謫臣僚終於貶所者,許令歸葬。



 秋七月丙辰,詔州縣官廩祿不時給者,御史臺覺察,或以他物折支,計贓論罪。壬戌,董槐薨,贈少師,謚文清。庚午,周、漢國公主薨,賜謚端孝。壬申,江州都統聶世興調遣入蜀,托疾憚行,詔奪二秩,押往京湖制司自效。戊寅,侍御史範純父言:「前四川制置使俞興,□石功啟戎,罷任鐫秩,罰輕,乞更褫奪,以紓眾怒。」奏可。辛巳,詔重修《吏部七司條法》。
 癸未,詔申嚴諸路郡縣苛取苗米之禁。甲申,夜有白氣亙天。



 八月甲午,海州石湫堰成,詔知州張漢英帶行遙郡刺史、馬步軍副總管,帶行環衛官。丁酉,築蘄州城。知州王益落階官,正任高州刺史;制置使汪立信上《新城圖》,詔獎諭。戊戌,李□兵敗,為大元所誅,事聞,詔沿邊諸郡嚴邊防。汪立信升直敷文閣、主管沿江制置司公事、知江州、主管江西安撫司公事。癸卯,太陰犯昴。乙巳,沿江制置使姚希得進寶章閣學士,職任依舊。



 九月壬申,
 召陳奕赴樞密院稟議。丙子,有流星大如太白。丁丑,溫州布衣李元老,讀書安貧,不事科舉,今已百四歲,詔補迪功郎致仕,本郡給奉。閏九月甲申朔,太白晝見。丙戌,流星透霞,大如太白。戊戌,詔刑部長貳、大理卿、少卿,歲終無評事可舉,即舉在京三獄官。庚子,有流星大如太白。丙午,詔應知縣罪罷,雖經赦,毋注緊、望闕,著為令。戊申,詔:「紹興府火,給貸居民錢,今及二載,民貧可憫,悉除勿征。」冬十月乙卯,詔蠲四川制總、州縣鹽酒榷額。己未,
 太陰犯歲星。甲子,以楊棟簽書樞密院事、兼權參知政事兼太子賓客,葉夢鼎端明殿學士、同簽書樞密院事兼太子賓客。丁卯,呂文德言遣將校御敵,多逗遛不進,且奏功失實,具姓名上聞。詔呂文煥、王達、趙真削兩秩,馬坤、王甫削一秩,餘貶降有差。太陰犯五車星。庚午,太白入氐。甲戌,歸化州岑從毅納土輸賦,獻丁壯為王臣。詔改歸化為來安州,從毅進秩修武郎、知州事,令世襲。丙子,詔安豐六安縣升軍使。



 十一月壬辰,丁大全竄貴
 州,招游手,立將校,置弓矢舟楫,縱僕隸淫虐軍民,詔奪大全貴州團練使,移置新州。癸巳,馬光祖乞祠祿,詔提舉臨安府洞霄宮、任便居住。丙申,徐清叟薨,贈少師,謚忠簡。丁酉,資陽砦主萬戶小哥及其子眾家奴叛來降,詔小哥賜姓王,名永堅,補武翼大夫、夔路副總管,重慶府駐扎。戊戌,以夏貴知廬州、淮西安撫副使。丁未,皇孫容州觀察使封資國公焯薨,贈保靜軍節度使、廣國公。熒惑、填星合在婁。十二月辛巳,呂文德累疏辭兼四川
 宣撫,詔仍兼四川策應使。



 四年春正月壬午朔,詔侍從、臺諫、給舍、卿監、郎官以上及制總、監司各舉所知,不拘員限,不如所舉,行連坐法。戊子,林希逸言蒲陽布衣林亦之、陳藻有道之士,林公遇幼承父澤,奉親不仕,詔林亦之、陳藻贈迪功郎,林公遇元官上進贈一官。詔董宋臣同提舉奉安符寶所,仍奉祠祿。己亥,嚴州火。丙午,詔革詞訴改送之弊。



 二月癸丑,詔吳潛、丁大全黨人遷謫已久,遠者量移,近者還本
 貫,並不復用。丁大全溺死藤州,詔許歸葬。詔俞興往歲失陷瀘城,更削一秩。丁巳,置官田所,以劉良貴為提領,陳誾為檢閱。戊午,日暈周匝。乙亥,呂文德浚築鄂州、常、澧城池訖事,詔獎之,守臣韓宣轉遙郡承宣使,蘇劉義吉州刺史。



 三月丁亥,以呂文德為寧武、保康軍節度使,職任依舊;劉雄飛樞密都承旨、四川安撫制置使兼知重慶府、四川總領財賦、夔路轉運使。加授姚希得刑部尚書,李庭芝兵部侍郎,朱祀孫太府卿,汪立信太府少
 卿,並依舊任。壬辰,太陽赤黃暈。丁酉,以王堅知和州兼管內安撫使,呂思望知濠州兼淮西招撫使。庚子,以何夢然兼權知樞密院事。丁未,詔知寧國府趙汝禖推行經界,不擾而辦,職事修舉,升直華文閣,依舊任。戊申,忠州防禦使貴傑授福州觀察使。



 夏四月乙卯,太陰犯權星。丙寅,官田所言,知嘉興縣段浚、知宜興縣葉哲佐買公田不遵元制,詔罷之。戊辰,太陽赤黃暈,不匝。



 五月庚寅,太陰入氐。丁酉,婺州布衣何基,建寧府布衣徐幾,皆
 得理學之傳。詔各補迪功郎,何基婺州教授兼麗澤書院山長,徐幾建寧府教授兼建安書院山長。戊戌,四川制司言:二月甲寅,大元兵攻嘉定城,馬坤出戰御之。詔馬坤援夔遷延,削一秩,令以所轉四官理作敘復。流星出自角宿距星。



 六月壬子,祈雨。乙卯,京城火。丙辰,詔饒虎臣敘復元官,依舊提舉太平興國宮。庚申,詔:平江、江陰、安吉、嘉興、常州、鎮江六郡已買公田三百五十餘萬畝,今秋成在邇,其荊湖、江西諸道,仍舊和糴。丙寅,詔公田
 竣事,劉良貴官兩轉,陳誾、廖邦傑洎六郡官進秩有差。丁卯,流星出自河鼓。庚午,宰執進《玉牒》、《日歷》、《會要》、《經武要略》及《徽宗長編》、《寧宗實錄》,詔賈似道以下官兩轉。



 秋七月壬辰,敕令所進《寧宗以來寬恤詔令》。戊戌,以董宋臣為入內內侍省押班。



 八月甲寅,董宋臣以病乞收回恩命,請祠,詔賜告五月。乙卯,流星出自天倉星。



 九月甲申,詔趙汝禖為太府少卿、淮東總領財賦。辛卯,祀明堂,大赦。甲午,以何夢然知樞密院事兼參知政事,楊棟同
 知樞密院事兼權參知政事,葉夢鼎簽書樞密院事。



 冬十月己未,詔發緡錢百四十萬,命浙西六郡置公田莊。甲子,命張玨興元府駐扎、御前諸軍都統制兼知合州。



 十一月己亥,福州火。十二月丁未朔,詔皇太子宮講官詹事以下,日輪一員,辰入酉出,專講讀,備咨問,以稱輔導之實。己未,詔在京置窠柵、私系囚並非法獄具,臺憲其嚴禁戢,違者有刑。辛未,太白、歲星順行。



 五年春正月丁丑朔,詔崇經術,考德行。癸巳,出奉宸庫
 珠,香、象、犀等貨下務場貨易,助收幣楮。庚子,太子右諭德湯漢三乞休致,授秘閣修撰、知福州、福建安撫使。



 二月壬戌,流星出自畢。甲子,太陰犯房。丁卯,太陰犯鬥。辛未,雨土。



 三月辛巳,王堅卒,賜謚忠壯。馬光祖依舊觀文殿學士、沿江制置使、知建康府、江東安撫使、行宮留守。己丑,日暈周匝。



 夏四月丙午,詔:管景模妻孥陷沒,效忠愈堅,平時所得奉入,率以撫恤將士,遂至空乏,特賜緡錢三十萬。尋賜金帶。丁未,以夏貴為樞密都承旨、四川
 安撫制置使兼知重慶府、四川總領、夔路轉運使。辛亥,詔郡邑行鄉飲酒禮。癸丑,太陰入太微垣。乙卯,信陽軍將領余元友等提兵防護春耕有功,補轉兩官資。戊午,太白晝見。乙丑,何夢然、馬天驥以臺臣劾罷。己巳,江萬里以資政殿學士知建寧府,李曾伯以觀文殿學士知慶元府、沿海制置使。庚午,太白、歲星合於婁。



 五月庚辰,何夢然以資政殿大學士知建寧府。辛卿,以楊棟參知政事,葉夢鼎同知樞密院事權參知政事,姚希得端
 明殿學士、同簽書樞密院事,馬天驥提舉洞霄宮。甲午,流星出自河鼓,大如太白。乙未,安南國奉表謝恩,進方物,詔卻之,仍賜金帛,以獎恭順。己亥,太白經天,晝見。



 六月甲辰朔,知衢州謝塈因寇焚掠常山縣棄城遁,詔削三秩,褫職不敘。臺臣言衢州詹沔之變,乃謝塈任都吏徐信苛取激之,塈罪重罰輕。詔斬信,籍其家,塈再削兩秩勒停。丁未,詔饒虎臣敘復資政殿學士,依前通奉大夫,差遣如故。甲寅,加授李庭芝寶章閣直學士,依舊任,
 朱祀孫右文殿修撰、知靜江府、廣西經略使,汪立信秘閣修撰、樞密副都承旨、沿江制置副使兼知江州、江西安撫使。詔呂文德職事修舉,與官一轉。太陰犯心。戊午,祈雨。太白犯天關星。乙丑,命董宋臣兼主管御前馬院、御前酒庫。戊辰,熒惑、歲星並行。己巳,太白、太陰並行入井。庚午,太陽赤黃暈。



 秋七月甲戌,彗星出柳。丁丑,詔避殿減膳,應中外臣僚許直言朝政闕失。己卯,流星出自右攝提星,彗星退於鬼。辛巳,彗星退於井。甲戌,京城大
 火。癸巳,謝奕昌卒,贈少保,追封臨海郡王,謚莊憲。甲午,填星守畢。乙未,馬天驥以臺臣劾其貪贓,奪職罷祠,其子時楙削一秩、罷新任。丙申,知嘉定府洪濤言:新繁縣御容殿前枯木再榮,殿有畫太祖像;又順化人楊嗣光等奉太宗、真宗、仁宗、英宗、神宗像來歸,令櫝藏府中天慶觀。詔本府選差武臣迎奉赴行在所,嗣光補武階兩資。祈雨。臺臣言太子賓客楊棟指彗為蚩尤旗,欺天罔君,詔棟罷職予祠。戊戌,彗星退於參。



 八月壬寅朔,熒惑
 與填星合。丙午,以楊棟知建寧府。戊午,彗星消伏。甲子,彗星復見於參。辛未,彗星化為霞氣。



 九月己丑,日生格氣。癸巳,內侍李忠輔以臺臣劾其貪肆欺罔,削兩秩放罷。乙未,建寧府教授謝枋得校文宣城及建康漕闈,發策十餘問,言權奸誤國,趙氏必亡。左司諫舒有開劾其怨望騰謗,大不敬,竄興國軍。



 冬十月丙午,太陰犯鬥。辛亥,詔十七界會浸輕,並以十八界會易之,限一月止。乙丑,詔行關子銅錢法,每百作七十七文足,以一準十八
 界會之三。帝有疾,不視朝。丙寅,大赦。丁卯,帝崩。遺詔皇太子祺即皇帝位。咸淳元年三月甲申,葬於會稽之永穆陵。二年十二月丙戌,謚曰建道備德大功復興烈文仁武聖明安孝皇帝,廟號理宗。



 贊曰:理宗享國久長,與仁宗同。然仁宗之世,賢相相繼。理宗四十年之間,若李宗勉、崔與之、吳潛之賢,皆弗究於用;而史彌遠、丁大全、賈似道竊弄威福,與相始終。治效之不及慶歷、嘉祐,宜也。蔡州之役,幸依大朝以定夾
 攻之策,及函守緒遺骨,俘宰臣天綱,歸獻廟社,亦可以刷會稽之恥,復齊襄之仇矣。顧乃貪地棄盟,入洛之師,事釁隨起,兵連禍結,境土日蹙。郝經來使,似道諱言其納幣請和,蒙蔽抑塞,拘留不報,自速滅亡。籲,可惜哉!由其中年嗜欲既多,怠於政事,權移奸臣,經筵性命之講,徒資虛談,固無益也。雖然,宋嘉定以來,正邪貿亂,國是靡定,自帝繼統,首黜王安石孔廟從祀,升濂、洛九儒,表章朱熹《四書》,丕變士習,視前朝奸黨之碑、偽學之禁,
 豈不大有徑庭也哉!身當季運,弗獲大效,後世有以理學復古帝王之治者,考論匡直輔翼之功,實自帝始焉。廟號曰「理」,其殆庶乎!



\end{pinyinscope}