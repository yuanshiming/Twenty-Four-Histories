\article{志第六十七 禮十七(嘉禮五)}

\begin{pinyinscope}

 巡幸養老視學賜進士宴幸秘書省進書儀大射儀鄉飲酒禮



 巡幸之制,唐《開元禮》有告至、肆覲、考制度之儀,《開寶通禮》因之。



 太祖幸西京,所過賜夏、秋田租之半。真宗朝諸
 陵及舉大禮,途中皆服折上巾、窄袍,出京、過京城,服靴袍、具鸞駕。群臣公服系鞋,供奉班及內朝官前導。凡從官並日赴行宮,合班起居,晚朝視事,群臣不赴。中頓侍食,百官就宿頓迎駕訖,先發,或道途隘遠,則免迎駕。將進發,近臣、諸軍賜裝錢。出京,留司馬、步諸軍夾道左右,至新城門外奉辭,留守辭於門內,百官、父老辭於苑前,召留守等賜飲苑中。州縣長吏、留司官待於境。所過賜巡警兵、守津梁行郵治道卒時服錢履,父老綾袍、茶
 帛,途中賜衛士緡錢。所幸寺、觀,賜道、釋茶帛,或加紫衣、師號。吏民有以饔餼、酒果、方物獻者,計值答之。命官籍所過系囚、逋負者,日引對,多原釋。仍採訪民間疾苦,振恤鰥、寡、孤、獨。車服、度量、權衡有不如法,則舉儀制禁之。有奇材、異德及政事尤異者,孝子、順孫、義夫、節婦為鄉里所稱者,其不守廉隅、昧於正理者,並條析以聞。官吏知民間疾苦者,亦許錄奏。所過州、府,結彩為樓,陳音樂百戲。道、釋以威儀奉迎者,悉有賜。東京留守遣官表請還京,
 優詔答之。駕還京,大陳兵衛以入。



 凡行幸,太祖、太宗不常其數。自咸平中,車駕每出,金吾將軍帥士二百人,執□周繞,謂之禁圍,春、夏緋衣,秋、冬紫衣。郊祀、省方並增二百,服錦襖,出京師則加執劍。親王、中書、樞密、宣徽行圍內,餘官圍外。大禮備儀衛,則有司先布土為黃道,自宮至祀所,左右設香臺、畫甕、青繩闌幹。巡省在途則不設。



 凡巡省,翰林進號傳詩付樞密院,每夕摘字,令衛士相應為識。東京舊城城門、西京皇城司並契勘,內外城、宮廟門
 並勘箭,出入皆然。入藩鎮外城、子城門亦勘箭。朝陵定扈從官人數,入柏城者,僕射以上三人,丞、郎以上二人,餘各一人。東封,定仗內導駕官從人數,親王、中書、樞密、宣徽、三司使四人,學士、尚書丞郎、節度使三人,大兩省、大卿監、三司副使、樞密承旨、客省閣門使副、金吾大將軍押仗鳴珂、內殿崇班以上二人,餘各一人。命諸司巡察之。自後舉大禮,皆循此制。



 建炎元年七月,詔曰:「祖宗都汴,垂二百年。比年以來,圖慮弗臧,禍生所忽。肆朕纂
 承,顧瞻宮室,何以為懷?是用權時之宜,法古巡狩,駐蹕近甸,號召軍馬。朕將親督六師,以援京城及河北、河東諸路,與之決戰。歸宅故都,迎還二聖,以稱朕夙夜憂勤之意。」十月一日,車駕登舟,巡幸淮甸,宰執、侍從、百司、三衛、禁旅五軍將佐扈衛以行,駐蹕揚州。



 三年,幸杭州,自杭州幸江寧府,尋幸浙西,自浙西幸浙東。乃下詔曰:「國家遭金人侵逼,無歲無兵。朕纂承以來,深軫念慮,謂父兄在難,而吾民未撫,不欲使之陷於鋒鏑。故包羞忍恥,
 為退避之謀,冀其逞志而歸,稍得休息。自南京移淮甸,自淮甸移建康而會稽,播遷之遠,極於海隅。卑詞厚禮,使介相望。以至願去尊稱,甘心貶屈,請用正朔,比於藩臣,遣使哀祈,無不曲盡。假使金石無情,亦當少動。累年卑屈,卒未見從。生民嗷嗷,何時寧息?今諸路之兵聚於江、浙之間,朕不憚親行,據其要害。如金人尚容朕為汝兵民之主,則朕於事大之禮,敢有不恭!或必用兵窺我行在,傾我宗社,塗炭生靈,竭取東西金帛、子女,則朕亦
 何愛一身,不臨行陣,以踐前言,以保群生。朕已取十一月二十五日移蹕,前去浙西,為迎敵計。惟我將士人民,念國家涵養之恩,二聖拘縻之辱,悼殺戮焚殘之禍。與其束手待斃,曷若並計合謀,同心戮力,奮勵而前,以存家國!」乃詔御前應奉官司自合扈從外,內太常寺據實用人數扈從,餘接續起發。四年正月,次臺州。二月,次溫州。三月,幸浙西。



 紹興元年,詔移蹕臨安府。六年,詔周視軍師,車駕進發,遣官奏告天地、社稷、宗廟。自臨安幸平
 江,尋幸建康。八年二月,還臨安。三十一年九月,詔:「金人背盟失信,今率精兵百萬,躬行天討,用十二月十日車駕進發,應行宮臨安府文武百僚城北奉辭。」其日,應文武百僚先詣城北幕次,俟車駕御舟將至,御史臺、閣門、太常寺分引文武百僚立班定,兩拜訖,俟御舟過,班退。三十二年正月,詔:「視師江上,北騎遁去,兩淮無警,已委重臣統護諸將經畫進討。今暫還臨安,畢恭文祔廟之禮。宜令有司增修建康百官吏舍、諸軍營砦,以備往來
 巡幸,可擇日進發。」車駕還宮。



 養老於太學,皇帝服通天冠、絳紗袍,乘金輅,至太學酌獻文宣王。三祭酒,再拜,歸御幄。比車駕初出,量時刻,遣使迎三老、五更於其第。三老、五更俱服朝服,乘安車,導從至太學就次;國老、庶老,有司預戒之,各服朝服,集於其次。大樂正帥工人、二舞入,立於庭。東上閣門、御史臺、太常寺、客省、四方館自下分引百官、宗室、客使、學生等,以次入就位,如視學班。太常博士贊三老、五更俱出次,
 引國老、庶老立於後,重行異位。



 禮直官、通事舍人引左輔奏請中嚴,少頃,又奏外辦,皇帝出大次,侍衛如常儀。大樂正令撞黃鐘之鐘,右五鐘皆應,協律郎跪,俯伏,舉麾興,宮架《乾安》之樂作,皇帝即御坐,樂止。典儀曰「再拜」,在位官皆再拜。三老、五更杖而入,各左右二人夾扶,太常博士前引,史臣執筆以從。三老、五更入門,宮架《和安》之樂作,至宮架北,北向立,以東為上。奉禮郎引群老隨入,位於其後,樂止。博士揖進,三老在前,五更在後,仍杖
 夾扶,宮架《和安》之樂作,至西階下,樂止。博士揖三老、五更自西階升堂,國老、庶老立堂下。三老、五更當御坐揖,群老亦揖,皇帝為興。次奉禮郎揖國老升堂,博士引三老、五更,奉禮郎引國老以下,各於席後立。典儀贊各就坐,贊者承傳,宮架《尊安》之樂作,三老、五更就坐。三公授幾、九卿正履訖,殿中監、尚食奉御進珍羞及黍稷等,先詣御坐前進呈,遂設於三老前,樂止。尚食奉御詣三老坐前,執醬而饋訖。尚醞奉御詣酒尊所,取爵酌酒,奉御
 執爵,奉於三老。次太官、良醞令以次進珍羞酒食於五更、群老之前,皆食。大樂正引工人升,登歌奏《惠安》之樂,三終。史臣既錄三老所論善言、善行,宮架作《申安》之樂。《憲言成福》之舞畢,文舞退,作《受成告功》之舞,畢,三老以下降筵,博士引三老、五更於堂下,當御坐前,奉禮郎引群老復位,俱揖,皇帝為興。三老、五更降階至堂下,宮架《和安》之樂作,出門,樂止。禮直官、通事舍人引左輔前奏禮畢,退,復位。興儀贊拜訖,皇帝降坐,太常卿導還大次,
 百僚以次退,車駕還宮。三老、五更升安車,導從還,翼日詣闕表謝。



 視學。哲宗始視學,遂幸國子監,詣至聖文宣王殿行釋奠禮,一獻再拜。御敦化堂,召從官賜坐,禮部、太常寺、本監官承事郎以上侍立,承務郎以下、三學生坐於東西廡,侍講吳安詩執經,祭酒豐稷講《尚書》無逸終篇,復命宰臣以下至三學生坐,賜茶,豐稷賜三品服,本監官、學官等賜帛有差。遂幸昭烈武成王廟,酌獻肅揖。



 徽宗幸
 太學,遂幸闢雍,奠獻如上儀。詔司業吳絪等轉官改秩,循資賜章服,文武學生授官,免省試、文解,賜帛有差。所司預設次於敦化堂後,又於堂上稍北當中兩間設次,南向設御坐。又設從官及講筵講書、執經官並太學官坐御坐之南,東西重行異位。太學生坐於兩廡,相向並北上。宰臣以下從官之次,設於中門外。



 皇帝酌獻文宣王畢,幸太學,降輦入次,簾垂更衣。禮直官、通事舍人引講官與侍立官入就堂下,皆系鞋。講書、執經官、學生各
 立堂下,東西相向。俟報班齊,皇帝升坐,班首奏萬福,在位者皆應喏訖,閣門使承旨臨階宣升堂,通事舍人喝拜,應在位者再拜訖,分左右升堂,各就位少立。起居郎、舍人分左右侍立。禮直官、通事舍人引講書及執經官就北向位,班首奏萬福,閣門使宣升堂,舍人喝再拜訖,分東西升堂,立於御坐左右。講書官在西,東向;執經官在東,西向;學生就北向位。舍人喝拜,在位者皆再拜,立於東西兩廡。內侍進書案,以經授執經官,稍前,進於案
 上。舍人喝拜就坐,宰臣以下並堂上坐,如閣門所進坐位圖。講書畢,通事舍人曰「可起」,群臣皆起,降階立。執經官降,講書官於御坐前致辭訖,亦降。舍人喝拜,如有宣答,即再喝拜。閣門宣坐賜茶,舍人喝拜訖,宰臣以下升堂,各立於位後,學生各就北向位。舍人喝拜,在位者俱拜訖,各分東西廡,以北為上下。舍人喝就坐,上下皆就坐。賜茶畢,禮直官、通事舍人引堂上官降階就位,舍人喝拜,在位者俱拜訖,禮直官引之以次出。學生就位,舍
 人喝拜,學生俱再拜,退。



 紹興十三年七月,國學大成殿告成,奉安廟像。明年二月,國子司業高閎請幸學,上從之。詔略曰:「偃革息民,恢儒建學。聲明丕闡,輪奐一新。請既方堅,理宜從欲。將款謁於先聖,仍備舉於舊章。」三月,上服靴袍,乘輦入監,止輦於大成殿門外。入幄,群臣列班於庭。帝出幄,升東階,跪上香,執爵三祭酒,再拜,群臣皆再拜,上降入幄。分奠從祀如常儀。尚舍先設次於崇化堂之後,及堂上之中南向設御坐。閣門設群臣班於
 堂下,如月朔視朝之儀。宰輔、從臣次於中門之外。上乘輦幸太學,降輦於堂,入次更衣。講官入就堂下講位,北向;執經官、學生皆立於堂下,東西相向。帝出次,升御坐,群臣起居如儀。乃命三公、宰輔以下升堂,皆就位,左右史侍立。講書及執經官北面起居再拜,皆命之升立於御坐左右。學生北面再拜,分立兩廡北上。內侍進書案牙簽,以經授執經官,賜三公、宰輔以下坐。講畢,群臣皆起,降階,東西相向立。執經官降,講官進前致詞,乃降,北
 面再拜,左右史降。乃賜茶,三公以下北面再拜,升,各立於位後。學生北面再拜,分兩廡立,上下就坐。賜茶畢,三公以下降階,學生自兩廡降階,北面再拜,群臣以次出。上降坐還次,乘輦還宮。時命禮部侍郎秦熹執經、司業高閎講《易》之《泰》,遂幸養正、持志二齋,賜閎三品服,學官遷秩,諸生授官免舉,賜帛有差。



 上既奠拜,注視貌象,翼翼欽慕,覽唐明皇及太祖、真宗、徽宗所制贊文,命有司悉取從祀諸贊,皆錄以進。帝遂作先聖及七十二子贊,
 冠以序文,親灑翰墨,以方載之,五月丙辰,登之彩殿,備儀衛作樂,命監學之臣,自行宮北門迎置學宮,揭之大成殿上及二廡。序曰:「朕自睦鄰息兵,首開學校。教養多士,以遂忠良。繼幸太學,延見諸生。濟濟在庭,意甚嘉之。因作《文宣王贊》。機政餘閑,歷取顏回而下七十二人,亦為制贊。用廣列聖崇儒右文之聲,復知『師弟子間纓弁森森、覃精繹思』之訓,其於世道人心庶幾焉。」二十六年十二月,言者謂:「陛下崇儒重道,制為贊辭,刻宸翰於琬琰,
 光昭往古。寰宇儒紳,敦不顧瞻《雲漢》之章?請奉石刻於國子監,以碑本遍賜郡學。」從之。



 淳熙四年,孝宗幸太學,如紹興之儀。命禮部侍郎李燾執經、祭酒林光朝講《大學》。尋幸武學,如太學之儀。帝肅揖武成王,不拜。嘉泰三年正月,寧宗幸太學,如淳熙之儀。淳祐元年正月,理宗幸太學,宗、武兩學官屬、生員並赴太學陪位,候車駕至學,詣先聖文宣王位,三上香,執爵三祭酒,俯伏,興,再拜,在位官皆再拜。皇帝至崇化堂,宰臣、使相、執政並起居。
 執經官由東階、講官由西階並升堂,於御前分東西相向立。次引國子監三學學官、學生一班北面再拜,贊各就坐,賜茶。俟講書畢,起,立班再拜。禮成,執經官、講書官、國子監三學官、生員各推恩轉官有差。咸淳三年正月戊辰,度宗幸太學祗謁,禮部尚書陳宗禮執經,國子祭酒雷宜中講《中庸》,餘並如儀。



 賜貢士宴,名曰「聞喜宴」。《政和新儀》:押宴官以下及釋褐貢士班首初入門,《正安》之樂作,至庭中望闕位立,樂止。
 預宴官就位,再拜訖。押宴官西向立,中使宣曰「有敕」,在位者皆再拜訖。中使宣曰「賜卿等聞喜宴」,在位者皆再拜,搢笏,舞蹈,又再拜。次引押宴官稍前謝坐再拜,在位者皆再拜。若賜敕書,即引貢士班首稍前,中使宣曰「有敕」,貢士再拜。中使宣曰「賜卿等敕書」,班首稍前,搢笏,跪,中使授敕書訖,少退,班首以敕書加笏上,俯伏,興,歸位再拜,在位者皆再拜。凡預宴官分東西升階就坐,貢士以齒。酒初行,《賓興賢能》之樂作,飲訖、食畢,樂止。酒再行,《
 於樂闢雍》之樂作。酒三行,《樂育人材》之樂作。酒四行,《樂且有儀》之樂作。酒五行,《正安》之樂作。再坐,酒行、樂作,節次如上儀。皆飲訖、食畢,樂止。押宴官以下俱興,就次,賜花有差。少頃,戴花畢,次引押宴官以下並釋褐貢士詣庭中望闕位立,謝花再拜,復升就坐,酒行、樂作,飲訖、食畢,樂止。酒四行訖,退。次日,預宴官及釋褐貢士入謝如常儀。



 寧宗慶元五年五月,賜新及第進士曾從龍以下聞喜宴於禮部貢院,上賜七言四韻詩,秘書監楊王休
 以下繼和以進,自後每舉並如之。



 幸秘書省。紹興十四年七月,新建秘書省成,秘書少監游操等援宣和故事,請車駕臨幸,詔從之。二十七日,幸秘書省,至右文殿降輦,頒手詔曰:「蓋聞周建外史,掌三皇、五帝之書;漢選諸儒,定九流、《七略》之奏。文德之盛,後世推焉。仰惟祖宗建開冊府,凡累朝名世之士,由是以興,而一代致治之原,蓋出於此。朕嘉興與學士大夫共宏斯道,乃一新史觀,新禦榜題,肆從望幸之忱,以示右文
 之意。嗚呼!士習為空言,而不為有用之學久矣。爾其勉修術業,益勵猷為,一德一心,以共赴亨嘉之會,用丕承我祖宗之大訓,顧不善歟!」遂陳累朝御書、御制、晉唐書畫、三代古器,次宣皇太子、宰臣以下觀訖,退。遂賜宴於右文殿,酒五行,群臣再拜退。車駕還內,賜少監游操三品服、御書扇,餘官筆墨,館閣官各轉一官。淳熙五年九月十三日,孝宗幸秘書省,如紹興十四年之儀,帝賦詩,群臣皆屬和。



 進書儀。紹興二十年五月八日,進呈《中興聖統》,太常博士丁屢明言:「乞比附進呈玉牒行禮。」二十四年,進呈《徽宗御集》,禮部言:「昨紹興十年,徽宗御制,擬以『敷文』名閣,今乞權安奉於天章閣,續俟崇建。」二十六年十月,進呈《太后回鑾事實》。二十七年三月,宰臣沉該言:「玉牒所官陳康伯等先次編修太祖皇帝玉牒,自誕聖至即位,自建隆元年至開寶九年,通修一十七年開基玉牒,舊制以梅紅羅面簽金字,今欲題曰《皇宋太祖皇帝玉牒》。又
 編修今上皇帝玉牒,自誕聖之後聖德祥瑞、建大元帥府事跡,至即帝位二十餘年,又自即位後編修至紹興二年,通修二十六年中興玉牒,今欲題曰《皇宋今上皇帝玉牒》。宣祖、太祖、太宗、魏王下各宗《仙源類譜》、五世昭穆,今已修寫進本,乞擇日進呈。」詔從其請。



 前期,儀鸞司、臨安府於玉牒殿上南向,設權安奉玉牒、類譜並《中興聖統》幄次;又至玉牒所向外,設騎從官及文武百官等侍班幕次;又於景靈宮內外,設騎從官幕次。進呈前一
 日,俟朝退,玉牒所提領官、都大提舉、諸司官、承受官、玉牒所官等赴本所幕次宿衛。俟儀仗樂人等排立,御史臺、閣門、太常寺分引玉牒所官詣玉牒殿下,北向立。禮直官引提領官詣玉牒殿下,北向立。禮直官揖、躬、拜,提領官拜,在位官皆再拜訖。次引提領官詣香案前,搢笏,三上香,執笏退,復位,皆再拜訖,班退,歸幕次宿衛。儀仗樂人作樂,晝夜更互排立。



 其日五更,御史臺、閣門、太常寺分引提領官、宰執、使相、侍從、臺諫、兩省官、知閣、禮
 官、南班宗室詣玉牒殿,北向立。禮直官揖、躬、拜,提領官拜,在位官皆再拜訖。次引提領官詣玉牒、類譜香案前,搢笏,三上香,執笏,退,復位。禮直官引提領官詣幄前,西向立。次騎從官分左右乘馬,俟玉牒所率輦官奉擎玉牒、類譜,腰輿進行,樂人作樂,儀衛、儀仗迎引。次引提領官、宰執、使相、侍從、臺諫、兩省官、知閣、禮官、南班宗室騎從,至和寧門下馬,執笏步從玉牒、類譜至垂拱殿門外幄次,步從官權歸幕次,樂止。儀衛、樂人等並於幄次前排
 立,俟進呈玉牒、類譜,並如閣門儀訖。



 俟玉牒、類譜出殿門,御史臺、閣門、太常寺分引提領官、宰執、使相、侍從、臺諫、兩省官、知閣、禮官、南班宗室分左右執笏步從,儀衛樂人前引,迎奉出皇城北宮門,步從等官上馬騎從,至和寧門外。前引將至玉牒所,御史臺、閣門、太常寺分引文武百官於玉牒所門內殿門外立班,內文臣厘務通直郎以上及承務郎見任寺監主簿執事官以上,武臣修武郎以上,迎拜訖。如值雨,地下沾濕,迎拜官吏不迎
 拜。騎從官至玉牒所,並下馬執笏步從,詣玉牒殿下,分東西相向立。禮直官引提領官詣玉牒、類譜幄前,西向立定。



 俟玉牒所率輦官奉擎玉牒、類譜入幄,儀仗、儀衛、輦官、樂人更互排立。提領官、宰執、使相、侍從、臺諫、兩省官、知閣、禮官、南班宗室及玉牒所官、分官赴景靈宮,迎奉《皇帝中興聖統》赴玉牒殿,同時安奉。



 俟安奉時將至,設香案畢,次御史臺、閣門、太常寺分引文武百官詣玉牒殿下,並北向立班定。禮直官引提領官詣幄前西立,
 俟日官報時及。次玉牒所安奉玉牒、類譜訖。次引提領官復位,北向立定。禮直官揖、躬、拜,提領官拜,在位官皆再拜訖。禮直官引提領官詣香案前,搢笏,三上香,執笏退,復位立定,在位皆再拜訖,退。儀衛、樂人等以次退。自是,凡進書並仿此,惟進太上皇聖政,則有詣德壽宮之儀。



 淳祐五年二月十二日,進孝宗、光宗兩朝御集、《寧宗實錄》及《理宗玉牒日歷》。其日,皇帝御垂拱殿,提舉官、禮儀使、宗室、使相、宰執以下,赴實錄院、右文殿、玉牒所、經
 武閣並行燒香禮畢,奉迎諸書至和寧門,步導至垂拱殿,以俟班齊,各隨腰輿入殿下,東西向立。



 皇帝服靴袍出宮,殿下鳴鞭,禁衛、諸班直、親從等並入內省執骨朵使臣,國史實錄院、日歷所、編修經武要略所、玉牒所點檢文字以下並腰輿下人,並迎駕,自贊常起居。內擎腰輿人不拜,止應喏。



 皇帝即御坐。先知閣門官以下,各班起居如常儀。



 次入內官下殿,各取合進呈書匣升殿,於殿上東壁各置案上,以南為上。知閣門官二員,自御坐
 前導皇帝起詣三朝諸書香案前褥位,東向立。閣門提點奏請上香,三上香訖,又奏請皇帝再拜訖,知閣門官前導皇帝復歸御坐。知閣門官歸東朵殿上侍立,儀鑾司徹香案、拜褥,降東朵殿。



 次舍人請國史實錄院以下提舉官、禮儀使、宰執並進讀官升殿,於御坐東面西立。國史實錄院、國史日歷所、編修經武要略所、玉牒所官,殿下依舊立。



 俟入內官進御案,《孝宗御集》提舉官並進讀御集官稍前立,分進讀御集官於御前過,西壁面東
 立。御集提舉諸司官於《孝宗御集》匣前,搢笏、啟封、開鑰訖,出笏,歸侍立位。御集承受官搢笏,於御集匣內取冊,轉授提舉官搢笏接訖,承受出笏,提舉官奉冊置御案上,出笏。皇帝起前立,提舉諸司官、承受官分東西相向立,並搢笏揭冊訖,各出笏。進讀御集官搢笏稍前,取篦子指讀,



 逐版揭冊指讀,並如上儀。俟進讀畢,皇帝復坐,進讀御集官置篦子於御案上,出笏,卻於御前東壁面西立以俟。提舉官搢笏、收冊,復授承受官搢笏接訖,提
 舉官出笏,稍後立。



 承受官奉冊入匣訖,出笏,提舉諸司官搢笏、鎖匣訖,出笏,歸侍立位。次讀《光宗御集》、《寧宗實錄》、《光宗經武要略》,並同上儀。



 次玉牒提舉官並進讀玉牒官稍前立,分進讀玉牒官於御前過,西壁面東立。玉牒提舉諸司官於玉牒匣前搢笏、啟封鑰訖,出笏,歸侍立位。玉牒承受官搢笏取冊,授提舉官置御案上,進讀亦如前儀,讀畢鎖匣,出笏,歸侍立位。次日歷提舉官並進讀日歷官啟封鑰,進讀亦同。



 俱畢,入內官徹案,



 承受
 官奉冊入匣訖,出笏,提舉諸司官搢笏、鑰匣訖,出笏,歸侍立位。儀鸞司徹卓子,降東朵殿。奉書匣下殿,各置腰輿上。國史實錄院、日歷所、編修經武要略所、玉牒所提舉官,禮儀使,宰執並降東階下殿,東壁面西立。舍人引各官及禮儀使一班當殿面北立定,引各直身出班、斂身稱謝訖,歸位立,揖,躬身贊拜,兩拜訖。贊各祗候直身立宣答,御藥下殿宣答,提舉官、禮儀使並斂身聽宣答訖,



 御藥升殿。揖,躬身贊拜,兩拜訖。贊各祗候直身立,舍
 人引赴東壁面西立。



 次引國史實錄院、日歷所、編修經武要略所、玉牒所官一班首直身出班、斂身稱謝訖,歸位立,揖,躬身贊拜,兩拜訖,贊各祗候直身立。如傳旨謝恩,知閣門官承旨訖,於折檻東面西立,傳與舍人承旨訖,再揖,躬身贊謝恩拜,兩拜訖,贊各祗候直身立。



 不該賜茶官先退。



 次引國史實錄院、日歷所、編修經武要略所、玉牒所提舉諸司官並承受官以下一班當殿面北立定,揖,躬身贊謝恩拜,兩拜訖,贊各祗候直身立,各歸
 位立。



 次引國史實錄院、日歷所、編修經武要略所、玉牒所點檢文字以下一班當殿面北立定,揖,躬身,贊謝恩拜,兩拜訖,贊各祗候直身立,各歸位立。傳旨宣坐賜茶訖,舍人奏閣門無公事,皇帝起還宮,百官導送,奏安兩朝《御集》、《實錄》於天章閣,《經武要略》於經武閣、《玉牒》於玉牒所、《日歷》於秘閣如儀。



 大謝之禮,廢於五季,太宗始命有司草定儀注。其群臣朝謁如元會。酒三行,有司言「請賜王、公以下射」,侍中稱
 制可。皇帝改服武弁,布七埒於殿下,王、公以次射,開樂縣東西廂,設熊虎等候。陳賞物於東階,以賚能者;設豐爵於西階,以罰否者。並圖其冠冕、儀式、表著、埻埒之位以進。帝覽而嘉之,謂宰臣曰:「俟弭兵,當與卿等行之。」



 凡游幸池苑,或命宗室、武臣射,每帝射中的,從官再拜奉觴、貢馬為賀。預射官中者,帝為之解,賜襲衣、金帶、散馬,不解則不賜。苑中皆有射棚、畫暈的。射則用招箭班三十人,服緋紫繡衣、帕首,分立左右,以唱中否。節序賜宴,
 則宗室、禁軍大校、牧伯、諸司使副皆令習射,遂為定制。外國使入朝,亦令帥臣伴,賜射於園苑。



 政和宴射儀:皇帝御射殿,侍宴官公服、系鞋,射官窄衣,奏聖躬萬福,再拜升殿。酒三行,引射官降,皆執弓矢,謝恩再拜,三公以下在右,射官在左,不射者依坐次分立。



 皇帝初射中,舍人贊拜,凡左右祗應臣僚,除內侍外,並階上下再拜。行門、禁衛、諸班、親從、諸司祗應人並自贊再拜。招箭班殿上躬奏訖,跪進碗。射官先傳弓箭與殿侍,側立。內侍接
 碗訖,就拜,起,降階再拜。有司進御茶床,天武引進奉馬列射垛前,員僚奏聖躬萬福,東上閣門官詣御前,躬奏班首姓名以下進酒。班首以下橫行立,贊再拜,班首奉酒進,樂作,飲畢,殿上臣僚再拜。舍人贊各賜酒,群官俱再拜,贊各就坐,群官皆立席後,引進司官臨階,宣進奉出,天武奉馬出,樂合,復贊就坐,飲訖,揖,興,諸司收坐物等。射官左側臨階,取弓箭侍立。皇帝再射中的或雙中,如上儀。



 進酒臨時取旨,得旨進酒,更不進奉中扁碗。及
 解中,更不賀、不進酒。



 臣僚射中,引降階再拜訖,殿下側立。御箭解中,招箭班進碗,如上儀。舍人再引射,中官當殿揖,躬宣「有敕,賜窄衣、金帶」。跪受,箱過再拜,過殿側服所賜訖,再引當殿再拜,更不謝。



 如宣再射,或更賜箭令射,如未退,即就位再拜。如再射中,御箭再解中,賜鞍轡馬如上儀。臣僚射中,御箭不解,引降階再拜,立。招箭班殿上躬奏訖,下殿,舍人宣「有敕,賜銀碗」。跪受執碗並箭,就拜,起,再拜。如合賜散馬,即同宣賜,宣「有敕,賜銀碗,兼
 賜散馬若干匹」。射訖,進御茶床,諸司復陳坐物等,群官各立席後,贊就坐,群官俱坐。酒五行,宣示盞、宣勸如儀,皆作樂。宴畢,內侍舉御茶床,三公以下降階再拜,退。



 乾道二年二月四日,車駕幸玉津園,皇帝射訖,次命皇太子,次慶王,次恭王,次管軍臣僚等射,如是者三。每射四發,帝前後四中的。



 淳熙元年九月,車駕幸玉津園,命從駕文武官行宴射之禮,皇太子、宰執以下,酒三行,樂作。皇帝臨軒,有司進弓矢。皇帝中的,皇太子進酒,率宰執
 以下再拜稱賀。宣皇太子射,射中,賜。宣預射臣僚射,使相鄭藻、起居舍人王卿月、環衛官蕭奪里懶射中,各賜襲衣、金帶。



 鄉飲之禮有三:《周禮》,鄉大夫,三年大比,興賢者、能者,鄉老及鄉大夫帥其吏,與其眾寡,以禮賓之,一也;黨正,國索鬼神而祭祀,則禮屬民而飲酒於序,以正齒位,二也;州長,春秋習射於序,先行鄉飲禮,三也。後世臘蠟百神、春秋習射、序賓飲酒之儀,不行於郡國,唯貢士日設鹿
 鳴宴,猶古者賓興賢能,行鄉飲之遺禮也。然古禮俯僎介,與今之禮不同。器以尊俎,與今之器不同。賓坐於西北,介坐於西南,主人坐東南,僎坐東北,與今之位不同。主人獻賓,賓酢主人,主人酬賓,次主人獻介,介酢主人,次主人獻眾賓,與今之儀不同。今制,州、軍貢士之月,以禮飲酒,且以知州、軍事為主人,學事司所在,以提舉學事為主人,其次本州官入行,上舍生當貢者,與州之群老為眾賓,亦古者序賓、養老之意也。是月也,會凡
 學之士及武士習射,亦古者習射於序之意也。



 唐貞觀所頒禮,惟明州獨存,淳化中會例行之。政和禮局定飲酒祭降之節,與舉酒作樂器用之屬,並參用闢雍宴貢士儀,其有古樂處,令用古樂。既又以河北轉運判官張孝純之言:「《周官》以六藝教士,必射而後行。古者諸侯貢士,天子試諸射宮,請詔諸路州郡,每歲宴貢士於學,因講射禮。」於是禮官參定射儀:鄉飲酒前一日,本州於射亭東西序,量地之宜,設提舉學事諸監司、知州、通判、州
 學教授、應赴鄉飲酒官貢士幕次,本州兵馬教諭備弓矢應用物,設樂。其日初筵,提舉學事、知州軍、通判帥應赴鄉飲酒官貢士詣射亭,執弓矢,揖人射,乘矢若中,則守帖者舉獲唱獲,執算者以算投壺畢,多算勝少算。射畢,贊者贊揖,酬酢如儀畢,揖退飲,如鄉飲酒。



 紹興七年,郡守仇悆置田以供費。十三年,比部郎中林保乞修定鄉飲儀制,遍下郡國,於是國子祭酒高閎草具其儀上之,僎介之位,皆與古制不合,諸儒莫解其指意。



 慶元中,
 朱熹以《儀禮》改定,知學者皆尊用之,主賓、僎介之位,始有定說。其主,則州以守、縣以令,位於東南;賓,以里居年高及致仕者,位俯人巽,則州以倅、縣以丞或簿,位東北;介,以次長,位西南。三賓,以賓之次者;司正,以眾所推服者;相及贊,以士之熟於禮儀者。其日質明,主人率賓以下,先釋菜於先聖先師,退各就次,以俟肅賓。介與眾賓既入,主人序賓祭酒,再拜,詣罍洗洗觶,至酒尊所酌實觶,授執事者,至賓席前跪以獻賓,賓酬主人,主人酬
 介,介酬眾賓,賓主以下各就席坐訖。酒再行,次沃洗,贊者請司正揚觶致辭,司正復位,主人以下復坐。主人興,復至阼俯僎從賓介復至西階下立,三賓至西階立,並南向。主人拜,賓介以下再拜。賓介與眾賓先自西趨出,主人少立,自東出。賓以下立於庠門外之右,東鄉;主人立於門外之左,西鄉。僎從主人再拜,賓介以下皆再拜,退。



\end{pinyinscope}