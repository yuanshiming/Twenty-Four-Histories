\article{列傳第四十八}

\begin{pinyinscope}

 高瓊子繼勛繼宣範廷召葛霸子懷敏



 高瓊,家世燕人。祖霸,父干。五代時,李景據江南,潛結契丹,歲遣單使往復。霸將契丹之命,以乾從行使景。方至江左,諜間北使與中夏構隙,以紓疆場之難,遂殺霸,居
 乾濠州,聲言為汴人所殺。乾在濠州生三子,以江左蹙弱,尋挈族歸中朝,給田亳州之蒙城,因土著焉。



 瓊少勇鷙無賴,為盜,事敗,將磔於市,暑雨創潰,伺守者稍怠,即掣釘而遁。事王審琦,太宗尹京邑,知其材勇,召置帳下。太宗嘗侍宴禁中,甚醉,及退,太祖送至苑門。時瓊與戴興、王超、李斌、桑贊從,瓊左手執靮,右手執鐙,太宗乃能乘馬。太祖顧瓊等壯之,因賜以控鶴官衣帶及器帛,且勖令盡心焉。



 太宗即位,擢御龍直指揮使。從征太原,命
 押弓弩兩班,合圍攻城。及討幽薊,屬車駕倍道還,留瓊與軍中鼓吹殿後,六班扈從不及,惟瓊首率所部見行在,太宗大悅,慰勞之。太平興國四年,遷天武都指揮使、領西州刺史。明年,改為神衛右廂都指揮使、領本州團練使。車駕巡師大名,命瓊與日騎右廂都指揮使朱守節分為京城內巡檢。坐事,出為許州馬步軍都指揮使。



 會有龍騎亡命卒數十人,因知州臧丙出郊,謀劫其導從以叛。瓊聞即白丙,趣還城,因自率從卒數十人,挾弓
 矢單騎追捕,至榆林村,及之。賊入村後舍,登墻以拒。賊首青腳狼者注弩將射瓊,瓊引弓一發斃之,遂悉擒送於州。丙上其事。會將北伐,召歸。授馬步軍都軍頭、領薊州刺史、樓船戰棹都指揮使,步船千艘赴雄州。又城易州。師還,為天武右廂都指揮使、領本州團練使。



 端拱初,遷左廂,改領富州團練使。是秋,出為單州防禦使,改貝州部署。其出守也,與範廷召、王超、孔守正並命焉。數月,廷召等皆復補兵職,瓊頗悒悒。時王承衍鎮貝丘,公主
 每入禁中,頗知上於瓊厚,承衍每寬慰之。二年,召還。故事,廉察以上入朝,始有茶藥之賜,至是特賜瓊焉。三月,遷朔、易帥臣,制授瓊侍衛步軍都指揮使、領歸義軍節度,廷召輩始加觀察使,不得與瓊比。出為並州馬步軍都部署,時潘美亦在太原,舊制,節度使領軍職者居上,瓊以美舊臣,表請居其下,從之。戍兵有以廩食陳腐嘩言者,瓊知之,一日,出巡諸營,士卒方聚食,因取其飯自啖之,謂眾曰:「今邊鄙無警,爾等坐飽甘豐,宜知幸也。」眾
 言遂息。改鎮州都部署。至道中,就改保大軍節度,典軍如故。



 真宗即位,加彰信軍節度,充太宗山陵部署,復為並代都部署。咸平中,契丹犯塞,其母車帳至狼山大夏。上親巡河朔,遣楊允恭馳往,召瓊率所部出土門,與石保吉會鎮、定。既而傅潛以逗留得罪,即召瓊代之。兵罷,復還本任。轉運使言其政績,詔褒之。



 咸平三年,代還,以手創不任持笏,詔執梃入謁,授殿前都指揮使。先是,範廷召、桑贊所將邊兵臨敵退衄,言者請罪之。以問瓊,瓊
 對曰:「兵違將令,於法當誅。然陛下去歲已釋其罪,今復行之,又方屯諸路,非時代易,臣恐眾心疑懼。」乃止。



 景德中,車駕北巡。時前軍已與敵接戰,上欲親臨營壘,或勸南還,瓊曰:「敵師已老,陛下宜親往,以督其成。」上悅,即日進幸澶淵。明年,以罷兵,料簡兵卒諸班直十年者出補軍校,年老者退為本班剩員。瓊進曰:「此非激勸之道,宿衛豈不勞乎?」自是八年者皆得敘補焉。



 馬軍都校葛霸權步軍司,會以疾在告,令瓊兼領二司。瓊從容上言曰:「
 臣衰老,儻又有犬馬之疾,則須一將總此二職。臣事先朝時,侍衛都虞候以上常至十員,職位相亞,易於遷改,且使軍伍熟其名望,邊藩緩急,亦可選用。」上深然之。未幾,以久疾求解兵柄,授檢校太尉、忠武軍節度。三年冬,疾甚,上欲親臨問之,宰相不可,乃止。卒,年七十二,贈侍中。



 瓊不識字,曉達軍政,然頗自任,罕與副將參議。善訓諸子:繼勛、繼宣、繼忠、繼密、繼和、繼隆、繼元。繼勛、繼宣最知名。



 繼勛字紹先,初補右班殿直。儀狀頎偉,太宗見而異之,召問其家世,以瓊子對。擢寄班祗候,累遷內殿崇班。



 咸平初,王均據益州。以崇儀副使為益州兵馬都監、提舉西川諸州軍巡檢公事。招安使雷有終以兵五百授繼勛,守東郭二門,會賊攻彌牟砦,繼勛引兵轉鬥至嘉州,敗之,獲黃傘、金塗鎗以還。有終益以勁兵復進攻二門,克之,乃建幟城上。諸將知城拔,有終乃引軍薄天長門,賊復來拒戰。會日暮,有終欲少休,繼勛曰:「賊窘矣,急擊
 之,無失也。」率十數騎鏖戰,身被數創,血濡甲;馬死,更馬以進。會入內都知秦翰來援,賊退保子城,不敢出。繼勛潛知賊欲夜遁,開圍使得潰去,均卒敗滅。以功遷崇儀使。賊餘黨保山藪中,時出剽劫,乃徙綿漢劍門路都巡檢使。繼勛募惡少年偵賊動靜,窮躡巖穴,掩其不備,悉擒殺之。



 又徙峽路鈐轄,還朝,遷洛苑使、並代州鈐轄。徙屯岢嵐軍。契丹聚兵五萬屯草城川,繼勛登高望之,謂軍使賈宗曰:「彼眾而陣不整,將不才也。我兵雖少,可以
 奇取勝。先伏兵山下,敵見我弱,必急攻我。我誘之南走,爾起乘之,當大潰。」轉戰至寒光嶺,伏發,契丹果敗,相蹂躪死者萬餘人,獲馬、牛、橐駝甚眾。遷弓箭庫使,賜金帶、錦袍,領榮州刺史,徙麟、府州鈐轄。



 時屯兵河外,饋運不屬。繼勛扼兔毛川,援送軍食,師乃濟。徙知環州,又徙瀛州。時歲饑,募富人出粟以給貧者。明年大稔,木生連理者四,郡人上治狀請留。遷內藏庫使,以宮苑使奉使契丹。還,知定州,遷西上閣門使、昭州團練使,徙鄜延路鈐
 轄,坐市馬虧價失官。已而復為西上閣門使、榮州刺史、知冀州、領果州團練使。徙貝州,復知瀛州。



 仁宗即位,改東上閣門使,真授隴州團練使、知雄州。其冬,契丹獵燕薊,候卒報有兵入鈔,邊州皆警。繼勛曰:「契丹歲賴漢金繒,何敢損盟好邪?」居自若,已,乃知渤海人叛契丹,行剽兩界也。擢捧日天武四廂都指揮使、連州防禦使,又知瀛州。歷步軍馬軍殿前都虞候、步軍副都指揮使、邕州觀察使、涇原路副都總管兼知渭州。入宿衛,出為天雄
 軍都總管,願復護邊,既而留不遣。後為真定府定州路都總管,改威武軍節度觀察留後,遂拜保順軍節度使、馬軍副都指揮使。



 恭謝禮成,徙昭信軍節度使,為莊獻明肅太后山陵、莊懿太后園陵都總管,以老病乞骸骨。召見便殿,許一子扶掖,俾勿拜,聽辭管軍。授建雄軍節度使、知滑州。河水暴溢,嚙堤岸,繼勛雖老,躬自督役,露坐河上,暮夜猶不輟,水乃殺怒,滑人德之。卒,年七十八,輟視朝一日,贈太尉。繼勛性謙,有機略,善撫御士卒,臨
 戰輒勝。在蜀有威名,號「神將」。



 子遵甫,官至北作坊副使。嘉祐八年,遵甫女正位皇后,神宗即位,冊皇太后。累贈繼勛太師、尚書令兼中書令,追封康王,謚穆武。熙寧九年,帝詔宰相王珪為神道碑,御篆碑首曰「克勤敏功鐘慶之碑」。遵甫亦贈太師、尚書令兼中書令,追封楚王。



 繼宣字舜舉。幼善騎射,頗工筆札,知讀書。以恩補西頭供奉官、惠民河巡督漕船。會歲饑多盜,兼沿河巡檢捉賊,遷閣門祗候、邠州兵馬都監。曹瑋守邠,數與言兵,薦其
 可用。



 乾興初,以內殿崇班為益州都監。蜀人富侈,元夕大張燈,知府薛奎戒以備盜,繼宣籍惡少年飲犒之,使夜中潛志盜背,明日皆獲。歷磁、相、邢、洺都巡檢使,知安肅軍,徙保州。累遷禮賓使、益州路兵馬鈐轄。還,為西上閣門使、涇原路鈐轄兼安撫使、知渭州,遷四方館使、昭州刺史、知雄州。



 初,元昊反,聲言侵關隴。繼宣請備麟府。未幾,羌兵果入寇河外,陷豐州。擢捧日天武四廂都指揮使、恩州團練使、知並州。俄寇麟府,繼宣帥兵營陵
 井,抵天門關。是夕大雨,及河,師半濟,黑凌暴合,舟不得進,乃具牲酒為文以禱。已而凌解,師濟,進屯府谷,間遣勇士夜亂賊營。又募黥配廂軍,得二千餘人,號清邊軍,命偏將王凱主之。軍次三松嶺,賊數萬眾圍之,清邊軍奮起,斬首千餘級。其相躪藉死者不可勝計。築寧遠砦,相視地脈,鑿石出泉。已而城五砦,遷眉州防禦使,卒。



 範廷召,冀州棗強人。父鐸,為里中惡少年所害。廷召年十八,手刃父仇,剖取其心以祭父墓。弱冠,身長七尺餘,
 有膂力。嘗為盜,以勇壯聞。周廣順初,應募為北面招收指揮使。世宗即位,入補衛士。從征高平,戰疾力,遷殿前指揮使。從征淮南,戰紫金山,流矢中左股。



 宋初,從平李筠、李重進,轉本班都知。又從征太原,再轉散都頭、都虞候、領費州刺史。太平興國中,以日騎軍都指揮使從平太原,徵範陽。秦王廷美嘗遣親吏閻懷忠、趙瓊犒禁軍列校,廷召預焉,坐出為唐州馬步軍都指揮使。



 雍熙三年,議北征,召入為馬步軍都軍頭、領平州刺史、幽州道
 前軍先鋒都指揮使。與賊遇固安南,破其眾三千,斬首千餘級,克固安、新城二縣,乘勝下涿州。廷召復與賊戰,中流矢,血漬甲縷,神色自若,督戰益急,詔褒之。師還,遷日騎右廂都指揮使、領本州圍練使,又遷左廂,移領高州。端拱初,出為齊州防禦使,數月,授捧日天武四廂都指揮使、領澄州防禦使。二年,轉殿前都虞候、領涼州觀察使、鎮州副都部署。大破契丹三萬眾於徐河,斬首數千級。



 淳化二年,為平虜橋砦都部署,歷並代、環慶兩路
 副部署。至道中,遣將從五路討李繼遷,命廷召副李繼隆為環慶靈都部署。廷召出延州路,與賊遇白池,獲米募軍主吃囉等兵器、鎧甲數萬。是役也,諸將失期,獨廷召與王超大小數十戰,屢克捷,上嘉之。俄又為並代兩路都部署。三年,遷侍衛馬軍都指揮使、領河西軍節度,為定州行營都部署。



 咸平二年,契丹入塞,車駕北巡。廷召與戰瀛州西,斬首二萬級,逐北至莫州東三十里,又斬首萬餘,奪其所掠老幼數萬口,契丹遁去。師還,錄功
 加檢校太傅,益賦邑,又改殿前都指揮使。四年正月被疾,車駕臨問,卒,年七十五,贈侍中。



 廷召在軍四十餘年,由顯德以來,凡親征,未嘗不從。善騎射,嘗出獵,有群鳥飛過,廷召發矢,並貫其三,觀者駭異。性惡飛禽,所至處彈射殆絕。尤不喜驢鳴,聞必擊殺之。



 子守均至散員都虞候、演州刺史;守信內殿承制、閣門祗候;守宣內殿崇班;守慶更名珪,後為西京作坊副使、淮南江浙荊湖制置發運副使。



 葛霸,真定人。姿表雄毅,善擊刺騎射。始事太宗於藩邸;踐阼,補殿前指揮使,稍遷本班都知,三遷至散員都虞候。雍熙中,幽州之師失律,大補軍校,以霸為驍騎軍都指揮使、領檀州刺史,戍定州。嘗遇敵唐河,與戰,敗走之,斬獲甚眾。俄召為御前忠佐馬步軍都軍頭。端拱初,出為博州團練使,歷潞、代二州部署。淳化元年,擢殿前都虞候、領潘州觀察使,為高陽關副都部署,進都部署。凡七戰。召還,制授保順軍節度,典軍如故。出為鎮州都部
 署,徙天雄軍。



 咸平三年,車駕勞師於大名,霸與石保吉同來覲。時康保裔沒於河間,即日以霸為貝、冀、高陽關前軍行營都部署。二月,就遷副都指揮使。未幾,改邠寧、涇原、環慶三路都部署。四年,遷侍衛馬軍都指揮使,領感德軍節度。



 景德元年,河決澶州橫□埽,命為修河都部署。未行,屬北邊有警,真宗議親征,以霸為駕前西面邢洺路都部署,又副李繼隆為駕前東面排陣使,駐澶州。明年召還,以功特加封邑。上言朝廷居明德心喪,尚
 遏音樂,請停迎授之制,奏可。是年冬,以霸久典兵,年且老,罷軍職,授昭德軍節度、並代都部署。時廷臣有隸麾下者,頗擾軍民,霸昏耄,為所罔,真宗知之,故有是召。



 四年夏,徙知耀州。霸雖懦,然能謹直自持。會東封,表求扈蹕。既以疾不能從,車駕還次衛南,疾少間,迎謁行在。上嘉其意,勞問久之。未幾卒,年七十五,贈太尉。



 子懷信、懷正、懷敏、懷煦。懷信至如京副使,懷煦內殿承制,懷正博州團練使、知滄、莫二州。



 懷敏以蔭授西頭供奉官,加閣門祗候。歷同提點益州路刑獄、襄鄧都巡檢。使契丹,知隰、莫、保三州,累遷東染院使、康州刺史、知雄州,就遷西上閣門使。上《平燕策》。會歲旱,塘水涸,懷敏慮契丹使至測知其廣深,乃擁界河水注之,塘復如故。召對邊事,復還雄州,改萊州團練使。濁流砦兵叛,殺官吏潰去,懷敏發兵掩襲,盡誅其黨。在雄州五年,徙滄州。



 懷敏為王德用妹婿,德用貶,亦絳知滁州。陜西用兵,起為涇原路馬步軍副總管兼涇原秦
 鳳兩路經略、安撫副使。既入對,以曹瑋嘗所被介冑賜之,令制置鄜延、環慶兩路存廢砦柵。擢龍神衛四廂都指揮、眉州防禦使、本路副都總管、知涇原。遷捧日天武四廂都指揮使、鄜延路副都總管。進殿前都虞候、知延州。範仲淹言其猾懦不知兵,復徙涇原路兼招討、經略、安撫副使。



 慶歷二年,元昊寇鎮戎軍,懷敏出瓦亭砦,督砦主都監許思純、環慶路都監劉賀、天聖砦主張貴,及緣邊都巡檢使向進、劉湛、趙瑜等禦敵。軍次安邊砦,給
 芻秣未絕,懷敏輒離軍,夜至開遠堡北一里而舍。既而自鎮戎軍西南,又先引從騎百餘以前,承受趙正曰:「敵近,不可輕進。」懷敏乃少止。日暮趨養馬城,與知鎮戎軍曹英及涇原路都監李知和王保王文、鎮戎軍都監李嶽、西路都巡檢使趙璘等會兵。聞元昊徙軍新壕外,懷敏議質明襲之,乃命諸命將分四路趣定川砦:劉湛、向進出西水口,涇原路都監趙珣出蓮華堡,曹英、李知和出劉璠堡,懷敏出定西堡。知和與英督軍夜發。翌日,湛、進
 行次趙福堡,遇敵,戰不勝,保向家峽,懷敏使珣、英並鎮戎軍西路巡檢李良臣、孟淵援之。



 俄報敵已拔柵逾邊壕,懷敏入保定川砦,敵毀板橋,斷其歸路,別為二十四道以過軍,環圍之。又絕定川水泉上流,以饑渴其眾。劉賀率蕃兵門於河西,不勝,餘眾潰去。懷敏為中軍屯塞門東偏,英等陣東北隅。敵自褊江三、葉燮會出,四面環之。先以銳兵沖中軍,不動,回擊英軍。會黑風起東北,部伍相失,陣遂擾。士卒攀城堞爭入,英面被流矢,僕壕中,
 懷敏部兵見之亦奔駭。懷敏為眾蹂躪幾死,輿致甕城,久之乃蘇。復選士據門橋,揮手刃以拒入城者。趙珣等以騎軍四合御敵,敵眾稍卻,然大軍無鬥志。珣馳入,勸懷敏還軍中。



 是夕,敵聚火圍城四隅,臨西北呼曰:「爾得非總管廳位圖者邪?爾固能軍,乃入我圍中,今復何往!」夜四鼓,懷敏召曹英、趙珣、李知和、王保、王文、許思純、劉賀、李良臣、趙瑜計議,莫知所出,遂謀結陣走鎮戎軍。雞鳴,懷敏自諭:「親軍左右及在後者皆毋得動,平明,從
 吾往安西堡。以英、珣為先鋒,賀、思純為左右翼,知和為殿,聽中軍鼓乃得行。」至卯,鼓未作,懷敏先上馬,而大軍按堵未動。懷敏周麾者再,將徑去,有執鞚者勸不可,懷敏不得已而還。使參謀郭京等取芻城中,未至,懷敏復上馬,叱執轡者使去,不聽,拔劍且擊之,士遂散。懷敏驅馬東南馳二百里,至長城壕,路已斷,敵周圍之,遂與諸將皆遇害。餘軍九千四百餘人,馬六百餘匹,為敵所斷。其子宗晟與趙正、郭京、承受王昭明等還保定川。



 初,懷
 敏令軍中步兵毋得動,及前陣已去,後軍多不知者,故皆得存。時韓質、郝從政、胡息以兵六千保蓮華堡,劉湛、向進兵一千保向家峽,皆不赴援。於是敵長驅抵渭州,幅員六七百里,焚蕩廬舍,屠掠民畜而去。奏至,帝嗟悼久之,贈懷敏鎮戎軍節度使兼太尉,英、知和、珣、保、文、質、岳、貴、璘、思純、良臣及同時戰沒者,及涇原巡檢楊遵、籠竿城巡檢姚奭、涇原都巡檢司監押董謙、同巡檢唐斌、指使霍達,皆贈官有差。復降向進等官,落郝從政、趙瑜
 職。



 懷敏通時事,善候人情,故多以才薦之。及用為將,而輕率昧於應變,遂至覆軍。帝念之,賜謚忠隱。子宗晟、宗壽、宗禮、宗師,皆遷官。



 論曰:真宗澶淵之役,高瓊之功亦盛矣。範廷召年十八,能手刃父仇;瓊將磔於市,幸以逃免;葛霸善擊刺馬射,給事藩邸:皆非素習韜略者也。及其出身戎行,迭居節鎮,而卓有可觀,由所遇之得其時也。或謂瓊頗自用,謀議不及參佐,而洞曉軍政;霸雖失於巽懦,而能謹直自
 持;廷召性雖癖,在軍中四十年,累從征討,所至有功:皆不害其為驍果也。廷召諸子,珪為最賢,霸子懷敏以戰死,固皆足稱。若繼宣、繼勛之將業,則過其父遠甚,此「克勤敏功鐘慶之碑」所由以立歟!夫以三子之自樹如此,而不得與狄青、郭逵同日而論者,豈非拳勇之有餘,而器識之不足也歟!



\end{pinyinscope}