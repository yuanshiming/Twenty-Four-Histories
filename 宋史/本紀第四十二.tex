\article{本紀第四十二}

\begin{pinyinscope}

 理宗
 二



 端平二年正月丁酉,太陰行犯太白。甲寅,詔議胡瑗、孫明復、邵雍、歐陽修、周敦頤、司馬光、蘇軾、張載、程顥、程頤等十人從祀孔子廟庭,升孔伋十哲。丙辰,詔主管侍衛
 馬軍孟珙黃州駐扎,措置邊防。丁巳,孟珙入見。辛酉,以御前寧淮軍統制、借和州防禦使程芾為大元通好使,從義郎王全副之,尋以武功郎杜顯為添差通好副使。



 二月甲子朔,日當虧不虧。癸酉,歲星守氐。壬午,太白、填星合於胃。



 三月乙未,詔太學生陳均編《宋長編綱目》,進士陳文蔚著《尚書解》,並補迪功郎。丁酉,楊谷、楊石並升太師,尋辭免。乙巳,曾從龍兼同知樞密院事,真德秀參知政事,兼給事中、兼侍讀陳卓同簽書樞密院事。



 夏四
 月甲子,詔:「前四川制置鄭損,城池失守,且盜陜西五路府庫財鉅萬,削官二秩,謫居溫州,簿錄其家。」丁卯,都城火。丁亥,太白晝見。戊子,大閱。有流星大如太白。



 五月乙未,雨雹。軍民交哄,御前諸軍都統制趙勝削三秩,罷,命韓昱代之。丙申,大雨、雹。甲辰,真德秀薨,贈銀青光祿大夫,謚文忠。庚戌,以喬行簡兼參知政事。



 六月壬申,太陰入氐。戊寅,以鄭清之為特進、左丞相兼樞密使,喬行簡金紫光祿大夫、右丞相兼樞密使。己卯,葛洪資政殿大
 學士,予祠祿。庚辰,流星晝隕。祈雨。壬午,以曾從龍知樞密院事兼參知政事,崔與之參知政事,鄭性之同知樞密院事,陳卓簽書樞密院事。賜進士吳叔告以下四百五十四人及第、出身有差。己丑,熒惑入太微垣。庚寅,詔鄭損更削兩秩,竄南劍州。



 秋七月丁酉,有流星大如太白。戊戌,太白經天。辛丑,流星晝隕。丙午,太白入東井。庚申,禮部尚書魏了翁上十事,不報。閏七月戊寅,詔錄開禧蜀難死事之臣,大安知軍楊震仲孫忠孫補下州文
 學;利州路常平乾官劉當可母王氏義不降曦,投江而死,追贈和義郡夫人,當可與升官差除。乙酉,賜少師、特進、銀青光祿大夫趙方謚忠肅。丙戌,故保寧軍節度使、魯國公安丙謚忠定。丁亥,全子才、劉子澄坐唐州之役棄兵宵遁,子才削二秩,謫居衡州,子澄削二秩,謫居瑞州。



 八月癸巳,歲星入氐。乙卯,以太師趙汝愚配享寧宗廟庭,仍圖像於昭勛崇德之閣。丁巳,太白犯太微垣右執法。



 九月癸未,崇國公主薨。



 冬十月辛卯,有流星大如
 太白。己未,填星犯畢,歲星、太白合於心。



 十一月乙丑,以曾從龍為樞密使、督視江淮軍馬,魏了翁同簽書樞密院事、督視京湖軍馬,鄭性之兼權參知政事。戊辰,詔兩督府各給金千兩、銀五萬兩、度牒千、緡錢五百萬為隨軍資。臺臣李鳴復論曾從龍、魏了翁督府事,不允。戊子,安南國貢方物。十二月庚寅,曾從龍六疏乞寢樞密使命,依舊知樞密院事、督視江淮軍馬。詔許辭樞密使。以魏了翁兼督視江淮軍馬。癸巳,四川制置司遣將斬
 叛軍首賊蒲世興於萬州。己亥,填星守天街星。庚子,詔官告院制修武郎以下告身給督視府。太陰入井。壬寅,魏了翁陛辭,詔事乾機速,許便宜行之。吳潛樞密都承旨、督府參謀官,趙善瀚、馬光祖督府參議官。甲辰,曾從龍薨,贈少師。餘嶸同簽書樞密院事。庚戌,故參知政事李壁謚文懿。辛亥,雷。



 三年春正月己未朔,以星行失度,雷發非時,罷天基節宴。詔勸農桑。賜安南國王封爵、襲衣、金帶。丁卯,填星犯
 畢。壬申,大元兵連攻洪山,張順、翁大成等以兵捍禦之。



 二月甲午,詔以大元兵攻江陵,統制李復明奮勇戰沒,其贈三秩,仍官其二子。死傷士卒,趣具姓名來上。壬寅,詔侍從、臺諫、給舍條具邊防事宜。甲辰,起居郎吳泳上疏論淮、蜀、京、襄捍禦十事,不報。詔魏了翁依舊端明殿學士、簽書樞密院事,其速赴闕。詔史嵩之淮西制置使兼副使。辛亥,日暈周匝。甲寅,左曹郎官趙以夫上備邊十策。



 三月乙亥,吳潛赴闕。是月,襄陽北軍主將王旻、李
 伯淵焚城郭倉庫,相繼降北。時城中官民兵四萬七千有奇,其財粟三十萬、軍器二十四庫皆亡,金銀鹽鈔不與焉。南軍主將李虎乘火縱掠,襄陽為空。制置使趙範坐失撫御,致南北軍交爭造亂,詔削官三秩,落龍圖閣學士,姑仍制置職任。階、岷、疊、宕十八族降。有諜者以檄招曹友聞軍降,友聞斬之以聞。



 夏四月丙申,太陰入太微垣。己酉,魏了翁乞歸田里,詔不允,以資政殿學士知潭州。癸丑,詔悔開邊,責己,其京湖、興、沔州軍縣鎮見系
 囚情理輕者釋之。



 五月戊寅,提舉萬壽觀洪咨夔依舊兼侍讀。己卯,有流星出心,大如太白。辛巳,太陰入畢。甲申,趙葵華文閣直學士、淮東安撫制置使兼知揚州。



 六月丁亥,流星夕隕。己亥,洪咨夔卒,詔與執政恩例,贈二秩,謚忠文。癸卯,熒惑、填星合於畢。丙午,熒惑犯填星。庚戌,大雨、雹。



 秋七月丁巳,祈睛。詔權徐州國安用力戰役而歿,已贈順昌軍節度使,仍官其子國興承節郎。庚申,以趙範失襄城,罪重罰輕,詔罷職奉祠。辛酉,太陰入氐。丁
 卯,以鄭性之參知政事,李嗚復簽書樞密院事,戊辰,監察御史杜範、吳昌裔以言事不報,上疏乞罷官,詔改授範太常少卿,昌裔太常卿。庚午,熒惑入井。戊寅,太陰入東井。甲申,雨血。



 八月丙戌,詔趙範更削兩秩、謫居建寧府,李虎削三秩、落刺史,罷御器械,各令任責捍禦自效。癸卯,詔前龍圖閣學士、光祿大夫、贈開府儀同三司傅伯成謚忠簡。



 九月庚申,太白、歲星合於尾。庚午,雷。辛未,祀明堂,大赦。雷雨。乙亥,左丞相兼樞密使鄭清之罷
 為觀文殿大學士、醴泉觀使兼侍讀,右丞相兼樞密使喬行簡罷為觀文殿大學士、醴泉觀使兼侍讀。以崔與之為右丞相兼樞密使。壬午,驍衛大將軍、利州駐扎御前諸軍統制曹友聞與大元兵大戰於大安軍陽平關,兵敗,死之。詔贈龍圖閣學士、大中大夫,謚毅節,立廟曰褒忠,官其二子承務郎。



 冬十月乙酉,詔:「殿前司將胡斌,曩死邵武之寇,贈武節大夫,有司為立後授官,因舊廟賜額。宗室師示賈死尤溪之戰,贈武節郎,官其一子進義校
 尉,立廟林嶺。」甲午,詔:「沿江制置使陳韡應援淮東,授淮西制置使兼沿江制置副使史嵩之應援江陵、峽州江面上流。」壬寅,大元兵破固始縣,淮西將呂文信、杜林率潰兵數萬叛,六安、霍丘皆為群盜所據。丙午,安南國貢方物,詔授金紫光祿大夫、靜海軍節度、觀察等使,賜襲衣、金銀帶。大元太子闊端兵離成都,大元兵破文州,守臣劉銳、通判趙汝向死之。



 十一月戊午,詔嗣秀王師彌授少師。丙寅,以喬行簡為特進、左丞相兼樞密使,封肅
 國公。大元兵圍光州,詔史嵩之援光,趙葵援合肥,陳韡遏和州,為淮西聲援。戊辰,魏了翁依舊資政殿學士、知紹興府、浙東安撫使,吳潛、袁甫、徐清叟赴闕。壬申,詔侍從、兩省、臺諫、卿監、宰掾、樞屬、郎官、鈐轄,各陳防邊方略。甲戌,太陰入太微垣。戊寅,復成都府。十二月戎戌,以吳淵戶部侍郎、淮東總領財賦兼知鎮江府。壬寅,詔改明年為嘉熙元年。癸卯,鄭清之辭免觀文殿大學士、醴泉觀使兼侍讀,詔仍舊觀文殿大學士、提舉洞霄宮。丁未,
 宣繒薨,以定策功,贈太師,謚忠靖。甲寅,池州都統趙邦永以援滁州功,詔邦永轉左武大夫,其餘立功將士具等第、姓名推賞。



 嘉熙元年春正月乙卯,以魏了翁知福州兼福建安撫使。丁巳,詔京西兵馬都監、隨州駐扎程再暹官三轉,帶行閣門宣贊舍人、京西鈐轄兼知隨州,賞其洪山戰功,餘有功將士趣以名上。辛酉,以李□同知樞密院事、四川宣撫使。甲子,詔:「兩淮、荊襄之民,避地江南,沿江州縣,
 間有招集振恤,尚慮恩惠不周,流離失所。江陰、鎮江、建寧、太平、池、江、興國、鄂、岳、江陵境內流民,其計口給米,期十日竣事以聞。」癸酉,熒惑守鬼宿。壬午,流星大如太白。



 二月癸未朔,以鄭性之知樞密院事兼參知政事,鄒應龍端明殿學士、簽書樞密院事,李宗勉同簽書樞密院事。李嗚復罷,以資政殿學士知紹興府。乙酉,葛洪薨。壬寅,雨雹。丙申,詔忠義選鋒張順、屈伸等,以舟師戰公安縣之巴芒有功,各官一轉,餘推恩有差。癸卯,詔以朱熹《
 通鑒綱目》下國子監,並進經筵。己酉,太白晝見,日暈周匝。



 三月癸亥,日生背氣。己巳,詔陣韡、史嵩之、趙葵各官兩轉。乙亥,魏了翁薨,贈少師,賜謚文靖。以孟珙為忠州團練使、知江陵府、京西湖北安撫副使,別之傑寶章閣待制、知太平州。



 夏四月壬午朔,以李□同知樞密院事、四川宣撫使、知成都府。壬辰,弟貴謙保康軍節度使,仍奉朝請,進封天水郡開國侯,加食邑;與芮武康軍節度使、提舉萬壽觀,仍奉朝請,進封開國子。丙申,詔:「兩淮策
 應軍戰宣化,兩軍殺傷相當,陳亡將校李仙、王海、李雄、廖雷各贈武翼大夫,餘贈官有差。」庚子,熒惑犯權星。丙午,詔:「沔州諸鎮將帥,昨以大元兵壓境,皆棄官遁。夔路鈐轄、知恩州田興隆,獨自大安德勝堡至潼川,逆戰數合,雖兵寡不敵,而忠節可尚,特與官一轉。」五月丙辰,袁韶薨。太陰犯熒惑。壬申,京城大火。丙子,熒惑犯將星。



 六月壬辰,詔賞蘄州都統制萬文勝、知州徐□守城之功,將士在行間者,論功補官有差。癸巳,以鄒應龍為資政
 殿學士、知慶元府、沿海制置使。乙未,太白、填星合於井。甲辰,祈雨。丙午,以吳潛為工部侍郎、知慶元府兼沿海制置使。知黃州兼淮西安撫使、本路提刑李壽朋一被命三月,不即便途之官,遂還私舍,詔削三秩,送建昌軍居住。詔建內小學,擇宗子十歲以下資質美者二三人,置師教之。



 秋七月壬子,湖北提舉董槐朝辭,奏楮幣物價重輕之弊。己未,樞密院言:「大元兵自光州、信陽抵合肥,制司參議官李曾伯、廬州守臣趙勝、都統王福戰守,俱
 有勞效。」詔曾伯等十一人各官一轉。辛酉,太陰犯歲星、填星入井,庚午,歲星守建星。壬申,日生背氣。癸酉,太陰入井。



 八月甲申,太師、秦國公汝愚追封福王。乙酉,填星犯井,癸巳,以李鳴復參知政事,李宗勉簽書樞密院事。甲辰,詔:蜀雞冠隘都統王宣戰歿,其總管吳桂棄所守走,又縱部伍剽劫,削三官勒停。



 九月壬子,填星留於井。癸丑,有流星出七公西星,至濁沒。丁巳,雷。



 冬十月戊戌,有流星大如桃。



 十一月戊辰,詔陳韡、史嵩之、趙葵於沿
 江、淮、漢州軍,備舟師戰具,防遏沖要堡隘。辛未,太史言十二月朔日食將既,日與金、木、水、火四星俱纏於斗。詔損膳避朝,庶圖消弭,其令有司檢會故實以聞。十二月戊寅朔,日有食之。



 二年春正月戊申朔,詔令侍從、臺諫、卿監、郎官、帥臣、監司、前宰執侍從舉曉暢兵財各二人,三衙、諸軍統制舉將材二人。己未,詔史嵩之、趙葵應援黃州、安豐,其立功將士等第,亟具名以聞;光州、信陽二城,共圖克復。辛酉,
 詔史嵩之進端明殿學士,視執政恩數;趙葵刑部尚書,制置並如舊;餘玠知招信軍兼淮東制置司參議官,進三秩;孟珙寧遠軍承宣使,依舊帶御器械。史嵩之端明殿學士,依是京湖安撫制置使兼沿江制置副使兼知鄂州,召赴闕。甲子,兩浙轉運判官王野察訪江面還,進對,劾吳潛知平江府不法厲民數事。詔野直華文閣、知建寧府。



 二月甲申,大理少卿朱揚祖充押伴使,借章服、金魚。庚寅,詔吏嵩之以參知政事督視京西荊湖南北
 路、江西軍馬,置司鄂州。癸己,大宗正丞賈似道奏言:「北使將至,地界、名稱、歲例,宜有成說。」又奏:「裕財之道,莫急於去贓吏,藝祖治贓吏,杖殺朝堂,孝宗真決刺面,今日行之,則財自裕。」戊戌,詔:「近覽李□奏,知蜀漸次收復,然創殘之餘,綏撫為急,宜施蕩宥之澤。淮西被兵,恩澤亦如之。其降德音,諭朕軫恤之意。」大元再遣王楫來。辛丑,楫還,以朱揚祖充送伴使。癸卯,以孟珙為京湖安撫制置副使,置司松滋縣。



 三月己丑,命將作監周次說為大
 元通好使。壬子,以李心傳為秘書少監、史館修撰,修高宗、孝宗、光宗、寧宗四朝國史實錄。癸丑,以高定子為中書舍人、京湖江西督視參贊軍事。庚申,詔史嵩之兼督視光、蘄、黃、夔、施州軍馬。戊辰,發行都會子二百萬、並湖廣九百萬,下都督參政行府犒師。乙亥,詔四川被兵州、軍、府、縣、鎮並轉輸勞役之所,見禁囚人情理輕者釋之。詔四川帥臣招集流民復業,給種與牛,優與振贍。



 夏四月癸未,以李□同簽書樞密院事,督視江淮、京湖軍馬。
 己酉,雨土。太陰入太微垣。



 閏月丁未,太陰入井。甲子,有流星大如太白。壬申,賜禮部進士周坦以下四百二十二人及第、出身有差。



 五月辛巳,太白晝見。癸未,以李鳴復知樞密院事,李宗勉參知政事,餘天錫簽書樞密院事。甲申,喬行簡請「以兵事委李鳴復,財用委李宗勉,楮幣委餘天錫,當會議者,臣則參酌行之」。詔允所請。詔嚴州布衣錢時、成忠郎吳如愚以隱居著書,並選為秘閣校勘。丙戌詔崔與之提舉洞霄宮,任便居住,李鳴復復
 參知政事。壬寅,歲星犯壁壘陣。



 六月甲辰朔,流星晝隕。戊申,吳淵知太平州、措置採石江防。以吳潛為淮東總領財賦、知鎮江府。丙寅,李□薨,特贈資政殿大學士。



 秋七月壬午,以霖雨不止,烈風大作,詔避殿、減膳、徹樂,令中外之臣極言闕失。辛卯,有流星大如太白。壬寅,熒惑犯鬼,積尸氣。



 八月辛酉,太白晝見,經天。癸亥,流星晝隕。



 九月壬午,熒惑犯權星。子維生。甲申,封宮人謝氏為永寧郡夫人。乙未,有流星大如太白。



 冬十月庚戌,雷。丁卯,
 吳潛言:「宗子趙時□更集真、滁、豐、濠四郡流民十餘萬,團結十七砦。其強壯二萬可籍為兵,近調五百援合肥,宜補時□更官。又沙上蘆場田可得二十餘萬畝,賣之以贍流民,以佐砦兵。」從之。熒惑入太微垣。戊辰,太白入於氐。己巳,日生黑子。辛未,復光州。



 十一月甲申,子維薨,追封祁王,謚沖昭。十二月丙午,光州守臣董堯臣伏誅,司戶柳臣舉配雷州。乙卯,詔四川諸州縣鹽酒榷額,自明年始更減免三年,其四路合發總所綱運者亦免。戊辰,詔
 諸路和糴給時直,平概量,毋科抑,申嚴收租苛取之禁。己巳,出祠牒、會子共七百萬紙,給四川制司為三年生券。



 三年春正月癸酉,以喬行簡為少傅、平章軍國重事,封益國公;李宗勉為左丞相兼樞密使;史嵩之右丞相兼樞密使,督視兩淮、四川、京湖軍馬;餘天錫參知政事;游似同簽書樞院事。



 二月丙午,詔史嵩之依舊兼都督江西、湖南軍馬。丁卯,又命嵩之都督江淮、京湖、四川軍
 馬。己巳,竄趙邦永,坐救滁不進兵。



 三月辛未朔,以吳潛為敷文閣直學士、沿海制置使兼知慶元府。甲戌,以別之傑權兵部尚書,依舊沿江制置安撫使兼都督行府參贊軍事,李曾伯兼都督行府參議官,孟珙兼都督行府參謀官。流星晝隕。辛卯,雨土。



 夏四月壬寅,祈雨。癸卯,以吳淵權工部尚書、沿江制置副使、知江州。



 五月辛未,熒惑犯太微垣執法星。戊寅,以吳潛為兵部尚書、浙西制置使、知鎮江府。辛卯,喬行簡五疏乞罷機政,詔不允。



 秋七月庚午,以董槐知江州兼都督行府參議官。甲申,以吳淵兼都督行府參贊軍事。



 八月戊戌朔,以浙江潮患,告天地、宗廟、社稷。以游似參知政事,許應龍簽書樞密院事,林略同簽書樞密院事。己亥,熒惑入氐。辛丑,太陰入氐。有流星大如太白。丁亥,熒惑犯房宿。



 九月辛巳,祀明堂,大赦。壬午,淮西敢勇將官陸旺、李威特與官三轉,同出戰二百人官兩轉,以賞廬州磨店北之功,其陣沒者優與撫恤。



 冬十月丁未,故太師魯王謝深甫賜謚
 惠正。己未,出祠牒百給濟處州。秉義郎李良守鄂州長壽縣,沒於戰陣,詔贈官三轉。癸亥,熒惑、太白合於斗。乙丑,虹見。



 十一月丙子,以範鐘簽書樞密院事。十二月己未,觀文殿大學士崔與之薨,贈少師,謚清獻。辛酉,太白晝見。甲子,復夔州,錄荊鄂都統張順、孟璋等將士戰功。



 四年春正月辛未,彗星出營室。庚辰,以星變,下詔罪己。辛巳,有流星大如太白。甲午,彗星犯王良第二星。



 二月丙申朔,日生背氣。戊戌,大赦。辛丑,流星晝隕。白虹貫日。
 丁未,太白晝見。癸丑,以孟珙為四川宣撫使兼知夔州,節制歸、峽、鼎、澧州軍馬。丙辰,白氣亙天。



 三月辛未,詔四川安撫制置副使彭大雅削三秩。彗星消伏。乙酉,流星晝隕。



 夏四月壬寅,前潼川運判吳申進對,因論蜀事,為上言:「鄭損棄邊郡不守,桂如淵啟潰卒為亂,趙彥吶忌忠勇不救,彭大雅險譎變詐,殊費關防。宜進孟珙於夔門。夔事力固乏,東南能助之,則夔足以自立。」又言:「張祥有保全趙彥吶、楊恢兩制置之功,敵人憚其果毅,宜
 見錄用。」上嘉納之。乙巳,詔史嵩之進三秩,依前右丞相兼樞密使,即日徹都督局。



 五月庚午,太陰入太微垣,歲星、太白合於婁。甲戌,太陰入氐。乙亥,子壽國公薨。戊子,命吳潛兼侍讀,李性傳兼侍講。



 六月甲午朔,江、浙、福建大旱,蝗。乙未,祈雨。己亥,太白犯畢。辛丑,追封閬州簽廳陳承己妻彭氏為恭人,賜廟閬州,以強寇入奉國縣市,承己為賊所創,彭罵賊死之。辛亥,追贈儒林郎王鞏為通直郎,官其一子為文學,以丙申蜀破,鞏闔門死於兵。癸
 丑,太白犯天關星。戊午,有流星大如太白。



 秋七月乙丑,詔:「今夏六月恆陽,飛蝗為孽,朕德未修,民瘼尤甚,中外臣僚其直言闕失毋隱。」又詔有司振災恤刑。太白入井。甲戌,太白、熒惑合於井。己丑,熒惑、太白合於鬼。



 八月己酉,熒惑、填星合於柳,太白犯權星大星。癸丑,熒惑犯填星。



 九月乙丑,詔餘玠進三秩,直華文閣、淮東提刑、節制招信軍屯戍軍馬。以玠昨帥舟師渡淮入河抵汴,所向有功,全師而還。至是,論功定賞,是役將士,趣以名上所
 司議推恩。



 冬十月癸巳,詔改明年為淳祐元年。丁巳,命餘玠兼節制應天府、泗、宿、永、海、邳、徐、漣水屯戍軍馬。



 十一月甲子,熒惑入太微垣。己巳,熒惑犯太微垣左執法星。癸酉,詔武功大夫、荊鄂都統制張順以私錢招襄、漢潰卒,創忠義、虎翼兩軍,及援安慶、池州有功,特與官兩轉。丙子,與芮妻錢氏封安康郡夫人。辛巳,熒惑犯太微上相垣。十二月甲辰,奉國軍節度使、提舉萬壽觀多謨薨。丙辰,地震。己未,詔求直言。閏十二月丙寅,李宗勉薨,
 贈少師,賜謚文清。以游似知樞密院事兼參知政事,範鐘參知政事,徐榮叟簽書樞密院事。庚午,詔系囚情理輕者釋之。乙亥,詔民間賦輸仍用錢會中半,其會半以十八界直納,半以十七界紐納。戊寅,以吳潛為福建安撫使,史宅之為浙東安撫使。



 淳祐元年春正月庚寅朔,詔舉文武才。庚子,雷。甲辰,詔:「朕惟孔子之道,自孟軻後不得其傳,至我朝周惇頤、張載、程顥、程頤,真見實踐,深探聖域,千載絕學,始有指歸。
 中興以來,又得朱熹精思明辨,表裏渾融,使《大學》、《論》、《孟》、《中庸》之書,本末洞徹,孔子之道,益以大明於世。朕每觀五臣論著,啟沃良多,今視學有日,其令學官列諸從祀,以示崇獎之意。」尋以王安石謂「天命不足畏,祖宗不足法,人言不足恤」,為萬世罪人,豈宜從祀孔子廟庭,黜之。丙午,封周惇頤為汝南伯,張載郿伯,程顥河南伯,程頤伊陽伯。丁未,太陰入氐。戊申,幸太學謁孔子,遂御崇化堂,命祭酒曹觱講《禮記·大學》篇,監學官各進一秩,諸生
 推恩錫帛有差。制《道統十三贊》,就賜國子監宣示諸生。



 二月戊寅,日生暈。壬午,喬行簡薨,謚文惠。



 夏四月丁丑,詔以與芮為開府儀同三司、萬壽觀使、嗣榮王,貴謙開府儀同三司、嗣沂王。辛巳,以賈似道為太府少卿、湖廣總領財賦。



 五月庚寅,以少師、保寧軍節度使、判大宗正事、嗣秀王師彌為太子少保,奉國軍節度使充萬壽觀使師貢為少師。己亥,詔沿江淮西制置使別之傑任責邊防。戊申,賜禮部進士徐儼夫以下三百六十七人及
 第、出身有差。



 六月庚申,太白晝見。螟。癸酉,有流星大如太白。己卯,流星晝隕。丙戌,熒惑入氐。



 秋七月壬辰,祈雨。



 八月辛巳,楊石薨,贈太師。



 冬十月庚辰,太白入氐。



 十一月戊戌,太白晝見。己亥,淮東提刑餘玠以舟師解安豐之圍。己巳,太白經天,晝見。十二月丁卯,餘天錫薨,贈太師,賜謚忠惠。丁丑,侍御史金淵言:彭大雅貪黷殘忍,蜀人銜怨,罪重罰輕,乞更竄責。詔除名、贛州居住。



 二年春正月甲申朔,詔作新吏治。戊戌,右丞相史嵩之
 等進《玉牒》及《中興四朝國史》、《孝宗經武要略》、《寧宗玉牒》《日歷》《會要實錄》。



 二月甲戌,以游似知紹興府、浙東安撫使,請祠祿,詔提舉洞霄宮。範鐘知樞密院事兼參知政事,徐榮叟參知政事,趙葵賜進士出身、同知樞密院事,別之傑簽書樞密院事。



 三月戊子,詔和州、無為軍、安慶府,並聽沿江制置司節制。詔今後州縣官有罪,諸帥司毋輒加杖責。



 夏四月甲寅,白氣亙天。壬申,雨雹。



 五月己亥,淮東制置副使餘玠進對。戊申,臺臣言知建寧府吳
 潛有三罪。詔奪職,罷新任。己酉,以趙葵為湖南安撫使、知潭州。



 六月壬子朔,徐榮叟乞歸田里,從之。丁巳,詔以餘玠為四川宣諭使,事乾機速,許同制臣共議措置,先行後奏,仍給金字符、黃榜各十,以備招撫。丙寅,以別之傑同知樞密院事兼權參知政事,高定子簽書樞密院事,杜範同簽書樞密院事。是月盛夏積雨,浙右大水。丁丑。歲星犯井。



 秋七月辛巳朔,常、潤、建康大水,兩淮尤甚。



 八月丁卯,詔淮東先鋒馬軍鄧淳、李海等揚州撻扒店
 之戰,宣勞居多,各官兩轉,餘推恩有差。



 九月庚辰朔,日有食之。己丑,雷。辛卯,祀明堂,大赦。癸巳,詔:「淮東忠勇軍統領王溫等二十四人戰天長縣東,眾寡不敵,皆沒於陣,贈溫武翼大夫、吉州刺史,其子興國補保義郎,更官其一子承信郎,厚賜其家。餘人恤典有差。」冬十月甲寅,史嵩之進封永國公。乙丑,大元兵大入通州。



 十一月辛卯,詔諭兩淮節制李曾伯,毋以通州被兵之故,不安厥職,其督勵諸將,勉圖後功。己亥,日南至,雷電交作,詔避
 殿減膳,求直言。癸卯,詔決中外系囚。十二月己未,詔:「通州守臣杜霆,兵至棄城弗守,載其私帑渡江以遁,遂致民被屠戮,雖已奪三秩,厥罰猶輕。其追毀出身以來文字,竄南雄州。」壬戌,太白晝見。癸亥,大元兵連攻敘州,帳前都統楊大全等水陸並進,自卯至午,戰十數合,歿於行伍。詔贈武節大夫、眉州防禦使,官其二子承節郎。丙寅,以孟珙為檢校少保,依舊寧武軍節度使、京湖安撫制置大使、夔路策應大使,餘玠權資政殿學士、湖南安
 撫大使兼知潭州,趙葵資政殿大學士、福建安撫使、知福州。



 三年春正月戊寅朔,以高定子兼參知政事。庚辰,熒惑入氐。乙未,以李曾伯為華文閣待制,依舊淮東西制置使、知揚州;杜杲敷文閣學士,依舊沿江制置使、知建康府;董槐秘閣修撰,依舊沿江制置副使、知江州、主管江西安撫司事。辛丑,詔安南國王陳日煚元賜功臣號,特增「守義」二字。



 二月乙丑,以呂文德為福州觀察使、侍衛
 馬軍副都指揮使,總統兩淮出戰軍馬,捍禦邊陲。庚午,以郢州推官黃從龍死節,詔贈通直郎,一子補下州文學。



 三月丁丑朔,日有食之。



 夏四月癸丑,左武衛中郎將、濠州措置捍禦王烈,閣門宣贊、淮西路鈐王傑,閣門祗候、江東路鈐李季實往馬帥王鑒軍前議事,遇大元兵戰死,贈官,仍各官其二子。乙卯,嘉定守臣程立之固守,詔官一轉。丙辰,安豐軍統領陳友直以王家堈戰功,與官兩轉,壬申,布衣王與之進所著《周禮訂議》,補下州文
 學。



 五月庚子,詔施州創築郡城及關隘六十餘所,本州將士及忠州戍卒執役三年者,各補轉一官。



 六月甲戌,有流星大如太白,出於氐。



 秋七月丁亥,詔海州屯駐借補保義郎申政,密州之役先登陷陣,後以戰沒,特贈保義郎,官其子進勇副尉。太白入井。壬辰,四川制司言:大元兵破大安軍,忠義副總管楊世威堅守魚孔隘,孤壘不降,有特立之操,可任責邊防。詔以世威就知大安軍。甲午,日生格氣。己亥,太白經天,晝見。



 八月乙卯,流星晝
 隕。癸亥,詔福州延祥、荻蘆兩砦並置武濟水軍,摘本州廂禁習水者充,千五百人為額。



 閏月丁丑,四川總領餘玠言,知巴州向牷、鈐轄譚淵白土坪等戰有功。詔佺等十八人各官三轉,餘轉官有差。其中創人各給緡錢百,陣沒者趣上姓名,贈恤其家。太白犯權星。壬寅,太白、填星合於翼。



 九月壬申,詔蠲高郵民耕荒田租。



 冬十月丙戌,太白入於氐。



 十二月己丑,史嵩之五請祠,不允。



\end{pinyinscope}