\article{志第一百二十六 食貨上一(農田)}

\begin{pinyinscope}

 昔武王克商,訪箕子以治道,箕子為之陳《洪範》九疇,五行五事之次,即曰「農用八政」,八政之目,即以食貨為先。五行,天道也;五事,人道也。天人之道治,而國家之政興
 焉。是故食貨而下,五卿之職備舉於是矣:宗伯掌邦禮,祀必有食貨而後儀物備,賓必有食貨而後委積豐;司空掌邦土,民必有食貨而後可奠於厥居;司徒掌邦教,民必有食貨而後可興於禮義;司寇掌邦禁,民必有食貨而後可遠於刑罰;司馬掌邦政,兵必有食貨而後可用於征戍。其曰「農用八政」,農,食貨之本也。唐杜祐作《通典》,首食貨而先田制,其能推本《洪範》八政之意歟。



 宋承唐、五季之後,太祖興,削平諸國,除藩鎮留州之法,而粟
 帛錢幣咸聚王畿;嚴守令勸農之條,而稻、粱、桑、枲務盡地力。至於太宗,國用殷實,輕賦薄斂之制,日與群臣講求而行之。傳至真宗,內則升中告成之事舉,外則和戎安邊之事滋,由是食貨之議,日盛一日。仁宗之世,契丹增幣,夏國增賜,養兵兩陲,費累百萬;然帝性恭儉寡欲,故取民之制,不至掊克。神宗欲伸中國之威,革前代之弊,王安石之流進售其強兵富國之術,而青苗、保甲之令行,民始罹其害矣。哲宗元祐更化,斯民稍望休息;紹
 聖而後,章惇倡紹述之謀,秕政復作。徽宗既立,蔡京為豐亨豫大之言,苛征暴斂,以濟多欲,自速禍敗。高宗南渡,雖失舊物之半,猶席東南地產之饒,足以裕國。然百五十年之間,公私粗給而已。



 考其祖宗立國初意,以忠厚仁恕為基,向使究其所為,勉而進於王道,亦孰能御之哉?然終宋之世,享國不為不長,其租稅征榷,規撫節目,煩簡疏密,無以大異於前世,何哉?內則牽於繁文,外則撓於強敵,供億既多,調度不繼,勢不但已,徵求於民;
 謀國者處乎其間,又多伐異而黨同,易動而輕變。殊不知大國之制用,如巨商之理財,不求近效而貴遠利。宋臣於一事之行,初議不審,行之未幾,既區區然較其失得,尋議廢格。後之所議未有以愈於前,其後數人者,又復訾之如前。使上之為君者莫之適從,下之為民者無自信守,因革紛紜,非是貿亂,而事弊日益以甚矣。世謂儒者論議多於事功,若宋人之言食貸,大率然也。又謂漢文、景之殷富,得諸黃、老之清靜,為黃、老之學者,大忌
 於紛更,宋法果能然乎?時有古今,世有升降,天地生財,其數有限,國家用財,其端無窮,歸於一是,則「生之者眾,食之者寡,為之者疾,用之者舒」之外,無他技也。



 宋舊史志食貨之法,或驟試而輒已,或亟言而未行。仍之則徒重篇帙,約之則不見其始末,姑去其泰甚,而存其可為鑒者焉。篇次離為上下:其一曰農田,二曰方田,三曰賦稅,四曰布帛,五曰和糴,六曰漕運,七曰屯田,八曰常平義倉,九曰課役,十曰振恤。或出或入,動關民生;國以民
 為本,故列之上篇焉。其一曰會計,二曰銅鐵錢,三曰會子,四曰鹽,五曰茶,六曰酒,七曰坑冶,八曰礬,九曰商稅,十曰市易,十一曰均輸,十二曰互市舶法。或損或益,有系國體;國不以利為利,故列之下篇焉。各疏其事,二十有二目,通為十有四卷雲。



 農田之制自五代以兵戰為務,條章多闕,周世宗始遣使均括諸州民田。太祖即位,循用其法,建隆以來,命官分詣諸道均田,苛暴失實者輒譴黜。申明周顯德三年之令,課民種樹,定民籍為
 五等,第一等種雜樹百,每等減二十為差,桑棗半之;男女十歲以上種韭一畦,闊一步,長十步;乏井者,鄰伍為鑿之;令、佐春秋巡視,書其數,秩滿,第其課為殿最。又詔所在長吏諭民,有能廣植桑棗、墾闢荒田者,止輸舊租;縣令、佐能招徠勸課,致戶口增羨、野無曠土者,議賞。諸州各隨風土所宜,量地廣狹,土壤瘠埆不宜種藝者,不須責課。遇豐歲,則諭民謹蓋歲,節費用,以備不虞。民伐桑棗為薪者罪之:剝桑三工以上,為首者死,從者流三
 千里;不滿三工者減死配役,從者徒三年。



 太宗太平興國中,兩京、諸路許民共推練土地之宜、明樹藝之法者一人,縣補為農師,令相視田畝肥瘠及五種所宜,某家有種,某戶有丁男,某人有耕牛;即同鄉三老、里胥召集餘夫,分畫曠土,勸令種蒔,候歲熟共取其利。為農師者蠲稅免役。民有飲博怠於農務者,農師謹察之,白州縣論罪,以警游惰。所墾田即為永業,官不取其租。其後以煩擾罷。初,農時,太宗嘗令取畿內青苗觀之,聽政之次,
 出示近臣。是歲,畿內菽粟苗皆長數尺。帝顧謂左右曰:「朕每念耕稼之勤,茍非兵食所資,固當盡復其租稅。」



 端拱初,親耕籍田,以勸農事。然畿甸民苦稅重,兄弟既壯乃析居,其田畝聚稅於一家,即棄去;縣歲按所棄地除其租,已而匿他舍,冒名佃作。帝聞而思革其弊,會知封丘縣竇玭言之,乃詔賜緋魚,絹百匹;擢太子中允,知開封府司錄事,俾按察京畿諸縣田租。玭專務苛刻以求課最,民實逃亡者,亦搜索於鄰里親戚之家,益造新籍,
 甚為勞擾,數月罷之。時州縣之吏多非其人,土地之利不盡出,租稅減耗,賦役不均,上下相蒙,積習成敝。乃詔:「諸知州、通判具如何均平賦稅,招輯流亡,惠恤孤貧,窒塞奸幸,凡民間未便事,限一月附疾置以聞。」而比年多稼不登,富者操奇贏之資,貧者取倍稱之息,一或小稔,富家責償愈急,稅調未畢,資儲罄然。遂令州縣戒里胥、鄉老察視,有取富民穀麥貲財,出息不得逾倍,未輸稅毋得先償私逋,違者罪之。



 言者謂江北之民雜植諸谷,
 江南專種粳稻,雖土風各有所宜,至於參植以防水旱,亦古之制。於是詔江南、兩浙、荊湖、嶺南、福建諸州長吏,勸民益種諸谷,民乏粟、麥、黍、豆種者,於淮北州郡給之;江北諸州,亦令就水廣種粳稻,並免其租。淳化五年,宋、亳數州牛疫,死者過半,官借錢令就江、淮市牛。未至,屬時雨沾足,帝慮其耕稼失時,太子中允武允成獻踏犁,運以人力,即分命秘書丞、直史館陳堯叟等即其州依式制造給民。



 凡州縣曠土,許民請佃為永業,蠲三歲租,
 三歲外,輸三分之一。官吏勸民墾田,悉書於印紙,以俟旌賞。至道二年,太常博士、直史館陳靖上言:



 「先王之欲厚生民,莫先於積穀而務農,鹽鐵榷酤斯為末矣。按天下土田,除江淮、湖湘、兩浙、隴蜀、河東諸路地裡□遠,雖加勸督,未遽獲利。今京畿周環二十三州,幅員數千里,地之墾者十才二三,稅之入者又十無五、六。復有匿里舍而稱逃亡,棄耕農而事游惰,賦額歲減,國用不充。



 詔書累下,許民復業,蠲其租調,寬以歲時。然鄉縣擾之,每
 一戶歸業,則刺報所由。朝耕尺寸之田,暮入差徭之籍,追胥責問,繼踵而來,雖蒙蠲其常租,實無補於損瘠。況民之流徙,始由貧困,或避私債,或逃公稅。亦既亡遁,則鄉里檢其資財,至於室廬、什器、桑棗、材木,咸計其直,或鄉官用以輸稅,或債主取以償逋;生計蕩然,還無所詣,以茲浮蕩,絕意歸耕。



 如授以閑曠之田,廣募游惰,誘之耕墾,未計賦租,許令別置版圖,便宜從事;酌民力豐寡、農畝肥磽,均配督課,令其不倦。其逃民歸業,丁口授田,
 煩碎之事,並取大司農裁決。耕桑之外,令益樹雜木蔬果,孳畜羊犬雞豚。給授桑土,潛擬井田,營造室居,使立保伍;養生送死之具,慶吊問遺之資,並立條制。候至三五年間,生計成立,即計戶定徵,量田輸稅。若民力不足,官借糴錢,或以市餱糧,或以營耕具。凡此給受,委於司農,比及秋成,乃令償直,依時價折納,以其成數關白戶部。」



 帝覽之喜,令靖條奏以聞。



 靖又言:「逃民復業及浮客請佃者,委農官勘驗以給授田土,收附版籍,州縣未得
 議其差役;乏糧種、耕牛者,令司農以官錢給借。其田制為三品:以膏沃而無水旱之患者為上品;雖沃壤而有水旱之患、埆瘠而無水旱之慮者為中品;既埆瘠復患於水旱者為下品。上田人授百畝,中田百五十畝,下田二百畝,並五年後收其租,亦只計百畝,十收其三。一家有三丁者,請加受田,如丁數五丁者從三丁之制,七丁者給五丁,十丁給七丁;至二十、三十丁者,以十丁為限。若寬鄉田多,即委農官裁度以賦之。其室廬、蔬韭及桑
 棗、榆柳種藝之地,每戶十丁者給百五十畝,七丁者百畝,五丁者七十畝,三丁者五十畝,不及三丁者三十畝。除桑功五年後計其租,餘悉蠲其課。」



 宰相呂端謂靖所立田制,多改舊法,又大費資用,以其狀付有司。詔鹽鐵使陳恕等共議,請如靖奏。乃以靖為京西勸農使,按行陳、許、蔡、穎、襄、鄧、唐、汝等州,勸民墾田,以大理寺丞皇甫選、光祿寺丞何亮副之。選、亮上言功難成,願罷其事。帝志在勉農,猶詔靖經度。未幾,三司以費官錢數多,萬一
 水旱,恐致散失,事遂寢。



 真宗景德初,詔諸州不堪牧馬閑田,依職田例招主客戶多方種蒔,以沃瘠分三等輸課。河朔戎寇之後,耕具頗闕,牛多瘠死。二年,內出踏犁式,詔河北轉運使詢於民間,如可用,則官造給之;且令有司議市牛送河北。又以兵罷,民始務農創什器,遂權除生熟鐵度河之禁。是歲,命權三司使丁謂取戶稅條敕及臣民所陳田農利害,與鹽鐵判官張若谷、戶部判官王曾等參詳刪定,成《景德農田敕》五卷,三年正月上
 之。謂等又取唐開元中宇文融請置勸農判田,檢戶口、田土偽濫;且慮別置官煩擾,而諸州長吏除當勸農,乃請少卿、監為刺史、合門使以上知州者,並兼管內勸農事,及通判並兼勸農事,諸路轉運使、副兼本路勸農使。詔可。



 大中祥符四年,詔曰:「火田之禁,著在《禮經》,山林之間,合順時令。其或昆蟲未蟄,草木猶蕃,輒縱燎原,則傷生類。諸州縣人畬田,並如鄉土舊例,自餘焚燒野草,須十月後方得縱火。其行路野宿人,所在檢察,毋使延燔。」
 帝以江、淮、兩浙稍旱即水田不登,遣使就福建取占城稻三萬斛,分給三路為種,擇民田高仰者蒔之,蓋早稻也。內出種法,命轉運使揭榜示民。後又種於玉宸殿,帝與近臣同觀;畢刈,又遣內侍持於朝堂示百官。稻比中國者穗長而無芒,粒差小,不擇地而生。六年,免諸路農器之稅。明年,諸州牛疫,又詔民買賣耕牛勿算;繼令群牧司選醫牛古方,頒之天下。



 天禧初,詔諸路自今候登熟方奏豐稔,或已奏豐稔而非時災沴者,即須上聞,違
 者重置其罪。先是,民訴水旱者,夏以四月,秋以七月,荊湖、淮南、江浙、川峽、廣南水田不得過期,過期者吏勿受;令、佐受訴,即分行檢視,白州遣官覆檢,三司定分數蠲稅;亦有朝旨特增免數及應輸者許其倚格,京畿則特遣官覆檢。太祖時,亦或遣官往外州檢視,不為常制;傷甚,有免覆檢者。至是,又以覆檢煩擾,止遣官就田所閱視,即定蠲數。時久罷畋游,令開封府諭民,京城四面禁圍草城,許其耕牧。三年,詔民有孝弟力田、儲蓄歲計者,
 長吏倍存恤之。



 初,朝議置勸農之名,然無職局。四年,始詔諸路提點刑獄朝臣為勸農使、使臣為副使,所至,取民籍視其差等,不如式者懲革之;勸恤農民,以時耕墾,招集逃散,檢括陷稅,凡農田事悉領焉。置局案,鑄印給之。凡奏舉親民之官,悉令條析勸農之績,以為殿最黜陟。



 自景德以來,四方無事,百姓康樂,戶口蕃庶,田野日闢。仁宗繼之,益務約己愛人。即位之初,下詔曰:「今宿麥既登,秋種向茂,其令州縣諭民,務謹蓋藏,無或妄費。」上
 書者言賦役未均,田制不立,因詔限田:公卿以下毋過三十頃,牙前將吏應復役者毋過十五頃,止一州之內,過是者論如違制律,以田賞告者。既而三司言:限田一州,而卜葬者牽於陰陽之說,至不敢舉事。又聽數外置墓田五頃。而任事者終以限田不便,未幾即廢。



 時又禁近臣置別業京師及寺觀毋得市田。初,真宗崩,內遣中人持金賜玉泉山僧寺市田,言為先帝植福,後毋以為例。繇是寺觀稍益市田。明道二年,殿中侍御史段少連
 言:「頃歲中人至漣水軍,稱詔市民田給僧寺,非舊制。」詔還民田,收其直入官。後承平浸久,勢官富姓,占田無限,兼並冒偽,習以成俗,重禁莫能止焉。



 帝敦本務農,屢詔勸劭,觀稼於郊,歲一再出;又躬耕籍田,以先天下。景祐初,患百姓多去農為兵,詔大臣條上兵農得失,議更其法。遣尚書職方員外郎沈厚載出懷、衛、磁、相、邢、洺、鎮、趙等州,教民種水田。京東轉運司亦言:「濟、兗間多閑田,而青州兵馬都監郝仁禹知田事,請命規度水利,募民耕
 墾。」從之。是秋,詔曰:「仍歲饑歉,民多失職。今秋稼甫登,方事斂獲,州縣毋或追擾,以妨農時。刑獄須證逮者速決之。」



 帝每以水旱為憂,寶元初,詔諸州旬上雨雪,著為令。慶歷三年,詔民犯法可矜者別為贖令,鄉民以穀麥,市人以錢帛。謂民重穀帛,免刑罰,則農桑自勸,然卒不果行。參知政事範仲淹言:「古者三公兼六卿之職,唐命相判尚書六曹,或兼諸道鹽鐵、轉運使。請於職事中擇其要者,以輔臣兼領。」於是以賈昌朝領農田,未及施為而
 仲淹罷,事遂止。皇祐中,於苑中作寶岐殿,每歲召輔臣觀刈穀麥,自是罕復出郊矣。



 帝聞天下廢田尚多,民罕土著,或棄田流徙為閑民。天聖初,詔民流積十年者,其田聽人耕,三年而後收,減舊額之半;後又詔流民能自復者,賦亦如之。既而又與流民限,百日復業,蠲賦役,五年減舊賦十之八;期盡不至,聽他人得耕。至是,每下赦令,輒以招輯流亡、募人耕墾為言。民被災而流者,又復其蠲復,緩其期招之。詔諸州長吏、令、佐能勸民修陂池、
 溝洫之久廢者,及墾闢荒田、增稅二十萬已上,議賞;監司能督責部吏經畫,賞亦如之。



 久之,天下生齒益蕃,闢田益廣。獨京西唐、鄧間尚多曠土,入草莽者十八、九,或請徙戶實之,或議置屯田,或欲遂廢唐州為縣。嘉祐中,唐守趙尚寬言土曠可闢,民希可招,而州不可廢。得漢邵信臣故陂渠遺跡而修復之,假牛犁、種食以誘耕者,勸課勞來。歲餘,流民自歸及淮南、湖北之民至者二十餘戶;引水溉田幾數萬頃,變磽瘠為膏腴。監司上其狀,
 三司使包拯亦以為言,遂留再任。治平中,歲滿當去。英宗嘉其勤,且倚以興輯,特進一官,賜錢二十萬,復留再任。時患守令數易,詔察其有實課者增秩再任,而尚寬應詔為天下倡。後太守高賦繼之,亦以能勸課被獎,留再任。



 天下墾田:景德中,丁謂著《會計錄》云,總得一百八十六萬餘頃。以是歲七百二十二萬餘戶計之,是四戶耕田一頃,繇是而知天下隱田多矣。又川峽、廣南之田,頃畝不備,第以五賦約之。至天聖中,國史則云:開寶末,
 墾田二百九十五萬二千三百二十頃六十畝;至道二年,三百一十二萬五千二百五十一頃二十五畝;天禧五年,五百二十四萬七千五百八十四頃三十二畝。而開寶之數乃倍於景德,則謂之所錄,固未得其實。皇祐、治平,三司皆有《會計錄》,而皇祐中墾田二百二十八萬餘頃,治平中四百四十萬餘頃,其間相去不及二十年,而墾田之數增倍。以治平數視天禧則猶不及,而敘《治平錄》者以謂此特計其賦租以知頃畝之數,而賦租所
 不如者十居其七。率而計之,則天下墾田無慮三千餘萬頃。是時,累朝相承,重於擾民,未嘗窮按,故莫得其實,而廢田見於籍者猶四十八萬頃。



 治平四年,詔曰:「歲比不登,今春時雨,農民桑蠶、穀麥,眾作勤勞,一歲之功,並在此時。其委安撫、轉運司敕戒州縣吏,省事息民,無奪其時。」「諸路逃田三十年者除其稅十四,四十年以上十五,五十年以上六分,百年以上七分;佃及十年輸五分,二十年輸七分,著為令。」



 神宗熙寧元年,襄州宜城令朱
 紘復修水渠,溉田六千頃,詔遷一官。權京西轉運使謝景溫言:「在法,請田戶五年內科役皆免。貶汝州四縣客戶,不一二年便為舊戶糾抉,與之同役,因此即又逃竄,田土荒萊。欲乞置墾田務,差官專領,籍四縣荒田,召人請射。更不以其人隸屬諸縣版籍,須五年乃撥附,則五年內自無差科。如招及千戶以上者,優獎。」詔不置務,餘從所請。



 明年,分遣諸路常平官,使專領農田水利。吏民能知土地種植之法,陂塘、圩堤、堤堰、溝洫利害者,皆得
 自言;行之有效,隨功利大小酬賞。民占荒逃田若歸業者,責相保任,逃稅者保任為輸之。已行新法縣分,田土頃畝、川港陂塘之類,令、佐受代,具墾闢開修之數授諸代者,令照籍有實乃代。



 中書議勸民栽桑。帝曰:「農桑,衣食之本。民不敢自力者,正以州縣約以為貲,升其戶等耳。宜申條禁。」於是司農寺諸立法,先行之開封,視可行,頒於天下。民種桑柘毋得增賦。安肅廣信順安軍、保州,令民即其地植桑榆或所宜木,因可限閡戎馬。官計其
 活茂多寡,得差減在戶租數;活不及數者罰,責之補種。



 興修水利田,起熙寧三年至九年,府界及諸路凡一萬七百九十三處,為田三十六萬一千一百七十八頃有奇。神宗元豐元年,詔開廢田,水利,民力不能給役者,貸以常平錢穀,京西南路流民買耕牛者免征。五年,都水使者範三淵奏:「自大名抵乾寧,跨十五州,河徙地凡七千頃,乞募人耕種。」從之。



 哲宗即位,宣仁太后臨朝,首起司馬光為門下侍郎,委之以政。詔天下臣民皆得以封
 事言民間疾苦。光抗疏曰:「四民之中,惟農最苦,寒耕熱耘,沾體塗足,戴日而作,戴星而息;蠶婦治繭、績麻、紡緯,縷縷而積之,寸寸而成之,其勤極矣。而又水旱、霜雹、蝗蜮間為之災,幸而收成,公私之債,交爭互奪。穀未離場,帛未下機,已非己有,所食者糠籺而不足,所衣者綈褐而不完。直以世服田畝,不知舍此之外有何可生之路耳。而況聚斂之臣,於租稅之外,巧取百端,以邀功賞。青苗則強散重斂,給陳納新;免役則刻剝窮民,收養浮食;
 保甲則勞於非業之作;保馬則困於無益之費,可不念哉!今者浚發德音,使畎畝之民得上封事。雖其言辭鄙雜,皆身受實患,直貢其誠,不可忽也。」



 初,熙寧六年,立法勸民栽桑,有不趨令,則仿屋粟、里布為之罰。然長民之吏不能究宣德意,民以為病。至是,楚丘民胡昌等言其不便,詔罷之,且蠲所負罰金。興平縣抑民田為牧地,民亦自言,詔悉還之。元祐四年,詔:「瀕河州縣,積水冒田。在任官能為民經畫疏導溝畎,退出良田自百頃至千頃,
 第賞。



 崇寧中,廣東南路轉運判官王覺,以開闢荒田幾及萬頃,詔遷一官。其後,知州、部使者以能課民種桑棗者,率優其第秩焉。政和六年,立管幹圩岸、圍岸官法,在官三年,無隳損堙塞者賞之。京畿提點刑獄王本言:「前任提舉常平,根括諸縣天荒瘠鹵地一萬二千餘頃入稻田務,已佃者五千三百餘頃,尚慮令、佐不肯究心。」詔比開墾堿地格推賞。平江府興修圍田二千餘頃,令、佐而下以差減磨勘年。



 八年,權淮南、江、浙、荊湖制置發運
 使任諒奏:「高郵軍有逃田四百四十六頃,楚州九百七十四頃,泰州五百七十二頃,平江府四百九十七頃,以六路計之,何可勝數。欲諸縣專選官按籍根括。」詔無丞處委他官,餘並從之。



 宣和二年,臣僚上言:「監司、守令官帶勸農,莫副上意,欲立四證驗之:按田萊荒治之跡,較戶產登降之籍,驗米穀貴賤之價,考租賦盈虧之數。四證具,則其實著矣。」命中書審定取旨。五年,詔:「江東轉運司根括到逃田一百六十頃一十六畝,兩浙根括到四
 百五十六頃,召人出租,專充今年增屯戍兵衣糧。」初,政和中,品官限田,一品百頃,以差降殺,至九品為十畝;限外之數,並同編戶差科。七年,又詔:「內外宮觀舍置田,在京不得過五十頃,在外不得過三十頃,不免科差、徭役、支移。雖奉御筆,許執奏不行。」



 建炎元年五月,高宗即位,命有司招誘農民,歸業者振貸之,蠲欠租,免耕牛稅。五年,廣州州學教授林勛獻《本政書》十三篇,大略謂:「國朝兵農之政,大抵因唐末之故。今農貧而多失職,兵驕而
 不可用,是以饑民竄卒,類為盜賊。宜仿古井田之制,使民一夫占田五十畝,其羨田之家毋得市田;其無田與游惰末作者,皆使為農,以耕田之羨。雜紐錢穀,以為什一之稅。本朝二稅之數,視唐增至七倍。今本政之制,每十六夫為一井,提封百里,為三千四百井,率稅米五萬一千斛,錢萬二千緡。每井賦二兵一馬,率為兵六千八百人,馬三千四百匹。



 此方百里之縣所出賦稅之數。



 歲取五之一以為上番之額,以給徵役;無事則又分為四番,以直官府,以
 給守衛。是民凡三十五年,而役始一遍也。悉上則歲食米萬九千餘斛,錢三千六百餘緡,無事則減四分之三,皆以一同之租稅供之。匹婦之貢,絹三尺,綿一兩,百里之縣,歲收絹四千餘匹,綿二千四百斤;非蠶鄉則布六尺,麻二兩,所收視綿絹倍之。行之十年,則民之口算,官之酒酤,與凡茶、鹽、香、礬之榷,皆可弛以予民。」其說甚備。尋以勛為桂州節度掌書記。



 建炎以來,內外用兵,所在多逃絕之田。紹興二年四月,詔兩浙路收買牛具,貸淮
 東人戶。七月,詔:知興國軍王綯、知永興縣陳升率先奉詔誘民墾田,各增一秩。三年九月,戶部言:「百姓棄產,已詔二年外許人請射,十年內雖已請射及充職田者,並聽歸業。孤幼及親屬應得財產者,守令驗實給還,冒占者論如律。州縣奉行不虔,監司按劾。」從之。



 先是,臣僚言:「近詔州縣拘籍被虜百姓稅賦,而苛酷之吏不考其實,其間有父母被虜兒女存者,有中道脫者,有全家被虜而親屬偶歸者,一概籍沒,人情皇皇。」故有是命。



 十月,募佃江東、西閑田,三等定租:上田畝輸米一斗五升,中田一斗,下田七升。四年,貸廬州
 民錢萬緡,以買耕牛。



 五年五月,立《守令墾田殿最格》,殘破州縣墾田增及一分,郡守升三季名次,增及九分,遷一官;虧及一分,降三季名次,虧及九分,鐫一官。縣令差減之。增虧各及十分者,取旨賞罰。其後以兩淮、荊湖等路民稍復業,而曠土尚多,戶部復立格上之:每州增墾田千頃,縣半之,守宰各進一秩;州虧五百頃,縣虧五之一,皆展磨勘年。詔頒之諸路。增,謂荒田開墾者;虧,謂熟田不因災傷而致荒者。



 又令縣具歸業民數及墾田多寡,月上之州,州季上轉運,轉運歲上戶部,戶部置籍以考之。七月,都督行府言:「潭、鼎、岳、澧、荊南歸業之民,其田已佃者,以附近閑田與之,免三年租稅;無產願受閑田者,亦與之。」上
 諭輔臣曰:「淮北之民襁負而至,亦可給田,以廣招徠之意。」



 六年,減江東諸路逃田稅額。知平江府章誼言:「民所甚苦者,催科無法,稅役不均。強宗巨室阡陌相望,而多無稅之田,使下戶為之破產。乞委通判一員均平賦役。」九年,宗正少卿方庭實言:「中原士民奔逃南州,十有四年,出違十年之限及流徙僻遠卒未能歸者,望詔有司別立限年。」戶部議:「自復降赦日為始,再期五年,如期滿無理認者,見佃人依舊承佃。中原士民流寓東南,往往
 有墳墓,或官拘籍,或民冒占,便行給還。」從之。十一年,復買牛貸淮南農戶。



 十二年,左司員外郎李椿年言經界不正十害,且言:「平江歲入昔七十萬有奇,今按籍雖三十九萬斛,然實入才二十萬耳。詢之土人,皆欺隱也。望考按核實,自平江始,然後施之天下,則經界正而仁政行矣。」上謂宰執曰:「椿年之論,頗有條理。」秦檜亦言其說簡易可行。程克俊曰:「比年百姓避役,正緣經界不正。行之,乃公私之利。」以椿年為兩浙路轉運副使,措置經界。椿
 年請先往平江諸縣,俟就緒即往諸州,要在均平,為民除害,不增稅額。十三年,以提舉洪州玉隆觀胡思、直顯謨閣徐林議沮經界,停官遠徙。以民田不上稅簿者沒官,稅簿不謹書者罪官吏。時量田不實者,罪至徒、流,江山尉汪大猷白椿年曰:「法峻,民未喻,固有田少而供多者,願許陳首追正。」椿年為之輕刑、省費甚眾。



 十四年,以椿年權戶部侍郎,措置經界。尋以母憂去,以兩浙轉運副使王鈇權戶部侍郎措置。十五年,詔戶部及所遣官
 委曲措置,務使賦稅均而無擾。又因興國軍守臣宋時言,詔諸州縣違期歸業者,其田已佃及官賣者,即以官田之可耕者給還。十六年,王鈇以疾罷。十七年,復以李椿年權戶部侍郎,措置經界。先是,真州兵燼之餘,瘡痍未復,洪興祖為守,請復租二年,明年又復請之,自是流民浸歸。十八年,墾荒田至七萬餘畝。



 十九年,詔敕令所刪定官鄭克行四川經界法。克頗峻責州縣,所謂「省莊田」者,雖蔬果、桑柘莫不有征,而邛、蜀民田至什稅其伍。
 通判嘉州楊承曰:「仁政而虐行之,非法意也。上不違令,下不擾民,則仁政得矣。」召諸邑令謂曰:「平易近民,美成在久,其謹行之。無愧於心,何畏焉?」事迄成,為列郡最。其後,民有訴不均者,殿中侍御史曹筠劾椿年,罷之。上謂秦檜曰:「若下田受重稅,將無以輸。」檜曰:「臣已諭戶部侍郎宋貺,有未均處亟與改正。」二十年,詔:兩淮沃壤宜穀,置力田科,募民就耕,以廣官莊。知資州楊師錫言:有司奉行失當,田畝不分腴瘠,市居丈尺隙田,亦充稅產。於
 是降詔曰:「椿年乞行經界,去民十害,今聞浸失本意。凡便民者依已行,害民者與追正。」二十一年四月,宋貺罷。二十六年正月,上謂輔臣曰:「經界事李椿年主之,若推行就緒,不為不善。今諸路往往中輟,願得一通曉經界者款曲議之。」會潼川府轉運判官王之望上書,言蜀中經界利害甚悉。明年,以之望提點刑獄,畢經界事。



 三月,戶部言:「蜀地狹人伙,而京西、淮南膏腴官田尚多,許人承佃,官貸牛、種,八年仍償。並邊免租十年,次邊半之,滿
 三年與其業。願往者給據津發。」上曰:「善。但貧民乍請荒田,安能便得牛、種?若不從官貸,未免為虛文,可令相度支給。」四月,通判安豐軍王時升言:「淮南土皆膏腴,然地未盡闢、民不加多者,緣豪強虛占良田,而無遍耕之力;流民襁負而至,而無開耕之地。望凡荒閑田許人鏟佃。」戶部議:期以二年,未墾者即如所請;京西路如之。詔以時升為司農寺丞。十月,用御史中丞湯鵬舉言,離軍添差之人,授以江、淮、湖南荒田,人一頃,為世業。所在郡以
 一歲奉充牛、種費,仍免租稅十年,丁役二十年。



 二十八年,王之望言:「去年分遣官詣經界不均縣裁正,今已迄事。此後吏民尚敢扇搖以疑百姓者,乞重置於法。」從之。二十九年,知潭州魏良臣言:「本州歸業之民,以熟田為荒,不輸租。今令給甲輸稅,自明年始,不實,許人告,以為田賞之。」戶部議:「期逾百日,依匿稅法。」詔可。三十年,初令純州平江縣民實田輸稅,畝輸米二升四合。



 孝宗隆興元年,詔:「凡百姓逃棄田宅,出三十年無人歸認者,依戶
 絕法。」乾道元年正月,都省言:「淮民復業,宜先勸課農桑。令、丞植桑三萬株至六萬株,守、倅部內植二十萬株以上,並論賞有差。」二月,三省、樞密院言:「歸正人貧乏者散居兩淮,去冬淮民種麥甚廣,逃亡未歸,無人收獲。」詔諸郡量口均給,其已歸業者毋例擾之。四年,知鄂州李椿奏:「州雖在江南,荒田甚多,請佃者開墾未幾,便起毛稅,度田追呼,不任其擾,旋即逃去。今欲召人請射,免稅三年;三年之後為世業,三分為率,輸苗一分,更三年增一
 分,又三年全輸。歸業者別以荒田給之。」又詔楚州給歸正人田及牛具、種糧錢五萬緡。



 六年二月,詔曰:「朕深惟治不加進,思有以正其本者。今欲均役法,嚴限田,抑游手,務農桑。凡是數者,卿等二三大臣為朕任之。」十有二月,監進奏院李結獻《治田三議》:一曰務本,二曰協力,三曰因時。大略謂:「浙西低田恃堤為固,若堤岸高厚,則水不能入。乞於蘇、湖、常、秀諸州水田塘浦要處,官以錢米貸田主,乘此農隙,作堰增令高闊,則堤成而水不為患。
 方此饑饉,俾食其力,因其所利而利之。秋冬旱涸,涇濱斷流,車畎修築,尤為省力。」詔令胡堅常相度以聞。其後,戶部以三議切當,但工力浩瀚,欲曉有田之家,各依鄉原畝步出錢米與租田之人,更相修築,庶官無所費,民不告勞。從之。



 七年二月,知揚州晁公武奏:「朝廷以沿淮荒殘之久,未行租稅,民復業與創戶者,雖阡陌相望,然聞之官者十才二三,咸懼後來稅重。昔晚唐民務稼穡則增其租,故播種少;吳越民墾荒田而不加稅,故無曠
 土。望詔兩淮更不增賦,庶民知勸。」詔可。十月,司馬伋請勸民種麥,為來春之計。於是詔江東西、湖南北、淮東西路帥漕,官為借種及諭大姓假貸農民廣種,依賑濟格推賞,仍上已種頃畝,議賞罰。九年,王之奇奏增定力田賞格,募人開耕荒田,給官告綾紙以備書填,及官會十萬緡充農具等用。以種糧不足,又詔淮東總領所借給稻三萬石。



 淳熙五年,詔:「湖北佃戶開墾荒田,止輸舊稅。若包占頃畝,未悉開耕,詔下之日,期以二年,不能遍耕
 者拘作營田,其增稅、鏟佃之令勿行。」六年五月,提舉浙西常平茶鹽顏師魯奏:「設勸課之法,欲重農桑、廣種植也。今鄉民於己田連接閑曠磽確之地,墾成田園,用力甚勤。或以未陳起稅,為人所訟,即以盜耕罪之,何以勸力田哉?止宜實田起稅,非特可戢告訐之風,亦見盛世重農之意。」詔可。十有一月,臣僚奏:「比令諸路帥、漕督守令勸諭種麥,歲上所增頃畝。然土有宜否,湖南一路唯衡、永等數郡宜麥,餘皆文具。望止諭民以時播種,免其
 歲上增種之數,庶得勸課之實。」



 七年,復詔兩浙、江、淮、湖南、京西路帥、漕臣督守令勸民種麥,務要增廣。自是每歲如之。八年五月,詔曰:「乃者得天之時,蠶麥既登,及命近甸取而視之,則穗短繭薄,非種植風厲之功有所未至歟?朕將稽勤惰而詔賞罰焉。」是歲連雨,下田被浸,詔兩浙諸州軍與常平司措置,再借種糧與下戶播種,毋致失時。十有一月,輔臣奏:「田世雄言,民有麥田,雖墾無種,若貸與貧民,猶可種春麥。臣僚亦言,江、浙旱田雖已
 耕,亦無麥種。」於是詔諸路帥、漕、常平司,以常平麥貸之。



 先是,知揚州鄭良嗣言:「兩淮民田,廣至包占,多未起稅。朝廷累限展首,今限滿適旱,乞更展一年。」詔如其請。九年,著作郎袁樞振兩淮還,奏:「民占田不知其數,二稅既免,止輸穀帛之課。力不能墾,則廢為荒地;他人請佃,則以疆界為詞,官無稽考。是以野不加闢,戶不加多,而郡縣之計益窘。望詔州縣畫疆立券,占田多而輸課少者,隨畝增之;其餘閑田,給與佃人,庶幾流民有可耕之地,
 而田萊不至多荒。」



 紹熙元年,初,朱熹為泉之同安簿,知二郡經界不行之害。至是,知漳州。會臣僚請行閩中經界,詔監司條具,事下郡。熹訪問講求,纖悉備至。乃奏言:「經界最為民間莫大之利,紹興已推行處,公私兩利,獨泉,漳、汀未行。臣不敢先一身之勞逸,而後一州之利病,切獨任其必可行也。然必推擇官吏,委任責成;度量步畝,算計精確;畫圖造帳,費從官給;隨產均稅,特許過鄉通縣均紐,庶幾百里之內,輕重齊同。今欲每畝隨九等
 高下定計產錢,而合一州租稅錢米之數,以產錢為母,每文輸米幾何,錢幾何,止於一倉一庫受納。既輸之後,卻視原額分隸為省計,為職田,為學糧,為常平,各撥入諸倉庫。版圖一定,則民業有經矣。但此法之行,貧民下戶固所深喜,然不能自達其情;豪家猾吏實所不樂,皆善為說辭,以惑群聽;賢士大夫之喜安靜、厭紛擾者,又或不深察而望風沮怯,此則不能無慮。」輔臣請行於漳州。明年春,詔漕臣陳公亮同熹協力奉行。會農事方興,
 熹益加講究,冀來歲行之。細民知其不擾而利於己,莫不鼓舞,而貴家豪右占田隱稅、侵漁貧弱者,胥為異論以搖之,前詔遂格。熹請祠去。五年,蠲廬州旱傷百姓貸稻種三萬二千一百石。



 慶元元年二月,上以歲兇,百姓饑病,詔曰:「朕德菲薄,饑饉薦臻,使民阽於死亡,夙夜慘怛,寧敢諉過於下耶?顧使者、守令所與朕分寄而共憂也,乃涉春以來,聞一二郡老稚乏食,去南畝,捐溝壑,咎安在耶?豈振給不盡及民歟?得粟者未必饑,饑者未必
 得歟?偏聚於所近,不能均濟歟?官吏視成而自不省歟?其各恪意措畫,務使實惠不壅,毋以虛文蒙上,則朕汝嘉。」



 寧宗開禧元年,夔路轉運判官範蓀言:「本路施、黔等州荒遠,綿亙山谷,地曠人稀,其占田多者須人耕墾,富豪之家誘客戶舉室遷去。乞將皇祐官莊客戶逃移之法校定:凡為客戶者,許役其身,毋及其家屬;凡典賣田宅,聽其離業,毋就租以充客戶;凡貸錢,止憑文約交還,毋抑勒以為地客;凡客戶身故、其妻改嫁者,聽其自便,
 女聽其自嫁。庶使深山窮谷之民,得安生理。」刑部以皇祐逃移舊法輕重適中,可以經久,淳熙比附略人之法太重,今後凡理訴官莊客戶,並用皇祐舊法。從之。



 嘉定八年,左司諫黃序奏:「雨澤愆期,地多荒白。知餘杭縣趙師恕請勸民雜種麻、粟、豆、麥之屬,蓋種稻則費少利多,雜種則勞多獲少。慮收成之日,田主欲分,官課責輸,則非徒無益;若使之從便雜種,多寡皆為己有,則不勸而勤,民可無饑。望如所陳,下兩浙、兩淮、江東西等路,凡有
 耕種失時者並令雜種,主毋分其地利,官毋取其秋苗,庶幾農民得以續食,官免振救之費。」從之。



 知婺州趙(與心)夫行經界於其州,整有倫緒,而(與心)夫報罷。士民相率請於朝,乃命趙師巖繼之。後二年,魏豹文代師巖為守,行之益力。於是向之上戶析為貧下之戶,實田隱為逃絕之田者,粲然可考。凡結甲冊、戶產簿、丁口簿、魚鱗圖、類姓簿二十三萬九千有奇,創庫匱以藏之,歷三年而後上其事於朝。



 淳祐二年九月,敕曰:「四川累經兵火,百姓
 棄業避難,官以其曠土權耕屯以給軍食,及民歸業,占據不還。自今凡民有契券,界至分明,析在州縣屯官隨即歸還。其有違戾,許民越訴,重罪之。」



 六年,殿中侍御史兼侍講謝方叔言:



 「豪強兼並之患,至今日而極,非限民名田有所不可,是亦救世道之微權也。國朝駐蹕錢塘,百有二十餘年矣。外之境土日荒,內之生齒日繁,權勢之家日盛,兼並之習日滋,百姓日貧,經制日壞,上下煎迫,若有不可為之勢。所謂富貴操柄者,若非人主之所
 得專,識者懼焉。夫百萬生靈資生養之具,皆本於穀粟,而穀粟之產,皆出於田。今百姓膏腴皆歸貴勢之家,租米有及百萬石者;少民百畝之田,頻年差充保役,官吏誅求百端,不得已,則獻其產於巨室,以規免役。小民田日減而保役不休,大官田日增而保役不及。以此弱之肉,強之食,兼並浸盛,民無以遂其生。於斯時也,可不嚴立經制以為之防乎?



 去年,諫官嘗以限田為說,朝廷付之悠悠。不知今日國用邊餉,皆仰和糴。然權勢多田之
 家,和糴不容以加之,保役不容以及之。敵人睥睨於外,盜賊窺伺於內,居此之時,與其多田厚貲不可長保,曷若捐金助國共紓目前?在轉移而開導之耳。乞諭二三大臣,摭臣僚論奏而行之,使經制以定,兼並以塞,於以尊朝廷,於以裕國計。陛下勿牽貴近之言以搖初意,大臣勿避仇怨之多而廢良策,則天下幸甚。」從之。



 十一年九月,敕曰:「監司、州縣不許非法估籍民產,戒非不嚴,而貪官暴吏,往往不問所犯輕重,不顧同居有分財產,一
 例估籍,殃及平民。或戶絕之家不與命繼;或經陳訴許以給還,輒假他名支破,竟成乾沒;或有典業不聽收贖,遂使產主無辜失業。違戾官吏,重置典憲。」是歲,信常饒州、嘉興府舉行經界。



 景定元年九月,敕曰:「州縣檢校孤幼財產,往往便行侵用,洎至年及陳乞,多稱前官用過,不即給還。自今如尚違戾,以吏業估償,官論以違制,不以去官、赦、降原減。」



 咸淳元年,監察御史趙順孫言:「經界將以便民,雖窮閻下戶之所深願,而未必豪宗大姓之
 所盡樂。自非有以深服其心,則亦何以使其情意之悉孚哉?且今之所謂推排,非昔之所謂自實也。推排者,委之鄉都,則徑捷而易行;自實者,責之於人戶,則散漫而難集。嘉定以來之經界,時至近也,官有正籍,鄉都有副籍,彪列昈分,莫不具在,為鄉都者不過按成牘而更業主之姓名。若夫紹興之經界,其時則遠矣,其籍之存者寡矣。因其鱗差櫛比而求焉,由一而至百,由百而至千,由千而至萬,稽其畝步,訂其主佃,亦莫如鄉都之便也。
 朱熹所以主經界而闢自實者,正謂是也。州縣能守朝廷鄉都任責之令,又隨諸州之便宜而為之區處,當必人情之悉孚,不令而行矣。」從之。



 三年,司農卿兼戶部侍郎李鏞言:「夫經界嘗議修明矣,而修明卒不行;嘗令自實矣,而自實卒不竟。豈非上之任事者每欲避理財之名,下之不樂其成者又每倡為擾民之說。故寧坐視邑政之壞,而不敢詰猾吏奸民之欺;寧忍取下戶之苛,而不敢受豪家大姓之怨。蓋經界之法,必多差官吏,必悉
 集都保,必遍走阡陌,必盡量步畝,必審定等色,必紐折計等,奸弊轉生,久不迄事。乃若推排之法,不過以縣統都,以都統保,選任才富公平者,訂田畝稅色,載之圖冊,使民有定產,產有定稅,稅有定籍而已。臣守吳門,巳嘗見之施行。今聞紹興亦漸就緒,湖南漕臣亦以一路告成。竊謂東南諸郡,皆奉行惟謹。其或田畝未實,則令鄉局厘正之;圖冊未備,則令縣局程督之。又必郡守察縣之稽違,監司察郡之怠弛,嚴其號令,信其賞罰,期之秋
 冬以竟其事,責之年歲以課其成,如《周官》日成、月要、歲會以綜核之。」於是詔諸路漕、帥施行焉。



 大抵南渡後水田之利,富於中原,故水利大興。而諸籍沒田募民耕者,皆仍私租舊額,每失之重,輸納之際,公私事例迥殊。私租額重而納輕,承佃猶可;公租額重而納重,則佃不堪命。州縣胥吏與倉庫百執事之人,皆得為侵漁之道於耕者也。季世金人乍和乍戰,戰則軍需浩繁,和則歲幣重大,國用常苦不繼,於是因民苦官租之重,命有司括
 賣官田以給用。其初弛其力役以誘之,其終不免於抑配,此官田之弊也。嘉定以後,又有所謂安邊所田,收其租以助歲幣。至其將亡,又限民名田,買其限外所有,謂之公田。初議欲省和糴以紓民力,而其弊極多,其租尤重;宋亡,遺患猶不息也。凡水田、官田之法,公田見於史者,匯其始末而悉載於篇,有足鑒者焉。



 紹興元年,詔宣州、太平州守臣修圩。二年,以修圩錢米及貸民種糧,並於宣州常平、義倉米撥借。三年,定州縣圩田租額充軍
 儲。建康府永豐圩租米,歲以三萬石為額。圩四至相去皆五六十里,有田九百五十餘頃,近歲墾田不及三之一。至是,始立額。



 五年,江東帥臣李光言:「明、越之境,皆有陂湖,大抵湖高於田,田又高於江、海。旱則放湖水溉田,澇則決田水入海,故無水旱之災。本朝慶歷、嘉祐間,始有盜湖為田者,其禁甚嚴。政和以來,創為應奉,始廢湖為田。自是兩州之民,歲被水旱之患。餘姚、上虞每縣收租不過數千斛,而所失民田常賦,動以萬計。莫若先罷
 兩邑湖田。」其會稽之鑒湖、鄞之廣德湖、蕭山之湘湖等處尚多,望詔漕臣盡廢之。其江東、西圩田,蘇、秀圍田,令監司守令條上。」於是詔諸路漕臣議之。其後議者雖稱合廢,竟仍其舊。



 初,五代馬氏於潭州東二十里,因諸山之泉,築堤瀦水,號曰龜塘,溉田萬頃。其後堤壞,歲旱,民皆阻饑。七年,守臣呂頤浩始募民修復,以廣耕稼。十六年,知袁州張成巳言:「江西良田,多占山岡,望委守令講陂塘灌溉之利。」其後比部員外郎李詠言,淮西高原處
 舊有陂塘,請給錢米,以時修浚。知江陰軍蔣及祖亦請浚治本軍五卸溝以洩水,修復橫河支渠以溉旱。乃並詔諸路常平司行之,每季以施行聞。



 二十三年,諫議大夫史才言:「浙西、民田最廣,而平時無甚害者,太湖之利也。近年瀕湖之地,多為兵卒侵據,累土增高,長堤彌望,名曰壩田。旱則據之以溉,而民田不沾其利;澇則遠近泛濫,不得入湖,而民田盡沒。望盡復太湖舊跡,使軍民各安,田疇均利。」從之。二十四年,大理寺丞周環言:「臨安、
 平江、湖、秀四州下田,多為積水所浸。緣溪山諸水並歸太湖,自太湖分二派:東南一派由松江入於海,東北一派由諸浦注之江。其松江洩水,惟白茅一浦最大。今泥沙淤塞,宜決浦故道,俾水勢分派流暢,實四州無窮之利。」詔兩浙漕臣視之。



 二十八年,兩浙轉運副使趙子潚、知平江府蔣璨言:「太湖者,數州之巨浸,而獨洩以松江之一川,宜其勢有所不逮。是以昔人於常熟之北開二十四浦,疏而導之江;又於昆山之東開一十二浦,分而
 納之海。三十六浦後為潮汐沙積,而開江之卒亦廢,於是民田有淹沒之患。天聖間,漕臣張綸嘗於常熟、昆山各開眾浦;景祐間,郡守範仲淹亦親至海浦,浚開五河;政和間提舉官趙霖復嘗開浚。今諸浦湮塞,又非前比,計用工三百三十餘萬,錢三十三萬餘緡,米十萬餘斛。」於是詔監察御史任古復視之。既而古至平江言:「常熟五浦通江誠便,若依所請,以五千功,月餘可畢。」詔以激賞庫錢、平江府上供米如數給之。二十九年,子潚又言:「
 父老稱福山塘與丁涇地勢等,若不浚福山塘,則水必倒注于丁涇。」乃命並浚之。



 隆興二年八月,詔:「江、浙水利,久不講修,勢家圍田,堙塞流水。諸州守臣按視以聞。」於是知湖州鄭作肅、知宣州許尹、知秀州姚憲、知常州劉唐稽並乞開圍田,浚港瀆。詔湖州委朱夏卿,秀州委曾□,平江府委陳彌作,常州、江陰軍委葉謙亨,宣州、太平州委沉樞措置。九月,刑部侍郎吳芾言:「昨守紹興,嘗請開鑒湖廢田二百七十頃,復湖之舊,水無泛濫,民田九
 千餘頃,悉獲倍收。今尚有低田二萬餘畝,本亦湖也,百姓交佃,畝直才兩三緡。欲官給其半,盡廢其田,去其租。」戶部請符浙東常平司同紹興府守臣審細標遷。從之。



 乾道二年四月,詔漕臣王炎開浙西勢家新圍田,草蕩、荷蕩、菱蕩及陂湖溪港岸際旋築塍畦、圍裹耕種者,所至守令同共措置。炎既開諸圍田,凡租戶貸主家種糧債負,並奏蠲之。六月,知秀州孫大雅代還,言:「州有柘湖、澱山湖、當湖、陳湖,支港相貫,西北可入於江,東南可達
 於海。旁海農家作壩以卻堿潮,雖利及一方,而水患實害鄰郡;設疏導之,則又害及旁海之田。若於諸港浦置閘啟閉,不惟可以洩水,而旱亦獲利。然工力稍大,欲率大姓出錢,下戶出力,於農隙修治之。」於是以兩浙轉運副使姜詵與守臣視之,詵尋與秀常州、平江府、江陰軍條上利便。詔:「秀州華亭縣張涇閘並澱山東北通陂塘港淺處,俟今年十一月興修;江陰軍、常州蔡涇閘及申港,明年春興修;利港俟休役一年興修;平江府姑緩之。」
 三年三月,詵使還,奏:「開浚畢功,通洩積水,久浸民田露出塍岸。臣已諭民趁時耕種。恐下戶闕本,良田復荒,望令浙西常平司貸給種糧。」又奏措置、提督、監修等官知江陰軍徐藏等減磨勘年有差。



 四年,以彭州守臣梁介修復三縣一十餘堰,灌溉之利及於鄰邦,詔介直秘閣、利路轉運判官。七年,王炎言:「興元府山河堰世傳漢蕭、曹所作。本朝嘉祐中,提舉史照上堰法,獲降敕書刻石堰上。紹興以來,戶口凋疏,堰事荒廢,遂委知興元府吳
 拱修復,發卒萬人助役。宣撫司及安撫、都統司共享錢三萬一千餘緡,盡修六堰,浚大小渠六十五里,凡溉南鄭、褒城田二十三萬三千畝有奇。」詔獎諭拱。



 八年,戶部侍郎兼樞密都承旨葉衡言:「奉詔核實寧國府、太平州圩岸,內寧國府惠民、化城舊圩四十餘里,新築九里餘;太平州黃池鎮福定圩周四十餘里,庭福等五十四圩周一百五十餘里,包圍諸圩在內,蕪湖縣圩周二百九十餘里,通當塗圩共四百八十餘里。並高廣堅致,瀕水
 一岸種植榆柳,足捍風濤,詢之農民,實為永利。」於是詔獎諭判寧國府魏王愷,略曰:「大江之堧,其地廣袤,使水之蓄洩不病而皆為膏腴者,圩之為利也。然水土斗嚙,從昔善壞。卿聿修稼政,巨防屹然,有懷勤止,深用嘆嘉。」九年八月,臣僚言江西連年荒旱,不能預興水利為之備。於是乃降詔曰:「朕惟旱乾、水溢之災,堯、湯盛時,有不能免。民未告病者,備先具也。豫章諸郡縣,但阡陌近水者,苗秀而實;高仰之地,雨不時至,苗輒就槁。意水利不
 修,失所以為旱備乎?唐韋丹為江西觀察使,治陂塘五百九十八所,灌田萬二千頃。此特施之一道,其利如此,矧天下至廣也。農為生之本也,泉流灌溉,所以毓五穀也。今諸道名山,川原甚眾,民未知其利。然則通溝瀆,瀦陂澤,監司、守令,顧非其職歟?其為朕相丘陵原隰之宜,勉農桑,盡地利,平繇行水,勿使失時。雖有豐兇,而力田者不至拱手受弊,亦天人相因之理也。朕將即勤惰而寓賞罰焉。」



 淳熙二年,兩浙轉運判官陳峴言:「昨奉詔遍
 走平江府、常州、江陰軍,諭民並力開浚利港諸處,並已畢功。始欲官給錢米,歲不下數萬,今皆百姓相率效力而成。」詔常熟知縣劉穎特增一秩,餘論賞有差。三年,賜皇子判明州魏王愷詔曰:「陂湖川澤之利,或通或塞,存乎其人。四明為州實治鄞,鄞之鄉東西凡十四,而錢湖之水實溉其東之七。吏惰不虔,葑菼蕪翳,利失其舊,農人病焉。卿臨是邦,乃能講求利便而浚治之,遂使並湖七鄉之田,無異時旱乾之患,其為澤豈淺哉。剡奏徹聞,
 不忘嘉嘆。」



 十年,大理寺丞張抑言:「陂澤湖塘,水則資之瀦洩,旱則資之灌溉。近者浙西豪宗,每遇旱歲,占湖為田,築為長堤,中植榆柳,外捍茭蘆,於是舊為田者,始隔水之出入。蘇、湖、常、秀昔有水患,今多旱災,蓋出於此。乞責縣令毋給據,尉警捕,監司覺察。有圍裹者,以違制論;給據與失察者,並坐之。」既而漕臣錢沖之請每圍立石以識之,共一千四百八十九所,令諸郡遵守焉。



 紹熙二年,詔守令到任半年後,具水源湮塞合開修處以聞;任
 滿日,以興修水利圖進,擇其勞效著明者賞之。慶元二年,戶部尚書袁說友等言:「浙西圍田相望,皆千百畝,陂塘漊瀆,悉為田疇,有水則無地可瀦,有旱則無水可戽。不嚴禁之,後將益甚,無復稔歲矣。」嘉泰元年,以大理司直留祐賢、宗正寺主簿李澄措置,自淳熙十一年立石之後,凡官民圍裹者盡開之。又令知縣並以「點檢圍田事」入銜,每歲三四月,同尉點檢有無奸民圍裹狀,上於州,州聞於朝。三年遣官審視,及委臺諫察之。二年二月,
 祐賢、澄使還,奏追毀臨安、平江、嘉興,湖、常開掘戶元給佃據。三月,右正言施康年言:「近屬貴戚不體九重愛民之心,止為一家營私之計,公然投牒以沮成法,乞戒飭:自今有陳狀者,指名奏劾,必罰無赦。」



 開禧二年,以淮農流移,無田可耕,詔兩浙州縣已開圍田,許元主復圍,專召淮農租種。嘉定三年,臣僚言:「竊聞豪民巨室並緣為奸,加倍圍裹,又影射包占水蕩,有妨農民灌溉。」於是復詔浙西提舉司俟農隙開掘。七年,復臨安府西湖舊界,
 盡蠲歲增租錢。十七年,臣僚言:「越之鑒湖,溉田幾半會稽,興化之木蘭陂,民田萬頃,歲飲其澤。今官豪侵占,填淤益狹。宜戒有司每歲省視,厚其瀦蓄,去其壅底,毋容侵占,以妨灌溉。」皆次第行之。



 寶慶元年,以右諫議大夫朱端常奏,除嘉泰間已開浙西圍田租錢,蓋稅額尚存,州縣迫民白納故也。寶祐元年,史館校勘黃國面對:「圍田自淳熙十一年識石者當存之,復圍者合權其利害輕重而為之存毀,其租或歸總所,或隸安邊所,或分隸
 諸郡。」上曰:「安邊所田,近已撥歸本所。」國又奏:「自丁未已來創圍之田,始因殿司獻草蕩,任事者欲因以為功,凡旱乾處悉圍之,利少害多,宜開掘以通水道。」上然之。咸淳十年,以江東水傷,除九年圩田租,減四分。



 紹興二十七年,趙子潚奉詔措置鎮江府沙田,欲輕立租課,令見佃者就耕;如勢家占吝,追日前所收租利。詔速拘其田措置,蠲其冒佃之租。二十八年正月,詔戶部員外郎莫蒙同浙西、江東、淮南漕臣趙子潚、鄧根、孫藎視諸路沙
 田、蘆場。先是,言者謂江、淮間沙田、蘆場為人冒占,歲失官課至多,故以命蒙等。既而殿中侍御史葉義問言:「奉行者不恤百姓,名為經量,實逼縣官按圖約紐,惟務增數,以希進用。有力之家初無加損,貧民下戶已受其害。因小利擾之,必致逃移,坐失稅額。」因極論之。二月,詔:「沙田、蘆場止為勢家詭名冒占,其三等以下戶勿例根括。」六月,以孫藎措置沙田滅裂,罷之。詔:「浙西江東沙田、蘆場,官戶十頃、民戶二十頃以上並增租,餘如舊。置提領
 官田所掌之,不隸戶部。」二十九年,以莫蒙經量沙田、蘆場失實,責監饒州景德鎮稅,遂詔盡罷所增租。



 三十二年九月,趙子潚言:「浙西、江東、淮東沙田,往年經量,有不盡不實處,為人戶包占。期以今冬自陳,給為己業,與免租稅之半;過期許人告,以全戶所租田賞之。其蘆場量力輕租。」詔以馮方措置。十有一月,方滋疏論沙田。上問:「沙田或以為可取,或以為可捐。」陳康伯等奏:「君子小人,各從其類。小人樂於生事,不惜為國斂怨;君子務存大
 體,唯恐有傷仁政,所以不同。」上然之,命止前詔勿行。



 乾道元年,臣僚言:「浙西、淮東、江東路沙田蘆場,頃畝浩瀚,宜立租稅,補助軍食。」詔復令梁俊彥與張津等措置。二年,輔臣奏:「俊彥所上沙田、蘆場之稅,或十取其一,或取其二,或取其三,皆不分主客。」朝廷疑之。六年,以俊彥所括沙田、蘆場二百八十餘萬畝,其間或已充己業,起稅不一,及包占未起租者,乞並估賣、立租。詔蔡洸、梁俊彥行在置司措置。八年七月,詔提領官田所所催三路沙
 田、蘆場租錢並歸戶部。十月,遣官實江、淮沙田、蘆場頃畝,悉追正之。



 建炎元年,籍蔡京、王黼等莊以為官田,詔見佃者就耕,歲減租二分。三年,凡天下官田,令民依鄉例自陳輸租。紹興元年,以軍興用度不足,詔盡鬻諸路官田。五年,詔諸官田比鄰田租,召人請買,佃人願買者聽,佃及三十年以上者減價十之二。六年,詔諸路總領諭民投買戶絕、沒官、及江漲沙田、海退泥田。七年,以賊徒田舍及逃田充官莊,其沒官田依舊出賣。二十年,凡
 沒官田、城空田、戶絕房廊及田,並撥隸常平司;轉運、提刑、茶鹽司沒入田亦如之。



 二十一年,以大理寺主簿丁仲京言,凡學田為勢家侵佃者,命提學官覺察。又命撥僧寺常住絕產以贍學。戶部議並撥無敕額庵院田,詔可。初,閩以福建八郡之田分三等:膏腴者給僧寺、道院,中下者給土著流寓。自劉夔為福州,始貿易取貲。迨張守帥閩,紹興二年秋。上倚以拊循凋瘵,存上等四十餘剎以待高僧,餘悉令民請買,歲入七、八萬緡以助軍衣,
 餘寬百姓雜科,民皆便之。



 二十六年,以諸路賣官田錢七分上供,三分充常平司糴本。初,盡鬻官田,議者恐佃人失業,未賣者失租。侍御史葉義問言:「今盡鬻其田,立為正稅,田既歸民,稅又歸官,不獨絕欺隱之弊,又可均力役之法。」浙東刑獄使者邵大受亦乞承買官田者免物力三年至十年。一千貫以下免三年,一千貫以上五年,五千貫以上十年。於是詔所在常平沒官、戶絕田,已佃未佃、已添租未添租,並拘賣。二十九年,初,兩浙轉運
 司官莊田四萬二千餘畝,歲收稻、麥等四萬八千餘斛;營田九十二萬六千餘畝,歲收稻、麥、雜豆等十六萬七千餘斛,充行在馬料及糴錢。四月,詔令出賣。七月,詔諸路提舉常平官督察欺弊,申嚴賞罰。分水令張升佐、宜興令陳𨑖以賣田稽違,各貶秩罷任。九月,浙東提舉常平都絜以賣田最多,增一秩。三十年,詔承買荒田者免三年租。



 乾道二年,戶部侍郎曾懷言:「江西路營田四千餘頃,已佃一千九百餘頃,租錢五萬五百餘貫,若出賣,
 可得六萬七千餘貫;及兩浙轉運司所括已佃九十餘萬畝,合而言之,為數浩瀚。今欲遵元詔,見佃願買者減價二分。」詔曾懷等提領出賣,其錢輸左藏南庫別貯之。四年四月,江東路營田亦令見佃者減價承買,期以三月賣絕,八月住賣;諸路未賣營田,轉運司收租。七年,提舉浙西常平李結乞以見管營田撥歸本司,同常平田立管莊。梁克家亦奏:「戶部賣營田,率為有力者下價取之,稅入甚微,不如置官莊,歲可得五十萬斛。」八年,以大
 理寺主簿薛季宣於黃岡、麻城立官莊二十二所。九年,以司農寺丞葉翥等出賣浙東、西路諸官田,以登聞檢院張孝賁等出賣江東、西路諸官田,以郎官薛元鼎拘催江、浙、閩、廣賣官田錢四百餘萬緡。



 淳熙元年,臣僚言:「出賣官田,二年之間,三省、戶部困於文移,監司、州郡疲於出賣。上下督責,不為不至,始限一季,繼限一年,已賣者才十三,已輸者才十二。蓋賣產之家,無非大姓。估價之初,以上色之產,輕立價貫,揭榜之後,率先投狀;若中
 下之產,無人屬意,所立之價,輕重不均。莫若且令元佃之家著業輸租,數猶可得數十萬斛。」從之。六年,詔諸路轉運、常平司,凡沒官田、營田、沙田、沙蕩之類,復括數賣之。紹熙四年,以臣僚言住賣。慶元元年八月,江東轉運提舉司以紹熙四年住賣以後續沒官田,依鄉價復召人承買,以其錢充常平糴本。十有一月,餘端禮、鄭僑言,福建地狹人稠,無以贍養,生子多不舉。福建提舉宋之瑞乞免鬻建、劍、汀、邵沒官田,收其租助民舉子之費,詔
 從之。四年,詔諸路召賣不行田,復實減價,其沙礫不可新處除之。



 開熙三年,韓侂冑既誅,金人講解。明年,用廷臣言,置安邊所,凡侂冑與其它權幸沒入之田,及圍田、湖田之在官者皆隸焉。輸米七十二萬二千七百斛有奇,錢一百三十一萬五千緡有奇,藉以給行人金、繒之費。迨與北方絕好,軍需邊用每於此取之。



 景定四年,殿中侍御史陳堯道、右正言曹孝慶、監察御史虞慮張晞顏等言廩兵、和糴、造楮之弊,「乞依祖宗限田議,自兩浙、
 江東西官民戶逾限之田,抽三分之一買充公田。得一千萬畝之田,則歲有六七百萬斛之入可以餉軍,可以免糴,可以重楮,可以平物而安富,一舉而五利具矣。」有旨從其言。朝士有異議者,丞相賈似道奏:「救楮之策莫切於住造楮,住造楮莫切於免和糴,免和糴莫切於買逾限田。」因歷詆異議者之非,帝曰:「當一意行之。」浙西安撫魏克愚言:「取四路民田立限回買,所以免和糴而益邦儲,議者非不自以為公且忠也。然而未見其利,而適見
 其害。近給事中徐經孫奏記丞相,言江西買田之弊甚詳,若浙西之弊,則尤有甚於經孫所言者。」因歷述其為害者八事,疏奏不省。



 六郡回買公田,畝起租滿石者償二百貫,九斗者償一百八十貫,八斗者償一百六十貫,七斗者償一百四十貫,六斗者償一百二十貫。五千畝以上,以銀半分、官告五分、度牒二分、會子二分半;五千畝以下,以銀半分、官告三分、度牒二分、會子三分半;千畝以下,度牒、會子各半;五百畝至三百畝,全以會子。是
 歲,田事成,每石官給止四十貫,而半是告、牒,民持之而不得售,六郡騷然。所遣劉良貴、陳時、趙與時、廖邦傑、成公策等推賞有差。邦傑之在常州,害民特甚,民至有本無田而以歸並抑買自經者。分置莊官催租,州縣督莊官及時交收運發。



 五年,選官充官田所分司,平江、嘉興,安吉各一員,常州、江陰、鎮江共一員,凡公田事悉以委之。是歲七月,彗見於東方。下詔求言,京學生蕭規、葉李等三學六館皆上封章;前秘書監高斯得亦應詔馳驛
 上封事,力陳買田之失人心、致天變;謝枋得校文江東運司,方山京校文天府,皆指陳得失。未幾,蕭規等真決黥隸,枋得、山京相繼被劾,斯得雖予郡,尋罷之。



 咸淳三年,京師糴貴,勒平江、嘉興上戶運米入京,鞭笞囚系,死於非命者十七八。太常寺簿陸逵謂:買田本以免和糴,今勒其運米,害甚於前。似道怒,出逵知臺州,未至,怖死。四年,以差置莊官弊甚,盡罷之。令諸郡公租以三千石為一莊,聽民於分司承佃,盜易者以盜賣官田論。其租
 於先減二分上更減一分。德祐元年三月,詔:「公田最為民害,稔怨召禍,十有餘年。自今並給田主,令率其租戶為兵。」而宋祚訖矣。



\end{pinyinscope}