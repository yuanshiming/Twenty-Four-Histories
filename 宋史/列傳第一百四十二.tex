\article{列傳第一百四十二}

\begin{pinyinscope}

 陳俊卿虞允文辛次膺



 陳俊卿,字應求,興化人。幼莊重,不妄言笑。父死,執喪如成人。紹興八年,登進士第,授泉州觀察推官。服勤職業,同僚宴集,恆謝不往。一日,郡中失火,守汪藻走視之,諸
 掾屬方飲某所,俊卿輿卒亦假之行,於是例以後至被詰,俊卿唯唯摧謝。已而知其實,問故,俊卿曰:「某不能止同僚之行,又資其僕,安得為無過。時公方盛怒,其忍幸自解,重人之罪乎?」藻嘆服,以為不可及。



 秩滿,秦檜當國,察其不附已,以為南外睦宗院教授。尋添通判南劍州,未上而檜死,乃以校書郎召。孝宗時為普安郡王,高宗命擇端厚靜重者輔導之,除著作佐郎兼王府教授。講經輒寓規戒,正色特立。王好鞠戲,因誦韓愈諫張建封
 書以諷,王敬納之。



 累遷監察御史、殿中侍御史。首言:「人主以兼聽為美,必本至公;人臣以不欺為忠,必達大體。御下之道,恩威並施,抑驕將,作士氣,則紀綱正而號令行矣。」遂劾韓仲通本以獄事附檜,冤陷無辜,檜黨盡逐而仲通獨全;劉寶總戎京口,恣掊克,且拒命不分戍;二人遂抵罪。湯思退專政,俊卿曰:「冬日無雲而雷,宰相上不當天心,下不厭人望。」詔罷思退。



 時災異數見,金人侵軼之勢已形。俊卿乃疏言:「張浚忠藎,白首不渝,竊聞讒
 言其陰有異志。夫浚之得人心、伏士論,為其忠義有素。反是,則人將去之,誰復與為變乎?」疏入,未報,因請對,力言之,上始悟。數月,以浚守建康。又言:「內侍張去為陰沮用兵,且陳避敵計,搖成算,請按軍法。」上曰:「卿可謂仁者之勇。」除權兵部侍郎。



 金主亮渡淮,俊卿受詔整浙西水軍,李寶因之遂有膠西之捷。亮死,詔俊卿治淮東堡砦屯田,所過安輯流亡。金主褒新立,申舊好,廷臣多附和議。俊卿奏:「和戎本非得已,若以得故疆為實利,得之未
 必能守,是亦虛文而已。今不若先正名,名正則國威強,歲幣可損。」因陳選將練兵、屯田減租之策,擇文臣有膽略者為參佐,俾察軍政、習戎務以儲將材。



 孝宗受禪,言:「為國之要有三:用人、賞功、罰罪,所以行之者至公而已,願留聖意。」遷中書舍人。時孝宗志在興復,方以閫外事屬張浚。以俊卿忠義,沉靖有謀,以本職充江、淮宣撫判官兼權建康府事。奏曰:「吳璘得孤軍深入,敵悉眾拒戰,久不決,危道也。兩淮事勢已急,盍分遣舟師直搗山東,彼
 必還師自救,而璘得乘勝定關中。我及其未至,潰其腹心,此不世之功也。」會主和議方堅,詔璘班師,亦召俊卿。奏陳十事:定規模,振紀綱,勵風俗,明賞罰,重名器,遵祖宗之法,蠲無名之賦。



 隆興初元,建都督府,俊卿除禮部侍郎參贊軍事。張浚初謀大舉北伐,俊卿以為未可。會諜報敵聚糧邊地,諸將以為秋必至,宜先其未動舉兵,浚乃請於朝出師。已而邵宏淵果以兵潰,俊卿退保揚州。主和議者幸其敗,橫議搖之。浚上疏待罪,俊卿亦乞
 從坐,詔貶兩秩。諫臣尹穡附思退,議罷浚都督,改宣撫使治揚州。俊卿奏:「浚果不可用,別屬賢將;若欲責其後效,降官示罰,古法也。今削都督重權,置揚州死地,如有奏請,臺諫沮之,人情解體,尚何後效之圖?議者但知惡浚而欲殺之,不復為宗社計。願下詔戒中外協濟,使浚自效。」疏再上,上悟,即命浚都督,且召為相,卒為思退、穡所擠,遣視師江、淮。俊卿累章請罪,以寶文閣待制知泉州,請祠,提舉太平興國宮。



 思退既竄,太學諸生伏闕下
 乞召俊卿。乾道元年,入對,上勞撫之,因極論朋黨之弊。除吏部侍郎、同修國史。論人才當以氣節為主,氣節者,小有過當容之;邪佞者,甚有才當察之。錢端禮起戚里為參政,窺相位甚急,館閣之士上疏斥之。端禮遣客密告俊卿,己即相,當引共政。深拒不聽。翌日,進讀《寶訓》,適及外戚,因言:「本朝家法,外戚不預政,有深意,陛下宜謹守。」上首肯,端禮憾之。知建康府。逾年,授吏部尚書。



 時上未能屏鞠戲,將游獵白石。俊卿引漢桓靈、唐敬穆及司
 馬相如之言力以為戒。上喜曰:「備見忠讜,朕決意用卿矣。朕在藩邸,知卿為忠臣。」後卿拜謝。



 受詔館金使,遂拜同知樞密院事。時曾覿、龍大淵怙舊恩,竊威福,士大夫頗出其門。及俊卿館伴,大淵副之,公見外,不交一語,大淵納謁,亦謝不接。洪邁白俊卿:「人言鄭聞除右史,某當除某官,信乎?」詰所從,邁以淵、覿告。具以邁語質於上,上曰:「朕曷嘗謀及此輩,必竊聽得之。」有旨出淵、覿,中外稱快。



 金移文邊吏,取前所俘。俊卿請報以「誓書云:俘虜叛
 亡是兩事,俘虜發已多,叛亡不應遣。且本朝兩淮民,上國俘虜亡慮數萬,本朝未嘗以為言,恐壞和議,使兩境民不安。或至交兵,則屈直勝負有在矣。」



 鎮江軍帥戚方刻削軍士,俊卿奏:「內臣中有主方者,當並懲之。」即詔罷方,以內侍陳瑤、李宗回付大理究臟狀。十一月,當郊而雷,上內出手詔,戒飭大臣,葉顒、魏杞坐罷。俊卿參知政事。時四明獻銀礦,將召冶工即禁中鍛之。俊卿奏:「不務帝王之大,而屑屑有司之細,恐為有識所窺。」從官梁克
 家、莫濟俱求補外,俊卿奏:「二人皆賢,其去可惜。」於是劾奏洪邁奸險讒佞,不宜在左右,罷之。減福建鈔鹽,罷江西和糴、廣西折米鹽錢,蠲諸道宿逋金谷錢帛以巨萬計,於是政事稍歸中書矣。



 龍大淵死,上憐曾覿,欲召之。俊卿曰:「自出此兩人,中外莫不稱頌。今復召,必大失天下望。臣請先罷。」遂不召。殿前指揮使王琪被旨按視兩淮城壁還,薦和州教授劉甄夫,得召。俊卿言:「琪薦兵將官乃其職,教官有才,何預琪事。」會揚州奏琪傳旨增築
 城已訖事,俊卿請於上,未嘗有是命。俊卿曰:「若詐傳上旨,非小故。」奏言:「人主萬幾,豈能盡防閑,所恃者紀綱、號令、賞罰耳。不誅琪,何所不為。」琪削秩罷官。



 先是,禁中密旨直下諸軍,宰相多不預聞,內官張方事覺,俊卿奏:「自今百司承御筆處分事,須奏審方行。」從之。既而以內諸司不樂,收前命。俊卿言:「張方、王琪事,聖斷已明,忽諭臣曰:『禁中取一飲一食,必待申審,豈不留滯。』臣所慮者,命令之大,如三衙發兵,戶部取財,豈為宮禁細微事。臣等
 備數,出內陛下命令耳。凡奏審欲取決陛下,非臣欲專之,且非新條,申舊制耳。已行復收,中外惶惑,恐小人以疑似激聖怒。」上曰:「朕豈以小人言疑卿等耶?」



 同知樞密院事劉珙進對,爭辨激切,忤旨,既退,手詔除珙端明殿學士,奉外祠。俊卿即藏去,密具奏:「前日奏札,臣實草定,以為有罪,臣當先罷。珙之除命,未敢奉詔。陛下即位以來,納諫諍,體大臣,皆盛德事。今珙以小事獲罪,臣恐自此大臣皆阿順持祿,非國家福。」上色悔久之,命珙帥江
 西。俊卿退自劾,上手札留之,且曰:「卿雖百請,朕必不從。」



 四年十月,制授尚書右僕射、同中書門下平章事兼樞密使。俊卿以用人為己任,所除吏皆一時選,獎廉退,抑奔競。或才可用,資歷淺,密薦於上,未嘗語人。每接朝士及牧守自遠至,必問以時政得失,人才賢否。



 虞允文宣撫四川,俊卿薦其才堪相。五年正月,上召允文為樞密使,至則以為右相,俊卿為左相。允文建議遣使金以陵寢為請,俊卿面陳,復手疏以為未可。上御孤矢,弦激致
 目眚,六月始御便殿。俊卿疏曰:「陛下經月不御外朝,口語籍籍,皆輔相無狀,不能先事開陳,虧損聖德。陛下憂勤恭儉,清靜寡欲,前代英主所不能免者皆屏絕,顧於騎射之末猶未能忘。臣知非樂此,志圖恢復,故俯而從事,以閱武備,激士氣耳。願陛下任智謀,明賞罰,恢信義,則英聲義烈,不越尊俎,固已震懾敵人於萬里之遠,豈待區區騎射於百步間哉。陛下一身,宗社生靈之休戚系焉,願以今日之事,永為後戒。」



 曾覿官滿當代,俊卿預
 請處以浙東總管。上曰:「覿意似不欲為此官。」俊卿曰:「前此陛下去二人,公論甚愜。願捐私恩,伸公議。」覿怏怏而去。樞密承旨張說為親戚求官,憚俊卿不敢言,會在告,請於允文,得之。俊卿聞敕已出,語吏留之。說皇恐來謝,允文亦愧,猶為之請,俊卿竟不與,說深憾之。吏部尚書汪應辰與允文議事不合,求去,俊卿數奏應辰剛毅正直,可為執政。上初然之,後竟出應辰守平江。自是上意向允文,而俊卿亦數求去。



 明年,允文復申陵寢之議,上
 手札諭俊卿,俊卿奏:「陛下痛念祖宗,思復故疆,臣雖疲駑,豈不知激昂仰贊聖謨,然於大事欲計其萬全,俟一二年間,吾之事力稍充乃可,不敢迎合意指誤國事。」即杜門請去,以觀文殿大學士帥福州。陛辭,猶勸上遠佞親賢,修政攘敵,泛使未可輕遣。既去,允文卒遣使,終不得要領。曾覿亦召還,建節鉞,躋保傅,而士大夫莫敢言。



 俊卿至福州,政尚寬厚,嚴於治盜,海道晏清,以功進秩。轉運判官陳峴建議改行鈔鹽法,俊卿移書宰執,極言
 福建鹽法與淮、浙異,遂不果行。明年,請祠,提舉洞霄宮。歸第,弊屋數楹,怡然不介意。



 淳熙二年,再命知福州。累章告歸,除特進,起判建康府兼江東安撫。召對垂拱殿,命坐賜茶,因從容言曰:「將帥當由公選,臣聞諸將多以賄得。曾覿、王抃招權納賄,進人皆以中批行之。臟吏已經結勘,而內批改正,將何所勸懲?」上曰:「卿言甚當。」朝辭,奏曰:「去國十年,見都城穀賤人安,惟士大夫風俗大變。」上曰:「何也?」俊卿曰:「向士大夫奔覿、抃之門,十纏一二,尚
 畏人知,今則公然趨附已七八,不復顧忌矣。人材進退由私門,大非朝廷美事。」上曰:「抃則不敢。覿雖時或有請,朕多抑之,自今不復從矣。」俊卿曰:「此曹聲勢既長,侍從、臺諫多出其門,毋敢為陛下言,臣恐壞朝廷紀綱,廢有司法度,敗天下風俗,累陛下聖德。」命二府飲餞浙江亭。



 俊卿去建康十五年,父老喜其再來。為政寬簡,罷無名之賦。時御前多行「白札」,用左右私人持送,俊卿奏非便,上手札獎諭。除少保,判建康府如故。八年上章告老,以少
 師、魏國公致仕。十三年十一月薨,年七十四。方屬疾,手書示諸子云:「遺表止謝聖恩,勿祈恩澤及功德,勿請謚樹碑。」上聞嗟悼,輟視朝,贈太保,命本路轉運司給葬事,賜謚正獻。



 俊卿孝友忠敬,得於天資,清嚴好禮,終日無惰容。平居恂恂若不能言,而在朝廷正色危論,分別邪正,斥權勢無顧避。凡所奏請,關治亂安危之大者。雅善汪應辰、李燾,尤敬朱熹,屢嘗論薦。其薨也,熹不遠千里往哭之,又狀其行。有集二十卷。



 子五人,宓有志於學,終
 承奉郎,朱熹為銘其墓。宓自有傳。



 虞允文,字彬甫,隆州仁壽人。父祺,登政和進士第,仕至太常博士、潼川路轉運判官。允文六歲誦《九經》,七歲能屬文。以父任入官。丁母憂,哀毀骨立。既葬,朝夕哭墓側,墓有枯桑,兩烏來巢。念父之鰥且疾,七年不調,跬步不忍離左右。父死,紹興二十三年始登進士第,通判彭州,權知黎州、渠州。



 秦檜當國,蜀士多屏棄。檜死,高宗欲收用之,中書舍人趙達首薦允文,召對,謂人君必畏天,必
 安民,必法祖宗。又論士風之弊,以文章進必抑其輕浮,以言語進必黜其巧偽,以政事進必去其苛刻,庶可任重致遠。且極論四川財賦科納之弊。上嘉納之。



 除秘書丞,累遷禮部郎官。金主亮修汴,已有南侵意。王綸還,言敵恭順和好。湯思退再拜賀,置邊備不問。及金使施宜生頗洩敵情,張燾密奏之。亮又隱畫工圖臨安湖山以歸。亮賦詩,情益露。允文上疏言:「金必敗盟,兵出有五道,願詔大臣豫思備御。」時三十年正月也。十月,借工部尚
 書充賀正使,與館伴賓射,一發破的,眾驚異之。允文見運糧造舟者多,辭歸,亮曰:「我將看花洛陽。」允文還,奏所見及亮語,申言淮、海之備。



 除中書舍人、直學士院。三衙管軍以宦寺充承受,允文言:「自古人主大權,不移於奸臣,則落於近幸。秦檜盜權十有八年,檜死,權歸陛下。邇來三衙交結中官,宣和、明受厥鑒未遠。」上大悟,立罷之。



 金使王全、高景山來賀生辰,口傳亮悖慢語,欲得淮南地,索將相大臣議事。於是召三衙大將趙密等議舉兵,
 侍從、臺諫集議。宰臣陳康伯傳上旨:「今日更不問和與守,直問戰當如何。」遣成閔為京、湖制置使,將禁衛五萬御襄、漢上流。允文曰:「兵來不除道,敵為虛聲以分我兵,成其出淮奸謀爾。」不聽,卒遣閔。七月,金主亮徙汴,允文復語康伯:「閔軍約程在江、池,宜令到池者駐池,到江者駐江。若敵兵出上流,則荊湖之軍捍於前,江、池之軍援於後;若出淮西,則池之軍出巢縣,江州軍出無為,可為淮西援,是一軍而兩用之。」康伯然其說,而閔軍竟屯武
 昌。



 九月,金主命李通為大都督,造浮梁於淮水上。金主自將,兵號百萬,氈帳相望,鉦鼓之聲不絕。十月,自渦口渡淮。先是,劉錡措置淮東,王權措置淮西。至是,權首棄廬州,錡亦回揚州,中外震恐。上欲航海,陳康伯力贊親征。是月戊午,樞臣葉義問督江、淮軍,允文參謀軍事。權又自和州遁歸,錡回鎮江,盡失兩淮矣。



 十一月壬申,金主率大軍臨採石,而別以兵爭瓜洲。朝命成閔代錡、李顯忠代權,錡、權皆召。義問被旨,命允文往蕪湖趣顯忠
 交權軍,且犒師採石,時權軍猶在採石。丙子,允文至採石,權已去,顯忠未來,敵騎充斥。我師三五星散,解鞍束甲坐道旁,皆權敗兵也。允文謂坐待顯忠則誤國事,遂立招諸將,勉以忠義,曰:「金帛、告命皆在此,待有功。」眾曰:「今既有主,請死戰。」或曰:「公受命犒師,不受命督戰,他人壞之,公任其咎乎?」允文叱之曰:「危及社稷,吾將安避?」



 至江濱,見江北已築高臺,對植絳旗二、繡旗二,中建黃屋,亮踞坐其下。諜者言,前一日刑白黑馬祭天,與眾盟,以
 明日濟江,晨炊玉麟堂,先濟者予黃金一兩。時敵兵實四十萬,馬倍之,宋軍才一萬八千。允文乃命諸將列大陣不動,分戈船為五,其二並東西岸而行,其一駐中流,藏精兵待戰,其二藏小港,備不測。部分甫畢,敵已大呼,亮操小紅旗麾數百艘絕江而來,瞬息,抵南岸者七十艘,直薄宋軍,軍小卻。允文入陣中,撫時俊之背曰:「汝膽略聞四方,立陣後則兒女子爾。」俊即揮雙刀出,士殊死戰。中流官軍亦以海鰍船沖敵,舟皆平沉,敵半死半戰,
 日暮未退。會有潰軍自光州至,允文授以旗鼓,從山後轉出,敵疑援兵至,始遁。又命勁弓尾擊追射,大敗之,殭尸凡四千餘,殺萬戶二人,俘千戶五人及生女真五百餘人。敵兵不死於江者,亮悉敲殺之,怒其不出江也。以捷聞,犒將士,謂之曰:「敵今敗,明必復來。」夜半,部分諸將,分海舟縋上流,別遣兵截楊林口。丁丑,果至,因夾擊之,復大戰,焚其舟三百,始遁去,再以捷聞。既而敵遣偽詔來諭王權,似有宿約。允文曰:「此反間也。」仍復書言:「權
 已置典憲,新將李世輔也,願一戰以決雌雄。」亮得書大怒,遂焚龍鳳車,斬梁漢臣及造舟者二人,乃趨瓜洲。漢臣,教亮濟江者也。



 顯忠至自蕪湖,允文語之曰:「敵入揚州,必與瓜洲兵合,京口無備,我當往,公能分兵相助乎?」顯忠分李捧軍萬六千往京口,葉義問亦命楊存中將所部來會。允文還建康,即上疏言:「敵敗於採石,將徼幸於瓜洲。今我精兵聚京口,持重待之,可一戰而勝。乞少緩六飛之發。」



 甲申,至京口。敵屯重兵滁河,造三閘儲水,
 深數尺,塞瓜洲口。時楊存中、成閔、邵宏淵諸軍皆聚京口,不下二十萬,惟海鰍船不滿百,戈船半之。允文謂遇風則使戰船,無風則使戰艦,數少恐不足用。遂聚材治鐵,改修馬船為戰艦,且借之平江,命張深守滁河口,扼大江之沖,以苗定駐下蜀為援。庚寅,亮至瓜洲,允文與存中臨江按試,命戰士踏車船中流上下,三周金山,回轉如飛,敵持滿以待,相顧駭愕。亮笑曰:「紙船耳。」一將跪奏:南軍有備,未可輕,願駐揚州,徐圖進取。亮怒,欲斬之,
 哀謝良久,杖之五十。乙未,亮為其下所殺。



 初,亮在瓜洲,聞李寶由海道入膠西,成閔諸軍方順流而下,亮愈怒。還揚州,召諸將約三日濟江,否則盡殺之。諸將謀曰:「進有渰殺之禍,退有敲殺之憂,奈何?」有萬戴者曰:「殺郎主,與南宋通和歸鄉則生矣。」眾曰:「諾。」亮有紫茸細軍,不臨陣,恆以自衛,眾患之,有蕭遮巴者紿之曰:「淮東子女玉帛皆聚海陵。」且嗾使往,細軍去而亮死。



 丙申,敵人退屯三十里,遣使議和。己亥,奏聞。召入對,上慰藉嘉嘆,謂陳
 俊卿曰:「虞允文公忠出天性,朕之裴度也。」詔免扈從,往兩淮措置。允文至鎮江,奏收兩淮三策,不報。



 明年正月,上至建康。尋議回鑾,詔以楊存中充江淮、荊襄路宣撫使,允文副之。給、舍繳存中除命,於是允文充川陜宣諭使。陛辭,言:「金亮既誅,新主初立,彼國方亂,天相我恢復也。和則海內氣沮,戰則海內氣伸。」上以為然。允文至蜀,與大將吳璘議經略中原,璘進取鳳翔,復鞏州。金治兵爭陜西新復州郡,蜀士欲棄之,允文持不可。



 孝宗受禪,
 朝臣有言西事者,謂官軍進討,東不可過寶雞,北不可過德順,且欲用忠義人守新復州郡,官軍退守蜀口。允文爭之不得,吳璘遂歸河池,蓋用參知政事史浩議,欲盡棄陜西,臺諫袁季、任古附和其說。允文再上疏,大略言:「恢復莫先於陜西,陜西五路新復州縣又系於德順之存亡,一旦棄之,則窺蜀之路愈多,西和、階、成,利害至重。」前後凡十五疏,且移書陳康伯,康伯牽於同列,不能回也。上將召允文問陜西事,執政忌其來,以顯謨閣直
 學士知夔州,尋又命奏事。



 隆興元年入對,史浩既素主棄地,及拜相,亟行之,且親為詔,有曰:「棄雞肋之無多,免鋃心之未已。」允文入對言:「今日有八可戰。」上問及棄地,允文以笏畫地,陳其利害。上曰:「此史浩誤朕。」以敷文閣待制知太平州,尋除兵部尚書、湖北京西宣撫使,改制置使。



 時朝廷遣盧仲賢使金議和,湯思退又欲棄唐、鄧、海、泗,手詔謂唐、鄧非險要,可置度外,允文五上疏力爭。思退怒,即奏曰:「此皆以利害不切於己,大言誤國,以邀
 美名。宗社大事,豈同戲劇。」上意遂定。思退陽請召允文,實欲去之也。允文上印,猶以四州不可棄為請,乞致仕。詔以顯謨閣學士知平江府。思退竟決和議,割唐、鄧。



 二年,金兵復至,思退貶,上悔不用允文言。陳俊卿亦薦允文堪大用,除端明殿學士、同簽書樞密院事。



 乾道元年,拜參知政事兼知樞密院事。是秋,金遣完顏仲有所議,偃蹇不敬,允文請斬之,廷有異論,不果。會錢端禮受李宏玉帶,事連允文,為御史章服所論,罷政,奉祠西歸。



 三
 年二月,召至闕,除知樞密院事兼參知政事。吳璘卒,議擇代,上諭允文曰:「吳璘既卒,汪應辰恐不習軍事,無以易卿。凡事不宜效張浚迂闊,軍前事,卿一一親臨之。」即拜資政殿大學士、四川宣撫使,尋詔依舊知樞密院事。歸蜀一月,召至闕,不數月復使蜀。太上賜御書《聖主得賢臣頌》,上又為之制跋,陛辭,復以所御雙履及甲冑賜焉。



 過郢,奏築黃鷹山城。過襄陽,奏修府城。八月至漢中,又往沔陽。九月,至益昌。先被手詔戒九事,洎至蜀,悉奉
 而行,尤以軍政為急。又奏閱實諸軍,第其壯怯為三,上備戰,中下備輜重,老者少者不預。汰兵凡萬人,減緡錢四百萬。汰去兵有勞績者,置員闕處之。興、洋義士,民兵也,紹興初以七萬計,大散之戰,將不授甲,驅之先官軍,死亡略盡。命利帥晁公武核實,得二萬三千九百餘人。又得陜西弓箭手法,參紹興制為一書,俾將吏守之。以馬政付張松,奏依舊制分茶馬為川、秦司。



 初在樞府,蕭遮巴以刷軍中人為言,允文嘗奏諭三衙撫存之。至是,
 金、洋、興元歸正人二萬,遮道訴系縲之苦,允文分給官田,俾咸振業。欲結敵將姜挺、白沂,遵御札募鞏人王嗣祖結外蕃以圖金人,又得蕃僧六彪者偕往,竟無成說。時邛、蜀十四郡告饑,荒政凡六十五事,劍倅獻羨錢五萬,卻之。



 五年八月,拜右僕射、同中書門下平章事兼樞密使。允文多薦知名士,如洪適、汪應辰。及為相,籍人才為三等,有所見聞即記之,號《材館錄》。凡所舉,上皆收用,如胡銓、周必大、王十朋、趙汝愚、晁公武、李燾其尤章明
 者也。上以兵冗財匱為憂,允文與陳俊卿議革三衙雜役,汰冗籍,三軍無怨言。



 六年,陳俊卿以奏留龔茂良忤上意,上震怒甚,俊卿待命浙江亭,兩日不報。允文請對,極論體貌之道,疊拜榻前,遂命判福州。



 詔以範成大為祈請使,為陵寢故。金不從,且諜報欲以三十萬騎奉遷陵寢來歸,中外洶洶,荊、襄將帥皆請增戍。允文謂:「金方懲亮,決不輕動,不過以虛聲撼我耳。」遂奏止之。朝論紛然,允文屹不動,敵卒無他。



 自莊文太子斃,儲位未定。允
 文上疏,且屢懇陳。七年正月,上兩宮尊號,議始定,下詔皇第三子恭王惇立為皇太子,皇子愷以雄武、保寧軍節度使判寧國府。皇太子尋尹臨安。侍衛馬軍司牧地舊在臨安,允文謂地狹不利芻牧,請令就牧鎮江,緩急用騎過江便。三軍有怨語,其後言者以此為言。



 胡銓以臺評去,允文奏留之經筵。銓薦朱熹,上問允文識熹否?允文謂熹不在程頤下,遂召熹,熹不至。檢鼓院以六條抑上書人,允文力言不可,從之。



 會慶節,金使烏林答天
 錫入見,金主婿也,驕倨甚,固請上降榻問金主起居,上不許,天錫跪不起,侍臣錯愕失措。允文請大駕還禁中,且諭之曰:「大駕既興,難再御殿,使人來且隨班上壽。」金使慚而退。



 上以僕射名不正,改為左、右丞相。八年二月,授允文特進、左丞相兼樞密使,梁克家為右丞相。允文嘗舉克家自代,上不許。是月,以病乞解機政,又薦克家靖重有宰相器,至是始同相,手詔付允文曰:「朕方欲武臣為樞密,曹勛如何?」允文謂勛人品卑凡,不可用。既而
 以張說簽書樞密院事,右正言王希呂與臺官交劾之。上怒希呂甚,手詔「與遠惡監當。」允文繳回,上益怒。梁克家曰:「希呂論張說,臺綱也,左相救希呂,國體也。」上怒稍解,卒薄希呂之罰。



 四月,御史蕭之敏劾允文,允文上章待罪。上過德壽宮,太上曰:「採石之功,之敏在何許?毋聽其去。」上為出之敏,且書扇制詩以留之。允文言之敏端方,請召歸以闢言路。上謂其言寬厚,命曾懷書之《時政記》。



 上命選諫官,允文以李彥穎、林光朝、王質對,三人皆
 鯁亮,又以文學推重於時,故薦之,久不報。曾覿薦一人,賜第,擢諫議大夫。允文、克家爭之,不從。允文力求去,授少保、武安軍節度使、四川宣撫使,進封雍國公。陛辭,上諭以進取之方,期以某日會河南。允文言:「異時戒內外不相應。」上曰:「若西師出而朕遲回,即朕負卿;若朕已動而卿遲回,即卿負朕。」上御正衙,酌酒賦詩以遣之,且賜家廟祭器。



 九年至蜀。大軍月給米一石五斗,不足贍其家,允文捐宣司錢三十萬易米,計口增給。立戶馬七條,
 括民馬,奏選良家子以儲戰用。初,北界有寇鄰者,擁眾數萬在商、虢間,允文秉政日納款,迨至蜀,復遣人致書允文,不報,羈縻之而已。既而鄰謀覺,金密遣人捕之。葉衡奏聞,允文上疏自辨,因請納祿,不報。



 上嘗謂允文曰:「丙午之恥,當與丞相共雪之。」又曰:「朕惟功業不如唐太宗,富庶不如漢文、景。」故允文許上以恢復。使蜀一歲,無進兵期,上賜密詔趣之,允文言軍需未備,上不樂。



 淳熙元年薨。後四年,上幸白石大閱,見軍皆少壯,謂輔臣曰:「
 虞允文行沙汰之效也。」尋詔贈太傅,賜謚忠肅。



 允文姿雄偉,長六尺四寸,慷慨磊落有大志,而言動有則度,人望而知為任重之器。早以文學致身臺閣,晚際時艱,出入將相垂二十年,孜孜忠勤無二焉。嘗注《唐書》、《五代史》,藏於家。有詩文十卷,《經筵春秋講義》三卷,《奏議》二十二卷,《內外志》十五卷,行於世。



 子三人:公亮、公著、杭孫。孫八人,皆好修,唯剛簡最知名,嘉定中,召不至,終利路提點刑獄。



 辛次膺,字起季,萊州人。幼孤,從母依外氏王聖美於丹徒。俊慧力學,日誦千言。甫冠,登政和二年進士第,歷官為單父丞。



 值山東亂,舉室南渡。屬閩寇範汝為陷建州,宰相呂頤浩以次膺宰浦城,遏賊沖。比至,寇黨熊志寧已焚其邑。於是披荊棘,坐瓦礫中,安輯吏民,料丁壯,治器械,厄險阻,號令不煩,邑民便之。數月,韓世忠破賊,復建州,除審計司。餘黨範黑龍破鄰邑,閩帥張守檄次膺,俟賊平而後行。乃募鄉兵習強弩,賊至,與之夾水而陣,
 矢齊發,賊奔潰,生致首領五人,餘悉宥之。



 用參政孟庾薦,召對,奏用人貴於務實,施令在於必行。遷駕部。願敕郡邑省耕薄征,務農抑末。又奏:「中原之人,棄墳墓生業,從巡江左,饑寒殞僕。願加存拊,可以堅中原徯後之心。」遷吏部郎、湖北運判,中途召還,見高宗於建康行宮,首言救世之弊,上稱善,敕以所奏榜朝堂。



 擢右正言。奏:「願閱兵將,親簡拔,攬恩威之柄,使人人知朝廷之尊。左右近習,久則干政,願杜其漸。兵連不解,十年於茲。一歲用
 錢三十萬、米四百萬石,諸路常賦僅足支其半,餘悉取諸民。乞罷不急之務,節姑息之澤,省冗官,汰心耎兵。」



 韓世忠男直秘閣,次膺奏曰:「攻城野戰,世忠功也,其子何與?石渠、東觀,圖書府也,武功何與?幸門一啟,援例者眾。」又奏:「今主議者見小利忽大計,偏師偶勝,遽思進討,便謂攻為有餘;警奏稍聞,首陳退舍,便謂守為不足。願嚴紀律,謹烽燧,明間探。」上皆信納。聞韓世忠將自楚州移軍鎮江,復陳可慮者五。王倫使北請和,次膺言:「宣和海上
 之約,靖康城下之盟,口血未幹,兵隨其後。今日之事當識其詐。」



 時秦檜在政府,為其妻兄王仲薿敘兩官。次膺劾仲薿奴事朱勉,投拜金酋,罪在不赦。又劾知撫州王喚違法佃官田,不輸租。其父仲山,先知撫州,屈膝金人,喚繼其後,何顏見吏民?喚,檜之妻兄也。章留中。次膺再論之曰:「近臣奏二人,繼聞追寢除命,是皆檜容私營救,陛下曲從其欲,國之紀綱,臣之責任,一切廢格。借使貴連宮掖,親如肺附,寵任非宜,臣亦得論之,而大臣之姻
 婭,乃不得繩之耶?望陛下奮乾剛之威,戒蒙蔽之漸。」



 求去,除直秘閣、湖南提刑。先是,湖南賊龍淵、李朝擁眾數萬,據衡之茶陵,檜匿不奏,乃以見闕處次膺。陛辭,上曰:「卿以將母為請,朕不得留。湖湘風物甚佳,且無盜賊,職名異恩,卒歲當召。」既抵長沙,賊勢方張,戍將抽回,始悟檜欲陷之。即單車趨茶陵,擒賊驍將戮之,募賊黨毛義、龍麟等,繼榜諭以朝廷抽回戍將,務欲招安,宜亟降,待以不死。龍淵、李朝相繼降,仍請料精銳,可得禁旅萬餘。
 次鷹笑曰:「是皆吾民,正當棄兵甲,持鋤櫌,趣令復業。」奏茶陵為軍。



 金好成,赦書至衡陽,次膺極陳其詐,略曰:「臣昨在諫列,嘗數論金人變詐無常,願陛下為宗社生靈深慮。近觀邸報,樞密院編修官胡銓妄議和好,歷詆大臣,除名遠竄。已而得銓書槁,乃知朝廷遽欲屈己稱藩,臣未知其可。大臣懷奸固位,不恤國計,媕婀趨和,謬以為便,臣不知天下之人以為便乎?『父之仇不與共戴天,兄弟之仇不反兵』。棄仇釋怨,盡除前事,降萬乘之尊,以
 求說於敵,天下之人,果能遂亡怨痛以從陛下之志乎?」書奏,不報。金陷三京。



 次膺罷,奉祠。秦檜以其負重名,欲先移書,當稍收用,次膺笑而不答。閱十六年,貧益甚,亡毫發求於人。檜死,起知婺州,三日被召。至國門,以足疾求去。加秘閣修撰,還郡。再召見,歷言仇怨當國,老母幾委溝壑,因奏國本未立,上改容曰:「誰可?」次膺曰:「知子莫若父。」上稱善。擢權給事中。蔣璨權戶部侍郎,次膺駁璨不守正,事交結,出璨知平江。御史中丞湯鵬舉劾次膺
 假權報怨,除待制、宮觀。起知泉州,移福建帥。丁母憂,乞納祿。



 孝宗即位,手詔趣召。既至,奏:「陛下用賢必考核事功,勿以一人譽用之,一人毀去之,出令要無反汗,納善要知轉圜。練兵恤民,經理兩淮,使敵不能乘虛而入。」是日,除御史中丞。朝德壽宮,高宗一見,謂「惜間卿於強健時。」



 上將以春饗迎高宗詣延祥觀,幸玉津園。次膺奏:「欽宗服未終,方停策士,且金人嫚書甫至,意在交兵,矧原野間禁衛稀少,當過為之慮,兼一出費十數萬緡,曷若
 以資兵食。」時兩淮盡為荒野,次膺奏:「乞集遺甿歸業,借種牛,或令在屯兵從便耕種,此足兵良法。」至若成閔之貪饕,湯思退之朋附,葉義問之奸罔,皆以次論劾。每章疏一出,天下韙之。上方厲精政事,次膺每以名實為言,多所裨益,呼其官不名。



 隆興改元三月,同知樞密院事。符離之師,捷奏日聞,次膺手疏千言,乞持重。未幾,軍果潰。及見,上顏色不樂,奏言:「師潰而歸,張浚彈壓必無他,此上天大儆戒於陛下。」上嘆其先見。



 拜參知政事,以疾
 力祈免。且奏曰:「王十朋除侍史,雖上親擢,天下皆知臣嘗薦其賢。湯思退召將至,亦知臣嘗疏其奸。臣不引避,人其謂何?」除資政殿學士、提舉洞霄宮。陛辭,賜茶,甚惜其去。次膺奏:「臣與思退,理難同列。」上曰:「有謂湯思退可用者。」次膺奏:「今日之事,恐非思退能辨。思退固不足道,竊恐誤國家事。」乾道六年閏五月卒,年七十九。



 次膺孝友清介,立朝謇諤。仕宦五十年,無絲毫掛吏議。為政貴清靜,先德化,所至人稱其不煩。善屬文,尤工於詩。



 論曰:孝宗志恢復,特任張浚,俊卿斥奸黨,明公道,以為之佐。洎居中書,知無不為,言無不盡,蓋其立志一以先哲為法,非他相可擬也。允文許國之忠,炳如丹青。金庶人亮之南侵,其鋒甚銳,中外倚劉錡為長城,錡以病不克進師。允文儒臣,奮勇督戰,一舉而挫之,亮乃自斃。昔赤壁一勝而三國勢成,淮淝一勝而南北勢定。允文採石之功,宋事轉危為安,實系乎此。及其罷相鎮蜀,受命興復,克期而往,志雖未就,其能慷慨任重,豈易得哉?次
 膺力排群邪,無負言責,蒞政不煩,居約有守。晚再立朝,謇諤尤著,南渡直言之臣,宜為首稱焉。



\end{pinyinscope}