\article{卷十四列傳第八衡陽獻王昌 南康愍王曇朗 子方泰 方慶}

\begin{pinyinscope}

 衡陽獻王
 昌,字敬業,高祖第六子也。梁太清末,高祖南征李賁,命昌與宣后隨沈恪還吳興。及高祖東討侯景,昌與宣后、世祖並為景所囚。景平,拜長城國世子、吳興
 太守,時年十六。



 昌容貌偉麗,神情秀朗,雅性聰辯,明習政事。高祖遣陳郡謝哲、濟陽蔡景歷輔昌為郡,又遣吳郡杜之偉授昌以經書。昌讀書一覽便誦,明於義理,剖析如流。



 尋與高宗俱往荊州,梁元帝除員外散騎常侍。荊州陷,又與高宗俱遷關右,西魏以高祖故,甚禮之。



 高祖即位,頻遣使請高宗及昌,周人許之而未遣,及高祖崩,乃遣之。是時王琳梗於中流,昌未得還,居于安陸。王琳平後,天嘉元年二月,昌發自安陸,由魯山濟江,而巴
 陵王蕭沇等率百僚上表曰:臣聞宗子維城,隆周之懋軌,封建籓屏,有漢之弘規,是以卜世斯永,式資邢、衛,鼎命靈長,實賴河、楚。伏惟陛下神猷光大,聖德欽明,道高日月,德侔造化。



 往者王業惟始,天步方艱,參奉權謨,匡合義烈,威略外舉,神武內定,故以再康禹迹,大庇生民者矣。及聖武升遐,王師遠次,皇嗣夐隔,繼業靡歸,宗祧危殆,綴旒非喻。既而傳車言反,公卿定策,纂我洪基,光昭景運,民心有奉,園寢克寧,后來其蘇,復在茲日,物情
 天意,皎然可求。王琳逆命,逋誅歲久,今者連結犬羊,乘流縱釁,舟旗野陣,綿江蔽陸,兵疲民弊,杼軸用空,中外騷然,蕃籬罔固。乃旰食當朝,憑流授律,蒼兕既馳,長蛇自翦,廓清四表,澄滌八紘,雄圖遐舉,仁聲遠暢,德化所覃,風行草偃,故以功深於微禹,道大於惟堯,豈直社稷用寧,斯乃黔黎是賴。



 第六皇弟昌,近以妙年出質,提契寇手,偏隔關徼,旋踵末由。陛下天倫之愛既深,克讓之懷常切。伏以大德無私,至公有在,豈得徇匹夫之恒情,
 忘王業之大計。憲章故實,式遵典禮,欽若姬、漢,建樹賢戚。湘中地維形勝,控帶川阜,扞城之寄,匪親勿居,宜啟服衡、疑,兼崇徽飾。臣等參議,以昌為使持節、散騎常侍、都督湘州諸軍事、驃騎將軍、湘州牧,封衡陽郡王,邑五千戶,加給皁輪三望車,後部鼓吹一部,班劍二十人。啟可奉行。



 詔曰「可」。三月入境,詔令主書舍人緣道迎接。丙子,濟江,於中流船壞,以溺薨。



 四月庚寅,喪柩至京師,上親出臨哭。乃下詔曰:「夫寵章所以嘉德,禮數所以崇親,
 乃歷代之通規,固前王之令典。新除使持節、散騎常侍、都督湘州諸軍事、驃騎將軍、湘州牧衡陽王昌,明哲在躬,珪璋早秀,孝敬內湛,聰睿外宣。梁季艱虞,宗社顛墜,西京淪覆,陷身關隴。及鼎業初基,外蕃逆命,聘問斯阻,音介莫通,睠彼機橋,將鄰烏白。今者群公戮力,多難廓清,輕傳入郛,無勞假道。周朝敦其繼好,驂駕歸來,欣此朝聞,庶歡昏定。報施徒語,曾莫輔仁,人之云亡,殄悴斯在,奄焉薨殞,倍增傷悼。津門之慟空在,恆岫之切不追,
 靜言念之,心焉如割。宜隆懋典,以協徽猷。可贈侍中、假黃鉞、都督中外諸軍事、太宰、揚州牧。



 給東園溫明秘器,九旒鑾輅,黃屋左纛,武賁班劍百人,轀輬車,前後部羽葆鼓吹。



 葬送之儀,一依漢東平憲王、齊豫章文獻王故事。仍遣大司空持節迎護喪事,大鴻臚副其羽衛,殯送所須,隨由備辦。」謚曰獻。無子,世祖以第七皇子伯信為嗣。



 南康愍王曇朗,高祖母弟忠壯王休先之子也。休先少
 倜儻有大志,梁簡文之在東宮,深被知遇。太清中既納侯景,有事北方,乃使休先召募得千餘人,授文德主帥,頃之卒。高祖之有天下也,每稱休先曰:「此弟若存,河、洛不足定也。」梁敬帝即位,追贈侍中、使持節、驃騎將軍、南徐州刺史,封武康縣公,邑一千戶。



 高祖受禪,追贈侍中、車騎大將軍、司徒,封南康郡王,邑二千戶,謚曰忠壯。



 曇朗少孤,尤為高祖所愛,寵踰諸子。有膽力,善綏御。侯景平後,起家為著作佐郎。高祖北濟江,圍廣陵,宿預人東
 方光據鄉建義,乃遣曇朗與杜僧明自淮入泗應赴之。齊援大至,曇朗與僧明築壘抗禦。尋奉命班師,以宿預義軍三萬家濟江。



 高祖誅王僧辯,留曇朗鎮京口,知留府事。紹泰元年,除中書侍郎,監南徐州。



 二年,徐嗣徽、任約引齊寇攻逼京邑,尋而請和,求高祖子姪為質。時四方州郡並多未賓,京都虛弱,糧運不斷,在朝文武咸願與齊和親,高祖難之,而重違眾議,乃言於朝曰:「孤謬輔王室,而使蠻夷猾夏,不能戡殄,何所逃責。今在位諸賢,
 且欲息肩偃武,與齊和好,以靜邊疆,若違眾議,必謂孤惜子姪,今決遣曇朗,棄之寇庭。且齊人無信,窺窬不已,謂我浸弱,必當背盟。齊寇若來,諸君須為孤力鬥也。」高祖慮曇朗憚行,或奔竄東道,乃自率步騎往京口迎之,以曇朗還京師,仍使為質於齊。



 齊果背約,復遣蕭軌等隨嗣徽渡江,高祖與戰,大破之,虜蕭軌、東方老等。



 齊人請割地并入馬牛以贖之,高祖不許。及軌等誅,齊人亦害曇朗于晉陽,時年二十八。是時既與齊
 絕,弗之知也。高祖踐祚,猶以曇朗襲封南康郡王,奉忠壯王祀,禮秩一同皇子。天嘉二年,齊人結好,方始知之。世祖詔曰:「夫追遠慎終,抑聞前誥。南康王曇朗,明哲懋親,蕃維是屬,入質北齊,用紓時難。皇運兆興,未獲旋反,永言跂予,日夜不忘。齊使始至,凶問奄及,追懷痛悼,兼倍常情,宜隆寵數,以光恆序。可贈侍中、安東將軍、開府儀同三司、南徐州刺史,謚曰愍。」乃遣兼郎中令隨聘使江德藻、劉師知迎曇朗喪柩,以三年春至都。



 初,曇朗未質於齊,生子
 方泰、方慶。及將適齊,以二妾自隨,在北又生兩子:方華、方曠,亦同得還。



 方泰少粗獷,與諸惡少年群聚,遊逸無度,世祖以南康王故,特寬貰之。天嘉元年,詔曰:「南康王曇朗,出隔齊庭,反身莫測,國廟方脩,奠饗須主,可以長男方泰為南康世子,嗣南康王。」後聞曇朗薨,於是襲爵南康嗣王。尋為仁威將軍、丹陽尹,置佐史。太建四年,遷使持節、都督廣、衡、交、越、成、定、明、新、合、羅、德、宜、黃、利、安、建、石、崖十九州諸
 軍事、平越中郎將、廣州刺史。



 為政殘暴,為有司所奏,免官。尋起為仁威將軍,置佐史。六年,授持節、都督豫章郡諸軍事、豫章內史。在郡不修民事,秩滿之際,屢放部曲為劫,又縱火延燒邑居,因行暴掠,驅錄富人,徵求財賄。代至,又淹留不還。至都,詔以為宗正卿,將軍、佐史如故。未拜,為御史中丞宗元饒所劾,免官,以王還第。



 十一年,起為寧遠將軍,直殿省。尋加散騎常侍,量置佐史。其年八月,高宗幸大壯觀,因大閱武,命都督任忠領步騎十
 萬,陳於玄武湖,都督陳景領樓艦五百,出於瓜步江,高宗登玄武門觀,宴群臣以觀之。因幸樂遊苑,設絲竹會。仍重幸大壯觀,集眾軍振旅而還。是時方泰當從,啟稱所生母疾,不行,因與亡命楊鐘期等二十人,微服往民間,淫人妻,為州所錄。又率人仗抗拒,傷禁司,為有司所奏。



 上大怒,下方泰獄。方泰初但承行淫,不承拒格禁司,上曰不承則上刑,方泰乃投列承引。於是兼御史中丞徐君敷奏曰:「臣聞王者之心,匪漏網而私物,至治之本,
 無屈法而申慈。謹案南康王陳方泰宗屬雖遠,幸託葭莩,刺舉莫成,共治罕績。聖上弘以悔往,許其錄用,宮闈寄切,宿衛是尸。豈有金門旦啟,玉輿曉蹕,百司馳騖,千隊騰驤,憚此翼從之勞,亡興晨昏之請?翻以危冠淇上,袨服桑中,臣子之諐,莫斯為大,宜從霜簡,允置秋官。臣等參議,請依見事,解方泰所居官,下宗正削爵土。謹以白簡奏聞。」上可其奏。尋復本官爵。禎明初,遷侍中,將軍如故。



 三年,隋師濟江,方泰與忠武將軍南豫州刺史樊
 猛、左衛將軍蔣元遜領水軍於白下,往來斷遏江路。隋遣行軍元帥、長史高熲領船艦溯流當之,猛及元遜並降,方泰所部將士離散,乃棄船走。及臺城陷,與後主俱入關。隋大業中為掖令。



 方慶少清警,涉獵書傳。及長,有幹略。天嘉中,封臨汝縣侯。尋為給事中、太子洗馬,權兼宗正卿,直殿省。太建九年,出為輕車將軍、假節、都督定州諸軍事、定州刺史。秩滿,又為散騎常侍,兼宗正卿。至德二年,進號智武將軍、
 武州刺史。初,廣州刺史馬靖久居嶺表,大得人心,士馬彊盛,朝廷疑之。至是以方慶為仁威將軍、廣州刺史,以兵襲靖。靖誅,進號宣毅將軍。方慶性清謹,甚得民和。



 四年,進號雲麾將軍。



 禎明三年,隋師濟江東衡州刺史王勇遣高州刺史戴智烈將五百騎迎方慶,欲令承制總督征討諸軍事。是時隋行軍總管韋洸帥兵度嶺,宣隋文帝敕云:「若嶺南平定,留勇與豊州刺史鄭萬頃且依舊職。」方慶聞之,恐勇賣己,乃不從,率兵以拒智烈。智烈與
 戰,敗之,斬方慶於廣州,虜其妻子。



 王勇,太建中為晉陵太守,在職有能名。方慶之襲馬靖也,朝廷以勇為超武將軍、東衡州刺史,領始興內史,以為方慶聲勢。靖誅,以功封龍陽縣子。及隋軍臨江,詔授勇使持節、光勝將軍、總督衡、廣、交、桂、武等二十四州諸軍事、平越中郎將,仍入援。會京城陷、勇因移檄管內,徵兵據守,使其同產弟鄧暠將兵五千,頓于嶺上。又遣使迎方慶,慾假以為名,而自執兵要。及方慶敗績,虜其妻子,收其貲產,分賞
 將帥。又令其將王仲宣、曾孝武迎西衡州刺史衡陽王伯信,伯信懼,奔于清遠郡,孝武追殺之。是時韋洸兵已上嶺,豊州刺史鄭萬頃據州不受勇召,而高梁女子洗氏舉兵以應隋軍,攻陷傍郡,勇計無所出,乃以其眾降。行至荊州,道病卒,隋贈大將軍、宋州刺史,歸仁縣公。



 鄭萬頃,滎陽人,梁司州刺史紹叔之族子也。父旻,梁末入魏。萬頃通達有材幹,周武帝時為司城大夫,出為溫州刺史。至德中,與司馬消難來奔。尋拜散騎常侍、昭武將軍、豊
 州刺史。在州甚有惠政,吏民表請立碑,詔許焉。



 初,萬頃之在周,深被隋文帝知遇,及隋文踐祚,常思還北。及王勇之殺方慶,萬頃乃率州兵拒勇,遣使由間道降于隋軍。拜上儀同,尋卒。



 史臣曰:獻、愍二王,聯華霄漢,或壤子之暱,或猶子之寵,而機橋為阻,驂駕無由,有隔於休辰,終之以早世。悲夫!



\end{pinyinscope}