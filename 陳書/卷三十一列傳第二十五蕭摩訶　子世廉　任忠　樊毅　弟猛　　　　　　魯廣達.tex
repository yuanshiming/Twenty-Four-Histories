\article{卷三十一列傳第二十五蕭摩訶 子世廉 任忠 樊毅 弟猛      魯廣達}

\begin{pinyinscope}

 蕭摩訶,字元胤,蘭陵人也。祖靚,梁右將軍。父諒,梁始興郡丞。摩訶隨父之郡,年數歲而父卒,其姑夫蔡路養時
 在南康,乃收養之。稍長,果毅有勇力。侯景之亂,高祖赴援京師,路養起兵拒高祖,摩訶時年十三,單騎出戰,軍中莫有當者。及路養敗,摩訶歸于侯安都,安都遇之甚厚,自此常隸安都征討。及任約、徐嗣徽引齊兵為寇,高祖遣安都北拒齊軍於鐘山龍尾及北郊壇。安都謂摩訶曰:「卿驍勇有名,千聞不如一見。」摩訶對曰:「今日令公見矣。」及戰,安都墜馬被圍,摩訶獨騎大呼,直衝齊軍,齊軍披靡,因稍解去,安都乃免。天嘉初,除本縣令,以平留
 異、歐陽紇之功,累遷巴山太守。



 太建五年,眾軍北伐,摩訶隨都督吳明徹濟江攻秦郡。時齊遣大將尉破胡等率眾十萬來援,其前隊有「蒼頭」、「犀角」、「大力」之號,皆身長八尺,膂力絕倫,其鋒甚銳。又有西域胡,妙於弓矢,弦無虛發,眾軍尤憚之。及將戰,明徹謂摩訶曰:「若殪此胡,則彼軍奪氣,君有關、張之名,可斬顏良矣。」摩訶曰:「願示其形狀,當為公取之。」明徹乃召降人有識胡者,云胡著絳衣,樺皮裝弓,兩端骨弭。明徹遣人覘伺,知胡在陣,乃自
 酌酒以飲摩訶。摩訶飲訖,馳馬衝齊軍,胡挺身出陣前十餘步,彀弓未發,摩訶遙擲銑鋧,正中其額,應手而仆。齊軍「大力」十餘人出戰,摩訶又斬之,於是齊軍退走。以功授明毅將軍、員外散騎常侍,封廉平縣伯,邑五百戶。尋進爵為侯,轉太僕卿,餘如故。七年,又隨明徹進圍宿預,擊走齊將王康德,以功除晉熙太守。九年,明徹進軍呂梁,與齊人大戰,摩訶率七騎先入,手奪齊軍大旗,齊眾大潰。以功授持節、武毅將軍、譙州刺史。



 及周武帝滅
 齊,遣其將宇文忻率眾爭呂梁,戰於龍晦。時忻有精騎數千,摩訶領十二騎深入周軍,縱橫奮擊,斬馘甚眾。及周遣大將軍王軌來赴,結長圍連鎖於呂梁下流,斷大軍還路。摩訶謂明徹曰:「聞王軌始鎖下流,其兩頭築城,今尚未立,公若見遣擊之,彼必不敢相拒。水路未斷,賊勢不堅,彼城若立,則吾屬且為虜矣。」明徹乃奮髯曰:「搴旗陷陣,將軍事也;長算遠略,老夫事也。」摩訶失色而退。一旬之間,周兵益至,摩訶又請於明徹曰:「今求戰不得,
 進退無路,若潛軍突圍,未足為恥。願公率步卒,乘馬輿徐行,摩訶領鐵騎數千,驅馳前後,必當使公安達京邑。」明徹曰:「弟之此計,乃良圖也。然老夫受脤專征,不能戰勝攻取,今被圍逼蹙,慚置無地。且步軍既多,吾為總督,必須身居其後,相率兼行。



 弟馬軍宜須在前,不可遲緩。」摩訶因率馬軍夜發。先是,周軍長圍既合,又於要路下伏數重,摩訶選精騎八十,率先衝突,自後眾騎繼焉,比旦達淮南。高宗詔徵還,授右衛將軍。十一年,周兵寇壽
 陽,摩訶與樊毅等眾軍赴援,無功而還。



 十四年,高宗崩,始興王叔陵於殿內手刃後主,傷而不死,叔陵奔東府城。時眾心猶預,莫有討賊者,東宮舍人司馬申啟後主,馳召摩訶,入見受敕,乃率馬步數百,先趣東府城西門屯軍。叔陵惶遽,自城南門而出,摩訶勒兵追斬之。以功授散騎常侍、車騎大將軍,封綏建郡公,邑三千戶,叔陵素所蓄聚金帛累巨萬,後主悉以賜之。尋改授侍中、驃騎大將軍,加左光祿大夫。舊制三公黃閣聽事置鴟尾,
 後主特賜摩訶開黃閣,門施行馬,聽事寢堂並置鴟尾。仍以其女為皇太子妃。



 會隋總管賀若弼鎮廣陵,窺覦江左,後主委摩訶備禦之任,授南徐州刺史,餘並如故。禎明三年正月元會,徵摩訶還朝,賀若弼乘虛濟江,襲京口,摩訶請兵逆戰,後主不許。及弼進軍鐘山,摩訶又請曰:「賀若弼懸軍深入,聲援猶遠,且其壘塹未堅,人情惶懼,出兵掩襲,必大克之。」後主又不許。及隋軍大至,將出戰,後主謂摩訶曰:「公可為我一決。」摩訶曰:「從來行
 陣,為國為身,今日之事,兼為妻子。」後主多出金帛,頒賞諸軍,令中領軍魯廣達陳兵白土崗,居眾軍之南偏,鎮東大將軍任忠次之,護軍將軍樊毅、都官尚書孔範次之,摩訶軍最居北,眾軍南北亙二十里,首尾進退,各不相知。賀若弼初謂未戰,將輕騎,登山觀望形勢,及見眾軍,因馳下置陣。廣達首率所部進薄,弼軍屢卻,俄而復振,更分軍趣北突諸將,孔範出戰,兵交而走,諸將支離,陣猶未合,騎卒潰散,駐之弗止,摩訶無所用力焉,為隋
 軍所執。



 及京城陷,賀若弼置後主於德教殿,令兵衛守,摩訶請弼曰:「今為囚虜,命在斯須,願得一見舊主,死無所恨。」弼哀而許之。摩訶入見後主,俯伏號泣,仍於舊廚取食而進之,辭訣而出,守衛者皆不能仰視。其年入隋,授開府儀同三司。



 尋從漢王諒詣并州,同諒作逆,伏誅,時年七十三。



 摩訶訥於語言,恂恂長者,至於臨戎對寇,志氣奮發,所向無前。年未弱冠,隨侯安都在京口,性好射獵,無日不畋遊。及安都東征西伐,戰勝攻取,摩訶功
 實居多。



 子世廉,少警俊,敢勇有父風。性至孝,及摩訶凶終,服闋後,追慕彌切。其父時賓故脫有所言及,世廉對之,哀慟不自勝,言者為之歔欷。終身不執刀斧,時人嘉焉。



 摩訶有騎士陳智深者,勇力過人,以平叔陵之功,為巴陵內史。摩訶之戮也,其妻子先已籍沒,智深收摩訶屍,手自殯斂,哀感行路,君子義之。



 潁川陳禹,亦隨摩訶征討,聰敏有識量,涉獵經史,解風角、兵書,頗能屬文,便騎射,官至王府諮議。



 任忠,字奉誠,小名蠻奴,汝陰人也。少孤微,不為鄉黨所齒。及長,譎詭多計略,膂力過人,尤善騎射,州里少年皆附之。梁鄱陽王蕭範為合州刺史,聞其名,引置左右。侯景之亂,忠率鄉黨數百人,隨晉熙太守梅伯龍討景將王貴顯於壽春,每戰卻敵。會土人胡通聚眾寇抄,範命忠與主帥梅思立并軍討平之。仍隨範世子嗣率眾入援,會京城陷,旋戍晉熙。侯景平,授蕩寇將軍。



 王琳立蕭莊,署忠為巴陵太守。琳敗還朝,遷明毅將軍、安湘太守,
 仍隨侯瑱真進討巴、湘。累遷豫寧太守、衡陽內史。華皎之舉兵也,忠預其謀。及皎平,高宗以忠先有密啟於朝廷,釋而不問。太建初,隨章昭達討歐陽紇於廣州,以功授直閣將軍。遷武毅將軍、廬陵內史,秩滿,入為右軍將軍。



 五年,眾軍北伐,忠將兵出西道,擊走齊歷陽王高景安於大峴,逐北至東關,仍克其東西二城。進軍蘄、譙,並拔之。徑襲合肥,入其郛。進克霍州。以功授員外散騎常侍,封安復縣侯,邑五百戶。呂梁之喪師也,忠全軍而還。尋
 詔忠都督壽陽、新蔡、霍州緣淮眾軍,進號寧遠將軍、霍州刺史。入為左衛將軍。十一年,加北討前軍事,進號平北將軍,率眾步騎趣秦郡。十二年,遷使持節、散騎常侍、都督南豫州諸軍事、平南將軍、南豫州刺史,增邑并前一千五百戶。仍率步騎趣歷陽。



 周遣王延貴率眾為援,忠大破之,生擒延貴。後主嗣位,進號鎮南將軍,給鼓吹一部。入為領軍將軍,加侍中,改封梁信郡公,邑三千戶。出為吳興內史,加秩中二千石。



 及隋兵濟江,忠自吳
 興入赴,屯軍朱雀門。後主召蕭摩訶以下於內殿定議,忠執議曰:「兵家稱客主異勢,客貴速戰,主貴持重。宜且益兵堅守宮城,遣水軍分向南豫州及京口道,斷寇糧運。待春水長,上江周羅珣等眾軍,必沿流赴援,此良計矣。」眾議不同,因遂出戰。及敗,忠馳入臺見後主,言敗狀,啟云:「陛下唯當具舟楫,就上流眾軍,臣以死奉衛。」後主信之,敕忠出部分,忠辭云:「臣處分訖,即當奉迎。」後主令宮人裝束以待忠,久望不至。隋將韓擒虎自新林進軍,
 忠乃率數騎往石子崗降之,仍引擒虎軍共入南掖門。臺城陷,其年入長安,隋授開府儀同三司。卒,時年七十七。子幼武,官至儀同三司。



 時有沈客卿者,吳興武康人,性便佞忍酷,為中書舍人,每立異端,唯以刻削百姓為事,由是自進。有施文慶者,吳興烏程人,起自微賤,有吏用,後主拔為主書,遷中書舍人,俄擢為湘州刺史。未及之官,會隋軍來伐,四方州鎮,相繼以聞。



 文慶、客卿俱掌機密,外有表啟,皆由其呈奏。文慶心悅湘州重鎮,冀欲
 早行,遂與客卿共為表裏,抑而不言,後主弗之知也,遂以無備,至乎敗國,實二人之罪。



 隋軍既入,並戮之於前闕。



 樊毅,字智烈,南陽湖陽人也。祖方興,梁散騎常侍、仁威將軍、司州刺史,魚復縣侯。父文熾,梁散騎常侍、信武將軍、益州刺史,新蔡縣侯。毅累葉將門,少習武善射。侯景之亂,毅率部曲隨叔父文皎援臺。文皎於青溪戰歿,毅將宗族子弟赴江陵,仍隸王僧辯,討河東王蕭譽,以功除假節、威戎將軍、右中郎將。代兄俊為梁興太守,領三
 州遊軍,隨宜豊侯蕭循討陸納於湘州。軍次巴陵,營頓未立,納潛軍夜至,薄營大噪,營中將士皆驚擾,毅獨與左右數十人,當營門力戰,斬十餘級,擊鼓申命,眾乃定焉。以功授持節、通直散騎常侍、貞威將軍,封夷道縣伯,食邑三百戶。尋除天門太守,進爵為侯,增邑并前一千戶。及西魏圍江陵,毅率兵赴援,會江陵陷,為岳陽王所執,久之遁歸。



 高祖受禪,毅與弟猛舉兵應王琳,琳敗奔齊,太尉侯瑱遣使招毅,毅率子弟部曲還朝。天嘉二年,授
 通直散騎常侍,仍隨侯瑱進討巴、湘。累遷武州刺史。太建初,轉豊州刺史,封高昌縣侯,邑一千戶。入為左衛將軍。五年,眾軍北伐,毅率眾攻廣陵楚子城,拔之,擊走齊軍於潁口,齊援滄陵,又破之。七年,進克潼州、下邳、高柵等六城。及呂梁喪師,詔以毅為大都督,進號平北將軍,率眾渡淮,對清口築城,與周人相抗,霖雨城壞,毅全軍自拔。尋遷中領軍。十一年,周將梁士彥將兵圍壽陽,詔以毅為都督北討前軍事,率水軍入焦湖。尋授鎮西將
 軍、都督荊、郢、巴、武四州水陸諸軍事。十二年,進督沔、漢諸軍事,以公事免。十三年,徵授中護軍。尋遷護軍將軍、荊州刺史。



 後主即位,進號征西將軍,改封逍遙郡公,邑三千戶,餘並如故。入為侍中、護軍將軍。及隋兵濟江,毅謂僕射袁憲曰:「京口、采石,俱是要所,各須銳卒數千,金翅二百,都下江中,上下防捍。如其不然,大事去矣。」諸將咸從其議。會施文慶等寢隋兵消息,毅計不行。京城陷,隨例入關,頃之卒。



 猛字智武,毅之弟也。幼倜儻,有幹略。既壯,便弓馬,膽氣過人。青溪之戰,猛自旦訖暮,與虜短兵接,殺傷甚眾。臺城陷,隨兄毅西上京,累戰功為威戎將軍。



 梁南安侯蕭方矩為湘州刺史,以猛為司馬。會武陵王蕭紀舉兵自漢江東下,方矩遣猛率湘、郢之卒,隨都督陸法和進軍以拒之。時紀已下,樓船戰艦據巴江,爭峽口,相持久之,不能決。法和揣紀師老卒墮,因令猛率驍勇三千,輕舸百餘乘,衝流直上,出其不意,鼓噪薄之。紀眾倉卒驚駭,
 不及整列,皆棄艦登岸,赴水死者以千數。時紀心膂數百人,猶在左右,猛將部曲三十餘人,蒙楯橫戈,直登紀舟,瞋目大呼,紀侍衛皆披靡,相枕藉不敢動。猛手擒紀父子三人,斬於絺中,盡收其船艦器械。以功授游騎將軍,封安山縣伯,邑一千戶。仍進軍撫定梁、益,蜀境悉平。



 軍還,遷持節、散騎常侍、輕車將軍、司州刺史,進爵為侯,增邑并前二千戶。



 永定元年,周文育等敗於沌口,為王琳所獲。琳乘勝將略南中諸郡,遣猛與李孝欽等將兵
 攻豫章,進逼周迪,軍敗,為迪斬執。尋遁歸王琳。王琳敗,還朝。天嘉二年,授通直散騎常侍、永陽太守。遷安成王府司馬。光大元年,授壯武將軍、廬陵內史。太建初,遷武毅將軍、始興平南府長史,領長沙內史。尋隸章昭達西討江陵,潛軍入峽,焚周軍船艦,以功封富川縣侯,邑五百戶。歷散騎常侍,遷使持節、都督荊信二州諸軍事、宣遠將軍、荊州刺史。入為左衛將軍。



 後主即位,增邑并前一千戶,餘並如故。至德四年,授使持節、都督南豫州諸
 軍事、忠武將軍、南豫州刺史。隋將韓擒虎之濟江也,猛在京師,第六子巡攝行州事,擒虎進軍攻陷之,巡及家口並見執。時猛與左衛將軍蔣元遜領青龍八十艘為水軍,於白下遊弈,以禦隋六合兵,後主知猛妻子在隋軍,懼其有異志,欲使任忠代之,又重傷其意,乃止。禎明三年入于隋。



 魯廣達,字遍覽,吳州刺史悉達之弟也。少慷慨,志立功名,虛心愛士,賓客或自遠而至。時江表將帥,各領部曲,
 動以千數,而魯氏尤多。釋褐梁邵陵王國右常侍,遷平南當陽公府中兵參軍。侯景之亂,與兄悉達聚眾保新蔡。梁元帝承制,授假節、壯武將軍、晉州刺史。王僧辯之討侯景也,廣達出境候接,資奉軍儲,僧辯謂沈炯曰:「魯晉州亦是王師東道主人。」仍率眾隨僧辯。景平,加員外散騎常侍,餘如故。



 高祖受禪,授徵遠將軍、東海太守。尋徙為桂陽太守,固辭不拜,入為員外散騎常侍。除假節、信武將軍、北新蔡太守。隨吳明徹討周迪於臨川,每戰
 功居最。



 仍代兄悉達為吳州刺史,封中宿縣侯,邑五百戶。



 光大元年,授通直散騎常侍、都督南豫州諸軍事、南豫州刺史。華皎稱兵上流,詔司空淳于量率眾軍進討。軍至夏口,皎舟師強盛,莫敢進者,廣達首率驍勇,直衝賊軍。戰艦既交,廣達憤怒大呼,登艦樓,獎勵士卒,風急艦轉,樓搖動,廣達足跌墮水,沈溺久之,因救獲免。皎平,授持節、智武將軍、都督巴州諸軍事、巴州刺史。



 太建初,與儀同章昭達入峽口,拓定安蜀等諸州鎮。時周
 氏將圖江左,大造舟艦於蜀,并運糧青泥,廣達與錢道戢等將兵掩襲,縱火焚之。以功增封并前二千戶,仍還本鎮。廣達為政簡要,推誠任下,吏民便之。及秩滿。皆詣闕表請,於是詔留二年。五年,眾軍北伐,略淮南舊地,廣達與齊軍會於大峴,大破之,斬其敷城王張元範,虜獲不可勝數。進克北徐州,乃授都督北徐州諸軍事、北徐州刺史。尋加散騎常侍,入為右衛將軍。八年,出為北兗州刺史,遷晉州刺史。十年,授使持節、都督合霍二州諸軍
 事,進號仁威將軍、合州刺史。十一年,周將梁士彥將兵圍壽春,詔遣中領軍樊毅、左衛將軍任忠等分部趣陽平、秦郡,廣達率眾入淮,為掎角以擊之。周軍攻陷豫、霍二州,南、北兗、晉等各自拔,諸將並無功,盡失淮南之地,廣達因免官,以侯還第。十二年,與豫州刺史樊毅率眾北討,克郭默城。尋授使持節、平西將軍、都督郢州以上十州諸軍事,率舟師四萬,治江夏。周安州總管元景將兵寇江外,廣達命偏師擊走之。



 後主即位,入為安左將
 軍。尋受平南將軍、南豫州刺史。至德二年,授安南將軍,徵拜侍中,又為安左將軍,改封綏越郡公,封邑如前。尋為中領軍。及賀若弼進軍鐘山,廣達率眾於白土崗南置陣,與弼旗鼓相對。廣達躬擐甲胄,手執桴鼓,率勵敢死,冒刃而前,隋軍退走,廣達逐北至營,殺傷甚眾,如是者數四焉。及弼攻敗諸將,乘勝至宮城,燒北掖門,廣達猶督餘兵,苦戰不息,斬獲數十百人。會日暮,乃解甲,面臺再拜慟哭,謂眾曰:「我身不能救國,負罪深矣。」士卒皆
 涕泣歔欷,於是乃就執。禎明三年,依例入隋。



 廣達愴本朝淪覆,遘疾不治,尋以憤慨卒,時年五十九。尚書令江總撫柩慟哭,乃命筆題其棺頭,為詩曰:「黃泉雖抱恨,白日自流名。悲君感義死,不作負恩生。」



 總又製廣達墓銘,其略曰:「災流淮海,險失金湯,時屯運極,代革天亡。爪牙背義,介胄無良,獨摽忠勇,率禦有方。誠貫皎日,氣勵嚴霜,懷恩感報,撫事何忘。」



 初,隋將韓擒虎之濟江也,廣達長子世真在新蔡,乃與其弟世雄及所部奔擒虎,擒虎
 遣使致書,以招廣達。廣達時屯兵京師,乃自劾廷尉請罪。後主謂之曰:「世真雖異路中大夫,公國之重臣,吾所恃賴,豈得自同嫌疑之間乎?」加賜黃金,即日還營。



 廣達有隊主楊孝辯,時從廣達在軍中,力戰陷陣,其子亦隨孝辯,揮刃殺隋兵十餘人,力窮,父子俱死。



 史臣曰:蕭摩訶氣冠三軍,當時良將,雖無智略,亦一代匹夫之勇矣;然口訥心勁,恂恂李廣之徒歟!任忠雖勇決彊斷,而心懷反覆,誣紿君上,自躓其惡,鄙矣!至於魯
 廣達全忠守道,殉義忘身,蓋亦陳代之良臣也。



\end{pinyinscope}