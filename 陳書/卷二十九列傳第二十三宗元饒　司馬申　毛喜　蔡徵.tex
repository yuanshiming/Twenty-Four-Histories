\article{卷二十九列傳第二十三宗元饒 司馬申 毛喜 蔡徵}

\begin{pinyinscope}

 宗元饒,南郡江陵人也。少好學,以孝敬聞。仕梁世,解褐本州主簿,遷征南府行參軍,仍轉外兵參軍。及司徒王僧辯幕府初建,元饒與沛國劉師知同為主簿。



 高祖受
 禪,除晉陵令。入為尚書功論郎。使齊還,為廷尉正。遷太僕卿,領本邑大中正,中書通事舍人。尋轉廷尉卿,加通直散騎常侍,兼尚書左丞。時高宗初即位,軍國務廣,事無巨細,一以咨之,臺省號為稱職。



 遷御史中丞,知五禮事。時合州刺史陳裦贓汙狼藉,遣使就渚斂魚,又於六郡乞米,百姓甚苦之。元饒劾奏曰:「臣聞建旟求瘼,實寄廉平,褰帷恤隱,本資仁恕。如或貪汙是肆,征賦無厭,天網雖疏,茲焉弗漏。謹案鐘陵縣開國侯、合州刺史臣褵,
 因藉多幸,預逢抽擢,爵由恩被,官以私加,無德無功,坐尸榮貴。譙、肥之地,久淪非所,皇威剋復,物仰仁風。新邦用輕,彌俟寬惠,應斯作牧,其寄尤重。爰降曲恩,祖行宣室,親承規誨,事等言提。雖廉潔之懷,誠無素蓄,而稟茲嚴訓,可以厲精。遂乃擅行賦斂,專肆貪取,求粟不厭,愧王沉之出賑,徵魚無限,異羊續之懸枯,置以嚴科,實惟明憲。臣等參議,請依旨免褵所應復除官,其應禁錮及後選左降本資,悉依免官之法。」遂可其奏。吳興太守武
 陵王伯禮,豫章內史南康嗣王方泰,並驕蹇放橫,元饒案奏之,皆見削黜。



 元饒性公平,善持法,諳曉故事,明練治體,吏有犯法、政不便民及於名教不足者,隨事糾正,多所裨益。遷貞威將軍、南康內史,以秩米三千餘斛助民租課,存問高年,拯救乏絕,百姓甚賴焉。以課最入朝,詔加散騎常侍、荊、雍、湘、巴、武五州大中正。尋以本官重領尚書左丞。又為御史中丞。歷左民尚書、右衛將軍、領前將軍,遷吏部尚書。太建十三年卒,時年六十四。詔贈侍
 中、金紫光祿大夫,官給喪事。



 司馬申,字季和,河內溫人也。祖慧遠,梁都水使者。父玄通,梁尚書左民郎。



 申早有風概,十四便善弈棋,嘗隨父候吏尚書到溉,時梁州刺史陰子春、領軍朱異在焉。子春素知申,即於坐所呼與為對,申每有妙思,異觀而奇之,因引申遊處。



 梁邵陵王為丹陽尹,以申為主簿。屬太清之難,父母俱沒,因此自誓,菜食終身。



 梁元帝承制,起為開遠將軍,遷鎮西外兵記室參軍。及侯景寇郢
 州,申隨都督王僧辯據巴陵,每進籌策,皆見行用。僧辯歎曰:「此生要鞬汗馬,或非所長,若使撫眾守城,必有奇績。」僧辯之討陸納也,申在軍中,于時賊眾奄至,左右披靡,申躬蔽僧辯,蒙楯而前,會裴之橫救至,賊乃退,僧辯顧申而笑曰:「仁者必有勇,豈虛言哉!」除散騎侍郎。紹泰初,遷儀同侯安都從事中郎。



 高祖受禪,除安東臨川王諮議參軍。天嘉三年,遷征北諮議參軍,兼廷尉監。



 五年,除鎮東諮議參軍,兼起部郎。出為戎昭將軍、江乘令,甚
 有治績。入為尚書金部郎。遷左民郎,以公事免。太建初,起為貞威將軍、征南鄱陽諮議參軍。九年,除秣陵令,在職以清能見紀,有白雀巢于縣庭。秩滿,頃之,預東宮賓客,尋兼東宮通事舍人。遷員外散騎常侍,舍人如故。及叔陵之肆逆也,事既不捷,出據東府,申馳召右衛蕭摩訶帥兵先至,追斬之,因入城中,收其府庫,後主深嘉之。以功除太子左衛率,封文招縣伯,邑四百戶,兼中書通事舍人。尋遷右衛將軍,加通直散騎常侍。以疾還第,
 就加散騎常侍,右衛、舍人如故。



 至德四年卒,後主嗟悼久之,下詔曰:「慎終追遠,欽若舊則,闔棺定謚,抑乃前典。故散騎常侍、右衛將軍、文招縣開國伯申,忠肅在公,清正立己,治繁處約,投軀殉義。朕任寄情深,方康庶績,奄然化往,傷惻于懷。可贈侍中、護軍將軍,進爵為侯,增邑為五百戶,謚曰忠。給朝服一具,衣一襲,剋日舉哀,喪事所須,隨由資給。」及葬,後主自製誌銘,辭情傷切。卒章曰:「嗟乎!天不與善,殲我良臣。」其見幸如此。



 申歷事三帝,內
 掌機密,至於倉卒之間,軍國大事,指麾斷決,無有滯留。子琇嗣,官至太子舍人。



 毛喜,字伯武,滎陽陽武人也。祖稱,梁散騎侍郎。父栖忠,梁尚書比部侍郎、中權司馬。喜少好學,善草隸。起家梁中衛西昌侯行參軍,尋遷記室參軍。高祖素知於喜,及鎮京口,命喜與高宗俱往江陵,仍敕高宗曰:「汝至西朝,可諮稟毛喜。」



 喜與高宗同謁梁元帝,即以高宗為領直,喜為尚書功論侍郎。及江陵陷,喜及高宗俱遷關右。世
 祖即位,喜自周還,進和好之策,朝廷乃遣周弘正等通聘。及高宗反國,喜於郢州奉迎。又遣喜入關,以家屬為請。周冢宰宇文護執喜手曰:「能結二國之好者,卿也。」仍迎柳皇后及後主還。天嘉三年至京師,高宗時為驃騎將軍,仍以喜為府諮議參軍,領中記室。府朝文翰,皆喜詞也。



 世祖嘗謂高宗曰:「我諸子皆以『伯』為名,汝諸兒宜用『叔』為稱。」高宗以訪于喜,喜即條牒自古名賢杜叔英、虞叔卿等二十餘人以啟世祖,世祖稱善。



 世祖崩,廢帝
 沖昧,高宗錄尚書輔政,僕射到仲舉等知朝望有歸,乃矯太后令遣高宗還東府,當時疑懼,無敢措言。喜即馳入,謂高宗曰:「陳有天下日淺,海內未夷,兼國禍併鐘,萬邦危懼。皇太后深惟社稷至計,令王入省,方當共康庶績,比德伊、周。今日之言,必非太后之意。宗社之重,願加三思。以喜之愚,須更聞奏,無使姦賊得肆其謀。」竟如其策。



 右衛將軍韓子高始與仲舉通謀,其事未發,喜請高宗曰:「宜簡選人馬,配與子高,并賜鐵炭,使修器甲。」高宗
 驚曰:「子高謀反,即欲收執,何為更如是邪?」



 喜答曰:「山陵始畢,邊寇尚多,而子高受委前朝,名為杖順,然甚輕狷,恐不時授首,脫其稽誅,或愆王度。宜推心安誘,使不自疑,圖之一壯士之力耳。」高宗深然之,卒行其計。



 高宗即位,除給事黃門侍郎,兼中書舍人,典軍國機密。高宗將議北伐,敕喜撰軍制,凡十三條,詔頒天下,文多不載。尋遷太子右衛率、右衛將軍。以定策功,封東昌縣侯,邑五百戶。又以本官行江夏、武陵、桂陽三王府國事。太建三
 年,丁母憂去職,詔追贈喜母庾氏東昌國太夫人,賜布五百匹,錢三十萬,官給喪事。又遣員外散騎常侍杜緬圖其墓田,高宗親與緬案圖指畫,其見重如此。尋起為明威將軍,右衛、舍人如故。改授宣遠將軍、義興太守。尋以本號入為御史中丞。服闋,加散騎常侍、五兵尚書,參掌選事。



 及眾軍北伐,得淮南地,喜陳安邊之術,高宗納之,即日施行。又問喜曰:「我欲進兵彭、汴,於卿意如何?」喜對曰:「臣實才非智者,安敢預兆未然。竊以淮左新平,邊
 氓未乂,周氏始吞齊國,難與爭鋒,豈以弊卒疲兵,復加深入。且棄舟楫之工,踐車騎之地,去長就短,非吳人所便。臣愚以為不若安民保境,寢兵復約,然後廣募英奇,順時而動,斯久長之術也。」高宗不從。後吳明徹陷周,高宗謂喜曰:「卿之所言,驗於今矣。」



 十二年,加侍中。十三年,授散騎常侍、丹陽尹。遷吏部尚書,常侍如故。及高宗崩,叔陵構逆,敕中庶子陸瓊宣旨,令南北諸軍,皆取喜處分。賊平,又加侍中,增封并前九百戶。至德元年,授信威
 將軍、永嘉內史,加秩中二千石。



 初,高宗委政於喜,喜亦勤心納忠,多所匡益,數有諫諍,事並見從,由是十餘年間,江東狹小,遂稱全盛。唯略地淮北,不納喜謀,而吳明徹竟敗,高宗深悔之,謂袁憲曰:「不用毛喜計,遂令至此,朕之過也。」喜既益親,乃言無回避,而皇太子好酒德,每共幸人為長夜之宴,喜嘗為言,高宗以誡太子,太子陰患之,至是稍見疏遠。



 初,後主為始興王所傷,及瘡愈而自慶,置酒於後殿,引江總以下,展樂賦詩,醉而命喜。于
 時山陵初畢,未及踰年,喜見之不懌,欲諫而後主已醉,喜升階,佯為心疾,仆于階下,移出省中。後主醒,乃疑之,謂江總曰:「我悔召毛喜,知其無疾,但欲阻我懽宴,非我所為,故姦詐耳。」乃與司馬申謀曰:「此人負氣,吾欲將乞鄱陽兄弟聽其報仇,可乎?」對曰:「終不為官用,願如聖旨。」傅縡爭之曰:「不然。若許報仇,欲置先皇何地?」後主曰:「當乞一小郡,勿令見人事也。」



 乃以喜為永嘉內史。



 喜至郡,不受俸秩,政弘清靜,民吏便之。遇豊州刺史章大寶舉
 兵反,郡與豊州相接,而素無備禦,喜乃修治城隍,嚴飾器械。又遣所部松陽令周磻領千兵援建安。賊平,授南安內史。禎明元年,徵為光祿大夫,領左驍騎將軍。喜在郡有惠政,乃徵入朝,道路追送者數百里。其年道病卒,時年七十二。有集十卷。子處沖嗣,官至儀同從事中郎、中書侍郎。



 蔡徵,字希祥,侍中、中撫軍將軍景歷子也。幼聰敏,精識彊記。年六歲,詣梁吏部尚書河南褚翔,翔字仲舉,嗟其
 穎悟。七歲,丁母憂,居喪如成人禮。繼母劉氏性悍忌,視之不以道,徵供侍益謹,初無怨色。徵本名覽,景歷以為有王祥之性,更名徵,字希祥。



 梁承聖初,高祖為南徐州刺史,召補迎主簿。尋授太學博士。天嘉初,遷始興王府法曹行參軍,歷外兵參軍事、尚書主客郎,所居以幹理稱。太建初,遷太子少傅丞、新安王主簿、通直散騎侍郎、晉安王功曹史、太子中舍人,兼東宮領直,中舍人如故。丁父憂去職,服闋,襲封新豊縣侯,授戎昭將軍、鎮右新安
 王諮議參軍。



 至德二年,遷廷尉卿,尋為吏部郎。遷太子中庶子、中書舍人,掌詔誥。尋授左民尚書,與僕射江總知撰五禮事。尋加寧遠將軍。後主器其材幹,任寄日重,遷吏部尚書、安右將軍,每十日一往東宮,於太子前論述古今得喪及當時政務。又敕以廷尉寺獄,事無大小,取徵議決。俄有敕遣徵收募兵士,自為部曲,徵善撫恤,得物情,旬月之間,眾近一萬。徵位望既重,兼聲勢熏灼,物議咸忌憚之。尋徙為中書令,將軍如故。中令清簡無
 事,或云徵有怨言,事聞後主,後主大怒,收奪人馬,將誅之,有固諫者獲免。



 禎明三年,隋軍濟江,後主以徵有乾用,權知中領軍。徵日夜勤苦,備盡心力,後主嘉焉,謂曰「事寧有以相報」。及決戰於鐘山南崗,敕徵守宮城西北大營,尋令督眾軍戰事。城陷,隨例入關。



 徵美容儀,有口辯,多所詳究。至於士流官宦,皇家戚屬,及當朝制度,憲章儀軌,戶口風俗,山川土地,問無不對。然性頗便佞進取,不能以退素自業。初拜吏部尚書,啟後主借鼓吹,後主
 謂所司曰:「鼓吹軍樂,有功乃授,蔡徵不自量揆,紊我朝章。然其父景歷既有締構之功,宜且如所啟,拜訖即追還。」徵不修廉隅,皆此類也。隋文帝聞其敏贍,召見顧問,言輒會旨,然累年不調,久之,除太常丞。



 歷尚書民部儀曹郎,轉給事郎,卒,時年六十七。子翼,治《尚書》,官至司徒屬、德教學士。入隋,為東宮學士。



 史臣曰:宗元饒夙夜匪懈,濟務益時。司馬申清恪在朝,攻苦立行,加之以忠節,美矣。毛喜深達事機,匡贊時主。
 蔡徵聰敏才贍,而擅權自躓,惜哉!



\end{pinyinscope}