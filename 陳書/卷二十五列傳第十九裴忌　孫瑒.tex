\article{卷二十五列傳第十九裴忌 孫瑒}

\begin{pinyinscope}

 裴忌,字無畏,河東聞喜人也。祖髦,梁中散大夫。父之平,倜儻有志略,召補文德主帥。梁普通中眾軍北伐,之平隨都督夏侯亶克定渦、潼,以功封費縣侯。



 會衡州部民
 相聚寇抄,詔以之平為假節、超武將軍、都督衡州五郡征討諸軍事。及之平至,即皆平殄,梁武帝甚嘉賞之。元帝承聖中,累遷散騎常侍、右衛將軍、晉陵太守。世祖即位,除光祿大夫,慈訓宮衛尉,並不就,乃築山穿池,植以卉木,居處其中,有終焉之志。天康元年卒,贈仁威將軍、光祿大夫,謚曰僖子。



 忌少聰敏,有識量,頗涉史傳,為當時所稱。解褐梁豫章王法曹參軍。侯景之亂,忌招集勇力,隨高祖征討,累功為寧遠將軍。及高祖誅王僧
 辯,僧辯弟僧智舉兵據吳郡,高祖遣黃他率眾攻之,僧智出兵於西昌門拒戰,他與相持,不能克。高祖謂忌曰:「三吳奧壤,舊稱饒沃,雖凶荒之餘,猶為殷盛,而今賊徒扇聚,天下搖心,非公無以定之,宜善思其策。」忌乃勒部下精兵,輕行倍道,自錢塘直趣吳郡,夜至城下,鼓噪薄之。僧智疑大軍至,輕舟奔杜龕,忌入據其郡。高祖嘉之,表授吳郡太守。



 高祖受禪,徵為左衛將軍。天嘉初,出為持節、南康內史。時義安太守張紹賓據郡反,世祖以忌為持
 節、都督嶺北諸軍事,率眾討平之。還除散騎常侍、司徒左長史。五年,授雲麾將軍、衛尉卿,封東興縣侯,邑六百戶。及華皎稱兵上流,高宗時為錄尚書輔政,盡命眾軍出討,委忌總知中外城防諸軍事。及皎平,高宗即位,太建無年,授東陽太守,改封樂安縣侯,邑一千戶。四年,入為太府卿。五年,轉都官尚書。



 吳明徹督眾軍北伐,詔忌以本官監明徹軍。淮南平,授軍師將軍、豫州刺史。



 忌善於綏撫,甚得民和。改授使持節、都督譙州諸軍事、譙州
 刺史。未及之官,會明徹受詔進討彭、汴,以忌為都督,與明徹掎角俱進。呂梁軍敗,陷於周,周授上開府。隋開皇十四年,卒於長安,時年七十三。



 孫瑒,字德璉,吳郡吳人也。祖文惠,齊越騎校尉、清遠太守。父循道,梁中散大夫,以雅素知名。瑒少倜儻,好謀略,博涉經史,尤便書翰。起家梁輕車臨川嗣王行參軍,累遷為安西邵陵王水曹中兵參軍事。王出鎮郢州,瑒盡室隨府,甚被賞遇。太清之難,授假節、宣猛將軍、軍主。王
 僧辯之討侯景也,王琳為前軍,琳與瑒同門,乃表薦為戎昭將軍、宜都太守,仍從僧辯救徐文盛於武昌。會郢州陷,乃留軍鎮巴陵,脩戰守之備。俄而侯景兵至,日夜攻圍,瑒督所部兵悉力拒戰,賊眾奔退。瑒從大軍沿流而下,及克姑熟,瑒力戰有功,除員外散騎常侍,封富陽縣侯,邑一千戶。尋受假節、雄信將軍、衡陽內史,未及之官,仍遷衡州平南府司馬。



 破黃洞蠻賊有功,除東莞太守,行廣州刺史。尋除智武將軍,監湘州事。敬帝嗣位,授
 持節、仁威將軍、巴州刺史。



 高祖受禪,王琳立梁永嘉王蕭莊於郢州,徵瑒為太府卿,加通直散騎常侍。及王琳入寇,以瑒為使持節、散騎常侍、都督郢、荊、巴、武、湘五州諸軍事、安西將軍、郢州刺史,總留府之任。周遣大將史寧率眾四萬,乘虛奄至,瑒助防張世貴舉外城以應之,所失軍民男女三千餘口。周軍又起土山高梯,日夜攻逼,因風縱火,燒其內城南面五十餘樓。時瑒兵不滿千人,乘城拒守,瑒親自撫巡,行酒賦食,士卒皆為之用命。
 周人苦攻不能克,乃矯授瑒柱國、郢州刺史,封萬戶郡公。瑒偽許以緩之,而潛修戰具,樓雉器械,一朝嚴設,周人甚憚焉。及聞大軍敗王琳,乘勝而進,周兵乃解。瑒於是盡有中流之地,集其將士而謂之曰:「吾與王公陳力協義,同獎梁室,亦已勤矣。今時事如此,天可違乎!」遂遣使奉表詣闕。



 天嘉元年,授使持節、散騎常侍、安南將軍、湘州刺史,封定襄縣侯,邑一千戶。瑒懷不自安,乃固請入朝,徵為散騎常侍、中領軍。未拜,而世祖從容謂瑒曰:「
 昔朱買臣願為本郡,卿豈有意乎?」仍改授持節、安東將軍、吳郡太守,給鼓吹一部。及將之鎮,乘輿幸近畿餞送,鄉里榮之。秩滿,徵拜散騎常侍、中護軍,鼓吹如故。留異之反東陽,詔瑒督舟師進討。異平,遷鎮右將軍,常侍、鼓吹並如故。



 頃之,出為使持節、安東將軍、建安太守。光大中,以公事免,尋起為通直散騎常侍。



 高宗即位,以瑒功名素著,深委任焉。太建四年,授都督荊、信二州諸軍事、安西將軍、荊州刺史,出鎮公安。瑒增修城池,懷服邊遠,
 為鄰境所憚。居職六年,又以事免,更為通直散騎常侍。及吳明徹軍敗呂梁,授使持節、督緣江水陸諸軍事、鎮西將軍,給鼓吹一部。尋授散騎常侍、都督荊、郢、巴、武、湘五州諸軍事、郢州刺史,持節、將軍、鼓吹並如故。十二年,坐疆埸交通抵罪。



 後主嗣位,復除通直散騎常侍,兼起部尚書。尋除中護軍,復爵邑,入為度支尚書,領步兵校尉。俄加散騎常侍,遷侍中、祠部尚書。後主頻幸其第,及著詩賦述勳德之美,展君臣之意焉。又為五兵尚書,領
 右軍將軍,侍中如故。以年老累乞骸骨,優詔不許。禎明元年卒官,時年七十二。後主臨哭盡哀,贈護軍將軍,侍中如故,給鼓吹一部,朝服一具,衣一襲,喪事量加資給,謚曰桓子。



 瑒事親以孝聞,於諸弟甚篤睦。性通泰,有財物散之親友。其自居處,頗失於奢豪,庭院穿築,極林泉之致,歌鐘舞女,當世罕儔,賓客填門,軒蓋不絕。及出鎮郢州,乃合十餘船為大舫,於中立亭池,植荷芰,每良辰美景,賓僚並集,泛長江而置酒,亦一時之勝賞焉。常於
 山齋設講肆,集玄儒之士,冬夏資奉,為學者所稱。而處己率易,不以名位驕物。時興皇寺朗法師該通釋典,瑒每造講筵,時有抗論,法侶莫不傾心。又巧思過人,為起部尚書,軍國器械,多所創立。有鑒識,男女婚姻,皆擇素貴。及卒,尚書令江總為其誌銘,後主又題銘後四十字,遣左民尚書蔡徵宣敕就宅鐫之。其詞曰:「秋風動竹,煙水驚波。幾人樵徑,何處山阿?今時日月,宿昔綺羅。天長路遠,地久雲多。功臣未勒,此意如何。」時論以為榮。



 瑒二
 十一子,咸有父風。世子讓,早卒。第二子訓,頗知名,歷臨湘令,直閣將軍、高唐太守。陳亡入隋。



 史臣曰:在梁之季,寇賊實繁,高祖建義杖旗,將寧區夏,裴忌早識攀附,每預戎麾,摧鋒卻敵,立功者數矣。孫瑒有文武幹略。見知時主,及行軍用兵,師司馬之法,至於戰勝攻取,屢著勛庸,加以好施接物,士咸慕向。然性不循恆,頻以罪免,蓋亦陳湯之徒焉。



\end{pinyinscope}