\article{卷一本紀第一高祖上}

\begin{pinyinscope}

 高祖武皇
 帝,諱霸先,字興國,小字法生,吳興長城下若里人,漢太丘長陳實之後也。世居潁川。實玄孫準,晉太尉。準生匡,匡生達,永嘉南遷,為丞相掾,歷太子洗馬,出
 為長城令,悅其山水,遂家焉。嘗謂所親曰:「此地山川秀麗,當有王者興,二百年後,我子孫必鐘斯運。」達生康,復為丞相掾,咸和中土斷,故為長城人。康生盱眙太守英,英生尚書郎公弼,公弼生步兵校尉鼎,鼎生散騎侍郎高,高生懷安令詠,詠生安成太守猛,猛生太常卿道巨,道巨生皇考文讚。高祖以梁天監二年癸未歲生。少倜儻有大志,不治生產。既長,讀兵書,多武藝,明達果斷,為當時所推服。身長七尺五寸,日角龍顏,垂手過膝。嘗遊
 義興,館於許氏,夜夢天開數丈,有四人朱衣捧日而至,令高祖開口納焉。及覺,腹中猶熱,高祖心獨負之。



 大同初,新喻侯蕭映為吳興太守,甚重高祖,嘗目高祖謂僚佐曰:「此人方將遠大。」及映為廣州刺史,高祖為中直兵參軍,隨府之鎮。映令高祖招集士馬,眾至千人,仍命高祖監宋隆郡。所部安化二縣元不賓,高祖討平之。尋監西江督護、高要郡守。先是,武林侯蕭諮為交州刺史,以裒刻失眾心,土人李賁連結數州豪傑同時反,臺遣高
 州刺史孫冏、新州刺史盧子雄將兵擊之,冏等不時進,皆於廣州伏誅。子雄弟子略與冏子侄及其主帥杜天合、杜僧明共舉兵,執南江督護沈顗,進寇廣州,晝夜苦攻,州中震恐。高祖率精兵三千,卷甲兼行以救之,頻戰屢捷,天合中流矢死,賊眾大潰。僧明遂降。梁武帝深歎異焉,授直閣將軍,封新安子,邑三百戶,仍遣畫工圖高祖容貌而觀之。



 其年冬,蕭映卒。明年,高祖送喪還都,至大庾嶺,會有詔高祖為交州司馬,領武平太守,與刺史
 楊蒨南討。高祖益招勇敢,器械精利。蒨喜曰:「能剋賊者,必陳司武也。」委以經略。高祖與眾軍發自番禺。是時蕭勃為定州刺史,於西江相會,勃知軍士憚遠役,陰購誘之,因詭說蒨。蒨集諸將問計,高祖對曰:「交趾叛渙,罪由宗室,遂使僭亂數州,彌歷年稔。定州復欲昧利目前,不顧大計,節下奉辭伐罪,故當生死以之。豈可畏憚宗室,輕於國憲?今若奪人沮眾,何必交州討賊,問罪之師,即回有所指矣。」於是勒兵鼓行而進。十一年六月,軍至交
 州,賁眾數萬於蘇歷江口立城柵以拒官軍。蒨推高祖為前鋒,所向摧陷,賁走典澈湖,於屈獠界立砦,大造船艦,充塞湖中,眾軍憚之,頓湖口不敢進。高祖謂諸將曰:「我師已老,將士疲勞,歷歲相持,恐非良計,且孤軍無援,入人心腹,若一戰不捷,豈望生全。今藉其屢奔,人情未固,夷獠烏合,易為摧殄,正當共出百死,決力取之,無故停留,時事去矣。」諸將皆默然,莫有應者。是夜江水暴起七丈,注湖中,奔流迅激。高祖勒所部兵,乘流先進,眾軍
 鼓噪俱前,賊眾大潰。賁竄入屈獠洞中,屈獠斬賁,傳首京師,是歲太清元年也。賁兄天寶遁入九真,與劫帥李紹隆收餘兵二萬,殺德州刺史陳文戒,進圍愛州,高祖仍率眾討平之。除振遠將軍、西江督護、高要太守、督七郡諸軍事。



 二年冬,侯景寇京師,高祖將率兵赴援,廣州刺史元景仲陰有異志,將圖高祖。



 高祖知其計,與成州刺史王懷明、行臺選郎殷外臣等密議戒嚴。三年七月,集義兵於南海,馳檄以討景仲。景仲窮蹙,縊于閣下,高
 祖迎蕭勃鎮廣州。是時臨賀內史歐陽頠監衡州,蘭裕、蘭京禮扇誘始興等十郡,共舉兵攻頠,頠請援於勃。勃令高祖率眾救之,悉擒裕等,仍監始興郡。



 十一月,高祖遣杜僧明、胡穎將二千人頓于嶺上,并厚結始興豪傑同謀義舉,侯安都、張人思等率千餘人來附。蕭勃聞之,遣鐘休悅說高祖曰:「侯景驍雄,天下無敵,前者援軍十萬,士馬精彊,然而莫敢當鋒,遂令羯賊得志。君以區區之眾,將何所之?如聞嶺北王侯又皆鼎沸,河東、桂陽相次
 屠戮,邵陵、開建親尋干戈,李遷仕許身當陽,便奪馬仗,以君疏外,詎可暗投?未若且住始興,遙張聲勢,保此太山,自求多福。」高祖泣謂休悅曰:「僕本庸虛,蒙國成造。往聞侯景渡江,即欲赴援,遭值元、蘭,梗我中道。今京都覆沒,主上蒙塵,君辱臣死,誰敢愛命!



 君侯體則皇枝,任重方岳,不能摧鋒萬里,雪此冤痛,見遣一軍。猶賢乎已,乃降後旨,使人慨然。僕行計決矣,憑為披述。」乃遣使間道往江陵,稟承軍期節度。



 時蔡路養起兵據南康,勃遣腹
 心譚世遠為曲江令,與路養相結,同遏義軍。大寶元年正月,高祖發自始興,次大庾嶺。路養出軍頓南野,依山水立四城以拒高祖。高祖與戰,大破之,路養脫身竄走,高祖進頓南康。湘東王承制授高祖員外散騎常侍、持節、明威將軍、交州刺史,改封南野縣伯。



 六月,高祖修崎頭古城,徙居焉。高州刺史李遷仕據大皋,遣主帥杜平虜率千人入贛石、魚梁。高祖命周文育將兵擊走之,遷仕奔寧都。承制授高祖通直散騎常侍、使持節、信威將
 軍、豫州刺史,領豫章內史,改封長城縣侯。尋授散騎常侍、使持節、都督六郡諸軍事、軍師將軍、南江州刺史,餘如故。時寧都人劉藹等資遷仕舟艦兵仗,將襲南康,高祖遣杜僧明等率二萬人據白口,築城以禦之,遷仕亦立城以相對。二年三月,僧明等攻拔其城,生擒遷仕送南康,高祖斬之。承制命高祖進兵定江州,仍授江州刺史,餘如故。



 六月,高祖發自南康。南康贛石舊有二十四灘,灘多巨石,行旅者以為難。高祖之發也,水暴起數丈,
 三百里間巨石皆沒。進軍頓西昌,有龍見于水濱,高五丈許,五采鮮耀,軍民觀者數萬人。是時承制遣征東將軍王僧辯督眾軍討侯景。八月,僧辯軍次湓城,高祖率杜僧明等眾軍及南川豪帥合三萬人將會焉。時西軍乏食,高祖先貯軍糧五十萬石,至是分三十萬以資之,仍頓巴丘。會侯景廢簡文帝,立豫章嗣王棟,高祖遣兼長史沈袞奉表於江陵勸進。十一月,承制授高祖使持節、都督會稽東陽新安臨海永嘉五郡諸軍事、平東將
 軍、東揚州刺史,領會稽太守、豫章內史,餘並如故。三年正月,高祖率甲士三萬人、彊弩五千張、舟艦二千乘,發自豫章。



 二月,次桑落洲,遣中記室參軍江元禮以事表江陵,承制加高祖鼓吹一部。是時僧辯已發湓城,會高祖於白茅灣,乃登岸結壇,刑牲盟約。進軍次蕪湖,侯景城主張黑棄城走。三月,高祖與諸軍進剋姑孰,仍次蔡洲。侯景登石頭城觀望形勢,意甚不悅,謂左右曰:「此軍上有紫氣,不易可當。」乃以貯石沈塞淮口,緣淮作
 城,自石頭迄青溪十餘里中,樓雉相接。諸將未有所決,僧辯遣杜崱問計於高祖,高祖曰:「前柳仲禮數十萬兵隔水而坐,韋粲之在青溪,竟不渡岸,賊乃登高望之,表裏俱盡,肆其凶虐,覆我王師。今圍石頭,須渡北岸。諸將若不能當鋒,請先往立柵。」高祖即於石頭城西橫隴築柵,眾軍次連八城,直出東北。賊恐西州路斷,亦於東北果林作五城以遏大路。景率眾萬餘人、鐵騎八百餘匹,結陣而進。



 高祖曰:「軍志有之,善用兵者,如常山之蛇,首
 尾相應。今我師既眾,賊徒甚寡,應分賊兵勢,以弱制強,何故聚其鋒銳,令必死於我?」乃命諸將分處置兵。賊直衝王僧志,僧志小縮,高祖遣徐度領弩手二千橫截其後,賊乃卻。高祖與王琳、杜龕等以鐵騎悉力乘之,賊退據其柵。景儀同盧輝略開石頭北門來降。盪主戴冕、曹宣等攻拔果林一城,眾軍又剋其四城。賊復還,殊死戰,又盡奪所得城柵。高祖大怒,親率攻之,士卒騰柵而入,賊復散走。景與百餘騎棄槊執刀,左右衝陣,陣不動,景
 眾大潰,逐北至西明門。景至闕下,不敢入臺,遣腹心取其二子而遁。高祖率眾出廣陵應接,會景將郭元建奔齊,高祖納其部曲三千人而還。僧辯啟高祖鎮京口。



 五月,齊遣辛術圍嚴超達於秦郡,高祖命徐度領兵助其固守。齊眾七萬,填塹,起土山,穿地道,攻之甚急。高祖乃自率萬人解其圍,縱兵四面擊齊軍,弓弩亂發,齊平秦王中流矢死,斬首數百級,齊人收兵而退。高祖振旅南歸,遣記室參軍劉本仁獻捷於江陵。



 七月,廣陵僑民朱
 盛、張象潛結兵襲齊刺史溫仲邕,遣使來告,高祖率眾濟江以應之。會齊人來聘,求割廣陵之地,王僧辯許焉,仍豹高祖,高祖於是引軍還南徐州,江北人隨軍而南者萬餘口。承制授高祖使持節、散騎常侍、都督南徐州諸軍事、征北大將軍、開府儀同三司、南徐州刺史,餘並如故。及王僧辯率眾征陸納於湘州,承制命高祖代鎮揚州。十一月,湘東王即位於江陵,改大寶三年為承聖元年。



 湘州平,高祖旋鎮京口。三年三月,進高祖位司空,
 餘如故。



 十一月,西魏攻陷江陵,高祖與王僧辯等進啟江州,請晉安王以太宰承制,又遣長史謝哲奉箋勸進。十二月,晉安王至自尋陽,入居朝堂,給高祖班劍二十人。



 四年五月,齊送貞陽侯深明還主社稷,王僧辯納之,即位,改元曰天成,以晉安王為皇太子。初,齊之請納貞陽也,高祖以為不可,遣使詣僧辯苦爭之,往返數四,僧辯竟不從。高祖居常憤歎,密謂所親曰:「武皇雖磐石之宗,遠布四海,至於剋雪仇恥,寧濟艱難,唯孝元而已,功
 業茂盛,前代未聞。我與王公俱受重寄,語未絕音,聲猶在耳,豈期一旦便有異圖。嗣主高祖之孫,元皇之子,海內屬目,天下宅心,竟有何辜,坐致廢黜,遠求夷狄,假立非次,觀其此情,亦可知矣。」乃密具袍數千領,及錦綵金銀,以為賞賜之具。九月壬寅,高祖召徐度、侯安都、周文育等謀之,仍部列將士,分賞金帛,水陸俱進。是夜發南徐謅討王僧辯。甲辰,高祖步軍至石頭前,遣勇士自城北踰入。時僧辯方視事,外白有兵。俄而兵自內出,僧辯
 遽走,與其第三子頠相遇,俱出閣,左右尚數十人,苦戰。高祖大兵尋至,僧辯眾寡不敵,走登城南門樓。高祖因風縱火,僧辯窮迫,乃就擒。是夜縊僧辯及頠。



 丙午,貞陽侯遜位,百僚奉晉安王上表勸進。十月己酉,晉安王即位,改承聖四年為紹泰元年。壬子,詔授高祖侍中、大都督中外諸軍事、車騎將軍、揚南徐二州刺史,持節、司空、班劍、鼓吹並如故。仍詔高祖甲仗百人,出入殿省。



 震州刺史杜龕據吳興,與義興太守韋載同舉兵反。高祖命
 周文育率眾攻載於義興,龕遣其從弟北叟將兵拒戰,北叟敗歸義興。辛未,高祖表自東討,留高州刺史侯安都、石州刺史杜棱宿衛臺省。甲戌,軍至義興。丙子,拔其水柵。秦州刺史徐嗣徽據其城以入齊,又要南豫州刺史任約共舉兵應龕、載,齊人資其兵食。嗣徽等以京師空虛,率精兵五千奄至闕下,侯安都領驍勇五百人出戰,嗣徽等退據石頭。



 丁丑,載及北叟來降,高祖撫而釋之。以嗣徽寇逼,卷甲還都,命周文育進討杜龕。



 十一月
 己卯,齊遣兵五千濟渡據姑孰。高祖命合州刺史徐度於冶城寺立柵,南抵淮渚。齊又遣安州刺史翟子崇、楚州刺史劉仕榮、淮州刺史柳達摩領兵萬人,於胡墅渡米粟三萬石、馬千匹,入於石頭。癸未,高祖遣侯安都領水軍夜襲胡墅,燒齊船千餘艘,周鐵虎率舟師斷齊運輸,擒其北徐州刺史張領州,獲運舫米數千石。仍遣韋載於大航築城,使杜棱據守。齊人又於倉門水南立二柵以拒官軍。甲辰,嗣徽等攻冶城柵,高祖領鐵騎精甲,
 出自西明門襲擊之,賊眾大潰。嗣徽留柳達摩等守城,自率親屬腹心,往南州采石,以迎齊援。十二月癸丑,高祖遣侯安都領舟師,襲嗣徽家口於秦州,俘獲數百人。官軍連艦塞淮口,斷賊水路。先是太白自十一月丙戌不見。乙卯出於東方。丙辰,高祖盡命眾軍分部甲卒,對冶城立航渡兵,攻其水南二柵。柳達摩等渡淮置陣,高祖督兵疾戰,縱火燒柵,煙塵張天。賊潰,爭舟相排擠,溺死者以千數。時百姓夾淮觀戰,呼聲震天地。軍士乘勝,
 無不一當百,盡收其船艦,賊軍懾氣。是日嗣徽、約等領齊兵水步萬餘人,還據石頭,高祖遣兵往江寧。據要險以斷賊路。賊水步不敢進,頓江寧浦口,高祖遣侯安都領水軍襲破之,嗣徽等乘單舸脫走,盡收其軍資器械。己未,官軍四面攻城,自辰訖酉,得其東北小城,及夜兵不解。庚申,達摩遣使侯子欽、劉仕榮等詣高祖請和,高祖許之,乃於城門外刑牲盟約,其將士部曲一無所問,恣其南北。辛酉,高祖出石頭南門,陳兵數萬,送齊人歸
 北者。



 壬戌,齊和州長史烏丸遠自南州奔還歷陽。江寧令陳嗣、黃門侍郎曹朗據姑孰反,高祖命侯安都、徐度等討平之,斬首數千級,聚為京觀。石頭、采石、南州悉平,收獲馬仗船米不可勝計。是月杜龕以城降。二年正月癸未,誅杜龕於吳興,龕從弟北叟、司馬沈孝敦並賜死。



 二月庚申,高祖遣侯安都、周鐵虎率舸艦備江州,仍頓梁山起柵。甲子,敕司空有軍旅之事,可騎馬出入城內。戊辰,前寧遠石城公外兵參軍王位於石頭沙際獲玉
 璽四紐,高祖表以送臺。



 三月戊戌,齊遣水軍儀同蕭軌、厙狄伏連、堯難宗、東方老、侍中裴英起、東廣州刺史獨孤辟惡、洛州刺史李希光,并任約、徐嗣徽等,率眾十萬出柵口,向梁山,帳內盪主黃叢逆擊,敗之,燒其前軍船艦,齊頓軍保蕪湖。高祖遣定州史沈泰、吳郡太守裴忌就侯安都,共據梁山以禦之。



 自去冬至是,甘露頻降於鐘山、梅崗、南澗及京口、江寧縣境,或至三數升,大如弈棋子,高祖表以獻臺。



 四月丁巳,高祖詣梁山軍巡撫。
 五月甲申,齊兵發自蕪湖,丙申,至秣陵故治。



 高祖遣周文育屯方山,徐度頓馬牧,杜棱頓大航南。己亥,高祖率宗室王侯及朝臣將帥,於大司馬門外白虎闕下刑牲告天,以齊人背約,發言慷慨,涕泗交流,同盟皆莫能仰視,士卒觀者益奮。辛丑,齊軍於秣陵故縣跨淮立橋柵,引渡兵馬。其夜至方山。侯安都、周文育、徐度等各引還京師。癸卯,齊兵自方山進及兒塘,游騎至臺。周文育、侯安都頓白土崗,旗鼓相望,都邑震駭。高祖潛撤精卒三
 千配沈泰,渡江襲齊行臺趙彥深於瓜步,獲舟艦百餘艘,陳粟萬斛。即日天子總羽林禁兵,頓于長樂寺。六月甲辰,齊兵潛至鐘山龍尾。丁未,進至莫府山。高祖遣錢明領水軍出江乘,要擊齊人糧運,盡獲其船米,齊軍於是大餒,殺馬驢而食之。庚戌,齊軍踰鐘山,高祖眾軍分頓樂遊苑東及覆舟山北,斷其衝要。壬子,齊軍至玄武湖西北莫府山南,將據北郊壇。眾軍自覆舟東移,頓郊壇北,與齊人相對。其夜大雨震電,暴風拔木,平地水丈
 餘,齊軍盡夜坐立泥中,懸鬲以爨,而臺中及潮溝北水退路燥,官軍每得番易。甲寅,少霽,高祖命眾軍秣馬蓐食,遲明攻之。乙卯旦,自率帳內麾下出莫府山南,吳明徹、沈泰等眾軍首尾齊舉,縱兵大戰,侯安都自白下引兵橫出其後,齊師大潰,斬獲數千人,相蹂藉而死者不可勝計,生執徐嗣徽及其弟嗣宗,斬之以徇。追奔至於臨沂。其江乘、攝山、鐘山等諸軍相次克捷,虜蕭軌、東方老、王敬寶、李希光、裴英起等將帥凡四十六人。其軍士
 得竄至江者,縛荻筏以濟,中江而溺,流屍至京口,翳水彌岸。丁巳,眾軍出南州,燒賊舟艦。己未,斬劉歸義、徐嗣彥、傅野豬於建康市。是日解嚴。庚申,蕭軌、東方老、王敬寶、李希光、裴英起皆伏誅。高祖表解南徐州以授侯安都。七月丙子,詔授高祖中書監、司徒、揚州刺史,進爵為公,增邑并前五千戶,侍中、使持節、都督中外諸軍事、將軍、尚書令、班劍、鼓吹、甲仗並如故,并給油幢皂輪車。是月侯瑱以江州入附。遣侯安都鎮上流,定南中諸郡。



 八
 月癸卯,太府卿何敱、新州刺史華志各上玉璽一。高祖表以送臺,詔歸之高祖。是日詔高祖食安吉、武康二縣,合五千戶。九月壬寅,改年曰太平元年。進高祖位丞相、錄尚書事、鎮衛大將軍,改刺史為牧,進封義興郡公,侍中、司徒、都督、班劍、鼓吹、甲仗、皁輪車並如故。丁未,中散大夫王彭箋稱今月五日平旦於御路見龍跡,自大社至象闕,亙三四里。庚申,詔追贈高祖考侍中、光祿大夫,加金章紫綬,封義興郡公,謚曰恭。十月甲戌,敕丞相自
 今入問訊,可施別榻以近扆坐。二年正月壬寅,天子朝萬國於太極東堂,加高祖班劍十人,并前三十人,餘如故。丁未,詔贈高祖兄道譚散騎常侍、使持節、平北將軍、南兗州刺史、長城縣公,謚曰昭烈;弟休先侍中、使持節、驃騎將軍、南徐州刺史、武康縣侯,謚曰忠壯,食邑各二千戶。甲寅,遣兼侍中謁者僕射陸繕策拜長城縣夫人章氏為義興國夫人。



 丁卯,詔贈高祖祖侍中、太常卿,謚曰孝。追封高祖祖母許氏吳郡嘉興縣君,謚曰敬;妣張
 氏義興國太夫人,謚曰宣。



 二月庚午,蕭勃舉兵,自廣州渡嶺,頓南康,遣其將歐陽頠、傅泰及其子孜為前軍,至於豫章,分屯要險,南江州刺史餘孝頃起兵應勃,高祖命周文育、侯安都率眾討平之。



 八月甲午,進高祖位太傅,加黃鉞,劍履上殿,入朝不趨,贊拜不名,并給羽葆鼓吹一部,其侍中、都督、錄尚書、鎮衛大將軍、揚州牧、義興郡公、班劍、甲仗、油幢皁輪車並如故。丙申,加高祖前後部羽葆鼓吹。是時,湘州刺史王琳擁兵不應命,高祖遣
 周文育、侯安都率眾討之。



 九月辛丑,詔曰:肇昔元胎剖判,太素氤氳,崇建人皇,必憑洪宰。故賢哲之后,牧伯征於四方,神武之君,大監治乎萬國。又有一匡九合,渠門之賜以隆,戮帶圍溫,行宮之寵斯茂,時危所以貞固,運泰所以光熙,斯乃千載同風,百王不刊之道也。太傅義興公,允文允武,乃聖乃神,固天生德,康濟黔首。昔在休期,早隆朝寄,遠踰滄海,大拯交、越。皇運不造,書契未聞,中國其亡,兵凶總至。哀哀噍類,譬彼窮牢,悠悠上天,莫
 云斯極。否終則泰,元輔應期,救此將崩,援茲已溺,乘舟履輂,架險浮深,經略中途,畢殲群醜。洎乎石頭、姑孰,流髓履腸,一朝指捴,六合清晏。



 是用光昭下武,翼亮中都,雪三后之勍仇,夷三靈之巨慝。堯台禹佐,未始能階,殷相周師,固非云擬。重之以屯剝餘象,荊楚大崩,天地無心,乘輿委御,五胡薦食,競謀諸夏,八方棋跱,莫有匡救,彊臣放命,黜我沖人,顧影於荼孺之魂,甘心於寧卿之辱。卻按下髻,求哀之路莫從,竊鈇逃責,容身之地無所。
 公神兵奄至,不日清澄,惟是孱蒙,再膺天錄。斯又巍巍蕩蕩,無德而稱焉。加以仗茲忠義,屠彼逆,震部夷氛,稽山罷昆,番禺、蠡澤,北鄙西郊,殲厥凶徒,罄無遺種。斯則兆民之命,修短所縣,率土之基,興亡是賴。於是刑禮兼訓,沿革有章,中外成平,遐邇寧一,用能使陽光合魄,曜象呈暉,棲閣遊庭,抱仁含信,宏勳該於厚地,大道格於玄天。羲、農、炎、昊以來,卷領垂衣之世,聖人濟物,未有如斯者也。



 夫備物典策,桓、文是膺,助理陰陽,蕭、曹不讓,
 未有功高於宇縣,而賞薄於伊、周,凡厥人祇,固懷延佇,是由公謙捴自牧,降損為懷,嘉數遲回,永言增歎。豈可申茲雅尚,久廢朝猷,宜戒司勳,敬升鴻典。且重華大聖。媯汭惟賢,盛德之祀無忘,公侯之門必復。是以殷嘉亶甫,繼后稷之官,堯命羲和,纂重黎之位。況其本枝攸建,宜誓山河者乎?其進公位相國,總百揆,封十郡為陳公,備九錫之禮,加璽紱,遠遊冠、綠綟綬,位在諸侯王上,其鎮衛大將軍、揚州牧如故。



 策曰:大哉乾元,資日月以貞
 觀,至哉坤元,憑山川以載物。故惟天為大,陟配者欽明,惟王建國,翼輔者齊聖。是以文、武之佐,磻溪蘊其玉璜,堯、舜之臣,榮河鏤其金版。況乎體得一之鴻姿,寧陽九之危厄,拯橫流於碣石,撲燎火於崑岑,驅馭於韋、彭,跨弩於齊、晉,神功行而靡用,聖道運而無名者乎?今將授公典策,其敬聽朕命:日者昊天不弔,鐘亂於我國家,網漏吞舟,彊胡內贔,茫茫宇宙,惵々黎元,方足圓顱,萬不遺一,太清否亢,橋山之痛已深,大寶屯如,平陽之禍相
 繼。上宰膺運,康救兆民,鞠旅於滇池之南,揚旌於桂嶺之北,懸三光於已墜,謐四海於群飛,屠猰窳於中原,斬鯨鯢於濛汜。蕩寧上國,光啟中興。此則公之大造於皇家者也。既而天未悔禍,夷醜薦臻,南夏崩騰,西京蕩覆,群胡孔熾,藉亂乘間,推納籓枝,盜假神器,冢司昏摐,旁引寇讎,既見貶於桐宮,方謀危於漢閣。



 皇運已殆,何殊贅旒,中國搖然,非徒如線。公赫然投袂,匡救本朝,復莒齊都,平戎王室。朕所以還膺寶歷,重履辰居,挹建武之
 風猷,歌宣王之雅頌。此又公之再造於皇家者也。公應務之初,登庸惟始,三川五嶺,莫不窺臨,銀洞珠宮,所在寧謐。孫、盧肇釁,越貊為災,番部阽危,勢將淪殄。公赤旗所指,祅壘洞開,白羽纔捴,兇徒粉潰。非其神武,久喪南籓。此又公之功也。大同之末,邊政不修,李賁狂迷,竊我交、愛,敢稱大號,驕恣甚於尉他,據有連州,雄豪熾於梁碩。公英謨雄算,電掃風行,馳御樓船,直跨滄海,新昌、典澈,備履艱難,蘇歷、嘉寧,盡為京觀。三山獠洞,八角蠻陬,
 逖矣水寓之鄉,悠哉火山之國,馬援之所不屆,陶璜之所未聞,莫不懼我王靈,爭朝邊候,歸賝天府,獻狀鴻臚。此又公之功也。



 自寇虜陵江,宮闈幽辱,公枕戈嘗膽,提劍拊心,氣涌青霄,神飛紫闥。而番禺連率,本自諸夷,言得其朋,是懷同惡。公仗此忠誠,乘機剿定,執沛令而釁鼓,平新野而據鞍。此又公之功也。世道初艱,方隅多難,勳門桀黠,作亂衡嶷,兵切池隍,眾兼夷獠。公以國盜邊警,知無不為,恤是同盟,誅其醜類,莫不魚驚鳥散,面縛
 頭懸。南土黔黎,重保蘇息。此又公之功也。長驅嶺嶠,夢想京畿,緣道酋豪,遞為榛梗,路養渠率,全據大都,蓄聚逋逃,方謀阻亂,百樓不戰,雲梯之所未窺,萬駑齊張,高輣之所非敵。公龍驤虎步,嘯吒風雲,山靡堅城,野無彊陣,清氛於贛石,滅沴氣於雩都。此又公之功也。遷仕凶慝,屯據大皋,乞活類馬騰之軍,流民多杜弢之眾,推鋒轉鬥,自北徂南,頻歲稽誅,實惟勍虜。公坐揮三略,遙制六奇,義勇同心,貔貅騁力,雷奔電擊,谷靜山空,列郡
 無犬吠之驚,叢祠罷狐鳴之盜。此又公之功也。王師討虜,次屆淪波,兵乏兼儲,士有飢色。公回麾蠡澤,積穀巴丘,億庾之詠斯豊,壺漿之迎是眾,軍民轉漕,曾無砥柱之難,艫舳相望,如運敖倉之府,犀渠貝胄,顧蔑雷霆,高艦層樓,仰捫霄漢,故使三軍勇銳,百戰無前,承此兵糧,遂殄凶逆。此又公之功也。若夫英圖邁俗,義旅如雲,湓壘猜攜,用淹戎略。公志唯同獎,師克在和,鵠塞非虞,鴻門是會,若晉侯之誓白水,如蕭王之推赤心,屈禮交盟,
 人祗感咽,故能使舟師並路,遠邇朋心。此又公之功也。



 姑孰襟要,崤函阻憑,寇虜據其關梁,大盜負其扃鐍。公一校裁捴,三雄並奮,左賢、右角,沙潰土崩,木甲殪於中原,氈裘赴於江水,他他藉藉,萬計千群,鄂阪之隘斯開,夷庚之道無塞。此又公之功也。義軍大眾,俱集帝京,逆豎凶徒,猶屯皇邑。若夫表裏山河,金湯險固,疏龍首以抗殿,揃華岳以為城,雜虜憑焉,彊兵自若。公回茲地軸,抗此天羅,曾不崇朝,俾無遺噍,軍容甚穆,國政方脩,物
 重睹於衣冠,民還瞻於禮樂,楚人滿道,爭睹於葉公,漢老銜悲,俱歡於司隸。此又公之功也。內難初靜,諸侯出關,外郡傳烽,鮮卑犯塞,莫非且渠、當戶,中貴名王,冀馬迾於淮南,胡笳動於徐北。公舟師步甲,亙野橫江,殲厥群羝,遂殫封豨,莫不絓木而止,戎車靡遺,遇濘而旋,歸驂盡殪。此又公之功也。公克黜禍難,劬勞皇室,而孫寧之黨,翻啟狄心,伊、洛之間,咸為虜戍,雖金陵佳氣,石壘天嚴,朝闇戎塵,夜喧胡鼓。公三籌既畫,八陣斯張,裁舉
 靈鉟,亦抽金僕,咸俘醜類,悉反高墉,異李廣之皆誅,同龐元之盡赦。此又公之功也。任約叛渙,梟聲不悛,戎羯貪婪,狼心無改,穹廬氈幕,抵北闕而為營,烏孫天馬,指東都而成陣。公左甄右落,箕張翼舒,掃是攙槍,驅其獫狁,長狄之種埋於國門,椎髻之酋烹於軍市,投秦坑而盡沸,噎濉水而不流。此又公之功也。一相居中,自折彝鼎,五湖小守,妄懷同惡。公夙駕兼道,秉羽杖戈,玉斧將揮,金鉦且戒,妖酋震懾,遽請灰釘,爇櫬以表其含弘,焚
 書以安其反側。此又公之功也。賊龕凶橫,陵虐具區,阻兵安忍,憑災怙亂,自古蟲言鳥跡,渾沌洪荒,凡或虔劉,未此殘酷。公雖宗居汝潁,世寓東南,育聖誕賢之鄉,含章挺生之地,眷言桑梓,公私憤切,卓爾英狀,丞規奉算,戮此大憝,如烹小鮮。此又公之功也。亂離永久,群盜孔多,浙左凶渠,連兵構逆,豈止千兵、五校、白雀、黃龍而已哉!公以中軍無率,選是親賢,奸寇途窮,涔然冰泮,刑溏之所,文命動其大威,雷門之間,句踐行其嚴戮,英
 規聖跡,異代同風。此又公之功也。同姓有扈,頑凶不賓,憑藉宗盟,圖危社稷,觀兵匯澤,勢震京師,驅率南蠻,已為東帝。公論兵於朝堂之上,決勝於樽俎之間,寇、賈、樊、滕,浮江下瀨,一朝揃撲,無待甸師,萬里澄清,非勞新息。此又公之功也。



 豫章妖寇,依憑山澤,繕甲完聚,多歷歲時,結從連橫,爰洎交、廣。呂嘉既獲,吳濞已鏦,命我還師,徵其不恪,連營盡拔,偽黨斯擒,曜聖武於匡山,回神旌於蠡派。此又公之功也。自八紘九野,瓜剖豆分,竊帝偷
 王,連州比縣。公武靈已暢,文德又宣,折簡馳書,風猷斯遠,至於蒼蒼浴日,杳杳無雷,北洎丈夫之鄉,南踰女子之國,莫不屈膝膜拜,求吏款關。此又公之功也。京師禍亂,亟積寒暄,雙闕低昂,九門寥豁。寧秦宮之可顧,豈魯殿之猶存!五都簪弁,百僚卿士,胡服縵纓,咸為戎俗,高冠厚履,希復華風,宋微子《麥遂》之歌,周大夫《黍離》之歎,方之於斯,未足為悲矣。公求衣昧旦,昃食高舂,興構宮闈,具瞻遐邇,郊癢稷宗之典,六符十等之章,還聞太始
 之風流,重睹永平之遺事。此又公之功也。公有濟天下之勳,重之以明德,凝神體道,合德符天,用百姓以為心,隨萬機而成務,恥一物非唐、虞之民,歸含靈於仁壽之域,上德不德,無為以為,夏長春生,顯仁藏用,忠信為寶,風雨弗愆,仁惠為基,牛羊勿踐,功成治定,樂奏《咸》、《雲》,安上治民,禮兼文質,物色丘園,衣裾里巷,朝多君子,野無遺賢,菽粟同水火之饒,工商富猗頓之旅。是以天無蘊寶,地有呈祥,潏露卿雲,朝團曉映,山車澤馬,服馭登閑,
 既景煥於圖書,方葳蕤於史諜。高勳踰於象緯,積德冠於嵩、華,固無德而稱者矣。朕又聞之,前王宰世,茂賞尊賢,式樹籓長,總征群伯,《二南》崇絕,四履遐曠,泱泱表海,祚土維齊,巖巖泰山,俾侯於魯;抑又勤王反鄭,夾輔遷周,召伯之命斯隆,河陽之禮咸備;況復經營宇宙,寧唯斷鰲足之功,弘濟蒼生,非直鑿龍門之險;而疇庸報德,寂爾無聞,朕所以垂拱當宁,載懷慚悸者也。今授公相國,以南豫州之陳留、南丹陽、宣城,揚州之吳興、東陽、新
 安、新寧,南徐州之義興,江州之鄱陽、臨川十郡,封公為陳公。錫茲青土,苴以白茅,爰定爾邦,用建塚社。昔旦、奭分陜,俱為保師,晉、鄭諸侯,咸作卿士,兼其內外,禮實攸宜。今命使持節兼太尉王通授相國印綬、陳公璽紱。使持節兼司空王瑒授陳公茅土,金虎符第一至第五左,竹使符第一至第十左。相國秩踰三鉉,任總百司,位絕朝班,禮由事革。其以相國總百揆,除錄尚書之號,上所假節侍中貂蟬、中書監印章、中外都督太傅印綬、義興公
 印策,其鎮衛大將軍、揚州牧如故。又加公九錫,其敬聽後命:以公禮為楨幹,律等銜策,四維皆舉,八柄有章,是用錫公大輅、戎輅各一,玄牡二駟。以公賤寶崇穀,疏爵待農,室富京坻,民知榮辱,是用錫公袞冕之服,赤舄副焉。以公調理陰陽,燮諧風雅,三靈允降,萬國同和,是用錫公軒縣之樂,六佾之舞。以公宣導王猷,弘闡風教,光景所照,鞮象必通,是用錫公朱戶以居。



 以公抑揚清濁,褒德進賢,髦士盈朝,幽人虛谷,是用錫公納陛以登。以
 公嶷然廊廟,為世鎔範,折衝四表,臨御八荒,是用錫公武賁之士三百人。以公執茲明罰,期在刑措,象恭無赦,干紀必誅,是用錫公斧、鉞各一。以公英猷遠量,跨厲嵩溟,包一車書,括囊寰宇,是用錫公彤弓一、彤矢百、甗弓十、甗矢千。以公天經地義,貫徹幽明,春露秋霜,允恭粢盛,是用錫公秬鬯一卣,圭瓚副焉。陳國置丞相已下,一遵舊式。往欽哉!其恭循朕命,克相皇天,弘建邦家,允興洪業,以光我高祖之休命!



 十月戊辰,進高祖爵為王,以
 揚州之會稽、臨海、永嘉、建安,南徐州之晉陵、信義,江州之尋陽、豫章、安成、廬陵并前為二十郡,益封陳國。其相國、揚州牧、鎮衛大將軍並如故。又命陳王冕十有二旒,建天子旌旗,出警入蹕,乘金根車,駕六馬,備五時副車,置旄頭雲罕,樂舞《八佾》,設鐘虡宮縣。王妃、王子、王女爵命之號,陳臺百官,一依舊典。辛未,梁帝禪位於陳,詔曰:五運更始,三正迭代,司牧黎庶,是屬聖賢,用能經緯乾坤,彌綸區宇,大庇黔首,闡揚鴻烈。革晦以明,積代同軌,
 百王踵武,咸由此則。梁德湮微,禍亂薦發,太清云始,見困長蛇,承聖之季,又罹封豕。爰至天成,重竊神器,三光亟沈,七廟乏祀,含生已泯,鼎命斯墜,我武、元之祚,有如綴旒,靜惟屯剝,夕惕載懷。



 相國陳王,有命自天,降神惟獄,天地合德,晷曜齊明,拯社稷之橫流,提億兆之塗炭,東誅逆叛,北殲獯醜,威加四海,仁漸萬國,復張崩樂,重興絕禮,儒館聿脩,戎亭虛候,大功在舜,盛績惟禹,巍巍蕩蕩,無得而稱。來獻白環,豈直皇虞之世,入貢素雉,非
 止隆周之日。固以效珍川陸,表瑞煙雲,甘露醴泉,旦夕凝涌,嘉禾朱草,孳植郊甸。道昭於悠代,勳格於皇穹,明明上天,光華日月,革故著於玄象,代德彰於圖讖,獄訟有歸,謳歌爰適,天之歷數,實有攸在。朕雖庸貌,闇於古昔,永稽崇替,為日已久,敢忘列代之遺典,人祇之至願乎。今便遜位別宮,敬禪於陳,一依唐、虞、宋、齊故事。



 策曰:咨爾陳王:惟昔上古,厥初生民,驪連、栗陸之前,容成、大庭之代,並結繩寫鳥,杳冥慌忽,故靡得而詳焉。自羲、農、
 軒、昊之君,陶唐、有虞之主,或垂衣而御四海,或無為而子萬姓,居之如馭朽索,去之如脫敝屣。裁遇許由,便能捨帝,暫逢善卷,即以讓王。故知玄扈璇璣,非關尊貴,金根玉輅,示表君臨。及南觀河渚,東沈刻璧,精華既竭,耄勤已倦,則抗首而笑,唯賢是與,言勞然作歌,簡能斯授,遺風餘烈,昭晰圖書。漢、魏因循,是為故實。宋、齊授受,又弘斯義。



 我高祖應期撫運,握樞御宇,三后重光,祖宗齊聖。及時屬陽九,封豕薦食,西都失馭,夷狄交侵,乃皋天成,
 輕弄龜鼎,喋喋黔首,若崩厥角,徽徽皇極,將甚綴旒。惟王乃聖乃神,欽明文思,二儀並運,四時合序,天錫智勇,人挺雄傑,珠庭日角,龍行武步,爰初投袂,日乃勤王,電掃番禺,雲撤彭蠡,揃其元惡,定我京畿。及王賀帝弘,貿茲冠屨,既行伊、霍,用保沖人。震澤、稽陰,並懷叛逆,獯羯醜虜,三亂皇都,裁命偏師,二邦自殄,薄伐獫狁,六戎盡殪。嶺南叛渙,湘、郢結連,賊帥既擒,凶渠傳首,用能百揆時序,四門允穆,無思不服,無遠不屆,上達穹昊,下漏淵
 泉,蛟魚並見,謳歌攸屬。況乎長彗橫天,已徵布新之兆,璧日斯既,實表更姓之符。是以始創義師,紫雲曜彩,肇惟尊主,黃龍負舟。苦矢素翬,梯山以至,白環玉玦,慕德而臻。若夫安國字萌,本因萬物之志,時乘御宇,良會樂推之心。七百無常期,皇王非一族,昔木德既季,而傳祚於我有梁,天之歷數,允集明哲。式遵前典,廣詢群議,王公卿尹,莫不攸屬,敬從人祇之願,授帝位於爾躬。四海困窮,天祿永終,王其允執厥中,軌儀前式,以副溥天之
 望。禋祀上帝,時膺大禮,永固洪業,豈不盛歟!



 又璽書曰:君子者自昭明德,達人者先天弗違,故能進退咸亨,動靜元吉。朕雖蒙寡,庶乎景行。何則?三才剖判,九有區分,情性相乖,亂離云起,是以建彼司牧,推乎聖賢,授受者任其時來,皇王者本非一族,人謀是與,屈己從萬物之心,天意斯歸,鞠躬奉百靈之命。謳歌所往,則攘袂以膺之,菁華已竭,乃褰裳而去之。昔在唐、虞,鑒於天道,舉其黎獻,授彼明哲,雖復質文殊軌,沿革不同,歷代因循,斯
 風靡替。我大梁所以考庸太室,接禮貳宮,月正元日,受終文祖。但運不常夷,道無恒泰,山岳傾偃,河海沸騰,電目雷聲之禽,鉤爪鋸牙之獸,咀齧含生,不知紀極。



 二后英聖,相仍在天,六夷貪狡,爭侵中國,縣王都帝,人懷干紀,一民尺土,皆非梁地。朕以不造,幼罹閔凶,仰憑衡佐,亟移年序。周成、漢惠,邈矣無階,惟是童蒙,必貽顛蹶。若使時無聖哲,世靡艱難,猶當高蹈於滄洲,自求於泰伯者矣。



 惟王應期誕秀,開籙握圖,性道故其難聞,嘉庸
 已其被物,乾行同其燾覆,日御比其貞明,登承聖於復禹之功,樹鞠子於興周之業,滅陸渾於伊、洛,殲驪戎於鎬京,大小二震之驍徒,東南兩越之勍寇,遽行天討,無遺神策。於是祖述堯舜,憲章文武,大樂與天地同和,大禮與天地同節,鼓之以雷霆,潤之以風雨,仁霑葭葦,信及豚魚,殷牖斯空,夏臺虛設,民惟大畜,野有同人,升平頌平,無偏無黨,固以雲飛紫蓋,水躍黃龍,東伐西征,晻映川陸。榮光曖曖,已冒郊廛,甘露瀼瀼,亟流庭苑。車轍馬
 跡,誰不率從?蟠水流沙,誰不懷德?祥圖遠至,非唯赤伏之符,靈命昭然,何止黃星之氣。海口河目,賢聖之表既彰,握旄執鉞,君人之狀斯偉。



 且自攝提無紀,孟陬殄滅,枉矢宵飛,天弧曉映,久矣夷羊之在牧,時哉蛟龍之出泉。革運之兆咸徵,惟新之符並集,朕所以欽若勛、華,屢回星琯。昔者水運斯盡,予高祖受焉。今歷去炎精,神歸樞紐,敬以火德,傳于爾陳。遠鑒前王,近謀群辟,明靈有悅,率土同心。今遣使持節兼太保侍中尚書左僕射平
 樂亭侯王通,兼太尉司徒左長史王錫奉皇帝璽綬。受終之禮,一依唐、虞故事。王其時陟元后,寧育兆民,光闡洪猷,以承昊天之休命!



 是日,梁帝遜于別宮。高祖謙讓再三,群臣固請,乃許。



\end{pinyinscope}