\article{卷十八列傳第十二沈眾 袁泌 劉仲威 陸山才 王質 韋載 族弟翽}

\begin{pinyinscope}

 沈眾,字仲師,吳興武康人也。祖約,梁特進。父旋,梁給事黃門侍郎。眾好學,頗有文詞,起家梁鎮衛南平王法曹
 參軍、太子舍人。是時,梁武帝制《千字詩》,眾為之注解。與陳郡謝景同時召見于文德殿,帝令眾為《竹賦》,賦成,奏,帝善之,手敕答曰:「卿文體翩翩,可謂無忝爾祖。」當陽公蕭大心為郢州刺史,以眾為限內記室參軍。尋除鎮南湘東王記室參軍。遷太子中舍人,兼散騎常侍。聘魏,還,遷驃騎廬陵王諮議參軍,舍人如故。



 侯景之亂,眾表於梁武,稱家代所隸故義部曲,並在吳興,求還召募以討賊,梁武許之。及景圍臺城,眾率宗族及義附五千餘人,
 入援京邑,頓于小航,對賊東府置陣,軍容甚整,景深憚之。梁武於城內遙授眾為太子右衛率。京城陷,眾降於景。景平,西上荊州,元帝以為太子中庶子、本州大中正。尋遷司徒左長史。江陵陷,為西魏所虜,尋而逃還,敬帝承制授御史中丞。紹泰元年,除侍中,遷左民尚書。高祖受命,遷中書令,中正如故。高祖以眾州里知名,甚敬重之,賞賜優渥,超於時輩。



 眾性吝嗇,內治產業,財帛以億計,無所分遺。其自奉養甚薄,每於朝會之中,衣裳破裂,
 或躬提冠屨。永定二年,兼起部尚書,監起太極殿。恆服布袍芒屩,以麻繩為帶,又攜乾魚蔬菜飯獨啖之,朝士共誚其所為。眾性狷急,於是忿恨,遂歷詆公卿,非毀朝廷。高祖大怒,以眾素有令望,不欲顯誅之,後因其休假還武康,遂於吳中賜死,時年五十六。



 袁泌,字文洋,左光祿大夫敬之弟也。清正有幹局,容體魁岸,志行修謹。釋褐員外散騎侍郎,歷諸王府佐。



 侯景之亂,泌欲求為將。是時泌兄君正為吳郡太守,梁簡文
 板泌為東宮領直,令往吳中召募士卒。及景圍臺城,泌率所領赴援。京城陷,退保東陽,景使兵追之,乃自會稽東嶺出湓城,依於鄱陽嗣王蕭範。範卒,泌乃降景。



 景平,王僧辯表泌為富春太守,兼丹陽尹。貞陽侯僭位,以泌為侍中,奉使於齊。高祖受禪,王琳據有上流,泌自齊從梁永嘉王蕭莊達琳所。及莊僭立,以泌為侍中、丞相長史。天嘉二年,泌與琳輔莊至於柵口,琳軍敗,眾皆奔散,唯泌獨乘輕舟送莊達於北境,屬莊於御史中丞劉仲
 威,令共入齊,然後拜辭而歸,詣闕請罪,文帝深義之。



 尋授寧遠始興王府法曹參軍,轉諮議參軍,除通直散騎常侍,兼侍中,領豫州大中正。聘於周,使還,授散騎常侍,御史中丞,其中正如故。高宗入輔,以泌為雲旗將軍、司徒左長史。光大元年卒,年五十八。臨終戒其子蔓華曰:「吾於朝廷素無功績,瞑目之後,斂手足旋葬,無得輒受贈謚。」其子述泌遺意,表請之,朝廷不許,贈金紫光祿大夫,謚曰質。



 劉仲威,南陽涅陽人也。祖虯,齊世以國子博士徵,不就。父之遲,荊州治中從事史。仲威少有志氣,頗涉文史。梁丞聖中為中書侍郎。蕭莊偽署御史中丞,隨莊入齊,終於鄴中。



 仲威從弟廣德,亦好學,負才任氣。父之亨,梁安西湘東王長史、南郡太守。



 廣德承聖中以軍功官至給事黃門侍郎、湘東太守。荊州陷後,依於王琳。琳平,文帝以廣德為寧遠始興王府限外記室參軍,仍領其舊兵。尋為太尉侯瑱湘州府司馬,歷樂山、豫章二郡太守,新
 安內史。光大中,假節、員外散騎常侍、雲旗將軍、河東太守。太建元年卒於郡,時年四十三,贈左衛將軍。



 陸山才,字孔章,吳郡吳人也。祖翁寶,梁尚書水部郎。父泛,散騎常侍。山才少倜儻,好尚文史,范陽張纘,纘弟綰,並欽重之。起家王國常侍,遷外兵參軍。



 尋以父疾,東歸侍養。承聖元年,王僧辯授山才儀同府西曹掾。高祖誅僧辯,山才奔會稽依張彪。彪敗,乃歸高祖。



 紹泰中,都督周文育出鎮南豫州,不知書疏,乃以山才為長史,政事
 悉以委之。



 文育南討,剋蕭勃,擒歐陽頠,計畫多出山才。及文育西征王琳,留山才監江州事,仍鎮豫章。文育與侯安都於沌口敗績,餘孝頃自新林來寇豫章,山才收合餘眾,依於周迪。擒餘孝頃、李孝欽等,遣山才自都陽之樂安嶺東道送於京師。除中書侍郎。



 復由樂安嶺綏撫南川諸郡。



 文育重鎮豫章金口,山才復為貞威將軍、鎮南長史、豫章太守。文育為熊曇朗所害,曇朗囚山才等,送於王琳。未至,而侯安都敗琳將常眾愛於宮亭湖,
 由是山才獲反,除貞威將軍、新安太守。為王琳未平,留鎮富陽,以捍東道。入為員外散騎常侍,遷宣惠始興王長史,行東揚州事。



 侯安都討留異,山才率王府之眾從焉。異平,除明威將軍、東陽太守。入為鎮東始興王長史,帶會稽郡丞,行東揚州事。未拜,改授散騎常侍,兼度支尚書,滿歲為真。



 高宗南征周迪,以山才為軍司。迪平,復職。餘孝頃自海道襲晉安,山才又以本官之會稽,指授方略。還朝,坐侍宴與蔡景歷言語過差,為有司所奏,免
 官。尋授散騎常侍,遷雲旗將軍、西陽武昌二郡太守。天康元年卒,時年五十八。贈右衛將軍,謚曰簡子。



 王質,字子貞,右光祿大夫通之弟也。少慷慨,涉獵書史。梁世以武帝甥封甲口亭侯,補國子《周易》生,射策高第。起家秘書郎、太子舍人、尚書殿中郎。遭母憂,居喪以孝聞。服闋,除太子洗馬、東宮領直。累遷中舍人、庶子。



 太清元年,除假節、寧遠將軍,領東宮兵,從貞陽侯北伐。及貞陽敗績,質脫身逃還。侯景於壽陽構逆,質又領舟師隨
 眾軍拒之。景軍濟江,質便退走。尋領步騎頓于宣陽門外。景軍至京師,質不戰而潰,乃翦髮為桑門,潛匿人間。及柳仲禮等會援京邑,軍據南岸,質又收合餘眾從之。京城陷後,西奔荊州,元帝承制,以質為右長史,帶河東太守。俄遷侍中。尋出為持節、都督吳州諸軍事、寧遠將軍、吳州刺史,領鄱陽內史。荊州陷,侯瑱鎮於湓城,與質不協,遣偏將羊亮代質,且以兵臨之,質率所部度信安嶺,依於留異。文帝鎮會稽,以兵助質,令鎮信安縣。



 永定
 二年,高祖命質率所部踰嶺出豫章,隨都督周文育以討王琳。質與琳素善,或譖云於軍中潛信交通,高祖命周文育殺質,文育啟請救之,獲免。尋授散騎常侍、晉陵太守。



 文帝嗣位,徵守五兵尚書。高宗為揚州刺史,以質為仁威將軍、驃騎府長史。



 天嘉二年,除晉安太守。高宗輔政,以為司徒左長史,將軍如故。坐公事免官。尋為通直散騎常侍,遷太府卿、都官尚書。太建二年卒,時年六十。贈本官,謚曰安子。



 韋載,字德基,京兆杜陵人也。祖叡,梁開府儀同三司,永昌嚴公。父政,梁黃門侍郎。載少聰惠,篤志好學。年十二,隨叔父棱見沛國劉顯,顯問《漢書》十事,載隨問應答,曾無疑滯。及長,博涉文史,沉敏有器局。起家梁邵陵王法曹參軍,遷太子舍人、尚書三公郎。



 侯景之亂,元帝承制以為中書侍郎。尋為建威將軍、尋陽太守,隨都督王僧辯東討侯景。是時僧辯軍於湓城,而魯悉達、樊俊等各擁兵保境,觀望成敗。元帝以載為假節、都督太原、高唐、
 新蔡三郡諸軍事、高唐太守。仍銜命喻悉達等令出軍討景。及大軍東下,載率三郡兵自焦湖出柵口,與僧辯會於梁山。景平,除冠軍將軍、琅邪太守。尋奉使往東陽、晉安,招撫留異、陳寶應等。仍授信武將軍、義興太守。



 高祖誅王僧辨,乃遣周文育輕兵襲載,未至而載先覺,乃嬰城自守。文育攻之甚急,載所屬縣卒並高祖舊兵,多善用弩,載收得數十人,繫以長鎖,命所親監之,使射文育軍,約曰十發不兩中者則死,每發輒中,所中皆斃。文
 育軍稍卻,因於城外據水立柵,相持數旬。高祖聞文育軍不利,乃自將征之,剋其水柵。仍遣載族弟翽齎書喻載以誅王僧辯意,并奉梁敬帝敕,敕載解兵。載得書,乃以其眾降於高祖。



 高祖厚加撫慰,即以其族弟翽監義興郡,所部將帥,並隨才任使,引載恒置左右,與之謀議。



 徐嗣徽、任約等引齊軍濟江,據石頭城,高祖問計於載,載曰:「齊軍若分兵先據三吳之路,略地東境,則時事去矣。今可急於淮南即侯景故壘築城,以通東道轉輸,別命
 輕兵絕其糧運,使進無所虜,退無所資,則齊將之首,旬日可致。」高祖從其計。



 永定元年,除和戎將軍、通直散騎常侍。二年,進號輕車將軍。尋加散騎常侍、太子右衛率,將軍如故。天嘉元年,以疾去官。載有田十餘頃,在江乘縣之白山,至是遂築室而居,屏絕人事,吉凶慶弔,無所往來,不入籬門者幾十載。太建中卒於家,時年五十八。



 載族弟翽。翽字子羽,少有志操。祖愛,梁輔國將軍。父乾向,汝陰太守。翽弱冠喪父,哀毀甚至,養母、撫孤兄弟子,以
 仁孝著稱。高祖為南徐州刺史,召為征北參軍,尋監義興郡。永定元年,授貞毅將軍、步兵校尉。遷驍騎將軍,領朱衣直閣。驍騎之職,舊領營兵,兼統宿衛。自梁代已來,其任踰重,出則羽儀清道,入則與二衛通直,臨軒則升殿俠侍。翽素有名望,每大事恒令俠侍左右,時人榮之,號曰「俠御將軍」。尋出為宣城太守。天嘉二年,預平王琳之功,封清源縣侯,邑二百戶。太建中卒官,贈明、霍、羅三州刺史。子宏,字德禮,有文學,歷官至永嘉王府諮議參
 軍。陳亡入隋。



 史臣曰:昔鄧禹基於文學,杜預出自儒雅,卒致軍功,名著前代。晉氏喪亂,播遷江左,顧榮、郗鑒之輩,溫嶠、謝玄之倫,莫非巾褐書生,晉紳素譽,抗敵以衛社稷,立勛而升臺鼎。自斯以降,代有其人。但梁室沸騰,懦夫立志,既身逢際會,見仗於時主,美矣!



\end{pinyinscope}