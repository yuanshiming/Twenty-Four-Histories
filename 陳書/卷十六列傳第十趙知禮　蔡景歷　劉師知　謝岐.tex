\article{卷十六列傳第十趙知禮 蔡景歷 劉師知 謝岐}

\begin{pinyinscope}

 趙知禮,字齊旦,天水隴西人也。父孝穆,梁候官令。知禮涉獵文史,善隸書。



 高祖之討元景仲也,或薦之,引為記室參軍。知禮為文贍速,每占授軍書,下筆便就,率皆稱
 旨。由是恒侍左右,深被委任,當時計劃,莫不預焉。知禮亦多所獻替。



 高祖平侯景,軍至白茅灣,上表於梁元帝及與王僧辯論述軍事,其文並知禮所製。



 侯景平,授中書侍郎,封始平縣子,邑三百戶。高祖為司空,以為從事中郎。



 高祖入輔,遷給事黃門侍郎,兼衛尉卿。高祖受命,遷通直散騎常侍,直殿省。尋遷散騎常侍,守太府卿,權知領軍事。天嘉元年,進爵為伯,增邑通前七百戶。王琳平,授持節、督吳州諸軍事、明威將軍、吳州刺史。



 知禮沈靜
 有謀謨,每軍國大事,世祖輒令璽書問之。秩滿,為明威將軍、太子右衛率。遷右衛將軍,領前軍將軍。六年卒,時年四十七。詔贈侍中,謚曰忠。子允恭嗣。



 蔡景歷,字茂世,濟陽考城人也。祖點,梁尚書左民侍郎。父大同,輕車岳陽王記室參軍,掌京邑行選。景歷少俊爽,有孝行。家貧好學,善尺牘,工草隸。解褐諸王府佐,出為海陽令,為政有能名。侯景亂,梁簡文帝為景所幽,景歷與南康嗣王蕭會理謀,欲挾簡文出奔,事泄見執,賊
 黨王偉保護之,獲免。因客遊京口。



 侯景平,高祖鎮朱方,素聞其名,以書要之。景歷對使人答書,筆不停綴,文不重改。曰:蒙降札書,曲垂引逮,伏覽循回,載深欣暢。竊以世求名駿,行地能致千里,時愛奇寶,照車遂有徑寸。但《雲》、《咸》斯奏,自輟《巴渝》,杞梓方雕,豈盼樗櫪。仰惟明將軍使君侯節下,英才挺茂,雄姿秀拔,運屬時艱,志匡多難,振衡嶽而綏五嶺,滌贛源而澄九派,帶甲十萬,彊弩數千,誓勤王之師,總義夫之力,鯨鯢式剪,役不踰時,氛霧
 廓清,士無血刃。雖漢誅祿、產,舉朝實賴絳侯,晉討約、峻,中外一資陶牧,比事論功,彼奚足算。加以抗威兗服,冠蓋通于北門,整旆徐方,詠歌溢于東道,能使邊亭臥鼓,行旅露宿,巷不拾遺,市無異價,洋洋乎功德政化,曠古未儔,諒非膚淺所能殫述。是以天下之人,向風慕義,接踵披衿,雜遝而至矣。或帝室英賢,貴遊令望,齊、楚秀異,荊、吳岐嶷。武夫則猛氣紛紜,雄心四據,陸拔山嶽,水斷虯龍,六鈞之弓,左右馳射,萬人之劍,短兵交接,攻壘若
 文鴦,焚艦如黃蓋,百戰百勝,貔貅為群。文人則通儒博識,英才偉器,雕麗暉煥,摛掞絢藻,子雲不能抗其筆,元瑜無以高其記,尺翰馳而聊城下,清談奮而嬴軍卻。復有三河辯客,改哀樂於須臾,六奇謀士,斷變反於倏忽。治民如子賤,踐境有成,折獄如仲由,片辭從理。直言如毛遂,能厲主威,銜使若相如,不辱君命。懷忠抱義,感恩徇己,誠斷黃金,精貫白日,海內雄賢,牢籠斯備。明將軍徹鞍下馬,推案止食,申爵以榮之,築館以安之,輕財重
 氣,卑躬厚士,盛矣哉!盛矣哉!



 抑又聞之,戰國將相,咸推引賓遊,中代岳牧,並盛延僚友,濟濟多士,所以成將軍之貴。但量能校實,稱才任使,員行方止,各盡其宜,受委責成,誰不畢力。



 至如走賤,亡庸人耳。秋冬讀書,終慚專學,刀筆為吏,竟闕異等。衡門衰素,無所聞達,薄宦輕資,焉能遠大。自陽九遘屯,天步艱阻,同彼貴仕,溺於巨寇,亟鄰危殆,備踐薄冰。今王道中興,殷憂啟運,獲存微命,足為幸甚,方歡飲啄,是謂來蘇。然皇鑾未反,宛、洛蕪曠,
 四壁固三軍之餘,長夏無半菽之產,遨遊故人,聊為借貸,屬此樂土,洵美忘歸。竊服高義,暫謁門下,明將軍降以顏色,二三士友假其餘論,菅蒯不棄,折簡賜留,欲以雞鶩廁鴛鴻於池沼,將移瓦礫參金碧之聲價。昔折脅遊秦,忽逢盼採,簷簦入趙,便致留連,今雖羈旅,方之非匹,樊林之賁,何用克堪。但眇眇纖蘿,憑喬松以自聳,蠢蠢輕蚋,託驂尾而遠騖。竊不自涯,願備下走,且為腹背之毛,脫充鳴吠之數,增榮改觀,為幸已多。海不厭深,山
 不讓高,敢布心腹,惟將軍覽焉。



 高祖得書,甚加欽賞。仍更賜書報答,即日板征北府中記室參軍,仍領記室。



 衡陽獻王昌時為吳興郡,昌年尚少,吳興王之鄉里,父老故人,尊卑有數,高祖恐昌年少,接對乖禮,乃遣景歷輔之。承聖中,授通直散騎侍郎,還掌府記室。



 高祖將討王僧辯,獨與侯安都等數人謀之,景歷弗之知也。部分既畢,召令草檄,景歷援筆立成,辭義感激,事皆稱旨。僧辯誅,高祖輔政,除從事中郎,掌記室如故。紹泰元年,遷給事
 黃門侍郎,兼掌相府記室。高祖受禪,遷秘書監,中書通事舍人,掌詔誥。永定二年,坐妻弟劉淹詐受周寶安餉馬,為御史中丞沈炯所劾,降為中書侍郎,舍人如故。



 三年,高祖崩,時外有彊寇,世祖鎮于南皖,朝無重臣,宣后呼景歷及江大權、杜棱定議,乃秘不發喪,疾召世祖。景歷躬共宦者及內人,密營斂服。時既暑熱,須治梓宮,恐斤斧之聲或聞于外,仍以蠟為祕器。文書詔誥,依舊宣行。世祖即位,復為秘書監,舍人如故。以定策功,封新豊
 縣子,邑四百戶。累遷散騎常侍。世祖誅侯安都,景歷勸成其事。天嘉三年,以功遷太子左衛率,進爵為侯,增邑百戶,常侍、舍人如故。六年,坐妻兄劉洽依倚景歷權勢,前後姦訛,并受歐陽武威餉絹百匹,免官。



 廢帝即位,起為鎮東鄱陽王諮議參軍,兼太府卿。華皎反,以景歷為武勝將軍、吳明徹軍司。皎平,明徹於軍中輒戮安成內史楊文通,又受降人馬仗有不分明,景歷又坐不能匡正,被收付治。久之,獲宥,起為鎮東鄱陽王諮議參軍。



 高
 宗即位,遷宣惠豫章王長史,帶會稽郡守,行東揚州府事。秩滿,遷戎昭將軍、宣毅長沙王長史、尋陽太守,行江州府事,以疾辭,遂不行。入為通直散騎常侍、中書通事舍人,掌詔誥,仍復封邑。遷太子左衛率,常侍、舍人如故。



 太建五年,都督吳明徹北伐,所向克捷,與周將梁士彥戰於呂梁,大破之,斬獲萬計,方欲進圖彭城。是時高宗銳意河南,以為指麾可定,景歷諫稱師老將驕,不宜過窮遠略。高宗惡其沮眾,大怒,猶以朝廷舊臣,不深罪責,
 出為宣遠將軍、豫章內史。未行,為飛章所劾,以在省之日,贓汙狼藉,帝令有司按問,景歷但承其半。於是御史中丞宗元饒奏曰:「臣聞士之行己,忠以事上,廉以持身,茍違斯道,刑茲罔赦。謹按宣遠將軍、豫章內史新豊縣開國侯景歷,因藉多幸,豫奉興王,皇運權輿,頗參締構。天嘉之世,贓賄狼藉,聖恩錄用,許以更鳴,裂壤崇階,不遠斯復。不能改節自勵,以報曲成,遂乃專擅貪汙,彰於遠近,一則已甚,其可再乎?宜置刑書,以明秋憲。臣等參
 議,以見事免景歷所居官,下鴻臚削爵土。謹奉白簡以聞。」詔曰「可。」於是徙居會稽。及吳明徹敗,帝思景歷前言,即日追還,復以為征南鄱陽王諮議參軍。數日,遷員外散騎常侍,兼御史中丞,復本封爵,入守度支尚書。舊式拜官在午後,景歷拜日,適值輿駕幸玄武觀,在位皆侍宴,帝恐景歷不豫,特令早拜,其見重如此。



 是歲,以疾卒官,時年六十。贈太常卿,謚曰敬。十三年,改葬,重贈中領軍。



 禎明元年,配享高祖廟庭。二年,輿駕親幸其宅,重贈
 景歷侍中、中撫將軍,謚曰忠敬,給鼓吹一部,并於墓所立碑。



 景歷屬文,不尚雕靡,而長於敘事,應機敏速,為當世所稱。有文集三十卷。



 劉師知,沛國相人也。家世素族。祖奚之,齊晉安王諮議參軍,淮南太守,有能政,齊武帝手詔頻褒賞。父景彥,梁尚書左丞、司農卿。師知好學,有當世才。



 博涉書史,工文筆,善儀體,臺閣故事,多所詳悉。梁世歷王府參軍。紹泰初,高祖入輔,以師知為中書舍人,掌詔誥。是時兵亂之
 後,禮儀多闕,高祖為丞相及加九錫并受禪,其儀注並師知所定焉。高祖受命,仍為舍人。性疏簡,與物多忤,雖位宦不遷,而委任甚重,其所獻替,皆有弘益。



 及高祖崩,六日成服,朝臣共議大行皇帝靈座俠御人所服衣服吉凶之制,博士沈文阿議,宜服吉服。師知議云:「既稱成服,本備喪禮,靈筵服物,皆悉縞素。



 今雖無大行俠御官事,按梁昭明太子薨,成服俠侍之官,悉著縗斬,唯著鎧不異,此即可擬。愚謂六日成服,俠靈座須服縗絰。」中書
 舍人蔡景歷亦云:「雖不悉準,按山陵有凶吉羽儀,成服唯凶無吉,文武俠御,不容獨鳴玉珥貂,情禮二三,理宜縗斬」。中書舍人江德藻、謝岐等並同師知議。文阿重議云「檢晉、宋《山陵儀》:『靈輿梓宮降殿,各侍中奏。』又《成服儀》稱:『靈輿梓宮容俠御官及香橙。』又檢《靈輿梓宮進止儀》稱:『直靈俠御吉服,在吉鹵簿中。』又云:『梓宮俠御衰服,在凶鹵簿中。』是則在殿吉凶兩俠御也。」時以二議不同,乃啟取左丞徐陵決斷。陵云:「梓宮祔山陵,靈筵祔宗廟,有
 此分判,便驗吉凶。按山陵鹵簿吉部伍中,公卿以下導引者,爰及武賁、鼓吹、執蓋、奉車,並是吉服,豈容俠御獨為縗鸑邪?斷可知矣。若言公卿胥吏並服縗苴,此與梓宮部伍有何差別?若言文物並吉,司事者凶,豈容衽絰而奉華蓋,縗衣而升玉輅邪?同博士議。」師知又議曰:「左丞引梓宮祔山陵,靈延祔宗廟,必有吉凶二部,成服不容上凶,博士猶執前斷,終是山陵之禮。若龍駕啟殯,鑾輿兼設,吉凶之儀,由來本備,準之成服,愚有未安。夫喪
 禮之制,自天子達。按王文憲《喪服明記》云:『官品第三,侍靈人二十。



 官品第四,下達士禮,侍靈之數,並有十人。皆白布褲褶,著白絹帽。內喪女侍數如外,而著齊縗。或問內外侍靈是同,何忽縗服有異?答云,若依君臣之禮,則外侍斬,內侍齊。頃世多故,禮隨事省。諸侯以下,臣吏蓋微,至於侍奉,多出義附,君臣之節不全,縗冠之費實闕,所以因其常服,止變帽而已。婦人侍者,皆是卑隸,君妾之道既純,服章所以備矣。』皇朝之典,猶自不然,以此而
 推,是知服斬。彼有侍靈,則猶俠御,既著白帽,理無彤服。且梁昭明《儀注》,今則見存,二文顯證,差為成準。且禮出人情,可得消息。凡人有喪,既陳延几,繐帷靈屏,變其常儀,蘆箔草廬,即其凶禮。堂室之內,親賓具來,齊斬麻緦,差池哭次,玄冠不弔,莫非素服。豈見門生故吏,綃縠間趨,左姬右姜,紅紫相糅?況四海遏密,率土之情是同,三軍縞素,為服之制斯一。逐使千門旦啟,非塗堊於彤闈,百僚戾止,變服粗於朱AX,而耀金在列,鳴玉節行,求之
 懷抱,固為未愜,準以禮經,彌無前事。豈可成服之儀,譬以山陵之禮?葬既始終已畢,故有吉凶之儀,所謂成服,本成喪禮,百司外內,皆變吉容,俠御獨不,何謂成服?若靈無俠御則已,有則必應縗服。」謝岐議曰:「靈延祔宗廟,梓宮祔山陵,實如左丞議。但山陵鹵簿,備有吉凶,從靈輿者儀服無變,從梓宮者皆服苴縗。爰至士禮,悉同此制,此自是山陵之儀,非關成服。今謂梓宮靈扆,共在西階,稱為成服,亦無鹵簿,直是爰自胥吏,上至王公,四海
 之內,必備縗絰,案梁昭明太子薨,略是成例,豈容凡百士庶,悉皆服重,而侍中至於武衛,最是近官,反鳴玉紆青,與平吉不異?左丞既推以山陵事,愚意或謂與成服有殊。若爾日俠御,文武不異,維侍靈之人,主書、宣傳、齊乾、應敕,悉應不改。」蔡景歷又議云:「俠御之官,本出五百,爾日備服居廬,仍於本省,引上登殿,豈應變服貂玉、若別攝餘官,以充簪珥,則爾日便有不成服者。山陵自有吉凶二議,成服凶而不吉,猶依前議,同劉舍人。」德藻又
 議云:「愚謂祖葬之辰,始終永畢,達官有追贈,須表恩榮,有吉鹵簿,恐由此義,私家放斅,因以成俗。上服本變吉為凶,理不應猶襲紈綺。劉舍人引王衛軍《喪儀》及檢梁昭明故事,此明據已審,博士、左丞乃各盡事衷,既未取證,須更詢詳,宜諮八座、詹事、太常、中丞及中庶諸通袁樞、張種、周弘正、弘讓、沈炯、孔奐。」



 時八座以下,並請:「案群議,斟酌舊儀,梁昭明太子《喪成服儀注》,明文見存,足為準的。成服日,侍官理不容猶從吉禮。其葬禮分吉,自是
 山陵之時,非關成服之日。愚謂劉舍人議,於事為允。」陵重答云:「老病屬纊,不能多說,古人爭議,多成怨府,傅玄見尤於晉代,王商取陷於漢朝,謹自三緘,敬同高命。若萬一不死,猶得展言,庶與朝賢更申揚搉。」文阿猶執所見,眾議不能決,乃具錄二議奏聞,從師知議。



 尋遷鴻臚卿,舍人如故。天嘉元年,坐事免。初,世祖敕師知撰《起居注》,自永定二年秋至天嘉元年冬,為十卷。起為中書舍人,復掌詔誥。天康元年,世祖不豫,師知與尚書僕射到
 仲舉等入侍醫藥。世祖崩,預受顧命。及高宗為尚書令,入輔,光大元年,師知與仲舉等遣舍人殷不佞矯詔令高宗還東府,事覺,於北獄賜死。



 謝岐,會稽山陰人也。父達,梁太學博士。岐少機警,好學,見稱於梁世。為尚書金部郎,山陰令。侯景亂,岐流寓東陽。景平,依于張彪。彪在吳郡及會稽,庶事一以委之。彪每征討,恆留岐監郡,知後事。彪敗,高祖引岐參預機密,以為兼尚書右丞。時軍旅屢興,糧儲多闕,岐所在幹理,
 深被知遇。永定元年,為給事黃門侍郎、中書舍人,兼右丞如故。天嘉二年卒,贈通直散騎常侍。



 岐弟嶠,篤學,為世通儒。



 史臣曰:高祖開基創業,克定禍亂,武猛固其立功,文翰亦乃展力。趙知禮、蔡景歷早識攀附,預締構之臣焉。劉師知博涉多通,而暗於機變,雖欲存乎節義,終陷極刑,斯不智矣。



\end{pinyinscope}