\article{卷二十二列傳第十六陸子隆 錢道戢 駱牙}

\begin{pinyinscope}

 陸子隆,字興世,吳郡吳人也。祖敞之,梁嘉興令。父悛,封氏令。子隆少慷慨,有志功名。起家東宮直後。侯景之亂,於鄉里聚徒。是時張彪為吳郡太守,引為將帥。彪徙鎮
 會稽,子隆隨之。及世祖討彪,彪將沈泰、吳寶真、申縉等皆降,而子隆力戰敗績,世祖義之,復使領其部曲,板為中兵參軍。歷始豊、永興二縣令。



 世祖嗣位,子隆領甲仗宿衛。尋隨侯安都拒王琳於柵口。王琳平,授左中郎將。



 天嘉元年,封益陽縣子,邑三百戶。出為高唐郡太守。二年,除明威將軍、廬陵太守。時周迪據臨川反,東昌縣人脩行師應之,率兵以攻子隆,其鋒甚盛。子隆設伏於外,仍閉門偃甲,示之以弱。及行師至,腹背擊之,行師大敗,
 因乞降,子隆許之,送于京師。



 四年,周迪引陳寶應復出臨川,子隆隨都督章昭達討迪。迪退走,因隨昭達踰東興嶺,討陳寶應。軍至建安,以子隆監郡。寶應據建安之湖際以拒官軍,子隆與昭達各據一營,昭達先與賊戰,不利,亡其鼓角,子隆聞之,率兵來救,大破賊徒,盡獲昭達所亡羽儀甲仗。晉安平,子隆功最,遷假節、都督武州諸軍事,將軍如故。



 尋改封朝陽縣伯,邑五百戶。廢帝即位,進號智武將軍,加員外散騎常侍,餘如故。



 華皎據湘
 州反,以子隆居其心腹,皎深患之,頻遣使招誘,子隆不從,皎因遣兵攻之,又不能剋。及皎敗於郢州,子隆出兵以襲其後,因與王師相會。授持節、通直散騎常侍、都督武州諸軍事,進爵為侯,增邑并前七百戶。尋遷都督荊、信、祐三州諸軍事、宣毅將軍、荊州刺史,持節、常侍如故。是時荊州新置,治于公安,城池未固,子隆修建城郭,綏集夷夏,甚得民和,當時號為稱職。三年,吏民詣都上表,請立碑頌美功績,詔許之。太建元年,進號雲麾將軍。二
 年卒,時年四十七。



 贈散騎常侍,謚曰威。子之武嗣。



 之武年十六,領其舊軍,隨吳明徹北伐有功,官至王府主簿、弘農太守,仍隸明徹。明徹於呂梁敗績,之武逃歸,為人所害,時年二十二。



 子隆弟子才,亦有幹略,從子隆征討有功,除南平太守,封始興縣子,邑三百戶。從吳明徹北伐,監安州,鎮于宿預。除中衛始興王咨議參軍,遷飆猛將軍、信州刺史。太建十三年卒,時年四十二,贈員外散騎常侍。



 錢道戢,字子韜,吳興長城人也。父景深,梁漢壽令。道戢少以孝行著聞,及長,頗有幹略,高祖微時,以從妹妻焉。從平盧子略於廣州,除濱江令。高祖輔政,遣道戢隨世祖平張彪于會稽,以功拜直閣將軍,除員外散騎常侍、假節、東徐州刺史,封永安縣侯,邑五百戶。仍領甲卒三千,隨侯安都鎮防梁山,尋領錢塘、餘杭二縣令。永定三年,隨世祖鎮于南皖口。天嘉元年,又領剡令,鎮于縣之南巖,尋為臨海太守,鎮巖如故。



 侯安都之討留異也,道
 戢帥軍出松陽以斷其後。異平,以功拜持節、通直散騎常侍、輕車將軍、都督東西二衡州諸軍事、衡州刺史,領始興內史。光大元年,增邑并前七百戶。



 高宗即位,徵歐陽紇入朝,紇疑懼,乃舉兵來攻衡州,道戢與戰,卻之。及都督章昭達率兵討紇,以道戢為步軍都督,由間道斷紇之後。紇平,除左衛將軍。



 太建二年,又隨昭達征蕭巋於江陵,道戢別督眾軍與陸子隆焚青泥舟艦,仍為昭達前軍,攻安蜀城,降之。以功加散騎常侍、仁武將軍,
 增邑并前九百戶。其年,遷仁威將軍、吳興太守。未行,改授使持節、都督郢、巴、武三州諸軍事、郢州刺史。王師北討,道戢與儀同黃法抃圍歷陽。歷陽城平,因以道戢鎮之。以功加雲麾將軍,增邑并前一千五百戶。其年十一月遘疾卒,時年六十三。贈本官,謚曰肅。



 子邈嗣。



 駱牙,字旗門,吳興臨安人也。祖秘道,梁安成王田曹參軍。父裕,鄱陽嗣王中兵參軍事。牙年十二,宗人有善相者,云:「此郎容貌非常,必將遠致。」梁太清末,世祖嘗避地
 臨安,牙母陵,睹世祖儀表,知非常人,賓待甚厚。及世祖為吳興太守,引牙為將帥,因從平杜龕、張彪等,每戰輒先鋒陷陣,勇冠眾軍,以功授真閣將軍。太平二年,以母憂去職。世祖鎮會稽,起為山陰令。永定三年,除安東府中兵參軍,出鎮冶城。尋從世祖拒王琳於南皖。世祖即位,授假節、威虜將軍、員外散騎常侍,封常安縣侯,邑五百戶。尋為臨安令,遷越州刺史,餘並如故。



 初,牙母之卒也,于時饑饉兵荒,至是始葬,詔贈牙母常安國太夫人,
 謚曰恭。



 遷牙為貞威將軍、晉陵太守。



 三年,以平周迪之功,遷冠軍將軍、臨川內史。太建三年,授安遠將軍、衡陽內史,未拜,徙為桂陽太守。八年,還朝,遷散騎常侍,入直殿省。十年,授豊州刺史,餘並如故。至德二年卒,時年五十七。贈安遠將軍、廣州刺史。子義嗣。



 史臣曰:陸子隆、錢道戢,或舉門願從,或舊齒樹勳,有統領之才,充師旅之寄。至於受任籓屏,功績並著,美矣!駱牙識真有奉,知世祖天授之德,蓋張良之亞歟?牙母智
 深先覺,符柏谷之禮,君子知鑒識弘遠,其在茲乎!



\end{pinyinscope}