\article{卷十七列傳第十一王沖 王通 弟勱 袁敬 兄子樞}

\begin{pinyinscope}

 王沖,字長深,琅邪臨沂人也。祖僧衍,齊侍中。父茂璋,梁給事黃門侍郎。



 沖母,梁武帝妹新安穆公主,卒於齊世,武帝以沖偏孤,深所鐘愛。年十八,起家梁秘書郎。尋為
 永嘉太守。入為太子舍人,以父憂去職。服闋,除太尉臨川王府外兵參軍、東宮領直。累遷太子洗馬、中舍人。出為招遠將軍、衡陽內史。遷武威將軍、安成嗣王長史、長沙內史,將軍如故。王薨於湘州,仍以沖監湘州事。入為太子庶子。遷給事黃門侍郎。大同三年,以帝甥賜爵安東亭侯,邑一百五十戶。歷明威將軍、南郡太守、太子中庶子、侍中。出監吳郡,滿歲即真。徵為通直散騎常侍,兼左民尚書。出為明威將軍、輕車當陽公府長史、江夏太
 守,行郢州事。遷平西邵陵王長史。轉驃騎廬陵王長史、南郡太守。王薨,行州府事。梁元帝鎮荊州,為鎮西長史,將軍、太守如故。沖性和順,事上謹肅,習於法令,政在平理,佐籓蒞人,鮮有失德,雖無赫赫之譽,久而見思,由是推重,累居二千石。又曉音樂,習歌舞,善與人交,貴游之中,聲名藉甚。



 侯景之亂,梁元帝於荊州承制,沖求解南郡,以讓王僧辯,并獻女妓十人,以助軍賞。元帝授持節、督衡、桂、成、合四州諸軍事、雲麾將軍、衡州刺史。元帝第
 四子元良為湘州刺史,仍以沖行州事,領長沙內史。侯景平,授翊左將軍、丹陽尹。



 武陵王舉兵至峽口,王琳偏將陸納等據湘州應之,沖為納所拘。納降,重授侍中、中權將軍,量置佐史,尹如故。江陵陷,敬帝為太宰,承制以沖為左長史。紹泰中,累遷左光祿大夫、尚書右僕射。遷左僕射、開府儀同三司,侍中、將軍如故。



 尋復領丹陽尹、南徐州大中正,給扶。



 高祖受禪,解尹,以本官領左光祿大夫。未拜,改領太子少傅。文帝嗣位,解少傅,加特進、左
 光祿大夫。尋又以本官領丹陽尹,參撰律令。廢帝即位,給親信十人。



 初,高祖以沖前代舊臣,特申長幼之敬。文帝即位,益加尊重,嘗從文帝幸司空徐度宅,宴筵之上,賜以几。其見重如此。光大元年薨,時年七十六。贈侍中、司空,謚曰元簡。



 沖有子三十人,並致通官。第十二子瑒,別有傳。



 王通,字公達,琅邪臨沂人也。祖份,梁左光祿大夫。父琳,司徒左長史。琳齊代娶梁武帝妹義興長公主,有子九
 人,並知名。



 通,梁世起家國子生,舉明經,為秘書郎、太子舍人。以帝甥封武陽亭侯。累遷王府主簿、限外記室參軍、司徒主簿、太子中庶子、驃騎廬陵王府給事中郎、中權何敬容府長史、給事黃門侍郎,坐事免。侯景之亂,奔于江陵,元帝以為散騎常侍,遷守太常卿。自侯景亂後,臺內宮室,並皆焚燼,以通兼起部尚書,歸于京師,專掌繕造。



 江陵陷,敬帝承制以通為吏部尚書。紹泰元年,加侍中,尚書如故。尋為尚書右僕射,吏部如故。高祖受禪,
 遷左僕射,侍中如故。文帝嗣位,領太子少傅。天康元年,為翊右將軍、右光祿大夫,量置佐史。廢帝即位,號安右將軍,又領南徐州大中正。太建元年,遷左光祿大夫。六年,加特進,侍中、將軍、光祿、佐史並如故。未拜卒,時年七十二。詔贈本官,謚曰成,葬日給鼓吹一部,弟質、弟固各有傳。



 勱字公濟,通之弟也。美風儀,博涉書史,恬然清簡,未嘗以利欲干懷。梁世為國子《周易》生,射策舉高第,除秘書
 郎、太子舍人、宣惠武陵王主簿、輕車河東王功曹史。王出鎮京口,勱將隨之籓,范陽張纘時典選舉,勱造纘言別,纘嘉其風采,乃曰:「王生才地,豈可遊外府乎?」奏為太子洗馬。遷中舍人,司徒左西屬。出為南徐州別駕從事史。



 大同末,梁武帝謁園陵,道出朱方,勱隨例迎候,敕勱令從輦側,所經山川,莫不顧問,勱隨事應對,咸有故實。又從登北顧樓,賦詩,辭義清典,帝甚嘉之。



 時河東王為廣州刺史,乃以勱為冠軍河東王長史、南海太守。王至
 嶺南,多所侵掠,因懼罪稱疾,委州還朝,勱行廣州府事。越中饒沃,前後守宰例多貪縱,勱獨以清白著聞。入為給事黃門侍郎。侯景之亂,西奔江陵,元帝承制以為太子中庶子,掌相府管記。出為寧遠將軍、晉陵太守。時兵饑之後,郡中凋弊,勱為政清簡,吏民便安之。徵為侍中,遷五兵尚書。



 及西魏寇江陵,元帝征湘州刺史宜豊侯蕭循入援,以勱監湘州。江陵陷,敬帝承制以為中書令。紹泰元年加侍中。高祖為司空,以勱兼司空長史。高祖
 為丞相,勱兼丞相長史,侍中、中書令並如故。時吳中遭亂,民多乏絕,乃以勱監吳興郡。



 及蕭勃平後,又以勱舊在嶺表,早有政績,乃授使持節、都督廣州等二十州諸軍事、平南將軍、平越中郎將、廣州刺史。未行,改為衡州刺史,持節、都督並如故。王琳據有上流,衡、廣攜貳,勱不得之鎮,留于大庾嶺。天嘉元年,徵為侍中、都官尚書,未拜,復為中書令。遷太子詹事,行東宮事,侍中並如故。加金紫光祿大夫,領度支尚書。廢帝即位,加散騎常侍。太
 建元年,遷尚書右僕射。時東境大水,百姓饑饉,以勱為仁武將軍、晉陵太守。在郡甚有威惠,郡人表請立碑,頌勱政績,詔許之。徵為中書監,重授尚書右僕射,領右軍將軍。四年五月卒,時年六十七。



 贈侍中、中書監,謚曰溫。



 袁敬,字子恭,陳郡陽夏人也。祖顗,宋侍中、吏部尚書、雍州刺史。父昂,梁侍中、司空,謚穆公。敬純孝有風格,幼便篤學,老而無倦。釋褐秘書郎,累遷太子舍人、洗馬、中舍人。江陵淪覆,流寓嶺表。高祖受禪,敬在廣州,依歐陽頠。



 及頠卒,其子紇據州,將有異志,敬累諫紇,為陳逆順之理,言甚切至,紇終不從。



 高宗即位,遣章昭達率眾討紇,紇將敗之時,恨不納敬言。朝廷義之,其年徵為太子中庶子、通直散騎常侍。俄轉司徒左長史。尋遷左民尚書,轉都官尚書,領豫州大中正。累遷太常卿、散騎常侍、金紫光祿大夫,加特進。至德三年卒,時年七十九,贈左光祿大夫,謚曰靖德。子元友嗣。弟泌自有傳。兄子樞。



 樞字踐言,梁吳郡太守君正之子也。美容儀,性沈靜,好
 讀書,手不釋卷。家世顯貴,貲產充積,而樞獨居處率素,傍無交往,端坐一室,非公事未嘗出遊,榮利之懷淡如也。起家梁秘書郎,歷太子舍人,輕車河東王主簿,安前邵陵王、中軍宣城王二府功曹史。侯景之亂,樞往吳郡省父,因丁父憂。時四方擾亂,人求茍免,樞居喪以至孝聞。王僧辯平侯景,鎮京城,衣冠爭往造請,樞獨杜門靜居,不求聞達。紹泰元年,徵為給事黃門侍郎。未拜,除員外散騎常侍,兼侍中。二年,兼吏部尚書。其年出為吳興
 太守。永定二年,徵為左民尚書。未至,改侍中,掌大選事。



 三年,遷都官尚書,掌選如故。



 樞博聞彊識,明悉舊章。初,高祖長女永世公主先適陳留太守錢蕆,生子岊,主及岊並卒于梁世。高祖受命,唯公主追封。至是將葬,尚書主客請詳議,欲加蕆駙馬都尉,并贈岊官。樞議曰:「昔王姬下嫁,必適諸侯,同姓為主,聞於《公羊》之說,車服不繫,顯於詩人之篇。漢氏初興,列侯尚主,自斯以後,降嬪素族。駙馬都尉置由漢武,或以假諸功臣,或以加於戚屬,
 是以魏曹植表駙馬、奉車趣為一號。《齊職儀》曰,凡尚公主必拜駙馬都尉,魏、晉以來,因為瞻準。蓋以王姬之重,庶姓之輕,若不加其等級,寧可合巹而酳,所以假駙馬之位,乃崇於皇女也。



 今公主早薨,伉儷已絕,既無禮數致疑,何須駙馬之授?案杜預尚晉宣帝第二女高陵宣公主,晉武踐祚,而主已亡,泰始中追贈公主,元凱無復駙馬之號。梁文帝女新安穆公主早薨,天監初王氏無追拜之事。遠近二例,足以據明。公主所生,既未及成人
 之禮,無勞此授,今宜追贈亭侯。」時以樞議為長。



 天嘉元年,守吏部尚書。三年,即真。尋領右軍將軍,又領丹陽尹,本官如故。



 五年,以葬父,拜表自解,詔賜絹布五十匹,錢十萬,令葬訖停宅視郡事,服闋,還復本職。其年秩滿,解尹,加散騎常侍,將軍、尚書並如故。是時,僕射到仲舉雖參掌選事,銓衡汲引,並出於樞,其所舉薦,多會上旨。謹慎周密,清白自居,文武職司,鮮有遊其門者。廢帝即位,遷尚書左僕射。光大元年卒,時年五十一。



 贈侍中、左
 光祿大夫,謚曰簡懿。有集十卷行於世。弟憲,自有傳。



 史臣曰:王沖、王通並以貴游,早升清貫,而允蹈禮節,篤誠奉上,斯為美焉。



 王勱之襟神夷淡,袁樞之端操沉冥,雖拘放為異,而勝概一揆,古所謂名士者,蓋在其人乎!






\end{pinyinscope}