\article{卷十九列傳第十三沈炯 虞荔 弟寄 馬樞}

\begin{pinyinscope}

 沈炯,字禮明,吳興武康人也。祖瑀,梁尋陽太守。父續,王府記室參軍。炯少有雋才,為當時所重。釋褐王國常侍,遷為尚書左民侍郎,出為吳令。侯景之難,吳郡太守袁
 君正入援京師,以炯監郡。京城陷,景將宋子仙據吳興,遣使召炯,委以書記之任。炯固辭以疾,子仙怒,命斬之。炯解衣將就戮,礙於路間桑樹,乃更牽往他所,或遽救之,僅而獲免。子仙愛其才,終逼之令掌書記。及子仙為王僧辯所敗,僧辯素聞其名,於軍中購得之,酬所獲者鐵錢十萬,自是羽檄軍書皆出於炯。



 及簡文遇害,四方岳牧皆上表於江陵勸進,僧辯令炯製表,其文甚工,當時莫有逮者。



 高祖南下,與僧辯會于白茅灣,登壇設盟,
 炯為其文。及侯景東奔至吳郡,獲炯妻虞氏,子行簡,並殺之,炯弟攜其母逃而獲免。侯景平,梁元帝愍其妻子嬰戮,特封原鄉縣侯,邑五百戶。僧辯為司徒,以炯為從事中郎。梁元帝徵為給事黃門侍郎,領尚書左丞。



 荊州陷,為西魏所虜,魏人甚禮之,授炯儀同三司。炯以母老在東,恒思歸國,恐魏人愛其文才而留之,恒閉門卻掃,無所交遊。時有文章,隨即棄毀,不令流布。



 嘗獨行經漢武通天臺,為表奏之,陳己思歸之意。其辭曰:「臣聞喬山
 雖掩,鼎湖之靈可祠,有魯既荒,大庭之迹無泯。伏惟陛下降德猗蘭,纂靈豊谷。漢道既登,神仙可望,射之罘於海浦,禮日觀而稱功,橫中流於汾河,指柏梁而高宴,何其樂也,豈不然歟!既而運屬上仙,道窮晏駕,甲帳珠簾,一朝零落,茂陵玉碗,宛出人間,陵雲故基,共原田而膴々,別風餘址,對陵阜而茫茫,羈旅縲臣,能不落淚!



 昔承明既厭,嚴助東歸,駟馬可乘,長卿西返,恭聞故實,竊有愚心。黍稷非馨,敢忘徼福。」奏訖,其夜炯夢見有宮禁之
 所,兵衛甚嚴,炯便以情事陳訴,聞有人言:「甚不惜放卿還,幾時可至。」少日,便與王克等並獲東歸。紹泰二年至都,除司農卿,遷御史中丞。



 高祖受禪,加通直散騎常侍,中丞如故。以母老表請歸養,詔不許。文帝嗣位,又表曰:「臣嬰生不幸,弱冠而孤,母子零丁,兄弟相長。謹身為養,仕不擇官,宦成梁朝,命存亂世,冒危履險,百死輕生,妻息誅夷,昆季冥滅,餘臣母子,得逢興運。臣母妾劉,今年八十有一,臣叔母妾丘,七十有五,臣門弟侄故自無人,
 妾丘兒孫又久亡泯,兩家侍養,餘臣一人。前帝知臣之孤煢,養臣以州里,不欲使頓居草萊,又復矜臣溫清,所以一年之內,再三休沐。臣之屢披丹款,頻冒宸鑒,非欲茍違朝廷,遠離畿輦。一者以年將六十,湯火居心,每跪讀家書,前懼後喜,溫枕扇席,無復成童。二者職居彞憲,邦之司直,若自虧身體,何問國章?前德綢繆,始許哀放,內侍近臣,多悉此旨。正以選賢與能,廣求明哲,趑趄荏苒,未始取才。而上玄降戾,奄至今日,德音在耳,墳土遽
 乾,悠悠昊天,哀此罔極。兼臣私心煎切,彌迫近時,縷縷之祈,轉忘塵觸。伏惟陛下睿哲聰明,嗣興下武,刑于四海,弘此孝治。寸管求天,仰歸帷扆,有感必應,實望聖明。特乞霈然申其私禮,則王者之德,覃及無方,矧彼翔沈,孰非涵養。」詔答曰:「省表具懷。卿譽馳咸、雒,情深宛、沛。日者理切倚閭,言歸異域,復牽時役,遂乖侍養。雖周生之思,每欲棄官,《戴禮》垂文,得遺從政,前朝光宅四海,劬勞萬機,以卿才為獨步,職居專席,方深委任,屢屈情禮。朕
 嗣奉洪基,思弘景業,顧茲寡薄,兼纏哀疚,實賴賢哲,同致雍熙,豈便釋簡南闈,解紱東路。當令馮親入舍,荀母從官,用睹朝榮,不虧家禮。尋敕所由,相迎尊累,使卿公私得所,並無廢也。」



 初,高祖嘗稱炯宜居王佐,軍國大政,多預謀謨,文帝又重其才用,欲寵貴之。



 會王琳入寇大雷,留異擁據東境,帝欲使炯因是立功,乃解中丞,加明威將軍,遣還鄉里,收合徒眾。以疾卒于吳中,時年五十九。文帝聞之,即日舉哀,並遣弔祭,贈侍中,謚曰恭子。有
 集二十卷行於世。



 虞荔,字山披,會稽餘姚人也。祖權,梁廷尉卿、永嘉太守。父檢,平北始興王諮議參軍。荔幼聰敏,有志操。年九歲,隨從伯闡候太常陸倕,倕問《五經》凡有十事,荔隨問輒應,無有遺失,倕甚異之。又嘗詣徵士何胤,時太守衡陽王亦造焉,胤言之於王,王欲見荔,荔辭曰:「未有板刺,無容拜謁。」王以荔有高尚之志,雅相欽重,還郡,即辟為主簿,荔又辭以年小不就。及長,美風儀,博覽墳籍,善屬文。
 釋褐梁西中郎行參軍,尋署法曹外兵參軍,兼丹陽詔獄正。梁武帝於城西置士林館,荔乃製碑,奏上,帝命勒之于館,仍用荔為士林學士。尋為司文郎,遷通直散騎侍郎,兼中書舍人。時左右之任,多參權軸,內外機務,互有帶掌,唯荔與顧協淡然靖退,居于西省,但以文史見知,當時號為清白。尋領大著作。



 及侯景之亂,荔率親屬入臺,除鎮西諮議參軍,舍人如故。臺城陷,逃歸鄉里。



 侯景平,元帝徵為中書侍郎,貞陽侯,授揚州別駕,並不就。



 張彪之據會稽也,荔時在焉。及文帝平彪,高祖遺荔書曰:「喪亂已來,賢哲凋散,君才用有美,聲聞許、洛,當今朝廷惟新,廣求英雋,豈可棲遲東土,獨善其身?今令兄子將接出都,想必副朝廷虛遲也。」文帝又與書曰:「君東南有美,聲譽洽聞,自應翰飛京許,共康時弊,而削迹丘園,保茲獨善,豈使稱空谷之望邪?



 必願便爾俶裝,且為出都之計。唯遲披覯,在於茲日。」迫切之不得已,乃應命至都。高祖崩,文帝嗣位,除太子中庶子,仍侍太子讀書。尋
 領大著作、東揚揚州二州大中正,庶子如故。



 初,荔母隨荔入臺,卒於臺內,尋而城陷,情禮不申,由是終身蔬食布衣,不聽音樂,雖任遇隆重,而居止儉素,淡然無營。文帝深器之,常引在左右,朝夕顧訪。荔性沉密,少言論,凡所獻替,莫有見其際者,故不列于後焉。



 時荔第二弟寄寓於閩中,依陳寶應,荔每言之輒流涕。文帝哀而謂曰:「我亦有弟在遠,此情甚切,他人豈知。」乃敕寶應求寄,寶應終不遣。荔因以感疾,帝數往臨視。令荔將家口人省,荔
 以禁中非私居之所,乞停城外,文帝不許,乃令住於蘭臺,乘輿再三臨問,手敕中使,相望於道。又以荔蔬食積久,非羸疾所堪,乃敕曰:「能敦布素,乃當為高,卿年事已多,氣力稍減,方欲仗委,良須克壯,今給卿魚肉,不得固從所執也。」荔終不從。天嘉二年卒,時年五十九。文帝甚傷惜之,贈侍中,謚曰德子。及喪柩還鄉里,上親出臨送,當時榮之。子世基、世南,並少知名。



 寄字次安,少聰敏。年數歲,客有造其父者,遇寄於門,因
 嘲之曰:「郎君姓虞,必當無智。」寄應聲答曰:「文字不辨,豈得非愚?」客大慚。入謂其父曰:「此子非常人,文舉之對不是過也。」及長,好學,善屬文。性沖靜,有棲遁之志。



 弱冠舉秀才,對策高第。起家梁宣城王國左常侍。大同中,嘗驟雨,殿前往往有雜色寶珠,梁武觀之甚有喜色,寄因上《瑞雨頌》。帝謂寄兄荔曰:「此頌典裁清拔,卿家之士龍也。將如何擢用?」寄聞之,歎曰:「美盛德之形容,以申擊壤之情耳。



 吾豈買名求仕者乎?」乃閉門稱疾,唯以書籍自娛。
 岳陽王為會稽太守,引寄為行參軍,遷記室參軍,領郡五官掾。又轉中記室,掾如故。在職簡略煩苛,務存大體,曹局之內,終日寂然。



 侯景之亂,寄隨兄荔入臺,除鎮南湘東王諮議參軍,加貞威將軍。京城陷,遁還鄉里。及張彪往臨川,彊寄俱行,寄與彪將鄭瑋同舟而載,瑋嘗忤彪意,乃劫寄奔于晉安。時陳寶應據有閩中,得寄甚喜。高祖平侯景,寄勸令自結,寶應從之,乃遣使歸誠。承聖元年,除和戎將軍、中書侍郎,寶應愛其才,託以道阻不
 遣。每欲引寄為僚屬,委以文翰,寄固辭,獲免。



 及寶應結婚留異,潛有逆謀,寄微知其意,言說之際,每陳逆順之理,微以諷諫,寶應輒引說他事以拒之。又嘗令左右誦《漢書》,臥而聽之,至蒯通說韓信曰「相君之背,貴不可言」,寶應蹶然起曰「可謂智士」。寄正色曰:「覆酈驕韓,未足稱智;豈若班彪《王命》,識所歸乎?」寄知寶應不可諫,慮禍及己,乃為居士服以拒絕之。常居東山寺,偽稱腳疾,不復起,寶應以為假託,使燒寄所臥屋,寄安臥不動。親近將
 扶寄出,寄曰:「吾命有所懸,避欲安往?」所縱火者,旋自救之。寶應自此方信。



 及留異稱兵,寶應資其部曲,寄乃因書極諫曰:東山虞寄致書於明將軍使君節下:寄流離世故,飄寓貴鄉,將軍待以上賓之禮,申以國士之眷,意氣所感,何日忘之。而寄沈痼彌留,心妻陰將盡,常恐卒填溝壑,涓塵莫報,是以敢布腹心,冒陳丹款,願將軍留須臾之慮,少思察之,則瞑目之日,所懷畢矣。



 夫安危之兆,禍福之機,匪獨天時,亦由人事。失之毫釐,差以千里。是
 以明智之士,據重位而不傾,執大節而不失,豈惑於浮辭哉?將軍文武兼資,英威不世,往因多難,仗劍興師,援旗誓眾,抗威千里,豈不以四郊多壘,共謀王室,匡時報主,寧國庇民乎?此所以五尺童子,皆願荷戟而隨將軍者也。及高祖武皇肇基草昧,初濟艱難。于時天下沸騰,民無定主,豺狼當道,鯨鯢橫擊,海內業業,未知所從。



 將軍運動微之鑒,折從衡之辯,策名委質,自託宗盟,此將軍妙算遠圖,發於衷誠者也。及主上繼業,欽明睿聖,選
 賢與能,群臣輯睦,結將軍以維城之重,崇將軍以裂土之封。豈非宏謨廟略,推赤心於物也?屢申明詔,款篤殷勤,君臣之分定矣,骨肉之恩深矣。不意將軍惑於邪說,遽生異計,寄所以疾首痛心,泣盡繼之以血。



 萬全之策,竊為將軍惜之。寄雖疾侵耄及,言無足採,千慮一得,請陳愚算。願將軍少戢雷霆,賒其晷刻,使得盡狂瞽之說,披肝膽之誠,則雖死之日,由生之年也。



 自天厭梁德,多難薦臻,寰宇分崩,英雄互起,不可勝紀,人人自以為
 得之。



 然夷凶翦亂,拯溺扶危,四海樂推,三靈眷命,揖讓而居南面者,陳氏也。豈非歷數有在,惟天所授,當璧應運?其事甚明一也。主上承基,明德遠被,天綱再張,地維重紐。夫以王琳之彊,侯瑱之力,進足以搖蕩中原,爭衡天下,退足以屈強江外,雄長偏隅。然或命一旅之師,或資一士之說,琳則瓦解冰泮,投身異域,瑱則厥角稽顙,委命闕廷。斯又天假之威,而除其患。其事甚明二也。今將軍以籓戚之重,東南之眾,盡忠奉上,戮力勤王,豈
 不勳高竇融,寵過吳芮,析珪判野,南面稱孤?其事甚明三也。且聖朝棄瑕忘過,寬厚得人,改過自新,咸加敘擢。至於餘孝頃、潘純陀、李孝欽、歐陽頠等,悉委以心腹,任以爪牙,胸中豁然,曾無纖芥。



 況將軍釁非張繡,罪異畢諶,當何慮於危亡,何失於富貴?此又其事甚明四也。方今周、齊鄰睦,境外無虞,并兵一向,匪朝伊夕,非劉、項競逐之機,楚、趙連從之勢,何得雍容高拱,坐論西伯?其事甚明五也。且留將軍狼顧一隅,亟經摧衄,聲實虧喪,膽
 氣衰沮。高瓖、向文政、留瑜、黃子玉,此數人者,將軍所知,首鼠兩端,唯利是視;其餘將帥,亦可見矣。孰能被堅執銳,長驅深入,繫馬埋輪,奮不顧命,以先士卒者乎?此又其事甚明六也。且將軍之彊,孰如侯景?將軍之眾,孰如王琳?武皇滅侯景於前,今上摧王琳於後,此乃天時,非復人力。且兵革已後,民皆厭亂,其孰能棄墳墓,捐妻子,出萬死不顧之計,從將軍於白刃之間乎?此又其事甚明七也。歷觀前古,鑒之往事,子陽、季孟,傾覆相尋,餘善、
 右渠,危亡繼及,天命可畏,山川難恃。況將軍欲以數郡之地,當天下之兵,以諸侯之資,拒天子之命,彊弱逆順,可得侔乎?此又其事甚明八也。且非我族類,其心必異。不愛其親,豈能及物?留將軍身縻國爵,子尚王姬,猶且棄天屬而弗顧,背明君而孤立,危急之日,豈能同憂共患,不背將軍者乎?至於師老力屈,懼誅利賞,必有韓、智晉陽之謀,張、陳井陘之勢。此又其事甚明九也。且北軍萬里遠鬥,鋒不可當,將軍自戰其地,人多顧後。梁安背
 向為心,修旿匹夫之力,眾寡不敵,將帥不侔,師以無名而出,事以無機而動,以此稱兵,示知其利。夫以漢朝吳、楚,晉室穎、顒,連城數十,長戟百萬,拔本塞源,自圖家國,其有成功者乎?此又其事甚明十也。



 為將軍計者,莫若不遠而復,絕親留氏,秦郎、快郎,隨遣入質,釋甲偃兵,一遵詔旨。且朝廷許以鐵券之要,申以白馬之盟,朕弗食言,誓之宗社。寄聞明者鑒未形,智者不再計,此成敗之效,將軍勿疑。吉凶之幾,間不容髮。方今籓維尚少,皇子
 幼沖,凡預宗枝,皆蒙寵樹。況以將軍之地,將軍之才,將軍之名,將軍之勢,而能克修籓服,北面稱臣,寧與劉澤同年而語其功業哉?豈不身與山河等安,名與金石相敝?願加三思,慮之無忽。



 寄氣力綿微,餘陰無幾,感恩懷德,不覺狂言,鈇鉞之誅,甘之如薺。



 寶應覽書大怒。或謂寶應曰:「虞公病勢漸篤,言多錯謬。」寶應意乃小釋。



 亦為寄有民望,且優容之。及寶應敗走,夜至蒲田,顧謂其子扞秦曰:「早從虞公計,不至今日。」扞秦但泣而已。寶應既
 擒,凡諸賓客微有交涉者,皆伏誅,唯寄以先識免禍。



 初,沙門慧摽涉獵有才思,及寶應起兵,作五言詩以送之,曰:「送馬猶臨水,離旗稍引風。好看今夜月,當入紫微宮。」寶應得之甚悅。慧摽齎以示寄,寄一覽便止,正色無言。摽退,寄謂所親曰:「摽既以此始,必以此終。」後竟坐是誅。



 文帝尋敕都督章昭達以理發遣,令寄還朝。及至,即日引見,謂寄曰:「管寧無恙?」其慰勞之懷若此。頃之,文帝謂到仲舉曰:「衡陽王既出閣,雖未置府僚,然須得一
 人旦夕遊處,兼掌書記,宜求宿士有行業者。」仲舉未知所對,文帝曰:「吾自得之。」乃手敕用寄,寄入謝,文帝曰:「所以暫屈卿遊籓者,非止以文翰相煩,乃令以師表相事也。」尋兼散騎常侍,聘齊,寄辭老疾,不行,除國子博士。



 頃之,又表求解職歸鄉里,文帝優旨報答,許其東還。仍除東揚州別駕,寄又以疾辭。高宗即位,徵授揚州治中及尚書左丞,並不就。乃除東中郎建安王諮議,加戎昭將軍,又辭以疾,不任旦夕陪列。王於是特令停王府公事,其
 有疑議,就以決之,但朔望箋修而已。太建八年,加太中大夫,將軍如故。十一年卒,時年七十。



 寄少篤行,造次必於仁厚,雖僮豎未嘗加以聲色,至於臨危執節,則辭氣凜然,白刃不憚也。自流寓南土,與兄荔隔絕,因感氣病,每得荔書,氣輒奔劇,危殆者數矣。前後所居官,未嘗至秩滿,纔期年數月,便自求解退。常曰:「知足不辱,吾知足矣。」及謝病私庭,每諸王為州將,下車必造門致禮,命釋鞭板,以几杖侍坐。常出游近寺,閭里傳相告語,老幼羅
 列,望拜道左。或言誓為約者,但指寄便不欺,其至行所感如此。所製文筆,遭亂多不存。



 馬樞,字要理,扶風郿人也。祖靈慶,齊竟陵王錄事參軍。樞數歲而父母俱喪,為其姑所養。六歲,能誦《孝經》、《論語》、《老子》。及長,博極經史,尤善佛經及《周易》、《老子》義。



 梁邵陵王綸為南徐州刺史,素聞其名,引為學士。綸時自講《大品經》,令樞講《維摩》、《老子》、《周易》,同日發題,道俗聽者二千人。王欲極觀優劣,乃謂眾曰:「與馬學士論義,必使屈伏,
 不得空立主客。」於是數家學者各起問端,樞乃依次剖判,開其宗旨,然後枝分流別,轉變無窮,論者拱默聽受而已。綸甚嘉之,將引薦於朝廷。尋遇侯景之亂,綸舉兵援臺,乃留書二萬卷以付樞。樞肆志尋覽,殆將周遍,乃喟然嘆曰:「吾聞貴爵位者以巢、由為桎梏,愛山林者以伊、呂為管庫,束名實則芻芥柱下之言,玩清虛則糠秕席上之說,稽之篤論,亦各從其好也。然支父有讓王之介,嚴子有傲帝之規,千載美談,所不廢也。比求志之士,
 望途而息。豈天之不惠高尚,何山林之無聞甚乎?」乃隱于茅山,有終焉之志。



 天嘉元年,文帝徵為度支尚書,辭不應命。時樞親故並居京口,每秋冬之際,時往遊焉。及鄱陽王為南徐州刺史,欽其高尚,鄙不能致,乃卑辭厚意,令使者邀之,前後數反,樞固辭以疾。門人或進曰:「鄱陽王待以師友,非關爵位,市朝之間,何妨靜默。」樞不得已,乃行。王別築室以處之,樞惡其崇麗,乃於竹林間自營茅茨而居焉。每王公饋餉,辭不獲已者,率十分受一。



 樞少屬亂離,每所居之處,盜賊不入,依託者常數百家。目精洞黃,能視闇中物。常有白燕一雙,巢其庭樹,馴狎紘廡,時集几案,春來秋去,幾三十年。太建十三年卒,時年六十。撰《道覺論》二十卷行於世。



 史臣曰:沈炯仕於梁室,年在知命,冀郎署之薄官,止邑宰之卑職,及下筆盟壇,屬辭勸表,激揚旨趣,信文人之偉者歟!虞荔之獻籌沈密,盡其誠款,可謂有益明時矣。



\end{pinyinscope}