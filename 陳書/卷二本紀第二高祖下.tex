\article{卷二本紀第二高祖下}

\begin{pinyinscope}

 永定元年冬十月乙亥,高祖即皇帝位于南郊,柴燎告天曰:「皇帝臣霸先,敢用玄牡昭告于皇皇后帝:梁氏以圮剝薦臻,歷運有極,欽若天應,以命于霸先。夫肇有烝
 民,乃樹司牧,選賢與能,未常厥姓。放勛、重華之世,咸無意於受終,當塗、典午之君,雖有心於揖讓,皆以英才處萬乘,高勳御四海,故能大庇黔首,光宅區縣。有梁末運,仍葉遘屯,獯醜憑陵,久移神器,承聖在外,非能祀夏,天未悔禍,復罹寇逆,嫡嗣廢黜,宗枝僭詐,天地蕩覆,紀綱泯絕。霸先爰初投袂,大拯橫流,重舉義兵,實戡多難,廢王立帝,實有厥功,安國定社,用盡其力。是謂小康,方期大道。既而煙云表色,日月呈瑞,緯聚東井,龍見譙邦,除
 舊布新,即彰玄象,遷虞事夏,且協謳歌,九域八荒,同布衷款,百神群祀,皆有誠願。梁帝高謝萬邦,授以大寶,霸先自惟菲薄,讓德不嗣,至于再三,辭弗獲許。僉以百姓須主,萬機難曠,皇靈眷命,非可謙拒。畏天之威,用膺嘉祚,永言夙志,能無慚德。敬簡元辰,升壇受禪,告類上帝,用答民心,永保于我有陳。惟明靈是饗!」



 先是氛霧,晝夜晦冥,至於是日,景氣清晏,識者知有天道焉。禮畢,輿駕還宮,臨太極前殿。詔曰:「五德更運,帝王所以御天,三正
 相因,夏、殷所以宰世,雖色分辭翰,時異文質,揖讓征伐,迄用參差,而育德振民,義歸一揆。朕以寡昧,時屬艱危,國步屢屯,天維三絕,肆勤先后,拯厥橫流,藉將帥之功,兼猛士之力,一匡天下,再造黔黎。梁氏以天祿永終,歷數攸在,遵與能之典,集大命于朕躬。



 顧惟菲德,辭不獲亮,式從天眷,俯協民心,受終文祖,升禋上帝,繼迹百王,君臨萬宇,若涉川水,罔知攸濟。寶業初建,皇祚惟新,思俾惠澤,覃被億兆。可大赦天下,改梁太平二年為永定
 元年。賜民爵二級,文武二等。鰥寡孤獨不能自存者人穀五斛。逋租宿債,皆勿復收。其有犯鄉里清議贓汙淫盜者,皆洗除先注,與之更始。長徒敕繫,特皆原之。亡官失爵,禁錮奪勞,一依舊典。」又詔曰:「《禮》陳杞、宋,《詩》詠二客,弗臣之重,歷代斯敦。梁氏欽若人祇,憲章在昔,濟河沈璧,高謝萬邦,茅賦所加,宜遵舊典。其以江陰郡奉梁主為江陰王,行梁正朔,車旗服色,一依前準,宮館資待,務盡優隆。」又詔梁皇太后為江陰國太妃,皇后為江陰國
 妃。又詔百司依位攝職。丙子,輿駕幸鐘山祠帝廟。戊寅,輿駕幸華林園,親覽詞訟,臨赦囚徒。己卯,分遣大使宣勞四方,下璽書敕州郡曰:「夫四王革代,商、周所以應天,五勝相推,軒、羲所以當運。梁德不造,喪亂積年,東夏崩騰,西都蕩覆。蕭勃干紀,非唯趙倫,侯景滔天,踰於劉載。貞陽反篡,賊約連兵,江左累屬於鮮卑,金陵久非於梁國。自有氤氳混沌之世,龍圖鳳紀之前,東漢興平之,西朝永嘉之亂,天下分崩,未有若於梁朝者也。朕以
 虛薄,屬當興運,自昔登庸,首清諸越,徐門浪泊,靡不征行,浮海乘山,所在戡定。冒朔風塵,騁馳師旅,六延梁祀,十翦彊寇,豈曰人謀,皆由天啟。梁氏以天祿斯改,期運永終,欽若唐、虞,推其鼎玉,朕東西退讓,拜手陳辭,避舜子于箕山之陽,求支伯於滄洲之野,而公卿敦逼,率土翹惶,天命難稽,遂享嘉祚。今月乙亥,升禮太壇,言念遷桐,但有慚德。自梁氏將末,頻月亢陽,火運斯終,秋霖奄降。翌日成禮,圓丘宿設,埃雲晚霽,星象夜張。朝景重輪,
 泫三危之膏露,晨光合璧,帶五色之卿雲。顧惟寡薄,彌慚休祉,昧旦丕顯,方思至治。卿等擁旄方岳,相任股肱,剖符名守,方寄恤隱。王歷惟新,念有欣慶,想深求民瘼,務在廉平,愛惠以撫孤貧,威刑以御彊猾。若有萑蒲之盜,或犯戎商,山谷之酋,擅彊幽險,皆從肆赦,咸使知聞。如或迷途,俾在無貸。今遣使人具宣往旨,念思善政,副此虛懷。」庚辰,詔出佛牙於杜姥宅,集四部設無遮大會,高祖親出闕前禮拜。初,齊故僧統法獻于烏纏國得之,
 常在定林上寺,梁天監末,為攝山慶雲寺沙門慧興保藏,慧興將終,以屬弟慧志,承聖末,慧志密送于高祖,至是乃出。辛巳,追尊皇考曰景皇帝,廟號太祖;皇妣董太夫人曰安皇后。追謚前夫人錢氏號為昭皇后,世子克為孝懷太子。立夫人章氏為皇后。癸未,尊景帝陵曰瑞陵,昭皇后陵曰嘉陵,依梁初園陵故事。立刪定郎,治定律令。戊子,遷景皇帝神主祔于太廟。辛卯,以中權將軍、開府儀同三司、丹陽尹王沖為左光祿大夫。癸巳,追贈
 皇兄梁故散騎常侍、平北將軍、兗州刺史長城縣公道譚驃騎大將軍、太尉,封始興郡王;弟梁故侍中、驃騎將軍、南徐州刺史武康縣侯休先車騎大將軍、司徒,封南康郡王。是月,西討都督周文育、侯安都於郢州敗績,囚於王琳。十一月丙申,詔曰:「東都齊國,義乃親賢,西漢城陽,事兼功烈。散騎常侍、使持節、都督會稽等十郡諸軍事、宣毅將軍、會稽太守長城縣侯蒨,學尚清優,神宇凝正,文參禮樂,武定妖氛,心力謀猷,為家治國,擁旄作守,
 期月有成,辟彼關河,功踰蕭、寇,萑蒲之盜,自反耕農,篁竹之豪,用稟聲朔。朕以虛寡,屬當興運,提彼三尺,賓于四門,王業艱難,賴乎此子,宜隆上爵,稱是元功。可封臨川郡王,邑二千戶。兄子梁中書侍郎頊襲封始興王,弟子梁中書侍郎曇朗襲封南康王,禮秩一同正王。」己亥,甘露降于鐘山松林,彌滿巖谷。庚子,開善寺沙門採之以獻,敕頒賜群臣。丙辰,以鎮西將軍、南豫州刺史徐度為鎮右將軍、領軍將軍。庚申,京師大火。十二月庚辰,皇
 后謁太廟。



 二年春正月乙未,詔曰:「夫設官分職,因事重輕,羽儀車馬,隨時隆替,晉之五校,鳴笳啟途,漢之九卿,傳呼並迾,虞官夏禮,豈曰同科,殷朴周文,固無恒格。朕膺茲寶歷,代是天工,留念官方,庶允時衷。梁天監中,左右驍騎領朱衣直閣,並給儀從,北徐州刺史昌義之初,首為此職。亂離歲久,朝典不存,後生年少,希聞舊則。今去左右驍騎,宜通文武,文官則用腹心,武官則用功臣,所給儀從,
 同太子二衛率。此外眾官,尚書詳為條制。」車騎將軍、開府儀同三司侯瑱進位司空,中權將軍、開府儀同三司、新除左光祿大夫王沖為太子少傅。左衛將軍徐世譜為護軍將軍,南兗州刺史吳明徹進號安南將軍,衡州刺史歐陽頠進號鎮南將軍。



 辛丑,輿駕親祠南郊。詔曰:「朕受命君臨,初移星琯,孟陬嘉月,備禮泰壇,景候昭華,人祗允慶,思令億兆,咸與惟新。且往代祅氛,于今猶梗,軍機未息,征賦咸繁,事不獲已,久知下弊,言念黔黎,無
 忘寢食。夫罪無輕重,已發覺未發覺,在今昧爽以前,皆赦除之。西寇自王琳以下,並許返迷,一無所問。近所募義軍,本擬西寇,並宜解遣,留家附業。晚訂軍資未送者並停,元年軍糧逋餘者原其半。



 州郡縣軍戍並不得輒遣使民間,務存優養。若有侵擾,嚴為法制。」乙巳,輿駕親祠北郊。甲辰,振遠將軍、梁州刺史張立表稱去乙亥歲八月,丹徒、蘭陵二縣界遺山側,一旦因濤水涌生,沙漲,周旋千餘頃,並膏腴,堪墾植。戊午,輿駕親祠明堂。二月
 壬申,南豫州刺史沈泰奔于齊。辛卯,詔車騎將軍、司空侯瑱總督水步眾軍以遏齊寇。三月甲午,詔曰:「罰不及嗣,自古通典,罪疑惟輕,布在方策。沈泰反覆無行,遐邇所知。昔有微功,仍荷朝寄,剖符名郡,推轂累籓,漢口班師,還居方岳,良田有逾於四百,食客不止於三千,富貴顯榮,政當如此。鬼害其盈,天奪之魄,無故猖狂,自投獯醜。雖復知人則哲,惟帝其難,光武有蔽於龐萌,魏武不知於于禁,但令朝廷,無我負人。其部曲妻兒,各令復業,
 所在及軍人若有恐脅侵掠者,皆以劫論。若有男女口為人所藏,並許詣臺申訴。若樂隨臨川王及節將立效者,悉皆聽許。」乙卯,高祖幸後堂聽訟,還於橋上觀山水,賦詩示群臣。是月,王琳立梁永嘉王蕭莊于郢州。夏四月甲子,輿駕親祠太廟。乙丑,江絮王薨,詔遣太宰弔祭,司空監護喪事,凶禮所須,隨由備辦。以梁武林侯蕭諮息季卿嗣為江陰王。丙寅,輿駕幸石頭,餞司空侯瑱。戊辰,重雲殿東鴟尾有紫煙屬天。五月乙未,京師地震。癸
 丑,齊廣陵南城主張顯和、長史張僧那各率其所部入附。辛酉,輿駕幸大莊巖寺捨身。壬戌,群臣表請還宮。六月己巳,詔司空侯瑱、領軍將軍徐度率舟師為前軍,以討王琳。秋七月戊戌,輿駕幸石頭,親送瑱等。己亥,江州刺史周迪擒王琳將李孝欽、樊猛、餘孝頃于工塘。甲辰,遣吏部尚書謝哲諭王琳。甲寅,嘉禾一穗六岐生五城。初,侯景之平也,火焚太極殿,承聖中議欲營之,獨闕一柱,至是有樟木大十八圍,長四丈五尺,流泊陶家後渚,
 監軍鄒子度以聞。詔中書令沈眾兼起部尚書,少府卿蔡儔兼將作大匠,起太極殿。八月丙寅,以廣梁郡為陳留郡。辛未,詔臨川王蒨西討,以舟師五萬發自京師,輿駕幸冶城寺親送焉。前開府儀同三司、南豫州刺史周文育,前鎮北將軍、南徐州刺史、新除開府儀同三司侯安都等於王琳所逃歸,自劾廷尉,即日引見,並宥之。戊寅,詔復文育等本官。



 壬午,追封皇子立為豫章王,謚曰獻;權為長沙王,謚曰思;長女為永世公主,謚曰懿。謝哲
 反命,王琳請還鎮湘川,詔追眾軍緩其伐。癸未,西討眾軍至自大雷。



 丁亥,以信威將軍、江州刺史周迪為開府儀同三司,進號平南將軍。改南徐州所領南蘭陵郡復為東海郡。冬十月庚午,遣鎮南將軍、開府儀同三司周文育都督眾軍出豫章,討餘孝勱。乙亥,輿駕幸莊嚴寺,發《金光明經》題。丁酉,以仁威將軍、高州刺史黃法抃為開府儀同三司,進號鎮南將軍。甲寅,太極殿成,匠各給復。十二月庚申,侍中、安東將軍臨川王蒨率百僚朝前
 殿,拜上牛酒。甲子,輿駕幸大莊嚴寺,設無珝大會,捨乘輿法物。群臣備法駕奉迎,即日輿駕還宮。丙寅,高祖於太極殿東堂宴群臣,設金石之樂,以路寢告成也。壬申,割吳郡鹽官、海鹽、前京三縣置海寧郡,屬揚州。以安成所部廣興六洞置安樂郡。丙戌,以寧遠將軍、北江州刺史熊曇朗為開府儀同三司,進號平西將軍。丁亥,詔曰:「梁時舊仕,亂離播越,始還朝廷,多未銓序。又起兵已來,軍勳甚眾。選曹即條文武簿及節將應九流者,量其所
 擬。」於是隨材擢用者五十餘人。



 三年春正月己丑,青龍見于東方。丁酉,以鎮南將軍、廣州刺史歐陽頠即本號開府儀同三司。是夜大雪,及旦,太極殿前有龍跡見。甲午,廣州刺史歐陽頠表稱白龍見於州江南岸,長數十丈,大可八九圍,歷州城西道入天井崗。仙人見於羅浮山寺小石樓,長三丈所,通身潔白,衣服楚麗。辛丑,詔曰:「南康、始興王諸妹,已有封爵,依禮止是籓主。此二王者,有殊恒情,宜隆禮數。諸主儀秩
 及尚主,可並同皇女。」戊申,詔臨川王蒨省揚、徐二州辭訟。二月辛酉,以平西將軍、桂州刺史淳于量為開府儀同三司,進號鎮西大將軍。壬午,司空侯瑱督眾軍自江入合州,焚齊舟艦。三月丙申,侯瑱至自合肥,眾軍獻捷。夏閏四月庚寅,詔曰:「開廩賑絕,育民之大惠,巡方恤患,前王之令典。朕當斯季俗,膺此樂推,君德未孚,民瘼猶甚,重茲多壘,彌疚納隍。良由四聰弗達,千里勿應。博施之仁,何其或爽?



 殘弊之軌,致此未康。吳州、縉州,去歲蝗
 旱,郢田雖疏,鄭渠終涸,室靡盈積之望,家有填壑之嗟。百姓不足,兆民何賴?近已遣中書舍人江德藻銜命東陽,與令長二千石問民疾苦,仍以入臺倉見米分恤。雖德非既飽,庶微慰阻飢。」甲午,詔依前代置西省學士,兼以伎術者預焉。丁酉,遣鎮北將軍徐度率眾城南皖口。是時久不雨,丙午,輿駕幸鐘山祠蔣帝廟,是日降雨,迄于月晦。五月丙辰朔,日有食之,有司奏:舊儀,御前殿,服朱紗袍、通天冠。詔曰:「此乃前代承用,意有未同。合朔仰
 助太陽,宜備袞冕之服。自今已去,永可為准。」丙寅,扶南國遣使獻方物。乙酉,北江州刺史熊曇朗殺都督周文育于軍,舉兵反。王琳遣其將常眾愛、曹慶率兵援餘孝勱。六月戊子,儀同侯安都敗眾愛等於左里,獲琳從弟襲、主帥羊暕等三十餘人,眾愛遁走,庚寅,廬山民斬之,傳首京師。甲午,眾師凱歸。詔曰:「曇朗噬逆,罪不容誅,分命眾軍,仍事掩討,方加梟磔,以明刑憲。」徵臨川王裝往皖口置城柵,以錢道戢守焉。丁酉,高祖不豫,遣兼太宰、
 尚書左僕射王通以疾告太廟,兼太宰、中書令謝哲告大社、南北郊。辛丑,高祖疾小瘳。故司空周文育之柩至自建昌。壬寅,高祖素服哭于東堂,哀甚。癸卯,高祖臨訊獄訟。是夜,熒惑在天尊。高祖疾甚。丙午,崩于璿璣殿,時年五十七。遺詔追臨川王蒨入纂。



 甲寅,大行皇帝遷殯于太極殿西階。秋八月甲午,群臣上謚曰武皇帝,廟號高祖。



 丙申,葬萬安陵。



 高祖智以綏物,武以寧亂,英謀獨運,人皆莫及,故能征伐四克,靜難夷凶。



 至升大麓之日,
 居阿衡之任,恒崇寬政,愛育為本。有須發調軍儲,皆出於事不可息。加以儉素自率,常膳不過數品,私饗曲宴,皆瓦器蚌盤,肴核庶羞,裁令充足而已,不為虛費。初平侯景,及立紹泰,子女玉帛,皆班將士。其充闈房者,衣不重綵,飾無金翠,哥鐘女樂,不列於前。及乎踐祚,彌厲恭儉。故隆功茂德,光有天下焉。



 陳吏部尚書姚察曰:高祖英略大度,應變無方,蓋漢高、魏武之亞矣。及西都盪覆,誠貫天人。王僧辯闕伊尹之
 才,空結桐宮之憤,貞陽假秦兵之送,不思穆嬴之泣。高祖乃蹈玄機而撫末運,乘勢隙而拯橫流,王迹所基,始自於此,何至戡黎升陑之捷而已焉。故於慎徽時序之世,變聲改物之辰,兆庶歸以謳歌,炎靈去如釋負,方之前代,何其美乎!



\end{pinyinscope}