\article{卷二十四列傳第十八周弘正 弟弘直 弘直子確 袁憲}

\begin{pinyinscope}

 周弘正,字思行,汝南安城人,晉光祿大夫顗之九世孫也。祖顒,齊中書侍郎,領著作。父寶始,梁司徒祭酒。弘正幼孤,及弟弘讓、弘直,俱為伯父侍中護軍捨所養。年十
 歲,通《老子》、《周易》,捨每與談論,輒異之,曰:「觀汝神情穎晤,清理警發,後世知名,當出吾右。」河東裴子野深相賞納,請以女妻之。十五,召補國子生,仍於國學講《周易》,諸生傳習其義。以季春入學,孟冬應舉,學司以其日淺,弗之許焉。博士到洽議曰:「周郎年未弱冠,便自講一經,雖曰諸生,實堪師表,無俟策試。」起家梁太學博士。晉安王為丹陽尹,引為主簿。出為鄴令,丁母憂去職。服闋,歷曲阿、安吉令。普通中,初置司文義郎,直壽光省,以弘正為司
 義侍郎。



 中大通三年,梁昭明太子薨,其嗣華容公不得立,乃以晉安王為皇太子,弘正乃奏記曰:竊聞捴謙之象,起於羲、軒爻畫,揖讓之源,生於堯、舜禪受,其來尚矣,可得而詳焉。夫以廟堂、汾水,殊途而同歸,稷、契、巢、許,異名而一貫,出者稱為元首,處者謂之外臣,莫不內外相資,表裏成治,斯蓋萬代同規,百王不易者也。



 暨於三王之世,浸以陵夷,各親其親,各子其子。乃至七國爭雄,劉項競逐,皇漢扇其俗,有晉揚其波,謙讓之道廢,多歷年所
 矣。夫文質遞變,澆淳相革,還樸反古,今也其時。



 伏惟明大王殿下,天挺將聖,聰明神武,百辟冠冕,四海歸仁。是以皇上發德音,下明詔,以大王為國之儲副,乃天下之本焉。雖復夏啟、周誦,漢儲、魏兩,此數君者,安足為大王道哉。意者願聞殿下抗目夷上仁之義,執子臧大賢之節,逃玉輿而弗乘,棄萬乘如脫屣,庶改澆競之俗,以大吳國之風。古有其人,今聞其語,能行之者,非殿下而誰?能使無為之化,復興於邃古,讓王之道,不墜於來葉,豈
 不盛歟!豈不盛歟!



 弘正陋學書生,義慚稽古,家自汝、潁,世傳忠烈,先人決曹掾燕抗辭九諫,高節萬乘,正色三府,雖盛德之業將絕,而狂直之風未墜。是以敢布腹心,肆其愚瞽。如使芻言野說,少陳於聽覽,縱復委身烹鼎之下,絕命肺石之上,雖死之日,猶生之年。



 其抗直守正,皆此類也。



 累遷國子博士。時於城西立士林館,弘正居以講授,聽者傾朝野焉。弘正啟梁武帝《周易》疑義五十條,又請釋《乾》、《坤》、《二繫》曰:「臣聞《易》稱立以盡意,繫辭以
 盡言,然後知聖人之情,幾可見矣。自非含微體極,盡化窮神,豈能通志成務,探賾致遠。而宣尼比之桎梏,絕韋編於漆字,軒轅之所聽瑩,遺玄珠於赤水。伏惟陛下一日萬機,匪勞神於瞬息,凝心妙本,常自得於天真,聖智無以隱其幾深,明神無以淪其不測。至若爻畫之苞於《六經》,文辭之窮於《兩系》,名儒劇談以歷載,鴻生抵掌以終年,莫有試游其籓,未嘗一見其涘。自制旨降談,裁成《易》道,析至微於秋毫,渙曾冰於幽谷。臣親承音旨,職司
 宣授,後進詵詵,不無傳業。但《乾》、《坤》之蘊未剖,《繫》表之妙莫詮,使一經深致,尚多所惑。臣不涯庸淺,輕率短陋,謹與受業諸生清河張譏等三百一十二人,於《乾》、《坤》、《二繫》、《象》、《爻》未啟,伏願聽覽之閑,曲垂提訓,得使微臣鑽仰,成其篤習,後昆好事,專門有奉。自惟多幸,懽沐道於堯年,肄業終身,不知老之將至。天尊不聞,而冒陳請,冰谷置懷,罔識攸厝。」詔答曰:「設《卦》觀象,事遠文高,作《繫》表言,辭深理奧,東魯絕編之思,西伯幽憂之作,事逾三古,人更
 七聖,自商瞿稟承,子庸傳授,篇簡湮沒,歲月遼遠。田生表菑川之譽,梁丘擅琅邪之學,代郡范生,山陽王氏,人藏荊山之寶,各盡玄言之趣,說或去取,意有詳略。近搢紳之學,咸有稽疑,隨答所問,已具別解。知與張譏等三百一十二人須釋《乾》、《坤》、《文言》及《二繫》,萬機小暇,試當討論。」



 弘正博物知玄象,善占候。大同末,嘗謂弟弘讓曰:「國家厄運,數年當有兵起,吾與汝不知何所逃之。」及梁武帝納侯景,弘正謂弘讓曰:「亂階此矣。」京城陷,弘直為衡
 陽內史,元帝在江陵,遺弘直書曰:「適有都信,賢兄博士平安。



 但京師搢紳,無不附逆,王克已為家臣,陸緬身充卒伍,唯有周生,確乎不拔。言及西軍,潺湲掩淚,恒思吾至,如望歲焉,松柏後凋,一人而已。」王僧辯之討侯景也,弘正與弘讓自拔迎軍,僧辯得之甚喜,即日啟元帝,元帝手書與弘正曰:「獯醜逆亂,寒暑亟離,海內相識,零落略盡。韓非之智,不免秦獄,劉歆之學,猶弊亡新,音塵不嗣,每以耿灼。常欲訪山東而尋子雲,問關西而求伯起,
 遇有今信,力附相聞,遲比來郵,慰其延佇。」仍遣使迎之,謂朝士曰:「晉氏平吳,喜獲二陸,今我破賊,亦得兩周,今古一時,足為連類。」及弘正至,禮數甚優,朝臣無與比者。授黃門侍郎,直侍中省。俄遷左民尚書,尋加散騎常侍。



 元帝嘗著《金樓子》,曰:「余於諸僧重招提琰法師,隱士重華陽陶貞白,士大夫重汝南周弘正,其於義理,清轉無窮,亦一時之名士也。」及侯景平,僧辯啟送秘書圖籍,敕弘正讎校。



 時朝議遷都,朝士家在荊州者,皆不欲遷,唯
 弘正與僕射王裒言於元帝曰:「若束脩以上諸士大夫微見古今者,知帝王所都本無定處,無所與疑。至如黔首萬姓,若未見輿駕入建鄴,謂是列國諸王,未名天子。今宜赴百姓之心,從四海之望。」



 時荊陜人士咸云王、周皆是東人,志願東下,恐非良計。弘正面折之曰:「若東人勸東,謂為非計,君等西人欲西,豈成良策?」元帝乃大笑之,竟不還都。



 及江陵陷,弘正遁圍而出,歸於京師,敬帝以為大司馬王僧辯長史,行揚州事。



 太平元年,授侍中,
 領國子祭酒,遷太常卿、都官尚書。高祖受禪,授太子詹事。



 天嘉元年,遷侍中、國子祭酒,往長安迎高宗。三年,自周還,詔授金紫光祿大夫,加金章紫綬,領慈訓太僕。廢帝嗣位,領都官尚書,總知五禮事。仍授太傅長史,加明威將軍。高宗即位,遷特進,重領國子祭酒,豫州大中正,加扶。太建五年,授尚書右僕射,祭酒、中正如故。尋敕侍東宮講《論語》、《孝經》。太子以弘正朝廷舊臣,德望素重,於是降情屈禮,橫經請益,有師資之敬焉。



 弘正特善玄言,
 兼明釋典,雖碩學名僧,莫不請質疑滯。六年,卒於官,時年七十九。詔曰:「追遠褒德,抑有恒規。故尚書右僕射、領國子祭酒、豫州大中正弘正,識宇凝深,藝業通備,辭林義府,國老民宗,道映庠門,望高禮閣,卒然殂殞,朕用惻然。可贈侍中、中書監,喪事所須,量加資給。」便出臨哭。謚曰簡子。



 所著《周易講疏》十六卷,《論語疏》十一卷,《莊子疏》八卷,《老子疏》五卷,《孝經疏》兩卷,《集》二十卷,行於世。子墳,官至吏部郎。



 弘正二弟:弘讓,弘直。弘讓性簡素,博學多
 通,天嘉初,以白衣領太常卿、光祿大夫,加金章紫綬。



 弘直字思方,幼而聰敏。解褐梁太學博士,稍遷西中郎湘東王外兵記室參軍,與東海鮑泉、南陽宗懍、平原劉緩、沛郡劉同掌書記。入為尚書儀曹郎。湘東王出鎮江、荊二州,累除錄事咨議參軍,帶柴桑、當陽二縣令。及梁元帝承制,授假節、英果將軍、世子長史。尋除智武將軍、衡陽內史。遷貞毅將軍、平南長史、長沙內史,行湘州府州事,湘濱縣侯,邑六百戶。歷邵陵、零陵太守、雲麾將
 軍、昌州刺史。王琳之舉兵也,弘直在湘州,琳敗,乃還朝。天嘉中,歷國子博士、廬陵王長史、尚書左丞、領羽林監、中散大夫、秘書監,掌國史。遷太常卿、光祿大夫,加金章紫綬。



 太建七年,遇疾且卒,乃遺疏敕其家曰:「吾今年已來,筋力減耗,可謂衰矣,而好生之情,曾不自覺,唯務行樂,不知老之將至。今時制云及,將同朝露,七十餘年,頗經稱足,啟手告全,差無遺恨。氣絕已後,便買市中見材,材必須小形者,使易提挈。斂以時服,古人通制,但下見
 先人,必須備禮,可著單衣裙衫故履。既應侍養,宜備紛兌,或逢善友,又須香煙,棺內唯安白布手巾、粗香爐而已,其外一無所用。」卒於家,時年七十六。有集二十卷。子確。



 確字士潛,美容儀,寬大有行檢,博涉經史,篤好玄言,世父弘正特所鐘愛。



 解褐梁太學博士、司徒祭酒、晉安王主簿。高祖受禪,除尚書殿中郎,累遷安成王限內記室。高宗即位,授東宮通事舍人,丁母憂,去職。及歐陽紇平,
 起為中書舍人,命於廣州慰勞,服闋,為太常卿。歷太子中庶子、尚書左丞、太子家令,以父憂去職。尋起為貞威將軍、吳令,確固辭不之官。至德元年,授太子左衛率、中書舍人,遷散騎常侍,加貞威將軍、信州南平王府長史,行揚州事,為政平允,稱為良吏。遷都官尚書。禎明初,遘疾、卒於官,時年五十九。詔贈散騎常侍、太常卿,官給喪事。



 袁憲,字德章,尚書左僕射樞之弟也。幼聰敏,好學,有雅
 量。梁武帝修建庠序,別開五館,其一館在憲宅西,憲常招引諸生,與之談論,每有新議,出人意表,同輩咸嗟服焉。



 大同八年,武帝撰《孔子正言章句》,詔下國學,宣制旨義。憲時年十四,被召為國子《正言》生,謁祭酒到溉,溉目而送之,愛其神彩。在學一歲,國子博士周弘正謂憲父君正曰:「賢子今茲欲策試不?」君正曰:「經義猶淺,未敢令試。」



 居數日,君正遣門下客岑文豪與憲候弘正,會弘正將登講坐,弟子畢集,乃延憲入室,授以麈尾,令憲樹義。
 時謝岐、何妥在坐,弘正謂曰:「二賢雖窮奧賾,得無憚此後生耶!」何、謝於是遞起義端,深極理致,憲與往復數番,酬對閑敏。弘正謂妥曰:「恣卿所問,勿以童稚相期。」時學眾滿堂,觀者重沓,而憲神色自若,辯論有餘。弘正請起數難,終不能屈,因告文豪曰:「卿還咨袁吳郡,此郎已堪見代為博士矣。」時生徒對策,多行賄賂,文豪請具束脩,君正曰:「我豈能用錢為兒買第耶?」學司銜之。及憲試,爭起劇難,憲隨問抗答,剖析如流,到溉顧憲曰:「袁君正其
 有後矣。」及君正將之吳郡,溉祖道於征虜亭,謂君正曰:「昨策生蕭敏孫、徐孝克,非不解義,至於風神器局,去賢子遠矣。」尋舉高第。以貴公子選尚南沙公主,即梁簡文之女也。



 大同元年,釋褐秘書郎。太清二年,遷太子舍人。侯景寇逆,憲東之吳郡,尋丁父憂,哀毀過禮。敬帝承制,徵授尚書殿中郎。高祖作相,除司徒戶曹。永定元年,授中書侍郎,兼散騎常侍。與黃門侍郎王瑜使齊,數年不遣,天嘉初乃還。四年,詔復中書侍郎,直侍中省。太建元
 年,除給事黃門侍郎,仍知太常事。二年,轉尚書吏部侍郎,尋除散騎常侍,侍東宮。三年,遷御史中丞,領羽林監。時豫章王叔英不奉法度,逼取人馬,憲依事劾奏,叔英由是坐免黜,自是朝野皆嚴憚焉。



 憲詳練朝章,尤明聽斷,至有獄情未盡而有司具法者,即伺閑暇,常為上言之,其所申理者甚眾。嘗陪宴承香閣,賓退之後,高宗留憲與衛尉樊俊徙席山亭,談宴終日。高宗目憲而謂俊曰「袁家故為有人」,其見重如此。



 五年,入為侍中。六年,除
 吳郡太守,以父任固辭不拜,改授明威將軍、南康內史。九年,秩滿,除散騎常侍,兼吏部尚書,尋而為真。憲以久居清顯,累表自求解任。高宗曰:「諸人在職,屢有謗書。卿處事已多,可謂清白,別相甄錄,且勿致辭。」十三年,遷右僕射,參掌選事。先是憲長兄簡懿子為左僕射,至是憲為右僕射,臺省內目簡懿為大僕射,憲為小僕射,朝廷榮之。



 及高宗不豫,憲與吏部尚書毛喜俱受顧命。始興王叔陵之肆逆也,憲指麾部分,預有力焉。後主被瘡病
 篤,執憲手曰:「我兒尚幼,後事委卿。」憲曰:「群情喁喁,冀聖躬康復,後事之旨,未敢奉詔。」以功封建安縣伯,邑四百戶,領太子中庶子,餘並如故。尋除侍中、信威將軍、太子詹事。



 至德元年,太子加元服,二年,行釋奠之禮,憲於是表請解職,後主不許,給扶二人,進號雲麾將軍,置佐史。皇太子頗不率典訓,憲手表陳諫凡十條,皆援引古今,言辭切直,太子雖外示容納,而心無悛改。後主欲立寵姬張貴妃子始安王為嗣,嘗從容言之,吏部尚書蔡徵
 順旨稱賞,憲厲色折之曰:「皇太子國家儲嗣,億兆宅心。卿是何人,輕言廢立!」夏,竟廢太子為吳興王。後主知憲有規諫之事,歎曰「袁德章實骨鯁之臣」,即日詔為尚書僕射。



 禎明三年,隋軍來伐,隋將賀若弼進燒宮城北掖門,宮衛皆散走,朝士稍各引去,惟憲衛侍左右。後主謂憲曰:「我從來待卿不先餘人,今日見卿,可謂歲寒知松柏後凋也。」後主遑遽將避匿,憲正色曰:「北兵之入,必無所犯,大事如此,陛下安之。臣願陛下正衣冠,御前殿,依
 梁武見侯景故事。」後主不從,因下榻馳去。憲從後堂景陽殿入,後主投下井中,憲拜哭而出。



 京城陷,入於隋,隋授使持節、昌州諸軍事、開府儀同三司、昌州刺史。開皇十四年,詔授晉王府長史。十八年卒,時年七十。贈大將軍,安城郡公,謚曰簡。



 長子承家,仕隋至祕書丞、國子司業。



 史臣曰:梁元帝稱士大夫中重汝南周弘正,信哉斯言也!觀其雅量標舉,尤善玄言,亦一代之國師矣。袁憲風
 格整峻,徇義履道。韓子稱為人臣委質,心無有二。



 憲弗渝終始,良可嘉焉。



\end{pinyinscope}