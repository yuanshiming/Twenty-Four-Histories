\article{卷六本紀第六後主}

\begin{pinyinscope}

 後主,諱叔寶,字元秀,小字黃奴,高宗嫡長子也。梁承聖二年十一月戊寅生于江陵。明年,江陵陷,高宗遷關右,留後主於穰城。天嘉三年,歸京師,立為安成王世子。天
 康元年,授寧遠將軍,置佐史。光大二年,為太子中庶子,尋遷侍中,餘如故。太建元年正月甲午,立為皇太子。



 十四年正月甲寅,高宗崩。乙卯,始興王叔陵作逆,伏誅。丁巳,太子即皇帝位于太極前殿。詔曰:「上天降禍,大行皇帝奄棄萬國,攀號擗踴,無所迨及。朕以哀煢,嗣膺寶歷,若涉巨川,罔知攸濟,方賴群公,用匡寡薄。思播遺德,覃被億兆,凡厥遐邇,咸與惟新。可大赦天下。在位文武及孝悌力田為父後者,並賜爵一級。孤老鰥寡不能自存
 者,賜穀人五斛、帛二匹。」癸亥,以侍中、翊前將軍、丹陽尹長沙王叔堅為驃騎將軍、開府儀同三司、揚州刺史,右衛將軍蕭摩訶為車騎將軍、南徐州刺史,鎮西將軍、荊州刺史樊毅進號征西將軍,平南將軍、豫州刺史任忠進號鎮南將軍,護軍將軍沈恪為特進、金紫光祿大夫,平西將軍魯廣達進號安西將軍,仁武將軍、豊州刺史章大寶為中護軍。乙丑,尊皇后為皇太后,宮曰弘範。



 丙寅,以冠軍將軍晉熙王叔文為宣惠將軍、丹陽尹。丁卯,
 立弟叔重為始興王,奉昭烈王祀,己巳,立妃沈氏為皇后。辛未,立皇弟叔儼為尋陽王,皇弟叔慎為岳陽王,皇弟叔達為義陽王,皇弟叔熊為巴山王,皇弟叔虞為武昌王。壬申,侍中、中權將軍、開府儀同三司鄱陽王伯山進號中權大將軍,軍師將軍、尚書左僕射晉安王伯恭進號翊前將軍、侍中,翊右將軍、中領軍廬陵王伯仁進號安前將軍,鎮南將軍、江州刺史豫章王叔英進號征南將軍,平南將軍、湘州刺史建安王叔卿進號安南將
 軍。



 以侍中、中書監、安右將軍徐陵為左光祿大夫,領太子少傅。甲戌,設無珝大會於太極前殿。三月辛亥,詔曰:「躬推為勸,義顯前經,力農見賞,事昭往誥。斯乃國儲是資,民命攸屬,豊儉隆替,靡不由之。夫入賦自古,輸槁惟舊,沃饒貴于十金,磽确至於三易,腴脊既異,盈縮不同。詐偽日興,簿書歲改。稻田使者,著自西京,不實峻刑,聞諸東漢。老農懼於祗應,俗吏因以侮文。輟耒成群,游手為伍,永言妨蠹,良可太息。今陽和在節,膏澤潤下,宜展
 春耨,以望秋坻。其有新闢塍畎,進墾蒿萊,廣袤勿得度量,征租悉皆停免。私業久廢,咸許占作,公田荒縱,亦隨肆勤。儻良守教耕,淳民載酒,有茲督課,議以賞擢。外可為格班下,稱朕意焉。」癸亥,詔曰:「夫體國經野,長世字氓,雖因革儻殊,馳張或異,至於旁求俊乂,爰逮側微,用適和羹,是隆大廈,上智中主,咸由此術。朕以寡薄,嗣膺景祚,雖哀疚在躬,情慮鋋舛,而宗社任重,黎庶務殷,無由自安拱默,敢忘康濟,思所以登顯髦彥,式備周行。但空
 勞宵夢,屢勤史卜,五就莫來,八能不至。



 是用申旦凝慮,丙夜損懷。豈以食玉炊桂,無因自達?將懷寶迷邦,咸思獨善?應內外眾官九品已上,可各薦一人,以會彙征之旨。且取備實難,舉長或易,小大之用,明言所施,勿得南箕北斗,名而非實。其有負能仗氣,擯壓當時,著《賓戲》以自憐,草《客嘲》以慰志,人生一世,逢遇誠難,亦宜去此幽谷,翔茲天路,趨銅駝以觀國,望金馬而來庭,便當隨彼方圓,飭之矩矱。」又詔曰:「昔睿后宰民,哲王御宇,雖德稱
 汪濊,明能普燭,猶復紆己乞言,降情訪道,高咨岳牧,下聽輿臺,故能政若神明,事無悔吝。朕纂承丕緒,思隆大業,常懼九重已邃,四聰未廣,欲聽昌言,不疲痺足,若逢廷折,無憚批鱗。而口柔之辭,儻聞於在位,腹誹之意,或隱於具僚,非所以弘理至公,緝熙帝載者也。內外卿士文武眾司,若有智周政術,心練治體,救民俗之疾苦,辯禁網之疏密者,各進忠讜,無所隱諱。朕將虛己聽受,擇善而行,庶深鑒物情,匡我王度。」己巳,以侍中、尚書左僕
 射、新除翊前將軍晉安王伯恭為安南將軍、湘州刺史,新除翊左將軍、永陽王伯智為尚書僕射,中護軍章大寶為豊州刺史。夏四月丙申,立皇子永康公胤為皇太子,賜天下為父後者爵一級,王公已下賚帛各有差。庚子,詔曰:「朕臨御區宇,撫育黔黎,方欲康濟澆薄,蠲省繁費,奢僭乖衷,實宜防斷。應鏤金銀薄及庶物化生土木人綵花之屬,及布帛幅尺短狹輕疏者,並傷財廢業,尤成蠹患。又僧尼道士,挾邪左道,不依經律,民間淫祀妖
 書諸珍怪事,詳為條制,並皆禁絕。」癸卯,詔曰:「中歲克定淮、泗,爰涉青、徐,彼土酋豪,並輸罄誠款,分遣親戚,以為質任。今舊土淪陷,復成異域,南北阻遠,未得會同,念其分乖,殊有愛戀。夷狄吾民,斯事一也,何獨譏禁,使彼離析?外可即檢任子館及東館并帶保任在外者,並賜衣糧,頒之酒食,遂其鄉路,所之阻遠,便發遣船仗衛送,必令安達。若已預仕宦及別有事義不欲去者,亦隨其意。」六月癸酉朔,以明威將軍、通直散騎常侍孫瑒為中護
 軍。秋七月辛未,大赦天下。是月,江水色赤如血,自京師至于荊州。八月癸未夜,天有聲如風水相擊。乙酉夜亦如之。丙戌,以使持節、都督緣江諸軍事、安西將軍魯廣達為安左將軍。九月丙午,設無珝大會於太極殿,捨身及乘輿御服,大赦天下。辛亥夜,天東北有聲如蟲飛,漸移西北。乙卯,太白晝見。丙寅,以驃騎將軍、開府儀同三司、揚州刺史長沙王叔堅為司空,征南將軍、江州刺史豫章王叔英即本號開府儀同三司。



 至德元年春正月壬寅,詔曰:「朕以寡薄,嗣守鴻基,哀惸切慮,疹恙纏織,訓俗少方,臨下靡算,懼甚踐冰,心慄同馭朽。而四氣易流,三光遄至,纓紱列陛,玉帛充庭,具物匪新,節序疑舊,緬思前德,永慕昔辰,對軒闥而哽心,顧枿筵而慓氣。思所以仰遵遺構,俯勵薄躬,陶鑄九流,休息百姓,用弘寬簡,取葉陽和。



 可大赦天下,改太建十五年為至德元年。」以征南將軍、江州刺史、新除開府儀同三司豫章王叔英為中衛大將軍,驃騎將軍、開府儀同三
 司、揚州刺史長沙王叔堅為江州刺史,征東將軍、開府儀同三司、東揚州刺史司馬消難進號車騎將軍,宣惠將軍、丹陽尹晉熙王叔文為揚州刺史,鎮南將軍、南豫州刺史任忠為領軍將軍,安左將軍魯廣達為平南將軍、南豫州刺史,祠部尚書江總為吏部尚書。癸卯,立皇子深為始安王。二月丁丑,以始興王叔重為揚州刺史。夏四月戊辰,交州刺史李幼榮獻馴象。己丑,以前輕車將軍、揚州刺史晉熙王叔文為江州刺史。秋八月丁卯,
 以驃騎將軍、開府儀同三司長沙王叔堅為司空。九月丁巳,天東南有聲如蟲飛。冬十月丁酉,立皇弟叔平為湘東王,叔敖為臨賀王,叔宣為陽山王,叔穆為西陽王。戊戌,侍中、安右將軍、左光祿大夫、太子少傅徐陵卒。癸丑,立皇弟叔儉為南安王,叔澄為南郡王,叔興為沅陵王,叔韶為岳山王,叔純為新興王。十二月丙辰,頭和國遣使獻方物。司空長沙王叔堅有罪免。戊午夜,天開自西北至東南,其內有青黃色,隆隆若雷聲。



 二年春正月丁卯,分遣大使巡省風俗。平南將軍、豫州刺史魯廣達進號安南將軍。癸巳,大赦天下。夏五月戊子,以尚書僕射永陽王伯智為平東將軍、東揚州刺史,輕車將軍、江州刺史晉熙王叔文為信威將軍、湘州刺史,仁威將軍、揚州刺史始興王叔重為江州刺史,信武將軍、南琅邪彭城二郡太守南平王嶷為揚州刺史,吏部尚書江總為尚書僕射。秋七月戊辰,以長沙王叔堅為侍中、鎮左將軍。壬午,太子加元服,在位文武賜帛各
 有差,孝悌力田為父後者各賜一級,鰥寡癃老不能自存者人穀五斛。九月癸未,太白晝見。冬十月己酉,詔曰:「耕鑿自足,乃曰淳風,貢賦之興,其來尚矣。蓋由庚極務,不獲已而行焉。但法令滋章,姦盜多有,俗尚澆詐,政鮮惟良。朕日旰夜分,矜一物之失所,泣辜罪己,愧三千之未措。望訂初下,使彊蔭兼出,如聞貧富均起,單弱重弊,斯豈振窮扇曷之意歟?是乃下吏箕斂之苛也。故云『百姓不足,君孰與足』。自太建十四年望訂租調逋未入者,
 並悉原除。在事百僚,辯斷庶務,必去取平允,無得便公害民,為己聲績,妨紊政道。」



 十一月丙寅,大赦天下。壬申,盤盤國遣使獻方物。戊寅,百濟國遣使獻方物。



 三年春正月戊午朔,日有蝕之。庚午,以鎮左將軍長沙王叔堅即本號開府儀同三司,征西將軍、荊州刺史樊毅為護軍將軍,守吏部尚書、領著作陸瓊為吏部尚書,金紫光祿大夫袁敬加特進。三月辛酉,前豊州刺史章大寶舉兵反。夏四月庚戌,豊州義軍主陳景詳斬大寶,
 傳首京師。秋八月戊子夜,老人星見。己酉,以左民尚書謝伷為吏部尚書。九月甲戌,特進、金紫光祿大夫袁敬卒。冬十月己丑,丹丹國遣使獻方物。十一月己未,詔曰:「宣尼誕膺上哲,體資至聖,祖述憲章之典,並天地而合德,樂正雅頌之奧,與日月而偕明,垂後昆之訓範,開生民之耳目。梁季湮微,靈寢忘處,鞠為茂草,三十餘年,敬仰如在,永惟愾息。今《雅道》雍熙,《由庚》得所,斷琴故履,零落不追,閱笥開書,無因循復。外可詳之禮典,改築舊廟,
 蕙房桂棟,咸使惟新,芳繁潔潦,以時饗奠。」辛巳,輿駕幸長干寺,大赦天下。十二月丙戌,太白晝見。辛卯,皇太子出太學,講《孝經》,戊戌,講畢。



 辛丑,釋奠于先師,禮畢,設金石之樂,會宴王公卿士。癸卯,高麗國遣使獻方物。



 是歲,蕭巋死,子琮代立。



 四年春正月甲寅,詔曰:「堯施諫鼓,禹拜昌言,求之異等,久著前徽,舉以淹滯,復聞昔典,斯乃治道之深規,帝王之切務。朕以寡昧,丕承鴻緒,未明虛己,日旰興懷,萬機
 多紊,四聰弗達,思聞蹇諤,採其謀計。王公已下,各薦所知,旁詢管庫,爰及輿皁,一介有能,片言可用,朕親加聽覽,佇於啟沃。」中權大將軍、開府儀同三司鄱陽王伯山進號鎮衛將軍,中衛大將軍、開府儀同三司豫章王叔英進號驃騎大將軍,鎮左將軍、開府儀同三司長沙王叔堅進號中軍大將軍,安南將軍晉安王伯恭進號鎮右將軍,翊右將軍宜都王叔明進號安右將軍。二月丙戌,以鎮右將軍晉安王伯恭為特進。丙申,立皇弟叔謨
 為巴東王,叔顯為臨江王,叔坦為新會王,叔隆為新寧王。夏五月丁巳,立皇子莊為會稽王。秋九月甲午,輿駕幸玄武湖,肆艫艦閱武,宴群臣賦詩。戊戌,以鎮衛將軍、開府儀同三司鄱陽王伯山為東揚州刺史,智武將軍岳陽王叔慎為丹陽尹。丁未,百濟國遣使獻方物。冬十月癸亥,尚書僕射江總為尚書令,吏部尚書謝伷為尚書僕射。十一月己卯,詔曰:「惟刑止暴,惟德成物,三才是資,百王不改。而世無抵角,時鮮犯鱗,渭橋驚馬,弗聞廷
 爭,桃林逸牛,未見其旨。雖剽悍輕侮,理從鉗棨,蠢愚杜默,宜肆矜弘,政乏良哉,明慚則哲,求諸刑措,安可得乎?是用屬寤寐以軫懷,負黼扆而於邑。復茲合璧輪缺,連珠緯舛,黃鐘獻呂,和氣始萌,玄英告中,履長在御,因時宥過,抑乃斯得。



 可大赦天下。」



 禎明元年春正月丙子,以安前將軍衡陽王伯信進號鎮前將軍,安東將軍、吳興太守廬陵王伯仁為特進,智武將軍、丹陽尹岳陽王叔慎為湘州刺史,仁武將軍義
 陽王叔達為丹陽尹。戊寅,詔曰:「柏皇、大庭,鼓淳和於曩日,姬王、嬴后,被澆風於末載,刑書已鑄,善化匪融,禮義既乖,姦宄斯作。何其淳朴不反,浮華競扇者歟?朕居中御物,納隍在眷,頻恢天網,屢絕三邊,元元黔庶,終罹五辟。蓋乃康哉寡薄,抑焉法令滋章。是用當宁弗怡,矜此向隅之意。今三元具序,萬國朝辰,靈芝獻於始陽,膏露凝於聿歲,從春施令,仰乾布德,思與九有,惟新七政。可大赦天下,改至德五年為禎明元年。」乙未,地震。癸卯,以
 鎮前將軍衡陽王伯信為鎮南將軍、西衡州刺史。二月丁未,以特進、鎮右將軍晉安王伯恭進號中衛將軍,中書令建安王叔卿為中書監。丁卯,詔至德元年望訂租調逋未入者,並原之。秋八月癸卯,老人星見。丁未,以車騎將軍蕭摩訶為驃騎將軍。九月乙亥,以驃騎將軍、開府儀同三司豫章王叔英為驃騎大將軍。庚寅,蕭琮所署尚書令、太傅安平王蕭巖,中軍將軍、荊州刺史義興王蕭獻,遣其都官尚書沈君公,詣荊州刺史陳紀請降。



 辛卯,巖等率文武男女十萬餘口濟江。甲午,大赦天下。冬十一月乙亥,割揚州吳郡置吳州,割錢塘縣為郡,屬焉。丙子,以蕭巖為平東將軍、開府儀同三司、東揚州刺史,蕭獻為安東將軍、吳州刺史。丁亥,以驃騎大將軍、開府儀同三司豫章王叔英兼司徒。十二月丙辰,以前鎮衛將軍、開府儀同三司、東揚州刺史鄱陽王伯山為鎮衛大將軍、開府儀同三司,前中衛將軍晉安王伯恭為中衛將軍、右光祿大夫。



 二年春正月辛巳,立皇子恮為東陽王,恬為錢塘王。是月,遣散騎常侍周羅嵒帥兵屯峽口。夏四月戊申,有群鼠無數,自蔡洲岸入石頭渡淮,至于青塘兩岸,數日死,隨流出江。戊午,以左民尚書蔡徵為吏部尚書。是月,郢州南浦水黑如墨。



 五月壬午,以安前將軍廬陵王伯仁為特進。甲午,東冶鑄鐵,有物赤色如數斗,自天墜熔所,有聲隆隆如雷,鐵飛出牆外,燒民家。六月戊戌,扶南國遣使獻方物。



 庚子,廢皇太子胤為吳興王,立軍師將軍、
 揚州刺史始安王深為皇太子。辛丑,平南將軍、江州刺史南平王嶷進號鎮南將軍;忠武將軍、南徐州刺史永嘉王彥進號安北將軍;會稽王莊為翊前將軍、揚州刺史;宣惠將軍、尚書令江總進號中權將軍;雲麾將軍、太子詹事袁憲為尚書僕射;尚書僕射謝伷為特進;寧遠將軍、新除吏部尚書蔡徵進號安右將軍。甲辰,以安右將軍魯廣達為中領軍。丁巳,大風至自西北激濤水入石頭城,淮渚暴益,漂沒舟乘。冬十月己亥,立皇子蕃為
 吳郡王。辛丑,以度支尚書、領大著作姚察為吏部尚書。己酉,輿駕幸莫府山,大校獵。十一月丁卯,詔曰:「夫議獄緩刑,皇王之所垂範,勝殘去殺,仁人之所用心。自畫冠既息,刻吏斯起,法令滋章,手足無措。朕君臨區宇,屬當澆末,輕重之典,在政未康,小大之情,興言多愧。眷茲狴犴,有軫哀矜,可克日於大政殿訊獄。」壬申,以鎮南將軍、江州刺史南平王嶷為征西將軍、郢州刺史,安北將軍、南徐州刺史永嘉王彥為安南將軍、江州刺史,軍師將
 軍南海王虔為安北將軍、南徐州刺史。丙子,立皇弟叔榮為新昌王,叔匡為太原王。是月,隋遣晉王廣眾軍來伐,自巴、蜀、沔、漢下流至廣陵,數十道俱入,緣江鎮戍,相繼奏聞。時新除湘州刺史施文慶、中書舍人沈客卿掌機密用事,並抑而不言,故無備禦。



 三年春正月乙丑朔,霧氣四塞。是日,隋總管賀若弼自北道廣陵濟京口,總管韓擒虎趨橫江,濟采石,自南道將會弼軍。丙寅,采石戍主徐子建馳啟告變。丁卯,召公
 卿入議軍旅。戊辰,內外戒嚴,以驃騎將軍蕭摩訶、護軍將軍樊毅、中領軍魯廣達並為都督,遣南豫州刺史樊猛帥舟師出白下,散騎常侍皋文奏將兵鎮南豫州。



 庚午,賀若弼攻陷南徐州。辛未,韓擒虎又陷南豫州,文奏敗還。至是隋軍南北道並進。後主遣驃騎大將軍、司徒豫章王叔英屯朝堂,蕭摩訶屯樂遊苑,樊毅屯耆闍寺,魯廣達屯白土岡,忠武將軍孔範屯寶田寺。己卯,鎮東大將軍任忠自吳興入赴,仍屯朱雀門。辛巳,賀若弼進
 據鐘山,頓白土岡之東南。甲申,後主遣眾軍與弼合戰,眾軍敗績。弼乘勝至樂遊苑,魯廣達猶督散兵力戰,不能拒。弼進攻宮城,燒北掖門。是時韓擒虎率眾自新林至于石子岡,任忠出降於擒虎,仍引擒虎經朱雀航趣宮城,自南掖門而入。於是城內文武百司皆遁出,唯尚書僕射袁憲在殿內。尚書令江總、吏部尚書姚察、度支尚書袁權、前度支尚書王瑗、侍中王寬在省中。後主聞兵至,從宮人十餘出後堂景陽殿,將自投于井。袁憲侍
 側,苦諫不從,後閣舍人夏侯公韻又以身蔽井,後主與爭久之,方得入焉。及夜,為隋軍所執。丙戌,晉王廣入據京城。三月己巳,後主與王公百司發自建鄴,入于長安。隋仁壽四年十一月壬子,薨於洛陽,時年五十二。追贈大將軍,封長城縣公,謚曰煬,葬河南洛陽之芒山。



 史臣侍中鄭國公魏徵曰:高祖拔起壟畝,有雄桀之姿。始佐下籓,奮英奇之略,弭節南海,職思靜亂。援旗北邁,義在勤王,掃侯景於既成,拯梁室於已墜。天網絕而復
 續,國步屯而更康,百神有主,不失舊物。魏王之延漢鼎祚,宋武之反晉乘輿,懋績鴻勳,無以尚也。于時內難未弭,外鄰勍敵,王琳作梗於上流,周、齊搖蕩於江、漢,畏首畏尾,若存若亡,此之不圖,遽移天歷,雖皇靈有眷,何其速也?



 然志度弘遠,懷抱豁如,或取士於仇讎,或擢才於亡命,掩其受金之過,宥其吠堯之罪,委以心腹爪牙,咸能得其死力,故乃決機百勝,成此三分,方諸鼎峙之雄,足以無慚權、備矣。世祖天姿睿哲,清明在躬,早預經綸,
 知民疾苦,思擇令典,庶幾至治。德刑並用,戡濟艱虞,群兇授首,彊鄰震懾。雖忠厚之化未能及遠,恭儉之風足以垂訓,若不尚明察,則守文之良主也。臨川年長於成王,過微於太甲,宣帝有周公之親,無伊尹之志,明避不復,桐宮遂往,欲加之罪,其無辭乎!高宗爰自在田,雅量宏廓,登庸御極,民歸其厚,惠以使下,寬以容眾。智勇爭奮,師出有名,揚旆分麾,風行電掃,闢土千里,奄有淮、泗,戰勝攻取之勢,近古未之有也。既而君侈民勞,將驕卒
 墮,帑藏空竭,折衄師徒,於是秦人方彊,遂窺兵於江上矣。李克以為吳之先亡,由乎數戰數勝,數戰則民疲,數勝則主驕,以驕主御疲民,未有不亡者也。信哉言乎!高宗始以寬大得人,終以驕侈致敗,文、武之業,墜于茲矣。後主生深宮之中,長婦人之手,既屬邦國殄瘁,不知稼穡艱難。初懼阽危,屢有哀矜之詔,後稍安集,復扇淫侈之風。賓禮諸公,唯寄情於文酒,暱近群小,皆委之以衡軸。謀謨所及,遂無骨鯁之臣,權要所在,莫匪侵漁之吏。
 政刑日紊,尸素盈朝,躭荒為長夜之飲,嬖寵同艷妻之孽。危亡弗恤,上下相蒙,眾叛親離,臨機不寤,自投於井,冀以茍生,視其以此求全,抑亦民斯下矣。遐觀列辟,纂武嗣興,其始也皆欲齊明日月,合德天地,高視五帝,俯協三王,然而靡不有初,克終蓋寡,其故何哉?並以中庸之才,懷可移之性,口存於仁義,心怵於嗜慾。仁義利物而道遠,嗜欲遂性而便身。便身不可久違,道遠難以固志。佞諂之倫,承顏候色,因其所好,以悅導之,若下阪以
 走丸,譬順流而決壅。非夫感靈辰象,降生明德,孰能遺其所樂,而以百姓為心哉?此所以成、康、文、景千載而罕遇,癸、辛、幽、厲靡代而不有,毒被宗社,身嬰戮辱,為天下笑,可不痛乎!古人有言,亡國之主,多有才藝,考之梁、陳及隋,信非虛論。然則不崇教義之本,偏尚淫麗之文,徒長澆偽之風,無救亂亡之禍矣。



 史臣曰:後主昔在儲宮,早標令德,及南面繼業,實允天人之望矣。至於禮樂刑政,咸遵故典,加以深弘六藝,廣闢
 四門,是以待詔之徒,爭趨金馬,稽古之秀,雲集石渠。且梯山航海,朝貢者往往歲至矣。自魏正始、晉中朝以來,貴臣雖有識治者,皆以文學相處,罕關庶務,朝章大典,方參議焉。文案簿領,咸委小吏,浸以成俗,迄至于陳。後主因循,未遑改革,故施文慶、沈客卿之徒,專掌軍國要務,奸黠左道,以裒刻為功,自取身榮,不存國計。是以朝經墮廢,禍生鄰國。斯亦運鐘百六,鼎玉遷變,非唯人事不昌,蓋天意然也。



\end{pinyinscope}