\article{卷三十六列傳第三十始興王叔陵 新安王伯固}

\begin{pinyinscope}

 始興王叔陵,字子嵩,高宗之第二子也。梁承聖中,高宗在江陵為直閣將軍,而叔陵生焉。江陵陷,高宗遷關右,叔陵留于穰城。高宗之還也,以後主及叔陵為質。天嘉
 三年,隨後主還朝,封康樂侯,邑五百戶。



 叔陵少機辯,徇聲名,彊梁無所推屈。光大元年,除中書侍郎。二年,出為持節、都督江州諸軍事、南中郎將、江州刺史。太建元年,封始興郡王,奉昭烈王祀。



 進授使持節、都督江、郢、晉三州諸軍事、軍師將軍,刺史如故。叔陵時年十六,政自己出,僚佐莫預焉。性嚴刻,部下懾憚。諸公子姪及罷縣令長,皆逼令事己。



 豫章內史錢法成詣府進謁,即配其子季卿將領馬仗,季卿慚恥,不時至,叔陵大怒,侵辱法成,
 法成憤怨自縊而死。州縣非其部內,亦徵攝案治之,朝貴及下吏有乖忤者,輒誣奏其罪,陷以重辟。尋進號雲麾將軍,加散騎常侍。三年,加侍中。四年,遷都督湘、衡、桂、武四州諸軍事、平南將軍、湘州刺史,侍中、使持節如故。諸州鎮聞其至,皆震恐股慄。叔陵日益暴橫,征伐夷獠,所得皆入己,絲毫不以賞賜。



 徵求役使,無有紀極。夜常不臥,燒燭達曉,呼召賓客,說民間細事,戲謔無所不為。性不飲酒,唯多置肴臠,晝夜食啖而已。自旦至中,方始
 寢寐。其曹局文案,非呼不得輒自呈。笞罪者皆繫獄,動數年不省視。瀟湘以南,皆逼為左右,廛里殆無遺者。其中脫有逃竄,輒殺其妻子。州縣無敢上言,高宗弗之知也。尋進號鎮南將軍,給鼓吹一部,遷中衛將軍。九年,除使持節、都督揚、徐、東揚、南豫四州諸軍事、揚州刺史,侍中、將軍、鼓吹如故。



 十年,至都,加扶,給油幢車。叔陵治在東府,事務多關涉省閣,執事之司,承意順旨,即諷上進用之,微致違忤,必抵以大罪,重者至殊死,道路籍籍,皆
 言其有非常志。叔陵脩飾虛名,每入朝,常於車中馬上執卷讀書,高聲長誦,陽陽自若。歸坐齋中,或自執斧斤為沐猴百戲。又好遊冢墓間,遇有塋表主名可知者,輒令左右發掘,取其石誌古器,并骸骨肘脛,持為玩弄,藏之庫中。府內民間少妻處女,微有色貌者,並即逼納。



 十一年,丁所生母彭氏憂去職。頃之,起為中衛將軍,使持節、都督、刺史如故。晉世王公貴人,多葬梅嶺,及彭卒,叔陵啟求於梅嶺葬之,乃發故太傅謝安舊墓,棄去安柩,
 以葬其母。初喪之日,偽為哀毀,自稱刺血寫《涅槃經》,未及十日,乃令庖廚擊鮮,日進甘膳。又私召左右妻女,與之姦合,所作尤不軌,侵淫上聞。高宗譴責御史中丞王政,以不舉奏免政官,又黜其典簽親事,仍加鞭捶。高宗素愛叔陵,不繩之以法,但責讓而已。服闋,又為侍中、中軍大將軍。



 及高宗不豫,太子諸王並入侍疾。高宗崩于宣福殿,翌日旦,後主哀頓俯伏,叔陵以剉藥刀斫後主中項。太后馳來救焉,叔陵又斫太后數下。後主乳媼吳
 氏,時在太后側,自後掣其肘,後主因得起。叔陵仍持後主衣,後主自奮得免。長沙王叔堅手搤叔陵,奪去其刀,仍牽就柱,以其褶袖縛之。時吳媼已扶後主避賊,叔堅求後主所在,將受命焉。叔陵因奮袖得脫,突走出雲龍門,馳車還東府,呼其甲士,散金銀以賞賜,外召諸王將帥,莫有應者,唯新安王伯固聞而赴之。



 叔陵聚兵僅千人,初欲據城保守,俄而右衛將軍蕭摩訶將兵至府西門,叔陵事急惶恐,乃遣記室韋諒送其鼓吹與摩訶,仍
 謂之曰:「如其事捷,必以公為台鼎。」



 摩訶紿報之,曰「須王心膂節將自來,方敢從命」。叔陵即遣戴溫、譚騏驎二人詣摩訶所,摩訶執以送臺,斬於閣道下。叔陵自知不濟,遂入內沈其妃張氏及寵妾七人于井中。叔陵有部下兵先在新林,於是率人馬數百,自小航渡,欲趨新林,以舟艦入北。行至白楊路,為臺軍所邀,伯固見兵至,旋避入巷,叔陵馳騎拔刃追之,伯固復還。叔陵部下,多棄甲潰散,摩訶馬容陳智深迎刺叔陵,僵斃于地,閹豎王飛
 禽抽刀斫之十數下,馬容陳仲華就斬其首,送于臺。自寅至巳乃定。



 尚書八座奏曰:「逆賊故侍中、中軍大將軍、始興王叔陵,幼而很戾,長肆貪虐。出撫湘南,及鎮九水,兩籓庶,掃地無遺。蜂目豺聲,狎近輕薄,不孝不仁,阻兵安忍,無禮無義,唯戮是聞。及居偏憂,淫樂自恣,產子就館,日月相接。晝伏夜遊,恆習姦詭,抄掠居民,歷發丘墓。謝太傅晉朝佐命,草創江左,斲棺露骸,事驚聽視。自大行皇帝寢疾,翌日未瘳,叔陵以貴介之地,參侍醫藥,
 外無戚容,內懷逆弒。大漸之後,聖躬號擗,遂因匍匐,手犯乘輿。皇太后奉臨,又加鋒刃,窮凶極逆,曠古未儔。賴長沙王叔堅誠孝懇至,英果奮發,手加挫拉,身蔽聖躬。



 叔陵仍奔東城,招集兇黨,餘毒方熾,自害妻孥。雖應時梟懸,猶未攄憤怨,臣等參議,請依宋代故事,流尸中江,汙瀦其室,并毀其所生彭氏墳廟,還謝氏之塋。」



 制曰:「凶逆梟獍,反噬宮闈,賴宗廟之靈,時從殄滅。撫情語事,酸憤兼懷,朝議有章,宜從所奏也。」



 叔陵諸子,即日並賜死。
 前衡陽內史彭暠諮議參軍兼記室鄭信、中錄事參軍兼記室韋諒、典簽俞公喜,並伏誅。暠,叔陵舅也,初隨高宗在關中,頗有勤效,因藉叔陵將領歷陽、衡陽二郡。信以便書記,有寵,謀謨皆預焉。諒,京兆人,梁侍中、護軍將軍粲之子也,以學業為叔陵所引。



 陳智深以誅叔陵之功為巴陵內史,封游安縣子。陳仲華為下巂太守,封新夷縣子。王飛禽除伏波將軍。賜金各有差。



 新安王伯固,字牢之,世祖之第五子也。生而龜胸,目通
 精揚白,形狀眇小,而俊辯善言論。天嘉六年,立為新安郡王,邑二千戶。廢帝嗣立,為使持節、都督南琅邪、彭城、東海三郡諸軍事、雲麾將軍、彭城、琅邪二郡太守。尋入為丹陽尹,將軍如故。



 太建元年,進號智武將軍,尹如故。秩滿,進號翊右將軍。尋授使持節、都督吳興諸軍事、平東將軍、吳興太守。四年,入為侍中、翊前將軍,遷安前將軍、中領軍。七年,出為使持節、散騎常侍、都督南徐、南豫、南、北兗四州諸軍事、鎮北將軍、南徐州刺史。伯固性嗜
 酒,而不好積聚,所得祿俸,用度無節。酣醉以後,多所乞丐,於諸王之中,最為貧窶。高宗每矜之,特加賞賜。伯固雅性輕率,好行鞭捶,在州不知政事,日出田獵,或乘眠轝至於草間,輒呼民下從遊,動至旬日,所捕麞鹿,多使生致。高宗頗知之,遣使責讓者數矣。



 十年,入朝,又為侍中、鎮右將軍,尋除護軍將軍。其年,為國子祭酒,領左驍騎將軍,侍中、鎮右並如故。伯固頗知玄理,而墮業無所通,至於擿句問難,往往有奇意。為政嚴苛,國學有墮
 遊不脩習者,重加檟楚,生徒懼焉,由是學業頗進。



 十二年,領宗正卿。十三年,為使持節、都督揚、南徐、東揚、南豫四州諸軍事、揚州刺史,侍中、將軍如故。齋後主初在東宮,與伯固甚相親狎,伯固又善嘲謔,高宗每宴集,多引之。叔陵在江州,心害其寵,陰求疵瑕,將中之以法。及叔陵入朝,伯固懼罪,諂求其意,乃共訕毀朝賢,歷詆文武,雖耆年高位,皆面折之,無所畏忌。伯因性好射雉,叔陵又好開發冢墓,出遊野外,必與偕行,於是情好大葉,遂謀不
 軌。伯固侍禁中,每有密語,必報叔陵。及叔陵出奔東府,遣使告之,伯固單馬馳赴,助叔陵指揮。



 知事不捷,便欲遁走,會四門已閉不得出,因同趣白揚道。臺馬容至,為亂兵所殺,屍於東昌館門,時年二十八。詔曰:「伯固同茲悖逆,殞身途路。今依外議,意猶弗忍,可特許以庶人禮葬。」又詔曰:「伯固隨同巨逆,自絕于天,俾無遺育,抑有恆典。但童孺靡識,兼預葭莩,置之甸人,良以惻憫,及伯固所生王氏,可並特宥為庶人。」國除。



 史臣曰:孔子稱「富與貴,是人之所欲,非其道得之,不處也」。上自帝王,至於黎獻,莫不嫡庶有差,長幼攸序。叔陵險躁奔競,遂行悖逆,轅褲形骸,未臻其罪,污瀦居處,不足彰過,悲哉!



\end{pinyinscope}