\article{卷十二列傳第六胡穎 徐度 子敬成 杜棱 沈恪}

\begin{pinyinscope}

 胡穎,字方秀,吳興東遷人也。其先寓居吳興,土斷為民。穎偉姿容,性寬厚。



 梁世仕至武陵國侍郎,東宮直前。出
 番禺,征討俚洞,廣州西江督護。高祖在廣州,穎仍自結高祖,高祖與其同郡,接遇甚隆。及南征交趾,穎從行役,餘諸將帥皆出其下。及平李賁,高祖旋師,穎隸在西江,出兵多以穎留守。



 侯景之亂,高祖克元景仲,仍渡嶺援臺,平蔡路養、李遷仕,穎皆有功。歷平固、遂興二縣令。高祖進軍頓西昌,以穎為巴丘縣令,鎮大皋,督糧運。下至豫章,以穎監豫章郡。高祖率眾與王僧辯會于白茅灣,同討侯景,以穎知留府事。



 梁承聖初,元帝授穎假節、
 鐵騎將軍、羅州刺史,封漢陽縣侯,邑五百戶。尋除豫章內史,隨高祖鎮京口。齊遣郭元建出關,都督侯瑱率師禦之。高祖選府內驍勇三千人配穎,令隨瑱,於東關大破之。三年,高祖圍廣陵,齊人東方光據宿預請降,以穎為五原太守,隨杜僧明援光,不克,退還,除曲阿令。尋領馬軍,從高祖襲王僧辯。又隨周文育於吳興討杜龕。紹泰元年,除假節、都督南豫州諸軍事、輕車將軍、南豫州刺史。太平元年,除持節、散騎常侍、仁威將軍。尋兼丹陽尹。



 高祖受禪,兼左衛將軍,餘如故。永定三年,隨侯安都征王琳,于宮亭破賊帥常眾愛等。世祖嗣位,除侍中、都督吳州諸軍事、宣惠將軍、吳州刺史。不行,尋為義興太守,將軍如故。天嘉元年,除散騎常侍、吳興太守。其年六月卒,時年五十四。贈侍中、中護軍,謚曰壯。二年,配享高祖廟庭。子六同嗣。



 穎弟鑠,亦隨穎將軍。穎卒,鑠統其眾。歷東海、豫章二郡守,遷員外散騎常侍。隨章昭達南平歐陽紇,為廣州東江督護。還預北伐,除雄信將軍、歷陽太
 守。



 太建六年卒,贈桂州刺史。



 徐度,字孝節,安陸人也。世居京師。少倜儻,不拘小節。及長,姿貌瑰偉,嗜酒好博。恆使僮僕屠酤為事。梁始興內史蕭介之郡,度從之,將領士卒,徵諸山洞,以驍勇聞。高祖征交趾,厚禮招之,度乃委質。



 侯景之亂,高祖克定廣州,平蔡路養,破李遷仕,計劃多出於度。兼統兵甲,每戰有功。歸至白茅灣,梁元帝授寧朔將軍、合州刺史。侯景平後,追錄前後戰功,加通直散騎常侍,封廣德縣侯,邑
 五百戶。遷散騎常侍。高祖鎮朱方,除信武將軍、蘭陵太守。高祖遣衡陽獻王往荊州,度率所領從焉。江陵陷,間行東歸。高祖平王僧辯,度與侯安都為水軍。紹泰元年,高祖東討杜龕,奉敬帝幸京口,以度領宿衛,并知留府事。



 徐嗣徽、任約等來寇,高祖與敬帝還都。時賊已據石頭城,市廛阜居民,並在南路,去臺遙遠,恐為賊所乘,乃使度將兵鎮于冶城寺,築壘以斷之。賊悉眾來攻,不能克。高祖尋亦救之,大敗約等。明年,嗣徽等又引齊寇濟江,
 度隨眾軍破之於北郊壇。以功除信威將軍、郢州刺史,兼領吳興太守。尋遷鎮右將軍、領軍將軍、徐州緣江諸軍事、鎮北將軍、南徐州刺史,給鼓吹一部。



 周文育、侯安都等西討王琳,敗績,為琳所拘,乃以度為前軍都督,鎮于南陵。



 世祖嗣位,遷侍中、中撫軍將軍、開府儀同三司,進爵為公。未拜,出為使持節、散騎常侍、鎮東將軍、吳郡太守。天嘉元年,增邑千戶。以平王琳功,改封湘東郡公,邑四千戶。秩滿,為侍中、中軍將軍。出為使持節、都督
 會稽、東陽、臨海、永嘉、新安、新寧、信安、晉安、建安九郡諸軍事、鎮東將軍、會稽太守。未行而太尉侯瑱薨于湘州,乃以度代瑱為都督湘、沅、武、巴、郢、桂六州諸軍事、鎮南將軍、湘州刺史。秩滿,為侍中、中軍大將軍,儀同、鼓吹並如故。



 世祖崩,度預顧命,以甲仗五十人入殿省。廢帝即位,進位司空。華皎據湘州反,引周兵下至沌口,與王師相持,乃加度使持節、車騎將軍,總督步軍,自安成郡由嶺路出于湘東,以襲湘州,盡獲其所留軍人家口以歸。光
 大二年薨,時年六十。



 贈太尉,給班劍二十人,謚曰忠肅。太建四年,配享高祖廟庭。子敬成嗣。



 敬成幼聰慧,好讀書,少機警,善占對,結交文義之士,以識鑒知名。起家著作郎。永定元年,領度所部士卒,隨周文育、侯安都征王琳,於沌口敗績,為琳所縶。二年,隨文育、安都得歸,除太子舍人,遷洗馬。度為吳郡太守,以敬成監郡。



 天嘉二年,遷太子中舍人,拜湘東郡公世子。四年,度自湘州還朝,士馬精銳,敬成盡領其眾。隨章昭達
 征陳寶應,晉安平,除貞威將軍、豫章太守。光大元年,華皎謀反,以敬成為假節、都督巴州諸軍事、雲旗將軍、巴州刺史。尋詔為水軍,隨吳明徹徵華皎,皎平還州。二年,以父憂去職。尋起為持節、都督南豫州諸軍事、壯武將軍、南豫州刺史。四年,襲爵湘東郡公,授太子右衛率。



 五年,除貞威將軍、吳興太守。其年隨都督吳明徹北討,出秦郡,別遣敬成為都督,乘金翅自歐陽引埭上溯江由廣陵。齊人皆城守,弗敢出。自繁梁湖下淮,圍淮陰
 城。仍監北兗州。淮、泗義兵相率響應,一二日間,眾至數萬,遂克淮陰、山陽、鹽城三郡,并連口、朐山二戍。仍進攻鬱州,克之。以功加通直散騎常侍、雲旗將軍,增邑五百戶。又進號壯武將軍,鎮朐山。坐于軍中輒科訂,并誅新附,免官。尋復為持節、都督安、元、潼三州諸軍事、安州刺史,將軍如故,鎮宿預。七年卒,時年三十六。贈散騎常侍,謚曰思。子敞嗣。



 杜棱,字雄盛,吳郡錢塘人也。世為縣大姓。棱頗涉書傳,
 少落泊,不為當世所知。遂遊嶺南,事梁廣州刺史新渝侯蕭映。映卒,從高祖,恒典書記。侯景之亂,命棱將領,平蔡路養、李遷仕皆有功。軍至豫章,梁元帝承制授棱仁威將軍、石州刺史,上陌縣侯,邑八百戶。



 侯景平,高祖鎮朱方,棱監義興、琅邪二郡。高祖誅王僧辯,引棱與侯安都等共議,棱難之。高祖懼其泄己,乃以手巾絞稜,棱悶絕于地,因閉於別室。軍發,召與同行。及僧辯平後,高祖東征杜龕等,留棱與安都居守。徐嗣徽、任約引齊寇濟
 江,攻臺城,安都與棱隨方抗拒,棱晝夜巡警,綏撫士卒,未常解帶。賊平,以功除通直散騎常侍、右衛將軍、丹陽尹。永定元年,加侍中、忠武將軍。尋遷中領軍,侍中,將軍如故。



 三年,高祖崩,世祖在南皖。時內無嫡嗣,外有彊敵,侯瑱、侯安都、徐度等並在軍中,朝廷宿將,唯棱在都,獨典禁兵,乃與蔡景歷等秘不發喪,奉迎世祖,事見景歷傳。世祖即位,遷領軍將軍。天嘉元年,以預建立之功,改封永城縣侯,增邑五百戶。出為雲麾將軍,晉陵太守,加
 秩中二千石。二年,徵為侍中、領軍將軍。尋遷翊左將軍、丹陽尹。廢帝即位,遷鎮右將軍、特進,侍中、尹如故。光大元年,解尹,量置佐史,給扶,重授領軍將軍。



 太建元年,出為散騎常侍、鎮東將軍、吳興太守,秩中二千石。二年,徵為侍中、鎮右將軍。尋加特進、護軍將軍。三年,以公事免侍中、護軍。四年,復為侍中、右光祿大夫,并給鼓吹一部,將軍、佐史、扶並如故。



 棱歷事三帝,並見恩寵。末年不預征役,優遊京師,賞賜優洽。頃之卒于官,時年七十。贈開
 府儀同三司,喪事所須,並令資給,謚曰成。其年配享高祖廟庭。



 子安世嗣。



 沈恪,字子恭,吳興武康人也。深沈有幹局。梁新渝侯蕭映為郡將,召為主簿。



 映遷北徐州,恪隨映之鎮。映遷廣州,以恪兼府中兵參軍,常領兵討伐俚洞。盧子略之反也。恪拒戰有功,除中兵參軍。高祖與恪同郡,情好甚暱,蕭映卒後,高祖南討李賁,仍遣妻子附恪還鄉。尋補東宮直後,以嶺南勳除員外散騎侍郎,仍令招集宗
 從子弟。



 侯景圍臺城,恪率所領入臺,隨例加右軍將軍。賊起東西二土山以逼城,城內亦作土山以應之,恪為東土山主,晝夜拒戰。以功封東興縣侯,邑五百戶。遷員外散騎常侍。京城陷,恪間行歸鄉里。高祖之討侯景,遣使報恪,乃於東起兵相應。



 賊平,恪謁高祖於京口,即日授都軍副。尋為府司馬。



 及高祖謀討王僧辯,恪預其謀。時僧辯女婿杜龕鎮吳興,高祖乃使世祖還長城,立柵備龕,又使恪還武康,招集兵眾。及僧辯誅,龕果遣副將杜泰
 率眾襲世祖於長城。恪時已率兵士出縣誅龕黨與,高祖尋遣周文育來援長城,文育至,泰乃遁走。



 世祖仍與文育進軍出郡,恪軍亦至,屯于郡南。及龕平,世祖襲東揚州刺史張彪,以恪監吳興郡。太平元年,除宣猛將軍、交州刺史。其年遷永嘉太守。不拜,復令監吳興郡。自吳興入朝。高祖受禪,使中書舍人劉師知引恪,令勒兵入,因衛敬帝如別宮。恪乃排闥入見高祖,叩頭謝曰:「恪身經事蕭家來,今日不忍見許事,分受死耳,決不奉命。」高
 祖嘉其意,乃不復逼,更以盪主王僧志代之。



 高祖踐祚,除吳興太守。永定二年,徙監會稽郡。會餘孝頃謀應王琳,出兵臨川攻周迪,以恪為壯武將軍,率兵踰嶺以救迪。餘孝頃聞恪至,退走。三年,遷使持節、通直散騎常侍、智武將軍、吳州刺史,便道之鄱陽。尋有詔追還,行會稽郡事。其年,除散騎常侍、忠武將軍、會稽太守。



 世祖嗣位,進督會稽、東陽、新安、臨海、永嘉、建安、晉安、新寧、信安九郡諸軍事,將軍、太守如故。天嘉元年,增邑五百戶。二年,
 徵為左衛將軍。俄出為都督郢、武、巴定四州諸軍事、軍師將軍、郢州刺史。六年,徵為中護軍。尋遷護軍將軍。光大二年,遷使持節、都督荊武右三州諸軍事、平西將軍、荊州刺史。



 未之鎮,改為護軍將軍。



 高宗即位,加散騎常侍、都督廣、衡、東衡、交、越、成、定、新、合、羅、愛、德、宜、黃、利、安、石、雙等十八州諸軍事、鎮南將軍、平越中郎將、廣州刺史。恪未至嶺,前刺史歐陽紇舉兵拒險,恪不得進,朝廷遣司空章昭達督眾軍討紇,紇平,乃得入州。州罹兵荒,所
 在殘毀,恪綏懷安緝,被以恩惠,嶺表賴之。



 太建四年,徵為領軍將軍。及代還,以途還不時至,為有司所奏免。十一年,起為散騎常侍、衛尉卿。其年授平北將軍、假節,監南兗州。十二年,改授散騎常侍、翊右將軍,監南徐州。又遣電威將軍裴子烈領馬五百匹,助恪緣江防戍。明年,入為衛尉卿,常侍、將軍如故。尋加侍中,遷護軍將軍。後主即位,以疾改授散騎常侍、特進、金紫光祿大夫。其年卒,時年七十四。贈翊左將軍,詔給東園秘器,仍出舉哀,
 喪事所須,並令資給,謚曰元。子法興嗣。



 史臣曰:胡穎、徐度、杜棱、沈恪並附騏驥而騰躍,依日月之光輝,始覯王佐之才,方悟公輔之量,生則肉食,終以配饗。盛矣哉!



\end{pinyinscope}