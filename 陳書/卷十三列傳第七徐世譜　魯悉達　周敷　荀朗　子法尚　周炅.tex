\article{卷十三列傳第七徐世譜 魯悉達 周敷 荀朗 子法尚 周炅}

\begin{pinyinscope}

 徐世譜,字興宗,巴東魚復人也。世居荊州,為主帥,征伐蠻、蜒。至世譜,尤敢勇有膂力,善水戰。梁元帝之為荊
 州刺史,世譜將領鄉人事焉。



 侯景之亂,因預征討,累遷至員外散騎常侍。尋領水軍,從司徒陸法和討景,與景戰於赤亭湖。時景軍甚盛,世譜乃別造樓船、拍艦、火舫、水車以益軍勢。將戰,又乘大艦居前,大敗景軍,生擒景將任約,景退走。因隨王僧辯攻郢州,世譜復乘大艦臨其倉門,賊將宋子仙據城降。以功除使持節、信武將軍、信州刺史,封魚復縣侯,邑五百戶。仍隨僧辯東下,恆為軍鋒。又破景將侯子鑒於湖熟。侯景平後,以功除通直散
 騎常侍、衡州刺史資,領河東太守,增邑并前一千戶。



 西魏來寇荊州,世譜鎮馬頭岸,據有龍洲,元帝授侍中、使持節、都督江南諸軍事、鎮南將軍、護軍將軍,給鼓吹一部。江陵陷沒,世譜東下依侯瑱。紹泰元年,徵為侍中、左衛將軍。高祖之拒王琳,其水戰之具,悉委世譜。世譜性機巧,諳解舊法,所造器械,並隨機損益,妙思出入。



 永定二年,遷護軍將軍。世祖嗣位,加特進,進號安右將軍。天嘉元年,增邑五百戶。二年,出為使持節、散騎常侍、都督
 宣城郡諸軍事、安西將軍、宣城太守,秩中二千石。還為安前將軍、右光祿大夫。尋以疾失明,謝病不朝。四年卒,時年五十五。贈本官,謚曰桓侯。



 世譜從弟世休,隨世譜自梁征討,亦有戰功。官至員外散騎常侍、安遠將軍,枳縣侯,邑八百戶。光大二年,隸都督淳于量徵華皎。卒,贈通直散騎常侍,謚曰壯。



 魯悉達,字志通,扶風郿人也。祖斐,齊通直散騎常侍、安遠將軍、衡州刺史,陽塘侯。父益之,梁雲麾將軍、新蔡、義
 陽二郡太守。悉達幼以孝聞,起家為梁南平嗣王中兵參軍。侯景之亂,悉達糾合鄉人,保新蔡,力田蓄穀。時兵荒饑饉,京都及上川餓死者十八九,有得存者,皆攜老幼以歸焉。悉達分給糧廩,其所濟活者甚眾,仍於新蔡置頓以居之。招集晉熙等五郡,盡有其地。使其弟廣達領兵隨王僧辯討侯景。景平,梁元帝授持節、仁威將軍、散騎常侍、北江州刺史。



 敬帝即位,王琳據有上流,留異、餘孝頃、周迪等所在蜂起,悉達撫綏五郡,甚得民和,士
 卒皆樂為之用。琳授悉達鎮北將軍,高祖亦遣趙知禮授征西將軍、江州刺史,各送鼓吹女樂,悉達兩受之,遷延顧望,皆不就。高祖遣安西將軍沈泰潛師襲之,不能克。齊遣行臺慕容紹宗以眾三萬來攻鬱口諸鎮,兵甲甚盛,悉達與戰,敗齊軍,紹宗僅以身免。



 王琳欲圖東下,以悉達制其中流,恐為己患,頻遣使招誘,悉達終不從。琳不得下,乃連結於齊,共為表裏,齊遣清河王高岳助之。相持歲餘,會裨將梅天養等懼罪,乃引齊軍入城。悉
 達勒麾下數千人,濟江而歸高祖。高祖見之,甚喜,曰:「來何遲也。」悉達對曰:「臣鎮撫上流,願為蕃屏,陛下授臣以官,恩至厚矣,沈泰襲臣,威亦深矣,然臣所以自歸於陛下者,誠以陛下豁達大度,同符漢祖故也。」



 高祖嘆曰:「卿言得之矣。」授平南將軍、散騎常侍、北江州刺史,封彭澤縣侯。



 世祖即位,進號安左將軍。



 悉達雖仗氣任俠,不以富貴驕人,雅好詞賦,招禮才賢,與之賞會。遷安南將軍、吳州刺史。遭母憂,哀毀過禮,因遘疾卒,時年三十八。贈
 安左將軍、江州刺史,謚曰孝侯。子覽嗣。弟廣達,別有傳。



 周敷,字仲遠,臨川人也。為郡豪族。敷形貌眇小,如不勝衣,而膽力勁果,超出時輩。性豪俠,輕財重士,鄉黨少年任氣者咸歸之。



 侯景之亂,鄉人周續合徒眾以討賊為名,梁內史始興王毅以郡讓續,續所部內有欲侵掠於毅,敷擁護之,親率其黨捍衛,送至豫章。時觀寧侯蕭永、長樂侯蕭基、豊城侯蕭泰避難流寓,聞敷信義,皆往依之。敷愍其危懼,屈體崇敬,厚加給恤,送之西上。俄
 而續部下將帥爭權,復反,殺續以降周迪。迪素無簿閥,恐失眾心,倚敷族望,深求交結。敷未能自固,事迪甚恭,迪大憑仗之,漸有兵眾。迪據臨川之工塘,敷鎮臨川故郡。侯景平,梁元帝授敷使持節、通直散騎常侍、信武將軍、寧州刺史,封西豊縣侯,邑一千戶。



 高祖受禪,王琳據有上流,餘孝頃與琳黨李孝欽等共圖周迪,敷大致人馬以助於迪。迪擒孝頃等,敷功居多。熊曇朗之殺周文育,據豫章,將兵萬餘人襲敷,徑至城下,敷與戰,大敗之,
 追奔五十餘里,曇朗單馬獲免,盡收其軍實。曇朗走巴山郡,收合餘黨,敷因與周迪、黃法鷿等進兵圍曇朗,屠之。王琳平,授散騎常侍、平西將軍、豫章太守。是時南江酋帥並顧戀巢窟,私署令長,不受召,朝廷未遑致討,但羈縻之,唯敷獨先入朝。天嘉二年,詣闕,進號安西將軍,給鼓吹一部,賜以女樂一部,令還鎮豫章。



 周迪以敷素出己下,超致顯貴,深不平,乃舉兵反,遣弟方興以兵襲敷。敷與戰,大破方興。仍率眾從都督吳明徹攻迪,破之,
 擒其弟方興并諸渠帥。詔以敷為安西將軍、臨川太守,餘並如故。尋徵為使持節、都督南豫、北江二州諸軍事、鎮南將軍、南豫州刺史,增邑五百戶,常侍、鼓吹如故。五年,迪又收合餘眾,還襲東興。世祖遣都督章昭達征迪,敷又從軍。至定川縣,與迪相對。迪紿敷曰:「吾昔與弟戮力同心,宗從匪他,豈規相害。今願伏罪還朝,因弟披露心腑,先乞挺身共立盟誓。」敷許之,方登壇,為迪所害,時年三十五。詔曰:「使持節、散騎常侍、都督南豫州緣江諸
 軍事、鎮南將軍、南豫州刺史西豊縣開國侯敷,受任遐征,淹時違律,虛衿姦詭,遂貽喪僕。但夙著勤誠,亟勞戎旅,猶深惻愴,愍悼于懷。



 可存其茅賦,量所賻恤,還葬京邑。」謚曰脫。子智安嗣。



 敷兄彖,共敷據本鄉,亦授臨川太守。



 荀朗,字深明,潁川潁陰人也。祖延祖,梁潁川太守,父伯道,衛尉卿。朗少慷慨,有將帥大略,起家梁廬陵王行參軍。侯景之亂,朗招率徒旅,據巢湖間,無所屬。臺城陷後,
 簡文帝密詔授朗雲麾將軍、豫州刺史,令與外籓討景。景使儀同宋子仙、任約等頻往征之,朗據山立砦自守,子仙不能克。時京師大饑,百姓皆於江外就食,朗更招致部曲,解衣推食,以相賑贍,眾至數萬人。侯景敗於巴陵,朗出自濡須截景,破其後軍。王僧辯東討,朗遣其將范寶勝及弟曉領兵二千助之。侯景平後,又別破齊將郭元建於踟躕山。梁承聖二年,率部曲萬餘家濟江,入宣城郡界立頓。梁元帝授朗持節、通直散騎常侍、安南
 將軍、都督南兗州諸軍事、南兗州刺史。未行而荊州陷。



 高祖入輔,齊遣蕭軌、東方老等來寇,據石頭城。朗自宣城來赴,因與侯安都等大破齊軍。永定元年,賜爵興寧縣侯,邑二千戶,以朗兄昂為左衛將軍,弟晷為太子右衛率。尋遣朗隨世祖拒王琳於南皖。



 高祖崩,宣太后與舍人蔡景歷秘不發喪,朗弟曉在都微知之,乃謀率其家兵襲臺。事覺,景歷殺曉,仍繫其兄弟。世祖即位,並釋之。因厚撫慰朗,令與侯安都等共拒王琳。琳平,遷使持
 節、安北將軍、散騎常侍、都督霍、晉、合三州諸軍事、合州刺史。天嘉六年卒,時年四十八。贈南豫州刺史,謚曰壯。子法尚嗣。



 法尚少倜儻,有文武幹略,起家江寧令,襲爵興寧縣侯。太建五年,隨吳明徹北伐。尋授通直散騎侍郎,除涇令,歷梁、安城太守。禎明中,為都督郢、巴、武三州諸軍事、郢州刺史。及隋軍濟江,法尚降于漢東道元帥秦王。入隋,歷邵、觀、綿、豊四州刺史,巴東、燉煌二郡太守。



 周炅,字文昭,汝南安城人也。祖彊,齊太子舍人、梁州刺史。父靈起,梁通直散騎常侍、廬、桂二州刺史,保城縣侯。炅少豪俠任氣,有將帥才。梁大同中為通直散騎侍郎、朱衣直閣。太清元年,出為弋陽太守。侯景之亂,元帝承制改授西陽太守,封西陵縣伯。景遣兄子思穆據守齊安,炅率驍勇襲破思穆,擒斬之。以功授持節、高州刺史。是時炅據武昌、西陽二郡,招聚卒徒,甲兵甚盛。景將任約來據樊山,炅與寧州長史徐文盛擊約,
 斬其部將叱羅子通、趙迦婁等。因乘勝追之,頻克,約眾殆盡。承聖元年,遷使持節、都督江、定二州諸軍事、戎昭將軍、江州刺史,進爵為侯,邑五百戶。



 高祖踐祚,王琳擁據上流,炅以州從之。及王琳遣其將曹慶等攻周迪,仍使炅將兵掎角而進,為侯安都所敗,擒炅送都。世祖釋炅,授戎威將軍、定州刺史,帶西陽、武昌二郡太守。



 天嘉二年,留異據東陽反,世祖召炅還都,欲令討異。未至而異平,炅還本鎮。



 天康元年,預平華皎之功,授員外散騎常侍。太建元
 年,遷持節、龍驤將軍、通直散騎常侍。五年,進授使持節、西道都督安、蘄、江、衡、司、定六州諸軍事、安州刺史,改封龍源縣侯,增邑并前一千戶。其年隨都督吳明徹北討,所向克捷,一月之中,獲十二城。齊遣尚書左丞陸騫以眾二萬出自巴、蘄,與炅相遇。炅留羸弱輜重,設疑兵以當之,身率精銳,由間道邀其後,大敗騫軍,虜獲器械馬驢,不可勝數。進攻巴州,克之。於是江北諸城及穀陽士民,並誅渠帥以城降。進號和戎將軍、散騎常侍,增邑并
 前一千五百戶。仍敕追炅入朝。



 初,蕭詧定州刺史田龍升以城降,詔以為振遠將軍、定州刺史,封赤亭王。及炅入朝,龍升以江北六州七鎮叛入于齊,齊遣歷陽王高景安帥師應之。於是令炅為江北道大都督,總統眾軍,以討龍升。龍升使弋陽太守田龍琰率眾二萬陣於亭川,高景安於水陵、陰山為其聲援,龍升引軍別營山谷。炅乃分兵各當其軍,身率驍勇先擊龍升,龍升大敗,龍琰望塵而奔,並追斬之,高景安遁走,盡復江北之地。以
 功增邑并前二千戶,進號平北將軍,定州刺史,持節、都督如故,仍賜女妓一部。



 太建八年卒官,時年六十四。贈司州刺史,封武昌郡公,謚曰壯。子法僧嗣,官至宣城太守。



 史臣曰:彼數子者,或驅馳前代,或擁據故鄉,並識運知歸,因機景附,位升列牧,爵致通侯,美矣。昔張耳、陳餘自同於至戚,周敷、周迪亦誓等暱親,尋鋒刃而誅殘,斯甚夫胡越矣。讎隙因於勢利,何其鄙歟!



\end{pinyinscope}