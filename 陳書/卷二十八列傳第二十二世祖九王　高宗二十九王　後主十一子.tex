\article{卷二十八列傳第二十二世祖九王 高宗二十九王 後主十一子}

\begin{pinyinscope}

 世祖十三男:沈皇后生廢帝、始興王伯茂,嚴淑媛生鄱陽王伯山、晉安王伯恭,潘容華生新安王伯固,劉昭華生衡陽王伯信,王充華生廬陵王伯仁,張脩容生江夏王伯義,韓脩華生武陵王伯禮,江貴妃生永陽王伯智,孔貴妃生桂陽王伯謀。其伯固犯逆別有傳。二男早卒,本書無名。



 始興王伯茂,字鬱之,世祖
 第二子也。初,高祖兄始興昭烈王道談仕於梁世,為東宮直閣將軍,侯景之亂,領弩手二千援臺,於城中中流矢卒。太平二年,追贈侍中、使持節、都督南兗州諸軍事、南兗州刺史,封長城縣公,謚曰昭烈。高祖受禪,重贈驃騎大將軍、太傅、揚州牧,改封始興郡王,邑
 二千戶。王生世祖及高宗。



 高宗以梁承聖末遷于關右,至是高祖遙以高宗襲封始興嗣王,以奉昭烈王祀。永定三年六月,高祖崩,是月世祖入纂帝位。時高宗在周未還,世祖以本宗乏饗,其年十月下詔曰:「日者皇基肇建,封樹枝戚,朕親地攸在,特啟大邦。弟頊嗣承
 門祀,雖土宇開建,薦饗莫由。重以遭家不造,閔凶夙遘,儲貳遐隔,轊車未返。猥以眇身,膺茲景命,式循龜鼎,冰谷載懷。今既入奉大宗,事絕籓裸,始興國廟蒸嘗無主,瞻言霜露,感尋慟絕。其徙封嗣王頊為安成王,封第二子伯茂為始興王,以奉昭烈王祀。賜天下為父後者爵一級。庶申罔極之情,永保山河之祚。」



 舊制諸王受封,未加戎號者,不置佐史,於是尚書八座奏曰:「夫增崇徽號,飾表車服,所以闡彰厥德,下變民望。第二皇子新除始興王伯茂,體自尊極,神姿明穎,玉暎觿辰,蘭芬綺歲,清暉美譽,日茂月升,道鬱平、河,聲超袞、植。皇情追感,聖性天深,以本宗闕緒,纂承籓嗣,雖珪社是膺,而戎章未襲,豈所以光崇睿哲,寵樹皇枝。臣等參議,宜加寧遠將軍,置佐史。」詔曰「可」。尋除使持節、都督南瑯邪彭城二郡諸軍事、彭城太守。天嘉二年,進號宣惠將
 軍、揚州刺史。



 伯茂性聰敏,好學,謙恭下士,又以太子母弟,世祖深愛重之。是時征北軍人於丹徒盜發晉郗曇墓,大獲晉右將軍王羲之書及諸名賢遺跡。事覺,其書並沒縣官,藏于秘府,世祖以伯茂好古,多以賜之,由是伯茂大工
 草隸,甚得右軍之法。三年,除鎮東將軍、開府儀同三司、東揚州刺史。



 廢帝即位,時伯茂在都,劉師知等矯詔出高宗也,伯茂勸成之。師知等誅後,高宗恐伯茂扇動朝廷,光大元年,乃進號中衛將軍,令入居禁中,專與廢帝遊處。



 是時四海之望,咸歸高宗,伯茂深不平,日夕憤怨,數肆惡言,高宗以其無能,不以為意。及建安人蔣裕與韓子高等謀反,伯茂並陰豫其事。二年十一月,皇太后令黜廢帝為臨海王,其日又下令曰:「伯茂輕薄,爰自弱
 齡,辜負嚴訓,彌肆凶狡。



 常以次居介弟,宜秉國權,不涯年德,逾逞狂躁,圖為禍亂,扇動宮闈,要招粗險,觖望臺閣,嗣君喪道,由此亂階,是諸凶德,咸作謀主。允宜罄彼司甸,刑斯蠙人。



 言念皇支,尚懷悲懣,可特降為溫麻侯,宜加禁止,別遣就第。不意如此,言增泫歎。」時六門之外有別館,以為諸王冠婚之所,名為婚第,至是命伯茂出居之。於路遇盜,殞于車中,時年十八。



 鄱陽王伯山,字靜之,世祖第三子也。偉容儀,舉止閑雅,
 喜慍不形於色,世祖深器之。初高祖時,天下草創,諸王受封儀注多闕,及伯山受封,世祖欲重其事,天嘉元年七月丙辰,尚書八座奏曰:「臣聞本枝惟允,宗周之業以弘,盤石既建,皇漢之基斯遠,故能協宣五運,規範百王,式固靈根,克隆卜世。第三皇子伯山,發睿德于齠年,表歧姿於丱日,光昭丹掖,暉暎青闈,而玉珪未秉,金錫靡駕,豈所以敦序維翰,建樹籓戚。臣等參議,宜封鄱陽郡王。」詔曰「可」。乃遣散騎常侍、度支尚書蕭睿持節兼太宰
 告于太廟;又遣五兵尚書王質持節兼太宰告于太社。



 其年十月,上臨軒策命之曰:「於戲!夫建樹籓屏,翼獎王室,欽若前典,咸必由之。惟爾夙挺珪璋,生知孝敬,令德茂親,僉譽所集,啟建大邦,實惟倫序,是用敬遵民瞻,錫此圭瑞。往欽哉!其勉樹聲業,永保宗社,可不慎歟!」策訖,敕令王公已下並宴於王第。仍授東中郎將、吳郡太守。六年,為緣江都督、平北將軍、南徐州刺史。天康元年,進號鎮北將軍。



 高宗輔政,不欲令伯山處邊,光大元年,徙
 為鎮東將軍、東揚州刺史。太建元年,徵為中衛將軍、中領軍。六年,又為征北將軍、南徐州刺史。尋為征南將軍、江州刺史。十一年,入為護軍將軍,加開府儀同三司,仍給鼓吹并扶。后主即位,進號中權大將軍。至德四年,出為持節、都督東揚、豊二州諸軍事、東揚州刺史,加侍中,餘並如故。禎明元年,丁所生母憂,去職。明年,起為鎮衛大將軍、開府儀同三司,給班劍十人。三年正月薨,時年四十。



 伯山性寬厚,美風儀,又於諸王最長,後主深敬重
 之,每朝廷有冠婚饗宴之事,恒使伯山為主。及丁所生母憂,居喪以孝聞。後主嘗幸吏部尚書蔡徵宅,因往弔之,伯山號慟殆絕,因起為鎮衛將軍,仍謂群臣曰:「鄱陽王至性可嘉,又是西第之長,豫章已兼司空,其亦須遷太尉。」未及發詔而伯山薨,尋值陳亡,遂無贈謚。



 長子君範,太建中拜鄱陽國世子,尋為貞威將軍、晉陵太守,未襲爵而隋師至。



 是時宗室王侯在都者百餘人,後主恐其為變,乃並召入,令屯朝堂,使豫章王叔英總督之,而
 又陰為之備。及六軍敗績,相率出降,因從後主入關。至長安,隋文帝並配於隴右及河西諸州,各給田業以處之。初,君範與尚書僕射江總友善,至是總贈君範書五言詩,以敘他鄉離別之意,辭甚酸切,當世文士咸諷誦之。大業二年,隋煬帝以後主第六女女婤為貴人,絕愛幸,因召陳氏子弟盡還京師,隨才敘用,由是並為守宰,遍於天下。其年君範為溫令。



 晉安王伯恭字肅之,世祖第六子也。天嘉六年,立為晉
 安王。尋為平東將軍、吳郡太守,置佐史。時伯恭年十餘歲,便留心政事,官曹治理。太建元年,入為安前將軍、中護軍,遷中領軍。尋為中衛將軍、揚州刺史,以公事免。四年,起為安左將軍,尋為鎮右將軍、特進,給扶。六年,出為安南將軍、南豫州刺史。九年,入為安前將軍、祠部尚書。十一年,進號軍師將軍、尚書右僕射。十二年,遷僕射。



 十三年,遷左僕射。十四年,出為安南將軍、湘州剌史,未拜。至德元年,為侍中、中衛將軍、光祿大夫,丁所生母憂,去
 職。禎明元年,起為中衛將軍、右光祿大夫,置佐史、扶並如故。三年入關。隋大業初,為成州刺史、太常卿。



 衡陽王伯信,字孚之,世祖第七子也。天嘉元年,衡陽獻王昌自周還朝,於道薨,其年世祖立伯信為衡陽王,奉獻王祀。尋為宣惠將軍、丹陽尹,置佐史。太建四年,為中護軍。六年,為宣毅將軍、揚州刺史。尋加侍中、散騎常侍。十一年,進號鎮前將軍,太子詹事,餘並如故。禎明元年,出為鎮南將軍、西衡州刺史。三年,隋軍濟江,與臨汝侯
 方慶並為東衡州刺史王勇所害,事在方慶傳。



 廬陵王伯仁,字壽之,世祖第八子也。天嘉六年,立為廬陵王。太建初,為輕車將軍,置佐史。七年,遷冠軍將軍、中領軍。尋為平北將軍、南徐州刺史。十二年,為翊左將軍、中領軍。貞明元年,加侍中、國子祭酒,領太子中庶子。三年入關,卒于長安。



 長子番,先封湘濱侯,隋大業中,不資陽令。



 江夏王伯義,字堅之,世祖第九子也。天嘉六年,立為江
 夏王。太建初,為宣惠將軍、東揚州刺史,置佐史。尋為宣毅將軍、持節、散騎常侍、都督合、霍二州諸軍事、合州刺史。十四年,徵為侍中、忠武將軍、金紫光祿大夫。禎明三年入關,遷于瓜州,於道卒。



 長子元基,先封湘潭侯,隋大業中為穀熟縣令。



 武陵王伯禮,字用之,世祖第十子也。天嘉六年,立為武陵王。太建初,為雲旗將軍、持節、都督吳興諸軍事、吳興太守。在郡恣行暴掠,驅錄民下,逼奪財貨,前後委積,百
 姓患之。太建九年,為有司所劾,上曰:「王年少,未達治道,皆由佐史不能匡弼所致,特降軍號,後若更犯,必致之以法,有司不言與同罪。」十一年春,被代征還,伯禮遂遷延不發。其年十月,散騎常侍、御史中丞徐君敷奏曰:「臣聞車屨不俟,君命之通規,夙夜匪懈,臣子之恒節。謹案雲旗將軍、持節、都督吳興諸軍事、吳興太守武陵王伯禮,早擅英猷,久馳令問,惟良寄重,枌鄉是屬。



 聖上愛育黔黎,留情政本,共化求瘼,早赴皇心,遂復稽緩歸驂,取
 移涼燠,遲回去鷁,空淹載路,淑慎未彰,違惰斯在,繩愆檢迹,以為懲誡。臣等參議以見事免伯禮所居官,以王還第,謹以白簡奏聞。」詔曰:「可」。禎明三年入關,隋大業中為散騎侍郎、臨洮太守。



 永陽王伯智,字策之,世祖第十二子也。少敦厚,有器局,博涉經史。太建中,立為永陽王。尋為侍中,加明威將軍,置佐史。尋加散騎常侍,累遷尚書左僕射,出為使持節、都督東揚、豊二州諸軍事、平東將軍,領會稽內史。至德
 二年,入為侍中、翊左將軍,加特進。禎明三年入關。隋大業中,為岐州司馬,遷國子司業。



 桂陽王伯謀,字深之,世祖第十三子也。太建中,立為桂陽王。七年,為明威將軍,置佐史。尋為信威將軍、丹陽尹。十年,加侍中。出為持節、都督吳興諸軍事、東中郎將、吳興太守。十一年,加散騎常侍。至德元年薨。



 子豊嗣,大業中,為番禾令。



 高宗四十二男:柳皇后生後主,彭貴人生始興王叔陵,
 曹淑華生豫章王叔英,何淑儀生長沙王叔堅、宜都王叔明,魏昭容生建安王叔卿,錢貴妃生河東王叔獻,劉昭儀生新蔡王叔齊,袁昭容生晉熙王叔文、義陽王叔達、新會王叔坦,王姬生淮南王叔彪、巴山王叔雄,吳姬生始興王叔重,徐姬生尋陽王叔儼,淳于姬生岳陽王叔慎,王修華生武昌王叔虞,韋修容生湘東王叔平,施姬生臨賀王叔敖、沅陵王叔興,曾姬生陽山王叔宣,楊姬生西陽王叔穆,申婕妤生南安王叔儉、南郡王叔澄、
 岳山王叔韶、太原王叔匡,袁姬生新興王叔純,吳姬生巴東王叔謨,劉姬生臨江王叔顯,秦姬生新寧王叔隆、新昌王叔榮。其皇子叔叡、叔忠、叔弘、叔毅、叔訓、叔武、叔處、叔封等八人,並未及封。叔陵犯逆,別有傳。三子早卒,本書無名。



 豫章王叔英,字子烈,高宗第三子也。少寬厚仁愛。天嘉元年,封建安侯。太建元年,改封豫章王,仍為宣惠將軍、都督東揚州諸軍事、東揚州刺史。五年,進號平北將軍、
 南豫州刺史。十一年,為鎮前將軍、江州刺史。後主即位,進號征南將軍,尋加開府儀同三司、中衛大將軍,餘並如故。四年,進號驃騎大將軍。禎明元年,給鼓吹一部,班劍十人。其年,遷司空。三年,隋師濟江,叔英知石頭軍戍事。尋令入屯朝堂。及六軍敗績,降于隋將韓擒虎。其年入關。隋大業中為涪陵太守。



 長子弘,至德元年,拜豫章國世子。



 長沙王叔堅,字子成,高宗第四子也。母本吳中酒家隸,
 高宗微時,嘗往飲,遂與通,及貴,召拜淑儀。叔堅少傑黠,凶虐使酒,尤好數術、卜筮、祝禁,金容金琢玉,並究其妙。天嘉中,封豊城侯。太建元年,立為長沙王,仍為東中郎將、吳郡太守。四年,為宣毅將軍、江州刺史,置佐史。七年,進號雲麾將軍、郢州刺史,未拜,轉為平越中郎將、廣州刺史。尋為平北將軍、合州刺史。八年,復為平西將軍、郢州刺史。十一年,入為翊左將軍、丹陽尹。



 初,叔堅與始興王叔陵並招聚賓客,各爭權寵,甚不平。每朝會鹵簿,不肯
 為先後,必分道而趨,左右或爭道而鬥,至有死者。及高宗弗豫,叔堅、叔陵等並從後主侍疾。叔陵陰有異志,乃命典藥吏曰:「切藥刀甚鈍,可礪之。」及高宗崩,倉卒之際,又命其左右於外取劍,左右弗悟,乃取朝服所佩木劍以進,叔陵怒。叔堅在側聞之,疑有變,伺其所為。及翌日小斂,叔陵袖銼藥刀趨進,斫後主,中項,後主悶絕于地,皇太后與後主乳母樂安君吳氏俱以身捍之,獲免。叔堅自後扼叔陵,擒之,并奪其刀,將殺之,問後主曰:「即盡
 之,為待也?」後主不能應。叔陵舊多力,須臾,自奮得脫,出雲龍門,入于東府城,召左右斷青溪橋道,放東城囚以充戰士。又遣人往新林,追其所部兵馬,仍自被甲,著白布帽,登城西門,招募百姓。是時眾軍並緣江防守,臺內空虛,叔堅乃白太后使太子舍人司馬申以後主命召蕭摩訶,令討之。即日擒其將戴溫、譚騏驎等,送臺,斬于尚書閣下,持其首徇于東城。叔陵恇擾不知所為,乃盡殺其妻妾,率左右數百人走趨新林。摩訶追之,斬于丹
 陽郡,餘黨悉擒。其年,以功進號驃騎將軍、開府儀同三司、揚州刺史。尋遷司空,將軍、刺史如故。



 是時後主患創,不能視事,政無小大,悉委叔堅決之,於是勢傾朝廷。叔堅因肆驕縱,事多不法,後主由是疏而忌之。孔範、管斌、施文慶之徒,並東宮舊臣,日夜陰持其短。至德元年,乃詔令即本號用三司之儀,出為江州刺史。未發,尋有詔又以為驃騎將軍,重為司空,實欲去其權勢。叔堅不自安,稍怨望,乃為左道厭魅以求福助,刻木為偶人,衣以
 道士之服,施機關,能拜跪,晝夜於日月下醮之,祝詛於上。其年冬,有人上書告其事,案驗並實,後主召叔堅囚于西省,將殺之。



 其夜,令近侍宣敕,數之以罪,叔堅對曰:「臣之本心,非有他故,但欲求親媚耳。



 臣既犯天憲,罪當萬死,臣死之日,必見叔陵,願宣明詔,責於九泉之下。」後主感其前功,乃赦之,特免所居官,以王還第。尋起為侍中、鎮左將軍。二年,又給鼓吹,油幢車。三年,出為征西將軍、荊州刺史。四年,進號中軍大將軍、開府儀同三司。禎
 明二年,秩滿還都。



 三年入關,遷于瓜州,更名叔賢。叔賢素貴,不知家人生產,至是與妃沈氏酤酒,以傭保為事。隋大業中,為遂寧郡太守。



 建安王叔卿,字子弼,高宗第五子也。性質直有材器,容貌甚偉。太建四年,立為建安王,授東中郎將、東揚州刺史。七年,為雲麾將軍、郢州刺史,置佐史。



 九年,進號平南將軍、湘州刺史。後主即位,進號安南將軍。又為侍中、鎮右將軍、中書令。遷中書監。禎明三年入關,隋大業中,為
 都官郎、上黨通守。



 宜都王叔明,字子昭,高宗第六子也。儀容美麗,舉止和弱,狀似婦人。太建五年,立為宜都王,尋授宣惠將軍,置佐史。七年,授東中郎將、東揚州刺史,尋為輕車將軍、衛尉卿。十三年,出為使持節、雲麾將軍、南徐州刺史。又為侍中、翊右將軍。至德四年,進號安右將軍。禎明三年入關,隋大業中為鴻臚少卿。



 河東王叔獻,字子恭,高宗第九子也。性恭謹,聰敏好學。
 太建五年,立為河東王。七年,授宣毅將軍,置佐史。尋為散騎常侍、軍師將軍、都督南徐州諸軍事、南徐州刺史。十二年薨,年十三。贈侍中、中撫將軍、司空,謚曰康簡。子孝寬嗣。



 孝寬以至德元年,襲爵河東王。禎明三年入關,隋大業中為汶城令。



 新蔡王叔齊,字子肅,高宗第十一子也。風彩明贍,博涉經史,善屬文。太建七年,立為新蔡王,尋為智武將軍,置佐史。出為東中郎將、東揚州刺史。至德二年,入為侍中,
 將軍、佐史如故。禎明元年,除國子祭酒,侍中、將軍、佐史如故。



 三年入關。隋大業中為尚書主客郎。



 晉熙王叔文,字子才,高宗第十二子也。性輕險,好虛譽,頗涉書史。太建七年,立為晉熙王。尋為侍中、散騎常侍、宣惠將軍,置佐史。進號輕車將軍、揚州刺史。至德元年,授持節、都督江州諸軍事、江州刺史。二年,遷信威將軍、督湘、衡、武、桂四州諸軍事、湘州刺史。禎明二年,秩滿,徵為侍中、宣毅將軍,佐史如故。未還,而隋軍濟江,破臺城,
 隋漢東道行軍元帥秦王至于漢口。時叔文自湘州還朝,至巴州,乃率巴州刺史畢寶等請降,致書於秦王曰:「竊以天無二日,晦明之序不差,土無二王,尊卑之位乃別。今車書混壹,文軌大同,敢披丹款,申其屈膝。」秦王得書,因遣行軍吏部柳莊與元帥府僚屬等往巴州迎勞叔文。叔文於是與畢寶、荊州刺史陳紀及文武將吏赴于漢口,秦王並厚待之,置于賓館。隋開皇九年三月,眾軍凱旋,文帝親幸溫湯勞之,叔文與陳紀、周羅睺、荀法
 尚等并諸降人,見于路次。數日,叔文從後主及諸王侯將相并乘輿、服御、天文圖籍等,並以次行列,仍以鐵騎圍之,隨晉王、秦王等獻凱而入,列于廟庭。明日,隋文帝坐于廣陽門觀,叔文又從後主至朝堂南。文帝使內史令李德林宣旨,責其君臣不能相弼,以致喪亡。後主與其群臣並慚懼拜伏,莫能仰視,叔文獨欣然而有自得之志。旬有六日,乃上表曰:「昔在巴州,已先送款,乞知此情,望異常例。」文帝雖嫌其不忠,而方欲懷柔江表,乃授
 開府,拜宜州刺史。



 淮南王叔彪,字子華,高宗第十三子也。少聰惠,善屬文。太建八年,立為淮南王。尋位侍中、仁威將軍,置佐史。禎明三年入關,卒于長安。



 始興王叔重,字子厚,高宗第十四子也。性質朴,無伎藝。高宗崩,始興王叔陵為逆。誅死,其年立叔重為始興王,以奉昭烈王後。至德元年,為仁威將軍、揚州刺史,置佐史。二年,加使持節、都督江州諸軍事、江州刺史。禎明三
 年入關。



 隋大業中為太府少卿,卒。



 尋陽王叔儼,字子思,高宗第十五子也。性凝重,舉止方正。後主即位,立為尋陽王。至德元年,為侍中、仁武將軍,置佐史。禎明三年入關,尋卒。



 岳陽王叔慎,字子敬,高宗第十六子也。少聰敏,十歲能屬文。太建十四年,立為岳陽王,時年十一。至德四年,拜侍中、智武將軍、丹陽尹。是時,後主尤愛文章,叔慎與衡陽王伯信、新蔡王叔齊等日夕陪侍,每應詔賦詩,恒被
 嗟賞。禎明元年,出為使持節、都督湘、衡、桂、武四州諸軍事、智武將軍、湘州刺史。三年,隋師濟江,破臺城,前刺史晉熙王叔文還至巴州,與巴州刺史畢寶、荊州刺史陳紀並降。隋行軍元帥清河公楊素兵下荊門,別遣其將龐暉將兵略地,南至湘州,城內將士,莫有固志,剋日請降。叔慎乃置酒會文武僚吏,酒酣,叔慎歎曰「君臣之義,盡於此乎!」長史謝基伏而流涕,湘州助防遂興侯正理在坐,乃起曰:「主辱臣死,諸君獨非陳國之臣乎?今天下
 有難,實是致命之秋也。縱其無成,猶見臣節,青門之外,有死不能。今日之機,不可猶豫,後應者斬。」眾咸許諾,乃刑牲結盟。仍遣人詐奉降書於龐暉,暉信之,克期而入,叔慎伏甲待之。暉令數百人屯于城門,自將左右數十人入于廳事,俄而伏兵發,縛暉以徇,盡擒其黨,皆斬之。叔慎坐于射堂,招合士眾,數日之中,兵至五千人。衡陽太守樊通、武州刺史鄔居業,皆請赴難。未至,隋遣中牟公薛胄為湘州刺史,聞龐暉死,乃益請兵,隋又遣行軍
 總管劉仁恩救之。未至,薛胄兵次鵝羊山,叔慎遣正理及樊通等拒之,因大合戰,自旦至于日昃,隋軍迭息迭戰,而正理兵少不敵,於是大敗。胄乘勝入城,生擒叔慎。



 是時,鄔居業率其眾自武州來赴,出橫橋江,聞叔慎敗績,乃頓于新康口。隋總管劉仁恩兵亦至橫橋,據水置營,相持信宿,因合戰,居業又敗。仁恩虜叔慎、正理、居業及其黨與十餘人,秦王斬之于漢口。叔慎時年十八。



 義陽王叔達,字子聰,高宗第十七子也。太建十四年,立
 為義陽王,尋拜仁武將軍,置佐史。禎明元年,除丹陽尹。三年入關。隋大業中為內史,至絳郡通守。



 巴山王叔雄,字子猛,高宗第十八子也。太建十四年,立為巴山王。禎明三年入關,卒于長安。



 武昌王叔虞,字子安,高宗第十九子也。太建十四年,立為武昌王,尋為壯武將軍,置佐史。禎明三年入關。隋大業中為高苑令。



 湘東王叔平,字子康,高宗第二十子也。至德元年,立為
 湘東王。禎明三年入關。隋大業中為胡蘇令。



 臨賀王叔敖,字子仁,高宗第二十一子也。至德元年,立為臨賀王,尋為仁武將軍,置佐史。禎明三年入關。隋大業初拜儀同三司。



 陽山王叔宣,字子通,高宗第二十二子也。至德元年,立為陽山王。禎明三年入關。隋大業中為涇城令。



 西陽王叔穆,字子和,高宗第二十三子也。至德元年,立為西陽王。禎明三年入關,卒于長安。



 南安王叔儉,字子約,高宗第二十四子也。至德元年,立為南安王。禎明三年入關,卒于長安。



 南郡王叔澄,字子泉,高宗第二十五子也。至德元年,立為南郡王。禎明三年入關。隋大業中為靈武令。



 沅陵王叔興,字子推,高宗第二十六子也。至德元年,立為沅陵王。禎明三年入關。隋大業中為給事郎。



 岳山王叔韶,字子欽,高宗第二十七子也。至德元年,立為岳山王,尋為智武將軍,置佐史。四年,除丹陽尹。禎明
 三年入關,卒于長安。



 新興王叔純,字子共,高宗第二十八子也。至德元年,立為新興王。禎明三年入關。隋大業中為河北令。



 巴東王叔謨,字子軌,高宗第二十九子也。至德四年,立為巴東王。禎明三年入關。隋大業中為岍陽令。



 臨江王叔顯,字子明,高宗第三十子也。至德四年,立為臨江王。禎明三年入關。隋大業中為鶉觚令。



 新會王叔坦,字子開,高宗第三十一子也。至德四年,立
 為新會王。禎明三年入關。隋大業中為涉令。



 新寧王叔隆,字子遠,高宗第三十二子也。至德四年,立為新寧王。禎明三年入關。卒于長安。



 新昌王叔榮,字子徹,高宗第三十三子也。禎明二年,立為新昌王。三年入關。



 隋大業中為內黃令。



 太原王叔匡,字子佐,高宗第三十四子也。禎明二年,立為太原王。三年入關。



 隋大業中為壽光令。



 後主二十二男:張貴妃生皇太子深、會稽王莊,孫姬生
 吳興王胤,高昭儀生南平王嶷,呂淑媛生永嘉王彥、邵陵王兢,龔貴嬪生南海王虔、錢塘王恬,張叔華生信義王祗,徐淑儀生東陽王恮,孔貴人生吳郡王蕃。其皇子總、觀、明、綱、統、沖、洽、縚、綽、威、辯十一人,並未及封。



 皇太子深,字承源,後主第四子也。少聰惠,有志操,容止儼然,雖左右近侍,未嘗見其喜慍。以母張貴妃故,特為後主所愛。至德元年,封始安王,邑二千戶。



 尋為軍師將軍、揚州刺史,置佐史。禎明二年,皇太子胤廢,後主乃立
 深為皇太子。



 三年,隋師濟江,六軍敗績,隋將韓擒虎自南掖門入,百僚逃散。深時年十餘歲,閉閣而坐,舍人孔伯魚侍焉。隋軍排閣而入,深使宣令勞之曰:「軍旅在途,不乃勞也?」軍人咸敬焉。其年入關。隋大業中為枹罕太守。



 吳興王胤,字承業,後主長子也。太建五年二月乙丑生于東宮,母孫姬因產卒,沈皇后哀而養之,以為己子。時後主年長,未有胤嗣,高宗因命以為嫡孫,其日下詔曰:「
 皇孫初誕,國祚方熙,思與群臣,共同斯慶,內外文武賜帛各有差,為父後者賜爵一級。」十年,封為永康公。後主即位,立為皇太子。胤性聰敏,好學,執經肄業,終日不倦,博通大義,兼善屬文。至德三年,躬出太學講《孝經》,講畢,又釋奠於先聖先師。其日設金石之樂於太學,王公卿士及太學生並預宴。是時張貴妃、孔貴嬪並愛幸,沈皇后無寵,而近侍左右數於東宮往來,太子亦數使人至后所,後主疑其怨望,甚惡之。而張、孔二貴妃又日夜構
 成后及太子之短,孔範之徒又於外合成其事,禎明二年,廢為吳興王,仍加侍中、中衛將軍。三年入關,卒于長安。



 南平王嶷,字承嶽,後主第二子也。方正有器局,年數歲,風采舉動,有若成人。至德元年,立為南平王。尋除信武將軍、南琅邪、彭城二郡太守,置佐史。遷揚州刺史,進號鎮南將軍。尋為使持節、都督郢、荊、湘三州諸軍事、征西將軍、郢州刺史。未行而隋軍濟江。禎明三年入關,卒于
 長安。



 永嘉王彥,字承懿,後主第三子也。至德元年,立為永嘉王。尋為忠武將軍、南徐州刺史,進號安南將軍。授散騎常侍、使持節、都督江、巴、東衡三州諸軍事、平南將軍、江州刺史。未行,隋師濟江。禎明三年入關。隋大業中為襄武令。



 南海王虔,字承恪,後主第五子也。至德元年,立為南海王。尋為武毅將軍,置佐史,進號軍師將軍。禎明二年,出
 為平北將軍、南徐州刺史。三年入關。隋大業中為涿令。



 信義王祗,字承敬,後主第六子也。至德元年,立為信義王。尋為壯武將軍,置佐史。授使持節、都督、智武將軍、琅邪、彭城二郡太守。禎明三年入關。隋大業中為通議郎。



 邵陵王兢,字承檢,後主第七子也。禎明元年,立為邵陵王,邑一千戶。尋為仁武將軍,置佐史。三年入關。隋大業中為國子監丞。



 會稽王莊,字承肅,後主第八子也。容貌蕞陋,性嚴酷,數
 歲,左右有不如意,輒剟刺其面,或加燒爇。以母張貴妃有寵,後主甚愛之。至德四年,立為會稽王。



 尋為翊前將軍,置佐史。除使持節、都督揚州諸軍事、揚州刺史。禎明三年入關。



 隋大業中為昌隆令。



 東陽王恮,字承厚,後主第九子也。禎明二年,立為東陽王,邑一千戶。未拜,三年入關。隋大業中為通議郎。



 吳郡王蕃,字承廣,後主第十子也。禎明二年,封吳郡王。三年入關。隋大業中為涪城令。



 錢塘王恬,字承惔,後主第十一子也。禎明二年,立為錢塘王,邑一千戶。三年入關,卒于長安。



 江左自西晉相承,諸王開國,並以戶數相差為大小三品。大國置上、中、下三將軍,又置司馬一人;次國置中、下二將軍;小國置將軍一人。餘官亦準此為差。



 高祖受命,自永定訖于禎明,唯衡陽王昌特加殊寵,至五千戶。自餘大國不過二千戶,小國即千戶。而舊史殘缺,不能別知其國戶數,故綴其遺事附于此。



 史臣曰:世祖、高宗、後主並建籓屏,以樹懿親,固乃本根,隆斯盤石。鄱陽王伯山有風采德器,亦一代令籓矣。岳陽王叔慎屬社稷傾危,情哀家國,竭誠赴敵,志不圖生。嗚呼!古之忠烈致命,斯之謂也。



\end{pinyinscope}