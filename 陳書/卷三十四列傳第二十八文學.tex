\article{卷三十四列傳第二十八文學}

\begin{pinyinscope}

 杜之偉顏晃江德藻庾持許亨褚玠岑之敬陸琰弟瑜
 何之元徐伯陽張正見蔡凝阮卓《易》曰「觀乎人文以化成天下」,孔子曰「煥乎其有文章」也。自楚、漢以降,辭人世出,洛汭、江左,其流彌暢。莫不思侔造化,明並日月,大則憲章典謨,裨贊王道,小則文理清正,申紓性靈。至於經禮樂,綜人倫,通古今,述美惡,莫尚乎此。後主嗣業,雅尚文詞,傍求學藝,煥乎俱集。每臣下表疏及獻上賦頌者,躬自省覽,其有辭工,則神筆賞
 激,加其爵位,是以搢紳之徒,咸知自勵矣。若名位文學晃著者,別以功迹論。今綴杜之偉等學既兼文,備于此篇云爾。



 杜之偉,字子大,吳郡錢塘人也。家世儒學,以《三禮》專門。父規,梁奉朝請,與光祿大夫濟陽江革、都官尚書會稽孔休源友善。



 之偉幼精敏,有逸才。七歲,受《尚書》,稍習《詩》、《禮》,略通其學。



 十五,遍觀文史及儀禮故事,時輩稱其早成。僕射徐勉嘗見其文,重其有筆力。中大通元年,梁武
 帝幸同泰寺捨身,敕勉撰定儀注,勉以臺閣先無此禮,召之偉草具其儀。乃啟補東宮學士,與學士劉陟等鈔撰群書,各為題目。所撰《富教》、《政道》二篇,皆之偉為序。及湘陰侯蕭昂為江州刺史,以之偉掌記室。昂卒,廬陵王續代之,又手教招引,之偉固辭不應命,乃送昂喪柩還京。仍侍臨城公讀。尋除揚州議曹從事、南康嗣王墨曹參軍,兼太學限內博士。大同七年,梁皇太子釋奠於國學,時樂府無孔子、顏子登歌詞,尚書參議令之偉製其文,伶
 人傳習,以為故事。



 轉補安前邵陵王田曹參軍,又轉刑獄參軍。之偉年位甚卑,特以彊識俊才,頗有名當世,吏部尚書張纘深知之,以為廊廟器也。



 侯景反,之偉逃竄山澤。及高祖為丞相,素聞其名,召補記室參軍。遷中書侍郎,領大著作。高祖受禪,除鴻臚卿,餘並如故。之偉啟求解著作,曰:「臣以紹泰元年,忝中書侍郎,掌國史,于今四載。臣本庸賤,謬蒙盼識,思報恩獎,不敢廢官。皇歷惟新,驅馭軒、昊,記言記事,未易其人,著作之材,更宜選眾。
 御史中丞沈炯、尚書左丞徐陵、梁前兼大著作虞荔、梁前黃門侍郎孔奐,或清文贍筆,或彊識稽古,遷、董之任,允屬群才,臣無容遽變市朝,再妨賢路。堯朝皆讓,誠不可追,陳力就列,庶幾知免。」優敕不許。尋轉大匠卿,遷太中大夫,仍敕撰梁史。永定三年卒,時年五十二。高祖甚悼惜之,詔贈通直散騎常侍,賻錢五萬,布五十匹,棺一具,克日舉哀。



 之偉為文,不尚浮華,而溫雅博贍。所製多遺失,存者十七卷。



 顏晃,字元明,琅邪臨沂人也。少孤貧,好學,有辭采。解褐梁邵陵王兼記室參軍。時東宮學士庾信嘗使於府中,王使晃接對,信輕其尚少,曰:「此府兼記室幾人?」晃答曰:「猶當少於宮中學士。」當時以為善對。



 侯景之亂,西奔荊州。承聖初,除中書侍郎。時杜龕為吳興太守,專好勇力,其所部多輕險少年,元帝患之,乃使晃管其書翰。仍敕龕曰:「卿年時尚少,習讀未晚,顏晃文學之士,使相毘佐,造次之間,必宜諮稟。」及龕誅,晃歸世祖,世祖委以書
 記,親遇甚篤。除宣毅府中錄事,兼記室參軍。



 永定二年,高祖幸大莊嚴寺,其夜甘露降,晃獻《甘露頌》,詞義該典,高祖甚奇之。天嘉初,遷員外散騎常侍,兼中書舍人,掌詔誥。三年卒,時年五十三。



 詔贈司農卿,謚曰貞子,并賜墓地。



 晃家世單門,傍無戚援,而介然脩立,為當世所知。其表奏詔誥,下筆立成,便得事理,而雅有氣質。有集二十卷。



 江德藻,字德藻,濟陽考城人也。祖柔之,齊尚書倉部郎
 中。父革,梁度支尚書、光祿大夫。德藻好學,善屬文。美風儀,身長七尺四寸。性至孝,事親盡禮。



 與異產昆弟居,恩惠甚篤。起家梁南中郎武陵王行參軍。大司馬南平王蕭偉聞其才,召為東閣祭酒。遷安西湘東王府外兵參軍,尋除尚書比部郎,以父憂去職。服闋之後,容貌毀瘠,如居喪時。除安西武陵王記室,不就。久之,授廬陵王記室參軍。



 除廷尉正,尋出為南兗州治中。及高祖為司空、征北將軍,引德藻為府諮議。轉中書侍郎,遷雲麾臨海
 王長史。陳臺建,拜尚書吏部侍郎。



 高祖受禪,授秘書監,兼尚書左丞。尋以本官兼中書舍人。天嘉四年,兼散騎常侍,與中書郎劉師知使齊,著《北征道理記》三卷。還拜太子中庶子,領步兵校尉。頃之遷御史中丞,坐公事免。尋拜振遠將軍、以通直散騎常侍。自求宰縣,出補新喻令,政尚恩惠,頗有異績。六年,卒於官,時年五十七。世祖甚悼惜之,詔贈散騎常侍。所著文筆十五卷。



 子椿,亦善屬文,歷太子庶子、尚書左丞。



 庾持,字允德,潁川鄢陵人也。祖佩玉,宋長沙內史。父沙彌,梁長城令。持少孤,性至孝,居父憂過禮。篤志好學,尤善書記,以才藝聞。解褐梁南平王國左常侍、輕車河東王府行參軍,兼尚書郎,尋而為真。出為安吉令,遷鎮東邵陵王府限外記室,兼建康令。天監初,世祖與持有舊,及世祖為吳興太守,以持為郡丞,兼掌書翰,自是常依文帝。文帝剋張彪,鎮會稽,又令持監臨海郡。以貪縱失民和,為山盜所劫,幽執十旬,世祖遣劉澄討平之,持乃獲
 免。高祖受禪,授安東臨川王府諮議參軍。天嘉初,遷尚書左丞。以預長城之功,封崇德縣子,邑三百戶。拜封之日,請令史為客,受其餉遺,世祖怒之,因坐免。尋為宣惠始興王府諮議參軍。



 除臨安令,坐杖殺縣民免封。遷為給事黃門侍郎。除棱威將軍、鹽官令。光大元年,遷祕書監,知國史事。又為少府卿,領羽林監。遷太中大夫,領步兵校尉。太建元年卒,時年六十二。詔贈光祿大夫。



 持善字書,每屬辭,好為奇字,文士亦以此譏之。有集十卷。



 許亨,字亨道,高陽新城人,晉徵士詢之六世孫也。曾祖珪,歷給事中,委桂陽太守,高尚其志,居永興之究山,即詢之所隱也。祖勇慧,齊太子家令、冗從僕射。父懋,梁始平天門二郡守、太子中庶子、散騎常侍,以學藝聞,撰《毛詩風雅比興義類》十五卷,《述行記》四卷。亨少傳家業,孤介有節行。博通群書,多識前代舊事,名輩皆推許之,甚為南陽劉之遴所重,每相稱述。解褐梁安東王行參軍,兼太學博士,尋除平西府記室參軍。太清初,為征西中
 記室,兼太常丞。



 侯景之亂,避地郢州,會梁邵陵王自東道至,引為諮議參軍。王僧辯之襲郢州也,素聞其名,召為儀同從事中郎。遷太尉從事中郎,與吳興沈炯對掌書記,府朝政務,一以委焉。晉安王承制,授給事黃門侍郎,亨奉箋辭府,僧辯答曰:「省告,承有朝授,良為德舉。卿操尚惇深,文藝該洽,學優而官,自致青紫。況久羈駿足,將成頓轡,匡輔虛闇,期寄實深。既欣遊處,用忘勞屈,而枳棘棲鵷,常以增歎。



 夕郎之選,雖為清顯,位以才升,差
 自無愧。且卿始云知命,方騁康衢,未有執戟之疲,便深夜行之慨,循復來翰,殊用憮然。古人相思,千里命駕,素心不昧,寧限城闉,存顧之深,荒慚無已。」



 高祖受禪,授中散大夫,領羽林監。遷太中大夫,領大著作,知梁史事。初,僧辯之誅也。所司收僧辯及其子頠屍,於方山同坎埋瘞,至是無敢言者。亨以故吏,抗表請葬之,乃與故義徐陵、張種、孔奐等,相率以家財營葬,凡七柩皆改窆焉。



 光大初,高宗入輔,以亨貞正有古人之風,甚相欽重,常以
 師禮事之。及到仲舉之謀出高宗也,毛喜知其詐,高宗問亨,亨勸勿奉詔。高宗即位,拜衛尉卿。太建二年卒,時年五十四。



 初撰《齊書》并《志》五十卷,遇亂失亡。後撰《梁史》,成者五十八卷。梁太清之後所製文筆六卷。



 子善心,早知名,官至尚書度支侍郎。



 褚玠,字溫理,河南陽翟人也。曾祖炫,宋昇明初與謝朏、江斅、劉俁入侍殿中,謂之四友。官至侍中、吏部尚書,謚貞子。祖涷,梁御史中丞。父蒙,太子舍人。玠九歲而孤,為
 叔父驃騎從事中郎隨所養。早有令譽,先達多以才器許之。及長,美風儀,善占對,博學能屬文,詞義典實,不好艷靡。起家王府法曹,歷轉外兵記室。天嘉中,兼通直散騎常侍,聘齊,還為桂陽王友。遷太子庶子、中書侍郎。



 太建中,山陰縣多豪猾,前後令皆以贓汙免,高宗患之,謂中書舍人蔡景歷曰:「稽陰大邑,久無良宰,卿文士之內,試思其人。」景歷進曰:「褚玠廉儉有乾用,未審堪其選不?」高宗曰:「甚善,卿言與朕意同。」乃除戎昭將軍、山陰令。縣
 民張次的、王休達等與諸猾吏賄賂通姦,全丁大戶,類多隱沒。玠乃鎖次的等,具狀啟臺,高宗手敕慰勞,并遣使助玠搜括,所出軍民八百餘戶。時舍人曹義達為高宗所寵,縣民陳信家富於財,諂事義達,信父顯文恃勢橫暴。玠乃遣使執顯文,鞭之一百,於是吏民股慄,莫敢犯者。信後因義達譖玠,竟坐免官。玠在任歲餘,守祿俸而已,去官之日,不堪自致,因留縣境,種蔬菜以自給。或嗤玠以非百里之才,玠答曰:「吾委輸課最,不後列城,除
 殘去暴,姦吏局蹐。若謂其不能自潤脂膏,則如來命。以為不達從政,吾未服也。」時人以為信然。皇太子知玠無還裝,手書賜粟米二百斛,於是還都。太子愛玠文辭,令入直殿省。十年,除電威將軍、仁威淮南王長史,頃之,以本官掌東宮管記。十二年,遷御史中丞,卒於官,時年五十二。



 玠剛毅有膽決,兼善騎射。嘗從司空侯安都於徐州出獵,遇有猛虎,玠引弓射之,再發皆中口入腹,俄而虎斃。及為御史中丞,甚有直繩之稱。自梁末喪亂,朝章
 廢弛,司憲因循,守而勿革,玠方欲改張,大為條例,綱維略舉,而編次未訖,故不列于後焉。及卒,太子親製誌銘,以表惟舊。至德二年,追贈祕書監。所製章奏雜文二百餘篇,皆切事理,由是見重於時。



 子亮,有才學,官至尚書殿中侍郎。



 岑之敬,字思禮,南陽棘陽人也。父善紆,梁世以經學聞,官至吳寧令、司義郎。之敬年五歲,讀《孝經》,每燒香正坐,親戚咸加歎異。年十六,策《春秋左氏》、制旨《孝經》義,擢
 為高第。御史奏曰:「皇朝多士,例止明經,若顏、閔之流,乃應高第。」梁武帝省其策曰:「何妨我復有顏、閔邪?」因召入面試,令之敬升講座,敕中書舍人朱異執《孝經》,唱《士孝章》,武帝親自論難。之敬剖釋縱橫,應對如響,左右莫不嗟服。乃除童子奉車郎,賞賜優厚。十八,預重雲殿法會,時武帝親行香,熟視之敬曰:「未几見兮,突而弁兮!」即日除太學限內博士。尋為壽光學士、司義郎,又除武陵王安西府刑獄參軍事。太清元年,表請試吏,除南沙令。



 侯景
 之亂,之敬率所部赴援京師。至郡境,聞臺城陷,乃與眾辭訣,歸鄉里。



 承聖二年,除晉安王宣惠府中記室參軍。是時蕭勃據嶺表,敕之敬宣旨慰喻,會江陵陷,仍留廣州。太建初,還朝,授東宮義省學士,太子素聞其名,尤降賞接。累遷鄱陽王中衛府記室、鎮北府中錄事參軍、南臺治書侍御史、征南府諮議參軍。



 之敬始以經業進,而博涉文史,雅有詞筆,不為醇儒。性廉謹,未嘗以才學矜物,接引後進,恂恂如也。每忌日營齋,必躬自洒掃,涕泣
 終日,士君子以篤行稱之。十一年卒,時年六十一。太子嗟惜,賻贈甚厚。有集十卷行於世。



 子德潤,有父風,官至中軍吳興王記室。



 陸琰,字溫玉,吏部尚書瓊之從父弟也。父令公,梁中軍宣城王記室參軍。琰幼孤、好學,有志操。州舉秀才。解褐宣惠始興王行參軍,累遷法曹外兵參軍,直嘉德殿學士。世祖聽覽餘暇,頗留心史籍,以琰博學,善占誦,引置左右。嘗使製《刀銘》,琰援筆即成,無所點竄,世祖嗟賞久
 之,賜衣一襲。俄兼通直散騎常侍,副琅邪王厚聘齊,及至鄴下而厚病卒,琰自為使主。時年二十餘,風神韶亮,占對閑敏,齊士大夫甚傾心焉。還為雲麾新安王主簿,遷安成王長史,寧遠府記室參軍。



 太建初,為武陵王明威府功曹史,兼東宮管記。丁母憂去官。五年卒,時年三十四。



 太子甚傷悼之,手令舉哀,加其賻贈,又自製誌銘。至德二年,追贈司農卿。



 琰寡嗜慾,鮮矜競,遊心經籍,晏如也。其所製文筆多不存本,後主求其遺文,撰成二卷。
 有弟瑜。



 瑜字幹玉。少篤學,美詞藻。州舉秀才。解褐驃騎安成王行參軍,轉軍師晉安王外兵參軍、東宮學士。兄琰時為管記,並以才學娛侍左右,時人比之二應。太建二年,太子釋奠于太學,宮臣並賦詩,命瑜為序,文甚贍麗。遷尚書祠部郎中,丁母憂去職。服闋,為桂陽王明威將軍功曹史,兼東宮管記。累遷永陽王文學、太子洗馬、中舍人。



 瑜幼長讀書,晝夜不廢,聰敏彊記,一覽無復遺失。嘗受《莊》、《老》於汝
 南周弘正,學《成實論》於僧滔法師,並通大旨。時皇太子好學,欲博覽群書,以子集繁多,命瑜鈔撰,未就而卒,時年四十四。太子為之流涕,手令舉哀,官給喪事,并親制祭文,遣使者弔祭。仍與詹事江總書曰:「管記陸瑜,奄然殂化,悲傷悼惜,此情何已。吾生平愛好,卿等所悉,自以學涉儒雅,不逮古人,欽賢慕士,是情尤篤。梁室亂離,天下糜沸,書史殘缺,禮樂崩淪,晚生後學,匪無牆面,卓爾出群,斯人而已。吾識覽雖局,未曾以言議假人,至於片善小
 才,特用嗟賞。況復洪識奇士,此故忘言之地。論其博綜子史,諳究儒墨,經耳無遺,觸目成誦,一褒一貶,一激一揚,語玄析理,披文摘句,未嘗不聞者心伏,聽者解頤,會意相得,自以為布衣之賞。吾監撫之暇,事隙之辰,頗用譚笑娛情,琴樽間作,雅篇艷什,迭互鋒起。每清風朗月,美景良辰,對群山之參差,望巨波之滉漾,或玩新花,時觀落葉,即聽春鳥,又聆秋鴈,未嘗不促膝舉觴,連情發藻,且代琢磨,間以嘲謔,俱怡耳目,並留情致。自謂百年
 為速,朝露可傷,豈謂玉折蘭摧,遽從短運,為悲為恨,當復何言。遺跡餘文,觸目增泫,絕絃投筆,恒有酸恨。以卿同志,聊復敘懷,涕之無從,言不寫意。」其見重如此。至德二年,追贈光祿卿。有集十卷。瑜有從父兄玠,從父弟琛。



 玠字潤玉,梁大匠卿晏子之子。弘雅有識度,好學,能屬文。舉秀才,對策高第。吏部尚書袁樞薦之於世祖,超授衡陽王文學,直天保殿學士。太建初,遷長沙王友,領記室。後主在東宮,聞其名,徵為管記。仍除中舍人,管記如故,
 甚見親待。尋以疾失明,將還鄉里,太子解衣贈玠,為之流涕。八年卒,時年三十七。有令舉哀,并加賵贈。至德二年,追贈少府卿。有集十卷。



 琛字潔玉,宣毅臨川王長史丘公之子。少警俊,事後母以孝聞。世祖為會稽太守,琛年十八,上《善政頌》,甚有詞采,由此知名,舉秀才。起家為衡陽王主簿,兼東宮管記。歷豫章王文學,領記室,司徒主簿,直宣明殿學士。尋遷尚書三公侍郎,兼通直散騎常侍,聘齊,還為司徒左西
 掾。又掌東宮管記,太子愛琛才辯,深禮遇之。後主嗣位,遷給事黃門侍郎、中書舍人,參掌機密。琛性頗疏,坐漏洩禁中語,詔賜死,時年四十二。



 何之元,廬江灊人也。祖僧達,齊南臺治書侍御史。父法勝,以行業聞。之元幼好學,有才思,居喪過禮,為梁司空袁昂所重。天監末,昂表薦之,因得召見。



 解褐梁太尉臨川王揚州議曹從事史,尋轉主簿。及昂為丹陽尹,辟為丹陽五官掾,總戶曹事。尋除信義令。之元宗人敬容者,
 勢位隆重,頻相顧訪,之元終不造焉。



 或問其故,之元曰:「昔楚人得寵於觀起,有馬者皆亡。夫德薄任隆,必近覆敗,吾恐不獲其利而招其禍。」識者以是稱之。



 會安西武陵王為益州刺史,以之元為安西刑獄參軍。侯景之亂,武陵王以太尉承制,授南梁州刺史、北巴西太守。武陵王自成都舉兵東下,之元與蜀中民庶抗表請無行,王以為沮眾,囚之元于艦中。及武陵兵敗,之元從邵陵太守劉恭之郡。俄而江陵陷,劉恭卒,王琳召為記室參軍。
 梁敬帝冊琳為司空,之元除司空府諮議參軍,領記室。



 王琳之立蕭莊也,署為中書侍郎。會齊文宣帝薨,令之元赴弔,還至壽春,而王琳敗,齊主以為揚州別駕,所治即壽春也。及在軍北伐,得淮南地,湘州刺史始興王叔陵遣功曹史柳咸齎書召之元。之元始與朝庭有隙,及書至,大惶恐,讀書至「孔璋無罪,左車見用」,之元仰而歎曰:「辭約若此,豈欺我哉!」遂隨咸至湘州。太建八年,除中衛府功曹參軍事,尋遷諮議參軍。



 及叔陵誅,之元乃屏
 絕人事,銳精著述。以為梁氏肇自武皇,終于敬帝,其興亡之運,盛衰之跡,足以垂鑒戒,定褒貶。究其始終,起齊永元元年,迄于王琳遇獲,七十五年行事,草創為三十卷,號曰《梁典》。其序曰:記事之史,其流不一,編年之作,無若《春秋》,則魯史之書,非帝皇之籍也。



 案三皇之簡為《三墳》,五帝之策為《五典》,此典義所由生也。至乃《尚書》述唐帝為《堯典》,虞帝為《舜典》,斯又經文明據。是以典之為義久矣哉。若夫《馬史》、《班漢》,述帝稱紀,自茲厥後,因相祖習。
 及陳壽所撰,名之曰志,總其三國,分路揚鑣。唯何法盛《晉書》變帝紀為帝典,既云師古,在理為優。故今之所作,稱為《梁典》。



 梁有天下,自中大同以前,區宇寧晏,太清以後,寇盜交侵,首尾而言,未為盡美,故開此一書,分為六意。以高祖創基,因乎齊末,尋宗討本,起自永元,今以前如乾卷為《追述》。高祖生自布衣,長於弊俗,知風教之臧否,識民黎之情偽。



 爰逮君臨,弘斯政術,四紀之內,實云殷阜。今以如乾卷為《太平》。世不常夷,時無恒治,非自我
 後,仍屬橫流,今以如乾卷為《敘亂》。洎高祖晏駕之年,太宗幽辱之歲,謳歌獄訟,向西陜不向東都;不庭之民,流逸之士,征伐禮樂,歸世祖不歸太宗。撥亂反正,厥庸斯在,治定功成,其勳有屬。今以如乾卷為《世祖》。



 至於四海困窮,五德升替,則敬皇紹立,仍以禪陳,今以如乾卷為《敬帝》。驃騎王琳,崇立後嗣,雖不達天命,然是其忠節,今以如乾卷為《後嗣主》。至在太宗,雖加美謚,而大寶之號,世所不遵,蓋以拘於賊景故也。承聖紀歷,自接太清,神
 筆詔書,非宜輒改,詳之後論,蓋有理焉。



 夫事有始終,人有業行,本末之間,頗宜詮敘。案臧榮緒稱史無裁斷,猶起居注耳,由此而言,實資詳悉。



 又編年而舉其歲次者,蓋取分明而易尋也。若夫獫狁孔熾,鯁我中原,始自一君,終為二主,事有相涉,言成混漫。今以未分之前為北魏,既分之後高氏所輔為東魏,宇文所挾為西魏,所以相分別也。重以蓋彰殊體,繁省異文,其間損益,頗有凡例。



 禎明三年,京城陷,乃移居常州之晉陵縣。隋開皇十
 三年,卒于家。



 徐伯陽,字隱忍,東海人也。祖度之,齊南徐州議曹從事史。父僧權,梁東宮通事舍人,領秘書,以善書知名。伯陽敏而好學,善色養,進止有節。年十五,以文筆稱。學《春秋左氏》。家有史書,所讀者近三千餘卷。試策高第,尚書板補梁河東王國右常侍、東宮學士、臨川嗣王府墨曹參軍。大同中,出為候官令,甚得民和。侯景之亂,伯陽浮海南至廣州,依於蕭勃,勃平還朝,仍將家屬之吳郡。



 天嘉
 二年,詔侍晉安王讀。尋除司空侯安都府記室參軍事,安都素聞其名,見之,降席為禮。甘露降樂遊苑,詔賜安都,令伯陽為謝表,世祖覽而奇之。太建初,中記室李爽、記室張正見、左民郎賀徹、學士阮卓、黃門郎蕭詮、三公郎王由禮、處士馬樞、記室祖孫登、比部賀循、長史劉刪等為文會之友,後有蔡凝、劉助、陳暄、孔範亦預焉。皆一時之士也。遊宴賦詩,勒成卷軸,伯陽為其集序,盛傳於世。



 及新安王為南徐州刺史,除鎮北新安王府中記室
 參軍,兼南徐州別駕,帶東海郡丞。鄱陽王為江州刺史,伯陽嘗奉使造焉,王率府僚與伯陽登匡嶺,置宴,酒酣,命筆賦劇韻二十,伯陽與祖孫登前成,王賜以奴婢雜物。及新安王還京,除臨海嗣王府限外諮議參軍。十一年春,皇太子幸太學,詔新安王於辟雍發《論語》題,仍命伯陽為《辟雍頌》,甚見嘉賞。除鎮右新安王府諮議參軍事。十三年,聞姊喪,發疾而卒,時年六十六。



 張正見,字見賾,清河東武城人也。祖蓋之,魏散騎常侍、
 勃海長樂二郡太守。



 父脩禮,魏散騎侍郎,歸梁,仍拜本職,遷懷方太守。正見幼好學,有清才。梁簡文在東宮,正見年十三,獻頌,簡文深贊賞之。簡文雅尚學業,每自升座說經,正見嘗預講筵,請決疑義,吐納和順,進退詳雅,四座咸屬目焉。太清初,射策高第,除邵陵王國左常侍。梁元帝立,拜通直散騎侍郎,遷彭澤令。屬梁季喪亂,避地於匡俗山,時焦僧度擁眾自保,遣使請交,正見懼之,遜辭延納,然以禮法自持,僧度亦雅相敬憚。



 高祖受禪,
 詔正見還都,除鎮東鄱陽王府墨曹行參軍,兼衡陽王府長史。歷宜都王限外記室、撰史著士,帶尋陽郡丞。累遷尚書度支郎、通直散騎侍郎,著士如故。太建中卒,時年四十九。有集十四卷,其五言詩尤善,大行於世。



 蔡凝,字子居,濟陽考城人也。祖撙,梁吏部尚書、金紫光祿大夫。父彥高,梁給事黃門侍郎。凝幼聰晤,美容止。既長,博涉經傳,有文辭,尤工草隸。天嘉四年,釋褐授祕書郎,轉廬陵王文學。光大元年,除太子洗馬、司徒主簿。太
 建元年,遷太子中舍人。以名公子選尚信義公主,拜駙馬都尉、中書侍郎。遷晉陵太守。



 及將之郡,更令左右緝治中書廨宇,謂賓友曰:「庶來者無勞,不亦可乎?」尋授寧遠將軍、尚書吏部侍郎。



 凝年位未高,而才地為時所重,常端坐西齋,自非素貴名流,罕所交接,趣時者多譏焉。高宗常謂凝曰:「我欲用義興主婿錢肅為黃門郎,卿意何如?」凝正色對曰:「帝鄉舊戚,恩由聖旨,則無所復問。若格以僉議,黃散之職,故須人門兼美,惟陛下裁之。」高宗
 默然而止。肅聞而有憾,令義興主日譖之於高宗,尋免官,遷交止。頃之,追還。



 後主嗣位,受晉安王諮議參軍,轉給事黃門侍郎。後主嘗置酒會,群臣歡甚,將移宴於弘範宮,眾人咸從,唯凝與袁憲不行。後主曰:「卿何為者?」凝對曰:「長樂尊嚴,非酒後所過,臣不敢奉詔。」眾人失色。後主曰:「卿醉矣。」即令引出。他日,後主謂吏部尚書蔡徵曰:「蔡凝負地矜才,無所用也。」尋遷信威晉熙王府長史,鬱鬱不得志,乃喟然歎曰:「天道有廢興,夫子云『樂天知命』,
 斯理庶幾可達。」因製《小室賦》以見志,甚有辭理。陳亡入隋,道病卒,時年四十七。



 子君知,頗知名。



 阮卓,陳留尉氏人。祖詮,梁散騎侍郎。父問道,梁寧遠岳陽王府記室參軍。



 卓幼而聰敏,篤志經籍,善談論,尤工五言詩。性至孝,其父隨岳陽王出鎮江州,遇疾而卒,卓時年十五,自都奔赴,水漿不入口者累日。屬侯景之亂,道路阻絕,卓冒履險艱,載喪柩還都。在路遇賊,卓形容毀瘁,號哭自陳,賊哀而不殺之,仍護送出境。及渡彭蠡
 湖,中流忽遇疾風,船幾沒者數四,卓仰天悲號,俄而風息,人皆以為孝感之至焉。



 世祖即位,除輕車鄱陽王府外兵參軍。天康元年,轉雲麾新安王府記室參軍,仍隋府轉翊右記室,帶撰史著士。遷鄱陽王中衛府錄事,轉晉安王府記室,著士如故。及平歐陽紇,交阯夷獠往往相聚為寇抄,卓奉使招慰。交阯通日南、象郡,多金翠珠貝珍怪之產,前後使者皆致之,唯卓挺身而還,衣裝無他,時論咸伏其廉。



 遷衡陽王府中錄事參軍。入為尚書
 祠部郎。遷始興王中衛府記室參軍。



 叔陵之誅也,後主謂朝臣曰:「阮卓素不同逆,宜加旌異。」至德元年,入為德教殿學士。尋兼通直散騎常侍,副王話聘隋。隋主夙聞卓名,乃遣河東薛道衡、琅邪顏之推等,與卓談宴賦詩,賜遺加禮。還除招遠將軍、南海王府諮議參軍。以目疾不之官,退居里舍,改構亭宇,脩山池卉木,招致賓友,以文酒自娛。禎明三年入于隋,行至江州,追感其父所終,因遘疾而卒,時年五十九。



 時有武威陰鏗,字子堅,梁左
 衛將軍子春之子。幼聰慧,五歲能誦詩賦,日千言。及長,博涉史傳,尤善五言詩,為當時所重。釋褐梁湘東王法曹參軍。天寒,鏗嘗與賓友宴飲,見行觴者,因回酒炙以授之,眾坐皆笑,鏗曰:「吾儕終日酣飲,而執爵者不知其味,非人情也。」及侯景之亂,鏗嘗為賊所擒,或救之獲免,鏗問其故,乃前所行觴者。天嘉中,為始興王府中錄事參軍。世祖嘗宴群臣賦詩,徐陵言之於世祖,即日召鏗預宴,使賦新成安樂宮,鏗授筆便就,世祖甚歎賞之。累
 遷招遠將軍、晉陵太守、員外散騎常侍,頃之卒。有集三卷行於世。



 史臣曰:夫文學者,蓋人倫之所基歟?是以君子異乎眾庶。昔仲尼之論四科,始乎德行,終於文學,斯則聖人亦所貴也。至如杜之偉之徒,值於休運,各展才用,之偉尤著美焉。



\end{pinyinscope}