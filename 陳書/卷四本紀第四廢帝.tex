\article{卷四本紀第四廢帝}

\begin{pinyinscope}

 廢帝,諱伯宗,字奉業,小字藥王,世祖嫡長子也。梁承聖三年五月庚寅生。



 永定二年二月戊辰,拜臨川王世子。三年,世祖嗣位,八月庚戌,立為皇太子。自梁室亂離,東
 宮焚燼,太子居于永福省。



 天康元年四月癸酉,世祖崩,其日,太子即皇帝位于太極前殿,詔曰:「上天降禍,大行皇帝奄棄萬國,攀號靡及,五內崩殞。朕以寡德,嗣膺寶命,煢煢在疚,懼甚綴旒,方賴宰輔,匡其不逮。可大赦天下。」又詔內外文武,各復其職,遠方悉停奔赴。五月己卯,尊皇太后曰太皇太后,皇后曰皇太后。庚寅,以驃騎將軍、司空、揚州刺史、新除尚書令安成王頊為驃騎大將軍,進位司徒、錄尚書、都督中外諸軍事。丁酉,中軍大將
 軍、開府儀同三司徐度進位司空;鎮南將軍、開府儀同三司、江州刺史章昭達為侍中,進號征南將軍;鎮東將軍、東揚州刺史始興王伯茂進號征東將軍、開府儀同三司;平北將軍、南徐州刺史鄱陽王伯山進號鎮北將軍;吏部尚書袁樞為尚書左僕射;雲麾將軍、吳興太守沈欽為尚書右僕射;新除中領軍吳明徹為領軍將軍;新除中護軍沈恪為護軍將軍;平南將軍、湘州刺史華皎進號安南將軍;散騎常侍、御史中丞徐陵為吏部尚
 書。六月辛亥,翊右將軍、右光祿大夫王通進號安右將軍。秋七月丁酉,立妃王氏為皇后。冬十月庚申,輿駕奉祠太廟。



 十一月乙亥,周遣使來弔。十二月甲子,高麗國遣使獻方物。



 光大元年春正月癸酉,尚書左僕射袁樞卒。乙亥,詔曰:「昔昊天成命,降集寶圖,二后重光,九區咸乂。閔余沖薄,王道未昭,荷茲神器,如涉靈海,庶親賢並建,牧伯惟良,天下雍熙,緬同刑措。今三元改歷,萬國充庭,清廟無追,
 具僚斯在,言瞻宁位,觸感崩心。思播遺恩,俾覃黎獻。可大赦天下。改天康二年為光大元年。孝悌力田賜爵一級。」己卯,以領軍將軍吳明徹為丹陽尹。辛卯,輿駕親祠南郊。二月辛亥,宣毅將軍、南豫州刺史餘孝頃謀反伏誅。癸丑,以征東將軍、開府儀同三司、東揚州刺史始興王伯茂為中衛大將軍,開府儀同三司黃法抃為鎮北將軍、南徐州刺史,鎮北將軍、南徐州刺史鄱陽王伯山為鎮東將軍、東陽州刺史。



 三月甲午,以尚書右僕射沈
 欽為侍中、尚書左僕射。夏四月乙卯,太白晝見。五月癸巳,以領軍將軍、丹陽尹吳明徹為安南將軍、湘州刺史。乙未,以鎮右將軍杜棱為領軍將軍。安南將軍、湘州刺史華皎謀反,丙申,以中撫大將軍淳于量為使持節、征南大將軍,總率舟師以討之。六月壬寅,以中軍大將軍、司空徐度進號車騎將軍,總督京邑眾軍,步道襲湘州。閏月癸巳,以雲麾將軍新安王伯固為丹陽尹。秋七月戊申,立皇子至澤為皇太子,賜天下為父後者爵一
 級,王公卿士已下賚帛各有差。



 九月乙巳,詔曰:「逆賊華皎,極惡窮凶,遂樹立蕭巋,謀危社稷。棄親即仇,人神憤惋,王師電邁,水陸爭前,梟剪之期,匪朝伊暮。其家口在北里尚方,宜從誅戮,用明國憲。」丙辰,百濟國遣使獻方物。是月,周將長胡公拓跋定率步騎二萬入郢州,與華皎水陸俱進,都督淳于量、吳明徹等與戰,大破之。皎單舸奔江陵,擒拓跋定,俘獲萬餘人,馬四千餘匹,送京師。冬十月辛巳,赦湘、巴二州為皎所詿誤者。甲申,輿駕親祠
 太廟。十一月己未,以護軍將軍沈恪為平西將軍、荊州刺史。甲子,侍中、中權將軍、開府儀同三司、特進、左光祿大夫王沖薨。十二月庚寅,以兼從事中郎孔英哲為奉聖亭侯,奉孔子祀。



 二年春正月己亥,侍中、都督中外諸軍事、驃騎大將軍、司徒、錄尚書、揚州刺史安成王頊進位太傅,領司徒,加殊禮,劍履上殿;侍中、征南將軍、開府儀同三司、江州刺史章昭達進號征南大將軍;中撫大將軍、新除征南大將軍淳于量為侍中、中軍大將
 軍、開府儀同三司;安南將軍、湘州刺史吳明徹即本號開府儀同三司,進號鎮南將軍;雲麾將軍、郢州刺史程靈洗進號安西將軍。庚子,詔討華皎軍人死王事者並給棺槥,送還本鄉,仍復其家。甲子,罷吳州,以鄱陽郡還屬江州。侍中、司空、車騎將軍徐度薨。夏四月辛巳,太白晝見。丁亥,割東揚州晉安郡為豊州。



 五月丙辰,太傅安成王頊獻玉璽一。六月丁卯,彗星見。秋七月丙午,輿駕親祠太廟。戊申,新羅國遣使獻方物。壬戌,立皇弟伯智
 為永陽王,伯謀為桂陽王。九月甲辰,林邑國遣使獻方物。丙午,狼牙脩國遣使獻方物。以侍中、征南大將軍、開府儀同三司、江州刺史章昭達為中撫大將軍。戊午,太白晝見。冬十月庚午,輿駕親祠太廟。十一月丙午,以前平西將軍、荊州刺史沈恪為護軍將軍。壬子,以鎮北將軍、開府儀同三司、南徐州刺史黃法抃為鎮西將軍、郢州刺史,新除中軍大將軍、開府儀同三司淳于量為鎮北將軍、南徐州刺史。甲寅,慈訓太后集群臣於朝堂,令
 曰:中軍儀同、鎮北儀同、鎮右將軍、護軍將軍、八座卿士:昔梁運季末,海內沸騰,天下蒼生,殆無遺噍。高祖武皇帝撥亂反正,膺圖御籙,重懸三象,還補二儀;世祖文皇帝克嗣洪基,光宣寶業,惠養中國,綏寧外荒;並戰戰兢兢,劬勞締構,庶幾鼎運,方隆殷、夏。伯宗昔在儲宮,本無令問,及居崇極,遂騁凶淫。居處諒闇,固不哀戚,嬪嬙弗隔,就館相仍,豈但衣車所納,是譏宗正,衰絰生子,得誚右師。七百之祚何憑,三千之罪為大。且費引金帛,令充
 椒閫,內府中藏,軍備國儲,未盈期稔,皆已空竭。太傅親承顧託,鎮守宮闈,遺誥綢繆,義深垣屏,而欑塗未御,翌日無淹,仍遣劉師知、殷不佞等顯言排斥。韓子高小豎輕佻,推心委仗,陰謀禍亂,決起蕭牆。元相雖持,但除君側。又以餘孝頃密邇京師,便相徵召,殃慝之咎,凶徒自擒,宗社之靈,祅氛是滅。於是密詔華皎,稱兵上流,國祚憂惶,幾移醜類。乃至要招遠近,葉力巴、湘,支黨縱橫,寇擾黟、歙。又別敕歐陽紇等攻逼衡州,嶺表紛紜,殊淹弦
 望。豈止罪浮於昌邑,非唯聲醜於太和。但賊豎皆亡,妖徒已散,日望懲改,猶加掩抑,而悖禮忘德,情性不悛,樂禍思亂,昏慝無已。張安國蕞爾凶狡,窮為小盜,仍遣使人蔣裕鉤出上京,即置行臺,分選凶黨。賊皎妻呂,舂徒為戮,納自奚官,藏諸永巷,使其結引親舊,規圖戕禍。盪主侯法喜等,太傅麾下,恒遊府朝,啖以深利,謀興肘腋。適又盪主孫泰等潛相連結,大有交通,兵力殊彊,指期挺亂。皇家有慶,歷數遐長,天誘其衷,同然開發。此諸文
 迹,今以相示,是而可忍,誰則不容?祖宗基業,將懼傾隕,豈可復肅恭禋祀,臨御兆民?式稽故實,宜在流放,今可特降為臨海郡王,送還籓邸。太傅安成王固天生德,齊聖廣深,二后鐘心,三靈佇眷。自前朝不悆,任總邦家,威惠相宣,刑禮兼設,指揮嘯詫,湘、郢廓清,闢地開疆,荊、益風靡,若太戊之承殷歷,中都之奉漢家,校以功名,曾何仿佛。且地彰靈璽,天表長彗,布新除舊,禎祥咸顯。



 文皇知子之鑒,事甚帝堯,傳弟之懷,又符太伯。今可還申曩
 志,崇立賢君,方固宗祧,載貞辰象。中外宜依舊典,奉迎輿駕。未亡人不幸屬此殷憂,不有崇替,容危社稷,何以拜祠高寢,歸祔武園?攬筆潸然,兼懷悲慶。



 是日,出居別第。太建二年四月薨,時年十九。



 帝仁弱無人君之器,世祖每慮不堪繼業。既居冢嫡,廢立事重,是以依違積載。



 及疾將大漸,召高宗謂曰:「吾欲遵太伯之事。」高宗初未達旨,後寤,乃拜伏涕泣,固辭。其後宣太后依詔廢帝焉。



 史臣曰:臨海雖繼體之重,仁厚懦弱,混一是非,不驚得
 喪,蓋帝摯、漢惠之流也。世祖知神器之重,諒難負荷,深鑒堯旨,弗傳寶祚焉。






\end{pinyinscope}