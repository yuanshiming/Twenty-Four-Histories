\article{卷八列傳第二杜僧明 周文育 子寶安 侯安都}

\begin{pinyinscope}

 杜僧明,字弘照,廣陵臨澤人也。形貌眇小,而膽氣過人,有勇力,善騎射。



 梁大同中,盧安興為廣州南江督護,僧明與兄天合及周文育並為安興所啟,請與俱行。頻征
 俚獠有功,為新州助防。天合亦有材幹,預在征伐。安興死,僧明復副其子子雄。及交州土豪李賁反,逐刺史蕭諮,諮奔廣州,臺遣子雄與高州刺史孫冏討賁。時春草已生,瘴癘方起,子雄請待秋討之,廣州刺史新渝侯蕭映不聽,蕭諮又促之,子雄等不得已,遂行。至合浦,死者十六七,眾並憚役潰散,禁之不可,乃引其餘兵退還。蕭諮啟子雄及冏與賊交通,逗留不進,梁武帝敕於廣州賜死。子雄弟子略、子烈並雄豪任俠,家屬在南江。天合
 謀於眾曰:「盧公累代待遇我等亦甚厚矣,今見枉而死,不能為報,非丈夫也。我弟僧明萬人之敵,若圍州城,召百姓,誰敢不從。城破,斬二侯祭孫、盧,然後待臺使至,束手詣廷尉,死猶勝生。縱其不捷,亦無恨矣。」眾咸慷慨曰:「是願也,唯足下命之。」乃與周文育等率眾結盟,奉子雄弟子略為主,以攻刺史蕭映。子略頓城南,天合頓城北,僧明、文育分據東西,吏人並應之,一日之中,眾至數萬。高祖時在高要,聞事起,率眾來討,大破之,殺天合,生擒
 僧明及文育等,高祖並釋之,引為主帥。



 高祖征交止及討元景仲,僧明、文育並有功。侯景之亂,俱隨高祖入援京師。



 高祖於始興破蘭裕,僧明為前鋒,擒裕斬之。又與蔡路養戰於南野,僧明馬被傷,高祖馳往救之,以所乘馬授僧明,僧明乘馬與數十人復進,眾皆披靡,因而乘之,大敗路養。高州刺史李遷仕又據大皋,入灨石,以逼高祖,高祖遣周文育為前軍,與僧明擊走之。遷仕與寧都人劉孝尚併力將襲南康,高祖又令僧明與文育等
 拒之,相持連戰百餘日,卒擒遷仕,送于高祖軍。及高祖下南康,留僧明頓西昌,督安成、廬陵二郡軍事。元帝承制授假節、清野將軍、新州刺史,臨江縣子,邑三百戶。



 侯景遣于慶等寇南江,高祖頓豫章,命僧明為前驅,所向克捷。高祖表僧明為長史,仍隨東討。軍至蔡洲,僧明率麾下燒賊水門大艦。及景平,以功除員外散騎常侍、明威將軍、南兗州刺史,進爵為侯,增邑併前五百戶,仍領晉陵太守。承聖二年,從高祖北圍廣陵,加使持節,遷通
 直散騎常侍、平北將軍、餘如故。荊州陷,高祖使僧明率吳明徹等隨侯瑱西援,於江州病卒,時年四十六。贈散騎常侍,謚曰威。世祖即位,追贈開府儀同三司。天嘉二年,配享高祖廟庭。子晉嗣。



 周文育,字景德,義興陽羨人也。少孤貧,本居新安壽昌縣,姓項氏,名猛奴。



 年十一,能反覆游水中數里,跳高五六尺,與群兒聚戲,眾莫能及。義興人周薈為壽昌浦口戍主,見而奇之,因召與語。文育對曰:「母老家貧,兄姊並
 長大,困於賦役。」薈哀之,乃隨文育至家,就其母請文育養為己子,母遂與之。及薈秩滿,與文育還都,見於太子詹事周捨,請製名字,捨因為立名文育,字景德。命兄子弘讓教之書計。弘讓善隸書,寫蔡邕《勸學》及古詩以遺文育,文育不之省也,謂弘讓曰:「誰能學此,取富貴但有大槊耳。」弘讓壯之,教之騎射,文育大悅。



 司州刺史陳慶之與薈同郡,素相善,啟薈為前軍軍主。慶之使薈將五百人往新蔡懸瓠,慰勞白水蠻,蠻謀執薈以入魏,事覺,
 薈與文育拒之。時賊徒甚盛,一日之中戰數十合,文育前鋒陷陣,勇冠軍中。薈於陣戰死,文育馳取其尸,賊不敢逼。



 及夕,各引去。文育身被九創,創愈,辭請還葬,慶之壯其節,厚加灊遺而遣之。



 葬訖,會廬安興為南江督護,啟文育同行。累徵俚獠,所在有功,除南海令。安興死後,文育與杜僧明攻廣州,為高祖所敗,高祖赦之,語在僧明傳。



 後監州王勱以文育為長流,深被委任。勱被代,文育欲與勱俱下,至大庾嶺,詣卜者,卜者曰:「君北下不
 過作令長,南入則為公侯。」文育曰:「足錢便可,誰望公侯。」卜人又曰:「君須臾當暴得銀至二千兩,若不見信,以此為驗。」其夕,宿逆旅,有賈人求與文育博,文育勝之,得銀二千兩。旦日辭勱,勱問其故,文育以告,勱乃遣之。高祖在高要,聞其還也,大喜,遣人迎之,厚加賞賜,分麾下配焉。



 高祖之討侯景,文育與杜僧明為前軍,克蘭裕,援歐陽頠,皆有功。高祖破蔡路養於南野,文育為路養所圍,四面數重,矢石雨下,所乘馬死,文育右手搏戰,左手解
 鞍,潰圍而出,因與杜僧明等相得,并力復進,遂大敗之。高祖乃表文育為府司馬。



 李遷仕之據大皋,遣其將杜平虜入灨石魚梁作城,高祖命文育擊之,平虜棄城走,文育據其城。遷仕聞平虜敗,留老弱於大皋,悉選精兵自將,以攻文育,其鋒甚銳,軍人憚之。文育與戰,遷仕稍卻,相持未解,會高祖遣杜僧明來援,別破遷仕水軍,遷仕眾潰,不敢過大皋,直走新淦。梁元帝授文育假節、雄信將軍、義州刺史。遷仕又與劉孝尚謀拒義軍,高祖遣
 文育與侯安都、杜僧明、徐度、杜棱築城於白口拒之。文育頻出與戰,遂擒遷仕。



 高祖發自南康,遣文育將兵五千,開通江路。侯景將王伯醜據豫章,文育擊走之,遂據其城。累前後功,除游騎將軍、員外散騎常侍,封東遷縣侯,邑五百戶。



 高祖軍至白茅灣,命文育與杜僧明常為軍鋒,平南陵、鵲頭諸城。及至姑熟,與景將侯子鑒戰,破之。景平,授通直散騎常侍,改封南移縣侯,邑一千戶,拜信義太守。累遷南丹陽、蘭陵、晉陵太守、智武將軍、散騎
 常侍。



 高祖誅王僧辯,命文育督眾軍會世祖於吳興,圍杜龕,克之。又濟江襲會稽太守張彪,得其郡城。及世祖為彪所襲,文育時頓城北香巖寺,世祖夜往趨之,因共立柵。頃之,彪又來攻,文育悉力苦戰,彪不能克,遂破平彪。



 高祖以侯瑱擁據江州,命文育討之,仍除都督南豫州諸軍事、武威將軍、南豫州刺史,率兵襲湓城。未克,徐嗣徽引齊寇渡江據蕪湖,詔徵文育還京。嗣徽等列艦於青墩,至于七磯,以斷文育歸路。及夕,文育鼓噪而發,
 嗣徽等不能制。至旦,反攻嗣徽,嗣徽驍將鮑砰獨以小艦殿軍,文育乘單舴艋與戰,跳入艦,斬砰,仍牽其艦而還。賊眾大駭,因留船蕪湖,自丹陽步上。時高祖拒嗣徽於白城,適與文育大會。將戰,風急,高祖曰:「兵不逆風。」文育曰:「事急矣,當決之,何用古法。」抽槊上馬,馳而進,眾軍從之,風亦尋轉,殺傷數百人。嗣微等移營莫府山,文育徙頓對之。頻戰功最,加平西將軍,進爵壽昌縣公,并給鼓吹一部。



 廣州刺史蕭勃舉兵踰嶺,詔文育督眾軍討之。
 時新吳洞主餘孝頃舉兵應勃,遣其弟孝勱守郡城,自出豫章,據于石頭。勃使其子孜將兵與孝頃相會,又遣其別將歐陽騑頓軍苦竹灘,傅泰據墌口城,以拒官軍。官軍船少,孝頃有舴艋三百艘、艦百餘乘在上牢,文育遣軍主焦僧度、羊柬潛軍襲之,悉取而歸,仍於豫章立柵。時官軍食盡,並欲退還,文育不許。乃使人間行遺周迪書,約為兄弟,并陳利害。迪得書甚喜,許饋糧餉。於是文育分遣老小乘故船舫,沿流俱下,燒豫章郡所立柵,
 偽退。孝頃望之,大喜,因不設備。文育由間道兼行,信宿達芊韶。芊韶上流則歐陽頠、蕭勃,下流則傅泰、餘孝頃,文育據其中間,築城饗士,賊徒大駭。歐陽頠乃退入泥溪,作城自守。文育遣嚴威將軍周鐵虎與長史陸山才襲頠,擒之。於是盛陳兵甲,與頠乘舟而讌,以巡傅泰城下,因而攻泰,克之。蕭勃在南康聞之,眾皆股慄,莫能自固。其將譚世遠斬勃欲降,為人所害。世遠軍主夏侯明徹持勃首以降。



 蕭孜、餘孝頃猶據石頭,高祖遣侯安都
 助文育攻之,孜降文育,孝頃退走新吳,文州平,廣育還頓豫章。以功授鎮南將軍、開府儀同三司、都督江廣衡交等州諸軍事、江州刺史。



 王琳擁據上流,詔命侯安都為西道都督,文育為南道都督,同會武昌。與王琳戰於沌口,為琳所執,後得逃歸,語在安都傳。尋授使持節、散騎常侍、鎮南將軍、開府儀同三司,壽昌縣公,給鼓吹一部。



 及周迪破餘孝頃,孝頃子公颺、弟孝勱猶據舊柵,扇動南土,高祖復遣文育及周迪、黃法抃等討之。豫章內
 史熊曇朗亦率軍來會,眾且萬人。文育遣吳明徹為水軍,配周迪運糧,自率眾軍入象牙江,城於金口。公颺領五百人偽降,謀執文育,事覺,文育囚之,送于京師,以其部曲分隸眾軍。乃捨舟為步軍,進據三陂。王琳遣將曹慶帥兵二千人以救孝勱,慶分遣主帥常眾愛與文育相拒,自帥所領徑攻周迪、吳明徹軍。迪等敗績,文育退據金口。熊曇朗因其失利,謀害文育,以應眾愛。文育監軍孫白象頗知其事,勸令先之。文育曰:「不可,我舊兵少,
 客軍多,若取曇朗,人人驚懼,亡立至矣,不如推心以撫之。」初,周迪之敗也,棄船走,莫知所在,及得迪書,文育喜,齎示曇朗,曇朗害之於座,時年五十一。高祖聞之,即日舉哀,贈侍中、司空、謚曰忠愍。



 初,文育之據三陂,有流星墜地,其聲如雷,地陷方一丈,中有碎炭數斗。又軍市中忽聞小兒啼,一市並驚,聽之在土下,軍人掘得棺長三尺,文育惡之。俄而迪敗,文育見殺。天嘉二年,有詔配享高祖廟庭。子寶安嗣。文育本族兄景曜,因文育官至新
 安太守。



 寶安字安民。年十餘歲,便習騎射,以貴公子驕蹇遊逸,好狗馬,樂馳騁,靡衣媮食。文育之為晉陵,以征討不遑之郡,令寶安監知郡事,尤聚惡少年,高祖患之。及文育西征敗績,縶於王琳,寶安便折節讀書,與士君子遊,綏御文育士卒,甚有威惠。除員外散騎侍郎。文育歸,復除貞威將軍、吳興太守。文育為熊曇朗所害,徵寶安還。起為猛烈將軍,領其舊兵,仍令南討。



 世祖即位,深器重之,
 寄以心膂,精卒利兵多配焉。及平王琳,頗有功。周迪之破熊曇朗,寶安南入,窮其餘燼。天嘉二年,重除雄信將軍、吳興太守,襲封壽昌縣公。三年,征留異,為侯安都前軍。異平,除給事黃門侍郎、衛尉卿。四年,授持節、都督南徐州諸軍事、貞毅將軍、南徐州刺史。徵為左衛將軍,加信武將軍。



 尋以本官領衛尉卿,又進號仁威將軍。天康元年卒,時年二十九。贈侍中、左衛將軍,謚曰成。



 子鷿嗣。寶安卒後,鷿亦為偏將。征歐陽紇,平定淮南,並有功,封
 江安縣伯,邑四百戶。歷晉陵、定遠二郡太守。太建九年卒,時年二十四,贈電威將軍。



 侯安都,字成師,始興曲江人也。世為郡著姓。父文捍,少仕州郡,以忠謹稱,安都貴後,官至光祿大夫、始興內史,秩中二千石。



 安都工隸書,能鼓琴,涉獵書傳,為五言詩,亦頗清靡,兼善騎射,為邑里雄豪。梁始興內史蕭子範辟為主簿。侯景之亂,招集兵甲,至三千人。高祖入援京邑,安都引兵從高祖,攻蔡路養,破李遷仕,克平侯景,並
 力戰有功。元帝授猛烈將軍、通直散騎常侍,富川縣子,邑三百戶。隨高祖鎮京口,除蘭陵太守。高祖謀襲王僧辯,諸將莫有知者,唯與安都定計,仍使安都率水軍自京口趨石頭,高祖自率馬步從江乘羅落會之。安都至石頭北,棄舟登岸,僧辯弗之覺也。石頭城北接崗阜,雉堞不甚危峻,安都被甲帶長刀,軍人捧之投於女垣內,眾隨而入,進逼僧辯臥室。



 高祖大軍亦至,與僧辯戰於聽事前,安都自內閣出,腹背擊之,遂擒僧辯。



 紹泰元年,
 以功授使持節、散騎常侍、都督南徐州諸軍事、仁威將軍、南徐州刺史。高祖東討杜龕,安都留臺居守。徐嗣徽、任約等引齊寇入據石頭,游騎至於闕下。安都閉門偃旗幟,示之以弱,令城中曰:「登陴看賊者斬。」及夕,賊收軍還石頭,安都夜令士卒密營禦敵之具。將旦,賊騎又至,安都率甲士三百人,開東西掖門與戰,大敗之,賊乃退還石頭,不敢復逼臺城。及高祖至,以安都為水軍,於中流斷賊糧運。又襲秦郡,破嗣徽柵,收其家口并馬驢輜
 重。得嗣徽所彈琵琶及所養鷹,遣信餉之曰:「昨至弟住處得此,今以相還。」嗣徽等見之大懼,尋而請和,高祖聽其還北。及嗣徽等濟江,齊之餘軍猶據采石,守備甚嚴,又遣安都攻之,多所俘獲。



 明年春,詔安都率兵鎮梁山,以備齊。徐嗣徽等復入丹陽,至湖熟,高祖追安都還,率馬步拒之於高橋。又戰於耕壇南,安都率十二騎,突其陣,破之,生擒齊儀同乞伏無勞。又刺齊將東方老墮馬,會賊騎至,救老獲免。賊北渡蔣山,安都又與齊將王敬
 寶戰於龍尾,使從弟曉、軍主張纂前犯其陣。曉被槍墜馬,張纂死之。



 安都馳往救曉,斬其騎士十一人,因取纂尸而還,齊軍不敢逼。高祖與齊軍戰於莫府山,命安都領步騎千餘人,自白下橫擊其後,齊軍大敗。安都又率所部追至攝山,俘獲首虜不可勝計。以功進爵為侯,增邑五百戶,給鼓吹一部。又進號平南將軍,改封西江縣公。



 仍都督水軍出豫章,助豫州刺史周文育討蕭勃。安都未至,文育已斬勃,并擒其將歐陽頠、傅泰等。唯餘孝
 頃與勃子孜猶據豫章之石頭,作兩城,孝頃與孜各據其一,又多設船艦,夾水而陣。安都至,乃銜枚夜燒其艦。文育率水軍,安都領步騎,登岸結陣。孝頃俄斷後路,安都乃令軍士多伐松木,豎柵,列營漸進,頻戰屢克,孜乃降。孝頃奔歸新吳,請入子為質,許之。師還,以功進號鎮北將軍,加開府儀同三司。



 仍率眾會於武昌,與周文育西討王琳。將發,王公已下餞於新林,安都躍馬渡橋,人馬俱墮水中,又坐絺內墜於櫓井,時以為不祥。至武昌,
 琳將樊猛棄城走。



 文育亦自豫章至。時兩將俱行,不相統攝,因部下交爭,稍不平。軍至郢州,琳將潘純陀於城中遙射官軍,安都怒,進軍圍之,未能克。而王琳至于弇口,安都乃釋郢州,悉眾往沌口以御之,遇風不得進。琳據東岸,官軍據西岸,相持數日,乃合戰,安都等敗績。安都與周文育、徐敬成並為琳所囚。琳總以一長鎖繫之,置於絺下,令所親宦者王子晉掌視之。琳下至湓城白水浦,安都等甘言許厚賂子晉。子晉乃偽以小船依絺
 而釣,夜載安都、文育、敬成上岸,入深草中,步投官軍。還都自劾,詔並赦之,復其官爵。



 尋為丹陽尹,出為都督南豫州諸軍事、鎮西將軍、南豫州刺史。令繼周文育攻餘孝勱及王琳將曹慶、常眾愛等。安都自宮亭湖出松門,躡眾愛後。文育為熊曇朗所害,安都回取大艦,值琳將周炅、周協南歸,與戰,破之,生擒炅、協。孝勱弟孝猷率部下四千家欲就王琳,遇炅、協敗,乃詣安都降。安都又進軍於禽奇洲,破曹慶、常眾愛等,焚其船艦。眾愛奔于廬
 山,為村人所殺,餘眾悉平。



 還軍至南皖,而高祖崩,安都隨世祖還朝,仍與群臣定議,翼奉世祖。時世祖謙讓弗敢當,太后又以衡陽王故,未肯下令,群臣猶豫不能決。安都曰:「今四方未定,何暇及遠,臨川王有功天下,須共立之。今日之事,後應者斬。」便按劍上殿,白太后出璽,又手解世祖髮,推就喪次。世祖即位,遷司空,仍為都督南徐州諸軍事、征北將軍、南徐州刺史,給扶。



 王琳下至柵口,大軍出頓蕪湖,時侯瑱為大都督,而指麾經略,多出
 安都。天嘉元年,增邑千戶。及王琳敗走入齊,安都進軍湓城,討琳餘黨,所向皆下。



 仍別奉中旨,迎衡陽獻王昌。初,昌之將入也,致書於世祖,辭甚不遜,世祖不懌,乃召安都從容而言曰:「太子將至,須別求一蕃,吾其老焉。」安都對曰:「自古豈有被代天子?臣愚不敢奉詔。」因請自迎昌,昌濟漢而薨。以功進爵清遠郡公,邑四千戶。自是威名甚重,群臣無出其右。



 安都父文捍,為始興內史,卒於官。世祖征安都還京師,為發喪。尋起復本官,贈其父散
 騎常侍、金紫光祿大夫,拜其母為清遠國太夫人。仍迎還都,母固求停鄉里,上乃下詔,改桂陽之汝城縣為盧陽郡,分衡州之始興、安遠二郡,合三郡為東衡州,以安都從弟曉為刺史,安都第三子秘年九歲,上以為始興內史,並令在鄉侍養。其年,改封安都桂陽郡公。



 王琳敗後,周兵入據巴、湘,安都奉詔西討。及留異擁據東陽,又奉詔東討。



 異本謂臺軍由錢塘江而上,安都乃步由會稽之諸暨,出于永康。異大恐,奔桃枝嶺,處嶺谷間,於巖
 口堅柵,以拒王師。安都作連城攻異,躬自接戰,為流矢所中,血流至踝,安都乘輿麾軍,容止不變。因其山壟之勢,迮而為堰。天嘉三年夏,潦,水漲滿,安都引船入堰,起樓艦與異城等,放拍碎其樓雉。異與第二子忠臣脫身奔晉安,安都虜其妻子,盡收其人馬甲仗,振旅而歸。以功加侍中、征北大將軍,增邑并前五千戶,仍還本鎮。其年,吏民詣闕表請立碑,頌美安都功績,詔許之。



 自王琳平後,安都勳庸轉大,又自以功安社稷,漸用驕矜,數招
 聚文武之士,或射馭馳騁,或命以詩賦,第其高下,以差次賞賜之。文士則褚玠、馬樞、陰鏗、張正見、徐伯陽,劉刪、祖孫登,武士則蕭摩訶、裴子烈等,並為之賓客,齋內動至千人。部下將帥,多不遵法度,檢問收攝,則奔歸安都。世祖性嚴察,深銜之。



 安都弗之改,日益驕橫。每有表啟,封訖,有事未盡,乃開封自書之,云又啟某事。



 及侍宴酒酣,或箕踞傾倚。嘗陪樂遊禊飲,乃白帝曰:「何如作臨川王時?」帝不應。安都再三言之,帝曰:「此雖天命,抑亦明公
 之力。」宴訖,又啟便借供帳水飾,將載妻妾於御堂歡會,世祖雖許其請,甚不懌。明日,安都坐於御坐,賓客居群臣位,稱觴上壽。初,重雲殿災,安都率將士帶甲入殿,帝甚惡之,自是陰為之備。又周迪之反,朝望當使安都討之,帝乃使吳明徹討迪,又頻遣臺使案問安都部下,檢括亡叛,安都內不自安。三年冬,遣其別駕周弘實自託於舍人蔡景歷,并問省中事。景歷錄其狀具奏之,希旨稱安都謀反。世祖慮其不受制,明年春,乃除安都為都
 督江吳二州諸軍事、征南大將軍、江州刺史。自京口還都,部伍入於石頭,世祖引安都宴於嘉德殿,又集其部下將帥會于尚書朝堂,於坐收安都,囚於嘉德西省,又收其將帥,盡奪馬仗而釋之。因出舍人蔡景歷表以示於朝。乃詔曰:「昔漢厚功臣,韓、彭肇亂,晉倚蕃牧,敦、約稱兵。託六尺於龐萌,野心竊發;寄股肱於霍禹,凶謀潛構。追惟往代,挻逆一揆,永言自古,患難同規。侯安都素乏遙圖,本慚令德,幸屬興運,預奉經綸,拔跡行間,假之毛
 羽,推於偏帥,委以馳逐。位極三槐,任居四獄,名器隆赫,禮數莫儔。而志唯矜己,氣在陵上,招聚逋逃,窮極輕狡,無賴無行,不畏不恭。受脤專征,剽掠一逞,推轂所鎮,裒斂無厭。寄以徐蕃,接鄰齊境,貿遷禁貨,鬻賣居民,椎埋發掘,毒流泉壤,睚眥僵尸,罔顧彞憲。朕以爰初締構,頗著功績,飛驂代邸,預定嘉謀,所以淹抑有司,每懷遵養,杜絕百辟,日望自新。款襟期於話言,推丹赤於造次,策馬甲第,羽林息警,置酒高堂,陛戟無衛。何嘗內隱片嫌,
 去柏人而勿宿,外協猜防,入成皋而不留?而勃戾不悛,驕暴滋甚,招誘文武,密懷異圖。去年十二月十一日,獲中書舍人蔡景歷啟,稱侯安都去月十日遣別駕周弘實來景歷私省宿,訪問禁中,具陳反計,朕猶加隱忍,待之如初。爰自北門,遷授南服,受命經停,姦謀益露。今者欲因初鎮,將行不軌。此而可忍,孰不可容?賴社稷之靈,近侍誠愨,醜情彰暴,逆節顯聞。外可詳案舊典,速正刑書,止在同謀,餘無所問。」明日,於西省賜死,時年四十四。



 尋有詔,宥其妻子家口,葬以士禮,喪事所須,務加資給。



 初,高祖在京城,嘗與諸將宴,杜僧明、周文育、侯安都為壽,各稱功伐。高祖曰:「卿等悉良將也,而並有所短。杜公志大而識闇,狎於下而驕於尊,矜其功不收其拙。周侯交不擇人,而推心過差,居危履險,猜防不設。侯郎傲誕而無厭,輕佻而肆志。並非全身之道。」卒皆如其言。



 安都長子敦,年十二,為員外散騎侍郎,天嘉二年墮馬卒,追謚桂陽國愍世子。



 太建三年,高宗追封安都為陳集縣
 侯,邑五百戶,子亶為嗣。



 安都從弟曉,累從安都征討有功,官至員外散騎常侍、明威將軍、東衡州刺史,懷化縣侯,邑五百戶。天嘉三年卒,年四十一。



 史臣曰:杜僧明、周文育並樹功業,成於興運,頗、牧、韓、彭,足可連類矣。



 侯安都情異向時,權逾曩日,因之以侵暴,加之以縱誕,茍曰非夫逆亂,奚用免於亡滅!昔漢高醢之為賜,宋武拉於坐右,良有以而然也。



\end{pinyinscope}