\article{曾鞏《陳書》目錄序}

\begin{pinyinscope}

 《
 陳書》六本紀,三十列傳,凡三十六篇,唐散騎常侍姚思廉撰。始思廉父察,梁、陳之史官也。錄二代之事,未就而陳亡。隋文帝見察甚重之,每就察訪梁陳故事,察因以所論載每一篇成輒奏之,而文帝亦遣虞世基就察求其書,又未就而察死。



 察之將死,屬思廉以繼其業。唐興,武德五年,高祖以自魏以來,二百餘歲,世統數更,史事
 放逸,乃詔撰次。而思廉遂受詔為《陳書》。久之,猶不就。貞觀三年,遂詔論撰於秘書內省。十年正月壬子,始上之。



 觀察等之為此書,歷三世,傳父子,更數十歲而後乃成,蓋其難如此。然及其既成,與宋、魏、梁、齊等書,世亦傳之者少,故學者於其行事之迹,亦罕得而詳也。而其書亦以罕傳,則自秘府所藏,往往脫誤。嘉祐六年八月,始詔校讎,使可鏤板行之天下。而臣等言:「梁、陳等書缺,獨館閣所藏,恐不足以定箸。願詔京師及州縣藏書之家,使
 悉上之。」先皇帝為下其事。至七年冬,稍稍始集,臣等以相校。至八年七月,《陳書》三十六篇者始校定,可傳之學者。其疑者亦不敢損益,特各書疏于篇末。其書舊無目,列傳名氏多闕謬,因別為目錄一篇,使覽者得詳焉。



 夫陳之為陳,蓋偷為一切之計,非有先王經紀禮義風化之美,制治之法,可章示後世。然而兼權尚計,明於任使,恭儉愛人,則其始之所以興;惑於邪臣,溺於嬖妾,忘患縱欲,則其終之所以亡。興亡之端,莫非自己致者。至於有
 所因造,以為號令威刑職官州郡之制,雖其事已淺,然亦各施於一時,皆學者之所不可不考也。



 而當時之士,自爭奪詐偽,茍得偷合之徒,尚不得不列以為世戒;而況於壞亂之中,蒼皇之際,士之安貧樂義,取舍去就不為患禍勢利動其心者,亦不絕於其間。若此人者,可謂篤於善焉。蓋古人之所思見而不可得,《風雨》之詩所為作者也,安可使之泯泯不少概見於天下哉!則陳之史,其可廢乎?



 蓋此書成之既難,其後又久不顯。及宋興已
 百年,古文遺事,靡不畢講,而始得盛行於天下,列於學者,其傳之之難又如此,豈非遭遇固自有時也哉!



 臣恂、臣穆、臣藻、臣覺、臣彥若、臣洙、臣鞏謹敘目錄昧死上



\end{pinyinscope}