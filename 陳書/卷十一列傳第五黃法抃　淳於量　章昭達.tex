\article{卷十一列傳第五黃法抃 淳於量 章昭達}

\begin{pinyinscope}

 黃法抃,字仲昭,巴山新建人也。少勁捷有膽力,步行日三百里,距躍三丈。



 頗便書疏,閑明簿領,出入郡中,為鄉閭所憚。侯景之亂,於鄉里合徒眾。太守賀詡下江州,法
 抃監知郡事。高祖將踰嶺入援建業,李遷仕作梗中途,高祖命周文育屯于西昌,法抃遣兵助文育。時法抃出頓新淦縣,景遣行臺于慶至豫章,慶分兵來襲新淦,法抃拒戰,敗之。高祖亦遣文育進軍討慶,文育疑慶兵彊,未敢進,法抃率眾會之,因進克笙屯,俘獲甚眾。



 梁元帝承制授超猛將軍、交州刺史資,領新淦縣令,封巴山縣子,邑三百戶。



 承聖三年,除明威將軍、游騎將軍,進爵為侯,邑五百戶。貞陽侯僭位,除左驍騎將軍。敬帝即位,改
 封新建縣侯,邑如前。太平元年,割江州四郡置高州,以法抃為使持節、散騎常侍、都督高州諸軍事、信武將軍、高州刺史,鎮于巴山。蕭勃遣歐陽頠攻法抃,法抃與戰,破之。



 永定二年,王琳遣李孝欽、樊猛、餘孝頃攻周迪,且謀取法抃,法抃率兵援迪,擒孝頃等三將。進號宣毅將軍,增邑并前一千戶,給鼓吹一部。又以拒王琳功,授平南將軍、開府儀同三司。熊曇朗於金口反,害周文育,法抃共周迪討平之,語在曇朗傳。



 世祖嗣位,進號安南將
 軍。天嘉二年,周迪反,法抃率兵會都督吳明徹,討迪於工塘。迪平,法抃功居多,徵為使持節、散騎常侍、都督南徐州諸軍事、鎮北大將軍、南徐州刺史,儀同、鼓吹並如故。未拜,尋又改授都督江、吳二州諸軍事、鎮南大將軍、江州刺史。六年,徵為中衛大將軍。廢帝即位,進爵為公,給扶。光大元年,出為使持節、都督南徐州諸軍事、鎮北將軍、南徐州刺史。二年,徙為都督郢、巴、武三州諸軍事、鎮西將軍、郢州刺史,持節如故。



 太建元年,進號征西大
 將軍。二年,徵為侍中、中權大將軍。四年,出為使持節、散騎常侍、都督南豫州諸軍事、征南大將軍、南豫州刺史。五年,大舉北伐,都督吳明徹出秦郡,以法抃為都督,出歷陽。齊遣其歷陽王步騎五萬來援,於小峴築城。法抃遣左衛將軍樊毅分兵於大峴禦之,大破齊軍,盡獲人馬器械。於是乃為拍車及步艦,豎拍以逼歷陽。歷陽人窘蹴乞降,法抃緩之,則又堅守,法抃怒,親率士卒攻城,施拍加其樓堞。時又大雨,城崩,克之,盡誅戍卒。進兵合
 肥,望旗降款,法抃不令軍士侵掠,躬自撫勞,而與之盟,並放還北。以功加侍中,改封義陽郡公,邑二千戶。其年,遷都督合、霍二州諸軍事、征西大將軍、合州刺史,增邑五百戶。七年,徙都督豫、建、光、朔、合、北徐六州諸軍事、豫州刺史,鎮壽陽,侍中、散騎常侍、持節、將軍、儀同、鼓吹、扶並如故。八年十月,薨,時年五十九。贈侍中、中權大將軍、司空,謚曰威。子玩嗣。



 淳于量,字思明。其先濟北人也,世居京師。父文成,仕梁
 為將帥,官至光烈將軍、梁州刺史。量少善自居處,偉姿容,有幹略,便弓馬。梁元帝為荊州刺史,文成分量人馬,令往事焉。起家湘東王國常侍,兼西中郎府中兵參軍。累遷府佐、常兼中兵、直兵者十餘載,兵甲士卒,盛於府中。



 荊、雍之界,蠻左數反,山帥文道期積為邊患,中兵王僧辯征之,頻戰不利,遣量助之。量至,與僧辯并力,大破道期,斬其酋長,俘虜萬計。以功封廣晉縣男,邑三百戶,授涪陵太守。歷為新興、武寧二郡太守。



 侯景之亂,梁元
 帝凡遣五軍入援京邑,量預其一。臺城陷,量還荊州。元帝承制以量為假節、通直散騎常侍、都督巴州諸軍事、信威將軍、巴州刺史。侯景西上攻巴州,元帝使都督王僧辯入據巴陵。量與僧辯并力拒景,大敗景軍,擒其將任約。



 進攻郢州,獲宋子仙。仍隨僧辯克平侯景。承聖元年,以功授左衛將軍,封謝沐縣侯,邑五百戶。尋出為持節、都督桂、定、東、西寧等四州諸軍事、信威將軍、安遠護軍、桂州刺史。



 荊州陷,量保據桂州。王琳擁割湘、郢,累遣
 召量,量外雖與琳往來,而別遣使從間道歸於高祖。高祖受禪,授持節、散騎常侍、平西大將軍,給鼓吹一部,都督、刺史並如故。尋進號鎮南將軍。仍授都督、鎮西大將軍、開府儀同三司。世祖嗣位,進號征南大將軍。王琳平後,頻請入朝,天嘉五年,徵為中撫大將軍,常侍、儀同、鼓吹並如故。量所部將帥,多戀本土,並欲逃入山谷,不願入朝。世祖使湘州刺史華皎征衡州界黃洞,且以兵迎量。天康元年,至都,以在道淹留,為有司所奏,免儀同,餘
 並如故。光大元年,給鼓吹一部。華皎構逆,以量為使持節、征南大將軍、西討大都督,總率大艦,自郢州樊浦拒之。皎平,并降周將長胡公拓跋定等。以功授侍中、中軍大將軍、開府儀同三司,進封醴陵縣公,增邑一千戶。未拜,出為使持節、都督南徐州諸軍事、鎮北將軍、南徐州刺史,侍中、儀同、鼓吹並如故。



 太建元年,進號征北大將軍,給扶。三年,坐就江陰王蕭季卿買梁陵中樹,季卿坐免,量免侍中。尋復加侍中。五年,徵為中護大將軍,侍中、
 儀同、鼓吹、扶並如故。



 吳明徹之西伐也,量贊成其事,遣第六子岑率所領從軍。淮南克定,量改封始安郡公,增邑一千五百戶。六年,出為使持節、都督郢、巴、南司、定四州諸軍事、征西大將軍、郢州刺史,侍中、儀同、鼓吹、扶並如故。七年,徵為中軍大將軍、護軍將軍。九年,以公事免侍中。尋復加侍中。十年,吳明徹陷沒,加量使持節、都督水陸諸軍事,仍授散騎常侍、都督南北兗、譙三州諸軍事、車騎將軍、南兗州刺史,餘並如故。十三年,加左光祿
 大夫,增邑五百戶,餘並如故。十四年四月薨,時年七十二。贈司空。



 章昭達,字伯通,吳興武康人也。祖道蓋,齊廣平太守。父法尚,梁揚州議曹從事。昭達性倜儻,輕財尚氣。少時,嘗遇相者,謂昭達曰:「卿容貌甚善,須小虧損,則當富貴。」梁大同中,昭達為東宮直後,因醉墜馬,鬢角小傷,昭達喜之,相者曰:「未也。」及侯景之亂,昭達率募鄉人援臺城,為流矢所中,眇其一目,相者見之,曰:「卿相善矣,不久當貴。」



 京城陷,昭達還鄉里,與世祖遊,因結君臣之分。侯景平,世祖為吳興太守,昭達杖策來謁世祖。世祖見之大喜,因委以將帥,恩寵優渥,超於儕等。及高祖討王僧辯,令世祖還長城招聚兵眾,以備杜龕,頻使昭達往京口,稟承計劃。僧辯誅後,龕遣其將杜泰來攻長城,世祖拒之,命昭達總知城內兵事。及杜泰退走,因從世祖東進,軍吳興,以討杜龕。龕平,又從世祖東討張彪於會稽,克之。累功除明威將軍、定州刺史。



 是時留異擁據東陽,私署
 守宰,高祖患之,乃使昭達為長山縣令,居其心腹。



 永定二年,除武康令。世祖嗣位,除員外散騎常侍。天嘉元年,追論長城之功,封欣樂縣侯,邑一千戶。尋隨侯安都等拒王琳于柵口,戰于蕪湖,昭達乘平虜大艦,中流而進,先鋒發拍中于賊艦。王琳平,昭達冊勳第一。二年,除使持節、散騎常侍、都督郢、巴、武沅四州諸軍事、智武將軍、郢州刺史,增邑并前千五百戶。尋進號平西將軍。



 周迪據臨川反,詔令昭達便道征之。及迪敗走,徵為護軍將
 軍,給鼓吹一部,改封邵武縣侯,增邑并前二千戶,常侍如故。四年,陳寶應納周迪,復共寇臨川,又以昭達為都督討迪。至東興嶺,而迪又退走。昭達仍踰嶺,頓于建安,以討陳寶應。寶應據建安、晉安二郡之界,水陸為柵,以拒官軍。昭達與戰不利,因據其上流,命軍士伐木帶枝葉為筏,施拍於其上,綴以大索,相次列營,夾于兩岸。寶應數挑戰,昭達按甲不動。俄而暴雨,江水大長,昭達放筏衝突寶應水柵,水柵盡破。



 又出兵攻其步軍。方大合
 戰,會世祖遣餘孝頃出自海道。適至,因并力乘之,寶應大潰,遂克定閩中,盡擒留異、寶應等。以功授鎮前將軍、開府儀同三司。



 初,世祖嘗夢昭達升於台鉉,及旦,以夢告之。至是侍宴,世祖顧詔達曰:「卿憶夢不?何以償夢?」昭達對曰:「當效犬馬之用,以盡臣節,自餘無以奉償。」



 尋又出為使持節、都督江、郢、吳三州諸軍事、鎮南將軍、江州刺史,常侍、儀同、鼓吹如故。



 廢帝即位,遷侍中、征南將軍,改封邵陵郡公。華皎之反也,其移書文檄,並假以昭達
 為辭,又頻遣使招之,昭達盡執其使,送于京師。皎平,進號征南大將軍,增邑并前二千五百戶。秩滿,徵為中撫大將軍,侍中、儀同、鼓吹如故。高宗即位,進號車騎大將軍,以還朝遲留,為有司所劾,降號車騎將軍。



 歐陽紇據有嶺南反,詔昭達都督眾軍討之。昭達倍道兼行,達于始興。紇聞昭達奄至,恇擾不知所為,乃出頓洭口,多聚沙石,盛以竹籠,置于水柵之外,用遏舟艦。昭達居其上流,裝造拍,以臨賊柵。又令軍人銜刀,潛行水中,以斫
 竹籠,籠篾皆解。因縱大艦隨流突之,賊眾大敗,因而擒紇,送於京師,廣州平。以功進車騎大將軍,遷司空,餘並如故。



 太建二年,率師征蕭巋于江陵。時蕭巋與周軍大蓄舟艦於青泥中,昭達分遣偏將錢道戢、程文季等,乘輕舟襲之,焚其舟艦。周兵又於峽下南岸築壘,名曰安蜀城,於江上橫引大索,編葦為橋,以度軍糧。昭達乃命軍士為長戟,施於樓船之上,仰割其索,索斷糧絕,因縱兵以攻其城,降之。三年,遘疾,薨,時年五十四。贈大將軍,
 增邑五百戶,給班劍二十人。



 昭達性嚴刻,每奉命出征,必晝夜倍道;然有所克捷,必推功將帥,廚膳飲食,並同於群下,將士亦以此附之。每飲會,必盛設女伎雜樂,備盡羌胡之聲,音律姿容,並一時之妙,雖臨對寇敵,旗鼓相望,弗之廢也。四年,配享世祖廟庭。



 子大寶,襲封邵陵郡公,累官至散騎常侍、護軍。出為豊州刺史,在州貪縱,百姓怨酷,後主以太僕卿李暈代之。至德三年四月,暈將到州,大寶乃襲殺暈,舉兵反,遣其將楊通寇建安。建
 安內史吳慧覺據郡城拒之,通累攻不克。官軍稍近,人情離異,大寶計窮,乃與通俱逃。臺軍主陳景詳率兵追躡大寶。大寶既入山,山路陰險,不復能行,通背負之,稍進。尋為追兵所及,生擒送都,於路死,傳首梟于朱雀航,夷三族。



 史臣曰:黃法抃、淳于量值梁末喪亂,劉、項未分,其有辯明暗見是非者蓋鮮,二公達向背之理,位至鼎司,亦其智也。昭達與世祖鄉壤惟舊,義等鄧、蕭,世祖纂歷,委任
 隆重,至於戰勝攻取,累平寇難,斯亦良臣良將,一代之吳、耿矣。






\end{pinyinscope}