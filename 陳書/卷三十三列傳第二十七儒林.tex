\article{卷三十三列傳第二十七儒林}

\begin{pinyinscope}

 沈文阿沈洙戚袞鄭灼張崖陸詡沈德威賀德基全緩張譏顧越沈不害
 王元規蓋今儒者,本因古之六學,斯則王教之典籍,先聖所以明天道,正人倫,致治之成法也。秦始皇焚書坑儒,六學自此缺矣。漢武帝立《五經》博士,置弟子員,設科射策,勸以官祿,其傳業者甚眾焉。自兩漢登賢,咸資經術。魏、晉浮蕩,儒教淪歇,公卿士庶,罕通經業矣。宋、齊之間,國學時復開置。梁武帝開五館,建國學,總以《五經》教授,經各置助教云。武帝或紆鑾駕,臨幸庠序,釋奠先師,躬親試
 胄,申之宴語,勞之束帛,濟濟焉斯蓋一代之盛矣。高祖創業開基,承前代離亂,衣冠殄盡,寇賊未寧,既日不暇給,弗遑勸課。世祖以降,稍置學官,雖博延生徒,成業蓋寡。今之採綴,蓋亦梁之遺儒云。



 沈文阿,字國衛,吳興武康人也。父峻,以儒學聞於梁世,授桂州刺史,不行。



 文阿性剛彊,有膂力,少習父業,研精章句。祖舅太史叔明、舅王慧興並通經術,而文阿頗傳之。又博採先儒異同,自為義疏。治《三禮》、《三傳》。察孝廉,為
 梁臨川王國侍郎,累遷兼國子助教、《五經》博士。



 梁簡文在東宮,引為學士,深相禮遇,及撰《長春義記》,多使文阿撮異聞以廣之。及侯景寇逆,簡文別遣文阿招募士卒,入援京師。城陷,與張乘共保吳興,乘敗,文阿竄于山野。景素聞其名。求之甚急,文阿窮迫不知所出,登樹自縊,遇有所親救之,便自投而下,折其左臂。及景平,高祖以文阿州里,表為原鄉令,監江陰郡。



 紹泰元年,入為國子博士,尋領步兵校尉,兼掌儀禮。自太清之亂,臺閣故事,
 無有存者,文阿父峻,梁武世嘗掌朝儀,頗有遺稿,於是斟酌裁撰,禮度皆自之出。



 及高祖受禪,文阿輒棄官還武康,高祖大怒,發使往誅之。時文阿宗人沈恪為郡,請使者寬其死,即面縛鎖頸致於高祖,高祖視而笑曰:「腐儒復何為者?」遂赦之。



 高祖崩,文阿與尚書左丞徐陵、中書舍人劉師知等議大行皇帝靈座俠御衣服之制,語在師知傳。及世祖即皇帝位,剋日謁廟,尚書右丞庾持奉詔遣博士議其禮。



 文阿議曰:民物推移,質文殊軌,聖
 賢因機而立教,王公隨時以適宜。夫千人無君,不散則亂,萬乘無主,不危則亡。當隆周之日,公旦叔父,呂、召爪牙,成王在喪,禍幾覆國。是以既葬便有公冠之儀,始殯受麻冕之策。斯蓋示天下以有主,慮社稷之艱難。逮乎末葉縱橫,漢承其弊,雖文、景刑厝,而七國連兵。或踰月即尊,或崩日稱詔,此皆有為而為之,非無心於禮制也。今國諱之日,雖抑哀於璽紱之重,猶未序於君臣之儀。古禮,朝廟退坐正寢,聽群臣之政,今皇帝拜廟還,宜御
 太極殿,以正南面之尊,此即周康在朝一二臣衛者也。其壤奠之節,周禮以玉作贄,公侯以珪,子男執璧,此瑞玉也。奠贄既竟,又復致享,天子以璧,王后用琮。秦燒經典,威儀散滅,叔孫通定禮,尤失前憲,奠贄不珪,致享無帛,公王同璧,鴻臚奏賀。



 若此數事,未聞於古,後相沿襲,至梁行之。夫稱觴奉壽,家國大慶,四廂雅樂,歌奏懽欣。今君臣吞哀,萬民抑割,豈同於惟新之禮乎?且周康賓稱奉珪,無萬壽之獻,此則前準明矣。三宿三吒,上宗曰
 饗,斯蓋祭儐受福,寧謂賀酒邪!愚以今坐正殿,止行薦璧之儀,無賀酒之禮。謹撰謁廟還升正寢、群臣陪薦儀注如別。



 詔可施行。尋遷通直散騎常侍,兼國子博士,領羽林監,仍令於東宮講《孝經》、《論語》。天嘉四年卒,時年六十一。詔贈廷尉卿。



 文阿所撰《儀禮》八十餘卷,《經典大義》十八卷,並行於世,諸儒多傳其學。



 沈洙,字弘道,吳興武康人也。祖休稚,梁餘杭令。父山卿,梁國子博士、中散大夫。洙少方雅好學,不妄交遊。治《三
 禮》、《春秋左氏傳》。精識彊記,《五經》章句,諸子史書,問無不答。解巾梁湘東王國左常侍,轉中軍宣城王限內參軍,板仁威臨賀王記室參軍,遷尚書祠部郎中,時年蓋二十餘。大同中,學者多涉獵文史,不為章句,而洙獨積思經術,吳郡朱異、會稽賀琛甚嘉之。及異、琛於士林館講制旨義,常使洙為都講。侯景之亂,洙竄於臨安,時世祖在焉,親就習業。



 及高祖入輔,除國子博士,與沈文阿同掌儀禮。



 高祖受禪,加員外散騎常侍,歷揚州別駕從事
 史、大匠卿。有司奏前寧遠將軍、建康令沈孝軌門生陳三兒牒稱主人翁靈柩在周,主人奉使關內,因欲迎喪,久而未返。此月晦即是再周,主人弟息見在此者,為至月末除靈,內外即吉?為待主人還情禮申竟?以事諮左丞江德藻,德藻議:「王衛軍云:『久喪不葬,唯主人不變,其餘親各終月數而除。』此蓋引《禮》文論在家內有事故未得葬者耳。孝軌既在異域,雖已迎喪,還期無指,諸弟若遂不除,永絕婚嫁,此於人情,或為未允。中原淪陷已後,
 理有事例,宜諮沈常侍詳議。」洙議曰:「禮有變正,又有從宜。《禮小記》云:『久而不葬者,唯主喪者不除,其餘以麻終月數者除喪則已。』《注》云:『其餘謂傍親。』如鄭所解,眾子皆應不除,王衛軍所引,此蓋禮之正也。但魏氏東關之役,既失亡屍柩,葬禮無期,議以為禮無終身之喪,故制使除服。晉氏喪亂,或死於虜庭,無由迎殯,江左故復申明其制。李胤之祖,王華之父,並存亡不測,其子制服依時釋縗,此並變禮之宜也。孝軌雖因奉使便欲迎喪,而戎
 狄難親,還期未剋。愚謂宜依東關故事,在此國內者,並應釋除縗麻,毀靈附祭,若喪柩得還,別行改葬之禮。自天下寇亂,西朝傾覆,流播絕域,情禮莫申,若此之徒,諒非一二,寧可喪期無數,而弗除衰服,朝庭自應為之限制,以義斷恩,通訪博識,折之禮衷。」德藻依洙議,奏可。



 世祖即位,遷通直散騎常侍,侍東宮讀。尋兼尚書左丞,領揚州大中正,遷光祿卿,侍讀如故。廢帝嗣位,重為通直散騎常侍,兼尚書左丞。遷戎昭將軍、輕車衡陽王長史,
 行府國事,帶琅邪、彭城二郡丞。梁代舊律,測囚之法,日一上,起自晡鼓,盡于二更。及比部郎范泉刪定律令,以舊法測立時久,非人所堪,分其刻數,日再上。廷尉以為新制過輕,請集八座丞郎并祭酒孔奐、行事沈洙五舍人會尚書省詳議。時高宗錄尚書,集眾議之,都官尚書周弘正曰:「未知獄所測人,有幾人款?幾人不款?須前責取人名及數并其罪目,然後更集。」得廷尉監沈仲由列稱,別制已後,有壽羽兒一人坐殺壽慧,劉磊渴等八
 人坐偷馬仗家口渡北,依法測之,限訖不款。劉道朔坐犯七改偷,依法測立,首尾二日而款。陳法滿坐被使封藏、阿法受錢,未及上而款。弘正議曰:「凡小大之獄,必應以情,正言依準五聽,驗其虛實,豈可全恣考掠,以判刑罪。且測人時節,本非古制,近代已來,方有此法。



 起自晡鼓,迄于二更,豈是常人所能堪忍?所以重械之下,危墮之上,無人不服,誣枉者多。朝晚二時,同等刻數,進退而求,於事為衷。若謂小促前期,致實罪不伏,如復時節延長,
 則無愆妄款。且人之所堪,既有彊弱,人之立意,固亦多途。



 至如貫高榜笞刺爇,身無完者,戴就熏針並極,困篤不移,豈關時刻長短,掠測優劣?夫與殺不辜,寧失不經,罪疑惟輕,功疑惟重,斯則古之聖王,垂此明法。愚謂依范泉著制,於事為允。」舍人盛權議曰:「比部范泉新制,尚書周弘正明議,咸允《虞書》惟輕之旨,《殷頌》敷正之言。竊尋廷尉監沈仲由等列新制以後,凡有獄十一人,其所測者十人,款者唯一。愚謂染罪之囚,獄官宜明加辯析,
 窮考事理。若罪有可疑,自宜啟審分判,幸無濫測;若罪有實驗,乃可啟審測立;此則枉直有分,刑宥斯理。范泉今牒述《漢律》,云『死罪及除名,罪證明白,考掠已至,而抵隱不服者,處當列上』。杜預注云『處當,證驗明白之狀,列其抵隱之意』。



 竊尋舊制深峻,百中不款者一,新制寬優,十中不款者九,參會兩文,寬猛實異,處當列上,未見釐革。愚謂宜付典法,更詳『處當列上』之文。」洙議曰:「夜中測立,緩急易欺,兼用晝漏,於事為允。但漏刻賒促,今古不
 同,《漢書·律歷》,何承天、祖沖之、釭之父子《漏經》,並自關鼓至下鼓,自晡鼓至關鼓,皆十三刻,冬夏四時不異。若其日有長短,分在中時前後。今用梁末改漏,下鼓之後,分其短長,夏至之日,各十七刻,冬至之日,各十二刻。伏承命旨,刻同勒令,檢一日之刻乃同,而四時之用不等,廷尉今牒,以時刻短促,致罪人不款。愚意願去夜測之昧,從晝漏之明,斟酌今古之間,參會二漏之義,捨秋冬之少刻,從夏日之長晷,不問寒暑,並依今之夏至,朝夕上
 測,各十七刻。比之古漏,則一上多昔四刻,即用今漏,則冬至多五刻。雖冬至之時,數刻侵夜,正是少日,於事非疑。庶罪人不以漏短而為捍,獄囚無以在夜而致誣,求之鄙意,竊謂允合。」眾議以為宜依范泉前制,高宗曰:「沈長史議得中,宜更博議。」左丞宗元饒議曰:「竊尋沈議非頓異范,正是欲使四時均其刻數,兼斟酌其佳,以會優劇。即同牒請寫還刪定曹詳改前制。」高宗依事施行。



 洙以太建元年卒,時年五十二。



 戚袞,字公文,吳郡鹽官人也。祖顯,齊給事中。父霸,梁臨賀王府中兵參軍。



 袞少聰慧,遊學京都,受《三禮》於國子助教劉文紹,一二年中,大義略備。年十九,梁武帝敕策《孔子正言》并《周禮》、《禮記》義,袞對高第。仍除揚州祭酒從事史。



 就國子博士宋懷方質《儀禮》義,懷方北人,自魏攜《儀禮》、《禮記》疏,秘惜不傳,及將亡,謂家人曰:「吾死後,戚生若赴,便以《儀禮》、《禮記》義本付之,若其不來,即宜隨屍而殯。」其為儒者推許如此。尋兼太學博士。



 梁簡文在東宮,
 召袞講論。又嘗置宴集玄儒之士,先命道學互相質難,次令中庶子徐摛馳騁大義,間以劇談。摛辭辯縱橫,難以答抗,諸人懾氣,皆失次序。袞時騁義,摛與往復,袞精采自若,對答如流,簡文深加歎賞。尋除員外散騎侍郎,又遷員外散騎常侍。敬帝承制,出為江州長史,仍隨沈泰鎮南豫州。泰之奔齊也,逼袞俱行,後自鄴下遁還。又隨程文季北伐,呂梁軍敗,袞沒于周,久之得歸。仍兼國子助教,除中衛始興王府錄事參軍。太建十三年
 卒,時年六十三。



 袞於梁代撰《三禮義記》,值亂亡失,《禮記義》四十卷行於世。



 鄭灼,字茂昭,東陽信安人也。祖惠,梁衡陽太守。父季徽,通直散騎侍郎、建安令。灼幼而聰敏,勵志儒學,少受業于皇侃。梁中大通五年,釋褐奉朝請。累遷員外散騎侍郎、給事中、安東臨川王府記室參軍,轉平西邵陵王府記室。簡文在東宮,雅愛經術,引灼為西省義學士。承聖中,除通直散騎侍郎,兼國子博士。尋為威戎將軍,兼中
 書通事舍人。高祖、世祖之世,歷安東臨川、鎮北鄱陽二王府諮議參軍,累遷中散大夫,以本職兼國子博士。未拜,太建十三年卒,時年六十八。



 灼性精勤,尤明《三禮》。少時嘗夢與皇侃遇於途,侃謂灼曰「鄭郎開口」,侃因唾灼口中,自後義理逾進。灼家貧,抄義疏以日繼夜,筆毫盡,每削用之。灼常蔬食,講授多苦心熱,若瓜時,輒偃臥以瓜鎮心,起便誦讀,其篤志如此。



 時有晉陵張崖、吳郡陸詡、吳興沈德威、會稽賀德基,俱以禮學自命。



 張崖傳《三禮》於同郡劉文紹,仕梁歷王府中記室。天嘉元年,為尚書儀曹郎,廣沈文阿《儀注》,撰五禮。出為丹陽令、王府諮議參軍。御史中丞宗元饒表薦為國子博士。



 陸詡少習崔靈恩《三禮義宗》,梁世百濟國表求講禮博士,詔令詡行。還除給事中、定陽令。天嘉初,侍始興王伯茂讀,遷尚書祠部郎中。



 沈德威字懷遠,少有操行。梁太清末,遁於天目山,築室以居,雖處亂離,而篤學無倦,遂治經業。天嘉元年,征出
 都,侍太子講《禮傳》。尋授太學博士,轉國子助教。每自學還私室以講授,道俗受業者數十百人,率常如此。遷太常丞,兼五禮學士,尋為尚書儀曹郎,後為祠部郎。俄丁母憂去職。禎明三年入隋,官至秦王府主簿。年五十五卒。



 賀德基字承業,世傳《禮》學。祖文發,父淹,仕梁俱為祠部郎,並有名當世。



 德基少遊學于京邑,積年不歸,衣資罄乏,又恥服故弊,盛冬止衣裌襦褲。嘗於白馬寺前逢一
 婦人,容服甚盛,呼德基入寺門,脫白綸巾以贈之。仍謂德基曰:「君方為重器,不久貧寒,故以此相遺耳。」德基問嫗姓名,不答而去。德基於《禮記》稱為精明,居以傳授,累遷尚書祠部郎。德基雖不至大官,而三世儒學,俱為祠部,時論美其不墜焉。



 全緩,字弘立,吳郡錢塘人也。幼受《易》于博士褚仲都,篤志研玩,得其精微。梁太清初,歷王國侍郎、奉朝請,俄轉國子助教,兼司義郎,專講《詩》、《易》。紹泰元年,除尚書水部
 郎。太建中,累遷鎮南始興王府諮議參軍,隨府詣湘州,以疾卒,時年七十四。緩治《周易》、《老莊》,時人言玄者咸推之。



 張譏,字直言,清河武城人也。祖僧寶,梁散騎侍郎、太子洗馬。父仲悅,梁廬陵王府錄事參軍、尚書祠部郎中。譏幼聰俊,有思理,年十四,通《孝經》、《論語》。篤好玄言,受學于汝南周弘正,每有新意,為先輩推伏。梁大同中,召補國子《正言》生。梁武帝嘗於文德殿釋《乾》、《坤》文言,譏與陳郡袁憲等預焉,敕令論議,諸儒莫敢先出,譏乃整容而進,
 諮審循環,辭令溫雅。梁武帝甚異之,賜裙襦絹等,仍云「表卿稽古之力」。



 譏幼喪母,有錯綵經帕,即母之遺製,及有所識,家人具以告之,每歲時輒對帕哽噎,不能自勝。及丁父憂,居喪過禮。服闋,召補湘東王國左常侍,轉田曹參軍,遷士林館學士。



 簡文在東宮,出士林館發《孝經》題,譏論議往復,甚見嗟賞,自是每有講集,必遣使召譏。及侯景寇逆,於圍城之中,猶侍哀太子於武德後殿講《老》、《莊》。



 梁臺陷,譏崎嶇避難,卒不事景,景平,歷臨安令。



 高
 祖受禪,除太常丞,轉始興王府刑獄參軍。天嘉中,遷國子助教。是時周弘正在國學,發《周易》題,弘正第四弟弘直亦在講席。譏與弘正論議,弘正乃屈,弘直危坐厲聲,助其申理。譏乃正色謂弘直曰:「今日義集,辯正名理,雖知兄弟急難,四公不得有助。」弘直曰:「僕助君師,何為不可?」舉座以為笑樂。弘正嘗謂人曰:「吾每登座,見張譏在席,使入懍然。」高宗世,歷建安王府記室參軍,兼東宮學士,轉武陵王限內記室,學士如故。



 後主在東宮,集宮僚
 置宴,時造玉柄麈尾新成,後主親執之,曰:「當今雖復多士如林,至於堪捉此者,獨張譏耳。」即手授譏。仍令於溫文殿講《莊》、《老》,高宗幸宮臨聽,賜御所服衣一襲。後主嗣位,領南平王府諮議參軍、東宮學士。尋遷國子博士,學士如故。後主嘗幸鐘山開善寺,召從臣坐於寺西南松林下,敕召譏豎義。時索麈尾未至,後主敕取松枝,手以屬譏,曰「可代麈尾」。顧謂群臣曰「此即是張譏後事」。禎明三年入隋,終於長安,時年七十六。



 譏性恬靜,不求榮利,
 常慕閑逸,所居宅營山池,植花果,講《周易》、《老》、《莊》而教授焉。吳郡陸元朗、朱孟博、一乘寺沙門法才、法雲寺沙門慧休、至真觀道士姚綏,皆傳其業。譏所撰《周易義》三十卷,《尚書義》十五卷,《毛詩義》二十卷,《孝經義》八卷,《論語義》二十卷,《老子義》十一卷,《莊子內篇義》十二卷,《外篇義》二十卷,《雜篇義》十卷,《玄部通義》十二卷,又撰《遊玄桂林》二十四卷,後主嘗敕人就其家寫入秘閣。



 子孝則,官至始安王記室參軍。



 顧越,字思南,吳郡鹽官人也。所居新坡黃岡,世有鄉校,由是顧氏多儒學焉。



 越少孤,以勤苦自立,聰慧有口辯,說《毛氏詩》,傍通異義,梁太子詹事周捨甚賞之。解褐揚州議曹史,兼太子左率丞。越於義理精明,尤善持論,與會稽賀文發俱為梁南平王偉所重,引為賓客。尋補《五經》博士。紹泰元年,遷國子博士。世祖即位,除始興王諮議參軍,侍東宮讀。世祖以越篤老,厚遇之,除給事黃門侍郎,又領國子博士,侍讀如故。廢帝嗣立,除通直
 散騎常侍、中書舍人。華皎之構逆也,越在東陽,或譖之於高宗,言其有異志,詔下獄,因坐免。太建元年卒於家,時年七十八。



 時有東陽龔孟舒者,亦治《毛氏詩》,善談名理。梁武世,仕至尋陽郡丞,元帝在江州,遇之甚重,躬師事焉。承聖中,兼中書舍人。天嘉初,除員外散騎常侍,兼國子助教、太中大夫。太建中卒。



 沈不害,字孝和,吳興武康人也。祖總,齊尚書祠部郎。父懿,梁邵陵王參軍。



 不害幼孤,而脩立好學。十四召補國
 子生,舉明經。累遷梁太學博士。轉廬陵王府刑獄參軍,長沙王府諮議,帶汝南令。天嘉初,除衡陽王府中記室參軍,兼嘉德殿學士。自梁季喪亂,至是國學未立,不害上書曰:臣聞立人建國,莫尚於尊儒,成俗化民,必崇於教學。故東膠西序,事隆乎三代,環林璧水,業盛於兩京。自淳源既遠,澆波已扇,物之感人無窮,人之逐欲無節,是以設訓垂範,啟導心靈,譬彼染藍,類諸琢玉,然後人倫以睦,卑高有序,忠孝之理既明,君臣之道攸固。執禮
 自基,魯公所以難侮,歌樂已細,鄭伯于是前亡,干戚舞而有苗至,泮宮成而淮夷服,長想洙、泗之風,載懷淹、稷之盛,有國有家,莫不尚已。



 梁太清季年,數鐘否剝,戎狄外侵,姦回內[B192],朝聞鼓鼙,夕炤烽火。洪儒碩學,解散甚於坑夷,《五典》、《九丘》,湮滅逾乎帷蓋。成均自斯墜業,瞽宗於是不脩,裒成之祠弗陳稞享,釋菜之禮無稱俎豆,頌聲寂寞,遂踰一紀。後生敦悅,不見函杖之儀,晚學鑽仰,徒深倚席之嘆。



 陛下繼歷升統,握鏡臨宇,道洽寰中,威
 加無外,濁流已清,重氛載廓,含生熙阜,品庶咸亨。宜其弘振禮樂,建立庠序,式稽古典,紆跡儒宮,選公卿門子,皆入于學,助教博士,朝夕講肄,使擔簦負笈,鏘鏘接衽,方領矩步,濟濟成林。



 如切如磋,聞《詩》聞《禮》,一年可以功倍,三冬於是足用。故能擢秀雄州,揚庭觀國,入仕登朝,資優學以自輔,蒞官從政,有經業以治身,轖駕列庭,青紫拾地。



 古者王世子之貴,猶與國子齒,降及漢儲,茲禮不墜,暨乎兩晉,斯事彌隆,所以見師嚴而道尊者也。皇
 太子天縱生知,無待審喻,猶宜晦跡俯同,專經請業,奠爵前師,肅若舊典。昔闕里之堂,草萊自闢,舊宅之內,絲竹流音,前聖遺烈,深以炯戒。況復江表無虞,海外有截,豈得不開闡大猷,恢弘至道?寧可使玄教儒風,弗興聖世,盛德大業,遂蘊堯年?臣末學小生,詞無足算,輕獻瞽言,伏增悚惕。



 詔答曰:「省表聞之。自舊章弛廢,微言將絕,朕嗣膺寶業,念在緝熙,而兵革未息,軍國草創,常恐前王令典,一朝泯滅。卿才思優洽,文理可求,弘惜大體,殷
 勤名教,付外詳議,依事施行。」又表改定樂章,詔使製三朝樂歌八首,合二十八曲,行之樂府。



 五年,除贛令。入為尚書儀曹郎,遷國子博士,領羽林監,敕治五禮,掌策文謚議。太建中,除仁武南康嗣王府長史,行丹陽郡事。轉員外散騎常侍、光祿卿。



 尋為戎昭將軍、明威武陵王長史,行吳興郡事。俄入為通直散騎常侍,兼尚書左丞。



 十二年卒,時年六十三。



 不害治經術,善屬文,雖博綜墳典,而家無卷軸。每製文,操筆立成,曾無尋檢。僕射汝南周
 弘正常稱之曰:「沈生可謂意聖人乎!」著治《五禮儀》一百卷,《文集》十四卷。



 子志道,字崇基,少知名。解褐揚州主簿,尋兼文林著士,歷安東新蔡王記室參軍。禎明三年入隋。



 王元規,字正範,太原晉陽人也。祖道寶,齊員外散騎常侍、晉安郡守。父瑋,梁武陵王府中記室參軍。元規八歲而孤,兄弟三人,隨母依舅氏往臨海郡,時年十二。郡土豪劉瑱者,資財巨萬,以女妻之。元規母以其兄弟幼弱,
 欲結彊援,元規泣請曰:「姻不失親,古人所重。豈得茍安異壤,輒婚非類!」母感其言而止。



 元規性孝,事母甚謹,晨昏未嘗離左右。梁時山陰縣有暴水,流漂居宅,元規唯有一小船,倉卒引其母妹并孤姪入船,元規自執楫棹而去,留其男女三人,閣於樹杪,及水退獲全,時人皆稱其至行。



 元規少好學,從吳興沈文阿受業,十八,通《春秋左氏》、《孝經》、《論語》、《喪服》。梁中大通元年,詔策《春秋》,舉高第,時名儒咸稱賞之。起家湘東王國左常侍,轉員外散騎
 侍郎。簡文之在東宮,引為賓客,每令講論,甚見優禮。除中軍宣城王府記室參軍。及侯景寇亂,攜家屬還會稽。天嘉中,除始興王府功曹參軍,領國子助教,轉鎮東鄱陽王府記室參軍,領助教如故。



 後主在東宮,引為學士,親受《禮記》、《左傳》、《喪服》等義,賞賜優厚。



 遷國子祭酒。新安王伯固嘗因入宮,適會元規將講,乃啟請執經,時論以為榮。俄除尚書祠部郎。自梁代諸儒相傳為《左氏》學者,皆以賈逵、服虔之義難駮杜預,凡一百八十條,元規引
 證通析,無復疑滯。每國家議吉凶大禮,常參預焉。丁母憂去職,服闋,除鄱陽王府中錄事參軍,俄轉散騎侍郎,遷南平王府限內參軍。王為江州,元規隨府之鎮,四方學徒,不遠千里來請道者,常數十百人。禎明三年入隋,為秦王府東閣祭酒。年七十四,卒於廣陵。



 元規著《春秋發題辭》及《義記》十一卷,《續經典大義》十四卷,《孝經義記》兩卷,《左傳音》三卷,《禮記音》兩卷。



 子大業,聰敏知名。



 時有吳郡陸慶,少好學,遍知《五經》,尤明《春秋左氏傳》,節操甚
 高。釋褐梁武陵王國右常侍,歷征西府墨曹行參軍,除婁令。值梁季喪亂,乃覃心釋典,經論靡不該究。天嘉初,徵為通直散騎侍郎,不就。永陽王為吳郡太守,聞其名,欲與相見,慶固辭以疾。時宗人陸榮為郡五官掾,慶嘗詣焉,王乃微服往榮第,穿壁以觀之。王謂榮曰:「觀陸慶風神凝峻,殆不可測,嚴君平、鄭子真何以尚茲。」



 鄱陽、晉安王俱以記室征,並不就。乃築室屏居,以禪誦為事,由是傳經受業者蓋鮮焉。



 史臣曰:夫砥身勵行,必先經術,樹國崇家,率由茲道,故王政因之而至治,人倫得之而攸序。若沈文阿之徒,各專經授業,亦一代之鴻儒焉。文阿加復草創禮儀,蓋叔孫通之流亞矣。






\end{pinyinscope}