\article{卷三十列傳第二十四蕭濟 陸瓊 子從典 顧野王 傅縡 章華}

\begin{pinyinscope}

 蕭濟,字孝康,東海蘭陵人也。少好學,博通經史,諮梁武帝《左氏》疑義三十餘條,尚書僕射范陽張纘、太常卿南
 陽劉之遴並與濟討論,纘等莫能抗對。解褐梁秘書郎,遷太子舍人。預平侯景之功,封松陽縣侯,邑五百戶。



 及高祖作鎮徐方,以濟為明威將軍、征北長史。承聖二年,徵為中書侍郎,轉通直散騎常侍。世祖為會稽太守,又以濟為宣毅府長史,遷司徒左長史。世祖即位,授侍中。尋遷太府卿,丁所生母憂,不拜。濟毘佐二主,恩遇甚篤,賞賜加於凡等。



 歷守蘭陵、陽羨、臨津、臨安等郡,所在皆著聲績。太建初,入為五兵尚書,與左僕射徐陵、特進周
 弘正、度支尚書王瑒、散騎常侍袁憲俱侍東宮。復為司徒長史。



 尋授度支尚書,領羽林監。遷國子祭酒,領羽林如故。加金紫光祿大夫,兼安德宮衛尉。尋遷仁威將軍、揚州長史。高宗嘗敕取揚州曹事,躬自省覽,見濟條理詳悉,文無滯害,乃顧謂左右曰:「我本期蕭長史長於經傳,不言精練繁劇,乃至於此。」



 遷祠部尚書,加給事中,復為金紫光祿大夫。未拜而卒,時年六十六。詔贈本官,官給喪事。



 陸瓊,字伯玉,吳郡吳人也。祖完,梁琅邪、彭城二郡丞。父雲公,梁給事黃門侍郎,掌著作。瓊幼聰惠有思理,六歲為五言詩,頗有詞采。大同末,雲公受梁武帝詔校定《棋品》,到溉、朱異以下並集。瓊時年八歲,於客前覆局,由是京師號曰神童。異言之武帝,有敕召見,瓊風神警亮,進退詳審,帝甚異之。十一,丁父憂,毀瘠有至性,從祖襄歎曰:「此兒必荷門基,所謂一不為少。」及侯景作逆,攜母避地于縣之西鄉,勤苦讀書,晝夜無怠,遂博學,善屬文。



 永
 定中,州舉秀才。天嘉元年,為寧遠始興王府法曹行參軍。尋以本官兼尚書外兵郎,以文學轉兼殿中郎,滿歲為真。瓊素有令名,深為世祖所賞。及討周迪、陳寶應等,都官符及諸大手筆,並中敕付瓊。遷新安王文學,掌東宮管記。及高宗為司徒,妙簡僚佐,吏部尚書徐陵薦瓊於高宗曰:「新安王文學陸瓊,見識優敏,文史足用,進居郎署,歲月過淹,左西掾缺,允膺茲選,階次小踰,其屈滯已積。」



 乃除司徒左西掾。尋兼通直散騎常侍,聘齊。



 太建
 元年,重以本官掌東宮管記。除太子庶子,兼通事舍人。轉中書侍郎、太子家令。長沙王為江州刺史,不循法度,高宗以王年少,授瓊長史,行江州府國事,帶尋陽太守。瓊以母老,不欲遠出,太子亦固請留之,遂不行。累遷給事黃門侍郎,領羽林監。轉太子中庶子,領步兵校尉。又領大著作,撰國史。



 後主即位。直中書省,掌詔誥。俄授散騎常侍,兼度支尚書,領揚州大中正。



 至德元年,除度支尚書,參掌詔誥,并判廷尉、建康二獄事。初,瓊父雲公奉
 梁武帝敕撰《嘉瑞記》,瓊述其旨而續焉,自永定訖于至德,勒成一家之言。遷吏部尚書,著作如故。瓊詳練譜諜,雅鑒人倫,先是,吏部尚書宗元饒卒,右僕射袁憲舉瓊,高宗未之用也,至是居之,號為稱職,後主甚委任焉。



 瓊性謙儉,不自封植,雖位望日隆,而執志愈下。園池室宇,無所改作,車馬衣服,不尚鮮華,四時祿俸,皆散之宗族,家無餘財。暮年深懷止足,思避權要,恒謝病不視事。俄丁母憂,去職。初,瓊之侍東宮也,母隨在官舍,後主賞賜
 優厚。



 及喪柩還鄉,詔加賻贈,并遣謁者黃長貴持冊奠祭,後主又自製誌銘,朝野榮之。



 瓊哀慕過毀,以至德四年卒,時年五十,詔贈領軍將軍,官給喪事。有集二十卷行於世。長子從宜,仕至武昌王文學。



 第三子從典,字由儀。幼而聰敏。八歲,讀沈約集,見回文研銘,從典援筆擬之,便有佳致。年十三,作《柳賦》,其詞其美。瓊時為東宮管記,宮僚並一時俊偉,瓊示以此賦,咸奇其異才。從父瑜特所賞愛,及瑜將終,家中墳籍皆付從典,從典乃集瑜
 文為十卷,仍製集序,其文甚工。



 從典篤好學業,博涉群書,於《班史》尤所屬意。年十五,本州舉秀才。解褐著作佐郎,轉太子舍人。時後主賜僕射江總并其父瓊詩,總命從典為謝啟,俄頃便就,文華理暢,總甚異焉。尋授信義王文學,轉太子洗馬。又遷司徒左西掾,兼東宮學士。丁父憂去職。尋起為德教學士,固辭不就,後主敕留一員,以待從典。俄屬金陵淪沒,隨例遷關右。仕隋為給事郎,兼東宮學士。又除著作佐郎。右僕射楊素奏從典續司
 馬遷《史記》迄于隋,其書未就。值隋末喪亂,寓居南陽郡,以疾卒,時年五十七。



 顧野王,字希馮,吳郡吳人也。祖子喬,梁東中郎武陵王府參軍事。父亙,信威臨賀王記室,兼本郡五官掾,以儒術知名。野王幼好學。七歲,讀《五經》,略知大旨。九歲能屬文,嘗製《日賦》,領軍朱異見而奇之。年十二,隨父之建安,撰《建安地記》二篇。長而遍觀經史,精記嘿識,天文地理、蓍龜占候、蟲篆奇字,無所不通。梁大同四年,除太學博
 士。遷中領軍臨賀王府記室參軍。宣城王為揚州刺史,野王及琅邪王褒並為賓客,王甚愛其才。野王又好丹青,善圖寫,王於東府起齋,乃令野王畫古賢,命王褒書贊,時人稱為二絕。



 及侯景之亂,野王丁父憂,歸本郡,乃召募鄉黨數百人,隨義軍援京邑。野王體素清羸,裁長六尺,又居喪過毀,殆不勝衣,及杖戈被甲,陳君臣之義,逆順之理,抗辭作色,見者莫不壯之。京城陷,野王逃會稽,尋往東陽,與劉歸義合軍據城拒賊。侯景平,太尉王僧辯深
 嘉之,使監海鹽縣。



 高祖作宰,為金威將軍、安東臨川王府記室參軍,尋轉府諮議參軍。天嘉元年,敕補撰史學士,尋加招遠將軍。光大元年,除鎮東鄱陽王諮議參軍。太建二年,遷國子博士。後主在東宮,野王兼東宮管記,本官如故。六年,除太子率更令,尋領大著作,掌國史,知梁史事,兼東宮通事舍人。時宮僚有濟陽江總,吳國陸瓊,北地傅縡,吳興姚察,並以才學顯著,論者推重焉。遷黃門侍郎,光祿卿,知五禮事,餘官並如故。十三年卒,時
 年六十三。詔贈秘書監。至德二年,又贈右衛將軍。



 野王少以篤學至性知名,在物無過辭失色,觀其容貌,似不能言,及其勵精力行,皆人所莫及。第三弟充國早卒,野王撫養孤幼,恩義甚厚。其所撰著《玉篇》三十卷,《輿地志》三十卷,《符瑞圖》十卷,《顧氏譜傳》十卷,《分野樞要》一卷,《續洞冥紀》一卷,《玄象表》一卷,並行於世。又撰《通史要略》一百卷,《國史紀傳》二百卷,未就而卒。有文集二十卷。



 傅縡,字宜事,北地靈州人也。父彝,梁臨沂令。縡幼聰敏,
 七歲誦古詩賦至十餘萬言。長好學,能屬文。梁太清末,攜母南奔避難,俄丁母憂,在兵亂之中,居喪盡禮,哀毀骨立,士友以此稱之。後依湘州刺史蕭循,循頗好士,廣集墳籍,縡肆志尋閱,因博通群書。王琳聞其名,引為府記室。琳敗,隨琳將孫瑒還都。時世祖使顏晃賜瑒雜物,瑒託縡啟謝,詞理優洽,文無加點,晃還言之世祖,尋召為撰史學士。除司空府記室參軍,遷驃騎安成王中記室,撰史如故。



 縡篤信佛教,從興皇惠朗法師受《三論》,盡
 通其學。時有大心暠法師著《無諍論》以詆之,縡乃為《明道論》,用釋其難。其略曰:《無諍論》言:比有弘《三論》者,雷同訶詆,恣言罪狀,歷毀諸師,非斥眾學,論中道而執偏心,語忘懷而競獨勝,方學數論,更為仇敵,仇敵既構,諍斗大生,以此之心,而成罪業,罪業不止,豈不重增生死,大苦聚集?答曰:《三論》之興,為日久矣。龍樹創其源,除內學之偏見,提婆揚其旨,蕩外道之邪執。欲使大化流而不擁,玄風闡而無墜。其言曠,其意遠,其道博,其流深。斯固
 龍象之騰驤,鯤鵬之摶運。蹇乘決羽,豈能觖望其間哉?頃代澆薄,時無曠士,茍習小學,以化蒙心,漸染成俗,遂迷正路,唯競穿鑿,各肆營造,枝葉徒繁,本源日翳,一師解釋,復異一師,更改舊宗,各立新意,同學之中,取寤復別,如是展轉,添糅倍多。總而用之,心無的準;擇而行之,何者為正?豈不渾沌傷竅,嘉樹弊牙?雖復人說非馬,家握靈蛇,以無當之卮,同畫地之餅矣。其於失道,不亦宜乎?攝山之學,則不如是。守一遵本,無改作之過;約文申
 意,杜臆斷之情。言無預說,理非宿構。睹緣爾乃應,見敵然後動。縱橫絡驛,忽恍杳冥。或彌綸而不窮。或消散而無所。煥乎有文章,蹤朕不可得;深乎不可量,即事而非遠。凡相酬對,隨理詳核。有何嫉詐,干犯諸師?且諸師所說,為是可毀?為不可毀?若可毀者,毀故為衰;若不可毀,毀自不及。法師何獨蔽護不聽毀乎?且教有大小,備在聖誥,大乘之文,則指斥小道。今弘大法,寧得不言大乘之意耶?斯則褒貶之事,從弘放學;與奪之辭,依經議論。
 何得見佛說而信順,在我語而忤逆?無諍平等心如是耶?且忿恚煩惱,凡夫恆性,失理之徒,率皆有此。豈可以三修未愜,六師懷恨,而蘊涅槃妙法,永不宣揚?但冀其忿憤之心既極,恬淡之寤自成耳。人面不同,其心亦異,或有辭意相反,或有心口相符。豈得必謂他人說中道而心偏執,己行無諍,外不違而內平等?仇敵鬥訟,豈我事焉;罪業聚集,鬥諍者所畏耳。



 《無諍論》言:攝山大師誘進化導,則不如此,即習行於無諍者也。導悟之德既往,
 淳一之風已澆,競勝之心,阿毀之曲,盛於茲矣。吾願息諍以通道,讓勝以忘德。何必排拂異家,生其恚怒者乎?若以中道之心行於《成實》,亦能不諍;若以偏著之心說於《中論》,亦得有諍。固知諍與不諍,偏在一法。答曰:攝山大師實無諍矣,但法師所賞,未衷其節。彼靜守幽谷,寂爾無為,凡有訓勉,莫匪同志,從容語嘿,物無間然,故其意雖深,其言甚約。今之敷暢,地勢不然。處王城之隅,居聚落之內,呼吸顧望之客,脣吻縱橫之士,奮鋒穎,勵羽
 翼,明目張膽,被堅執銳,聘異家,衒別解,窺伺間隙,邀冀長短,與相酬對,捔其輕重,豈得默默無言,唯唯應命?必須掎摭同異,發擿玼瑕,忘身而弘道,忤俗而通教,以此為病,益知未達。若令大師當此之地,亦何必默己,而為法師所貴耶?法師又言:「吾願息諍以通道,讓勝以忘德。」道德之事,不止在諍與不諍,讓與不讓也。此語直是人間所重,法師慕而言之,竟未知勝若為可讓也。若他人道高,則自勝不勞讓矣;他人道劣,則雖讓而無益矣。欲
 讓之辭,將非虛設?中道之心,無處不可。《成實三論》,何事致乖?但須息守株之解,除膠柱之意,是事皆中也。來旨言「諍與不諍,偏在一法」。何為獨褒無諍耶?詎非矛盾?



 《無諍論》言:邪正得失,勝負是非,必生於心矣,非謂所說之法,而有定相論勝劣也。若異論是非,以偏著為失言,無是無非,消彼得失,以此論為勝妙者,他論所不及,此亦為失也。何者?凡心所破,豈無心於能破,則勝負之心不忘,寧不存勝者乎?斯則矜我為得,棄他之失,即有取舍,
 大生是非,便是增諍。答曰:言為心使,心受言詮;和合根塵,鼓動風氣,故成語也。事必由心,實如來說。至於心造偽以使口,口行詐以應心,外和而內險,言隨而意逆,求利養,引聲名,入道之人,在家之士,斯輩非一。聖人所以曲陳教誡,深致防杜,說見在之殃咎,敘將來之患害,此文明著,甚於日月,猶有忘愛軀,冒峻制,蹈湯炭,甘齏粉,必行而不顧也。豈能悅無諍之作,而回首革音耶?若弘道之人,宣化之士,心知勝也,口言勝也,心知劣也,口言
 劣也,亦無所苞藏,亦無所忌禪,但直心而行之耳。他道雖劣,聖人之教也;己德雖優,亦聖人之教也。我勝則聖人勝,他劣則聖人劣。



 聖人之優劣,蓋根緣所宜爾。於彼於此,何所厚薄哉?雖復終日按劍,極夜擊柝,瞋目以爭得失,作氣以求勝負,在誰處乎?有心之與無心,徒欲分別虛空耳。何意不許我論說,而使我謙退?此謂鷦褷已翔於寥廓,而虞者猶窺藪澤而求之。嗟乎!



 丈夫當弘斯道矣。



 《無諍論》言:無諍之道,通於內外。子所言須諍者,此
 用末而救本,失本而營末者也。今為子言之。何則?若依外典,尋書契之前,至淳之世,朴質其心,行不言之教,當于此時,民至老死不相往來,而各得其所,復有何諍乎?固知本末不諍,是物之真矣。答曰:諍與無諍,不可偏執。本之與末,又安可知?由來不諍,寧知非末?於今而諍,何驗非本?夫居后而望前,則為前;居前而望後,則為後。



 而前後之事猶如彼此,彼呼此為彼,此呼彼為彼,彼此之名,的居誰處?以此言之,萬事可知矣。本末前後,是非善
 惡,可恆守邪?何得自信聰明,廢他耳目?夫水泡生滅,火輪旋轉,入牢阱,受羈紲,生憂畏,起煩惱,其失何哉?不與道相應,而起諸見故也。相應者則不然,無為也,無不為也。善惡不能偕,而未曾離善惡,生死不能至,亦終然在生死,故得永離而任放焉。是以聖人念繞桎之不脫,愍黏膠之難離,故殷勤教示,備諸便巧。希向之徒,涉求有類,雖驎角難成,象形易失,寧得不仿佛遐路,勉勵短晨?且當念己身之善惡,莫揣他物,而欲分別,而言我聰明,
 我知見,我計校,我思惟,以此而言,亦為疏矣。他人者實難測,或可是凡夫真爾,亦可是聖人俯同,時俗所宜見,果報所應睹。安得肆胸衿,盡情性,而生譏誚乎?



 正應虛己而遊乎世,俯仰於電露之間耳。明月在天,眾水咸見,清風至林,群籟畢響。吾豈逆物哉?不入鮑魚,不甘腐鼠。吾豈同物哉?誰能知我,共行斯路,浩浩乎!堂堂乎!豈復見有諍為非,無諍為是?此則諍者自諍,無諍者自無諍,吾俱取而用之。寧勞法師費功夫,點筆紙,但申於無諍;弟
 子疲脣舌,消晷漏,唯對於明道?戲論哉!糟粕哉!必欲且考真偽,蹔觀得失,無過依賢聖之言,檢行藏之理,始終研究,表裏綜覈,使浮辭無所用,詐道自然消。請待後筵,以觀其妙矣。



 尋以本官兼通直散騎侍郎使齊,還除散騎侍郎、鎮南始興王諮議參軍,兼東宮管記。歷太子庶子、僕,兼管記如故。後主即位,遷秘書監、右衛將軍,兼中書通事舍人,掌詔誥。



 縡為文典麗,性又敏速,雖軍國大事,下筆輒成,未嘗起草,沉思者亦無以加焉,甚為後主
 所重。然性木彊,不持檢操,負才使氣,陵侮人物,朝士多銜之。會施文慶、沈客卿以便佞親幸,專制衡軸,而縡益疏。文慶等因共譖縡受高麗使金,後主收縡下獄。縡素剛,因憤恚,乃於獄中上書曰:「夫君人者,恭事上帝,子愛下民,省嗜欲,遠諂佞,未明求衣,日旰忘食,是以澤被區宇,慶流子孫。陛下頃來酒色過度,不虔郊廟之神,專媚淫昏之鬼;小人在側,宦豎弄權,惡忠直若仇讎,視生民如草芥;後宮曳綺繡,廄馬餘菽粟,百姓流離,僵尸蔽野;
 貨賄公行,帑藏損耗,神怒民怨,眾叛親離。恐東南王氣,自斯而盡。」書奏,後主大怒。頃之,意稍解,遣使謂縡曰:「我欲赦卿,卿能改過不?」縡對曰:「臣心如面,臣面可改,則臣心可改。」後主於是益怒,令宦者李善慶窮治其事,遂賜死獄中,時年五十五。有集十卷行於世。



 時有吳興章華,字仲宗,家世農夫,至華獨好學,與士君子遊處,頗覽經史,善屬文。侯景之亂,乃遊嶺南,居羅浮山寺,專精習業。歐陽頠為廣州刺史,署為南海太守。及歐陽紇敗,乃還
 京師。太建中,高宗使吏部侍郎蕭引喻廣州刺史馬靖,令入子為質,引奏華與俱行。使還,而高宗崩。後主即位,朝臣以華素無伐閱,競排詆之,乃除大市令,既雅非所好,乃辭以疾,鬱鬱不得志。禎明初,上書極諫,其大略曰:「昔高祖南平百越,北誅逆虜;世祖東定吳會,西破王琳;高宗克復淮南,辟地千里:三祖之功,亦至勤矣。陛下即位,于今五年,不思先帝之艱難,不知天命之可畏,溺於嬖寵,惑於酒色,祠七廟而不出,拜妃嬪而臨軒,老臣宿
 將,棄之草莽,諂佞讒邪,昇之朝廷。今疆埸日蹙,隋軍壓境,陛下如不改弦易張,臣見麋鹿復遊於姑蘇臺矣。」書奏,後主大怒,即日命斬之。



 史臣曰:蕭濟、陸瓊,俱以才學顯著,顧野王博極群典,傅縡聰警特達,並一代之英靈矣。然縡不能循道進退,遂置極網,悲夫!



\end{pinyinscope}