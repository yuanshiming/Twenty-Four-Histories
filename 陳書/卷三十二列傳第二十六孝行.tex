\article{卷三十二列傳第二十六孝行}

\begin{pinyinscope}

 殷不害弟不佞謝貞司馬皓張昭孔子曰:「夫聖人之德,何以加於孝乎!」孝者百行之本,人
 倫之至極也。凡在性靈,孰不由此。若乃奉生盡養,送終盡哀,或泣血三年,絕漿七日,思《蓼莪》之慕切,追顧復之恩深,或德感乾坤,誠貫幽顯,在於歷代,蓋有人矣。陳承梁室喪亂,風漓化薄,及迹隱閻閭,無聞視聽,今之採綴,以備闕云。



 殷不害,字長卿,陳郡長平人也。祖任,齊豫章王行參軍。父高明,梁尚書中兵郎。不害性至孝,居父憂過禮,由是少知名。家世儉約,居甚貧窶,有弟五人,皆幼弱,不害事
 老母,養小弟,勤劇無所不至,士大夫以篤行稱之。



 年十七,仕梁廷尉平。不害長於政事,兼飾以儒術,名法有輕重不便者,輒上書言之,多見納用。大同五年,遷鎮西府記室參軍,尋以本官兼東宮通事舍人。是時朝廷政事多委東宮,不害與舍人庾肩吾直日奏事,梁武帝嘗謂肩吾曰:「卿是文學之士,吏事非卿所長,何不使殷不害來邪?」其見知如此。簡文又以不害善事親,賜其母蔡氏錦裙襦、氈席被褥,單復畢備。七年,除東宮步兵校尉。太
 清初,遷平北府諮議參軍,舍人如故。



 侯景之亂,不害從簡文入臺。及臺城陷,簡文在中書省,景帶甲將兵入朝陛見,過謁簡文。景兵士皆羌、胡雜種,衝突左右,甚不遜,侍衛者莫不驚恐辟易,唯不害與中庶子徐摛侍側不動。及簡文為景所幽,遣人請不害與居處,景許之,不害供侍益謹。簡文夜夢吞一塊土,意甚不悅,以告不害,不害曰:「昔晉文公出奔,野人遺之塊,卒反晉國,陛下此夢,事符是乎?」簡文曰:「若天有徵,冀斯言不妄。」



 梁元帝立,以
 不害為中書郎,兼廷尉卿,因將家屬西上。江陵之陷也,不害先於別所督戰,失母所在。于時甚寒,冰雪交下,老弱凍死者填滿溝塹。不害行哭道路,遠近尋求,無所不至,遇見死人溝水中,即投身而下,扶捧閱視,舉體凍濕,水漿不入口,號泣不輟聲,如是者七日,始得母屍。不害憑屍而哭,每舉音輒氣絕,行路無不為之流涕。即於江陵權殯,與王裒、庾信俱入長安,自是蔬食布衣,枯槁骨立,見者莫不哀之。



 太建七年,自周還朝,其年詔除司農
 卿,尋遷光祿大夫。八年,加明威將軍、晉陵太守。在郡感疾,詔以光祿大夫征還養疾。後主即位,加給事中。初,不害之還也,周留其長子僧首,因居關中。禎明三年,京城陷,僧首來迎,不害道病卒,時年八十五。



 不佞字季卿,不害弟也。少立名節,居父喪以至孝稱。好讀書,尤長吏術,仕梁,起家為尚書中兵郎,甚有能稱。梁元帝承制,授戎昭將軍、武陵王諮議參軍。



 承聖初,遷武康令。時兵荒饑饉,百姓流移,不佞巡撫招集,繈負而至
 者以千數。



 會江陵陷,而母卒,道路隔絕,久不得奔赴,四載之中,晝夜號泣,居處飲食,常為居喪之禮。高祖受禪,起為戎昭將軍,除婁令。至是,第四兄不齊始之江陵,迎母喪柩歸葬。不佞居處之節,如始聞問,若此者又三年。身自負土,手植松柏,每歲時伏臘,必三日不食。



 世祖即位,除尚書左民郎,不就。後為始興王諮議參軍,兼尚書右丞,遷東宮通事舍人。及世祖崩,廢帝嗣立,高宗為太傅,錄尚書輔政,甚為朝望所歸。不佞素以名節自立,又
 受委東宮,乃與僕射到仲舉、中書舍人劉師知、尚書右丞王暹等,謀矯詔出高宗。眾人猶豫,未敢先發,不佞乃馳詣相府,面宣敕,令相王還第。及事發,仲舉等皆伏誅,高宗雅重不佞,特赦之,免其官而已。



 高宗即位,以為軍師始興王諮議參軍,加招遠將軍。尋除大匠卿,未拜,加員外散騎常侍,又兼尚書右丞。俄遷通直散騎常侍,右丞如故。太建五年卒,時年五十六。詔贈秘書監。



 第三兄不疑,次不占,次不齊,並早亡。不佞最小,事第二寡嫂張
 氏甚謹,所得祿俸,不入私室。長子梵童,官至尚書金部郎。



 謝貞,字元正,陳郡陽夏人,晉太傅安九世孫也。祖綏,梁著作佐郎、太子舍人。父藺,正員外郎,兼散騎常侍。貞幼聰敏,有至性。祖母阮氏先苦風眩,每發便一二日不能飲食,貞時年七歲,祖母不食,貞亦不食,往往如是,親族莫不奇之。



 母王氏,授貞《論語》、《孝經》,讀訖便誦。八歲,嘗為《春日閒居》五言詩,從舅尚書王筠奇其有佳致,謂所親
 曰:「此兒方可大成,至如『風定花猶落』,乃追步惠連矣。」由是名輩知之。年十三,略通《五經》大旨。尤善《左氏傳》,工草隸蟲篆。十四,丁父艱,號頓於地,絕而復蘇者數矣。初,父藺居母阮氏憂,不食泣血而卒,家人賓客懼貞復然,從父洽、族兄皓乃共往華嚴寺,請長爪禪師為貞說法,仍謂貞曰:「孝子既無兄弟,極須自愛,若憂毀滅性,誰養母邪?」自後少進饘粥。



 太清之亂,親屬散亡,貞於江陵陷沒,皓逃難番禺,貞母出家於宣明寺。及高祖受禪,皓還鄉
 里,供養貞母,將二十年。太建五年,貞乃還朝,除智武府外兵參軍事。俄遷尚書駕部郎中,尋遷侍郎。及始興王叔陵為揚州刺史,引祠部侍郎阮卓為記室,辟貞為主簿,貞不得已乃行。尋遷府錄事參軍,領丹陽丞。貞度叔陵將有異志,因與卓自疏於王,每有宴遊,輒辭以疾,未嘗參預,叔陵雅欽重之,弗之罪也。俄而高宗崩,叔陵肆逆,府僚多相連逮,唯貞與卓獨不坐。



 後主仍詔貞入掌中宮管記,遷南平王友,加招遠將軍,掌記室事。府長史
 汝南周確新除都官尚書,請貞為讓表,後主覽而奇之。嘗因宴席問確曰:「卿表自製邪?」



 確對曰:「臣表謝貞所作。」後主因敕舍人施文慶曰:「謝貞在王處,未有祿秩,可賜米百石。」至德三年,以母憂去職。頃之,敕起還府,仍加招遠將軍,掌記室。



 貞累啟固辭,敕報曰:「省啟具懷,雖知哀煢在疚,而官俟得才,禮有權奪,可便力疾還府也。」貞哀毀羸瘠,終不能之官舍。時尚書右丞徐祚、尚書左丞沈客卿俱來候貞,見其形體骨立,祚等愴然歎息,徐喻之
 曰:「弟年事已衰,禮有恒制,小宜引割自全。」貞因更感慟,氣絕良久,二人涕泣,不能自勝,憫默而出。祚謂客卿曰:「信哉,孝門有孝子。」客卿曰:「謝公家傳至孝,士大夫誰不仰止,此恐不能起,如何?」吏部尚書吳興姚察與貞友善,及貞病篤,察往省之,問以後事,貞曰:「孤子颭禍所集,將隨灰壤。族子凱等粗自成立,已有疏付之,此固不足仰塵厚德。即日迷喘,時不可移,便為永訣。弱兒年甫六歲,名靖,字依仁,情累所不能忘,敢以為託耳。」是夜卒,敕賻
 米一百斛,布三十匹。後主問察曰:「謝貞有何親屬?」察因啟曰:「貞有一子年六歲。」即有敕長給衣糧。



 初,貞之病亟也,遺疏告族子凱曰:「吾少罹酷罰,十四傾外蔭,十六鐘太清之禍,流離絕國,二十餘載。號天蹐地,遂同有感,得還侍奉,守先人墳墓,於吾之分足矣。不悟朝廷採拾空薄,累致清階,縱其殞絕,無所酬報。今在憂棘,晷漏將盡,斂手而歸,何所多念。氣絕之後,若直棄之草野,依僧家屍陀林法,是吾所願,正恐過為獨異耳。可用薄板周身,
 載以靈車,覆以葦席,坎山而埋之。又吾終鮮兄弟,無他子孫,靖年幼少,未閑人事,但可三月施小床,設香水,盡卿兄弟相厚之情,即除之,無益之事,勿為也。」



 初,貞在周嘗侍趙王讀,王即周武帝之愛弟也,厚相禮遇。王嘗聞左右說貞每獨處必晝夜涕泣,因私使訪問,知貞母年老,遠在江南,乃謂貞曰:「寡人若出居籓,當遣侍讀還家供養。」後數年,王果出,因辭見,面奏曰:「謝貞至孝而母老,臣願放還。」帝奇王仁愛而遣之,因隨聘使杜子暉還國。
 所有文集,值兵亂多不存。



 司馬皓,字文昇,河內溫人也。高祖晉侍中、光祿勳柔之,以南頓王孫紹齊文獻王攸之後。父子產,梁尚書水部侍郎、後陽太守,即梁武帝之外兄也。



 皓幼聰警,有至性。年十二,丁內艱,孺慕過禮,水漿不入口,殆經一旬。每至號慟,必致悶絕,內外親戚,皆懼其不勝喪。父子產每曉喻之,逼進饘粥,然毀瘠骨立。服闋,以姻戚子弟,預入問訊,梁武帝見皓羸瘦,歎息良久,謂其父子產曰:「昨見羅
 兒面顏憔悴,使人惻然,便是不墜家風,為有子矣。」羅兒,即皓小字也。釋褐太學博士,累遷正員郎。丁父艱,哀毀逾甚,廬于墓側,一日之內,唯進薄麥粥一升。墓在新林,連接山阜,舊多猛獸,皓結廬數載,豺狼絕迹。常有兩鳩棲宿廬所,馴狎異常,新林至今猶傳之。



 承聖中,除太子庶子。江陵陷,隨例入關,而梁室屠戮,太子瘞殯失所,皓以宮臣,乃抗表周朝,求還江陵改葬,辭甚酸切。周朝優詔答曰:「昔主父從戮,孔車有長者之風,彭越就誅,欒布
 得陪臣之禮。庶子鄉國已改,猶懷送往之情,始驗忠貞,方知臣道,即敕荊州,以禮安厝。」



 太建八年,自周還朝,高宗特降殊禮,賞錫有加。除宜都王諮議參軍事,徙安德宮長秋卿、通直散騎常侍、太中大夫、司州大中正,卒于官。有集十卷。



 子延義,字希忠,少沈敏好學。江陵之陷,隨父入關。丁母憂,喪過于禮。及皓還都,延義乃躬負靈櫬,晝伏宵行,冒履冰霜,手足皆皸瘃。及至都,以中風冷,遂致攣廢,數年方愈。稍遷鄱陽王錄事參軍、沅陵王友、司
 徒從事中郎。



 張昭,字德明,吳郡吳人也。幼有孝性,色養甚謹,禮無違者。父,常患消渴,嗜鮮魚,昭乃身自結網捕魚,以供朝夕。弟乾,字玄明,聰敏博學,亦有至性。



 及父卒,兄弟並不衣綿帛,不食鹽醋,日唯食一升麥屑粥而已。每一感慟,必致嘔血,鄰里聞其哭聲,皆為之涕泣。父服未終,母陸氏又亡,兄弟遂六年哀毀,形容骨立,親友見者莫識焉。家貧,未得大葬,遂布衣蔬食,十有餘年,杜門不出,屏絕
 人事。時衡陽王伯信臨郡,舉乾孝廉,固辭不就。兄弟並因毀成疾,昭失一眼,乾亦中冷苦癖,年並未五十終於家,子胤俱絕。



 高宗世有太原王知玄者,僑居於會稽剡縣,居家以孝聞。及丁父憂,哀毀而卒,高宗嘉之,詔改其所居清苦里為孝家裏雲。



 史臣曰:人倫之德,莫大於孝,是以報本反始,盡性窮神,孝乎惟孝,不可不勖矣。故《記》云「塞乎天地」,盛哉!



\end{pinyinscope}