\article{卷十列傳第四周鐵虎 程靈洗 子文季}

\begin{pinyinscope}

 周鐵虎,不知何許人也,梁世南渡。語音傖重,膂力過人,便馬槊,事梁河東王蕭譽,以勇敢聞,譽板為府中兵參軍。譽為廣州刺史,以鐵虎為興寧令。譽遷湘州,又為臨
 蒸令。侯景之亂,元帝於荊州遣世子方等代譽,且以兵臨之。譽拒戰,大捷,方等死,鐵虎功最,譽委遇甚重。及王僧辯討譽,於陣獲鐵虎,僧辯命烹之,鐵虎呼曰:「侯景未滅,柰何殺壯士!」僧辯奇其言,乃宥之,還其麾下。



 及侯景西上,鐵虎從僧辯克任約,獲宋子仙,每戰皆有功。元帝承制授仁威將軍、潼州刺史,封沌陽縣子,邑三百戶。又從僧辯克定京邑,降謝答仁,平陸納於湘州。承聖二年,以前後戰功,進爵為侯,增邑并前五百戶。仍為散騎常
 侍,領信義太守,將軍如故。高祖誅僧辯,鐵虎率所部降,因復其本職。



 徐嗣徽引齊寇渡江,鐵虎於板橋浦破其水軍,盡獲甲仗船舸。又攻歷陽,襲齊寇步營,並皆克捷。嗣徽平,紹泰二年,遷散騎常侍、嚴威將軍、太子左衛率。



 尋隨周文育於南江拒蕭勃,恒為前軍。文育又命鐵虎偏軍,於苦竹灘襲勃前軍歐陽頠。又隨文育西征王琳,於沌口敗績,鐵虎與文育、侯安都並為琳所擒。琳引見諸將,與之語,唯鐵虎辭氣不屈,故琳盡宥文育之徒,獨
 鐵虎見害,時年四十九。



 高祖聞之,下詔曰:「天地之寶,所貴曰生,形魄之徒,所重唯命。至如捐生立節,效命酬恩,追遠懷昔,信宜加等。散騎常侍、嚴威將軍、太子左衛率、潼州刺史、領信義太守沌陽縣開國侯鐵虎,器局沈厚,風力勇壯,北討南征,竭忠盡力。推鋒江夏,致陷凶徒,神氣彌雄,肆言無撓。豈直溫序見害,方其理鬚,龐德臨危,猶能瞋目。忠貞如此,惻愴兼深,可贈侍中、護軍將軍、青、冀二州刺史,加封一千戶,并給鼓吹一部,侯如故。」天嘉
 五年,世祖又詔曰:「漢室功臣,形寫宮觀,魏朝猛將,名配宗祧,功烈所以長存,世代因之不朽。故侍中、護軍將軍、青、冀二州刺史沌陽縣開國侯鐵虎,誠節梗亮,力用雄敢,王業初基,行間累及,垂翅賊壘,正色寇庭,古之遺烈,有識同壯。隕身不屈,雖隆榮等,營魂易遠,言追嘉惜。



 宜仰陪需寢,恭頒饗奠,可配食高祖廟庭。」子瑜嗣。



 時有盱眙馬明,字世朗,梁世事鄱陽嗣王蕭範。侯景之亂,據廬江之東界,拒賊臨城柵。元帝授散騎常侍、平北將軍、北
 兗州刺史,領廬江太守。荊州陷沒,歸于高祖。紹泰中,復官位,封西華縣侯,邑二千戶。亦隨文育西征王琳,於沌口軍敗,明力戰死之,贈使持節、征西將軍、郢州刺史。



 程靈洗,字玄滌,新安海寧人也。少以勇力聞,步行日二百餘里,便騎善游。



 梁末,海寧、黟、歙等縣及鄱陽、宣城郡界多盜賊,近縣苦之。靈洗素為鄉里所畏伏,前後守長恒使召募少年,逐捕劫盜。



 侯景之亂,靈洗聚徒據黟、歙以拒景。景軍據有新安,新安太守湘西鄉侯蕭隱奔依
 靈洗,靈洗奉以主盟。梁元帝於荊州承制,又遣使間道奉表。劉神茂自東陽建義拒賊,靈洗攻下新安,與神茂相應。元帝授持節、通直散騎常侍、都督新安郡諸軍事、雲麾將軍、譙州刺史資,領新安太守,封巴丘縣侯,邑五百戶。神茂為景所破,景偏帥呂子榮進攻新安,靈洗退保黟、歙。及景敗,子榮退走,靈洗復據新安。



 進軍建德,擒賊帥趙桑乾。以功授持節、散騎常侍、都督青、冀二州諸軍事、青州刺史,增邑并前一千戶,將軍、太守如故。



 仍令
 靈洗率所部下揚州,助王僧辯鎮防。遷吳興太守,未行,僧辯命靈洗從侯瑱西援荊州。荊州陷,還都。高祖誅僧辯,靈洗率所領來援,其徒力戰於石頭西門,軍不利,遣使招諭,久之乃降,高祖深義之。紹泰元年,授使持節、信武將軍、蘭陵太守,常侍如故,助防京口。及平徐嗣徽,靈洗有功,除南丹陽太守,封遂安縣侯,增邑并前一千五百戶,仍鎮采石。



 隨周文育西討王琳,於沌口敗績,為琳所拘。明年,與侯安都等逃歸。兼丹陽尹,出為高唐、太原
 二郡太守,仍鎮南陵。遷太子左衛率。高祖崩,王琳前軍東下,靈洗於南陵破之,虜其兵士,并獲青龍十餘乘。以功授持節、都督南豫州緣江諸軍事、信武將軍、南豫州刺史。侯瑱等敗王琳于柵口,靈洗乘勝逐北,據有魯山。徵為左衛將軍,餘如故。



 天嘉四年,周迪重寇臨川,以靈洗為都督,自鄱陽別道擊之,迪又走山谷間。



 五年,遷中護軍,常侍如故。出為使持節、都督郢、巴、武三州諸軍事、宣毅將軍、郢州刺史。廢帝即位,進號雲麾將軍。



 華皎之
 反也,遣使招誘靈洗,靈洗斬皎使,以狀聞。朝廷深嘉其忠,增其守備,給鼓吹一部,因推心待之,使其子文季領水軍助防。是時周遣其將長胡公拓跋定率步騎二萬助皎攻圍靈洗,靈洗嬰城固守。及皎退,乃出軍躡定,定不獲濟江,以其眾降。因進攻周沔州,克之,擒其刺史裴寬。以功進號安西將軍,改封重安縣公,增邑并前二千戶。



 靈洗性嚴急,御下甚苛刻,士卒有小罪,必以軍法誅之,造次之間,便加捶撻,而號令分明,與士卒同甘苦,眾
 亦以此依附。性好播植,躬勤耕稼,至於水陸所宜,刈獲早晚,雖老農不能及也。伎妾無游手,並督之紡績。至於散用貲財,亦弗儉吝。



 光大二年,卒於州,時年五十五。贈鎮西將軍、開府儀同三司,謚曰忠壯。太建四年,詔配享高祖廟庭。子文季嗣。



 文季字少卿。幼習騎射,多幹略,果決有父風。弱冠從靈洗征討,必前登陷陣。



 靈洗與周文育、侯安都等敗於沌口,為王琳所執,高祖召陷賊諸將子弟厚遇之,文季最
 有禮容,深為高祖所賞。永定中,累遷通直散騎侍郎、句容令。世祖嗣位,除宣惠始興王府限內中直兵參軍。是時王為揚州刺史,鎮冶城,府中軍事,悉以委之。



 天嘉二年,除貞毅將軍、新安太守,仍隨侯安都東討留異。異黨向文政據有新安,文季率精甲三百,輕往攻之。文政遣其兄子瓚來拒,文季與戰,大破瓚軍,文政乃降。



 三年,始興王伯茂出鎮東州,復以文季為鎮東府中兵參軍,帶剡令。



 四年,陳寶應與留異連結,又遣兵隨周迪更出臨
 川,世祖遣信義太守餘孝頃自海道襲晉安,文季為之前軍,所向克捷。陳寶應平,文季戰功居多,還,轉府諮議參軍,領中直兵。出為臨海太守。尋乘金翅助父鎮郢城。華皎平,靈洗及文季並有扞禦之功。及靈洗卒,文季盡領其眾,起為超武將軍,仍助防郢州。文季性至孝,雖軍旅奪禮,而毀瘠甚至。



 太建二年,為豫章內史,將軍如故。服闋,襲封重安縣公。隨都督章昭達率軍往荊州征蕭巋。巋與周軍多造舟艦,置于青泥水中。時水長漂疾,昭
 達乃遣文季共錢道戢輕舟襲之,盡焚其舟艦。昭達因蕭巋等兵稍怠,又遣文季夜入其外城,殺傷甚眾。既而周兵大出,巴陵內史雷道勤拒戰死之,文季僅以身免。以功加通直散騎常侍、安遠將軍,增邑五百戶。



 五年,都督吳明徹北討秦郡,秦郡前江浦通塗水,齊人並下大柱為杙,柵水中,乃前遣文季領驍勇拔開其柵,明徹率大軍自後而至,攻秦郡克之。又別遣文季圍涇州,屠其城,進攻盱眙,拔之。仍隨明徹圍壽陽。



 文季臨事謹急,御
 下嚴整,前後所克城壘,率皆迮水為堰,土木之功,動踰數萬。每置陣役人,文季必先諸將,夜則早起,迄暮不休,軍中莫不服其勤幹。每戰恆為前鋒,齊軍深憚之,謂為程虎。以功除散騎常侍、明威將軍,增邑五百戶。又帶新安內史,進號武毅將軍。



 八年,為持節、都督譙州諸軍事、安遠將軍、譙州刺史。其年,又督北徐仁州諸軍事、北徐州刺史,餘並如故。九年,又隨明徹北討,於呂梁作堰,事見明徹傳。



 十年春,敗績,為周所囚,仍授開府儀同三司。
 十一年,自周逃歸,至渦陽,為邊吏所執,還送長安,死于獄中。後主是時既與周絕,不之知也。至德元年,後主始知之,追贈散騎常侍。尋又詔曰:「故散騎常侍、前重安縣開國公文季,纂承門緒,克荷家聲。早歲出軍,雖非元帥,啟行為最,致果有聞,而覆喪車徒,允從黜削。



 但靈洗之立功捍禦,久而見思,文季之埋魂異域,有足可憫。言念勞舊,傷茲廢絕,宜存廟食,無使餒而。可降封重安縣侯,邑一千戶,以子饗襲封。」



 史臣曰:程靈洗父子並御下嚴苛,治兵整肅,然與眾同其勞苦,匪私財利,士多依焉,故臨戎克辦矣。



\end{pinyinscope}