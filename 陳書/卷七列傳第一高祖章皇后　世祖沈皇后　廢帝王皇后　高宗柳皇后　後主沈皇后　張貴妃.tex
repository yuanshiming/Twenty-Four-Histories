\article{卷七列傳第一高祖章皇后 世祖沈皇后 廢帝王皇后 高宗柳皇后 後主沈皇后 張貴妃}

\begin{pinyinscope}

 周禮,王者
 立后,六宮,
 三夫人,九嬪,二十七世婦,八十一御妻,以聽天下之內治。然受命繼體之主,非獨外相佐也,蓋亦有內德助焉。漢魏已來,六宮之職,因襲增置,代不同矣。高祖承微接亂,光膺天歷,以朴素自處,故後宮員位多闕。



 世祖天嘉初,詔立後宮員數,始置貴妃、貴嬪、貴姬三人,以擬古之
 三
 夫人。
 又置淑媛、淑儀、淑容、昭華、昭容、昭儀、脩華、脩儀、脩容九人,以擬古之九嬪。



 又置婕妤、容華、充華、承徽、列榮五人,謂之五職,亞於九嬪。又置美人、才人、良人三職,其職無員數,號為散位。世祖性恭
 儉,而嬪嬙多闕,高宗、後主內職無所改作。今之所綴,略備此篇。



 高祖宣皇后章氏,諱要兒,吳興烏程人也。本姓鈕,父景明為章氏所養,因改焉。景明,梁代官至散騎侍郎。后母蘇,嘗遇道士以小龜遺己,光采五色,曰:「三年有徵。」及期后生,而紫光照室,因失龜所在。少聰慧,美容儀,手爪長五寸,色並紅白,每有期功之服,則一爪先折。高祖先娶同郡錢仲方女,早卒,後乃聘后。后善書計,能誦《詩》及《楚
 辭》。



 高祖自廣州南征交止,命后與衡陽王昌隨世祖由海道歸于長城。侯景之亂,高祖下至豫章,后為景所囚。景平,而高祖為長城縣公,后拜夫人。及高祖踐祚,永定元年立為皇后。追贈后父景明特進、金紫光祿大夫,加金章紫綬,拜后母蘇安吉縣君。二年,安吉君卒,與后父合葬吳興。明年,追封后父為廣德縣侯,邑五百戶,謚曰溫。高祖崩,后與中書舍人蔡景歷定計,秘不發喪,召世祖入纂,事在蔡景歷及侯安都傳。世祖即位,尊后為皇
 太后,宮曰慈訓。廢帝即位,尊后為太皇太后。光大二年,后下令黜廢帝為臨海王,命高宗嗣位。太建元年,尊后為皇太后。



 二年三月丙申,崩于紫極殿,時年六十五。遺令喪事所須,並從儉約,諸有饋奠,不得用牲牢。其年四月,群臣上謚曰宣太后,祔葬萬安陵。



 后親屬無在朝者,唯族兄鈕洽官至中散大夫。



 世祖沈皇后,諱妙容,吳興武康人也。父法深,梁安前中錄事參軍。后年十餘歲,以梁大同中歸于世祖。高祖之
 討侯景,世祖時在吳興,景遣使收世祖及后。景平,乃獲免。高祖踐祚,永定元年,后為臨川王妃。世祖即位,為皇后。追贈后父法深光祿大夫,加金章紫綬,封建城縣侯,邑五百戶,謚曰恭,追贈后母高綏安縣君,謚曰定。廢帝即位,尊后為皇太后,宮曰安德。



 時高宗與僕射到仲舉、舍人劉師知等並受遺輔政,師知與仲舉恒居禁中參決眾事,而高宗為揚州刺史,與左右三百人入居尚書省。師知見高宗權重,陰忌之,乃矯敕謂高宗曰:「今四方
 無事,王可還東府,經理州務。」高宗將出,而諮議毛喜止之曰:「今若出外,便受制於人,譬如曹爽,願作富家翁不可得也。」高宗乃稱疾,召師知留之與語,使毛喜先入言之於后。后曰:「今伯宗年幼,政事並委二郎,此非我意。」喜又言於廢帝,帝曰:「此自師知等所為,非朕意也。」喜出以報高宗,高宗因囚師知,自入見后及帝,極陳師知之短,仍自草敕請畫,以師知付廷尉治罪。其夜,於獄中賜死。自是政無大小,盡歸高宗。后憂悶,計無所出,乃密賂宦
 者蔣裕,令誘建安人張安國,使據郡反,冀因此以圖高宗。安國事覺,並為高宗所誅。時后左右近侍頗知其事,后恐連逮黨與,並殺之。高宗即位,以后為文皇后。



 陳亡入隋,大業初,自長安歸於江南,頃之,卒。



 后兄欽,隨世祖征伐,以功至貞威將軍、安州刺史。世祖即位,襲爵建城侯,加通直散騎常侍、持節、會稽等九郡諸軍事、明威將軍、會稽太守,入為侍中、左衛將軍、衛尉卿。光大中,為尚書右僕射,尋遷左僕射。欽素無技能,奉己而已。



 高宗即
 位,出為雲麾將軍、義興太守,秩中二千石。太建元年卒,時年六十七,贈侍中、特進、翊左將軍,謚曰成。子觀嗣,頗有學識,官至御史中丞。



 廢帝王皇后,金紫光祿大夫固之女也。天嘉元年,為皇太子妃,廢帝即位,立為皇后。廢帝為臨海王,后為臨海王妃。至德中薨。



 后生臨海嗣王至澤。至澤以光大元年為皇太子。太建元年,襲封臨海嗣王。尋為宣惠將軍,置佐史。陳亡入長安。



 高宗柳皇后,諱敬言,河東解人也。曾祖世隆,齊侍中、司空、尚書令、貞陽忠武公。祖惲,有重名於梁代,官至秘書監,贈侍中、中護軍。父偃,尚梁武帝女長城公主,拜駙馬都尉,大寶中,為鄱陽太守,卒官。后時年九歲,幹理家事,有若成人。侯景之亂,后與弟盼往江陵依梁元帝,元帝以長城公主之故,待遇甚厚。



 及高宗赴江陵,元帝以后配焉。承聖二年,后生後主於江陵。明年,江陵陷,高宗遷于關右,后與後主俱留穰城。天嘉二年,與後主還朝,后
 為安成王妃。高宗即位,立為皇后。



 后美姿容,身長七尺二寸,手垂過膝。初,高宗居鄉里,先娶吳興錢氏女,及即位,拜為貴妃,甚有寵,后傾心下之。每尚方供奉之物,其上者皆推於貴妃,而己御其次焉。高宗崩,始興王叔陵為亂,後主賴后與樂安君吳氏救而獲免,事在叔陵傳。後主即位,尊后為皇太后,宮曰弘範。當是之時,新失淮南之地,隋師臨江,又國遭大喪,後主病瘡,不能聽政,其誅叔陵、供大行喪事、邊境防守及百司眾務,雖假以後
 主之命,實皆決之於后。後主瘡愈,乃歸政焉。陳亡入長安,大業十一年薨於東都,年八十三,葬洛陽之邙山。



 后性謙謹,未嘗以宗族為請,雖衣食亦無所分遺。



 弟盼,太建中尚世祖女富陽公主,拜駙馬都尉。後主即位,以帝舅加散騎常侍。



 盼性愚戇,使酒,常因醉乘馬入殿門,為有司所劾,坐免官,卒於家。贈侍中、中護軍。



 后從祖弟莊,清警有鑒識,太建末,為太子洗馬,掌東宮管記。後主即位,稍遷至散騎常侍、衛尉卿。禎明元年,轉右衛將軍,兼
 中書舍人,領雍州大中正。自盼卒後,太后宗屬唯莊為近,兼素有名望,猶是深被恩遇。尋遷度支尚書。陳亡入隋,為岐州司馬。



 後主沈皇后,諱婺華,儀同三司望蔡貞憲侯君理女也。母即高祖女會稽穆公主。



 主早亡,時后尚幼,而毀瘠過甚。及服畢,每至歲時朔望,恒獨坐涕泣,哀動左右,內外咸敬異焉。太建三年,納為皇太子妃。後主即位,立為皇后。



 后性端靜,寡嗜慾,聰敏彊記,涉獵經史,工書翰。初,後
 主在東宮,而后父君理卒,后居憂,處於別殿,哀毀逾禮。後主遇后既薄,而張貴妃寵傾後宮,後宮之政並歸之,后澹然未嘗有所忌怨。而居處儉約,衣服無錦繡之飾,左右近侍纔百許人,唯尋閱圖史、誦佛經為事。陳亡,與後主俱入長安。及後主薨,后自為哀辭,文甚酸切。隋煬帝每所巡幸,恆令從駕。及煬帝為宇文化及所害,后自廣陵過江還鄉里,不知所終。



 后無子,養孫姬子胤為己子。后宗族多有顯官,事在君理傳。



 后叔君公,自梁元帝
 敗後,常在江陵。禎明中,與蕭獻、蕭巖率眾叛隋歸朝,後主擢為太子詹事。君公博學有才辯,善談論,後主深器之。陳亡,隋文帝以其叛己,命斬于建康。



 後主張貴妃,名麗華,兵家女也。家貧,父兄以織席為事。後主為太子,以選入宮。是時龔貴嬪為良娣,貴妃年十歲,為之給使,後主見而說焉,因得幸,遂有娠,生太子深。後主即位,拜為貴妃。性聰惠,甚被寵遇。後主每引貴妃與賓客遊宴,貴妃薦諸宮女預焉,後宮等咸德之,兢言
 貴妃之善,由是愛傾後宮。又好厭魅之術,假鬼道以惑後主,置淫祀於宮中,聚諸妖巫使之鼓舞。因參訪外事,人間有一言一事,妃必先知之,以白後主。由是益重妃,內外宗族,多被引用。及隋軍陷臺城,妃與後主俱入於井,隋軍出之,晉王廣命斬貴妃,榜於青溪中橋。



 史臣侍中鄭國公魏徵考覽記書,參詳故老,云:後主初即位,以始興王叔陵之亂,被傷臥于承香閣下,時諸姬並不得進,唯張貴妃侍焉。而柳太后猶居柏梁殿,即皇
 后之正殿也。後主沈皇后素無寵,不得侍疾,別居求賢殿。至德二年,乃於光照殿前起臨春、結綺、望仙三閣。閣高數丈,並數十間,其窗牖、壁帶、懸楣、欄檻之類,並以沈檀香木為之,又飾以金玉,間以珠翠,外施珠廉,內有寶床、寶帳、其服玩之屬,瑰奇珍麗,近古所未有。每微風暫至,香聞數里,朝日初照,光映後庭。其下積石為山,引水為池,植以奇樹,雜以花藥。後主自居臨春閣,張貴妃居結綺閣,龔、孔二貴嬪居望仙閣,並復道交相往來。又有
 王、李二美人、張、薛二淑媛、袁昭儀、何婕妤、江脩容等七人,並有寵,遞代以遊其上。以宮人有文學者袁大捨等為女學士。後主每引賓客對貴妃等遊宴,則使諸貴人及女學士與狎客共賦新詩,互相贈答,採其尤艷麗者以為曲詞,被以新聲,選宮女有容色者以千百數,令習而歌之,分部迭進,持以相樂。其曲有《玉樹後庭花》、《臨春樂》等,大指所歸,皆美張貴妃、孔貴嬪之容色也。其略曰:「璧月夜夜滿,瓊樹朝朝新。」而張貴妃髮長七尺,鬒黑如
 漆,其光可鑒。特聰惠,有神采,進止閑暇,容色端麗。



 每瞻視盼睞,光采溢目,照映左右。常於閣上靚妝,臨於軒檻,宮中遙望,飄若神仙。才辯彊記,善候人主顏色。是時後主怠於政事,百司啟奏,並因宦者蔡脫兒、李善度進請,後主置張貴妃於膝上共決之。李、蔡所不能記者,貴妃並為條疏,無所遺脫。由是益加寵異,冠絕後庭。而後宮之家,不遵法度,有挂於理者,但求哀於貴妃,貴妃則令李、蔡先啟其事,而後從容為言之。大臣有不從者,亦因
 而譖之,所言無不聽。於是張、孔之勢,薰灼四方,大臣執政,亦從風而靡。閹宦便佞之徒,內外交結,轉相引進,賄賂公行,賞罰無常,綱紀瞀亂矣。



 史臣曰:《詩》表《關雎》之德,《易》著《乾坤》之基,然夫婦之際,人道之大倫也。若夫作儷天則,燮贊王化,則宣太后有其懿焉。



\end{pinyinscope}