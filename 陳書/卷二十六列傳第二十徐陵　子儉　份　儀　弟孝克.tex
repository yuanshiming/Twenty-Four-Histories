\article{卷二十六列傳第二十徐陵 子儉 份 儀 弟孝克}

\begin{pinyinscope}

 徐陵,字孝穆,東海郯人也。祖超之,齊鬱林太守,梁員外散騎常侍。父摛,梁戎昭將軍、太子左衛率,贈侍中、太子詹事,謚貞子。母臧氏,嘗夢五色雲化而為鳳,集左肩上,
 已而誕陵焉。時寶誌上人者,世稱其有道,陵年數歲,家人攜以候之,寶誌手摩其頂,曰:「天上石麒麟也。」光宅惠雲法師每嗟陵早成就,謂之顏回。八歲能屬文,十二通《莊》、《老》義。既長,博涉史籍,縱橫有口辯。



 梁普通二年,晉安王為平西將軍、寧蠻校尉,父摛為王咨議,王又引陵參寧蠻府軍事。中大通三年,王立為皇太子,東宮置學士,陵充其選。稍遷尚書度支郎。



 出為上虞令,御史中丞劉孝儀與陵先有隙,風聞劾陵在縣贓汙,因坐免。久之,起為
 南平王府行參軍,遷通直散騎侍郎。梁簡文在東宮撰《長春殿義記》,使陵為序。



 又令於少傅府述所製《莊子義》。尋遷鎮西湘東王中記室參軍。



 太清二年,兼通直散騎常侍。使魏,魏人授館宴賓。是日甚熱,其主客魏收嘲陵曰:「今日之熱,當由徐常侍來。」陵即答曰:「昔王肅至此,為魏始制禮儀;今我來聘,使卿復知寒暑。」收大慚。



 及侯景寇京師,陵父摛先在圍城之內,陵不奉家信,便蔬食布衣,若居憂恤。



 會齊受魏禪,梁元帝承制於江陵,復通使
 於齊。陵累求復命,終拘留不遣,陵乃致書於僕射楊遵彥曰:夫一言所感,凝暉照於魯陽,一志冥通,飛泉涌於疏勒,況復元首康哉,股肱良哉,鄰國相聞,風教相期者也!天道窮剝,鐘亂本朝,情計馳惶,公私哽懼,而骸骨之請,徒淹歲寒,顛沛之祈,空盈卷軸,是所不圖也,非所仰望也。



 執事不聞之乎:昔分鰲命勣之世,觀河拜洛之年,則有日烏流災,風禽騁暴,天傾西北,地缺東南,盛旱坼三川,長波含五嶽。我大梁應金圖而有亢,纂玉鏡而猶
 屯。何則?聖人不能為時,斯固窮通之恒理也。至如荊州刺史湘東王,機神之本,無寄名言,陶鑄之餘,猶為堯、舜,雖復六代之舞,陳於總章,九州之歌,登於司樂,虞夔拊石,晉曠調鐘,未足頌此英聲,無以宣其盛德者也。若使郊禋楚翼,寧非祀夏之君,戡定艱難,便是匡周之霸,豈徒豳王徙雍,期月為都,姚帝遷河,周年成邑。方今越常藐藐,馴雉北飛,肅紵茫茫,風牛南偃,吾君之子,含識知歸,而答旨云何所投身,斯其未喻一也。



 又晉熙等郡,
 皆入貴朝,去我尋陽,經途何幾。至於鐺鐺曉漏,的的宵烽,隔漵浦而相聞,臨高臺而可望。泉流寶碗,遙憶湓城,峰號香爐,依然廬嶽。日者鄱陽嗣王治兵匯派,屯戍淪波,朝夕箋書,春秋方物,吾無從以躡屩,彼何路而齊鑣。



 豈其然乎?斯不然矣。又近者邵陵王通和此國,郢中上客,雲聚魏都,鄴下名卿,風馳江浦,豈盧龍之徑於彼新開,銅駝之街於我長閉?何彼途甚易,非勞於五丁,我路為難,如登於九折?地不私載,何其爽歟?而答旨云還路無
 從,斯所未喻二也。



 晉熙、廬江,義陽、安陸,皆云款附,非復危邦,計彼中途,便當靜晏,自斯以北,桴鼓不鳴,自此以南,封疆未壹。如其境外,脫殞輕軀,幸非邊吏之羞,何在匹夫之命。又此賓遊,通無貨殖,忝非韓起聘鄭,私買玉環,吳札過徐,躬要寶劍。由來宴錫,凡厥囊裝,行役淹留,皆已虛罄,散有限之微財,供無期之久客,斯可知矣。且據圖刎首,愚者不為,運斧全身,庸流所鑒。何則?生輕一髮,自重千鈞,不以賈盜明矣。骨肉不任充鼎俎,皮毛不
 足入貨財,盜有道焉,吾無憂矣。



 又公家遣使,脫有資須,本朝非隆平之時,遊客豈皇華之勢。輕裝獨宿,非勞聚橐之儀,微騎閑行,寧望輶軒之禮。歸人將從,私具驢騾,緣道亭郵,唯希蔬粟。若曰留之無煩於執事,遣之有費於官司,或以顛沛為言,或云資裝可懼,固非通論,皆是外篇。斯所未喻三也。



 又若以吾徒應還侯景,侯景凶逆,殲我國家,天下含靈,人懷憤厲,既不獲投身社稷,衛難乘輿,四冢磔蚩尤,千刀剸王莽,安所謂俯首頓膝,歸奉
 寇仇,佩弭腰鞬,為其皁隸?日者通和,方敦曩睦,凶人狙詐,遂駭狼心,頗疑宋萬之誅,彌懼荀幹之請,所以奔蹄勁角,專恣憑陵,凡我行人,偏膺仇憾。政復俎筋醢骨,抽舌探肝,於彼凶情,猶當未雪,海內之所知也,君侯之所具焉。又聞本朝公主,都人士女,風行雨散,東播西流,京邑丘墟,姦蓬蕭瑟,偃師還望,咸為草萊,霸陵回首,俱霑霜露,此又君之所知也。彼以何義,爭免寇仇?我以何親,爭歸委質?



 昔鉅平貴將,懸重於陸公,叔向名流,深知於
 鬷篾。吾雖不敏,常慕前修,不圖明庶有懷,翻其以此量物。昔魏氏將亡,群凶挺爭,諸賢戮力,想得其朋。為葛榮之黨邪?為邢杲之徒邪?如曰不然,斯所未喻四也。



 假使吾徒還為凶黨,侯景生於趙代,家自幽恒,居則台司,行為連率,山川形勢,軍國彝章,不勞請箸為籌,便當屈指能算。景以逋逃小醜,羊豕同群,身寓江皋,家留河朔,舂舂井井,如鬼如神。其不然乎?抑又君之所知也。且夫宮闈秘事,並若雲霄,英俊訏謨,寧非帷幄,或陽驚以定策,
 或焚槁而奏書,朝廷之士,猶難參預,羈旅之人,何階耳目。至於禮樂沿革,刑政寬猛,則謳歌已遠,萬舞成風,不知手之舞之足之蹈之也。安在搖其牙齒,為間諜者哉?若謂復命西朝,終奔東虜,雖齊、梁有隔,尉候奚殊?豈以河曲之難浮,而曰江關之可濟?河橋馬度,寧非宋典之姦?關路雞鳴,皆曰田文之客。何其通蔽,乃爾相妨?斯所未喻五也。



 又兵交使在,雖著前經,儻同徇僕之尤,追肆寒山之怒,則凡諸元帥,並釋縲囚,爰及偏裨,同無翦馘。
 乃至鐘儀見赦,朋笑遵途,襄老蒙歸,《虞歌》引路。



 吾等張拭玉,修好尋盟,涉泗之與浮河,郊勞至於贈賄,公恩既被,賓敬無違,今者何愆,翻蒙貶責?若以此為言,斯所未喻六也。



 若曰妖氛永久,喪亂悠然,哀我奔波,存其形魄,固已銘茲厚德,戴此洪恩,譬渤澥而俱深,方嵩、華而猶重。但山梁飲啄,非有意於籠樊,江海飛浮,本無情於鐘鼓。況吾等營魂已謝,餘息空留,悲默為生,何能支久,是則雖蒙養護,更夭天年。若以此為言,斯所未喻七也。



 若云逆豎殲夷,當聽反命,高軒繼路,飛蓋相隨,未解其言,何能善謔?夫屯亨治亂,豈有意於前期。謝常侍今年五十有一,吾今年四十有四,介已知命,賓又杖鄉,計彼侯生,肩隨而已。豈銀臺之要,彼未從師,金灶之方,吾知其決,政恐南陽菊水,竟不延齡,東海桑田,無由可望。若以此為言,斯所未喻八也。



 足下清襟勝托,書囿文林,凡自洪荒,終乎幽、厲,如吾今日,寧有其人,爰至《春秋》,微宜商略。夫宗姬殄墜,霸道昏凶,或執政之多門,或陪臣之
 涼德,故臧孫有禮,翻囚與國之賓,周伯無愆,空怒天王之使,遷箕卿於兩館,縶驥子於三年。斯匪貪亂之風邪?寧當今之高例也?至於雙崤且帝,四海爭雄,或構趙而侵燕,或連韓而謀魏,身求盟於楚殿,躬奪璧於秦庭,輸寶鼎以託齊王,馳安車而誘梁客。其外膏脣販舌,分路揚鑣,無罪無辜,如兄如弟。逮乎中陽受命,天下同規,巡省諸華,無聞幽辱。及三方之霸也,孫甘言以娬媚,曹屈詐以羈縻,旍軫歲到於句吳,冠蓋年馳於庸蜀,則客嘲
 殊險,賓戲已深,共盡遊談,誰云猜忤,若使搜求故實,脫有前蹤,恐是叔世之姦謀,而非為邦之勝略也。



 抑又聞之,雲師火帝,澆淳乃異其風,龍躍麟驚,王霸雖殊其道,莫不崇君親以銘物,敦敬養以治民,預有邦司,曾無隆替。吾奉違溫清,仍屬亂離,寇虜猖狂,公私播越。蕭軒靡御,王舫誰持?瞻望鄉關,何心天地?自非生憑廩竹,源出空桑,行路含情,猶其相愍。常謂擇官而仕,非曰孝家,擇事而趨,非云忠國。況乎欽承有道,驂駕前王,郎吏明經,
 鴟鳶知禮,巡省方化,咸問高年,東序西膠,皆尊耆耋。吾以圭璋玉帛,通聘來朝,屬世道之屯期,鐘生民之否運,兼年累載,無申元直之祈,銜泣吞聲,長對公閭之怒,情禮之訴,將同逆鱗,忠孝之言,皆應齚舌,是所不圖也,非所仰望也。



 且天倫之愛,何得忘懷?妻子之情,誰能無累?夫以清河公主之貴,餘姚書佐之家,莫限高卑,皆被驅略。自東南醜虜,抄販饑民,臺署郎官,俱餒墻壁,況吾生離死別,多歷暄寒,孀室嬰兒,何可言念。如得身還鄉土,
 躬自推求,猶冀提攜,俱免凶虐。



 夫四聰不達,華陽君所謂亂臣,百姓無冤,孫叔敖稱為良相。足下高才重譽,參贊經綸,非豹非貔,聞《詩》聞《禮》,而中朝大議,曾未矜論,清禁嘉謀,安能相及,諤諤非周舍,容容類胡廣,何其無諍臣哉?歲月如流,平生何幾,晨看旅鴈,心赴江、淮,昏望牽牛,情馳揚、越,朝千悲而掩泣,夜萬緒而回腸,不自知其為生,不自知其為死也。足下素挺詞鋒,兼長理窟,匡丞相解頤之說,樂令君清耳之談,向所咨疑,誰能曉喻。若
 鄙言為謬,來旨必通,分請灰釘,甘從斧鑊,何但規規默默,齰舌低頭而已哉。若一理存焉,猶希矜眷,何必期令我等必死齊都,足趙、魏之黃塵,加幽、并之片骨,遂使東平拱樹,長懷向漢之悲,西洛孤墳,恒表思鄉之夢。乾祈以屢,哽慟增深。



 遵彥竟不報書。及江陵陷,齊送貞陽侯蕭淵明為梁嗣,乃遣陵隨還。太尉王僧辯初拒境不納,淵明往復致書,皆陵詞也。及淵明之入,僧辯得陵大喜,接待饋遺,其禮甚優。以陵為尚書吏部郎,掌詔誥。其年
 高祖率兵誅僧辯,仍進討韋載。時任約、徐嗣徽乘虛襲石頭,陵感僧辯舊恩,乃往赴約。及約等平,高祖釋陵不問。尋以為貞威將軍、尚書左丞。



 紹泰二年,又使于齊,還除給事黃門侍郎、秘書監。高祖受禪,加散騎常侍,左丞如故。天嘉初,除太府卿。四年,遷五兵尚書,領大著作。六年,除散騎常侍、御史中丞。時安成王頊為司空,以帝弟之尊,勢傾朝野。直兵鮑僧叡假王威權,抑塞辭訟,大臣莫敢言者。陵聞之,乃為奏彈,導從南臺官屬,引奏案而
 入。世祖見陵服章嚴肅,若不可犯,為斂容正坐。陵進讀奏版時,安成王殿上侍立,仰視世祖,流汗失色。陵遣殿中御史引王下殿,遂劾免侍中、中書監。自此朝廷肅然。



 天康元年,遷吏部尚書,領大著作。陵以梁末以來,選授多失其所,於是提舉綱維,綜覈名實。時有冒進求官,喧競不已者,陵乃為書宣示曰:「自古吏部尚書者,品藻人倫,簡其才能,尋其門胄,逐其大小,量其官爵。梁元帝承侯景之凶荒,王太尉接荊州之禍敗,爾時喪亂,無復典
 章,故使官方,窮此紛雜。永定之時,聖朝草創,干戈未息,亦無條序。府庫空虛,賞賜懸乏,白銀難得,黃札易營,權以官階,代於錢絹,義存撫接,無計多少,致令員外、常侍,路上比肩,咨議、參軍,市中無數,豈是朝章,應其如此?今衣冠禮樂,日富年華,何可猶作舊意,非理望也。所見諸君,多踰本分,猶言大屈,未喻高懷。若問梁朝朱領軍異亦為卿相,此不踰其本分邪?此是天子所拔,非關選序。梁武帝云『世間人言有目色,我特不目色范悌』。宋文帝
 亦云『人世豈無運命,每有好官缺,輒憶羊玄保。』此則清階顯職,不由選也。秦有車府令趙高直至丞相,漢有高廟令田千秋亦為丞相,此復可為例邪?既忝衡流,應須粉墨。所望諸賢,深明鄙意。」自是眾咸服焉。時論比之毛玠。



 廢帝即位,高宗入輔,謀黜異志者,引陵預其議。高宗纂歷,封建昌縣侯,邑五百戶。太建元年,除尚書右僕射。三年,遷尚書左僕射,陵抗表推周弘正、王勱等,高宗召陵入內殿,曰:「卿何為固辭此職而舉人乎?」陵曰:「周弘正
 從陛下西還,舊籓長史,王勱太平相府長史,張種帝鄉賢戚,若選賢與舊,臣宜居後。」



 固辭累日,高宗苦屬之,陵乃奉詔。



 及朝議北伐,高宗曰:「朕意已決,卿可舉元帥。」眾議咸以中權將軍淳于量位重,共署推之。陵獨曰:「不然。吳明徹家在淮左,悉彼風俗,將略人才,當今亦無過者。」於是爭論累日不能決。都官尚書裴忌曰:「臣同徐僕射。」陵應聲曰:「非但明徹良將,裴忌即良副也。」是日,詔明徹為大都督,令忌監軍事,遂克淮南數十州之地。高宗因
 置酒,舉杯屬陵曰:「賞卿知人。」陵避席對曰:「定策出自聖衷,非臣之力也。」其年加侍中,餘並如故。七年,領國子祭酒、南徐州大中正。以公事免侍中、僕射。尋加侍中,給扶,又除領軍將軍。八年,加翊右將軍、太子詹事,置佐史。俄遷右光祿大夫,餘並如故。十年,重為領軍將軍。尋遷安右將軍、丹陽尹。十三年,為中書監,領太子詹事,給鼓吹一部,侍中、將軍、右光祿、中正如故。陵以年老累表求致仕,高宗亦優禮之,乃詔將作為造大齋,令陵就第攝事。



 後
 主即位,遷左光祿大夫、太子少傅,餘如故。至德元年卒,時年七十七。詔曰:「慎終有典,抑乃舊章,令德可甄,諒宜追遠。侍中、安右將軍、左光祿大夫、太子少傅、南徐州大中正建昌縣開國侯陵,弱齡學尚,登朝秀穎,業高名輩,文曰詞宗。朕近歲承華,特相引狎,雖多臥疾,方期克壯,奄然殞逝,震悼于懷。可贈鎮右將軍、特進,其侍中、左光祿、鼓吹、侯如故,并出舉哀,喪事所須,量加資給。謚曰章。」



 陵器局深遠,容止可觀,性又清簡,無所營樹,祿俸與親
 族共之。太建中,食建昌邑,邑戶送米至於水次,陵親戚有貧匱者,皆令取之,數日便盡,陵家尋致乏絕。府僚怪而問其故,陵云:「我有車牛衣裳可賣,餘家有可賣不?」其周給如此。



 少而崇信釋教,經論多所精解。後主在東宮,令陵講大品經,義學名僧,自遠雲集,每講筵商較,四座莫能與抗。目有青睛,時人以為聰惠之相也。自有陳創業,文檄軍書及禪授詔策,皆陵所製,而《九錫》尤美。為一代文宗,亦不以此矜物,未嘗詆訶作者。其於後進之徒,
 接引無倦。世祖、高宗之世,國家有大手筆,皆陵草之。



 其文頗變舊體,緝裁巧密,多有新意。每一文出手,好事者已傳寫成誦,遂被之華夷,家藏其本。後逢喪亂,多散失,存者三十卷。有四子:儉,份,儀,僔。



 儉一名眾。幼而修立,勤學有志操,汝南周弘正重其為人,妻以女。梁太清初,起家豫章王府行參軍。侯景亂,陵使魏未反,儉時年二十一,攜老幼避於江陵,梁元帝聞其名,召為尚書金部郎中。嘗侍宴賦詩,元帝歎賞曰:「徐
 氏之子,復有文矣。」江陵陷,復還於京師。永定初,為太子洗馬,遷鎮東從事中郎。天嘉三年,遷中書侍郎。



 太建初,廣州刺史歐陽紇舉兵反,高宗令儉持節喻旨。紇初見儉,盛列仗衛,言辭不恭,儉曰:「呂嘉之事,誠當已遠,將軍獨不見周迪、陳寶應乎?轉禍為福,未為晚也。」紇默然不答,懼儉沮其眾,不許入城,置儉於孤園寺,遣人守衛,累旬不得還。紇嘗出見儉,儉謂之曰:「將軍業已舉事,儉須還報天子,儉之性命雖在將軍,將軍成敗不在於儉,幸
 不見留。」紇於是乃遣儉從間道馳還。高宗乃命章昭達率眾討紇,仍以儉悉其形勢,敕儉監昭達軍。紇平,高宗嘉之,賜奴婢十人,米五百斛,除鎮北鄱陽王咨議參軍,兼中書舍人。累遷國子博士、大匠卿,餘並如故。尋遷黃門侍郎,轉太子中庶子,加通直散騎常侍,兼尚書左丞,以公事免。尋起為中衛始興王限外咨議參軍,兼中書舍人。又為太子中庶子,遷貞威將軍、太子左衛率,舍人如故。



 後主立,授和戎將軍、宣惠晉熙王長史,行丹陽郡
 國事。俄以父憂去職。尋起為和戎將軍,累遷尋陽內史,為政嚴明,盜賊靜息。遷散騎常侍,襲封建昌侯,入為御史中丞。儉性公平,無所阿附,尚書令江總望重一時,亦為儉所糾劾,後主深委任焉。又領右軍。禎明二年卒。



 份少有父風,年九歲,為《夢賦》,陵見之,謂所親曰:「吾幼屬文,亦不加此。」解褐為秘書郎。轉太子舍人。累遷豫章王主簿、太子洗馬。出為海鹽令,甚有治績。秩滿,入為太子洗馬。份性孝悌,陵嘗遇疾,甚篤,份燒香泣涕,跪誦《孝經》,晝
 夜不息,如此者三日,陵疾豁然而愈,親戚皆謂份孝感所致。太建二年卒,時年二十二。



 儀少聰警,以《周易》生舉高第為祕書郎,出為烏傷令。禎明初,遷尚書殿中郎,尋兼東宮學士。陳亡入隋。開皇九年,隱于錢塘之赭山,煬帝召為學士,尋除著作郎。大業四年卒。



 孝克,陵之第三弟也。少為《周易》生,有口辯,能談玄理。既長,遍通《五經》,博覽史籍,亦善屬文,而文不逮義。梁太清初,起家為太學博士。



 性至孝,遭父憂,殆不勝喪,事所生
 母陳氏,盡就養之道。梁末,侯景寇亂,京邑大饑,餓死者十八九。孝克養母,饘粥不能給,妻東莞臧氏,領軍將軍臧盾之女也,甚有容色,孝克乃謂之曰:「今饑荒如此,供養交闕,慾嫁卿與富人,望彼此俱濟,於卿意如何?」臧氏弗之許也。時有孔景行者,為侯景將,富於財,孝克密因媒者陳意,景行多從左右,逼而迎之,臧涕泣而去,所得穀帛,悉以供養。孝克又剃髮為沙門,改名法整,兼乞食以充給焉。臧氏亦深念舊恩,數私自饋餉,故不乏絕。後
 景行戰死,臧伺孝克於途中,累日乃見,謂孝克曰:「往日之事,非為相負,今既得脫,當歸供養。」孝克默然無答。於是歸俗,更為夫妻。



 後東遊,居於錢塘之佳義里,與諸僧討論釋典,遂通《三論》。每日二時講,旦講佛經,晚講《禮傳》,道俗受業者數百人。天嘉中,除剡令,非其好也,尋復去職。太建四年,徵為秘書丞,不就,乃蔬食長齋,持菩薩戒,晝夜講誦《法華經》,高宗甚嘉其操行。



 六年,除國子博士,遷通直散騎常侍,兼國子祭酒,尋為真。孝克每侍宴,無
 所食啖,至席散,當其前膳羞損減,高宗密記以問中書舍人管斌,斌不能對。自是斌以意伺之,見孝克取珍果內紳帶中,斌當時莫識其意,後更尋訪,方知還以遺母。



 斌以實啟,高宗嗟歎良久,乃敕所司,自今宴享,孝克前饌,並遣將還,以餉其母,時論美之。



 至德中,皇太子入之釋奠,百司陪列,孝克發《孝經》題,後主詔皇太子北面致敬。禎明元年,入為都官尚書。自晉以來,尚書官僚皆攜家屬居省。省在臺城內下舍門,中有閣道,東西跨路,通
 于朝堂。其第一即都官之省,西抵閣道,年化久遠,多有鬼怪,每昏夜之際,無故有聲光,或見人著衣冠從井中出,須臾復沒,或門閣自然開閉。居省者多死亡,尚書周確卒於此省。孝克代確,便即居之,經涉兩載,妖變皆息,時人咸以為貞正所致。



 孝克性清素而好施惠,故不免飢寒,後主敕以石頭津稅給之,孝克悉用設齋寫經,隨得隨盡。二年,為散騎常侍,侍東宮。陳亡,隨例入關。家道壁立,所生母患,欲粳米為粥,不能常辦。母亡之後,孝克
 遂常啖麥,有遺粳米者,孝克對而悲泣,終身不復食之焉。



 開皇十年,長安疾疫,隋文帝聞其名行,召令於尚書都堂講《金剛般若經》。



 尋授國子博士。後侍東宮講《禮傳》。十九年,以疾卒,時年七十三。臨終,正坐念彿,室內有非常異香氣,鄰里皆驚異之。子萬載,仕至晉安王功曹史、太子洗馬。



 史臣曰:徐孝穆挺五行之秀,稟天地之靈,聰明特達,籠罩今古。及締構興王,遭逢泰運,位隆朝宰,獻替謀猷,蓋
 亮直存矣。孝克砥身厲行,養親逾禮,亦參、閔之志歟!



\end{pinyinscope}