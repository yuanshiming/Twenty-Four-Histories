\article{卷五本紀第五宣帝}

\begin{pinyinscope}

 高宗孝宣
 皇帝諱頊,字紹世,小字師利,始興昭烈王第二子也。梁中大通二年七月辛酉生,有赤光滿堂室。少寬大,多智略。及長,美容儀,身長八尺三寸,手垂過膝。有勇力,善騎射。高祖平侯景,鎮京口,梁元帝征高祖子姪入侍,高祖遣高宗赴江陵,累
 官為直閣將軍、中書侍郎。時有馬軍主李總與高宗有舊,每同遊處。



 高宗嘗夜被酒,張燈而寐,總適出,尋返,乃見高宗身是大龍,總便驚駭,走避他室。及江陵陷,高宗遷于關右。永定元年,遙襲封始興郡王,邑二千戶。三年,世祖嗣位,改封安成王。天嘉三年,自周還,授侍中、中書監、中衛將軍,置佐史。



 尋授使持節、都督揚南徐東揚南豫北江五州諸軍事、揚州刺史,進號驃騎將軍,餘如故。
 四年,加開府儀同三司。六年,遷司空。天康元年,授尚書令,餘並如故。



 廢帝
 即位,拜司徒,進號驃騎大將軍,錄尚書,都督中外諸軍事,給班劍三十人。



 光大二年正月,進位太傅,領司徒,加殊禮,劍履上殿,增邑並前三千戶,餘並如故。十一月甲寅,慈訓太后令廢帝為臨海王,以高宗入纂。



 太建元年春正月甲午,即皇帝位于太極前殿,詔曰:「夫聖人受命,王者中興,並由懿德,方作元
 后。高祖武皇帝揖拜堯圖,經綸禹跡,配天之業,光辰象而利貞,格地之功,侔川岳而長遠。世祖文皇帝體上聖之姿,當下武之運,築宮示儉,所務唯德,定鼎初基,厥謀斯在。朕以寡薄,才非聖賢,夙荷前規,方傳景祚。雖復親承訓誨,志守籓維,詠季子之高風,思城陽之遠託,自元儲紹國,正位君臨,無道非幾,佇聞刑措。豈圖王室不造,頻謀亂階,天步艱難,將傾寶歷,仰惟嘉命,爰集朕躬。我心貞確,堅誓蒼昊,而群辟啟請,相喧渭橋,文母尊嚴,懸心長樂,對揚璽紱,非止殷湯之三辭,履涉春冬,何但代王之五讓。今便肅奉天策,欽承介圭。



 若據滄溟,踰增兢業。思所以雲行雨施,品物咸亨,當與黔
 黎,普同斯慶。可改光大三年為太建元年。大赦天下。在位文武賜位一階,孝悌力田及為父後者賜爵一級,異等殊才,並加策序。鰥寡孤獨不能自存者,人賜穀五斛。」復太皇太后尊號曰皇太后。立妃柳氏為皇后,世子叔寶為皇太子,皇子南中郎將、江州刺史康樂侯叔陵為始興王,奉昭烈王祀。乙未,輿駕謁太廟。丁酉,分命大使巡行四方,觀省風俗。



 征南大將軍、開府儀同三司、新除中撫大將軍章昭達進號車騎大將軍,新除中軍大將軍、開府儀同三司、南徐州刺史淳于量為征北大將軍,鎮北將軍、開府儀同三司、南徐州刺史、新除鎮西將軍、郢州刺史黃法抃進號征西大將軍,新除安南將軍、開府儀同三司、湘州刺史吳明徹進號鎮南將軍,鎮東將軍、揚州刺史、鄱陽王伯山進號中衛將軍,尚書僕射沈欽為尚書左僕
 射,度支尚書王勱為尚書右僕射,護軍將軍沈恪為鎮南將軍、廣州刺史。辛丑,輿駕親祠南郊。壬寅,以皇子建安侯叔英為宣惠將軍、東揚州刺史,改封豫章王。豊城侯叔堅改封長沙
 王。癸卯,以明威將軍周弘正為特進。戊午,輿駕親祠太廟。二月庚午,皇后謁太廟。辛未,皇太子謁太廟。



 乙亥,輿駕親耕藉田。夏五月甲午,齊遣使來聘。丁巳,以吏部尚書、領大著作徐陵為尚書右僕射,太子詹事、駙馬都尉沈君理為吏部尚書。秋七月辛卯,皇太子納妃沈氏,王公已下賜帛各有差。
 丁酉,以平東將軍、吳郡太守晉安王伯恭為中護軍,進號安南將軍。九月甲辰,以新除中護軍晉安王伯恭為中領軍。冬十月,新除左衛將軍歐陽紇據廣州舉兵反。辛未,遣車騎將軍、開府儀同三司章昭達率眾討
 之。壬午,輿駕親祠太廟。



 二
 年春正月乙酉,以征西大將軍、開府儀同三司、郢州刺史黃法抃為中權大將軍。丙午,輿駕親祠太廟。二月癸未,儀同章昭達擒歐陽紇送都,斬于建康市,廣州平。
 三月丙申,皇太后崩。丙午,曲赦廣、衡二州。丁未,大赦天下。又詔自討周迪、華皎已來,兵交之所有死亡者,並令收斂,并給棺槥,送還本鄉;瘡痍未瘳者,各給醫藥。夏四月乙卯,臨海王伯宗薨。戊寅,皇太后祔葬萬安陵。閏月戊申,輿駕謁太廟。己酉,太白晝見。五月乙卯,儀同黃法抃獻瑞璧一。壬午,齊遣使來弔。六月戊子,新羅國遣使獻方物。辛卯,大雨雹。乙巳,分遣大使巡行州郡,省理冤屈。戊申,車騎將軍、開府儀同三司章昭達進號車騎大
 將軍,安南將軍、廣州刺史沈恪進號鎮南將軍。秋八月甲申,詔曰:「懷遠以德,抑惟恒典,去戎即華,民之本志。頃年江介襁負相隨,崎嶇歸化,亭候不絕,宜加恤養,答其誠心。維是荒境自皞,有在都邑及諸州鎮,不問遠近,並蠲課役。若克平舊土,反我侵地,皆許還鄉,一無拘限。州郡縣長明加甄別,良田廢村,隨便安處。若輒有課訂,即以擾民論。」又詔曰:「民惟邦本,著在典謨,治國愛民,抑又通訓。朕聽朝晏罷,日昃劬勞,方流惠澤,覃被億兆。有梁
 之季,政刑廢缺,條綱弛紊,僭盜薦興,役賦征徭,尤為煩刻。大陳御宇,拯茲餘弊,滅扈戡黎,弗遑創改,年代彌流,將及成俗,如弗解張,物無與厝,夕惕疚懷,有同首疾。思從卑菲,約己濟民,雖府帑末充,君孰與足,便可刪革,去其甚泰,冀永為定准,令簡而易從。自今維作田,值水旱失收,即列在所,言上折除。軍士年登六十,悉許放還。巧手於役死亡及與老疾,不勞訂補。其籍有巧隱,并王公百司輒受民為程蔭,解還本屬,開恩聽首。



 在職治事之
 身,須遞相檢示,有失不推,當局任罪。令長代換,具條解舍戶數,付度後人。戶有增進,即加擢賞;若致減散,依事準結。有能墾起荒田,不問頃畝少多,依舊蠲稅。」戊子,太白晝見。九月乙丑,以散騎常侍鎮東將軍吳興太守杜棱為特進、護軍將軍。冬十月乙酉,輿駕親祠太廟。十一月辛酉,高麗國遣使獻方物。



 十二月癸巳夜,西北有雷聲。



 三年春正月癸丑,以尚書右僕射、領大著作徐陵為尚
 書僕射。辛酉,輿駕親祠南郊。辛未,親祠北郊。二月辛巳,輿駕親祠明堂。丁酉,親耕籍田。三月丁丑,大赦天下。自天康元年訖太建元年,逋餘軍糧、祿秩、夏調未入者,悉原之。又詔犯逆子弟支屬逃亡異境者,悉聽歸首;見縶繫者,量可散釋;其有居宅,並追還。



 夏四月壬辰,齊遣使來聘。五月戊申,太白晝見。辛亥,遼東、新羅、丹丹、天竺、盤盤等國並遣使獻方物。六月丁亥,江陰王蕭季卿以罪免。甲辰,封東中郎將長沙王府諮議參軍蕭彞為江陰
 王。秋八月辛丑,皇太子親釋奠于太學,二傅、祭酒以下賚帛各有差。九月癸酉,太白晝見。冬十月甲申,輿駕親祠太廟。乙酉,周遣使來聘。己亥,丹丹國遣使獻方物。十二月壬辰,車騎大將軍、司空章昭達薨。



 四年春正月丙午,以雲麾將軍、江州刺史始興王叔陵為湘州刺史,進號平南將軍;東中郎將、吳郡太守長沙王叔堅為宣毅將軍、江州刺史;尚書僕射、領大著作徐陵為尚書左僕射;中書監王勱為尚書右僕射。庚申,以
 丹陽尹衡陽王伯信為信威將軍、中護軍。庚午,輿駕親祠太廟。二月乙酉,立皇子叔卿為建安王,授東中郎將、東揚州刺史。三月壬子,以散騎常侍孫瑒為安西將軍、荊州刺史。乙丑,扶南、林邑國並遣使來獻方物。夏四月戊子,以中權大將軍、開府儀同三司黃法抃為征南大將軍、南豫州刺史。五月癸卯,尚書右僕射王勱卒。六月辛巳,侍中、鎮右將軍、右光祿大夫杜棱卒。秋八月辛未,周遣使來聘。丁丑,景雲見。戊寅,詔曰:「國之大事,受賑興
 戎。師出以律,稟策於廟,所以乂安九有,克成七德。自頃掃滌群穢,廓清諸夏,乃貔貅之戮力,亦帷幄之運籌。雖左衽已戡,干戈載戢,呼韓來謁,亭鄣無警,但不教民戰,是謂棄之,仁必有勇,無忘武備。磻溪之傳韜訣,穀城之授神符,文叔懸制戎規,孟德頗言兵略。朕既慚暗合,良皆披覽。兼昔經督戎,備嘗行陣,齊以七步,肅之三鼓,得自胸襟,指掌可述。今並條制,凡十三科,宜即班宣,以為永准。」乙未,詔停督湘、江二州逋租,無錫等十五縣流民,
 並蠲其徭賦。九月庚子朔,日有蝕之。辛亥,大赦天下。又詔曰:「舉善從諫,在上之明規;進賢謁言,為臣之令範。朕以寡德,嗣守寶圖,雖世襲隆平,治非寧一。辨方分職,旰食早衣;傍闕爭臣,下無貢士。何其闕爾,鮮能抗直。豈余獨運,匪薦讜言。置鼓公車,罕論得失;施石象魏,莫陳可否。朱雲摧檻,良所不逢;禽息觸楹,又為難值。至如衣褐以見,簷簦以遊,或耆艾絕倫,或妙年異等,乾時而不偶,左右莫之譽,黑貂改弊,黃金且殫,終其滯淹,可為太
 息。又貴為百辟,賤有十品,工拙並騖,勸沮莫分,街謠徒擁,廷議斯闕。實朕之弗明,而時無獻替。永言至治,何迺爽歟?外可通示文武:凡厥在位,風化乖殊,朝政糸比蠹,正色直辭,有犯無隱。



 兼各舉所知,隨才明試。其蒞政廉穢,在職能否,分別矢言,俟茲黜陟。」丙寅,以故太尉徐度、儀同杜棱、儀同程靈洗配食高祖廟庭,故車騎將軍章昭達配食世祖廟庭。冬十月乙酉,輿駕親祠太廟。戊戌,以鎮南將軍、廣州刺史沈恪為領軍將軍。



 十一月己亥夜地
 震。閏月辛未,詔曰:「姑熟饒曠,荊河斯擬,博望關畿,天限嚴峻,龍山南指,牛渚北臨,對熊繹之餘城,邇全琮之故壘,良疇美柘,畦畎相望,連宇高甍,阡陌如繡。自梁末兵災,凋殘略盡,比雖務優寬,猶未克復,咫尺封畿,宜須殷阜。且眾將部下,多寄上下,軍民雜俗,極為蠹秏。自今有罷任之徒,許分留部下;其已在江外,亦令迎還,悉住南州津裏安置。有無交貨,不責市估;萊荒墾闢,亦停租稅。臺遣鎮監一人,共刺史、津主分明檢押,給地賦田,各立
 頓舍。」



 十二月壬寅,甘露降樂遊苑。甲辰,輿駕幸樂遊苑,採甘露,宴群臣。丁卯,詔曰:「梁氏之季,兵火薦臻,承華焚蕩,頓無遺構。寶命惟新,迄將二紀,頻事戎旅,未遑脩繕。今工役差閑,椽楹有擬,來歲開肇,創築東宮,可權置起部尚書、將作大匠,用主監作。」



 五年春正月癸酉,以征北大將軍、開府儀同三司、南徐州刺史淳于量為中權大將軍;宣惠將軍、豫章王叔英為南徐州刺史,進號平北將軍;吏部尚書、駙馬都尉沈
 君理為尚書右僕射,領吏部。辛巳,輿駕親祠南郊。甲午,輿駕親祠太廟。二月辛丑,輿駕親祠明堂。乙卯,夜有白氣如虹,自北方貫北斗紫宮。三月壬午,分命眾軍北伐,以鎮前將軍、開府儀同三司吳明徹都督征討諸軍事。丙戌,西衡州獻馬生角。己丑,皇孫胤生,內外文武賜帛各有差,為父後者爵一級。北討大都督吳明徹統眾十萬,發自白下。夏四月癸卯,前巴州刺史魯廣達克齊大峴城。辛亥,吳明徹克秦州水柵。庚申,齊遣兵十萬援歷
 陽,儀同黃法抃破之。辛酉,齊軍救秦州,吳明徹又破之。癸亥,詔北伐眾軍所殺齊兵,並令埋掩。甲子,南譙太守徐槾克石梁城。五月己巳,瓦梁城降。癸酉,陽平郡城降。甲戌,徐槾克廬江郡城。丙子,黃法抃克歷陽城。己卯,北高唐郡城降。辛巳,詔征南大將軍、開府儀同三司、南豫州刺史黃法抃徙鎮歷陽,齊改縣為郡者並復之。乙酉,南齊昌太守黃詠克齊昌外城。丙戌,廬陵內史任忠軍次東關,克其東西二城,進克蘄城。戊子,又克譙郡城,秦
 州城降。癸巳,瓜步、胡墅二城降。六月庚子,郢州刺史李綜克灄口城。乙巳,任忠克合州外城。庚戌,淮陽、沭陽郡並棄城走。癸丑,景雲見。豫章內史程文季克涇州城。乙卯,宣毅司馬湛陀克新蔡城。癸亥,周遣使來聘。黃法抃克合州城。



 吳明徹師次仁州,甲子,克其州城。是月,治明堂。秋七月乙丑,鎮前將軍、開府儀同三司吳明徹進號征北大將軍。戊辰,齊遣眾二萬援齊昌,西陽太守周炅破之。



 己巳,吳明徹軍次峽口,克其北岸城,南岸守者棄
 城走。周炅克巴州城。淮北絳城及穀陽士民,並誅其渠帥,以城降。丙戌,吳明徹克壽陽外城。八月乙未,山陽城降。壬寅,盱眙城降。戊申,罷南齊昌郡。壬子,戎昭將軍徐敬辯克海安城。青州東海城降。戊午,平固侯陳敬泰等克晉州城。九月甲子,陽平城降。壬申,高唐太守沈善度克馬頭城。甲戌,齊安城降。丙子,左衛將軍樊毅克廣陵楚子城。癸未,尚書右僕射、領吏部、駙馬都尉沈君理卒。丁亥,前鄱陽內史魯天念克黃城小城,齊軍退保大城。
 戊子,割南兗州之盱眙郡屬譙州。壬辰晦,夜明。黃城大城降。冬十月甲午,郭默城降。戊戌,以中書令王瑒為吏部尚書。己亥,以特進、領國子祭酒周弘正為尚書右僕射。乙巳,吳明徹克壽陽城,斬王琳,傳首京師,梟于朱雀航。



 丁未,齊兵萬人至潁口。樊毅擊走之。辛亥,齊遣兵援蒼陵,又破之。丙辰,詔曰:「梁末得懸瓠,以壽陽為南豫州,今者克復,可還為豫州。以黃城為司州,治下為安昌郡,鳷湍為漢陽郡,三城依梁為義陽郡,並屬司州。」以征北
 大將軍、開府儀同三司吳明徹為豫州刺史,進號車騎大將軍;征南大將軍、開府儀同三司、南豫州刺史黃法抃為征西大將軍、合州刺史。戊午,湛陀克齊昌城。十一月甲戌,淮陰城降。庚辰,威虜將軍劉桃根克朐山城。辛巳,樊毅克濟陰城。己丑,魯廣達等克北徐州。十二月壬辰朔,詔曰:「古者反噬叛逆,盡族誅夷,所以藏其首級,誡之後世。比者所戮止在一身,子胤或存,梟懸自足,不容久歸武庫,長比月支。惻隱之懷,有仁不忍。維熊曇朗、留
 異、陳寶應、周迪、鄧緒等及今者王琳首,並還親屬,以弘廣宥。」乙未,譙城降。乙巳,立皇子叔明為宜都王,叔獻為河東王。壬午,任忠克霍州城。



 六年春正月壬戌朔,詔曰:「王者以四海為家,萬姓為子,一物乖方,夕惕猶厲,六合未混,旰食彌憂。朕嗣纂鴻基,思弘經略,上符景宿,下葉人謀,命將興師,大拯淪溺。灰琯未周,凱捷相繼,拓地數千,連城將百。蠢彼餘黎,毒茲異境,江淮年少,猶有剽掠,鄉閭無賴,摘出陰私,將帥軍
 人,罔顧刑典,今使苛法蠲除,仁聲載路。且肇元告慶,邊服來荒,始睹皇風,宜覃曲澤,可赦江右淮北南司、定、霍、光、建、朔、合、豫、北徐、仁、北兗、青、冀,南譙、南兗十五州,郢州之齊安、西陽,江州之齊昌、新蔡、高唐,南豫州之歷陽、臨江郡土民,罪無輕重,悉皆原宥。將帥職司,軍人犯法,自依常科。」以翊前將軍新安王伯固為中領軍,進號安前將軍;安前將軍、中領軍晉安王伯恭為安南將軍、南豫州刺史。壬午,輿駕親祠太廟。甲申,廣陵金城降。周遣使
 來聘。高麗國遣使獻方物。二月壬辰朔,日有蝕之。辛亥,輿駕親耕籍田。丙辰,以中權大將軍、開府儀同三司淳于量為征西大將軍、郢州刺史。三月癸亥,詔曰:「去歲南川頗言失稔,所督田租于今未即。



 豫章等六郡太建五年田租,可申半至秋。豫章又逋太建四年檢首田稅,亦申至秋。



 南康一郡,嶺下應接,民間尤弊,太建四年田租未入者,可特原除。庶脩墾無廢,歲取方實。」夏四月庚子,彗星見。辛丑,詔曰:「戢情懷善,有國之令圖,拯弊救危,聖
 範之通訓。近命師薄伐,義在濟民,青、齊舊隸,膠、光部落,久患凶戎,爭歸有道,棄彼農桑,忘其衣食。而大軍未接,中途止憩,朐山、黃郭,車營布滿,扶老攜幼,蓬流草跋,既喪其本業,咸事遊手,饑饉疾疫,不免流離。可遣大使精加慰撫,仍出陽平倉穀,拯其懸磬,並充糧種。勸課士女,隨近耕種。石鱉等屯,適意脩墾。」六月壬辰,尚書右僕射、領國子祭酒周弘正卒。乙巳,以中衛將軍、揚州刺史鄱陽王伯山為征北將軍、南徐州刺史,中護軍衡陽王伯
 信為宣毅將軍、揚州刺史。冬十一月乙亥,詔北討行軍之所,並給復十年。十二月癸巳,平南將軍、湘州刺史始興王叔陵進號鎮南將軍。戊戌,以吏部尚書王瑒為尚書右僕射,度支尚書孔奐為吏部尚書。丙午,安右將軍、左光祿大夫王通加特進。



 七年春正月辛未,輿駕親祠南郊。乙亥,左衛將軍樊毅克潼州城。辛巳,輿駕親祠北郊。二月戊申,樊毅克下邳、高柵等六城。三月辛未,詔豫、二兗、譙、徐、合、霍、南司、定九
 州及南豫、江、郢所部在江北諸郡置雲旗義士,往大軍及諸鎮備防。戊寅,以新除征西大將軍、合州刺史、開府儀同三司黃法抃為豫州刺史。改梁東徐州為安州,武州為沅州。移譙州鎮於新昌郡,以秦郡屬之。盱眙、神農二郡還隸南兗州。夏四月丙戌,有星孛於大角。庚寅,監豫州陳桃根於所部得青牛,獻之,詔遣還民。甲午,輿駕親祠太廟。乙未,陳桃根又表上織成羅文錦被各二,詔於雲龍門外焚之。壬子,郢州獻瑞鐘六。五月乙卯,割
 譙州之秦郡還隸南兗州。分北譙縣置北譙郡,領陽平所屬北譙、西譙二縣。合州之南梁郡,隸入譙州。六月丙戌,為北討將士死王事者克日舉哀。壬辰,以尚書右僕射王瑒為尚書僕射。己酉,改作雲龍、神虎門。秋八月壬寅,移西陽郡治保城。癸卯,周遣使來聘。閏九月壬辰,都督吳明徹大破齊軍於呂梁。是月,甘露頻降樂遊苑。丁未,輿駕幸樂遊苑,採甘露,宴群臣,詔於苑龍舟山立甘露亭。冬十月戊午,以征北將軍、南徐州刺史鄱陽王
 伯山為征南將軍、江州刺史;安前將軍、中領軍新安王伯固為南徐州刺史,進號鎮北將軍;信威將軍、江州刺史長沙王叔堅為雲麾將軍、中領軍。己巳,立皇子叔齊為新蔡王,叔文為晉熙王。十一月庚戌,以征西大將軍、開府儀同三司、郢州刺史淳于量為中軍大將軍。十二月丙辰,以新除雲麾將軍、郢州刺史長沙王叔堅為平越中郎將、廣州刺史,東中郎將、東揚州刺史建安王叔卿為雲麾將軍、郢州刺史,宣惠將軍宜都王叔明為東
 揚州刺史。壬戌,以尚書僕射王瑒為尚書左僕射,太子詹事、揚州大中正陸繕為尚書右僕射,國子祭酒徐陵為領軍將軍。甲子,南康郡獻瑞鐘。



 八年春正月庚辰,西南有紫雲見。二月壬申,車騎大將軍、開府儀同三司吳明徹進位司空。丁丑,詔江東道太建五年以前租稅夏調逋在民間者,皆原之。夏西月甲寅,詔曰:「元戎凱旋,群師振旅,旌功策賞,宜有饗宴。今月十七日,可幸樂遊苑,設絲竹之樂,大會文武。」己未,輿駕
 親祠太廟。五月庚寅,尚書左僕射王瑒卒。六月癸丑,以雲麾將軍、廣州刺史長沙王叔堅為合州刺史,進號平北將軍。



 甲寅,以尚書右僕射陸繕為尚書左僕射,新除晉陵太守王克為尚書右僕射。秋八月丁卯,以車騎大將軍、司空吳明徹為南兗州刺史。九月戊戌,以皇子叔彪為淮南王。



 冬十一月乙酉,以平南將軍、湘州刺史長沙王叔堅為平西將軍、郢州刺史。丁酉,分江州晉熙、高唐、新蔡三郡為晉州。辛丑,以冠軍將軍廬陵王伯仁為中領
 軍。十二月丁卯,以新除太子詹事徐陵為右光祿大夫。



 九年春正月辛卯,輿駕親祠北郊。壬寅,以湘州刺史、新除中衛將軍始興王叔陵為揚州刺史;雲麾將軍建安王叔卿為湘州刺史,進號平南將軍。二月壬子,輿駕親耕藉田。夏五月丙子,詔曰:「朕昧旦求衣,日旰方食,思弘億兆,用臻俾乂,而牧守蒞民,廉平未洽,年常租賦,多致逋餘,即此務農,宜弘寬省。可起太建已來訖八年流移叛戶所帶租調,七年八年叛義丁、五年訖八年叛軍丁、
 六年七年逋租田米粟夏調綿絹絲布麥等,五年訖七年逋貲絹,皆悉原之。」秋七月乙亥,以輕車將軍、丹陽尹江夏王伯義為合州刺史。己卯,百濟國遣使獻方物。庚辰,大雨,震萬安陵華表。己丑,震慧日寺剎及瓦官寺重門,一女子於門下震死。冬十月戊午,司空吳明徹破周將梁士彥眾數萬于呂梁。十二月戊申,東宮成,皇太子移于新宮。



 十年春正月己巳朔,以中領軍廬陵王伯仁為平北將
 軍、南徐州刺史,翊左將軍、右光祿大夫、領太子詹事徐陵為領軍將軍。二月甲子,北討眾軍敗績於呂梁,司空吳明徹及將卒已下,並為周軍所獲。三月辛未,震武庫。丙子,分命眾軍以備周:中軍大將軍、開府儀同三司淳于量為大都督,總水陸諸軍事;明威將軍孫瑒都督荊、郢水陸諸軍事,進號鎮西將軍;左衛將軍樊毅為大都督,督朱沛、清口上至荊山緣淮眾軍,進號平北將軍;武毅將軍任忠都督壽陽、新蔡、霍州等眾
 軍,進號寧遠將軍。乙酉,大赦天下。丁酉,以中軍大將軍、開府儀同三司、護軍將軍淳于量為南兗州刺史,進號車騎將軍。夏四月庚戌,詔曰:「懋賞之言,明於訓誥,挾纊之美,著在撫巡。近歲薄伐,廓清淮、泗,摧鋒致果,文武畢力,櫛風沐雨,寒暑亟離,念功在茲,無忘終食。宜班榮賞,用酬厥勞。應在軍者可並賜爵二級,并加賚恤,付選即便量處。」又詔曰:「惟堯葛衣鹿裘,則天為大,伯禹弊衣菲食,夫子曰『無間然』,故儉德之恭,約失者鮮。朕君臨宇宙,十變年籥,旰日
 勿休,乙夜忘寢,跂予思治,若濟巨川,念茲在茲,懍同馭朽。非貪四海之富,非念黃屋之尊,導仁壽以置群生,寧勞役以奉諸己。但承梁季,亂離斯瘼,宮室禾黍,有名亡處,雖輪奐未睹,頗事經營,去泰去甚,猶為勞費。加以戎車屢出,千金日損,府帑未充,民疲征賦。百姓不足,君孰與足?興言靜念,夕惕懷抱,垂訓立法,良所多慚。



 斲雕為朴,庶幾可慕,雉頭之服既焚,弋綈之衣方襲,損撤之制,前自朕躬,草偃風行,冀以變俗。應御府堂署所營造禮
 樂儀服軍器之外,其餘悉皆停息;掖庭常供、王侯妃主諸有俸恤,並各量減。」丁巳,以新除鎮右將軍新安王伯固為護軍將軍。



 戊午,樊毅遣軍度淮北對清口築城。庚申,大雨雹。壬戌,清口城不守。五月甲申,太白晝見。六月丁卯,大雨,震大皇寺剎、莊嚴寺露盤、重陽閣東樓、千秋門內槐樹、鴻臚府門。秋七月戊戌,新羅國遣使獻方物。乙巳,以散騎常侍、兼吏部尚書袁憲為吏部尚書。八月乙丑朔,改秦郡為義州。戊寅,隕霜,殺稻菽。九月壬寅,以
 平北將軍樊毅為中領軍。乙巳,立方明壇于婁湖。戊申,以中衛將軍、揚州刺史始興王叔陵兼王官伯臨盟。甲寅,輿駕幸婁湖臨誓。乙卯,分遣大使以盟誓班下四方,上下相警戒也。壬戌,以宣惠將軍江夏王伯義為東揚州刺史。冬十月戊寅,罷義州及琅邪、彭城二郡。立建興,領建安、同夏、烏山、江乘、臨沂、湖熟等六縣,屬揚州。戊子,以尚書左僕射陸繕為尚書僕射。十一月辛丑,以鎮西將軍孫瑒為郢州刺史。十二月乙亥,合州廬江蠻田伯
 興出寇樅陽,刺史魯廣達討平之。



 十一年春正月丁酉,龍見於南兗州永寧樓側池中。二月癸亥,輿駕親耕藉田。



 三月丁未,詔淮北義人率戶口歸國者,建其本屬舊名,置立郡縣,即隸近州,賦給田宅,喚訂一無所預。夏五月乙巳,詔曰:「昔軒轅命于風后、力牧,放勛咨爾稷、契、朱武,冕旒垂拱,化致隆平。爰逮漢列五曹,周分六職,設官理務,各有攸司,亦幾期刑措,卜世彌永,並賴群才,用康庶績。朕日昃劬勞,思弘治要,而機事
 尚擁,政道未凝,夕惕于懷,罔知攸濟。方欲仗茲舟楫,委成股肱,徵名責實,取寧多士。自今應尚書曹、府、寺、內省監、司文案,悉付局參議分判。其軍國興造、徵發、選序、三獄等事,前須詳斷,然後啟聞。凡諸辯決,務令清乂,約法守制,較若畫一,不得前後舛互,自相矛盾,致有枉滯。紆意舞文,糾聽所知,靡有攸赦。」



 甲寅,詔曰:「舊律以枉法受財為坐雖重,直法容賄其制甚輕,豈不長彼貪殘,生其舞弄?事涉貨財,寧不尤切?今可改不枉法受財者,科同
 正盜。」六月庚辰,以鎮前將軍豫章王叔英為鎮南將軍、江州刺史。丙戌,以征南將軍、江州刺史鄱陽王伯山為中權將軍、護軍將軍。秋七月辛卯,初用大貨六銖錢。八月甲子,青州義主朱顯宗等率所領七百戶入附。丁卯,輿駕幸大壯觀閱武。戊寅,輿駕還宮。冬十月甲戌,以安前將軍、祠部尚書晉安王伯恭為軍師將軍,尚書僕射陸繕為尚書左僕射。



 十一月辛卯,詔曰:「畫冠弗犯,革此澆風,孥戮是蹈,化於薄俗。朕肅膺寶命,迄將一紀,思經
 邦濟治,憂國愛民,日仄劬勞,夜分輟寢,而還淳反樸,其道靡階,雍熙盛美,莫云能致。遂乃鞫訊之牒,盈於聽覽,舂釱之人,煩於牢犴。周成刑措,漢文斷獄,杼軸空勞,邈焉既遠。加以蕞爾醜徒,軼我彭、汴,淮、汝氓庶,企踵王略,治兵誓旅,義存拯救。飛芻挽粟,征賦頗煩,暑雨祁寒,寧忘咨怨。兼宿度乖舛,次舍違方,若曰之誠,責歸元首,愧心斯積,馭朽非懼。即建子令月,微陽初動,應此嘉辰,宜播寬澤,可大赦天下。」甲午,周遣柱國梁士彥率眾至肥
 口。



 戊戌,周軍進圍壽陽。辛丑,以車騎將軍、開府儀同三司、南兗州刺史淳于量為上流水軍都督;中領軍樊毅都督北討諸軍事,加安北將軍;散騎常侍、左衛將軍任忠都督北討前軍事,加平北將軍;前豊州刺史皋文奏率步騎三千趣陽平郡。癸卯,任忠率步騎七千趣秦郡。丙午,新除仁威將軍、右衛將軍魯廣達率眾入淮。是日,樊毅領水軍二萬自東關入焦湖,武毅將軍蕭摩訶率步騎趣歷陽。戊申,豫州陷。辛亥,霍州又陷。癸丑,以新除
 中衛大將軍、揚州刺史始興王叔陵為大都督,總督水步眾軍。十二月乙丑,南北兗、晉三州,及盱眙、山陽、陽平、馬頭、秦、歷陽、沛、北譙、南梁等九州,並自拔還京師。譙、北徐州又陷。自是淮南之地盡沒于周矣。



 己巳,詔曰:「昔堯、舜在上,茅屋土階,湯、禹為君,藜杖韋帶。至如甲帳珠絡,華榱璧璫,未能雍熙,徒聞侈欲。朕企仰前聖,思求訟平,正道多違,澆風靡乂。



 至今貴里豪家,金鋪玉舄,貧居陋巷,彘食牛衣,稱物平施,何其遼遠。爟烽未息,役賦兼勞,
 文吏姦貪,妄動科格。重以旗亭關市,稅斂繁多,不廣都內之錢,非供水衡之費,逼遏商賈,營謀私蓄。靖懷眾弊,宜事改張。弗弘王道,安拯民蠹?今可宣勒主衣、尚方諸堂署等,自非軍國資須,不得繕造眾物。後宮僚列,若有游長,掖庭啟奏,即皆量遣。大予秘戲,非會禮經,樂府倡優,不合雅正,並可刪改。市估津稅,軍令國章,更須詳定,唯務平允。別觀離宮,郊間野外,非恒饗宴,勿復脩治。并勒內外文武車馬宅舍,皆循儉約,勿尚奢華。違我嚴規,
 抑有刑憲。所由具為條格,標榜宣示,令喻朕心焉。」癸酉,遣平北將軍沈恪、電威將軍裴子烈鎮南徐州,開遠將軍徐道奴鎮柵口,前信州刺史楊寶安鎮白下。戊寅,以中領軍樊毅為鎮西將軍、都督荊郢巴武四州水陸諸軍事。



 十二年春正月戊戌,以散騎常侍、左衛將軍任忠為平南將軍、南豫州刺史,督緣江軍防事。三月壬辰,以平北將軍廬陵王伯仁為翊左將軍、中領軍。夏四月癸亥,尚
 書左僕射陸繕卒。乙丑,以宣毅將軍河東王叔獻為南徐州刺史。己卯,大雩。壬午,雨。五月癸巳,以軍師將軍、尚書右僕射晉安王伯恭為尚書僕射。六月壬戌,大風壞皋門中闥。秋八月己未,周使持節、上柱國、鄖州總管滎陽郡公司馬消難以鄖、隨、溫、應、土、順、沔、儇、岳等九州,魯山、甑山、沌陽、應城、平靖、武陽、上明、溳水等八鎮內附。詔以消難為使持節、侍中、大都督、總督安隨等九州八鎮諸軍事、車騎將軍、司空,封隨郡公,給鼓吹、女樂各一部。
 庚申,詔鎮西將軍樊毅進督沔、漢諸軍事。遣平南將軍、南豫州刺史任忠率眾趣歷陽;通直散騎常侍、超武將軍陳慧紀為前軍都督,趣南兗州。戊辰,以新除司空司馬消難為大都督水陸諸軍事。庚午,通直散騎常侍淳于陵克臨江郡。癸酉,智武將軍魯廣達克郭默城。甲戌,大雨霖。丙子,淳于陵克祐州城。九月癸未,周臨江太守劉顯光率眾內附。是夜,天東南有聲,如風水相擊,三夜乃止。丙戌,改安陸郡為南司州。丁亥,周將王延貴率眾
 援歷陽,任忠擊破之,生擒延貴等。己酉,周廣陵義主曹藥率眾入附。冬十月癸丑,大雨雹震。十一月己丑,詔曰:「朕君臨四海,日旰劬勞,思弘至治,未臻斯道。而兵車驟出,軍費尤煩,芻漕控引,不能征賦。夏中亢旱傷農,畿內為甚,民失所資,歲取無託。此則政刑未理,陰陽舛度,黎元阻饑,君孰與足?靖言興念,餘責在躬,宜布惠澤,溥沾氓庶。其丹陽、吳興、晉陵、建興、義興、東海、信義、陳留、江陵等十郡,并諸署即年田稅、祿秩,並各原半,其丁租半申
 至來歲秋登。」十二月庚辰,宣毅將軍、南徐州刺史河東王叔獻薨。



 十三年春正月壬午,以車騎將軍、開府儀同三司淳于量為左光祿大夫;中權將軍、護軍將軍鄱陽王伯山即本號開府儀同三司;鎮右將軍、國子祭酒新安王伯固為揚州刺史;軍師將軍、尚書僕射晉安王伯恭為尚書左僕射;安右將軍、丹陽尹徐陵為中書監,領太子詹事;吏部尚書袁憲為尚書右僕射。庚寅,以輕車將軍、衛尉卿
 宜都王叔明為南徐州刺史。二月甲寅,詔賜司馬消難所部周大將軍田廣等封爵各有差。乙亥,輿駕親耕藉田。夏四月乙巳,分衡州始興郡為東衡州,衡州為西衡州。



 五月丙辰,以前鎮西將軍樊毅為中護軍。六月辛卯,以新除中護軍樊毅為護軍將軍。



 秋九月癸亥,夜,大風至自西北,發屋拔樹,大雷震雹。冬十月癸未,以散騎常侍、丹陽尹毛喜為吏部尚書,護軍將軍樊毅為鎮西將軍、荊州刺史。改鄱陽郡為吳州。



 壬寅,丹丹國遣使獻方
 物。十二月辛巳,彗星見。己亥,以翊右將軍、衛尉卿沈恪為護軍將軍。



 十四年春正月己酉,高宗弗豫。甲寅,崩于宣福殿,時年五十三。遺詔曰:「朕爰自遘疾,曾未浹旬,醫藥不瘳,便屬大漸,終始定分,夫復奚言。但君臨寰宇,十有四載,誠則雖休勿休,日慎一日,知宗廟之負重,識王業之艱難。而邊鄙多虞,生民未乂,方欲蕩清四海,包吞八荒,有志莫從,遺恨幽壤。皇太子叔寶繼體正嫡,年業韶茂,纂統
 洪基,社稷有主。群公卿士,文武內外,俱罄心力,同竭股肱,送往事居,盡忠誠之節,當官奉職,引翼亮之功。務在葉和,無違朕意。凡厥終制,事從省約。金銀之飾,不須入壙,明器之具,皆令用瓦。唯使儉而合禮,勿得奢而乖度。以日易月,既有通規,公除之制,悉依舊準。在位百司,三日一臨,四方州鎮,五等諸侯,各守所職,並停奔赴。」二月辛卯,上謚孝宣皇帝,廟號高宗。癸巳,葬顯寧陵。



 高宗在田之日,有大度幹略,及乎登庸,實允天人之望。梁室喪亂,
 淮南地並入齊,高宗太建初,志復舊境,乃運神略,授律出師,至於戰勝攻取,獻捷相繼,遂獲反侵地,功實懋焉。及周滅齊,乘勝略地,還達江際矣。



 史臣曰:高宗器度弘厚,亦有人君之量焉。世祖知冢嗣仁弱,弗可傳於寶位,高宗地居姬旦,世祖情存太伯,及乎弗悆,大事咸委焉。至於纂業,萬機平理,命將出師,克淮南之地,開拓土宇,靜謐封疆。享國十餘年,志大意逸,呂梁覆軍,大喪師徒矣。江左削弱,抑此之由。嗚呼!蓋德
 不逮文,智不及武,雖得失自我,無禦敵之略焉。



\end{pinyinscope}