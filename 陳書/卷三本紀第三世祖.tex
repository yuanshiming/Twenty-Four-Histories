\article{卷三本紀第三世祖}

\begin{pinyinscope}

 世祖文皇帝,諱蒨,字子華,始興昭烈王長子也。少沈敏有識量,美容儀,留意經史,舉動方雅,造次必遵禮法。高祖甚愛之,常稱「此兒吾宗之英秀也」。梁太清初,夢兩日
 鬥,一大一小,大者光滅墜地,色正黃,其大如斗,世祖因三分取一而懷之。侯景之亂,鄉人多依山湖寇抄,世祖獨保家無所犯。時亂日甚,乃避地臨安。及高祖舉義兵,侯景遣使收世祖及衡陽獻王,世祖乃密袖小刀,冀因入見而害景。至便屬吏,故其事不行。高祖大軍圍石頭,景欲加害者數矣。會景敗,世祖乃得出赴高祖營。起家為吳興太守。時宣城劫帥紀機、郝仲等各聚眾千餘人,侵暴郡境,世祖討平之。承聖二年,授信武將軍,監南徐
 州。三年,高祖北征廣陵,使世祖為前軍,每戰克捷。高祖之將討王僧辯也,先召世祖與謀。時僧辯女婿杜龕據吳興,兵眾甚盛,高祖密令世祖還長城,立柵以備龕。世祖收兵纔數百人,戰備又少,龕遣其將杜泰領精兵五千,乘虛奄至。將士相視失色,而世祖言笑自若,部分益明,於是眾心乃定。泰知柵內人少,日夜苦攻。世祖激厲將士,身當矢石,相持數旬,泰乃退走。及高祖遣周文育率兵討龕,世祖與并軍往吳興。時龕兵尚眾,斷據衝要,
 水步連陣相結,世祖命將軍劉澄、蔣元舉率眾攻龕,龕軍大敗,窘急,因請降。東揚州刺史張彪起兵圍臨海太守王懷振,懷振遣使求救,世祖與周文育輕兵往會稽以掩彪。後彪將沈泰開門納世祖,世祖盡收其部曲家累,彪至,又破走,若邪村民斬彪,傳其首。以功授持節、都督會稽等十郡諸軍事、宣毅將軍、會稽太守。



 山越深險,皆不賓附,世祖分命討擊,悉平之,威惠大振。高祖受禪,立為臨川郡王,邑二千戶,拜侍中、安東將軍。及周文育、
 侯安都敗於沌口,高祖詔世祖入衛,軍儲戎備,皆以委焉。尋命率兵城南皖。



 永定三年六月丙午,高祖崩,遺詔徵世祖入纂。甲寅,至自南皖,入居中書省。



 皇后令曰:「昊天不弔,上玄降禍。大行皇帝奄捐萬國,率土哀號,普天如喪,窮酷煩冤,無所迨及。諸孤藐爾,反國無期,須立長主,以寧宇縣。侍中、安東將軍、臨川王蒨,體自景皇,屬惟猶子。建殊功於牧野,敷盛業於戡黎,納麓時敘之辰,負扆乘機之日,並佐時雍,是同草創,祧祏所繫,遐邇宅心,
 宜奉大宗,嗣膺寶錄,使七廟有奉,兆民寧晏。未亡人假延餘息,嬰此百罹,尋繹纏綿,興言感絕。」世祖固讓,至于再三,群公卿士固請,其日即皇帝位於太極前殿。詔曰:「上天降禍,奄集邦家,大行皇帝背離萬國,率土崩心,若喪考妣。龍圖寶歷,眇屬朕躬,運鐘擾攘,事切機務,南面須主,西讓禮輕,令便式膺景命,光宅四海。可大赦天下,罪無輕重,悉皆蕩滌。逋租宿債,吏民愆負,可勿復收。文武內外,量加爵敘。孝悌力田為父後者,賜爵一級。庶祗
 畏在心,公卿畢力,勝殘去殺,無待百年。興言號哽,深增慟絕。」又詔州郡悉停奔赴。秋七月丙辰,尊皇后為皇太后,己未,以鎮南將軍、開府儀同三司、廣州刺史歐陽頠進號征南將軍,平南將軍、開府儀同三司周迪進號鎮南將軍,平南將軍、開府儀同三司、高州刺史黃法抃進號安南將軍。



 庚申,以鎮南大將軍、開府儀同三司、桂州刺史淳于量進號征南大將軍。辛酉,以侍中、車騎將軍、司空侯瑱為太尉,鎮西將軍、開府儀同三司、南豫州刺
 史侯安都為司空,侍中、中權將軍、開府儀同三司王沖為特進、左光祿大夫,鎮北將軍、南徐州刺史徐度為侍中、中撫軍將軍、開府儀同三司。壬戌,以侍中、護軍將軍徐世譜為特進、安右將軍;侍中、忠武將軍杜棱為領軍將軍。乙丑,重雲殿災。八月癸巳,以平北將軍、南徐州刺史留異為安南將軍、縉州刺史,平南將軍、北江州刺史魯悉達進號安左將軍。庚戌,封皇子伯茂為始興王,奉昭烈王後。徙封始興嗣王頊為安成王。九月辛酉,立皇
 子伯宗為皇太子,王公以下賜帛各有差。乙亥,立妃沈氏為皇后。冬十一月乙卯,王琳寇大雷,詔遣太尉侯瑱、司空侯安都、儀同徐度率眾以禦之。



 天嘉元年春正月癸丑,詔曰:「朕以寡昧,嗣纂洪業,哀煢在疚,治道弗昭,仰惟前德,幽顯遐暢,恭己不言,庶幾無改。雖宏圖懋軌,日月方弘,而清廟廓然,聖靈浸遠,感尋永往,瞻言罔極。今四象運周,三元告獻,華夷胥洎,玉帛駿奔,思覃遺澤,播之億兆。其大赦天下。改永定四年為
 天嘉元年。鰥寡孤獨不能自存立者,賜穀人五斛。孝悌力田殊行異等,加爵一級。」甲寅,分遣使者宣勞四方。辛酉,輿駕親祠南郊,詔曰:「朕式饗上玄,虔奉牲玉,高禋禮畢,誠敬兼弘。且陰霾浹辰,褰霽在日,雲物韶朗,風景清和,慶動人祇,忭流庶俗,思俾黎元,同此多祐。可賜民爵一級。」辛未,輿駕親祠北郊。日有冠。二月辛卯,老人星見。乙未,高州刺史紀機自軍叛還宣城,據郡以應王琳,涇令賀當遷討平之。丙申,太尉侯瑱敗王琳于梁山,攻齊
 兵于博望,生擒齊將劉伯球,盡收其資儲船艦,俘馘以萬計,王琳及其主蕭莊奔于齊。戊戌,詔曰:「夫五運遞來,三靈眷命,皇王因之改創,殷、周所以樂推。朕統歷承基,丕隆鼎運,期理攸屬,數祚斯在,豈僥倖所至,寧卜祝可求。故知神器之重,必在符命。是以逐鹿貽譏,斷蛇定業,亂臣賊子,異世同尤。王琳識暗挈瓶,智慚衛足,干紀亂常,自貽顛沛,而縉紳君子,多被縶維,雖涇渭合流,蘭鮑同肆,求之厥理,或有脅從。今九罭既設,八紘斯掩,天網
 恢恢,吞舟是漏。至如伏波遊說,永作漢蕃,延壽脫歸,終為魏守,器改秦、虞,材通晉、楚,行藏用捨,亦豈有恒,宜加寬仁,以彰雷作。其衣冠士族,預在凶黨,悉皆原宥;將帥戰兵,亦同肆眚,並隨才銓引,庶收力用。」又詔師旅以來,將士死王事者,並加贈謚。己亥,詔曰:「日者凶渠肆虐,眾軍進討,舟艦輸積,權倩民丁,師出經時,役勞日久。今氣昆廓清,宜有甄被。可蠲復丁身。夫妻三年,於役不幸者,復其妻子。」庚子,分遣使者齎璽書宣勞四方。乙巳,遣太
 尉侯瑱鎮湓城。庚戌,以高祖第六子昌為驃騎將軍、湘州牧,立為衡陽王。三月丙辰,詔曰:「自喪亂以來,十有餘載,編戶凋亡,萬不遺一,中原氓庶,蓋云無幾。頃者寇難仍接,算斂繁多,且興師已來,千金日費,府藏虛竭,杼軸歲空。近所置軍資,本充戎備,今元惡克殄,八表已康,兵戈靜戢,息肩方在,思俾餘黎,陶此寬賦,今歲軍糧通減三分之一。尚書申下四方,稱朕哀矜之意。守宰明加勸課,務急農桑,庶鼓腹含哺,復在茲日。」蕭莊所署郢州刺
 史孫瑒舉州內附。丁巳,江州刺史周迪平南中,斬賊率熊曇朗,傳首京師。先是,齊軍守魯山城,戊午,齊軍棄城走,詔南豫州刺史程靈洗守之。甲子,分荊州之天門、義陽、南平,郢州之武陵四郡,置武州。其刺史督沅州,領武陵太守,治武陵郡。其都尉所部六縣為沅州。別置通寧郡,以刺史領太守,治都尉城,省舊都尉。以安南將軍、南兗州刺史、新除右衛將軍吳明徹為安西將軍、武州刺史,偽郢州刺史孫瑒為安南將軍、湘州刺史。丙子,衡陽
 王昌薨。丁丑,詔曰:「蕭莊偽署文武官屬還朝者,量加錄序。」夏四月丁亥,立皇子伯信為衡陽王,奉獻王後。乙未,以安南將軍荀朗為安北將軍、合州刺史。五月乙卯,改桂陽之汝城縣為廬陽郡。分衡州之始興、安遠二郡,置東衡州。六月辛巳,改謚皇祖妣景安皇后曰景文皇后。壬辰,詔曰:「梁孝元遭離多難,靈櫬播越,朕昔經北面,有異常倫,遣使迎接,以次近路。江寧既有舊塋,宜即安卜,車旗禮章,悉用梁典,依魏葬漢獻帝故事。」甲午,追策故
 始興昭烈王妃曰孝妃。丁酉,以開府儀同三司徐度為侍中、中軍將軍。辛丑,國哀周忌,上臨于太極前殿,百僚陪哭。



 赦京師殊死已下。是月,葬梁元帝於江寧。秋七月甲寅,詔曰:「朕以眇身,屬當大寶,負荷至重,憂責實深,而庶績未康,胥怨猶結,佇咨賢良,發於夢想,每有一言入聽,片善可求,何嘗不褒獎抽揚,緘書紳帶。而傅巖虛往,穹谷尚淹,蒲幣空陳,旌弓不至。豈當有乖則哲,使草澤遺才?將時運澆流,今不逮古?側食長懷,寢興增歎。新安
 太守陸山才有啟,薦梁前征西從事中郎蕭策,梁前尚書中兵郎王暹,並世胄清華,羽儀著族,或文史足用,或孝德可稱,並宜登之朝序,擢以不次。王公已下,其各進舉賢良,申薦淪屈,庶眾才必萃,大廈可成,使《棫樸》載歌,《由庚》在詠。」乙卯,詔曰:「自頃喪亂,編戶播遷,言念餘黎,良可哀惕。其亡鄉失土,逐食流移者,今年內隨其適樂,來歲不問僑舊,悉令著籍,同土斷之例。」



 丙辰,立皇子伯山為鄱陽王。八月庚辰,老人星見。壬午,詔曰:「菽粟之貴,重
 於珠玉。自頃寇戎,游手者眾,民失分地之業,士有佩犢之譏。朕哀矜黔庶,念康弊俗,思俾阻饑,方存富教。麥之為用,要切斯甚,今九秋在節,萬實可收,其班宣遠近,並令播種。守宰親臨勸課,務使及時。其有尤貧,量給種子。」癸未,世祖臨景陽殿聽訟。戊子,詔曰:「汙罇土鼓,誠則難追,畫卵彫薪,或可易革。梁氏末運,奢麗已甚,芻豢厭於胥史,歌鐘列於管庫,土木被朱丹之采,車馬飾金玉之珍,逐欲澆流,遷訛遂遠。朕自諸生,頗為內足,而家敦朴
 素,室靡浮華,觀覽時俗,常所扼腕。今妄假時乘,臨馭區極,屬當淪季,思聞治道,菲食卑宮,自安儉陋,俾茲薄俗,獲反淳風。維雕鏤淫飾,非兵器及國容所須,金銀珠玉,衣服雜玩,悉皆禁斷。」甲午,周將賀若敦率馬步一萬,奄至武陵,武州刺史吳明徹不能拒,引軍還巴陵。丁酉,上幸正陽堂閱武。九月癸丑,彗星見。乙卯,周將獨孤盛領水軍將趣巴、湘,與賀若敦水陸俱進,太尉侯瑱自尋陽往禦之。辛酉,遣儀同徐度率眾會瑱于巴丘。丙子,太白
 晝見。丁丑,詔侯瑱眾軍進討巴、湘。十月癸巳,侯瑱襲破獨孤盛於楊葉洲,盡獲其船艦,盛收兵登岸,築城以保之。丁酉,詔司空侯安都率眾會侯瑱南討。十二月乙未,詔曰:「古者春夏二氣,不決重罪。蓋以陽和布澤,天秩是弘,寬網省刑,義符含育,前王所以則天象地,立法垂訓者也。朕屬當澆季,思求民瘼,哀矜惻隱,念甚納隍,常欲式遵舊軌,用長風化。自今孟春訖于夏首,罪人大辟事已款者,宜且申停。」己亥,周巴陵城主尉遲憲降,遣巴州
 刺史侯安鼎守之。庚子,獨孤盛將餘眾自楊葉州潛遁。



 二年春正月庚戌,大赦天下。以雲麾將軍、晉陵太守杜棱為侍中、領軍將軍。



 辛亥,以始興王伯茂為宣惠將軍、揚州刺史。乙卯,合州刺史裴景徽奔于齊。辛未,周湘州城主殷亮降,湘州平。二月丙戌,以太尉侯瑱為車騎將軍、湘州刺史。庚寅,曲赦、湘州諸郡。三月乙卯,太尉、車騎將軍、湘州刺史侯瑱薨。丁丑,以鎮東將軍、會稽太守徐度為鎮南將軍、湘州刺史。夏四月,分荊州之南平、宜都、
 羅、河東四郡,置南荊州,鎮河東郡。以安西將軍、武州刺史吳明徹為南荊州刺史。庚寅,以安左將軍魯悉達為安南將軍、吳州刺史。辛卯,老人星見。秋七月丙午,周將賀若敦自拔遁歸,人畜死者十七八。武陵、天門、南平、義陽、河東、宜都郡悉平。



 九月甲寅,詔曰:「姬業方闡,望載渭濱,漢歷既融,道通圮上。若乃摛精辰宿,降靈惟岳,風雲有感,夢寐是求,斯固舟楫鹽梅,遞相表裏,長世建國,罔或不然。



 至於銘德太常,從祀清廟,以貽厥後來,垂諸不
 朽者也。前皇經濟區宇,裁成品物,靈貺式甄,光膺寶命,雖謨明濬發,幽顯協從,亦文武賢能,翼宣王業。故大司馬、驃騎大將軍瑱,故司空文育,故平北將軍、開府儀同三司僧明,故中護軍穎,故領軍將軍擬,或締構艱難,經綸夷險;或摧鋒冒刃,殉義遺生;或宣哲協規,綢繆帷幄;或披荊汗馬,終始勤劬;莫不罄誠悉力,屯泰以之。朕以寡昧,嗣膺丕緒,永言勳烈,思弘典訓,便可式遵故實,載揚盛軌,可並配食高祖廟庭,俾茲大猷,永傳宗祏。」丙辰,
 以侍中、中權將軍、特進、左光祿大夫、開府儀同三司王沖為丹陽尹;丹陽尹沈君理為左民尚書,領步兵校尉。冬十月乙巳,霍州西山蠻率部落內屬。十一月乙卯,高麗國遣使獻方物。甲子,以武昌、國川為竟陵郡,以安流民。



 十二月辛巳,以安東將軍、吳郡太守孫瑒為中護軍。甲申,立始興國廟於京師,用王者之禮。太子中庶子虞荔、御史中丞孔奐以國用不足,奏立煮海鹽賦及榷酤之科,詔並施行。先是,縉州刺史留異應于王琳等反,丙
 戌,詔司空侯安都率眾討之。



 三年春正月庚戌,設帷宮於南郊,幣告胡公以配天。辛亥,輿駕親祠南郊。詔曰:「朕負荷寶圖,亟回星琯,兢兢業業,庶幾治定,而德化不孚,俗弊滋甚,永言念之,無忘日夜。陽和布氣,昭事上玄,躬奉犧玉,誠兼饗敬,思與黎元,被斯寬惠,可普賜民爵一級,其孝悌力田,別加一等。」辛酉,輿駕親祠北郊。閏二月己酉,以百濟王餘明為撫東大將軍,高句驪王高湯為寧東將軍。江州刺史周迪舉
 兵應留異,襲湓城,攻豫章郡,並不剋。辛亥,以南荊州刺史吳明徹為安右將軍。甲子,改鑄五銖錢。三月丙子,安成王頊至自周,詔授侍中、中書監中衛將軍,置佐史。丁丑,以安右將軍吳明徹為安南將軍、江州刺史,督眾軍南討。甲申,大赦天下。庚寅,司空侯安都破留異於桃支嶺,異脫身奔晉安,東陽郡平。夏四月癸卯,曲赦東陽郡。乙巳,齊遣使來聘。六月丙辰,以侍中、中衛將軍安成王頊為驃騎將軍、揚州刺史。以會稽、東陽、臨海、永嘉、新安、
 新寧、晉安、建安八郡置東揚州。以揚州刺史始興王伯茂為鎮東將軍、東揚州刺史,征北將軍、司空、南徐州刺史侯安都為侍中、征北大將軍。秋七月己丑,皇太子納妃王氏。在位文武賜帛各有差,孝悌力田為父後者賜爵二級。九月戊辰朔,日有食之。以侍中、都官尚書到仲舉為尚書右僕射、丹陽尹。丁亥,周迪請降,詔安成王頊督眾軍以招納之。是歲,周所立梁王蕭詧死,子巋代立。



 四年春正月丙子,乾陀利國遣使獻方物。甲申,周迪棄
 城走,閩州刺史陳寶應納之,臨川郡平。壬辰,以平西將軍、郢州刺史章昭達為護軍將軍,仁武將軍、新州刺史華皎進號平南將軍,鎮南將軍、開府儀同三司、高州刺史黃法抃為鎮北大將軍、南徐州刺史,安西將軍、領臨川太守周敷為南豫州刺史,中護軍孫瑒為鎮右將軍。罷高州隸入江州。二月戊戌,征南將軍、開府儀同三司、廣州刺史歐陽頠進號征南大將軍。庚戌,以侍中、司空、征北大將軍侯安都為征南大將軍、江州刺史。



 庚申,以
 平南將軍華皎為南湘州刺史。三月辛未,以鎮南將軍、開府儀同三司徐度為侍中、中軍大將軍。辛巳,詔贈討周迪將士死王事者。夏四月辛丑,設無珝大會於太極前殿。乙卯,以侍中、中書監、中衛將軍、驃騎將軍、揚州刺史安成王頊為開府儀同三司。五月丁卯,安前將軍、右光祿大夫徐世譜卒。六月癸巳,太白晝見。



 司空侯安都賜死。七月丁丑,以鎮北大將軍、開府儀同三司、南徐州刺史黃法甗為鎮南大將軍、江州刺史。九月壬戌,開府
 儀同三司、廣州刺史歐陽頠薨。癸亥,曲赦京師。辛未,周迪復寇臨川,詔護軍章昭達率眾討之。十一月辛酉,章昭達大破周迪,悉擒其黨與,迪脫身潛竄。十二月丙申,大赦天下。詔護軍將軍章昭達進軍建安,以討陳寶應。信威將軍、益州刺史餘孝頃督會稽、東陽、臨海、永嘉諸軍自東道會之。癸丑,以前安南將軍、江州刺史吳明徹為鎮前將軍。



 五年春正月庚辰,以吏部尚書、領右軍將軍袁樞為丹
 陽尹。辛巳,輿駕親祠北郊。乙酉,江州湓城火,燒死者二百餘人。三月丁丑,以征南大將軍、開府儀同三司、桂州刺史淳于量為中撫軍大將軍。壬午,詔以故護軍將軍周鐵虎配食高祖廟庭。



 夏四月庚子,周遣使來聘。五月庚午,罷南丹陽郡。是月,周、齊並遣使來聘。六月丁未,夜,有白氣兩道,出于北斗東南,屬地。秋七月丁丑,詔曰:「朕以寡昧,屬當負重,星籥亟改,冕旒弗曠,不能仰協璇衡,用調玉燭,傍慰蒼生,以安黔首。



 兵無寧歲,民乏有年,移
 風之道未弘,習俗之患猶在,致令氓多觸網,吏繁筆削,獄犴滋章,雖由物犯,囹圄淹滯,亦或有冤。念俾納隍,載勞負扆,加以膚湊不適,攝衛有虧,比獲微痊,思覃寬惠,可曲赦京師。」九月,城西城。冬十一月丁亥,以左衛將軍程靈洗為中護軍。己丑,章昭達破陳寶應于建安,擒寶應、留異,送京師,晉安郡平。甲辰,以護軍將軍章昭達為鎮前將軍、開府儀同三司。十二月甲子,曲赦建安、晉安二郡。討陳寶應將士死王事者,並給棺槥,送還本鄉,并
 復其家。



 瘡痍未瘳者,給其醫藥。癸未,齊遣使來聘。



 六年春正月甲午,皇太子加元服,王公以下賜帛各有差,孝悌力田為父後者賜爵一級,鰥寡孤獨不能自存者穀人五斛。庚戌,以領軍將軍杜棱為翊左將軍、丹陽尹,丹陽尹袁樞為吏部尚書,衛尉卿沈欽為中領軍。三月乙未,詔侯景以來遭亂移在建安、晉安、義安郡者,並許還本土,其被略為奴婢者,釋為良民。夏四月甲寅,以侍中、中書監、中衛將軍、驃騎將軍、開府儀同三司、揚州
 刺史安成王頊為司空。



 辛酉,有彗星見。周遣使來聘。秋七月癸未,大風至自西南,廣百餘步,激壞靈臺候樓。甲申,儀賢堂無故自壞。丙戌,臨川太守駱文牙斬周迪,傳首京師,梟於朱雀航。丁酉,太白晝見。八月丁丑,詔曰:「梁室多故,禍亂相尋,兵甲紛紜,十年不解,不逞之徒虐流生氣,無賴之屬暴及徂魂。江左肇基,王者攸宅,金行水位之主,木運火德之君,時更四代,歲逾二百。若其經綸王業,縉紳民望,忠臣孝子,何世無才,而零落山丘,變移
 陵谷,或皆剪伐,莫不侵殘。玉杯得於民間,漆簡傳於世載,無復五株之樹,罕見千年之表。自大祚光啟,恭惟揖讓,爰暨朕躬,聿脩祖武,雖復旗旗服色,猶行杞、宋之邦,每車駕巡遊,眇瞻河、雒之路,故喬山之祀,蘋藻弗虧,驪山之墳,松柏恒守。唯戚籓舊壟,士子故塋,掩殣未周,樵牧猶眾。或親屬流隸,負土無期,子孫冥滅,手植何寄。漢高留連於無忌,宋祖惆悵於子房,丘墓生哀,性靈共惻者也。朕所以興言永日,思慰幽泉。維前代王侯,自古忠
 烈,墳冢被發絕無後者,可檢行修治,墓中樹木,勿得樵採,庶幽顯咸暢,稱朕意焉。」己卯,立皇子伯固為新安郡王,伯恭為晉安王,伯仁為廬陵王,伯義為江夏王。九月癸未,罷豫章郡。是月,新作大航。冬十月辛亥,齊遣使來聘。十二月乙卯,立皇子伯禮為武陵王。丁巳,以鎮前將軍、開府儀同三司章昭達為鎮南將軍、江州刺史,鎮南大將軍、江州刺史黃法抃為中衛大將軍,中護軍程靈洗為宣毅將軍、郢州刺史,軍師將軍、郢州刺史沈恪為
 中護軍,鎮東將軍、吳興太守吳明徹為中領軍。戊午,以東中郎將、吳郡太守鄱陽王伯山為平北將軍、南徐州刺史。癸亥,詔曰:「朕自居民牧之重,託在王公之上,顧其寡昧,鬱于治道。加以屢虧聽覽,事多壅積,冤滯靡申,幽枉弗鑒。念茲罪隸,有甚納隍。而惠澤未流,愆陽累月,今歲序云暮,元正向肇,欲使幽圄之內,同被時和,可曲赦京師。」



 天康元年春二月丙子,詔曰:「朕以寡德,纂承洪緒,日昃
 劬勞,思弘景業,而政道多昧,黎庶未康,兼疹患淹時,亢陽累月,百姓何咎,實由朕躬,念茲在茲,痛如疾首。可大赦天下,改天嘉七年為天康元年。三月己卯,以驃騎將軍、開府儀同三司、揚州刺史、司空安成王頊為尚書令。夏四月乙卯,皇孫至澤生,在位文武賜絹帛各有差,為父後者賜爵一級。癸酉,世祖疾甚。是日,崩于有覺殿。遺詔曰:「朕疾苦彌留,遂至不救,脩短有命,夫復何言。但王業艱難,頻歲軍旅,生民多弊,無忘愧惕。今方隅乃定,俗
 教未弘,便及大漸,以為遺恨。社稷任重,太子可即君臨,王侯將相,善相輔翊,內外協和,勿違朕意!山陵務存儉速。大斂竟,群臣三日一臨,公除之制,率依舊典。」六月甲子,群臣上謚曰文皇帝,廟號世祖。



 丙寅,葬永寧陵。



 世祖起自艱難,知百姓疾苦。國家資用,務從儉約。常所調斂,事不獲已者,必咨嗟改色,若在諸身。主者奏決,妙識真偽,下不容姦,人知自勵矣。一夜內刺閨取外事分判者,前後相續。每雞人伺漏,傳更簽於殿中,乃敕送者必投
 簽於階石之上,令鎗然有聲,云「吾雖眠,亦令驚覺也」。始終梗概,若此者多焉。



 陳吏部尚書姚察曰:世稱繼體守文,宗枝承統,得失之間,蓋亦祥矣。大抵以奉而勿墜為賢能,撓而易之為不肖;其有光揚前軌,克荷曾構,固以少焉。世祖自初發跡,功庸顯著,寧亂靜寇,首佐大業。及國禍奄臻,入承寶祚,兢兢業業,其若馭朽,加以崇尚儒術,愛悅文義,見善如弗及,用人如由己,恭儉以御身,勤勞以濟物,自昔允文
 允武之君,東征西怨之后,賓實之迹,可為聯類。至於杖聰明,用鑒識,斯則永平之政,前史其論諸。



\end{pinyinscope}