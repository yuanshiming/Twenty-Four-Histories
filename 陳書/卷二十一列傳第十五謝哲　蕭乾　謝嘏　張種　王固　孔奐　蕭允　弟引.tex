\article{卷二十一列傳第十五謝哲 蕭乾 謝嘏 張種 王固 孔奐 蕭允 弟引}

\begin{pinyinscope}

 謝哲,字穎豫,陳郡陽夏人也。祖朏,梁司徒。父譓,梁右光祿大夫。哲美風儀,舉止醞藉,而襟情豁然,為士君子所
 重。起家梁秘書郎,累遷廣陵太守。侯景之亂,以母老因寓居廣陵,高祖自京口渡江應接郭元建,哲乃委質,深被敬重。高祖為南徐州刺史,表哲為長史。荊州陷,高祖使哲奉表於晉安王勸進。敬帝承製徵為給事黃門侍郎,領步兵校尉。貞陽侯僭位,以哲為通直散騎常侍,侍東宮。敬帝即位,遷長兼侍中。高祖受命,遷都官尚書、豫州大中正、吏部尚書。出為明威將軍、晉陵太守,入為中書令。世祖嗣位,為太子詹事。出為明威將軍、衡陽內史,秩
 中二千石。遷長沙太守,將軍、加秩如故。還除散騎常侍、中書令。廢帝即位,以本官領前將軍。高宗為錄尚書,引為侍中、仁威將軍、司徒左長史。未拜,光大元年卒,時年五十九。贈侍中、中書監,謚康子。



 蕭乾,字思惕,蘭陵人也。祖嶷,齊丞相豫章文獻王。父子範,梁秘書監。乾容止雅正,性恬簡,善隸書,得叔父子雲之法。年九歲,召補國子《周易》生,梁司空袁昂時為祭酒,深敬重之。十五,舉明經。釋褐東中郎湘東王法曹參軍,
 遷太子舍人。建安侯蕭正立出鎮南豫州,又板錄事參軍。累遷中軍宣城王中錄事諮議參軍。侯景平,高祖鎮南徐州,引乾為貞威將軍、司空從事中郎。遷中書侍郎、太子家令。



 永定元年,除給事黃門侍郎。是時熊曇朗在豫章,周迪在臨川,留異在東陽,陳寶應在建、晉,共相連結,閩中豪帥,往往立砦以自保,高祖甚患之,乃令乾往使,諭以逆順,并觀虛實。將發,高祖謂乾曰:「建、晉恃險,好為姦宄,方今天下初定,難便出兵。昔陸賈南征,趙佗歸
 順,隨何奉使,黥布來臣,追想清風,仿佛在目。況卿坐鎮雅俗,才高昔賢,宜勉建功名,不煩更勞師旅。」乾既至,曉以逆順,所在渠帥並率部眾開壁款附。其年,就除貞威將軍、建安太守。



 天嘉二年,留異反,陳寶應將兵助之,又資周迪兵糧,出寇臨川,因逼建安。



 乾單使臨郡,素無士卒,力不能守,乃棄郡以避寶應。時閩中守宰,並為寶應迫脅,受其署置,乾獨不為屈,徙居郊野,屏絕人事。及寶應平,乃出詣都督章昭達,昭達以狀表聞,世祖甚嘉之,
 超授五兵尚書。光大元年卒,謚曰靜子。



 謝嘏,字含茂,陳郡陽夏人也。祖,齊金紫光祿大夫。父舉,梁中衛將軍、開府儀同三司。嘏風神清雅,頗善屬文。起家梁秘書郎,稍遷太子中庶子,掌東宮管記,出為建安太守。侯景之亂,嘏之廣州依蕭勃,承聖中,元帝徵為五兵尚書,辭以道阻,轉授智武將軍。蕭勃以為鎮南長史、南海太守。勃敗,還至臨川,為周迪所留。久之,又度嶺之晉安依陳寶應,世祖前後頻召之,嘏崎嶇寇虜,不能
 自拔。



 及寶應平,嘏方詣闕,為御史中丞江德藻所舉劾,世祖不加罪責,以為給事黃門侍郎。尋轉侍中,天康元年,以公事免,尋復本職。光大元年,為信威將軍、中衛始興王長史。遷中書令、豫州大中正、都官尚書,領羽林監,中正如故。太建元年卒,贈侍中、中書令,謚曰光子。有文集行於世。



 二子儼、伷。儼官至散騎常侍、侍中、御史中丞、太常卿,出監東揚州。禎明二年卒於會稽,贈中護軍。



 張種,字士苗,吳郡人也。祖辯,宋司空右長史、廣州刺史。
 父略,梁太子中庶子、臨海太守。種少恬靜,居處雅正,不妄交遊,傍無造請,時人為之語曰:「宋稱敷、演,梁則卷、充。清虛學尚,種有其風。」仕梁王府法曹,遷外兵參軍,以父憂去職。服闋,為中軍宣城王府主簿。種時年四十餘,家貧,求為始豊令,入除中衛西昌侯府西曹掾。時武陵王為益州刺史,重選府僚,以種為征西東曹掾,種辭以母老,抗表陳請,為有司所奏,坐黜免。



 侯景之亂,種奉其母東奔,久之得達鄉里。俄而母卒,種時年五十,而毀瘠過
 甚,又迫以凶荒,未獲時葬,服制雖畢,而居處飲食,恆若在喪。及景平,司徒王僧辯以狀奏聞,起為貞威將軍、治中從事史,并為具葬禮,葬訖,種方即吉。僧辯又以種年老,傍無胤嗣,賜之以妾,及居處之具。



 貞陽侯僭位,除廷尉卿、太子中庶子。敬帝即位,為散騎常侍,遷御史中丞,領前軍將軍。高祖受禪,為太府卿。天嘉元年,除左民尚書。二年,權監吳郡,尋徵復本職。遷侍中,領步兵校尉,以公事免,白衣兼太常卿,俄而即真。廢帝即位,加領右軍
 將軍,未拜,改領弘善宮衛尉,又領揚、東揚二州大中正。高宗即位,重為都官尚書,領左驍騎將軍,遷中書令,驍騎、中正並如故。以疾授金紫光祿大夫。



 種沈深虛靜,而識量宏博,時人皆以為宰相之器。僕射徐陵嘗抗表讓位於種曰:「臣種器懷沈密,文史優裕,東南貴秀,朝庭親賢,克壯其猷,宜居左執。」其為時所推重如此。太建五年卒,時年七十,贈特進,謚曰元子。



 種仁恕寡欲,雖歷居顯位,而家產屢空,終日晏然,不以為病。太建初,女為始興
 王妃,以居處僻陋,特賜宅一區,又累賜無錫、嘉興縣侯秩。嘗於無錫見有重囚在獄,天寒,呼出曝日,遂失之,世祖大笑,而不深責。有集十四卷。



 種弟棱,亦清靜有識度,官至司徒左長史,太建十一年卒,時年七十,贈光祿大夫。



 種族子稚才,齊護軍沖之孫。少孤介特立,仕為尚書金部郎中。遷右丞,建康令、太府卿、揚州別駕從事史,兼散騎常侍。使于周,還為司農、廷尉卿。所歷並以清白稱。



 王固,字子堅,左光祿大夫通之弟也。少清正,頗涉文史,
 以梁武帝甥封莫口亭侯。舉秀才。起家梁秘書郎,遷太子洗馬,掌東宮管記,丁所生母憂去職。服闋,除丹陽尹丞。侯景之亂,奔于荊州,梁元帝承制以為相國戶曹屬,掌管記。尋聘于西魏,魏人以其梁氏外戚,待之甚厚。承聖元年,遷太子中庶子,尋為貞威將軍、安南長史、尋陽太守。荊州陷,固之鄱陽,隨兄質度東嶺,居信安縣。紹泰元年,徵為侍中,不就。永定中,移居吳郡。世祖以固清靜,且欲申以婚姻。天嘉二年,至都,拜國子祭酒。三年,遷中
 書令。四年,又為散騎常侍、國子祭酒。其年,以固女為皇太子妃,禮遇甚重。



 廢帝即位,授侍中、金紫光祿大夫。時高宗輔政,固以廢帝外戚,妳媼恒往來禁中,頗宣密旨,事洩,比將伏誅,高宗以固本無兵權,且居處清潔,止免所居官,禁錮。



 太建二年,隨例為招遠將軍、宣惠豫章王諮議參軍。遷太中大夫、太常卿、南徐州大中正。七年,卒官,時年六十三。贈金紫光祿大夫。喪事所須,隨由資給。



 至德二年改葬,謚曰恭子。



 固清虛寡欲,居喪以孝聞。又
 崇信佛法,及丁所生母憂,遂終身蔬食,夜則坐禪,晝誦佛經,兼習《成實論》義,而於玄言非所長。嘗聘于西魏,因宴饗之際,請停殺一羊,羊於固前跪拜。又宴於昆明池,魏人以南人嗜魚,大設罟網,固以佛法咒之,遂一鱗不獲。



 子寬,官至司徒左史、侍中。



 孔奐,字休文,會稽山陰人也。曾祖琇之,齊左民尚書、吳興太守。祖臶,太子舍人、尚書三公郎。父稚孫,梁寧遠枝江公主簿、無錫令。奐數歲而孤,為叔父虔孫所養。好學,
 善屬文,經史百家,莫不通涉。沛國劉顯時稱學府,每共奐討論,深相歎服,乃執奐手曰:「昔伯喈墳素悉與仲宣,吾當希彼蔡君,足下無愧王氏。」



 所保書籍,尋以相付。



 州舉秀才,射策高第。起家揚州主簿、宣惠湘東王行參軍,並不就。又除鎮西湘東王外兵參軍,入為尚書倉部郎中,遷儀曹侍郎。時左民郎沈炯為飛書所謗,將陷重辟,事連臺閣,人懷憂懼,奐廷議理之,竟得明白。丹陽尹何敬容以奐剛正,請補功曹史。出為南昌侯相,值侯景亂,
 不之官。



 京城陷,朝士並被拘縶,或薦奐於賊帥侯子鑒,子鑒命脫桎梏,厚遇之,令掌書記。時景軍士悉恣其凶威,子鑒景之腹心,委任又重,朝士見者,莫不卑俯屈折,奐獨敖然自若,無所下。或諫奐曰:「當今亂世,人思茍免,獯羯無知,豈可抗之以義?」奐曰:「吾性命有在,雖未能死,豈可取媚凶醜,以求全乎?」時賊徒剝掠子女,拘逼士庶,奐每保持之,得全濟者甚眾。



 尋遭母憂,哀毀過禮。時天下喪亂,皆不能終三年之喪,唯奐及吳國張種,在寇亂
 中守持法度,並以孝聞。



 及景平,司徒王僧辯先下辟書,引奐為左西曹掾,又除丹陽尹丞。梁元帝於荊州即位,徵奐及沈炯並令西上,僧辯累表請留之。帝手敕報僧辯曰:「孔、沈二士,今且借公。」其為朝廷所重如此。仍除太尉從事中郎。僧辯為揚州刺史,又補揚州治中從事史。時侯景新平,每事草創,憲章故事,無復存者,奐博物彊識,甄明故實,問無不知,儀注體式,箋表書翰,皆出於奐。



 高祖作相,除司徒右長史,遷給事黃門侍郎。齊遣東方
 老、蕭軌等來寇,軍至後湖,都邑搔擾,又四方壅隔,糧運不繼,三軍取給,唯在京師,乃除奐為貞威將軍、建康令。時累歲兵荒,戶口流散,勍敵忽至,徵求無所,高祖剋日決戰,乃令奐多營麥飯,以荷葉裹之,一宿之間,得數萬裹,軍人旦食訖,棄其餘,因而決戰,遂大破賊。



 高祖受禪,遷太子中庶子。永定二年,除晉陵太守。晉陵自宋、齊以來,舊為大郡,雖經寇擾,猶為全實,前後二千石多行侵暴,奐清白自守,妻子並不之官,唯以單船監郡,所得秩
 俸,隨即分贍孤寡,郡中大悅,號曰「神君」。曲阿富人殷綺,見奐居處素儉,乃餉衣一襲,氈被一具。奐曰:「太守身居美祿,何為不能辦此,但民有未周,不容獨享溫飽耳。勞卿厚意,幸勿為煩。」



 初,世祖在吳中,聞奐善政,及踐祚,徵為御史中丞,領揚州大中正。奐性剛直,善持理,多所糾劾,朝廷甚敬憚之。深達治體,每所敷奏,上未嘗不稱善,百司滯事,皆付奐決之。遷散騎常侍,領步兵校尉,中書舍人,掌詔誥,揚、東揚二州大中正。天嘉四年,重除御史
 中丞,尋為五兵尚書,常侍、中正如故。時世祖不豫,臺閣眾事,並令僕射到仲舉共奐決之。及世祖疾篤,奐與高宗及仲舉並吏部尚書袁樞、中書舍人劉師知等入侍醫藥。世祖嘗謂奐等曰:「今三方鼎峙,生民未乂,四海事重,宜須長君。朕欲近則晉成,遠隆殷法,卿等須遵此意。」奐乃流涕歔欷而對曰:「陛下御膳違和,痊復非久,皇太子春秋鼎盛,聖德日躋,安成王介弟之尊,足為周旦,阿衡宰輔,若有廢立之心,臣等愚誠,不敢聞詔。」世祖曰:「古
 之遺直,復見於卿。」天康元年,乃用奐為太子詹事,二州中正如故。



 世祖崩,廢帝即位,除散騎常侍、國子祭酒。光大二年,出為信武將軍、南中郎康樂侯長史、尋陽太守,行江州事。高宗即位,進號仁威將軍、雲麾始興王長史,餘並如故。奐在職清儉,多所規正,高宗嘉之,賜米五百斛,并累降敕書殷勤勞問。



 太建三年,徵為度支尚書,領右軍將軍。五年,改領太子中庶子,與左僕射徐陵參掌尚書五條事。六年,遷吏部尚書。七年,加散騎常侍。八年,
 改加侍中。時有事北討,剋復淮、泗,徐、豫酋長,降附相繼,封賞選敘,紛紜重疊,奐應接引進,門無停賓。加以鑒識人物,詳練百氏,凡所甄拔,衣冠縉紳,莫不悅伏。



 性耿介,絕請託,雖儲副之尊,公侯之重,溺情相及,終不為屈。始興王叔陵之在湘州,累諷有司,固求台鉉。奐曰:「袞章之職,本以德舉,未必皇枝。」因抗言於高宗。高宗曰:「始興那忽望公,且朕兒為公,須在鄱陽王後。」奐曰:「臣之所見,亦如聖旨。」後主時在東宮,欲以江總為太子詹事,令管記
 陸瑜言之於奐。奐謂瑜曰:「江有潘、陸之華,而無園、綺之實,輔弼儲宮,竊有所難。」



 瑜具以白後主,後主深以為恨,乃自言於高宗。高宗將許之,奐乃奏曰:「江總文華之人,今皇太子文華不少,豈藉於總!如臣愚見,願選敦重之才,以居輔導。」



 帝曰:「即如卿言,誰當居此?」奐曰:「都官尚書王廓,世有懿德,識性敦敏,可以居之。」後主時亦在側,乃曰:「廓王泰之子,不可居太子詹事。」奐又奏曰:「宋朝范曄即范泰之子,亦為太子詹事,前代不疑。」後主固爭之,帝
 卒以總為詹事,由是忤旨。其梗正如此。



 初,後主欲官其私寵,以屬奐,奐不從。及右僕射陸繕遷職,高宗欲用奐,已草詔訖,為後主所抑,遂不行。九年,遷侍中、中書令、領左驍騎將軍、揚、東揚、豊三州大中正。十一年,轉太常卿,侍中、中正並如故。十四年,遷散騎常侍、金紫光祿大夫,領前軍將軍,未拜,改領弘範宮衛尉。至德元年卒,時年七十。贈散騎常侍,本官如故。有集十五卷,彈文四卷。



 子紹薪、紹忠。紹忠字孝揚,亦有才學,官至太子洗馬、儀同
 鄱陽王東曹掾。



 蕭允,字叔佐,蘭陵人也。曾祖思話,宋征西將軍、開府儀同三司、尚書右僕射,封陽穆公。祖惠蒨,散騎常侍、太府卿、左民尚書。父介,梁侍中、都官尚書。



 允少知名,風神凝遠,通達有識鑒,容止醞藉,動合規矩。起家邵陵王法曹參軍,轉湘東王主簿,遷太子洗馬。侯景攻陷臺城,百僚奔散,允獨整衣冠坐于宮坊,景軍人敬而弗之逼也。尋出居京口。時寇賊縱橫,百姓波駭,衣冠士族,四出奔散,
 允獨不行。人問其故,允答曰:「夫性命之道,自有常分,豈可逃而獲免乎?但患難之生,皆生於利,茍不求利,禍從何生?方今百姓爭欲奮臂而論大功,一言而取卿相,亦何事於一書生哉?莊周所謂畏影避迹,吾弗為也。」乃閉門靜處,并日而食,卒免於患。



 侯景平後,高祖鎮南徐州,以書召之,允又辭疾。永定中,侯安都為南徐州刺史,躬造其廬,以申長幼之敬,天嘉三年,徵為太子庶子。三年,除棱威將軍、丹陽尹丞。五年,兼侍中,聘于周,還拜中書
 侍郎、大匠卿。高宗即位,遷黃門侍郎。



 五年,出為安前晉安王長史。六年,晉安王為南豫州,允復為王長史。時王尚少,未親民務,故委允行府州事。入為光祿卿。允性敦重,未嘗以榮利幹懷。及晉安出鎮湘州,又苦攜允,允少與蔡景歷善,景歷子徵脩父黨之敬,聞允將行,乃詣允曰:「公年德並高。國之元老,從容坐鎮,旦夕自為列曹,何為方復辛苦在外!」允答曰:「已許晉安,豈可忘信。」其恬於榮勢如此。



 至德三年,除中衛豫章王長史,累遷通直散
 騎常侍、光勝將軍、司徒左長史、安德宮少府。鎮衛鄱陽王出鎮會稽,允又為長史,帶會稽郡丞。行經延陵季子廟,設萍藻之薦,託為異代之交,為詩以敘意,辭理清典。後主嘗問蔡徵曰:「卿世與蕭允相知,此公志操何如?」徵曰:「其清虛玄遠,殆不可測,至於文章,可得而言。」因誦允詩以對,後主嗟賞久之。其年拜光祿大夫。



 及隋師濟江,允遷于關右。是時朝士至長安者,例並授官,唯允與尚書僕射謝伷辭以老疾,隋文帝義之,並厚賜錢帛。尋以
 疾卒於長安,時年八十四。弟引。



 引字叔休。方正有器局,望之儼然,雖造次之間,必由法度。性聰敏,博學,善屬文。釋褐著作佐郎,轉西昌侯儀同府主簿。侯景之亂,梁元帝為荊州刺史,朝士多往歸之。引曰:「諸王力爭,禍患方始,今日逃難,未是擇君之秋。吾家再世為始興郡,遺愛在民,正可南行以存家門耳。」於是與弟彤及宗親等百餘人奔嶺表。



 時始興人歐陽頠為衡州刺史,引往依焉。頠後遷為廣州,病死,子紇領其
 眾。引每疑紇有異,因事規正,由是情禮漸疏。及紇舉兵反,時京都士人岑之敬、公孫挺等並皆惶駭,唯引恬然,謂之敬等曰:「管幼安、袁曜卿亦但安坐耳。君子正身以明道,直己以行義,亦復何憂懼乎?」及章昭達平番禺,引始北還。高宗召引問嶺表事,引具陳始末,帝甚悅,即日拜金部侍郎。



 引善隸書,為當時所重。高宗嘗披奏事,指引署名曰:「此字筆勢翩翩,似鳥之欲飛。」引謝曰:「此乃陛下假其羽毛耳。」又謂引曰:「我每有所忿,見卿輒意解,何
 也?」引曰:「此自陛下不遷怒,臣何預此恩。」太建七年,加戎昭將軍。



 九年,除中衛始興王咨議參軍,兼金部侍郎。



 引性抗直,不事權貴,左右近臣無所造請,高宗每欲遷用,輒為用事者所裁。



 及呂梁覆師,戎儲空匱,乃轉引為庫部侍郎,掌知營造弓弩槊箭等事。引在職一年,而器械充牣。頻加中書侍郎、貞威將軍、黃門郎。十二年,吏部侍郎缺,所司屢舉王寬、謝燮等,帝並不用,乃中詔用引。



 時廣州刺史馬靖甚得嶺表人心,而兵甲精練,每年深入
 俚洞,又數有戰功,朝野頗生異議。高宗以引悉嶺外物情,且遣引觀靖,審其舉措,諷令送質。引奉密旨南行,外託收督賧物。既至番禺,靖即悟旨,盡遣兒弟下都為質。還至贛水,而高宗崩,後主即位,轉引為中庶子,以疾去官。明年,京師多盜,乃復起為貞威將軍、建康令。



 時殿內隊主吳璡,及宦官李善度、蔡脫兒等多所請屬,引一皆不許。引族子密時為黃門郎,諫引曰:「李、蔡之勢,在位皆畏憚之,亦宜小為身計。」引曰:「吾之立身,自有本末,亦安
 能為李、蔡改行。就令不平,不過解職耳。」吳璡竟作飛書,李、蔡證之,坐免官,卒於家,時年五十八。子德言,最知名。



 引宗族子弟,多以行義知名。弟彤,以恬靜好學,官至太子中庶子、南康王長史。密字士機,幼而聰敏,博學有文詞。祖琛,梁特進。父遊,少府卿。密太建八年,兼散騎常侍,聘於齊。歷位黃門侍郎、太子中庶子、散騎常侍。



 史臣曰:謝、王、張、蕭,咸以清凈為風,文雅流譽,雖更多難,終克成名。



 奐謇諤在公,英飆振俗,詳其行事,抑古之遺
 愛矣。固之蔬菲禪悅,斯乃出俗者焉,猶且致絓於黜免,有懼於傾覆。是知上官、博陸之權勢,閻、鄧、梁、竇之震動,籲可畏哉!



\end{pinyinscope}