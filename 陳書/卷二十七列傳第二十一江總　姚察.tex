\article{卷二十七列傳第二十一江總 姚察}

\begin{pinyinscope}

 江總,字總持,濟陽考城人也,晉散騎常侍統之十世孫。五世祖湛,宋左光祿大夫、開府儀同三司,忠簡公。祖蒨,梁光祿大夫,有名當代。父紑,本州迎主簿,少居父憂,以
 毀卒,在《梁書孝行傳》。



 總七歲而孤,依於外氏。幼聰敏,有至性。舅吳平光侯蕭勱,名重當時,特所鐘愛,嘗謂總曰:「爾操行殊異,神采英拔,後之知名,當出吾右。」及長,篤學有辭采,家傳賜書數千卷,總晝夜尋讀,未嘗輟手。年十八,解褐宣惠武陵王府法曹參軍。中權將軍、丹陽尹何敬容開府,置佐史,並以貴胄充之,仍除敬容府主簿。



 遷尚書殿中郎。梁武帝撰《正言》始畢,製《述懷詩》,總預同此作,帝覽總詩,深降嗟賞。仍轉侍郎。尚書僕射范陽張纘,
 度支尚書琅邪王筠,都官尚書南陽劉之遴,並高才碩學,總時年少有名,纘等雅相推重,為忘年友會。之遴嘗酬總詩,其略曰:「上位居崇禮,寺署鄰栖息。忌聞曉騶唱,每畏晨光赩。高談意未窮,晤對賞無極。探急共遨遊,休沐忘退食。曷用銷鄙吝,枉趾覯顏色。下上數千載,揚搉吐胸臆。」其為通人所欽挹如此。遷太子洗馬,又出為臨安令,還為中軍宣城王府限內錄事參軍,轉太子中舍人。



 及魏國通好,敕以總及徐陵攝官報聘,總以疾不行。
 侯景寇京都,詔以總權兼太常卿,守小廟。臺城陷,總避難崎嶇,累年,至會稽郡,憩於龍華寺,乃製《修心賦》,略序時事。其辭曰:太清四年秋七月,避地于會稽龍華寺。此伽藍者,餘六世祖宋尚書右僕射州陵侯元嘉二十四年之所構也。侯之王父晉護軍將軍彪,昔蒞此邦,卜居山陰都陽里,貽厥子孫,有終焉之志。寺域則宅之舊基,左江右湖,面山背豁,東西連跨,南北紆縈,聊與苦節名僧,同銷日用,曉脩經戒,夕覽圖書,寢處風雲,憑棲水月。
 不意華戎莫辨,朝市傾淪,以此傷情,情可知矣。啜泣濡翰,豈攄鬱結,庶後生君子,憫余此概焉。



 嘉南斗之分次,肇東越之靈秘。表《檜風》於韓什,著鎮山於周記。蘊大禹之金書,鐫暴秦之石字。太史來而探穴,鐘離去而開笥。信竹箭之為珍,何珷玞之罕值。奉盛德之鴻祀,寓安禪之古寺。實豫章之舊圃,成黃金之勝地。遂寂默之幽心,若鏡中而遠尋。面曾阜之超忽,邇平湖之迥深。山條偃蹇,水葉侵淫。掛猿朝落,飢鼯夜吟。果叢藥苑,桃蹊橘林。
 梢雲拂日,結暗生陰。保自然之雅趣,鄙人間之荒雜。望島嶼之邅回,面江源之重沓。泛流月之夜迥,曳光煙之曉匝。風引蜩而嘶噪,雨鳴林而修颯,鳥稍狎而知來,雲無情而自合。爾迺野開靈塔,地築禪居,喜園迢遰,樂樹扶疏。經行籍草,宴坐臨渠,持戒振錫,度影甘蔬。堅固之林可喻,寂滅之場蹔如。異曲終而悲起,非木落而悲始。豈降志而辱身,不露才而揚己。鐘風雨之如晦,倦雞鳴之聒耳。幸避地而高棲,憑調御之遺旨。折四辯之微言,
 悟三乘之妙理。遣十纏之繫縛,祛五惑之塵滓。久遺榮於勢利,庶忘累於妻子。感意氣於疇日,寄知音於來祀。何遠客之可悲,知自憐其何已。



 總第九舅蕭勃先據廣州,總又自會稽往依焉。梁元帝平侯景,徵總為明威將軍、始興內史,以郡秩米八百斛給總行裝。會江陵陷,遂不行,總自此流寓嶺南積歲。



 天嘉四年,以中書侍郎徵還朝,直侍中省。累遷司徒右長史,掌東宮管記,給事黃門侍郎,領南徐州大中正。授太子中庶子、通直散騎常
 侍,東宮、中正如故。遷左民尚書,轉太子詹事,中正如故。以與太子為長夜之飲,養良娣陳氏為女,太子微行總舍,上怒免之。尋為侍中,領左驍騎將軍。復為左民尚書領左軍將軍,未拜,又以公事免。尋起為散騎常侍、明烈將軍、司徒左長史,遷太常卿。



 後主即位,除祠部尚書,又領左驍騎將軍,參掌選事。轉散騎常侍、吏部尚書。



 尋遷尚書僕射,參掌如故。至德四年,加宣惠將軍,量置佐史。尋授尚書令,給鼓吹一部,加扶,餘並如故。策曰:「於戲,夫
 文昌政本,司會治經,韋彪謂之樞機,李固方之斗極。況其五曹斯綜,百揆是諧,同冢宰之司,專臺閣之任。惟爾道業標峻,寓量弘深,勝範清規,風流以為准的,辭宗學府,衣冠以為領袖。故能師長六官,具瞻允塞,明府八座,儀形載遠,其端朝握揆,朕所望焉。往欽哉,懋建爾徽猷,亮采我邦國,可不慎歟!」禎明二年,進號中權將軍。京城陷,入隋,為上開府。開皇十四年,卒於江都,時年七十六。



 總嘗自敘其略曰:歷升清顯,備位朝列,不邀世利,不涉
 權幸。嘗撫躬仰天太息曰:莊青翟位至丞相,無迹可紀;趙元叔為上計吏,光乎列傳。官陳以來,未嘗逢迎一物,干預一事。悠悠風塵,流俗之士,頗致怨憎,榮枯寵辱,不以介意。太建之世,權移群小,諂嫉作威,屢被摧黜,奈何命也。後主昔在東朝,留意文藝,夙荷昭晉,恩紀契闊。



 嗣位之日,時寄謬隆,儀形天府,釐正庶績,八法六典,無所不統。昔晉武帝策荀公曾曰「周之塚宰,今之尚書令也」。況復才未半古,尸素若茲。晉太尉陸玩云「以我為三公,
 知天下無人矣」。軒冕儻來之一物,豈是預要乎?弱歲歸心釋教,年二十餘,入鐘山就靈曜寺則法師受菩薩戒。暮齒官陳,與攝山布上人遊款,深悟苦空,更復練戒,運善於心,行慈於物,頗知自勵,而不能蔬菲,尚染塵勞,以此負愧平生耳。



 總之自敘,時人謂之實錄。



 總篤行義,寬和溫裕。好學,能屬文,於五言七言尤善;然傷於浮艷,故為後主所愛幸。多有側篇,好事者相傳諷玩,於今不絕。後主之世,總當權宰,不持政務,但日與後主遊宴後庭,
 共陳暄、孔範、王瑳等十餘人,當時謂之狎客。由是國政日頹,綱紀不立,有言之者,輒以罪斥之,君臣昏亂,以至于滅。有文集三十卷,並行於世焉。



 長子溢,字深源,頗有文辭。性傲誕,恃勢驕物,雖近屬故友,不免詆欺。歷官著作佐郎、太子舍人、洗馬、中書黃門侍郎、太子中庶子。入隋,為秦王文學。



 第七子漼,駙馬都尉、祕書郎、隋給事郎,直秘書省學士。



 姚察,字伯審,吳興武康人也。九世祖信,吳太常卿,有名
 江左。察幼有至性,事親以孝聞。六歲,誦書萬餘言。弱不好弄,博弈雜戲,初不經心。勤苦厲精,以夜繼日。年十二,便能屬文。父上開府僧垣,知名梁武代,二宮禮遇優厚,每得供賜,皆回給察兄弟,為遊學之資,察並用聚蓄圖書,由是聞見日博。年十三,梁簡文帝時在東宮,盛脩文義,即引於宣猷堂聽講論難,為儒者所稱。及簡文嗣位,尤加禮接。起家南海王國左常侍,兼司文侍郎。除南郡王行參軍,兼尚書駕部郎。



 值梁室喪亂,於金陵隨二親
 還鄉里。時東土兵荒,人飢相食,告糴無處,察家口既多,並採野蔬自給。察每崎嶇艱阻,求請供養之資,糧粒恒得相繼。又常以己分減推諸弟妹,乃至故舊乏絕者皆相分恤,自甘唯藜藿而已。在亂離之間,篤學不廢。



 元帝於荊州即位,父隨朝士例往赴西臺,元帝授察原鄉令。時邑境蕭條,流亡不反,察輕其賦役,勸以耕種,於是戶口殷盛,民至今稱焉。



 中書侍郎領著作杜之偉與察深相眷遇,表用察佐著作,仍撰史。永定初,拜始興王府功
 曹參軍,尋補嘉德殿學士,轉中衛、儀同始興王府記室參軍。吏部尚書徐陵時領著作,復引為史佐,及陵讓官致仕等表,並請察製焉,陵見歎曰:「吾弗逮也。」太建初,補宣明殿學士,除散騎侍郎、左通直。尋兼通直散騎常侍,報聘于周。江左耆舊先在關右者,咸相傾慕。沛國劉臻竊于公館訪《漢書》疑事十餘條,並為剖析,皆有經據。臻謂所親曰:「名下定無虛士。」著《西聘道里記》,所敘事甚詳。



 使還,補東宮學士。于時濟陽江總、吳國顧野王、陸瓊、從
 弟瑜、河南褚玠、北地傅縡等,皆以才學之美,晨夕娛侍。察每言論製述,咸為諸人宗重。儲君深加禮異,情越群僚,宮內所須方幅手筆,皆付察立草。又數令共野王遞相策問,恒蒙賞激。



 遷尚書祠部侍郎。此曹職司郊廟,昔魏王肅奏祀天地,設宮縣之樂,八佾之舞,爾後因循不革。梁武帝以為事人禮縟,事神禮簡,古無宮縣之文。陳初承用,莫有損益。高宗欲設備樂,付有司立議,以梁武帝為非。時碩學名儒、朝端在位者,咸希上旨,並即注同。
 察乃博引經籍,獨違群議,據梁樂為是,當時驚駭,莫不慚服,僕射徐陵因改同察議。其不順時隨俗,皆此類也。



 拜宣惠宜都王中錄事參軍,帶東宮學士。歷仁威淮南王、平南建安王二府咨議參軍,丁內憂去職。俄起為戎昭將軍,知撰梁史事,固辭不免。後主纂業,敕兼東宮通事舍人,將軍、知撰史如故。又敕專知優冊謚議等文筆。至德元年,除中書侍郎,轉太子僕,餘並如故。



 初,梁季淪沒,父僧垣入於長安,察蔬食布衣,不聽音樂,至是凶問
 因聘使到江南。時察母韋氏喪制適除,後主以察羸瘠,慮加毀頓,乃密遣中書舍人司馬申就宅發哀,仍敕申專加譬抑。爾後又遣申宣旨誡喻曰:「知比哀毀過禮,甚用為憂。



 卿迥然一身,宗奠是寄,毀而滅性,聖教所不許。宜微自遣割,以存禮制。憂懷既深,故有此及。」



 尋以忠毅將軍起兼東宮通事舍人。察志在終喪,頻有陳讓,並抑而不許。又推表其略曰:「臣私門禍,併罹殃罰,偷生晷漏,冀申情禮,而尪疹相仍,苴緌穢質,非復人流,將畢苫
 壤。豈期朝恩曲覃,被之纓紱,尋斯寵服,彌見慚靦。且宮闥祕奧,趨奏便繁,寧可以茲荒毀所宜叨預。伏願至德孝治,矜其理奪,使殘魂喘息,以遂餘生。」詔答曰:「省表具懷。卿行業淳深,聲譽素顯,理徇情禮,未膺刀筆。但參務承華,良所期寄,允茲抑奪,不得致辭也。」俄敕知著作郎事,服闋,除給事黃門侍郎,領著作。



 察既累居憂服,兼齋素日久,自免憂後,因加氣疾。後主嘗別召見,見察柴瘠過甚,為之動容,乃謂察曰:「朝廷惜卿,卿宜自惜,即蔬菲
 歲久,可停持長齋。」



 又遣度支尚書王瑗宣旨,重加慰喻,令從晚食。手敕曰:「卿羸瘠如此,齋菲累年,不宜一飯,有乖將攝,若從所示,甚為佳也。」察雖奉此敕,而猶敦宿誓。



 又詔授秘書監,領著作如故,乃累進讓,並優荅不許。察其秘書省大加刪正,又奏撰中書表集。拜散騎常侍,尋授度支尚書,旬月遷吏部尚書,領著作並如故。



 察既博極墳素,尤善人物,至於姓氏所起,枝葉所分,官職姻娶,興衰高下,舉而論之,無所遺失。且澄鑒之職,時人久以
 梓匠相許,及遷選部,雅允朝望。初,吏部尚書蔡徵移中書令,後主方擇其人,尚書令江總等咸共薦察,敕答曰:「姚察非唯學藝優博,亦是操行清修,典選難才,今得之矣。」乃神筆草詔,讀以示察,察辭讓甚切。



 別日召入論選事,察垂涕拜請曰:「臣東皋賤族,身才庸近,情忘遠致,念絕脩途。頃來忝竊,久知逾分,特以東朝攀奉,恩紀謬加。今日叨濫,非由才舉,縱陛下特升庸薄,其如朝序何?臣九世祖信,名高往代,當時纔居選部,自後罕有繼蹤。臣
 遭逢成擢,沐浴恩造,累致非據,每切妨賢。臣雖無識,頗知審己,言行所踐,無期榮貴,豈意銓衡之重,妄委非才。且皇明御歷,事高昔代,羽儀世胄,帷幄名臣,若授受得宜,方為稱職。臣夙陶教義,必知不可。」後主曰:「選眾之舉,僉議所歸,昔毛玠雅量清恪,盧毓心平體正,王蘊銓量得地,山濤舉不失才,就卿而求,必兼此矣。且我與卿雖君臣禮隔,情分殊常,藻鏡人倫,良所期寄,亦以無慚則悊也。」



 察自居顯要,甚勵清潔,且廩錫以外,一不交通。嘗
 有私門生不敢厚餉,止送南布一端,花綀一匹。察謂之曰:「吾所衣著,止是麻布蒲綀,此物於吾無用。既欲相款接,幸不煩爾。」此人遜請,猶冀受納,察厲色驅出,因此伏事者莫敢饋遺。



 陳滅,入隋,開皇九年,詔授秘書丞,別敕成梁、陳二代史。又敕於朱華閣長參。文帝知察蔬菲,別日乃獨召入內殿,賜果菜,乃指察謂朝臣曰:「聞姚察學行當今無比,我平陳唯得此一人。」十三年,襲封北絳郡公。察往歲之聘周也,因得與父僧垣相見,將別之際,絕
 而復蘇,至是承襲,愈更悲感,見者莫不為之歔欷。



 察幼年嘗就鐘山明慶寺尚禪師受菩薩戒,及官陳,祿俸皆舍寺起造,并追為禪師樹碑,文甚遒麗。及是,遇見梁國子祭酒蕭子雲書此寺禪齋詩,覽之愴然,乃用蕭韻述懷為詠,詞又哀切,法俗益以此稱之。丁後母杜氏喪,解職。在服制之中,有白鳩巢於戶上。



 仁壽二年,詔曰:「前秘書丞北絳郡開國公姚察,彊學待問,博極群典,脩身立德,白首不渝,雖在哀疚,宜奪情禮,可員外散騎常侍,封
 如故。」又敕侍晉王昭讀。煬帝初在東宮,數被召見,訪以文籍。即位之始,詔授太子內舍人,餘並如故。車駕巡幸,恒侍從焉。及改易衣冠,刪正朝式,切問近對,察一人而已。



 年七十四,大業二年,終於東都,遺命薄葬,務從率儉。其略曰:「吾家世素士,自有常法。吾意斂以法服,並宜用布,土周於身。又恐汝等不忍行此,必不爾,須松板薄棺,纔可周身,土周於棺而已。葬日,止粗車,即送厝舊塋北。吾在梁世,當時年十四,就鐘山明慶寺尚禪師受菩薩
 戒,自爾深悟苦空,頗知回向矣。嘗得留連山寺,一去忘歸。及仕陳代,諸名流遂許與聲價,兼時主恩遇,宦途遂至通顯。



 自入朝來,又蒙恩渥。既牽纏人世,素志弗從。且吾習蔬菲五十餘年,既歷歲時,循而不失。瞑目之後,不須立靈,置一小床,每日設清水,六齋日設齋食果菜,任家有無,不須別經營也。」初,察願讀一藏經,並已究竟,將終,曾無痛惱,但西向坐,正念,云「一切空寂」。其後身體柔軟,顏色如恒。兩宮悼惜,賵賻甚厚。



 察性至孝,有人倫鑒
 識。沖虛謙遜,不以所長矜人。終日恬靜,唯以書記為樂,於墳籍無所不睹。每有製述,多用新奇,人所未見,咸重富博。且專志著書,白首不倦,手自抄撰,無時蹔輟。尤好研核古今,諟正文字,精采流贍,雖老不衰。兼諳識內典,所撰寺塔及眾僧文章,特為綺密,在位多所稱引,一善可錄,無不賞薦。



 若非分相干,咸以理遣。盡心事上,知無不為。侍奉機密,未嘗淹漏。且任遇已隆,衣冠攸屬,深懷退靜,避於聲勢。清潔自處,貲產每虛,或有勸營生計,笑
 而不答。



 穆於親屬,篤於舊故,所得祿賜,咸充周恤。



 後主所製文筆,卷軸甚多,乃別寫一本付察,有疑悉令刊定,察亦推心奉上,事在無隱。後主嘗從容謂朝士曰:「姚察達學洽聞,手筆典裁,求之於古,猶難輩匹,在於今世,足為師範。且訪對甚詳明,聽之使人忘倦。」察每製文筆,敕便索本,上曰:「我于姚察文章,非唯玩味無已,故是一宗匠。」



 徐陵名高一代,每見察製述,尤所推重。嘗謂子儉曰:「姚學士德學無前,汝可師之也。」尚書令江總與察尤篤
 厚善,每有製作,必先以簡察,然後施用。總為詹事時,嘗製登宮城五百字詩,當時副君及徐陵以下諸名賢並同此作。徐公後謂江曰:「我所和弟五十韻,寄弟集內。」及江編次文章,無復察所和本,述徐此意,謂察曰:「高才碩學,庶光拙文,今須公所和五百字,用偶徐侯章也。」察謙遜未付,江曰:「若不得公此製,僕詩亦須棄本,復乖徐公所寄,豈得見令兩失。」察不獲已,乃寫本付之。為通人推挹,例皆如此。



 所著《漢書訓纂》三十卷,《說林》十卷,《四聘》、《玉
 璽》、《建康三鐘》等記各一卷,悉窮該博,并《文集》二十卷,並行於世。察所撰梁、陳史雖未畢功,隋文帝開皇之時,遣內史舍人虞世基索本,且進上,今在內殿。梁、陳二史本多是察之所撰,其中序論及紀、傳有所闕者,臨亡之時,仍以體例誡約子思廉,博訪撰續,思廉泣涕奉行。思廉在陳為衡陽王府法曹參軍,轉會稽王主簿。入隋,補駐王府行參軍,掌記室,尋除河間郡司法。大業初,內史侍郎虞世基奏思廉踵成梁、陳二代史,自爾以來,稍就補
 續。



 史臣曰:江總持清標簡貴,加潤以辭採,及師長六官,雅允朝望。史臣先臣稟茲令德,光斯百行,可以厲風俗,可以厚人倫。至於九流、《七略》之書,名山石室之記,汲郡、孔堂之書,玉箱金板之文,莫不窮研旨奧,遍探坎井,故道冠人師,晉紳以為準的。既歷職貴顯,國典朝章,古今疑議,後主皆取先臣斷決焉。



\end{pinyinscope}