\article{卷九列傳第三侯瑱 歐陽頠 子紇 吳明徹 裴子烈}

\begin{pinyinscope}

 侯瑱,字伯
 玉,巴西充國人也。父弘遠,世為西蜀酋豪。蜀賊張文萼據白崖山,有眾萬人,梁益州刺史鄱陽王蕭範命弘遠討之。弘遠戰死,瑱固請復仇,每戰必先鋒陷
 陣,遂斬文萼,由是知名。因事範,範委以將帥之任,山谷夷獠不賓附者,並遣瑱征之。累功授輕車府中兵參軍、晉康太守。範為雍州刺史。瑱除超武將軍、馮翊太守。範遷鎮合肥,瑱又隨之。



 侯景圍臺城,範乃遣瑱輔其世子嗣,入援京邑。京城陷,瑱與嗣退還合肥,仍隨範徙鎮湓城。俄而範及嗣皆卒,瑱領其眾,依于豫章太守莊鐵。鐵疑之,瑱懼不自安,詐引鐵謀事,因而刃之,據有豫章之地。侯景將于慶南略地至豫章,城邑皆下,瑱窮蹙,乃降
 於慶。慶送瑱於景,景以瑱與己同姓,託為宗族,待之甚厚,留其妻子及弟為質。遣瑱隨慶平定蠡南諸郡。及景敗於巴陵,景將宋子仙、任約等並為西軍所獲,瑱乃誅景黨與,以應義軍,景亦盡誅其弟及妻子。梁元帝授瑱武臣將軍、南兗州刺史,郫縣侯,邑一千戶。仍隨都督王僧辯討景,恆為前鋒,每戰卻敵。



 既復臺城,景奔吳郡,僧辯使瑱率兵追之,與景戰於吳松江,大敗景,盡獲其軍實。



 進兵錢塘,景將謝答仁、呂子榮等皆降。以功除南豫
 州刺史,鎮于姑熟。



 承聖二年,齊遣郭元建出自濡須,僧辯遣瑱領甲士三千,築壘於東關以扞之,大敗元建。除使持節、鎮北將軍,給鼓吹一部,增邑二千戶。西魏來寇荊州,王僧辯以瑱為前軍,赴援,未至而荊州陷。瑱之九江,因衛晉安王還都。承制以瑱為侍中、使持節、都督江晉吳齊四州諸軍事、江州刺史,改封康樂縣公,邑五千戶,進號車騎將軍。司徒陸法和據郢州,引齊兵來寇,乃使瑱都督眾軍西討,未至,法和率其部北度入齊。齊遣
 慕容恃德鎮于夏首,瑱控引西還,水陸攻之,恃德食盡,請和,瑱還鎮豫章。僧辯使其弟僧心音率兵與瑱共討蕭勃,及高祖誅僧辯,僧心音陰欲圖瑱而奪其軍,瑱知之,盡收僧心音徒黨,僧心音奔齊。



 紹泰二年,以本號加開府儀同三司,餘並如故。是時,瑱據中流,兵甚彊盛,又以本事王僧辯,雖外示臣節,未有入朝意。初,餘孝頃為豫章太守,及瑱鎮豫章,乃於新吳縣別立城柵,與瑱相拒。瑱留軍人妻子於豫章,令從弟奫知後事,悉眾以攻孝頃。自夏
 及冬,弗能克,乃長圍守之,盡收其禾稼。奫與其部下侯方兒不協,方兒怒,率所部攻奫,虜掠瑱軍府妓妾金玉,歸于高祖。瑱既失根本,兵眾皆潰,輕歸豫章,豫章人拒之,乃趨湓城,投其將焦僧度。僧度勸瑱投齊,瑱以高祖有大量,必能容己,乃詣闕請罪,高祖復其爵位。



 永定元年,授侍中、車騎將軍。二年,進位司空。王琳至於沌口,周文育、侯安都並沒,乃以瑱為都督西討諸軍事。瑱至于梁山。世祖即位,進授太尉,增邑千戶。王琳至于柵口,又
 以瑱為都督,侯安都等並隸焉。瑱與琳相持百餘日,未決。



 天嘉元年二月,東關春水稍長,舟艦得通,琳引合肥漅湖之眾,舳艫相次而下,其勢甚盛。瑱率軍進獸檻洲,琳亦出船列于江西,隔洲而泊。明日合戰,琳軍少卻,退保西岸。及夕,東北風大起,吹其舟艦,舟艦並壞,沒于沙中,溺死者數十百人。



 浪大不得還浦,夜中又有流星墜于賊營。及旦風靜,琳入浦治船,以荻船塞于浦口,又以鹿角繞岸,不敢復出。是時,西魏遣大將軍史寧躡其上
 流,瑱聞之,知琳不能持久,收軍卻據湖浦,以待其敝。及史寧至,圍郢州,琳恐眾潰,乃率船艦來下,去蕪湖十里而泊,擊柝聞於軍中。明日,齊人遣兵數萬助琳,琳引眾向梁山,欲越官軍以屯險要。齊儀同劉伯球率兵萬餘人助琳水戰,行臺慕容恃德子子會領鐵騎二千,在蕪湖西岸博望山南,為其聲勢。瑱令軍中晨炊蓐食,分搥盪頓蕪湖洲尾以待之。將戰,有微風至自東南,眾軍施拍縱火。定州刺史章昭達乘平虜大艦,中江而進,發拍
 中于賊艦,其餘冒突、青龍,各相當值。又以牛皮冒蒙衝小船,以觸賊艦,并熔鐵灑之。琳軍大敗。其步兵在西岸者,自相蹂踐,馬騎並淖于蘆荻中,棄馬脫走以免者十二三。盡獲其舟艦器械,并禽齊將劉伯球、慕容子會,自餘俘馘以萬計。



 琳與其黨潘純陀等乘單舴艋冒陣走至湓城,猶欲收合離散,眾無附者,乃與妻妾左右十餘人入齊。



 其年,詔以瑱為都督湘、巴、郢、江、吳等五州諸軍事,鎮湓城。周將賀若敦、獨孤盛等寇巴、湘,又以瑱為西
 討都督,與盛戰於西江口,大敗盛軍,虜其人馬器械,不可勝數。以功授使持節、都督湘、桂、郢、巴、武、沅六州諸軍事、湘州刺史,改封零陵郡公,邑七千戶,餘如故。二年,以疾表求還朝。三月,於道薨,時年五十二。贈侍中、驃騎大將軍、大司馬,加羽葆、鼓吹、班劍二十人,給東園秘器,謚曰壯肅。其年九月,配享高祖廟庭。子凈藏嗣。



 凈藏尚世祖第二女富陽公主,以公主除員外散騎侍郎。太建三年卒,贈司徒主簿。凈藏無子,弟就襲封。



 歐陽頠,字靖世,長沙臨湘人也。為郡豪族。祖景達,梁代為本州治中。父僧寶,屯騎校尉。頠少質直有思理,以言行篤信著聞於嶺表。父喪毀瘠甚至。家產累積,悉讓諸兄。州郡頻辟不應,乃廬於麓山寺傍,專精習業,博通經史。年三十,其兄逼令從宦,起家信武府中兵參軍,遷平西邵陵王中兵參軍事。



 梁左衛將軍蘭欽之少也,與頠相善,故頠常隨欽征討。欽為衡州,仍除清遠太守。欽南征夷獠,擒陳文徹,所獲不可勝計,獻大銅鼓,累代所無,
 頠預其功。還為直閣將軍,仍除天門太守,伐蠻左有功。刺史廬陵王蕭續深嘉之,引為賓客。欽征交州,復啟頠同行。欽度嶺以疾終,頠除臨賀內史,啟乞送欽喪還都,然後之任。



 時湘衡之界五十餘洞不賓,敕令衡州刺史韋粲討之,粲委頠為都督,悉皆平殄。粲啟梁武,稱頠誠幹,降詔褒賞,仍加超武將軍,征討廣、衡二州山賊。



 侯景構逆,粲自解還都征景,以頠監衡州。京城陷後,嶺南互相吞併,蘭欽弟前高州刺史裕攻始興內史蕭紹基,奪
 其郡。裕以兄欽與頠有舊,遣招之,頠不從。



 乃謂使云:「高州昆季隆顯,莫非國恩,今應赴難援都,豈可自為跋扈。」及高祖入援京邑,將至始興,頠乃深自結託。裕遣兵攻頠,高祖援之,裕敗,高祖以王懷明為衡州刺史,遷頠為始興內史。高祖之討蔡路養、李遷仕也,頠率兵度嶺,以助高祖。及路養等平,頠有功,梁元帝承制以始興郡為東衡州,以頠為持節、通直散騎常侍、都督東衡州諸軍事、雲麾將軍、東衡州刺史,新豊縣伯,邑四百戶。



 侯景平,
 元帝遍問朝宰:「今天下始定,極須良才,卿各舉所知。」群臣未有對者。帝曰:「吾已得一人。」侍中王褒進曰:「未審為誰?」帝云:「歐陽頠公正有匡濟之才,恐蕭廣州不肯致之。」乃授武州刺史,尋授郢州刺史,欲令出嶺,蕭勃留之,不獲拜命。尋授使持節、散騎常侍、都督衡州諸軍事、忠武將軍、衡州刺史,進封始興縣侯。



 時蕭勃在廣州,兵彊位重,元帝深患之,遣王琳代為刺史。琳已至小桂嶺,勃遣其將孫信監州,盡率部下至始興,避琳兵鋒。頠別據一
 城,不往謁勃,閉門高壘,亦不拒戰。勃怒,遣兵襲頠,盡收其此貲財馬仗。尋赦之,還復其所,復與結盟。



 荊州陷,頠委質於勃。及勃度嶺出南康,以頠為前軍都督,頓豫章之苦竹灘,周文育擊破之,擒送于高祖,高祖釋之,深加接待。蕭勃死後,嶺南擾亂,頠有聲南土,且與高祖有舊,乃授頠使持節、通直散騎常侍、都督衡州諸軍事、安南將軍、衡州刺史,始興縣侯。未至嶺南,頠子紇已克定始興。及頠至嶺南,皆懾伏,仍進廣州,盡有越地。改授都督廣、
 交、越、成、定、明、新、高、合、羅、愛、建、德、宜、黃、利、安、石、雙十九州諸軍事、鎮南將軍、平越中郎將、廣州刺史,持節、常侍、侯並如故。王琳據有中流,頠自海道及東嶺奉使不絕。永定三年,進授散騎常侍,增都督衡州諸軍事,即本號開府儀同三司。世祖嗣位,進號征南將軍,改封陽山郡公,邑一千五百戶,又給鼓吹一部。



 初,交州刺史袁曇緩密以金五百兩寄頠,令以百兩還合浦太守龔翽,四百兩付兒智矩,餘人弗之知也。頠尋為蕭勃所破,貲財並盡,
 唯所寄金獨在。曇緩亦尋卒,至是頠並依信還之,時人莫不嘆伏。其重然諾如此。



 時頠弟盛為交州刺史,次弟邃為衡州刺史,合門顯貴,名振南土。又多致銅鼓、生口,獻奉珍異,前後委積,頗有助於軍國焉。頠以天嘉四年薨,時年六十六。贈侍中、車騎大將軍、司空、廣州刺史,謚曰穆。子紇嗣。



 紇字奉聖,頗有幹略。天嘉中,除黃門侍郎、員外散騎常侍。累遷安遠將軍、衡州刺史。襲封陽山郡公,都督交、廣
 等十九州諸軍事、廣州刺史。在州十餘年,威惠著於百越,進號輕車將軍。



 光大中,上流蕃鎮並多懷貳,高宗以紇久在南服,頗疑之。太建元年,下詔徵紇為左衛將軍。紇懼,未欲就徵,其部下多勸之反,遂舉兵攻衡州刺史錢道戢。道戢告變,乃遣儀同章昭達討紇,屢戰兵敗,執送京師,伏誅,時年三十三。家口籍沒。子詢以年幼免。



 吳明徹,字通昭,秦郡人也。祖景安,齊南譙太守。父樹,梁右軍將軍。明徹幼孤,性至孝,年十四,感墳塋未備,家貧
 無以取給,乃勤力耕種。時天下亢旱,苗稼焦枯,明徹哀憤,每之田中,號泣,仰天自訴。居數日,有自田還者,云苗已更生。明徹疑之,謂為紿己,及往田所,竟如其言。秋而大獲,足充葬用。時有伊氏者,善占墓,謂其兄曰:「君葬之日,必有乘白馬逐鹿者來經墳所,此是最小孝子大貴之徵。」至時果有此應,明徹即樹之最小子也。



 起家梁東宮直後。及侯景寇京師,天下大亂,明徹有粟麥三千餘斛,而鄰里飢餧,乃白諸兄曰:「當今草竊,人不圖久,柰何
 有此而不與鄉家共之?」於是計口平分,同其豊儉,群盜聞而避焉,賴以存者甚眾。



 及高祖鎮京口,深相要結,明徹乃詣高祖,高祖為之降階,執手即席,與論當世之務。明徹亦微涉書史經傳,就汝南周弘正學天文、孤虛、遁甲,略通其妙,頗以英雄自許,高祖深奇之。



 承聖三年,授戎昭將軍、安州刺史。紹泰初,隨周文育討杜龕、張彪等。東道平,授使持節、散騎常侍、安東將軍、南兗州刺史,封安吳縣侯。高祖受禪,拜安南將軍,仍與侯安都、周文育
 將兵討王琳。及眾軍敗沒,明徹自拔還京。世祖即位,詔以本官加右衛將軍。王琳敗,授都督武沅二州諸軍事、安西將軍、武州刺史,餘並如故。周遣大將軍賀若敦率馬步萬餘人奄至武陵,明徹眾寡不敵,引軍巴陵,仍破周別軍於雙林。



 天嘉三年,授安西將軍。及周迪反臨川,詔以明徹為安南將軍、江州刺史,領豫章太守,總督眾軍,以討迪。明徹雅性剛直,統內不甚和,世祖聞之,遣安成王頊慰曉明徹,令以本號還朝。尋授鎮前將軍。五年,
 遷鎮東將軍、吳興太守。及引辭之郡,世祖謂明徹曰:「吳興雖郡,帝鄉之重,故以相授。君其勉之!」及世祖弗豫,徵拜中領軍。



 廢帝即位,授領軍將軍,尋遷丹陽尹,仍詔明徹以甲仗四十人出入殿省。到仲舉之矯令出高宗也,毛喜知其謀,高宗疑懼,遣喜與明徹籌焉。明徹謂喜曰:「嗣君諒闇,萬機多闕,外鄰彊敵,內有大喪。殿下親實周、邵,德冠伊、霍,社稷至重,願留中深計,慎勿致疑。」



 及湘州刺史華皎陰有異志,詔授明徹使持節、散騎常侍、都督
 湘、桂、武三州諸軍事、安南將軍、湘州刺史,給鼓吹一部,仍與征南大將軍淳于量等率兵討皎。



 皎平,授開府儀同三司,進爵為公。太建元年,授鎮南將軍。四年,徵為侍中、鎮前將軍,餘並如故。



 會朝議北伐,公卿互有異同,明徹決策請行。五年,詔加侍中、都督征討諸軍事,仍賜女樂一部。明徹總統眾軍十餘萬,發自京師,緣江城鎮,相續降款。軍至秦郡,克其水柵。齊遣大將尉破胡將兵為援,明徹破走之,斬獲不可勝計,秦郡乃降。高宗以秦郡
 明徹舊邑,詔具太牢,令拜祠上塚,文武羽儀甚盛,鄉里以為榮。



 進克仁州,授徵北大將軍,進爵南平郡公,增邑并前二千五百戶。次平峽石岸二城。進逼壽陽,齊遣王琳將兵拒守。琳至,與刺史王貴顯保其外郭。明徹以琳初入,眾心未附,乘夜攻之,中宵而潰,齊兵退據相國城及金城。明徹令軍中益脩治攻具,又迮肥水以灌城。城中苦濕,多腹疾,手足皆腫,死者十六七。會齊遣大將軍皮景和率兵數十萬來援,去壽春三十里,頓軍不進。諸
 將咸曰:「堅城未拔,大援在近,不審明公計將安出?」明徹曰:「兵貴在速,而彼結營不進,自挫其鋒,吾知其不敢戰明矣。」於是躬擐甲胄,四面疾攻,城中震恐,一鼓而克,生禽王琳、王貴顯、扶風王可朱渾孝裕、尚書廬潛、左丞李騊駼,送京師。景和惶懼遁走,盡收其駝馬輜重。琳之獲也,其舊部曲多在軍中,琳素得士卒心,見者皆歔欷不能仰視。明徹慮其有變,遣左右追殺琳,傳其首。詔曰:「壽春者古之都會,襟帶淮、汝,控引河、洛,得之者安,是稱要
 害。侍中、使持節、都督征討諸軍事、征北大將軍、開府儀同三司南平郡開國公明徹,雄圖克舉,宏略蓋世。在昔屯夷,締構皇業,乃掩衡、岳,用清氛沴,實吞雲夢,即敘上游。今茲蕩定,恢我王略,風行電掃,貔虎爭馳,月陣雲梯,金湯奪險,威陵殊俗,惠漸邊氓。惟功與能,元戎是屬,崇麾廣賦,茂典恒宜,可都督、豫、合、建、光、朔、北徐六州諸軍事、車騎大將軍、豫州刺史,增封并前三千五百戶,餘如故。」詔遣謁者蕭淳風就壽陽冊明徹,於城南設壇,士卒
 二十萬,陳旗鼓戈甲,明徹登壇拜受,成禮而退,將卒莫不踴躍焉。



 初,秦郡屬南兗州,後隸譙州,至是,詔以譙之秦、盱眙、神農三郡還屬南兗州,以明徹故也。



 六年,自壽陽入朝,輿駕幸其第,賜鐘磬一部,米一萬斛,絹布二千匹。



 七年,進攻彭城。軍至呂梁,齊遣援兵前後至者數萬,明徹又大破之。八年,進位司空,餘如故。又詔曰:「昔者軍事建旌,交鋒作鼓,頃日訛替,多乖舊章,至於行陣,不相甄別。今可給司空、大都督泬鉞龍麾,其次將各有差。」尋
 授都督南北兗、南北青譙五州諸軍事、南兗州刺史。



 會周氏滅齊,高宗交事徐、兗,九年,詔明徹進軍北伐,令其世子戎昭將軍、員外散騎侍郎惠覺攝行州事。明徹軍至呂梁,周徐州總管梁士彥率眾拒戰,明徹頻破之,因退兵守城,不復敢出。明徹仍迮清水以灌其城,環列舟艦於城下,攻之甚急。周遣上大將軍王軌將兵救之。軌輕行自清水入淮口,橫流豎木,以鐵鎖貫車輪,遏斷船路。諸將聞之,甚惶恐,議欲破堰拔軍,以舫載馬。馬主裴
 子烈議曰:若決堰下船,船必傾倒,豈可得乎?不如前遣馬出,於事為允。」適會明徹苦背疾甚篤,知事不濟,遂從之,乃遣蕭摩訶帥馬軍數千前還。明徹仍自決其堰,乘水勢以退軍,冀其獲濟。及至清口,水勢漸微,舟艦並不得渡,眾軍皆潰,明徹窮蹙,乃就執。



 尋以憂憤遘疾,卒於長安,時年六十七。



 至德元年詔曰:「李陵矢竭,不免請降,于禁水漲,猶且生獲,固知用兵上術,世罕其人。故侍中、司空南平郡公明徹,爰初躡足,迄屆元戎,百戰百勝之
 奇,決機決死之勇,斯亦侔於古焉。及拓定淮、肥,長驅彭、汴,覆勍寇如舉毛,掃銳帥同沃雪,風威慴於異俗,功郊著於同文。方欲息駕陰山,解鞍浣海,既而師出已老,數亦終奇,不就結纓之功,無辭入褚之屈,望封崤之為易,冀平翟之非難,雖志在屈伸,而奄中霜露,埋恨絕域,甚可嗟傷。斯事已往,累逢肆赦,凡厥罪戾,皆蒙灑濯,獨此孤魂,未霑寬惠,遂使爵土湮沒,饗醊無主。棄瑕錄用,宜在茲辰,可追封邵陵縣開國侯,食邑一千戶,以其息惠
 覺為嗣。」



 惠覺歷黃門侍郎,以平章大寶功,授豊州刺史。



 明徹兄子超,字逸世。少倜儻,以幹略知名。隨明徹征伐,有戰功,官至忠毅將軍、散騎常侍、桂州刺史,封汝南縣侯,邑一千戶。卒,贈廣州刺史,謚曰節。



 裴子烈,字大士,河東聞喜人,梁員外散騎常侍猗之子。子烈少孤,有志氣。



 遇梁末喪亂,因習武藝,以驍勇聞。頻從明徹征討,所向必先登陷陣。官至電威將軍、北譙太守、岳陽內史,海安縣伯,邑三百戶。至德四年卒。



 史臣曰:高祖撥亂創基,光啟天歷,侯瑱、歐陽頠並歸身有道,位貴鼎司,美矣。吳明徹居將帥之任,初有軍功,及呂梁敗績,為失算也。斯以勇非韓、白,識異孫、吳,遂使蹙境喪師,金陵虛弱,禎明淪覆,蓋由其漸焉。



\end{pinyinscope}