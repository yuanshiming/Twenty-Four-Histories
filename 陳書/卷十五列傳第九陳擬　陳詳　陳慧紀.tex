\article{卷十五列傳第九陳擬 陳詳 陳慧紀}

\begin{pinyinscope}

 陳擬,字公正,高祖疏屬也。少孤貧,性質直彊記,高祖南征交趾,擬從焉。



 又進討侯景,至豫章,以擬為羅州刺史,與胡穎共知後事,并應接軍糧。高祖作鎮朱方,擬除步
 兵校尉、曲阿令。紹泰元年,授貞威將軍、義興太守。二年,入知衛尉事,除員外散騎常侍、明威將軍、雍州刺史資,監南徐州。



 高祖踐祚,詔曰:「維城宗子,實固有周,盤石懿親,用隆大漢。故會盟則異姓為後,啟土則非劉勿王,所以糾合枝幹,廣樹蕃屏,前王懋典,列代恒規。從子持節、員外散騎常侍、明威將軍、雍州刺史、監南徐州擬,持節、通直散騎侍郎、貞威將軍、北徐州刺史褒,從子晃、炅,從孫假節、員外散騎常侍、明威將軍訬,假節、信威將軍、北
 徐州刺史吉陽縣開國侯諠,假節、通直散騎侍郎、信武將軍祏,假節、散騎侍郎、雄信將軍、青州刺史、廣梁太守詳,貞戚將軍、通直散騎侍郎慧紀,從孫敬雅、敬泰,並枝戚密近,劬勞王室,宜列河山,以光利建。擬可永脩縣開國侯,褒鐘陵縣開國侯,晃建城縣開國侯,炅上饒縣開國侯,訬虔化縣開國侯,諠仍前封,祏豫章縣開國侯,詳遂興縣開國侯,慧紀宜黃縣開國侯,敬雅寧都縣開國侯,敬泰平固縣開國侯,各邑五百戶。」擬尋除輕車將軍,
 兼南徐州刺史,常侍如故。其年,授通直散騎常侍、中領軍。三年,復以本官監南徐州。世祖嗣位,除丹陽尹,常侍如故。坐事,又以白衣知郡,尋復本職。天嘉元年卒,時年五十八。



 贈領軍將軍,凶事所須,並官資給。謚曰定。二年,配享高祖廟廷。子黨嗣。



 陳詳,字文幾,少出家為桑門。善書記,談論清雅。高祖討侯景,召詳,令反初服,配以兵馬,從定京邑。高祖東征杜龕,詳別下安吉、原鄉、故鄣三縣。龕平,以功授散騎侍郎、
 假節、雄信將軍、青州刺史資,割故鄣、廣德置廣梁郡,以詳為太守。高祖踐祚,改廣梁為陳留,又以為陳留太守。永定二年,封遂興縣侯,食邑五百戶。其年除明威將軍、通直散騎常侍。三年,隨侯安都破王琳將常眾愛於宮亭湖。世祖嗣位,除宣城太守,將軍如故。王琳下據柵口,詳隨吳明徹襲湓城,取琳家口,不克,因入南湖,自鄱陽步道而歸。琳平,詳與明徹並無功。天嘉元年,隨例增邑并前一千五百戶。仍除通直散騎常侍,兼右衛將軍。三
 年,出為假節、都督吳州諸軍事、仁威將軍、吳州刺史。



 周迪據臨川舉兵,詳自州從他道襲迪於濡城別營,獲其妻子。迪敗走,詳還復本鎮。五年,周迪復出臨川,乃以詳為都督,率水步討迪。軍至南城,與賊相遇,戰敗,死之,時年四十二。以所統失律,無贈謚。子正理嗣。



 陳慧紀,字元方,高祖之從孫也。涉獵書史,負才任氣。高祖平侯景,慧紀從焉。尋配以兵馬。景平,從征杜龕。除貞威將軍、通直散騎常侍。高祖踐祚,封宜黃縣侯,邑五百
 戶,除黃門侍郎。世祖即位,出為安吉縣令。遷明威將軍軍副。司空章昭達征安蜀城,慧紀為水軍都督,於荊州燒青泥船艫。光大元年,以功除持節、通直散騎常侍、宣遠將軍、豊州刺史,增邑并前一千戶。太建十年,吳明徹北討敗績,以慧紀為持節、智武將軍、緣江都督、兗州刺史,增邑并前二千戶,餘如故。



 周軍乘勝據有淮南,江外騷擾,慧紀收集士卒,自海道還都。尋除使持節、散騎常侍、宣毅將軍、都督郢、巴二州諸軍事、郢州刺史,增邑并
 前二千五百戶。至德二年,遷使持節、散騎常侍、雲麾將軍、都督荊、信二州諸軍事、荊州刺史,賜女伎一部,增邑並前三千戶。禎明元年,蕭琮尚書左僕射安平王蕭巖、晉熙王蕭獻等,率其部眾男女二萬餘口,詣慧紀請降,慧紀以兵迎之。其年,以應接之功,加侍中、金紫光祿大夫、開府儀同三司、征西將軍、增邑并前六千戶,餘如故。



 及隋師濟江,元帥清河公楊素下自巴硤,慧紀遣其將呂忠肅、陸倫等拒之,戰敗,素進據馬頭。是時,隋將韓擒虎
 及賀若弼等已濟江據蔣山,慧紀聞之,留其長史陳文盛等居守,身率將士三萬人,樓船千餘乘,沿江而下,欲趣臺城。至漢口,為秦王軍所拒,不得進,因與湘州刺史晉熙王叔文、巴州刺史畢寶等請降。入隋,依例授儀同三司。頃之卒。子正平,頗有文學。



 史臣曰:《詩》云:「宗子維城,無俾城壞。」又曰:「綿綿瓜瓞,葛藟累之。」



 西京皆豊、沛故人,東都亦南陽多顯,有以哉!



\end{pinyinscope}