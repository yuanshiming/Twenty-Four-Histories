\article{卷二十列傳第十四到仲舉 韓子高 華皎}

\begin{pinyinscope}

 到仲舉,字德言,彭城武原人也。祖坦,齊中書侍郎。父洽,梁侍中。仲舉無他藝業,而立身耿正。釋褐著作佐郎、太子舍人、王府主簿。出為長城令,政號廉平。文帝居鄉里,
 嘗詣仲舉,時天陰雨,仲舉獨坐齋內,聞城外有簫鼓之聲,俄而文帝至,仲舉異之,乃深自結託。文帝又嘗因飲,夜宿仲舉帳中,忽有神光五采照于室內,由是祗承益恭。侯景之亂,仲舉依文帝。及景平,文帝為吳興郡守,以仲舉為郡丞,與潁川庾持俱為文帝賓客。文帝為宣毅將軍,以仲舉為長史,尋帶山陰令。文帝嗣位,授侍中,參掌選事。天嘉元年,守都官尚書,封寶安縣侯,邑五百戶。三年,除都官尚書。其年,遷尚書右僕射、丹陽尹,參掌並
 如故。尋改封建昌縣侯。仲舉既無學術,朝章非所長,選舉引用,皆出自袁樞。性疏簡,不干涉世務,與朝士無所親狎,但聚財酣飲而已。六年,秩滿,解尹。



 是時,文帝積年寢疾,不親御萬機,尚書中事,皆使仲舉斷決。天康元年,遷侍中、尚書僕射,參掌如故。文帝疾甚,入侍醫藥。及文帝崩,高宗受遺詔為尚書令入輔,仲舉與左丞王暹、中書舍人劉師知、殷不佞等,以朝望有歸,乃遣不佞矯宣旨遣高宗還東府。事發,師知下北獄賜死,暹、不佞並付
 治,乃以仲舉為貞毅將軍、金紫光祿大夫。



 初,仲舉子郁尚文帝妹信義長公主,官至中書侍郎,出為宣城太守,文帝配以士馬,是年遷為南康內史,以國哀未之任。仲舉既廢居私宅,與郁皆不自安。時韓子高在都,人馬素盛,郁每乘小輿蒙婦人衣與子高謀。子高軍主告言其事,高宗收子高、仲舉及郁並付廷尉。詔曰:「到仲舉庸劣小才,坐叨顯貴,受任前朝,榮寵隆赫,父參王政,子據大邦,禮盛外姻,勢均戚里。而肆此驕闇,凌傲百司,遏密之
 初,擅行國政,排黜懿親,欺蔑台袞。韓子高蕞爾細微,擢自卑末,入參禁衛,委以腹心,蜂蠆有毒,敢行反噬。仲舉、子高,共為表裏,陰構姦謀,密為異計。



 安成王朕之叔父,親莫重焉。受命導揚,稟承顧託,以朕沖弱,屬當保祐。家國安危,事歸宰輔,伊、周之重,物無異議,將相舊臣,咸知宗仰。而率聚凶徒,欲相掩襲,屯據東城,進逼崇禮,規樹仲舉,以執國權,陵斥司徒,意在專政,潛結黨附,方危社稷。賴祖宗之靈,姦謀顯露。前上虞令陸昉等具告其事,
 並有據驗,并剋今月七日,縱其凶謀。領軍將軍明徹,左衛將軍、衛尉卿寶安及諸公等,又並知其事。二三颭迹,彰於朝野,反道背德,事駭聞見。今大憝克殲,罪人斯得,並可收付廷尉,肅正刑書。罪止仲舉父子及子高三人而已,其餘一從曠蕩,並所不問。」



 仲舉及郁並於獄賜死,時年五十一。郁諸男女,以帝甥獲免。



 韓子高,會稽山陰人也。家本微賤。侯景之亂,寓在京都。景平,文帝出守吳興,子高年十六,為總角,容貌美麗,狀
 似婦人,於淮渚附部伍寄載慾還鄉。文帝見而問之,曰「能事我乎?」子高許諾。子高本名蠻子,文帝改名之。性恭謹,勤於侍奉,恒執備身刀及傳酒炙。文帝性急,子高恒會意旨。及長,稍習騎射,頗有膽決,願為將帥,及平杜龕,配以士卒。文帝甚寵愛之,未嘗離於左右。文帝嘗夢見騎馬登山,路危欲墮,子高推捧而升。



 文帝之討張彪也,沈泰等先降,文帝據有州城,周文育鎮北郭香巖寺。張彪自剡縣夜還襲城,文帝自北門出,倉卒闇夕,軍人擾
 亂,文育亦未測文帝所在,唯子高在側,文帝乃遣子高自亂兵中往見文育,反命,酬答於闇中,又往慰勞眾軍。文帝散兵稍集,子高引導入文育營,因共立柵。明日,與彪戰,彪將申縉復降,彪奔松山,浙東平。文帝乃分麾下多配子高,子高亦輕財禮士,歸之者甚眾。



 文帝嗣位,除右軍將軍。天嘉元年,封文招縣子,邑三百戶。王琳至于柵口,子高宿衛臺內。及琳平,子高所統益多,將士依附之者,子高盡力論進,文帝皆任使焉。二年,遷員外散騎
 常侍、壯武將軍、成州刺史。及征留異,隨侯安都頓桃支嶺巖下。時子高兵甲精銳,別御一營,單馬入陳,傷項之左,一髻半落。異平,除假節、貞毅將軍、東陽太守。五年,章昭達等自臨川征晉安,子高自安泉嶺會于建安,諸將中人馬最為彊盛。晉安平,以功遷通直散騎常侍,進爵為伯,增邑并前四百戶。六年,徵為右衛將軍,至都,鎮領軍府。文帝不豫,入侍醫藥。廢帝即位,遷散騎常侍,右衛如故,移頓于新安寺。



 高宗入輔,子高兵權過重,深不自
 安,好參訪臺閣,又求出為衡、廣諸鎮。光大元年八月,前上虞縣令陸昉及子高軍主告其謀反,高宗在尚書省,因召文武在位議立皇太子,子高預焉,平旦入省,執之,送廷尉,其夕與到仲舉同賜死,時年三十。父延慶及子弟並原宥。延慶因子高之寵,官至給事中、山陰令。



 華皎,晉陵暨陽人。世為小吏。皎梁代為尚書比部令史。侯景之亂,事景黨王偉。高祖南下,文帝為景所囚,皎遇文帝甚厚。景平,文帝為吳興太守,以皎為都錄事,軍府
 穀帛,多以委之。皎聰慧,勤於簿領。及文帝平杜龕,仍配以人馬甲仗,猶為都錄事。御下分明,善於撫養。時兵荒之後,百姓饑饉,皎解衣推食,多少必均,因稍擢為暨陽、山陰二縣令。文帝即位,除開遠將軍,左軍將軍。天嘉元年,封懷仁縣伯,邑四百戶。



 王琳東下,皎隨侯瑱拒之。琳平,鎮湓城,知江州事。時南州守宰多鄉里酋豪,不遵朝憲,文帝令皎以法馭之。王琳奔散,將卒多附於皎。三年,除假節、通直散騎常侍、仁武將軍、新州刺史資,監江州。
 尋詔督尋陽、太原、高唐、南北新蔡五郡諸軍事、尋陽太守,假節、將軍、州資、監如故。周迪謀反,遣其兄子伏甲於船中,偽稱賈人,欲於湓城襲皎。未發,事覺,皎遣人逆擊之,盡獲其船仗。其年,皎隨都督吳明徹征迪,迪平,以功授散騎常侍、平南將軍、臨川太守,進爵為侯,增封并前五百戶。未拜,入朝,仍授使持節、都督湘、巴等四州諸軍事、湘州刺史,常侍、將軍如故。



 皎起自下吏,善營產業,湘川地多所出,所得並入朝廷,糧運竹木,委輸甚眾;至于
 油蜜脯菜之屬,莫不營辦。又征伐川洞,多致銅鼓、生口,並送于京師。廢帝即位,進號安南將軍,改封重安縣侯,食邑一千五百戶。文帝以湘州出杉木舟,使皎營造大艦金翅等二百餘艘,并諸水戰之具,欲以入漢及峽。



 韓子高誅後,皎內不自安,繕甲聚徒,厚禮所部守宰。高宗頻命皎送大艦金翅等,推遷不至。光大元年,密啟求廣州,以觀時主意。高宗偽許之,而詔書未出。



 皎亦遣使句引周兵,又崇奉蕭巋為主,士馬甚盛。詔乃以吳明徹為
 湘州刺史,實欲以輕兵襲之。是時慮皎先發,乃前遣明徹率眾三萬,乘金翅直趨郢州,又遣撫軍大將軍淳于量率眾五萬,乘大艦以繼之,又令假節、冠武將軍楊文通別從安成步道出茶陵,又令巴山太守黃法慧別從宜陽出澧陵,往掩襲,出其不意,并與江州刺史章昭達、郢州刺史程靈洗等參謀討賊。



 是時蕭巋遣水軍為皎聲援。周武又遣其弟衛國公宇文直率眾屯魯山,又遣其柱國長胡公拓跋定人馬三萬,攻圍郢州。蕭巋授皎
 司空,巴州刺史戴僧朔,衡陽內史任蠻奴,巴陵內史潘智虔,岳陽太守章昭裕,桂陽太守曹宣,湘東太守錢明,並隸於皎。又長沙太守曹慶等本隸皎下,因為之用。帝恐上流宰守並為皎扇惑,乃下詔曰:「賊皎輿皁微賤,特逢獎擢,任據籓牧,屬當寵寄,背斯造育,興構姦謀,樹立蕭氏,盟約彰露,鴆毒存心,志危宗社,扇結邊境,驅逼士庶,蟻聚巴、湘,豕突鄢、郢,逆天反地,人神忿嫉。征南將軍量、安南將軍明徹、郢州刺史靈洗,受律專征,備盡心力,
 撫勞驍雄,舟師俱進,義烈爭奮,兇惡奔殄,獻捷相望,重氛載廓,言念泣罪,思與惟新。可曲赦湘、巴二州:凡厥為賊所逼制,預在凶黨,悉皆不問;其賊主帥節將,並許開恩出首,一同曠蕩。」



 先是,詔又遣司空徐度與楊文通等自安成步出湘東,以襲皎後。時皎陣于巴州之白螺,列舟艦與王師相持未決。及聞徐度趨湘州,乃率兵自巴、郢因便風下戰。



 淳于量、吳明徹等募軍中小艦,多賞金銀,令先出當賊大艦,受其拍。賊艦發拍皆盡,然後官軍
 以大艦拍之,賊艦皆碎,沒于中流。賊又以大艦載薪,因風放火,俄而風轉自焚,賊軍大敗。皎乃與戴僧朔單舸走,過巴陵,不敢登城,徑奔江陵。拓跋定等無復船渡,步趨巴陵,巴陵城邑為官軍所據,乃向湘州。至水口,不得濟,食且盡,詣軍請降。俘獲萬餘人,馬四千餘匹,送于京師。皎黨曹慶、錢明、潘智虔、魯閑、席慧略等四十餘人並誅,唯任蠻奴、章昭裕、曹宣、劉廣業獲免。



 戴僧朔,吳郡錢塘人也。有膂力,勇健善戰,族兄右將軍僧錫甚愛之。僧
 錫年老,征討恒使僧朔領眾。平王琳有功,僧錫卒,仍代為南丹陽太守。鎮采石。從征留異,侯安都於巖下出戰,為賊斫傷,僧朔單刀步援。以功除壯武將軍、北江州刺史,領南陵太守。又從征周迪有功,遷巴州刺史,假節、將軍如故。至是同皎為逆,伏誅於江陵。



 曹慶,本王琳將,蕭莊偽署左衛將軍、吳州刺史,部領亞於潘純陀。琳敗,文帝以配皎,官至長沙太守。錢明,本高祖主帥,後歷湘州諸郡守。潘智虔,純陀之子,少有志氣,年二十為巴陵內
 史。魯閑,吳郡錢塘人。席慧略,安定人。閑本張彪主帥,慧略王琳部下,文帝皆配于皎,官至郡守。並伏誅。



 章昭裕,昭達之弟;劉廣業,廣德之弟;曹宣,高祖舊臣;任蠻奴嘗有密啟於朝廷;由是並獲宥。



 史臣曰:韓子高、華皎雖復瓶筲小器,輿臺末品,文帝鑒往古之得人,救當今之急弊,達聰明目之術,安黎和眾之宜,寄以腹心,不論胄閥。皎早參近暱,嘗預艱虞,知其無隱,賞以悉力,有見信之誠,非可疑之地。皎據有上游,
 忠於文帝。



 仲舉、子高亦無爽於臣節者矣。



\end{pinyinscope}