\article{卷三十五列傳第二十九熊曇朗 周迪 留異 陳寶}

\begin{pinyinscope}

 熊曇朗,豫章南昌人也。世為郡著姓。曇朗跅弛不羈,有膂力,容貌甚偉。侯景之亂,稍聚少年,據豊城縣為柵,桀黠劫盜多附之。梁元帝以為巴山太守。荊州陷,曇朗兵
 力稍彊,劫掠鄰縣,縛賣居民,山谷之中,最為巨患。



 及侯瑱鎮豫章,曇朗外示服從,陰欲圖瑱。侯方兒之反瑱也,曇朗為之謀主。



 瑱敗,曇朗獲瑱馬仗子女甚多。及蕭勃踰領,歐陽頠為前軍,曇朗紿頠共往巴山襲黃法抃,又報法抃期共破頠,約曰「事捷與我馬仗」。及出軍,與頠掎角而進,又紿頠曰「餘孝頃欲相掩襲,須分留奇兵,甲仗既少,恐不能濟」。頠乃送甲三百領助之。及至城下,將戰,曇朗偽北,法抃乘之,頠失援,狼狽退衄,曇朗取其馬仗
 而歸。時巴山陳定亦擁兵立寨,曇朗偽以女妻定子。又謂定曰「周迪、餘孝頃並不願此婚,必須以彊兵來迎」。定乃遣精甲三百并土豪二十人往迎,既至,曇朗執之,收其馬仗,並論價責贖。



 紹泰二年,曇朗以南川豪帥,隨例除游騎將軍。尋為持節、飆猛將軍、桂州刺史資,領豊城令,歷宜新、豫章二郡太守。王琳遣李孝欽等隨餘孝頃於臨川攻周迪,曇朗率所領赴援。其年,以功除持節、通直散騎常侍、寧遠將軍,封永化縣侯,邑一千戶,給鼓吹
 一部。又以抗禦王琳之功,授平西將軍、開府儀同三司,餘並如故。



 及周文育攻餘孝勱於豫章,曇朗出軍會之,文育失利,曇朗乃害文育,以應王琳,事見文育傳。於是盡執文育所部諸將,據新淦縣,帶江為城。



 王琳東下,世祖征南川兵,江州刺史周迪、高州刺史黃法抃欲沿流應赴,曇朗乃據城列艦斷遏,迪等與法抃因帥南中兵築城圍之,絕其與琳信使。及王琳敗走,曇朗黨援離心,迪攻陷其城,虜其男女萬餘口。曇朗走入村中,村民斬
 之,傳首京師,懸于朱雀觀。於是盡收其宗族,無少長皆棄市。



 周迪,臨川南城人也。少居山谷,有膂力,能挽彊弩,以弋獵為事。侯景之亂,迪宗人周續起兵於臨川,梁始興王蕭毅以郡讓續,迪召募鄉人從之,每戰必勇冠眾軍。續所部渠帥,皆郡中豪族,稍驕橫,續頗禁之,渠帥等並怨望,乃相率殺續,推迪為主,迪乃據有臨川之地,築城于工塘。梁元帝授迪持節、通直散騎常侍、壯武將軍、高州
 刺史,封臨汝縣侯,邑五百戶。



 紹泰二年,除臨川內史。尋授使持節、散騎常侍、信威將軍、衡州刺史,領臨川內史。周文育之討蕭勃也,迪按甲保境,以觀成敗。文育使長史陸山才說迪,迪乃大出糧餉,以資文育。勃平,以功加振遠將軍,遷江州刺史。



 高祖受禪,王琳東下,迪欲自據南川,乃總召所部八郡守宰結盟,聲言入赴,朝廷恐其為變,因厚慰撫之。琳至湓城,新吳洞主餘孝頃舉兵應琳。琳以為南川諸郡可傳檄而定,乃遣其將李孝欽、樊
 猛等南徵糧餉。猛等與餘孝頃相合,眾且二萬,來趨工塘,連八城以逼迪。迪使周敷率眾頓臨川故郡,截斷江口,因出與戰,大敗之,屠其八城,生擒李孝欽、樊猛、餘孝頃送于京師,收其軍實,器械山積,並虜其人馬,迪並自納之。永定二年,以功加平南將軍、開府儀同三司,增邑一千五百戶,給鼓吹一部。



 世祖嗣位,進號安南將軍。熊曇朗之反也,迪與周敷、黃法抃等率兵共圍曇朗,屠之,盡有其眾。王琳敗後,世祖徵迪出鎮湓城,又徵其子入
 朝,迪趑趄顧望,並不至。豫章太守周敷本屬於迪。至是與黃法抃率其所部詣闕,世祖錄其破熊曇朗之功,並加官賞,迪聞之,甚不平,乃陰與留異相結。及王師討異,迪疑懼不自安,乃使其弟方興率兵襲周敷,敷與戰,破之。又別使兵襲華皎於湓城,事覺,盡為皎所擒。天嘉三年春,世祖乃下詔赦南川士民為迪所詿誤者,使江州刺史吳明徹都督眾軍,與高州刺史黃法抃、豫章太守周敷討迪。於是尚書下符曰:告臨川郡士庶:昔西京為盛,信、
 越背誕;東都中興,萌、寵違戾。是以鷹鸇競逐,菹醢極誅,自古有之,其來尚矣。逆賊周迪,本出輿臺,有梁喪亂,暴掠山谷。我高祖躬率百越,師次九川,濯其泥沙,假以毛羽,裁解豚佩,仍剖獸符,卵翼之恩,方斯莫喻。皇運肇基,頗布誠款,國步艱阻,竟微效力。龍節繡衣,藉王爵而御下,熊旗組甲,因地險而陵上。日者王琳始貳,蕭勃未夷,西結三湘,南通五嶺,衡、廣戡定,既安反側,江、郢紛梗,復生攜背,擁據一郡,茍且百心,志貌常違,言迹不副。特以
 新吳未靜,地遠兵彊,互相兼并,成其形勢。收獲器械,俘虜士民,並曰私財,曾無獻捷。時遣一介,終持兩端。朝廷光大含弘,引納崇遇,遂乃位等三槐,任均四嶽,富貴隆赫,超絕功臣。加以出師逾嶺,遠相響援,按甲斷江,翻然猜拒。故司空愍公,敦以宗盟,情同骨肉,城池連接,勢猶脣齒,彭亡之禍,坐觀難作,階此颭故,結其黨與。于時北寇侵軼,西賊憑陵,扉屨餱糧,悉以資寇,爵號軍容,一遵偽黨。及王師凱振,大定區中,天網恢弘,棄之度外,璽書
 綸誥,撫慰綢繆,冠蓋縉紳,敦授重疊。至於熊曇朗剿滅,豊城克定,蓋由儀同法抃之元功,安西周敷之效力,司勳有典,懋賞斯舊,惡直醜正,自為仇仇,悖禮姦謀,因此滋甚。征出湓城,歷年不就,求遣侍子,累載未朝。外誘逋亡,招集不逞,中調京輦,規冀非常。擅斂征賦,罕歸九府,擁遏二賈,害及四民,潛結賊異,共為表裏,同惡相求,密加應援。謂我六軍薄伐,三越未寧,屠破述城,虜縛妻息,分襲湓鎮,稱兵蠡邦,拘逼酋豪,攻圍城邑,幸國有備,應
 時衄殄。



 假節、通直散騎常侍、仁武將軍、尋陽太守懷仁縣伯華皎,明威將軍、廬陵太守益陽縣子陸子隆,並破賊徒,剋全郡境。持節、散騎常侍、安西將軍、定州刺史、領豫章太守西豊縣侯周敷,躬扞溝壘,身當矢石,率茲義勇,以寡摧眾,斬馘萬計,俘虜千群。迪方收餘燼,還固墉堞。使持節、安南將軍、開府儀同三司、高州刺史新建縣侯法抃,雄績早宣,忠誠夙著,未奉王命,前率義旅,既援敷等,又全子隆,裹糧擐甲,仍躡飛走,批熊之旅,驅馳越
 電,振武之眾,叱吒移山,以此追奔,理無遺類。雖復朽株將拔,非待尋斧,落葉就殞,無勞烈風;但去草絕根,在於未蔓,撲火止燎,貴乎速滅,分命將帥,實資英果。今遣鎮南儀同司馬、湘東公相劉廣德,兼平西司馬孫曉,北新蔡太守魯廣達,持節、安南將軍、吳州刺史彭澤縣侯魯悉達,甲士萬人,步出興口。又遣前吳興太守胡鑠,樹功將軍、前宣城太守錢法成,天門、義陽二郡太守樊毅,雲麾將軍、合州刺史南固縣侯焦僧度,嚴武將軍、建州刺
 史辰縣子張智達,持節、都督江吳二州諸軍事、安南將軍、江州刺史安吳縣侯吳明徹,樓艦馬步,直指臨川。前安成內史劉士京,巴山太守蔡僧貴,南康內史劉峰,廬陵太守陸子隆,安成內史闕慎,並受儀同法抃節度,同會故郡。又命尋陽太守華皎,光烈將軍、巴州刺史潘純陀,平西將軍、郢州刺史欣樂縣侯章昭達,並率貔豹,逕造賊城。使持節、散騎常侍、鎮南將軍、開府儀同三司、湘州刺史湘東郡公度,分遣偏裨,相繼上道,戈船蔽水,彀
 騎彌山。又詔鎮南將軍、開府儀同三司歐陽頠,率其子弟交州刺史盛、新除太子右率邃、衡州刺史侯曉等,以勁越之兵,踰嶺北邁。



 千里同期,百道俱集,如脫稽誅,更淹旬晦,司空、大都督安都已平賊異,凱歸非久,飲至禮畢,乘勝長驅,剿撲凶醜,如燎毛髮。已有明詔,罪唯迪身,黎民何辜,一皆原宥。其有因機立功,賞如別格;執迷不改,刑茲罔赦。



 吳明徹至臨川,令眾軍作連城攻迪,相拒不能剋,世祖乃遣高宗總督討之,迪眾潰,妻子悉擒,乃
 脫身踰嶺之晉安,依于陳寶應。寶應以兵資迪,留異又遣第二子忠臣隨之。



 明年秋,復越東興嶺,東興、南城、永成縣民,皆迪故人,復共應之。世祖遣都督章昭達征迪,迪又散于山谷。初,侯景之亂也,百姓皆棄本業,群聚為盜,唯迪所部,獨不侵擾,並分給田疇,督其耕作,民下肆業,各有贏儲,政教嚴明,徵斂必至,餘郡乏絕者,皆仰以取給。迪性質朴,不事威儀,冬則短身布袍,夏則紫紗襪腹,居常徒跣,雖外列兵衛,內有女伎,挼繩破篾,傍若無人。
 然輕財好施,凡所周贍,毫釐必鈞,訥於言語,而襟懷信實,臨川人皆德之。至是並共藏匿,雖加誅戮,無肯言者。昭達仍度嶺,頓于建安,與陳寶應相抗,迪復收合出東興。時宣城太守錢肅鎮東興,以城降迪。吳州刺史陳詳,率師攻迪,詳兵大敗,虔化侯陳訬、陳留太守張遂並戰死,於是迪眾復振。世祖遣都督程靈洗擊破之,迪又與十餘人竄于山穴中。日月轉久,相隨者亦稍苦之。後遣人潛出臨川郡市魚鮭,足痛,舍於邑子,邑子告臨川太
 守駱牙,牙執之,令取迪自效。因使腹心勇士隨入山中,誘迪出獵,伏兵於道傍,斬之,傳首京都,梟于朱雀觀三日。



 留異,東陽長山人也。世為郡著姓。異善自居處,言語醞藉,為鄉里雄豪。多聚惡少,陵侮貧賤,守宰皆患之。梁代為蟹浦戍主,歷晉安、安固二縣令。侯景之亂,還鄉里,召募士卒,東陽郡丞與異有隙,引兵誅之,及其妻子。太守沈巡援臺,讓郡於異,異使兄子超監知郡事,率兵隨巡
 出都。



 及京城陷,異隨臨城公蕭大連,大連板為司馬,委以軍事。異性殘暴,無遠略,督責大連軍主及以左右私樹威福,眾並患之。會景將軍宋子仙濟浙江,異奔還鄉里,尋以其眾降于子仙。是時大連亦趣東陽之信安嶺,欲之鄱陽,異乃為子仙鄉導,令執大連。侯景署異為東陽太守,收其妻子為質。景行臺劉神茂建義拒景,異外同神茂,而密契於景。及神茂敗績,為景所誅,異獨獲免。



 侯景平後,王僧辯使異慰勞東陽,仍糾合鄉閭,保據巖
 阻,其徒甚盛,州郡憚焉。元帝以為信安令。荊州陷,王僧辯以異為東陽太守。世祖平定會稽,異雖轉輸糧饋,而擁擅一郡,威福在己。紹泰二年,以應接之功,除持節、通直散騎常侍、信武將軍、縉州刺史,領東陽太守,封永興縣侯,邑五百戶。其年遷散騎常侍、信威將軍,增邑三百戶,餘並如故。又以世祖長女豊安公主配異第三子貞臣。永定二年,徵異為使持節、散騎常侍、都督南徐州諸軍事、平北將軍、南徐州刺史,異遷延不就。



 世祖即位,改
 授都督縉州諸軍事、安南將軍、縉州刺史,領東陽太守。異頻遣其長史王澌為使入朝,澌每言朝廷虛弱,異信之,雖外示臣節,恒懷兩端,與王琳自鄱陽信安嶺潛通信使。王琳又遣使往東陽,署守宰。及琳敗,世祖遣左衛將軍沈恪代異為郡,實以兵襲之。異出下淮抗禦,恪與戰,敗績,退還錢塘,異乃表啟遜謝。是時眾軍方事湘、郢,乃降詔書慰喻,且羈縻之,異亦知朝廷終討於己,乃使兵戍下淮及建德,以備江路。湘州平,世祖乃下詔曰:昔
 四罪難弘,大媯之所無赦,九黎亂德,少昊之所必誅。自古皇王,不貪征伐,茍為時蠹,事非獲已。逆賊留異,數應亡滅,繕甲完聚,由來積年。進謝群龍,自躍於千里,退懷首鼠,恒持於百心。中歲密契番禺,既弘天網,賜以名爵,敦以國姻,儻望懷音,猶能革面。王琳竊據中流,翻相應接,別引南川之嶺路,專為東道之主人,結附凶渠,唯欣禍亂。既妖氛盪定,氣沮心孤,類傷鳥之驚弦,等窮獸之謀觸。雖復遣家入質,子陽之態轉遒;侍子還朝,隗囂之
 心方熾。



 朕志相成養,不計疵慝,披襟解帶,敦喻殷勤。蜂目彌彰,梟聲無改,遂置軍江口,嚴戍下淮,顯然反叛,非可容匿。且縉邦膏腴,稽南殷曠,永割王賦,長壅國民,竹箭良材,絕望京輦,萑蒲小盜,共肆貪殘,念彼餘,兼其慨息。西戎屈膝,自款重關,秦國依風,並輸侵地,三邊已乂,四表咸寧,唯此微妖,所宜清殄。



 可遣使持節、都督南徐州諸軍事、征北將軍、司空、南徐州刺史桂陽郡開國公安都指往擒戮,罪止異身,餘無所問。



 異本謂官軍自
 錢塘江而上,安都乃由會稽、諸暨步道襲之。異聞兵至,大恐,棄郡奔于桃支嶺,於嶺口立柵自固。明年春,安都大破其柵,異與第二子忠臣奔于陳寶應,於是虜其餘黨男女數千人。天嘉五年,陳寶應平,并擒異送都,斬于建康市,子姪及同黨無少長皆伏誅,唯第三子貞臣以尚主獲免。



 陳寶應,晉安候官人也。世為閩中四姓。父羽,有材幹,為郡雄豪。寶應性反覆,多變詐。梁代晉安數反,累殺郡將,
 羽初並扇惑合成其事,後復為官軍鄉導破之,由是一郡兵權皆自己出。



 侯景之亂,晉安太守、賓化侯蕭雲以郡讓羽,羽年老,但治郡事,令寶應典兵。



 是時東境饑饉,會稽尤甚,死者十七八,平民男女,並皆自賣,而晉安獨豊沃。寶應自海道寇臨安、永嘉及會稽、餘姚、諸暨,又載米粟與之貿易,多致玉帛子女,其有能致舟乘者,亦並奔歸之,由是大致貲產,士眾彊盛。侯景平,元帝因以羽為晉安太守。



 高祖輔政,羽請歸老,求傳郡于寶應,高祖
 許之。紹泰元年,授壯武將軍、晉安太守,尋加員外散騎常侍。二年,封候官縣侯,邑五百戶。時東西嶺路,寇賊擁隔,寶應自海道趨于會稽貢獻。高祖受禪,授持節、散騎常侍、信武將軍、閩州刺史,領會稽太守。世祖嗣位,進號宣毅將軍,又加其父光祿大夫,仍命宗正錄其本系,編為宗室,并遣使條其子女,無大小並加封爵。



 寶應娶留異女為妻,侯安都之討異也,寶應遣兵助之,又資周迪兵糧,出寇臨川。及都督章昭達於東興、南城破迪,世祖因
 命昭達都督眾軍,由建安南道渡嶺,又命益州刺史領信義太守餘孝頃都督會稽、東陽、臨海、永嘉諸軍自東道會之,以討寶應,並詔宗正絕其屬籍。於是尚書下符曰:告晉安士庶:昔隴西旅拒,漢不稽誅,遼東叛換,魏申宏略。若夫無諸漢之策勳,有扈夏之同姓,至於納吳濞之子,致橫海之師,違姒啟之命,有《甘誓》之討。



 況乃族不繫於宗盟,名無紀於庸器,而顯成三叛,颭深四罪者乎?



 案閩寇陳寶應父子,卉服支孽,本迷愛敬。梁季喪亂,閩
 隅阻絕,父既豪俠,扇動蠻陬,椎髻箕坐,自為渠帥,無聞訓義,所資姦諂,爰肆蜂豺,俄而解印。炎行方謝,網漏吞舟,日月居諸,棄之度外。自東南王氣,實表聖基,斗牛聚星,允符王迹,梯山航海,雖若款誠,擅割瑰珍,竟微職貢。朝廷遵養含弘,寵靈隆赫,起家臨郡,兼晝繡之榮,裂地置州,假籓麾之盛。即封戶牖,仍邑櫟陽,乘華轂者十人,保弊廬而萬石。又以盛漢君臨,推恩婁敬,隆周朝會,乃長滕侯,由是紫泥青紙,遠賁恩澤,鄉亭龜組,頒及嬰孩。
 自谷遷喬,孰復為擬,而苞藏鴆毒,敢行狼戾。連結留異,表裏周迪,盟歃婚姻,自為脣齒,屈彊山谷,推移歲時。及我彀騎防山,定秦望之西部,戈船下瀨,克匯澤之南川,遂敢舉斧,並助凶孽,莫不應弦摧衄,盡殪醜徒。每以罪在酋渠,憫茲驅逼,所收俘馘,並勒矜放。仍遣中使,爰降詔書,天網恢弘,猶許改思。異既走險,迪又逃刑,誑侮王人,為之川藪,遂使袁熙請席,遠歎頭行,馬援觀蛙,猶安井底。至如遏絕九賦,剽掠四民,闔境資財,盡室封奪,凡
 厥蒼頭,皆略黔首。蝥賊相扇,葉契連蹤,乃復踰超瀛冥,寇擾浹口,侵軼嶺嶠,掩襲述城,縛掠吏民,焚燒官寺,此而可縱,孰不可容?



 今遣沙州刺史俞文冏,明威將軍程文季,假節、宣猛將軍、成州刺史甘他,假節、雲旗將軍譚瑱,假節、宣猛將軍、前監臨海郡陳思慶,前軍將軍徐智遠,明毅將軍宜黃縣開國侯慧紀,開遠將軍、新除晉安太守趙彖,持節、通直散騎常侍、壯武將軍、定州刺史康樂縣開國侯林馮,假節、信威將軍、都督東討諸軍事、益
 州刺史餘孝頃,率羽林二萬,蒙衝蓋海,乘跨滄波,掃蕩巢窟。此皆明恥教戰,濡須鞠旅,累從楊僕,亟走孫恩,斬蛟中流,命馮夷而鳴鼓,黿鼉為駕,闌方壺而建旗。



 義安太守張紹賓,忠誠款到,累使求軍,南康內史裴忌,新除輕車將軍劉峰,東衡州刺史錢道戢,並即遣人仗,與紹賓同行。



 故司空歐陽公,昔有表奏,請宣薄伐,遙途意合,若伏波之論兵,長逝遺誠,同子顏之勿赦。征南薨謝,上策無忘,周南餘恨,嗣子弗忝。廣州刺史歐陽紇,克符家
 聲,聿遵廣略,舟師步卒,二萬分趨,水扼長鯨,陸製封犬希,董率衡、廣之師,會我六軍。



 潼州刺史李者,明州刺史戴晃,新州刺史區白獸,壯武將軍脩行師,陳留太守張遂,前安成內史闕慎,前廬陵太守陸子隆,前豫寧太守任蠻奴,巴山太守黃法慈,戎昭將軍、湘東公世子徐敬成,吳州刺史魯廣達,前吳州刺史遂興縣開國侯詳,使持節、都督征討諸軍事、散騎常侍、護軍將軍昭達,率緹騎五千,組甲二萬,直渡邵武,仍頓晉安。按轡揚旌,夷山堙
 谷,指期掎角,以制飛走。



 前宣城太守錢肅,臨川太守駱牙,太子左衛率孫詡,尋陽太守莫景隆,豫章太守劉廣德,並隨機鎮遏,絡驛在路。



 使持節、散騎常侍、鎮南將軍、開府儀同三司、江州刺史新建縣開國侯法抃,戒嚴中流,以為後殿。



 斧鉞所臨,罪唯元惡及留異父子。其黨主帥,雖有請泥函谷,相背淮陰,若能翻然改圖,因機立效,非止肆眚,仍加賞擢。其建、晉士民,久被驅迫者,大軍明加撫慰,各安樂業,流寓失鄉,既還本土。其餘立功立事,
 已具賞格。若執迷不改,同惡趑趄,斧鉞一臨,罔知所赦。



 昭達既剋周迪,踰東興嶺,頓于建安,餘孝頃又自臨海道襲于晉安,寶應據建安之湖際,逆拒王師,水陸為柵。昭達深溝高壘,不與戰,但命軍士伐木為簰。俄而水盛,乘流放之,突其水柵,仍水步薄之,寶應眾潰,身奔山草間,窘而就執,并其子弟二十人送都,斬于建康市。



 史臣曰:梁末之災沴,群凶競起,郡邑巖穴之長,村屯鄔壁之豪,資剽掠以致彊,恣陵侮而為大。高祖應期撥亂,
 戡定安輯,熊曇朗、周迪、留異、陳寶應雖身逢興運,猶志在亂常。曇朗奸慝翻覆,夷滅斯為幸矣。寶應及異,世祖或敦以婚姻,或處其類族,豈有不能威制,蓋以德懷也。遂乃背恩負義,各立異圖,地匪淮南,有為帝之志,勢非庸、蜀,啟自王之心。嗚呼,既其迷暗所致,五宗屠剿,宜哉!



\end{pinyinscope}