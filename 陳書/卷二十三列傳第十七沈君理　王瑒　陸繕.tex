\article{卷二十三列傳第十七沈君理 王瑒 陸繕}

\begin{pinyinscope}

 沈君理,字仲倫,吳興人也。祖僧畟,梁左民尚書。父巡,素與高祖相善,梁太清中為東陽太守。侯景平後,元帝徵為少府卿。荊州陷,蕭詧署金紫光祿大夫。



 君理美風儀,
 博涉經史,有識鑒。起家湘東王法曹參軍。高祖鎮南徐州,巡遣君理自東陽謁于高祖,高祖器之,命尚會稽長公主,辟為府西曹掾,稍遷中衛豫章王從事中郎,尋加明威將軍,兼尚書吏部侍郎。遷給事黃門侍郎,監吳郡。高祖受禪,拜駙馬都尉,封永安亭侯。出為吳郡太守。是時兵革未寧,百姓荒弊,軍國之用,咸資東境,君理招集士卒,脩治器械,民下悅附,深以幹理見稱。



 世祖嗣位,徵為侍中,遷守左民尚書,未拜,為明威將軍、丹陽尹。天嘉
 三年,重授左民尚書,領步兵校尉,尋改前軍將軍。四年,侯安都徙鎮江州,以本官監南徐州。六年,出為仁威將軍、東陽太守。天康元年,以父憂去職。君理因自請往荊州迎喪柩,朝議以在位重臣,難令出境,乃遣令長兄君嚴往焉。及還,將葬,詔贈巡侍中、領軍將軍,謚曰敬子。其年起君理為信威將軍、左衛將軍。又起為持節、都督東衡、衡二州諸軍事、仁威將軍、東衡州刺史,領始興內史。又起為明威將軍、中書令。前後奪情者三,並不就。



 太建元
 年,服闋,除太子詹事,行東宮事,遷吏部尚書。二年,高宗以君理女為皇太子妃,賜爵望蔡縣侯,邑五百戶。四年,加侍中。五年,遷尚書右僕射,領吏部,侍中如故。其年有疾,輿駕親臨視,九月卒,時年四十九。詔贈侍中、太子少傅。喪事所須,隨由資給。重贈翊左將軍、開府儀同三司,侍中如故。謚曰貞憲。



 君理子遵儉早卒,以弟君高子遵禮為嗣。



 君理第五叔邁,亦方正有幹局,仕梁為尚書金部郎。永定中,累遷中書侍郎。



 天嘉中,歷太僕、廷尉,出為
 鎮東始興王長史、會稽郡丞,行東揚州事。光大元年,除尚書吏部郎。太建元年,遷為通直散騎常侍,侍東宮。二年卒,時年五十二,贈散騎常侍。



 君理第六弟君高,字季高,少知名,性剛直,有吏能。以家門外戚,早居清顯,歷太子舍人、洗馬、中舍人、高宗司空府從事中郎、廷尉卿。太建元年,東境大水,百姓饑弊,乃以君高為貞威將軍、吳令。尋除太子中庶子、尚書吏部郎、衛尉卿。



 出為宣遠將軍、平南長沙王長史、南海太守,行廣州事。以女為王妃,
 固辭不行,復為衛尉卿。八年,詔授持節、都督廣等十八州諸軍事、寧遠將軍、平越中郎將、廣州刺史。嶺南俚、獠世相攻伐,君高本文吏,無武幹,推心撫御,甚得民和。十年,卒于官,時年四十七。贈散騎常侍,謚曰祁子。



 王瑒,字子璵,司空沖之第十二子也。沈靜有器局,美風儀,舉止醞藉。梁大同中,起家秘書郎,遷太子洗馬。元帝承制,徵為中書侍郎,直殿省,仍掌相府管記。出為東宮內史,遷太子中庶子。丁所生母憂,歸于丹陽。江陵陷,梁
 敬帝承制,除仁威將軍、尚書吏部郎中。貞陽侯僭位,以敬帝為太子,授瑒散騎常侍,侍東宮。



 尋遷長史兼侍中。



 高祖入輔,以為司徒左長史。永定元年,遷守五兵尚書。世祖嗣位,授散騎常侍,領太子庶子,侍東宮。遷領左驍騎將軍、太子中庶子,常侍、侍中如故。瑒為侍中六載,父沖嘗為瑒辭領中庶子,世祖顧謂沖曰:「所以久留瑒於承華,政欲使太子微有瑒風法耳。」廢帝嗣位,以侍中領左驍騎將軍。光大元年,以父憂去職。



 高宗即位,太建
 元年,復除侍中,領左驍騎將軍。遷度支尚書,領羽林監。出為信威將軍、雲麾始興王長史,行州府事。未行,遷中書令,尋加散騎常侍,除吏部尚書,常侍如故。瑒性寬和,及居選職,務在清靜,謹守文案,無所抑揚。尋授尚書右僕射,未拜,加侍中,遷左僕射,參掌選事,侍中如故。瑒兄弟三十餘人,居家篤睦,每歲時饋遺,遍及近親,敦誘諸弟,並稟其規訓。太建八年卒,時年五十四。贈侍中、特進、護軍將軍。喪事隨所資給。謚曰光子。



 瑒第十三弟瑜,字子
 珪,亦知名,美容儀,早歷清顯,年三十,官至侍中。永定元年,使於齊,以陳郡袁憲為副,齊以王琳之故,執而囚之。齊文宣帝每行,載死囚以從,齊人呼曰「供御囚」,每有他怒,則召殺之,以快其意。瑜及憲並危殆者數矣,齊僕射楊遵彥憫其無辜,每救護之。天嘉二年還朝,詔復侍中,頃之卒,時年四十。贈本官,謚曰貞子。



 陸繕,字士繻,吳郡吳人也。祖惠曉,齊太常卿。父任,梁御史中丞。繕幼有志尚,以雅正知名。起家梁宣惠武陵王
 法曹參軍。承聖中,授中書侍郎,掌東宮管記。江陵陷,繕微服遁還京師。紹泰元年,除司徒右長史,御史中丞,以父任所終,固辭不就。高祖引繕為司徒司馬,遷給事黃門侍郎、領步兵校尉、通直散騎常侍,兼侍中。永定元年,遷侍中。時留異擁割東陽,新安人向文政與異連結,因據本郡,朝廷以繕為貞威將軍、新安太守。



 世祖嗣位,徵為太子中庶子,領步兵校尉,掌東宮管記。繕儀表端麗,進退閑雅,世祖使太子諸王咸取則焉。其趨步躡履,皆
 令習繕規矩。除尚書吏部郎中,步兵如故,仍侍東宮。陳寶應平後,出為貞毅將軍、建安太守。秩滿,為散騎常侍、御史中丞,猶以父之所終,固辭,不許,乃權換廨宇徙居之。



 太建初,遷度支尚書、侍中、太子詹事,行東宮事,領揚州大中正。及太子親蒞庶政,解行事,加散騎常侍,改加侍中。遷尚書右僕射,尋遷左僕射,參掌選事,侍中如故。更為尚書僕射,領前將軍。重授左僕射,領揚州大中正,別敕令與徐陵等七人參議政事。十二年卒,時年六十
 三。贈侍中、特進、金紫光祿大夫,謚曰安子。太子以繕東宮舊臣,特賜祖奠。



 繕子辯惠,年數歲,詔引入殿內,辯惠應對進止有父風,高宗因賜名辯惠,字敬仁云。



 繕兄子見賢,亦方雅,高宗為揚州牧,乃以為治中從事史,深被知遇。歷給事黃門侍郎,長沙、鄱陽二王長史,帶尋陽太守,少府卿。太建十年卒,時年五十。



 贈廷尉卿,謚曰平子。



 史臣曰:夫衣冠雅道,廊廟嘉猷,諒以操履敦修,局宇詳正。經曰「容止可觀」,《詩》言「其儀罔忒」,彼三子者,其有斯風
 焉。



\end{pinyinscope}