\article{卷五本紀第五 文帝}

\begin{pinyinscope}

 太祖文皇帝諱義隆,小字車兒,武帝第三子也。晉安帝義熙三年,生於京口。



 盧循之難,上年四歲,高祖使諮議參軍劉粹輔上鎮京城。十一年,封彭城縣公。高祖伐羌
 至彭城,將進路,板上行冠軍將軍留守。晉朝加授使持節、監徐兗青冀四州諸軍事、徐州刺史,將軍如故。關中平定,高祖還彭城,又授監司州豫州之淮西兗州之陳留諸軍事、前將軍、司州刺史,持節如故,將鎮洛陽。仍改授都督荊益寧雍梁秦六州豫州之河南廣平揚州之義成松滋四郡諸軍事、西中郎將、荊州刺史,持節如故。永初元年,封宜都王,食邑三千戶。進督北秦,并前七州。進號鎮西將軍,給鼓吹一部。又進督湘州,是歲入朝,時
 年十四。長七尺五寸,博涉經史,善隸書。



 景平二年七月中,少帝廢。百官備法駕奉迎,入奉皇統。行臺至江陵,進璽紱。



 侍中臣琇、散騎常侍臣嶷之、中書監尚書令護軍將軍建城縣公臣亮、左衛將軍臣景仁、給事中游擊將軍龍鄉縣侯臣隆、越騎校尉都亭侯臣綱、給事黃門侍郎臣孔璩之、散騎侍郎臣劉思考、員外散騎侍郎臣潘盛、中書侍郎臣何尚之、羽林監封陽縣開國侯臣蕭思話、長兼尚書左丞德陽縣侯臣孫康、吏部郎中騎都尉
 臣張茂度、儀曹郎中臣徐長琳、倉部郎中臣庾俊之、都官郎中臣袁洵等上表曰:「臣聞否泰相革,數窮則變,天道所以不謟,卜世所以靈長。乃者運距陵夷,王室艱晦,九服之命,靡所適歸;高祖之業,將墜于地。賴基厚德深,人神同獎,社稷以寧,有生獲乂。伏惟陛下君德自然,聖明在御,孝悌著於家邦,風猷宣於蕃牧。是以徵祥雜沓,符瑞輝。宗廟神靈,乃眷西顧;萬邦黎獻,望景託生。臣等忝荷朝列,豫充將命,復集休明之運,再
 睹太平之業。行臺至止,瞻望城闕,不勝喜說鳧藻之情,謹詣門拜表以聞。」上答曰:「皇運艱弊,數鐘屯夷,仰惟崇基,感尋國故,永慕厥躬,悲慨交集。賴七百祚永,股肱忠賢,故能休否以泰,天人式序。猥以不德,謬降大命,顧己兢悸,何以克堪。輒當暫歸朝庭,展哀陵寢,並與賢彥申寫所懷。望體其心,勿為辭費。」府州佐史並稱臣,請題枿諸門,一依宮省,上不許。甲戌,發江陵。



 八月丙申,車駕至京城。丁酉,謁初寧陵,還於中堂即皇帝位。



 元嘉元年秋八月丁酉,大赦天下,改景平二年為元嘉元年。文武賜位二等,逋租宿債勿復收。庚子,以行撫軍將軍、荊州刺史謝晦為撫軍將軍、荊州刺史。癸卯,司空、錄尚書事、揚州刺史徐羨之進位司徒,衛將軍、江州刺史王弘進位司空,中書監、護軍將軍傅亮加左光祿大夫、開府儀同三司,撫軍將軍、荊州刺史謝晦進號衛將軍,鎮北將軍、南兗州刺史檀道濟進號征北將軍。甲辰,追尊所生胡婕妤為皇太后,謚曰章后。衛將軍、南徐州
 刺史彭城王義康進號驃騎將軍,冠軍將軍、南豫州刺史義恭進號撫軍將軍,封江夏王。立第六皇弟義宣為竟陵王,第七皇弟義季為衡陽王。戊申,以豫州刺史劉粹為雍州刺史,驍騎將軍管義之為豫州刺史,南蠻校尉到彥之為中領軍。己酉,減荊、湘二州今年稅布之半。九月丙子,立妃袁氏為皇后。



 二年春正月丙寅,司徒徐羨之、尚書令傅亮奉表歸政,上始親覽。車駕祠南郊,大赦天下。三月乙丑,左將軍、徐
 州刺史王仲德進號安北將軍。夏五月戊寅,特進謝澹卒。秋八月甲申,以關中流民出漢川,置京兆、扶風、馮翊等郡。乙酉,驃騎將軍、南徐州刺史彭城王義康為開府儀同三司,新除司空王弘為車騎大將軍、開府儀同三司,以右軍長史江恒為廣州刺史。冬十一月癸酉,以前將軍楊玄為征西將軍、北秦州刺史。



 三年春正月丙寅,司徒、錄尚書事、揚州刺史徐羨之,尚書令、護軍將軍、左光祿大夫傅亮,有罪伏誅。遣中領軍
 到彥之、征北將軍檀道濟討荊州刺史謝晦。上親率六師西征,大赦天下。丁卯,以車騎大將軍、江州刺史王弘為司徒、錄尚書事、揚州刺史,驃騎將軍、南徐州刺史彭城王義康改為荊州刺史,撫軍將軍、南豫州刺史江夏王義恭改為南徐州刺史。己巳,以前護軍將軍趙倫之為鎮軍將軍。閏月丙戌,皇子劭生。二月乙卯,繫囚見徒,一皆原赦。戊午,以金紫光祿大夫王敬弘為尚書左僕射,豫章太守鄭鮮之為尚書右僕射。建安太守潘盛有
 罪伏誅。庚申,特進範泰加光郤大夫。是日,車駕發京師。戊辰,到彥之、檀道濟大破謝晦於隱磯。丙子,車駕自蕪湖反旆。己卯,擒晦於延頭,送京師伏誅。三月辛巳,車駕還宮。夏五月乙未,以征北將軍、南兗州刺史檀道濟為征南大將軍、江州刺史,中領軍到彥之為南豫州刺史。戊戌,以後將軍長沙王義欣為南兗州刺史。乙巳,驃騎大將軍、涼州牧大沮渠蒙遜改為車騎大將軍。詔曰:「夫哲王宰世,廣達四聰,猶巡嶽省方,採風觀政。所以情偽
 必審,幽遐罔滯,王澤無擁,九皋有聞者也。朕以寡薄,猥纂洪緒。雖永念治道,志存昧旦,願言傅巖,發想宵寐,而丘園之秀,藏器未臻,物情民隱,尚隔視聽。乃眷區域,輟寐忘餐。今氛昆祛蕩,宇內寧晏,旌賢弘化,於是乎始。可遣大使巡行四方。其宰守稱職之良,閨蓽一介之善,詳悉列奏,勿或有遺。



 若刑獄不恤,政治乖謬,傷民害教者,具以事聞。其高年、鰥寡、幼孤、六疾不能自存者,可與郡縣優量賑給。博採輿誦,廣納嘉謀,務盡銜命之旨,俾若
 朕親覽焉。」



 丙午,車駕臨延賢堂聽訟。六月己未,以鎮軍將軍趙倫之為左光祿大夫、領軍將軍。



 丙寅,車駕又於延賢堂聽訟。丙子,又聽訟。以右衛王華為中護軍。冬十一月戊寅,以梁、南秦二州刺史吉翰為益州刺史,驃騎參軍劉道產為梁、南秦二州刺史。己亥,以南蠻校尉劉遵考為雍州刺史。十二月癸丑,以中書侍郎蕭思話為青州刺史。壬戌,前吳郡太守徐佩之謀反,及黨與皆伏誅。



 四年春正月乙亥朔,曲赦都邑百里內。辛巳,車駕親祠南郊。二月乙卯,行幸丹徒,謁京陵。三月丙子,詔曰:「丹徒桑梓綢繆,大業攸始,踐境永懷,觸感罔極。昔漢章南巡,加恩元氏,況情義二三,有兼曩日。思播遺澤,酬慰士民。其蠲此縣今年租布,五歲刑以下皆悉原遣;登城三戰及大將家,隨宜隱恤。」丁亥,車駕還宮。戊子,尚書右僕射鄭鮮之卒。壬寅,禁斷夏至日五絲命縷之屬,富陽令諸葛闡之之議也。夏四月庚戌,以廷尉王徽之為交州刺
 史。五月壬午,中護軍王華卒。



 京師疾疫。甲午,遣使存問,給醫藥;死者若無家屬,賜以棺器。六月癸卯朔,日有蝕之。庚申,以金紫光祿大夫殷穆為護軍將軍。



 五年春正月乙亥,詔曰:「朕恭承洪業,臨饗四海,風化未弘,治道多昧,求之人事,鑒寐惟憂。加頃陰違序,旱疫成患,仰惟災戒,責深在予。思所以側身剋念,議獄詳刑,上答天譴,下恤民瘼。群后百司,其各獻讜言,指陳得失,勿有所諱。」甲申,車駕臨玄武館閱武。戊子,京邑大火,遣
 使巡慰賑賜。夏四月己亥,以南蠻校尉蕭摹之為湘州刺史。戊午,以始興太守徐豁為廣州刺史。五月己卯,以湘州刺史張邵為雍州刺史。六月庚戌,司徒王弘降為衛將軍、開府儀同三司。京邑大水。乙卯,遣使檢行賑贍。以江夏內史程道惠為廣州刺史。秋八月壬戌,特進、左光祿大夫范泰卒。冬十月甲辰,車駕於延賢堂聽訟。閏月癸未,以右軍司馬劉德武為豫州刺史。辛卯,安陸公相周籍之為寧州刺史。十二月庚寅,左光祿大夫、領軍
 將軍趙倫之卒。是歲,天竺國遣使獻方物。



 六年春正月辛丑,車駕親祠南郊。癸丑,以驃騎將軍、荊州刺史彭城王義康為司徒、錄尚書事,領平北將軍、徐州刺史。三月丁巳,立皇子劭為皇太子。戊午,大赦天下,賜文武位一等。辛酉,以左衛將軍殷景仁為中領軍。夏四月癸亥,以尚書左僕射王敬弘為尚書令,丹陽尹臨川王義慶為尚書左僕射,吏部尚書江夷為尚書右僕射。五月壬辰朔,日有蝕之。癸巳,以新除尚書令王敬弘
 為特進、左光祿大夫。



 甲午,以撫軍司馬劉道濟為益州刺史。乙卯,於雍州置馮翊郡。七月己酉,以尚書左丞孔默之為廣州刺史。是月,百濟王遣使獻方物。九月戊午,於秦州置隴西、宋康二郡。冬十月壬申,中領軍殷景仁丁艱去職。十一月己丑朔,日有蝕之。十二月丁亥,河南國、河西王遣使獻方物。



 七年春正月癸巳,以吐谷渾慕容璝為征西將軍、沙州刺史。是月,倭國王遣使獻方物。三月戊子,遣右將軍到彥
 之北伐,水軍入河。甲午,以前征虜司馬尹沖為司州刺史。甲寅,以前中領軍殷景仁為領軍將軍。夏四月癸未,訶羅單國遣使獻方物。六月己卯,以冠軍將軍氐楊難當為秦州刺史。秋七月戊子,索虜確磝戍棄城走。



 丙申,以平北諮議參軍甄法護為梁、南秦二州刺史。戊戌,索虜滑臺戍棄城走。甲寅,林邑國、訶羅佗國、師子國遣使獻方物。冬十月甲寅,罷南豫州並豫州。以左將軍竟陵王義宣為徐州刺史。戊午,立錢署,鑄四銖錢。戊寅,金墉
 城為索虜所陷。



 十一月癸未,虎牢城復為索虜所陷。壬辰,遣征南大將軍檀道濟北討,右將軍到彥之自滑臺奔退。十二月辛酉,以南兗州刺史長沙王義欣為豫州刺史,司徒司馬吉翰為司州刺史。乙亥,京邑火,延燒太社北墻。兗州刺史竺靈秀有罪伏誅。



 八年春正月庚寅,於交州復立珠崖郡。癸巳,以左軍將軍申宣為兗州刺史。丁酉,征南大將軍檀道濟破索虜於東平壽張。二月乙卯,以平北司馬韋郎為青州刺史。



 戊
 午,以尚書右僕射江夷為湘州刺史。辛酉,滑臺為索虜所陷。癸酉,征南大將軍檀道濟引軍還。丁丑,青州刺史蕭思話棄城走。以太子右衛率劉遵考為南兗州刺史。



 三月甲申,車駕於延賢堂聽訟。戊申,詔曰:「自頃軍役殷興,國用增廣,資儲不給,百度尚繁。宜存簡約,以應事實。內外可通共詳思,務令節儉。」夏四月甲寅,以衡陽王師阮萬齡為湘州刺史。乙卯,以後軍參軍徐遵之為兗州刺史。六月乙丑,大赦天下。己卯,割江南及揚州晉陵郡
 屬南徐州,江北屬兗州。以徐州刺史竟陵王義宣為南兗州刺史,司徒司馬吉翰為徐州刺史。閏月庚子,詔曰:「自頃農桑惰業,遊食者眾,荒萊不闢,督課無聞。一時水旱,便有罄匱,茍不深存務本,豐給靡因。



 郡守賦政方畿,縣宰親民之主,宜思獎訓,導以良規。咸使肆力,地無遺利,耕蠶樹藝,各盡其力。若有力田殊眾,歲竟條名列上。」揚州旱。乙巳,遣侍御史省獄訟,申調役。丙午,以左軍諮議參軍劉道產為雍州刺史。秋八月甲辰,臨川王義慶解
 尚書僕射。丁未,割豫州秦郡屬南兗州。冬十二月,罷湘州,還并荊州。



 九年春三月庚戌,衛將軍王弘進位太保,加中書監。丁巳,征南大將軍、江州刺史檀道濟進位司空。夏四月乙亥,以護軍將軍殷穆為特進、右光祿大夫,建昌縣公到彥之為護軍將軍。五月壬申,中書監、錄尚書事、衛將軍、揚州刺史王弘薨。



 六月甲戌,以左軍諮議參軍申宣為青州刺史。分青州置冀州。戊寅,司徒、南徐州刺史彭城
 王義康改領揚州刺史。己卯,以司徒參軍崔諲為冀州刺史。壬午,以吐谷渾慕容延為平東將軍,吐谷渾拾虔為平北將軍,吐谷渾輝伐為鎮軍將軍。癸未,詔曰:「益、梁、交、廣,境域幽遐,治宜物情,或多偏擁。可更遣大使,巡求民瘼。」



 置積射、彊弩將軍官。乙未,以征西將軍、沙州刺史吐谷渾慕容璝為征西大將軍、西秦河二州刺史、隴西王。北秦州刺史氐楊難當加號征西將軍。壬寅,以撫軍將軍、荊州刺史江夏王義恭為征北將軍、開府儀同三司、
 南兗州刺史;前將軍臨川王義慶為平西將軍、荊州刺史;南兗州刺史竟陵王義宣為中書監、中軍將軍;征虜將軍衡陽王義季為南徐州刺史。秋七月戊辰,以尚書王仲德為鎮北將軍、徐州刺史。庚午,以領軍將軍殷景仁為尚書僕射,太子詹事劉湛為領軍將軍。壬申,河南國、河西王遣使獻方物。九月,妖賊趙廣寇益州,陷沒郡縣,州府討平之。冬十一月壬子,以少府甄法崇為益州刺史。癸丑,於廣州立宋康郡。十二月甲戌,以右軍參軍
 李秀之為交州刺史。庚寅,立第五皇子紹為廬陵王,江夏王義恭子郎為南豐縣王。



 十年春正月甲寅,竟陵王義宣改封南譙王,鎮北將軍、徐州刺史王仲德加領兗州刺史,淮南太守段宏為青州刺史。己未,大赦天下。孤老、六疾不能自存者,人賜穀五斛。後將軍、豫州刺史長沙王義欣進號鎮軍將軍。夏四月戊戌,青州刺史段宏加冀州刺史。封陽縣侯蕭思話為梁、南秦二州刺史。五月,林邑王遣使獻方物。



 六月
 乙亥,以前青州刺史韋郎為廣州刺史。闍婆州訶羅單國遣使獻方物。秋七月戊戌,曲赦益、梁、秦三州。於益州立宋寧、宋興二郡。八月丁丑,於青州立太原郡。



 辛巳,護軍將軍到彥之卒。冬十一月,氐楊難當寇漢川。丁未,梁州刺史甄法護棄城走,難當據有梁州。



 十一年春正月,亡命馬大玄群黨數百人寇秦、梁,州郡討平之。二月癸酉,以交址太守李耽之為交州刺史。夏四月,梁、秦二州刺史蕭思話破氐楊難當,梁州平。



 五月
 丁卯,曲赦梁、南秦二州劍閣北。戊寅,以大沮渠茂虔為征西大將軍、涼州刺史。是月,京邑大水。六月丁未,省魏郡。是歲,林邑國、扶南國、訶羅單國遣使獻方物。



 十二年春正月辛酉,大赦天下。辛未,車駕親祠南郊。癸酉,封黃龍國主馮弘為燕主。夏四月乙酉,尚書僕射殷景仁加中護軍。丙辰,詔曰:「周宗以寧,實由多士;漢室之隆,亦資得人。朕寐寤樂賢,為日已久,而俊哲難階,明揚莫效。用令遺才在野,管庫虛朝,永懷前載,慚德深矣。夫
 舉爾所知,宣尼之篤訓,貢士任官,先代之成準。便可宣敕內外,各有薦舉。當依方銓引,以觀厥用。」是夜,京郡地震。六月,丹陽、淮南、吳興、義興大水,京邑乘船。己酉,以徐豫南兗三州、會稽宣城二郡米數百萬斛賜五郡遭水民。是月,斷酒。師子國遣使獻方物。秋七月乙酉,闍婆娑達國、扶南國並遣使獻方物。八月壬申,於益州立南晉壽、南新巳、北巴西三郡。乙亥,原遭水郡諸逋負。九月,蜀郡賊張尋為寇。冬十一月,以右軍行參軍茍道覆為交州刺史。



 十三年春正月癸丑,上有疾,不朝會。三月己未,司空、江州刺史檀道濟有罪伏誅。庚申,大赦天下。以中軍將軍南譙王義宣為鎮南將軍、江州刺史。夏五月戊辰,鎮北將軍、徐兗二州刺史王仲德進號鎮北大將軍。庚辰,以征北司馬王方俳為兗州刺史。六月,高麗國、武都王遣使獻方物。秋七月己未,零陵王太妃薨。追崇為晉皇后,葬以晉禮。八月庚寅,尚書僕射、中護軍殷景仁改為護軍將軍。九月癸丑,立第二皇子濬為始興王,第三皇
 子諱為武陵王。



 十四年春正月辛卯,車駕親祠南郊,大赦天下。文武賜位一等;孤老、六疾不能自存者,人賜穀五斛。二月壬子,以步兵校尉劉道真為梁、南秦二州刺史。夏四月丁未,以輔國將軍周籍之為益州刺史。秋八月戊午,以尚書金部郎中徐森之為交州刺史。冬十二月辛酉,停賀雪。河南國、河西王、訶羅單國並遣使獻方物。



 十五年春二月丁未,以平東將軍吐谷渾慕容延為鎮
 西將軍、秦河二州刺史。夏四月甲辰,燕王年遣使獻方物。立皇太子妃殷氏,賜王公以下各有差。己巳,以倭國王珍為安東將軍。五月己丑,特進、右光祿大夫殷穆卒。辛卯,鎮北大將軍、徐州刺史王仲德卒。壬辰,以右衛將軍劉遵考為徐、兗二州刺史。秋七月辛未,地震。



 甲戌,以陳、南頓二郡太守徐循為寧州刺史。八月辛丑,以左衛將軍趙伯符為徐、兗二州刺史。甲寅,以始興內史陸徽為廣州刺史。丁巳,以兗州刺史王方俳為青、冀二州刺
 史。是歲,武都王、河南國、高麗國、倭國、扶南國、林邑國並遣使獻方物。



 十六年春正月戊寅,車駕於北郊閱武。庚寅,司徒、錄尚書事、揚州刺史彭城王義康進位大將軍,領司徒,餘如故。征北將軍、開府儀同三司、南兗州刺史江夏王義恭進位司空,刺史如故。特進、左光祿大夫王敬弘開府儀同三司。癸巳,復分荊州置湘州。二月己亥,以南徐州刺史衡陽王義季為安西將軍、荊州刺史。丁未,以始興王
 濬為湘州刺史。癸亥,割梁州之巴西梓潼南宕渠南漢中、南秦州之南安懷寧凡六郡,屬益州。分長沙江夏郡立巴陵郡,屬湘州。夏四月丁巳,以鎮南將軍、江州刺史南譙王義宣為征北將軍、南徐州刺史。平西將軍臨川王義慶為衛將軍、江州刺史。六月己酉,隴西吐谷渾慕容延改封河南王。癸丑,以吐谷渾拾寅為平西將軍,吐谷渾繁暱為撫軍將軍。秋八月庚子,立第四皇子鑠為南平王。閏月乙未,鎮軍將軍、豫州刺史長沙王義欣薨。
 戊戌,復分豫州之淮南為南豫州。癸卯,以左衛將軍劉遵考為豫州刺史。戊申,以湘州刺史始興王浚為南豫州之刺史,武陵王諱為湘州刺史。冬十二月乙亥,皇太子冠,大赦天下。是歲,武都王、河南王、林邑國、高麗國並遣使獻方物。



 十七年夏四月戊午朔,日有蝕之。五月癸巳,領軍將軍劉湛母憂去職。秋七月壬寅,以征虜諮議參軍杜驥為青州刺史。壬子,皇后袁氏崩。八月,徐、兗、青、冀四州大水。
 己未,遣使檢行賑恤。九月壬子,葬元皇后於長寧陵。冬十月戊午,前丹陽尹劉湛有罪,及同黨伏誅。大赦天下,文武賜爵一級。以大將軍、領司徒、錄尚書、揚州刺史彭城王義康為江州刺史,大將軍如故。以司空、南兗州刺史江夏王義恭為司徒、錄尚書事。戊寅,衛將軍臨川王義慶以本號為南兗州刺史,尚書僕射、護軍將軍殷景仁為揚州刺史,僕射如故。十一月丙戌,以尚書劉義融為領軍將軍,祕書監徐湛之為中護軍。丁亥,詔曰:「前所
 給揚、南徐二州百姓田糧種子,兗、兩豫、青、徐諸州比年所寬租穀應督入者,悉除半。今半有不收處,都原之。



 凡諸逋債,優量申減。又州郡估稅,所在市調,多有煩刻。山澤之利,猶或禁斷;役召之品,遂及稚弱。諸如此比,傷治害民。自今咸依法令,務盡優允。如有不便,即依事別言,不得茍趣一時,以乖隱恤之旨。主者明加宣下,稱朕意焉。」癸丑,尚書僕射、揚州刺史殷景仁卒。十二月癸亥,以光祿大夫王琳為尚書僕射。戊辰,以南豫州刺史始興
 王濬為揚州刺史,湘州刺史武陵王諱為南豫州刺史,南平王鑠為湘州刺史。是歲,武都王、河南王、百濟國遣使獻方物。



 十八年春二月乙卯,以豫章太守庾登之為江州刺史。夏五月壬申,衛將軍南兗州刺史臨川王義慶、征北將軍南徐州刺史南譙王義宣並開府儀同三司。癸巳,於交州置宋熙郡。是月,沔水泛溢。六月戊辰,遣使巡行賑贍。辛未,領軍將軍劉義融卒。秋七月戊戌,以徐、兗二州
 刺史趙伯符為領軍將軍。冬十月辛亥,以巴東、建平二郡太守臧質為徐、兗二州刺史。乙卯,省南徐州之南燕、濮陽、南廣平郡。十一月戊子,尚書僕射王琳卒。己亥,以丹陽尹孟顗為尚書僕射。氐楊難當又寇漢川。



 十二月癸亥,遣龍驤將軍裴方明與梁、秦二州刺史劉真道討之。是月,晉寧太守公爨松子反叛,寧州刺史徐循討平之。是歲,肅特國、高麗國、蘇靡黎國、林邑國並遣使獻方
 物。



 十九年正月乙巳,詔曰:「夫所因者本,聖哲之遠教;本立化成,教學之為貴。



 故詔以三德,崇以四術,用能納諸義方,致之軌度。盛王祖世,咸必由之。永初受命,憲章弘遠,將陶鈞庶品,混一殊風。有詔典司,大啟庠序,而頻溝屯夷,未及修建。永瞻前猷,思敷鴻烈,今方隅乂寧,戎夏慕響,廣訓胄子,實維時務。便可式遵成規,闡揚景業。」夏四月甲戌,以久疾愈,始奉初祠,大赦天下。五月庚寅,梁秦二州刺史劉真道、龍驤將軍裴方明破氐楊難當,仇池
 平。閏月,京邑雨水;丁巳,遣使巡行賑恤。六月壬午,以大沮渠無諱為征西大將軍,涼州刺史。秋七月,以梁、秦二州刺史劉真道為雍州刺史,龍驤將軍裴方明為梁、南秦二州刺史。甲戌晦,日有蝕之。冬十月甲申,芮芮國遣使獻方物。己亥,以晉寧太守周萬歲為寧州刺史。十二月丙申,詔曰:「胄子始集,學業方興。自微言泯絕,逝將千祀,感事思人,意有慨然。奉聖之胤,可速議繼襲。於先廟地,特為營造,依舊給祠置令,四時饗祀。闕里往經寇亂,
 黌校殘毀,并下魯郡脩復學舍,採召生徒。昔之賢哲及一介之善,猶或衛其丘壟,禁其芻牧,況尼父德表生民,功被百代,而墳塋荒蕪,荊棘弗翦。可蠲墓側數戶,以掌洒掃。」魯郡上民孔景等五戶居近孔子墓側,蠲其課役,供給洒掃,并種松柏六百株。是歲,婆皇國遣使獻方物。



 二十年春正月,於臺城東西開萬春、千秋二門。二月甲戌,江州刺史庾登之為中護軍。庚申,以廬陵王紹為江州刺史。仇池為索虜所沒。甲申,車駕於白下閱武。



 三月
 辛亥,安西將軍、荊州刺史衡陽王義季進號征西大將軍。以巴西、梓潼二郡太守申坦為梁、南秦二州刺史。夏四月甲午,立第六皇子誕為廣陵王。五月癸丑,中護軍庾登之卒。秋七月癸丑,以楊文德為征西將軍、北秦州刺史,封武都王。辛酉,以南蠻校尉蕭思話為雍州刺史。甲子,前雍州刺史劉真道、梁南秦二州刺史裴方明有罪,下獄死。八月癸未,以廷尉陶愍祖為廣州刺史。冬十二月庚午,以始興內史檀和之為交州刺史。壬午,詔曰:「
 國以民為本,民以食為天。故一夫輟稼,饑者必及,倉廩既實,禮節以興。自頃在所貧罄,家無宿積。賦役暫偏,則人懷愁墊;歲或不稔,而病乏比室。誠由政德弗孚,以臻斯弊;抑亦耕桑未廣,地利多遺。宰守微化道之方,氓庶忘勤分之義。永言弘濟,明發載懷,雖制令亟下,終莫征勸,而坐望滋殖,庸可致乎!有司其班宣舊條,務盡敦課。遊食之徒,咸令附業,考覈勤惰,行其誅賞,觀察能殿,嚴加黜陟。古者躬耕帝籍,敬供粢盛,仰瞻前王,思遵令典。
 便可量處千畝,考卜元辰。朕當親率百辟,致禮郊甸,庶幾誠素,將被斯民。」是歲,河西國、高麗國、百濟國、倭國並遣使獻方物。是歲,諸州郡水旱傷稼,民大饑。遣使開倉賑恤,給賜糧種。



 二十一年春正月己亥,南徐、南豫州、揚州之浙江西,並禁酒。大赦天下,諸逋債在十九年以前,一切原除。去歲失收者,疇量申減。尤弊之處,遣使就郡縣隨宜賑恤。凡欲附農,而種糧匱乏者,並加給貸,營千畝諸統司役人,
 賜布各有差。



 戊午,衛將軍臨川王義慶薨。辛酉,以太子詹事劉義宗為南兗州刺史。二月庚午,以領軍將軍趙伯符為豫州刺史。己丑,司徒、錄尚書事江夏王義恭進位太尉,領司徒。庚寅,以右衛將軍沈演之為中領軍。辛卯,立第七皇子宏為建平王。甲午,以廣陵王誕為南兗州刺史。夏四月,晉陵延陵民徐耕以米千斛助恤饑民。五月壬戌,以尚書何尚之為中護軍,諮議參軍劉道錫為廣州刺史。六月,連雨水。丁亥,詔曰:「霖雨彌日,水潦為
 患,百姓積儉,易致乏匱。二縣官長及營署部司,各隨統檢實,給其柴米,必使周悉。」秋七月丁酉,揚州刺史始興王濬加中軍將軍,南豫州刺史武陵王贊加撫軍將軍。乙巳,詔曰:「比年穀稼傷損,淫亢成災,亦由播殖之宜,尚有未盡,南徐、兗、豫及揚州浙江西屬郡,自今悉督種麥,以助闕乏。速運彭城下邳郡見種,委刺史貸給。徐、豫土多稻田,而民間專務陸作,可符二鎮,履行舊陂,相率脩立,并課墾闢,使及來年。凡諸州郡,皆令盡勤地利,勸導
 播殖,蠶桑麻棨,各盡其方,不得但奉行公文而已。」八月戊辰,征西大將軍、荊州刺史衡陽王義季為征北大將軍、開府儀同三司、南兗州刺史;征北將軍、徐州刺史南譙王義宣為車騎將軍、荊州刺史。南兗州刺史廣陵王誕為南徐州刺史。九月甲辰,以大沮渠安周為征西將軍、涼州刺史,封河西王。冬十月己卯,以左軍將軍徐瓊為兗州刺史,大將軍參軍申恬為冀州刺史。



 二十二年春正月辛卯朔,改用御史中丞何承天元嘉
 新歷。壬辰,撫軍將軍、南豫州刺史武陵王諱改為雍州刺史,湘州刺史南平王鑠為南豫州刺史。二月辛巳,以侍中王僧朗為湘州刺史。甲戌,立第八皇子褘為東海王,第九皇子昶為義陽王。夏六月辛亥,以南豫州刺史南平王鑠為豫州刺史。秋七月己未,以尚書僕射孟顗為尚書左僕射,中護軍何尚之為尚書右僕射。雍州刺史武陵王諱討緣沔蠻,移一萬四千餘口於京師。乙酉,征北大將軍、南兗州刺史衡陽王義季改為徐州刺史。
 九月己未,開酒禁。冬十月,起湖熟廢田千頃。十二月乙未,太子詹事范曄謀反,及黨與皆伏誅。丁酉,免大將軍彭城王義康為庶人。庚戌,以前豫州刺史趙伯符為護軍將軍。



 二十三年春正月丁巳,以長沙內史陸徽為益州刺史。庚申,尚書左僕射孟顗去職。遷漢州流民於沔次。二月癸卯,以左衛將軍劉義賓為南兗州刺史。三月,索虜寇兗、豫,青、冀,刺史申恬破之。夏四月丁未,大赦天下。六月
 癸未朔,日有蝕之。交州刺史檀和之伐林邑國,剋之。秋七月辛未,以散騎常侍杜坦為青州刺史。



 八月癸卯,揭陽赭賊攻建安郡,燔燒城府。九月己卯,車駕幸國子學,策試諸生,答問凡五十九人。冬十月戊子,詔曰:「癢序興立累載,胄子肄業有成。近親策試,睹濟濟之美,緬想洙、泗,永懷在昔。諸生答問,多可採覽。教授之官,並宜沾賚。」



 賜帛各有差。十二月丁酉,以龍驤司馬蕭景憲為交州刺史。是歲,大有年。築北堤,立玄武湖,築景陽山於華林
 園。



 二十四年春正月甲戌,大赦天下,文武賜位一等。繫囚降宥,諸逋負寬減各有差。孤老、六疾不能自存,人賜穀五斛。蠲建康、秣陵二縣今年田租之半。三月壬申,護軍將軍趙伯符遷職。夏五月甲戌,青州刺史杜坦加冀州刺史。六月,京邑疫癘。丙戌,使郡縣及營署部司,普加履行,給以醫藥。是月,以貨貴,制大錢一當兩。秋七月乙卯,以林邑所獲金銀寶物,班賚各有差。八月乙未,征北大
 將軍、徐州刺史衡陽王義季薨。癸卯,以南兗州刺史劉義賓為徐州刺史。九月己未,以中領軍沈演之為領軍將軍。辛未,以太子詹事徐湛之為南兗州刺史。冬十月壬午,豫章胡誕世反,殺太守桓隆之。前交州刺史檀和之南還至豫章,因討平之。壬辰,以建平王宏為中護軍。十一月甲寅,立第十皇子渾為汝陰王。



 二十五年春正月戊辰,詔曰:「比者冰雪經旬,薪粒貴踴,貧弊之室,多有窘罄。可檢行京邑二縣及營署,賜以柴
 米。」二月庚寅,詔曰:「安不忘虞,經世之所同;治兵教戰,有國之恒典。故服訓明恥,然後少長知禁,頃戎政雖脩,而號令未審。今宣武場始成,便可剋日大習眾軍。當因校獵,肄武講事。」閏月己酉,大搜于宣武場。三月庚辰,車駕校獵。夏四月乙巳,新作閶闔、廣莫二門,改先廣莫門曰承明,開陽曰津陽。乙卯,以撫軍將軍、雍州刺史武陵王諱為安北將軍、徐州刺史。癸亥,以右衛將軍蕭思話為雍州刺史。五月己卯,罷大錢當兩。六月庚戌,零陵王司
 馬元瑜薨。庚申,安北將軍、徐州刺史武陵王諱加兗州刺史。丙寅,車騎將軍、荊州刺史南譙王義宣進位司空。秋七月壬午,左光祿大夫王敬弘薨。八月己酉,以撫軍參軍劉秀之為梁、南秦二州刺史。甲子,立第十一皇子諱為淮南王。九月辛未,以尚書右僕射何尚之為尚書左僕射,領軍將軍沈演之遷職,吳興太守劉遵考為領軍將軍。



 二十六年春正月辛巳,車駕親祠南郊。二月己亥,車駕
 陸道幸丹徒,謁京陵。



 三月丁巳,詔曰:「朕違北京,二十餘載,雖云密邇,瞻塗莫從。今因四表無塵,時和歲稔,復獲拜奉舊塋,展罔極之思,饗宴故老,申追遠之懷。固以義兼於桑梓,情加於過沛;永言慷慨,感慰實深。宜聿宣仁惠,覃被率土。其大赦天下,復丹徒縣僑舊今歲租布之半。行所經縣,蠲田租之半。二千石官長並勤勞王務,宜有沾錫。



 登城三戰及大將戰亡墜沒之家,老病單弱者,普加贍恤。遣使巡行百姓,問所疾苦。



 孤老、鰥寡、六疾不
 能自存者,人賜穀五斛。」遣使祭晉故司空忠肅公何無忌之墓。



 乙丑,申南北沛下邳三郡復。又詔曰:「京口肇祥自古,著符近代,衿帶江山,表裏華甸,經塗四達,利盡淮、海,城邑高明,土風淳壹,苞總形勝,實唯名都。故能光宅靈心,克昌帝業。頃年岳牧遷回,軍民徙散,廛里廬宇,不逮往日。皇基舊鄉,地兼蕃重,宜令殷阜,式崇形望。可募諸州樂移者數千家,給以田宅,并蠲復。」



 五月丙寅,詔曰:「吾生於此城。及盧循肆亂,害流茲境。先帝以桑梓根本,
 實同休戚,復以蒙稚,猥同艱難,情義繾綣,夷險兼備,舊物遺蹤,猶存心目。歲月不居,逝踰三紀,時人故老,與運零落。眷惟既往,倍深感歎。可搜訪于時士庶文武今尚存者,具以名聞。人身已亡而子孫見在,優量賜賚之。」車駕水路發丹徒,壬午,至京師。丙戌,婆皇國,壬辰,婆達國,並遣使獻方物。秋七月辛未,以江州刺史廬陵王紹為南徐州刺史,廣陵王誕為雍州刺史。八月己酉,以中護軍建平王宏為江州刺史。癸丑,以南豐王朗為湘州刺
 史。冬十月,廣陵王誕改封隨郡王。甲辰,以中軍將軍、揚州刺史始興王濬為征北將軍、開府儀同三司、南徐兗二州刺史;南徐州刺史廬陵王紹為揚州刺史。



 二十七年春正月辛未,制交、寧二州假板郡縣,俸祿聽依臺除。辛卯,百濟國遣使獻方物。二月辛丑,右將軍、豫州刺史南平王鑠進號平西將軍。辛巳,索虜寇汝南諸郡,陳南頓二郡太守鄭琨、汝陽潁川二郡太守郭道隱委守走。索虜攻懸瓠城,行汝南郡事陳憲拒之。以軍興減
 百官俸三分之一。三月乙丑,淮南太守諸葛闡求減俸祿同內百官,於是州及郡縣丞尉並悉同減。戊寅,罷國子學。乙酉,以新除吏部尚書蕭思話為護軍將軍。夏四月壬子,安北將軍、徐兗二州刺史武陵王贊降號鎮軍將軍。六月丁酉,侍中蕭斌為青、冀二州刺史。秋七月庚午,遣寧朔將軍王玄謨北伐。太尉江夏王義恭出次彭城,總統諸軍。乙亥,索虜確磝戍委城走。冬閏月癸亥,玄謨攻滑臺,不克,為虜所敗,退還確磝。辛未,雍州刺史隨
 王誕遣軍攻弘農城,克之。丙戌,又克關城。十一月戊子,索虜陷鄒山,魯、陽平二郡太守崔邪利沒。



 甲午,隨王誕所遣軍又攻陜城,克之。癸卯,左軍將軍劉康祖於壽陽尉武戍與虜戰敗見殺。丁未,大赦天下。十二月戊午,內外纂嚴。乙丑,冗從僕射胡崇之、太子積弩將軍臧澄之、建威將軍毛熙祚於盱眙與虜戰敗,並見殺。庚午,虜偽主率大眾至瓜步。壬午,內外戒嚴。



 二十八年春正月丙戌朔,以寇逼不朝會。丁亥,索虜自
 瓜步退走。丁酉,攻圍盱眙城。是月,寧朔將軍王玄謨自確磝退還歷下。二月丙辰,索虜自盱眙奔走。癸酉,詔曰:「玁狁孔熾,難及數州,眷言念之,寤寐興悼。兇羯痍挫,迸跡遠奔,凋傷之民,宜時振理。凡遭寇賊郡縣,令還復居業,封屍掩骼,賑贍饑流。東作方始,務盡勸課。貸給之宜,事從優厚。其流寓江、淮者,並聽即屬,并蠲復稅調。」



 甲戌,太尉、領司徒江夏王義恭降為驃騎將軍、開府儀同三司。辛巳,鎮軍將軍、徐兗二州刺史武陵王諱降號北中
 郎將。壬午,車駕幸瓜步,是日解嚴。三月乙酉,車駕還宮。壬辰,征北將軍始興王濬解南兗州。庚子,以輔國將軍臧質為雍州刺史。



 戊申,徐州刺史武陵王諱為南兗州刺史。甲寅,護軍將軍蕭思話為撫軍將軍、徐兗二州刺史。夏四月癸酉,婆達國遣使獻方物。索虜偽寧南將軍魯爽、中書郎魯秀歸順。戊寅,以爽為司州刺史。五月乙酉,亡命司馬順則自號齊王,據梁鄒城。丁巳,婆皇國,戊戌,河南王,並遣使獻方物。己巳,驃騎將軍江夏王義
 恭領南兗州刺史。



 戊申,以尚書左僕射何尚之為尚書令,太子詹事徐湛之為尚書僕射、護軍將軍。壬子,以後將軍隨王誕為安南將軍、廣州刺史。六月壬戌,以北中郎將武陵王諱為江州刺史,以振武將軍、秦郡太守劉興祖為青、冀二州刺史。秋七月甲辰,安東將軍倭王倭濟進號安東大將軍。八月癸亥,梁鄒平,斬司馬順則。冬十月癸亥,高麗國遣使獻方物。十一月壬寅,曲赦二兗、徐、豫、青、冀六州。是冬,徙彭城流民於瓜步,淮西流民於姑
 孰,合萬許家。



 二十九年春正月甲午,詔曰:「經寇六州,居業未能,仍值菑澇,饑困薦臻。



 可速符諸鎮,優量救恤。今農事行興,務盡地利。若須田種,隨宜給之。」二月庚申,虜帥拓跋燾死。庚午,立第十二皇子休仁為建安王。夏四月戊午,訶羅單國遣使獻方物。以驃騎參軍張永為冀州刺史。五月甲午,罷湘州並荊州。以始興、臨賀、始安三郡屬廣州。丙申,詔曰:「惡稔身滅,戎醜常數,虐虜窮凶,著於自昔。未勞
 資斧,已伏天誅,子孫相殘,親黨離貳,關、洛偽帥,並懷內款,河朔遺民,注誠請效。拯溺蕩穢,今其會也。可符驃騎、司空二府,各部分所統,東西應接。歸義建績者,隨勞酬獎。」是月,京邑雨水。六月己酉,遣部司巡行,賜樵米,給船。



 撫軍將軍蕭思話率眾北伐。以征北從事中郎劉瑀為益州刺史。秋七月壬辰,汝陰王渾改封武昌王,淮陽王諱改封湘東王。丁酉,省大司農、太子僕、廷尉監官。八月丁卯,蕭思話攻確磝,不拔,退還。九月丁亥,以平西將軍
 吐谷渾拾寅為安西將軍、秦河二州刺史。己丑,撫軍將軍、徐兗二州刺史蕭思話加冀州刺史,兗州如故。冬十月癸亥,司州刺史魯爽攻虎牢不拔,退還。十一月壬寅,揚州刺史廬陵王紹薨。



 十二月辛未,以驃騎將軍、南兗州刺史江夏王義恭為大將軍、南徐州刺史,錄尚書事如故。



 三十年春正月戊寅,以司空、荊州刺史南譙王義宣為司徒、中軍將軍、揚州刺史。以南兗州並南徐州。庚辰,以
 領軍將軍劉遵考為平西將軍、豫州刺史。壬午,以征北將軍、南徐州刺史始興王濬為衛將軍、荊州刺史。戊子,江州刺史武陵王諱統眾軍伐西陽蠻。癸巳,以豫州刺史南平王鑠為撫軍將軍、領軍將軍。青、徐州饑。



 二月壬子,遣運部賑恤。甲子,上崩于含章殿,時年四十七。謚曰景皇帝,廟曰中宗。三月癸巳,葬長寧陵。世祖踐阼,追改謚及廟號。



 史臣曰:太祖幼年特秀,顧無保傅之嚴,而天授和敏之
 姿,自稟君人之德。及正位南面,歷年長久,綱維備舉,條禁明密,罰有恆科,爵無濫品。故能內清外晏,四海謐如也。昔漢氏東京常稱建武、永平故事,自茲厥後,亦每以元嘉為言,斯固盛矣!授將遣帥,乖分閫之命,才謝光武,而遙制兵略,至於攻日戰時,莫不仰聽成旨。雖覆師喪旅,將非韓、白,而延寇蹙境,抑此之由。及至言漏衾衽,難結商豎,雖禍生非慮,蓋亦有以而然也。嗚呼哀哉!



\end{pinyinscope}