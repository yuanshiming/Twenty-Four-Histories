\article{卷四十八列傳第八 朱齡石 弟超石 毛脩之 傅弘之}

\begin{pinyinscope}

 朱齡石,字伯兒,沛郡沛人也。家世將帥。祖騰,建威將軍、吳國內史。伯父憲及斌,並為西中郎袁真將佐,憲為梁國內史,斌為汝南內史。大司馬桓溫伐真於壽陽,真以
 憲兄弟與溫潛通,並殺之。齡石父綽逃走歸溫,攻戰常居先,不避矢石。



 壽陽平,真已死,綽輒發棺戮尸;溫怒,將斬之,溫弟沖苦請得免。綽為人忠烈,受沖更生之恩,事沖如父。參沖車騎軍事、西陽廣平太守。及沖薨,綽歐血死。沖諸子遇齡石如兄弟。



 齡石少好武事,頗輕佻,不治崖檢。舅淮南蔣氏,人才儜劣,齡石使舅臥於聽事一頭,剪紙方一寸,帖著舅枕,自以刀子懸擲之,相去八九尺,百擲百中。舅雖危懼戰慄,為畏齡石,終不敢動。舅頭有
 大瘤,齡石伺舅眠,密往割之,舅即死。



 初為殿中將軍,常追隨桓脩兄弟,為脩撫軍參軍。在京口,高祖克京城,以為建武參軍。從至江乘,將戰,齡石言於高祖曰:「世受桓氏厚恩,不容以兵刃相向,乞在軍後。」高祖義而許之。事定,以為鎮軍參軍,遷武康令,加寧遠將軍。



 喪亂之後,武康人姚係祖招聚亡命,專為劫盜,所居險阻,郡縣畏憚不能討。



 齡石至縣,偽與係祖親厚,召為參軍。係祖恃其兄弟徒黨彊盛,謂齡石必不敢圖己,乃出應召。齡石潛結腹
 心,知其居北塗徑,乃要係祖宴會,叱左右斬之。乃率吏人馳至其家,掩其不備,莫有得舉手者,悉斬係祖兄弟,殺數十人,自是一郡得清。



 高祖又召為參軍,補徐州主簿,遷尚書都官郎,尋復為參軍。從征鮮卑,坐事免官。廣固平,復為參軍。盧循至石頭,領中軍。循選敢死之士數千人上南岸,高祖遣齡石領鮮卑步槊,過淮擊之。率厲將士,皆殊死戰,殺數百人,賊乃退。齡石既有武幹,又練吏職,高祖甚親委之。盧循平,以為寧遠將軍、寧蠻護軍、
 西陽太守。義熙八年,高祖西伐劉毅,齡石從至江陵。九年,遣諸軍伐蜀,令齡石為元帥,以為建威將軍、益州刺史,率寧朔將軍臧熹、河間太守蒯恩、下邳太守劉鍾、龍驤將軍朱林等,凡二萬人,發自江陵。尋加節益州諸軍事。初,高祖與齡石密謀進取,曰:「劉敬宣往年出黃虎,無功而退。賊謂我今應從外水往,而料我當出其不意,猶從內水來也。如此,必以重兵守涪城,以備內道。若向黃虎,正陊其計。今以大眾自外水取成都,疑兵出內水,此
 制敵之奇也。」而慮此聲先馳,賊審虛實,別有函書,全封付齡石,署函邊曰:「至白帝乃開。」諸軍雖進,未知處分所由。至白帝,發書,曰:「眾軍悉從外水取成都,臧熹、朱林於中水取廣漢,使羸弱乘高艦十餘,由內水向黃虎。」眾軍乃倍道兼行,譙縱果備內水,使其大將譙道福以重兵戍涪城,遣其前將軍秦州刺史侯輝、尚書僕射蜀郡太守譙詵等率眾萬餘屯彭模,夾水為城。



 十年六月,齡石至彭模,諸將以賊水北城險阻眾多,咸欲先攻其南,齡石
 曰:「不然。雖寇在北,今屠南城,不足以破北;若盡銳以拔北壘,南城不麾而自散也。」



 七月,齡石率劉鐘、蒯恩等攻城,詰朝戰,至日昃,焚其樓櫓,四面並登,斬侯輝、譙詵,仍回軍以麾,南城即時散潰。凡斬大將十五級,諸營守以次土崩,眾軍乃舍船步進。



 龍驤將軍臧熹至廣漢,病卒。朱林至廣漢,復破譙道福,別軍乘船陷牛脾城,斬其大將譙撫。譙縱聞諸處盡敗,奔於涪城,巴西人王志斬送。偽尚書令馬耽封府庫以待王師。道福聞彭模不守,率
 精銳五千兼行來赴,聞縱已走,道福眾亦散,乃逃於獠中。巴西民杜瑤縛送之,斬於軍門。桓謙弟恬隨謙入蜀,為寧蜀太守,至是亦斬焉。



 高祖之伐蜀也,將謀元帥而難其人,乃舉齡石。眾咸謂自古平蜀,皆雄傑重將,齡石資名尚輕,慮不克辦,諫者甚眾,高祖不從。乃分大軍之半,猛將勁卒,悉以配之。臧熹,敬皇后弟,咸服高祖之知人,又美齡石之善於其事。



 齡石遣司馬沈叔任戍涪,蜀人侯產德作亂,攻涪城,叔任擊破之,斬產德。初,齡石平
 蜀,所戮止縱一祖之後,產德事起,多所連結,乃窮加誅剪,死者甚眾。進號輔國將軍,尋進監益州之巴西、梓潼、宕渠、南漢中、秦州之安固、懷寧六郡諸軍事,以平蜀功,封豐城縣侯,食邑千戶。



 十一年,徵為太尉咨議參軍,加冠軍將軍。十二年北伐,遷左將軍,本號如故,配以兵力,守衛殿省,劉穆之甚加信仗,內外諸事,皆與謀焉。高祖還彭城,以齡石為相國右司馬。十四年,安西將軍桂陽公義真被徵,以齡石持節督關中諸軍事、右將軍、雍州刺
 史。敕齡石,若關右必不可守,可與義真俱歸。齡石亦舉城奔走。



 龍驤將軍王敬先戍曹公壘,齡石自潼關率餘眾就敬先,虜斷其水道,眾渴不能戰,城陷。虜執齡石及敬先還長安,見殺,時年四十。



 子景符嗣。景符卒,子祖宣嗣,坐輒之封,八年不反,及不分姑國秩,奪爵。



 更以祖宣弟隆紹封。齊受禪,國除。



 齡石弟超石,亦果銳善騎乘,雖出自將家,兄弟並閑尺牘。桓謙為衛將軍,以補行參軍。又參何無忌輔國右軍軍事。徐道覆破無忌,得超石,以
 為參軍。至石頭,超石說其同舟人乘單舸走歸高祖,高祖甚喜之,以為徐州主簿。超石收迎桓謙身首,躬營殯葬。遷車騎參軍事,尚書都官郎;尋復補中兵參軍、寧朔將軍、沛郡太守。



 西伐劉毅,使超石率步騎出江陵,未至而毅平。及討司馬休之,遣冠軍將軍檀道濟及超石步軍出大薄,魯宗之聞超石且至,自率軍逆之,未戰而江陵平。從至襄陽,領新野太守,追宗之至南陽而還。



 義熙十二年北伐,超石為前鋒入河,索虜托跋嗣,姚興之婿也,
 遣弟黃門郎鵝青、冀州刺史安平公乙旃眷、襄州刺史托跋道生、青州刺史阿薄干,步騎十萬,屯河北,常有數千騎,緣河隨大軍進止。時軍人緣河南岸,牽百丈,河流迅急,有漂渡北岸者,輒為虜所殺略。遣軍裁過岸,虜便退走,軍還,即復東來。高祖乃遣白直隊主丁旿,率七百人,及車百乘,於河北岸上,去水百餘步,為卻月陣,兩頭抱河,車置七仗士,事畢,使豎一白毦。虜見數百人步牽車上,不解其意,並未動。



 高祖先命超石馳往赴之,並齎
 大弩百張,一車益二十人,設彭排於轅上。虜見營陣既立,乃進圍營。超石先以軟弓小箭射虜,虜以眾少兵弱,四面俱至。嗣又遣南平公托跋嵩三萬騎至,遂肉薄攻營。於是百弩俱發,又選善射者叢箭射之,虜眾既多,弩不能制。超石初行,別齎大錘并千餘張槊,乃斷槊長三四尺,以錘錘之,一槊輒洞貫三四虜,虜眾不能當,一時奔潰。臨陣斬阿薄干首,虜退還半城。超石率胡蕃、劉榮祖等追之,復為虜所圍,奮擊盡日,殺虜千計,虜乃退走。高
 祖又遣振武將軍徐猗之五千人向越騎城,虜圍猗之,以長戟結陣。超石赴之,未至,悉奔走。大軍進克蒲阪,以超石為河東太守,戍守之。賊以超石眾少,復還攻城,超石戰敗退走,數日乃及大軍。



 高祖自長安東還,超石常令人水道至彭城,除中書侍郎,封興平縣五等侯。關中擾亂,高祖遣超石慰勞河、洛。始至蒲阪,值齡石自長安東走至曹公壘,超石濟河就之,與齡石俱沒,為佛佛所殺,時年三十七。



 毛脩之,字敬文,滎陽陽武人也。祖虎生,伯父璩,並益州刺史。父瑾,梁、秦二州刺史。



 脩之有大志,頗讀史籍,荊州刺史殷仲堪以為寧遠參軍。桓玄克荊州,仍為玄佐,歷後軍、太尉、相國參軍。解音律,能騎射,玄甚遇之。及篡位,以為屯騎校尉。隨玄西奔,玄敗於崢嶸洲,復還江陵,人情離散,議欲西奔漢川。脩之誘令入蜀,馮遷斬玄於枚回洲,脩之力也。



 晉安帝反正於江陵,除驍騎將軍。下至京師,高祖以為鎮軍咨議參軍,加寧朔將軍。旬月,遷右
 衛將軍。既有斬玄之謀,又伯、父並在蜀土,高祖欲引為外助,故頻加榮爵。及父瑾為譙縱所殺,高祖表為龍驤將軍,配給兵力,遣令奔赴。又遣益州刺史司馬榮期及文處茂、時延祖等西討。脩之至宕渠,榮期為參軍楊承祖所殺,承祖自稱鎮軍將軍、巴州刺史。脩之退還白帝,承祖自下攻之,不拔。脩之使參軍嚴綱等收兵眾,漢嘉太守馮遷率兵來會,討承祖斬之。時文處茂猶在邑郡,脩之遣振武將軍張季仁五百兵係處茂等。荊州刺史道
 規又遣奮武將軍原導之領千人,受脩之節度。脩之遣原導之與季仁俱進。



 時益州刺史鮑陋不肯進討,脩之下都上表曰:「臣聞在生所以重生,實有生理可保。臣之情地,生途已竭,所以未淪於泉壤,借命於朝露者,以日月貞照,有兼映之輝,庶憑天威,誅夷仇逆。自提戈西赴,備嘗時難,遂使齊斧停柯,狡豎假息。



 誠由經路有暨,亦緣制不自己。撫影窮號,泣望西路。益州刺史陋始以四月二十九日達巴東,頓白帝,以俟廟略。可乘之機宜踐,
 投袂之會屢愆。臣雖效死寇庭,而理絕救援,是以束骸載馳,訴冤象魏。昔宋害申丹,楚莊有遺履之憤,況忘家殉國,鮮有臣門,節冠風霜,人所矜悼。伍員不虧君義,而申包不忘國艱,俟會佇鋒,因時乃發。今臣庸踰在昔,未蒙宵邁之旗,是以仰辰極以希照,眷西土以灑淚也。公私懷恥,仰望洪恩,豈宜遂享名器,比肩人伍。求情既所不容,即實又非所繼,但以方仗威靈,要須綜攝,乞解金紫寵私之榮,賜以鷹揚折衝之號。臣之於國,理無虛請。
 自臣涉道,情慮荒越,疹毒交纏,常慮性命隕越,要當躬先士卒,身馳賊庭,手斬凶醜,以攄莫大之釁。然後就死之日,即化如歸,闔門靈爽,豈不謝先帝於玄宮。」高祖哀其情事,乃命冠軍將軍劉敬宣率文處茂、時延祖諸軍伐蜀。軍次黃虎,無功而退。譙縱由此送脩之父、伯及中表喪,口累並得俱還。



 盧循逼京邑,脩之服未除,起為輔國將軍,尋加宣城內史,戍姑孰。為循黨阮賜所攻,擊破之。循走,劉毅還姑孰,脩之領毅後軍司馬,坐長置吏僮,
 免將軍、內史官。毅西鎮江陵,以為衛軍司馬、輔國將軍、南郡太守。脩之雖為毅將佐,而深自結高祖。高祖討毅,先遣王鎮惡襲江陵,脩之與咨議參軍任集之等並力戰,高祖宥之。



 時遣朱齡石伐蜀,脩之固求行,高祖慮脩之至蜀,必多所誅殘,士人既與毛氏有嫌,亦當以死自固,故不許。還都,除黃門侍郎,復為右衛將軍。



 脩之不信鬼神,所至必焚除房廟。時蔣山廟中有佳牛好馬,脩之並奪取之。高祖討司馬休之,以為咨議參軍、冠軍將軍、
 領南郡相。



 高祖將伐羌,先遣脩之復芍陂,起田數千頃。及至彭城,又使營立府舍,轉相國右司馬,將軍如故。時洛陽已平,即本號為河南、河內二郡太守,行西州事,戍洛陽,脩治城壘。高祖既至,案行善之,賜衣服玩好,當時計直二千萬。先是,劉敬宣女嫁,高祖賜錢三百萬,雜彩千匹,時人並以為厚賜。王鎮惡死,脩之代為安西司馬,將軍如故。值桂陽公義真已發長安,為佛佛虜所邀,軍敗。脩之與義真相失,走將免矣。始登一阪,阪甚高峻,右
 衛軍人叛走,已上阪,嘗為脩之所罰者,以戟擲之,傷額,因墜阪,遂為佛佛所擒。佛佛死,其子赫連昌為索虜托跋燾所獲,脩之并沒。



 初,脩之在洛,敬事嵩高山寇道士,道士為燾所信敬,營護之,故得不死,遷於平城。脩之嘗為羊羹,以薦虜尚書,尚書以為絕味,獻之於燾;燾大喜,以脩之為太官令。稍被親寵,遂為尚書、光祿大夫、南郡公,太官令、尚書如故。其後朱脩之沒虜,亦為燾所寵。脩之相得甚歡。脩之問南國當權者為誰,朱脩之答云:「殷
 景仁。」脩之笑曰:「吾昔在南,殷尚幼少,我得歸罪之日,便應巾韝到門邪!」



 經年不忍問家消息,久之乃訊訪,脩之具答,并云:「賢子元矯,甚能自處,為時人所稱。」脩之悲不得言,直視良久,乃長歎曰:「嗚呼!」自此一不復及。初,荒人去來,言脩之勸誘燾侵邊,并教燾以中國禮制,太祖甚疑責之。脩之後得還,具相申理,上意乃釋。脩之在虜中,多畜妻妾,男女甚多。元嘉二十三年,死於虜中,時年七十二。元矯歷宛陵、江乘、溧陽令。



 傅弘之,字仲度,北地泥陽人。傅氏舊屬靈州,漢末郡境為虜所侵,失土寄寓馮翊,置泥陽、富平二縣,靈州廢不立,故傅氏還屬泥陽。晉武帝太康三年,復立靈州縣,傅氏悉屬靈州。弘之高祖晉司徒祗,後封靈州公,不欲封本縣,故祗一門還復泥陽。曾祖暢,祕書丞,沒胡,生子洪,晉穆帝永和中,胡亂得還。洪生韶,梁州刺史,散騎常侍。韶生弘之。



 少倜儻有大志,為本州主簿,舉秀才,不行。桓玄將篡,新野人庾仄起兵於南陽,襲雍州刺史馮該,該
 走。弘之時在江陵,與仄兄子彬謀殺荊州刺史桓石康,以荊州刺史應仄。彬從弟宏知其謀,以告石康,石康收彬殺之,繫弘之於獄。桓玄以弘之非造謀,又白衣無兵眾,原不罪。



 義旗建,輔國將軍道規以為參軍、寧遠將軍、魏興太守。盧循作亂,桓石綏自上洛甲口自號荊州刺史,征陽令王天恩自號梁州刺史,襲西城。時韶為梁州,遣弘之討石綏等,並斬之。除太尉行參軍。從征司馬休之,署後部賊曹,仍為建威將軍、順陽太守。高祖北伐,弘之
 與扶風太守沈田子等七軍自武關入,偽上洛太守囗脫奔走,進據藍田,招懷戎、晉。晉人龐斌之、戴養、胡人康橫等各率部落歸化。弘之素善騎乘,高祖至長安,弘之於姚泓馳道內,緩服戲馬,或馳或驟,往反二十里中,甚有姿制。羌胡聚觀者數千人,並驚惋歎息。初上馬,以馬鞭柄策,挽致兩股內,及下馬,柄孔猶存。



 進為桂陽公義真雍州治中從事史,除西戎司馬、寧朔將軍。略陽太守徐師高反叛,弘之討平之。高祖歸後,佛佛偽太子赫連
 瑰率眾三萬襲長安,弘之又領步騎五千,於池陽大破之,殺傷甚眾。瑰又抄掠渭南,弘之又於寡婦人渡破瑰,獲賊三百,掠七千餘口。又義真東歸,佛佛傾國追躡,於青泥大戰,弘之身貫甲胄,氣冠三軍。



 軍敗,陷沒,佛佛逼令降,弘之不為屈。時天寒,裸弘之,弘之叫罵見殺。時年四十二。



 史臣曰:三代之隆,畿服有品,東漸西被,無遺遐荒。及漢氏闢土,通譯四方,風教淺深,優劣已遠。晉室播遷,來宅
 揚、越,關、朔遙阻,隴、水開遐荒,區甸分其內外,山河判其表裏,而羌、戎雜合,久絕聲教,固宜待以荒服,羈縻而已也。



 若其懷道畏威,奉王受職,則通以書軌,班以王規。若負其岨遠,屈彊邊垂,則距險閉關,禦其寇暴。桓溫一世英人,志移晉鼎,自非兵屈霸上,戰衄枋頭,則光宅之運,中年允集。高祖無周世累仁之基,欲力征以君四海,實須外積武功,以收天下人望。止欲掛旆龍門,折衝冀、趙,跨功桓氏,取高昔人,地未闢於東晉,威獨振於江南,然後
 可以變國情,愜民志,撫歸運而膺寶策。豈不知秦川不足供養,百二難以傳後哉!至舉咸陽而棄之,非失算也。此四將藉歸眾難固之情,已至於俱陷,為不幸矣。






\end{pinyinscope}