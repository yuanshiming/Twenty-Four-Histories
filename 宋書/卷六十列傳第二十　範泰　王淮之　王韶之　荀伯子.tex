\article{卷六十列傳第二十 範泰 王淮之 王韶之 荀伯子}

\begin{pinyinscope}

 范泰,字伯倫,順陽山陰人也。祖汪,晉安北將軍、徐兗二州刺史。父寧,豫章太守。泰初為太學博士,衛將軍謝安、驃騎將軍會稽王道子二府參軍。荊州刺史王忱,泰外
 弟也,請為天門太守。忱嗜酒,醉輒累旬,及醒,則儼然端肅。泰謂忱曰:「酒雖會性,亦所以傷生。游處以來,常欲有以相戒,當卿沈湎,措言莫由,及今之遇,又無假陳說。」忱嗟嘆久之,曰:「見規者眾矣,未有若此者也。」或問忱曰:「范泰何如謝邈?」忱曰:「茂度慢。」又問:「何如殷覬?」忱曰:「伯通易。」忱常有意立功,謂泰曰:「今城池既立,軍甲亦充,將欲掃除中原,以申宿昔之志。伯通意銳,當令擁戈前驅。以君持重,欲相委留事,何如?」泰曰:「百年逋寇,前賢挫屈者多
 矣。功名雖貴,鄙生所不敢謀。」會忱病卒。召泰為驃騎諮議參軍,遷中書侍郎。時會稽王世子元顯專權,內外百官請假,不復表聞,唯簽元顯而已。泰建言以為非宜,元顯不納。父憂去職,襲爵陽遂鄉侯。桓玄輔晉,使御史中丞祖台之奏泰及前司徒左長史王準之、輔國將軍司馬珣之並居喪無禮,泰坐廢徙丹徒。



 義旗建,國子博士。司馬休之為冠軍將軍、荊州刺史,以泰為長史、南郡太守。



 又除長沙相,散騎常侍,並不拜。入為黃門郎,御史中
 丞。坐議殷祠事謬,白衣領職。出為東陽太守。盧循之難,泰預發兵千人,開倉給稟,高祖加泰振武將軍。明年,遷侍中,尋轉度支尚書。時僕射陳郡謝混,後進知名,高祖嘗從容問混:「泰名輩可以比誰?」對曰:「王元太一流人也。」徙為太常。



 初,司徒道規無子,養太祖,及薨,以兄道憐第二子義慶為嗣。高祖以道規素愛太祖,又令居重。道規追封南郡公,應以先華容縣公賜太祖。泰議曰:「公之友愛,即心過厚。禮無二嗣,諱宜還本屬。」從之。轉大司馬左
 長史,右衛將軍,加散騎常侍。復為尚書,常侍如故。兼司空,與右僕射袁湛授宋公九錫,隨軍到洛陽。



 高祖還彭城,與共登城,泰有足疾,特命乘輿。泰好酒,不拘小節,通率任心,雖在公坐,不異私室,高祖甚賞愛之。然拙於為治,故不得在政事之官。遷護軍將軍,以公事免。高祖受命,拜金紫光祿大夫,加散騎常侍。明年,議建國學,以泰領國子祭酒。泰上表曰:臣聞風化興於哲王,教訓表於至世。至說莫先講習,甚樂必寄朋來。古人成童入學,易
 子而教,尋師無遠,負糧忘艱,安親光國,莫不由此。若能出不由戶,則斯道莫從。是以明詔爰發,已成渙汗,學制既下,遠近遵承。臣之愚懷,少有未達。



 今惟新告始,盛業初基,天下改觀,有志景慕。而置生之制,取少停多,開不來之端,非一塗而已。臣以家推國,則知所聚不多,恐不足以宣大宋之風,弘濟濟之美。臣謂合選之家,雖制所未達,父兄欲其入學,理合開通;雖小違晨昏,所以大弘孝道。不知《春秋》,則所陷或大,故趙盾忠而書弒,許子孝
 而得罪,以斯為戒,可不懼哉!十五志學,誠有其文,若年降無幾,而深有志尚者,何必限以一格,而不許其進邪!揚烏豫《玄》,實在弱齒;五十學《易》,乃無大過。



 昔中朝助教,亦用二品。潁川陳載已辟太保掾,而國子取為助教,即太尉淮之弟。所貴在於得才,無系於定品。教學不明,獎厲不著,今有職閑而學優者,可以本官領之,門地二品,宜以朝請領助教,既可以甄其名品,斯亦敦學之一隅。其二品才堪,自依舊從事。會今生到有期,而學校未立。
 覆簣實望其速,回轍已淹其遲。



 事有似賒而宜急者,殆此之謂。古人重寸陰而賤尺璧,其道然也。



 時學竟不立。時言事者多以錢貨減少,國用不足,欲悉市民銅,更造五銖錢。



 泰又諫曰:流聞將禁私銅,以充官銅。民雖失器,終於獲直,國用不足,其利實多。臣愚意異,不寧寢默。臣聞治國若烹小鮮,拯敝莫若務本。百姓不足,君孰與足。未有民貧而國富,本不足而末有餘者也。故囊漏貯中,識者不吝;反裘負薪,存毛實難。



 王者不言有無,諸侯不
 言多少,食祿之家,不與百姓爭利。故拔葵所以明治,織蒲謂之不仁,是以貴賤有章,職分無爽。



 今之所憂,在農民尚寡,倉廩未充,轉運無已,資食者眾,家無私積,難以禦荒耳。夫貨存貿易,不在少多,昔日之貴,今者之賤,彼此共之,其揆一也。但令官民均通,則無患不足。若使必資貨廣以收國用者,則龜貝之屬,自古所行。尋銅之為器,在用也博矣。鐘律所通者遠,機衡所揆者大。夏鼎負《圖》,實冠眾瑞,晉鐸呈象,亦啟休徵。器有要用,則貴賤同
 資;物有適宜,則家國共急。今毀必資之器,而為無施之錢,於貨則功不補勞,在用則君民俱困,校之以實,損多益少。



 陛下勞謙終日,無倦庶務,以身率物,勤素成風,而頌聲不作,板、渭不至者,良由基根未固,意在遠略。伏願思可久之道,賒欲速之情,弘山海之納,擇芻收之說,則嘉謀日陳,聖慮可廣。其亡存心,然後苞桑可繫。愚誠一至,用忘寢食。



 景平初,加位特進。明年,致仕,解國子祭酒。少帝在位,多諸愆失,上封事極諫,曰:伏聞陛下時在後
 園,頗習武備,鼓鞞在宮,聲聞于外;黷武掖庭之內,喧譁省闥之間,不聞將帥之臣,統御之主,非徒不足以威四夷,祗生遠近之怪。近者東寇紛擾,皆欲伺國瑕隙,今之吳會,寧過二漢關、河,根本既搖,于何不有。如水旱成災,役夫不息,無寇而戒,為費漸多。河南非復國有,羯虜難以理期,此臣所以用忘寢食,而干非其位者也。



 陛下踐阼,委政宰臣,實同高宗諒闇之美。而更親狎小人,不免近習,懼非社稷至計,經世之道。王言如絲,其出如綸,下
 觀而化,疾於影響。伏願陛下思弘古道,式遵遺訓,從理無滯,任賢勿疑,如此則天下歸德,宗社惟永。《書》云:「一人有慶,兆民賴之。」天高聽卑,無幽不察,興衰在人,成敗易曉,未有政治在於上而人亂於下者也。



 臣蒙先朝過遇,陛下殊私,實欲盡心竭誠,少報萬分;而惛耄已及,百疾互生,便為永違聖顏,無復自盡之路,貪及視息,陳其狂瞽。陛下若能哀其所請,留心覽察,則臣夕殞于地,無恨九泉。



 少帝雖不能納,亦不加譴。徐羨之、傅亮等與泰素
 不平,及廬陵王義真、少帝見害,泰謂所親曰:「吾觀古今多矣,未有受遺顧託,而嗣君見殺,賢王嬰戮者也。」



 元嘉二年,表賀元正,并陳旱災,曰:元正改律,品物惟新。陛下藉日新以畜德,仰乾元以履祚,吉祥集室,百福來庭。頃旱魃為虐,亢陽愆度,通川燥流,異井同竭。老弱不堪遠汲,貧寡單於負水。



 租輸既重,賦稅無降,百姓怨咨。臣年過七十,未見此旱。陰陽并隔,則和氣不交,豈惟凶荒,必生疾疫,其為憂虞,不可備序。



 雩絜之典,以誠會事,巫祝
 常祈,罕能有感,上天之譴,不可不察。漢東海枉殺孝婦,亢旱三年;及祭其墓,澍雨立降,歲以有年。是以衛人伐邢,師興而雨。



 伏願陛下式遵遠猷,思隆高構,推忠恕之愛,矜冤枉之獄,遊心下民之瘼,厝思幽冥之紀。令謗木豎闕,諫鼓鳴朝,察芻牧之言,總統御之要。如此,則苞桑可系,危幾無兆。斯而災害不消,未之有也。故夏禹引百姓之罪,殷湯甘萬方之過,太戊資桑穀以進德,宋景藉熒惑以脩善,斯皆因敗以轉成,往事之昭晰也。循末俗
 者難為風,就正路者易為雅。臣疾患日篤,夕不謀朝,會及歲慶,得一聞達,微誠少亮,無恨泉壤,永違聖顏,拜表悲咽。



 遂輕舟遊東陽,任心行止,不關朝廷。有司劾奏之,太祖不問也。時太祖雖當陽親覽,而羨之等猶秉重權,復上表曰:「伏承廬陵王已復封爵,猶未加贈。陛下孝慈天至,友于過隆,伏揆聖心,已自有在。但司契以不唱為高,冕旒以因寄成用。



 臣雖言不足採,誠不亮時,但猥蒙先朝忘醜之眷,復沾廬陵矜顧之末,息晏委質,有兼常
 款,契闊戎陣,顛狽艱危,厚德無報,授令路絕,此老臣兼不能自已者也。



 朽謝越局,無所逃刑。」泰諸子禁之,表竟不奏。



 三年,羨之等伏誅,進位侍中、左光祿大夫、國子祭酒,領江夏王師,特進如故。上以泰先朝舊臣,恩禮甚重,以有腳疾,起居艱難,宴見之日,特聽乘輿到坐。



 累陳時事,上每優容之。其年秋,旱蝗,又上表曰:陛下昧旦丕顯,求民之瘼,明斷庶獄,無倦政事,理出群心,澤謠民口,百姓翕然,皆自以為遇其時也。災變雖小,要有以致之。守
 宰之失,臣所不能究;上天之譴,臣所不敢誣。有蝗之處,縣官多課民捕之,無益於枯苗,有傷於殺害。臣聞桑穀時亡,無假斤斧,楚昭仁愛,不絜自瘳,卓茂去無知之蟲,宋均囚有異之虎,蝗生有由,非所宜殺。石不能言,星不自隕,《春秋》之旨,所宜詳察。



 禮,婦人有三從之義,而無自專之道;《周書》父子兄弟,罪不相及,女人被宥,由來尚矣。謝晦婦女,猶在尚方,始貴後賤,物情之所甚苦,匹婦一至,亦能有所感激。臣於謝氏,不容有情,蒙國重恩,寢處
 思報,伏度聖心,已當有在。



 禮春夏教詩,無一而闕也。臣近侍坐,聞立學當在入年。陛下經略粗建,意存民食,入年則農功興,農功興則田里闢,入秋治庠序,入冬集遠生,二塗並行,事不相害。夫事多以淹稽為戒,不遠為患,任臣學官,竟無微績,徒墜天施,無情自處。臣之區區,不望目睹盛化,竊慕子囊城郢之心,庶免荀偃不瞑之恨。臣比陳愚見,便是都無可採,徒煩天聽,愧作反側。



 書奏,上乃原謝晦婦女。



 時司徒王弘輔政,泰謂弘曰:「天下務
 廣,而權要難居;卿兄弟盛滿,當深存降挹。彭城王,帝之次弟,宜徵還入朝,共參朝政。」弘納其言。



 時旱災未已,加以疾疫,泰又上表曰:「頃亢旱歷時,疾疫未已,方之常災,實為過差,古以為王澤不流之徵。陛下昧旦臨朝,無懈治道,躬自菲薄,勞心民庶,以理而言,不應致此。意以為上天之於賢君,正自殷勤無已。陛下同規禹、湯引百姓之過,言動於心,道敷自遠。桑穀生朝而殞,熒惑犯心而退,非唯消災弭患,乃所以大啟聖明;靈雨立降,百姓改
 瞻,應感之來,有同影響。陛下近當仰推天意,俯察人謀,升平之化,尚存舊典,顧思與不思,行與不行耳。大宋雖揖讓受終,未積有虞之道,先帝登遐之日,便是道消之初。至乃嗣主被殺,哲籓嬰禍,九服俳徊,有心喪氣,佐命託孤之臣,俄為戎首。天下蕩蕩,王道已淪,自非神英,撥亂反正,則宗社非復宋有。革命之與隨時,其義尤大。是以古今異用,循方必壅,大道隱於小成,欲速或未必達。深根固蒂之術,未洽於愚心,是用猖狂妄作而不能緘
 默者也。



 臣既頑且鄙,不達治宜,加之以篤疾,重之以昏耄,言或非言而復不能無言,陛下錄其一毫之誠,則臣不知厝身之所。」



 泰博覽篇籍,好為文章,愛獎後生,孜孜無倦。撰《古今善言》二十四篇及文集,傳於世。暮年事佛甚精,於宅西立祗洹精舍。五年,卒,時年七十四。追贈車騎將軍,侍中、特進、王師如故。謚曰宣侯。



 長子昂,早卒。次子暠,宜都太守。次晏,侍中、光祿大夫。次曄,太子詹事,謀反伏誅,自有傳。少子廣淵,善屬文,世祖撫軍諮議參軍,
 領記室,坐曄事從誅。



 王淮之,字元曾,琅邪臨沂人。高祖彬,尚書僕射。曾祖彪之,尚書令。祖臨之,父納之,並御史中丞。彪之博聞多識,練悉朝儀,自是家世相傳,並諳江左舊事,緘之青箱,世人謂之「王氏青箱學」。



 淮之兼明《禮傳》,贍於文辭。起家為本國右常侍,桓玄大將軍行參軍。玄篡位,以為尚書祠部郎。義熙初,又為尚書中兵郎,遷參高祖車騎中軍軍事,丹陽丞,中軍太尉主簿,出為山陰令,有能名。預討盧
 循功,封都亭侯。又為高祖鎮西、平北、太尉參軍,尚書左丞,本郡大中正。宋臺建,除御史中丞,為僚友所憚。淮之父納之、祖臨之、曾祖彪之至淮之,四世居此職。淮之嘗作五言,范泰嘲之曰:「卿唯解彈事耳。」淮之正色答:「猶差卿世載雄狐。」坐世子右衛率謝靈運殺人不舉,免官。



 高祖受命,拜黃門侍郎。永初二年,奏曰:「鄭玄注《禮》,三年之喪,二十七月而吉,古今學者多謂得禮之宜。晉初用王肅議,祥衣覃共月,故二十五月而除,遂以為制。江左以來,
 唯晉朝施用;縉紳之士,多遵玄義。夫先王制禮,以大順群心。喪也寧戚,著自前訓。今大宋開泰,品物遂理。愚謂宜同即物情,以玄義為制,朝野一禮,則家無殊俗。」從之。



 遷司徒左長史,出為始興太守。元嘉二年,為江夏王義恭撫軍長史、歷陽太守,行州府之任,綏懷得理,軍民便之。尋入為侍中。明年,徙為都官尚書,改領吏部。



 性峭急,頗失縉紳之望。出為丹陽尹。淮之究識舊儀,問無不對,時大將軍彭城王義康錄尚書事,每歎曰:「何須高論玄
 虛,正得如王淮之兩三人,天下便治矣。」



 然寡乏風素,不為時流所重。撰《儀注》,朝廷至今遵用之。十年,卒,時年五十六。追贈太常。子興之,征虜主簿。



 王韶之,字休泰,琅邪臨沂人也。曾祖暠,晉驃騎將軍。祖羨之,鎮軍掾。父偉之,本國郎中令。韶之家貧,父為烏程令,因居縣境。好史籍,博涉多聞。初為衛將軍謝琰行參軍。偉之少有志尚,當世詔命表奏,輒自書寫。泰元、隆安時事,小大悉撰錄之,韶之因此私撰《晉安帝陽秋》。既成,
 時人謂宜居史職,即除著作佐郎,使續後事,訖義熙九年。善敘事,辭論可觀,為後代佳史。遷尚書祠部郎。



 晉帝自孝武以來,常居內殿,武官主書於中通呈,以省官一人管司詔誥,任在西省,因謂之西省郎。傅亮、羊徽相代,領西省事。轉中書侍郎。安帝之崩也,高祖使韶之與帝左右密加鴆毒。恭帝即位,遷黃門侍郎,領著作郎,西省如故。凡諸詔奏,皆其辭也。



 高祖受禪,加驍騎將軍、本郡中正,黃門如故,西省職解,復掌宋書。有司奏東冶士朱
 道民禽三叛士,依例放遣,韶之啟曰:「尚書金部奏事如右,斯誠檢忘一時權制,懼非經國弘本之令典。臣尋舊制,以罪補士,凡有十餘條,雖同異不紊,而輕重實殊。至於詐列父母死,誣罔父母淫亂,破義反逆,此四條,實窮亂抵逆,人理必盡。雖復殊刑過制,猶不足以塞莫大之罪。既獲全首領,大造已隆,寧可復遂拔徒隸,緩帶當年,自同編戶,列齒齊民乎?臣懼此制永行,所虧實大。方今聖化惟新,崇本棄末,一切之令,宜加詳改。愚謂此四條
 不合加贖罪之恩。」侍中褚淡之同韶之三條,卻宜仍舊。詔可。又駮員外散騎侍郎王實之請假事曰:「伏尋舊制,群臣家有情事,聽併急六十日。太元中改制,年賜假百日。又居在千里外,聽並請來年限,合為二百日。此蓋一時之令,非經通之旨。會稽雖途盈千里,未足為難,百日歸休,於事自足。若私理不同,便應自表陳解,豈宜名班朝列,而久淹私門?臣等參議,謂不合開許。或家在河、洛及嶺、沔、漢者,道阻且長,猶宜別有條品,請付尚書詳為
 其制。」從之。坐璽封謬誤,免黃門,事在《謝晦傳》。



 韶之為晉史,序王珣貨殖,王廞作亂。珣子弘,廞子華,並貴顯,韶之懼為所陷,深結徐羨之、傅亮等。少帝即位,遷侍中,驍騎如故。景平元年,出為吳興太守。羨之被誅,王弘入為相,領揚州刺史。弘雖與韶之不絕,諸弟未相識者,皆不復往來。韶之在郡,常慮為弘所繩,夙夜勤厲,政績甚美,弘亦抑其私憾。太祖兩嘉之。在任積年,稱為良守,加秩中二千石。十年,徵為祠部尚書,加給事中。坐去郡長取送
 故,免官。十二年,又出為吳興太守。其年卒,時年五十六。七廟歌辭,韶之制也。文集行於世。子曄,尚書駕部外兵郎,臨賀太守。



 荀伯子,潁川潁陰人也。祖羨,驃騎將軍。父猗,秘書郎。伯子少好學,博覽經傳,而通率好為雜戲,遨遊閭里,故以此失清塗。解褐為駙馬都尉,奉朝請,員外散騎侍郎。著作郎徐廣重其才學,舉伯子及王韶之並為佐郎,助撰晉史及著桓玄等傳。遷尚書祠部郎。



 義熙九年,上表曰:「
 臣聞咎由亡後,臧文以為深歎;伯氏奪邑,管仲所以稱仁。功高可百世不泯,濫賞無崇朝宜許。故太傅鉅平侯祜,明德通賢,宗臣莫二,勳參佐命,功成平吳,而後嗣闕然,烝嘗莫寄。漢以蕭何元功,故絕世輒紹。愚謂鉅平之封,宜同酂國。故太尉廣陵公陳淮,黨翼孫秀,禍加淮南,竊饗大國,因罪為利。值西朝政刑失裁,中興復因而不奪。今王道惟新,豈可不大判臧否?謂廣陵之國,宜在削除。故太保衛瓘,本爵蕭陽縣公,既被橫禍,及進弟秩,始
 贈蘭陵,又轉江夏。中朝公輔,多非理終,瓘功德不殊,亦無緣獨受偏賞,宜復本封,以正國章。」詔付門下。



 前散騎常侍江夏公衛璵上表自陳曰:「臣乃祖故太保瓘,於魏咸熙之中,太祖文皇帝為元輔之日,封蕭陽侯;大晉受禪,進爵為公。歷位太保,總錄朝政。于時賈庶人及諸王用事,忌瓘忠節,故楚王瑋矯詔致禍。前朝以瓘秉心忠正,加以伐蜀之勳,故追封蘭陵郡公。永嘉之中,東海王越食蘭陵,換封江夏,戶邑如舊。臣高祖散騎侍郎璪,囗之
 嫡孫,纂承封爵。中宗元皇帝以曾祖故右衛將軍崇承襲,逮于臣身。伏聞祠部郎荀伯子表,欲貶降復封蕭陽。夫趙氏之忠,寵延累葉,漢祖開封,誓以山河。伏願陛下錄既往之勳,垂罔極之施,乞出臣表,付外參詳。」潁川陳茂先亦上表曰:「祠部郎荀伯子表臣七世祖太尉淮禍加淮南,不應濫賞。尋先臣以剪除賈謐,封海陵公,事在淮南遇禍之前。後廣陵雖在擾攘之際,臣祖乃始蒙殊遇,歷位元、凱。後被遠外,乃作平州,而猶不至除國。良以
 先勳深重,百世不泯故也。



 聖明御世,英輔係興,曾無疑議,以為濫賞。臣以微弱,未齒人倫,加始勉視息,封爵兼嗣。伏願陛下遠錄舊勳,特垂矜察。」詔皆付門下,並不施行。



 伯子為世子征虜功曹,國子博士。妻弟謝晦薦達之,入為尚書左丞,出補臨川內史。車騎將軍王弘稱之曰:「沈重不華,有平陽侯之風。」伯子常自矜蔭藉之美,謂弘曰:「天下膏粱,唯使君與下官耳。宣明之徒,不足數也。」遷散騎常侍,本邑大中正。又上表曰:「伏見百官位次,陳留
 王在零陵王上,臣愚竊以為疑。昔武王剋殷,封神農之後於焦,黃帝之後於祝,帝堯之後於薊,帝舜之後於陳,夏後於杞,殷後於宋。杞、陳並為列國,而薊、祝、焦無聞焉。斯則褒崇所承,優於遠代之顯驗也。是以《春秋》次序諸侯,宋居杞、陳之上。考之近世,事亦有征。



 晉泰始元年,詔賜山陽公劉康子弟一人爵關內侯,衛公姬署、宋侯孔紹子一人駙馬都尉。又泰始三年,太常上博士劉跂等議,稱衛公署於大晉在三恪之數,應降稱侯。



 臣以零陵
 王位宜在陳留之上。」從之。



 遷太子僕,御史中丞,蒞職勤恪,有匪躬之稱;立朝正色,外內憚之。凡所奏劾,莫不深相謗毀,或延及祖禰,示其切直;又頗雜嘲戲,故世人以此非之。出補司徒左長史,東陽太守。元嘉十五年,卒官,時年六十一。文集傳於世。



 子赤松,為尚書左丞,以徐湛之黨,為元凶所殺。伯子族弟昶,字茂祖,與伯子絕服五世。元嘉初,以文義至中書郎。昶子萬秋,字元寶,亦用才學自顯。世祖初,為晉陵太守。坐於郡立華林閣,置主書、
 主衣,下獄免。前廢帝末,為御史中丞,卒官。



 史臣曰:夫令問令望,詩人所以作詠;有禮有法,前謨以之垂美。荀、範、二王,雖以學義自顯,而在朝之譽不弘,蓋由才有餘而智未足也,惜矣哉!



\end{pinyinscope}