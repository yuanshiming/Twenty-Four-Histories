\article{卷十六志第六 禮三}

\begin{pinyinscope}

 「國之大事,在祀與戎」。自書契經典,咸崇其義,而聖人之德,莫大於嚴父者也。故司馬遷著《封禪書》,班固備《郊祀志》,上紀皇王正祀,下錄郡國百神。



 司馬彪又著《祭祀志》,
 以續終漢。中興以後,其舊制誕章,粲然弘備。自茲以降,又有異同,故復撰次雲爾。



 漢獻帝延康元年十一月己丑,詔公卿告祠高廟。遣兼御史大夫張音奉皇帝璽綬策書,禪帝位于魏。是時魏文帝繼王位,南巡在潁陰。有司乃為壇於潁陰之繁陽故城。庚午,登壇。魏相國華歆跪受璽紱以進於王。既受畢,降壇視燎,成禮而返,未有祖配之事。魏文帝黃初二年正月,郊祀天地明堂。是時魏都洛京,而神祇兆域明
 堂靈臺,皆因漢舊事。四年七月,帝將東巡,以大軍當出,使太常以一特牛告祠南郊,自后以為常。及文帝崩,太尉鐘繇告謚南郊,皆是有事於郊也。



 明帝太和元年正月丁未,郊祀武皇帝以配天,宗祀文皇帝於明堂以配上帝。是時二漢郊禋之制具存,魏所損益可知也。



 四年八月,帝東巡,過繁昌,使執金吾臧霸行太尉事,以特牛祠受禪壇。《後漢紀》,章帝詔高邑祠即位壇。此雖前
 代已行之事,然為壇以祀天,而壇非神也。



 今無事於上帝,而致祀於虛壇,未詳所據也。



 景初元年十月乙卯,始營洛陽南委粟山為圓丘,詔曰:「蓋帝王受命,莫不恭承天地,以彰神明;尊祀世統,以昭功德。故先代之典既著,則禘郊祖宗之制備也。



 昔漢氏之初,承秦滅學之後,採摭殘缺,以備郊祀。自甘泉、后土、雍宮、五畤神祗兆位,多不經見,並以興廢無常,一彼一此,四百餘年,廢無禘禮。古代之所更立者,遂有闕焉。曹
 氏世系,出自有虞氏。今祀圓丘,以始祖帝舜配,號圓丘曰皇皇帝天;方丘所祭曰皇皇后地,以舜妃伊氏配;天郊所祭曰皇天之神,以太祖武皇帝配;地郊所祭曰皇地之祗,以武宣皇後配;宗祀皇考高祖文皇帝于明堂,以配上帝。」十二月壬子冬至,始祀皇皇帝天於圓丘,以始祖有虞帝舜配。自正始以後,終魏世,不復郊祀。



 孫權初稱尊號於武昌,祭南郊告天。文曰:「皇帝臣孫權,敢用玄牡,昭告皇皇后帝。漢饗國二十有四世,歷年四
 百三十,行氣數終,祿胙運盡,普天弛絕,率土分崩。孽臣曹丕,遂奪神器;丕子睿繼世作慝,竊名亂制。權生於東南,遭值期運,承乾秉戎,志在拯世,奉辭行罰,舉足為民。群臣將相州郡百城執事之人,咸以為天意已去於漢,漢氏已終於天。皇帝位虛,郊祀無主,休徵嘉瑞,前後雜沓,歷數在躬,不得不受。權畏天命,敢不敬從。謹擇元日,登壇柴燎,即皇帝位。唯爾有神饗之!左右有吳,永綏天極。」其後自以居非中土,不復脩設。中年,群臣奏議,宜脩
 郊祀,權曰:「郊祀當於中土,今非其所。」重奏曰:「普天之下,莫非王土。王者以天下為家。昔周文、武郊於禜、鎬,非必中土。」權曰:「武王伐紂,即阼於鎬京,而郊其所也。文王未為天子,立郊於禜,見何經典?」復奏曰:「伏見《漢書·郊祀志》,匡衡奏徙甘泉河東郊於長安,言文王郊於禜。」權曰:「文王德性謙讓,處諸侯之位,明未郊也。經傳無明文,由匡衡俗儒意說,非典籍正義,不可用也。」虞喜《志林》曰:「吳主糾駁郊祀,追貶匡衡,凡在見者,莫不慨然稱善也。」何承
 天曰:「案權建號繼天,而郊享有闕,固非也。末年雖一南郊,而遂無北郊之禮。環氏《吳紀》:『權思崇嚴父配天之義,追上父堅尊號為吳始祖。』如此說,則權末年所郊,堅配天也。權卒後,三嗣主終吳世不郊祀,則權不享配帝之禮矣。」



 劉備章武元年,即皇帝位,設壇。「建安二十六年夏四月丙午,皇帝臣備,敢用玄牡,昭告皇天上帝、后土神祗。漢有天下,歷數無疆。曩者王莽篡盜,光武皇帝震怒致誅,
 社稷復享。今曹操阻兵安忍,子丕載其凶逆,竊居神器。群臣將士以為社稷墮廢,備宜脩之,嗣武二祖,龔行天罰。備惟否德,懼忝帝位,詢于庶民,外及蠻夷君長,僉曰天命不可以不答,祖業不可以久替,四海不可以無主,率土式望,在備一人。備畏天之威,又懼漢邦將湮于地。謹擇元日,與百僚登壇,受皇帝璽綬。修燔瘞,告類于大神。惟大神尚饗!祚於漢家,永綏四海。」章武二年十月,詔丞相諸葛亮營南北郊于成都。



 魏元帝咸熙二年十二月甲子,使持節侍中太保鄭沖、兼太尉司隸校尉李喜奉皇帝璽綬策書,禪帝位於晉。丙寅,晉設壇場於南郊,柴燎告類,未有祖配。其文曰:「皇帝臣炎,敢用玄牡,明告于皇皇后帝。魏帝稽協皇運,紹天明命,以命炎曰:『昔者唐堯禪位虞舜,虞舜又以禪禹,邁德垂訓,多歷年載。暨漢德既衰,太祖武皇帝撥亂濟民,扶翼劉氏,又用受禪於漢。粵在魏室,仍世多故,幾於顛墜,實賴有晉匡拯之德,用獲保厥肆祀,弘濟于艱難,
 此則晉之有大造于魏也。誕惟四方之民,罔不祗順,開國建侯,宣禮明刑,廓清梁、岷,苞懷揚、越,函夏興仁,八紘同軌,遐邇弛義,祥瑞屢臻,天人協應,無思不服。肆予憲章三后,用集大命于茲。』炎惟德不嗣,辭不獲命。于是群公卿士,百辟庶僚,黎獻陪隸,暨於百蠻君長,僉曰:『皇天鑒下,求民之瘼,既有成命,固非克讓所得距違。』天序不可以無統,人神不可以曠主,炎虔奉皇運,畏天之威,敢不欽承休命,敬簡元辰,升壇受禪,告類上帝,以永答民
 望,敷佑萬國。惟明德是饗。」



 泰始二年正月,詔曰:「有司前奏郊祀權用魏禮。朕不慮改作之難,今便為永制。眾議紛互,遂不時定,不得以時供饗神祀,配以祖考,日夕歎企,貶食忘安。



 其便郊祀。」時群臣又議:「五帝,即天也。五氣時異,故殊其號。雖名有五,其實一神。明堂南郊,宜除五帝之坐。五郊改五精之號,皆同稱昊天上帝,各設一坐而已。北郊又除先后配祀。」帝悉從之。二月丁丑,郊祀宣皇帝以配天,宗祀文
 皇帝于明堂,以配上帝。是年十一月,有司又議奏:「古者丘郊不異,宜并圓丘方澤於南北郊,更脩治壇兆。其二至之祀,合於二郊。」帝又從之,一如宣帝所用王肅議也。是月庚寅冬至,帝親祠圓丘於南郊。自是後,圓丘方澤不別立至今矣。太康十年十月,乃更詔曰:「《孝經》『郊祀后稷以配天,宗祀文王於明堂,以配上帝』。



 而《周官》云:『祀天旅上帝。』又曰:『祀地旅四望。』四望非地,則明上帝不得為天也。往者眾議除明堂五帝位,考之禮文正經不通。且《詩
 序》曰:『文、武之功,起於后稷。』故推以配天焉。宣帝以神武創業,既已配天,復以先帝配天,於義亦不安。其復明堂及南郊五帝位。」晉武帝太康三年正月,帝親郊禮。皇太子、皇弟、皇子悉侍祠,非前典也。愍帝都長安,未及立郊廟而敗。



 元帝中興江南,太興元年,始更立郊兆。其制度皆太常賀循依據漢、晉之舊也。



 三月辛卯,帝親郊祀,饗配之禮,一依武帝始郊故事。初,尚書令刁協、國子祭酒杜夷,議
 宜須旋都洛邑乃脩之。司徒荀組據漢獻帝居許,即便立郊,自宜於此脩奉。



 驃騎王導、僕射荀崧、太常華恒、中書侍郎庾亮皆同組議,事遂施行。按元帝紹命中興,依漢氏故事,宜享明堂宗祀之禮。江左不立明堂,故闕焉。明帝太寧三年七月,始詔立北郊。未及建而帝崩,故成帝咸和八年正月,追述前旨,於覆舟山南立之。是月辛未,祀北郊,始以宣穆張皇后配地。魏氏故事,非晉舊也。



 康帝建元元年正月,將北郊,有疑議。太常顧和表曰:「泰
 始中,合二至之祀於二郊。北郊之月,古無明文,或以夏至,或同用陽復。漢光武正月辛未,始建北郊,此則與南郊同月。及中興草創,百度從簡,合北郊於一丘。憲章未備,權用斯禮,蓋時宜也。至咸和中,議別立北郊,同用正月。魏承後漢,正月祭天,以地配,而稱周禮,三王之郊,一用夏正。」於是從和議。是月辛未,南郊。辛巳,北郊。



 帝皆親奉。



 安帝元興三年三月,宋高祖討桓玄走之。己卯,告義功
 于南郊。是年,帝蒙塵江陵未返。其明年應郊,朝議以為宜依周禮,宗伯攝職,三公行事。尚書左丞王納之獨曰:「既殯郊祀,自是天子當陽,有君存焉,稟命而行,何所辨也。齋之與否,豈如今日之比乎?議者又云今宜郊,故是承制所得命三公行事。又郊天極尊,唯一而巳,故非天子不祀也。庶人以上,莫不蒸嘗,嫡子居外,庶子執事,禮文炳然。



 未有不親受命而可祭天者。又武皇受禪,用二月郊,元帝中興,以三月郊。今郊時未過,日望輿駕。無為欲
 速而無據,使皇輿旋返,更不得親奉。」遂從納之議。



 晉恭帝元熙二年五月,遣使奉策,禪帝位于宋。永初元年六月丁卯,設壇南郊,受皇帝璽紱,柴燎告類。策曰:「皇帝臣諱,敢用玄牡,昭告皇皇后帝。晉帝以卜世告終,歷數有歸,欽若景運,以命于諱。夫樹君司民,天下為公,德充帝王,樂推攸集。越俶唐、虞,降暨漢、魏,靡不以上哲格文祖,元勳陟帝位,故能大拯黔黎,垂訓無窮。晉自東遷,四維弗樹,宰輔焉依,為日已久。難棘隆安,禍成元興,遂
 至帝王遷播,宗祀湮滅。諱雖地非齊、晉,眾無一旅,仰憤時難,俯悼橫流,投袂一麾,則皇祚剋復。及危而能持,顛而能扶,姦宄具殲,僭偽必滅。誠否終必泰,興廢有期。至於撥亂濟民,大造晉室,因藉時運,以尸其勞。加以殊俗慕義,重譯來款,正朔所暨,咸服聲教。至乃三靈垂象,山川告祥,人神和協,歲月茲著。是以群公卿士,億兆夷人,僉曰皇靈降監於上,晉朝款誠於下;天命不可以久淹,宸極不可以暫曠。遂逼群議,恭茲大禮。猥以寡德,託於
 兆民之上。雖仰畏天威,略是小節,顧深永懷,祗懼若厲。敬簡元日,升壇受禪,告類上帝,用酬萬國之嘉望。



 克隆天保,永祚于有宋。惟明靈是饗。」



 永初元年,皇太子拜告南北郊。永初二年正月上辛,上親郊祀。文帝元嘉三年,車駕西征謝晦,幣告二郊。



 孝武帝孝建元年六月癸巳,八座奏:「劉義宣、臧質,乾時犯順,滔天作戾,連結淮、岱,謀危宗社。質反之始,戒嚴之
 日,二郊廟社,皆已遍陳。其義宣為逆,未經同告。輿駕將發,醜徒冰消,質既梟懸,義宣禽獲,二寇俱殄,並宜昭告。檢元嘉三年討謝晦之始,普告二郊、太廟。賊既平蕩,唯告太廟、太社,不告二郊。」



 禮官博議。太學博士徐宏、孫勃、陸澄議:「《禮》無不報。始既遍告,今賊已禽,不應不同。」國子助教蘇瑋生議:「案《王制》,天子巡狩,『歸,假于祖禰』。



 又《曾子問》:『諸侯適天子,告于祖,奠于禰,命祝史告于社稷宗廟山川。告用牲幣,反亦如之。諸侯相見,反必告于祖禰,乃
 命祝史告至于前所告者。』又云:『天子諸侯將出,必以幣帛皮圭,告于祖禰。反必告至。』天子諸侯,雖事有小大,其禮略鈞,告出告至,理不得殊。鄭云:『出入禮同。』其義甚明。天子出征,類于上帝,推前所告者歸必告至,則宜告郊,不復容疑。元嘉三年,唯告廟社,未詳其義。或當以《禮記》唯云『歸假祖禰』,而無告郊之辭。果立此義,彌所未達。



 夫《禮記》殘缺之書,本無備體,折簡敗字,多所闕略。正應推例求意,不可動必徵文。天子反行告社,亦無成記,何故
 告郊,獨當致嫌。但出入必告,蓋孝敬之心。



 既以告歸為義,本非獻捷之禮。今輿駕竟未出宮,無容有告至之文;若陳告不行之禮,則為未有前準。愚謂祝史致辭,以昭誠信。茍其義舛於禮,自可從實而闕。臣等參議,以應告為允,宜並用牲告南北二郊、太廟、太社,依舊公卿行事。」詔可。



 孝建二年正月庚寅,有司奏:「今月十五日南郊。尋舊儀,廟祠至尊親奉,以太尉亞獻;南郊親奉,以太常亞獻。又
 廟祠行事之始,以酒灌地;送神則不灌。而郊初灌,同之於廟,送神又灌,議儀不同,於事有疑。輒下禮官詳正。」太學博士王祀之議:「案《周禮》,大宗伯『佐王保國,以吉禮事鬼神祗,禋祀昊天。』則今太常是也。以郊天,太常亞獻。又《周禮》外宗云:『王后不與,則贊宗伯。』鄭玄云:『后不與祭,宗伯攝其事。』又說云:『君執圭瓚稞尸,大宗伯執璋瓚亞獻。』中代以來,后不廟祭,則應依禮大宗伯攝亞獻也。而今以太尉亞獻。鄭注《禮·月令》云:『三王有司馬,無太尉。太尉,
 秦官也。』蓋世代彌久,宗廟崇敬,攝後事重,故以上公亞獻。」又議:「履時之思,情深於霜露;室戶之感,有懷於容聲。不知神之所在,求之不以一處。鄭注《儀禮》有司云,天子諸侯祭於祊而繹。繹又祭也。今廟祠闕送神之稞,將移祭於祊繹,明在於留神,未得而殺。禮郊廟祭殊,故灌送有異。」



 太常丞朱膺之議:「案《周禮》,大宗伯使掌典禮,以事神為上,職總祭祀,而昊天為首。今太常即宗伯也。又尋袁山松《漢·百官志》云:『郊祀之事,太尉掌亞獻,光祿掌三
 獻。太常每祭祀,先奏其禮儀及行事,掌贊天子。』無掌獻事。



 如儀志,漢亞獻之事,專由上司,不由秩宗貴官也。今宗廟太尉亞獻,光祿三獻,則漢儀也。又賀循制太尉由東南道升壇,明此官必預郊祭。古禮雖由宗伯,然世有因革,上司亞獻,漢儀所行。愚謂郊祀禮重,宜同宗廟。且太常既掌贊天子,事不容兼。又尋灌事,《禮記》曰:『祭求諸陰陽之義也。殷人先求諸陽。』『樂三闋然後迎牲。』則殷人後灌也。『周人先求諸陰』,『灌用鬯,達於淵泉。既灌,然後迎
 牲。』則周人先灌也。此謂廟祭,非謂郊祠。案《周禮》天官:『凡祭祀贊王祼將之事。』鄭注云:『祼者,灌也。唯人道宗廟有灌,天地大神至尊不灌。』而郊未始有灌,於禮未詳。淵儒注義,炳然明審。謂今之有灌,相承為失,則宜無灌。」



 通關八座丞郎博士,並同膺之議。尚書令建平王宏重參議,謂膺之議為允。詔可。



 大明二年正月丙午朔,有司奏:「今月六日南郊,輿駕親奉。至時或雨。魏世值雨,高堂隆謂應更用後辛。晉時既
 出遇雨,顧和亦云宜更告。徐禪云:『晉武之世,或用丙,或用己,或用庚。』使禮官議正并詳。若得遷日,應更告廟與不?」



 博士王燮之議稱:「遇雨遷郊,則先代成議。《禮》傳所記,辛日有征。《郊特牲》曰:『郊之用辛也,周之始郊日以至。』鄭玄注曰:『三王之郊,一用夏正。用辛者,取其齋戒自新也。』又《月令》曰:『乃擇元日,祈穀于上帝。』注曰:『元日,謂上辛。郊祭天也。』又《春秋》載郊有二,成十七年九月辛丑,郊。《公羊》曰:『曷用郊?用正月上辛。』哀元年四月辛巳,郊。《穀梁》曰:『自
 正月至于三月,郊之時也。以十二月下辛卜正月上辛,如不從,以正月下辛卜二月上辛;如不從,以二月下辛卜三月上辛。』以斯明之,則郊祭之禮,未有不用辛日者也。晉氏或丙、或己、或庚,並有別議。武帝以十二月丙寅南郊受禪,斯則不得用辛也。



 又泰始二年十一月己卯,始并圓丘方澤二至之祀合於二郊。三年十一月庚寅冬至祠天,郊于圓丘。是猶用圓丘之禮,非專祈穀之祭,故又不得用辛也。今之郊饗,既行夏時,雖得遷卻,謂宜
 猶必用辛也。徐禪所據,或為未宜。又案《郊特牲》曰:『受命于祖廟,作龜于禰宮。』鄭玄注曰:『受命,謂告退而卜也。』則告義在郊,非為告日。今日雖有遷,而郊祀不異,愚謂不宜重告。」



 曹郎朱膺之議:「案先儒論郊,其議不一。《周禮》有冬至日圓丘之祭。《月令》孟春有祈穀于上帝。鄭氏說,圓丘祀昊天上帝,以帝嚳配,所謂禘也。祈穀祀五精之帝,以后稷配,所謂郊也。二祭異時,其神不同。諸儒云,圓丘之祭,以后稷配。取其所在,名之曰郊。以形體言之,謂之
 圓丘。名雖有二,其實一祭。晉武捨鄭而從諸儒,是以郊用冬至日。既以至日,理無常辛。然則晉代中原不用辛日郊,如徐禪議也。江左以來,皆用正月,當以傳云三王之郊,各以其正,晉不改正朔,行夏之時,故因以首歲,不以冬日,皆用上辛,近代成典也。夫祭之禮,『過時不舉』。今在孟春,郊時未過,值雨遷日,於禮無違。既已告日,而以事不從,禋祀重敬,謂宜更告。高堂隆云:『九日南郊,十日北郊。』是為北郊可不以辛也。」



 尚書何偃議:「鄭玄注《禮記》,
 引《易》說三王之郊,一用夏正。《周禮》,凡國大事,多用正歲。《左傳》又啟蟄而郊。則鄭之此說,誠有據矣。眾家異議,或云三王各用其正郊天,此蓋曲學之辯,於禮無取。固知《穀梁》三春皆可郊之月,真所謂膚淺也。然用辛之說,莫不必同。晉郊庚己,參差未見前征。愚謂宜從晉遷郊依禮用辛。燮之所受命作龜,知告不在日,學之密也。」右丞徐爰議以為:「郊祀用辛,有礙遷日,禮官祠曹,考詳已備。何偃據禮,不應重告,愚情所同。尋告郊剋辰,於今宜改,
 告事而已。次辛十日,居然展齋,養牲在滌,無緣三月。謂毛血告泬之後,雖有事礙,便應有司行事,不容遷郊。」眾議不同。參議:「宜依《經》,遇雨遷用後辛,不重告。若殺牲薦血之後值雨,則有司行事。」詔可。



 明帝泰始二年十一月辛酉,詔曰:「朕載新寶命,仍離多難,戎車遄駕,經略務殷,禋告雖備,弗獲親禮。今九服既康,百祀咸秩,宜聿遵前典,郊謁上帝。」



 有司奏檢,未有先準。黃門侍郎徐爰議:「虞稱肆類,殷述昭告。蓋以創世成
 功,德盛業遠,開統肇基,必享上帝。漢、魏以來,聿遵斯典。高祖武皇帝克伐偽楚,晉安帝尚在江陵,即於京師告義功于郊兆。伏惟泰始應符,神武英斷,王赫出討,戎戒淹時,雖司奉弗虧,親謁尚闕。謹尋晉武郊以二月,晉元禋以三月。有非常之慶,必有非常之典,不得拘以常祀,限以正月上辛。愚謂宜下史官,考擇十一月嘉吉,車駕親郊,奉謁昊天上帝,高祖武皇帝配饗。其餘祔食,不關今祭。」尚書令建安王休仁等同爰議。參議為允,詔可。



 泰始六年正月乙亥,詔曰:「古禮王者每歲郊享,爰及明堂。自晉以來,間年一郊,明堂同日。質文詳略,疏數有分。自今可間二年一郊,間歲一明堂。外可詳議。」有司奏:「前兼曹郎虞願議:『郊祭宗祀,俱主天神,而同日殷薦,於義為黷。明詔使圓丘報功,三載一享。明堂配帝,間歲昭薦。詳辰酌衷,實允懋典。』緣諮參議並同。曹郎王延秀重議:『改革之宜,實如聖旨。前虞願議,蓋是仰述而已,未顯後例。謹尋自初郊間二載,明堂間一年,第二郊與第三明
 堂,還復同歲。



 願謂自始郊明堂以後,宜各間二年。以斯相推,長得異歲。』通關八座,同延秀議。」



 後廢帝元徽二年十月丁巳,有司奏郊祀明堂,還復同日,間年一修。



 漢文帝初祭地祇於渭陽,以高帝配;武帝立后土社祠於汾陰,亦以高帝配。漢氏以太祖兼配天地,則未以后配地也。王莽作相,引《周禮》享先妣為配北郊。夏至祭后土,以高后配,自此始也。光武建武中,不立北郊,故后地之祇,常配食天壇,山川群望皆在營內,凡一千五百一
 十四神。中元年,建北郊,使司空馮魴告高廟,以薄後代呂后配地。江左初,未立北壇,地祇眾神,共在天郊也。



 晉成帝立二郊,天郊則六十二神,五帝之佐、日月五星、二十八宿、文昌、北斗、三台、司命、軒轅、后土、太一、天一、太微、鉤陳、北極、雨師、雷電、司空、風伯、老人六十二神也。地郊則四十四神,五嶽、四望、四海、四瀆、五湖、五帝之佐、沂山、嶽山、白山、霍山、醫無閭山、蔣山、松江、會稽山、錢唐江、先農凡四十四也。江南諸小山,蓋江左所立,猶如漢西京
 關中小水,皆有祭秩也。



 二郊所秩,官有其注。



 宋武帝永初三年九月,司空羨之、尚書令亮等奏曰:「臣聞崇德明祀,百王之令典;憲章天人,自昔之所同。雖因革殊時,質文異世,所以本情篤教,其揆一也。



 伏惟高祖武皇帝允協靈祗,有命自天,弘日靜之勤,立蒸民之極,帝遷明德,光宅八表,太和宣被,玄化遐通。陛下以聖哲嗣徽,道孚萬國。祭禮久廢,思光鴻烈,饗帝嚴親,今實宜之。高祖武皇帝宜配天郊;至於地祗之配,雖禮無明文,
 先代舊章,每所因循,魏、晉故典,足為前式。謂武敬皇后宜配北郊。蓋述懷以追孝,躋聖敬於無窮,對越兩儀,允洽幽顯者也。明年孟春,有事於二郊,請宣攝內外,詳依舊典。」詔可。



 晉武帝太康二年冬,有司奏:「三年正月立春祠,時日尚寒,可有司行事。」



 詔曰:「郊祀禮典所重,中間以軍國多事,臨時有所妨廢,故每從奏可。自今方外事簡,唯此為大,親奉禋享,固常典也。」



 成帝祠南郊,遇雨。侍中顧和啟:「宜
 還,更剋日。」詔可。漢明帝據《月令》有五郊迎氣服色之禮,因採元始中故事,兆五郊于洛陽,祭其帝與神,車服各順方色。魏、晉依之。江左以來,未遑修建。



 宋孝武大明五年四月庚子,詔曰:「昔文德在周,明堂崇祀;高烈惟漢,汶邑斯尊。所以職祭罔愆,氣令斯正,鴻名稱首,濟世飛聲。朕皇考太祖文皇帝功耀洞元,聖靈昭俗,內穆四門,仁濟群品,外薄八荒,威憺殊俗,南腦勁越,西髓剛戎。



 裁禮興稼穡之根,張樂協四氣之紀。匡飾墳
 序,引無題之外;旌延寶臣,盡盛德之範。訓深劭農,政高刑厝。萬物棣通,百神薦祉。動協天度,下沿地德。故精緯上靈,動殖下瑞,諸侯軌道,河溓海夷。朕仰憑洪烈,入子萬姓,皇天降祐,迄將一紀。思奉揚休德,永播無窮。便可詳考姬典,經始明堂,宗祀先靈,式配上帝,誠敬克展,幽顯咸秩。惟懷永元,感慕崩心。」



 有司奏:「伏尋明堂辟雍,制無定文,經記參差,傳說乖舛。名儒通哲,各事所見,或以為名異實同,或以為名實皆異。自漢暨晉,莫之能辨。周
 書云,清廟明堂路寢同制。鄭玄注《禮》,義生於斯。諸儒又云明堂在國之陽,丙巳之地,三里之內。至於室宇堂個,戶牖達向,世代湮緬,難得該詳。晉侍中裴頠,西都碩學,考詳前載,未能制定。以為尊祖配天,其義明著,廟宇之制,理據未分,直可為殿,以崇嚴祀。其餘雜碎,一皆除之。參詳鄭玄之注,差有準據;裴頠之奏,竊謂可安。



 國學之南,地實丙巳,爽塏平暢,足以營建。其墻宇規範,宜擬則太廟,唯十有二間,以應期數。依漢汶上圖儀,設五帝位,
 太祖文皇帝對饗。祭皇天上帝,雖為差降,至於三載恭祀,理不容異。自郊徂宮,亦宜共日。《禮記》郊以特牲,《詩》稱明堂羊牛,吉蠲雖同,質文殊典。且郊有燔柴,堂無禋燎,則鼎俎彝簋,一依廟禮。班行百司,搜材簡工,權置起部尚書、將作大匠,量物商程,剋今秋繕立。」



 乃依頠議,但作大殿屋雕畫而已,無古三十六戶七十二牖之制。六年正月,南郊還,世祖親奉明堂,祠祭五時之帝,以文皇帝配,是用鄭玄議也。官有其注。



 大明五年九月甲子,有司奏:「南郊祭用三牛;廟四時祠六室用二牛。明堂肇建,祠五帝,太祖文皇帝配,未詳祭用幾牛?」太學博士司馬興之議:「案鄭玄注《禮記大傳》:稱『《孝經》郊祀后稷以配天,配靈威仰也。宗祀文王於明堂,以配上帝,配五帝也。』夫五帝司方,位殊功一,牲牢之用,理無差降。太祖文皇帝躬成天地,則道兼覆載;左右群生,則化洽四氣。祖、宗之稱,不足彰無窮之美;金石之音,未能播勳烈之盛。故明堂聿脩,聖心所以昭玄極;泛配
 宗廟,先儒所以得禮情。愚管所見,謂宜用六牛。」博士虞龢議:「祀帝之名雖五,而所生之實常一。五德之帝,迭有休王,各有所司,故有五室。宗祀所主,要隨其王而饗焉。主一配一,合用二牛。」祠部郎顏奐議:「祀之為義,並五帝以為言。帝雖云五,牲牢之用,謂不應過郊祭廟祀。宜用二牛。」



 明帝泰始七年十月庚子,有司奏:「來年正月十八日,祠明堂。尋舊南郊與明堂同日,並告太廟。未審今祀明堂,
 復告與不?」祠部郎王延秀議:「案鄭玄云:『郊者祭天之名,上帝者,天之別名也。神無二主,故明堂異處,以避后稷。』謹尋郊宗二祀,既名殊實同,至於應告,不容有異。」守尚書令袁粲等並同延秀議。



 魏明帝世,中護軍蔣濟奏曰:「夫帝王大禮,巡狩為先;昭祖揚禰,封禪為首。



 是以自古革命受符,未有不蹈梁父,登泰山,刊無竟之名,紀天人之際者也。故司馬相如謂有文以來七十二君,或從所由於前,謹遺跡於後。太史
 公曰:『主上有聖明而不宣布,有司之過也。』然則元功懿德,不刊山、梁之石,無以顯帝王之功,布生民不朽之觀也。語曰,當君而歎堯、舜之美,譬猶人子對厥所生,譽他人之父。



 今大魏振前王之弊亂,拯流遁之艱危,接千載之衰緒,繼百世之廢始。自武、文至于聖躬,所以參成天地之道,綱維人神之化,上天報應,嘉瑞顯祥,以比往古,其優衍豐隆,無所取喻。至於歷世迄今,未發大禮。雖志在掃盡殘盜,蕩滌餘穢,未遑斯事。若爾,三苗堀彊於江海,
 大舜當廢東巡之儀;徐夷跳梁於淮、泗,周成當止岱嶽之禮也。且昔歲破吳虜於江、漢,今茲屠蜀賊於隴右。其震蕩內潰,在不復淹,就當探其窟穴,無累於封禪之事也。此儀久廢,非倉卒所定。宜下公卿,廣纂其禮,卜年考時,昭告上帝,以副天下之望。臣待罪軍旅,不勝大願,冒死以聞。」



 詔曰:「聞濟斯言,使吾汗出流足。自開闢以來,封禪者七十餘君爾。故太史公曰:『雖有受命之君,而功有不洽,是以中間曠遠者,千有餘年,近數百載。其儀闕不
 可得記。』吾何德之脩,敢庶茲乎!濟豈謂世無管仲,以吾有桓公登泰山之志乎?



 吾不敢欺天也。濟之所言,華則華矣,非助我者也。公卿侍中、尚書、常侍省之而已。勿復有所議,亦不須答詔也。」帝雖拒濟議,而實使高堂隆草封禪之儀。以天下未一,不欲便行大禮。會隆卒,故不行。



 晉武帝平吳,混一區宇。太康元年九月庚寅,尚書令衛瓘、尚書左僕射山濤、魏舒、尚書劉實、張華等奏曰:「聖德隆茂,光被四表,諸夏乂清,幽荒率從。神策廟算,席卷吳
 越,孫皓稽顙,六合為家,巍巍之功,格于天地。宜同古典,勒封東嶽,告三府太堂為儀制。」瓘等又奏:「臣聞肇自生民,則有后辟,載祀之數,莫之能紀。立德濟世,揮揚仁風,以登封泰山者七十有四家,其謚號可知者,十有四焉。沉淪寂寞,曾無遺聲者,不可勝記。自黃帝以前,古傳昧略,唐、虞以來,典謨炳著。三王代興,體業繼襲,周道既沒,秦氏承之,至于漢、魏,而質文未復。



 大晉之德,始自重、黎,實佐顓頊。至于夏、商,世序天地,其在于周,不失其緒。



 金
 德將升,世濟明聖,外平蜀漢,海內歸心,武功之盛,實由文德。至于陛下受命踐阼,弘建大業,群生仰流,唯獨江湖沅湘之表,凶桀負固,歷代不賓。神謀獨斷,命將出討,兵威暫加,數旬蕩定,羈其鯨鯢,赦其罪逆。雲覆雨施,八方來同,聲教所被,達于四極。雖黃軒之徵,大禹遠略,周之奕世,何以尚今。若夫玄石素文,底號前載,象以姓表,言以事告,《河圖》、《洛書》之徵,不是過也。加以騶虞麟趾,眾瑞並臻。昔夏、殷以丕崇為祥,周武以烏魚為美,咸曰休
 哉;然符瑞之應,備物之盛,未有若今之富者也。宜宣大典,禮中嶽,封泰山,禪梁父,發德號,明至尊,享天休,篤黎庶,勒千載之表,播流後之聲,俾百代之下,莫不興起。斯帝王之盛業,天人之至望也。」詔曰:「今逋寇雖殄,外則障塞有警,內則民黎未康,此盛德之事,所未議也。」



 瓘等又奏:「今東漸於海,西被流沙,大漠之陰,日南北戶,莫不通屬。茫茫禹跡,今實過之,則天人之道已周,巍巍之功已著。宜有事梁父,脩禮地祗,登封泰山,致誠上帝,以答人
 神之願。乞如前奏。」詔曰:「今陰陽未和,政刑未當,百姓未得其所,豈可以勒功告成邪!」瓘又奏:「臣聞處帝王之位者,必有歷運之期,天命之應;濟生民之大功者,必有盛德之容,告成之典。無不可誣,有不可讓,自古道也。而明詔謙沖,屢辭其禮。雖盛德攸在,推而未居。夫三公職典天地,實掌民物,國之大事取議於此。漢氏封禪,非是官也,不在其事。臣等前奏,蓋陳祖考之功,天命又應,陛下之德,合同四海,述古考今,宜循此禮。至於剋定歲月,須五
 府上議,然後奏聞。請寫詔及奏,如前下議。」詔曰:「雖蕩清江表,皆臨事者之勞,何足以告成。方望群后,思隆大化,以寧區夏,百姓獲乂,與之休息,此朕日夜之望。無所復下諸府矣。勿復為煩。」瓘等又奏:「臣聞唐、虞二代,濟世弘功之君,莫不仰答天心,俯協民志,登介丘,履梁父,未有辭焉者,蓋不可讓也。



 今陛下勛高百王,德無與二,茂績宏規,巍巍之業,固非臣等所能究論。而聖旨勞謙,屢自抑損,時至弗應,推美不居,闕皇代之上儀,塞
 神祇之款望,使大晉之典謨,不同風於三、五。臣等誠不敢奉詔,請如前奏施行。」詔曰:「方當共弘治道,以康庶績,且俟他年,無復紛紜也。」



 太康元年冬,王公有司又奏:「自古聖明,光宅四海,封禪名山,著於史籍,作者七十四君矣。舜、禹之有天下,巡狩四嶽,躬行其道。《易》著『觀民省方』,《禮》有『升中于天』,《詩》頌『陟其高山』,皆載在方策。文王為西伯,以服事殷;周公以魯蕃,列于諸侯,或享於岐山,或有事泰山。徒以聖德,猶得
 為其事。



 自是以來,功薄而僭其義者,不可勝言,號謚不泯,以至于今。況高祖宣皇帝肇開王業,海外有截;世宗景皇帝濟以大功,輯寧區夏;太祖文皇帝受命造晉,蕩定蜀漢;陛下應期龍興,混壹六合,澤被群生,威震無外。昔漢氏失統,吳、蜀鼎峙,兵興以來,近將百年。地險俗殊,民望絕塞,以為分外,其日久矣。大業之隆,重光四葉,不羈之寇,二世而平。非聰明神武,先天弗違,孰能巍巍其有成功若茲者歟!臣等幸以千載,得遭運會,親奉大化,
 目睹太平,至公之美,誰與為讓!宜祖述先朝,憲章古昔,勒功岱嶽,登封告成,弘禮樂之制,正三雍之典,揚名萬世,以顯祖宗。是以不勝大願,敢昧死以聞。請告太常具禮儀。」上復詔曰:「所議誠前烈之盛事也;方今未可以爾。便報絕之。」



 宋太祖在位長久,有意封禪。遣使履行泰山舊道,詔學士山謙之草封禪儀注。



 其後索虜南寇,六州荒毀,其意乃息。



 世祖大明元年十一月戊申,太宰江夏王義恭表曰:「惟皇天崇稱大道,始行揖讓。迄于有晉,雖聿修前緒,而跡淪言廢,蔑記於竹素者,焉可單書。紹乾維,建徽號,流風聲,被絲管,自無懷以來,可傳而不朽者,七十有四君。罔仁厚而道滅,鮮義澆而德宣,鐘律之先,曠世綿絕,難得而聞。《丘》、《索》著明者,尚有遺炳。故《易》稱先天弗違,後天奉時。蓋陶唐姚姒商姬之主,莫不由斯道也。是以風化大洽,光熙于後。炎漢二帝,亦踵曩則,因百姓之心,聽輿人
 之頌,龍駕帝服,鏤玉梁甫,昌言明稱,告成上靈。況大宋表祥唐虞,受終素德,山龍啟符,金玉顯瑞,異採騰於軫墟,紫煙藹於邦甸,錫冕兆九五之徵,文豹赴天歷之會。誠二祖之幽慶,聖后之冥休。道冠軒、堯,惠深亭毒;而猶執沖約,未言封禪之事,四海竊以恧焉。臣聞惟皇配極,惟帝祀天,故能上稽乾式,照臨黔首,協和穹昊,膺茲多福。高祖武皇帝明並日月,光振八區,拯已溺之晉,濟橫流之世,撥亂寧民,應天受命,鴻徽洽于海表,威棱震乎
 沙外。太祖文皇帝體聖履仁,述業興禮,正樂頌,作象歷,明達通於神祇,玄澤被乎上下。仁孝命世,睿武英挺,遭運屯否,三才湮滅,乃龍飛五洲,鳳翔九江,身先八百之期,斷出人鬼之表,慶煙應高牙之建,風耀符發迹之辰,親翦兇逆,躬清昏壒,天地革始,夫婦更造,豈與彼承業繼緒,拓復禹跡,車一其軌,書罔異文者,同年而議哉!今龍麟已至,鳳皇已儀,比李已實,靈茅已茂,雕氣降霧於宮榭,珍露呈味於禁林,嘉禾積穗於殿甍,連理合干於
 園禦,皆耀質離宮,植根蘭囿。至夫霜毫玄文,素翮赬羽,泉河山嶽之瑞,草木金石之祥,方畿憬塗之謁,抗驛絕祖之奏,彪炳雜沓,粵不可勝言。太平之應,茲焉富矣。宜其從天人之誠,遵先王之則,備萬乘,整法駕,脩封泰山,瘞玉岱趾,延喬、松於東序,詔韓、岐於西廂,麾天閽,使啟關,謁紫宮,朝太一,奏《鈞天》,詠《雲門》,贊揚幽奧,超聲前古,豈不盛哉!伏願時命宗伯,具茲典度。」詔曰:「太宰表如此。昔之盛王,永保鴻名,常為稱首,由斯道矣。朕遭家多難,
 入纂絕業,德薄勳淺,鑒寐崩愧。頃麟鳳表禎,茅禾兼瑞,雖符祥顯見,恧乎猶深,庶仰述先志,拓清中宇,禮祇謁神,朕將試哉。」



 四年四月辛亥,有司奏曰:臣聞崇號建極,必觀俗以樹教;正位居體,必採世以立言。是以重代列聖,咸由厥道。玄勳上烈,融章未分,鳴光委緒,歇而罔臧。若其顯謚略騰軌,則系綴聲采,徵略聞聽。爰洎姬、漢,風流尚存,遺芬餘榮,綿映紀緯。雖年絕世祀,代革精華,可得騰金彩,奏玉潤,鏤迹以熏今,鐫德以麗遠。而四望埋
 禋歌之禮,日觀弛脩封之容,豈非神明之業難崇,功基之迹易泯。自茲以降,訖于季末,莫不欲英弘徽位,詳固洪聲。豈徒深默脩文,淵幽馭世而已。諒以縢非虛奏,書匪妄埋,擊雨恕神,淳廕復樹,安得紫壇肅祗,竹宮載佇,散火投郊,流星奔座。寶緯初基,厭靈命歷,德振弛維,功濟淪象,玄浸紛流,華液幽潤,規存永馭,思詳樹遠。



 太祖文皇帝以啟遘泰運,景望震凝,採樂調風,集禮宣度,祖宗相映,軌跡重暉。聖上韞籙蕃河,佇翔衡漢,金波掩照,
 華耀停明,運動時來,躍飛風舉,澄氛海、岱,開景中區,歇神還靈,頹天重耀,儲正凝位於兼明,哀嶽蕃華於元列。故以祥映昌基,繫發篆素。重以班朝待典,飾令詳儀,纂綜淪蕪,搜騰委逸,奏玉郊宮,禋珪玄畤,景集天廟,脈壤祥農,節至昕陽,川丘夙禮,綱威巡袪止,表綏中甸,史流其詠,民挹其風。於是涵迹視陰,振聲威響,歷代之渠,沉□望內,安侯之長,賢王入侍,殊生詭氣,奉俗還鄉,羽族卉儀,懷音革狀,邊帛絕書,權光弛燭。天岱發靈,宗河開
 寶,崇丘淪鼎,振采泗淵,雲皇王嶽,離藻□漢,並角即音,栖翔禁禦,袞甲霜味,翾舞川肆,榮泉流鏡,後昭河源,故以波沸外關,雲蒸內澤。



 若其雪趾青毳,玄文硃彩,日月郊甸,擇木弄音,重以榮露騰軒,蕭雲掩閣,鎬潁孳萌,移華淵禁,山輿佇衡,云鶼竦翼,海鰈泳流,江茅吐廕。校書之列,仰筆以飾辭,濟、代之蕃,獻邑以待禮。豈非神勰氣昌,物瑞雲照,蒱軒龜軫,□泉淳芳。



 太宰江夏王臣義恭咀道遵英,抽奇麗古,該潤圖史,施詳閟載,表以功懋往初,
 德耀炎、昊,升文中岱,登牒天關,耀冠榮名,摛振聲號。而道謙稱首,禮以虛挹,將使玄祇缺觀,幽瑞乖期,梁甫無盛德之容,介丘靡聲聞之響。加窮泉之野,獻八代之駟,交木之鄉,奠絕金之楛,肅靈重表,珍符兼貺。伏惟陛下謨詳淵載,衍屬休章,依征聖靈,潤色聲業,諏辰稽古,肅齊警列,儒僚展采,禮官相儀,懸蕤動音,洪鐘竦節,陽路整衛,正途清禁。於是績環佩,端玉藻,鳴鳳佇律,騰駕流文,間彩比象之容,昭明紀數之服。徽焯天陣,容藻神行,
 翠蓋懷陰,羽華列照。乃詔聯事掌祭,賓客贊儀,金支宿縣,鏞石潤響。命五神以相列,闢九關以集靈,警衛兵而開雲,先雨祇以灑路。霞凝生闕,煙起成宮,臺冠丹光,壇浮素靄。爾乃臨中壇,備盛禮,天降祥錫,壽固皇根,谷動神音,山傳稱響。然後辨年問老,陳詩觀俗,歸薦告神,奉遺清廟。光美之盛,彰乎萬古;淵祥之烈,溢乎無窮。豈不盛歟!



 臣等生接昌辰,肅懋明世,束教管聞,未足言道。且章志湮微,代往淪絕,拘採遺文,辯明訓誥□
 □□□簉訪鄒、魯,草縢書堙玉之禮,具竦石繩金之儀,和芝潤瑛,鐫璽乾封。懼弗軌屬上徽,輝當王則。謹奉儀注以聞。



 詔曰:「天生神物,昔王稱愧,況在寡德,敢當鴻貺。今文軌未一,可停此奏。」



 漢獻帝建安十八年五月,以河北十郡封魏武帝為魏公。是年七月,始建宗廟於鄴,自以諸侯禮立五廟也。後雖進爵為王,無所改易。延康元年,文帝繼王位。七月,追尊皇祖為太王,丁夫人曰太王后。黃初元年十一月受
 禪,又追尊太王曰太皇帝,皇考武王曰武皇帝。明帝太和三年六月,又追尊高祖大長秋曰高皇,夫人吳氏曰高皇后,並在鄴廟廟所祠。則文帝之高祖處士、曾祖高皇、祖太皇帝共一廟。考太祖武皇帝特一廟百世不毀,然則所祠止於親廟四室也。至明帝太和三年十一月,洛京廟成,則以親盡遷處士主,置園邑,使令丞奉薦。而使行太傅太常韓暨、行太常宗正曹恪持節迎高皇以下神主共一廟,猶為四室而已。至景初元年六月,群公
 有司始更奏定七廟之制,曰:「大魏三聖相承,以成帝業。武皇帝肇建洪基,撥亂夷險,為魏太祖。文皇帝繼天革命,應期受禪,為魏高祖。上集成大命,清定華夏,興制禮樂,宜為魏烈祖。」更於太祖廟北為二祧,其左為文帝廟,號曰高祖,昭祧,其右擬明帝號曰烈祖,穆祧。三祖之廟,萬世不毀,其餘四廟,親盡迭遷,一如周后稷、文、武廟祧之禮。孫盛《魏氏春秋》曰:「夫謚以表行,廟以存容,皆於既歿然後著焉。所以原始要終,以示百世者也。未有當年
 而逆制祖宗,未終而豫自尊顯。昔華樂以厚斂致譏,周人以豫凶違禮,魏之群司,於是乎失正矣。」



 文帝甄后賜死,故不列廟。明帝即位,有司奏請追謚曰文昭皇后,使司空王朗持節奉策告祠于陵。三公又奏曰:「自古周人始祖后稷,又特立廟以祀姜嫄。今文昭皇后之於後嗣,聖德至化,豈有量哉!夫以皇家世妃之尊,神靈遷化,而無寢廟以承享祀,非以報顯德,昭孝敬也。稽之古制,宜依周禮,先妣別立寢廟。」奏可。



 以太和元年二月,立廟於
 鄴。四月,洛邑初營宗廟,掘地得玉璽方一寸九分,其文曰:「天子羨思慈親。」明帝為之改容,以太牢告廟。至景初元年十二月己未,有司又奏文昭皇后立廟京師,永傳享祀。樂舞與祖廟同,廢鄴廟。



 魏文帝黃初二年六月,以洛京宗廟未成,乃祠武帝於建始殿,親執饋奠如家人禮。何承天曰:「案禮,將營宮室,宗廟為先。庶人無廟,故祭於寢。帝者行之,非禮甚矣。」



 漢獻帝延康元年七月,魏文帝幸譙,親祠譙陵,此漢禮
 也。漢氏諸陵皆有園寢者,承秦所為也。說者以為古前廟後寢,以象人君前有朝後有寢也。廟以藏主,四時祭祀,寢有衣冠象生之具以薦新。秦始出寢起於墓側,漢因弗改。陵上稱寢殿,象生之具,古寢之意也。及魏武帝葬高陵,有司依漢,立陵上祭殿。至文帝黃初三年,乃詔曰:「先帝躬履節儉,遺詔省約。子以述父為孝,臣以系事為忠。古不墓祭,皆設於廟。高陵上殿屋皆毀壞,車馬還廄,衣服藏府,以從先帝儉德之志。」



 及文帝自作終制,又
 曰:「壽陵無立寢殿,造園邑。」自後至今,陵寢遂絕。



 孫權不立七廟,以父堅嘗為長沙太守,長沙臨湘縣立堅廟而已。權既不親祠,直是依後漢奉南頓故事,使太守祠也。堅廟又見尊曰始祖廟,而不在京師。又以民人所發吳芮塚材為屋,未之前聞也。於建鄴立兄長沙桓王策廟於硃爵橋南。權疾,太子所禱,即策廟也。權卒,子亮代立。明年正月,於宮東立權廟曰太祖廟,既不在宮南,又無昭穆之序。



 及孫皓初立,追尊父和曰文皇帝。皓
 先封烏程侯,即改葬和於烏程西山,號曰明陵,置園邑二百家。於烏程立陵寢,使縣令丞四時奉祠。寶鼎元年,遂於烏程分置吳興郡,使太守執事。有司尋又言宜立廟京邑。寶鼎二年,遂更營建,號曰清廟。



 遣守丞相孟仁、太常姚信等備官僚中軍步騎,以靈輿法駕迎神主於明陵,親引仁拜送於庭。比仁還,中吏手詔日夜相繼,奉問神靈起居動止。巫覡言見和被服顏色如平日,皓悲喜,悉召公卿尚書詣皞下受賜。靈輿當至,使丞相陸凱
 奉三牲祭於近郊。



 皓於金城外露宿。明日,望拜於東門之外,又拜廟薦饗。比七日,三祭,倡伎晝夜娛樂。有司奏:「『祭不欲數,數則黷』,宜以禮斷情。」然後止。



 劉備章武元年四月,建尊號於成都。是月,立宗廟,袷祭高祖已下。備紹世而起,亦未辨繼何帝為禰,亦無祖宗之號。劉禪面縛,北地王諶哭於昭烈之廟,此則備廟別立也。



 魏元帝咸熙元年,增封晉文帝進爵為王,追命舞陽宣
 文侯為晉宣王,忠武侯為晉景王。是年八月,文帝崩,謚曰文王。武帝泰始元年十二月丙寅,受禪。丁卯,追尊皇祖宣王為宣皇帝,伯考景王為景皇帝,考文王為文皇帝,宣王妃張氏為宣穆皇后,景王夫人羊氏為景皇后。二年正月,有司奏天子七廟,宜如禮營建。帝重其役,詔宜權立一廟。於是君臣奏議:「上古清廟一宮,尊遠神祇,逮至周室,制為七廟,以辨宗祧。聖旨深弘,遠跡上世,敦崇唐、虞。舍七廟之繁華,遵一宮之尊遠。昔舜承堯禪,受
 終文祖,遂陟帝位,蓋三十載,月正元日,又格于文祖。此則虞氏不改唐廟,因仍舊宮。可依有虞氏故事,即用魏廟。」奏可。於是追祭征西將軍、豫章府君、潁川府君、京兆府君,與宣皇帝、景皇帝、文皇帝為三昭三穆。是時宣皇未升,太祖虛位,所以祠六世與景帝為七廟,其禮則據王肅說也。七月,又詔曰:「主者前奏就魏舊廟,誠亦有準。然於祗奉神明,情猶未安。宜更營造,崇正永制。」於是改創宗廟。十一月,追尊景帝夫人夏侯氏為景懷皇后。



 太
 康元年,靈壽公主脩麗祔于太廟,周、漢未有其準。魏明帝則別立廟,晉又異魏也。六月,因廟陷當改治。群臣又議奏曰:「古者七廟異所,自宜如禮。」詔又曰:「古雖七廟,自近代以來,皆一廟七室,於禮無廢,於情為敘,亦隨時之宜也。其便仍舊。」至十年,乃更改築於宣陽門內,窮壯極麗。然坎位之制,猶如初爾。廟成,帝率百官遷神主于新廟,自征西以下,車服導從,皆如帝者之儀。摯虞之議也。至世祖武皇帝崩,則遷征西;及惠帝崩,又遷豫章。而
 惠帝世,愍懷太子、太子二子哀太孫臧、沖太孫尚並祔廟。元帝世,懷帝殤太子又被廟,號為陰室四殤。



 懷帝初,又策謚武帝楊后曰武悼皇后,改葬峻陽陵側。別立弘訓宮,不列於廟。元帝既即尊位,上繼武帝,於禮為禰,如漢光武上繼元帝故事也。是時西京神主堙滅虜庭,江左建廟,皆更新造。尋以登懷帝之主,又遷潁川。位雖七室,其實五世,蓋從刁協,以兄弟為世數故也。于時百度草創,舊禮未備,三祖毀主,權居別室。



 太興三年,將登愍
 帝之主,於是乃定更制,還復豫章、潁川二主于昭穆之位,以同惠帝嗣武帝故事;而惠、懷、愍三帝自從《春秋》尊尊之義,在廟不替也。至元帝崩,則豫章復遷。然元帝神位,猶在愍帝之下,故有坎室者十也。至明帝崩,而潁川又遷,猶十室也。于時續廣太廟,故三遷主並還西儲,名之曰祧,以準遠廟。



 成帝咸和三年,蘇峻覆亂京都,溫嶠等入伐,立行廟於白石,告先帝先后曰:「逆臣蘇峻,傾覆社稷,毀棄三正,汙辱海內。臣亮等手刃戎首,龔行天罰。
 惟中宗元皇帝、肅祖明皇帝、明穆皇后之靈,降鑒有罪,剿絕其命,翦此群兇,以安宗廟。臣等雖隕首摧軀,猶生之年。」咸康七年五月,始作武悼皇后神主,祔于廟,配饗世祖。成帝崩而康帝承統,以兄弟一世,故不遷京兆,始十一室也。康帝崩,京兆遷入西儲,同謂之祧,如前三祖遷主之禮。故正室猶十一也。穆帝崩而哀帝、海西並為兄弟,無所登降。咸安之初,簡文皇帝上繼元皇帝,世秩登進。於是潁川、京兆二主,復還昭穆之位。至簡文崩,潁
 川又遷。孝武皇帝太元十六年,改作太廟,殿正室十六間,東西儲各一間,合十八間。棟高八丈四尺,堂基長三十九丈一尺,廣十丈一尺。堂集方石,庭以磚。尊備法駕,遷神主于行廟。征西至京兆四主,及太子太孫,各用其位之儀服。四主不從帝者之儀,是與太康異也。諸主既入廟,設脯醢之奠。及新廟成,帝主還室,又設脯醢之奠。十九年二月,追尊簡文母會稽太妃鄭氏為簡文皇帝宣太后,立廟太廟道西。及孝武崩,京兆又遷,如穆帝之
 世四祧故事。安帝隆安四年,以孝武母簡文李太后、帝母宣德陳太后祔于宣鄭太后之廟。



 元興三年三月,宗廟神主在尋陽,已立新主于太廟,權告義事。四月,輔國將軍何無忌奉送神主還。丙子,百官拜迎于石頭。戊寅,入廟。安帝崩,未及禘,而天祿終焉。



 宋武帝初受晉命為宋王,建宗廟於彭城,依魏、晉故事,立一廟。初祠高祖開封府君、曾祖武原府君、皇祖東安府君、皇考處士府君、武敬臧後,從諸侯五廟之禮也。既
 即尊位,及增祠七世右北平府君、六世相國掾府君為七廟。永初初,追尊皇考處士為孝穆皇帝,皇妣趙氏為穆皇后。三年,孝懿蕭皇后崩,又祔廟。高祖崩,神主升廟,猶從昭穆之序,如魏、晉之制,虛太祖之位也。廟殿亦不改構,又如晉初之因魏也。文帝元嘉初,追尊所生胡婕妤為章皇太后,立廟西晉宣太后地。孝武昭太后、明帝宣太后並祔章太后廟。



 晉元帝太興三年正月乙卯,詔曰:「吾雖上繼世祖,然於
 懷、愍皇帝,皆北面稱臣。今祠太廟,不親執觴酌,而令有司行事,於情禮不安。可依禮更處。」太常華恆議:「今聖上繼武皇帝,宜準漢世祖故事,不親執觴爵。」又曰:「今上承繼武帝,而廟之昭穆,四世而已。前太常賀循、博士傅純以為惠、懷及愍宜別立廟。



 然臣愚謂廟室當以客主為限,無拘常數。殷世有二祖三宗,若拘七室,則當祭禰而已。推此論之,宜還復豫章、潁川,全祠七廟之禮。」驃騎長史溫嶠議:「凡言兄弟不相入廟,既非禮文。且光武奮劍
 振起,不策名於孝平,豫神其事,以應九世之讖;又古不共廟,故別立焉。今上以策名而言,殊於光武之事,躬奉烝嘗,於經既正,於情又安矣。太堂恆欲還二府君以全七世,嶠謂是宜。」驃騎將軍王導從嶠議。



 嶠又曰:「其非子者,可直言皇帝敢告某皇帝。又若以一帝為一世,則不祭禰,反不及庶人。」於是帝從嶠議,悉施用之。孫盛《晉春秋》曰:「《陽秋傳》云,『臣子一例也』。雖繼君位,不以後尊,降廢前敬。昔魯僖上嗣莊公,以友于長幼而升之,為逆。準之
 古義,明詔是也。」



 穆帝永和二年七月,有司奏:「十月殷祭,京兆府君當遷祧室。昔征西、豫章、潁川三府君毀主,中興之初,權居天府,在廟門之西。咸康中,太常馮懷表續奉還於西儲夾室,謂之為祧,疑亦非禮。今京兆遷入,是為四世遠祖,長在太祖之上。



 昔周室太祖世遠,故遷有所歸。今晉廟宣皇為主,而四祖居之,是屈祖就孫也。殷袷在上,是代太祖也。」領司徒蔡謨議:「四府君宜改築別室,若未展者,當
 入就太廟之室。人莫敢卑其祖,文、武不先不窋。殷祭之日,征西東面,處宣皇之上。



 其後遷廟之主,藏於征西之祧,祭薦不絕。」護軍將軍馮懷表議:「《禮》,『無廟者,為壇以祭』。可別立室藏之,至殷禘,則祭于壇也。」輔國將軍譙王司馬無忌等議:「諸儒謂太王王季遷主藏於文、武之祧,如此,府君遷主,宜在宣皇帝廟中。然今無寢室,宜變通而改築。又殷袷太廟,征西東面。」尚書郎孫綽與無忌議同,曰:「太祖雖位始九五,而道以從暢,贊人爵之尊,篤天倫
 之道,所以成教本而光百代也。」尚書郎徐禪議:「《禮》,『去祧為壇,去壇為鸑,歲袷則祭之』。



 今四祖遷主,可藏之石室。有禱則祭於壇鸑。」又遣禪至會稽訪處士虞喜。喜答曰:「漢世韋玄成等以毀主瘞於園。魏朝議者云應埋兩階之間。且神主本在太廟,若今別室而祭,則不如永藏。又四君無追號之禮,益明應毀而無祭。」於是撫軍將軍會稽王司馬昱、尚書劉劭等奏:「四祖同居西祧,藏主石室,禘袷乃祭,如先朝舊儀。」



 時陳留范宣兄子問此禮。宣答
 曰;「舜廟所祭,皆是庶人。其後世遠而毀,不居舜上,不序昭穆。今四君號猶依本,非以功德致禮也。若依虞主之瘞,則猶藏子孫之所;若依夏主之埋,則又非本廟之階。宜思其變,別築一室,親未盡則禘袷,處宣帝之上;親盡則無緣下就子孫之列。」其後太常劉遐等同蔡謨議。博士張憑議:「或疑陳於太祖者,皆其後毀之主。憑案古義,無別前後之文也。禹不先鯀,則遷主居太祖之上,亦可無疑矣。」



 安帝義熙九年四月,將殷祭,詔博議遷毀之禮。大
 司馬琅邪王司馬德文議:「泰始之初,虛太祖之位,而緣情流遠,上及征西,故世盡則宜毀,而宣皇帝正太祖之位。又漢光武帝移十一帝主於洛邑,則毀主不沒,理可推矣。宜從范宣之言,築別室以居四府君之主,永藏而不祀也。」大司農徐廣議:「四府君嘗處廟室之首,歆率土之祭。若埋之幽壤,於情理未必咸盡。謂可遷藏西儲,以為遠祧,而禘饗永絕也。」太尉諮議參軍袁豹議:「仍舊無革。殷祠猶及四府君,情理為允。」祠部郎臧燾議:「四府君
 之主,享祀禮廢,則亦神所不依。宜同虞主之瘞埋矣。」時高祖輔晉,與大司馬議同。須後殷祀行事改制。



 晉孝武帝太元十二年五月壬戌,詔曰:「昔建太廟,每事從儉約,思與率土,致力備禮。又太祖虛位,明堂未建。郊祀,國之大事,而稽古之制闕然。便可詳議。」



 祠部郎徐邈議:「圓丘郊祀,經典無二,宣皇帝嘗辨斯義。而檢以聖典,爰及中興,備加研極,以定南北二郊,誠非異學所可輕改也。謂仍舊為安。武皇帝建廟,六世三昭三穆,宣皇帝
 創基之主,實惟太祖,親則王考,四廟在上,未及遷世,故權虛東向之位也。兄弟相及,義非二世,故當今廟祀,世數未足,而欲太祖正位,則違事七之義矣。又《禮》曰『庶子王亦禘祖立廟』。蓋謂支胤授位,則親近必復。京兆府君於今六世,宜復立此室,則宣皇未在六世之上,須前世既遷,乃太祖位定爾。



 京兆遷毀,宜藏主於石室。雖禘袷猶弗及。何者?傳稱毀主升合乎太祖,升者自下之名,不謂可降尊就卑也。太子太孫陰室四主,儲嗣之重,升祔
 皇祖所配之廟,世遠應遷,然後從食之孫,與之俱毀。明堂圓方之制,綱領已舉,不宜闕配帝之祀。



 且王者以天下為家,未必一邦,故周平、光武無廢於二京也。周公宗祀文王,漢明配以世祖,自非惟新之考,孰配上帝。」邈又曰:「明堂所配之神,積疑莫辨。按《易》,『殷薦上帝,以配祖考』。祖考同配,則上帝亦為天,而嚴父之義顯。



 《周禮》,旅上帝者有故,告天與郊祀常禮同用四圭,故並言之。若上帝者五帝,經文何不言祀天旅五帝,祀地旅四望乎?人帝
 之與天帝,雖天人之通謂,然五方不可言上帝,諸侯不可言大君也。書無全證,而義容彼此,故太始、太康二紀之間,興廢迭用矣。」侍中車胤議同。又曰:「明堂之制,既其難詳。且樂主於和,禮主於敬,故質文不同,音器亦殊。既茅茨廣廈,不一其度,何必守其形範,而不知弘本順民乎!九服咸寧,河朔無塵,然後明堂辟雍,可崇而修之。」中書令王氏意與胤同。太常孔汪議:「太始開元,所以上祭四府君,誠以世數尚近,可得饗祠,非若殷、周先世,王迹所
 因也。向使京兆爾時在七世之外,自當不祭此四王。推此知既毀之後,則殷禘所絕矣。」吏部郎王忱議:「明堂則天象地,儀觀之大,宜俟皇居反舊,然後脩之。」驃騎將軍會稽王司馬道子、尚書令謝石意同忱議,於是奉行,一無所改。



 晉安帝義熙二年六月,白衣領尚書左僕射孔安國啟云:「元興三年夏,應殷祠。



 昔年三月,皇輿旋軫。其年四月,便應殷,而太常博士徐乾等議云:『應用孟秋。』臺尋校自
 太和四年相承皆用冬夏,乾等既伏應孟冬,回復追明孟秋非失。御史中丞範泰議:『今雖既祔之後,得以烝嘗,而無殷薦之比。太元二十一年十月應殷,烈宗以其年九月崩。至隆安三年,國家大吉,乃修殷事。又禮有喪則廢吉祭,祭新主於寢。今不設別寢,既祔,祭於廟。故四時烝嘗,以寄追遠之思,三年一禘,以習昭穆之序,義本各異。三年喪畢,則合食太祖,遇時則殷,無取於限三十月也。當是內臺常以限月成舊。』就如所言,有喪可殷。隆安
 之初,果以喪而廢矣。月數少多,復遲速失中。至於應寢而脩,意所未譬。」安國又啟:「范泰云:『今既祔,遂祭於廟,故四時烝嘗。』如泰此言,殷與烝嘗,其本不同。既祔之後,可親烝嘗而不得親殷也。太常劉瑾云:『章后喪未一周,不應祭。』臣尋升平五年五月,穆皇帝崩,其年七月,山陵,十月,殷。興寧三年二月,哀皇帝崩,太和元年五月,海西夫人庾氏薨,時為皇后,七月,葬,十月,殷。此在哀皇再周之內,庾夫人既葬之後,二殷策文見在廟。又文皇太后以
 隆安四年七月崩,陛下追述先旨,躬服重制,五年十月,殷。再周之內,不以廢事。今以小君之哀,而泰更謂不得行大禮。



 臣尋永和十年至今五十餘載,用三十月輒殷,皆見於注記,是依禮,五年再殷。而泰所言,非真難臣,乃以聖朝所用,遲速失中。泰為憲司,自應明審是非,群臣所啟不允,即當責失奏彈,而愆墮稽停,遂非忘舊。請免泰、瑾官。」丁巳,詔皆白衣領職。於是博士徐乾皆免官。



 初,元興三年四月,不得殷祠進用十月,計常限,則義熙三
 年冬又當殷;若更起端,則應用來年四月。領司徒王謐、丹陽尹孟昶議:「有非常之慶,必有非常之禮。殷祭舊準不差,蓋施於經常爾。至於義熙之慶,經古莫二,雖曰旋幸,理同受命。愚謂理運惟新,於是乎始。宜用四月。」中領軍謝混、太常劉瑾議:「殷無定日,考時致敬,且禮意尚簡。去年十月祠,雖於日有差,而情典允備,宜仍以為正。」



 太學博士徐乾議:「三年一袷,五年一禘,經傳記籍,不見補殷之文。」員外散騎侍郎領著作郎徐廣議:「尋先事,海西
 公泰和六年十月,殷祠。孝武皇帝寧康二年十月,殷祠。若依常去前三十月,則應用四月也。于時蓋當有故,而遷在冬,但未詳其事。太元元年十月殷祠,依常三十月,則應用二年四月也。是追計辛未歲十月,來合六十月而再殷。何邵甫注《公羊傳》云,袷從先君來,積數為限。『自僖八年至文二年,知為袷祭』。如此,履端居始,承源成流,領會之節,遠因宗本也。昔年有故推遷,非其常度。寧康、太元前事可依。雖年有曠近之異,然追計之理同矣。



 愚
 謂從復常次者,以推歸正之道也。」左丞劉潤之等議:「太元元年四月應殷,而禮官墮失,建用十月。本非正期,不應即以失為始也。宜以反初四月為始。當用三年十月。」尚書奏從王謐議,以元年十月為始也。



 宋孝武帝孝建元年十二月戊子,有司奏:「依舊今元年十月是殷祠之月,領曹郎范泰參議,依永初三年例,須再周之外殷祭。尋祭再周來二年三月,若以四月殷,則猶在禫內。」下禮官議正。國子助教蘇瑋生議:「案《禮》,三年
 喪畢,然後袷於太祖。又云『三年不祭,唯天地社稷,越紼行事』。且不禫即祭,見譏《春秋》。



 求之古禮,喪服未終,固無祼享之義。自漢文以來,一從權制,宗廟朝聘,莫不皆吉。雖祥禫空存,無FH縞之變,烝嘗薦祀,不異平日。殷祠禮既弗殊,豈獨以心憂為礙。」太學博士徐宏議:「三年之喪,雖從權制,再周祥變,猶服縞素,未為純吉,無容以祭。謂來四月,未宜便殷,十月則允。」太常丞臣朱膺之議:「《虞禮》云:『中月而禫,是月也吉祭,猶未配。』謂二十七月既禫祭,
 當四時之祭日,則未以其妃配,哀未忘也。推此而言,未示覃不得祭也。又《春秋》閔公二年,吉禘于莊公。鄭玄云:『閔公心懼於難,務自尊成以厭其禍,凡二十二月而除喪,又不禫。』云又不禫,明禫內不得禘也。案王肅等言於魏朝云,今權宜存古禮,俟畢三年。舊說三年喪畢,遇禘則禘,遇袷則袷。鄭玄云:『禘以孟夏,祫以孟秋。』今相承用十月。如宏所上《公羊》之文,如為有疑,亦以魯閔設服,因言喪之紀制爾。何必全許素冠可吉禘。縱《公羊》異說,官以
 禮為正,亦求量宜。」郎中周景遠參議:「永初三年九月十日奏傅亮議:『權制即吉,御世宜爾。宗廟大禮,宜依古典。』則是皇宋開代成準。謂博士徐宏、太常丞朱膺之議用來年十月殷祠為允。」



 詔可。



 宋殷祭皆即吉乃行。大明七年二月辛亥,有司奏:「四月應殷祠,若事中未得為,得用孟秋與不?」領軍長史周景遠議:「案《禮記》云:『天子祫禘祫嘗祫烝。』依如禮文,則夏秋冬三時皆殷,不唯用冬夏也。晉義熙初,僕射孔安國啟
 議,自泰和四年相承殷祭,皆用冬夏。安國又啟,永和十年至今五十餘年,用三十月輒殷祠。



 博士徐乾據《禮》難安國。乾又引晉咸康六年七月殷祠,是不專用冬夏。於時晉朝雖不從乾議,然乾據禮及咸康故事,安國無以奪之。今若以來四月未得殷祠,遷用孟秋,於禮無違。參議據禮有證,謂用孟秋為允。」詔可。



 晉武帝咸寧五年十一月己酉,弘訓羊太后崩,宗廟廢一時之祀,天地明堂去樂,且不上胙。升平五年十月己
 卯,殷祠,以穆帝崩後,不作樂。初,永嘉中,散騎常侍江統議曰:「《陽秋》之義,去樂卒事。」是為吉祭有廢樂也,故升平末行之。



 其後太常江逌表:「穆帝山陵之後十月殷祭,從太常丘夷等議,撤樂。逌尋詳今行漢制,無特祀之別。既入廟吉禘,何疑於樂。」



 史臣曰:聞樂不怡,故申情於遏密。至於諒闇奪服,慮政事之荒廢,是以乘權通以設變,量輕重而降屈。若夫奏音之與寢聲,非有損益於機務,縱復回疑於兩端,固宜
 緣恩而從戚矣。宋世國有故,廟祠皆懸而不樂。



\end{pinyinscope}