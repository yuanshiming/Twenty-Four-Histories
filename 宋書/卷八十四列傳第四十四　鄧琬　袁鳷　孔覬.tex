\article{卷八十四列傳第四十四 鄧琬 袁鳷 孔覬}

\begin{pinyinscope}

 鄧琬,字元琬,豫章南
 昌人
 也。高祖混,曾祖玄,並為晉尚書吏部郎。祖潛之,鎮南長史。父胤之,世祖征虜長史,吏部郎,彭城王義康大將軍長史、豫章太守,光祿勳。琬初
 為州西曹主簿,南譙王義宣征北行參軍,轉參軍事,又隨府轉車騎參軍,仍轉府主簿,江州治中從事史。世祖起義,版琬為輔國將軍、南海太守,率軍伐蕭簡於廣州,攻圍踰年,乃克。以臧質反,為江州刺史宗愨所執,值赦原。琬弟璩,與臧質同逆,質敗從誅;琬弟環亦坐誅。琬在遠,又有功,免死遠徙,仍停廣州。久之,得還,除給事中,尚書庫部郎,都水使者,丹陽丞,本州大中正。大明七年,車駕幸歷陽,追思在籓之舊,下詔曰:「故光祿勳、前征虜長
 史鄧胤之體局沈隱,累任著績。朕昔當籓重,首先佐務,心力款盡,弗忘于懷。往歲息璩凶悖,自取誅翦,沿恩及琬,特免釁戮。今可擢為給事黃門侍郎,以旌胤之宿誠。」



 明年,出為晉安王子勛鎮軍長史、尋陽內史,行江州事。前廢帝狂悖無道,以太祖、世祖並第數居三以登極位,子勛次第既同,深構嫌隙,因何邁之謀,乃遣使齎藥賜子勛死。使至,子勛典簽謝道遇、齋帥潘欣之、侍書褚靈嗣等馳以告琬,泣涕請計。琬曰:「身南土寒士,蒙先殊
 恩,以愛子見託,豈得惜門戶百口,其當以死報效。幼主昏暴,社稷危殆,雖曰天子,事猶獨夫。今便指率文武,直造京邑,與群公卿士,廢昏立明。」景和元年十一月十九日,稱子勛教,即日戒嚴。子勛戎服出聽事,集僚佐,使潘欣之口宣旨曰:「少主昏狂悖戾,並是諸君所見聞。顧命重臣,悉皆誅戮。驅逼王公,幽辱太后。不逞之徒,共成其釁。京師諸王,並見囚逼,委厄虎口,思奮莫因。身義兼家國,豈可坐視橫流!今便欲舉九江之眾,馳檄近遠,以謀
 王室。於諸君何如?」四座未答,錄事參軍陶亮曰:「少主昏狂,醜毒已積。伊、霍行之於古,殿下當之於今。鄙州士子,世習忠節,況屬千載之會,請效死前驅。」眾並奉旨。文武普進位一階。轉亮為諮議參軍事,領中兵,加寧朔將軍,總統軍事。功曹張沈為諮議參軍,統作舟艦。參軍事顧昭之、沈伯玉、荀道林等參管書記。南陽太守沈懷寶、岷山太守薛常寶之郡,始至尋陽,與新蔡太守韋希直並為諮議參軍,領中兵,及彭澤令陳紹宗並為將帥。



 初,廢
 帝使荊州錄送前軍長史、荊州行事張悅下至盆口,琬稱子勛命,釋其桎梏,迎以所乘之車,以為司馬,加征虜將軍。加琬冠軍將軍,二人共掌內外眾事。



 遣將軍俞伯奇率五百人出斷大雷,禁絕商旅,及公私使命。遣使上諸郡民丁,收斂器械。十日之內,得甲士五千人,出頓大雷,於兩岸築壘。巴東、建平二郡太守孫沖之之郡,始至孤石,琬以沖之為子勳諮議參軍,領中兵,加輔國將軍,與陶亮並統前軍。使記室參軍荀道林造檄文,馳告遠
 近。



 會太宗定亂,進子勛號車騎將軍、開府儀同三司。令書至,諸佐吏並喜,造琬曰:「暴亂既除,殿下又開黃閣,實為公私大慶。」琬以子勛次第居三,又以尋陽起事,有符世祖,理必萬克。乃取令書投地曰:「殿下當開端門,黃閣是吾徒事耳!」



 眾並駭愕。琬與陶亮等繕治器甲,徵兵四方。郢州刺史安陸王子綏、荊州刺史臨海王子頊、會稽太守尋陽王子房、雍州刺史袁顗、梁州刺史柳元怙、益州刺史蕭惠開、廣州刺史袁曇遠、徐州刺史薛安都、青
 州刺史沈文秀、冀州刺史崔道固、湘州行事何慧文、吳郡太守顧琛、吳興太守王曇生、晉陵太守袁標、義興太守劉延熙並同叛逆。



 先是,廢帝以邵陵王子元為冠軍將軍、湘州刺史,中兵參軍沈仲玉為道路行事。



 至鵲頭,聞尋陽兵起,停住,白太宗進止之宜。太宗以子勛起兵,本在幼主,雖疑其不即解甲,不欲先彰同異,敕令進道。信未報,琬聞子元停鵲頭不進,遣數百人劫迎之。乃建牙於桑尾,傳檄京師曰:陽六數艱,雲雷相襲。高皇受歷,
 時乘雲轡,頓於促路。文祖定祥,係昭睿化,翦於中年。二凶縱禍,三綱理滅,宗王俯首,姑息逆朝,枕戈無聞,偷榮有秩。孝武皇帝釋位泣血,糾義入討,投袂戎首,親戮鯨鯢,九服還輝,兩儀更造。而穹旻不惠,棄離萬國,皇運重替,嗣王荒淫。孤以不才,任居籓長,大懼宗稷,殲覆待日。故招徒楚郢,飛檄京甸,志遵前典,黜幽陟明,庶七廟復安,海昏有紹。豈圖宋未悔禍,弒亂奄臻,遂矯害明茂,篡竊天寶,反道效尤,蔑我皇德,干我昭穆,寡我兄弟,恣鴟
 鴞之心,蹈倫、穎之志,覆移鼎祚,誣罔天人。藐孤同氣,猶有十三,聖靈何辜,而當乏饗。



 昔隆周弛御,晉、鄭是依;盛漢中陵,居、章抗節。支苗輕屬,猶或忘驅,況孤忝惟臣子,情地兼切,號感一隅,心與事痛。是用飲血衽金,誓復宗祀。今遣輔國將軍諮議領中直兵孫沖之、龍驤將軍陳紹宗,率螭虎之士,組甲二萬,沿流電發,徑取白下。龍驤將軍領中直兵薛常寶、建威將軍領中直兵沈懷寶,長戟萬刃,羽騎千群,徑出南州,直造朱雀。寧朔將軍諮議
 領中直兵陶亮、龍驤將軍焦度,總中黃之旅,梟雄三萬,風掩江介,雲臨石頭。建威將軍張冽,龍驤將軍何休明,提育、獲之徒,勁悍之卒,邪趨金陵,北指閶闔。龍驤將軍張係伯、龍驤將軍陳慶,勒輕銳五千,彊弩一萬,飛鋒班瀆,齊會西明。冠軍將軍、尋陽內史鄧琬,撮湘、雍之兵,勇敢四萬,授律總威,飆集京邑。征虜將軍領府司馬張悅,蒼兕千艘,水軍五萬,大董群校,絡繹繼道。冠軍將軍豫章內史劉衍、寧朔將軍武昌太守劉弼、寧朔將軍西陽
 太守謝稚、建威將軍領中直兵晉熙太守閻湛之,皆掃境勝兵,薦誠請效。



 後將軍、郢州刺史安陸王子綏懷恩纏慕,鞠旅先辰。冠軍將軍、湘州刺史邵陵王子元席颿陵波,整眾遄至。前將軍、荊州刺史臨海王子頊練甲陜西,獻徒萬數。輔國將軍、冠軍長史、長沙內史何惠文,見拔先皇,誠深投袂。冠軍將軍、雍州刺史袁顗,不謀同契,雷發漢南。建武將軍、順陽太守劉道憲,懷忠抱慨,不遠三千。梁、益、青、徐、兗、豫、吳、會,皆密介歸誠,誓為表裏。孤親
 總烝徒,十有餘萬,白羽咽川,霜鋒照野,金聲振谷,鳴鼙聒天。凡諸將帥,皆忠無匿情,智無遺計,果乾剛鷙,譎略多奇。水陸長驅,數道並進,發舟踰險,背水爭先。以此眾戰,孰能斯禦,推此義銳,滄海可垔。諸君或荷寵前朝,感恩舊日;或弈世貞淳,見危授命。而逼迫寇手,效節莫由。今大軍密邇,形援已接,見幾而作,豈俟終日!便宜轉禍趣福,因變立功。夫旦、奭與三監並時,金、霍與上官共主,邪正粈雜,何世無之!但績亮則名播,姦騁則道消耳。紀
 季入齊,陳平歸漢,身尊譽遠,明誓是裒,成範全規,殷監匪遠。若玩咎惟休,告舍罔悟,則誅及五族,有殄無遺。軍科爵賞,信如皦日,巫山既燎,芝艾共煙,幸遵良塗,無守毀轍。檄到宣告,咸使聞知。



 購太宗萬戶侯,布絹二萬匹,金銀五百斤,其餘各有差。太宗遣荊州典簽邵宰乘驛還江陵,經過襄陽,袁顗馳書報琬,勸勿解甲,並奉表勸子勛即位。郢州承子勛初檄,及聞太宗定大事,即解甲下標。繼聞尋陽不息,而鳷又響應,郢府行事錄事參軍
 荀卞之大懼,慮為琬所咎責,即遣諮議領中兵參軍鄭景玄率軍馳下,并送軍糧。琬乃稱說符瑞,造乘輿御服,云松滋縣生豹自來,柴桑縣送竹有「來奉天子」



 字,又云青龍見東淮,白鹿出西岡。令顧昭之撰為《瑞命記》。立宗廟,設壇場,矯作崇憲太后璽,令群僚上偽號於子勛。泰始二年正月七日,即位於尋陽城,改景和二年為義嘉元年。以安陸王子綏為司徒、驃騎將軍、揚州刺史,尋陽王子房車騎將軍,臨海王子頊衛將軍,並開府儀同三
 司,邵陵王子元撫軍將軍。其日雲雨晦合,行禮忘稱萬歲。取子勛所乘車,除腳以為輦,置偽殿之西。其夕,有鳩棲其中,鴞鳥集其憲;又有禿鶖集城上。子綏拜司徒日,雷電晦冥,震其黃閣柱,鴟尾墮地;又有鴟棲其帳上。以鄧琬為左將軍、尚書右僕身,張悅領軍將軍、吏部尚書,征虜將軍如故;進袁顗號安北將軍,加尚書左僕射。臨川內史張淹為侍中。府主簿顧昭之、武昌太守劉弼並為黃門侍郎。廬江太守王子仲委郡奔尋陽,亦為黃門侍
 郎。鄱陽內史丘景先、廬陵內史殷損、西陽太守謝稚、後軍府記室參軍孫詵、長沙內史孔靈產、參軍事沈伯玉、荀道林並為中書侍郎。荀卞之為尚書左丞,府主簿江乂為右丞,府主簿蕭寶欣為通直郎。琬大息粹、悅息洵並正員郎,粹領衛尉,洵弟洌司徒主簿。建武將軍、領軍主、晉熙太守閻湛之加寧朔將軍。廬陵內史王僧胤為秘書丞。



 桂陽太守劉卷為尚書殿中郎。褚靈嗣、潘欣之、沈光祖,中書通事舍人。餘諸州郡,並加爵號。



 琬性鄙暗,
 貪吝過甚,財貨酒食,皆身自量校。至是父子並賣官鬻爵,使婢僕出市道販賣,酣歌博奕,日夜不休。大自矜遇,賓客到門者,歷旬不得前。內事悉委褚靈嗣等三人,群小橫恣,競為威福,士庶忿怨,內外離心矣。



 太宗遣散騎常侍、領軍將軍王玄謨領水軍南討,吳興太守張永為其後繼;又遣寧朔將軍尋陽內史沈攸之、寧朔將軍江方興、龍驤將軍劉靈遺率眾屯虎檻。時東賊甚急,張永、江方興回軍東討。尚書下符曰:夫晦明遞運,崇替相沿,
 帝宋之基,懋業維永,聖祖重光,氤氳上業。狂昏承祀,國維以紊,毒流九縣,釁穢三靈,搢紳戮辱,黔庶塗炭,人神同憤,朝野泣血。



 聖上明睿在躬,膺符握曜,眷懷家國,夙夜劬勞,懼社稷湮蕪,彞倫左衽。天威雷發,氛沴冰消,殄凶譙門,不俟鳴條之旅;殲虐牧野,無勞孟津之鉞。華、夷即晏,晷緯還光,鏗鏘聞於管絃,趨翔被於冠冕,同軌仰化,異域懷風。劉子勛昏世稱兵,義同翦惡,明朝不戟,罔識邪正。窺窬畿甸,逼遏兩江,陵上無君,暴於遐邇。王赫
 斯怒,興言討違,命彼上將,治兵薄伐。



 今遣寧朔將軍、尋陽內史沈攸之,輕銳七千,飛舟先邁。龍驤將軍劉靈遺,羽林虎旅,連鋒繼造。假節、督南討前鋒諸軍事、冠軍將軍、兗州刺史殷孝祖,驅濟、河勁卒,電擊雷動。使持節、車騎將軍、江州刺史曲江縣開國侯王玄謨,烝徒五萬,董統前師。使持節、侍中、司徒、揚州刺史建安王休仁,擁神州之眾,總督群帥。



 龍驤將軍劉勔、寧朔將軍劉懷珍,步騎五千,直指大雷。寧朔將軍柳倫、司州刺史龐孟虯,淮、
 潁突騎,邪趣西陽。使持節、驃騎大將軍、豫州刺史山陽王休祐,總勒步師,連旗百萬,河舟代馬,遄鶩江水賁,越棘吳鉤,交曜畿服,笳鼓動坤維,金甲震雲漢,掎角相望,水陸俱發。冠軍將軍武念,率雍、司之銳,已據樊、沔。



 徐州刺史申令孫,提彭、宋剽勇,陸塗焱奮。皇上當親馭六師,降臨江服,旌旆掩雲,舳艫咽海。



 昔吳、楚連衡,燕、淮勁悍,塵擾區內,聲沸秦中,霧散埃滅,豈非先鑒。而嬰彼孤城,以待該天之網,迫此烏合,以抗絡宇之師。雲羅四掩,霜鋒
 交集,猶勁飆之拂細草,烈火之掃寒原,燋卷之形,昭然已著。朝廷惻愍我僚吏,哀矜我士民,並亦何辜,拘誤迷黨。故加宣示,令得自新。如其淪惑不改,抵冒王威,同焚既至,雖悔奚補。奉詔以四王幼弱,不幸陷難,兵交之日,不得妄加侵犯,若有逼損,誅翦無貸。左右主帥,嚴相衛奉,詿誤之罪,一無所問。



 琬遣孫沖之率陳紹宗、胡靈秀、薛常寶、張繼伯、焦度等前鋒一萬,來據赭圻。



 沖之於道與子勛書曰:「舟楫已辦,器械亦整,三軍踴躍,人爭效命,
 便欲沿流掛颿,直取白下。願速遣陶亮眾軍,兼行相接,分據新亭、南州,則一麾定矣。」乃加沖之左衛將軍,以陶亮為右衛將軍,統諸州兵俱下。郢州軍主鄭景玄、荊州軍主劉亮、湘州軍主何昌、梁州軍主柳登、雍州軍主宗庶等合二萬人,一時俱下。亮本無幹略,聞建安王休仁自上,殷孝祖又至,不敢進,屯軍鵲洲。



 時琬遣閻湛之來寇廬江,臺軍主、龍驤將軍段佛榮受命討之。更使佛榮領鐵騎一千,回軍南討。三月三日,水陸攻赭圻,亮等率
 眾來救,殷孝祖為流矢所中死,軍主朱輔之、申謙之、張靈符並失利,輔之副正員將軍皇甫仲遠、謙之副虎賁中郎將徐稚賓並沒。孝祖支軍主範潛率五百人投亮。時東軍已捷,江方興復還虎檻,建安王休仁遣方興、劉靈遺各領三千人助赭圻,以方興領孝祖軍,沈攸之代孝祖為前鋒都督。沖之謂陶亮曰:「孝祖梟將,一戰便死。天下事定矣,不須復戰,便當直取京都。」亮不從。太宗遣員外散騎侍郎王道隆至赭圻督戰。孝祖死之明日,建
 安王休仁又遣軍主郭季之馬步三千就攸之,攸之乃率季之及輔國將軍步兵校尉杜幼文、寧朔將軍屯騎校尉垣恭祖、龍驤將軍朱輔之、員外散騎侍郎高遵世、馬軍主龍驤將軍頓生、段佛榮等三萬人,詰旦進戰,奮擊,大破之,斬獲數千,追奔至姥山而反。



 沖之等於湖、白口築二城,為軍主張興世所拔。陶亮聞湖、白二城陷沒,大懼,急呼沖之還鵲尾,留薛常寶代沖之守赭圻。先於姥山及諸岡分立營寨,亦悉敗還,共保濃湖。濃湖即在
 鵲尾。



 時軍旅大起,國用不足,募民上米二百斛,錢五萬,雜穀五百斛,同賜荒縣除。



 上米三百斛,錢八萬,雜穀千斛,同賜四品正令史;滿報,若欲署四品在家,亦聽。



 上米四百斛,錢十二萬,雜穀一千三百斛,同賜四品正令史;滿報,若欲署三品在家,亦聽。上米五百斛,錢十五萬,雜穀一千五百斛,同賜三品令史;滿報,若欲署內監在家,亦聽。上米七百斛,錢二十萬,雜穀二千斛,同賜荒郡除;若欲署諸王國三令在家,亦聽。



 琬又遣輔國將軍、豫州刺
 史劉胡率眾三萬,鐵騎二千,來屯鵲尾。胡宿將,屢有戰功,素多狡詐,為眾推伏,攸之等甚憚之。時胡鄉人蔡那、佼長生、張敬兒各領軍隸攸之在赭圻,胡以書招之,那等並拒絕。胡因要那等共語,陳說平生,那等詰誚,說令歸順。胡回軍入鵲尾,無他權略。輔國將軍吳喜平定三吳,率所領五千人,并運資實,至于赭圻,於戰鳥山築壘,分遣千人,乘輕舸二百,與佼長生為游軍。



 薛常寶糧盡,告胡求援。三月二十九日,胡率步卒一萬,夜斫山開道,
 以布囊運米,來餉赭圻。平旦至城下,猶隔小塹,未能得入。沈攸之率眾軍攻之,軍主郭季之、荀僧韶、幢主韓欣宗等,率眾三千,為攸之勢援。胡發所由橋道,僧韶等接盾行戰,復橋得渡。軍主劉沙彌輕騎深入,至胡麾下,遂見殺。攸之策馬陷陳,回還,為追騎所刺;馬軍主段佛榮、武保救之得免。並殊死戰,多所傷殺。胡眾大敗,舍糧棄甲,緣山遁走,乘勝追之,斬獲甚眾。胡被創,僅得還營。常寶惶懼無計,遣信告胡,欲突圍奔出。四月四日,胡自率
 數千人迎之,常寶等開城突圍走。



 攸之率輔國將軍沈懷明、軍主周普孫、江方興、申謙之等諸軍悉力擊之。吳喜率眾來赴,為胡別軍所圍,甚急。有人來捉喜馬,將蔡保以刀斫之,斷手,然後得免。



 正員將軍幢主卜伯宗、江夏國侍郎幢主張渙力戰沒陳。伯宗,益州刺史天與子也。



 攸之、喜等苦戰移日,常寶、張繼伯、胡靈秀、焦度等皆被重創,走還胡軍。赭圻城陷,斬偽寧朔將軍南陽太守沈懷寶、偽奉朝請領中舍人督戰謝道遇,納降數千。



 陳
 紹宗單舸奔西岸,與其部曲俱還鵲尾。建安王休仁自虎檻進據赭圻。劉胡遣陳紹宗、陳慶率輕艓二百,大艦五十,出鵲外挑戰;吳喜、張興世、佼長生等擊之。喜支軍主吳獻之飛舸衝突,所向摧陷,斬獲及投水死甚多,追至鵲裏而還。太宗慮胡等或於步路向京邑,使寧朔將軍、廣德令王蘊千人防魯顯。



 時胡等兵眾彊盛,遠近疑惑。太宗欲綏慰人情,遣吏部尚書褚淵至虎檻選用將帥以下,申謙之、杜幼文因此求黃門郎,沈懷明、劉亮求
 中書郎。建安王休仁即使褚淵擬選,上不許,曰:「忠臣殉國,不謀其報,臨難以干朝典,豈臣下之節邪?」



 始安內史王職之、建安內史趙道生、安成太守劉襲,並舉郡奉順。琬遣龍驤將軍廖琰率數千人,并發廬陵白丁攻襲。襲與郡丞檀玢拒戰,大敗,玢臨陳見殺,襲棄郡走,據險自守。琰虜掠而退,襲復出據郡。



 時齊王率眾東北征討,而齊王世子為南康贛令,琬遣使收世子;世子腹心蕭欣祖、桓康等數十人,奉世子長子奔竄草澤,召募得百餘
 人,攻郡出世子。世子自號寧朔將軍,與南康相沈用之、前南海太守何曇直、晉康太守劉紹祖、北地傅浩、東莞童禽等,據郡起義。琬徵始興相殷孚為御史中丞,并令率郡人俱下。孚眾盛,世子避之於揭陽山。琬遣武昌戴凱之為南康相,世子率眾攻之,凱之戰敗遁走。世子遣幢主檀文起千人戍西昌,與襲相應。琬又遣廖琰與其中兵參軍胡昭等築壘於西昌,堅壁相守。琬召豫章太守劉衍以為右將軍、中護軍,殷孚代為豫章太守,督上
 流五郡,以防襲等。



 衡陽內史王應之率郡文武五百許人,起義兵襲何慧文於長沙,徑至城下。慧文率左右出城與戰,應之勇氣奮發,擊殺數人,遂與慧文交手戰,斫慧文八創,慧文斫應之斷足,遂殺之。時湘東國侍郎虞洽為太宗督國秩,在湘東,勸太守顏躍發兵應朝廷,躍不從。洽乃投桂陽,收募得數百人,還欲攻躍,躍懼求和,許之;有眾二千。時琬征慧文率眾下尋陽,發長沙,已行數百里,聞洽起兵,乃回還攻洽,洽尋戰敗奔走。



 殷孚既
 去始興,以郡五官掾譚伯初留知郡事。士人劉嗣祖等斬伯初,據郡起義。



 琬遣始興太守韋希真、鷹揚將軍楊弘之領眾一千討嗣祖。嗣祖亦遣眾出南康,與齊王世子合。希真等以義徒彊盛,住廬陵不敢進。廣州刺史袁曇遠聞始興起義,遣將李萬周、陳伯紹率眾討嗣祖。嗣祖遣兵戍湞陽,萬周亦築壘相守。嗣祖遣人誑萬周曰:「尋陽已平,臺遣劉勔為廣州,垂至。」萬周信之,便回還襲番禺,夜以長梯入城;曇遠怯弱無防,聞萬周反,便徒跣
 出奔,萬周追斬之於城內。交州刺史檀翼被代還至廣州,資貨鉅萬,萬周誣以為逆,襲而殺之。遂劫掠公私銀帛,藉略袁、檀珍寶,悉以自入。



 袁顗悉雍州之眾,來赴尋陽。時孔道存為衛軍長史,行荊州事。琬以黃門侍郎劉道憲代之,以道存為侍中,行雍州事。柳元景之誅也,元景弟子世隆為上庸太守,民吏共藏匿之。顗起兵,召世隆,不至。顗既下,世隆乃合率蠻、宋二千餘人,起義於上庸,來襲襄陽。道存遣將王式民、康元隆等迎擊於萬山,
 世隆大敗,還郡自守。



 沈攸之等與劉胡相持久不決,上又遣彊弩將軍任農夫、振武將軍武會倉、冗從僕射全景文、軍主劉伯符等領兵繼至。攸之繕治船舸,材板不周,計無所出。會琬送五千片榜供胡軍用,俄而風潮奔迅,榜捍突柵出江,胡等力不能制,自撞船艦,殺沒數十人,赴流而下,來泊攸之等營,於是材板大足。



 琬進袁顗都督征討諸軍事,給鼓吹一部。六月十八日,顗率樓船千艘,來入鵲尾,張興世建議越鵲尾上據錢溪,斷其糧
 道。胡累攻之,不能剋,事在《興世傳》。



 劉亮率所領至胡寨下,胡遣其副孫犀及張靈、焦度鐵騎五匹,越磵取亮,不能得,犀回馬去,亮使左右善射者夾身之,墜馬,斬犀首。張繼伯副馬可率所領來降。劉亮營寨,深入賊地,袁顗畏憚之,曰:「賊入我肝臟里,何由得活!」劉胡率輕舸四百,由鵲頭內路,欲攻錢溪。既而謂其長史王念叔曰:「吾少習步戰,未閑水斗。



 若步戰,恒在數萬人中,水戰在一舸之上,舸舸各進,不復相關,正在三十人中取,此非萬全
 之計,吾不為也。」乃託瘧疾,住鵲頭不進。遣龍驤將軍陳慶領三百舸向錢溪,戒慶不須戰:「張興世、武會倉,吾之所悉,自當走耳。」陳慶至錢溪,不敢攻。越錢溪,於梅根立寨。胡別遣將王起領百舸攻興世,興世擊,大破之。胡率其餘舸馳還,謂顗曰:「興世營寨已立,不可卒攻,昨日小戰,未足為損。陳慶已與南陵、大雷諸軍共遏其上,大軍在此,鵲頭諸將又斷其下流,已墮圍中,不足復慮。」顗怒胡不戰,謂曰:「糧運梗塞,當如此何?」胡曰:「彼尚得溯流越我
 而上,此運何以不得沿流越彼而下邪!」顗更使胡率步卒二萬,鐵馬一千,往攻興世。



 休仁因此命沈攸之、吳喜、佼長生、劉靈遺、劉伯符等進攻濃湖,造皮艦十乘,拔其營柵,苦戰移日,大破之。顗被攻既急,馳信召胡令還。



 張興世既據錢溪,江路岨斷,胡軍乏食,琬大送資糧,畏興世不敢下。胡遣將迎之,為錢溪所破,資實覆沒都盡,燒米三十萬斛,胡眾駭懼。胡副張喜來降,說胡欲叛。八月二十四日,胡誑顗云:「更率步騎二萬,上取興世,兼下大
 雷餘餫。」



 令顗悉度馬配之,其夜,委顗奔走,徑趣梅根。先令薛常寶辦船舸,悉撥南陵諸軍,燒大雷諸城而走。顗聞胡走,亦棄眾西奔,至青林見殺。



 胡率數百舸二萬人向尋陽,報子勛詐云:「袁顗已降,軍皆散,唯己率所領獨反。宜速處分,為一戰之資,當停據盆城,誓死不貳。」乃於江外夜取沔口。琬聞胡去,惶擾無復計,呼褚靈嗣等謀之,並不知所出,唯云更集兵力,加賞五階,或云三階者。張悅始發兄子浩喪,乃稱疾呼琬計事,令左右伏甲帳
 後,戒之:「若聞索酒,便出。」琬既至,悅曰:「卿首唱此謀,今事已急,計將安出?」琬曰:「正當斬晉安王,封府庫,以謝罪耳。」悅曰:「今日寧可賣殿下求活邪?」因呼求酒,再呼,左右震懾不能應。第二子洵提刀走出,餘人續至,即斬琬。琬死時,年六十。時中護軍劉順在座,驚起抱悅,左右人欲殺之,悅顧曰:「無關護軍。」



 乃止。



 潘欣之聞琬死,勒兵而至,悅使人語之曰:「鄧琬謀反,即已梟戮。」欣之乃回還,取琬兒並殺之。悅因單舸齎琬首馳下,詣建安王休仁降。蔡那
 子道淵,以父為太宗效力,被繫作部,因亂脫鎖入城,執子勛囚之。沈攸之諸軍至江州,斬子勛於桑尾牙下,傳首京都。劉順及餘同逆,並伏誅。吳喜、張興世進向荊州,沈懷明向郢州,劉亮、張敬兒向雍州,孫超之向湘州,沈思仁、任農夫向豫章,所至皆平定。



 劉胡走入沔,眾稍散,比至石城,裁餘數騎。竟陵郡丞陳懷真,憲子也,聞胡經過,率數十人斷道邀之。胡人馬既疲,自度不免,因隨懷真入城,告渴,與之酒,胡飲酒畢,引佩刀自刺,不死,斬首
 送京邑。張興世弟僧產追胡,未至石城數十里,逢送胡首信,將還竟陵,殺懷真,竊有其功。郢州行事張沈、偽竟陵太守丘景先聞敗,變形為沙門逃走,追擒伏誅。



 荊州聞濃湖平,議欲更遣軍與郢州合勢,又欲斷據巴陵,經日不決。乃遣將趙道始於江津築壘,任演戍沙橋,諸門津要,皆有屯兵。人情轉離,將士漸逃散。更議奉子頊奔益州,就蕭惠開,典簽阮道預、邵宰不同,曰:「近奉別詔,諸籓若改迷歸順者,悉復本爵。且任叔兒已斷白帝,楊僧
 嗣據梁州,雖復欲西,豈可得至。」



 道預、邵宰即與劉道憲解遣白丁,遣使歸罪。荊州治中宗景、土人姚儉等勒兵入城,殺道憲、預、記室參軍鮑照,劫掠府庫,無復孑遺,執子頊以降。



 初,鄧琬徵兵巴東,巴東太守羅寶稱辭以郡接凶蠻,兵力不足分。巴東人任叔兒聚徒起義,遣信要寶稱,寶稱持疑未決,暴疾死。叔兒乃自號輔國將軍,引兵據白帝,殺寶稱二子,阻守三陜。蕭惠開遣費欣壽等五千人攻叔兒,叔兒與戰,大破之,斬欣壽。子頊又遣中
 兵參軍何康之領宜都太守,討叔兒。軍至陜口,為夷帥向子通所破,挺身走還。叔兒遂固白帝。



 孔道存知尋陽已平,遣使歸順。尋聞柳世隆、劉亮當至,眾悉奔逃,道存及三子同時自殺。何慧文始謀同逆,其母禁之不從,母乃攜女歸江陵,遽嫁之。慧文才兼將吏,幹略有施,雖害王應之,上特加原宥,吳喜宣旨赦之。慧文曰:「既陷逆節,手害忠義,天網雖復恢恢,何面目以見天下之士。」和藥將飲,門生覆之,乃不食而死。



 顏躍慮虞洽還都,說其始
 時同逆,密使人殺之。



 初,淮南定陵人賈襲宗本縣已為劉胡所得,率二十人投沈攸之。攸之言之建安王休仁,休仁版為司徒參軍督護,使還鄉里招集,為胡所禽,以火炙之,問臺軍消息,一無所言,瞋目謂胡曰:「君稱兵內侮,窺覦神器,未聞奇謀遠略,而為炮烙之刑。僕本以身奉義,死亦何有。」胡乃斬之。前軍典簽範道興志不同逆,為琬所誅,其餘奉順見害者,並為上所愍。詔曰:「前鎮軍參軍督護範道興,朕之舊隸,經從北籓,徒役南畿,遭離
 命會,抱恩固節,受害群凶,言念純誠,良有憫愴。可贈員外散騎侍郎。南城令鮑法度、後軍典簽馮次民、永新令應生、新建令庫延寶、上饒令黃難等,違逆識順,同被誅滅,言念既往,宜在追榮。可贈生奉朝請,法度南臺御史,次民、延寶、難並員外將軍。」



 有司奏:「寧朔將軍、督豫州之梁郡諸軍事、豫州刺史、領南梁郡太守竟陵張興世,都統水軍,屢戰克捷,仍進斷賊上流錢溪,貴口苦戰,平定凶逆,今封南平郡作唐縣開國侯,食邑一千戶。寧朔將
 軍、參司徒中直兵軍事廣平佼長生,同統水軍屢戰,及興世上據錢溪,長生獨距賊衝要,功次興世,今封武陵郡遷陵縣開國侯,食邑八百戶。寧朔將軍試守西陽太守吳興全景文、尚書比部郎吳縣孫超之、假輔國將軍右衛將軍南彭城劉亮等三人,並經晉陵苦戰,景文、超之仍又北討破釜,水軍斷賊糧運,及經葛塚、石梁二處破賊,亮南伐經大戰,又最處險劇。景文今封西陽郡孝寧縣,超之封長沙郡羅縣,亮封順陽縣,並開國侯,食邑
 各六百戶。假輔國將軍驃騎司馬劉靈遺、寧朔將軍右軍蔡那、寧朔將軍屯騎校尉段佛榮等三人,統治攻道,並經苦戰,靈遺今封新野郡新野縣,那封始平郡平陽縣,佛榮封湘東郡臨蒸縣,並開國伯,食邑各五百戶。假輔國將軍左軍吳興沈懷明、龍驤將軍積射將軍東平周盤龍、司徒參軍南彭城李安民等三人,懷明經晉陵破賊,又水軍南伐,統治攻道,盤龍雖不統軍,並經大戰,先登陷陳,安民又隨張興世遏斷錢溪,別統軍貴口破
 賊,今封懷明建安郡吳興縣,盤龍封晉安郡晉安縣,安民封建安郡邵武縣,並開國子,食邑各四百戶。假輔國將軍游擊將軍彭城杜幼文、龍驤將軍羽林監太原王穆之、龍驤將軍羽林監濟北頓生、龍驤將軍羽林監沛郡周普孫、員外散騎侍郎硃重恩等五人,幼文經晉陵破賊,在軍統攻道,南伐濃湖,普孫副沈攸之都統眾軍,穆之、生、重恩並南伐有功。今封幼文邵陵郡邵陽縣,穆之封衡陽郡衡山縣,生封始平郡武功縣,普孫封順陽
 郡清水縣,重恩封南海郡龍川縣,並開國男,食邑各三百戶。」



 江方興以戰功為太子左衛率,賊未平,病卒,追封武當縣侯,食邑五百戶。方興,濟陽考城人,衣冠之舊也。龍驤將軍、虎賁中郎將董凱之,隨張興世破胡、白城,先登,封河隆縣子,食邑四百戶。軍主張靈符,東南征討有功,封上饒縣男,食邑三百戶。前征北長兼行參軍楊覆,以貴口有功,封綏城縣男,食邑二百戶。追贈虞洽、檀玢給事中。以李萬周為步兵校尉。陳懷真以斬劉胡功,追
 封永豐縣男,食邑三百戶。



 劉胡,南陽涅陽人也,本名坳胡,以其顏面坳黑似胡,故以為名。及長,以坳胡難道,單呼為胡。出身郡將,捷口,善處分,稍至隊主,討伐諸蠻,往無不捷,蠻甚畏憚之。太祖元嘉二十八年,為振威將軍,率步騎三千,討上如、南山就溪蠻,大破之。孝建元年,朱脩之為雍州,以胡為西外兵參軍、寧朔將軍、建昌太守。擊魯秀有功,除建武將軍、東平陽平二郡太守。入為江夏王義恭太宰參軍,加龍驤將軍。前廢帝景和中,建安
 王休仁嘗為雍州,以胡為休仁安西中兵參軍、馮翊太守,將軍如故,仍轉諮議參軍。太宗即位,除越騎校尉。蠻至今畏之,小兒啼,語之云「劉胡來!」便止。



 段佛榮,京兆人也。泰始五年,自游擊將軍為輔師將軍、豫州刺史,蒞任清謹,為西土所安。後廢帝元徽二年,徵為散騎常侍,領長水校尉。明年,遷衛尉,領右軍將軍,未拜,復出為冠軍將軍、南豫州刺史、歷陽太守。四年,卒,追贈前將軍,改封雲杜縣,謚曰烈侯。



 劉靈遺,襄陽人也。元徽元年,自輔師
 將軍、淮南太守,為南豫州刺史、歷陽太守,將軍如故。明年,徵為散騎常侍,領步兵校尉、南蘭陵太守。病卒,謚曰壯侯。



 袁顗,字景章,陳郡陽夏人,太尉淑兄子也。父洵,吳郡太守。顗初為豫州主簿,舉秀才,不行。後補始興王浚後軍行參軍,著作佐郎,廬陵王紹南中郎主簿,世祖征虜、撫軍主簿,廬江太守,尚書都官郎,江夏王義恭驃騎記室參軍,汝陰王文學,太子洗馬。時顗父為吳郡,鳷隨父在
 官。值元凶弒立,安東將軍隨王誕舉兵入討,板鳷為諮議參軍。事寧,除正員郎,晉陵太守。遭父憂,服闋,為中書侍郎,又除晉陵太守,襲南昌縣五等子。大明二年,除東海王禕平南司馬、尋陽太守,行江州事。復為義陽王昶前軍司馬,太守如故。昶尋罷府,司馬職解,加寧朔將軍,改太守為內史。復為尋陽王子房冠軍司馬,將軍如故,行淮南、宣城二郡事。五年,召為太子中庶子,御史中丞,領本州大中正。七年,遷侍中。明年,除晉安王子勛鎮軍
 長史、襄陽太守,加輔國將軍。未行,復為永嘉王子仁左軍長史、廣陵太守,將軍如故。未拜,復為侍中,領前軍將軍。



 大明末,新安王子鸞以母嬖有盛寵,太子在東宮多過失,上微有廢太子立子鸞之意,從容頗言之。顗盛稱太子好學,有日新之美。世祖又以沈慶之才用不多,言論頗相蚩毀,顗又陳慶之忠勤有幹略,堪當重任。由是前廢帝深感顗,慶之亦懷其德。景和元年,誅群公,欲引進顗,任以朝政,遷為吏部尚書。又下詔曰:「宗社多故,釁
 因冢司,景命未淪,神祚再乂,自非忠謀密契,豈伊剋殄。侍中祭酒、領前軍將軍、新除吏部尚書顗,游擊將軍、領著作郎、兼尚書左丞徐爰,誠心內款,參聞嘉策,匡贊之效,實監朕懷。宜甄茅社,以獎義概。顗可封新隆縣子,爰可封吳平縣子,食邑各五百戶。」俄而意趣乖異,寵待頓衰。始令顗與沈慶之、徐爰參知選事,尋復反以為罪,使有司糾奏,坐白衣領職。從幸湖熟,往反數日,不被喚召。



 顗慮及禍,詭辭求出,沈慶之為顗固陳,乃見許。除建安
 王休仁安西長史、襄陽太守,加冠軍將軍。休仁不行,即以顗為使持節、督雍、梁、南北秦四州、郢州之竟陵、隨二郡諸軍事、領寧蠻校尉、雍州刺史,將軍如故。顗舅蔡興宗謂之曰:「襄陽星惡,豈可冒邪?」顗曰:「白刃交前,不救流矢,事有緩急故也。今者之行,本願生出虎口。且天道遼遠,何必皆驗,如其有征,當脩德以禳之耳。」於是狼狽上路,恒慮見追,行至尋陽,喜曰:「今始免矣!」與鄧琬款狎相過,常請間,必盡日窮夜。顗與琬人地本殊,眾知其有異
 志矣。



 既至襄陽,便與劉胡繕脩兵械,纂集士卒。會太宗定大事,進顗號右將軍。以荊州典簽邵宰乘驛還江陵,道由襄陽。顗反意已定,而糧仗未足,且欲奉表於太宗。



 顗子祕書丞戩曰:「一奉表疏,便為彼臣,以臣伐君,於義不可。」顗從之。顗詐云被太皇太后令,使其起兵。便建牙馳檄,奉表勸晉安王子勛即大位,與琬書,使勿解甲。子勛即位,進顗號安北將軍,加尚書左僕射。



 太宗使朝士與顗書曰:夫夷陂相因,興革遞數,或多難而固其國,或
 殷憂而啟聖明,此既著於前史,亦彰於聞見。王室不造,昏凶肆虐,神鼎將淪,宗稷幾泯,幸天未亡宋,乾歷有歸。



 主上體自聖文,繼明作睿,而辱均牖里,屯踰夏臺。既天地俱憤,義勇同奮,剋殄鯨鯢,三靈更造,應天順民,爰集寶命,四海屬息肩之歡,華戎見來蘇之泰。吾等獲免刀鋸,僅全首領,復身奉惟新,命承亨運,緩帶談笑,擊壤聖世。



 汝雖劬勞于外,跡阻京師,然心期所寄,江、漢何遠。自九江告變,皆謂鄧氏狂惑,比日國言藉藉,頗塵吾子。道
 路之議,豈其或然,聞此之日,能無駭惋。



 凶人反道敗德,日夜滋深,暱近狡慝,取謀豺虎,非惟毒流外物,惡積中朝,乃欲毀陵邑,虐崇憲,燒宗廟,鹵御物,然後蕩覆京都,必使蘭蕕俱盡。自非聖上廟算靈圖,俯眉遜避,維持內外,擁衛臣下,則赤縣為戎,百姓其魚矣。此事此理,寧可孰念!



 既天道輔順,謳歌有奉,高祖之孫,文皇之子,德洞九幽,功貫三曜,匡拯家國,提毓黔首,若不子民南面,將使神器何歸。而群小構慝,妄生窺覬,成軫惑燕,貫高亂
 趙,讒人罔極,自古有之。汝中京冠冕,儒雅世襲,多見前載,縣鑒忠邪,何遠遺郎中之清軌,近忘太尉之純概。相與,或群從舅甥,或姻婭周款,一旦胡、越,能無悵恨。若疑誑所至,邪詖無窮,汝當誓眾奮戈,翦此朝食。若自延過聽,迷塗未遠,聖上臨物以仁,接下以愛,豈直雍齒先封,乃當射鉤見相矣!當由力窘跡屈,丹誠未亮邪。跂予南服,寤寐延首,若反棹沿流,歸誠鳳闕,錫珪開宇,非爾而誰。吾等並過荷曲慈,俱叨非服,紆金拖玉,改觀蓬門,入
 奉舜、禹之渥,出見羲、唐之化,雍容揄揚,信白駒空谷之時也。奈何毀擲先基,自蹈凶戾,山門蕭瑟,松庭誰掃,言念楚路,豈不思父母之邦。幸納惡石,以蠲美疹。裁書表意,爾其圖之。



 時尚書右僕射蔡興宗是顗舅,領軍將軍袁粲是顗從父弟,故舊云群從舅甥也。



 子勛征顗下尋陽,遣侍中孔道存行雍州事。顗乃率眾馳下,使子戩領家累俱還。



 時劉胡屯鵲尾,久不決。泰始二年夏,加顗都督征討諸軍事,給鼓吹一部,率樓船千艘,戰士二萬,來
 入鵲尾。顗本無將略,性又怯撓,在軍中未嘗戎服,語不及戰陳,唯賦詩談義而已。不能撫接諸將,劉胡每論事,酬對甚簡,由此大失人情,胡常切齒恚恨。胡以南運未至,軍士匱乏,就顗換襄陽之資。顗答曰:「都下兩宅未成,亦應經理,不可損徹。」又信往來之言,京師米貴,斗至數百,以為不勞攻伐,行自離散,於是擁甲以待之。太宗使顗舊門生徐碩奉手詔譬顗曰:「卿歷觀古今,嶮之與彊,何嘗可恃。自朕踐阼,塗路梗塞,卿無由奉表,未經為臣。
 今追蹤竇融,猶未為晚也。」



 及劉胡叛走,不告顗,顗至夜方知,大怒罵曰:「今年為小子所誤!」呼取飛燕,謂其眾曰:「我當自出追之。」因又遁走。至鵲頭,與戍主薛伯珍及其所領數千人步取青林,欲向尋陽。夜止山間宿,殺馬勞將士,顗顧謂伯珍曰:「我舉八州以謀王室,未一戰而散,豈非天邪!非不能死,豈欲草間求活,望一至尋陽,謝罪主上,然後自刎耳。」因慷慨叱左右索節,無復應者。及旦,伯珍請以間言,乃斬顗首詣錢溪馬軍主襄陽俞湛之。
 湛之因斬伯珍,併送首以為己功。顗死時年四十七。



 太宗忿顗違叛,流尸於江,弟子彖微服求訪,四十一日乃得,密致喪瘞於石頭後罔,與一舊奴,躬共負土。後廢帝即位,方得改葬。



 顗子戩為偽黃門侍郎,加輔國將軍,戍盆城。尋陽敗,戩棄城走,討禽伏誅。



 孔覬,字思遠,會稽山陰人,太常琳之孫也。父邈,揚州治中。覬少骨梗有風力,以是非為己任。口吃,好讀書,早知名。初舉揚州秀才,補主簿,長沙王義欣鎮軍功曹,衡陽
 王義季安西主簿,戶曹參軍,領南義陽太守,轉署記室,奉箋固辭,曰:「記室之局,實惟華要,自非文行秀敏,莫或居之。覬遜業之舉,無聞於鄉部;惰遊之貶,有編於疲農。直山淵藏引,用不遐棄,故得抃風舞潤,憑附彌年。今日之命,非所敢冒。昔之學優藝富,猶尚斯難,況覬能薄質魯,亦何容易。覬聞居方辨物,君人所以官才;陳力就列,自下所以奉上。覬雖不敏,常服斯言。今寵藉惟舊,舉非尚德,恐無以提衡一隅,僉允視聽者也。伏願天明照其
 心請,乞改今局,授以閑曹,則鳧鶴從方,所憂去矣。」又曰:「夫以記室之要,宜須通才敏忠,加性情勤密者。覬學不綜貫,性又疏惰,何可以屬知祕記,秉筆文閨。假吹之尤,方斯非濫。覬少淪常檢,本無遠植,榮進之願,何能忘懷。若實有螢爝,增暉光景,固其騰聲之日,飛藻之辰也,豈敢自求從容,保其淡逸。伏願矜其魯拙,業之有地,則曲成之施,終始優渥。」義季不能奪,遂得免。召為通直郎,太子中舍人,建平王友,秘書丞,中書侍郎,隨王誕安東諮
 議參軍,領記室,黃門侍郎,建平王宏中軍長史。復為黃門,臨海太守。



 初,晉世散騎常侍選望甚重,與侍中不異,其後職任閑散,用人漸輕。孝建三年,世祖欲重其選,詔曰:「散騎職為近侍,事居規納,置任之本,實惟親要,而頃選常侍,陵遲未允,宜簡授時良,永置清轍。」於是吏部尚書顏竣奏曰:「常侍華選,職任俟才,新除臨海太守孔覬意業閑素,司徒左長史王彧懷尚清理,並任為散騎常侍。」世祖不欲威權在下,其後分吏部尚書置二人,以輕其
 任。侍中蔡興宗謂人曰:「選曹要重,常侍閑淡,改之以名而不以實,雖主意欲為輕重,人心豈可變邪!」既而常侍之選復卑,選部之貴不異。



 覬領本州大中正。大明元年,改太子中庶子,領翊軍校尉,轉祕書監。欲以為吏部郎,不果。遷廷尉卿,御史中丞,坐鞭令史,為有司所糾,原不問。六年,除義興太守,未之任,為尋陽王子房冠軍長史,加寧朔將軍,行淮南、宣城二郡事。



 其年,復除安陸王子綏冠軍長史、江夏內史,復隨府轉後軍長史如故。



 為人
 使酒仗氣,每醉輒彌日不醒,僚類之間,多所凌忽,尤不能曲意權幸,莫不畏而疾之。不治產業,居常貧罄,有無豐約,未嘗關懷。為二府長史,典簽諮事,不呼不敢前,不令去不敢去。雖醉日居多,而明曉政事,醒時判決,未嘗有壅。眾咸云:「孔公一月二十九日醉,勝他人二十九日醒也。」世祖每欲引見,先遣人覘其醉醒。性真素,不尚矯飾,遇得寶玩,服用不疑,而他物粗敗,終不改易。時吳郡顧覬之亦尚儉素,衣裘器服,皆擇其陋者。宋世言清約,
 稱此二人。覬弟道存,從弟徽,頗營產業。二弟請假東還,覬出渚迎之,輜重十餘船,皆是綿絹紙席之屬。



 覬見之,偽喜,謂曰:「我比困乏,得此甚要。」因命上置岸側,既而正色謂道存等曰:「汝輩忝預士流,何至還東作賈客邪!」命左右取火燒之,燒盡乃去。先是,庾徽之為御史中丞,性豪麗,服玩甚華,覬代之,衣冠器用,莫不粗率。蘭臺令史並三吳富人,咸有輕之之意,覬蓬首緩帶,風貌清嚴,皆重迹屏氣,莫敢欺犯。庾徽之,字景猷,潁川鄢陵人也。自
 中丞出為新安王子鸞北中郎長史、南東海太守,卒官。



 八年,覬自郢州行真,徵為右衛將軍,未拜,徙司徒左長史;道存代覬為後軍長史、江夏內史。時東土大旱,都邑米貴,一斗將百錢。道存慮覬甚乏,遣吏載五百斛米餉之。覬呼吏謂之曰:「我在彼三載,去官之日,不辦有路糧。二郎至彼未幾,那能便得此米邪?可載米還彼。」吏曰:「自古以來,無有載米上水者,都下米貴,乞於此貨之。」不聽,吏乃載米而去。永光元年,遷侍中,未拜,復為江夏王義恭
 太宰長史,復出為尋陽王子房右軍長史,加輔國將軍,行會稽郡事。



 太宗即位,召覬為太子詹事,遣故佐平西司馬庾業為右軍司馬,代覬行會稽郡事。時上流反叛,上遣都水使者孔璪入東慰勞。璪至,說覬以:「廢帝侈費,倉儲耗盡,都下罄匱,資用已竭。今南北並起,遠近離叛,若擁五郡之銳,招動三吳,事無不克。」覬然其言,遂發兵馳檄。覬子長公、璪二子淹、玄並在都,馳信密報。



 泰始二年正月,並叛逃東歸。遣書要吳郡太守顧琛,琛以母年
 篤老,又密邇京邑,與長子寶素謀議,未叛。少子寶先時為山陰令,馳書報琛,以南師已近,朝廷孤弱,不時順從,必有覆滅之禍。覬前鋒軍已渡浙江,琛遂據郡同反。吳興太守王曇生、義興太守劉延熙、晉陵太守袁標,一時響應。庾業既東,太宗即以代延熙為義興,加建威將軍,以延熙為巴陵王休若鎮東長史。業至長塘湖,即與延熙合。



 太宗遣建威將軍沈懷明東討,尚書張永係進,鎮東將軍巴陵王休若董統東討諸軍事。移檄東土曰:蓋
 聞釁集有兆,禍至無門,倚伏之來,實惟人致。故囂、述貪亂,終殄宗祀;昌、憲構氛,旋潤斧鉞。斯則昭章記牒,炯戒今古者也。



 自國步時艱,三綱道盡,神歇靈繹,璇業綴旒。皇上仁雄集瑞,英睿應歷,鳳儀熛昇,龍輝電舉。盪穢紫樞,不俟鳴條之誓,凝政中宇,不肆漂杵之威。是以墜維再造,虧天重構,幽明裁紀,標配斯光。而群凶恣虐,協扇童孺,蕞爾東垂,復淪醜迹,邪回從慝,蜂動蟻附。聖圖霆發,神威四臨,羽馹所屆,義旅雲屬,欃鉞所麾,逆徒冰泮,
 勝負之效,皎然已顯。



 司徒建安王英猷冠世,董率元戎。驃騎山陽王風略夙昭,撫厲中陳。或振霜江、蠡,或騰焱荊、河,金甲燭天庭,囂聲震海浦。前將軍、吳興太守張永,東南標秀,協贊戎機。建威將軍沈懷明、鎮東中兵參軍劉亮、武衛將軍壽寂之,霜銳五千,熊騰虎步。龍驤將軍王穆之、龍驤將軍頓生,鐵騎連群,風驅電邁。右軍將軍齊王、射聲校尉姚道和,樓艦千艘,覆川蓋汜。左軍垣恭祖、步兵校尉杜幼文、冗從僕射全景文、員外散騎侍郎
 孫超之,並率虎旅,駱驛雲赴。殿中將軍杜敬真、殿中將軍陸攸之、建武將軍吳喜,甲楯一萬,分趣義興。予猥承人乏,總司戎統,聳劍東馳,申憤海曲。噴氣則白日盡晦,刷馬則清江倒流。以此伐叛,何勍不剿,以此柔服,何順不懷。愍彼群迷,弗辨堯、桀,螳黽微命,擬雷霆之衝;已枯之葉,當霜飆之隊。尺豎所為寒心,匹婦所為歎息。夫因禍致慶,資敗為成,前監不忘,後事明筮。



 若能相率歸順,投兵效款,則福鐘當年,祉覃來裔,孰如身轘宗屠,鬼喂
 魂泣者哉!



 詳鏡安危,自求多福。



 購生禽覬千五百戶開國縣侯;生擒琛千戶開國縣侯。斬送者半賞。時將士多是東人,父兄子弟皆已附逆,上因送軍普加宣示曰:「朕方務德簡刑,使四罪不相及,助順同逆者,一以所從為斷。卿等當深達此懷,勿以親戚為慮也。」眾於是大悅。



 覬所遣孫曇瓘等軍,頓晉陵九里,部陳甚盛。懷明至奔牛,所領寡弱,乃築壘自固。張永至曲阿,未知懷明安否,百姓驚擾,將士咸欲離散。永退還延陵,就休若;諸將帥咸
 勸退保破岡。其日大寒,風雪甚猛,塘埭決壞,眾無固心。休若宣令:「敢有言退者,斬!」眾小定,乃築壘息甲。尋得懷明書,賊定未進。軍主劉亮又繼至,兵力轉加,人情乃安。



 時永世令孔景宣復反,柵縣西江峴山,斷遏津徑,劉延熙加其寧朔將軍。杜敬真、陸攸之、溧陽令劉休文攻景宣別寨,斬其中兵參軍史覽之等十五人。永世人徐崇之率鄉里起義,攻縣斬景宣。吳喜至,板崇之領縣事。太宗嘉休文等誠效,除休文寧朔將軍,縣如故;崇之殿中
 將軍,行永世縣事,並賜侯爵。喜、敬真及員外散騎侍郎竺超之等至國山縣界,遇東軍於虎檻村,擊大破之。自國山進吳城,去義興十五里。劉延熙遣楊玄、孫矯之、沈靈秀、黃泰四軍拒喜。喜等兵力甚弱,眾寡勢懸,交戰盡日,臨陳斬楊玄、孫矯之、黃泰,餘眾一時奔走,因進義興南郭外。延熙屯軍南射堂,喜遣步騎擊之,即退還水北,乃柵斷長橋,保郡自守。喜築壘與之相持。庾業於長塘湖口夾岸築城,有眾七千餘人,器甲甚盛,與延熙遙相掎角。沈
 懷明、張永與晉陵軍相持,久不決。



 太宗每遣軍,輒多所求須,不時上道。外監朱幼舉司徒參軍督護任農夫,驍果有膽力,性又簡率,資給甚易,乃以千人配之,使助東討。時庾業兵盛,農夫於延陵出長塘,雖云千兵,至者裁四百。未至數十里,遣人參候,云:「賊築城猶未合。」



 農夫率廣武將軍高志之、永興令徐崇之馳往攻之。因其城壘未立,農夫親持刀楯,赴城入陳,大破之,庾業棄城走義興。先是,龍驤將軍阮佃夫募得蜀人數百,多壯勇便戰,
 皆著犀皮鎧,執短兵。本應就佃夫向晉陵,未發,會農夫須人,分以配之。



 及戰,每先登,東人並畏憚,又怪其形飾殊異,舊傳狐獠食人,每見之,輒奔走。



 農夫收其船杖,與高志之進義興援吳喜。二月一日,喜乃度水攻郡,分兵擊諸壘柵。



 農夫雖至,眾力尚少,兵勢不敵。喜乃與數騎登高東西指麾,若招引四面俱進者。



 東軍大駭,諸營一時奔散,唯龍驤將軍孔睿一柵未拔。喜以殺傷者多,乃開圍緩之。



 其夜,庾業、孔睿相率奔走,義興平。劉延熙投
 水死,有人告之,乃斬尸,傳首京邑。義興諸縣唯綏安令巢邃秉節不移,不受偽爵。



 時齊王率軍東討,與張永、劉亮、杜幼文、沈懷明等於晉陵九里西結營,與東軍相持。義興軍既為吳喜等所破,奔散者多投晉陵,東軍震恐。上又遣積射將軍江方興、南臺御史王道隆至晉陵視賊形勢。賊帥孫曇瓘、程捍宗、陳景遠凡有五城,互相連帶;捍宗城猶未固。其月三日,道隆與齊王、張永共議:「捍宗城既未立,可以籍手。上副聖旨,下成眾氣。」道隆便率
 所領急攻之,俄頃城陷,斬捍宗首。



 劉亮果勁便刀楯,朝士先不相悉,上亦弗聞,唯尚書左丞徐爰知之,白太宗,稱其驍敢。至是,每戰以刀楯直盪,往輒陷決,張永嫌其過銳,不令居前。賊連柵周亙,塘道迫狹,將士力不得展,亮乃負楯而進,直入重柵,眾軍因之,即皆摧破。袁標遣千人繼至,齊王與永等乘勝馳擊,又大破之,屠其兩城。曇瓘率眾數百,鼓噪而至,標又遣千人繼之,眾軍駭懼,將欲散矣。江方興率勇士迎射之,應弦倒者相繼,曇瓘
 因此敗走。



 吳喜軍至義鄉,偽輔國將軍、車騎司馬孔璪屯吳興南亭,太守王曇生詣璪計事,會信還,云:「臺軍已近。」璪大懼,墮床,曰:「懸賞所購,唯我而已,今不遽走,將為人禽。」左右聞之,並各散走。璪與曇生焚燒倉庫,東奔錢塘。喜至吳興,頓置郡城,倉廩遇雨不然,無所損失。初,曇生遣寧朔將軍沈靈寵率八千人向黃鵠嶠,欲從候道出蕪湖,迎接南軍。廣德令王蘊發兵據嶮,靈寵不得進,屯住故鄣。



 曇生既走,靈寵乃與弟靈昭、軍副姚天覆率偏裨
 以下十七軍歸順。太宗嘉之,擢為鎮東參軍事,因率所領東討。喜分遣軍主沈思仁、吳係公追躡璪等。



 陸攸之、任農夫自東遷進向吳郡,臺遣軍主張靈符即晉陵。其月四日,齊王急攻之。其夜,孫曇瓘、陳景遠一時奔潰。諸軍至晉陵,袁標棄郡東走。晉陵既平,吳中震動。吳興軍又將至,顧琛與子寶素攜其老母泛海奔會稽,海鹽令王孚邀討不及。



 太宗以四郡平定,留吳喜統全景文、沈懷明、劉亮、孫超之、壽寂之等東平會稽,追齊王、張永、
 姚道和、杜幼文、垣恭祖、張靈符北討,王穆之、頓生、江方興南伐。其月九日,喜等至錢唐,錢唐令顧昱及孔璪、王曇生等奔渡江東。喜仍進軍柳浦,諸暨令傅琰將家歸順。喜遣鎮北參軍沈思仁、彊弩將軍任農夫、龍驤將軍高志之、南臺御史阮佃夫、揚武將軍盧僧澤等率軍向黃山浦。東軍據岸結寨,農夫等攻破之,乘風舉帆,直趣定山,破其大帥孫會之,於陳斬首。自定山進向漁浦,戍主孔睿率千餘人據壘拒戰。佃夫使隊主闕法炬射殺
 樓上弩手,睿眾驚駭,思仁縱兵攻之,斬其軍主孔奴,於是敗散。其月十九日,吳喜使劉亮由鹽官海渡,直指同浦;壽寂之濟自漁浦,邪趣永興;喜自柳浦渡,趣西陵。西陵諸軍皆悉散潰,斬庾業、顧法直、吳恭,傳首京都。東軍主卜道濟、督戰許天賜請降。庾業,新野人也。



 父彥達,以幹局為太祖所知,為益州刺史。世祖世,官至豫章太守,太常卿。劉亮、全景文、孫超之進次永興同市,遇覬所遣陸孝伯、孔豫兩軍,與戰破之,斬孝伯、豫首。



 會稽聞西軍
 稍近,將士多奔亡,覬不能復制。二十日,上虞令王晏起兵攻郡,覬以東西交逼,憂遽不知所為。其夕,率千餘人聲云東討,實趣石瀃。先已具船海浦,值潮涸不得去,眾叛都盡,門生載以小船,竄于山村。偽車騎從事中郎張綏先遣人於錢唐詣喜歸誠,及覬走,綏閉封倉庫,以待王師。二十一日,晏至郡,入自北門,囚綏付作部,其夜殺之。執尋陽王子房於別署,縱兵大掠,府庫空盡。若邪村民錄送偽龍驤將軍、車騎中兵參軍軍主孔睿,將斬
 之。睿曰:「吾年已過立,未霑官伍,蒙知己之顧,以身許之。今日就死,亦何所恨!」含笑就戮。孔璪叛投門生陸林夫,林夫斬首送之。二十二日,山民縛覬送詣晏,晏謂之曰:「此事孔璪所為,無豫卿事。可作首辭,當相為申上。」覬曰:「江東處分,莫不由身,委罪求活,便是君輩行意耳!」晏乃斬之東閣外。臨死求酒,曰:「此是平生所好。」



 時年五十一。顧琛、王曇生、袁標等並詣喜歸罪,喜皆宥之。琛子寶素與父相失,自縊死。東軍主凡七十六人,於陳斬十七
 人,其餘皆原宥。初,遣庾業向會稽,追使奉朝請孫長度送仗與之,并令召募。行達晉陵,袁標就其求仗,長度不與,為標所殺。追贈給事中。



 先是,鄧琬遣臨川內史張淹自南路出東陽,淹遣龍驤將軍桂遑、征西行參軍劉越緒屯據定陽縣。巴陵王休若遣沈思仁討之,思仁遣軍主崔公烈攻其營,斬幢主朱伯符首,桂遑、劉越緒諸軍並奔逸。晉安太守劉瞻據郡同逆,建安內史趙道生起義討之,聚徒未合。七月,思仁遣軍主姚宏祖、鮑伯奮、應
 寄生等討破瞻,斬之於羅江縣。



 鄧琬先遣新安太守陽伯子及軍主任獻子襲黟縣,縣令吳茹公固守,力不敵,棄城走,伯子等屯據縣城。茹公與臺軍主丘敬文、李靈賜、蕭柏壽等攻圍彌時,八月乃克,斬伯子、獻子首。張淹屯軍上饒縣,聞劉胡敗,軍副鄱陽太守費曇欲圖之,詐云:「得鄧琬信,急宜諮論。」欲因此斬淹。淹素事佛,方禮佛,不得時進。曇復誑云捕虎,借大鼓及仗士二百人,淹信而與之。曇因率眾入山,饗士約誓,揚言虎走城西,鳴鼓
 大呼,直來趣城;城門守衛,悉委仗觀之,曇率眾突入,淹正禮佛,聞難走出,因斬首。



 史臣曰:自江左以來,舉干戈以圖宗國,十有一焉,其能克振者,四而已矣。



 元皇外守虛器,政由王氏;蘇峻事雖暫申,旋受屠磔;桓玄宣武之子,運屬橫流;世祖仗順入討,民無異望。其餘皆漆顙夷宗,作戒於後,何哉?夫勝敗之數,實由眾心,社廟尊嚴,民情所係,安以義動,猶或稱難,況長戟指闕,志在陵暴者乎!



 泰始交爭,逆順未辨,太
 宗身劋悖亂,事惟拯溺,國道屯詖,宜立長君,太祖之昭,義無不可。子勛體自世祖,家運已絕,當璧之命,屬有所歸。曲直二途,未知攸適。



 徒以據有神甸,擅資天府,宗稷之重,威臨四方,以中制外,故能式清區宇。夫帝王所居,目以眾大之號,名曰京師,其義趣遠有以也。



\end{pinyinscope}