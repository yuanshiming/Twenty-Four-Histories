\article{卷六十四列傳第二十四 鄭鮮之 裴松之 何承天}

\begin{pinyinscope}

 鄭鮮之,字道子,滎陽開封人也。高祖渾,魏將作大匠。曾祖襲,大司農。父遵,尚書郎。襲初為江乘令,因居縣境。鮮之下帷讀書,絕交游之務。初為桓偉輔國主簿。先是,兗
 州刺史滕恬為丁零、翟遼所沒,屍喪不反,恬子羨仕宦不廢,議者嫌之。桓玄在荊州,使群僚博議,鮮之議曰:名教大極,忠孝而已,至乎變通抑引,每事輒殊,本而尋之,皆是求心而遺跡。



 跡之所乘,遭遇或異。故聖人或就跡以助教,或因跡以成罪,屈申與奪,難可等齊,舉其阡陌,皆可略言矣。天可逃乎?而伊尹廢君;君可脅乎?而鬻權見善;忠可愚乎?而箕子同仁。自此以還,殊實而齊聲,異譽而等美者,不可勝言。而欲令百代之下,聖典所闕,正
 斯事於一朝,豈可易哉!



 然立言明理,以古證今,當使理厭人情。如滕羨情事者,或終身隱處,不關人事;或昇朝理務,無譏前哲。通滕者則以無譏為證,塞滕者則以隱處為美。折其兩中,則異同之情可見矣。然無譏前哲者,厭情之謂也。若王陵之母,見烹於楚,陵不退身窮居,終為社稷之臣,非為榮也。鮑勛蹇諤魏朝,亡身為效,觀其志非貪爵也。凡此二賢,非滕之諭。



 夫聖人立教,猶云「有禮無時,君子不行」。有禮無時,政以事有變通,不可守一
 故耳。若滕以此二賢為證,則恐人人自賢矣;若不可人人自賢,何可獨許其證。



 譏者兼在於人,不但獨證其事。漢、魏以來,記闕其典,尋而得者無幾人。至乎大晉中朝及中興之後,楊臻則七年不除喪,三十餘年不關人事,溫公則見逼於王命,庾左丞則終身不著袷,高世遠則為王右軍、何驃騎所勸割,無有如滕之易者也。若以縗麻非為哀之主,無所復言矣。文皇帝以東關之役,尸骸不反者,制其子弟,不廢婚宦。明此,孝子已不自同於人
 倫,有識已審其可否矣。若其不爾,居宗輔物者,但當即聖人之教,何所復明制於其間哉!及至永嘉大亂之後,王敦復申東關之制於中興,原此是為國之大計,非謂訓範人倫,盡於此也。



 何以言之?父仇明不同戴天日,而為國不可許復仇,此自以法奪情,即是東關、永嘉之喻也。何妨綜理王務者,布衣以處之。明教者自謂世非橫流,凡士君子之徒,無不可仕之理,而雜以情譏,謂宜在貶裁爾。若多引前事以為通證,則孝子可顧法而不復
 仇矣。文皇帝無所立制於東關,王敦無所明之於中興。每至斯會,輒發之於宰物,是心可不喻乎!



 且夫求理當先以遠大,若滄海橫流,家國同其淪溺,若不仕也,則人有餘力;人有餘力,則國可至乎亡,家可至乎滅。當斯時也,匹婦猶亡其身,況大丈夫哉!



 既其不然,天下之才,將無所理。滕但當盡《陟岵》之哀,擬不仕者之心,何為證喻前人,以自通乎?且名為大才之所假,而小才之所榮,榮與假乘常,已有慚德,無欣工進,何有情事乎?若其不然,
 則工進無欣,何足貴於千載之上邪!茍許小才榮其位,則滕不當顧常疑以自居乎。所謂柳下惠則可,我則不可也。



 且有生之所宗者聖人,聖人之為教者禮法,即心而言,則聖人之法,不可改也。



 而秦以郡縣治天下,莫之能變;漢文除肉刑,莫之能復。彼聖人之為法,猶見改於後王,況滕賴前人,而當必通乎?若人皆仕,未知斯事可俟後聖與不?況仕與不仕,各有其人,而不仕之所引,每感三年之下。見議者弘通情紀,每傍中庸,又云若許譏
 滕,則恐亡身致命之仕,以此而不盡。何斯言之過與!夫忠烈之情,初無計而後動。若計而後動,則懼法不盡命。若有不盡,則國有常法。故古人軍敗於外,而家誅於內。茍忠發自內,或懼法於外,復有踟躕顧望之地邪!若有功不賞,有罪不誅,可致斯喻爾。無有名教翼其子弟,而子弟不致力於所天。不致力於所天,則王經忠不能救主,孝不顧其親,是家國之罪人爾,何所而稱乎?夫恩宥十世,非不隆也;功高賞厚,非不報也。若國憲無負於滕
 恬,則羨之通塞,自是名教之所及,豈是勸沮之本乎?



 議者又以唐虞邈矣,孰知所歸,尋言求意,將所負者多乎。後漢亂而不亡,前史猶謂數公之力。魏國將建,荀令君正色異議,董昭不得枕蘇則之膝,賈充受辱於庾純。以此而推,天下之正義,終自傳而不沒,何為發斯嘆哉!若以時非上皇,便不足復言多者,則夷齊於奭、望,子房於四人,亦無所復措其言矣。至於陳平默順避禍,以權濟屈,皆是衛生免害,非為榮也。滕今生無所衛,鞭塞已冥,
 義安在乎?



 昔陳壽在喪,使婢丸藥,見責鄉閭;阮咸居哀,騎驢偷婢,身處王朝。豈可以阮獲通於前世,便無疑於後乎!且賢聖抑引,皆是究其始終,定其才行。故雖事有驚俗,而理必獲申。郗詵葬母後園,而身登宦,所以免責,以其孝也。日磾殺兒無譏,以其忠也。今豈可以二事是忠孝之所為,便可許殺兒葬母後園乎?不可明矣。既其不可,便當究定滕之才行,無所多辯也。



 滕非下官鄉親,又不周旋,才能非所能悉。若以滕謀能決敵,才能周用,
 此自追蹤古人,非議所及。若是士流,故謂宜如子夏受曾參之詞,可謂善矣,而子夏無不孝之稱也。意之所懷,都盡於此,自非名理,何緣多其往復;如其折中,裁之居宗。



 桓偉進號安西,轉補功曹,舉陳郡謝絢自代,曰:「蓋聞知賢弗推,臧文所以竊位;宣子能讓,晉國以之獲寧。鮮之猥承人乏,謬蒙過眷,既恩以義隆,遂再叨非服。知進之難,屢以上請,然自退之志,未獲暫申,夙夜懷冰,敢忘其懼。伏見行參軍謝絢,清悟審正,理懷通美,居以端右,
 雖未足舒其采章,升庸以漸,差可以位擬人。請乞愚短,甘充下列,授為賢牧,實副群望。」入為員外散騎侍郎,司徒左西屬,大司馬琅邪王錄事參軍,仍遷御史中丞。



 性剛直,不阿強貴,明憲直繩,甚得司直之體。外甥劉毅,權重當時,朝野莫不歸附,鮮之盡心高祖,獨不屈意於毅,毅甚恨焉。義熙六年,鮮之使治書侍御史丘洹奏彈毅曰:「上言傳詔羅道盛輒開箋,遂盜發密事,依法棄市,奏報行刑,而毅以道盛身有侯爵,輒復停宥。按毅勳德光
 重,任居次相,既殺之非己,無緣生之自由。又奏之於先,而弗請於後,閫外出疆,非此之謂。中丞鮮之於毅舅甥,制不相糾,臣請免毅官。」詔無所問。



 時新制長吏以父母疾去官,禁錮三年。山陰令沈叔任父疾去職,鮮之因此上議曰:「夫事有相權,故制有與奪,此有所屈,而彼有所申。未有理無所明,事無所獲,而為永制者也。當以去官之人,或容詭託之事。詭託之事,誠或有之,豈可虧天下之大教,以末傷本者乎?且設法蓋以眾苞寡,而不以寡
 違眾,況防杜去官而塞孝愛之實。且人情趨於榮利,辭官本非所防,所以為其制者,蒞官不久,則奔競互生,故杜其欲速之情,以申考績之實。省父母之疾,而加以罪名,悖義疾理,莫此為大。謂宜從舊,於義為允。」從之。於是自二品以上父母沒者,墳墓崩毀及疾病族屬輒去,並不禁錮。



 劉毅當鎮江陵,高祖會於江寧,朝士畢集。毅素好摴蒱,於是會戲。高祖與毅斂局,各得其半,積錢隱人,毅呼高祖併之。先擲得雉,高祖甚不說,良久乃答之。



 四
 坐傾矚,既擲,五子盡黑,毅意色大惡,謂高祖曰:「知公不以大坐席與人!」



 鮮之大喜,徒跣繞床大叫,聲聲相續。毅甚不平,謂之曰:「此鄭君何為者!」無復甥舅之禮。高祖少事戎旅,不經涉學,及為宰相,頗慕風流,時或言論,人皆依違之,不敢難也。鮮之難必切至,未嘗寬假,要須高祖辭窮理屈,然後置之。高祖或有時慚恧,變色動容,既而謂人曰:「我本無術學,言義尤淺。比時言論,諸賢多見寬容,唯鄭不爾,獨能盡人之意,甚以此感之。」時人謂為「格
 佞」。



 自中丞轉司徒左長史,太尉咨議參軍,俄而補侍中,復為太尉咨議。十二年,高祖北伐,以為右長史。鮮之曾祖墓在開封,相去三百里,乞求拜省,高祖以騎送之。宋國初建,轉奉常。



 佛佛虜陷關中,高祖復欲北討,行意甚盛。鮮之上表諫曰:「伏思聖略深遠,臣之愚管無所措其意。然臣愚見,竊有所懷。虜凶狡情狀可見,自關中再敗,皆是帥師違律,非是內有事故,致外有敗傷。虜聞殿下親御六軍,必謂見伐,當重兵守潼關,其勢然也。若陵威長
 驅,臣實見其未易;若輿駕頓洛,則不足上勞聖躬。如此,則進退之機,宜在熟慮。賊不敢乘勝過陜,遠懾大威故也。今盡用兵之算,事從屈申,遣師撲討,而南夏清晏,賊方懼將來,永不敢動。若輿駕造洛而反,凶醜更生揣量之心,必啟邊戎之患,此既必然。江南顒顒,傾注輿駕,忽聞遠伐,不測師之深淺,必以殿下大申威靈,未還,人情恐懼,事又可推。往年西征,劉鐘危殆,前年劫盜破廣州,人士都盡。三吳心腹之內,諸縣屢敗,皆由勞役所致。又
 聞處處大水,加遠師民敝,敗散,自然之理。殿下在彭城,劫盜破諸縣,事非偶爾,皆是無賴凶慝。凡順而撫之,則百姓思安;違其所願,必為亂矣。古人所以救其煩穢,正在於斯。漢高身困平城,呂后受匈奴之辱,魏武軍敗赤壁,宣武喪師枋頭,神武之功,一無所損。況偏師失律,無虧於廟堂之上者邪!即之事實,非敗之謂,唯齡石等可念爾。若行也,或速其禍。反覆思惟,愚謂不煩殿下親征小劫。西虜或為河、洛之患,今正宜通好北虜,則河南安。
 河南安,則濟、泗靜。伏願聖鑑察臣愚懷。」



 高祖踐阼,遷太常,都官尚書。鮮之為人通率,在高祖坐,言無所隱,時人甚憚焉。而隱厚篤實,贍恤親故。性好游行,命駕或不知所適,隨御者所之。尤為高祖所狎,上嘗於內殿宴飲,朝貴畢至,唯不召鮮之。坐定,謂群臣曰:「鄭鮮之必當自來。」俄而外啟:「尚書鮮之詣神虎門求啟事。」高祖大笑引入,其被親遇如此。



 永初二年,出為丹陽尹,復入為都官尚書,加散騎常侍。以從征功,封龍陽縣五等子。出為豫章
 太守,秩中二千石。元嘉三年,王弘入為相,舉鮮之為尚書右僕射。四年,卒,時年六十四。追贈散騎常侍、金紫光祿大夫。文集傳於世。子愔,位至尚書郎,始興太守。



 裴松之,字世期,河東聞喜人也。祖昧,光祿大夫。父珪,正員外郎。松之年八歲,學通《論語》、《毛詩》。博覽墳籍,立身簡素。年二十,拜殿中將軍。此官直衛左右,晉孝武太元中革選名家以參顧問,始用琅邪王茂之、會稽謝輶,皆南北之望。舅庾楷在江陵,欲得松之西上,除新野太守,以
 事難不行。拜員外散騎侍郎。義熙初,為吳興故鄣令,在縣有績。入為尚書祠部郎。



 松之以世立私碑,有乖事實,上表陳之曰:「碑銘之作,以明示後昆,自非殊功異德,無以允應茲典。大者道勛光遠,世所宗推;其次節行高妙,遺烈可紀。若乃亮采登庸,績用顯著,敷化所蒞,惠訓融遠,述詠所寄,有賴鐫勒,非斯族也,則幾乎僭黷矣。俗敝偽興,華煩已久,是以孔悝之銘,行是人非;蔡邕制文,每有愧色。而自時厥後,其流彌多,預有臣吏,必為建立,勒
 銘寡取信之實,刊石成虛偽之常,真假相蒙,殆使合美者不貴,但論其功費,又不可稱。不加禁裁,其敝無已。」以為「諸欲立碑者,宜悉令言上,為朝議所許,然後聽之。庶可以防遏無征,顯彰茂實,使百世之下,知其不虛,則義信於仰止,道孚於來葉。」由是並斷。



 高祖北伐,領司州刺史,以松之為州主簿,轉治中從事史。既克洛陽,高祖敕之曰:「裴松之廊廟之才,不宜久尸邊務,今召為世子洗馬,與殷景仁同,可令知之。」于時議立五廟樂,松之以妃
 臧氏廟樂亦宜與四廟同。除零陵內史,徵為國子博士。



 太祖元嘉三年,誅司徒徐羨之等,分遣大使,巡行天下。通直散騎常侍袁渝、司徒左司掾孔邈使揚州,尚書三公郎陸子真、起部甄法崇使荊州,員外散騎常侍范雍、司徒主簿龐遵使南兗州,前尚書右丞孔默使南北二豫州,撫軍參軍王歆之使徐州,冗從僕射車宗使青、兗州,松之使湘州,尚書殿中郎阮長之使雍州,前竟陵太守殷道鸞使益州,員外散騎常侍李耽之使廣州,郎中
 殷斌使梁州、南秦州,前員外散騎侍郎阮園客使交州,駙馬都尉、奉朝請潘思先使寧州,並兼散騎常侍。班宣詔書曰:「昔王者巡功,群后述職,不然則有存省之禮,聘眺之規。所以觀民立政,命事考績,上下偕通,遐邇咸被,故能功昭長世,道歷遠年。朕以寡暗,屬承洪業,夤畏在位,昧于治道,夕惕惟憂,如臨淵谷。懼國俗陵頹,民風凋偽,眚厲違和,水旱傷業。雖躬勤庶事,思弘攸宜,而機務惟殷,顧循多闕,政刑乖謬,未獲具聞。



 豈誠素弗孚,使群
 心莫盡,納隍之愧,在予一人。以歲時多難,王道未壹,卜征之禮,廢而未修,眷被氓庶,無忘欽恤。今使兼散騎常侍渝等申令四方,周行郡邑,親見刺史二千石官長,申述至誠,廣詢治要,觀察吏政,訪求民隱,旌舉操行,存問所疾。禮俗得失,一依周典,每各為書,還具條奏,俾朕昭然,若親覽焉。大夫君子,其各悉心敬事,無惰乃力。其有咨謀遠圖,謹言中誠,陳之使者,無或隱遺。



 方將敬納良規,以補其闕。勉哉勖之,稱朕意焉。」



 松之反使,奏曰:「臣聞
 天道以下濟光明,君德以廣運為極。古先哲后,因心溥被,是以文思在躬,則時雍自洽,禮行江漢,而美化斯遠。故能垂大哉之休詠,廓造周之盛則。伏惟陛下神睿玄通,道契曠代,冕旒華堂,垂心八表。咨敬敷之未純,慮明揚之靡暢。清問下民,哀此鰥寡,渙焉大號,周爰四達。遠猷形於《雅》、《誥》,惠訓播乎遐陬。是故率土仰詠,重譯咸說,莫不謳吟踴躍,式銘皇風。或有扶老攜幼,稱歡路左,誠由亭毒既流,故忘其自至,千載一時,於是乎在。臣謬蒙
 銓任,忝廁顯列,猥以短乏,思純八表,無以宣暢聖旨,肅明風化,黜陟無序,搜揚寡聞,慚懼屏營,不知所措。奉二十四條,謹隨事為牒。伏見癸卯詔書,禮俗得失,一依周典,每各為書,還具條奏。謹依事為書以繫之後。」松之甚得奉使之議,論者美之。



 轉中書侍郎、司冀二州大中正。上使注陳壽《三國志》,松之鳩集傳記,增廣異聞,既成奏上。上善之,曰:「此為不朽矣!」出為永嘉太守,勤恤百姓,吏民便之。入補通直為常侍,復領二州大中正。尋出為南
 琅邪太守。十四年致仕,拜中散大夫,尋領國子博士。進太中大夫,博士如故。續何承天國史,未及撰述,二十八年,卒,時年八十。子駰,南中郎參軍。松之所著文論及《晉紀》,駰注司馬遷《史記》,並行於世。



 何承天,東海郯人也。從祖倫,晉右衛將軍。承天五歲失父,母徐氏,廣之姊也,聰明博學,故承天幼漸訓議,儒史百家,莫不該覽。叔父肹為益陽令,隨肹之官。



 隆安四年,南蠻校尉桓偉命為參軍。時殷仲堪、桓玄等互舉兵以
 向朝廷,承天懼禍難未已,解職還益陽。義旗初,長沙公陶延壽以為其輔國府參軍,遣通敬於高祖,因除瀏陽令,尋去職還都。撫軍將軍劉毅鎮姑孰,版為行參軍。毅嘗出行,而鄢陵縣史陳滿射鳥,箭誤中直帥,雖不傷人,處法棄市。承天議曰:「獄貴情斷,疑則從輕。昔驚漢文帝乘輿馬者,張釋之劾以犯蹕,罪止罰金。何者?明其無心於驚馬也。故不以乘輿之重,加以異制。今滿意在射鳥,非有心於中人。按律過誤傷人,三歲刑,況不傷乎?微罰
 可也。」出補宛陵令。趙惔為寧蠻校尉、尋陽太守,請為司馬。尋去職。



 高祖以為太尉行參軍。高祖討劉毅,留諸葛長民為監軍。長民密懷異志,劉穆之屏人問承天曰:「公今行濟否云何?」承天曰:「不憂西不時,別有一慮爾。公昔年自左里還入石頭,甚脫爾,今還,宜加重復。」穆之曰:「非君不聞此言。頃日願丹徒劉郎,恐不復可得也。」除太學博士。義熙十一年,為世子征虜參軍,轉西中郎中軍參軍,錢唐令。高祖在壽陽,宋臺建,召為尚書祠部郎,與傅
 亮共撰朝儀。永初末,補南臺治書侍御史。



 謝晦鎮江陵,請為南蠻長史。時有尹嘉者,家貧,母熊自以身貼錢,為嘉償責。



 坐不孝當死。承天議曰:「被府宣令,普議尹嘉大辟事,稱法吏葛滕簽,母告子不孝,欲殺者許之。法云,謂違犯教令,敬恭有虧,父母欲殺,皆許之。其所告惟取信於所求而許之。謹尋事原心,嘉母辭自求質錢,為子還責。嘉雖虧犯教義,而熊無請殺之辭。熊求所以生之而今殺之,非隨所求之謂。始以不孝為劾,終於和賣結刑,
 倚旁兩端,母子俱罪,滕簽法文,為非其條。嘉所存者大,理在難申,但明教爰發,矜其愚蔽。夫明德慎罰,文王所以恤下;議獄緩死,《中孚》所以垂化。言情則母為子隱,語敬則禮所不及。今舍乞宥之評,依請殺之條,責敬恭之節,於饑寒之隸,誠非罰疑從輕,寧失有罪之謂也。愚以謂降嘉之死,以普春澤之恩;赦熊之愆,以明子隱之宜。則蒲亭雖陋,可比德於盛明;豚魚微物,不獨遺於今化。」



 事未判,值赦,並免。



 晦進號衛將軍,轉咨議參軍,領記室。
 元嘉三年,晦將見討,其弟黃門郎爵密信報之,晦問承天曰:「若果爾,卿令我云何?」承天曰:「以王者之重,舉天下以攻一州,大小既殊,逆順又異,境外求全,上計也。其次,以腹心領兵戍於義陽,將軍率眾於夏口一戰,若敗,即趨義陽以出北境,其次也。」晦良久曰:「荊楚用武之國,兵力有餘,且當決戰,走不晚也。」使承天造立表檄。晦以湘州刺史張邵必不同己,欲遣千人襲之;承天以為邵意趨未可知,不宜便討。時邵兄茂度為益州,與晦素善,故
 晦止不遣兵。前益州刺史蕭摹之、前巴西太守劉道產去職還江陵,晦將殺之,承天盡力營救,皆得全免。晦既下,承天留府不從。及到彥之至馬頭,承天自詣歸罪,彥之以其有誠,宥之,使行南蠻府事。



 七年,彥之北伐,請為右軍錄事。及彥之敗退,承天以才非軍旅,得免刑責。



 以補尚書殿中郎,兼左丞。吳興餘杭民薄道舉為劫。制同籍期親補兵。道舉從弟代公、道生等並為大功親,非應在補謫之例,法以代公等母存為期親,則子宜隨母補
 兵。承天議曰:「尋劫制,同籍期親補兵,大功不在此例。婦人三從,既嫁從夫,夫死從子。今道舉為劫,若其叔尚存,制應補謫,妻子營居,固其宜也。但為劫之時,叔父已沒,代公、道生並是從弟,大功之親,不合補謫。今若以叔母為期親,令代公隨母補兵,既違大功不謫之制,又失婦人三從之道。由於主者守期親之文,不辨男女之異,遠嫌畏負,以生此疑,懼非聖朝恤刑之旨。謂代公等母子並宜見原。」



 故司徒掾孔邈奏事未御,邈已喪殯,議者謂不宜
 仍用邈名,更以見官奏之。承天又議曰:「既沒之名不合奏者,非有它義,正嫌於近不祥耳。奏事一卻,動經歲時,盛明之世,事從簡易,曲嫌細忌,皆應蕩除。」



 承天為性剛愎,不能屈意朝右,頗以所長侮同列,不為僕射殷景仁所平,出為衡陽內史。昔在西與士人多不協,在郡又不公清,為州司所糾,被收繫獄,值赦免。



 十六年,除著作佐郎,撰國史。承天年已老,而諸佐郎並名家年少,潁川荀伯子嘲之,常呼為奶母。承天曰:「卿當云鳳凰將九子,奶母
 何言邪!」尋轉太子率更令,著作如故。



 時丹陽丁況等久喪不葬,承天議曰:「禮所云還葬,當謂荒儉一時,故許其稱財而不求備。丁況三家,數年中,葬輒無棺櫬,實由淺情薄恩,同於禽獸者耳。竊以為丁寶等同伍積年,未嘗勸之以義,繩之以法。十六年冬,既無新科,又未申明舊制,有何嚴切,欻然相糾。或由鄰曲分爭,以興此言。如聞在東諸處,此例既多,江西淮北尤為不少。若但謫此三人,殆無整肅。開其一端,則互相恐動,里伍縣司,競為姦
 利。財賂既逞,獄訟必繁,懼虧聖明烹鮮之美。臣愚謂況等三家,且可勿問,因此附定制旨,若民人葬不如法,同伍當即糾言,三年除服之後,不得追相告列,於事為宜。」



 十九年,立國子學,以本官領國子博士。皇太子講《孝經》,承天與中庶子顏延之同為執經。頃之,遷御史中丞。時索虜侵邊,太祖訪群臣威戎御遠之略,承天上表曰:伏見北籓上事,虜犯青、兗,天慈降鑑,矜此黎元,博逮群策,經綸戎政,臣以愚陋,預聞訪及。竊尋獫狁告難,爰自上
 古,有周之盛,南仲出車,漢氏方隆,衛、霍宣力。雖飲馬瀚海,揚旍祁連,事難役繁,天下騷動,委興負海,貲及舟車。



 凶狡倔強,未肯受弱,得失報復,裁不相補。宣帝末年,值其乖亂,推亡固存,始獲稽服。自晉喪中原,戎狄侵擾,百餘年間,未暇以北虜為念。大宋啟祚,兩耀靈武,而懷德畏威,用自款納。陛下臨御以來,羈縻遵養,十餘年中,貢譯不絕。去歲三王出鎮,思振遠圖,獸心易駭,遂生猜懼,背違信約,深構攜隙。貪禍恣毒,無因自反,恐烽燧之警,
 必自此始。臣素庸懦,才不經武,率其管窺,謹撰《安邊論》。意及淺末,懼無可採。若得詢之朝列,辨覈同異,庶或開引群慮,研盡眾謀,短長畢陳,當否可見。其論曰:漢世言備匈奴之策,不過二科,武夫盡征伐之謀,儒生講和親之約,課其所言,互有遠志。加塞漠之外,胡敵掣肘,必未能摧鋒引日,規自開張。當由往年冀土之民,附化者眾,二州臨境,三王出籓,經略既張,宏圖將舉,士女延望,華、夷慕義。故昧於小利,且自矜侈,外示餘力,內堅偽眾。今
 若務存遵養,許其自新,雖未可羈致北闕,猶足鎮靜邊境。然和親事重,當盡廟算,誠非愚短,所能究言。若追蹤衛、霍瀚海之志,時事不等,致功亦殊。寇雖習戰未久,又全據燕、趙,跨帶秦、魏,山河之險,終古如一。自非大田淮、泗,內實青、徐,使民有贏儲,野有積穀,然後分命方、召,總率虎旅,精卒十萬,使一舉盪夷,則不足稍勤王師,以勞天下。何以言之?今遺黎習亂,志在偷安,非皆恥為左衽,遠慕冠冕,徒以殘害剝辱,視息無寄,故繦負歸國,先後
 相尋。虜既不能校勝循理,攻城略地,而輕兵掩襲,急在驅殘,是其所以速怨召禍,滅亡之日。今若遣軍追討,報其侵暴,大翦幽、冀,屠城破邑,則聖朝愛育黎元,方濟之以道。若但欲撫其歸附,伐罪弔民,則駿馬奔走,不肯來征,徒興巨費,無損於彼。復奇兵深入,殺敵破軍,茍陵患未盡,則困獸思鬥,報復之役,將遂無已。斯秦、漢之末策,輪臺之所悔也。



 安邊固守,於計為長。臣以安邊之計,備在史策,李牧言其端,嚴尤申其要,大略舉矣。曹、孫之霸,
 才均智敵,江、淮之間,不居各數百里。魏舍合肥,退保新城,江陵移民南涘,濡須之戍,家停羨溪。及表陵之屯,民夷散雜,晉宣王以為宜從江南以北岸,曹爽不許,果亡柤中,此皆前代之殷鑒也。何者?斥候之郊,非畜牧之地,非耕桑之邑。故堅壁清野,以俟其來,整甲繕兵,以乘其敝。雖時有古今,勢有強弱,保民全境,不出此塗。要而歸之有四:一曰移遠就近;二曰浚復城隍;三曰纂偶車牛;四曰計丁課仗。良守疆其土田,驍帥振其風略。搜獵宣
 其號令,俎豆訓其廉恥。縣爵以縻之,設禁以威之。徭稅有程,寬猛相濟。比及十載,民知義方。然後簡將授奇,揚旌雲朔,風卷河冀,電掃嵩恒,燕弧折卻,代馬摧足,秦首斬其右臂,吳蹄絕其左肩,銘功於燕然之阿,饗徒於金微之曲。



 寇雖亂亡有征,昧弱易取,若天時人事,或未盡符,抑銳俟機,宜審其算。若邊戍未增,星居布野,勤惰異教,貧富殊資,疆場之民,多懷彼此,虜在去就,不根本業,難可驅率,易在振蕩。又狡虜之性,食肉衣皮,以馳騁為
 儀容,以游獵為南畝,非有車輿之安,宮室之衛。櫛風沐雨,不以為勞;露宿草寢,維其常性;勝則競利,敗不羞走,彼來或驟,而此已奔疲。且今春踰濟,既獲其利,乘勝忸心犬,未虞天誅,比及秋末,容更送死。猋騎蟻聚,輕兵鳥集,並踐禾稼,焚爇閭井,雖邊將多略,未審何以禦之。若盛師連屯,廢農必眾,馳車奔馹,起役必遲,散金行賞,損費必大,換土客戍,怨曠必繁。孰若因民所居,並修農戰,無動眾之勞,有扞衛之實,其為利害,優劣相縣也。



 一曰移遠
 就近,以實內地。今青、兗舊民,冀州新附,在界首者二萬家,此寇之資也。今悉可內徙,青州民移東萊、平昌、北海諸郡,泰山以南,南至下邳,左沭右沂,田良野沃,西阻蘭陵,北扼大峴,四塞之內,其號險固。民性重遷,暗於圖始,無虜之時,喜生咨怨。今新被鈔掠,餘懼未息,若曉示安危,居以樂土,宜其歌抃就路,視遷如歸。



 二曰浚復城隍,以增阻防。舊秋冬收斂,民人入保,所以警備暴客,使防衛有素也。古之城池,處處皆有,今雖頹毀,猶可修治。
 粗計戶數,量其所容,新徙之家,悉著城內,假其經用,為之閭伍,納稼築場,還在一處。婦子守家,長吏為師,丁夫匹婦,春夏佃牧。寇至之時,一城千室,堪戰之士,不下二千,其餘羸弱,猶能登陴鼓噪。十則圍之,兵家舊說,戰士二千,足抗群虜三萬矣。



 三曰纂偶車牛,以飾戎械。計千家之資,不下五百耦牛,為車五百兩。參合鉤連,以衛其眾。設使城不可固,平行趨險,賊所不能干。既已族居,易可檢括。號令先明,民知夙戒。有急徵發,信宿可聚。



 四曰
 計丁課仗,勿使有闕。千家之邑,戰士二千,隨其便能,各自有仗,素所服習,銘刻由己,還保輸之於庫,出行請以自衛。弓干利鐵,民不辦得者,官以漸充之,數年之內,軍用粗備矣。



 臣聞軍國異容,施於封畿之內;兵農並修,在於疆場之表。攻守之宜,皆因其習,任其怯勇。山陵川陸之形,寒暑溫涼之氣,各由本性,易則害生。是故戍申作師,遠屯清濟,功費既重,嗟怨亦深。以臣料之,未若即用彼眾之易也。管子治齊,寄令在民;商君為秦,設以耕戰。
 終申威定霸,行其志業,非茍任強,實由有數。



 梁用走卒,其邦自滅;齊用技擊,厥眾亦離。漢、魏以來,茲制漸絕,搜田非復先王之禮,治兵徒逞耳目之欲,有急之日,民不知戰,至乃廣延賞募,奉以厚秩,發遽奔救,天下騷然。方伯刺史,拱手坐聽,自無經略,唯望朝廷遣軍,此皆忘戰之害,不教之失也。



 今移民實內,浚治城隍,族居聚處,課其騎射,長吏簡試,差品能不,甲科上第,漸就優別,明其勳才,表言州郡。如此則屯部有常,不遷其業。內護老弱,
 外通官塗,朋曹素定,同憂等樂,情由習親,藝因事著,晝戰見貌足相識,夜戰聞聲足相救,斯教戰之一隅,先哲之遺術。論者必以古城荒毀,難可修復。今不謂頓便加功,整麗如舊,但欲先定民,營其閭術,墉壑存者,因而即之,其有毀缺,權時柵斷。足以禦彼輕兵,防遏游騎,假以方將,漸就只立。車牛之賦,課仗之宜,攻守所資,軍國之要,今因民所利,導而率之。耕農之器,為府庫之寶,田蠶之氓,兼城之用,千家總倍旅之兵,萬戶具全軍之眾,兵
 強而敵不戒,國富而民不勞,比於優復隊伍,坐食廩糧者,不可同年而校矣。



 今承平未久,邊令弛縱,弓竿利鐵,既不都斷,往歲棄甲,垂二十年,課其所住,理應消壞。謂宜申明舊科,嚴加禁塞,諸商賈往來,幢隊挾藏者,皆以軍法治之。又界上嚴立關候,杜廢間蹊。城保之境,諸所課仗,並加雕鐫,別造程式。若有遺鏃亡刃,及私為竊盜者,皆可立驗,於事為長。又鉅野湖澤廣大,南通洙、泗,北連青、齊,有舊縣城,正在澤內。宜立式修復舊堵,利其埭
 遏,給輕艦百艘。寇若入境,引艦出戰,左右隨宜應接,據其師津,毀其航漕。此以利制車,運我所長,亦微徹敵之要也。



 承天素好弈棋,頗用廢事。太祖賜以局子,承天奉表陳謝,上答:「局子之賜,何必非張武之金邪!」承天又能彈箏,上又賜銀裝箏一面。承天與尚書左丞謝元素不相善,二人競伺二臺之違,累相糾奏。太尉江夏王義恭歲給資費錢三千萬,布五萬匹,米七萬斛。義恭素奢侈,用常不充,二十一年,逆就尚書換明年資費。而舊制出
 錢二十萬,布五百匹以上,並應奏聞,元輒命議以錢二百萬給太尉。事發覺,元乃使令史取僕射孟顗命。元時新除太尉咨議參軍,未拜,為承天所糾。上大怒,遣元長歸田里,禁錮終身。元時又舉承天賣茭四百七十束與官屬,求貴價。承天坐白衣領職。元字有宗,陳郡陽夏人,臨川內史靈運從祖弟也。以才學見知,卒於禁錮。



 二十四年,承天遷廷尉,未拜,上欲以為吏部,已受密旨,承天宣漏之,坐免官。卒於家,年七十八。先是,《禮論》有八百卷,
 承天刪減并合,以類相從,凡為三百卷,并《前傳》、《雜語》、《纂文》、論並傳於世。又改定《元嘉歷》,語在《律歷志》。



 史臣曰:治邊之術,前世言之詳矣。夫戎夷狡黠,飄迅難虞,必宜完其障塞,謹其烽柝,使來徑可防,去塗易梗,然後乃能禁暴止姦,養威攘寇。漢世案秦舊迹,嚴塞以限外夷,吳、魏交戰,亦以江、淮為疆場,莫不先憑地險,卻保民和,且守且耕,伺隙乘釁。高祖受命,王略未遠,雖綿河作守,而兵孤援闊,盛衰既兆,用啟戎心。蓋由王業始基,
 經創多闕,先內後外,以至於此乎。自茲以降,分青置境,無圍守之宜,闕耕戰之略,恃寇不來,遂無其備。周、漢二策,在宋頓亡,遂致胡馬橫行,曾無籓落之固,使士民跼蒼天,蹐厚地,系虜俘囚,而無所控告,哀哉!



 承天《安邊論》,博而篤矣,載之云爾。



\end{pinyinscope}