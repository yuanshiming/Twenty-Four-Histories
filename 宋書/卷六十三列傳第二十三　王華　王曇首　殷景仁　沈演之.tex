\article{卷六十三列傳第二十三 王華 王曇首 殷景仁 沈演之}

\begin{pinyinscope}

 王華,字子陵,瑯邪臨沂
 人,太保弘從祖弟也。祖薈,衛將軍,會稽內史。父,廞,太子中庶子,司徒左長史。居在吳,晉隆安初,王恭起兵討王國寶,時廞丁母憂在家,恭檄令
 起兵,廞即聚眾應之,以女為貞烈將軍,以女人為官屬。國寶既死,恭檄廞罷兵。廞起兵之際,多所誅戮,至是不復得已,因舉兵以討恭為名。恭遣劉牢之擊廞,廞敗走,不知所在。長子泰為恭所殺。華時年十三,在軍中,與廞相失,隨沙門釋曇永逃竄。時牢之搜檢覓華甚急,曇永使華提衣襆隨後,津邏咸疑焉。華行遲,永呵罵云:「奴子怠懈,行不及我!」以杖捶華數十,眾乃不疑,由此得免。



 遇赦還吳。



 少有志行,以父存亡不測,布衣蔬食不交游,如此十餘
 年,為時人所稱美。高祖欲收其才用,乃發廞喪問,使華制服。服闋,高祖北伐長安,領鎮西將軍、北徐州刺史,辟華為州主簿,仍轉鎮西主簿,治中從事史,歷職著稱。太祖鎮江陵,以為西中郎主簿,遷咨議參軍,領錄事。太祖進號鎮西,復隨府轉。太祖未親政,政事悉委司馬張邵。華性尚物,不欲人在己前;邵性豪,每行來常引夾轂,華出入乘牽車,從者不過二三以矯之。嘗於城內相逢,華陽不知是邵,謂左右:「此鹵簿甚盛,必是殿下出行。」乃下
 牽車,立於道側;及邵至,乃驚。邵白服登城,為華所糾,坐被徵;華代為司馬、南郡太守,行府州事。



 太祖入奉大統,以少帝見害,疑不敢下。華建議曰:「羨之等受寄崇重,未容便敢背德,廢主若存,慮其將來受禍,致此殺害。蓋由每生情多,寧敢一朝頓懷逆志。且三人勢均,莫相推伏,不過欲握權自固,以少主仰待耳。今日就徵,萬無所慮。」太祖從之,留華總後任。上即位,以華為侍中,領驍騎將軍,未拜,轉右衛將軍,侍中如故。



 先是,會稽孔寧子為太
 祖鎮西咨議參軍,以文義見賞,至是為黃門侍郎,領步兵校尉。寧子先為高祖太尉主簿,陳損益曰:「隆化之道,莫先於官得其才;枚卜之方,莫若人慎其舉。雖復因革不同,損益有物,求賢審官,未之或改。師錫僉曰,煥乎欽明之誥,拔茅征吉,著於幽《賁》之爻。晉師有成,瓜衍作賞,楚乘無入,蒍賈不賀。今舊命惟新,幽人引領,《韶》之盡美,已備於振綱;《武》之未盡,或存於理目。雖九官之職,未可備舉,親民之選,尤宜在先。愚欲使天朝四品官,外及守
 牧,各舉一人堪為二千石長吏者,以付選官,隨缺敘用,得賢受賞,失舉任罰。夫惟帝之難,豈庸識所易,然舉爾所知,非求多人,因百官之明,孰與一識之見,執咎在己,豈容徇物之私。今非以選曹所銓,果於乖謬,眾職所舉,必也惟良,蓋宜使求賢闢其廣塗,考績取其少殿。若才實拔群,進宜尚德,治阿之宰,不必計年,免徒之守,豈限資秩。自此以還,故當才均以資,資均以地。宰蒞之官,誠曰吏職,然監觀民瘼,翼化宣風,則隱厚之求,急於刀筆,
 能事之功,接於德心,以此論才,行之年歲,豈惟政無秕蠹,民庇手足而已,將使公路日清,私請漸塞。士多心競,仁必由己,處士砥自求之節,仕子藏交馳之情。寧子庸微,不識治體,冒昧陳愚,退懼違謬。」



 寧子與華並有富貴之願,自羨之等秉權,日夜構之於太祖。寧子嘗東歸,至金昌亭,左右欲泊船,寧子命去之,曰:「此弒君亭,不可泊也。」華每閑居諷詠,常誦王粲《登樓賦》曰:「冀王道之一平,假高衢而騁力。」出入逢羨之等,每切齒憤吒,歎曰:「當見
 太平時不?」元嘉二年,寧子病卒。三年,誅羨之等,華遷護軍,侍中如故。



 宋世惟華與南陽劉湛不為飾讓,得官即拜,以此為常。華以情事異人,未嘗預宴集,終身不飲酒,有燕不之詣。若宜有論事者,乘車造門,主人出車就之。及王弘輔政,而弟曇首為太祖所任,與華相埒,華嘗謂己力用不盡,每歎息曰:「宰相頓有數人,天下何由得治!」四年,卒,時年四十三。追贈散騎常侍、衛將軍。九年,上思誅羨之之功,追封新建縣侯,食邑千戶,謚曰宣侯。世祖
 即位,配饗太祖廟庭。



 子定侯嗣,官至左衛將軍,卒。子長嗣,太宗泰始二年,坐罵母奪爵,以長弟終紹封。後廢帝元徽三年,終上表乞以封還長,許之。齊受禪,國除。華從父弟鴻,五兵尚書,會稽太守。



 王曇首,琅邪臨沂人,太保弘少弟也。幼有業尚,除著作郎,不就。兄弟分財,曇首唯取圖書而已。辟琅邪王大司馬屬,從府公脩復洛陽園陵。與從弟球俱詣高祖,時謝晦在坐,高祖曰:「此君並膏粱盛德,乃能屈志戎旅。」曇首
 答曰:「既從神武之師,自使懦夫有立志。」晦曰:「仁者果有勇。」高祖悅。行至彭城,高祖大會戲馬臺,豫坐者皆賦詩;曇首文先成,高祖覽讀,因問弘曰:「卿弟何如卿?」



 弘答曰:「若但如民,門戶何寄。」高祖大笑。曇首有識局智度,喜慍不見於色,閨門之內,雍雍如也。手不執金玉,婦女不得為飾玩,自非祿賜所及,一毫不受於人。



 太祖為冠軍、徐州刺史,留鎮彭城,以曇首為府功曹。太祖鎮江陵,自功曹為長史,隨府轉鎮西長史。高祖甚知之,謂太祖曰:「王
 曇首,沈毅有器度,宰相才也。汝每事咨之。」景平中,有龍見西方,半天騰上,廕五綵雲,京都遠近聚觀,太史奏曰:「西方有天子氣。」太祖入奉大統,上及議者皆疑不敢下,曇首與到彥之、從兄華固勸,上猶未許。曇首又固陳,并言天人符應,上乃下。率府州文武嚴兵自衛,臺所遣百官眾力,不得近部伍,中兵參軍朱容子抱刀在平乘戶外,不解帶者數旬。既下在道,有黃龍出負上所乘舟,左右皆失色,上謂曇首曰:「此乃夏禹所以受天命,我何堪
 之。」及即位,又謂曇首曰:「非宋昌獨見,無以致此。」以曇首為侍中,尋領右衛將軍,領驍騎將軍。以朱容子為右軍將軍。誅徐羨之等,平謝晦,曇首及華之力也。



 元嘉四年,車駕出北堂,嘗使三更竟開廣莫門,南臺云:「應須白虎幡,銀字棨。不肯開門。尚書左丞羊玄保奏免御史中丞傅隆以下,曇首繼啟曰:「既無墨敕,又闕幡棨,雖稱上旨,不異單刺。元嘉元年、二年,雖有再開門例,此乃前事之違。



 今之守舊,未為非禮。但既據舊史,應有疑卻本末,曾無此
 狀,猶宜反咎其不請白虎幡、銀字棨,致門不時開,由尚書相承之失,亦合糾正。」上特無所問,更立科條。遷太子詹事,侍中如故。



 晦平後,上欲封曇首等,會宴集,舉酒勸之,因拊御床曰:「此坐非卿兄弟,無復今日。」時封詔已成,出以示曇首,曇首曰:「近日之事,釁難將成,賴陛下英明速斷,故罪人斯戮。臣等雖得仰憑天光,效其毫露,豈可因國之災,以為身幸。



 陛下雖欲私臣,當如直史何?」上不能奪,故封事遂寢。



 時兄弘錄尚書事,又為揚州刺史,曇
 首為上所親委,任兼兩宮。彭城王義康與弘並錄,意常怏怏,又欲得揚州,形於辭旨。以曇首居中,分其權任,愈不悅。曇首固乞吳郡,太祖曰:「豈有欲建大廈而遺其棟梁者哉?賢兄比屢稱疾,固辭州任,將來若相申許者,此處非卿而誰?亦何吳郡之有。」時弘久疾,屢遜位,不許。義康謂賓客曰:「王公久疾不起,神州詎合臥治。」曇首勸弘減府兵力之半以配義康,義康乃悅。



 七年,卒。太祖為之慟,中書舍人周赳侍側,曰:「王家欲衰,賢者先殞。」



 上曰:「直是
 我家衰耳。」追贈左光祿大夫,加散騎常侍,詹事如故。九年,以預誅羨之等謀,追封豫寧縣侯,邑千戶,謚曰文侯。世祖即位,配饗太祖廟庭。子僧綽嗣,別有傳。少子僧虔,升明末,為尚書令。



 殷景仁,陳郡長平人也。曾祖融,晉太常。祖茂,散騎常侍、特進、左光祿大夫。父道裕,蚤亡。景仁少有大成之量,司徒王謐見而以女妻之。初為劉毅後軍參軍,高祖太尉行參軍。建議宜令百官舉才,以所薦能否為黜陟。遷宋
 臺祕書郎,世子中軍參軍,轉主簿,又為驃騎將軍道憐主簿。出補衡陽太守,入為宋世子洗馬,仍轉中書侍郎。景仁學不為文,敏有思致,口不談義,深達理體;至於國典朝儀,舊章記注,莫不撰錄,識者知其有當世之志也。高祖甚知之,遷太子中庶子。



 少帝即位,入補侍中,累表辭讓,又固陳曰:「臣志乾短弱,歷著出處。值皇塗隆泰,身荷恩榮,階牒推遷,日月頻積,失在饕餮,患不自量。而奉聞今授,固守愚心者,竊惟殊次之寵,必歸器望;喉脣之
 任,非才莫居。三省諸躬,無以克荷,豈可茍順甘榮,不知進退,上虧朝舉,下貽身咎,求之公私,未見其可。顧涯審分,誠難庶幾,踰方越序,易以誡懼。所以俯仰周偟,無地寧處。若惠澤廣流,蘭艾同潤,回改前旨,賜以降階,雖實不敏,敢忘循命。臣迕違之愆,既已屢積,寧當徒尚浮采,塵黷天聽。丹情悾款,仰希照察。」詔曰:「景仁退挹之懷,有不可改,除黃門侍郎,以申君子之請。」尋領射聲。頃之,轉左衛將軍。



 太祖即位,委遇彌厚,俄遷侍中,左衛如故。時
 與侍中右衛將軍王華、侍中驍騎將軍王曇首、侍中劉湛四人,並時為侍中,俱居門下,皆以風力局幹,冠冕一時,同升之美,近代莫及。元嘉三年,車駕征謝晦,司徒王弘入居中書下省,景仁長直,共掌留任。晦平,代到彥之為中領軍,侍中如故。



 太祖所生章太后早亡,上奉太后所生蘇氏甚謹。六年,蘇氏卒,車駕親往臨哭,下詔曰:「朕夙罹偏罰,情事兼常,每思有以光隆懿戚,少申罔極之懷。而禮文遺逸,取正無所,監之前代,用否又殊,故惟疑
 累年,在心未遂。蘇夫人奄至傾殂,情禮莫寄,追思遠恨,與事而深,日月有期,將卜窀穸,便欲粗依《春秋》以貴之義,式遵二漢推恩之典。但動藉史筆,傳之後昆,稱心而行,或容未允。可時共詳論,以求其中。執筆永懷,益增感塞。」景仁議曰:「至德之感,靈啟厥祥,文母伣天,實熙皇祚。主上聿遵先典,號極徽崇,以貴之義,禮盡於此。蘇夫人階緣戚屬,情以事深,寒泉之思,實感聖懷,明詔爰發,詢求厥中。謹尋漢氏推恩加爵,于時承秦之弊,儒術蔑如,
 自君作故,罔或前典,懼非盛明所宜軌蹈。晉監二代,朝政之所因,君舉必書,哲王之所慎。體至公者,懸爵賞於無私;奉天統者,每屈情以申制。所以作孚萬國,貽則後昆。臣豫蒙博逮,謹露庸短。」上從之。



 丁母憂,葬竟,起為領軍將軍,固辭。上使綱紀代拜,遣中書舍人周赳輿載還府。九年,服闋,遷尚書僕射。太子詹事劉湛代為領軍,與景仁素善,皆被遇於高祖,俱以宰相許之。湛尚居外任,會王弘、華、曇首相係亡,景仁引湛還朝,共參政事。湛既
 入,以景仁位遇本不踰己,而一旦居前,意甚憤憤。知太祖信仗景仁,不可移奪,乃深結司徒彭城王義康,欲倚宰相之重以傾之。



 十二年,景仁復遷中書令,護軍、僕射如故。尋復以僕射領吏部,護軍如故。



 湛愈忿怒。義康納湛言,毀景仁於太祖;太祖遇之益隆。景仁對親舊歎曰:「引之令入,入便噬人。」乃稱疾解職,表疏累上,不見許,使停家養病。發詔遣黃門侍郎省疾。湛議遣人若劫盜者於外殺之,以為太祖雖知,當有以,終不能傷至親之愛。



 上微聞之,遷景仁於西掖門外晉鄱陽主第,以為護軍府,密邇宮禁,故其計不行。



 景仁臥疾者五年,雖不見上,而密表去來,日中以十數;朝政大小,必以問焉,影跡周密,莫有窺其際者。收湛之日,景仁使拂拭衣冠,寢疾既久,左右皆不曉其意。其夜,上出華林園延賢堂召景仁,猶稱腳疾,小床輿以就坐,誅討處分,一皆委之。



 代義康為揚州刺史,僕射領吏部如故。遣使者授印綬,主簿代拜,拜畢,便覺其情理乖錯。性本寬厚,而忽更苛暴,問左
 右曰:「今年男婚多?女嫁多?」是冬大雪,景仁乘輿出聽事觀望,忽驚曰:「當閣何得有大樹?」既而曰:「我誤邪?」



 疾轉篤。太祖謂不利在州司,使還住僕射下省,為州凡月餘卒。或云見劉湛為祟。



 時年五十一,追贈侍中、司空,本官如故。謚曰文成公。



 上與荊州刺史衡陽王義季書曰:「殷僕射疾患少日,奄忽不救。其識具經遠,奉國竭誠,周游繾綣,情兼常痛。民望國器,遇之為難,惋歎之深,不能已已。汝亦同不?往矣如何!」世祖大明五年,行幸經景仁墓,詔曰:「
 司空文成公景仁德量淹正,風識明允,徽績忠謨,夙達先照,惠政茂譽,實留民屬。近瞻丘墳,感往興悼,可遣使致祭。」



 子道矜,幼而不慧,官至太中大夫。道矜子恆,太宗世為侍中,度支尚書,屬父疾積久,為有司所奏。詔曰:「道矜生便有病,無更橫疾。恆因愚習惰,久妨清序,可降為散騎常侍。」



 沈演之,字臺真,吳興武康人也。高祖充,晉車騎將軍,吳國內史。曾祖勁,冠軍陳祐長史,戍金墉城,為鮮卑慕容
 恪所陷,不屈節,見殺,追贈東陽太守。祖赤黔,廷尉卿。父叔任,少有幹質,初為揚州主簿,高祖太尉參軍,吳、山陰令,治皆有聲。硃齡石伐蜀,為齡石建威府司馬,加建威將軍。平蜀之功,亞於元帥,即本號為西夷校尉、巴西梓潼郡太守,戍涪城。東軍既反,二郡強宗侯勱、羅奧聚眾作亂,四面雲合,遂至萬餘人,攻城急。叔任東兵不滿五百,推布腹心,眾莫不為用,出擊大破之,逆黨皆平。高祖討司馬休之,齡石遣叔任率軍來會。時高祖領鎮西將
 軍,命為司馬。及軍還,以為揚州別駕從事史。以平蜀全涪之功,封寧新縣男,食邑四百四十戶。出為建威將軍、益州刺史,以疾還都。義熙十四年,卒,時年五十。長子融之,蚤卒。



 演之年十一,尚書僕射劉柳見而知之,曰:「此童終為令器。」家世為將,而演之折節好學,讀《老子》日百遍,以義理業尚知名。襲父別爵吉陽縣五等侯。郡命主簿,州辟從事史,西曹主簿,舉秀才,嘉興令,有能名。入為司徒祭酒,南譙王義宣左軍主簿,錢唐令,復有政績。復為
 司徒主簿。丁母憂。起為武康令,固辭不免,到縣百許日,稱疾去官。服闋,除司徒左西掾,州治中從事史。



 元嘉十二年,東諸郡大水,民人饑饉,吳義興及吳郡之錢唐,升米三百。以演之及尚書祠部郎江邃並兼散騎常侍,巡行拯恤,許以便宜從事。演之乃開倉廩以賑饑民,民有生子者,口賜米一斗,刑獄有疑枉,悉制遣之,百姓蒙賴。轉別駕從事史,領本郡中正,深為義康所待,故在府州前後十餘年。後劉湛、劉斌等結黨,欲排廢尚書僕射殷
 景仁,演之雅仗正義,與湛等不同,湛因此讒之於義康。嘗因論事不合旨,義康變色曰:「自今而後,我不復相信!」演之與景仁素善,盡心於朝庭,太祖甚嘉之,以為尚書吏部郎。



 十七年,義康出籓,誅湛等,以演之為右衛將軍。景仁尋卒,乃以後軍長史範曄為左衛將軍,與演之對掌禁旅,同參機密。二十年,遷侍中,右衛將軍如故。太祖謂之曰:「侍中領衛,望實優顯,此蓋宰相便坐,卿其勉之。」上欲伐林邑,朝臣不同,唯廣州刺史陸徽與演之贊成
 上意。及平,賜群臣黃金、生口、銅器等物,演之所得偏多。上謂之曰:「廟堂之謀,卿參其力,平此遠夷,未足多建茅土。廓清京都,鳴鸞東岱,不憂河山不開也。」二十一年,詔曰:「總司戎政,翼贊東朝,惟允之舉,匪賢莫授。侍中領右衛將軍演之,清業貞審,器思沈濟。右衛將軍曄,才應通敏,理懷清要。並美彰出內,誠亮在公,能克懋厥猷,樹績所蒞。演之可中領軍,曄可太子詹事。」曄懷逆謀,演之覺其有異,言之太祖,曄尋事發伏誅。遷領國子祭酒,本州
 大中正,轉吏部尚書,領太子右衛率。雖未為宰相,任寄不異也。



 素有心氣,疾病歷年,上使臥疾治事。性好舉才,申濟屈滯,而謙約自持,上賜女伎,不受。二十六年,車駕拜京陵,演之以疾不從。上還宮,召見,自勉到坐,出至尚書下省,暴卒,時年五十三。太祖痛惜之,追贈散騎常侍、金紫光祿大夫,謚曰貞侯。



 演之昔與同使江邃字玄遠,濟陽考城人。頗有文義。官至司徒記室參軍,撰《文釋》,傳於世。演之子睦,至黃門郎,通直散騎常侍。世祖大明初,
 坐要引上左右俞欣之訪評殿省內事,又與弟西陽王文學勃忿鬩不睦,坐徙始興郡,勃免官禁錮。



 勃好為文章,善彈琴,能圍棋,而輕薄逐利。歷尚書殿中郎。太宗泰始中,為太子右衛率,加給事中。時欲北討,使勃還鄉里募人,多受貨賄。上怒,下詔曰:「沈勃琴書藝業,口有美稱,而輕躁耽酒,幼多罪愆。比奢淫過度,妓女數十,聲酣放縱,無復劑限。自恃吳興土豪,比門義故,脅說士庶,告索無已。又輒聽募將,委役還私,託注病叛,遂有數百。周旋
 門生,競受財貨,少者至萬,多者千金,考計臟物,二百餘萬,便宜明罰敕法,以正典刑。故光祿大夫演之昔受深遇,忠績在朝,尋遠矜懷,能無弘律,可徙勃西垂,令一思愆悔。」於是徙付梁州。廢帝元徽初,以例得還。結事阮佃夫、王道隆等,復為司徒左長史。為廢帝所誅。順帝即位,追贈本官。



 勃弟統,大明中為著作佐郎。先是,五省官所給幹僮,不得雜役,太祖世,坐以免官者,前後百人。統輕役過差,有司奏免。世祖詔曰:「自頃幹僮,多不祗給,主可量
 聽行杖。」得行幹杖,自此始也。



 演之兄融之子暢之,襲寧新縣男。大明中,為海陵王休茂北中郎咨議參軍,為休茂所殺,追贈黃門郎。子曄嗣,齊受禪,國除。



 史臣曰:元嘉初,誅滅宰相,蓋王華、孔寧子之力也。彼群公義雖往結,恩實今疏,而任即曩權,意非昔主,居上六之窮爻,當來寵之要轍,顛覆所基,非待他釁,況於廢殺之重,其隙易乘乎!夫殺人而取其璧,不知在己興累;傾物而移其寵,不忌自我難持。若二子永年,亦未知來禍
 所止也。有能戒彼而悟此,則所望於來哲。



\end{pinyinscope}