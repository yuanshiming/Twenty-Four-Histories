\article{卷二十五志第十五 天文三}

\begin{pinyinscope}

 晉簡文咸安元年十二月辛卯,熒惑逆行入太微,二年
 三月猶不退。占曰:「國不安,有憂。」是時,帝有桓溫之逼,恆懷憂慘。七月,帝崩。



 咸安二年正月己酉,歲星犯填星,在須女。占曰:「為內亂。」五月,歲星形色如太白。占曰:「進退如度,奸邪息。變色亂行,主無福。歲星囚於仲夏,當細小而不明,此其失常也;又為臣強。」六月,太白晝見在七星。乙酉,太白犯輿鬼。



 占曰:「國有憂。」七月,帝疾甚,詔桓溫
 曰:「
 少子
 可輔者輔之;如不可,君自取之。」賴侍中王坦之毀手詔,改使如王導輔政故事。溫聞之大怒,將誅坦之等,內亂之應也。是月,帝崩。咸安二年五月丁未,太白犯天關。占曰:「兵起。」六月,庾希入
 京城。十一月,盧悚入宮,並誅滅。



 晉孝武寧康元年正月戊申,月奄心大星。案占,災不在王者,則在豫州。一曰:「主命惡之。」三月丙午,月奄南斗第五星。占曰:「大臣有憂,憂死亡。」一曰:「將軍死。」七月,桓溫薨。



 寧康二年正月丁巳,有星孛于女虛,經氐、亢、角、軫、翼、張。九月丁丑,有星孛于天市。十一月癸酉,太白奄熒惑,在營室。占曰:「金火合為爍,此災皆為兵喪。」太元元年五月,氐賊苻堅伐涼州。七月,氐破涼州,虜張天錫。十一月,桓
 沖發三州軍軍淮、泗,桓豁亦遣軍備境上。寧康二年閏月己未,月奄牽牛南星。



 占曰:「左將軍死。」三年五月,北中郎將王坦之薨。



 寧康三年六月辛卯,太白犯東井。占曰:「秦地有兵。」九月戊申,熒惑奄左執法。占曰:「執法者死。」太元元年,苻堅破涼州。十月,尚書令王彪之卒。



 晉孝武太元元年四月丙戌,熒惑犯南斗第三星。丙申,又奄第四星。占曰:「兵大起,中國飢。」一曰:「有赦。」八月癸酉,
 太白晝見在氐。氐,兗州分野。



 九月,熒惑犯哭泣星,遂入羽林。占曰:「天子有哭泣事,中軍兵起。」十一月己未,月奄左角。占曰:「天子有兵。」一曰:「國有憂。」三年六月,熒惑守羽林。



 占曰:「禁兵大起。」九月壬午,太白晝見在角,兗州分。元年五月,大赦。三年八月,氐賊韋鐘入漢中東下,苻融寇樊、鄧,慕容暐圍襄陽,氐兗州刺史彭超圍彭城。四年二月,襄陽城陷,賊獲朱序。彭超捨彭城,獲吉挹。彭超等聚廣陵三河眾五萬。於是征虜謝石次塗中,右衛毛安之、
 游擊河間王曇之等次堂邑,發丹陽民丁,使尹張涉屯衛京都。六月,兗州刺史謝玄討賊,大破之,餘燼皆走。是時中外連兵,比年荒儉。是年,又發揚州萬人戍夏口。



 太元四年十一月丁巳,太白犯哭星。占曰:「天子有哭泣事。」五年七月丙子,辰星犯軒轅。占曰:「女主當之。」九月癸未,皇后王氏崩。



 太元六年十月乙卯,有奔星東南經翼軫,聲如雷。《星說》曰:「光跡相連曰流,絕跡而去曰奔。」案占:「楚地有兵。」一曰:「
 軍破民流。」十二月,氐荊州刺史梁成、襄陽太守閻震率眾伐竟陵,桓石虔擊大破之,生禽震,斬首七千,獲生萬人。聲如雷,將帥怒之象也。七年九月,朱綽擊襄陽,拔將六百餘家而還。



 太元七年十一月,太白晝見,在斗。占曰:「吳有兵喪。」八年四月甲子,太白又晝見,在參。占曰:「魏有兵喪。」是月,桓沖徵沔漢,楊亮伐蜀,並拔城略地。八月,苻堅自將號百萬,九月,攻沒壽陽。十月,劉牢之破堅將梁成斬之,殺獲萬
 餘人。謝玄等又破堅於淝水,斬其弟融,堅大眾奔潰。九年六月,皇太后褚氏崩。八月,謝玄出屯彭城,經略中州。十年八月,苻堅為其將姚萇所殺。



 太元十年十二月己丑,太白犯歲星。占曰:「為兵飢。」是時河朔未一,兵連在外,冬,大飢。



 太元十一年三月戊申,太白晝見,在東井。占曰:「秦有兵,臣彊。」六月甲午,歲星晝見,在胃。占曰:「魯有兵,臣彊。」十二年,慕容垂寇東阿,翟遼寇河上,姚萇假號安定,苻登自
 立隴上,呂光竊據涼土。太元十一年三月,客星在南斗,至六月乃沒。占曰:「有兵。」一曰:「有赦。」是後司、雍、兗、冀常有兵役。十二年正月,大赦。八月,又赦。



 太元十二年二月戊寅,熒惑入月。占曰:「有亂臣死,相若有戮者。」一曰:「女親為敗,天下亂。」是時琅邪王輔政,王妃從兄國寶以姻暱受寵。又陳郡人袁悅昧私茍進,交遘主相,扇揚朋黨。十三年,帝殺悅。於是主相有隙,亂階興
 矣。



 太元十二年十月庚午,太白晝見,在斗。十三年閏月戊辰,天狗東北下有聲。十二月戊子,辰星入月,在危。占曰:「賊臣欲殺主,不出三年,必有內惡。」是月,熒惑在角亢,形色猛盛。占曰:「熒惑失其常,吏且棄其法,諸侯亂其政。」自是後慕容垂、翟遼、姚萇、苻登、慕容永並阻兵爭彊。十四年正月,彭城妖賊又稱號於皇丘,劉牢之破滅之。三月,張道破合鄉,圍泰山,向欽之擊走之。是年,翟遼又攻沒
 滎陽,侵略陳、項。于時政事多弊,治道陵遲矣。



 太元十四年十二月,熒惑入羽林。乙未,月犯歲星。占並同上。十五年,翟遼陸掠司、兗,眾軍累討弗克。鮮卑又跨略并、冀。七月,旱。八月,諸郡大水,兗州又蝗。



 太元十五年七月壬申,有星孛于北河戒,經太微、三台、文昌,入北斗,長十餘丈。八月戊戌,入紫微,乃滅。占曰:「北河戒,一名胡門。胡門有兵喪。掃太微,入紫微,王者當之。三台為三公,文昌為將相,將相三公有災。入北斗,彊國
 發兵,諸侯爭權,大夫憂。」十一月,太白入羽林。占曰:「天子為軍自守,有反臣。」二十一年九月,孝武帝崩。隆安元年,王恭、殷仲堪、桓玄等並發兵表誅王國寶,朝廷從而殺之,并斬其從弟緒,司馬道子由是失勢,禍亂成矣。



 太元十六年十一月癸巳,月奄心前星。占曰:「太子憂。」是時太子常有篤疾。



 太元十七年九月丁丑,歲星、熒惑、填星同在亢氐。占曰:「三星合,是謂驚位絕行,內外有兵喪與飢,改立王公。」



 太元十八年正月乙酉,熒惑入月。占曰:「憂在宮中,非賊乃盜也。」一曰:「有亂臣,若有戮者。」二十一年九月,帝暴崩內殿,兆庶宣言夫人張氏潛行大逆。



 于時朝政暗緩,不加顯戮,但默責而已。又王國寶邪狡,卒伏其辜。太元十八年二月,有客星在尾中,至九月乃滅。占曰:「燕有兵喪。」十九年四月己巳,月奄歲星,在尾。占曰:「為饑,燕國亡。」二十年,慕容垂遣息寶伐什圭,為圭所破,死者數萬人。二十一年,垂死,國遂衰亡。



 太元十九年十月癸丑,太白犯歲星,在斗。占曰:「為饑,為內兵。斗,吳、越分。」至隆安元年,王恭等舉兵顯王國寶之罪,朝廷赦之。是後連歲水旱民饑。



 太元二十年六月,熒惑入天囷。占曰:「天下飢。」七月丁亥,太白入太微。



 占曰:「太白入太微,國有憂。晝見,為兵喪。」九月,有蓬星如粉絮,東南行,歷女虛至哭星。占曰:「蓬星見,不出三年,必有亂臣戮死於市。」十二月己巳,月犯楗閉及東西咸。占曰:「楗閉司心腹喉舌,東西咸主陰謀。」是時
 王國寶交構朝政。二十一年九月,帝崩;隆安元年,王恭等舉兵,而朝廷戮王國寶、王緒。又連歲水旱,兼三方動眾,民饑。



 太元二十一年三月,太白連晝見,在羽林。占曰:「有強臣,有兵喪,中軍兵起。」四月壬午,太白入天囤。占曰:「為飢。」六月,歲星犯哭星。占曰:「有哭泣事。」是年九月,孝武帝崩。隆安元年,王恭舉兵脅朝廷,於是中外戒嚴,戮王國寶以謝之。



 晉安帝隆安元年正月癸亥,熒惑犯哭星。占曰:「有哭泣事。」二月,歲星熒惑皆入羽林。占曰:「軍兵起。」四月丁丑,太白晝見,在東井。秦有兵喪。是月,王恭舉兵,內外戒嚴。尋殺王國寶等。六月,羌賊攻洛陽,郗恢遣兵救之。姚萇死,子略代立。什圭自號於中山。隆安元年六月庚午,月奄太白,在太微端門外。占曰:「國受兵。」乙酉,月奄歲星,在東壁。占曰:「為饑。衛地有兵。」八月,熒惑守井鉞。占曰:「大臣有誅。」二年六月戊辰,攝提移度
 失常,歲星晝見在胃。胃,兗州分。是年六月,郗恢遣鄧啟方等以萬人殘虜於滑臺。滑臺,衛地也。啟方等敗而還。九月,王恭、庾楷、殷仲堪、桓玄等並舉兵表誅王愉、司馬尚之兄弟。於是內外戒嚴,大發民眾。仲堪軍至尋陽,禽江州刺史王愉,楷將段方攻尚之於楊湖,為所敗,方死。王恭司馬劉牢之反恭,恭敗。桓玄至白石,亦奔退。仲堪還江陵。



 三年冬,荊州刺史殷仲堪為桓玄所殺。



 隆安二年閏月,太白晝見,在羽林。丁丑,月犯東上相。三年五月辛
 酉,月又奄東上相。辛未,辰星犯軒轅星。占悉同上。是年正月,楊佺期破郗恢,奪其任,殷仲堪又殺之。六月,鮮卑攻沒青州。十月,羌賊攻沒洛陽。桓玄破荊、雍,殺殷仲堪、楊佺期。孫恩聚眾攻沒會稽,殺內史王凝之,劉牢之東討走之。四年七月,太皇太后李氏崩。



 隆安四年正月乙亥,月犯填星,在牽牛。占曰:「吳、越有兵喪。女主憂。」



 二月己丑,有星孛於奎,長三丈,上至閣道紫宮西蕃,入斗魁,至三台、太微、帝座、端門。占曰:「彗拂天子
 廷閣,易主之象。」經三台,入北斗,占同上條。六月己未,月又犯填星,在牽牛。辛酉,又犯哭星。十月,奄歲星在北河。占曰:「為饑。」十二月戊寅,有星孛于貫索、天市、天津。占曰:「貴臣獄死,內外有兵喪。天津為賊斷,王道天下不通。」十二月,太白在斗晝見,至五年正月乙卯。



 案占,災在吳、越。三月甲寅,流星赤色眾多,西行經牽牛、虛、危、天津、閣道,貫太微、紫宮。占曰:「星者庶民,類眾多西流之象。徑行天子庭,主弱臣彊,諸侯兵不制。」七月癸亥,大角星散搖五
 色。占曰:「王者流散。」丁卯,月犯天關。



 占曰:「王者憂。」九月庚子,熒惑犯少微,又守之。占曰:「處士誅。」十月戊子,月犯東蕃次相。四年五月,孫恩復破會稽,殺內史謝琰;遣高雅之等討之。七月,太皇太后李氏崩。十月,妖賊大破高雅之於餘姚,死者十七八。五年二月,孫恩攻句章,高祖拒之。五月,吳郡內史袁山松出戰,為所殺,死者數千人。六月,孫恩至京口,高祖擊破之。恩軍蒲洲,於是內外戒嚴,營陣屯守,柵斷淮口。恩遣別將攻廣陵,殺三千餘人。恩遁
 據郁洲。是月,高祖又追破之。九月,桓玄表至,逆旨陵上。十月,司馬元顯大治水軍,將以伐玄。元興元年正月,桓玄東下。是月,孫恩在臨海,人眾餓死散亡,恩亦投水死。盧循自稱征虜將軍,領其餘眾,略有永嘉、晉安之地。二月,帝戎服遣西軍。丁卯,桓玄至姑孰,破歷陽,司馬尚之見殺,劉牢之降于玄。三月,玄克京都,殺司馬元顯,放太傅道子。七月,大饑,人相食。



 浙江東餓死流亡十六七,吳郡、吳興戶口減半;又流奔而西者萬計。十月,桓玄遣
 將擊劉軌,破走奔青州。四年,玄遂篡位,遷帝尋陽。



 晉安帝元興元年三月戊子,太白犯五諸侯,因晝見。四月辛丑,月奄辰星。七月戊寅,熒惑在東井,熒惑犯輿鬼、積尸,占並同上。八月庚子,太白犯歲星,在上將東南。占曰:「楚兵飢。」一曰:「災在上將。」丙寅,太白奄右執法。九月癸未,太白犯進賢。占曰:「賢者誅。」十月,客星色白如粉絮,在太微西,至十二月,入太微。占曰:「兵入天子庭。」二年二月,歲星犯西上將。六月甲辰,奄斗第四星。占曰:「大臣誅,不
 出三年。」八月癸丑,太白犯房北第二星。九月己丑,歲星犯進賢,熒惑犯西上將。十月甲戌,太白犯泣星。十一月丁丑,熒惑犯填星。辛巳,月犯熒惑。十二月乙巳,月奄軒轅第二星,占悉同上。元年冬,索頭破羌軍。二年十二月,桓玄篡位,放遷帝后於尋陽,以永安何皇后為零陵君。三年二月,高祖盡誅桓氏。



 元興三年正月戊戌,熒惑逆行犯太微西上相。占曰:「天子戰於野,上相死。」



 二月甲辰,月奄歲星於左角。占曰:「天
 下兵起。」丙辰,熒惑逆行在左執法西北。



 占曰:「執法者憂。」四月甲午,月奄軒轅第二星,填星入羽林,十二月,熒惑太白皆犯羽林,占同上。是年二月丙辰,高祖殺桓修等。三月己未,破走桓玄,遣軍西討。辛巳,誅左僕射王愉及子荊州刺史綏。桓玄劫帝如江陵。五月,玄下至崢嶸洲,義軍破滅之。桓振又攻沒江陵,幽劫天子。明年正月,眾軍攻之,振走,乘輿乃旋。七月,永安何皇后崩。三月,桓振又襲江陵,荊州刺史司馬休之敗走。是月,劉懷肅擊振
 滅之。其年二月,巴西人譙縱殺益州刺史毛璩及璩弟西夷校尉瑾,跨有西土,自號蜀王。



 晉安帝義熙元年三月壬辰,月奄左執法,占同上。丁酉,月奄心前星。占曰:「豫州有災。」太白犯東井。占曰:「秦有兵。」四月己卯,月犯填星,在東壁。



 占曰:「其地亡國。」一曰:貴人死。」七月庚辰,太白比晝見,在翼、軫。占曰:「為臣強。荊州有兵喪。」己未,月奄填星,在東壁。占曰:「其國以伐亡。」一曰:「民流。」八月丁巳,月犯斗第一星。占曰:「天下有兵。」一曰:「大臣
 憂。」



 案江左來,南斗有災,則吳越會稽、丹陽、豫章、廬江各隨其星應之。淮南失土,殆不占耳。史闕其說,故不列焉。九月戊子,熒惑犯少微。占曰:「處士誅。」庚寅,熒惑犯右執法。癸卯,熒惑犯左執法。占並同上。十月丁巳,月奄填星營室,占同七月。十一月丙戌,太白奄鉤金句鈐。占曰:「喉舌臣憂。」十二月己卯,歲星犯天江。占曰:「有兵亂,河津不通。」是年六月,索頭寇沛土,使偽豫州刺史索度真戍相縣,太傅長沙景王討破走之。十一月,荊州刺史魏詠之薨。二
 年二月,司馬國璠等攻沒弋陽。四月,羌伐仇池,仇池公楊盛擊走之。九月,益州刺史司馬榮期為其參軍楊承祖所害,時文處茂討蜀屢有功,會榮期死,乃退。三年十二月,司徒揚州刺史王謐薨。四年正月,太保武陵王遵薨。三月,左僕射孔安國卒。五年,高祖討鮮卑,并定舊兗之地。



 義熙二年二月己丑,月犯心後星。占曰:「豫州有災。」四月癸丑,月犯太微西將。己未,月犯房南第二星。乙丑,歲星
 犯天江,占悉同上。五月癸未,月犯左角。占曰:「左將軍死,天下有兵。」壬寅,熒惑犯氐。占曰:「氐為宿宮,人主憂。」六月庚午,熒惑犯房北第二星。八月癸亥,熒惑犯斗第五星。丁巳,犯建星。



 九月壬午,熒惑犯哭星,又犯泣星,占悉同上。十二月丙午,月奄太白,在危。占曰:「齊亡國。」一曰:「彊國君死。」丁未,熒惑、太白皆入羽林。是年二月甲戌,司馬國璠等攻沒弋陽。三年正月,鮮卑寇北徐州,至下邳。八月,遣劉敬宣伐蜀。十二月,司徒王謐薨。四年正月,武陵王
 遵薨。五年,鮮卑復寇淮北。四月,高祖大軍討之。六月,大戰臨朐城,進圍廣固。十月,什圭為其子偽清河公所殺。



 六年二月,拔廣固,禽慕容超,坑斬其眾三千餘人。



 義熙三年正月丙子,太白晝見,在奎。二月庚寅,月奄心後星,占悉同上。癸亥,熒惑、填星、太白、辰星聚於奎、婁,從填星也。其說見上九年。五月己丑,太白晝見,在參。占曰:「益州有兵喪,臣彊。」六月辛卯,熒惑犯辰星,在翼。



 占曰:「天下兵起。」八月己卯,太白奄熒惑,又犯執法。占曰:「奄熒惑,有
 大兵。」辛卯,熒惑犯左執法。九月壬子,熒惑犯進腎。是年正月丁巳,鮮卑寇北徐,至下邳。八月,劉敬宣伐蜀,不克而旋。四年三月,左僕射孔安國卒。七月,司馬國璠等攻沒鄒山,魯郡太守徐邕破走之。姚略遣眾征佛佛,大為所破。五年,高祖討鮮卑。六年三月,妖賊徐道覆殺鎮南將軍、江州刺史何無忌於豫章。四月,妖賊盧循寇湘中巴陵。五月丙子,循、道覆敗撫軍將軍、豫州刺史劉毅於桑落洲,毅僅以身免。丁丑,循等至蔡洲,遣別將焚京口。
 庚辰,賊攻焚查浦,查浦戍將距戰不利,高祖遣軍渡淮擊,大破之。司馬國璠寇碭山,竺夔討破之。七月,妖賊南走據尋陽,高祖遣劉鐘等追之。八月,孫季高乘海伐廣州。桓謙以蜀眾聚枝江,盧循將荀林略華容,相去百里。臨川烈武王討謙之,又討林,林退走。鄱陽太守虞丘進破賊別帥於上饒。九月,烈武王使劉遵擊荀林於巴陵,斬之。桓道兒率蔡猛向大薄,又遣劉基討之,斬猛。十月,高祖以舟師南征。是時徐道覆率二萬餘人攻荊州,烈
 武王距之。戰於江津,大破之,梟殄其十八九。道覆棄戰船走。十一月,劉鐘破賊軍於南陵。癸丑,益州刺史鮑陋卒于白帝,譙道福攻沒其眾。庚戌,孫季高襲廣州,克之。十二月,高祖在大雷,與賊交戰,大破之。賊走左里,進擊,又破,死者十八九。賊還廣州,劉籓等追之。七年二月,籓拔始興城,斬徐道覆、盧循還番禺,攻圍孫季高,不能剋。走交州,交州刺史杜慧度斬之。四月,到彥之攻譙道福於白帝,拔之。



 義熙四年正月庚子,熒惑犯天江,占同上。五月丁未,月奄斗第二星,占同上。



 壬子,填星犯天廩。占曰:「天下饑,倉粟少。」六月己丑,太白犯太微西上將。



 己卯,又犯左執法。十月戊子,熒惑入羽林。占悉同上。五年,高祖討鮮卑。六年,左僕射孟昶仰藥卒。是後南北軍旅,運轉不息。



 義熙五年二月甲子,月犯昴。占曰:「胡不安;天子破匈奴。」四月甲戌,熒惑犯辰星,在東井,占同三年。五月戊戌,歲星入羽林,占同上。九月壬寅,月犯昴,占同二月。十月,熒
 惑犯氐,占同二年。閏月丁酉,月犯昴,占同二月。辛亥,熒惑犯鉤鈐。占同元年。十二月辛丑,太白犯歲星,在奎。占曰:「大兵起。魯有兵。」己酉,月奄心大星。占曰:「王者惡之。」是年四月,高祖討鮮卑。什圭為其子所殺。十一月,西虜攻安定,姚略自以大眾救之。六年二月,鮮卑滅。皆胡不安之應也。是時鮮卑跨魯地,又魯有兵之應也。五月,盧循逼郊甸,宮衛被甲。



 義熙六年三月丁卯,月奄房南第二星。占曰:「災在次相。」
 己巳,又奄斗第五星。占曰:「斗主兵,兵起。」一曰:「將軍死。」太白犯五諸侯。占曰:「諸侯有誅。」五月甲子,月奄斗第五星,占同三月。己亥,月奄昴。占曰:「國有憂。」



 一曰:「有白衣之會。」六月己丑,月犯房南第二星。甲午,太白晝見,占並同上。



 七月己亥,月犯輿鬼。占曰:「國有憂。」一曰:「秦有兵。」八月壬午,太白犯軒轅大星。甲申,月犯心前星。災在豫州。丙戌,月犯斗第五星,占悉同上五月。



 丁亥,月奄牛宿南星。占曰:「天下有大誅。」乙未,太白犯少微。丙午,太白在少微而
 晝見。九月甲寅,太白犯左執法。丁丑,填星犯畢。占曰:「有邊兵。」是年三月,始興太守徐道覆反,江州刺史何無忌討之,大敗於豫章,無忌死之。四月,盧循寇湘中,沒巴陵。五月,循等大破豫州刺史劉毅,毅僅以身免;循率眾逼京畿。



 是月,左僕射孟昶懼王威不振,仰藥自殺。七年二月,劉籓梟徐道覆首,杜慧度斬盧循,並傳首京都。八年六月,臨川烈武王道規薨,時為豫州。八月,皇后王氏崩。



 九月,兗州刺史劉籓、尚書僕射謝混伏誅。高祖西討劉毅,
 斬之。十二月,遣益州刺史朱齡石伐蜀。九年,諸葛長民伏誅。林邑王范胡達將萬餘人寇九真,九真太守杜彗期距破之。七月,硃齡石滅蜀。



 義熙七年四月辛丑,熒惑入輿鬼。占曰:「秦有兵。」一曰:「雍州有災。」



 六月,太白晝見在翼,占同元年。己亥,填星犯天關。占曰:「臣謀主。」庚子,月犯歲星,在畢。占曰:「有邊兵,且飢。」七月丁卯,歲星在參。占曰:「歲、填合為內亂。」一曰:「益州戰不勝,亡地。」五虹見東方。占曰:「天子黜,聖人出。」八月乙未,
 月犯歲星,在參。占曰:「益州兵飢。」太白犯房南第二星。



 十一月丙午,太白犯哭泣星。占悉同上。七月,朱齡石剋蜀,蜀民尋又反,又討滅之。八年,誅劉籓、謝混,滅劉毅。皇后王氏崩。九年,誅諸葛長民。十一年,討荊州刺史司馬休之、雍州刺史魯宗之破之也。



 義熙八年正月庚戌,月犯歲星,在畢,占同上。七月癸亥,月奄房北第二星,占同上。甲申,太白犯填星,在東井。占曰:「秦有大兵。」己未,月犯井鉞。八月戊申,月犯泣星。十月
 辛亥,月奄天關。占曰:「有兵。」十月丁丑,填星犯東井。占曰:「大人憂。」十二月癸卯,填星犯井鉞。是年八月,皇后王氏崩。九月,誅劉籓、謝混,滅劉毅。九年三月,誅諸葛長民。西虜攻羌安定戍,克之。十二月,朱齡石伐蜀。九年七月,朱齡石滅蜀。



 義熙九年二月丙午,熒惑、填星皆犯東井。占曰:「秦有兵。」壬辰,歲星、熒惑、填星、太白聚于東井,從歲星也。熒惑入輿鬼,太白犯南河。初,義熙三年,四星聚奎,奎、婁,徐州分。
 是時慕容超僭號於齊,侵略徐、兗,連歲寇抄,至于淮、泗。姚興、譙縱僭偽秦、蜀。盧循、木末,南北交侵。五年,高祖北殄鮮卑,是四星聚奎之應也。九年,又聚東井。東井,秦分。十三年,高祖定關中,又其應也。而縱、循群凶之徒,皆己剪滅,於是天人歸望,建國舊徐。元熙二年,受終納禪,皆其徵也。《星傳》曰:「四星若合,是謂太陽,其國兵喪並起,君子憂,小人流。五星若合,是謂易行。有德受慶,改立王者,奄有四方;無德受罰,離其國家,滅其宗廟。」今案遺文所
 存,五星聚者有三:周漢以王齊以霸,周將伐殷,五星聚房。齊桓將霸,五星聚箕。漢高入秦,五星聚東井。齊則永終侯伯,卒無更紀之事。是則五星聚有不易行者矣。四星聚者有九:漢光武、晉元帝並中興,而魏、宋並更紀。是則四星聚有以易行者矣。昔漢平帝元始四年,四星聚柳、張,各五日。



 柳、張,三河分。後有王莽、赤眉之亂,而光武興復於洛。晉懷帝永嘉六年,四星聚牛、女,後有劉聰、石勒之亂,而元皇興復揚土。漢獻帝初平元年,四星聚心,
 又聚箕、尾。心,豫州分。後有董卓、李傕暴亂,黃巾、黑山熾擾,而魏武迎帝都許,遂以兗、豫定,是其應也。一曰:「心為天王,大兵升殿,天下大亂之兆也。」



 韓馥以為尾箕燕興之祥,故奉幽州牧劉虞,虞既距之,又尋滅亡,固已非矣。尾為燕,又為吳,此非公孫度,則孫權也。度偏據僻陋,然亦郊祀備物,皆為改漢矣。



 建安二十二年,四星又聚。二十五年而魏文受禪,此為四星三聚而易行矣。蜀臣亦引後聚為劉備之應。案太元十九年、義熙三年九月,四
 星各一聚,而宋有天下,與魏同也。魚豢云:「五星聚冀方,而魏有天下。」熒惑入輿鬼。占曰:「兵喪。」



 太白犯南河。占曰:「兵起。」後皆有應。



 五月壬辰,太白犯右執法,晝見,占同上。七月庚午,月奄鉤鈐。占曰:「喉舌臣憂。」九月庚午,歲星犯軒轅大星。己丑,月犯左角。十年正月丁卯,月犯畢。



 占曰:「將相有以家坐罪者。」二月己酉,月犯房北星。五月壬寅,月犯牽牛南星。



 乙丑,歲星犯軒轅大星,占悉同上。六月丙申,月奄氐。占
 曰:「將死之,國有誅者。」七月庚辰,月犯天關。占曰:「兵起。」熒惑犯井鉞,填星犯輿鬼,遂守之。



 占曰:「大人憂,宗廟改。」八月丁酉,月奄牽牛南星,占同上。九月,填星犯輿鬼。占曰:「人主憂。」丁巳,太白入羽林。十二月己酉,月犯西咸。占曰:「有陰謀。」十一年三月丁巳,月入畢。占曰:「天下兵起。」一曰:「有邊兵。」己卯,填星入輿鬼。閏月丙午,填星又入輿鬼。占曰:「為旱,為疫,為亂臣。」五月甲申,彗星出天市,掃帝座,在房、心。房、心,宋之分野。案占,得彗柄者興,除舊布新,宋興之
 象。癸卯,熒惑從行入太微。甲辰,犯右執法。六月己未,太白犯東井。占曰:「秦有兵。」戊寅,犯輿鬼。占曰:「國有憂。」七月辛丑,月犯異,占同上。八月壬子,月犯氐,占同上。庚申,太白從行從右掖門入太微。丁卯,奄左執法。十一月癸亥,月入畢,占同上。乙未,月入輿鬼而暈。占曰:「主憂,財寶出。」一曰:「暈,有赦。」



 十二年五月甲申,月犯歲星,在左角。占曰:「為飢。留房、心之間,宋之分野,與武王伐紂同,得歲者王。」于時晉始封高祖為宋公。六月壬子,太白從行入太微
 右掖門。己巳,月犯畢,占同上。七月,月犯牛宿。占曰:「天下有大誅。」



 十月丙戌,月入畢,占同上。



 十三年五月丙子,月犯軒轅。丁亥,犯牽牛。癸巳,熒惑犯右執法。八月己酉,月犯牽牛。丁卯,月犯太微。占曰:「人君憂。」九月壬辰,熒惑犯軒轅。十月戊申,月犯畢,占悉同上。月犯箕。占曰:「國有憂。」甲寅,月犯畢,占同上。乙卯,填星犯太微,留積七十餘日。占曰:「亡君之戒。」壬戌,月犯太微,占同上。



 十一月,月於太微,奄填星。占曰:「王者惡之。」



 十四年三月癸丑,太白犯五
 諸侯,占同上。四月壬申,月犯填星,於張。占曰:「天下有大喪。」五月庚子,月犯太微,占同上。壬子,有星孛於北斗魁中。占曰:「有聖人受命。」七月甲辰,熒惑犯輿鬼。占曰:「秦有兵。」丁巳,月犯東井。



 占曰:「軍將死。」癸亥,彗星出太微西,柄起上相星下,芒漸長至十餘丈,進歸北斗紫微中臺。占曰:「彗出太微,社稷亡,天下易王。入北斗紫微,帝宮空。」



 一曰:「天下得聖主。」八月甲子,太白犯軒轅。癸酉,填星入太微,犯右執法,因留太微中,積二百餘日乃去。占曰:「填星
 守太微,亡君之戒,有徙王。」九月乙未,太白入太微,犯左執法。丁巳,月入太微。占曰:「大人憂。」十月癸巳,熒惑入太微,犯西蕃上將,仍從行至左掖門內,留二十日乃逆行。至恭帝元熙元年三月五日,出西蕃上將西三尺許,又從還入太微。時填星在太微,熒惑繞填星成鉤已。其年四月二十七日丙戌,從端門出。占曰:「熒惑與填星鉤已,天下更紀。」



 甲申,月入太微,占同上。



 十一年正月,高祖討司馬休之、魯宗之等,潰奔長安。五月,林邑寇交州,交州刺史杜慧度距戰
 於九真,大為所敗。十二年七月,高祖伐羌。十月,前驅定陜、洛。十三年三月,索頭大眾緣河為寇,高祖討之奔退,其別帥托跋嵩交戰,又大破之,嵩眾殲焉。進復攻關。八月,擒姚泓,司、兗、秦、雍悉平,索頭兇懼。十四年,高祖還彭城,受宋公。十一月,左僕射前將軍劉穆之卒。明年,西虜寇長安,雍州刺史朱齡石諸軍陷沒,官軍舍而東。十二月,安帝崩,母弟琅邪王踐阼,是曰恭帝。



 晉恭帝元熙元年正月丙午,三月壬寅,月犯太微,占悉
 同上。乙卯,辰星犯軒轅。六月庚辰,太白犯太微。七月,月犯歲星。己卯,月犯太微,太白晝見。占悉同上。自義熙元年至是,太白經天者九,日蝕者四,皆從上始。革代更王,臣民失君之象也。是夜,太白犯哭星。十二月丁巳,月、太白俱入羽林。二年二月庚午,填星犯太微。占悉同上。元年七月,高祖受宋王。二年六月,晉帝遜位,高祖
 入宮。



\end{pinyinscope}