\article{卷十本紀第十 順帝}

\begin{pinyinscope}

 順皇帝諱準,字仲謀,小字智觀,明帝第三子也。泰始五年七月癸丑生。七年,封安成王,食邑三千戶。仍拜撫軍將軍,置佐史。廢帝即位,為揚州刺史。元徽二年,進號車
 騎將軍、都督揚、南豫二州諸軍事,給鼓吹一部,刺史如故。四年,又進號驃騎大將軍、開府儀同三司,班劍三十人,都督、刺史如故。元徽五年七月戊子夜,廢帝殞,奉迎王入居朝堂。壬辰,即皇帝位。



 昇明元年,改元,大赦天下,賜文武位二等。甲午,鎮軍將軍齊王出鎮東城,輔政作相。丙申,詔曰:「露臺息構,義光漢德;雉裘焚制,事隆晉道。故以檢奢軌化,敦儉馭俗。頃甸服未靜,師旅連年,委蓄屢空,勞敝莫偃。而丹雘之飾,糜耗難訾,寶賂之費,征賦
 靡計。今車服儀制,實宜約損,使徽章有序,勿得侈溢。



 可罷省御府二署。凡工麗彫鐫,傷風毀治,一皆禁斷。庶永昭憲則,弘茲始政。」



 征西大將軍、荊州刺史沈攸之進號車騎大將軍、開府儀同三司;尚書左僕射、中領軍、鎮軍將軍、南兗州刺史齊王為司空、錄尚書事、驃騎大將軍,刺史如故。中書令、衛將軍、開府儀同三司、撫軍將軍劉秉為尚書令,加中軍將軍;鎮西將軍、郢州刺史晉熙王燮為撫軍將軍、揚州刺史;南陽王翽為郢州刺史。辛丑,
 尚書右僕射王僧虔為尚書僕射,右衛將軍劉韞為中領軍,金紫光祿大夫王琨為右光祿大夫。給司空齊王錢五百萬,布五千匹。癸卯,車駕謁太廟。丙午,以安西參軍明慶符為青、冀二州刺史,武陵王贊為郢州刺史,新除郢州刺史南陽王翽為湘州刺史,司空、南兗州刺史齊王改領南徐州刺史,征虜將軍李安民為南兗州刺史。雍州大水,八月壬子,遣使賑恤,蠲除稅調。以驃騎長史劉澄之為南豫州刺史。山陽太守于天寶、新吳縣子
 秦立有罪,下獄死。戊午,改平準署。辛酉,以宣城太守李靈謙為兗州刺史。



 癸亥,司空袁粲鎮石頭。丁卯,原除元年以前逋調;復郡縣祿田。戊辰,崇拜帝所生陳昭華為皇太妃。庚午,司空長史謝朏、衛軍長史江斅、中書侍郎褚炫、武陵王文學劉侯入直殿省,參侍文義。齊王固讓司空,庚辰,以為驃騎大將軍、開府儀同三司。九月己丑,詔曰:「昔聖王既沒,淳風已衰,龜書永湮,龍圖長秘。故三代之末,德刑相擾。世淪物競,道陂人諛。然猶正士比轂,
 奇才接軫。朕襲運金樞,纂靈瑤極,負扆巡政,日晏忘疲,永言興替,望古盈慮。姬、夏典載,猶傳緗帙,漢、魏餘文,布在方冊。故元封興茂才之制,地節創獨行之品。振維務本,存乎得人。今可宣下州郡,搜揚幽仄,摽采鄉邑,隨名薦上。朕將親覽,甄其茂異。庶野無遺彥,永激遐芬。」己酉,廬陵王暠薨。冬十一月己酉,倭國遣使獻方物。丙午,員外散騎侍郎胡羨生行越州刺史,以交州刺史沈景德為廣州刺史。十二月丁巳,以驍騎將軍王廣之為徐州
 刺史。車騎大將軍、荊州刺史沈攸之舉兵反。丁卯,錄公齊王入守朝堂,侍中蕭嶷鎮東府。戊辰,內外纂嚴。己巳,以郢州刺史武陵王贊為安西將軍、荊州刺史,征虜將軍、雍州刺史張敬兒進號鎮軍將軍。右衛將軍黃回為平西將軍、郢州刺史,督諸軍前鋒南討。征虜將軍呂安國為湘州刺史,都官尚書王寬加平西將軍。庚午,新除左衛將軍齊王世子奉新除撫軍將軍、揚州刺史晉熙王燮鎮尋陽之盆城。壬申,以驍騎將軍周槃龍為廣州
 刺史。是日,司徒袁粲據石頭反,尚書令劉秉、黃門侍郎劉述、冠軍王蘊率眾赴之。黃回及輔國將軍孫曇瓘、屯騎校尉王宜興、輔國將軍任候伯、左軍將軍彭文之密相響應。中領軍劉韞、直皞將軍卜伯興在殿內同謀。錄公齊王誅韞等於省內。軍主蘇烈、王天生、薛道淵、戴僧靜等陷石頭,斬粲於城內。秉、述、蘊躍城走,追擒之,並伏誅;其餘無所問。豫州刺史劉懷珍、雍州刺史張敬兒、廣州刺史陳顯達並舉義兵。司州刺史姚道和、梁州刺史
 范柏年、湘州行事庾佩玉擁眾懷貳。甲戌,大赦天下。乙亥,以尚書僕射王僧虔為尚書左僕射,新除中書令王延之為尚書右僕射。吳郡太守劉遐據郡反,輔國將軍張環討斬之。閏月辛巳,屯騎校尉王宜興有罪伏誅。癸巳,沈攸之攻圍郢城,前軍長史柳世隆固守。攸之弟登之作亂於吳興,吳興太守沈文秀討斬之。己亥,內外戒嚴。



 假錄公齊王黃鉞。辛丑,寧朔將軍、北秦州刺史武都王楊文度進號征西將軍。乙巳,錄公齊王出頓新亭。



 二年春正月,沈攸之遣將公孫方平據西陽。辛酉,建寧太守張謨擊破之。丁卯,沈攸之自郢城奔散。己巳,華容縣民斬送之。左將軍、豫州刺史劉懷珍進號平南將軍。辛未,鎮軍將軍、雍州刺史張敬兒克江陵,斬攸之子光琰,荊州平,同逆皆伏誅。丙子,解嚴。以新除侍中柳世隆為尚書右僕射。是日,錄公齊王旋鎮東府。丁丑,以江州刺史邵陵王友為安南將軍、豫州刺史。左衛將軍齊王世子為江州刺史,侍中蕭嶷為領軍,鎮軍將軍、雍州刺
 史張敬兒進號征西將軍,平西將軍、郢州刺史黃回進號鎮西將軍。二月庚辰,以尚書左僕射王僧虔為尚書令,尚書右僕射王延之為尚書左僕射。癸未,錄公齊王加授太尉,衛將軍褚淵為中書監、司空。甲申,曲赦荊州。丙戌,撫軍將軍、揚州刺史晉熙王燮進號中軍將軍、開府儀同三司。戊子,蠲雍州緣沔居民前被水災者租布三年。辛卯,郢州刺史、新除鎮南將軍黃回為鎮北將軍、南兗州刺史,南兗州刺史李安民為郢州刺史。癸巳,以
 山陰令傅琰為益州刺史。丙申,左軍將軍彭文之有罪,下獄死。行湘州事任侯伯殺前湘州行事庾佩玉,傳首京邑。三月庚戌,以廣州刺史周槃龍為司州刺史,輔國將軍劉悛為廣州刺史。



 丙子,給太尉齊王羽葆、鼓吹。夏四月己卯,以遊擊將軍垣崇祖為兗州刺史。辛卯,新除鎮北將軍、南兗州刺史黃回有罪賜死。甲午,輔國將軍、淮南宣城二郡太守蕭映行南兗州刺史。五月戊午,倭國王武遣使獻方物,以武為安東大將軍。輔國將軍、行
 湘州事任侯伯有罪伏誅。六月己丑,以前新會太守趙超民為交州刺史。丁酉,以輔國將軍楊文弘為北秦州刺史、武都王。八月辛卯,太尉齊王表斷奇飾麗服,凡十有四條。乙未,以江州刺史齊王世子為領軍將軍、撫軍將軍。丙申,以領軍蕭嶷為江州刺史。九月乙巳朔,日有蝕之。丙午,加太尉齊王黃鉞、都督中外諸軍事、太傅,領揚州牧,劍履上殿,入朝不趨,贊拜不名。置左右長史、司馬、從事中郎、掾、屬各四人。中軍將軍、揚州刺史晉熙
 王燮為司徒。戊申,行南兗州刺史蕭映為南兗州刺史。甲寅,給太傅齊王三望車。己未,芮芮國遣使獻方物。癸酉,武陵內史張澹有罪,下獄死。冬十月丁丑,寧朔將軍、淮南宣城二郡太守蕭晃為豫州刺史。



 孫曇瓘先逃亡,己卯,擒獲,伏誅。壬寅,立皇后謝氏,減死罪一等,五歲刑以下悉原。十一月壬子,立故武昌太守劉琨息頒為南豐縣王。癸亥,臨灃侯劉晃謀反,晃及黨與皆伏誅。甲子,改封南陽王翽為隨郡王,改隨陽郡。十二月丙戌,皇后
 見于太廟。戊子,高麗國遣使獻方物。



 三年春正月甲辰,以江州刺史蕭嶷為鎮西將軍、荊州刺史,尚書左僕射王延之為安南將軍、江州刺史。安西長史蕭順之為郢州刺史。乙卯,太傅齊王表諸負官物質役者,悉原除。辛亥,以驍騎將軍王玄邈為梁、南秦二州刺史。領軍將軍、撫軍將軍齊王世子加尚書僕射,進號中軍大將軍、開府儀同三司。丙辰,加太傅齊王前部羽葆、鼓吹。丁巳,詔太傅府依舊辟召。以征西將軍、雍州
 刺史張敬兒為護軍將軍,新除給事黃門侍郎蕭諱為雍州刺史。二月丙子,安南將軍、南豫州刺史邵陵王友薨。三月癸卯朔,日有蝕之。甲辰,崇太傅為相國,總百揆,封十郡,為齊公,備九錫之禮,加璽紱遠游冠,位在諸王上;加相國綠綟綬,其驃騎大將軍、揚州牧、南徐州刺史如故。丙午,以中軍大將軍蕭諱為南豫州刺史、齊公世子,副貳相國,綠綟綬。庚戌,臨川王綽謀反,綽及黨與皆伏誅。丁巳,以齊國初建,給錢五百萬,布五千匹,絹千匹。夏
 四月壬申,進齊公爵為齊王,增封十郡。甲戌,安西將軍武陵王贊薨。丙戌,命齊王冕十有二旒,建天子旌旗,出警入蹕,乘金根車,駕六馬,備五時副車,置旄頭雲罕,樂舞八佾,設鐘虡宮縣。進世子為太子,王子、王女、王孫爵命之號,壹如舊儀。辛卯,天祿永終,禪位于齊,壬辰,帝遜位于東邸。既而遷居丹陽宮。齊王踐阼,封帝為汝陰王,待以不臣之禮。行宋正朔,上書不為表,答表不為詔。



 建元元年五月己未,殂于丹陽宮,時年十三。謚曰順帝。六月
 乙酉,葬於遂寧陵。



 史臣曰:聖王膺錄,自非接亂承微,則天歷不至也。自三、五以來,受命之主,莫不乘淪亡之極,然後符樂推之運。水德遷謝,其來久矣。豈止於區區汝陰揖禪而已哉!



\end{pinyinscope}