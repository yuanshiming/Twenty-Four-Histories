\article{卷三十八志第二十八 州郡四}

\begin{pinyinscope}

 益州寧州廣州交州越州益州刺史,漢武帝分梁州立,所治別見梁州,領郡二十九,縣一百二十八,戶五萬三千一百四十一,口二十四萬八千二百九十三。去京都水九千九百七十。



 蜀郡太守,秦立。晉武帝太康中,改曰成都國,後復舊。領縣五,戶一萬一千九百二,口六萬八百七十六。



 成都令,漢舊縣。



 郫令,漢舊縣。



 繁縣令,漢舊縣。



 鞞縣令,二漢、《晉太康地志》並曰牛鞞,屬犍為,何志晉穆帝度此。



 永昌令,孝建二年,以僑戶立。



 廣漢太守,漢高帝六年立。《晉太康地志》屬梁州。領縣六,戶四千五百八十六,口二萬七千一百四十九。去州陸六百;去京都水九千九百。



 雒縣令,漢舊縣。



 什邡令,漢舊縣。



 郪縣令,漢舊縣。



 新都令,漢舊縣,晉武帝為王國,太康六年省為縣。屬廣漢。



 陽泉令,蜀分綿竹立。



 伍城令,晉武帝咸寧四年立,太康六年省,七年又立。何志劉氏立。



 巴西太守,譙周《巴記》,建安六年,劉璋分巴郡墊江以上為巴西郡。徐志本南陽冠軍流民,寓入蜀漢,晉武帝立,非也。本屬梁州,文帝元嘉十六年度。何志梁、益二州無此郡。領縣九,戶四千九百五十四,口三萬三千三百四十六。



 閬中令,漢舊縣,屬巴郡。



 西充國令,《漢書地理志》,巴郡有充國縣。《續漢郡國志》,和帝永元二年,分閬中立充國縣。



 二志不同。《
 晉太康地志》有西南二充國,屬巴西。



 南充國令,譙周《巴記》,初平六年,分充國為南充國。



 安漢令,舊縣,屬巴郡。



 漢昌令,和帝永元中立。



 晉興令,徐志不注置立。



 平州令,晉武帝太康元年,以野民歸化立。



 懷歸令,徐志不注置立。



 益昌令,徐志不注置立。



 梓潼太守,《晉太康地志》劉氏分廣漢立。本屬梁州,文帝元嘉十六年,度益州。《永初郡國》又有漢德、新興,徐同。徐云,新興,義熙九年立;漢德,舊縣。案二漢並無漢德縣,《晉太康地志》、王隱並有,疑是劉氏所立。何益、梁二州無此郡。領縣四,戶三千三十四,口二萬一千九百七十六。



 涪令,漢舊縣,屬廣漢。



 梓潼令,漢舊縣,屬廣漢。



 西浦令,徐志義熙九年立。



 萬安令,徐志舊縣。二漢、晉並無。



 巴郡太守,秦立。領縣四,戶三千七百三十四,口一萬三千一百八十三。去州內水一千八百,陸五百,外水二千二百;去京都水六千。



 江州令,漢舊縣。



 臨江令,漢舊縣。



 墊江令,漢舊縣,獻帝建安六年度巴西,劉禪建興十五年復舊。



 枳令,漢舊縣。



 遂寧太守,《永初郡國》有,何無,徐云舊立。領縣四,戶三千三百二十。



 巴興令,徐志不注置立,疑是李氏所立。



 德陽令,前漢無,後漢、《晉太康地志》屬廣漢。



 廣漢令,漢舊縣,屬廣漢。寧蜀郡復有此縣,未知孰是。



 晉興令,徐志不注置立。



 江陽太守,劉璋分犍為立。中失本土,寄治武陽。領縣四,戶一千五百二十五,口八千二十七。



 江陽令,漢舊縣,屬犍為。


綿水令。
 \gezhu{
  別見}


漢安令。
 \gezhu{
  別見}



 常安令,晉孝武立。



 懷寧太守,秦、雍流民,晉安帝立。本屬南秦,文帝元嘉十六年度益州。領縣三,戶一千三百一十五,口五千九百
 五十。寄治成都。


始平令。
 \gezhu{
  別見}



 西平令,《永初郡國》直云西。何志故屬天水,名西縣。



 萬年令,漢舊名,屬馮翊。



 寧蜀太守,《永初郡國》有而何無,徐云舊立。《永初郡國》及徐並有西墊江縣,今無。領縣四,戶一千六百四十三。


廣漢令
 \gezhu{
  別見}
 ,遂寧郡復有此縣。



 廣都令,漢舊縣,屬蜀郡。



 升遷令,《晉太康地志》屬汶山。



 西鄉令,本名南鄉,屬漢中,晉武太康三年更名。



 越巂太守,漢武帝元鼎六年立,故邛都國。何志無。領縣八,戶一千三百四十九。



 邛都令,漢舊縣。



 新興令,《永初郡國》有。



 臺登長,漢舊縣。



 晉興長,《永初郡國》有。



 會無長,漢舊縣。



 卑水長,漢舊縣。



 定莋長,漢舊縣。



 蘇利長,漢縣曰蘇示,囗曰蘇利。



 汶山太守,《晉太康地志》漢武帝立,孝宣地節三年合蜀郡,劉氏又立。領縣二,戶一千一百七,口六千一百五。去州陸一百;去京都水一萬。



 都安侯相,蜀立。



 晏官令,何志魏平蜀立。《晉太康地志》無。


南陰平太守
 \gezhu{
  陰平郡別見}
 ,永嘉流寓來屬,寄治萇陽。領縣二。戶一千二百四十,口七千五百九十七。


陰平令。
 \gezhu{
  別見}



 綿竹令,漢舊縣,屬廣漢。



 犍為太守,漢武帝建元六年,開夜郎國立。領縣五,戶一千三百九十,口四千五十七。去州陸九十;去京都水一萬。



 武陽令,漢舊縣。



 南安令,漢舊縣。



 資中令,漢舊縣。



 僰道令,漢舊縣。



 治官令,晉安帝義熙十年立。



 始康太守,關隴流民,晉安帝立。領縣四,戶一千六十三,口四千二百二十六。寄治成都。



 始康令,晉安帝立。



 新城子相,晉安帝立。



 談令,晉安帝立。



 晉豐令,晉安帝立。



 晉熙太守,秦州流民,晉安帝立。領縣二,戶七百八十五,口三千九百二十五。



 晉熙令,晉安帝立。



 萇陽令,晉安帝立。



 晉原太守,李雄分蜀郡為漢原,晉穆帝更名。領縣五,戶
 一千二百七十二,口四千九百六十。



 去州陸一百二十;去京都水一萬。



 江原男相,漢舊縣,屬蜀郡。



 臨邛令,漢舊縣,屬蜀郡。



 晉樂令,何志故屬沈黎。《晉太康地志》無沈黎郡及晉樂縣。



 徙陽令,前漢徙縣屬蜀郡,後漢屬蜀郡屬國都尉。《晉太康地志》有徙陽縣,屬漢嘉。



 漢嘉令,前漢青衣縣屬蜀郡,順帝陽嘉二年更名。劉氏立為漢嘉郡,晉江右猶為郡,江左省為縣。



 宋寧太守,文帝元嘉十年,免吳營僑立。領縣三,戶一千三十六,口八千三百四十二。寄治成都。



 欣平令,與郡俱立。



 宜昌令,與郡俱立。



 永安令,與郡俱立。



 安固太守,張氏於涼州立。晉哀帝時,民流入蜀,僑立此
 郡,本屬南秦,文帝元嘉十六年度益州。領縣六,戶一千一百二十,口六千五百五十七。去州一百三十。去京都水一萬。


略陽令。
 \gezhu{
  別見}



 桓陵令,張氏立。



 臨渭令,《晉太康地志》屬略陽。


清水令。
 \gezhu{
  別見}
 。



 下邽令,何志漢舊縣。案二漢、晉並無此縣。



 興固令,何志新立。



 南漢中太守,晉地記,孝武太元十五年,梁州刺史周瓊表立。徐志,北漢中民流寓,孝武大明三年立。《起居注》,本屬梁州,元嘉十六年度。《永初郡國》屬梁州,領縣與此同。以《永初郡國》及《起居》檢,則是太元所立,而何志無此郡,當是永初以後省,大明三年復立也。領縣五,戶一千八十四,口五千二百四十六。



 南長樂令,徐志與郡俱立。



 南鄭令,徐志與郡俱立。



 南苞中令,徐志與郡俱立。



 南沔陽令,徐志與郡俱立。



 南城固令,徐志與郡俱立。



 北陰平太守,徐志本屬秦州,文帝元嘉二十六年度。《永初郡國》、何志,秦、梁、益並無。



 領縣四,戶一千五十三,口六千七百六十四。


陰平令。
 \gezhu{
  已見}



 南陽令,徐志本南陽白民流寓立。



 桓陵令,徐志本安固郡民流寓立。



 順陽令,徐志本南陽民流寓立。


武都太守
 \gezhu{
  別見}
 ,《永初郡國》、何志益州並無此郡。徐志本屬秦州,流寓立。領縣五。



 戶九百八十二,口四千四百一。



 武都令,漢舊名。


下辯令。
 \gezhu{
  別見}



 漢陽令,漢舊名。



 略陽令,漢屬略陽郡,流寓配。



 安定令,舊安定郡,流寓配。


新城太守,何志新分廣漢立。領縣二,戶七百五十三,口五千九百七十一。去州
 \gezhu{
  闕}
 去京都九千五百三十。



 北五城令,何志新分五城立。



 懷歸令,何志新立。


南新巴太守
 \gezhu{
  新巴郡別見}
 ,《起居注》新巴民流寓,文帝元嘉十二年,於劍南立。何志新立,新巴民先屬梁州,既立害配。
 領縣六,戶一千七十,口二千六百八十三。



 新巴令,何志晉安帝立。



 晉城令,何志晉安帝立。



 晉安令,何志晉安帝立。



 漢昌令,何志晉安帝立。



 桓陵令,何志晉哀帝立。按《起居注》,南新巴,元嘉十二年立。何云新立,則非先有此郡,而云此諸縣晉哀帝、安帝立,不詳。



 綏歸令,何無此。徐有,不注置立。



 南晉壽太守,梁州元有晉壽,文帝元嘉十二年,於劍南以僑流立。領縣五,戶一千五十七,口一千九百四十三。去州一百二十;去京都水一萬。


晉壽令。
 \gezhu{
  別見}


興安令。
 \gezhu{
  別見}



 興樂令,二漢、魏無。《晉太康地記》云:「元年更名。本曰白馬,屬汶山。」何志,漢舊縣。檢二漢益部,無白馬
 縣。


邵歡令。
 \gezhu{
  別見}


白馬令。
 \gezhu{
  別見}



 宋興太守,文帝元嘉十年,免建平營立。領南陵、建昌二縣。何志無復南陵,有南漢、建忠。



 徐無建忠,有永川,何云建忠新立。領縣三,戶四百九十六,口一千九百四十三。寄治成都。



 南漢令,何志晉穆帝立。故屬漢中,流寓來配。



 建昌令,何志新立。



 永川令,徐志新立。



 南宕渠太守,徐志本南中民,蜀立。《起居注》,本屬梁州,元嘉十六年度。《永初郡國》梁州有宕渠郡,領縣三,與此同。而無「南」字,何同。若此郡元嘉十六年度益,則何志應在益部,不詳。領縣三,戶五百四,口三千一百二十七。



 宕渠令,二漢、《晉太康地志》屬巴郡。



 漢興令,二漢、魏無,晉地志有,屬興古郡。



 宣漢令,前漢無,後漢屬巴郡,《晉太康地志》無。


天水太守
 \gezhu{
  別見}
 ,《永初郡國》、何志益州無此郡。徐志與今同。領縣三,戶四百六十一。



 宋興令,徐志不注置立。


上邽令。
 \gezhu{
  別見}


西縣長。
 \gezhu{
  別見}



 東江陽太守,何志晉安帝初,流寓入蜀,今新復舊土為郡。領縣二,戶一百四十二,口七百四十。去州一千五百
 八十;去京都水八千九十。



 漢安令,前漢無,後漢屬犍為,《晉太康地志》屬江陽。



 綿水令,何志晉孝武立。



 沈黎太守,《蜀記》云:「漢武元鼎十一年,分蜀西部邛莋為沈黎郡,十四年罷。」案元鼎至六年,云十一年,非也。又二漢、晉並無此郡,《永初郡國》有,何無,徐云舊郡。領縣四,戶六十五。



 城陽令,徐不注置立。



 蘭令,漢舊縣,屬越巂,作「闌」,《晉太康地志》無。



 旄牛令,前漢屬蜀郡,後漢屬蜀郡屬國都尉,《晉太康地志》屬漢嘉。



 寧州刺史,晉武帝太始七年分益州南中之建寧、興古、雲南、永昌四郡立。太康三年省,立南夷校尉。惠帝太安二年復立,增牂牁、越巂、硃提三郡。成帝咸康四年,分牂牁、夜郎、朱提、越巂四郡為安州,尋罷並寧州。越巂復還益州。今領郡十五,縣八十一,戶一萬二百五十三。去京都一萬三
 千三百。



 建寧太守,漢益州郡滇王國,劉氏更名。領縣十三,戶二千五百六十二。



 味縣令,漢舊縣。



 同樂令,晉武帝立。



 談槁令,漢舊縣,屬牂牁。晉武帝立。



 牧麻令,漢舊縣,作牧靡。



 漏江令,漢舊縣,屬牂牁。晉武帝立。



 同瀨長,漢舊縣。「同」作「銅」。



 昆澤長,漢舊縣。



 新定長,《晉太康地志》有。



 存邑囗,《晉太康地志》有。



 同並長,漢舊縣,前漢作同並,屬牂牁。晉武帝咸寧五年省,哀帝復立。



 萬安長,江左立。



 毋單長,漢舊縣,屬牂牁,《晉太康地志》屬建寧。



 新興長,江左立。



 晉寧太守,晉惠帝太安二年,分建寧西七縣為益州郡,晉懷帝更名。領縣七,戶六百三十七。



 去州七百三十;去京都水一萬三千七百。



 建伶令,舊漢縣,屬益州郡,《晉太康地志》屬建寧。



 連然令,漢舊縣,屬益州郡,《晉太康地志》屬建寧。



 滇池令,漢舊縣,屬益州郡,《晉太康地志》屬建寧。



 穀昌長,漢舊縣,屬益州郡,《晉太康地志》屬建寧。


秦臧長,漢舊縣,屬益州郡,《晉太康地志》屬建寧。
 \gezhu{
  闕}



 〔俞元長,漢舊縣,屬益州,《晉太康地志》〕雙柏長,漢舊縣,屬益州郡,《晉太康地志》屬建寧。



 牂牁太守,漢武帝元鼎六年立。領縣六,戶一千九百七十。去州一千五百;去京都水一萬二千。



 萬壽令,晉武帝立。



 且蘭令,漢舊縣云故且蘭,《晉太康地志》無。



 故毋斂令,漢舊縣。



 晉樂令,江左立。



 丹南長,江左立。



 新寧長,何、徐不注置立。



 平蠻太守,晉懷帝永嘉五年,寧州刺史王遜分牂牁、朱提、建寧立平夷郡,後避桓溫諱改。領縣二,戶二百四十五。去京都水一萬三千。



 平蠻令,漢舊縣,屬牂牁,故名平夷。



 鄨令,漢舊縣,屬牂牁。



 夜郎太守,晉懷帝永嘉五年,寧州刺史王遜分牂牁、朱提、建寧立。領縣四,戶二百八十八。



 去州一千;去京都水一萬四千。



 夜郎令,漢舊縣,屬牂牁。



 廣談長,《晉太康地志》屬牂牁。



 談樂長,江左立。



 談柏令,漢舊縣,屬牂牁。



 朱提太守,劉氏分犍為立。領縣五,戶一千一十。去州七
 百二十;去京都水一萬四千六百。



 朱提令,前漢屬犍為,後漢屬犍為屬國都尉。



 堂狼令,前漢屬犍為,「狼」作「琅」。後漢、《晉太康地志》屬朱提。



 臨利長,江左立。



 漢陽長,前漢屬犍為,後漢無,《晉太康地志》屬朱提。



 南秦長,本名南昌,晉武帝太康元年更名。



 南廣太守,晉懷帝分朱提立。領縣四,戶四百四十。去州
 水二千三百;去京都水一萬四百。



 南廣令,漢舊縣,屬犍為,《晉太康地志》屬朱提。



 新興令,何志不注置立。



 晉昌令,江左立。



 常遷長,江左立。



 建都太守,晉成帝分建寧立。領縣六,戶一百七。去州二千;去京都水一萬五十。



 新安令,晉成帝立。



 經雲令,晉成帝立。



 永豐令,晉成帝立。



 臨江令,晉成帝立。



 麻應長,晉成帝立。



 遂安長,晉成帝立。



 西平太守,晉懷帝永嘉五年,寧州刺史王遜分興古之東立。何志晉成帝立,非也。《永初郡國》、何志並有西寧縣,何云晉成帝立,今無。領縣五,戶一百七十六。去州二千
 三百;去京都水一萬五千三百。



 西平令,何志晉成帝立。



 溫江令,何志晉成帝立。



 都陽令,何志晉成帝立。案《晉起居注》,太康二年置興古之都唐縣,疑是。



 晉綏長,何志晉成帝立。



 義成長,何志晉成帝立。案此五縣應與郡俱立。



 西河陽太守,晉成帝分河陽立。領縣三,戶三百六十九。去
 州二千五百;去京都水一萬五千五百。



 芘蘇令,前漢屬益州郡,後漢、《晉太康地志》屬永昌。「芘」作「比」。



 成昌令,晉成帝立。



 建安長,晉成帝立。


東河陽太守,晉懷帝永嘉五年,寧州刺史王遜分永昌、雲南立。《永初郡國》又有西河陽,領楪榆、遂段、新豐三縣,何、徐無。
 \gezhu{
  遂段、新豐二縣,二漢、晉並無。}
 領縣二,戶一百五十二。去州二
 千;去京都水一萬五千。東河陽令,何不注置立,疑與郡俱立。



 楪榆長,前漢屬益州郡,後漢屬永昌,《晉太康地志》屬雲南。前漢「枼」作「葉」。


雲南太守,《晉太康地志》云,故屬永昌。何志劉氏分建寧、永昌立。領縣五,
 \gezhu{
  疑}
 戶三百八十一。去州一千五百;去京都水一萬四千五百。



 雲南令,前漢屬益州郡,後漢屬永昌,《晉太康地志》
 屬雲南。



 雲平長,晉武帝咸寧五年立。



 東古復長,漢屬越巂,《晉太康地志》屬雲南,並云姑復。《永初郡國》、何並云東古復,何不注置立。



 西古復長,《永初郡國》有。何不注置立。



 興寧太守,晉成帝分雲南立。領縣二,戶七百五十三。去州一千五百;去京都水一萬四千五百。



 梇棟令,漢舊縣,屬益州,《晉太康地志》屬雲南。



 青蛉令,漢舊縣,屬越巂,《晉太康地志》屬雲南。



 興古太守,漢舊郡,《晉太康地志》故牂牁。何志劉氏分建寧、牂牁立,則是後漢末省也。



 領縣六,戶三百八十六。去州二千三百;去京都水一萬六千。



 漏臥令,漢舊縣。屬牂牁。



 宛暖令,漢舊,屬牂牁。本名宛溫,為桓溫改。



 律高令,漢舊縣,屬益州郡,後省。晉武帝咸寧元年,分建寧郡修雲、俞元二縣間流民復立律高縣。
 修雲、俞元二縣,二漢無。



 西安令,江左立。



 句町令,漢舊縣,屬牂牁。



 南興長。江左立。



 梁水太守,晉成帝分興古立。領縣七,戶四百三十一。去州水三千;去京都水一萬六千。



 梁水令,與郡俱立。



 騰休長,漢舊縣,屬益州郡,《晉太康地志》屬興古,何
 志故屬建寧,晉武帝從興古治之,遂以屬焉。



 西隋令,漢舊縣,屬牂牁,《晉太康地志》屬興古。並作「隨」。



 毋棳令,漢舊縣,屬益州郡,《晉太康地志》屬興古。劉氏改曰西豐,晉武帝泰始五年,復為毋棳。



 新豐長,何志不注置立。



 建安長,何志不注置立。



 鐔封長,漢舊縣,屬牂牁,《晉太康地志》屬興古。



 廣州刺史,吳孫休永安七年,分交州立。領郡十七,縣一百三十六,戶四萬九千七百二十六,口二十萬六千六百九十四。去京都水五千二百。



 南海太守,秦立。秦敗,尉他王此地,至漢武帝元鼎六年,開屬交州。領縣十,戶八千五百七十四,口四萬九千一百五十七。



 番禺男相,漢舊縣。



 熙安子相,文帝立。



 增城令,前漢無,後漢有。



 博羅男相,漢舊縣。二漢皆作「傅」字,《晉太康地志》作「博」。



 酉平令,《永初郡國》有。



 龍川令,舊縣。



 懷化令,晉安帝立。



 綏寧男相,文帝立。



 高要子相,漢舊縣,屬蒼梧,文帝廢。



 始昌令,文帝立。


蒼梧太守,漢武帝元鼎六年立。《永初郡國》又有高要、建陵、寧新、都羅、端溪、撫寧六縣。建陵、寧新,吳立。都羅,晉武分建陵立。晉武帝太康元年,改新寧曰寧新。端溪
 \gezhu{
  別見}
 、撫寧始見《永初郡國》。高要,何志無,餘與《永初郡國》同。徐志無建陵、寧新、撫寧三縣。何、徐二志並有懷熙一縣。思安、封興、蕩康、僑寧四縣,疑是宋末度此也。今領縣十一,戶六千五百九十三,口萬一千七百五十三。去州水八百;
 去京都水五千五百九十。



 廣信令,漢舊縣。



 猛陵令,漢舊縣。



 懷熙令,文帝立。



 思安令,《永初郡國》有,及何志並屬晉康,徐志度此。



 封興令,《永初郡國》有,及何志並屬晉康,徐志度此。



 蕩康令,《永初郡國》有,及何志並屬晉康,徐志度此。



 僑寧令,《永初郡國》有,及何志並屬晉康,徐志度此。



 遂成令,《永初郡國》有。



 丁留令,晉武帝太康七年,以蒼梧蠻夷賓服立,囗作「丁溜」;溜音留。



 廣陵令,《永初郡國》有。



 武化令,徐志以前無,疑是宋末所立。


晉康太守,晉穆帝永和七年分蒼梧立,治元溪。《永初郡國》治龍鄉。何志無復龍鄉縣,當是晉末立。元嘉二十年前,以龍鄉併端溪也。《永初郡國》又有封興、蕩康、思安、遼安、
 開平縣。何志無遼安、開平二縣,餘與《永初郡國》同。封興、蕩康、思安
 \gezhu{
  別見}
 、遼安、開平,應是晉末立,元嘉二十年前省。今領縣十四,戶四千五百四十七,口一萬七千七百一十。去州水五百,去京都水五千八百。端溪令,漢舊縣,何志屬蒼梧,徐志屬此。



 晉化令,何志不注置立,疑是晉末所立。



 都城令,何志晉初分建陵立,今無建陵縣。按《太康地志》唯有都羅、武城縣。



 樂城令,何志無,徐志有。



 賓江令,何志無,徐志有。



 說城令,何志無,徐志有。



 元溪令,《晉太康地志》屬蒼梧。



 夫阮令,《永初郡國》有。



 僑寧令,何志云漢舊縣,檢二漢《地理》《郡國》,無。蒼梧又有僑寧縣。



 安遂令,文帝立。



 永始令,文帝立。



 武定令,文帝立。



 文招令,何志無,徐志有二文招,一屬綏建,一屬晉康。



 熙寧令,何志無,徐志有。



 新寧太守,晉穆帝永和七年,分蒼梧立。《永初郡國》有平興、永城縣,何、徐志有永城,無平興。此二縣當是晉末立。平興當是元嘉二十年以前省,永城當是大明八年以
 後省。何志又有熙寧縣,云新立,當是文帝所立。徐志無,當是元嘉二十年後省也。今領縣十四,戶二千六百五十三,口一萬五百一十四。去州水六百二十;去京都水五千六百。



 南興令,何志漢舊縣。檢二漢《地理》《郡國》、《晉太康地志》並無。《永初郡國》有。



 臨允令,漢舊縣,屬合浦,《晉太康地志》屬蒼梧。何志,吳度蒼梧。



 新興令,《永初郡國》有,何志不注置立。



 博林令,《永初郡國》有,何志不注置立。



 甘東令,《永初郡國》有,何志不注置立。



 單牒令,《永初郡國》有,何志不注置立。



 威平令,《永初郡國》有,何志不注置立。



 龍潭令,文帝立。



 平鄉令,文帝立。



 城陽令,文帝立。



 威化令,文帝立。



 初興令,文帝立。



 撫納令,徐志有。



 歸順令,徐志有。


永平太守,晉穆帝升平五年,分蒼梧立。《永初郡國》有雷鄉、盧平、員鄉、逋寧、開城五縣,當是與郡俱立。何、徐志無雷鄉、員鄉,又有熙平,云新立,疑是文帝所立。雷鄉、員鄉當是元嘉二十年以前省。盧平、逋寧、開城當是大明八
 年以後省。今領縣七,
 \gezhu{
  疑}
 戶一千六百九,口一萬七千二百二。去州水一千二百;去京都水五千四百。



 安沂令,《永初郡國》有,何志不注置立。



 豐城令,吳立,屬蒼梧。《永初郡國》併安沂,當是宋初併。何志有,當是元嘉中復立。



 蘇平令,《永初郡國》有,何志不注置立。徐曰藉平。



 叔安令,《永初郡國》有,何志不注置立。



 夫寧令,《永初郡國》有,何志不注置立。



 武林令,文帝立。


鬱林太守,秦桂林郡,屬尉他,武帝元鼎六年復,更名。《永初郡國》有安遠、程安、威定
 \gezhu{
  三縣別見}
 、中胄、歸化五縣。中胄疑即桂林之中溜。歸化,二漢、《晉太康地志》無,疑是是江左所立。何志無中胄、歸化,餘三縣屬桂林,徐志同。今領縣十七,戶一千一百二十一,口五千七百二十七。去州水一千六百;去京都水七千九百。



 布山令,漢舊縣。



 領方令,漢舊縣,吳改曰臨浦,晉武復舊。



 阿林令,漢舊縣。



 鬱平令,吳立曰陰平,晉武太康元年更名。



 新邑令,吳立。



 建初令,《永初郡國》有,何志不注置立,徐同。



 賓平令,《永初郡國》有,何志不注置立。



 威化令,《永初郡國》有,何志不注置立。



 新林令,《永初郡國》有,何志不注置立。



 龍平令,《永初郡國》有,何志不注置立。



 安始令,吳立曰建始,晉武帝太康元年更名。



 懷安令,何志吳改,未知先何名。《吳錄》地理無懷安縣名。《太康地志》無。《永初郡國》有。



 晉平令,吳立曰長平,晉武帝太康元年更名。



 綏寧令,《永初郡國》併領方,何無徐有。



 歸代令,徐志有。



 中胄令,徐志有。



 建安令,《永初郡國》有,何無,徐有。



 桂林太守,本縣名,屬鬱林。吳孫皓鳳皇三年,分鬱林,治武熙縣,不知何時徙。《永初郡國》有常安、夾陽二縣。夾陽,晉武帝太康元年分龍岡立。常安,《太康地志》有而王隱無。何、徐並無此二縣。今領縣七,戶五百五十八,口二千二百五。去州水一千五百七十五;去京都水六千八百。



 中溜令,漢舊縣,屬鬱林,《晉太康地志》無。



 龍定令,晉武帝太康元年立桂林之龍岡,疑是。《永
 初郡國》、何、徐並云龍定。



 武熙令,本曰武安,應是吳立,晉武帝太康元年更名。故屬鬱林。



 陽平令,《永初郡國》、何、徐並有。何云新置。按晉武帝太康元年,立桂林之洋平縣,疑是。



 安遠令,晉武帝太康六年立,屬鬱林。《永初郡國》猶屬鬱林,何、徐屬此。



 程安令,《永初郡國》屬鬱林,何、徐屬此。疑是江左立。



 威定令,《永初郡國》屬鬱林,何、徐屬此。疑是江左立。



 高涼太守,二漢有高涼縣,屬合浦。漢獻帝建安二十三年,吳分立,治思平縣,不知何時徙。吳又立高熙郡,太康中省併高涼,宋世又經立,尋省。《永初郡國》高涼又有石門、廣化、長度、宋康四縣。何、徐並無宋康,當是宋初所立,元嘉二十年以前省,其餘當是江左所立。領縣七,戶一千四百二十九,口八千一百二十三。去州水一千一百;去京都水六千六百。



 思平令,《晉太康地志》有。



 莫陽令,《晉太康地志》有,屬高興。



 平定令,何志有,不注置立。



 安寧令,吳立。



 羅州令,何志新立。



 西鞏令,何志新立。



 禽鄉令,何志新立。



 新會太守,晉恭帝元熙二年,分南海立。《廣州記》云:「永初
 元年,分新寧立,治盆允。」未詳孰是。領縣十二,戶一千七百三十九,口萬五百九。去州三百五十。



 宋元令,《永初郡國》無,文帝元嘉九年,割南海、新會、新寧三郡界上新民立宋安、新熙、永昌、始成、招集五縣。二十七年,改宋安為宋元。



 新熙令。



 永昌令。



 始成令。



 招集令。



 盆允令,《永初郡國》故屬南海,何、徐同。



 新夷令,吳立曰平夷,晉武帝太康元年更名,故屬南海。



 封平令,《永初郡國》云故屬新寧,何云故屬南海,徐同。



 封樂令,文帝元嘉十二年,以盆允、新夷二縣界歸化民立。



 初賓令,何志新立。



 義寧令,何志新立。



 始康令,何志新立。



 東官太守,何志故司鹽都尉,晉成帝立為郡。《廣州記》,晉成帝咸和六年,分南海立。領縣六,戶一千三百三十二,口一萬五千六百九十六。去州水三百七十;去京都水五千六百七十。



 寶安男相,《永初郡國》、何、徐並不注置立。



 安懷令,《永初郡國》、何、徐並不注置立。



 興寧令,江左立。



 海豐男相,《永初郡國》、何、徐並不注置立。



 海安男相,吳曰海寧,晉武改名。《太康地志》屬高興。



 欣樂男相,本屬南海,宋末度。



 義安太守,晉安帝義熙九年,分東官立。領縣五,戶一千一百一十九,口五千五百二十二。去州三千五百;去京都水八千九百。



 海陽令,何志晉初立。《晉太康地志》無。晉地記故屬東官。



 綏安令,何志與郡俱立。晉地記故屬東官。



 海寧令,何志與郡俱立。晉地記故屬東官。



 潮陽令,何志與郡俱立。晉地記故屬東官。



 義招令,晉安帝義熙九年,以東官五營立。



 宋康太守,本高涼西營,文帝元嘉九年立。領縣九,戶一千五百一十三,口九千一百三十一。去州水九百五十;
 去京都水五千九百七十。



 廣化令,《晉太康地志》有,屬高興,《永初郡國》屬高涼。



 單城令,何志新立。



 逐度令,何志新立。



 海鄰令,何志新立。



 化隆令,何志新立。



 開寧令,何志新立。



 綏定令,何志新立。



 石門長,何志故屬高涼。



 威覃長,徐志有。


綏建太守,文帝元嘉十三年立。孝武孝建元年,有司奏化注、永固、綏南、宋昌、宋泰五縣,舊屬綏建,中割度臨賀,相去既遠,疑還綏建。今唯有綏南,餘並無。何、徐又有新招縣,云本屬蒼梧,元嘉十九年改配。徐志晉康復有此縣,疑誤。今領縣七,
 \gezhu{
  疑}
 戶三千七百六十四,口一萬四千四百九十一。去州
 \gezhu{
  闕}
 。



 新招令,本四會之官細鄉,元嘉十三年分為縣。



 化蒙令,本四會古蒙鄉,元嘉十三年分為縣。



 懷集令,本四會之銀屯鄉,元嘉十三年分為縣。



 四會男相,漢舊縣,屬南海。



 化穆令,何志新立。



 綏南令,《永初郡國》、徐並無。



 海昌太守,文帝元嘉十六年立。何有覃化縣,徐無。領縣五,戶一千七百二十四,口四千七十四。



 去州水六百五
 十;去京都水五千四百九十四。



 寧化令,徐志新立。



 威寧令,徐志新立。



 永建令,徐志新立。



 招懷令,徐志新立。



 興定令,文帝元嘉九年立,屬新會,後度此。



 宋熙太守,文帝元嘉十八年,以交州流寓立昌國、義懷、綏寧、新建四縣為宋熙郡,今無此四縣。



 二十七年,更名
 宋隆。孝武孝建中,復改為宋熙。領縣七,戶二千八十四,口六千四百五十。去州水三百四十五;去京都水五千二百。



 平興令,徐志新立。



 初寧令,徐志新立。



 建寧令,徐志新立。



 招興令,徐志新立。



 崇化令,徐志新立。



 熙穆令,徐志新立。



 崇德令,徐志新立。



 寧浦太守,《晉太康地志》,武帝太康七年改合浦屬國都尉立。《廣州記》,漢獻帝建安二十三年,吳分鬱林立,治平山縣。《吳錄》孫休永安三年,分合浦立為合浦北部尉,領平山、興道、寧浦三縣。又云晉分平山為始定,寧浦為澗陽,未詳孰是。《永初郡國》有安廣縣,無始定縣。何、徐並無此郡。領縣六。



 澗陽令,晉武帝太康七年立。《永初郡國》作「簡陽」。



 興道令,晉武帝太康元年,以合浦北部營之連道立。《吳錄》有此縣,未詳。



 寧浦令,《晉太康地記》本名昌平,武帝太康元年更名。《吳錄》有此縣,未詳。



 吳安令,《吳錄》無。



 平山令,《晉太康地記》有。



 始定令,《晉太康地記》有,《永初郡國》無。



 晉興太守,晉元帝太興元年,分鬱林立。



 晉興。



 熙注。



 桂林。



 增翊。



 安廣。



 廣鬱。



 晉城。



 鬱陽。



 樂昌郡。



 樂昌令。



 始昌令。



 宋元令。



 樂山令。



 義立令安樂令。



 交州刺史,漢武帝元鼎六年開百越,交趾刺史治龍編。漢獻帝建安八年,改曰交州,治蒼梧廣信縣;十六年,徙治南海番禺縣。及分為廣州,治番禺。交州還治龍編。領郡八,縣五十三,戶一萬四百五十三。去京都水一萬。交趾太守,漢武帝元鼎六年開。領縣十二,戶四千二百三十三。



 龍編令,漢舊縣。



 句漏令,漢舊縣。



 朱涘令,漢舊縣。



 吳興令,吳立。



 西于令,漢舊縣。



 定安令,漢舊縣。



 望海令,漢光武建武十九年立。



 海平令,吳立曰軍平,晉武改名。



 武寧令,吳立。


羸
 \gezhu{
  力知反}
 婁令,漢舊縣。


曲昜
 \gezhu{
  音陽}
 令,漢舊縣。



 南定令,吳立曰武安,晉武改。何志無。


武平太守,吳孫皓建衡三年討扶嚴夷,以其地立。領縣六。
 \gezhu{
  上闕}
 戶一千四百九十。去州水二百一十,陸
 \gezhu{
  下闕}
 。
 \gezhu{
  上闕}
 《吳錄》無,《晉太康地志》有。



 吳定長,吳立。



 新道長,江左立。



 晉化長,江左立。


九真太守,漢武元鼎六年立。領縣十二,
 \gezhu{
  疑}
 戶二千三百二十八。去州水八百;去京都水一萬一百八十。



 移風令,漢舊縣。故名居風,吳更名。



 胥浦令,漢舊縣。



 松原令,晉武帝分建初立。



 高安令,何志晉武帝立。《太康地志》無。《吳錄》晉分常樂立。



 建初令,吳立。



 常樂令,吳立。



 軍安長,何志晉武帝立。《太康地志》無此縣,而交趾有軍平縣。



 武寧令,吳立,何志武帝立。《太康地志》無此縣,而交趾有。


都龐
 \gezhu{
  音龍}
 長,漢舊縣。《吳錄》有,《晉太康地志》無。



 寧夷長,何志晉武帝立,《太康地志》無。



 津梧長,晉武帝分移風立。



 九德太守,故屬九真,吳分立。何志領縣七,今領縣十一,戶八百九。去州水九百;去京都水一萬九百。



 浦陽令,晉武帝分陽遠立。陽遠,吳立曰陽成,太康二年更名,後省。



 九德令,何志吳立。



 咸驩令,漢舊縣。



 都禋長,何志晉武帝分九德立。



 西安長,何志晉武帝立。《太康地志》無。《吳錄》亦無。



 南陵長,何志晉武帝立。《太康地志》無,王隱有。



 越常長,何志吳立,《太康地志》無。



 宋泰令,宋末立。



 宋昌令,宋末立。



 希平令,宋末立。



 日南太守,秦象郡,漢武元鼎六年更名,吳省,晉武帝太康三年復立。領縣七,戶四百二。去州水二千四百;去京都水一萬六百九十。



 西卷令,漢舊縣作「手卷」。



 盧容令,漢舊縣。



 象林令,漢舊縣。



 壽泠令,晉武太康十年,分西卷立。



 朱吾令,漢舊縣。



 無勞長,晉武分北景立。



 北景長,漢舊縣。



 義昌郡,宋末立。



 宋平郡,孝武世,分日南立宋平縣,後為郡。



 越州刺史,明帝泰始七年立。



 百梁太守,新立。



 心龍蘇太守,新立。



 永寧太守,新立。



 安昌太守,新立。



 富昌太守,新立。



 南流太守,新立。



 臨漳太守,先屬廣州。



 合浦太守,漢武帝立,孫權黃武七年,更名珠官,孫亮復舊。先屬交州。領縣七,戶九百三十八。去京都水一萬八百。



 合浦令,漢舊縣。



 徐聞令,故屬朱崖。晉平吳,省朱崖,屬合浦。



 朱官長,吳立,「朱」作「珠」。



 蕩昌長,晉武分合浦立。



 硃盧長,吳立。



 晉始長,晉武帝立。



 新安長,江左立。



 宋壽太守,先屬交州。



\end{pinyinscope}