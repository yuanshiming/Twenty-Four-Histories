\article{卷七十六列傳第三十六 硃修之 宗愨 王玄謨}

\begin{pinyinscope}

 朱脩之,字恭祖,義興平氏人也。曾祖燾,晉平西將軍。祖序,豫州刺史。父諶,益州刺史。脩之自州主簿遷司徒從事中郎,文帝謂曰:「卿曾祖昔為王導丞相中郎,卿今又
 為王弘中郎,可謂不忝爾祖矣。」後隨到彥之北伐。彥之自河南回,留脩之戍滑臺,為虜所圍,數月糧盡,將士熏鼠食之,遂陷於虜。初,脩之母聞其被圍既久,常憂之,忽一旦乳汁驚出,母號泣告家人曰:「吾今已老,忽復有乳汁,斯不祥矣。吾兒其不利乎!」後問至,修之果以此日陷沒。



 托跋燾嘉其守節,以為侍中,妻以宗室女。脩之潛謀南歸,妻疑之,每流涕問其意,脩之深嘉其義,竟不告也。後鮮卑馮弘稱燕王,治黃龍城,托跋燾伐之,脩之與同
 沒人邢懷明並從。又有徐卓者,復欲率南人竊發,事泄被誅。脩之、懷明懼奔馮弘,弘不禮。留一年,會宋使傳詔至,脩之名位素顯,傳詔見即拜之。彼國敬傳詔,謂為「天子邊人」,見其致敬於脩之,乃始加禮。時魏屢伐弘,或說弘遣人脩之歸求救,遂遣之。泛海至東萊,遇猛風柁折,垂以長索,船乃復正。海師望見飛鳥,知其近岸,須臾至東萊。



 元嘉九年,至京邑,以為黃門侍郎,累遷江夏內史。雍州刺史劉道產卒,群蠻大動,脩之為征西司馬討蠻,失
 利。孝武初,為寧蠻校尉、雍州刺史,加都督。脩之在政寬簡,士眾悅附。及荊州刺史南郡王義宣反,檄脩之舉兵;脩之偽與之同,而遣使陳誠於帝。帝嘉之,以為荊州刺史,加都督。義宣聞脩之不與己同,乃以魯秀為雍州刺史,擊襄陽。脩之命斷馬鞍山道,秀不得前,乃退。及義宣敗於梁山,單舟南走,脩之率眾南定遺寇。時竺超民執義宣,修之至,乃殺之,以功封南昌縣侯。



 修之治身清約,凡所贈貺,一無所受。有餉,或受之,而旋與佐吏賭之,終
 不入己,唯以撫納群蠻為務。徵為左民尚書,轉領軍將軍。去鎮,秋毫不犯,計在州然油及牛馬穀草,以私錢十六萬償之。然性儉克少恩情,姊在鄉里,饑寒不立,修之未嘗供贍。嘗往視姊,姊欲激之,為設菜羹粗飯,脩之曰:「此乃貧家好食。」



 致飽而去。先是,新野庾彥達為益州刺史,攜姊之鎮,分祿秩之半以供贍之,西土稱焉。



 脩之後墜車折腳,辭尚書,領崇憲太僕,仍加特進、金紫光祿大夫。以腳疾不堪獨行,特給扶侍。卒,贈侍中,特進如故。謚
 貞侯。



 宗愨,字元幹,南陽人也。叔父炳,高尚不仕。愨年少時,炳問其志,愨曰:「願乘長風破萬里浪。」炳曰:「汝不富貴,即破我家矣。」兄泌娶妻,始入門,夜被劫。愨年十四,挺身拒賊,賊十餘人皆披散,不得入室。



 時天下無事,士人並以文義為業,炳素高節,諸子群從皆好學,而愨獨任氣好武,故不為鄉曲所稱。江夏王義恭為征北將軍、南兗州刺史,愨隨鎮廣陵。時從兄綺為征北府主簿,綺嘗入直,而
 給吏牛泰與綺妾私通,愨殺泰,綺壯其意,不責也。



 元嘉二十二年,伐林邑,愨自奮請行。義恭舉愨有膽勇,乃除振武將軍,為安西參軍蕭景憲軍副,隨交州刺史檀和之圍區粟城。林邑遣將范毗沙達來救區粟,和之遣偏軍拒之,為賊所敗。又遣愨,愨乃分軍為數道,偃旗潛進,討破之,拔區粟,入象浦。林邑王范陽邁傾國來拒,以具裝被象,前後無際,士卒不能當。愨曰:「吾聞師子威服百獸。」乃製其形,與象相禦,象果驚奔,眾因潰散,遂克林邑。



 收其異寶雜物,不可勝計。愨一無所取,衣櫛蕭然,文帝甚嘉之。



 後為隨郡太守,雍州蠻屢為寇,建威將軍沈慶之率愨及柳元景等諸將,分道攻之,群蠻大潰。又南新郡蠻帥田彥生率部曲反叛,焚燒郡城,屯據白楊山。元景攻之未能下,愨率其所領先登,眾軍隨之,群蠻由是畏服。二十年,孝武伐元凶,以愨為南中郎諮議參軍,領中兵。孝武即位,以為左衛將軍,封洮陽侯,功次柳元景。



 孝建中,累遷豫州刺史,監五州諸軍事。先是,鄉人庾業,
 家甚富豪,方丈之膳,以待賓客;而愨至,設以菜菹粟飯,謂客曰:「宗軍人,慣啖粗食。」愨致飽而去。



 至是業為愨長史,帶梁郡,愨待之甚厚,不以前事為嫌。



 大明三年,竟陵王誕據廣陵反,愨表求赴討,乘驛詣都,面受節度;上停輿慰勉,愨聳躍數十,左右顧盻,上壯之。及行,隸車騎大將軍沈慶之。初,誕誑其眾云:「宗愨助我。」及愨至,躍馬繞城呼曰:「我宗愨也!」事平,入為左衛將軍。



 五年,從獵墮馬,腳折不堪朝直,以為光祿大夫,加金紫。愨有佳牛堪進
 御,官買不肯賣,坐免官。明年,復職。廢帝即位,為寧蠻校尉、雍州刺史,加都督。卒,贈征西將軍,謚曰肅侯。泰始二年,詔以愨配食孝武廟。子羅雲,卒,子元寶嗣。



 王玄謨,字彥德,太原祁人也。六世祖宏,河東太守,綿竹侯,以從叔司徒允之難,棄官北居新興,仍為新興、鴈門太守,其自敘云爾。祖牢,仕慕容氏為上谷太守,陷慕容德,居青州。父秀,早卒。



 玄謨幼而不群,世父蕤有知人鑒,常笑曰:「此兒氣概高亮,有太尉彥雲之風。」



 武帝臨徐州,
 闢為從事史,與語異之。少帝末,謝晦為荊州,請為南蠻行參軍、武昌太守。晦敗,以非大帥見原。元嘉中,補長沙王義欣鎮軍中兵將軍,領汝陰太守。



 時虜攻陷滑臺,執朱修之以歸。玄謨上疏曰:「王途始開,隨復淪塞,非惟天時,抑亦人事。虎牢、滑臺,豈惟將之不良,抑亦本之不固。本之不固,皆由民憚遠役。



 臣請以西陽之魯陽,襄陽之南鄉,發甲卒,分為兩道,直趣淆、澠,徵士無遠徭之思,吏士有屢休之歌。若欲以東國之眾,經營牢、洛,道途既遠,
 獨克實難。」玄謨每陳北侵之策,上謂殷景仁曰:「聞王玄謨陳說,使人有封狼居意。」後為興安侯義賓輔國司馬、彭城太守。義賓薨,玄謨上表,以彭城要兼水陸,請以皇子撫臨州事,乃以孝武出鎮。



 及大舉北征,以玄謨為寧朔將軍,前鋒入河,受輔國將軍蕭斌節度。玄謨向確磝,戍主奔走,遂圍滑臺,積旬不克。虜主拓跋燾率大眾號百萬,鞞鼓之聲,震動天地。玄謨軍眾亦盛,器械甚精,而玄謨專依所見,多行殺戮。初圍城,城內多茅屋,眾求以
 火箭燒之,玄謨恐損亡軍實,不從。城中即撤壞之,空地以為窟室。及魏救將至,眾請發車為營,又不從。將士多離怨,又營貨利,一匹布責人八百梨,以此倍失人心。及拓跋燾軍至,乃奔退,麾下散亡略盡。蕭斌將斬之,沈慶之固諫曰:「佛狸威震天下,控弦百萬,豈玄謨所能當。且殺戰將以自弱,非良計也。」



 斌乃止。初,玄謨始將見殺,夢人告曰:「誦《觀音經》千遍,則免。」既覺,誦之得千遍,明日將刑,誦之不輟,忽傳呼停刑。遣代守確磝,江夏王義恭為
 征討都督,以為確磝不可守,召令還,為魏軍所追,大破之,流矢中臂。二十八年正月,還至歷城,義恭與玄謨書曰:「聞因敗為成,臂上金瘡,得非金印之徵也。」



 元凶弒立,玄謨為益州刺史。孝武伐逆,玄謨遣濟南太守垣護之將兵赴義。事平,除徐州刺史,加都督。及南郡王義宣與江州刺史臧質反,朝庭假玄謨輔國將軍,拜豫州刺史,與柳元景南討。軍屯梁山,夾岸築偃月壘,水陸待之。義宣遣劉諶之就臧質,陳軍城南,玄謨留老弱守城,悉精
 兵接戰,賊遂大潰。加都督、前將軍,封曲江縣侯。中軍司馬劉沖之白孝武,言:「玄謨在梁山,與義宣通謀。」上意不能明,使有司奏玄謨多取寶貨,虛張戰簿,與徐州刺史垣護之並免官。



 尋復為豫州刺史。淮上亡命司馬黑石推立夏侯方進為主,改姓李名弘,以惑眾,玄謨討斬之。遷寧蠻校尉、雍州刺史,加都督。雍土多僑寓,玄謨請土斷流民,當時百姓不願屬籍,罷之。其年,玄謨又令九品以上租,使貧富相通,境內莫不嗟怨。



 民間訛言玄謨欲
 反,時柳元景當權,元景弟僧景為新城太守,以元景之勢,制令南陽、順陽、上庸、新城諸郡並發兵討玄謨。玄謨令內外晏然,以解眾惑,馳啟孝武,具陳本末。帝知其虛,馳遣主書吳喜公撫慰之,又答曰:「梁山風塵,初不介意,君臣之際,過足相保,聊復為笑,伸卿眉頭。」玄謨性嚴,未嘗妄笑,時人言玄謨眉頭未曾伸,故帝以此戲之。後為金紫光祿大夫,領太常。及建明堂,以本官領起部尚書,又領北選。



 孝武狎侮群臣,隨其狀貌,各有比類,多鬚者
 謂之羊。顏師伯缺齒,號之曰齴。



 劉秀之儉吝,呼為老慳。黃門侍郎宗靈秀體肥,拜起不便,每至集會,多所賜與,欲其瞻謝傾踣,以為歡笑。又刻木作靈秀父光祿勳叔獻像,送其家事。柳元景、垣護之並北人,而玄謨獨受「老傖」之目。凡所稱謂,四方書疏亦如之。嘗為玄謨作四時詩曰:「堇荼供春膳,粟漿充夏飧。瓟醬調秋菜,白醝解冬寒。」又寵一昆侖奴子,名曰主。常在左右,令以杖擊群臣,自柳元景以下,皆罹其毒。



 玄謨尋遷平北將軍、徐州
 刺史,加都督。時北土饑饉,乃散私穀十萬斛、牛千頭以振之。轉領軍將軍。孝武崩,與柳元景等俱受顧命,以外監事委玄謨。時朝政多門,玄謨以嚴直不容,徙青、冀二州刺史,加都督。少帝既誅顏師伯、柳元景等,狂悖益甚,以領軍征玄謨。子侄咸勸稱疾,玄謨曰:「吾受先帝厚恩,豈可畏禍茍免。」遂行。及至,屢表諫諍,又流涕請緩刑去殺,以安元元。少帝大怒。



 明帝即位,禮遇甚優。時四方反叛,以玄謨為大統,領水軍南討,以腳疾,聽乘輿出入。尋
 除車騎將軍、江州刺史,副司徒建安王於赭圻,賜以諸葛亮筒袖鎧。



 頃之,為左光祿大夫、開府儀同三司,領護軍。遷南豫州刺史,加都督。玄謨性嚴克少恩,而將軍宗越御下更苛酷,軍士謂之語曰:「寧作五年徒,不逢王玄謨。玄謨猶自可,宗越更殺我。」年八十一薨,謚曰莊公。子深早卒,深子繢嗣。



 史臣曰:脩之、宗愨,皆以將帥之材,懷廉潔之操,有足稱焉。玄謨雖苛克少恩,然觀其大節,亦足為美。當少帝失
 道,多所殺戮,而能冒履不測,傾心輔弼,斯可謂忘身徇國者歟!



\end{pinyinscope}