\article{卷六十五列傳第二十五 吉翰 劉道產 杜驥 申恬}

\begin{pinyinscope}

 吉翰,字休文,馮翊池陽人也。初為龍驤將軍道憐參軍,隨府轉征虜左軍參軍,員外散騎侍郎。隨道憐北征廣固,賜爵建城縣五等男。轉道憐驃騎中兵參軍,從事中
 郎。為將佐十餘年,清謹剛正,甚為高祖所知賞。永初三年,轉道憐太尉司馬。



 太祖元嘉元年,出督梁、南秦二州諸軍事、龍驤將軍、西戎校尉、梁、南秦二州刺史。三年,仇池氐楊興平遣使歸順,并兒弟為質,翰遣始平太守龐咨據武興。



 仇池大帥楊玄遣弟難當率眾拒咨,又遣將強鹿皮向白水。咨擊破,難當等並退走。



 其年,徙督益、寧二州、梁州之巴西、梓潼、宕渠、南漢中、秦州之安固、懷寧六郡諸軍事、益州刺史,將軍如故。在益州著美績,甚得方
 伯之體,論者稱之。六年,以老疾徵還,除彭城王義康司徒司馬,加輔國將軍。



 時太祖經略河南,以翰為持節、監司、雍、并三州諸軍事、司州刺史,將軍如故。會前鋒諸軍到彥之等敗退,明年,復為司徒司馬,將軍如故。其年,又假節、監徐、兗二州、豫州之梁郡諸軍事、徐州刺史,將軍如故。時有死罪囚,典簽意欲活之,因翰入關齎呈其事。翰省訖,語「今且去,明可便呈」。明旦,典簽不敢復入,呼之乃來,取昨所呈事視訖,謂之曰:「卿意當欲宥此囚
 死命。昨於齋坐見其事,亦有心活之。但此囚罪重,不可全貸,既欲加恩,卿便當代任其罪。」因命左右收典簽付獄殺之,原此囚生命。其刑政如此,其下畏服,莫敢犯禁。明年卒官,時年六十。追贈征虜將軍,持節、監、刺史如故。



 劉道產,彭城呂人,太尉咨議參軍簡之子也。簡之事在弟子《康祖傳》。道產初為輔國參軍,無錫令,在縣有能名。高祖版為中軍行參軍,又為道憐驃騎參軍,襲父爵晉安縣五等侯。廣州群盜因刺史謝道欣死為寇,攻沒州
 城,道憐加道產振武將軍南討,會始興謙之已平廣州,道產未至而反。



 元年,除寧遠將軍、巴西、梓潼二郡太守。郡人黃公生、任肅之、張石之等並譙縱餘燼,與姻親侯攬、羅奧等招引白水氐,規欲為亂。道產誅公生等二十一家,宥其餘黨。還為彭城王義康驃騎中兵參軍。元嘉三年,督梁、南秦二州諸軍事、寧遠將軍、西戎校尉、梁、南秦二州刺史。在州有惠化,關中流民,前後出漢川歸之者甚多。六年,道產表置隴西、宋康二郡以領之。七年,徵
 為後軍將軍。明年,遷竟陵王義宣左將軍咨議參軍,仍為持節、督雍、梁、南秦三州、荊州之南陽、竟陵、順陽、襄陽、新野、隨六郡諸軍事、寧遠將軍、寧蠻校尉、雍州刺史、襄陽太守。



 善於臨民,在雍部政績尤著,蠻夷前後叛戾不受化者,並皆順服,悉出緣沔為居。



 百姓樂業,民戶豐贍,由此有《襄陽樂歌》,自道產始也。



 十三年,進號輔國將軍。十九年卒,追贈征虜將軍,謚曰襄侯。道產惠澤被於西土,及喪還,諸蠻皆備衰絰,號哭追送,至于沔口。荊州刺
 史衡陽王義季啟太祖曰:「故輔國將軍劉道產患背癰,疾遂不救。道產自鎮漢南,境接凶寇,政績既著,威懷兼舉。年時猶可,方宣其用,奄至殞沒,傷怨特深。伏惟聖懷,愍惜兼至。」



 長子延孫,別有傳。延孫弟延熙,因延孫之廕,大明中,為司徒右長史,黃門郎,臨海、義興太守。泰始初,與四方同反,伏誅。



 道產弟道錫,巴西、梓潼二郡太守。元嘉十八年,為氐寇所攻,道錫保城退敵,太祖嘉之。下詔曰:「前者兵寇攻逼,邊情波駭,廣威將軍、巴西、梓潼二郡
 太守劉道錫,將率文武,盡心固守,保全之績,厥效可書。可冠軍。咨議參軍、前建威將軍、晉壽太守申坦,孤城弱眾,厲志致果,死傷參半,壯氣不衰,雖力屈陷沒,在誠宜甄。可建威將軍、巴西梓潼二郡太守。」初,氐寇至,城內眾寡,道錫募吏民守城,復租布二十年。及賊退,朝議:「賊雖攻城,一戰便走,聽依本要,於事為優。」右衛將軍沈演之、丹陽尹羊玄保、後軍長史范曄並謂:「宜隨功勞裁量,不可全用本誓,多者不得過十年。」從之。二十一年,遷揚烈
 將軍、廣州刺史。二十七年,坐貪縱過度,自杖治中荀齊文垂死,乘輿出城行,與阿尼同載,為有司所糾。值赦,明年散征。又以赦後餘贓,收下廷尉,被宥病卒。



 杜驥,字度世,京兆杜陵人也。高祖預,晉征南將軍。曾祖耽,避難河西,因仕張氏。苻堅平涼州,父祖始還關中。兄坦,頗涉史傳。高祖征長安,席卷隨從南還。太祖元嘉中,任遇甚厚,歷後軍將軍,龍驤將軍,青、冀二州刺史,南平王鑠右將軍司馬。晚度北人,朝廷常以傖荒遇之,雖復
 人才可施,每為清塗所隔,坦以此慨然。嘗與太祖言及史籍,上曰:「金日磾忠孝淳深,漢朝莫及,恨今世無復如此輩人。」坦曰:「日磾之美,誠如聖詔。假使生乎今世,養馬不暇,豈辦見知。」



 上變色曰:「卿何量朝廷之薄也。」坦曰:「請以臣言之。臣本中華高族,亡曾祖晉氏喪亂,播遷涼土,世葉相承,不殞其舊。直以南度不早,便以荒傖賜隔。日磾胡人,身為牧圉,便超入內侍,齒列名賢。聖朝雖復拔才,臣恐未必能也。」上默然。



 北土舊法,問疾必遣子弟。驥
 年十三,父使候同郡韋華。華子玄有高名,見而異之,以女妻焉。桂陽公義真鎮長安,辟為州主簿,後為義真車騎行參軍,員外散騎侍郎,江夏王義恭撫軍刑獄參軍,尚書都官郎,長沙王義欣後軍錄事參軍。



 元嘉七年,隨到彥之入河南,加建武將軍。索虜撤河南戍悉歸河北,彥之使驥守洛陽。洛陽城不治既久,又無糧食,及彥之敗退,驥欲棄城走,慮為太祖所誅。



 初,高祖平關洛,致鐘虡舊器南還,一大鐘墜洛水。至是太祖遣將姚聳夫領
 千五百人迎致之。時聳夫政率所領牽鐘於洛水,驥乃誑之曰:「虜既南渡,洛城勢弱,今修理城池,並已堅固,軍糧又足,所乏者人耳。君率眾見就,共守此城,大功既立,取鐘無晚。」聳夫信之,率所領就驥。既至,見城不可守,又無糧食,於是引眾去。



 驥亦委城南奔,白太祖曰:「本欲以死固守,姚聳夫及城便走,人情沮敗,不可復禁。」上大怒,使建威將軍鄭順之殺聳夫於壽陽。聳夫,吳興武康人。勇果有氣力,宋世偏裨小將莫及。始隨到彥之北伐,與
 虜遇,聳夫手斬托跋燾叔父英文特勒首,燾以馬百匹贖之。



 以驥為通直郎,射聲校尉,世祖征虜咨議參軍。十七年,出督青、冀二州、徐州之東莞、東安二郡諸軍事、寧遠將軍、青、冀二州刺史。在任八年,惠化著於齊土。自義熙至于宋末,刺史唯羊穆之及驥,為吏民所稱詠。二十四年,徵左軍將軍,兄坦代為刺史,北土以為榮焉。坦長子琬為員外散騎侍郎,太祖嘗有函詔敕坦,琬輒開視。信未發又追取之,敕函已發,大相推檢。丞都答云:「諸郎
 開視。」上遣主書詰責,驥答曰:「開函是臣第四子季文,伏待刑坐。」上特原不問。二十七年,卒,時年六十四。



 長子長文,早卒。第五子幼文,薄於行。太宗初,以軍功為驍騎將軍,封邵陽縣男,食邑三百戶。尋坐巧佞奪爵。後以發太尉廬江王禕謀反事,拜黃門侍郎。出為輔國將軍、梁、南秦二州刺史。廢帝元徽中,為散騎常侍。幼文所蒞貪橫,家累千金,女伎數十人,絲竹晝夜不絕,與沈勃、孫超之居止接近,常相從,又並與阮佃夫厚善。佃夫死,廢帝深
 疾之。帝微行夜出,輒在幼文門牆之間,聽其弦管,積久轉不能平,於是自率宿衛兵誅幼文、勃、超之等。幼文兄叔文為長水校尉,及諸子姪在京邑方鎮者並誅。唯幼文兄季文、弟希文等數人,逃亡得免。



 申恬,字公休,魏郡魏人也。曾祖鐘,為石虎司徒。高祖平廣固,恬父宣、宣從父兄永皆得歸國,並以幹用見知。永歷青、兗二州刺史。高祖踐祚,拜太中大夫。



 宣,太祖元嘉初,亦歷兗、青二州刺史。恬兄謨,與朱修之守滑臺,為虜
 所沒,後得叛還。元嘉中,為竟陵太守。



 恬初為驃騎道憐長兼行參軍。高祖踐祚,拜東宮殿中將軍,度還臺。直省十載,不請休息。轉員外散騎侍郎,出為綏遠將軍、下邳太守。轉在北海,加寧遠將軍。



 所至皆有政績。又為北譙、梁二郡太守,將軍如故。郡境邊接任榛,屢被寇抄。恬到,密知賊來,仍伏兵要害,出其不意,悉皆禽殄。元嘉十二年,遷督魯、東平、濟北三郡軍事、泰山太守,將軍如故。惠威兼著,吏民便之。臨川王義慶鎮江陵,為平西中兵參軍、
 河東太守。衡陽王義季代義慶,又度安西府,加寧朔將軍。召拜太子屯騎校尉,母憂去職。



 二十一年,冀州移鎮歷下,以恬督冀州、青州之濟南、樂安、太原三郡諸軍事、揚烈將軍、冀州刺史,明年,加濟南太守。時又遷換諸郡守,恬上表曰:「伏聞朝恩當加臣濟南太守,仰惟優旨,荒心散越。臣殃咎之餘,遭蒙踰忝,寵私罔己,復兼今授,豈其愚迷,所能上答。臣近至止,即履行所統,究其形宜。河、濟之間,應置戍扞,其中四處,急須修立,甕口故城,又是
 要所,宜移太原,委以邊事。緣山諸邏,並得除省,防衛綏懷,利便非一。呂綽誠效益著,深同臣意,百姓聞者,咸皆附說,急有同異,二三未宜。但房紹之蒞郡經年,軍民粗狎,改以帶臣,有乖永事。遠牽太原,於民為苦。而甕口之計,復成交互,人情非樂,容有不安。疆場威刑,患不開廣,若得依先處分,公私允緝。」上從之。詔有司曰:「恬所陳當是事宜,近諸除授可悉停。」



 北虜入寇,恬摧擊之,為虜所破,被徵還都。二十七年,起為通直常侍。是歲,索虜南寇,
 其武昌王向青州。遣恬援東陽,因與輔國司馬、齊郡太守龐秀之保城固守。蕭斌遣青州別駕解榮之率垣護之還援恬等,仍傍南山得入。賊朝來脅城,日晚輒退。城內乃出車北門外,環塹為營,欲挑戰,賊不敢逼。停五日,東過抄略清河郡及驛道南數千家,從東安、東莞出下邳。下邳太守垣閬閉城距守,保全二千餘家。



 虜退,以恬為寧朔將軍、山陽太守。善於治民,所蒞有績。世祖踐阼,遷青州刺史,將軍如故。尋加督徐州之東莞、東安二郡
 諸軍事。明年,又督冀州。齊地連歲興兵,百姓凋弊,恬初防衛邊境,勸課農桑,二三年間,遂皆優實。性清約,頻處州郡,妻子不免飢寒,世以此稱之。進號輔國將軍。



 孝建二年,遷督豫州軍事、寧朔將軍、豫州刺史。明年,疾病徵還,於道卒,時年六十九。死之日,家無遺財。子實,南譙郡太守,早卒。



 謨子元嗣,海陵、廣陵太守。元嗣弟謙,太始初,以軍功歷軍校,官至輔國將軍、臨川內史。永子坦,自巴西、梓潼遷梁、南秦二州刺史。元嘉二十六年,為世
 祖鎮軍咨議參軍,與王玄謨圍滑臺不克,免官。青州刺史蕭斌板行建威將軍、濟南、平原二郡太守,復攻確磝,敗退,下歷城。蕭思話起義討元兇,假坦輔國將軍,為前鋒。世祖至新亭,坦亦進克京城。孝建初,為太子右衛率,寧朔將軍、徐州刺史。



 大明元年,虜寇兗州,世祖遣太子衛率薛安都、新除東陽太守沈法系北討,至兗州,虜已去。坦建議:「任榛亡命,屢犯邊民,軍出無功,宜因此翦撲。」上從之。亡命先已聞知,舉村逃走,安都與法系坐白衣
 領職,坦棄市。群臣為之請,莫能得。



 將行刑,始興公沈慶之入市抱坦慟哭曰:「卿無罪,為朝廷所枉誅,我入市亦當不久。」市官以白上,乃原生命,系尚方。尋被宥,復為驍騎將軍,病卒。



 子令孫,前廢帝景和中,為永嘉王子仁左軍司馬、廣陵太守。太宗以為寧朔將軍、徐州刺史,討薛安都。行至淮陽,即與安都合。弟闡,時為濟陰太守,戍睢陵城,奉順不同安都,安都攻圍不能克。會令孫至,遣往睢陵令說闡降,闡既降,殺之,令孫亦見殺。先是,清河崔
 諲亦以將吏見知高祖,永初末,為振威將軍、東萊太守。少帝初,亡命司馬靈期、司馬順之千餘人圍東萊,諲擊之,斬靈期等三十級。



 太祖元嘉中,至青州刺史。



 史臣曰:漢之良吏,居官者或長子孫,孫、曹之世,善職者亦二三十載,皆敷政以盡民和,興讓以存簡久。及晚代風烈漸衰,非才有起伏,蓋所遭之時異也。劉道產之在漢南,歷年逾十,惠化流於樊沔,頗有前世遺風,故能樹績垂名,斯為美矣!



\end{pinyinscope}