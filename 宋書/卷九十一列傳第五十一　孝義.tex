\article{卷九十一列傳第五十一 孝義}

\begin{pinyinscope}

 《易》曰:「立人之道,曰仁與義。」夫仁義者,合君親之至理,實忠孝之所資。雖義發因心,情非外感,然企及之旨,聖哲詒言。至於風漓化薄,禮違道喪,忠不樹國,孝亦愆家,而
 一世之民,權利相引;仕以勢招,榮非行立,乏翱翔之感,棄舍生之分;霜露未改,大痛已忘於心,名節不變,戎車遽為其首。斯並斬訓之理未弘,汲引之途多闕。若夫情發於天,行成乎己,損軀舍命,濟主安親,雖乘理暗至,匪由勸賞,而宰世之人,曾微誘激。乃至事隱閭閻,無聞視聽,故可以昭被圖篆,百不一焉。今采綴湮落,以備闕文云爾。



 龔穎,遂寧人也。少好學,益州刺史毛璩辟為勸學從事。
 璩為譙縱所殺,故佐吏並逃亡,穎號哭奔赴,殯送以禮。縱後設宴延穎,不獲已而至。樂奏,穎流涕起曰:「北面事人,亡不能死,何忍聞舉樂,蹈跡逆亂乎!」縱大將譙道福引出,將斬之。道福母即穎姑,跣出救之,故得免。縱既僭號,備禮徵,又不至。乃收穎付獄,脅以兵刃,執志彌堅,終無回改。至于蜀平,遂不屈節。



 其後刺史至,輒加辟引,歷府參軍,州別駕從事史。太祖元嘉二十四年,刺史陸徵上表曰:「臣聞運纏明夷,則艱貞之節顯;時屬棟橈,則獨
 立之操彰。昔之元興,皇綱弛紊,譙縱乘釁,肆虐巴、庸,害殺前益州刺史毛璩,竊據蜀土,涪、岷士庶,怵迫受職。璩故吏襲穎,獨秉身貞白,抗志不撓,殯送舊君,哀敬盡禮,全操九載,不染偽朝。縱雖殘凶,猶重義概,遂延以旌命,劫以兵威。穎忠誠奮發,辭色方壯,雖桎梏在身,踐危愈信其節;白刃臨頸,見死不更其守。若王蠋之抗辭燕軍,同周苛之肆詈楚王,方之於穎,蔑以加焉。誠當今之忠壯,振古之遺烈。而名未登於王府,爵猶齒於鄉曹,斯實
 邊氓遠土,所為於邑。臣過叨恩私,宣風萬里,志存砥竭,有懷必聞,故率愚愨,舉其所知。追懼紕妄,伏增悚慄。」穎遂不被朝命,終於家。



 劉瑜,歷陽人也。七歲喪父,事母至孝。年五十二,又喪母,三年不進鹽酪,號泣晝夜不絕聲。勤身運力,以營葬事。服除後,二十餘年布衣蔬食,言輒流涕。



 常居墓側,未嘗暫違。太祖元嘉初,卒。



 賈恩,會稽諸暨人也。少有志行,為鄉曲所推重。元嘉三
 年,母亡,居喪過禮。



 未葬,為鄰火所逼,恩及妻桓氏號哭奔救,鄰近赴助,棺櫬得免。恩及桓俱見燒死。



 有司奏改其里為孝義里,蠲租布三世。追贈天水部顯親縣左尉。



 郭世道,會稽永興人也。生而失母,父更娶,世道事父及後母,孝道淳備。年十四,又喪父,居喪過禮,殆不勝喪。家貧,無產業,傭力以養繼母。婦生一男,夫妻共議曰:「勤身供養,力猶不足,若養此兒,則所費者大。」乃垂泣瘞之。母亡,負土成墳,親戚咸共賻助,微有所受。葬畢,傭賃倍還
 先直。服除後,哀戚思慕,終身如喪者,以為追遠之思,無時去心,故未嘗釋衣。仁厚之風,行於鄉黨,鄰村小大,莫有呼其名者。嘗與人共於山陰市貨物,誤得一千錢,當時不覺,分背方悟。請其伴求以此錢追還本主,伴大笑不答。世道以己錢充數送還之,錢主驚嘆,以半直與世道,世道委之而去。元嘉四年,遣大使巡行天下,散騎常侍袁愉表其淳行,太祖嘉之,敕郡榜表閭門,蠲其稅調,改所居獨楓里為孝行焉。太守孟顗察孝廉,不就。



 子
 原平,字長泰,又稟至行,養親必己力。性閑木功,傭賃以給供養。性謙虛,每為人作匠,取散夫價。主人設食,原平自以家貧,父母不辦有肴味,唯飧鹽飯而已。若家或無食,則虛中竟日,義不獨飽;要須日暮作畢,受直歸家,於里中買糴,然後舉爨。父抱篤疾彌年,原平衣不解帶,口不嘗鹽菜者,跨積寒暑;又未嘗睡臥。



 父亡,哭踴慟絕,數日方蘇。以為奉終之義,情禮所畢,營壙凶功,不欲假人。本雖智巧,而不解作墓,乃訪邑中有營墓者,助人運力,
 經時展勤,久乃閑練。又自賣十夫,以供眾費。窀穸之事,儉而當禮,性無術學,因心自然。葬畢,詣所買主,執役無懈,與諸奴分務。每讓逸取勞,主人不忍使,每遣之,原平服勤,未曾暫替。



 所餘私夫,傭賃養母,有餘聚以自贖。本性智巧,既學構塚,尤善其事,每至吉歲,求者盈門。原平所赴,必自貧始,既取賤價,又以夫日助之。父喪既終,自起兩間小屋,以為祠堂。每至節歲烝嘗,於此數日中,哀思,絕飲粥。父服除後,不復食魚肉。於母前,示有所啖,在
 私室,未曾妄嘗。自此迄終,三十餘載。高陽許瑤之居在永興,罷建安郡丞還家,以綿一斤遺原平。原平不受,送而復反者前後數十。



 瑤之乃自往曰:「今歲過寒,而建安綿好,以此奉尊上下耳。」原平乃拜而受之。



 及母終,毀瘠彌甚,僅乃免喪。墓前有數十畝田,不屬原平,每至農月,耕者恒裸袒,原平不欲使人慢其墳墓,乃販質家資,貴買此田。三農之月,輒束帶垂泣,躬自耕墾。每出市賣物,人問幾錢,裁言其半,如此積時,邑人皆共識悉,輒加本
 價與之。彼此相讓,欲買者稍稍減價,要使微賤,然後取直。居宅下濕,繞宅為溝,以通淤水。宅上種少竹,春月夜有盜其筍者,原平偶起見之,盜者奔走墜溝。原平自以不能廣施,至使此人顛沛,乃於所植竹處溝上立小橋,令足通行,又采筍置籬外。鄰曲慚愧,無復取者。



 太祖崩,原平號哭致慟,日食麥料一枚,如此五日。人或問之曰:「誰非王民,何獨如此?」原平泣而答曰:「吾家見異先朝,蒙褒贊之賞,不能報恩,私心感慟耳。」又以種爪為業。世祖
 大明七年大旱,瓜瀆不復通船,縣官劉僧秀愍其窮老,下瀆水與之。原平曰:「普天大旱,百姓俱困,豈可減溉田之水,以通運瓜之船。」



 乃步從他道往錢唐貨賣。每行來,見人牽埭未過,輒迅楫助之;己自引船,不假旁力。若自船已渡,後人未及,常停住須待,以此為常。嘗於縣南郭鳳埭助人引船,遇有相鬥者,為吏所錄,聞者逃散,唯原平獨住。吏執以送縣,縣令新到,未相諳悉,將加嚴罰。原平解衣就罪,義無一言。左右小大咸稽顙請救,然後得
 免。由來不謁官長,自此以後,乃脩民敬。



 太守王僧郎察教廉,不就。太守蔡興宗臨郡,深加貴異,以私米饋原平及山陰朱百年妻,教曰:「秩年之貺,著自國書,餼貧之典,有聞甲令。況高柴窮老,萊婦屯暮者哉。永興郭原平世稟孝德,洞業儲靈,深仁絕操,追風曠古,棲貞處約,華耇方嚴。山陰朱百年道終物表,妻孔耋齒孀居,窶迫殘日,欽風撫事,嗟慨滿懷。



 可以帳下米,各餉百斛。」原平固讓頻煩,誓死不受。人或問曰:「府君嘉君淳行,敏君貧老,故
 加此贍,豈宜必辭。」原平曰:「府君若以吾義行邪,則無一介之善,不可濫荷此賜。若以其貧老邪,耋齒甚多,屢空比室,非吾一人而已。」終不肯納。



 百年妻亦辭不受。



 會稽貴重望計及望孝,盛族出身,不減秘、著。太宗泰始七年,興宗欲舉山陰孔仲智長子為望計,原平次息為望孝。仲智會土高門,原平一邦至行,欲以相敵。



 會太宗別敕用人,故二選並寢。泰豫元年,興宗征還京師,表其殊行,宜舉拔顯選,以勸風俗。舉為太學博士。會興宗薨,事不
 行。明年,元徽元年,卒於家。原平少長交物,無忤辭於人,與其居處者數十年,未嘗見喜慍之色。三子一弟,並有門行。



 長子伯林,舉孝廉,次子靈馥,儒林祭酒,皆不就。



 嚴世期,會稽山陰人也。好施慕善,出自天然。同里張邁三人,妻各產子,時歲飢儉,慮不相存,欲棄而不舉。世期聞之,馳往拯救,分食解衣,以贍其乏,三子並得成長。同縣俞陽妻莊年九十,莊女蘭七十,並各老病,單孤無所依,世期衣飴之二十餘年,死並殯葬。宗親嚴弘、鄉人潘
 伯等十五人,荒年並餓死,露骸不收,世期買棺器殯埋,存育孩幼。山陰令何曼之表言之。元嘉四年,有司奏榜門曰:「義行嚴氏之閭」,復其身徭役,蠲租稅十年。



 吳逵,吳興烏程人也。經荒飢饉,係以疾疫,父母兄弟嫂及群從小功之親,男女死者十三人。逵時病困,鄰里以葦席裹之,埋於村側。既而逵疾得瘳,親屬皆盡,唯逵夫妻獲全。家徒壁立,冬無被褲,晝則庸賃,夜則伐木燒磚,此誠無有懈倦。



 逵夜行遇虎,虎輒下道避之。期年中,成
 七墓,葬十三棺。鄰里嘉其志義,葬日悉出赴助,送終之事,亦儉而周禮。逵時逆取鄰人夫直,葬畢,眾悉以施之;逵一無所受,皆傭力報答焉。太守張崇之三加禮命,太守王韶之擢補功曹史,逵以門寒,固辭不就,舉為孝廉。



 潘綜,吳興烏程人也。孫恩之亂,妖黨攻破村邑,綜與父驃共走避賊。驃年老行遲,賊轉逼,驃語綜:「我不能去,汝走可脫,幸勿俱死。」驃困乏坐地,綜迎賊叩頭曰:「父年老,乞賜生命。」賊至,驃亦請賊曰:「兒年少,自能走,今為老子
 不走去。老子不惜死,乞活此兒。」賊因斫驃,綜抱父於腹下,賊斫綜頭面,凡四創,綜當時悶絕。有一賊從傍來,相謂曰:「卿欲舉大事,此兒以死救父,云何可殺。殺孝子不祥。」賊良久乃止,父子並得免。


綜鄉人秘書監丘繼祖、廷尉沈赤黔以綜異行,廉補左民令史,除遂昌長,歲滿還家。太守王韶之臨郡,發教曰:「前被符,孝廉之選,必審其人,雖四科難該,文質寡備,必能孝義邁俗,拔萃著聞者,便足以顯應明易又,允將符旨。烏程潘綜守死孝道,全親濟
 難。烏程吳逵義行純至,列墳成行。咸精誠內淳,休聲外著,可並察孝廉,并列上州臺,陳其行跡。」及將行,設祖道,贈以四言詩曰:東寶惟金,南木有喬。發煇曾崖,竦幹重霄。美哉茲土,世載英髦。育翮幽林,養音九皋。
 \gezhu{
  其一}


唐后明易又,漢宗蒲輪。我皇降鑑,思樂懷人。群臣競薦,舊章惟新。餘亦奚貢,曰義與仁。
 \gezhu{
  其二}


仁義伊在,惟吳惟潘。心積純孝,事著艱難。投死如歸,淑問若蘭。吳實履仁,心力偕單。固此苦節,易彼歲寒。霜雪雖厚,松柏丸丸。
 \gezhu{
  其三}


人亦有言,無
 善不彰。二子徽猷,彌久彌芳。拔叢出類,景行朝陽。誰謂道遐,弘之則光。咨爾庶士,無然怠荒。
 \gezhu{
  其四}


江革奉摯,慶祿是荷。姜詩入貢,漢朝咨嗟。勖哉行人,敬爾休嘉。俾是下國,照煇京華。
 \gezhu{
  其五}


伊余朽駘,竊服懼盜。無能禮樂,豈暇聲教。順彼康夷,懿德是好。聊綴所懷,以贈二孝。
 \gezhu{
  其六}



 元嘉四年,有司奏改其里為純孝里,蠲租布三世。



 張進之,永嘉安固人也。為郡大族。少有志行,歷郡五官主簿,永寧、安固二縣領校尉。家世富足,經荒年散其財,
 救贍鄉里,遂以貧罄,全濟者甚多。進之為太守王味之吏,味之有罪當見收,逃避投進之家,供奉經時,盡其誠力。以本村淺近,移入池溪,味之墮水沈沒,進之投水拯救,相與沈淪,危而得免。時劫掠充斥,每入村抄暴,至進之門,輒相約勒,不得侵犯,其信義所感如此。元嘉初,詔在所蠲其徭役。孫恩之亂,永嘉太守司馬逸之被害,妻子並死,兵寇之際,莫敢收藏。



 郡吏俞僉以家財買棺斂逸之等六喪,送致還都,葬畢乃歸鄉里。元嘉中,老病卒。



 王彭,盱眙直瀆人也。少喪母。元嘉初,父又喪亡,家貧力弱,無以營葬,兄弟二人,晝則傭力,夜則號感。鄉里並哀之,乃各出夫力助作磚。磚須水而天旱,穿井數十丈,泉不出;墓處去淮五里,荷簷遠汲,困而不周。彭號天自訴,如此積日。一旦大霧,霧歇,磚灶前忽生泉水,鄉鄰助之者,并嗟歎神異,縣邑近遠,悉往觀之。葬事既竟,水便自竭。元嘉九年,太守劉伯龍依事表言,改其里為通靈里,蠲租布三世。



 蔣恭,義興臨津人也。元嘉中,晉陵蔣崇平為劫見禽,云與恭妻弟吳晞張為侶。



 晞張先行不在,本村遇水,妻息五口避水移寄恭家,討錄晞張不獲,收恭及兄協付獄治罪。恭、協並款舍住晞張家口,而不知劫情。恭列晞張妻息是婦之親,親今有罪,恭身甘分,求遣兄協。協列協是戶主,延制所由,有罪之日,關協而已,救遣弟恭。兄弟二人,爭求受罪,郡縣不能判,依事上詳。州議之曰:「禮讓者以義為先,自厚者以利為上,末世俗薄,靡不自私。伏
 膺聖教,猶或不逮,況在野夫,未達誥訓,而能互發天倫之憂,甘受莫測之罪,若斯情義,實為殊特。蔑爾恭、協,而能行之,茲乃終古之所希,盛世之嘉事。二子乘舟,無以過此。豈宜拘執憲文,加以罪戮!且晞張封筒遠行,他界為劫,造釁自外,贓不還家,所寓村伍,容有不知,不合加罪。」勒縣遣之,還復民伍。乃除恭義成令,協義怡令。



 徐耕,晉陵延陵人也。自令史除平原令。元嘉二十一年,大旱民飢,耕詣縣陳辭曰:「今年亢旱,禾稼不登。氓黎飢
 餒,採掇存命,聖上哀矜,已垂存拯。但饉罄來久,困殆者眾,米穀轉貴,糴索元所。方涉春夏,日月悠長,不有微救,永無濟理。不惟凡瑣,敢憂身外,《鹿鳴》之求,思同野草,氣類之感,能不傷心。民糴得少米,資供朝夕。志欲自竭,義存分飧,今以千斛,助官賑貸。此境連年不熟,今歲尤甚,晉陵境特為偏祐。此郡雖弊,猶有富室,承陂之家,處處而是,並皆保熟,所失蓋微。陳積之谷,皆有巨萬,旱之所弊,實鐘貧民,溫富之家,各有財寶。



 謂此等並宜助官,得
 過儉月,所損至輕,所濟甚重。今敢自勵,為勸造之端。實願掘水揚塵,崇益山海。」縣為言上。當時議者以耕比漢卜式,詔書褒美,酬以縣令。



 大明八年,東土飢旱,東海嚴成、東莞王道蓋各以穀五百斛助官賑恤。



 孫法宗,吳興人也。父遇亂被害,尸骸不收,母兄並餓死。法宗年小流迸,至年十六,方得還。單身勤苦,霜行草宿,營辦棺槨,造立塚墓,葬送母兄,儉而有禮。以父喪不測,於部境之內,尋求枯骨,刺血以灌之,如此者十餘年不
 獲,乃縗絰。終身不娶,饋遺無所受。世祖初,揚州辟為文學從事,不就。



 範叔孫,吳郡錢唐人也。少而仁厚,固窮濟急。同里范法先父母兄弟七人,同時疫死,唯餘法先,病又危篤,喪尸經月不收。叔孫悉備棺器,親為殯埋。又同里施淵夫疾病,父母死不殯;又同里范苗父子並亡;又同里危敬宗家口六人俱得病,二人喪沒,親鄰畏遠,莫敢營視。叔孫並殯葬,躬恤病者,並皆得全。鄉曲貴其義行,莫有呼其
 名者。世祖孝建初,除竟陵王國中軍將軍,不就。



 義興吳國夫,亦有義讓之美。人有竊其稻者,乃引還,為設酒食,以米送之。



 卜天與,吳興餘杭人也。父名祖,有勇幹,徐赤將為餘杭令,祖依隨之。赤將死,高祖聞其有幹力,召補隊主,從征伐,封關中侯,歷二縣令。天與善射,弓力兼倍,容貌嚴正,笑不解顏。太祖以其舊將子,便教皇子射。居累年,以白衣領東掖防關隊。元嘉二十七年,臧質救懸瓠,劉興祖
 守白石,並率所領隨之,虜退罷。



 遷領輦後第一隊,撫恤士卒,甚得眾心。二十九年,以為廣威將軍,領左細仗,兼帶營祿。



 元凶入弒,事變倉卒,舊將羅訓、徐罕皆望風屈附,天與不暇被甲,執刀持弓,疾呼左右出戰。徐罕曰:「殿下入,汝欲何為?」天與罵曰:「殿下常來,云何即時方作此語。只汝是賊。」手射賊劭於東堂,幾中。逆徒擊之,臂斷倒地,乃見殺。



 其隊將張泓之、朱道欽、陳滿與天與同出拒戰,並死。世祖即位,詔曰:「日者逆豎犯蹕,釁變卒起,廣威
 將軍關中侯卜天與提戈赴難,挺身奮節,斬殪凶黨,而旋受虐刃。勇冠當時,義侔古烈,興言追悼,傷痛於心。宜加甄贈,以旌忠節。可贈龍驤將軍、益州刺史,謚曰壯侯。」車駕臨哭。泓之等各贈郡守,給天與家長稟。



 子伯宗,殿中將軍。太宗泰始初,領幢,擊南賊於赭圻,戰沒。伯宗弟伯興,官至前將軍、南平昌太守,直閣,領細仗主。順帝昇明元年,與袁粲同謀,伏誅。



 天與弟天生,少為隊將,十人同火。屋後有一大坑,廣二丈餘,十人共跳之皆渡,唯天
 生墜坑。天生乃取實中苦竹,剡其端使利,交橫布坑內,更呼等類共跳,並畏懼不敢。天生曰:「我向已不渡,今者必墜此坑中。丈夫跳此不渡,亦何須活。」



 乃復跳之,往反十餘,曾無留礙,眾並歎服。以兄死節,為世祖所留心,稍至西陽王子尚撫軍參軍,加龍驤將軍。隸沈慶之攻廣陵城,天生推車塞塹,率數百人先登西北角,徑至城上。賊為重柵斷攻道,苦戰移日,不拔,乃還。詔曰:「天生始受戎任,甫造寇壘,而投輪越塹,率果先騰,驍壯之氣,嘉歎
 無已。可且賜布千匹,以厲眾校。」大明末,為弋陽太守。太宗泰始初,與殷琰同逆,邊城令宿僧護起義討斬之。



 許昭先,義興人也。叔父肇之,坐事繫獄,七年不判。子姪二十許人,昭先家最貧薄,專獨料訴,無日在家。餉饋肇之,莫非珍新,家產既盡,賣宅以充之。肇之諸子倦怠,昭先無有懈息,如是七載。尚書沈演之嘉其操行,肇之事由此得釋。



 昭先舅夫妻並疫病死亡,家貧無以殯送,昭先賣衣物以營殯葬。舅子三人並幼,贍護皆得成長。昭先
 父母皆老病,家無僮役,竭力致養,甘旨必從,宗黨嘉其孝行。



 雍州刺史劉真道板為征虜參軍,昭先以親老不就。本邑補主簿,昭先以叔未仕,又固辭。元嘉初,西陽董陽五世同財,為鄉邑所美。會稽姚吟,事親至孝,孝建初,揚州辟文學從事,不就。



 餘齊民,晉陵晉陵人也。少有孝行,為邑書吏。父殖,大明二年,在家病亡,家人以父病報之。信未至,齊民謂人曰:「比者肉痛心煩,有若割截,居常遑駭,必有異故。」信尋至,
 便歸,四百餘里,一日而至。至門,方詳父死,號踴慟絕,良久乃蘇。問母:「父所遺言。」母曰:「汝父臨終,恨不見汝。」曰:「相見何難。」於是號叫殯所,須臾便絕。州郡上言,有司奏曰:「收賢旌善,萬代無殊,心至自天,古今豈異。齊民至性由中,情非外感,淳情凝至,深心天徹,跪訊遺旨,一慟殞亡。雖跡異參、柴,而誠均丘、趙。方今聖務彪被,移革華夏,實乃風淳以禮,治本惟孝,靈祥歸應,其道先彰。齊民越自氓隸,行貫生品,旌閭表墓,允出在茲。」改其里為孝義里,
 蠲租布,賜其母穀百斛。



 孫棘,彭城彭城人也。世祖大明五年,發三五丁,弟薩應充行,坐違期不至。



 依制,軍法,人身付獄。未及結竟,棘詣郡辭:「不忍令當一門之苦,乞以身代薩。」



 薩又辭列:「門戶不建,罪應至此,狂愚犯法,實是薩身,自應依法受戮。兄弟少孤,薩三歲失父,一生恃賴,唯在長兄;兄雖可垂愍,有何心處世。」太守張岱疑其不實,以棘、薩各置一處,語棘云:「已為諮詳,聽其相代。」棘顏色甚悅,答云:「得爾,旦則
 為不死。」又語薩,亦欣然曰:「死自分甘,但令兄免,薩有何恨!」棘妻許又寄語屬棘:「君當門戶,豈可委罪小郎。且大家臨亡,以小郎屬君,竟未妻娶,家道不立,君已有二兒,死復何恨。」岱依事表上,世祖詔曰:「棘、薩氓隸,節行可甄,特原罪。」州加辟命,并賜許帛二十匹。



 先是,新蔡徐元妻許,年二十一,喪夫,子甄年三歲,父攬愍其年少,以更適同縣張買。許自誓不行,父逼載送買。許自經氣絕,家人奔赴,良久乃蘇。買知不可奪,夜送還攬。許歸徐氏,養元
 父季。元嘉中,年八十餘,卒。



 太宗泰始二年,長城奚慶思殺同縣錢仲期。仲期子延慶屬役在都,聞父死,馳還,於庚浦埭逢慶思,手刃殺之,自繫烏程縣獄。吳興太守郗顒表不加罪,許之。



 何子平,廬江灊人也。曾祖楷,晉侍中。祖友,會稽王道子驃騎諮議參軍。父子先,建安太守。子平世居會稽,少有志行,見稱於鄉曲。事母至孝。揚州辟從事史,月俸得白米,輒貨市粟麥。人或問曰:「所利無幾,何足為煩?」子平曰:「
 尊老在東,不辦常得生米,何心獨饗白粲。」每有贈鮮肴者,若不可寄致其家,則不肯受。



 母本側庶,籍注失實,年未及養,而籍年已滿,便去職歸家。時鎮軍將軍顧覬之為州上綱,謂曰:「尊上年實未八十,親故所知。州中差有微祿,當啟相留。」



 子平曰:「公家正取信黃籍,籍年既至,便應扶侍私庭,何容以實年未滿,茍冒榮利。且歸養之願,又切微情。」覬之又勸令以母老求縣,子平曰:「實未及養,何假以希祿。」覬之益重之。既歸家,竭身運力,以給供養。



 元嘉三十年,元凶弒逆,安東將軍隨王誕入討,以為行參軍。子平以凶逆滅理,普天同奮,故廢己受職,事寧,自解。又除奉朝請,不就。末除吳郡海虞令,縣祿唯以養母一身,而妻子不犯一毫。人或疑其儉薄,子平曰:「希祿本在養親,不在為己。」問者慚而退。母喪去官,哀毀踰禮,每至哭踴,頓絕方蘇。值大明末,東士飢荒,繼以師旋,八年不得營葬,晝夜號絕擗踴,不闋俄頃,叫慕之音,常如袒括之日。冬不衣絮,暑避清涼,日以數合米為粥,不進鹽
 菜。所居屋敗,不蔽雨日,兄子伯興採伐茅竹,欲為葺治,子平不肯,曰:「我情事未申,天地一罪人耳,屋何宜覆。」蔡興宗為會稽太守,甚加旌賞。泰始六年,為營塚槨。子平居喪毀甚,困瘠踰久,及至免喪,支體殆不相屬。幼持操檢,敦厲名行,雖處暗室,如接大賓。



 學義堅明,處之以默,安貧守善,不求榮進,好退之士,彌以貴之。順帝升明元年,卒,時年六十。



 史臣曰:漢世士務治身,故忠孝成俗,至乎乘軒服冕,非
 此莫由。晉、宋以來,風衰義缺,刻身厲行,事薄膏腴。若夫孝立閨庭,忠被史策,多發溝畎之中,非出衣簪之下。以此而言聲教,不亦卿大夫之恥乎!



\end{pinyinscope}