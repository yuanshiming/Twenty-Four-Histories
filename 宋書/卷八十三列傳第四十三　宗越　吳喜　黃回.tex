\article{卷八十三列傳第四十三 宗越 吳喜 黃回}

\begin{pinyinscope}

 宗越,南陽葉人也。本河南人,晉亂,徙南陽宛縣,又土斷屬葉。本為南陽次門,安北將軍趙倫之鎮襄陽,襄陽多雜姓,倫之使長史范覬之條次氏族,辨其高卑,覬之點
 越為役門。出身補郡吏。父為蠻所殺,殺其父者嘗出郡,越於市中刺殺之,太守夏侯穆嘉其意,擢為隊主。蠻有為寇盜者,常使越討伐,往輒有功。家貧無以市馬,常刀楯步出,單身挺戰,眾莫能當。每一捷,郡將輒賞錢五千,因此得市馬。



 後被召,出州為隊主。世祖鎮襄陽,以為揚武將軍,領臺隊。



 元嘉二十四年,啟太祖求復次門,移戶屬冠軍縣,許之。二十七年,隨柳元景北伐,領馬幢,隸柳元怙,有戰功,事在元景傳。還補後軍參軍督護,隨王誕
 戲之曰:「汝何人,遂得我府四字。」越答曰:「佛貍未死,不憂不得諮議參軍。」誕大笑。



 隨元景伐西陽蠻,因值建義,轉南中郎長兼行參軍,新亭有戰功。世祖即位,以為江夏王義恭大司馬行參軍,濟陽太守,尋加龍驤將軍。臧質、魯爽反,越率軍據歷陽。爽遣將軍鄭德玄前據大峴,德玄分遣偏師楊胡興、劉蜀馬步三千,進攻歷陽。越以步騎五百於城西十餘里拒戰,大破斬胡興、蜀等。爽平,又率所領進梁山拒質,質敗走,越戰功居多。因追奔至江陵。
 時荊州刺史朱修之未至,越多所誅戮。



 又逼略南郡王義宣子女,坐免官系尚方。尋被宥,復本官,追論前功,封築陽縣子,食邑四百戶。遷西陽王子尚撫軍中兵參軍,將軍如故。大明三年,轉長水校尉。



 竟陵王誕據廣陵反,越領馬軍隸沈慶之攻誕。及城陷,世祖使悉殺城內男丁,越受旨行誅,躬臨其事,莫不先加捶撻,或有鞭其面者,欣欣然若有所得,所殺凡數千人。四年,改封始安縣子,戶邑如先。八年,遷新安王子鸞撫軍中兵參軍,加輔
 國將軍。其年,督司州、豫州之汝南、新蔡、汝陽、潁川四郡諸軍事、寧朔將軍、司州刺史,尋領汝南、新蔡二郡太守。



 前廢帝景和元年,召為遊擊將軍,直閣。頃之,領南濟陰太守,進爵為侯,增邑二百戶。又加冠軍將軍,改領南東海太守,遊擊如故。帝凶暴無道,而越及譚金、童太壹並為之用命,誅戮群公及何邁等,莫不盡心竭力。故帝憑其爪牙,無所忌憚。



 賜與越等美女金帛,充牣其家。越等武人,粗彊識不及遠,咸一往意氣,皆無復二心。帝將欲
 南巡,明旦便發,其夕悉聽越等出外宿,太宗因此定亂。明晨,越等並入,上撫接甚厚,越改領南濟陰太守,本官如故。



 越等既為廢帝盡力,慮太宗不能容之,上接待雖厚,內並懷懼。上亦不欲使其居中,從容謂之曰:「卿等遭罹暴朝,勤勞日久,苦樂宜更,應得自養之地。兵馬大郡,隨卿等所擇。」越等素已自疑,及聞此旨,皆相顧失色,因謀作難。以告沈攸之,攸之具白太宗,即日收越等下獄死。越時年五十八。



 越善立營陣,每數萬人止頓,越自騎
 馬前行,使軍人隨其後,馬止營合,未嘗參差。及沈攸之代殷孝祖為南討前鋒,時孝祖新死,眾並懼,攸之歎曰:「宗公可惜,故有勝人處。」而御眾嚴酷,好行刑誅,睚眥之間,動用軍法。時王玄謨御下亦少恩,將士為之語曰:「寧作五年徒,不逐王玄謨。玄謨尚可,宗越殺我。」



 譚金,荒中傖人也。在荒中時,與薛安都有舊,後出新野,居牛門村。及安都歸國,金常隨征討。自北入崤陜,及巴口建義,恒副安都,排堅陷陣,氣力兼人,平元凶及梁山破臧質,每
 有戰功。稍至建平王宏中軍參軍事,加建武將軍,尋轉龍驤將軍、南下邳太守,參軍如故。孝建三年,遷屯騎校尉、直閣,領南清河太守。



 景和元年,前廢帝誅群公,金等並為之用。帝下詔曰:「屯騎校尉南清河太守譚金、彊弩將軍童太壹、車騎中兵參軍沈攸之,誠略沈果,忠幹勇鷙,消盪氛翳,首制鯨凶,宜裂河山,以酬勳義。金可封平都縣男,太壹宜陽縣男,攸之東興縣男,食邑各三百戶。」金遷驍騎將軍,增邑百戶。太壹,東莞人也。自彊弩遷左
 軍將軍,增邑百戶。金、太壹並與宗越俱死。



 越州里劉胡、武念、佼長生、蔡那、曹欣之,並以將帥顯。劉胡事在《鄧琬傳》。



 武念,新野人也。本三五門,出身郡將。蕭思話為雍州,遣土人龐道符統六門田,念為道符隨身隊主。後大府以念有健名,且家富有馬,召出為將。世祖臨雍州,念領隊奉迎。時沔中蠻反,世祖之鎮,緣道討伐,部伍至大堤巖洲,蠻數千人忽至,乘高矢射雨下。念馳赴奮擊,應時摧退,即擢為參軍督護。其後每軍旅,常有戰功。



 世祖孝
 建中,為建威將軍、桂陽太守。竟陵王誕反,念以江夏王義恭太宰參軍、龍驤將軍,隸沈慶之攻廣陵城。誕出城走,既而復還,念追之不及,坐免官。復以為冗從僕射,出為龍驤將軍、南陽太守。前廢帝景和中,為右軍將軍,直閣,封開國縣男,食邑三百戶。太宗初即位,四方反叛,遣念乘驛還雍州,綏慰西土,因以為南陽太守。念既至,人情並向之,劉胡遣腹心數騎詐詣念降,於坐縛念,袁顗斬之,送首詣晉安王子勛。念黨袁處珍逃亡至壽陽,為
 逆黨劉順所得,考楚備至,秉義不移,後得叛奔劉勔;太宗嘉之,以為奉朝請。追贈念冠軍將軍、南陽、新野二郡太守,封綏安縣侯,食邑四百戶。泰始四年,綏安縣省,改封邵陵縣。



 佼長生,廣平人也。出身為縣將,大府以其有膂力,召為府將。朱修之拒魯秀於峴南,長生有戰功,稍見任使。太宗初,為建安王休仁司徒中兵參軍,加寧朔將軍。南討有功,封遷陵縣侯,食邑八百戶。後為張悅寧遠司馬,寧蠻校尉。泰始五年,卒,追贈征虜將軍、雍州刺
 史。



 蔡那,南陽冠軍人也。家素富,而那兄局善接待賓客,客至無少多,皆資給之,以此為郡縣所優異,蠲其調役。那始為建福戍主,漸至大府將佐。太宗初,為建安王休仁司徒中兵參軍,南討。那子弟皆在襄陽,為劉胡所執,胡每戰輒懸之城外,那進戰愈猛。以功封平陽縣侯,食邑五百戶。稍至劉韞撫軍司馬、寧蠻校尉,加寧朔將軍。泰豫元年,以本號為益州刺史、宋寧太守。未拜,卒,追贈輔師將軍,餘如故,謚曰平侯。



 曹欣之,新野人也。積勤勞,
 後廢帝元徽初,為軍主。以平桂陽王休範功,封新市縣子,食邑五百戶。為左軍驍騎將軍,加輔國將軍。元徽四年,以本號為徐州刺史、鐘離太守,進號冠軍將軍。順帝昇明二年,徵為散騎常侍、驍騎將軍。三年,卒。



 吳喜,吳興臨安人也。本名喜公,太宗減為喜。初出身為領軍府白衣吏。少知書,領軍將軍沈演之使寫起居注,所寫既畢,闇誦略皆上口。演之嘗作讓表,未奏,失本,喜經一見,即便寫赴,無所漏脫,演之甚知之。因此涉獵《史》、《
 漢》,頗見古今。演之門生朱重民入為主書,薦喜為主書書史,進為主圖令史。太祖嘗求圖書,喜開卷倒進之,太祖怒,遣出。



 會太子步兵校尉沈慶之征蠻,啟太祖請喜自隨,使命去來,為世祖所知賞。世祖於巴口建義,喜遇病,不堪隨慶之下。事平,世祖以喜為主書,稍見親遇,擢為諸王學官令,左右尚方令,河東太守,殿中御史。大明中,黟、歙二縣有亡命數千人,攻破縣邑,殺害官長。豫章王子尚為揚州,在會稽,再遣主帥,領三千人水陸討
 伐;遂再往,失利。世祖遣喜將數十人至二縣,誘說群賊,賊即日歸降。



 太宗初即位,四方反叛,東兵尤急。喜請得精兵三百,致死於東,上大說,即假建武將軍,簡羽林勇士配之。議者以喜刀筆主者,不嘗為將,不可遣。中書舍人巢尚之曰:「喜昔隨沈慶之,屢經軍旅,性既勇決,又習戰陳,若能任之,必有成績。諸人紛紛,皆是不別才耳。」喜乃率員外散騎侍郎竺超之、殿中將軍杜敬真馬步東討。既至永世,得庾業、劉延熙書,送尋陽王子房檄文。與
 喜書曰:「知統戎旅,已次近路,卿所在著名,今日何為立忠於彼邪?想便倒戈,共受河、山之賞。」



 喜報書曰:「前驅之人,忽獲來翰,披尋狂惑,良深悵駭。聖主以神武撥亂,德盛勳高,群逆交扇,滅在晷刻。君等勳義之烈,世荷國恩,事愧鳴鴞,不懷食椹。今練勒所部,星言進邁,相見在近,不復多陳。」喜,孝武世見驅使,常充使命,性寬厚,所至人並懷之。及東討,百姓聞吳河東來,便望風降散,故喜所至克捷,事在《孔覬傳》。



 遷步兵校尉,將軍如故。封竟陵縣
 侯,食邑千戶。東土平定,又率所領南討,遷輔國將軍、尋陽太守。南賊退走,喜追討平定荊州,遷前軍將軍,增邑三百戶。



 泰始四年,改封東興縣侯,戶邑如先。仍除使持節、督交州、廣州之鬱林、寧浦二郡諸軍事、輔國將軍、交州刺史。不行,又除右軍將軍、淮陵太守,假輔師將軍,兼太子左衛率。



 五年,轉驍騎將軍,假號、太守、兼率如故。其年,虜冠豫州,喜統諸軍出討,大破虜於荊亭,偽長社公遁走,戍主帛乞奴歸降。軍還,復以本位兼左衛將軍。六
 年,又率軍向豫州拒索虜,加節、督豫州諸軍事,假冠軍將軍,驍騎、太守如故。



 明年,還京都。



 初,喜東征,白太宗得尋陽王子房及諸賊帥,即於東梟斬。東土既平,喜見南賊方熾,慮後翻覆受禍,乃生送子房還都;凡諸大主帥顧琛、王曇生之徒,皆被全活。上以喜新立大功,不問也,而內密銜之。及平荊州,恣意剽虜,贓私萬計;又嘗對賓客言漢高、魏武本是何人,上聞之,益不說。其後誅壽寂之,喜內懼,因啟乞中散大夫,上尤疑駭。至是會上有疾,
 為身後之慮,以喜素得人情,疑其將來不能事幼主,乃賜死,時年四十五。喜將死之日,上召入內殿與共言謔,酬接甚款。



 既出,賜以名饌,并金銀御器,敕將命者勿使食器宿喜家。上素多忌諱,不欲令食器停凶禍之室故也。喜未死一日,上與劉勔、張興世、齊王詔曰:吳喜出自卑寒,少被驅使,利口任詐,輕狡萬端。自元嘉以來,便充刀筆小役,賣弄威恩,茍取物情,處處交結,皆為黨與,眾中常以正直為詞,而內實阿媚。每仗計數,運其佞巧,甘
 言說色,曲以事人,不忠不平,彰於觸事。從來作諸署,主意所不協者,覓罪委頓之,以示清直;而餘人恣意為非,一不檢問,故甚得物情。



 昔大明中,黟、歙二縣有亡命數千人,攻破縣邑,殺害官長。劉子尚在會稽,再遣為主帥,領三千精甲水陸討伐,再往失利。孝武以喜將數十人至二縣說誘群賊,賊即歸降。詭數幻惑,乃能如此,故每豫驅馳,窮諸狡慝。及泰始初東討,正有三百人,直造三吳,凡再經薄戰,而自破岡以東至海十郡,無不清蕩。百
 姓聞吳河東來,便望風自退,若非積取三吳人情,何以得弭伏如此。其統軍寬慢無章,放恣諸將,無所裁檢,故部曲為之致力。觀其意趣,止在賊平之後,應力為國計。



 喜初東徵發都,指天畫地,云得劉子房即當屏除,袁標等皆加斬戮,使略無生口。既平之後,緩兵施恩,納罪人之貨,誘諸賊帥,令各逃藏,受賂得物,不可稱紀。聽諸賊帥假稱為降,而擁衛子房遂得生歸朝庭。收羅群逆,皆作爪牙,撫接優密,過於義士。推此意,正是聞南賊大盛,
 殷孝祖戰亡,人情大惡,慮逆徒得志,規以自免。喜善為姦變,每以計數自將,於朝廷則三吳首獻慶捷,於南賊則不殺其黨,頗著陰誠。當云東人恇怯,望風自散,皆是彼無處分,非其苦相逼迫,保全子房及顧琛等,足表丹誠,進退二塗,可以無患。



 南賊未平,唯以軍糧為急,西南及北道斷不通,東土新平,商運稀簡,朝廷乃至鬻官賣爵,以救災困,斗斛收斂,猶有不充。喜在赭圻,軍主者頓偷一百三十斛米,初不問罪;諸軍主皆云宜治,喜不獲
 已,止與三十鞭,又不責備,凡所曲意,類皆如此。



 喜至荊州,公私殷富,錢物無復孑遺。喜乘兵威之盛,誅求推檢,凡所課責,既無定科,又嚴令驅蹙,皆使立辦。所使之人,莫非姦猾。因公行私,迫肋在所。



 入官之物,侵竊過半。納資請託,不知厭已。西難既殄,便應還朝,而解故槃停,託云扞蜀。實由貨易交關,事未回展。又遣人入蠻,矯詔慰勞,賧伐所得,一以入私。又遣部下將吏,兼因土地富人,往襄陽或蜀、漢,屬託郡縣,侵官害民,興生求利,千端萬
 緒。從西還,大め小艒,爰及草舫,錢米布絹,無船不滿。自喜以下,迨至小將,人人重載,莫不兼資。



 喜本小人,多被使役,經由水陸,州郡殆遍;所至之處,輒結物情,妄竊善稱。



 聲滿天下,密懷姦惡,人莫之知。喜軍中諸將,非劫便賊,唯云:「賊何須殺,但取之,必得其用。」雖復羸弱,亦言:「健兒可惜,天下未平,但令以功贖罪。」



 處遇料理,反勝勞人,此輩所感唯喜,莫云恩由朝廷。凶惡不革,恒出醜聲,勞人義士,相與歎息,並云:「我等不愛性命,擊擒此賊,朝廷
 不肯殺去,反與我齊。



 今天下若更有賊,我不復能擊也。」此等既隨喜行,多無功效,或隱在眾後,或在幔屋中眠。賊即破散,與勞人同受爵賞。既被詰問,辭白百端,云:「此輩既見原宥,擊賊有功,那得不依例加賞。」褚淵往南選諸將卒,喜為軍中經為賊者,就淵求官,倍於義士。淵以喜最前獻捷,名位已通,又為統副,難相違拒,是以得官受賞,反多義人。義人雖忿喜不平,又懷其寬弛。



 往歲竺超之聞四方反叛,人情畏賊,無敢求為朝廷行者,乃慨
 然攘步,隨喜出征,為其軍副。身經臨敵,自東還,失喜意。說超之多酒,不堪驅使,遂相委棄。



 高敬祖年雖少宿,氣力實健,其有處分,為軍中所稱,喜薄其衰老,云無所施。正以二人忠清,與己異行。超之為人,乃多飲酒,計喜軍中主帥,豈無飲酒者?特是不利超之,故以酒致言耳。敬祖既無餘事,直云年老,託為乞郡,潛相遣斥。其餘主帥,並貪濁謅媚之流,皆提攜東西,不相離舍。喜聞天壤間有罪人死或應繫者,必啟以入軍,皆得官爵,厚被處遇。
 應入死之人,緣己得活,非唯得活,又復如意。



 人非木石,何能不感!設令吾攻喜門,此輩誰不致力,但是喜不敢生心耳。喜軍中人皆是喜身爪牙,豈關於國。



 喜自得軍號以來,多置吏佐,是人加板,無復限極。為兄弟子侄及其同堂群從,乞東名縣,連城四五,皆灼然巧盜,侵官奪私。亡命罪人,州郡不得討;崎嶇蔽匿,必也黨護。臺州符旨,殆不復行。船車牛犢,應為公家所假借者,託之於喜,吏司便不敢問。它縣奴婢,入界便略。百姓牛犢,輒索殺
 啖。州郡應及役者,並入喜家。



 喜兄茹公等悉下取錢,盈村滿里。諸吳姻親,就人間徵求,無復紀極,百姓嗷然,人人悉苦。喜具知此,初不禁呵。



 索惠子罪不甚江悆,既已被恩,得免憲辟,小小忤意,輒加刑斬。張悅賊中大帥,逼迫歸降,沈攸之錄付喜,云:「殺活當由朝廷。」將帥征伐,既有常體,自應執歸之有司。喜即便打鎖,解襦與著,對膝圍棋,仍造重義,私惠招物,觸事如斯。張靈度凶愚小人,背叛之首,喜在西輒恕其罪,私將下都,與之周旋,情若
 同體。狼子野心,獨懷毒性,遂與柳欣慰等謀立劉禕。吾使喜錄之,而喜密報令去,去未得遠,為建康所錄。喜背國親惡,乃至於是。



 初從西反,圖兼右丞,貪因事物,以行私詐。吾患其諂曲,抑而不許,從此怨懟,意用不平。喜西救汝陰,縱肆兵將,掠暴居民,姦人婦女,逼奪雞犬,虜略縱橫,緣路官長,莫敢呵問。脫誤有縛錄一人,喜輒大怒。百姓呼嗟,人人失望。近段佛榮求還,乃欲用喜代之。西人聞其當來,皆欲叛走,云:「吳軍中人皆是生劫,若作刺
 史,吾等豈有活路。既無他計,正當叛投虜耳。」夫伐罪弔民,用清國道。



 豈有殘虐無辜,剝奪為務,害政妨國,罔上附下,罪釁若此,而可久容!臧文仲有云:「見有善於其君,如孝子之養父母;見有惡於君,若鷹鸇之逐鳥雀」。耿弇不以賊遺君父,前史以為美談。而喜軍中五千人,皆親經反逆,攜養左右,豈有奉上之心!



 喜意志張大,每稱漢高、魏武,本是何人。近忽通啟,求解軍任,乞中散大夫。



 喜是何人,乃敢作此舉止!且當今邊疆未寧,正是喜輸蹄
 領之日,若以自處之宜,當節儉廉慎,靜掃閉門,不興外物交關;專心奉上,何得以其蜼螭,高自比擬。當是自顧愆釁,事宣遐邇,又見壽寂之流徙,施修林被擊,物惡傷類,內懷憂恐,故興此計,圖欲自安。



 朝廷之士及大臣籓鎮,喜殆無所畏者,畏者唯吾一人耳。人生修短,不可豫量,若吾壽百年,世間無喜,何所虧損。若使吾四月中疾患不得治力,天下豈可有喜一人。尋喜心迹,不可奉守文之主,豈可遭國家間隙,有可乘之會邪!世人多云,「時
 可畏,國政嚴」。歷觀有天下,御億兆,仗威齊眾,何代不然。故上古象刑,民淳不犯;後聖征偽,易以剠墨。唐堯至仁,不赦四凶之罪;漢高大度,而急三傑之誅。且太公為治,先華士之刑;宣尼作宰,肆少正之戮。自昔力安社稷,功濟蒼生,班劍引前,笳鼓陪後,不能保此者,歷代無數。養之以福,十分有一耳。至若喜之深罪,其得免乎?



 夫富之與貴,雖以功績致之,必由道德守之。故善始者未足稱奇,令終者乃可重耳。凡置官養士,本在利國,當其為利,
 愛之如赤子;及其為害,畏之若仇讎,豈暇遠尋初功,而應忍受終敝耳。將之為用,譬如餌藥,當人羸冷,資散石以全身;及熱勢發動,去堅積以止患。豈憶始時之益,不計後日之損;存前者之賞,抑當今之罰。非忘其功,勢不獲已耳。喜罪釁山積,志意難容,雖有功效,不足自補,交為國患,焉得不除。且欲防微杜漸,憂在未萌,不欲方幅露其罪惡,明當嚴詔切之,令自為其所。卿諸人將相大臣,股肱所寄,賞罰事重,應與卿等論之,卿意並謂云何?



 及喜死,發詔賻賜。子徽民,襲爵。齊受禪,國除。



 黃回,竟陵郡軍人也。出身充郡府雜役,稍至傳教。臧質為郡,轉齋帥,及去職,將回自隨。質為雍州,回復為齋帥。質討元凶,回隨從有功,免軍戶。質在江州,擢領白直隊主。隨質於梁山敗走向豫章,為臺軍主謝承祖所錄,付江州作部,遇赦得原。回因下都,於宣陽門與人相打,詐稱江夏王義恭馬客,鞭二百,付右尚方。會中書舍人戴明寶被繫,差回為戶伯,性便辟勤緊,奉事明寶,竭盡心
 力。明寶尋得原赦,委任如初,啟免回,以領隨身隊,統知宅及江西墅事。性有功藝,觸類多能,明寶甚寵任之。



 回拳捷果勁,勇力兼人,在江西與諸楚子相結,屢為劫盜。會太宗初即位,四方反叛,明寶啟太宗使回募江西楚人,得快射手八百,假回寧朔將軍、軍主,隸劉勔西討。於死虎破杜叔寶軍,除山陰王休祐驃騎行參軍、龍驤將軍。攻合肥,破之,累遷至將校,以功封葛陽縣男,食邑二百戶。



 後廢帝元徽初,桂陽王休範為逆,回以屯騎校尉
 領軍隸齊王,於新亭創詐降之計,事在《休範傳》。回見休範可乘,謂張敬兒曰:「卿可取之,我誓不殺諸王。」



 敬兒即日斬休範。事平,轉回驍騎將軍,加輔師將軍,進爵為侯,改封聞喜縣,增邑千戶。四年,遷冠軍將軍、南琅邪、濟陽二郡太守。建平王景素反,回又率軍前討,假節。城平之日,回軍先入,又以景素讓張倪奴,回增邑五百戶,進號征虜將軍,加散騎常侍,太守如故。明年,遷右衛將軍,常侍如故。



 沈攸之反,以回為使持節、督郢州、司州之義陽
 諸軍事、平西將軍、郢州刺史,給鼓吹一部,率眾出新亭為前鋒。未發,而袁粲據石頭為亂,回與新亭諸將帥任候伯、彭文之、王宜興、孫曇瓘等謀應粲。粲事發,候伯等並乘船赴石頭,唯曇瓘先至得入,候伯等至,而粲已平。回本期詰旦率所領從御道直向臺門,攻齊王於朝堂,事既不果,齊王撫之如舊。回與宜興素不協,慮或反告,因其不從處分,斬之。宜興,吳興人也。形狀短小,而果勁有膽力。少年時為劫,不須伴,郡討逐圍繞數十重,終莫
 能擒。太宗泰始中,為將,在壽陽間擊索虜,每以少制多,挺身深入,無所畏憚,虜眾值宜興,皆引避不敢當。稍至寧朔將軍,羽林監。以平建平王景素功,封長壽縣男,食邑三百戶。至是,為屯騎校尉,加輔國將軍。



 回進軍未至郢州,而沈攸之敗走。回至鎮,進號鎮西將軍,改督為都督。回不樂停郢州,固求南兗,遂率部曲輒還。改封安陸郡公,增邑二千戶,并前三千七百戶。改都督南兗、徐、兗、青、冀五州諸軍事、鎮北將軍、南兗州刺史,加散騎常侍,持
 節如故。



 齊王以回終為禍亂,乃上表曰:「黃回出自廝伍,本無信行,仰值泰始,謬被驅馳,階藉風雲,累叨顯伍。及沈攸之作逆,事切戎機,臣闇於知人,冀其搏噬,遣統前鋒,竟不接刃。軍至郢城,乘威迫肋,陵掠所加,必先尊貴。武陵王馬器服咸被虜奪,城內文武,剝剔靡遺。及至還都,縱恣彌甚,先朝御服,猶有二輿,弓劍遺思,尚在車府。回遂啟求,以擬私用,僭侮無厭,罔顧天極。又廣納逋亡,多受劫盜,親信此等,並為爪牙。觀其凶狡,憂在不測,惡
 積罪著,非可含忍,應加鏟除,以明國憲。尋其釁狀,實宜極法,但嘗經將帥,微有塵露,罪疑從輕,事炳前策,請在降減,特原餘嗣。臣過荷隆寄,言必罄誠,謹陳管穴,式遵弘典,伏願聖明,特垂允鑒。臣思不出位,誠昧甄才,追言既往,伏增慚恧。」詔曰:「黃回擢自凡豎,夙負疵釁,貰以憲綱,收基搏噬。雖勤效累著,而屢懷干紀。新亭背叛,投拜寇場,異規既扇,廟律幾殆,幸得張敬兒提戈直奮,元惡受戮。及景素結逆,履霜歲久,乃密通音譯,潛送器杖,氛
 沴克霽,狡謀方顯。每存容掩,冀能悛革,故裂茅升爵,均榮勳寵。凶詖有本,險慝滋深,構誘敬兒,志相攻陷,悖圖未遂,很戾彌甚。近軍次郢鎮,劫逼府主,兼挾私計,多所徵索,主局咨疑,便加捶楚,專肆暴慢,罔顧彞則。膺牧西蕃,徽賁惟厚,曾不知感,猶懷忿怨。李安民述任河、濟,星管未周,貪據襟要,苦祈回奪。黷謁弗已,叨侈無度,遂請求御輿,僭擬私飾。又招萃賊黨,初不啟聞,傷風蠹化,莫此之甚。宜明繩裁,肅正刑書,便收付廷尉,依法窮治。」



 回
 死時,年五十二。子僧念,尚書左民郎,竟陵相,未發,從誅。



 回既貴,祗事戴明寶甚謹,言必自名。每至明寶許,屏人獨進,未嘗敢坐。躬至帳下及入內,料檢有無,隨乏供送,以此為常。



 先是,王蘊為湘州,潁川庾佩玉為蘊寧朔府長史、長沙內史。蘊去職,南中郎將、湘州刺史南陽王翽未之任,權以佩玉行府州事。先遣中兵參軍、臨湘令韓幼宗領軍戍防湘州,與佩玉共事,不美。及沈攸之為逆,佩玉、幼宗各不相信,幼宗密圖,佩玉知其謀,襲殺幼宗。
 回至郢州,遣輔國將軍任候伯行湘州事,候伯以佩玉兩端,輒殺之。湘州刺史呂安國之鎮,齊王使安國誅候伯。



 彭文之,泰山人也。以軍功稍至龍驤將軍。討建平王景素功,封葛陽縣男,食邑三百戶。順帝初,為輔國將軍、左軍將軍、南濮陽太守、直閣,領右細杖盪主。



 沈攸之平後,齊王收之下獄,賜死。



 孫曇瓘,吳郡富陽人也。驍果有氣力,以軍功稍進,至是為寧朔將軍、越州刺史。於石頭叛走,逃竄經時,後於秣陵縣禽獲,伏誅。



 回同時為將者,
 臨淮任農夫,沛郡周寧民,南郡高道慶,並以武用顧。農夫稍至彊弩將軍。太宗初,以東討功,封廣晉縣子,食邑五百戶。東土平定,仍又南討,增邑二百戶。歷射聲校尉,左軍將軍。時桂陽王休範在江州,有異志,朝廷慮其下,以農夫為輔師將軍、淮南太守,戍姑孰以防之。休範尋率眾向京邑,奄至近道,農夫棄戍還都。休範平,以戰功改封孱陵縣侯,增邑千戶,并前千七百戶。出為輔師將軍、豫州刺史,尋進號冠軍將軍。明年,入為驍騎將軍,加
 通直散騎常侍。前世加官,唯散騎常侍,無通直員外之文。太宗以來,多因軍功至大位,資輕加常侍者,往往通直員外焉。五年,加征虜將軍,改通直為散騎常侍,驍騎如故。其年卒,追贈左將軍,常侍如故,謚曰貞肅。候伯,即農夫弟也。



 周寧民於鄉里起義討薛安都,亦以軍功至軍校。泰始初,封贛縣男,食邑三百戶。官至寧朔將軍、徐州刺史,鐘離太守。



 高道慶亦至軍校驍游,以平桂陽王休範功,封樂安縣男,食邑三百戶。建平王景素反,道慶
 領軍北討,而與景素通謀。及事平,自啟求增邑五百戶,詔加二百,並前五百戶。道慶兇險暴橫,求欲無已,有失其意,輒加捶拉,往往有死者,朝廷畏之如虎狼。齊王與袁粲等議,收付廷尉,賜死。



 史臣曰:夫豎人匹夫,濟其身業,非世亂莫由也。以亂世之情,用於治日,其得不亡,亦為幸矣!



\end{pinyinscope}