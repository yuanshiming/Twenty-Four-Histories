\article{卷五十八列傳第十八 王惠 謝弘微 王球}

\begin{pinyinscope}

 王惠,字令明,琅邪臨沂人,太保弘從祖弟也。祖劭,車騎將軍。父默,左光祿大夫。惠幼而夷簡,為叔父司徒謐所知。恬靜不交遊,未嘗有雜事。陳郡謝瞻才辯有風氣,嘗與
 兄弟群從造惠,談論鋒起,文史間發,惠時相酬應,言清理遠,瞻等慚而退。高祖聞其名,以問其從兄誕,誕曰:「惠後來秀令,鄙宗之美也。」即以為行太尉參軍事,府主簿,從事中郎。世子建府,以為征虜長史,仍轉中軍長史。



 時會稽內使劉懷敬之郡,送者傾京師,惠亦造別,還過從弟球。球問:「向何所見?」



 惠曰:「惟覺即時逢人耳。」常臨曲水,風雨暴至,座者皆馳散,惠徐起,姿貌不異常日。世子為荊州,惠長史如故。領南郡太守,不拜。宋國初建,當置郎中
 令,高祖難其人,謂傅亮曰:「今用郎中令,不可令減袁曜卿也。」既而曰:「吾得其人矣。」乃以惠居之。遷世子詹事,轉尚書,吳興太守。



 少帝即位,以蔡廓為吏部尚書,不肯拜,乃以惠代焉。惠被召即拜,未嘗接客,人有與書求官者,得輒聚置閣上,及去職,印封如初時。談者以廓之不拜,惠之即拜,雖事異而意同也。兄鑒,頗好聚斂,廣營田業,惠意甚不同,謂鑒曰:「何用田為?」鑒怒曰:「無田何由得食!」惠又曰:「亦復何用食為。」其標寄如此。



 元嘉三年,卒,時年
 四十二。追贈太常。無子。



 謝弘微,陳郡陽夏人也。祖韶,車騎司馬。父思,武昌太守。從叔峻,司空琰第二子也,無後,以弘微為嗣。弘微本名密,犯所繼內諱,故以字行。



 童幼時,精神端審,時然後言。所繼叔父混名知人,見而異之,謂思曰:「此兒深中夙敏,方成佳器。有子如此,足矣。」年十歲出繼。所繼父於弘微本緦麻,親戚中表,素不相識,率意承接,皆合禮衷。義熙初,襲峻爵建昌縣侯。弘微家素貧儉,而所繼豐泰,唯受
 書數千卷,國吏數人而已,遺財祿秩,一不關豫。混聞而驚嘆,謂國郎中令漆凱之曰:「建昌國祿,本應與北舍共之,國侯既不措意,今可依常分送。」弘微重違混言,乃少有所受。



 混風格高峻,少所交納,唯與族子靈運、瞻、曜、弘微並以文義賞會。嘗共宴處,居在烏衣巷,故謂之烏衣之遊。混五言詩所云「昔為烏衣遊,戚戚皆親姪」者也。其外雖復高流時譽,莫敢造門。瞻等才辭辯富,弘微每以約言服之,混特所敬貴,號曰微子。謂瞻等曰:「汝諸人雖
 才義豐辯,未必皆愜眾心;至於領會機賞,言約理要,故當與我共推微子。」常云:「阿遠剛躁負氣;阿客博而無檢;曜恃才而持操不篤;晦自知而納善不周,設復功濟三才,終亦以此為恨;至如微子,吾無間然。」又云:「微子異不傷物,同不害正,若年迨六十,必至公輔。」嘗因酣宴之餘,為韻語以獎勸靈運、瞻等曰:「康樂誕通度,實有名家韻,若加繩染功,剖瑩乃瓊瑾。宣明體遠識,穎達且沈俊,若能去方執,穆穆三才順。阿多標獨解,弱冠纂華胤,質勝
 誡無文,其尚又能峻。通遠懷清悟,采采標蘭訊,直轡鮮不躓,抑用解偏吝。微子基微尚,無倦由慕藺,勿輕一簣少,進往將千仞。數子勉之哉,風流由爾振,如不犯所知,此外無所慎。」靈運等並有誡厲之言,唯弘微獨盡褒美。



 曜,弘微兄,多,其小字也。遠即瞻字。靈運小名客兒。



 晉世名家身有國封者,起家多拜員外散騎侍郎,弘微亦拜員外散騎,琅邪王大司馬參軍。義熙八年,混以劉毅黨見誅,妻晉陵公主改適琅邪王練,公主雖執意不行,而
 詔其與謝氏離絕,公主以混家事委之弘微。混仍世宰輔,一門兩封,田業十餘處,僮僕千人,唯有二女,年數歲。弘微經紀生業,事若在公,一錢尺帛出入,皆有文簿。遷通直郎。高祖受命,晉陵公主降為東鄉君,以混得罪前代,東鄉君節義可嘉,聽還謝氏。自混亡,至是九載,而室宇修整,倉廩充盈,門徒業使,不異平日,田疇墾闢,有加於舊。東鄉君嘆曰:「僕射平生重此子,可謂知人。僕射為不亡矣。」中外姻親,道俗義舊,見東鄉之歸者,入門莫不歎
 息,或為之涕流,感弘微之義也。性嚴正,舉止必循禮度,事繼親之黨,恭謹過常。伯叔二母,歸宗兩姑,晨夕瞻奉,盡其誠敬。內或傳語通訊,輒正其衣冠。婢僕之前,不妄言笑,由是尊卑小大,敬之若神。



 太祖鎮江陵,宋初封宜都王,以瑯邪王球為友,弘微為文學。母憂去職。居喪以孝稱,服闋踰年,菜蔬不改。除鎮西咨議參軍。太祖即位,為黃門侍郎,與王華、王曇首、殷景仁、劉湛等號曰五臣。遷尚書吏部郎,參預機密。尋轉右衛將軍。諸故吏臣佐,並
 委弘微選擬。居身清約,器服不華,而飲食滋味,盡其豐美。



 兄曜歷御史中丞,彭城王義康驃騎長史,元嘉四年卒。弘微蔬食積時,哀戚過禮,服雖除,猶不啖魚肉。沙門釋慧琳詣弘微,弘微與之共食,猶獨蔬素。慧琳曰:「檀越素既多疾,頃者肌色微損,即吉之後,猶未復膳。若以無益傷生,豈所望於得理。」弘微答曰:「衣冠之變,禮不可踰。在心之哀,實未能已。」遂廢食感咽,歔欷不自勝。弘微少孤,事兄如父,兄弟友穆之至,舉世莫及也。弘微口不
 言人短長,而曜好臧否人物,曜每言論,弘微常以它語亂之。



 六年,東宮始建,領中庶子,又尋加侍中。弘微志在素官,畏忌權寵,固讓不拜,乃聽解中庶子。每有獻替及論時事,必手書焚草,人莫之知。上以弘微能營膳羞,嘗就求食。弘微與親故經營,既進之後,親人問上所御,弘微不答,別以餘語酬之,時人比漢世孔光。八年秋,有疾,解右衛,領太子右衛率,還家。議欲解弘微侍中,以率加吏部尚書,固陳疾篤,得免。



 九年,東鄉君薨,資財鉅萬,園
 宅十餘所,又會稽、吳興、琅邪諸處,太傅、司空琰時事業,奴僮猶有數百人。公私咸謂室內資財,宜歸二女,田宅僮僕,應屬弘微。弘微一無所取,自以私祿營葬。混女夫殷睿素好樗蒱,聞弘微不取財物,乃濫奪其妻妹及伯母兩姑之分以還戲責,內人皆化弘微之讓,一無所爭。弘微舅子領軍將軍劉湛性不堪其非,謂弘微曰:「天下事宜有裁衷。卿此不治,何以治官。」



 弘微笑而不答。或有譏之曰:「謝氏累世財產,充殷君一朝戲責,理之不允,莫
 此為大。卿親而不言,譬棄物江海以為廉耳。設使立清名,而令家內不足,亦吾所不取也。」弘微曰:「親戚爭財,為鄙之甚。今內人尚能無言,豈可導之使爭。今分多共少,不至有乏,身死之後,豈復見關。」東鄉君葬,混墓開,弘微牽疾臨赴,病遂甚。十年,卒,時年四十二。



 時有一長鬼寄司馬文宣家,云受遣殺弘微,弘微疾增劇,輒豫告文宣。弘微既死,與文宣分別而去。弘微臨終,語左右曰:「有二封書,須劉領軍至,可於前燒之,慎勿開也。」書皆是太祖
 手敕。上甚痛惜之,使二衛千人營畢葬事。追贈太常。



 子莊,別有傳。



 王球,字倩玉,琅邪臨沂人,太常惠從父弟也。父謐,司徒。球少與惠齊名,美容止。除著作佐郎,不拜。尋除琅邪王大司馬行參軍,轉主簿,豫章公世子中軍功曹。宋國建,初拜世子中舍人。高祖受命,仍為太子中舍人,宜都王友,轉咨議參軍,以疾去職。元嘉四年,起為義興太守。從兄弘為揚州,服親不得相臨,加宣威將軍,在郡有寬惠
 之美,徙太子右衛率。入為侍中,領冠軍將軍,又領本州大中正,徙中書令,侍中如故。遷吏部尚書。



 球公子簡貴,素不交遊,筵席虛靜,門無異客。尚書僕射殷景仁、領軍劉湛並執重權,傾動內外,球雖通家姻戚,未嘗往來。頗好文義,唯與琅邪顏延之相善。



 居選職,接客甚希,不視求官書疏,而銓衡有序,朝野稱之。本多羸疾,屢自陳解。



 遷光祿大夫,加金章紫綬,領廬陵王師。



 兄子履進利為行,深結劉湛,委誠大將軍彭城王義康,與劉斌、孔胤秀
 等並有異志,球每訓厲,不納。自大將軍從事中郎,轉太子中庶子,流涕訴義康不願違離,以此復為從事中郎。太祖甚銜之。及湛誅之夕,履徒跣告球。球命為取履,先溫酒與之,謂曰:「常日語汝,何如?」履怖懼不得答,球徐曰:「阿父在,汝亦何憂。」



 命左右:「扶郎還齋。」上以球故,履得免死,廢於家。



 十七年,球復為太子詹事,大夫、王師如故。未拜,會殷景仁卒,因除尚書僕射,王師如故。素有腳疾。錄尚書江夏王義恭謂尚書何尚之曰:「當今乏才,群下宜
 加戮力,而王球放恣如此,恐宜以法糾之。」尚之曰:「球有素尚,加又多疾,應以淡退求之,未可以文案責也。」猶坐白衣領職。時群臣詔見,多不即前,卑疏者或至數十日,大臣亦有十餘日不被見者。唯球輒去,未嘗肯停。十八年,卒,時年四十九。追贈特進、金紫光祿大夫,加散騎常侍。無子,從孫奐為後。大明末,吳興太守。



 或人問史臣曰:「王惠何如?」答之曰:「令明簡。」又問:「王球何如?」



 答曰:「倩玉淡。」又問:「謝弘微何如?」曰:「簡而不失,淡而不
 流,古之所謂名臣,弘微當
 之矣。」



\end{pinyinscope}