\article{卷九本紀第九 後廢帝}

\begin{pinyinscope}

 廢
 帝諱昱,字德融,小字慧震,明帝長子也。大明七年正
 月辛丑,生於衛尉府。



 太宗諸子在孕,皆以《周易》筮之,即以所得之卦為小字,故帝字慧震,其餘皇子亦如
 此。泰始二年,立為皇太子。三年,始制太子改名昱。安車乘象輅。六年,出東宮。又制太子元正朝賀,服袞冕九章衣。



 泰豫元
 年
 四月
 己亥,太宗崩。庚子,太子即皇帝位,大赦天下。尚書令袁粲、護軍將軍褚淵共輔朝政。乙巳,以護軍將軍張永為右光祿大夫,撫軍將軍安成王為揚州刺史。己酉,特進、右光祿大夫劉遵考改為左光祿大夫。五月丁巳,以吳興太守張岱為益州刺史。戊辰,緣江戍兵老疾者,悉聽還。班劍依舊入殿。六月壬辰,詔曰:「夫興王
 經制,實先民隱,方求廣教,刑於四維。朕以煢眇,夙膺寶歷,永言民政,未接聽覽,眷言乃顧,無忘鑒寐。可遣大使分行四方,觀採風謠,問其疾苦。令有咈民,法不便俗者,悉各條奏。若守宰威恩可紀,廉勤允著,依事騰聞;如獄訟誣枉,職事紕繆,惰公存私,害民利己者,無或隱昧。廣納芻輿之議,博求獻藝之規。巡省之道,務令精洽,深簡行識,俾若朕親覽焉。」又詔曰:「夫寢夢期賢,往誥垂美,物色求良,前書稱盛。朕以沖昧,嗣膺寶業,思仰述聖猷,勉弘
 政道,興言多士,常想得人。可普下牧守,廣加搜採。其有孝友聞族,義讓光閭,或匿名屠釣,隱身耕牧,足以整厲澆風,扶益淳化者,凡厥一善,咸無遺逸。虛輪佇帛,俟聞嘉薦。」京師雨水,詔賑恤二縣貧民。乙巳,尊皇后曰皇太后,立皇后江氏。秋七月戊辰,崇拜帝所生陳貴妃為皇太妃。閏月丁亥,罷宋安郡還屬廣興。



 己丑,割南豫州南汝陰郡屬西豫州,西豫州廬江郡屬豫州。甲辰,以新除征西將軍、開府儀同三司、荊州刺史蔡興宗為中書監、
 光祿大夫,安西將軍、郢州刺史沈攸之為鎮西將軍、荊州刺史,南徐州刺史劉秉為平西將軍、郢州刺史,新除太常建平王景素為鎮軍將軍、南徐州刺史。八月戊午,新除中書監、左光祿大夫、開府儀同三司蔡興宗薨。冬十月辛卯,撫軍將軍劉韞有罪免官。辛未,護軍將軍褚淵母憂去職。



 十一月己亥,新除平西將軍、郢州刺史劉秉為左僕射。辛丑,護軍將軍褚淵還攝本任。芮芮國、高麗國遣使獻方物。十二月,索虜寇義陽。丁巳,司州
 刺史王瞻擊破之。



 元徽元年春正月戊寅朔,改元,大赦天下。壬寅,詔曰:「夫緩法昭恩,裁風茂典,蠲憲貸眚,訓俗彞義。朕臨馭宸樞,夤制氓宇,式存寬簡,思孚矜惠。今開元肆宥,萬品惟新,凡茲流斥,宜均弘洗。自元年以前貽罪徙放者,悉聽還本。」



 二月乙亥,以晉熙王燮為郢州刺史。三月丙申,以撫軍長史何恢為廣州刺史。婆利國遣使獻方物。戊戌,以前淮南太守劉靈遺為南豫州刺史。夏五月辛卯,以
 輔師將軍李安民為司州刺史。丙申,河南王遣使獻方物。六月壬子,以越州刺史陳伯紹為交州刺史。乙卯,特進、左光祿大夫劉遵考卒。壽陽大水,己未,遣殿中將軍賑恤慰勞。丙寅,以左軍將軍孟次陽為兗州刺史。秋七月丁丑,散騎常侍顧長康、長水校尉何翌之表上所撰《諫林》,上自虞、舜,下及晉武,凡十二卷。八月辛亥,詔曰:「分方正俗,著自虞冊,川谷異制,煥乎姬典。故井遂有辨,閭伍無雜,用能七教克宣,八政斯序。雖綿代殊軌,沿革異
 儀,或民懷遷俗,或國尚興徙,漢陽列燕、代之豪,關西熾齊、楚之族,並通籍新邑,即居成舊。洎金行委御,禮樂南移,中州黎庶,襁負揚、越。聖武造運,道一閎區,貽長世之規,申土斷之制。而夷險相因,盈晦遞襲,歲饉凋流,戎役惰散,違鄉寓境,漸至繁積。宜式遵鴻軌,以為永憲,庶阜俗昌民,反風定保。夷胥山之險,澄瀚海之波,括《河圖》於九服,振玉軔於五都矣。」秘書丞王儉表上所撰《七志》三十卷。京師旱。甲寅,詔曰:「比亢序愆度,留熏耀晷,有傷秋
 稼,方貽民瘼。朕以眇疚,未弘政道,囹圄尚繁,枉滯猶積,夕厲晨矜,每惻于懷。尚書令可與執法以下,就訊眾獄,使冤訟洗遂,困弊昭蘇。頒下州郡,咸令無壅。」癸亥,鎮軍將軍、南徐州刺史建平王景素進號鎮北將軍。庚午,陳留王曹銑薨。九月壬午,詔曰:「國賦氓稅,蓋有恒品,往屬戎難,務先軍實,徵課之宜,或乖昔準。湘、江二州,糧運偏積,調役既繁,庶徒彌擾。因循權政,容有未革,民單力弊,歲月愈甚。永言矜歎,情兼宵寐。可遣使到所,明加詳察。
 其輸違舊令,役非公限者,並即蠲改,具條以聞。」丁亥,立衡陽王嶷子伯玉為南平王。冬十月壬子,以撫軍司馬王玄載為梁、南秦二州刺史。癸酉,割南兗州之鐘離、豫州之馬頭,又分秦郡、梁郡、歷陽置新昌郡,立徐州。十一月丙子,以散騎常侍垣閎為徐州刺史。丁丑,尚書令袁粲母喪去職。十二月癸卯朔,日有蝕之。乙巳,司空、江州刺史桂陽王休範進位太尉,尚書令袁粲還攝本任,加號衛將軍。癸亥,立前建安王世子伯融為始安縣王。丙
 寅,河南王遣使獻方物。



 二年春正月庚子,以右光祿大夫張永為征北將軍、南兗州刺史。二月己巳,加護軍將軍褚淵中軍將軍。三月癸酉,以左衛將軍王寬為南豫州刺史。夏四月癸亥,詔曰:「頃列爵敘勳,銓榮酬義,條流積廣,又各淹闕。歲往事留,理至逋壅,在所參差,多違甄飭。賞未均洽,每疚厥心。可悉依舊準,並下注職。」五月壬午,太尉、江州刺史桂陽王休範舉兵反。庚寅,內外戒嚴。加中領軍劉勔鎮軍將
 軍,加右衛將軍齊王平南將軍,前鋒南討,出屯新亭。征北將軍張永屯白下,前南兗州刺史沈懷明戍石頭,衛將軍袁粲、中軍將軍褚淵入衛殿省。壬辰,賊奄至,攻新亭壘。



 齊王拒擊,大破之。越騎校尉張敬兒斬休範。賊黨杜黑蠡、丁文豪分軍向朱雀航,劉勔拒賊敗績,力戰死之;右軍將軍王道隆奔走遇害。張永潰於白下,沈懷明自石頭奔散。戊午,撫軍典簽茅恬開東府納賊,賊入屯中堂。羽林監陳顯達擊大破之。



 丙申,張敬兒等破賊於
 宣陽門、莊嚴寺、小市,進平東府城,梟擒群賊。賞賜封爵各有差。丁酉,詔京邑二縣埋藏所殺賊,并戰亡者,復同京城。是日解嚴,大赦天下,文武賜位一等。戊戌,原除江州逋債,其有課非常調、役為民蠹者,悉皆蠲停。



 詔曰:「頃國賦多騫,公儲罕給。近治戎雖淺,而軍費已多,廩藏虛罄,難用馭遠。



 宜矯革淫長,務在節儉。其供奉服御,悉就減撤,雕文靡麗,廢而勿脩。凡諸游費,一皆禁斷,外可詳為科格。」荊州刺史沈攸之、南徐州刺史建平王景素、郢
 州刺史晉熙王燮、湘州刺史王僧虔、雍州刺史張興世並舉義兵赴京師。己亥,以第七皇弟友為江州刺史。芮芮國遣使獻方物。六月庚子,以平南將軍齊王為中領軍、鎮軍將軍、南兗州刺史。癸卯,晉熙王燮遣軍克尋陽,江州平。戊申,以淮南太守任農夫為豫州刺史,右將軍、南豫州刺史王寬進號平西將軍。壬戌,改輔師將軍還為輔國。



 秋七月庚辰,立第七皇弟友為邵陵王。辛巳,以撫軍司馬孟次陽為兗州刺史。乙酉,鎮西將軍、荊州刺
 史沈攸之進號征西大將軍,鎮北將軍、南徐州刺史建平王景素進號征北將軍,並開府儀同三司。征虜將軍、郢州刺史晉熙王燮進號安西將軍,前將軍、湘州刺史王僧虔進號平南將軍。八月辛酉,以征虜行參軍劉延祖為寧州刺史。



 九月壬辰,以遊擊將軍呂安國為兗州刺史。丁酉,以尚書令、新除衛將軍袁粲為中書監,即本號開府儀同三司,領司徒;加護軍將軍褚淵尚書令;撫軍將軍、揚州刺史安成王進號車騎將軍。冬十月庚申,以
 新除侍中王蘊為湘州刺史。甲子,以遊擊將軍陳顯達為廣州刺史。十一月丙戌,御加元服,大赦天下。賜民男子爵一級;為父後及三老孝悌力田者爵二級;鰥寡孤獨篤癃不能自存者,穀五斛;年八十以上,加帛一匹。大酺五日,賜王公以下各有差。十二月癸亥,立第八皇弟躋為江夏王,第九皇弟贊為武陵王。



 三年春正月辛巳,車駕親祠南郊、明堂。三月丙寅,河南王遣使獻方物。己巳,以車騎將軍張敬兒為雍州刺史。
 其日,京師大水,遣尚書郎官長檢行賑賜。閏月戊戌,詔曰:「頃民俗滋弊,國度未殷,歲時屢騫,編戶不給。且邊虞尚警,徭費彌繁,永言夕惕,寢興增疚。思弘豐耗之制,以惇約素之風,庶偫蓄拯民,以康治道。



 大官珍膳,御府麗服,諸所供擬,一皆減撤,可詳為其格,務從簡衷。」夏四月,遣尚書郎到諸州檢括民戶,窮老尤貧者,蠲除課調;丁壯猶有生業,隨宜寬申;貲財足以充限者,督令洗畢。丙戌,車駕幸中堂聽訟。六月癸未,北國使至。兼司徒袁粲、
 尚書令褚淵並固讓。秋七月庚戌,以粲為尚書令。壬戌,以給事黃門侍郎劉懷珍為豫州刺史。八月庚子,加護軍將軍褚淵中書監。九月丙辰,征西大將軍河南王吐谷渾拾夤進號車騎大將軍。冬十月丙戌,高麗國遣使獻方物。十二月乙丑,以冠軍將軍姚道和為司州刺史。



 四年春正月己亥,車駕躬耕籍田,大赦天下。賜力田爵一級;貸貧民糧種。壬子,以梁、南秦二州刺史王玄載為益州刺史。二月壬戌,以步兵校尉範柏年為梁、南秦二
 州刺史。丁卯,加金紫光祿大夫王琨特進。夏五月,以寧朔將軍武都王楊文度為北秦州刺史。乙未,尚書右丞虞玩之表陳時事曰:天府虛散,垂三十年。江、荊諸州,稅調本少,自頃以來,軍募多乏。其穀帛所入,折供文武。豫、兗、司、徐,開口待哺;西北戎將,裸身求衣。委輸京都,蓋為寡薄。天府所資,唯有淮、海。民荒財單,不及曩日。而國度弘費,四倍元嘉。



 二衛臺坊人力,五不餘一;都水材官朽散,十不兩存。備豫都庫,材竹俱盡;東西二嵒,磚瓦雙匱。
 敕令給賜,悉仰交市。尚書省舍,日就傾頹,第宅府署,類多穿毀。視不遑救,知不暇及。尋所入定調,用恒不周,既無儲畜,理至空盡。積弊累耗,鐘於今日。昔歲奉敕,課以揚、徐眾逋,凡入米穀六十萬斛,錢五千餘萬,布絹五萬匹,雜物在外,賴此相贍,故得推移。即今所懸轉多,興用漸廣,深懼供奉頓闕,軍器輟功,將士飢怨,百官騫祿。署府謝雕麗之器,土木停緹紫之容,國戚無以贍,勳求無以給。如愚管所慮,不月則歲矣。



 經國遠謀,臣所不敢言,
 朝夕祗勤,心存於匪懈。起伏震遽,事屬冒聞。伏願陛下留須臾之鑒,垂永代之計,發不世之詔,施必行之典,則氓祗齊懽,高卑同泰。



 帝優詔答之。庚戌,以驍騎將軍曹欣之為徐州刺史。六月乙亥,加鎮軍將軍齊王尚書左僕射。秋七月戊子,征北將軍、南徐州刺史建平王景素據京城反。己丑,內外纂嚴。遣驍騎將軍任農夫、領軍將軍黃回北討,鎮軍將軍齊王總統眾軍。曲赦南徐州;始安王伯融、都鄉侯伯猷賜死。辛卯,南豫州刺史段佛榮統
 前鋒馬步眾軍。



 甲午,軍主、左軍將軍張保戰敗見殺。黃回等至京城,與景素諸軍戰,連破之。乙未,克京城,斬景素,同逆皆伏誅;其日解嚴。丙申,大赦天下,封賞各有差。原京邑二縣元年以前逋調。辛丑,以武陵王贊為南徐州刺史。八月丁卯,立第十皇弟翽為南陽王,第十一皇弟嵩為新興王,第十二皇弟禧為始建王。庚午,以給事黃門侍郎阮佃夫為南豫州刺史。乙酉,以行青、冀二州刺史劉善明為青、冀二州刺史。



 九月丁亥,割郢州之隨
 郡屬司州。戊子,驍騎將軍高道慶有罪,賜死。己丑,車騎將軍、揚州刺史安成王進號驃騎大將軍、開府儀同三司,安西將軍、郢州刺史晉熙王燮進號鎮西將軍。冬十月辛酉,以吏部尚書王僧虔為尚書右僕射。宕昌王梁彌機為安西將軍、河涼二州刺史。丙寅,中書監、護軍將軍褚淵母憂去職。十一月庚戌,詔攝本任。



 五年春二月壬申,以建寧太守柳和為寧州刺史。四月甲戌,豫州刺史阮佃夫、步兵校尉申伯宗、朱幼謀廢立,
 佃夫、幼下獄死,伯宗伏誅。五月己亥,以左軍將軍沈景德為交州刺史,驍騎將軍全景文為南豫州刺史。丙午,以屯騎校尉孫曇瓘為越州刺史。六月甲戌,誅司徒左長史沈勃、散騎常侍杜幼文、遊擊將軍孫超之、長水校尉杜叔文,大赦天下。



 七月戊子夜,帝殞於仁壽殿,時年十五。己丑,皇太后令曰:衛將軍、領軍、中書監、八座:昱以塚嫡,嗣登皇統,庶其體識日弘,社稷有寄。豈意窮兇極悖,自幼而長,善無細而不違,惡有大而必蹈。前後訓誘,
 常加隱蔽,險戾難移,日月滋甚。棄冠毀冕,長襲戎衣,犬馬是狎,鷹隼是愛,皁櫪軒殿之中,韝紲宸扆之側。至仍單騎遠郊,獨宿深野,手揮矛金延,躬行刳斮,白刃為弄器,斬害為恒務。捨交戟之衛,委天畢之儀,趨步闤闠,酣歌壚肆,宵遊忘反,宴寢營舍,奪人子女,掠人財物,方策所不書,振古所未聞。沈勃儒士,孫超功臣,幼文兄弟,並豫勛效,四人無罪,一朝同戮。飛鏃鼓劍,孩稚無遺,屠裂肝腸,以為戲謔,投骸江流,以為懽笑。又淫費無度,帑藏空
 竭,橫賦關河,專充別蓄,黔庶嗷嗷,厝生無所。吾與其所生,每厲以義方,遂謀鴆毒,將騁凶忿。沈憂假日,慮不終朝。自昔辛、癸,爰及幽、厲,方之於此,未譬萬分。民怨既深,神怒已積,七廟阽危,四海褫氣。



 廢昏立明,前代令範,況乃滅義反道,天人所棄,釁深牧野,理絕桐宮。故密令蕭領軍潛運明略,幽顯協規,普天同泰。驃騎大將軍安王體自太宗,天挺淹睿,風神凝遠,德映在田。地隆親茂,皇歷攸歸,億兆係心,含生屬望。宜光奉祖宗,臨享萬國。
 便依舊典,以時奉行。未亡人追往傷懷,永言感絕。



 太后又令曰:「昱窮凶極暴,自取灰滅,雖曰罪招,能無傷悼。棄同品庶,顧所不忍。可特追封蒼梧郡王。」葬丹陽秣陵縣郊壇西。



 初,昱在東宮,年五六歲時,始就書學,而惰業好嬉戲,主師不能禁。好緣漆賬竿,去地丈餘,如此者半食久,乃下。年漸長,喜怒乖節,左右有失旨者,輒手加撲打。徒跣蹲踞,以此為常。主師以白太宗,上輒敕昱所生,嚴加捶訓。及嗣位,內畏太后,外憚諸大臣,猶未得肆志。自
 加元服,變態轉興,內外稍無以制。三年秋冬間,便好出遊行,太妃每乘青篾車,隨相檢攝。昱漸自放恣,太妃不復能禁。



 單將左右,棄部伍,或十里、二十里,或入市里,或往營署,日暮乃歸。四年春夏,此行彌數。自京城剋定,意志轉驕,於是無日不出。與左右人解僧智、張五兒恒相馳逐,夜出,開承明門,夕去晨反,晨出暮歸。從者並執金延矛,行人男女,及犬馬牛驢,值無免者。民間擾懼,晝日不敢開門,道上行人殆絕。常著小褲褶,未嘗服衣冠。或有
 忤意,輒加以虐刑。有白棓數十枚,各有名號,鍼椎鑿鋸之徒,不離左右。嘗以鐵椎椎人陰破,左右人見之有斂眉者,昱大怒,令此人袒胛正立,以矛刺胛洞過。於耀靈殿上養驢數十頭,所自乘馬,養於御床側。先是民間訛言,謂太宗不男,陳太妃本李道兒妾,道路之言,或云道兒子也。昱每出入去來,常自稱李統,或自號李將軍。與右衛翼輦營女子私通,每從之遊,持數千錢,供酒肉之費。



 阮佃夫腹心人張羊為佃夫所委信。佃夫敗,叛走,後
 捕得,昱自於承明門以車轢殺之。杜延載、沈勃、杜幼文、孫超,皆躬運矛金延,手自臠割。執幼文兄叔文於玄武湖北,昱馳馬執槊,自往刺之。制露車一乘,其上施篷,乘以出入,從者不過數十人。羽儀追之恒不及,又各慮禍,亦不敢追尋,唯整部伍,別在一處瞻望而已。



 凡諸鄙事,過目則能,鍛煉金銀,裁衣作帽,莫不精絕。未嘗吹篪,執管便韻,天性好殺,以此為懽,一日無事,輒慘慘不樂。內外百司,人不自保,殿省憂遑,夕不及旦。



 齊王順天人之心,
 潛圖廢立,與直閣將軍王敬則謀之。七月七日,昱乘露車,從二百許人,無復鹵簿羽儀,往青園尼寺,晚至新安寺就曇度道人飲酒。醉,夕扶還於仁壽殿東阿氈幄中臥。時昱出入無恒,省內諸皞,夜皆不閉。且群下畏相逢值,無敢出者。宿衛並逃避,內外無相禁攝。王敬則先結昱左右楊玉夫、楊萬年、呂欣之、湯成之、陳奉伯、張石留、羅僧智、鐘千載、嚴道福、雷道賜、戴昭祖、許啟、戚元寶、盛道泰、鐘千秋、王天寶、公上延孫、俞成、錢道寶、馬敬之、陳
 寶直、吳璩之、劉印魯、唐天寶、俞孫等二十五人,謀共取昱。其夕,敬則出外,玉夫見昱醉熟無所知,乃與萬年同入氈幄內,以昱防身刀斬之。奉伯提昱首,依常行法,稱敕開承明門出,以首與敬則,馳至領軍府,以首呈齊王。王乃戎服,率左右數十人,稱行還,開承明門入。昱他夕每開門,門者震懾不敢視,至是弗之疑。齊王既入,曉,乃奉太后令奉迎安成王。



 史臣曰:喪國亡家之主,雖適末同途,發軫或異也。前廢
 帝卑游褻幸,皆龍駕帝飾,傳警清路;蒼梧王則藏璽懷紱,魚服忘反,危冠短服,匹馬孤征。至於殞身覆祚,其理若一。姬、夏之隆,質文異尚,亡國之道,其亦然乎!



\end{pinyinscope}