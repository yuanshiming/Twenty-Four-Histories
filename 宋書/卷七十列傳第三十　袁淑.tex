\article{卷七十列傳第三十 袁淑}

\begin{pinyinscope}

 袁淑,字陽源,陳郡陽夏人,丹陽尹豹少子也。少有風氣,年數歲,伯湛謂家人曰:「此非凡兒。」至十餘歲,為姑夫王弘所賞。不為章句之學,而博涉多通,好屬文,辭采遒艷,
 縱橫有才辯。本州命主簿,著作佐郎,太子舍人,並不就。彭城王義康命為軍司祭酒。義康不好文學,雖外相禮接,意好甚疏。劉湛,淑從母兄也,欲其附己,而淑不以為意,由是大相乖失,以久疾免官。補衡陽王義季右軍主簿,遷太子洗馬,以腳疾不拜。衛軍臨川王義慶雅好文章,請為諮議參軍。頃之,遷司徒左西屬。出為宣城太守,入補中書侍郎,以母憂去職。服闋,為太子中庶子。



 元嘉二十六年,遷尚書吏部郎。其秋,大舉北伐,淑侍坐從容
 曰:「今當鳴鑾中岳,席卷趙、魏,檢玉岱宗,今其時也。臣逢千載之會,願上《封禪書》一篇。」



 太祖笑曰:「盛德之事,我何足以當之。」出為始興王征北長史、南東海太守。淑始到府,濬引見,謂曰:「不意舅遂垂屈佐。」淑答曰:「朝廷遣下官,本以光公府望。」還為御史中丞。時索虜南侵,遂至瓜步,太祖使百官議防禦之術,淑上議曰:臣聞函車之獸,離山必斃;絕波之鱗,宕流則枯。羯寇遺醜,趨致畿甸,蟻萃螽集,聞已崩殪。天險巖曠,地限深遐,故全魏戢其圖,盛
 晉輟其議,情屈力殫,氣挫勇竭,諒不虞於來臨,本無怵於能濟矣。乃者燮定攜遠,阻違授律,由將有弛拙,故士少鬥志。圍潰之眾,匪寇傾淪,攻制之師,空自班散,濟西勁騎,急戰蹴旅,淮上訓卒,簡備靡旗。是由綏整寡衷,戎昭多昧,遂使栲潞入患,泉伊來擾,紛殄姬風,泯毒禹績,騰書有渭陰之迫,懸烽均咸陽之警。然而切揣虛實,伏匿先彰,校索伎能,譎詭既顯。綿地千里,彌行阻深,表裏躓硋,後先介逼。捨陵衍之習,競湍沙之利。今虹見萍生,
 土膏泉動,津陸陷溢,痁禍洊興,芻稿已單,米粟莫系,水宇衿帶,進必傾殞,河隘扁固,退亦墮滅。所謂棲烏於烈火之上,養魚於叢棘之中。



 或謂損緩江右,寬繕淮內。竊謂拯扼閩城,舊史為允,棄遠涼士,前言稱非。



 限此要荒,猶弗委割。況聯被京國,咫尺神甸,數州摧掃,列邑殲痍,山淵反覆,草木塗地。今丘賦千乘,井算萬集,肩摩倍於長安,締袂百於臨淄,什一而籍,實慊氓願,履畝以稅,既協農和。戶競戰心,人含銳志,皆欲贏糧請奮,釋緯乘城。



 謂宜懸金鑄印,要壯果之士,重幣甘辭,招摧決之將,舉薦板築之下,抽登臺皁之間,賞之以焚書,報之以相爵,俄而昭才賀闕,異能間至。



 戎貪而無謀,肆而不整,迷乎向背之次,謬於合散之宜,犯軍志之極害,觸兵家之甚諱。咸畜憤矣,僉策戰矣,稱願影從,謠言緡命。宜選敢悍數千,騖行潛掩,偃旗裹甲,鉗馬銜枚,檜稽而起,晨壓未陣,旌噪亂舉,火鼓四臨,使景不暇移,塵不及起,無不禽鎩獸讋,冰解霧散,掃洗哨類,漂鹵浮山。如有決罦漏網,
 逡窠逗穴,命淮、汝戈船,遏其還徑,兗部勁卒,梗其歸塗。必剪元雄,懸首麾下,乃將隻輪不反,戰轊無旋矣。於是信臣騰威,武士繕力,緹組接陰,鞞柝聯響。



 若其偽遁羸張,出沒無際,楚言漢旆,顯默如神,固已日月蔽虧,川谷蕩貿。



 負塞殘孽,阻山燼黨,收險竊命,憑城借土,則當因威席卷,乘機芟剿。泗、汴秀士,星流電燭,徐、阜嚴兵,雨湊雲集,蹶亂桑溪之北,搖潰浣海以南,絕其心根,勿使能植,銜索之枯,幾何不蠹。是由涸澤而漁,焚林而狩,若浚
 風之人舞輕籜,杲日之拂浮霜。既而尉洽荷掠之餘,望吊網悲之鬼。然後天行樞運,猋舉煙升,青蓋西巡,翠華東幸,經啟州野,滌一軫策,俾高闕再勒,燕然後銘。方乃奠山沉河,創禮輯策,闡耀炎、昊之遺則,貫軼商、夏之舊文。



 今眾賈拳勇,而將術疏怯,意者稔泰日積,承平歲久,邑無驚赴之急,家緩饋戰之勤,闕閱訓之禮,簡參屬之飾,且亦薦採之法,庸未蔇歟。若乃邦造里選,攉論深切,躬擐盡幽,斬帶尋遠,設有沉明能照,俊偉自宣,誠感泉雨,
 流通金石,氣懾飛、賁,知窮苴、起,審邪正順逆之數,達昏明益損之宜,能睽合民心,愚睿物性,登丹墀而敷策,躡青蒲而揚謀,上說辰鑒,下弭素言,足以安民紓國,救災恤患。則宜拔過寵貴之上,褒升戚舊之右,別其旂章,榮其班祿,出得專譽,使不稟命。降席折節,同廣武之請;設壇致禮,均淮陰之授。必有要盟之功,竊符之捷。



 夷裔暴狠,內外侮棄,始附之眾,分茷無序,蠱以威利,勢必攜離,首順之徒,靡然自及。今淶繹故典,瀍土纓緌,翦焉幽播,
 折首凶狡。是猶眇者願明,痿之思步,動商遄會,功終易感。劫晉在於善覘,全鄭實寄良諜,多縱反間,汨惑心耳,發險易之前,抵興喪之術,衝其猜伏,拂其嫌嗜,汨以連率之貴,餌以析壤之資。



 罄筆端之用,展辭鋒之銳,振辯則堅圍可解,馳羽而巖邑易傾。必府鬲土崩,枝幹瓦裂,故燕、樂相悔,項、范交疑矣。



 或乃言約功深,事邇應廣,齊圉反駕,趙養還君,盡輿誦之道,畢能事之效。



 臣幸得出內層禁,游心明代,澤與身泰,恩隨年行,無以逢迎昌運,
 潤飾鴻法。今塗有遺鏃,蠆未息蜂,敢思涼識,少酬閎施。但坐幕既乏昭文,免胄不能致果,竊觀都護之邊論,屬國之兵謨,終、晁之抗辭,杜、耿之言事,咸云及經之棘,猶闕上算,燭郛之敬,裁收下策。自恥懦木,智不綜微,敢露昧見,無會昭採。



 淑喜為誇誕,每為時人所嘲。始興王浚嘗送錢三萬餉淑,一宿復遣追取,謂使人謬誤,欲以戲淑。淑與浚書曰:「袁司直之視館,敢寓書於上國之宮尹。日者猥枉泉賦,降委弊邑。弊邑敬事是遑,無或違貳。懼
 非郊贈之禮,覲饗之資,不虞君王惠之於是也,是有懵焉。弗圖旦夕發咫尺之記,籍左右而請,以為胥授失旨,爰速先幣。曾是附庸臣委末學孤聞者,如之何勿疑。且亦聞之前志曰,七年之中,一與一奪,義士猶或非之。況密邇旬次,何其裒益之亟也。藉恐二三諸侯,有以觀大國之政。是用敢布心腹。弊室弱生,砥節清廉,好是潔直,以不邪之故,而貧聞天下。寧有昧夫嗟金者哉。不腆供賦,束馬先璧以俟命。唯執事所以圖之。」



 遷太子左衛率。
 元凶將為弒逆,其夜淑在直,二更許,呼淑及蕭斌等流涕謂曰:「主上信讒,將見罪廢。內省無過,不能受枉。明旦便當行大事,望相與戮力。」



 淑及斌並曰:「自古無此,願加善思。」劭怒變色,左右皆動。斌懼,乃曰:「臣昔忝伏事,常思效節,況憂迫如此,輒當竭身奉令。」淑叱之曰:「卿便謂殿下真有是邪?殿下幼時嘗患風,或是疾動耳。」劭愈怒,因問曰:「事當克不?」淑曰:「居不疑之地,何患不克。但既克之後,為天地之所不容,大禍亦旋至耳。願急息之。」劭左右
 引淑等褲褶,又就主衣取錦,截三尺為一段,又中破,分斌、淑及左右,使以縛褲。淑出環省,繞床行,至四更乃寢。劭將出,已與蕭斌同載,呼淑甚急,淑眠終不起。劭停車奉化門,催之相續。徐起至車後,劭使登車,又辭不上。



 劭因命左右:「與手刃。」見殺於奉化門外,時年四十六。劭即位,追贈太常,賜賵甚厚。



 世祖即位,使顏延之為詔曰:「夫輕道重義,亟聞其教;世弊國危,希遇其人。



 自非達義之至,識正之深者,孰能抗心衛主,遺身固節者哉!故太子
 左衛率淑,文辯優洽,秉尚貞愨。當要逼之切,意色不橈,厲辭道逆,氣震凶黨。虐刃交至,取斃不移。古之懷忠隕難,未云出其右者。興言嗟悼,無廢乎心。宜在加禮,永旌宋有臣焉。可贈侍中、太尉,謚曰忠憲公。」又詔曰:「袁淑以身殉義,忠烈邈古。



 遺孤在疚,特所矜懷。可厚加賜恤,以慰存亡。」淑及徐湛之、江湛、王僧綽、卜天與四家,於是長給稟祿。文集傳於世。



 子幾、敳、稜、凝、標。敳,世祖步兵校尉。凝,太宗世御史中丞,出為晉陵太守。太宗初與四方同
 反,兵敗歸降,以補劉湛冠軍府主簿。淑諸子並早卒。



 史臣曰:天長地久,人道則異於斯。蕣華朝露,未足以言也。其間夭遽,曾何足云。宜任心去留,不以存沒嬰心。徒以靈化悠遠,生不再來,雖天行路險,而未之斯遇,謂七尺常存,百年可保也。所以據洪圖而輕天下,吝寸陰而敗尺璧。若乃義重乎生,空炳前誥,投軀殉主,世罕其人。若無陽源之節,丹青何貴焉爾!



\end{pinyinscope}