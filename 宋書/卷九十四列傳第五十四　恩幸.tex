\article{卷九十四列傳第五十四 恩幸}

\begin{pinyinscope}

 夫君子小人,類物之稱。蹈道則為君子,違之則為小人。屠釣,卑事也;版築,賤役也,太公起為周師,傅說去為殷相。非論公侯之世,鼎食之資,明揚幽仄,唯才是與。逮于
 二漢,茲道未革,胡廣累世農夫,伯始致位公相;黃憲牛醫之子,叔度名重京師。且任子居朝,咸有職業,雖七葉珥貂,見崇西漢,而侍中身奉奏事,又分掌御服。東方朔為黃門侍郎,執戟殿下。郡縣掾史,並出豪家,負戈宿衛,皆由勢族,非若晚代,分為二途者也。漢末喪亂,魏武始基,軍中倉卒,權立九品。



 蓋以論人才優劣,非為世族高卑。因此相沿,遂為成法。自魏至晉,莫之能改,州都郡正,以才品人,而舉世人才,升降蓋寡。徒以馮藉世資,用相
 陵駕,都正俗士,斟酌時宜,品目少多,隨事俯仰,劉毅所云「下品無高門,上品無賤族」者也。歲月遷訛,斯風漸篤,凡厥衣冠,莫非二品,自此以還,遂成卑庶。周、漢之道,以智役愚,臺隸參差,用成等級;魏晉以來,以貴役賤,士庶之科,較然有辨。夫人君南面,九重奧絕,陪奉朝夕,義隔卿士,階闥之任,宜有司存。既而恩以幸生,信由恩固,無可憚之姿,有易親之色。孝建、泰始,主威獨運,官置百司,權不外假,而刑政糾雜,理難遍通,耳目所寄,事歸近習。
 賞罰之要,是謂國權,出內王命,由其掌握,於是方途結軌,輻湊同奔。人主謂其身卑位薄,以為權不得重。曾不知鼠憑社貴,狐藉虎威,外無逼主之嫌,內有專用之功,勢傾天下,未之或悟。



 挾朋樹黨,政以賄成,鈇鉞創磐,構于筵笫之曲,服冕乘軒,出乎言笑之下。南金北毳,來悉方艚,素縑丹魄,至皆兼兩,西京許、史,蓋不足云,晉朝王、庾,未或能比。及太宗晚運,慮經盛衰,權幸之徒,懾憚宗戚,欲使幼主孤立,永竊國權,構造同異,興樹禍隙,帝弟
 宗王,相繼屠劋。民忘宋德,雖非一途,寶祚夙傾,實由於此。嗚呼!《漢書》有《恩澤侯表》,又有《佞倖傳》。今採其名,列以為《恩倖篇》云。



 戴法興,會稽山陰人也。家貧,父碩子,販珝為業。法興二兄延壽、延興並脩立,延壽善書,法興好學。山陰有陳載者,家富,有錢三千萬,鄉人咸云:「戴碩子三兒,敵陳載三千萬錢。」



 法興少賣葛于山陰市,後為吏傳署,入為尚書倉部令史。大將軍彭城王義康於尚書中覓了了令史,
 得法興等五人,以法興為記室令史。義康敗,仍為世祖征虜、撫軍記室掾。上為江州,仍補南中郎典簽。上於巴口建義,法興與典簽戴明寶、蔡閑俱轉參軍督護。上即位,並為南臺侍禦史,同兼中書通事舍人。法興等專管內務,權重當時。孝建元年,加建武將軍、南魯郡太守,解舍人,侍太子於東宮。大明二年,三典簽並以南下預密謀,封法興吳昌縣男,明寶湘鄉縣男,閑高昌縣男,食邑各三百戶。閑時已卒,追加爵封。法興轉員外散騎侍郎,
 給事中,太子旅賁中郎將,太守如故。



 世祖親覽朝政,不任大臣,而腹心耳目,不得無所委寄。法興頗知古今,素見親待,雖出侍東宮,而意任隆密。魯郡巢尚之,人士之末,元嘉中,侍始興王濬讀書,亦涉獵文史,為上所知。孝建初,補東海國侍郎,仍兼中書通事舍人。凡選授遷轉誅賞大處分,上皆與法興、尚之參懷,內外諸雜事,多委明寶。



 上性嚴暴,睚眥之間,動至罪戮,尚之每臨事解釋,多得全免,殿省甚賴之。



 而法興、明寶大通人事,多納貨
 賄,凡所薦達,言無不行,天下輻湊,門外成市,家產並累千金。明寶驕縱尤甚,長子敬為揚州從事,與上爭買御物。六宮嘗出行,敬盛服騎馬於車左右,馳驟去來。上大怒,賜敬死,繫明寶尚方,尋被原釋,委任如初。



 世祖崩,前廢帝即位,法興遷越騎校尉。時太宰江夏王義恭錄尚書事,任同總己,而法興、尚之執權日久,威行內外,義恭積相畏服,至是懾憚尤甚。廢帝未親萬機,凡詔敕施為,悉決法興之手;尚書中事無大小,專斷之。顏師伯、義恭
 守空名而已。廢帝年已漸長,凶志轉成,欲有所為,法興每相禁制,每謂帝曰:「官所為如此,欲作營陽耶?」帝意稍不能平。所愛幸閹人華願兒有盛寵,賜與金帛無算,法興常加裁減,願兒甚恨之。帝常使願兒出入市里,察聽風謠,而道路之言,謂法興為真天子,帝為應天子。願兒因此告帝曰:「外間云宮中有兩天子,官是一人,戴法興是一人。官在深宮中,人物不相接;法興與太宰、顏、柳一體,吸習往來,門客恒有數百,內外士庶,莫不畏服之。法
 興是孝武左右,復久在宮闈,今將他人作一家,深恐此坐席非復官許。」帝遂發怒,免法興官,遣還田里,仍復徙付遠郡,尋又於家賜死,時年五十二。法興臨死,封閉庫藏,使家人謹錄鑰牡。死一宿,又殺其二子,截法興棺,焚之,籍沒財物。法興能為文章,頗行於世。



 死後,帝敕巢尚之曰:「吾纂承洪基,君臨萬國,推心勳舊,著於遐邇。不謂戴法興恃遇負恩,專作威福,冒憲黷貨,號令自由,積釁累愆,遂至於此。卿等忠勤在事,吾乃具悉,但道路之言,
 異同紛糾,非唯人情駭愕,亦玄象違度,委付之旨,良失本懷。吾今日親覽萬機,留心庶事,卿等宜竭誠盡力,以副所期。」尚之時為新安王子鸞撫軍中兵參軍、淮陵太守。乃解舍人,轉為撫軍諮議參軍,太守如故。



 太宗泰始二年,詔曰:「故越騎校尉吳昌縣開國男戴法興,昔從孝武,誠勤左右,入定社稷,預誓河山。及出侍東儲,竭盡心力,嬰害兇悖,朕甚愍之。可追復削注,還其封爵。」有司奏以法興孫靈珍襲封。又詔曰:「法興小人,專權豪姿,雖虐
 主所害,義由國討,不宜復貪人之封,封爵可停。」太宗初,復以尚之兼中書通事舍人、南清河太守。二年,遷中書侍郎,太守如故。未拜,改除前軍將軍,太守如故,侍太子於東宮。晉安王子勛平後,以軍守管內,封邵陵縣男,食邑四百戶,固辭不受。轉黃門侍郎,出為新安太守,病卒。



 戴明寶,南東海丹徒人也。亦歷員外散騎侍郎,給事中。世祖世,帶南清河太守。前廢帝即阼,權任悉歸法興,而明寶輕矣,以為宣威將軍、南東莞太守。景和末,增邑百
 戶。太宗初,天下反叛,軍務煩擾,以明寶舊人,屢經戎事,復委任之,以為前軍將軍。事平,遷宣威將軍、晉陵太守,進爵為侯,增邑四百戶。泰始三年,坐參掌戎事,多納賄貨,削增封官爵,繫尚方,尋被宥。復為安陸太守,加寧朔將軍,游擊、驍騎將軍,武陵內史,宣城太守,順帝驃騎司馬。昇明初,年老,拜太中大夫,病卒。



 武陵國典書令董元嗣,與法興、明寶等俱為世祖南中郎典簽。元嘉三十年,奉使還都,值元凶弒立,遣元嗣南還,報上以徐湛之等反。
 上時在巴口,元嗣具言弒狀。上遣元嗣下都,奉表於劭。既而上舉義兵,劭責元嗣,元嗣答曰:「始下,未有反謀。」劭不信,備加考掠,不服,遂死。世祖事克,追贈員外散騎侍郎,使文士蘇寶生為之誄焉。



 大明中,又有奚顯度者,南東海郯人也。官至員外散騎侍郎。世祖常使主領人功,而苛虐無道,動加捶撲,署雨寒雪,不聽暫休,人不堪命,或有自經死者。人役聞配顯度,如就刑戮。時建康縣考囚,或用方材壓額及踝脛,民間謠曰:「寧得建康壓額,不
 能受奚度拍。」又相戲曰:「勿反顧,付奚度。」其酷暴如此。前廢帝嘗戲云:「顯度刻虐,為百姓所疾,比當除之。」左右因倡「諾」。即日宣旨殺焉。時人比之孫皓殺岑昏。



 徐爰,字長玉,南琅邪開陽人也。本名瑗,後以與傅亮父同名,改為爰。初為晉琅邪王大司馬府中典軍,從北征。微密有意理,為高祖所知。少帝在東宮,入侍左右。太祖初,又見親任,歷治吏勞,遂至殿中侍御史。元嘉十二年,轉南臺侍御史,始興王濬後軍。復侍太子於東宮,遷員
 外散騎侍郎。太祖每出軍行師,常懸授兵略。二十九年,重遣王玄謨等北伐,配爰五百人,隨軍向確磝,銜中旨,臨時宣示。



 世祖至新亭,大將軍江夏王義恭南奔,爰時在殿內,誑劭追義恭,因得南走。



 時世祖將即大位,軍府造次,不曉朝章。爰素諳其事,既至,莫不喜說,以兼太常丞,撰立儀注。孝建初,補尚書水部郎,轉為殿中郎,兼右丞。孝建三年,索虜寇邊,詔問群臣防禦之策,爰議曰:詔旨「虜犯邊塞,水陸遼遠,孤城危棘,復不可置」。臣以戎虜
 猖狂,狡焉滋廣,列卒擬候,伺覘間隙,不勞大舉,終莫永寧。然連於千里,費固巨萬,而中興造創,資儲未積,是以齊斧徘徊,朔氣稽掃。今皇運洪休,靈威遐懾,蠢爾遺燼,懼在誅剪,思肆蜂蠆,以表有餘,雖不敢深入濟、沛,或能草竊邊塞。羽林鞭長,太倉遙阻,救援之日,勢不相及。且當使緣邊諸戍,練卒嚴城,凡諸督統,聚糧蓄田,籌計資力,足相抗擬。小鎮告警,大督電赴,塢壁邀斷,州郡犄角,儻有自送,可使匹馬不反。



 詔旨「胡騎倏忽,抄暴無漸,出
 耕見虜,野粒資寇,比及少年,軍實無擬,江東根本,不可俱竭,宜立何方,可以相贍?」臣以為方鎮所資,實宜且田且守,若使堅壁而春墾輟耕,清野而秋登莫擬,私無生業,公成虛罄,遠引根本,二三非宜。



 救之之術,唯在盡力防衛,來必拒戰,去則邀躡,據險保隘,易為首尾。胡馬既退,則民豐稟實,比及三載,可以長驅。



 詔旨「賊之所向,本無前謀,兵之所進,亦無定所。比歲戎戍,倉庫多虛,先事聚眾,則消費糧粟,敵至倉卒,又無以相應。」臣以為推鋒
 前討,大須資力,據本應末,不俟多眾。今寇無傾國豕突,列城勢足脣齒,養卒得勇,所任得才,臨事而懼,應機無失,豈煩空聚兵眾,以待未然。



 詔旨「戎狄貪婪,唯利是規,不挫凶圖,姦志歲結。」臣以為不擊則必侵掠,侵掠不已,則民失農桑;農桑不收,則王戍不立,為立之方,擊之為要。



 詔旨「若令邊地歲驚,公私失業,經費困於遙輸,遠圖決無遂事,寢弊贊略,逆應有方」。臣以為威虜之方,在於積粟塞下。若使邊民失業,列鎮寡儲,非唯無以遠圖,亦
 不能制其侵抄。今當使小戍制其始寇,大鎮赴其入境,一被毒手,便自吹齏鳥逝矣。



 尋即真,遷左丞。先是元嘉中,使著作郎何承天草創國史。世祖初,又使奉朝請山謙之、南臺御史蘇寶生踵成之。六年,又以爰領著作郎,使終其業。爰雖因前作,而專為一家之書。上表曰:臣聞虞史炳圖,原光被之美,夏載昭策,先隨山之勤。天飛雖王德所至,終陟固有資田躍,神宗始於俾乂,上日兆於納揆。其在《殷頌》,《長發》玄王,受命作周,實唯雍伯,考行之
 盛則,振古之弘軌。降逮二漢,亦同茲義,基帝創乎豐郊,紹祚本於昆邑。魏以武命《國志》,晉以宣啟《陽秋》,明黃初非更姓之本,泰始為造物之末,又近代之令準,式遠之鴻規。典謨緬邈,紀傳成準,善惡具書,成敗畢記。然餘分紫色,滔天泯夏,親所芟夷,而不序於始傳,涉、聖、卓、紹,煙起雲騰,非所誅滅,而顯冠乎首述,豈不以事先歸之前錄,功偕著之後撰。



 伏惟皇宋承金行之澆季,鐘經綸之屯極,擁玄光以鳳翔,秉神符而龍舉,喿刂定鯨鯢,天人佇
 屬。晉祿數終,上帝臨宋,便應奄膺珣宇,對越神工,而恭服勤於三分,讓德邁於不嗣,其為巍巍蕩蕩,赫赫明明,歷觀逖聞,莫或斯等。宜依銜書改文,登舟變號,起元義熙,為王業之始,載序宣力,為功臣之斷。其偽玄纂竊,同於新莽,雖靈武克殄,自詳之晉錄。及犯命干紀,受戮霸朝,雖揖禪之前,皆著之宋策。國典體大,方垂不朽,請外詳議,伏須遵承。



 於是內外博議,太宰江夏王義恭等三十五人同爰議,宜以義熙元年為斷。散騎常侍巴陵王
 休若、尚書金部郎檀道鸞二人謂宜以元興三年為始。太學博士虞和謂宜以開國為宋公元年。詔曰:「項籍、聖公,編錄二漢,前史已有成例。桓玄傳宜在宋典,餘如爰議。」



 七年,爰遷游擊將軍。其年,世祖南巡,權以本官兼尚書左丞,車駕還宮,罷。



 明年,又兼左丞,著作兼如故。世祖崩,營景寧陵,爰以本官兼將作大匠。爰便僻善事人,能得人主微旨,頗涉書傳,尤悉朝儀。元嘉初便入侍左右,預參顧問,既長於附會,又飾以典文,故為太祖所任遇。
 大明世,委寄尤重,朝廷大體儀注,非爰議不行。雖復當時碩學所解過人者,既不敢立異議,所言亦不見從。世祖崩,公除後,晉安王子勛侍讀博士咨爰宜習業與不?爰答:「居喪讀喪禮,習業何嫌。」



 少日,始安王子真博士又咨爰,爰曰:「小功廢業,三年喪何容讀書。」其專斷乖謬皆如此。



 前廢帝凶暴無道,殿省舊人,多見罪黜,唯爰巧於將迎,始終無迕。誅群公後,以爰為黃門侍郎,領射聲校尉,著作如故。封吳平縣子,食邑五百戶。寵待隆密,群臣
 莫二。帝每出行,常與沈慶之、山陰公主同輦,爰亦預焉。太宗即位,例削封,以黃門侍郎改領長水校尉,兼尚書左丞。明年,除太中大夫,著作並如故。



 爰秉權日久,上昔在籓,素所不說。及景和世,屈辱卑約,爰禮敬甚簡,益銜之。泰始三年,詔曰:夫事君無禮,教道弗容;訕上炫己,人倫所棄。太中大夫徐爰拔迹廝猥,推斥饕逢,遂官參時望,門伍豪族,遷位轉榮,莫非超荷。而諂側輕險,與性自俱,利口讒妄,自少及長,奉公在事,釐毫蔑聞,初無愧滿,
 常有窺進。先朝嘗以芻輩之中,粗有學解,故得漸蒙驅策,出入兩宮。太初偽立,盡心佞事,義師已震,方得南奔。及孝武居統,唯極諂諛,附會承旨,專恣厥性,致使治政苛縱,興造乖法,損德害民,皆由此豎。景和悖險,深相贊協,茍取偷存,罔顧節義,任算設數,取合人主,崎嶇姦矯,所志必從,故歷事七朝,白首全貴。自以體含德厚,識鑒機先,迷塗遂深,罔知革悟。



 朕撥亂反正,勳濟天下,靈祗助順,群逆必夷,況爰恩養,而無輸效,遂內挾異心,著於
 形跡,陽愚杜口,罔所陳聞,惰事緩文,庶申詭略。當今朝列賢彥,國無佞邪,而秉心弗純,累蠹時政。以其自告之辰,用賜歸老之職,榮禮優崇,寧非號饕過。不謂潛怨斥外,進競不已,勤言託意,觸遇斯發。小人之情,雖所先照,猶許其當改,未忍加法。遂恃朕仁弘,必永容貸。昨因觴宴,肆意譏毀,謂制詔所為,皆資傍說。又宰輔無斷,朝要非才,恃老與舊,慢戾斯甚。比邊難未靜,安眾以惠,戎略是務,政網從簡,故得使此小物,乘寬自縱。乃合投畀豺虎,
 以清王猷,但朽悴將盡,不足窮法,可特原罪,徙付交州。



 爰既行,又詔曰:「八議緩罪,舊在一條;五刑所抵,耆必加貸。徐爰前後釁跡,理無可申,廢棄海埵,實允國憲。但蚤蒙朕識,曲矜愚朽,既經大宥,思沾殊渥。可特除廣州統內郡。」有司奏以為宋隆太守。除命既下,爰已至交州,值刺史張牧病卒,土人李長仁為亂,悉誅北來流寓,無或免者。長仁素聞爰名,以智計誑誘,故得無患。久之聽還,仍除南康郡丞。太宗崩,還京都,以爰為南濟陰太守,復
 除中散大夫。元徽三年,卒,時年八十二。



 阮佃夫,會稽諸暨人也。元嘉中,出身為臺小史。太宗初出閣,選為主衣。世祖召還左右,補內監。永光中,太宗又請為世子師,甚見信待。景和末,太宗被拘於殿內,住在秘書省,為帝所疑,大禍將至,惶懼計無所出。佃夫與王道隆、李道兒及帝左右琅邪淳于文祖謀共廢立。時直閣將軍柳光世亦與帝左右蘭陵繆方盛、丹陽周登之有密謀,未知所奉。登之與太宗有舊,方盛等乃使登之
 結佃夫,佃夫大說。



 先是,帝立皇后,普暫徹諸王奄人,太宗左右錢藍生亦在其例。事畢,未被遣,密使藍生候帝,慮事洩,藍生不欲自出,帝動止輒以告淳于文祖,令文祖報佃夫。



 景和元年十一月二十九日晡時,帝出幸華林園,建安王休仁、山陽王休祐、山陰公主並侍側。太宗猶在秘書省,不被召,益憂懼。佃夫以告外監典事東陽朱幼,又告主衣吳興壽寂之、細鎧主南彭城姜產之,產之又語所領細鎧將臨淮王敬則,幼又告中書舍人戴
 明寶,並響應。明寶、幼欲取其日向曉,佃夫等勸取開鼓後。幼豫約勒內外,使錢藍生密報建安王休仁等。時帝欲南巡,腹心直閣將軍宋越等其夕並聽出外裝束,唯有隊主樊僧整防華林閣,是柳光世鄉人,光世要之,僧整即受命。



 姜產之又要隊副陽平聶慶及所領壯士會稽富靈符、吳郡俞道龍、丹陽宋逵之、陽平田嗣,並聚於慶省。佃夫慮力少不濟,更欲招合,壽寂之曰:「謀廣或洩,不煩多人。」



 時巫覡云:「後堂有鬼。」其夕,帝於竹林堂前,與
 巫共射之。建安王休仁等山陰主並從。帝素不說寂之,見輒切齒。寂之既與佃夫成謀,又慮禍至,抽刀前入;姜產之隨其後,淳于文祖、繆方盛、周登之、富靈符、聶慶、田嗣、王敬則、俞道龍、宋逵之又繼進。休仁聞行聲甚疾,謂休佑曰:「事作矣。」相隨奔景陽山。帝見寂之至,引弓射之,不中,乃走,寂之追而殞之。事定,宣令宿衛曰:「湘東王受太后令,除狂主。今已平定。」太宗即位,論功行賞,壽寂之封應城縣侯,食邑千戶;姜產之汝南縣侯,佃夫建城縣
 侯,食邑八百戶。王道隆吳平縣侯,淳于文祖陽城縣侯,食邑各五百戶。李道兒新塗縣侯,繆方盛劉陽縣侯,周登之曲陵縣侯,食邑各四百戶。富靈符惠懷縣子,聶慶建陽縣子,田嗣將樂縣子,王敬則重安縣子,俞道龍茶陵縣子,宋逵之零陵縣子,食邑各三百戶。



 佃夫遷南臺侍御史。薛索兒渡淮為寇,山陽太守程天祚又反,佃夫與諸軍討之,破索兒,降天祚。遷龍驤將軍、司徒參軍,率所領南助赭圻,轉太子步兵校尉、南魯郡太守,侍太子
 於東宮。太始四年,以破薛索兒功,增封二百戶,并前千戶;以本官兼游擊將軍,假寧朔將軍,與輔國將軍兼驍騎將軍孟次陽與二衛參員直。次陽字崇基,平昌安丘人也。泰始初,為山陽王休祐驃騎參軍。薛安都子道標攻合肥,次陽擊破之,以功封攸縣子,食邑三百戶。歷右軍、驃騎參軍;六年,出為輔師將軍、兗州刺史,戍淮陰。立北兗州,自此始也。進號冠軍將軍。元徽四年,卒。



 時佃夫、王道隆、楊運長並執權柄,亞於人主。巢、戴大明之世方
 之蔑如也。



 嘗值正旦應合朔,尚書奏遷元會,佃夫曰:「元正慶會,國之大禮,何不遷合朔日邪?」其不稽古如此。大通貨賄,凡事非重賂不行。人有餉絹二百匹,嫌少,不答書。宅舍園池,諸王邸第莫及。妓女數十,藝貌冠絕當時,金玉錦繡之飾,宮掖不逮也。每製一衣,造一物,京邑莫不法效焉。於宅內開瀆,東出十許里,塘岸整潔,汎輕舟,奏女樂。中書舍人劉休嘗詣之,值佃夫出行,中路相逢,要休同反;就席,便命施設,一時珍羞,莫不畢備。凡諸火
 劑,並皆始熟,如此者數十種。佃夫嘗作數十人饌,以待賓客,故造次便辦,類皆如此,雖晉世王、石,不能過也。泰始初,軍功既多,爵秩無序,佃夫僕從附隸,皆受不次之位。捉車人虎賁中郎,傍馬者員外郎。朝士貴賤,莫不自結,而矜傲無所降意,入其室者,唯吳興沈勃、吳郡張澹數人而已。



 泰豫元年,除寧朔將軍、淮南太守,遷驍騎將軍,尋加淮陵太守。太宗晏駕,後廢帝即位,佃夫權任轉重,兼中書通事舍人,加給事中、輔國將軍,餘如故。欲用
 張澹為武陵郡,衛將軍袁粲以下皆不同,而佃夫稱敕施行,粲等不敢執。元徽三年,遷黃門侍郎,領右衛將軍,太守如故。明年,改領驍騎將軍。其年,遷使持節、督南豫州諸軍事、冠軍將軍、南豫州刺史、歷陽太守,猶管內任。以平建平王景素功,增邑五百戶。



 時廢帝猖狂,好出遊走,始出宮,猶整羽儀,引隊仗;俄而棄部伍,單騎與數人相隨,或出郊野,或入市廛,內外莫不懼憂。佃夫密與直閣將軍申伯宗、步兵校尉朱幼、于天寶謀共廢帝,立安
 成王。五年春,帝欲往江乘射雉。帝每北出,常留隊仗在樂遊苑前,棄之而去。佃夫欲稱太后令喚隊仗還,閉城門,分人守石頭、東府,遣人執帝廢之,自為揚州刺史輔政。與幼等已成謀,會帝不成向江乘,故其事不行。於天寶因以其謀告帝,帝乃收佃夫、幼、伯宗於光祿外部,賜死。佃夫、幼罪止身,其餘無所問。佃夫時年五十一。



 幼,泰始初為外監,配張永諸軍征討,有濟辦之能,遂官涉三品,為奉朝請、南高平太守,封安浦縣侯,食邑二百戶。
 于天寶,其先胡人,預竹林堂功。元徽中,自陳功勞,求加封爵,乃封為鄂縣子,食邑二百戶。發佃夫之謀,以為清河太守,右軍將軍。升明元年,出為山陽太守。齊王以其反覆,賜死。



 壽寂之,泰始初,以軍功增邑二百戶。為羽林監,遷太子屯騎校尉,尋加寧朔將軍、南泰山太守。多納貨賄,請謁無窮,有一不從,切齒罵詈,常云:「利刀在手,何憂不辦。」鞭尉吏,斫邏將。七年,為有司所奏,徙送越州,行至豫章,謀
 欲逃叛,乃殺之。



 姜產之,泰始初,以軍功增邑二百戶。為晉平王休祐驃騎中兵參軍,龍驤將軍、南濟陰太守。三年北伐,與虜戰,軍敗見殺。追贈左軍將軍,太守如故。



 李道兒,臨淮人。本為湘東王師,稍至湘東國學官令。太宗即位,稍進至員外散騎侍郎,淮陵太守。泰始二年,兼中書通事舍人,轉給事中。四年,病卒。



 王道隆,吳興烏程人。兄道迄,涉學善書,形貌又美,吳興
 太守王韶之謂人曰:「有子弟如王道迄,無所少。」始興王浚以為世子師。以書補中書令史。道隆亦知書,為主書書吏,漸至主書。世祖使傳命,失旨,遣出,不聽復入六門。太宗鎮彭城,以補典簽,署內監。及即位,為南臺侍御史,稍至員外散騎侍郎,南蘭陵太守。



 泰始二年,兼中書通事舍人。以破晉陵功,增邑百戶,並前六百戶。五年,出侍東宮,復兼中書通事舍人。後廢帝即位,自太子翊軍校尉遷右軍將軍,太守、兼舍人如故。道隆為太宗所委,過
 於佃夫,和謹自保,不妄毀傷人。執權既久,家產豐積,豪麗雖不及佃夫,而精整過之。



 元徽二年,太尉桂陽王休範奄至新亭,佃夫留守殿內,而道隆領羽林精兵向朱雀門。時賊已至航南,道隆忽召鎮軍將軍劉勔於石頭,勔至,命開航,道隆怒曰:「賊至但當急擊,寧可開航自弱邪!」勔不敢復言。催勔進戰,勔度航便敗,賊乘勝徑進,道隆棄眾走向臺,所乘馬連聳跼不肯前,遂為賊兵及,見殺。事平,車駕臨哭,贈輔國將軍、益州刺史。子法貞嗣。齊
 受禪,國除。



 楊運長,宣城懷安人。初為宣城郡吏,太守范曄解吏名。素善射,太宗初為皇子,出運長為射師。性謹愨,為太宗所委信。及即位,親遇甚厚,與佃夫、道隆、李道兒等並執權要,稍至員外散騎侍郎,南平昌太守。泰始七年,出侍東宮。後廢帝即位,與佃夫俱兼通事舍人,加龍驤將軍,轉給事中。以平桂陽王休範功,封南城縣子,食邑八百戶。元徽三年,自安成王車騎中兵參軍,遷後軍將軍,兼
 舍人如故。



 運長質木廉正,治身甚清,不事園宅,不受餉遺,而凡鄙無識知,唯與寒人潘智、徐文盛厚善,動止施為,必與二人量議。文盛為奉朝請,預平桂陽王休範,封廣晉縣男,食邑四百戶。順帝即位,出運長為寧朔將軍、宣城太守,尋去郡還家。



 沈攸之反,運長有異志,齊王遣驃騎司馬崔文仲討誅之。



 史臣曰:竭忠盡節,仕子恒圖;隨方致用,明君盛典。舊非本舊,因新以成舊者也;狎非先狎,因疏以成狎者也。而
 任隔疏情,殊塗一致,權歸近狎,異世同規。



 雖復漢高之簡易,光武之謹厚,猶豐、沛多顯,白水先華,況世祖之泥滯鄙近,太宗之拘攣愛習,欲不紛惑床笫,豈可得哉!



\end{pinyinscope}