\article{卷四十志第三十 百官下}

\begin{pinyinscope}

 給事黃門侍郎,四人,與侍中俱掌門下眾事。郊廟臨軒,則一人執麾。《漢百官表》秦曰給事黃門,無員,掌侍從左右,漢因之。漢東京曰給事黃門侍郎,亦無員,掌侍從左右,關
 通中外,諸王朝見,則引王就坐。應劭曰:「每日莫向青瑣門拜,謂之夕郎。」史臣按,劉向與子歆書曰:「黃門郎,顯處也。」然則前漢世已為黃門侍郎矣。董巴《漢書》曰:「禁門曰黃闥,中人主之,故號曰黃門令。」然則黃門郎給事黃闥之內,故曰黃門郎也。魏、晉以來員四人,秩六百石。



 公車令,一人。掌受章奏。秦有公車司馬令,屬衛尉,漢因之,掌宮南闕門。



 凡吏民上章,四方貢獻,及徵詣公車者,皆掌之。晉江左以來,直云公車令。



 太醫令,一人。丞一人。《周官》為醫師,秦為太醫令,至二漢屬少府。太官令,一人。丞一人。《周官》為膳夫,秦為太官令,至漢屬少府。



 驊騮廄丞,一人。漢西京為龍馬長,漢東京為未央廄令,魏為驊騮令。自公車令至此,隸侍中。



 散騎常侍,四人。掌侍左右。秦置散騎,又置中常侍,散騎並乘輿車後;中常侍得入禁中。皆無員,並為加官。漢東京初省散騎,而中常侍因用宦者。魏文帝黃初初,置散
 騎,合於中常侍,謂之散騎常侍,始以孟達補之。久次者為祭酒散騎常侍,秩比二千石。



 通直散騎常侍,四人。魏末散騎常侍又有在員外者,晉武帝使二人與散騎常侍通直,故謂之通直散騎常侍。晉江左置五人。員外散騎常侍,魏末置,無員。



 散騎侍郎,四人。魏初與散騎常侍同置。魏、晉散騎常侍、侍郎,與侍中、黃門侍郎共平尚書奏事,江左乃罷。通直散騎侍郎,四人。初晉武帝置員外散騎侍郎四人,元帝
 使二人與散騎侍郎通直,故謂之通直散騎侍郎,後增為四人。員外散騎侍郎,晉武帝置,無員。



 給事中,無員。漢西京置。掌顧問應對,位次中常侍。漢東京省,魏世復置。



 奉朝請,無員,亦不為官。漢東京罷省三公、外戚、宗室、諸侯,多奉朝請。



 奉朝請者,奉朝會請召而已。晉武帝亦以宗室外戚為奉車、駙馬、騎都尉,而奉朝請焉。元帝為晉王,以參軍為奉車都尉,掾、屬為駙馬都尉,行參軍、舍人
 為騎都尉,皆奉朝請。後省奉車、騎都尉,唯留駙馬都尉、奉朝請。永初已來,以奉朝請選雜,其尚主者唯拜駙馬都尉。三都尉並漢武帝置。孝建初,奉朝請省。駙馬都尉、三都尉秩比二千石。



 中書令,一人。中書監人,一人。中書侍郎,四人。中書通事舍人,四人。漢武帝游後廷,始使宦者典尚書
 事,謂之中書謁者,置令、僕射。元帝時,令弘恭,僕射石顯,秉勢用事,權傾內外。成帝改中書謁者令曰中謁者令,罷僕射。漢東京省中謁者令,而有中宮謁者令,非其職也。魏武帝為王,置祕書令,典尚書奏事,又其任也。文帝黃初初,改為中書令,又置監,及通事郎,次黃門郎。黃門郎已署事過,通事乃奉以入,為帝省讀書可。晉改曰中書侍郎,員四人。晉江左初,改中書侍郎曰通事郎,尋復為中書侍郎。晉初置舍人一人,通事一人。江左初,合舍
 人通事謂之通事舍人,掌呈奏案章。後省通事,中書差侍郎一人直西省,又掌詔命。



 宋初又置通事舍人,而侍郎之任輕矣。舍人直閣內,隸中書。其下有主事,本用武官,宋改用文吏。



 秘書監,一人。祕書丞,一人。秘書郎,四人。漢桓帝延熹二年,置祕書監。



 皇甫規與張奐書云「從兄祕書它何動靜」是也。應劭《漢官》曰:「祕書監
 一人,六百石。」後省。魏武帝為魏王,置秘書令、祕書丞。祕書典尚書奏事。文帝黃初初,置中書令,典尚書奏事,而秘書改令為監。後欲以何楨為秘書丞,而祕書先自有丞,乃以楨為祕書右丞。後省。掌藝文圖籍。《周官》外史掌四方之志、三皇五帝之書,即其任也。漢西京圖籍所藏,有天府、石渠、蘭臺、石室、延閣、廣內之府是也。東京圖書在東觀。晉武帝以祕書并中書,省監,謂丞為中書祕書丞。惠帝復置著作郎一人,佐郎八人,掌國史。周世左史
 記事,右史記言,即其任也。漢東京圖籍在東觀,故使名儒碩學,著作東觀,撰述國史。著作之名,自此始也。魏世隸中書。晉武世,繆徵為中書著作郎。元康中,改隸祕書,後別自為省,而猶隸祕書。著作郎謂之大著作,專掌史任。晉制,著作佐郎始到職,必撰名臣傳一人。宋氏初,國朝始建,未有合撰者,此制遂替矣。



 領軍將軍,一人。掌內軍。漢有南北軍,衛京師。武帝置中壘校尉,掌北軍營。



 光武省中壘校尉,置北軍中候,監五
 校營。魏武為丞相,相府自置領軍,非漢官也。



 文帝即魏王位,魏始置領軍,主五校、中壘、武衛三營。晉武帝初省,使中軍將軍羊祜統二衛前後左右驍騎七軍營兵,即領軍之任也。祜遷罷,復置北軍中候。北軍中候置丞一人。懷帝永嘉中,改曰中領軍。元帝永昌元年,復改曰北軍中候;尋復為領軍。成帝世,復以為中候,而陶回居之;尋復為領軍。領軍今猶有南軍都督。



 護軍將軍,一人。掌外軍。秦時護軍都尉,漢因之。陳平為
 護軍中尉,盡護諸將。然則復以都尉為中尉矣。武帝元狩四年,以護軍都尉屬大司馬,于時復為都尉矣。《漢書·李廣傳》,廣為驍騎將軍,屬護軍將軍。蓋護軍護諸將軍。哀帝元壽元年,更名護軍都尉曰司寇。平帝元始元年,更名護軍都尉。東京省,班固為大將軍中護軍,隸將軍莫府,非漢朝列職。魏武為相,以韓浩為護軍,史奐為領軍,非漢官也。建安十二年,改護軍為中護軍,領軍為中領軍,置長史、司馬。魏初因置護軍,主武官選,隸領軍,晉世
 則不隸也。晉元帝永昌元年,省護軍并領軍。明帝太寧二年,復置。魏、晉江右領、護各領營兵;江左以來,領軍不復別置營,總統二衛驍騎材官諸營,護軍猶別有營也。領、護資重者為領軍、護軍將軍,資輕者為中領軍、中護軍。官屬有長史、司馬、功曹、主簿、五官。受命出征,則置參軍。



 左衛將軍,一人。右衛將軍,一人。二衛將軍掌宿衛營兵。二漢、魏不置。晉文帝為相國,相國府置中衛將軍。武帝初,分中衛置左
 右衛將軍,以羊琇為左衛,趙序為右衛。二衛江右有長史、司馬、功曹、主簿,江左無長史。



 驍騎將軍,漢武帝元光六年,李廣為驍騎將軍。魏世置為內軍,有營兵,高功者主之。先有司馬、功曹、主簿,後省。



 遊擊將軍,漢武帝時,韓說為遊擊。是為六軍。



 左軍將軍、右軍將軍、前軍將軍、
 後軍將軍。魏明帝時,有左軍將軍,然則左軍魏官也。晉武帝初,置前軍、右軍;泰始八年,又置後軍。是為四軍。



 左中郎將、右中郎將,秦官,漢因之。與五官中郎將領三署郎,魏無三署郎,猶置其職。晉武帝省。宋世祖大明中,又置。



 屯騎校尉、步兵校尉、越騎校尉、
 長水校尉、射聲校尉。五校並漢武帝置。屯騎、步兵掌上林苑門屯兵;越騎掌越人來降,因以為騎也;一說取其材力超越也。



 長水掌長水宣曲胡騎。長水,胡部落名也。胡騎屯宣曲觀下。韋曜曰:「長水校尉,典胡騎,廄近長水,故以為名。長水,蓋關中小水名也。」射聲掌射聲士,聞聲則射之,故以為名。漢光武初,改屯騎為驍騎,越騎為青巾。建武十五年,復舊。漢東京五校,典宿衛士。自遊擊至五校,魏、晉
 逮于江左,初猶領營兵,並置司馬、功曹、主簿,後省。二中郎將本不領營也。五營校尉,秩二千石。



 虎賁中郎將,《周官》有虎賁氏。漢武帝建元三年,始微行出遊,選材力之士執兵從送,期之諸門,故名期門。無員,多至千人。平帝元始元年,更名曰虎賁郎,置中郎將領之。虎賁舊作虎奔,言如虎之奔走也。王莽輔政,以古有勇士孟賁,故以奔為賁。比二千石。



 冗從僕射,漢東京有中黃門冗從僕射,非其職也。魏世
 因其名而置冗從僕射。



 羽林監,漢武帝太初元年,初置建章營騎,亦掌從送次期門,後更名羽林騎,置令、丞。宣帝令中郎將騎都尉監羽林,謂之羽林中郎將。漢東京又置羽林左監、羽林右監,至魏世不改。晉罷羽林中郎將,又省一監,置一監而已。自虎賁至羽林,是為三將。哀帝省。宋高祖永初初,復置。江右領營兵,江左無復營兵。羽林監六百石。



 積射將軍、
 彊弩將軍。漢武帝以路博德為彊弩校尉,李沮為彊弩將軍。宣帝以許延壽為彊弩將軍。彊弩將軍至東漢為雜號,前漢至魏無積射。晉太康十年,立射營、弩營,置積射、彊弩將軍主之。自驍騎至彊弩將軍,先並各置一人;宋太宗泰始以來,多以軍功得此官,今並無復員。



 殿中將軍、殿中司馬督。晉武帝時,殿內宿衛,號曰三部司馬,置此二官,分隸左右二衛。江右初,員十人。朝會宴饗,則將軍
 戎服,直侍左右,夜開城諸門,則執白虎幡監之。晉孝武太元中,改選,以門閥居之。宋高祖永初初,增為二十人。



 其後過員者,謂之殿中員外將軍、員外司馬督。其後並無復員。



 武衛將軍,無員。初,魏王始置武衛中郎將,文帝踐阼,改為衛將軍,主禁旅,如今二衛,非其任也。晉氏不常置。宋世祖大明中,復置,代殿中將軍之任,比員外散騎侍郎。



 武騎常侍,無員。漢西京官。車駕游獵,常從射猛獸。後漢、
 魏、晉不置。宋世祖大明中,復置。比奉朝請。



 御史丞,一人。掌奏劾不法。秦時御史大夫有二丞,其一曰御史丞,其二曰御史中丞。殿中蘭臺秘書圖籍在焉,而中丞居之。外督部刺史,內領侍御史,受公卿奏事,舉劾按章。時中丞亦受奏事,然則分有所掌也。成帝綏和元年,更名御史大夫為大司空,置長史,而中丞官職如故。哀帝建平二年,復為御史大夫。元壽二年,復為大司空。而中丞出外為御史臺主,名御史長史。光武還曰中丞,
 又屬少府。獻帝時,更置御史大夫,自置長史一人,不復領中丞也。漢東京御史中丞遇尚書丞郎,則中丞止車執版揖,而丞郎坐車舉手禮之而已。不知此制何時省。中丞每月二十五日,繞行宮垣白壁。史臣按《漢志》執金吾每月三繞行宮城,疑是省金吾,以此事併中丞。中丞秩千石。



 治書侍御史,掌舉劾官品第六已上。漢宣帝齋居決事,令御史二人治書,因謂之治書御史。漢東京使明法律
 者為之,天下讞疑事,則以法律當其是非。魏、晉以來,則分掌侍御史所掌諸曹,若尚書二丞也。



 侍御史,於周為柱下史。《周官》有御史,掌治令,亦其任也。秦置侍御史,漢因之。二漢員並十五人。掌察舉非法,受公卿奏事,有違失者舉劾之。凡有五曹,一曰令曹,掌律令;二曰印曹,掌刻印,三曰供曹,掌齋祠;四曰尉馬曹,掌官廄馬;五曰乘曹,掌護駕。魏置御史八人,有治書曹,掌度支運,課第曹,掌考課,不知其餘曹也。晉西朝凡有吏
 曹、課第曹、直事曹、印曹、中都督曹、外都督曹、媒曹、符節曹、水曹、中旂曹、營軍曹、算曹、法曹,凡十三曹,而置御史九人。



 晉江左初,省課第曹,置庫曹,掌廄牧牛馬市租。後復分庫曹,置外左庫、內左庫二曹。宋太祖元嘉中,省外左庫,而內左庫直云左庫。世祖大明中,復置。廢帝景和元年又省。順帝初,省營軍併水曹,省算曹併法曹,吏曹不置御史,凡十御史焉。



 魏又有殿中侍御史二人,蓋是蘭臺遣二御史居殿內察非法也。晉西朝四人,江左二
 人。秦、漢有符節令,隸少府,領符璽郎、符節令史。蓋《周禮》典瑞、掌節之任也。漢至魏別為一臺,位次御史中丞,掌授節、銅虎符、竹使符。晉武帝泰始九年,省併蘭臺,置符節御史掌其事焉。



 謁者僕射,一人。掌大拜授及百官班次。領謁者十人。謁者掌小拜授及報章。



 蓋秦官也。謁,請也。應氏《漢官》曰,堯以試舜,賓于四門,是其職也。秦世謁者七十人,漢因之。後漢《百官志》,謁者僕射掌奉引。和帝世,陳郡何熙為謁
 者僕射,贊拜殿中,音動左右。然則又掌唱贊。有常侍謁者五人,謁者則置三十五人,半減西京也。二漢並隸光祿勳。魏世置謁者十人。晉武帝省僕射,以謁者隸蘭臺。



 江左復置僕射,後又省。宋世祖大明中,復置。秩比千石。



 都水使者,一人。掌舟航及運部。秦、漢有都水長、丞,主陂池灌溉,保守河渠,屬太常。漢東京省都水,置河隄謁者,魏因之。漢世水衡都尉主上林苑,魏世主天下水軍舟船器械。晉武帝省水衡,置都水使者,而河堤為都水官
 屬。有參軍二人,謁者一人,令史減置無常員。晉西朝有參軍而無謁者,謁者則江左置也。懷帝永嘉六年,胡入洛陽,都水使者爰浚先出督運得免。然則武帝置職,便掌運矣。江左省河隄。



 太子太傅,一人。丞一人。太子少傅,一人。丞一人。傅,古官也。《文王世子》曰:「凡三王教世子,太傅在前,少傅在後,並以輔導為職。」漢高帝九年,以叔孫通為太子太傅,位次太常。二漢並無丞。魏世
 無東宮,然則晉氏置丞也。晉武帝泰始五年,詔太子拜太傅、少傅,如弟子事師之禮;二傅不得上疏曲敬。二傅並有功曹、主簿、五官。太傅中二千石,少傅二千石。



 太子詹事,一人。丞一人。職比臺尚書令、領軍將軍。詹,省也。漢西京則太子門大夫、庶子、洗馬、舍人屬二傅,率更令、家令、僕、衛率屬詹事。皆秦官也。



 後漢省詹事,太子官屬悉屬少傅,而太傅不復領官屬。晉初,太子官屬通屬二傅。



 咸寧元年,復置詹事,二傅不復領官屬。詹事,二千
 石。



 家令,一人。丞一人。晉世置。漢世太子食湯沐邑十縣,家令主之。又主刑獄飲食,職比廷尉、司農、少府。漢東京主食官令。食官令,晉世自為官,不復屬家令。



 率更令,一人。主宮殿門戶及賞罰事,職如光祿勳、衛尉。漢東京掌庶子、舍人,晉世則不也。自漢至晉,家令在率更下;宋則居上。



 僕,一人。漢世太子五日一朝,非入朝日,遣僕及中允旦
 入請問起居,主車馬、親族,職如太僕、宗正。自家令至僕,為太子三卿。三卿,秩千石。



 門大夫,二人。漢東京置,職如中郎將,分掌遠近表箋。秩六百石。



 中庶子,四人。職如侍中。漢東京員五人,晉減為四人。秩六百石。



 中舍人,四人。漢東京太子官屬有中允之職,在中庶子下,洗馬上,疑若今中書舍人矣。中舍人,晉初置,職如黃
 門侍郎。



 食官令,一人。職如太官令。漢東京官也。今屬中庶子。



 庶子,四人。職比散騎常侍、中書監令。晉制也。漢西京員五人,漢東京無員,職如三署中郎。古者諸侯世子,有庶子之官,秦因其名也。秩四百石。



 舍人,十六人。職如散騎、中書侍郎。晉制也。二漢無員,掌宿衛如三署中郎。



 洗馬,八人。職如謁者、秘書郎也。二漢員十六人。太子出,
 則當直者前驅導威儀。秩比六百石。



 太子左衛率,七人。太子右衛率,二人。二率職如二衛。秦時直云衛率,漢因之。主門衛。晉初曰中衛率,泰始分為左右,各領一軍。惠帝時,愍懷太子在東宮,加置前後二率。成都王穎為太弟,又置中衛,是為五率。江左初,省前後二率。孝武太元中又置。皆有丞,晉初置。宋世止置左右二率。秩舊四百石。



 太子屯騎校尉。太子步兵校尉。太子翊軍校尉。三校尉各七人,並宋初置。屯騎、步兵,因臺校尉;翊軍,晉武帝太康初置,始為臺校尉,而以唐彬居之,江左省。



 太子冗從僕射,七人。宋初置。



 太子旅賁中郎將,十人。職如虎賁中郎將。宋初置。《周官》有旅賁氏。漢制,天子有虎賁,王侯有旅賁。旅,眾也。



 太子左積弩將軍,十人。太子右積弩將軍,二人。漢東京積弩將軍,雜號也,無左右之積弩。魏世至晉江左,左右積弩為臺職,領營兵。宋世度東宮,無復營矣。



 殿中將軍,十人。殿中員外將軍,二十人。宋初置。



 平越中郎將,晉武帝置,治廣州,主南越。



 南蠻校尉,晉武帝置,治襄陽。江左初省。尋又置,治江陵。宋世祖孝建中省。



 西戎校尉,晉初置,治長安。安帝義熙中又置,治漢中。



 寧蠻校尉,晉武帝置,治襄陽,以授魯宗之。



 南夷校尉,晉武帝置,治寧州。江左改曰鎮蠻校尉。四夷中郎校尉,皆有長史、司馬、參軍。魏、晉有雜號護軍,如將軍,今猶有鎮蠻、安遠等護軍。鎮蠻以加廬江、晉熙、西陽太守。安遠以加武陵內史。



 刺史,每州各一人。黃帝立四監以治萬國,唐、虞世十二牧,是其職也。周改曰典,秦曰監御史,而更遣丞相史分
 刺諸州,謂之刺史。刺之為言,猶參覘也。寫書亦謂之刺。漢制,不得刺尚書事是也。刺史班行六條詔書,其一條曰,彊宗豪右,田宅踰制,以彊陵弱,以眾暴寡;其二條曰,二千石不奉詔書,遵承典制,背公向私,旁詔守利,侵漁百姓,聚斂為姦;其三條曰,二千石不恤疑獄,風厲殺人,怒則加罰,喜則任賞,煩擾苛暴,剝戮黎元,為百姓所疾,山崩石裂,妖詳訛言;其四條曰,二千石選署不平,茍阿所愛,蔽賢寵頑;其五條曰,二千石子弟恃怙榮勢,請託
 所監;其六條曰,二千石違公下比,阿附豪彊,通行貨賂,割損正令。歲終則乘傳詣京師奏事。成帝綏和元年,改為牧。哀帝建平二年,復為刺史。前漢世,刺史乘傳周行郡國,無適所治。後漢世,所治始有定處,止八月行部,不復奏事京師。



 晉江左猶行郡縣詔,棗據《追遠詩》曰:「先君為鉅鹿太守,迄今三紀。忝私為冀州刺史,班詔次于郡傳」是也。靈帝世,天下漸亂,豪桀各據有州郡,而劉焉、劉虞並自九卿出為益州、幽州牧,其任漸重矣。官屬有別
 駕從事史一人,從刺史行部;治中從事史一人,主財穀簿書;兵曹從事史一人,主兵事;部從事史每郡各一人,主察非法;主簿一人,錄閣下眾事,省署文書;門亭長一人,主州正門;功曹書佐一人,主選用;《孝經》師一人,主試經;月令師一人,主時節祠祀;律令師一人,平律;簿曹書佐一人,主簿書;典郡書佐每郡各一人,主一郡文書:漢制也。今有別駕從事史、治中從事史、主簿、西曹書佐、祭酒從事史、議曹從事史、部郡從事史,自主簿以下,置人
 多少,各隨州,舊無定制也。晉成帝咸康中,江州又有別駕祭酒,居僚職之上,而別駕從事史如故,今則無也。別駕、西曹主吏及選舉事,治中主眾曹文書事。西曹,即漢之功曹書佐也。祭酒分掌諸曹兵、賊、倉、戶、水、鎧之屬。揚州無祭酒,而主簿治事。荊州有從事史,在議曹從事史下,大較應是魏、晉以來置也。今廣州、徐州有月令從事,若諸州之曹史,漢舊名也。漢武元封四年,令諸州歲各舉秀才一人。後漢避光武諱,改茂才。魏復曰秀才。晉江
 左揚州歲舉二人,諸州舉一人,或三歲一人,隨州大小,並對策問。晉東海王越為豫州牧,牧置長史、參軍,庾凱為長史,謝鯤為參軍,此為牧者則無也。牧,二千石;刺史,六百石。



 郡守,秦官。秦滅諸侯,隨以其地為郡,置守、丞、尉各一人。守治民,丞佐之。郡當邊戍者,丞為長史。晉江左皆謂之丞。尉典兵,備盜賊。漢景帝中二年,更名守曰太守,尉為都尉。光武省都尉,後又往往置東部、西部都尉。有蠻夷
 者,又有屬國都尉。漢末及三國,多以諸部都尉為郡。晉成帝咸康七年,又省諸郡丞。



 宋太祖元嘉四年,復置。郡官屬略如公府,無東西曹,有功曹史,主選舉,五官掾,主諸曹事,部縣有都郵、門亭長,又有主記史,催督期會,漢制也,今略如之。諸郡各有舊俗,諸曹名號,往往不同。漢武帝納董仲舒之言,元光元年,始令郡國舉孝廉,制郡口二十萬以上,歲察一人;四十萬以上,二人;六十萬,三人;八十萬,四人;百萬,五人;百二十萬,六人;不滿二十萬,
 二歲一人;不滿十萬,三歲一人。限以四科,一曰德行高妙,志節清白;二曰學通行修,經中博士;三曰明習法令,足以決疑,能案章覆問,文中御史;四曰剛毅多略,遭事不惑,明足決斷,材任三輔縣令。魏初,更制口十萬以上,歲一人,有秀異,不拘戶口。江左以丹陽、吳、會稽、吳興並大郡,歲各舉二人。漢制,歲遣上計掾史各一人,條上郡內眾事,謂之階簿,至今行之。太守,二千石;丞,六百石。



 縣令、長,秦官也。大者為令,小者為長,侯國為相。漢制,置
 丞一人,尉大縣二人,小縣一人。五家為伍,伍長主之;二五為什,什長主之;十什為里,里魁主之;十里為亭,亭長主之;十亭為鄉,鄉有鄉佐、三老、有秩、嗇夫、游徼各一人。鄉佐、有秩主賦稅,三老主教化,嗇夫主爭訟,游徼主姦非。其餘諸曹,略同郡職。以五官為廷掾,後則無復丞,唯建康有獄丞,其餘眾職,或此縣有而彼縣無,各有舊俗,無定制也。晉江右洛陽縣置六部都尉,餘大縣置二人,次縣、小縣各一人。宋太祖元嘉十五年,縣小者又省之。
 諸官府至郡,各置五百者,舊說古君行師從,卿行旅從。旅,五百人也。今縣令以上,古之諸侯,故立四五百以象師從旅從,依古義也。韋曜曰,五百字本為伍伯。伍,當也;伯,道也。使之導引當道伯中以驅除也。周制,五百為旅,帥皆大夫,不得卑之如此說也。又《周禮》秋官有條狼氏,掌執鞭以趨辟,王出入則八人夾道,公則六人,侯伯則四人,子男則二人,近之矣,名之異爾。又《漢官》中有伯使,主為諸官驅使辟路於道伯中,故言伯使,此其比也。縣
 令,千石至六百石;長,五百石。



 漢初,王國置太傅,掌輔導;內史主治民;丞相統眾官;中尉掌武職。分官置職,略同京師。至景帝懲七國之亂,更制諸王不得治國,漢為置吏,改丞相曰相,省御史大夫、廷尉、少府、宗正、博士官,其大夫、謁者、諸官長丞,皆損其員數。



 後改漢內史為京兆尹,中尉為執金吾,郎中令為光祿勳,而王國如故;又太僕為僕,司農為大農。成帝更令相治民如郡太守,省內史。其中尉如郡尉,太傅但曰
 傅。漢東京亦置傅一人,王師事之;相一人,主治民;中尉一人,主盜賊;郎中令一人,掌郎中宿衛;僕一人,治書一人,治書本曰尚書,後更名治書;中大夫,無員,掌奉使京師及諸國;謁者及禮樂、衛士、醫工、永巷、祀禮長各一人;郎中,無員。



 魏氏謁者官屬,史闕不知次第。晉武帝初置師、友、文學各一人。師即傅也,景帝諱師,改為傅。宋世復改曰師。其文學,前漢已置也。友者,因文王、仲尼四友之名也。改太守為內史,省相及僕。有郎中令、中尉、大農為三
 卿。大國置左右常侍各三人,省郎中,置侍郎二人。大國又置上軍、中軍、下軍三將軍;次國上軍將軍、下軍將軍各一人;小國上軍而已。典書、典祠、典衛、學官令、典書令丞各一人,治書四人,中尉、司馬、世子庶子陵廟、牧長各一人,謁者四人,中大夫六人,舍人十人,典醫丞、典府丞各一人。宋氏以來,一用晉制,雖大小國,皆有三軍。晉制,典書令在常侍下,侍郎上;江左則侍郎次常侍,而典書令居三軍下矣。江左以來,公國則無中尉、常侍、三軍,侯
 國又無大農、侍郎,伯子男唯典書以下,又無學官令矣。吏職皆以次損省焉。晉江右公侯以下置官屬,隨國小大,無定制也。晉江左諸國,並三分食一。元帝太興元年,始制九分食一太傅,太保,太宰,太尉,司徒,司空,大司馬,大將軍,諸位從公。
 右第一品。



 特進,驃騎,車騎,衛將軍,諸大將軍,諸持節都督。右第二品。



 侍中,散騎常侍,尚書令,僕射,尚書,
 中書監,令,祕書監,諸征、鎮至龍驤將軍,光祿大夫,諸卿,尹,太子二傅,大長秋,太子詹事,領、護軍,
 縣侯。



 右第三品。



 二衛至五校尉,寧朔至五威、五武將軍,四中郎將,刺史領兵者,戎蠻校尉,御史中丞,都水使者,
 鄉侯。右第四品。



 給事中,黃門、散騎、中書侍郎,謁者僕射,三將,積射、彊弩將軍,太子中庶子,庶子,三卿,率,鷹揚至陵江將軍,刺史不領兵者,
 郡國太守,內史,相,亭侯。右第五品。



 尚書丞,郎,治書侍御史,侍御史,三都尉,博士,撫軍以上及持節都督領護長史,司馬,
 公府從事中郎將,廷尉正,監,評,祕書著作丞,郎,王國公三卿,師,友,文學,諸縣署令千石者,太子門大夫,殿中將軍,司馬督,雜號護軍,
 闕內侯。右第六品。



 謁者,殿中監,諸卿尹丞,太子傅詹事率丞,諸軍長史、司馬六百石者,諸府參軍,
 戎蠻府長史,司馬,公府掾,屬,太子洗馬,舍人,食官令,諸縣令六百石者。右第七品。



 內臺正令史,郡丞,諸縣署長,
 雜號宣威將軍以下。右第八品。



 內臺書令史,外臺正令史,諸縣署丞,尉。右第九品。凡新置不見此諸條者,隨秩位所視,蓋囗囗右所定也。



\end{pinyinscope}