\article{卷五十四列傳第十四 孔季恭 羊玄保 沈曇慶}

\begin{pinyinscope}

 孔靖,字季恭,會稽山陰人也。名與高祖祖諱同,故稱字。祖愉,晉車騎將軍。



 父誾,散騎常侍。季恭始察郡孝廉,功曹史,著作佐郎,太子舍人,鎮軍司馬,司徒左西掾。未拜,
 遭母憂。隆安五年,於喪中被起建威將軍、山陰令,不就。高祖東征孫恩,屢至會稽,季恭曲意禮接,贍給甚厚。高祖後討孫恩,時桓玄篡形已著,欲於山陰建義討之。季恭以為山陰去京邑路遠,且玄未居極位,不如待其篡逆事彰,釁成惡稔,徐於京口圖之,不憂不克。高祖亦謂為然。虞嘯父為征東將軍、會稽內史,季恭初求為府司馬,不得。及帝定桓玄,以季恭為內史,使齎封板拜授,季恭相值,季恭便舟夜還。至即叩扉告嘯父,並令掃拂別
 齋,即便入郡。嘯父本為桓玄所授,聞玄敗,震懼,開門請罪。季恭慰勉,使且安所住,明旦乃移。季恭到任,務存治實,敕止浮華,翦罰游惰,由是寇盜衰止,境內肅清。



 徵為右衛將軍,加給事中,不拜。尋除侍中,領本國中正,徙瑯邪王大司馬司馬。尋出為吳興太守,加冠軍。先是,吳興頻喪太守,云項羽神為卞山王,居郡聽事,二千石至,常避之;季恭居聽事,竟無害也。遷尚書右僕射,固讓。義熙八年,復督五郡諸軍、征虜、會稽內史。修飾學校,計課調
 習。十年,復為尚書右僕射,加散騎常侍,又讓不拜。頃之,除領軍將軍,加散騎常侍,本州大中正。十二年,致仕,拜金紫光祿大夫,常侍如故。是歲,高祖北伐,季恭求從,以為太尉軍諮祭酒、後將軍。從平關、洛。高祖為相國,又隨府遷。



 宋臺初建,令書以為尚書令,加散騎常侍,又讓不受,乃拜侍中、特進、左光祿大夫。辭事東歸,高祖餞之戲馬臺,百僚咸賦詩以述其美。及受命,加開府儀同三司,辭讓累年,終以不受。永初三年,薨,時年七十六。追贈侍
 中、左光祿大夫、開府儀同三司。



 子山士,歷顯位,侍中,會稽太守,坐小弟駕部郎道穰逼略良家子女,白衣領郡。元嘉二十七年,卒官。



 弟靈符,元嘉末,為南譙王義宣司空長史、南郡太守,尚書吏部郎。世祖大明初,自侍中為輔國將軍、郢州刺史,入為丹陽尹。山陰縣土境褊狹,民多田少,靈符表徙無貲之家於餘姚、鄞、鄮三縣界,墾起湖田。上使公卿博議,太宰江夏王義恭議曰:「夫訓農修本,有國所同,土著之民,習玩日久,如京師無田,不聞徙居
 他縣。尋山陰豪族富室,頃畝不少,貧者肆力,非為無處,耕起空荒,無救災歉。



 又緣湖居民,魚鴨為業,及有居肆,理無樂徙。」尚書令柳元景、右僕射劉秀之、尚書王瓚之、顧凱之、顏師伯、嗣湘東王彧議曰:「富戶溫房,無假遷業;窮身寒室,必應徙居。葺宇疏皋,產粒無待,資公則公未易充,課私則私卒難具。生計既完,畬功自息,宜募亡叛通恤及與樂田者,其往經創,須粗修立,然後徙居。」侍中沈懷文、王景文、黃門侍郎劉敳、卻顒議曰:「百姓雖不親
 農,不無資生之路,若驅以就田,則坐相違奪。且鄞等三縣,去治並遠,既安之民,忽徙他邑,新垣未立,舊居已毀,去留兩困,無以自資。謂宜適任民情,從其所樂,開宥逋亡,且令就業,若審成腴壤,然後議遷。」太常王玄謨議曰:「小民貧匱,遠就荒疇,去舊即新,糧種俱闕,習之既難,勸之未易。謂宜微加資給,使得肆勤,明力田之賞,申怠惰之罰。」光祿勳王升之議曰:「遠廢之疇,方翦荊棘,率課窮乏,其事彌難,資徙粗立,徐行無晚。」上違議,從其徙民,並
 成良業。



 靈符自丹陽出為會稽太守,尋加豫章王子尚撫軍長史。靈符家本豐,產業甚廣,又於永興立墅,周回三十三里,水陸地二百六十五頃,含帶二山,又有果園九處。



 為有司所糾,詔原之,而靈符答對不實,坐以免官。後復舊官,又為尋陽王子房右軍長史,太守如故。愨實有材幹,不存華飾,每所蒞官,政績脩理。前廢帝景和中,犯忤近臣,為所讒搆,遣鞭殺之。二子湛之、淵之,於都賜死。太宗即位,追贈靈符金紫光祿大夫。



 淵之,大明中為
 尚書比部郎。時安陸應城縣民張江陵與妻吳共罵母黃令死,黃忿恨自經死,值赦。律文,子賊殺傷毆父母,梟首;罵詈,棄市;謀殺夫之父母,亦棄市。值赦,免刑補冶。江陵罵母,母以之自裁,重於傷毆。若同殺科,則疑重;用毆傷及罵科,則疑輕。制唯有打母,遇赦猶梟首,無罵母致死值赦之科。淵之議曰:「夫題里逆心,而仁者不入,名且惡之,況乃人事。故毆傷咒詛,法所不原,詈之致盡,則理無可宥。罰有從輕,蓋疑失善,求之文旨,非此之謂。江陵
 雖值赦恩,故合梟首,。婦本以義,愛非天屬,黃之所恨,情不在吳,原死補冶,有允正法。」詔如淵之議,吳免棄市。



 羊玄保,太山南城人也。祖楷,尚書都官郎。父綏,中書侍郎。玄保起家楚臺太常博士,遭母憂,服闋,右將軍何無忌、前將軍諸葛長民俱板為參軍,並不就。



 除臨安令。劉穆之舉為高祖鎮軍參軍,庫部郎,永世令。復為高祖太尉參軍,轉主簿,丹陽丞。少帝景平二年,入為尚書右丞。轉左丞,司徒右長史。府公王弘甚知重之,謂左長史庾登
 之、吏部尚書王準之曰:「卿二賢明美朗識,會悟多通,然弘懿之望,故當共推羊也。」頃之,入為黃門侍郎。



 善弈棋,棋品第三,太祖與睹郡戲,勝,以補宣城太守。先是,劉式之為宣城,立吏民亡叛制,一人不禽,符伍里吏送州作部,若獲者賞位二階。玄保以為非宜,陳之曰:「臣伏尋亡叛之由,皆出於窮逼,未有足以推存而樂為此者也。今立殊制,於事為苦。臣聞苦節不可貞,懼致流弊。昔龔遂譬民於亂繩,緩之然後可理;黃霸以寬和為用,不以嚴
 刻為先。臣愚以謂單身逃役,便為盡戶。今一人不測,坐者甚多,既憚重負,各為身計,牽挽逃竄,必致繁滋。又能禽獲叛身,類非謹惜,既無堪能,坐陵勞吏,名器虛假,所妨實多,將階級不足供賞,服勤無以自勸。又尋此制,施一邦而已,若其是邪,則應與天下為一;若其非邪,亦不宜獨行一郡。民離憂患,其弊將甚。臣忝守所職,懼難遵用,致率管穴,冒以陳聞。」由此此制得停。



 玄保在郡一年,為廷尉。數月,遷尚書吏部郎,御史中丞,衡陽王義季右
 軍長史、南東海太守,加輔國將軍。入為都官尚書、左衛將軍,加給事中,丹陽尹,會稽太守。又徙吳郡太守,加秩中二千石。太祖以玄保廉素寡欲,故頻授名郡。為政雖無幹績,而去後常見思。不營財利,處家儉薄。太祖嘗曰:「人仕宦非唯須才,然亦須運命;每有好官缺,我未嘗不先憶羊玄保。」



 元兇弒立,為吏部尚書,領國子祭酒,尋加光祿大夫。及世祖入討,朝野多南奔,劭集群僚,橫刀怒曰:「卿等便可去矣!」眾戰懼莫敢言,玄保容色不異,徐曰:「
 臣以死奉朝。」劭乃解。世祖即位,以為散騎常侍,領崇憲衛尉。尋遷金紫光祿大夫。又以謹敬見知,賜賚甚厚。大明初,進位光祿大夫。五年,遷散騎常侍,特進。玄保自少至老,謹於祭奠,四時珍新,未得祠薦者,口不妄嘗。八年,卒,時年九十四。謚曰定子。



 子戎,有才氣,而輕薄少行檢,玄保嘗云:「此兒必亡我家。」官至通直郎。



 與王僧達謗議時政,賜死。死後世祖引見玄保,玄保謝曰:「臣無日磾之明,以此上負。」上美其言,戎二弟,太祖並賜名,曰咸,曰粲。
 謂玄保曰:「欲令卿二子有林下正始餘風。」



 玄保既善棋,而何尚之亦雅好棋。吳郡褚胤,年七歲,入高品。及長,冠絕當時。胤父榮期與臧質同逆,胤應從誅,何尚之請曰:「胤弈棋之妙,超古冠今。魏犨犯令,以才獲免。父戮子宥,其例甚多。特乞與其微命,使異術不絕。」不許。



 時人痛惜之。



 玄保兄子希,字泰聞,少有才氣。大明初,為尚書左丞。時揚州刺史西陽王子尚上言:「山湖之禁,雖有舊科,民俗相因,替而不奉,熂山封水,保為家利。自頃以來,頹弛
 日甚,富強者兼嶺而占,貧弱者薪蘇無託,至漁採之地,亦又如茲。



 斯實害治之深弊,為政所宜去絕,損益舊條,更申恒制。」有司檢壬辰詔書:「占山護澤,彊盜律論,贓一丈以上,皆棄市。」希以「壬辰之制,其禁嚴刻,事既難遵,理與時弛。而占山封水,漸染復滋,更相因仍,便成先業,一朝頓去,易致嗟怨。今更刊革,立制五條。凡是山澤,先常熂爈種養竹木雜果為林,及陂湖江海魚梁鰍鮆場,常加功脩作者,聽不追奪。官品第一、第二,聽占山三頃;第
 三、第四品,二頃五十畝;第五、第六品,二頃;第七、第八品,一頃五十畝;第九品及百姓,一頃。皆依定格,條上貲簿。若先已占山,不得更占;先占闕少,依限占足。



 若非前條舊業,一不得禁。有犯者,水土一尺以上,並計贓,依常盜律論。停除咸康二年壬辰之科。」從之。



 益州刺史劉瑀,先為右衛將軍,與府司馬何季穆共事不平。季穆為尚書令建平王宏所親待,屢毀瑀於宏。會瑀出為益州,奪士人妻為妾,宏使羊希彈之;瑀坐免官,瑀恨希切齒。有門
 生謝元伯往來希間,瑀令訪訊被免之由。希曰:「此奏非我意。」瑀即日到宏門奉箋陳謝,云聞之羊希。希坐漏泄免官。



 大明末,為始安王子真征虜司馬,黃門郎,御史中丞。泰始三年,出為寧朔將軍、廣州刺史。希初請女夫鎮北中兵參軍蕭惠徽為長史,帶南海太守,太宗不許。



 又請為東莞太宙。希既到鎮,長史、南海太守陸法真喪官,希又請惠徽補任。詔曰:「希卑門寒士,累世無聞,輕薄多釁,備彰歷職。徒以清刻一介,擢授嶺南,乾上逞欲,求訴
 不已,可降號橫野將軍。」



 初,李萬周、劉嗣祖籍略廣州,事在《鄧琬傳》。太宗以萬周為步兵校尉,加寧朔將軍,權行廣州事。希既至,而萬周等並有異圖,希誅之。希以沛郡劉思道行晉康太守,領軍伐俚。思道違節度,失利,希遣收之。思道不受命,率所領攻州,希遣平越長史鄒琰於朝亭拒戰,軍敗見殺。思道進攻州城,司馬鄒嗣之拒之西門,戰敗又死。希踰城走,思道獲而殺之。府參軍鄒曼率數十人襲思道,已得入城,力不敵,又敗。東莞太守蕭
 惠徽率郡文武千餘人攻思道,戰敗,又見殺。時龍驤將軍陳伯紹率軍伐俚,還擊思道,定之。贈希輔國將軍,惠徽中書郎,嗣之越騎校尉。



 希子崇,字伯遠,尚書主客郎。丁母憂,哀毀過禮。及聞廣州亂,即日便徒跣出新亭,不能步涉,頓伏江渚。門義以小船致之,於是進路。父葬畢,不勝哀,卒。



 沈曇慶,吳興武康人,侍中懷文從父兄也。父發,員外散騎侍郎,早卒;吳興太守王韶之為之誄焉。



 曇慶初辟主
 簿,州從事,西曹主簿,長沙王義欣後軍鎮軍主簿。遭母憂,哀毀致稱,本縣令諸葛闡之公解言上。服釋,復為主簿。義欣又請為鎮軍記室參軍。出為餘杭令,遷司徒主簿,江夏王義恭太尉錄事參軍,尚書右丞。時歲有水旱,曇慶議立常平倉以救民急,太祖納其言,而事不行。領本邑中正,少府,揚州治中從事史,始興王濬衛軍長史。元凶弒立,世祖入討,劭遣曇慶還東募人,安東將軍隨王誕收付永興縣獄,久之,被原。



 世祖踐阼,除東海王禕
 撫軍長史,入為尚書吏部郎,江夏王義恭大司馬長史,南東海太守,左衛將軍。大明元年,督徐兗二州及梁郡諸軍事、輔國將軍、徐州刺史。時殿中員外將軍裴景仁助戍彭城,本傖人,多悉戎荒事。曇慶使撰《秦記》十卷,敘苻氏僭偽本末,其書傳於世。明年,復徵為左衛將軍,加給事中,領本州大中正。三年,遷祠部尚書。其年,卒,時年五十七。追贈本官。曇慶謹實清正,所蒞有稱績。常謂子弟曰:「吾處世無才能,政圖作大老子耳。」世以長者稱之。



 史臣曰:江南之為國,盛矣。雖南包象浦,西括邛山,至於外奉貢賦,內充府實,止於荊、揚二州。自漢氏以來,民戶凋耗,荊楚四戰之地,五達之郊,井邑殘亡,萬不餘一也。自元熙十一年司馬休之外奔,至于元嘉末,三十有九載,兵車勿用,民不外勞,役寬務簡,氓庶繁息,至餘糧栖畝,戶不夜扃,蓋東西之極盛也。



 既揚部分析,境極江南,考之漢域,惟丹陽會稽而已。自晉氏遷流,迄於太元之世,百許年中,無風塵之警,區域之內,晏如也。及孫恩寇亂,
 殲亡事極,自此以至大明之季,年踰六紀,民戶繁育,將曩時一矣。地廣野豐,民勤本業,一歲或稔,則數郡忘饑。會土帶海傍湖,良疇亦數十萬頃,膏腴上地,畝直一金,鄠、杜之間,不能比也。荊城跨南楚之富,揚部有全吳之沃,魚鹽杞梓之利,充仞八方;絲綿布帛之饒,覆衣天下。而田家作苦,役難利薄,亙歲從務,無或一日非農,而經稅橫賦之資,養生送死之具,莫不咸出於此。穰歲糶賤,糶賤則稼苦;饑年糴貴,糴貴則商倍。常平之議,行於漢
 世。元嘉十三年,東土潦浸,民命棘矣。太祖省費減用,開倉廩以振之,病而不兇,蓋此力也。大明之末,積旱成災,雖敝同往困,而救非昔主,所以病未半古,死已倍之。並命比室,口減過半。若常平之計,興於中年,遂切扶患,或不至是。若籠以平價,則官苦民優,議屈當時,蓋由於此。



\end{pinyinscope}