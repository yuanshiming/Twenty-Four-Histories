\article{卷八十六列傳第四十六 殷孝祖 劉勔}

\begin{pinyinscope}

 殷孝祖,陳郡長平人也。曾祖羨,晉光祿勳。父祖並不達。孝祖少誕節,好酒色,有氣幹。太祖元嘉末,為奉朝請,員外散騎侍郎。世祖以其有武用,除奮武將軍、濟北太守。
 入為積射將軍。大明初,索虜寇青州,上遣孝祖北援,受刺史顏師伯節度,累與虜戰,頻大破之,事在師伯傳。還授太子旅賁中郎將,加龍驤將軍。



 竟陵王誕據廣陵為逆,孝祖隸沈慶之攻誕,又有戰功,遷西陽王子尚撫軍、寧朔將軍、南濟陰太守。出為盱眙太守,將軍如故。還為虎賁中郎將,仍除寧朔將軍、陽平東平二郡太守。又遷濟南、南郡,將軍如故。



 前廢帝景和元年,以本號督兗州諸軍事、兗州刺史。太宗初即位,四方反叛,孝祖外甥司
 徒參軍潁川葛僧韶建議銜命徵孝祖入朝,上遣之。時徐州刺史薛安都遣薛索兒等屯據津徑,僧韶間行得至,說孝祖曰:「景和凶狂,開闢未有,朝野危極,假命漏刻。主上聖德天挺,神武在躬,曾不浹辰,夷凶翦暴,更造天地,未足為言。



 國亂朝危,宜立長生,公卿百辟,人無異議,泰平之隆,非旦則夕。而群小相煽,構造無端,貪利幼弱,競懷希望。使天道助逆,群凶事申,則主幼時艱,權柄不一,兵難互起,豈有自容之地。舅少有立功之志,長以氣
 節成名,若便能控濟、河義勇,還奉朝廷,非唯匡主靜亂,乃可以垂名竹帛。」孝祖具問朝廷消息,僧韶隨方酬譬,并陳兵甲精彊,主上欲委以前驅之任。孝祖即日棄妻子,率文武二千人隨僧韶還都。



 時普天同逆,朝廷唯保丹陽一郡,而永世縣尋又反叛。義興賊垂至延陵,內外憂危,咸欲奔散。孝祖忽至,眾力不少,並傖楚壯士,人情於是大安。進孝祖號冠軍,假節、督前鋒諸軍事,遣向虎檻,拒對南賊。御仗先有諸葛亮筒袖鎧帽,二十五石弩射
 之不能入,上悉以賜孝祖。孝祖負其誠節,凌轢諸將,臺軍有父子兄弟在南者,孝祖並欲推治。由是人情乖離,莫樂為用。進使持節、都督兗州青冀幽四州諸軍事、撫軍將軍,刺史如故。時賊據赭圻,孝祖將進攻之,與大統王玄謨別,悲不自勝,眾並駭怪。泰始二年三月三日,與賊合戰,常以鼓蓋自隨,軍中人相謂曰:「殷統軍可謂死將矣。今與賊交鋒,而以羽儀自標顯,若善射者十士攢射,欲不斃,得乎?」是日,於陣為矢所中死,時年五十二。追贈
 散騎常侍、征北將軍,持節、都督如故。封秭歸縣侯,食邑千戶。四年,追改封建安縣,謚曰忠侯。孝祖子悉為薛安都所殺,以從兄子慧達繼封。齊受禪,國除。



 劉勔,字伯猷,彭城人也。祖懷義,始興太守。父穎之,汝南、新蔡二郡太守,征林邑,遇疾卒。勔少有志節,兼好文義。家貧,為廣州增城令,廣州刺史劉道錫引為揚烈府主簿。元嘉二十七年,索虜南侵,道錫遣勔奉使詣京都,太祖引見之,酬對稱旨,除寧遠將軍、綏遠太守。元嘉末,蕭
 簡據廣州為亂,勔起義討之,燒其南門。廣州刺史宗愨又命為軍府主簿,以功封大亭侯。除員外散騎侍郎。孝建初,荊、江反叛,宗愨以勔行寧朔將軍、湘東內史,領軍出安陸。會事平,以本號為晉康太守,又徙鬱林太守。大明初還都,徐州刺史劉道隆請為寧朔司馬。竟陵王誕據廣陵為逆,勔隨道隆受沈慶之節度,事平,封金城縣五等侯。除西陽王子尚撫軍參軍,入直閣。先是,遣費沈伐陳檀,不克,乃除勔龍驤將軍、西江督護、鬱林太守。



 勔既至,
 率軍進討,隨宜翦定,大致名馬,并獻珊瑚連理樹,上甚悅。還除新安王子鸞撫軍中兵參軍,遭母憂,不拜。前廢帝即位,起為振威將軍、屯騎校尉,入直閣。



 太宗即位,加寧朔將軍,校尉如故。江州刺史晉安王子勛為逆,四方響應,勔以本官領建平王景素輔國司馬,進據梁山。會豫州刺史殷琰反叛,徵勔還都,假輔國將軍,率眾討琰,甲仗三十人入六門。復兼山陽王休祐驃騎司馬,餘如故。破琰將劉順於宛唐,杜叔寶於橫塘,事在琰傳。除輔
 國將軍、山陽王休祐驃騎諮議參軍、梁郡太守、假節,不拜。琰嬰城固守,自始春至于末冬,薛道標、龐孟虯並向壽陽,勔內攻外禦,戰無不捷。善撫將帥,以寬厚為眾所依。將軍王廣之求勔所自乘馬,諸將帥並忿廣之叨冒,勸勔以法裁之,勔歡笑,即時解馬與廣之。復除使持節、督廣交二州諸軍事、平越中郎將、廣州刺史,將軍如故,不拜。及琰開門請降,勔約令三軍,不得妄動。城內士民,秋毫無所失,百姓感悅,咸曰來蘇。百姓生為立碑。



 改督益
 寧二州諸軍事、益州刺史,持節、將軍如故,又不拜。還京都,拜太子左衛率,封鄱陽縣侯,食邑千戶。



 琰初求救索虜,虜大眾屯據汝南。泰始三年,以勔為征虜將軍、督西討前鋒諸軍事,假節、置佐、本官如故。先是,常珍奇據汝南,與琰為逆,琰降,因據戍降虜,事在琰傳。至是引虜西河公、長社公攻圍輔國將軍、汝陰太守張景遠。景遠與軍主楊文萇拒擊,大破之。景遠尋病卒,太宗嘉其功,追贈冠軍將軍、豫州刺史,追封含洭縣男,食邑三百戶,以
 文萇代為汝陰太守。除勔右衛將軍,仍以為使持節、都督豫司二州諸軍事、征虜將軍、豫州刺史,餘如故。四年,除侍中,領射聲校尉,又不受。進號右將軍。其年,虜遣汝陽司馬趙懷仁步騎五百,寇武津縣。勔遣龍驤將軍曲元德輕兵進討,虜眾驚散。虜子都公閼於拔又率三百人防運車囗囗千兩,於汝陽臺東水上結營。元德單騎直入,斬拔首,因進攻汝陽臺,即陷外壘,獲車一千三百乘,斬首一百五十級。勔又使司徒參軍孫曇瓘督弋陽
 以西,會虜寇義陽,曇瓘大破之。虜上其北豫州租,有車二千兩,勔招荒人,邀擊於許昌,虜眾奔散,焚燒米穀。



 淮西人賈元友上書太宗,勸北攻懸瓠,可收陳郡、南頓、汝南、新蔡四郡之地。



 上以所陳示勔,使具條答。勔對曰:元友稱:「虜主幼弱,姦偽競起,內外規亂,天亡有期。」臣以為獯醜侵縱,乘藉王境,盤據州郡,百姓殘亡。去冬眾軍失耕,今春連城圍逼,國家復境之略,實有不遑,滅虜未及。元友又云:「有七千餘家,穀米豐積,可供二萬人數年
 資儲。」



 臣又以為二萬人歲食米四十八萬斛,五年合須米二百四十萬斛,既理不容有,恐事難稱言。元友又云:「虜於懸瓠開驛保,虜已先據,若不足恃,此不須囗。」俱是攻城,便應先圖懸瓠,何更越先取郾,以受腹背之災。且七千餘家豐積,而虜猶當遠運為糧,是威不制民,民非異計。元友又云:「虜欲水陸運糧,以救軍命,可襲之機,在於今日。」臣又以為開立驛道,據守堅城,觀其形候,不似蹙弱。可乘之機,恐為難驗。元友又云:「四郡民人,遭虜二
 十七年之毒,皆欲雪仇報恥,伏待朝威。」臣又以為垣式寶等受國重恩,今猶驅略車營,翻還就賊,蓋是戀本之情深,非報怨之宜,何可輕試。元友又云:「請敕荊、雍兩州,遣二千精兵,從義陽依西山北下,直據郾城。」臣又以為郾城是賊驛路要戍,且經蠻接險,數百里中,裹糧潛進,方出平地,攻賊堅城,自古名將,未有能以此濟者。假其克捷,不知足南抗懸瓠,北捍長社與不?且賊擁據數城,水陸通便,而今使官以二千斷其資運,於事為難。元友
 又云:「虜圍逼汝陰,遊魂二歲,為張景遠所挫,不敢渡淮。」臣又以為景遠兵力寡弱,不能自固,遠遣救援,方得少克。今定是為賊所畏不?景遠前所摧傷,裁至數百,虜步騎四萬,猶不敢前,而今必勸國家以輕兵遠討,指掌可克,言理相背,莫復過此。元友又云:「龍山雉水,魯奴、王景直等並受朝爵,馬步萬餘。進討之宜,唯須敕命。」臣以為魯奴與虜交關,彌歷年世,去歲送誠朝廷,誓欲立功。自蒙榮爵,便即逃遁,殊類姦猾,豈易暗期。兼王景直是一
 亡命,部曲不過數十人,既不可言,又未足恃。萬餘之言,似不近實。元友又云:「四郡恨忿此非類,車營連結,廢田二載,生業已盡,賊無所資,糧儲已罄。斷其運道,最是要略。」臣又以斷運須兵,兵應資食,而當此過懸瓠二百里中,使兵食兼足,何處求辦?



 臣竊尋元嘉以來,傖荒遠人,多干國議,負儋歸闕,皆勸討虜。魯爽誕說,實挫國威,徒失兵力,虛費金寶。凡此之徒,每規近說,從來信納,皆詒後悔。界上之人,唯視彊弱,王師至境,必壺漿候塗,裁見
 退軍,便抄截蜂起。首領回師,何嘗不為河畔所弊。



 太宗納之,元友議遂寢。勔與常珍奇書,勸令反虜,珍奇乃與子超越、羽林監式寶,於譙殺虜子都公費拔等凡三千餘人。勔馳驛以聞,太宗大喜,以珍奇為使持節、都督司北豫二州諸軍事、平北將軍、司州刺史,汝南新蔡縣侯,食邑千戶;超越輔國將軍、北豫州刺史,潁川汝陽囗囗三郡太守,安陽縣男;式寶輔國將軍、陳南頓二郡太守,真陽縣男,食邑三百戶。珍奇為虜所攻,引軍南出,虜追
 擊破之,珍奇走依山,得至壽陽,超越、式寶為人所殺。



 五年,汝陰太守楊文萇又頻破虜於荊亭及戍西。詔進勔號平西將軍、豫州刺史,餘如故,不拜。其年,徵拜散騎常侍、中領軍。勔以世路糾紛,有懷止足,求東陽郡。上以勔啟遍示朝臣,自尚書僕射袁粲以下,莫不稱贊,咸謂宜許。上曰:「巴陵、建平二王,並有獨往之志。若世道寧晏,皆當申其所請。」勔經始鐘嶺之南,以為棲息,聚石蓄水,仿佛丘中,朝士愛素者,多往游之。六年,改常侍為侍中。



 其
 年,南兗州刺史齊王出鎮淮陰,以勔為使持節、都督南徐兗青冀囗五州諸軍事、平北將軍,侍中、中領軍如故,出鎮廣陵。固辭侍中、軍號,許之,以為假平北將軍。七年,解都督、假號、并節。太宗臨崩,顧命以為守尚書右僕射,中領軍如故,給鼓吹一部。廢帝即位,加兵五百人。



 元徽初,月犯右執法,太白犯上將,或勸勔解職。勔曰:「吾執心行己,無愧幽明。若才輕任重,災眚必及,天道密微,避豈得免。」桂陽王休範為亂,奄至京邑,加勔使持節、領軍,置佐
 史,鎮捍石頭。既而賊眾屯朱雀航南,右軍王道隆率宿衛向朱雀,聞賊已至,急信召勔。勔至,命閉航,道隆不聽,催勔渡航進戰。率所領於航南戰敗,臨陳死之,時年五十七。事平,詔曰:「夫義實天經,忠惟人則,篆素流采,金石宣輝,自非識洞情靈,理感生極,豈有捐軀衛主,舍命匡朝者哉!



 故持節、鎮軍將軍、守尚書右僕射、中領軍鄱陽縣開國侯勔,思懷亮粹,體業淹明,弘勳樹績,譽洽華野。綢繆顧託,契闊屯夷,方倚謀猷,翌康帝道。逆蕃扇禍,逼
 擾京甸,援桴誓旅,奉律行師。身與事滅,名隨操遠。朕用傷悼,震慟于厥心。昔王允秉誠,卞壺峻節,均風往德,歸茂先軌。泉途就永,冤逝無追,思崇徽策,式光惇史。可贈散騎常侍、司空,本官、侯如故,謚曰忠昭公。」



 子悛嗣,順帝昇明末,為廣州刺史。齊受禪,國除。勔弟斅,泰始中,為寧朔將軍、交州刺史,於道遇病卒。先有都鄉侯爵,謚曰質侯。



 史臣曰:吳漢平蜀,城內流血霑踝,而其後無聞於漢;陸
 抗定西陵,步氏禍及嬰孩,而機、雲為戮上國。劉勔克壽春,士民無遺芻委粒之嘆;莫不扶老攜幼,歌唱而出重圍,美矣!



\end{pinyinscope}