\article{卷十二志第二 歷中}

\begin{pinyinscope}

 夫天地之所貴者生也,萬物之所尊者人也。役智窮神,無幽不察,是以動作云為,皆應天地之象。古先聖哲,擬辰極,制渾儀。夫陰陽二氣,陶育群品,精象所寄,是為日
 月。群生之性,章為五才,五才之靈,五星是也。歷所以擬天行而序七耀,紀萬國而授人時。黃帝使大撓造六甲,容成制歷象,羲和占日,常儀占月。少昊氏有鳳鳥之瑞,以鳥名官,而鳳鳥氏司歷。顓頊之代,南正重司天,北正黎司地。



 堯復育重黎之後,使治舊職,分命羲和,欽若昊天。故《虞書》曰:「期三百有六旬六日,以閏月定四時成歲。」其後授舜,曰:「天之歷數在爾躬。」舜亦以命禹。



 爰及殷、周二代,皆創業革制,而服色從之。順其時氣以應天道,萬
 物群生,蒙其利澤。三王既謝,史職廢官,故孔子正《春秋》,以明司歷之過。秦兼天下,自以為水德,以十月為正,服色上黑。



 漢興,襲秦正朔,北平侯張蒼首言律歷之事,以《顓頊歷》比於六歷,所失差近。施用至武帝元封七年,太中大夫公孫卿、壺遂、太史令司馬遷等,言歷紀廢壞,宜改正朔,易服色,所以明受之於天也。乃詔遂等造漢歷。選鄧平、長樂司馬可及人間治歷者,二十餘人。方士唐都分天部,落下閎運算轉歷。其法積八十一寸,則一日
 之分也,閎與鄧平所治同。於是皆觀星度,日月行,更以算推,如閎、平法,一月之日二十九日八十一分日之四十三。詔遷用鄧平所造八十一分律歷,以平為太史丞。至元鳳三年,太史令張壽王上書,以為元年用黃帝《調歷》,「令陰陽不調,更歷之過」。詔下主歷使者鮮于妄人與治歷大司農中丞麻光等二十餘人雜候晦朔弦望二十四氣。又詔丞相、御史、大將軍、右將軍史各一人雜候上林清臺,課諸疏密,凡十一家,起三年盡五年。壽王課
 疏遠。又漢元年不用黃帝《調歷》,效劾壽王逆天地,大不敬,詔勿劾。復候,盡六年,《太初歷》第一。壽王歷乃太史官《殷歷》也。壽王再劾不服,竟下吏。至孝成時,劉向總六歷,列是非,作《五紀論》。向子歆作《三統歷》以說《春秋》,屬辭比事,雖盡精巧,非其實也。班固謂之密要,故漢《歷志》述之。校之何承天等六家之歷,雖六元不同,分章或異,至今所差,或三日,或二日數時,考其遠近,率皆六國及秦時有人所造。其術斗分多,上不可檢於《春秋》,下不驗於漢、魏,
 雖復假稱帝王,只足以惑時人耳。



 光武建武八年,太僕朱浮上言歷紀不正,宜當改治,時所差尚微,未遑考正。



 明帝永平中,待詔楊岑、張盛、景防等典治歷,但改易加時弦望,未能綜校歷元也。



 至元和二年,《太初》失天益遠,宿度相覺浸多,候者皆知日宿差五度,冬至之日在斗二十一度,晦朔弦望,先天一日。章帝召治歷編欣、李梵等綜核意狀。遂下詔書稱:「《春秋保乾圖》曰:『三百年斗歷改憲。』史官用《太初》鄧平術,有餘分
 一,在三百年之域,行度轉差,浸以繆錯,璇璣不正,文象不稽。冬至之日,日在斗二十二度,先立春一日,則《四分》之立春日也。而以折獄斷大刑,於氣已逆;用望平和,蓋亦遠矣。今改行《四分》,以遵堯順孔,奉天之文,同心敬授,儻獲咸熙。」於是《四分法》施行。黃帝以來諸歷以為冬至在牽牛初者,皆黜焉。



 和帝永元十四年,待詔太史霍融上言:「官漏刻率九日增減一刻,不與天相應,或時差至二刻半,不如夏歷密。」其年十一月甲寅,詔曰:「漏所以節
 時分,定昏明。昏明長短,起於日去極遠近,日道周圜,不可以計率分。官漏九日增減一刻,違失其實,以晷景為刻,密近有驗,今下晷景漏刻四十八箭。」其二十四氣日所在,并黃道去極、晷景、漏刻、昏明中星,並列載於《續漢律歷志》。安帝延光三年,中竭者亶誦上書言當用甲寅元,河南梁豐云當復用《太初》。尚書郎張衡、周興皆審歷,數難誦、豐,或不能對,或云失誤。衡等參案儀注,考往校今,以為《九道法》最密。詔下公卿詳議。太尉愷等參議:「《太
 初》過天一度,月以晦見西方。



 元和改從《四分》,《四分》雖密於《太初》,復不正,皆不可用。甲寅元與天相應,合圖讖,可施行。」議者不同。尚書令忠上奏:「天之歷數,不可任疑從虛,以非易是。」亶等遂寢。



 靈帝熹平四年,五官郎中馮光、沛相上計掾陳晃等言:「歷元不正,故盜賊為害。歷當以甲寅為元,不用庚申,乞本庚申元經緯明文。」詔下三府,與儒林明道術者詳議。群臣會司徒府集議。議郎蔡邕曰:「歷數精微,術無常是。漢興承秦,歷用《顓頊》,元用乙卯;
 百有二歲,孝武皇帝始改《太初》,元用丁丑。行之百八十九歲,孝章帝改從《四分》,元用庚申。今光等以庚申為非,甲寅為是。按歷法,黃帝、顓頊、夏、殷、周、魯,各自有元。光、晃所援,則殷歷元也。昔始用《太初》丁丑之後,六家紛錯,爭訟是非。張壽王挾甲寅元以非漢歷,雜候清臺,課在下第。《太初》效驗,無所漏失。是則雖非圖讖之元,而有效於前者也。及用《四分》以來,考之行度,密於《太初》,是又新元有效於今者也。故延光中,亶誦亦非《四分》,言當用甲寅
 元,公卿參議,竟不施行。且三光之行,遲速進退,不必若一。故有古今之術。今術之不能通於古,亦猶古術不能下通於今也。又光、晃以《考靈耀》為本,二十八宿度數至日所在,錯異不可參校。元和二年用至今九十二歲,而光、晃言陰陽不和,姦臣盜賊,皆元之咎。元和詔書,文備義著,非群臣議者所能變易。」三公從邕議,以光、晃不敬,正鬼薪法,詔書勿治罪。



 何承天曰:夫歷數之術,若心所不達,雖復通人前識,無
 救其為敝也。是以多歷年歲,未能有定。《四分》於天,出三百年而盈一日。積代不悟,徒云建歷之本,必先立元,假言讖緯,遂關治亂,此之為蔽,亦已甚矣。劉歆《三統法》尤復疏闊,方於《四分》,六千餘年又益一日。揚雄心惑其說,采為《太玄》,班固謂之最密,著於《漢志》;司彪因曰「自太初元年始用《三統歷》,施行百有餘年」。曾不憶劉歆之生,不逮太初,二三君子言歷,幾乎不知而妄言歟!



 光和中,穀城門候劉洪始悟《四分》於天疏闊,更以五百
 八十九為紀法;百四十五為斗分,造《乾象法》。又制遲疾歷以步月行。方於《太初》、《四分》,轉精微矣。魏文帝黃初中,太史丞韓翊以為《乾象》減鬥分太過,後當先天,造《黃初歷》,以四千八百八十三為紀法,一千二百五為斗分。其後尚書令陳群奏,以為「歷數難明,前代通儒多共紛爭。《黃初》之元,以《四分歷》久遠疏闊,大魏受命,宜正歷明時。韓翊首建《黃初》,猶恐不審,故以《乾象》互相參校。歷三年,更相是非,舍本即末,爭長短而疑尺丈,竟無時而決。按三
 公議,皆綜盡曲理,殊塗同歸,欲使效之璇璣,各盡其法,一年之間,得失足定,合於事宜。」奏可。明帝時,尚書郎楊偉制《景初歷》,施用至於晉、宋。古之為歷者,鄧平能修舊制新,劉洪始減《四分》,又定月行遲疾,楊偉斟酌兩端,以立多少之衷,因朔積分設差,以推合朔月蝕。此三人,漢、魏之善歷者,然而洪之遲疾,不可以檢《春秋》;偉之五星,大乖於後代,斯則洪用心尚疏,偉拘於同出上元壬辰故也。



 魏明帝景初元年,改定歷數,以建丑之月為正,改其年三月為孟夏四月。其孟仲季月,雖與正歲不同,至於郊祀、迎氣、祭祠、烝嘗,巡狩、搜田,分至啟閉,班宣時令,皆以建寅為正。三年正月,帝崩,復用夏正。



 楊偉表曰:「臣攬載籍,斷考歷數,時以紀農,月以紀事,其所由來,遐而尚矣。乃自少昊,則玄鳥司分;顓頊帝嚳,則重、黎司天;唐帝、虞舜,則羲、和掌日。三代因之,則世有日官。日官司歷,則頒之諸侯,諸侯受之,則頒於境內。夏后之代,羲、和湎淫,廢
 時亂日,則《書》載《胤征》。由此觀之,審農時而重人事者,歷代然也。逮至周室既衰,戰國橫騖,告朔之羊,廢而不紹,登臺之禮,滅而不遵。閏分乖次而不識,孟陬失紀而莫悟,大火猶西流,而怪蟄蟲之不藏也。是時也,天子不協時,司歷不書日,諸侯不受職,日御不分朔,人事不恤,廢棄農時。



 仲尼之撥亂於《春秋》,託褒貶糾正,司歷失閏,則譏而書之,登臺頒朔,則謂之有禮。自此以降,暨於秦、漢,乃復以孟冬為歲首,閏為後九月,中節乖錯,時月紕繆,
 加時後天,蝕不在朔,累載相襲,久而不革也。至武帝元封七年,始乃寤其繆焉。於是改正朔,更歷數,使大才通人,造《太初歷》。校中朔所差,以正閏分;課中星得度,以考疏密,以建寅之月為正朔,以黃鐘之月為歷初。其歷斗分太多,後遂疏闊。至元和二年,復用《四分歷》。施而行之。至於今日,考察日蝕,率常在晦,是則鬥分太多,故先密後疏而不可用也。是以臣前以制典餘日,推考天路,稽之前典,驗之食朔,詳而精之,更建密歷,則不先不後,古今
 中天。以昔在唐帝,協日正時,允厘百工,咸熙庶績也。欲使當今國之典禮,凡百制度,皆韜合往古,郁然備足,乃改正朔,更歷數,以大呂之月為歲首,以建子之月為歷初。臣以為昔在帝代,則法曰《顓頊》,曩自軒轅,則歷曰《黃帝》。暨至漢之孝武,革正朔,更歷數,改元曰太初,因名《太初歷》。今改元為景初,宜曰《景初歷》。臣之所建《景初歷》,法數則約要,施用則近密,治之則省功,學之則易知。雖復使研、桑心算,隸首運籌,重、黎司晷,羲、和察景,以考天路,
 步驗日月,究極精微,盡術數之極者,皆未如臣如此之妙也。是以累代歷數,皆疏而不密,自黃帝以來,改革不已。



 壬辰元以來,至景初元年丁巳,歲積四千四十六,算上。此元以天正建子黃鐘之月為歷初,元首之歲夜半甲子朔旦冬至。



 元法,萬一千五十八。



 紀法,千八百四十三。



 紀月,二萬二千七百九十五。



 章歲,十九。



 章月,二百三十五。



 章閏,七。



 通數,十三萬四千六百三十。



 日法,四千五百五十九。



 餘數,九千六百七十。



 周天,六十七萬三千一百五十。



 紀日歲中,十二。



 氣法,十二。



 沒分,六萬七千三百一十五。



 沒法,九百六十七。



 月周,二萬四千六百三十八。



 通法,四十七。



 會通,七十九萬一百二十。



 朔望合數,六萬七千三百一十五。



 入交限數,七十二萬二千七百九十五。



 通周,十二萬五千六百二十一。



 周日日餘,二千五百二十八。



 周虛,二千三十一。



 斗分,四百五十五。



 甲子紀第一:紀首合朔,月在日道裏。



 交會差率,四十一萬二千九百一十九。



 遲疾差率,十萬三千九百四十七。



 甲戌紀第二:紀首合朔,月在日道里。



 交會差率,五十一萬六千五百二十九。



 遲疾差率,七萬三千七百六十七。



 甲申紀第三:紀首合朔,月在日道里。



 交會差率,六十二萬一百三十九。



 遲疾差率,四萬三千五百八十七。



 甲午紀第四:紀首合朔,月在日道裏。



 交會差率,七十二萬三千七百四十九。



 遲疾差率,一萬三千四百七。



 甲辰紀第五:紀首合朔,月在日道裏。



 交會差率,三萬七千二百四十九。



 遲疾差率,一十萬八千八百四十八。



 甲寅紀第六:紀首合朔,月在日道裏。



 交會差率,十四萬八百五十九。



 遲疾差率,七萬八千六百六十八。



 交會紀差,十萬三千六百一十。求其數之所生者,置一紀積月以通數乘之,會通去之,所去之餘,紀差之數也。以之轉加前紀,則得後紀。加之未滿會通者,則紀首之
 歲天正合朔,月在日道裏,滿去之,則月在日道表。加表滿在裏,加裏滿在表。



 遲疾紀差,三萬一百八十。求其數之所生者,置一紀積月,以通數乘之,通周去之,餘以減通周,所減之餘,紀差之數也。以之轉減前紀,則得後紀。不足減者,加通周。求次元紀差率,轉減前元甲寅紀差率,餘則次元甲子紀差率也。求次紀,如上法也。



 推朔積月術曰:置壬辰元以來,盡所求年,外所求,以紀
 法除之,所得算外,所入紀第也,餘則入紀年數。年以章月乘之,如章歲而一為積月,不盡為閏餘。閏餘十二以上,其年有閏。閏月以無中氣為正。推朔術曰:以通數乘積月,為朔積分,如日法而一為積日,不盡為小餘。以六十去積日,餘為大餘。大餘命以紀,算外,所求年天正十一月朔日也。求次月,加大餘二十九,小餘二千四百一十九,小餘滿日法從大餘,命如前,次月朔日也。小餘二千一百四十
 以上,其月大也。推弦望,加朔大餘七,小餘千七百四十四,小分一,小分滿二從小餘,上餘滿日法從大餘,大餘滿六十去之,餘命以紀,算外,上弦日也。又加得望、下弦、後月朔。其月蝕望者,定小餘,如所近中節間限,限數以下者,算上為日。望在中節前後各四日以還者,視限數;望在中節前後各五日以上者,視間限。



 推二十四氣術曰:置所入紀年,外所求,以餘數乘之,滿
 紀法為大餘,不盡為小餘。大餘滿六十去之,餘命以紀,算外,天正十一月冬至日也。求次氣,加大餘十五,小餘四百二,小分十一,小分滿氣法從小餘,小餘滿紀法從大餘,命如前,次氣日也。



 推閏月術曰:以閏餘減章歲,餘以歲中乘之,滿章閏得一月,餘滿半法以上亦得一月。數從天正十一月起,算外,閏月也。閏有進退,以無中氣御之。


冬至,十一月中。
 \gezhu{
  限數千二百五十四。間限千二百四十五。}


小寒,十二月節。
 \gezhu{
  限數千二百三十五。間限千二百二十四。}


大寒,十二月中。
 \gezhu{
  限數千二百一十三。間限千一百九十二。}


立春,正月節。
 \gezhu{
  限數千一百七十二。間限千一百三十七。}


雨水,正月中。
 \gezhu{
  限數千一百一十二。間限千九十三。}


驚蟄,二月節。
 \gezhu{
  限數千六十五。間限千二十六。}


春分,二月中。
 \gezhu{
  限數千八。間限九百七十九。}


清明,三月節
 \gezhu{
  限數九百五十一。間限九百二十五。}


穀雨,三月中。
 \gezhu{
  限數九百。間限八百七十九。}


立夏,四月節。
 \gezhu{
  限數八百五十七。間限八百四十。}


小滿,四月中。
 \gezhu{
  限數八百二十二。間限八百一十三。}


芒種,五月節。
 \gezhu{
  限數八百。間限七百九十九。}


夏至,五月中。
 \gezhu{
  限數七百九十八。間限八百。}


小暑,六月節。
 \gezhu{
  限數八百五。間限八百一十五。}


大暑,六月中。
 \gezhu{
  限數八百二十五。間限八百四十二。}


立秋,七月節。
 \gezhu{
  限數八百五十九。間限八百八十三。}


處暑,七月中。
 \gezhu{
  限數九百七。間限九百三十五。}


白露,八月節。
 \gezhu{
  限數九百六十二。間限九百九十二。}


秋分,八月中。
 \gezhu{
  限數千二十一。間限千五十一。}


寒露,九月節。
 \gezhu{
  限數千八十。間限千一百七。}


霜降,九月中。
 \gezhu{
  限數千一百三十三。間限千一百五十七。}


立冬,十月節。
 \gezhu{
  限數千一百八十一。間限千一百九十八。}


小雪,十月中。
 \gezhu{
  限數千二百一十五。間限千二百二十九。}



 推沒滅術曰:因冬至積日有小餘者,加積一,以沒分乘
 之,以沒法除之,所得為大餘,不盡為小餘。大餘滿六十去之,餘命以紀,算外,即去年冬至後沒日也。



 求次沒,加大餘六十九,小餘五百九十二,小餘滿沒法得一,從大餘,命如前。



 小餘盡,為滅也。



 推五行用事日:立春、立夏、立秋、立冬者,即木、火、金、水始用事日也。



 各減其大餘十八,小餘四百八十三,小分六,餘命以紀,算外,各四立之前土用事日也。大餘不足減者,加六十;小餘不足減者,減大餘一,加紀法;小分不足
 減者,減小餘一,加氣法。



 推卦用事日:因冬至大餘,六其小餘,《坎卦》用事日也。加小餘萬九十一,滿元法從大餘,即《中孚》用事日也。求次卦,各加大餘六,小餘九百六十七。其四正各因其中日,六其小餘。



 推日度術曰:以紀法乘朔積日,滿周天去之,餘以紀法除之,所得為度,不盡為分。命度從牛前五起,宿次除之,不滿宿,則天正十一月朔夜半日所在度及分也。



 求次日,日加一度,分不加,經斗除斗分,分少退一度。推月度術曰:以月周乘朔積日,滿周天去之,餘以紀法除之,所得為度,不盡為分,命如上法,則天正十一月朔夜半月所在度及分也。求次月,小月加度二十二,分八百六;大月又加一日,度十三,分六百七十九;分滿紀法得一度,則次月朔夜半月所在度及分也。其冬下旬,夕在張心署之。



 推合朔度術曰:以章歲乘朔小餘,滿通法為大分,不盡
 為小分。以大分從朔夜半日度分,分滿紀法從度,命如前,則天正十一月合朔日月所共合度也。



 求次月,加度二十九,大分九百七十七,小分四十二,小分滿通法從大分,大分滿紀法從度。經斗除其分,則次月合朔日月所共合度也。



 推弦望日所在度:加合朔度七,大分七百五,小分十,微分一,微分滿二從小分,小分滿通法從大分,大分滿紀法從度,命如前,則上弦日所在度也。又加得望、下弦、後
 月合也。推弦望月所在度:加合朔度九十八,大分千二百七十九,小分三十四,數滿命如前,即上弦月所在度也。又加得望下弦後月合也。



 推日月昏明度術曰:日以紀法,月以月周,乘所近節氣夜漏,二百而一,為明分。日以減紀法,月以減月周,餘為昏分。各以加夜半,如法為度。



 推合朔交會月蝕術曰:置所入紀朔積分,以所入紀交
 會差率之數加之,以會通去之,餘則所求年天正十一月合朔去交度分也。以通數加之,滿會通去之,餘則次月合朔去交度分也。以朔望合數各加其月合朔去交度分,滿會通去之,餘則各其月望去交度分也。朔望去交分如朔望合數以下,入交限數以上者,朔則交會,望則月蝕。推合朔交會月蝕月在日道表裏術曰:置所入紀朔積分,以所入紀下交會差率之數加之,倍會通去之,餘不
 滿會通者,紀首表,天正合朔月在表,紀首裏,天正合朔月在裏。滿會通去之,表在裏,裏在表。



 求次月,以通數加之,滿會通去之,加裏滿在表,加表滿在裏。先交會後月蝕者,朔在表則望在表,朔在裏則望在裏。先月蝕後交會者,看食月朔在裏則望在表,朔在表則望在裏。交會月蝕如朔望會數以下,則前交後會;如入交限數以上,則前會後交。其前交後會近於限數者,則豫伺之前月;前會後交近於限數者,則後伺之後
 月。



 求去交度術曰:其前交後會者,今去交度分如日法而一,所得則卻去交度也。



 其前會後交者,以去交度分減會通,餘如日法而一,所得則前去交度,餘皆度分也。



 去交度十五以上,雖交不蝕也。十以下是蝕,十以上虧蝕微少,光晷相及而已。虧之多少,以十五為法。



 求日蝕虧起角術曰:其月在外道,先交後會者,虧蝕西南角起;先會後交者,虧蝕東南角起。其月在內道,先交後會者,虧食西北角起;先會後交者,虧食東北角起。虧
 食分多少,如上以十五為法。會交中者,蝕盡。月蝕在日之衝,虧角與上反也。



 月行遲疾度損益率盈縮積分月行分一日十四度十四分益二十六盈初二百八十二日十四度十一分益二十三盈積分一十一萬八千五百三十四二百七十七
 三日十四度八分益二十盈積分二十二萬三千三百九十一二百七十四四日十四度五分益十七盈積分三十一萬四千五百七十一二百七十一五日十四度一分益十三盈積分三十九萬二千七十四
 二百六十七六日十三度十四分益七盈積分四十五萬一千三百四十一二百六十一七日十三度七分損盈積分四十八萬三千二百五十四二百五十四八日十三度一分損六盈積分四十八
 萬三千二百五十四二百四十八九日十二度十六分損十盈積分四十五萬五千九百二百四十四十日十二度十三分損十三盈積分四十一萬三百一十二百四十一
 十一日十二度十一分損十五盈積分三十五萬一千四十三二百三十九十二日十二度八分損十八盈積分二十八萬二千六百五十八二百三十六十三日十二度五分損二十一盈積分二十萬五百九十六二
 百三十三十四日十二度三分損二十三盈積分十萬四千八百五十七二百三十一十五日十二度五分益二十一縮初二百三十三十六日十二度七分益十九縮積分九萬五千七百三十九
 二百三十五十七日十二度九分益十七縮積分十八萬二千三百六十二百三十七十八日十二度十二分益十四縮積分二十五萬九千八百六十三二百四十十九日十二度十五分益十一縮積分三十二
 萬三千六百八十九二百四十三二十日十二度十八分益八縮積分三十七萬三千八百三十八二百四十六廿一日十三度三分益四縮積分四十一萬三百一十二百五
 十廿二日十三度七分損縮積分四十二萬八千五百四十六二百五十四廿三日十三度十二分損五縮積分四十二萬八千五百四十六二百五十九廿四日十三度十八分損十一縮積分四十萬五千七百五十一
 二百六十五廿五日十四度五分損十七縮積分三十五萬五千六百二二百七十一廿六日十四度十一分損二十三縮積分二十七萬八千九十九二百七十七廿七日十四度十一分損二十四縮積分十七萬
 三千二百四十二二百七十八周日十四度十三分損二十五縮積分六萬三千八百二十六二百七十九有小分六百二有小分二百十六二十六推合朔交會月蝕入遲疾歷術曰:置所入紀朔積分,以所入紀下遲疾差率之數加之,以通周去之,餘滿日法得一日,不盡為日餘,命日算外,則所求年天正十一月
 合朔入歷日也。



 求次月,加一日,日餘四千四百五十。求望,加十四日,日餘三千四百八十九。



 日餘滿日法成日,日滿二十七去之。又除餘如周日餘,日餘不足者,減一日,加周虛。



 推合朔交會月蝕定大小餘:以入歷日餘,乘所入歷損益率,以損益盈縮積分為定積分。以章歲減所入歷月行分,餘以除之,所得以盈減縮加本小餘。加之滿日法者,交會加時在後日;減之,不足者,交會加時在前日。月蝕
 者,隨定大小餘為日加時。入歷在周日者,以周日日餘乘縮積分,為定積分。以率損乘入歷日餘,又以周日日餘乘之,以周日日度小分并之,以損定積分,餘為後定積分。以章歲減周日月行分,餘以周日日餘乘之,以周日度小分并之,以除後定積分,所得以加本小餘,如上法。



 推加時:以十二乘定小餘,滿日法得一辰,數從子起,算外,則朔望加時所在辰也。有餘不盡者四之,如日法而
 一為少,二為半,三為太。又有餘者三之,如日法而一為強,半法以上排成之,不滿半法廢棄之。以強并少為少強,并半為半強,并太為太強。得二強者為少弱,以之并少為半弱,以之并半為太弱,以之并太為一辰弱。以所在辰命之,則各得其少、太、半及強、弱也。其月蝕望在中節前後四日以還者,視限數;五日以上者,視間限。定小餘如間限、限數以下者,以算上為日。


斗二十六
 \gezhu{
  分四百五十五} 牛八女十二虛十危十七室十六壁九北方九十八度
 \gezhu{
  分四百五十五}


奎十六
 婁十二胃十四昴十一畢十六觜二參九西方八十度井三十三
 鬼四柳十五星七張十八翼十八軫十七南方百一十二度角十二
 亢九氐十五房五心五尾十八箕十一東方七十五度中節日所在度
 日行黃道去極度日中晷景冬至
 \gezhu{
  十一月中}
 斗二十一
 \gezhu{
  少}
 百一十五度
 丈三尺小寒
 \gezhu{
  十二月節}
 女二
 \gezhu{
  少}
 百一十三
 \gezhu{
  強}
 丈二尺三寸
 大寒
 \gezhu{
  十二月中}
 虛五
 \gezhu{
  半弱}
 百一十
 \gezhu{
  太弱}
 丈一尺
 立春
 \gezhu{
  正月節}
 危十
 \gezhu{
  太弱}


百六
 \gezhu{
  少弱}
 九尺六寸雨水
 \gezhu{
  正月中}
 室八
 \gezhu{
  太強}
 百一
 \gezhu{
  強}


七尺九寸
 \gezhu{
  五分} 驚蟄
 \gezhu{
  二月節}
 壁八
 \gezhu{
  強}
 九十五
 \gezhu{
  強}


六尺五寸
 春分
 \gezhu{
  二月中}
 奎十四
 \gezhu{
  少強}
 八十九
 \gezhu{
  少強}
 五尺


二寸
 \gezhu{
  五分}
 清明
 \gezhu{
  三月節}
 胃一
 \gezhu{
  半} 八十三
 \gezhu{
  少弱}
 四尺一寸
 \gezhu{
  五分}
 穀雨
 \gezhu{
  三月中}
 昴二
 \gezhu{
  太}
 七十七
 \gezhu{
  太強} 三尺二寸立夏
 \gezhu{
  四月節}
 畢六
 \gezhu{
  太}
 七十三
 \gezhu{
  少弱}
 二尺五寸
 \gezhu{
  二分}
 小滿
 \gezhu{
  四月中}
 參四
 \gezhu{
  少弱}
 六十九
 \gezhu{
  太}
 尺九寸
 \gezhu{
  八分}
 芒種
 \gezhu{
  五月節}


井十
 \gezhu{
  半弱}
 六十七
 \gezhu{
  少弱}
 尺六寸
 \gezhu{
  八分} 夏至
 \gezhu{
  五月中}
 井二十五
 \gezhu{
  半強}
 六十七
 \gezhu{
  強}
 尺五寸小暑
 \gezhu{
  六月
  節}
 柳三
 \gezhu{
  太強}


六十七
 \gezhu{
  太強}
 尺七寸大暑
 \gezhu{
  六
  月中}
 星四
 \gezhu{
  強}
 七十


二尺立秋
 \gezhu{
  七月節}
 張十二
 \gezhu{
  少} 七十三
 \gezhu{
  半強}
 二尺五寸


\gezhu{
  五分}
 處暑
 \gezhu{
  七月中}
 翼九
 \gezhu{
  半}
 七十八
 \gezhu{
  半強} 三尺三寸
 \gezhu{
  三分}
 白露
 \gezhu{
  八月節}
 軫六
 \gezhu{
  太}
 八十四
 \gezhu{
  少強}
 四尺三寸
 \gezhu{
  五
  分}


秋分
 \gezhu{
  八月中}
 角五
 \gezhu{
  弱}
 九十
 \gezhu{
  半強}
 五尺五寸
 寒露
 \gezhu{
  九月節}


亢八
 \gezhu{
  半弱}
 九十六
 \gezhu{
  太強}
 六尺八寸
 \gezhu{
  五分} 霜降
 \gezhu{
  九月中}
 氐十四
 \gezhu{
  少強}
 百二
 \gezhu{
  少強}
 八尺四寸
 立冬
 \gezhu{
  十月節}
 尾四
 \gezhu{
  半強}


百七
 \gezhu{
  少強}
 丈小雪
 \gezhu{
  十
  月中}
 箕一
 \gezhu{
  太強}
 百一十一
 \gezhu{
  弱}


丈一尺四寸大雪
 \gezhu{
  十一月節}
 斗六
 百一十三
 \gezhu{
  太強}
 丈二尺五寸
 \gezhu{
  六分}
 中節晝漏刻夜漏刻昏中星明中星冬至四十五五十五奎六
 \gezhu{
  弱}
 亢二
 \gezhu{
  少強}
 小寒四十五
 \gezhu{
  八分}
 五十四
 \gezhu{
  二分}
 婁六
 \gezhu{
  半強}
 氐七
 \gezhu{
  強}
 大寒四十六
 \gezhu{
  八分}
 五十三
 \gezhu{
  二分}
 胃十一
 \gezhu{
  太強}
 心
 \gezhu{
  半}
 立春四十八
 \gezhu{
  六分}


五十一
 \gezhu{
  四分}
 畢五
 \gezhu{
  少弱}
 尾七
 \gezhu{
  半弱}
 雨水五
 十
 \gezhu{
  八分}
 四十九
 \gezhu{
  二分}
 參六
 \gezhu{
  半弱}
 箕
 \gezhu{
  半弱}
 驚蟄五十三
 \gezhu{
  三分}
 四十六
 \gezhu{
  七分}


井十七
 \gezhu{
  少弱}
 鬥初
 \gezhu{
  少}
 春分五十五
 \gezhu{
  八分}
 四十四
 \gezhu{
  二分}
 鬼四


斗十一
 \gezhu{
  弱}
 清明五十八
 \gezhu{
  三分}
 四十一
 \gezhu{
  七分}
 星四
 \gezhu{
  太}


斗二十一
 \gezhu{
  半}
 穀雨六十
 \gezhu{
  五分}
 三十九
 \gezhu{
  五分}
 張十七牛六
 \gezhu{
  半}
 立夏六十二
 \gezhu{
  四分}
 三十七
 \gezhu{
  六分}
 翼十七
 \gezhu{
  太}
 女十
 \gezhu{
  少弱}


小滿六十三
 \gezhu{
  九分}
 三十六
 \gezhu{
  一分}
 角
 \gezhu{
  太弱}
 危
 \gezhu{
  太弱}
 芒種六十四
 \gezhu{
  九分}
 三十五
 \gezhu{
  一分}
 亢五
 \gezhu{
  太}
 危十四
 \gezhu{
  強}
 夏至六十五三十五氐十二
 \gezhu{
  少弱}
 室十二
 \gezhu{
  強}
 小暑六十四
 \gezhu{
  七分}


三十五
 \gezhu{
  三分}
 尾一
 \gezhu{
  太強}
 奎二
 \gezhu{
  太強}
 大暑六十三
 \gezhu{
  八分}
 三
 十六
 \gezhu{
  二分}
 尾十五
 \gezhu{
  半強}
 婁三
 \gezhu{
  太}
 立秋六十二
 \gezhu{
  三分}
 三十七
 \gezhu{
  七分}


箕九
 \gezhu{
  太強}
 胃九
 \gezhu{
  太弱}
 處暑六十
 \gezhu{
  二分}
 三十九
 \gezhu{
  八分}
 斗十
 \gezhu{
  少}
 畢三
 \gezhu{
  太}
 白露五十七
 \gezhu{
  八分}
 四十二
 \gezhu{
  二分}
 斗二十一
 \gezhu{
  強}


參五
 \gezhu{
  少強}
 秋分五十五
 \gezhu{
  二分}
 四十四
 \gezhu{
  八分}
 牛五
 \gezhu{
  少}
 井十六
 \gezhu{
  少強}
 寒露五十二
 \gezhu{
  六分}
 四十七
 \gezhu{
  四分}
 女七
 \gezhu{
  太}
 鬼三
 \gezhu{
  少強}
 霜降五十
 \gezhu{
  三分}
 四十九
 \gezhu{
  七分}
 虛六
 \gezhu{
  太}
 星三
 \gezhu{
  太}
 立冬四十八
 \gezhu{
  二分}
 五十一
 \gezhu{
  八分}
 危八
 \gezhu{
  強}
 張十五
 \gezhu{
  太強}
 小雪四十六
 \gezhu{
  七分}
 五十三
 \gezhu{
  三分}
 室三
 \gezhu{
  半強}
 翼十五
 \gezhu{
  太}
 大雪四十五
 \gezhu{
  五分}
 五十四
 \gezhu{
  五分}
 壁
 \gezhu{
  半強}
 軫十五
 \gezhu{
  少強}



 右中節二十四氣,如術求之,得冬至十一月中也。加之得次月節,加節得其月中。中星以日所在為正。置所求年二十四氣小餘四之,如法得一為少,不盡少三之,如法為強。所以減其節氣昏明中星各定。



 推五星術:五星者,木曰歲
 星,火曰熒惑,土曰填星,金曰太白,水曰辰星。凡五星之行,有遲有疾,有留有逆。曩自開闢,清濁始分,則日月五星聚於星紀。發自星紀,並而行天,遲疾留逆,互相逮及。星與日會,同宿共度,則謂之合。從合至合之日,則謂之終。各以一終之日與一歲之日,通分相約,終而率之,歲數歲則謂之合終歲數,歲終則謂之合終合數。二率既定,則法數生焉。以章歲乘合數為合月法,以紀法乘合數為日度法,以章月乘歲數為
 合月分,如合月法為合月數,合月之餘為月餘。以通數乘合月數,如日法而一為大餘,以六十去大餘,餘為星合朔大餘。大餘之餘為朔小餘。



 以通數乘月餘,以合月法乘朔小餘,並之,以日法乘合月法除之,所得星合入月日數也。餘以通法約之,為入月日。以朔小餘減日法,餘為朔虛分。以歷斗分乘合數,為
 星度鬥分。木、火、土各以合數減歲數,餘以周天乘之,如日度法而一,所得則行星度數也,餘則度餘。金、水以周天乘歲數,如日度法而一,所得則行星度數也,餘則度餘。



 木:合終歲數,千二百五十五。



 合終合數,千一百四十九。



 合月法,二萬一千八百三十一。



 日度法,二百一十一萬七千六百七。



 合月數,十三。



 月餘,萬一千一百二十二。



 朔大餘,二十三。



 朔小餘,四千九十三。



 入月日,十五。



 日餘,百九十九萬五千六百六十四。



 朔虛分,四百六十六。



 斗分,五十二萬二千七百九十五。



 行星度,三十三。



 度餘,百四十七萬二千八百。



 火:
 合終歲數,五千一百五。



 合終合數,二千三百八十八。



 合月法,四萬五千三百七十二。



 日度法,四百四十萬一千八十四。



 合月數,二十六。



 月餘,二萬三。



 朔大餘,四十七。



 朔小餘,三千六百二十七。



 入
 月日,十三。



 日餘,三百五十八萬五千二百三十。



 朔虛分,九百三十二。



 斗分,百八萬六千五百四十。



 行星度,五十。



 度餘,百四十一萬二千一百五十。



 土:合終歲數,三千九百四十三。



 合終合數,三千八百九。



 合月法,七萬二千三百七十一。



 日度法,七百一萬九千九百八十七。合月數,十二。



 月餘,五萬八千一百五十三。



 朔大餘,五十四。



 朔小餘,千六百七十四。



 入月日,二十四。



 日餘,六十七萬五千三百六十
 四。



 朔虛分,二千八百八十五。



 斗分,百七十三萬三千九十五。



 行星度,十二。



 度餘,五百九十六萬二千二百五十六。



 金:合終歲數,千九百七。



 合終合數,二千三百八十五。



 合月法,四萬五千三百一十五。



 日度法,四百三十九萬五千五百五十五。



 合月數,九。



 月餘,四萬三百一十。



 朔大餘,二十五。



 朔小餘,三千五百三十五。



 入月日,二十七。



 日餘,十九萬四千九百九十。



 朔虛分,千二十四。



 斗分,百八萬五千一百七十五。



 行星度,二百九十二。



 度餘,十九萬四千九百九十。



 水:合終歲數,一千八百七十。



 合科合數,萬一千七百八十九。



 合月法,二十二萬三千九百九十一。



 日度法,二千一百七十二萬七千一百二十七。



 合月數,一。



 月餘,二十一萬五千四百五十九。



 朔大餘,二十九。



 朔小餘,二千四百一十九。



 入月日,二十八。



 日餘,二千三十四萬四千二百六十一。



 朔虛分,二千一百四十。



 斗分,五百三十六萬三千九百九十五。



 行星度,五十七。



 度餘,二千三十四萬四千二百六十一。



 推五星術曰:置壬辰元以來盡所求年,以合終合數乘之,滿合終歲數得一,名積合,不盡名合餘。以合終合數減合餘,得一者星合往年,得二者合前往年,無所得,合其年。餘以減合終合數,為度分。金、水積合,偶為晨,奇為夕。



 推五星合月:以月數月餘各乘積合,餘滿合月法從月,
 為積月,不盡為月餘。



 以紀月除積月,所得算外,所入紀也,餘為入紀月。副以章閏乘之,滿章月得一為閏,以減入紀月,餘以歲中去之,餘為入歲月,命以天正起,算外,星合月也。其在閏交際,以朔御之。



 推合月朔:以通數乘入紀月,滿日法得一為積日,不盡為小餘。以六十去積日,餘為大餘,命以所入紀,算外,星合朔日也。推入月日:以通數乘月餘,合月法乘朔小餘,并之,通法
 約之,所得滿日度法得一,則星合入月日也,不滿為日餘。命日以朔,算外,入月日也。



 推星合度:以周天乘度分,滿日度法得一為度,不盡為餘,命以牛前五度起,算外,星所合度也。



 求後合月,以月數加入歲月,以餘加月餘,餘滿合月法得一月,月不滿歲中,即在其年;滿去之,有閏計焉,餘為後年;再滿,在後二年。金、水加晨得夕,加夕得晨也。求後合朔,以朔大小餘數加合朔月大小餘,其月餘上
 成月者,又加大餘二十九,小餘一千四百一十九,小餘滿日法從大餘,命如前法。求後入月日,以入月日、日餘加入月日及餘,餘滿日度法得一。其前合朔小餘滿其虛分者,去一日;後小餘滿二千四百一十九以上,去二十九日;不滿,去三十日,其餘則後合入月日,命以朔。求後合度,以度數及分,如前合宿次命之。



 木:晨與日合,伏,順,十六日九十九萬七千八百三十二
 分,行星二度百七十九萬五千二百三十八分,而晨見東方,在日後。順,疾,日行五十七分之十一,五十七日行十一度。順,遲,日行九分,五十七日行九度而留。不行,二十七日而旋。



 逆,日行七分之一,八十四日退十二度,而復留二十七日。復遲,日行九分,五十七日行九度而復順。疾,日行十一分,五十七日行十一度,在曰前,夕伏西方。順,十六日九十九萬七千八百三十二分,行星二度百七十九萬五千二百三十八分,而與日合。凡一終,三
 百九十八日百九十九萬五千六百六十四分,行星三十三度百四十七萬二千八百六十九分。



 火:晨與日合,伏,七十二日百七十九萬二千六百一十五分,行星五十六度百二十四萬九千三百四十五分,而晨見東方,在日後。順,日行二十三分之十四,百八十四日行百一十二度。更順,遲,日行十二分,九十二日行四十八度而留。不行,十一日而旋。逆,日行六十二分之十七,六十二日退十七度,而復留十一日。復順,遲,日行
 十二分,九十二日,行四十八度而復疾。日行十四分,百八十四日行百一十二度,在日前,夕伏西方。順,七十二日百七十九萬二千六百一十五分,行星五十六度百二十四萬九千三百四十五分,而與日合。凡一終,七百八十日三百五十八萬五千二百三十分,行星四百一十五度二百四十九萬八千六百九十分。



 土:晨與日合,伏,十九日三百八十四萬七千六百七十五分半,行星二度六百四十九萬一千一百二十一分
 半,而晨見東方,在日後。順,行百七十二分之十三,八十六日行六度半而留。不行,三十二日半而旋。逆,日行十七分之一,百二日退六度而復留。不行,三十二日半復順,日行十三分,八十六日行六度半,在日前,夕伏西方。順,十九日三百八十四萬七千六百七十五分半,行星二度六百四十九萬一千一百二十一分半,而與日合。凡一終,三百七十八日六十七萬五千三百六十四分,行星十二度五百九十六萬二千二百五十六分。



 金:晨與日合,伏,六日退四度,而晨見東方,在日後而逆。遲,日行五分之三,十日退六度。留,不行,七日而旋。順,遲,日行四十五分之三十三,四十五日行三十三度而順。疾,日行一度九十一分之十四,九十一日行百五度而順。益疾,日行一度九十一分之二十一,九十一日行百一十二度,在日後,而晨伏東方。順,四十二日十九萬四千九百九十分,行星五十二度十九萬四千九百九十分,而與日合。



 一合,二百九十二日十九萬四千九百九
 十分,行星如之。



 金:夕與日合,伏,順,四十二日十九萬四千九百九十分,行星五十二度十九萬四千九百九十分,而夕見西方,在日前。順,疾,日行一度九十一分之二十一,九十一日行百一十二度而更順。遲,日行一度十四分,九十一日行百五度而順。益遲,日行四十五分之三十三,四十五日行三十三度而留。不行,七日而旋。逆,日行五分之三,十日退六度,在日前,夕伏西方。逆,六日,退四度,而與日
 合。凡再合一終,五百八十四日三十八萬九千九百八十分,行星如之。



 水:晨與日合,伏,十一日退七度,而晨見東方,在日後。逆,疾,一日退一度而留。不行,一日而旋。順,遲,日行八分之七,八日行七度而順。疾,日行一度十八分之四,十八日行二十二度,在日後,晨伏東方。順,十八日二千三十四萬四千二百六十一分,行星三十六度二千三十四萬四千二百六十一分,而與日合。凡一合,五十七日二千
 三十四萬四千二百六十一分,行星如之。



 水:夕與日合,伏,十八日二千三十四萬四千二百六十一分,行星三十六度二千三十四萬四千二百六十一分,而夕見西方,在日前。順,疾,日行一度十八分之四,十八日行二十二度而更順。遲,日行八分之七,八日行七度而留。不行,一日而旋。逆,一日退一度,在日前,夕伏西方。逆,十一日退七度,而與日合。凡再合一終,百一十五日千八百九十六萬一千三百九十五分,行星如之。



 五星歷步術:以法伏日度餘,加星合日度餘,餘滿日度法得一從全,命之如前,得星見日及度餘也。以星行分母乘見度分,如日度法得一,分不盡,半法以上,亦得一,而日加所行分,分滿其母得一度。逆順母不同,以當行之母乘故分,如故母而一,當行分也。留者承前,逆則減之,伏不書度,除斗分,以行母為率。分有損益,前後相御。



 凡五星行天,遲疾留逆,雖大率有常,至犯守逆順,難以術推。月之行天,猶有遲疾,況五星乎!唯日之行天有常,
 進退有率,不遲不疾,不外不內,人君德也。



 求木合終歲數法,以木日度法乘一木終之日,內分,周天除之,即得也。求木合終合數法,以木日度法乘周天,滿紀法,所得復以周天除之,即得。五星皆放此也。



 魏黃初元年十一月小,己卯蔀首,己亥歲,十一月己卯朔旦冬至,臣偉上。」



 劉氏在蜀,不見改歷,當是仍用漢《四分法》。吳中書令闞
 澤受劉洪《乾象法》於東萊徐岳字公河。故孫氏用《乾象歷》,至於吳亡。



 晉武帝泰始元年,有司奏:「王者祖氣而奉其囗終,晉於五行之次應尚金,金生於己,事於酉,終於丑,宜祖以酉日,臘以丑日。改《景初歷》為《泰始歷》。」



 奏可。



 史臣按,鄒衍五德,周為火行。衍生在周時,不容不知周氏行運。且周之為歷年八百,秦氏即有周之建國也。周之火木,其事易詳。且五德更王,唯有二家之說。



 鄒衍以
 相勝立體,劉向相生為義。據以為言,不得出此二家者。假使即劉向之說,周為木行,秦氏代周,改其行運。若不相勝,則克木者金;相生則木實生火。秦氏乃稱水德,理非謬然,斯則劉氏所證為不值矣。臣以為張蒼雖是漢臣,生與周接,司秦柱下,備睹圖書。且秦雖滅學,不廢術數,則有周遺文雖不畢在,據漢水行,事非虛作。賈誼《取秦》云:「漢土德。」蓋以是漢代秦。詳論二說,各有其義。



 張蒼則以漢水勝周火,廢秦不班五德。賈誼則以漢土勝
 秦水,以秦為一代。論秦、漢雖殊,而周為火一也。然則相勝之義,於事為長。若同蒼黜秦,則漢水、魏土、晉木、宋金;若同賈誼《取秦》,則漢土、魏木、晉金、宋火也。難者云:「漢高斷蛇而神母夜哭,云赤帝子殺白帝子,然則漢非火而何?」斯又不然矣。漢若為火,則當云赤帝,不宜云赤帝子也。白帝子又何義況乎?蓋由漢是土德,土生乎火,秦是水德,水生乎金,斯則漢以土為赤帝子,秦以水德為白帝子也。難者又曰:「向云五德相勝,今復云土為赤帝子,
 何也?」答曰:「五行自有相勝之義,自有相生之義。不得以相勝廢相生,相生廢相勝也。相勝者,以土勝水耳;相生者,土自火子,義豈相關。」



 崔寔《四人月令》曰:祖者,道神。黃帝之子曰累祖,好遠遊,死道路,故祀以為道神。合《祖賦序》曰:漢用丙午,魏用丁未,晉用孟月之酉。曰莫識祖之所由。說者云祈請道神,謂之祖有事於道者,君子行役,則列之於中路,喪者將遷,則稱名於階庭。或云,百代遠祖,名謚凋滅,墳塋不復存於銘表,游魂不得託於廟祧,
 故以初歲良辰,建華蓋,揚彩旌,將以招靈爽,庶眾祖之來憑云爾。



 晉江左時,侍中平原劉智,推三百年斗歷改憲,以為《四分法》三百年而減一日,以百五十為度法,三十七為斗分。飾以浮說,以扶其理。江左中領軍琅邪王朔之以其上元歲在甲子,善其術,欲以九萬七千歲之甲子為開闢之始,何承天云「悼於立意」者也。《景初》日中晷景,即用漢《四分法》,是以漸就乖差。其推五星,則甚疏闊。晉江左
 以來,更用《乾象五星法》以代之,猶有前卻。



 宋太祖頗好歷數,太子率更令何承天私撰新法。元嘉二十年,上表曰:臣授性頑惰,少所關解。自昔幼年,頗好歷數,耽情注意,迄於白首。臣亡舅故秘書監徐廣,素善其事,有既往《七曜歷》,每記其得失。自太和至泰元之末,四十許年。臣因比歲考校,至今又四十載。故其疏密差會,皆可知也。



 夫圓極常動,七曜運行,離合去來,雖有定勢,以新故相涉,自然有毫末之差,連日累歲,積微成著。
 是以《虞書》著欽若之典,《周易》明治歷之訓,言當順天以求合,非為合以驗天也。漢代雜候清臺,以昏明中星,課日所在,雖不可見,月盈則蝕,必當其衝,以月推日,則躔次可知焉。捨易而不為,役心於難事,此臣所不解也。



 《堯典》云:「日永星火,以正仲夏」。今季夏則火中。又「宵中星虛,以殷仲秋」。今季秋則虛中。爾來二千七百餘年,以中星檢之,所差二十七八度。則堯冬令至,日在須女十度左右也。漢之《太初歷》,冬至在牽牛初,後漢《四分》及魏《景初
 法》,同在斗二十一。臣以月蝕檢之,則《景初》今之冬至,應在斗十七。



 又史官受詔,以土圭測景,考校二至,差三日有餘。從來積歲及交州所上,檢其增減,亦相符驗。然則今之二至,非天之二至也。天之南至,日在斗十三四矣。此則十九年七閏,數微多差。復改法易章,則用算滋繁,宜當隨時遷革,以取其合。案《後漢志》,春分日長,秋分日短,差過半刻。尋二分在二至之間,而有長短,因識春分近夏至,故長;秋分近冬至,故短也。楊偉不悟,即用之,上歷
 表云:「自古及今,凡諸歷數,皆未能並己之妙。」何此不曉,亦何以云。是故臣更建《元嘉歷》,以六百八為一紀,半之為度法,七十五為室分,以建寅之月為歲首,雨水為氣初,以諸法閏餘一之歲為章首。冬至從上三日五時。日之所在,移舊四度。又月有遲疾,合朔月蝕,不在朔望,亦非歷意也。故元嘉皆以盈縮定其小餘,以正朔望之日。



 伏惟陛下允迪聖哲,先天不違,劬勞庶政,寅亮鴻業,究淵思於往籍,探妙旨於未聞,窮神知化,罔不該覽。是以
 愚臣欣遇盛明,效其管穴。伏願以臣所上《元嘉法》下史官考其疏密,若謬有可採,庶或補正闕謬,以備萬分。



 詔曰:「何承天所陳,殊有理據。可付外詳之。」太史令錢樂之、兼丞嚴粲奏曰:太子率更令領國子博士何承天表更改《元嘉歷法》,以月蝕檢今冬至日在斗十七,以土圭測影,知冬至已差三日。詔使付外檢署。以元嘉十一年被敕,使考月蝕,土圭測影,檢署由來用偉《景初法》,冬至之日,日在斗二十一度少。檢十一年七月十六日望月蝕,
 加時在卯,到十五日四更二唱丑初始蝕,到四唱蝕既,在營室十五度末。《景初》其日日在軫三度。以月蝕所衝考之,其日日應在翼十五度半。又到十三年十二月十六日望月蝕,加時在酉,到亥初始食,到一更三唱蝕既,在鬼四度。《景初》其日日在女三。以衝考之,其日日應在牛六度半。又到十四年十二月十六日望月蝕,加時在戌之半,到二更四唱亥末始蝕,到三更一唱食既,在井三十八度。《景初》其日日在斗二十五。以衝考之,其日日
 應在斗二十二度半。到十五年五月十五日望月蝕,加時在戌,其日月始生而已,蝕光已生四分之一格,在斗十六度許。《景初》其日日在井二十四。考取其衝,其日日應在井二十。又到十七年九月十六日望月蝕,加時在子之少,到十五日未二更一唱始蝕,到三唱蝕十五分之十二格,在昴一度半。《景初》其日在房二。以衝考之,則其日日在氐十三度半。



 凡此五蝕。以月衝一百八十二度半考之,冬至之日,日並不在斗二十一度少,並在斗
 十七度半間,悉如承天所上。



 又去十一年起,以土圭測影。其年《景初法》十一月七日冬至,前後陰不見影。



 到十二年十一月十八日冬至,其十五日影極長。到十三年十一月二十九日冬至,其二十六日影極長。到十四年十一月十一日冬至,其前後並陰不見。到十五年十一月二十一日冬至,十八日影極長。到十六年十一月二日冬至,其十月二十九日影極長。



 到十七年十一月十三日冬至,其十日影極長。到十八年十一月二十五日
 冬至,二十一日影極長。到十九年十一月六日冬至,其三日影極長。到二十年十一月十六日冬至,其前後陰不見影。尋校前後,以影極長為冬至,並差三日。以月蝕檢日所在,已差四度。土圭測影,冬至又差三日。今之冬至,乃在斗十四間,又如承天所上。



 又承天法,每月朔望及弦,皆定大小餘,於推交會時刻雖審,皆用盈縮,則月有頻三大、頻二小,比舊法殊為異。舊日蝕不唯在朔,亦有在晦及二日。《公羊傳》所謂「或失之前,或失之後」。愚謂
 此一條自宜仍舊。



 員外散騎郎皮延宗又難承天:「若晦朔定大小餘,紀首值盈,則退一日,便應以故歲之晦,為新紀之首。」承天乃改新法依舊術,不復每月定大小餘,如延宗所難,太史所上。



 有司奏:「治歷改憲,經國盛典,爰及漢、魏,屢有變革。良由術無常是,取協當時。方今皇猷載暉,舊域光被,誠應綜核晷度,以播維新。承天歷術,合可施用。宋二十二年,普用《元嘉歷》。」詔可。



\end{pinyinscope}