\article{卷二十二志第十二 樂四}

\begin{pinyinscope}

 漢《鼙舞歌》五篇:《關東有賢女》、《章和二年中》、《
 樂久長》、《四方皇》、《殿前生桂樹》。



 魏《鼙舞歌》五篇:《明明魏皇帝》、《太和有聖帝》、《魏歷長》、《天生烝民》、《
 為君既不易》。



 魏陳思王《鼙舞歌》五篇:《聖皇篇》,當《章和二年中》:聖皇應歷數,正康帝道休。九州咸賓服,威德洞八幽。三公奏諸公,不得久淹留。蕃位任至重,舊章咸率由。侍臣省文奏,陛下體仁慈。
 沉吟有愛戀,不忍聽可之。迫有官典憲,不得顧恩私。諸王當就國,璽綬何纍縗。便時舍外殿,宮省寂無人。主上增顧念,皇母懷苦辛。何以為贈賜,傾府竭寶珍。文錢百億萬,采帛若煙雲。乘輿服御物,錦羅與金銀。
 龍旗垂九旒,羽蓋參斑輪。諸王自計念,無功荷厚德。思一效筋力,糜軀以報國。鴻臚擁節衛,副使隨經營。貴戚並出送,夾道交輜軿。車服齊整設,鞾曄耀天精。武騎衛前後,鼓吹簫笳聲。祖道魏東門,淚下霑冠纓。
 扳蓋因內顧,俯仰慕同生。行行將日莫,何時還闕庭。車輪為裴回,四馬躊躇鳴。路人尚酸鼻,何況骨肉情。



 《靈芝篇》,當《殿前生桂樹》:靈芝生玉地,硃草被洛濱。榮華相晃耀,光采曄若神。古時有虞舜,父母頑且嚚。
 盡孝於田隴,烝烝不違仁。伯瑜年七十,采衣以娛親,慈母笞不痛,歔欷涕沾巾。丁蘭少失母,自傷蚤孤煢,刻木當嚴親,朝夕致三牲。暴子見陵侮,犯罪以亡形,丈人為泣血,免戾全其名。董永遭家貧,父老財無遺。
 舉假以供養,傭作致甘肥。責家填門至,不知何用歸。天靈感至德,神女為秉機。歲月不安居,烏乎我皇考!生我既已晚,棄我何期蚤!《蓼莪》誰所興,念之令人老。退詠《南風》詩,灑淚滿褘抱。亂曰:
 聖皇君四海,德教朝夕宣。萬國咸禮讓,百姓家肅虔。庠序不失儀,孝悌處中田。戶有曾閔子,比屋皆仁賢。髫齔無夭齒,黃髮盡其年。陛下三萬歲,慈母亦復然。



 《大魏篇》,當《漢吉昌》:大魏應靈符,天祿方甫始。
 聖德致泰和,神明為驅使。左右宜供養,中殿宜皇子。陛下長壽考,群臣拜賀咸說喜。積善有餘慶,榮祿固天常。眾善填門至,臣子蒙福祥。無患及陽遂,輔翼我聖皇。眾吉咸集會,凶邪姦惡並滅亡。黃鵠游殿前,神鼎周四阿。
 玉馬充乘輿,芝蓋樹九華。白虎戲西除,舍利從辟邪。騏驎躡足舞,鳳凰拊翼歌。豐年大置酒,玉尊列廣庭。樂飲過三爵,硃顏暴己形。式宴不違禮,君臣歌《鹿鳴》。樂人舞鼙鼓,百官雷抃贊若驚。儲禮如江海,積善若陵山。
 皇嗣繁且熾,孫子列曾玄。群臣咸稱萬歲,陛下長樂壽年!



 御酒停未飲,貴戚跪東廂。侍人承顏色,奉進金玉觴。此酒亦真酒,福祿當聖皇。



 陛下臨軒笑,左右咸歡康。杯來一何遲,群僚以次行。賞賜累千億,百官並富昌。



 《
 精微篇》,當《關東有賢女》:精微爛金石,至心動神明。杞妻哭死夫,梁山為之傾。子丹西質秦,烏白馬角生。鄒羨囚燕市,繁霜為夏零。關東有賢女,自字蘇來卿。壯年報父仇,身沒垂功名。女休逢赦書,白刃幾在頸。
 俱上列仙籍,去死獨就生。太倉令有罪,遠征當就拘。自悲居無男,禍至無與俱。緹縈痛父言,何擔西上書。槃桓北闕下,泣淚何漣如。乞得并姊弟,沒身贖父軀。漢文感其義,肉刑法用除。其父得以免,辨義在列圖。
 多男亦何為,一女足成居。簡子南渡河,津吏廢舟船。執法將加刑,女娟擁櫂前。「妾父聞君來,將涉不測淵。畏懼風波起,禱祝祭名川。備禮饗神祗,為君求福先。不勝釂祀誠,至令犯罰艱。君必欲加誅,乞使知罪愆。
 妾願以身代」,至誠感蒼天。國君高其義,其父用赦原。河激奏中流,簡子知其賢。歸娉為夫人,榮寵超後先。辯女解父命,何況健少年。黃初發和氣,明堂德教施。治道致太平,禮樂風俗移。刑錯民無枉,怨女復何為。
 聖皇長壽考,景福常來儀。



 《孟冬篇》,當《狡兔》:孟冬十月,陰氣厲清。武官誡田,講旅統兵。元龜襲吉,元光著明。蚩尤蹕路,風弭雨停。乘輿啟行,鸞鳴幽軋。虎賁採騎,飛象珥鶡。鐘鼓鏗鏘,簫管嘈喝。萬騎齊鑣,千乘等蓋。夷山填谷,平林滌藪。
 張羅萬里,盡其飛走。翟翟狡兔,揚白跳翰。獵以青骹,掩以修竿。韓盧宋鵲,呈才騁足。噬不盡厓,牽麋掎鹿。魏氏發機,養基撫弦。都盧尋高,搜索猴猨。慶忌孟賁,蹈谷超巒。張目決眥,髮怒穿冠。頓熊扼虎,蹴豹搏貙。氣有餘勢,負象而趨。獲車既盈,日側樂終。
 罷役解徒,大饗離宮。亂曰:聖皇臨飛軒,論功校獵徒。死禽積如京,流血成溝渠。明詔大勞賜,大官供有無。走馬行酒醴,驅車布肉魚。鳴鼓舉觴爵,鐘擊位無餘。絕網縱麟麑,弛罩出鳳雛。
 收功在羽校,威靈振鬼區。陛下長懽樂,永世合天符。



 《晉鼙舞歌》五篇:《洪業篇》、《鼙舞歌》,當魏曲《明明魏皇帝》,古曲《關東有賢女》:宣文創洪業,盛德在太始。聖皇應靈符,受命君四海。萬國何所樂,上有明天子。
 唐堯禪帝位,虞舜惟恭己。恭己正南面,道化與時移。大赦盪萌漸,文教被黃支。象天則地,體無為。聰明配日月,神聖參兩儀。雖有三凶類,靜言無所施。象天則地,體無為。稷契並佐命,伊呂升王臣。
 蘭芷登朝肆,下無失宿民。聲發響自應,表立景來附。虓虎從羈制,潛龍升天路。備物立成器,變通極其數。百事以時敘,萬機有常度。訓之以克讓,納之以忠恕。群下仰清風,海外同懽慕。象天則地,化雲布。



 昔日貴雕飾,今尚儉與素。昔日多纖介,今去情與故。象天則地,化雲布。濟濟大朝士,夙夜綜萬機。萬機無廢理,明明降疇咨。臣譬列星景,君配朝日暉。事業並通濟,功烈何巍巍。五帝繼三皇,三王世所歸。
 聖德應期運,天地不能違。仰之彌已高,猶天不可階。將復御龍氏,鳳皇在庭棲。



 《天命篇》、《鼙舞歌》,當魏曲《太和有聖帝》,古曲《章和二年中》:聖祖受天命,應期輔魏皇。入則綜萬機,出則徵四方。朝廷無遺理,方表寧且康。
 道隆舜臣堯,積德踰太王。孟度阻窮險,造亂天一隅。神兵出不意,奉命致天誅。赦善戮有罪,元惡宗為虛。威風震勁蜀,武烈懾彊吳。諸葛不知命,肆逆亂天常。擁徒十餘萬,數來寇邊疆。我皇邁神武,秉鉞鎮雍涼。
 亮乃畏天威,未戰先仆僵。盈虛自然運,時變固多難。東征陵海表,萬里梟賊淵。受遺齊七政,曹爽又滔天。群兇受誅殛,百祿咸來臻。黃華應福始,王凌為禍先。



 《景皇帝》、《鼙舞歌》,當魏曲《魏歷長》,古曲《樂久長》:景皇帝,聰明命世生,盛德參天地。
 帝王道,創基既已難,繼世亦未易。外則夏侯玄,內則張與李,三凶稱逆,亂帝紀。從天行誅,窮其姦宄。遏將御其漸,潛謀不得起。罪人咸伏辜,威風震萬里。平衡綜萬機,萬機無不理。召陵桓不君,內外何紛紛,眾小便成群。蒙昧恣心,
 治亂不分。睿聖獨斷,濟武常以文。從天惟廢立,掃霓披浮雲。



 雲霓既已闢,清和未幾間。羽檄首尾至,變起東南蕃。儉欽為長蛇,外則馮吳蠻。



 萬國紛騷擾,戚戚天下懼不安。神武御六軍,我皇秉鉞征。儉欽起壽春,
 前鋒據項城。出其不意,並縱奇兵。奇兵誠難御,廟勝實難支。兩軍不期遇,敵退計無施。



 虎騎惟武進,大戰沙陽陂。欽乃亡魂走,奔虜若雲披。天恩赦有罪,東土放鯨鯢。



 《大晉篇》、《鼙舞歌》,當魏曲《天生烝民》,古曲《四方皇》:赫赫大晉,於穆文皇。蕩蕩巍巍,
 道邁陶唐。世稱三皇五帝,及今重其光。九德克明,文既顯,武又章。恩弘六合,兼濟萬方。內舉元凱,朝政以綱。外簡虎臣,時惟鷹揚。靡從不懷,逆命斯亡。仁配春日,威踰秋霜。濟濟多士,同茲蘭芳。唐虞至治,四凶滔天。致討儉欽,罔不肅虔。化感海外,海外來賓。獻其聲樂,
 並稱妾臣。西蜀猾夏,僭號方域。命將致討,委國稽服。吳人放命,馮海阻江。飛書告諭,響應來同。先王建萬國,九服為蕃衛。亡秦壞諸侯,序胙不二世。歷代不能復,忽踰五百歲。我皇邁聖德,應期創典制。分土五等,蕃國正封界。莘莘文武佐,千秋遘嘉會。洪業溢區內,仁風翔海外。



 《
 明君篇》、《鼙舞歌》,當魏曲《為君既不易》,古曲《殿前生桂樹》:明君御四海,聽鑑盡物情。顧望有譴罰,竭忠身必榮。蘭茝出荒野,萬里升紫庭。茨草穢堂階,掃截不得生。能否莫相蒙,百官正其名。恭己慎有為,有為無不成。暗君不自信,群下執異端。正直罹譖潤,姦臣奪其權。雖欲盡忠誠,結舌不敢言。
 結舌亦何憚,盡忠為身患。清流豈不潔,飛塵濁其源。歧路令人迷,未遠勝不還。忠臣立君朝,正色不顧身。邪正不並存,譬若胡與秦。秦胡有合時,邪正各異津。忠臣遇明君,乾乾惟日新。群目統在綱,眾星拱北辰。設令遭闇主,斥退為凡民。雖薄供時用,白茅猶可珍。冰霜晝夜結,蘭桂摧為薪。邪臣多端變,用心何委曲。
 便辟從情指,動隨君所欲。偷安樂目前,不問清與濁。積偽罔時主,養交以持祿。言行恒相違,難饜甚溪谷。昧死射乾沒,覺露則滅族。右五篇《鼙舞歌行》。



 《鐸舞》歌詩二篇。《聖人制禮樂篇》:昔皇文武邪彌彌舍善誰吾時吾
 行許帝道銜來治路萬邪治路萬邪赫赫意黃運道吾治路萬邪善道明邪金邪善道明邪金邪帝邪近帝武武邪邪聖皇八音偶邪尊來聖皇八音及來義邪同邪烏及來義邪善草供國吾咄等邪烏近帝邪武邪近帝武邪武邪應節合用武邪尊邪應節合用酒期義邪同邪酒期義邪
 善草供國吾咄等邪烏近帝邪武邪近帝武武邪邪下音足木上為鼓義邪應眾義邪樂邪邪延否已邪烏已禮祥咄等邪烏素女有絕其聖烏烏武邪《雲門篇》、《鐸舞歌行》,當魏《太和時》:黃《雲門》,唐《咸池》,虞《韶舞》,夏殷《濩》。列代有五,振鐸鳴金,近《大武》。清歌發倡,刑為主。



 聲和八音,協律呂。身不虛動,手不徒舉。應節合度,周其敘。時奏宮商,雜之以徵羽。下饜眾目,上從鐘鼓。樂以移風,與德禮相輔,安有失其所。



 右二篇《鐸舞歌行》。



 《拂舞》歌詩五篇:《白鳩篇》:
 翩翩白鳩,再飛再鳴。懷我君德,來集君庭。白雀呈瑞,素羽明鮮。翔庭舞翼,以應仁乾。交交鳴鳩,或丹或黃。樂我君惠,振羽來翔。東壁餘光,魚在江湖。惠而不費,敬我微軀。策我良駟,習我驅馳。與君周旋,樂道亡餘。我心虛靜,我志霑濡。彈琴鼓瑟,聊以自娛。
 陵雲登臺,浮游太清。扳龍附鳳,日望身輕。



 《濟濟篇》:暢飛暢舞,氣流芳。追念三五,大綺黃。去失有,時可行。去來同時,此未央。時冉冉,近桑榆。但當飲酒,為歡娛。衰老逝,有何期。多憂耿耿,
 內懷思。淵池廣,魚獨希。願得黃浦,眾所依。恩感人,世無比。悲歌具舞,無極已。



 《獨祿篇》:獨祿獨祿,水深泥濁。泥濁尚可,水深殺我。雍雍雙鴈,游戲田畔。我欲射鴈,念子孤散。翩翩浮蘋,得風遙輕。我心何合,與之同并。
 空床低帷,誰知無人。夜衣錦繡,誰別偽真。刀嗚削中,倚床無施。父冤不報,欲活何為。猛虎班班,游戲山間。虎欲嚙人,不避豪賢。



 《碣石篇》:東臨碣石,以觀滄海。水何澹澹,山島竦峙。樹木叢生,百草豐茂。秋風蕭瑟,洪波湧起。日月之行,
 若出其中。星漢粲爛,若出其里。幸甚至哉!



 歌以詠志。



 《觀滄海》。



 孟冬十月,北風裴回。天氣肅清,繁霜霏霏。鵾雞晨鳴,鴈過南飛。鷙烏潛藏,熊羆窟棲。錢褲停置,農收積場。逆旅整設,以通賈商。幸甚至哉!歌以詠志。



 《冬十月》。



 鄉土不同,河朔隆寒。流澌浮漂,
 舟船行難。錐不入地,豐籟深奧,水竭不流,冰堅可蹈。士隱者貧,勇俠輕非。心常歎怨,戚戚多悲。幸甚至哉!歌以詠志。



 《土不同》。



 神龜雖壽,猶有竟時;騰蛇乘霧,終為土灰。老驥伏櫪,志在千里;烈士莫年,壯心不已。盈縮之期,不但在天;養怡之福,可得永年。
 幸甚至哉!歌以詠志。



 《龜雖壽》。



 《淮南王篇》:淮南王,自言尊,百尺高樓與天連。後園鑿井銀作床,金瓶素綆汲寒漿。汲寒漿,飲少年。少年窈窕何能賢?揚聲悲歌音絕天。我欲度河河無梁,願化雙黃鵠,還故鄉。還故鄉,入故里。徘徊故鄉,苦身不已。繁舞寄聲無不泰,徘徊桑梓遊天
 外。



 右五篇《拂舞行》。



 《杯槃舞》歌詩一篇:晉世寧,四海平,普天安樂永大寧。四海安,天下歡,樂治興隆舞杯盤。舞杯槃,何翩翩,舉坐翻覆壽萬年。天與日,終與一,左回右轉不相失。箏笛悲,酒舞疲,心中慷慨可健兒。樽酒甘,絲竹清,願令諸君醉復醒。



 醉復醒,時合同,四坐歡樂皆言工。絲竹音,可不聽,亦舞此槃左右輕。自相當,合坐歡樂人命長。人命長,當結友,千秋萬歲皆老壽。



 右《杯槃舞歌行》。



 《巾舞》歌詩一篇:吾不見公莫時吾何嬰公來嬰姥時吾哺聲何為茂時為來嬰當思吾明月之上轉起吾何嬰土來嬰轉去吾
 哺聲何為土轉南來嬰當去吾城上羊下食草吾何嬰下來吾食草吾哺聲汝何三年針縮何來嬰吾亦老吾平平門淫涕下吾何嬰何來嬰涕下吾哺聲昔結吾馬客來嬰吾當行吾度四州洛四海吾何嬰海何來嬰海何來嬰四海吾哺聲熇西馬頭香來嬰吾洛道吾治五丈度汲水吾噫邪哺誰當求兒母何意零邪錢健步哺誰當吾求兒母何吾哺聲三針一發交時還弩心意何零意弩心遙來嬰弩心哺聲復相頭巾意何零何邪相
 哺頭巾相吾來嬰頭巾母何何吾復來推排意何零相哺推相來嬰推非母何吾復車輪意何零子以邪相哺轉輪吾來嬰轉母何吾使君去時意何零子以邪使君去時使來嬰去時母何吾思君去時意何零子以邪思君去時思來嬰吾去時母何何吾吾右《公莫巾舞歌行》。



 《白珝舞》歌詩三篇:高舉兩手白鵠翔。輕軀徐起何洋洋。
 凝停善睞容儀光。



 宛若龍轉乍低昂。隨世而變誠無方。如推若引留且行。宋世方昌樂未央。舞以盡神安可忘。愛之遺誰贈佳人。質如輕雲色如銀。袍以光軀巾拂塵。制以為袍餘作巾。



 四坐歡樂胡可陳。清歌徐舞降祗神。



 右一篇。



 雙袂齊舉鸞鳳翔。羅裾飄濆昭儀光。
 趨步生姿進流芳。鳴弦清歌及三陽。人生世間如電過。樂時每少苦日多。幸及良辰曜春花。齊倡獻舞趙女歌。羲和馳景逝不停。春露未晞嚴霜零。百草凋索花落英。蟋蟀吟牖寒蟬鳴。百年之命忽若傾。蚤知迅速秉燭行。東造扶桑游紫庭。西至崑崙戲曾城。



 右一篇。



 陽春白日風花香。趨步明玉舞瑤璫。聲發金石媚笙簧。羅袿徐轉紅袖揚。清歌流響繞鳳梁。如矜若思凝且翔。轉盼遺精艷輝光。將流將引雙鴈翔。歡來何晚意何長。明君御世永歌倡。



 右一篇。《白珝》舊新合三篇。


宋泰始歌舞曲詞:《皇業頌》
 \gezhu{
  歌自堯至楚元王、高祖,世世載聖德。}
 明帝造:
 皇業沿德建,帝運資勳融。胤唐重盛軌,胄楚載休風。堯帝兆深祥,元王衍遐慶。積善傳上業,祚福啟英聖。衰數隨金祿,登歷昌水命。維宋垂光烈,世美流舞詠。



 《聖祖頌》:聖祖惟高德,積勛代晉歷。
 永建享鴻基,萬古盛音冊。睿文纘宸馭,廣運崇帝聲。衍德被仁祉,留化洽民靈。孝建締孝業,允協天人謀。宇內齊政軌,宙表燭威流。鐘管騰列聖,彞銘賁重猷。



 《明君大雅》,虞龢造:明君應乾數,撥亂紐頹基。
 民慶來蘇日,國頌《薰風》詩。天步或暫難,列蕃扇迷慝。廟勝敷九代,神謨洞七德。文教洗昏俗,武誼清昆埏。英勛冠帝則,萬壽永衍天。



 《通國風》,明帝造:開寶業,資賢昌,謨明盛,弼諧光。烈武惟略,景王勳。



 南康華容,變政文。猛績爰著,有左軍。三王到氏,文武贊。丞相作輔,屬伊旦。



 沈柳宗侯,皆殄亂。泰始開運,超百王。司徒驃騎,勳德康。江安謀效,殷誠彰。



 劉沈承規,功名揚。慶歸我后,祚無疆。



 《天符頌》,明帝造:
 天符革運,世誕英皇。在館神炫,既壯龍驤。六鐘集表,四緯駢光。於穆配天,永休厥祥。



 《明德頌》,明帝造:明德孚教,幽符麗紀。山鼎見奇,醴液涵祉。鵷雛耀儀,騶虞游趾。福延億祚,慶流萬祀。



 《帝圖頌》:
 帝圖凝遠,瑞美昭宣。濟流月鏡,鹿毳霜鮮。甘露降和,花雪表年。孝德載衍,芳風永傳。



 《龍躍大雅》:龍躍式符,玉耀蕃宮。歲淹豫野,璽屬嬪中。江波澈映,石柏開文。觀毓花蕊,樓凝景雲。白烏三獲,甘液再呈。嘉穟表沃,連理協成。
 德充動物,道積通神。宋業允大,靈瑞方臻。



 《淮祥風》:淮祥應,賢彥生。翼贊中興,致太平。



 《宋世大雅》,虞龢造:宋世寧,在太始。醉酒歡,飽德喜。萬國朝,上壽酒。帝同天,惟長久。



 《
 治兵大雅》,明帝造:王命治兵,有征無戰。巾拂以凈,醜類革面。王儀振旅,載戢在辰。中虛巾拂,四表靜塵。



 《白珝篇大雅》,明帝造:在心曰志發言詩,聲成于文被管絲。手舞足蹈欣泰時,移風易俗王化基。琴角揮韻白雲舒,《簫韶》協音神鳳來。
 拊擊和節詠在初,章曲乍畢情有餘。文同軌壹道德行,國靖民和禮樂成。四縣庭響美勛英,八列陛倡貴人聲。舞飾麗華樂容工,羅裳皎日袂隨風。金翠列輝蕙麝豐,淑姿委體允帝衷。


漢鼓吹鐃歌十八曲,《硃鷺曲》:硃鷺,魚以烏路訾邪。鷺何食,食茄下。不之食,不以吐,將以問誅
 \gezhu{
  一作諫}
 者。


《
 思悲翁曲》:思悲翁,唐思,奪我美人侵以遇,悲翁也,但我思。蓬首
 \gezhu{
  一作蕞}
 狗,逐狡兔,食交君,梟子五。梟母六,拉沓高飛莫安宿。



 《艾如張曲》:艾而張羅,夷於何。行成之,四時和。山出黃雀亦有羅,雀以高飛柰雀何?為此倚欲,誰肯礞室。



 《上之回曲》:上之回,所中益。夏將至,行將北。以承甘泉宮,寒暑德。游
 石關,望諸國,月支臣,匈奴服。令從百官疾驅馳,千秋萬歲樂無極。



 《翁離曲》:擁離趾中,可築室,何用葺之蕙用蘭。擁離趾中。



 《戰城南曲》:戰城南,死郭北,野死不葬烏可食。為我謂烏:「且為客豪!



 野死諒不葬,腐肉安能去子逃」?水深激激,蒲葦冥冥;梟騎戰鬥死,駑馬裴回鳴。



 梁築室,何以南?梁何北?禾黍不
 獲君何食?願為忠臣安可得?思子良臣,良臣誠可思:朝行出攻,莫不夜歸!



 《巫山高曲》:巫山高,高以大;淮水深,難以逝。我欲東歸,害梁不為。我集無高,曳水何梁。湯湯回回,臨水遠望。泣下霑衣,遠道之人心思歸。謂之何?



 《上陵曲》:上陵何美美,下津風以寒。問客從何來,言從水中央。桂
 樹為君船,青絲為君笮,木蘭為君櫂,黃金錯其間。滄海之雀赤翅鴻,白鴈隨,山林乍開乍合,曾不知日月明。醴泉之水,光澤何蔚蔚。芝為車,龍為馬。覽遨游,四海外。甘露初二年,芝生銅池中,仙人下來飲,延壽千萬歲。



 《將進酒曲》:將進酒,乘太白。辨加哉,詩審搏。放故歌,心所作。同陰氣,詩悉索。使禹良工,觀者苦。



 《君馬黃歌》:
 君馬黃,臣馬蒼,三馬同逐臣馬良。易之有騩蔡有赭,美人歸以南,駕車馳馬。美人傷我心!佳人歸以北,駕車馳馬。佳人安終極!



 《芳樹曲》:芳樹,日月君亂,如於風,芳樹不上無心。溫而鵠,三而為行。



 臨蘭池,心中懷我悵。心不可匡,目不可顧,妬人之子悲殺人。君有它心,樂不可禁。王將何似?如孫如魚乎?悲矣!



 《
 有所思曲》:有所思,乃在大海南。何用問遣君,雙珠玳瑁簪,用玉紹繚之。



 聞君有它心,拉雜摧燒之。摧燒之,當風揚其灰。從今以往,勿復相思!相思與君絕。雞鳴狗吠,兄嫂當知之。妃呼犬希!秋風肅肅晨風穀,東方須臾高知之。



 《雉子曲》:雉子,班如此,之于雉梁,無以吾翁孺。雉子,知得雉子高飛止,黃鵠蜚之以千里,王可思。雄來蜚從雌,視子趨一
 雉。雉子車大駕馬滕,被王送行所中,堯芊蜚從王孫行。



 《聖人出曲》:聖人出,陰陽和。美人出,游九河。佳人來,騑離哉何。駕六飛龍四時和。君之臣明護不道,美人哉,宜天子。免甘星筮樂甫始,美人子,含四海。



 《上邪曲》:上邪!我欲與君相知,長命無絕衰。山無陵,江水為竭,冬雷震震夏雨雪,天地合,乃敢與君絕。



 《
 臨高臺曲》:臨高臺以軒,下有清水清且寒。江有香草目以蘭,黃鵠高飛離哉翻。關弓射鵠,令我主壽萬年。收中吾。



 《遠如期曲》:遠如期,益如壽,處天左側,大樂,萬歲與天無極。雅樂陳,佳哉紛,單于自歸,動如驚心。虞心大佳,萬人還來,謁者引,鄉殿陳,累世未嘗聞之。增壽萬年亦誠哉!冷?《石留曲》:
 石留涼陽涼石水流為沙錫以微河為香向始冷將風陽北逝肯無敢與于楊心邪懷蘭志金安薄北方開留離蘭魏鼓吹曲十二篇,繆襲造:漢第一曲《硃鷺》,今第一曲《初之平》,言魏也。



 初之平,義兵征。神武奮,金鼓鳴。邁武德,揚洪名。漢室微,社稷傾。皇道失,桓與靈。閹宦熾,群雄爭。
 邊韓起,亂金城。中國擾,無紀經。赫武皇,起旗旌。麾天下,天下平。濟九州,九州寧。創武功,武功成。越五帝,邈三王。興禮樂,定紀綱。普日月,齊暉光。



 右《初之平曲》凡三十句,句三字。



 漢第二曲《思悲翁》,今第二曲《戰滎陽》,言曹公也。



 戰滎陽,汴水陂。戎士憤怒,貫甲馳。
 陳未成,退徐榮。二萬騎,塹壘平。戎馬傷,六軍驚。勢不集,眾幾傾。白日沒,時晦冥,顧中牟,心屏營。同盟疑,計無成。賴我武皇,萬國寧。



 右《戰滎陽》曲凡二十句,其十八句句三字,二句句四字。



 漢第三曲《艾如張》,今第三曲《獲呂布》,言曹公東圍臨淮,生擒呂布也。



 獲呂布,戮陳宮。芟夷鯨鯢,驅騁群雄。囊括天下,運掌中。



 右《獲呂布曲》凡六句,其三句句三字,三句句四字。



 漢第四曲《上之回》,今第四曲《克官渡》,言曹公與袁紹戰,破之於官渡也。



 克紹官渡,由白馬。僵屍流血,被原野。賊眾如犬羊,王師尚寡。沙醿傍,風飛揚。
 轉戰不利,士卒傷。今日不勝,後何望!土山地道,不可當。卒勝大捷,震冀方。屠城破邑,神武遂章。



 右《克官渡曲》凡十八句,其八句句三字,一句句五字,九句句四字。



 漢第五曲《翁離》,今第五曲《舊邦》,言曹公勝袁紹於官渡,還譙收藏士卒死亡也。



 舊邦蕭條,心傷悲。孤魂翩翩,當何依。
 游士戀故,涕如摧。兵起事大,令願違。博求親戚,在者誰。立廟置後,魂來歸。



 右《舊邦曲》凡十二句,其六句句三字,六句句四字。



 漢第六曲《戰城南》,今第六曲《定武功》,言曹公初破鄴,武功之定,始乎此也。



 定武功,濟黃河。河水湯湯,旦莫有橫流波。袁氏欲衰,兄弟尋干戈。決漳水,水流滂沱。
 嗟城中如流魚,誰能復顧室家!計窮慮盡,求來連和。和不時,心中憂戚。賊眾內潰,君臣奔北。撥鄴城,奄有魏國。王業艱難,覽觀古今,可為長嘆。



 右《定武功曲》凡二十一句,其五句句三字,三句句六字,十二句句四字,一句五字。



 漢第七曲《巫山高》,今第七曲《屠柳城》,言曹公越北塞,歷白檀,破三郡烏桓於柳城也。



 屠柳城,功誠難。越度隴塞,路漫漫。北踰岡平,但聞悲風正酸。蹋頓授首,遂登白狼山。神武慹海外,永無北顧患。



 右《屠柳城曲》凡十句,其三句句三字,三句句四字,三句句五字,一句六字。



 漢第八曲《上陵》,今第八曲《平南荊》,言曹公南平荊州也。



 南荊何遼遼,江漢濁不清。菁茅久不貢,王師赫南征。劉琮據襄陽,賊備屯樊城。
 六軍廬新野,金鼓震天庭。劉子面縛至,武皇許其成。許與其成,撫其民。陶陶江漢間,普為大魏臣。大魏臣,向風思自新。思自新,齊功古人。在昔虞與唐,大魏得與均。多選忠義士,為喉脣。天下一定,萬世無風塵。



 右《平南荊曲》凡二十四句,其十七句句五字,四句句三字,三句句四字。



 漢第九曲《將進酒》,今第九曲《平關中》,言曹公征馬超,定關中也。



 平關中,路向潼。濟濁水,立高墉。斗韓馬,離群兇。選驍騎,縱兩翼,虜崩潰,級萬億。



 右《平關中曲》凡十句,句三字。



 漢第十曲《有所思》,今第十曲《應帝期》,言曹文帝以聖德受命,應運期也。



 應帝期,於昭我文皇,歷數承天序,龍飛自許昌。聰明昭四表,恩德動遐方。



 星辰為垂耀,日月為重光。河洛吐符瑞,草木挺嘉祥。麒麟步郊野,黃龍游津梁。



 白虎依山林,鳳凰鳴高岡。考圓定篇籍,功配上古羲皇。羲皇無遺文,仁聖相因循。



 運期三千歲,一生聖明君。堯授舜萬國,萬國皆附親。四門為穆穆,教化常如神。



 大魏興盛,與之為鄰。



 右《應帝期曲》凡二十六句,其一句三字,二句四字,二十二句句五字,一句六字。



 漢第十一曲《芳樹》,今第十一曲《邕熙》,言魏氏臨其國,君臣邕穆,庶績咸熙也。



 邕熙,君臣合德,天下治。隆帝道,獲瑞寶,頌聲並作,洋洋浩浩。吉日臨高堂,置酒列名倡。
 歌聲一何紆餘,雜笙簧。八音諧,有紀綱。子孫永建萬國,壽考樂無央。



 右《邕熙曲》凡十五句,其六句句三字,三句句四字,一句二字,三句句五字,二句句六字。



 漢第十二曲《上邪》,今第十二曲《太和》,言魏明帝繼體承統,太和改元,德澤流布。



 惟太和元年,皇帝踐阼,聖且仁,德澤為流布。災蝗一時為絕息,
 上天時雨露。



 五穀溢田疇,四民相率遵軌度。事務澄清,天下獄訟察以情。元首明,魏家如此,那得不太平?



 右《太和曲》凡十三句,其二句句三字,五句句五字,三句句四字,三句句七字。



 晉鼓吹歌曲二十二篇,傅玄作:《靈之祥》,古《硃鷺行》。《靈之祥》,言宣皇帝之佐魏,猶虞舜之事堯也。既有石瑞
 之徵,又能用武以誅孟度之逆命也。



 靈之祥,石瑞章。旌金德,出西方。天命降,授宣皇。應期運,時龍驤。繼大舜,佐陶唐。贊武文,建帝綱。孟氏叛,據南疆。追有扈,亂五常。吳寇勁,蜀虜彊。交誓盟,連遐荒。宣赫怒,奮鷹揚。震乾威,耀電光。陵九天,陷石城。梟逆命,拯有生。
 萬國安,四海寧。



 《宣受命》,古《思悲翁行》。《宣受命》,言宣皇帝禦諸葛亮,養威重,運神兵,亮震怖而死。



 宣受命,應天機。風雲時動,神龍飛。禦葛亮,鎮雍涼。邊境安,民夷康。務節事,勤定傾。覽英雄,保持盈。淵穆穆,赫明明。沖而泰,天之經。
 養威重,運神兵。亮乃震死,平下寧。



 《征遼東》、古《艾而張行》。《征遼東》,言宣皇帝陵大海之表,討滅公孫淵而梟其首也。



 征遼東,敵失據,威靈邁日域。淵既授首,群逆破膽,咸震怖。朔北響應,海表景附。武功赫赫,德雲布。



 《宣輔政》,古《上之回行》:《
 宣輔政》,言宣皇帝聖道深遠,撥亂反正,網羅文武之才,以定二儀之序也。



 宣皇輔政,聖列深。撥亂反正,從天心。網羅文武才,慎厥所生。所生賢,遺教施,安上治民,化風移。肇創帝基,洪業垂。於鑠明明,時赫戲。功濟萬世,定二儀。定二儀,雲澤雨施,海外風馳。



 《時運多難》,古《擁離行》:《
 時運》,言宣皇帝致討吳方,有征無戰也。



 時運多難,道教痡。天地變化,有盈虛。蠢爾吳蠻,虎視江湖。我皇赫斯,致天誅。有征無戰,弭其圖。天威橫被,震東隅。



 《景龍飛》,古《戰城南行》。《景龍飛》,言景帝克明威教,賞從夷逆,祚隆無疆,崇此洪基也。



 景龍飛,御天威。聰鑒玄發,動與神明協機。
 從之者顯,逆之者滅夷。文教敷,武功巍。普被四海,萬邦望風,莫不來綏。聖德潛斷,先天弗違。弗違祥,享世永長。猛以致寬,道化光。赫明明,祚隆無疆。帝績惟期,有命既集,崇此洪基。



 《平玉衡》,古《巫山高行》。《平玉衡》,言景皇帝一萬國之殊風,齊四海之乖心,禮賢養士,而纂洪業也。



 平玉衡,糾姦回。萬國殊風,四海乖。禮賢養士,羈御英雄思心齊。纂成洪業,崇皇階。品物咸亨,聖敬日躋。聰鑒盡下情,明明綜天機。



 《文皇統百揆》,古《上陵行》。《百揆》,言文皇帝始統百揆,用人有序,以敷泰平之化也。



 文皇統百揆,繼天理萬方。武將鎮四隅,英佐盈朝堂。謀言協秋蘭,清風發其芳。
 洪澤所漸潤,礫石為珪璋。大道侔五帝,盛德踰三王。咸光大,上參天與地,至化無內外。無內外,六合並康乂。並康乂,遘茲嘉會。在昔羲與農,大晉德斯邁。



 鎮征及諸州,為蕃衛。功濟四海,洪烈流萬世。



 《因時運》,古《將進酒行》。《因時運》,言文皇帝因時運變,聖謀潛施,解長蛇之交,離
 群桀之黨,以武濟文,審其大計,以邁其德也。



 因時運,聖策施。長蛇交解,群桀離。勢窮奔吳,虎騎厲。惟武進,審大計。



 時邁其德,清一世。



 《惟庸蜀》,古《有所思行》。《惟庸蜀》,言文皇帝既平萬乘之蜀,封建萬國,復五等之爵也。



 惟庸蜀,僭號天一隅。劉備逆帝命,
 禪亮承其餘。擁眾數十萬,窺隙乘我虛。



 驛騎進羽檄,天下不遑居。姜維屢寇邊,隴上為荒墟。文皇愍斯民,歷世受罪辜。



 外謨蕃屏臣,內謀眾士夫。爪牙應指授,腹心獻良圖。良圖協成文,大興百萬軍。



 雷鼓震地起,猛勢陵浮雲。逋虜畏天誅,面縛造壘門。萬里同風教,逆命稱妾臣。



 光建五等,紀綱天人。



 《
 天序》,古《芳樹行》。《天序》,言聖皇應歷受禪,弘濟大化,用人各盡其才也。



 天序,應歷受禪,承靈祜。御群龍,勒螭虎。弘濟大化,英俊作輔。明明統萬機,赫赫鎮四方。咎由稷契之疇,協蘭芳。禮王臣,覆兆民。化之如天與地,誰敢愛其身。



 《大晉承運期》,古《上邪行》。《
 大晉承運期》,言聖皇應籙受圖,化象神明也。



 大晉承運期,德隆聖皇。時清晏,白日垂光。慶籙圖,陟帝位,繼天正玉衡,化行象神明。至哉道隆虞與唐。元首敷洪化,百僚股肱並忠良,民大康。隆隆赫赫,福祚盈無疆。



 《金靈運》,古《君馬黃行》。《靈運》,言聖皇踐阼,致敬宗廟,而孝道施於天下也。



 金靈運,天符發。聖徵見,參日月。惟我皇,體神聖。受魏禪,應天命。皇之興,靈有征。登大麓,御萬乘。皇之輔,若虓虎。爪牙奮,莫之禦。皇之佐,贊清化。百事理,萬邦賀。神祗應,嘉瑞章。恭享祀,薦先皇。樂時奏,磬管鏘。鼓淵淵,鐘喤喤。奠尊俎。實玉觴。神歆饗,咸說康。
 宴孫子,祐無疆。大孝烝烝,德教被萬方。



 《於穆我皇》,古《雉子行》。《於穆》,言聖皇受命,德合神明也。



 於穆我皇,盛德聖且明。受禪君世,光濟群生。普天率土,莫不來庭。顒顒六合內,望風仰泰清。萬國雍雍,興頌聲。大化洽,地平而天成。七政齊,玉衡惟平。



 峨峨佐命,
 濟濟群英。夙夜乾乾,萬機是經。雖治興,匪荒寧。謙道光,沖不盈。



 天地合德,日月同榮。赫赫煌煌,耀幽冥。三光克從,於顯天垂景星。龍鳳臻,甘露宵零。肅神祗,祗上靈。萬物欣戴,自天效其成。



 《仲春振旅》,古《聖人出行》。《
 仲春》,言大晉申文武之教,田獵以時也。



 仲春振旅,大致民,武教於時日新。師執提,工執鼓,坐作從,節有序,盛矣允文允武。搜田表祃,申法誓,遂圍禁,獻社祭,允矣時明國制。文武並用,禮之經,列車如戰,大教明,古今誰能去兵。大晉繼天,濟群生。



 《
 夏苗田》,古《臨高臺行》。《苗田》,言大晉田狩從時,為苗除害也。



 夏苗田,運將徂,軍國異容,文武殊。乃命群吏,選車徒,辯其名號,贊契書。



 王軍啟八門,行同上帝居。時路建大麾,雲旗翳紫虛。百官象其事,疾則疾,徐則徐。回衡旋軫,罷陳敝車。獻禽享祠,烝烝配有虞。
 惟大晉,德參兩儀,化雲敷。



 《仲秋獮田》,古《遠期行》。《仲秋》,言大晉雖有文德,不廢武事,從時以殺伐也。仲秋獮田,金德常剛。涼風清且厲,凝露結為霜。白虎司辰,蒼隼時鷹揚。鷹揚猶周尚父,從天以殺伐。春秋時敘,雷霆震威耀,進退由鉦鼓。致禽祀惣,羽毛之用充軍府。赫赫大晉德,芬烈陵三五,
 敷化以文,雖治不廢武。光宅四海,永享天之祜。



 《從天道》,古《石留行》。《從天道》,言仲冬大閱,用武脩文,大晉之德配天也。



 從天道,握神契。三時亦講武事,冬大閱。鳴鐲振鼓鐸,旌旗象虹霓。文制其中,武不窮武,動軍誓眾,禮成而義舉。三驅以崇仁,進止不失其序。
 兵卒練,將如虎。惟虓虎,氣陵青雲。解圍三面,殺不殄群。偃旌麾,班六軍。獻享烝,脩典文。嘉大晉,德配天。祿報功,爵俟賢。饗燕樂,受茲百祿,嘉萬年。



 《唐堯》,《古務成行》,古曲亡。《唐堯》,言聖皇陟帝位,德化光四表也。



 唐堯咨務成,謙謙德所興。積漸終光大,履霜致堅冰。神明道自然,河海猶可凝。舜禹統百揆,元凱以次升。禪讓應天歷,睿聖世相承。我皇陟帝位,平衡正準繩。德化飛四表,祥氣見其征。興王坐俟旦,亡主恬自矜。致遠由近始,覆簣成山陵。披圖按先籍,有其證靈液。



 《玄雲》,古《玄雲行》,古曲亡。《
 玄云》,言聖皇用人,各盡其材也。



 玄雲起山嶽,祥氣萬里會。龍飛何蜿蜿,鳳翔何翽翽。昔在唐虞朝,時見青雲際。今親遊方國,流光溢天外。鶴鳴在後園,清音隨風邁。成湯隆顯命,伊摯來如飛。周文獵渭濱,遂載呂望歸。符合如影響,先天天弗違。輟耕總地綱,解褐衿天維。元功配二主,芬馨世所稀。我皇敘群才,
 洪烈何巍巍。桓桓征四表,濟濟理萬機。神化感無方,髦才盈帝畿。丕顯惟昧旦,日新孔所咨。茂哉聖明德,日月同光輝。



 《伯益》,古《黃爵行》,古曲亡。《伯益》,言赤烏銜書,有周以興;今聖皇受命,神雀來也。



 伯益佐舜禹,職掌山與川。德侔十六相,思心入無間。智理周萬物,下知眾鳥言。黃雀應清化,翔集何翩翩。和鳴棲庭樹,
 徘徊雲日間。夏桀為無道,密網施山阿。酷祝振纖網,當柰黃雀何。殷湯崇天德,去其三面羅。逍遙群飛來,鳴聲乃復和。朱雀作南宿,鳳皇統羽群。赤鳥銜書至,天命瑞周文。神雀今來遊,為我受命君。嘉祥致天和,膏澤降青雲。蘭風發芳氣,闔世同其芬。


《釣竿》,古《釣竿行》,
 \gezhu{
  漢《鐃歌》二十二無《釣竿》。}
 《
 釣竿》,言聖皇德配堯、舜,又有呂望之佐以濟大功致太平也。



 釣竿何冉冉,甘餌芳且鮮。臨川運思心,微綸沈九淵。太公寶此術,乃在靈秘篇。機變隨物移,精妙貫未然。遊魚驚著釣,潛龍飛戾天。戾天安所至,撫翼翔太清。太清一何異,兩儀出渾成。玉衡正三辰,造化賦群形。退願輔聖君,與神合其靈。
 我君弘遠略,天人不足并。天人初并時,昧昧何茫茫。日月有徵兆,文象興二皇。蚩尤亂生民,黃帝用兵征萬方。逮夏禹而德衰,三代不及虞與唐。我皇聖德配堯舜,受禪即阼享天祥。率土蒙祐,靡不肅,庶事康。庶事康,穆穆明明。荷百祿,保無極,永泰平。



 吳鼓吹曲十二篇,韋昭造:《炎精缺》者,言漢室衰,武烈皇帝奮迅猛志,念在匡救,然
 而王迹始乎此也。漢曲有《硃鷺》,此篇當之。第一。



 炎精缺,漢道微。皇綱弛,政德違。眾姦熾,民罔依。赫武烈,越龍飛。陟天衢,耀靈威。鳴雷鼓,抗電麾。撫乾衡,鎮地機。厲虎旅,騁熊羆。發神聽,吐英奇。張角破,邊韓羈。宛潁平,南土綏。神武章,渥澤施。金聲震,仁風馳。顯高門,啟皇基。
 統罔極,垂將來。



 右《炎精缺曲》凡三十句,句三字。



 《漢之季》者,武烈皇帝悼漢之微,痛卓之亂,興兵奮擊,功蓋海內也。漢曲有《曲悲翁》,此篇當之。第二。



 漢之季,董卓亂。桓桓武烈,應時運。義兵興,雲旗建。厲六師,羅八陳。飛鳴鏑,接白刃。輕騎發,介士奮。醜虜震,使眾散。劫漢主,遷西館。
 雄豪怒,元惡僨。赫赫皇祖,功名聞。



 右《漢之季曲》凡二十句,其十八句句三字,二句句四字。



 《攄武師》者,言大皇帝卒武烈之業而奮徵也。漢曲有《艾如張》,此篇當之。



 第三。



 攄武師,斬黃祖。肅夷凶族,革平西夏。炎炎大烈,震天下。



 右《攄武師曲》凡六句,其三句句三字,三句句四
 字。



 《烏林》者,言曹操既破荊州,從流東下,欲來爭鋒。大皇帝命將周瑜逆擊之於烏林而破走也。漢曲有《上之回》,此篇當之。第四。



 曹操北伐,拔柳城。乘勝席卷,遂南征。劉氏不睦,八郡震驚。眾既降,操屠荊。舟車十萬,揚風聲。議者狐疑,慮無成。賴我大皇,發聖明。虎臣雄烈,周與程。
 破操烏林,顯章功名。



 右《伐烏林曲》凡十八句,其十句句四字,八句句三字。



 《秋風》者,言大皇帝說以使民,民忘其死。漢曲有《擁離》,此篇當之。第五。



 秋風揚沙塵,寒露霑衣裳。角弓持弦急,鳩鳥化為鷹。邊垂飛羽檄,寇賊侵界疆。跨馬披介胄,慷慨懷悲傷。辭親向長路,
 安知存與亡。窮達固有分,志士思立功。邀之戰場,身逸獲高賞,身沒有遺封。



 右《秋風曲》凡十五句,其十四句句五字,一句四字。



 《克皖城》者,言曹操志圖并兼,而令硃光為廬江太守。上親征光,破之於皖城也。漢曲有《戰城南》,此篇當之。第六。



 克滅皖城,遏寇賊。惡此兇孽,阻姦慝。王師赫征,眾傾覆。除穢去暴,戢兵革。
 民得就農,邊境息。誅君弔臣,昭至德。



 右《克皖城曲》凡十二句,其六句句三字,六句句四字。



 《關背德》者,言蜀將關羽背棄吳德,心懷不軌。大皇帝引師浮江而禽之也。



 漢曲有《巫山高》,此篇當之。第七。



 關背德,作鴟張。割我邑城,圖不祥。稱兵北伐,圍樊襄陽。嗟臂大於股,將受其殃。巍巍吳聖主,睿德與玄通。與玄通,親任呂蒙。
 泛舟洪氾池,溯涉長江。



 神武一何桓桓!聲烈正與風翔。歷撫江安城,大據郢邦。虜羽授首,百蠻咸來同,盛哉無比隆。



 右《關背德曲》凡二十一句,其八句句四字,二句句六字,七句句五字,四句句三字。



 《通荊門》者,言大皇帝與蜀交好齊盟,中有關羽自失之愆,戎蠻樂亂,生變作患,蜀疑其眩,吳惡其詐,乃大治兵,終復初好也。漢曲有《上陵》,此篇當之。



 第八。



 荊門限巫山,高峻與雲連。蠻夷阻其險,歷世懷不賓。漢王據蜀郡,崇好結和親。乖微中情疑,讒夫亂其間。大皇赫斯怒,虎臣勇氣震。蕩滌幽藪,討不恭。觀兵揚炎耀,厲鋒整封疆。整封疆,闡揚威武容。功赫戲,洪烈炳章。邈矣帝皇世,聖吳同厥風。荒裔望清化,化恢弘。煌煌大吳,延祚永未央。



 右《通荊門曲》凡二十四句,其十七句句五字,四句句三字,三句句四字。



 《章洪德》者,言大皇帝章其大德,而遠方來附也。漢曲有《將進酒》,此篇當之。第九。



 章洪德,邁威神。感殊風,懷遠鄰。平南裔,齊海濱。越裳貢,扶南臣。珍貨充庭,所見日新。



 右《章洪德曲》凡十句,其八句句三字,二句句四
 字。



 《從歷數》者,言大皇帝從籙圖之符,而建大號也。漢曲有《有所思》,此篇當之。第十。



 從歷數,於穆我皇帝。聖哲受之天,神明表奇異。建號創皇基,聰睿協神思。



 德澤浸及昆蟲,浩蕩越前代。三光顯精耀,陰陽稱至治。肉角步郊畛,鳳凰棲靈囿。



 神龜游沼池,圖讖摹文字。黃龍覿鱗,
 符祥日月記。覽往以察今,我皇多噲事。上欽昊天象,下副萬姓意。光被彌蒼生,家戶蒙惠賚。風教肅以平,頌聲章嘉喜。大吳興隆,綽有餘裕。



 右《從歷數曲》凡二十六句,其一句句三字,三句句四字,二十二句句五字,一句六字。



 《承天命》者,言上以聖德踐位,道化至盛也。漢曲有《芳樹》,此篇當之。



 第十一。



 承天命,於昭聖德。三精垂象,符靈表德。巨石立,九穗植。龍金其鱗,烏赤其色。輿人歌,億夫歎息。超龍升,襲帝服。躬淳懿,體玄默。夙興臨朝,勞謙日昃。易簡以崇仁,放遠讒與慝。舉賢才,親近有德。均田疇,茂稼穡。審法令,定品式。考功能,明黜陟。人思自盡,惟心與力。家國治,王道直。思我帝皇,壽萬億。
 長保天祿,祚無極。



 右《承天命曲》凡三十四句,其十九句句三字,二句句五字,十三句句四字。



 《玄化》者,言上修文訓武,則天而行,仁澤流洽,天下喜樂也。漢曲有《上邪》,此篇當之。第十二。



 玄化象以天,陛下聖真。張皇綱,率道以安民。惠澤宣流而雲布,上下睦親。



 君臣酣宴樂,激發弦歌揚妙新。脩文籌廟勝,
 須時備駕巡洛津。康哉泰,四海歡忻,越與三五鄰。



 右《玄化曲》凡十三句,其五句句五字,二句句三字,三句句四字,三句句七字。


今鼓吹鐃歌詞。
 \gezhu{
  樂人以音聲相傳,訓詁不可復解。}



 大竭夜烏自云何來堂吾來聲烏奚姑悟姑尊盧聖子黃尊來餭清嬰烏白日為隨來郭吾微令吾應龍夜烏由道何來直子為烏奚如悟姑尊盧雞子聽烏虎行為
 來明吾微令吾詩則夜烏道祿何來黑洛道烏奚悟如尊爾尊盧起黃華烏伯遼為國日忠雨令吾伯遼夜烏若國何來日忠雨烏奚如悟姑尊盧面道康尊錄龍永烏赫赫福胙夜音微令吾右四解,《上邪曲》。


幾令吾幾令諸韓亂發正令吾幾令吾諸韓從聽心令吾若里洛何來韓微令吾尊盧
 忌盧文盧子路子路為路雞如文盧炯烏諸胙微令吾幾令諸韓或公隨令吾幾令吾幾諸或言隨令吾黑洛何來諸韓微令吾尊盧安成隨來免路路子為吾路奚如文盧炯烏諸胙微令吾右九解,《晚芝曲》。
 \gezhu{
  漢曲有《遠期》,疑是。}



 幾令吾呼歷舍居執來隨咄武子邪令烏銜針相風其右其右
 幾令吾呼群議破葫執來隨吾咄武子邪令烏今烏今狖入海相風及後幾令吾呼無公赫吾執來隨吾咄武子邪令烏無公赫吾婮立諸布始布右三解,《艾如張曲》。



 鼓吹鐃歌十五篇,何承天義熙中私造:《硃路篇》:朱路揚和鸞,翠蓋耀金華。玄牡飾樊纓,
 流旌拂飛霞。雄戟闢曠塗,班劍翼高車。三軍且莫喧,聽我奏鐃歌。清鞞驚短簫,朗鼓節鳴笳。人心惟愷豫,茲音亮且和。輕風起紅塵,渟瀾發微波。逸韻騰天路,頹響結城阿。仁聲被八表,威震振九遐。嗟嗟介胄士,勖哉念皇家。



 《思悲公篇》:思悲公,懷袞衣。東國何悲,公西歸。
 公西歸,流二叔,幼主既悟,偃禾復。偃禾復,聖志申。營都新邑,從斯民。從斯民,德惟明。制禮作樂,興頌聲。興頌聲,致嘉祥。鳴鳳爰集,萬國康。萬國康,猶弗已。握髮吐餐,下群士。惟我君,繼伊周。親睹盛世,復何求。



 《雍離篇》:雍士多離心,荊民懷怨情。二凶不量德,
 構難稱其兵。王人銜朝命,正辭糾不庭。上宰宣九伐,萬里舉長旌。樓船掩江濆,駟介飛重英。歸德戒後夫,賈勇尚先鳴。逆徒既不濟,愚智亦相傾。霜鋒未及染,鄢郢忽已清。西川無潛鱗,北渚有奔鯨。凌威致天府,一戰夷三城。江漢被美化,宇宙歌太平。惟我東郡民,曾是深推誠。



 《戰城南篇》:
 戰城南,衡黃塵。丹旌電烻,鼓雷震。勍敵猛,戎馬殷。橫陳亙野,若屯雲。仗大從,應三靈。義之所感,士忘生。長劍擊,繁弱鳴。飛鏑炫晃,亂奔星。虎騎躍,華毦旋。朱火延起,騰飛煙。驍雄斬,高旗搴。長角浮叫,響清天。夷群寇,殪逆徒。餘黎霑惠,詠來蘇。奏愷樂,歸皇都。班爵獻俘,邦國娛。



 《
 巫山高篇》:巫山高,三峽峻。青壁千尋,深谷萬仞。崇巖冠靈,林冥冥。



 山禽夜響,晨猿相和鳴。洪波迅澓,載逝載停。心妻心妻商旅之客,懷苦情。在昔陽九,皇綱微。李氏竊命,宣武耀靈威。蠢爾逆縱,復踐亂機。王旅薄伐,傳首來至京師。古之為國,
 惟德是貴。力戰而虛民,鮮不顛墜。矧乃叛戾,伊胡能遂。咨爾巴子,無放肆。



 《上陵者篇》:上陵者,相追攀。被服纖麗,振綺紈。攜童幼,升崇巒。南望城闕,鬱槃桓。王公第,通衢端。高甍華屋,列硃軒。臨浚谷,掇秋蘭。士女悠奕,映隰原。
 指營丘,感牛山。爽鳩既沒,景君嘆。嗟歲聿,游不還。志氣衰沮,玄鬢斑。野莽宿,墳土乾。顧此纍纍,中心酸。生必死,亦何怨。取樂今日,展情懽。



 《將進酒篇》:將進酒,慶三朝。備繁禮,薦嘉肴。榮枯換,霜霧交。緩春帶,命朋僚。車等旗,馬齊鑣。懷溫克,樂林濠。
 士失志,慍情勞。思旨酒,寄游遨。



 敗德人,甘醇醪。耽長夜,或淫妖。興屢舞,厲哇謠。形人差人差,聲號呶。首既濡,志亦荒。性命天,國家亡。嗟後生,節酣觴。匪酒辜,孰為殃。



 《君馬篇》:君馬麗且閑,揚鑣騰逸姿。駿足躡流景,高步追輕飛。冉冉六轡柔,奕奕金華暉。
 輕霄翼羽蓋,長風靡淑旂。願為範氏驅,雍容步中畿。豈效詭遇子,馳騁趣危機。鉛陵策良駟,造父為之悲。不怨吳阪峻,但恨伯樂稀。赦彼岐山盜,實躋韓原師。柰何漢魏主,縱情營所私。疲民甘藜藿,廄馬患盈肥。人畜貿厥養,蒼生將焉歸。



 《芳樹篇》:芳樹生北庭,豐隆正裴徊。翠穎陵冬秀,
 紅葩迎春開。佳人閑幽室,惠心婉以諧。蘭房掩綺幌,綠草被長階。日夕游雲際,歸禽命同棲。皓月盈素景,涼風拂中閨。哀弦理虛堂,要妙清且心妻。嘯歌流激楚,傷此碩人懷。梁塵集丹帷,微飆揚羅袿。豈怨嘉時莫,徒惜良願乖。



 《有所思篇》:有所思,思昔人。曾閔二子,善養親。
 和顏色,奉晨昏。至誠烝烝,通明神。鄒孟軻,為齊卿。稱身受祿,不貪榮。道不用,獨擁楹。三徙既誶,禮義明。飛鳥集,猛獸附。功成事畢,乃更娶。哀我生,遘凶旻。幼罹荼毒,備艱辛。慈顏絕,見無因。長懷永思,託丘墳。



 《雉子遊原澤篇》:雉子遊原澤,初懷耿介心。飲啄雖勤苦,
 不願棲園林,古有避世士,抗志清霄岑。浩然寄卜肆,揮棹通川陰。消搖風塵外,散髮撫鳴琴。卿相非所眄,何況於千金。功名豈不美,寵辱亦相尋。冰炭結六府,憂虞纏胸襟。當世須大度,量己不克任。三復泉流誡,自驚良已深。



 《上邪篇》:上邪下難正,眾枉不可矯。音和響必清,
 端影緣直表。大化揚仁風,齊人猶偃草。聖王既已沒,誰能弘至道。開春湛柔露,代終肅嚴霜。承平貴孔孟,政敝侯申商。孝公明賞罰,六世猶克昌。李斯肆濫刑,秦氏所以亡。漢宣隆中興,魏祖寧三方。譬彼針與石,效疾故稱良。《行葦》非不厚,悠悠何詎央。琴瑟時永調,改弦當更張。矧乃治天下,此要安可忘。



 《
 臨高臺篇》:臨高臺,望天衢,飄然輕舉,陵太虛。攜列子,超帝鄉。雲衣雨帶,乘風翔。肅龍駕,會瑤臺。清暉浮景,溢蓬萊。濟四海,濯洧盤。佇立雲岳,結幽蘭。馳迅風,遊炎州。願言桑梓,思舊遊。傾霄蓋,靡電旌。降彼天塗,頹窈冥。辭仙族,歸人群。懷忠抱義,奉明君。
 任窮達,隨所遭。何為遠想,令心勞。



 《遠期篇》:遠期千里客,肅駕候良辰。近命城郭友,具爾惟懿親。高門啟雙闈,長筵列嘉賓。中唐儛六佾,三廂羅樂人。簫管激悲音,羽毛揚華文。金石響高宇,絃歌動梁塵。修標多巧捷,九劍亦入神。遷善自雅調,成化由清均。主人垂隆慶,群士樂亡身。
 願我聖明君,邇期保萬春。



 《石流篇》:石上流水,湔湔其波。發源幽岫,永歸長河。瞻彼逝者,歲月其偕。子在川上,惟以增懷。嗟我殷憂,載勞寤寐。遘此百罹,有志不遂。行年倏忽,長勤是嬰。永言沒世,悼茲無成。幸遇開泰,沐浴嘉運。緩帶安寢,亦又何慍。古之為仁,自求諸己。虛情遙慕,終於徒已。



 《
 聖人制禮樂》一篇,《巾舞歌》一篇,按《景祐廣樂記》言,字訛謬,聲辭雜書。宋鼓吹鐃歌辭四篇,舊史言,詁不可解。漢鼓吹鐃歌十八篇,按《古今樂錄》,皆聲、辭、艷相雜,不復可分。



\end{pinyinscope}