\article{卷五十七列傳第十七 蔡廓 子興宗}

\begin{pinyinscope}

 蔡廓,字子度,濟陽考城人也。曾祖謨,晉司徒。祖系,撫軍長史。父綝,司徒左西屬。廓博涉群書,言行以禮。
 起家著作佐郎,時桓玄輔晉,議復肉刑,廓上議曰:「夫建封立法,
 弘治稽化,必隨時置制,德刑兼施。貞一以閑其邪,教禁以檢其慢,灑湛露以膏潤,厲嚴霜以肅威,晞風者陶和而安恬,畏戾者聞憲而警慮。



 雖復質文迭用,而斯道莫革。肉刑之設,肇自哲王。蓋由曩世風淳,民多惇謹,圖像既陳,則機心冥戢,刑人在塗,則不逞改操,故能勝殘去殺,化隆無為。季末澆偽,法網彌密,利巧之懷日滋,恥畏之情轉寡,終身劇役,不足止其姦,況乎黥劓,豈能反其善!徒有酸慘之聲,而無濟治之益。至于棄市之條,實非
 不赦之罪,事非手殺,考律同歸,輕重均科,減降路塞,鐘、陳以之抗言,元皇所為留愍。今英輔翼贊,道邈伊、周,雖閉否之運甫開,而遐遺之難未已。誠宜明慎用刑,愛民弘育,申哀矜以革濫,移大辟於支體,全性命之至重,恢繁息於將來。使將斷之骨,荷更榮於三陽,乾時之華,監商飆而知懼。威惠俱宣,感畏偕設,全生拯暴,於是乎在。」



 遷司徒主簿,尚書度支殿中郎,通直郎,高祖太尉參軍,司徒屬,中書、黃門郎。以方鯁閑素,為高祖所知。及高祖
 領兗州,廓為別駕從事史,委以州任。尋除中軍咨議參軍,太尉從事中郎。未拜,遭母憂。性至孝,三年不櫛沐,殆不勝喪。



 服闋,相國府復板為從事中郎,領記室。宋臺建,為侍中,建議以為:「鞫獄不宜令子孫下辭明言父祖之罪,虧教傷情,莫此為大。自今但令家人與囚相見,無乞鞫之訴,使足以明伏罪,不須責家人下辭。」朝議咸以為允,從之。



 世子左衛率謝靈運輒殺人,御史中丞王準之坐不糾免官,高祖以廓剛直,不容邪枉,補御史中丞。多所糾
 奏,百僚震肅。時中書令傅亮任寄隆重,學冠當時,朝廷儀典,皆取定於亮,每咨廓然後施行。亮意若有不同,廓終不為屈。時疑揚州刺史廬陵王義真朝堂班次,亮與廓書曰:「揚州自應著刺史服耳。然謂坐起班次,應在朝堂諸官上,不應依官次坐下。足下試更尋之。《詩序》云『王姬下嫁於諸侯,衣服禮秩,不係其夫,下王后一等。』推王姬下王后一等,則皇子居然在王公之上。



 陸士衡《起居注》,式乾殿集,諸皇子悉在三司上。今抄疏如別。又海西
 即位赦文,太宰武陵王第一,撫軍將軍會稽王第二,大司馬第三。大司馬位既最高,又都督中外,而次在二王之下,豈非下皇子邪?此文今具在也。永和中,蔡公為司徒,司馬簡文為撫軍開府,對錄朝政。蔡為正司,不應反在儀同之下,而於時位次,相王在前,蔡公次之耳。諸例甚多,不能復具疏。揚州反乃居卿君之下,恐此失禮,宜改之邪?」廓答曰:「揚州位居卿君之下,常亦惟疑。然朝廷以位相次,不以本封,復無明文云皇子加殊禮。齊獻王
 為驃騎,孫秀來降,武帝欲優異之,以秀為驃騎,轉齊王為鎮軍,在驃騎上。若如足下言,皇子便在公右,則齊王本次自尊,何改鎮軍,令在驃騎上,明知故依見位為次也。又齊王為司空,賈充為太尉,俱錄尚書署事,常在充後。潘正叔奏《公羊》事,于時三錄,梁王肜為衛將軍,署在太尉隴西王泰、司徒王玄沖下。近太元初,駕新宮成,司馬太傅為中軍,而以齊王柔之為賀首。立安帝為太子,上禮,徐邈為郎,位次亦以太傅在諸王下;又謁李太后,
 宗正尚書符令以高密王為首,時王東亭為僕射。王、徐皆是近世識古今者。足下引式乾公王,吾謂未可為據。其云上出式乾,召侍中彭城王植、荀組、潘岳、嵇紹、杜斌,然後道足下所疏四王,在三司之上,反在黃門郎下,有何義?且四王之下則云大將軍梁王肜、車騎趙王倫,然後云司徒王戎耳。梁、趙二王亦是皇子,屬尊位齊,在豫章王常侍之下,又復不通。蓋書家指疏時事,不必存其班次;式乾亦是私宴,異於朝堂。如今含章西堂,足下在
 僕射下,侍中在尚書下耳。來示又云曾祖與簡文對錄,位在簡文下。吾家故事則不然,今寫如別。王姬身無爵位,故可得不從夫而以王女為尊。皇子出任則有位,有位則依朝,復示之班序。唯引泰和赦文,差可為言。



 然赦文前後,亦參差不同。太宰上公,自應在大司馬前耳。簡文雖撫軍,時已授丞相殊禮,又中外都督,故以本任為班,不以督中外便在公右也。今護軍總方伯,而位次故在持節都督下,足下復思之。」



 遷司徒左長史,出為豫章
 太守,徵為吏部尚書。廓因北地傅隆問亮:「選事若悉以見付,不論;不然,不能拜也。」亮以語錄尚書徐羨之,羨之曰:「黃門郎以下,悉以委蔡,吾徒不復厝懷;自此以上,故宜共參同異。」廓曰:「我不能為徐幹木署紙尾也。」遂不拜。干木,羨之小字也。選案黃紙,錄尚書與吏部尚書連名,故廓云:「署紙尾」也。羨之亦以廓正直,不欲使居權要。徙為祠部尚書。



 太祖入奉大統,尚書令傅亮率百僚奉迎,廓亦俱行。至尋陽,遇疾,不堪前。



 亮將進路,詣廓別,廓謂
 曰:「營陽在吳,宜厚加供奉。營陽不幸,卿諸人有弒主之名,欲立於世,將可得邪!」亮已與羨之議害少帝,乃馳信止之,信至,已不及。



 羨之大怒曰:「與人共計議,云何裁轉背,便賣惡於人。」及太祖即位,謝晦將之荊州,與廓別,屏人問曰:「吾其免乎?」廓曰:「卿受先帝顧命,任以社稷,廢昏立明,義無不可。但殺人二昆,而以之北面,挾震主之威,據上流之重,以古推今,自免為難也。」



 廓年位並輕,而為時流所推重,每至歲時,皆束帶到門。奉兄軌如父,家事小
 大,皆咨而後行;公祿賞賜,一皆入軌,有所資須,悉就典者請焉。從高祖在彭城,妻郗氏書求夏服,廓答書曰:「知須夏服,計給事自應相供,無容別寄。」時軌為給事中。元嘉二年,廓卒,時年四十七。高祖嘗云:「羊徽、蔡廓,可平世三公。」



 少子興宗。



 興宗年十歲失父,哀毀有異凡童。廓罷豫章郡還,起二宅。先成東宅,與軌;廓亡而館宇未立,軌罷長沙郡還,送錢五十萬以補宅直。興宗年十歲,白母曰:「一家由來豐
 儉必共,今日宅價不宜受也。」母悅而從焉。軌有愧色,謂其子淡曰:「我年六十,行事不及十歲小兒。」尋喪母。



 少好學,以業尚素立見稱。初為彭城王義康司徒行參軍,太子舍人,南平穆王冠軍參軍,武昌太守。又為太子洗馬,義陽王友,中書侍郎。中書令建平王宏、侍中王僧綽並與興宗厚善。元凶弒立,僧綽被誅,凶威方盛,親故莫敢往,興宗獨臨哭盡哀。出為司空何尚之長史。又遷太子中庶子。



 世祖踐阼,還先職,遷臨海太守,徵為黃門郎,太
 子中庶子,轉游擊將軍,俄遷尚書吏部郎。時尚書何偃疾患,上謂興宗曰:「卿詳練清濁,今以選事相付,便可開門當之,無所讓也。」轉司徒左長史,復為中庶子,領前軍將軍,遷侍中。每正言得失,無所顧憚,由是失旨。竟陵王誕據廣陵城為逆,事平,興宗奉旨慰勞。



 州別駕范義與興宗素善,在城內同誅。興宗至廣陵,躬自收殯,致喪還豫章舊墓。



 上聞之,甚不悅。廬陵內史周朗以正言得罪,鎖付寧州,親戚故人,無敢瞻送;興宗在直,請急,詣朗別。
 上知尤怒。坐屬疾多日,白衣領職。尋左遷司空沈慶之長史,行兗州事,還為廷尉卿。



 有解士先者,告申坦昔與丞相義宣同謀。時坦已死,子令孫時作山陽郡。自繫廷尉。興宗議曰:「若坦昔為戎首,身今尚存,累經肆眚,猶應蒙宥。令孫天屬,理相為隱。況人亡事遠,追相誣訐,斷以禮律,義不合關。若士先審知逆謀,當時即應聞啟,苞藏積年,發因私怨,況稱風聲路傳,實無定主,而千黷欺罔,罪合極法。」又有訟民嚴道恩等二十二人,事未洗正,敕
 以當訊,權繫尚方。興宗以訟民本在求理,故不加械,即若系尚方,於事為苦。又司徒前劾送武康令謝沈及郡縣尉還職司十一人,坐仲良鑄錢不禽,久已判結。又送郡主簿丘元敬等九人,或下疾假,或去職已久。又加執啟,事悉見從。



 出為東陽太守,遷安陸王子綏後軍長史、江夏內史,行郢州事。徵還,未拜,留為左民尚書。頃之,轉掌吏部。時上方盛淫宴,虐侮群臣,自江夏王義恭以下,咸加穢辱,唯興宗以方直見憚,不被侵媟。尚書僕射顏
 師伯謂議曹郎王耽之曰:「蔡尚書常免暱戲,去人實遠。」耽之曰:「蔡豫章昔在相府,亦以方嚴不狎,武帝宴私之日,未嘗相召,每至官賭,常在勝朋。蔡尚書今日可謂能負荷矣。」



 大明末,前廢帝即位,興宗告太宰江夏王義恭,應須策文。義恭曰:「建立儲副,本為今日,復安用此。」興宗曰:「累朝故事,莫不皆然。近永初之末,營陽王即位,亦有文策,今在尚書,可檢視也。」不從。興宗時親奉璽綬,嗣主容色自若,了無哀貌。興宗出謂親故曰:「魯昭在戚而有
 嘉容,終之以釁結大臣,昭子請死。國家之禍,其在此乎。」時義恭錄尚書事,受遺輔政,阿衡幼主,而引身避事,政歸近習。越騎校尉戴法興、中書舍人巢尚之專制朝權,威行近遠。興宗職管九流,銓衡所寄,每至上朝,輒與令錄以下,陳欲登賢進士之意,又箴規得失,博論朝政。



 義恭素性恇橈,阿順法興,常慮失旨,聞興宗言,輒戰懼無計。先是大明世,奢侈無度,多所造立,賦調煩嚴,徽役過苦。至是發詔,悉皆削除,由此紫極殿南北馳道之屬,皆
 被毀壞。自孝建以來至大明末,凡諸制度,無或存者。興宗於都坐慨然謂顏師伯曰:「先帝雖非盛德主,要以道始終。三年無改,古典所貴。今殯宮始徹,山陵未遠,而凡諸制度興造,不論是非,一皆刊削。雖復禪代,亦不至爾。天下有識,當以此窺人。」師伯不能用。



 興宗每陳選事,法興、尚之等輒點定回換,僅有在者。興宗於朝堂謂義恭及師伯曰:「主上諒暗,不親萬機,而選舉密事,多被刪改,復非公筆,亦不知是何天子意。」王景文、謝莊等遷授失
 序,興宗又欲為美選。時薛安都為散騎常侍、征虜將軍、太子左率,殷常為中庶子。興宗先選安都為左衛將軍,常侍如故;殷常為黃門,領校。太宰嫌安都為多,慾單為左衛,興宗曰:「率衛相去,唯阿之間。且已失征虜,非乃超越,復奪常侍,頓為降貶。若謂安都晚達微人,本宜裁抑,令名器不輕,宜有貫序。謹依選體,非私安都。」義恭曰:「若宮官宜加超授者,殷常便應侍中,那得為黃門而已。」興宗又曰:「中庶、侍中,相去實遠。且安都作率十年,殷恆中庶百
 日,今又領校,不為少也。」使選令史顏禕之、薛慶先等往復論執,義恭然後署案。



 既中旨以安都為右衛,加給事中,由是大忤義恭及法興等,出興宗吳郡太守。



 固辭郡,執政愈怒,又轉為新安王子鸞撫軍司馬、輔國將軍、南東海太守,行南徐州事。又不拜,苦求益州。義恭於是大怒,上表曰:「臣聞慎節言語,《大易》有規,銓序九流,無取裁囗。若乃結黨連群,譏訴互起,街談巷議,罔顧聽聞,乃撤實憲制所宜禁經之巨蠹。侍中祕書監臣彧自表父疾,
 必求侍養,聖旨矜體,特順所陳,改授臣府元僚,兼帶軍郡。雖臣駑劣,府任非輕,準之前人,不為屈後。京郡本以為祿,不計戶之少多,遇缺便用,無關高下。撫軍長史莊滯府累朝,每陳危苦,內職外守,稱未堪依。唯王球昔比,賜以優養,恩慈之厚,不近於薄。前新除吳郡太守興宗,前居選曹,多不平允,鴻渥含宥,恕其不閑,改任大都,寵均阿輔,仍苦請益州,雅違成命。伏尋揚州刺史子尚、吳興太守休若,並國之茂戚,魯、衛攸在,猶牧守東山,竭誠
 撫蒞,而辭擇適情,起自庶族,逮佐北籓,尤無欣荷。御史中丞永,昔歲餘愆,從恩今授,光祿勳臣淹,雖曰代臣,累經降黜,後效未申,以何取進。司徒左長史孔覬,前除右衛,尋徙今職,回換之宜,不為乃少。竊外談謂彧等咸為失分,又聞興宗躬自怨懟,與尚書右僕射師伯疏,辭旨甚苦。臣雖不見,所聞不虛。臣以凡才,不應機務,謬自幸會,受任三朝,進無古人興賢之美,退無在下獻替之績,致茲紛紜,伏增慚悚。然此源不塞,此風弗變,將虧正道,
 塵穢盛猷。伏顧聖德,賜垂覽察。」詔曰:「太宰表如此,省以憮然。朕恭承洪緒,思弘盛烈,而在朝倰競,驅扇成風,將何以式揚先德,克隆至化。公體國情深,保釐攸託,便可付外詳議。」



 義恭因使尚書令柳元景奏曰:「臣義恭表、詔書如右。攝曹辨核尚書袁愍孫牒:『此月十七日,詣僕射顏師伯,語次,因及尚書蔡興宗有書固辭今授,仍出疏見示,乃者數紙,不意悉何所道,緣此因及朝士。當今聖世,不可使人以為少。今牒。』數之,朝廷處之實得所,臣等
 亦自謂得分,常多在門,袁愍孫無或措多,而愚意欲啟更量出內之宜,芻蕘管見,願在聞徹。選令史宣傳密事,故因附上聞,亦外人言此。今薛慶先列:『今月十八日,往尚書袁愍孫論選事。愍孫云,昨詣顏修射,出蔡尚書疏見示,言辭甚苦。又云所得亦少。主上踐阼始爾,朝士有此人不多,物議謂應美用,乃更恨少,使咨事便啟錄公。又謝莊囗時未老,其疾以轉差,今居此任,復為非宜,謂宜中書令才望為允。又孔覬南士之美,所歷已多,近頻
 授即復回改,於理為屈,門下無人,此是名選。又張永人地可論,其去歲愆戾,非為深罪,依其望復門下一人。張淹昔忝南下,預同休戚,雖屢經愆黜,事亦已久,謂應祕書監。』帶授興宗手跡數紙,文翰炳然,事證明白,不假覈辨。愍孫任居官人,職掌銓裁,若有未允,則宜顯言,而私加許與,自相選署,託云物論,終成虛詭,隱末出端,還為矛楯。臣聞九官成讓,虞風垂則,誹主怨時,漢罪夙斷。況義為身發,言謗朝序,亂辟害政,混穢大猷,紛紜彰謬,上
 延詔旨,不有霜準,軌憲斯淪。請解興宗新附官,須事御,收付廷尉法獄治罪,免愍孫所居官。」詔曰:「興宗首亂朝典,允當明憲,以其昔經近侍,未忍盡法,可令思愆遠封。愍孫竊評自己,委咎物議,可以子領職。」



 除興宗新昌太守,郡屬交州。朝廷莫不嗟駭。先是,興宗納何后寺尼智妃為妾,姿貌甚美,有名京師,迎車已去,而師伯密遣人誘之,潛往載取,興宗迎人不覺。



 及興宗被徙,論者並云由師伯,師伯甚病之。法興等既不欲以徙大臣為名,師
 伯又欲止息物議,由此停行。頃之,法興見殺,尚之被繫,義恭、師伯誅,復起興宗為臨海王子頊前軍長史、輔國將軍、南郡太守,行荊州事,不行。



 時前廢帝凶暴,興宗外甥袁顗為雍州刺史,勸興宗行,曰:「朝廷形勢,人所共見,在內大臣,朝夕難保。舅今出居陜西,為八州行事,顗在襄、沔,地勝兵強,去江陵咫尺,水陸通便。若朝廷有事,可共立桓、文之功,豈與受制凶狂,禍難不測,同年而語乎。今不去虎口,而守此危逼,後求復出,豈得哉!」興宗曰:「吾素
 門平進,與主上甚疏,未容有患。宮省內外,人不自保,會應有變。若內難得弭,外釁未必可量。汝欲在外求全,我欲居內免禍,各行所見,不亦善乎。」時京城危懼,衣冠咸欲遠徙,後皆流離外難,百不一存。



 重除吏部尚書。太尉沈慶之深慮危禍,閉門不通賓客,嘗遣左右范羨詣興宗屬事。興宗謂羨曰:「公閉門絕客,以避悠悠請託耳,身非有求,何為見拒。」還造慶之,慶之遣羨報命,要興宗令往。興宗因說之曰:「先帝雖無功於天下,要能定平凶逆,
 在位十一年,以道晏駕。主上紹臨,四海清謐,即位正是舉止違衷,小小得失耳,亦謂春秋尚富,進德可期。而比者所行,人倫道盡。今所忌憚,唯在於公;百姓喁喁,無復假息之望,所冀正在公一人而已。若復坐視成敗者,非唯身禍不測,四海重責,將有所歸。公威名素著,天下所服,今舉朝遑遑,人人危怖,指麾之日,誰不景從;如其不斷,旦暮禍及。僕者昔佐貴府,蒙眷異常,故敢盡言,願公思為其計。」慶之曰:「僕皆日前,慮不復自保,但盡忠奉國,始終
 以之,正當委天任命耳。加老罷私門,兵力頓闕,雖有其意,事亦無從。」興宗曰:「當今懷謀思奮者,非要富貴,求功賞,各欲免死朝夕耳。殿內將帥,正聽外間消息,若一人唱首,則俯仰可定。況公威風先著,統戎累朝,諸舊部曲,布在宮省,宋越、譚金之徒,出公宇下,並受生成;攸之、恩仁,公家口子弟耳,誰敢不從。且公門徒義附,並三吳勇士,宅內奴僮,人有數百。陸攸之今入東討賊,又大送鎧仗,在青溪未發。



 攸之公之鄉人,驍勇有膽力,取其器仗,
 以配衣宇下,使攸之率以前驅,天下之事定矣。僕在尚書中,自當率百僚案前世故事,更簡賢明,以奉社稷。昔太甲罪不加民,昌邑虐不及下,伊尹、霍光猶成大事,況今蒼生窘急,禍百往代乎。又朝廷諸所行造,民間皆云公悉豫之。今若沈疑不決,當有先公起事者,公亦不免附從之禍。



 車駕屢幸貴第,醉酣彌留,又聞屏左右獨入閣內,此萬世一時,機不可失。僕荷眷深重,故吐去梯之言,宜詳其禍福。」慶之曰:「深感君無已。意此事大,非僕所
 能行,事至故當抱忠以沒耳。」頃之,慶之果以見忌致禍。



 時領軍王玄謨大將有威名,邑里訛言云已見誅,市道喧擾。玄謨典簽包法榮者,家在東陽,興宗故郡民也,為玄謨所信,見使至,興宗因胃曰:「領軍殊當憂懼。」



 法榮曰:「領軍比日殆不復食,夜亦不眠,常言收已在門,不保俄頃。」興宗曰:「領軍憂懼,當為方略,那得坐待禍至。」初,玄謨舊部曲猶有三千人,廢帝頗疑之,徹配監者。玄謨太息深怨,啟留五百人巖山營墓,事猶未畢,少帝欲獵,又悉
 喚還城。巖兵在中堂,興宗勸以此眾舉事,曰:「當今以領軍威名,率此為朝廷唱始,事便立克。領軍雖復失腳,自可乘輿處分。禍殆不測,勿失事機。君還,可白領軍如此。」玄謨遣法榮報曰:「此亦未易可行,期當不泄君言。」太宗踐祚,玄謨責所親故吏郭季產、女婿韋希真等曰:「當艱難時,周旋輩無一言相扣發者。」



 季產曰:「蔡尚書令包法榮所道,非不會機,但大事難行爾,季產言亦何益。」玄謨有慚色。



 右衛將軍劉道隆為帝所寵信,專統禁兵,乘輿
 嘗夜幸著作佐郎江斅宅,興宗馬車從道隆從車後過,興宗謂曰:「劉公!比日思一閑寫。」道隆深達此旨,掐興宗手曰:「蔡公!勿多言。」帝每因朝宴,捶毆群臣,自驃騎大將軍建安王休仁以下,侍中袁愍孫等,咸見陵曳,唯興宗得免。頃之,太宗定大事。是夜,廢帝橫尸在大醫閣口,興宗謂尚書右僕射王景文曰:「此雖凶悖,要是天下之主,宜使喪禮粗足。



 若直如此,四海必將乘人。」



 時諸方並舉兵反,國家所保,唯丹陽、淮南數郡,其間諸縣,或已應賊。
 東兵已至永世,宮省危懼,上集群臣以謀成敗。興宗曰:「今普天圖逆,人有異志,宜鎮之以靜,以至信侍人。比者逆徒親戚,布在宮省,若繩之以法,則土崩立至,宜明罪不相及之義。物情既定,人有戰心,六軍精勇,器甲犀利,以待不習之兵,其勢相萬耳。願陛下勿憂。」上從之。



 加游擊將軍,未拜,遷尚書右僕射,尋領衛尉,又領兗州大中正。太宗謂興宗曰:「諸處未定,殷琰已復同逆。頃日人情云何?事當濟不?」興宗曰:「逆之與順,臣無以辨。今商旅斷絕,
 而米甚豐賤,四方雲合,而人情更安,以此卜之,清蕩可必。但臣之所憂,更在事後,猶羊公言既平之後,方當勞聖慮耳。」尚書褚淵以手板築興宗,興宗言之不已,上曰:「如卿言。」赭圻平,函送袁顗首,敕從登南掖門樓觀之,興宗漼然流涕,上不悅。事平,封興宗始昌縣伯,食邑五百戶;固讓不許,封樂安縣伯,邑三百戶,國秩吏力,終以不受。



 時殷琰據壽陽為逆,遣輔國將軍劉勔攻圍。四方既平,琰嬰城固守,上使中書為詔譬琰,興宗曰:「天下既定,
 是琰思過之日,陛下宜賜手詔數行以相私慰。今直中書為詔,彼必疑謂非真,未是所以速清方難也。」不從。琰得詔,謂劉勔詐造,果不敢降。攻戰經時,久乃歸順。



 先徐州刺史薛安都據彭城反,後遣使歸順。泰始二年冬,遣張永率軍迎之。興宗曰:「安都遣使歸順,此誠不虛。今宜撫之以和,即安所蒞,不過須單使及咫尺書耳。若以重兵迎之,勢必疑懼,或能招引北虜,為患不測。叛臣釁重,必宜翦戮,則比者所宥,亦已弘矣。況安都外據強地,密
 邇邊關,考之國計,憂宜馴養。如其遂叛,將生旰食之憂。彭城險固,兵強將勇,圍之既難,攻不可拔,疆塞之虞,二三宜慮,臣為朝廷憂之。」時張永已行,不見從。安都聞大軍過淮,嬰城自守,要取索虜。永戰大敗,又值寒雪,死者十八九,遂失淮北四州。其先見如此。初,永敗問至,上在乾明殿,先召司徒建安王休仁,又召興宗,謂休仁曰:「吾慚蔡僕射。」



 以敗書示興宗,曰:「我愧卿。」



 三年春,出為使持節、都督郢州諸軍事、安西將軍、郢州刺史。坐詣尚書切
 論以何始真為咨議參軍,初不被許,後又重陳,上怒,貶號平西將軍,尋又復號。初,吳興丘珍孫言論常侵興宗。珍孫子景先,人才甚美,興宗與之周旋。及景先為鄱陽郡,值晉安王子勛為逆,轉在竟陵,為吳喜所殺。母老女稚,流離夏口。興宗至郢州,親自臨哭,致其喪柩家累,令得東還。在任三年,遷鎮東將軍、會稽太守,加散騎常侍,尋領兵置佐,加都督會稽、東陽、新安、永嘉、臨海五郡諸軍事,給鼓吹一部。會稽多諸豪右,不遵王憲。又幸臣近
 習,參半宮省,封略山湖,妨民害治。



 興宗皆以法繩之。會土全實,民物殷阜,王公妃主,邸舍相望,橈亂在所,大為民患,子息滋長,督責無窮。興宗悉啟罷省。又陳原諸逋負,解遣雜役,並見從。三吳舊有鄉射禮,久不復修,興宗行之,禮儀甚整。先是元嘉中,羊玄保為郡,亦行鄉射。



 太宗崩,興宗與尚書令袁粲、右僕射褚淵、中領軍劉勔、鎮軍將軍沈攸之同被顧命。以興宗為使持節、都督荊湘雍益梁寧南北秦八州諸軍事、征西將軍、開府儀同三
 司、荊州刺史,加班劍二十人,常侍如故。被徵還都。時右軍將軍王道隆任參內政,權重一時,躡履到前,不敢就席,良久方去,竟不呼坐。元嘉初,中書舍人秋當詣太子詹事王曇首,不敢坐。其後中書舍人王弘為太祖所愛遇,上謂曰:「卿欲作士人,得就王球坐,乃當判耳。殷、劉並雜,無所知也。若往詣球,可稱旨就席。」球舉扇曰:「若不得爾。」弘還,依事啟聞,帝曰:「我便無如此何。」五十年中,有此三事。道隆等以興宗強正,不欲使擁兵上流,改為中書
 監、左光祿大夫,開府儀同三司、常侍如故,固辭不拜。



 興宗幼立風概,家行尤謹,奉宗姑,事寡嫂,養孤兄子,有聞於世。太子左率王錫妻范,聰明婦人也,有才藻學見,與錫弟僧達書,詰讓之曰:「昔謝太傅奉嫂王夫人如慈母,今蔡興宗亦有恭和之稱。」其為世所重如此。妻劉氏早卒,一女甚幼,外甥袁顗始生彖而妻劉氏亦亡。興宗姊,即顗母也,一孫一侄,躬自撫養,年齒相比,欲為婚姻,每見興宗,輒言此意。



 大明初,詔興宗女與南平王敬猷婚,
 興宗以姊生平之懷,屢經陳啟,答曰:「卿諸人欲各行己意,則國家何由得婚?且姊言豈是不可違之處邪?」舊意既乖,彖亦他娶。其後彖家好不終,顗又禍敗,彖等淪廢當時,孤微理盡。敬猷遇害,興宗女無子嫠居,名門高胄,多欲結姻,明帝亦敕適謝氏,興宗並不許,以女適彖。



 北地傅隆與廓相善,興宗修父友敬。



 泰豫元年,薨,時年五十八。遺令薄葬,奏還封爵。追贈後授,子景玄固辭不受,又奏還封,表疏十餘上,見許。詔曰:「景玄表如此。故散騎
 常侍、中書監、左光祿大夫、開府儀同三司、樂安縣開國伯興宗,忠恪立朝,謀猷宣著,往屬時難,勳亮帷幄,錫珪分壤,實允通誥。而懇誠慊訴,備彰存沒,廉概素情,有潔聲軌。



 景玄固陳先志,良以惻然。雖彞典宜全,而哀款難奪,可特申不瞑之請,永矜克讓之風。」初,興宗為郢州府參軍,彭城顏敬以式卜曰:「亥年當作公,官有大字者,不可受也。」及有開府之授,而太歲在亥,果薨於光祿大夫之號焉。文集傳於世。



 景玄雅有父風,為中書郎,晉陵太
 守,太尉從事中郎。升明末卒。



 史臣曰:世重清談,士推素論,蔡廓雖業力弘正,而年位未高,一世名臣,風格皆出其下。及其固辭銓衡,恥為志屈,豈不知選錄同體,義無偏斷乎!良以主暗時難,不欲居通塞之任也。遠矣哉!



\end{pinyinscope}