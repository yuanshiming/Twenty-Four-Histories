\article{卷九十二列傳第五十二 良吏}

\begin{pinyinscope}

 高祖起自匹庶,知民事艱難,及登庸作宰,留心吏職,而王略外舉,未遑內務。



 奉師之費,日耗千金,播茲寬簡,雖所未暇,而絀華屏欲,以儉抑身,左右無幸謁之私,閨房
 無文綺之飾,故能戎車歲駕,邦甸不憂。太祖幼而寬仁,入纂大業,及難興陜方,六戎薄伐,命將動師,經略司、兗,費由府實,役不及民。自此區宇宴安,方內無事,三十年間,氓庶蕃息,奉上供徭,止於歲賦,晨出莫歸,自事而已。



 守宰之職,以六期為斷,雖沒世不徙,未及曩時,而民有所係,吏無茍得。家給人足,即事雖難,轉死溝渠,於時可免。凡百戶之鄉,有市之邑,歌謠舞蹈,觸處成群,蓋宋世之極盛也。暨元嘉二十七年,北狄南侵,戎役大起,傾資
 掃蓄,猶有未供,於是深賦厚斂,天下騷動。自茲至於孝建,兵連不息,以區區之江東,地方不至數千里,戶不盈百萬,薦之以師旅,因之以兇荒,宋氏之盛,自此衰矣。



 晉世諸帝,多處內房,朝宴所臨,東西二堂而已。孝武末年,清暑方構,高祖受命,無所改作,所居唯稱西殿,不制嘉名;太祖因之,亦有合殿之稱。及世祖承統,制度奢廣,犬馬餘菽粟,土木衣綈繡,追陋前規,更造正光、玉燭、紫極諸殿。



 雕欒綺節,珠窗網戶,嬖女幸臣,賜傾府藏,竭四海
 不供其欲,單民命未快其心。



 太宗繼阼,彌篤浮侈,恩不恤下,以至橫流。蒞民之官,遷變歲屬,灶不得黔,席未暇煖,蒲、密之化,事未易階。豈徒吏不及古,民偽於昔,蓋由為上所擾,致治莫從。今採其風跡粗著者,以為《良吏篇》云。



 王鎮之,字伯重,琅邪臨沂人,徵士弘之兄也。曾祖暠,晉驃騎將軍。祖耆之,中書郎。父隨之,上虞令。鎮之初為琅邪王衛軍行參軍,出補剡、上虞令,並有能名。內史謝輶
 請為山陰令,復有殊績。遷衛軍參軍,本國郎中令,加寧朔將軍。桓玄輔晉,以為大將軍錄事參軍。時三吳饑荒,遣鎮之銜命賑恤,而會稽內史王愉不奉符旨,鎮之依事糾奏。愉子綏,玄之外甥,當時貴盛,鎮之為所排抑,以母老求補安成太守。及玄敗,玄將苻宏寇亂郡境,鎮之拒戰彌年,子弟五人,並臨陣見殺。



 母憂去職,在官清潔,妻子無以自給,乃棄家致喪還上虞舊基。畢,為子標之求安復令,隨子之官。服闋,為征西道規司馬、南平太守。
 徐道覆逼江陵,加鎮之建威將軍,統檀道濟、到彥之等討道覆,以不經將帥,固辭,不見聽。既而前軍失利,白衣領職,尋復本官。以討道覆功,封華容縣五等男,徵廷尉。晉穆帝何皇后山陵,領將作大匠。遷御史中丞,秉正不撓,百僚憚之。



 出為使持節、都督交廣二州諸軍事、建威將軍、平越中郎將、廣州刺史。高祖謂人曰:「王鎮之少著清績,必將繼美吳隱之。嶺南之弊,非此不康也。」在鎮不受俸祿,蕭然無所營。去官之日,不異始至。高祖初建相
 國府,以為諮議參軍,領錄事。善於吏職,嚴而不殘。遷宋臺祠部尚書。高祖踐阼,鎮之以腳患自陳,出為輔國將軍、琅邪太守,遷宣訓衛尉,領本州大中正。永初三年,卒官,時年六十六。



 弟弘之,在《隱逸傳》。



 杜慧度,交趾朱鸘人也。本屬京兆。曾祖元,為寧浦太守,遂居交趾。父瑗,字道言,仕州府為日南、九德、交趾太守。初,九真太守李遜父子勇壯有權力,威制交土,聞刺史騰遁之當至,分遣二子斷遏水陸津要。瑗收眾斬遜,州
 境獲寧。除龍驤將軍。遁之在州十餘年,與林邑累相攻伐。遁之將北還,林邑王范胡達攻破日南、九德、九真三郡,遂圍州城。時遁之去已遠,瑗與第三子玄之悉力固守,多設權策,累戰,大破之。追討於九真、日南。連捷,故胡達走還林邑。乃以瑗為龍驤將軍、交州刺史。義旗進號冠軍將軍。盧循竊據廣州,遣使通好,瑗斬之。義熙六年,年八十四,卒,追贈右將軍,本官如故。



 慧度,瑗第五子也。初為州主簿,流民督護,遷九真太守。瑗卒,府州綱佐以
 交土接寇,不宜曠職,共推慧度行州府事,辭不就。七年,除使持節、督交州諸軍事、廣武將軍、交州刺史。詔書未至,其年春,盧循襲破合浦,徑向交州。慧度乃率文武六千人距循於石碕,交戰,禽循長史孫建之。循雖敗,餘黨猶有三千人,皆習練兵事。李子遜李弈、李脫等奔竄石碕,盤結俚、獠,各有部曲。循知弈等與杜氏有怨,遣使招之,弈等引諸俚帥眾五六千人,受循節度。六月庚子,循晨造南津,命三軍入城乃食。慧度悉出宗族私財,以充
 勸賞。弟交趾太守慧期、九真太守章民並督率水步軍,慧度自登高艦,合戰,放火箭雉尾炬,步軍夾兩岸射之。循眾艦俱然,一時散潰,循中箭赴水死。斬循及父嘏,并循二子,親屬錄事參軍阮靜、中兵參軍羅農夫、李脫等,傳首京邑。封慧度龍編縣侯,食邑千戶。



 高祖踐阼,進號輔國將軍。其年,率文武萬人南討林邑,所殺過半,前後被抄略,悉得還本。林邑乞降,輸生口、大象、金銀、古貝等,乃釋之。遣長史江悠奉表獻捷。慧度布衣蔬食,儉約質
 素,能彈琴,頗好《莊》、《老》。禁斷淫祀,崇修學校。歲荒民饑,則以私祿賑給。為政纖密,有如治家,由是威惠沾洽,姦盜不起,乃至城門不夜閉,道不拾遺。少帝景平元年,卒,時年五十,追贈左將軍。



 以慧度長子員外散騎侍郎弘文為振威將軍、刺史。初,高祖北征關、洛,慧度板弘文為鷹揚將軍,流民督護,配兵三千,北係大軍。行至廣州,關、洛已平,乃歸。統府板弘文行九真太守。及繼父為刺史,亦以寬和得眾,襲爵龍編侯。太祖元嘉四年,以廷尉王徽
 為交州刺史,弘文就徵。會得重疾,牽以就路,親舊見其患篤,勸表待病愈。弘文曰:「吾世荷皇恩,杖節三世,常欲投軀帝庭,以報所荷。況親被徵命,而可宴然者乎!如其顛沛,此乃命也。」弘文母既年老,見弘文輿疾就路,不忍分別,相與俱行。到廣州,遂卒。臨死,遣弟弘猷詣京,朝廷甚哀之。



 徐豁,字萬同,東莞姑幕人也,中散大夫廣兄子。父邈,晉太子左衛率。豁晉安帝隆安末為太學博士。桓玄輔政,
 為中外都督,豁議:「致敬唯內外武官,太宰、司徒,並非軍職,則琅邪王不應加敬。」玄諷中丞免豁官。玄敗,以為祕書郎,尚書倉部郎,右軍何無忌功曹,仍為鎮南參軍;又祠部,永世令,建武司馬,中軍參軍,尚書左丞。永初初,為徐羨之鎮軍司馬,尚書左丞,山陰令。歷二丞三邑,精練明理,為一世所推。



 元嘉初,為始興太守。三年,遣大使巡行四方,并使郡縣各言損益。豁因此表陳三事,其一曰:「郡大田,武吏年滿十六,便課米六十斛,十五以下至十
 三,皆課米三十斛,一戶內隨丁多少,悉皆輸米。且十三歲兒,未堪田作,或是單迥,無相兼通,年及應輸,便自逃逸,既遏接蠻、俚,去就益易。或乃斷截支體,產子不養,戶口歲減,實此之由。謂宜更量課限,使得存立。今若減其米課,雖有交損,考之將來,理有深益。」其二曰:「郡領銀民三百餘戶,鑿坑採砂,皆二三丈。功役既苦,不顧崩壓,一歲之中,每有死者。官司檢切,猶致逋違,老少相隨,永絕農業;千有餘口,皆資他食,豈唯一夫不耕,或受其饑而
 已。所以歲有不稔,便致甚困。尋臺邸用米,不異於銀,謂宜準銀課米,即事為便。」其三曰:「中宿縣俚民課銀,一子丁輸南稱半兩。尋此縣自不出銀,又俚民皆巢居鳥語,不閑貨易之宜,每至買銀,為損已甚。又稱兩受入,易生姦巧,山俚愚怯,不辨自申,官所課甚輕,民以所輸為劇。今若聽計丁課米,公私兼利。」



 在郡著績,太祖嘉之。下詔曰:「始興太守豁,潔己退食,恪居在官,政事脩理,惠澤沾被。近嶺南荒弊,郡境尤甚,拯恤有方,濟厥饑饉,雖古之
 良守,蔑以尚焉。宜蒙褒賁,以旌清績,可賜絹二百匹,穀千斛。」五年,以為持節、督廣交二州諸軍事、寧還將軍、平越中郎將、廣州刺史。未拜,卒,時年五十一。太祖又下詔曰:「豁廉清勤恪,著稱所司,故擢授南服,申其才志。不幸喪殞,朕甚悼之。



 可賜錢十萬,布百匹,以營葬事。」



 陸徽,字休猷,吳郡吳人也。郡辟命主簿,仍除衛軍、車騎二府參軍,揚州主簿,王弘衛將軍主簿,除尚書都官郎,出補建康令。清平無私,為太祖所善,遷司徒左西掾。元
 嘉十四年,為始興太守。明年,仍除使持節、交廣二州諸軍事、綏遠將軍、平越中郎將、廣州刺史。清名亞王鎮之,為士民所愛詠。上表薦士曰:「臣聞陵雪褒潁,貞柯必振;尊風賞流,清原斯挹。是以衣囊揮譽於西京,折轅延高於東帝。伏見廣州別駕從事史朱萬嗣,年五十三,字少豫,理業沖夷,秉操純白,行稱私庭,能著官政。雖氏非世祿,宦無通資,而隨牒南服,位極僚首,九綜州綱,三端府職,頻掌蕃機,屢績符守。年暨知命,廉尚愈高,冰心與貪
 流爭激,霜情與晚節彌茂。歷宰金山,家無寶鏤之飾;連組珠海,室靡璫珥之珍。確然守志,不求聞達,實足以澄革汙吏,洗鏡貪氓。臣謬忝司牧,任專萬里,雖情祗慎擢,才闕豪露,敢罄愚陋,舉其所知。如得提名禮闈,抗跡朝省,摶嶺表之清風,負冰宇之潔望,則恩融一臣,而施光萬物。敢緣天澤雲行,時德雨施,每甄外州,榮加遠國。



 是以獻其瞽言,希垂聽覽。」



 二十一年,徵以為南平王鑠冠軍司馬、長沙內史,行湘州府事。母憂去職。張尋、趙廣為
 亂於益州,兵寇之餘,政荒民擾。二十三年,乃追徽為持節、督益寧二州諸軍事、寧朔將軍、益州刺史。隱恤有方,威惠兼著,寇盜靜息,民物殷阜,蜀土安說,至今稱之。二十九年,卒,時年六十二。身亡之日,家無餘財。太祖甚痛惜之,詔曰:「徽厲志廉潔,歷任恪勤,奉公盡誠,克己無倦。褒榮未申,不幸夙殞,言念在懷,以為傷恨。可贈輔國將軍,本官如故。」賜錢十萬,米二百斛。謚曰簡子。子睿,正員外郎。弟展,臧質車騎長史、尋陽太守,質敗,從誅。



 阮長之,字茂景,陳留尉氏人也。祖思曠,金紫光祿大夫。父普,驃騎諮議參軍。長之年十五喪父,有孝性,哀感傍人。服除,蔬食者猶積載。閑居篤學,未嘗有惰容。初為諸府參軍,除員外散騎侍郎。母老,求補襄垣令,督郵無禮,鞭之,去職。尋補廬陵王義真車騎行正參軍,平越長史,東莞太守。入為尚書殿中郎,出為武昌太守。時王弘為江州,雅相知重,引為車騎從事中郎。入為太子中舍人,中書侍郎,以母老,固辭朝直,補彭城王義康平北諮議
 參軍。元嘉九年,遷臨川內史,以南土卑濕,母年老,非所宜,辭不就。十一年,復除臨海太守。至郡少時而母亡,葬畢,不勝憂,十四年,卒,時年五十九。



 時郡縣田祿,芒種為斷,此前去官者,則一年秩祿皆入前人;此後去官者,則一年秩祿皆入後人。始以元嘉末改此科,計月分祿。長之去武昌郡,代人未至,以芒種前一日解印綬。初發京師,親故或以器物贈別,得便緘錄,後歸,悉以還之。



 在中書省直,夜往鄰省,誤著履出閣,依事自列門下;門下以
 暗夜人不知,不受列。



 長之固遣送之,曰:「一生不侮暗室。」前後所蒞官,皆有風政,為後人所思。宋世言善治者,咸稱之。子師門,原鄉令。



 江秉之,字玄叔,濟陽考城人也。祖逌,晉太常。父纂,給事中。秉之少孤,弟妹七人,並皆幼稚,撫育姻娶,罄其心力。初為劉穆之丹陽前軍府參軍。高祖督徐州,轉主簿,仍為世子中軍參軍。宋受禪,隨例為員外散騎侍郎,補太子詹事丞。



 少帝即位,入為尚書都官郎,出為永世、烏程
 令,以善政著名東土。征建康令,為治嚴察,京邑肅然。殷景仁為領軍,請為司馬。復出為山陰令,民戶三萬,政事煩擾,訟訴殷積,階庭常數百人,秉之御繁以簡,常得無事。宋世唯顧覬之亦以省務著績,其餘雖復刑政修理,而未能簡事。以在縣有能,遷補新安太守。



 元嘉十二年,轉在臨海,並以簡約見稱。所得祿秩,悉散之親故,妻子常饑寒。



 人有勸其營田者,秉之正色曰:「食祿之家,豈可與農人競利!」在郡作書案一枚,及去官,留以付庫。十七
 年,卒,時年六十。



 子徽,尚書都官郎,吳令。元兇殺徐湛之,徽以黨與見誅。子謐,升明末為尚書吏部郎。元嘉初,太祖遣大使巡行四方,兼散騎常侍孔默之、王歆之等上言:「宣威將軍、陳南頓二郡太守李元德,清勤均平,姦盜止息。彭城內史魏恭子,廉恪脩慎,在公忘私,安約守儉,久而彌固。前宋縣令成浦,治政寬濟,遺詠在民。



 前鮦陽令李熙國,在事有方,民思其政。山桑令何道,自少清廉,白首彌厲。應加褒齎,以勸於後。」乃進元德號寧朔將軍,
 恭子賜絹五十匹,穀五百斛;浦、熙國、道各賜絹三十匹,穀二百斛。



 王歆之,字叔道,河東人也。曾祖愆期,有名晉世,官至南蠻校尉。祖尋之,光祿大夫。父肇之,豫章公相。歆之被遇於太祖,歷顯官左民尚書,光祿大夫,卒官。元嘉九年,豫州刺史長沙王義欣上言:「所統威遠將軍、北譙梁二郡太守關中侯申季歷,自奉職邦畿,於茲五年,信惠並宣,威化兼著,外清姦暴,內輯民黎,役賦均平,閭井齊肅,綏
 穆初附,招攜荒遠,郊境之外,仰澤懷風,爵賞之授,績能是顯,宜升階秩,以崇獎勸。」進號寧朔將軍。



 其後晉壽太守郭啟玄亦有清節,卒官。元嘉二十八年,詔曰:「故綏遠將軍、晉壽太守郭啟玄往銜命虜庭,秉意不屈,受任白水,盡勤靡懈,公奉私餼,纖毫弗納,布衣蔬食,飭躬惟儉。故超授顯邦,以甄廉績。而介誠苦節,終始匪貳,身死之日,妻子凍餒,志操殊俗,良可哀悼。可賜其家穀五百斛。」



 時有北地傅僧祐、潁川陳氏、高平張祐,並以吏才見知。
 僧祐事在《臧燾傳》。



 氏為吳令,善發姦伏,境內以為神明。祐祖父湛,晉孝武世,以才學為中書侍郎,光祿勛。祐歷臨安、武康、錢塘令,並著能名,宋世言長吏者,以三人為首。元嘉中,高平太守潘詞,有清節。子亮為昌慮令,亦著廉名,大明中,為徐州刺史劉道隆所表。世祖世,吳郡陸法真歷官有清節,嘗為劉秀之安北錄事參軍。泰山羊希與安北諮議參軍孫詵書曰:「足下同僚似有陸錄事者,此生東南名地,又張玄外孫,持身至清,雅有志節。年
 高官下,秉操不衰,計當日夕相與申意。」太宗初,為南海太守,卒官。



 太宗世,琅邪王悅,亦蒞官清正見知。悅字少明,晉右將軍羲之曾孫也。父靖之,官至司徒左長史。靖之為劉穆之所厚,就穆之求侍中,如此非一。穆之曰:「卿若不求,久自得也。」遂不果。悅泰始中,為黃門郎,御史中丞。上以其廉介,賜良田五頃。遷尚書吏部郎,侍中,在門下,盡其心力。五年,卒官,追贈太常。



 初,悅為侍中,檢校御府、太官、太醫諸署,得姦巧甚多。及悅死,眾咸謂諸署詋
 詛之,上乃收典掌者十餘人,桎梏云送淮陰,密令渡瓜步江,投之中流。



 史臣曰:夫善政之於民,猶良工之於埴也,用功寡而成器多。漢世戶口殷盛,刑務簡闊,郡縣治民,無所橫擾,勸賞威刑,事多專斷,尺一詔書,希經邦邑,龔、黃之化,易以有成。降及晚代,情偽繁起,民減昔時,務多前世,立績垂風,艱易百倍。若以上古之化,治此世之民,今吏之良,撫前代之俗,則武城弦歌,將有未暇;淮陽臥治,如或可勉。
 未必今才陋古,蓋化有淳薄也。



\end{pinyinscope}