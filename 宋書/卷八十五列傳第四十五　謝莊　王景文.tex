\article{卷八十五列傳第四十五 謝莊 王景文}

\begin{pinyinscope}

 謝莊,字希逸,陳郡陽夏人,太常弘微子也。年七歲,能屬文,通《論語》。



 及長,韶令美容儀,太祖見而異之,謂尚書僕射殷景仁、領軍將軍劉湛曰:「藍田出玉,豈虛也哉!」初為
 始興王浚後軍法曹行參軍,轉太子舍人,廬陵王文學,太子洗馬,中舍人,廬陵王紹南中郎諮議參軍。又轉隨王誕後軍諮議,並領記室。分左氏《經傳》,隨國立篇,製木方丈,圖山川土地,各有分理,離之則州別郡殊,合之則宇內為一。元嘉二十七年,索虜寇彭城,虜遣尚書李孝伯來使,與鎮軍長史張暢共語,孝伯訪問莊及王徽,其名聲遠布如此。二十九年,除太子中庶子。時南平王鑠獻赤鸚鵡,普詔群臣為賦。太子左衛率袁淑文冠當時,
 作賦畢,齎以示莊;莊賦亦竟,淑見而歎曰:「江東無我,卿當獨秀。我若無卿,亦一時之傑也。」遂隱其賦。



 元凶弒立,轉司徒左長史。世祖入討,密送檄書與莊,令加改治宣布。莊遣腹心門生具慶奉啟事密詣世祖曰:「賊劭自絕於天,裂冠毀冕,窮弒極逆,開闢未聞,四海泣血,幽明同憤。奉三月二十七日檄,聖迹昭然,伏讀感慶。天祚王室,睿哲重光。殿下文明在嶽,神武居陜,肅將乾威,龔行天罰,滌社稷之仇,雪華夷之恥,使弛墜之構,更獲締造,垢
 辱之氓,復得明目。伏承所命,柳元景、司馬文恭、宗愨、沈慶之等精甲十萬,已次近道。殿下親董銳旅,授律繼進。荊、鄢之師,岷、漢之眾,舳艫萬里,旌旆虧天,九土冥符,群后畢會。今獨夫醜類,曾不盈沴,自相暴殄,省闥橫流,百僚屏氣,道路以目。檄至,輒布之京邑,朝野同欣,里頌途歌,室家相慶,莫不望景聳魂,瞻雲佇足。先帝以日月之光,照臨區宇,風澤所漸,無幽不洽。況下官世荷寵靈,叨恩踰量,謝病私門,幸免虎口,雖志在投報,其路無由。今
 大軍近次,永清無遠,欣悲踴躍,不知所裁。」



 世祖踐阼,除侍中。時索虜求通互市,上詔群臣博議。莊議曰:「臣愚以為獯獫棄義,唯利是視,關市之請,或以覘國,順之示弱,無明柔遠,距而觀釁,有足表彊。且漢文和親,豈止彭陽之寇;武帝脩約,不廢馬邑之謀。故有餘則經略,不足則閉關。何為屈冠帶之邦,通引弓之俗,樹無益之軌,招塵點之風。交易爽議,既應深杜;和約詭論,尤宜固絕。臣庸管多蔽,豈識國儀,恩誘降逮,敢不披盡。」



 時驃騎將軍竟
 陵王誕當為荊州,徵丞相、荊州刺史南郡王義宣入輔,義宣固辭不入,而誕便克日下船。莊以:「丞相既無入志,驃騎發便有期,如似欲相逼切,於事不便。」世祖乃申誕發日,義宣竟亦不下。



 上始踐阼,欲宣弘風則,下節儉詔書,事在《孝武本紀》。莊慮此制不行,又言曰:「詔云『貴戚競利,興貨廛肆者,悉皆禁制』。此實允愜民聽。其中若有犯違,則應依制裁糾;若廢法申恩,便為令有所屈。此處分伏願深思,無緣明詔既下,而聲實乖爽。臣愚謂大臣在
 祿位者,尤不宜與民爭利,不審可得在此詔不?拔葵去織,實宜深弘。」



 孝建元年,遷左衛將軍。初,世祖嘗賜莊寶劍,莊以與豫州刺史魯爽送別。爽後反叛,世祖因宴集,問劍所在,答曰:「昔以與魯爽別,竊為陛下杜郵之賜。」



 上甚說,當時以為知言。于時搜才路狹,乃上表曰:臣聞功照千里,非特燭車之珍;德柔鄰國,豈徒秘璧之貴,故《詩》稱殄悴,《誓》述榮懷,用能道臻無積,化至恭己。伏惟陛下膺慶集圖,締宇開縣,夕爽選政,昃旦調風,采言廝輿,觀
 謠仄遠,斯實辰階告平,頌聲方製。臣竊惟隆陂所漸,治亂之由,何嘗不興資得才,替因失士。故楚書以善人為寶,《虞典》以則哲為難。



 進選之軌,既弛中代,登造之律,未闡當今。必欲崇本康務,庇民濟俗,匪更怗懘,奚取九成。升歷中陽,英賢起於徐、沛;受籙白水,茂異出於荊、宛。寧二都智之所產,七諲愚之所集,實遇與不遇,用與不用耳。



 今大道光亨,萬務俟德,而九服之曠,九流之艱,提鈞懸衡,委之選部。一人之鑒易限,而天下之才難原;以易
 限之鑒,鏡難原之才,使國罔遺授,野無滯器,其可得乎?昔公叔與僎同升,管仲取臣於盜,趙文非親士疏嗣,祁奚豈諂讎比子,茹茅以彙,作範前經,舉爾所知,式昭往牒。且自古任薦,賞罰弘明,成子舉三哲而身致魏輔,應侯任二士而已捐秦相,臼季稱冀缺而疇以田采,張勃進陳湯而坐以褫爵。此先事之盛準,亦後王之彞鑒。如臣愚見,宜普命大臣,各舉所知,以付尚書,依分銓用。若任得其才,舉主延賞;有不稱職,宜及其坐。重者免黜,輕
 者左遷,被舉之身,加以禁錮,年數多少,隨愆議制。若犯大辟,則任者刑論。



 又政平訟理,莫先親民,親民之要,實歸守宰。故黃霸治潁川累稔,杜畿居河東歷載,或就加恩秩,或入崇輝寵。今蒞民之職,自非公私必應代換者,宜遵六年之制,進獲章明庸墮,退得民不勤擾。如此則下無浮謬之愆,上靡棄能之累,考績之風載泰,薪之歌克昌。臣生屬亨路,身漸鴻猷,遂得奉詔左右,陳愚於側,敢露芻言,懼氛恒典。



 有詔莊表如此,可付外詳議,事
 不行。其年,拜吏部尚書。莊素多疾,不願居選部,與大司馬江夏王義恭箋自陳,曰:下官凡人,非有達概異識,俗外之志,實因羸疾,常恐奄忽,故少來無意於人間,豈當有心於崇達邪。頃年乘事回薄,遂果饕非次,既足貽誚明時,又亦取愧朋友。前以聖道初開,未遑引退,及此諸夏事寧,方陳微請。款志未伸,仍荷今授,被恩之始,具披寸心,非惟在己知尤,實懼塵穢彞序。



 稟生多病,天下所悉,兩脅癖疾,殆與生俱,一月發動,不減兩三,每至一惡,
 痛來逼心,氣餘如綖。利患數年,遂成痼疾,吸吸惙惙,常如行尸。恒居死病,而不復道者,豈是疾痊,直以荷恩深重,思答殊施,牽課尪瘵,以綜所忝。眼患五月來便不復得夜坐,恆閉帷避風日,晝夜愍懵,為此不復得朝謁諸王,慶弔親舊,唯被敕見,不容停耳。此段不堪見賓,已數十日,持此苦生,而使銓綜九流,應對無方之訴,實由聖慈罔已,然當之信自苦劇。若才堪事任,而體氣休健,承寵異之遇,處自效之途,豈茍欲思閑辭事邪!家素貧弊,
 宅舍未立,兒息不免粗糲,而安之若命,寧復是能忘微祿,正以復有切於此處,故無復他願耳。今之所希,唯在小閑。



 下官微命,於天下至輕,在己不能不重。屢經披請,未蒙哀恕,良由誠淺辭訥,不足上感。



 家世無年,亡高祖四十,曾祖三十二,亡祖四十七,下官新歲便三十五,加以疾患如此,當復幾時見聖世,就其中煎憹若此,實在可矜。前時曾啟願三吳,敕旨云「都不須復議外出」。莫非過恩,然亦是下官生運,不應見一閑逸。今不敢復言此,
 當付之來生耳。但得保餘年,無復物務,少得養痾,此便是志願永畢。在衡門下有所懷,動止必聞,亦無假居職,患於不能裨補萬一耳。識淺才常,羸疾如此,孤負主上擢授之恩,私心實自哀愧。入年便當更申前請,以死自固。但庸近所訴,恐未能仰徹。公恩盼弘深,粗照誠懇,願侍坐言次,賜垂拯助,則苦誠至心,庶獲哀允。若不蒙降祐,下官當於何希冀邪?仰憑愍察,願不垂吝。



 三年,坐辭疾多,免官。大明元年,起為都官尚書,奏改定刑獄,曰:臣
 聞明慎用刑,厥存姬典;哀矜折獄,實暉呂命。罪疑從輕,既前王之格範;寧失弗經,亦列聖之恒訓。用能化致升平,道臻恭己。逮漢文傷不辜之罰,除相坐之令,孝宣倍深文之吏,立鞫訊之法,當是時也,號令刑存。陛下踐位,親臨聽訟,億兆相賀,以為無冤民矣。而比囹圄未虛,頌聲尚缺。臣竊謂五聽之慈,弗宣於宰物;三宥之澤,未洽於民謠。頃年軍旅餘弊,劫掠猶繁,監司計獲,多非其實。或規免咎,不慮國患,楚對之下,鮮不誣濫。身遭鈇金質之
 誅,家嬰孥戮之痛,比伍同閈,莫不及罪。是則一人罰謬,坐者數十。昔齊女告天,臨淄臺殞;教婦冤戮,東海愆陽,此皆符變靈祗,初咸景緯。臣近兼訊,見重囚八人,旋觀其初,死有餘罪,詳察其理,實並無辜。恐此等不少,誠可怵惕也。



 舊官長竟囚畢,郡遣督郵案驗,仍就施刑。督郵賤吏,非能異於官長,有案驗之名,而無研究之實。愚謂此制宜革。自今入重之囚,縣考正畢,以事言郡,并送囚身,委二千石親臨覈辯,必收聲吞釁,然後就戮。若二千
 石不能決,乃度廷尉。



 神州統外,移之刺史;刺史有疑,亦歸臺獄。必令死者不怨,生者無恨。庶鬻棺之諺,輟歎於終古;兩造之察,流詠於方今。臣學暗申、韓,才寡治術,輕陳庸管,懼乖國憲。



 上時親覽朝政,常慮權移臣下,以吏部尚書選舉所由,欲輕其勢力。二年,下詔曰:「八柄馭下,以爵為先;九德咸事,政典居首。銓衡治樞,興替攸寄。頃世以來,轉失厥序,徒秉國鈞,終貽權謗。今南北多士,勳勤彌積,物情善否,實繫斯任。官人之詠,維聖克允;則哲
 之美,粵帝所難。加澆季在俗,讓議成風,以一人之識,當群品之誚,望沈浮自得,庸可致乎!吏部尚書可依郎分置,并詳省閑曹。」



 又別詔太宰江夏王義恭曰:分選詔旦出,在朝論者,亦有同異。誠知循常甚易,改舊生疑。但吏部尚書由來與錄共選,良以一人之識,不辦洽通,兼與奪威權,不宜專一故也。前述宣先旨,敬從來奏,省錄作則,永貽後昆。自此選舉之要,唯由元、凱一人。若通塞乖衷,而訴達者鮮,且違令與物,理至隔閡。前王盛主,猶或
 難之,況在寡暗,尤見其短。



 又選官裁病,即嗟誚滿道,人之四體,會盈有虛,旬日之間,便至怨詈,況實有假託,不由寢頓者邪!一詣不前,貧苦交困,則兩邊致患,互不相體,校之以實,並有可哀。若職置二人,則無此弊。兼選曹樞要,歷代斯重,人經此職,便成貴塗,己心外議,咸不自限,故範曄、魯爽,舉兵滅門。以此言之,實由榮厚勢驅,殷繁所至。設可擬議此授,唯有數人,本積歲月,稍加引進,而理無前期,多生慮表;或嬰艱抱疾,事至回移。官人之
 任,決不可闕,一來一去,向人已周,非有黜責,已貴難賤;既成妨長,置之無所,盛衰遞襲,便是一段世臣相處之方。臣主生疑,所以彌覺此職,宜在降階。監令端右,足處時望,無人則闕,異於九流。今但直銓選部,有減前資。物情好猜,橫立別解,本旨向意,終不外宣。唯有從郎分置,視聽自改。選既輕先,民情已變,有堪其任,大展遷回。兼常之宜,以時稍進,本職非復重官可得,不須帶帖數過,居之盡無詒怪。



 自中分荊、揚,于時便有意於此,正訝改
 革不少,容生駭惑。爾來多年,欲至歲下處分,會何偃致故,應有親人,故近因此施行。本意詔文不得委悉,故復紙墨具陳。



 於是置吏部尚書二人,省五兵尚書,莊及度支尚書顧覬之並補選職。遷右衛將軍,加給事中。時河南獻舞馬,詔群臣為賦,莊所上其詞曰:天子馭三光,總萬宇,挹雲經之留憲,裁河書之遺矩。是以德澤上昭,天下漏泉,符瑞之慶咸屬,榮懷之應必躔。月晷呈祥,乾維效氣,賦景河房,承靈天駟,陵原郊而漸影,躍采淵而泳
 質,辭水空而南傃,去輪臺而東洎,乘玉塞而歸寶,奄芝庭而獻秘。及其養安騏校,進駕龍涓,輝大馭於國皁,賁上襄於帝閑,超益野而踰綠地,軼蘭池而轢紫燕。五王晦其術,十氏懵其玄,東門豈或狀,西河不能傳。



 既秣苞以均性,又佩蘅以崇躅,卷雄神於綺文,蓄奔容於帷燭,蘊鷫雲之銳景,戢追電之逸足,方疊熔於丹縞,亦聯規於朱駁。觀其雙璧應範,三封中圖,玄骨滿,燕室虛,陽理竟,潛策紆,汗飛赭,沫流朱。至於《肆夏》已升,《采齊》既薦,始
 徘徊而龍俯,終沃若而鸞眄,迎調露於飛鐘,赴承雲於驚箭,寫秦坰之彌塵,狀吳門之曳練,窮虞庭之蹈蹀,究遺野之環袨。若夫蹠實之態未卷,凌遠之氣方攄,歷岱野而過碣石,跨滄流而軼姑餘,朝送日於西阪,夕歸風於北都,尋瓊宮於倏瞬,望銀臺於須臾。



 若乃日宣重光,德星昭衍,國稱梁、岱佇蹕,史言壇場望踐。鄗上之瑞彰,江間之禎闡,榮鏡之運既臻,會昌之歷已辨,感五繇之程符,鑒群后之薦典。聖主將有事於東嶽,禮也。於是順
 斗極,乘次躔,戒懸日於昭旦,命月題於上年。騑騑翼翼,泛脩風而浮慶煙,肅肅雍雍,引八神而詔九仙。下齊郊而掩配林,集嬴里而降祊田,蒲軒次巘,瑄璧承巒,金檢茲發,玉牒斯刊,盛節之義洽,升中之禮殫,億兆悅,精祗歡,聆萬歲於曾岫,燭神光於紫壇。是以擊轅之蹈,撫埃之舞,相與而歌曰:「聳朝蓋兮泛晨霞,靈之來兮雲漢華。山有壽兮松有茂,祚神極兮貺皇家。」



 然後悟聖朝之績,號慶榮之烈,比盛乎天地,爭明乎日月,茂實冠於胥、庭,
 鴻名邁於勛、發。業底於告成,道臻乎報謁,巍巍乎,蕩蕩乎,民無得而稱焉。



 又使莊作《舞馬歌》,令樂府歌之。五年,又為侍中,領前軍將軍。于時世祖出行,夜還,敕開門。莊居守,以棨信或虛,執不奉旨,須墨詔乃開。上後因酒宴從容曰:「卿欲效郅君章邪?」對曰:「臣聞蒐巡有度,郊祀有節,盤于遊田,著之前誡。陛下今蒙犯塵露,晨往宵歸,容恐不逞之徒,妄生矯詐。臣是以伏須神筆,乃敢開門耳。」改領游擊將軍,又領本州大中正,晉安王子勛征虜長
 史、廣陵太守,加冠軍將軍。改為江夏王義恭太宰長史,將軍如故。六年,又為吏部尚書,領國子博士,坐選公車令張奇免官,事在《顏師伯傳》。



 時北中郎將新安王子鸞有盛寵,欲令招引才望,乃使子鸞板莊為長史,府尋進號撫軍,仍除長史、臨淮太守。未拜,又除吳郡太守。莊多疾,不樂去京師,復除前職。前廢帝即位,以為金紫光祿大夫。初,世祖寵姬殷貴妃薨,莊為誄云:「贊軌堯門。」引漢昭帝母趙婕妤堯母門事,廢帝在東宮,銜之。至是遣人
 詰責莊曰:「卿昔作殷貴妃誄,頗知有東宮不?」將誅之。或說帝曰:「死是人之所同,政復一往之苦,不足為深困。莊少長富貴,今且繫之尚方,使知天下苦劇,然後殺之未晚也。」帝然其言,繫於左尚方。太宗定亂,得出。及即位,以莊為散騎常侍、光祿大夫,加金章紫綬,領尋陽王師。頃之,轉中書令,常侍、王師如故。尋加金紫光祿大夫,給親信二十人,本官並如故。泰始二年,卒,時年四十六,追贈右光祿大夫,常侍如故,謚曰憲子。所著文章四百餘首,
 行於世。長子颺,晉平太守。女為順帝皇后,追贈金紫光祿大夫。



 王景文,琅邪臨沂人也。名與明帝諱同。祖穆,臨海太守。伯父智,少簡貴,有高名,高祖甚重之,常云:「見王智,使人思仲祖。」與劉穆之謀討劉毅,而智在焉。它日,穆之白高祖曰:「伐國,重事也,公云何乃使王智知?」高祖笑曰:「此人高簡,豈聞此輩論議。」其見知如此。為太尉諮議參軍,從征長安,留為桂陽公義真安西將軍司馬、天水太守。還
 為宋國五兵尚書,晉陵太守,加秩中二千石,封建陵縣五等子,追贈太常。父僧朗,亦以謹實見知。元嘉中,為侍中,勤於朝直,未嘗違惰。太祖嘉之,以為湘州刺史。世祖大明末,為尚書左僕射。太宗初,以后父為特進、左光祿大夫,又進開府儀同三司,固讓,乃加侍中、特進。尋薨,追贈開府,謚曰元公。



 景文出繼智,幼為從叔球所知。美風姿,好言理,少與陳郡謝莊齊名。太祖甚相欽重,故為太宗娶景文妹,而以景文名與太宗同。高祖第五女新安
 公主先適太原王景深,離絕,當以適景文,固辭以疾,故不成婚。起家太子太傅主簿,轉太子舍人,襲爵建陵子。出為江夏王義恭、始興王浚征北後軍二府主簿,武陵王文學,世祖撫軍記室參軍,南廣平太守,轉諮議參軍,仍度安北、鎮軍府,出為宣城太守。



 元凶弒立,以為黃門侍郎,未及就,世祖入討,景文遣間使歸款。以父在都邑,不獲致身,及事平,頗見嫌責,猶以舊恩,除南平王鑠司空長史,不拜。出為東陽太守,入為御史中丞,秘書監,領
 越騎校尉,不拜,遷司徒左長史。上以散騎常侍舊與侍中俱掌獻替,欲高其選,以景文及會稽孔覬俱南北之望,並以補之。尋復為左長史。坐姊墓開不臨赴,免官。大明二年,復為祕書監,太子右衛率,侍中。五年,出為安陸王子綏冠軍長史、輔國將軍、江夏內史,行郢州事。又徵為侍中,領射聲校尉,右衛將軍,加給事中,太子中庶子,右衛如故。坐與奉朝請毛法因蒱戲,得錢百二十萬,白衣領職。尋復為侍中,領中庶子,未拜。前廢帝嗣位,徙秘
 書監,侍中如故。以父老自解,出為江夏王義恭太宰長史,輔國將軍、南平太守。永光初,為吏部尚書。景和元年,遷右僕射。



 太宗即位,加領左衛將軍。時六軍戒嚴,景文仗士三十人入六門。諸將咸云:「平殄小賊,易於拾遺。」景文曰:「敵固無小,蜂蠆有毒,何可輕乎?諸軍當臨事而懼,好謀而成,先為不可勝,乃制勝之術耳。」尋遷丹陽尹,僕射如故;遭父憂,起為冠軍將軍,尚書左僕射,丹陽尹,固辭僕射,改授散騎常侍、中書令、中軍將軍,尹如故,又辭
 不拜。仍出為使持節、散騎常侍、都督江州郢州之西陽豫州之新蔡晉熙三郡諸軍事、安南將軍、江州刺史。讓常侍,服闋乃受。



 太宗翦除暴主,又平四方,欲引朝望以佐大業,乃下詔曰:「夫良圖宣國,賞崇彞命;殊績顯朝,策勤王府。安南將軍、江州刺史景文,風度淹粹,理懷清暢,體兼望實,誠備夷岨。寶歷方啟,密贊義機,妖徒干紀,預毗廟略。宜登茅社,永傳厥祚。朕澄氛寧樞,實資多士,疏爵疇庸,實膺徽烈。尚書右僕射、領衛尉興宗,識懷詳正,
 思局通敏。吏部尚書、領太子左衛率淵,器情閑茂,風業韶遠。並謀參軍政,績亮時艱,拓宇開邑,實允勳典。景文可封江安縣侯,食邑八百戶,興宗可始昌縣伯,淵可南城縣伯,食邑五百戶。」景文固讓,不許,乃受五百戶。進號鎮南將軍,尋給鼓吹一部。後以江州當徙鎮南昌,領豫章太守,餘如故;州不果遷。



 頃之,徵為尚書左僕射,領吏部,揚州刺史,加太子詹事,常侍如故。不願還朝,求為湘州刺史,不許。



 時又謂景文在江州,不能潔己。景文與上
 幸臣王道龍書曰:「吾雖寡於行己,庶不負心,既愧殊效,誓不上欺明主。竊聞有為其貝錦者,云營生乃至巨萬,素無此能,一旦忽致異術,必非平理。唯乞平心精檢,若此言不虛,便宜肆諸市朝,以正風俗。脫其妄作,當賜思罔昧之由。吾踰忝轉深,足以致謗,念此驚懼,何能自測。區區所懷,不願望風容貸。吾自了不作偷,猶如不作賊。故以密白,想為申啟。」



 景文屢辭內授,上手詔譬之曰:「尚書左僕射,卿已經此任,東宮詹事,用人雖美,職次正可
 比中書令耳。庶姓作揚州,徐干木、王休元、殷鐵並處之不辭。卿清令才望,何愧休元;毗贊中興,豈謝干木;綢繆相與,何後殷鐵邪?司徒以宰相不應帶神州,遠遵先旨,京口鄉基義重,密邇畿內,又不得不用驃騎,陜西任要,由來用宗室。驃騎既去,巴陵理應居之,中流雖曰閑地,控帶三江,通接荊、郢,經塗之要,由來有重鎮。如此,則揚州自成闕刺史,卿若有辭,更不知誰應處之。此選大備,與公卿疇懷,非聊爾也。」固辭詹事領選,徙為中書令,常侍、僕
 射、揚州如故。又進中書監,領太子太傅,常侍、揚州如故。景文固辭太傅,上遣新除尚書右僕射褚淵宣旨,以古來比例六事詰難之,不得已,乃受拜。



 時太子及諸皇子並小,上稍為身後之計,諸將帥吳喜、壽寂之之徒,慮其不能奉幼主,並殺之;而景文外戚貴盛,張永累經軍旅,又疑其將來難信,乃自為謠言曰:「一士不可親,弓長射殺人。」一士,王字;弓長,張字也。景文彌懼,乃自陳求解揚州,曰:臣凡猥下劣,方圜無算,特逢聖私,頻叨不次,乘非
 其任,理宜覆折。雖加恭謹,無補橫至,夙夜燋戰,無地容處。六月中,得臣外甥女殷恒妻蔡疏,欲令其兒啟聞乞祿,求臣署入,云凡外人通啟,先經臣署。于時驚怖,即欲封疏上呈;更思此家落漠,庶非通謗,且廣聽察,幸無復所聞。比日忽得兗州都送迎西曹解季遜板云是臣屬,既不識此人,即問郗顒,方知虛託。比十七日晚,得征南參軍事謝儼口信,云臣使人略奪其婢。臣遣李武之問儼元由,答云「使人謬誤」。誤之與實,雖所不知,聞此之日,
 唯有憂駭。



 臣之所知,便有此三變,臣所不覺,尤不可思。若守爵散輩,寧當招此,誠由暗拙,非復可防。自竊州任,倏已七月,無德而祿,其殃將至。且傅職清峻,亢禮儲極,以臣凡走,豈可暫安。荷恩懼罪,不敢執固,焦魂褫氣,憂迫失常。況臣髮醜人群,病絕力效,穢朝點列,顧無與等,獨息易駭,慚懼難持。伏願薄回矜愍,全臣身計,大夫之俸,足以自周,久懷欣羨,未敢干請,仰希慈宥,照臣款誠。



 上詔答曰:去五月中,吾病始差,未堪勞役,使卿等看選
 牒,署竟,請敕施行。此非密事,外間不容都不聞。然傳事好訛,由來常患。殷恒妻,匹婦耳,閨閣之內,傳聞事復作一兩倍落漠,兼謂卿是親故,希卿署,不必云選事獨關卿也。恒妻雖是傳聞之僻,大都非可駭異。且舉元薦凱,咸由疇諮,可謂唐堯不明,下干其政邪?悠悠好詐貴人及在事者,屬卿偶不悉耳,多是其周旋門生輩,作其屬託,貴人及在事者,永無由知。非徒止於京師,乃至州郡縣中,或有詐作書疏,灼然有文跡者。諸舍人右丞輩,及
 親近驅使人,慮有作其名,載禁物,求停檢校,彊賣猥物與官,仍求交直,或屬人求乞州郡資禮,希蠲呼召及虜發船車,並啟班下在所,有即駐錄。但卿貴人,不容有此啟。由來有是,何故獨驚!



 人居貴要,但問心若為耳。大明之世,巢、徐、二戴,位不過執戟,權亢人主;顏師伯白衣僕射,橫行尚書中。令袁粲作僕射領選,而人往往不知有粲。粲遷為令,居之不疑。今既省錄,令便居昔之錄任,置省事及幹童,並依錄格。粲作令來,亦不異為僕射。人情
 向粲,淡淡然亦復不改常。以此居貴位要任,當有致憂兢理不?



 卿今雖作揚州,太子傅位雖貴,而不關朝政,可安不懼,差於粲也。想卿虛心受榮,而不為累。



 貴高有危殆之懼,卑賤有溝壑之憂,張、單雙災,木雁兩失,有心於避禍,不如無心於任運。夫千仞之木,既摧於斧斤;一寸之草,亦瘁於踐蹋。高崖之脩幹,與深谷之淺條,存亡之要,巨細一揆耳。晉畢萬七戰皆獲,死於牖下;蜀相費禕從容坐談,斃於刺客。故甘心於履危,未必逢禍;縱意於
 處安,不必全福。但貴者自惜,故每憂其身;賤者自輕,故易忘其己。然為教者,每誡貴不誡賤,言其貴滿好自恃也。凡名位貴達,人以在懷,泰則觸人改容,不則行路嗟愕。至如賤者,否泰不足以動人,存亡不足以絓數,死於溝瀆,死於塗路者,天地之間,亦復何限,人不以係意耳。



 以此而推,貴何必難處,賊何必易安。但人生也自應卑慎為道,行己用心,務思謹惜。若乃吉凶大期,正應委之理運,遭隨參差,莫不由命也。既非聖人,不能見吉兇之
 先,正是依俙於理,言可行而為之耳。得吉者是其命吉,遇不吉者是其命凶。以近事論之,景和之世,晉平庶人從壽陽歸亂朝,人皆為之戰慄,而乃遇中興之運;袁顗圖避禍於襄陽,當時皆羨之,謂為陵霄駕鳳,遂與義嘉同滅。駱宰見幼主,語人云:「越王長頸鳥喙,可與共憂,不可與共樂。范蠡去而全身,文種留而遇禍。今主上口頸,頗有越王之狀,我在尚書中久,不去必危。」遂求南江小縣。



 諸都令史住京師者,皆遭中興之慶,人人蒙爵級;宰
 值義嘉染罪,金木纏身,性命幾絕。卿耳眼所聞見,安危在運,何可預圖邪!



 時上既有疾,而諸弟並已見殺,唯桂陽王休範人才本劣,不見疑,出為江州刺史。慮一旦晏駕,皇后臨朝,則景文自然成宰相,門族彊盛,藉元舅之重,歲暮不為純臣。泰豫元年春,上疾篤,乃遣使送藥賜景文死,手詔曰:「與卿周旋,欲全卿門戶,故有此處分。」死時年六十。追贈車騎將軍、開府儀同三司,常侍、中書監、刺史如故,謚曰懿侯。



 長子絢,字長素。年七歲,讀《論語》至「
 周監於二代」,外祖何尚之戲之曰:「耶耶乎文哉。」絢即答曰:「草蓊風必偃。」少以敏惠見知。及長,篤志好學,官至秘書丞。年二十四,先景文卒,謚曰恭世子。子婼襲封,齊受禪,國除。



 景文兄子蘊,字彥深。父楷,太中大夫,人才凡劣,故蘊不為群從所禮,常懷恥慨。家貧,為廣德令,會太宗初即位,四方叛逆,蘊遂感激為將,假寧朔將軍,建安王休仁司徒參軍,令如故。景文甚不悅,語之曰:「阿益,汝必破我門戶。」



 阿益者,蘊小字也。事寧,封吉陽縣男,食邑三
 百戶。為中書、黃門郎,晉陵、義興太守,所蒞並貪縱。在義興應見收治,以太后故,止免官。



 廢帝元徽初,復為黃門郎,東陽太守。未之郡,值桂陽王休範逼京邑,蘊領兵於朱雀門戰敗被創,事平,除侍中,出為寧朔將軍、湘州刺史。蘊輕躁,薄於行業,時沈攸之為荊州刺史,密有異志,蘊與之結厚。及齊王輔朝政,蘊、攸之便連謀為亂,會遭母憂,還都,停巴陵十餘日,更與攸之成謀。時齊王世子為郢州行事,蘊至郢州,謂世子必下慰之,欲因此為變,
 據夏口,與荊州連橫。世子覺其意,稱疾不往,又嚴兵自衛,蘊計不得行,乃下。及攸之為逆,蘊密與司徒袁粲等結謀,事在粲傳。事敗,走鬥場,追禽,斬於秣陵市。



 景文弟子孚,大明末,為海鹽令。泰始初,天下反叛,唯孚獨不同逆,官至司徒記室參軍。



 史臣曰:王景文弱年立譽,聲芳籍甚,榮貴之來,匪由勢至。若泰始之朝,身非外戚,與袁粲群公方驂並路,傾覆之災,庶幾可免。庾元規之讓中書令,義在此乎!



\end{pinyinscope}