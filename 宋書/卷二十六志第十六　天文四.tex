\article{卷二十六志第十六 天文四}

\begin{pinyinscope}

 宋武帝
 永初元年十月辛丑,熒惑犯進賢。占曰:「進賢官誅。」十一月乙卯,熒惑犯填星於角。占曰:「為喪,大人惡之。」一曰:「兵起。」十二月庚子,月犯熒惑於亢。占曰:「為內亂。」一
 曰:「貴人憂。角為天門,亢為朝廷。」三年五月,宮車晏駕。七月,太傅長沙景王道憐薨。索頭攻略青、司、兗三州。於是禁兵大出,是後司徒徐羨之、尚書令傅亮、領軍謝晦等廢少帝,內亂之應。



 永初元年十二月甲辰,月犯南斗。占曰:「大臣憂。」三年七月,長沙王薨。



 索虜寇青、司二州,大軍出救。



 永初二年六月甲申,太白晝見。占:「為兵喪,為臣彊。」三年五月,宮車晏駕。尋遣兵出救青、司。其後徐羨之等秉權,
 臣彊之應也。永初二年六月乙酉,熒惑犯氐。乙巳,犯房。占曰:「氐為宿宮,房為明堂,人主有憂。房又為將相,將相有憂。氐、房又兗、豫分。」三年五月,宮車晏駕。七月,長沙王薨,王領兗州也。景平元年,廬陵王義真廢,王領豫州也。永初二年十月,太白犯填星於亢。亢,兗州分,又為鄭。占曰:「大星有大兵,金土合為內兵。」三年,索頭攻略青、冀、兗三州,禁兵大出,兗州失守,虎牢沒。



 永初三年正月丁卯,月犯南斗,占同元年。一曰:「女主當之。」二月辛卯,有星孛于虛危,向河津,掃河鼓。占曰:「為兵喪。」五月,宮車晏駕。明年,遣軍救青、司。二月,太后蕭氏崩。永初三年二月壬辰,填星犯亢。占曰:「諸侯有失國者,民多流亡。」一曰:「廷臣為亂。亢,兗州分,又為鄭。」其年,索頭攻圍司、兗,兗州刺史徐琰委守奔敗,司州刺史毛德祖距守陷沒,緣河吏民,多被侵略。永初三年三月壬戌,月犯南斗,占同正月。五月丙午,犯
 軒轅。占曰:「女主當之。」六月辛巳,月犯房。占曰:「將相有憂,豫州有災。」癸巳,犯歲星於昴。



 占曰:「趙、魏兵飢。」其年,虜攻略青、兗、司三州。廬陵王義真廢,王領豫州也。二月,太后蕭氏崩。元嘉三年,司徒徐羨之等伏誅。永初三年九月癸卯,熒惑經太微犯左執法。己未,犯右執法,占悉同上。十月癸酉,太白犯南斗。占曰:「國有兵事,大臣有反者。」辛巳,熒惑犯進賢。占曰:「進賢官誅。」明年,師出救青、司。景平二年,徐羨之等廢帝徙王。元嘉三年,羨
 之及傅亮、謝晦悉誅。



 永初三年十一月戊午,有星孛於室壁。占曰:「為兵喪。」明年,兵救青、司。二月,太后蕭氏崩。營室,內宮象也。永初三年十一月癸亥,月犯亢、氐。占曰:「國有憂。」十二月戊戌,熒惑犯房。房為明堂,王者惡之。一曰:「將相憂。」



 景平二年,羨之等廢帝,因害之。元嘉三年,羨之等伏誅。



 少帝景平元年正月乙卯,有星孛于東壁南,白色,長二丈餘,拂天苑,二十日滅。二月,太后蕭氏崩。十月戊午,有
 星孛於氐北,尾長四丈,西北指,貫攝提,向大角,東行,日長六七尺,十餘日滅。明年五月,羨之等廢帝。



 文帝元嘉元年十月,熒惑犯心。元嘉三年正月甲寅夜,天東南有黑氣,廣一丈,長十餘丈。元嘉六年五月,太白晝見經天。元嘉七年三月,太白犯歲星於奎。六月,熒惑犯東井輿鬼,入軒轅。月犯歲星。十一月癸未,西南有氣,上下赤,中央黑,廣三尺,長三十餘丈,狀如旌旗。十二月丙戌,有流星頭如甕,尾長二十餘丈,大如數十斛船,赤
 色,有光照人面,從西行經奎北大星南過,至東壁止。其年,索虜寇青、司,殺刺史,掠居民。遣征南大將軍檀道濟討伐,經歲乃歸。



 元嘉八年四月辛未,太白晝見,在胃。五月,犯天關東井。六月庚午,熒惑入東井。七月壬戌夜,白虹見東方。丁丑,太白犯上將。八月癸未,太白入太微右掖門內,犯左執法。乙未,熒惑犯積尸。九月丙寅,流星大如斗,赤色,發太微西蕃,北行,未至北斗沒,餘光長三丈許。十月丙辰,金
 土相犯,在須女,月奄天關東井。



 十二月,月犯房鉤鈐。十年,仇池氐寇漢中,梁州失戍。



 元嘉九年正月庚午,熒惑入輿鬼。三月,月犯軒轅。四月,犯左角,歲星入羽林。月犯房鉤鈐。己丑,太白入積尸。五月,犯軒轅,月掩南斗第六星。辛酉,熒惑入太微右掖門,犯右執法。七月丙午,月蝕左角。八月癸未,太白犯心前星。乙酉,犯心明堂星。元嘉十年十月,有流星大如甕,尾長二十餘丈。元嘉十一年二月庚子,月犯畢,入畢口而
 出,因暈昴、畢,西及五車,東及參。三月丙辰,太白晝見,在參。閏月戊寅,太白犯五諸侯。己丑,月入東井,犯太白。于時司徒彭城王義康專權。



 元嘉十二年五月壬戌,月犯右執法。七月壬戌,熒惑犯積尸,奄上將。十月丙午,月犯右執法。十二月甲申,太白犯羽林。十七年,上將執法皆被誅。



 元嘉十三年正月庚午,月犯熒惑。二月,月犯太微東蕃第一星。十一月辛亥,歲星犯積尸。十二月戊子,熒惑入
 羽林。後年,廢大將軍彭城王義康及其黨與;凡所收掩,皆羽林兵出。



 元嘉十四年正月,有星晡前晝見東北維,在井左右,黃赤色,大如橘。月犯東井。四月丁未,太白犯輿鬼。五月丙子,太白晝見,在太微。七月辛卯,歲星入軒轅。八月庚申,熒惑犯上將。九月丙戌,熒惑犯左執法。其後,皇后袁氏崩,丹陽尹劉湛誅,尚書僕射殷景仁薨。



 元嘉十五年四月己卯,月犯氐。十月壬戌,流星大如鴨
 子,出文昌,入紫宮,聲如雷。十一月癸未,熒惑入羽林。丁未,月犯東井鉞星。其後,誅丹陽尹劉湛等。



 元嘉十六年二月,歲星逆行犯左執法。五月丁卯,太白晝見胃、昴間。月入羽林,太白犯畢,歲星犯左執法。七月,月會填星。八月,太白犯軒轅。明年,皇后袁氏崩。熒惑犯太微西上將,太白晝見,在翼。九月,熒惑同入太微相犯。太白犯左執法,熒惑犯右執法。十月,歲星熒惑相犯,在亢。十一月,熒惑犯房北第一星。



 明年,大將軍義康出徙豫
 章,誅其黨與。尚書僕射、揚州刺史殷景仁薨。



 元嘉十九年九月,客星見北斗,漸為彗星,至天苑末滅。元嘉二十年二月二十四日乙未,有流星大如桃,出天津,入紫宮;須臾,有細流星或五或三相續,又有一大流星從紫宮出,入北斗魁;須臾,又一大流星出,貫索中,經天市垣,諸流星并向北行,至曉不可稱數。流星占並云:「天子之使。」又曰:「庶民惟星。星流,民散之象。」至二十七年,索虜殘破青、冀、徐、兗、南兗、豫六州,民死太半。



 元嘉二十二年二月,金火木合東井。四月,月犯心,太白入軒轅。七月,太白晝見。其冬,太子詹事范曄謀反伏誅。元嘉二十三年正月,金火相爍。其月,索虜寇青州,驅略民戶。元嘉二十四年正月,月犯心大星。天星并西流,多細,大不過如雞子,尾有長短,當有數百。至旦,日光定乃止,有入北斗紫宮者。占:「流星群趨所之者,兵聚其下,有大急。」又占:「眾星並流,將軍並舉兵。隨星所之,以應天氣。」又占:「
 流星入紫宮,有喪,水旱不調。」又占:「流星入北斗,大臣有系者。」又占:「流星為民,大星大臣流,小星小民流。」四月,太白晝見。



 八月,征北大將軍衡陽王義季薨;豫章民胡誕世率其宗族破郡縣,殺太守及縣令元嘉二十五年正月,火、水入羽林。月犯歲星,太白晝見經天。元嘉二十六年十月,彗星入太微。十一月,白氣貫北斗。二十七年夏,太白晝見經天。九月,太白犯歲星。十月,熒惑入太微。元嘉二十八年五月,彗星見卷舌,入太
 微,逼帝座,犯上相,拂屏,出端門,滅翼、軫。翼、軫,荊州分。太白晝見犯哭星。三十年,太子巫蠱咒詛事覺,遂殺害朝臣。孝建元年,荊、江二州反,皆夷滅。卷舌,咒詛之象。彗之所起,是其應也。元嘉二十九年正月,太白晝見,經天。明年,東宮弒逆。



 孝武孝建元年二月,有流星大如月,西行。其年,豫州刺史魯爽反誅。孝建元年九月壬寅,熒惑犯左執法。尚書左僕射建平
 王宏表解職,不許。孝建元年十月乙丑,熒惑犯進賢星。吏部尚書謝莊表解職,不許。



 孝建二年五月乙未,熒惑入南斗。十月甲辰,又入南斗。大明元年夏,京師疾疫。



 孝建三年四月戊戌,太白犯輿鬼。占曰:「民多疾。」明年夏,京邑疫疾。孝建三年八月甲午,太白入心。占曰:「後九年,大飢至。」大
 明八年,東土大飢,民死十二三。



 大明元年三月癸亥,太白在奎南,犯歲星。占曰:「有滅諸侯。」三年,司空竟陵王誕反誅。大明元年六月丙申,月在東壁,掩熒惑。占曰:「將軍有憂,期不出三年。」至三年,司空竟陵王誕反。



 大明二年三月辛未,熒惑入東井。四月己亥,熒惑在東井犯北軒轅第二星。井,雍州分。其年四月,海陵王休茂為雍州刺史,五年,休茂反誅。
 大明二年七月己巳,月掩軒轅第二星。十月辛卯,月掩軒轅。十一月丙戌,月又掩軒轅。軒轅,女主。



 時民間喧言人主帷薄不修。大明二年十一月庚戌,熒惑犯房及鉤鈐。壬子,熒惑又犯鉤鈐。占曰:「有兵。」其年,索虜寇歷下,遣羽林軍討破之。



 大明三年春正月夜,通天薄雲,四方生赤氣,長三四尺,乍沒乍見,尋皆消滅。



 占名隧星,一曰刀星,天下有兵,戰鬥流血。月入太微,犯次將。占曰:「有反臣死,將誅。」三月,月
 在房,犯鉤鈐,因蝕。占曰:「人主惡之,將軍死。」三月,土守牽牛。占曰:「大人憂疾,兵起,大赦,姦臣賊子謀欲殺主。」四月,犯五諸侯。占曰:「諸侯誅。」金、水合西方。占曰:「兵起。」五月,歲星犯東井鉞。



 占曰:「斧鉞用,大臣誅。」六月,月入南斗。占曰:「大臣大將軍誅。」南兗州刺史竟陵王誕尋據廣陵反,遣車騎大將軍沈慶之領羽林勁兵及豫州刺史宗愨、徐州刺史劉道隆眾軍攻戰。及屠城,城內男女道俗,梟斬靡遺。將軍宗越偏用虐刑,先刳腸決眼,或笞面鞭腹,苦
 酒灌創,然後方加以刀鋸。大兵之應也。八月,月犯太白,太白犯房。占曰:「人君有憂,天子惡之。」熒惑守畢。占曰:「萬民饑,有大兵。」九月,太白犯南斗。占曰:「大臣有反者。」九月,月在胃而蝕,既,又於昴犯熒惑。占曰:「兵起,女主當之,人主惡之。」一曰:「女主憂,國王死,民飢。」十月,太白犯哭星。占曰:「人主有哭泣之聲。」自後六宮多喪,公主薨亡,天子舉哀相係。歲大旱,民飢。



 大明四年正月,月奄氐。占曰:「大將死。」又犯房北第二星。
 占曰:「有亂臣謀其主。」二月,有赤氣,長一尺餘,在太白帝坐北。占曰:「兵起,臣欲謀其君。」五月,月入太微。占曰:「有反臣,大臣死。」六月,太白犯井鉞。占曰:「兵起,斧鉞用,大臣誅。」月犯心前星。占曰:「有亂臣,太子惡之。」月入南斗魁中。占曰:「大人憂,女主惡之。」七月,歲星犯積尸。占曰:「大臣誅。」



 十二月,月犯心中央大星。占曰:「大人憂。」十二月,通天有雲,西及東北並生,合八所,並長四尺,乍沒乍見,尋消盡。占曰:「天下有兵。」十二月,月犯箕東北星,女主惡之。明年,雍
 州刺史海陵王休茂反。太白犯東井,雍州兵亂之應也。



 大明五年正月,歲星犯輿鬼積尸。占曰:「大臣誅,主有憂,財寶散。」月入南斗魁中。占曰:「大人憂,天下有兵。」火、土同在須女。占曰:「女主惡之。」



 三月,月掩軒轅。占曰:「女主惡之。」有流星數千萬,或長或短,或大或小,並西行,至曉而止。占曰:「人君惡之,民流亡。」四月,太白犯東井北轅。占曰:「大臣為亂,斧鉞用。」太白犯輿鬼。占曰:「大臣誅,斧鉞用,人主憂。」六月,有流星白色,大如甌,出王良,西南行,沒天市中,
 尾長數十丈,沒後餘光良久。



 占曰:「天下亂。」八月,熒惑入東井。占曰:「大臣當之。」十月,歲星犯太微上將星。太白入亢,犯南第二星。占曰:「上將有憂,輔臣有誅者,人君惡之。」



 十月,太白入氐中,熒惑入井中。占曰:「王者亡地,大赦,兵起,為飢。」月入太微,掩西蕃上將,犯歲星。占曰:「有反臣死。」大星大如斗,出柳北行,尾十餘丈,入紫宮沒,尾後餘光良久乃滅。占曰:「天下凶,有兵喪,天下惡之。」十一月,月掩心前星,又犯大星。占曰:「大人憂,兵起,大旱。」十二月,太白
 犯西建中央星。占曰:「大臣相譖。」月犯左角。占曰:「天子惡之。」後三年,孝武帝、文穆皇后相系崩;嗣主即位一年,誅滅宰輔將相,虐戮朝臣,禍及宗室,因自受害。



 大明六年正月,月在張,犯歲星。占曰:「民飢流亡。」月犯心後星。占曰:「庶子惡之。」二月,月掩左角。占曰:「天子惡之。」三月,熒惑入輿鬼。占曰:「有兵,大臣誅,天下多疾疫。」五月,月在張,又入太微,犯熒惑。占曰:「國主不安,女主憂。」火犯木在翼。占曰:「為飢,為旱,近臣大臣謀主。」有星前赤後白,大
 如甌,尾長十餘丈,出東壁北,西行沒天市,啾啾有聲。占曰:「其下有兵,天下亂。」月掩昴七星。占曰:「貴臣誅,天子破匈奴,胡主死。」歲星犯上將。占曰:「輔臣誅,上將憂。」六月,月入太微,犯右執法。占曰:「人主不安,天下大驚,主不吉,執法誅。」月犯心後星。占曰:「庶子惡之。」七月,月犯箕。占曰:「女主惡之。」八月,月入南斗魁中。占曰:「大臣誅,斧鉞用,吳、越有憂。」明年,揚、南徐州大旱,田穀不收,民流死亡。自後三年,帝后仍崩,宰輔及尚書令僕誅戮,索虜主死,新安王
 兄弟受害,司徒豫章王子尚薨,羽林兵入三吳討叛逆。



 大明七年正月夜,通天薄雲,四方合有八氣,蒼白色,長二三丈,乍見乍沒,名刀星。占曰:「天下有兵。」三月,月犯心後星。占曰:「庶子惡之。」四月,火犯金,在婁。占曰:「有喪,有兵,大戰。」六月,月犯箕。占曰:「女主惡之。」



 太白入東井。占曰:「大臣當之。」太白犯東井。占曰:「大臣為亂,斧鉞用。」



 七月,熒惑入東井。占曰:「兵起,大將當之。」月入南斗魁,犯第二星。占曰:「大人憂,吳郡當之。」太白犯輿鬼。占曰:「兵起,大將誅,人
 主憂,財帛出。」



 八月,月入哭星中間,太白犯軒轅少微星。占曰:「人主憂,哭泣之聲,民飢流亡。」



 太白入太微。占曰:「近臣起兵,國不安。」熒惑犯鬼,太白犯右執法。占曰:「大臣誅。」十月,金水相犯。占曰:「天下飢。」熒惑守軒轅第二星。占曰:「宮中憂,有哀。」十一月,歲星入氐。占曰:「諸侯人君有入宮者。」十二月,月犯五車。占曰:「天庫兵動。」後二年,帝后崩,大臣將相誅滅,皇子被害,皇太后崩,四方兵起,分遺諸軍推鋒外討。



 大明八年正月,月掩輿鬼。占曰:「大臣誅。」月入南斗魁中,掩第二星。占曰:「大人憂,女主惡之。」二月,月犯南斗第四星,入魁中。占曰:「大人有憂,女主當之。豫章受災。」四月,月入南斗魁中,犯第三星。占曰:「大人有憂,女主惡之。丹陽當之。」太白入東井,入太微,犯執法。占曰:「執法誅,近臣起兵,國不安。」六月,歲星犯氐。占曰:「歲大飢。」有流星大如五斗甌,赤色有光,照見人面,尾長一丈餘,從參北東行,直下經東井,過南河,沒。占曰:「民飢,吳、越有兵。」七月,歲星入
 氐。十月,太白守房。占曰:「有兵,大喪。」月掩食房。占曰:「有喪,大飢。」此後國仍有大喪,丹陽尹顏師伯、豫章王子尚死。



 明年,昭太后崩,四方賊起,王師水陸征伐,義興晉陵縣大戰,殺傷千計。



 前廢帝永光元年正月丁酉,太白掩牽牛。牽牛,越分。其月庚申,月在虛宿,犯太白。虛,齊地。二月甲申,月入南斗。南斗,揚州分野,又為貴臣。三月庚子,月入輿鬼,犯積尸。輿鬼,主斬戮。六月庚午,熒惑入東井。東井,雍州分。其月
 壬午,有大流星,前赤後白,入紫宮。景和元年九月丁酉,熒惑入軒轅,在女主大星北。十月,熒惑入太微,犯西上將。十一月丁未,太白犯哭星。其月乙卯,月犯心,心為天王。其年,太宰江夏王義恭、尚書令柳元景、尚書僕射顏師伯等並誅。



 太尉沈慶之薨。廬陵王敬先、南平王敬猷、南安侯敬淵並賜死。廢帝殞。明年,會稽太守尋陽王子房、廣州刺史袁曇遠、雍州刺史袁顗、青州刺史沈文秀並反。昭太后崩。



 明帝泰始元年十二月己巳,太白入羽林。占曰:「羽林兵動。」乙亥,白氣入紫宮。占曰:「有喪事。」明年,羽林兵出討。昭太后崩。



 泰始二年正月甲午,熒惑逆行在屏西南。占曰:「有兵在中。」其月丙申,月暈五車,通畢、昴。占曰:「女主惡之。」其月庚子,月犯輿鬼。占曰:「將軍死。」



 其月甲寅,流星從五車出,至紫宮西蕃沒。占曰:「有兵。」其月丙辰,黑氣貫宿。



 占曰:「王侯有歸骨者。」三月乙未,有流星大小西行,不可稱數,至曉
 乃息。占曰:「民流之象。」四月壬午,熒惑入太微,犯右執法。月在丙子,歲星晝見南斗度中。占曰:「其國有軍容,大敗。」其月己卯,竟夜有流星百餘西南行,一大如甌,尾長丈餘,黑色,從河鼓出。又曰:「有兵。」其月壬午,太白在月南並出東方,為犯。占曰:「有破軍死將,王者亡地。」七月甲午,月犯心。心為宋地。其月丙午,月犯南斗。占曰:「大臣誅。」其月乙卯,熒惑犯氐。氐,兗州分野。十月辛巳,太白入氐。占曰:「春穀貴。」十一月癸巳,太白犯房。占曰:「牛多死。」



 其年,四方
 反叛,內兵大出,六師親戎。昭太后崩。大將殷孝祖為南賊所殺。尚書右僕射蔡興宗以熒惑犯右執法,自解,不許。九月,諸方反者皆平,多有歸降者。



 後失淮北四州地,彭城、兗州並為虜所沒,民流之驗也。彭城,宋分也。是春,穀貴民飢。明年,牛多疾死,詔太官停宰牛。



 泰始三年六月甲辰,月犯東井。占曰:「軍將死。」熒惑犯輿鬼。占曰:「金錢散。」又曰:「不出六十日,必大赦。」八月癸卯,天子以皇后六宮衣服金釵雜物賜北征將士。明年二月,
 護軍王玄謨薨。



 泰始四年六月壬寅,太白犯輿鬼。占曰:「民大疾,死不收。」其年,普天大疫。



 泰始五年二月丙戌,月犯左角。占曰:「三年天子惡之。」三月庚申,月犯建星。占曰:「易相。」十月壬午,月犯畢。占曰:「天子用法,誅罰急,貴人有死者。」其月丙申,太白犯亢。占曰:「收斂國兵以備北方。」其年冬,建安王休仁解揚州,桂陽王休範為揚州。揚州牧前後常宰相居之,易相之驗也。
 七年,晉平王休祐、建安王休仁並見殺。時失淮北,立戍以備防北虜。後三年,宮車晏駕。



 泰始六年正月辛巳,月犯左角,同前占。八月壬辰,熒惑犯南斗。南斗,吳分。



 十一月乙亥,月犯東北轅。占曰:「大人當之。」又曰:「大臣有誅者。」二年,殺揚州刺史王景文。宮車晏駕。



 後廢帝元徽三年七月丙申,太白入角,犯歲星。占曰:「角為天門,國將有兵事。」占,於角太白與木星會,殺軍在外,
 破軍殺將。其月丁巳,太白入氐。氐為天子宿宮,太白兵凶之星。八月己巳,太白犯房北頭第二星。占曰:「王失德。」



 九月癸卯,太白犯南斗第三星。占曰:「大人當之,國易政。」十月丙戌,歲星入氐。占曰:「諸侯人君有來入宮者。」十一月庚戌,月入太微,奄屏西南星。占曰:「貴者失勢。」四年七月,建平王景素據京口反。時廢主凶慝無度,五年七月殞,安成王入篡皇阼。三年,齊受禪。



 元徽四年三月乙巳,月犯房北頭第一星,進犯鍵閉
 星。占曰:「有謀伏甲兵在宗廟中,天子不可出宮下堂,多暴事。」九月甲辰,填星犯太微西蕃。占曰:「立王。」一曰:「徙王。」又曰:「大人憂。」時廢帝出入無度,卒以此殞,安成王立。



 元徽五年正月戊申,月犯南斗第五星。與前同占。四月丁巳,熒惑犯輿鬼西北星。占曰:「大人憂,近期六十日,遠期六百日。」又曰:「人君惡之。」其月丙子,太白犯輿鬼西北星。占曰:「大赦。」五月戊申,太白晝見午上,光明異常。



 占曰:「更姓。」六月壬戌,月犯鉤鈐星。占曰:「有大令。」其月乙丑,月
 犯南斗第四星。與前同占。七月,廢帝殞,大赦天下。後二年,齊受禪。



 順帝升明元年八月庚申,月入南斗,犯第三星,與前同占。九月丁亥,太白在翼,晝見經天。占曰:「更姓。」閏十二月癸卯夜,月奄南斗第四星,與前同占。



\end{pinyinscope}