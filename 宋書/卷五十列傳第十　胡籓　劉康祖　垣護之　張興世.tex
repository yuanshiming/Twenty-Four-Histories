\article{卷五十列傳第十 胡籓 劉康祖 垣護之 張興世}

\begin{pinyinscope}

 胡籓,
 字道序,豫章南
 昌人
 也。祖隨,散騎常侍。父仲任,治書侍御史。籓少孤,居喪以毀稱。太守韓伯見之,謂籓叔尚書少廣曰:「卿此侄當以義烈成名。」



 州府辟召,不就。須二
 弟冠婚畢,乃參郗恢征虜軍事。時殷仲堪為荊州刺史,籓外兄羅企生為仲堪參軍,籓請假還,過江陵省企生。仲堪要籓相見,接待甚厚。籓因說仲堪曰:「桓玄意趣不常,每怏怏於失職。節下崇待太過,非將來之計也。」仲堪色不悅。籓退而謂企生曰:「倒戈授人,必至之禍。若不早規去就,後悔無及。」



 玄自夏口襲仲堪,籓參玄後軍軍事。仲堪敗,企生果以附從及禍。籓轉參太尉、大將軍、相國軍事。



 義旗起,玄戰敗將出奔,籓於南掖門捉玄馬控,曰:「今
 羽林射手猶有八百,皆是義故西人,一旦捨此,欲歸可復得乎?」玄直以馬鞭指天而已,於是奔散相失。



 追及玄於蕪湖,玄見籓,喜謂張須無曰:「卿州故為多士,今乃復見王叔治。」桑落之戰,籓艦被燒,全鎧入水潛行三十許步,方得登岸。義軍既迫,不復得西,乃還家。



 高祖素聞籓直言於殷氏,又為玄盡節,召為員外散騎侍郎,參軍軍事。從征鮮卑,賊屯聚臨朐,籓言於高祖曰:「賊屯軍城外,留守必寡,今往取其城,而斬其旗幟,此韓信所以克趙
 也。」高祖乃遣檀韶與籓等潛往,既至,即克其城。賊見城陷,一時奔走,還保廣固累月。將拔之夜,佐史並集,忽有鳥大如鵝,蒼黑色,飛入高祖帳裏,眾皆駭愕,以為不祥。籓起賀曰:「蒼黑者,胡虜之色,胡虜歸我,大吉之祥也。」明旦,攻城,陷之。從討盧循於左里,頻戰有功,封吳平縣五等子,除正員郎。尋轉寧遠將軍、鄱陽太守。



 從伐劉毅。毅初當之荊州,表求東道還京辭墓,去都數十里,不過拜闕。高祖出倪塘會之。籓勸於坐殺毅,高祖不從。至是謂籓
 曰:「昔從卿倪塘之謀,無今舉也。」又從征司馬休之。復為參軍,加建武將軍,領游軍於江津。徐逵之敗沒,高祖怒甚,即日於馬頭岸渡江,而江津岸峭,壁立數丈,休之臨岸置陣,無由可登。



 高祖呼籓令上,籓有疑色,高祖奮怒,命左右錄來,欲斬之。籓不受命,顧曰:「籓寧前死耳!」以刀頭穿岸,少容腳指,於是徑上,隨之者稍多。既得登岸,殊死戰,賊不能當,引退。因而乘之,一時奔散。



 高祖伐羌,假籓寧朔將軍,參太尉軍事,統別軍。至河東,暴風漂籓重
 艦渡北岸,索虜牽得此艦,取其器物。籓氣厲心憤,率左右十二人,乘小船徑往河北。賊騎五六百見籓來,並笑之。籓素善射,登岸射,賊應弦而倒者十許人,賊皆奔退,悉收所失而反。又遣籓及朱超石等追索虜於半城,虜騎數重,籓及超石所領皆割配新軍,不盈五千,率厲力戰,大破之。又與超石等擊姚業於蒲阪,超石失利退還,籓收超石成捨資實,徐行而反,業不敢追。高祖還彭城,參相國軍事。時盧循餘黨與蘇淫賊大相聚結,以為始
 興相。論平司馬休之及廣固功,封陽山縣男,食邑五百戶。



 少帝景平元年,坐守東府,開掖門,免官,尋復其職。元嘉四年,遷建武將軍、江夏內史。七年,徵為遊擊將軍。到彥之北伐,南兗州刺史長沙王義欣進據彭城,籓出戍廣陵,行府州事。轉太子左衛率。十年,卒,時年六十二,謚曰壯侯。



 子隆世嗣,官至西陽太守。隆世卒,子乾秀嗣。籓庶子六十人,多不遵法度。



 籓第十四子遵世,為臧質寧遠參軍,去職還家,與孔熙先同逆謀,高祖以籓功臣,不欲顯
 其事,使江州以他事收殺之。二十四年,籓第十六子誕世、第十七子茂世率群從二百餘人攻破郡縣,殺太守桓隆之、令諸葛和之,欲奉庶人義康。值交州刺史檀和之至豫章,討平之。誕世兄車騎參軍新興太守景世、景世弟寶世,詣廷尉歸罪,並徙遠州。乾秀奪國。世祖初,徙者並得還。



 劉康祖,彭城呂人,世居京口。伯父簡之,有志干,為高祖所知。高祖將謀興復,收集才力之士,嘗再造簡之,值有
 賓客。簡之悟其意,謂弟虔之曰:「劉下邳頻再來,必當有意。既不得共語,汝可試往見之。」既至,高祖已克京城,虔之即便投義。簡之聞之,殺耕牛,會聚徒眾,率以赴高祖。簡之歷官至通直常侍,少府,太尉咨議參軍。簡之弟謙之,好學,撰《晉紀》二十卷;義熙末,為始興相。東海人徐道期流寓廣州,無士行,為僑舊所陵侮。因刺史謝欣死,合率群不逞之徒作亂,攻沒州城,殺士庶素憾者百餘,傾府軍、招集亡命,出攻始興。謙之破走之,進平廣州,誅其
 黨與,仍行州事。即以為振威將軍、廣州刺史。後為太中大夫。虔之誕節,不營產業,輕財好施。高祖西征司馬休之、魯宗之等,遣參軍檀道濟、朱超石步騎出襄陽,虔之時為江夏相,率府郡兵力出溳城,屯三連,立橋聚糧以待。道濟等積日不至,為宗之子軌所襲,眾寡不敵。參軍孫長庸流涕勸還軍,虔之厲色曰:「我仗順伐罪,理無不克。如其不幸,命也。」戰敗見殺,追贈梁、秦二州刺史,封新康縣男,食邑五百戶。



 康祖,虔之子也,襲封,為長沙王義
 欣鎮軍參軍,轉員外散騎侍郎。便弓馬,膂力絕人,在閭里不治士業,以浮蕩手莆酒為事。每犯法,為郡縣所錄,輒越屋踰墻,莫之能禽。夜入人家,為有司所圍守,康祖突圍而去,並莫敢追。因夜還京口,半夕便至。明旦,守門詣府州要職。俄而建康移書錄之,府州執事者並證康祖其夕在京口,遂見無恙。前後屢被糾劾,太祖以勳臣子,每原貸之。為員外郎十年,再坐摴手莆戲免。



 轉太子左積弩將軍,隨射聲校尉裴方明西征仇池,與方明同下廷
 尉,康祖免官。



 頃之,世祖為豫州刺史,鎮歷陽,以康祖為征虜中兵參軍,既被委任,折節自修。



 轉太子翊軍校尉。久之,遷南平王鑠安蠻府司馬。元嘉二十七年春,索虜托拔燾親率大眾攻圍汝南,太祖遣諸軍救援,康祖總統為前驅。軍次新蔡,與虜戰,俱前百餘里,濟融水。虜眾大至,奮擊破之,斬偽殿中尚書任城公乞地真,去縣瓠四十里,燾燒營退走。轉左軍將軍。



 太祖欲大舉北伐,康祖以歲月已晚,請待明年。上以河北義徒並起,若頓兵
 一周,沮向義之志,不許。其年秋,蕭斌、王玄謨、沈慶之等入河,康祖率豫州軍出許、洛。玄謨等敗歸,虜引大眾南度。南平王鑠在壽陽,上慮為所圍,召康祖速反。



 康祖回軍,未至壽陽數十里,會虜永昌王庫仁真以長安之眾八萬騎,與康祖相及於尉武。康祖凡有八千人,軍副胡盛之欲附山依險,間行取至。康祖怒曰:「吾受命本朝,清蕩河洛。寇今自送,不復遠勞王師,犬羊雖多,實易摧滅。吾兵精器練,去壽陽裁數十里,援軍尋至,亦何患乎!」乃
 結車營而進。虜四面來攻,大戰一日一夜,殺虜填積。虜分眾為三,且休且戰,以騎負草燒車營。康祖率厲將士,無不一當百,虜死者太半。會矢中頸死,於是大敗,舉營淪覆,為虜所殺盡,自免者裁數十人。虜傳康祖首示彭城,面如生。



 胡盛之為虜生禽,托跋燾寵之,常在左右。盛之有勇力,初為長沙王義欣鎮軍參軍督護,討劫譙郡,縣西劫有馬步七十,逃隱深榛,盛之挺身獨進,手斬五十八級。



 二十八年,詔曰:「康祖班師尉武,戎律靡忒。對眾
 以寡,殲殄太半。猛氣雲騰,志申力屈,沒世徇節,良可嘉悼。宜加甄寵,以旌忠烈。可贈益州刺史,謚曰壯男。」傳國至齊受禪,國除。



 垣護之,字彥宗,略陽桓道人也。祖敞,仕苻氏,為長樂國郎中令。慕容德入青州,以敞為車騎長史。德兄子超襲偽位,伯父遵、父苗復見委任。遵為尚書,苗京兆太守。高祖圍廣固,遵、苗踰城歸降,並以為太尉行參軍。太祖元嘉中,遵為員外散騎常侍,苗屯騎校尉。



 護之少倜儻,不
 拘小節,形狀短陋,而氣干彊果。從高祖征司馬休之,為世子中軍府長史,兼行參軍。永初中,補奉朝請。元嘉初,為殿中將軍。隨到彥之北伐,彥之將回師,護之為書諫曰:「外聞節下欲回師反旆,竊所不同。何者?殘虜畏威,望風奔迸,八載侵地,不戰克復。方當長驅朔漠,窮掃遺醜,況乃自送,無假遠勞。



 宜使竺靈秀速進滑臺助朱修之固守,節下大軍進擬河北,則牢、洛遊魂,自然奔退。



 且昔人有連年攻戰,失眾乏糧者,猶張膽爭前,莫肯輕退。況
 今青州豐穰,濟漕流通,士馬飽逸,威力無損。若空棄滑臺,坐喪成業,豈是朝廷受任之旨。」彥之不納,散敗而歸。太祖聞而善之,以補江夏王義恭征北行參軍、北高平太守。以載禁物繫尚方,久之蒙宥。又補衡陽王義季征北長流參軍,遷宣威將軍、鐘離太守。



 隨王玄謨入河,玄謨攻滑臺,護之百舸為前鋒,進據石濟;石濟在滑臺西南百二十里。及虜救至,又馳書勸玄謨急攻,曰:「昔武皇攻廣固,死沒者亦眾。況事殊曩日,豈得計士眾傷疲,願
 以屠城為急。」不從。玄謨敗退,不暇報護之。護之聞知,而虜悉已牽玄謨水軍大艚,連以鐵鎖三重斷河,欲以絕護之還路。河水迅急,護之中流而下,每至鐵鎖,以長柯斧斷之,虜不能禁。唯失一舸,餘舸並全。留戍靡溝城。還為江夏王義恭驃騎戶曹參軍,戍淮陰。加建武將軍,領濟北太守。率二千人復隨張永攻確磝,先據委慄津。虜杜道俊與偽尚書伏連來援確磝,護之拒之,賊因引軍東去。蕭思話遣護之迎軍至梁山,偽尚書韓元興率精騎
 卒至,護之依險拒戰,斬其都軍長史,甲首數十,賊乃退。思話將引還,誑護之云:「沈慶之救軍垂至,可急於濟口立橋。」護之揣知其意,即分遣白丁。思話復令度河戍乞活堡以防追軍。



 三十年春,太祖崩,遷屯歷下。聞世祖入討,率所領馳赴,上嘉之,以為督冀州青州之濟南樂安太原三郡諸軍事、寧遠將軍、冀州刺史。孝建元年,南郡王義宣反,兗州刺史徐遺寶,護之妻弟也。遠相連結,與護之書,勸使同逆。護之馳使以聞。遺寶時戍湖陸,護之留子
 恭祖守歷城,自率步騎襲遺寶。道經鄒山,破其別戍。



 未至湖陸六十里,遺寶焚城西走。袞土既定,徵為游擊將軍。



 隨沈慶之等擊魯爽,加輔國將軍。義宣率大眾至梁山,與王玄謨相持。柳元景率護之及護之弟詢之、柳叔仁、鄭琨等諸軍,出鎮新亭。玄謨見賊強盛,遣司馬管法濟求救甚急。上遣元景等進據南州,護之水軍先發。賊遣將龐法起率眾襲姑孰,適值護之、鄭琨等至,奮擊,大破之,斬獲及投水死略盡。玄謨馳信告元景曰:「西城不
 守,唯餘東城,眾寡相懸,請退還姑孰,更議進取。」元景不許,將悉眾赴救,護之勸分軍援之。元景然其計,乃以精兵配護之赴梁山。及戰,護之見賊舟艦累沓,謂玄謨曰:「今當以火平之。」即使隊主張談等燒賊艦,風猛水急,賊軍以此奔散。梁山平,護之率軍追討,會朱修之已平江陵,至尋陽而還。遷督徐袞二州豫州之梁郡諸軍事、寧朔將軍、徐州刺史,封益陽縣侯。食邑千戶。



 弟詢之,驍敢有氣力,元凶夙聞其名,以副輔國將軍張柬。時張超首
 行大逆,亦領軍隸柬。詢之規殺之,慮柬不同,柬宿有此志,又未測詢之同否,互相觀察。



 會超來論事,柬色動,詢之覺之,即共定謀,遣信召超。超疑之不至,改宿他所。



 詢之不知其移,徑斫之,殺其僕於床,因與柬南奔。柬溺淮死,詢之得至。時世祖已即位,以為積弩將軍。梁山之役力戰,為流矢所中。死,追贈冀州刺史。



 二年,護之坐論功挾私,免官。復為游擊將軍。俄遷大司馬,輔國將軍,領南東海太守。未拜,復督青冀二州諸軍事、寧遠將軍、青冀二州刺史,
 鎮歷城。明年,進號寧朔將軍。進督徐州之東莞東安二郡軍事。世祖以歷下要害,欲移青州并鎮歷城。議者多異。護之曰:「青州北有河、濟,又多陂澤,非虜所向。每來寇掠,必由歷城,二州并鎮,此經遠之略也。北又近河,歸順者易,近息民患,遠申王威,安邊之上計也。」由是遂定。



 大明三年,徵為右衛將軍,還,於道聞司空竟陵王誕於廣陵反叛,護之即率部曲受車騎大將軍沈慶之節度。事平,轉西陽王子尚撫軍司馬、臨淮太守。明年,出為使持節、
 督豫司二州諸軍事、輔國將軍、豫州刺史、淮南太守。復隸沈慶之伐西陽蠻。護之所蒞多聚斂,賄貨充積。七年,坐下獄,免官。明年,復起為太中大夫。



 未拜,其年卒,時年七十,謚曰壯侯。前廢帝永光元年,追贈冠軍將軍、豫州刺史。



 子承祖嗣。承祖卒,子顯宗嗣。齊受禪,國除。護之次子恭祖,勇果有父風。



 太宗泰始初,以軍功為梁、南秦二州刺史。



 遵子閬,元嘉中,為員外散騎侍郎。母墓為東阿寺道人曇洛等所發,閬與弟殿中將軍閎共殺曇洛等
 五人,詣官歸罪,見原。閬,大明三年,自義興太守為寧朔將軍、兗州刺史,為竟陵王誕所殺。追贈征虜將軍,刺史如故。閎,順帝昇明末,右衛將軍。



 張興世,字文德,竟陵竟陵人也。本單名世,太宗益為興世。少時家貧,南郡宗珍之為竟陵郡,興世依之為客。竟陵舊置軍府,以補參軍督護,不就。白衣隨王玄謨伐蠻,每戰,輒有禽獲,玄謨舊部曲諸將不及也,甚奇之。興世還都,白太祖,稱其膽力。後隨世祖鎮尋陽,以補南中參
 軍督護。入討元凶,隸柳元景為前鋒。事定,轉員外將軍,領從隊。南郡王義宣反,又隨玄謨出梁山,有戰功。除建平王宏中軍行參軍,領長刀。又隸西平王子尚為直衛。坐從子尚入臺,棄仗游走,下獄,免官。復以白衣充直衛。



 大明末,除員外散騎侍郎,仍除宣威將軍、隨郡太守。未行,太宗即位,四方反叛。進興世號龍驤將軍,領水軍,距南賊於赭圻。築二城於湖口,偽龍驤將軍陳慶領舸於前為游軍。興世率龍驤將軍佼長生、董凱之攻克二城,
 因擊慶,慶戰大敗,投水死者數千人。時臺軍據赭圻,南賊屯鵲尾,相持久不決。興世建議曰:「賊據上流,兵彊地勝。我今雖相持有餘,而制敵不足。今若以兵數千,潛出其上,因險自固,隨宜斷截,使其首尾周遑,進退疑沮,中流一梗,糧運自艱。制賊之奇,莫過於此。」沈攸之、吳喜並贊其計。時豫州刺史殷琰之據壽陽同逆,為劉勔所攻,南賊遣龐孟虯率軍助琰,劉勔遣信求援甚急。建安王休仁欲遣興世救之,問沈攸之。



 攸之曰:「孟虯蟻寇,必無
 能為。遣別將馬步數千,足以相制。若有意外,且以江西餌之。上流若捷,不憂不殄。興世之行,是安危大機,必不可輟。」乃遣段佛榮等援勔。



 興世欲率所領直取大雷,而軍旅未集,不足分張。會薛索兒平定,太宗使張永以步騎五千留戍盱眙,餘眾二萬人悉遣南討。山陽又尋平。征阮佃夫所領諸軍,悉還南伐,眾軍大集。乃分戰士七千配興世,興世乃令輕舸溯流而上,旋復回還,一二日中,輒復如此,使賊不為之備。劉胡聞興世欲上,笑之曰:「
 我尚不敢越彼下取揚州,張興世何物人,欲輕據我上!」興世謂攸之等曰:「上流唯有錢谿可據,地既險要,江又甚狹,去大眾不遠,應赴無難。江有洄洑,船下必來泊,岸有橫浦,可以藏船舸,二三為宜。」乃夜渡湖口,至鵲頭,因復回下疑之。其夜四更,值風,仍舉颿直前。賊亦遣胡靈秀諸軍,於東岸相翼而上。興世夕住景江浦宿,賊亦不進。



 夜潛遣黃道標領七十舸,徑據錢谿,營立城柴。明旦,興世與軍齊集。停一宿,劉胡自領水步二十六軍平旦
 來攻。將士欲迎擊之,興世禁曰:「賊來尚遠,而氣盛矢驟,驟既力盡,盛亦易衰,此曹劌之所以破齊也。」令將士不得妄動,治城如故。



 俄而賊來轉近,舫入洄洑,興世乃命壽寂之、任農夫率壯士數百擊之,眾軍相繼進,胡於是敗走。斬級數百,投水者甚眾,胡收軍而下。



 時興世城壘未固,司徒建安王休仁慮賊并力更攻錢溪,欲分其形勢,命沈攸之、吳喜、佼長生、劉靈遺等以皮艦二十,攻賊濃湖,苦戰連日,斬獲千數。是日,劉胡果率眾軍,欲更攻
 興世。未至錢谿數十里,袁顗以濃湖之急遽追之,錢溪城柴由此得立。賊連戰轉敗,興世又遏其糧道,尋陽遣運至南陵,不敢下,賊眾漸饑。劉胡乃遣顗安北府司馬、偽右軍沈仲玉領千人步取南陵,迎接糧運。仲玉至南陵,領米三十萬斛,錢布數十舫,豎榜為城,規欲突過。行至貴口,不敢進,遣間信報胡,令遣重軍援接。興世、壽寂之、任農夫、李安民等三千人至貴口擊之,與仲玉相值。



 交戰盡日,仲玉走還顗營,悉虜其資實;賊眾大敗,胡
 棄軍遁走,顗仍亦奔散。興世率軍追討,與吳喜共平江陵。遷左軍將軍,尋為督豫司二州南豫州之梁郡諸軍事,封作唐縣侯,食邑千戶。徵為游擊將軍。



 海道北伐,假輔國將軍,加節置佐,無功而還。四年,遷太子右衛率,又以本官領驍騎將軍,與左衛將軍沈攸之參員置。五年,轉左衛將軍。六年,中領軍劉勔當鎮廣陵,興世權兼領軍。泰豫元年,為持節、督雍梁南北秦郢州之竟陵隨二郡諸軍事、冠軍將軍、雍州刺史,尋加寧蠻校尉。桂陽王休
 範反,興世遣軍赴朝廷,未發而事平。進號征虜將軍。廢帝元徽三年,徵為通直散騎常侍、左衛將軍。五年,以疾病,徙光祿大夫,常侍如故。順帝昇明二年,卒,時年五十九。追贈本官。



 興世居臨沔水,沔水自襄陽以下,至于九江,二千里中,先無洲嶼。興世初生,當其門前水中,一旦忽生洲,年年漸大,及至興世為方伯,而洲上遂十餘頃。父仲子,由興世致位給事中。興世欲將往襄陽,愛戀鄉里,不肯去。嘗謂興世:「我雖田舍老公,樂聞鼓角,可送一部,
 行田時吹之。」興世素恭謹畏法憲,譬之曰:「此是天子鼓角,非田舍老公所吹。」興世欲拜墓,仲子謂曰:「汝衛從太多,先人必當驚怖。」興世減撤而後行。



 興世子欣業,當嗣封,會齊受禪,國除。



 史臣曰:兵固詭道,勝在用奇。當二帝爭雄,天人之分未決,南北連兵,相厄而不得進者,半歲矣。蓋乃趙壁拔幟之機,官渡熸師之日,至於鵲浦投戈,實興世用奇之力也。建旆垂組,豈徒然哉!



\end{pinyinscope}