\article{卷四十七列傳第七 劉懷肅 孟懷玉 弟龍符 劉敬宣 檀祗}

\begin{pinyinscope}

 劉懷肅,彭城人,高祖從母兄也。家世貧窶,而躬耕好學。初為劉敬宣寧朔府司馬,東征孫恩,有戰功,又為龍驤司馬、費令。聞高祖起義,棄縣來奔。京邑平定,振武將軍
 道規追桓玄,以懷肅為司馬。玄留何澹之、郭銓等戍桑落洲,進擊破之。潁川太守劉統平,除高平太守。玄既死,從子振大破義軍於楊林,義軍退尋陽。



 懷肅與江夏相張暢之攻澹之於西塞,破之。偽鎮東將軍馮該戍夏口東岸,孟山圖據魯山城,桓仙客守偃月壘,皆連壁相望。懷肅與道規攻之,躬擐甲胄,陷二城,馮該走石城,生擒仙客。義熙元年正月,振敗走,道規遣懷肅平石城,斬馮該及其子山靖。三月,桓振復襲江陵,荊州刺史司馬休之
 出奔,懷肅自雲杜馳赴,日夜兼行,七日而至。振勒兵三萬,旗幟蔽野,躍馬橫矛,躬自突陳。流矢傷懷肅額,眾懼欲奔,懷肅瞋目奮戰,士氣益壯。於是士卒爭先,臨陣斬振首。江陵既平,休之反鎮,執懷肅手曰:「微子之力,吾無所歸矣。」偽輔國將軍符嗣、馬孫、偽龍驤將軍金符青、樂志等屯結江夏,懷肅又討之,梟樂志等。道規加懷肅督江夏九郡,權鎮夏口。



 除通直郎。仍為輔國將軍、淮南歷陽二郡太守。二年,又領劉毅撫軍司馬,軍、郡如故。以義
 功封東興縣侯,食邑千戶。其冬,桓石綏、司馬國璠、陳襲於胡桃山聚眾為寇,懷肅率步騎討破之。江淮間群蠻及桓氏餘黨為亂,自請出討,既行失旨,毅上表免懷肅官。三年,卒,時年四十一。追贈左將軍。無子,弟懷慎以子蔚祖嗣封,官至江夏內史。



 蔚祖卒,子道存嗣。太祖元嘉末,為太尉江夏王義恭咨議參軍。世祖伐元凶,義軍至新亭,道存出奔,元凶殺其母以徇。前廢帝景和中,為義恭太宰從事中郎。



 義恭敗,以黨與下獄死。



 懷肅次弟懷
 敬,澀訥無才能。初,高祖產而皇妣殂,孝皇帝貧薄,無由得乳人,議欲不舉高祖。高祖從母生懷敬,未期,乃斷懷敬乳,而自養高祖。高祖以舊恩,懷敬累見寵授,至會稽太守,尚書,金紫光祿大夫。



 懷敬子真道,為錢唐令。元嘉十三年,東土饑,上遣揚州治中從事史沈演之巡行在所,演之上表曰:「宰邑輔政,必其簡惠成能;蒞職闡治,務以利民著績。故王奐見紀於前,叔卿流稱於後。竊見錢唐令劉真道、餘杭令劉道錫,皆奉公恤民,恪勤匪懈,百
 姓稱詠,訟訴希簡。又翦蕩凶非,屢能擒獲。災水之初,餘杭高堤崩潰,洪流迅激,勢不可量;道錫躬先吏民,親執板築,塘既還立,縣邑獲全。經歷諸縣,訪覈名實,並為二邦之首最,治民之良宰。」上嘉之,各賜穀千斛,以真道為步兵校尉。



 十四年,出為梁、南秦二州刺史。十八年,氐賊楊難當侵寇漢中,真道率軍討破之。而難當寇盜猶不已,太祖遣龍驤將軍裴方明率禁兵五千,受真道節度。十九年,方明至武興,率太子積弩將軍劉康祖、後軍參
 軍梁坦、陳彌、裴肅之、安西參軍段叔文、魯尚期、始興王國常侍劉僧秀、綏遠將軍馬洗、振武將軍王奐之等,進次潭谷,去蘭皋數里。難當遣其建節將軍苻弘祖、啖元等固守蘭皋,鎮北將軍苻德義於外為游軍,難當子撫軍大將軍和重兵繼其後。方明進擊,大破之於濁水,斬弘祖并三千餘級。遣康祖追之,過蘭皋二千餘里。和又遣德義助戰,康祖又大破之,和退保脩城。難當遣建忠將軍楊林、振威將軍姚憲領二千騎就和,方明又率諸
 將攻之。和敗走,追至赤亭,難當席卷奔叛。方明遣康祖直趣百頃,偽丞相楊萬壽等一時歸降。難當第三息虎先戍陰平,難當既走,虎逃竄民間,生禽之,送京都,斬於建康市。



 秦州刺史胡從之西鎮百頃,行至濁水,為索虜所邀擊,敗沒。以真道為建威將軍、雍州刺史,方明輔國將軍、梁南秦二州刺史。方明辭不拜。詔曰:「往年氐豎楊難當造為叛亂,俯首者眾。其長史楊萬壽、建節將軍姚憲,情不違順,屢進矢言。



 及凶醜宵遁,闔境崩擾,建忠將
 軍呂訓衛倉儲以候王師。寧朔將軍姜檀果烈懇到,志在宣力,濁水之捷,厥庸顯然,近者協贊義奮,乃心無替。略陽苻昭,誠係本朝,亦同斯舉,俘擒偽將,獨克武興,推鋒致效,隕命寇手。並事著屯險,感於予懷,宜蒙旌敘,榮慰存亡。可贈萬壽龍驤將軍,昭武都太守;憲補員外散騎侍郎,訓駙馬都尉、奉朝請;檀征西大將軍司馬、仇池太守,宜並內徙。可符雍、梁二州,厚加贍恤。」呂訓,略氐人呂先子也。又詔曰:「故晉壽太守姜道盛,前討仇池,志輸
 誠力,即戎著效,臨財能清。近先登濁水,殞身鋒鏑,誠節俱亮,矜悼於懷。可贈給事中,賜錢千萬。」道盛注《古文尚書》,行於世。



 真道、方明並坐破仇池,斷割金銀諸雜寶貨,又藏難當善馬,下獄死。劉康祖等繫免各有差。方明,河東人,為劉道濟振武中兵參軍,立功蜀土,歷潁川南平昌太守,皆坐贓私免官。



 孟懷玉,平昌安丘人也。高祖珩,晉河南尹。祖淵,右光祿大夫。父綽,義旗後為給事中,光祿勳,追贈金紫光祿大
 夫。世居京口。



 高祖東伐孫恩,以懷玉為建武司馬。豫義旗,從平京城,進定京邑。以功封鄱陽縣侯,食邑千戶。高祖鎮京口,以懷玉為鎮軍參軍、下邳太守。義熙三年,出為寧朔將軍、西陽太守、新蔡內史,除中書侍郎,轉輔國將軍,領丹陽府兵,戍石頭。



 盧循逼京邑,懷玉於石頭岸連戰有功,為中軍咨議參軍。賊帥徐道覆屢欲以精銳登岸,畏懷玉不敢上。及循南走,懷玉與眾軍追躡,直至嶺表。徐道覆屯結始興,懷玉攻圍之,身當矢石,旬月乃
 陷。仍南追循,循平,又封陽豐縣男,食邑二百五十戶。復為太尉咨議參軍,征虜將軍。八年,遷江州刺史,尋督江州豫州之西陽新蔡汝南潁川司州之恆農揚州之松滋六郡諸軍事、南中郎將,刺史如故。時荊州刺史司馬休之居上流,有異志,故授懷玉此任以防之。十一年,加持節。丁父艱,懷玉有孝性。因抱篤疾,上表陳解,不許。又自陳弟仙客出繼,喪主唯己,乃見聽。未去任,其年卒官。時年三十一。追贈平南將軍。子元卒,無子,國除。懷玉別封陽豐男,
 子慧熙嗣,坐廢祭祀奪爵。慧熙子宗嗣,竟陵太守,中大夫。



 龍符,懷玉弟也。驍果有膽氣,幹力絕人。少好游俠,結客於閭里。早為高祖所知,既克京城,以龍符為建武參軍。江乘、羅落、覆舟三戰,並有功。參鎮軍軍事,封平昌縣五等子,加寧遠將軍、淮陵太守。與劉籓、向彌征桓歆、桓石康,破斬之。除建威將軍、東海太守。索虜斛蘭、索度真侵邊,彭、沛騷擾,高祖遣龍符、建威將軍道憐北討,一戰破之。追斛蘭至光水溝邊,被創奔走。



 高祖伐廣固,以龍
 符為車騎參軍,加龍驤將軍、廣川太守,統步騎為前鋒。軍達臨朐,與賊爭水,龍符單騎衝突,應手破散,即據水源,賊遂退走。龍符乘勝奔逐,後騎不及,賊數千騎圍繞攻之。龍符奮槊接戰,每一合輒殺數人,眾寡不敵,遂見害,時年三十三。高祖深加痛悼,追贈青州刺史。又表曰:「故龍驤將軍、廣川太守孟龍符,忠勇果毅,隕身王事,宜蒙甄表,以顯貞節,聖恩嘉悼,寵贈方州。



 龍符投袂義初,前驅效命,推鋒三捷,每為眾先。及西劋桓歆,北殄索虜,
 朝議爵賞,未及施行。會今北伐,復統前旅,臨朐之戰,氣冠三軍。於時逆徒實繁,控弦掩澤,龍符匹馬電躍,所向摧靡,奪戈深入,知死弗吝。賊超奔遁,依險鳥聚,大軍因勢,方軌長驅。考其庸績,豫參濟不,竊謂宜班爵土,以褒勳烈。」乃追封臨沅縣男,食邑五百戶。無子,弟仙客以子微生嗣封。太祖元嘉中,有罪奪爵,徙廣州,以微生弟彥祖子佛護襲爵。齊受禪,國除。孝武大明初,諸流徒者悉聽還本,微生已死,子係祖歸京都,有筋幹異力,能人詹負
 數人,入隸羽林,為殿中將軍。



 二年,索虜寇青、冀,世祖遣軍援之,係祖自占求行。戰於杜梁,挺身入陳,所殺狼籍,遂見殺。詔書追贈潁川郡太守。



 劉敬宣,字萬壽,彭城人,漢楚元王交後也。祖建,征虜將軍。父牢之,鎮北將軍。敬宣八歲喪母,晝夜號泣,中表異之。輔國將軍桓序鎮蕪湖,牢之參序軍事。



 四月八日,敬宣見眾人灌佛,乃下頭上金鏡以為母灌,因悲泣不自勝,序歎息,謂牢之曰:「卿此兒既為家之孝子,必為國之
 忠臣。」起家為王恭前軍參軍,又參會稽世子元顯征虜軍事。



 隆安三年,王恭起兵於京口,以誅司馬尚之兄弟為名。牢之時為恭前軍司馬、輔國將軍、晉陵太守,置佐領兵。而恭以豪戚自居,甚相陵忽,牢之心不能平。及恭此舉,使牢之為前鋒。太傅會稽王道子與牢之書,備言禍福,使以兵反恭。牢之呼敬宣謂曰:「王恭昔蒙先帝殊恩,今居伯舅之重,義心未彰,唯兵是縱。吾不能審恭事捷之日,必能奉戴天子,緝穆宰相與不。今欲奉國威靈,
 以明逆順,汝以為何如?」敬宣曰:「朝廷雖無成、康之隆,未有桓、靈之亂,而恭怙亂阻兵,志陵京邑。大人與恭親無骨肉,分非君臣,雖共事少時,意好不協。今日討之,於情何有?」牢之至竹里,斬恭大將顏延,遣敬宣率高雅之等還京襲恭。恭方出城耀軍,馳騎橫擊之,一時散潰。元顯進號後將軍,以敬宣為咨議參軍,加寧朔將軍。



 三年,孫恩為亂,東土騷擾,牢之自表東討,軍次虎矰。賊皆死戰,敬宣請以騎傍南山趣其後,吳賊畏馬,又懼首尾受敵,
 遂大敗。進平會稽,尋加臨淮太守,遷後軍從事中郎。五年,孫恩又入浹口,高祖戍句章,賊頻攻不能拔。敬宣請往為援,賊恩於是退遠入海。是時四方雲擾,朝廷微弱,敬宣每慮艱難未已,高祖既累破妖賊,功名日盛,故敬宣深相憑結,情好甚隆。元顯進號驃騎,敬宣仍隨府轉,軍、郡如故。元顯驕淫縱肆,群下化之;敬宣每預燕會,未嘗欽酒,調戲之來,無所酬答,元顯甚不說。尋進號輔國將軍,餘如故。



 元興元年,牢之南討桓玄,元顯為征討大
 都督,日夜昏酣,牢之驟詣門,不得相見;帝出餞行,方遇公坐而已。桓玄既至溧州,遣信說牢之;牢之以道子昏暗,元顯淫凶,慮平玄之日,亂政方始,假手於玄,誅除執政,然後乘玄之隙,可以得志於天下,將許玄降。敬宣諫曰:「方今國家亂擾,四海鼎沸,天下之重,在大人與玄。玄藉先父之基,據荊南之勢,雖無姬文之德,實為參分之形。一朝縱之,使陵朝廷,威望既成,則難圖也。董卓之變,將生於今。」牢之怒曰:「吾豈不知今日取玄如反覆手,但
 平玄之後,令我那驃騎何?」遺敬宣為任,玄板為其府咨議參軍。



 玄既得志,害元顯,廢道子,以牢之為征東將軍、會稽太守。牢之與敬宣謀共襲玄,期以明旦。值爾日大霧,府門晚開,日旰,敬宣不至,牢之謂所謀已泄,率部曲向白洲,欲奔廣陵。而敬宣還京口迎家,牢之尋求不得,謂已為玄所擒,乃自縊死。敬宣奔喪,哭畢,即渡江就司馬休之、高雅之等,俱奔洛陽,往來長安,各以子弟為質,求救於姚興。興與之符信,令關東募兵,得數千人,復還
 至彭城間,收聚義故。玄遣孫無終討冀州刺史劉軌,軌要敬宣、雅之等共據山陽破之,不剋。



 又進昌平澗,戰不利,眾各離散,乃俱奔鮮卑慕容德。



 敬宣素曉天文,知必有興復晉室者。尋夢丸土服之,既覺,喜曰:「丸者桓也。



 桓既吞矣,吾復本土乎!」乃結青州大姓諸崔、封,并要鮮卑大帥免逵,謀滅德,推休之為主,克日垂發。時劉軌為德司空,大被委任,雅之又欲要軌。敬宣曰:「此公年老,吾觀其有安齊志,必不動,不可告也。」雅之以為不然,遂告軌,
 軌果不從。謀頗泄,相與殺軌而去。至淮、泗間,會高祖平京口,手書召敬宣;左右疑其詐,敬宣曰:「吾固知其然矣。下邳不誘我也。」即便馳還。既至京師,以敬宣為輔國將軍、晉陵太守,襲封武岡縣男。是歲,安帝元興三年也。



 桓歆率氐賊楊秋寇歷陽,敬宣與建威將軍諸葛長民大破之。歆單騎走渡淮,斬楊秋於練固而還。遷建威將軍、江州刺史。敬宣固辭,言於高祖曰:「仇恥既雪,四海清蕩,所願反身草澤,以終餘年。恩遇不遣,遂復僶俛,即目所
 忝,已為優渥。



 且盤龍、無忌猶未遇寵,賢二弟位任尚卑,一朝先之,必貽朝野之責。」不許。敬宣既至江州,課集軍糧,搜召舟乘,軍戎要用,常有儲擬。故囗徵諸軍雖失利退據,因之每即振復。其年,桓玄兄子亮自號江州刺史,寇豫章;亮又遣苻宏寇廬陵,敬宣並討破之。



 初,劉毅之少也,為敬宣寧朔參軍。時人或以雄傑許之,敬宣曰:「夫非常之才,當別有調度,豈得便謂此君為人豪邪?其性外寬而內忌,自伐而尚人,若一旦遭逢,亦當以陵上取
 禍耳。」毅聞之,深以為恨。及在江陵,知敬宣還,乃使人言於高祖曰:「劉敬宣父子,忠國既昧,今又不豫義始。猛將勞臣,方須敘報,如敬宣之比,宜令在後。若使君不忘平生,欲相申起者,論資語事,正可為員外常侍耳。



 聞已授其郡,實為過優;尋知復為江州,尤所駭惋。」敬宣愈不自安。安帝反正,自表解職。於是散徹,賜給宅宇,月給錢三十萬。高祖數引與游宴,恩款周洽,所賜錢帛車馬及器服玩好,莫與比焉。尋除冠軍將軍、宣城內史、襄城太守。
 宣城多山縣,郡舊立屯以供府郡費用,前人多發調工巧,造作器物。敬宣到郡,悉罷私屯,唯伐竹木,治府舍而已。亡叛多首出,遂得三千餘戶。



 高祖方大相寵任,欲先令立功。義熙三年,表遣敬宣率眾五千伐蜀。國子博士周祗書諫高祖曰:「自義旗之建,所征無不必克,此可謂天人交助,信順之徵也。



 今大難已夷,君臣俱泰。頃五穀轉豐,民無饑苦,劫盜之患,亦為弭息,比誠漸足無事,宜大寧治本。蜀賊宜平,六合宜一,非為不爾也。古人有言,
 天時不如地利,地利不如人和。今往伐蜀,萬有餘里,溯流天險,動經時歲。若此軍直指成都,徑禽譙氏者,復是將帥奮威,一快之舉耳。然益士荒殘,野無青草,成都之內,殆無孑遺。計得彼利,與今行軍之費,不足相補也。而今往艱險,雨雪方降,驅三州三吳之人,投之三巴三蜀之土,其中疾病死亡,豈可稱計。此一疑也。賊必不守窮城,將決力戰。今我往勞困,彼來甚逸。若忽使師行不利,人情波駭,大勢挫衄。此二疑也。且千里饋糧,士有饑色。
 況今溯險萬里,所在無儲。若連兵不解,運漕不繼,雖韓、白之將,何以成功。此三疑也。今云可徵者皆云:『彼親離眾叛。』愚謂不然。彼以一匹夫,而能致今日之事,若眾力離散,亦何以至此。官所遣兵皆烏合受募之人,亦必無千人一心,有前無退矣。為治者固先定其內而理其外,先安其近而懷其遠。自頃狂狡不息,誅戮相繼,未可謂人和也。天險如彼,未可謂地利也。毛修之家仇不雪,不應以得死為恨;劉敬宣蒙生存之恩,亦宜性命仰報。今將軍
 欲驅二死之甘心,而忘國家之重計,愚情竊所未安。闕門之外,非所宜豫,茍其有心,不覺披盡。」不從。



 假敬宣節,監征蜀諸軍事,郡如故。既入峽,分遣振武將軍、巴東太守溫祚以二千人揚聲外水,自率益州刺史鮑陋、輔國將軍文處茂、龍驤將軍時延祖由墊江而進。敬宣率先士卒,轉戰而前,達遂寧郡之黃虎,去成都五百里。偽輔國將軍譙道福等悉眾距險,相持六十餘日,大小十餘戰,賊固守不敢出。敬宣不得進,食糧盡,軍中多疾疫,死
 者太半,引軍還。譙縱送毛璩一門諸喪,其妻女、文處茂母何,并諸士人喪柩,浮之中流,敬宣皆拯接致歸。為有司所奏,免官,削封三分之一。



 五年,高祖伐鮮卑,除中軍咨議參軍,加冠軍將軍。從至臨朐,慕容超出軍距戰,敬宣與兗州刺史劉籓等奮擊,大破之。龍驤將軍孟龍符戰沒,敬宣并領其眾,圍廣固,屢獻規略。盧循逼京師,敬宣分領鮮卑虎班突騎,置陣甚整,循等望而畏之。遷使持節、督馬頭淮西諸軍郡事、鎮蠻護軍、淮南安豐二郡
 太守、梁國內史,將軍如故。循既走,仍從高祖南討,轉左衛將軍,加散騎常侍。



 敬宣寬厚善待士,多伎藝,弓馬音律,無事不善。時尚書僕射謝混自負才地,少所交納,與敬宣相遇,便盡禮著歡。或問混曰:「卿未嘗輕交於人,而傾蓋於萬壽,何也?」混曰:「人之相知,豈可以一塗限。孔文舉禮太史子義,夫豈有非之者邪!」



 初,敬宣回師於蜀,劉毅欲以重法繩之;高祖既相任待,又何無忌明言於毅,謂不宜以私憾傷至公,若必文致為戮,己當入朝以廷
 議決之。毅雖止,猶謂高祖曰:「夫生平之舊,豈可孤信。光武悔之於龐萌,曹公失之於孟卓,公宜深慮之。」毅出為荊州,謂敬宣曰:「吾忝西任,欲屈卿為長史、南蠻,豈有見輔意乎?」敬宣懼禍及,以告高祖。高祖笑曰:「但令老兄平安,必無過慮。」出為使持節、督北青州軍郡事、征虜將軍、北青州刺史,領青河太守,尋領冀州刺史。



 時高祖西討劉毅,豫州刺史諸葛長民監太尉軍事,貽敬宣書曰:「盤龍狼戾專恣,自取夷滅,異端將盡,世路方夷,富貴之事,
 相與共之。」敬宣報曰:「下官自義熙以來,首尾十載,遂忝三州七郡。今此杖節,常懼福過禍生,實思避盈居損;富貴之旨,非所敢當。」遣使呈長民書,高祖謂王誕曰:「阿壽故為不負我也。」



 十一年正月,進號右將軍。



 司馬道賜者,晉宗室之賤屬也。為敬宣參軍。至高祖西征司馬休之,道賜乃陰結同府辟閭道秀及左右小將王猛子等謀反。道賜自號齊王,以道秀為青州刺史,規據廣固,舉兵應休之。敬宣召道秀有所論,因屏人,左右悉出戶,猛子
 逡巡在後,取敬宣備身刀殺敬宣,時年四十五。文武佐吏即討道賜、猛子等,皆斬之。先是,敬宣未死,嘗夜與僚佐宴集,空中有放一只芒屩於坐中,墜敬宣食槃上,長三尺五寸,已經人著,耳鼻間並欲壞。頃之而敗。喪至,高祖臨哭甚哀。子祖嗣。宋受禪,國除。



 檀祗,字恭叔,高平金鄉人,左將軍歆第二弟也。少為孫無終輔國參軍,隨無終東征孫恩,屢有戰功。復為王誕龍驤參軍。從高祖克京城,參建武軍事。至羅落,檀憑之
 戰沒之後,仍以憑之所領兵配祗。京邑既平,參鎮軍事,加振武將軍,隸振武大將軍道規追討桓玄,每戰克捷。江陵平定,道規遣祗征溳、沔亡命桓道兒、張靖、苻嗣等,皆悉平之。除龍驤將軍、秦郡太守、北陳留內史;又為寧朔將軍、竟陵太守,不拜。破桓亮於長沙,苻宏於湘東。武陵內史庾悅疾病,道規以祗代悅,加寧朔將軍,封西昌縣侯,食邑千戶。五年,入為中書侍郎。



 盧循逼京邑,加輔國將軍,領兵屯西明門外。循退走,祗率所領,步道援江
 陵,未發,遇疾停。八年,遷右衛將軍,出為輔國將軍、宣城內史,即本號督江北淮南軍郡事、青州刺史、廣陵相。進號征虜將軍,加節。十年,亡命司馬國璠兄弟自北徐州界聚眾數百,潛得過淮,因天夜陰暗,率百許人緣廣陵城得入,叫喚直上聽事。



 祗驚起,出門將處分,賊射之,傷股,乃入。祗語左右:「賊乘暗得入,欲掩我不備。但打五鼓,懼曉,必走矣。」賊聞鼓鳴,謂為曉,於是奔散,追討殺百餘人。


祗降號建武將軍。十一年,進號右將軍。十二年,高
 祖北伐,而亡命司馬囗寇塗
 \gezhu{
  塗或作滁}
 中,秦郡太守劉基求救,分軍掩討,即破斬之。



 十四年,宋國初建,天子詔曰:「宋國始立,內外草創,禁旅王要,總司須才。



 右將軍祗可為宋領軍將軍,加散騎常侍。」祗性矜豪,樂在外放恣,不願內遷,甚不得志。發疾不自治,其年卒廣陵,時年五十一。贈散騎常侍、撫軍將軍,謚曰威侯。



 子獻嗣,元熙中卒,無子,祗次子朗紹封。朗卒,子宣明嗣。宣明卒,子逸嗣。



 齊受禪,國除。



 史臣曰:劉敬宣與高祖恩結龍潛,義分早合,雖興復之始,事隔逢迎,而深期久要,未之或爽。隆赫之任,義止於人存;飾終之數,無聞於身後。恩禮之有厚薄者,將有以乎!



\end{pinyinscope}