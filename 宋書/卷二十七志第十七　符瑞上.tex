\article{卷二十七志第十七 符瑞上}

\begin{pinyinscope}

 夫體睿窮幾,含靈獨秀,謂之聖人,所以能君四海而役萬物,使動植之類,莫不各得其所。百姓仰之,懽若親戚,芬若椒蘭,故為旗章輿服以崇之,玉璽黃屋以尊之。以
 神器之重,推之於兆民之上,自中智以降,則萬物之為役者也。性識殊品,蓋有愚暴之理存焉。見聖人利天下,謂天下可以為利;見萬物之歸聖人,謂之利萬物。力爭之徒,至以逐鹿方之,亂臣賊子,所以多於世也。夫龍飛九五,配天光宅,有受命之符,天人之應。《易》曰:「河出《圖》,洛出《書》,而聖人則之。」



 符瑞之義大矣。



 赫胥、燧人之前,無聞焉。太昊帝宓犧氏,母曰華胥。燧人之世,有大跡出雷澤,華
 胥履之,而生伏犧於成紀。蛇身人首,有聖德。燧人氏沒,宓犧代之,受《龍圖》,畫八卦,所謂「河出《圖》」者也。有景龍之瑞。炎帝神農氏,母曰女登,遊於華陽,有神龍首感女登於常羊山,生炎帝。人身牛首,有聖德,致大火之瑞。嘉禾生,醴泉出。



 黃帝軒轅氏,母曰附寶,見大電光繞北斗樞星,照郊野,感而孕。二十五月而生黃帝於壽丘。弱而能言,龍顏,有
 聖德,劾百神朝而使之。應龍攻蚩尤,戰虎、豹、熊、羆四獸之力。以女魃止淫雨。天下既定,聖德光被,群瑞畢臻。有屈軼之草生於庭,佞人入朝,則草指之,是以佞人不敢進。有景雲之瑞,有赤方氣與青方氣相連;赤方中有兩星,青方中有一星,凡三星,皆黃色,以天清明時見於攝提,名曰景星。黃帝黃服齋於中宮,坐於玄扈洛水之上,有鳳皇集,不食生蟲,不履生草,或止帝之東園,或巢於阿閣,或鳴於庭,其雄自歌,其雌自舞。麒麟在囿,神鳥來
 儀。有大螻如羊,大螾如虹。黃帝以土氣勝,遂以土德王。五十年秋七月庚申,天霧三日三夜,晝昏。黃帝以問天老、力牧、容成曰:「於公何如?」天老曰:「臣聞之,國安,其主好文,則鳳凰居之;國亂,其主好武,則鳳凰去之。今鳳凰翔於東郊而樂之,其鳴音中夷則,與天相副。以是觀之,天有嚴教以賜帝,帝勿犯也。」乃召史卜之,龜燋。史曰:「臣不能占也。其問之聖人。」帝曰:「已問天老、力牧、容成矣。」史北面再拜曰:「龜不違聖智,故燋。」霧除,遊於洛水之上,見大
 魚,殺五牲以醮之,天乃甚雨;七日七夜,魚流於海,得《圖》、《書》焉。《龍圖》出河,《龜書》出洛,赤文篆字,以授軒轅。軒轅接萬神於明庭,今寒門谷口是也。



 帝摯少昊氏,母曰女節,見星如虹,下流華渚,既而夢接意感,生少昊。登帝位,有鳳皇之瑞。帝顓頊高陽氏,母曰女樞,見瑤光之星,貫月如虹,感己於幽房之宮,生顓頊於若水。首戴干戈,有聖德。生十年而佐少昊氏,二十而登帝位。
 帝嚳高辛氏,生而駢齒,有聖德,代高陽氏王天下。使鼓人拊鞞鼓,擊鐘磬,鳳凰鼓翼而舞。



 帝堯之母曰慶都,生於斗維之野,常有黃雲覆護其上。及長,觀於三河,常有龍隨之。一旦龍負《圖》而至,其文要曰:「亦受天祐。」眉八彩,鬢髮長七尺二寸,面銳上豐下,足履翼宿。既而陰風四合,赤龍感之。孕十四月而生堯於丹陵,其狀如圖。及長,身長十尺,有聖德,封於唐。夢攀天而上。高辛氏衰,天下歸之。



 在帝位七十年,景星出翼,鳳
 凰在庭,朱草生,嘉禾秀,甘露潤,醴泉出,日月如合璧,五星如連珠。廚中自生肉,其薄如箑,搖動則風生,食物寒而不臭,名曰「箑脯。」又有草夾階而生,月朔始生一莢,月半而生十五莢,十六日以後,日落一莢,及晦而盡;月小則一莢焦而不落,名曰「蓂莢,一曰「歷莢」。歸功於舜,將以天下禪之。乃潔齋修壇場於河、洛,擇良日,率舜等升首山,遵河渚。有五老游焉,蓋五星之精也。相謂曰:「《河圖》將來告帝以期,知我者重瞳黃姚。」五老因飛為流星,上入
 昴。二月辛丑昧明,禮備,至於日昃,榮光出河,休氣四塞,白雲起,回風搖,乃有龍馬銜甲,赤文綠色,臨壇而止,吐《甲圖》而去。甲似龜,背廣九尺,其圖以白玉為檢,赤玉為字,泥以黃金,約以青繩。檢文曰:「闓色授帝舜。」言虞、夏、殷、周、秦、漢當授天命。帝乃寫其言,藏於東序。後二年二月仲辛,率群臣沈璧於洛。禮畢,退俟,至於下昃,赤光起,玄龜負書而出,背甲赤文成字,止於壇。其書言當禪舜,遂讓舜。



 帝舜有虞氏,母曰握登,見大虹意感,而生舜於姚墟。目重瞳子,故名重華。



 龍顏大口,黑色,身長六尺一寸。舜父母憎舜,使其塗廩,自下焚之,舜服鳥工衣服飛去。又使浚井,自上填之以石,舜服龍工衣自傍而出。耕於歷山,夢眉長與髮等。及即帝位,蓂莢生於階,鳳凰巢於庭,擊石拊石,百獸率舞,景星出房,地出乘黃之馬,西王母獻白環、玉玦。舜在位十有四年,奏鐘石笙筦未罷,而天大雷雨,疾風發屋拔木,桴鼓播地,鐘磬亂行,舞人頓伏,樂
 正狂走。舜乃擁璇持衡而笑曰:「明哉!夫天下非一人之天下也,亦乃見於鐘石笙筦乎!」乃薦禹於天,使行天子事。於時和氣普應,慶雲興焉,若煙非煙,若雲非雲,郁郁紛紛,蕭索輪囷,百工相和而歌《慶雲》。帝乃倡之曰:「慶雲爛兮,糾縵縵兮。日月光華,旦復旦兮。」



 群臣咸進,稽首曰:「明明上天,爛然星陳。日月光華,弘予一人。」帝乃再歌曰:「日月有常,星辰有行。四時從經,萬姓允誠。於予論樂,配天之靈。遷於聖腎,莫不咸聽。鼚乎鼓之,軒乎舞之。精華
 以竭,褰裳去之。」於是八風修通,慶雲業聚,蟠龍奮迅於其藏,蛟魚踴躍於其淵,龜鱉咸出其穴,遷虞而事夏。舜乃設壇於河,依堯故事。至於下昃,榮光休氣至,黃龍負《圖》,長三十二尺,廣九尺,出於壇畔,赤文綠錯,其文言當禪禹。



 帝禹有夏氏,母曰修己,出行,見流星貫昴,夢接意感,既而吞神珠。修己背剖,而生禹於石紐。虎鼻大口,兩耳參鏤,首戴鉤鈐,胸有玉斗,足文履己,故名文命。長有聖德。長
 九尺九寸,夢自洗於河,以手取水飲之;又有白狐九尾之瑞。



 當堯之世,舜舉之。禹觀於河,有長人白面魚身,出曰:「吾河精也。」呼禹曰:「文命治淫。」言訖,授禹《河圖》,言治水之事,乃退入於淵。禹治水既畢,天錫玄珪,以告成功。夏道將興,草木暢茂,青龍止於郊,祝融之神,降於崇山。乃受舜禪,即天子之位。洛出《龜書》六十五字,是為《洪範》,此謂「洛出《書》」



 者也。南巡狩,濟江,中流有二黃龍負舟,舟人皆懼。禹笑曰:「吾受命於天,屈力以養人。生,性也;死,命也。
 奚憂龍哉!」龍於是曳尾而逝。



 高辛氏之世妃曰簡狄,以春分玄鳥至之日,從帝祀郊禖,與其妹浴於玄丘之水。



 有玄鳥銜卵而墜之,五色甚好,二人競取,覆以玉筐。簡狄先得而吞之,遂孕。胸剖而生契。長為堯司徒,成功於民,受封於商。後十三世,生主癸。主癸之妃曰扶都,見白氣貫月,意感,以乙日生湯,號天乙。豐下銳上,晰而有髯,句身而揚聲,身長九尺,臂有四肘,是曰殷湯。湯在亳,能修其德。伊摯將應湯命,夢乘
 船過日月之傍。湯乃東至於洛,觀帝堯之壇,沈璧退立,黃魚雙踴,黑鳥隨魚止於壇,化為黑玉。又有黑龜,並赤文成字,言夏桀無道,湯當代之。檮杌之神,見於邳山。



 有神牽白狼銜鉤而入商朝。金德將盛,銀自山溢。湯將奉天命放桀,夢及天而舓之,遂有天下。商人後改天下之號曰殷。



 高辛氏之世妃曰姜嫄,助祭郊禖,見大人跡履之。當時歆如有人道感己,遂有身而生男。以為不詳,棄之厄巷,
 羊牛避而不踐;又送之山林之中,會伐林者薦覆之;又取而置寒冰上,大鳥來以一翼藉覆之。姜嫄以為異,乃收養焉,名之曰棄。



 枝頤有異相,長為堯稷官,有功於民。后稷之孫曰公劉,有德,諸侯皆以天子之禮待之。初黃帝之世,讖言曰:「西北為王,期在甲子,昌制命,發行誅,旦行道。」



 及公劉之後,十三世而生季歷。季歷之十年,飛龍盈於殷之牧野,此蓋聖人在下位將起之符也。季歷之妃曰太任,夢長人感己,溲於豕牢而生昌,是為周文王。
 龍顏虎肩,身長十尺,胸有四乳。太王曰:「吾世當有興者,其在昌乎!」季歷之兄曰太伯,知天命在昌,適越終身不反。弟仲雍從之,故季歷為嗣以及昌。昌為西伯,作邑於豐。文王之妃曰太姒,夢商庭生棘,太子發植梓樹於闕間,化為松柏棫柞。



 以告文王,文王幣告群臣,與發並拜告夢。季秋之甲子,赤爵銜書及豐,止於昌戶,昌拜稽首受之。其文要曰:「姬昌,蒼帝子,亡殷者紂王。」將畋,史遍卜之,曰:「將大獲,非熊非羆,天遺汝師以佐昌。臣太祖史疇
 為禹卜畋,得皋陶。其兆如此。」



 王至於磻谿之水,呂尚釣於涯,王下趨拜曰:「望公七年,乃今見光景於斯。」尚立變名答曰:「望釣得玉璜,其文要曰:『姬受命,昌來提,撰爾雒鈐報在齊。』」



 尚出游,見赤人自雒出,授尚書曰:「命曰呂,佐昌者子。」文王夢日月著其身,又摐甗鳴於岐山。孟春六旬,五緯聚房。後有鳳凰銜書,游文王之都。書又曰:「殷帝無道,虐亂天下。皇命已移,不得復久。靈祗遠離,百神吹去。五星聚房,昭理四海。」文王既沒,太子發代立,是為武
 王。



 武王駢齒望羊。將伐紂,至於孟津,八百諸侯,不期而會。咸曰:「紂可伐矣。」



 武王不從。及紂殺比干,囚箕子,微子去之,乃伐紂。度孟津,中流,白魚躍入王舟。王俯取魚,長三尺,目下有赤文成字,言紂可伐。王寫以世字,魚文消。燔魚以告天,有火自天止於王屋,流為赤烏,烏銜穀焉。穀者,紀后稷之德;火者,燔魚以告天,天火流下,應以吉也。遂東伐紂,勝於牧野,兵不血刃,而天下歸之。



 乃封呂望於齊。周德既隆,草木茂盛,蒿堪為宮室,因名蒿宮。武
 王沒,成王少,周公旦攝政七年,制體作樂,神鳥鳳凰見,蓂莢生。乃與成王觀於河、洛,沈璧。



 禮畢,王退俟,至於日昧,榮光並出幕河,青雲浮至,青龍臨壇,銜玄甲之圖,吐之而去。禮於洛,亦如之。玄龜青龍蒼兕止於壇,背甲刻書,赤文成字。周公援筆以世文寫之,書成文消,龜墮甲而去。其言自周公訖於秦、漢盛衰之符。麒麟遊苑,鳳凰翔庭,成王援琴而歌曰:「鳳凰翔兮於紫庭,餘何德兮以感靈,賴先王兮恩澤臻,于胥樂兮民以寧。」



 魯哀公十四年,孔子夜夢三槐之間,豐、沛之邦,有赤煙氣起,乃呼顏淵、子夏往視之。驅車到楚西北范氏街,見芻兒摘麟,傷其左前足,薪而覆之。孔子曰:「兒來,汝姓為赤誦,名子喬,字受紀。」孔子曰:「汝豈有所見邪?」兒曰:「見一禽,巨如羔羊,頭上有角,其未有肉。」孔子曰:「天下已有主也,為赤劉,陳、項為輔,五星入井從歲星。」兒發薪下麟示孔子,孔子趨而往,麟蒙其耳,吐三卷《圖》,廣三寸,長八寸,每卷二十四字,其言赤劉當起,曰:「周亡,赤氣起,大耀興,
 玄丘制命,帝卯金。」孔子作《春秋》,制《孝經》;既成,使七十二弟子向北辰星罄折而立,使曾子抱《河》、《洛》事北向。孔子齋戒向北辰而拜,告備於天曰:「《孝經》四卷,《春秋》、《河》、《洛》凡八十一卷,謹已備。」



 天乃洪鬱起白霧摩地,赤虹自上下,化為黃玉,長三尺,上有刻文。孔子跪受而讀之曰:「寶文出,劉季握。卯金刀,在軫北。字禾子,天下服。」



 漢高帝父曰劉執嘉。執嘉之母,夢赤鳥若龍戲己,而生執嘉,是為太上皇帝。



 母名含始,是為昭靈后。昭靈后游
 於洛池,有玉雞銜赤珠,刻曰玉英,吞此者王。



 昭靈后取而吞之;又寢於大澤,夢與神遇。是時雷電晦冥,太上皇視之,見蛟龍在其上,遂有身而生季,是為高帝。高帝隆準而龍顏,美須髯,左股有七十二黑子。



 微時,數從王媼、武負貰酒,醉臥,上常有光怪。每留飲,售輒數倍。武負異之,輒折其契。單父人呂公好相人,見高帝,謂曰:「臣少好相人,相人多矣,無如季相,願季自愛。臣有息女,願為箕掃妾。」呂公妻媼怒呂公曰:「公常奇此女,欲為貴人。沛令
 善公,求不與。何妄許劉季?」呂公曰:「非女子所知。」卒與高帝。



 生惠帝、魯元公主。呂后嘗與兩子居田中,有一老父過,請飲,呂后因饋之食。老父相呂后曰:「夫人,天下貴人也。」令相二子,見惠帝曰:「夫人所以貴者,乃此男。」相魯元公主,亦貴。老父已去,高帝適從傍舍來,呂后具言之。高帝追問老父。老父曰:「向者夫人、兒子之貴,皆以君相。君貴不可言。」高帝被飲,夜行徑澤中。前人反曰:「有大蛇當道,願還。」高帝醉,曰:「壯士行,何畏?」



 乃前,拔劍斬蛇,蛇分為
 兩,道開而過。後人來者,見老嫗守蛇曰:「向者赤帝子過,殺之。」見者疑嫗為詐,欲笞之,忽然不見。具以狀告高帝,帝心喜。秦始皇帝曰:「東南有天子氣。」於是東遊以厭之。高帝隱於芒、碭山澤之間,呂后常知其處。高帝怪問之,對曰:「季所居,上常有雲氣,故知之。」高帝為沛公,入秦,五星聚於東井,歲星先至,而四星從之。占曰:「以義取天下。」



 初,張良遊於下邳沂水之上,有老父來,直至良前,而墮其履。顧謂良曰:「孺子,下取履!」良愕然,欲毆之,以其老,乃
 下取跪進。父以足受,笑而去,良殊大驚。父去里所復來,曰:「孺子可教也。後五日平明,與我會此。」良怪之,跪應曰:「諾。」五日,良往,父已先來,怒曰:「何與長者期而後也?五日,更與我會此。」凡三期而良先至。老父喜曰:「不當如是邪!」即出懷中一卷書與之,曰:「讀之,此為王者師。後十三年,孺子見我濟北穀城山下,黃石即我也。」旦視其書,乃《太公兵法》。良以黃石篇為他人說,皆不省,唯高帝說焉。良曰:「此殆天所授矣。」五年而成帝業。後十三年,張良果得
 穀城山下黃石,寶而祠之,死與合葬。



 文帝之母薄姬,魏豹為魏王,納之後宮。許負相之,當生天子,魏王豹於是背漢。漢高帝擊虜,而薄姬輸織室。高帝見而美之,內於後宮,歲餘乃得幸。將見幸,薄姬言:「妾昨夢青龍據妾心。」高帝曰:「我是也。吾為爾成之。」一御而生文帝。



 景帝王皇后初嫁為金王孫妻,母臧兒卜筮曰:「當貴。」乃奪金氏而內太子宮,生男。男方在身,夢日入其懷,以告
 太子。太子曰:「是貴徵也。」生男,是為武帝。



 武帝趙婕妤,家在河間,生而兩手皆拳,不可開。武帝巡狩過河間,望氣者言,此有奇女天子氣,召而見之。武帝自披其手,既時申,得一玉鉤。由是見幸,號曰:「拳夫人。」進為婕妤,居鉤弋宮,大有寵。十四月生男,是為昭帝,號曰:「鉤弋子。」武帝曰:「聞昔堯十四月而生,今鉤弋子亦然。」乃名其門曰堯母門。



 昭帝元鳳三年正月,泰山、萊蕪山南,民夜聞洶洶有數
 千人聲,晨往視之,見大石自立,高丈五尺,大四十八圍,入地八尺,三石為足,立後,白烏數千集其旁。



 又上林苑中柳樹斷臥地,一朝自起生枝葉,蟲齧其葉成文,曰:「公孫病已立。」



 陳留襄邑王社忽移至長安。博士眭孟占之曰:「石,陰類。泰山,岱宗,王者禪代之處。將有廢故之家,姓公孫,名病已,從白衣為天子者。」時昭帝幼少,霍光輔政,以孟妖言誅之。及昭帝崩,昌邑王又廢,光立宣帝,武帝曾孫,本名病己,在民間白衣三世,如孟言焉。



 元帝王皇后,齊田氏之苗裔。祖父翁孺,自東平陵徙元城。元城建公曰:「昔《春秋》沙鹿崩,晉史卜之,陰為陽雄,土火相乘,故沙鹿崩。後六百四十五年,宜有聖女興,其齊田乎?今翁孺之徙,正值其地,日月當之。元城郭東有五鹿之墟,即沙鹿地。後八十年,當有貴女興天下。」翁孺生禁。禁妻李氏方任身,夢月入其懷,生女,是為元后。每許嫁,未行,所許者輒死。卜相者云:「當大貴。」遂為元帝皇后,生
 成帝。



 初,秦始皇世,有長人十二,身長五丈,足跡六尺,見於隴西臨洮,前史以為秦亡之徵,史臣以為漢興之符也。自高帝至於平帝,十二主焉。



 光武皇帝,父為濟陽令。濟陽有武帝行宮,常封閉。哀帝建平元年十二月甲子夜,光武將產,乃開而居之。時有赤光,室中盡明,皇考異焉。使卜者王長卜之。



 長辟左右曰:「此善事,不可言。」是歲,有嘉禾生產屋景天中,一莖九穗,異於凡禾,縣界大豐,故名光武曰秀。時又有鳳凰集
 濟陽,於是畫宮為鳳凰之象。明年,方士有夏賀良者,上言哀帝云:「漢家歷運中衰,當再受命。」於是改號為太初元將元年,稱陳聖劉太平皇帝以厭勝之。王莽時,善望氣者蘇伯阿望光武所居縣舂陵城郭,唶曰:「氣佳哉!鬱鬱蔥蔥然。」莽忌惡漢,而錢文有金,乃改鑄貨泉以易之。既而光武起於舂陵之白水鄉,貨泉之文為「白水真人」也。



 初起兵,望見家南有火光,以為人持火,呼之而光遂盛,赫然上屬天,有頃不見。及在河北,為王郎所逼,將南
 濟滹沱河。導吏還云:「河水流澌,無船可渡。」



 左右皆恐懼。帝更遣王霸視之。霸往視,如吏言。霸慮還以實對,驚動眾心,乃謬云:「冰堅可渡。」帝馳進。比至,而河水皆合,其堅可乘。既渡,餘數乘車未畢而冰陷。前至下博城西,疑所之。有一白衣老公在道旁,曰:「努力!信都為長安城守,去此八十里耳。」言畢,失所在。遂至信都,投太守任光。初,光武微時,穰人蔡少公曰:「讖言劉秀發兵捕不道,卯金修德為天子。」國師公劉子駿名秀。



 少公曰:「國師公是也。」光
 武笑曰:「何用知非僕?」道士西門君惠等並云:「劉秀當為天子。」光武平定河北,還至中山,將軍萬修得《赤伏符》,言光武當受命。群臣上尊號,光武辭。前至鄗縣,諸生彊華又自長安詣鄗,上《赤伏符》,文與修合。群下又請曰:「受命之符,人應為大。」光武又夢乘赤龍登天,乃即位,都洛陽,營宮闕。一夕,有門材自至。



 是時琅邪開陽縣城門,一夕無故自亡,檢所得材,即是也,遂名其門曰開陽門。



 先是秦穆公時,陳倉人掘地得物,若羊非羊,若豬非豬,怪,將
 獻之。道逢二僮子,謂之曰:「子知彼乎,名為襜,常在地下食死人腦。若欲殺之,以柏東南枝指之,則死矣。」襜因言曰:「此二僮子,名為寶。得其雄者王,得其雌者霸。」於是陳倉人遂棄襜而逐二僮子,二僮子化為雉,飛入林。陳倉人以告穆公,穆公發徒大獵,得其雌者,化而為石,置之汧、渭之間。至文公,為之立祠,名曰陳寶祠。雄南飛集南陽穰縣,其後光武興於南陽。



 光武之初興也,隗囂擁眾隴右,招集英俊,而公孫述稱帝於蜀,天下雲擾,大者連
 州郡,小者據縣邑。囂問扶風人班彪曰:「往者周亡,戰國並爭,天下分裂,數世然後定。縱橫之事,復起於今乎?將承運迭興,在於一人也?願先生論之。」



 對曰:「周之廢興與漢異。昔周立爵五等,諸侯從政,本根既微,枝葉彊大,故其末流有縱橫之事,其勢然也。漢家承秦之制,郡縣治民,主有專己之威,臣無百年之柄。至於成帝,假借外家,哀、平短祚,國嗣三絕,禍自上起,傷不及下。故王氏之貴,傾擅朝廷,能竊號位,而不根於民,是以即真之後,天下
 莫不引領而歎。



 十餘年間,中外騷擾,遠近俱發,假號雲合,咸稱劉氏,不謀而同辭。方今雄桀帶州域者,皆無七國世業之資。《詩》云:『皇矣上帝,臨下有赫。鑒觀四方,求民之瘼。』今民皆謳吟思漢,向仰劉氏,已可知矣。」隗囂曰:「先生言周、漢之勢,可也。至於但見愚民習識劉氏姓號之故,而謂漢復興,疏矣。昔秦失其鹿,劉季逐而掎之,時民復知漢乎?」彪既感囂言,又愍狂狡之不息,乃著《王命論》以救時難。辭曰:昔在帝堯之禪曰:「咨爾舜,天之歷數在
 爾躬。」舜亦以命禹。洎於稷、契,咸佐唐、虞,光濟四海,奕世載德,至於湯、武,而有天下。雖其遭遇異時,禪代不同,至於應天從民,其揆一焉。是故劉氏承堯之祚,氏族之世,著於《春秋》。



 唐據火德,而漢紹之。始起沛澤,則神母夜號,以章赤帝之符。由是言之,帝王之祚,必有明聖顯懿之德,豐功厚利積累之業,然後精誠通於神明,流澤加於生民。



 故能為鬼神所福嚮,天下所歸往。未見運世無本,功德不紀,而得掘起在此位者也。



 世俗見高祖興於布
 衣,不達其故,以為適遭暴亂,得奮其劍。游說之士,至比天下於逐鹿,幸捷而得之。不知神器有命,不可以智力求也。悲夫!此世之所以多亂臣賊子者也。若然者,豈徒暗於天道哉,又不觀之於人事矣。



 夫餓饉流隸,饑寒道路,思有裋褐之褻,擔石之畜,所願不過一金,然終於轉死溝壑。何則?貧窮亦有命也。況乎天子之貴,四海之富,神明之祚,可得而妄據哉!故雖遭罹厄會,竊其權柄,勇如信、布,彊如梁、籍,成如王莽,然卒潤鑊伏鑕,烹菹分裂;
 又況麼麼不及數子,而欲暗干天位者乎?是故駑蹇之乘,不騁千里之塗;燕雀之儔,不奮六翮之用;楶棁之材,不荷棟梁之任;斗筲之子,不秉帝王之重。《易》曰:「鼎折足,覆公餗。」不勝其任也。當秦之末,豪桀共推陳嬰而王之。嬰母止嬰曰:「自吾為子家婦,而世貧賤,卒富貴,不祥。不如以兵屬人,事成,少受其利;不成,禍有所歸。」嬰從其言,而陳氏以寧。王陵之母,亦見項氏之必亡,而劉氏之將興也。是時陵為漢將,而母獲於楚。有漢使來,陵母見之,
 謂曰:「願告吾子,漢王長者,必得天下,子謹事之,無有二心。」遂對漢使,伏劍而死,以固勉陵。其後果定於漢,陵為宰相封侯。夫以匹婦之明,猶能推事理之致,探禍福之機,全宗祀於無窮,垂冊書於《春秋》,而況大丈夫之事乎!是故窮達有命,吉凶由人,嬰母知廢,陵母知興,審此二者,帝王之分決矣。!



 蓋在高祖,其興也有五:一曰帝堯之苗裔,二曰體貌多奇異,三曰神武有徵應,四曰寬明而仁恕,五曰知人善任使。加之以信誠好謀,達於聽受,見
 善如不及,用人如由己,從諫如從流,趨時如響赴;當食吐哺,納子房之策;拔足揮洗,揖酈生之說;寤戍卒之言,斷懷土之情;高四皓之名,割肌膚之愛;舉韓信於行陣,收陳平於亡命;英雄陳力,群才畢舉,此高祖之大略所以成帝業也。若乃靈瑞符應,又可略聞矣。初,劉媼妊高祖而夢與神遇,震雷晦冥,有龍蛇之怪。及長多靈異,有殊於眾,是以王、武感物而折契,呂公睹貌而進女;秦皇東遊以厭其氣,呂后望雲而知所處;始受命則白蛇分,
 西入關則五星聚。故淮陰、留侯謂之天授,非人力也。



 歷古今之得失,驗行事之成敗,稽帝王之世運,考五者之所謂,取舍不厭斯位,符應不同斯度,而欲昧於權利,越次妄據,外不量力,內不知命,則必喪保家之主,失天年之壽,遇折足之凶,伏鈇鉞之誅。英雄誠知覺寤,畏若禍戒,超然遠覽,淵然深識,收陵、嬰之明分,絕信、布之覬覦,距逐鹿之瞽說,審神器之有授,無貪不可幾,為二母之所笑,則福祚流於子孫,天祿其永終矣。



 隗囂不納,果敗。
 漢元、成世,道士言:「讖者云:『赤厄三七。』三七,二百一十年,有外戚之篡。祚極三六,當有龍飛之秀,興復祖宗。」及莽篡漢,漢二百一十年矣。莽十八年而敗,光武興焉。



 明帝初生,豐下兌上,赤色似堯,終登帝位。



 和帝鄧皇后,祖父禹,佐命光武,常曰:「我將百萬人,未嘗妄殺一人,子孫當大興。」后少時,相者蘇文見后,驚曰:「此成湯之骨法也,貴不可言。」後嘗夢登梯,以手捫天,天體蕩蕩正青而滑,有若鐘乳者,后仰吮之。以訊之占夢。占
 夢者曰:「堯夢攀天而上,湯夢及天而呧之,此皆非常夢也。」既而入宮,遂登尊位。安帝未即大位,在邸,數有神光赤蛇嘉應,照曜室內,磐紆殿屋床第之間,後遂入承大統。



 初,桓帝之世,有黃星見於楚、宋之分。遼東殷馗曰:「後五十年,當有真人起於譙、沛之間,其鋒不可當。」靈帝熹平五年,黃龍見譙。光祿大夫喬玄問太史令單颺曰:「此何祥也?」颺曰:「其國後當有王者興,不及五十年,亦當復見天事恆象,此其徵也。」內黃殷登默記之。其後曹操起於
 譙,是為魏武帝。建安五年,於黃星見之,歲五十年矣,而武帝破袁紹,天下莫敵。



 《春秋讖》曰:「代漢者,當塗高也。」漢有周舒者,善內學。人或問之,舒曰:「當塗高者,魏也。」舒既沒,譙周又問術士杜瓊曰:「周徵君以為當塗高,魏也。其義何在?」瓊曰:「魏,闕名也。當塗而高,聖人以類言耳。」又問周曰:「寧復有所怪邪?」周曰:「未達也。」瓊曰:「古者名官職不言曹,自漢以來,名官盡言曹,吏言屬曹,卒言侍曹,此殆天意也。」周曰:「魏者,大也;
 曹者,眾也。眾而且大,天下之所歸乎?」建安十八年,武帝為公,又進爵為王。二十五年,武帝薨,太子丕嗣為魏王,是為文帝。



 文帝始生,有雲青色,員如車蓋,當其上終日。望氣者以為至貴之祥,非人臣之氣。善相者高元呂曰:「其貴不可言。」延康元年三月,黃龍又見譙,殷登猶存,歎曰:「黃龍見於熹平也,單颺云:『不及五十年,亦當復見。』今四十五年矣,颺之言其驗茲乎。」四月,饒安言白虎見。八月,石邑言鳳凰集,又有麒麟見。十月,漢帝禪位於魏,魏
 王辭讓不受。博士蘇林、董巴上言:「臣聞天之去就,固有常分,聖人當之,昭然不疑。故堯捐骨肉而禪有虞,終無吝色。舜發壟畝而居天下,若固有之。其相授間,不稽漏刻,天下已傳矣。所以急天命,明天下不可一日無君。



 今漢期運已終,妖異絕之已審。陛下受天之命,符瑞告徵,丁寧詳悉,反覆備至,雖言語相諭,無以代此。今既發詔書,璽綬未御,固執謙讓,上稽天命,下違民情。



 臣謹按古之典籍,參以圖緯,魏之行運及天道所在,即尊之驗,在
 於今年此月,昭晢分明。謹條奏如左。唯陛下遷思易慮,以時即位,顯告上帝,布詔天下。然後改正朔,易服色,正大號,天下幸甚。」其所陳事曰:天有十二次,以為分野,王公之國,各有所屬。周在鶉火,魏在大梁,歲星行歷,凡十二次,所在國天子受命,諸侯以封。周文王始受命,歲星在鶉火。至武王伐紂,十三年,歲星復在鶉火。故《春秋傳》曰:「武王伐紂,歲在鶉火。」又曰:「歲之所在,則我有周之分野也。」昔光和七年,歲在大梁,武王始受命為將,討黃
 巾。是歲,改年為中平元年。建安元年,歲復在大梁,始拜大將軍。十三年,復在大梁,始拜丞相。今二十五年,歲復在大梁,陛下受命。此魏得歲與周文、武受命相應。



 今年青龍在庚子,《詩推度災》曰:「庚者,更也;子者,茲也。聖人制法天下治。」又曰:「王者布德於子,治成於丑。」此言今年天更命聖人,制法天下,布德於民也。魏以政制天下,與《詩》協矣。顓頊受命,歲在豕韋,衛居其地,亦在豕韋。故《春秋傳》曰:「衛,顓頊之墟也。」今十月,斗之所建,則顓頊受命之
 分也。魏以十月受禪,此同符始祖受命之驗也。魏之氏族,出自顓頊,與舜同祖,見於《春秋世家》。舜以土德承堯之火,今魏亦以土德承漢之火,其於行運合於堯、舜授受之次。



 魏王猶未許。大史丞許芝又上天文祥瑞:自建安三年十二月戊辰,有新天子氣見於東南,到今積二十三年。建安十年,茀星出庫樓,歷犯氐、房宿,北入天市,犯北斗、紫微。氐為天子宿宮,路寢所止。



 房為天子明堂政教之首。北斗七星,主尊輔象近臣。紫微者,北極最尊。
 此除掃漢家之大異也。建安十八年秋,歲星、鎮星、熒惑俱入太微,逆行留守帝坐百有餘日。



 歲星入太微,人主改姓。鎮星入太微,內有兵亂,人主以弱。三者,漢改姓易代之異也。建安十九年正月,白虹貫日。《易傳》曰:「后妃擅國,白虹貫日。」建安二十一年五月朔己亥,日蝕。建安二十三年三月,茀星晨見東方二十餘日,夕出西方,犯歷五車、東井、五諸侯、文昌、軒轅、太微,鋒炎刺帝坐。茀者除舊布新,亡惡興聖之異也。建安二十四年二月晦壬子,
 日蝕。日者陽精,月為侯王,而以亥子日蝕,皆水滅火之異也。延康元年九月十日黃昏時,月蝕熒惑,過人定時,熒惑出營室,宿羽林。月為大臣侯王之象;熒惑火精,漢氏之行。占曰:「漢家以兵亡。」



 延康元年九月二十日,《剝》卦天子氣不見,皆崩亡之異也。熒惑火精,行縮日一度有餘。故太史令王昱以為漢家衰亡之極。熒惑大而赤色;光不明,赤而小,與小星無別,皆漢家衰亡之異也。



 《易傳》曰:「上下流通聖賢昌,厥應帝德鳳凰翔,萬民喜樂無咎
 殃。」《易傳》又曰:「聖人受命,厥應鳳凰下,天子虜。」《易傳》又曰:「黃龍見,天災將至,天子絀,聖人出。」黃龍以戊己日見,五色文章皆具,聖人得天受命。黃龍以戊寅見,此帝王受命之符瑞最著明者也。《易傳》又曰:「聖人清靜行中正,賢人至,民從命,厥應麒麟來。」《春秋玉版讖》曰:「代赤者魏公子。」《春秋佐助期》曰:「漢以許昌失天下。」故白馬令甘陵李雲上事,言許昌氣見,當塗高已萌,欲使漢家防絕萌芽。今漢都許,日以微弱,當居許昌以失天下。當塗高者,魏
 也;魏者,象魏兩闕之名當道而高大者也。魏當代漢,如李雲之言也。《春秋佐助期》又曰:「漢以蒙孫亡。」說者以蒙孫直漢二十四帝,童蒙愚惑以弱亡。漢帝少時名為董侯,名不正,蒙亂荒惑,其子孫以弱亡也。《孝經中黃讖》曰:「日載東,紀火光。不橫一,聖明聰。四百之外,易姓而王。天下歸功致太平。」此魏王之姓諱著見圖讖也。《易運期》曰:「言居東,西有午,兩日並光日居下。其為主,反為輔,五八四十,黃氣受,真人出。」言午「許」字,兩日「昌」字,漢當以許亡,
 魏當以許昌。今際會之期在許,是其大效也。《易運期》又曰:「鬼在山,禾女運,王天下。」



 於是魏王受漢禪,柴於繁陽,有黃鳥銜丹書,集於尚書臺,於是改元為黃初。



 漢中平二年,洛陽民訛言虎賁寺有黃人,觀者日數萬,道路斷絕。中平元年,黃巾賊起,云:「蒼天已死,黃天當立。」此魏氏依劉向自云土德之符也。先是,周敬王之四十七年,宋景公問大夫邢史子臣:「天道何祥?」對曰:「後五年五月丁亥,臣將死。死後五年五月丁卯,吳將亡。亡後五年,君將
 終。終後四百年,邾王天下。」



 皆如其言。邾王天下,蓋謂魏國之後。言四百年則錯。疑年代久遠,傳記者謬誤。



 高貴鄉公初生,有光氣照耀室屋,其後即大位。



 劉備身長七尺七寸,垂手過膝,顧自見耳。《洛書甄耀度》曰:「赤三,德昌九世會備,合為帝際。」《洛書寶予命》曰:「天度帝道備稱皇,以統握契,百成不敗。」《洛書錄運期》曰:「九侯七傑爭民命,炊骸道路,誰使主者玄且來。」



 備字玄德,故云:「玄且來」也。《孝經鉤命決》曰:「帝三建,九會備。」先是,術士周
 群言,西南數有黃氣,直立數丈,如此積年,每有景雲祥風,從璇璣下應之。



 建安二十二年中,屢有氣如旗,從西竟東,中天而行。圖書曰「必有天子出其方。」



 太白、熒惑、鎮星從歲星,又黃龍見犍為武陽之赤水,九日乃去。關羽在襄陽,男子張嘉、王休獻玉璽,備後稱帝於蜀。



 孫堅之祖名鐘,家在吳郡富春,獨與母居。性至孝,遭歲荒,以種瓜為業。忽有三少年詣鐘乞瓜,鐘厚待之。三人謂鐘曰:「此山下善,可作塚,葬之,當出天子。君可下山百
 步許,顧見我去,即可葬也。」鐘去三十步,便反顧,見三人並乘白鶴飛去。鐘死,即葬其地。地在縣城東,塚上數有光怪,雲氣五色上屬天,衍數里。父老相謂,此非凡氣,孫氏其興矣。堅母任堅,夢腸出繞吳昌門。以告鄰母,鄰母曰:「安知非吉祥也。」昌門,吳郭門也。堅生而容貌奇異。堅妻吳氏初妊子策,夢月入其懷;後孕子權,又夢日入懷。告堅曰:「昔妊策,夢月入懷,今又夢日入懷,何也?」堅曰:「日月陰陽之精,極貴之象,吾子孫其興乎!」權方頤大口,紫
 髯,長上短下。漢世有劉琬者,能相人,見權兄弟,曰:「孫氏兄弟,雖各才智明達,然祿胙不終。唯中弟孝廉,形貌奇偉,骨體不恆,有大貴之表,年又最壽。爾其識之。」權時為孝廉。初,秦始皇東巡,濟江。望氣者云:「五百年後,江東有天子氣出於吳,而金陵之地,有王者之勢。」於是秦始皇乃改金陵曰秣陵,鑿北山以絕其勢。至吳,又令囚徒十餘萬人掘汙其地,表以惡名,故曰囚卷縣,今嘉興縣也。漢世術士言:「黃旗紫蓋,見於斗、牛之間,江東有天子氣。」
 獻帝興平中,吳中謠言:「黃金車,斑蘭耳。開昌門,出天子。」魏文帝黃初三年,舉兵武昌,並言黃龍、鳳凰見。其年,權稱尊號,年至七十一而薨。權子休,初封琅邪王,夢乘龍上天,顧不見尾。後得大位,其子被廢。



 漢元、成之世,先識之士有言曰:「魏年有和,當有開石於西三千餘里,繫五馬,文曰討曹。」及魏之初興也,張掖刪丹縣金山柳谷有石生焉,周圍七尋,中高一仞,蒼質素章,有五馬、麟、鹿、鳳凰、仙人之象。始見於建安,形成於黃初,
 文備於太和。至青龍三年,柳谷之玄川溢湧,石形改易,狀似云龜,廣六尺,長一丈七尺一寸,圍五丈八寸,立於川西。有石馬十二,其一仙人騎之,其一羈靽,其五有形而不善成,其五成形。又有一牛八卦列宿彗星之象。有玉匣開蓋於前,有玉玦二,玉璜一。又有麒麟、鳳凰、白虎、馬、牛於中布列。有文字曰:「上上三天王述大會討大曹金但取之金立中大金馬一疋中正大吉關壽此馬甲寅述水」凡三十五字。石色蒼,而物形及字,並白石書之,皆隆起。
 魏明帝惡其文有「討曹」,鑿去為「計」,以蒼石塞之,宿昔而白石滿焉。當時稱為祥瑞,班下天下。處士張臶曰:「夫神兆未然,不追往事,此蓋將來之休徵,當今之怪異也。」既而晉以司馬氏受禪。太尉屬程猗說曰:「夫大者,盛之極也。金者,晉之行也。中者,物之會也。



 吉者,福之始也。此言司馬氏之王天下,感德而生,應正吉而王之符也。」猗又為贊曰:「皇德遐通,實降嘉靈。乾生其象,坤育其形。玄石既表,素文以成。瑞虎合仁,白麟耀精。神馬自圖,金言其
 形。體正而王,中允克明。關壽無疆,於萬斯齡。」



 宣帝有狼顧之相,能使面正向後,而身形不異。魏武帝嘗夢有三匹馬在一槽中共食,其後宣帝及景、文相繼為宰相,遂傾曹氏。文帝未立世子,有意於齊獻王攸。



 武帝時為中撫軍,懼不立,以相貌示裴秀,秀言於文帝曰:「中撫軍振髮籍地,垂手過膝,天表如此,非人臣之相也。」由是得立。及嗣晉位,其月,襄武縣言有大人相,長三丈餘,足跡三尺一寸,白髮,黃單衣,黃巾,柱杖呼民王始語云:「今當太
 平。」頃之,受魏禪。



 武帝咸寧元年,大風吹帝社樹折,有青氣出社中。占者以為東莞有天子氣。時琅邪武王伷封東莞,伷,元帝祖也。元帝以咸寧二年夜生,有光照室,室內盡明,有白毛生於日角之左,眼有精光耀。隨惠帝幸鄴。成都王穎殺東安王繇,繇,元帝叔父也。帝懼,欲出奔,而月明,邀候急,四衢斷絕,不得去。有頃,天陰,風雨大至,候者皆休,乃得去。



 初,武帝伐吳,琅邪武王伷率眾出塗中,而王渾逼歷
 陽,王濬已次近路。孫皓欲降,送天子璽綬,近越二將,而遠送詣伷,識者咸怪之。吳之未亡也,吳郡臨平湖一旦自開,湖邊得石函,中有小青石,刻作皇帝字。舊言臨平湖塞天下亂,開則天下太平。吳人以為美祥。俄而吳滅。後元帝興於江左。吳亡後,蔣山上常有紫雲,數術者亦云,江東猶有帝王氣。又謠言曰:「五馬游度江,一馬化為龍。」元帝與西陽、汝南、南頓、彭城五王過江,而元帝升天位。讖書曰:「銅馬入海建業期。」



 元帝小字銅環。



 永嘉初,元
 帝以安東將軍鎮建業。時歲、鎮星、辰、太白四星聚於牛、女之間,常裴回進退。愍帝建興四年,晉陵武進人陳龍在田中得銅鐸五枚,柄口皆有龍虎形;又有將雛雞雀集其前,皆驅去復還,至於再三;又有鵝三四頭,高飛且鳴,周回東西,晝夜不下,如此者六七日。會稽剡縣陳清又於井中得棧鐘,長七寸二分,口徑四寸,其器雖小,形制甚精,上有古文書十八字,其四字可識,云:「會稽徽命。」



 豫章有大樟樹,大三十五圍,枯死積久,永嘉中,忽更榮
 茂。景純並言是元帝中興之應。初,武帝太康三年,建鄴有寇,餘姚人伍振筮之,曰:「寇已滅矣。三十八年,揚州有天子。」至元帝即天位,果三十八年。



 先是,宣帝有寵將牛金,屢有功,宣帝作兩口榼,一口盛毒酒,一口盛善酒,自飲善酒,毒酒與金,金飲之即斃。景帝曰:「金名將,可大用,云何害之?」宣帝曰:「汝忘石瑞,馬後有牛乎?」元帝母夏侯妃與琅邪國小史姓牛私通,而生元帝。愍帝之立也,改毗陵為晉陵,時元帝始霸江、揚,而戎翟稱制,西都微弱。
 干寶以為晉將滅於西而興於東之符也。



 宋武帝居在丹徒,始生之夜,有神光照室;其夕,甘露降於墓樹。皇考以高祖生有奇異,名為奇奴。皇妣既殂,養於舅氏,改為寄奴焉。少時誕節嗜酒,自京都還,息於逆旅。逆旅嫗曰:「室內有酒,自入取之。」帝入室,飲於盎側,醉臥地。



 時司徒王謐有門生居在丹徒,還家,亦至此逆旅。逆旅嫗曰:「劉郎在室內,可入共飲酒。」此門生入室,驚出謂嫗曰:「室內那得此異物?」嫗遽入之,見帝已覺矣。嫗密
 問:「向何所見?」門生曰:「見有一物,五采如蛟龍,非劉郎。」門生還以白謐,謐戒使勿言,而與結厚。帝嘗行至下邳,遇一沙門,沙門曰:「江表尋當喪亂,拯之必君也。」帝患手創積年,沙門出懷中黃散一裹與帝,曰:「此創難治,非此藥不能瘳也。」倏忽不見沙門所在。以散傅創即愈。餘散帝寶錄之,後征伐屢被傷,通中者數矣,以散傅之,無不立愈。自少至長,目中常見二龍在前,始尚小,及貴轉大。晉陵人車藪善相人,相帝曰:「君貴不可言,願無相忘。」晉安
 帝義熙初,帝始康晉亂,而興霸業焉。



 廬江霍山常有鐘聲十二。帝將征關、洛,霍山崩,有六鐘出,制度精奇,上有古文書一百六十字。冀州有沙門法稱將死,語其弟子普嚴曰:「嵩皇神告我云,江東有劉將軍,是漢家苗裔,當受天命。吾以三十二璧,鎮金一餅,與將軍為信。三十二璧者,劉氏卜世之數也。」普嚴以告同學法義。法義以十三年七月,於嵩高廟石壇下得玉璧三十二枚,黃金一餅。漢中城固縣水際,忽有雷聲,俄而岸崩,得銅鐘十二
 枚。又鞏縣民宋耀得嘉禾九穗。後二年而受晉禪。孔子《河雒讖》曰:「二口建戈不能方,兩金相刻發神鋒,空穴無主奇入中,女子獨立又為雙。」二口建戈,「劉」字也。晉氏金行,劉姓又有金,故曰兩金相刻。空穴無主奇入中,為「寄」



 字。女子獨立又為雙,「奴」字。



 晉既禪宋,太史令駱達奏陳天文符讖曰:「去義熙元年,至元熙元年十月,太白星晝見經天凡七。占曰:『天下革民更王,異姓興。』義熙元年至元熙元年十一月朔,日有蝕之凡四,皆蝕從上始,臣民
 失君之象也。義熙十一年五月三日,彗星出天市,其芒掃帝坐。天市在房、心之北,宋之分野。得彗柄者興,此除舊布新之徵。義熙七年七月二十五日,五虹見於東方。占曰:『五虹見,天子黜,聖人出。』義熙七年八月十一日,新天子氣見東南。十二年,北定中原,崇進宋公。歲星裴回房、心之間,大火,宋之分野。與武王克殷同,得歲星之分者應王也。十一年以來至元熙元年,月行失道,恆北入太微中。占:『月入太微廷,王入為主。』十三年十月,鎮星入
 太微,積留七十餘日,到十四年八月十日,又入太微不去,到元熙元年,積二百餘日。占:『鎮星守太微,亡君之戎。有立王,有徙王。』十四年五月十七日,茀星出北斗魁中。占曰:『星茀北斗中,聖人受命。』十四年七月二十九日,彗星出太微中,彗柄起上相星下,芒尾漸長至十餘丈,進掃北斗及紫微中。占曰:『彗星出太微,社稷亡,天下易政。入北斗,帝宮空。』一占:『天下得召人。』召人,聖主也。一曰:『彗孛紫微,天下易主。』十四年十月一日,熒惑從入太微鉤
 己,至元年四月二十七日,從端門出積屍,留二百六日,繞鎮星。熒惑與填星鉤己天廷,天下更紀。十四年十二月,歲、太白、辰裴回居斗、牛之間經旬。斗、牛,歷數之起。占曰:『三星合,是謂改立。』元熙元年十二月二十四日,四黑龍登天。



 《易傳》曰:『冬龍見,天子亡社稷,大人應天命之符。』《金雌詩》云:『大火有心水抱之,悠悠百年是其時。』火,宋之分野。水,宋之德也。《金雌詩》又曰:『云出而兩漸欲舉,短如之何乃相岨,交哉亂也當何所,唯有隱巖殖禾黍,西南
 之朋困桓父。』兩云「玄」字也。短者,云胙短也。巖隱不見,唯應見谷,殖禾谷邊,則聖諱炳明也。《易》曰:『西南得朋。』故能困桓父也。劉向讖曰:『上五盡寄致太平,草付合成集群英。』前句則陛下小諱,後句則太子諱也。十一年五月,西明門地陷,水涌出,毀門扉閾。西者,金鄉之門,為水所毀,此金德將衰,水德方興之象也。太興中,民於井中得棧鐘,上有古文十八字,晉自宣帝至今,數滿十八傳。義熙八年,太社生桑,尤著明者也。夫六,亢位也。漢建安二十
 五年,一百九十六年而禪魏。魏自黃初至咸熙二年,四十六年而禪晉。晉自泰始至今元熙二年,一百五十六年。三代數窮,咸以六年。」



 少帝即位,景平三年四月,有五色雲見西方。時文帝為荊州刺史,鎮江陵,尋即大位。文帝元嘉中,謠言錢唐當出天子,乃於錢唐置戍軍以防之。其後,孝武帝即大位於新亭寺之禪堂。「禪」之與「錢」,音相近也。太宗為徐州刺史,出鎮彭城,昭太后賜以大珠鹿盧劍,此劍是御服,占者以為嘉祥。前廢帝永光初,又
 訛言湘州出天子,幼主欲南幸湘川以厭之。既而湘東王即尊位,是為明帝。



 史臣謹按,冀州道人法稱所云玉璧三十二枚,宋氏卜世之數者,蓋卜年之數也。



 謂卜世者,謬其言耳。三十二者,二三十,則六十矣。宋氏受命至於禪齊,凡六十年云。



\end{pinyinscope}