\article{卷十五志第五 禮二}

\begin{pinyinscope}

 古者天子巡狩之禮,布在方策。至秦、漢巡幸,或以厭望氣之祥,或以希神仙之應,煩擾之役,多非舊典。唯後漢諸帝,頗有古禮焉。魏文帝值參分初創,方隅事多,皇輿
 亟動,略無寧歲。蓋應時之務,又非舊章也。明帝凡三東巡,所過存問高年,恤人疾苦,或賜穀帛,有古巡幸之風焉。齊王正始元年,巡洛陽,賜高年、力田各有差。



 晉武帝泰始四年,詔刺史二千石長吏曰:「古之王者,以歲時巡狩方嶽,其次則二伯述職,不然則行人巡省,撣人誦志。故雖幽遐側微,心無壅隔。人情上通,上指遠喻。至于鰥寡,罔不得所。用垂風遺烈,休聲猶存。朕在位累載,如臨深泉,夙興夕惕,明發不寐,坐而待旦。思四方水旱災眚,
 為之怛然。勤躬約己,欲令事事當宜。常恐眾吏用情,誠心未著,萬機兼猥,慮有不周;政刑失謬,而弗獲備覽。



 百姓有過,在予一人。惟歲之不易,未遑卜征巡省之事。人之未乂,其何以恤之。



 今使使持節侍中、副給事黃門侍郎,銜命四出,周行天下,親見刺史二千石長吏,申喻朕心懇誠至意,訪求得失損益諸宜,觀省政治,問人間患苦。周典有之曰:『其萬人利害為一書,其禮俗政事教治刑禁之逆順為一書,其悖逆暴亂作慝犯令為一書,其
 札喪凶荒厄貧為一書,其康樂和親安平為一書。每國辯異之,以反命於王,以周知天下之故。』斯舊章前訓,今率由之。還具條奏,俾朕昭然鑒于幽遠,若親行焉。大夫君子,其各悉乃心,各敬乃事,嘉謀令圖,苦言至戒,與使者盡之,無所隱諱。方將虛心以俟。其勉哉勖之,稱朕意焉。」摯虞新禮儀曰:「魏氏無巡狩故事,新禮則巡狩方岳,柴望告至,設壝宮,如禮諸侯之覲者。擯及執贄,皆如朝儀,而不建其旗。臣虞案覲禮,諸侯覲天子,各建其旗章,
 所以殊爵命,示等威。



 《詩》稱『君子至止,言觀其旂』。宜定新禮建旗如舊禮。」然終晉世,巡狩廢矣。



 宋武帝永初元年,詔遣大使分行四方,舉善旌賢,問其疾苦。元嘉四年二月己卯,太祖東巡。丁卯,至丹徒。己巳,告覲園陵。三月甲戌,幸丹徒離宮,升京城北顧。乙亥,饗父老舊勳于丹徒行宮,加賜衣裳各有差,蠲丹徒縣其年租布之半,繫囚見徒五歲刑以下,悉皆原遣。登城三戰及先大將軍并貴泥關頭敗沒餘口。老疾單孤,又諸
 戰亡家不能自存者,並隨宜隱恤。二十六年二月己亥,上東巡。辛丑,幸京城。辛亥,謁二陵。丁巳,會舊京故老萬餘人,往還饗勞,孤疾勤勞之家,咸蒙恤賚,發赦令,蠲徭役。其時皇太子監國,有司奏儀注。



 某曹關某事云云。被令,儀宜如是。請為箋如左。謹關。



 右署眾官如常儀。



 尚書僕射、尚書左右丞某甲,死罪死罪。某事云云。參議以為宜如是事諾。奉行。某年月日。某曹上。



 右箋儀準於啟事年月右方,關門下位及尚書官署。其言選事者,依舊不經它官。



 太常主者寺押。某署令某甲辭。言某事云云。求告報如所稱。詳檢相應。今聽如所上處事諾。明詳旨申勤,依承不得有虧。符到奉行。年月日。起尚書某曹。



 右符儀。



 某曹關太常甲乙啟辭。押。某署令某甲上言。某事云云。請臺告報如所稱。主者詳檢相應。請聽如所上事諾。別
 符申攝奉行。謹關。



 年月日。



 右關事儀準於黃案年月日右方,關門下位年月下左方,下附列尚書眾官署。其尚書名下應云奏者,今言關。餘皆如黃案式。



 某曹關司徒長史王甲啟辭。押。某州刺史丙丁解騰某郡縣令長李乙書言某事云云。請臺告報如所稱。尚書某甲參議,以為所論正如法令,報聽如所上。請為令
 書如左。謹關。



 右關門下位及尚書署,如上儀。



 司徒長史王甲啟辭。押。某州刺史丙丁解騰某郡縣令長李乙書言某事云云。州府緣案允。值。請臺告報。



 年月日。尚書令某甲上。



 建康宮無令,稱僕射。



 令日下司徒,令報聽如某所上。某宣攝奉行如故事。文書如千里驛行。



 年月朔日甲子。尚書令某甲下。無令稱僕射。司徒承書從事到上起某曹。



 右外上事,內處報,下令書儀。



 某曹關某事云云。令如是,請為令書如右。謹關。



 右關署如前式。



 令司徒。某事云云。令如是,其下所屬,奉行如故事。文書如千里驛行。



 年月日子,下起某曹。



 右令書自內出下外儀。



 令書前某官某甲。令以甲為某官,如故事。



 右令書板文準於昭事板文。



 年月日。侍御史某甲受。



 尚書下云云。奏行如故事。



 右以準尚書敕儀。起某曹。



 右並白紙書。凡內外應關箋之事,一準此為儀。其經宮臣者,依臣禮。



 拜刺史二千石誡敕文曰制詔云云。某動靜屢聞。



 右若拜詔書除者如舊文。其拜令書除者,「令」代「制詔」,餘如常儀。辭關板文云:「某官糞土臣某甲臨官。稽首再拜辭。」制曰右除糞土臣及稽首云云。



 某官某甲再拜辭。以「令日」代「制曰」。某官宮臣者,稱臣。



 皇太子夜開諸門,墨令,銀字啟傳令信。



 太史每歲上某年歷。先立春立夏大暑立秋立冬,常讀五時令。皇帝所服,各隨五時之色。帝升御坐,尚書令以
 下就席位,尚書三公郎以令著錄案上,奉以入,就席伏讀訖,賜酒一卮。官有其注。傅咸曰:「立秋一日,白路光於紫庭,白旗陳於玉階。」然則其日旂、路皆白也。



 晉成帝咸和五年六月丁未,有司奏讀秋令。兼侍中散騎侍郎荀弈、兼黃門侍郎散騎侍郎曹宇駮曰:「尚書三公曹奏讀秋令儀注。新荒以來,舊典未備。臣等參議,光祿大夫臣華恒議,武皇帝以秋夏盛暑,常闕不讀令,在春冬不廢也。夫先王所以從時讀令者,蓋後天而奉天
 時。正服,尊嚴之所重,今服章多闕如。比熱隆赫,臣等謂可如恆議,依故事闕而不讀。」詔可。六年三月,有司奏:「今月十六日立夏。



 案五年六月三十日門下駮,依武皇夏闕讀令。今正服漸備,四時讀令,是祗述天和隆赫之道。謂今故宜讀夏令。」奏可。



 宋文帝元嘉六年六月辛酉朔,駙馬都尉奉朝請徐道娛上表曰:「謹案晉博士曹弘之議,立秋御讀令,上應著緗幘,遂改用素,相承至今。臣淺學管見,竊有惟疑。



 伏尋《禮記·月令》,王者四時之服正見
 駕蒼龍,載赤旗,衣白衣,服黑玉。季夏則黃,文極於此,無白冠則某履某耑也。且幘又非古服,出自後代。上附於冠,下不屬衣。冠固不革,而幘豈容異色。愚謂應恒與冠同色,不宜隨節變採。土令在近,謹以上聞。如或可採,乞付外詳議。」太學博士荀萬秋議:「伏尋幘非古者冠冕之服,《禮》無其文。案蔡邕《獨斷》云:『幘是古卑賤供事不冠人所服。』又董仲舒《止雨書》曰:『其執事皆赤幘。』知並不冠之服也。漢元始用,眾臣率從。



 故司馬彪《輿服志》曰:『尚書幘
 名曰納言。迎氣五郊,各如其色,從章服也。』自茲相承,迄于有晉。大宋受命,禮制因循。斯既歷代成準,謂宜仍舊。」有司奏:「謹案道娛啟事,以土令在近,謂幘不宜變。萬秋雖云幘宜仍舊,而不明無讀土令之文。今書舊事于左。《魏臺雜訪》曰:『前後但見讀春夏秋冬四時令,至於服黃之時,獨闕不讀。今不解其故。』魏明帝景初元年十二月二十一日,散騎常侍領太史令高堂隆上言曰:『黃於五行,中央土也。王西季各十八
 日。土生於火,故於火用事之末服黃,三季則否。其令則隨四時,不以五行為分也。是以服黃無令。』」



 其後太祖常謂土令,三公郎每讀時令,皇帝臨軒,百僚備位,多震悚失常儀。宋唯世祖世劉勰、太宗世謝緯為三公郎,善於其事,人主及公卿並屬目稱歎。勰見《宗室傳》。緯,謝綜弟也。



 舊說後漢有郭虞者,有三女。以三月上辰產二女,上巳產一女。二日之中,而三女並亡,俗以為大忌。至此月此日,不敢止家,皆於東流水上為祈禳,自潔濯,謂之禊
 祠。分流行觴,遂成曲水。史臣案《周禮》,女巫掌歲時祓除釁浴,如今三月上巳如水上之類也。釁浴謂以香薰草藥沐浴也。《韓詩》曰:「鄭國之俗,三月上巳,之溱、洧兩水之上,招魂續魄。秉蘭草,拂不祥。」此則其來甚久,非起郭虞之遺風、今世之度水也。《月令》,暮春,天子始乘舟。蔡邕章句曰:「陽氣和暖,鮪魚時至,將取以薦寢廟,故因是乘舟禊於名川也。《論語》,暮春浴乎沂。



 自上及下,古有此禮。今三月上巳,祓於水濱,蓋出此也。」邕之言然。張衡《南都賦》
 祓於陽濱又是也。或用秋,《漢書》八月祓於霸上。劉楨《魯都賦》:「素秋二七,天漢指隅,人胥祓除,國子水嬉。」又是用七月十四日也。自魏以後但用三日,不以巳也。魏明帝天淵池南,設流杯石溝,燕群臣。晉海西鐘山後流杯曲水,延百僚,皆其事也。宮人循之至今。



 漢文帝始革三年喪制。臨終詔曰:「天下吏民臨三日,皆釋服。無禁取婦、嫁女、祠祀、飲酒、食肉。其當給喪事者,無跣。絰帶無過三寸。當臨者,皆旦夕各十五舉音。服大紅
 十五日,小紅十四日,纖七日而釋服。」文帝以己亥崩,乙巳葬,其間凡七日。自是之後,天下遵令,無復三年之禮。案《尸子》,禹治水,為喪法,曰毀必杖,哀必三年,是則水不救也。故使死於陵者葬於陵,死於澤者葬於澤。桐棺三寸,制喪三日。然則聖人之於急病,必為權制也。但漢文治致升平,四海寧晏,廢禮開薄,非也。宣帝地節四年,詔曰:「今百姓或遭衰絰凶災,而吏徭事不得葬,傷孝子心。自今諸有大父母、父母喪者,勿徭事,使得收斂送終,盡其
 子道。」至成帝時,丞相翟方進事父母孝謹,母終,既葬,三十六日,除服視事。自以為身備漢相,不敢踰國家典章。然而原涉行父喪三年,顯名天下。河間惠王行母喪三年,詔書褒稱,以為宗室儀表。薛修服母喪三年,而兄宣曰:「人少能行之。」遂兄弟不同,宣卒以此獲譏於世。是則喪禮見貴常存矣。至漢平帝崩,王莽欲眩惑天下,示忠孝,使六百石以上皆服喪三年。及莽母死,但服天子弔諸侯之服,一吊再會而已。而令子新都侯宇服喪三年。
 及元后崩,莽乃自服三年之禮。事皆姦妄,天下疾之。漢安帝初,長吏多避事棄官。乃令自非父母服,不得去職。是後吏又守職居官,不行三年喪服。其後又開長吏以下告寧,言事者或以為刺史二千石宜同此制,帝從之。建元元年,尚書孟布奏宜復如建武、永平故事,絕刺史二千石告寧及父母喪服,又從之。至桓帝永興二年,復令刺史二千石行三年服。永壽二年,又使中常侍以下行三年服。至延熹元年,又皆絕之。



 後漢世,諸帝不豫,並告泰山、弘農、廬江、常山、潁川、南陽、河東、東郡、廣陵太守禱祠五岳四瀆,遣司徒分詣郊廟社稷。



 魏武臨終遺令曰:「天下尚未安定,未得遵古。百官臨殿中者,十五舉音。葬畢,便除服。其將兵屯戍者,不得離部。」帝以正月庚子崩,辛丑即殯。是月丁卯葬,葬畢反吉,是為不逾月也。諸葛亮受劉備遺詔,既崩,群臣發喪,滿三日除服,到葬復如禮。其郡國太守、相、尉、縣令長三日便除服。此則魏、蜀喪制,又並異於漢也。孫權令諸居
 任遭三年之喪,皆須交代乃去,然多犯者。嘉禾六年,使群臣議立制,胡綜以為宜定大辟之科。又使代未至,不得告,告者抵罪。顧雍等同綜議,從之。其後吳令孟仁聞喪輒去,陸遜陳其素行,得減死一等,自此遂絕。



 晉宣帝崩,文、景並從權制。及文帝崩,國內行服三日。武帝亦遵漢、魏之典,既葬除喪,然猶深衣素冠,降席撤膳。太宰司馬孚、太傅鄭沖、太保王祥、太尉何曾、司徒領中領軍司馬望、司空荀顗、車騎將軍賈充、尚書令裴秀、尚書僕射武
 陔、都護大將軍郭建、侍中郭綏、中書監荀勖、中軍將軍羊祜等奏曰:「臣聞禮典軌度,豐殺隨時,虞、夏、商、周,咸不相襲,蓋有由也。大晉紹承漢、魏,有革有因,期於足以興化致治而已。故未皆得返情太素,同規上古也。陛下既已俯遵漢、魏降喪之典,以濟時務;而躬蹈大孝,情過乎哀,素冠深衣,降席撤膳。雖武丁行之於殷世,曾閔履之於布衣,未足以喻。方今荊蠻未夷,庶政未乂,萬機事殷,動勞神慮。豈遑全遂聖旨,以從至情。加歲時變易,期
 運忽過,山陵彌遠,攀慕永絕。臣等以為陛下宜回慮割情,以康時濟治。輒敕御府易服,內省改坐,太官復膳。諸所施行,皆如舊制。」詔曰:「每感念幽冥,而不得終苴絰於草土,以存此痛,況當食稻衣錦,誠佹然激切其心,非所以相解也。吾本諸生家,傳禮來久,何心一旦便易此情於所天。相從已多,可試省孔子答宰我之言,無事紛紜也。言及悲剝,奈何奈何!」孚等重奏:「伏讀明詔,感以悲懷。輒思仲尼所以抑宰我之問,聖思所以不能已已,甚深甚篤。
 然今者干戈未戢,武事未偃,萬機至重,天下至眾。陛下以萬乘之尊,履布衣之禮,服粗席槁,水飲疏食,殷憂內盈,毀悴外表,而躬勤萬機,坐而待旦,降心接下,仄不遑食,所以勞力者如斯之甚。是以臣等悚息不寧,誠懼神氣用損,以疚大事。輒敕有司改坐復常,率由舊典。惟陛下察納愚款,以慰皇太后之心。」又詔曰:「重覽奏議,益以悲剝,不能自勝,奈何奈何!三年之喪,自古達禮,誠聖人稱心立哀,明恕而行也。神靈日遠,無所告訴;雖薄於情,
 食旨服美,朕更所不堪也。不宜反覆,重傷其心,言用斷絕,奈何奈何!」帝遂以此禮終三年。後居太后之喪,亦如之。



 泰始二年八月,詔書曰:「此上旬,先帝棄天下日也,便以周年。吾煢煢,常復何時壹得敘人子情邪?思慕煩毒,欲詣陵瞻侍,以盡哀憤。主者具行備。」太宰司馬孚、尚書令裴秀、尚書僕射武陔等奏:「陛下至孝蒸蒸,哀思罔極。衰麻雖除,毀顇過禮,疏食粗服,有損神和。今雖秋節,尚有餘暑,謁見山陵,悲感摧傷,群下竊用悚息。平議以為
 宜惟遠體,降抑聖情,以慰萬國。」詔曰:「孤煢忽爾,日月已周,痛慕摧感,永無逮及。欲奉瞻山陵,以敘哀僨。體氣自佳,其又已涼,便當行,不得如所奏也。主者便具行備。」又詔曰:「昔者哀適三十日,便為梓宮所棄,遂離衰絰,感痛豈可勝言!顧漢文不使天下盡哀,亦先帝至謙之志,是以自割,不以副諸君子。有三年之愛,而身禮廓然,當見山陵,何心而無服,其以衰絰行。」



 孚等重奏:「臣聞上古喪期無數,後世乃有年月之漸。漢文帝隨時之義,制為短
 喪,傳之於後。陛下以社稷宗廟之重,萬方億兆之故,既從權制,釋降衰麻;群臣庶僚吉服。今者謁陵,以敘哀慕,若加衰絰,近臣期服,當復受制進退無當,不敢奉詔。」



 詔曰:「亦知不在此麻布耳。然人子情思,為欲令哀喪之物在身,蓋近情也。群臣自當案舊制。期服之義,非先帝意也。」孚等又奏:「臣聞聖人制作,必從時宜。



 故五帝殊樂,三王異禮。此古今所以不同,質文所以迭用也。陛下隨時之宜,既降心克己,俯就權制;既除衰麻,而行心喪之禮。
 今復制服,義無所依。若君服而臣不服,雖先帝厚恩,亦未之敢安也。參量平議,宜如前奏。臣等敢固以請。」詔曰:「患情不能企及耳,衣服何在?諸君勤勤之至,豈茍相違!」



 泰始四年,皇太后崩。有司奏:「前代故事,倚廬中施白縑帳蓐,素床,以布巾裹草。軺輦板輿細犢車皆施縑里。」詔不聽,但令以布衣車而已。其餘居喪之制,一如禮文。有司又奏:「大行皇太后當以四月二十五日安厝。故事,虞著衰服,既虞而除。其內外官僚,皆就朝晡臨位。御除
 服訖,各還所次除衰服。」詔曰:「夫三年之喪,天下之達禮也。受終身之愛,而無數年之報,奈何葬而便即吉,情所不忍也。」有司又奏:「世有險易,道有洿隆,所遇之時異。誠有由然,非忽禮也。方今戎馬未散,王事至殷,更須聽斷,以熙庶績。昔周康王始登翌室,猶戴冕臨朝。降於漢、魏,既葬除釋,諒暗之禮,自遠代而廢矣。唯陛下割高宗之制,從當時之宜。敢固以請。」詔曰:「攬省奏事,益增感剝。夫三年之喪,所以盡情致禮。葬已便除,所不堪也。當敘吾
 哀懷,言用斷絕,奈何奈何!」有司又固請。



 詔曰:「不能篤孝,勿以毀傷為憂也。誠知衣服末事耳。然今思存草土,率常以吉奪之,乃所以重傷至心,非見念也。每代禮典質文皆不同,此身何為限以近制,使達喪闕然乎!」群臣又固請,帝流涕久之,乃許。



 文帝崇陽陵先開一日,遣侍臣侍梓宮,又遣將軍校尉當直尉中監各一人,將殿中將軍以下及先帝時左右常給使詣陵宿衛。文明皇后崩及武元楊后崩,天下將吏發哀三日止。



 泰始元年,詔諸將吏二千石以下遭三年喪,聽歸終寧,庶人復除徭役。太康七年,大鴻臚鄭默母喪,既葬,當依舊攝職,固陳不起。於是始制大臣得終喪三年。



 然元康中,陳準、傅咸之徒,猶以權奪,不得終禮。自茲至今,往往以為成比也。



 晉文帝之崩也,羊祜謂傅玄曰:「三年之喪,自天子達;漢文除之,毀禮傷義。今上有曾、閔之性,實行喪禮。喪禮實行,何為除服。若因此守先王之法,不亦善乎?」



 玄曰;「漢文以
 末世淺薄,不能復行國君之喪,故因而除之。數百年一旦復古,恐難行也。」祜曰:「且使主上遂服,猶為善乎?」玄曰:「若上不除而臣下除,此為但有父子,無復君臣,三綱之道虧矣。」習鑿齒曰:「傅玄知無君臣之傷教,而不知兼無父子為重,豈不蔽哉!且漢廢君臣之喪,不降父子之服,故四海黎庶,莫不盡情於其親。三綱之道,二服恒用於私室,而王者獨盡廢之,豈所以孝治天下乎?



 《詩》云『猷之未遠』,其傅玄之謂
 也。」



 泰始十年,武元楊皇后崩。博士張靖議:「太子宜依漢文權制,割情除服。」



 博士陳逵議:「太子宜令服重。」尚書僕射盧欽、尚書魏舒、杜預奏:「諒暗之制,乃因自古。是以高宗無服喪之文,唯稱不言而已。漢文限三十六日,魏氏以既虞為斷。皇太子與國為體,理宜釋服。」博士段暢承述預旨,推引《禮》傳以成其說。



 既卒哭,太子及三夫人以下皆隨御除服。自漢文用權禮,無復□禁,歷代遵用之。



 至晉孝武崩,太傅錄尚書會稽王道子議:「山陵之後通婚
 嫁,不得作樂,以一期為限。」宋高祖崩,葬畢,吏民至于宮掖,悉通樂,唯殿內禁。



 宋武帝永初元年,黃門侍郎王準之議:「鄭玄喪制二十七月而終,學者多云得禮。晉初用王肅議,祥禫共月,遂以為制。江左以來,唯晉朝施用;搢紳之士,猶多遵玄議。宜使朝野一體。」詔可。



 晉惠帝永康元年,愍懷太子薨,帝依禮服長子三年,群臣服齊衰期。
 晉孝武太元二十一年,孝武帝崩,李太后制三年之制。



 宋武帝永初三年,武帝崩,蕭太后制三年之服。



 晉惠帝太安元年三月,皇太孫尚薨。有司奏:「御服齊衰期。」詔通議。散騎常侍謝衡以為諸侯之太子,誓與未誓,尊卑體殊,《喪服》云,為嫡子長殤,謂未誓也;已誓則不殤也。中書令卞粹曰:「太子始生,故已尊重,不待命誓。若衡議已誓不殤,則元服之子,當斬衰三年;未誓而殤,則雖十九,當大功九月。誓與未誓,其為升降也微;斬與大功,
 其為輕重也遠。而今注云,諸侯不降嫡殤,重嫌於無,以大功為重嫡之服。大功為重嫡之服,則雖誓,無復有三年之理明矣。男能奉衛社稷,女能奉婦道,各以可成之年,而有已成之事,故可無殤,非孩齔之謂也。



 謂殤後者,尊之如父,猶無所加,而止殤服。況以天子之尊,為無服之殤,行成人之制邪!凡諸宜重之殤,皆士大夫不加服,而令至尊獨居其重,未之前聞也。」博士蔡克同粹。祕書監摯虞議:「太子初生,舉以成人之禮,則殤理除矣。太孫
 亦體君重,由位成而服全,非以年也。天子無服殤之儀,絕期故也。」於是御史以上皆服齊衰。



 晉康帝建元元年正月晦,成恭杜皇后周忌。有司奏;「至尊期年應改服。」詔曰:「君親,名教之重也。權制出於近代耳。」於是素服如舊,非漢、魏之典。晉孝武太元九年,崇德太后褚氏崩。后於帝為從嫂,或疑其服。太學博士徐藻議:「資父事君而敬同。又《禮》傳,其夫屬乎父道者,妻皆母道也。則夫屬君道,妻亦后道矣。
 服后宜以資母之義。魯譏逆祀,以明尊尊。今上躬奉康、穆、哀皇及靖后之祀,致敬同於所天,豈可敬之以君道,而服廢於本親!謂應服齊衰期。」於是帝制期服。



 晉安帝隆安四年,太皇太后李氏崩。尚書祠部郎徐廣議:「太皇太后名位允正,體同皇極,理制備盡,情禮彌申。《陽秋》之義,母以子貴。既稱夫人,禮服從正。



 故成風顯夫人之號,昭公服三年之喪。子於父之所生,體尊義重。且禮祖不厭孫,宜遂服無屈。而緣情立制,若嫌明文不存,
 則疑斯從重。謂應同於為祖母後齊衰期。



 永安皇后無服,但一舉哀。百官亦一期。」詔可。



 宋文帝元嘉十七年七月壬子,元皇后崩。兼司徒給事中劉溫持節監喪。神虎門設凶門柏歷至西上皞,皇太子於東宮崇正殿及永福省並設廬。諸皇子未有府第者,於西廨設廬。元嘉十七年,元皇后崩。皇太子心喪三年。禮心喪者,有禫無禫,禮無成文,世或兩行。皇太子心喪畢,詔使博議。
 有司奏:「喪禮有示覃,以祥變有漸,不宜便除即吉,故其間服以FH縞也。心喪已經十三月,大祥十五月,祥禫變除,禮畢餘一期,不應復有禫。宣下以為永制。」詔可。



 孝武孝建三年三月,有司奏:「故散騎常侍、右光祿大夫、開府儀同三司義陽王師王偃喪逝。至尊為服緦三月,成服,仍即公除。至三月竟,未詳當除服與不?



 又皇后依朝制服心喪,行喪三十日公除。至祖葬日,臨喪當著何服?又舊事,皇后心喪,服終除之日,更還著未公除時服,
 然後就除。未詳今皇后除心制日,當依舊更服?為但釋心制中所著布素而已?勒禮官處正。」太學博士王膺之議:「尊卑殊制,輕重有級,五服雖同,降厭則異。禮,天子止降旁親;外舅緦麻,本在服例,但衰絰不可臨朝饗,故有公除之議。雖釋衰襲冕,尚有緦麻之制。愚謂至尊服三月既竟,猶宜除釋。」又議:「吉兇異容,情禮相稱。皇后一月之限雖過,二功之服已釋。哀情所極,莫深於尸柩,親見之重,不可以無服。案周禮,為兄弟既除喪已,及其葬
 也,反服其服。輕喪雖除,猶齊衰以臨葬。舉輕明重,則其理可知也。愚謂王右光祿祖葬之日,皇后宜反齊衰。」又議:「喪禮即遠,變除漸輕;情與日殺,服隨時改。權禮既行,服制已變,豈容終除之日,而更重服乎?案晉泰始三年,武帝以期除之月,欲反重服拜陵,頻詔勤勤,思申棘心。于時朝議譬執,亦遂不果。



 愚謂皇后終除之日,不宜還著重服,直當釋除布素而已。」太常丞硃膺之議:「凡云公除,非全除之稱。今朝臣私服,亦有公除,猶自窮其本制。
 膺之云,晉武拜陵不遂反服,此時是權制。既除衰麻,不可以重制耳,與公除不同。愚謂皇后除心制日,宜如舊反服未公除時服,以申創巨之情。」餘同膺之議。國子助教蘇瑋生議:「案三日成服即除,及皇后行喪三十日,禮無其文。若並謂之公除,則可粗相依準。



 凡諸公除之設,蓋以王制奪禮。葬及祥除,皆宜反服。未有服之於前,不除於後。



 雖有齊斬重制,猶為功緦除喪。夫公除暫奪,豈可遂以即吉邪?愚謂至尊三月服竟,故應依禮除釋。皇
 后臨祖,及一周祥除,並宜反服齊衰。」尚書令、中軍將軍建平王宏議謂:「至尊緦制終,止舉哀而已,不須釋服。」餘同朱膺之議。前祠部郎中周景遠議:「權事變禮,五服俱革,緦麻輕制,不容獨異。」謂:「至尊既已公除,至三月竟,不復有除釋之義。」其餘同硃膺之議。重加研詳,以宏議為允。詔可。



 大明二年正月,有司奏:「故右光祿大夫王偃喪,依格皇后服期,心喪三年,應再周來二月晦。檢元嘉十九年舊
 事,武康公主出適,二十五月心制終盡,從禮即吉。昔國哀再周,孝建二年二月,其月末,諸公主心制終,則應從吉。于時猶心禫素衣,二十七月乃除,二事不同。」領儀曹郎朱膺之議:「詳尋禮文,心喪不應有禫,皇代考檢,已為定制。元嘉季年,禍難深酷,聖心天至,喪紀過哀。是以出適公主,還同在室,即情變禮,非革舊章。今皇后二月晦,宜依元嘉十九年制,釋素即吉。」文帝元嘉十五年,皇太子妃祖父右光祿大夫殷和喪,變除之禮,儀同皇后。



 晉孝武太元十五年,淑媛陳氏卒,皇太子所生也。有司參詳母以子貴,贈淑媛為夫人,置家令典喪事。太子前衛率徐邈議:「《喪服》傳稱,與尊者為體,則不服其私親。又君父所不服,子亦不敢服。故王公妾子服其所生母,練冠麻衣,既葬而除。非五服之常,則謂之無服。」從之。宋孝武大明五年閏月,皇太子妃薨。樟木為櫬,號曰樟宮。載以龍輴。造陵於龍山,置大匠卿斷草,司空告后土,謂葬曰山塋。祔文元皇后廟之陰室,在正堂後壁之外,
 北向。御服大功九月,設位太極東宮堂殿。中監、黃門侍郎、僕射並從服。從服者,御服衰乃從服,他日則否。宮臣服齊衰三月,其居宮者處寧假。



 大明五年閏月,有司奏:「依禮皇太后服太子妃小功五月,皇后大功九月。」



 右丞徐爰參議:「宮人從服者,若二御哭臨應著衰時,從服者悉著衰,非其日如常儀。太子既有妃期服,詔見之日,還著公服。若至尊非哭臨日幸東宮,太子見亦如之。宮臣見至尊,皆著硃
 衣。」大明五年閏月,有司奏:「皇太子妃薨,至尊、皇后並服大功九月,皇太后小功五月,未詳二御何當得作鼓吹及樂?」博士司馬興之議:「案《禮》,『齊衰大功之喪,三月不從政。』今臨軒拜授,則人君之大典,今古既異,賒促不同。愚謂皇太子妃祔廟之後,便可臨軒作樂及鼓吹。」右丞徐爰議:「皇太子妃雖未山塋,臨軒拜官,舊不為礙。梓棺在殯,應縣而不作。祔後三御樂,宜使學官擬禮上。」興之又議:「案禮,大功至則辟琴瑟,誠無自奏之理。



 但王者體大,理
 絕凡庶。故漢文既葬,悉皆復吉,唯縣而不樂,以此表哀。今準其輕重,侔其降殺,則下流大功,不容撤樂以終服。夫金石賓饗之禮,簫管警塗之衛,實人君之盛典,當陽之威飾,固亦不可久廢於朝。又禮無天王服嫡婦之文,直後學推貴嫡之義耳。既已制服成喪,虛懸終窆,亦足以甄崇塚正,標明禮歸矣。」爰參議,皇太子期服內,不合作樂及鼓吹。



 明帝泰始中,陳貴妃父金寶卒,貴妃制服三十日滿,公
 除。晉穆帝時,東海國言哀王薨踰年,嗣王乃來繼,不復追服,群臣皆已反吉,國妃亦宜同除。詔曰:「朝廷所以從權制者,以王事奪之,非為變禮也。婦人傳重義大,若從權制,義將安託?」於是國妃終三年之制。孫盛曰:「廢三年之禮,開偷薄之源,漢、魏失之大者也。今若以丈夫宜奪以王事,婦人可終本服,是為吉凶之儀,雜陳於宮寢;彩素之制,乖異於內外,無乃情禮俱違,哀樂失所乎!蕃國寡務,宜如聖典,可無疑矣。」



 宋文帝元嘉四年八月,太傅長沙景王神主隨子南兗州刺史義欣鎮廣陵,備所加殊禮下船。及至鎮,入行廟。大司馬臨川烈武王神主隨子荊州刺史義慶江陵,亦如之。



 元嘉二十三年七月,白衣領御史中丞何承天奏:尚書刺:「海鹽公主所生母蔣美人喪。海鹽公主先離婚,今應成服,撰儀注參詳,宜下二學禮官博士議公主所服輕重。太學博士顧雅議:『今既咸用士禮,便宜同齊衰削杖,
 布帶疏履,期,禮畢,心喪三年。』博士周野王議又云:『今諸王公主咸用士禮。譙王、衡陽王為所生太妃皆居重服,則公主情禮,亦宜家中期服為允。』其博士庾邃之、顏測、殷明、王淵之四人同雅議;何惔、王羅雲二人同野王議。」



 如所上臺案。今之諸王,雖行士禮,是施於傍親及自己以下。至於為帝王所厭,猶一依古典。又永初三年九月,符修儀亡,廣德三主以餘尊所厭,猶服大功。海鹽公主體自宸極,當上厭至尊,豈得遂服?臺據《經》、傳正文,並
 引事例,依源責失。



 而博士顧雅、周野王等捍不肯怗,方稱「自有宋以來,皇子蕃王,皆無厭降,同之士禮,著於故事。緦功之服,不廢於末戚,顧獨貶於所生,是申其所輕,奪其所重;奪其所重,豈緣情之謂?」臺伏尋聖朝受終於晉,凡所施行,莫不上稽禮文,兼用晉事。又太元中,晉恭帝時為皇子,服其所生陳氏,練冠縓緣,此則前代施行故事,謹依禮文者也。又廣德三公主為所生母符脩儀服大功,此先君餘尊之所厭者也。元嘉十三年,第七皇
 子不服曹婕妤,止於麻衣,此厭乎至尊者也。博士既不據古,又不依今,背違施行見事,而多作浮辭自衛。乃云五帝之時,三王之季。又言長子去斬衰,除禫杖,皆是古禮,不少今世。博士雖復引此諸條,無救於失。又詰臺云「蕃國得遂其私情,此義出何經記?」臣案南譙、衡陽太妃並受朝命,為國小君,是以二王得遂其服,豈可為美人比例?尋蕃王得遂者,聖朝之所許也。皇子公主不得申者,由有厭而然也。臺登重更責失制不得過十日,而復不
 酬答。既被催攝二三日,甫輸怗辭。雖理屈事窮,猶聞義恥服。臣聞喪紀有制,禮之大經;降殺攸宜,家國舊典。古之諸侯眾子,猶以尊厭;況在王室,而欲同之士庶。此之僻謬,不俟言而顯。太常統寺,曾不研卻,所謂同乎失者,亦未得之。宜加裁正,弘明國典。



 謹案太學博士顧雅、國子助教周野王、博士王羅雲、顏測、殷明、何惔、王淵之、前博士遷員外散騎侍郎庾邃之等,咸蒙抽飾,備位前疑,既不謹守舊文,又不審據前準,遂上背經典,下違故事,
 率意妄作,自造禮章。太常臣敬叔位居宗伯,問禮所司,騰述往反,了無研卻,混同茲失,亦宜及咎。請以見事並免今所居官,解野王領國子助教。雅、野王初立議乖舛,中執捍愆失,未違十日之限。雖起一事,合成三愆,羅雲掌押捍失,三人加禁固五年。



 詔敬叔白衣領職。餘如奏。元嘉二十九年,南平王鑠所生母吳淑儀薨。依禮無服,麻衣練冠,既葬而除。有司奏:「古者與尊者為體,不得服其私親。而比世諸侯咸用士禮,五服之內悉皆成服,於
 其所生,反不得遂。」於是皇子皆申母服。



 孝武帝孝建元年六月己巳,有司奏:「故第十六皇弟休倩薨夭,年始及殤,追贈謚東平沖王。服制未有成準,輒下禮官詳議。」太學博士陸澄議:「案禮有成人道,則不為殤。今既追胙土宇,遠崇封秩,圭黻備典,成孰大焉。典文式昭,殤名去矣。夫典文垂式,元服表身,猶以免孺子之制,全丈夫之義。安有名頒爵首,而可服以殤禮!」有司尋澄議無明證,卻使秉正更上。澄重議:「竊謂贈之為義,所
 以追加名器。故贈公者便成公,贈卿者便成卿。贈之以王,得不為王乎?然則有在生而封,或既沒而爵,俱受帝命,不為吉兇殊典;同備文物,豈以存亡異數?今璽策咸秩,是成人之禮;群后臨哀,非下殤之制。若喪用成人,親以殤服,末學含疑,未之或辨。敢求詳衷如所稱。」左丞臣羊希參議:「尋澄議,既無畫然前例,不合準據。案《禮》,子不殤父,臣不殤君。君父至尊,臣子恩重,不得以幼年而降。



 又曰,『尊同則服其親服』,推此文旨,旁親自宜服殤,所不
 殤者唯施臣子而已。」



 詔可。



 孝建元年六月,湘東國刺稱「國太妃以去三十年閏六月二十八日薨。未詳周忌當在六月?為取七月?勒禮官議正」。博士丘邁之議:「案吳商議,閏月亡者,應以本正之月為忌。謂正閏論雖各有所執,商議為允。宜以今六月為忌。」左僕射建平王宏謂:「邁之議不可準據。案晉世及皇代以來,閏月亡者,以閏之後月祥。宜以來七月為祥忌。」及大明元年二月,有司又奏:「太常鄱陽哀王去年閏
 三月十八日薨。今為何月末祥除?」下禮官議正。博士傅休議:「尋《三禮》,喪遇閏,月數者數閏,歲數者沒閏,閏在期內故也。鄱陽哀王去年閏三月薨,月次節物,則定是四月之分,應以今年四月末為祥。晉元、明二帝,並以閏二月崩,以閏後月祥,先代成準,則是今比。」



 太常丞庾蔚之議:「禮,正月存親,故有忌日之感。四時既已變,人情亦已衰,故有二祥之殺。是則祥忌皆以同月為議,而閏亡者,明年必無其月,不可以無其月而不祥忌,故必宜用閏
 所附之月。閏月附正,《公羊》明議,故班固以閏九月為後九月,月名既不殊,天時亦不異。若用閏之後月,則春夏永革,節候亦舛。設有人以閏臘月亡者,若用閏後月為祥忌,則祥忌應在後年正月。祥涉三載,既失周期之義,冬亡而春忌;又乖致感之本。譬今年末三十日亡,明年末月小,若以去年二十九日親尚存,則應用後年正朝為忌,此必不然。則閏亡可知也。」通關並同蔚之議,三月
 末祥。



 大明五年七月,有司奏:「故永陽縣開國侯劉叔子夭喪,年始四歲,傍親服制有疑。」太學博士虞龢、領軍長史周景遠、司馬朱膺之、前太常丞庾蔚之等議,並云「宜同成人之服。東平沖王服殤,實由追贈,異於已受茅土」。博士司馬興之議:「應同東平殤服。」左丞荀萬秋等參議:「南面君國,繼體承家,雖則佩觿,未闕成人,得君父名也,不容服殤,故云『臣不殤君,子不殤父』。推此,則知傍親故依殤制。東平沖王已經前議。若升仕朝列,則為大成,故鄱陽
 哀王追贈太常,親戚不降。愚謂下殤以上,身居封爵,宜同成人。年在無服之殤,以登官為斷。今永陽國臣,自應全服。至於傍親,宜從殤禮。」詔:「景遠議為允」。後廢帝元徽二年七月,有司奏:「第七皇弟訓養母鄭修容喪。未詳服制,下禮官正議。」太學博士周山文議:「案庶母慈己者,小功五月。鄭玄云:『其使養之不命為母子,亦服庶母慈己之服。』愚謂第七皇弟宜從小功之制。」參議並同。



 漢、魏廢帝喪親三年之制,而魏世或為舊君服三
 年者。至晉太始四年,尚書何楨奏:「故辟舉綱紀吏,不計違適,皆反服舊君齊衰三月。」於是詔書下其奏,所適無貴賤,悉同依古典。



 魏武以正月崩,魏文以其年七月設伎樂百戲,是魏不以喪廢樂也。晉武帝以來,國有大喪未除,正會亦廢樂。太安元年,太子喪未除,正會亦廢樂。穆帝永和中,為中原山陵未脩復,頻年會,輒廢樂。是時太后臨朝,后父褚裒薨,元會又廢樂。



 晉世孝武太元六年,為皇后王氏喪,亦廢樂。宋大喪則廢樂。



 漢獻帝建安末,魏武帝作終令曰:「古之葬者,必在瘠薄之地,其規西原上為壽陵。因高為基,不封不樹。《周禮》,塚人掌公墓之地,凡諸侯居左右以前,卿大夫居後。漢制亦謂之陪陵。其公卿大臣列將有功者,宜陪壽陵。其廣為兆域,使足相容。」魏武以送終制衣服四篋,題識其上,春秋冬夏日有不諱,隨時以斂;金珥珠玉銅鐵之物,一
 不得送。文帝遵奉,無所增加。及受禪,刻金璽,追加尊號。



 不敢開埏,乃為石室,藏璽埏首,示陵中無金銀諸物也。漢禮明器甚多,自是皆省矣。



 文帝黃初三年,又自作終制:「禮,國君即位,為椑,存不忘亡也。壽陵因山為體,無封無樹,無立寢殿,造圓邑,通神道。夫葬者,藏也。欲人之不能見也。



 禮不墓祭,欲存亡之不黷也。皇后及貴人以下,不隨王之國者,有終沒,皆葬澗西,前又已表其處矣。」此詔藏之宗廟,副在尚書、祕書
 三府,明帝亦遵奉之。明帝性雖崇奢,然未遽營陵墓也。晉宣帝豫自於首陽山為土藏,不墳不樹,作顧命終制,斂以時服,不設明器。文、景皆謹奉成命,無所加焉。景帝崩,喪事制度,又依宣帝故事。



 武帝泰始四年,文明王皇后崩,將合葬,開崇陽陵。使太尉司馬望奉祭,進皇帝密璽綬於便房神坐。魏氏金璽,此又儉矣。



 泰始二年,詔曰:「昔舜葬蒼梧,農不易畝;禹葬會稽,市不改肆。上惟祖考清簡之旨,外欲移陵十里內居人,一切
 停之。」江左元、明崇儉,且百度草創,山陵奉終,省約備矣。



 成帝咸康七年,杜后崩。詔外官五日一入臨,內官旦一入而已。過葬虞祭禮畢止。有司奏;「大行皇后陵所作凶門柏歷,門號顯陽端門。」詔曰:「門如所處,兇門柏歷,大為煩費,停之。」案蔡謨說,以二瓦器盛死者之祭,系於木表,以葦席置於庭中近南,名為重。今之凶門,是其象也。《禮》,既虞而作主。今未葬,未有主,故以重當之。《禮》稱為主道,此其義也。範堅又曰:「凶門非古,古有懸重,形似凶門。後
 人出之門外以表喪,俗遂行之。薄帳,既古弔幕之類也。」是時又詔曰:「重壤之下,豈宜崇飾無用。陵中唯潔掃而已。」有司又奏,依舊選公卿以下六品子弟六十人為挽郎,詔又停之。



 孝武帝太元四年九月,皇后王氏崩。詔曰:「終事唯從儉速。」又詔:「遠近不得遣山陵使。」有司奏選挽郎二十四人,詔停。宋元帝元嘉十七年,元皇后崩,詔亦停選挽郎。漢儀五
 供畢則上陵,歲歲以為常,魏則無定禮。齊王在位九載,始一謁高平陵,而曹爽誅,其後遂廢,終魏世。



 晉宣帝遺詔:「子弟群官,皆不得謁陵。」於是景、文遵旨。至武帝猶再謁崇陽陵,一謁峻平陵,然遂不敢謁高原陵。至惠帝復止也。逮江左初,元帝崩後,諸公始有謁陵辭陵之事,蓋由眷同友執,率情而舉,非洛京之舊也。成帝時,中宮亦年年拜陵,議者以為非禮,於是遂止,以為永制。至穆帝時,褚太后臨朝,又拜陵,帝幼故也。至孝武崩,
 驃騎將軍司馬道子命曰:「今雖權制釋服,至於朔望諸節,自應展情陵所,以一周為斷。」於是至陵變服單衣夾,煩瀆無準,非禮意也。至安帝元興元年,尚書左僕射桓謙奏曰:「百僚拜陵,起於中興,非晉舊典。積習生常,遂為近法。尋武皇帝詔,乃不使人主諸王拜陵,豈唯百僚。謂宜遵奉。」於是施行。及義熙初,又復江左之舊。



 宋明帝又斷群臣初拜謁陵,而辭如故。自元嘉以來,每歲正月,輿駕必謁初寧陵,復漢儀也。世祖、太宗亦每歲
 拜初寧、長寧陵。



 漢以後,天下送死奢靡,多作石室石獸碑銘等物。建安十年,魏武帝以天下雕弊,下令不得厚葬,又禁立碑。魏高貴鄉公甘露二年,大將軍參軍太原王倫卒,倫兄俊作《表德論》,以述倫遺美,云「祗畏王典,不得為銘,乃撰錄行事,就刊於墓之陰云爾」。此則碑禁尚嚴也,此後復弛替。晉武帝咸寧四年,又詔曰:「此石獸碑表,既私褒美,興長
 虛偽,傷財害人,莫大於此;一禁斷之。其犯者雖會赦令,皆當毀壞。」至元帝太興元年,有司奏:「故驃騎府主簿故恩營葬舊君顧榮,求立碑。」詔特聽立。自是後,禁又漸頹。大臣長吏,人皆私立。義熙中,尚書祠部郎中裴松之又議禁斷,於是至今。



 順帝昇明三年四月壬辰,御臨軒,遣使奉璽綬禪位於齊王,懸而不樂。



 宋明帝泰始二年九月,有司奏:「皇太子所生陳貴妃禮
 秩既同儲宮,未詳宮臣及朝臣並有敬不?妃主在內相見,又應何儀?」博士王慶緒議:「百僚內外禮敬貴妃,應與皇太子同。其東朝臣隸,理歸臣節。」太常丞虞愿等同慶緒。尚書令建安王休仁議稱:「禮云,妾既不得體君,班秩視子為序。母以子貴,經著明文。內外致敬貴妃,誠如慶緒議。天子姬嬪,不容通音介於外,雖義可致虔,不應有箋表。」



 參詳休仁議為允。詔可。



 泰豫元年,後廢帝即位,崇所生陳貴妃為皇太妃。有司
 奏:「皇太妃位亞尊極,未詳國親舉哀格當一同皇太后?為有降異?又於本親期以下,當猶服與不?」前曹郎王燮之議:「案《喪服傳》,『妾服君之黨,得與女君同』。如此,皇太妃服宗與太后無異。但太后既以尊降無服,太妃儀不應殊,故悉不服也。計本情舉哀,其禮不異。又《禮》,『諸侯絕期』。皇太妃雖云不居尊極,不容輕於諸侯。謂本親期以下,一無所服。有慘自宜舉哀。親疏二儀,準之太后。」兼太常丞司馬燮之議:「《禮》,『妾服君之庶子及女君之黨』。皆謂大
 夫士耳。妾名雖總,而班有貴賤。



 三夫人九嬪,位視公卿。大夫猶有貴妾,而況天子!諸侯之妾為他妾之子無服,既不服他妾之子,豈容服君及女君餘親。況皇太后妃貴亞相極,禮絕群后,崇輝盛典,有踰東儲,尚不服期,太妃豈應有異。若本親有慘,舉哀之儀,宜仰則太后。」參議以燮之議為允。太妃於國親無服,故宜緣情為諸王公主於至尊是期服者反,其太妃王妃三夫人九嬪各舉哀。



 宋孝武帝孝建三年八月戊子,有司奏:「雲杜國解稱國子檀和之所生親王,求除太夫人。檢無國子除太夫人先例,法又無科。下禮官議正。」太學博士孫豁之議:「《春秋》,『母以子貴』。王雖為妾,是和之所生。案五等之例,鄭伯許男同號夫人,國子體例,王合如國所生。」太常丞庾蔚之議:「『母以子貴』,雖《春秋》明義,古今異制,因革不同。自頃代以來,所生蒙榮,唯有諸王。既是王者之嬪禦,故宜見尊於蕃國。若功高勳重,列為公侯,亦有拜太夫人之禮。凡
 此皆朝恩曲降,非國之所求。子男妾母,未有前比。」祠部郎中朱膺之議以為:「子不得爵父母,而《春秋》有『母以子貴』。當謂傳國君母,本先公嬪媵,所因藉有由故也。始封之身,所不得同。若殊績重勳,恩所特錫,時或有之,不由司存。」所議參議,以蔚之為允。詔可。



 大明二年六月,有司奏:「凡侯伯子男世子喪,無嗣,求進次息為世子。檢無其例,下禮官議正。」博士孫武議:「案晉濟北侯荀勖長子連卒,以次子輯拜世子。



 先代成準,宜
 為今例。」博士傅郁議:「《禮記》,微子立衍,商禮斯行。仲子舍孫,姬典攸貶。歷代遵循,靡替于舊。今胙土之君在而世子卒,厥嗣未育,非孫之謂。愚以為次子有子,自宜紹為世孫。若其未也,無容遠搜輕屬,承綱繼體,傳之有由。父在立子,允稱情典。」曹郎諸葛雅之議:「案《春秋傳》云,『世子死,有母弟則弟,無則立長;年均則賢,義均則卜』。古之制也。今長子早卒,無嗣,進立次息以為世子,取諸《左氏》,理義無違。又孫武所據晉濟北侯荀勖長子卒,立次子,亦
 近代成例。依文採比,竊所允安。謂宜開許,以為永制。」參議為允。



 詔可。



 大明十二年十一月,有司奏:「興平國解稱國子袁愍孫母王氏,應除太夫人。



 檢無國母除太夫人例。下禮官議正。」太學博士司馬興之議:「案禮,下國卿大夫之妻,皆命天子。以斯而推,則子男之母,不容獨異。」博士程彥議以為:「五等雖差,而承家事等。公侯之母,崇號得從,子男於親,尊秩宜顯。故《春秋》之義,『母以子貴』。固知從子尊與國
 均也。彥參議,以興之議為允。除王氏為興平縣開國子太夫人。」詔可。



 大明四年九月,有司奏:「陳留國王曹虔季長兄虔嗣早卒,季襲封之後,生子銑以繼虔嗣。今依例應拜世子,未詳應以銑為世子?為應立次子鍇?」太學博士王溫之、江長議,並為應以銑為正嗣;太常陸澄議立鍇。右丞徐爰議謂:「禮後大宗,以其不可乏祀。諸侯世及,《春秋》成義。虔嗣承家傳爵,身為國王,雖薨沒無子,猶列昭穆。立後之
 日,便應即纂國統。於時既無承繼,虔季以次襲紹。虔嗣既列廟饗,故自與世數而遷,豈容蒸嘗無闕,橫取他子為嗣!為人胤嗣,又應恭祀先父。



 案禮文,公子不得禰諸侯。虔嗣無緣降廟就寢。銑本長息,宜還為虔季世子。」詔如爰議。



 宋文帝元嘉十三年七月,有司奏:「御史中丞劉式之議,『每至出行,未知制與何官分道,應有舊科。法唯稱中丞專道,傳詔荷信,詔喚眾官,應詔者行,得制令無分別他
 官之文,既無盡然定則,準承有疑。謂皇太子正議東儲,不宜與眾同例,中丞應與分道。揚州刺史、丹陽尹、建康令,並是京輦土地之主,或檢校非違,或赴救水火,事應神速,不宜稽駐,亦合分道。又尋六門則為行馬之內,且禁衛非違,並由二衛及領軍,未詳京尹、建康令門內之徒及公事,亦得與中丞分道與不?其準參舊儀。告報參詳所宜分道。』聽如臺所上,其六門內,既非州郡縣部界,則不合依門外。其尚書令、二僕射所應分道,亦悉與中
 丞同。」



 孝武帝大明六年五月,詔立凌室藏冰。有司奏,季冬之月,冰壯之時,凌室長率山虞及輿隸取冰於深山窮谷涸陰冱寒之處,以納於凌陰。務令周密,無泄其氣。



 先以黑牡翽黍祭司寒於凌室之北。仲春之月,春分之日,以黑羔翽黍祭司寒。啟冰室,先薦寢廟。二廟夏祠用鑑盛冰,室一鑒,以御溫氣蠅蚋。三御殿及太官膳羞,並以鑒供冰。自春分至立秋,有臣妾喪,詔贈祕器。自立夏至立秋,
 不限稱數以周喪事。繕制夷盤,隨冰借給。凌室在樂游苑內,置長一人,保舉吏二人。



 三公黃皞,前史無其義。史臣按,《禮記》「士韠與天子同,公侯大夫則異」。



 鄭玄注:「士賤,與君同,不嫌也。」夫朱門洞啟,當陽之正色也。三公之與天子,禮秩相亞,故黃其皞,以示謙不敢斥天子,蓋是漢來制也。張超與陳公箋,「拜黃皞將有日月」是也。



 史臣按:今朝士詣三公,尚書丞、郎詣令、僕射、尚書,並門外下車,履,度門閫乃納屐。漢世朝臣見三公,並拜。丞、郎
 見八座,皆持板揖,事在《漢儀》及《漢舊儀》,然則並有敬也。陳蕃為光祿勳,範滂為主事,以公儀詣蕃,執板入皞,至坐,蕃不奪滂板,滂投板振衣而去。郭泰責蕃曰:「以階級言之,滂宜有敬;以類數推之,至皞宜省。」然後敬止在門,其來久矣。



\end{pinyinscope}