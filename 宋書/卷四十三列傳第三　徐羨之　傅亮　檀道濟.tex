\article{卷四十三列傳第三 徐羨之 傅亮 檀道濟}

\begin{pinyinscope}

 徐羨
 之,字宗文,東海郯人也。祖寧,
 尚書吏部郎,江州刺史,未拜卒。父祚之,上虞令。羨之少為王雅太子少傅主簿,劉牢之鎮北功曹,尚書祠部郎,不拜,桓修撫軍中兵
 曹參軍。與高祖同府,深相親結。義旗建,高祖版為鎮軍參軍,尚書庫部郎,領軍司馬。與謝混共事,混甚知之。補琅邪王大司馬參軍,司徒左西屬,徐州別駕從事史,太尉咨議參軍。義熙十一年,除鷹揚將軍、琅邪內史,仍為大司馬從事中郎,將軍如故。高祖北伐,轉太尉左司馬,掌留任,以副貳劉穆之。



 初,高祖議欲北伐,朝士多諫,唯羨之默然。或問何獨不言,羨之曰:「吾位至二品,官為二千石,志願久充。今二方已平,拓地萬里,唯有小羌未定,而
 公寢食不忘。意量乖殊,何可輕豫。」劉穆之卒,高祖命以羨之為吏部尚書、建威將軍、丹陽尹,總知留任,甲仗二十人出入。轉尚書僕射,將軍、尹如故。



 十四年,大司馬府軍人朱興妻周坐息男道扶年三歲,先得癇病,周因其病發,掘地生埋之,為道扶姑女所告,正周棄市刑。羨之議曰:「自然之愛,虎狼猶仁。



 周之凶忍,宜加顯戮。臣以為法律之外,故尚弘物之理。母之即刑,由子明法,為子之道,焉有自容之地。雖伏法者當罪,而在宥者靡容。愚謂
 可特申之遐裔。」從之。



 高祖踐阼,進號鎮軍將軍,加散騎常侍。上初即位,思佐命之功,詔曰:「散騎常侍、尚書僕射、鎮軍將軍、丹陽尹徐羨之,監江州豫州之西陽新蔡諸軍事、撫軍將軍、江州刺史華容侯王弘,散騎常侍、護軍將軍作唐男檀道濟,中書令、領太子詹事傅亮,侍中、中領軍謝晦,前左將軍、江州刺史宜陽侯檀韶,使持節、雍梁南北秦四州荊州之河北諸軍事、後將軍、雍州刺史關中侯趙倫之,使持節、督北徐兗青三州諸軍事、征虜
 將軍、北徐州刺史南城男劉懷慎,散騎常侍、領太子左衛率新淦侯王仲德,前冠軍將軍、北青州刺史安南男向彌,左衛將軍灄陽男劉粹,使持節、南蠻校尉佷山子到彥之,西中郎司馬南郡宜陽侯張邵,參西中郎將軍事、建威將軍、河東太守資中侯沈林子等,或忠規遠謀,扶贊洪業;或肆勤樹績,弘濟艱難。



 經始圖終,勳烈惟茂,並宜與國同休,饗茲大賚。羨之可封南昌縣公,弘可華容縣公,道濟可改封永修縣公,亮可建城縣公,晦可武
 昌縣公,食邑各二千戶;韶可更增邑二千五百戶,仲德可增邑二千二百戶;懷慎、彥之各進爵為侯,粹改封建安縣侯,並增邑為千戶;倫之可封霄城縣侯,食邑千戶;邵可封臨沮縣伯,林子可封漢壽縣伯,食邑六百戶。開國之制,率遵舊章。」



 羨之遷尚書令、揚州刺史,加散騎常侍。進位司空、錄尚書事,常侍、刺史如故。羨之起布衣,又無術學,直以志力局度,一旦居廊廟,朝野推服,咸謂有宰臣之望。沈密寡言,不以憂喜見色。頗工弈棋,觀戲
 常若未解,當世倍以此推之。傅亮、蔡廓常言:「徐公曉萬事,安異同。」



 高祖不豫,加班劍三十人。宮車晏駕,與中書令傅亮、領軍將軍謝晦、鎮北將軍檀道濟同被顧命。少帝詔曰:「平理獄訟,政道所先。朕哀荒在疚,未堪親覽。



 司空、尚書令可率眾官月一決獄。」



 帝後失德,羨之等將謀廢立,而廬陵王義真輕動多過,不任四海,乃先廢義真,然後廢帝。時謝晦為領軍,以府舍內屋敗應治,悉移家人出宅,聚將士於府內。鎮北將軍、南兗州刺史檀道濟
 先朝舊將,威服殿省,且有兵眾,召使入朝,告之以謀。



 事將發,道濟入宿領軍府。中書舍人邢安泰、潘盛為內應,其日守關。道濟領兵居前,羨之等繼其後,由東掖門雲龍門入,宿衛先受處分,莫有動者。先是帝於華林園為列肆,親自酤賣,又開瀆聚土,以像破崗,率左右唱呼引船為樂。是夕,寢於龍舟,在天淵池。兵士進殺二人,又傷帝指。扶帝出東閣,收璽綬。群臣拜辭,衛送故太子宮,遷於吳郡。侍中程道惠勸立第五皇弟義恭,羨之不許。遣使
 殺義真於新安,殺帝於吳縣。時為帝築宮未成,權居金昌亭,帝突走出昌門,追者以門關擊之倒地,然後加害。



 太祖即阼,進羨之司徒,餘如故,改封南平郡公,食邑四千戶,固讓加封。有司奏車駕依舊臨華林園聽訟,詔曰:「政刑多所未悉,可如先二公推訊。」



 元嘉二年,羨之與左光祿大夫傅亮上表歸政,曰:「臣聞元首司契,運樞成務;臣道代終,事盡宣翼。冕旒之道,理絕於上皇;拱己之事,不行於中古。故高宗不言,以三齡為斷;塚宰聽政,以
 再期為節。百王以降,罔或不然。陛下聖德紹興,負荷洪業,憶兆顒顒,思陶盛化。而聖旨謙挹,委成群司。自大禮告終,鑽燧三改,大明佇照,遠邇傾屬。臣等雖率誠屢聞,未能仰感,敢藉品物之情,謹因蒼生之志。



 伏願陛下遠存周文日昃之道,近思皇室締構之艱,時攬萬機,躬親朝政,廣闢四聰,博詢庶業,則雍熙可臻,有生幸甚。」上未許。羨之等重奏曰:「近寫下情,言為心罄,奉被還詔,鑒許未回。豈惟愚臣,秉心有在,詢之朝野,人無異議。何者?



 形
 風四方,實繫王德,一國之事,本之一人。雖世代不同,時殊風異,至於主運臣贊,古今一揆。未有渾心委任,而休明可期,此之非宜,布自遐邇。臣等荷遇二世,休戚以均,情為國至,豈容順默。重披丹心,冒昧以請。」上猶辭。羨之等又固陳曰:「比表披陳,辭誠俱盡,詔旨沖遠,未垂聽納,三復屏營,伏增憂歎。臣聞克隆先構,幹蠱之盛業;昧旦丕顯,帝王之高義。自皇宋創運,英聖有造,殷憂未闕,艱患仍纏。賴天命有底,聖明承業,時屯國故,猶在民心。泰
 山之安,未易可保,昏明隆替,繫在聖躬。斯誠周詩夙興之辰,殷王待旦之日,豈得無為拱己,復玄古之風,逡巡虛挹,徇匹夫之事。伏願以宗廟為重,百姓為心,弘大業以嗣先軌,隆聖道以增前烈。愚瞽所獻,情盡於此。」上乃許之。羨之仍遜位退還私第,兄子佩之及侍中程道惠、吳興太守王韶之等並謂非宜,敦勸甚苦,復奉詔攝任。



 三年正月,詔曰:「民生於三,事之如一,愛敬同極,豈惟名教,況乃施侔造物,義在加隆者乎!徐羨之、傅亮、謝晦,皆
 因緣之才,荷恩在昔,擢自無聞,超居要重,卵翼而長,未足以譬。永初之季,天禍橫流,大明傾曜,四海遏密,實受顧託,任同負圖。而不能竭其股肱,盡其心力,送往無復言之節,事居闕忠貞之效,將順靡記,匡救蔑聞,懷寵取容,順成失德。雖末因懼禍,以建大策,而逞其悖心,不畏不義。播遷之始,謀肆鴆毒,至止未幾,顯行怨殺,窮凶極虐,荼酷備加,顛沛皁隸之手,告盡逆旅之館,都鄙哀愕,行路飲涕。故廬陵王英秀明遠,徽風夙播,魯衛之寄,朝
 野屬情。羨之等暴蔑求專,忌賢畏逼,造構貝錦,成此無端,罔主蒙上,橫加流屏,矯誣朝旨,致茲禍害。寄以國命,而翦為仇讎,旬月之間,再肆鴆毒,痛感三靈,怨結人鬼。自書契以來,棄常安忍,反易天明,未有如斯之甚者也。



 昔子家從弒,鄭人致討;宋肥無辜,蕩澤為戮。況逆亂倍於往釁,情痛深於國家,此而可容,孰不可忍!即宜誅殛,告謝存亡。而于時大事甫爾,異同紛結,匡國之勳實著,莫大之罪未彰。是以遠酌民心,近聽輿訟,雖欲討亂,慮
 或難圖,故忍戚含哀,懷恥累載。每念人生實難,情事未展,何嘗不顧影慟心,伏枕泣血。今逆臣之釁,彰暴遐邇,君子悲情,義徒思奮,家仇國恥,可得而雪,便命司寇,肅明典刑。晦據有上流,或不即罪,朕當親率六師,為其遏防。可遣中領軍到彥之即日電發,征北將軍檀道濟絡驛繼路,符衛軍府州以時收翦。已命征虜將軍劉粹斷其走伏。



 罪止元凶,餘無所問。感惟永往,心情崩絕。氛霧既袪,庶幾治道。」


爾日詔召羨之。行至西明門外,時謝晦
 弟㬭
 \gezhu{
  子肖反}
 為黃門郎,正直,報亮云:「殿內有異處分。」亮馳報羨之。羨之回還西州,乘內人問訊車出郭,步走至新林,入陶灶中自剄死,時年六十三。羨之初不應召,上遣中領軍到彥之、右衛將軍王華追討。羨之死,野人以告,載尸付廷尉。子喬之,尚高祖第六女富陽公主,官至竟陵王文學。喬之及弟乞奴從誅。



 初,羨之年少時,嘗有一人來,謂之曰:「我是汝祖。」羨之因起拜之。此人曰:「汝有貴相,而有大厄,可以錢二十八文埋宅四角,可以免災。過
 此可位極人臣。」後羨之隨親之縣,住在縣內,嘗暫出,而賊自後破縣;縣內人無免者,雞犬亦盡,唯羨之在外獲全。隨從兄履之為臨海樂安縣,嘗行經山中,見黑龍長丈餘,頭有角,前兩足皆具,無後足,曳尾而行。及拜司空,守關將入,彗星晨見危南。



 又當拜時,雙鶴集太極東鴟尾鳴喚。



 兄子佩之,輕薄好利,高祖以其姻戚,累加寵任,為丹陽尹,吳郡太守。景平初,以羨之秉權,頗豫政事。與王韶之、程道惠、中書舍人邢安泰、潘盛相結黨與。



 時謝
 晦久病,連灸,不堪見客。佩之等疑其託疾有異圖,與韶之、道惠同載詣傅亮,稱羨之意,欲令亮作詔誅之。亮答以為:「己等三人,同受顧命,豈可相殘戮!若諸君果行此事,便當角巾步出掖門耳。」佩之等乃止。羨之既誅,太祖特宥佩之,免官而已。其年冬,佩之又結殿中監茅亨謀反,并告前寧州刺史應襲,以亨為兗州,襲為豫州。亨密以聞,襲亦告司徒王弘。佩之聚黨百餘人,殺牛犒賜,條牒時人,並相署置,期明年正會,於殿中作亂。未及數日,
 收斬之。



 傅亮,字季友,北地靈州人也。祖咸,司隸校尉。父瑗,以學業知名,位至安成太守。瑗與郗超善,超嘗造瑗,瑗見其二子迪及亮。亮年四五歲,超令人解亮衣,使左右持去,初無吝色。超謂瑗曰:「卿小兒才名位宦,當遠踰於兄。然保家傳祚,終在大者。」迪字長猷,亦儒學,官至五兵尚書。永初二年卒,追贈太常。



 亮博涉經史,尤善文詞。初為建威參軍,桓謙中軍行參軍。桓玄篡位,聞其博學有文采,
 選為秘書郎,欲令整正秘閣,未及拜而玄敗。義旗初,丹陽尹孟昶以為建威參軍。義熙元年,除員外散騎侍郎,直西省,典掌詔命。轉領軍長史,以中書郎滕演代之。亮未拜,遭母憂,服闋,為劉毅撫軍記室參軍,又補領軍司馬。七年,遷散騎侍郎,復代演直西省。仍轉中書黃門侍郎,直西省如故。高祖以其久直勤勞,欲以為東陽郡,先以語迪,迪大喜告亮。亮不答,即馳見高祖曰:「伏聞恩旨,賜擬東陽,家貧忝祿,私計為幸。但憑廕之願,實結本心,
 乞歸天宇,不樂外出。」



 高祖笑曰:「謂卿之須祿耳,若能如此,甚協所望。」會西討司馬休之,以為太尉從事中郎,掌記室。以太尉參軍羊徽為中書郎,代直西省。



 亮從征關、洛,還至彭城。宋國初建,令書除侍中,領世子中庶子。徙中書令,領中庶子如故。從還壽陽。高祖有受禪意,而難於發言,乃集朝臣宴飲,從容言曰:「桓玄暴篡,鼎命已移,我首唱大義,復興皇室,南征北伐,平定四海,功成業著,遂荷九錫。今年將衰暮,崇極如此,物戒盛滿,非可久安。
 今欲奉還爵位,歸老京師。」群臣唯盛稱功德,莫曉此意。日晚坐散,亮還外,乃悟旨,而宮門已閉;亮於是叩扉請見,高祖即開門見之。亮入便曰:「臣暫宜還都。」高祖達解此意,無復他言,直云:「須幾人自送?」亮曰:「須數十人便足。」於是即便奉辭。亮既出,已夜,見長星竟天。亮拊髀曰:「我常不信天文,今始驗矣!」至都,即征高祖入輔。



 永初元年,遷太子詹事,中書令如故。以佐命功,封建城縣公,食邑二千戶。



 入直中書省,專典詔命。以亮任總國權,聽於省
 見客。神虎門外,每旦車常數百兩。



 高祖登庸之始,文筆皆是記室參軍滕演;北征廣固,悉委長史王誕;自此後至于受命,表策文誥,皆亮辭也。演字彥將,南陽西鄂人,官至黃門郎,祕書監。義熙八年卒。二年,亮轉尚書僕射,中書令、詹事如故。明年,高祖不豫,與徐羨之、謝晦並受顧命,給班劍二十人。



 少帝即位,進為中書監,尚書令。景平二年,領護軍將軍。少帝廢,亮率行臺至江陵奉迎太祖。既至,立行門於江陵城南,題曰「大司馬門。」率行臺百
 僚詣門拜表,威儀禮容甚盛。太祖將下,引見亮,哭慟甚,哀動左右。既而問義真及少帝薨廢本末,悲號嗚咽,侍側者莫能仰視。亮流汗沾背,不能答。於是布腹心於到彥之、王華等,深自結納。太祖登阼,加散騎常侍、左光祿大夫、開府儀同三司,本官悉如故。司空府文武即為左光祿府。又進爵始興郡公,食邑四千戶,固讓進封。



 元嘉三年,太祖欲誅亮,先呼入見;省內密有報之者,亮辭以嫂病篤,求暫還家。遣信報徐羨之,因乘車出郭門,騎馬
 奔兄迪墓。屯騎校尉郭泓收付廷尉,伏誅。



 時年五十三。初至廣莫門,上遣中書舍人以詔書示亮,并謂曰:「以公江陵之誠,當使諸子無恙。」初,亮見世路屯險,著論名曰《演慎》,曰:大道有言,慎終如始,則無敗事矣。《易》曰:「括囊無咎。」慎不害也。又曰:「藉之用茅,何咎之有。」慎之至也。文王小心,《大雅》詠其多福;仲由好勇,馮河貽其苦箴。《虞書》著慎身之譽,周廟銘陛坐之側。因斯以談,所以保身全德,其莫尚於慎乎!夫四道好謙,三材忌滿,祥萃虛室,鬼瞰
 高屋,豐屋有蔀家之災,鼎食無百年之貴。然而徇欲厚生者,忽而不戒;知進忘退者,曾莫之懲。前車已摧,後鑾不息,乘危以庶安,行險而徼幸,於是有顛墜覆亡之禍,殘生夭命之釁。其故何哉?流溺忘反,而以身輕於物也。



 故昔之君子,同名爵於香餌,故傾危不及;思憂患而豫防,則鍼石無用。洪流壅於涓涓,合拱挫於纖蘗,介焉是式,色斯而舉,悟高鳥以風逝,鑒醴酒而投紱。



 夫豈敝著而後謀通,患結而後思復云爾而已哉!故《詩》曰:「慎爾侯
 度,用戒不虞。」言防萌也。夫單以營內喪表,張以治外失中,齊、秦有守一之敗,偏恃無兼濟之功,冰炭滌於胸心,巖牆絕於四體。夫然,故形神偕全,表裏寧一,營魄內澄,百骸外固,邪氣不能襲,憂患不能及,然可以語至而言極矣!



 夫以嵇子之抗心希古,絕羈獨放,五難之根既拔,立生之道無累,人患殆乎盡矣。徒以忽防於鐘、呂,肆言於禹、湯,禍機發於豪端,逸翩鎩於垂舉。觀夫貽書良友,則匹厚味於甘鴆,囗囗囗囗囗囗囗囗其懼患也,若無
 轡而乘奔,其慎禍也,猶履冰而臨谷。或振褐高棲,揭竿獨往,或保約違豐,安于卑位。故漆園外楚,忌在龜犧;商洛遐遁,畏此駟馬。平仲辭邑,殷鑒於崔、慶,張臨挹滿,灼戒乎桑、霍。若君子覽茲二塗,則賢鄙之分既明,全喪之實又顯。非知之難,慎之惟難,慎也者,言行之樞管乎!



 夫據圖揮刃,愚夫弗為,臨淵登峭,莫不惴慄。何則?害交故慮篤,患切而懼深。故《詩》曰:「不敢暴虎,不敢馮河。」慎微之謂也。故庖子涉族,怵然為戒,差之一毫,弊猶如此。況乎
 觸害犯機,自投死地。禍福之具,內充外斥,陵九折於邛僰,泛衝波於呂梁,傾側成於俄頃,性命哀而莫救。嗚呼!嗚呼!故語有之曰,誠能慎之,福之根也。曰是何傷,禍之門爾。言慎而已矣。



 亮布衣儒生,僥幸際會,既居宰輔,兼總重權。少帝失德,內懷憂懼,作《感物賦》以寄意焉。其辭曰:余以暮秋之月,述職內禁,夜清務隙,遊目藝苑。于時風霜初戒,蟄類尚繁,飛蛾翔羽,翩翾滿室,赴軒幌,集明燭者,必以燋滅為度。雖則微物,矜懷者久之。



 退感莊生
 異鵲之事,與彼同迷而忘反鑒之道,此先師所以鄙智,及齊客所以難日論也。悵然有懷,感物興思,遂賦之云爾。



 在西成之暮晷,肅皇命於禁中。聆蜻蛚於前廡,鑒朗月於房櫳。風蕭瑟以陵幌,霜皚皚而被墉。憐鳴蜩之應節,惜落景之懷東。嗟勞人之萃感,何夕永而慮充。眇今古以遐念,若循環之無終。詠倚相之遺矩,希董生之方融。鉆光燈而散袠,溫聖哲之遺蹤。墳素杳以難暨,九流紛其異封。領三百於無邪,貫五千於有宗。考舊聞於前
 史,訪心跡於汙隆。豈夷阻之在運,將全喪之由躬。遊翰林之彪炳,嘉美手於良工。辭存麗而去穢,旨既雅而能通。雖源流之深浩,且揚榷而發蒙。



 習習飛蚋,飄飄纖蠅,緣幌求隙,望爓思陵。糜蘭膏而無悔,赴朗燭而未懲。



 瞻前軌之既覆,忘改轍於後乘。匪微物之足悼,悵永念而捬膺。彼人道之為貴,參二儀而比靈。稟清曠以授氣,修緣督而為經。照安危於心術,鏡纖兆於未形。有徇末而捨本,或耽欲而忘生。碎隨侯於微爵,捐所重而要輕。矧
 昆蟲之所昧,在智士其猶嬰。悟雕陵於莊氏,幾鑒濁而迷清。仰前修之懿軌,知吾跡之未并。雖宋元之外占,曷在予之克明。豈知反之徒爾,喟投翰以增情。



 初,奉迎大駕,道路賦詩三首,其一篇有悔懼之辭,曰:「夙櫂發皇邑,有人祖我舟。餞離不以幣,贈言重琳球。知止道攸貴,懷祿義所尤。四牡倦長路,君轡可以收。張邴結晨軌,疏董頓夕輈。東隅誠已謝,西景逝不留。性命安可圖,懷此作前修。敷衽銘篤誨,引帶佩嘉謀。迷寵非予志,厚德良未
 酬。撫躬愧疲朽,三省慚爵浮。重明照蓬艾,萬品同率由。忠誥豈假知,式微發直謳。」亮自知傾覆,求退無由,又作辛有、穆生、董仲道贊,稱其見微之美。



 長子演,祕書郎,先亮卒。演弟悝、湛逃亡。湛弟都,徙建安郡;世祖孝建之中,並還京師。



 檀道濟,高平金鄉人,左將軍韶少弟也。少孤,居喪備禮。奉姊事兄,以和謹致稱。高祖創義,道濟從入京城,參高祖建武軍事,轉征西。討平魯山,禽桓振,除輔國參軍、南
 陽太守。以建義勳,封吳興縣五等侯。盧循寇逆,群盜互起,郭寄生等聚作唐,以道濟為揚武將軍、天門太守討平之。又從劉道規討柏謙、荀林等,率厲文武,身先士卒,所向摧破。及徐道覆來逼,道規親出拒戰,道濟戰功居多。



 遷安遠護軍、武陵內史。復為太尉參軍,拜中書侍郎,轉寧朔將軍,參太尉軍事。



 以前後功封作唐縣男,食邑四百戶。補太尉主簿、咨議參軍。豫章公世子為征虜將軍鎮京口,道濟為司馬、臨淮太守。又為世子西中郎司
 馬、梁國內史。復為世子征虜將軍司馬,加冠軍將軍。



 義熙十二年,高祖北伐,以道濟為前鋒出淮、肥,所至諸城戍望風降服。進剋許昌,獲偽寧朔將軍、潁川太守姚坦及大將楊業。至成皋,偽兗州刺史韋華降。徑進洛陽,偽平南將軍陳留公姚洸歸順。凡拔城破壘,俘四千餘人。議者謂應悉戮以為京觀。道濟曰:「伐罪吊民,正在今日。」皆釋而遣之。於是戎夷感悅,相率歸之者甚眾。進據潼關,與諸軍共破姚紹。長安既平,以為征虜將軍、琅邪內
 史。世子當鎮江陵,復以道濟為西中郎司馬、持節、南蠻校尉。又加征虜將軍。遷宋國侍中,領世子中庶子,兗州大中正。高祖受命,轉護軍,加散騎常侍,領石頭戍事。



 聽直入殿省。以佐命功,改封永修縣公,食邑二千戶。徙為丹陽尹,護軍如故。高祖不豫,給班劍二十人。



 出監南徐兗之江北淮南諸郡軍事、鎮北將軍、南兗州刺史。景平元年,虜圍青州刺史竺夔於東陽城,夔告急。加道濟使持節、監征討諸軍事,與王仲德救東陽。



 未及至,虜燒營,
 焚攻具遁走。將追之,城內無食,乃開窖取久穀;窖深數丈,出穀作米,已經再宿;虜去已遠,不復可追,乃止。還鎮廣陵。



 徐羨之將廢廬陵王義真,以告道濟,道濟意不同,屢陳不可,不見納。羨之等謀欲廢立,諷道濟入朝;既至,以謀告之。將廢之夜,道濟入領軍府就謝晦宿。晦其夕竦動不得眠,道濟就寢便熟,晦以此服之。太祖未至,道濟入守朝堂。上即位,進號征北將軍,加散騎常侍,給鼓吹一部。進封武陵郡公,食邑四千戶。固辭進封。



 又增督
 青州、徐州之淮陽下邳琅邪東莞五郡諸軍事。



 及討謝晦,道濟率軍繼到彥之。彥之戰敗,退保隱圻,會道濟至。晦本謂道濟與羨之等同誅,忽聞來上,人情兇懼,遂不戰自潰。事平,遷都督江州之江夏豫州之西陽新蔡晉熙四郡諸軍事、征南大將軍、開府儀同三司、江州刺史,持節、常侍如故;增封千戶。



 元嘉八年,到彥之伐索虜,已平河南,尋復失之;金墉、虎牢並沒,虜逼滑臺。



 加道濟都督征討諸軍事,率眾北討。軍至東平壽張縣,值虜安平
 公乙旃眷。道濟率寧朔將軍王仲德、驍騎將軍段宏奮擊,大破之。轉戰至高梁亭,虜寧南將軍、濟州刺史壽昌公悉頰庫結前後邀戰,道濟分遣段宏及臺隊主沈虔之等奇兵擊之,即斬悉頰庫結。道濟進至濟上,連戰二十餘日,前後數十交,虜眾盛,遂陷滑臺。道濟於歷城全軍而反。進位司空,持節、常侍、都督、刺史並如故。還鎮尋陽。



 道濟立功前朝,威名甚重;左右腹心,並經百戰,諸子又有才氣,朝廷疑畏之。



 太祖寢疾累年,屢經危殆,彭城
 王義康慮宮車晏駕,道濟不可復制。十二年,上疾篤,會索虜為邊寇,召道濟入朝。既至,上間。十三年春,將遣道濟還鎮,已下船矣,會上疾動,召入祖道,收付廷尉。詔曰:「檀道濟階緣時幸,荷恩在昔,寵靈優渥,莫與為比。曾不感佩殊遇,思答萬分,乃空懷疑貳,履霜日久。元嘉以來,猜阻滋結,不義不暱之心,附下罔上之事,固已暴之民聽,彰於遐邇。謝靈運志凶辭醜,不臣顯著,納受邪說,每相容隱。又潛散金貨,招誘剽猾,逋逃必至,實繁彌廣,日
 夜伺隙,希冀非望。鎮軍將軍仲德往年入朝,屢陳此迹。朕以其位居台鉉,豫班河岳,彌縫容養,庶或能革。而長惡不悛,凶慝遂遘,因朕寢疾,規肆禍心。



 前南蠻行參軍龐延祖具悉奸狀,密以啟聞。夫君親無將,刑茲罔赦。況罪釁深重,若斯之甚。便可收付廷尉,肅正刑書。事止元惡,餘無所問。」於是收道濟及其子給事黃門侍郎植、司徒從事中郎粲、太子舍人隰、征北主簿承伯、秘書郎遵等八人,並於廷尉伏誅。又收司空參軍薛彤,付建康伏
 法。又遣尚書庫部郎顧仲文、建武將軍茅亨至尋陽,收道濟子夷、邕、演及司空參軍高進之,誅之。薛彤、進之並道濟腹心,有勇力,時以比張飛、關羽。初,道濟見收,脫幘投地曰:「乃復壞汝萬里之長城!」邕子孺乃被宥,世祖世,為奉朝請。



 史臣曰:夫彈冠出里,結組登朝,道申於夷路,運艱於險轍,是以古人裴回於出處,交戰乎臨岐。若其任重於身,恩結自主,雖復據鼎承劍,悠然不以存歿為懷。



 當二公
 受言西殿,跪承顧托,若使死而可再,固以赴蹈為期也。及逢權定之機,當震主之地,甫欲攘抑後禍,御蔽身災,使桐宮有卒迫之痛,淮王非中霧之疾。若以社稷為存亡,則義異於此。但彭城無燕剌之釁,而有楚英之戮。若使一昆延歷,亦未知定終所在也。謝晦言不以賊遺君父,豈徒言哉!



\end{pinyinscope}