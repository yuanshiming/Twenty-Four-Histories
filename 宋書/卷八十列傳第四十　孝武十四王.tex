\article{卷八十列傳第四十 孝武十四王}

\begin{pinyinscope}

 孝武帝二十八男:文穆皇后生廢帝子業、豫章王子尚,陳淑媛生晉安王子勛;阮容華生安陸王子綏;徐昭容生皇子子深;何淑儀生松滋侯子房;史昭華生臨海王
 子頊;殷貴妃生始平孝敬王子鸞;次永嘉王子仁,與皇子子深同生。何婕妤生皇子子鳳;謝昭容生始安王子真;江婕妤生皇子子玄;史昭儀生邵陵王子元;次齊敬王子羽,與始平孝敬王子鸞同生。江美人生皇子子衡;楊婕妤生淮南王子孟;次皇子子況,與皇子子玄同生。次南平王子產,與永嘉王子仁同生。次晉陵孝王子雲,次皇子子文,並與始平孝敬王子鸞同生。次廬陵王子輿,與淮南王子孟同生。次南海哀王子師,與始平孝敬
 王子鸞同生。次淮陽思王子霄,與皇子子玄同生。次皇子子雍,與始安王子真同生。次皇子子趨,與皇子子鳳同生,次皇子子期,與皇子子衡同生。次東平王子嗣,與始安王子真同生。杜容華生皇子子悅。安陸王子綏、南平王子產、廬陵王子輿並出繼。皇子子深、子鳳、子玄、子衡、子況、子文、子雍未封,早夭。子趨、子期、子悅未封,為明帝所殺。



 豫章王子尚,字孝師,孝武帝第二子也。孝建三年,年六
 歲,封西陽王,食邑二千戶。仍都督南徐、兗二州諸軍事、北中郎將、南兗州刺史。其年,遷揚州刺史。



 大明二年,加撫軍將軍。三年,分浙江西立王畿,以浙江東為揚州,命王子尚都督揚州江州之鄱陽、晉安、建安三郡諸軍事、揚州刺史,將軍如故,給鼓吹一部。



 五年,改封豫章王,戶邑如先,領會稽太守。七年,加使持節,進號車騎將軍。其年,又加散騎常侍,以本號開府儀同三司。時東土大旱,鄞縣多矰田,世祖使子尚上表至鄞縣勸農。又立左學,
 召生徒,置儒林祭酒一人,學生師敬,位比州治中;文學祭酒一人,比西曹;勸學從事二人,比祭酒從事。前廢帝即位,罷王畿復舊,徵子尚都督揚、南徐二州諸軍事,領尚書令,解督東揚州,餘如故。



 初孝建中,世祖以子尚太子母弟,上甚留心。後新安王子鸞以母幸見愛,子尚之寵稍衰。既長,人才凡劣,凶慝有廢帝風。太宗殞廢帝,稱太皇后令曰:「子尚頑凶極悖,行乖天理。楚玉淫亂縱慝,義絕人經。並可於第賜盡。」子尚時年十六。



 楚玉,山陰公
 主也。廢帝改封為會稽郡長公主,食湯沐邑二千戶,給鼓吹一部,加班劍二十人。未及拜受,而廢帝敗。楚玉肆情淫縱,以尚書吏部郎褚淵貌美,請自侍十日,廢帝許之。淵雖承旨而行,以死自固,楚玉不能制也。



 晉安王子勛,字孝德,孝武帝第三子也。大明四年,年五歲,封晉安王,食邑二千戶。仍都督南兗州、徐州之東海諸軍事、征虜將軍、南兗州刺史。七年,改督江州、南豫州之晉熙、新蔡、郢州之西陽三郡諸軍事、前將軍、江州刺
 史。八年,遷使持節、都督雍、梁、南北秦四州、郢州之竟陵、隨二郡諸軍事、鎮軍將軍、寧蠻校尉、雍州刺史。未拜而世祖崩,以鎮軍將軍還為江州,本官如故。眼患風,為世祖所不愛。景和元年,加使持節。



 時廢帝狂凶,多所誅害。前撫軍諮議參軍何邁少好武,頗招集才力之士。邁先尚太祖女新蔡公主,帝詐云主薨,殺宮人代之,顯加殯葬,而納主於後宮。深忌邁,邁慮禍及,謀因帝出行為變,迎立子勛。事泄,帝自率宿衛兵誅邁,使八座奏子勛與
 邁通謀。又手詔子勛曰:「何邁殺我立汝,汝自計孰若孝武邪?可自為其所。」



 遣左右朱景雲送藥賜子勛死。景雲至盆口,停不進,遣信使報長史鄧琬。琬等因奉子勛起兵,以廢立為名。



 太宗定亂,進子勛號車騎將軍、開府儀同三司。琬等不受命,傳檄京邑。泰始二年正月七日,奉子勛為帝,即偽位於尋陽城,年號義嘉元年,備置百官,四方並響應,威震天下。是歲四方貢計,並詣尋陽。遣左衛將軍孫沖之等下據赭圻,又遣豫州刺史劉胡率大
 眾來屯鵲尾,又遣安北將軍袁顗總統眾軍。臺軍屯據前谿斷顗等糧援,胡遣將攻之,大敗,於是焚營遁走。顗聞胡去,亦棄眾南奔。沈攸之諸軍至尋陽,誅子勛及其母,同逆皆夷滅。子勛死時,年十一,即葬尋陽廬山。



 松滋侯子房,字孝良,孝武帝第六子也。大明四年,年五歲,封尋陽王,食邑二千戶。仍為冠軍將軍、淮南、宣城二郡太守。五年,遷豫州刺史,將軍、淮南太守如故。六年,改領宣城太守。七年,進號右將軍,解宣城,餘如故。前廢帝
 永光元年,遷東揚州刺史,將軍如故。景和元年,罷東揚州,子房以本號督會稽、東陽、新安、臨海、永嘉五郡諸軍事、會稽太守。



 太宗即位,改督為都督,進號安東將軍,太守如故。又徵為撫軍,領太常。長史孔覬不受命,舉兵反,應晉安王。子勛即偽位,進子房號車騎將軍、開府儀同三司。三吳、晉陵並受命於覬。太宗遣衛將軍巴陵王休若督諸將吳喜等東討,戰無不捷,以次平定。上虞令王晏起兵殺覬,囚子房,送還京都,上宥之,貶為松滋縣侯,
 食邑千戶。



 司徒建安王休仁以子房兄弟終為禍難,勸上除之。乃下詔曰:「不虞之釁,著自終古,情為法屈,聖達是遵。朕掃穢定傾,再全寶業,遠惟鴻基,猥當負荷。思弘治道,務盡敦睦,而妖豎遘扇,妄造異圖。自西南阻兵,東夏侵斥,都邸群凶,密相脣齒。路休之兄弟,專作謀主,規興禍亂,令舍人嚴龍覘覦宮省,以羽林出討,宿衛單罄,候隙伺間,將謀竊發。劉祗在蕃,規相應援,通言北寇,引令過淮。頃休範濟江,潛欲拒捍,賴卜祚靈長,姦回弗逞。
 陰慝已露,宜盡憲辟,實以方難未夷,曲加遵養。今王化帖泰,宜辨忠邪,涓流不壅,燎火難滅。便可委之有司,肅正刑典。松滋侯子房等淪陷逆徒,協同醜悖,遂與簽帥群小,潛通南釁,連結祗等,還圖朕躬。雖咎戾已彰,在法無宥,猶子之情,良所未忍。可廢為庶人,徙付遠郡。」



 於是並殺之,房時年十一。



 路休之等以崇憲太后既崩,自慮將來不立,不自安。劉祗在南兗州,有志為逆。



 嚴龍,太祖元嘉中,已為中書舍人、南臺御史,世祖又以為舍人,
 甚見委信。景和、泰始之際,至越騎校尉,右軍將軍。至是懷異端,故及於誅。



 臨海王子頊,字孝列,孝武帝第七子也。大明四年,年五歲,封歷陽王,食邑二千戶。仍為冠軍將軍、吳興太守。五年,改封臨海王,戶邑如先。其年,遷使持節、都督廣交二州、湘州之始興、始安、臨賀三郡諸軍事、征虜將軍、平越中郎將、廣州刺史。未之鎮,徙荊州刺史,將軍如故。八年,進號前將軍。



 前廢帝即位,以本號都督荊、湘、雍、益、梁、寧、
 南北秦八州諸軍事,刺史如故。明帝即位,解督雍州,以為鎮軍將軍、丹陽尹。尋留本任,進督雍州,又進號平西將軍。長史孔道存不受命,舉兵反,以應晉安王子勛。子勛即偽位,進號衛將軍、開府儀同三司。鵲尾奔敗,吳喜、張興世等軍至,子頊賜死,時年十一。葬巴陵。



 始平孝敬王子鸞,字孝羽,孝武帝第八子也。大明四年,年五歲,封襄陽王,食邑二千戶。仍為東中郎將、吳郡太守。其年,改封新安王,戶邑如先。五年,遷北中郎將、南徐
 州刺史,領南琅邪太守。母殷淑儀,寵傾後宮,子鸞愛冠諸子,凡為上所盼遇者,莫不入子鸞之府、國。及為南徐州,又割吳郡以屬之。



 六年,丁母憂。追進淑儀為貴妃,班亞皇后,謚曰宣。葬給轀輬車,虎賁、班劍,鑾輅九旒,黃屋左纛,前後部羽葆、鼓吹。上自臨南掖門,臨過喪車,悲不自勝,左右莫不感動。上痛愛不已,擬漢武《李夫人賦》,其詞曰:朕以亡事棄日,閱覽前王詞苑,見《李夫人賦》,悽其有懷,亦以嗟詠久之,因感而會焉。巡靈周之殘冊,略鴻
 漢之遺篆。弔新宮之奄映,喭璧臺之蕪踐。賦流波之謠思,詔河濟以崇典。雖媛德之有載,竟滯悲其何遣。訪物運之榮落,訊雲霞之舒卷。念桂枝之秋霣,惜瑤華之春翦。桂枝折兮沿歲傾,瑤華碎兮思聯情。彤殿閉兮素塵積,翠所蕪兮紫苔生。寶羅曷兮春幌垂,珍簟空兮夏幬扃。秋臺惻兮碧煙凝,冬宮冽兮朱火清。流律有終,深心無歇。徙倚雲日,裴回風月。思玉步於鳳墀,想金聲於鸞闕。竭方池而飛傷,損園淵而流咽。端蚤朝之晨罷,泛輦
 路之晚清。



 轥南陸,蹕閶闔,轢北津,警承明。面縞館之酸素,造松帳之蔥青。俯眾胤而慟興,撫藐女而悲生。雖哀終其已切,將何慰於爾靈。存飛榮於景路,沒申藻於服車。垂葆旒於昭術,竦鸞劍於清都。朝有儷於征準,禮無替於粹圖。閟瑤光之密陛,宮虛梁之餘陰。俟玉羊之晨照,正金雞之夕臨。升雲鼛以引思,鏘鴻鐘以節音。文七星於霜野,旗二耀於寒林。中雲枝之夭秀,寓坎泉之曾岑。屈封嬴之自古,申反周乎在今。遣雙靈兮達孝思,附
 孤魂兮展慈心。伊鞠報之必至,諒顯晦之同深。予棄西楚之齊化,略東門之遙衣金。淪漣兩拍之傷,奄抑七萃之箴。



 又諷有司曰:「典禮云,天子有后,有夫人。《檀弓》云,舜葬蒼梧,二妃不從。《昏義》云,后立六宮,有三夫人。然則三妃則三夫人也。后之有三妃,猶天子之有三公也。按《周禮》,三公八命,諸侯七命。三公既尊於列國諸侯,三妃亦貴於庶邦夫人。據《春秋傳》,仲子非魯惠公之元嫡,尚得考彼別宮;今貴妃蓋天秩之崇班,理應創立新廟。」尚書
 左丞徐爰之又議:「宣貴妃既加殊命,禮絕五宮,考之古典,顯有成據。廟堂克構,宜選將作大匠卿。」



 葬畢,詔子鸞攝職,以本官兼司徒,進號撫軍、司徒,給鼓吹一部,禮儀並依正公。又加都督南徐州諸軍事。八年,加中書令,領司徒。前廢帝即位,解中書令,領司徒,加持節之鎮。帝素疾子鸞有寵,既誅群公,乃遣使賜死,時年十歲。子鸞臨死,謂左右曰:「願身不復生王家。」同生弟妹並死,仍葬京口。



 太宗即位,詔曰:「夫紓冤申痛,雖往必追,緣情惻愛,感
 事彌遠。故使持節、都督南徐州諸軍事、撫軍將軍、南徐州刺史新安王子鸞,夙表成器,蚤延殊寵,方樹美業,克光蕃維。而凶心肆忌,奄羅橫禍,興言永傷,有兼常懷,宜旍夭秀,以雪沈魂。可贈使持節、侍中、都督南徐、兗二州諸軍事、司徒、南徐州刺史,王如故。第十二皇女、第二皇子子師,俱嬰謬酷,有增酸悼。皇女可贈縣公主,子師復先封為南海王,並加徽謚。」又曰:「哀枉追遠,仁道所弘,興滅繼絕,盛典斯貴。



 朕務古思治,恩禮必敷,異族猶敦,況
 在近戚。故新除使持節、侍中、都督南徐、兗二州諸軍事、司徒、南徐州刺史新安王子鸞,年雖沖弱,性識早茂,鐘慈世祖,冠寵列蕃。值景和凶虐,橫羅酷禍,國胤無主,冤祀莫寄,尋念痛悼,夙軫于懷。



 可以建平王景素息延年為嗣。」追改子鸞封為始平王,食邑千戶,改葬秣陵縣龍山。



 延年,字德沖,泰始四年薨,時年四歲,謚曰沖王。明年,復以長沙王纂子延之為始平王,紹子鸞後。順帝昇明三年薨,國除。



 永嘉王子仁,字孝和,孝武帝第九子也。大明五年,年五歲,監雍、梁、南北秦四州、郢州之竟陵、隨二郡諸軍事、北中郎將、寧蠻校尉、雍州刺史,封永嘉王,食邑二千戶。仍遷東中郎將、吳郡太守。六年,又遷丹陽尹。七年,兼衛尉。前廢帝即位,加征虜將軍,領衛尉,丹陽尹如故。尋出為左將軍、南兗州刺史。景和元年,遷南徐州刺史,將軍如故。泰始元年,又遷中軍將軍,領太常。未拜,徙護軍將軍。四方平定,以為使持節、都督湘、廣、交三州諸軍事、平南
 將軍、湘州刺史。



 太宗遣主書趙扶公宣旨於子仁曰:「汝一家門戶不建,幾覆社稷。天未亡宋,景命集我。上流迷愚相扇,四海同惡,若非我修德御天下,三祖基業,一朝墜地,汝輩便應淪於異族之手。我昔兄弟近二十人,零落相繼,存者無幾。唯司徒年長,令德作輔,皇家門戶所憑,唯我與司徒二人而已,尚未能厭百姓姦心,餘諸王亦未堪贊治。我惟有太子一人,司徒世子,年又幼弱,桂陽、巴陵並未有繼體,正賴汝輩兄弟,相倚為彊,庶使天
 下不敢窺覘王室。汝輩始十餘歲,裁知俯仰,當今諸舍細弱,殆不免人輕陵。若非我為主,劉氏不辦今日。汝諸兄弟沖眇,為群凶所逼誤,遂與百姓還圖骨肉,於汝在心,不得無媿。即日四海就寧,恩化方始,方今處汝湘州。汝年漸長,足知善惡,當每思刻厲,奉朝廷為心,爵秩自然與年俱進。我垂猶子之情,著於萬物;汝亦當知好,憶我敕旨。」時司徒建安王休仁南討猶未還,既還,白上,以將來非社稷計,宜並為之所。未拜,賜死,時年十歲。



 始安王子真,字孝貞,孝武帝第十一子。大明五年,年五歲,封始安王,食邑二千戶。仍為輔國將軍、吳興太守。七年,遷使持節、監廣交二州始興、始安、臨賀三郡諸軍事、平越中郎將、廣州刺史,將軍如故,不之鎮。遷征虜將軍、南彭城太守,領石頭戍事。景和元年,為丹陽尹,將軍如故。尋復為南兗州刺史,將軍如故。泰始二年,遷左將軍、丹陽尹。未拜,賜死,時年十歲。



 邵陵王子元,字孝善,孝武帝第十三子也。大明六年,年
 五歲,封邵陵王,食邑二千戶。八年,以為度支校尉、秦、南沛二郡太守。仍為冠軍將軍、南琅邪、泰山二郡太守。景和元年,出為湘州刺史,將軍如故,未之鎮。至尋陽,值晉安王子勛為逆,留不之鎮。進號撫軍將軍。事平,賜死,時年九歲。



 齊敬王子羽,字孝英,孝武帝第十四子也。大明二年生,三年卒,追加封謚。



 淮南王子孟,字孝光,孝武帝第十六子也。大明七年,年
 五歲,封淮南王,食邑二千戶。時世祖改豫州之南梁郡為淮南國,罷南豫州之淮南郡并宣城。前廢帝即位,二郡並復舊,子孟仍國名度食淮南郡。景和元年,為冠軍將軍、南琅邪、彭城二郡太守。泰始二年,改封安成王,戶邑如先。未拜,賜死,時年八歲。



 晉陵孝王子雲,字孝舉,孝武帝第十九子也。大明六年,年四歲,封晉陵王,食邑二千戶。未拜,其年薨。



 南海哀王子師,字孝友,孝武帝第二十二子也。大明七
 年,年四歲,封南海王,食邑二千戶。未拜,景和元年,為前廢帝所害,時年六歲。太宗即位,追謚。



 淮陽思王子霄,字孝雲,孝武帝第二十三子也。大明五年生,八年薨,追加封謚。



 東平王子嗣,字孝叔,孝武帝第二十七子也。大明七年生,仍封東平王,食邑二千戶。繼東平沖王休倩。休倩母顏性理嚴酷,泰始二年,子嗣所生母景寧園昭容謝上表曰:「故東平沖王休倩託荄璇極,岐嶷夙表,降年弗永,
 遺胤莫傳。孝武皇帝敕妾子臣子嗣出繼為後,既承國祀,方奉烝薦,庶覃遐慶,式延于遠。而妾顏訓養非恩,撫導乖理,情闕引進,義違負螟。昔世祖平日,詭申慈愛;崩背未幾,真性便發,猶逼畏崇憲,少欲藏掩。自茲以後,專縱嚴酷,實顯布宗戚,宣灼宮闈,用傷人倫,爰惻行路。妾天屬冥至,感切實深,伏願乾渥廣臨,曲垂照賜,復改命還依本屬,則妾母子雖隕之辰,猶生之年。」許之。其年賜死,時年四歲。



 武陵王贊,字仲敷,明帝第九子也。泰始六年生。其年,詔曰:「世祖孝武皇帝雖恃尊墮惠,勳狹政弛,樂飲無饜,事因於寧泰,任威縱費,義緣於務寡。故以積怨動天,流殃胤嗣,景和肇釁,義嘉成禍,世祖繼體,陷憲無遺。昔皇家中圮,含生懼滅,賴英孝感奮,掃雪冤恥,勳纘墜歷,拯茲窮氓。繼絕追遠,禮訓攸尚,況既帝且兄,而缺斯典。今以第九子智隨奉世祖為子,武陵郡大明之世,事均代邦,可封智隨武陵王,食邑五千戶。尋世祖一門女累不少,
 既無釐總,義須防閑,諸侯雖不得祖稱天子,而事有一家之切。且歸寧有所,疹疾相營,得失是任,閨房有稟。



 朕應天在位,恩深九族,庶此足申追睦之懷,敷愛之旨。」



 後廢帝元徽四年,出為使持節、督南徐、兗、青、冀五州諸軍事、北中郎將、南徐州刺史。順帝升明元年,遷持節、督郢州、司州之義陽諸軍事、前將軍、郢州刺史。二年,為沈攸之所圍,徙都督荊、湘、雍、益、梁、寧、南北秦八州諸軍事、安西將軍、荊州刺史,持節如故。攸之平,乃之鎮。其年薨,時
 年九歲,國除。



 史臣曰:晉安諸王,提挈群下,以成其釁亂,遂至九域沸騰,難結天下,而世祖之胤亦殲焉。強不如弱,義在於此也。



\end{pinyinscope}