\article{卷十四志第四 禮一}

\begin{pinyinscope}

 夫有國有家者,禮儀之用尚矣。然
 而
 歷代損益,每有不同,非務相改,隨時之宜故也。漢文以人情季薄,國喪革三年之紀;光武以中興崇儉,七廟有共堂之制;魏祖以
 侈惑宜矯,終斂去襲稱之數;晉武以丘郊不異,二至并南北之祀。互相即襲,以訖于今,豈三代之典不存哉,取其應時之變而已。且閔子譏古禮,退而致事;叔孫創漢制,化流後昆。由此言之,任己而不師古,秦氏以之致亡;師古而不適用,王莽所以身滅。然則漢、魏以來,各揆古今之中,以通一代之儀。司馬彪集後漢眾注,以為《禮儀志》,校其行事,已與前漢頗不同矣。況三國鼎峙,歷晉至宋,時代移改,各隨事立。自漢末剝亂,舊章乖弛,魏初則
 王粲、衛覬典定眾儀;蜀朝則孟光、許慈創理制度;晉始則荀鳷、鄭沖詳定晉禮;江左則荀崧、刁協緝理乖紊。



 其間名儒通學,諸所論敘,往往新出,非可悉載。今抄魏氏以後經國誕章,以備此志云。



 魏文帝雖受禪於漢,而以夏數為得天,故黃初元年詔曰:「孔子稱『行夏之時,乘殷之輅,服周之冕,樂則《韶舞》。』此聖人集群代之美事,為後王制法也。



 《傳》曰『夏數為得天』。朕承唐、虞之美,至於正朔,當依虞、夏故事。若殊徽號,異
 器械,制禮樂,易服色,用牲幣,自當隨土德之數。每四時之季月,服黃十八日,臘以丑,牲用白,其飾節旄,自當赤,但節幡黃耳。其餘郊祀天地朝會四時之服,宜如漢制。宗廟所服,一如《周禮》。」尚書令桓階等奏:「據三正周復之義,國家承漢氏人正之後,當受之以地正,犧牲宜用白,今從漢十三月正,則犧牲不得獨改。今新建皇統,宜稽古典先代,以從天命,而告朔犧牲,壹皆不改,非所以明革命之義也。」詔曰:「服色如所奏。其餘宜如虞承唐,但臘
 日用丑耳,此亦聖人之制也。」



 明帝即位,便有改正朔之意,朝議多異同,故持疑不決。久乃下詔曰:「黃初以來,諸儒共論正朔,或以改之為宜,或以不改為是,意取駮異,于今未決。朕在東宮時聞之,意常以為夫子作《春秋》,通三統,為後王法。正朔各從色,不同因襲。自五帝、三王以下,或父子相繼,同體異德;或納大麓,受終文祖;或尋干戈,從天行誅。雖遭遇異時,步驟不同,然未有不改正朔,用服色,表明文物,以章受命之符也。由此言之,何必以
 不改為是邪!」



 於是公卿以下博議。侍中高堂隆議曰:「按自古有文章以來,帝王之興,受禪之與干戈,皆改正朔,所以明天道,定民心也。《易》曰:『《革》,元亨利貞。』『有孚改命吉。』『湯武革命,應乎天,從乎人。』其義曰,水火更用事,猶王者必改正朔易服色也。《易通卦驗》曰:『王者必改正朔,易服色,以應天地三氣三色。』《書》曰:『若稽古帝舜曰重華,建皇授政改朔。』初『高陽氏以十一月為正,薦玉以赤繒。高辛氏以十三月為正,薦玉以白繒。』《尚書傳》曰:『舜定鐘石,論
 人聲,乃及鳥獸,咸變於前。故更四時,改堯正。』《詩》曰:『一之日觱發,二之日栗烈,三之日于耜。』《傳》曰:『一之日,周正月,二之日,殷正月,三之日,夏正月。』《詩推度災》曰:『如有繼周而王者,雖百世可知。以前檢後,文質相因,法度相改。三而復者,正色也,二而復者,文質也。』以前檢後,謂軒轅、高辛、夏后氏、漢,皆以十三月為正;少昊、有唐、有殷,皆以十二月為正;高陽、有虞、有周,皆以十一月為正。後雖百世,皆以前代三而復也。《禮大傳》曰:『聖人南面而治天下,必
 正度量,考文章,改正朔,易服色,殊徽號。』《樂稽曜嘉》曰:『禹將受位,天意大變,迅風雷雨,以明將去虞而適夏也。是以舜禹雖繼平受禪,猶制禮樂,改正朔,以應天從民。夏以十三月為正,法物之始,其色尚黑。殷以十二月為正,法物之牙,其色尚白。周以十一月為正,法物之萌,其色尚赤。能察其類,能正其本,則嶽瀆致雲雨,四時和,五稼成,麟皇翔集。』《春秋》『十七年夏六月甲子朔,日有蝕之。』《傳》曰:『當夏四月,是謂孟夏。』《春秋元命苞》曰:『王者受命,昭然
 明於天地之理,故必移居處,更稱號,改正朔,易服色,以明天命聖人之寶,質文再而改,窮則相承,周則復始,正朔改則天命顯。』凡典籍所記,不盡於此,略舉大較,亦足以明也。」太尉司馬懿、尚書僕射衛臻、尚書薛悌、中書監劉放、中書侍郎刁幹、博士秦靜、趙怡、中候中詔季岐以為宜改;侍中繆襲、散騎常侍王肅、尚書郎魏衡、太子舍人黃史嗣以為不宜改。



 青龍五年,山茌縣言黃龍見。帝乃詔三公曰:昔在庖犧,繼天而王,始據木德,為群代首。自
 茲以降,服物氏號,開元著統者,既膺受命歷數之期,握皇靈遷興之運,承天改物,序其綱紀。雖炎、黃、少昊、顓頊、高辛、唐、虞、夏后,世系相襲,同氣共祖,猶豫昭顯所受之運,著明天人去就之符,無不革易制度,更定禮樂,延群后,班瑞信,使之煥炳可述于後也。至于正朔之事,當明示變改,以彰異代,曷疑其不然哉!



 文皇帝踐阼之初,庶事草創,遂襲漢正,不革其統。朕在東宮,及臻在位,每覽書籍之林,總公卿之議。夫言三統相變者,有明文;云虞、
 夏相因者,無其言也。



 《歷志》曰:「天統之正在子,物萌而赤;地統之正在丑,物化而白;人統之正在寅,物成而黑。」但含生氣,以微成著。故太極運三辰五星於上,元氣轉三統五行於下,登降周旋,終則又始,言天地與人所以相通也。仲尼以大聖之才,祖述堯、舜,範章文、武,制作《春秋》,論究人事,以貫百王之則。故於三微之月,每月稱王,以明三正迭相為首。夫祖述堯、舜,以論三正,則其明義,豈使近在殷、周而已乎!朕以眇身,繼承洪緒,既不能紹上
 聖之遺風,揚先帝之休德,又使王教之弛者不張,帝典之闕者未補,亹亹之德不著,亦惡可已乎!



 今推三統之次,魏得地統,當以建丑之月為正。考之群藝,厥義彰矣。改青龍五年春三月為景初元年孟夏四月。服色尚黃,犧牲用白,戎事乘黑首之白馬,建大赤之旗,朝會建大白之旗。春夏秋冬孟仲季月,雖與正歲不同,至於郊祀迎氣,礿、祀、烝、嘗、巡獰、搜田,分至啟閉,班宣時令,中氣晚早,敬授民事,諸若此者,皆以正歲斗建為節。此歷數之序,乃
 上與先聖合符同契,重規疊矩者也。今遵其義,庶可以顯祖考大造之基,崇有魏維新之命。於戲!王公群后,百辟卿士,靖康厥職,帥意無怠,以永天休。司徒露布,咸使聞知,稱朕意焉。



 案服色尚黃,據土行也。犧牲旂旗,一用殷禮,行殷之時故也。《周禮》巾車職,「建大赤以朝」,「大白以即戎」,此則周以正色之旗朝,以先代之旗即戎。



 魏用殷禮,變周之制,故建大白朝,大赤即戎也。明帝又詔曰:「以建寅之月為正者,其牲用玄;以建丑之月為正者,其牲
 用白;以建子之月為正者,其牲用騂。此為牲色各從其正,不隨所祀之陰陽也。祭天不嫌於用玄,則祭地不得獨疑於用白也。



 天地用牲,得無不宜異邪?更議。」於是議者各有引據,無適可從。又詔曰:「諸議所依據各參錯,若陽祀用騂,陰祀用黝,復云祭天用玄,祭地用黃,如此,用牲之義,未為通也。天地至尊,用牲當同以所尚之色,不得專以陰陽為別也。今祭皇皇帝天、皇皇后地、天地郊、明堂、宗廟,皆宜同。其別祭五郊,各隨方色,祭日月星辰
 之類用騂,社稷山川之屬用玄,此則尊卑方色,陰陽眾義暢矣。」



 三年正月,帝崩,齊王即位。是年十二月,尚書盧毓奏:「烈祖明皇帝以今年正日棄離萬國。《禮》,忌日不樂,甲乙之謂也。烈祖明皇帝建丑之月棄天下,臣妾之情,於此正日,有甚甲乙。今若以建丑正朝四方,會群臣,設盛樂,不合於禮。」



 博士樂祥議:「正日旦受朝貢,群臣奉贄;後五日,乃大宴會作樂。」太尉屬硃誕議:「今因宜改之際,還修舊則,元首建寅,於制為便。」大將軍屬劉肇議:「宜過
 正一日乃朝賀大會,明令天下,知崩亡之日不朝也。」詔曰:「省奏事,五內斷絕,奈何奈何!烈祖明皇帝以正日棄天下,每與皇太后念此日至,心有剝裂。不可以此日朝群辟,受慶賀也。月二日會,又非故也。聽當還夏正月。雖違先帝通三統之義,斯亦子孫哀慘永懷。又夏正朔得天數者,其以建寅之月為歲首。」



 晉武帝泰始二年九月,群公奏:「唐堯、舜、禹不以易祚改制;至於湯、武,各推行數。宣尼答為邦之問,則曰行夏之
 時,輅冕之制,通為百代之言。蓋期於從政濟治,不繫於行運也。今大晉繼三皇之蹤,踵舜、禹之跡,應天從民,受禪有魏,宜一用前代正朔服色,皆如有虞遵唐故事,於義為弘。」奏可。孫盛曰:「仍舊,非也。且晉為金行,服色尚赤,考之天道,其違甚矣。」及宋受禪,亦如魏、晉故事。



 魏明帝初,司空王朗議:「古者有年數,無年號,漢初猶然。或有世而改,有中元、後元。元改彌數,中、後之號不足,故更假取美名,非古也。述春秋之事,曰隱公元年,則簡而
 易知。載漢世之事,曰建元元年,則後不見。宜若古稱元而已。」



 明帝不從。乃詔曰:「先帝即位之元,則有延康之號,受禪之初,亦有黃初之稱。


今名年可也。」於是尚書奏:「《易》曰:『乾道變化,各正性命。保合大和,乃利貞。首出庶物,萬國咸寧。』宜為太和元年。」詔
 \gezhu{
  缺七字}



 周之五禮,其五為嘉。嘉□□《春秋左氏傳》曰:「晉侯問襄公年,季武子對曰:『會于沙隨之歲,寡君以生。』晉侯曰:『十二年矣,是謂一終,一星終也。



 國君十五而生子。冠而生子,禮也。君
 可以冠矣。大夫盍為冠具。』武子對曰:『君冠必以祼享之禮行之,以金石之樂節之,以先君之祧處之。今君在行,未可具也。請及兄弟之國而假備焉。』晉侯許諾。還及衛,冠于成公之廟,假鐘磬焉,禮也。」賈、服說皆以為人君禮十二而冠也。《古尚書》說武王崩,成王年十三。推武王以庚辰歲崩,周公以壬午歲出居東,以癸未歲反。《禮》周公冠成王,命史祝辭。辭,告也;是除喪冠也。周公居東未反,成王冠弁以開金滕之書,時十六矣。



 是成王年十五服
 除,周公冠之而後出也。按《禮》、《傳》之文,則天子諸侯近十二,遠十五,必冠矣。《周禮》雖有服冕之數,而無天子冠文。《儀禮》云:「公侯之有冠禮,夏之末造。」王、鄭皆以為夏末上下相亂,篡弒由生,故作公侯冠禮,則明無天子冠禮之審也。大夫又無冠禮。古者五十而後爵,何大夫冠禮之有?周人年五十而有賢才,則試以大夫之事,猶行士禮也。故筮日筮賓,冠於阼以著代,醮於客位,三加彌尊,皆士禮耳。然漢氏以來,天子諸侯,頗采其議。《志》曰「儀從《冠
 禮》」是也。漢順帝冠,又兼用曹褒新禮;褒新禮今不存。《禮儀志》又云:「乘輿初加緇布進賢,次爵弁、武弁,次通天,皆於高廟。王公以下,初加進賢而已。」按此文始冠緇布,從古制也,冠於宗廟是也。魏天子冠一加,其說曰,士禮三加,加有成也。至於天子諸侯,無加數之文者,將以踐阼臨民,尊極德備,豈得復與士同?此言非也。夫以聖人之才,猶三十而立,況十二之年,未及志學,便謂德成,無所勸勉,非理實也。魏氏太子再加,皇子、王公世子乃三加。
 孫毓以為一加再加皆非也。《禮》詞曰「令月吉日」,又「以歲之正,以月之令」。魯襄公冠以冬,漢惠帝冠以三月,明無定月也。後漢以來,帝加元服,咸以正月。晉咸寧二年秋閏九月,遣使冠汝南王柬,此則晉禮亦有非必歲首也。《禮》冠於廟,魏以來不復在廟。然晉武、惠冠太子,皆即廟見,斯亦擬在廟之儀也。晉穆帝、孝武將冠,先以幣告廟,訖又廟見也。



 晉惠帝之為太子將冠也,武帝臨軒,使兼司徒高陽王
 珪加冠,兼光祿勳、屯騎校尉華暠贊冠。江左諸帝將冠,金石宿設,百僚陪位。又豫於殿上鋪大床。御府令奉冕幘簪導袞服,以授侍中、常侍。太尉加幘,太保加冕。將加冕,太尉跪讀祝文曰:「令月吉日,始加元服。皇帝穆穆,思弘袞職。欽若昊天,六合是式。率遵祖考,永永無極。眉壽惟祺,介茲景福。」加冕訖,侍中系玄紞。侍中脫絳紗服,加袞服。冠事畢,太保率群臣奉觴上壽,王公以下三稱萬歲,乃退。按儀注,一加幘冕而已。
 宋冠皇太子及蕃王,亦一加也。官有其注。晉武帝泰始十年,南宮王承年十五,依舊應冠。有司議奏:「禮十五成童。國君十五而生子,以明可冠之宜。又漢、魏遣使冠諸王,非古典。」於是制諸王十五冠,不復加命。元嘉十一年,營道侯將冠。詔曰:「營道侯義綦可克日冠。外詳舊施行。」何楨《冠儀約制》及王堪私撰《冠儀》,亦皆家人之可遵用者也。魏齊王正始四年,立皇后甄氏,其儀不存。



 晉武帝咸寧
 二年,臨軒,遣太尉賈充策立后楊氏,納悼后也。因大赦,賜王公以下各有差。百僚上禮。太康八年,有司奏:「昏禮納徵,大昏用玄纁,束帛加珪,馬二駟;王侯玄纁,束帛加璧,乘馬;大夫用玄纁,束帛加羊。古者以皮馬為庭實,天子加穀珪,諸侯加大璋。可依《周禮》改璧用璋,其羊、鴈、酒、米、玄纁如故。



 諸侯昏禮加納采吉期迎各帛五匹,及納徵馬四匹,皆令夫家自備,唯璋官為具之。」



 尚書朱整議:「按魏氏故事,王娶妃、公主嫁之禮,天子諸侯以皮馬為
 庭實,天子加以穀珪,諸侯加以大璋。漢高后制,聘后黃金二百斤,馬十二匹;夫人金五十斤,馬四匹。魏聘后、王娶妃、公主嫁之禮,用絹百九十匹。晉興,故事用絹三百匹。」



 詔曰:「公主嫁由夫氏,不宜皆為備物,賜錢使足而已。唯給璋,餘如故事。」



 成帝咸康二年,臨軒,遣使兼太保領軍將軍諸葛恢、兼太尉護軍將軍孔愉六禮備物,拜皇后杜氏,即日入宮。帝御太極殿,群臣畢賀,非禮也。王者昏禮,禮無其制。《春秋》祭公逆王后于《紀》。《穀梁》、《左氏》說與《
 公羊》又不同,而漢、魏遺事闕略者眾。晉武、惠納后,江左又無復儀注,故成帝將納杜后,太常華恒始與博士參定其儀。據杜預《左氏傳》說主婚,是供其婚禮之幣而已。又周靈王求婚於齊,齊侯問於晏桓子,桓子對曰:「夫婦所生若而人,姑姊妹則稱先守某公之遺女若而人。」此則天子之命,自得下達,臣下之答,徑自上通。先儒以為丘明詳錄其事,蓋為王者婚娶之禮也。故成帝臨軒遣使稱制拜后。然其儀注,又不具存。



 康帝建元元年,納后
 褚氏。而儀注陛者不設旄頭。殿中御史奏:「今迎皇后,依昔成恭皇后入宮御物,而儀注至尊袞冕升殿,旄頭不設,求量處。又案昔迎恭皇后,唯作青龍旂,其餘皆即御物。今當臨軒遣使,而立五牛旂旗,旄頭畢罕並出。



 即用舊制,今闕。」詔曰:「所以正法服升太極者,以敬其始,故備其禮也。今云何更闕所重而撤法物邪?又恭后神主入廟,先帝詔后禮宜有降,不宜建五牛旗,而今猶復設之邪?既不設五牛旗,則旄頭畢罕之器易具也。」又詔曰:「舊制
 既難準,且於今而備,亦非宜。府庫之儲,唯當以供軍國之費耳。法服儀飾粗令舉,其餘兼副雜器,停之。」及至穆帝升平元年,將納皇后何氏,太常王彪之始更大引經傳及諸故事,以正其禮,深非公羊婚禮不稱主人之義。又曰:「王者之於四海,無非臣妾。雖復父兄之親,師友之賢,皆純臣也。夫崇三綱之始,以定乾坤之儀,安有天父之尊,而稱臣下之命,以納伉儷;安有臣下之卑,而稱天父之名,以行大禮。遠尋古禮,無王者此制;近求史籍,無
 王者此比。於情不安,於義不通。案咸寧二年,納悼皇后時,弘訓太后母臨天下,而無命戚屬之臣為武皇父兄主婚之文。又考大晉已行之事,咸寧故事,不稱父兄師友,則咸康華恒所上合於舊也。臣愚謂今納后儀制,宜一依咸康故事。」於是從之。



 華恆所定六禮,云宜依漢舊及大晉已行之制,此恒猶識前事,故王彪之多從咸康,由此也。惟以取婦之家,三日不舉樂,而咸康群臣賀為失禮;故但依咸寧上禮,不復賀也。其告廟六禮版文等
 儀,皆彪之所定也。詳推有典制,其納采版文璽書曰:「皇帝咨前太尉參軍何琦,渾元資始,肇經人倫,爰及夫婦,以奉天地宗廟社稷,謀于公卿,咸以為宜率由舊典。今使使持節太常彪之、宗正綜以禮納采。」主人曰:「皇帝嘉命,訪婚陋族,備數采擇。臣從祖弟故散騎侍郎準之遺女,未閑教訓,衣履若而人,欽承舊章,肅奉典制。前太尉參軍都鄉侯糞土臣何琦稽首再拜承制詔。」



 次問名版文曰:「皇帝曰,咨某官某姓,兩儀配合,承天統物,正位于內,
 必俟令族,重章舊典。今使使持節太常某、宗正某,以禮問名。」主人曰:「皇帝嘉命,使者某到,重宣中詔,問臣名族。臣族女父母所生先臣故光祿大夫雩婁侯楨之遺玄孫,先臣故豫州刺史關中侯惲之曾孫,先臣安豐太守關中侯睿之孫,先臣故散騎侍郎準之遺女。外出自先臣故尚書左丞胄之外曾孫,先臣故侍中關內侯夷之外孫女,年十七。欽承舊章,肅奉典制。」次納吉版文曰:「皇帝曰,咨某官某姓,人謀龜從,僉曰貞吉,敬從典禮。今使
 持節太常某、宗正某,以禮納吉。」主人曰:「皇帝嘉命,使者某重宣中詔,太卜元吉。臣陋族卑鄙,憂懼不堪。欽承舊章,肅奉典制。」次納徵版文:「皇帝曰,咨某官某姓之女,有母儀之德,窈窕之姿,如山如河,宜奉宗廟,永承天祚。以玄絺皮帛馬羊錢璧,以章典禮。今使使持節司徒某、太常某,以禮納徵。」主人曰:「皇帝嘉命,降婚卑陋,崇以上公,寵以典禮,備物典策。欽承舊章,肅奉典制。」次請期版文:「皇帝曰,咨某官某姓,謀于公卿,大筮元龜,罔有不臧,率
 遵典禮。今使使持節太常某、宗正某,以禮請期。」主人曰:「皇帝嘉命,使某重宣中詔,吉日惟某可迎。臣欽承舊章,肅奉典制。」次親迎版文:「皇帝曰,咨某官某姓,歲吉月令,吉日惟某,率禮以迎。今使使持節太保某、太尉某以迎。」主人曰:「皇帝嘉命,使者某重宣中詔。令月吉辰,備禮以迎。上公宗卿,兼至副介,近臣百兩,臣蝝蟻之族,猥承大禮,憂懼戰悸。欽承舊章,肅奉典制。」其稽首承詔皆如初答。



 孝武納王皇后,其禮亦如之。其納採、問名、納吉、請期、親迎,
 皆用白鴈白羊各一頭,酒米各十二斛。唯納徵羊一頭,玄絺用帛三匹,絳二匹,絹二百匹,虎皮二枚,錢二百萬,玉璧一枚,馬六頭,酒米各十二斛,鄭玄所謂五鴈六禮也。其珪馬之制,備物之數,校太康所奏,又有不同,官有其注。古者昏、冠皆有醮,鄭氏醮文三首具存。



 宋文帝元嘉十五年四月,皇太子納妃,六禮文與納后不異。百官上禮。其月壬戌,於太極殿西堂敘宴二宮隊主副、司徒征北鎮南三府佐、揚兗江三州綱、彭城江夏南譙始興
 武陵廬陵南豐七國侍郎以上,諸二千石在都邑者,並豫會。又詔今小會可停妓樂,時有臨川曹太妃服。



 明帝泰始五年十一月,有司奏:「按晉江左以來,太子昏,納徵,禮用玉一,虎皮二,未詳何所準況。或者虎取其威猛有彬炳,玉以象德而有溫潤。慄珪璋既玉之美者,豹皮義兼炳蔚,熊羆亦昏禮吉徵,以類取象,亦宜並用,未詳何以遺文。



 晉氏江左,禮物多闕,後代因襲,未遑研考。今法章徽儀,方將大備。宜憲範經籍,稽諸舊典。今皇太子昏,納
 徵,禮合用珪璋豹皮熊羆皮與不?下禮官詳依經記更正。



 若應用者,為各用一?為應用兩?」博士裴昭明議:「案《周禮》,納徵,玄纁束帛儷皮。鄭玄注云:束帛,以儀注,以虎皮二。太元中,公主納徵,以虎豹皮各一具。豈謂婚禮不辨王公之序,故取虎豹皮以尊革其事乎?虎豹雖文,而徵禮所不用。



 熊羆吉祥,而婚典所不及。珪璋雖美,或為用各異。今帝道弘明,徽則光闡,儲皇聘納,宜準經誥。凡諸僻謬,並合詳裁。雖禮代不同,文質或異,而鄭為儒宗,既
 有明說,守文淺見,蓋有惟疑。兼太常丞孫詵議以為:『聘幣之典,損益惟義,歷代行事,取制士婚。若珪璋之用,實均璧品,采豹之彰,義齊虎文,熊羆表祥,繁衍攸寄。今儲后崇聘,禮先訓遠,皮玉之美,宜盡暉備。《禮》稱束帛儷皮,則珪璋數合同璧,熊羆文豹,各應用二。』長兼國子博士虞龢議:『案《儀》《禮》納徵,直云玄絺束帛雜皮而已。《禮記郊特牲》云虎豹皮與玉璧,非虛作也。則虎豹之皮,居然用兩,珪璧宜仍舊各一也。』參詵、龢二議不異,今加珪璋各
 一,豹熊羆皮各二,以龢議為允。」詔可。



 晉武帝太始十年,將聘拜三夫人九嬪。有司奏:「禮,皇后聘以穀珪,無妾媵禮贄之制。」詔曰:「拜授可依魏氏故事。」於是臨軒使使持節兼太常拜夫人,兼御史中丞拜九嬪。漢、魏之禮,公主居第,尚公主者來第成婚。司空王朗以為不可,其後乃革。



 凡遣大使拜皇后、三公,及冠皇太子,及拜蕃王,帝皆臨
 軒。其儀,太樂令宿設金石四廂之樂於殿前。漏上二刻,侍中、侍臣、冗從僕射、中謁者、節騎郎、虎賁,旄頭遮列,五牛旗皆入。虎賁中郎將、羽林監分陛端門內。侍御史、謁者各一人監端門。廷尉監、平分陛東、西中華門。漏上三刻,殿中侍御史奏開殿之殿門、南止車門、宣陽城門。軍校、侍中、散騎常侍、給事黃門侍郎、散騎侍郎升殿夾御座。尚書令以下應階者以次入。治禮引大鴻臚入,陳九賓。漏上四刻,侍中奏:「外辦。」皇帝服袞冕之服,升太極殿,
 臨軒南面。謁者前北面一拜,跪奏:「大鴻臚臣某稽首言,群臣就位。謹具。」侍中稱制曰:「可。」謁者贊拜,在位皆再拜。大鴻臚稱臣一拜,仰奏:「請行事。」侍中稱制曰;「可。」鴻臚舉手曰:「可行事。」謁者引護當使者當拜者入就拜位。四廂樂作。將拜,樂止。禮畢出。



 官有其注。舊時歲旦,常設葦茭桃梗,磔雞於宮及百寺門,以禳惡氣。《漢儀》,則仲夏之月設之,有桃卯,無磔雞。案明帝大修禳禮,故何晏禳祭議據雞牲供禳釁之事,磔雞宜起於魏也。桃卯本漢所以輔,
 卯金又宜魏所除也,但未詳改仲夏在歲旦之所起耳。宋皆省,而諸郡縣此禮往往猶存。



 上代聘享之禮,雖頗見經傳,然首尾不全。《叔孫通傳》載通所制漢元會儀,綱紀粗舉,施於今,又未周備也。魏國初建,事多兼闕,故黃初三年,始奉璧朝賀。



 何承天云,魏元會儀無存者。案何楨《許都賦》曰:「元正大饗,壇彼西南。旗幕峨峨,簷宇弘深。」王沈《正會賦》又曰:「華幄映於飛雲,朱幕張於前庭。絙青帷於兩階,象紫極之崢嶸。延百辟于和門,等尊卑而奉
 璋。」此則大饗悉在城外,不在宮內也。臣案魏司空王朗奏事曰:「故事,正月朔,賀。殿下設兩百華燈,對於二階之間。端門設庭燎火炬,端門外設五尺、三尺燈。月照星明,雖夜猶晝矣。」



 如此,則不在城外也。何、王二賦,本不在洛京。何云《許都賦》,時在許昌也。



 王賦又云「朝四國於東巡」,亦賦許昌正會也。



 晉武帝世,更定元會注,今有《咸寧注》是也。傅玄《元會賦》曰:「考夏后之遺訓,綜殷、周之典藝,採秦、漢之舊儀,定元正之嘉會。」此則兼採眾代可知矣。《咸
 寧注》,先正一日,守宮宿設王公卿校便坐於端門外,大樂鼓吹又宿設四廂樂及牛馬帷皞於殿前。夜漏未盡十刻,群臣集到,庭燎起火。上賀謁報,又賀皇后。還從雲龍東中華門入謁,詣東皞下便坐。漏未盡七刻,群司乘車與百官及受贄郎下至計吏,皆入,詣陛部立。其陛衛者,如臨軒儀。漏未盡五刻,謁者僕射、大鴻臚各奏:「群臣就位定。」漏盡,侍中奏:「外辦。」皇帝出,鐘鼓作,百官皆拜伏。太常導皇帝升御座,鐘鼓止,百官起。大鴻臚跪奏:「請朝
 賀。」治禮郎贊:「皇帝延王登。」大鴻臚跪贊:「蕃王臣某等奉白璧各一,再拜賀。」太常報:「王悉登。」謁者引上殿,當御座。皇帝興,王再拜。皇帝坐,復再拜,跪置璧御座前,復再拜。成禮訖,謁者引下殿,還故位。治禮郎引公、特進、匈奴南單于子、金紫將軍當大鴻臚西,中二千石、二千石、千石、六百石當大行令西,皆北面伏。



 大鴻臚跪贊:「太尉、中二千石等奉璧皮帛羔鴈雉,再拜賀。」太常贊:「皇帝延君登。」禮引公至金紫將軍上殿,當御座。皇帝興,皆再拜。皇帝
 坐,又再拜。跪置璧皮帛御座前,復再拜。成禮訖,贊者引下殿,還故位。王公置璧成禮時,大行令並贊,殿下中二千石以下同。成禮訖,以贄授受贄郎,郎以璧帛付謁者,羔鴈雉付太官。太樂令跪請奏雅樂,以次作樂。乘黃令乃出車,皇帝罷入,百官皆坐。晝漏上水六刻,諸蠻夷胡客以次入,皆再拜訖,坐。御入三刻,又出。鐘鼓作。謁者僕射跪奏:「請群臣上。」謁者引王公至二千石上殿,千石、六百石停本位。謁者引王詣尊酌壽酒,跪授侍中。侍中跪置
 御座前。王還自酌,置位前。謁者跪奏:「蕃王臣某等奉觴再拜,上千萬歲壽。」侍中曰:「觴已上。」百官伏稱萬歲,四廂樂作,百官再拜。已飲,又再拜。謁者引諸王等還本位。陛者傳就席,群臣皆跪諾。侍中、中書令、尚書令各於殿上上壽酒,登歌樂升,太官令又行御酒。御酒升階,太官令跪授侍郎,侍郎跪進御座前。乃行百官酒。太樂令跪奏:「奏登歌。」



 三。終,乃降。太官令跪請御飯到陛,群臣皆起。太官令持羹跪授司徒;持飯跪授大司農;尚食持案並授
 侍郎,侍郎跪進御座前。群臣就席,太樂令跪奏:「食。舉樂。」太官行百官飯案遍。食畢,太樂令跪奏:「請進儛。」儛以次作。鼓吹令又前跪奏:「請以次進眾伎。」乃召諸郡計吏前,授敕戒於階下。宴樂畢,謁者一人跪奏:「請罷退。」鐘鼓作,群臣北面再拜出。江左更隨事立位,大體亦無異也。



 宋有天下,多仍舊儀,所損益可知矣。



 晉江左注,皇太子出會者,則在三恪下、王公上。宋文帝元嘉十一年,升在三恪上。
 魏制,蕃王不得朝覲。明帝時有朝者,皆由特恩,不得以為常。晉泰始中,有司奏:「諸侯之國,其王公以下入朝者,四方各為二番,三歲而周,周則更始。



 若臨時有解,卻在明年。來朝之後,更滿三歲乃復,不得從本數。朝禮執璧如舊朝之制。不朝之歲,各遣卿奉聘。」奏可。江左王侯不之國,其有授任居外,則同方伯刺史二千石之禮,亦無朝聘之制,此禮遂廢。



 正旦元會,設白虎樽於殿庭。樽蓋上施白虎,若有能獻
 直言者,則發此樽飲酒。



 案《禮記》,知悼子卒,未葬,平公飲酒,師曠、李調侍,鼓鐘。杜蕢自外來,聞鐘聲曰:「安在?」曰:「在寢。」杜蕢入寢,歷階而升,酌曰:「曠飲斯。」又酌曰:「調飲斯。」又酌,堂上北面坐飲之。降,趨而出。平公呼而進之曰:「蕢,曩者爾心或開予,是以不與爾言,爾飲曠,何也?」曰:「子卯不樂,知悼子在堂,斯其為子卯也大矣。曠也,太師也。不以詔,是以飲之也。」「爾飲調,何也。」



 曰:「調也,君之褻臣也。為一飲一食,忘君之疾,是以飲之也。」「爾飲,何也?」



 曰:「蕢也宰夫,
 唯刀匕是供,又敢與知防,是以飲也。」平公曰;「寡人亦有過焉。酌而飲寡人。」杜蕢洗而揚觶。公謂侍者曰:「如我死,則必無廢斯爵。」至于今,既畢獻,斯揚觶,謂之「杜舉」。白虎樽,蓋杜舉之遺式也。畫為虎,宜是後代所加,欲令言者猛如虎,無所忌憚也。



 漢以高帝十月定秦旦為歲首,至武帝雖改用夏正,然朔猶常饗會,如元正之儀。



 魏、晉則冬至日受萬國及百僚稱賀,因小會。其儀亞於歲旦,晉有其注。宋永初元年
 八月,詔曰:「慶冬使或遣不,役宜省,今可悉停。唯元正大慶,不得廢耳。郡縣遣冬使詣州及都督府者,亦宜同停。」



 孫權始都武昌及建業,不立郊兆。至末年太元元年十一月,祭南郊,其地今秣陵縣南十餘里郊中是也。晉氏南遷,立南郊於巳地,非禮所謂陽位之義也。宋武大明三年九月,尚書右丞徐爰議:「郊祀之位,遠古蔑聞。《禮記》『燔柴於太壇,祭天也。』『兆於南郊,就陽位也。』漢初甘泉河東禋埋易位,終亦徙於長安南北。



 光武紹祚,定二郊洛
 陽南北。晉氏過江,悉在北。及郊兆之議,紛然不一。又南出道狹,未議開闡,遂於東南巳地創立丘壇。皇宋受命,因而弗改。且居民之中,非邑外之謂。今聖圖重造,舊章畢新,南驛開塗,陽路修遠。謂宜移郊正午,以定天位。」博士司馬興之、傅郁、太常丞陸澄並同爰議。乃移郊兆於秣陵牛頭山西,正在宮之午地。世祖崩,前廢帝即位,以郊舊地為吉祥,移還本處。北郊,晉成帝世始立,本在覆舟山南。宋太祖以其地為樂游苑,移於山西北。後以其
 地為北湖,移於湖塘西北。其地卑下泥濕,又移於白石村東。其地又以為湖,乃移於鐘山北京道西,與南郊相對。後罷白石東湖,北郊還舊處。



 南郊,皇帝散齋七日,致齋三日。官掌清者亦如之。致齋之朝,御太極殿幄坐。



 著絳紗袍,黑介幘,通天金博山冠。先郊日未晡五刻,夕牲。公卿京兆尹眾官悉壇東就位,太祝史牽牲入。到榜,稟犧令跪白:「請省牲。」舉手曰:「盾。」太祝令繞牲,舉手曰:「充。」太祝令牽牲詣庖。以二陶豆酌毛血,其一奠皇天神座前,
 其一奠太祖神座前。郊之日未明八刻,太祝令進饌,郎施饌。牲用繭栗二頭,群神用牛一頭。醴用翽鬯,藉用白茅。玄酒一器,器用匏陶,以瓦樽盛酒,瓦圩斟酒。璧用蒼玉。蒯席各二,不設茵蓐。古者席槁,晉江左用蒯。車駕出,百官應齋及從駕填街先置者,各隨申攝從事。上水一刻,御服龍袞,平天冠,升金根車,到壇東門外。博士、太常引入到黑攢。太祝令跪執匏陶,酒以灌地。皇帝再拜,興。



 群臣皆再拜伏。治禮曰:「興。」博士、太常引皇帝至南階,脫
 舄升壇,詣罍盥。



 黃門侍郎洗爵,跪授皇帝。執樽郎授爵,酌秬鬯授皇帝。跪奠皇天神座前,再拜,興。次詣太祖配天神座前,執爵跪奠,如皇天之禮。南面北向,一拜伏。太祝令各酌福酒,合置一爵中,跪進皇帝,再拜伏。飲福酒訖,博士、太常引帝從東階下,還南階。謁者引太常升壇,亞獻。謁者又引光祿升壇,終獻。訖,各降階還本位。



 太祝送神,跪執匏陶,酒以灌地。興。直南行出壇門,治禮舉手白,群臣皆再拜伏。



 皇帝盤,治禮曰:「興。」博士跪曰:「祠事畢,
 就燎。」博士、太常引皇帝就燎位,當壇東階,皇帝南向立。太祝令以案奉玉璧牲體爵酒黍飯諸饌物,登柴壇施設之。治禮舉手曰:「可燎。」三人持火炬上。火發。太祝令等各下壇。壇東西各二十人,以炬投壇,火半柴傾。博士仰白:「事畢。」皇帝出便坐。解嚴。天子有故,則三公行事,而太尉初獻,其亞獻、終獻,猶太常、光祿勛也。北郊齋、夕牲、進熟,及乘輿百官到壇三獻,悉如南郊之禮;唯事訖,太祝令牲玉饌物詣坎置牲上訖,又以一牲覆其上。治禮舉
 手曰:「可埋。」二十人俱時下土。填坎欲半,博士仰白:「事畢。」帝出。自魏以來,多使三公行事,乘輿罕出矣。魏及晉初,儀注雖不具存,所損益漢制可知也。江左以後,官有其注。



 魏文帝詔曰:「漢氏不拜日於東郊,而旦夕常於殿下東面拜日,煩褻似家人之事,非事天郊神之道也。」黃初二年正月乙亥,朝日于東門之外。按《禮》,天子以春分朝日於東,秋分夕月於西,今正月,非其時也。《漢郊祀志》,帝郊
 泰畤,平旦出竹宮東向揖日,其夕西向揖月。此為即用郊日,不俟二分也。明帝太和元年二月丁亥朔,朝日于東郊,八月己丑,夕月于西郊,此古禮也。《白虎通》:「王者父天、母地,兄日、姊月」,此其義也。《尚書大傳》,迎日之詞曰:「維某年某月上日。明光于上下,勤施于四方,旁作穆穆,維予一人。某敬拜迎日于郊。」



 吳時郎陳融奏《東郊頌》,吳時亦行此禮也。晉武帝太康二年,有司奏:「春分依舊車駕朝日,寒溫未適,可不親出。」詔曰:「禮儀宜有常;如所奏,與
 故太尉所撰不同,復為無定制。間者方難未平,故每從所奏。今戎事弭息,唯此為大。」案此詔,帝復為親朝日也。此後廢。



 殷祠,皇帝散齋七日,致齋三日。百官清者亦如之。致齋之日,御太極殿幄坐,著絳紗袍,黑介幘,通天金博山冠。祠之日,車駕出,百官應齋從駕留守填街先置者,各依宣攝從事。上水一刻,皇帝著平冕龍袞之服,升金根車,到廟北門訖。治禮、謁者各引太樂、太常、光祿勳、三公等皆
 入在位。皇帝降車入廟,脫舄盥,及洗爵,訖,升殿。初獻,奠爵,樂奏。太祝令跪讀祝文,訖,進奠神座前,皇帝還本位。博士引太尉亞獻,訖,謁者又引光祿勳終獻。凡禘祫大祭,則神主悉出廟堂,為昭穆以安坐,不復停室也。晉氏又有陰室四殤,治禮引陰室以次奠爵於饌前。其功臣配饗者,設坐於庭,謁者奠爵于饌前。皇帝不親祠,則三公行事,而太尉初獻,太常亞獻,光祿勳終獻也。四時祭祀,亦皆於將祭必先夕牲,其儀如郊。



 晉武帝太始七年
 四月,帝將親祠,車駕夕牲,而儀注還不拜。詔問其故。博士奏:「歷代相承如此。」帝曰:「非致敬宗廟之禮也。」於是實拜而還,遂以為制。



 太康中,有司奏議,十一月一日合朔奠、冬烝、夕牲同日,可有司行事。詔曰:「夕牲而令有司行事,非也。改擇上旬他日。」案此則武帝夕牲必躬臨拜,而江左以來復止也。晉元帝建武元年十月辛卯,即晉王位,行天子殷祭之禮,非常之事也。



 孝武太元十一年九月,皇女亡及應烝祠。中書侍郎范寧奏:「案《喪服》傳,有死
 宮中者,三月不舉祭,不別長幼之與貴賤也。皇女雖在嬰孩,臣竊以為疑。」於是尚書奏使三公行事。昔漢靈帝世,立春尚齋迎氣東郊,尚書左丞毆殺陌使於南書寺,於是詔書曰:「議郎蔡邕、博士任敏,問可齋祠不?得無不宜?」邕等對曰:「按上帝之祠,無所為廢。宮室至大,陌使至微,日又寬,可齋無疑。」寧非不知有此議,然不從也。魏及晉初,祭儀雖不具存,江左則備矣。官有其注。



 祠太社、帝社、太稷,常以歲二月八月二社日祠之。太祝令
 夕牲進熟,如郊廟儀。司空、太常、大司農三獻也。官有其注。周禮王親祭,漢以來,有司行事。漢安帝元初六年,立六宗祠於國西北戌城地,祠儀比泰社。日月將交會,太史上合朔。



 尚書先事三日,宣攝內外,戒嚴。摯虞《決疑》曰:「凡救蝕者,皆著赤幘,以助陽也。日將蝕,天子素服避正殿,內外嚴警,太史登靈臺,伺候日變。更伐鼓於門,聞鼓音,侍臣皆著赤幘,帶劍入侍。三臺令史以上,皆各持劍立其戶前。衛尉卿馳繞宮,伺察守備,
 周而復始。日復常,乃皆罷。」魯昭公十七年,六月朔,日有蝕之。祝史請所由,叔孫昭子曰:「日有蝕之,天子不舉樂,伐鼓於社;諸侯用敝於社,伐鼓於朝,禮也。」又以赤絲為繩繫社,祝史陳辭以責之。社,勾龍之神,天子之上公,故責之。合朔,官有其注。



 昔漢建安中,將王會,而太史上言正旦當日蝕,朝士疑會不。共詣尚書令荀文若諮之,時廣平計吏劉劭在坐,曰:「梓慎、裨灶,古之良史,猶占水火,錯失天時。《禮》諸侯旅見天子,入門不得終禮者四,日蝕在
 一。然則聖人垂制,不為變異豫廢朝禮者,或災消異伏,或推術謬誤也。」文若及眾人咸喜而從之,遂朝會如舊,日亦不蝕。劭由此顯名,魏史美而書之。魏高貴鄉公正元二年三月朔,太史奏日蝕而不蝕。晉文王時為大將軍,大推史官不驗之負。史官答曰;「合朔之時,或有日掩月,或有月掩日。月掩日,則蔽障日體,使光景有虧,故謂之日蝕;日掩月,則日於月上過,謂之陰不侵陽,雖交無變。日月相掩必食之理,無術以知,是以嘗禘郊社,日蝕
 則接祭,是亦前代史官不能審蝕也。自漢故事,以為日蝕必當於交。



 每至其時,申警百官,以備日變。故《甲寅詔》有備蝕之制,無考負之法。古來黃帝、顓頊、夏、殷、周、魯六歷,皆無推日蝕法,但有考課疏密而已。負坐之條,由本無術可課,非司事之罪。」乃止。



 晉武帝咸寧三年、四年,並以正旦合朔卻元會,改魏故事也。晉元帝太興元年四月合朔,中書侍郎孔愉奏曰:「《春秋》日有蝕之,天子伐鼓于社,攻諸陰也。



 諸侯伐鼓於朝,臣自攻也。案尚書符,若
 日有變,便伐鼓於諸門,有違舊典。」詔曰:「所陳有正義,輒敕外改之。」



 至康帝建元元年,太史上元日合朔,朝士復疑應卻會與否。庾冰輔政,寫劉劭議以示八坐,于時有謂劭為不得禮意,荀文若從之,是勝人之一失。故蔡謨遂著議非之曰:「劭論災消異伏,又以慎、灶猶有錯失,太史上言亦不必審,其理誠然也。



 而云聖人垂制,不為變異豫廢朝禮,此則謬矣。災祥之發,所以譴告人君,王者所重誡。故素服廢樂,退避正寢,百官降物,用幣伐鼓,躬
 親而救之。夫敬誡之事,與其疑而廢之,寧慎而行之。故孔子、老聃助葬於巷黨,以喪不見星而行,故日蝕而止柩,曰安知其不見星也。今史官言當蝕,亦安知其不蝕乎?夫子、老聃豫行見星之防,而劭廢之,是棄聖賢之成規也。魯桓公壬申有災,而以乙亥嘗祭,《春秋》譏之。災事既過,猶退懼未已,故廢宗廟之祭;況聞天眚將至,行慶樂之會,於禮乖矣。《禮記》所云『諸侯入門不得終禮者』,謂日官不豫言,諸侯既入,見蝕乃知耳;非先聞當蝕,而朝會
 不廢也。別此,可謂失其義指。劉劭所執者《禮記》也;夫子、老聃巷黨之事,亦《禮記》所言,復違而反之,進退無據。荀令所善,漢朝所從,遂使此言至今見稱,莫知其謬。後來君子,將擬以為式,故正之云爾。」於是冰從眾議,遂以卻會。至永和中,殷浩輔政,又欲從劉劭議不卻會。王彪之據咸寧、建元故事,又曰:「《禮》云,諸侯旅見天子,不得終禮而廢者四,自謂卒暴有之,非為先存其事而徼幸史官推術繆錯,故不豫廢朝禮也。」於是又從彪之,相承至今。



 耕籍之禮尚矣,漢文帝修之。及昭帝幼即大位,耕於鉤盾弄田。明帝永平十五年二月,東巡,耕於下邳。章帝元和三年正月北巡,耕於懷縣。魏三祖皆親耕籍。



 晉武帝泰始四年,有司奏始耕祠先農,可令有司行事。詔曰:「夫民之大事,在祀與農。是以古之聖王,躬耕帝籍,以供郊廟之粢盛,且以訓化天下。近代以來,耕籍止於數步中,空有慕古之名,曾無供祀訓農之實,而有百官車徒之費。今修千畝之制,當與群公卿士,躬稼穡之艱難,以帥先天
 下。主者詳具其制,并下河南處田地於東郊之南,洛水之北,平良中水者。若無官田,隨宜便換,不得侵民人也。」



 自此之後,其事便廢,史注載多有闕。止元、哀二帝,將修耕籍,賀循等所上注,及裴憲為胡中所定儀,又未詳允。



 元嘉二十年,太祖將親耕,以其久廢,使何承天撰定儀注。史學生山謙之已私鳩集,因以奏聞。乃下詔曰:「國以民為本,民以食為天。一夫輟耕,饑者必及。



 倉廩既實,禮節以興。自頃在所貧耗,家無宿積,陰陽暫偏,則人懷愁
 墊;年或不稔,而病乏比室。誠由政德未孚,以臻斯弊,抑亦耕桑未廣,地利多遺。宰守微化導之方,氓庶忘勤分之義。永言弘濟,明發載懷。雖制令亟下,終莫懲勸,而坐望滋殖,庸可致乎!有司其班宣舊條,務盡敦課。遊食之徒,咸令附業。考覈勤惰,行其誅賞;觀察能殿,嚴加黜陟。古者從時脈土,以訓農功,躬耕帝籍,敬供粢盛。



 仰瞻前王,思遵令典,便可量處千畝,考卜元辰。朕當親率百辟,致禮郊甸。庶幾誠素,獎被斯民。」於是斟酌眾條,造定圖
 注。先立春九日,尚書宣攝內外,各使隨局從事。司空、大農、京尹、令、尉,度宮之辰地八里之外,整制千畝,開阡陌。



 立先農壇於中阡西陌南,御耕壇於中阡東陌北。將耕,宿設青幕于耕壇之上。皇后帥六宮之人出種穜之種,付籍田令。耕日,太祝以一太牢告祠先農,悉如祠帝社之儀。孟春之月,擇上辛後吉亥日,御乘耕根三蓋車,駕蒼駟,青旂,著通天冠,青幘,朝服青袞,帶佩蒼玉。蕃王以下至六百石皆衣青。唯三臺武衛不耕,不改服章。



 車駕
 出,眾事如郊廟之儀。車駕至籍田,侍中跪奏:「尊降車。」臨壇,大司農跪奏:「先農已享,請皇帝親耕。」太史令贊曰:「皇帝親耕。」三推三反。於是群臣以次耕,王公五等開國諸侯五推五反,孤卿大夫七推七反,士九推九反。籍田令率其屬耕,竟畝,灑種,即櫌,禮畢。魏氏雖天子耕籍,其蕃鎮諸侯,並闕百畝之禮。晉武帝末,有司奏:「古諸侯耕籍百畝,躬秉耒耜,以奉社稷宗廟,以勸率農功。今諸王治國,宜修耕籍之義。」然未施行。宋太祖東耕後,乃班下州
 郡縣,悉備其禮焉。



 周禮,王后帥內外命婦,蠶於北郊。漢則東郊,非古也。魏則北郊,依周禮也。



 晉則西郊,宜是與籍田對其方也。魏文帝黃初七年正月,命中宮蠶于北郊。按韋誕《后蠶頌》,則于時漢注已亡,更考撰其儀也。及至晉氏,先蠶多采魏法。晉武帝太康六年,散騎常侍華嶠奏:「先王之制,天子諸侯親耕千畝,后夫人躬蠶桑。今陛下以聖明至仁,修先王之緒,皇后體資生之德,合配乾之義,而教道未
 先,蠶禮尚闕。以為宜依古式,備斯盛典。」詔曰:「古者天子親籍以供粢盛,后夫人躬蠶以備祭服。所以聿遵孝敬,明教示訓也。今籍田有制,而蠶禮不修。中間務多,未暇崇備。今天下無事,宜修禮以示四海。其詳依古典及近代故事,以參今宜。明年施行。」於是使侍中袁粲草定其儀。皇后采桑壇在蠶室西,帷宮中門之外,桑林在其東,先蠶壇在宮外門之外而東南。取民妻六人為蠶母。蠶將生,擇吉日,皇后著十二笄,依漢魏故事,衣青衣,乘油
 蓋雲母安車,駕六馬。女尚書著貂蟬,佩璽,陪乘,載筐鉤。公主、三夫人、九嬪、世婦、諸太妃、公太夫人、公夫人,及縣鄉君、郡公侯特進夫人、外世婦、命婦,皆步搖、衣青,各載筐鉤從。蠶桑前一日,蠶官生蠶著薄上。桑日,太祝令以一太牢祠先蠶。皇后至西郊,升壇,公主以下陪列壇東。皇后東面躬桑,采三條;諸妃公主各采五條;縣鄉君以下各采九條。悉以桑授蠶母。還蠶室。事訖,皇后還便坐,公主以下以次就位,設饗賜絹各有差。宋孝武大明四
 年,又修此禮。



 漢獻帝建安二十二年,魏國作泮宮于鄴城南。魏文帝黃初五年,立太學於洛陽。



 齊王正始中,劉馥上疏曰:「黃初以來,崇立太學,二十餘年,而成者蓋寡。由博士選輕,諸生避役,高門子弟,恥非其倫,故無學者。雖有其名,而無其實;雖設其教,而無其功。宜高選博士,取行為人表,經任人師者,掌教國子。依遵古法,使二千石以上子孫,年從十五,皆入太學。明制黜陟,陳榮辱之路。」不從。晉武帝
 泰始八年,有司奏:「太學生七千餘人,才任四品,聽留。」詔:「已試經者留之,其餘遣還郡國。大臣子弟堪受教者,令入學。」咸寧二年,起國子學。蓋《周禮》國之貴遊子弟所謂國子,受教於師氏者也。太康五年,修作明堂、辟雍、靈臺。



 孫休永安元年,詔曰:「古者建國,教學為先。所以導世治性,為時養器也。自建興以來,時事多故,吏民頗以目前趨務,棄本就末,不循古道。夫所尚不淳,則傷化敗俗。其按舊置學官,立《五經》博士,覈取應選,加其寵祿。科見史
 之中及將吏子弟有志好者,各令就業。一歲課試,差其品第,加以位賞。使見之者樂其榮,聞之者羨其譽。以淳王化,以隆風俗。」於是立學。



 元帝為晉王,建武初,驃騎將軍王導上疏:夫治化之本,在於正人倫。人倫之正,存乎設庠序。庠序設而五教明,則德化洽通,彞倫攸敘,有恥且格也。父子兄弟夫婦長幼之序順,而君臣之義固矣。《易》所謂正家而天下定者也。故聖王蒙以養正,少而教之,使化沾肌骨,習以成性,有若自然,日遷善遠罪,而不
 自知。行成德立,然後裁之以位,雖王之嫡子,猶與國子齒,使知道而後貴。其取才用士,咸先本之于學。故《周禮》,鄉大夫「獻賢能之書于王,王拜而受之」。所以尊道而貴士也。人知士之所貴,由乎道存;則退而修其身,修其身以及其家,正家以及於鄉,學於鄉以登於朝。反本復始,各求諸己,敦素之業著,浮偽之道息,教使然也。故以之事君則忠,用之蒞下則仁,即孟軻所謂「未有仁而遺其親,義而後其君者也」。



 自頃皇綱失統,禮教陵替,頌聲不
 興,于今二紀。《傳》曰:三年不為禮,禮必壞;三年不為樂,樂必崩」。而況如此其久者乎?先進漸忘揖讓之容,後生唯聞金革之響。干戈日尋,俎豆不設,先王之道彌遠,華偽之風遂滋,非所以習民靖俗,端本抑末之謂也。殿下以命世之資,屬當傾危之運,禮樂征伐,翼成中興,將滌穢蕩瑕,撥亂反正。誠宜經綸稽古,建明學校;闡揚六藝,以訓後生,使文武之道,墜而復興。方今《小雅》盡廢,戎虜扇熾,節義陵遲,國恥未雪。忠臣義士,所以扼腕拊心;禮樂
 政刑,當並陳以俱濟者也。茍禮義膠固,純風載洽,則化之所陶者廣,而德之所被者大,義之所屬者深,而威之所震者遠矣。由斯而進,則可朝服濟河,使帝典闕而復補,王綱弛而更張;饕餮改情,獸心革面,揖讓而蠻夷服,緩帶而天下從,得乎其道者,豈難也哉!故有虞舞干戚而三苗化,魯僖作泮宮而淮夷平,桓、文之霸,皆先教而後戰。今若聿遵前典,興復教道,使朝之子弟,並入于學,立德出身者咸習之而後通。德路開而偽塗塞,則其化
 不肅而成,不嚴而治矣。選明博修禮之士以為之師,隆教貴道,化成俗定,莫尚於斯也。



 散騎常侍戴邈又上表曰:臣聞天道之所運,莫大於陰陽;帝王之至務,莫重於禮學。是以古之建國,教學為先。國有明堂辟雍之制,鄉有庠序黌校之儀,皆所以抽導幽滯,啟廣才思,蓋以六四有《困》《蒙》之吝,君子大養正之功也。昔仲尼列國之大夫耳,興禮修學於洙、泗之間,四方髦俊,斐然向風,受業身通者七十餘人。自茲以來,千載寂漠。



 豈天下小於魯
 國,賢哲乏於曩時?厲與不厲故也。



 自頃遭無妄之禍,社稷有綴旒之危;寇羯飲馬於長江,凶狡虎步於萬里,遂使神州蕭條,鞠為茂草;四海之內,人跡不交。霸主有旰食之憂,黎民懷荼毒之痛,戎首交并于中原,何遽籩豆之事哉!然「三年不為禮,禮必壞;三年不為樂,樂必崩」。況曠載累紀,如此之久邪!今末進後生,目不睹揖讓升降之禮,耳不聞鐘鼓管弦之音,文章散滅胡馬之足,圖讖無復孑遺於世。此蓋聖達之所深悼,有識之所咨嗟也。
 夫治世尚文,遭亂尚武,文武迭用,久長之道。譬之天地昏明之術,自古以來,未有不由之者也。今以天下未一,非興禮學之時,此言似是而非。夫儒道深奧,不可倉卒而成,古之俊乂,必三年而通一經。比須寇賊清夷,天下平泰,然後修之,則功成事定,誰與制禮作樂者哉!又貴遊之子,未必有斬將搴旗之才,亦未有從軍征戍之役,不及盛年講肄道義,使明珠加瑩磨之功,荊、隨發采琢之美,不亦良可惜乎!



 愚以世喪道久,民情玩於所習,純風日
 去,華競日彰,猶火之消膏而莫之覺也。



 今天地造始,萬物權輿,聖朝以神武之德,值革命之運,蕩近世之流弊,繼千載之絕軌,篤道崇儒,創立大業。明主唱之於上,宰輔篤之於下,夫上之所好,下必有過之者焉。是故雙劍之節崇,而飛白之俗成;挾琴之容飾,而赴曲之和作。君子之德風,小人之德草,實在所以感之而已。臣以暗淺,不能遠識格言,謂宜以三時之隙,漸就經始。



 太興初,議欲修立學校,唯《周易》王氏、《尚書》鄭氏、《古文》孔氏、《毛詩》、《周
 官》、《禮記》、《論語》、《孝經》鄭氏、《春秋左傳》杜氏、服氏,各置博士一人。其《儀禮》、《公羊》、《穀梁》及鄭《易》,皆省不置博士。太常荀崧上疏曰:臣聞孔子有云,「才難,不其然乎」。自喪亂以來,經學尤寡。儒有席上之珍,然後能弘明道訓。今處學則闕朝廷之秀,仕朝則廢儒學之美。昔咸寧、太康、元康、永嘉之中,侍中、常侍、黃門之深博道奧,通洽古今,行為世表者,領國子博士。



 一則應對殿堂,奉酬顧問;二則參訓門子,以弘儒學;三則祠、儀二曹,及太常之職,以得藉用
 質疑。今皇朝中興,美隆往初,宜憲章令軌,祖述前典。世祖武皇帝聖德欽明,應運登禪,受終于魏。崇儒興學,治致升平。經始明堂,營建辟雍,告朔班政,鄉飲大射,西閣東序,圖書禁籍,臺省有宗廟太府金墉故事,太學有《石經》《古文》。先儒典訓,賈、馬、鄭、杜、服、孔、王、何、顏、尹之徒,章句傳注眾家之學,置博士十九人。九州之中,師徒相傳,學士如林,猶是選張華、劉寔居太常之官,以重儒教。



 《傳》稱「孔子沒而微言絕,七十子終而大義乖」。自頃中夏殄瘁,
 講誦遏密,斯文之道,將墜于地。陛下聖哲龍飛,闡弘祖烈,申命儒術,恢崇道教,樂正《雅》、《頌》,於是乎在。江、揚二州,先漸聲教,學士遺文,於今為盛;然方之疇昔,猶千之一也。臣學不章句,才不弘道,階緣光寵,遂忝非服。方之華、實,儒風邈遠;思竭駑駘,庶增萬分,願斯道隆於百代之上,搢紳詠於千載之下。



 伏聞節省之制,皆三分置二,博士舊員十有九人,今五經合九人。準古計今,猶未中半。九人以外,猶宜增四。願陛下萬機餘暇,時垂省覽。《周易》
 一經,有鄭玄注,其書根源,誠可深惜,宜為鄭《易》博士一人。《儀禮》一經,所謂曲禮,鄭玄於《禮》特明,皆有證據,宜置鄭《儀禮》博士一人。《春秋公羊》,其書精隱,明於斷獄,宜置博士一人。《穀梁》簡約隱要,宜存於世,置博士一人。昔周之衰,下陵上替,臣弒其君,子弒其父;上無天子,下無方伯;善者誰賞,惡者誰罰,綱紀亂矣。孔子懼而作《春秋》,諸侯諱石,懼犯時禁,是以微辭妙旨,義不顯明,故曰「知我者其唯《春秋》,罪我者其唯《春秋》。」時左丘明、子夏造膝親
 受,無不精究。孔子既沒,微言將絕,於是丘明退撰所聞而為之《傳》。其書善禮,多膏腴美辭;張本繼末,以發明經意,信多奇偉,學者好之。儒者稱公羊高親受子夏,立於漢朝,辭義清俊,斷決明審,多可採用,董仲舒之所善也。穀梁赤師徒相傳,暫立於漢,時劉向父子,漢之名儒,猶執一家,莫肯相從。其書文清約,諸所發明,或是《左氏》、《公羊》所不載,亦足有所訂正,是以《三傳》並行於先代,通才未能孤廢。今去聖久遠,斯文將墜,與其過廢,寧過而立也。
 臣以為《三傳》雖同一《春秋》,而發端異趣。案如三家異同之說,義則戰爭之場,辭亦劍戟之鋒,於理不可得共。博士宜各置一人,以傳其學。



 元帝詔曰:「崧表如此,皆經國大務,而為治所由。息馬投戈,猶可講藝。今雖日不暇給,豈忘本而道存邪!可共博議之。」有司奏宜如崧表。詔曰:「《穀梁》膚淺,不足立博士。餘如所奏。」會王敦之難,事不施行。



 成帝咸康三年,國子祭酒袁環、太常馮懷又上疏曰:臣聞先王之教也,崇典訓,明禮學,以示後生,道萬物之
 性,暢為善之道也。



 宗周既興,文史載煥,端委治於南蠻,頌聲逸於四海。故延州入聘,聞《雅》音而嗟咨;韓起適魯,觀《易》象而歎息。何者?立人之道,於此為首也。孔子恂恂,道化洙、泗;孟軻皇皇,誨誘無倦。是以仁義之聲,于今猶存,禮讓之風,千載未泯。



 疇昔陵替,喪亂屢臻,儒林之教暫頹,庠序之禮有闕。國學索然,墳卷莫啟,有心之徒,抱志無由。昔魏武身親介胄,務在武功,猶尚息鞍披覽,投戈吟詠,以為世之所須者,治之本宜崇。況今陛下以聖
 明臨朝,百官以虔恭蒞事,朝野無虞,江外靜謐。如之何泱泱之風,漠焉無聞;洋洋之美,墜於聖世乎!古人有言,《詩》《書》義之府,禮樂德之則。實宜留心經籍,闡明學義,使諷頌之音,盈於京室;味道之賢,典謨是詠,豈不盛哉!



 疏奏,帝有感焉。由是議立國學,徵集生徒,而世尚莊、老,莫肯用心儒訓。



 穆帝永和八年,殷浩西征,以軍興罷遣,由此遂廢。征西將軍庾亮在武昌,開置學官。教曰:人情重交而輕財,好逸而惡勞。學業致苦,而祿答未厚,由捷徑
 者多,故莫肯用心。洙、泗邈遠,《風》、《雅》彌替,後生放任,不復憲章典謨;臨官宰政者,務目前之治,不能閑以典誥。遂令《詩》、《書》荒塵,頌聲寂漠,仰瞻俯省,能弗歎慨!自胡夷交侵,殆三十年矣。而未革面響風者,豈威武之用盡,抑文教未洽,不足綏之邪?昔魯秉周禮,齊不敢侮;范會崇典,晉國以治。楚、魏之君,皆阻帶山河,憑城據漢,國富民殷,而不能保其強大,吳起、屈完所以為歎也。由此言之,禮義之固,孰與金城湯池?季路稱攝乎大國之間,加之以
 師旅,因之以饑饉,為之三年,猶欲行其義方。況今江表晏然,王道隆盛,而不能弘敷禮樂,敦明庠序,其何以訓彞倫而來遠人乎!魏武帝於馳鶩之時,以馬上為家,逮于建安之末,風塵未弭。然猶留心遠覽,大學興業,所謂顛沛必於是,真通才也。



 今使三時既務,五教並修,軍旅已整,俎豆無廢,豈非兼善者哉!便處分安學校處所,籌量起立講舍。參佐大將子弟,悉令入學,吾家子弟,亦令受業。四府博學識義通涉文學經綸者,建儒林祭酒,使
 班同三署,厚其供給;皆妙選邦彥,必有其宜者,以充此舉。近臨川、臨賀二郡,並求修復學校,可下聽之。若非束修之流,禮教所不及,而欲階緣免役者,不得為生。明為條制,令法清而人貴。



 又繕造禮器俎豆之屬,將行大射之禮。亮尋薨,又廢。



 孝武帝太元九年,尚書謝石又陳之曰:立人之道,曰仁與義。翼善輔性,唯禮與學。雖理出自然,必須誘導。故洙、泗闡弘道之風,《詩》、《書》垂軌教之典。敦《詩》悅《禮》,王化以斯而隆;甄陶九流,群生於是乎穆。世不
 常治,道亦時亡。光武投戈而習誦,魏武息馬以修學,懼墜斯文,若此之至也。大晉受命,值世多陰。雖聖化日融,而王道未備。庠序之業,或廢或興。遂令陶鑄闕日用之功,民性靡素絲之益,亹亹玄緒,翳焉莫抽,臣所以遠尋伏念,寤寐永歎者也。今皇威遐震,戎車方靜,將灑玄風於四區,導斯民於至德。豈可不弘敷禮樂,使煥乎可觀!請興復國學,以訓胄子;班下州郡,普修鄉校。雕琢琳琅,和寶必至;大啟群蒙,茂茲成德。匪懈于事,必由之以通,
 則人競其業,道隆學備矣。



 烈宗納其言。其年,選公卿二千石子弟為生,增造廟屋一百五十五間。而品課無章,士君子恥與其列。國子祭酒殷茂言之曰:臣聞弘化正俗,存乎禮教,輔性成德,必資於學。先王所以陶鑄天下,津梁萬物,閑邪納善,潛被於日用者也。故能疏通玄理,窮綜幽微,一貫古今,彌綸治化。



 且夫子稱回,以好學為本;七十希仰,以善誘歸宗。《雅》、《頌》之音,流詠千載。聖賢之淵範,哲王所同風。



 自大晉中興,肇基江左,崇明學校,修
 建庠序,公卿子弟,並入國學。尋值多故,訓業不終。陛下以聖德玄一,思隆前美,順通居方,導達物性,興復儒肆,僉與後生。自學建彌年,而功無可名。憚業避役,就存者無幾;或假託親疾,真偽難知,聲實渾亂,莫此之甚。臣聞舊制,國子生皆冠族華胄,比列皇儲。而中者混雜蘭艾,遂令人情恥之。子貢去朔之餼羊,仲尼猶愛其禮。況名實兼喪,面牆一世者乎!若以當今急病,未遑斯典,權宜停廢者,別一理也。若其不然,宜依舊準。竊謂群臣內外,
 清官子姪,普應入學,制以程課。今者見生,或年在扞格,方圓殊趣,宜聽其去就,各從所安。所上謬合,乞付外參議。



 烈宗下詔褒納,又不施行。朝廷及草萊之人有志於學者,莫不發憤歎息。清河人李遼又上表曰:「臣聞教者,治化之本,人倫之始,所以誘達群方,進德興仁,譬諸土石,陶冶成器。雖復百王殊禮,質文參差,至於斯道,其用不爽。自中華湮沒,闕里荒毀,先王之澤寢,聖賢之風絕。自此迄今,將及百年。造化有靈,否終以泰,河、濟夷徙,海、
 岱清通,黎庶蒙蘇,鳧藻奮化。而典訓弗敷,《雅》、《頌》寂蔑,久凋之俗,大弊未改。非演迪斯文,緝熙宏猷,將何以光贊時邕,克隆盛化哉!事有如賒而實急,此之謂也。亡父先臣回,綏集邦邑,歸誠本朝。以太元十年,遣臣奉表。路經闕里,過覲孔廟,庭宇傾頓,軌式頹弛,萬世宗匠,忽焉淪廢;仰瞻俯慨,不覺涕流。既達京輦,表求興復聖祀,修建講學。至十四年十一月十七日,奉被明詔,採臣鄙議,敕下兗州魯郡,準舊營飾。故尚書令謝石令臣所須列上,
 又出家布,薄助興立。故鎮北將軍譙王恬版臣行北魯縣令,賜許供遣。二臣薨徂,成規不遂。陛下體唐堯文思之美,訪宣尼善誘之勤,矜荒餘之凋昧,愍聲教之未浹。愚謂可重符兗州刺史,遂成舊廟,蠲復數戶,以供掃灑。并賜給《六經》,講立庠序,延請宿學,廣集後進,使油然入道,發剖琢之功。運仁義以征伐,敷道德以服遠,何招而不懷,何柔而不從!所為者微,所弘甚大。臣自致身輦轂,于今八稔,違親轉積,夙夜匪寧。振武將軍何澹之今震
 捍三齊,臣當隨反。裴回天邑,感戀罔極。乞臣表付外參議。」又不見省。



 宋高祖受命,詔有司立學,未就而崩。太祖元嘉二十年,復立國子學,二十七年廢。魏高貴鄉公甘露二年,車駕親率群司行養老之禮於太學。於是王祥為三老,鄭小同為五更。今無其注,然漢禮具存也。



 晉武帝泰始六年十二月,帝臨辟雍,行鄉飲酒之禮。詔
 曰:「禮儀之廢久矣,乃今復講肄舊典。賜太常絹百匹,丞、博士及學生牛酒。」咸寧三年,惠帝元康九年,復行其禮。魏齊王正始中,齊王每講經,使太常釋奠先聖先師於避雍,弗躬親。



 晉惠帝、明帝之為太子,及愍懷太子講經竟,並親釋奠於太學。太子進爵於先師,中庶子進爵於顏淵。元帝詔曰:「吾識太子此事,祠訖便請王公以下者,昔在洛時,嘗豫清坐也。」成、穆、孝武三帝,亦皆親釋奠。孝武時,以太學在水南縣遠,有司議依升平元年,於中堂
 權立行太學。于時無復國子生,有司奏:「應須二學生百二十人。太學生取見人六十,國子生權銓大臣子孫六十人,事訖罷。」奏可。釋奠禮畢,會百官六品以上。元嘉二十二年,太子釋奠,採晉故事,官有其注。祭畢,太祖親臨學宴會,太子以下悉豫。



 兵者,守國之備。孔子曰:「以不教民戰,是謂棄之。」兵,凶事,不可空設,因搜狩而習之。而凡師出曰治兵,入曰振旅,皆戰陳之事。辨鼓鐸鐲鐃之用,以教坐作進退疾徐疏數
 之節,遂以搜田。獻禽以祭社。仲夏教茇舍,如振旅之陳,遂以苗田,如搜之法。獻禽以享礿。仲秋教治兵,如振旅之陳,遂以獮田。如搜之法,致禽以祀方。仲冬教大閱,遂以狩田。獻禽以享蒸。搜者,搜索取其不孕者也。苗者,為苗除害而已。獮者,殺也。從秋氣所殺多也。狩者,冬物畢成,獲則取之,無所擇也。



 漢儀,立秋日,郊禮畢,始揚威武,斬牲於郊,以薦陵廟,名曰貙劉。其儀,乘輿御戎路,白馬朱鬣,躬執弩射牲。太宰令以獲車送陵廟。於是乘輿還
 宮,遣使以束帛賜武官,肄孫、吳兵法戰陳之儀,率以為常。至獻帝建安二十一年,魏國有司奏:「古四時講武,皆於農隙。漢西京承秦制,三時不講,唯十月都試。今兵革未偃,士民素習,可無四時講武。但以立秋擇吉日大朝車騎,號曰治兵。上合禮名,下承漢制。」奏可。是冬,治兵。魏王親金鼓以令進退。



 延康元年,魏文帝為魏王,是年六月立秋,治兵于東郊,公卿相儀。王御華蓋,親令金鼓之節。
 明帝太和元年十月,治兵于東郊。晉武帝泰始四年、九年、咸寧元年、太康四年、六年冬,皆自臨宣武觀,大習眾軍,然不自令進退也。自惠帝以後,其禮遂廢。元帝太興四年,詔左右衛及諸營教習,依大習儀作鴈羽仗。成帝咸和中,詔內外諸軍戲兵於南郊之場,故其地因名鬥場。自後蕃鎮桓、庾諸方伯,往往閱習,然朝廷無事焉。



 太祖在位,依故事肄習眾軍,兼用漢、魏之禮。其後以時講武於宣武堂。元嘉二十五年閏二月,大搜於
 宣武場,主胄奉詔列奏申攝,克日校獵,百官備辦。設行宮殿便坐武帳於幕府山南岡,設王公百官便坐幔省如常儀,設南北左右四行旌門;建獲旗以表獲車。殿中郎一人典獲車,主者二人收禽,吏二十四人配獲車。備獲車十二兩。校獵之官著褲褶。有帶武冠者,脫冠者上纓。二品以上擁刀,備槊、麾幡,三品以下帶刀。皆騎乘。將領部曲先獵一日,遣屯布圍。領軍將軍一人督右甄;護軍一人督左甄;大司馬一人居中,董正諸軍,悉受節度。
 殿中郎率獲車部曲,在司馬之後。尚書僕射、都官尚書、五兵尚書、左右丞、都官諸曹郎、都令史、都官諸曹令史幹、蘭臺治書侍御史令史、諸曹令史幹,督攝糾司,校獵非違。至日,會於宣武場,列為重圍。設留守填街位於雲龍門外內官道北,外官道南,以西為上。設從官位於雲龍門內大官階北,小官階南,以西為上。設先置官位於行止車門外內官道西,外官道東,以北為上。設先置官還位於廣莫門外道之東西,以南為上。校獵日平旦,正直侍中奏嚴。
 上水一刻,奏:「搥一鼓。」為一嚴。上水二刻,奏:「搥二鼓。」為再嚴。殿中侍御史奏開東中華雲龍門,引仗為小駕鹵簿。百官非校獵之官,著朱服,集列廣莫門外。應還省者還省。留守填街後部從官就位;前部從官依鹵簿;先置官先行。上水三刻,奏:「搥三鼓。」為三嚴。上水四刻,奏:「外辦。」正次直侍中、散騎常侍、給事黃門侍郎、軍校劍履進夾上皞。正直侍郎負璽,通事令史帶龜印中書之印。上水五刻,皇帝出,著黑介幘單衣,乘輦。正直侍中負璽陪乘,不
 帶劍。殿中侍御史督攝黃麾以內。次直侍中、次直黃門侍郎護駕在前。又次直侍中佩信璽、行璽,與正直黃門侍郎從護駕在後。不鳴鼓角,不得喧譁,以次引出,警蹕如常儀。東駕出,騶贊,陛者再拜。皇太子入守。車駕將至,威儀唱:「引先置前部從官就位。」再拜。車駕至行殿前回輦,正直侍中跪奏:「降輦。」次直侍中稱制曰:「可。」正直侍中俯伏起。皇帝降輦登御坐,侍臣升殿。直衛鈒戟虎賁,旄頭文衣,鶡尾,以次列階。正直侍中奏:「解嚴。」先置從駕百
 官還便坐幔省。



 帝若躬親射禽,變御戎服,內外從官以及虎賁悉變服,如校獵儀。鈒戟抽鞘,以備武衛。黃麾內官,從入圍裏。列置部曲,廣張甄圍,旗鼓相望,銜枚而進。甄周圍會,督甄令史奔騎號法施令曰:「春禽懷孕,搜而不射;鳥獸之肉不登於俎,不射;皮革齒牙骨角毛羽不登於器,不射。」甄會。大司馬鳴鼓蹙圍,眾軍鼓噪警角,至宣武場止。大司馬屯北旌門;二甄帥屯左右旌門;殿中中郎率獲車部曲入次北旌門內之右。皇帝從南旌門入射
 禽。謁者以獲車收載,還陳於獲旗北。王公以下以次射禽,各送詣獲旗下,付收禽主者。事畢,大司馬鳴鼓解圍復屯,殿中郎率其屬收禽,以實獲車,充庖廚。列言統曹正廚,置尊酒俎肉于中逵,以犒饗校獵眾軍。



 至晡,正直侍中量宜奏嚴,從官還著朱服,鈒戟復鞘。再嚴,先置官先還。三嚴後二刻,正直侍中奏:「外辦。」皇帝著黑介幘單衣。正次直侍中、散騎常侍、給事黃門侍郎、軍校進夾御坐。正直侍中跪奏:「還宮。」次直侍中稱制曰:「可。」



 正直侍中
 俯伏起。乘輿登輦還,衛從如常儀。大司馬鳴鼓散屯,以次就舍。車駕將至,威儀唱:「引留守填街先置前部從官就位。」再拜。車駕至殿前回輦,正直侍中跪奏:「降輦。」次直侍中稱制曰:「可。」正直侍中俯伏起。乘輿降入。正直次直侍中、散騎常侍、給事黃門侍郎、散騎侍郎、軍校從至皞,亦如常儀。正直侍中奏:「解嚴。」內外百官拜表問訊如常儀,訖,罷。



\end{pinyinscope}