\article{卷七本紀第七 前廢帝}

\begin{pinyinscope}

 前廢帝諱子業,小字法師,孝武帝長子也。元嘉二十六年正月甲申生。世祖鎮尋陽,子業留京邑。三十年,世祖入伐元兇,被囚侍中下省,將見害者數矣,卒得無恙。世
 祖踐祚,立為皇太子。始未之東宮,中庶子、二率並入直永福省。大明二年,出居東宮。四年,講《孝經》於崇正殿。七年,加元服。八年閏王月庚申,世祖崩,其日,太子即皇帝位。大赦天下。太宰江夏王義恭解尚書令,加中書監,驃騎大將軍柳元景加尚書令。甲子,置錄尚書,太宰江夏王義恭錄尚書事。驃騎大將軍柳元景加開府儀同三司。丹陽尹永嘉王子仁為南豫州刺史。



 六月辛未,詔曰:「朕以眇身,夙紹洪業,敬御天威,欽對靈命。仰遵凝緒,日鑒
 前圖,實可以拱默守成,詒風長世。而寶位告始,萬宇改屬,惟德弗明,昧于大道。思宣睿範,引茲簡恤,可具詢執事,詳訪民隱。凡曲令密文,繁而傷治,關市僦稅,事施一時,而姦吏舞文,妄興威福,加以氣緯舛玄,偏頗滋甚。宜其寬徭輕憲,以救民切。御府諸署,事不須廣,雕文篆刻,無施於今。悉宜并省,以酬氓願。籓王貿貨,壹皆禁斷。外便具條以聞。」戊寅,以豫州之淮南郡復為南梁郡,復分宣城還置淮南郡。庚辰,以南海太守袁曇遠為廣州刺
 史。秋七月己亥,鎮軍將軍、雍州刺史晉安王子勛改為江州刺史,中護軍宗愨為安西將軍、雍州刺史,鎮北將軍、徐州刺史湘東王諱為護軍將軍,中軍將軍義陽王昶為征北將軍、徐州刺史。



 庚戌,婆皇國遣使獻方物。崇皇太后曰太皇太后,皇后曰皇太后。乙卯,罷南北二馳道。孝建以來所改制度,還依元嘉。丙辰,追崇獻妃為皇后。乙丑,撫軍將軍、南徐州刺史新安王子鸞解領司徒。八月丁卯,領軍將軍王玄謨為鎮北將軍、南徐州刺
 史。新安王子鸞為青、冀二州刺史。己未,以青、冀二州刺史蕭惠開為益州刺史。



 己丑,皇太后崩。京師雨水。庚子,遣御史與官長隨宜賑恤。九月辛丑,護軍將軍湘東王諱為領軍將軍。癸卯,以尚書左僕射劉遵考為特進、右光祿大夫。乙卯,文穆皇后祔葬景寧陵。冬十月甲戌,太常建安王休仁為護軍將軍。戊寅,輔國將軍宋越為司州刺史。庚辰,原除揚、南徐州大明七年逋租。十二月乙酉,以尚書右僕射顏師伯為尚書左僕射。壬辰,以王畿
 諸郡為揚州,以揚州為東揚州。癸巳,以車騎將軍、揚州刺史豫章王子尚為司徒、揚州刺史。去歲及是歲,東諸郡大旱,甚者米一升數百,京邑亦至百餘,餓死者十有六七。孝建以來,又立錢署鑄錢,百姓因此盜鑄,錢轉偽小,商貨不行。



 永光元年春正月乙未朔,改元,大赦天下。乙巳,省諸州臺傳。戊午,以領軍將軍湘東王諱為衛將軍、南豫州刺史,護軍將軍建安王休仁為領軍將軍,秘書監山陽王
 休祐為豫州刺史,左衛將軍桂陽王休範為中護軍,南豫州刺史尋陽王子房為東揚州刺史。二月丁丑,減州郡縣田租之半。庚寅,鑄二銖錢。三月甲辰,罷臨江郡。



 五月己亥,割郢州隨郡屬雍州。丙午,以後軍司馬張牧為交州刺史。六月己巳,左軍長史劉道隆為梁、南秦二州刺史。乙亥,安西將軍、雍州刺史宗愨卒。壬午,衛將軍、南豫州刺史湘東王諱改為雍州刺史。尚書令、驃騎大將軍柳元景加南豫州刺史。秋八月辛酉,越騎校尉戴法興
 有罪,賜死。庚午,以尚書左僕射顏師伯為尚書僕射,吏部尚書王景文為尚書右僕射。癸酉,帝自率宿衛兵,誅太宰江夏王義恭、尚書令、驃騎大將軍柳元景、尚書僕射顏師伯、廷尉劉德願。改元為景和元年,文武賜位二等。以領軍將軍建安王休仁為安西將軍、雍州刺史,衛將軍湘東王諱還為南豫州刺史。甲戌,司徒、揚州刺史豫章王子尚領尚書令,射聲校尉沈文秀為青州刺史,左軍司馬崔道固為冀州刺史。乙亥,詔曰:「昔凝神佇逸,
 磻溪贊道,湛慮思才,傅巖毗化。朕位御三極,風澄萬宇,資鈇電斷,正卯斯戮。思所以仰宣遺烈,俯弘景祚,每結夢庖鼎,瞻言板築,有劬日昃,無忘昧旦。可甄訪郡國,招聘閭部:其有孝性忠節,幽居遁棲,信誠義行,廉正表俗,文敏博識,幹事治民,務加旌舉,隨才引擢。庶官克順,彞倫咸敘。主者精加詳括,稱朕意焉。」以始興公沈慶之為太尉,鎮北將軍、青冀二州刺史王玄謨為領軍將軍。庚辰,以石頭城為長樂宮,東府城為未央宮。罷東揚州並
 揚州。甲申,以北邸為建章宮,南第為長楊宮。以冠軍將軍邵陵王子元為湘州刺史。丙戌,原除吳、吳興、義興、晉陵、琅邪五郡大明八年以前逋租。己丑,復立南北二馳道。九月癸巳,車駕幸湖熟,奏鼓吹。戊戌,車駕還宮。庚子,以南兗州刺史永嘉王子仁為南徐州刺史,丹陽尹始安王子真為南兗州刺史。辛丑,撫軍將軍、南徐州刺史新安王子鸞免為庶人,賜死。丙午,以兗州刺史薛安都為平北將軍、徐州刺史。丁未,衛將軍湘東王諱加開府儀
 同三司,特進、右光祿大夫劉遵考為安西將軍、南豫州刺史,寧朔將軍殷孝祖為兗州刺史。戊申,以前梁、南秦二州刺史柳元怙復為梁、南秦二州刺史。己酉,車駕討征北將軍、徐州刺史義陽王昶,內外戒嚴。昶奔于索虜。辛亥,右將軍、豫州刺史山陽王休祐進號鎮西大將軍。甲寅,以安西長史袁鳷為雍州刺史。戊午,以左民尚書劉思考為益州刺史。是日解嚴,車駕幸瓜步。開百姓鑄錢。冬十月癸亥,曲赦徐州。丙寅,車駕還宮。以建安
 休仁為護軍將軍。己卯,東陽太守王藻下獄死。以宮人謝貴嬪為夫人,加虎賁靸戟,鸞輅龍旂,出警入蹕,實新蔡公主也。乙酉,以鎮北大將軍、豫州刺史山陽王休祐為鎮軍大將軍、開府儀同三司。十一月壬辰,寧朔將軍何邁下獄死。新除太尉沈慶之薨。壬寅,立皇后路氏,四廂奏樂。赦揚、南徐二州。護軍將軍建王休仁加特進、左光祿大夫。中護軍桂陽王休範遷職。丁未,皇子生,少府劉勝之子也。大赦天下,贓汙淫盜,悉皆原除。賜為父
 後者爵一級。壬子,以特進、左光祿大夫、護軍將軍建安王休仁為驃騎大將軍、開府儀同三司。戊午,南平王敬猷、廬陵王敬先、安南侯敬淵並賜死。



 時帝兇悖日甚,誅殺相繼,內外百司,不保首領。先是訛言云:「湘中出天子。」



 帝將南巡荊、湘二州以厭之。先欲誅諸叔,然後發引。太宗與左右阮佃夫、王道隆、李道兒密結帝左右壽寂之、姜產之等十一人,謀共廢帝。戊午夜,帝於華林園竹堂射鬼。時巫覡云:「此堂有鬼。」故帝自射之。壽寂之懷刀
 直入,姜產之為副。帝欲走,寂之追而殞之,時年十七。太皇太后令曰:司徒領護軍八座:子業雖曰嫡長,少稟兇毒,不仁不孝,著自髫齔。孝武棄世,屬當辰歷。自梓宮在殯,喜容靦然,天罰重離,歡恣滋甚。逼以內外維持,忍虐未露,而兇慘難抑,一旦肆禍,遂縱戮上宰,殄害輔臣。子鸞兄弟,先帝鐘愛,含怨既往,枉加屠酷。昶茂親作扞,橫相征討。新蔡公主逼離夫族,幽置深宮,詭云薨殞。襄事甫爾,喪禮頓釋,昏酣長夜,庶事傾遺。朝賢舊勳,棄若遺
 土。管絃不輟,珍羞備膳。詈辱祖考,以為戲謔。行游莫止,淫縱無度。肆宴園陵,規圖發掘。誅剪無辜,籍略婦女。建樹偽豎,莫知誰息。拜嬪立后,慶過恒典。宗室密戚,遇若婢僕,鞭捶陵曳,無復尊卑。南平一門,特鐘其酷。反天滅理,顯暴萬端。苛罰酷令,終無紀極,夏桀、殷辛,未足以譬。闔朝業業,人不自保;百姓遑遑,手足靡厝。行穢禽獸,罪盈三千。高祖之業將泯,七廟之享幾絕。吾老疾沉篤,每規禍鴆,憂煎漏刻,氣命無幾。開闢以降,所未嘗聞。遠近
 思奮,十室而九。



 衛將軍湘東王體自太祖,天縱英聖,文皇鐘愛,寵冠列籓。吾早識神睿,特兼常禮。潛運宏規,義士投袂,獨夫既殞,懸首白旗,社稷再興,宗祐永固,人鬼屬心,大命允集。且勳德高邈,大業攸歸,宜遵漢、晉,纂承皇極。主者詳舊典以時奉行。



 未亡人餘年不幸嬰此百艱,永尋情事,雖存若殞。當復奈何!當復奈何!



 葬廢帝丹陽秣陵縣南郊壇西。帝幼而狷急,在東宮每為世祖所責。世祖西巡,子業啟參承起居,書迹不謹,上詰讓之。子
 業啟事陳謝,上又答曰:「書不長進,此是一條耳。聞汝素都懈怠,狷戾日甚,何以頑固乃爾邪!」初踐阼,受璽紱,悖然無哀容。始猶難諸大臣及戴法興等,既殺法興,諸大臣莫不震懾。於是又誅群公,元凱以下,皆被毆捶牽曳。內外危懼,殿省騷然。初太后疾篤,遣呼帝。帝曰:「病人間多鬼,可畏,那可往。」太后怒,語侍者:「將刀來,破我腹,那得生如此寧馨兒!」及太后崩後數日,帝夢太后謂之曰:「汝不孝不仁,本無人君之相。



 子尚愚悖如此,亦非運祚所
 及。孝武險虐滅道,怨結人神,兒子雖多,並無天命。



 大運所歸,應還文帝之子。」其後湘東王紹位,果文帝子也。故帝聚諸叔京邑,慮在外為患。山陰公主淫恣過度,謂帝曰:「妾與陛下,雖男女有殊,俱託體先帝。



 陛下六宮萬數,而妾唯駙馬一人。事不均平,一何至此!」帝乃為主置面首左右三十人;進爵會稽郡長公主,秩同郡王侯,湯沐邑二千戶,給鼓吹一部,加班劍二十人。帝每出,與朝臣常共陪輦。主以吏部郎褚淵貌美,就帝請以自侍,帝許
 之。淵侍主十日,備見逼迫,誓死不回,遂得免。帝所幸閹人華願兒,官至散騎常侍,加將軍帶郡。帝少好講書,頗識古事,自造《世祖誄》及雜篇章,往往有辭采。以魏武帝有發丘中郎將、摸金校尉,乃置此二官。以建安王休祐領之。其餘事迹,分見諸傳。



 史臣曰:廢帝之事行著于篇。若夫武王數殷紂之釁,不能掛其萬一;霍光書昌邑之過,未足舉其毫釐。假以中才之君,有一於此,足以霣社殘宗,污宮瀦廟,況總斯惡
 以萃一人之體乎!其得亡,亦為幸矣。



\end{pinyinscope}