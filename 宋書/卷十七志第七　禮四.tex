\article{卷十七志第七 禮四}

\begin{pinyinscope}

 宋
 文帝元嘉三年五月庚午,以誅徐羨之等,仇恥已雪,幣告太廟。元嘉三年十二月甲寅,西征謝晦,告太廟、太社。晦平,車
 駕旋軫,又告。



 元嘉六年七月,太學博士徐道娛上議曰:「伏見太廟烝嘗儀注,皇帝行事畢,出便坐,三公已上獻,太祝送神于門,然後至尊還拜,百官贊拜,乃退。謹尋清廟之道,所以肅安神也。《禮》曰,廟者,貌也;神靈所馮依也。事亡如存,若常在也。既不應有送神之文,自陳豆薦俎,車駕至止,並弗奉迎。夫不迎而送,送而後辭,暗短之情,實用未達。按時人私祠,誠皆迎送,由於無廟,庶感降來格。因心立意,
 非王者之禮也。《儀禮》雖太祝迎尸于門,此乃延尸之儀,豈是敬神之典!



 恐於禮有疑。謹以議上。」有司奏下禮官詳判。



 博士江邃議:「在始不逆,明在廟也;卒事而送,節孝思也。若不送而辭,是舍親也;辭而後送,是遣神也。故孝子不忍違其親,又不忍遣神。是以祝史送神以成烝嘗之義。」博士賀道期議:「樂以迎來,哀以送往。《祭統》『迎牲而不迎尸』。



 《詩》云:『鐘鼓送尸。』鄭云:『尸,神象也。』與今儀注不迎而後送,若合符契。」博士荀萬秋議:「古之事尸,與今之事
 神,其義一也。周禮,尸出,送于廟門,拜,尸不顧。《詩》云:『鐘鼓送尸。』則送神之義,其來久矣。《記》曰:『迎牲而不迎尸,別嫌也。尸在門外,則疑於臣;入廟中,則全於君。君在門外,則疑於君;入廟,則全於臣。是故不出者,明君臣之義。』」邃等三人謂舊儀為是,唯博士陳氏同道娛議。參詳「邃等議雖未盡,然皆依擬經禮。道娛、氏所據難從。



 今眾議不一,宜遵舊體」。詔可。



 元嘉六年九月,太學博士徐道娛上議曰:「祠部下十月三
 日殷祠,十二日烝祀。



 謹按禘袷之禮,三年一,五年再。《公羊》所謂五年再殷祭也。在四時之間,《周禮》所謂凡四時之間祀也。蓋歷歲節月無定,天子諸侯,先後弗同。《禮》稱『天子袷嘗,諸侯烝袷。有田則祭,無田則薦』。鄭注:『天子先袷然後時祭,諸侯先時祭然後祫。有田者既祭又薦新。祭以首時,薦以仲月。』然則大祭四祀,其月各異。天子以孟月殷,仲月烝,諸侯孟月嘗,仲月袷也。《春秋》僖公八年秋七月,禘。文公二年八月,大事于太廟。《穀梁傳》曰:『著袷
 嘗也。』昭公十五年二月,『有事于武宮』。《左傳》曰:『禮也。』又《周禮》『仲冬享烝』。《月令》『季秋嘗稻』。晉春烝曲沃,齊十月嘗太公,此並孟仲區別不共之明文矣。凡祭必先卜,日用丁巳,如不從,進卜遠日。卜未吉,豈容二事,推期而往,理尤可知。



 尋殷烝祀重,祭薦禮輕。輕尚異月,重寧反同。且『祭不欲數,數則瀆』。今隔旬頻享,恐於禮為煩。自經緯墳誥,都無一月兩獻。先儒舊說,皆云殊朔。晉代相承,未審其原。國事之重,莫大乎祀。愚管膚淺,竊以惟疑。請詳告下
 議。」寢不報。



 元嘉七年四月乙丑,有司奏曰:「《禮·喪服》傳云:『有死於宮中者,則為之三月不舉祭。』今礿祀既戒,而掖庭有故。下太常依禮詳正。太學博士江邃、袁朗、徐道娛、陳氏等議,參互不同。殿中曹郎中領祠部謝元議以為:『遵依《禮》傳,使有司行事,於義為安。』輒重參詳。宗廟敬重,饗祀精明。雖聖情罔極,必在親奉。然茍曰有疑,則情以禮屈。無所稱述,於義有據。請聽如元所上。」詔可。



 元嘉十年十二月癸酉,太祝令徐閏刺署:「典宗廟社稷祠祀薦五牲,牛羊豕雞並用雄。其一種市買,由來送雌。竊聞周景王時,賓起見雄雞自斷其尾,曰:『雞憚犧,不詳。』今何以用雌,求下禮官詳正。」勒太學依禮詳據。博士徐道娛等議稱:「案《禮》孟春之月,『是月也,犧牲無用牝』。如此,是春月不用雌爾,秋冬無禁。雄雞斷尾,自可是春月。」太常丞司馬操議:「尋《月令》孟春『命祀山林川澤,犧牲無用牝。』若如學議,春祠三牲以下,便應一時俱改,以從《月令》,
 何以偏在一雞。」重更勒太學議答。博士徐道娛等又議稱:「凡宗祀牲牝不一,前惟《月令》不用牝者,蓋明在春必雄,秋冬可雌,非以山林同宗廟也。四牲不改,在雞偏異,相承來久,義或有由,誠非末學所能詳究。求詳議告報,如所稱令。」



 參詳閏所稱粗有證據,宜如所上。自今改用雄雞。



 孝武帝孝建三年五月丁巳,詔以第四皇子出紹江夏王太子睿為後。有司奏:「皇子出後,檢未有告廟先例,輒
 勒二學禮官議正,應告與不?告者為告幾室?」



 太學博士傅休議:「禮無皇子出後告廟明文。晉太康四年,封北海王寔紹廣漢殤王後,告于太廟。漢初帝各異廟,故告不必同。自漢明帝以來,乃共堂各室,魏、晉依之。今既共堂,若獨告一室,而闕諸室,則于情未安。」太常丞庾亮之議:「案《禮》,『大事則告祖禰,小事則特告禰』。今皇子出嗣,宜告禰廟。」祠部朱膺之議以為:「有事告廟,蓋國之常典。今皇子出紹,事非常均,愚以為宜告。賀循云,古禮異廟,唯謁
 一室是也。既皆共廟,而闕於諸帝,於情未安。謂循言為允,宜在皆告。」兼右丞殿中郎徐爰議以為:「國之大事,必告祖禰。皇子出嗣,不得謂小。昔第五皇子承統廬陵,備告七廟。」參議以爰議為允,詔可。



 大明元年六月己卯朔,詔以前太子步兵校尉祗男歆紹南豐王朗。有司奏:「朗先嗣營陽,告廟臨軒。檢繼體為舊,不告廟臨軒。」下禮官議正。太學博士王燮之議:「南豐昔別開土宇,以紹營陽,義同始封,故有臨軒告廟之禮。
 今歆奉詔出嗣,則成繼體,先爵猶存,事是傳襲,不應告廟臨軒。」祠部郎朱膺之議:「南豐王嗣爵封已絕,聖恩垂矜,特詔繼茅土,復申義同始封,為之告廟臨軒。」殿中郎徐爰議:「營陽繼體皇基,身亡封絕,恩詔追封,錫以一城。既始啟建茅土,故宜臨軒告廟。今歆繼後南豐,彼此俱為列國,長沙、南豐,自應各告其祖,豈關太廟?事非始封,不合臨軒。同博士王燮之議。」參詳,爰議為允,詔可。



 大明三年六月乙丑,有司奏:「來七月十五日,嘗祠太廟、
 章皇太后廟,輿駕親奉。而乘輿辭廟親戎,太子合親祠與不?且今月二十四日,第八皇女夭。案《禮》,『宮中有故,三月不舉祭』。皇太子入住上宮,於事有疑。」下禮官議正。太學博士司馬興之議:「竊惟『國之大事,在祀與戎』。皇太子有撫軍之道,而無專御之義,戎既如之,祀亦宜然。案《祭統》,『夫祭之道,孫為王父尸』。又云,『祭有昭穆,所以別父子』。太子監國,雖不攝,至於宗廟,則昭穆實存,謂事不可亂。



 又云,『有故則使人』。準此二三,太子無奉祀之道。又皇女
 夭札,則實同宮一體之哀,理不得異。設令得祀,令猶無親奉之義。」博士郁議:「案《春秋》,太子奉社稷之粢盛,長子主器,出可守宗廟,以為祭主,《易彖》明文。監國之重,居然親祭。皇女夭札,時既同宮,三月廢祭,於禮宜停。」二議不同。尚書參議,宜以郁議為允。詔可。



 太明三年十一月乙丑朔,有司奏:「四時廟祠,吉日已定,遇雨及舉哀,舊停親奉,以有司行事。先下使禮官博議,於禮為得遷日與不?」博士江長議:「《禮記·祭統》:『君之祭也,
 有故則使人,而君不失其儀。』鄭玄云:『君雖不親,祭禮無闕,君德不損。』愚以為有故則必使人者,明無遷移之文。茍有司充事,謂不宜改日。」太常丞陸澄議:「案《周禮》宗伯之職,『若王不與祭祀則攝位』。



 鄭君曰:『王有故,行其祭事也。』臣以為此謂在致齋,祭事盡備,神不可瀆,齋不可久,而王有他故,則使有司攝焉。晉泰始七年四月,世祖將親祠于太廟。庚戌,車駕夕牲。辛亥,雨,有司行事。此雖非人故,蓋亦天硋也。求之古禮,未乖周制。



 案《禮記》,『孔子答
 曾子,當祭而日蝕太廟火,如牲至未殺,則廢』。然則祭非無可廢之道也,但權所為之輕重耳。日蝕廟火,變之甚者,故乃牲至尚猶可廢。推此而降,可以理尋。今散齋之內,未及致齋,而有輕哀甚雨,日時展事,可以延敬。



 不愆義情,無傷正典,改擇令日,夫何以疑。愚謂散齋而有舉哀若雨,可更遷日。



 唯入致齋及日月逼晚者,乃使有司行事耳。又前代司空顧和啟,南郊車駕已出遇雨,宜遷日更郊,事見施用。郊之與廟,其敬可均,至日猶遷,況散
 齋邪!」殿中郎殷淡議:「《曾子問》『日蝕太廟火,牲未殺則廢』。縱有故則使人。清廟敬重,郊禋禮大,故廟焚日蝕,許以可遷;輕哀微故,事不合改。是以鼷鼠食牛,改卜非禮。



 晉世祖有司行事,顧司空之改郊月,既不見其當時之宜,此不足為準。愚謂日蝕廟火,天譴之變,乃可遷日。至於舉哀小故,不宜改辰。」眾議不同。參議,既有理據,且晉氏遷郊,宋初遷祠,並有成準。謂孟月散齋之中,遇雨及舉輕哀,宜擇吉更遷,無定限數。唯入致齋及侵仲月節者,
 使有司行事。詔可。



 大明五年十月甲寅,有司奏:「今月八日烝祠二廟,公卿行事。有皇太子獻妃服。」前太常丞庾蔚之議:「禮所以有喪廢祭,由祭必有樂。皇太子以元嫡之重,故主上服妃,不以尊降。既正服大功,愚謂不應祭。有故,三公行事,是得祭之辰,非今之比。卿卒猶不繹,況於太子妃乎?」博士司馬興之議:「夫緦則不祭,《禮》之大經;卿卒不繹,《春秋》明義。又尋魏代平原公主薨,高堂隆議不應三月廢祠,而
 猶云殯葬之間,權廢事改吉,芬馥享祠。尋此語意,非使有司。此無服之喪,尚以未葬為廢,況皇太子妃及大功未祔者邪?上尋禮文,下準前代,不得烝祠。」領軍長史周景遠議:「案《禮》,『緦不祭』。大功廢祠,理不俟言。今皇太子故妃既未山塋,未從權制,則應依禮廢烝嘗。至奠以大功之服,於禮不得親奉,非有故之謂,亦不使公卿行事。」右丞徐爰議以為:「《禮》,『緦不祭』,蓋惟通議。



 大夫以尊貴降絕,及其有服,不容復異。《祭統》云『君有故使人可』者,謂於禮
 應祭,君不得齋,祭不可闕,故使臣下攝奉。不謂君不應祭,有司行事也。晉咸寧四年,景獻皇后崩,晉武帝伯母,宗廟廢一時之祀,雖名號尊崇,粗可依準。今太子妃至尊正服大功,非有故之比。既未山塋,謂烝祠宜廢。尋蔚之等議,指歸不殊,闕烝為允。過卒哭祔廟,一依常典。」詔可。



 大明七年二月丙辰,有司奏:「鑾輿巡搜江左,講武校獵,獲肉先薦太廟、章太后廟,并設醢酒,公卿行事,及獻妃
 陰室,室長行事。」太學博士虞龢議:「檢《周禮》,四時講武獻牲,各有所施。振旅春搜,則以祭社;茇舍夏苗,則以享礿;治兵秋獮,則以祀祊;大閱冬狩,則以享烝。案《漢祭祀志》:『唯立秋之日,白郊事畢,始揚威武,名曰:「貙劉」。乘輿入囿,躬執弩以射,牲以鹿麑。太宰令謁者各一人,載獲車馳送陵廟。』然則春田薦廟,未有先準。」兼太常丞庾蔚之議:「龢所言是搜狩不失其時,此禮久廢。今時龢表晏,講武教人,又虔供乾豆,先薦二廟,禮情俱允。社主土神,司空
 土官,故祭社使司空行事。太廟宜使上公。參議搜狩之禮,四時異議,禮有損益,時代不同。今既無復四方之祭,三殺之儀,曠廢來久,禽獲牲物,面傷翦毛,未成禽不獻。太宰令謁者擇上殺奉送,先薦廟社二廟,依舊以太尉行事。」詔可。



 明帝泰豫元年七月庚申,有司奏:「七月嘗祠,至尊諒闇之內,為親奉與不?



 使下禮官通議。伏尋三年之制,自天子達。漢文愍秦餘之弊,於是制為權典。魏、晉以來,卒哭
 而祔則就吉。案《禮記王制》,『三年不祭,唯祭天地社稷,為越紼而行事。』鄭玄云:『唯不敢以卑廢尊也。』范宣難杜預、段暢,所以闕宗廟祭者,皆人理所奉,哀戚之情,同於生者。譙周《祭志》稱:『禮,身有喪,則不為吉祭。



 緦麻之喪,於祖考有服者,則亦不祭,為神不饗也。』尋宮中有故,雖在無服,亦廢祭三月,有喪不祭。如或非若三年之內必宜親奉者,則應禘序昭穆。而今必須免喪,然後禘袷,故知未祭之意,當似可思。《起居注》,晉武有二喪,兩期之中,並不
 自祠,亦近代前事也。伏惟至尊孝越姬文,情深明發,公服雖釋,純哀內纏。



 推訪典例,則未應親奉。有司祗應,祭不為曠。仰思從敬,竊謂為允。臣等參議,甚有明證,宜如所上。」詔可。



 後廢帝元徽二年十月丙寅,有司奏:「至尊親祠太廟文皇帝太后之日,孝武皇帝及昭皇太后,雖親非正統,而嘗經北面,未詳應親執爵與不?」下禮官議。太學博士周山文議:「案禮,尊者尊統上,卑者尊統下。孝武皇帝於至
 尊雖親非正統,而祖宗之號,列于七廟。愚謂親奉之日,應執觴爵。昭皇太后既親非禮正,宜使三公行事。」博士顏燮等四人同山文。兼太常丞韓賁議:「晉景帝之於世祖,肅祖之於孝武,皆傍尊也,親執觴杓。今孝武皇帝於至尊,親為伯父,功列祖宗,奉祠之日,謂宜親執。按昭皇太后於主上,親無名秩,情則疏遠,庶母在我,猶子祭孫止,況伯父之庶母。愚謂昭后觴爵,可付之有司。」前左丞孫緬議:「晉世祖宗祠顯宗、烈宗、肅祖,並是晉帝之伯,今
 朝明準,而初無有司行事之禮。愚謂主上親執孝武皇帝觴爵,有愜情敬。昭皇太后君母之貴,見尊一時,而與章、宣二廟同饗閟宮,非唯不躬奉,乃宜議其毀替。請且依舊,三公行事。」詔緬議為允。



 宋孝武帝孝建元年十月戊辰,有司奏章皇太后廟毀置之禮。二品官議者六百六十三人。太傅江夏王義恭以為:「經籍殘偽,訓傳異門,諒言之者罔一,故求之者鮮究。是以六宗之辯,舛於兼儒,迭毀之論,亂於群學。章皇
 太后誕神啟聖,禮備中興,慶流胙胤,德光義遠。宜長代崇芬,奕葉垂則。豈得降侔通倫,反遵常典。



 夫議者成疑,實傍紀傳,知一爽二,莫窮書旨。按《禮記》不代祭,爰及慈母,置辭令有所施。《穀梁》於孫止,別主立祭。則親執虔祀,事異前志。將由大君之宜,其職彌重,人極之貴,其數特中。且漢代鴻風,遂登配祔,晉氏明規,咸留薦祀。



 遠考史策,近因暗見,未應毀之,於義為長。所據《公羊》,祇足堅秉。安可以貴等帝王,祭從士庶,緣情訪制,顛越滋甚。謂應
 同七廟,六代乃毀。」六百三十六人同義恭不毀,散騎侍郎王法施等二十七人議應毀。領曹郎中周景遠重參議,義恭等不毀議為允。詔可。



 大明二年二月庚寅,有司奏:「皇代殷祭,無事於章后廟。高堂隆議魏文思后依周姜嫄廟禘袷,及徐邈答晉宣太后殷薦舊事,使禮官議正。」博士孫武議:「按《禮記祭法》,『置都立邑,設廟祧壇鸑而祭之,乃為親疏多少之數。是故王立七廟,遠廟為祧』。鄭云:『天子遷廟之主,昭穆合藏
 於二祧之中,袷乃祭之。』《王制》曰:『袷禘。』鄭云:『袷,合也。合先君之主於祖廟而祭之,謂之袷。



 三年而夏禘,五年而秋袷,謂之五年再殷祭。』又『禘,大祭也』。《春秋》文公二年,『大事于太廟』。《傳》曰:『毀廟之主,陳于太祖;未毀廟之主,皆升合食太祖。』《傳》曰:『合族以食,序以昭穆。』《祭統》曰:『有事于太廟,則群昭群穆咸在,不失其倫。』今殷祠是合食太祖,而序昭穆。章太后既屈於上,不列正廟。若迎主入太廟,既不敢配列於正序,又未聞於昭穆之外別立為位。若徐邈議,今
 殷祠就別廟奉薦,則乖禘袷大祭合食序昭穆之義。邈云:『陰室四殤,不同袷就祭。』此亦其義也。《喪服小記》,『殤與無後,從祖祔食』。《祭法》,『王下祭殤』。鄭玄云:『祭適殤於廟之奧,謂之陰厭。』既從祖食於廟奧,是殤有位於奧,非就祭別宮之謂。今章太后廟,四時饗薦,雖不於孫止,若太廟禘袷,獨祭別宮,與四時烝嘗不異,則非禘大祭之義,又無取於袷合食之文。謂不宜與太廟同殷祭之禮。高堂隆答魏文思后依姜嫄廟禘袷,又不辨袷之義,而改祫
 大饗,蓋有由而然耳。守文淺學,懼乖禮衷。」博士王燮之議:「按禘小袷大,禮無正文,求之情例如有。推尋袷之為名,雖在合食,而祭典之重,於此為大。夫以孝饗親,尊愛罔極,既殷薦太祖,亦致盛祀於小廟。譬有事於尊者,可以及卑。故高堂隆所謂獨以袷故而祭之也。是以魏之文思,晉之宣后,雖並不序於太廟,而猶均禘於姜嫄,其意如此。又徐邈所引四殤不袷,就而祭之,以為別饗之例,斯其證矣。愚謂章皇太后廟,亦宜殷薦。」太常丞孫緬
 議以為:「袷祭之名,義在合食,守經據古,孫武為詳。竊尋小廟之禮,肇自近魏,晉之所行,足為前準。高堂隆以袷而祭,有附情敬。徐邈引就祭四殤,以證別饗。孫武據殤祔於祖,謂廟有殤位。尋事雖同廟,而祭非合食。且七廟同宮,始自後漢,禮之祭殤,各附厥祖。既豫袷,則必異廟而祭。愚謂章廟殷薦,推此可知。」祠部朱膺之議:「閟宮之祀,高堂隆、趙怡並云周人袷,歲俱袷祭之。魏、晉二代,取則奉薦,名儒達禮,無相譏非,不愆不忘,率由舊章。愚意
 同王燮之、孫緬議。」詔曰:「章皇太后追尊極號,禮同七廟,豈容獨闕殷薦,隔茲盛祠。閟宮遙袷,既行有周,魏、晉從饗,式範無替。宜述附前典,以宣情敬。」



 明帝泰始二年正月,孝武昭太后崩。五月甲寅,有司奏:「晉太元中,始正太后尊號,徐邈議廟制,自是以來,著為通典。今昭皇太后於至尊無親,上特制義服,祔廟之禮,宜下禮官詳議。」博士王略、太常丞虞愿議:「正名存義,有國之徽典;臣子一例,史傳之明文。今昭皇太后正位母
 儀,尊號允著,祔廟之禮,宜備彞則。



 母以子貴,事炳聖文。孝武之祀,既百代不毀,則昭后之祔,無緣有虧。愚謂神主應入章后廟。又宜依晉元皇帝之於愍帝,安帝之於永安后,祭祀之日,不親執觴爵,使有司行事。」時太宗宣太后已祔章太后廟,長兼儀曹郎虞龢議以為:「《春秋》之義,庶母雖名同崇號,而實異正嫡。是以猶考別宮,而公子主其祀。今昭皇太后既非所生,益無親奉之理。《周禮》宗伯職云:『若王不與祭,則攝位。』然則宜使有司行其禮
 事。又婦人無常秩,各以夫氏為定,夫亡以子為次。昭皇太后即正位在前,宣太后追尊在後,以從序而言,宜躋新禰于上。」參詳,龢議為允。詔可。



 泰始二年六月丁丑,有司奏:「來七月嘗祀二廟,依舊車駕親奉。孝武皇帝室至尊親進觴爵及拜伏。又昭皇太后室應拜,及祝文稱皇帝諱。又皇后今月二十五日虔見於禰,拜孝武皇帝、昭皇太后,並無明文,下禮官議正。」太學博士劉緄議;「尋晉元北面稱臣於愍帝,烝嘗奉薦,亦
 使有司行事。且兄弟不相為後,著於魯史。



 以此而推,孝武之室,至尊無容親進觴爵拜伏。其日親進章皇太后廟,經昭皇太后室過,前議既使有司行事,謂不應進拜。昭皇太后正號久定,登列廟祀,詳尋祝文,宜稱皇帝諱。案禮,婦無見兄之典,昭后位居傍尊,致虔之儀,理不容備。孝武、昭后二室,牲薦宜闕。」太常丞虞願議:「夫烝嘗之禮,事存繼嗣,故傍尊雖近,弟侄弗祀。君道雖高,臣無祭典。按晉景帝之於武帝,屬居伯父,武帝至祭之日,猶進
 觴爵。今上既纂祠文皇,於孝武室謂宜進拜而已,觴爵使有司行事。按《禮》,『過墓則軾,過祀則下』。凡在神祇,尚或致恭;況昭太后母臨四海,至尊親曾北面,兄母有敬,謂宜進拜,祝文宜稱皇帝諱。尋皇后廟見之禮,本修虔為義,今於孝武,論其嫂叔,則無通問之典;語其尊卑,亦無相見之義。又皇后登御之初,昭后猶正位在宮,敬謁之道,久已前備。愚謂孝武、昭太后二室,並不復薦告。」參議以愿議為允。詔可。



 後廢帝元徽二年十月壬寅,有司奏昭太后廟毀置,下禮官詳議。太常丞韓賁議:「按君母之尊,義發《春秋》,庶後饗薦無間。周典七廟承統,猶親盡則毀。況伯之所生,而無服代祭,稽之前代,未見其準。」都令史殷匪子議:「昭皇太后不係於祖宗,進退宜毀。議者云,『妾祔於妾祖姑』,祔既必告,毀不容異。應告章皇太后一室。按《記》云:『妾祔於妾祖姑,無妾祖姑,則易牲而祔於女君可也。』始章太后於昭太后,論昭穆而言,則非妾祖姑,又非女君,於義不
 當。伏尋昭太后名位允極,昔初祔之始,自上祔於趙后,即安于西廟,並皆幣告諸室。古者大事必告,又云每事必告。禮,牲幣雜用。檢魏、晉以來,互有不同。元嘉十六年,下禮官辨正。太學博士殷靈祚議稱:『吉事用牲,凶事用幣。』自茲而後,吉凶為判,已是一代之成典。今事雖不全凶,亦未近吉,故宜依舊,以幣遍告二廟。又尋昭太后毀主,無義陳列於太祖,博士欲依虞主埋於廟兩階之間。按階間本以埋告幣埋虞主之所。昔虞喜云,依五經典
 議,以毀主附於虞主,埋於廟之北牆,最為可據。昭太后神主毀之埋之後,上室不可不虛置,太后便應上下升之。既升之頃,又應設脯醢以安神。今禮官所議,謬略未周。遷毀事大,請廣詳訪。」左僕射劉秉等七人同匪子。左丞王諶重參議,謂:「以幣遍告二廟,埋毀殷主於北牆。宣太后上室,仍設脯醢以安神,匪子議為允。」詔可。



 魏明帝太和三年,詔曰:「禮,王后無嗣,擇建支子以繼大宗,則當纂正統而奉公義,何得顧私親哉!漢宣繼昭帝,
 後加悼考以皇號;哀帝以外蕃援立,而董宏等稱引亡秦,或誤朝議。遂尊恭皇,立廟京師,又寵蕃妾,使比長信,僭差無禮,人神弗佑,非罪師丹忠正之諫,用致丁、傅焚如之禍。自是之後,相踵行之。其令公卿有司,深以前代為誡。後嗣萬一有由諸侯入奉大統,則當明為人後之義。敢為佞邪,導諛君上,妄建非正之號,謂考為皇,稱妣為后,則股肱大臣,誅之無赦。



 其書之金策,藏之宗廟,著于令典。」是後高貴、常道援立,皆不外尊也。



 晉愍帝建興四
 年,司徒梁芬議追尊之禮。帝既不從,而左僕射索綝等亦稱引魏制,以為不可,故追贈吳王為太保而已。元帝太興二年,有司言琅邪恭王宜稱皇考。



 賀循議云:「禮典之義,子不敢以己爵加其父號。」帝又從之。二漢此典棄矣。



 魏明帝有愛女曰淑涉,三月而夭,帝痛之甚,追封謚為平原懿公主,葬於南陵,立廟京師。無前典,非禮也。宋孝武帝孝建元年七月辛酉,有司奏:「東平沖王年稚
 無後,唯殤服五月。雖不殤君,應有主祭,而國是追贈,又無其臣。未詳毀靈立廟,為當它祔與不?輒下禮官詳議。」太學博士臣徐宏議:「王既無後,追贈無臣,殤服既竟,靈便合毀。《記》曰:『殤與無後者,從祖祔食。』又曰:『士大夫不得祔於諸侯,祔於祖之為士大夫者。』按諸侯不得祔於天子,沖王則宜祔諸祖之廟為王者,應祔付長沙景王廟。」詔可。



 大明四年丁巳,有司奏:「安陸國土雖建,而奠酹之所,未
 及營立,四時薦饗,故祔江夏之廟,宣王所生夫人當應祠不?」太學博士傅郁議:「應廢祭。」右丞徐爰議:「按《禮》,『慈母妾母不世祭』。鄭玄注:『以其非正,故傳曰子祭孫止。』又云:『為慈母後者,為祖庶母可也。』注稱:『緣為慈母後之義,父妾無子,亦可命己庶子為之後也。』考尋斯義,父母妾之祭,不必唯子。江夏宣王太子,體自元宰,道戚之胤,遭時不幸,聖上矜悼。降出皇愛,嗣承徽緒,光啟大蕃,屬國為祖。始王夫人載育明懿,則一國之正,上無所厭,哀敬得
 申。既未獲祔享江夏,又不從祭安陸,即事求情,愚以為宜依祖母有為後之義,謂合列祀于廟。」二議不同,參議以爰議為允。詔可。



 大明六年十月丙寅,有司奏:「故晉陵孝王子雲未有嗣,安廟後三日,國臣從權制除釋,朔望周忌,應還臨與不?祭之日,誰為主?」太常丞庾蔚之議:「既葬三日,國臣從權制除釋。而靈筵猶存,朔望及期忌,諸臣宜還臨哭,變服衣夾,使上卿主祭。王既未有後,又無三年服者,期親服
 除之,而國尚存,便宜立廟,為國之始祖。服除之日,神主暫祔食祖廟。諸王不得祖天子,宜祔從祖國廟,還居新廟之室。未有嗣之前,四時饗薦,常使上卿主之。」左丞徐爰參議,以蔚之議為允。



 詔可。



 大明七年正月庚子,有司奏:「故宣貴妃加殊禮,未詳應立廟與不?」太學博士虞龢議:「《曲禮》云:『天子有后,有夫人。』《檀弓》云:『舜葬蒼梧,三妃未之從。』《昏義》云:『后之立六宮,有三夫人。』然則三妃即三夫人也。后之有三妃,猶天子之
 有三公也。按《周禮》,三公八命,諸侯七命。三公既尊於列國諸侯,三妃亦貴於庶邦夫人。據《春秋傳》,仲子非魯惠元嫡,尚得考彼別宮。



 今貴妃是秩,天之崇班,理應立此新廟。」左丞徐爰議:「宣貴妃既加殊命,禮絕五宮,考之古典,顯有成據。廟堂克構,宜選將作大匠。」參詳以龢、爰議為允。



 詔可。



 大明七年三月戊戌,有司奏:「新安王服宣貴妃齊衰期,十一月練,十三月縞,十五月禫,心喪三年。未詳宣貴妃
 祔廟,應在何時?入廟之日,當先有祔,但入新廟而已?若在大祥及禫中入廟者,遇四時便得祭不?新安王在心制中,得親奉祭不?」



 太學博士虞龢議:「《春秋傳》云:『祔而作主,烝嘗禘於廟。』嘗為吉祭之名,大祥及禫,未得入廟,應在禫除之後也。新安王心喪之內,若遇時節,便應吉祭於廟,親奉亦在無嫌。祔之為言,以後亡者祔於先廟也。《小記》云:『諸侯不得祔於天子。』今貴妃爵視諸侯,居然不得祔於先后。又別考新宮,無所宜祔。且卒哭之後,益無祔
 理。」左丞徐爰議以:「禮有損益,古今異儀,雖云卒哭而祔,祔而作主,時之諸侯,皆禫終入廟。且麻衣縓緣,革服於元嘉,苫絰變除,申情於皇宋。



 況宣貴妃誕育睿蕃,葬加殊禮,靈筵廬位,皆主之哲王,考宮創祀,不得關之朝廷。



 謂禫除之後,宜親執奠爵之禮。若有故,三卿行事。貴妃上厭皇姑,下絕列國,無所應祔。」參議,龢議大體與爰不異,宜以爰議為允。詔可。



 大明七年十一月癸未,有司奏:「晉陵國刺:孝王廟依廬
 陵等國例,一歲五祭。



 二國以三卿主祭。應同有服之例與不?」博士顏僧道議:「《禮記》云:『所祭者亡服則不祭。』今晉陵王於衡陽小功,宜依二國同廢。」太常丞庾蔚之議:「緦不祭者,據主為言也。晉陵雖未有嗣,宜依有嗣致服,依闕祭之限。衡陽為族伯緦麻,則應祭三月。」兼左丞徐爰議:「嗣王未立,將來承胤未知疏近,豈宜空計服屬,以虧祭敬。」參議以爰議為允,詔可。



 大明八年正月壬辰,有司奏:「故齊敬王子羽將來立後,
 未詳便應作主立廟?



 為須有後之日?未立廟者,為於何處祭祀?」游擊將軍徐爰議以為:「國無後,於制除罷。始封之君,宜存繼嗣。皇子追贈,則為始祖。臣不殤君,事著前準,豈容虛闕烝嘗,以俟有後。謂宜立廟作主,三卿主祭依舊。」通關博議,以爰議為允。



 令便立廟,廟成作主,依晉陵王近例,先暫祔廬陵孝獻王廟。祭竟,神主即還新廟。



 未立後之前,常使國上卿主祭。



 《禮》云:「共工氏之霸九州,其子句龍曰后土,能平九土,故
 祀以為社。」



 周以甲日祭之,用日之始也。「社所以神地之道。地載萬物,天垂象。取財於地,取法於天。是以尊天而親地,故教人美報焉。家主中溜而國主社,示本也。」故言報本反始。烈山氏之有天下,其子曰農,能殖百穀。其裔曰柱,佐顓頊為稷官,主農事,周棄係之,法施於人,故祀以為稷。



 《禮》:「王為群姓立社曰太社,王自為立社曰王社。」故國有二社,而稷亦有二也。漢、魏則有官社,無稷,故常二社一稷也。晉初仍魏,無所增損。至太康九年,改建宗廟,
 而社稷壇與廟俱徙。乃詔曰:「社實一神,其并二社之禮。」於是車騎司馬傅咸表曰:「《祭法》二社各有其義。天子尊事郊廟,故冕而躬耕也者,所以重孝享之粢盛,致殷薦於上帝也。《穀梁傳》曰:『天子親耕以供粢盛。』親耕,謂自報,自為立社者,為籍而報也。國以人為本,人以穀為命,故又為百姓立社而祈報焉。事異報殊,此社之所以有二也。王景侯之論王社,亦謂春祈籍田,秋而報之也。其論太社,則曰『王者布下圻內,為百姓立之,謂之太社,不自
 立之於京師也』。景侯此論,據《祭法》,『大夫以下,成群立社,曰置社』。景侯解曰:『今之里社是也。』景侯解《祭法》,則以置社為人間之社矣。而別論復以太社為人間之社,未曉此旨也。太社,天子為民而祀,故稱天子社。《郊特牲》曰:『天子太社,必受霜露風雨。』夫以群姓之眾,王者通為立社,故稱太社。若夫置社,其數不一,蓋以里所為名。《左氏傳》盟于清丘之社是也。人間之社,既已不稱太矣。若復不立之京都,當安所立乎?《祭法》又曰:『王為群姓立七祀。自
 為立七祀。』言自為者,自為而祀也;為群姓者,為群姓而祀也。太社與七祀,其文正等。



 說者窮此,因云墳籍但有五祀,無七祀也。按祭五祀,國之大祀,七者小祀。《周禮》所云祭凡小祀,則墨冕之屬也。景侯解大厲曰:『如周杜伯,鬼有所歸,乃不為厲。』今云無二社者,稱景侯《祭法》不謂無二,則曰口傳無其文也。夫以景侯之明,擬議而後為解,而欲以口論除明文。如此,非但二社,當是思惟景侯之後解,亦未易除也。前被敕,《尚書召誥》:『社于新邑,唯一
 太牢,』不立二社之明義也。按《郊特牲》曰:『社稷太牢。』必援一牢之文,以明社之無二,則稷無牲矣。



 說者則曰,舉社以明稷。何獨不可舉一以明二。『國之大事,在祀與戎』。若有過而除之,不若過而存之。況存之有義,而除之無據乎。《周禮》封人『掌設社紘』。



 無稷字。今帝社無稷,蓋出於此。然國主社稷,故經傳動稱社稷。《周禮》,王祭稷則絺冕。此王社有稷之文也。封人設紘之無稷字,說者以為略文,從可知也。謂宜仍舊立二社,而加立帝社之稷。」



 時成粲議
 稱:「景侯論太社不立京都,欲破鄭氏學。」咸重表以為:「如粲之論,景侯之解文以此壞。《大雅》云:『乃立塚土。』毛公解曰:『塚土,太社也。』景侯解《詩》,即用此說。《禹貢》『惟土五色』。景侯解曰:『王者取五色土為太社,封四方諸侯。各割其方色土者覆四方也。』如此,太社復為立京都也。不知此論從何出而與解乖。上違經記明文,下壞景侯之解。臣雖頑蔽,少長學問,不能默已,謹復續上。」劉寔與咸議同。詔曰:「社實一神,而相襲二位,眾議不同,何必改作,其使仍舊,
 一如魏制。」至元帝建武元年,又依洛京立二社一稷。其太社之祝曰:「地德普施,惠存無疆。乃建太社,保佑萬邦。悠悠四海,咸賴嘉祥。」



 其帝社之祝曰:「坤德厚載,王畿是保。乃建帝社,以神地道。明祝惟辰,景福來造。」《禮》,左宗廟,右社稷,歷代遵之,故洛京社稷在廟之右,而江左又然也。



 吳時宮東門雩門,疑吳社亦在宮東,與其廟同所也。宋仍舊,無所改作。



 魏氏三祖皆親耕籍,此則先農無廢享也。其禮無異聞,
 宜從漢儀。執事告祠以太牢。晉武、哀帝並欲籍田而不遂,儀注亦闕略。宋文帝元嘉二十一年春,親耕,乃立先農壇於籍田中阡西陌南,高四尺,方二丈。為四出陛,陛廣五尺,外加埒。



 去阡陌各二十丈。車駕未到,司空、大司農率太祝令及眾執事質明以一太牢告祠。



 祭器用祭社稷器。祠畢,班餘胙於奉祠者。舊典先農又常列於郊祭云。



 漢儀,皇后親桑東郊苑中。蠶室祭蠶神曰:「苑灒婦人,寓
 氏公主。」祠用少牢。晉武帝太康九年,楊皇后躬桑于西郊,祀先蠶。壇高一丈,方二丈;為四出陛,陛廣五尺。在採桑壇東南帷宮之外,去帷宮十丈。皇后未到,太祝令質明以一太牢告祠。謁者一人監祠。畢,徹饌,班餘胙於從桑及奉祠者。



 魏文帝黃初二年六月庚子,初禮五嶽四瀆,咸秩群祀,瘞沈珪璋。六年七月,帝以舟軍入淮。九月壬戌,遣使者沈璧于淮,禮也。



 魏明帝太和四年八月,帝東巡,遣使者以特牛祠中嶽,禮也。魏元帝咸熙元年,帝行幸長安,遣使者以璧幣禮華山,禮也。晉穆帝升平中,何琦論修五嶽祠曰:「唐、虞之制,天子五載一巡狩,省時之方,柴燎五嶽,望于山川,遍于群神。故曰『因名山升中于天』。所以昭告神祇,饗報功德。是以災厲不作,而風雨寒暑以時。降逮三代,年數雖殊,而其禮
 不易。五嶽視三公,四瀆視諸侯,著在經記,所謂有其舉之,莫敢廢也。及秦、漢都西京,涇、渭長水,雖不在祀典,以近咸陽,故盡得比大川之祠。而正立之禮,可以闕哉!自永嘉之亂,神州傾覆,茲事替矣。



 唯灊之天柱,在王略之內,舊臺選百石吏卒,以奉其職。中興之際,未有官守,廬江郡常遣大吏兼假,四時禱賽,春釋寒而冬請冰。咸和迄今,已復墮替。計今非典之祠,可謂非一。考其正名,則淫昏之鬼;推其糜費,則四人之蠹。而山川大神,更為簡
 闕,禮俗頹紊,人神雜擾,公私奔蹙,漸以滋繁。良由頃國家多難,日不暇給,草建廢滯,事有未遑。今元憝已殲,宜修舊典。嶽瀆之域,風教所被,來蘇之人,咸蒙德澤,而神祇禋祀,未之或甄,巡狩柴燎,其廢尚矣。崇明前典,將俟皇輿北旋,稽古憲章,大厘制度。其五嶽、四瀆宜遵修之處,但俎豆牲牢,祝嘏文辭,舊章靡記。可令禮官作式,歸諸誠簡,以達明德馨香,如斯而已。其諸妖孽,可俱依法令,先去其甚。俾邪正不瀆。」不見省。



 宋孝武帝大明七年六月丙辰,有司奏;「詔奠祭霍山,未審應奉使何官?用何牲饌?進奠之日,又用何器?」殿中郎丘景先議:「修祀川嶽,道光列代;差秩珪璋,義昭聯冊。但業曠中葉,儀漏典文。尋姬典事繼宗伯,漢載持節侍祠,血祭埋沉,經垂明範,酒脯牢具,悉有詳例。又名山著珪幣之異,大塚有嘗禾之加。山海祠霍山,以太牢告玉,此準酌記傳,其可言者也。今皇風緬暢,輝祀通嶽,愚謂宜使以太常持節,牲以太牢之具,羞用酒脯時穀,禮以赤
 璋纁幣。又鬯人之職,『凡山川四方用蜃』,則盛酒當以蠡杯,其餘器用,無所取說。按郊望山瀆,以質表誠,器尚陶匏,籍以茅席,近可依準。山川以兆,宜為壇域。」參議景先議為允。令以兼太常持節奉使,牲用太牢,加以璋幣,器用陶匏,時不復用蜃,宜同郊祀,以爵獻。凡肴饌種數,一依社祭為允。詔可。



 晉武帝咸寧二年春,久旱。四月丁巳,詔曰:「諸旱處廣加祈請。」五月庚午,始祈雨于社稷山川。六月戊子,獲澍雨。
 此雩禜舊典也。太康三年四月、十年二月,又如之。是後,修之至今。



 魏文帝黃初二年正月,詔曰;「昔仲尼資大聖之才,懷帝王之器,當衰周之末,無受命之運,乃退考五代之禮,修素王之事,因魯史而制《春秋》,就太師而正《雅》、《頌》,俾千載之後,莫不宗其文以述作,仰其聖以成謀。茲可謂命世大聖,億載之師表者也。以遭天下大亂,百祀隳廢,舊居之廟,毀而不修,褒成之後,絕而莫繼,闕里不聞講頌之
 聲,四時不睹烝嘗之位,斯豈所謂崇化報功,盛德百世必祀者哉!其以議郎孔羨為宗聖侯,邑百戶,奉孔子祀。命魯郡修舊廟,置百戶吏卒,以守衛之。」晉武帝泰始三年十一月,改封宗聖侯孔震為奉聖亭侯。又昭太學及魯國四時備三牲,以祀孔子。明帝太寧三年,詔給事奉聖亭侯孔亭四時祠孔子,祭宜如泰始故事。亭五代孫繼之博塞無度,常以祭直顧進,替慢不祀。宋文帝元嘉八年,有司奏奪爵。至十九年,
 又授孔隱之。兄子熙先謀逆,又失爵。二十八年,更以孔惠云為奉聖侯。後有重疾,失爵。孝武大明二年,又以孔邁為奉聖侯。邁卒,子莽嗣,有罪,失爵。



 魏齊王正始二年三月,帝講《論語》通;五年五月,講《尚書》通;七年十二月,講《禮記》通;並使太常釋奠,以太牢祀孔子於辟雍,以顏淵配。晉武帝泰始七年,皇太子講《孝經》通;咸寧三年,講《詩》通;太康三年,講《論語》通。



 元帝太興三年,皇太子講《論語》通,
 太子並親釋奠,以太牢祠孔子,以顏淵配。



 成帝咸康元年,帝講《詩》通,穆帝升平元年三月,帝講《孝經》通;孝武寧康三年七月,帝講《孝經》通,並釋奠如故事。



 穆帝、孝武並權以中堂為太學。宋文帝元嘉二十二年四月,皇太子講《孝經》通,釋奠國子學,如晉故事。



 漢東海恭王薨,明帝出幸津門亭發哀。魏時會喪及使者弔祭,用博士杜希議,皆去玄冠,加以布巾。
 魏武帝少時,漢太尉橋玄獨先禮異焉。故建安中,遣使祠以太牢。文帝黃初六年十二月,過梁郡,又以太牢祠之。黃初二年正月,帝校獵至原陵,遣使者以太牢祠漢世祖。宋文帝元嘉二十五年四月丙辰,車駕行幸江寧,經司徒劉穆之墓,遣使致祭焉。孝武帝大明三年二月戊申,行幸籍田,經左光祿大夫袁湛墓,遣使致祭。
 大明五年九月庚午,車駕行幸,經司空殷景仁墓,遣使致祭。大明七年十一月,南巡。乙酉,遣使祭晉大司馬桓溫、征西將軍毛璩墓。



 劉禪景耀六年,詔為丞相諸葛亮立廟於沔陽。先是所居各請立廟,不許,百姓遂私祭之。而言事者或以為可立於京師,乃從人意,皆不納。步兵校尉習隆、中書侍郎向允等言於禪曰:「昔周人懷邵伯之美,甘棠為之不伐;越王思范蠡之功,鑄金以存其象。自漢興以來,小善小
 德,而圖形立廟者多矣;況亮德範遐邇,勛蓋季世,興王室之不壞,實斯人是賴。而烝嘗止於私門,廟象闕而莫立,百姓巷祭,戎夷野祀,非所以存德念功,述追在昔也。今若盡從人心,則瀆而無典;建之京師,又逼宗廟,此聖懷所以惟疑也。愚以為宜因近其墓,立之於沔陽,使屬所以時賜祭。



 凡其故臣欲奉祠者,皆限至廟。斷其私祀,以崇正禮。」於是從之。何承天曰:「《周禮》:『凡有功者祭於大烝。』故後代遵之,以元勳配饗。允等曾不是式,禪又從之,
 並非禮也。」



 漢時城陽國人以劉章有功於漢,為之立祠。青州諸郡,轉相放效,濟南尤盛。



 至魏武帝為濟南相,皆毀絕之。及秉大政,普加除翦,世之淫祀遂絕。至文帝黃初五年十一月,詔曰:「先王制祀,所以昭孝事祖,大則郊社,其次宗廟,三辰五行,名山川澤,非此族也,不在祀典。叔世衰亂,崇信巫史,至乃宮殿之內,戶牖之間,無不沃酹,甚矣其惑也。自今其敢設非禮之祭,巫祝之言,皆以執左道論,
 著于令。」



 明帝青龍元年,又詔:「郡國山川不在祀典者,勿祠。」



 晉武帝泰始元年十二月,詔:「昔聖帝明王,修五嶽、四瀆,名山川澤,各有定制。所以報陰陽之功,而當幽明之道故也。然以道蒞天下者,其鬼不神,其神不傷人也。故史薦而無媿詞,是以其人敬慎幽冥,而淫祀不作。末代信道不篤,僭禮瀆神,縱欲祈請,曾不敬而遠之,徒偷以求幸,妖妄相扇,舍正為邪,故魏朝疾之。



 其按舊禮,具為之
 制,使功著於人者,必有其報,而妖淫之鬼,不亂其間。」二年正月,有司奏:「春分祠厲殃及禳祠。」詔曰:「不在祀典,除之。」



 宋武帝永初二年,普禁淫祀。由是蔣子文祠以下,普皆毀絕。孝武孝建初,更修起蔣山祠,所在山川,漸皆修復。明帝立九州廟於雞籠山,大聚群神。蔣侯宋代稍加爵,位至相國、大都督、中外諸軍事,加殊禮,鐘山王。蘇侯驃騎大將軍。四方諸神,咸加爵秩。



 漢安帝元初四年,詔曰:「《月令》,『仲秋,養衰老,授幾杖,行糜鬻』。



 方今八月按比之時,郡縣多不奉行。雖有糜鬻,糠秕泥土相和半,不可飲食。」按此詔,漢時猶依《月令》施政事也。



\end{pinyinscope}