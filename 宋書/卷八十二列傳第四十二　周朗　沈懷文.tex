\article{卷八十二列傳第四十二 周朗 沈懷文}

\begin{pinyinscope}

 周朗,字義利,汝
 南安城人也。祖文,黃
 門侍郎。父淳,宋初貴達,官至侍中,太常。兄嶠,尚高祖第四女宣城德公主。二女適建平王宏、廬江王禕。以貴戚顯官,元嘉末,為吳
 興太守。賊劭弒立,隨王誕舉義於會稽,劭加嶠冠軍將軍,誕檄又至。



 嶠素懼怯,回惑不知所從,為府司馬丘珍孫所殺。朝庭明其本心,國婚如故。



 朗少而愛奇,雅有風氣,與嶠志趨不同,嶠甚疾之。初為南平王鑠冠軍行參軍,太子舍人,司徒主簿,坐請急不待對,除名。又為江夏王義恭太尉參軍。元嘉二十七年春,朝議當遣義恭出鎮彭城,為北討大統。朗聞之解職。及義恭出鎮,府主簿羊希從行,與朗書戲之,勸令獻奇進策。朗報書曰:羊生
 足下:豈當適使人進哉,何卿才之更茂也。宅生結意,可復佳耳,屬華比彩,何更工邪!視己反覆,慰亦無已。觀諸紙上,方審卿復逢知己。動以何術,而能每降恩明,豈不為足下欣邪,然更憂不知卿死所處耳。



 夫匈奴之不誅有日,皇居之亡辱舊矣。天下孰不憤心悲腸,以忿胡人之患,靡衣偷食,以望國家之師。自智士鉗口,雄人蓄氣,不得議圖邊之事者,良淹歲紀。



 今天子以炎、軒之德,塚輔以姬、呂之賢,故赫然發怒,將以匈奴釁旗,惻然動仁,
 欲使餘氓被惠。及取士之令朝發,宰士暮登英豪;調兵之詔夕行,主公旦升雄俊。



 延賢人者,固非一日,況復加此焉。



 夫天下之士,砥行磨名,欲不辱其志氣;選奇蓄異,將進善於所天。非但有建國之謀不及,安民之論不與,至反以孝潔生議於鄉曲,忠烈起謗於君寀。身不絓王臣之籙,名不廁通人之班。顛倒國門,湮銷丘里者,自數十年以往,豈一人哉!若吾身無他伎,而出值明君,變官望主,歲增恩價,竟不能柔心飾帶,取重左右。校於向士,
 則榮已多;料於今職,則笑亦廣。而足下方復廣吾以馳志之時,求予以安邊之術,何足下不知言也。若以賢未登,則今之登賢如此;以才應進,則吾之非才若是。豈可欲以殞海之鬐,望鼓鰓於豎鱗之肆;墜風之羽,覬振翮於軒毳之間。其不能俱陪淥水,並負青天,可無待於明見。若乃闕奇謀深智之術,無悅主狎俗之能,亦不可復稍為卿說。但觀以上國再毀之臣,望府一逐之吏,當復是天下才否,此皆足下所親知。



 吾雖疲冗,亦嘗聽君子
 之餘論,豈敢忘之。凡士之置身有三耳:一則雲戶岫寢,欒危桂榮,秣芝浮霜,翦松沈雪,憐肌蓄髓,寶氣愛魂,非但土石侯卿,腐鴆梁錦,實乃佇意天后,睨目羽人。次則刳心掃智,剖命驅生,橫議於雲臺之下,切辭於宣室之上,衍王德而批民患,進貞白而鴆姦猾,委玉入而齊聲禮,揭金出而烹勍寇,使車軌一風,甸道共德,令功日濟而己無跡,道日富而君難名,致諸侯斂手,天子改觀。其末則饜臺而出,望旃而入,結冤兩宮之下,鼓袖六王之
 間,俯眉脅肩,言天下之道德,瞋目扼腕,陳從橫於四海,理有泰則止而進,調覺迕則反而還,閑居違官,交造頓罷,捐慕遺憂,夷毀銷譽,呼吸以補其氣,繕嚼以輔其生。凡此三者,皆志士仁人之所行,非吾之所能也。



 若吾幸病不及死,役不至身,蓬藜既滿,方杜長者之轍;穀稼是諮,自絕世豪之顧。塵生床帷,苔積階月,又簷中山木,時華月深,池上海草,歲榮日蔓。且室間軒左,幸有陳書十篋,席隅奧右,頗得宿酒數壺。按絃拭徽,讎方校石,時復
 陳局露初,奠爵星晚,歡然不覺是羲、軒後也。近春田三頃,秋園五畦,若此無災,山裝可具。候振飲之罷,俟封勒之畢,當敬觀邠、酆,蕭尋伊、鄗,傍眺燕、隴,邪履遼、衛,覛我周之軫跡,弔他賢之憂天。當其少涉,未休此欲,但理實詭固,物好交加,或征勢而笑其言,或觀謀而害其意。夫楊朱以此,猶見嗤於梁人,況才減楊子之器,物甚魏君之意者哉!若如漢宗之言李廣,此固許天下之有才,又知天下之時非也。豈若黨巷閭里之間,忌見貞士之遭
 遇,便謂是臧獲庸人之徒耳。士固願呈心於其主,露奇於所歸。卿相,末事也。若廣者,何用侯為。至乃復有致謁於為亂之日,被訕於害正之徒,心奇而無由露,事直而變為枉,豈不痛哉!豈不痛哉!



 若足下可謂冠負日月,籍踐淵海,心支身首,無不通照。今復出入燕、河,交關姬、衛,整笏振豪,已議於帷筵之上,提鞭鳴劍,復呵於軍場之間,身超每深恩之所集,心動必明主之所亮。可不直議正身,輔人君之過誤。明目張膽,謀軍家之得失,操志勇
 之將,薦俊正之士,此乃足下之所以報也。不爾,便擐甲修戈,徘徊左右,衛君王之身,當馬首之鏑,關必固之壘,交死進之戰,使身分而主豫,寇滅而兵全,此亦報之次也。如是,則繫匈奴於北闕無日矣。亡但默默,窺寵而坐。謂子有心,敢書薄意。



 朗之辭意倜儻,類皆如此。復起為通直郎。世祖即位,除建平王宏中軍錄事參軍。時普責百官讜言,朗上書曰:昔仲尼有言:「治天下若寘諸掌。」豈徒言哉!方策之政,息舉在人,蓋當世之君不為之耳。況
 乃運鐘澆暮,世膺亂餘,重以宮廟遭不更之酷,江服被未有之痛,千里連死,萬井共泣。而秦、漢餘敝,尚行於今,魏、晉遺謬,猶布於民,是而望國安於今,化崇於古,卻行及前之言,積薪待然之譬,臣不知所以方。然陛下既基之以孝,又申之以仁,民所疾苦,敢不略薦。



 凡治者何哉?為教而已。今教衰已久,民不知則,又隨以刑逐之,豈為政之道歟!欲為教者,宜二十五家選一長,百家置一師,男子十三至十七,皆令學經;十八至二十,盡使修武。訓
 以書記圖律,忠孝仁義之禮,廉讓勤恭之則;授以兵經戰略,軍部舟騎之容,挽彊擊刺之法。官長皆月至學所,以課其能。習經者五年有立,則言之司徒;用武者三年善藝,亦升之司馬。若七年而經不明,五年而勇不達,則更求其言政置謀,跡其心術行履,復不足取者,雖公卿子孫,長歸農畝,終身不得為吏。其國學則宜詳考占數,部定子史,令書不煩行,習無糜力。凡學,雖凶荒不宜廢也。



 農桑者,實民之命,為國之本,有一不足,則禮節不興。
 若重之,宜罷金錢,以穀帛為賞罰。然愚民不達其權,議者好增其異。凡自淮以北,萬匹為市;從江以南,千斛為貨,亦不患其難也。今且聽市至千錢以還者用錢,餘皆用絹布及米,其不中度者坐之。如此,則墾田自廣,民資必繁,盜鑄者罷,人死必息。又田非疁水,皆播麥菽,地堪滋養,悉藝珝麻,蔭巷緣籓,必樹桑柘,列庭接宇,唯植竹栗。若此令既行,而善其事者,庶民則敘之以爵,有司亦從而加賞。若田在草間,木物不植,則撻之而伐其餘樹,
 在所以次坐之。



 又取稅之法,宜計人為輸,不應以貲。云何使富者不盡,貧者不蠲。乃令桑長一尺,圍以為價,田進一畝,度以為錢,屋不得瓦,皆責貲實。民以此,樹不敢種,土畏妄墾,棟焚榱露,不敢加泥。豈有剝善害民,禁衣惡食,若此苦者。方今若重斯農,則宜務削茲法。



 凡為國,不患威之不立,患恩之不下;不患土之不廣,患民之不育。自華、夷爭殺,戎、夏競威,破國則積屍竟邑,屠將則覆軍滿野,海內遺生,蓋不餘半。重以急政嚴刑,天災歲疫,
 貧者但供吏,死者弗望霾,鰥居有不願娶,生子每不敢舉。



 又戍淹徭久,妻老嗣絕,及淫奔所孕,皆復不收。是殺人之日有數途,生人之歲無一理,不知復百年間,將盡以草木為世邪?此最是驚心悲魂慟哭太息者。法雖有禁殺子之科,設蚤娶之令,然觸刑罪,忍悼痛而為之,豈不有酷甚處邪!今宜家寬其役,戶減其稅。女子十五不嫁,家人坐之。特雉可以娉妻妾,大布可以事舅姑,若待足而行,則有司加糾。凡宮中女隸,必擇不復字者。庶家
 內役,皆令各有所配。



 要使天下不得有終獨之生,無子之老。所謂十年存育,十年教訓,如此,則二十年間,長戶勝兵,必數倍矣。



 又亡者亂郊,饉人盈甸,皆是不為其存計,而任之遷流,故饑寒一至,慈母不能保其子,欲其不為寇盜,豈可得邪?既御之使然,復止之以殺,彼於有司,何酷至是!且草樹既死,皮葉皆枯,是其梁肉盡矣。冰霜已厚,苫蓋難資,是其衣裘敗矣。比至陽春,生其餘幾。今自江以南,在所皆穰,有食之處,須官興役,宜募遠近能
 食五十口一年者,賞爵一級。不過千家,故近食十萬口矣。使其受食者,悉令就佃淮南,多其長帥,給其糧種。凡公私遊手,歲發佐農,令堤湖盡修,原陸並起。



 仍量家立社,計地設閭,檢其出入,督其游惰。須待大熟,可移之復舊。淮以北悉使南過江,東旅客盡令西歸。



 故毒之在體,必割其緩處,函、渭靈區,闃為荒窟,伊、洛神基,蔚成茂草,豈可不懷歟?歷下、泗間,何足獨戀。議者必以為胡衰不足避,而不知我之病甚於胡矣!若謂民之既徙,狄必就
 之,若其來從,我之願也。胡若能來,必非其種,不過山東雜漢,則是國家由來所欲覆育。既華得坐實,戎空自遠,其為來,利固善也。



 今空守孤城,徒費財役,亦行見淮北必非境服有矣,不亦重辱喪哉!使虜但發輕騎三千,更互出入,春來犯麥,秋至侵禾,水陸漕輸,居然復絕。於賊不勞,而邊已困,不至二年,卒散民盡,可蹻足而待也。設使胡滅,則中州必有興者,決不能有奉土地、率民人以歸國家矣。誠如此,則徐、齊終逼,亦不可守。



 且夫戰守之
 法,當恃人之不敢攻。頃年兵之所以敗,皆反此也。今人知不以羊追狼,蟹捕鼠,而令重車弱卒,與肥馬悍胡相逐,其不能濟,固宜矣。漢之中年能事胡者,以馬多也;胡之後服漢者,亦以馬少也。既兵不可去,車騎應蓄。今宜募天下使養馬一匹者,蠲一人役。三匹者,除一人為吏。自此以進,階賞有差,邊亭徼驛,一無發動。



 又將者,將求其死也。自能執干戈,幸而不亡,筋力盡於戎役,其於望上者,固已深矣。重有澄風掃霧之勤,驅波滌塵之力,此
 所自矜,尤復為甚。近所功賞,人知其濃,然似頗謬虛實,怨怒實眾。垂臂而反脣者,往往為部,耦語而呼望者,處處成群。凡武人意氣,特易崩沮,設一旦有變,則向之怨者皆為敵也。今宜國財與之共竭,府粟與之同罄,去者應遣,濃加寵爵,發所在祿之,將秩未充,餘費宜闕,他事負輦,長不應與,唯可教以搜狩之禮,習以鉦鼓之節。若假勇以進,務黜其身。老至而罷,賞延於嗣。



 又緣淮城壘,皆宜興復,使烽鼓相達,兵食相連。若邊民請師,皆宜莫許。
 遠夷貢至,止於報答,語以國家之未暇,示以何事而非君。須內教既立,徐料寇形,辦騎卒四十萬,而國中不擾,取穀支二十歲,而遠邑不驚,然後越淮窮河,跨隴出漠,亦何適而不可。



 又教之不敦,一至於是。今士大夫以下,父母在而兄弟異計,十家而七矣。庶人父子殊產,亦八家而五矣。凡甚者,乃危亡不相知,飢寒不相恤,又嫉謗讒害,其間不可稱數。宜明其禁,以革其風,先有善於家者,即務其賞;自今不改,則沒其財。



 又三年之喪,天下之
 達喪,以其哀並衷出,故制同外興;日久均痛,故愈遲齊典。漢氏節其臣則可矣,薄其子則亂也。云何使衰苴之容盡,鳴號之音息。夫佩玉啟旒,深情弗忍,冕珠視朝,不亦甚乎!凡法有變於古而刻於情,則莫能順焉。至乎敗於禮而安於身,必遽而奉之,何乃厚於惡,薄於善歟!今陛下以大孝始基,宜反斯謬。



 且朝享臨御,當近自身始,妃主典制,宜漸加矯正。凡舉天下以奉一君,何患不給。或帝有集皁之陋,后有帛布之鄙,亦無取焉。且一體炫
 金,不及百兩,一歲美衣,不過數襲,而必收寶連櫝,集服累笥,目豈常視,身未時親,是為櫝帶寶,笥著衣,空散國家之財,徒奔天下之貨。而主以此惰禮,妃以此傲家,是何糜蠹之劇,惑鄙之甚!逮至婢豎,皆無定科,一婢之身,重婢以使,一豎之家,列豎以役。



 瓦金皮繡,漿酒藿肉者,故不可稱紀。至有列軿以遊遨,飾兵以驅叱,不亦重甚哉!



 若禁行賜薄,不容致此。且細作始并,以為儉節,而市造華怪,即傳於民。如此,則遷也,非罷也。凡天下得治者
 以實,而治天下者常虛,民之耳目,既不可誑,治之盈耗,立亦隨之。故凡厥庶民,制度日侈,商販之室,飾等王侯,傭賣之身,製均妃后。凡一袖之大,足斷為兩,一裾之長,可分為二;見車馬不辨貴賤,視冠服不知尊卑。尚方今造一物,小民明已睥睨。宮中朝制一衣,庶家晚已裁學。侈麗之原,實先宮閫。又妃主所賜,不限高卑,自今以去,宜為節目。金魄翟玉,錦繡縠羅,奇色異章,小民既不得服,在上亦不得賜。若工人復造奇伎淫器,則皆焚之,而
 重其罪。



 又置官者,將以燮天平氣,贊地成功,防姦御難,治煩理劇,使官稱事立,人稱官置,無空樹散位,繁進冗人。今高卑貿實,大小反稱,名之不定,是謂官邪。



 而世廢姬公之制,俗傳秦人之法,惡明君之典,好闇主之事,其憎聖愛愚,何其甚矣。今則宜先省事,從而并官,置位以周典為式,變名以適時為用,秦、漢末制,何足取也。當使德厚者位尊,位尊者祿重;能薄者官賤,官賤者秩輕。纓冕紱佩,稱官以服;車騎容衛,當職以施。



 又寄土州郡,宜
 通廢罷,舊地民戶,應更置立。豈吳邦而有徐邑,揚境而宅兗民,上淆辰紀,下亂畿甸。其地如朱方者,不宜置州,土如江都者,應更建邑。



 又民少者易理,君近者易歸,凡吏皆宜每詳其能,每厚其秩,為縣不得復用恩家之貧,為郡不得復選勢族之老。



 又王侯識未堪務,不應彊仕,須合冠而啟封,能政而議爵。且帝子未官,人誰謂賤。但宜詳置賓友,選擇正人,亦何必列長史、參軍、別駕、從事,然後為貴哉!



 又世有先後,業有難易,明帝能令其兒不
 匹光武之子,馬貴人能使其家不比陰后之族。盛矣哉,此於後世不可忘也。至當輿抑碎首之忿,陛殿延辟戟之威,此亦復不可忘也。



 內外之政,實不可雜。若妃主為人請官者,其人宜終身不得為官;若請罪者,亦終身不得赦罪。



 凡天下所須者才,而才誠難知也。有深居而言寡,則蘊學而無由知;有卑處而事隔,則懷奇而無由進。或復見忌於親故,或亦遭讒於貴黨,其欲致車右而動御席,語天下而辯治亂,焉可得哉!漫言舉賢,則斯人固
 未得矣。宜使世之所稱通經達史、辨詞精數、吏能將謀、偏術小道者,使獵纓危膝,博求其用。制內外官與官之遠近及仕之類,令各以所能而造其室,降情以誘之,卑身以安之。然後察其擢脣吻,樹頰胲,動精神,發意氣,語之所至,意之所執,不過數四間,不亦盡可知哉!若忠孝廉清之比,彊正惇柔之倫,難以檢格立,不可須臾定。宜使鄉部求其行,守宰察其能,竟皆見之於選貴,呈之於相主,然後處其職宜,定其位用。如此,故應愚鄙盡捐,賢
 明悉舉矣。又俗好以毀沈人,不知察其所以致毀;以譽進人,不知測其所以致譽。毀徒皆鄙,則宜擢其毀者;譽黨悉庸,則宜退其譽者。如此,則毀譽不妄,善惡分矣。又既謂之才,則不宜以階級限,不應以年齒齊。凡貴者好疑人少,不知其少於人矣。老者亦輕人少,不知其不及少矣。



 自釋氏流教,其來有源,淵檢精測,固非深矣。舒引容潤,既亦廣矣。然習慧者日替其修,束誡者月繁其過,遂至糜散錦帛,侈飾車從。復假精醫術,託雜卜數,延妹
 滿室,置酒浹堂,寄夫託妻者不無,殺子乞兒者繼有。而猶倚靈假像,背親傲君,欺費疾老,震損宮邑,是乃外刑之所不容戮,內教之所不悔罪,而橫天地之間,莫不糾察。人不得然,豈其鬼歟!今宜申嚴佛律,裨重國令,其疵惡顯著者,悉皆罷遣,餘則隨其藝行,各為之條,使禪義經誦,人能其一,食不過蔬,衣不出布。



 若應更度者,則令先習義行,本其神心,必能草腐人天,竦精以往者,雖侯王家子,亦不宜拘。



 凡鬼道惑眾,妖巫破俗,觸木而言怪
 者不可數,寓采而稱神者非可算。其原本是亂男女,合飲食,因之而以祈祝,從之而以報請,是亂不誅,為害未息。凡一苑始立,一神初興,淫風輒以之而甚。今修隄以北,置園百里,峻山以右,居靈十房,糜財敗俗,其可稱限。又針藥之術,世寡復修,診脈之伎,人鮮能達。民因是益徵於鬼,遂棄於醫,重令耗惑不反,死夭復半。今太醫宜男女習教,在所應遣吏受業。



 如此,故當愈於媚神之愚,征正腠理之敝矣。



 凡無世不有言事,未時不有令下,然
 而升平不至,昏危是繼,何哉?蓋設令之本非實也。又病言不出於謀臣,事不便於貴黨,輕者抵訾呵駭,重者死壓窮擯,故西京有方調之誅,東郡有黨錮之戮。陛下若欲申常令,循末典,則群臣在焉;若欲改舊章,興王道,則微臣存矣。敢昧死以陳,唯陛下察之。



 書奏,忤旨,自解去職。又除太子中舍人,出為廬陵內史。郡後荒蕪,頻有野獸,母薛氏欲見獵,朗乃合圍縱火,令母觀之。火逸燒郡廨,朗悉以秩米起屋,償所燒之限,稱疾去官,遂為州司
 所糾。還都謝世祖曰:「州司舉臣愆失,多有不允。



 臣在郡,虎三食人,蟲鼠犯稼,以此二事上負陛下。」上變色曰:「州司不允,或可有之。蟲虎之災,寧關卿小物。」朗尋丁母艱,有孝性,每哭必慟,其餘頗不依居喪常節。大明四年,上使有司奏其居喪無禮,請加收治。詔曰:「朗悖禮利口,宜令翦戮,微物不足亂典刑,特鎖付邊郡。」於是傳送寧州,於道殺之,時年三十六。子仁昭,順帝昇明末,為南海太守。



 沈懷文,字思明,吳興武康人也。祖寂,晉光祿勳。父宣,新安太守。懷文少好玄理,善為文章,嘗為楚昭王二妃詩,見稱於世。初州辟從事,轉西曹,江夏王義恭司空行參軍,隨府轉司徒參軍事,東閣祭酒。丁父憂,新安郡送故豐厚,奉終禮畢,餘悉班之親戚,一無所留。太祖聞而嘉之,賜奴婢六人。服闋,除尚書殿中郎。隱士雷次宗被徵居鐘山,後南還廬岳,何尚之設祖道,文義之士畢集,為連句詩,懷文所作尤美,辭高一座。以公事例免,同輩皆
 失官,懷文乃獨留。隨王誕鎮襄陽,出為後軍主簿,與諮議參軍謝莊共掌辭令,領義成太守。元嘉二十八年,誕當為廣州,欲以懷文為南府記室,先除通直郎,懷文固辭南行,上不悅。



 弟懷遠納東陽公主養女王鸚鵡為妾。元凶行巫蠱,鸚鵡預之,事泄,懷文因此失調,為治書侍御史。元凶弒立,以為中書侍郎。世祖入討,劭呼之使作符檄,懷文固辭,劭大怒,投筆於地曰:「當今艱難,卿欲避事邪!」旨色甚切。值殷沖在坐,申救得免。託疾落馬,間行
 奔新亭。以為竟陵王誕衛軍記室參軍、新興太守。



 又為誕驃騎錄事參軍、淮南太守。時國哀未釋,誕欲起內齋,懷文以為不可,乃止。



 尋轉揚州治中從事史。



 時議省錄尚書,懷文以為非宜,上議曰:「昔天官正紀,六典序職,載師掌均,七府成務,所以翼平辰衡,經贊邦極。故總屬之原,著夫官典,和統之要,昭于國言。夏因虞禮,有深冢司之則;周承殷法,無損掌邦之儀。用乃調佐王均,緝亮帝度。而式憲之軌,弘正漢庭;述章之範,崇明魏室。雖條錄
 之名,立稱於中代,總釐之實,不愆於自古,比代相沿,歷朝罔貳。及乎爵以事變,級以時改,皆興替之道,無害國章,八統元任,靡或省革。按台輔之職,三曰禮典,以和邦國,以統百官。四曰政典,以平邦國,以正百官。鄭康成云『冢宰之於庶僚,無所不總也。』考于茲義,備於典文,詳古準今,不宜虛廢。」不從。遷別駕從事史,江夏王義恭遷,西陽王子尚為揚州,居職如故。



 時熒惑守南斗,上乃廢西州舊館,使子尚移居東城以厭之。懷文曰:「天道示變,宜
 應之以德。今雖空西州,恐無益也。」不從,而西州竟廢矣。大明二年,遷尚書吏部郎。時朝議欲依古制置王畿,揚州移治會稽,猶以星變故也。懷文曰:「周制封畿,漢置司隸,各因時宜,非存相反,安民寧國,其揆一也。茍民心所安,天亦從之,未必改今追古,乃致平壹。神州舊壤,歷代相承,異於邊州,或罷或置,既物情不說,容虧化本。」又不從。三年,子尚移鎮會稽,遷撫軍長史,行府州事。



 時囚系甚多,動經年月,懷文到任,訊五郡九百三十六獄,眾咸稱
 平。



 入為侍中,寵待隆密,將以為會稽,其事不行。竟陵王誕據廣陵反,及城陷,士庶皆裸身鞭面,然後加刑,聚所殺人首於石頭南岸,謂之髑髏山。懷文陳其不可,上不納。揚州移會稽,上忿浙江東人情不和,欲貶其勞祿,唯西州舊人不改。懷文曰:「揚州徒治,既乖民情,一州兩格,尤失大體。臣謂不宜有異。」上又不從。



 懷文與顏竣、周朗素善,竣以失旨見誅,朗亦以忤意得罪,上謂懷文曰:「竣若知我殺之,亦當不敢如此。」懷文默然。嘗以歲夕與謝
 莊、王景文、顏師伯被敕入省,未及進,景文因言次稱竣、朗人才之美,懷文與相酬和,師伯後因語次白上,敘景文等此言。懷文屢經犯忤,至此上倍不說。上又壞諸郡士族,以充將吏,並不服役,至悉逃亡,加以嚴制不能禁。乃改用軍法,得便斬之,莫不奔竄山湖,聚為盜賊。懷文又以為言。齋庫上絹,年調鉅萬匹,綿亦稱此。期限嚴峻,民間買絹一匹,至二三千,綿一兩亦三四百,貧者賣妻兒,甚者或自縊死。懷文具陳民困,由是綿絹薄有所減,
 俄復舊。子尚諸皇子皆置邸舍,逐什一之利,為患遍天下。懷文又言之曰:「列肆販賣,古人所非,故卜式明不雨之由,弘羊受致旱之責。若以用度不充,頓止為難者,故宜量加減省。」不聽。



 孝建以來,抑黜諸弟,廣陵平後,復欲更峻其科。懷文曰:「漢明不使其子比光武之子,前史以為美談。陛下既明管、蔡之誅,願崇唐、衛之寄。」及海陵王休茂誅,欲遂前議,太宰江夏王義恭探得密旨,先發議端,懷文固謂不可,由是得息。



 時游幸無度,太后及六宮
 常乘副車在後,懷文與王景文每陳不宜亟出。後同從坐松樹下,風雨甚驟。景文曰:「卿可以言矣。」懷文曰:「獨言無係,宜相與陳之。」江智淵臥草側,亦謂言之為善。俄而被召俱入雉場,懷文曰:「風雨如此,非聖躬所宜冒。」景文又曰:「懷文所啟宜從。」智淵未及有言,上方注弩,作色曰:「卿欲效顏竣邪?何以恒知人事。」又曰:「顏竣小子,恨不得鞭其面!」上每宴集,在坐者咸令沈醉,懷文素不飲酒,又不好戲調,上謂故欲異己。謝莊嘗誡懷文曰:「卿每與人
 異,亦何可久。」懷文曰:「吾少來如此,豈可一朝而變。非欲異物,性所得耳。」



 五年,乃出為晉安王子勛征虜長史、廣陵太守。明年,坐朝正,事畢,被遣還北,以女病求申。臨辭,又乞停三日,訖猶不去。為有司所糾,免官,禁錮十年,既被免,買宅欲還東。上大怒,收付廷尉,賜死,時年五十四。三子:淡、淵、沖。



 弟懷遠,為始興王浚征北長流參軍,深見親待。坐納王鸚鵡為妾,世祖徙之廣州,使廣州刺史宗愨於南殺之。會南郡王義宣反,懷遠頗閑文筆,愨起義,
 使造檄書,并銜命至始興,與始興相沈法系論起義事。事平,愨具為陳請,由此見原;終世祖世不得還。懷文雖親要,屢請終不許。前廢帝世,流徙者並聽歸本,官至武康令。撰《南越志》及懷文文集,並傳於世。



 史臣曰:昔婁敬戍卒,委輅而遷帝都;馮唐老賤,片詞以悟明主。素無王公卿士之貴,非有積譽取信之資,徒以一言合旨,仰感萬乘。自此山壑草萊之人,布衣韋帶之士,莫不踵闕縣書,煙霏霧集。自漢至魏,此風未爽。暨于
 晉氏,浮偽成俗,人懷獨善,仕貴遺務。降及宋祖,思反前失,雖革薄捐華,抑揚名教,而闢聰之路未啟,採言之制不弘。至於賤隸卑臣,義合朝算,徒以事非己出,知允莫從。昔之開之若彼,今之塞之若此,非為徐樂、嚴安,偏富漢世,東方、主父,獨闕宋時,蓋由用與不用也。徒置乞言之旨,空下不諱之令,慕古飾情,義非側席,文士因斯,各存炫藻。周朗辯博之言,多切治要,而意在摛詞,文實忤主。文詞之為累,一至此乎!



\end{pinyinscope}