\article{卷二本紀第二 武帝中}

\begin{pinyinscope}

 七年正月己未,振旅於京師,改授大將軍、揚州牧,給班劍二十人,本官悉如故,固辭。凡南北征伐戰亡者,並列上賻贈。尸喪未反,遣主帥迎接,致還本土。



 二月,盧循至
 番禺,為孫季高所破,收餘眾南走。劉籓、孟懷玉斬徐道覆于始興。



 晉自中興以來,治綱大弛,權門並兼,彊弱相凌,百姓流離,不得保其產業。



 桓玄頗欲釐改,竟不能行。公既作輔,大示軌則,豪彊肅然,遠近知禁。至是,會稽餘姚虞亮復藏匿亡命千餘人,公誅亮,免會稽內史司馬休之。



 天子又申前命,公固辭。於是改授太尉、中書監,乃受命。奉送黃鉞,解冀州。



 交州刺史杜慧度斬盧循,傳首京師。先是,諸州郡所遣秀才、孝廉,多非其人,公表天子,
 申明舊制,依舊策試。



 征西將軍、荊州刺史道規疾患求歸。八年四月,改授豫州刺史,以後將軍、豫州刺史劉毅代之。毅與公俱舉大義,興復晉室,自謂京城、廣陵,功業足以相抗。



 雖權事推公,而心不服也。毅既有雄才大志,厚自矜許,朝士素望者多歸之。與尚書僕射謝混、丹陽尹郗僧施並深相結。及西鎮江陵,豫州舊府,多割以自隨,請僧施為南蠻校尉。既知毅不能居下,終為異端,密圖之。毅至西,稱疾篤,表求從弟兗州刺史籓以為副貳,
 偽許焉。九月,籓入朝,公命收籓及謝混,並於獄賜死。自表討毅,又假黃鉞,率諸軍西征。以前鎮軍將軍司馬休之為平西將軍、荊州刺史,兗州刺史道憐鎮丹徒,豫州刺史諸葛長民監太尉留府事,加太尉司馬,丹陽尹劉穆之建威將軍,配以實力。壬午,發自京師。遣參軍王鎮惡、龍驤將軍蒯恩前襲江陵。



 十月,鎮惡剋江陵,毅及黨與皆伏誅。十一月己卯,公至江陵,下書曰:夫去弊拯民,必存簡恕,舍網脩綱,雖煩易理。江、荊凋殘,刑政多闕;頃
 年事故,綏撫未周。遂令百姓疲匱,歲月滋甚,財傷役困,慮不幸生。凋殘之餘,而不減舊,刻剝徵求,不循政道。宰蒞之司,或非良幹,未能菲躬儉,茍求盈給,積習生常,漸不知改。



 近因戎役,來涉二州,踐境親民,愈見其瘼;思欲振其所急,恤其所苦。凡租稅調役,悉宜以見戶為正。州郡縣屯田池塞,諸非軍國所資,利人守宰者,今一切除之。州郡縣吏,皆依尚書定制實戶置。臺調癸卯梓材,庚子皮毛,可悉停省,別量所出。巴陵均折度支,依舊兵運。
 原五歲刑已下,凡所質錄賊家餘口,亦悉原放。



 以荊州十郡為湘州,公乃進督,以西陽太守朱齡石為益州刺史,率眾伐蜀。進公太傅、揚州牧,加羽葆鼓吹,班劍二十人。



 九年二月乙丑,公至自江陵。初,諸葛長民貪淫驕橫,為士民所患苦。公以其同大義,優容之。劉毅既誅,長民謂所親曰:「昔年醢彭越,今年誅韓信,禍其至矣。」將謀作亂。公克期至京邑,而每淹留不進,公卿以下頻日奉候於新亭,長民亦驟出。既而公輕舟密至,已還東府矣。長
 民到門,引前,卻人閑語,凡平生於長民所不盡者,皆與及之;長民甚說。已密命左右壯士丁旿等自幔後出,於坐拉焉。



 長民墜床,又於地毆之,死於床側。輿尸付廷尉;并誅其弟黎民。午驍勇有氣力,時人為之語曰:「勿跋扈,付丁旿。」



 先是,山湖川澤,皆為豪彊所專,小民薪採漁釣,皆責稅直,至是禁斷之。時民居未一,公表曰:臣聞先王制治,九土攸序;分境畫疆,各安其居;在昔盛世,人無遷業,故井田之制,三代以隆。秦革斯政,漢遂不改;富彊兼
 并,於是為弊。然九服弗擾,所託成舊,在漢西京,大遷田、景之族,以實關中,即以三輔為鄉閭,不復係之於齊、楚。自永嘉播越,爰托淮、海,朝有匡復之算,民懷思本之心,經略之圖,日不暇給。是以寧民綏治,猶有未遑。及至大司馬桓溫,以民無定本,傷治為深,庚戌土斷,以一其業。于時財阜國豐,實由於此。自茲迄今,彌歷年載,畫一之制,漸用頹弛。雜居流寓,閭伍弗脩,王化所以未純,民瘼所以猶在。



 臣荷重任,恥責實深,自非改調解張,無以濟
 治。夫人情滯常,難與慮始,所謂父母之邦以為桑梓者,誠以生焉終焉,敬愛所託耳。今所居累世,墳壟成行,敬恭之誠,豈不與事而至。請準庚戌土斷之科,庶子本所弘,稍與事著。然後率之以仁義,鼓之以威武,超大江而跨黃河,撫九州而復舊土,則戀本之志,乃速由於當年,在始暫勤,要終所以能易。伏惟陛下,垂矜萬民,憐其所失,永懷《鴻鴈》之詩,思隆中興之業。既委臣以國重,期臣以寧濟,若所啟合允,請付外施行。



 於是依界土斷,唯徐、
 兗、青三州居晉陵者,不在斷例。諸流寓郡縣,多被併省。以公領鎮西將軍、豫州刺史。公固讓太傅、州牧及班劍,奉還黃鉞。七月,朱齡石平蜀,斬偽蜀王譙縱,傳首京師。九月,封公次子義真為桂陽縣公,以賞平齊及定盧循也。天子重申前命,授公太傅、揚州牧,加羽葆、鼓吹、班劍二十人。將吏百餘敦勸,乃受羽葆、鼓吹、班劍,餘固辭。十年,息民簡役。築東府,起府舍。



 平西將軍、荊州刺史司馬休之,宗室之重,又得江漢人心,公疑其有異志。而休之
 兄子譙王文思在京師,招集輕俠,公執文思送還休之,令自為其所。休之表廢文思,并與公書陳謝。十一年正月,公收休之子文寶、兄子文祖,並於獄賜死。率眾軍西討,復加黃鉞,領荊州刺史。辛巳,發京師,以中軍將軍道憐監留府事。休之上表自陳曰:臣聞運不常一,治亂代有,陽九既謝,圮終則泰。昔篡臣肆逆,皇綱絕紐。十世未改,鼎祚再隆。太尉臣諱威武明斷,首建義旗,除蕩元凶,皇居反正。布衣匹夫,匡復社稷,南剿盧循,北定廣固,千
 載以來,功無與等。由是四海歸美,朝野推崇。既位窮臺牧,權傾人主,不能以道處功,恃寵驕溢。自以酬賞既極,便情在無上;刑戮逆濫,政用暴苛。問鼎之跡日彰,人臣之禮頓缺。陛下四時膳御,觸事縣空,宮省供奉,十不一在。皇后寢疾之際,湯藥不周;手與家書,多所求告。皆是朝士共所聞見,莫不傷懷憤歎,口不敢言。前揚州刺史元顯第五息法興,桓玄之釁,逃遠於外,王路既開,始得歸本。太傅之胤,絕而復興,凡在有懷,誰不感慶。



 諱吞噬
 之心,不避輕重,以法興聰敏明慧,必為民望所歸;芳蘭既茂,內懷憎惡,乃妄扇異言,無罪即戮。大司馬臣德文及王妃公主,情計切逼,並狼狽請命,逆肆禍毒,誓不矜許,冤酷之痛,感動行路。自以地卑位重,荷恩崇大,乃以庶孽與德文嫡婚,致茲非偶,實由威逼。故衛將軍劉毅、右將軍劉籓、前將軍諸葛長民、尚書僕射謝混、南蠻校尉郗僧施,或盛勳德胤,令望在身,皆社稷輔弼,協贊所寄,無罪無辜,一旦夷滅。猜忍之性,終古所希。



 臣自惟門
 戶衰破,賴之獲存,皇家所重,終古難匹。是以公私歸馮,事盡祗順。



 再授荊州,輒苦陳告。自以才弱位隆,不宜久荷分陜,屢求解任,必不見聽。前經攜侍老母,半家俱西,凡諸子侄,悉留京輦。臣兄子譙王文思,雖年少常人,粗免咎悔,性好交遊,未知防遠,群醜交構,為其風聲。諱遂翦戮人士,遠送文思。臣順其此旨,表送章節,請廢文思,改襲大宗,遣息文寶送女東歸。自謂推誠奉順,理不過此。豈意諱包藏禍心,遂見討伐,加惡文思,構生罪釁。群
 小之言,遠近噂,而臣純愚,暗信必謂不然。尋臣府司馬張茂度狼狽東歸,南平太守檀範之復以此月三日委郡叛逆,尋有審問,東軍已上。諱今此舉,非有怨憎,正以臣王室之乾,位居籓岳,時賢既盡,唯臣獨存,規以翦滅,成其篡殺。鎮北將軍臣宗之、青州刺史臣敬宣,並是諱所深忌憚,欲以次除蕩,然後傾移天日,於事可易。



 今荊、雍義徒,不召而集,子來之眾,其會如林,豈臣無德所能綏致?蓋七廟之靈,理貫幽顯,輒授文思振武將軍、南
 郡太守,宗之子竟陵太守魯軌進號輔國將軍。臣今與宗之親御大眾,出據江津,案甲抗威,隨宜應赴。今絳旗所指,唯諱兄弟父子而已。須克蕩寇逆,尋續馳聞。由臣輕弱,致諱凌橫,上慚俯愧,無以厝顏。



 休之府錄事參軍韓延之,故吏也,有乾用才能。公未至江陵,密使與之書曰:「文思事源,遠近所知,去秋遣康之送還司馬軍者,推至公之極也。而了不遜愧,又無表疏,文思經正不反,此是天地之不容。吾受命西討,止其父子而已。彼土僑舊,
 為所驅逼,一無所問。往年郗僧施、謝邵、任集之等,交構積歲,專為劉毅謀主,所以至此。卿等諸人,一時逼迫,本無纖釁。吾處懷期物,自有由來。今在近路,正是諸人歸身之日。若大軍登道,交鋒接刃,蘭艾吾誠不分,故具示意,并同懷諸人。」延之報曰:承親率戎馬,遠履西畿,闔境士庶,莫不蒨駭。何者?莫知師出之名故也。今辱來疏,始知以譙王前事,良增歎息。司馬平西體國忠貞,款愛待物,當於古人中求耳。以君公有匡復之勳,家國蒙賴,
 推德委誠,每事詢仰。譙王往以微事見劾,猶自表遜位;況以大過而當默然邪!但康之前言有所不盡,故重使胡道諮白所懷。



 道未及反,已奏表廢之,所不盡者命耳。推寄相與之懷,正當如此?有何不可,便興兵戈。自義旗秉權以來,四方方伯,誰敢不先相諮疇,而徑表天子邪?譙王為宰相所責,又表廢之,經正何歸,表使何因,可謂「欲加之罪,其無辭乎」!



 劉諱足下,海內之人,誰不見足下此心,而復欲欺誑國士!天地所不容,在彼不在此矣。來
 示言「處懷期物,自有由來」。今伐人之君,啖人以利,真可謂「處懷期物,自有由來」者矣。劉籓死於閭闔之內;諸葛斃於左右之手;甘言詫方伯,襲之以輕兵,遂使席上靡款懷之士,閫外無自信諸侯,以是為得算,良可恥也。貴府將佐及朝廷賢德,寄性命以過日,心企太平久矣。吾誠鄙劣,嘗聞道於君子。以平西之至德,寧可無授命之臣乎!未能自投虎口,比迹郗、任之徒明矣。假令天長喪亂,九流渾濁,當與臧洪遊於地下,不復多言。



 公視書歎
 息,以示諸佐曰:「事人當如此。」三月,軍次江陵。初,雍州刺史魯宗之常慮不為公所容,與休之相結,至是率其子竟陵太守軌會于江陵。江夏太守劉虔之邀之,軍敗見殺。公命彭城內史徐逵之、參軍王允之出江夏口,復為軌所敗,並沒。時公軍泊馬頭,即日率眾軍濟江,躬督諸將登岸,莫不奮踴爭先。休之眾潰,與軌等奔襄陽。江陵平,加領南蠻校尉。



 將拜,值四廢日,佐史鄭鮮之、褚叔度、王弘、傅亮白遷日,不許。下書曰:「此州積弊,事故相仍,民
 疲田蕪,杼軸空匱。加以舊章乖昧,事役頻苦,童耄奪養,老稚服戎,空戶從役,或越紼應召,每永懷民瘼,宵分忘寢,誠宜蠲除苛政,弘茲簡惠。庶令凋風弊政,與事而新,寧一之化,成於期月。荊、雍二州,西局、蠻府吏及軍人年十二以還,六十以上,及扶養孤幼,單丁大艱,悉仰遣之。窮獨不能存者,給其長賑。府州久勤將吏,依勞銓序;并除今年租稅。」



 四月,公復率眾進討,至襄陽,休之奔羌。天子復重申前命,授太傅、揚州牧,劍履上殿,入朝不趨,贊
 拜不名,加前部羽葆、鼓吹,置左右長史、司馬、從事中郎四人。封公第三子義隆為北彭城縣公。以中軍將軍道憐為荊州刺史。八月甲子,公至自江陵,奉還黃鉞,固辭太傅、州牧、前部羽葆、鼓吹,其餘受命。朝議以公道尊勳重,不宜復施敬護軍,既加殊禮,奏事不復稱名,以世子為兗州刺史。



 十二年正月,詔公依舊辟士,加領平北將軍、兗州刺史。增都督南秦,凡二十二州。公以平北文武寡少,不宜別置,於是罷平北府,以併大府,以世子為豫
 州刺史。三月,加公中外大都督。



 初,公平齊,仍有定關、洛之意,值盧循侵逼,故其事不諧。荊、雍既平,方謀外略。會羌主姚興死,子泓立,兄弟相殺,關中擾亂,公乃戒嚴北討。加領征西將軍、司豫二州刺史。以世子為徐、兗二州刺史。下書曰:「吾倡大義,首自本州,克復皇祚,遂建勳烈。外夷勍敵,內清姦宄,皆邦人州黨竭誠盡力之效也。情若風霜,義貫金石。今當奉辭西旆,有事關、河,弱嗣叨蒙,復忝今授,情事纏綿,可謂深矣。頃軍國務殷,刑辟未息。
 眷言懷之,能不多歎。其犯罪五歲以還,可一原遣。文武勞滿未蒙榮轉者,便隨班序報。」



 公受中外都督及司州,並辭大司馬琅邪王禮敬,朝議從之。公欲以義聲懷遠,奉琅邪王北伐。五月,羌偽黃門侍郎尹沖率兄弟歸順。又加公北雍州刺史,前部羽葆、鼓吹,增班劍為四十人,解中書監。八月丁巳,率大眾發京師。以世子為中軍將軍,監太尉留府事。尚書右僕射劉穆之為左僕射,領監軍、中軍二府軍司,入居東府,總攝內外。九月,公次于
 彭城,加領徐州刺史。



 先是,遣冠軍將軍檀道濟、龍驤將軍王鎮惡步向許、洛,羌緣道屯守,皆望風降服。偽兗州刺史韋華先據倉垣,亦率眾歸順。公又遣北兗刺史王仲德先以水軍入河。仲德破索虜於東郡涼城,進平滑臺。十月,眾軍至洛陽,圍金墉。泓弟偽平南將軍洸請降,送于京師,脩復晉五陵,置守衛。天子詔曰:夫嵩、岱配極,則乾道增輝;籓岳作屏,則帝王成務。是以夏、殷資昆、彭之伯,有周倚齊、晉之輔。鑒諸前典,儀刑萬代,翼治
 扶危,靡不由此。



 太尉公命世天縱,齊聖廣淵,明燭四方,道光宇宙。爰自初迪,則投勤王國,妖蝥孔熾,則功存社稷。固以四維是荷,萬邦攸賴者矣。暨桓玄僭逆,傾蕩四海。公深秉大節,靈武霆震,弘濟朕躬,再造王室。每惟勳德,銘于厥心,遂北清海、岱,南夷百越,荊、雍稽服,庸、氓順軌,剋黜方難,式遏寇虐。及阿衡王猷,班序內外,仰興絕風,傍嗣逸業。秉禮以整俗,遵王以垂訓,聲教遠被,無思不洽。



 爰暨木居海處之酋,被髮雕題之長,莫不忘其
 陋險,九譯來庭,此蓋播諸徽策,靡究其詳者也。曩者永嘉不綱,諸夏幅裂,終古帝居,淪胥戎虜,永言園陵,率土同慕。公明發遐慨,撫機電征,親董侯伯,棱威致討。旗旝首塗,則八表響震;偏師先路,則多壘雲徹。舊都載清,五陵復禮,百城屈膝,千落影從。自篇籍所載,生民以來,勳德懋功,未有若此之盛者也。



 昔周、呂佐睿聖之主,因三分之形,把旄仗鉞,一時指麾,皆大啟疆宇,跨州兼國。其在桓、文,方茲尤儉,然亦顯被寵章,光錫殊品。況乃獨絕
 百代,顧邈前烈者哉!朕每弘鑒古訓,思遵令圖。以公深秉沖挹,用闕大禮,天人引領,于茲歷載。況今禹迹齊軌,九隩同文,司勳抗策,普天增佇。遂公高挹,大愆國章。三靈眷屬,朕實祗懼。便宜顯答群望,允崇盛典。其進位相國,總百揆,揚州牧,封十郡為宋公,備九錫之禮,加璽綬、遠遊冠,位在諸侯王上,加相國綠綟綬。



 策曰:朕以寡昧,仰贊洪基,夷羿乘釁,蕩覆王室,越在南鄙,遷于九江。宗祀絕饗,人神無位,提挈群凶,寄命江滸。則我祖宗之業,
 奄墜于地,七百之祚,翦焉既傾,若涉淵海,罔知攸濟。天未絕晉,誕育英輔,振厥弛維,再造區宇,興亡繼絕,俾昏作明。元勛至德,朕實賴焉。今將授公典策,其敬聽朕命:乃者桓玄肆僭,滔天泯夏,拔本塞源,顛倒六位,庶僚俯眉,四方莫恤。公精貫朝日,氣凌霄漢,奮其靈武,大殲群慝,剋復皇邑,奉帝歆神。此公之大節,始於勤王者也。授律群后,溯流長騖,薄伐崢嶸,獻捷南郢,大憝折首,群逆畢夷,三光旋採,舊物反正。此又公之功也。出籓入輔,弘茲
 保弼,阜財利用,繁殖生民,編戶歲滋,疆宇日啟,導德明刑,四境有截。此又公之功也。鮮卑負眾,僭盜三齊,狼噬冀、青,虔劉沂、岱,介恃遐阻,仍為邊毒。公搜乘秣駟,夐入遠疆,衝櫓四臨,萬雉俱潰,竊號之虜,顯戮司寇,拓土三千,申威龍漠。此又公之功也。盧循妖凶,伺隙五嶺,乘虛肆逆,侵覆江、豫,旍拂寰內,矢及王城,朝野喪沮,莫有固志,家獻徙卜之計,國議遷都之規。公乘轅南濟,義形于色,嶷然內湛,視險若夷,攄略運奇,英謨不世,狡寇窮恤,
 喪旗宵遁,俾我畿甸,拯於將墜。此又公之功也。追奔逐北,揚旌江濆,偏旅浮海,指日遄至。番禺之功,俘級萬數,左里之捷,魚潰鳥散。元兇遠迸,傳首萬里,海南肅清,荒服來款。此又公之功也。劉毅叛渙,負釁西夏,凌上罔主,志肆奸暴,附麗協黨,扇蕩王畿。公御軌以刑,消之不日,倉兕電溯,神兵風掃,罪人斯得,荊、衡清晏。此又公之功也。譙縱怙亂,寇竊一隅,王化阻閡,三巴淪溺。公指命偏師,授以良圖,凌波浮湍,致屆井絡,僭豎伏金質,梁、岷草偃。
 此又公之功也。馬休、魯宗,阻兵內侮,驅率二方,連旗稱亂。公投袂星言,研其上略,江津之師,勢踰風電,回旆沔川,實繁震懾,二叛奔迸,荊、雍來蘇,玄澤浸育,溫風潛被。此又公之功也。永嘉不競,四夷擅華,五都幅裂,山陵幽辱,祖宗懷沒世之憤,遺氓有匪風之思。公遠齊伊宰納隍之仁,近同小白滅亡之恥,鞠旅陳師,赫然大號,公命群帥,北徇司、兗。許、鄭風靡,鞏、洛載清,偽牧逆籓,交臂請罪,百年榛穢,一朝掃濟。此又公之功也。



 公有康宇內之
 勳,重之以明德。爰初發迹,則奇謨冠古,電擊彊妖,則鋒無前對,聿寧東畿,大造黔首。若乃草昧經綸,化融於歲計,扶危靜亂,道固於苞桑。



 辯方正位,納之軌度,蠲削煩苛,較若畫一,淳風美化,盈塞宇宙。是以絕域獻琛,遐夷納貢,王略所宣,九服率從。雖文命之東漸西被,咎繇之邁于種德,何以尚茲。



 朕聞先王之宰世也,庸勛尊賢,建侯胙土,褒以寵章,崇其徽物,所以協輔皇家,永隆籓屏。故曲阜光啟,遂荒徐宅,營丘表海,四履有聞。其在襄王,
 亦賴匡霸,又命晉文,備物光錫。惟公道冠前烈,勳高振古,而殊典未加,朕甚懵焉。今進授相國,以徐州之彭城沛蘭陵下邳淮陽山陽廣陵、兗州之高平魯泰山十郡,封公為宋公。錫茲玄土,苴以白茅,爰定爾居,用建塚社。昔晉、鄭啟籓,入作卿士,周、邵保傅,出總二南,內外之重,公實兼之。命使持節、太尉、尚書左僕射、晉寧縣五等男湛授相國印綬,宋公璽紱,使持節、兼司空、散騎常侍、尚書、陽遂鄉侯泰授宋公茅土,金虎符第一至第五左,竹
 使符第一至第十左。相國位無不總,禮絕朝班,居常之名,宜與事革。其以相國總百揆,去「錄尚書」之號。上送所假節、侍中、中外都督、太傅太尉印綬,豫章公印策。進揚州牧,領征西將軍、司豫北徐雍四州刺史如故。



 公紀綱禮度,萬國是式,秉介蹈方,罔有遷志。是以錫公大輅、戎輅各一,玄牡二駟。公抑末敦本,務農重積,採蘩實殷,稼穡惟阜。是用錫公兗冕之服,赤舄副焉。公閑邪納正,移風改俗,陶鈞品物,如樂之和。是用錫公軒縣之樂,六佾之
 舞。公宣美王化,導揚休風,華夷企踵,遠人胥萃。是用錫公硃戶以居,公官方任能,網羅幽滯,九皋辭野,髦士盈朝。是用錫公納陛以登,公當軸處中,率下以義,式遏寇仇,清除苛慝,是用錫公虎賁之士三百人。公明罰恤刑,庶獄詳允,放命干紀,罔有攸縱。是用錫公鈇、鉞各一。公龍驤鳳矯,咫尺八紱,括囊四海,折衝無外。是用錫公彤弓一,彤矢百,盧弓十,盧矢千。公溫恭孝思,致虔禋祀,忠肅之志,儀刑萬方。是用錫公秬鬯一卣,圭瓚副焉。宋國
 置丞相以下,一遵舊儀。欽哉!



 其祗服往命,茂對天休,簡恤庶邦,敬敷顯德,以終我高祖之嘉命。



 置宋國侍中、黃門侍郎、尚書左丞、相,隨大使奉迎。桴罕虜乞佛熾盤遣使詣公求效力討羌,拜平西將軍、河南公。



 十三年正月,公以舟師進討,留彭城公義隆鎮彭城。軍次留城,經張良廟,令曰:「夫盛德不泯,義在祀典,微管之嘆,撫事彌深。張子房道亞黃中,照鄰殆庶,風雲言感,蔚為帝師,大拯橫流,夷項定漢,固以參軌伊、望,冠德如仁。若乃神交圯
 上,道契商洛,顯晦之間,窈然難究,源流淵浩,莫測其端矣。塗次舊沛,佇駕留城,靈廟荒殘,遺象陳昧,撫迹懷人,慨然永歎。過大梁者,或佇想於夷門;遊九原者,亦流連於隨會。可改構榱桷,脩飾丹青,蘩行潦,以時致薦。以紓懷古之情,用存不刊之烈。」天子追贈公祖為太常,父為左光祿大夫,讓不受。



 二月,冠軍將軍檀道濟等次潼關。三月庚辰,大軍入河。索虜步騎十萬,營據河津。公命諸軍濟河擊破之。公至洛陽。七月,至陜城。龍驤將軍王
 鎮惡伐木為舟,自河浮渭。八月,扶風太守沈田子大破姚泓於藍田。王鎮惡剋長安,生擒泓。九月,公至長安。長安豐稔,帑藏盈積。公先收其彞器、渾儀、土圭之屬,獻于京師;其餘珍寶珠玉,以班賜將帥。執送姚泓,斬于建康市。謁漢高帝陵,大會文武於未央殿。



 十月,天子詔曰:朕聞先王之蒞天下也,上則大寶以尊德,下則建侯以褒功。是以成勳告就,文命有玄圭之錫,四海來王,姬旦饗龜、蒙之封。夫翼聖宣績,輔德弘猷,禮窮元賞,寵章希世,
 況明保沖昧,獨運陶鈞者哉!



 朕以不德,遭家多難,雲雷作屯,夷羿竊命,失位京邑,遂播蠻荊,艱難卑約,制命凶醜。相國宋公,天縱睿聖,命世應期,誠貫三靈,大節宏發。拯朕躬於巢幕,回靈命於已崩,固已道窮北面,暉格八表者矣。及外積全國之勳,內累戡黎之伐,芟夷彊妖之始,蘊崇奸猾之源,顯仁藏用之道,六府孔脩之績,莫不雲行雨施,能事必舉,諒已方軌於三、五,不容於典策者焉。自永嘉喪師,綿踰十紀,五都分崩,然正朔時暨;唯三
 秦懸隔,未之暫賓。至令羌虜襲亂,淫虐三世,資百二之易守,恃函谷之可關,廟算韜略,不謀之日久矣。公命世撫運,闡曜威靈,內研諸侯之慮,外致上天之罰。故能倉兕甫訓,則許、鄭風偃;鉦鉞未指,則瀍、洛霧披。俾舊闕之陽,復集萬國之軫,東京父老,重睹司隸之章。俾朕負扆高拱,而保大洪烈。是用遠鑒前典,延即群謀,敬授殊錫,光啟疆宇。乘馬之制,有陋舊章。徽稱之美,未窮上爵。豈足以顯報懋功,允塞民望;籓輔王畿,長轡六合者乎!實
 以公每秉謙德,卑不可踰,難進之道,以寵為戚。是故降損盛制,且有後命也。自茲迄今,洪勳彌劭,棱威九河,魏、趙底服,回轅崤、潼,連城冰泮。遂長驅灞滻、懸旌龍門,逆虜姚泓,係頸就擒。百稔梗穢,滌於崇朝;祖宗遺憤,雪於一旦。涉禹之跡,方行天下,至於海外,罔有不服。功固萬世,其寧惟永,豈金石《雅頌》所能贊揚,實可以告於神明,勒銘嵩、岱者已。



 朕又聞之,周道方遠,則摐甗鳴岐,二南播德,則麟騶呈瑞。自公大號初發,爰暨告成,靈祥炳煥,
 不可勝紀,豈伊素雉遠至,嘉禾近歸而已哉!朕每仰鑒玄應,俯察人謀,進惟道勳,退惟國典,豈得遂公沖挹,而久蘊盛策。便宜敬行大禮,允副幽顯之望。其進宋公爵為王,以徐州之海陵、東安、北琅邪、北東莞、北東海、北譙、北梁、豫州之汝南、北潁川、北南頓凡十郡,益宋國。其相國、揚州牧、領征西將軍、司豫北徐雍四州刺史如故。



 十一月,前將軍劉穆之卒,以左司馬徐羨之代掌留任。大事昔所決於穆之者,皆悉以諮。公欲息駕長安,經略趙、魏,
 會穆之卒,乃歸。十二月庚子,發自長安,以桂陽公義真為安西將軍、雍州刺史,留腹心將佐以輔之。閏月,公自洛入河,開汴渠以歸。



 十四年正月壬戌,公至彭城,解嚴息甲。以輔國將軍劉遵考為並州刺史,領河東太守,鎮蒲阪。公解司州,領徐、冀二州刺史,固讓進爵。六月,受相國宋公九錫之命。令曰:「孤以寡薄,負荷殊重,守位奉籓,危溢是懼。朝恩隆泰,委美推功,遂方軌齊、晉,擬議國典。雖亮誠守分,十稔于今,而成命弗回,百辟胥暨內外庶
 僚,敦勉周至。籍運來之功,參休明之迹,乘菲薄之資,同盛德之事,監寐永言,未知攸托。隆祚之始,思覃斯慶,其赦國內殊死以下,今月二十三日昧爽以前,悉皆原宥。鰥寡孤獨不能自存者,人賜粟五斛。府州刑罪,亦同蕩然。其餘詳依舊準。」詔崇豫章公太夫人為宋公太妃,世子為中軍將軍,副貳相國府。以太尉軍諮祭酒孔季恭為宋國尚書令,青州刺史檀祗為領軍將軍,相國左長史王弘為尚書僕射。



 其餘百官悉依天朝之制。又詔宋
 國所封十郡之外,悉得除用。



 先是,安西中兵參軍沈田子殺安西司馬王鎮惡,諸將軍復殺安西長史王脩,關中亂。十月,公遣右將軍硃齡石代安西將軍桂陽公義真為雍州刺史。義真既還,為佛佛虜所追,大敗,僅以身免。諸將帥及齡石並沒。領軍檀祗卒,以中軍司馬檀道濟為中領軍。十二月,天子崩,大司馬琅邪王即帝位。



 元熙元年正月,詔遣大使征公入輔。又申前命,進公爵為王。以徐州之海陵東海北譙北梁、豫州之新蔡、兗州
 之北陳留、司州之陳郡汝南潁川滎陽十郡,增宋國。



 七月,乃受命,赦國內五歲刑以下。遷都壽陽。以尚書劉懷慎為北徐州刺史,鎮彭城。九月,解揚州。十二月,天子命王冕十有二旒,建天子旌旗,出警入蹕,乘金根車,駕六馬,備五時副車,置旄頭雲罕,樂舞八佾,設鐘虡宮縣。進王太妃為太后,王妃為王后,世子為太子,王子、王孫爵命之號,一如舊儀。



 二年四月,徵王入輔。六月,至京師。晉帝禪位于王,詔曰:
 夫天造草昧,樹之司牧,所以陶鈞三極,統天施化。故大道之行,選賢與能,隆替無常期,禪代非一族,貫之百王,由來尚矣。晉道陵遲,仍世多故,爰暨元興,禍難既積,至三光貿位,冠履易所,安皇播越,宗祀墮泯,則我宣元之祚,永墜于地,顧瞻區域,翦焉已傾。相國宋王,天縱聖德,靈武秀世,一匡頹運,再造區夏,固以興滅繼絕,舟航淪溺矣。若夫仰在璇璣,旁穆七政,薄伐不庭,開復疆宇。遂乃三俘偽主,開滌五都,雕顏卉服之鄉,龍荒朔漠之長,
 莫不回首朝陽,沐浴玄澤。



 故四靈效瑞,川岳啟圖,嘉祥雜遝,休應炳著,玄象表革命之期,華裔注樂推之願。



 代德之符,著乎幽顯,瞻烏爰止,允集明哲,夫豈延康有歸,咸熙告謝而已哉!



 昔火德既微,魏祖底績,黃運不競,三后肆勤。故天之歷數,實有攸在。朕雖庸暗,昧於大道,永鑒廢興,為日已久。念四代之高義,稽天人之至望,予其遜位別宮,歸禪于宋,一依唐虞、漢魏故事。



 詔草既成,送呈天子使書之,天子即便操筆,謂左右曰:「桓玄之時,天
 命已改,重為劉公所延,將二十載。今日之事,本所甘心。」甲子,策曰:咨爾宋王:夫玄古權輿,悠哉邈矣,其詳靡得而聞。爰自書契,降逮三、五,莫不以上聖君四海,止戈定大業。然則帝王者,宰物之通器;君道者,天下之至公。



 昔在上葉,深鑒茲道,是以天祿既終,唐、虞弗得傳其嗣;符命來格,舜、禹不獲全其謙。所以經緯三才,澄序彞化,作範振古,垂風萬葉,莫尚於茲。自是厥後,歷代彌劭,漢既嗣德於放勛,魏亦方軌於重華。諒以協謀乎人鬼,而以
 百姓為心者也。



 昔我祖宗欽明,辰居其極,而明晦代序,盈虧有期。翦商兆禍,非唯一世,曾是弗剋,矧伊在今,天之所廢,有自來矣。惟王體上聖之姿,苞二儀之德,明齊日月,道合四時。乃者社稷傾覆,王拯而存之;中原蕪梗,又濟而復之。自負固不賓,干紀放命,肆逆滔天,竊據萬里。靡不潤之以風雨,震之以雷霆。九伐之道既敷,八法之化自理。豈伊博施於民,濟斯黔庶;固以義洽四海,道威八荒者矣。至於上天垂象,四靈效征,圖讖之文既明,
 人神之望已改;百工歌於朝,庶民頌於野,億兆抃踴,傾佇惟新。自非百姓樂推,天命攸集,豈伊在予,所得獨專!是用仰祗皇靈,俯順群議,敬禪神器,授帝位于爾躬。大祚告窮,天祿永終。於戲!王其允執其中,敬遵典訓,副率土之嘉願,恢洪業於無窮,時膺休祐,以答三靈之眷望。



 又璽書曰:蓋聞天生蒸民,樹之以君。帝皇寄世,實公四海。崇替係於勳德,升降存乎其人。故有國必亡,卜年著其數;代謝無常,聖哲握其符。昔在上世,三聖係軌,疇咨
 四岳,以弘揖讓,惟先王之有作,永垂範於無窮。及劉氏致禪,實堯是法;有魏告終,亦憲茲典。我世祖所以撫歸運而順人事,乘利見而定天保者也。而道不常泰,戎夷亂華,喪我洛食,蹙國江表,仍遘否運,淪沒相因,逮于元興,遂傾宗祀。幸賴神武光天,大節宏發,匡復我社稷,重造我國家。惟王聖德欽明,則天光大,應期誕載,明保王室。內紓國難,外播宏略,誅大憝於漢陽,逋僭盜於沂渚,澄氛西岷,肅清南越,再靜江、湘,拓定樊、沔。若乃永懷區
 宇,思一聲教,王師首路,則伊、洛澄流;棱威崤、潼,則華嶽褰靄,偽酋銜璧,咸陽即序。雖彞器所銘,詩書所詠,庸勳之盛,莫之與二也。遂偃武脩文,誕敷德政,八統以馭萬民,九職以刑邦國,思兼三王,以施四事。故能信著幽顯,義感殊方。自歷世所賓,舟車所暨,靡不謳歌仁德,抃舞來庭。



 朕每敬惟道勳,永察符運,天之歷數,實在爾躬。是以五緯升度,屢示除舊之迹;三光協數,必昭布新之祥。圖讖禎瑞,皎然斯在。加以龍顏英特,天授殊姿,君人之
 表,煥如日月。傳稱「惟天為大,惟堯則之。」《詩》云:「有命自天,命此文王。」夫「或躍在淵」者,終以饗九五之位;「勳格天地」者,必膺大寶之業。



 昔土德告沴,傳祚于我有晉;今歷運改卜,永終于茲,亦以金德而傳于宋。仰四代之休義,鑒明昏之定期,詢于群公,爰逮庶尹,咸曰休哉,罔違朕志。今遣使持節、兼太保、散騎常侍、光祿大夫澹,兼太尉、尚書宣範奉皇帝璽綬,受終之禮,一如唐虞、漢魏故事。王其允答人神,君臨萬國,時膺靈祉,酬于上天之眷命。



 王奉
 表陳讓,晉帝已遜瑯邪王第,表不獲通。於是陳留王虔嗣等二百七十人,及宋臺群臣,並上表勸進,上猶不許。太史令駱達陳天文符瑞數十條,群臣又固請,王乃從之。






\end{pinyinscope}