\article{卷九十列傳第五十 明四王}

\begin{pinyinscope}

 明帝十二子:陳貴妃生後廢帝,謝脩儀生皇子法良,陳昭華生順帝,徐婕妤生第四皇子,鄭脩容生皇子智井,次晉熙王燮,與皇子法良同生。泉美人生邵陵殤王友;
 次江夏王躋,與第四皇子同生。徐良人生武陵王贊,杜脩華生隨陽王翽。次新興王嵩,與武陵王贊同生。又泉美人生始建王禧。智井、燮、躋、贊並出繼。法良未封,第四皇子未有名,早夭。



 邵陵殤王友,字仲賢,明帝第七子也。後廢帝元徽二年,太尉、江州刺史桂陽王休範反誅,皇室寡弱,友年五歲,出為使持節、督江州豫州之西陽新蔡晉熙三郡諸軍事、南中郎將、江州刺史,封邵陵王,食邑二千戶。府州文
 案及臣吏不諱有無之有。順帝即位,進號左將軍,改督為都督。升明元年,徙都督南豫豫司三州諸軍事、安南將軍、南豫州刺史、歷陽太守。三年,薨,無子,國除。



 隨陽王翽,字仲儀,明帝第十子也。元徽四年,年六歲,封南陽王,食邑二千戶。升明元年,為使持節、督郢州司州之義陽諸軍事、西中郎將、郢州刺史。未拜,徙督湘州諸軍事、南中郎將、湘州刺史,持節如故。未之鎮,進號前將軍。二年,以南陽荒遠,改封隨陽王,以本號停京師。齊受
 禪,降封舞陰縣公,食邑千五百戶。



 謀反,賜死。



 新興王嵩,字仲岳,明帝第十一子。元徽四年,年六歲,封新興王,食邑二千戶。齊受禪,降封定襄縣公,食邑千五百戶。謀反,賜死。



 始建王禧,字仲安,明帝第十二子也。元徽四年,年六歲,封始建王,食邑二千戶。齊受禪,降封荔封縣公,食邑千五百戶。謀反,賜死。



 史臣曰:太宗負螟之慶,事非己出,枝葉不茂,豈能庇其
 本根。侯服於周,斯為幸矣。



\end{pinyinscope}