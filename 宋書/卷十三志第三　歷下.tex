\article{卷十三志第三 歷下}

\begin{pinyinscope}

 《元嘉歷法》:上元庚辰甲子紀首至太甲元年癸亥,三千五百二十三年,至元嘉二十年癸未,五千七百三年,算外。



 元法,三千六百四十八。



 章歲,十九。



 紀法,六百八。



 章月,二百三十五。



 紀月,七千五百二十。



 章閏,七。



 紀日,二十二萬二千七十。



 度分,七十五。



 度法,三百四。



 氣法,二十四。



 餘數,一千五百九十五。



 歲中,十二。



 日法,七百五十二。



 沒餘,三十六。



 通數,二萬二千二百七。



 通法,四十七。



 沒法,三百一十九。



 月周,四千六十四。



 周天,十一萬一千二十五。



 通周,二萬七百二十一。



 周日日餘,四百一十七。



 周虛,三百三十五。



 會數,一百六十。



 交限數,八百五十九。



 會月,九百二十九。



 朔望合數,八十。


甲子紀第一
 \gezhu{
  遲疾差一萬
  七千六百六十三,交會差八百七十七}


甲戌紀第二
 \gezhu{
  遲
  疾差三千四十三,交會差二
  百七十九}


甲申紀第三
 \gezhu{
  遲疾差九千一百四十四,交會差六百二十}


甲午
 紀第四
 \gezhu{
  遲疾差一萬五千二百四十五,交會差
  二十二}


甲辰紀第五
 \gezhu{
  遲疾差六百二十五,交會差三百六十
  三}


甲寅紀第六
 \gezhu{
  遲疾差六千七百二十六,交會差七百四}


推入紀法:置上元庚辰盡所求年,以元法除之,不滿元法,以紀法除之,餘不滿紀法,入紀年也。滿法去之,得後紀。
 \gezhu{
  入甲午紀壬辰歲來,至今元嘉二十年歲在癸末,二百三十一年,算外。}



 推積月術:置入紀年數算外,以章月乘之,如章歲為積月,不盡為閏餘。閏餘十二以上,其年閏。



 推朔術:以通數乘積分,為朔積分,滿日法為積日,不盡為小餘。以六旬去積日,不盡為大餘,命以紀,算外,所求年正月朔日也。



 求次月,加大餘二十九,小餘三百九十九,小餘滿日法從大餘,即次月朔也。



 小餘三百五十三以上,其月大也。



 推弦望法:加朔大餘七,小餘二百八十七,小分三,小分滿四從小餘,小餘滿日法從大餘,命如前,上弦日也。又加之得望,又加之得下弦。


推二十四氣術:置入紀年算外,以餘數乘之,滿度法三百四為積沒,不盡為小餘。以六旬去積沒,不盡為大餘,命以紀,算外,所求年雨水日也。求次氣,加大餘十五,小餘六十六,小分十一,小分滿氣法從小餘,小餘滿度法從大餘,次氣日也。
 \gezhu{
  雨水在十六日以後者,如法減之,得立春。}



 推閏月法:以閏餘減章歲,餘以歲中乘之,滿章閏得一,數從正月起,閏所在也。閏有進退,以無中氣御之。


立春正月節
 \gezhu{
  限數一百九十四,間數一百九十}


雨水正月中
 \gezhu{
  限數一百八十六,間數一百八十二}


驚蟄二月節
 \gezhu{
  限數一百七十七,間數一百七十二}


春分二月中
 \gezhu{
  限數一百六十七,間數一百六十二}


清明三月節
 \gezhu{
  限數一百五十八,間數一百五十四}


穀雨三月中
 \gezhu{
  限數一百四十九,間數一百四十五}


立夏四月節
 \gezhu{
  限數一百四十二,間數一百三十九}


小滿四月中
 \gezhu{
  限數一百三十六,間數一百三十四}


芒種五月節
 \gezhu{
  限數一百三十三,間數一百三十
  二}


夏至五月中
 \gezhu{
  限數一百三十一,間數一百三十二}


小暑六月節
 \gezhu{
  限數一百三十三,間數一百三十四}


大暑六月中
 \gezhu{
  限數一百三十六,間數一百三十九}


立秋七月節
 \gezhu{
  限數一百四十二,間數一百四十五}


處暑七月中
 \gezhu{
  限數一百四十九,間數一百五十三}


白露八月節
 \gezhu{
  限數一百五十七,間數一百六十二}


秋分八月中
 \gezhu{
  限數一百六十七,間數一百七十二}


寒露九月節
 \gezhu{
  限數一百七十七,間數一百八十二}


霜降九月中
 \gezhu{
  限數一百八十六,間數一百九十}


立冬十月節
 \gezhu{
  限數一百九十四,間數一百九十七}


小雪十月中
 \gezhu{
  限數二百,間數二百三}


大雪十一月節
 \gezhu{
  限數二百五,間數二百六}


冬至十一月中
 \gezhu{
  限數二百七,間數二百六}


小寒十二月節
 \gezhu{
  限數二百五,間數二百三}


大寒十二月中
 \gezhu{
  限數二百,間數一百九十七}


推沒滅術:因雨水積,以沒餘乘之,滿沒法為大餘,不盡
 為小餘,如前,所求年為雨水前沒日也。求次沒,加大餘六十九,小餘一百九十六,滿沒法從大餘,命如前,雨水後沒日也。
 \gezhu{
  雨水前沒多在故歲,常有五沒,官以沒正之,一年常有五沒或六沒。小餘盡為滅日也。}
 雨水小餘三十九以還,雨水六旬後乃有。



 推土用事法:置立春大小餘小分之數,減大餘十八,小餘七十九,小分十八,命以紀,算外,立春前土用事日也。大餘不足加六十,小餘不足減,減大餘一,加度法而後減之。立夏、立冬求土用事皆如上法。



 推日所在度法:以度法乘朔積度,不盡為分。命度起室二,次宿除之,算外,正月朔夜半日在度及分也。求次日,日加一度,經室去度分。



 推月所在度法:以月周乘朔積日,周天去之,餘滿度法為積度,不盡為分,命度如前,正月朔夜半月所在度及分。求次月,小月加度二十二,分一百三十三,大月加度三十五,分二百四十五,分滿度法成一度,命如前,次月朔月所在度及分也。



 歷先月法:以十六除月行分為大
 分,如所入遲疾加之,經室去度分。



 推合朔月食術:置所求年積月,以會數一百六十乘之,以所入交會紀差二十二加之,滿會月去之,餘則其年正月朔去交分也。求次月,以會數加之,滿會月去之。



 求望,加合數。朔望去交分如合數以下,交限數以上,朔則交會,望則月食。


推入遲疾歷法:置所求年朔積分,所入遲疾差
 \gezhu{
  一萬五千二百四十五}
 加之,滿通周去之,餘滿日法得一日,不盡為日餘,命
 日算外,所求年正月朔入歷。求次月,加一日,日餘七百三十四。求望,加十四日,日餘五百七十五半。餘滿日法成一日,日滿二十七去之,除日餘如周日日餘,不足減,減一日,加周虛。
 \gezhu{
  日滿二十七而日餘不滿周日日餘,為損。周日滿去之,為入歷一日。}


推合朔月食定大小餘法:以入歷日餘乘入歷下損益率,
 \gezhu{
  入一日,益二十五是也。}
 以損益盈縮積分,
 \gezhu{
  值損則損之,值益則益之。}
 為定積分。以入歷日餘乘列差,滿日法盈減縮加差法,為定差法。以除定積分,所得減加本朔望小餘,
 \gezhu{
  值盈則減,縮則加之。}
 為定
 小餘。加之滿日法,合朔月食進一日;減之不足減者,加日法而後減之,則退一日。值周日者,用周日定數。



 推加時:以十二乘定小餘,滿日法得一辰,數從子起,算外,則朔望加時所在辰也。有餘者四之,滿日法得一為少,二為半,三為太半。又有餘者三之,滿日法得一為強,半法以上排成一,不滿半法棄之。以強并少為少強,并半為半強,并太為太強。得二者為小弱,以并少為半弱,以并半為太弱,以并太為一辰弱。以所在辰名之。



 推合朔月食加時滿刻法:各以百刻乘定小餘,如日法而一;不盡什之,求分。



 先除夜漏之半,即晝漏加時刻及分也。晝漏盡,又入夜漏。在中節前後四日以還者,視限數。在中節前後五日以上者,視間限數。月食加時定小餘不滿限數、間數者,皆以算上為日。



 月行遲疾度損益率盈縮積分列差差法一日十四度十三分益二十五
 盈二二百六十二日十四度十一分益二十三盈萬八千八百三二百五十八三日十四度八分益二十盈三萬六千九十六四二百五十五四日十四度四分益十六盈五萬一千一百三十六五二百五十一五日十三度十八分益十一
 盈六萬三千一百六十八五二百四十六六日十三度十三分益六



 盈七萬一千四百四十六二百四十一七日十三度七分益



 盈七萬五千九百五十二五二百三十五八日十三度二分損五



 盈七萬五千九百五十二四二百三十九日十二度十七分損九
 盈七萬二千一百九十二三二百二十六十日十二度十四分損十二盈六萬五千四百二十四三二百二十三十一日十二度十一分損十五盈五萬六千四百三二百二十十二日十二度八分損十八盈四萬五千一百二十二二百一十七十三日十二度六分損二十
 盈三萬一千五百八十四二二百一十五十四日十二度四分損二十二盈一萬六千五百四十四二二百一十三十五日十二度二分益二十四縮



 二二百一十一十六日十二度四分益二十二縮一萬八千四十



 八二二百一十三十七日十二度六分益二十
 縮三萬四千五百九十二三二百一十五十八日十二度九分益十七縮四萬九千六百三十二五二百一十八十九日十二度十四分益十二縮六萬二千四百一十六六二百二十三二十日十三度一分益六縮七萬一千四百四十六二百二十九廿一日十三度七分益
 縮七萬五千九百五十二五二百三十五廿二日十三度十二分損五縮七萬五千九百五十二四二百四十廿三日十三度十六分損九縮七萬二千一百九十二四二百四十四廿四日十四度一分損十三縮六萬五千四百二十四四二百四十八廿五日十四度五分損十七
 縮五萬五千六百四十八三二百五十二廿六日十四度八分損二十縮四萬二千八百六十四三二百五十五廿七日十四度十一分損二十三縮二萬七千八百二十四二二百五十八周日十四度十三分損二十五定縮一萬五百二十八定備二百六十定小分一百三損二百二十四九萬三千四百八意差法二千三百九推合朔度:以章歲乘朔小餘,滿通法為大分,不盡為小
 分。以大分從朔夜半日日分,滿度命如前,正月朔日月合朔所在共合度也。



 求次月,加度二十九,大分一百六十一,小分十四,小分滿通法從大分,大分滿度法從度。


經室除度分。求望,加十四度,大分二百三十二,小分三十半。
 \gezhu{
  求望月所在度,加日度一百八十二,分一百八十九,小分二十三半。}


推五星法:二十四氣日所在度日中晷影
 晝漏刻夜漏刻雨水室太
 \gezhu{
  強}
 八尺二寸二分五十五分四十九五分驚蟄壁一
 \gezhu{
  強}
 六尺七寸二分五十二九分四十七一分
 春分奎七
 \gezhu{
  少強}
 五尺三寸九分五十五五分四十四五分清明婁六
 \gezhu{
  半}
 四尺二寸五分五十八四十二穀雨胃九
 \gezhu{
  太弱}
 三盡二寸五分六十三分三十九七分立夏昴十一
 \gezhu{
  弱}
 二尺五寸六十二三分三十七七分
 小滿畢十五
 \gezhu{
  少弱}
 一尺九寸七分六十三九分三十六一分芒種井三半
 \gezhu{
  弱}
 一尺六寸九分六十四八分三十五二分夏至井十八一尺五寸六十五三十五小暑鬼一
 \gezhu{
  弱}


一尺六寸九分六十四八分三十五二分
 大暑柳十二
 \gezhu{
  弱}
 一尺九寸七分六十三九分三十六一分立秋張五
 \gezhu{
  半強}
 二尺五寸六十二三分三十七七分處暑翼二
 \gezhu{
  半}
 三尺二寸五分六十三分三十九七分白露翼十七
 \gezhu{
  太弱}
 四尺二寸五分五十八四十二
 秋分軫十五五尺三寸九分五十五五分四十四五分寒露亢一
 \gezhu{
  少}
 六尺七寸二分五十二九分四十七一分霜降氐七
 \gezhu{
  半}
 八尺二寸八分五十五分四十九五分立冬心二
 \gezhu{
  半弱}


九尺九寸一分四十八四分五十一六分
 小雪尾十二
 \gezhu{
  太強}
 一丈一尺三寸四分四十六七分五十三三分大雪箕十一丈二尺四寸八分四十五六分五十四四分冬至斗十四
 \gezhu{
  強}
 一丈三尺四十五五十五小寒牛三半
 \gezhu{
  強}
 一丈二尺四寸八分四十五六分五十四
 四分大寒女十半
 \gezhu{
  強}
 一丈一尺三寸四分四十六七分五十三三分立春危四九尺九寸一分四十八四分五十一六分二十四氣昏中星明中星雨水觜一少強尾十一強驚蟄井九半箕四少弱春分
 井二十九半強斗四弱清明柳十二太鬥十四半穀雨張十斗二十五半立夏翼十太弱女三少小滿軫十弱虛二弱芒種角十太弱危七弱夏至氐五少弱室五少強小暑房四太弱壁六太弱大暑尾八太弱奎十二太弱立秋箕三胃二太弱處暑斗三半昴七太弱白露斗十四半弱畢十六半弱秋分斗二十五少強井九少強寒露牛八半強井二十
 九弱霜降女十一半弱柳十一半強立冬危二弱張八太弱小雪危十三半強翼八太強大雪室九半強軫八少強冬至壁八太強角七少強小寒奎十五少亢九大寒胃四半強氐十三太強立春昴九少心四強推五星法:合歲合數日度法室分木三百四十四三百一十五九萬五千七百六十二萬三千六百二十五火四百五十九二百一十五六萬五千三百六十一萬六千一百二十五土三百八十三三百七十一十一萬二千四百八十二萬七千七百五十金二百六十七一百六十七五萬七百六十八一萬二千五百二十五水七十九二百四十九七萬五千六百九十六一萬八千六百七十五木後元丙戌,晉咸和元年,至元嘉二十年癸未,百十八年算上。



 火後元乙亥,元嘉十二年,至元嘉二十年癸未,九年算上。



 土後元甲戌,元嘉十一年,至元嘉二十年癸未,十年算上。



 金後元甲
 申,晉太元九年,至元嘉二十年癸未,六十年算上。



 水後元乙丑,元嘉二年,至元嘉二十年癸未,十九年算上。


推五星法:各設其元至所求年算上,以合數乘之,滿合歲為積合,不盡曰合餘,多者以合數除之,得一,星合往年,得二,合前往年,不滿合數,其年。
 \gezhu{
  木、土、金則有往年合,火有前往年合,水一年三合或四合也。}
 以合餘減合數為度分,
 \gezhu{
  水度分滿合歲則去之也。}
 以周天
 \gezhu{
  十一萬一千三十五}
 乘度分,滿日度法為積度,不盡曰度餘。命度以室二,算外,星合所在度也。以合數乘其年,內雨水小餘,
 并度餘為日餘,滿日度法從積度為日,命以雨水,算外,星合日也。求星見日法,以法伏日及餘
 \gezhu{
  木則十六日及金是也。}
 加星合日及餘,滿日度法成一日,命如前,星見日也。求星見度法,以法伏度及餘
 \gezhu{
  木則二度及餘是也。}
 加星合度及餘,滿日度法成一度,命如前,所見度也。以星行分母
 \gezhu{
  木則二十三見也。}
 乘見度餘,滿日度法得一,分乃日加所行分。
 \gezhu{
  木順日行四分。}
 分滿其母成一度,逆順母不同,
 \gezhu{
  木逆分母七也。}
 當各乘度餘,留者承前,逆則減之,伏不書度,經室去分,不足減者,破全度。
 \gezhu{
  五星室分各異,若在行分,各依室分去之。}


木:初與日合,伏,十六日,日餘四萬一千七百八十,行二度,餘七萬七千八百四十七半,晨見東方。
 \gezhu{
  去日十三度半強。}
 順,日行二十三分之四,一百一十五日行二十度。留,不行,二十六日而逆。日行七分之一,八十四日退十二度。又留二十六日。順,一百一十五日行二十度,夕伏西方,日度餘如初,與日合。一終三百九十八日,日餘八萬三千五百六十,行星三十三度,餘五萬九千九百三十五。


火:初與日合,伏,七十一日,日餘二萬四千八百一十二半,行五十四度,度餘四萬九千四百三十,晨見東方。
 \gezhu{
  去日十七度半強。}
 順,疾,日行七分之五,一百八日半行七十七度半。小遲,日行七分之四,一百二十六日行七十二度而大遲。



 日行七分之二,四十二日行十二度。留,不行,十二日而遲。日行十分之三,六十日退十八度。又留十二日。順,遲,四十二日行十二度。小疾,一百二十六日,行七十二度。一百八日半行七十七度半,夕伏西方,日度餘如初,
 與日合。一終七百七十九日,日餘四萬九千六百二十五,行星四百一十四,度餘三萬三千五百。除一周,定四十九度,度餘一萬七千三百七十五。


土:初與日合,伏,十八日,日餘四千四百八十二半,行二度,度餘四萬六千八百四十七半,晨見東方。
 \gezhu{
  去日十五度半強,}
 順,日行十二分之一,八十四日,行七度。留,不行,三十六日而逆。日行十七分之一,一百二日退六度。又留三十六日。順,八十四日行七度,夕伏西方,日度餘如初,與日
 合。一終三百七十八日,日餘八千九百六十五,行星十二度,度餘九萬三千六百九十五。


金:初與日合,伏,四十一日,日餘四萬九千六百八十四半,行五十一度,度餘四萬九千六百八十四半,見西方。
 \gezhu{
  去日十度。}
 順,疾,日行一度十三分之三,九十一日行一百十二度而小遲。日行一度十三分之二,九十一日行一百五度。又大遲。日行十五之十一,四十五日行三十三度。留,不行,八日而遲。日行三分之二,九日退六度,伏
 西方。伏六日,退四度而與日合。又六日退四度,晨見東方。逆,九日退六度。又留八日。順,四十五日行三十三度。小疾,九十一日行一百五度。



 大疾,九十一日行百一十二度,晨伏東方,日度餘如初,與日合。一終五百八十三日,日餘四萬八千六百一。除一周,行星定二百一十八度,度餘三萬六千七十六。



 一合二百九十一日,餘四萬九千六百八十四半,行星如之。


水:初與日合,伏,十七日,日餘七萬一千二百一十半,行
 三十四度,度餘七萬一千二百一十半,見西方。
 \gezhu{
  去日十七度。}
 順,疾,日行一度三分之一,十八日行二十四度而遲。日行七分之五,七日行五度。留,不行,四日,夕伏西方。伏十一日,退六度,而與日合。又十一日退六度,而晨見東方。留四日。順,遲,七日行五度。疾,十八日行二十四度,晨伏東方,日度餘如初,與日合。一終一百一十五日,日餘六萬六千七百二十五,行星如之。一合五十七日,日餘七萬一千二百一十半,行星亦如之。盈加縮減,十六除月
 行分,日法除盈縮分,以減度分,盈加縮減。



 推卦:因雨水大小餘,加大餘六,小餘三百一十九,小餘滿三千六百四十八成日。日滿二十七日餘不足加減不加周虛。



 元嘉二十年,承天奏上尚書:「今既改用《元嘉歷》,漏刻與先不同,宜應改革。按《景初歷》春分日長,秋分日短,相承所用漏刻,冬至後晝漏率長於冬至前。



 且長短增減,進退無漸,非唯先法不精,亦各傳寫謬誤。今二至二分,各
 據其正。



 則至之前後,無復差異。更增損舊刻,參以晷影,刪定為經,改用二十五箭。請臺勒漏郎將考驗施用。」從之。



 前世諸儒依圖緯云,月行有九道。故畫作九規,更相交錯,檢其行次,遲疾換易,不得順度。劉向論九道云:「青道二出黃道東,白道二出黃道西,黑道二出北,赤道二出南。」又云:「立春、春分,東從青道;立夏、夏至,南從赤道。秋白冬黑,各隨其方。」按日行黃道,陽路也,月者陰精,不由陽路,故或出其外,或入其內,出入去黃道不得過六度。
 入十三日有奇而出,出亦十三日有奇而入,凡二十七日而一入一出矣。交於黃道之上,與日相掩,則蝕焉。漢世劉洪推檢月行,作陰陽歷法。元嘉二十年,太祖使著作令史吳癸依洪法,制新術,令太史施用之。


《元嘉歷》月行陰陽法:陰陽歷損益率兼數一日益十七初二日
 \gezhu{
  前限餘六百六十五微分一千七百三十八}
 益十六十七
 三日益十五三十三四日益十二四十八五日益八六十六日益四六十八七日益一七十二八日損二七十三九日損六七十一十日損十六十五
 十一日損十三五十五十二日損十五四十二十三日
 \gezhu{
  後限餘二千一十九微分一千七十九}
 損十六二十七分日
 \gezhu{
  二千六百八十五半}
 損十六大
 \gezhu{
  大者五千三百七十一分之三千四百七十二十一歷周,五萬五千五百一十七半。}



 差率,一萬一百九十。



 微分法,一千八百七十八。



 推入陰陽歷術曰:以會月去入紀積月,餘以會數乘之,以所入紀交會差加之,周天乘之,滿微分法為大分,不
 盡為微分。大分滿周天去之,餘不滿歷周者為入陽歷。餘皆如月周得一日,算外,所求年正月合朔入歷也。不盡為日餘。



 求次月,加二日,日餘一千三百三十一,微分一千五百九十八,如法成日,日滿十三去之,除日餘如分日。陰陽歷竟平入端,入歷在前限餘前,後限餘後者,月行中道。



 求朔弦望定數:各置入遲疾歷盈縮定積分,以章歲乘之,差法除之,所得滿通法為大分。不盡,以微分法乘之,
 如法為微分。盈減縮加陰陽日餘,盈不足,以月周進退日而定,以定日餘乘損益兼數,為加時定數。



 推夜半入歷:以差率朔小餘,如微分法得一,以減入歷餘,不足,加月周而減之,卻一日,卻得分日,加其分,半微分為小分,即朔日夜半入歷歷餘小分也。



 求次日,加一日,日餘十六,小分三百二十,小分如會從餘,餘滿月周去之,又加一日。歷竟,下日餘滿分日去之,于入歷初也。不滿分日者,值之,加餘一千二百九十四,
 小分七百八十九半,為入次歷。



 求夜半定日:以朔小餘減入遲疾歷日餘,不足一日,卻得周日,加餘四百一十七,即月夜半入歷日及餘也。以日餘乘損益率,盈縮積分,為定積分。滿通法為大分,不盡以會月乘之,如法為小分,以盈加縮減入陰陽日餘,盈不足進退日而定也。



 以定日餘乘損益率,如月周,以損益兼數,為夜半定數。



 求昏明數:以損益率乘所近節氣夜漏,二百而一為明,
 以減損益率為昏,而以損益夜半數為昏明定數也。



 求月去黃道度:置加時若昏明定數,以十二除之為度,其餘三而一為少,不盡為強,二少弱也。所得為月去黃道度。



 大明六年,南徐州從事史祖沖之上表曰:古歷疏舛,頗不精密,群氏糾紛,莫審其要。何承天所奏,意存改革,而置法簡略,今已乖遠。以臣校之,三睹厥謬:日月所在,差覺三度;二至晷影,幾失一日;五星見伏,至差四旬,留逆
 進退,或移兩宿。分至乖失,則節閏非正;宿度違天,則伺察無準。臣生屬聖辰,逮在昌運,敢率愚瞽,更創新歷。謹立改易之意有二,設法之情有三。



 改易者,其一,以舊法一章十九歲有七閏,閏數為多,經二百年,輒差一日。



 節閏既移,則應改法,歷紀屢遷,實由此條。今改章法,三百九十一年有一百四十四閏。令卻合周、漢,則將來永用,無復差動。其二,以《堯典》云:「日短星昴,以正仲冬。」以此推之,唐代冬至,日在今宿之左五十許度。漢代之初,即用秦
 歷,冬至日在牽牛六度。漢武改立《太初歷》,冬至日在牛初。後漢《四分法》,冬至日在斗二十二。晉時姜岌以月蝕檢日,知冬至在斗十七。今參以中星,課以蝕望,冬至之日,在斗十一。通而計之,未盈百載,所差二度。舊法並令冬至日有定處,天數既差,則七曜宿度漸與歷舛。乖謬既著,輒應改制,僅合一時,莫能通遠,遷革不已,又由此條。今令冬至所在,歲歲微差,卻檢漢注,並皆審密,將來久用,無煩屢改。



 又設法者,其一,以子為辰首,位在正北,
 爻應初九,斗氣之端,虛為北方,列宿之中,元氣肇初,宜在此次。前儒虞喜,備論其義。今歷上元日度,發自虛一。



 其二,以日辰之號,甲子為先,歷法設元,應在此歲。而黃帝以來,世代所用,凡十一歷,上元之歲,莫值此名。今歷上元,歲在甲子。其三,以上元之歲,歷中眾條,並應以此為始,而《景初歷》交會遲疾,亦置紀差,裁合朔氣而已。條序紛互,不及古意。今設法,日月五緯,交會遲疾,悉以上元歲首為始。則合璧之曜,信而有征,連珠之暉,於是乎
 在,群流共源,實精古法。



 若夫測以定形,據以實效,縣象著明,尺表之驗可推,動氣幽微,寸管之候不忒。今臣所立,易以取信。但深練始終,大存整密,革新變舊,有約有繁。用約之條,理不自懼,用繁之意,顧非謬然。何者?夫紀閏參差,數各有分,分之為體,非細不密。臣是用深惜毫厘,以全求妙之準,不辭積累,以成永定之制。非為思而莫悟,知而不改也,竊恐贊有然否,每崇遠而隨近;論有是非,或貴耳而遺目。所以竭其管穴,俯洗同異之嫌,披
 心日月,仰希葵藿之照。若臣所上,萬一可采,伏願頒宣群司,賜垂詳究,庶陳錙銖,少增盛典。



 歷法上元甲子至宋大明七年癸卯,五萬一千九百三十九年算外。



 元法,五十九萬二千三百六十五。



 紀法,三萬九千四百九十一。



 章歲,三百九十一。



 章月,四千八百三十六。



 章閏,一百四十四。



 閏法,十二。



 月法,十一萬六千三百二十一。



 日法,三千九百三十九。



 餘數,二十萬七千四十四。



 歲餘,九千五百八十九。



 沒分,三百六十萬五千九百五十一。



 沒法,五萬一千七百六十一。



 周天,一千四百四十二萬四千六百六十四。



 虛分,萬四百四十九。



 行分法,二十三。



 小分法,一千七百一十七。



 通周,七十二萬六千八百一十。



 會周,七十一萬七千七百七十七。



 通法,二萬六千三百七十七。



 差率,三十九。



 推朔術:置入上元年數,算外,以章月乘之,滿章歲為積月,不盡為閏餘。閏餘二百四十七以上,其年有閏。以月法乘積月,滿日法為積日,不盡為小餘。六旬去積日,不盡為大餘。大餘命以甲子,算外,所求年天正十一月朔也。小餘千八百四十九以上,其月大。
 求次月,加大餘二十九,小餘二千九十,小餘滿日法從大餘,大餘滿六旬去之,命如前,次月朔也。求弦望:加朔大餘七,小餘千五百七,小分一,小分滿四從小餘,小餘滿日法從大餘,命如前,上弦日也。又加得望,又加得下弦,又加得後月朔也。



 推閏術:以閏餘減章歲,餘滿閏法得一月,命以天正,算外,閏所在也。閏有進退,以無中氣為正。推二十四氣術:置入上元年數,算外,以餘數乘之,滿紀
 法為積日,不盡為小餘。六旬去積日,不盡為大餘。大餘命以甲子,算外,天正十一月冬至日也。求次氣,加大餘十五,小餘八千六百二十六,小分五,小分滿六從小餘,小餘滿紀法從大餘,命如前,次氣日也。求土用事:加冬至大餘二十七,小餘萬五千五百二十八,季冬土用事日也。又加大餘九十一,小餘萬二千二百七十,次土用事日也。推沒術:以九十乘冬至小餘,以減沒分,滿沒法為日,不
 盡為日餘,命日以冬至,算外,沒日也。



 求次沒,加日六十九,日餘三萬四千四百四十二,餘滿沒法從日,次沒日也。



 日餘盡為滅。



 推日所在度術:以紀法乘朔積日為度實,周天去之,餘滿紀法為積度,不盡為度餘,命以虛一,次宿除之,算外,天正十一月朔夜半日所在度也。求次月,大月加度三十,小月加度二十九,入虛去度分。求行分,以小分法除度餘,所得為行分,不盡為小分。小
 分滿法從行分,行分滿法從度。求次日,加一度。入虛去行分六,小分百四十七。



 推月所在度術:以朔小餘乘百二十四為度餘。又以朔小餘乘八百六十為微分。



 微分滿月法從度餘,度餘滿紀法為度,以減朔夜半日所在,則月所在度。



 求次月,大月加度三十五,度餘三萬一千八百三十四,微分七萬七千九百六十七,小月加度二十二,度餘萬七千二百六十一,微分六萬三千七百三十六,入虛去
 度分也。



 遲疾歷:月行度損益率盈縮積分差法一日十四行分十三益七十盈初5304二日十四十一益六十五盈百八十四萬二千三百一十六
 5270三日十四八益五十七盈三百五十五萬七百六5219四日十四四益四十七盈五百五萬八千二百八5151五日十三二十二益三十四
 盈六百二十九萬七千八百五十七5066六日十三十七益二十二盈七百二十萬二千六百九十一4981七日十三十一益六盈七百七十七萬二千七百一十4
 879八日十三五損九盈七百九十四萬九百五十二4777九日十二二十二損二十四盈七百七十萬七千四百一十五4675十日十二十六損三十九盈七百七萬二千一百
 4573十一日十二十一損五十二盈六百三萬五千七4488十二日十二八損六十盈四百六十六萬三千一百4437十三日十二六損六十五
 盈三百九萬三百二4403十四日十二四損七十盈百三十八萬三千五百八十4369十五日十二五益六十七縮四十五萬七千六十九
 4368十六日十二七益六十二縮二百二十三萬七百五十五4420十七日十二十益五十五縮三百八十七萬五百一十四4471十八日十二十四益四十四縮五百三十萬九千三百八十五
 4539十九日十二十九益三十二縮六百四十八萬四百四4624二十日十三一益十九縮七百三十一萬六千六百八4709二十一日十三七益四縮七百八十一萬七千九百九十六
 4811二十二日十三十三損十一縮七百九十一萬七千六百七4913二十三日十三十九損二十七縮七百六十一萬五千四百四十5015二十四日十四一損三十九
 縮六百九十萬一千四百九十五5100二十五日十四六損五十二縮五百八十七萬二千七百三十五5185二十六日十四十損六十二縮四百四十九萬九千一百五十九
 5253二十七日十四十二損六十七縮二百八十五萬七千七百三十二5287二十八日十四十四損七十四縮百八萬二千三百七十九5321推入遲疾歷術:以通法乘朔積日為通實,通周去之,餘滿通法為日,不盡為日餘。命日算外,天正十一月朔夜
 半入歷日也。求次月,大月加二日,小月加一日,日餘皆萬一千七百四十六。歷滿二十七日,日餘萬四千六百三十一,則去之。



 求次日,加一日。求日所在定度:以夜半入歷日餘乘損益率,以損益盈縮積分,如差率而一,所得滿紀法為度,不盡為度餘,以盈加縮減平行度及餘為定度。益之或滿法,損之或不足,以紀法進退。



 求度行分如上法。求次
 日,如所入遲疾加之,虛去分如上法。



 陰陽歷損益率兼數一日益十六初二日益十五十六三日益十四三十一四日益十二四十五五日益九五十七六日益五六十六
 七日益一七十一八日損二七十二九日損六七十十日損十六十四十一日損十三五十四十二日損十五四十一十三日損十六二十六十四日損十六十
 推入陰陽歷術:置通實以會周去之,不滿交數三十五萬八千八百八十八半為朔入陽歷分,滿去之,為朔入陰歷分。各滿通法得一日,不盡為日餘,命日算外,天正十一月朔夜半入歷日也。



 求次月,大月加二日,小月加一日,日餘皆二萬七百七十九。歷滿十三日,日餘萬五千九百八十七半則去之。陽竟入陰,陰竟入陽。求次日,加一日。求朔望差,以二千二十九乘朔小餘,滿
 三百三為日餘,不盡倍之為小分,則朔差數也。加一十四日,日餘二萬一百八十六,小分百二十五,小分滿六百六從日餘,日餘滿通法為日,即望差數也。又加之,後月朔也。



 求合朔月食:置朔望夜半入陰陽歷日及餘,有半者去之,置小分三百三,以差數加之,小分滿六百六從日餘,日餘滿通法從日,日滿一歷去之。命日算外,則朔望加時入歷也。朔望加時入歷一日,日餘四千一百九十八,
 小分四百二十八以下,十二日,日餘萬一千七百八十八,小分四百八十一以上,朔則交會,望則月食。



 求合朔月食定大小餘:令差數日餘加夜半入遲疾歷餘,日餘滿通法從日,則朔望加時入歷也。以入歷餘乘損益率,以損益盈縮積分,如差法而一,以盈減縮加本朔望小餘,為定小餘。益之或滿法,損之或不足,以日法進退日。



 求合朔月食加時:以十二乘定小餘,滿日法得一辰,命
 以子,算外,加時所在辰也。有餘者四之,滿日法得一為少,二為半,三為太。又有餘者三之,滿日法得一為強,以強並少為少強,並半為半強,並太為太強。得二者為少弱,以並太為一辰弱,以前辰名之。



 求月去日道度:置入陰陽歷餘乘損益率,如通法而一,以損益兼數為定,定數十二而一為度,不盡三而一,為少、半、太。又不盡者,一為強,二為少弱,則月去日道數也。陽歷在表,陰歷在裏。



 (
 表略



 求昏明中星:各以度數加夜半日所在,則中星度也。



 推五星術:木率:千五百七十五萬三千八十二。火率:三千八十萬四千一百九十六。土率:千四百
 九十三萬三百五十四。金率:二千三百六萬一十四。水率:四
 百五十七萬六千二百四。推五星術:置度實各以率去之,餘以減率,其餘如紀法而一,為入歲日,不盡為日餘。命以天正朔,算外,星合日。



 求星合度:以入歲日及餘從天正朔日積度及餘,滿紀法從度,滿三百六十餘度分則去之,命以虛一,算外,星合所在度也。求星見日術:以伏日及餘,加星合日及餘,餘滿紀法從日,命如前,見日也。求星見度術:以伏度及餘,加星合度及餘,餘滿紀法從度,入虛去度分,命如前,星見度也。行五星法:以小分法除度餘,所得為行分,不盡為小分,及日加所行分滿法從度,留者因前,逆則減之,伏不盡度。



 從行入虛,去行分六,小分百四十七;逆行出虛,則加之。


木:初與日合,伏,十六日,餘萬七
 千八百三十二,行二度,度餘三萬七千五百四,晨見東方。從,日行四分,百一十二日,
 \gezhu{
  行十九度十一分。}
 留二十八日。


逆,日行三分,八十六日,
 \gezhu{
  退十一度五分。}
 又留二十八日。從,日行四分,百一十二日,夕伏西方。日度餘如初。一終,三百九十八日,日
 餘三萬五千六百六十四,行三十三度,度餘二萬五千二百一十五。


火:初與日合,伏,七十二日,日餘六百八,行五十五度,度餘二萬八千八百六十五,晨見東方。從,疾,日行十七分,九十二日,
 \gezhu{
  行六十八度。}
 小遲,日行十四
 分,九十二日,
 \gezhu{
  行五十六度。}
 大遲,日行九分,九十二日,
 \gezhu{
  行三十六度。}
 留十日。逆,日行六分,六十四日,
 \gezhu{
  退十六度十六分。}
 又留十日。從,遲,日行九分,九十二日。小疾,日行十四分,九十二日。大疾,日行十七分,九十二日,夕伏西方,日度餘如初。一終,七百八十日,
 日餘千二百一十六,行四百一十四度,度餘三萬二百五十八。除一周,定行四十九度,度餘萬九千八百九。


土:初與日合,伏,十七日,日餘千三百七十八,行一度,度餘萬九千三百三十三,晨見東方。行順,日行二分,八十四
 日,
 \gezhu{
  行七度七分。}
 留三十三日。行逆,日行一分,百一十日,
 \gezhu{
  退四度十八分。}
 又留三十三日。從,日行二分,八十四日,夕伏西方,日度餘如初。一終,三百七十八日,日
 餘二千七百五十六,行十二度,度餘三萬一千七百九十八。


金:初與日合,伏,三十九日,餘三萬八千一百二十六,行四十九度,度餘三萬八千一百二十六,夕見西方。從,疾,日行一度五分,九十二日,
 \gezhu{
  行百十
  二度。}


小遲,日行一度四分,九十二日,
 \gezhu{
  行百八度。}
 大遲,日行十七分,四十五日,
 \gezhu{
  行三十三度六分。}
 留九日。遲,日行十六分,
 \gezhu{
  退六度六分。}
 夕伏西方。伏五日,退五度,而與日合。又五日退五度,而晨見東方。逆,日行十六分,九日。



 留九日。從,遲,日行十七分,
 四十五日。小疾,日行一度四分,九十二日。大疾,日行一度五分,九十二日,晨伏東方,日度餘如初。一終,五百八十三日,日餘三萬六千七百六十一,行星如之。除一周,定行二百十八度,度餘二萬六千三百一十二。一合,二百九十一日,日餘三萬八千一百二十六,行星亦如
 之。


水:初與日合,伏,十四日,日餘三萬七千一百十五,行三十度,度餘三萬七千一百一十五,夕見西方。從,疾,日行一度六分,二十三日,
 \gezhu{
  行二十九度。}


遲,日行二十分,八日,
 \gezhu{
  行六度二十二分。}
 留二日。遲,日行十一分,二日,
 \gezhu{
  退二十二分。}
 夕伏西方。伏八日,退八
 度,而與日合。又八日,退八度,晨見東方。逆,日行十一分,二日。留二日。從,遲,日行二十分,八日。疾,日行一度六分,二十三日,晨伏東方,日度餘如初。一終,百一十五日,日餘三萬四千七百三十九,行星如之。一合,五十七日,日餘三萬七千一百一十五,行星亦如之。



 上元之歲,歲在甲子,天正甲子朔夜半冬至,日月五星,聚于虛度之初,陰陽遲疾,並自此始。



 世祖下之有司,使內外博議,時人少解歷數,竟無異同之辨。唯太子旅賁中郎將戴法興議,以為:三精數微,五緯會始,自非深推測,窮識晷變,豈能刊古革今,轉正圭宿。案沖之所議,每有違舛,竊以愚見,隨事辨問。案沖之新推歷術,「今冬至所在,歲歲微差」。臣法興議:夫二至發斂,南北之極,日有恆度,而宿無改位。古歷冬至,皆在建星。戰國橫騖,史官
 喪紀,爰及漢初,格候莫審,後雜覘知在南斗二十二度,元和所用,即與古歷相符也。逮至景初,而終無毫忒。《書》云:「日短星昴,以正仲冬。」直以月維四仲,則中宿常在衛陽,羲、和所以正時,取其萬世不易也。



 沖之以為唐代冬至日在今宿之左五十許度,遂虛加度分,空撤天路。其置法所在,近違半次,則四十五年九月,率移一度。在《詩》「七月流火」,此夏正建申之時也。「定之方中」,又小雪之節也。若冬至審差,則豳公火流,晷長一尺五寸,楚宮之作,
 晝漏五十三刻,此詭之甚也。仲尼曰:「丘聞之,火伏而後蟄者畢。今火猶西流,司歷過也。」就如沖之所誤,則星無定次,封有差方。名號之正,古今必殊,典誥之音,代不通軌,堯之開、閉,今成建、除。今之壽星,乃周之鶉尾,即時東壁,已非玄武,軫星頓屬蒼龍,誣天痛經,乃至於此。



 沖之又改章法三百九十一年有一百四十四閏。臣法興議:夫日有緩急,故斗有闊狹,古人制章,立為中格,年積十九,常有七閏,晷或虛盈,此不可革。沖之削閏壞章,倍減
 餘數,則一百三十九年二月,於四分之科,頓少一日;七千四百二十九年,輒失一閏。夫日少則先時,閏失則事悖。竊聞時以作事,事以厚生,以此乃生人之大本,歷數之所先,愚恐非沖之淺慮妄可穿鑿。



 沖之又命上元日度發自虛一,云虛為北方列宿之中。臣法興議:沖之既云冬至歲差,又謂虛為北中,舍形責影,未足為迷。何者?凡在天非日不明,居地以斗而辨。借令冬至在虛,則黃道彌遠,東北當為黃鐘之宮,室壁應屬玄枵之位,虛宿
 豈得復為北中乎?曲使分至屢遷,而星次不改,招搖易繩,而律呂仍往,則七政不以璣衡致齊,建時亦非攝提所紀,不知五行何居,六屬安託?



 沖之又令上元年在甲子。臣法興議:夫置元設紀,各有所尚,或據文於圖讖,或取效於當時。沖之云,「群氏糾紛,莫審其會」。昔《黃帝》辛卯,日月不過;《顓頊》乙卯,四時不忒;《景初》壬辰,晦無差光;《元嘉》庚辰,朔無錯景,豈非承天者乎!沖之茍存甲子,可謂為合以求天也。



 沖之又令日月五緯,交會遲疾,悉以上
 元為始。臣法興議:夫交會之元,則食既可求,遲疾之際,非凡夫所測。昔賈逵略見其差,劉洪觕著其術。至於疏密之數,莫究其極。且五緯所居,有時盈縮,即如歲星在軫,見超七辰,術家既追算以會今,則往之與來,斷可知矣。《景初》所以紀首置差,《元嘉》兼又各設後元者,其並省功於實用,不虛推以為煩也。沖之既違天於改易,又設法以遂情,愚謂此治歷之大過也。



 臣法興議:日有八行,各成一道,月有一道,離為九行,左交右疾,倍半相違,其
 一終之理,日數宜同。沖之通周與會周相覺九千四十,其陰陽七十九周有奇,遲疾不及一匝。此則當縮反盈,應損更益。



 沖之隨法興所難辯折之曰:臣少銳愚尚,專功數術,搜練古今,博采沈奧,唐篇夏典,莫不揆量,周正漢朔,咸加該驗。罄策籌之思,究疏密之辨。至若立圓舊誤,張衡述而弗改;漢時斛銘,劉歆詭謬其數,此則算氏之劇疵也。《乾象》之弦望定數,《景初》之交度周日,匪謂測候不精,遂乃乘除翻謬,斯又歷家之甚失也。及鄭玄、闞
 澤、王蕃、劉徽,並綜數藝,而每多疏舛。臣昔以暇日,撰正眾謬,理據炳然,易可詳密,此臣以俯信偏識,不虛推古人者也。按何承天歷,二至先天,閏移一月,五星見伏,或違四旬,列差妄設,當益反損,皆前術之乖遠,臣歷所改定也。既沿波以討其源,刪滯以暢其要,能使躔次上通,晷管下合,反以譏詆,不其惜乎!尋法興所議六條,並不造理難之關楗。謹陳其目。



 其一,日度歲差,前法所略,臣據經史辨正此數,而法興設難,徵引《詩》《書》,三事皆謬。其
 二,臣校晷景,改舊章法,法興立難,不能有詰,直云「恐非淺慮,所可穿鑿」。其三,次改方移,臣無此法,求術意誤,橫生嫌貶。其四,歷上元年甲子,術體明整,則茍合可疑。其五,臣其歷七曜,咸始上元,無隙可乘,復云「非凡夫所測」。其六,遲疾陰陽,法興所未解,誤謂兩率日數宜同。凡此眾條,或援謬目譏,或空加抑絕,未聞折正之談,厭心之論也。謹隨詰洗釋,依源徵對。仰照天暉,敢罄管穴。



 法興議曰:「夫二至發斂,南北之極,日有恒度,而宿無改位。故
 古歷冬至,皆在建星」。沖之曰:周漢之際,疇人喪業,曲技競設,圖緯實繁,或借號帝王以崇其大,或假名聖賢以神其說。是以讖記多虛,桓譚知其矯妄;古歷舛雜,杜預疑其非直。按《五紀論》黃帝歷有四法,顓頊、夏、周並有二術,詭異紛然,則孰識其正,此古歷可疑之據一也。夏歷七曜西行,特違眾法,劉向以為後人所造,此可疑之據二也。殷歷日法九百四十,而《乾鑿度》云殷歷以八十一為日法。若《易緯》非差,殷歷必妄,此可疑之據三也。《顓頊》
 歷元,歲在乙卯,而《命歷序》云:「此術設元,歲在甲寅。」此可疑之據四也。《春秋》書食有日朔者凡二十六,其所據歷,非周則魯。以周歷考之,檢其朔日,失二十五,魯歷校之,又失十三。二歷並乖,則必有一偽,此可疑之據五也。古之六術,並同《四分》,《四分》之法,久則後天。以食檢之,經三百年,輒差一日。古歷課今,其甚疏者,朔後天過二日有餘。以此推之,古術之作,皆在漢初周末,理不得遠。且卻校《春秋》,朔並先天,此則非三代以前之明徵矣,此可疑
 之據六也。尋《律歷志》,前漢冬至日在斗牛之際,度在建星,其勢相鄰,自非帝者有造,則儀漏或闕,豈能窮密盡微,纖毫不失。建星之說,未足證矣。



 法興議曰:「戰國橫騖,史官喪紀,爰及漢初,格候莫審,後雜覘知在南斗二十二度,元和所用,即與古歷相符也。逮至景初,終無毫忒。」沖之曰:古術訛雜,其詳闕聞,乙卯之歷,秦代所用,必有效於當時,故其言可徵也。漢武改創,檢課詳備,正儀審漏,事在前史,測星辨度,理無乖遠。今議者所是不實見,
 所非徒為虛妄,辨彼駭此,既非通談,運今背古,所誣誠多,偏據一說,未若兼今之為長也。



 《景初》之法,實錯五緯,今則在衝口,至曩已移日。蓋略治朔望,無事檢候,是以晷漏昏明,並即《元和》,二分異景,尚不知革,日度微差,宜其謬矣。



 法興議曰:「《書》云『日短星昴,以正仲冬』。直以月推四仲,則中宿常在衛陽,羲、和所以正時,取其萬代不易也。沖之以為唐代冬至,日在今宿之左五十許度,遂虛加度分,空撤天路。」沖之曰:《書》以上四星昏中審分至者,據
 人君南面而言也。且南北之正,其詳易準,流見之勢,中天為極。先儒注述,其義僉同,而法興以為《書》說四星,皆在衛陽之位,自在巳地,進失向方,退非始見,迂回經文,以就所執,違訓詭情,此則甚矣。舍午稱巳,午上非無星也。必據中宿,餘宿豈復不足以正時。若謂舉中語兼七列者,觜參尚隱,則不得言,昴星雖見,當云伏矣,奎婁已見,復不得言伏見囗囗不得以為辭,則名將何附。若中宿之通非允,當實謹檢經旨,直云星昴,不自衛陽,衛陽無
 自顯之義,此談何因而立。茍理無所依,則可愚辭成說,曾泉、桑野,皆為明證,分至之辨,竟在何日,循復再三,竊深嘆息。



 法興議曰:「其置法所在,近違半次,則四十五年九月率移一度。」沖之曰:《元和》日度,法興所是,唯征古歷在建星,以今考之,臣法冬至亦在此宿,斗二十二了無顯證,而虛貶臣歷乖差半次,此愚情之所駭也。又年數之餘有十一月,而議云九月,涉數每乖,皆此類也。月盈則食,必在日衝,以檢日則宿度可辨,請據效以課疏密。
 按太史註記,元嘉十三年十二月十六日中夜月蝕盡,在鬼四度,以衝計之,日當在牛六。依法興議:「在女七」。又十四年五月十五日丁夜月蝕盡,在斗二十六度,以沖計之,日當在井三十,依法興議曰:「日在柳二。」又二十八年八月十五日丁夜月蝕,在奎十一度,以衝計之,日當在角二;依法興議曰:「日在角十二。」又大明三年九月十五日乙夜月蝕盡,在胃宿之末,以衝計之,日當在氐十二;依法興議曰:「日在心二。」凡此四蝕,皆與臣法符同,
 纖毫不爽,而法興所據,頓差十度,違衝移宿,顯然易睹。故知天數漸差,則當式遵以為典,事驗昭晰,豈得信古而疑今。



 法興議曰:「在《詩》『七月流火』,此夏正建申之時也。『定之方中』,又小雪之節也。若冬至審差,則豳公火流,晷長一尺五寸,楚宮之作,晝漏五十三刻,此詭之甚也。」沖之曰:臣按此議三條皆謬。《詩》稱流火,蓋略舉西移之中,以為驚寒之候。流之為言,非始動之辭也。就如始說,冬至日度在斗二十二,則火星之中,當在大暑之前,豈鄰
 建申之限。此專自攻糾,非謂矯失。《夏小正》:「五月昏,大火中。」此復在衛陽之地乎?又謂臣所立法,楚宮之作,在九月初。按《詩》傳箋皆謂定之方中者,室辟昏中,形四方也。然則中天之正,當在室之八度。



 臣歷推之,元年立冬後四日,此度昏中,乃處十月之初,又非寒露之日也。議者之意,蓋誤以周世為堯時,度差五十,故致此謬。小雪之節,自信之談,非有明文可據也。



 法興議曰:「仲尼曰:『丘聞之,火伏而後蟄者畢。今火猶西流,司歷過也。』就如沖之
 所誤,則星無定次,卦有差方,名號之正,古今必殊,典誥之音,時不通軌。堯之開、閉,今成建、除,今之壽星,乃周之鶉尾也。即時東壁,已非玄武,軫星頓屬蒼龍,誣天背經,乃至於此。」沖之曰:臣以為辰極居中,而列曜貞觀,群像殊體,而陰陽區別,故羽介咸陳,則水火有位,蒼素齊設,則東西可準,非以日之所在,定其名號也。何以明之?夫陽爻初九,氣始正北,玄武七列,虛當子位。



 若圓儀辨方,以日為主,冬至所舍,當在玄枵;而今之南極,乃處東維,
 違體失中,其義何附。若南北以冬夏稟稱,則卯酉以生殺定號,豈得春躔義方,秋麗仁域,名舛理乖,若此之反哉!因茲以言,因知天以列宿分方,而不在於四時,景緯環序,日不獨守故轍矣。至於中星見伏,記籍每以審時者,蓋以歷數難詳,而天驗易顯,各據一代所合,以為簡易之政也。亦猶夏禮未通商典,《濩》容豈襲《韶》節,誠天人之道同差,則藝之興,因代而推移矣。月位稱建,諒以氣之所本,名隨實著,非謂斗杓所指。近校漢時,已差半次,
 審斗節時,其效安在。或義非經訓,依以成說,將緯候多詭,偽辭間設乎?次隨方名,義合宿體。分至雖遷,而厥位不改,豈謂龍火貿處,金水亂列,名號乖殊之譏,抑未詳究。至如壁非玄武,軫屬蒼龍,瞻度察晷,實效咸然。《元嘉歷法》,壽星之初,亦在翼限,參校晉注,顯驗甚眾。



 天數差移,百有餘載,議者誠能馳辭騁辯,令南極非冬至,望不在衝,則此談乃可守耳。若使日遷次留,則無事屢嫌,乃臣歷之良證,非難者所宜列也。尋臣所執,必據經史,遠
 考唐典,近征漢籍,讖記碎言,不敢依述,竊謂循經之論也。月蝕檢日度,事驗昭著,史注詳論,文存禁閣,斯又稽天之說也。《堯典》四星,並在衛陽,今之日度,遠淮元和,誣背之誚,實此之謂。



 法興議曰:「夫日有緩急,故斗有闊狹,古人制章,立為中格,年積十九,常有七閏,晷或盈虛,此不可革。沖之削閏壞章,倍減餘數,則一百三十九年二月,於四分之科,頓少一日;七千四百二十九年,輒失一閏。夫日少則先時,閏失則事悖。竊聞時以作事,事以厚
 生,此乃生民之所本,歷數之所先。愚恐非沖之淺慮,妄可穿鑿。」沖之曰:按《後漢書》及《乾象說》,《四分歷法》,雖分章設篰創自元和,而晷儀眾數定於嘉平三年。《四分志》,立冬中影長一丈,立春中影九尺六寸。尋冬至南極,日晷最長,二氣去至,日數既同,則中影應等,而前長後短,頓差四寸,此歷景冬至後天之驗也。二氣中影,日差九分半弱,進退均調,略無盈縮。以率計之,二氣各退二日十二刻,則晷影之數,立冬更短,立春更長,並差二寸,二氣
 中影俱長九尺八寸矣。即立冬、立春之正日也。以此推之,歷置冬至,後天亦二日十二刻也。嘉平三年,時歷丁丑冬至,加時正在日中。以二日十二刻減之,天定以乙亥冬至,加時在夜半後三十八刻。又臣測景歷紀,躬辨分寸,銅表堅剛,暴潤不動,光晷明潔,纖毫盡然。據大明五年十月十日,影一丈七寸七分半,十一月二十五日,一丈八寸一分太,二十六日,一丈七寸五分彊,折取其中,則中天冬至,應在十一月三日。求其蚤晚,令後二日
 影相減,則一日差率也。倍之為法,前二日減,以百刻乘之為實,以法除實,得冬至加時在夜半後三十一刻,在《元嘉歷》後一日,天數之正也。量檢竟年,則數減均同,異歲相課,則遠近應率。臣因此驗,考正章法。今以臣歷推之,刻如前,竊謂至密,永為定式。尋古歷法並同《四分》,《四分》之數久則後天,經三百年,朔差一日。是以漢載四百,食率在晦。魏代已來,遂革斯法,世莫之非者,誠有效於天也。章歲十九,其疏尤甚,同出前術,非見經典。而議云
 此法自古,數不可移。若古法雖疏,永當循用,謬論誠立,則法興復欲施《四分》於當今矣,理容然乎?臣所未譬也。若謂今所革創違舛失衷者,未聞顯據有以矯奪臣法也。《元嘉歷》術,減閏餘二,直以襲舊分粗,故進退未合。



 至於棄盈求正,非為乖理。就如議意,率不可易,則分無增損,承天置法,復為違謬。節氣蚤晚,當循《景初》,二至差三日,曾不覺其非,橫謂臣歷為失,知日少之先時,未悟增月甚惑也。誠未睹天驗,豈測歷數之要,生民之本,諒
 非率意所斷矣。又法興始云窮識晷變,可以刊舊革今,復謂晷數盈虛,不可為準,互自違伐,罔識所依。若推步不得準,天功絕於心目,未詳歷紀何因而立。案《春秋》以來千有餘載,以食檢朔,曾無差失,此則日行有恒之明徵也。且臣考影彌年,窮察毫微,課驗以前,合若符契,孟子以為千歲之日至,可坐而知,斯言實矣。日有緩急,未見其證,浮辭虛貶,竊非所懼。



 法興議曰:「沖之既云冬至歲差,又謂虛為北中,舍形責影,未足為迷。何者?



 凡在天非
 日不明,居地以斗而辨,借令冬至在虛,則黃道彌遠,東北當為黃鐘之宮,室壁應屬玄枵之位,虛宿豈得復為北中乎?曲使分至屢遷,而星次不改,招搖易繩,而律呂仍往,則七政不以璣衡致齊,建時亦非攝提所紀,不知五行何居,六屬安託。」



 沖之曰:此條所嫌,前牒已詳。次改方移,虛非中位,繁辭廣證,自構紛惑,皆議者所謬誤,非臣法之違設也。七政致齊,實謂天儀,鄭、王唱述,厥訓明允,雖有異說,蓋非實義。



 法興議曰:「夫置元設紀,各有所
 尚,或據文於圖讖,或取效於當時。沖之云『群氏糾紛,莫審其會。』昔《黃帝》辛卯,日月不過,《顓頊》乙卯,四時不忒,《景初》壬辰,晦無差光,《元嘉》庚辰,朔無錯景,豈非承天者乎?沖之茍存甲子,可謂為合以求天也。」沖之曰:夫歷存效密,不容殊尚,合讖乖說,訓義非所取,雖驗當時,不能通遠,又臣所未安也。元值始名,體明理正。未詳辛卯之說何依,古術詭謬,事在前牒,溺名喪實,殆非索隱之謂也。若以歷合一時,理無久用,元在所會,非有定歲者,今以
 效明之。夏、殷以前,載籍淪逸,《春秋》漢史,咸書日蝕,正朔詳審,顯然可徵。以臣歷檢之,數皆協同,誠無虛設,循密而至,千載無殊,則雖遠知矣。備閱曩法,疏越實多,或朔差三日,氣移七晨,未聞可以下通於今者也。元在乙丑,前說以為非正,今值甲子,議者復疑其茍合,無名之歲,自昔無之,則推先者,將何從乎?歷紀之作,幾於息矣。夫為合必有不合,願聞顯據,以覈理實。



 法興曰:「夫交會之元,則蝕既可求,遲疾之際,非凡夫所測。昔賈逵略見
 其差,劉洪粗著其術,至於疏密之數,莫究其極。且五緯所居,有時盈縮,即如歲星在軫,見超七辰,術家既追算以會今,則往之與來,斷可知矣。《景初》所以紀首置差,《元嘉》兼又各設後元者,其並省功於實用,不虛推以為煩也。沖之既違天於改易,又設法以遂情,愚謂此治歷之大過也。」沖之曰:遲疾之率,非出神怪,有形可檢,有數可推,劉、賈能述,則可累功以求密矣。議又云「五緯所居,有時盈縮」。「歲星在軫,見超七辰」。謂應年移一辰也。案歲星
 之運,年恒過次,行天七匝,輒超一位。代以求之,歷凡十法,並合一時,此數咸同,史注所記,天驗又符。此則盈次之行,自其定準,非為衍度濫徙,頓過其衝也。若審由盈縮,豈得常疾無遲。夫甄耀測象者,必料分析度,考往驗來,準以實見,據以經史。曲辯碎說,類多浮詭,甘、石之書,互為矛盾。今以一句之經,誣一字之謬,堅執偏論,以罔正理,此愚情之所未厭也。算自近始,眾法可同,但《景初》之二差,承天之後元,實以奇偶不協,故數無盡同,為遺前
 設後,以從省易。夫建言倡論,豈尚矯異,蓋令實以文顯,言勢可極也。稽元曩歲,群數咸始,斯誠術體,理不可容譏;而譏者以為過,謬之大者。然則《元嘉》置元,雖七率舛陳,而猶紀協甲子,氣朔俱終,此又過謬之小者也。必當虛立上元,假稱歷始,歲違名初,日避辰首,閏餘朔分,月緯七率,並不得有盡,乃為允衷之製乎?設法情實,謂意之所安;改易違天,未睹理之譏者也。



 法興曰:「日有八行,合成一道,月有一道,離為九行,左交右疾,倍半相違,其
 一終之理,日數宜同。沖之通同與會周相覺九千四十,其陰陽七十九周有奇,遲疾不及一匝,此則當縮反盈,應損更益。」沖之曰:此議雖游漫無據,然言迹可檢。



 按以日八行譬月九道,此為月行之軌,當循一轍,環匝於天,理無差動也。然則交會之際,當有定所,豈容或斗或牛,同麗一度。去極應等,安得南北無常。若日月非例,則八行之說是衍文邪?左交右疾,語甚未分,為交與疾對?為舍交即疾?若舍交即疾,即交在平率入歷七日及二十
 一日是也。值交蝕既當在盈縮之極,豈得損益,或多或少。若交與疾對,則在交之衝,當為遲疾之始,豈得入歷或深或淺,倍半相違,新故所同,復摽此句,欲以何明。臣覽歷書,古今略備,至如此說,所未前聞,遠乖舊準,近背天數,求之愚情,竊所深惑。尋遲疾陰陽不相生,故交會加時,進退無常,昔術著之久矣,前儒言之詳矣。而法興云日數同。竊謂議者未曉此意,乖謬自著,無假驟辯,既云盈縮失衷,復不備記其數,或自嫌所執,故汎略其說
 乎?又以全為率,當互因其分,法興所列二數皆誤,或以八十為七十九,當縮反盈,應損更益,此條之謂矣。總檢其議,豈但臣歷不密,又謂何承天法乖謬彌甚。



 若臣歷宜棄,則承天術益不可用。法興所見既審,則應革創。至非景極,望非日沖,凡諸新說,必有妙辯乎?



 時法興為世祖所寵,天下畏其權,既立異議,論者皆附之。唯中書舍人巢尚之是沖之之術,執據宜用。上愛奇慕古,欲用沖之新法,時大明八年也。故須明年改元,因此改歷。未及
 施用,而宮車晏駕也。



\end{pinyinscope}