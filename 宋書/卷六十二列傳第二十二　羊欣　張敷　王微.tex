\article{卷六十二列傳第二十二 羊欣 張敷 王微}

\begin{pinyinscope}

 羊欣,字敬元,泰山南城人也。曾祖忱,晉徐州刺史。祖權,黃門郎。父不疑,桂陽太守。欣少靖默,無競於人,美言笑,善容止。汎覽經籍,尤長隸書。不疑初為烏程令,欣時年
 十二,時王獻之為吳興太守,甚知愛之。獻之嘗夏月入縣,欣著新絹裙晝寢,獻之書裙數幅而去。欣本工書,因此彌善。起家輔國參軍,府解還家。



 隆安中,朝廷漸亂,欣優游私門,不復進仕。會稽王世子元顯每使欣書,常辭不奉命,元顯怒,乃以為其後軍府舍人。此職本用寒人,欣意貌恬然,不以高卑見色,論者稱焉。欣嘗詣領軍將軍謝混,混拂席改服,然後見之。時混族子靈運在坐,退告族兄瞻曰:「望蔡見羊欣,遂易衣改席。」欣由此益知名。



 桓玄輔政,領平西將軍,以欣為平西參軍,仍轉主簿,參預機要。欣欲自疏,時漏密事,玄覺其此意,愈重之,以為楚臺殿中郎。謂曰:「尚書政事之本,殿中禮樂所出。卿昔處股肱,方此為輕也。」欣拜職少日,稱病自免,屏居里巷,十餘年不出。



 義熙中,弟徽被遇於高祖,高祖謂咨議參軍鄭鮮之曰:「羊徽一時美器,世論猶在兄後,恨不識之。」即板欣補右將軍劉籓司馬,轉長史,中軍將軍道憐諮議參軍。出為新安太守。在郡四年,簡惠著稱。除臨川王
 義慶輔國長史,廬陵王義真車騎諮議參軍,並不就。太祖重之,以為新安太守,前後凡十三年,游玩山水,甚得適性。轉在義興,非其好也。頃之,又稱病篤自免歸。除中散大夫。



 素好黃老,常手自書章,有病不服藥,飲符水而已。兼善醫術,撰《藥方》十卷。欣以不堪拜伏,辭不朝覲,高祖、太祖並恨不識之。自非尋省近親,不妄行詣,行必由城外,未嘗入六關。元嘉十九年,卒,時年七十三。子俊,早卒。



 弟徽,字敬猷,世譽多欣。高祖鎮京口,以為記室參軍掌
 事。八年,遷中書郎,直西省。後為太祖西中郎長史、河東太守。子瞻,元嘉末為世祖南中郎長史、尋陽太守,卒官。



 張敷,字景胤,吳郡人,吳興太守邵子也。生而母沒。年數歲,問母所在,家人告以死生之分,敷雖童蒙,便有思慕之色。年十許歲,求母遺物,而散施已盡,唯得一畫扇,乃緘錄之,每至感思,輒開笥流涕。見從母,常悲感哽咽。性整貴,風韻甚高,好讀玄書,兼屬文論,少有盛名。高祖見而愛之,以為世子中軍參軍,數見接引。永初初,遷秘書
 郎。嘗在省直,中書令傅亮貴宿權要,聞其好學,過候之;敷臥不即起,亮怪而去。



 父邵為湘州,去官侍從。太祖版為西中郎參軍。元嘉初,為員外散騎侍郎,秘書丞。江夏王義恭鎮江陵,以為撫軍功曹,轉記室參軍。時義恭就太祖求一學義沙門,比沙門求見發遣,會敷赴假還江陵,太祖謂沙門曰:「張敷應西,當令相載。」



 及敷辭,上謂曰:「撫軍須一意懷道人,卿可以後め載之,道中可得言晤。」敷不奉旨,曰:「臣性不耐雜。」上甚不說。



 遷正員郎。中書舍
 人狄當、周赳並管要務,以敷同省名家,欲詣之。赳曰:「彼若不相容,便不如不往。詎可輕往邪?」當曰:「吾等並已員外郎矣,何憂不得共坐。」敷先設二床,去壁三四尺,二客就席,酬接甚歡,既而呼左右曰:「移我遠客。」赳等失色而去。其自摽遇如此。善持音儀,盡詳緩之致,與人別,執手曰:「念相聞。」餘響久之不絕。張氏後進至今慕之,其源流起自敷也。



 遷黃門侍郎,始興王濬後軍長史,司徒左長史。未拜,父在吳興亡,報以疾篤,敷往奔省,自發都至吳
 興成服,凡十餘日,始進水漿。葬畢,不進鹽菜,遂毀瘠成疾。世父茂度每止譬之,輒更感慟,絕而復續。茂度曰:「我冀譬汝有益,但更甚耳。」自是不復往。未期而卒,時年四十一。



 琅邪顏延之書弔茂度曰:「賢弟子少履貞規,長懷理要,清風素氣,得之天然。



 言面以來,便申忘年之好,比雖艱隔成阻,而情問無睽。薄莫之人,冀其方見慰說,豈謂中年,奄為長往,聞問悼心,有兼恒痛。足下門教敦至,兼實家寶,一旦喪失,何可為懷。」其見重如此。世祖即位,
 詔曰:「司徒故左長史張敷,貞心簡立,幼樹風規。居哀毀滅,孝道淳至,宜在追甄,於以報美。可追贈侍中。」於是改其所居稱為孝張里。無子。



 王微,字景玄,琅邪臨沂人,太保弘弟子也。父孺,光祿大夫。微少好學,無不通覽,善屬文,能書畫,兼解音律、醫方、陰陽術數。年十六,州舉秀才,衡陽王義季右軍參軍,並不就。起家司徒祭酒,轉主簿,始興王濬後軍功曹記室參軍,太子中舍人,始興王友。父憂去官,服闋,除南平王
 鑠右軍咨議參軍。微素無宦情,稱疾不就。仍除中書侍郎,又擬南琅邪、義興太守,並固辭。吏部尚書江湛舉微為吏部郎,微與湛書曰:弟心病亂度,非但蹇蹙而已,此處朝野所共知。騶會忽扣蓽門,閭里咸以為祥怪,君多識前世之載,天植何其易傾。弟受海內駭笑,不過如燕石禿鶖邪,未知君何以自解於良史邪?今雖王道鴻鬯,或有激朗於天表,必欲探援潛寶,傾海求珠,自可卜肆巫祠之間,馬棧牛口之下,賞劇孟於博徒,拔卜式於芻
 牧。亦有西戎孤臣,東都戒士,上窮範馳之御,下盡詭遇之能,兼鱗雜襲者,必不乏於世矣。且廬於承明,署乎金馬,皆明察之官,又賢於管庫之末。何為劫勒通家疾病人,塵穢難堪之選,將以靖國,不亦益囂乎。《書》云「任官維賢才」。而君擢士先疹廢,芃耳棫樸,似不如此。且弟曠違兄姊,迄將十載,姊時歸來,終不任輿曳入閣,兄守金城,永不堪扶抱就路,若不憊疾,非性僻而何。比君曰表裏,無假長目飛耳也。



 常謂生遭太公,將即華士之戮;幸
 遇管叔,必蒙僻儒之養。光武以馮衍才浮其實,故棄而不齒。諸葛孔明云:「來敏亂郡,過於孔文舉。」況無古人之才概,敢干周、漢之常刑。彼二三英賢,足為曉治與否?恐君逄此時,或亦不免高閣,乃復假名不知己者,豈欲自比衛賜邪?君欲高斅山公,而以仲容見處,徒以捶提禮學,本不參選,鄙夫瞻彼,固不任下走,未知新沓何如州陵耳。而作不師古,坐亂官政,誣飾蚯蚓,冀招神龍,如復託以真素者,又不宜居華留名,有害風俗。君亦不至期
 人如此,若交以為人賜,舉未以己勞,則商販之事,又連所不忍聞也。豈謂不肖易擢,貪者可誘,凡此數者,君必居一焉。雖假天口於齊駢,藉鬼說於周季,公孫碎毛髮之文,莊生縱漭瀁之極,終不能舉其契,為之辭矣。子將明魂,必靈咍於萬里,汝、潁餘彥,將拂衣而不朝。浮華一開,風俗或從此而爽。鬼谷以揣情為最難,何君忖度之輕謬。



 今有此書,非敢叨擬中散,誠不能顧影負心,純盜虛聲,所以綿絡累紙,本不營尚書虎爪板也。成童便往
 來居舍,晨省復經周旋,加有諸甥,亦何得頓絕慶吊。



 然生平之意,自於此都盡。君平公云:「生我名者殺我身。」天爵且猶滅名,安用吏部郎哉!其舉可陋,其事不經,非獨搢紳者不道,僕妾皆將笑之。忽忽不樂,自知壽不得長,且使千載知弟不詐諼耳。



 微既為始興王濬府吏,浚數相存慰,微奉答箋書,輒飾以辭采。微為文古甚,頗抑揚,袁淑見之,謂為訴屈。微因此又與從弟僧綽書曰:吾雖無人鑒,要是早知弟,每共宴語,前言何嘗不以止足為
 貴。且持盈畏滿,自是家門舊風,何為一旦落漠至此,當局苦迷,將不然邪!詎容都不先聞,或可不知耳。衣冠胄胤,如吾者甚多,才能固不足道,唯不傾側溢詐,士頗以此容之。至於規矩細行,難可詳料。疹疾日滋,縱恣益甚,人道所貴,廢不復脩。幸值聖明兼容,置之教外,且舊恩所及,每蒙寬假。吾亦自揆疾疹重侵,難復支振,民生安樂之事,心死久矣。所以解日偷存,盡於大布糲粟,半夕安寢,便以自度,血氣盈虛,不復稍道,長以大散為和羹,
 弟為不見之邪?疾廢居然,且事一己,上不足敗俗傷化,下不至毀辱家門,泊爾尸居,無方待化。凡此二三,皆是事實。吾與弟書,不得家中相欺也。州陵此舉,為無所因,反覆思之,了不能解。豈見吾近者諸箋邪,良可怪笑。



 吾少學作文,又晚節如小進,使君公欲民不偷,每加存飾,酬對尊貴,不厭敬恭。且文詞不怨思抑揚,則流澹無味。文好古,貴能連類可悲,一往視之,如似多意。當見居非求志,清論所排,便是通辭訴屈邪。爾者真可謂真素寡
 矣!其數旦見客小防,自來盈門,亦不煩獨舉吉也。此輩乃云語勢所至,非其要也。弟無懷居今地,萬物初不以相非,然魯器齊虛,實宜書紳。今三署六府之人,誰表裏此內,儻疑弟豫有力,於素論何如哉。則吾長厄不死,終誤盛壯也。



 江不過彊吹拂吾,云是巖穴人。巖穴人情所高,吾得當此,則雞鶩變作鳳皇,何為乾飾廉隅,秩秩見於面目,所惜者大耳。諸舍闔門皆蒙時私,此既未易陳道,故常因含聲不言。至兄弟尤為叨竊,臨海頻煩二郡,
 謙亦越進清階,吾高枕家巷,遂至中書郎,此足以闔棺矣。



 又前年優旨,自弟所宣,雖夏后撫辜人,周宣及鰥寡,不足過也。語皆循檢校跡,不為虛飾也。作人不阿諛,無緣頭髮見白,稍學諂詐。且吾何以為,足不能行,自不得出戶;頭不耐風,故不可扶曳。家本貧餒,至於惡衣蔬食,設使盜跖居此,亦不能兩展其足,妄意珍藏也。正令選官設作此舉,於吾亦無劍戟之傷,所以勤勤畏人之多言也。管子晉賢,乃關人主之輕重,此何容易哉。州陵亦
 自言視明聽聰,而返區區飾吾,何辯致而下英俊。夫奇士必龍居深藏,與蛙蝦為伍,放勳其猶難之,林宗輩不足識也。似不肯眷眷奉箋記,彫琢獻文章,居家近市廛,親戚滿城府,吾猶自知袁陽源輩當平此不?飾詐之與直獨,兩不關吾心,又何所耿介。弟自宜以解塞群賢矣,兼悉怒此言自爾家任兄故能也。



 日日望弟來,屬病終不起,何意向與江書,粗布胸心,無人可寫,比面乃具與弟。書便覺成,本以當半日相見,吾既惡勞,不得多語,樞
 機幸非所長,相見亦不勝讀此書也。親屬欲見自可示,無急付手。



 時論者或云微之見舉,廬江何偃亦豫其議,慮為微所咎,與書自陳。微報之曰:卿昔稱吾於義興,吾常謂之見知,然復自怪鄙野,不參風流,未有一介熟悉於事,何用獨識之也。近日何見綽送卿書,雖知如戲,知卿固不能相哀。茍相哀之未知,何相期之可論。



 卿少陶玄風,淹雅脩暢,自是正始中人。吾真庸性人耳,自然志操不倍王、樂。



 小兒時尤粗笨無好,常從博士讀小小章
 句,竟無可得,口吃不能劇讀,遂絕意於尋求。至二十左右,方復就觀小說,往來者見床頭有數帙書,便言學問,試就檢,當何有哉。乃復持此擬議人邪。尚獨愧笑揚子之褒贍,猶恥辭賦為君子,若吾篆刻,菲亦甚矣。卿諸人亦當尤以此見議。或謂言深博,作一段意氣,鄙薄人世,初不敢然。是以每見世人文賦書論,無所是非,不解處即日借問,此其本心也。



 至於生平好服上藥,起年十二時病虛耳。所撰服食方中,粗言之矣。自此始信攝養有
 征,故門冬昌術,隨時參進。寒溫相補,欲以扶護危羸,見冀白首。家貧乏役,至於春秋令節,輒自將兩三門生,入草采之。吾實倦遊醫部,頗曉和藥,尤信《本草》,欲其必行,是以躬親,意在取精。世人便言希仙好異,矯慕不羈,不同家頗有罵之者。又性知畫繢,蓋亦鳴鵠識夜之機,盤紆糾紛,或記心目,故兼山水之愛,一往跡求,皆仿像也。不好詣人,能忘榮以避權右,宜自密應對舉止,因卷慚自保,不能勉其所短耳。由來有此數條,二三諸賢,因復
 架累,致之高塵,詠之清壑。瓦礫有資,不敢輕廁金銀也。



 而頃年嬰疾,沉淪無已,區區之情,心妻於生存,自恐難復,而先命猥加,魂氣褰籞,常人不得作常自處疾苦,正亦臥思已熟,謂有記自論。既仰天光,不夭庶類,兼望諸賢,共相哀體,而卿首唱誕言,布之翰墨,萬石之慎,或未然邪。好盡之累,豈其如此。綽大駭歎,便是闔朝見病者。吾本佇人,加疹意惛,一旦聞此,便惶怖矣。五六日來,復苦心痛,引喉狀如胸中悉腫,甚自憂。力作此答,無復條貫,
 貴布所懷,落漠不舉。卿既不可解,立欲便別,且當笑。



 微常住門屋一間,尋書玩古,如此者十餘年。太祖以其善筮,賜以名蓍。弟僧謙,亦有才譽,為太子舍人,遇疾,微躬自處治,而僧謙服藥失度,遂卒。微深自咎恨,發病不復自治,哀痛僧謙不能已,以書告靈曰:弟年十五,始居宿於外,不為察慧之譽,獨沉浮好書,聆琴聞操,輒有過目之能。討測文典,斟酌傳記,寒暑未交,便卓然可述。吾長病,或有小間,輒稱引前載,不異舊學。自爾日就月將,著名
 邦黨,方隆夙志,嗣美前賢,何圖一旦冥然長往,酷痛煩冤,心如焚裂。



 尋念平生,裁十年中耳。然非公事,無不相對,一字之書,必共詠讀;一句之文,無不研賞,濁酒忘愁,圖籍相慰,吾所以窮而不憂,實賴此耳。奈何罪酷,煢然獨坐。憶往年散發,極目流涕,吾不舍日夜,又恒慮吾羸病,豈圖奄忽,先歸冥冥。反覆萬慮,無復一期,音顏仿佛,觸事歷然,弟今何在,令吾悲窮。昔仕京師,分張六旬耳,其中三過,誤云今日何意不來,鐘念懸心,無物能譬。方
 欲共營林澤,以送餘年,念茲有何罪戾,見此夭酷,沒於吾手,觸事痛恨。吾素好醫術,不使弟子得全,又尋思不精,致有枉過,念此一條,特復痛酷。痛酷奈何!吾罪奈何!



 弟為志,奉親孝,事兄順,雖僮僕無所叱咄,可謂君子不失色於人,不失口於人。沖和淹通,內有皁白,舉動尺寸,吾每咨之。常云:「兄文骨氣,可推英麗以自許。又兄為人矯介欲過,宜每中和。」道此猶在耳,萬世不復一見,奈何!唯十紙手迹,封拆儼然,至於思戀不可懷。及聞吾病,肝
 心寸絕,謂當以幅巾薄葬之事累汝,奈何反相殯送!



 弟由來意,謂「婦人雖無子,不宜踐二庭。此風若行,便可家有孝婦」。仲長《昌言》,亦其大要。劉新婦以刑傷自誓,必留供養;殷太妃感柏舟之節,不奪其志。僕射篤順,范夫人知禮,求得左率第五兒,廬位有主。此亦何益冥然之痛,為是存者意耳。



 吾窮疾之人,平生意志,弟實知之。端坐向窗,有何慰適,正賴弟耳。過中未來,已自心妻望,今云何得立,自省惛毒,無復人理。比煩冤困憊,不能作刻石文,若
 靈響有識,不得吾文,豈不為恨。儻意慮不遂謝能思之如狂,不知所告訴,明書此數紙,無復詞理,略道阡陌,萬不寫一。阿謙!何圖至此!誰復視我,誰復憂我!



 他日寶惜三光,割嗜好以祈年,今也唯速化耳。吾豈復支,冥冥中竟復云何。弟懷隨、和之寶,未及光諸文章,欲收所一集,不知忽忽當辦此不?今已成服,吾臨靈,取常共飲杯,酌自釀酒,寧有仿像不?冤痛!冤痛!



 元嘉三十年,卒,時年三十九。僧謙卒後四旬而微終。遺令薄葬,不設轜旐鼓挽
 之屬,施五尺床,為靈二宿便毀。以嘗所彈琴置床上,何長史來,以琴與之。何長史者,偃也。無子。家人遵之。所著文集,傳於世。世祖即位,詔曰:「微棲志貞深,文行惇洽,生自華宗,身安隱素,足以賁茲丘園,惇是薄俗。不幸蚤世,朕甚悼之。可追贈秘書監。」



 史臣曰:燕太子吐一言,田先生吞舌而死;安邑令戒屠者,閔仲叔去而之沛。



 良由內懷耿介,峻節不可輕幹。袁淑笑謔之間,而王微弔詞連牘,斯蓋好名之士,欲以身
 為珪璋,皦皦然使塵玷之累,不能加也。



\end{pinyinscope}