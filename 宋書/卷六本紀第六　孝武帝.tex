\article{卷六本紀第六 孝武帝}

\begin{pinyinscope}

 世祖孝武
 皇帝諱駿,字休龍,小字道民,文帝第三子也。元嘉七年秋八月庚午生。十二年,立為武陵王,食邑二千戶。十六年,都督湘州諸軍事、征虜將軍、湘州刺史,領石頭戍事。
 十七年,遷使持節、都督南豫豫司雍並五州諸軍事、南豫州刺史,將軍如故,猶戍石頭。
 二十一
 年,加督秦州,進號撫軍將軍。明年,徙都督雍梁南北秦四州荊州之襄陽竟陵南陽順陽新野隨六郡諸軍事、寧蠻校尉、雍州刺史,持節、將軍如故。自晉氏江左以來,襄陽未有皇子重鎮,時太祖欲經略關、河,故有此授。尋給鼓吹一部。



 二十五年,改授都督南兗徐兗青冀幽六州豫州之梁郡諸軍事、安北將軍、徐州刺史,持節如故,北鎮
 彭城。尋領兗州刺史。始興王浚為南兗州,上解督南兗。二十七年,坐汝陽戰敗,降號鎮軍將軍。又以索虜南侵,降為北中郎將。二十八年,進督南兗州、南兗州刺史,當鎮山陽。尋遷都督江州荊州之江夏豫州之西陽晉熙新蔡四郡諸軍事、南中郎將、江州刺史,持節如故。時緣江蠻為寇,太祖遣太子步兵校尉沈慶之等伐之,使上總統眾軍。



 三十年正月,上出次西陽之五洲。會元兇弒逆,以上為征南將軍,加散騎常侍。



 上率眾入討,荊州刺
 史南譙王義宣、雍州刺史臧質並舉義兵。四月辛酉,上次溧洲。



 癸亥,冠軍將軍柳元景前鋒至新亭,修建營壘。甲子,賊劭親率眾攻元景,大敗退走。丙寅,上次江寧。丁卯,大將軍江夏王義恭來奔,奉表上尊號。戊辰,上至于新亭。己巳,即皇帝位,大赦天下,文武賜爵一等,從軍者二等。贓污清議,悉皆蕩除。高年、鰥寡、孤幼、六疾不能自存,人賜穀五斛。逋租宿債勿復收。長徒之身,優量降宥。崇改太祖號謚。以大將軍江夏王義恭為太尉、錄尚書
 六條事、南徐州刺史。庚午,以荊州刺史南譙王義宣為中書監、丞相、錄尚書六條事、揚州刺史;安東將軍隨王誕為衛將軍、開府儀同三司、荊州刺史;雍州刺史臧質為車騎將軍、開府儀同三司、江州刺史;征虜將軍沈慶之為領軍將軍;撫軍將軍、兗冀二州刺史蕭思話為尚書左僕射。壬申,以征虜將軍王僧達為尚書右僕射。改新亭為中興亭。



 五月甲戌,輔國將軍申坦克京城。乙亥,輔國將軍朱修之克東府。丙申,克定京邑。



 劭及始興王
 濬諸同逆,並伏誅。庚辰,詔曰:「天步艱難,國道用否,雖基構永固,而氣數時愆。朕以眇身,奄承皇業,奉尋歷命,鑒寐震懷。萬邦風政,人治之本,感念陵替,若疚在心。可分遣大使巡省方俗。」是日解嚴。辛巳,車駕幸東府城。



 甲申,尊所生路淑媛為皇太后。乙酉,立妃王氏為皇后。戊子,以左衛將軍柳元景為雍州刺史。壬辰,以太尉江夏王義恭為太傅,領大司馬。甲午,曲赦京邑二百里內,并蠲今年租稅。戊戌,以撫軍將軍南平王鑠為司空,建平王
 宏為尚書左僕射,東海王禕為撫軍將軍,新除尚書左僕射蕭思話遷職。六月壬寅,以驃騎參軍坦護之為冀州刺史。甲辰,以山陽太守申恬為青州刺史。丙午,車駕還宮,初置殿門及上皞屯兵。以江夏內史硃修之為平西將軍、雍州刺史,御史中丞王曇生為廣州刺史。



 戊申,以新除雍州刺史柳元景為護軍將軍。己酉,以司州刺史魯爽為豫州刺史。庚戌,以梁、南秦二州刺史劉秀之為益州刺史;太尉司馬龐秀之為梁、南秦二州刺史;
 衛軍司馬徐遺寶為兗州刺史;寧朔將軍王玄謨為徐州刺史;衛將軍隨王誕進號驃騎大將軍。尚書右僕射王僧達遷職,丹陽尹褚湛之為尚書右僕射。丙辰,以侍中南譙王世子恢為湘州刺史。丁亥,詔曰:「興王立訓,務弘治節,輔臣佐時,勤獻政要,仰惟聖規,每存茲道。猥以眇躬,屬承景業,闡揚遺澤,無廢厥心。夫量入為出,邦有恒典;而經給之宜,多違常度。兵役糜耗,府藏散減,外內眾供,未加損約,非所以聿遵先旨,敬奉遺圖。自今諸可
 薄己厚民、去煩從簡者,悉宜施行,以稱朕意。」庚申,詔有司論功班賞,各有差。辛酉,安西將軍、西秦河二州刺史吐谷渾拾寅進號鎮西大將軍、開府儀同三司。庚午,還分南徐,立南兗州。辛未,改封南譙王義宣為南郡王,隨王誕為竟陵王,義宣次子宜陽侯愷為宜陽縣王。閏月壬申,以領軍將軍沈慶之為鎮軍將軍、南兗州刺史。癸酉,以護軍將軍柳元景為領軍將軍。



 丙子,遣兼散騎常侍樂詢等十五人巡行風俗。甲申,蠲尋陽、西陽郡租布
 三年。甲午,丞相南郡王義宣改為荊、湘二州刺史,驃騎大將軍、荊州刺史竟陵王誕改為揚州刺史,南蠻校尉王僧達為護軍將軍。是月,置衛尉官。秋七月辛丑朔,日有蝕之。



 甲寅,詔曰:「世道未夷,惟憂在國。夫使群善畢舉,固非一才所議,況以寡德,屬衰薄之期,夙宵寅想,永懷待旦。王公卿士,凡有嘉謀善政,可以維風訓俗,咸達乃誠,無或依隱。」辛酉,詔曰:「百姓勞弊,徭賦尚繁,言念未乂,宜崇約損。



 凡用非軍國,宜悉停功。可省細作并尚方,雕
 文靡巧,金銀塗飾,事不關實,嚴為之禁。供御服膳,減除遊侈。水陸捕採,各順時日。官私交市,務令優衷。其江海田池公家規固者,詳所開弛。貴戚競利,悉皆禁絕。」戊戌,以右衛將軍宗愨為廣州刺史。己巳,司空南平王鑠薨。八月辛未,武皇帝舊役軍身,嘗在齋內,人身猶存者,普賜解戶。乙亥,尚書左僕射建平王宏加中書監、中軍將軍。丁亥,以沛郡太守垣閎為寧州刺史。撫軍司馬費沈為梁、南秦二州刺史。甲午,護軍將軍王僧達遷職。九月
 丁巳,以前尚書劉義綦為中護軍。壬戌,新亭戰亡者,復同京城。劭黨南海太守蕭簡據廣州反。丁卯,輔國將軍鄧琬討平之。冬十月癸未,車駕於閱武堂聽訟。十一月丙午,以左軍將軍魯秀為司州刺史。丙辰,停臺省眾官朔望問訊。丙寅,高麗國遣使獻方物。十二月甲戌,省都水臺,罷都水使者官,置水衡令官。癸未,以將置東宮,省太子率更令、步兵、翊軍校尉、旅賁中郎將、冗從僕射、左右積弩將軍官。中庶子、中舍人、庶子、舍人、洗馬,各減舊
 員之半。



 孝建元年春正月己亥朔,車駕親祠南郊,改元,大赦天下。壬寅,以丹陽尹蕭思話為安北將軍、徐州刺史。甲辰,護軍將軍劉義綦遷職,以尚書令何尚之為左光祿大夫、護軍將軍。戊申,詔曰:「首食尚農,經邦本務,貢士察行,寧朝當道。



 內難甫康,政訓未洽;衣食有仍耗之弊,選造無觀國之美。昔衛文勤民,高宗恭默,卒能收賢巖穴,大殷季年。朕每側席疚懷,無忘鑒寐。凡諸守蒞親民之官,
 可詳申舊條,勤盡地利。力田善蓄者,在所具以名聞。褒甄之科,精為其格。四方秀孝,非才勿舉,獻答允值,即就銓擢。若止無可採,猶賜除署;若有不堪酬奉,虛竊榮薦,遣還田里,加以禁錮。尚書百官之元本,庶績之樞機,丞郎列曹,局司有在。



 而頃事無巨細,悉歸令僕,非所以眾材成構,群能濟業者也。可更明體制,咸責厥成,糾核勤惰,嚴施賞罰。」壬戌,更鑄四銖錢。丙寅,立皇子子業為皇太子。賜天下為父後者爵一級。孝子、順孫、義夫、節婦粟
 帛各有差。是月,起正光殿。二月庚午,豫州刺史魯爽、車騎將軍江州刺史臧質、丞相荊州刺史南郡王義宣、兗州刺史徐遺寶舉兵反。乙亥,撫軍將軍東海王禕遷職。己卯,領軍將軍柳元景加撫軍將軍。壬午,曲赦豫州。辛卯,左衛將軍王玄謨為豫州刺史。癸巳,玄謨進據梁山。



 丙申,以安北司馬夏侯祖歡為兗州刺史。三月癸亥,內外戒嚴。辛丑,以安北將軍、徐州刺史蕭思話為安南將軍、江州刺史,撫軍將軍柳元景即本號為雍州刺史。癸
 卯,以太子左衛率龐秀之為徐州刺史。徐遺寶為夏侯祖歡所破,棄眾走。丙寅,以輔國長史明胤為冀州刺史。夏四月戊辰,以後將軍劉義綦為湘州刺史。甲申,以平西將軍、雍州刺史朱修之為安西將軍、荊州刺史。丙戌,鎮軍將軍、南兗州刺史沈慶之大破魯爽於歷陽之小峴,斬爽。癸巳,進慶之號鎮北大將軍。封第十六皇弟休倩為東平王。未拜,薨。五月甲寅,義宣等攻梁山,王玄謨大破之。己未,解嚴。癸亥,以吳興太守劉延孫為尚書
 右僕射。六月戊辰,臧質走至武昌,為人所斬,傳首京師。



 甲戌,撫軍將軍柳元景進號撫軍大將軍,鎮北大將軍沈慶之並開府儀同三司。丙子,以征虜將軍武昌王渾為雍州刺史。癸未,分揚州立東揚州;分荊、湘、江、豫州立郢州。罷南蠻校尉。戊子,省錄尚書事。庚寅,義宣於江陵賜死。秋七月丙申朔,日有蝕之。丙辰,大赦天下。文武賜爵一級;逋租宿債勿復收。辛酉,於雍州立建昌郡。以會稽太守義陽王昶為東揚州刺史。八月庚午,撫軍大將軍
 柳元景復為領軍將軍,本號如故。壬申,以游擊將軍垣護之為徐州刺史。壬辰,以安西司馬梁坦為梁、南秦二州刺史。九月丙申,以彊弩將軍尹懷順為寧州刺史。丁酉,左光祿大夫何尚之解護軍將軍。甲辰,加尚之特進。丙午,以安南將軍、江州刺史蕭思話為鎮西將軍、郢州刺史。冬十月戊寅,詔曰:「仲尼體天降德,維周興漢,經緯三極,冠冕百王。爰自前代,咸加褒述。典司失人,用闕宗祀。先朝遠存遺範,有詔繕立,世故妨道,事未克就。國難
 頻深,忠勇奮厲,實憑聖義,大教所敦。永惟兼懷,無忘待旦。可開建廟制,同諸侯之禮。詳擇爽塏,厚給祭秩。」丁亥,以秘書監東海王禕為撫軍將軍、江州刺史。於郢州立安陸郡。十一月癸卯,復立都水臺,置都水使者官。是歲,始課南徐州僑民租。



 二年正月壬寅,以冠軍將軍湘東王諱為中護軍。二月己丑,婆皇國遣使獻方物。



 丙寅,以鎮北大將軍、南兗州刺史沈慶之為左光祿大夫、開府儀同三司。辛巳,以尚
 書右僕射劉延孫為南兗州刺史。三月辛亥,以吳興太守劉遵考為湘州刺史。壬子,以行征西將軍楊文智為征西將軍、北秦州刺史。夏四月壬申,河南國遣使獻方物。



 壬午,以豫章太守檀和之為豫州刺史。五月戊戌,以湘州刺史劉遵考為尚書右僕射,前軍司馬垣閎為交州刺史。庚子,以輔國將軍申坦為徐、兗二州刺史。癸卯,以右衛將軍顧覬之為湘州刺史。丁未,以金紫光祿大夫王偃為右光祿大夫。六月甲子,以國哀除釋,大赦天
 下。庚辰,以曲江縣侯王玄謨為豫州刺史。秋七月癸巳,立第十三皇弟休祐為山陽王,第十四皇弟休茂為海陵王,第十五皇弟休業為鄱陽王。戊戌,鎮西將軍蕭思話卒。己酉,以益州刺史劉秀之為郢州刺史。槃槃國遣使獻方物。



 甲寅,以義興太守到元度為益州刺史。八月庚申,雍州刺史武昌王渾有罪,廢為庶人,自殺。辛酉,以南兗州刺史劉延孫為鎮軍將軍、雍州刺史。斤陀利國遣使獻方物。三吳民饑,癸酉,詔所在賑貸。丙子,詔曰:「諸
 苑禁制綿遠,有妨肄業。可詳所開弛,假與貧民。」壬午,以新除豫州刺史王玄謨為青、冀二州刺史,青州刺史申恬為豫州刺史。甲申,以右衛將軍檀和之為南兗州刺史。九月丁亥,車駕於宣武場閱武。庚戌,詔曰:「國道再屯,艱虞畢集。朕雖寡德,終膺鴻慶。惟新之祉,實深百王;而惠宥之令,未殊常渥。永言勤慮,寤寐載懷。在朕受命之前,凡以罪徙放,悉聽還本。犯釁之門,尚有存者,子弟可隨才署吏。」冬十月壬午,太傅江夏王義恭領揚州刺史,
 驃騎大將軍、揚州刺史竟陵王誕為司空、南徐州刺史,中書監、尚書左僕射、中軍將軍建平王宏為尚書令,將軍如故。十一月戊子,中護軍湘東王諱遷職,鎮軍將軍劉延孫為護軍將軍。青、冀二州刺史王玄謨為雍州刺史。甲午,以大司馬垣護之為青、冀二州刺史。辛亥,高麗國遣使獻方物。十二月癸亥,以前交州刺史蕭景憲為交州刺史。



 三年春正月庚寅,立第十八皇弟休範為順陽王,第十
 九皇弟休若為巴陵王。戊戌,立第二皇子子尚為西陽王。辛丑,車駕親祠南郊。壬子,立皇太子妃何氏。甲寅,大赦天下。二月癸亥,右光祿大夫王偃卒。甲子,以廣州刺史宗愨為平西將軍、豫州刺史。丁卯,以新除御史中丞王翼為廣州刺史。丁丑,始制朔望臨西堂接群下,受奏事。壬午,內外官有田在近道,聽遣所給吏僮附業。三月癸丑,以西陽王子尚為南兗州刺史。閏月戊午,尚書右僕射劉遵考遷職。癸酉,鄱陽王休業薨。庚辰,停元嘉三
 十年以前兵工考剔。夏五月辛酉,制荊、徐、兗、豫、雍、青、冀七州統內,家有馬一匹者,蠲復一丁。壬戌,以右衛將軍劉瑀為益州刺史。六月,上於華林園聽訟。秋七月,太傅江夏王義恭解揚州。丙子,以南兗州刺史西陽王子尚為揚州刺史,秘書監建安王休仁為南兗州刺史。八月戊戌,以北軍中郎諮議參軍費淹為交州刺史。丁未,以尚書吏部王琨為廣州刺史。九月壬戌,以丹陽尹劉遵考為尚書左僕射。冬十月癸未,以尋陽太守張悅為益
 州刺史。丙午,太傅江夏王義恭進位太宰,領司徒。丁未,領軍將軍柳元景加驃騎將軍,尚書令建平王宏加中書監、衛將軍,撫軍將軍、江州刺史東海王禕進號平南將軍。十一月癸丑,淮南太守袁景有罪棄市。十二月丙午,以侍中孔靈符為郢州刺史。



 大明元年春正月辛亥朔,改元,大赦天下。賜高年、孤疾粟帛各有差。庚午,護軍將軍劉延孫遷職,右衛將軍湘東王諱為中護軍。京邑雨水,辛未,遣使檢行,賜以樵米。
 二月己亥,復親民職公田。索虜寇兗州。三月壬戌,制大臣加班劍者,不得入宮城門。梁州獠求內屬,立懷漢郡。夏四月,京邑疾疫。丙申,遣使按行,賜給醫藥。死而無收斂者,官為斂埋。庚子,省湘州宋建郡并臨賀。五月,吳興、義興大水,民饑。乙卯,遣使開倉賑恤。癸酉,於華林園聽訟。乙亥,以左衛將軍沈曇慶為徐州刺史,輔國將軍梁瑾蔥為河州刺史、宕昌王。六月己卯,以前太子步兵校尉劉祗子歆繼南豐王朗。辛巳,以長水校尉山陽王休
 祐為東揚州刺史。丁亥,休祐改為湘州刺史。以丹陽尹顏竣為東揚州刺史。秋七月辛未,土斷雍州諸僑郡縣。



 八月戊戌,於兗州立陽平郡。壬寅,於華林園聽訟。甲辰,司空、南徐州刺史竟陵王誕改為南兗州刺史,太子詹事劉延孫為鎮軍將軍、南徐州刺史。冬十月丙申,詔曰:「旒纊之道,有孚於結繩,日昃之勤,已切於姬后。況世弊教淺,歲月澆季。



 朕雖戮力宇內,未明求衣,而識狹前王,務廣昔代,永言菲德,其愧良深。朝咨野怨,自達者寡,
 惠民利公,所昧實眾。自今百辟庶尹,下民賤隸,有懷誠抱志,擁鬱衡閭,失理負謗,未聞朝聽者,皆聽躬自申奏,小大以聞。朕因聽政之日,親對覽焉。」甲辰,以百濟王餘慶為鎮東大將軍。十二月丁亥,順陽王休範改封桂陽王。



 戊戌,於華林園聽訟。



 二年春正月辛亥,車駕祀南郊。壬子,詔曰:「去歲東土多經水災,春務已及,宜加優課。糧種所須,以時貸給。」丙辰,復郡縣田秩,并九親祿俸。壬戌,詔曰:「先帝靈命初興,龍
 飛西楚,歲紀浸遠,感往纏心。奉迎文武,情深常隸,思弘殊澤,以申永懷。吏身可賜爵一級,軍戶免為平民。」二月丙子,詔曰:「政道未著,俗弊尚深,豪侈兼并,貧弱困窘,存闕衣裳,沒無斂槥,朕甚傷之。其明敕守宰,勤加存恤。賻贈之科,速為條品。」乙酉,以金紫光祿大夫褚湛之為尚書左僕射。



 丙戌,中書監、尚書令、衛將軍建平王宏以本號開府儀同三司,中書監如故。丁酉,驃騎將軍柳元景以本號開府儀同三司。甲辰,散騎常侍義陽王昶為中
 軍將軍。三月丁未,中書監、尚書令、衛將軍建平王宏薨。乙卯,以田農要月,太官停殺牛。丁卯,上於華林園聽訟。癸酉,以寧朔將軍劉季之為司州刺史。夏四月甲申,立皇子子綏為安陸王。甲午,以海陵王休茂為雍州刺史。辛丑,地震。五月戊申,復西陽郡。六月戊寅,增置吏部尚書一人,省五兵尚書。丁亥,左光祿大夫何尚之加開府儀同三司。戊子,以金紫光祿大夫羊玄保為右光祿大夫。丙申,詔曰:「往因師旅,多有逋亡。或連山染逆,懼致軍
 憲;或辭役憚勞,茍免刑罰。雖約法從簡,務思弘宥,恩令驟下,而逃伏猶多,豈習愚為性,忸惡難反;將在所長吏,宣導乖方。可普加寬申,咸與更始。」秋七月甲辰,彭城民高闍等謀反伏誅。癸亥,以右衛將軍顏師伯為青、冀二州刺史。八月乙酉,河南王遣使獻方物。丙戌,中書令王僧達有罪,下獄死。己丑,以彊弩將軍杜叔文為寧州刺史,交州刺史費淹為廣州刺史,南海太守垣閬為交州刺史。甲午,以寧朔將軍沈僧榮為兗州刺史。九月癸卯,
 於華林園聽訟。壬戌,以寧朔將軍劉道隆為徐州刺史。襄陽大水,遣使巡行賑贍。庚午,置武衛將軍、武騎常侍官。冬十月甲午,以中軍將軍義陽王昶為江州刺史。乙未,高麗國遣使獻方物。十一月壬子,揚州刺史西陽王子尚加撫軍將軍。十二月己亥,諸王及妃主庶姓位從公者,喪事聽設兇門,餘悉斷。閏月庚子,詔曰:「夫山處巖居,不以魚鱉為禮。頃歲多虞,軍調繁切,違方設賦,本濟一時,而主者玩習,遂為常典。杶桿瑤琨,任土作貢,積羽
 群輕,終致深弊。永言弘革,無替朕心。凡寰衛貢職,山淵採捕,皆當詳辨產殖,考順歲時,勿使牽課虛懸,睽忤氣序。庶簡約之風,有孚於品性;惠敏之訓,無漏於幽仄。」庚申,上於華林園聽訟。壬戌,林邑國遣使獻方物。是冬,索虜寇青州,刺史顏師伯頻大破之。



 三年春正月丁亥,割豫州梁郡屬徐州。己丑,以驃騎將軍、領軍將軍柳元景為尚書令,尚書右僕射劉遵考為領軍將軍。丙申,婆皇國遣使獻方物。二月乙卯,以揚州
 所統六郡為王畿。以東揚州隸揚州。時欲立司隸校尉,以元兇已立乃止。撫軍將軍、揚州刺史西陽王子尚徙為揚州刺史。甲子,復置廷尉監官。荊州饑,三月甲申,原田租布各有差。庚寅,以義興太守垣閬為兗州刺史。壬辰,中護軍湘東王諱遷職,以中書令東海王禕為衛將軍、護軍將軍。癸巳,太宰江夏王義恭加中書監。



 夏四月癸卯,上於華林園聽訟。丙午,以建寧太守苻仲子為寧州刺史。乙卯,司空、南兗州刺史竟陵王誕有罪,貶爵;誕
 不受命,據廣陵城反,殺兗州刺史垣閬。以始興公沈慶之為車騎大將軍、開府儀同三司、南兗州刺史討誕。甲子,上親御六師,車駕出頓宣武堂。司州刺史劉季之反叛,徐州刺史劉道隆討斬之。秋七月己巳,克廣陵城,斬誕。悉誅城內男丁,以女口為軍賞;是日解嚴。辛未,大赦天下。尚方長徒、奚官奴婢老疾者原放。孝子、順孫、義夫、節婦,賜粟帛各有差。王畿下貧之家,與近行頓所由,並蠲租一年。丙子,以丹陽尹劉秀之為尚書右僕射。丙
 戌,分淮南北復置二豫州。以新除車騎大將軍、開府儀同三司、南兗州刺史沈慶之為司空,刺史如故。戊子,以衛將軍、護軍將軍東海王禕為南豫州刺史,衛將軍如故。



 江州刺史義陽王昶為護軍將軍,冠軍將軍桂陽王休範為江州刺史。癸巳,以前左衛將軍王玄謨為郢州刺史。八月丙申,詔曰:「近北討文武,於軍亡沒,或殞身矢石,或癘疾死亡,並盡勤王事,而斂槥卑薄。可普更賻給,務令豐厚。」己酉,以車騎長史庾深之為豫州刺史。甲子,
 詔曰:「昔姬道方凝,刑法斯厝;漢德初明,犴圄用簡。良由上一其道,下淳其性。今民澆俗薄,誠淺偽深,重以寡德,弗能心化。



 故知方者鮮,趣辟實繁,向因巡覽,見二尚方徒隸,嬰金屨校,既有矜復。加國慶民和,獨隔凱澤,益以慚焉。可詳所原宥。」九月己巳,詔曰:「夫五闢三刺,自古所難;巧法深文,在季彌甚。故沿情察訟,魯師致捷;市獄勿擾,漢史飛聲。廷尉遠邇疑讞,平決攸歸,而一蹈幽圄,動逾時歲。民嬰其困,吏容其私。自今囚至辭具,並即以聞,
 朕當悉詳斷,庶無留獄。若繁文滯劾,證逮遐廣,必須親察,以盡情狀。自後依舊聽訟。」壬辰,於玄武湖北立上林苑。冬十月丁酉,詔曰:「古者薦鞠青壇,聿祈多慶,分繭玄郊,以供純服。來歲,可使六宮妃嬪修親桑之禮。」



 庚子,鎮軍將軍、南徐州刺史劉延孫進號車騎將軍。戊申,河西國遣使獻方物。庚戌,以河西王大沮渠安周為征虜將軍、涼州刺史。十一月己巳,高麗國遣使獻方物;肅慎國重譯獻楛矢、石砮;西域獻舞馬。十二月戊午,上於華林
 園聽訟。辛酉,置謁者僕射官。



 四年春正月辛未,四駕祠南郊。甲戌,宕昌王奉表獻方物。乙亥,車駕躬耕藉田,大赦天下。尚方徒系及逋租宿債,大明元年以前,一皆原除。力田之民,隨才敘用。孝悌義順,賜爵一級。孤老貧疾,人穀十斛。藉田職司,優沾普賚。百姓乏糧種,隨宜貸給。吏宣勸有章者,詳加褒進。壬午,以北中郎司馬柳叔仁為梁、南秦二州刺史。左將軍、荊州刺史硃修之進號鎮軍將軍。庚寅,立第三皇子
 勛為晉安王,第六皇子房為尋陽王,第七皇子子頊為歷陽王,第八皇子子鸞為襄陽王。二月庚子,侍中建安王休仁為湘州刺史。己未,以員外散騎侍郎費景緒為寧州刺史。三月甲子,以冠軍將軍巴陵王休若為徐州刺史。丁卯,以安陸王子綏為郢州刺史。癸酉,以徐州刺史劉道隆為青、冀二州刺史。索虜寇北陰平孔堤,太守楊歸子擊破之。



 甲申,皇后親桑于西郊。夏四月癸卯,以南琅邪隸王畿。丙午,詔曰:「昔紩衣御宇,貶甘示節;土
 簋臨天,飭儉昭度。朕綈帛之念,無忘於懷。雖深詔有司,省游務實,而歲用兼積,年量虛廣。豈以捐豐從損,允稱約心。四時供限,可詳減太半。



 庶裘絺順典,有偃民華;纂組傷工,無競廛市。」辛酉,詔曰:「都邑節氣未調,癘疫猶眾,言念民瘼,情有矜傷。可遣使存問,並給醫藥;其死亡者,隨宜恤贍。」



 五月庚辰,於華林園聽訟。乙酉,以徐州之梁郡還屬豫州。丙戌,尚書左僕射褚湛之卒。以撫軍長史劉思考為益州刺史。庚寅,以南下邳併南彭城郡。秋七
 月甲戌,左光祿大夫、開府儀同三司何尚之薨。八月壬寅,宕昌王遣使獻方物。己酉,以晉安王子勛為南兗州刺史。雍州大水,甲寅,遣軍部賑給。九月辛未,以冠軍將軍垣護之為豫州刺史。甲申,上於華林園聽訟。丁亥,改封襄陽王子鸞為新安王。冬十月庚寅,遣新除司空沈慶之討沿江蠻。壬辰,制郡縣減祿,並先充公限。十一月戊辰,改細作署令為左右御府令。丙戌,復置大司農官。十二月乙未,上於華林園聽訟。辛巳,車駕幸廷尉寺,凡
 囚系咸悉原遣。索虜遣使請和。丁未,車駕幸建康縣,原放獄囚。倭國遣使獻方物。



 五年春正月丁卯,以宕昌王梁唐子為河州刺史。二月癸己,車駕閱武。詔曰:「昔人稱人道何先,於兵為首,雖淹紀勿用,忘之必危。朕以聽覽餘閑,因時講事,坐作有儀,進退無爽。軍幢以下,普量班錫。頃化弗能孚,而民未知禁,逭役違調,起觸刑網。凡諸逃亡,在今昧爽以前,悉皆原赦;已滯囹圄者,釋還本役;其逋負在大明三年以前,
 一賜原停。自此以還,鰥貧疾老,詳所申減,伐蠻之家,蠲租稅之半。近籍改新制,在所承用,殊謬實多,可普更符下,聽以今為始。若先已犯制,亦同盪然。」甲寅,加右光祿大夫羊玄保特進。夏四月癸巳,改封西陽王子尚為豫章王。丙申,加尚書令柳元景左光祿大夫、開府儀同三司。戊戌,詔曰:「南徐、兗二州去歲水潦傷年,民多困窶。逋租未入者,可申至秋登。」丙午,雍州刺史海陵王休茂殺司馬庾深之,舉兵反,義成太守薛繼考討斬之。甲寅,以
 第九皇子子仁為雍州刺史。五月癸亥,制帝室期親,朝官非祿官者,月給錢十萬。丙辰,車駕幸閱武堂聽訟。六月丙午,以護軍將軍義陽王昶為中軍將軍。壬子,分廣陵置沛郡,省東平郡並廣陵。秋七月丙辰,詔曰:「雨水猥降,街衢泛溢,可遣使巡行。窮弊之家,賜以薪粟。」丁卯,高麗國遣使獻方物。庚午,曲赦雍州。八月戊子,立第九皇子子仁為永嘉王,第十一皇子子真為始安王。以北中郎參軍費伯弘為寧州刺史。



 己丑,詔曰:「自靈命初基,聖
 圖重遠。參正樂職,感神明之應;崇殖禮囿,奮至德之光。聲實同和,文以均節,化調其俗,物性其情。故臨經式奠,煥乎炳發,道喪世屯,學落年永。獄訟微衰息之術,百姓忘退素之方。今息警夷嶂,恬波河渚,棧山航海,向風慕義,化民成俗,茲時篤矣。來歲可修葺庠序,旌延國胄。」庚寅,制方鎮所假白板郡縣,年限依臺除,食祿三分之一,不給送故。衛將軍東海王禕以本號開府儀同三司。九月甲寅朔,日有食之。丁卯,行幸瑯邪郡,囚系悉原遣。甲
 戌,移南豫州治淮南于湖縣。丁丑,以冠軍將軍尋陽王子房為南豫州刺史。閏月戊子,皇太子妃何氏薨。丙申,初立馳道,自閶闔門至于朱雀門,又自承明門至于玄武湖。壬寅,改封歷陽王子頊為臨海王。冬十月甲寅,以車騎將軍、南徐州刺史劉延孫為尚書左僕射、領護軍將軍,尚書右僕射劉秀之為安北將軍、雍州刺史。以冠軍將軍臨海王子頊為廣州刺史。乙卯,以東中郎將新安王子鸞為南徐州刺史。十一月壬辰,詔曰:「王畿內奉
 京師,外表眾夏,民殷務廣,宜思簡惠。可遣尚書就加詳檢,并與守宰平治庶獄。其有疑滯,具以狀聞。」丁酉,增置少府丞一人。十二月壬申,以領軍將軍劉遵考為尚書右僕射。甲戌,制天下民戶歲輸布四匹。庚辰,以太常王玄謨為平北將軍、徐州刺史。



 六年春正月己丑,湘州刺史建安王休仁加平南將軍。辛卯,車駕親祠南郊。是日,又宗祀明堂,大赦天下。孝子、順孫、義夫、悌弟,賜爵一級;慈姑、節婦及孤老、六疾,賜帛
 五匹,穀十斛。下四方旌賞茂異,其有懷真抱素,志行清白,恬退自守,不交當世,或識通古今,才經軍國,奉公廉直,高譽在民,具以名奏。乙未,置五官中郎將、左右中郎將官。二月乙卯,復百官祿。三月庚寅,立第十三皇子子元為邵陵王。壬寅,以倭國王世子興為安東將軍。乙巳,改豫州南梁郡為淮南郡,舊淮南郡并宣城。丁未,輔國將軍、征虜長史、廣陵太守沈懷文有罪,下獄死。



 四月庚申,原除南兗州大明三年以前逋租。新作大航門。五月
 丙戌,置凌室,修藏冰之禮。壬寅,太宰江夏王義恭解領司徒。六月辛酉,尚書左僕射、護軍將軍劉延孫卒。秋七月庚辰,以荊州刺史硃修之為領軍將軍,廣州刺史臨海王子頊為荊州刺史。甲申,地震。戊子,以輔國將軍王翼之為廣州刺史。辛卯,以西陽太守檀翼之為交州刺史。乙未,立第十九皇子子云為晉陵王。八月癸亥,原除雍州大明四年以前逋租。乙亥,置清臺令。九月戊寅,制沙門致敬人主。戊子,以前金紫光祿大夫宗愨為中護
 軍。乙未,尚書右僕射劉遵考為尚書左僕射,丹陽尹王僧朗為尚書右僕射。冬十月丁巳,以山陽王休祐子士弘繼鄱陽哀王休業。上林苑內民庶丘墓慾還合葬者,勿禁。十一月己卯,陳留王曹虔秀薨。辛巳,以尚書令柳元景為司空,尚書令如故。


七年春正月癸未,詔曰:「春搜之禮,著自周令;講事之語,書于魯史。所以昭宣德度,示民軌則。今歲稔氣榮,中外寧晏。當因農隙,葺是舊章。可克日於玄武湖大閱水師,
 并巡江右,講武校獵。」丁亥,以尚書右僕射王僧朗為太常,衛將軍顏師伯為尚書右僕射。己丑,以尚書令柳元景為驃騎大將軍、開府儀同三司。庚寅,以南兗州刺史晉安王子勛為江州刺史。癸巳,割吳郡屬南徐州。二月甲寅,車駕巡南豫、南兗二州。丙辰,詔曰:「江漢楚望,咸秩周禋,禮九疑於盛唐,祀蓬萊於渤海,皆前載流訓,列聖遺式。霍山是曰南嶽,實維國鎮,韞靈呈瑞,肇光宋道。朕駐驆于野,有事岐陽,瞻睇風雲,徘徊以想。可遣使奠祭。」
 丁巳,車駕校獵於歷陽之烏江。己未,車駕登烏江縣六合山。庚申,割歷陽秦郡置臨江郡。壬戌,詔曰:「朕受天慶命,十一年於茲矣。憑七廟之靈,獲上帝之力,禮橫四海,威震八荒。方巡三湘而奠衡岳,次九河而檢云、岱。今恢覽功成,省風畿表,觀民六合,搜校長洲。騰沙飛礫,平嶽盪海,
 \gezhu{
  卉鼓}
 晉合序,鐃鉦協節,獻鬯如禮,饁獸傾郊,敬舉王公之觴,廣納士民之壽。八風循通,卿云叢聚,盡天罄瑞,率宇竭歡。思散太極之泉,以福無方之外。可大赦天下,
 行幸所經,無出今歲租布。其逋租餘債,勿復收。賜民爵一級,女子百戶牛酒。刺守邑宰及民夫從搜者,普加洽賚。」又詔曰:「朕弱年操製,出牧司雍,承政宣風,薦歷年紀。國步中阻,治戎江甸,難夷情義,實系于懷。今或練搜訓旅,涉茲境閭,故邑耆舊,在目罕存。年世未遠,殲亡太半,撫跡惟事,傾慨兼著。太宗燕故,晉陽洽恩;世祖流仁,濟畿暢澤。永言往猷,思廣前賚。可蠲歷陽郡租輸三年。遣使巡慰,問民疾苦,鰥寡、孤老、六疾不能自存者,厚賜粟帛。
 高年加以羊酒。凡一介之善,隨才銓貫;前國名臣及府州佐吏,量所沾錫。人身已往,施及子孫。」壬申,車駕還宮。夏四月甲寅,以領軍將軍朱修之為特進。丙辰,以尚書湘東王諱為領軍將軍。甲子,詔曰:「自非臨軍戰陳,一不得專殺。其罪甚重辟者,皆如舊先上須報,有司嚴加聽察。犯者以殺人罪論。」五月乙亥,撫軍將軍、揚州刺史豫章王子尚進號車騎將軍,輔國將軍始安王子真為廣州刺史。丙子,詔曰:「自今刺史守宰,動民興軍,皆須手詔
 施行。唯邊隅外警,及姦釁內發,變起倉卒者,不從此例。」六月甲辰,以北中郎司馬柳元怙為梁、南秦二州刺史。戊申,芮芮國、高麗國遣使獻方物。戊辰,以秦郡太守劉德願為豫州刺史。七月乙亥,征東大將軍高麗王高璉進號車騎大將軍、開府儀同三司。秋七月丙申,詔曰:「前詔江海田池,與民共利。歷歲未久,浸以弛替。名山大川,往往占固。有司嚴加檢糾,申明舊制。」八月丁巳,詔曰:「昔匹婦含怨,山焦北鄙;孀妻哀慟,臺傾東國。良以誠之所
 動,在微必著;感之所震,雖厚必崩。



 朕臨察九野,志深待旦,弗能使爛然成章,各如其節。遂令炎精損河,陽偏不施,歲云不稔,咎實朕由。大官供膳,宜從貶撤。近道刑獄,當親料省。其王畿內及神州所統,可遣尚書與所在共詳;畿外諸州,委之刺史。并詳省律令,思存利民。其考謫貿襲,在大明七年以前,一切勿治;尤弊之家,開倉賑給。」乙丑,立第十六皇子子孟為淮南王,第十八皇子子產為臨賀王。車駕幸建康秣陵縣,訊獄囚。九月巳卯,詔曰:「
 近炎精亢序,苗稼多傷。今二麥未晚,甘澤頻降,可下東境郡,勤課墾殖。尤弊之家,量貸麥種。」戊子,詔曰:「昔周王驥迹,實窮四溟;漢帝鸞軫,夙遍五嶽。皆所以上對幽靈,下理民土。自天昌替馭,臨宮創圖,禮代夭鬱,世貿興毀。皇家造宋,日月重光,璇璣得序,五星順命,而戎車歲動,陳詩義闕。



 朕聿含五光,奄一天下,思盡寶戒之規,以塞謀危之路。當沿時省方,觀察風俗。



 外詳考舊典,以副側席之懷。」庚寅,南徐州刺史新安王子鸞兼司徒。乙未,
 車駕幸廷尉,訊獄囚。丙申,立第十七皇子子嗣為東平王。冬十月壬寅,太子冠,賜王公以下帛各有差。戊申,車駕巡南豫州。詔曰:「朕巡幸所經,先見百年者,及孤寡老疾,並賜粟帛。獄繫刑罪,並親聽訟。其士庶或怨鬱危滯,受抑吏司,或隱約潔立,負擯州里,皆聽進朕前,面自陳訴。若忠信孝義,力田殖穀,一介之能,一藝之美,悉加旌賞。雖秋澤頻降,而夏旱嬰弊。可即開行倉,並加賑賜。」癸丑,行幸江寧縣,訊獄囚。車騎將軍、揚州刺史豫章王子
 尚加開府儀同三司。癸亥,衛將軍、開府儀同三司東海王禕為司空,中軍將軍義陽王昶加開府儀同三司。丙寅,詔曰:「賞慶刑威,奄國彞軌,黜幽升明,闢宇恆憲。故採言聆風,式觀侈質,貶爵加地,於是乎在。今類帝宜社,親巡江甸,因覲嶽守,躬求民瘼。思弘明試之典,以申考績之義。行幸所經,蒞民之職,功宣於德,即加甄賞;若廢務亂民,隨愆議罰,主者詳察以聞。」己巳,車駕校獵於姑孰。十一月丙子,曲赦南豫州殊死以下。



 巡幸所經,詳減今
 歲田租。乙酉,詔遣祭晉大司馬桓溫、征西將軍毛璩墓。上於行所訊溧陽、永世、丹陽縣囚。癸巳,車駕習水軍於梁山,有白爵二集華蓋,有司奏改大明七年為神爵元年,詔不許。乙未,原放行獄徒繫。東諸郡大獄,壬寅,遣使開倉貸恤,聽受雜物當租。十二月丙午,行幸歷陽。甲寅,大赦天下。南豫州別署敕系長徒,一切原散。其兵期考襲謫戍,悉停。歷陽郡女子百戶牛酒;高年孤疾,賜帛十匹,蠲郡租十年。己未,太宰江夏王義恭加尚書令。於博
 望梁山立雙闕。癸亥,車駕至自歷陽。



 八年春正月甲戌,詔曰:「東境去歲不稔,宜廣商貨。遠近販鬻米粟者,可停道中雜稅。其以仗自防,悉勿禁。」癸未,安北將軍、雍州刺史劉秀之卒。戊子,以平南將軍、湘州刺史建安王休仁為安南將軍、江州刺史,晉安王子勛為鎮軍將軍、雍州刺史,徐州刺史新安王子鸞為撫軍將軍,領司徒、刺史如故,輔國將軍江夏王世子伯禽為湘州刺史。二月辛丑,特進朱修之卒。壬寅,詔曰:「去歲東境
 偏旱,田畝失收。使命來者,多至乏絕。或下窮流冗,頓伏街巷,朕甚閔之。可出倉米付建康、秣陵二縣,隨宜贍恤。若濟拯不時,以至捐棄者,嚴加糾劾。」乙巳,以鎮軍將軍湘東王諱為鎮北將軍、徐州刺史。平北將軍、徐州刺史王玄謨為領軍將軍。



 夏閏五月辛丑,以前御史中丞蕭惠開為青、冀二州刺史。壬寅,太宰江夏王義恭領太尉。特進、右光祿大夫羊玄保卒。庚申,帝崩於玉燭殿,時年三十五。秋七月丙午,葬丹陽秣陵縣巖山景寧陵。



 史臣曰:役己以利天下,堯、舜之心也;利己以及萬物,中主之志也;盡民命以自養,桀、紂之行也。觀大明之世,其將盡民命乎!雖有周公之才之美,猶終之以亂,
 何
 益哉!



\end{pinyinscope}