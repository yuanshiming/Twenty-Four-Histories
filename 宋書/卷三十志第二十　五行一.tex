\article{卷三十志第二十 五行一}

\begin{pinyinscope}

 昔
 八卦兆而天人之理著,九疇序而帝王之應明。雖可以知從德獲自天之祐,違道陷神聽之罪,然未詳舉徵效,備考幽明,雖時列鼎雉庭穀之異,然而未究者眾矣。



 至於鑑悟後王,多有所闕。故仲尼作《春秋》,具書祥眚,以驗行事。是則九疇陳其義於前,《春秋》列其效於後也。逮至伏生創紀《大傳》,五行之體始詳;劉向廣演《洪範》,休咎之文益備。故班固斟酌《經》、《傳》,詳紀條流,誠以一王之典,不可獨闕故也。夫天道雖無聲無臭,然而應若影響,天人之驗,理不可誣。



 司馬彪纂集光武以來,以究漢事;王沈《魏書》志篇闕,凡厥災異,但編帝紀而已。



 自黃初以降,二百餘年,覽其災妖,以考之事,常若重規沓矩,不謬前
 說。又高堂隆、郭景純等,據經立辭,終皆顯應。闕而不序,史體將虧。今自司馬彪以後,皆撰次論序,斯亦班固遠采《春秋》,舉遠明近之例也。又按言之不從,有介蟲之孽,劉歆以為毛蟲;視之不明,有蠃蟲之孽,劉歆以為羽蟲。按《月令》,夏蟲羽,秋蟲毛,宜如歆說,是以舊史從之。五行精微,非末學所究。凡已經前議者,並即其言以釋之;未有舊說者,推準事理,以俟來哲。



 《五行傳》曰:「田獵不宿,飲食不享,出入不節,奪民農時,及
 有姦謀,則木不曲直,謂木失其性而為災也。」又曰:「貌之不恭,是謂不肅。厥咎狂,厥罰恒雨,厥極惡。時則有服妖,時則有龜孽,時則有雞禍,時則有下體生上之痾,時則有青眚、青祥。惟金沴木。」班固曰:「蓋工匠為輪矢者多傷敗,及木為變怪。」



 皆為不曲直也。



 木不曲直:魏文帝黃初六年正月,雨,木冰。按劉歆說,木不曲直也。劉向曰:「冰者陰之盛,木者少陽,貴臣象也。此人將有害,
 則陰氣脅木,木先寒,故得雨而冰也。」



 是年六月,利成郡兵蔡方等殺太守徐質,據郡反,多所脅略,並聚亡命。遣二校尉與青州刺史共討平之。太守,古之諸侯,貴臣有害之應也。一說以木冰為甲兵之象。



 是歲,既討蔡方;又八月,天子自將以舟師征吳,戎卒十餘萬,連旍數百里,臨江觀兵。



 晉元帝太興三年二月辛未,雨,木冰。後二年,周顗、戴淵、刁協、劉隗皆遇害,與《春秋》同事,是其應也。一曰,是後王
 敦攻京師,又其象也。



 晉穆帝永和八年正月乙巳,雨,木冰。是年,殷浩北伐;明年,軍敗;十年,廢黜。又曰,荀羨、殷浩北伐,桓溫入關之象也。



 晉孝武帝太元十四年十二月乙巳,雨,木冰。明年二月,王恭為北蕃;八月,庾楷為西蕃;九月,王國寶為中書令,尋加領軍將軍;十七年,殷仲堪為荊州。雖邪正異規,而終同摧滅,是其應也。一曰,苻堅雖敗,關、河未一,丁零鮮
 卑,侵略司、兗,竇揚勝扇逼梁、雍,兵役不已,又其象也。



 吳孫亮建興二年,諸葛恪征淮南,行後,所坐聽事棟中折。恪妄興征役,奪民農時,作為邪謀,傷國財力,故木失其性,致毀折也。及旋師而誅滅,於《周易》又為棟橈之凶也。



 晉武帝太康五年五月,宣帝廟地陷梁折。八年正月,太廟殿又陷,改作廟,築基及泉。其年九月,遂更營新廟,遠致名材,雜以銅柱。陳勰為匠,作者六萬人。



 十年四月,乃
 成。十一月庚寅,梁又折。按地陷者,分離之象;梁折者,木不曲直也。孫盛曰:于時後宮殿有孽火,又廟梁無故自折。先是帝多不豫,益惡之。明年,帝崩,而王室頻亂,遂亡天下。



 晉惠帝太安二年,成都王穎使陸機率眾向京師,擊長沙王乂。軍始引而牙竿折,俄而戰敗,機被誅。穎尋奔潰,卒賜死。初,河間王顒謀先誅長沙,廢太子,立穎。



 長沙知之,誅其黨卞粹等,故穎來伐。機又以穎得遐邇心,將為
 漢之代王,遂委質於穎,為犯從之將。此皆姦謀之罰,木不曲直也。



 王敦在武昌,鈴下儀仗生華如蓮花狀,五六日而萎落,此木失其性而為變也。



 干寶曰:「鈴合,尊貴者之儀;鈴下,主威儀之官。今狂花生於枯木,又在鈴合之間,言威儀之富,榮華之盛,皆如狂花之發,不可久也。」其後終以逆命,沒又加戮,是其應也。一說此花孽也,於《周易》為「枯楊生華」。



 桓玄始篡,龍旂竿折。玄田獵出入,不絕昏夜,飲食恣奢,土水妨農,又多姦謀,故木失其性也。夫旂所以擬三辰,章著明也。旂竿之折,高明去矣。在位八十日而敗。



 宋明帝泰始二年五月丙午,南琅邪臨沂黃城山道士盛道度堂屋一柱自然,夜光照室內。此木失其性也。或云木腐自光。廢帝昇明元年,吳興餘杭舍亭禾蕈樹生李實。禾蕈樹,民間所謂胡頹樹。



 貌不恭:魏文帝居諒暗之始,便數出遊獵,體貌不重,風尚通脫。故戴凌以直諫抵罪,鮑勛以迕旨極刑。天下化之,咸賤守節,此貌之不恭也。是以享國不永,後祚短促。



 《春秋》魯君居喪不哀,在戚而有嘉容,穆叔謂之不度,後終出奔。蓋同事也。



 魏尚書鄧颺,行步弛縱,筋不束體,坐起傾倚,若無手足。此貌之不恭也。管輅謂之鬼躁。鬼躁者,凶終之徵。後卒
 誅死。



 晉惠帝元康中,貴遊子弟相與為散髮惈身之飲,對弄婢妾。逆之者傷好,非之者負譏。希世之士,恥不與焉。蓋胡、翟侵中國之萌也。豈徒伊川之民,一被髮而祭者乎?晉惠帝元康中,賈謐親貴,數入二宮,與儲君游戲,無降下心。又嘗同弈棋爭道,成都王穎厲色曰:「皇太子,國之儲貳,賈謐何敢無禮!」謐猶不悛,故及於禍。



 齊王冏既誅趙倫,因留輔政,坐拜百官,符敕臺府,淫F
 W專驕,不一朝覲。



 此狂恣不肅之容也。天下莫不高其功,而慮其亡也。冏終弗改,遂至夷滅。



 太元中,人不復著帩頭。頭者,元首;帩者,令髮不垂,助元首為儀飾者也。



 今忽廢之,若人君獨立無輔,以至危亡也。其後桓玄篡位。舊為屐者,齒皆達楄上,名曰「露卯」。太元中,忽不徹,名曰:「陰卯」。其後多陰謀,遂致大亂。



 晉安帝義熙七年,晉朝拜授劉毅世子。毅以王命之重,
 當設饗宴親,請吏佐臨視。至日,國僚不重白,默拜於廄中。王人將反命,毅方知,大以為恨,免郎中令劉敬叔官。識者怪焉。此墮略嘉禮,不肅之妖也。



 陳郡謝靈運有逸才,每出入,自扶接者常數人。民間謠曰「四人挈衣裙,三人捉坐席」是也。此蓋不肅之咎,後坐誅。



 宋明帝泰始中,幸臣阮佃夫勢傾朝廷,室宇豪麗,車服鮮明,乘車常偏向一邊,違正立執綏之體。時人多慕效。
 此亦貌不恭之失也。時偏左之化行,方正之道廢矣。



 後廢帝常單騎游遨,出入市里營寺,未嘗御輦。終以殞滅。



 恆雨:魏明帝太和元年秋,數大雨,多暴雷電,非常,至殺鳥雀。案楊阜上疏,此恒雨之罰也。時帝居喪不哀,出入弋獵無度,奢侈繁興,奪民農時,故木失其性而恆雨為災也。太和四年八月,大雨霖三十餘日,伊、洛、河、漢皆溢,歲以
 兇饑。



 孫亮太平二年二月甲寅,大雨震電;乙卯,雪,大寒。案劉歆說,此時當雨而不當大,大雨,恒雨之罰也。於始震電之明日而雪大寒,又恒寒之罰也。劉向以為既已震電,則雪不當復降,皆失時之異也。天戒若曰,為君失時,賊臣將起。先震電而後雪者,陰見間隙,起而勝陽,逆殺之禍將及也。亮不悟,尋見廢。此與《春秋》魯隱同也。



 晉武帝泰始六年六月,大雨霖;甲辰,河、洛、沁水同時並
 溢,流四千九百餘家,殺二百餘人,沒秋稼千三百六十餘頃。晉武太康五年七月,任城、梁國暴雨,害豆麥。太康五年九月,南安霖雨暴雪,折樹木,害秋稼;魏郡、淮南、平原雨水,傷秋稼。是秋,魏郡、西平郡九縣霖雨暴水,霜傷秋稼。



 晉惠帝永寧元年十月,義陽、南陽、東海霖雨,淹害秋麥。



 晉成帝咸康元年八月乙丑,荊州之長沙攸、醴陵、武陵之龍陽三縣,雨水浮漂屋室,殺人,傷損秋稼。



 宋文帝元嘉二十一年六月,京邑連雨百餘日,大水。



 孝武帝大明元年正月,京邑雨水。大明五年七月,京邑雨水。大明八年八月,京邑雨水。



 明帝太始二年六月,京邑雨水。



 順帝升明三年四月乙亥,吳郡桐廬縣暴風雷電,揚砂折木,水平地二丈,流漂居民。



 服妖:
 魏武帝以天下凶荒,資財乏匱,始擬古皮弁,裁縑帛為白帢,以易舊服。傅玄曰:「白乃軍容,非國容也。」干寶以為縞素,凶喪之象;帢,毀辱之言也。蓋革代之後,攻殺之妖也。初為白帢,橫縫其前以別後,名之曰「顏」,俗傳行之。至晉永嘉之間,稍去其縫,名「無顏帢」。而婦人束髮,其緩彌甚,紒之堅不能自立,髮被于額,目出而已。無顏者,愧之言也;覆額者,慚之貌;其緩彌甚,言天下忘禮與義,放縱情性,及其終極,至乎大恥也。永嘉之後,二帝不反,天下
 愧焉。魏明帝著繡帽,被縹紈半袖,嘗以見直臣楊阜。阜諫曰:「此於禮何法服邪?」帝默然。近服妖也。縹,非禮之色,褻服不貳。今之人主,親御非法之章,所謂自作孽不可禳也。帝既不享永年,身沒而祿去王室,後嗣不終,遂亡天下。



 魏明帝景初元年,發銅鑄為巨人二,號曰「翁仲」,置之司馬門外。案古長人見,為國亡;長狄見臨洮,為秦亡之禍。始皇不悟,反以為嘉祥,鑄銅人以象之。



 魏法亡國之器,
 而於義竟無取焉。蓋服妖也。



 魏尚書何晏,好服婦人之服。傅玄曰:「此服妖也。」夫衣裳之制,所以定上下,殊內外也。《大雅》云:「玄袞赤舄,鉤膺鏤錫。」歌其文也。《小雅》云:「有嚴有翼,共武之服。」詠其武也。若內外不殊,王制失敘,服妖既作,身隨之亡。末喜冠男子之冠,桀亡天下;何晏服婦人之服,亦亡其家。其咎均也。



 吳婦人之修容者,急束其髮,而劘角過于耳。蓋其俗自操束大急,而廉隅失中之謂也。故吳之風俗,相驅以急,
 言論彈射,以刻薄相尚。居三年之喪者,往往有致毀以死。諸葛患之,著《正交論》,雖不可以經訓整亂,蓋亦救時之作也。孫休後,衣服之制,上長下短,又積領五六而裳居一二。干寶曰:「上饒奢,下儉逼,上有餘下不足之妖也。」至孫皓,果奢暴恣情於上,而百姓彫困於下,卒以亡國。



 是其應也。



 晉興後,衣服上儉下豐,著衣者皆厭腰蓋裙。君衰弱,臣放縱,下掩上之象也。



 陵遲至元康末,婦人出兩襠,加乎
 脛之上,此內出外也。為車乘者,茍貴輕細,又數變易其形,皆以白蔑為純,古喪車之遺象。乘者,君子之器,蓋君子立心無恒,事不崇實也。干寶曰:「及晉之禍,天子失柄,權制寵臣,下掩上之應也。永嘉末,六宮才人,流徙戎、翟,內出外之應也。及天下亂擾,宰輔方伯,多負其任,又數改易,不崇實之應也。」



 晉武帝泰始後,中國相尚用胡床、貊盤,及為羌煮、貊炙。貴人富室,必置其器,吉享嘉會,皆此為先。太康中,天下
 又以氈為絈頭及絡帶、衿口,百姓相戲曰,中國必為胡所破也。氈產於胡,而天下以為絈頭帶身、衿口,胡既三制之矣,能無敗乎。干寶曰:「元康中,氐、羌反,至于永嘉,劉淵、石勒遂有中都。自後四夷迭據華土,是其應也。」



 晉武帝太康後,天下為家者,移婦人於東方,空萊北庭,以為園囿。干寶曰:「夫王朝南向,正陽也;后北宮,位太陰也;世子居東宮,位少陽也。今居內於東,是與外俱南面也。亢陽無陰,婦人失位而乾少陽之象也。賈后讒戮愍
 懷,俄而禍敗亦及。」



 昔初作履者,婦人圓頭,男子方頭。圓者,順從之義,所以別男女也。晉太康初,婦人皆履方頭,此去其圓從,與男無別也。太康之中,天下為《晉世寧》之舞,手接杯槃反覆之,歌曰:「晉世寧,舞杯槃。」夫樂生人心,所以觀事。故《記》曰:「總干山立,武王之事也;發揚蹈厲,太公之志也;《武》亂皆坐,周、召之治也。」又曰:「其治民勞者,舞行綴遠;其治民逸者,舞行
 綴近。今接杯盤於手上而反覆之,至危也。杯槃者,酒食之器也。而名曰《晉世寧》者,言晉世之士,偷茍於酒食之間,而其知不及遠,晉世之寧,猶杯槃之在手也。」



 晉惠帝元康中,婦人之飾有五兵佩,又以金、銀、玳瑁之屬為斧、鉞、戈、戟,以當笄囗。干寶曰:「男女之別,國之大節,故服物異等,贄幣不同。今婦人而以兵器為飾,又妖之大也。遂有賈后之事,終以兵亡天下。」



 元康中,婦人結發者,既成,以繒急束其環,名曰擷子紒。
 始自中宮,天下化之。其後賈后果害太子。元康中,天下始相仿為樢杖,以柱掖其後,稍施其錞,住則植之。夫木,東方之行,金之臣也。杖者,扶體之器,樢其頭者,尤便用也。



 必傍柱掖者,傍救之象也。王室多故,而元帝以蕃臣樹德東方,維持天下,柱掖之應也。至社稷無主,海內歸之,遂承天命,建都江外,獨立之應也。



 元康末至太安間,江、淮之域,有敗編自聚于道,多者或至四五十量。干寶嘗使人散而去之,或投林草,或投坑
 谷。明日視之,悉復如故。民或云見狸銜而聚之,亦未察也。寶說曰:「夫編者,人之賤服,最處于下,而當勞辱,下民之象也。敗者,疲斃之象也。道者,地理四方,所以交通王命所由往來也。故今敗編聚於道者,象下民罷病,將相聚為亂,絕四方而壅王命之象也。在位者莫察。太安中,發壬午兵,百姓嗟怨。江夏男子張昌遂首亂荊楚,從之者如流。於是兵革歲起,天下因之,遂大破壞。此近服妖
 也。」



 晉孝懷永嘉以來,士大夫竟服生箋單衣。遠識者怪之,竊指摘曰:「此則古者繐衰之布,諸侯大夫所以服天子也。今無故畢服之,殆有應乎?」其後愍、懷晏駕,不獲厥所。



 晉元帝太興以來,兵士以絳囊縛紒。紒在首,莫上焉。《周易》《乾》為首,《坤》為囊。《坤》,臣道也。晉金行,赤火色,金之賊也。以硃囊縛紒,臣道上侵之象也。到永昌元年,大將軍王敦舉兵內攻,六軍散潰。



 舊為羽扇,柄刻木,象其骨形,羽用十,取全數也。晉中興
 初,王敦南征,始改為長柄下出,可捉,而減其羽用八。識者尤之曰:「夫羽扇,翼之名也。創為長柄者,執其柄制羽翼也。以十改八者,將以未備奪已備也。」是時為衣者,又上短,帶至于掖;著帽者,以帶縛項。下逼上,上無地也。下褲者,直幅為口無殺,下大失裁也。尋有兵亂,三年而再攻京師。晉海西初嗣位,迎官忘設豹尾。識者以為不終之象,近服妖也。



 晉司馬道子於府北園內為酒金盧列肆,使姬人酤鬻酒
 肴,如裨販者,數遊其中,身自巘易,因醉寓寢,動連日夜。漢靈帝嘗若此。干寶以為:「君將失位,降在皁隸之象也。」道子卒見廢徙,以庶人終。



 桓玄篡立,殿上施絳綾帳,鏤黃金為顏,四角金龍,銜五色羽葆流蘇。群下竊相謂曰:「頗類蒐車。」此服妖也。



 晉末皆冠小冠,而衣裳博大,風流相仿,輿臺成俗,識者曰:「此禪代之象也。」



 永初以後,冠還大云。



 宋文帝元嘉六年,民間婦人結發者,三分髮,抽其鬟直
 向上,謂之「飛天紒」。



 始自東府,流被民庶。時司徒彭城王義康居東府,其後卒以陵上徙廢。



 孝武帝世,豫州刺史劉德願善御車,世祖嘗使之御畫輪,幸太宰江夏王義恭第。



 德願挾牛杖催世祖云:「日暮宜歸!」又求益僦車。世祖甚歡。此事與漢靈帝西園蓄私錢同也。孝武世,幸臣戴法興權亞人主,造圓頭履,世人莫不效之。其時圓進之俗大行,方格之風盡矣。



 明帝初,司徒建安王休仁統軍赭圻,制烏紗帽,反抽帽裙,民間謂之「司徒狀」,京邑翕然相尚。休仁後果以疑逼致禍。



 龜孽:晉惠帝永熙初,衛瓘家人炊飯,墮地,盡化為螺,出足起行。螺,龜類,近龜孽也。干寶曰:「螺被甲,兵象也。於《周易》為《離》,《離》為戈兵。」明年,瓘誅。



 雞禍:
 魏明帝景初二年,廷尉府中有雌雞變為雄,不鳴不將。干寶曰:「是歲,晉宣帝平遼東,百姓始有與能之義,此其象也。」然晉三后並以人臣終,不鳴不將,又天意也。



 晉惠帝元康六年,陳國有雞生雄雞無翅,既大,墜坑而死。王隱曰:「雄,胤嗣象,坑地事為母象,賈后誣殺愍懷,殆其應也。」晉惠帝太安中,周𤣱家有雌雞逃承溜中,六七日而下,奮翼鳴將,獨毛羽不變。其後有陳敏之事。敏雖控制江
 表,終無綱紀文章,殆其象也。卒為𤣱所滅。雞禍見𤣱家,又天意也。



 晉元帝太興中,王敦鎮武昌,有雌雞化為雄。天戒若曰:「雌化為雄,臣陵基上。」其後王敦再攻京師。



 晉孝武太元十三年四月,廣陵高平閻嵩家雄雞,生無右翅;彭城到象之家雞,無右足。京房《易傳》曰:「君用婦人言,則生雞妖。」



 晉安帝隆安元年八月,琅邪王道子家青雌雞化為赤
 雄,不鳴不將。後有桓玄之事,具如其象。隆安四年,荊州有雞生角,角尋墮落。是時桓玄始擅西夏,狂慢不肅,故有雞禍。角,兵象;尋墮落者,暫起不終之妖也。晉安帝元興二年,衡陽有雌雞化為雄,八十日而冠萎。衡陽,桓玄楚國封略也。後篡位八十日而敗,徐廣以為玄之象也。



 宋文帝元嘉十二年,華林園雌雞漸化為雄。後孝武即
 位,皇太后令行于外,亦猶漢宣帝時,雌雞為雄,至哀帝時,元后與政也。



 明帝泰始中,吳興東遷沈法符家雞有四距。



 青眚青祥:晉武帝咸寧元年八月丁酉,大風折太社樹,有青氣出焉;此青祥也。占曰:「東莞當有帝者。」明年,元帝生。是時,帝大父武王封東莞,由是徙封琅邪。孫盛以為中興之表。晉室之亂,武帝子孫無孑遺,社樹折之應,又恒風之罰
 也。



 晉惠帝元康中,洛陽南山有虻作聲曰:「韓屍屍。」識者曰:「韓氏將死也。



 言屍屍者,盡死意也。」其後韓謐誅而韓族殲焉。此青祥也。



 金沴木:魏文帝黃初七年正月,幸許昌。許昌城南門無故自崩,帝心惡之,遂不入,還洛陽。此金沴木,木動也。五月,宮車晏駕。京房《易傳》曰:「上下咸悖,厥妖城門壞。」



 晉元帝太興二年六月,吳郡米廩無故自壞。是歲大饑,死者數千。



 晉明帝太寧元年,周延自歸王敦,既立宅宇,而所起五間六架,一時躍出墮地,餘桁猶亙柱頭。此金沴木也。明年五月,錢鳳謀亂,遂族滅筵,而湖熟尋亦為墟矣。



 晉安帝元興元年正月丙子,司馬元顯將西討桓玄,建牙揚州南門,其東者難立,良久乃正。近沴妖也。尋為桓玄所禽。
 元興三年五月,樂賢堂壞。天意若曰,安帝囂眊,不及有樂賢之心,故此堂見沴也。晉安帝義熙九年五月乙酉,國子聖堂壞。



 宋文帝元嘉十七年,劉斌為吳郡,郡堂屋西頭鴟尾無故落地,治之未畢,東頭鴟尾復落。頃之,斌誅。



\end{pinyinscope}