\article{卷三十一志第二十一 五行二}

\begin{pinyinscope}

 《五行傳》曰:「好戰攻,輕百姓,飾城郭,侵邊境,則金不從革。謂金失其性而為災也。」又曰:「言之不從,是謂不乂。厥咎僭,厥罰恆晹,厥極憂。時則有詩妖,時則有介蟲之孽,時
 則有犬禍,時則有口舌之痾,時則有白眚、白祥。惟木沴金。」介蟲,劉歆傳以為毛蟲。



 金不從革:魏世張掖石瑞,雖是晉氏之符命,而於魏為妖。好攻戰,輕百姓,飾城郭,侵邊境,魏氏三祖皆有其事。劉歆以為金石同類,石圖發非常之文,此不從革之異也。



 晉定大業,多敝曹氏,石瑞文「大討曹」之應也。魏明帝青龍中,盛修宮室,西取長安金狄,承露槃折,聲
 聞數十里,金狄泣,於是因留霸城。此金失其性而為異也。



 吳時,歷陽縣有巖穿似印,咸云「石印封發,天下太平」。孫皓天璽元年印發。



 又陽羨山有石穴,長十餘丈。皓初修武昌宮,有遷都之意。是時武昌為離宮。班固云:「離宮與城郭同占。」飾城郭之謂也。寶鼎三年,皓出東關,遣丁奉至合肥;建衡三年,皓又大舉出華里。侵邊境之謂也。故令金失其性,卒面縛而吳亡。



 晉惠帝永興元年,成都伐長沙,每夜戈戟鋒有火光如縣燭。此輕民命,好攻戰,金失其性而為變也。天戒若曰,兵猶火也,不戢將自焚。成都不悟,終以敗亡。



 晉懷帝永嘉元年,項縣有魏豫州刺史賈逵石碑,生金可采。此金不從革而為變也。五月,汲桑作亂,群寇飆起。



 晉清河王覃為世子時,所佩金鈴忽生起如粟者。康王母疑不祥,毀棄之。及後為惠帝太子,不終于位,卒為司馬越所殺。



 晉元帝永昌元年,甘卓將襲王敦,既而中止。及還家,多變怪,照鏡不見其頭。



 此金失其性而為妖也。尋為敦所襲,遂夷滅。



 石虎時,鄴城鳳陽門上金鳳凰二頭,飛入漳河。



 晉海西太和中,會稽山陰縣起倉,鑿地得兩大船,滿中錢,錢皆輪文大形。時日向莫,鑿者馳以告官。官夜遣防守甚嚴。至明旦,失錢所在,唯有船存,視其狀,悉有錢處。



 晉安帝義熙初,東陽太守殷仲文照鏡不見其頭,尋亦
 誅翦。占與甘同。



 宋後廢帝元徽四年,義熙、晉陵二郡,並有霹靂車墜地,如青石,草木燋死。



 言之不從:魏齊王嘉平初,東郡有訛言云,白馬河出妖馬,夜過官牧邊鳴呼,眾馬皆應。



 明日見其迹,大如斛,行數里,還入河。楚王彪本封白馬,兗州刺史令狐愚以彪有智勇,及聞此言,遂與王凌謀共立之。遣人謂曰:「天下事未可知,
 願王自愛。」



 彪答曰:「知厚意。」事洩,凌、愚被誅,彪賜死。此言不從之罰也。詩云:「民之訛言,寧莫之懲。」



 劉禪嗣位,譙周引晉穆侯、漢靈帝命子事譏之曰:「先主諱備,其訓具也。後主諱禪,其訓授也。若言劉已具矣,當授與人,甚於穆侯、靈帝之詳也。」蜀果亡,此言之不從也。



 劉備卒,劉禪即位,未葬,亦未踰月,而改元為建興。此言之不從也。習鑿齒曰:「禮,國君即位踰年而後改元者,緣臣子之心,不忍一年而有二君也。今可謂亟而不知禮
 矣。君子是以知蜀之不能東遷也。」後又降晉。吳孫亮、晉惠帝、宋元凶亦然。亮不終其位,惠帝號令非己,元凶尋誅。言不從也。魏太和中,姜維歸蜀,失其母。魏人使其母手書呼維令反,并送當歸以譬之。維報書曰:「良田百頃,不計一畝。但見遠志,無有當歸。」維卒不免。



 魏明帝景初元年,有司奏帝為烈祖,與太祖、高祖並為不毀之廟。從之。按宗廟之制,祖宗之號,皆身沒名成,乃
 正其禮。故雖功赫天壤,德邁前王,未有豫定之典。此蓋言之不從,失之甚者也。後二年而宮車晏駕,於是統微政逸。



 吳孫休世,烏程民有得困疾,及差,能以響言者,言於此而聞於彼。自其所聽之,不覺其聲之大也;自遠聽之,如人對言,不覺聲之自遠來也。聲之所往,隨其所向,遠者不過十數里。其鄰人有責息於外,歷年不還。乃假之使為責讓,懼以禍福,負物者以為鬼神,即傾倒畀之。其人
 亦不自知所以然也。言不從之咎也。



 魏世起安世殿,晉武帝後居之。安世,武帝字也。晉武帝每延群臣,多說平生常事,未嘗及經國遠圖。此言之不從也。何曾謂子遵曰:「國家無貽厥之謀,及身而已,後嗣其殆乎,此子孫之憂也。」自永熙後,王室漸亂。永嘉中,天下大壞。



 及何綏以非辜被誅,皆如曾言。



 趙王倫廢惠帝於金墉城,改號金墉為永安宮。帝尋復位而倫誅。
 晉惠帝永興元年,詔廢太子覃還為清河王,立成都王穎為皇太弟,猶加侍中,大都督,領丞相,備九錫,封二十郡,如魏王故事。案周禮,傳國以胤不以勳,故雖公旦之聖,不易成王之嗣。所以遠絕覬覦,永一宗祧。後代遵履,改之則亂。今擬非其實,僭差已甚。且既為國副,則不應復開封土,兼領庶職。此言之不從,進退乖爽。故帝既播越,穎亦不終,是其咎也。後猶不悟,又立懷帝為皇太弟。懷終流弒,不永厥祚,又其應也。語曰:「變古易常,不亂則
 亡。」此之謂乎?



 晉惠帝太安中,周𤣱於陽羨起宅,始成,而邊戶有聲如人嘆吒者。𤣱亡後,家誅滅。此近言不從也。



 晉元帝太興四年,吳郡民訛言有大蟲在棨中及樗樹上,嚙人即死。晉陵民又言曰,見一老女子居市,被髮從肆人乞飲,自言:「天帝令我從水門出,而我誤由蟲門。若還,天帝必殺我。如何?」於是百姓共相恐動,云死者已十數也。西及京都,諸家有樗棨者,伐去之。無幾自止。
 晉元帝永昌元年,寧州刺史王遜遣子澄入質,將渝、濮雜夷數百人。京邑民忽訛言寧州人大食人家小兒,親有見其蒸煮滿釜甑中者。又云失兒皆有主名,婦人尋道,拊心而哭。於是百姓各禁錄小兒,不得出門。



 尋又言已得食人之主,官當大航頭大杖考竟。而日有四五百人晨聚航頭,以待觀行刑。朝廷之士相問者,皆曰信然,或言郡縣文書已上。王澄大懼,檢測之,事了無形,民家亦未嘗有失小兒者,然後知其訛言也。此二事,干寶云「
 未之能論」。



 永昌二年,大將軍王敦下據姑熟。百姓訛言行蟲病,食人大孔,數日入腹,入腹則死。治之有方,當得白犬膽以為藥。自淮、泗遂及京都,數日之間,百姓驚擾,人人皆自云已得蟲病。又云,始在外時,當燒鐵以灼之。於是翕然被燒灼者十七八矣。而白犬暴貴,至相請奪,其價十倍。或有自云能行燒鐵者,賃灼百姓,日得五六萬,憊而後已。四五日漸靜。說曰,夫裸蟲人類,而人為之主,今云蟲
 食人,言本同臭類而相殘賊也。自下而上,斯其逆也。必入腹者,言害由中不由外也。犬有守禦之性,白者金色,而膽用武之主也。帝王之運,五霸會於戌,戌主用兵。金者晉行,火燒鐵以治疾者,言必去其類而來,火與金合德,共治蟲害也。案中興之際,大將軍本以腹心受伊、呂之任,而元帝末年,遂攻京邑,明帝諒暗,又有異謀。是以下逆上,腹心內爛也。及錢鳳、沈充等逆兵四合,而為王師所挫,踰月而不能濟。



 北中郎將劉遐及淮陵內史蘇
 峻率淮、泗之眾以救朝廷,故其謠言首作於淮、泗也。



 朝廷卒以弱制強,罪人授首,是用白犬膽可救之效也。



 晉海西時,庾晞四五年中,喜為挽歌,自搖大鈴為唱,使左右齊和。又燕會,輒令倡妓作新安人歌舞離別之辭,其聲悲切。時人怪之,後亦果敗。晉海西公太和以來,大家婦女,緩鬢傾髻,以為盛飾。用髮既多,不恒戴。乃先作假髻,施於木上,呼曰「假頭」。人欲借,名曰「借頭」,遂布天下。自此以來,人士多離事故,或亡
 失頭首,或以草木為之。假頭之言,此其先兆也。



 晉孝武泰元中,立內殿名曰清暑,少時而崩。時人曰,「清暑」者,反言楚聲也。果有哀楚之聲。有人曰:「非此之謂,豈可極言乎。讖云,代晉者楚,其在茲乎?」及桓玄篡逆,自號曰楚。太元中,小兒以兩鐵相打於土中,名曰「斗族」。



 後王國寶、王孝伯一姓之中,自相攻擊也。



 桓玄出鎮南州,立齋名曰蟠龍。後劉毅居此齋。蟠龍,毅
 小字也。桓玄初改年為大亨,遐邇沄言曰:「二月了。」故義謀以仲春發也。玄篡立,又改年為建始,以與趙王倫同,又易為永始。永始,復是王莽受封之年也。始徙司馬道子于安成,晉主遜位,出永安宮,封晉主為平固王,琅邪王德文為石陽公,並使住尋陽城。識者皆以為言不從之妖也。厥咎僭。



 晉興,何曾薄太官御膳,自取私食,子劭又過之,而王愷又過劭。王愷、羊琇之疇,盛致聲色,窮珍極麗。至
 元康中,夸恣成俗,轉相高尚。石崇之侈,遂兼王、何而儷人主矣。崇既誅死,天下尋亦淪喪。僭踰之咎也。



 恆暘:魏明帝太和二年五月,大旱。元年以來,崇廣宮府之應也。又是春,晉宣帝南禽孟達,置二郡;張郃西破諸葛亮,斃馬謖。亢陽自大,又其應也。京房《易傳》曰:「欲德不用,茲謂張。厥災荒。其旱陰雲不雨,變而赤煙四際。眾出過時,茲謂廣。其旱不生。上下皆蔽,茲謂隔。其旱天赤三月,時
 有雹殺飛禽。上緣求妃,茲謂僭。其旱三月大溫亡雲。君高臺府,茲謂犯。陰侵陽,其旱萬物根死,數有火災。庶位踰節,茲謂僭。其旱澤物枯,為火所傷。」太和五年三月,自去冬十月至此月不雨,辛巳,大雩。是春,諸葛亮寇天水,晉宣王距卻之,亢陽動眾。



 又是時二隅分據,眾出多過時也。《春秋》說曰:「傷二穀,謂之不雨。」



 魏齊王正始元年二月,自去冬十二月至此月不雨。去歲正月,明帝崩。二月,曹爽白嗣主,轉晉宣王為太傅,外
 示尊崇,內實欲令事先由己。是時宣王功蓋魏朝,欲德不用之應也。



 魏高貴鄉公甘露三年正月,自去秋至此月旱。時晉文王圍諸葛誕,眾出過時之應也。初,壽春秋夏常雨潦,常淹城,而此旱踰年,城陷乃大雨。咸以為天亡。



 吳孫亮五鳳二年,大旱,民饑。是歲閏月,魏將文欽以淮南眾數萬口來奔;孫峻又破魏將曹珍于高亭。三月,朱異襲安豐,不克。七月,城廣陵、東海二郡。十二月,以馮朝
 為監軍使者,督徐州諸軍,軍士怨叛。此亢陽自大,勞民失眾之罰也。



 其役彌歲,故旱亦竟年。



 吳孫皓寶鼎元年春夏旱。是時皓遷都武昌,勞民動眾之應也。



 晉武帝泰始七年五月閏月,旱,大雩。是春,孫皓出華里,大司馬望帥眾次于準北。四月,北地胡寇金城西平,涼州刺史牽弘出戰,敗沒。泰始八年五月,旱。



 是時帝納荀勖邪說,留賈充不復西
 鎮,而任愷稍疏,上下皆蔽之應也。又李喜、魯芝、李胤等並在散職,近欲德不用之謂也。泰始九年,自正月旱,至于六月,祈宗廟社稷山川,癸未雨。去年九月,吳西陵督步闡據城來降,遣羊祜統楊肇等眾八萬救迎闡。十二月,陸抗大破肇軍,攻闡滅之。泰始十年四月,旱。去年秋冬,采擇卿校諸葛沖等女,是春五十餘人入殿簡選。又取小將吏女數十人,母子號哭於宮中,聲聞于外,行人悲酸。是殆積陰生陽之應也。



 晉武帝咸寧二年五月,旱,大雩,及社稷山川。至六月,乃澍雨。



 晉武帝太康二年,自去冬旱,至此春平吳,亢陽動眾自大之應也。太康三年四月,旱。乙酉,詔司空齊王攸與尚書、廷尉、河南尹錄訊系囚,事從蠲宥。太康五年六月,旱。此年正月,天陰,解而復合。劉毅上疏曰:「必有阿黨之臣,奸以事君者,當誅而不赦也。」帝不答。
 是時荀勖、馮紞僭作威福,亂朝尤甚。太康六年三月,青、涼、幽、冀郡國旱。



 太康六年六月,濟陰、武陵旱,傷麥。太康七年夏,郡國十三旱。太康八年四月,冀州旱。太康九年夏,郡國三十三旱。太康九年六月,扶風、始平、京兆、安定旱,傷麥。太康十年二月,旱。



 晉武帝太熙元年二月,旱。自太康以後,雖正人滿朝,不被親仗;而賈充、荀勖、楊駿、馮紞憐等,迭居要重。所以無年不旱者,欲德不用,上下皆蔽,庶位踰節之罰也。



 晉惠帝元康元年七月,雍州大旱,殞霜疾疫。關中飢,米斛萬錢。元康七年七月,秦雍二州大旱。故其年氐羌反叛,雍州刺史解系敗績。是年正月,周處、盧播等復敗,關西震亂。交兵彌歲,至是飢疫薦臻,戎、晉並困,朝廷不能振,詔聽
 相賣鬻。元康七年九月,郡國五旱。



 晉惠帝永寧元年,自夏及秋,青、徐、幽、並四州旱。是年春,三王討趙王倫,六旬之中,大小數十戰,死者十餘萬人。十二月,郡國十二又旱。



 晉懷帝永嘉三年五月,大旱。襄平縣梁水淡淵竭,河、洛、江、漢皆可涉。是年三月,司馬越歸京都,遣兵入宮,收中書令繆播等九人殺之。此僭踰之罰也。又四方諸侯,多
 懷無君之心,劉淵、石勒、王彌、李雄之徒,賊害民命,流血成泥,又其應也。永嘉五年,自去冬旱至此春。去歲十二月,司馬越棄京都,以大眾南出,多將王公朝士,及以行臺自隨,斥黜禁衛,代以國人。宮省蕭然,無復君臣之節矣。



 《晉陽秋》云:「愍帝在西京,旱傷薦臻。」無注記年月也。晉愍帝建武元元年六月,揚州旱。去年十二月,淳于伯冤死,其年即旱,而太興元年六月又旱。干寶曰「殺伯之後
 旱三年」是也。案前漢殺孝婦則旱,後漢有囚亦旱,見謝見理,並獲雨澍,此其類也。班固曰:「刑罰妄加,群陰不附,則陽氣勝,故其罰恆暘。」



 建武元年四月,曲允等悉眾禦寇。五月,祖逖攻譙。其冬,周訪討杜曾。又眾出之應也。



 晉元帝太興四年五月,旱。是時,王敦彊僭之釁漸著。又去歲蔡豹、祖逖等,並有征役。晉元帝永昌元年,大旱。是年三月,王敦有石頭之變,二
 宮陵辱,大臣誅死。僭踰無上,故旱尤甚也。永昌元年閏十一月,京都大旱,川谷並竭。



 晉明帝太寧三年,自春不雨,至於六月。去年秋,滅王敦,亢陽動眾自大之應也。



 晉成帝咸和元年秋,旱。是時庾太后臨朝稱制,群臣奏事稱「皇太后陛下」。



 此婦人專王事,言不從而僭踰之罰也。與漢鄧太后同事。咸和二年夏,旱。
 咸和五年五月,旱。去年殄蘇峻之黨,此春又討郭默滅之。亢陽動眾之應也。咸和六年四月,旱。去年八月,石勒遣郭敬寇襄陽,南中郎將周撫奔武昌。十月,李雄使李壽寇建平,建平太守楊謙奔宜都。此正月,劉征略婁縣,於是起眾警備。咸和八年七月,旱。咸和九年,自四月不雨,至于八月。



 晉成帝咸康元年六月,旱。是時成帝沖弱,不親萬機,內
 外之政,委之將相。



 此僭踰之罰,故連歲旱也。至四年,王導固讓太傅,復子明辟,是後不旱,殆其應也。時天下普旱,會稽餘姚特甚,米斗直五百,民有相鬻。咸康二年三月,旱。咸康三年六月,旱。



 晉康帝建元年五月,旱。是時宰相專政,方伯擅重兵,又與咸康初同事也。



 晉穆帝永和元年五月,旱。有司奏依董仲舒術,徙市開
 水門,遣謁者祭太社。



 是時帝在衣強抱,褚太后臨朝如明穆太后故事。永和五年七月,不雨,至于十月。



 是年二月,征北將軍褚裒遣軍伐沛,納其民以歸。六月,又遣西中郎將陳達進據壽陽,自以舟師二萬至于下邳,喪其前驅而還,達亦退。永和六年閏月,旱。是春,桓溫以大眾出夏口,上疏欲以舟軍北伐,朝廷駭之。蕭敬文盜涪,四蠻校尉采壽敗績。



 晉穆帝升平三年十二月,大旱。此冬十月,北中郎將郗曇帥萬餘人出高平,經略河、兗;又遣將軍諸葛悠以舟軍入河,敗績。西中郎將謝萬次下蔡,眾潰而歸。



 升平四年十二月,大旱。



 晉哀帝隆和元年夏,旱。是時桓溫強恣,權制朝廷,僭踰之罰也。又去年慕容恪圍冀州刺史呂護,桓溫出次宛陵,范汪、袁真並北伐,眾出過時也。



 晉海西太和四年十二月,涼州春旱至夏。



 晉簡文帝咸安二年十月,大旱,民飢。是時嗣主幼沖,桓溫陵僭。



 晉孝武帝寧康元年二月,旱。是時桓溫入覲高平陵,合朝致拜,踰僭之應也。



 寧康三年冬,旱。先是,氐賊破梁、益州,刺史楊亮、周仲孫奔退。明年,威遠將軍桓石虔擊姚萇墊江,破之,退至五城。益州刺史竺瑤帥眾戍巴東。



 晉孝武帝太元四年六月,大旱。去歲,氐賊圍南中郎將
 朱序於襄陽,又圍揚威將軍戴遁於彭城。桓嗣以江州之眾次鄀援序,北府發三州民配何謙救遁。是春,襄陽、順陽、魏興城皆沒,賊遂略淮南,向廣陵。征虜將軍謝石率水軍次塗中,兗州刺史謝玄督諸將破之。太元八年六月,旱。夏初,桓沖徵襄陽,遣冠軍將軍桓石虔進據樊城。朝廷又遣宣城內史胡彬次峽石為沖聲勢也。太元十年七月,旱饑。初八年,破苻堅;九年,諸將略地,有
 事徐、豫;楊亮、趙統攻討巴、沔。是年正月,謝安又出鎮廣陵,使子琰進次彭城。太元十三年六月,旱。去歲,北府遣戍胡陸,荊州經略河南。是年,郭銓置戍野王,又遣軍破黃淮。太元十五年七月,旱。是春,丁零略兗、豫,鮮卑寇河上。硃序、桓不才等北至太行,東至滑臺,踰時攻討,又戍石門。太元十七年秋,旱,至冬。是時茹千秋為驃騎諮議,竊弄主相威福;又丘尼乳母親黨及婢僕之子,階緣近習,臨
 民領眾。又在所多上春竟囚,不以其辜,建康獄吏枉暴尤甚。此僭踰不從,冤濫之罰也。



 晉安帝隆安四年五月,旱。去冬桓玄迫殺殷仲堪,而朝廷即授以荊州之任;司馬元顯又諷百僚悉使敬己。此皆陵僭之罰也。隆安五年夏秋,大旱,十二月不雨。



 去年夏,孫恩入會稽,殺內史謝琰;此年夏,略吳,又殺內史袁山松。軍旅東討,眾出過時。



 晉安帝元興元年七月,大饑;九月十月不雨。是年正月,司馬元顯以大眾將討桓玄,既而玄至,殺元顯。五月,又遣東征孫恩餘黨,十月,北討劉軌。元興二年六月,不雨,冬,又旱。是時桓玄奢僭,十二月,遂篡位。元興三年八月,不雨。



 是時王旅四伐,西夏未平。



 晉安帝義熙六年九月,不雨。是時王師北討廣固,疆理三州。
 義熙八年十月,不雨。是秋,王師西討劉毅;分遣伐蜀。義熙十年九月,旱;十二月,又旱,井瀆多竭。



 宋文帝元嘉二年夏,旱。元嘉四年秋,京都旱。元嘉八年五月,揚州諸郡旱。



 元嘉十九年、二十年,南兗、豫州旱。元嘉二十七年八月,不雨,至二十八年三月。



 時索虜南寇。



 孝武帝大明七年、八年,東諸郡大旱,民飢,死者十六七。先是江左以來,制度多闕,孝武帝立明堂,造五輅。是時大發徒眾,南巡校獵,盛自矜大,故致旱災。



 後廢帝元徽元年八月,京都旱。



 詩妖:魏明帝太和中,京師歌《兜鈴曹子》,其唱曰:「其奈汝曹何。」此詩妖也。



 其後曹爽見誅,曹氏遂廢。魏明帝景初中,童謠曰:「阿公阿公駕馬車,不意阿公東
 渡河。阿公東還當奈何!」及宣王平遼東,歸至白屋,當還鎮長安。會帝疾篤,急召之。乃乘追鋒車東渡河,終翦魏室,如童謠之言也。



 魏齊王嘉平中,有謠曰:「白馬索羈西南馳,其誰乘者硃虎騎。」朱虎者,楚王彪小字也。王凌、令狐愚聞此謠,謀立彪。事發,凌等伏誅,彪賜死。



 吳孫亮初,童謠曰:「吁汝恪,何若若,蘆葦單衣篾鉤絡,於何相求成子閣。」



 成子閣者,反語石子堈也。鉤落,鉤帶也。
 及諸葛恪死,果以葦席裹身,篾束其要,投之石子堈。後聽恪故吏收斂,求之此堈云。孫亮初,公安有白鼉鳴。童謠曰:「白鼉鳴,龜背平,南郡城中可長生,守死不去義無成。」南郡城可長生者,有急,易以逃也。明年,諸葛恪敗,弟融鎮公安,亦見襲。融刮金印龜,服之而死。鼉有鱗介,甲兵之象。又曰白祥也。



 孫休永安二年,將守質子群聚嬉戲,有異小子忽來,言曰:「三公鋤,司馬如。」



 又曰:「我非人,熒惑星也。」言畢上升,仰
 視若曳一匹練,有頃沒。干寶曰,後四年而蜀亡,六年而魏廢,二十一年而吳平,於是九服歸晉。魏與吳、蜀,並為戰國,「三公鋤,司馬如」之謂也。



 孫皓初,童謠曰:「寧飲建業水,不食武昌魚。寧還建業死,不止武昌居。」



 皓尋遷都武昌,民溯流供給,咸怨毒焉。孫皓遣使者祭石印山下妖祠。使者因以丹書巖曰:「楚九州渚,吳九州都。揚州士,作天子。四世治,太平矣。」皓聞之,意益張,曰:「從大皇帝至朕四世,太平之主,非朕復誰?」
 恣虐踰甚,尋以降亡。



 近詩妖也。孫皓天紀中,童謠曰:「阿童復阿童,銜刀游渡江。不畏岸上虎,但畏水中龍。」晉武帝聞之,加王濬龍驤將軍。及征吳,江西眾軍無過者,而王浚先定秣陵。



 晉武帝太康後,江南童謠曰:「局縮肉,數橫目,中國當敗吳當復。」又曰:「宮門柱,且莫朽,吳當復,在三十年後。」又曰:「雞鳴不拊翼,吳復不用力。」



 于時吳人皆謂在孫氏子孫,故竊發亂者相繼。按橫目者「四」字,自吳亡至晉元帝興,
 幾四十年,皆如童謠之言。元帝懦而少斷,局縮肉,直斥之也。干寶云「不知所斥」,諱之也。太康末,京、洛始為「折楊柳」之歌,其曲始有兵革苦辛之詞,終以禽獲斬截之事。是時三楊貴盛而族滅,太后廢黜而幽死。



 晉惠帝永熙中,河內溫縣有人如狂,造書曰:「光光文長,大戟為牆。毒藥雖行,戟還自傷。」又曰:「兩火沒地,哀哉秋蘭。歸形街郵,路人為歎。」及楊駿居內府,以戟為衛,死時,
 又為戟所害。楊太后被廢,賈后絕其膳,八日而崩,葬街郵亭北,百姓哀之。兩火,武帝諱;蘭,楊后字也。永熙中,童謠曰:「二月末,三月初,荊筆楊版行詔書,宮中大馬幾作驢。」楊駿初專權,楚王尋用事,故言「荊筆楊版」也。二人不誅,則君臣禮悖,故云「幾作驢。」



 晉惠帝元康中,京、洛童謠曰:「南風起,吹白沙,遙望魯國何嵯峨,千歲髑髏生齒牙。」又曰:「城東馬子莫嚨哅,比至三月纏汝鬃。」南風,賈后字也。白,晉行也。沙門,太子小名
 也。魯,賈謐國也。言賈后將與謐為亂,以危太子;而趙王因釁咀嚼豪賢,以成篡奪也。是時愍懷頗失眾望,卒以廢黜,不得其死。元康中,天下商農通著大鄣日,童謠曰:「屠蘇鄣日覆兩耳,當見瞎兒作天子。」及趙王篡位,其目實眇焉。趙王倫既篡,洛中童謠曰:「虎從北來鼻頭汗,龍從南來登城看,水從西來何灌灌。」數月而齊王、成都、河間義兵同會誅倫。按成都西蕃而在鄴,故曰:「虎從北來」;齊東蕃而在許,
 故曰「龍從南來」;河間水區而在關中,故曰「水從西來」。齊留輔政,居宮西,有無君之心,故言「登城看」也。



 晉惠帝太安中,童謠曰:「五馬游度江,一馬化為龍。」後中原大亂,宗蕃多絕,唯琅邪、汝南、西陽、南頓、彭城同至江表,而元帝嗣晉矣。



 司馬越還洛,有童謠曰:「洛中大鼠長尺二,若不蚤去大狗至。」及茍希將破汲桑,又謠曰:「元超兄弟大落度,上桑打椹為茍作。」由是越惡希,奪其兗州,隙難遂構。



 晉愍帝建興中,江南歌謠曰:「訇如白坑破,合集持作甒。揚州破換敗,吳興覆瓿甊。」按白者晉行,坑器有口,屬甕,瓦質剛,亦金之類也。「訇如白坑破」



 者,言二都傾覆,王室大壞也。「合集持作甒」者,言元皇帝鳩集遺餘,以主社稷,未能克復中原,偏王江南,故其喻小也。及石頭之事,六軍大潰,兵人抄掠京邑,爰及二宮。其後三年,錢鳳復攻京邑,阻水而守,相持月餘日,焚燒城邑,井堙木刊矣。鳳等敗退,沈充將其黨還吳興,官軍踵之,蹈藉郡縣。充父
 子授首,黨與誅者以百數。所謂「揚州破換敗,吳興覆瓿甊。」瓿甊,瓦器,又小於甒也。



 晉明帝太寧初,童謠歌曰:「惻力惻力,放馬山側。大馬死,小馬餓,高山崩,石自破。」及明帝崩,成帝幼,為蘇峻所逼,遷于石頭,御饍不足。「高山崩」,言峻尋死;「石」,峻弟蘇石也,峻死後,石據石頭,尋為諸公所破也。



 晉成帝之末,民間謠曰:「郤郤何隆隆,駕車入梓宮。」少日而宮車晏駕。
 晉成帝咸康二年十二月,河北謠語曰:「麥入土,殺石虎。」後如謠言。



 庾亮初出鎮武昌,出石頭,百姓於岸上歌曰:「庾公上武昌,翩翩如飛鳥。庾公還揚州,白馬牽旒旐」又曰:「庾公初上時,翩翩如飛鳥。庾公還揚州,白馬牽流蘇。」後連徵不入,及薨,還都葬。



 庾義在吳郡,吳中童謠曰:「寧食下湖荇,不食上湖FF。庾吳沒命喪,復殺王領軍。」無幾而庾義、王洽相繼亡。



 晉
 穆帝升平中,童子輩忽歌於道曰「阿子聞」,曲終輒云「阿子汝聞不」。無幾而穆帝崩,太后哭曰:「阿子汝聞不?」升平末,民間忽作廉歌。有扈謙者聞之,曰:「廉者臨也。歌云『白門廉,宮廷廉』,內外悉臨,國家其大諱乎?」少時而穆帝晏駕。



 晉哀帝隆和初,童兒歌曰:「升平不滿斗,隆和那得久!桓公入石頭,陛下徒跣走。」帝聞而惡之,復改年曰興寧。民復歌曰:「雖復改興寧,亦復無聊生。」



 哀帝尋崩。升平五年,
 穆帝崩。不滿斗,不至十年也。



 晉海西公太和中,民歌曰:「青青御路楊,白馬紫游韁。汝非皇太子,那得甘露漿。」白者金行;馬者國族;紫為奪正之色,明以紫間朱也。海西公尋廢,三子非海西子,並死,縊以馬韁死之。明日,南方獻甘露。太和末,童謠云:「犁牛耕御路,白門種小麥。」及海西被廢,處吳,民犁耕其門前,以種小麥,如謠言。



 晉海西公生皇子,百姓歌云:「鳳凰生一雛,天下莫不喜。
 本言是馬駒,今定成龍子。」其歌甚美,其旨甚微。海西公不男,使左右向龍與內侍接,生子以為己子。



 桓石民為荊州,鎮上明,民忽歌曰「黃曇子」。曲終又曰:「黃曇英,揚州大佛來上明。」頃之而石民死,王忱為荊州。「黃曇子」乃是王忱之字也。忱小字佛大,是「大佛來上明」也。



 太元末,京口謠曰:「黃雌雞,莫作雄父啼。一旦去毛衣,衣被拉颯棲。」尋王恭起兵誅王國寶,旋為劉牢之所敗也。



 司馬道子於東府造土山,名曰靈秀山。無幾而孫恩作
 亂,再踐會稽。會稽,道子所封。靈秀,恩之字也。庾楷鎮歷陽,民歌曰:「重羅犁,重羅犁,使君南上無還時。」後楷南奔桓玄,為玄所誅。殷仲堪在荊州,童謠曰:「芒籠目,繩縛腹。



 殷當敗,桓當復。」無幾而仲堪敗,桓玄有荊州。



 王恭鎮京口,舉兵誅王國寶。百姓謠云:「昔年食白飯,今年食麥麩。天公誅謫汝,教汝捻嚨喉。嚨喉喝復喝,京口敗復敗。」「昔年食白飯」,言得志也。



 「今年食麥麩」,麩,粗穢,其
 精已去,明將敗也,天公將加譴謫而誅之也。「捻嚨喉」,氣不通,死之祥也。「敗復敗」,丁寧之辭也。恭尋死,京都大行咳疾,而喉並喝焉。王恭在京口,民間忽云:「黃頭小人欲作賊,阿公在城下,指縛得。」



 又云:「黃頭小人欲作亂,賴得金刀作蕃捍。」「黃」字上,「恭」字頭也;「小人」,「恭」字下也。尋如謠者言焉。



 晉安帝隆安中,民忽作《懊惱歌》,其曲中有「草生可攬結,女兒可攬抱」之言。桓玄既篡居天位,義旗以三月二日
 掃定京都,玄之宮女及逆黨之家子女伎妾,悉為軍賞。東及甌、越,北流淮、泗,皆人有所獲焉。時則草可結,事則女可抱,信矣。



 桓玄既篡,童謠曰:「草生及馬腹,烏啄桓玄目。」及玄敗走至江陵,五月中誅,如其期焉。桓玄時,民謠語云:「征鐘落地桓迸走。」徵鐘,至穢之服;桓,四體之下稱。玄自下居上,猶徵鐘之廁歌謠,下體之詠民口也。而云「落地」,墜地之祥,迸走之言,其驗明矣。



 司馬元顯時,民謠詩云:「當有十一口,當為兵所傷。木亙當北度,走入浩浩鄉。」又云:「金刀既以刻,娓娓金城中。」此詩云襄陽道人竺曇林所作,多所道,行於世。孟顗釋之曰,「十一口」者,玄字象也;「木亙」,桓也。桓氏當悉走入關、洛,故云「浩浩鄉」也。「金刀」,劉也。倡義諸公,皆多姓劉。「娓娓」,美盛貌也。



 桓玄得志,童謠曰:「長干巷,巷長干。今年殺郎君,明年斬諸桓。」及玄走而諸桓悉誅焉。郎君,司馬元顯也。



 晉安帝義熙初,童謠曰:「官家養蘆化成荻,蘆生不止自成積。」其時官養盧龍,寵以金紫,奉以名州,養之已極,而不能懷我好音,舉兵內伐,遂成讎敵也。



 「蘆生不止自成積」,及盧龍作亂,時人追思童謠,惡其有成積之言。識者曰:「芟夷蘊崇之,又行火焉,是草之窮也。伐斫以成積,又以為薪,亦蘆荻之終也。



 其盛既極,亦將芟夷而為積焉。」龍既窮其兵勢,盛其舟艦,卒以滅亡,僵屍如積焉。



 盧龍據有廣州,民間謠云:「蘆生漫漫竟天半。」後擁有上
 流數州之地,內逼京輦,應「天半」之言。



 義熙三年中,小兒相逢於道,輒舉其兩手曰「盧健健」,次曰「斗嘆斗嘆」,末復曰「翁年老,翁年老」。當時莫知所謂。其後盧龍內逼,舟艦蓋川,「健健」



 之謂也。既至查浦,屢剋期欲與官斗,「斗嘆」之應也。「翁年老」,群公有期頤之慶,知妖逆之徒,自然消殄也。其時復有謠言曰:「盧橙橙,逐水流,東風忽如起,那得入石頭。」盧龍果敗,不得入石頭。昔溫嶠令郭景純卜己與庾亮吉凶。景純云「元吉」。嶠語亮:「景純每筮,當是不敢盡言。吾等與國家同安危而曰元吉,事有成也。」於是協同討滅
 王敦。



 苻堅中,童謠曰:「阿堅連牽三十年,後若欲敗時,當在江湖邊。」後堅敗於淝水,在偽位凡三十年。苻堅中,謠語云:「河水清復清,苻詔死新城。」堅為姚萇所殺,死於新城。苻堅中,歌云:「魚羊田斗當滅秦。」「魚羊」,鮮也;「田斗」,卑也。堅自號秦,言滅之者鮮卑也。其群臣諫堅,令盡誅鮮卑,堅
 不從。及淮南敗還,為慕容沖所攻,亡奔姚萇,身死國滅。



 毛蟲之孽:晉武帝太康六年,南陽送兩足虎,此毛蟲之孽也。識者為其文曰:「武形有虧,金虎失儀,聖主應天,斯異何為。」言非亂也。京房《易傳》曰:「足少者,下不勝任也。」干寶曰:「虎者陰精,而居于陽,金獸也。南陽,火名也。金精入火,而失其形,王室亂之妖也。六,水數,言水數既極,火慝得作,而金受其敗也。至元康九年,始殺太子,距此十四年。二七十
 四,火始終相乘之數也。自帝受命,至愍懷之廢,凡三十五年。」太康九年,荊州獻兩足玃。太康七年十一月丙辰,四角獸見于河間,河間王顒獲以獻。角,兵象也。董仲舒以四角為四方之象。後河間王數連四方之兵,作為亂階,殆其應也。



 晉懷帝永嘉五年,偃鼠出延陵,此毛蟲之孽也。郭景純筮之曰:「此郡東之縣,當有妖人欲稱制者,亦尋自死矣。」
 其後吳興徐馥作亂,殺太守袁琇,馥亦時滅,是其應也。



 晉成帝咸和六年正月丁巳,會州郡秀孝於樂賢堂,有躭見於前,獲之。孫盛曰:「夫秀孝,天下之彥士,樂賢堂,所以樂養賢也。晉自喪亂以後,風教凌夷,秀無策試之才,孝乏四行之寶。躭興於前,或斯故乎。」



 晉哀帝隆和元年十月甲申,有麈入東海第。百姓沄言曰:「主入東海第。」識者怪之。及海西廢為東海王,先送此
 第。



 晉孝武太元十三年四月癸巳,礿祠畢,有兔行廟堂上。兔,野物也,而集宗廟之堂,不祥莫甚焉。



 宋文帝元嘉二十四年二月,雍州送六足麞,刺史武陵王表為祥瑞。此毛蟲之孽。



 宋順帝升明元年,象三頭度蔡洲,暴稻穀及園野。



 犬禍:公孫淵家有犬冠幘絳衣上屋,此犬禍也。屋上亢陽高危之地。天戒若曰,淵亢陽無上,偷自尊高,狗而冠者也。
 及自立為燕王,果為魏所滅。京房《易傳》曰:「君不正,臣欲篡,厥妖狗出朝門。」



 魏侍中應璩在直廬,欻見一白狗,問眾人無見者。踰年卒。近犬禍也。



 諸葛恪征淮南歸,將朝會,犬銜引其衣。恪曰:「犬不欲我行乎?」還坐,有頃復起,犬又銜衣。乃令逐犬。遂升車入而被害。



 晉武帝太康九年,幽州有犬,鼻行地三百餘步。



 晉惠帝元康中,吳郡婁縣民家聞地中有犬聲,掘視得雌雄各一。還置窟中,覆以磨石,宿昔失所在。元帝太興中,吳郡府舍又得二物頭如此。其後太守張茂為吳興兵所殺。案《夏鼎志》曰:「掘地得狗名曰賈。」《尸子》曰:「地中有犬,名曰地狼。」同實而異名也。晉惠帝永興元年,丹陽內史朱逵家犬生三子,皆無頭。



 後逵為揚州刺史曹武所殺。



 晉孝懷帝永嘉五年,吳郡嘉興張林家狗人言云:「天下人
 餓死。」



 晉安帝隆安初,吳郡治下狗恒夜吠,聚高橋上。人家狗有限,而吠聲甚眾。或有夜出覘之者,云一狗假有兩三頭,皆前向亂吠。無幾,孫恩亂於吳會。



 桓玄將拜楚王,已設拜席,群官陪位,玄未及出,有狗來便其席,萬眾旺候,莫不驚怪。玄性猜暴,竟無言者,逐狗改席而已。



 宋武帝永初二年,京邑有狗人言。



 文帝元嘉二十九年,吳興東遷孟慧度婢蠻與狗通好如夫妻彌年。



 孝武孝建初,顏竣為左衛,於省內聞犬子聲在地中,掘焉得烏犬子。養久之,後自死。



 明帝初,晉安王子勛稱偽號於尋陽,柴桑有狗與女人交,三日不分離。明帝泰始中,秣陵張僧護家犬生豕子。



 白眚白祥:
 晉武帝太康十年,洛陽宮西宜秋里石生地中,始高三尺,如香金盧形,後如傴人,盤薄不可掘。案劉向說,此白眚也。明年,宮車晏駕,王室始騷,卒以亂亡。



 京房《易傳》曰:「石立如人,庶人為天下雄。」此近之矣。



 晉成帝咸康初,地生毛,近白眚也。孫盛以為民勞之異。是後胡滅而中原向化,將相皆甘心焉。於是方鎮屢革,邊戍仍遷,皆擁帶部曲,動有萬數。其間征伐征賦,役無寧歲,天下擾動,民以疲怨。
 咸康三年六月,地生毛。



 晉孝武太元二年五月,京都地生毛。至四年而氐賊攻襄陽,圍彭城,向廣陵,征戍仍出,兵連不解。太元十四年四月,京都地生毛。是時苻堅滅後,經略多事。



 太元十七年四月,地生毛。



 晉安帝隆安四年四月乙未,地生毛,或白或黑。晉安帝元興三年五月,江陵地生毛。是後江陵見襲,交
 戰者數矣。晉安帝義熙三年三月,地生白毛。義熙十年三月,地生白毛。明年,王旅西討司馬休之。又明年,北掃關、洛。



 魏明帝青龍三年正月乙亥,隕石于壽光。按《左氏傳》,隕石,星也。劉歆說曰:「庶民,惟星隕於宋者,象宋襄公將得諸侯而不終也。」秦始皇時有隕石,班固以為石陰類,又白祥,臣將危君。是後司馬氏得政。



 晉武帝太康五年五月丁巳,隕石于溫及河陽各二。太康六年正月,隕石于溫三。



 晉成帝咸和八年五月,星隕于肥鄉一。咸和九年正月,隕石於涼州。



 吳孫亮五鳳二年五月,陽羨縣離里山大石自立。按京房《易傳》曰:「庶士為天子之祥也。」其說曰:「石立於山,同姓;平地,異姓。」干寶以為孫皓承廢故之家得位,其應也;或曰孫休見立之祥也。



 晉惠帝元康五年十二月,有石生于宜年里。晉惠帝永康元年,襄陽郡上言得鳴石,撞之,聲聞七八里。晉惠帝太安元年,丹陽湖熟縣夏架湖有大石浮二百步而登岸。民驚噪相告曰:「石來!」干寶曰:「尋有石冰入建業。」



 晉武帝泰始八年五月,蜀地雨白毛。此白祥也。是時益州刺史皇甫晏冒暑伐汶山胡,從事何旅固諫,不從。牙
 門張弘等因眾之怨,誣晏謀逆,害之。京房《易傳》曰:「前樂後憂,厥妖天雨羽。」又曰:「邪人進,賢人逃,天雨毛。」其《易妖》曰:「天雨毛羽,貴人出走。」三占皆應也。



 晉惠帝永寧元年,齊王冏舉義軍。軍中有小兒出於襄城繁昌縣,年八歲,髮體悉白,頗能卜。於《洪範》,則白祥也。



 晉車騎大將軍東嬴王騰自并州遷鎮鄴,行次真定。時久積雪,而當門前方數尺獨消釋,騰怪而掘之,得玉馬高尺許,口齒缺。騰以馬者國姓,上送之以為瑞。然論者
 皆云馬而無齒,則不得食,妖祥之兆,衰亡之徵。案占,此白祥也。是後騰為汲桑所殺,而晉室遂亡。



 宋文帝元嘉中,徐湛之為丹陽尹。夜西門內有氣如練,西南指,長數十丈。又白光覆屋,良久而轉駃乃消。此白祥也。



 前廢帝景和元年,鄧琬在尋陽,種紫花皆白,白眚也。



 木沴金:魏齊王正始末,河南尹李勝治聽事,有小材激墮,楇受
 符石虎項斷之。此木沴金也。勝後旬日而敗。



 晉惠帝元康八年三月,郊禖壇石中破為二。此木沴金也。郊禖壇者,求子之神位,無故而自毀,太子將危之妖也。明年,愍懷廢死。



 晉孝武帝太元十年四月,謝安出鎮廣陵,始發石頭,金鼓無故自破。此木沴金之異也。天意若曰,安徒揚經略之聲,終無其實,鉦鼓不用之象也。八月,以疾還,是月薨。



\end{pinyinscope}