\article{卷八本紀第八 明帝}

\begin{pinyinscope}

 太
 宗明皇帝諱彧,字休炳,小字榮期,文帝第十一子也。元嘉十六年十月戊寅生。二十五年,封淮陽王,食邑二千戶。二十九年,改封湘東王。元凶弒立,以為驍騎將軍,
 加給事中。世祖踐阼,為秘書監,遷冠軍將軍、南蘭陵下邳二郡太守,領石頭戍事。孝建元年,徙為南彭城、東海二郡太守,將軍如故,鎮京口。其年,徵為中護軍。二年,遷侍中,領遊擊將軍。三年,徙衛尉,侍中如故。又為左衛將軍,衛尉如故。大明元年,轉中護軍,衛尉如故。三年,為都官尚書,領遊擊將軍,衛尉如故。七年,遷領軍將軍。八年,出為使持節、都督徐兗二州豫州之梁郡諸軍事、鎮北將軍、徐州刺史,給鼓吹一部。其年,徵為侍中、護軍將軍。
 未拜,復為領軍將軍,侍中如故。



 永光元年,又出為使持節、散騎常侍、都督南豫豫司江四州揚州之宣城諸軍事、衛將軍、南豫州刺史,鎮姑孰。又徙為都督雍梁南北秦四州郢州之竟陵諸軍事、寧蠻校尉、雍州刺史,持節、常侍、將軍如故。未拜,復本位。尋以本號開府儀同三司。



 廢帝景和末,上入朝,被留停都。廢帝誅害宰輔,殺戮大臣,恒慮有圖之者,疑畏諸父,並拘之殿內,遇上無禮,事在《文諸王傳》。遂收上付廷尉,一宿被原。



 將加禍害者,前
 後非一。既而害上意定,明旦便應就禍。上先已與腹心阮佃夫、李道兒等密共合謀。于時廢帝左右常慮禍及,人人有異志。唯有直皞將軍宋越、譚金、童太一等數人為其腹心,並虓虎有幹力,在殿省久,眾並畏服之,故莫敢動。是夕,越等並外宿。佃夫、道兒因結壽寂之等殞廢帝於後堂,十一月二十九日夜也。事定,上未知所為。建安王休仁便稱臣奉引升西堂,登御坐,召見諸大臣。于時事起倉卒,上失履,跣至西堂,猶著烏帽。坐定,休仁呼
 主衣以白帽代之,令備羽儀。雖未即位,凡眾事悉稱令書施行。己未,司徒揚州刺史豫章王子尚、山陰公主並賜死。宗越、譚金、童太一謀反伏誅。十二月庚申朔,令書以司空東海王禕為中書監、太尉,鎮軍將軍、江州刺史晉安王子勛進號車騎將軍、開府儀同三司。癸亥,以新除驃騎大將軍建安王休仁為司徒、尚書令、揚州刺史,鎮軍將軍、開府儀同三司山陽王休祐進號驃騎大將軍、荊州刺史。崇憲衛尉桂陽王休範為鎮北將軍、南徐
 州刺史。乙丑,改封安陸王子綏為江夏王。



 泰始元年冬十二月丙寅,上即皇帝位。詔曰:高祖武皇帝德洞四瀛,化綿九服。太祖文皇帝以大明定基;世祖孝武皇帝以下武寧亂。日月所照,梯山航海;風雨所均,削衽襲帶。所以業固盛漢,聲溢隆周。



 子業凶嚚自天,忍悖成性,人面獸心,見於齠日,反道敗德,著自比年。其狎侮五常,怠棄三正,矯誣上天,毒流下國,實開闢所未有,書契所未聞。再罹遏密,而無一日之哀;齊斬在躬,方深
 北里之樂。虎兕難匣,憑河必彰,遂誅滅上宰,窮釁逆之酷,虐害國輔,究孥戮之刑。子鸞同生,以昔憾殄殪;敬猷兄弟,以睚眥殲夷。



 征逼義陽,將加屠膾。陵辱戚籓,檟楚妃主。奪立左右,竊子置儲,肆酗于朝,宣淫於國。事穢東陵,行汙飛走。積釁罔極,日月滋深。比遂圖犯玄宮,志窺題湊,將肆梟、獍之禍,騁商、頓之心。又欲鴆毒崇憲,虐加諸父,事均宮閫,聲遍國都。



 鴟梟小豎,莫不寵暱,朝廷忠誠,必加戮挫。收掩之旨,虓虎結轍;掠奪之使,白刃相望。
 百僚危氣,首領無有全地;萬姓崩心,妻子不復相保。所以鬼哭山鳴,星鉤血降,神器殆於馭索,景祚危於綴旒。



 朕假寐凝憂,泣血待旦,慮大宋之基,於焉而泯,武、文之業,將墜于淵。賴七廟之靈,藉八百之慶,巨猾斯殄,鴻沴時褰。皇綱絕而復紐,天緯缺而更張。猥以寡薄,屬承乾統,上緝三光之重,俯顧庶民之艱。業業矜矜,若履冰谷,思與億兆,同此維新。可大赦天下,改景和元年為泰始元年。賜民爵二級。鰥寡孤獨不能自存者,人穀五斛。逋
 租宿債勿復收。犯鄉論清議,贓污淫盜,並悉洗除。長徒之身,特賜原遣。亡官失爵,禁錮舊勞,一依舊典。其昏制謬封,並皆刊削。



 己巳,以安西將軍、南豫州刺史劉遵考為特進、右光祿大夫,輔國將軍、歷陽南譙二郡太守建平王景素為南豫州刺史。庚午,以荊州刺史臨海王子頊為鎮軍將軍,南徐州刺史永嘉王子仁為中軍將軍,左衛將軍劉道隆為中護軍。辛未,改封臨賀王子產為南平王,晉熙王子輿為廬陵王。壬申,以尚書左僕射王景
 文為尚書僕射。新除中護軍劉道隆卒。壬午,詔曰:「朕戡亂寧民,屬膺景祚。鴻制初造,革道惟新。



 而國故頻罹,仁澤偏壅。每鑒寐疚心,罔識攸濟。巡方問俗,弘政所先,可分遣大使,廣求民瘼,考守宰之良,採衡閭之善。若獄犴淹枉,傷民害教者,具以事聞;鰥寡孤獨,癃殘六疾,不能自存者,郡縣優量賑給;貞婦孝子,高行力田,許悉條奏。務詢輿誦,廣納嘉謀,每盡皇華之旨,俾若朕親覽焉。」乙亥,追尊所生沈婕妤曰宣皇太后。後軍將軍垣閎為司
 州刺史,前右將軍長史殷琰為豫州刺史。丙子,詔曰:「皇室多故,糜費滋廣,且久歲不登,公私歉弊。方刻意從儉,弘濟時艱,政道未孚,慨愧兼積。大官供膳,可詳所減撤,尚方御府雕文篆刻無益之物,一皆蠲省,務存簡約,以稱朕心。」戊寅,崇太后為崇憲皇太后,立皇后王氏。鎮軍將軍、江州刺史晉安王子勛舉兵反,鎮軍長史鄧琬為其謀主,雍州刺史袁鳷率眾赴之。



 辛巳,驃騎大將軍、前荊州刺史山陽王休祐改為江州刺史,荊州刺史臨海
 王子頊即留本任。加領軍將軍王玄謨鎮軍將軍。壬午,車駕謁太廟。甲申,後將軍、郢州刺史安陸王子綏進號征南將軍,右將軍、會稽太守尋陽王子房進號安東將軍,前將軍、荊州刺史臨海王子頊進號平西將軍。子綏、子房、子頊並不受命,舉兵同逆。戊子,新除中軍將軍永嘉王子仁為護軍將軍。



 二年春正月己丑朔,以軍事不朝會。庚寅,以金紫光祿大夫王僧朗為左光祿大夫、開府儀同三司。壬辰,驃騎
 大將軍、江州刺史山陽王休祐改為南豫州刺史;鎮歷陽。鎮軍將軍、領軍將軍王玄謨為車騎將軍、江州刺史,平北將軍、徐州刺史薛安都進號安北將軍。安都亦不受命。癸巳,以左衛將軍巴陵王休若為鎮東將軍;新除安東將軍尋陽王子房為撫軍將軍;司徒左長史袁愍孫為領軍將軍。甲午,中外戒嚴。司徒建安王休仁都督征討諸軍事,統眾軍南討。以青州刺史劉祗為南兗州刺史。



 丙申,以征虜司馬申令孫為徐州刺史,義陽內史
 龐孟虯為司州刺史。令孫、孟虯及豫州刺史殷琰、青州刺史沈文秀、冀州刺史崔道固、湘州行事何慧文、廣州刺史袁曇遠、益州刺史蕭惠開、梁州刺史柳元怙並同叛逆。兗州刺史殷孝祖入衛京都,仍遣孝祖前鋒南伐。甲辰,加孝祖撫軍將軍。丙午,車駕親御六師,出頓中興堂。辛亥,驃騎大將軍、南豫州刺史山陽王休祐改為豫州刺史,統眾軍西討。吳郡太守顧琛、吳興太守王曇生、義興太守劉延熙、晉陵太守袁摽、山陽太守程天祚
 並舉兵反。



 鎮東將軍巴陵王休若統眾軍東討。壬子,崇憲皇太后崩。是日,軍主任農夫、劉懷珍平定義興。永世縣民史逸宗據縣為逆,殿中將軍陸攸之討平之。丙辰,以新除左光祿大夫、開府儀同三司王僧朗為特進,左光祿大夫如故。二月乙丑,僧朗卒。尚書僕射王文景父憂去職。曲赦吳、吳興、義興、晉陵四郡。吏部尚書蔡興宗為尚書左僕射,吳興太守張永、右軍將軍齊王東討,平晉陵。癸未,曲赦浙江東五郡。丁亥,鎮東將軍巴陵王休
 若進號衛將軍。建武將軍吳喜公率諸軍破賊於吳、吳興、會稽,平定三郡,同逆皆伏誅。輔國將軍齊王前鋒北討,輔國將軍劉緬前鋒南討。賊劉胡領眾四萬據赭圻。三月庚寅,撫軍將軍沈攸之代為南討前鋒。賊眾稍盛,袁鳷頓鵲尾,聯營迄至濃湖,眾十餘萬。壬辰,以新除太子詹事張永為青、冀二州刺史。



 丙申,鎮北將軍、南徐州刺史桂陽王休範總統北討諸軍事。丁酉,以尚書劉思考為徐州刺史。戊戌,貶尋陽王子房爵為松滋縣侯。乙
 巳,以奉朝請鄭黑為司州刺史。



 辛亥,鎮北將軍、南徐州刺史桂陽王休範領南兗州刺史。壬子,斷新錢,專用古錢。



 癸丑,原赦揚、南徐二州囚系,凡逋亡一無所問。夏四月壬午,以散騎侍郎明僧暠為青州刺史。五月壬辰,以輔國將軍沈攸之為雍州刺史。丁酉,曲赦豫州。丁未,新除尚書僕射王景文為中軍將軍,以青、冀二州刺史張永為鎮軍將軍。庚戌,以寧朔將軍劉乘民為冀州刺史。甲寅,葬崇憲皇太后於攸寧陵。冠軍將軍、益州刺史蕭
 惠開進號平西將軍。六月辛酉,鎮軍將軍張永領徐州刺史。京師雨水,丁卯,遣殿中將軍檢行賜恤。以左軍將軍垣恭祖為梁、南秦二州刺史。秋七月己丑,鎮北將軍、南徐兗二州刺史桂陽王休範進號征北大將軍。辛卯,鎮軍將軍、徐州刺史張永改為南兗州刺史。丁酉,以仇池太守楊僧嗣為北秦州刺史、武都王。壬寅,以男子時朗之為北豫州刺史。乙巳,龍驤將軍劉道符平山陽。辛亥,又以義軍主鄭叔舉為北豫州刺史,鎮軍將軍、南兗
 州刺史張永復領徐州刺史。甲寅,復以冀州刺史崔道固為徐州刺史。八月己卯,司徒建安王休仁率眾軍大破賊,斬偽尚書修射袁鳷,進討江、郢、荊、雍、湘五州,平定之。晉安王子勛、安陸王子綏、臨海王子頊、邵陵王子元並賜死;同黨皆伏誅。諸將軍帥封賞各有差。甲申,以護軍將軍、永嘉王子仁為平南將軍、湘州刺史。九月乙酉,曲赦江、郢、荊、雍、湘五州;守宰不得離職。



 壬辰,驃騎大將軍、豫州刺史山陽王休祐改為荊州刺史。分豫州立南
 豫州。癸巳,六軍解嚴。大赦天下,賜民爵一級。甲午,以中軍將軍王景文為安南將軍、江州刺史。戊戌,以車騎將軍、江州刺史王玄謨為左光祿大夫、開府儀同三司、護軍將軍。



 庚子,以建安王休仁世子伯融為豫州刺史。辛丑,衛將軍巴陵王休若即本號為雍州刺史。雍州刺史沈攸之為郢州刺史。庚戌,以太子左衛率建平王景素為南兗州刺史。



 十月乙卯,永嘉王子仁、始安王子真、淮南王子孟、南平王子產、廬陵王子輿、松滋侯子房並賜
 死。丁卯,以郢州刺史沈攸之為中領軍,與張永俱北討。庚午,以吳郡太守顧覬之為湘州刺史。戊寅,立皇子昱為皇太子。曲赦揚、南徐二州。以輔國將軍劉勔為廣州刺史,左軍將軍張世為豫州刺史。十一月甲申,以安成太守劉襲為郢州刺史。壬辰,詔曰:「治崇簡易,化疾繁侈,遠關隆替,明著軌跡者也。朕拯斯墜運,屬此屯極,仍之以凋耗,因之以師旅,而識昧前王,務艱昔代。俾夫舊賦既繁,為費彌廣,鑒寐萬務,每思弘革。方欲緩徭優調,愛民
 為先,有司詳加寬惠,更立科品。其方物職貢,各順土宜,出獻納貢,敬依時令。凡諸蠹俗妨民之事,趣末違本之業,雕繪靡麗,奇器異技,並嚴加裁斷,務歸要實。左右尚方御府諸署,供御制造,咸存儉約。庶淳風至教,微遵太古,阜財興讓,少敦季俗。」又詔曰:「夫秉機詢政,立教之攸本;舉賢聘逸,弘化之所基。故負鼎進策,殷代以康;釋釣作輔,周祚斯乂。朕甫承大業,訓道未敷,雖側席忠規,佇夢巖築,而良圖莫薦,奇士弗聞,永鑒通古,無忘宵寐。今籓
 隅克晏,敷化維始,屢懷存治,實望箴闕。



 王公卿尹,群僚庶官,其有嘉謀直獻,匡俗濟時,咸切事陳奏,無或依隱。若乃林澤貞棲,丘園耿潔,博洽古今,敦崇孝讓,四方大任,可明書搜揚,具即以聞,隨就褒立。」以建平王景素子延年為新安王。以新除左光祿大夫、開府儀同三司王玄謨為車騎將軍、南豫州刺史。丙申,制使東土經荒流散,並各還本,蠲眾調二年。



 十二月己未,以尚書金部郎劉善明為冀州刺史。乙丑,詔曰:「近眾籓稱亂,多染釁科。
 或誠係本朝,事緣逼迫,混同證錮,良以悵然。夫天道尚仁,德刑並用,雷霆時至,雲雨必解。朕眷言靜念,思弘風澤,凡應禁削,皆從原蕩。其文武堪能,隨才銓用。」辛未,以新除廣州刺史劉勔為益州刺史,前巴西、梓潼二郡太守費混為廣州刺史。劉勔克壽陽,豫州平。辛巳,以輔國將軍劉靈遺為梁、南秦二州刺史。



 薛安都要引索虜,張永、沈攸之大敗,於是遂失淮北四州及豫州淮西地。



 三年春正月庚子,以農役將興,太官停宰牛。癸卯,曲赦
 豫、南豫二州。衛將軍巴陵王休若降號鎮西將軍。閏月庚午,京師大雨雪,遣使巡行,賑賜各有差。戊寅,以遊擊將軍垣閬為益州刺史。二月甲申,以御史中丞羊南為廣州刺史。是日,車駕為戰亡將士舉哀。己丑,以鎮西司馬劉亮為梁、南秦二州刺史。索虜寇汝陰,太守張景遠擊破之。丙申,曲赦青、冀二州。三月丙子,以尚書左僕射蔡興宗為安西將軍、郢州刺史。戊寅,以冠軍將軍王玄載為徐州刺史,寧朔將軍崔平為兗州刺史。夏四月癸
 巳,以前司州刺史鄭黑為司州刺史。乙未,冠軍將軍、北秦州刺史楊僧嗣進號征西將軍。庚子,立桂陽王休範第二子德嗣為廬陵王,立侍中劉韞第二子銑為南豐王。丙午,安西將軍蔡興宗降號平西將軍。五月丙辰,宣太后崇寧陵禁內墳屋瘞遷徙者,給葬直,蠲復家丁。戊午,以車騎將軍、南豫州刺史王玄謨為左光祿大夫、開府儀同三司。辛酉,罷南豫州并豫州。壬戌,以太子詹事袁粲為尚書僕射。六月乙酉,以侍中劉韞為湘州刺史。
 秋七月壬子,以左光祿大夫、開府儀同三司王玄謨為特進、左光祿大夫、護軍將軍。薛安都子伯令略據雍州四郡,刺史巴陵王休若討斬之。八月丁酉,詔曰:「古者衡虞置制,蝝蚳不收;川澤產育,登器進御。所以繁阜民財,養遂生德。頃商販逐末,競早爭新。折未實之果,收豪家之利,籠非膳之翼,為戲童之資。豈所以還風尚本,捐華務實。宜修道布仁,以革斯蠹。



 自今鱗介羽毛,肴核眾品,非時月可採,器味所須,可一皆禁斷,嚴為科制。」壬寅,以
 中領軍沈攸之行南兗州刺史,率眾北討。癸卯,詔曰:「法網之用,期世而行,寬惠之道,因時而布。況朕尚德戡亂,依仁馭俗,宜每就弘簡,以隆至治。而頻罹兵革,徭賦未休,軍民巧偽,興事甚多。蹈刑入憲,諒非一科。至乃假名戎伍,竊爵私庭,因戰散亡,託懼逃役。且往諸淪逼,雖經累宥,逋竄之黨,猶為實繁。



 宵言永懷。良兼矜疚。思所以重播至澤,覃被區宇。可大赦天下。」加新除左光祿大夫王玄謨車騎將軍。丙午,遣吏部尚書褚淵慰勞緣淮將
 帥,隨宜量賜。戊申,以新除右衛將軍劉勔為豫州刺史。九月癸丑,鎮西將軍、雍州刺史巴陵王休若進號衛將軍,平西將軍、郢州刺史蔡興宗進號安西將軍。乙卯,以越騎校尉周寧民為兗州刺史。戊午,以皇后六宮以下雜衣千領,金釵千枚,班賜北征將士。庚申,前將軍兼冀州刺史崔道固進號平北將軍。甲子,曲赦徐、兗、青、冀四州。冬十月壬午,改封新安王延年為始平王。戊子,芮芮國遣使獻方物。辛丑,復郡縣公田。鎮西大將軍、西秦河
 二州刺史吐谷渾拾寅進號征西大將軍。十一月,立建安王休仁第二子伯猷為江夏王,改封義陽王昶為晉熙王。乙卯,分徐州置東徐州,以輔國將軍張讜為刺史。高麗國、百濟國遣使獻方物。十二月庚辰,以寧朔將軍劉休賓為兗州刺史。



 四年春正月己未,車駕親祠南郊,大赦天下。庚午,衛將軍巴陵王休若降號左將軍。乙亥,零陵王司馬勖薨。二月辛丑,以前龍驤將軍常珍奇為平北將軍、司州刺史,
 珍奇子超越為北冀州刺史。乙巳,右光祿大夫、車騎將軍、護軍將軍王玄謨薨。三月乙未,以游擊將軍劉懷珍為東徐州刺史。戊辰,以軍司馬劉靈遺為梁、南秦二州刺史,南譙太守孫奉伯為交州刺史。交州人李長仁據州叛,妖賊攻廣州,殺刺史羊南,龍驤將軍陳伯紹討平之。夏四月己卯,復減郡縣田租之半。東海王禕改封廬江王,山陽王休祐改封晉平王,改晉安郡為晉平郡。辛丑,芮芮國及河南王並遣使獻方物。甲辰,以豫章太守
 張辯為廣州刺史。五月乙未,曲赦廣州。癸亥,以行雍州刺史巴陵王休若行湘州刺史。會稽太守張永為雍州刺史,湘州刺史劉韞為南兗州刺史。秋七月乙巳朔,以吳郡太守王琨為中領軍。丙辰,始平王延年薨。己未,以侍中劉襲為中護軍。庚申,以驍騎將軍齊王為南兗州刺史。八月戊子,以南康相劉勃為交州刺史。辛卯,分青州置東青州,以輔國將軍沈文靖為東青州刺史。丁酉,安南將軍、江州刺史王景文進號鎮南將軍。九月丙辰,
 以驃騎長史張悅為雍州刺史。



 戊辰,詔曰:「夫愆有小大,憲隨寬猛,故五刑殊用,三典異施。而降辟次網,便暨鉗撻,求之法科,差品滋遠。朕務存欽恤,每有矜貸。尋劫制科罪,輕重同之大辟,即事原情,未為詳衷。自今凡竊執官仗,拒戰邏司,或攻剽亭寺,及害吏民者,凡此諸條,悉依舊制。五人以下相逼奪者,可特賜黥刖,投畀四遠,仍用代殺,方古為優,全命長戶,施同造物。庶簡惠之化,有孚群萌,好生之德,無漏幽品。」



 庚午,曲赦揚、南徐、兗、豫四
 州。冬十月癸酉朔,日有蝕之。發諸州兵北討。南康、建安、安成、宣城四郡,昔不同南逆,並不在徵發之例。甲戌,割揚州之義興郡屬南徐州。



 五年春正月癸亥,車駕躬耕藉田。大赦天下,賜力田爵一級。二月丙申,分豫州、揚州為南豫州。以太尉廬江王禕為車騎將軍、開府儀同三司、南豫州刺史。三月乙卯,於南豫州立南義陽郡。丙寅,車駕幸中堂聽訟。己巳,河南王遣使獻方物。



 夏四月辛未,割雍州隨郡屬郢州。乙
 酉,割豫州義陽郡屬郢州,郢州西陽郡屬豫州。



 戊子,以寧朔將軍崔公烈為兗州刺史。戊戌,新除給事黃門侍郎杜幼文為梁、南秦二州刺史。六月辛未,晉平王休祐子宣曜為南平王。壬申,以安西將軍、郢州刺史蔡興宗為鎮東將軍。癸酉,以左衛將軍沈攸之為郢州刺史。以軍興已來,百官斷俸,並給生食。丁丑,車騎將軍、南豫州刺史廬江王禕免官爵。戊寅,以左將軍、行湘州刺史巴陵王休若為征南將軍、湘州刺史。壬午,罷南豫州。丙戌,
 以新除給事黃門侍郎劉亮為益州刺史。秋七月己酉,以輔國將軍王亮為徐州刺史,東莞太守陳伯紹為交州刺史。甲寅,以山陽太守李靈謙為兗州刺史。壬戌,改輔國將軍為輔師將軍。八月己丑,以右將軍行豫州刺史劉勔為平西將軍、豫州刺史。壬辰,以海陵太守劉崇智為冀州刺史。九月甲寅,立長沙王纂子延之為始平王。戊午,中領軍王琨遷職。己未,詔曰:「夫箕、潁之操,振古所貴,沖素之風,哲王攸重。朕屬橫流之會,接難晦之辰,
 龕暴剪亂,日不暇給。今雖關、隴猶靄,區縣澄氛,偃武修文,於是乎在。思崇廉恥,用靜馳薄,固已物色載懷,寢興佇歎。其有貞栖隱約,自事衡樊,鑿壞遺榮,負釣辭聘,志恬江海,行高塵俗者,在所精加搜括,時以名聞。



 將賁園矜德,茂昭厥禮。群司各舉所知,以時授爵。」乙丑,以新除平西將軍、豫州刺史劉勔為中領軍。冬十月丁卯朔,日有蝕之。十一月丁未,索虜遣使獻方物。



 閏月戊子,驃騎大將軍、荊州刺史晉平王休祐以本號為南徐州刺史,
 征南將軍、湘州刺史巴陵王休若為征西將軍、荊州刺史,輔師將軍孟陽為兗州刺史,義陽太守呂安國為司州刺史。十二月戊戌,司徒建安王休仁解揚州刺史。己未,以征北大將軍、南徐州刺史桂陽王休範為中書監、中軍將軍、揚州刺史,吳興太守建平王景素為湘州刺史,輔師將軍建安王世子融為廣州刺史。庚申,分荊、益州五郡置三巴校尉。



 六年春正月乙亥,初制間二年一祭南郊,間一年一祭
 明堂。二月壬寅,司徒建安王休仁為太尉,領司徒。癸丑,皇太子納妃。甲寅,大赦天下,巧注從軍,不在赦例。班賜各有差。三月乙亥,中護軍劉襲卒。丁丑,以太子詹事張永為護軍將軍。



 夏四月癸亥,立第六皇子燮為晉熙王。五月丁丑,以前軍將軍陳胤宗為徐州刺史。



 丁亥,以冠軍將軍吐谷渾拾虔為平西將軍。戊子,奉朝請孔玉為寧州刺史。六月己亥,以第五皇子智井繼東平沖王休倩。庚子,以侍中劉韞為撫軍將軍、雍州刺史,前將軍、郢
 州刺史沈攸之進號鎮軍將軍,揚州刺史桂陽王休範為征南大將軍、江州刺史。癸卯,以鎮南將軍、江州刺史王景文為尚書左僕射、揚州刺史,尚書僕射袁粲為尚書右僕射。己未,改臨賀郡為臨慶郡,追改東平王休倩為臨慶沖王。七月丙戌,第五皇子智井薨。九月乙丑,中領軍劉勔加平北將軍。戊寅,立總明觀,徵學士以充之。置東觀祭酒。癸未,以第八皇子智渙繼臨慶沖王休倩。冬十月辛卯,立第九皇子贊為武陵王。乙巳,以前右軍
 馬詵為北雍州刺史。己酉,車駕幸東堂聽訟。



 十一月己巳,高麗國遣使獻方物。十二月癸巳,以邊難未息,制父母陷異域,悉使婚宦。戊戌,以始興郡為宋安郡。丙辰,護軍將軍張永遷職。



 七年春正月甲戌,置散騎奏舉郎。二月癸巳,征南將軍、荊州刺史巴陵王休若進號征西大將軍、開府儀同三司。戊戌,置百梁、隴蘇、永寧、安昌、富昌、南流郡,又分廣、交州三郡,合九郡,立越州。己亥,以前將軍劉康為平東
 將軍。妖寇宋逸攻合肥,殺汝陰太守王穆之,郡縣討平之。甲寅,驃騎大將軍、開府儀同三司、南徐州刺史晉平王休祐薨。戊午,以征西大將軍、荊州刺史巴陵王休若為征北大將軍、南徐州刺史,湘州刺史建平王景素為荊州刺史。三月辛酉,索虜遣使獻方物。



 壬戌,芮芮國遣使奉獻。夏四月辛丑,減天下死罪一等,凡敕繫悉遣之。甲辰,於南兗州置新平郡。癸丑,金紫光祿大夫張永領護軍。五月戊午,司徒建安王休仁有罪,自殺。辛酉,以寧
 朔長史孫超之為廣州刺史,尚書左僕射、揚州刺史王景文以刺史領中書監。庚午,以尚書右僕射袁粲為尚書令,新除吏部尚書褚淵為尚書左僕射。辛未,監吳郡王僧虔行湘州刺史。丙戌,追免晉平王休祐為庶人。六月丁酉,以征南大將軍、江州刺史桂陽王休範為驃騎大將軍、南徐州刺史,征北大將軍巴陵王休若為車騎大將軍、江州刺史。甲辰,芮芮國遣使獻方物。秋七月丁巳,罷散騎奏舉郎。乙丑,新除車騎大將軍、江州刺史巴
 陵王休若薨。桂陽王休範以新除驃騎大將軍,還為江州。庚午,以第三皇子準為撫軍將軍。辛未,以太子詹事劉秉為南徐州刺史。戊寅,以寧朔將軍沈懷明為南兗州刺史。乙酉,於冀州置西海郡。八月戊子,第八皇子躋繼江夏文獻王義恭。庚寅,以疾愈,大赦天下。冀州刺史劉崇智加青州刺史。戊戌,立第三皇子準為安成王。九月辛未,以越騎校尉周寧民為徐州刺史。冬十一月戊午,百濟國遣使獻方物。十二月丁酉,分豫州、南兗州立南
 豫州,以歷陽太守王玄載為南豫州刺史。



 泰豫元年春正月甲寅朔,上有疾不朝會。以疾患未痊,故改元。賜孤老貧疾粟帛各有差。戊午,皇太子會萬國於東宮,並受貢計。二月辛丑,以給事黃門侍郎王瞻為司州刺史。三月癸丑朔,林邑國遣使獻方物。己未,中書監、揚州刺史王景文卒。夏四月辛卯,以撫軍司馬蔡那為益州刺史。癸巳,以右衛將軍張興為雍州刺史。



 己亥,上大漸。驃騎大將軍、江州刺史桂陽王休範進位司空,
 尚書右僕射褚淵為護軍將軍,中領軍劉勔加尚書右僕射,鎮東將軍蔡興宗為征西將軍、開府儀同三司、荊州刺史,鎮軍將軍、郢州刺史沈攸之進號安西將軍。詔曰:「朕自臨御億兆,仍屬戎寇,雖每存弘化,而惠弗覃遠,軍國凋弊,刑訟未息。今大漸維危,載深矜歎,可緩徭優調,去繁就約。因改之宜,詳有簡衷。務以愛民為先,以宣朕遺意。」袁粲、褚淵、劉勔、蔡興宗、沈攸之同被顧命。是日,上崩于景福殿,時年三十四。



 五月戊寅,葬臨沂縣莫府
 山高寧陵。



 帝少而和令,風姿端雅。早失所生,養於太后宮內。大明世,諸弟多被猜忌,唯上見親,常侍路太后醫藥。好讀書,愛文義,在籓時,撰《江左以來文章志》,又續衛瓘所注《論語》二卷,行於世。及即大位,四方反叛,以寬仁待物。諸軍帥有父兄子弟同逆者,並授以禁兵,委任不易,故眾為之用,莫不盡力。平定天下,逆黨多被全;其有才能者,並見授用,有如舊臣。才學之士,多蒙引進,參侍文籍,應對左右。於華林園芳堂講《周易》,常自臨聽。末年
 好鬼神,多忌諱,言語文書,有禍敗凶喪及疑似之言應回避者,數百千品,有犯必加罪戮。改「騧」為馬邊瓜,亦以「騧」字似「禍」字故也。以南苑借張永,云「且給三百年,期訖更啟」。其事類皆如此。宣陽門,民間謂之白門,上以白門之名不祥,甚諱之。尚書右丞江謐嘗誤犯,上變色曰:「白汝家門!」謐稽顙謝,久之方釋。太后停屍漆床先出東宮,上嘗幸宮,見之怒甚,免中庶子官,職局以之坐者數十人。內外常慮犯觸,人不自保。宮內禁忌尤甚,移床治壁,必
 先祭土神,使文士為文詞祝策,如大祭饗。泰始、泰豫之際,更忍虐好殺,左右失旨忤意,往往有斮刳斷截者。時經略淮、泗,軍旅不息,荒弊積久,府藏空竭。內外百官,並日料祿俸;而上奢費過度,務為雕侈。



 每所造制,必為正御三十副,御次、副又各三十,須一物輒造九十枚,天下騷然,民不堪命。其餘事跡,別見眾篇。親近讒慝,剪落皇枝,宋氏之業,自此衰矣。



 史臣曰:聖人立法垂制,所以必稱先王,蓋由遺訓餘風,
 足以貽之來世也。太祖負扆南面,實有君人之懿焉,經國之義雖弘,而隆家之道不足。彭城王照不窺古,本無卓爾之資,徒見昆弟之義,未識君臣之禮,冀以此家情,行之國道,主猜而猶犯,恩薄而未悟,致以呵訓之微行,遂成滅親之大禍。開端樹隙,垂之後人。雖天倫之重,義殊凡戚,而中人以下,情由恩變。至於易衣而出,分苦而食,與夫別宮異門,形疏事隔者,宜有降矣。太宗因易隙之情,據已行之典,剪落洪枝,願不待慮。既而本根無庇,
 幼主孤立,神器以勢弱傾移,靈命隨樂推回改。斯蓋履霜有漸,堅冰自至,所從來遠也。



\end{pinyinscope}