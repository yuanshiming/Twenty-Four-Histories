\article{卷四十一列傳第一 后妃}

\begin{pinyinscope}

 帝祖母號
 太皇太后,母號皇太后,妃號皇后,漢舊制也。晉武帝採漢、魏之制,置貴嬪、夫人、貴人,是為三夫人,位視三公。淑妃、淑媛、淑儀、修華、修容、修儀、婕妤、容華、充華,是為九嬪,位視九卿。其餘有美人、才人、中才人,爵視千石以下。高祖受命,省二才人,其餘仍用晉制。貴
 嬪,魏文帝所制。夫人,魏武帝初建魏國所制。貴人,漢光武所
 製。
 淑妃,魏明帝所制。淑媛,魏文帝所制。



 淑儀、修華,晉武帝所制。修容,魏文帝所制。修儀,魏明帝所制。婕妤、容華,前漢舊號。充華,晉武帝所制。美人,漢光武所制。世祖孝建三年,省夫人、修華、修容,置貴妃,位比相國;進貴嬪,位比丞相;貴人,位比三司,以為三夫人。又置昭儀、昭容、昭華,以
 代修華、修儀、修容。又置中才人、充衣,以為散位。昭儀,漢元帝所制。昭容,世祖所制。昭華,魏明帝所制。中才人,晉武帝所制。充衣,前漢舊制。太宗泰始元年,省淑妃、昭華、中才人、充衣,復置修華、修儀、修容、才人、良人。三年,又省貴人,置貴姬,以備三夫人之數。又置昭華,增淑容、承徽、列榮。以淑媛、淑儀、淑容、昭華、昭儀、昭容、修華、修儀、修容為九嬪。婕妤、容華、充華、承徽、列榮凡五職,班亞九嬪。美人、中才人、才人三職為散役。其後太宗留心後房,擬外百官,備
 位置內職。列其名品于後。


後宮通尹,準錄尚書,紫極戶主,光興戶主。官品第一
 \gezhu{
  各置一人,並銓六宮}
 。



 後宮列敘,準尚書令,銓六宮。紫極中監尹,銓六宮。光興中監尹,銓六宮。


宣融戶主,銓六宮。
 紫極房帥,置一人。光興房帥,置一人。官品第二
 \gezhu{
  各置一人}
 。


後宮司儀,準左僕射,銓人士。後宮司政,準右僕射,銓人士。參議女林,準銀青光祿,銓人士。中臺侍御尹,銓六宮。宣融便殿中監尹,銓六宮。
 採藝房主,銓六宮。南房主,銓六宮。中藏女典,銓六宮。典坊,銓六宮。樂正,銓六宮。內保,銓人士。學林祭酒,銓人士。昭陽房帥,置一人。
 徽音房帥,置一人。宣融房帥,置一人。官品第三
 \gezhu{
  各置一人}
 。


後宮都掌治職,置二人
 \gezhu{
  準左右丞,位比尚書,銓人士}
 。後宮殿中治職,置一人
 \gezhu{
  準左民尚書,銓人士}
 。後宮源典治職,置一人
 \gezhu{
  準祠部尚書,銓人士}
 。


後宮穀帛治職,置一人
 \gezhu{
  準度支尚書}
 。中傅,置一人
 \gezhu{
  銓人士}
 。
 後宮校事女史,置一人
 \gezhu{
  銓人士}
 。紫極中監女史,置一人
 \gezhu{
  銓人士}
 。光興中監女史,置一人
 \gezhu{
  銓人士}
 。紫極房參事,置人無定數
 \gezhu{
  銓人士。有限外}
 。宣融房參事,置人無定數
 \gezhu{
  銓人士。有限外}
 。中臺侍御奏案女史,置一人
 \gezhu{
  銓人士}
 。贊樂女史,置一人
 \gezhu{
  銓人士}
 。中訓女史,置一人
 \gezhu{
  銓人士}
 。
 女祝史,置一人。紫極中監典,置一人。光興中監典,置一人。典樂帥,置人無定數
 \gezhu{
  有限外}
 。紫極房廉帥祭酒,置一人。光興房廉帥祭酒,置一人。宣融房廉帥祭酒,置一人。官品第四。


後宮通關參事,置一人。景德房參事,置人無定數
 \gezhu{
  銓人士}
 。采藝房參事。


置人無定數
 \gezhu{
  銓人士}
 。南房參事,置人無定數
 \gezhu{
  銓人士}
 。內房參事,置一人
 \gezhu{
  銓人士}
 。校學女史,置一人
 \gezhu{
  銓人士}
 。後宮中房帥,置二人。後宮源典帥,置二人。
 後宮穀帛帥,置二人。中臺帥,置一人。中臺侍御起居帥,置二人。中臺侍御詔誥帥,置二人。斯男房帥,置一人。宣豫房帥,置一人。景德房帥,置一人。


采藝房帥,置一人。
 中藏帥,置一人。內坊帥,置一人。南房帥,置一人。外華房帥,置一人。招慶房帥,置一人。紫極諸房廉帥,置人無定數
 \gezhu{
  有限外}
 。紫極中監省帥,置一人。紫極殿帥,置六人。
 光興殿帥,置四人。徽音監帥,置一人。徽章監帥,置一人。宣融便殿中監典,置一人。清商帥,置人無定數。總章帥,置人無定數。左西章帥,置人無定數。右西章帥,置人無定數。
 中廚師,置一人。官品第五。


中臺侍御執衛,置人無定數。中臺侍御監閨帥,置二人。中臺侍御監司帥,置二人。宣融便殿帥,置一人。永巷帥,置一人。後宮都掌內史,置二人。
 後宮殿中內史,置一人。後宮源典內史,置一人。後宮穀帛內史,置二人。後宮監臨內史,置二人。中臺侍御執法內史,置一人。中臺侍御典內史,置二人。中臺侍御節度內史,置二人。中臺侍御應內史,置六人。
 紫極房內史,置一人。光興房內史,置一人。助教,置一人。彩製帥,置人無定數。裝飾帥,置人無定數。繡帥,置人無定數。織帥,置人無定數。學林館帥,置一人。
 宮閨帥,置一人。教堂帥,置人無定數,
 \gezhu{
  有限外}
 。監解帥,置人無定數。累室帥,置人無定數。行病帥,置人無定數。官品第六。



 合堂帥,置二人。御清帥,置一人。
 監夜帥,置一人。諸房禁防,置人無定數。



 三廂禁防,置三人。諸房廚帥,各置一人。中廚廉,置三人。應閨,置六人。諸應閣,置人無定數。宮閨史,置一人。
 官品第七。



 諸房中掾,各置一人。中藏掾,各置二人。比五品敕吏。



 紫極供殿直倀。光興供殿直倀。總章伎倀。侍御扶持。
 主衣。準二衛五品,敕吏比六品。


供殿左右。
 \gezhu{
  紫極置二十人。光興置十人。}



 左右守藏,置四人。



 典樂人。比諸房禁防。



 作倀。比王官。


供殿給使。
 \gezhu{
  紫極置二十人。光興置十人}
 。



 典殿,置人無定數。比官人。



 紫極三廂給事,置十人。



 全堂給使,置五人。



 宮閨給使,置六人。比房。



 孝穆趙皇后,諱安宗,下邳僮人也。祖彪,字世範,治書侍御
 史。父裔,字彥胄,平原太守。后以晉穆帝升平四年嬪孝皇,晉哀帝興寧元年四月二日生高祖。其日,后以產疾殂於丹徒官舍,時年二十一。葬晉陵丹徒縣東鄉練璧里雩山。宋初追崇號謚,陵曰興寧。



 永初二年,有司奏曰:「大孝之德,盛於榮親。一人有慶,光被萬國。是以靈文寵於西京,壽張顯於隆漢。故平原太守趙裔、故洮陽令蕭卓,並外屬尊戚,不逮休寵。臣等仰述聖思,遠稽舊章,並可追贈光祿大夫,加金章紫綏。裔命婦孫可豫章郡建
 昌縣君,卓命婦趙可吳郡壽昌縣君。」孫氏,東莞人也。其年,又詔曰:「推恩之禮,在情所同。故內樹宗子,外崇后屬,爰自漢、魏,咸遵斯典。外祖趙光祿、蕭光祿,名器雖隆,茅土未建,並宜追封開國縣侯,食邑五百戶。」於是追封裔臨賀縣侯。裔長子宣之,仕至江乘令。蚤卒,無子,以弟孫襲之繼宣之紹封。



 襲之卒,子祖憐嗣。齊受禪,國除。宣之弟倫之,自有傳。



 孝懿蕭皇后,諱文壽,蘭陵蘭陵人也。祖亮,字保祚,侍御
 史。父卓,字子略,洮陽令。孝穆后殂,孝皇帝娉后為繼室,生長沙景王道憐、臨川烈武王道規。義熙七年,拜豫章公太夫人。高祖為宋王,又加太妃之號。高祖以十二年北伐,仍停彭城、壽陽,至元熙二年入朝,因受晉禪;在外凡五年,后常留東府。高祖踐阼,有司奏曰:「臣聞道積者慶流,德洽者禮備。故祗敬表於崇高,嘉號彰於盛典。伏惟太妃母儀之德,化穆不言,保翼之訓,光被洪業。雖幽明同慶,而稱謂未窮。稽之前代,禮有恒準,宜式遵舊章,
 允副群望。臣等請上宋王太后號皇太后。」故有司奏猶稱太妃也。



 上以恭孝為行,奉太后素謹,及即大位,春秋已高,每旦入朝太后,未嘗失時刻。



 少帝即位,加崇曰太皇太后。景平元年,崩於顯陽殿,時年八十一。遺令曰:「孝皇背世五十餘年,古不祔葬。且漢世帝后陵皆異處,今可於塋域之內,別為一壙。孝皇陵墳本用素門之禮,與王者制度奢儉不同,婦人禮有所從,可一遵往式。」



 乃開別壙,與興寧陵合墳。初,高祖微時,貧約過甚。孝皇之殂,葬禮多
 闕;高祖遺旨,太后百歲後不須祔葬。至是故稱后遺旨施行。



 卓,初與趙裔俱贈金紫光祿大夫,又追封封陽縣侯,妻下邳趙氏封吳郡壽昌縣君。卓子源之襲爵,源之見子《思話傳》。



 武敬臧皇后,諱愛親,東莞人也。祖汪,字山甫,尚書郎。父俊,字宣乂,郡功曹。后適高祖,生會稽宣長公主興弟。高祖以儉正率下,后恭謹不違。及高祖興復晉室,居上相之重,而后器服粗素,不為親屬請謁。義熙四年正月甲
 午,殂於東城,時年四十八。追贈豫章公夫人,還葬丹徒。高祖臨崩,遺詔留葬京師,於是備法駕,迎梓宮祔葬初寧陵。



 宋初,追贈俊金紫光祿大夫,妻高密叔孫氏封遷陵永平鄉君。俊子燾,燾弟熹,熹子質,自有傳。



 武帝張夫人,諱闕,不知何郡縣人也。義熙初,得幸高祖,生少帝,又生義興恭長公主惠媛。永初元年,拜為夫人。少帝即位,有司奏曰:「臣聞嚴親敬始,所因者本,充孝之道,由中被外。伏惟夫人德並坤元,徽音光劭,發祥兆慶,
 誕啟聖明。宜崇極徽號,允備盛則。從《春秋》母以子貴之義,遵漢、晉推愛之典,謹上尊號為皇太后,宮曰永樂。」少帝既廢,太后還璽紱,隨居吳縣。太祖元嘉元年,拜營陽王太妃。三年,薨。



 少帝司馬皇后,諱茂英,河內溫人,晉恭帝女也。初封海鹽公主,少帝以公子尚焉。宋初,拜皇太子妃。少帝即位,立為皇后。元嘉元年,降為營陽王妃,又為南豐王太妃。十六年薨,時年四十七。



 武帝胡婕妤,諱道安,淮南人。義熙初,為高祖所納,生文帝。五年,被譴賜死,時年四十二。葬丹徒。高祖踐阼,追贈婕妤。太祖即位,有司奏曰:「臣聞德厚者禮尊,慶深者位極。故閟宮既構,詠歌先妣;園陵崇衛,聿追來孝。伏惟先婕妤柔明塞淵,光備六列,德昭坤範,訓洽母儀。用能啟祚聖明,奄宅四海。嚴親莫逮,天祿永違。臣等遠準《春秋》,近稽漢、晉。謹上尊號曰章皇太后,陵曰熙寧。」



 立廟於京師。



 太后兄子元慶,位至奉朝請。



 文帝袁皇后,諱齊媯,陳郡陽夏人,左光祿大夫敬公湛之庶女也。母本卑賤,后年五六歲,方見舉。後適太祖,初拜宜都王妃。生子劭、東陽獻公主英娥。上待后恩禮甚篤,袁氏貧薄,后每就上求錢帛以贍與之;上性節儉,所得不過三五萬、三五十匹。後潘淑妃有寵,愛傾後宮,咸言所求無不得。后聞之,欲知信否,乃因潘求三十萬錢與家,以觀上意,信宿便得。因此恚恨甚深,稱疾不復見上。上每入,必他處回避。上數掩伺之,不能得。始興王濬
 諸庶子問訊,后未嘗視也。后遂憤恚成疾。元嘉十七年,疾篤,上執手流涕問所欲言,后視上良久,乃引被覆面。崩於顯陽殿,時年三十六。上甚相悼痛,詔前永嘉太守顏延之為哀策,文甚麗。其辭曰:龍輁纚綍,容翟結驂。皇塗昭列,神路幽嚴。皇帝親臨祖饋,躬瞻宵載。飾遺儀於組旒,想徂音乎珩佩。悲黼筵之移御,痛翬褕之重晦。降輿客位,撤奠殯階。



 乃命史臣,誄德述懷。其辭曰:倫昭儷昇,有物有憑。圓精初鑠,方只始凝。昭哉世族,祥發慶
 膺。祕儀景胄,圖光玉繩。昌輝在陰,柔明將進。率禮蹈和,稱詩納順。爰自待年,金聲夙振。亦既有行,素章增絢。象服是加,言觀惟則。俾我王風,始基嬪德。蕙問川流,芳猷淵塞。方江泳漢,再謠南國。伊昔不造,洪化中微。用集寶命,仰陟天機。釋位公宮,登耀紫闈。欽若皇姑,允迪前徽。孝達寧親,敬行宗祀。進思才淑,傍綜圖史。



 發音在詠,動容成紀。壺政穆宣,房樂昭理。坤則順成,星軒潤飾。德之所屆,惟深必測。下節震騰,上清朓側。有來斯雍,無思不
 極。謂道輔仁,司化莫晰。



 象物方臻,眡祲告沴。太和既融,收華委世。蘭殿長陰,椒塗弛衛。嗚呼哀哉!



 戒涼在律,杪秋即穸。霜夜流唱,曉月升魄。八神警引,五輅遷迹。噭噭儲嗣,哀哀列辟。灑零玉墀,雨泗丹掖,撫存悼亡,感今懷昔。嗚呼哀哉!南背國門,北首山園。僕人案節,服馬顧轅。遙酸紫蓋,眇泣素軒。滅彩清都,夷體壽原。邑野淪藹,戎夏悲沄。來芳可述,往駕弗援。嗚呼哀哉!



 策既奏,上自益「撫存悼亡,感今懷昔」八字,以致其意焉。有司奏謚宣皇
 后,上特詔曰「元」。



 初,后生劭,自詳視之,馳白太祖:「此兒形貌異常,必破國亡家,不可舉。」



 便欲殺之。太祖狼狽至后殿戶外,手撥幔禁之,乃止。后亡後,常有小小靈應。沈美人者,太祖所幸也。嘗以非罪見責,應賜死。從后昔所住徽音殿前度。此殿有五間,自后崩後常閉。美人至殿前,流涕大言曰:「今日無罪就死,先后若有靈,當知之!」殿諸窗戶應聲豁然開。職掌遽白太祖,太祖驚往視之。美人乃得釋。



 大明五年,世祖詔曰:「昔漢道既靈,博平輝絕,魏
 國方安,嘉憲啟策,皆因心所弘,酌典沿誥。亡外祖親王夫人柔德淑範,光啟坤載。屬內位闕正,攝饋閨庭,儀被芳闈,聞宣戚里。永言感遠,思追榮秩,宜式傍鴻則,敬登徽序。」乃追贈豫章郡新淦縣平樂鄉君。后之所生母也。又詔:「趙、蕭、臧光祿、袁敬公、平樂郡君墓,先未給塋戶。加世數已遠,胤嗣衰陵,外戚尊屬,不宜使墳塋蕪穢。可各給蠻戶三,以供灑掃。」后父湛,自有傳。



 文帝路淑媛,諱惠男,丹陽建康人也。以色貌選入後宮,
 生孝武帝,拜為淑媛。



 年既長,無寵,常隨世祖出蕃。世祖入討元凶,淑媛留尋陽。上即位,遣建平王宏奉迎。有司奏曰:「臣聞歷集周邦,徽音克嗣,氣淳漢國,沙麓發祥。昔在上代,業隆祚遠,未有不敷陰教以闡洪基,膺淑慶以載聖哲者也。伏惟淑媛柔明內昭,徽儀外範,合靈初迪,則庶姬仰耀;引訓蕃閫,則家邦被德。民應惟和,神屬惟祉,故能誕鐘睿躬,用集大命,固靈根於既殞,融盛烈乎中興。載厚化深,聲詠允緝,宜式諧舊典,恭享極號。謹奉尊
 號曰皇太后,宮曰崇憲。」太后居顯陽殿。



 上於閨房之內,禮敬甚寡,有所御幸,或留止太后房內,故民間喧然,咸有醜聲。宮掖事秘,莫能辨也。孝建二年,追贈太后父興之散騎常侍,興之妻徐氏餘杭縣廣昌鄉君。大明四年,太后弟子撫軍參軍瓊之上表曰:「先臣故懷安令道慶賦命乖辰,自違明世。敢緣衛戍請名之典,特乞雲雨,微垂灑潤。」詔付門下。有司承旨奏贈給事中。瓊之及弟休之、茂之並超顯職。太后頗豫政事,賜與瓊之等財物,家
 累千金;居處服器,與帝子相侔。



 瓊之宅與太常王僧達並門。嘗盛車服衛從造僧達,僧達不為之禮。瓊之以訴太后,太后大怒,告上曰:「我尚在,而人皆陵我家;死後,乞食矣!」欲罪僧達。



 上曰:「瓊之年少,自不宜輕造詣。王僧達貴公子,豈可以此事加罪!」



 大明五年,太后隨上巡南豫州,妃主以下並從。廢帝即位,號太皇太后。



 太宗踐阼,號崇憲太后。初,太宗少失所生,為太后所攝養,太宗盡心祗事,而太后撫愛亦篤。及上即位,供奉禮儀,不異舊日。
 有司奏曰:「夫德敷於內,典章必遠;化覃於外,徽號宜宣。伏惟皇太后懿聖自天,母儀允著,義明八遠,道變九圍。聖明登御,景胙攸改,皇太后宜即前號,別居外宮。」詔曰:「朕備丁艱罰,蚤嬰孤苦,特蒙崇憲太后聖訓撫育。昔在蕃閫,常奉藥膳,中迫凶威,抱懷莫遂。



 今泰運初啟,情典獲申,方欲親奉晨昏,盡歡閨禁。不得如所奏。」尋崩,時年五十五。遷殯東宮,門題曰宗憲宮。上又詔曰:「朕幼集荼蓼,夙憑德訓,龕虣定業,實資仁範,恩著屯夷,有兼常慕。
 夫禮沿情施,義循事立,可特齊衰三月,以申追仰之心。」謚曰昭皇太后,葬世祖陵東南,號曰修寧陵。



 先是,晉安王子勛未平,巫者謂宜開昭太后陵以為厭勝。修復倉卒,不得如禮。



 上性忌,慮將來致災。泰始四年夏,詔有司曰:「崇憲昭太后修寧陵地,大明之世,久所考卜。前歲遭諸蕃之難,禮從權宜。奉營倉卒,未暇營改。而塋隧之所,山原卑陋。頃年頹壞,日有滋甚,恒費修整,終無永固。且詳考地形,殊乖相勢。朕蚤蒙慈遇,情禮兼常,思使終始
 之義,載彰幽顯。史官可就巖山左右,更宅吉地。明審龜筮,須選令辰,式遵舊典,以禮創制。今中宇雖寧,邊虜未息,營就之功,務在從簡。舉言尋悲,情如切割。」有司奏:「北疆未緝,戎役是務,禮之詳略,各沿時宜。臣等參議,修寧陵玄宮補治毀壞,權施油殿,暫出梓宮,事畢即窆,於事為允。」詔可。



 瓊之為衡陽內史,先后卒。廢帝景和中,以休之為黃門侍郎,茂之左軍將軍,並封開國縣侯,邑千戶。又追贈興之侍中、金紫光祿大夫,謚曰孝侯;道慶散騎
 常侍、光祿大夫、開府儀同三司,謚曰敬侯。立道慶女為皇后,以休之為侍中,茂之黃門郎。太宗廢幼主,欲說太后之心,乃下令書曰:「太皇太后蚤垂愛遇,沿情即事,同於天屬。前車騎諮議參軍路休之、前丹陽丞路茂之,崇憲密戚,蚤延榮貫,並懷所勳,宜殊恒飾。休之可黃門侍郎,領步兵校尉;茂之可中書侍郎。」太宗未即位,故稱令書。茂之又遷司徒從事中郎,休之桂陽王休範鎮北諮議參軍。太宗殺世祖諸子,因此陷休之等,宥其諸子。



 孝武文穆王皇后,諱憲嫄,琅邪臨沂人。元嘉二十年,拜武陵王妃。生廢帝、豫章王子尚、山陰公主楚玉、臨淮康哀公主楚佩、皇女楚琇、康樂公主修明。世祖在蕃,后甚有寵。上入伐凶逆,后留尋陽,與太后同還京都,立為皇后。



 大明四年,后率六宮躬桑於西郊,皇太后觀禮。上下詔曰:「朕卜祥大昕,測辰拂羽,爰詔六宮,親蠶川室。皇太后降鑾從御,佇蹕觀禮。綠蘧既具,玄紞方修,庶儀發椒,闈化動中。縣妃主以下,可量加班錫。」廢帝即位,尊曰皇
 太后,宮曰永訓。其年,崩於含章殿,時年三十八。祔葬景寧陵。



 后父偃,字子游,晉丞相導玄孫,尚書嘏之子也。母晉孝武帝女鄱陽公主,宋受禪,封永成君。偃尚高祖第二女吳興長公主諱榮男,少歷顯官,黃門侍郎,祕書監,侍中。元嘉末,為散騎常侍、右衛將軍。世祖即位,以后父,授金紫光祿大夫,領義陽王師,常侍如故。遷右光祿大夫,常侍、王師如故。偃謙虛恭謹,不以世事關懷。孝建二年卒,時年五十四。追贈開府儀同三司,本官如故,謚曰恭
 公。



 長子藻,位至東陽太守。尚太祖第六女臨川長公主諱英媛。公主性妒,而藻別愛左右人吳崇祖。前廢帝景和中,主讒之於廢帝,藻坐下獄死,主與王氏離婚。泰始初,以主適豫章太守庾沖遠,未及成禮而沖遠卒。



 宋世諸主,莫不嚴妒,太宗每疾之。湖熟令袁慆妻以妒忌賜死,使近臣虞通之撰《妒婦記》。左光祿大夫江湛孫斅當尚世祖女,上乃使人為斅作表讓婚,曰:伏承詔旨,當以臨汝公主降嬪,榮出望表,恩加典外。顧審輶蔽,伏用憂
 惶。



 臣寒門顇族,人凡質陋,閭閻有對,本隔天姻。如臣素流,室貧業寡,年近將冠,皆已有室,荊釵布裙,足得成禮。每不自解,無偶迄茲,媒訪莫尋,素族弗問。自惟門慶,屬降公主,天恩所覃,容及醜末。懷憂抱惕,慮不獲免,徵命所當,果膺茲舉。雖門泰宗榮,於臣非幸,仰緣聖貸,冒陳愚實。



 自晉氏以來,配上王姬者,雖累經美胄,亟有名才,至如王敦懾氣,桓溫斂威,真長佯愚以求免,子敬灸足以違詔,王偃無仲都之質,而惈露於北階,何瑀闕龍工
 之姿,而投軀於深井,謝莊殆自同於矇叟,殷沖幾不免於彊鉏。彼數人者,非無才意,而勢屈於崇貴,事隔於聞覽,吞悲茹氣,無所逃訴。制勒甚於僕隸,防閑過於婢妾。往來出入,人理之常;當賓待客,朋從之義。而令掃轍息駕,無窺門之期;廢筵抽席,絕接對之理。非唯交友離異,乃亦兄弟疏闊。第令受酒肉之賜,制以動靜;監子荷錢帛之私,節其言笑。姆妳爭媚,相勸以嚴;妮媼競前,相諂以急。第令必凡庸下才,監子皆葭萌愚豎,議舉止則未閑
 是非,聽言語則謬於虛實。姆妳敢恃耆舊,唯贊妒忌;尼媼自倡多知,務檢口舌。其間又有應答問訊,卜筮師母,乃至殘餘飲食,詰辯與誰,衣被故敝,必責頭領。又出入之宜,繁省難衷,或進不獲前,或入不聽出。不入則嫌於欲疏,求出則疑有別意,召必以三晡為期,遣必以日出為限,夕不見晚魄,朝不識曙星。至於夜步月而弄琴,晝拱袂而披卷,一生之內,與此長乖。又聲影裁聞,則少婢奔迸;裾袂向席,則老醜叢來。左右整刷,以疑寵見嫌;賓
 客未冠,以少容致斥。禮則有列媵,象則有貫魚,本無嫚嫡之嫌,豈有輕婦之誚。況今義絕傍私,虔恭正匹,而每事必言無儀適,設辭輒言輕易我。又竊聞諸主集聚,唯論夫族。緩不足為急者法,急則可為緩者師,更相扇誘,本其恆意,不可貸借,固實常辭。或言野敗去,或言人笑我,雖家曰私理,有甚王憲,發口所言,恒同科律。王藻雖復彊佷,頗經學涉,戲笑之事,遂為冤魂。褚曖憂憤,用致夭絕。傷理害義,難以具聞。



 夫螽斯之德,實致克昌;專妒
 之行,有妨繁衍,是以尚主之門,往往絕嗣;駙馬之身,通離釁咎。以臣凡弱,何以克堪。必將毀族淪門,豈伊身眚。前後嬰此,其人雖眾,然皆患彰遐邇,事隔天朝,故吞言咽理,無敢論訴。臣幸屬聖明,矜照由道,弘物以典,處親以公,臣之鄙懷,可得自盡。如臣門分,世荷殊榮,足守前基,便預提拂,清官顯宦,或由才升,一叨婚戚,咸成恩假。是以仰冒非宜,披露丹實。非唯止陳一己,規全身願;實乃廣申諸門憂患之切。伏願天慈照察,特賜蠲停,使燕
 雀微群,得保叢蔚,蠢物含生,自己彌篤。若恩詔難降,披請不申,便當刊膚剪發,投山竄海。



 太宗以此表遍示諸主。於是臨川長公主上表曰:「妾遭隨奇薄,絕於王氏,私庭囂戾,致此分異。今孤疾煢然,假息朝夕,情寄所鐘,唯在一子。契闊荼炭,持兼憐愍,否泰枯榮,系以為命。實願申其門釁,還為母子。推遷僶俛,未及自聞。



 先朝慈愛,鑒妾丹衷。若賜使息徹歸第定省,仰揆天旨,或有可尋。今事迫誠切,不顧典憲,敢緣恩燾,觸冒披聞。特乞還身王
 族,守養弱嗣,雖死之日,實甘於生。」



 許之。



 藻弟懋,昇明末貴達。懋弟攸,太宰從事中郎,蚤卒,追贈黃門侍郎。弟臻,昇明末顯宦。



 前廢帝何皇后,諱令婉,廬江灊人也。孝建三年,納為皇太子妃。大明五年,薨於東宮徽光殿,時年十七。葬囗囗,謚曰獻妃。上更為太子置內職二等,曰保林,曰良娣。納南中郎長史太山羊瞻女為良娣,宜都太守袁僧惠女為保林。廢帝即位,追崇獻妃曰獻皇后。太宗踐阼,遷后
 與廢帝合葬龍山北。



 后父瑀,字稚玉,晉尚書左僕射澄曾孫也。祖融,大司農。瑀尚高祖少女豫章康長公主諱欣男。公主先適徐喬,美容色,聰敏有智數。太祖世,禮待特隆。瑀豪競於時,與平昌孟靈休、東海何勖等,並以輿馬驕奢相尚。公主與瑀情愛隆密,何氏外姻疏戚,莫不沾被恩紀。瑀歷位清顯,至衛將軍。大明八年,公主薨,瑀墓開,世祖追贈金紫光祿大夫,加散騎常侍。



 子邁,尚太祖第十女新蔡公主諱英媚。邁少以貴戚居顯宦,好犬
 馬馳逐,多聚才力之士。有墅在江乘縣界,去京師三十里。邁每游履,輒結駟連騎,武士成群。



 大明末,為豫章王子尚撫軍諮議參軍,加寧朔將軍、南濟陰太守。廢帝納公主於後宮,偽言薨殞,殺一婢送出邁第嬪葬行喪禮。常疑邁有異圖,邁亦招聚同志,欲因行幸廢立。事覺,廢帝自出討邁誅之。太宗即位,追封建寧縣侯,食邑五百戶。子曼倩嗣,齊受禪,國除。



 瑀兄子亮,孝建初,為桂陽太守。丞相南郡王義宣為逆,遣參軍王師壽斷桂陽道,以
 防廣州刺史宗愨,亮收斬之。官至新安內史。亮弟恢,廢帝元徽初,為廣州刺史,未之鎮,坐國哀期晦不到,免官。復起為都官尚書,未拜,卒。恢弟誕,司徒右長史。誕弟衍,最知名。性躁動。太宗初,為建安王休仁司徒從事中郎,仍除黃門郎,未拜竟,求轉司徒司馬。得司馬,復求太子右率。拜右率一二日,復求侍中。旬日之間,求進無已。不得侍中,以怨詈賜死。



 文帝沈婕妤,諱容姬,不知何許人也。納於後宮,為美人。生
 明帝,拜為婕妤。



 元嘉三十年卒,時四十。葬建康之莫府山。世祖即位,追贈湘東國太妃。太宗即位,有司奏曰:「昔豳都追遠,正邑纏哀,緬慕德義,敬奉園陵。先太妃德履端華,徽景明峻,風光宸掖,訓流國闈,鞠聖誕靈,蚤捐鴻祚。臣等遠模漢冊,近儀晉典,謹上尊號為皇太后。」下禮官議謚,謚曰宣太后,陵號曰崇寧。



 以太后弟道慶為給事中,泰始三年卒,追贈通直散騎常侍,賜爵縣侯。又追贈太后父散騎常侍,母王氏成樂鄉君。



 明恭王皇后,諱貞風,琅邪臨沂人也。元嘉二十五年,拜淮陽王妃;太宗改封,又為湘東王妃。生晉陵長公主伯姒、建安長公主伯媛。太宗即位,立為皇后。上常宮內大集,而瑀婦人觀之,以為懽笑。后以扇障面,獨無所言。帝怒曰:「外舍家寒乞,今共為笑樂,何獨不視?」后曰:「為樂之事,其方自多。豈有姑姊妹集聚,而瑀婦人形體。以此為樂,外舍之為歡適,實與此不同。」帝大怒,遣后令起。后兄揚州刺史景文以此事語從舅陳郡謝緯曰:「后在家為儜弱
 婦人,不知今段遂能剛正如此。」



 廢帝即位,尊為皇太后,宮曰弘訓。廢帝失德,太后每加勖譬,始者猶見順從,後狂慝轉甚,漸不悅。元徽五年五月五日,太后賜帝玉柄毛扇,帝嫌其毛柄不華,因此欲加鴆害,已令太醫煮藥,左右人止之曰:「若行此事,官便應作孝子,豈復得出入狡獪。」帝曰:「汝語大有理。」乃止。



 順帝即位,齊王秉權,宗室劉晃、劉綽、卜伯興等有異志,太后頗與相關。順帝禪位,太后與帝遜於東邸,因遷居丹陽宮,拜汝陰王太妃。順
 帝殂於丹陽,更立第京邑。建元元年,薨於第,時年四十四。追加號謚,葬以宋后禮。父僧朗,事別見《景文傳》。



 明帝陳貴妃,諱妙登,丹陽建康人,屠家女也。世祖常使尉司採訪民間子女有姿色者。太妃家在建康縣界,家貧,有草屋兩三間。上出行,問尉曰:「御道邊那得此草屋,當由家貧。」賜錢三萬,令起瓦屋。尉自送錢與之,家人並不在,唯太妃在家,時年十二三。尉見其容質甚美,即以白世祖,於是迎入宮。在路太后房內,經二三年,再呼,不
 見幸。太后因言於上,以賜太宗。始有寵,一年許衰歇,以乞李道兒。尋又迎還,生廢帝,故民中皆呼廢帝為李氏子。廢帝後每自稱李將軍,或自謂李統。



 太宗即位,拜貴妃,禮秩同皇太子妃。廢帝踐阼,有司奏曰:「臣聞河龍啟聖,理浹民神;郊電基皇,慶爍天地。故資敬之道,粹古銘風;沿貴之誼,眇代凝則。


伏惟貴妃含和日晷,表淑星樞,徽音峻古,柔光照世,聲華帝掖,軌秀天嬪,景發皇明,祚昌睿命。而備物之章,未煥彞策。遠酌前王,允陟鴻典。臣
 等參議,謹上尊號曰皇太妃。輿服一如晉孝武帝太后故事。置家令一人。改諸國太妃曰太妃
 \gezhu{
  妃音怡}
 。宮曰弘化。」追贈太妃父金寶散騎常侍,金寶妻王氏永世縣成樂鄉君。昇明初,降為蒼梧王太妃。



 伯父照宗,中書通事舍人。叔佛念,步兵校尉。兄敬元,通直郎,南魯郡太守。



 佛念大通貨賄,侵亂朝政。昇明初,賜死。



 後廢帝江皇后,諱簡珪,濟陽考城人,北中郎長史智淵孫女。泰始五年,太宗訪求太子妃,而雅信小數,名家女
 多不合。后弱小,門無彊蔭,以卜筮最吉,故為太子納之。諷朝士州郡令獻物,多者將直百金。始興太守孫奉伯止獻琴書,其外無餘物。上大怒,封藥賜死,既而原之。太子即帝位,立為皇后。帝既廢,降為蒼梧王妃。智淵自有傳。



 明帝陳昭華,諱法容,丹陽建康人也。太宗晚年,痿疾不能內御,諸弟姬人有懷孕者,輒取以入宮;及生男,皆殺其母,而以與六宮所愛者養之。順帝,桂陽王休範子也,
 以昭華為母焉。明帝崩,昭華拜安成王太妃。順帝即位,進為皇太妃。



 順帝禪位,去皇太妃之號。



 順帝謝皇后,諱梵境,陳郡陽夏人,右光祿大夫莊孫女也。昇明二年,立為皇后。順帝禪位,降為汝陰王妃。莊自有傳。



 史臣曰:飲食男女,人之大欲存焉。故聖人順民情而為之度,王宮六列,士室二等,皆司事設防,典文曲立。若夫義篤閫闈,化形邦國,古先哲王有以之致治者矣。夫后
 妃專夕,配以德升;姬嬙並御,進非色幸。欲使情有覃被,愛罔偏流,專貞內表,妖蠱外息。至於降班在四,簪珥成行;同列者三,環珮係響,乃可以燮理陰教,輔佐君德。宋氏藉晉世令典,娉納有章,伣天作儷,必四岳之後。雖正位天閨,禮亢尊極,而衰懨易兆,恩宴難留,一謝屬車之塵,永隔青蒲之地。是故元后憤終,良有以也。自元嘉以降,內職稍繁,椒庭綺觀,千門萬戶,而淫妝怪飾,變炫無窮。自漢氏昭陽之輪奐,魏室九華之照曜,曾不能概其
 萬一。徒以所選止於軍署之內,徵引極乎廝皁之間,非晉氏採擇濫及冠冕也。且愛止帷房,權無外授,戚屬餼賚,歲時不過肴漿,斯為美矣。及太祖之傾惑潘嫗,謀及婦人;大明之淪溺殷姬,並后匹嫡,至使多難起於肌膚,並命行於同產,又況進於此者乎!以斯言之,三代、二漢之亡於淫嬖,非不幸也。






\end{pinyinscope}