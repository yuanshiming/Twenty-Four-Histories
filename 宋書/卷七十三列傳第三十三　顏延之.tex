\article{卷七十三列傳第三十三 顏延之}

\begin{pinyinscope}

 顏延之,字延年,瑯邪臨沂人也。曾
 祖含,右光祿大夫。祖約,零陵太守。父顯,護軍司馬。延之少孤貧,居負郭,室巷甚陋。好讀書,無所不覽,文章之美,冠絕當時。飲酒不護
 細行,年三十,猶未婚。妹適東莞劉憲之,穆之子也。穆之既與延之通家,又聞其美,將仕之;先欲相見,延之不往也。後將軍、吳國內史劉柳以為行參軍,因轉主簿,豫章公世子中軍行參軍。



 義熙十二年,高祖北伐,有宋公之授,府遣一使慶殊命,參起居;延之與同府王參軍俱奉使至洛陽,道中作詩二首,文辭藻麗,為謝晦、傅亮所賞。宋國建,奉常鄭鮮之舉為博士,仍遷世子舍人。高祖受命,補太子舍人。鴈門人周續之隱居廬山,儒學著稱,永
 初中,徵詣京師,開館以居之。高祖親幸,朝彥畢至,延之官列猶卑,引升上席。上使問續之三義,續之雅仗辭辯,延之每折以簡要。既連挫續之,上又使還自敷釋,言約理暢,莫不稱善。徙尚書儀曹郎,太子中舍人。



 時尚書令傅亮自以文義之美,一時莫及,延之負其才辭,不為之下,亮甚疾焉。



 廬陵王義真頗好辭義,待接甚厚;徐羨之等疑延之為同異,意甚不悅。少帝即位,以為正員郎,兼中書,尋徙員外常侍,出為始安太守。領軍將軍謝晦謂
 延之曰:「昔荀勖忌阮咸,斥為始平郡,今卿又為始安,可謂二始。」黃門郎殷景仁亦謂之曰:「所謂俗惡俊異,世疵文雅。」延之之郡,道經汨潭,為湘州刺史張紀祭屈原文以致其意,曰:恭承帝命,建旋舊楚。訪懷沙之淵,得捐佩之浦。弭節羅潭,艤舟汨渚,敬祭楚三閭大夫屈君之靈:蘭薰而摧,玉貞則折。物忌堅芳,人諱明潔。曰若先生,逢辰之缺。溫風迨時,飛霜急節。嬴、芊遘紛,昭、懷不端。謀折儀、尚,貞蔑椒、蘭。身絕郢闕,跡遍湘幹。比物荃蓀,連類龍
 鸞。聲溢金石,志華日月。如彼樹芬,實穎實發。望汨心欷,瞻羅思越。藉用可塵,昭忠難闕。



 元嘉三年,羨之等誅,徵為中書侍郎,尋轉太子中庶子。頃之,領步兵校尉,賞遇甚厚。延之好酒疏誕,不能斟酌當世,見劉湛、殷景仁專當要任,意有不平,常云:「天下之務,當與天下共之,豈一人之智所能獨了!」辭甚激揚,每犯權要。



 謂湛曰:「吾名器不升,當由作卿家吏。」湛深恨焉,言於彭城王義康,出為永嘉太守。延之甚怨憤,乃作《五君詠》以述竹林七賢,山
 濤、王戎以貴顯被黜,詠嵇康曰:「鸞翮有時鎩,龍性誰能馴。」詠阮籍曰:「物故可不論,途窮能無慟。」



 詠阮咸曰:「屢薦不入官,一麾乃出守。」詠劉伶曰:「韜精日沉飲,誰知非荒宴。」



 此四句,蓋自序也。湛及義康以其辭旨不遜,大怒。時延之已拜,欲黜為遠郡,太祖與義康詔曰:「降延之為小邦不政,有謂其在都邑,豈動物情,罪過彰著,亦士庶共悉,直欲選代,令思愆裏閭。猶復不悛,當驅往東土。乃志難恕,自可隨事錄治。殷、劉意咸無異也。」乃以光祿勳車
 仲遠代之。



 延之與仲遠世素不協,屏居里巷,不豫人間者七載。中書令王球名公子,遺務事外,延之慕焉;球亦愛其材,情好甚款。延之居常罄匱,球輒贍之。晉恭思皇后葬,應須百官,湛之取義熙元年除身,以延之兼侍中。邑吏送札,延之醉,投札於地曰:「顏延之未能事生,焉能事死!」閑居無事,為《庭誥》之文。今刪其繁辭,存其正,著於篇。曰:《庭誥》者,施於閨庭之內,謂不遠也。吾年居秋方,慮先草木,故遽以未聞,誥爾在庭。若立履之方,規鑒之明,已
 列通人之規,不復續論。今所載咸其素畜,本乎性靈,而致之心用。夫選言務一,不尚煩密,而至於備議者,蓋以網諸情非。



 古語曰得鳥者羅之一目,而一目之羅,無時得鳥矣。此其積意之方。



 道者識之公,情者德之私。公通,可以使神明加嚮;私塞,不能令妻子移心。



 是以昔之善為士者,必捐情反道,合公屏私。



 尋尺之身,而以天地為心;數紀之壽,常以金石為量。觀夫古先垂戒,長老餘論,雖用細制,每以不朽見銘;繕築末迹,咸以可久承志。況
 樹德立義,收族長家,而不思經遠乎。曰身行不足遺之後人。欲求子孝必先慈,將責弟悌務為友。雖孝不待慈,而慈固植孝;悌非期友,而友亦立悌。



 夫和之不備,或應以不和;猶信不足焉,必有不信。儻知恩意相生,情理相出,可使家有參、柴,人皆由、損。夫內居德本,外夷民譽,言高一世,處之逾默;器重一時,體之滋沖。不以所能干眾,不以所長議物,淵泰入道,與天為人者,士之上也。若不能遺聲,欲人出已,知柄在虛求,不可校得,敬慕謙通,畏
 避矜踞,思廣監擇,從其遠猷,文理精出,而言稱未達,論問宣茂,而不以居身,此其亞也。



 若乃聞實之為貴,以辯畫所克,見聲之取榮,謂爭奪可獲,言不出於戶牖,自以為道義久立,才未信於僕妾,而曰我有以過人,於是感茍銳之志,馳傾觖之望,豈悟已掛有識之裁,入修家之誡乎!記所云「千人所指,無病自死」者也。行近於此者,吾不願聞之矣。



 凡有知能,預有文論,不練之庶士,校之群言,通才所歸,前流所與,焉得以成名乎。若呻吟於牆室
 之內,喧囂於黨輩之間,竊議以迷寡聞,妲語以敵要說,是短算所出,而非長見所上。適值尊朋臨座,稠覽博論,而言不入於高聽,人見棄於眾視,則慌若迷塗失偶,黶如深夜撤燭,銜聲茹氣,腆默而歸,豈識向之夸慢,祗足以成今之沮喪邪!此固少壯之廢,爾其戒之。



 夫以怨誹為心者,未有達無心救得喪,多見誚耳。此蓋臧獲之為,豈識量之為事哉!是以德聲令氣,愈上每高,忿言懟議,每下愈發。有尚於君子者,寧可不務勉邪!雖曰恒人,情
 不能素盡,故當以遠理勝之,麼算除之,豈可不務自異,而取陷庸品乎。



 富厚貧薄,事之懸也。以富厚之身,親貧薄之人,非可一時同處。然昔有守之無怨,安之不悶者,蓋有理存焉。夫既有富厚,必有貧薄,豈其證然,時乃天道。



 若人皆厚富,是理無貧薄。然乎?必不然也。若謂富厚在我,則宜貧薄在人。可乎?



 又不可矣。道在不然,義在不可,而橫意去就,謬生希幸,以為未達至分。



 蠶溫農飽,民生之本,躬稼難就,止以僕役為資,當施其情願,庀其衣食,
 定其當治,遞其優劇,出之休饗,後之捶責,雖有勸恤之勤,而無沾曝之苦。務前公稅,以遠吏讓,無急傍費,以息流議,量時發斂,視歲穰儉,省贍以奉己,損散以及人,此用天之善,御生之得也。



 率下多方,見情為上;立長多術,晦明為懿。雖及僕妾,情見則事通;雖在畎畝,明晦則功博。若奪其常然,役其煩務,使威烈雷霆,猶不禁其欲;雖棄其大用,窮其細瑕,或明灼日月,將不勝其邪。故曰:「孱焉則差,的焉則暗。」是以禮道尚優,法意從刻。優則人自
 為厚,刻則物相為薄。耕收誠鄙,此用不忒,所謂野陋而不以居心也。



 含生之氓,同祖一氣,等級相傾,遂成差品,遂使業習移其天識,世服沒其性靈。至夫願欲情嗜,宜無間殊,或役人而養給,然是非大意,不可侮也。隅奧有灶,齊侯蔑寒,犬馬有秩,管、燕輕饑。若能服溫厚而知穿弊之苦,明周之德;厭滋旨而識寡嗛之急,仁恕之功。豈與夫比肌膚於草石,方手足於飛走者,同其意用哉!



 罰慎其濫,惠戒其偏。罰濫則無以為罰,惠偏則不如無惠,
 雖爾眇末,猶扁庸保之上,事思反己,動類念物,則其情得,而人心塞矣。



 抃博蒱塞,會眾之事,諧調哂謔,適坐之方,然失敬致侮,皆此之由。方其剋瞻,彌喪端儼,況遭非鄙,慮將醜折。豈若拒其容而簡其事,靜其氣而遠其意,使言必諍厭,賓友清耳;笑不傾嫵,左右悅目。非鄙無因而生,侵侮何從而入,此亦持德之管龠,爾其謹哉。



 嫌惑疑心,誠亦難分,豈唯厚貌蔽智之明,深情怯剛之斷而已哉。必使猜怨愚賢,則顰笑入戾,期變犬馬,則步顧成
 妖。況動容竊斧,束裝濫金,又何足論。是以前王作典,明慎議獄,而僭濫易意;朱公論璧,光澤相如,而倍薄異價。此言雖大,可以戒小。



 游道雖廣,交義為長。得在可久,失在輕絕。久由相敬,絕由相狎。愛之勿勞,當扶其正性;忠而勿誨,必藏其枉情。輔以藝業,會以文辭,使親不可褻,疏不可間,每存大德,無挾小怨。率此往也,足以相終。



 酒酌之設,可樂而不可嗜,嗜而非病者希,病而遂眚者幾。既眚既病,將蔑其正。若存其正性,紓其妄發,其唯善戒
 乎?聲樂之會,可簡而不可違,違而不背者鮮矣,背而非弊者反矣。既弊既背,將受其毀。必能通其礙而節其流,意可為和中矣。



 善施者豈唯發自人心,乃出天則。與不待積,取無謀實,並散千金,誠不可能。



 贍人之急,雖乏必先,使施如王丹,受如杜林,亦可與言交矣。



 浮華怪飾,滅質之具;奇服麗食,棄素之方。動人勸慕,傾人顧盼,可以遠識奪,難用近欲從。若睹其淫怪,知生之無心,為見奇麗,能致諸非務,則不抑自貴,不禁自止。



 夫數相者,必有之
 徵,既聞之術人,又驗之吾身,理可得而論也。人者兆氣二德,稟體五常。二德有奇偶,五常有勝殺,及其為人,寧無葉沴。亦猶生有好醜,死有夭壽,人皆知其懸天;至於丁年乖遇,中身迂合者,豈可易地哉!是以君子道命愈難,識道愈堅。



 古人恥以身為溪壑者,屏欲之謂也。欲者,性之煩濁,氣之蒿蒸,故其為害,則燻心智,耗真情,傷人和,犯天性。雖生必有之,而生之德,猶火含煙而妨火,桂懷蠹而殘桂,然則火勝則煙滅,蠹壯則桂折。故性明者
 欲簡,嗜繁者氣惛,去明即惛,難以生矣。其以中外群聖,建言所黜,儒道眾智,發論是除。然有之者不患誤深,故藥之者恒苦術淺,所以毀道多而於義寡。頓盡誠難,每指可易,能易每指,亦明之末。



 廉嗜之性不同,故畏慕之情或異,從事於人者,無一人我之心,不以己之所善謀人,為有明矣。不以人之所務失我,能有守矣。己所謂然,而彼定不然,弈棋之蔽;悅彼之可,而忘我不可,學顰之蔽。將求去蔽者,念通怍介而已。



 流言謗議,有道所不免,
 況在闕薄,難用算防。接應之方,言必出己。或信不素積,嫌間所襲,或性不和物,尤怨所聚,有一於此,何處逃毀。茍能反悔在我,而無責於人,必有達鑒,昭其情遠,識跡其事。日省吾躬,月料吾志,寬默以居,潔靜以期,神道必在,何恤人言。



 諺曰,富則盛,貧則病矣。貧之病也,不唯形色粗黶,或亦神心沮廢;豈但交友疏棄,必有家人誚讓。非廉深識遠者,何能不移其植。故欲蠲憂患,莫若懷古。



 懷古之志,當自同古人,見通則憂淺,意遠則怨浮,昔有琴
 歌於編蓬之中者,用此道也。



 夫信不逆彰,義必出隱,交賴相盡,明有相照。一面見旨,則情固丘岳;一言中志,則意入淵泉。以此事上,水火可蹈,以此託友,金石可弊。豈待充其榮實,乃將議報,厚之篚筐,然後圖終。如或與立,茂思無忽。



 祿利者受之易,易則人之所榮;蠶穡者就之艱,艱則物之所鄙。艱易既有勤倦之情,榮鄙又間向背之意,此二塗所為反也。以勞定國,以功施人,則役徒屬而擅豐麗;自埋於民,自事其生,則督妻子而趨耕織。必
 使陵侮不作,懸企不萌,所謂賢鄙處宜,華野同泰。



 人以有惜為質,非假嚴刑;有恒為德,不慕厚貴。有惜者,以理葬;有恒者,與物終。世有位去則情盡,斯無惜矣。又有務謝則心移,斯不恒矣。又非徒若此而已,或見人休事,則勤蘄結納,及聞否論,則處彰離貳,附會以從風,隱竊以成釁,朝吐面譽,暮行背毀,昔同稽款,今猶叛戾,斯為甚矣。又非唯若此而已,或憑人惠訓,藉人成立,與人餘論,依人揚聲,曲存稟仰,甘赴塵軌。衰沒畏遠,忌聞影迹,又
 蒙之,毀之無度,心短彼能,私樹己拙,自崇恒輩,罔顧高識,有人至此,實蠹大倫。每思防避,無通閭伍。



 睹驚異之事,或無涉傳;遭卒迫之變,反思安順。若異從己發,將尸謗人,迫而又迕,愈使失度。能夷異如裴楷,處逼如裴遐,可稱深士乎。



 喜怒者有性所不能無,常起於褊量,而止於弘識。然喜過則不重,怒過則不威,能以恬漠為體,寬愉為器者,大喜蕩心,微抑則定,甚怒煩性,小忍即歇。故動無愆容,舉無失度,則物將自懸,人將自止。



 習之所變亦
 大矣,豈唯蒸性染身,乃將移智易慮。故曰:「與善人居,如入芷蘭之室,久而不聞其芬。」與之化矣。「與不善人居,如入鮑魚之肆,久而不知其臭」。與之變矣。是以古人慎所與處。唯夫金真玉粹者,乃能盡而不污爾。故曰:「丹可滅而不能使無赤,石可毀而不可使無堅。」茍無丹石之性,必慎浸染之由。



 能以懷道為人,必存從理之心。道可懷而理可從,則不議貧,議所樂爾。或云:「貧何由樂?」此未求道意。道者,瞻富貴同貧賤,理固得而齊。自我喪之,未為通
 議,茍議不喪,夫何不樂。



 或曰,溫飽之貴,所以榮生,饑寒在躬,空曰從道,取諸其身,將非篤論,此又通理所用。凡養生之具,豈間定實,或以膏腴夭性,有以菽藿登年。中散云,所足與,不由外。是以稱體而食,貧歲愈嗛;量腹而炊,豐家餘餐。非粒實息耗,意有盈虛爾。況心得復劣,身獲仁富,明白入素,氣志如神,雖十旬九飯,不能令饑,業席三屬,不能為寒。豈不信然!



 且以己為度者,無以自通彼量。渾四游而幹五緯,天道弘也。振河海而載山川,地道
 厚也。一情紀而合流貫,人靈茂也。昔之通乎此數者,不為剖判之行,必廣其風度,無挾私殊,博其交道,無懷曲異。故望塵請友,則義士輕身,一遇拜親,則仁人投分。此倫序通允,禮俗平一,上獲其用,下得其和。



 世務雖移,前休未遠,人之適主,吾將反本。三人至生,暫有之識,幼壯驟過,衰耗騖及。其間夭鬱,既難勝言,假獲存遂,又云無幾。柔麗之身,亟委土木,剛清之才,遽為丘壤,回遑顧慕,雖數紀之中爾。以此持榮,曾不可留,以此服道,亦何能
 平。進退我生,遊觀所達,得貴為人,將在含理。含理之貴,惟神與交,幸有心靈,義無自惡,偶信天德,逝不上慚。欲使人沈來化,志符往哲,勿謂是賒,日鑿斯密。著通此意,吾將忘老,如固不然,其誰與歸。值懷所撰,略布眾修;若備舉情見,顧未書一。贍身之經,別在田家節政;奉終之紀,自著燕居畢義。



 劉湛誅,起延之為始興王浚後軍諮議參軍,御史中丞。在任縱容,無所舉奏。



 遷國子祭酒、司徒左長史,坐啟買人田,不肯還直。尚書左丞荀赤松奏
 之曰:「求田問舍,前賢所鄙。延之唯利是視,輕冒陳聞,依傍詔恩,拒捍餘直,垂及周年,猶不畢了,昧利茍得,無所顧忌。延之昔坐事屏斥,復蒙抽進,而曾不悛革,怨誹無已。交游闒茸,沈迷曲蘗,橫興譏謗,詆毀朝士。仰竊過榮,增憤薄之性;私恃顧盼,成彊梁之心。外示寡求,內懷奔競,干祿祈遷,不知極已,預燕班觴,肆罵上席。山海含容,每存遵養,愛兼彫蟲,未忍遐棄,而驕放不節,日月彌著。臣聞聲問過情,孟軻所恥,況聲非外來,問由己出,雖心
 智薄劣,而高自比擬,客氣虛張,曾無愧畏,豈可復弼亮五教,增曜台階。請以延之訟田不實,妄干天聽,以彊凌弱,免所居官。」詔可。



 復為秘書監,光祿勳,太常。時沙門釋慧琳,以才學為太祖所賞愛,每召見,常升獨榻,延之甚疾焉。因醉白上曰:「昔同子參乘,袁絲正色。此三台之坐,豈可使刑餘居之。」上變色。延之性既褊激,兼有酒過,肆意直言,曾無遏隱,故論者多不知云。居身清約,不營財利,布衣蔬食,獨酌郊野,當其為適,傍若無人。


二十九年,
 上表自陳曰:「臣聞行百里者半於九十,言其末路之難也。愚心常謂為虛,方今乃知其信。臣延之人薄寵厚,宿塵國言,而雪效無從,榮牒增廣,歷盡身彫,日叨官次,雖容載有途,而妨穢滋積。早欲啟請餘算,屏蔽醜老。但時制行及,歸慕無賒,是以腆冒愆非,簡息干黷耗歇難支,質用有限,自去夏侵暑,入此秋變,頭齒眩疼,根痼漸劇,手足冷痺,左胛尤甚。素不能食,頃向減半。本猶賴服,比倦悸晚,年疾所催,顧景引日。臣班叨首卿,位尸封典,肅
 祗朝校,尚恧匪任,而陵廟眾事,有以疾怠,宮府覲慰,轉闕躬親。息
 \gezhu{
  大}
 庸微,過宰近邑,回澤爰降,實加將監,乞解所職,隨就藥養。伏願聖慈,特垂矜許。稟恩明世,負報冥暮,仰企端闈,上戀罔極。」不許。明年致事。元凶弒立,以為光祿大夫。



 先是,子竣為世祖南中郎諮議參軍。及義師入討,竣參定密謀,兼造書檄。劭召延之,示以檄文,問曰:「此筆誰所造?」延之曰:「竣之筆也。」又問:「何以知之?」延之曰:「竣筆體,臣不容不識。」劭又曰:「言辭何至乃爾。」延之曰:「竣
 尚不顧老父,何能為陛下。」劭意乃釋,由是得免。



 世祖登阼,以為金紫光祿大夫,領湘東王師。子竣既貴重,權傾一朝,凡所資供,延之一無所受,器服不改,宅宇如舊。常乘羸牛笨車,逢竣鹵簿,即屏往道側。



 又好騎馬,遨游里巷,遇知舊輒據鞍索酒,得酒必頹然自得。常語竣曰:「平生不喜見要人,今不幸見汝。」竣起宅,謂曰:「善為之,無令後人笑汝拙也。」表解師職,加給親信三十人。



 孝建三年,卒,時年七十三。追贈散騎常侍、特進,金紫光祿大夫如
 故。謚曰憲子。延之與陳郡謝靈運俱以詞彩齊名,自潘岳、陸機之後,文士莫及也,江左稱顏、謝焉。所著並傳於世。



 竣別有傳。竣弟測,亦以文章見知,官至江夏王傅義恭大司徒錄事參軍,蚤卒。


太宗即位,詔曰:「延之昔師訓朕躬,情契兼款。前記室參軍、濟陽太守
 \gezhu{
  大}
 伏勤蕃朝,綢繆恩舊。可擢為中書侍郎。」
 \gezhu{
  大}
 ,延之第三子也。



 史臣曰:出身事主,雖義在忘私,至於君親兩事,既無同濟,為子為臣,各隨其時可也。若夫馳文道路,軍政恒儀,
 成敗所因,非系乎此。而據筆數罪,陵仇犯逆,餘彼慈親,垂之虎吻,以此為忠,無聞前誥。夫自忍其親,必將忍人之親;自忘其孝,期以申人之孝。食子放鹿,斷可識矣。《記》云:「八十者一子不從政,九十者家不從政。」豈不以年薄桑榆,憂患將及,雖有職王朝,許以辭事,況顛沛之道,慮在未測者乎!自非延年之辭允而義愜,夫豈或免。



\end{pinyinscope}