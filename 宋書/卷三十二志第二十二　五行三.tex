\article{卷三十二志第二十二 五行三}

\begin{pinyinscope}

 《五行傳》
 曰:「棄法律,逐功臣,殺太子,以妾為妻,則火不炎上。」謂火失其性而為災也。又曰:「視之不明,是謂不哲。厥咎舒,厥罰恆燠,厥極疾。時則有草妖,時則有裸蟲之孽,
 時則有羊禍,時則有目痾,時則有赤眚、赤祥。惟水沴火。」裸蟲,劉歆傳以為羽蟲。



 火不炎上:魏明帝太和五年五月,清商殿災。初,帝為平原王,納河南虞氏為妃。及即位,不以為后,更立典虞車工卒毛嘉女,是為悼皇后。后本仄微,非所宜升。以妾為妻之罰也。魏明帝青龍元年六月,洛陽宮鞠室災。二年四月,崇華殿災,延于南閣。



 繕復之。至三年七月,此
 殿又災。帝問高堂隆:「此何咎也?於禮寧有祈禳之義乎?」



 對曰:「夫災變之發,皆所以明教誡也。唯率禮修德,可以勝之。《易傳》曰:『上不儉,下不節,孽火燒其室。』又曰:『君高其臺,天火為災。』此人君茍飾宮室,不知百姓空竭,故天應之以旱,火從高殿起也。案舊占,災火之發,皆以臺榭宮室為誡。今宜罷散民役,務從節約,清掃所災之處,不敢於此有所營造。萐莆嘉禾,必生此地,以報陛下虔恭之德。」不從。遂復崇華殿,改曰九龍。以郡國前後言龍見者
 九,故以為名。多棄法度,疲民逞欲,以妾為妻之應也。



 吳孫亮建興元年十二月,武昌端門災;改作端門,又災內殿。案《春秋》魯雉門及兩觀災,董仲舒以為天意欲使定公誅季氏,若曰去其高顯而奢僭者也。漢武帝世,遼東高廟災,其說又同。今此與二事頗類也。且門者,號令所出;殿者,聽政之所。是時諸恪屬秉政,而矜慢放肆;孫峻總禁旅,而險害終著。武昌,孫氏尊號所始,天戒若曰,宜除其貴耍之首者。恪果喪眾殄民,峻授政於綝,綝廢
 亮也。或曰孫權毀徹武昌,以增太初宮,諸葛恪有遷都意,更起門殿,事非時宜,故見災也。



 京房《易傳》曰:「君不思道,厥妖火燒宮。」吳孫亮太平元年二月朔,建業火。



 人火之也。是秋,孫綝始秉政,矯以亮詔殺呂據、滕胤。明年,又輒殺朱異。棄法律、逐功臣之罰也。



 吳孫休永安五年二月,白虎門北樓災。六年十月,石頭小城火,燒西南百八十丈。是時嬖人張布專擅國勢,多
 行無禮,而韋昭、盛沖終斥不用,兼遣察戰等為使,驚擾州郡,致使交趾反亂。是其咎也。



 吳孫皓建衡二年三月,大火,燒萬餘家,死者七百人。案《春秋》,齊火,劉向以為桓公好內,聽女口,妻妾數更之罰也。皓制令詭暴,蕩棄法度,勞臣名士,誅斥甚眾。後宮萬餘,女謁數行,其中隆寵佩皇后璽者又多矣。故有大火。



 晉武帝太康八年三月乙丑,震災西閣、楚王所止坊,及臨商觀窗。
 十年四月癸丑,崇賢殿災。十月庚辰,含章鞠室、修成堂前廡、內坊東屋、煇章殿南閣火。時有上書者曰:「漢王氏五侯兄弟迭任,今楊氏三公並在大位,天變屢見,竊為陛下憂之。」楊珧由是乞退。是時帝納馮紞之間,廢張華之功;聽楊駿之讒,離衛瓘之寵。此逐功臣之罰也。明年,宮車晏駕。其後楚王承竊發之旨,戮害二公,身亦不免。震災其坊,又天意乎!



 晉惠帝元康五年閏月庚寅,武庫火。張華疑有亂,先固
 守,然後救災。是以累代異寶,王莽頭,孔子履,漢高斷白蛇劍及二百萬人器械,一時蕩盡。是後愍懷見殺,殺太子之罰也。天戒若曰,夫設險擊柝,所以固其國;儲積戎器,所以戒不虞。



 今塚嗣將傾,社稷將泯,禁兵無所復施,皇旅又將誰衛!帝后不悟,終喪四海,是其應也。張華、閻纂皆曰,武庫火而氐、羌反,太子見廢,則四海可知矣。元康八年十一月,高原陵火。是時賈后凶恣,賈謐擅朝,惡積罪稔,宜見誅絕。天戒若曰,臣妾之不可者,雖親貴
 莫比,猶宜忍而誅之,如吾燔高原陵也。帝既眊弱,而張華又不納裴頠、劉卞之謀,故后遂與謐誣殺太子也。干寶云:「高原陵火,太子廢,其應也。漢武帝世,高園便殿火,董仲舒對與此占同。」



 晉惠帝永康元年,帝納皇后羊氏。后將入宮,衣中忽有火,眾咸怪之。太安二年,后父玄之以成都之逼,憂死。永興元年,成都遂廢后,處之金墉城,而殺其叔父同之。是後還立,立而復廢者四,又詔賜死,荀籓表全之。雖末還
 在位,然憂逼折辱,終古未聞。此孽火之應。晉惠帝永興二年七月甲午,尚書諸曹火,延崇禮闥及閣道。夫百揆王化之本,王者棄法律之應也。清河王覃入為晉嗣,不終于位,又殺太子之罰也。



 晉孝懷帝永嘉四年十一月,襄陽火,死者三千餘人。是時王如自號大將軍、司雍二州牧,眾四五萬,攻略郡縣,以為己邑。都督力屈,嬰城自守,賊遂攻逼襄陽。



 此下陵上,陽失節,火災出也。



 晉元帝太興中,王敦鎮武昌。武昌火起,興眾救之。救於此而發於彼,東西南北數十處俱應,數日不絕。班固所謂濫炎妄起,雖興師不能救之之謂也。干寶曰:「此臣而君行,亢陽失節之災也。」晉元帝永昌二年正月癸巳,京都大火。三月,饒安、東光、安陵三縣火,燒七千餘家,死者萬五千人。



 晉明帝太寧元年正月,京都火。是時王敦威侮朝廷,多行無禮,內外臣下,咸懷怨毒。極陰生陽,故有火災。與董
 仲舒說《春秋》陳火同事也。



 晉穆帝永和五年六月,震災石虎太武殿及兩廂、端門,光爛照天,金石皆盡,火月餘乃滅。是年四月,石虎死矣。其後胡遂滅亡。



 晉海西太和中,郗愔為會稽。六月,大旱災,火燒數千家,延及山陰倉米數百萬斛。炎煙蔽天,不可撲滅。



 晉孝武帝寧康元年三月,京都風,火大起。是時桓溫入朝,志在陵上;少主踐位,人懷憂恐。此與太寧火同事。
 晉孝武帝太元十年正月,立國子學。學生多頑嚚,因風放火,焚房百餘間。是後考課不厲,賞黜無章,有育才之名,無收賢之實。



 《書》云:「知人則哲。」此不哲之罰先兆也。太元十三年十二月乙未,延賢堂災。



 丙申,螽斯、則百堂及客館、驃騎庫皆災。于時朝多弊政,衰陵日兆。不哲之罰,皆有象類。主相不悟,終至亂亡云。



 晉安帝隆安二年三月,龍舟二乘災。是水沴火也。晉安帝元興元年八月庚子,尚書下舍曹火。
 元興三年,盧循攻略廣州,刺史吳隱之閉城固守。是年十月壬戌夜,大火起。時民人避寇,盈滿城內。隱之懼有應賊,但務嚴兵,不先救火,由是府舍焚燒蕩盡,死者萬餘人,因遂散潰,悉為賊擒。殆與襄陽火同占也。



 晉安帝義熙四年七月丁酉,尚書殿中吏部曹火。義熙十一年,京都所在大行火災,吳界尤甚。火防甚峻,猶自不絕。王弘時為吳郡,白日在聽事上,見天上有一赤物下,狀如信幡,徑集路南人家屋上,火即復大發。弘
 知天為之災,不罪火主。



 宋文帝元嘉五年正月戊子,京邑大火。元嘉七年十二月乙亥,京邑火,延燒太社北墻。元嘉二十九年三月壬午,京邑大火,風雷甚壯。



 後廢帝元徽三年正月己巳,京邑大火。元徽三年三月戊辰,京邑大火,燒二岸數千家。



 恆燠:庶徵之恆燠,劉向、班固以冬亡冰及霜不殺草應之。京
 房《易傳》又曰:「夏則暑殺人,冬則物華實。」



 吳孫亮建興元年九月,桃李華。孫權世,政煩賦重,民雕於役。是時諸葛恪始輔政,息校官,原逋責,除關梁,崇寬厚。此舒緩之應也。一說桃李寒華為草妖,或屬華孽。



 魏元帝景元三年十月,桃李華。自高貴弒死之後,晉文王深樹恩德,事崇優緩,此其應也。



 晉穆帝永和九年十二月,桃李華。是時簡文輔政,事多弛略,舒緩之應也。



 宋順帝昇明元年十月,於潛桃、李、柰結實。



 草妖:漢獻帝建安二十五年春正月,魏武帝在洛陽,將起建始殿,伐濯龍祠樹而血出;又掘徙梨,根傷亦血出。帝惡之,遂寢疾,是月崩。蓋草妖,又赤祥也。是歲,魏文帝黃初元年也。



 吳孫亮五鳳元年六月,交趾稗草化為稻。昔三苗將亡,五穀變種。此草妖也。



 其後亮廢。



 蜀劉禪景耀五年,宮中大樹無故自折。譙周憂之,無所與之言,乃書柱曰:「眾而大,其之會,具而授,若何復。」言曹者眾也;魏者大也。眾而大,天下其當會也;具而授,如何復有立者乎。蜀果亡,如周言。此草妖也。



 吳孫皓天璽元年,吳郡臨平湖自漢末穢塞,是時一夕忽開除無草。長老相傳,此湖塞,天下亂;此湖開,天下平。吳尋亡,而九服為一。吳孫皓天紀三年八月,建業有鬼目菜生工黃狗家,依
 緣棗樹,長丈餘,莖廣四寸,厚三分。又有賣菜生工吳平家,高四尺,如枇杷形,上圓徑一尺八寸,下莖廣五寸,兩邊生葉綠色。東觀案圖,名鬼目作芝草,賣菜作平慮。遂以狗為侍芝郎,平為平慮郎,皆銀印青綬。



 干寶曰:「明年晉平吳,王浚止船,正得平渚,姓名顯然,指事之徵也。黃狗者,吳以土運承漢,故初有黃龍之瑞,及其季年,而有鬼目之妖,託黃狗之家,黃稱不改,而貴賤大殊。天道精微之應也。」



 晉惠帝元康二年春,巴西郡界竹生花,紫色,結實如麥,外皮青,中赤白,味甘。元康九年六月庚子,有桑生東宮西廂,日長尺餘;甲辰,枯死。此與殷太戊同妖。太子不能悟,故至廢戮也。班固稱「野木生朝而暴長,小人將暴居大臣之位,危亡國家,象朝將為墟也」。是後孫秀、張林尋用事,遂至大亂。



 晉惠帝永康元年四月丁巳,立皇孫臧為皇太孫。五月甲子,就東宮。桑又生於西廂。明年,趙倫篡位,鴆殺臧。此
 與愍懷同妖也。永康元年四月,壯武國有桑化為柏。是月,張華遇害。



 晉孝懷帝永嘉三年冬,項縣桑樹有聲如解材,民謂之桑林哭。案劉向說,桑者喪也,又為哭聲,不祥之甚。是時京師虛弱,胡寇交逼,司馬越無衛上國之心。四年冬,委而南出,至五年春,薨于此城。石勒邀其眾,圍而射之,王公以下至庶人,死者十餘萬人,又剖越棺焚其尸。是敗也,中原無所請命,洛京尋沒。桑哭之應也。



 永嘉六年五月,無錫縣有四株茱萸樹,相樛而生,狀若連理。先是,郭景純筮延陵偃鼠,遇《臨》之《益》,曰:「後當復有妖樹生,若瑞而非,辛螫之木也。



 儻有此,東南數百里必有作逆者。」其後徐馥作亂。此草妖也,郭以為木不曲直。



 永嘉六年七月,豫章郡有樟樹久枯,是月忽更榮茂,與昌邑枯社復生同占。懷帝不終其祚,元帝由支族興之應也。



 晉明帝太寧元年九月,會稽剡縣木生如人面。是後王
 敦稱兵作逆,禍敗無成。



 漢哀、靈之世,並有此妖,而人貌備具,故其禍亦大。今此但人面而已,故其變亦輕。



 晉成帝咸和六年五月癸亥,曲阿有柳樹倒地六載,是月忽復起生。咸和九年五月甲戌,吳雄家有死榆樹,是日因風雨起生。與漢上林斷柳起生同象。初,康帝為吳王,于時雖改封瑯邪,而猶食吳郡為邑。是帝越正體饗國之象也。曲阿先亦吳地,象見吳邑雄舍,又天意也。



 晉哀帝興寧三年五月癸卯,廬陵西昌縣修明家有死
 慄樹,是日忽起生。時孝武年四歲,而簡文居蕃,四海宅心。及得位垂統,則祚隆孝武。識者竊曰,西昌修明之祥,帝諱實應之矣。是與漢宣帝頗同象也。



 晉海西太和元年,涼州楊樹生松。天戒若曰,松不改柯易葉,楊者柔脆之木,此永久之業,將集危亡之地。是後張天錫降氐。



 晉孝武太元十四年六月,建寧同樂縣枯木斷折,忽然自立相屬。京房《易傳》曰:「棄正作淫,厥妖木斷自屬。妃后
 有專,木仆反立。」是時治道方僻,多失其正。其後張夫人專寵,及帝崩,兆庶歸咎張氏焉。



 晉安帝元興三年,荊、江二界生竹實如麥。晉安帝義熙二年九月,揚州營揚武將軍營士陳蓋家有苦賣菜,莖高四尺六寸,廣三尺二寸。此殆與吳終同象也。義熙中,宮城上御道左右皆生蒺藜,草妖也。蒺藜有刺,不可踐而行,生宮牆及馳道,天戒若曰,人君拱默不能聽政,雖居宸極,猶若空宮;雖有御道,未嘗馳騁,皆生蒺藜
 若空廢也。義熙八年,太社生薰樹于壇側。薰於文尚黑,宋水德將王之符也。



 羽蟲之孽:魏文帝黃初四年五月,有鵜鶘鳥集靈芝池。案劉向說,此羽蟲之孽,又青祥也。



 詔曰:「此詩人所謂污澤者也。《曹詩》刺恭公遠君子,近小人。今豈有賢智之士,處于下位,否則斯鳥胡為而至哉?其博舉天下俊德茂才,獨行君
 子,以答曹人之刺。」



 於是楊彪、管寧之徒,咸見薦舉。此謂睹妖知懼者也。雖然不能優容亮直,而多溺偏私矣。京房《易傳》曰:「辟退有德,厥妖水鳥集于國井。」黃初末,宮中有生鷹,口爪俱赤。此與商紂、宋隱同象。



 景初元年,又有生鉅彀於衛國涓桃里李蓋家,形若鷹,吻似燕。案劉向說,此羽蟲之孽,又赤眚也。高堂隆曰:「此魏室之大異,宜防鷹揚之臣於蕭牆之內。」



 其後晉宣王起,遂有魏室。



 漢
 獻帝建安二十三年,禿鶖鳥集鄴宮文昌殿後池。明年,魏武王薨。



 魏文帝黃初三年,又集雒陽芳林園池。七年,又集。其夏,文帝崩。景初末,又集芳林園池。前世再至,輒有大喪,帝惡之。其年,明帝崩。



 蜀劉禪建興九年十月,江陽至江州有鳥從江南飛渡江北,不能達,墮水死者以千餘。是時諸葛亮連年動眾,志吞中夏,而終死渭南,所圖不遂。又諸將分爭,頗喪徒
 旅。鳥北飛不能達,墮水死者,皆有其象也。亮竟不能過渭,又其應乎!此與漢、楚國烏鬥墮泗水粗類矣。



 魏明帝青龍三年,戴頠巢鉅鹿人張臶家。臶博學有高節,不應袁紹、高幹之命,魏太祖辟亦不至,優游嘉遁,門徒數百,太守王肅雅敬焉。時年百餘歲,謂門人曰:「戴頠陽鳥,而巢于門陰,此凶祥也。」乃援琴歌詠,作詩一首,旬日而卒。按占,羽蟲之孽也。魏明帝景初元年,陵霄閣始構,有鵲巢其上。鵲體白黑
 雜色。此羽蟲之孽,又白黑祥也。帝以問高堂隆,對曰:「《詩》云:『惟鵲有巢,惟鳩居之。』今興起宮室,而鵲來巢,此宮室未成,身不得居之之象。天意若曰,宮室未成,將有它姓制御之,不可不深慮。」於是帝改容動色。



 吳孫權赤烏十二年四月,有兩烏銜鵲墮東館。權使領丞相朱據燎鵲以祭。案劉歆說,此羽蟲之孽,又黑祥也。視不明,聽不聰之罰也。是時權意溢德衰,信讒好殺,二子將危,將相俱殆。睹妖不悟,加之以燎,昧道之甚者也。
 明年,太子和廢,魯王霸賜死,朱據左遷,陸議憂卒,是其應也。東館,典教之府;鵲墮東館,又天意乎!



 吳孫權太元二年正月,封前太子和為南陽王,遣之長沙。有鵲巢其帆檣。和故宮僚聞之,皆憂慘,以為檣末傾危,非久安之象。是後果不得其死。



 吳孫亮建興二年十一月,大鳥五見于春申。吳人以為鳳凰,明年,改元為五鳳。



 漢桓帝時,有五色大鳥。司馬彪云:「政治衰缺,無以致鳳,乃羽蟲孽耳。」孫亮未有德政,孫
 峻驕暴方甚,此與桓帝同事也。案《瑞應圖》,大鳥似鳳而為孽者非一,疑皆是也。吳孫皓建衡三年,西苑言鳳凰集,以之改元。義同於亮。



 晉武帝泰始四年八月,翟雉飛上閶闔門。趙倫既篡,洛陽得異鳥,莫能名。倫使人持出,周旋城邑匝以問人。積日,宮西有小兒見之,逆自言曰:「服留鳥翳。」



 持者即還白倫。倫使更求小兒。至,又見之,將入宮,密籠鳥,閉兒戶中。明日視,悉不見。此羽蟲之孽,又妖之甚者也。



 趙倫篡位,有鶉入太極殿,雉集東堂。按太極、東堂,皆朝享聽政之所。而鶉、雉同日集之者,天意若曰,不當居此位也。《詩》云「鵲之疆疆,鶉之奔奔。人之無良,我以為君。」其此之謂乎!昔殷宗感雉雊,懼而修德;倫睹二物,曾不知戒,故至滅亡也。



 晉孝懷帝永嘉元年二月,洛陽東北步廣里地陷,有鵝出,蒼色者飛翔沖天,白者止焉。此羽蟲之孽,又黑白祥也。董養曰:「步廣,周之狄泉,盟會地也。白者金色,蒼為胡
 象,其可盡言乎。」是後劉淵、石勒相繼擅華,懷、愍二帝淪滅非所。



 晉孝懷帝世,周𤣱家有鵝在籠中,而頭斷籠外。𤣱亡後家誅。



 晉明帝太寧三年八月庚戌,有鳥二,蒼黑色,翼廣一丈四尺。其一集司徒府,射而殺之;其一集市北家人舍,亦獲焉。此羽蟲之孽,又黑祥也。閏月戊子,帝崩。



 後有蘇峻、祖約之亂。



 晉成帝咸和二年正月,有五鷗鳥集殿庭。此又白祥也。是時庾亮茍違眾謀,將召蘇峻,有言不從之咎,故白祥先見也。三年二月,峻果作亂,宮室焚毀,化為污萊,其應也。晉成帝咸康八年七月,白鷺集殿屋。是時康帝始即位,此不永之祥也。



 後涉再期而帝崩。劉向曰:「野鳥入處,宮室將空。」張瓘在涼州正朝,放隹雀諸鳥,出手便死;左右放者悉飛去。



 晉孝武帝太元十六年正月,鵲巢太極東頭鴟尾,又巢國子學堂西頭。十八年,東宮始成。十九年正月,鵲又巢其西門。此殆與魏景初同占。學堂,風教所聚;西門,金行之祥也。



 晉安帝義熙三年,龍驤將軍硃猗戍壽陽。婢炊飯,忽有群烏集灶,競來啄啖,婢驅逐不去。有獵狗咋殺烏鵲,餘者因共啄狗即死,又啖其肉,唯餘骨存。五年六月,猗死。



 宋武帝永初三年,臨軒拜徐羨之為司空,百僚陪位,有
 二野鸛集太極鴟尾鳴呼。



 少帝景平二年春,鸛巢太廟西鴟尾,驅去復還。



 文帝元嘉二年春,有江鷗鳥數百,集太極殿前小階內。明年,誅徐羨之等。



 羊禍:晉成帝咸和二年五月,司徒王導廄,羊生無後足。此羊禍也。京房《易傳》曰:「足少者,下不勝任也。」明年,蘇峻入京都,導與成帝俱幽石頭,僅乃免身。是其應也。



 宋孝武帝大明七年,永平郡獻三角羊。羊禍也。



 赤眚赤祥:公孫淵時,襄平北市生肉,長圍各數尺,有頭目口喙,無手足,而動搖。此赤眚也。占曰:「有形不成,有體無聲,其國滅亡。」淵尋為魏所誅。



 吳戍將鄧嘉殺豬祠神,治畢縣之,忽見一人頭往食肉。嘉引弓射中之,咋咋作聲,繞屋三日。近赤祥也。後人白嘉謀北叛,闔門被誅。京房《易妖》曰:「山見葆,江于邑,邑有
 兵,狀如人頭赤色。」吳諸葛恪將見誅,盥洗水血臭;侍者授衣,衣亦臭。此近赤祥也。



 晉武帝太康七年十一月,河陰有赤雪二頃。此赤祥也。後涉四載而帝崩,王宮遂亂。



 晉惠帝元康五年三月,呂縣有流血,東西百餘步。此赤祥也。元康末,窮凶極亂,僵尸流血之應也。干寶以為後八載而封雲亂徐州,殺傷數萬人,是其應也。
 晉惠帝永康元年三月,尉氏雨血。夫政刑舒緩,則有常燠赤祥之妖。此歲正月,送愍懷太子幽于許宮。天戒若曰,不宜緩恣姦人,將使太子冤死。惠帝愚眊不悟,是月愍懷遂斃。於是王室釁成,禍流天下。淖齒殺齊閔王日,天雨血沾衣,天以告也,此之謂乎?京房《易傳》曰:「歸獄不解,茲謂追非,厥咎天雨血,茲謂不親,民有怨心,不出三年,無其宗人。」又曰:「佞人祿,功臣戮,天雨血。」



 晉愍帝建興四年十二月丙寅,丞相府斬督運令史淳
 于伯,血逆流上柱二丈三尺。



 此赤祥也。是時後將軍褚裒鎮廣陵,丞相揚聲北伐,伯以督運稽留及役使臧罪,依征軍法戮之。其息訴稱:「伯督運事訖,無所稽乏,受賕役使,罪不及死。兵家之勢,先聲後實,實是屯戍,非為征軍。自四年以來,運漕稽停,皆不以軍興法論。」



 僚佐莫之理。及有此變,司直彈劾眾官,元帝又無所問。於是頻旱三年。干寶以為冤氣之應也。郭景純曰:「血者水類,同屬於《坎》,《坎》為法家。水平潤下,不宜逆流。此政有咎失之徵也。」



\end{pinyinscope}