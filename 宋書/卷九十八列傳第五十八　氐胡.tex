\article{卷九十八列傳第五十八 氐胡}

\begin{pinyinscope}

 略陽清水氐楊氏,秦、
 漢
 以來,世居隴右,為豪族。漢獻帝建安中,有楊騰者,為部落大帥。騰子駒,勇健多計略,始徙仇池。仇池地方百頃,因以百頃為號,四面斗絕,高平
 地方二十餘里,羊腸蟠道,三十六回。山上豐水泉,煮土成鹽。駒後有名千萬者,魏拜為百頃氐王。千萬子孫名飛龍,漸彊盛,晉武假征西將軍,還居略陽。無子,養外甥令狐氏子為子,名戊搜。晉惠帝元康六年,避齊萬年之亂,率部落四千家,還保百頃,自號輔國將軍、右賢王。關中人士奔流者多依之,戊搜延納撫接,欲去者則衛護資遣之。愍帝以為驃騎將軍、左賢王。時南陽王保在上邽,又以戊搜子難敵為征南將軍。建興五年,戊搜卒,難敵
 襲位。與堅頭分部曲,難敵號左賢王,屯下辯,堅頭號右賢王,屯河池。元帝太興四年,劉曜伐難敵,與堅頭俱奔晉壽,臣於李雄,曜退,復還仇池。



 成帝咸和九年,難敵卒,子毅立,自號使持節、龍驤將軍、左賢王、下辯公。



 以堅頭子槃為使持節、冠軍將軍、右賢王、河池公。咸康元年,遣使稱蕃於晉,以毅為征南,盤征東將軍。三年,毅族兄初襲殺毅,并有其眾,自立為仇池公,臣於石虎。後遣使稱蕃於穆帝。永和三年,以初為使持節、征南將軍、雍州刺
 史、平羌校尉、仇池公。初子國為鎮東將軍、武都太守。十年,改封初天水公。十一年,毅小弟宋奴使姑子梁式王因侍直手刃殺初,子國率左右誅式王及宋奴,復自立。征西將軍桓溫表國為鎮北將軍、秦州刺史、平羌校尉,國子安為振威將軍、武都太守。



 十二年,國從父楊俊復殺國自立,安奔苻生,俊遣使歸順。



 升平三年,以俊為平西將軍、平羌校尉、仇池公。四年,俊卒,子世立,復以為冠軍將軍、平羌校尉、武都太守、仇池公,海西公太和三年,
 遷征西將軍、秦州刺史,以世弟統為寧東將軍、武都太守。五年,世卒,統廢世子纂自立。纂一名德,聚黨殺統,遣使詣簡文帝自陳,復以纂為平羌校尉、秦州刺史、仇池公。咸安元年,苻堅遣楊安、苻雅等討纂克之,徙其民於關中,空百頃之地。纂後為楊安所殺。



 宋奴之死也,二子佛奴、佛狗奔逃關中,苻堅以佛奴為右將軍,佛狗為撫夷護軍。後以女妻佛奴子定,以定為尚書、領軍將軍。孝武帝太元八年,苻堅敗於淮南,關中擾亂,定盡力奉堅。
 堅死,乃將家奔隴右,徙治歷城,城在西縣界,去仇池百二十里。置倉儲於百頃。招合夷、晉,得千餘家,自號龍驤將軍、平羌校尉、仇池公,稱蕃於晉孝武帝,孝武帝即以其自號假之。求割天水之西縣、武都之上祿為仇池郡,見許。十五年,又以定為輔國將軍、秦州刺史,定已自署征西將軍。又進持節、都督隴右諸軍事、輔國大將軍、開府儀同三司,校尉、刺史如故。其年,進平天水略陽郡,遂有秦州之地,自號隴西王。至十九年,攻隴西虜乞佛乾歸,
 定軍敗見殺。無子,佛狗子盛先為監國,守仇池,襲位,自號使持節、征西將軍、秦州刺史、平羌校尉、仇池公。謚定為武王。分諸四山氐、羌為二十部護軍,各為鎮戍,不置郡縣。



 安帝隆安三年,遣使稱蕃,奉獻方物。安帝以盛為輔國將軍、平羌校尉、仇池公。元興三年,桓玄輔晉,進盛平北將軍、涼州刺史、西戎校尉。義熙元年。姚興伐盛,盛懼,遣子難當為質。興遣將王敏攻城,因梁州別駕呂瑩,求救於盛,盛遣軍次濜口,敏退。以盛為都督隴右諸軍
 事、征西大將軍、開府儀同三司。時益州刺史毛璩討桓玄所置梁州刺史桓希,敗走,漢中空虛,盛遣兄子平南將軍撫守漢中。



 三年,又假盛使持節、北秦州刺史。盛又遣將苻寧行梁州刺史代撫。九年,梁州刺史索邈鎮南城,寧乃還。高祖踐阼,進盛車騎大將軍,加侍中。永初三年,改封武都王,以長子玄為武都王世子,加號前將軍,難當為冠軍將軍,撫為安南將軍。盛嗣位三十年,太祖元嘉二年六月卒,時年六十二,私謚曰惠文王。



 玄字黃
 眉,自號使持節、都督隴右諸軍事、征西大將軍、開府儀同三司、平羌校尉、秦州刺史、武都王。雖為蕃臣,猶奉義熙之號。善待士,為流、舊所懷。安南將軍撫有文武智略,玄不能容,三年,因其子殺人,并誅之。明帝即以玄為使持節、征西將軍、平羌校尉、北秦州刺史、武都王。乃改義熙之號,奉元嘉正朔。初,盛謂玄曰:「吾年已老,當為晉臣,汝善事宋帝。」故玄奉焉。追贈盛驃騎大將軍,餘如故。六年六月,玄卒,私謚曰孝昭王。



 弟難當廢玄子保宗,一名
 羌奴而自立,號使持節、都督雍涼諸軍事、秦州刺史、平羌校尉、武都王。太祖以為冠軍將軍、秦州刺史、武都王。九年,進號征西將軍,加持節、都督、校尉之號。難當拜保宗為鎮南將軍,鎮宕昌;以次子順為鎮東將軍、秦州刺史,守上邽。保宗謀襲難當,事泄,收繫之。先是,四方流民有許穆之、郝恢之二人投難當,並改姓為司馬。穆之自云名飛龍,恢之自云名康之。云是晉室近戚,康之尋為人所殺。十年,難當以益州刺史劉道濟失蜀土人情,以
 兵力資飛龍,使入蜀為寇,道濟擊斬之。時梁州刺史甄法護刑法不理,太祖遣刺史蕭思話代任。



 難當因思話未至,法護將軍下,舉兵襲梁州,破白馬,獲晉昌太守張範。法護遣參軍魯安期、沈法慧等拒之,並各奔退。難當又遣建忠將軍趙進攻葭萌,獲晉壽太守范延郎。其年十一月,法護委鎮奔洋川,難當遂有漢中之地。以氐苻粟持為梁州刺史,又以其凶悍,殺之,以司馬趙溫代為梁州。十年正月,思話使司馬蕭承之先驅進討,所向剋捷,
 遂平梁州,事在《思話傳》。四月,難當遣使奉表謝罪,曰:臣聞生成之德,含氣同係,而榮悴殊塗,遭遇異兆,至於恩降自然,誠無答謝。



 夫以狂聖道隔,猶存克念之誠,況君親莫二,不期自感者哉!每思自竭,奉遵光訓,丹誠未諒,大謗已臻。梁州刺史甄法護誣臣遣司馬飛龍擾亂西蜀,諸所譖引,言非一事,長塗萬里,無路自明,風塵之聲,日有滋甚。與其逆生,寧就清滅,文武同憤,制不自由。遣參軍姚道賢齎書詣梁州刺史蕭思話,尋續又遣詣臺
 歸罪。道賢至西城,為守兵所殺,行李蔽擁,日月莫照。法護恇擾,望風奔逃,臣即回軍,秋毫無犯,權留少守,以俟會通。其後數旬,官軍尋至,守兵單弱,懼不自免,續遣輕兵,共相迎接。值秦流民,懷土及本,行將既旋,不容禁制,由臣約防無素,以致斯闕。



 臣本歷代守蕃,世荷殊寵,王化始基,順天委命,要名期義,不在今日,豈可假托妖妄,毀敗成功,如此之形,灼然易見,仰恃聖明,必垂鑒察。但臣微心不達,跡違忠順,至乃聲聞朝庭,勞煩師旅,負辱
 之深,罪當誅責。遠隔遐荒,告謝無地,謹遣兼長史齊亮聽命有司,并奉送所授第十一符策,伏待天旨。



 太祖以其邊裔,下詔曰:「楊難當表如此,悔謝前愆,可特恕宥,并特還章節。」



 十二年,難當釋保宗,遣鎮童亭。保宗奔,索虜主拓跋燾以為都督隴西諸軍事、征西大將軍、開府儀同三司、平羌校尉、南秦王,遣襲上邽。難當子順失守,退,以為雍州刺史,守下辯。十三年三月,難當自立為大秦王,號年曰建義,立妻為王后,世子為太子,置百官,具擬
 天朝;然猶奉朝庭,貢獻不絕。十七年,其國大旱,多災異,降大秦王復為武都王。



 十八年十月,傾國南寇,規有蜀土,慮漢中軍出,遣建忠將軍苻沖出東洛以防之。梁州刺史劉道真擊斬沖。十一月,難當剋葭萌,獲晉壽太守申坦,遂圍涪城。



 巴西太守劉道銀嬰城固守,難當攻之十餘日,不剋,乃還。十九年正月,太祖遣龍驤將軍裴方明、太子左積弩將軍劉康祖、後軍參軍梁坦甲士三千人,又發荊、雍二州兵討難當,受劉道真節度。五月,方明
 等至漢中,長驅而進。道真到武興,攻偽建忠將軍苻隆,剋之。安西參軍韋俊、建武將軍姜道盛別向下辯,道真又遣司馬夏侯穆季西取白水,難當子雍州刺史順、建忠將軍楊亮拒之,並望風奔走。閏月,方明至蘭皋,難當鎮北將軍苻義德、建節將軍苻弘祖萬餘人列陣拒戰,方明擊破之,斬弘祖,殺二千餘人,義德遁去。天水任愈之率部曲歸順。難當世子撫軍大將軍和據修城,方明又遣軍率愈之攻和,大破之。於是難當將妻子奔索虜,
 死于虜中。安西參軍魯尚期追難當出寒峽,生禽建節將軍楊保熾、安昌侯楊虎頭。初,難當遣第二子虎為鎮南將軍、益州刺史,守陰平。聞父走,逃還,至下辯。方明使子肅之要之,生禽虎,傳送京師,斬于建康市。



 仇池平。以輔國司馬胡崇之為龍驤將軍、秦州刺史、平羌校尉,守仇池。索虜拓跋燾遣安西大將軍吐奚弼、平北將軍拓跋齊等二萬人邀崇之。二十年二月,崇之至濁水,去仇池八十里,遇齊等,戰敗沒,餘眾奔還漢中。



 三月,前鎮東
 司馬苻達、征西從事中郎任朏等舉義,立保宗弟文德為主。拓跋齊聞兵起遁走,達追擊斬齊,因據白崖,分平諸戍。文德自號使持節、都督秦河涼三州諸軍事、征西大將軍、秦河涼三州牧、平羌校尉、仇池公,遣露板馳告朝廷。



 太祖詔曰:「近者校尉仇池公表虜縱逸,寇竊仇池,將士挫傷,民萌塗炭,眷言西顧,矜慨在懷。楊文德世篤忠順,誠感家國,糾率義徒,奄殄凶醜,鋒旗所向,殲潰無遺,氛昆澄清,蕃境寧一,念功惟事,良有欣嘉。便可遣使
 慰勞,宣示朝旨,并敕梁州刺史申坦隨宜應援。」又詔曰:「顯錄勳效,蓋惟國典,施賞務速,無或踰時。楊文德志氣果到,文武兼全,乘機潛奮,殊功仍集,告捷歸誠,獻俘萬里,朝無暫土,樹難自肅,休烈昭著,朕甚嘉焉。楊氏世祖西勞,方忠累葉,宜紹先緒,膺受寵榮。可使持節、散騎常侍、都督北秦雍二州諸軍事、征西大將軍、平羌校尉、北秦州刺史,封武都王。」任朏祖父岐,伯父祚,父綜,並仕楊氏,為諮議從事中郎。朏有志幹,文德以為左司馬。



 文德
 既受朝命,進戍茄蘆城。二十五年,為索虜所攻,奔于漢中。時世祖鎮襄陽,執文德歸之于京師,以失守,免官,削爵土。二十七年,王師北討,起文德為輔國將軍,率軍自漢中西入,搖動汧、隴。文德宗人楊高率陰平、平武群氐,據唐魯橋以拒文德,文德水陸俱攻,大破之,眾並奔散。高遁走奔羌,文德追之至黎仰嶺,高單身投羌仇阿弱家,追斬之,陰平、平武悉平。又遣文德伐啖提氐,不剋,梁州刺史劉秀之執送荊州,使文德從祖兄頭戍茄蘆。荊
 州刺史南郡王義宣反,文德不同見殺,世祖追贈征虜將軍、秦州刺史。



 孝建二年,以保宗子元和為征虜將軍,以頭為輔國將軍。元和繼楊氏正統,群氐欲相宗推,年小才弱,不能綏御所部,頭母妻子弟並為索虜所執,頭至誠奉順,無所顧懷。朝廷既不正元和號位,部落未有定主,雍州刺史王玄謨上表曰:「被敕令臣遣使與楊元和、楊頭相聞,并致信餉。即遣中軍行參軍呂智宗齎書并信等,亦自遣使隨智宗。及頭語智宗,頃破家為國,母妻
 子弟并墜沒虜中,不顧孝道,陳力邊捍,竭忠盡誠,未為朝廷所識。若以元和承統,宜授王爵;若以其年小未堪大任,則應別有所委。頃來公私紛紜,華、戎交構,皆此之由。臣伏尋頭元嘉以來,實有忠誠於國,棄親遺愛,誠在可嘉。氐、羌負遠,又與虜咫尺,急之則反,緩之則怨。



 觀頭使人言語,不敢便望仇池公,所希政在西秦州假節而已。如臣愚見,蕃捍漢川,使無慮患,頭實有力,四千戶荒州,殆不足吝。元和小弱,若未可專委。復數年之後,必堪
 嗣業,用之不難。若才用不稱,則應歸頭。若茄蘆不守,漢川亦無立理。」



 上不許。其後立元和為武都王,治白水,不能自立,復走奔索虜。



 元和從弟僧嗣,復自立,還戍茄蘆,以為寧朔將軍、仇池太守。太宗泰始二年,詔曰:「僧嗣遠守西疆,世篤忠款,宜加旌顯,以甄義概。可冠軍將軍、北秦州刺史、武都王,太守如故。」三年,加持節、都督北秦雍二州諸軍事,進號征西將軍、校尉,刺史如故。僧嗣卒,從弟文度復自立。泰豫元年,以為龍驤將軍、略陽太守,封
 武都王,又改龍驤為寧朔將軍。



 後廢帝元徽四年,加督北秦州諸軍事、平羌校尉、北秦州刺史、將軍如故。文度遣弟龍驤將軍文弘伐仇池,破戍兵於蘭皋。順帝昇明元年,詔曰:「茂賞有章,實昭國度,疇庸斯炳,載宣史冊。督北秦州諸軍事、寧朔將軍、平羌校尉、北秦州刺史、武都王文度門乘輝寵,世榮邊邑,忠果既亮,才勁兼彰。龍驤將軍楊文弘肅協成規,躬提桴鼓,申棱百頃,席卷蘭皋,功烈之美,並足嘉歎,宜膺爵授,以酬勳緒。文度可使持
 節、都督北秦雍二州諸軍事、征西將軍,刺史、校尉悉如故。文弘輔國將軍、略陽太守。」其年,虜破茄蘆,文度見殺,追贈本官,加散騎常侍。



 以文弘督北秦州諸軍事、平羌校尉、北秦州刺史,襲封武都王,將軍如故。退治武興。



 大且渠蒙遜,張掖臨松盧水胡人也。匈奴有左且渠、右且渠之官,蒙遜之先為此職,羌之酋豪曰大,故且渠以位為氏,而以大冠之。世居盧水為酋豪。蒙遜高祖暉仲歸,曾祖遮,皆雄健有勇名。祖祁復延,封狄地王。父法弘
 襲爵,苻氏以為中田護軍。



 蒙遜代父領部曲,有勇略,多計數,為諸胡所推服。呂光自王於涼州,使蒙遜自領營人配箱直,又以蒙遜叔父羅仇為西平太守。安帝隆安三年春,呂光遣子鎮東將軍纂率羅仇伐桴罕虜乞佛乾歸,為乾歸所敗,光委罪羅仇,殺之。四月,蒙遜求還葬羅仇,因聚萬餘人叛光,殺臨松護軍,屯金山。五月,光揮纂擊破蒙遜,蒙遜將六七人,逃山中,家戶悉亡散。時蒙遜兄男成將兵西守晉昌,聞蒙遜反,引軍還,殺酒泉太守疊
 滕,推建康太守段業為主。業自號龍驤大將軍、涼州牧、建康公,以男成為輔國將軍。男成及晉昌太守王德圍張掖,剋之,業因據張掖。蒙遜率部曲投業,業以蒙遜為鎮西將軍、臨池太守,王德為酒泉太守。尋又以蒙遜領張掖太守。



 三年四月,業使蒙遜將萬人攻光弟子純於西郡,經旬不剋,乃引水灌城,窘急乞降,執之以歸。時王德叛業,自稱河州刺史。業使蒙遜西討,德焚城,將部曲走投晉昌太守唐瑤;蒙遜追德至沙頭,大破之,虜其妻
 子部落而還。轉西安太守,將軍如故。四年五月,蒙遜與男成謀殺業,男成不許,蒙遜反譖男成於業,業殺男成。



 蒙遜乃謂其部曲曰:「段公無道,枉殺輔國。吾為輔國報仇。」遂舉兵攻張掖,殺段業,自稱車騎大將軍,建號永安元年。



 是月,敦煌太守李皓亦起兵,自號冠軍大將軍、西胡校尉、沙州刺史,太守如故。稱庚子元年,與蒙遜相抗。其冬,皓遣唐瑤及鷹揚將軍宋繇攻酒泉,獲太守大且渠益生,蒙遜從叔也。



 呂光死,子纂立。元年,為從弟隆所
 篡。姚興攻涼州,隆稱臣請降,蒙遜亦遣使詣興,興以為鎮西將軍、沙州刺史、西海侯。二年二月,蒙遜與西平虜禿發傉檀共攻涼州,為隆所破。十月,傉檀復攻隆。三年三月,隆以蒙遜;傉檀交逼,遣弟超詣姚興求迎。七月,興遣將齊難迎隆,隆說難伐蒙遜,蒙遜懼,遣弟為質,獻寶貨於難,乃止,以武衛將軍王尚行涼州刺史而還。



 義熙元年正月,李皓改稱大將軍、大都督、涼州牧、護羌校尉、涼公;五月,移據酒泉。姚興假傉檀涼州刺史,代王尚屯
 姑臧。二年九月,蒙遜襲李皓,至安彌,去城六十里,皓乃覺。引軍出戰,大敗,退還,閉城自守,蒙遜亦歸。六年,蒙遜攻破傉檀,傉檀走屯樂都。武威人焦朗入姑臧,自號驃騎大將軍,臣于李皓。八年,蒙遜攻焦朗,殺之。據姑臧,自號大都督、大將軍、河西王,改稱玄始元年,立子正德為世子。



 十三年五月,李皓死,子歆立。六月,歆伐蒙遜,至建康,蒙遜拒之,歆退走,追到西支澗,蒙遜大敗,死者四千餘人,乃收餘眾,增築建康城,置兵戍而還。



 十四年,蒙遜
 遣使詣晉,奉表稱蕃,以蒙遜為涼州刺史。高祖踐阼,以歆為使持節、都督高昌敦煌晉昌酒泉西海玉門堪泉七郡諸軍事、護羌校尉、征西大將軍、酒泉公。



 永初元年七月,蒙遜東略浩釁,李歆乘虛攻張掖;蒙遜回軍西歸,歆退走,追至臨澤,斬歆兄弟三人,進攻酒泉,剋之。歆弟敦煌太守恂據郡,自稱大將軍。十月,蒙遜遣世子正德攻恂,不下。三年正月,蒙遜自往築長堤引水灌城,數十日,又不下。三月,恂武衛將軍宋丞、廣武將軍弘舉城降,
 恂自殺,李氏由是遂亡。於是鄯善王比龍入朝,西域三十六國皆稱臣貢獻。



 高祖以蒙遜為使持節、散騎常侍、都督涼州諸軍事、鎮軍大將軍、開府儀同三司、涼州刺史、張掖公。十二月,晉昌太守唐契反,復遣正德攻契。景平元年三月,克之,契奔伊吾。八月,芮芮來抄,蒙遜遣正德距之,正德輕騎進戰,軍敗見殺。



 乃以次子興國為世子。是歲,進蒙遜侍中、都督涼秦河沙四州諸軍事、驃騎大將軍、領護匈奴中郎將、西夷校尉、涼州牧,河西王,開
 府、持節如故。



 太祖元嘉元年,桴罕虜乞佛熾盤出貂渠谷攻河西白草嶺,臨松郡皆沒,執蒙遜從弟成都、從子日蹄、頗羅等而去。三年,改驃騎為車騎。世子與國遣使奉表,請《周易》及子集諸書,太祖並賜之,合四百七十五卷。蒙遜又就司徒王弘求《搜神記》,弘寫與之。六年,蒙遜征桴罕,時乞佛熾盤死矣,子茂蔓大破蒙遜,生禽興國,殺三千殺人。蒙遜贖興國,遂穀三十萬斛,竟不遣。蒙遜乃立興國母弟菩提為世子,朝廷未知也。七年,以興國
 為冠軍將軍、河西王世子。其年夏四月,西虜赫連定為索虜拓跋燾所破,奔上邽。十一月,茂蔓聞定敗,將家戶及興國東征,欲移居上邽。八年正月至南安,定率眾御茂蔓,大破之,殺茂蔓,執興國而還。四月,定避拓跋燾,欲渡河西擊蒙遜。五月,率部曲至治城峽口,渡河,濟未半,為吐谷渾慕璝所邀,見獲,興國被創數日死。



 九年,以菩提為冠軍將軍、河西王世子。十年四月,蒙遜卒,時年六十六。私謚曰武宣王。菩提年幼,蒙遜第三子茂虔時為
 酒泉太守,眾議推茂虔為主,襲蒙遜位號。十一年,茂虔上表曰:「臣聞功以濟物為高,非竹帛無以述德,名以當實為美,非謚號無以休終。先臣蒙遜西復涼城,澤憺昆裔,芟夷群暴,清灑區夏。暨運鐘有道,備大宋之宗臣,爵班九服,享惟永之丕祚,功名昭著,剋固貞節。考終由正,而請名之路無階,懿跡雖弘,而述敘之美有缺。臣子痛感,咸用不安。謹案謚法,剋定禍亂曰武,善聞周達曰宣。先臣廓清河外,勳光天府,標榜稱跡,實兼斯義。輒上謚
 為武宣王。若允天聽,垂之史筆,則幽顯荷榮,始終無恨。」詔曰:「使持節、侍中、都督秦河沙涼四州諸軍事、車騎大將軍、開府儀同三司、領護匈奴中郎將、西夷校尉、涼州牧河西王蒙遜,才兼文武,勳濟西服,爰自萬里,款誠夙著,方仗忠果,翼宣遠略,奄至薨隕,悽悼于懷。便遣使吊祭,并加顯謚。嗣子茂虔,纂戎前軌,乃心彌彰,宜蒙寵授,紹茲蕃業。可持節、散騎常侍、都督涼秦河沙四州諸軍事、征西大將軍、領護匈奴中郎將、西夷校尉、涼州刺史、
 河西王。」



 河西人趙匪又善歷算。十四年,茂虔奉表獻方物,並獻《周生子》十三卷,《時務論》十二卷,《三國總略》二十卷,《俗問》十一卷,《十三州志》十卷,《文檢》六卷,《四科傳》四卷,《敦煌實錄》十卷,《涼書》十卷,《漢皇德傳》二十五卷,《亡典》七卷,《魏駮》九卷,《謝艾集》八卷,《古今字》二卷,《乘丘先生》三卷,《周髀》一卷,《皇帝王歷三合紀》一卷,《趙匪又傳》并《甲寅元歷》一卷,《孔子贊》一卷,合一百五十四卷。茂虔又求晉、趙《起居注》諸雜書數十件,太祖賜之。



 十六年閏八月,拓跋燾攻
 涼州,茂虔兄子萬年為虜內應,茂虔見執。茂虔弟安彌縣侯無諱先為征西將軍、沙州刺史、都督建康以西諸軍事、酒泉太守,第六弟武興縣侯儀德為征東將軍、秦州刺史、都督丹嶺以西諸軍事、張掖太守。燾既獲茂虔,遣軍擊儀德,棄城奔無諱。於是無諱、儀德擁家戶西就從弟敦煌太守唐兒。燾使將守武威、酒泉、張掖而還。十七年正月,無諱使唐兒守敦煌,自與儀德伐酒泉,三月,剋之。攻張掖、臨松,得四萬餘戶,還據酒泉。



 十八年五月,
 唐兒反,無諱留從弟天周守酒泉,復與儀德討唐兒。唐兒將萬餘人出戰,大敗,執唐兒殺之,復據敦煌。七月,拓跋燾遣軍圍酒泉。十月,城中饑,萬餘口皆餓死,天周殺妻以食戰士;食盡,城乃陷,執天周至平城,殺之。于時虜兵甚盛,無諱眾饑,懼不自立,欲引眾西行。十一月,遣弟安周五千人伐鄯善,堅守不下。十九年四月,無諱自率萬餘家棄敦煌,西就安周,未至而鄯善王比龍將四千餘家走,因據鄯善。初,唐契自晉昌奔伊吾,是年攻高昌,
 高昌城主闕爽告急。



 八月,無諱留從子豐周守鄯善,自將家戶赴之。未至,而芮芮遣軍救高昌,殺唐契,部曲奔無諱。九月,無諱遣將衛崿夜襲高昌,爽奔芮芮,無諱復據高昌。



 遣常侍氾俊奉表使京師,獻方物。太祖詔曰:「往年狡虜縱逸,侵害涼土,西河王茂虔遂至不守,淪陷寇逆,累世著誠,以為矜悼。次弟無諱克紹遺業,保據方隅,外結鄰國,內輯民庶,係心闕庭,踐修貢職,宜加朝命,以褒篤勳。可持節、散騎常侍、都督涼河沙三州諸軍事、征
 西大將軍、領護匈奴中郎將、西夷校尉、涼州刺史、河西王。」



 無諱卒,弟安周立。二十一年,詔曰:「故征西大將軍、河西王無諱弟安周,才略沈到,世篤忠疑,統承遺業,民眾歸懷。雖亡士喪師,孤立異所,而能招率殘寡,攘寇自今,宜加榮授,垂軌先烈。可使持節、散騎常侍、都督涼河沙三州諸軍事、領西域戊己校尉、涼州刺史、河西王。」世祖大明三年,安周奉獻方物。



 史臣曰:氐藉世業之資,胡因倔起之眾,結根百頃,跨有
 河西,雖戎夷猾夏,自擅荒服,而財力雄富,頗尚禮文。楊氏兵精地險,境接華漢,伺隙邊關,首鼠疆場,遂西入白馬,東出黃金,乘晉燾之捷,構圍涪之釁,規吞黑水,志傾井絡,紀、郢之勢方危,樊、鄧之心屢駭。天子聽朝不怡,有懷辛、李之將,而齊之宣皇,率偏旅數百,定命先驅,推鋒直指,勢踰風電,雲徹席卷,致屆南城,逐北追奔,全勝萬里,敵人皆裹骨輿屍,越至險而自竄,其餘皆膏身山野,委骸川澤。既而裴、劉二將,藉其威聲,故使濁水靡旗,蘭
 皋失險,氐族轉徙奔亡,遺燼不滅者若線,梁土獲乂,以迄於今。由此而言,功烈可謂盛矣!



\end{pinyinscope}