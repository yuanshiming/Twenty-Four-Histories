\article{卷九十七列傳第五十七 夷蠻}

\begin{pinyinscope}

 南夷、西南夷,大抵在交州之南及西南,居大海中洲上,相去或三五千里,遠者二三萬里,乘舶舉帆,道里不可詳知。外國諸夷雖言里數,非定實也。



 南夷林邑國,高祖永初二年,林邑王范陽邁遣使貢獻,即加除授。太祖元嘉初,侵暴日南、九德諸郡,交州刺史杜弘文建牙聚眾欲討之,聞有代,乃止。七年,陽邁遣使自陳與交州不睦,求蒙恕宥。八年,又遣樓船百餘寇九德,入四會浦口,交州刺史阮彌之遣隊主相道生三千人赴討,攻區粟城不剋,引還。林邑欲伐交州,借兵於扶南王,扶南不從。十年,陽邁遣使上表獻方物,求領交州,詔答以道遠,不許。十二、十五、十六、十八年,頻遣貢獻,而
 寇盜不已,所貢亦陋薄。



 太祖忿其違傲,二十三年,使龍驤將軍、交州刺史檀和之伐之,遣太尉府振武將軍宗愨受和之節度。和之遣府司馬蕭景憲為前鋒,愨仍領景憲軍副。陽邁聞將見討,遣使上表,求還所略日南民戶,奉獻國珍。太祖詔和之:「陽邁果有款誠,許其歸順。」其年二月,軍至朱梧戍,遣府戶曹參軍日南太守姜仲基、前部賊曹參軍蟜弘民隨傳詔畢願、高精奴等宣揚恩旨,陽邁執仲基、精奴等二十八人,遣弘民反命,外言歸
 款,猜防愈嚴。景憲等乃進軍向區粟城,陽邁遣大帥范扶龍大戍區粟,又遣水步軍徑至。景憲破其外救,盡銳致城。五月,剋之,斬扶龍大首,獲金銀雜物不可勝計。乘勝追討,即克林邑,陽邁父子並挺身奔逃,所獲珍異,皆是未名之寶。上嘉將帥之功,詔曰:「林邑介恃遐險,久稽王誅。龍驤將軍、交州刺史檀和之忠果到列,思略經濟,稟命致討,萬里推鋒,法命肅齊,文武畢力,潔己奉公,以身率下,故能立勳海外,震服殊俗。宜加褒飾,參管近侍,
 可黃門侍郎,領越騎校尉、行建武將軍。龍驤司馬蕭景憲協贊軍首,勤捷顯著,總勒前驅,克殄巢穴,必能威服荒夷,撫懷民庶。可持節、督交州、廣州之鬱林、寧浦二郡諸軍事、建威將軍、交州刺史。龍驤司馬童林之、九真太守傅蔚祖戰死,並贈給事中。」



 世祖孝建二年,林邑又遣長史范龍跋奉使貢獻,除龍跋揚武將軍。大明二年,林邑王范神成又遣長史范流奉表獻金銀器及香布諸物。太宗泰豫元年,又遣使獻方物。初,檀和之被徵至豫
 章,值豫章民胡誕世等反,因討平之,并論林邑功,封雲杜縣子,食邑四百戶。和之,高平金鄉人,檀憑子也。太祖元嘉二十七年,自太子左衛率為世祖鎮軍司馬、輔國將軍、彭城太守。元凶弒立,以為西中郎將、雍州刺史。世祖入討,加輔國將軍,統豫州戍事,因出南奔。世祖即位,以為右衛將軍。



 孝建二年,除輔國將軍、豫州刺史,不行,復為右衛,加散騎常侍。三年,出為南兗州刺史,坐酣飲黷貨,迎獄中女子入內,免官禁錮。其年卒,追贈左將軍。
 謚曰襄子。



 廣州諸山並俚、獠,種類繁熾,前後屢為侵暴,歷世患苦之。世祖大明中,合浦大帥陳檀歸順,拜龍驤將軍。四年,檀表乞官軍征討未附,乃以檀為高興太守,將軍如故。遣前硃提太守費沈、龍驤將軍武期率眾南伐,并通朱崖道,並無功,輒殺檀而反,沈下獄死。



 扶南國,太祖元嘉十一、十二、十五年,國王持黎跋摩遣使奉獻。



 西南夷訶羅駝國,元嘉七年,遣使奉表曰:伏承聖主,信
 重三寶,興立塔寺,周滿國界。城郭莊嚴,清凈無穢,四衢交通,廣博平坦。臺殿羅列,狀若眾山,莊嚴微妙,猶如天宮。聖王出時,四兵具足,導從無數,以為守衛。都人士女,麗服光飾,市廛豐富,珍賄無量,王法清整,無相侵奪。學徒遊集,三乘競進,敷演正法,雲布雨潤。四海流通,萬國交會,長江眇漫,清凈深廣,有生咸資,莫能銷穢,陰陽調和,災厲不行。誰有斯美,大宋揚都,聖王無倫,臨覆上國。有大慈悲,子育萬物,平等忍辱,怨親無二,濟乏周窮,無
 所藏積,靡不照達,如日之明,無不受樂,猶如凈月。宰輔賢良,群臣貞潔,盡忠奉主,心無異想。



 伏惟皇帝,是我真主。臣是訶羅駝國王,名曰堅鎧,今敬稽首聖王足下,惟願大王知我此心久矣,非適今也。山海阻遠,無緣自達,今故遣使,表此丹誠。所遣二人,一名毗紉,一名婆田,今到天子足下。堅鎧微蔑,誰能知者,是故今遣二人,表此微心,此情既果,雖死猶生。仰惟大國,籓守曠遠,我即邊方籓守之一。上國臣民,普蒙慈澤,願垂恩逮,等彼僕臣。
 臣國先時人眾殷盛,不為諸國所見陵迫,今轉衰弱,鄰國競侵。伏願聖王,遠垂覆護,并市易往反,不為禁閉。若見哀念,願時遣還,令此諸國,不見輕侮,亦令大王名聲普聞,扶危救弱,正是今日。今遣二人,是臣同心,有所宣啟,誠實可信。願敕廣州時遣舶還,不令所在有所陵奪。



 願自今以後,賜年年奉使。今奉微物,願垂哀納。



 呵羅單國,治闍婆洲。元嘉七年,遣使獻金剛指鈽、赤鸚鵡鳥、天竺國白壘古貝、葉波國古貝等物。十年,呵羅單
 國王毗沙跋摩奉表曰:常勝天子陛下:諸佛世尊,常樂安隱,三達六通,為世間道,是名如來,應供正覺,遺形舍利,造諸塔像,莊嚴國土,如須彌山,村邑聚落,次第羅匝,城郭館宇,如忉利天宮,宮殿高廣,樓閣莊嚴,四兵具足,能伏怨敵,國土豐樂,無諸患難。奉承先王,正法治化,人民良善,慶無不利,處雪山陰,雪水流注,百川洋溢,八味清凈,周匝屈曲,順趣大海,一切眾生,咸得受用。於諸國土,殊勝第一,是名震旦,大宋揚都,承嗣常勝大王之業,
 德合天心,仁廕四海,聖智周備,化無不順,雖人是天,護世降生,功德寶藏,大悲救世,為我尊主常勝天子。是故至誠五體敬禮。呵羅單國王毗沙跋摩稽首問訊。



 其後為子所纂奪。十三年,又上表曰:大吉天子足下:離淫怒癡,哀愍群生,想好具足,天龍神等,恭敬供養,世尊威德,身光明照,如水中月,如日初囗間自豪,普照十方,其白如雪,亦如月光,清凈如華,顏色照耀,威儀殊勝,諸天龍神之所恭敬,以正法寶,梵行眾僧,莊嚴國土,人民熾盛,
 安隱快樂。城閣高峻,如乾他山,眾多勇士,守護此城,樓閣莊嚴,道巷平正,著種種衣,猶如天服,於一切國,為最殊勝吉。揚州城無憂天主,愍念群生,安樂民人,律儀清凈,慈心深廣,正法治化,共養三寶,名稱遠至,一切並聞。民人樂見,如月初生,譬如梵王,世界之主,一切人天,恭敬作禮。呵羅單跋摩以頂禮足,猶如現前,以體布地,如殿陛道,供養恭敬,如奉世尊,以頂著地,曲躬問訊。



 忝承先業,嘉慶無量,忽為惡子所見爭奪,遂失本國。今唯一
 心歸誠天子,以自存命。今遣毗紉問訊大家,意欲自往,歸誠宣訴,復畏大海,風波不達。今命得存,亦由毗紉此人忠志,其恩難報。此是大家國,今為惡子所奪,而見驅擯,意頗忿惋,規欲雪復。伏願大家聽毗紉買諸鎧仗袍襖及馬,願為料理毗紉使得時還。前遣闍邪仙婆羅訶,蒙大家厚賜,悉惡子奪去,啟大家使知。今奉薄獻,願垂納受。



 此後又遣使。二十六年,太祖詔曰:「訶羅單、媻皇、媻達三國,頻越遐海,款化納貢,遠誠宜甄,可並加除授。」乃
 遣使策命之曰:「惟汝慕義款化,效誠荒遐,恩之所洽,殊遠必甄,用敷典章,顯茲策授。爾其欽奉凝命,永固厥職,可不慎歟。」二十九年,又遣長史媻和沙彌獻方物。



 媻皇國,元嘉二十六年,國王舍利媻羅跋摩遣使獻方物四十一種,太祖策命之為媻皇國王曰:「惟爾仰政邊城,率貢來庭,皇澤凱被,無幽不洽。宜班典策,授茲嘉命。爾其祗順禮度,式保厥終,可不慎歟。」二十八年,復貢獻。世祖孝建三年,又遣長史竺那媻智奉表獻方物。以那
 媻智為振威將軍。大明三年,獻赤白鸚鵡。



 大明八年、太宗泰始二年,又遣使貢獻。太宗以其長史竺須羅達、前長史振威將軍竺那媻智並為龍驤將軍。



 媻達國,元嘉二十六年,國王舍利不陵伽跋摩遣使獻方物。太祖策命之為婆婆達國王曰:「惟爾仰化懷誠,馳慕聲教,皇風遐暨,荒服來款,是用加茲顯策,式甄義順。爾其祗順憲典,永終休福,可不慎歟。」二十六年、二十八年,復遣使獻方物。



 闍婆婆達國,元嘉十二年,國王師黎婆達駝阿羅跋摩遣使奉表曰:宋國大主大吉天子足下:敬禮一切種智安隱,天人師降伏四魔,成等正覺,轉尊法輪,度脫眾生,教化已周,入于涅槃,舍利流布,起無量塔,眾寶莊嚴,如須彌山,經法流布,如日照明,無量凈僧,猶如列宿。國界廣大,民人眾多,宮殿城郭,如忉利天宮。名大宋揚州大國大吉天子,安處其中,紹繼先聖,王有四海,閻浮提內,莫不來服。悉以茲水,普飲一切,我雖在遠,亦霑靈潤,是
 以雖隔巨海,常遙臣屬,願照至誠,垂哀納受。若蒙聽許,當年遣信,若有所須,惟命是獻,伏願信受,不生異想。今遣使主佛大駝婆、副使葛抵奉宣微誠,稽首敬禮大吉天子足下,駝婆所啟,願見信受,諸有所請,唯願賜聽。今奉微物,以表微心。



 師子國,元嘉五年,國王剎利摩訶南奉表曰:謹白大宋明主,雖山海殊隔,而音信時通。伏承皇帝道德高遠,覆載同於天地,明照齊乎日月,四海之外,無往不伏,方國
 諸王,莫不遣信奉獻,以表歸德之誠。



 或泛海三年,陸行千日,畏威懷德,無遠不至。我先王以來,唯以修德為正,不嚴而治,奉事三寶,道濟天下,欣人為善,慶若在己,欲與天子共弘正法,以度難化。



 故託四道人遣二白衣送牙臺像以為信誓,信還,願垂音告。



 至十二年,又復遣使奉獻。



 天竺迦毗黎國,元嘉五年,國王月愛遣使奉表曰:伏聞彼國,據江傍海,山川周固,眾妙悉備,莊嚴清凈,猶如化
 城,宮殿莊嚴,街巷平坦,人民充滿,歡娛安樂。聖王出遊,四海隨從,聖明仁愛,不害眾生,萬邦歸仰,國富如海。國中眾生,奉順正法,大王仁聖,化之以道,慈施群生,無所遺惜。帝修凈戒,軌道不及,無上法船,濟諸沈溺,群僚百官,受樂無怨,諸天擁護,萬神侍衛,天魔降伏,莫不歸化。王身莊嚴,如日初出,仁澤普潤,猶如大雲,聖賢承業,如日月天,於彼真丹,最為殊勝。



 臣之所住,名迦毗河,東際于海,其城四邊,悉紫紺石,首羅天護,令國安隱。



 國王相
 承,未嘗斷絕,國中人民,率皆修善,諸國來集,共遵道法,諸寺舍子,皆七寶形像,眾妙供具,如先王法。臣自修檢,不犯道禁,臣名月愛,棄世王種。



 惟願大王聖體和善,群臣百官,悉自安隱。今以此國群臣吏民,山川珍寶,一切歸屬,五體歸誠大王足下。山海遐隔,無由朝覲,宗仰之至,遣使下承。使主父名天魔悉達,使主名尼駝達,此人由來良善忠信,是故今遣奉使表誠。大王若有所須,珍奇異物,悉當奉送,此之境土,便是王國,王之法令,治國
 善道,悉當承用。



 願二國信使往來不絕,此反使還,願賜一使,具宣聖命,備敕所宜。款至之誠,望不空反,所白如是,願加哀愍。



 奉獻金剛指環、摩勒金環諸寶物、赤白鸚鵡各一頭。太宗泰始二年,又遣使貢獻,以其使主竺扶大、竺阿彌並為建威將軍。



 元嘉十八年,蘇摩黎國王那鄰那羅跋摩遣使獻方物。世祖孝建二年,斤駝利國王釋婆羅那鄰駝遣長史竺留駝及多獻金銀寶器。後廢帝元徽元年,婆黎國遣使貢獻。



 凡此諸國,皆事佛道。



 佛
 道自後漢明帝,法始東流,自此以來,其教稍廣,自帝王至於民庶,莫不歸心。經誥充積,訓義深遠,別為一家之學焉。元嘉十二年,丹陽尹蕭摩之奏曰:「佛化被于中國,已歷四代,形像塔寺,所在千數,進可以擊心,退足以招勸。而自頃以來,情敬浮末,不以精誠為至,更以奢競為重。舊宇頹弛,曾莫之修,而各務造新,以相姱尚。甲第顯宅,於茲殆盡,材竹銅彩,糜損無極,無關神祗,有累人事。建中越制,宜加裁檢,不為之防,流道未息。請自今以後,
 有欲鑄銅像者,悉詣臺自聞;興造塔寺精舍,皆先詣在所二千石通辭,郡依事列言本州;須許報,然後就功。其有輒造寺舍者,皆依不承用詔書律,銅宅林苑,悉沒入官。」詔可。



 又沙汰沙門,罷道者數百人。



 世祖大明二年,有曇標道人與羌人高闍謀反,上因是下詔曰:「佛法訛替,沙門混雜,未足扶濟鴻教,而專成逋藪。加奸心頻發,凶狀屢聞,敗亂風俗,人神交怨。可付所在,精加沙汰,後有違犯,嚴加誅坐。」於是設諸條禁,自非戒行精苦,並使還
 俗。而諸寺尼出入宮掖,交關妃后,此制竟不能行。



 先是,晉世庾冰始創議,欲使沙門敬王者,後桓玄復述其義,並不果行。大明六年,世祖使有司奏曰:「臣聞邃宇崇居,非期宏峻,拳跪盤伏,非止敬恭,將以施張四維,締制八宇。故雖儒法枝派,名墨條分,至於崇親嚴上,厥由靡爽。唯浮圖為教,逖自龍堆,反經提傳,訓遐事遠,練生瑩識,恆俗稱難,宗旨緬謝,微言淪隔,拘文蔽道,在末彌扇。遂乃陵越典度,偃倨尊戚,失隨方之眇跡,迷製化之淵義。
 夫佛法以謙儉自牧,忠虔為道,不輕比丘,遭道人斯拜,目連桑門,過長則禮,寧有屈膝四輩,而簡禮二親,稽顙耆臘,而直體萬乘者哉。故咸康創議,元興載述,而事屈偏黨,道挫餘分。今鴻源遙洗,群流仰鏡,九仙盡寶,百神聳職,而畿輦之內,舍弗臣之氓,陛席之間,延抗體之客,懼非所以澄一風範,詳示景則者也。臣等參議,以為沙門接見,比當盡虔禮敬之容,依其本俗,則朝徽有序,乘方兼遂矣。」詔可。前廢帝初,復舊。



 世祖寵姬殷貴妃薨,為
 之立寺,貴妃子子鸞封新安王,故以新安為寺號。前廢帝殺子鸞,乃毀廢新安寺,驅斥僧徒,尋又毀中興、天寶諸寺。太宗定亂,下令曰:「先帝建中興及新安諸寺,所以長世垂範,弘宣盛化。頃遇昏虐,法像殘毀,師徒奔迸,甚以矜懷。妙訓淵謨,有扶名教。可招集舊僧,普各還本,並使材官,隨宜修復。」



 宋世名僧有道生。道生,彭城人也。父為廣戚令。生出家為沙門法大弟子。幼而聰悟,年十五,便能講經。及長,有異解,立頓悟義,時人推服之。元嘉十
 一年,卒於廬山。沙門慧琳為之誄。



 慧琳者,秦郡秦縣人,姓劉氏。少出家,住冶城寺,有才章,兼外內之學,為廬陵王義真所知。嘗著《均善論》,其詞曰:有白學先生,以為中國聖人,經綸百世,其德弘矣,智周萬變,天人之理盡矣;道無隱旨,教罔遺筌,聰睿迪哲,何負於殊論哉。有黑學道士陋之,謂不照幽冥之途,弗及來生之化,雖尚虛心,未能虛事,不逮西域之深也。於是白學訪其所以不逮云爾。



 白曰:「釋氏所論之空,與老氏所言之空,無同異乎?」
 黑曰:「異。釋氏即物為空,空物為一。老氏有無兩行,空有為異,安得同乎!」白曰:「釋氏空物,物信空邪?」黑曰:「然。空又空,不翅於空矣。」白曰:「三儀靈長於宇宙,萬品盈生於天地,孰是空哉?」黑曰:「空其自性之有,不害因假之體也。今構群材以成大廈,罔專寢之實,積一毫以致合抱,無檀木之體,有生莫俄頃之留,泰山蔑累息之固,興滅無常,因緣無主,所空在於性理,所難據於事用,吾以為誤矣。」



 白曰:「所言實相,空者其如是乎?」黑曰:「然。」白曰:「浮變之理,
 交於目前,視聽者之所同了邪?解之以登道場,重之以輕異學,誠未見其淵深。」黑白:「斯理若近,求之實遠。夫情之所重者虛,事之可重者實。今虛其真實,離其浮偽,愛欲之惑,不得不去。愛去而道場不登者,吾不知所以相曉也。」白曰:「今析豪空樹,無囗乘蔭之茂,離材虛室,不損輪奐之美,明無常增其心妻蔭之情,陳若偏篤其競辰之慮。貝錦以繁采發輝,和羹以鹽梅致旨,齊侯追爽鳩之樂,燕王無延年之術,恐和合之辯,危脆之教,正足戀其
 嗜好之欲,無以傾其愛競之惑也。」黑曰:「斯固理絕於諸華,墳素莫之及也。」白曰:「山高累卑之辭,川樹積小之詠,舟壑火傳之談,堅白唐肆之論,蓋盈於中國矣,非理之奧,故不舉以為教本耳。子固以遺情遺累,虛心為道,而據事剖析者,更由指掌之間乎!」黑曰:「周、孔為教,正及一世,不見來生無窮之緣,積善不過子孫之慶,累惡不過餘殃之罰,報效止於榮祿,誅責極於窮賤,視聽之外,冥然不知,良可悲矣。釋迦關無窮之業,拔重關之險,陶方
 寸之慮,宇宙不足盈其明,設一慈之救,群生不足勝其化,敘地獄則民懼其罪,敷天堂則物歡其福,指泥洹以長歸,乘法身以遐覽,神變無不周,靈澤靡不覃,先覺翻翔於上世,後悟騰翥而不紹,坎井之局,何以識大方之家乎!」白曰:「固能大其言矣,今效神光無徑寸之明,驗靈變罔纖介之異,勤誠者不睹善救之貌,篤學者弗剋陵虛之實,徒稱無量之壽,孰見期頤之叟,咨嗟金剛之固,安覿不朽之質。茍於事不符,宜尋立言之指,遺其所寄
 之說也。且要天堂以就善,曷若服義而蹈道,懼地獄以敕身,孰與從理以端心。禮拜以求免罪,不由祗肅之意,施一以徼百倍,弗乘無吝之情。美泥洹之樂,生耽逸之慮,贊法身之妙,肇好奇之心,近欲未弭,遠利又興,雖言菩薩無欲,群生固以有欲矣。甫救交敝之氓,永開利競之俗,澄神反道,其可得乎?」黑曰:「不然。若不示以來生之欲,何以權其當生之滯。



 物情不能頓至,故積漸以誘之。奪此俄頃,要彼無窮,若弗勤春稼,秋墻何期。端坐井底,
 而息意庶慮者,長淪於九泉之下矣。」白曰:「異哉!何所務之乖也。道在無欲,而以有欲要之,北行求郢,西征索越,方長迷於幽都,永謬滯於昧谷。遼遼閩、楚,其可見乎!所謂積漸者,日損之謂也。當先遺其所輕,然後忘其所重,使利欲日去,淳白自生耳。豈得以少要多,以粗易妙,俯仰之間,非利不動,利之所蕩,其有極哉!乃丹青眩媚彩之目,土木夸好壯之心,興糜費之道,單九服之財,樹無用之事,割群生之急,致營造之計,成私樹之權,務權化
 之業,結師黨之勢,苦節以要厲精之譽,護法以展陵競之情,悲矣!夫道其安寄乎?是以周、孔敦俗,弗關視聽之外;老、莊陶風,謹守性分而已。」黑曰:「三游本於仁義,盜跖資於五善,聖跡之敝,豈有內外。且黃、老之家,符章之偽,水祝之誣,不可勝論。子安於彼,駭於此,玩於濁水,違於清淵耳。」白曰:「有跡不能不敝,有術不能無偽,此乃聖人所以桎梏也。今所惜在作法於貪,遂以成俗,不正其敝,反以為高耳。



 至若淫妄之徒,世自近鄙,源流蔑然,因不
 足論。」黑曰:「釋氏之教,專救夷俗,便無取於諸華邪?」白曰:「曷為其然。為則開端,宜懷屬緒,愛物去殺,尚施周人,息心遺榮華之願,大士布兼濟之念,仁義玄一者,何以尚之。惜乎幽旨不亮,末流為累耳。」黑曰:「子之論善殆同矣,便事盡於生乎?」白曰:「幽冥之理,固不極於人事矣。周、孔疑而不辨,釋迦辨而不實,將宜廢其顯晦之跡,存其所要之旨。請嘗言之。夫道之以仁義者,服理以從化;帥之以勸戒者,循利而遷善。故甘辭興於有欲,而滅於悟理,
 淡說行於天解,而息於貪偽。是以示來生者,蔽虧於道、釋不得已,杜幽暗者,冥符於姬、孔閉其兌。由斯論之,言之者未必遠,知之者未必得,不知者未必失,但知六度與五教並行,信順與慈悲齊立耳。殊塗而同歸者,不得守其發輪之轍也。」



 論行於世。舊僧謂其貶黜釋氏,欲加擯斥。太祖見論賞之,元嘉中,遂參權要,朝廷大事,皆與議焉。賓客輻湊,門車常有數十兩,四方贈賂相係,勢傾一時。注《孝經》及《莊子逍遙篇》、文論,傳於世。



 又有慧嚴、慧
 議道人,並住東安寺,學行精整,為道俗所推。時鬥場寺多禪僧,京師為之語曰:「鬥場禪師窟,東安談義林。」世祖大明四年,於中興寺設齋。有一異僧,眾莫之識,問其名,答言名明慧,從天安寺來,忽然不見。天下無此寺名,乃改中興曰天安寺。大明中,外國沙門摩訶衍苦節有精理,於京都多出新經,《勝鬘經》尤見重內學。



 東夷高句驪國,今治漢之遼東郡。高句驪王高璉,晉軍帝義熙九年,遣長史高翼奉表獻赭白馬。以璉為使持
 節、都督營州諸軍事、征東將軍、高句驪王、樂浪公。



 高祖踐阼,詔曰:「使持節、都督營州諸軍事、征東將軍、高句驪王、樂浪公璉,使持節、督百濟諸軍事、鎮東將軍、百濟王映,並執義海外,遠修貢職。惟新告始,宜荷國休,璉可征東大將軍,映可鎮東大將軍。持節、都督、王、公如故。」三年,加璉散騎常侍,增督平州諸軍事。



 少帝景平二年,璉遣長史馬婁等詣闕獻方物,遣使慰勞之。曰:「皇帝問使持節、散騎常侍、都督營平二州諸軍事、征東大將軍、高句
 驪王、樂浪公,纂戎東服,庸績繼軌,厥惠既彰,款誠亦著,踰遼越海,納貢本朝。朕以不德,忝承鴻緒,永懷先蹤,思覃遺澤。今遣謁者朱邵伯、副謁者王邵子等,宣旨慰勞。其茂康惠政,永隆厥功,式昭往命,稱朕意焉。」



 先是,鮮卑慕容寶治中山,為索虜所破,東走黃龍。義熙初,寶弟熙為其下馮跋所殺,跋自立為主,自號燕王,以其治黃龍城,故謂之黃龍國。跋死,子弘立,屢為索虜所攻,不能下。太祖世,每歲遣使獻方物。元嘉十二年,賜加除授。十五
 年,復為索虜所攻,弘敗走,奔高驪北豐城,表求迎接。太祖遣使王白駒、趙次興迎之,并令高驪料理資遣;璉不欲使弘南,乃遣將孫漱、高仇等襲殺之。白駒等率所領七千餘人掩討漱等,生禽漱,殺高仇等二人。璉以白駒等專殺,遣使執送之,上以遠國,不欲違其意,白駒等下獄,見原。



 璉每歲遣使。十六年,太祖欲北討,詔璉送馬,璉獻馬八百匹。世祖孝建二年,璉遣長史董騰奉表慰國哀再周,并獻方物。大明三年,又獻肅慎氏楛矢石砮。七
 年,詔曰:「使持節、散騎常侍、督平營二州諸軍事、征東大將軍、高句驪王、樂浪公璉,世事忠義,作籓海外,誠係本朝,志剪殘險,通譯沙表,克宜王猷。宜加褒進,以旌純節。可車騎大將軍、開府儀同三司,持節、常侍、都督、王、公如故。」太宗泰始、後廢帝元徽中,貢獻不絕。



 百濟國,本與高驪俱在遼東之東千餘里,其後高驪略有遼東,百濟略有遼西。



 百濟所治,謂之晉平郡晉平縣。義熙十二年,以百濟王餘映為使持節、都督百濟諸軍
 事、鎮東將軍、百濟王。高祖踐阼,進號鎮東大將軍。少帝景平二年,映遣長史張威詣闕貢獻。元嘉二年,太祖詔之曰:「皇帝問使持節、都督百濟諸軍事、鎮東大將軍、百濟王。累葉忠順,越海效誠,遠王纂戎,聿修先業,慕義既彰,厥懷赤款,浮桴驪水,獻騕執贄,故嗣位方任,以籓東服,勉勖所蒞,無墜前蹤。今遣兼謁者閭丘恩子、兼副謁者丁敬子等宣旨慰勞稱朕意。」其後,每歲遣使奉表,獻方物。七年,百濟王餘毗復修貢職,以映爵號授之。二十
 七年,毗上書獻方物,私假臺使馮野夫西河太守,表求《易林》、《式占》、腰弩,太祖並與之。



 毗死,子慶代立。世祖大明元年,遣使求除授,詔許。二年,慶遣使上表曰:「臣國累葉,偏受殊恩,文武良輔,世蒙朝爵。行冠軍將軍右賢王餘紀等十一人,忠勤宜在顯進,伏願垂愍,並聽賜除。」仍以行冠軍將軍右賢王餘紀為冠軍將軍。



 以行征虜將軍左賢王餘昆、行征虜將軍餘暈並為征虜將軍。以行輔國將軍餘都、餘乂並為輔國將軍。以行龍驤將軍沐衿、
 餘爵並為龍驤將軍。以行寧朔將軍餘流、麋貴並為寧朔將軍。以行建武將軍于西、餘婁並為建武將軍。太宗泰始七年,又遣使貢獻。



 倭國,在高驪東南大海中,世修貢職。高祖永初二年,詔曰:「倭贊萬里修貢,遠誠宜甄,可賜除授。」太祖元嘉二年,贊又遣司馬曹達奉表獻方物。贊死,弟珍立,遣使貢獻。自稱使持節、都督倭百濟新羅任那秦韓慕韓六國諸軍事、安東大將軍、倭國王。表求除正,詔除安東將軍、倭
 國王。珍又求除正倭隋等十三人平西、征虜、冠軍、輔國將軍號,詔並聽。二十年,倭國王濟遣使奉獻,復以為安東將軍、倭國王。二十八年,加使持節、都督倭新羅任那加羅秦韓慕韓六國諸軍事,安東將軍如故。并除所上二十三人軍、郡。濟死,世子興遣使貢獻。世祖大明六年,詔曰:「倭王世子興,奕世載忠,作籓外海,稟化寧境,恭修貢職。新嗣邊業,宜授爵號,可安東將軍、倭國王。」興死,弟武立,自稱使持節、都督倭百濟新羅任那加羅秦韓慕
 韓七國諸軍事、安東大將軍、倭國王。



 順帝升明二年,遣使上表曰:「封國偏遠,作籓於外,自昔祖禰,躬擐甲胄,跋涉山川,不遑寧處。東征毛人五十五國,西服眾夷六十六國,渡平海北九十五國,王道融泰,廓土遐畿,累葉朝宗,不愆于歲。臣雖下愚,忝胤先緒,驅率所統,歸崇天極,道遙百濟,裝治船舫,而句驪無道,圖欲見吞,掠抄邊隸,虔劉不已,每致稽滯,以失良風。雖曰進路,或通或不。臣亡考濟實忿寇仇,壅塞天路,控弦百萬,義聲感激,方欲
 大舉,奄喪父兄,使垂成之功,不獲一簣。居在諒暗,不動兵甲,是以偃息未捷。至今欲練甲治兵,申父兄之志,義士虎賁,文武效功,白刃交前,亦所不顧。若以帝德覆載,摧此彊敵,克靖方難,無替前功。竊自假開府儀同三司,其餘咸各假授,以勸忠節。」詔除武使持節、都督倭新羅任那加羅秦韓慕韓六國諸軍事、安東大將軍、倭王。



 荊、雍州蠻,盤瓠之後也。分建種落,布在諸郡縣。荊州置南蠻,雍州置寧蠻校尉以領之。世祖初,罷南蠻併大府,
 而寧蠻如故。蠻民順附者,一戶輸穀數斛,其餘無雜調,而宋民賦役嚴苦,貧者不復堪命,多逃亡入蠻。蠻無徭役,彊者又不供官稅,結黨連群,動有數百千人,州郡力弱,則起為盜賊,種類稍多,戶口不可知也。所在多深險,居武陵者有雄溪、褭溪、辰谿、酉谿、舞谿,謂之五谿蠻。而宜都、天門、巴東、建平、江北諸郡蠻,所居皆深山重阻,人跡罕至焉。前世以來,屢為民患。



 少帝景平二年,宜都蠻帥石寧等一百一十三人詣闕上獻。太祖元嘉六年,建
 平蠻張雍之等五十人,七年,宜都蠻田生等一百一十三人,並詣闕獻見。其後沔中蠻大動,行旅殆絕。天門漊中令宗僑之徭賦過重,蠻不堪命。十八年,蠻田向求等為寇,破漊中,虜略百姓。荊州刺史衡陽王義季遣行參軍曹孫念討破之,獲生口五百餘人,免僑之官。二十四年,南郡臨沮當陽蠻反,縛臨沮令傅僧驥。荊州刺史南譙王義宣遣中兵參軍王諶討破之。



 先是,雍州刺史劉道產善撫諸蠻,前後不附官者,莫不順服,皆引出平土,
 多緣沔為居。及道產亡,蠻又反叛。及世祖出為雍州,群蠻斷道,擊大破之。臺遣軍主沈慶之連年討蠻,所向皆平殄,事在《慶之傳》。二十八年正月,龍山雉水蠻寇抄涅陽縣,南陽太守朱曇韶遣軍討之,失利,殺傷三百餘人;曇韶又遣二千人係之,蠻乃散走。是歲,滍水諸蠻因險為寇,雍州刺史隨王誕遣使說之曰:「頃威懷所被,覃自遐遠,順化者寵祿,逆命者無遺,此亦爾所知也。聖朝今普天肆眚,許以自新,便宜各還舊居,安堵復業,改過革
 心,於是乎始。」



 先是,蠻帥魯奴子擄龍山,屢為邊患。魯軌在長社,奴子歸之,軌言於虜主,以為四山王。軌子爽歸國,奴子亦求內附,隨王誕又遣軍討沔北諸蠻,襲濁山、如口、蜀松三柴,剋之,又圍升錢、柏義諸柴,蠻悉力距戰。軍以具裝馬夾射,大破之,斬首二百級,獲生蠻千口,牛馬八十頭。



 世祖大明中,建平蠻向光侯寇暴峽川,巴東太守王濟、荊州刺史朱修之遣軍討之,光侯走清江。清江去巴東千餘里。時巴東、建平、宜都、天門四郡蠻為寇,
 諸郡民戶流散,百不存一。太宗、順帝世尤甚,雖遣攻伐,終不能禁,荊州為之虛敝。



 大明中,桂陽蠻反,殺荔令晏珍之,臨賀蠻反,殺關建令邢伯兒,振武將軍蕭沖之討之,獲少費多,抵罪。



 豫州蠻,廩君後也。盤瓠及廩君事,並具前史。西陽有巴水、蘄水、希水、赤亭水、西歸水,謂之五水蠻,所在並深岨,種落熾盛,歷世為盜賊。北接淮、汝,南極江、漢,地方數千里。元嘉二十八年,西陽蠻殺南川令劉臺,並其家口。二
 十九年,新蔡蠻二千餘人破大雷戍,略公私船舫,悉引入湖。有亡命司馬黑石在蠻中,共為寇盜。太祖遣太子步兵校尉沈慶之率江、荊、雍、豫諸州軍討之。世祖大明四年,又遣慶之討西陽蠻,大剋獲而反。司馬黑石徒黨三人,其一人名智,黑石號曰「太公」,以為謀主;一人名安陽,號譙王;一人名續之,號梁王。蠻文小羅等討禽續之,為蠻世財所篡,小羅等相率斬世財父子六人。豫州刺史王玄謨遣殿中將軍郭元封慰勞諸蠻,使縛送亡命,
 蠻乃執智黑石、安陽二人送詣玄謨。世祖使於壽陽斬之。



 太宗初即位,四方反叛,及南賊敗於鵲尾,西陽蠻田益之、田義之、成邪財、田光興等起義攻郢州,剋之。以益之為輔國將軍,都統四山軍事,又以蠻戶立宋安、光城二郡,以義之為宋安太守,光興為龍驤將軍、光城太守。封益之邊城縣王,食邑四百一十一戶,成邪財陽城縣王,食邑三千戶。益之徵為虎賁中郎將,將軍如故。



 順帝昇明初,又轉射聲校尉、冠軍將軍。成邪財死,子婆思襲
 爵,為輔國將軍、武騎常侍。晉熙蠻梅式生亦起義,斬晉熙太守閻湛之、晉安王子勛典簽沈光祖,封高山侯,食所統牛崗、下柴二村三十戶。



 史臣曰:漢世西譯遐通,兼途累萬,跨頭痛之山,越繩度之險,生行死徑,身往魂歸。晉氏南移,河、隴夐隔,戎夷梗路,外域天斷。若夫大秦、天竺,迥出西溟,二漢銜役,特艱斯路,而商貨所資,或出交部,汎海陵波,因風遠至。又重峻參差,氏眾非一,殊名詭號,種別類殊,山琛水寶,由茲
 自出,通犀翠羽之珍,蛇珠火布之異,千名萬品,並世主之所虛心,故舟舶繼路,商使交屬。太祖以南琛不至,遠命師旅,泉浦之捷,威震滄溟,未名之寶,入充府實。夫四夷孔熾,患深自古,蠻、僰殊雜,種眾特繁,依深傍岨,充積畿甸,咫尺華氓,易興狡毒,略財據土,歲月滋深。自元嘉將半,寇慝彌廣,遂盤結數州,搖亂邦邑。於是命將出師,恣行誅討,自江漢以北,廬江以南,搜山蕩谷,窮兵罄武,繫頸囚俘,蓋以數百萬計。至於孩年耋齒,執訊所遺,將
 卒申好殺之憤,干戈窮酸慘之用,雖雲積怨,為報亦甚。張奐所云:「流血於野,傷和致災。」斯固仁者之言矣。



\end{pinyinscope}