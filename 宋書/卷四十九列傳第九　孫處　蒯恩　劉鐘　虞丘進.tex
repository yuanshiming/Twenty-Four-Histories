\article{卷四十九列傳第九 孫處 蒯恩 劉鐘 虞丘進}

\begin{pinyinscope}

 孫處,字季高,會稽永興人也。籍注季高,故字行於世。少任氣。高祖東征孫恩,季高義樂隨。高祖平定京邑,以為振武將軍,封新夷縣五等侯。廣固之役,先登有功。



 盧循
 之難,於石頭扞柵,戍越城、查浦,破賊於新亭。高祖謂季高曰:「此賊行破,應先傾其巢窟,令奔走之日,無所歸投,非卿莫能濟事。」遣季高率眾三千,汎海襲番禺。初,賊不以海道為防,季高至東沖,去城十餘里,城內猶未知。循守戰士猶有數千人,城池甚固。季高先焚舟艦,悉力登岸,會天大霧,四面陵城,即日克拔。循父嘏、長史孫建之、司馬虞尪夫等,輕舟奔始興。即分遣振武將軍沈田子等討平始興、南康、臨賀、始安嶺表諸郡。循於左里奔走,
 而眾力猶盛,自嶺道還襲廣州。季高距戰二十餘日,循乃破走,所殺萬餘人。追奔至鬱林,會病,不得窮討,循遂得走向交州。



 義熙七年四月,季高卒於晉康,時年五十三。追贈龍驤將軍、南海太守,封侯官縣侯,食邑千戶。九年,高祖念季高之功,乃表曰:「孫季高嶺南之勳,已蒙褒贈。臣更思惟盧循稔惡一紀,據有全域。若令根本未拔,投奔有所,招合餘燼,猶能為虞;縣師遠討,方勤廟算。而季高汎海萬里,投命洪流,波激電邁,指日遄至,遂奄定
 南海,覆其巢窟,使循進退靡依,輕舟遠迸。曾不旬月,妖凶殲殄。蕩滌之功,實庸為大。往年所贈,猶為未優。愚謂宜更贈一州,即其本號,庶令忠勳不湮,勞臣增厲。」重贈交州刺史,將軍如故。子宗世卒,子欽公嗣。欽公卒,子彥祖嗣。



 齊受禪,國除。



 蒯恩,字道恩,蘭陵承人也。高祖征孫恩,縣差為徵民,充乙士,使伐馬芻。



 恩常負大束,兼倍餘人,每捨芻於地,歎曰:「大丈夫彎弓三石,柰何充馬士!」



 高祖聞之,即給器仗,
 恩大喜。自征妖賊,常為先登,多斬首級。既習戰陣,膽力過人,誠心忠謹,未嘗有過失,甚見愛信。於婁縣戰,箭中左目。



 從平京城,進定京邑,以寧遠將軍領幢。隨振武將軍道規西討,虜桓仙客,克偃月疊,遂平江陵。義熙二年,賊張堅據應城反,恩擊破之,封都鄉侯。從伐廣固,又有戰功。盧循逼京邑,恩戰於查浦,賊退走。與王仲德等追破循別將范崇民於南陵。循既走還廣州,恩又領千餘人隨劉籓追徐道覆於始興,斬之。遷龍驤將軍、蘭陵太
 守。



 高祖西征劉毅,恩與王鎮惡輕軍襲江陵,事在《鎮惡傳》。以本官為太尉長兼行參軍,領眾二千,隨益州刺史朱齡石伐蜀。至彭模,恩所領居前,大戰,自朝至日昃,勇氣益奮,賊破走。進平成都,擢為行參軍,改封北至縣五等男。高祖伐司馬休之及魯宗之,恩與建威將軍徐逵之前進。逵之敗沒,恩陳於隄下。宗之子軌乘勝擊恩,矢下如雨,呼聲震地,恩整厲將士,置陣堅嚴。軌屢衝之不動,知不可攻,乃退。高祖善其能將軍持重。江陵平定,復
 追魯軌於石城。軌棄城走,恩追至襄陽,宗之奔羌,恩與諸將追討至魯陽關乃還。恩自從征討,每有危急,輒率先諸將,常陷堅破陣,不避艱惸。凡百餘戰,身被重創。高祖錄其前後功勞,封新寧縣男,食邑五百戶。高祖世子為征虜將軍,恩以大府佐領中兵參軍,隨府轉中兵參軍。高祖北伐,留恩侍衛世子,命朝士與之交。恩益自謙損,與人語常呼官位,而自稱為鄙人。撫待士卒,甚有紀綱,眾咸親附之。遷咨議參軍,轉輔國將軍、淮陵太守。世
 子開府,又為從事中郎,轉司馬,將軍、太守如故。



 入關迎桂陽公義真。義真還至青泥,為佛佛虜所追,恩斷後,力戰連日。義真前軍奔散,恩軍人亦盡,為虜所執,死於虜中。子國才嗣。國才卒,子慧度嗣。慧度卒,無子,國除。



 劉鐘,字世之,彭城彭城人也。少孤,依鄉人中山太守劉固共居。幼有志力,常慷慨於貧賤。隆安四年,高祖伐孫恩,鐘願從餘姚、浹口攻句章、海鹽、婁縣,皆摧堅陷陣,每有戰功。為劉牢之鎮北參軍督護。高祖每有戎事,鐘不
 辭艱劇,專心盡力,甚見愛信。



 義旗將建,高祖版鐘為郡主簿。明日,從入京城。將向京邑,高祖命曰:「預是彭沛鄉人赴義者,並可依劉主簿。」於是立為義隊,恆在左右,連戰皆捷。明日,桓謙屯於東陵,卞範之屯覆舟山西,高祖疑賊有伏兵,顧視左右,正見鐘,謂之曰:「此山下當有伏兵,卿可率部下稍往撲之。」鐘應聲馳進,果有伏兵數百,一時奔走。桓玄西奔,其夕,高祖止桓謙故營,遣鐘宿據東府,轉鎮軍參軍督護。桓歆寇歷陽,遣鐘助豫州刺史
 魏詠之討之,歆即奔迸。除南齊國內史,封安丘縣五等侯。



 自陳情事,改葬父祖及親屬十喪,高祖厚加資給。轉車騎長史,兼行參軍。司馬叔璠與彭城劉謚、劉懷玉等自蕃城攻鄒山,魯郡太守徐邕失守,鐘率軍討平之。從征廣固。孟龍符陷沒,鐘率左右直入,取其尸而反。除振武將軍、中兵參軍,代龍符領廣川太守。



 盧循逼京邑,徐赤軍違處分,敗於南岸。鐘率麾下距柵,身被重創,賊不得入。



 循南走,鐘與輔國將軍王仲德追之。循先留別帥範
 崇民以精兵高艦據南陵,夾屯兩岸。鐘自行覘賊,天霧,賊鉤得其舸;鐘因率左右攻艦戶,賊遽閉戶距之,鐘乃徐還。與仲德攻崇民,崇民敗走。鐘追討百里,燒其船乘。又隨劉籓追徐道覆於始興,斬之。補太尉行參軍、寧朔將軍、下邳太守。代孟懷玉領石頭戍事。



 高祖討劉毅,鐘率軍繼王鎮惡。江陵平定,仍隨朱齡石伐蜀,為前鋒,由外水,至於彭城模,去成都二百里。偽冠軍征討督護譙亢等兩岸連營,層樓重柵,眾號三萬。鐘於時腳疾不能行,
 齡石乃詣鐘謀曰:「今天時盛熱,而賊嚴兵固險,攻之未必可拔,只增疲困。計其人情恇撓,必不久安,且欲養銳息兵,以伺其隙;隙而乘之,乃可捷事。然決機兩陳,公本有所委,卿意謂何?」鐘曰:「不然。前揚聲言大眾向內水,譙道福不敢舍涪城。今重軍卒至,出其不意,蜀人已破膽矣。賊今阻兵守險,是其懼不敢戰,非能持久堅守也。因其兇懼,盡銳攻之,其勢必克。鼓行而進,成都必不能守矣。今若緩兵相守,彼將知人虛實,涪軍忽并來力距我,
 人情既安,良將又集,此求戰不獲,軍食無資,當為蜀子虜耳。」齡石從之。明日進攻,陷其二城,斬其大將侯輝、譙詵,逕平成都。以廣固功,封永新縣男,食邑五百戶。



 遷給事中、太尉參軍事、龍驤將軍、高陽內史,領石頭戍事。



 高祖討司馬休之,前軍將軍道憐留鎮東府,領屯兵。冶亭群盜數百,夜襲鐘壘,距擊破之。時大軍外討,京邑擾懼,鐘以不能鎮遏,降號建威將軍。平蜀功,應封四百戶男,以先有封爵,減戶以賜次子敬順高昌縣男,食邑百戶。
 尋復本號龍驤將軍。十二年,高祖北伐,復留鎮居守,增其兵力,又命府置佐史。荊州刺史道憐獻名馬三匹,并精麗乘具,高祖悉以賜鐘三子。十四年,遷右衛將軍,龍驤將軍如故。



 元熙元年卒,時年四十三。



 子敬義嗣。敬義官至馬頭太守,卒。子國重嗣,齊受禪,國除。鐘次子高昌男敬順,卒,子國須嗣。須卒,無子,國除。



 虞丘進,字豫之,東海郯人也。少時隨謝玄討苻堅,有功,封關內侯。隆安中,從高祖征孫恩,戍句章城,被圍數十
 日,無日不戰,身被數創。至餘姚呵浦,破賊張驃,追至海鹽故治及婁縣。於蒲濤口與孫恩水戰,又被重創。追恩至鬱州,又至石鹿頭,還海鹽大柱,頻戰有功。元興元年,又從高祖東征臨海,於石步固與盧循相守二十餘日。二年,又從高祖至東陽,破徐道覆。其年,又至臨松穴破賊,追至永嘉千江,又至安固,累戰皆有功。三年,從平京城,定京邑,除燕國內史。



 義熙二年,除龍驤將軍,封龍川縣五等侯。從高祖伐廣固,於臨朐破賊。盧循逼京邑,孟
 昶、諸葛長民等建議奉天子過江,進廷議不可,面折昶等,高祖甚嘉之。



 獻計伐樹,樹柵石頭。除鄱陽太守,將軍如故。統馬步十八隊,於東道出鄱陽,至五畝嶠。循遣將英紏為上饒令。千餘人守故城,進攻破之。循又遣童敏之為鄱陽太守,據郡,進從餘干步道趣鄱陽,敏之退走,追破之,斬首數百。復隨劉籓至始興,討斬徐道覆。



 八年,除寧蠻護軍、尋陽太守,領文武二年從征劉毅。事平,補太尉行參軍,尋加振威將軍。九年,以前後功封望蔡縣
 男,食邑五百戶,加龍驤將軍。討司馬休之,又有戰功。軍還,除輔國將軍、山陽太守。宋臺令書除秦郡太守,督陳留郡事,將軍如故。元熙二年,宋王令書以為高祖第四子義康右將軍司馬。永初二年,遷太子右衛率。明年,卒官。時年六十。追論討司馬休之功,進爵為子,增邑三百戶。



 子耕嗣。耕卒,子襲祖嗣。襲祖卒,世寶嗣。齊受禪,國除。



 史臣曰:《詩》云:「無言不酬,無德不報。」此諸將並起自豎夫,出於皁隸芻牧之下,徒以心一乎主,故能奮其鱗翼。至
 於推鋒轉戰,百死而不顧一生,蓋由其心一也。遂饗封侯之報,詩人之言,信矣!



\end{pinyinscope}