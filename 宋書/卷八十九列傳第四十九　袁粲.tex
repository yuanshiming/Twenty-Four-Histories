\article{卷八十九列傳第四十九 袁粲}

\begin{pinyinscope}

 袁粲,字景倩,陳郡陽夏人,太尉淑兄子也。父濯,揚州秀才,蚤卒。祖母哀其幼孤,名之曰愍孫。伯叔並當世榮顯,而愍孫饑寒不足。母琅邪王氏,太尉長史誕之女也,躬
 事績紡,以供朝夕。愍孫少好學,有清才,有欲與從兄顗婚者,伯父洵即顗父,曰:「顗不堪,政可與愍孫婚耳。」時愍孫在坐,流涕起出。蚤以操立志行見知。初為揚州從事,世祖安北、鎮軍、北中郎行參軍,侍中郎主簿。世祖伐逆,轉記室參軍。及即位,除尚書吏部郎,太子右衛率,侍中。孝建元年,世祖率群臣並於中興寺八關齋,中食竟,愍孫別與黃門郎張淹更進魚肉食。尚書令何尚之奉法素謹,密以白世祖,世祖使御史中丞王謙之糾奏,並免
 官。二年,起為廷尉,太子中庶子,領右軍將軍。出為輔國將軍、西陽王子尚北中郎長史、廣陵太守,行兗州事。仍為永嘉王子仁冠軍長史,將軍、太守如故。



 大明元年,復為侍中,領射聲校尉,封興平縣子,食邑五百戶,事在《顏師伯傳》。三年,坐納山陰民丁彖文貨,舉為會稽郡孝廉,免官。尋為西陽王子尚撫軍長史,又為中庶子,領左軍將軍。四年,出補豫章太守,加秩中二千石。五年,復還為侍中,領長水校尉,遷左衛將軍,加給事中。七年,轉吏部
 尚書,左衛如故。



 其年,皇太子冠,上臨宴東宮,愍孫勸顏師伯酒;師伯不飲,愍孫因相裁辱。師伯見寵於上,上常嫌愍孫以寒素凌之,因此發怒,出為海陵太守。前廢帝即位,除御史中丞,不拜。復為吏部尚書。永光元年,徙右衛將軍,加給事中。景和元年,復入為侍中,領驍騎將軍。太宗泰始元年,轉司徒左長史,冠軍將軍,南東海太守。



 愍孫清整有風操,自遇甚厚,常著《妙德先生傳》以續嵇康《高士傳》以自況,曰:有妙德先生,陳國人也。氣志淵虛,
 姿神清映,性孝履順,棲沖業簡,有舜之遺風。先生幼夙多疾,性疏懶,無所營尚,然九流百氏之言,雕龍談天之藝,皆泛識其大歸,而不以成名。家貧嘗仕,非其好也。混其聲迹,晦其心用,故深交或迕,俗察罔識。所處席門常掩,三徑裁通,雖揚子寂漠,嚴叟沈冥,不是過也。脩道遂志,終無得而稱焉。



 又嘗謂周旋人曰:「昔有一國,國中一水,號曰狂泉。國人飲此水,無不狂,唯國君穿井而汲,獨得無恙。國人既並狂,反謂國主之不狂為狂。於是聚謀,
 共執國主,療其狂疾。火艾針藥,莫不畢具。國主不任其苦,於是到泉所酌水飲之,飲畢便狂。君臣大小,其狂若一,眾乃歡然。我既不狂,難以獨立,比亦欲試飲此水。」



 愍孫幼慕荀奉倩之為人,白世祖,求改名為粲,不許。至是言於太宗,乃改為粲,字景倩焉。二年,遷領軍將軍,仗士三十人入六門。其年,徙中書令,領太子詹事,增封三百戶,固辭不受。三年,轉尚書僕射,尋領吏部。五年,加中書令,又領丹陽尹。六年,上於華林園茅堂講《周易》,粲為執
 經。又知東宮事,徙為右僕射。七年,領太子詹事,僕射如故。未拜,遷尚書令,丹陽尹如故。坐前選武衛將軍江柳為江州刺史,柳有罪,降為守尚書令。



 太宗臨崩,粲與褚淵、劉勔並受顧命,加班劍二十人,給鼓吹一部。後廢帝即位,加兵五百人。帝未親朝政,下詔曰:「比元序愆度,留熏耀晷,有傷秋稼,方貽民瘼。朕以眇疚,未弘政道,囹圄尚繁,枉滯猶積,晨兢夕厲,每惻于懷。尚書令可與執法以下,就訊眾獄,使冤訟洗遂,困弊昭蘇。頒下州郡,咸令
 無壅。」元徽元年,丁母憂,葬竟,攝令親職,加衛將軍,不受。敦逼備至,中使相望,粲終不受。性至孝,居喪毀甚,祖日及祥變,常發詔衛軍斷客。



 二年,桂陽王休範為逆,粲扶曳入殿,詔加兵自隨,府置佐史。時兵難危急,賊已至南掖門,諸將意沮,咸莫能奮。粲慷慨謂諸將帥曰:「寇賊已逼,而眾情離沮。孤子受先帝顧託,本以死報,今日當與褚護軍同死社稷!」因命左右被馬,辭色哀壯。於是陳顯達等感激出戰,賊即平殄。事寧,授中書監,即本號開府
 儀同三司,領司徒,以揚州解為府,固不肯移。



 三年,徙尚書令,衛軍、開府如故,並固辭,服終乃受。加侍中,進爵為侯,又不受。時粲與齊王、褚淵、劉秉入直,平決萬機,時謂之「四貴」。粲閑默寡言,不肯當事,主書每往諮決,或高詠對之,時立一意,則眾莫能改。宅宇平素,器物取給。好飲酒,善吟諷,獨酌園庭,以此自適。居負南郭,時杖策獨遊,素寡往來,門無雜客。及受遺當權,四方輻湊,閑居高臥,一無所接,談客文士,所見不過一兩人。



 順帝即位,遷中
 書監,司徒、侍中如故。時齊王居東府,故使粲鎮石頭。粲素靜退,每有朝命,多不即從,逼切不得已,然後方就。及詔移石頭,即便順旨。有周旋人解望氣,謂粲曰:「石頭氣甚乖,往必有禍。」粲不答。又給油絡通憲車,仗士五十人入殿。時齊王功高德重,天命有歸,粲自以身受顧託,不欲事二姓,密有異圖。丹陽尹劉秉,宋代宗室;前湘州刺史王蘊,太后兄子,素好武事,並慮不見容於齊王,皆與粲相結。將帥黃回、任候伯、孫曇瓘、王宜興、彭文之、卜伯
 興等,並與粲合。



 昇明元年,荊州刺史沈攸之舉兵,齊王自詣粲,粲稱疾不見。粲宗人通直郎袁達以為不宜示異同,粲曰:「彼若以主幼時艱,與桂陽時不異,劫我入臺,便無辭以拒。一如此,不復得出矣。」時齊王入屯朝堂,秉從父弟領軍將軍韞入直門下省,伯興為直閣,黃回諸將皆率軍出新亭。粲謀克日矯太后令,使韞、伯興率宿衛兵攻齊王於朝堂,回率軍來應。秉、候伯等並赴石頭,本期夜發。其日秉恇擾不知所為,晡後便束裝,未暗,載
 婦女席卷就粲,由此事洩。先是,齊王遣將薛淵、蘇烈、王天生等領兵戍石頭,云以助粲,實禦之也。又令腹心王敬則為直閣,與伯興共總禁兵。王蘊聞秉已奔,歎曰:「今年事敗矣!」時齊王使蘊募人,已得數百,乃狼狽率部曲向石頭。本期開南門,時已暗夜,薛淵等據門射之,蘊謂粲已敗,即便散走。



 齊王以報敬則,率所領收蘊殺之,并誅伯興。又遣軍主戴僧靜向石頭助薛淵,自倉門得入。時粲與秉等列兵登東門,僧靜分兵攻府西門。粲與秉
 欲還赴府,既下城,列燭自照,僧靜挺身暗往,粲子最覺有異人,以身衛粲,僧靜直前斬之,父子俱殞,左右各分散。粲死時,年五十八。任候伯等其夜並乘輕舸,自新亭赴石頭,聞粲敗,乃馳還;其後並誅。秉事在《宗室傳》。



 齊永明元年,詔曰:「昔魏矜袁紹,恩給丘墳;晉亮兩王,榮覃餘裔。斯蓋懷舊流仁,原心興宥,二代弘義,前載美談。袁粲、劉秉,並與先朝同獎宋室;沈攸之於景和之世,特有乃心,雖末節不終,而始誠可錄。歲月彌往,宜沾優隆。粲、秉
 前年改葬,塋兆未脩,材官可為經略,粗合周禮。攸之及其諸子喪柩在西,可符荊州以時致送,還反舊墓,在所營葬事。」



 史臣曰:闢運創基,非機變無以通其務;世及繼體,非忠貞無以守其業。闢運之君,千載一有,世及之主,無乏於時,囗囗須機變之用短,資忠貞之路長也。故漢室囗囗,文舉不屈曹氏;魏鼎將移,夏侯義不北面。若悉以二子為心,則兩代宜不亡矣。袁粲清標簡貴,任屬負圖,朝野
 之望雖隆,然未以大節許也。及其赴危亡,審存滅,豈所謂義重於生乎!雖不達天命,而其道有足懷者。昔王經被旌於晉世,粲等亦改葬於聖朝,盛代同符,美矣!



\end{pinyinscope}