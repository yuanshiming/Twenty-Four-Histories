\article{卷五十五列傳第十五 臧燾 徐廣 傅隆}

\begin{pinyinscope}

 臧燾,字
 德仁,東莞莒人,武敬皇后兄也。少好學,善《三禮》,貧約自立,操行為鄉里所稱。晉孝武帝太元中,衛將軍謝安始立國學,徐、兗二州刺史謝玄舉燾為助教。



 孝武
 帝追崇庶祖母宣太后,議者或謂宜配食中宗。燾議曰:「《陽秋》,之義,母以子貴,故仲子、成風,咸稱夫人。《經》云『考仲子之宮』。若配食惠廟,則宮無緣別築。前漢孝文、孝昭太后,並繫子為號,祭於寢園,不配於高祖、孝武之廟。後漢和帝之母曰恭懷皇所,安帝祖母曰敬隱皇后,順帝之母曰恭愍皇后,雖不系子為號,亦祭於陵寢。不配章、安二帝。此則二漢雖有太后、皇后之異,至于並不配食,義同《陽秋》。唯光武追廢呂后,故以薄后配高祖廟。又衛后既廢,霍光追尊李夫人
 為皇后,配孝武廟,此非母以子貴之例,直以高、武二廟無配故耳。夫漢立寢於陵,自是晉制所異。謂宜遠準《陽秋》考宮之義,近摹二漢不配之典。尊號既正,則罔極之情申,別建寢廟,則嚴禰之義顯,系子為稱,兼明母貴之所由,一舉而允三義,固哲王之高致也。」議者從之。



 頃之,去官。以母老家貧,與弟熹俱棄人事,躬耕自業,約己養親者十餘載。



 父母喪亡,居喪六年,以毀瘠著稱。服闋,除臨沂令。義旗建,為太學博士,參右將軍何無忌軍事,隨
 府轉鎮南參軍。



 高祖鎮京口,與燾書曰:「頃學尚廢弛,後進頹業,衡門之內,清風輟響。良由戎車屢警,禮樂中息,浮夫恣志,情與事染,豈可不敷崇墳籍,敦厲風尚。此境人士,子侄如林,明發搜訪,想聞令軌。然荊玉含寶,要俟開瑩,幽蘭懷馨,事資扇發,獨習寡悟,義著周典。今經師不遠,而赴業無聞,非唯志學者鮮,或是勸誘未至邪。想復弘之。」參高祖中軍軍事,入補尚書度支郎,改掌祠部。襲封高陵亭侯。



 時太廟鴟尾災,燾謂著作郎徐廣曰:「昔孔
 子在齊,聞魯廟災,曰必桓、僖也。



 今征西、京兆四府君,宜在毀落,而猶列廟饗,此其徵乎?」乃上議曰:「臣聞國之大事,在祀與戎,將營宮室,宗廟為首。古先哲王,莫不致肅恭之誠心,盡崇嚴乎祖考,然後能流淳化於四海,通幽感於神明。固宜詳廢興於古典,循情禮以求中者也。禮,天子七廟,三昭三穆,與太祖而七。自考廟以至祖考五廟,皆月祭之,遠廟為祧,有二祧,享嘗乃止。去祧為壇,去壇為墠,有禱然後祭之。此宗廟之次,親疏之序也。鄭玄
 以為祧者文王、武王之廟,王肅以為五世六世之祖。尋去祧之言,則祧非文、武之廟矣。文、武周之祖宗,何云去祧為壇乎?明遠廟為祧者,無服之祖也。又遠廟則有享嘗之禮,去祧則有壇墠之殊,明世遠者,其義彌疏也。若祧是文、武之廟,宜同月祭於太祖,雖推后稷以配天,由功德之所始,非尊崇之義每有差降也。又禮有以多貴者,故傳稱德厚者流光,德薄者流卑。又云自上以下,降殺以兩,禮也。此則尊卑等級之典,上下殊異之文。而云天子
 諸侯俱祭五廟,何哉?



 又王祭嫡殤,下及來孫,而上祀之禮,不過高祖。推隆恩於下流,替誠敬於尊屬,亦非聖人制禮之意也。是以泰始建廟,從王氏議,以禮父為士,子為天子諸侯,祭以天子諸侯,其尸服以士服。故上及征西,以備六世之數,宣皇雖為太祖,尚在子孫之位,至於殷祭之日,未申東向之禮,所謂子雖齊聖,不先父食者矣。今京兆以上既遷,太祖始得居正,議者以昭穆未足,欲屈太祖於卑坐,臣以為非禮典之旨。



 所與太祖而七,
 自是昭穆既足,太祖在六世之外,非為須滿七廟,乃得居太祖也。



 議者又以四府君神主宜永同於殷祫,臣又以為不然。傳所謂毀廟之主,陳乎太祖,謂太祖以下先君之主也。故《白虎通》云『禘祫祭遷廟者,以其繼君之體,持其統而不絕也。」豈如四府君在太祖之前乎。非繼統之主,無靈命之瑞,非王業之基,昔以世近而及,今則情禮已遠,而當長饗殷祫,永虛太祖之位,求之禮籍,未見其可。昔永和之初,大議斯禮,于時虞喜、范宣並以淵儒碩
 學,咸謂四府君神主,無緣永存於百世。或欲瘞之兩階,或欲藏之石室,或欲為之改築,雖所秉小異,而大歸是同。若宣皇既居群廟之上,而四主禘祫不已,則大晉殷祭,長無太祖之位矣。



 夫理貴有中,不必過厚;禮與世遷,豈可順而不斷!故臣子之情雖篤,而靈厲之謚彌彰;追遠之懷雖切,而遷毀之禮為用。豈不有心於加厚,顧禮制不可踰爾。石室則藏於廟北,改築則未知所處,虞主所以依神,神移則有瘞埋之禮。四主若饗祀宜廢,亦神
 之所不依也,準傍事例,宜同虞主之瘞埋。然經典難詳,群言紛錯,非臣卑淺所能折中。」時學者多從燾議,竟未施行。



 遷通直郎,高祖鎮軍、車騎、中軍、太尉咨議參軍。高祖北伐關、洛,大司馬琅邪王同行,除大司馬從事中郎,總留府事。義熙十四年,除侍中。元熙元年,以腳疾去職。高祖受命,徵拜太常,雖外戚貴顯,而彌自沖約,茅屋蔬餐,不改其舊。



 所得奉祿,與親戚共之。永初三年,致仕,拜光祿大夫,加金章紫綬。其年卒,時年七十。少帝追贈左
 光祿大夫,加散騎常侍。



 長子邃,護軍司馬,宜都太守。少子綽,太子中舍人,新安太守。邃長子諶之,尚書都官郎,烏程令。諶之弟凝之,學涉有當世才具,與司空徐湛之為異常之交。



 年少時與北地傅僧祐俱以通家子始為太祖所引見,時上與何尚之論鑄錢事,凝之便干其語,上因回與論之。僧祐引凝之衣令止,凝之大言謂僧祐曰:「明主難再遇,便應正盡所懷。」上與往復十餘反,凝之詞韻銓序,兼有理證,上甚賞焉。歷隨王誕後軍記室錄
 事,欲以為青州,其事不果。遷尚書右丞,以徐湛之黨,為元兇所殺。



 子夤,尚書主客郎,沈攸之征西功曹,為攸之盡節,事在《攸之傳》。凝之弟潭之,亦有美譽。太宗世,歷尚書吏部郎,御史中丞。後廢帝元徽中,為左民尚書,卒官。



 潭之弟澄之,太子左積弩將軍。元嘉二十七年,領軍於盱眙,為索虜所破,見殺,追贈通直郎。綽子煥,順帝升明中,為武昌太守。沈攸之攻郢城,煥棄郡赴之;攸之敗,伏誅。



 傅僧祐,祖父弘仁,高祖外弟也。以中表歷顯官,征虜
 將軍、南譙太守,太常卿。子邵,員外散騎侍郎,妻燾女也,生僧祐,有吏才,再為山陰令,甚有能名,末世令長莫及。亦以徐湛之黨,為元凶所殺。



 徐廣,字野民,東莞姑幕人也。父藻,都水使者。兄邈,太子前衛率。家世好學,至廣尤精,百家數術,無不研覽。謝玄為州,辟廣從事西曹。又譙王司馬恬鎮北參軍。晉孝武帝以廣博學,除為祕書郎,校書秘閣,增置職僚。轉員外散騎侍郎,領校書如故。隆安中,尚書令王珣舉為祠部
 郎。



 李太后薨,廣議服曰:「太皇太后名位允正,體同皇極,理制備盡,情禮彌申。



 《陽秋》之義,母以子貴。既稱夫人,禮服從正,故成風顯夫人之號,文公服三年之喪。子於父之所生,體尊義重。且禮祖不厭孫,固宜遂服無屈。而緣情立制,若嫌明文不存,則疑斯從重。謂應同於為祖母後,齊衰三年。」服從其議。



 時會稽王世子元顯錄尚書,欲使百僚致敬,臺內使廣立議,由是內外並執下官禮,廣常為愧恨焉。元顯引為中軍參軍,遷領軍長史。桓玄輔政,
 以為大將軍文學祭酒。



 義熙初,高祖使撰車服儀注,乃除鎮軍咨議參軍,領記室。封樂成縣五等侯。



 轉員外散騎常侍,領著作郎。二年,尚書奏曰:「臣聞左史述言,右官書事,《乘》、《志》顯於晉、鄭,《陽秋》著乎魯史。自皇代有造,中興晉祀,道風帝典,煥乎史策。而太和以降,世歷三朝,玄風聖迹,倏為疇古。臣等參詳,宜敕著作郎徐廣撰成國史。」詔曰:「先朝至德光被,未著方策,宜流風緬代,永貽將來者也。便敕撰集。」



 六年,遷散騎常侍,又領徐州大中正,轉
 正員常侍。時有風雹為災,廣獻書高祖曰:「風雹變未必為災,古之聖賢輒懼而修己,所以興政化而隆德教也。嘗忝服事,宿眷未忘,思竭塵露,率誠於習。明公初建義旗,匡復宗社,神武應運,信宿平夷。且恭儉謙約,虛心匪懈,來蘇之化,功用若神。頃事故既多,刑德並用,戰功殷積,報敘難盡,萬機繁湊,固應難速,且小細煩密,群下多懼。又穀帛豐賤,而民情不勸;禁司互設,而劫盜多有,誠由俗弊未易整,而望深未易炳。追思義熙之始,如有不
 同,何者?好安願逸,萬物之大趣,習舊駭新,凡識所不免。要當俯順群情,抑揚隨俗,則朝野歡泰,具瞻允康矣。言無可採,願矜其愚款之志。」又轉大司農,領著作郎皆如故。十二年,《晉紀》成,凡四十六卷,表上之。遷秘書監。



 初,桓玄篡位,安帝出宮,廣陪列悲慟,哀動左右。及高祖受禪,恭帝遜位,廣又哀感,涕泗交流。謝晦見之,謂之曰:「徐公將無小過?」廣收淚答曰:「身與君不同。君佐命興王,逢千載嘉運;身世荷晉德,實眷戀故主。」因更歔欷。



 永初元年,
 詔曰:「秘書監徐廣,學優行謹,歷位恭肅,可中散大夫。」廣上表曰:「臣年時衰耄,朝敬永闕,端居都邑,徒增替怠。臣墳墓在晉陵,臣又生長京口,戀舊懷遠,每感暮心。息道玄謬荷朝恩,忝宰此邑,乞相隨之官,歸終桑梓。



 微志獲申,殞沒無恨。」許之,贈賜甚厚。性好讀書,老猶不倦。元嘉二年,卒,時年七十四。《答禮問》百餘條,用於今世。廣兄子豁,在《良吏傳》。



 傅隆,字伯祚,北地靈州人也。高祖咸,晉司隸校尉。曾祖
 晞,司徒屬。父祖早亡。隆少孤,又無近屬,單貧有學行,不好交游。義熙初,年四十,始為孟昶建威將軍,員外散騎侍郎。坐辭兼,免。復為會稽征虜參軍。家在上虞,及東歸,便有終焉之志。歷佐三軍,首尾八年。除給事中。尚書僕射、丹陽尹徐羨之置建威府,以為錄事參軍,尋轉尚書祠部郎、丹陽丞,入為尚書左丞。以族弟亮為僕射,緦服不得相臨,徙太子率更令,廬陵王義真車騎咨議參軍,出補山陰令。太祖元嘉初,除司徒右長史,遷御史中
 丞。當官而行,甚得司直之體。轉司徒左長史。



 時會稽剡縣民黃初妻趙打息載妻王死亡,遇赦,王有父母及息男稱、息女葉,依法徙趙二千里外。隆議之曰:「原夫禮律之興,蓋本之自然,求之情理,非從天墮,非從地出也。父子至親,分形同氣,稱之於載,即載之於趙,雖云三世,為體猶一,未有能分之者也。稱雖創巨痛深,固無仇祖之義。若稱可以殺趙,趙當何以處載?將父子孫祖,互相殘戮,懼非先王明罰,咎繇立法之本旨也。向使石厚之子、日磾之孫,
 砥鋒挺鍔,不與二祖同戴天日,則石碏、秺侯何得流名百代,以為美談者哉!舊令云,『殺人父母,徙之二千里外』。不施父子孫祖明矣。趙當避王期功千里外耳。令亦云,『凡流徙者,同籍親近欲相隨者,聽之』。此又大通情體,因親以教愛者也。趙既流移,載為人子,何得不從;載從而稱不行,豈名教所許?如此,稱、趙竟不可分。趙雖內愧終身,稱當沈痛沒齒,孫祖之義,自不得永絕,事理固然也。」從之。



 又出為義興太守,在郡有能名。徵拜左民尚書,坐
 正直受節假,對人未至,委出,白衣領職。尋轉太常。十四年,太祖以新撰《禮論》付隆使下意,隆上表曰:「臣以下愚,不涉師訓,孤陋閭閻,面牆靡識,謬蒙詢逮,愧懼流汗。原夫禮者,三千之本,人倫之至道。故用之家國,君臣以之尊,父子以之親;用之婚冠,少長以之仁愛,夫妻以之義順;用之鄉人,友朋以之三益,賓主以之敬讓。所謂極乎天,播乎地,窮高遠,測深厚,莫尚於禮也。其樂之五聲,《易》之八象,《詩》之《風雅》,《書》之《典誥》,《春秋》之微婉勸懲,無不本
 乎禮而後立也。其源遠,其流廣,其體大,其義精,非夫睿哲大賢,孰能明乎此哉。況遭暴秦焚亡,百不存一。漢興,始徵召故老,搜集殘文,其體例紕繆,首尾脫落,難可詳論。幸高堂生頗識舊義,諸儒各為章句之說,既明不獨達,所見不同,或師資相傳,共枝別幹。故聞人、二戴,俱事后蒼,俄已分異;盧植、鄭玄,偕學馬融,人各名象。又後之學者,未逮曩時,而問難星繁,充斥兼兩,摛文列錦,煥爛可觀。然而五服之本或差,哀敬之制舛雜,國典未一於四
 海,家法參駁於縉紳,誠宜考詳遠慮,以定皇代之盛禮者也。伏惟陛下欽明玄聖,同規唐、虞,疇咨四岳,興言《三禮》,而伯夷未登,微臣竊位,所以大懼負乘,形神交惡者,無忘夙夜矣。而復猥充搏採之數,與聞爰發之求,實無以仰酬聖旨萬分之一。不敢廢默,謹率管穴所見五十二事上呈。蚩鄙茫浪,伏用竦赧。」



 明年,致仕,拜光祿大夫。歸老在家,手不釋卷,博學多通,特精《三禮》。



 謹於奉公,常手抄書籍。二十八年,卒,時年八十三。



 史臣曰:選賢於野,則治身業弘;求士子朝,則飾智風起。《六經》奧遠,方軌之正路;百家淺末,捷至之偏道。漢世登士,閭黨為先,崇本務學,不尚浮詭,然後可以俯拾青組,顧蔑籝金。於是人厲從師之志,家競專門之術,藝重當時,所居一旦成市,黌舍暫啟,著錄或至萬人。是故仕以學成,身由義立。自魏氏膺命,主愛雕蟲,家棄章句,人重異術。又選賢進士,不本鄉閭,銓衡之寄,任歸臺閣。



 以一人之耳目,究山川之險情,賢否臆斷,萬不值一。由是仕
 憑借譽,學非為己,崇詭遇之巧速,鄙稅駕之遲難,士自此委笥植《經》,各從所務,早往晏退,以取世資。庠序黌校之士,傳經聚徒之業,自黃初至于晉末,百餘年中,儒教盡矣。高祖受命,議創國學,宮車早晏,道未及行。迄于元嘉,甫獲克就,雅風盛烈,未及曩時,而濟濟焉,頗有前王之遺典。天子鸞旗警蹕,清道而臨學館,儲后冕旒黼黻,北面而禮先師,後生所不嘗聞,黃髮未之前睹,亦一代之盛也。臧燾、徐廣、傅隆、裴松之、何承天、雷次宗,並服膺
 聖哲,不為雅俗推移,立名於世,宜矣。潁川庾蔚之、雁門周野王、汝南周王子、河內向琰、會稽賀道養,皆托志經書,見稱於後學。蔚之略解《禮記》,並注賀循《喪服》,行於世云。



\end{pinyinscope}