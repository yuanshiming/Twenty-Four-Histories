\article{卷三十七志第二十七 州郡三}

\begin{pinyinscope}

 荊州郢州湘州
 雍州梁州秦州荊州刺史,漢治武陵漢壽,魏、晉治江陵,王敦治武昌,陶侃前治沔陽,後治武昌,王暠治江陵,庾亮治武昌,庾翼進襄陽,復還夏口;桓溫治江陵,桓沖治上明,王忱還江陵,此後遂治江陵。宋初領郡三十一,後分南陽、順陽、襄陽、新野、竟陵為雍州;湘川十郡為湘州,江夏、武陵屬郢
 州,隨郡、義陽屬司州,北義陽省,凡餘十一郡。文帝世,又立宋安左郡,領拓邊、綏慕、樂寧、慕化、仰澤、革音、歸德七縣,後省改。汶陽郡又度屬。今領郡十二,縣四十八,戶六萬五千六百四。



 去京都水三千三百八十。



 南郡太守,秦立。漢高帝元年,為臨江國,景帝中二年復故。晉武帝太康元年改曰新郡,尋復故。宋初領縣九,後州陵、監利度屬巴陵;旌陽,文帝元嘉十八年省併枝江。二漢無旌陽,見《晉太康地志》,疑是吳所立。凡餘六縣,戶
 一萬四千五百四十四,口七萬五千八十七。



 江陵公相,漢舊縣。



 華容公相,漢舊縣,晉武太康元年省,後復立。



 當陽男相,漢舊縣。



 臨沮伯相,漢舊縣。《晉太康》、《永守地志》屬襄陽,後度。



 編縣男相,漢舊縣。



 枝江侯相,漢舊縣。



 南平內史,吳南郡治江南,領江陵、華容諸縣。晉武帝太
 康元年,分南郡江南為南平郡,治作唐,後治江安。領縣四,戶一萬二千三百九十二,口四萬五千四十九。去州水二百五十,去京都水三千五百,無陸。



 江安侯相,晉武帝太康元年立。



 孱陵侯相,二漢舊縣,屬武陵,《晉太康地志》屬南平。



 作唐侯相,前漢無,後漢屬武陵,《晉太康地志》屬南平。



 南安令,晉武帝分江安立。



 天門太守,吳孫休永安六年,分武陵立。充縣有松梁山,山有石,石開處數十丈,其高以努仰射不至,其上名「天門」,因此名郡。充縣後省。孝武孝建元年,度郢州;明帝泰始三年,復舊。領縣四,戶三千一百九十五。去州水一千二百,陸六百;去京都水三千五百。



 澧陽令,晉武帝太康四年立。



 臨澧令,晉武帝太康四年立。



 零陽令,漢舊縣,屬武陵。



 漊中令,二漢無,《晉太康地志》有,疑是吳立。



 宜都太守,《太康地志》、王隱《地道》、何志並云吳分南郡立;張勃《吳錄》云劉備立。按《吳志》,呂蒙平南郡,據江陵,陸遜別取宜都,獲秭歸、枝江、夷道縣。初權與劉備分荊州,而南郡屬備,則是備分南郡立宜都,非吳立也。習鑿齒云,魏武平荊州,分南郡枝江以西為臨江郡;建安十五年,劉備改為宜都。領縣四,戶一千八百四十三,口三萬四千二百二十。去州水三百五十,無陸;去京都水三千七
 百三十。



 夷道令,漢舊縣。



 佷山男相,前漢屬武陵,後漢屬南郡,晉武帝太康元年改為興山,後復舊。



 宜昌令,何志晉武帝立。按《太康》、《永寧地志》並無,疑是此後所立。



 夷陵令,漢舊縣,吳改曰西陵,晉武帝太康元年復舊。



 巴東公相,譙周《巴記》云,初平元年,荊州帳下司馬趙韙建議分巴郡諸縣漢安以下為永寧郡。建安六年,劉璋改永寧為巴東郡,以涪陵縣分立丹興、漢葭二縣,立巴東屬國都尉,後為涪陵郡。《晉太康地志》,巴東屬梁州,惠帝太安二年度益州;穆帝永和初平蜀,度屬荊州。《永初郡國志》無巴渠、黽陽二縣。領縣七,戶一萬三千七百九十五,口四萬五千二百三十七。去州水一千三百;去京都水四千六百八十。



 魚復侯相,漢舊縣,屬巴郡,劉備章武二年,改為永安,晉武帝太康元年復舊。



 朐令,漢舊縣,屬巴郡。



 新浦令,何志新立。



 南浦令,劉禪建興八年十月,益州牧閻宇表改羊渠立。羊渠不詳,何志吳立。



 漢豐令,何志不注置立。《太康地志》巴東有漢昌縣,疑是。



 巴渠令,何志不注置立。



 黽陽令,何志不注置立。晉末平吳時,峽中立武陵郡,有黽陽、黔陽縣,咸寧元年並省。



 汶陽太守,何志新立。先屬梁州,文帝元嘉十一年度。宋初有四縣,後省汶陽縣。今領三縣,戶九百五十八,口四千九百一十四。去州水七百,陸四百;去京都四千一百。



 僮陽令,何志新立。



 沮陽令,何志新立。



 高安令,何志新立。


南義陽太守
 \gezhu{
  義陽郡別見}
 ,晉末以義陽流民僑立。宋初有四縣,孝武孝建二年,以平陽縣併厥西。平陽本為郡,江左僑立。魏世分河東為平陽郡,晉末省為縣。



 今領縣二,戶一千六百七,口九千七百四十一。



 厥西令,二漢無,《晉太康地志》屬義陽。



 平氏令,漢舊名,屬南陽。


新興太守,《魏志》建安二十年,省雲中、定襄、五原、朔方四
 郡,郡立一縣,合為此郡,屬并州。晉江左僑立。宋初六縣,後省雲中
 \gezhu{
  漢舊名,屬雲中。}
 ;孝武孝建二年,又省九原縣
 \gezhu{
  漢舊名,屬五原。}
 併定襄,宕渠
 \gezhu{
  流寓立。}
 併廣牧。



 凡今領縣三,戶二千三百一,口九千五百八十四。



 定襄令,漢舊名。



 廣牧男相,漢舊名,屬朔方。



 新豐令,漢舊名,屬京兆。僑流立。



 南河東太守,河東郡,秦立。晉成帝咸康三年,征西將軍
 庾亮以司州僑戶立。


宋初八縣,孝武孝建二年,以廣戚
 \gezhu{
  前漢屬沛,後漢、《晉太康地志》屬彭城。江左流寓立。}
 並聞喜,弘農
 \gezhu{
  江左立僑郡,後併省為縣。}
 、臨汾併松滋,安邑並永安。
 \gezhu{
  臨汾、安邑,漢舊名。臨汾後屬平陽。}
 今領縣四,戶二千四百二十三,口一萬四百八十七。去州水一百二十;去京都水三千五百。



 聞喜令,故曲沃,秦改為左邑。漢武帝元鼎六年,行幸至此,聞南越破,改名聞喜。



 永安令,前漢彘縣,順帝陽嘉二年更名,後屬平陽。



 松滋令,前漢屬廬江,後漢無,晉屬安豐。疑是有流民寓荊土,故立。


譙縣令
 \gezhu{
  別見}
 ,譙流民寓立。



 建平太守,吳孫休永安三年,分宜都立,領信陵、興山、秭歸、沙渠四縣。晉又有建平都尉,領巫、北井、泰昌、建始四縣。晉武帝咸寧元年,改都尉為郡,於是吳、晉各有建平郡。太康元年吳平,併合。五年,省建始縣,後復立。《永初郡國》有南陵、建始、信陵、興山、永新、永寧、平樂七縣,今並無。
 按《太康地志》無南陵、永新、永寧、平樂、新鄉五縣,疑是江左所立。信陵、興山、沙渠,疑是吳立。建始,晉初所立也。領縣七,戶一千三百二十九,口二萬八百一十四。去州水陸一千;去京都水四千三百八十。



 巫令,漢舊縣。



 秭歸侯相,漢舊縣。



 歸鄉公相,何志,故屬秭歸,吳分。按《太康地志》云,秭歸有歸鄉,故夔子國,楚滅之,而無歸鄉縣,何志
 所言非也。



 北井令,《晉太康地志》有。先屬巴東,晉武帝泰始五年度建平。



 泰昌令,《晉太康地志》有。



 沙渠令,《晉起居注》,太康元年立。按沙渠是吳建平郡所領,吳平不應方立,不詳。



 新鄉令。


永寧太守,晉安帝僑立為長寧郡;宋明帝以名與文帝
 陵同,改為永寧。宋初五縣,後省綏安
 \gezhu{
  晉安帝立}
 。孝武孝建二年後,以僮陽
 \gezhu{
  晉安帝立}
 併長寧,綏寧
 \gezhu{
  晉安帝立}
 併上黃。今領縣二,戶一千一百五十七,口四千二百七十四。去州陸六十;去京都三千四百三十。



 長寧侯相,晉安帝立。



 上黃男相,宋初屬襄陽,後度。二漢、晉並無此縣。



 武寧太守,晉安帝隆安五年,桓玄以沮、漳降蠻立。領縣二,戶九百五十八,口四千九百一十四。



 樂鄉令,晉安帝立。



 長林男相,晉安帝立。



 郢州刺史,魏文帝黃初三年,以荊州江北諸郡為郢州,其年罷并荊,非今地。



 吳又立郢州。孝武孝建元年,分荊州之江夏、竟陵、隨、武陵、天門,湘州之巴陵,江州之武昌,豫州之西陽,又以南郡之州陵、監利二縣度屬巴陵,立郢州。天門後還荊。領郡六,縣三十九,戶二萬九千四百六十九,口十五萬八千五百八十七。去京都水二千一
 百。



 江夏太守,漢高帝立,本屬荊州。《永初郡國》及何志並治安陸,此後治夏口。



 又有安陸、曲陵,曲後別郡。領縣七,戶五千七十二,口二萬三千八百一十。



 汝南侯相,本沙羨土,晉末汝南郡民流寓夏口,因立為汝南縣。沙羨令,漢舊縣,吳省。晉武太康元年復立,治夏口。孝武太元三年,省併沙陽,後以其地為汝南實土。



 沌陽子相,江左立。



 孝昌侯相,《永初郡國》、何志並無,徐志有,疑是孝武世所立。



 惠懷子相,江左立。



 沙陽男相,二漢舊縣,本名沙羨,屬武昌,晉武帝太康元年更名;又立沙羨,而沙陽徙今所治。文帝元嘉十六年度巴陵,孝武孝建元年度江夏。



 羨陽子相,晉惠帝世,安陸人朱伺為陶侃將,求分
 安陸東界為此縣。



 蒲圻男相,晉武帝太康元年立。本屬長沙,文帝元嘉十六年度巴陵,孝武孝建元年度江夏。



 竟陵太守,晉惠帝元康九年,分江夏西界立。何志又有宋縣,徐無。領縣六,戶八千五百九十一,口四萬四千三百七十五。去州水一千四百;去京都水三千四百。



 萇壽令,明帝泰始六年立。



 竟陵侯相,漢舊縣,屬江夏。



 新市子相,漢舊縣,屬江夏。



 霄城侯相,《永初郡國》有,何、徐不注置立。



 新陽男相,《永初郡國》有,何、徐不注置立。



 雲杜侯相,漢舊縣,屬江夏。



 武陵太守,《前漢地理志》,高帝立。《續漢郡國志》云,秦昭王立,名黔中郡,高帝五年更名。本屬荊州。領縣十,戶五千九十,口三萬七千五百五十五。去州水一千;去京都水三千。



 臨沅男相,漢舊縣。



 龍陽侯相,《晉太康地理志》、何志吳立。



 漢壽伯相,前漢立,後漢順帝陽嘉三年更名。吳曰吳壽,晉武帝復舊。



 沅南令,漢光武建武二十六年立。



 遷陵侯相,漢舊縣。



 辰陽男相,漢舊縣。



 舞陽令,前漢作無陽,後漢無,《晉太康地志》有。



 酉陽長,漢舊縣。



 黚陽長,二漢無,《晉太康地志》有。



 沅陵令,漢舊縣。



 巴陵太守,文帝元嘉十六年,分長沙之巴陵、蒲圻、下雋,江夏之沙陽四縣立,屬湘州;孝武孝建元年,割南郡之監利、州陵度江夏,屬郢州。二年,又度長寧之綏安屬巴陵。何志訖元嘉二十年,巴陵郡以十六年立,應在何志而闕。領縣四,戶五千一百八十七,口二萬五千三百一
 十六。去州水五百,去京都水二千五百。



 巴陵男相,晉武帝太康元年立,屬長沙。本領度支校尉,立郡省。



 下雋侯相,漢舊縣,屬長沙。



 監利侯相,按《晉起居注》,太康四年,復立南郡之監利縣,尋復省之。言由先有而被省也,疑是吳所立,又是吳所省。孝武孝建元年度。



 州陵侯相,漢舊縣,屬南郡,晉武帝太康元年復立,
 疑是吳所省也。孝武孝建元年度。明帝泰始四年,以綏安縣併州陵。



 武昌太守,《晉起居注》,太康元年,改江夏為武昌郡。領縣三,戶二千五百四十六,口一萬一千四百一十一。去京都水一千一百。



 武昌侯相,魏文帝黃初二年,孫權改鄂為武昌。



 陽新侯相,吳立。



 鄂令,漢舊縣,屬江夏。吳改鄂為武昌,晉武帝太康
 元年,復立鄂縣,而武昌如故。



 西陽太守,本縣名,二漢屬江夏,魏立弋陽郡,又屬焉。晉惠帝又分弋陽為西陽國,屬豫州;宋孝武孝建元年,度郢州;明帝泰始五年,又度豫,後又還郢。



 《永初郡國》、何、徐並有弋陽縣。今領縣十,戶二千九百八十三,口一萬六千一百二十。去州水二百八十;去京都水一千七百二十。



 西陽令,漢舊縣,屬江夏,後屬弋陽。



 西陵男相,漢舊縣,屬江夏,後屬弋陽。



 孝寧侯相,本軑縣,漢舊縣。孝武自此伐逆,即位改名。



 蘄陽令,二漢江夏郡有蘄春縣,吳立為郡;晉武帝太康元年,省蘄春郡,而縣屬弋陽,後屬新蔡;孝武大明八年,還西陽。



 義安令,明帝泰始二年以來流民立。



 蘄水左縣長,文帝元嘉二十五年,以豫部蠻民立
 建昌、南川、長風、赤亭、魯亭、陽城、彭波、遷溪、東丘、東安、西安、南安、房田、希水、高坡、直水、蘄水、清石十八縣,屬西陽。孝武大明八年,赤亭、彭波併陽城,其餘不詳何時省。



 東安左縣長,前廢帝永光元年,復以西陽蘄水、直水、希水三屯為縣。



 建寧左縣長,孝武大明八年省建寧左郡為縣,屬西陽。徐志有建寧縣,當是此後為郡。



 希水左縣長。



 陽城左縣長,本屬建寧左郡,孝武大明八年,省西陽之赤亭、陽城、彭城三縣併建寧之陽城縣,而以縣屬西陽。



 湘州刺史,晉懷帝永嘉元年,分荊州之長沙、衡陽、湘東、邵陵、零陵、營陽、建昌,江州之桂陽八郡立,治臨湘。成帝咸和三年省。安帝義熙八年復立,十二年又省。宋武帝永初三年又立,文帝元嘉八年省;十六年又立,二十九
 年又省。孝武孝建元年又立。建昌郡,晉惠帝元康九年,分長沙東北下雋諸縣立,成帝咸康元年省。元嘉十六年,立巴陵郡屬湘州,後度郢。領郡十,縣六十二,戶四萬五千八十九,口三十五萬七千五百七十二。去京都水三千三百。



 長沙內史,秦立。宋初十縣,下雋、蒲圻、巴陵屬巴陵。今領縣七,戶五千六百八十四,口四萬六千二百一十三。



 臨湘侯相,漢舊縣。



 醴陵侯相,後漢立。



 瀏陽侯相,吳立。



 吳昌侯相,後漢立,曰漢昌,吳更名。



 羅縣侯相,漢舊縣。



 攸縣子相,漢舊縣。



 建寧子相,吳立。



 衡陽內史,吳孫亮太平二年,分長沙西部都尉立。領縣七,戶五千七百四十六,口二萬八千九百九十一。去州
 水二百二十;去京都水三千七百。



 湘西令,吳立。



 湘南男相,漢舊縣,屬長沙。



 益陽侯相,漢舊縣,屬長沙。



 湘鄉男相,前漢無,後漢屬零陵。



 新康男相,吳曰新陽,晉武帝太康元年更名。



 重安侯相,前漢曰鐘武,後漢順帝永建三年更名,屬零陵。



 衡山男相,吳立曰衡陽,晉惠帝更名。



 桂陽太守,漢高立,屬荊州,晉惠帝元康元年度江州。領縣六,戶二千二百一十九,口二萬二千一百九十二。去州水一千四百,去京都水四千九百四十。



 郴縣伯相,漢舊縣。



 耒陽子相,漢舊縣。



 南平令,漢舊縣。



 臨武令,漢舊縣。



 汝城令,江左立。



 晉寧令,漢順帝永和元年立,曰漢寧,吳改曰陽安,晉武帝太康元年改曰晉寧。



 零陵內史,漢武帝元鼎六年立。領縣七,戶三千八百二十八,口六萬四千八百二十八。去州一千四百;去京都水四千八百。



 泉陵子相,漢舊縣。



 洮陽侯相,漢舊縣。



 零陵子相,漢舊縣。



 祁陽子相,吳立。明帝泰始初度湘東,五年復舊。



 應陽男相,晉惠帝分觀陽立。



 觀陽男相,吳立。



 永昌令,吳立。



 營陽太守,江左分零陵立。領縣四,戶一千六百八,口二萬九百二十七。去州水一千七百一;去京都水五千五百五十。



 營浦侯相,漢舊縣,屬零陵。



 營道侯相,漢舊縣,屬零陵。



 舂陵令,前漢舊縣,舂陵侯徙國南陽,省。吳復立,屬零陵。



 泠道令,漢舊縣,屬零陵。


湘東太守,吳孫亮太平二年,分長沙東部都尉立。晉世七縣,孝武太元二十年,省酃
 \gezhu{
  漢舊縣}
 、利陽、新平
 \gezhu{
  張勃《吳錄》有此二縣,利作梨,晉作利音。}
 三縣。今領縣五,戶一千三百九十六,口一萬
 七千四百五十。去州水陸七百;去京都水三千六百。



 臨烝伯相,吳屬衡陽,《晉太康地志》屬湘東。



 新寧令,吳立。



 茶陵子相,漢舊縣,屬長沙。



 湘陰男相,後廢帝元徽二年,分益陽、羅、湘西及巴、硤流民立。



 陰山令,陰山乃是漢舊縣,而屬桂陽。吳湘東郡有此陰山縣,疑是吳所立。



 邵陵太守,吳孫皓寶鼎元年,分零陵北部都尉立。領縣七,戶一千九百一十六,口二萬五千五百六十五。去州水七百,陸一千三百;去京都水四千五百。



 邵陵子相,何志屬長沙。按二漢無,《吳錄》屬邵陵。



 武剛令,晉武分都梁立。



 建興男相,晉武帝分邵陵立。



 高平男相,吳立。晉武帝太康元年,改曰南高平,後更曰高平。



 都梁令,漢舊縣,屬零陵。



 邵陽男相,吳立曰昭陽,晉武改。



 扶縣令,漢舊縣,至晉曰夫夷。漢屬零陵,晉屬邵陵。案今云扶者,疑是避桓溫諱去「夷」,「夫」不可為縣名,故為「扶」云。



 廣興公相,吳孫皓甘露元年,分桂陽南部都尉,立為始興郡。晉武帝平吳,以屬廣州,成帝度荊州;宋文帝元嘉二十九年,又度廣州;三十年,復度湘州。明帝泰始六年,立
 岡湲縣,割始興之封陽、陽山、含洭三縣,立宋安郡,屬湘州。泰豫元年復囗,省岡湲縣,改始興曰廣興。領縣七,戶一萬一千七百五十六,口七萬六千三百二十八。去州水二千三百九十;去京都水五千。



 曲江侯相,漢舊縣,屬桂陽。



 桂陽令,漢舊縣,屬桂陽。



 陽山侯相,漢舊縣,後漢曰陰山,屬桂陽。吳始興郡無此縣,當是晉後立。



 貞陽侯相,漢舊縣,名湞陽,屬桂陽。宋明帝泰始三年,改「湞」為「貞」。



 含洭男相,漢舊縣,屬桂陽。



 始興令,吳立。



 中宿令,漢舊縣,屬南海,吳度。



 臨慶內史,吳分蒼梧立為臨賀郡,屬廣州;晉成帝度荊州;宋文帝元嘉二十九年,度廣州;三十年,復度湘州。明帝改名。領縣九,戶三千七百一十五,口三萬一千五百
 八十七。去州水陸二千八百;去京都水陸五千五百七十。



 臨賀侯相,漢舊縣。《晉太康地志》、王隱云屬南海,而二漢屬蒼梧,當是吳所度。



 馮乘侯相,漢舊縣,屬蒼梧。



 富川令,漢舊縣,屬蒼梧。



 封陽侯相,漢舊縣。



 興安侯相,吳立曰建興,晉武帝太康元年更名。



 謝沐長,漢舊縣,屬蒼梧。



 寧新令,二漢無,當是吳所立,屬蒼梧,晉武帝太康元年更名。


開建令,文帝分封陽立宋昌、宋興、開建、武化、徃徃
 \gezhu{
  徃音生}
 、永固、綏南七縣。後又分開建、武化、宋昌三縣立宋建郡,屬廣州。孝武大明元年悉省,唯餘開建縣。



 撫寧令,宋末立。



 始建內史,吳孫皓甘露元年,分零陵南部都尉立始安郡,屬廣州;晉成帝度荊州;宋文帝元嘉二十九年,度廣州;三十年,復度湘州。明帝改名。領縣七,戶三千八百三十,口二萬二千四百九十。去州水二千八十,陸二千六百三十;去京都水五千五百九十。



 始安子相,漢舊縣,屬零陵。



 熙平令,吳立為尚安,晉武改。



 永豐男相,吳立。



 荔浦令,漢舊縣,屬蒼梧。



 平樂侯相,吳立。



 建陵男相,吳立,屬蒼梧,宋末度。



 樂化左令,宋末立。



 雍州刺史,晉江左立。胡亡氐亂,雍、秦流民多南出樊、沔,晉孝武始於襄陽僑立雍州,並立僑郡縣。宋文帝元嘉二十六年,割荊州之襄陽、南陽、新野、順陽、隨五郡為雍州,而僑郡縣猶寄寓在諸郡界。孝武大明中,又分實土
 郡縣以為僑郡縣境。徐志雍州有北上洛、北京兆、義陽三郡。北上洛,晉孝武立,領上洛、北商、酆陽、陽亭、北拒陽五縣。北京兆領北藍田、霸城、山北三縣。並云景平中立。義陽,云晉安帝立,領平氏、襄鄉二縣。酆陽、陽亭、北拒陽,並云安帝立,餘縣不注置立。今並無此三郡。今領郡十七,縣六十,戶三萬八千九百七十五,口十六萬七千四百六十七。去京都水四千四百,陸二千一百。


襄陽公相,魏武帝平荊州,分南郡編以北及南陽之山
 都立,屬荊州。魚豢云,魏文帝立。《永初郡國》、何志並有宜城
 \gezhu{
  漢舊縣,屬南郡。}
 、鄀、上黃縣
 \gezhu{
  並別見}
 。徐志無。領縣三,戶四千二十四,口一萬六千四百九十六。



 襄陽令,漢舊縣,屬南郡。



 中廬令,漢舊縣,屬南郡。



 巳阜縣令,漢舊縣,屬南郡。


南陽太守,秦立,屬荊州。《永初郡國》有比陽、魯陽、赭陽、西鄂、俯犨、葉、雉、博望八縣
 \gezhu{
  並漢舊縣}
 。何志無犨、雉。徐志無比陽、魯
 陽、赭陽、西鄂、博望,而有葉,餘並同。孝武大明元年,省葉縣。領縣七,戶四千七百二十七,口三萬八千一百三十二。去州三百六十,去京都水四千四百。



 宛縣令,漢舊縣。



 涅陽令,漢舊縣。



 云陽男相,漢舊縣。故名育陽,晉孝武改。



 冠軍令,漢舊縣,武帝分穰立。



 酈縣令,漢舊縣。



 舞陰令,漢舊縣。



 許昌男相,徐志無,此後所立。本屬潁川。


新野太守,何志晉惠帝分南陽立。《永初郡國》、何志有棘陽
 \gezhu{
  別見}
 、蔡陽、鄧縣
 \gezhu{
  並漢舊縣}
 。徐無。孝武大明元年,省蔡陽。今領縣五,戶四千二百三十五,口一萬四千七百九十三。去州一百八十;去京都水四千五百八十。



 新野侯相,漢舊縣,屬南陽。文帝元嘉末省,孝武大明元年復立。



 山都男相,漢舊縣,屬南陽,《晉太康地志》屬襄陽,《永初郡國》及何、徐屬新野。



 池陽令,漢舊名,屬馮翊,《晉太康地志》屬京兆。僑立亦屬京兆。孝武大明中土斷,又屬此。



 穰縣令,漢舊縣,屬南陽。



 交木令,孝武大明元年立。


順陽太守,魏分南陽立曰南鄉,晉武帝更名。成帝咸康四年,復立南鄉,後復舊。《永初郡國》及何志有朝陽、武當、
 酂、陰、汎陽、築
 \gezhu{
  並別見}
 、析
 \gezhu{
  前漢屬弘農,後漢屬南陽。}
 修陽
 \gezhu{
  唯見《永初郡國》}
 凡八縣。徐志唯增朝陽。朝陽,孝武大明元年省。領縣七,戶四千一百六十三,口二萬三千一百六十三。



 南鄉令,前漢無,後漢有,屬南陽。



 槐里男相,漢舊名,屬扶風,《晉太康地志》屬始平。僑立亦屬始平。大明土斷屬此。



 順陽侯相,前漢曰博山,後漢明帝更名,屬南陽。



 清水令,前漢屬天水,後漢為天水漢陽,無此縣。《晉
 太康地志》屬略陽。僑立屬始平。大明土斷屬此。



 朝陽令,漢舊縣。



 丹水令,前漢屬弘農,後漢屬南陽。何志魏立,非也。



 鄭縣令,漢舊名,屬京兆。僑立亦屬京兆,後度此。



 京兆太守,故秦內史。漢高帝元年,屬塞國;二年,更為渭南郡;九年罷,復為內史。武帝建元六年,分為右內史;太初元年,更為京兆尹,魏改為京兆郡。初僑立,寄治襄陽。朱序沒氐。孝武太元十一年復立。大明土斷,割襄陽西
 界為實土。


雍州僑郡先屬府,武帝永初元年屬州。《永初郡國》有藍田
 \gezhu{
  漢舊縣}
 、鄭、池陽
 \gezhu{
  並別見}
 、南霸城
 \gezhu{
  本霸陵,漢舊縣。《太康地志》曰,霸城何志魏地。}
 、新康五縣。何志無新康而有新豐。徐無。孝武大明元年,省京兆之盧氏、藍田、霸城縣。盧氏當是何志後所立,二漢屬弘農,《晉太康地志》屬上洛。新康疑是晉末所立。領縣三,戶二千三百七,口九千二百二十三。



 杜令,二漢曰杜陵,魏改。



 鄧縣令,漢舊縣,屬南陽。



 新豐令,漢舊縣。


始平太守,晉武帝泰始二年,分京兆、扶風立。後分京兆、扶風僑立,治襄陽;今治武當。《永初郡國》唯有始平、平陽、清水
 \gezhu{
  別見}
 三縣。何志有槐里
 \gezhu{
  別見}
 、宋寧、宋嘉
 \gezhu{
  何志新立}
 三縣,而清水、始平與《永初郡國》同。領縣四,戶二千七百九十七,口五千五百十二。



 武當侯相,漢舊縣,屬南陽,後屬順陽。



 始平令,魏立。



 武功令,漢舊名,故屬扶風,《晉太康地志》屬始平。



 平陽子相,江左平陽郡民流寓,立此。


扶風太守,故秦內史。高帝元年,屬雍國;二年,更為中地郡,九年罷。後為內史。武帝建元六年,分為右內史;太初元年,更名為右扶風。僑立,治襄陽,今治築口。《永初郡國》及何志唯有郿、魏昌縣
 \gezhu{
  魏昌,魏立,屬中山}
 。孝武大明元年省魏昌。領縣三,戶二千一百五十七,口七千二百九十。



 築陽令,漢舊縣,屬南陽,又屬順陽。大明土斷屬此。



 郿縣令,漢舊名,屬扶風,《晉太康地志》屬秦國。



 汎陽令,晉武帝太康五年立,屬南鄉,仍屬順陽。大明土斷屬此。



 南上洛太守,《永初郡國》、何志雍州並有南上洛郡,寄治魏興,今梁州之上洛是也。此上洛蓋是何志以後僑立耳。今治臼。何、徐志雍州南上洛,晉武帝立,北上洛云晉孝武立,非也。徐有南北陽亭、陽安縣,不注置立。今領縣二,戶一百四十四,口四百七十七。


上洛男相。
 \gezhu{
  別見}


商縣令。
 \gezhu{
  別見}



 河南太守,故秦三川郡,漢高帝更名。光武都雒陽,建武十五年,改曰河南尹。


僑立,始治襄陽,孝武大明中,分沔北為境。《永初郡國》及何志並又有陽城、緱氏縣
 \gezhu{
  漢舊名,並屬河南。}
 ,徐無此二縣,而有僑洛陽。
 \gezhu{
  漢舊名。}
 、陽城縣,孝武大明元年省。洛陽,當是何志後立。領縣五,戶三千五百四十一,口一萬三千四百七十。去州陸三十五。



 河南令,漢舊名。



 新城令,漢舊名。



 河陰子相,魏立。



 棘陽令,漢縣,故屬南陽,《晉太康地志》屬義陽,後屬新野。大明土斷屬此。



 襄鄉令,前漢無,後漢有,屬南陽。徐志屬義陽。當是大明土斷屬此。


廣平太守
 \gezhu{
  別見}
 ,江左僑立,治襄陽,今為實土。《永初郡國》及
 何志並又有易陽、曲周、邯鄲
 \gezhu{
  並見在}
 ,無酂、比陽。徐無復邯鄲縣。易陽、曲周,孝武大明元年省。邯鄲應是土斷省。領縣四,戶二千六百二十七,口六千二百九十三。



 廣平令,漢舊名。徐志,南度以朝陽縣境立。



 酂縣令,漢舊縣,屬南陽,後屬順陽。



 比陽令,漢舊縣,屬南陽。



 陰縣令,漢舊縣,屬南陽。


義成太守,晉孝武立,治襄陽,今治均。《永初郡國》又有下
 蔡、平阿縣
 \gezhu{
  二縣前漢屬沛,後漢屬九江,《晉太康地志》屬淮南。}
 ,何同。孝武大明元年省下蔡,始亦流寓立也。平阿當是何志後省。領縣二,戶一千五百二十一,口五千一百一。



 義成侯相,晉孝武立。



 萬年令,漢舊名,屬馮翊。


馮翊太守,故秦內史。高帝元年,屬塞國,二年,更名為河上郡;九年罷,復為內史。武帝建元六年,分為左內史。太初元年,更名。三輔流民出襄陽,文帝元嘉六年立,則何
 志應有而無。治襄陽。今治鄀。領縣三,
 \gezhu{
  疑}
 戶二千七十八,口五千三百二十一。



 鄀縣令,漢舊縣,屬南郡,作「若」字。《晉太康地志》作「鄀」。《永初郡國》及何志屬襄陽,徐屬此。



 高陸令,《晉太康地志》屬京兆。《永初郡國》、何志並無,孝武大明元年復立。


南天水太守
 \gezhu{
  天水郡別見}
 ,徐志本西戎流寓。今治巖州。《永初郡國》、何志並無,當是何志後所立。又有冀縣
 \gezhu{
  漢舊名}
 ,孝
 武大明元年省。領縣四,戶六百八十七,口三千一百二十二。



 華陰令,前漢屬京兆,後漢、魏、晉屬弘農。



 西縣令,前漢屬隴西,後漢屬漢陽,即天水。魏、晉屬天水。


略陽侯相。
 \gezhu{
  別見}


河陽令。
 \gezhu{
  別見}



 建昌太守,孝建元年,刺史朱脩之免軍戶為永興、安寧
 二縣,立建昌郡;又立永寧為昌國郡,並寄治襄陽。昌國後省。徐志,建昌又有永寧縣,今無。領縣二,戶七百三十二,口四千二百六十四。



 永興令。



 安寧男相。



 華山太守,胡人流寓,孝武大明元年立。今治大堤。領縣三,戶一千三百九十九,口五千三百四十二。



 華山令,與郡俱立。



 藍田令,漢舊名,本屬京兆。



 上黃令,本屬襄陽,立郡割度。



 北河南太守,晉孝武太元十年立北河南郡,後省。《永初郡國》、何、徐志並無。明帝泰始末復立。寄治宛中。領縣八。


新蔡令。
 \gezhu{
  別見}
 。


汝陰令。
 \gezhu{
  別見}


苞信令。
 \gezhu{
  別見}


上蔡令。
 \gezhu{
  別
  見}


固始令。
 \gezhu{
  別見}


緱氏令。
 \gezhu{
  別見}


新安令。
 \gezhu{
  別見}


洛陽令。
 \gezhu{
  別見}



 弘農太守,漢武帝元鼎四年立。宋明帝末立,寄治五壟。領縣三。



 邯鄲令,漢舊名,屬趙國。《晉太康地志》無此縣。



 圉縣令,前漢屬淮陽,後漢屬陳留。《晉太康地志》無
 此縣。


盧氏令。
 \gezhu{
  別見}



 梁州刺史,《禹貢》舊州,周以梁併雍,漢以梁為益,治廣漢雒縣。魏元帝景元四年平蜀,復立梁州,治漢中南鄭,而益州治成都。李氏據梁、益,江左於襄陽僑立梁州。李氏滅,復舊。譙縱時,又治漢中。刺史治魏興。縱滅,刺史還治漢中之苞中縣,所謂南城也。文帝元嘉十年,刺史甄法護於南城失守,刺史蕭思話還治南鄭。《永初郡國》又有
 宕渠郡、北宕渠郡。《宋起居注》,元嘉十六年,割梁州宕渠郡度益州。今益部宕渠郡曰南宕渠。何、徐並有北宕渠郡,唯領宕渠一縣。何云,本巴西流民。今無。


漢中太守,秦立。漢獻帝建安二十年,魏武平張魯,復漢寧郡為漢中,疑是此前改漢中曰漢寧也。晉地記云,孝武太元十五年,梁州刺史周瓊表立。又疑是李氏所省,李氏平後復立。《永初郡國》又有苞中、懷安
 \gezhu{
  漢、晉、何、徐並無二縣。}



 二縣。領縣四,戶一千七百八十六,口一萬三百三十四。



 南鄭令,漢舊縣。



 城固令,漢舊縣。



 沔陽令,漢舊縣。



 西鄉令,蜀立曰南鄉,晉武帝太康二年更名。



 魏興太守,魏文帝以漢中遺民在東垂者立,屬荊州。江左還本。領縣十三。


\gezhu{
  疑}
 去州一千二百;去京都水六千七百。



 西城令,漢舊縣,屬漢中。



 鄖鄉令,本錫縣,二漢舊縣,屬漢中,後屬魏興;魏、晉世為郡,後省。武帝太康五年,改為鄖鄉。何志晉惠帝立,非也。



 錫縣令,前漢長利縣,屬漢中,後漢省。晉武帝太康四年復立,屬魏興。五年,改長利為錫。



 廣城令,《永初郡國》、何、徐並有,不注置立。



 興晉令,魏立曰平陽,晉武帝太康元年更名。



 旬陽令,前漢有,後漢無,晉武帝太康四年復立。



 上庸令,《晉太康地志》、《永初郡國》、徐並屬上庸,何無。



 長樂令,《永初郡國》、何、徐並屬晉昌。本蜀郡流民。



 廣昌子相,何志屬上庸,晉成帝立。晉地記,武帝太康元年,改上庸之廣昌為庸昌,二年省。疑是魏所立。



 安晉令,《永初郡國》、何、徐屬晉昌。本蜀郡流民。



 延壽令,《永初郡國》、何、徐屬晉昌。本蜀郡流民。



 宣漢令,《永初郡國》、何、徐屬晉昌。本建平流離民。



 新興太守,《永初郡國》、何、徐云新興、吉陽、東關三縣,屬晉昌郡。何云晉元帝立,本巴、漢流民。宋末省晉昌郡,立新興郡,以晉昌之長樂、安晉、延壽、安樂屬魏興郡,宣漢屬巴渠郡,寧都屬安康郡。《永初郡國》有永安縣,何、徐無。



 今亦無復新興縣。何云巴東夷人。今領縣二。



 吉陽令,本益州流民。



 東關令,本建平流民。



 新城太守,故屬漢中,魏文帝分立,屬荊州。江左還本。領
 縣六,戶一千六百六十八,口七千五百九十四。去州陸一千五百;去京都水五千三百。



 房陵令,漢舊縣,屬漢中,《太康地志》、王隱無。



 綏陽令,魏立,後改為秭歸,晉武帝太康二年,復為綏陽。



 昌魏令,魏立。


祁鄉令,何志魏立。《晉太康地志》作「沶」
 \gezhu{
  音祁}
 。



 閬陽令,何志不注置立。



 樂平令,何志不注置立。



 上庸太守,魏明帝太和二年,分新城之上庸、武陵、北巫為上庸郡。景初元年,又分魏興之魏陽,錫郡之安富、上庸為郡。疑是太和後省,景初又立也。魏屬荊州,江左還本。《永初郡國》有上庸、廣昌。何有廣昌。領縣七,戶四千五百五十四,口二萬六百五十三。去州陸二千三百;去京都水六千七百。



 上庸令,漢舊縣,屬漢中。



 安富令,《晉太康地志》、《永初郡國》、何、徐並有。



 北巫令,何志晉武帝立。按魏所分新城之北巫,應即是此縣,然則非晉武立明矣。



 微陽令,魏立曰建始,晉武帝改。



 武陵令,前漢屬漢中,後漢、《晉太康地志》、王隱並無。



 新安令,《永初郡國》、何、徐有。何云本建平流民。



 吉陽令,《永初郡國》云北吉陽,何、徐無。



 晉壽太守,晉地記云,孝武太元十五年,梁州刺史周瓊
 表立。何志故屬梓潼。



 而益州南晉壽郡悉有此諸縣。《永初郡國》、徐又有南晉壽、南興、樂南、興安縣。



 何無南興樂,云南晉壽,惠帝立,餘並不注置立。今領縣四,去州陸一千二百;去京都水一萬。



 晉壽令,屬梓潼。何志晉惠帝立。按《晉起居注》,武帝太康元年,改梓潼之漢壽曰晉壽。漢壽之名,疑是蜀立,云惠帝立,非也。



 白水令,漢舊縣,屬廣漢,《晉太康地志》屬梓潼。



 邵歡令,《永初郡國》、何、徐並有,不注置立,疑是蜀立曰昭歡,晉改也。



 興安令,《永初郡國》、何、徐並有,不注置立。



 華陽太守,徐志新立。《永初郡國》、何並無,寄治州下。領縣四。戶二千五百六十一,口萬五千四百九十四。



 華陽令。



 興宋令。



 宕渠令。



 嘉昌令,徐不注置立。



 新巴太守,晉安帝分巴西立。何、徐又有新歸縣,何云新立,今無。領縣三。



 戶三百九十三,口二千七百四十九。



 新巴令,晉安帝立。



 晉城令,晉安帝立。



 晉安令,晉安帝立。



 北巴西太守,何志不注置立。《宋起居注》,文帝元嘉十二年,於劍南立北巴西郡,屬益州。今益州無此郡。又《永初
 郡國》、何、徐梁州並有北巴西而益州無,疑是益部僑立,尋省;梁州北巴西是晉末所立也。《永初郡國》領閬中、漢昌二縣。


何又有宋昌縣,云新立。徐無宋昌,有宋壽。何、徐並領縣四,今六。
 \gezhu{
  疑}
 去州一千四百;去京都水九千九百。


閬中令。
 \gezhu{
  別見}


安漢令。
 \gezhu{
  別見}


南國令。
 \gezhu{
  即南充國,別見。}


西國令。
 \gezhu{
  即西充國,
  別見。}



 平周令,益州巴西有平州縣。



 北陰平太守,《晉太康地志》故廣漢屬國都尉。何志蜀分立。《永初郡國》曰北陰平,領陰平、綿竹、平武、資中、胄旨五縣。何、徐直曰陰平,領二縣與此同。



 戶五百六,口二千一百二十四。寄治州下。


陰平令,前漢、後漢屬廣漢屬國,名宙底。《晉太康地志》陰平郡陰平縣注云,宙底。當是故宙底為陰平。《永初郡國》胄旨縣,即宙底也。
 \gezhu{
  當是後又立此縣,而字誤也。}



 平武令,蜀立曰廣武,晉武帝太康元年更名。



 南陰平太守,《永初郡國》唯領陰平一縣。徐志無南字,云陰平舊民流寓立,唯領懷舊一縣。何無。今領縣二,戶四百七。



 陰平令。



 懷舊令,徐志不注置立。



 巴渠太守,何志新立。領縣七,戶五百,口二千一百八十三。


宣漢令
 \gezhu{
  別見}
 ,與郡新立。



 始興令,何志新立。



 巴渠令,何志新立。



 東關令,何志新立。



 始安令,何志新立。



 下蒲令,何志無,徐志不注置立。



 晉興令,何志晉安帝立。案《永初郡國》,梁部諸郡,唯巴西有此縣,不容是此晉興。若是晉安帝時立,
 便應在《永初郡國》,疑何謬也。



 懷安太守,何志新立。領縣二,戶四百七,口二千三百六十六。寄治州下。



 懷安令,何志新立。



 義存令,何志新立。



 宋熙太守,何、徐志新立。領縣五,戶一千三百八十五,口三千一百二十八。



 去州七百;去京都九千八百。



 興樂令。



 歸安令。



 宋安令。



 元壽令。



 嘉昌令,何志五縣並新立。



 白水太守,《永初郡國》、何並無,徐志仇池氐流寓立。有漢昌縣。今領縣六。



 戶六百五。



 新巴令。



 漢德令。



 晉壽令。



 益昌令。



 興安令。



 平周令,徐志作「平州」。此五縣,徐並不注置立。



 南上洛太守,《晉太康地志》分京兆立上洛郡,屬司隸。《永初郡國》、何志並屬雍州,僑寄魏興,即此郡也。徐志巴民新立,徐志時已屬梁州矣。《永初郡國》無豐陽而有陽亭,何、徐有,何不注陽亭置立。領縣六。



 上洛令,前漢屬弘農,後漢屬京兆。何云魏立,非也。



 商縣令,上洛同。



 流民令,何不注置立。



 豐陽長,《永初郡國》無,何作酆陽,新立。徐作豐。



 渠陽令,《永初郡國》、何、徐並作拒陽。



 義縣令,《永初郡國》、何、徐並無。



 北上洛太守,徐志巴民新立。領縣七,戶二百五十四。



 北上洛令。



 豐陽令。



 流民令。



 陽亭令。



 拒陽令,「拒」字與南上洛不同。



 商縣令,徐志無。



 西豐陽令,徐志無。



 安康太守,宋末分魏興之安康縣及晉昌之寧都縣立。



 安康令,二漢安陽縣,屬漢中,漢末省。魏復立,屬魏
 興。晉武帝太康元年更名。何云魏立,非也。



 寧都令,蜀郡流民。



 南宕渠太守,《永初郡國》有宕渠郡,領宕渠、漢興、宣漢三縣,屬梁州;元嘉十六年,度屬益州,非此南宕渠也。何、徐梁並無此郡,疑是徐志後所立。



 宕渠令。



 漢安令。



 宣漢令。



 宋康令。三縣並新置。



 懷漢太守,孝武孝建二年立。領縣三,戶四百十九。



 永豐長。



 綏來長。



 預德長。



 秦州刺史,晉武帝太始五年,分隴右五郡及涼州金城、梁州陰平并七郡為秦州,治天水冀縣;太康三年併雍州,惠帝元康七年復立。何志晉孝武復立,寄治襄陽。



 安
 帝世在漢中南鄭。領郡十四,縣四十二,戶八千七百三十二,口四萬八百八十八。


武都太守,漢武帝元鼎六年立。《永初郡國》又有河池、故道縣
 \gezhu{
  並漢舊縣}
 。



 今領縣三,戶一千二百七十四,口六千一百四十。



 下辨令,漢舊縣。



 上祿令,漢舊縣,後省,晉武帝太康三年又立。



 陳倉令,漢舊縣,屬扶風,《晉太康地志》屬秦國。



 略陽太守,《晉太康地志》屬天水。何志故曰漢陽,魏分立曰廣魏,武帝更名。


《永初郡國》有清水縣
 \gezhu{
  別見}
 ,何、徐無。領縣三,戶一千三百五十九,口五千六百五十七。



 略陽令,前漢屬天水,後漢漢陽即天水,《晉太康地志》屬略陽。雍州南天水、益州安固郡又有此縣。



 臨漢令,何志新立。



 上邽令,前漢屬隴西,後漢屬漢陽,《晉太康地志》屬天水。何志流寓割配。



 安固太守,《永初郡國志》有安固郡,又有南安固郡,元嘉十六年度益州。今領縣二,戶一千五百五,口二千四十四。


桓陵令。
 \gezhu{
  別見}



 南桓陵令,《永初郡國》及何志安固郡唯領桓陵一縣,徐志又有此縣。



 西京兆太守,晉末三輔流民出漢中僑立。領縣三,戶六百九十三,口四千五百五十二。


藍田令
 \gezhu{
  別見}
 ,《永初郡國志》無。


杜令。
 \gezhu{
  別見}



 鄠令,二漢屬扶風,《晉太康地志》屬始平。


南太原太守
 \gezhu{
  太原別見}
 ,何志云,故屬并州,流寓割配。《永初郡國》又有清河
 \gezhu{
  別見}
 、高堂縣
 \gezhu{
  別見翼州平原郡,作高唐。}
 。領縣一,戶二百三十三,口一千一百五十六。



 平陶令,漢舊名。



 南安太守,何志云故屬天水,魏分立。《永初郡國》無。領縣
 二,戶六百二十,口三千八十九。



 桓道令,漢舊名,屬天水,後漢屬漢陽,作「獂」。



 中陶令,何志魏立。《晉太康地志》有。



 馮翊太守,三輔流民出漢中,文帝元嘉二年僑立。領縣五,戶一千四百九十,口六千八百五十四。


蓮芍令。
 \gezhu{
  別見}



 頓陽令,漢舊名。



 下辨令,徐志故屬略陽,流寓割配。何無此縣。



 高陸令,二漢、魏無,《晉太康地志》有,屬京兆。何志流寓割配。


萬年令。
 \gezhu{
  別見}



 隴西太守,秦立。文帝元嘉初,關中民三千二百三十六戶歸化,六年立,今領縣六。戶一千五百六十一,口七千五百三十。



 襄武令,漢舊名。



 臨洮令,漢舊名。



 河關令,前漢屬金城,後漢、《晉太康地志》屬隴西。



 狄道令,漢舊名。



 大夏令,漢舊名,《晉太康地志》無。



 首陽令。


始平太守,
 \gezhu{
  別見}
 ,《永初郡國》無。領縣三,戶八百五十九,口五千四百四十一。



 始平令,《太康地志》有,何志晉武帝立;而雍州始平郡之始平縣,何云魏立。



 按此縣末雖各立,本是
 一縣,何為不同?


槐里令。
 \gezhu{
  別見}



 宋熙令,何無,徐新立。



 金城太守,漢昭帝始元六年立。《永初郡國》無,何、徐領縣二,戶三百七十五,口一千。



 金城令,漢舊名。



 榆中令,漢舊名。



 安定太守,漢武帝元鼎三年立。《永初郡國志》無。領縣二,
 戶六百四十,口二千五百一十八。



 朝那令,漢舊名。



 宋興令,何志新立。



 天水太守,漢武元鼎三年立,明帝改曰漢陽。雍州已有此郡。《永初郡國》無。



 領縣二,戶八百九十三,口五千二百二十八。



 阿陽令,漢舊名,《晉太康地志》無。



 新陽令,《晉太康地志》有,何志魏立。


西扶風太守
 \gezhu{
  扶風郡別見}
 ,晉末三輔流民出漢中僑立。領縣二,戶百四十四。


郿令。
 \gezhu{
  別見}


武功令。
 \gezhu{
  別見}



 北扶風太守,孝武孝建二年,以秦、雍流民立。領縣三,時又有廣長郡,又立成階縣,領氐民,尋省。


武功令。
 \gezhu{
  別見}


華陰令。
 \gezhu{
  別見}


始平縣。
 \gezhu{
  別見}



\end{pinyinscope}