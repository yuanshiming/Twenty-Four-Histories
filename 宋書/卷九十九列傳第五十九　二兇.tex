\article{卷九十九列傳第五十九 二兇}

\begin{pinyinscope}

 元凶劭,字休遠,文帝長子也。帝即位後生劭,時上猶在諒闇,故秘之。三年閏正月,方云劭生。自前代以來,未有人君即位後皇后生太子,唯殷帝乙既踐阼,正妃生紂,
 至是又有劭焉。體元居正,上甚喜說。



 年六歲,拜為皇太子,中庶子二率入直永福省。更築宮,制度嚴麗。年十二,出居東宮,納黃門侍郎殷淳女為妃。十三,加元服。好讀史傳,尤愛弓馬。及長,美須眉,大眼方口,長七尺四寸。親覽宮事,延接賓客,意之所欲,上必從之。東宮置兵,與羽林等。十七年,劭拜京陵,大將軍彭城王義康、竟陵王誕、尚書桂陽侯義融並從,司空江夏王義恭自江都來會京口。



 二十七年,上將北伐,劭與蕭思話固諫,不從。索虜
 至瓜步,京邑震駭。劭出鎮石頭,總統水軍。善於撫御。上登石頭城,有憂色,劭曰:「不斬江湛、徐湛之,無以謝天下。」上曰:「北伐自我意,不關二人也。」



 上時務在本業,勸課耕桑,使宮內皆蠶,欲以諷厲天下。有女巫嚴道育,本吳興人,自言通靈,能役使鬼物。夫為劫,坐沒入奚官。劭姊東陽公主應閣婢王鸚鵡白公主云:「道育通靈有異術。」主乃白上,託云善蠶,求召入,見許。道育既入,自言服食,主及劭並信惑之。始興王浚素佞事劭,與劭並多過失,慮
 上知,使道育祈請,欲令過不上聞。道育輒云:「自上天陳請,必不泄露。」劭等敬事,號曰天師。後遂為巫蠱,以玉人為上形像,埋於含章殿前。



 初,東陽主有奴陳天興,鸚鵡養以為子,而與之淫通。鸚鵡、天興及寧州所獻黃門慶國並預巫蠱事。劭以天興補隊主。東陽主薨,鸚鵡應出嫁,劭慮言語難密,與浚謀之。時吳興沈懷遠為浚府佐,見待異常,乃嫁鸚鵡與懷遠為妾,不以啟上,慮後事泄,因臨賀公主微言之。上後知天興領隊,遣閹人奚承祖
 詰讓劭曰:「臨賀公主南第先有一下人欲嫁,又聞此下人養他人奴為兒,而汝用為隊主,抽拔何乃速。



 汝間用主、副,並是奴邪?欲嫁置何處?」劭答曰:「南第昔屬天興,求將驅使,臣答曰:『伍那可得,若能擊賊者,可入隊。』當時蓋戲言耳,都不復憶。後天興道上通辭乞位,追存往為者,不忍食言,呼視見其形容粗健,堪充驅使,脫爾使監禮兼隊副。比用人雖取勞舊,亦參用有氣幹者。謹條牒人囗名上呈。下人欲嫁者,猶未有處。」時鸚鵡已嫁懷遠矣。
 劭懼,馳書告浚,并使報臨賀主:「上若問嫁處,當言未有定所。」濬答書曰:「奉令,伏深惶怖,啟此事多日,今始來問,當是有感發之者,未測源由爾。計臨賀故當不應翻覆言語,自生寒熱也。此姥由來挾兩端,難可孤保,正爾自問臨賀,冀得審實也。其若見問,當作依違答之。天興先署佞人府位,不審監上當無此簿領爾。急宜犍之。殿下已見王未?宜依此具令嚴自躬上啟聞。彼人若為不已,正可促其餘命,或是大慶之漸。」凡劭、濬相與書疏類如
 此,所言皆為名號,謂上為「彼人」,或以為「其人」;以太尉江夏王義恭為「佞人」;東陽主第在西掖門外,故云「南第」,王即鸚鵡姓,躬上啟聞者,令道育上天白天神也。



 鸚鵡既適懷遠,慮與天興私通事泄,請劭殺之。劭密使人害天興。慶國謂宣傳往來,唯有二人,天興既死,慮將見及,乃具以其事白上。上驚惋,即遣收鸚鵡,封籍其家,得劭、濬書數百紙,皆咒詛巫蠱之言,得所埋上形像於宮內。道育叛亡,討捕不得。上大怒,窮治其事,分遣中使入東諸
 郡搜討,遂不獲。上詰責劭、濬,劭、浚惶懼無辭,唯陳謝而已。道育變服為尼,逃匿東宮,濬往京口,又載以自隨,或出止民張旿家。



 江夏王義恭自盱眙還朝,上以巫蠱告之,曰:「常見典籍有此,謂之書傳空言,不意遂所親睹。劭雖所行失道,未必便亡社稷,南面之日,非復我及汝事。汝兒子多,將來遇此不幸爾。」



 先是二十八年,彗星起畢、昴,入太微,掃帝座端門,滅翼、軫。二十九年,熒惑逆行守氐,自十一月霖雨連雪,太陽罕曜。三十年正月,大風飛
 霰且雷。上憂有竊發,輒加劭兵眾,東宮實甲萬人。車駕出行,劭入守,使將白直隊自隨。



 其年二月,浚自京口入朝,當鎮江陵,復載道育還東宮,欲將西上。有告上云:「京口民張旿家有一尼,服食,出入征北內,似是嚴道育。」上初不信,試使掩錄,得其二婢,云:「道育隨征北還都。」上謂劭、濬已當斥遣道育,而猶與往來,惆悵惋駭。乃使京口以船送道育二婢,須至檢覈,廢劭,賜浚死,以語浚母潘淑妃,淑妃具以告濬。浚馳報劭,劭因是異謀,每夜輒饗
 將士,或親自行酒,密與腹心隊主陳叔兒、詹叔兒、齋帥張超之、任建之謀之。



 道育婢將至,其月二十一日夜,詐上詔云:「魯秀謀反,汝可平明守闕,率眾入。」因使超之等集素所畜養兵士二千餘人,皆使被甲,召內外幢隊主副,豫加部勒,云有所討。宿召前中庶子、右軍長史蕭斌,夜呼斌及左衛率袁淑、中舍人殷仲素、左積弩將軍王正見,並入宮,告以大事,自起拜斌等,因流涕,眾並驚愕,語在淑傳。明旦未開鼓,劭以朱服加戎服上,乘畫輪車,
 與蕭斌同載,衛從如常入朝之儀,守門開,從萬春門入。舊制,東宮隊不得入城,劭與門衛云:「受敕,有所收討。」令後隊速來,張超之等數十人馳入雲龍、東中華門及齋閣,拔刀徑上合殿。



 上其夜與尚書僕射徐湛之屏人語,至旦燭猶未滅,直衛兵尚寢。超之手行弒逆,并殺湛之。劭進至合殿中閣,太祖已崩,出坐東堂,蕭斌執刀侍直。呼中書舍人顧嘏,嘏震懼不時出,既至,問曰:「欲共見廢,何不蚤啟?」未及答,即於前斬之。遣人於崇禮闥殺吏部
 尚書江湛。太祖左細杖主卜天與攻劭於東堂,見殺。又使人從東閣入殺潘淑妃,又殺太祖親信左右數十人。急召始興王浚,率眾屯中堂。又召太尉江夏王義恭、尚書令何尚之。



 劭即偽位,為書曰:「徐湛之、江湛弒逆無狀,吾勒兵入殿,已無所及,號惋崩恤,肝心破裂。今罪人斯得,元凶克殄,可大赦天下。改元嘉三十年為太初元年。



 文武並賜位二等,諸科一依丁卯。」初,使蕭斌作詔,斌辭以不文,乃使侍中王僧綽為之。使改元為太初,劭素與
 道育所定。斌曰:「舊踰年改元。」劭以問僧綽,繒綽曰:「晉惠帝即位,便改號。」劭喜而從之。百僚至者裁數十人,劭便遽即位。



 即位畢,稱疾還入永福省,然後遷大行皇帝升太極前殿。是日,以蕭斌為散騎常侍、尚書僕射、領軍將軍;何尚之為司空;前右衛率檀和之戍石頭;侍中營道侯義綦為征虜將軍、晉陵南下邳二郡太守,鎮京城;尚書殷仲景為侍中、中護軍。大行皇帝大斂,劭辭疾不敢出。先給諸王及諸處兵杖,悉收還武庫。殺徐湛之、江湛
 親黨新除始興內史荀赤松、新除尚書左丞臧凝之、山陰令傅僧祐、吳令江徽、前征北行參軍諸葛詡、右衛司馬江文綱。以殷仲素為黃門侍郎,王正見為左軍將軍,張超之及諸同逆聞人文子、徐興祖、詹叔兒、陳叔兒、任建之等,並將校以下龍驤將軍帶郡,各賜錢二十萬。遣人謂魯秀曰:「徐湛之常欲相危,我已為卿除之矣。」使秀與屯騎校尉龐秀之對掌軍隊。以侍中王僧綽為吏部尚書,司徒左長史何偃為侍中。成服日,劭登殿臨靈,號
 慟不自持。博訪公卿,詢求治道,薄賦輕徭,損諸遊費。田苑山澤,有可弛者,假與貧民。



 三月,遣大使分行四方,分浙以東五郡為會州,省揚州立司隸校尉,以殷沖補之。以大將軍江夏王義恭為太保,司徒南譙王義宣為太尉,衛將軍、荊州刺史始興王浚進號驃騎將軍。王僧綽以先預廢立,見誅。長沙王瑾、瑾弟楷、臨川王燁、桂陽侯覬、新諭侯球,並以宿恨下獄死。禮官希旨,謚太祖不敢盡美稱,上謚曰中宗景皇帝。以雍州刺史臧質為丹陽
 尹,進世祖號征南將軍,加散騎常侍,撫軍將軍南平王鑠中軍將軍,會稽太守隨王誕會州刺史。江夏王義恭以太保領大宗師,諮稟之科,依晉扶風王故事。



 世祖及南譙王義宣、隨王誕諸方鎮並舉義兵。劭聞義師大起,悉聚諸王及大臣於城內,移江夏王義恭住尚書下舍,義恭諸子住侍中下省。自永初元年以前,相國府入齋、傳教、給使,免軍戶,屬南彭城薛縣。劭下書,以中流起兵,當親率六師,觀變江介,悉召下番將吏。加三吳太守軍
 號,置佐領兵。四月,立妻殷氏為皇后。



 世祖檄京邑曰:夫運不常隆,代有莫大之釁。爰自上葉,或因多難以成福,或階昏虐以兆亂,咸由君臣義合,理悖恩離。故堅冰之遘,每鐘澆末,未有以道御世,教化明厚,而當梟鏡反噬,難發天屬者也。先帝聖德在位,功格區宇,明照萬國,道洽無垠,風之所被,荒隅變識;仁之所動,木石開心。而賊劭乘藉塚嫡,夙蒙寵樹,正位東朝,禮絕君后,凶慢之情,發於齠昪,猜忍之心,成於幾立。賊浚險躁無行,自幼而
 長,交相倚附,共逞奸回。



 先旨以王室不造,家難亟結,故含蔽容隱,不彰其釁,訓誘啟告,冀能革音。



 何悟狂慝不悛,同惡相濟,肇亂巫蠱,終行弒逆,聖躬離荼毒之痛,社稷有翦墜之哀,四海崩心,人神泣血,生民以來,未聞斯禍。奉諱驚號,肝腦塗地,煩冤腷臆,容身無所。大將軍、諸王幽間窮省,存亡未測。徐僕射、江尚書、袁左率,皆當世標秀,一時忠貞,或正色立朝,或聞逆弗順,並橫分階闥,懸首都市。宗黨夷滅,豈伊一姓,禍毒所流,未知其極。



 昔
 周道告難,齊、晉勤王,漢歷中圮,虛、牟立節,異姓末屬,猶或亡軀,況幕府職同昔人,義兼臣子。所以枕戈嘗膽,茍全視息,志梟元凶,少雪仇恥。今命冠軍將軍領諮議中直兵柳元景、寧朔將軍領中直兵馬文恭等,統勁卒三萬,風馳徑造石頭,分趨白下;輔國將軍領諮議中直兵宗愨等,勒甲楯二萬,征虜將軍領司馬武昌內史沈慶之等,領壯勇五萬,相尋就路;支軍別統,或焚舟破釜,步自姑孰;或迅楫蕪湖,入據雲陽。凡此諸帥,皆英果權奇,
 智略深贍,名震中土,勳暢遐疆。



 幕府親董精悍一十餘萬,授律枕戈,駱驛繼邁。司徒睿哲淵謨,赫然震發,徵甲八州,電起荊郢;冠軍將軍臧質忠烈協舉,雷動漢陰;冠軍將軍朱修之誠節亮款,悉力請奮。荊、雍百萬,稍次近塗,蜀、漢之卒,續已出境。又安東將軍誕、平西將軍遵考、前撫軍將軍蕭思話、征虜將軍魯爽、前寧朔將軍王玄謨,並密信俱到,不契同期,傳檄三吳,馳軍京邑,遠近俱發,揚旍萬里。樓艦騰川,則滄江霧咽;銳甲赴野,則林薄
 摧根。謀臣智士,雄夫毅卒,畜志須時,懷憤待用。先聖靈澤,結在民心,逆順大數,冥發天理,無父之國,天下無之。羽檄既馳,華素響會,以此眾戰,誰能抗禦,以此義動,何往不捷!況逆醜無親,人鬼所背,計其同惡,不盈一旅,崇極群小,是與此周,哲人君子,必加積忌。傾海注螢,頹山壓卵,商、周之勢,曾何足云。



 諸君或奕世貞賢,身囗皇渥,或勳烈肺腑,休否攸同。拘逼凶勢,俯眉寇手,含憤茹戚,不可為心。大軍近次,威聲已接,便宜因變立功,洗雪滓
 累;若事有不獲,能背逆歸順,亦其次也;如有守迷遂往,黨一凶類,刑茲無赦,戮及五宗。賞罰之科,信如日月。原火一燎,異物同灰,幸求多福,無貽後悔。書到宣告,咸使聞知。



 劭自謂素習武事,語朝士曰:「卿等但助我理文書,勿措意戎陳。若有寇難,吾當自出,唯恐賊虜不敢動爾。」司隸校尉殷沖掌綜文符,左衛將軍尹弘配衣軍旅,蕭斌總眾事,中外戒嚴。防守世祖子於侍中下省,南譙王義宣諸子於太倉空屋。劭使濬與世祖書曰:「聞弟忽起
 狂檄,阻兵反噬,縉紳憤歎,義夫激怒。古來陵上內侮,誰不夷滅,弟洞覽墳籍,豈不斯具。今主上天縱英聖,靈武宏發,自登宸極,威澤兼宣,人懷甘死之志,物競舍生之節。弟蒙眷遇,著自少長,東宮之懽,其來如昨,而信惑姦邪,忘茲恩友,此之不義,人鬼同疾。今水步諸軍悉已備辦,上親御六師,太保又乘鉞臨統,吾與烏羊,相尋即道。所以淹霆緩電者,猶冀弟迷而知返爾。故略示懷,言不盡意,主上聖恩,每厚法師,今在殿內住,想弟欲知消息,
 故及。」烏羊者,南平王鑠;法師,世祖世子小名也。



 劭欲殺三鎮士庶家口,江夏王義恭、何尚之說之曰:「凡舉大事者,不顧家口。



 且多是驅逼,今忽誅其餘累,正足堅彼意耳。」劭謂為然,乃下書一無所問。使褚湛之戍石頭,劉思考鎮東府。浚及蕭斌勸劭勒水軍自上決戰,若不爾,則保據梁山。



 江夏王義恭慮義兵倉卒,船舫陋小,不宜水戰。乃進策曰:「賊駿少年未習軍旅,遠來疲弊,宜以逸待之。今遠出梁山,則京都空弱,東軍乘虛,容能為患。若分
 力兩赴,則兵散勢離。不如養銳待期,坐而勸釁。」劭善其議,蕭斌厲色曰:「南中郎二十年少,業能建如此大事,豈復可量。三方同惡,勢據上流,沈慶之甚練軍事,柳元景、宗愨屢嘗立功。形勢如此,實非小敵。唯宜及人情未離,尚可決力一戰。



 端坐臺城,何由得久。主相咸無戰意,此自天也。」劭不納。疑朝廷舊臣悉不為己用,厚接王羅漢、魯秀,悉以兵事委之,多賜珍玩美色,以悅其意。羅漢先為南平王鑠右軍參軍,劭以其有將用,故以心膂委焉。或勸
 劭保石頭城者,劭曰:「昔人所以固石頭,俟諸侯勤王爾。我若守此,誰當見救。唯應力戰決之,不然不剋。」



 日日自出行軍,慰勞將士,親督都水治船艦,焚南岸,驅百姓家悉渡水北。使有司奏立子偉之為皇太子,以褚湛之為後將軍、丹陽尹,置佐史,驃騎將軍始興王浚為侍中、中書監、司徒、錄尚書六條事,中軍將軍南平王鑠為使持節、都督南兗兗青徐冀五州諸軍事、征北將軍、開府儀同三司、南兗州刺史,新除左將軍、丹陽尹建平王宏為
 散騎常侍、鎮軍將軍、江州刺史。



 龐秀之自石頭先眾南奔,人情由是大震。以征虜將軍營道侯義綦即本號為湘州刺史,輔國將軍檀和之為西中郎將、雍州刺史。十九日,義軍至新林,劭登石頭烽火樓望之。二十一日,義軍至新亭。時魯秀屯白石,劭召秀與王羅漢共屯硃雀門。



 蕭斌統步軍,褚湛之統水軍。二十二日,使蕭斌率魯秀、王羅漢等精兵萬人攻新亭壘,劭登朱雀門躬自督率,將士懷劭重賞,皆為之力戰。將克,而秀斂軍遽止,為
 柳元景等所乘,故大敗。劭又率腹心同惡自來攻壘,元景復破之;劭走還硃雀門,蕭斌臂為流矢所中。褚湛之攜二子與檀和之同共歸順。劭駭懼,走還臺城。其夜,魯秀又南奔。時江夏王義恭謀據石頭,會劭已令浚及蕭斌備守。劭並焚京都軍籍,置立郡縣,悉屬司隸為民。以前軍將軍、輔國將軍王羅漢為左衛將軍,輔國如故,左軍王正見為太子左衛率。二十五日,義恭單馬南奔,自東掖門出,於冶渚過淮。



 東掖門隊主吳道興是臧質門
 人,冶渚軍主原稚孫是世祖故史,義恭得免。劭遣騎追討,騎至冶渚,義恭始得渡淮。義恭佐史義故二千餘人,隨從南奔,多為追兵所殺。



 遣浚殺義恭諸子。以輦迎蔣侯神像於宮內,啟顙乞恩,拜為大司馬,封鐘山郡王,食邑萬戶,加節鉞。蘇侯為驃騎將軍。使南平王鑠為祝文,罪狀世祖。



 加濬使持節、都督南徐會二州諸軍事、領太子太傅、南徐州刺史,給班劍二十人;征北將軍、南兗州刺史南平王鑠進號驃騎將軍,與浚並錄尚書事。二十
 七日,臨軒拜息偉之為太子,百官皆戎服,劭獨袞衣。下書大赦天下,唯世祖、劉義恭、義宣、誕不在原例,餘黨一無所問。先遣太保參軍庾道、員外散騎侍郎朱和之,又遣殿中將軍燕欽東拒誕。五月,世祖所遣參軍顧彬之及誕前軍,並至曲阿,與道相遇,與戰,大破之。劭遣人焚燒都水西裝及左尚方,決破柏崗方山埭以絕東軍。又悉以上守家之丁巷居者,緣淮豎舶船為樓,多設大弩。又使司隸治中監琅邪郡事羊希柵斷班瀆、白石諸水
 口。于時男丁既盡,召婦女親役。



 其月三日,魯秀等募勇士五百人攻大航,鉤得一舶。王羅漢副楊恃德命使復航,羅漢昏酣作伎,聞官軍已渡,驚懼放仗歸降。緣渚幢隊,以次奔散,器仗鼓蓋,充塞街衢。是夜,劭閉守六門,於門內鑿塹立柵,以露車為樓,城內沸亂,無復綱紀。



 丹陽尹尹弘、前軍將軍孟宗嗣等下及將吏,並踰城出奔。劭使詹叔兒燒輦及袞冕服。



 蕭斌聞大航不守,惶窘不知所為,宣令所統,皆使解甲,自石頭遣息約詣闕請罪,尋
 戴白幡來降,即於軍門伏誅。四日,太尉江夏王義恭登朱雀門,總群帥,遣魯秀、薛安都、程天祚等直趣宣陽門。劭軍主徐興祖、羅訓、虞丘要兒等率眾來降。劭先遣龍驤將軍陳叔兒東討,事急,召還。是日,始入建陽門,遙見官軍,所領並棄仗走。劭腹心白直同諸逆先屯閶闔門外,並走還入殿。天祚與安都副譚金因而乘之,即得俱入。安都及軍主武念、宋越等相繼進,臧質大軍從廣莫門入,同會太極殿前,即斬太子左衛率王正見。建平、東
 海等七王並號哭俱出。劭穿西垣入武庫井中,隊副高禽執之。浚率左右數十人,與南平王鑠於西明門出,俱共南奔。於越城遇江夏王義恭,濬下馬曰:「南中郎今何所作?」義恭曰:「四海無統,百司固請,上已俯順群心,君臨萬國。」又曰:「虎頭來得無晚乎?」義恭曰:「殊當恨晚。」又曰:「故當不死耶?」義恭曰:「可詣行闕請罪。」又曰:「未審猶能賜一職自效不?」義恭又曰:「此未可量。」勒與俱歸,於道斬首。



 浚字休明,將產之夕,有鵩鳥鳴於屋上。元嘉十三年,年八
 歲,封始興王。十六年,都督湘州諸軍事、後將軍、湘州刺史。仍遷使持節、都督南豫豫司雍并五州諸軍事、南豫州刺史,將軍如故。十七年,為揚州刺史,將軍如故,置佐領兵。十九年,罷府。二十一年,加散騎常侍,進號中軍將軍。



 明年,浚上言:「所統吳興郡,衿帶重山,地多汙澤,泉流歸集,疏決遲壅,時雨未過,已至漂沒。或方春輟耕,或開秋沈稼,田家徒苦,防遏無方。彼邦奧區,地沃民阜,一歲稱稔,則穰被京城;時或水潦,由數郡為災。頃年以來,儉
 多豐寡,雖賑賚周給,傾耗國儲,公私之弊,方在未已。州民姚嶠比通便宜,以為二吳、晉陵、義興四郡,同注太湖,而松江滬瀆壅噎不利,故處處涌溢,浸漬成災。欲從武康珝溪開漕谷湖,直出海口,一百餘里,穿渠浛必無閡滯。自去踐行量度,二十許載。去十一年大水,已詣前刺史臣義康欲陳此計,即遣主簿盛曇泰隨嶠周行,互生疑難,議遂寢息。既事關大利,宜加研盡,登遣議曹從事史虞長孫與吳興太守孔山士同共履行,準望地勢,格
 評高下,其川源由歷,莫不踐校,圖畫形便,詳加算考,如所較量,決謂可立。尋四郡同患,非獨吳興,若此浛獲通,列邦蒙益。不有暫勞,無由永晏。然興創事大,圖始當難。今欲且開小漕,觀試流勢,輒差烏程、武康、東遷三縣近民,即時營作。若宜更增廣,尋更列言。昔鄭國敵將,史起畢忠,一開其說,萬世為利。嶠之所建,雖側芻蕘,如或非妄,庶幾可立。」從之;功竟不立。



 二十三年,給鼓吹一部。二十六年,出為使持節、都督南徐兗二州諸軍事、征北將
 軍、開府儀同三司、南徐兗二州刺史,常侍如故。二十八年,遣浚率眾城瓜步山,解南兗州。三十年,徙都督荊雍益梁寧南北秦七州諸軍事、衛將軍、開府儀同三司、荊州刺史、領護南蠻校尉,持節、常侍如故。



 浚少好文籍,姿質端妍。母潘淑妃有盛寵,時六宮無主,潘專總內政。浚人才既美,母又至愛,太祖甚留心。建平王宏、侍中王僧綽、中書侍郎蔡興宗並以文義往復。初,元皇后性忌,以潘氏見幸,遂以恚恨致崩,故劭深疾潘氏及浚。浚慮將
 來受禍,乃曲意事劭,劭與之遂善。多有過失,屢為上所詰讓,憂懼,乃與劭共為巫蠱。及出鎮京口,聽將揚州文武二千人自隨,優遊外籓,甚為得意。在外經年,又失南兗,於是復願還朝。廬陵王紹以疾患解揚州,時江夏王義恭外鎮,浚謂州任自然歸己,而上以授南譙王義宣,意甚不悅。乃因員外散騎侍郎徐爰求鎮江陵,又求助於尚書僕射徐湛之。而尚書令何尚之等咸謂浚太子次弟,不宜遠出。上以上流之重,宜有至親,故以授浚。時
 浚入朝,遣還京,為行留處分。至京數日而巫蠱事發,時二十九年七月也。上惋歎彌日,謂潘淑妃曰:「太子圖富貴,更是一理。虎頭復如北,非復思慮所及。汝母子豈可一日無我耶!」浚小名虎頭。使左右硃法瑜密責讓浚,辭甚哀切,并賜書曰:「鸚鵡事想汝已聞,汝亦何至迷惑乃爾。且沈懷遠何人,其詎能為汝隱此耶?故使法瑜口宣,投筆惋慨。」濬慚懼,不知所答。濬還京,本暫去,上怒,不聽歸。其年十二月,中書侍郎蔡興宗問建平王宏曰:「歲無
 復幾,征北何當至?」宏歎息良久曰:「年內何必還。」在京以沈懷遠為長流參軍,每夕輒開便門為微行。上聞,殺其嬖人楊承先。明年正月,荊州事方行,二月,浚還朝。十四日,臨軒受拜。其日,藏嚴道育事發,明旦浚入謝,上容色非常。其夕,即加詰問,浚唯謝罪而已。潘淑妃抱持濬,泣涕謂曰:「汝始咒詛事發,猶冀刻己思愆,何意忽藏嚴道育耶?上責汝深,至我叩頭乞恩,意永不釋。今日用活何為,可送藥來,當先自取盡,不忍見汝禍敗。」浚奮衣而去,
 曰:「天下事尋自當判,願小寬憂煎,必不上累。」



 劭入弒之旦,浚在西州,府舍人硃法瑜奔告浚曰:「臺內叫喚,宮門皆閉,道上傳太子反,未測禍變所至。」浚陽驚曰:「今當奈何?」法瑜勸入據石頭。浚未得劭信,不知事之濟不,騷擾未知所為。將軍王慶曰:「今宮內有變,未知主上安危,預在臣子。當投袂赴難。憑城自守,非臣節也。」浚不聽,乃從南門出,徑向石頭,文武從者千餘人。時南平王鑠守石頭,兵士亦千餘人。俄而劭遣張超之馳馬召浚,浚屏人
 問狀,即戎服乘馬而去。朱法瑜固止浚,浚不從。出至中門,王慶又諫曰:「太子反逆,天下怨憤。明公但當堅閉城門,坐食積粟,不過三日,凶黨自離。公情事如此,今豈宜去。」浚曰:「皇太子令,敢有復言者斬!」既入,見劭,勸殺荀赤松等。劭謂濬曰:「潘淑妃遂為亂兵所害。」濬曰:「此是下情由來所願。」



 其悖逆乃如此。



 及劭將敗,勸劭入海,輦珍寶繒帛下船,與劭書曰:「船故未至,今晚期當於此下物令畢,願速敕謝賜出船艦。尼已入臺,願與之明日決也。臣
 猶謂車駕應出此,不爾無以鎮物情。」人情離散,故行計不果。浚書所云尼,即嚴道育也。及劭入井,高禽於井中牽出之。劭問禽曰:「天子何在?」禽曰:「至尊近在新亭。」將劭至殿前,臧質見之慟哭,劭曰:「天地所不覆載,丈人何為見哭。」質因辨其逆狀,答曰:「先朝當見枉廢,不能作獄中囚,問計於蕭斌,斌見勸如此。」又語質曰:「可得為啟,乞遠徙不?」質答曰:「主上近在航南,自當有處分。」縛劭於馬上,防送軍門。既至牙下,據鞍顧望,太尉江夏王義恭與諸
 王皆共臨視之。義恭詰劭曰:「我背逆歸順,有何大罪,頓殺我家十二兒?」劭答曰:「殺諸弟,此事負阿父。」



 江湛妻庾氏乘車罵之,龐秀之亦加誚讓,劭厲聲曰:「汝輩復何煩爾!」先殺其四子,謂南平王鑠曰:「此何有哉。」乃斬劭于牙下。臨刑歎曰:「不圖宗室一至於此。」



 劭、浚及劭四子偉之、迪之、彬之、其一未有名;濬三子長文、長仁、長道,並梟首大航,暴尸於市。劭妻殷氏賜死於廷尉,臨死,謂獄丞江恪曰:「汝家骨肉相殘害,何以枉殺天下無罪人。」恪曰:「受
 拜皇后,非罪而何?」殷氏曰:「此權時爾,當以鸚鵡為后也。」浚妻褚氏,丹陽尹湛之女,湛之南奔之始,即見離絕,故免於誅。其餘子女妾媵,並於獄賜死。投劭、濬尸首於江,其餘同逆,及王羅漢等,皆伏誅。張超之聞兵入,遂走至合殿故基,正於御床之所,為亂兵所殺。割腸刳心,臠剖其肉,諸將生啖之,焚其頭骨。當時不見傳國璽,問劭,云:「在嚴道育處。」就取得之。道育、鸚鵡並都街鞭殺,於石頭四望山下焚其尸,揚灰於江。



 毀劭東宮所住齋,汙瀦其
 處。



 封高禽新陽縣男,食邑三百戶。追贈潘淑妃長寧園夫人,置守塚。偽司隸校尉殷沖,丹陽尹尹弘,並賜死。沖為劭草立符文,又妃叔父也。弘二月二十一日平旦入直,至西掖門,聞宮中有變,率城內禦兵至閣道下。及聞劭入,惶怖通啟,求受處分,又為劭簡配兵士,盡其心力。弘,天水冀人,司州刺史沖弟也。為太祖所委任。元嘉中,歷太子左右衛率、左右衛將軍,囗人官爵高下,皆以委之。



 史臣曰:甚矣哉,宋氏之家難也。自赫胥以降,立號皇王,統天南面,未聞斯禍。唯荊、莒二國,棄夏即戎,武靈胡服,亦背華典,戕賊之釁,事起肌膚,而因心之重,獨止此代。難興天屬,穢流床笫,愛敬之道,頓滅一時,生民得無左衽,亦為幸矣!



\end{pinyinscope}