\article{卷七十九列傳第三十九 文五王}

\begin{pinyinscope}

 竟陵王誕廬江王禕武昌王渾海陵王休茂桂陽王休範竟陵王誕,字休文,文帝第六子也。元嘉二十年,年十一,
 封廣陵王,食邑二千戶。二十一年,監南兗州諸軍事、北中郎將、南兗州刺史,出鎮廣陵。尋以本號徙南徐州刺史。二十六年,出為都督雍、梁、南北秦四州、荊州之竟陵、隨二郡諸軍事、後將軍、雍州刺史。



 以廣陵彫弊,改封隨郡王。上欲大舉北討,以襄陽外接關、河,欲廣其資力,乃罷江州軍府,文武悉配雍州,湘州入臺稅租雜物,悉給襄陽。及大舉北伐,命諸蕃並出師,莫不奔敗;唯誕中兵參軍柳元景先克弘農、關、陜三城,多獲首級,關、洛震動,
 事在《元景傳》。會諸方並敗退,故元景引還。徵誕還京師,遷都督廣交二州諸軍事、安南將軍、廣州刺史,當鎮始興,未行;改授都督會稽、東陽、新安、臨海、永嘉五郡諸軍事、安東將軍、會稽太守,給鼓吹一部。



 元凶弒立,以揚州浙江西屬司隸校尉,浙江東五郡立會州,以誕為刺史。世祖入討,遣沈慶之兄子僧榮間報誕,又遣寧朔將軍顧彬之自魯顯東入,受誕節度。誕遣參軍劉季之與彬之并勢,自頓西陵,以為後繼。劭遣將華欽、庾導東討,與彬
 之弟相逢於曲阿之奔牛塘,路甚狹,左右皆悉入菰封,彬之軍人多齎籃屐,於菰葑中夾射之,欽等大敗。事平,徵誕為持節、都督荊、湘、雍、益、寧、梁、南北秦八州諸軍事、衛將軍、開府儀同三司、荊州刺史。誕以位號正與濬同,惡之,請求回改。乃進號驃騎將軍,加班劍二十人,餘如故。南譙王義宣不肯就徵,以誕為侍中、驃騎大將軍、揚州刺史,開府如故。改封竟陵王,食邑五千戶。顧彬之以奔牛之功,封陽新縣侯,食邑千戶,季之零陽縣侯,食邑
 五百戶。



 明年,義宣舉兵反,有荊、江、兗、豫四州之力,勢震天下。上即位日淺,朝野大懼;上欲奉乘輿法物,以迎義宣,誕固執不可,然後處分。加誕節,仗士五十人,出入六門。上流平定,誕之力也。初討元凶,與上同舉兵,有奔牛之捷,至是又有殊勳。上性多猜,頗相疑憚。而誕造立第舍,窮極工巧,園池之美,冠於一時。



 多聚才力之士,實之第內,精甲利器,莫非上品,上意愈不平。孝建二年,乃出為使持節、都督南徐、兗二州諸軍事、太子太傅、南徐州
 刺史,侍中如故。上以京口去都密邇,猶疑之。大明元年秋,又出為都督南兗、南徐、兗、青、冀、幽六州諸軍事、南兗州刺史,餘如故。誕既見猜,亦潛為之備,至廣陵,因索虜寇邊,修治城隍,聚糧治仗。嫌隙既著,道路常云誕反。



 三年,建康民陳文紹上書曰:「私門有幸,亡大姑元嘉中蒙入臺六宮,薄命早亡,先朝賜贈美人,又聽大姑二女出入問訊。父饒,司空誕取為府史,恒使入山圖畫道路,勤劇備至,不敢有辭,不復聽歸,消息斷絕。姑二女去年冒
 啟歸訴,蒙陛下聖恩,賜敕解饒吏名。誕見符至,大怒,喚饒入交問:『汝欲死邪?訴臺求解。』饒即答:『官比不聽通家信,消息斷絕。若是姊為啟聞,所不知。』誕因問饒:『汝那得入臺?』饒被問,依實啟答。既出,誕主衣莊慶、畫師王強語饒:『汝今年敗,汝姊誤汝。官云小人輩敢持臺家逼我。』饒因叛走歸,誕即遣王強將數人逐,突入家內縛錄,將還廣陵。至京口客舍,乃陊死井中,託云『饒懼罪自殺』。抱痛懷冤,冒死歸訴。」吳郡民劉成又詣闕上書,告誕謀反,稱:「
 息道龍昔伏事誕,親見姦狀。又見誕在石頭城內,修乘輿法物,習倡警蹕。道龍私獨憂懼,向伴侶言之,語頗漏泄,誕使大吏令監內執道龍,道龍逸走,誕怒鞭殺監,又捕殺道龍。」



 又豫章民陳談之上書訴枉,稱:「弟詠之昔蒙誕采錄,隨從歷鎮;大駕南下,為誕奉送箋書,經涉危險,時得上聞。聖明登阼,恩澤周普,回改小人,使命微勤,賜署臺位。詠之恒見誕與左右小人莊慶、傅元祀潛圖姦逆,言詞醜悖,每云:『天下方是我家有,汝等不憂不富貴。』
 又常疏陛下年紀姓諱,往巫鄭師憐家祝詛。詠之既聞此語,又不見其事,恐一旦事發,橫罹其罪,密以告建康右尉黃宣達,并有啟聞,希以自免。元祀弟知詠之與宣達來往,自嫌言語漏泄,即具以告誕。誕大怒,令左右飲詠之酒,逼使大醉,因言詠之乘酒罵詈,遂被害。自顧冤枉,事有可哀。」



 其年四月,上乃使有司奏曰:臣聞神極尊明,大儀所以貞觀;皇天峻邈,玄化所以幽宣。故能經緯氓俗,大庇黔首。庶道被八紘,不遺疏賤之賞;威格天區,
 豈漏親貴之罰。此不刊之鴻則,古今之恆訓。



 謹按元嘉之末,天綱崩褫,人神哀憤,含生喪氣。司空竟陵王誕義兼臣子,任居籓維。進不能泣血提戈,忘身徇節;退不能閉關拒險,焚符斬使。遂至拜受偽爵,欣承榮寵,沈淪姦逆,肆於昏放。以妻故司空臣湛之女,誅亡餘類,單舟遄遣,披猖千里,事哀行路,賊忍無親,莫此為甚。



 故山陰傅僧祐,誠亮國朝,義均休戚。重門峻衛,不能拒折簡之使;巖險千里,不能庇匹夫之身。乃更助虐憑凶,抽兵勒刃,
 遂使頓仆牢井,死不旋踵,妻子播流,庭筵莫立,見之者流涕,聞之者含歎。及神鋒首路,欃槍東指,風卷四嶽,電埽三江。誕猶持疑兩端,陰規進退。陛下頻遣書檄,告譬殷勤,方改姦圖,末乃奉順。



 分遣弱旅,永塞符文,宴安所蒞,身不越境,悖禮忘情,不顧物議,彎弧躍馬,務是畋游,致奔牛有崩碎之陳,新亭無獨克之術。假威義銳,乞命皇旅,竟有何勞,而論功伐。既妖祲廓清,大明升曜,幽顯宅心,遠邇雲集。誕忽星行之悲,違開泰之慶,遲回顧望,
 淹踰旬朔。逆黨陳叔兒等,泉寶鉅億,資貨不貲,誕收籍所得,不歸天府,辭稱天軍,實入私室。又太官東傳,舊有獻御,喪亂既平,猶加斷遏,珍羞庶品,回充私膳。於號諱之辰,遽甘滋之品,當惟新之始,絕苞苴之貢,忠孝兩忘,敬愛俱盡。乃徵引巫史,潛考圖緯,自謂體應符相,富貴可期,悖意醜言,不可勝載。遂復遙諷朝廷,占求官爵,侮蔑宗室,詆毀公卿,不義不暱,人道將盡。



 荷任神州,方懷姦慝,每窺向宸御,妄生規幸;多樹淫祀,顯肆祅詛,遂在
 石頭,潛修法物;傳警稱蹕,擬則天行,皆已駭暴觀聽,彰布朝野。



 昔內難甫寧,珍瑋散佚,有御刀利刃,擅價諸夏,天府禁器,歷代所珍。誕密加購賞,頓藏私室。賊義宣初平,餘黨逃命,誕含縱罔忌,私竊招納,名工細巧,悉匿私第。又引義宣故將裘興為己腹心,事既彰露,猶執欺罔,公文面啟,矯稱舊隸。加以營乾制館,僭擬天居,引石征材,專擅興發,驅迫士族,役同輿皁,殫木土之姿,窮吞并之勢。故會稽宣長公主受遇二祖,禮級尊崇,臣湛之亡
 身徇國,追榮典軍。誕以廣拓宅宇,地妨藝植,輒逼遺孤,頓相驅徙。遂令神主宵遷,改卜委巷,宗戚含傷,行路掩涕。又緣溪兩道,積代通衢,誕拓宇開垣,擅斷其一。致使徑塗擁隔,川陸阻礙,神怒民怨,毒遍幽顯。



 故丞相臨川烈武王臣道規,名德茂親,勳光常策,異禮殊榮,受自先旨者。嗣王臣義慶受任西夏,靈寢暫移,先帝親枉鑾輿,拜辭路左,恩冠終古,事絕常班。



 誕又以廟居宅前,固請毀換,詔旨不許,怨懟彌極。



 有靦面目,豺狼為性,規牧江
 都,希廣兵力,天德尚弘,甫申所請,仍謂應住東府,宜為中台,貪冒無厭,人莫與比。雖聖慈全救,每垂容納,而虐戾不悛,姦詖彌甚。受命還鎮,猜怨愈深,忠規正諫,必加鴆毒,諂瀆膚躁,是與比周。又矯稱符敕,設榜開募,事發辭寢,委罪自下。及錄事徐靈壽以常署受坐,將就囚執,舀韓近恭,中護軍遣吏夏嗣伯密相屬請,求寬桎梏。且王僧達臨刑之啟事,高闍即戮之辭,皆稱潛驛往來,遙相要契,醜聲穢問,宣著遐邇,含識能言,孰不憤歎。



 又獲
 吳郡民劉成、豫章民陳談之、建康民陳文紹等並如訴狀,則奸情猜志,歲月增積。



 昔周德初升,公旦有流言之釁,魯道方泰,季子斷逵泉之誅。近則淮厲覆車於前,義康襲軌於後,變發柴奇,禍成范、謝,亦皆以義奪親,情為憲屈。況乃上悖天經,下誣政道,結釁於無妄之辰,希幸於文明之日,皇穹所不覆,厚土所不容。



 夫無禮之誡,臣子所宜服膺;干紀之刑,有國所應慎守。



 臣等參議,宜下有司,絕誕屬籍,削爵土,收付延尉法獄治罪。諸所連坐,
 別下考論。伏願遠尋宗周之重,近監興亡之由,割恩棄私,俯順群議,則卜世靈根,於茲克固,鴻勳盛烈,永永無窮。陛下如復隱忍,未垂三思,則覆皇基於七百,擠生民於塗炭。此臣等所以夙夜危懼,不敢避鈇鉞之誅者也。



 上不許,有司又固請,乃貶爵為侯,遣令之國。上將誅誕,以義興太守垣閬為兗州刺史,配以羽林禁兵,遣給事中戴明寶隨閬襲誕,使閬以之鎮為名。閬至廣陵,誕未悟也。明寶夜報誕典簽蔣成,使明晨開門為內應。成以
 告府舍人許宗之,宗之奔入告誕。誕驚起,呼左右及素所畜養數百人,執蔣成,勒兵自衛。明旦將曉,明寶與閬率精兵數百人卒至,天明而門不開,誕已列兵登陴,自在門上斬蔣成,焚兵籍,赦作部徒繫囚,開門遣腹心率壯士擊明寶等,破之。閬即遇害,明寶奔逃,自海陵界得還。



 上乃遣車騎大將軍沈慶之率大眾討誕。誕焚燒郭邑,驅居民百姓,悉使入城,分遣書檄,要結近遠。時山陽內史梁曠家在廣陵,誕執其妻子,遣使要曠,曠斬使拒之。
 誕怒,滅其家。誕奉表投之城外,曰:「往年元凶禍逆,陛下入討,臣背凶赴順,可謂常節。及丞相構難,臧、魯協從,朝野恍惚,咸懷憂懼,陛下欲百官羽儀,星馳推奉,臣前後固執。方賜允俞,社稷獲全,是誰之力?陛下接遇殷勤,累加榮寵,驃騎、揚州,旬月移授,恩秩頻加,復賜徐、兗,仰屈皇儲,遠相餞送。



 臣一遇之感,感此何忘,庶希偕老,永相娛慰。豈謂陛下信用讒言,遂令無名小人來相掩襲,不任枉酷即加誅剪。雀鼠貪生,仰違詔敕。今親勒部曲,鎮
 扞徐、兗。



 先經何福,同生皇家;今有何愆,便成胡、越?陵鋒奮戈,萬沒豈顧,盪定以期,冀在旦夕。右軍、宣蘭,爰及武昌,皆以無罪,並遇枉酷,臣有何過,復致於此。



 陛下宮帷之醜,豈可三糸咸。臨紙悲塞,不知所言。」世祖忿誕,左右復心同籍期親並誅之,死者以千數。或有家人已死,方自城內叛出者。



 車駕出頓宣武堂,內外纂嚴。慶之進廣陵,誕幢主韓道元來降。豫州刺史宗愨、徐州刺史劉道隆率眾來會。誕中兵參軍柳光宗、參軍何康之、劉元邁、幢
 主索智朗謀開城北門歸順,未期而康之所鎮隊主石貝子先眾出奔,康之懼事泄,夜與智朗斬關而出。誕禽光宗殺之。光宗,柳元景從弟也。康之母在城內,亦為誕所殺。



 誕見眾軍大集,欲棄城北走,留中兵參軍申靈賜居守,自將騎步數百人,親信並隨,聲云出戰,邪趨海陵道。誕將周豐生馳告慶之,慶之遣龍驤將軍武念追躡。



 誕行十餘里,眾並不欲去,請誕還城。誕曰:「我還,卿能為我盡力不?」眾皆曰:「願盡力。」左右楊承伯牽誕馬曰:「死生
 且還保城,欲持此安之?速還尚得入,不然,敗矣。」慶之所遣將戴寶之單騎前至,刺誕殆獲,誕懼,乃馳還。武念去誕遠,未及至,故誕得向城。既至,曰:「城上白鬚,非沈公邪?」左右曰:「申中兵。」誕乃入。以靈賜為驃騎府錄事參軍,王璵之為中軍長史,世子景粹為中軍將軍,州別駕范義為中軍長史,其餘府州文武,皆加秩。



 先是,右衛將軍垣護之、左軍將軍崔道固、屯騎校尉龐番虯、太子旅賁中郎將殷孝祖破索虜還,至廣陵,上並使受慶之節度。司
 州刺史劉季之,誕故佐也,驍果有膂力,梁山之役,又有戰功,增邑五百戶。在州貪殘,司馬翟弘業諫爭甚苦,季之積忿,置毒藥食中殺之。少年時,宗愨共蒱戲,曾手侮加愨,愨深銜恨。至是愨為豫州刺史,都督司州,季之慮愨為禍,乃委官間道欲歸朝廷。會誕反,季之至盱眙,盱眙太守鄭瑗以季之素為誕所遇,疑其同逆,因邀道殺之,送首詣道隆。時誕亦遣間信要季之,及季之首至,沈慶之送以示誕。季之缺齒,垣護之亦缺,誕謂眾曰:「此垣
 護之頭,非劉季之也。」



 太宗初即位,鄭瑗為山陽王休祐驃騎中兵參軍。豫州刺史殷琰與晉安王子勛同逆,休祐遣瑗及左右邢龍符說琰,琰不受。鄭氏,壽陽強族。瑗即使琰鎮軍。子勛責琰舉兵遲晚,琰欲自解釋,乃殺龍符送首,瑗固爭不能得。及壽陽城降,瑗隨輩同出,龍符兄僧愍時在城外,謂瑗構殺龍符,輒殺瑗。即為劉勔所錄,後見原。僧愍尋擊虜於淮西戰死。此四人者,並由橫殺,旋受身禍,論者以為有天道焉。



 誕幢主公孫安期率
 兵隊出降。誕初閉城拒使,記室參軍賀弼固諫再三,誕怒,抽刃向之,乃止。或勸弼出降,弼曰:「公舉兵向朝廷,此事既不可從;荷公厚恩,又義無違背,唯當死明心耳。」乃服藥自殺。弼字仲輔,會稽山陰人也。有文才。



 贈車騎將軍、山陽、海陵二郡太守,長史如故。幢主王璵之賞募數百人,從東門出攻龍驤將軍程天祚營,斷其弩弦,天祚擊破之,即走還城。誕又加申靈賜南徐州刺史。軍主馬元子踰城歸順,追及殺之,乃於城內建列立壇誓,誕將
 歃血,其所署輔國將軍孟玉秀曰:「陛下親歃。」群臣皆稱萬歲。



 初,誕使黃門呂曇濟與左右素所信者,將世子景粹藏於民間,謂曰:「事若濟,斯命全脫,如其不免,可深埋之。」分以金寶,齊送出門,並各散走。唯曇濟不去,攜負景粹,十餘日,乃為沈慶之所捕得,斬之。



 誕所署平南將軍虞季充又出降書。上使慶之於桑里置烽火三所。誕又遣千餘人自北門攻強弩將軍茍思達營,龍驤將軍宗越擊破之。開東門掩攻劉道隆營,復為殷孝祖及員外
 散騎侍郎沈攸之所破。誕又加申靈賜左長史,王璵之右長史,范義左司馬、左將軍,孟玉秀右司馬、右將軍。范義母妻子並在城內,有勸義出降,義曰:「我人吏也,且豈能作何康活邪!」義字明休,濟陽考城人也。早有世譽。



 五月十九日夜,有流星大如斗桿,尾長十餘丈,從西北來墜城內,是謂天狗。



 占曰:「天狗所墜,下有伏尸流血。」誕又遣二百人出東門攻劉道產營,別遣疑兵二百人出北門。沈攸之於東門奮短兵接戰,大破之。門者又為茍思
 達所破。誕又遣數百人出東門攻寧朔司馬劉勔營,攸之又破之。廣陵城舊不開南門,云開南門者,不利其主,至誕乃開焉。彭城邵領宗在城內,陰結死士,欲襲誕。先欲布誠於慶之,乃說誕求為間諜,見許。領宗既出,致誠畢,復還城內,事泄,誕鞭二百,考問不服,遂支解之。



 上遣送章二紐,其一曰竟陵縣開國侯,食邑一千戶,募賞禽誕;其二曰建興縣開國男,三百戶,募賞先登。若克外城,舉一烽;克內城,舉兩烽;禽誕,舉三烽。



 上又遣屯騎校尉
 譚金、前虎賁中郎將鄭景玄率羽林兵隸慶之。誕復遣三百人自南門攻劉勔土山,為勔所破。



 慶之填塹治攻道,值夏雨,不得攻城。上每璽書催督之,前後相繼。及晴,再怒,使太史擇發日,將自濟江。太宰江夏王義恭上表諫曰:「誕素無才略,畜養又寡,自拒王命,士庶離散。城內乏糧,器械不足,徒賴免兵倉頭三四百人,造次相附,恩怨夙結。臣始短慮,謂一旬可殄,而假息流遷,七十餘日。上將受律,群蕃岳峙,銳卒精旅,動以萬計,大威所震,未
 有成功。臣雖凡怯,猶懷憤踴。陛下入翦封豕,出討長蛇,兵不血刃,再興七百。而蕞爾小醜,遂延晷漏,致皇赫斯怒,將動乘輿。此實臣下素食駑鈍之責,行留百司,莫不仰慚俯愧。今盛暑被甲,日費千金,天威一麾,孰不幸甚。臣伏尋晉文王征淮南,淹師出二百日,方能制寇。今誕餱糧垂竭,背逆者多;慶之等轉悟遲重之非,漸見乘機之利。且成旨頻降,必應旦夕夷殄。愚又以廣陵塗近,人信易達,雖為江水,約示不難。且睹理者寡,暗塞者眾,忽
 見雲旗移次,京都既當祗悚,四方之志,必有未達。臣愚伏重思計,今寧不當計小醜,省生命,以安遐邇之情。又以長江險闊,風波難期,王者尚不乘危,況乃泛不測之水。昔魏文濟江,遂有遺州之名,今雖先天不違,動干休慶,龍舟所幸,理必利涉,然居安慮危,不可不懼。私誠款款,冒啟赤心,追用悚汗,不自宣盡。」



 七月二日,慶之率眾軍進攻,剋其外城,乘勝而進,又克小城。誕聞軍入,與申靈賜走趨後園。隊主沈胤之、義征客周滿、胡思祖馳至,
 誕執玉鈽刀與左右數人散走,胤之等追及誕於橋上,誕舉刀自衛,胤之傷誕面,因墜水,引出殺之,傳首京邑。時年二十七,因葬廣陵,貶姓留氏。同黨悉誅,殺城內男為京觀,死者數千,女口為軍賞。誕母殷、妻徐,並自殺。追贈殷長寧園淑妃。嘉梁曠誠節,擢為後將軍。封周滿山陽縣侯,食邑四百五十戶,胤之萊陽子,食邑三百五十戶。胡思祖高平縣男食邑二百戶。臨川內羊璇之以先協附誕,伏誅。



 誕為南徐州刺史,在京,夜大風飛落屋
 瓦,城門鹿床倒覆,誕心惡之。及遷鎮廣陵,入城,沖風暴起揚塵,晝晦。又中夜閑坐,有赤光照室,見者莫不怪愕。左右侍直,眠中夢人告之曰:「官須髮為槊毦。」既覺,已失髻矣,如此者數十人,誕甚怪懼。大明二年,發民築治廣陵城,誕循行,有人乾輿揚聲大罵曰:「大兵尋至,何以辛苦百姓!」誕執之,問其本末,答曰:「姓夷名孫,家在海陵。天公去年與道佛共議,欲除此間民人,道佛苦諫得止。大禍將至,何不立六慎門。」誕問:「六慎門云何?」答曰:「古時有
 言,禍不入六慎門。」誕以其言狂悖,殺之。又五音士忽狂易見鬼,驚怖啼哭曰:「外軍圍城,城上張白布帆。」誕執錄二十餘日,乃赦之。城陷之日,雲霧晦暝,白虹臨北門,亙屬城內。



 八年,前廢帝即位,義陽王昶為征北將軍、徐州刺史,道經廣陵,上表曰:「竊聞淮南中霧,眷求遺緒;楚英流殛,愛存丘墓。並難結兩臣,義開二主,法雖事斷,禮或情申。伏見故賊劉誕,稱戈犯節,自貽逆命,膏斧嬰戮,在憲已彰。但尋屬忝皇枝,位叨列辟,一以罪終,魂骸莫赦。
 生均宗籍,死同匹豎,旅窆委雜,封樹不修。今歲月愈邁,愆流釁往,踐境興懷,感事傷目。陛下繼明升運,咸與惟新,大德方臨,哀矜未及。夫欒布哭市,義犯雷霆;田叔鉗赭,志於夷戮。況在天倫,何獨無感。伏願稽若前準,降申丹志,乞薄改褊祔,微表窀穸。則朽骨知榮,窮泉識荷。臨紙哽慟,辭不自宣。」詔曰:「征北表如此。省以慨然。誕及妻女,並可以庶人禮葬,并置守衛。」太宗泰始四年,又更改葬,祭以少牢。



 廬江王禕,字休秀,文帝第八子也。元嘉二十二年,年十歲,封東海王,食邑二千戶。二十六年,以為侍中、後軍將軍,領石頭戍事。遷冠軍將軍、南彭城、下邳二郡太守、散騎常侍,領戍如故。出為會稽太守,將軍如故。二十九年,遷使持節、都督廣交二州荊州之始興臨安二郡諸軍事、車騎將軍、平越中郎將、廣州刺史。



 元凶弒立,進號安南將軍,未之鎮。世祖踐阼,復為會稽太守,加撫軍將軍。



 明年,徵為秘書監,加散騎常侍。尋出為撫軍將軍、江州
 刺史,進號平南將軍,置吏。大明二年,徵為散騎常侍、中書令,領驍騎將軍,給鼓吹一部,常侍如故。又出為南豫州刺史,常侍、將軍如故。以本號開府儀同三司,領國子祭酒,常侍如故。



 五年,詔曰:「昔韓、衛異姓,宗周之明憲;三封殊級,往晉之令典。唯皇家創典,盡弘斯義。朕應天命,光宅四海,思所以憲章前式,崇建懿親,永垂畫一,著於甲令。諸弟國封,並可增益千戶。」七年,進司空,常侍、祭酒如故。前廢帝即位,加中書監。太宗踐阼,進太尉,加侍中、
 中書監,給班劍二十人。改封廬江王。



 太祖諸子,禕尤凡劣,諸兄弟蚩鄙之。南平王鑠蚤薨,鑠子敬淵婚,禕往視之,白世祖借伎。世祖答曰:「婚禮不舉樂,且敬淵等孤苦,倍非宜也。」至是太宗與建安王休仁詔曰:「人既不比數西方公,汝便為諸王之長。」時禕住西州,故謂之西方公也。泰始五年,河東柳欣慰謀反,欲立禕,禕與相酬和。欣慰要結征北諮議參軍杜幼文、左軍參軍宋祖珍、前郡令王隆伯等。禕使左右徐虎兒以金合一枚餉幼文,銅
 缽二枚餉祖珍、隆伯。幼文具奏其事。上乃下詔曰:昔周室既盛,二叔流言,漢祚方隆,七蕃迷叛,斯實事彰往代,難興自古。雖聖賢御極,宇內紓患。太尉廬江王藉慶皇枝,蚤升寵樹,幼無立德,長缺修聲,淡薄親情,厚結行路,狎暱群細,疏澀人士。



 自朕撥亂定宇,受命應天,實尚敦睦,克敷友于,故崇殊爵,超居上台。而公常懷不平,表於事跡。公若德深望重,宜膺大統,朕初平暴亂,豈敢當璧,自然推符奉璽,天祚有歸。且朕雖居尊極,不敢自恃,宗
 室之事,無不諮公。不虞志欲難滿,妄生窺怨,積慝在衿,遂謀社稷。



 曩者四方遘禍,兵斥畿甸,搢紳憂惶,親賢同憤。唯公獨幸厥災,深抃時難,晝則從禽遊肆,夜則縱酒絃歌,側耳視陰,企賊休問。司徒休仁等並各令弟,事兼家國,推鋒履險,各伐一方,蒙霜踐棘,辛勤已甚。況身被矢石,否泰難虞,悠悠之人,尚有信分。公未曾有一函之使,遺半紙之書,志棄五弟,以餌仇賊。自謂身非勳烈,義不參謀,必期凶逆道申,以圖輔相。及皇威既震,群凶肅
 蕩,九有同慶,萬國含欣。而公容氣更沮,下帷晦跡,每覘天察宿,懷協左道,咒詛禱請,謹事邪巫,常被髮跣足,稽首北極,遂圖畫朕躬,勒以名字,或加以矢刃,或烹之鼎鑊。



 公在江州,得一漢女,云知吉凶,能行厭咒,大設供養,朝夕拜伏,衣裝嚴整,敬事如神;令其祝詛孝武,並及崇憲,祈皇室危弱,統天稱己;巫稱神旨,必得如願,後事發覺,委罪所生,徼幸τ,僅得自免。近又有道士張寶,為公見信,事既彰露,肆之於法。公不知慚懼,猶加營理,遣
 左右二人,主掌殯含。顯行邪志,罔顧吏司。又挾閹豎陳道明交關不逞,傳驛音意,投金散寶,以為信誓。又使府史徐虎兒招引邊將,要結禁旅,規害台輔,圖犯宮掖。



 公受性不仁,才非治用,昔忝江州,無稱被徵,前蒞會稽,以罪左黜。公稽古寡聞,嚴而無理,言不暢寒暑,惠不及帷房,朝野所輕,搢紳同侮,豈堪輔相之地,寧任蒞民之職,非唯一朝,有自來矣。



 大明之世,迄於永光,公常留中,未嘗外撫,何以在今,方起嫌怨。公少即長人,情無哀戚,侍
 拜長寧,從祀宗廟,顏無戚狀,淚不垂臉,兄弟長幼,靡有愛心。



 昔因孝武御筵置酒,心誠不著,于時義陽念遇本薄,遭公此譖,益被猜嫌。朕當時狼狽,不暇自理,賴崇憲太后譬解百端,少蒙申亮,得免殃責。景和狂主,醜毒橫流,初誅宰輔,豺志方扇。於建章宮召朕兄弟,逼酒使醉,公因酒勢,遂肆苦言,云朕及休仁,與太宰親數,往必清閑,贈貺豐厚。朕當時惶駭,五內崩墜,于其語次,劣得小止。往又經在尋陽長公主第,兄弟共集,忽中坐忿怒,厲
 色見指,以朕行止出入,每不能同,若得稱心,規肆忿憾。惟公此旨,蚤欲見滅,而天道愛善,朕獲南面,不長惡逆,挫公毒心。



 自大明積費,國弊民凋,加景和奢虐,府藏罄盡。朕在位甫爾,恤義具瞻,仍值終阻蜂起,日耗萬金,公卿庶民,傾產歸獻。積受台奉,貲畜優廣。朕踐阼之初,公請故太宰東傳餘錢,見入數百萬,內不充養,外不助國,散賜諂諛,遍惠趨隸。



 推心考行,事類斯比。群小交構,遂生異圖,籍籍之義,轉盈民口。公若地居衡寄,任專八柄,
 德育於民,勳高於物,勢不自安,於事為可。公既才均櫟木,牽以曲全,因高無民,得守虛靜,而坐作凶咎,自囗深釁。由朕誠感無素,爰至於此,永尋多難,惋慨實深。



 凡人所行,各有本志。朕博愛尚仁,為日已久,尚能含仇恕罪,著於觸事,豈容於公,不相隱忍。但禍萌易漸,去惡宜疾,負荷之重,寧得坐觀。且蔓草難除,燎火須撲,狡扇之徒,宜時誅剪。已詔司戮,肅正典刑。公身居戚長,情禮兼至,準之常科,顧有惻怛,宜少申國憲,以弔不臧。今以淮南、
 宣城、歷陽三郡還立南豫州,降公為車騎將軍、開府儀同三司、南豫州刺史,削邑千戶,侍中、王如故。



 出鎮宣城,上遣腹心楊運長領兵防衛。同黨柳欣慰、徐虎兒、陳道明、寧敬之、閭丘邈之、樊平祖、孟敬祖並伏誅。明年六月,上又令有司奏:「禕忿懟有怨言,請免官,削爵土,付宛陵縣獄,依法窮治。」不許。乃遣大鴻臚持節,兼宗正為副奉詔責禕,逼令自殺,時年三十五,即葬宣城。



 子充明,輔國將軍、南彭城、東莞二郡太守。廢徙新安歙縣。後廢帝即
 位,聽還京邑。順帝昇明二年卒,時年二十八,無子。



 武昌王渾,字休淵,文帝第十子也。元嘉二十四年,年九歲,封汝陰王,食邑二千戶。為後軍將軍,加散騎常侍。索虜南寇,破汝陰郡,徙渾為武昌王。少而凶戾,嘗出石頭,怨左右人,援防身刀斫之。元凶弒立,以為中書令。山陵夕,裸身露頭,往散騎省戲,因彎弓射通直郎周朗,中其枕,以為笑樂。世祖即位,授征虜將軍、南彭城、東海二郡太守,出鎮京口。



 孝建元年,遷使持節、監雍、梁、南北秦四
 州、荊州之竟陵隨二郡諸軍事、寧蠻校尉、雍州刺史,將軍如故。渾至鎮,與左右人作文檄,自號楚王,號年為永光元年,備置百官,以為戲笑。長史王翼之得其手跡,封呈世祖。上使有司奏免為庶人,下太常,絕其屬籍,徙付始安郡。



 上遣員外散騎侍郎戴明寶詰渾曰:「我與汝親則同氣,義則君臣,遣任西蕃,以同盤石,云何一旦反欲見圖?文檄處分,事跡炳然,不忠不義,乃可至此。豈唯天道助順,逆志難充,如其凶圖獲逞,天下誰當相容?前事
 不遠,足為鑒戒。加以頻歲釁難,非起外人,唯應相與厲精,以固七百。汝忽復構此,良可悲惋。國雖有典,我亦何忍極法,好自將養,以保松、喬之壽。」逼令自殺,即葬襄陽,時年十七。大明四年,聽還葬母江太妃墓次。太宗即位,追封為武昌縣侯。



 王翼之,字季弼,琅邪臨沂人,晉黃門侍郎徽之孫也。官至御史中丞,會稽太守,廣州刺史。謚曰肅子。



 海陵王休茂,文帝第十四子也。孝建二年,年十一,封海
 陵王,食邑二千戶。



 大明二年,以為使持節、都督雍、梁、南北秦四州、郢州之竟陵、隨二郡諸軍事、北中郎將、寧蠻校尉、雍州刺史。進號左將軍,增邑千戶。時司馬庾深之行府事,休茂性急疾,欲自專,深之及主帥每禁之,常懷忿怒。左右張伯超至所親愛,多罪過,主帥常加呵責,伯超懼罪,謂休茂曰:「主帥密疏官罪過,欲以啟聞,如此,恐無好。」休茂曰:「為何計?」伯超曰:「唯當殺行事及主帥,且舉兵自衛。此去都數千里,縱大事不成,不失入虜中為王。」
 休茂從之。夜挾伯超及左右黃靈期、蔡捷世、滕穆之、王寶龍、來承道、彭叔兒、魏公子、陳伯兒、張駟奴、楊興、劉保、餘雙等,率夾轂隊,於城內殺典簽楊慶,出金城,殺司馬庾深之、典簽戴雙。



 集征兵眾,建牙馳檄,使佐吏上車騎大將軍、開府儀同三司,加黃鉞。侍讀博士荀銑諫爭,見殺。伯超專任軍政,殺害自己。休茂左右曹萬期挺身斫休茂,被創走,見殺。休茂出城行營,諮議參軍沈暢之等率眾閉門拒之。休茂馳還,不得入。義成太守薛繼考為
 休茂盡力攻城,殺傷甚眾,暢之不能自固,遂得入城,斬暢之及同謀數十人。



 其日,參軍尹玄慶起義,攻休茂,生禽之,將出中門斬首,時年十七。母妻皆自殺,同黨悉伏誅。城中撓亂,無相統領。時尚書右僕射劉秀之弟恭之為休茂中兵參軍,眾共推行府州事。繼考以兵肋恭之,使作啟事云立義,自乘驛還都,上以為永嘉王子仁北中郎諮議參軍、河南太守,封冠軍縣侯,食邑四百戶。尋事泄,伏誅。



 恭之坐繫尚方。以玄慶為射聲校尉。有司奏
 絕休茂屬籍,貶姓為留,上不許。即葬襄陽。



 庾深之,字彥靜,新野人也。以事先朝見知。元嘉二十九年,自輔國長史為長沙內史。南郡王義宣為荊、湘二州,加深之寧朔將軍,督湘州七郡。明年,義宣為逆,深之據巴陵拒之。轉休茂司馬。見害之旦,子孫亦死。追贈深之冠軍將軍、雍州刺史,荀銑員外散騎侍郎,曹萬期始平太守。



 桂陽王休範,文帝第十八子也。孝建三年,年九歲,封順陽王,食邑二千戶。



 大明元年,改封桂陽王。為冠軍將軍、
 南彭城、下邳太守。三年,出為江州刺史,尋加征虜將軍,邑千戶。入為秘書監,領前軍將軍。七年,遷左衛將軍,加給事中。



 前廢帝永光元年,轉中護軍,領崇憲衛尉。



 太宗定亂,以為使持節、都督南徐、徐、南兗、兗四州諸軍事、鎮北將軍、南徐州刺史,給鼓吹一部。時薛安都據彭城反叛,遣從子索兒南侵,休範進據廣陵,督北討諸軍事,加南兗州刺史,進征北大將軍,加散騎常侍,還京口,解兗州,增邑二千戶,受五百戶。泰始五年,徵為中書監、中軍
 將軍、揚州刺史,常侍如故。



 明年,出為使持節、都督江、郢、司、廣、交五州豫州之西陽、新蔡、晉熙、湘州之始興四郡諸軍事、征南大將軍、江州刺史。尋加開府儀同三司,未拜,改授都督南徐、徐、南兗、兗、青、冀六州諸軍事、驃騎大將軍、南徐州刺史,持節、常侍、開府如故。未拜,以驃騎大將軍還為江州,進督越州諸軍事,給三望車一乘。太宗遺詔,進位司空,改常侍為侍中,加班劍三十人。



 休範素凡訥,少知解,不為諸兄所齒遇。太宗常指左右人謂王
 景文曰:「休範人才不及此,以我弟故,生便富貴。釋氏願生王家,良有以也。」及太宗晚年,晉平王休祐以狠戾致禍,建安王休仁以權逼不見容,巴陵王休若素得人情,又以此見害。唯休範謹澀無才能,不為物情所向,故得自保;而常懷憂懼,恒慮禍及。



 及太宗晏駕,主幼時艱,素族當權,近習秉政,休範自謂宗戚莫二,應居宰輔,事既不至,怨憤彌結。招引勇士,繕治器械,行人經過尋陽者,莫不降意折節,重加問遺,囗囗留則傾身接引,厚相資給。
 於是遠近同應,從者如歸。朝廷知其有異志,密相防禦,雖未表形跡,而釁難已成。母荀太妃薨,葬廬山,以示不還之志。



 解侍中。



 時夏口闕鎮,朝議以居尋陽上流,欲樹置腹心,重其兵力。元徽元年,乃以第五皇弟晉熙王燮為郢州刺史,長史王奐行府州事,配以資力,出鎮夏口。慮為休範所撥留,自太子洑去,不過尋陽。休範大怒,欲舉兵襲朝廷,密與典簽新蔡人許公輿謀之。表治城池,修起樓堞,多解榜板,擬以備用。其年,進位太尉。明年五
 月,遂舉兵反。虜發百姓船乘,使軍隊稱力請受,付以榜解板,合手裝治,二三日間,便悉整辦。率眾二萬,鐵騎數百匹,發自尋陽,晝夜取道。書與袁粲、褚淵、劉秉曰:夫治政任賢,宜親疏相輔,得其經緯,則結繩可及;失其規矩,則危亡可期。



 漢承戰國之餘,傷周室衰殄,立磐石之宗,而致七國之亂。魏革漢典,創於前失,遂使諸王絕朝聘之禮,是以根疏葉枯,政移異族。今宗室衰微,自昔未有,泰寧之世,足以為譬。孤子忝枝皇族,預關興毀,雖欲忘
 言,其可得乎!



 高祖武皇帝升睿三光,滌紛四表。太祖文皇帝欽明冠古,資乾承歷,秉鉞西服,鳴鑾東京,搜賢選能,納奇賞異。孝武皇帝歧嶷天縱,先機雷發,陵波靜亂,宏業中興,儲嗣不腆,遂貽禍難。於時建安王以家難頻遘,宜立長主,明皇帝恢郎淵懿,仁潤含遠,奉戴南面,允合天人。而太尉以年長居卑,怨心形色,柳欣慰等規行不軌,事跡披猖。驃騎以忤顏失旨,應對不順,在蕃刻削,怨結人鬼。先帝明於號令,豈枉法為親,二王之釁,實自
 由己。但司徒巴陵王勞謙為國,中流事難,有不世之勳,奉時如天,事兄猶父,非唯令友,信為國器。唐叔之忠,而受管、蔡之罪,親戚哀憤,行路嗟歎。王地籍光潔,德厭民望,並無寸罪,受斃讒邪。先帝穆於友于,留心親戚,去昔事平之後,面受詔誨,禮則君臣,樂則兄弟,升級賜賞,動不移年,撫慰孜孜,恒如不足,豈容一旦鬩牆,致此禍害,良有由也。



 先帝寢疾彌年,體疲膳少,雖神照無虧,而慮有失德,補闕拾遺,責在左右。



 於時出入臥內,唯有運長、
 道隆,群細無狀,因疾遘禍,見上不和,知無瘳拯,慮晏駕之日,長王作輔,奪其寵柄,不得自專。是以內假帝旨,外託朝議,諛辭詭貌,萬類千端,升進姦回,屠斥賢哲,外矯天則,內誣人鬼。是以星紀違常,義望失度。



 昔魏顆擇命,《春秋》美之;秦穆殉良,《詩》有明刺。臣子之節,得失必書,不及匡諫,猶以為罪。交間蒼蠅,驅扇禍戮,爵以貨重,才由貧輕,先帝舊人,無罪黜落,薦致鄉親,遍布朝省。諂諛親狎者,飛榮玉除;靜立貞粹者,柴門生草。事先關己,雖非
 必行;若不諮詢,雖是必抑。海內遠近,人誰不知,未解執事,不加斧鉞,遂致先帝有殺弟之名,醜聲遺於君父,格以古義,豈得為忠!先帝崩殂,若無天地,理痛常情,便應赴泣。但兄弟枉酷,已陷讒細,孤子已下,復觸姦機。是以望陵墳而摧裂,想鑾旂而抽慟。雖復才違寄寵,而地屬負荷,顧命之辰,曾不見及。分崩之際,詔出兩豎,天誘其衷,得居乎外。若受制群邪,則玉石同碎矣!以宇宙之基,一旦受制卑瑣,劉氏家國,使小人處分,終古以來,未有
 斯酷。昔石顯、曹節,方今為優,而望之、仲舉,由以致弊。至於遭逢醜慝,豈有古今者乎!



 諸賢胄籍冠冕,世歷忠貞,位非恩樹,勳豈寵結,憂國勤王,社稷之鎮,豈可含縱讒凶,坐觀傾覆。自惟宋室未殞,得以推移者,正內賴諸賢,防勒姦軌;外有孤子,跨據中流。而人非金石,何能支久,使一虧落,則本根莫庇。當今主上沖幼,宜明典章,征虜之鎮,不見慰省,逆旅往來,尚有顧眄,骨肉何仇,逼使離隔。禽獸之心,橫生疑貳,經由此者,每加約截,同惡相求,
 有若市賈。以孤子知其情狀,恒恐以此乘之,鉗勒州郡,過見防禦。近遣西南二使,統內宣傳,不容恐懼,即遣啟并有別書。若以孤子有過,便應鳴鼓見伐;如其不爾,宜令各有所歸。與殺不辜,憲有常辟,三公之使,無罪而斬,鄙雖不肖,天子之季父,卑小主者,敢不如是乎!



 孤子承奉今上,如事先朝,夙宵恭謹,散心雲日,晦望表驛,相從江衢,有何虧違,頓至於此。既已甘心,其可再乎!如往來所說,以孤子納士為尤,此輩懼其身罪,豈為國計。



 在昔
 四豪,列國公子,猶博引廣納,門客三千。況孤子位居鼎司,捍衛畿甸,且今與昔異,咸所知也。狡虜陵掠,江、淮侵逼,主上年稚,宗室衰微,邪僭用命,親賢結舌,疆場嬰塗炭之苦,征夫有勤役之勞,瓜時不代,齊猶致禍,況長淮戍卒,歷年怨思,不務拓遠強邊,而先事國君親戚,以此求心,何事非亂。又以繕治盆壘,復致囂聲。自晉、宋之災,積貯百萬,孤子到鎮,曾不數千里,且修城池,整郭邑,為治常理,復何足致嫌邪?若以中流清蕩,則任農夫不應
 實力強兵,作鎮姑孰,俱防寇害,豈得獨嫌於此。昔成王之明,而為流言致惑,若使金縢不開,則周公無以自保。樂毅歸趙,不忍謀燕,況孤子禮則君臣,恩猶父子者乎!所以枕戈泣血,只以兄弟之仇爾。觀其不逞之意,豈可限量。設使遂其虐志,諸君欲安坐得乎!脣亡齒寒,理不難見。桂蠹必除,人邪必翦,枉突徙薪,何勞多力。望便執錄二豎,以謝冤魂,則先帝不失順悌之名,宋世無枉筆之史。



 此州地居形要,路枕九江,控弦跨馬,越關而至。重
 氣輕死,排藪競出,練甲照水,總戈成林,劋此纖隸,何患不克。但千鈞之弩,不為鼷鼠發機,欲使薰蕕內辨,晉陽外息爾。功有所歸,不亦可乎!便當投命有司,謝罪天闕,同奉溫凊,齊心庶事。伊、霍之任,非君而誰;周、邵之職,頗以自許。左提右挈,無愧古人。



 昔平、勃剛斷,產、祿蚤誅;張、溫趑趄,文臺扼腕。事之樞機,得失俄頃,往車今轍,庶無惑焉。近持此意,申之沈攸,其憤難不解諸王致此!既知禍原,銳然奮發,蓄兵厲卒,以俟同舉。張興世發都日,受
 制凶黨,揚颿直逝,遂不見遇,孤子近遣信申述姦禍,方大惆惋,追恨前迷,比者信使,每申勤款。王奐佐郢,兵權在握,厥督屠枉,朝野嗟痛,猶父之怨,寧可與之比肩。孤子此舉,增其慷慨,義之所勸,其應猶響。諸君或未得此意,故先告懷。徙倚一隅,遲及委問。孤子哀疾尪毀,窮盡無日,庶規史鰍,死不忘本。臨紙荒哽,言不詮第。



 大雷戍主杜道欣馳下告變。道欣至一宿,休範已至新林,朝廷震動。平南將軍齊王出次新亭壘,領軍將軍劉勔、前兗
 州刺史沈懷明據石頭,征北將軍張永屯白下,衛將軍袁粲、中軍褚淵、尚書左僕射劉秉等入衛殿省。時事起倉卒,不暇得更處分,開南北二武庫,隨將士意取。



 休範於新林步上,及新亭壘,自臨城南,於臨滄囗上,以數十人自衛。屯騎校尉黃回見其可乘,乃偽往請降,並宣齊王意旨,休範大悅,以二子德宣、德嗣付回與為質,至即斬之。回與越騎校尉張敬兒直前斬休範首,持還,左右並奔散。



 初,休範自新林分遣同黨杜耳、丁文豪、杜墨蠡
 等,直向硃雀。休範雖死,墨蠡等不相知聞。王道隆率羽林兵在朱雀門內,聞賊至,急召劉勔。勔自石頭來赴,仍進桁南,戰敗,死之。墨蠡等乘勝直入朱雀門,王道隆為亂兵所殺。墨蠡等唱:「太尉至。」休範之死也,齊王遣隊主陳靈寶齎首詣臺,道逢賊,棄首於水,挺身得達。雖唱云已平,而無以為據,眾愈疑惑。張永棄眾於白下,沈懷明於石頭奔散,撫軍典簽茅恬開東府納賊。墨蠡徑至杜姥宅,中書舍人孫千齡開囗明門出降,宮省恇擾,
 無復固志。時庫藏賞賜已盡,皇太后、太妃剔取宮內金銀器物以充用。羽林監陳顯達率所領於杜姥宅與墨蠡戰,破之。至宣陽御道,諸賊一時奔散,斬墨蠡、文豪及同黨姜伯玉、柳中虔、任天助等。許公輿走還新茶,村民斬送之。晉熙王燮自夏口遣軍平尋陽,德嗣弟青牛、智藏並伏誅。詔建康、秣陵二縣收斂諸軍死者,並殺賊屍,並加藏埋。



 史臣曰:語有之,投鼠而忌器,信矣。阮佃夫、王道隆專用
 主命,臣行君道,識義之徒,咸思戮以馬劍。休範馳兵象魏,矢及君屋,忠臣義士,莫不銜膽爭先。



 夫以邪附君,猶或自免,況於仗正順以爭主哉!



\end{pinyinscope}