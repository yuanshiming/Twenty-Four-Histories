\article{卷八十七列傳第四十七 蕭惠開 殷琰}

\begin{pinyinscope}

 蕭惠開,南蘭陵人,征西
 將軍思
 話子也。初名慧開,後改慧為惠。少有風氣,涉獵文史,家雖貴戚,而居服簡素。初為秘書郎,著作並名家年少。惠開意趣與人多不同,比
 肩或三年不共語。外祖光祿大夫沛郡劉成戒之曰:「汝恩戚家子,當應將迎時俗,緝外內之歡。如汝自業,將無小傷多異,以取天下之疾患邪?」惠開曰:「人間宜相緝和,甚如慈旨。但不幸耿介,恥見作凡人,畫龍未成,故遂至於多忤耳。」轉太子舍人。與汝南周朗同官友善,以偏奇相尚。轉尚書水部郎,始興王浚征北府主簿,南徐州治中從事史,徙汝陰王友;又為南徐州別駕,中書侍郎,江夏王義恭大將軍大司馬從事中郎。



 孝建元年,自太子
 中庶子轉黃門侍郎,與侍中何偃爭積射將軍徐沖之事。偃任遇甚隆,惠開不為之屈,偃怒,使門下推彈之。惠開乃上表解職曰:「陛下未照臣愚,故引參近侍。臣以職事非長,故委能何偃,凡諸當否,不敢參議。竊見積射將軍徐沖之為偃命所黜,臣愚懷謂有可申,故聊設微異。偃恃恩使貴,欲使人靡二情,便訶脅主者。手定文案,割落臣議,專載己辭。雖天照廣臨,竟未見察臣理,違顏咫尺,致茲壅濫,則臣之受劾,蓋何足悲。但不順侍中,臣有
 其咎,當而行之,不知何過。且議之不允,未有彈科,省心揆天,了知在宥。臣不能謝愆右職,改意重臣,刺骨鑠金,將在朝夕。乞解所忝,保拙私庭。」時偃寵方隆,由此忤旨,別敕有司以屬疾多,免惠開官。思話素恭謹,操行與惠開不同,常以其峻異,每加嫌責。



 及見惠開自解表,自歎曰:「兒子不幸與周朗周旋,理應如此。」杖之二百。尋重除中庶子。



 丁父艱,居喪有孝性,家素事佛,凡為父起四寺,南岸南岡下,名曰禪岡寺;曲阿舊鄉宅,名曰禪鄉寺;京
 口墓亭,名曰禪亭寺;所封封陽縣,名曰禪封寺。謂國僚曰:「封秩蓋鮮,而兄弟甚多,若使全關一人,則在我所讓。若使人人等分,又事可悲恥。寺眾既立,自宜悉供僧眾。」由此國秩不復下均。服除,除司徒左長史。大明二年,出為海陵王休茂北中郎長史、寧朔將軍、襄陽太守,行雍州州府事。



 善於為政,威行禁止。襲封封陽縣侯。還為新安王子鸞冠軍長史,行吳郡事。惠開妹當適桂陽王休範,女又當適世祖子,發遣之資,應須二千萬。乃以為豫
 章內史,聽其肆意聚納,由是在郡著貪暴之聲。入為尚書吏部郎,不拜,徙御史中丞。世祖與劉秀之詔曰:「今以蕭惠開為憲司,冀當稱職。但一往服領,已自殊有所震。」



 及在任,百僚畏憚之。



 八年,入為侍中。詔曰:「惠開前在憲司,奉法直繩,不阿權戚,朕甚嘉之。



 可更授御史中丞。」母憂去職。起為持節、督青冀二州諸軍事、輔國將軍、青冀二州刺史,不行。改督益寧二州刺史,持節、將軍如故。惠開素有大志,至蜀,欲廣樹經略,善於述事,對賓僚及士
 人說收牂牁、越巂以為內地,綏討蠻、濮,闢地征租;聞其言者,以為大功可立。太宗即位,進號冠軍將軍,又進平西將軍,改督為都督。晉安王子勛反,惠開乃集將佐謂之曰:「湘東太祖之昭,晉安世祖之穆,其於當璧,並無不可。但景和雖昏,本是世祖之嗣,不任社稷,其次猶多。吾奉武、文之靈,兼荷世祖之眷,今便當投袂萬里,推奉九江。」乃遣巴郡太守費欣壽領二千人東下,為巴東人任叔兒起義所邀,欣壽敗沒,陜口道不復通。更遣州治中
 程法度領三千人步出梁州,又為氐賊楊僧嗣所斷。



 先是,惠開為治,多任刑誅,蜀土咸懷猜怨。及聞欣壽沒,法度又不得前,晉原一郡遂反,於是諸郡悉應之,並來圍城。城內東兵不過二千,凡蜀人惠開疑之,皆悉遣出。子勛尋平,蜀人並欲屠城,以望厚賞。惠開每遣軍出戰,未嘗不捷,前後所摧破殺傷不可勝計。外眾逾合,勝兵者十餘萬人。時天下已平,太宗以蜀土險遠,赦其誅責,遣惠開弟惠基步道使蜀,具宣朝旨。惠基既至涪,而蜀人
 志在屠城,不欲使王命遠達,遏留惠基不聽進。惠基率部曲破其渠帥馬興懷等,然後得前。惠開奉旨歸順,城圍得解。



 時太宗遣惠開宗人寶首水路慰勞益州,寶首欲以平蜀為功。更獎說蜀人,於是處處蜂起,凡諸離散者,一時還合。渠帥趙燕、句文章等,與寶首屯軍於上,去成都六十里,眾號二十萬人。惠開欲遣擊之,將佐咸曰:「攻破蜀賊,誠不為難。但慰勞使至,未獲奉受,而遣兵相距,何以自明本心。」惠開曰:「今水陸四斷,表啟路絕,寶首
 或相誣陷,謂我不奉朝旨。我之欲戰,本在通使;使若得通,則誠心達矣。」乃作啟事,具陳事情,使腹心二人帶啟,戒之曰:「須賊破路開,便躍馬馳去。」遣永寧太守蕭惠訓、別駕費欣業萬兵並進,與戰,大破之,生禽寶首,囚於成都縣獄。所遣使至,上使執送寶首,除惠開晉平王休祐驃騎長史、南郡太守,不拜。泰始四年,還至京師。



 初,惠開府錄事參軍到希微負蜀人債將百萬,為責主所制,未得俱還。惠開與希微共事不厚,以為隨其同上,不能攜
 接得還,意恥之。廄中凡有馬六十匹,悉以乞希微償責,其意趣不常皆如是。先劉瑀為益州,張悅代之,瑀去任,凡所攜將佐有不樂反者,必逼制將還。語人曰:「隨我上,豈可為張悅作西門客邪!」惠開自蜀還,資財二千餘萬,悉散施道路,一無所留。



 五年,又除桂陽王休範征北長史、南東海太守。其年,會稽太守蔡興宗之郡,而惠開自京口請假還都,相逢於曲阿。惠開先與興宗名位略同,又經情款,自以負釁摧屈,慮興宗不能詣己,戒勒部下:「
 蔡會稽部伍若借問,慎不得答。」惠開素嚴,自下莫敢違犯。興宗見惠開舟力甚盛,不知為誰,遣人歷舫訊,惠開有舫十餘,事力二三百人,皆低頭直去,無一人答者。



 復為晉平王休祐驃騎長史,太守如故。六年,除少府,加給事中。惠開素剛,至是益不得志,寺內所住齋前,有嚮種花草甚美,惠開悉刬除,列種白揚樹。每謂人曰:「人生不得行胸懷,雖壽百歲,猶為夭也。」發病歐血,吐如肝肺者甚多。



 除巴陵王休若征西長史、寧朔將軍、南郡太守,未
 拜。七年,卒,時年四十九。子睿嗣,齊受禪,國除。惠開與諸弟並不睦,惠基使益州,遂不相見。與同產弟惠明亦著嫌隙云。



 殷琰,陳郡長平人也。父道鸞,衡陽王義季右軍長史。琰少為太祖所知,見遇與琅邪王景文相埒。初為江夏王義恭征北行參軍,始興王浚後軍主簿,出為鄱陽、晉熙太守,豫州治中從事史,廬陵內史。臧質反,棄郡奔北皖。琰性有計數,欲進退保全,故不還都邑。事平,坐繫尚方,
 頃之被宥。除海陵王國郎中令,不拜。臨海王子頊為冠軍將軍、吳興太守,以琰為錄事參軍,行郡事。復為豫州別駕,太宰戶曹屬,丹陽丞,尚書左丞,少府,尋陽王子房冠軍司馬,行南豫州,隨府轉右軍司馬,又徙巴陵王休若左軍司馬。



 前廢帝永光元年,除黃門侍郎,出為山陽王休祐右軍長史、南梁郡太守。休祐入朝,琰仍行府州事。太宗泰始元年,以休祐為荊州,欲以吏部郎張岱為豫州刺史。



 會晉安王子勛反,即以琰督豫司二州南豫
 州之梁郡諸軍事、建武將軍、豫州刺史,以西汝陰太守龐道隆為琰長史,殿中將軍劉順為司馬。順勸琰同子勛。琰家累在京邑。意欲奉順,而土人前右軍參軍杜叔寶、前陳南頓二郡太守皇甫道烈、道烈從弟前馬頭太守景度、前汝南潁川二郡太守龐天生、前睢陽令夏侯季子等,並勸琰同逆。



 琰素無部曲,門義不過數人,無以自立,受制於叔寶等。太宗遣冗從僕射柳倫領軍助,驃騎大將軍山陽王休祐又遣中兵參軍鄭瑗說琰令還。
 二人至,即與叔寶合。叔寶者,杜坦之子,既土豪鄉望,內外諸軍事並專之。



 弋陽太守卜天生據郡同逆,斷梁州獻馬得百餘匹。邊城令宿僧護起義斬天生,傳首京邑。太宗嘉之,以為龍驤將軍,封建興縣侯,食邑三百戶。時綏戎將軍、汝南新蔡二郡太守周矜起義於懸瓠,收兵得千餘人。袁顗遺信誘矜司馬汝南人常珍奇,以金鈴為信。珍奇即日斬矜,送首詣顗,顗以珍奇為汝南、新蔡二郡太守。太宗追贈矜本官,以義陽內史龐孟虯為司
 州刺史,領隨郡太守。孟虯不受命,起兵同子勛。



 子勛召孟虯出尋陽,而以孟虯子定光行義陽郡事。



 太宗知琰逼迫土人,事不獲已,猶欲羈縻之。以琰兄前中書郎瑗為司徒右長史,子邈為山陽王休祐驃騎參軍。子勛遣使以琰為輔國將軍、梁郡太守,後又加豫州,假節督南豫數郡。杜叔寶求琰上佐,龐道隆慮其為禍,乃請奉表使尋陽。琰即以叔寶為長史、梁郡太守。休祐步入朝,家內猶分停壽陽,琰資給供贍,事盡豐厚。



 二年正月,太宗
 遣輔國將軍劉勔率寧朔將軍呂安國西討,休祐出鎮歷陽,為諸軍總統。時徐州刺史薛安都亦據彭城反,募能生禽琰、安都,封千戶縣侯,賜布絹各二千匹。二月,勔進軍小峴。初,合肥戍主、南汝陰太守薛元寶委郡奔子勛,前太守硃輔之據城歸順。琰遣攻輔之,輔之敗走。琰以前右軍參軍裴季為南汝陰太守,季又歸順,太宗即而授之。琰所用象縣令許道蓮亦率二百人歸降,太宗以為馬頭太守。三月,上又遣寧朔將軍劉懷珍、段僧愛、
 龍驤將軍姜產之馬步三軍,助勔討琰。



 義軍主黃回募江西楚人千餘,斬子勛所置馬頭太守王廣元,以回為龍驤將軍。淮西人前奉朝請鄭墨率子弟部曲及淮右郡起義於陳郡城,有眾一萬,太宗以為司州刺史。



 後虜寇淮西,戰敗見殺,追贈冠軍將軍。



 是月,劉順、柳倫、皇甫道烈、龐天生等馬步八千人,東據宛唐,去壽陽三百里。勔率眾軍並進,去順數里立營。在道遇雨,旦始至,壘塹未立,順欲擊之。時琰所遣諸軍並受節度,而以皇甫道
 烈、土豪柳倫,臺之所遣,順本卑微,不宜統督,唯二軍不受命。至是道烈、倫不同,順不能獨進,乃止。既而勔營壘漸立,不可復攻,因相持守。四月,勔錄事參軍王起、前部賊曹參軍甄澹等五人委勔奔順,順因此出軍攻勔。順幢主樊僧整與臺馬軍主驃騎中兵參軍段僧愛交槊鬥,僧整刺僧愛,殺之,追贈屯騎校尉。僧愛勇冠三軍,軍中並懼。太宗又遣太尉司馬垣閎率軍來會,步兵校尉龐沈之助裴季戍合肥。初,淮南人周伯符說休祐求起
 義兵,休祐不許,固請,乃遣之。杖策單行,至安豐,收得八百餘人,於淮西為遊兵。珍奇所置弋陽太守郭確遣將軍郭慈孫擊伯符於金丘,琰又遣中兵參軍杜叔寶助之。慈孫等為伯符所敗,並投水死。太宗以伯符為驃騎參軍。



 叔寶本謂臺軍停住歷陽不辦進,順等至,無不瓦解,唯齎一月日糧。既與勔相持,軍食盡,報叔寶送食;叔寶乃發車千五百乘,載米餉順,自以五千精兵防送之。



 勔聞之,軍副呂安國曰:「劉順精甲八千,而我眾不能居
 半,相持既久,彊弱勢殊,茍復推遷,則無以自立,所賴在彼糧將竭,我食有餘耳。若使叔寶米至,非唯難可復圖,我亦不能持久。今唯有間道襲其米車,出彼不意。若能制之,將不戰走矣。」



 勔以為然,乃以疲弱守營,簡選千百精手,配安國及軍主黃回等,間路出順後,於橫塘抄之。安國始行,計叔寶尋至,止齎二日熟食,食盡,叔寶不至,將士並欲還。



 安國曰:「卿等旦已一食,今晚米車不容不至。若其不至,夜去不晚。」叔寶果至,以米車為函箱陣,叔
 寶於外為遊軍,幢主楊仲懷領五百人居前,與安國、回等相會。



 仲懷部曲並欲退就叔寶,并力擊安國。仲懷曰:「賊至不擊,復欲何待?且統軍在後,政三二里間,比吾交手,何憂不至。」即便前戰,回所領並淮南楚子,天下精兵,眾力既倍,合戰,便破之。於陣殺仲懷,仲懷所領五百人死盡。叔寶至,而仲懷及士卒伏尸蔽野,回等欲乘勝擊之,安國曰:「彼將自走,不假復擊。」退軍三十里止宿,夜遣騎參候,叔寶果棄米車奔走。安國即復夜往,燒米車,驅
 牛二千餘頭而還。劉順聞米車見燒,叔寶又走,五月一日夜,眾潰,奔還壽陽,仍走淮西就常珍奇。勔於是方軌而進。



 叔寶斂居民及散卒,嬰城自守。勔與諸軍分營城外,黃回立航渡肥水。叔寶遣馬步三千,欲破航,并柵斷小峴埭,回擊大破之,焚其船柵。



 休祐與琰書曰:「君本文弱,素無武幹,是遠近所悉,且名器清顯,不應復有分外希覬。近者之事,當是劫於凶豎,不能守節。今大軍長驅,已造城下,勢孤援絕,禍敗交至,顧昔情款,猶有惻然。聖
 上垂天地之仁,開不世之澤,好生惡殺,遐邇所聞。顧琛、王曇生等皆軍敗迸走,披草乞活,尚蒙恩恕,晏處私門。今神鋒所臨,前無橫陳,況窮城弱眾,殘傷之餘,而欲自固乎!若開門歸順,自可不失富貴;將佐小大,並保榮爵。何故茍困士民,自求齏膾,身膏斧鑊,妻息並盡,老兄垂白,東市受刑邪!幸自思之。信言不爽,有如皎日。」上又遣王道隆齎詔宥琰罪。



 勔又與琰書曰:「昔景和凶悖,行絕人倫,昏虐險穢,諫諍杜塞,遂殘毀陵廟,芟刈百僚,縱毒
 窮凶,靡有紀極。于時人神回遑,莫能自保,中外士庶,咸願一匡。



 予職在直衛,目所備睹。主上神機天發,指麾克定,橫流塗炭,一朝太平,扶危拯急,實冠終古。而四方持疑,成此乖逆,資斧所臨,每從偃簡。足下以衣冠華胄,信概夙昭,附戾從違,猶見容養。賢兄長史,階升清列;賢子參軍,亦塞國網。間者進軍宛唐,計由劉順,退眾閉城,當時未了。過蒙朝恩,謬充將帥,蚤承風素,情有依然。今皇威遠申,三方蹙弱,勝敗之勢,皎然可覽。王御史昨至,主
 上敕、驃騎教、賢兄賢子書,今悉遣送。百代以來,未有弘恩曲宥,乃至於此。且朝廷方宣示大義,惟新王道,何容摽虛辭於士女,失國信於一州。以足下明識淵見,想必不俟終日。如其孤背亭毒,弗忌屠陷者,便當窮兵肆武,究法極刑。將恐貴門無復祭祀之主,墳壟乏掃灑之望。進謝忠臣,退慚孝子,名實兩喪,沒有餘責。扶力略白,幸加研覽。」琰本無反心,事由力屈,叔寶等有降意,前後屢遣送誠箋,而眾心持疑,莫能相一,故歸順之計,每多愆
 塞,嬰城愈固。弋陽西山蠻田益之起義,攻郭確於弋陽,以益之為輔國將軍,督弋陽西山事。六月,勔築長圍始合。田益之率蠻眾萬餘人攻龐定光於義陽,定光遣從兄文生拒之,為益之所破,見殺,遂圍其城。定光求救於子勛,子勛以定光父孟虯為司州刺史,率精兵五千救義陽,并解壽陽之圍。常珍奇又自懸瓠遣三千人援定光,屯軍柳水。益之不戰,望風奔散。孟虯乘勝進軍向壽陽。初,常珍奇遣周當、垣式寶率數百人送仗與琰。式寶
 驍勇絕眾,因留守北門,乃率所領,開門掩襲勔,入其營;勔逃避得免,式寶得勔衣帽而去。



 勔於是乃豎長圍,治攻道於東南角,并填塹。東南角有高樓,隊主趙法進計曰:「外若進攻,必先攻樓,樓頹落,既傷將士,又使人情沮壞,不如先自毀之。」從其言。勔用草茅苞土,擲以塞塹。擲者如雲,城內乃以火箭射之,草未及燃,後土續至,一二日,塹便欲滿。趙法進復獻計,以鐵珠子灌之。珠子流滑,悉緣隙得入,草於是火燃,二日間草盡,塹中土不過二
 三寸。勔乃作大蝦蟆車載土,牛皮蒙之,三百人推以塞塹。琰戶曹參軍虞挹之造確車,擊之以石,車悉破壞。



 初,廬江太守王子仲棄郡奔尋陽,廬江人起義,休祐遣員外散騎侍郎陸悠之助之。劉胡遣其輔國將軍薛道標渡江煽動群蠻,規自廬江掩襲歷陽,悠之眾弱,退保譙城。司徒建安王休仁遣參軍沈靈寵馳據廬江,道標後一日方至,悠之自譙城來會,因與道標相持。七月,龐孟虯至弋陽,勔遣呂安國、垣閎、龍驤將軍陳顯達、驃騎參
 軍孟次陽拒之。孟虯軍副呂興壽與安國有舊,率所領降。安國進軍,破孟虯於蓼潭,義軍主陳肫又破之於汝水,孟虯走向義陽;義陽已為王玄謨子曇善起義所據,乃逃於蠻中。淮西人鄭叔舉起義擊常珍奇,以為北豫州刺史。



 八月,皇甫道烈、柳倫等二十一人聞孟虯敗,並開門出降。勔因此又與琰書曰:「柳倫來奔,具相申述,方承足下跡纏穢亂,心秉忠誠,惘默窮愁,不親戎政。去冬開天之始,愚迷者多,如足下流比,進非社稷宗臣,退無
 顧命寄託,朝廷既不偏相嫌責,足下亦復無所獨愧。程天祚已舉城歸順,龐孟虯又繼迹奔亡,劉胡困於錢溪,袁顗欲戰不得,推理揆勢,亦安能久。且南方初起,連州十六,擁徒百萬,仲春以來,無戰不北,摧陷殄滅,十無一二。南憑袁顗弱卒,北恃足下孤城,以茲定業,恐萬無一理。方今國網疏略,示舉宏維,比日相白,想亦已具矣。且倫等皆是足下腹心牙爪,所以攜手相舍,非有怨恨也,了知事不可濟,禍害已及故耳。夫擁數千烏合,抗天下
 之兵,傾覆之狀,豈不易曉。假令六蔽之人,猶當不為其事,況復足下少祖名教,疾沒世無稱者邪。所以復有此白者,實惜華州重鎮,鞠為茂草,兼傷貴門一日屠滅。足下若能封府庫,開四門,宣語文武,示以禍福,先遣咫尺之書,表達誠款,然後素車白馬,來詣轅門,若令足下髮膚不全,兒姪彫耗者,皇天后土,實聞此言。至辭不華,寧復多白。」



 薛道標猶在廬江,劉胡又分兵揚聲向壽陽及合肥。勔遣許道蓮馳赴合肥,助裴季文,又遣黃回、孟次
 陽乃屯騎校尉段佛榮、武衛將軍王廣之繼之。道標率其黨薛元寶等攻合肥,勔所遣諸軍未至,為道標所陷,季文及武衛將軍葉慶祖力戰死之。



 勔馳遣垣閎總統諸軍攻合肥。是月,劉胡敗走,尋陽平定。太宗遣叔寶從父弟季文至琰城下,與叔寶語,說四方已定,勸令時降。叔寶曰:「我乃信汝,恐為人所誑耳!」叔寶閉絕子勛敗問,有傳者即殺之。時琰子邈東在京邑,繫建康,太宗送邈與琰,令說南賊已平之問,自建康出,便防送就道。議者
 以為宜聽邈與伯父瑗私相見,不爾無以解城內之惑,不從。邈至,叔寶等果疑,守備方固。十月,薛道標突圍,與十餘騎走奔淮西,投常珍奇,薛元寶歸降。



 先是,晉熙太守閻湛之據郡同逆,至是沈靈寵自廬江攻之。湛之未知尋陽已敗,固守不降。靈寵乃取諸將破劉胡文書置車中,攻城偽敗,棄車而走。湛之得書大駭,其夜奔逃。十一月,常珍奇乞降,慮不見納,又求救於索虜。太宗即以珍奇為司州刺史,領汝南、新蔡二郡太守。虜亦遣偽帥
 張窮奇騎萬匹救之。十二月,虜至汝南,珍奇開門納虜,淮西七縣民並連營南奔,劉順亦棄虜歸順。



 南賊降者,太宗並送琰城下,令與城內交言,由是人情沮喪。琰將降,先送休祐內人出城,然後開門。時琰有疾,以板自輿,與諸將帥面縛請罪。勔並撫宥,無所誅戮,自將帥以下,財物資貨,皆以還之,纖毫無所失。虜騎救琰,至師水,聞城陷,乃破義陽,殺掠數千人而去。垣式寶尋復反叛,投常珍奇。以平琰功,劉懷珍封艾縣侯,食邑四百戶,垣
 閎樂鄉縣侯,孟次陽攸縣子,王廣之蒲圻縣子,陳顯達彭澤縣子,呂安國鐘武縣子,食邑各三百戶,黃回葛陽縣男,食邑二百戶。送琰及偽節還京都。



 久之,為王景文鎮南諮議參軍,兼少府。泰豫元年,除少府,加給事中。後廢帝元徽元年,卒,時年五十九。琰性和雅靜素,寡嗜欲,諳前世舊事,事兄甚謹,少以名行見稱。在壽陽被攻圍積時,為城內所懷附。揚州刺史王景文、征西將軍蔡興宗、司空褚淵,並與之友善云。



 史臣曰:夫求忠臣必於孝子之門,蓋以類得之也。昔啟方說主,跡表遺親,鄧攸淳行,愛兼猶子,雖稟分參差,情紀難一,而均薄等厚,未之或偏。惠開親禮雖篤,弟隙尤著,方寸之內,孝友異情,險於山川,有驗於此也。



\end{pinyinscope}