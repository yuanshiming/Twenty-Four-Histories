\article{卷四十二列傳第二 劉穆之 王弘}

\begin{pinyinscope}

 劉穆
 之,字道和,小字道民,東莞莒人,漢齊悼惠王肥後也,世居京口。少好《書》、《傳》,博覽多通,為濟陽江敳所知。敳為建武將軍、琅邪內史,以為府主簿。



 初,穆之嘗夢與高
 祖俱泛海,忽值大風,驚懼。俯視船下,見有二白龍夾舫。



 既而至一山,峰崿聳秀,林樹繁密,意甚悅之。及高祖克京城,問何無忌曰:「急須一府主簿,何由得之?」無忌曰:「無過劉道民。」高祖曰:「吾亦識之。」即馳信召焉。時穆之聞京城有叫噪之聲,晨起出陌頭,屬與信會。穆之直視不言者久之。既而反室,壞布裳為褲,往見高祖。高祖謂之曰:「我始舉大義,方造艱難,須一軍吏甚急,卿謂誰堪其選?」穆之曰:「貴府始建,軍吏實須其才,倉卒之際,當略無見踰
 者。」高祖笑曰:「卿能自屈,吾事濟矣。」即於坐受署。



 從平京邑,高祖始至,諸大處分,皆倉卒立定,並穆之所建也。遂委以腹心之任,動止咨焉;穆之亦竭節盡誠,無所遺隱。時晉綱寬弛,威禁不行,盛族豪右,負勢陵縱,小民窮蹙,自立無所。重以司馬元顯政令違舛,桓玄科條繁密。穆之斟酌時宜,隨方矯正,不盈旬日,風俗頓改。遷尚書祠部郎,復為府主簿,記室錄事參軍,領堂邑太守。以平桓玄功,封西華縣五等子。



 義熙三年,揚州刺史王謐薨。高
 祖次應入輔,劉毅等不欲高祖入,議以中領軍謝混為揚州。或欲令高祖於丹徒領州,以內事付尚書僕射孟昶。遣尚書右丞皮沈以二議咨高祖。沈先見穆之,具說朝議。穆之偽起如廁,即密疏白高祖曰:「皮沈始至,其言不可從。」高祖既見沈,且令出外,呼穆之問曰:「卿云沈言不可從,其意何也?」穆之曰:「昔晉朝失政,非復一日,加以桓玄篡奪,天命已移。公興復皇祚,勳高萬古。既有大功,便有大位。位大勳高,非可持久。公今日形勢,豈得居謙
 自弱,遂為守籓之將邪?劉、孟諸公,與公俱起布衣,共立大義,本欲匡主成勳,以取富貴耳。事有前後,故一時推功,非為委體心服,宿定臣主之分也。力敵勢均,終相吞咀。揚州根本所係,不可假人。前者以授王謐,事出權道,豈是始終大計必宜若此而已哉!今若復以他授,便應受制於人。一失權柄,無由可得。而公功高勳重,不可直置,疑畏交加,異端互起,將來之危難,可不熟念。今朝議如此,宜相酬答,必云在我,厝辭又難。唯應云『神州治本,
 宰輔崇要,興喪所階,宜加詳擇。此事既大,非可懸論,便暫入朝,共盡同異。』公至京,彼必不敢越公更授餘人,明矣!」高祖從其言,由是入輔。



 從征廣固,還拒盧循,常居幕中畫策,決斷眾事。劉毅等疾穆之見親,每從容言其權重,高祖愈信仗之。穆之外所聞見,莫不大小必白,雖復閭里言謔,途陌細事,皆一二以聞。高祖每得民間委密消息以示聰明,皆由穆之也。又愛好賓遊,坐客恒滿,布耳目以為視聽,故朝野同異,穆之莫不必知。雖復親暱
 短長,皆陳奏無隱。人或譏之,穆之曰:「以公之明,將來會自聞達。我蒙公恩,義無隱諱,此張遼所以告關羽欲叛也。」高祖舉止施為,穆之皆下節度。高祖書素拙,穆之曰:「此雖小事,然宣彼四遠,願公小復留意。」高祖既不能厝意,又稟分有在。穆之乃曰:「便縱筆為大字,一字徑尺,無嫌。大既足有所包,且其勢亦美。」高祖從之,一紙不過六七字便滿。凡所薦達,不進不止,常云:「我雖不及荀令君之舉善,然不舉不善。」穆之與朱齡石並便尺牘,常於高
 祖坐與齡石答書。自旦至日中,穆之得百函,齡石得八十函,而穆之應對無廢也。轉中軍太尉司馬。八年,加丹陽尹。



 高祖西討劉毅,以諸葛長民監留府,總攝後事。高祖疑長民難獨任,留穆之以輔之。加建威將軍,置佐吏,配給實力。長民果有異謀,而猶豫不能發,乃屏人謂穆之曰:「悠悠之言,皆云太尉與我不平,何以至此?」穆之曰:「公溯流遠伐,而以老母稚子委節下,若一毫不盡,豈容如此邪?」意乃小安。高祖還,長民伏誅。



 十年,進穆之前將軍,
 給前軍府年布萬匹,錢三百萬。十一年,高祖西伐司馬休之,中軍將軍道憐知留任,而事無大小,一決穆之。遷尚書右僕射,領選,將軍、尹如故。十二年,高祖北伐,留世子為中軍將軍,監太尉留府,轉穆之左僕射,領監軍、中軍二府軍司,將軍、尹、領選如故。甲仗五十人,入殿。入居東城。



 穆之內總朝政,外供軍旅,決斷如流,事無擁滯。賓客輻輳,求訴百端,內外咨稟,盈階滿室,目覽辭訟,手答箋書,耳行聽受,口並酬應,不相參涉,皆悉贍舉。又數客暱賓,言
 談賞笑,引日亙時,未嘗倦苦。裁有閑暇,自手寫書,尋覽篇章,校定墳籍。性奢豪,食必方丈,旦輒為十人饌。穆之既好賓客,未嘗獨餐,每至食時,客止十人以還者,帳下依常下食,以此為常。嘗白高祖曰:「穆之家本貧賤,瞻生多闕。自叨忝以來,雖每存約損,而朝夕所須,微為過豐。自此以外,一毫不以負公。」十三年,疾篤,詔遣正直黃門郎問疾。十一月卒,時年五十八。



 高祖在長安,聞問驚慟,哀惋者數日。本欲頓駕關中,經略趙、魏。穆之既卒,京邑
 任虛,乃馳還彭城,以司馬徐羨之代管留任,而朝廷大事常決穆之者,並悉北諮。穆之前軍府文武二萬人,以三千配羨之建威府,餘悉配世子中軍府。追贈穆之散騎常侍、衛將軍、開府儀同三司。



 高祖又表天子曰:「臣聞崇賢旌善,王教所先;念功簡勞,義深追遠。故司勛秉策,在勤必書,德之休明,沒而彌著。故尚書左僕射、前將軍臣穆之,爰自布衣,協佐義始,內端謀猷,外勤庶政,密勿軍國,心力俱盡。及登庸朝右,尹司京畿,翼新王化,敷
 贊百揆。頃戎軍遠役,居中作扞,撫寄之勛,實洽朝野。方宣贊盛猷,緝隆聖世,志績示究,遠邇悼心。皇恩褒述,班同三事,榮哀兼備,寵靈已厚。臣伏思尋,自義熙草創,艱患未弭,外虞既殷,內難彌結,時屯世故,靡歲暫寧。豈臣以寡乏,負荷國重,實賴穆之匡翼之益。豈唯讜言嘉謀,益於民聽;若乃忠規遠畫,潛慮密謨,造膝詭辭,莫見其際。功隱於視聽,事隔於皇朝者,不可稱記。所以陳力一紀,克遂有成,出征入輔,幸不辱命,微夫人之左右,未有寧
 濟其事者矣。



 履謙居寡,守之彌固,每議及封賞,輒深自抑絕。所以勳高當年,而未沾茅社,撫事永傷,胡寧可昧。謂宜加贈正司,追甄土宇,俾大賚所及,永秩於善人,忠正之烈,不泯於身後。臣契闊屯泰,旋觀始終,金蘭之分,義深情密。是以獻其乃懷,布之朝聽。」於是重贈侍中、司徒,封南昌縣侯,食邑千五百戶。



 高祖受禪,思佐命元勛,詔曰:「故侍中、司徒南昌侯劉穆之,深謀遠猷,肇基王跡,勳造大業,誠實匪躬。今理運惟新,蕃屏並肇,感事懷人,
 實深心妻悼。



 可進南康郡公,邑三千戶。故左將軍、青州刺史王鎮惡,荊、郢之捷,克翦放命,北伐之勛,參跡方叔。念勤惟績,無忘厥心。可進龍陽縣侯,增邑千五百戶。」謚穆之曰文宣公。太祖元嘉九年,配食高祖廟庭;二十五年四月,車駕行幸江寧,經穆之墓,詔曰:「故侍中、司徒、南康文宣公穆之,秉德佐命,翼亮景業,謀猷經遠,元勳克茂,功銘鼎彞,義彰典策,故已嗣徽前哲,宣風後代者矣。近因遊踐,瞻其塋域,九原之想,情深悼歎。可致祭墓所,以
 申永懷。」



 穆之三子,長子慮之嗣,仕至員外散騎常侍卒。子邕嗣。先是,郡縣為封國者,內史、相並於國主稱臣,去任便止。至世祖孝建中,始革此制,為下官致敬。河東王歆之嘗為南康相,素輕邕。後歆之與邕俱豫元會,並坐。邕性嗜酒,謂歆之曰:「卿昔嘗見臣,今不能見勸一杯酒乎?」歆之因斅孫晧歌答之曰:「昔為汝作臣,今與汝比肩。既不勸汝酒,亦不願汝年。」邕所至嗜食瘡痂,以為味似鰒魚。嘗詣孟靈休,靈休先患灸瘡,瘡痂落床上,因取食
 之。靈休大驚。答曰:「性之所嗜。」



 靈休瘡痂未落者,悉褫取以飴邕。邕既去,靈休與何勖書曰:「劉邕向顧見啖,遂舉體流血。」南康國吏二百許人,不問有罪無罪,遞互與鞭,鞭瘡痂常以給膳。卒,子肜嗣。大明四年,坐刀砍妻,奪爵土,以弟彪紹封。齊受禪,降為南康縣侯,食邑千戶。



 穆之中子式之字延叔,通《易》好士。累遷相國中兵參軍,太子中舍人,黃門侍郎,寧朔將軍、宣城淮南二郡太守。在任贓貨狼藉,揚州刺史王弘遣從事檢校。



 從事呼攝吏民,
 欲加辨覆。式之召從事謂曰:「治所還白使君,劉式之於國家粗有微分,偷數百萬錢何有,況不偷邪!吏民及文書章之互在。」從事還具白弘,弘曰:「劉式之辯如此奔!」亦由此得停。還為太子右率,左衛將軍,吳郡太守。卒,追贈征虜將軍。從征關、洛有功,封德陽縣五等侯,謚曰恭侯。長子敳,世祖初,黃門侍郎。敳弟衍,大明末,以為黃門郎,出為豫章內史。晉安王子勛稱偽號,以為中護軍。事敗伏誅。



 衍弟瑀,字茂琳,少有才氣,為太祖所知。始與王濬為
 南徐州,以瑀補別駕從事史,為浚所遇。瑀性陵物護前,不欲人居己上。時浚征北府行參軍吳郡顧邁輕薄而有才能,浚待之甚厚,深言密事,皆與參之。瑀乃折節事邁,深布情款,家內婦女間事,言語所不得至者,莫不倒寫備說。邁以瑀與之款盡,深相感信。浚所言密事,悉以語瑀。瑀與邁共進射堂下,瑀忽顧左右索單衣幘,邁問其所以,瑀曰:「公以家人待卿,相與言無所隱,而卿於外宣洩,致使人無不知。我是公吏,何得不啟。」因而白之。浚
 大怒,啟太祖徙邁廣州。邁在廣州,值蕭簡為亂,為之盡力,與簡俱死。



 瑀遷從事中郎,領淮南太守。元嘉二十九年,出為寧遠將軍、益州刺史。元凶弒立,以為青州刺史。瑀聞問,即起義遣軍,并送資實於荊州。世祖即位,召為御史中丞。還至江陵,值南郡王義宣為逆,瑀陳其不可,言甚切至。義宣以為丞相左司馬,俱至梁山。瑀猶乘其蜀中船舫,又有義宣故部曲潛於梁山洲外下投官軍。除司徒左長史。明年,遷御史中丞。瑀使氣尚人,為憲司
 甚得志。彈王僧達云:「廕籍高華,人品冗末。」朝士莫不畏其筆端。尋轉右衛將軍。瑀願為侍中,不得,謂所親曰:「人仕宦不出當入,不入當出,安能長居戶限上。」因求益州。世祖知其此意,許之。孝建三年,除輔國將軍、益州刺史。既行,甚不得意。至江陵,與顏竣書曰:「朱修之三世叛兵,一旦居荊州,青油幙下,作謝宣明面見向,使齋帥以長刀引吾下席。於吾何有,政恐匈奴輕漢耳。」其年,坐奪人妻為妾,免官。大明元年,起為東陽太守。明年,遷吳興太
 守。侍中何偃嘗案云:「參伍時望。」瑀大怒曰:「我於時望何參伍之有!」遂與偃絕。及為吏部尚書,意彌憤憤。族叔秀之為丹陽尹,瑀又與親故書曰:「吾家黑面阿秀,遂居劉安眾處,朝廷不為多士。」



 其年,疽發背,何偃亦發背癰。瑀疾已篤,聞偃亡,歡躍叫呼,於是亦卒。謚曰剛子。子卷,南徐州別駕。卷弟藏,尚書左丞。



 穆之少子貞之,中書黃門侍郎,太子右衛率。寧朔將軍、江夏內史。卒官。子裒,始興相,以贓貨繫東冶內。穆之女適濟陽蔡祐,年老貧窮。世
 祖以祐子平南參軍孫為始安太守。



 王弘,字休元,琅邪臨沂人也。曾祖導,晉丞相。祖洽,中領軍。父珣,司徒。



 弘少好學,以清恬知名,與尚書僕射謝混善。弱冠,為會稽王司馬道子驃騎參軍主簿。時農務頓息,末役繁興,弘以為宜建屯田,陳之曰:「近面所咨立屯田事,已具簡聖懷。南畝事興,時不可失,宜早督田畯,以要歲功。而府資役單刻,控引無所,雖復厲以重勸,肅以嚴威,適足令囹圄充積,而無救於事實也。伏見南局諸冶,
 募吏數百,雖資以廩贍,收入甚微。愚謂若回以配農,必功利百倍矣。然軍器所須,不可都廢,今欲留銅官大冶及都邑小冶各一所,重其功課,一準揚州;州之求取,亦當無乏,餘者罷之,以充東作之要。又欲二局田曹,各立典軍募吏,依冶募比例,并聽取山湖人,此皆無損於私,有益於公者也。其中亦應疇量,分判番假,及給廩多少,自可一以委之本曹。親局所統,必當練悉,且近東曹板水曹參軍納之領此任,其人頗有幹能,自足了其事耳。
 頃年以來,斯務弛廢,田蕪廩虛,實亦由此。弘過蒙飾擢,志輸短效,豈可相與寢默,有懷弗聞邪!至於當否,尊自當裁以遠鑒。若所啟謬允者,伏願便以時施行,庶歲有務農之勤,倉有盈廩之實,禮節之興,可以垂拱待也。」道子欲以為黃門侍郎,珣以其年少固辭。



 珣頗好積聚,財物布在民間。珣薨,弘悉燔燒券書,一不收責;餘舊業悉以委付諸弟。未免喪,後將軍司馬元顯以為咨議參軍,加寧遠將軍,知記室事,固辭不就。道子復以為諮議參
 軍,加建威將軍,領中兵,又固辭。時內外多難,在喪者皆不終其哀,唯弘固執得免。桓玄克京邑。收道子付廷尉,臣吏畏恐,莫敢瞻送。弘時尚在喪,獨於道側拜,攀車涕泣,論者稱焉。



 高祖為鎮軍,召補咨議參軍。以功封華容縣五等侯,遷琅邪王大司馬從事中郎。



 出為寧遠將軍、琅邪內史,尚書吏部郎中,豫章相。盧循寇南康諸郡,弘奔尋陽。



 高祖復命為中軍咨議參軍,遷大司馬右長史,轉吳國內史。義熙十一年,徵為太尉長史,轉左長史。從
 北征,前鋒已平洛陽,而未遣九錫,弘銜使還京師,諷旨朝廷。



 時劉穆之掌留任,而旨反從北來,穆之愧懼,發病遂卒。而高祖還彭城,弘領彭城太守。



 宋國初建,遷尚書僕射領選,太守如故。奏彈謝靈運曰:「臣聞閑厥有家,垂訓《大易》,作威專戮,致誡《周書》,斯典或違,刑茲無赦。世子左衛率康樂縣公謝靈運,力人桂興淫其嬖妾,殺興江涘,棄尸洪流。事發京畿,播聞遐邇。宜加重劾,肅正朝風。案世子左衛率康樂縣公謝靈運過蒙恩獎,頻叨榮授,
 聞禮知禁,為日已久。而不能防閒閫闈,致茲紛穢,罔顧憲軌,忿殺自由。此而勿治,典刑將替。請以事見免靈運所居官,上臺削爵土,收付大理治罪。御史中丞都亭侯王準之,顯居要任,邦之司直,風聲噂𠴲,曾不彈舉。若知而弗糾,則情法斯撓;如其不知,則尸昧已甚。豈可復預班清階,式是國憲。請免所居官,以侯還散輩中。內臺舊體,不得用風聲舉彈,此事彰赫,曝之朝野,執憲蔑聞,群司循舊,國典既頹,所虧者重。臣弘忝承人乏,位副朝端,
 若復謹守常科,則終莫之糾正。所以不敢拱默,自同秉彞。違舊之愆,伏須準裁。」高祖令曰:「靈運免官而已,餘如奏。端右肅正風軌,誠副所期,豈拘常儀,自今為永制。」



 十四年,遷監江州豫州之西陽新蔡二郡諸軍事、撫軍將軍、江州刺史。至州,省賦簡役,百姓安之。永初元年,加散騎常侍。以佐命功,封華容縣公,食邑二千戶。三年,入朝,進號衛將軍,開府儀同三司。高祖因宴集,謂群公曰:「我布衣,始望不至此。」傅亮之徒並撰辭欲盛稱功德。弘率
 爾對曰:「此所謂天命,求之不可得,推之不可去。」時人稱其簡舉。



 少帝景平二年,徐羨之等謀廢立,召弘入朝。太祖即位,以定策安社稷,進位司空,封建安郡公,食邑千戶。上表固辭曰:「臣聞趙武稱隨會夫子之家事治,言於晉國無隱情。臣千載幸會,謬荷榮遇,雖以智能虛薄,政績蔑聞,而言無隱情,竊所庶幾。向令天啟其心,預定大策,而名編司勳,功不見紀,固將請不賞之罪,懸龍蛇之書,豈當稽違成命,茍修小節。但無功勤,暴之四海,進闕
 君子勞心之謀,退微小人勞力之效,而聖朝僭賞於上,愚臣茍忝於下,則為厚誣當時,永貽口實。



 竊財之誚,比此為輕,惟塵盛猷,虧玷為大。微躬所惜,一朝亦盡,非唯仰塵國紀,實亦俯畏友朋。憂心彌疹,胡顏靡託。且凡人之交,尚申知己,況在明主,可用理干。所以敢遂愚狷,守之以死。」乃見許。加使持節、侍中,改監為都督,進號車騎大將軍,開府、刺史如故。



 徐羨之等以廢弒之罪將見誅,弘既非首謀,弟曇首又為上所親委,事將發,密使報弘。
 羨之等誅,徵弘為侍中、司徒、揚州刺史,錄尚書,給班劍三十人。上西征謝晦,弘與驃騎彭城王義康居守,入住中書下省,引隊仗出入。司徒府權置參軍。



 五年春,大旱,弘引咎遜位,曰:「臣聞三才雖殊,其致則一。故世道休明,五福攸應;政有失德,咎徵必顯。臣抑又聞之,台輔之職,論道贊契,上佐人主,燮理陰陽。位以德授,則和氣淳穆;寇竊非據,則謫見於天。是以陳平有辭,不濫主者之局;邴吉停駕,大懼牛喘之由。斯固有國之所同,天人之遠
 旨。陛下聖哲御世,光隆中興,宜休徵表祥,醴泉毖涌。而頃陰陽隔并,亢旱成災,秋無嚴霜,冬無積雪,疾厲之氣,彌歷四時。此豈非任失其人,覆餗之咎。臣以庸短,自畢凡流,謬逢嘉運,叨恩在昔。陛下忘其不腆,又重之以今任。正位槐鼎,統理神州,珥貂衣袞,總錄朝端,內外要重,頓萃微躬,窮極寵貴,人臣莫比。令德居之,猶或難稱,矧伊陋昧,何以克任。此之易了,不俟明識。但受命之始,屬值時艱,六戎親戒,憂及社稷,誠是臣下致節忘身之時,當有
 何心,塵撓聖聽。所以僶俛從事,循牆馳驅,志在宣力,慮不及遠。既鯨鯢折首,西夏底定,便宜訴其本懷,避賢謝拙。



 而常人偷安,日甘一日,實亦仰佩天眷,未能自已。荏苒推遷,忽及三載。遂令負乘之釁,彰著幽明,愆伏之災,患纏氓庶。上缺皇朝緝熙之美,下增官謗覆折之災。



 伏念惶赧,五情飛散,雖曰厚顏,何以寧處。不遠而復,《大易》攸稱,小懲大戒,細人之福。近復之美,非所敢觖,懲戒之幸,竊懷庶幾。今履端惟始,朝慶禮畢,輒還私門,思愆家
 巷,庶微塞天譴,少弭謗讟。伏願鑒其所守,即而許之。臨啟愧塞,不自宣盡。」



 先是,彭城王義康為荊州刺史,鎮江陵。平陸令河南成粲與弘書曰:「僕聞軌物設教,必隨時制宜;世代盈虛,亦與之消息。夫勢之所處,非親不居。是以周之宗盟,異姓為後。權軸之要,任歸二南,斯前代之明謨,當今之顯轍。明公位極台鼎,四海具瞻;劬勞夙夜,義同吐握。而總錄百揆,兼牧畿甸,功實盛大,莫之與儔。天道福謙,宜存挹損。驃騎彭城王道德昭備,上之懿弟,
 宗本歸源,所應推先,宜出據列蕃,齊光魯、衛。明公高枕論道,燮理陰陽,則天下和平,災害不作;福慶與大宋升降,享年與松、喬齊久,名垂萬代,豈不美歟!」弘本有退志,挾粲言,由是固自陳請,乃降為衛將軍、開府儀同三司。



 六年,弘又上表曰:「臣聞異姓為後,宗周之明義;親不在外,有國之所先。



 故魯長滕君,《春秋》所美,楚出棄疾,前史垂誡。矧乃茂親明德,道光一時,述職侯甸,朝政弗及,而以庶族庸陋浮華之臣,超踰先典,居中贊契,豈所以憲
 章古式,緝熙治道?驃騎將軍臣義康,徽猷淵邈,明德彌劭,敷政江漢,化被荊南,搢紳屬情,想樂當務,周旦之寄,不謀同詞,分陜雖重,比此為輕。臣實空暗,階恩踰越,俯積素餐,仰玷盛化,公私二三,無一而可。昔孫叔未進,優孟見弓又;展季在下,臧文貽譏。況道隆地暱,義兼前禮。臣於古人,無能為役,負乘竊位,萬物謂何,雖曰厚顏,胡寧以處。斯亡之懼,實疚其心。乞解州錄,以允民望。伏願陛下遠存至公,近鑒丹款,俯順朝野,改授親賢。豈惟下臣,
 獲免大戾,凡厥眾隸,孰不慶幸。若天眷罔已,脫復遲回,請出臣表,逮聞外內,朝議輿誦,或有可擇。」



 詔曰:「省表,遠擬隆周經國之體,近述《大易》卑牧之志,三復沖旨,良用憮然。



 公體道淵虛,明識經遠,毗翼艱難,勳猷光茂,俾朕獲辰居垂拱,司契委成。豈容高遜總錄,固辭神州,使成務有虧,以重朕之不德邪!深存禮國,所望夤亮。驃騎親賢之寄,地均旦、奭,還入內輔,參贊機務,輒敬從所執。」義康由是代弘為司徒,與之分錄。



 弘又表曰:「近冒表聞,披
 陳愚管,實冀天鑒,體其至誠。而奉被還詔,未蒙酬察,徒塵聖覽,仰延優旨,顧影慚惶,罔識攸厝。臣忝荷要重,四載於今。既違前史量力之誡,又微古人進賢之美,尸位固寵,日積官謗,旋觀周行,興愧已後。



 況在親賢,朝野歸德,甫思引身,曷云能補,惟塵大典,虧喪已多。不悟天眷之隆,復垂恩獎,名器弗改,蒙寵如舊,感遇自揆,茫若無涯。臣義康既總錄百揆,毗贊盛化,忝廁下風,咨憑有所。內朝細務,庶可免竭,神州任重,望實兼該,臣何人斯,寇
 竊不已。為爾推遷,覆敗將及,就無人事之愆,必有陰陽之患。伏念惟憂,疢如疾首,不知何理,可以自安。但成旨已決,渙汗難反,加臣懦劣,少無此志,進不能抗言陳辭,以死自固;退不能重繭置冰,鮮食為瘠。祗畏天威,遂復俯仰。



 至於攝督所部,料綜文案,曹局吏役,所須不多,其餘文武,皆為冗長。相府初建,或有未充,請留職僚同事而已,自此以外,及諸資實,一送司徒。臣受恩深重,休戚是預,義無虛飾,茍自貶損。伏願聖察,特垂許順,不令誠
 訴,其見抑奪。」上又詔曰:「衛軍表如此,司徒宜須事力,可順公雅懷,割二千人配府,資儲不煩事送。」



 弘博練治體,留心庶事,斟酌時宜,每存優允。與八座丞郎疏曰:「同伍犯法,無士人不罪之科。然每至詰謫,輒有請訴。若垂恩宥,則法廢不可行;依事糾責,則物以為苦怨。宜更為其制,使得憂苦之衷也。又主守偷五匹,常偷四十匹,並加大辟,議者咸以為重,宜進主守偷十匹、常偷五十匹死,四十匹降以補兵。既得少寬民命,亦足以有懲也。想各言
 所懷。」



 左丞江奧議:「士人犯盜贓不及棄市者,刑竟,自在贓汙淫盜之目,清議終身,經赦不原。當之者足以塞愆,聞之者足以鑒誡。若復雷同群小,謫以兵役,愚謂為苦。符伍雖比屋鄰居,至於士庶之際,實自天隔,舍藏之罪,無以相關。奴客與符伍交接,有所藏蔽,可以得知,是以罪及奴客。自是客身犯愆,非代郎主受罪也。



 如其無奴,則不應坐。」右丞孔默之議:「君子小人,既雜為符伍,不得不以相檢為義。士庶雖殊,而理有聞察,譬百司居上,所
 以下不必躬親而後同坐。是故犯違之日,理自相關。今罪其養子、典計者,蓋義存戮僕。如此,則無奴之室,豈得宴安!但既云復士,宜令輸贖。常盜四十匹,主守五匹,降死補兵,雖大存寬惠,以紓民命。然官及二千石及失節士大夫,時有犯者,罪乃可戮,恐不可以補兵也。謂此制可施小人,士人自還用舊律。」



 尚書王準之議:「昔為山陰令,士人在伍,謂之押符。同伍有愆,得不及坐,士人有罪,符伍糾之。此非士庶殊制,實使即刑當罪耳。夫束脩之胄,
 與小人隔絕,防檢無方,宜及不逞之士,事接群細,既同符伍,故使糾之。於時行此,非唯一處。



 左丞議奴客與鄰伍相關,可得檢察,符中有犯,使及刑坐。即事而求,有乖實理。



 有奴客者,類多使役,東西分散,住家者少。其有停者,左右驅馳,動止所須,出門甚寡,典計者在家十無其一。奴客坐伍,濫刑必眾,恐非立法當罪本旨。右丞議士人犯偷,不及大辟者,宥補兵。雖欲弘士,懼無以懲邪。乘理則君子,違之則小人。制嚴於上,猶冒犯之,以其宥科,
 犯者或眾。使畏法革心,乃所以大宥也。且士庶異制,意所不同。」



 殿中郎謝元議謂:「事必先正其本,然後其末可理。本所以押士大夫於符伍,而所以檢小人邪?可使受檢於小人邪?士犯坐奴,是士庶天隔,則士無弘庶之由,以不知而押之於伍,則是受檢於小人也。然則小人有罪,士人無事,僕隸何罪,而令坐之。若以實相交關,貴其聞察,則意有未因。何者?名實殊章,公私異令,奴不押符,是無名也。民乏資財,是私賤也,以私賤無名之人,豫公家
 有實之任,公私混淆,名實非允。由此而言,謂不宜坐。還從其主,於事為宜。無奴之士,不在此例。若士人本檢小人,則小人有過,已應獲罪,而其奴則義歸戮僕,然則無奴之士,未合宴安,使之輸贖,於事非謬。二科所附,惟制之本耳。此自是辨章二本,欲使各從其分。至於求之管見,宜附前科,區別士庶,於義為美。盜制,按左丞議,士人既終不為兵革,幸可同寬宥之惠;不必依舊律,於議咸允。」



 吏部郎何尚之議:「按孔右丞議,士人坐符伍為罪,有
 奴罪奴,無奴輸贖。既許士庶緬隔,則聞察自難,不宜以難知之事,定以必知之法。夫有奴不賢,無奴不必不賢。今多僮者傲然於王憲,無僕者怵迫於時網,是為恩之所霑,恒在程、卓;法之所設,必加顏、原,求之鄙懷,竊所未愜。謝殿中謂奴不隨主,於名分不明,誠是有理。然奴僕實與閭里相關,今都不問,恐有所失。意同左丞議。」



 弘議曰:「尋律令既不分別士庶,又士人坐同伍罹謫者,無處無之,多為時恩所宥,故不盡親謫耳。吳及義興適有許、
 陸之徒,以同符合給,二千石論啟丹書。



 己未間,會稽士人云十數年前,亦有四族坐此被責,以時恩獲停。而王尚書云人舊無同伍坐,所未之解。恐蒞任之日,偶不值此事故邪。聖明御世,士人誠不憂至苦,然要須臨事論通,上干天聽為紛擾,不如近為定科,使輕重有節也。又尋甲符制,蠲士人不傳符耳,令史復除,亦得如之。共相押領,有違糾列,了無等衰,非許士人閭里之外也。諸議云士庶緬絕,不相參知,則士人犯法,庶民得不知。若庶
 民不許不知,何許士人不知。小民自非超然簡獨,永絕塵秕者,比門接棟,小以為意,終自聞知,不必須日夕來往也。右丞百司之言,粗是其況。如衰陵士人,實與里巷關接,相知情狀,乃當於冠帶小民。今謂之士人,便無小人之坐;署為小民,輒受士人之罰。於情於法,不其頗歟?且都令不及士流,士流為輕,則小人令使徵預其罰,便事至相糾,閭伍之防,亦為不同。謂士人可不受同伍之謫耳,罪其奴客,庸何傷邪?無奴客,可令輸贖,又或無奴
 僮為眾所明者,官長二千石便當親臨列上,依事遣判。又主偷五匹、常偷四十匹,謂應見優量者,實以小吏無知,臨財易昧,或由疏慢,事蹈重科,求之於心,常有可愍,故欲小進匹數,寬其性命耳。至於官長以上,荷蒙祿榮,付以局任,當正己明憲,檢下防非,而親犯科律,亂法冒利,五匹乃已為弘矣。士人無私相偷四十匹理,就使至此,致以明罰,固其宜耳,並何容復加哀矜。且此輩士人,可殺不可謫,有如諸論,本意自不在此也。近聞之道路,聊欲共論,
 不呼乃爾難精。既眾議糾紛,將不如其已。若呼不應停寢,謂宜集議奏聞,決之聖旨。」太祖詔:「衛軍議為允。」



 弘又上言:「舊制,民年十三半役,十六全役。當以十三以上,能自營私及公,故以充役。而考之見事,猶或未盡。體有彊弱,不皆稱年。且在家自隨,力所能堪,不容過苦。移之公役,動有定科,循吏隱恤,可無其患,庸宰守常,已有勤劇,況值苛政,豈可稱言。乃有務在豐役,增進年齒,孤遠貧弱,其敝尤深。至令依寄無所,生死靡告,一身之切,逃竄
 求免,家人遠計,胎孕不育,巧避羅憲,實亦由之。



 今皇化惟新,四方無事,役召之宜,應存乎消息。十五至十六,宜為半,十七為全。」



 從之。



 其後,弘寢疾,弘表屢乞骸骨,上輒優詔不許。九年,進位太保,領中書監,餘如故。其年,薨,時年五十四。即贈太保、中書監,給節,加羽葆、鼓吹,增班劍為六十人,侍中、錄尚書、刺史如故。謚曰文昭公,配食高祖廟廷。其年,詔曰:「乃者三逆煽禍,實繁有徒,爰初遵養,暨於明罰,外虞內慮,實維艱難。故太保華容縣公弘、故
 衛將軍華、故左光祿大夫曇首,抱義懷忠,乃情同至,籌謀廟堂,竭盡智力,經營夷險,簡自朕心。國恥既雪,允膺茅土,而並執謙挹,志不命踰,故用佇朝典,將有後命。盛業不究,相係殞落,永懷傷歎,痛恨無已。弘可增封千戶,華、曇首封開國縣侯,食邑各千戶。護軍將軍建昌公彥之,深誠密謨,比蹤齊望,其復先食邑,以酬忠勛。」又詔:「聞王太保家便已匱乏,清約之美,同規古人。言念始終,情增悽歎。可賜錢百萬,米千斛。」



 世祖大明五年,車駕遊幸,
 經弘墓。下詔曰:「故侍中、中書監、太保、錄尚書事、揚州刺史華容文昭公弘,德猷光劭,鑒識明遠。故散騎常侍、左光祿大夫、太子詹事豫章文侯曇首,夙尚恬素,理心貞正。並綢繆先眷,契闊屯夷,內亮王道,外流徽譽。以國圖令勛,民思茂惠。朕薄巡都外,瞻覽墳塋,永言想慨,良深於懷。



 便可遣使致祭墓所。」



 弘明敏有思致,既以民望所宗,造次必存禮法,凡動止施為,及書翰儀體,後人皆依仿之,謂為王太保家法。雖歷任籓輔,不營財利,薨亡之
 後,家無餘業。而輕率少威儀,性又褊隘,人忤意者,輒面加責辱。少時嘗摴蒱公城子野舍,及後當權,有人就弘求縣,辭訴頗切。此人嘗以蒱戲得罪,弘詰之曰:「君得錢會戲,何用祿為!」答曰:「不審公城子野何在?」弘默然。



 子錫嗣。少以宰相子,起家為員外散騎,歷清職,中書郎,太子左衛率,江夏內史。高自位遇。太尉江夏王義恭當朝,錫箕踞大坐,殆無推敬。卒官。子僧亮嗣。



 齊受禪,降爵為侯,食邑五百戶。弘少子僧達,別有傳。弘弟虞,廷尉卿。虞子
 深,有美名,官至新安太守。虞弟抑,光祿大夫。抑弟孺,侍中。孺弟曇首,別有傳。



 弘從父弟練,晉中書令氏子也。元嘉中,歷顯官,侍中,度支尚書。練子釗,世祖大明中,亦經清職,黃門郎,臨海王子頊晉安王子勛征虜、前軍長史,左民尚書。太宗初,為司徒左長史。隨司徒建安王休仁出赭圻,時居母憂,加冠軍將軍。



 忤犯休仁,出為始興相。休仁恚之不已,太宗乃收付廷尉,賜死。



 史臣曰:晉綱弛紊,其漸有由。孝武守文於上,化不下及,
 道子昏德居宗,憲章墜矣。重之以國寶啟亂,加之以元顯嗣虐,而祖宗之遺典,群公之舊章,莫不葉散冰離,掃地盡矣。主威不樹,臣道專行,國典人殊,朝綱家異,編戶之命,竭於豪門,王府之蓄,變為私藏。由是禍基東妖,難結天下,蕩蕩然王道不絕者若綖。



 高祖一朝創義,事屬橫流,改亂章,布平道,尊主卑臣之義,定於馬棰之間。威令一施,內外從禁,以建武、永平之風,變太元、隆安之俗,此蓋文宣公之為也。為一代宗臣,配饗清廟,豈徒然哉!



\end{pinyinscope}