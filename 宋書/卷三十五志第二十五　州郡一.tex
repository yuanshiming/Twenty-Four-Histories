\article{卷三十五志第二十五 州郡一}

\begin{pinyinscope}

 揚州南徐
 州徐州
 南兗州兗州唐堯之世,置十有二牧,及禹平水土,更制九州,冀州堯都,土界廣遠,濟、河為兗州,海、岱為青州,海、岱及淮為徐州,淮、海為揚州,荊及衡陽為荊州,荊、河為豫州,華陽、黑水為梁州,黑水、西河為雍州。自虞至殷,無所改變。周氏既有天下,以徐并青,以梁并雍,分冀州之地以為幽、并。漢初又立徐、梁二州。



 武帝攘卻胡、越,開地斥境,南置交
 趾,北置朔方,改雍曰涼,改梁曰益,凡為十三州,而司隸部三輔、三河諸郡。東京無復朔方,改交趾曰交州,凡十二州;司隸所部如故。及三國鼎歭,吳得揚、荊、交三州,蜀得益州,魏氏猶得九焉。吳又分交為廣。魏末平蜀,又分益為梁。晉武帝太康元年,天下一統,凡十有六州。後又分涼、雍為秦,分荊、揚為江,分益為寧,分幽為平,而為二十矣。



 自夷狄亂華,司、冀、雍、涼、青、並、兗、豫、幽、平諸州一時淪沒,遺民南渡,並僑置牧司,非舊土也。江左又分荊為
 湘,或離或合,凡有揚、荊、湘、江、梁、益、交、廣,其徐州則有過半,豫州唯得譙城而已。及至宋世,分揚州為南徐,徐州為南兗,揚州之江西悉屬豫州;分荊為雍,分荊、湘為郢,分荊為司,分廣為越,分青為冀,分梁為南北秦。太宗初,索虜南侵,青、冀、徐、兗及豫州淮西,並皆不守;自淮以北,化成虜庭。於是於鐘離置徐州,淮陰為北兗,而青、冀二州治贛榆之縣。今志大較以大明八年為正,其後分派,隨事記列。內史、侯、相,則以昇明末為定焉。



 地理參差,其
 詳難舉,實由名號驟易,境土屢分,或一郡一縣,割成四五;四五之中,亟有離合,千回百改,巧歷不算,尋校推求,未易精悉。今以班固馬彪二志、太康元康定戶、王隱《地道》、晉世《起居》、《永初郡國》、何徐《州郡》及地理雜書,互相考覆。且三國無志,事出帝紀,雖立郡時見,而置縣不書。今唯以《續漢郡國》校《太康地志》,參伍異同,用相徵驗。自漢至宋,郡縣無移改者,則注云「漢舊」,其有回徙,隨源甄別。若唯云「某無」者,則此前皆有也。若不注置立,史闕也。


揚州刺史,前漢刺史未有所治
 \gezhu{
  它州同}
 ,後漢治歷陽,魏、晉治壽春,晉平吳治建業。成帝咸康四年,僑立魏郡
 \gezhu{
  別見}
 ,肥鄉
 \gezhu{
  別見}
 、元城
 \gezhu{
  漢舊縣,晉屬陽平}
 二縣,後省元城。又僑立廣川郡
 \gezhu{
  別見}
 ,領廣川一縣,宋初省為縣,隸魏郡。江左又立高陽
 \gezhu{
  別見}
 、堂邑二郡
 \gezhu{
  別見}
 ,高陽領北新城
 \gezhu{
  別見}
 、博陸
 \gezhu{
  博陸縣,霍光所封,而二漢無,晉屬高陽。}
 二縣。堂邑,領堂邑一縣,後省堂邑并高陽,又省高陽并魏郡,並隸揚州,寄治京邑。文帝元嘉十一年省,以其民併建康。孝建元年,分揚州之會稽、東陽、新安、永嘉、臨海
 五郡為東揚州。大明三年罷州,以其地為王畿,以南臺侍御史部諸郡,如從事之部傳焉,而東揚州直云揚州。八年,罷王畿,復立揚州,揚州還為東揚州。前廢帝永光元年,省東揚州并揚州。順帝升明三年,改揚州刺史曰牧。領郡十,領縣八十。戶一十四萬三千二百九十六,口一百四十五萬五千六百八十五。



 丹陽尹,秦鄣郡,治今吳興之故鄣縣。漢初屬吳國,吳王濞反敗,屬江都國。



 武帝元封二年,為丹陽郡,治今宣城
 之宛陵縣。晉武帝太康二年,分丹陽為宣城郡,治宛陵,而丹陽移治建業。元帝太興元年,改為尹。領縣八,戶四萬一千一十,口二十三萬七千三百四十一。



 建康令,本秣陵縣。漢獻帝建安十六年置縣,孫權改秣陵為建業。晉武帝平吳,還為秣陵。太康三年,分秣陵之水北為建業。愍帝即位,避帝諱,改為建康。



 秣陵令,其地本名金陵,秦始皇改。本治去京邑六
 十里,今故治村是也。晉安帝義熙九年,移治京邑,在鬥場。恭帝元熙元年,省揚州府禁防參軍,縣移治其處。



 丹楊令,漢舊縣。



 江寧令,晉武帝太康元年,分秣陵立臨江縣。二年,更名。



 永世令,吳分溧陽為永平縣,晉武帝太康元年更名。惠帝世,度屬義興,尋復舊。義興又有平陵縣,
 董覽《吳地誌》云:「晉分永世。」《太康》、《永寧地誌》並無,疑是江左立。文帝元嘉九年,以併永世、溧陽二縣。



 溧陽令,漢舊縣。吳省為屯田。晉武帝太康元年復立。



 湖熟令,漢舊縣。吳省為典農都尉。晉武帝太康元年復立。



 句容令,漢舊縣。



 會稽太守,秦立,治吳。漢順帝永建四年,分會稽為吳郡,會稽移治山陰。領縣十,戶五萬二千二百二十八,口三十四萬八千一十四。去京都水一千三百五十五,陸同。



 山陰令,漢舊縣。



 永興令,漢舊餘暨縣,吳更名。



 上虞令,漢舊縣。



 餘姚令,漢舊縣。



 剡令,漢舊縣。



 諸暨令,漢舊縣。



 始寧令,何承天志,漢末分上虞立。賀《續會稽記》云:「順帝永建四年,分上虞南鄉立。」《續漢志》無。《晉太康三年地志》有。



 句章令,漢舊縣。



 鄮令,漢舊縣。



 鄞令,漢舊縣。



 吳郡太守,分會稽立。孝武大明七年,度屬南徐。八年,復
 舊。領縣十二,戶五萬四百八十八,口四十二萬四千八百一十二。去京都水六百七十,陸五百二十。



 吳令,漢舊縣。



 婁令,漢舊縣。



 嘉興令,此地本名長水,秦改曰由拳。吳孫權黃龍四年,由拳縣生嘉禾,改曰禾興。孫皓父名和,又改名曰嘉興。



 海虞令,晉武帝太康四年,分吳縣之虞鄉立。



 海鹽令,漢舊縣。《吳記》云:「本名武原鄉,秦以為海鹽縣。」



 鹽官令,漢舊縣。《吳記》云:「鹽官本屬嘉興,吳立為海昌都尉治,此後改為縣。」非也。



 錢唐令,漢舊縣。



 富陽令,漢舊縣。本曰富春。孫權黃武四年,以為東安郡;七年,省。晉簡文鄭太后諱「春」,孝武改曰富陽。



 新城令,浙江西南名為桐溪,吳立為新城縣,後并桐廬。《晉太康地志》無。



 張勃云:「晉末立。」疑是太康末立,尋復省也。晉成帝咸和九年又立。



 建德令,吳分富春立。



 桐廬令,吳分富春立。



 壽昌令,吳分富春立。新昌縣,晉武帝太康元年更名。



 吳興太守,孫皓寶鼎元年,分吳、丹陽立。領縣十,戶四萬
 九千六百九,口三十一萬六千一百七十三。去京都水九百五十,陸五百七十。



 烏程令,漢舊縣,先屬吳。



 東遷令,晉武帝太康三年,分烏程立。後廢帝元徽四年,更名東安。順帝昇明元年復舊。



 武康令,吳分烏程、餘杭立永安縣,晉武帝太康元年更名。



 長城令,晉武帝太康三年,分烏程立。



 原鄉令,漢靈帝中平二年,分故鄣立。



 故鄣令,漢舊縣,先屬丹陽。



 安吉令,漢靈帝中平二年,分故鄣立。



 餘杭令,漢舊縣,先屬吳。



 臨安令,吳分餘杭為臨水縣,晉武帝太康元年更名。



 於潛令,漢舊縣,先屬丹陽。


淮南太守,秦立為九江郡,兼得廬江豫章。漢高帝四年,
 更名淮南國,分立豫章郡,文帝又分為廬江郡。武帝元狩元年,復為九江郡,治壽春縣。後漢徙治陰陵縣。魏復曰淮南,徙治壽春。晉武帝太康元年,復立歷陽
 \gezhu{
  別見}
 、當塗、逡道諸縣,二年,復立鐘離縣
 \gezhu{
  別見}
 ,並二漢舊縣也。三國時,江淮為戰爭之地,其間不居者各數百里,此諸縣並在江北淮南,虛其地,無復民戶。吳平,民各還本,故復立焉。其後中原亂,胡寇屢南侵,淮南民多南度。成帝初,蘇峻、祖約為亂於江淮,胡寇又大至,民南度江者轉多,乃於
 江南僑立淮南郡及諸縣。晉末,遂割丹陽之于湖縣為淮南境。宋孝武大明六年,以淮南郡并宣城,宣城郡徙治于湖。八年,復立淮南郡,屬南豫州。明帝泰始三年,還屬揚州。領縣六,戶五千三百六十二,口二萬五千八百四十。去京都水一百七十,陸一百四十。



 于湖令,晉武帝太康二年,分丹楊縣立,本吳督農校尉治。



 當塗令,晉成帝世,與逡道俱立為僑縣,晉末分于
 湖為境。



 繁昌令,漢舊名,本屬潁川。魏分潁川為襄城,又屬焉。晉亂,省襄城郡,以此縣屬淮南,割于湖為境。



 襄垣令,其地本蕪湖。蕪湖縣,漢舊縣。至于晉末,立襄垣縣,屬上黨。上黨民南過江,立僑郡縣,寄治蕪湖,後省上黨郡為縣,屬淮南。文帝元嘉九年,省上黨縣併襄垣。



 定陵令,漢舊名,本屬襄城,後割蕪湖為境。



 逡道令,漢作逡遒,晉作逡道,後分蕪湖為境。



 宣城太守,晉武帝太康元年,分丹陽立。領縣十,戶一萬一百二十,口四萬七千九百九十二。去京都水五百八十,陸五百。



 宛陵令,漢舊縣。



 廣德令,何志云:「漢舊縣。」《二漢志》並無,疑是吳所立。



 懷安令,吳立。



 寧國令,吳立。



 宣城令,漢舊縣。



 安吳令,吳立。



 涇令,漢舊縣。



 臨城令,吳立。



 廣陽令,漢舊縣曰陵陽,子明得仙於此縣山,故以為名。晉成帝杜皇后諱「陵」,咸康四年更名。



 石城令,漢舊縣。



 東陽太守,本會稽西部都尉,吳孫皓寶鼎元年立。領縣
 九,戶一萬六千二十二,口一十萬七千九百六十五。去京都水一千七百,陸同。



 長山令,漢獻帝初平二年,分烏傷立。



 太末令,漢舊縣。



 烏傷令。



 永康令,赤烏八年,分烏傷上浦立。



 信安令,漢獻帝初平三年,分太末立曰新安。晉武帝太康元年更名。



 吳寧令,漢獻帝興平二年,孫氏分諸暨立。



 豐安令,漢獻帝興平二年,孫氏分諸暨立。



 定陽令,漢獻帝建安二十三年,孫氏分信安立。



 遂昌令,孫權赤烏二年,分太末立曰平昌。晉武帝太康元年更名。



 臨海太守,本會稽東部都尉。前漢都尉治鄞,後漢分會稽為吳郡,疑是都尉徙治章安也。孫亮太平二年立。領縣五,戶三千九百六十一,口二萬四千二百二十六。



 去
 京都水二千一十九,陸同。



 章安令,《續漢志》:「故治,閩中地,光武更名。」《晉太康記》:「本鄞縣南之回浦鄉,漢章帝章和中立。」未詳孰是。



 臨海令,吳分章安立。



 始豐令,吳立曰始平,晉武帝太康元年更名。



 寧海令,何志,漢舊縣。按《二漢志》、《晉太康地志》無。



 樂安令,晉康帝分始豐立。



 永嘉太守,晉明帝太寧元年,分臨海立。領縣五,戶六千二百五十,口三萬六千六百八十。去京都水二千八百,陸二千六百四十。



 永寧令,漢順帝永建四年,分章安東甌鄉立,或云順帝永和三年立。



 安固令,吳立曰羅陽,孫皓改曰安陽;晉武帝太康元年更名。



 松陽令,吳立。



 樂成令,晉孝武寧康三年,分永寧立。



 橫陽令,晉武帝太康四年,以橫藇船屯為始陽,仍復更名。



 新安太守,漢獻帝建安十三年,孫權分丹陽立曰新都,晉武帝太康元年更名。



 領縣五,戶一萬二千五十八,口三萬六千六百五十一。去京都水一千八百六十,陸一千八百。



 始新令,孫權分歙立。



 遂安令,孫權分歙為新定縣,晉武帝太康元年更名。



 歙令,漢舊縣。



 海寧令,孫權分歙為休陽縣,晉武帝太康元年更名。分歙置諸縣之始,又分置黎陽,大明八年,省併海寧。



 黟令,漢舊縣。



 南徐州刺史,晉永嘉大亂,幽、冀、青、並、兗州及徐州之淮
 北流民,相率過淮,亦有過江在晉陵郡界者。晉成帝咸和四年,司空郗鑑又徙流民之在淮南者於晉陵諸縣,其徙過江南及留在江北者,並立僑郡縣以司牧之。徐、兗二州或治江北,江北又僑立幽、冀、青、並四州。安帝義熙七年,始分淮北為北徐,淮南猶為徐州。



 後又以幽、冀合徐,青、并合兗。武帝永初二年,加徐州曰南徐,而淮北但曰徐。



 文帝元嘉八年,更以江北為南兗州,江南為南徐州,治京口,割揚州之晉陵、兗州之九郡僑在江南者
 屬焉,故南徐州備有徐、兗、幽、冀、青、並、揚七州郡邑。



 《永初二年郡國志》又有南沛、南下邳、廣平、廣陵、盱眙、鐘離、海陵、山陽八郡。南沛、廣陵、海陵、山陽、盱眙、鐘離割屬南兗,南下邳併南彭城,廣平並南泰山。今領郡十七,縣六十三,戶七萬二千四百七十二,口四十二萬六百四十。去京都水二百四十,陸二百。


南東海太守
 \gezhu{
  東海郡別見}
 ,晉元帝初,割吳郡海虞縣之北境為東海郡,立郯、朐、利城三縣,而祝其、襄賁等縣寄治曲
 阿。穆帝永和中,郡移出京口,郯等三縣亦寄治於京。文帝元嘉八年立南徐,以東海為治下郡,以丹徒屬焉。郯、利城並為實土。《永初郡國》有襄賁
 \gezhu{
  別見}
 、祝其、厚丘
 \gezhu{
  並漢舊名}
 、西隰
 \gezhu{
  何江左立}



 四縣,文帝元嘉十二年,省厚丘併襄賁。何、徐無厚丘,餘與《永初郡國》同。其襄賁、祝其、西隰,是徐志後所省也。領縣六,戶五千三百四十二,口三萬三千六百五十八。



 郯令,漢舊名。文帝元嘉八年,分丹徒之峴西為境。



 丹徒令,本屬晉陵,古名硃方,後名谷陽,秦改曰丹徒。孫權嘉禾三年,改曰武進。晉武帝太康三年,復曰丹徒。



 武進令,晉武帝太康二年,分丹徒、曲阿立。



 毗陵令,宋孝武大明末,度屬此。



 朐令,漢舊名。晉江左僑立。宋孝武世,分郯西界為土。



 利城令,漢舊名。晉江左僑立。宋文帝世,與郡俱為
 實土。


南琅邪太守
 \gezhu{
  瑯邪郡別見}
 ,晉亂,琅邪國人隨元帝過江千餘戶,太興三年,立懷德縣。丹楊雖有琅邪相而無此地。成帝咸康元年,桓溫領郡,鎮江乘之蒲洲金城上,求割丹陽之江乘縣境立郡,又分江乘地立臨沂縣。《永初郡國》有陽都
 \gezhu{
  前漢屬城陽,後漢、《晉太康地志》屬瑯邪。}
 、費、即丘
 \gezhu{
  並別見}
 三縣,並割臨沂及建康為土。費縣治宮城之北。元嘉八年,省即丘並陽都。十五年,省費併建康、臨沂。孝武大明五年,省陽都
 併臨沂。今領縣二,戶二千七百八十九,口一萬八千六百九十七。去州水二百,陸一百;去京都水一百六十。



 臨沂令,漢舊名。前漢屬東海,後漢、晉屬琅邪。



 江乘令,漢舊縣。本屬丹陽,吳省為典農都尉。晉武帝太康元年復立。



 晉陵太守,吳時分吳郡無錫以西為毗陵典農校尉。晉武帝太康二年,省校尉,立以為毗陵郡,治丹徒,後復還毗陵。東海王越世子名毗,而東海國故食毗陵。永嘉五
 年,帝改為晉陵。始自毗陵徙治丹徒。太興初,郡及丹徒縣悉治京口,郗鑑復徙還丹徒。安帝義熙九年,復還晉陵。本屬揚州,文帝元嘉八年,度屬南徐。領縣六,戶一萬五千三百八十二,口八萬一百一十三。去州水一百七十五,陸同;去京都水四百,陸同。



 晉陵令,本名延陵,漢改曰毗陵,後與郡俱改。



 延陵令,晉武帝太康二年,分曲阿之延陵鄉立。



 無錫令,漢舊縣。吳省,晉武帝太康元年復立。



 南沙令,本吳縣司鹽都尉署。吳時名沙中。吳平後,立暨陽縣割屬之。晉成帝咸康七年,罷鹽署,立以為南沙縣。



 曲阿令,本名雲陽,秦始皇改曰曲阿。吳嘉禾三年,復曰雲陽。晉武帝太康二年,復曰曲阿。



 暨陽令,晉武帝太康二年,分無錫、毗陵立。



 義興太守,晉惠帝永興元年,分吳興之陽羨、丹陽之永世立。永世尋還丹陽。



 本揚州,明帝泰始四年,度南徐。領
 縣五,戶一萬三千四百九十六,口八萬九千五百二十五。去州水四百,陸同;去京都水四百九十,陸同。



 陽羨令,漢舊縣。



 臨津令,故屬陽羨,立郡分立。



 義鄉令,故屬長城、陽羨,立郡分立。



 國山令,故屬陽羨,立郡分立。



 綏安令,武帝永初三年,分宣城之廣德、吳興之故鄣、長城及陽羨、義鄉五縣立。


南蘭陵太守
 \gezhu{
  蘭陵郡別見}
 ,領縣二,戶一千五百九十三,口一萬六百三十四。


蘭陵令。
 \gezhu{
  別見}


承令
 \gezhu{
  別見}
 ,文帝元嘉十二年,以合鄉縣併承。《永初郡國》、何、徐並無合鄉縣。


南東莞太守
 \gezhu{
  東莞郡別見}
 ,《永初郡國》又有蓋縣
 \gezhu{
  別見}
 。領縣三,戶一千四百二十四,口九千八百五十四。


莒令。
 \gezhu{
  別
  見}


東莞令
 \gezhu{
  別見}
 ,文帝元嘉十二年,以蓋縣併此。



 姑幕令,漢舊名。



 臨淮太守,漢武帝元狩六年立,光武以並東海。明帝永平十五年,復分臨淮之故地為下邳郡。晉武帝太康元年,復分下邳之淮南為臨淮郡,治盱眙。江左僑立。



 《永初郡國》又有盱眙縣,何、徐無。領縣七,戶三千七百一十一,口二萬二千八百八十六。



 海西令,前漢屬東海,後漢、晉屬廣陵。



 射陽令,前漢屬臨淮,後漢屬廣陵,三國時廢,晉武帝太康元年復立。



 凌令,前漢屬泗水,後漢屬廣陵,三國時廢,晉武帝太康二年又立,屬廣陵。



 淮浦令,前漢屬臨淮,後漢屬下邳,《晉太康地志》屬廣陵。



 淮陰令,前漢屬臨淮,後漢屬下邳,《晉太康地志》屬廣陵。



 東陽令,前漢屬臨淮,後漢屬廣陵,《晉太康地志》屬臨淮。


長樂令,本長樂郡
 \gezhu{
  別見}
 ,並合為縣。


淮陵太守,本淮陵縣,前漢屬臨淮,後漢屬下邳,晉屬臨淮,惠帝永寧元年,以為淮陵國。《永初郡國》又有下相
 \gezhu{
  前漢屬臨淮,後漢屬下邳,《晉太康地志》屬臨淮。}
 、廣陽
 \gezhu{
  廣陽,漢高立為燕國,昭帝更名。光武省併上谷,和帝永元八年復立。魏、晉復為燕國。前漢廣陽縣,後漢無,晉復有此也。}
 二縣。今領縣三,戶一千九百五,口一萬六百三十。



 司吾令,前漢屬東海,後漢屬下邳,《晉太康地志》屬臨淮。後廢帝元徽五年五月,改名桐梧,順帝昇明元年復舊。



 徐令,前漢屬臨淮,後漢屬下邳,《晉太康地志》屬臨淮。



 陽樂令,漢舊名,本屬遼西。文帝元嘉十三年,以下相併陽樂。


南彭城太守
 \gezhu{
  彭城郡別見}
 ,江左僑立。晉明帝又立南下邳郡,
 成帝又立南沛郡。文帝元嘉中,分南沛為北沛,屬南兗,而南沛猶屬南徐。孝武大明四年,以二郡並併南彭城。領縣十二,戶一萬一千七百五十八,口六萬八千一百六十三。


呂令。
 \gezhu{
  別見}



 武原令,漢舊名。



 傅陽令,漢舊名。


蕃令
 \gezhu{
  別見}
 ,義旗初,免軍戶立遂誠縣。武帝永初元年,
 改從舊名。


薛令
 \gezhu{
  別見}
 ,義旗初,免軍戶為建熙縣。永初元年,改從舊名。



 開陽令,前漢屬東海,章帝建初五年屬琅邪。晉僑立,猶屬琅邪,安帝度屬彭城。



 杼秋令,漢舊名。



 洨令,前漢屬梁,後漢、晉屬沛。


下邳令
 \gezhu{
  別見}
 ,本屬南下邳。



 北凌令,本屬南下邳,二漢無,《晉太康地志》屬下邳。本名凌,而廣陵郡舊有凌縣,晉武帝太康二年,以下邳之凌縣非舊土而同名,改為北凌。


僮令
 \gezhu{
  別見}
 ,本屬南下邳。南下邳有良城縣
 \gezhu{
  別見}
 ,文帝元嘉十二年併僮。


南清河太守
 \gezhu{
  清河郡別見}
 ,領縣四,戶一千八百四十九,口七千四百四。


清河令。
 \gezhu{
  別見}


東武城令。
 \gezhu{
  別見}


繹幕令。
 \gezhu{
  別見}


貝丘令。
 \gezhu{
  別見}


南高平太守
 \gezhu{
  高平郡別見}
 ,《永初郡國》又有鉅野、昌邑二縣
 \gezhu{
  並漢舊名}
 。



 今領縣三,戶一千七百一十八,口九千七百三十一。


金鄉令。
 \gezhu{
  別見}



 湖陸令,前漢曰湖陵,漢章帝更名。


高平令
 \gezhu{
  別見}
 。文帝元嘉十八年,以鉅野併高平。


南平昌太守
 \gezhu{
  平昌郡別見}
 ,領縣四,戶二千一百七十八,口一萬一千七百四十一。


安丘令。
 \gezhu{
  別見}



 新樂令,二漢無,魏分平原為樂陵郡,屬冀州,而新樂縣屬焉。晉江左立樂陵郡及諸縣,後省,以新樂縣屬此。


東武令。
 \gezhu{
  別見}


高密令
 \gezhu{
  別見}
 ,江左立高密國,後為南高密郡。文帝元
 嘉十八年,省為高密縣,屬此。


南濟陰太守,二漢、晉屬兗州,前漢初屬梁國,景帝中六年,別為濟陰國,宣帝甘露二年,更名定陶國,後還曰濟陰。《永初郡國》又有句陽、定陶二縣
 \gezhu{
  並漢舊名}
 今領縣四,戶一千六百五十五,口八千一百九十三。


城武令。
 \gezhu{
  別見}



 冤句令,漢舊名。



 單父令,前漢屬山陽。



 城陽令,漢舊名。


南濮陽太守,本東郡,屬兗州。晉武帝咸寧二年,以封子允,以東不可為國名,東郡有濮陽縣,故曰濮陽國。濮陽,漢舊名也,允改封淮南,還曰東郡。趙王倫篡位,廢太孫臧為濮陽王,王尋廢,郡名遂不改。《永初郡國》又有鄄城縣。
 \gezhu{
  二漢屬濟陰,《晉太康地志》屬濮陽也。}
 今領縣二,戶二千二十六,口八千二百三十九。



 廩丘令,前漢及《晉太康地志》有廩丘縣,後漢無。文
 帝元嘉十二年,以鄄城併廩丘。



 榆次令,漢舊名,至晉屬太原。


南泰山太守
 \gezhu{
  泰山郡別見}
 ,《永初郡國》有廣平
 \gezhu{
  漢武帝征和二年,立為平干國。宣帝五鳳二年,改為廣平。光武建武十三年,省併鉅鹿。魏分鉅鹿、魏郡復為廣平。江左僑立郡,晉成帝咸康四年省,後又立。}
 ,寄治丹徒,領廣平、易陽
 \gezhu{
  易陽,二漢屬趙,《晉太康地志》屬廣平。}
 、曲周
 \gezhu{
  前漢屬廣平,作曲周。後漢屬鉅鹿。《晉太康地志》屬廣平,作曲梁。}
 三縣。文帝元嘉十八年,省廣平郡為廣平縣,屬南泰山。今領縣三,戶二千四百九十九,口一萬三千六百。


南城令。
 \gezhu{
  別見}


武陽令。
 \gezhu{
  別見}



 廣平令,前漢屬廣平,後漢屬鉅鹿,《太康地志》屬廣平。



 濟陽太守,晉惠分陳留為濟陽國。領縣二,戶一千二百三十二,口八千一百九十二。



 考城令,前漢曰甾,屬梁國,章帝更名,屬陳留。《太康地志》無。


鄄城令。
 \gezhu{
  別見}


南魯郡太守
 \gezhu{
  魯郡別見}
 ,又有樊縣。
 \gezhu{
  前漢屬東平,後漢、《晉太康地志》屬任城也。}
 今領縣二,戶一千二百一十一,口六千八百一十八。


魯令。
 \gezhu{
  別見}



 西安令,漢舊名,本屬齊郡。齊郡過江僑立,後省,以西安配此。文帝元嘉十八年,以樊併西安。《永初郡國》無西安縣。



 徐州刺史,後漢治東海郯縣,魏、晉、宋治彭城。明帝世,淮
 北沒寇,僑立徐州,治鐘離。泰豫元年,移治東海朐。後廢帝元徽元年,分南兗州之鐘離、豫州之馬頭,又分秦郡之頓丘、梁郡之穀熟、歷陽之酂,立新昌郡,置徐州,還治鐘離。



 今先列徐州舊郡於前,以新割係。舊領郡十二,縣三十四。戶二萬三千四百八十五,口十七萬五千九百六十七。今領郡三,縣九。彭城去京都水一千三百六十,陸一千。



 彭城太守,漢高立為楚國,宣帝地節元年,改為彭城郡;
 黃龍元年,又為楚國;章帝還為彭城。領縣五,戶八千六百二十七,口四萬一千二百三十一。



 彭城令,漢縣。



 呂令,漢舊縣。



 蕃令,漢舊縣,屬魯。晉惠帝元康中度。蕃音皮;漢末太傅陳蕃子逸為魯相,改音。



 薛令,漢舊縣,屬魯。晉惠帝元康中度。



 留令,漢舊縣。



 沛郡太守,秦泗水郡,漢高更名。舊屬豫州,江左改配。領縣三,戶五千二百九,口二萬五千一百七十。去州陸六十;去京都一千。



 蕭令,漢舊縣。



 相令,漢舊縣。



 沛令,漢舊縣。



 下邳太守,前漢本臨淮郡,武帝立,明帝改為下邳。晉武帝分下邳之淮南為臨淮,而下邳如故。領縣三;戶三千
 九十九,口一萬六千八十八。去州水二百,陸一百八十;去京都水一千一百六十,陸八百。



 下邳令,前漢屬東海,後漢、《晉太康地志》屬下邳。



 良成令,前漢屬東海,後漢、《晉太康地志》屬下邳。



 僮令,前漢屬臨淮,後漢、《晉太康地志》屬下邳。



 蘭陵太守,晉惠帝元康元年,分東海立。領縣三,戶三千一百六十四,口一萬四千五百九十七。去州陸二百;去京都水一千六百,陸一千三百。



 昌慮令,漢舊縣。



 承令,漢舊縣。



 合鄉令,漢舊縣。



 東海太守,秦郯郡,漢高更名。明帝失淮北,僑立青州於贛榆縣。泰始七年,又立東海縣屬東海郡,又割贛檢置鬱縣,立西海郡,並隸僑青州。領縣二,戶二千四百一十一,口一萬三千九百四十一。去州水一千,陸八百;去京都水一千,陸六百七十。



 襄賁令,漢舊縣。



 贛榆令,前漢屬琅邪,後漢屬東海。魏省,晉武帝太康元年復立。



 東莞太守,晉武帝泰始元年,分琅邪立。咸寧三年,復以合琅邪,太康十年復立。領縣三,戶八百八十七;口七千三百二十。去州陸七百。去京都水二千,陸一千四百。



 莒令,前漢屬城陽,後漢屬琅邪。孝武大明五年改為長。



 諸令,前漢屬城陽,後漢屬琅邪,《晉太康地志》屬城陽。



 東莞令,漢舊縣。



 東安太守,東安故縣名,前漢屬城陽,後漢屬琅邪,《晉太康地志》屬東莞,晉惠帝分東莞立。領縣三,戶一千二百八十五,口一萬七百五十五。去州陸七百;去京都陸一千三百。



 蓋令,前漢屬瑯邪,後漢屬太山,《晉太康地志》屬樂
 安。孝武大明五年改為長。



 新泰令,魏立,屬泰山。



 發干令,漢舊名,屬東郡。《太康地志》無,江左來配。



 琅邪太守,秦立。領縣二,戶一千八百一十八,口八千二百四十三。去州陸四百;去京都水一千五百,陸一千一百。



 費令,前漢屬東海,後漢屬泰山,《晉太康地志》屬琅邪。



 即丘令,前漢屬東海,後漢、《晉太康地志》屬琅邪。



 淮陽太守,晉安帝義熙中土斷立。領縣四,戶二千八百五十五,口一萬五千三百六十三。去州水六百,陸五百;去京都水七百,陸五百五十。



 角城令,晉安帝義熙中土斷立。



 晉寧令,故屬濟岷,流寓來配。



 宿預令,晉安帝立。



 上黨令,本流寓郡,併省來配。


陽平太守,陽平本縣名,屬東郡。魏分東郡及魏郡為陽平郡。故屬司州,流寓來配。《永初郡國》又有廩丘縣
 \gezhu{
  別置}
 。今領縣三,戶一千七百二十五,口一萬三千三百三十。



 館陶令,漢舊名。



 陽平令,漢舊名。



 濮陽令,本流寓郡,併省來配。



 濟陰太守,漢景帝立,屬兗州。流寓徐土,因割地為境。領縣三,戶二千三百五,口一萬一千九百二十八。



 睢陵令,前漢屬臨淮,後漢屬下邳。孝武大明元年度。



 定陶令,漢舊名。孝武大明五年改為長。



 頓丘令,屬頓丘,流寓割配。



 北濟陰太守,孝武孝建元年升立。領縣三,戶九百二十七,口三千八百十。



 城武令,前漢屬山陽,後漢、《晉太康地志》屬濟陰。



 豐令,漢舊名,屬沛。孝武大明元年復立。



 離狐令,前漢屬東郡,後漢、《晉太康地志》屬濟陰。



 鐘離太守,本屬南兗州,晉安帝分立。案漢九江郡、晉淮南郡有鐘離縣,即此地也。領縣三,戶三千二百七十二,口一萬七千八百三十二。去京都陸六百二十,水一千三十。


燕縣令
 \gezhu{
  別見}
 ,故屬東燕。流寓因配。



 朝歌令,本屬河內,晉武帝分河內為汲,又屬焉。流寓因配。



 樂平令,前漢曰清,屬東郡,章帝更名,《晉太康地志》無。流寓因配。



 馬頭太守,屬南豫州,故淮南當塗縣地,晉安帝立,因山形立名。領縣三,戶一千三百三十二,口一萬二千三百一十。去京都水一千七百五十,陸六百七十。



 虞縣令,漢舊名,屬梁郡。流寓因配。



 零縣令,晉安帝立。



 濟陽令,故屬濟陽。流寓因配。



 新昌太守,後廢帝元徽元年立。



 頓丘令,二漢屬東郡,魏屬陽平;晉武帝泰始二年,分淮陽置頓丘郡,頓丘縣又屬焉。江左流寓立,屬秦。先有沛縣,元嘉八年併頓丘,後廢帝元徽元年度屬此。



 穀熟令,前漢無,後漢、晉屬梁。《永初郡國》、何、徐志並屬南梁。後廢帝元徽元年度。



 酂令,漢屬沛,晉屬譙。文帝元嘉八年,自南譙度屬
 歷陽,後廢帝元徽元年度屬此。


南兗州刺史,中原亂,北州流民多南渡,晉成帝立南兗州,寄治京口。時又立南青州及並州,武帝永初元年,省并併南兗。文帝元嘉八年,始割江淮間為境,治廣陵。《永初郡國》領十四郡。南高平、南平昌、南濟陰、南濮陽、南泰山、濟陽、南魯山郡,今並屬徐州。又有東燕郡,江左分濮陽所立也,領燕縣
 \gezhu{
  前漢曰南燕,後漢曰燕,並屬東郡。《太康地志》屬濮陽。}
 、白馬、平昌、考城凡四縣。文帝元嘉十八年,省考城併燕。十九年,省
 東燕郡為東燕縣,屬南濮陽,後又省東燕郡。


南東平郡領范、蛇丘、歷城凡三縣。高密郡領淳于、黔陬、營陵、夷安凡四縣。南齊郡領安西、臨菑凡二縣。南平原郡領平原、高唐、茌平
 \gezhu{
  並別見}
 凡三縣。濟岷郡
 \gezhu{
  江左立}
 ,領營城、晉寧
 \gezhu{
  江左立}
 凡二縣。雁門郡
 \gezhu{
  漢舊郡}
 領樓煩
 \gezhu{
  別見}
 、陰館
 \gezhu{
  前漢作「觀」,後漢、晉作「館」也。}
 、廣武
 \gezhu{
  前漢屬太原,後漢、《晉太康地志》屬雁門也。}
 、崞、馬邑
 \gezhu{
  並漢舊名}
 凡五縣。凡七郡,二十三縣,並省屬南徐州。諸僑郡縣何志又有鐘離、雁門、平原、東平、北沛五郡。鐘離今屬徐州。雁門領樓煩、陰館、廣武三縣。平
 原領茌平、臨菑、營城、平原四縣。東平領范、朝陽、歷城三縣。北沛領符離、蕭、相、沛四縣。
 \gezhu{
  符離,漢舊縣。餘並別見。}


凡十四縣。《起居注》,元嘉十一年,以南兗州東平之平陸併範,壽張並朝陽,平原之濟岷、晉寧併營城
 \gezhu{
  先是,省濟岷郡為縣。}
 ,高唐並茌平。按此五縣,元嘉十一年所省,則平陸、壽張疑在《永初郡國志》,而無此二縣,未詳。徐志有南東平郡,領范、朝陽、歷城、樓煩、陰觀、廣武、茌平、營城、臨菑、平原十縣,則是雁門、平原併東平也。孝武大明五年,以東平併廣陵。宋又僑
 立新平、北淮陽、北濟陰、北下邳、東莞五郡。元嘉二十八年,南兗州徙治盱眙。三十年,省南兗州併南徐,其後復立,還治廣陵。徐志領郡九,縣三十九,戶三萬一千一百一十五,口十五萬九千三百六十二。宋末領郡十一,縣四十四。去京都水二百五十,陸一百八十。


廣陵太守,漢高六年立,屬荊國,十一年,更屬吳;景帝四年,更名江都國;武帝元狩三年,更名廣陵。舊屬徐州。晉武帝太康三年,治淮陰故城,後又治射陽
 \gezhu{
  射陽別見}
 。江左治
 廣陵。《永初郡國》又有輿
 \gezhu{
  前漢屬臨淮,後漢省臨淮屬廣陵,文帝元嘉十三年並江都也。}
 、肥如、潞、真定、新市五縣。
 \gezhu{
  並二漢舊名。}



 肥如屬遼西,潞屬上黨,真定前漢屬真定,後漢省真定屬常山,晉亦屬常山。新市二漢、晉屬中山。《永初郡國》云四縣本屬遼西,則是晉末遼西僑郡省併廣陵也。



 何有肥如、新市,徐與今同也。



 廣陵令,漢舊縣。



 海陵令,前漢屬臨淮,後漢、晉屬廣陵,三國時廢,晉武帝太康元年復立。



 高郵令,漢舊縣。三國時廢,晉武帝太康元年復立。



 江都令,漢舊縣。三國時廢,晉武帝太康六年復立。江左又省併輿縣,元嘉十三年復立,以並江都。



 海陵太守,晉安帝分廣陵立。《永初郡國》屬徐州。領縣六,戶三千六百二十六,口二萬一千六百六十。去州水一百三十,陸同;去京都水三百九十,陸同。



 建陵令,晉安帝立。



 臨江令,晉安帝立。



 如皋令,晉安帝立。



 寧海令,晉安帝立。



 蒲濤令,晉安帝立。



 臨澤令,明帝泰豫元年立。



 山陽太守,晉安帝義熙中土斷分廣陵立。案漢景帝分梁為山陽,非此郡也。



 《永初郡國》屬徐州。領縣四,戶二千八百一十四,口二萬二千四百七十。去州水三百,陸同;去京都水五百,陸同。



 山陽令,射陽縣境,地名山陽,與郡俱立。



 鹽城令,舊曰鹽瀆,前漢屬臨淮,後漢、晉屬廣陵;三國時廢,晉武帝太康二年復立。晉安帝更名。



 東城令,晉安帝立。



 左鄉令,晉安帝立。



 盱眙太守,盱眙本縣名,前漢屬臨淮,後漢屬下邳,晉屬臨淮,晉安帝分立。



 領縣五,戶一千五百一十八,口六千八百二十五。去州水四百九十,陸二百九;去京都水七
 百,陸五百。


考城令。
 \gezhu{
  別見}



 陽城令,晉安帝立。



 直瀆令,晉安帝立。



 信都令,信都雖漢舊名,其地非也。地在河北,宋末立。



 睢陵令,前漢屬臨淮,後漢屬下邳,《晉太康地志》無。宋末立。


秦郡太守,晉武帝分扶風為秦國,中原亂,其民南流,寄居堂邑。堂邑本為縣,前漢屬臨淮,後漢屬廣陵,晉又屬臨淮。晉惠帝永興元年,分臨淮淮陵立堂邑郡,安帝改堂邑為秦郡。《永初郡國》屬豫州,元嘉八年度南兗。《永初郡國》又領臨塗
 \gezhu{
  晉、宋立}
 、平丘
 \gezhu{
  漢舊,屬陳留,《晉太康地志》無。}
 、外黃
 \gezhu{
  漢舊名,屬陳留。}
 、沛、雍丘、浚儀、頓丘
 \gezhu{
  別見}
 凡七縣。何無雍丘、外黃、平丘、沛,徐又無浚儀。元嘉八年,以沛併頓丘。後廢帝元徽元年,割頓丘屬新昌。領縣四,戶三千三百三十三,口一萬五千
 二百九十六。去州水二百四十一,陸一百八十;去京都水一百五十,陸一百四十。



 秦令,本屬秦國,流寓立。文帝元嘉八年,以臨塗併秦,以外黃並浚儀。孝武孝建元年,以浚儀並秦。



 義成令,江左立。



 尉氏令,漢舊名,屬陳留。文帝元嘉八年,以平丘併尉氏。



 懷德令,孝武大明五年立。又以歷陽之烏江,并此
 為二縣,立臨江郡。前廢帝永光元年,省臨江郡。懷德即住郡治,烏江還本也。


南沛太守
 \gezhu{
  沛郡別見}
 ,何志云,北沛新立;徐云南沛。《永初郡國》又有符離、洨
 \gezhu{
  並別見}
 、竹邑
 \gezhu{
  前漢曰竹。李奇曰,今邑也。後漢曰竹邑。至晉並屬沛。}


杼秋
 \gezhu{
  前漢屬梁,後漢、《晉太康地志》屬沛。}
 四縣。杼秋治無錫,餘並治廣陵。



 文帝元嘉十二年,以北沛郡竹邑并杼秋,何、徐並無此二縣,不詳。《起居注》,孝武大明五年,分廣陵為沛郡,治肥如縣。時無復肥如縣,當是肥如故縣處也。二漢、《晉太康地志》
 並無肥如縣。沛郡宜是大明五年以前省,其時又立也。今領縣三,戶一千一百九,口一萬二千九百七十。


蕭縣令。
 \gezhu{
  別見}


相縣令。
 \gezhu{
  別見}


沛縣令。
 \gezhu{
  別見}



 新平太守,明帝泰始七年立。



 江陽令。郡同立。



 海安令。郡同立。



 北淮陽太守,宋末僑立。


晉寧令。
 \gezhu{
  別見}


宿預令。
 \gezhu{
  別見}


角城令。
 \gezhu{
  別見}


北濟陰太守
 \gezhu{
  濟陰郡別見}
 ,宋失淮北僑立。



 廣平令,前漢臨淮有廣平縣,後漢以後無。


定陶令。
 \gezhu{
  別見}


陽平令。
 \gezhu{
  別
  見}


上黨令。
 \gezhu{
  別見}


冤句令。
 \gezhu{
  別見}


館陶令。
 \gezhu{
  別見}


北下邳太守
 \gezhu{
  下邳郡別見}
 ,宋失淮北僑立。


僮縣令。
 \gezhu{
  別見}


下邳令。
 \gezhu{
  別見}


寧城令。
 \gezhu{
  別見}


東莞太守
 \gezhu{
  東莞郡別見}
 ,宋失淮北僑立。


莒縣令。
 \gezhu{
  別見}


諸縣令。
 \gezhu{
  別見}


東莞令。
 \gezhu{
  別見}



 柏人令,漢舊名,屬趙國。宋失淮北僑立。


兗州刺史,後漢治山陽昌邑,魏、晉治廩丘;武帝平河南,治滑臺;文帝元嘉十三年,治鄒山,又寄治彭城。二十年,省兗州,分郡屬徐、冀州。三十年六月復立,治瑕丘。
 \gezhu{
  二漢山陽有瑕丘縣}
 。《永初郡國》有東郡、陳留、濮陽三郡,而無陽平。東郡領
 白馬
 \gezhu{
  別見}
 、涼城
 \gezhu{
  二漢東郡有聊城縣,《晉太康地志》無,疑此是。}
 、東燕
 \gezhu{
  別見}
 三縣。陳留郡領酸棗
 \gezhu{
  漢舊縣}
 、小黃、雍丘、白馬、襄邑、尉氏六縣。
 \gezhu{
  郡縣並別見。}
 濮陽郡領濮陽、廩丘
 \gezhu{
  並別見}
 二縣。宋末失淮北,僑立兗州,寄治淮陰
 \gezhu{
  淮陰別見}
 。兗州領郡六,縣三十一,戶二萬九千三百四十,口一十四萬五千五百八十一。


泰山太守,漢高立。《永初郡國》又有山茌
 \gezhu{
  別見}
 、萊蕪
 \gezhu{
  漢舊名}
 、太原
 \gezhu{
  本郡,僑立此縣}
 三縣,而無鉅平縣。今領縣八,戶八千一百七十七,口四萬五千五百八十一。去州陸八百;去京都陸
 一千八百。



 奉高令,漢舊縣。



 鉅平令,漢舊縣。



 嬴令,漢舊縣。



 牟令,漢舊縣。



 南城令,前漢屬東海,後漢、晉屬泰山。



 武陽令,漢舊縣。



 梁父令,漢舊縣。



 博令,漢舊縣。


高平太守,故梁國,漢景帝中六年,分為山陽國;武帝建元五年為郡;晉武帝泰始元年更名。《永初郡國》及徐並又有任城縣
 \gezhu{
  前漢屬東平,章帝元和元年,分東平為任城,又屬焉。晉亦屬任城。江左省郡為縣也。}
 ,後省。今領縣六,戶六千三百五十八,口二萬一千一百一十二。去州陸二百二十;去京都陸一千三百三十。



 宋明帝泰始五年,僑立於淮南當塗縣界,領高平、金鄉二縣。其年,又立睢陵縣。



 高平令,前漢名稿,章帝更名。



 方與令,漢舊縣。



 金鄉令,前漢無,後漢、晉有。



 鉅野令,漢舊縣。



 平陽令,漢舊縣。曰南平陽。



 亢父令,漢舊縣。舊屬任城。



 魯郡太守,秦薛郡,漢高后更名。本屬徐州,光武改屬豫州,江左屬兗州。領縣六,戶四千六百三十一,口二萬八
 千三百七。去州陸三百五十;去京都陸一千一百。



 鄒令,漢舊縣。



 汶陽令,漢舊縣。



 魯令,漢舊縣。



 陽平令,孝武大明元年立。



 新陽令,孝武大明中立。



 卞令,明帝泰始二年立。



 東平太守,漢景帝分梁為濟東國,宣帝更名。領縣五,戶
 四千一百五十九,口一萬七千二百九十五。去州水五百,陸同;去京都水二千,陸一千四百。宋末又僑立於淮陰。



 無鹽令,漢舊縣。



 平陸令,漢舊縣。



 須昌令,前漢屬東郡,後漢、《晉太康地志》屬東平。



 壽昌令,春秋時曰良,前漢曰壽良,屬東郡;光武改曰壽張,屬東平。



 范令,漢舊縣。四縣並治郡下。



 陽平太守,魏分魏郡立。文帝元嘉中,流寓來屬,後省,孝武大明元年復立。



 領縣五,戶二千八百五十七,口一萬一千二百七十一。



 館陶令,漢舊名,寄治無鹽。



 樂平令,魏立,屬陽平。後漢東郡有樂平,非也。寄治下平陸。



 元城令,漢舊。寄治無鹽。


平原令
 \gezhu{
  別見}
 ,孝武大明中立。


頓丘令
 \gezhu{
  別見}
 ,孝武大明中立。


濟北太守,漢和帝永元二年,分泰山立。《永初郡國》有臨邑
 \gezhu{
  二漢屬東郡,《晉太康地志》屬濟北。}
 、東阿
 \gezhu{
  二漢屬東郡,晉無。}
 二縣,孝武大明元年省,應在何志而無,未詳。領縣三,戶三千一百五十八,口一萬七千三。去州陸七百;去京都水二千,陸一千五百。宋末又僑立於淮陽。



 蛇丘令,前漢屬泰山,後漢、《晉太康地志》屬濟北。



 盧令,前漢屬泰山,後漢、《晉太康地志》屬濟北。



 穀城令,前漢無,後漢屬東郡,《晉太康地志》屬濟北。



\end{pinyinscope}