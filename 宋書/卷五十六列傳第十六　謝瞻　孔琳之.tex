\article{卷五十六列傳第十六 謝瞻 孔琳之}

\begin{pinyinscope}

 謝瞻,字宣遠,一名簷,字通遠,陳郡陽夏人,衛將軍晦第三兄也。年六歲,能屬文,為《紫石英贊》、《果然詩》,當時才士,莫不歎異。初為桓偉安西參軍,楚臺秘書郎。瞻幼孤,叔
 母劉撫養有恩紀,兄弟事之,同於至親。劉弟柳為吳郡,將姊俱行,瞻不能違,解職隨從,為柳建威長史。尋為高祖鎮軍、琅邪王大司馬參軍,轉主簿,安成相,中書侍郎,宋國中書、黃門侍郎,相國從事中郎。



 弟晦時為宋臺右衛,權遇已重,於彭城還都迎家,賓客輻輳,門巷填咽。時瞻在家,驚駭謂晦曰:「汝名位未多,而人歸趣乃爾。吾家以素退為業,不願干預時事,交遊不過親朋,而汝遂勢傾朝野,此豈門戶之福邪?」乃籬隔門庭,曰:「吾不忍見此。」
 及還彭城,言於高祖曰:「臣本素士,父、祖位不過二千石。弟年始三十,志用凡近,榮冠臺府,位任顯密,福過災生,其應無遠。特乞降黜,以保衰門。」前後屢陳。高祖以瞻為吳興郡,又自陳請,乃為豫章太守。晦或以朝廷密事語瞻,瞻輒向親舊陳說,以為笑戲,以絕其言。晦遂建佐命之功,任寄隆重,瞻愈憂懼。



 永初二年,在郡遇疾,不肯自治,幸於不永。晦聞疾奔往,瞻見之,曰:「汝為國大臣,又總戎重,萬里遠出,必生疑謗。」時果有訴告晦反者。瞻疾篤
 還都,高祖以晦禁旅,不得出宿,使瞻居於晉南郡公主婿羊賁故第,在領軍府東門。瞻曰:「吾有先人弊廬,何為於此!」臨終,遣晦書曰:「吾得啟體幸全,歸骨山足,亦何所多恨。弟思自勉厲,為國為家。」遂卒,時年三十五。



 瞻善於文章,辭採之美,與族叔混、族弟靈運相抗。靈運父瑛,無才能。為祕書郎,早年而亡。靈運好臧否人物,混患之,欲加裁折,未有方也。謂瞻曰:「非汝莫能。」乃與晦、曜、弘微等共遊戲,使瞻與靈運共車;靈運登車,便商較人物,瞻謂之曰:「祕書早
 亡,談者亦互有同異。」靈運默然,言論自此衰止。



 弟晙,字宣鏡,幼有殊行。年數歲,所生母郭氏,久嬰痼疾,晨昏溫清,嘗藥捧膳,不闕一時,勤容戚顏,未嘗暫改。恐僕役營疾懈倦,躬自執勞。母為病畏驚,微踐過甚,一家尊卑,感㬭至性,咸納屢而行,屏氣而語,如此者十餘年。初為州主簿,中軍行參軍,太子舍人,俄遷秘書丞。自以兄居權貴,己蒙超擢,固辭不就。徐羨之請為司空長史,黃門郎。元嘉三年,從坐伏誅,時年三十一。有詔宥其子世平,又
 早卒,無後。



 孔琳之,字彥琳,會稽人。祖沈,晉丞相掾。父曌,光祿大夫。琳之彊正有志力,好文義,解音律,能彈棋,妙善草隸。郡命主簿,不就,後辟本國常侍。桓玄輔政為太尉,以為西閣祭酒。桓玄時議欲廢錢用穀帛,琳之議曰:「《洪範》八政,以貨次食,豈不以交易之所資,為用之至要者乎?若使不以交易,百姓用力於為錢,則是妨其為生之業,禁之可也。今農自務穀,工自務器,四民各肄其業,何嘗致勤於錢。故聖王制無用之
 貨,以通有用之財,既無毀敗之費,又省運置之苦,此錢所以嗣功龜貝,歷代不廢者也。穀帛為寶,本充衣食,今分以為貨,則致損甚多。又勞毀於商販之手,耗棄於割截之用,此之為敝,著於自曩。故鐘繇曰:『巧偽之民,競蘊濕穀以要利,制薄絹以充資。』魏世制以嚴刑,弗能禁也。是以司馬芝以為用錢非徒豐國,亦所以省刑。錢之不用,由於兵亂積久,自至於廢,有由而然,漢末是也。今既用而廢之,則百姓頓亡其財。今括囊天下之穀,以周天下
 之食,或倉庾充衍,或糧靡斗儲,以相資通,則貧者仰富,致之之道,實假於錢。一朝斷之,便為棄物,是有錢無糧之民,皆坐而饑困,此斷錢之立敝也。且據今用錢之處不為貧,用穀之處不為富。又民習來久,革之必惑。語曰:『利不百,不易業。』況又錢便於穀邪?魏明帝時,錢廢穀用,三十年矣。以不便於民,乃舉朝大議。精才達治之士,莫不以為宜復用錢,民無異情,朝無異論。彼尚舍穀帛而用錢,足以明穀帛之弊,著於已試。世或謂魏氏不用錢久,積
 累巨萬,故欲行之,利公富國。斯殆不然。



 昔晉文後舅犯之謀,而先成季之信,以為雖有一時之勳,不如萬世之益。于時名賢在列,君子盈朝,大謀天下之利害,將定經國之要術。若穀實便錢,義不昧當時之近利,而廢永用之通業,斷可知矣。斯實由困而思革,改而更張耳。近孝武之末,天下無事,時和年豐,百姓樂業,便自穀帛殷阜,幾乎家給人足,驗之事實,錢又不妨民也。頃兵革屢興,荒饉薦及,饑寒未振,實此之由。公既援而拯之,大革視
 聽,弘敦本之教,明廣農之科,敬授民時,各順其業,游蕩知反,務末自休,固以南畝競力,野無遺壤矣。于是以往,升平必至,何衣食之足恤。愚謂救弊之術,無取於廢錢。」



 玄又議復肉刑,琳之以為:「唐、虞象刑,夏禹立辟,蓋淳薄既異,致化實同,寬猛相濟,惟變所適。《書》曰『刑罰世輕世重』,言隨時也。夫三代風純而事簡,故罕蹈刑辟;季末俗巧而務殷,故動陷憲網。若三千行於叔世,必有踴貴之尤,此五帝不相循法,肉刑不可悉復者也。漢文發仁惻
 之意,傷自新之路莫由,革古創制,號稱刑厝,然名輕而實重,反更傷民。故孝景嗣位,輕之以緩。緩而民慢,又不禁邪,期于刑罰之中,所以見美在昔,歷代詳論而未獲厥中者也。兵荒後,罹法更多。



 棄市之刑,本斬右趾,漢文一謬,承而弗革,所以前賢恨恨,議之而未辯。鐘繇、陳群之意,雖小有不同,而欲右趾代棄市。若從其言,則所活者眾矣。降死之生,誠為輕法,然人情慎顯而輕昧,忽遠而驚近,是以盤盂有銘,韋弦作佩,況在小人,尤其所惑,
 或目所不睹,則忽而不戒,日陳於前,則驚心駭矚。由此言之,重之不必不傷,輕之不必不懼,而可以全其性命,蕃其產育,仁既濟物,功亦益眾。又今之所患,逋逃為先,屢叛不革,逃身靡所,亦以肅戒未犯,永絕惡原。至於餘條,宜依舊制。豈曰允中,貴獻管穴。」



 玄好人附悅,而琳之不能順旨,是以不見知。遷楚臺員外散騎侍郎。遭母憂,去職。服闋,除司徒左西掾,以父致仕自解。時司馬休之為會稽內史、後將軍,仍以琳之為長史。父憂,去官。服闋,
 補太尉主簿,尚書左丞,揚州治中從事史,所居著績。



 時責眾官獻便宜,議者以為宜修庠序,恤典刑,審官方,明黜陟,舉逸拔才,務農簡調。琳之於眾議之外,別建言曰:「夫璽印者,所以辯章官爵,立契符信。



 官莫大於皇帝,爵莫尊於公侯。而傳國之璽,歷代迭用,襲封之印,奕世相傳,貴在仍舊,無取改作。今世唯尉一職,獨用一印,至於內外群官,每遷悉改,討尋其義,私所未達。若謂官各異姓,與傳襲不同,則未若異代之為殊也。若論其名器,雖有公
 卿之貴,不若帝王之重;若以或有誅夷之臣,忌其凶穢,則漢用秦璽;延祚四百,未聞以子嬰身戮國亡,而棄之不佩。帝王公侯之尊,不疑於傳璽,人臣眾僚之卑,何嫌於即印。載籍未聞其說,推例自乖其準。而終年刻鑄,喪功肖實,金銀銅炭之費,不可稱言,非所以因循舊貫易簡之道。愚謂眾官即用一印,無煩改作。



 若有新置官,又官多印少,文或零失,然後乃鑄,則仰裨天府,非唯小益。」



 又曰:「凶門柏裝,不出禮典,起自末代,積習生常,遂成舊
 俗。爰自天子,達于庶人,誠行之有由,卒革必駭。然茍無關於情,而有愆禮度,存之未有所明,去之未有所失,固當式遵先典,釐革後謬,況復兼以游費,實為民患者乎!凡人士喪儀,多出閭里,每有此須,動十數萬,損民財力,而義無所取。至于寒庶,則人思自竭,雖復室如懸磬,莫不傾產殫財,所謂葬之以禮,其若此乎。謂宜謹遵先典,一罷凶門之式,表以素扇,足以示凶。」



 又曰:「昔事故飢荒,米穀綿絹皆貴,其後米價登復,而絹于今一倍。綿絹既
 貴,蠶業者滋,雖勤厲兼倍,而貴猶不息。愚謂致此,良有其由。昔事故之前,軍器正用鎧而已,至於袍襖裲襠,必俟戰陣,實在庫藏,永無損毀。今儀從直衛及邀羅使命,或有防衛送迎,悉用袍襖之屬,非唯一府,眾軍皆然。綿帛易敗,勢不支久。又晝以禦寒,夜以寢臥,曾未周年,便自敗裂。每絲綿新登,易折租以市,又諸府競收,動有千萬,積貴不已,實由於斯,私服為脂艱貴,官庫為之空盡。愚謂若侍衛所須,固不可廢,其餘則依舊用鎧。小小使命
 送迎之屬,止宜給仗,不煩鎧襖。用之既簡,則其價自降」



 又曰:「夫不恥惡食,唯君子能之。肴饌尚奢,為日久矣。今雖改張是弘,而此風未革。所甘不過一味,而陳必方丈,適口之外,皆為悅目之費,富者以之示誇,貧者為之殫產,眾所同鄙,而莫能獨異。愚謂宜粗為其品,使奢儉有中;若有不改,加以貶黜,則德儉之化,不日而流。」



 遷尚書吏部郎。義熙六年,高祖領平西將軍,以為長史,大司馬琅邪王從事中郎。又除高祖平北、征西長史,遷侍中。宋
 臺初建,除宋國侍中。出為吳興太守,公事免。



 永初二年,為御史中丞。明憲直法,無所屈橈。奏劾尚書令徐羨之曰:「臣聞事上以奉憲為恭,臨下以威嚴為整。然後朝典惟明,蒞眾必肅。斯道或替,則憲綱其頹。臣以今月七日,預皇太子正會。會畢車去,並猥臣停門待闕。有何人乘馬,當臣車前,收捕驅遣命去。何人罵詈收捕,咨審欲錄。每有公事,臣常慮有紛紜,語令勿問,而何人獨罵不止,臣乃使錄。何人不肯下馬,連叫大喚,有兩威儀走來,擊
 臣收捕。尚書令省事倪宗又牽威儀手力,擊臣下人。宗云:『中丞何得行凶,敢錄令公人。凡是中丞收捕,威儀悉皆縛取。』臣敕下人一不得鬥,凶勢輈張,有頃乃散。又有群人就臣車側,錄收捕樊馬子,互行築馬子頓伏,不能還臺。臣自錄非,本無對校,而宗敢乘勢兇恣,篡奪罪身。尚書令臣羨之,與臣列車,紛紜若此,或云羨之不禁,或云羨之禁而不止。縱而不禁,既乖國憲;禁而不止,又不經通。陵犯監司,凶聲彰赫,容縱宗等,曾無糾問,虧損國
 威,無大臣之體,不有準繩,風裁何寄。羨之內居朝右,外司輦轂,位任隆重,百辟所瞻。而不能弘惜朝章,肅是風軌。致使宇下縱肆,凌暴憲司,凶赫之聲,起自京邑,所謂己有短垣,而自踰之。



 又宗為篡奪之主,縱不糾問,二三虧違,宜有裁貶。請免羨之所居官,以公還第。



 宗等篡奪之愆,已屬掌故御史隨事檢處。」詔曰:「小人難可檢御,司空無所問,餘如奏。」羨之任居朝端,不欲以犯憲示物。時羨之領揚州刺史,琳之弟璩之為治中,羨之使璩之解
 釋琳之,停寢其事。琳之不許。璩之固陳,琳之謂曰:「我觸忤宰相,正當罪止一身爾,汝必不應從坐,何須勤勤邪!」自是百僚震肅莫敢犯禁。



 高祖甚嘉之,行經蘭臺,親加臨幸。又領本州大中正,遷祠部尚書。不治產業,家尤貧素。景平元年,卒,時年五十五。追贈太常。



 子邈,有父風,官至揚州治中從事史。邈子覬,別有傅。覬弟道存,世祖大明中,歷黃門吏部郎,臨海王子頊前軍長史、南郡太守。晉安王子勛建偽號,為侍中,行雍州事。事敗自殺。



 史臣曰:民生所貴,曰食與貨。貨以通幣,食為民天。是以九棘播於農皇,十朋興於上代。昔醇民未離,情嗜疏寡,奉生贍己,事有易周。一夫躬稼,則餘食委室;匹婦務織,則兼衣被體。雖懋遷之道,通用濟乏,龜貝之益,為功蓋輕。而事有訛變,姦敝代起,昏作役苦,故穡人去而從商,商子事逸,末業流而浸廣,泉貨所通,非復始造之意。於是競收罕至之珍,遠蓄未名之貨,明珠翠羽,無足而馳,絲罽文犀,飛不待翼,天下蕩蕩,咸以棄本為事。豐衍則
 同多稌之資,饑凶又減田家之蓄。錢雖盈尺,既不療饑於堯年;貝或如輪,信無救渴於湯世,其蠹病亦已深矣。固宜一罷錢貨,專用穀帛,使民知役生之路,非此莫由。夫千匹為貨,事難於懷璧;萬斛為市,未易於越鄉,斯可使末伎自禁,游食知反。而年世推移,民與事習,或庫盈朽貫,而高廩未充,或家有藏鏹,而良疇罕闢。若事改一朝,廢而莫用,交易所寄,旦夕無待,雖致乎要術,而非可卒行。先宜削華止偽,還淳反古,抵璧幽峰,捐珠清壑。然後驅
 一世之民,反耕桑之路,使縑粟羨溢,同於水火。既而蕩滌圓法,銷鑄勿遺,立制垂統,永傳於後,比屋稱仁,豈伊唐世。桓玄知其始而不覽其終,孔琳之睹其末而不統其本,豈慮有開塞,將一往之談可然乎。



\end{pinyinscope}