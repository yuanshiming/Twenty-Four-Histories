\article{卷一本紀第一 武帝上}

\begin{pinyinscope}

 高祖武皇帝諱裕,字德輿,小名寄奴,彭城縣綏輿里人,漢高帝弟楚元王交之後也。交生紅懿侯富,富生宗正闢強,闢強生陽城繆侯德,德生陽城節侯安民,安民生陽
 城釐侯慶忌,慶忌生陽城肅侯岑,岑生宗正平,平生東武城令某,某生東萊太守景,景生明經洽,洽生博士弘,弘生瑯邪都尉悝,悝生魏定襄太守某,某生邪城令亮,亮生晉北平太守膺,膺生相國掾熙,熙生開封令旭孫,旭孫生混,始過江,居晉陵郡丹徒縣之京口里,官至武原令。混生東安太守靖,靖生郡功曹翹,是為皇考。高祖以晉哀帝興寧元年歲次癸亥三月壬寅夜生。及長,身長七尺六寸,風骨奇特。家貧,有大志,不治廉隅。事繼母
 以孝謹稱。



 初為冠軍孫無終司馬。安帝隆安三年十一月,妖賊孫恩作亂於會稽,晉朝衛將軍謝琰、前將軍劉牢之東討。牢之請高祖參府軍事。十二月,牢之至吳,而賊緣道屯結,牢之命高祖與數十人,覘賊遠近。會遇賊至,眾數千人,高祖便進與戰。所將人多死,而戰意方厲,手奮長刀,所殺傷甚眾。牢之子敬宣疑高祖淹久,恐為賊所困,乃輕騎尋之。既而眾騎並至,賊乃奔退,斬獲千餘人,推鋒而進,平山陰,恩遁還入海。四年五月,
 恩復入會稽,殺衛將軍謝琰。十一月,劉牢之復率眾東征,恩退走。牢之屯上虞,使高祖戍句章城。句章城既卑小,戰士不盈數百人。高祖常被堅執銳,為士卒先,每戰輒摧鋒陷陣,賊乃退還浹口。于時東伐諸帥,御軍無律,士卒暴掠,甚為百姓所苦。唯高祖法令明整,所至莫不親賴焉。



 五年春,孫恩頻攻句章,高祖屢摧破之,恩復走入海。三月,恩北出海鹽,高祖追而翼之,築城于海鹽故治。賊日來攻城,城內兵力甚弱,高祖乃選敢死之士數
 百人,咸脫甲胄,執短兵,並鼓噪而出。賊震懼奪氣,因其懼而奔之,並棄甲散走,斬其大帥姚盛。雖連戰克勝,然眾寡不敵,高祖獨深慮之。一夜,偃旗匿眾,若已遁者。明晨開門,使羸疾數人登城。賊遙問劉諱所在,曰:「夜已走矣。」賊信之,乃率眾大上。高祖乘其懈怠,奮擊,大破之。恩知城不可下,乃進向滬瀆。高祖復棄城追之。海鹽令鮑陋遣子嗣之以吳兵一千,請為前驅。高祖曰:「賊兵甚精,吳人不習戰。若前驅失利,必敗我軍,可在後為聲援。」不
 從。是夜,高祖多設伏兵,兼置旗鼓,然一處不過數人。明日,賊率眾萬餘迎戰。前驅既交,諸伏皆出,舉旗鳴鼓。賊謂四面有軍,乃退。嗣之追奔,為賊所沒。高祖且戰且退,賊盛,所領死傷且盡。高祖慮不免,至向伏兵處,乃止,令左右脫取死人衣。賊謂當走反停,疑猶有伏。高祖因呼更戰,氣色甚猛,賊眾以為然,乃引軍去。高祖徐歸,然後散兵稍集。五月,孫恩破滬瀆,殺吳國內史袁山松,死者四千人。是月,高祖復破賊於婁縣。六月,恩乘勝浮海,奄
 至丹徒,戰士十餘萬。劉牢之猶屯山陰,京邑震動。



 高祖倍道兼行,與賊俱至。于時眾力既寡,加以步遠疲勞,而丹徒守軍莫有鬥志。



 恩率眾數萬,鼓噪登蒜山,居民皆荷擔而立。高祖率所領奔擊,大破之,投巘赴水死者甚眾。恩以彭排自載,僅得還船。雖被摧破,猶恃其眾力,徑向京師。樓船高大,值風不得進,旬日乃至白石。尋知劉牢之已還,朝廷有備,遂走向鬱洲。八月,以高祖為建武將軍、下邳太守,領水軍追恩至鬱洲,復大破恩,恩南
 走。十一月,高祖追恩於滬瀆,及海鹽,又破之。三戰,並大獲,俘馘以萬數。恩自是饑饉疾疫,死者太半,自浹口奔臨海。



 元興元年正月,驃騎將軍司馬元顯西伐荊州刺史桓玄,玄亦率荊楚大眾,下討元顯。元顯遣鎮北將軍劉牢之拒之,高祖參其軍事,次溧洲。玄至,高祖請擊之,不許,將遣子敬宣詣玄請和。高祖與牢之甥東海何無忌並固請,不從。遂遣敬宣詣玄,玄克京邑,殺元顯,以牢之為會稽內史。懼而告高祖曰:「便奪我兵,禍其至矣。今
 當北就高雅於廣陵舉事,卿能從我去乎?」答曰:「將軍以勁卒數萬,望風降服。彼新得志,威震天下。三軍人情,都已去矣,廣陵豈可得至邪!諱當反復還京口耳。」牢之叛走,自縊死。何無忌謂高祖曰:「我將何之?」高祖曰:「鎮北去必不免,卿可隨我還京口。桓玄必能守節北面,我當與卿事之。不然,與卿圖之。



 今方是玄矯情任算之日,必將用我輩也。」桓玄從兄修以撫軍鎮丹徒,以高祖為中兵參軍,軍、郡如故。



 孫恩自奔敗之後,徒旅漸散,懼生見獲,
 乃於臨海投水死。餘眾推恩妹夫盧循為主。桓玄欲且緝寧東土,以循為永嘉太守。循雖受命,而寇暴不已。五月,玄復遣高祖東征。時循自臨海入東陽。二年正月,玄復遣高祖破循於東陽。循奔永嘉,復追破之,斬其大帥張士道,追討至于晉安,循浮海南走。六月,加高祖彭城內史。



 桓玄為楚王,將謀篡盜。玄從兄衛將軍謙屏人問高祖曰:「楚王勳德隆重,四海歸懷。朝廷之情,咸謂宜有揖讓,卿意以為何如?」高祖既志欲圖玄,乃遜辭答曰:「楚
 王,宣武之子,勛德蓋世。晉室微弱,民望久移,乘運禪代,有何不可!」



 謙喜曰:「卿謂可爾,便當是真可爾。」十二月,桓玄篡帝位,遷天子於尋陽。桓修入朝,高祖從至京邑。玄見高祖,謂司徒王謐曰:「昨見劉諱,風骨不恒,蓋人傑也。」每遊集,輒引接殷勤,贈賜甚厚。高祖愈惡之。或說玄曰:「劉諱龍行虎步,視瞻不凡,恐不為人下,宜蚤為其所。」玄曰:「我方欲平蕩中原,非劉諱莫可付以大事。關隴平定,然後當別議之耳。」玄乃下詔曰:「劉諱以寡制眾,屢摧妖
 鋒,汎海窮追,十殄其八。諸將力戰,多被重創。自元帥以下至于將士,並宜論賞,以敘勳烈。」



 先是,高祖東征盧循,何無忌隨至山陰,勸於會稽舉義。高祖以為玄未據極位,且會稽遙遠,事濟為難,俟其篡逆事著,徐於京口圖之,不憂不克。至是桓脩還京,高祖託以金創疾動,不堪步從,乃與無忌同船共還,建興復之計。於是與弟道規、沛郡劉毅、平昌孟昶、任城魏詠之、高平檀憑之、琅邪諸葛長民、太原王元德、隴西辛扈興、東莞童厚之,並同義
 謀。時桓修弟弘為征虜將軍、青州刺史,鎮廣陵。



 道規為弘中兵參軍,昶為州主簿。乃令毅潛往就昶,聚徒於江北,謀起兵殺弘。長民為豫州刺史刁逵左軍府參軍,謀據歷陽相應。元德、厚之謀於京邑,聚眾攻玄,並剋期齊發。



 三年二月己丑朔,乙卯,高祖託以遊獵,與無忌等收集義徒,凡同謀何無忌、魏詠之、詠之弟欣之、順之、檀憑之、憑之從子韶、弟祗、隆與叔道濟、道濟從兄範之、高祖弟道憐、劉毅、毅從弟籓、孟昶、昶族弟懷玉、河內向彌、管
 義之、陳留周安穆、臨淮劉蔚、從弟珪之、東莞臧熹、從弟寶符、從子穆生、童茂宗、陳郡周道民、漁陽田演、譙國范清等二十七人;願從者百餘人。丙辰,詰旦,城開,無忌服傳詔服,稱詔居前。義眾馳入,齊聲大呼,吏士驚散,莫敢動,即斬脩以徇。



 高祖哭甚慟,厚加殯斂。孟昶勸弘其日出獵。未明開門,出獵人,昶、道規、毅等率壯士五六十人因開門直入。弘方啖粥,即斬之,因收眾濟江。義軍初剋京城,脩司馬刁弘率文武佐吏來赴。高祖登城謂之曰:「
 郭江州已奉乘輿反正於尋陽,我等並被密詔,誅除逆黨,同會今日。賊玄之首,已當梟於大航矣。諸君非大晉之臣乎,今來欲何為?」弘等信之,收眾而退。毅既至,高祖命誅弘。



 毅兄邁先在京師,事未發數日,高祖遣同謀周安穆報之,使為內應。邁外雖酬許,內甚震懼。安穆見其惶駭,慮事必泄,乃馳歸。時玄以邁為竟陵太守,邁不知所為,便下船欲之郡。是夜,玄與邁書曰:「北府人情云何?卿近見劉諱何所道?」



 邁謂玄已知其謀,晨起白之。玄驚
 懼,封邁為重安侯,既而嫌邁不執安穆,使得逃去,乃殺之。誅元德、扈興、厚之等。召桓謙、卞範之等謀拒高祖。謙等曰:「亟遣兵擊之。」玄曰:「不然。彼兵速銳,計出萬死。若行遣水軍,不足相抗;如有蹉跌,則彼氣成而吾事敗矣!不如屯大眾於覆舟山以待之。彼空行二百里,無所措手,銳氣已挫,既至,忽見大軍,必驚懼駭愕。我案兵堅陣,勿與交鋒。彼求戰不得,自然散走。此計之上也。」謙等固請,乃遣頓丘太守吳甫之、右衛將軍皇甫敷北拒義軍。玄
 自聞軍起,憂懼無復為計。或曰:「劉諱等眾力甚弱,豈辦之有成,陛下何慮之甚!」玄曰:「劉諱足為一世之雄,劉毅家無擔石之儲,摴蒲一擲百萬;何無忌,劉牢之甥,酷似其舅。共舉大事,何謂無成。」



 眾推高祖為盟主,移檄京邑,曰:夫治亂相因,理不常泰,狡焉肆虐,或值聖明。自我大晉,陽九屢構。隆安以來,難結皇室。忠臣碎於虎口,貞良弊於豺狼。逆臣桓玄,陵虐人鬼,阻兵荊郢,肆暴都邑。天未亡難,凶力繁興,踰年之間,遂傾皇祚。主上播越,流幸
 非所;神器沉淪,七廟毀墜。夏后之罹浞、豷,有漢之遭莽、卓,方之於玄,未足為喻。自玄篡逆,于今歷年,亢旱彌時,民無生氣。加以士庶疲於轉輸,文武困於造築,父子乖離,室家分散,豈唯《大東》有杼軸之悲,《摽梅》有傾筐之怨而已哉!仰觀天文,俯察人事,此而能久,孰有可亡!凡在有心,誰不扼腕。諱等所以叩心泣血,不遑啟處者也。是故夕寐宵興,援獎忠烈,潛構崎嶇,險過履虎。輔國將軍劉毅、廣武將軍何無忌、鎮北主簿孟昶、兗州主簿魏詠
 之、寧遠將軍劉道規、龍驤將軍劉籓、振威將軍檀憑之等,忠烈斷金,精貫白日,荷戈奮袂,志在畢命。益州刺史毛璩,萬里齊契,掃定荊楚。江州刺史郭昶之,奉迎主上,宮于尋陽。鎮北參軍王元德等,並率部曲,保據石頭。揚武將軍諸葛長民,收集義士,已據歷陽。征虜參軍庾賾之等,潛相連結,以為內應。同力協規,所在蜂起,即日斬偽徐州刺史安城王脩、青州刺史弘首。義眾既集,文武爭先,咸謂不有一統,則事無以輯。諱辭不獲已,遂總軍
 要。庶上憑祖宗之靈,下罄義夫之力,翦馘逋逆,蕩清京輦。公侯諸君,或世樹忠貞,或身荷爵寵,而並俯眉猾豎,自效莫由,顧瞻周道,寧不弔乎!今日之舉,良其會也。諱以虛薄,才非古人,接勢於已替之機,受任於既頹之運。丹誠未宣,感慨憤躍,望霄漢以永懷,眄山川以增厲。授檄之日,神馳賊廷。



 以孟昶為長史,總攝後事;檀憑之為司馬。百姓願從者千餘人。三月戊午朔,遇吳甫之於江乘。甫之,玄驍將也,其兵甚銳。高祖躬執長刀,大呼以衝
 之,眾皆披靡,即斬甫之。進至羅落橋,皇甫敷率數千人逆戰。寧遠將軍檀憑之與高祖各御一隊,憑之戰敗見殺,其眾退散。高祖進戰彌厲,前後奮擊,應時摧破,即斬敷首。



 初,高祖與何無忌等共建大謀,有善相者相高祖及無忌等並當大貴,其應甚近,惟云憑之無相。高祖與無忌密相謂曰:「吾等既為同舟,理無偏異。吾徒咸皆富貴,則檀不應獨殊。」深不解相者之言。至是而憑之戰死,高祖知其事必捷。



 玄聞敷等並沒,愈懼,使桓謙屯東陵
 口,卞範之屯覆舟山西,眾合二萬。己未旦,義軍食畢,棄其餘糧,進至覆舟山東,使丐士張旗幟於山上,以為疑兵;玄又遣武騎將軍庾禕之,配以精卒利器,助謙等。高祖躬先士卒以奔之,將士皆殊死戰,無不一當百,呼聲動天地。時東北風急,因命縱火,煙焰張天,鼓噪之音震京邑。



 謙等諸軍,一時土崩。玄始雖遣軍置陣,而走意已決,別使領軍將軍殷仲文具舟於石頭,仍將子姪浮江南走。庚申,高祖鎮石頭城,立留臺,總百官,焚桓溫神主
 於宣陽門外,造晉新主,立于太廟。遣諸將帥追玄,尚書王嘏率百官奉迎乘輿。司徒王謐與眾議推高祖領揚州,固辭。乃以謐為錄尚書事,領揚州刺史。於是推高祖為使持節、都督揚徐兗豫青冀幽并八州諸軍事、領軍將軍、徐州刺史。



 先是,朝廷承晉氏亂政,百司縱弛,桓玄雖欲釐整,而眾莫從之。高祖以身範物,先以威禁內外,百官皆肅然奉職。二三日間,風俗頓改。且桓玄雖以雄豪見推,而一朝便有極位,晉氏四方牧守及在朝大臣,
 盡心伏事,臣主之分定矣。高祖位微於朝,眾無一旅,奮臂草萊之中,倡大義以復皇祚。由是王謐等諸人時眾民望,莫不愧而憚焉。



 諸葛長民失期不得發,刁逵執送之,未至而玄敗。玄經尋陽,江州刺史郭昶之備乘輿法物資之。玄收略得二千餘人,挾天子走江陵。冠軍將軍劉毅、輔國將軍何無忌、振武將軍劉道規率諸軍追討。尚書左僕射王愉、愉子荊州刺史綏等,江左冠族。綏少有重名,以高祖起自布衣,甚相凌忽。綏,桓氏甥,亦有自
 疑之志。高祖悉誅之。四月,奉武陵王遵為大將軍,承制,大赦天下,唯桓玄一祖後不在赦例。



 初,高祖家貧,嘗負刁逵社錢三萬,經時無以還。逵執錄甚嚴,王謐造逵見之,密以錢代還,由是得釋。高祖名微位薄,盛流皆不與相知,唯謐交焉。桓玄將篡,謐手解安帝璽紱,為玄佐命功臣。及義旗建,眾並謂謐宜誅,唯高祖保持之。劉毅嘗因朝會,問謐璽紱所在,謐益懼。及王愉父子誅,謐從弟諶謂謐曰:「王駒無罪,而義旗誅之,此是剪除勝己,以絕
 民望。兄既桓氏黨附,名位如此,欲求免得乎?」



 駒,愉小字也。謐懼,奔于曲阿。高祖箋白大將軍,深相保謐,迎還復位。光祿勳丁承之、左衛將軍褚粲、游擊將軍司馬秀役使官人,為御史中丞王禎之所糾察,謝箋言辭怨忿。承之造司宜藏。高祖與大將軍箋,白「粲等備位大臣,所懷必盡,執憲不允,自應據理陳訴,而橫興怨忿,歸咎有司,宜加裁當,以清風軌」。並免官。



 桓玄兒子韶,聚眾向歷陽,高祖命輔國將軍諸葛長民擊走之。無忌、道規破玄大
 將郭鈐等于桑落洲,眾軍進據尋陽。加高祖督江州諸軍事。玄既還荊郢,大聚兵眾,召水軍造樓船、器械,率眾二萬,挾天子發江陵,浮江東下,與冠軍將軍劉殷等相遇於崢嶸洲,眾軍下擊,大破之。玄棄眾,復挾天子還復江陵。玄黨殷仲文奉晉二皇后還京師。玄至江陵,因西走。南郡太守王騰之、荊州別駕王康產奉天子入南郡府。初,征虜將軍、益州刺史毛璩,遣從孫祐之與參軍費恬送弟喪下,有眾二百。璩弟子脩之時為玄屯騎
 校尉,誘玄以入蜀。至枚回洲,恬與祐之迎射之。益州督護馮遷斬玄首,傳京師,又斬玄子昇於江陵市。



 初,玄敗於崢嶸洲,義軍以為大事已定,追躡不速。玄死幾一旬,眾軍猶不至。



 玄從子振逃於華容之湧中,招聚逆黨數千人,晨襲江陵城,居民競出赴之。騰之、康產皆被殺。桓謙先匿於沮川,亦聚眾以應。振為玄舉哀,立喪廷。謙率眾官奉璽綬于安帝。無忌、道規既至江陵,與桓振戰于靈溪。玄黨馮該又設伏于楊林,義軍奔敗,退還尋陽。兗
 州刺史辛禺懷貳。會北青州刺史劉該反,禺求徵該,次淮陰,又反。禺長史羊穆之斬禺,傳首京師。十月,高祖領青州刺史。甲仗百人入殿。



 劉毅諸軍復進至夏口。毅攻魯城,道規攻偃月壘,皆拔之。十二月,諸軍進平巴陵。義熙元年正月,毅等至江津,破桓謙、桓振,江陵平。天子反正。三月,天子至自江陵。詔曰:古稱大者天地,其次君臣,所以列貫三辰,神人代序,諒理本於造昧,而運周於萬葉。故盈否時襲,四靈通其變;王道或昧,貞賢拯其危。天
 命所以永固,人心所以攸穆。雖夏、周中傾,賴靡、申之績,莽、倫載竊,實二代是維,或乘資藉號,或業隆異世,猶詩書以之休詠,記策用為美談。未有因心撫民,而誠發理應,援神器於已淪,若在今之盛者也。朕以寡昧,遭家不造,越自遘閔,屬當屯極。逆臣桓玄,乘釁縱慝,窮凶恣虐,滔天猾夏。遂誣罔人神,肆其篡亂。祖宗之基既湮,七廟之饗胥殄,若墜淵谷,未足斯譬。



 皇度有晉,天縱英哲,使持節、都督揚徐兗豫青冀幽並江九州諸軍事、鎮軍將
 軍、徐青二州刺史,忠誠天亮,神武命世,用能貞明協契,義夫響臻。故順聲一唱,二溟卷波;英風振路,宸居清翳。暨冠軍將軍毅、輔國將軍無忌、振武將軍道規,舟旗遄邁,而元凶傳首;回戈疊揮,則荊、漢霧廓。俾宣、元之祚,永固於嵩、岱;傾基重造,再集於朕躬。宗廟歆七百之祜,皇基融載新之命。念功惟德,永言銘懷。



 固已道冠開闢,獨絕終古,書契以來,未之前聞矣。雖則功高靡尚,理至難文,而崇庸命德,哲王攸先者,將以弘道制治,深關盛衰。
 故伊、望膺殊命之錫,桓、文饗備物之禮,況宏征不世,顧邈百代者,宜極名器之隆,以光大國之盛。而鎮軍謙虛自衷,誠旨屢顯。朕重逆仲父,乃所以愈彰德美也。鎮軍可進位侍中、車騎將軍、都督中外諸軍事,使持節、徐青二州刺史如故。顯祚大邦,啟茲疆宇。



 高祖固讓;加錄尚書事,又不受,屢請歸籓。天子不許,遣百僚敦勸,又親幸公第。高祖惶懼,詣闕陳請,天子不能奪。是月,旋鎮丹徒。天子重遣大使敦勸,又不受。乃改授都督荊、司、梁、益、寧、
 雍、涼七州,并前十六州諸軍事,本官如故。於是受命解青州,加領兗州刺史。



 盧循浮海破廣州,獲刺史吳隱之。即以循為廣州刺史,以其同黨徐道覆為始興相。二年三月,督交、廣二州。十月,高祖上言曰:「昔天禍皇室,巨狡縱篡,臣等義惟舊隸,豫蒙國恩,仰契信順之符,俯厲人臣之憤,雖社稷之靈,抑亦事由眾濟。其翼獎忠勤之佐,文武畢力之士,敷執在己之謙,用虧國體之大,輒申攝眾軍先上,同謀起義,始平京口、廣陵二城。臣及撫軍將軍
 毅等二百七十二人,并後赴義出都,緣道大戰,所餘一千五百六十六人。又輔國將軍長民、故給事中王元德等十人,各一千八百四十八人,乞正封賞。其西征眾軍,須論集續上。」於是尚書奏封唱義謀主鎮軍將軍諱豫章郡公,食邑萬戶,賜絹三萬匹。其餘封賞各有差。鎮軍府佐吏,降故太傅謝安府一等。十一月,天子重申前令,加高祖侍中,進號車騎將軍、開府儀同三司。固讓。詔遣百僚敦勸。三年二月,高祖還京師,將詣廷尉;天子先詔
 獄官不得受,詣闕陳讓,乃見聽。旋于丹徒。



 閏月,府將駱冰謀作亂,將被執,單騎走,追斬之。誅冰父永嘉太守球。球本東陽郡史,孫恩之亂,起義於長山,故見擢用。初,桓玄之敗,以桓沖忠貞,署其孫胤。至是冰謀以胤為主,與東陽太守殷仲文潛相連結。乃誅仲文及仲文二弟。凡桓玄餘黨,至是皆誅夷。



 天子遣兼太常葛籍授公策曰:「有扈滔天,夷羿乘釁,亂節干紀,實撓皇極。



 賊臣桓玄,怙寵肆逆,乃摧傾華、霍,倒拔嵩、岱,五嶽既夷,六地易所。公
 命世英縱,藏器待時,因心資敬,誓雪國恥。慨憤陵夷,誠發宵寐。既而歲月屢遷,神器已遠,忠孝幽寄,實貫三靈。爾乃介石勝機,宣契畢舉,訴蒼天以為正,揮義旅而一驅;奔鋒數百,勢烈激電,百萬不能抗限,制路日直植城。遂使衝鯨潰流,暴鱗奔漢,廟勝遠加,重氛載滌,二儀廓清,三光反照,事遂永代,功高開闢,理微稱謂,義感朕心。若夫道為身濟,猶縻厥爵,況乃誠德俱深,勳冠天人者乎!是用建茲邦國,永祚山河,言念載懷,匪云足報。往欽
 哉!俾屏余一人,長弼皇晉,流風垂祚,暉烈無窮。其降承嘉策,對揚朕命。」十二月,司徒、錄尚書、揚州刺史王謐薨。



 四年正月,征公入輔,授侍中、車騎將軍、開府儀同三司、揚州刺史、錄尚書、徐兗二州刺史如故。表解兗州。先是,遣冠軍劉敬宣伐蜀賊譙縱,無功而返。九月,以敬宣挫退,遜位,不許。乃降為中軍將軍,開府如故。



 初,偽燕王鮮卑慕容德僭號於青州,德死,兄子超襲位,前後數為邊患。五年二月,大掠淮北,執陽平太守劉千載、濟南太守
 趙元,驅略千餘家。三月,公抗表北討,以丹陽尹孟昶監中軍留府事。四月,舟師發京都,溯淮入泗。五月,至下邳,留船艦輜重,步軍進琅邪;所過皆築城留守。鮮卑梁父、莒城二戍並奔走。慕容超聞王師將至,其大將公孫五樓說超:「宜斷據大峴,刈除粟苗,堅壁清野以待之。



 彼僑軍無資,求戰不得,旬月之間,折棰以笞之耳。」超不從,曰:「彼遠來疲勞,勢不能久;但當引令過峴,我以鐵騎踐之,不憂不破也。豈有預芟苗稼,先自蹙弱邪!」初,公將行,議
 者以為賊聞大軍遠出,必不敢戰。若不斷大峴,當堅守廣固,刈粟清野,以絕三軍之資,非唯難以有功,將不能自反。公曰:「我揣之熟矣。鮮卑貪,不及遠計,進利剋獲,退惜粟苗。謂我孤軍遠入,不能持久,不過進據臨朐,退守廣固。我一得入峴,則人無退心,驅必死之眾,向懷貳之虜,何憂不克!彼不能清野固守,為諸君保之。」公既入峴,舉手指天曰:「吾事濟矣!」



 六月,慕容超遣五樓及廣寧王賀賴盧先據臨朐城。既聞大軍至,留羸老守廣固,乃悉
 出。臨朐有巨蔑水,去城四十里,超告五樓曰:「急往據之,晉軍得水,則難擊也。」五樓馳進。龍驤將軍孟龍符領騎居前,奔往爭之,五樓乃退。眾軍步進,有車四千兩,分車為兩翼,方軌徐行,車悉張幔,御者執槊,又以輕騎為遊軍。軍令嚴肅,行伍齊整。未及臨朐數里,賊鐵騎萬餘,前後交至。公命兗州刺史劉籓、弟并州刺史道憐、諮議參軍劉敬宣、陶延壽、參軍劉懷玉、慎仲道、索邈等,齊力擊之。日向昃,公遣諮議參軍檀韶直趨臨朐。韶率建威將
 軍向彌、參軍胡籓馳往,既日陷城,斬其牙旗,悉虜超輜重。超聞臨朐已拔,引眾走。公親鼓之,賊乃大破。



 超遁還廣固。獲超馬、偽輦、玉璽、豹尾等,送于京師;斬其大將段暉等十餘人,其餘斬獲千計。明日,大軍進廣固,既屠大城。超退保小城。於是設長圍守之,圍高三丈,外穿三重塹。停江、淮轉輸,館穀於齊土。撫納降附,華戎歡悅;援才授爵,因而任之。七月,詔加公北青、冀二州刺史。超大將垣遵、遵弟苗並率眾歸順。



 公方治攻具,城上人曰:「汝不
 得張綱,何能為也。」綱者,超偽尚書郎,其人有巧思。會超遣綱稱籓於姚興,乞師請救。興偽許之,而實憚公,不敢遣。綱從長安還,泰山太守申宣執送之。乃升綱於樓上,以示城內,城內莫不失色。於是使綱大治攻具。超求救不獲,綱反見虜,轉憂懼,乃請稱籓,求割大峴為界,獻馬千匹。



 不聽,圍之轉急。河北居民荷戈負糧至者,日以千數。



 錄事參軍劉穆之,有經略才具,公以為謀主,動止必諮焉。時姚興遣使告公云:「慕容見與鄰好,又以窮告急,
 今當遣鐵騎十萬,徑據洛陽。晉軍若不退者,便當遣鐵騎長驅而進。」公呼興使答曰:「語汝姚興,我定燕之後,息甲三年,當平關、洛。今能自送,便可速來!」穆之聞有羌使,馳入,而公發遣已去。以興所言并答,具語穆之。穆之尤公曰:「常日事無大小,必賜與謀之。此宜善詳之,云何卒爾便答?公所答興言,未能威敵,正足怒彼耳。若燕未可拔,羌救奄至,不審何以待之?」



 公笑曰:「此是兵機,非卿所解,故不語耳。夫兵貴神速,彼若審能遣救,必畏我知,寧
 容先遣信命。此是其見我伐燕,內已懷懼,自張之辭耳。」九月,進公太尉、中書監,固讓。偽徐州刺史段宏先奔索虜,十月,自河北歸順。



 張綱治攻具成,設諸奇巧,飛樓木幔之屬,莫不畢備。城上火石弓矢,無所用之。六年二月丁亥,屠廣固。超踰城走,征虜賊曹喬胥獲之,殺其亡命以下,納口萬餘,馬二千匹。送超京師,斬于建康市。



 公之北伐也,徐道覆仍有窺窬之志,勸盧循乘虛而出,循不從。道覆乃至番禺說循曰:「本住嶺外,豈以理極於此,正以
 劉公難與為敵故也。今方頓兵堅城之下,未有旋日。以此思歸死士,掩襲何、劉之徒,如反掌耳。不乘此機而保一日之安,若平齊之後,小息甲養眾,不過一二年間,必璽書徵君。若劉公自率眾至豫章,遣銳師過嶺,雖復將軍神武,恐必不能當也。今日之機,萬不可失。既剋都邑,傾其根本。劉公雖還,無能為也。」循從之,乃率眾過嶺。是月,寇南康、廬陵、豫章,諸郡守皆委任奔走。于時平齊問未至,既馳使征公。公之初克齊也,欲停鎮下邳,清蕩河、
 洛,既而被徵使至,即日班師。



 鎮南將軍何無忌與徐道覆戰于豫章,敗績,無忌被害,內外震駭。朝廷欲奉乘輿北走就公,尋知賊定未至,人情小安。公至下邳,以船運輜重,自率精銳步歸。



 至山陽,聞無忌被害,則慮京邑失守,乃卷甲兼行,與數十人至淮上,問行旅以朝廷消息。人曰:「賊尚未至,劉公若還,便無所憂也。」公大喜,單船過江,徑至京口,眾乃大安。四月癸未,公至京師,解嚴息甲。



 撫軍將軍劉毅抗表南征,公與毅書曰:「吾往習擊妖賊,
 曉其變態,新獲奸利,其鋒不可輕。宜須裝嚴畢,與弟同舉。」又遣毅從弟籓往止之。毅不從,舟師二萬,發自姑孰。循之初下也,使道覆向尋陽,自寇湘中諸郡。荊州刺史道規遣軍至長沙,為循所敗。徑至巴陵,將向江陵。道覆聞毅上,馳使報循曰:「毅兵眾甚盛,成敗事係之於此,宜并力摧之。若此克捷,天下無復事矣。根本既定,不憂上面不平也。」



 循即日發巴陵,與道覆連旗而下。別有八艚艦九枚,起四層,高十二丈。公以南籓覆沒,表送章綬,詔
 不聽。五月,劉毅敗績于桑落洲,棄船步走,餘眾不得去者,皆為賊所擒。初,循至尋陽,聞公已還,不信也。既破毅,乃審凱入之問,並相視失色。循欲退還尋陽,進平江陵,據二州以抗朝廷。道覆謂宜乘勝徑進,固爭之。



 疑議多日,乃見從。



 毅敗問至,內外洶擾。于時北師始還,多創痍疾病。京師戰士,不盈數千。賊既破江、豫二鎮,戰士十餘萬,舟車百里不絕。奔敗還者,並聲其雄盛。孟昶、諸葛長民懼寇漸逼,欲擁天子過江,公不聽,昶固請不止。公曰:「
 今重鎮外傾,彊寇內逼,人情危駭,莫有固志。若一旦遷動,便自瓦解土崩,江北亦豈可得至!設令得至,不過延日月耳。今兵士雖少,自足以一戰。若其克濟,則臣主同休;茍厄運必至,我當以死衛社稷,橫尸廟門,遂其由來以身許國之志,不能遠竄於草間求活也。我既決矣,卿勿復言!」昶恐其不濟,乃為表曰:「臣諱北討,眾並不同,唯臣贊諱行計,致使彊賊乘間,社稷危逼,臣之罪也。今謹引分以謝天下。」封表畢,乃仰藥而死。



 於是大開賞募,投
 身赴義者,一同登京城之科。發居民治石頭城,建牙戒嚴。



 時議者謂宜分兵守諸津要。公以為:「賊眾我寡,若分兵屯,則人測虛實。且一處失利,則沮三軍之心。今聚眾石頭,隨宜應赴,既令賊無以測多少,又於眾力不分。



 若徒旅轉集,徐更論之耳。」移屯石頭,乃柵淮斷查浦。既而群賊大至,公策之曰:「賊若於新亭直進,其鋒不可當,宜且回避,勝負之事,未可量也;若回泊西岸,此成擒耳。」



 道覆欲自新亭、白石焚舟而上。循多疑少決,每欲以萬全
 為慮,謂道覆曰:「大軍未至,孟昶便望風自裁,大勢言之,自當計日潰亂。今決勝負於一朝,既非必定之道,且殺傷士卒,不如按兵待之。」公于時登石頭城以望循軍,初見引向新亭,公顧左右失色;既而回泊蔡洲。道覆猶欲上,循禁之。自是眾軍轉集,脩治越城,築查浦、藥園、廷尉三壘,皆守以實眾。冠軍將軍劉敬宣屯北郊,輔國將軍孟懷玉屯丹陽郡西,建武將軍王仲德屯越城,廣武將軍劉默屯建陽門外。使寧朔將軍索邈領鮮卑具裝虎
 班突騎千餘匹,皆被練五色,自淮北至于新亭。賊並聚觀,咸畏憚之;然猶冀京邑及三吳有應之者。遣十餘艦來拔石頭柵。公命神弩射之,發輒摧陷,循乃止,不復攻柵。設伏兵於南岸,使羸老悉乘舟艦向白石。公憂其從白石步上,乃率劉毅、諸葛長民北出拒之,留參軍徐赤特戍南岸,命堅守勿動。公既去,賊焚查浦步上,赤特軍戰敗,死沒有百餘人。赤特棄餘眾,單舸濟淮,賊遂率數萬屯丹陽郡。公率諸軍馳歸,眾憂賊過,咸謂公當徑還
 拒戰,公先分軍還石頭,眾莫之曉。解甲息士,洗浴飲食之,乃出列陳於南塘。以赤特違處分,斬之。命參軍諸葛叔度、朱齡石率勁勇士千餘人過淮。群賊數千,皆長刀矛金延,精甲曜日,奮躍爭進。齡石所領多鮮卑,善步槊,並結陳以待之。賊短兵弗能抗,死傷者數百人,乃退走。會日暮,眾亦歸。



 劉毅之敗,豫州主簿袁興國反叛,據歷陽以應賊。瑯邪內史魏順之遣將謝寶討斬之。興國司馬襲寶,順之不救而退,公怒斬之。順之,詠之之弟也。於是功
 臣震懾,莫敢不用命。六月,更授公太尉、中書監,加黃鉞。受黃鉞,餘固辭。以司馬庾悅為建威將軍、江州刺史,自東陽出豫章。七月庚申,群賊自蔡洲南走,還屯尋陽。遣輔國將軍王仲德、廣川太守劉鐘、河間太守蒯恩追之。公還東府,大治水軍,皆大艦重樓,高者十餘丈。盧循遣其大將荀林寇江陵,桓謙先於江陵奔羌,又自羌入蜀,偽主譙縱以為荊州刺史。謙及譙道福率軍二萬,出寇江陵,適與林會,相去百餘里。荊州刺史道規斬謙于枝
 江,破林於江津,追至竹町,斬之。初,循之走也,公知其必寇江陵,登遣淮陵內史索邈領馬軍步道援荊州;又遣建威將軍孫季高率眾三千,自海道襲番禺。江州刺史庾悅至五畝嶠,賊遣千餘人據斷嶠道,悅前驅鄱陽太守虞丘進攻破之。公治兵大辦。十月,率兗州刺史劉籓、寧朔將軍檀韶等舟師南伐。以後將軍劉毅監太尉留守府,後事皆委焉。是月,徐道覆率眾三萬寇江陵。荊州刺史道規又大破之,斬首萬餘級,道覆走還盆口。初,公
 之遣索邈也,邈在道為賊所斷,道覆敗後方達。自循東下,江陵斷絕京邑之問,傳者皆云已沒。及邈至,方知循走。



 循初自蔡洲南走,留其親黨范崇民五千人,高艦百餘,戍南陵。王仲德等聞大軍且至,乃進攻之。十一月,大破崇民軍,焚其舟艦,收其散卒。循廣州守兵,不以海道為防。是月,建威將軍孫季高乘海奄至,而城池峻整,兵猶數千。季高焚賊舟艦,悉力而上,四面攻之,即日屠其城。循父以輕舟奔始興。季高撫其舊民,戮其親黨,勒兵
 謹守。初,公之遣季高也,眾咸以海道艱遠,必至為難;且分撤見力,二三非要。公不從。敕季高曰:「大軍十二月之交,必破妖虜。卿今時當至廣州,傾其巢窟,令賊奔走之日,無所歸投。」季高受命而行,如期剋捷。



 循方治兵旅舟艦,設諸攻備。公欲御以長算,乃屯軍雷池。賊揚聲不攻雷池,當乘流徑下。公知其欲戰,且慮賊戰敗,或於京江入海,遣王仲德以水艦二百於吉陽下斷之。十二月,循、道覆率眾數萬,方艦而下,前後相抗,莫見舳艫之際。公
 悉出輕利鬥艦,躬提幡鼓,命眾軍齊力擊之;又上步騎於西岸。右軍參軍庾樂生乘艦不進,斬而徇之,於是眾軍並踴騰爭先。軍中多萬鈞神弩,所至莫不摧陷。公中流蹙之,因風水之勢,賊艦悉泊西岸,上軍先備火具,乃投火焚之。煙焰張天,賊眾大敗,追奔至夜乃歸。循等還尋陽。初分遣步軍,莫不疑怪,及燒賊艦,眾乃悅服。召王仲德,請還為前驅,留輔國將軍孟懷玉守雷池。循聞有大軍上,欲走向豫章,乃悉力柵斷左里。大軍至左里,將戰,
 公所執麾竿折,折幡沈水,眾並怪懼。



 公歡笑曰:「往年覆舟之戰,幡竿亦折;今者復然,賊必破矣。」即攻柵而進。循兵雖殊死戰,弗能禁。諸軍乘勝奔之,循單舸走。所殺及投水死,凡萬餘人。納其降附,宥其逼略。遣劉籓、孟懷玉輕軍追之。循收散卒,尚有數千人,徑還廣州。



 道覆還保始興。公旋自左里,天子遣侍中、黃門勞師于行所。






\end{pinyinscope}