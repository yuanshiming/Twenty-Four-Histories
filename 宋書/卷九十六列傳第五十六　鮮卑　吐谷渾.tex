\article{卷九十六列傳第五十六 鮮卑 吐谷渾}

\begin{pinyinscope}

 阿柴虜吐谷渾,遼東鮮卑也。父弈洛韓,有二子,長曰吐谷渾,少曰若洛廆。



 若洛廆別為慕容氏,渾庶長,廆正嫡。父在時,分七百戶與渾。渾與廆二部俱牧馬,馬鬥相傷,
 廆怒,遣信謂渾曰:「先公處分,與兄異部,牧馬何不相遠,而致鬥爭相傷?」渾曰:「馬是畜生,食草飲水,春氣發動,所以致鬥。斗在於馬,而怒及人邪?乖別甚易,今當去汝萬里。」於是擁馬西行,日移一頓,頓八十里。經數頓,廆悔悟,深自咎責,遣舊父老及長史乙那樓追渾,令還。渾曰:「我乃祖以來,樹德遼右,又卜筮之言,先公有二子,福胙並流子孫。我是卑庶,理無並大,今以馬致別,殆天所啟。諸君試擁馬令東,馬若還東,我當相隨去。」樓喜拜曰:「處可
 寒。」虜言「處可寒」,宋言爾官家也。即使所從二千騎共遮馬令回,不盈三百步,炎然悲鳴突走,聲若頹山。如是者十餘輩,一向一遠。樓力屈,又跪曰:「可寒,此非復人事。」渾謂其部落曰:「我兄弟子孫,並應昌盛,廆當傳子及曾孫玄孫,其間可百餘年,我乃玄孫間始當顯耳。」於是遂西附陰山。遭晉亂,遂得上隴。後廆追思渾,作《阿干之歌》。鮮卑呼兄為「阿干」。廆子孫竊號,以此歌為輦後大曲。



 渾既上隴,出罕開、西零。西零,今之西平郡;罕開,今桴罕縣。自
 桴罕以東千餘里,暨甘松,西至河南,南界昂城、龍涸。自洮水西南,極白蘭,數千里中,逐水草,廬帳居,以肉酪為糧。西北諸雜種謂之為阿柴虜。



 渾年七十二死,有子六十人,長吐延嗣。吐延身長七尺八寸,勇力過人,性刻暴,為昂城羌酋姜聰所刺;劍猶在體,呼子葉延,語其大將絕拔渥曰:「吾氣絕,棺斂訖,便遠去保白蘭。白蘭地既險遠,又土俗懦弱,易為控御。葉延小,意乃欲授與餘人,恐倉卒終不能相制。今以葉延付汝,汝竭股肱之力以輔
 之,孺子得立,吾無恨矣!」抽劍而死。嗣位十三年,年三十五,有子十二人。



 葉延少而勇果,年十歲,縛草為人,號曰姜聰,每旦輒射之,射中則喜,不中則號叫泣涕。其母曰:「仇賊諸將已屠膾之,汝年小,何煩朝朝自苦如此!」葉延嗚咽不自勝。答母曰:「誠知無益,然葉延罔極之心,不勝其痛耳。」性至孝,母病,三日不能食,葉延亦不食。頗視書傳,自謂曾祖弈洛韓始封昌黎公,曰:「吾為公孫之子,案禮,公孫之子,得氏王父字。」命姓為吐谷渾氏。嗣立二十
 三年,年三十三。有子四人。



 長子碎奚立。碎奚性純謹,三弟專權,碎奚不能制,諸大將共誅之。碎奚憂哀不復攝事,遂立子視連為世子,委之事,號曰:「莫賀郎」。「莫賀」,宋言父也。



 碎奚遂以憂死。在位二十五年,年四十一。有子六人。子視連以父憂卒,不遊娛,不酣宴。在位十五年,年四十二。有子二人,長曰視羆,次烏紇提。視羆嗣立十一年,年四十二;子樹洛干等並小,弟烏紇提立。紇提立八年,年三十五。視羆子樹洛干立,自稱車騎將軍,義熙初也。



 樹洛干死,弟阿豺自稱驃騎將軍。譙縱亂蜀,阿豺遣其從子西彊公吐谷渾敕來泥拓土至龍涸、平康。少帝景平中,阿豺遣使上表獻方物。詔曰:「吐谷渾阿豺介在遐表,募義可嘉,宜有寵任。今酬其來款,可督塞表諸軍事、安西將軍、沙州刺史、澆河公。」未及拜受,太祖元嘉三年,又詔加除命。未至而阿犲死,弟慕璝立。



 六年,表曰:「大宋應運,四海宅心,臣亡兄阿犲慕義天朝,款情素著。去年七月五日,謁者董湛至,宣傳明詔,顯授榮爵,而臣私門
 不幸,亡兄見背。臣以懦弱,負荷後任,然天恩所報,本在臣門,若更反覆,懼停信命。輒拜受寵任,奉遵上旨,伏願詳處,更授章策。」七年,詔曰:「吐谷渾慕璝兄弟慕義,至誠可嘉,宜授策爵,以甄忠款。可督塞表諸軍事、征西將軍、沙州刺史、隴西公。」



 先是晉末,金城東允街縣胡人乞伏乾歸擁部眾據洮河、罕開,自號隴西公。乾歸死,子熾磐立,遣使詣晉朝歸順,以為使持節、都督河西諸軍事、平西將軍,公如故。高祖即位,進號安西大將軍。熾磐死,子
 茂蔓立。慕璝前後屢遣軍擊,茂蔓率部落東奔隴右,慕璝據有其地。是歲,赫連定於長安為索虜拓跋燾所攻,擁秦戶口十餘萬西次罕開,欲向涼州。慕璝距擊,大破之,生擒定。燾遣使求,慕璝以定與之。九年,慕璝遣司馬趙敘奉貢獻,并言二萬人捷。太祖加其使持節、散騎常侍、都督西秦河沙三州諸軍事、征西大將軍、西秦河二州刺史、領護羌校尉,進爵隴西王。弟慕延為平東將軍,慕璝兄樹洛干子拾寅為平北將軍,阿豺子煒代鎮軍
 將軍。



 詔慕璝南國將士,昔沒在佛佛者,並悉致。慕璝遣送朱昕之等五十五戶,一百五十四人。



 慕璝死,弟慕延立,遣使奉表。十五年,除慕延使持節、散騎常侍、都督西秦河沙三州諸軍事、征西大將軍、領護羌校尉、西秦河二州刺史、隴西王。十六年,改封河南王。其年,以拾虔弟拾寅為平西將軍,慕延庶長子繁暱為撫軍將軍,慕延嫡子瑍為左將軍、河南王世子。十九年,追贈阿豺本號安西、秦沙三州諸軍事、沙州刺史、領護羌校尉、隴西王。
 索虜拓跋燾遣軍擊慕延,大破之,慕延率部落西奔白蘭,攻破於闐國。慮虜復至,二十七年,遣使上表云:「若不自固者,欲率部曲入龍涸越巂門。」并求牽車,獻烏丸帽、女國金酒器、胡王金釧等物。太祖賜以牽車,若虜至不自立,聽入越巂。虜竟不至也。



 慕延死,拾寅自立。二十九年,以拾寅為使持節、督西秦河沙三州諸軍事、安西將軍、領護羌校尉、西秦河二州刺史、河南王。拾寅東破索虜,加開府儀同三司。



 世祖大明五年,拾寅遣使獻善舞
 馬,四角羊。皇太子、王公以下上《舞馬歌》者二十七首。太宗泰始三年,進號征西大將軍。五年,拾寅奉表獻方物,以弟拾皮為平西將軍、金城公。前廢帝又進號車騎大將軍。其國西有黃沙,南北一百二十里,東西七十里,不生草木,沙州因此為號。屈真川有鹽池,甘谷嶺北有雀鼠同穴,或在山嶺,或在平地,雀色白,鼠色黃,地生黃紫花草,便有雀鼠穴。白蘭土出黃金、銅、鐵。其國雖隨水草,大抵治慕賀川。



 史臣曰:吐谷渾逐草依泉,擅強塞表,毛衣肉食,取資佃畜,而錦組繒紈,見珍殊俗,徒以商譯往來,故禮同北面。自昔哲王,雖存柔遠,要荒回隔,禮文弗被,大不過子,義著《春秋》。晉、宋垂典,不修古則,遂爵班上等,秩擬臺光。辮發稱賀,非尚簪冕,言語不通,寧敷袞職。雖復苞篚歲臻,事惟賈道,金罽氈毦,非用斯急,送迓煩擾,獲不如亡。若令肅慎年朝,越裳歲饗,固不容以異見書,取高前策。聖人謂之荒服,此言蓋有以也。



\end{pinyinscope}