\article{卷四十六列傳第六 趙倫之 到彥之闕 王懿 張邵}

\begin{pinyinscope}

 趙倫之,字幼成,下邳僮人也。孝穆皇后之弟。幼孤貧,事母以孝稱。武帝起兵,以軍功封閬中縣五等侯,累遷雍州刺史。武帝北伐,倫之遣順陽太守傅弘之、扶風太守
 沈田子出嶢柳,大破姚泓於藍田。及武帝受命,以佐命功,封霄城縣侯,安北將軍,鎮襄陽。少帝即位,徵拜護軍。元嘉三年,拜鎮軍將軍,尋遷左光祿大夫,領軍將軍。



 倫之雖外戚貴盛,而以儉素自處。性野拙,人情世務,多所不解。久居方伯,頗覺富盛,入為護軍,資力不稱,以為見貶。光祿大夫范泰好戲謂曰:「司徒公缺,必用汝老奴。我不言汝資地所任,要是外戚高秩次第所至耳。」倫之大喜,每載酒肴詣泰。五年,卒。子伯符嗣。



 伯符,字潤遠。少好弓馬。倫之在襄陽,伯符為竟陵太守。時竟陵蠻屢為寇,伯符征討,悉破之,由是有將帥之稱。後為寧遠將軍,總領義徒,以居宮城北,每有火起及賊盜,輒身貫甲胄,助郡縣赴討,武帝甚嘉之。文帝即位,累遷徐、兗二州刺史。為政苛暴,吏人畏之若豺虎,然而寇盜遠竄,無敢犯境。元嘉十八年,徵為領軍將軍。先是,外監不隸領軍,宜相統攝者,自有別詔,至此始統領焉。二十一年,轉豫州刺史。明年,為護軍將軍,復為丹陽尹。在
 郡嚴酷,吏人苦之,或至委叛被錄赴水而死;典筆吏取筆不如意,鞭五十。子倩,尚文帝第四女海鹽公主。



 初,始興王濬以潘妃之寵,故得出入後宮,遂與公主私通。及適倩,倩入宮而怒,肆詈搏擊,引絕帳帶。事上聞,有詔離婚,殺主所生蔣美人,伯符慚懼發病卒。謚曰肅。傳國至孫勖,齊受禪,國除。



 王懿,字仲德,太原祁人。自言漢司徒允弟幽州刺史懋七世孫也。祖宏,事石季龍;父苗,事苻堅,皆為二千石。



 仲
 堅德少沈審,有意略,通陰陽,解聲律。苻氏之敗,仲德年十七,與兄睿同起義兵,與慕容垂戰,敗;仲德被重創走,與家屬相失。路經大澤,不能前,困臥林中。忽有青衣童兒騎牛行,見仲德,問曰:「食未?」仲德告飢。兒去,頃之復來,攜食與之。仲德食畢欲行,會水潦暴至,莫知所如。有一白狼至前,仰天而號,號訖,銜仲德衣,因渡水;仲德隨之,獲濟,與睿相及。渡河至滑臺,復為翟遼所留,使為將帥。積年,仲德欲南歸,乃奔太山,遼遣騎追之急,夜行,忽有炬
 火前導,仲德隨之,行百許里,乃免。



 晉太元末,徙居彭城。兄弟名犯晉宣、元二帝諱,並以字稱。睿字元德。北土重同姓,謂之骨肉,有遠來相投者,莫不竭力營贍;若不至者,以為不義,不為鄉里所容。仲德聞王愉在江南,是太原人,乃往依之;愉禮之甚薄,因至姑孰投桓玄。



 值玄篡,見輔國將軍張暢,言及世事,仲德曰:「自古革命,誠非一族,然今之起者,恐不足以成大事。」



 元德果敢有智略,武帝甚知之,告以義舉,使於都下襲玄。仲德聞其謀,謂元
 德曰:「天下之事,不可不密,應機務速,不在巧遲。玄每冒夜出入,今若圖之,正須一夫力耳。」事泄,元德為玄所誅,仲德奔竄。會義軍克建業,仲德抱元德子方回出候武帝,帝於馬上抱方回與仲德相對號泣,追贈元德給事中,封安復縣侯,以仲德為中兵參軍。



 武帝伐廣固,仲德為前鋒,大小二十餘戰,每戰輒克。及盧循寇逼,敗劉毅於桑落,帝北伐始還,士卒創痍,堪戰者可數千人。賊眾十萬,舳艫百里,奔敗而歸者,咸稱其雄。眾議並欲遷都,
 仲德正色曰:「今天子當陽而治,明公命世作輔,新建大功,威震六合。妖賊豕突,乘我遠征,既聞凱入,將自奔散。今自投草間,則同之匹夫;匹夫號令,何以威物?義士英豪,當自求其主爾。此謀若行,請自此辭矣。」帝悅之,以仲德屯越城。及賊自蔡洲南走,遣仲德追之。賊留親黨范崇民五千人,高艦百餘,城南陵。仲德攻之,大破崇民,焚其舟艦,收其散卒,功冠諸將,封新淦縣侯。義熙十二年北伐,進仲德征虜將軍,加冀州刺史,為前鋒諸軍事。



 冠
 軍將軍檀道濟、龍驤將軍王鎮惡向洛陽,寧朔將軍劉遵考、建武將軍沈林子出石門,寧朔將軍朱超石、胡籓向半城,咸受統於仲德。仲德率龍驤將軍朱牧、寧遠將軍竺靈秀、嚴綱等開鉅野入河,乃總眾軍,進據潼關。長安平,以仲德為太尉咨議參軍。



 武帝欲遷都洛陽,眾議咸以為宜。仲德曰:「非常之事,常人所駭。今暴師日久,士有歸心,固當以建業為王基,俟文軌大同,然後議之可也。」帝深納之,使衛送姚泓先還彭城。武帝受命,累遷徐
 州刺史,加都督。



 元嘉三年,進號安北將軍,與到彥之北伐,大破虜軍。諸軍進屯靈昌津。司、兗既定,三軍咸喜,仲德獨有憂色,曰:「胡虜雖仁義不足,而凶狡有餘,今斂戈北歸,并力完聚,若河冰冬合,豈不能為三軍之憂!」十月,虜於委粟津渡河,進逼金墉,虎牢、洛陽諸軍,相繼奔走。彥之聞二城不守,欲焚舟步走,仲德曰:「洛陽既陷,則虎牢不能獨全,勢使然也。今賊去我千里,滑臺猶有彊兵,若便舍舟奔走,士卒必散。且當入濟至馬耳谷口,更詳
 所宜。」乃回軍沿濟南歷城步上,焚舟棄甲,還至彭城。仲德與彥之並免官。尋與檀道濟救滑臺,糧盡而歸。



 九年,又為鎮北將軍、徐州刺史。明年,加領兗州刺史。仲德三臨徐州,威德著於彭城,立佛寺作白狼、童子像於塔中,以河北所遇也。十三年,進號鎮北大將軍。十五年,卒,謚曰桓侯。亦於廟立白狼、童子壇,每祭必祠之。子正脩嗣,為家僮所殺。



 張邵,字茂宗,會稽太守裕之弟也。初為晉琅邪內史王
 誕龍驤府功曹,桓玄徙誕於廣州,親故咸離棄之,惟邵情意彌謹,流涕追送。時變亂饑饉,又饋送其妻子。



 桓玄篡位,父敞先為尚書,以答事微謬,降為廷尉卿。及武帝討玄,邵白敞表獻誠款,帝大說,命署其門曰:「有犯張廷尉者,以軍法論。」後以敞為吳郡太守。



 王謐為揚州,召邵為主簿。劉毅為亞相,愛才好士,當世莫不輻水奏,獨邵不往。



 或問之,邵曰:「主公命世人傑,何煩多問。」劉穆之聞以白,帝益親之,轉太尉參軍,署長流賊曹。盧循寇迫京師,
 使邵守南城。時百姓臨水望賊,帝怪而問邵,邵曰:「若節鉞未反,奔散之不暇,亦何能觀望。今當無復恐耳。」尋補州主簿。



 邵悉心政事,精力絕人。及誅劉籓,邵時在西州直廬,即夜誡眾曹曰:「大軍當大討,可各脩舟船倉庫,及曉取辦。」旦日,帝求諸簿署,應時即至;怪問其速,諸曹答曰:「昨夜受張主簿處分。」帝曰:「張邵可謂同我憂慮矣。」九年,世子始開征虜府,補邵錄事參軍,轉號中軍,遷咨議參軍,領記室。十二年,武帝北伐,邵請見,曰:「人生危脆,必
 當遠慮。穆之若邂逅不幸,誰可代之?尊業如此,茍有不諱,事將如何?」帝曰:「此自委穆之及卿耳。」青州刺史檀祗鎮廣陵,時滁州結聚亡命,祗率眾掩之。劉穆之恐以為變,將發軍。邵曰:「檀韶據中流,道濟為軍首,若疑狀發露,恐生大變。宜且遣慰勞,以觀其意。」既而祗果不動。及穆之卒,朝廷恇懼,便欲發詔以司馬徐羨之代之,邵對曰:「今誠急病,任終在徐,且世子無專命,宜須北咨。」信反,方使世子出命曰:「朝廷及大府事,悉咨徐司馬,其餘啟還。」
 武帝重其臨事不撓,有大臣體。十四年,以世子鎮荊州,邵諫曰:「儲貳之重,四海所系,不宜處外,敢以死請。」從之。



 文帝為中郎將、荊州刺史,以邵為司馬,領南郡相,眾事悉決於邵。武帝受命,以佐命功,封臨沮伯。分荊州立湘州,以邵為刺史。將署府,邵以為長沙內地,非用武之國,置署妨人,乖為政要。帝從之。謝晦反,遺書要邵,邵不發函,馳使呈帝。



 元嘉五年,轉征虜將軍,領寧蠻校尉、雍州刺史,加都督。初,王華與邵有隙,及華參要,親舊為之危
 心。邵曰:「子陵方弘至公,必不以私仇害正義。」是任也,華實舉之。及至襄陽,築長圍,修立隄堰,開田數千頃,郡人賴之富贍。丹、淅二川蠻屢為寇,邵誘其帥,因大會誅之,悉掩其徒黨。既失信群蠻,所在並起,水陸斷絕。子敷至襄陽定省,當還都,群蠻伺欲取之。會蠕蠕國遣使朝貢,賊以為敷,遂執之,邵坐降號揚烈將軍。



 江夏王義恭鎮江陵,以邵為撫軍長史,持節、南蠻校尉。坐在雍州營私蓄取贓貨二百四十五萬,下廷尉,免官,削爵土。後為吳
 興太守,卒,追復爵邑,謚曰簡伯。邵臨終,遺命祭以菜果,葦席為轜車,諸子從焉。子敷、演、鏡,有名於世。



 敷字景胤。生而母亡,年數歲,問知之,雖童蒙,便有感慕之色。至十歲許,求母遺物,而散施已盡,唯得一扇,乃緘錄之。每至感思,輒開笥流涕。見從母,悲咸嗚咽。性整貴,風韻端雅,好玄言,善屬文。初,父邵使與南陽宗少文談《繫》、《象》,往復數番,少文每欲屈,握麈尾歎曰:「吾道東矣。」於是名價日重。武帝聞其美,召見奇之,曰:「真千里駒也。」以為世子中
 軍參軍,數見接引。累遷江夏王義恭撫軍記室參軍。義恭就文帝求一學義沙門,會敷赴假江陵,入辭,文帝令以後車載沙門往,謂曰:「道中可得言晤。」敷不奉詔,上甚不說。遷正員中書郎。敷小名查,父邵小名梨,文帝戲之曰:「查何如梨?」敷曰:「梨為百果之宗,查何可比。」



 中書舍人狄當、周赳並管要務,以敷同省名家,欲詣之。赳曰:「彼恐不相容接,不如勿往。」當曰:「吾等並已員外郎矣,何憂不得共坐。」敷先設二床,去壁三四尺,二客就席,敷呼左右
 曰:「移我遠客!」赳等失色而去。其自標遇如此。



 善持音儀,盡詳緩之致,與人別,執手曰:「念相聞。」餘響久之不絕。張氏後進皆慕之,其源起自敷也。遷黃門侍郎、始興王濬後將軍司徒左長史。未拜,父在吳興亡,成服凡十餘日,方進水漿,葬畢,不進鹽菜,遂毀瘠成疾。伯父茂度每譬止之,敷益更感慟,絕而復續。茂度曰:「我比止汝,而乃益甚。」自是不復往,未期年而卒。孝武即位,旌其孝道,追贈侍中,改其所居為孝張里。



 敷弟柬,襲父封,位通直郎。柬
 有勇力,手格猛獸,元凶以為輔國將軍。孝武至新亭,柬出奔,墜淮死。子式嗣。



 暢字少微,邵兄偉之子也。偉少有操行,為晉琅邪王國郎中令,從王至洛,還京都,武帝封藥酒一罌付偉,令密加鴆毒,受命於道,自飲而卒。



 暢少與從兄敷、演、敬齊名,為後進之秀。起家為太守徐佩之主簿,佩之被誅,暢馳出奔赴,制服盡哀,時論美之。弟牧嘗為猘犬所傷,醫者云食楎蟆可療,牧難之。暢含笑先嘗,牧因此乃食,由是
 遂愈。累遷太子中庶子。



 孝武鎮彭城,暢為安北長史、沛郡太守。元嘉二十七年,魏主托跋燾南征,太尉江夏王義恭統諸軍出鎮彭城。虜眾近城數十里,彭城眾力雖多,而軍食不足,義恭欲棄彭城南歸,計議彌日不定。時歷城眾少食多,安北中兵參軍沈慶之議欲以車營為函箱陣,精兵為外翼,奉二王及妃媛直趨歷城,分城兵配護軍將軍蕭思話留守。



 太尉長史何勖不同,欲席卷奔鬱洲,自海道還都。二議未決,更集群僚議之。暢曰:「若
 歷城、鬱洲可至,下官敢不高贊。今城內乏食,人無固心,但以關扃嚴密,不獲走耳。若一搖動,則潰然奔散,雖欲至所在,其可得乎!今食雖寡,名朝夕未至窘乏,豈可舍萬全之術,而即危亡之道。此計必行,下官請以頸血污君馬跡!」孝武聞暢議,謂義恭曰:「張長史言,不可違也。」義恭乃止。



 魏主既至,登城南亞父塚,於戲馬臺立氈屋。先是,隊主蒯應見執,其日晡時,遣送應至小市門,致意求甘蔗及酒。孝武遣送酒二器,甘蔗百挺。求駱駝。明日,魏
 主又自上戲馬臺,復遣使至小市門,求與孝武相見,遣送駱駝,并致雜物,使於南門受之。暢於城上與魏尚書李孝伯語,孝伯問:「君何姓?」答曰:「姓張。」



 孝伯曰:「張長史乎?」暢曰:「君何得見識?」孝伯曰:「君名聲遠聞,足使我知。」城內有具思者,嘗在魏,義恭使視,知是孝伯,乃開門餉物。魏主又求酒及甘橘,孝武又致螺盃雜物,南土所珍。魏主復令孝伯傳語曰:「魏主有詔借博具。」



 暢曰:「博具當為申致,有詔之言,正可施於彼國,何得施之於此?」孝伯曰:「以
 鄰國之臣耳。」孝伯又言:「太尉、鎮軍,久闕南信,殊當憂邑。若遣信,當為護送。」暢曰:「此中間道甚多,亦不須煩魏。」孝伯曰:「亦知有水路,似為白賊所斷。」暢曰:「君著白衣,故號白賊也。」孝伯笑曰:「今之白賊,亦不異黃巾、赤眉,但不在江南耳。」又求博具,俄送與。魏主又遣送氈及九種鹽并胡豉,云:「此諸鹽,各有宜。白鹽是魏主所食;黑者療腹脹氣滿,刮取六銖,以酒服之;胡鹽療目痛。柔鹽不用食,療馬脊創;赤鹽、駁鹽、臭鹽、馬齒鹽四種,並不中食。



 胡豉亦
 中啖。」又求黃甘,并云:「魏主致意太尉、安北,何不遣人來問,觀我儀貌,察我為人。」暢又宣旨答曰:「魏主形狀才力,久為來往所見。李尚書親自銜命,不忍彼此不盡,故不復遣。」又云「魏主恨向所送馬殊不稱意,安北若須大馬,當送之,脫須蜀馬,亦有佳者。」暢曰:「安北不乏良駟,送在彼意,此非所求。」



 義恭又送炬燭十挺,孝武亦致錦一匹。又曰:「知更須黃甘,若給彼軍,即不能足;若供魏主,未當乏絕,故不復致。」孝伯又曰:「君南土膏粱,何為著屩?君且
 如此,將士云何?」暢曰:「膏粱之言,誠以為愧。但以不武,受命統軍,戎陣軍間,不容緩服。」魏主又遣就二王借箜篌、琵琶等器及棋子,孝伯足詞辯,亦北土之美。



 暢隨宜應答,甚為敏捷,音韻詳雅,魏人美之。



 時魏聲云當出襄陽,故以暢為南譙王義宣司空長史、南郡太守。元凶弒逆,義宣發哀之日,即便舉兵。暢為元佐,舉哀畢,改服著黃褲褶,出射堂簡人,音儀容止,眾皆矚目,見者皆為盡命。事平,徵為吏部尚書,封夷道縣侯。及義宣有異圖,蔡超
 等以暢人望,勸義宣留之,乃解南蠻校尉以授暢,加冠軍將軍,領丞相長史。



 暢遣門生荀僧寶下都,因顏竣陳義宣釁狀。僧寶有私貨,止巴陵不時下。會義宣起兵,津路斷絕,遂不得前。義宣將為逆,使嬖人翟靈寶告暢,暢陳必無此理,請以死保之。靈寶還白義宣,云暢必不可回,請殺以徇眾,賴丞相司馬竺超民得免。進號撫軍,別立軍部,以收人望。暢雖署文檄,飲酒常醉,不省其事。及義宣敗於梁山,暢為軍人所掠,衣服都盡。遇右將軍王
 玄謨乘輿出營,暢已得敗衣,遂排玄謨上輿,玄謨甚不悅。諸將請殺之,隊主張榮救之得免。執送都下,付廷尉,見原。



 起為都官尚書,轉侍中。孝武宴朝賢,暢亦在坐。何偃因醉曰:「張暢信奇才也,與義宣作賊,而卒無咎。茍非奇才,安能致此!」暢曰:「太初之時,誰黃其閣?」帝曰:「何事相苦。」初,尚之為元兇司空,及義師至新林門,人皆逃,尚之父子共洗黃閣,故暢以此譏之。孝建二年,出為會稽太守。卒,謚曰宣。暢愛弟子輯,臨終遺命與輯合墳,時議非之。



 弟悅,亦有美稱,歷侍中、臨海王子頊前將軍長史、南郡太守。晉安王子勛建偽號,召拜為吏部尚書,與鄧琬共輔偽政。及事敗,悅殺琬歸降,復為太子中庶子。



 後拜雍州刺史。泰始六年,明帝於巴郡置三巴校尉,以悅補之,加持節、輔師將軍,領巴郡太守。未拜,卒。



 暢子浩,官至義陽王昶征北咨議參軍。浩弟淹,黃門郎,封廣晉縣子,太子右衛率,東陽太守。逼郡吏燒臂照佛,百姓有罪,使禮佛贖刑,動至數千拜。免官禁錮。起為光祿勳,與晉安王
 子勛同逆,軍敗見殺焉。



 臣穆等案《高氏小史》,《趙倫之傳》下有《到彥之傳》,而此書獨闕。約之史法,諸帝稱廟號,而謂魏為虜。今帝稱帝號,魏稱魏主,與《南史》體同,而傳末又無史臣論,疑非約書。然其辭差與《南史》異,故特存焉。



\end{pinyinscope}