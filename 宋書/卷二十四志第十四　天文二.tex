\article{卷二十四志第十四 天文二}

\begin{pinyinscope}

 晉惠帝元康二年二月,天西北大裂。按劉向說:「天裂,陽不足;地動,陰有餘。」是時人主拱默,婦后專制。



 元康三年四月,熒惑守太微六十日。占曰:「諸侯三公謀
 其上,必有斬臣。」



 一曰:「天子亡國。」是春,太白守畢,至是百餘日。占曰:「有急令之憂。」一曰:「相亡。又為邊境不安。」是年,鎮、歲、太白三星聚于畢昴。占曰:「為兵喪。畢昴,趙地也。」後賈后陷殺太子,趙王廢后,又殺之,斬張華、裴頠,遂篡位,廢帝為太上皇。天下從此遘亂連禍。



 元康五年四月,有星孛于奎,至軒轅、太微,經三台、大陵。占曰:「奎為魯,又為庫兵,軒轅為後宮,太微天子廷,三台為三司,大陵有積屍死喪之事。」明年,武庫火,西羌反。後
 五年,司空張華遇禍,賈后廢死,魯公賈謐誅。又明年,趙王儉篡位。於是三王興兵討倫,士民戰死十餘萬人。



 元康六年六月丙午夜,有枉矢自斗魁東南行。按占曰:「以亂伐亂。北斗主執殺,出斗魁,居中執殺者不直象也。」十月,太白晝見。後趙王殺張、裴,廢賈后,以理太子之冤,因自篡盜,以至屠滅。以亂伐亂,兵喪臣彊之應也。



 元康九年二月,熒惑守心。占曰:「王者惡之。」八月,熒惑入羽林。占曰:「禁兵大起。」後二年,惠帝見廢為太上皇,俄而
 三王起兵討倫。倫悉遣中軍兵,相距累月。



 晉惠帝永康元年三月,妖星見南方,中台星坼,太白晝見。占曰:「妖星出,天下大兵將起。台星失常,三公憂。太白晝見為不臣。」是月,賈后殺太子,趙王倫尋廢殺后及司空張華,又廢帝自立。於是三王並起,迭總大權。永康元年五月,熒惑入南斗。占曰:「宰相死,兵大起。斗又吳分也。」是時趙王倫為相,明年篡位,三王興師誅之。太安二年,石冰破揚州。
 永康元年八月,熒惑入箕。占曰:「人主失位,兵起。」十二月,彗出牽牛之西,指天市。占曰:「牛者七政始,彗出之,改元易號之象也。」天市一名天府,一名天子禖,帝座在其中。明年,趙王篡位,改元,尋為大兵所滅。



 永康二年二月,太白出西方,逆行入東井。占曰:「國失政,臣為亂。」四月,彗星見齊分。占曰:「齊有兵喪。」是時齊王冏起兵討趙王倫。倫滅,冏擁兵不朝,專權淫侈,明年誅死。



 晉惠帝永寧元年,自正月至于閏月,五星互經天。《星傳》
 曰:「日陽,君道也。星陰,臣道也。日出則星亡,臣不得專也。晝而星見午上者為經天,其占為不臣,為更王。今五星悉經天,天變所未有也。」石氏說曰:「辰星晝見,其國不亡,則大亂。」是後台鼎方伯,互秉大權。二帝流亡,遂至六夷彊,迭據華夏,亦載籍所未有也。永寧元年五月,太白晝見。占同前條。七月,歲星守虛危。占曰:「木守虛危,有兵憂。」一曰:「守虛飢;守危徭役煩,下屈竭。」辰星入太微。占曰:「為內亂。」一曰:「群臣相殺。」太白守右
 掖門。占曰:「為兵,為亂,為賊。」



 八月戊午,鎮星犯左執法,又犯上相。占曰:「上相憂。」熒惑守昴。占曰:「趙、魏有災。」辰星守輿鬼。占曰:「秦有災。」九月丁未,月犯左角。占曰:「人主憂。」一曰:「左將軍死,天下有兵。」



 二年四月癸酉,歲星晝見。占曰:「為臣彊。」十月,熒惑太白鬥于虛危。占曰:「大兵起,破軍殺將。虛危,又齊分也。」十二月,熒惑襲太白于營室。占曰:「天下兵起,亡君之戒。」一曰:「易相。」初齊王冏定京都,因留輔政,遂專慠無君。是月,成
 都、河間檄長沙王乂討之。冏、乂交戰,攻焚宮闕。冏兵敗夷滅,又殺其兄上軍將軍實以下二十餘人。太安二年,成都攻長沙,於是公私飢困,百姓力屈。



 晉惠帝太安二年二月,太白入昴。占曰:「天下擾,兵大起。」三月,彗星見東方,指三台。占曰:「兵喪之象。三台為三公。」七月,熒惑入東井。占曰:「兵起國亂。」是秋,太白守太微上將。占曰:「上將將以兵亡。」是年冬,成都、河間攻洛陽。三年正月,東海王越執長沙王乂,張方又殺之。
 太安二年八月,長沙王奉帝出距二王,庚午,舍于玄武館。是日,天中裂為二,有聲如雷。三占同元康,臣下專僭之象也。是時長沙王擅權,後成都、河間、東海又迭專威命,是其應也。



 太安二年十一月辛巳,有星晝隕中天,北下有聲如雷。按占:「名曰熒首,營首所在,下有大兵流血。」明年,劉淵、石勒攻略并州,多所殘滅。王浚起燕、代,引鮮卑攻掠鄴中,百姓塗地。有聲如雷,怒之象也。



 太安二年十一月庚辰,歲星入月中。占曰:「國有逐相。」十二月壬寅,太白犯月。占曰:「天下有兵。」太安三年正月己卯,月犯太白,占同青龍。熒惑入南斗,占同永康。是月,熒惑又犯歲星。占曰:「有大戰。」七月,左衛將軍陳率眾奉帝伐成都,六軍敗績,兵逼乘輿。九月,王浚又攻成都于鄴,鄴潰,成都王由是喪亡。帝還洛,張方脅如長安。是時天下盜賊群起,張昌尤盛。後二年,惠帝崩。



 晉惠帝永興元年五月,客星守畢。占曰:「天子絕嗣。」一曰:「
 大臣有誅。」



 七月庚申,太白犯角、亢,經房、心,歷尾、箕。九月,入南斗。占曰:「犯角,天下大戰;犯亢,有大兵,人君憂;入房、心,為兵喪;犯尾,將軍與民人為變;犯箕,女主憂。」一曰:「天下亂。入南斗,有兵喪。」一曰:「將軍為亂。」其所犯守,又兗、豫、幽、冀、揚州之分也。是年七月,有蕩陰之役。九月,王浚殺幽州刺史和演,攻鄴,鄴潰。於是兗、豫為天下兵衝。陳敏又亂揚土,劉淵、石勒、李雄等並起微賤,跨有州郡。皇后羊氏數被幽廢。光熙元年,惠帝崩,終無繼嗣。



 永興元年七月乙丑,星隕有聲。二年十月,星又隕有聲。按劉向說,民去其土之象也。是後遂亡中夏。永興元年十二月壬寅夜,赤氣亙天,砰隱有聲。二年十月丁丑,赤氣見在北方,東西竟天。占曰:「並為大兵。砰隱有聲,怒之象也。」是後四海雲擾,九服交兵。



 永興二年四月丙子,太白犯狼星。占曰:「大兵起。」九月,歲星守東井。占曰:「有兵。井又秦分也。」是年,茍晞破公師籓,張方破范陽王虓,關西諸將攻河間王顒,顒奔走,東海
 王迎殺之。永興二年八月,星孛于昴、畢。占曰:「為兵喪。」昴、畢,又趙、魏分也。十月丁丑,有星孛于北斗。占曰:「璇璣更授,天子出走。」又曰:「彊國發兵,諸侯爭權。」是後皆有其應。明年,惠帝崩。



 晉惠帝光熙元年四月,太白失行,自翼入尾、箕。占曰:「太白失行而北,是謂返生。不有破軍,必有屠城。」五月,汲桑攻鄴,魏郡太守馮嵩出戰大敗,桑遂害東燕王騰,殺萬
 餘人,焚燒魏時宮室皆盡。光熙元年五月,枉矢西南流。占曰:「以亂伐亂之象也。」是時司馬越西破河間,奉迎大駕。尋收繆胤、何綏等,肆其無君之心,天下惡之。死而石勒焚其屍柩,是其應也。光熙元年九月丁未,熒惑守心。占曰:「王者惡之。」己亥,填星守房、心,又犯歲星。占曰:「土守房,多禍喪。守心,國內亂,天下赦。」又曰:「填與歲合為內亂。」是時司馬越秉權,終以無禮破滅,內亂之應也。十一月,惠帝崩,懷帝即位,大赦
 天下。



 光熙元年十二月癸未,太白犯填星。占曰:「為內兵,有大戰。」是後河間王為東海王越所殺。明年正月,東海王越殺諸葛玫等。五月,汲桑破馮嵩,殺東燕王。



 八月,茍晞大破汲桑。光熙元年十二月甲申,有白氣若虹,中天北下至地,夜見五日乃滅。占曰:「大兵起。」明年,王彌起青、徐,汲桑亂河北,毒流天下。



 孝懷帝永嘉元年九月辛亥,有大星自西南流于東北,小者如升相隨,天盡赤,聲如雷。占曰:「流星為貴使。」是年五月,汲桑殺東燕王騰,遂據河北。十一月,始遣和郁為征北將軍鎮鄴,而田甄等大破汲桑,斬于樂陵。於是以甄為汲郡太守,弟蘭鉅鹿太守。小星相隨,小將別帥之象也。司馬越忿魏郡以東,平原南,皆黨於桑,悉以賞甄等,於是侵略赤地,有聲如雷,怒之象也。永嘉元年十二月丁亥,星流震散。案劉向說:「天官列宿,
 在位之象,小星無名者,庶民之類。此百官庶民將流散之象也。」是後天下大亂,百官萬民,流移轉死矣。



 永嘉二年正月庚午,太白伏不見。二月庚子,始晨見東方。是謂當見不見,占同上條。其後破軍殺將,不可勝數。帝崩虜庭,中夏淪覆。



 永嘉三年正月庚子,熒惑犯紫微。占曰:「當有野死之王。又為火燒宮。」是時太史令高堂沖奏,乘輿宜遷幸,不然必無洛陽。五年六月,劉曜、王彌入京都,燒宮廟,帝崩于
 平陽。永嘉三年,鎮星久守南斗。占曰:「鎮星所居者,其國有福。」



 是時安東琅邪王始有揚土。其年十一月,地動,陳卓以為是地動應也。永嘉三年十二月乙亥,有白氣如帶出東南北方各二,起地至天,貫參伐。占曰:「天下大兵起。」



 四年三月,司馬越收繆胤、繆播等;又三方雲擾,攻戰不休。五年三月,司馬越死於寧平城,石勒攻破其眾,死者十餘萬人。六月,京
 都焚滅,帝劫虜庭。



 永嘉五年十月,熒惑守心。後二年,帝崩于虜庭。



 永嘉六年七月,熒惑、歲星、鎮星、太白聚牛女之間,裴回進退。按占曰:「牛,揚州分。」是後兩都傾覆,而元帝中興揚土,是其應也。愍帝建武元年五月癸未,太白熒惑合於東井。占曰:「金火合曰爍,為喪。」是時帝雖劫于平陽,天下猶未敢居其虛位,災在帝也。六月丁卯,太白犯太微。占曰:「兵入天子
 廷,王者惡之。」七月,愍帝崩于寇庭,天下行服大臨。



 晉元帝太興元年七月,太白犯南斗。占曰:「吳、越有兵,大人憂。」二年二月甲申,熒惑犯東井。占曰:「兵起,貴臣相戮。」八月己卯,太白犯軒轅大星。



 占曰:「後宮憂。」乙未,太白犯歲星,在翼。占曰:「為兵亂。」三年四月壬辰,枉矢出虛、危,沒翼、軫。占曰:「枉矢所觸,天下之所伐。翼、軫,荊州之分也。」



 五月戊子,太白入太微,又犯上將。占曰:「天子自將,上將誅。」六月丙辰,太白與歲星合于房。占曰:「為兵饑。」九月,太白
 犯南斗,占同元年。十月己亥,熒惑在東井,居五諸侯南,踟躕留止,積三十日。占曰:「熒惑守井二十日以上,大人憂;守五諸侯,諸侯有誅者。」十二月己未,太白入月,在斗。郭景純曰:「月屬坎,陰府法象也。太白金行而來犯之,天意若曰刑理失中,自毀其法也。」



 四年十二月丁亥,月犯歲星在房。占曰:「其國兵饑,民流亡。」永昌元年三月,王敦率江、荊之眾,來攻京都,六軍距戰,敗績。於是殺護軍將軍周顗、尚書令刁協,驃騎將軍劉隗出奔。四月,又殺湘
 州刺史譙王承、鎮南將軍甘卓。閏十二月,元帝崩。間一年,敦亦梟夷,枉矢觸翼之應也。十月,石他入豫州,略城父、霡二縣民以北,刺史祖約遣軍追之,為其所沒,遂退守壽春。



 明帝太寧三年正月,熒惑逆行入太微。占曰:「為兵喪,王者惡之。」閏八月,帝崩。咸和二年,蘇峻反,攻宮室,太后以憂逼崩,天子幽劫于石頭,遠近兵亂,至四年乃息。



 成帝咸和四年七月,有星孛于西北,二十三日滅。占曰:「
 為兵亂。」十二月,郭默殺江州刺史劉胤,荊州刺史陶侃討默,明年,斬之。是時,石勒又始僭號。



 咸和六年正月丙辰,月入南斗。占曰:「有兵。」一曰:「有大赦。」是月胡賊殺略婁、武進二縣民,於是遣戍中洲。明年,胡賊又略南沙、海虞民。是年正月,大赦,伐淮南,討襄陽,平之。咸和六年十一月,熒惑守胃、昴。占曰:「趙、魏有兵。」八年七月,石勒死,石虎自立,多所殘滅。是時雖勒、虎僭號,而其
 強弱常占於昴,不關太微紫宮也。



 咸和八年三月己巳,月入南斗,與六年占同。其年七月,石勒死,彭彪以譙,石生以長安,郭權以秦州,並歸從。於是遣督護高球率眾救彪,彪敗球退。又石虎、石斌攻滅生、權。咸康元年正月,大赦。咸和八年七月,熒惑入昴。占曰:「胡王死。」石虎多所攻滅。八月,月犯昴。占曰:「胡不安。」九年六月,月又犯昴。



 是時石弘雖襲勒位,而石虎擅威暴橫。十月,廢弘自立,遂幽殺
 之。



 咸和九年三月己亥,熒惑入輿鬼,犯積屍。占曰:「兵在西北,有沒軍死將。」



 四月,鎮西將軍、雍州刺史郭權始以秦州歸從,尋為石斌所滅,徙其眾於青、徐。



 晉成帝咸康元年二月己亥,太白犯昴。占曰:「兵起,歲大旱。」四月,石虎掠騎至歷陽。朝廷慮其眾也,加司徒王導大司馬,治兵動眾。又遣慈湖、牛渚、蕪湖三戍。五月乃罷。是時胡賊又圍襄陽,征西將軍庾亮遣寧距退之。六月,
 旱。咸康元年八月戊戌,熒惑入東井。占曰:「無兵兵起;有兵兵止。」是年夏,發眾列戍。加王導大司馬,以備胡賊。咸康元年三月丙戌,月入昴。占曰:「胡王死。」



 十一月,月犯昴。二年八月,月又犯昴。占同。咸和三年,石虎發眾七萬,四年二月,自襲段遼于薊,遼奔敗。又攻慕容皝於棘城,不剋引退。皝追之,殺數百人。



 虎留其將麻秋屯令支,皝破秋,并虜遼殺之。



 咸康二年正月辛巳,彗星夕見西方,在奎。占曰:「為兵喪。奎又為邊兵。」



 四年,石虎伐慕容皝不剋,皝追擊之,又破麻秋。時皝稱蕃,邊兵之應也。咸康二年正月辛卯,月犯房南第二星。占曰:「將相有憂。」五年七月,丞相王導薨。八月,太尉郗鑒薨。六年正月,征西大將軍庾亮薨。咸康二年九月庚寅,太白犯南斗,因晝見。占曰:「斗為宰相,又揚州分,金犯之,死喪象。晝見為不臣,又為兵喪。」



 三
 年,石虎僭稱天王。四年,虎滅段遼而敗於慕容皝。皝,國蕃臣。五年,王導薨。



 咸康三年六月辛未,有流星大如二斗魁,色青,赤光耀地,出奎中,沒婁北。



 案占為飢,五穀不藏。是月,大旱。咸康三年八月,熒惑入輿鬼,犯積屍。占曰:「貴人憂。」三年八月甲戌,月犯東井距星。占曰:「國有憂,將死。」三年九月戊子,月犯建星。占曰:「易相。」一曰:「大將死。」五年,丞相王導薨,庾冰代輔政。太尉郗鑒、征西大將軍庾亮薨。
 咸康三年十一月乙丑,太白犯歲星。占曰:「為兵飢。」四年二月,石虎破幽州,遷其人萬餘家。李壽殺李期。五年,胡眾五萬寇沔南,略七千餘家而去。又騎二萬圍陷邾城,殺略五千餘人。



 咸康四年四月己巳,太白晝見在柳。占曰:「為兵,為不臣。」七月乙巳,月掩太白。占曰:「王者亡地,大兵起。」明年,胡賊大寇沔南,陷邾城,豫州刺史毛寶、西陽太守樊峻皆棄城投江死。於是內外戒嚴,左衛桓監、匡術等諸軍至武
 昌,乃退。七年,慕容皝自稱為燕王。咸康四年五月戊午,熒惑犯右執法。占曰:「大臣死,執政者憂。」九月,太白犯右執法。案占,「五星災同,金火尤甚。」十一月戊子,太白犯房上星。占曰:「上相憂。」五年七月己酉,月犯房上星,亦同占。



 是月庚申,丞相王導薨。



 咸康五年四月辛未,月犯歲星,在胃。占曰:「國飢民流。」乙未,月犯畢距星。占曰:「兵起。」是夜,月又犯歲星,在昴。及冬,有沔南、邾城之敗,百姓流亡萬餘家。



 咸康六年二月庚午朔,流星大如斗,光耀地,出天市,西行入太微。占曰:「大人當之。」乙未,太白入月。占曰:「人主死。」四月甲午,月犯太白。占曰:「人主惡之。」八年六月,成帝崩。咸康六年三月甲寅,熒惑從行犯太微上將星。



 占曰:「上將憂。」四月丁丑,熒惑犯右執法。占曰:「執法者憂。」六月乙亥,月犯牽牛中央星。占曰:「大將憂。」是時尚書令何充為執法,有譴欲避其咎,明年,求為中書令。建元二年,庾冰薨,皆大將執政之應也。是歲正月,征西將軍庾亮薨。三
 月,而熒惑犯上將。九月,石虎大將夔安死。庾冰後積年方薨。豈冰能修德,移禍於夔安乎?咸康六年四月丙午,太白犯畢距星。占曰:「兵革起。」一曰:「女主憂。」六月乙卯,太白犯軒轅大星。占曰:「女主憂。」七年三月,皇后杜氏崩。



 咸康七年三月壬午,月犯房。占曰:「將相憂。」八年六月,熒惑犯房上第二星。占曰:「次相憂。」建元二年,車騎將軍江州刺史庾冰薨。是時驃騎將軍何充居內,冰為次相也。
 咸康七年四月己丑,太白入輿鬼。占曰:「兵革起。」五月,太白晝見。以晷度推之,非秦、魏,則楚也。占曰:「為臣彊,為有兵。」八月辛丑,月犯輿鬼。占曰:「人主憂。」八年六月,成帝崩。



 咸康八年八月壬寅,月犯畢赤星。占曰:「下犯上,兵革起。」十月,月又掩畢赤星,占同。己酉,太白犯熒惑。占曰:「大兵起。」其後庾翼大發兵謀伐胡,專制上流,朝廷憚之。



 康帝建元元年正月壬午,太白入昴。占曰:「趙地有兵。」又曰:「天下兵起。」



 四月乙酉,太白晝見。八月丁未,太白犯歲
 星。占曰:「有大兵。」是年,石虎殺其太子遂及其妻子徒屬二百餘人。又遣將劉寧寇沒狄道,又使將張舉將萬餘人屯薊東,謀慕容皝。建元元年十一月六日,彗星見亢,長七尺,尾白色。占曰:「亢為朝廷,主兵喪。」二年九月,康帝崩。建元元年,歲星犯天關。安西將軍庾翼與兄冰書曰:「歲星犯天關,占云:『關梁當澀。』比來江東無他故,江道亦不艱難;而石虎頻年再閉關不通信使,此復是天公憒憒
 無皁白之徵也。」



 建元二年閏月乙酉,太白犯斗。占曰:「為喪,天下受爵祿。」九月,康帝崩,太子立,大赦賜爵也。



 晉穆帝永和元年正月丁丑,月入畢。占曰:「兵大起。」戊寅,月犯天關。占曰:「有亂臣更天子之法。」五月辛巳,太白晝見,在東井。占曰:「為臣彊,秦有兵。」六月辛丑,入太微,犯屏西南。占曰:「輔臣有免罷者。」七、八月,月皆犯畢。占同正月。己未,月犯輿鬼。占曰:「大臣有誅。」九月庚戌,月又犯畢。



 是
 年初,庾翼在襄陽,七月,翼疾將終,輒以子爰之為荊州刺史,代己任;爰之尋被廢。明年,桓溫又輒率眾伐蜀,執李勢,送至京都。蜀本秦地也。



 永和二年二月壬子,月犯房上星。四月丙戌,月又犯房上星。占同前。八月壬申,太白犯左執法。是歲,司徒蔡謨被廢。



 永和三年正月壬午,月犯南斗第五星。占曰:「將軍死,近臣去。」五月壬申,月犯南斗第四星,因入魁。占曰:「有兵。」一
 曰:「有大赦。」六月,月犯東井距星。占曰:「將死,國有憂。」戊戌,月犯五諸侯。占曰:「諸侯有誅。」九月庚寅,太白犯南斗第五星。占曰:「為喪兵。」四年七月丙申,太白犯左執法。甲寅,月犯房。丁巳,月入南斗犯第二星。乙丑,太白犯左執法。占悉同上。十月甲戌,月犯亢。占曰:「兵起,軍將死」。十一月戊戌,犯上將星。三年六月,大赦。



 是月,陳逵征壽春,敗而還。七月,氐蜀餘寇反亂益土。九月,石虎伐涼州,不克。



 永和四年四月,太白入昴。五月,熒惑入婁,犯鎮星。七月,
 太白犯軒轅。占在趙,及為兵喪,女主憂。其年八月,石虎太子宣殺弟韜,宣亦死。五年正月,石虎僭稱皇帝,尋病死。是年,褚裒北伐喪眾,又尋薨,太后素服。六年正月,朝會廢樂。



 永和五年四月丁未,太白犯東井。占曰:「秦有兵。九月戊戌,太白犯左角。



 占曰:「為兵。」十月,月犯昴。占曰:「朝廷有憂,軍將死。」十一月乙卯,彗星見于亢,芒西向,色白,長一丈。占曰:「為兵喪。」是年八月,褚裒北征兵敗。



 十月,關中二十餘
 壁舉兵歸從,石遵攻沒南陽。十一月,冉閔殺石遵,又盡殺胡十餘萬人,於是中土大亂。十二月,褚裒薨。八年,劉顯、苻健、慕容俊並僭號。殷浩北伐敗,見廢。



 永和六年二月辛酉,月犯心大星。占曰:「大人憂。心豫州分也。」丁丑,月犯房。占曰:「將相憂。」三月戊戌,熒惑犯歲星。占曰:「為戰。」六月己丑,月犯昴。占同上。乙未,月犯五諸侯。占同三年。七月壬寅,月始出西方,犯左角。



 占曰:「大將軍死。」一曰:「天下有兵。」丁未,月犯箕。占曰:「軍將死。」



 丙寅,熒惑
 犯鉞星。占曰:「大臣有誅。」八月辛卯,月犯左角,太白晝見,在南斗。月犯右執法,占並同上。七年二月,太白犯昴,占同上。乙卯,熒惑輿鬼,犯積屍。占曰:「貴人憂。」五月乙未,熒惑犯軒轅大星。占曰:「女主憂。」太白入畢口,犯左股。占曰:「將相當之。」六月乙亥,月犯箕。丙子,月犯斗。丁丑,熒惑入太微,犯右執法。八月庚午,太白犯軒轅。戊子,太白犯右執法。占悉同上。



 七年,劉顯殺石祗及諸胡帥,中土大亂,戎、晉十萬數,各還舊土,互相侵略及疾疫死亡,能達者
 十二三。是年,桓溫輒以大眾求浮江入淮北伐,朝廷震懼。八年,豫州刺史謝尚討張遇,為苻雄所敗。殷浩北伐敗,被廢。十年,桓溫伐苻健,不克而還。



 永和八年三月戊戌,月犯軒轅大星。癸丑,月入南斗犯第二星。五月,月犯心星。四月癸酉,月犯房。六月辛巳,日未入,有流星如三斗魁,從辰巳上東南行。



 晷度推之,在箕、斗之間,蓋燕分也。案占為營首,營首之下,流血滂沲。七月壬子,歲星犯東井距星。占曰:「內亂兵起。」八月戊戌,
 熒惑入輿鬼。占曰:「忠臣戮死。」丙辰,太白入南斗,犯第四星。占曰:「將為亂。」一曰:「丞相免。」



 九年二月乙巳,入南斗,犯第三星。三月戊辰,月犯房。八月,歲星犯輿鬼東南星。



 占:「東南星主兵,兵起」。十二月,月在東井,犯歲星。占曰:「秦飢民流。」



 是時帝主幼沖,母后稱制,將相有隙,兵革連起。慕容俊僭稱大燕,攻伐無已,故災異數見,殷浩見廢也。



 永和十年正月乙卯,月食昴。占曰:「趙、魏有兵。」癸酉,填星奄鉞星。占曰:「斧鉞用。」二月甲申,月犯心大星。占曰:「王者
 惡之。」四月癸未,流星大如斗,色赤黃,出織女,沒造父,有聲如雷。占曰:「燕、齊有兵,民流。」戊午,月犯心大星。七月庚午,太白晝見。晷度推之,災在秦、鄭。九月辛酉,太白犯左執法。十一月,月奄填星,在輿鬼。占曰:「秦有兵。」十一年三月辛亥,月奄軒轅。占同上。四月庚寅,月犯牛宿南星。占曰:「國有憂。」八月己未,太白犯天江。占曰:「河津不通。」十二年六月庚子,太白晝見,在東井,占如上。己未,月犯鉞星。七月丁卯,太白犯填星,在柳。占曰:「周地有大兵。」八月癸
 酉,月奄建星。九月戊寅,熒惑入太微,犯西蕃上將星。十一月丁丑,熒惑犯太微東蕃上相。十年四月,桓溫伐苻健,破其堯柳眾軍。健壁長安,溫退。十二月,慕容恪攻齊。十二年八月,桓溫破姚襄於伊水,定周地。十一月,齊城陷,執段龕,殺三千餘人。永和末,鮮卑侵略河、冀,升平元年,慕容俊遂據臨漳,盡有幽、并、青、冀之地。緣河諸將漸奔散,河津隔絕矣。三年,會稽王以郗曇、謝萬敗績,求自貶三等。是時權在方伯,九服交兵,故譴象仍見。



 晉穆帝升平元年四月壬子,太白入輿鬼。丁亥,月奄東井南轅西頭第二星。占曰:「秦地有兵。」一曰:「將死。」六月戊戌,太白晝見,在軫,占同上。軫,楚分也。壬子,月犯畢。占曰:「為邊兵。」七月辛巳,熒惑犯天江。占曰:「河津不通。」十一月,歲星犯房。壬午,月奄歲星,在房。占曰:「民飢。」一曰:「豫州有災。」二年二月辛卯,填星犯軒轅大星。甲午,月犯東井。閏月乙亥,月犯歲星,在房。占悉同上。五月丁亥,彗出天船,在胃度中。彗為兵喪,除舊布新,出天船,外夷陵。一曰:「為
 大水。」六月辛酉,月犯房。八月戊午,熒惑犯填星,在張。占曰:「兵大起。張,三河分。」十月己未,太白犯哭星。十二月,枉矢自東南流于西北,其長半天。三年正月壬辰,熒惑犯楗閉。案占:「人主憂。」三月乙酉,熒惑逆行犯金句鈐。案占:「王者惡之。」月犯太白,在昴。占曰:「人君死。」一曰:「趙地有兵,朝廷不安。」六月,太白犯東井。七月乙酉,熒惑犯天江。丙戌,太白犯輿鬼。占悉同上。戊子,月犯牽牛中央大星。占曰:「牽牛,天將也。犯中央星,大將軍死。」八月丁未,太白犯軒
 轅大星。甲子,月犯畢大星。



 占曰:「為邊兵。」一曰:「下犯上。」庚午,太白犯填星,在太微中。占曰:「王者惡之。」二年五月,關中氐帥殺苻生立堅。十二月,慕容俊入屯鄴。八月,安西將軍、豫州刺史謝奕薨。三年十月,諸葛攸舟軍入河,敗績。豫州刺史謝萬入潁,眾潰而歸,除名為民。十一月,司徒會稽王以二鎮敗,求自貶三等。四年正月,慕容俊死,子暐代立。慕容恪殺其尚書令陽鶩等。五月,天下大水。五年五月,穆帝崩。



 升平四年正月乙亥,月犯牽牛中央大星。占曰:「大將死。」六月辛亥,辰星犯軒轅。占曰:「女主憂。」己未,太白入太微右掖門,從端門出。占曰:「貴奪勢。」一曰:「有兵。」又曰:「出端門,臣不臣。」八月戊申,太白犯氐。占曰:「國有憂。」丙辰,熒惑犯太微西蕃上將。九月壬午,太白入南斗口,犯第四星。



 占曰:「為喪,有赦,天下受爵祿。」十月庚戌,天狗見西南。占曰:「有大兵流血。」十二月甲寅,熒惑犯房。丙寅,太白晝見。庚寅,月犯楗閉。占曰:「人君惡之。」五年正月乙巳,填星逆行犯太
 微。乙丑辰時,月在危宿奄太白。占曰:「天下民靡散。」三月丁未,月犯填星在軫。占曰:「為大喪。」五月壬寅,月犯太微。庚戌,月犯建星。占曰:「大臣相譖。」辛亥,月犯牽牛宿。占曰:「國有憂。」五年正月,北中郎將郗曇薨。五月,穆帝崩,哀帝立,大赦賜爵,褚後失勢。



 七月,慕容恪攻冀州刺史呂護於野王,拔之,護奔滎陽。是時桓溫以大眾次宛,聞護敗乃退。



 升平五年六月癸酉,月奄氐東北星。占曰:「大將當之。」九
 月乙酉,奄畢。



 占曰:「有邊兵。」十月丁卯,熒惑犯歲星,在營室。占曰:「大臣有匿謀。」一曰:「衛地有兵。」丁未,月犯畢赤星。占曰:「下犯上。」又曰:「有邊兵。」



 八月,范汪廢。隆和元年,慕容暐遣傅末波寇河陰,陳祐危逼。



 晉哀帝興寧元年八月,星孛大角亢,入天市。按占:「為兵喪」。三年正月,皇后王氏崩。二月,哀帝崩。三月,慕容恪攻洛陽,沈勁等戰死。興寧元年十月丙戌,月奄太白,在須女。占曰:「天下民靡
 散。」一曰:「災在揚州。」三年,洛陽沒。其後桓溫傾揚州資實,討鮮卑敗績,死亡太半。及征袁真,淮南殘破。後氐及東胡侵逼,兵役無已。



 興寧三年正月乙卯,月奄歲星,在參。參,益州分也。六月,鎮西將軍、益州刺史周撫薨。十月,梁州刺史司馬勳入益州以叛,朱序率眾助刺史周楚討平之。興寧三年七月庚戌,月犯南斗。占曰:「女主憂。」歲星犯輿鬼。占曰:「人君憂。」



 十月,太白晝見,在亢。占曰:「亢為朝廷,有
 兵喪,為臣彊。」哀帝是年二月崩,其災皆在海西也。明年五月,皇后庾氏崩。



 晉海西太和元年二月丙子,月奄熒惑,在參。占曰:「為內亂。」一曰:「參,魏地。」二年正月,太白入昴。五年,慕容暐為苻堅所滅,司、冀、幽、并四州並屬氐。



 太和二年八月戊午,太白犯歲星,在太微。三年六月甲寅,太白奄熒惑,在太微端門中。六年,海西公廢。



 太和四年二月,客星見紫宮西垣,至七月乃滅。占曰:「客
 星守紫宮,臣殺主。」



 閏月乙亥,月暈軫,復有白暈貫月,北暈斗柄三星。占曰:「王者惡之。」六年,桓溫廢帝。太和四年十月壬申,有大流星西下,聲如雷。按占:「流星為貴使,星大者使大。」明年,遣使免袁真為庶人。桓溫征壽春,真病死,息瑾代立,求救於苻堅,溫破氐軍。六年,壽春城陷,聲如雷,將士怒之象也。



 太和六年閏月,熒惑守太微端門。占曰:「天子亡國。」又曰:「諸侯三公謀其上。」一曰:「有斬臣。」辛卯,月犯心大星。占曰:「
 王者惡之。」十一月,桓溫廢帝,並奏誅武陵王,簡文不許,溫乃徙之新安。



\end{pinyinscope}