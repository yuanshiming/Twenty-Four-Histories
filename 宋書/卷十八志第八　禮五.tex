\article{卷十八志第八 禮五}

\begin{pinyinscope}

 秦滅
 禮學,事多違古。漢初崇簡,不存改作,車服之儀,
 多因秦舊。至明帝始乃修復先典,司馬彪《輿服志》詳之矣。魏代唯作指南車,其餘雖累有改易,不足相變。晉立服
 制令,辨定眾儀,徐廣《車服注》,略明事目,並行於今者也。故復敘列,以通數代典事。



 上古聖人見轉蓬,始為輪,輪何可載,因為輿。任重致遠,流運無極。後代聖人觀北斗魁方杓曲攜龍角,為帝車,曲其輈以便駕。《系本》云:「奚仲始作車。」



 案庖羲畫《八卦》而為大輿,服牛乘馬,以利天下。奚仲乃夏之車正,安得始造乎?



 《系本》之言,非也。「車服以庸」,著在唐《典》。夏建旌旗,以表貴賤。周有六職,百工居其一焉。一器而群工致其巧,
 車最居多。《明堂記》曰:「鸞車,有虞氏之路也。大路,殷路也。乘路,周路也。」殷有山車之瑞,謂桑根車,殷人制為大路。《禮緯》曰:「山車垂句。」句,曲也。言不揉治而自曲也。周之五路,則有玉、金、象、革、木。五者之飾,備於《考工記》。輿方法地,蓋員象天,輻以象日月,二十八弓以象列宿。玉、金、象者,飾車諸末,因為名也。革者漆革,木者漆木也。玉路,建大常以祀;金路,建大旂以賓;象路,建大赤以朝;革路,建大白以戎;木路,建大麾以田。黑色,夏所尚也。



 秦閱三代之
 車,獨取殷制。古曰桑根車,秦曰金根車也。漢氏因秦之舊,亦為乘輿,所謂乘殷之路者也。《禮論·輿駕議》曰:「周則玉輅最尊,漢之金根,亦周之玉路也。」漢制乘輿金根車,輪皆朱斑,重轂兩轄,飛軨。轂外復有轂,施轄,其外復設轄,施銅貫其中。《東京賦》曰:「重輪二轄,疏轂飛軨。」飛軨以赤油為之,廣八寸,長注地,繫軸頭,謂之飛軨也。以金薄繆龍,為輿倚較。較在箱上,颭文畫蕃。蕃,箱也。文虎伏軾,龍首銜軛,鸞雀立衡,颭文畫轅,翠羽蓋黃裹,所謂黃
 屋也。金華施橑末,建太常十二旒,畫日月升龍,駕六黑馬,施十二鸞,金為叉髦,插以翟尾。又加犛牛尾,大如斗,置左騑馬軛上,所謂左纛輿也。路如周玉路之制。應劭《漢官鹵簿圖》,乘輿大駕,則御鳳皇車,以金根為副。又五色安車、五色立車各五乘。建龍旂,駕四馬,施八鸞,餘如金根之制,猶周金路也。



 其車各如方色,所謂五時副車,俗謂為「五帝車」也。江左則闕矣。白馬者,朱其鬣,安車者,坐乘。又有建華蓋九重。甘泉鹵簿者,道車五乘,游車九乘,
 在乘輿車前。又有象車,最在前,試橋道。晉江左駕猶有之。凡婦人車皆坐乘,故《周禮》王后有安車而王無也。漢制乘輿乃有之。天子所御駕六,其餘副車皆駕四。案《書》稱朽索御六馬。逸禮《王度記》曰:「天子駕六,諸侯駕五,卿駕四,大夫三,士二,庶人一。」楚平王駕白馬。梁惠王以安車駕三送淳于髡,大夫之儀。《周禮》,四馬為乘。毛詩,「天子至大夫同駕四,士駕二」。袁盎諫漢文馳六飛。魏時天子亦駕六。晉《先蠶儀》,皇后安車駕六,以兩轅安車駕五為
 副。江左以來,相承無六,駕四而已。



 宋孝武大明三年,使尚書左丞荀萬秋造五路。《禮圖》,金路,通赤旗,無蓋,改造依擬金根,而赤漆颭畫,玉飾諸末,建青旂,十有二旒,駕玄馬四,施羽葆蓋,以祀。即以金根為金路,建大青旂,十有二旒,駕玄馬四,羽葆蓋,以賓。象、革、木路,《周官》、《輿服志》、《禮圖》並不載其形段,並依擬玉路,漆颭畫,羽葆蓋,象飾諸末,建立赤旂,十有二旒,以視朝。革路,建赤旂,十有二旒,以即戎。木路,建赤麾,以田。象、革駕玄,木駕赤,四馬。舊
 有大事,法駕出,五路各有所主,不俱出也。大明中,始制五路俱出。親耕籍田,乘三蓋車,一名芝車,又名耕根車,置耒耜於軾上。戎車立乘,夏曰鉤車,殷曰寅車,周曰元戎。建牙麾,邪注之,載金鼓羽幢,置甲弩於軾上。獵車,輞𨏥,輪畫繆龍繞之。一名蹋豬車。



 魏文帝改曰蹋虎車。



 指南車,其始周公所作,以送荒外遠使。地域平漫,迷於東西,造立此車,使常知南北。鬼谷子云:「鄭人取玉,必載
 司南,為其不惑也。」至於秦、漢,其制無聞。後漢張衡始復創造。漢末喪亂,其器不存。魏高堂隆、秦朗,皆博聞之士,爭論於朝,云無指南車,記者虛說。明帝青龍中,令博士馬鈞更造之而車成。晉亂復亡。石虎使解飛,姚興使令狐生又造焉。安帝義熙十三年,宋武帝平長安,始得此車。其制如鼓車,設木人於車上,舉手指南。車雖回轉,所指不移。大駕鹵簿,最先啟行。此車戎狄所制,機數不精,雖曰指南,多不審正。回曲步驟,猶須人功正之。範陽人
 祖沖之,有巧思,常謂宜更構造。宋順帝升明末,齊王為相,命造之焉。車成,使撫軍丹陽尹王僧虔、御史中丞劉休試之。其制甚精,百屈千回,未嘗移變。晉代又有指南舟。索虜拓跋燾使工人郭善明造指南車,彌年不就。扶風人馬岳又造,垂成,善明鴆殺之。



 記里車,未詳所由來,亦高祖定三秦所獲。制如指南,其上有鼓,車行一里,木人輒擊一槌。大駕鹵簿,以次指南。



 輦車,《周禮》王后五路之卑者也。后宮中從容所乘,非王
 車也。漢制乘輿御之,或使人輓,或駕果下馬。漢成帝欲與班婕妤同輦是也。後漢陰就外戚驕貴,亦輦。井丹譏之曰:「昔桀乘人車,豈此邪!」然則輦夏后氏末代所造也。井丹譏陰就乘人,而不云僭上,豈貴臣亦得乘之乎?未知何代去其輪。《傅玄子》曰:「夏曰餘車,殷曰胡奴,周曰輜車。」輜車,即輦也。魏、晉御小出,常乘馬,亦多乘輿車。輿車,今之小輿。



 犢車,軿車之流也。漢諸侯貧者乃乘之,其後轉見貴。孫
 權云「車中八牛」,即犢車也。江左御出,又載儲偫之物。漢代賤軺車而貴輜軿,魏晉賤輜軿而貴軺車。



 又有追鋒車,去小平蓋,加通幔,如軺車,而駕馬。又以雲母飾犢車,謂之雲母車,臣下不得乘,時以賜王公。晉氏又有四望車,今制亦存。又漢制,唯賈人不得乘馬車,其餘皆乘之矣。除吏赤蓋杠,餘則青蓋杠云。



 《周禮》,王后亦有五路,重翟、厭翟、安車、翟車、輦車,凡五也。漢制,太皇太后、皇太后、皇后法駕乘重翟羽蓋金根車,
 駕青交路,青帷裳,云颭畫轅,黃金塗五末,蓋爪施金華,駕三馬,左右騑。其法駕則紫罽軿車。按《字林》,軿車有衣蔽,無後轅。其有後轅者謂之輜。應劭《漢官》,明帝永平七年,光烈陰皇后葬,魂車,鸞路青羽蓋,駕駟馬,龍旂九旒,前有方相。鳳皇車,大將軍妻參乘,太僕妻、御女騎夾轂,此前漢舊制也。



 晉《先蠶儀注》,皇后乘油畫雲母安車,駕六騩馬。騩,淺黑色也。油畫兩轅安車,駕五騩馬為副。公主油畫安車,駕
 三。三夫人青交路安車,駕三。皆以紫絳罽軿車,駕三為副。九嬪世婦軿車,駕二。宮入輜車,駕一。王妃、公侯特進夫人、封君皁交路安車,駕三。



 漢制,貴人、公主、王妃、封君油軿皆駕二,右騑而已。漢制,太子、皇子皆安車,朱斑輪,倚虎較,伏鹿軾,黑颭文畫蕃,青蓋,金華施橑末,黑颭文畫轅,金塗五末。皇子為王,錫以此乘,故曰王青蓋車。皆左右騑駕,五旂,旂九颭,畫降龍。皇孫乘綠車,亦駕三。魏、晉之制,太子及諸王皆駕四。



 晉元帝太興三年,太子釋奠。詔曰:「未有高車,可乘安車。」高車,即立乘車也。公及列侯安車,朱斑輪、倚鹿較、伏熊軾、黑蕃者謂之軒,皁繒蓋,駕二,右騑。王公旂八旒,侯七旒,卿五旒,皆降龍。公卿中二千石二千石郊陵法駕出,皆大車立乘,駕四。後導從大車,駕二,右騑。他出乘安車。其去位致仕,皆賜安車四馬。中二千石皆皁蓋、朱蕃,銅五末,駕二,右騑。《晉令》,王公之世子攝命治國者,安車,駕三,旂七旒,其侯世子,五旒。



 傅暢《故事》,三公安車,駕三;特進駕二;卿一。漢制,公、列侯、中二千石、二千石夫人會廟及蠶,各乘其夫之安車,右騑,加皁交路,帷裳。非公會,則乘漆布輜軿,銅五末。晉武帝太康四年,詔依漢故事,給九卿朝車駕及安車各一乘。



 傅暢《故事》,尚書令軺車,黑耳後戶。僕射但後戶無耳。中書監令如僕射。



 漢制,乘輿御大駕,公卿奉引,太僕、大將軍參乘,備千乘萬騎,屬車八十一乘。古者諸侯貳車九乘,秦滅九國,兼
 其車服,故八十一乘也。漢遵弗改。漢都長安時,祠天於甘泉用之。都洛陽,上原陵,又用之,大喪又用之。法駕則河南尹、洛陽令奉引,奉車郎御,侍中參乘,屬車三十六乘。凡屬車皆皁蓋赤裏。後漢祠天郊用法駕,祠宗廟用小駕。小駕,減損副車也。前驅有九游雲罕,皮軒鸞旗,車皆大夫載之。鸞旗者,編羽旄列繫幛傍也。金鉦黃鉞,黃門鼓車,乘輿之後有屬車,尚書、御史載之。最後一車懸豹尾。豹尾以前,比於省中。每出警蹕清道,建五旗。



 太僕
 奉駕條上鹵簿,尚書郎侍御史令史皆執注以督整車騎,所謂護駕也。春秋上陵,尤省於小駕。直事尚書一人從,其餘令史以下皆從行,所謂先置也。薛綜《東京賦》注以雲罕九游為旌旗別名,亦不辨其形。案魏命晉王建天子旌旗,置旄頭雲罕。是知雲罕非旌旗也。徐廣《車服注》以為九游,游車九乘。雲罕疑是璟罕。《詩敘》曰:「齊侯田獵璟弋,百姓苦之。」璟罕勣施遊獵,遂為行飾乎?潘岳《籍田賦》先敘五路九旗,次言瓊璟雲罕。若罕為旗,則岳不
 應頻句於九旗之下。又以其物匹璟戟,宜是今畢網明矣。此說為得之。皮軒,以虎皮為軒也。徐又引《淮南子》「軍正執豹皮以制正其眾。」《禮記》「前月士師,則載虎皮」。乘輿豹尾,亦其義類乎?五旗者,五色各一旗,以木牛承其下。徐又云「木牛,蓋取其負重而安穩也。」五旗纏竿,即《禮記》德車結旌不盡飾也。戎事乃散之。又武車綏旌,垂舒之也。史臣案:今結旌綏旌同,而德車武車之所不建。又木牛之義,亦未灼然可曉。



 又案《周禮》辨載法物,莫不詳究,
 然無相風、璟網、旄頭之屬,此非古制明矣。



 何承天謂戰國並爭,師旅數出,懸烏之設,務察風昆,宜是秦矣。晉武嘗問侍臣:「旄頭何義?」彭推對曰:「秦國有奇怪,觸山截水,無不崩潰,唯畏旄頭,故虎士服之,則秦制也。」張華曰:「有是言而事不經。臣謂壯士之怒,髮踴衝冠,義取於此。」摯虞《決疑》無所是非也。徐爰曰:「彭、張之說,各言意義,無所承據。案天文畢昴之中謂之天街,故車駕以璟罕前引,畢方昴員,因其象。《星經》,昴一名旄頭,故使執之者冠皮
 毛之冠也。」



 輕車,古之戰車也。輪輿洞硃,不巾不蓋,建矛戟幢麾,置弩於軾上,駕二。



 射聲校尉司馬吏士載,以次屬車。



 《漢儀》曰:「出稱警,入稱蹕。」說者云,車駕出則應稱警,入則應稱蹕也,而今俱唱之。史臣以為警者,警戒也;蹕者,止行也。今從乘輿而出者,並警戒以備非常也。從外而入乘輿相干者,蹕而止之也。董巴、司馬彪云:「諸侯王遮迾出入,稱警設蹕。」
 武剛車,有巾有蓋,在前為先驅。又在輕車之後為殿也。駕一。



 《史記》,衛青征匈奴,以武剛車為營是也。



 漢制,大行載轀輬車,四輪。其飾如金根,加施組連璧,交絡,四角金龍首銜璧垂五采,析羽流蘇,前後雲氣畫帷裳,颭文畫曲蕃,長與車等。太僕御,駕六白駱馬,以黑藥灼其身為虎文,謂之布施馬。既下,馬斥賣,車藏城北秘宮。今則馬不虎文,不斥賣;車則毀也。自漢霍光、晉安平、齊王、賈充、王導、謝安、宋江夏王葬以殊禮者,皆大輅黃
 屋,載紵輬車。



 《晉令》曰:「乘傳出使,遭喪以上,即自表聞,聽得白服乘騾車,到副使攝事。」徐廣《車服注》:「傳聞騾車者,犢車裝而馬車轅也。」又車無蓋者曰科車。



 晉武帝時,護軍將軍羊琇乘羊車,司隸校尉劉毅奏彈之。詔曰:「羊車雖無制,猶非素者所服。」江左來無禁也。



 舊有充庭之制,臨軒大會,陳乘輿車輦旌鼓於殿庭。張衡《東京賦》云:「龍路充庭,鸞旗拂霓。」晉江左廢絕。宋孝武
 大明中修復。



 上古寢處皮毛,未有制度。後代聖人見鳥獸毛羽及其文章與草木華採之色,因染絲彩以作衣裳,為玄黃之服,以法乾坤上下之儀:觀鳥獸冠胡之形,制冠冕纓蕤之飾。虞氏作繢,採章彌文,夏后崇約,猶美黻冕。咎繇陳《謨》,則稱五服五章。



 皆後王所不得異也。周監二代,典制詳密,故弁師掌六冕,司服掌六服,設擬等差,各有其序。《禮記·冠義》曰:「冠者禮之始,事之重者也。」太古布冠,齊則
 緇之。夏曰毋追,殷曰章甫,周曰委貌,此皆三代常所□□周之祭冕,繅采備飾,故夫子曰「服周之冕」,以盡美稱之。



 至秦以戰國即天子位,滅去古制,郊祭之服,皆以袀玄。至漢明帝始採《周官》、《禮記》、《尚書》諸儒說,還備袞冕之服。魏明帝以公卿袞衣黼黻之文,擬於至尊,復損略之。晉以來無改更也。天子禮郊廟,則黑介幘,平冕,今所謂平頂冠也。



 皁表朱綠裏,廣七寸,長尺二寸,垂珠十二旒。以組為纓,衣皂上絳下,前三幅,後四幅,衣畫而裳繡,為日、
 月、星辰、山、龍、華、蟲、藻、火、粉米、黼、黻之象,凡十二章也。素帶廣四寸,朱裏,以硃緣裨飾其側。中衣以絳緣其領袖,赤皮蔽膝。蔽膝,古之韍也。絳褲,絳襪,赤幹。未元服者,空頂介幘。其釋奠先聖,則皁紗裙,絳緣中衣,絳褲襪,黑幹。其臨軒亦袞冕也。其朝服,通天冠,高九寸,金博山顏,黑介幘,絳紗裙,皁緣中衣。其拜陵,黑介幘,緌單衣。其雜服,有青赤黃白緗黑色介幘,五色紗裙,五梁進賢冠,遠游冠,平上幘,武冠。其素服,白𢂿單衣。《漢儀》,立秋日獵服緗幘。
 晉哀帝初,博士曹弘之等議:「立秋御讀令,不應緗幘,求改用素。」詔從之。宋文帝元嘉六年,奉朝請徐道娛表:「不應素幘。」詔門下詳議,帝執宜如舊,遂不改。



 進賢冠,前高七寸,後高三寸,長八寸,梁數隨貴賤,古之緇布冠也。文儒者之所服。上公、卿助祭於郊廟,皆平冕,王公八旒,卿七旒,以組為纓,色如其綬。



 王公衣山龍以下,九章也;卿衣華蟲以下,七章也。行鄉射禮,則公卿委貌冠,以皁絹為之,形如覆杯,與皮弁同制。長七寸,高四
 寸。衣黑而裳素。其中衣以皁緣領袖;其執事之人皮弁,以鹿皮為之。



 武冠,昔惠文冠,本趙服也,一名大冠。凡侍臣則加貂蟬。應劭《漢官》曰:「說者以金取堅剛,百煉不耗;蟬居高食潔,口在腋下;貂內勁悍而外溫潤。」此因物生義,非其實也。其實趙武靈王變胡,而秦滅趙,以其君冠賜侍臣,故秦、漢以來,侍臣有貂蟬也。徐廣《車服注》稱其意曰:「北土寒涼,本以貂皮暖額,附施於冠,因遂變成首飾乎?」侍中左
 貂,常侍右貂。



 法冠,本楚服也。一名柱後,一名獬豸。說者云:「獬豸獸知曲直,以角觸不正者也。」秦滅楚,以其君冠賜法官。



 謁者高山冠,本齊服也。一名側注冠。秦滅齊,以其君冠賜謁者。魏明帝以其形似通天、遠遊,乃毀變之。



 樊噲冠,廣九寸,制似平冕,殿門衛士服之。漢將樊噲常持鐵盾。鴻門之會,項羽欲害漢王,乃裂裳以苞盾,戴入見羽。漢承秦制,冠有十三種。魏、晉以來,不盡施用。今志
 其施用者也。



 幘者,古賤人不冠者之服也。漢元帝額有壯髮,始引幘服之。王莽頂禿,又加其屋也。《漢注》曰:「冠進賢者宜長耳,今介幘也;冠惠文者宜短耳,今平中幘也。知時各隨所宜,後遂因冠為別。」介幘服文吏,平上服武官也。童子幘無屋者,示未成人也。又有納言幘,後收,又一重,方三寸。又有赤幘,騎吏、武史、乘輿鼓吹所服。救日蝕,文武官皆免冠,著赤幘,對朝服,示威武也。宋乘輿鼓吹,黑幘武冠。



 漢制,祀事五郊,天子與執事所服各如方色;百官不執事者,自服常服以從。



 常服,絳衣也。魏祕書監秦靜曰:「漢氏承秦,改六冕之制,俱玄冠絳衣而已。」



 晉名曰五時朝服;有四時朝服,又有朝服。



 凡兵事,總謂之戎。《尚書》云:「一戎衣而天下定。」《周禮》:「革路以即戎。」又曰:「兵事韋弁服。」以韎韋為弁,又以為衣裳。《春秋左傳》:「戎服將事。」又云:「晉郤至衣韎韋之跗。」注,先儒云:「韎,絳色。」今時伍伯衣。說者云,五霸兵戰,猶有綬紱、冠纓、
 漫胡,則戎服非褲褶之制,未詳所起。



 近代車駕親戎中外戒嚴之服,無定色,冠黑帽,綴紫褾。褾以繒為之,長四寸,廣一寸。腰有絡帶,以代鞶革。中官紫褾。外官絳褾。又有纂嚴戎服,而不綴褾。行留文武悉同。其畋獵巡幸,則唯從官戎服,帶鞶革;文官不下纓,武官脫冠。宋文帝元嘉中,巡幸搜狩皆如之;救宮廟水火,亦如之。



 漢制,太后入廟祭神服,紺上皁下;親蠶,青上縹下,皆深衣。深衣,即單衣也。首飾剪犛幗。
 漢制,皇后謁廟服,紺上皂下;親蠶,青上縹下。首飾,假髻,步搖,八雀,九華,加以翡翠。晉《先蠶儀注》,皇后十二金奠,步搖,大手髻,衣純青之衣,帶綬佩。今皇后謁廟服袿襡大衣,謂之褘衣。公主三夫人大手髻,七金奠蔽髻。九嬪及公夫人五金奠。世婦三金奠。公主會見,大手髻。其長公主得有步搖。公主封君以上皆帶綬,以采組為緄帶,各如其綬色。公特進列侯夫人、卿校世婦、二千石命婦年長者,紺繒幗。佐祭則皁絹上下;助蠶則青絹上下。自皇后至二千石
 命婦,皆以蠶衣為朝服。



 劉向曰:「古者天子至于士,王后至于命婦,必佩玉,尊卑各有其制。」《禮記》曰:「天子佩白玉而玄組綬,公侯山玄玉而硃組綬,卿大夫水蒼玉而緇組綬,士佩瓀玟而縕組綬。」縕,赤黃色。綬者,所貫佩相承受也。上下施韍如蔽膝,貴賤亦各有殊。五霸之後,戰兵不息,佩非兵器,韍非戰儀,於是解去佩韍,留其繫襚而已。秦乃以采組連結於襚,轉相結受,謂之綬。漢承用之。至明帝始復制佩,而漢
 末又亡絕。魏侍中王粲識其形,乃復造焉。今之佩,粲所制也。皇后至命婦所佩,古制不存,今與外同制,秦組綬,仍又施之。



 漢制,自天子至于百官,無不佩刀。司馬彪志具有其制。漢高祖為泗水亭長,拔劍斬白蛇。雋不疑云:「劍者,君子武備。」張衡《東京賦》「紆黃組,腰干將。」



 然則自人君至士人,又帶劍也。自晉代以來,始以木劍代刃劍。



 乘輿六璽,秦制也。《漢舊儀》曰:「皇帝行璽,皇帝之璽,皇帝
 信璽,天子行璽,天子之璽,天子信璽。」此則漢遵秦也。初,高祖入關,得秦始皇藍田玉璽,螭虎紐,文曰「受天之命,皇帝壽昌」。高祖佩之,後代名曰傳國璽,與斬白蛇劍俱為乘輿所寶。傳國璽,魏、晉至今不廢;斬白蛇劍,晉惠帝武庫火燒之,今亡。



 晉懷帝沒胡,傳國璽沒於劉聰,後又屬石勒。及石勒弟石虎死,胡亂,晉穆帝代,乃還天府。虞喜《志林》曰:「傳國璽,自在六璽之外,天子凡七璽也。」《漢注》曰:「璽,印也。自秦以前,臣下皆以金玉為印,龍虎紐,唯所
 好。秦以來,以璽為稱,又獨以玉,臣下莫得用。」漢制,皇帝黃赤綬,四採,黃、赤、縹、紺。皇后金璽,綬亦如之。於禮,士綬之色如此,後代變古也。吳無刻玉工,以金為璽。



 孫皓造金璽六枚是也。又有麟鳳龜龍璽,駝馬鴨頭雜印,今代則闕也。



 皇太子,金璽,龜紐,纁朱綬,四采,赤、黃、縹、紺。給五時朝服,遠遊冠,亦有三梁進賢冠,佩瑜玉。



 諸王,金璽,龜紐,纁朱綬,四採,赤、黃、縹、紺。給五時朝服,遠
 遊冠,亦有三梁進賢冠,佩山玄玉。



 郡公,金章,玄朱綬。給五時朝服,進賢三梁冠,佩山玄玉。太宰、太傅、太保、丞相、司徒、司空,金章,紫綬,給五時朝服,進賢三梁冠,佩山玄玉。相國則綠綟綬,三采,綠、紫、紺。綟,草名也,其色綠。大司馬、大將軍、太尉、凡將軍位從公者,金章,紫綬,給五時朝服,武冠。佩山玄玉。郡侯,金章,青朱綬,給五時朝服,進賢三梁冠。佩水蒼玉。驃騎、車騎將軍,凡諸將軍加大者,徵、鎮、安、平、中軍、鎮軍、
 撫軍、前、左、右、後將軍、征虜、冠軍、輔國、龍驤將軍,金章,紫綬。給五時朝服,武冠,佩水蒼玉。



 貴嬪、夫人、貴人,金章,文曰貴嬪、夫人、貴人之章。紫綬,佩于闐玉。淑妃、淑媛、淑儀、修華、修容、修儀、婕妤、容華、充華,銀印,文曰淑妃、淑媛、淑儀、修華、修容、修儀、婕妤、容華、充華之印。青綬。佩五采瓊玉。



 皇太子妃,金璽,龜紐,纁硃綬。佩瑜玉。
 諸王太妃、諸長公主、公主、封君,金印,紫綬,佩山玄玉。諸王世子,金印,紫綬。五時朝服,進賢兩梁冠,佩山玄玉。郡公侯太夫人,銀印,青綬,佩水蒼玉。郡公侯太子,銀印,青綬。給五時朝服,進賢兩梁冠,佩水蒼玉。



 侍中、散騎常侍及中常侍,給五時朝服,武冠。貂蟬,侍中左,右常侍,皆佩水蒼玉。尚書令、僕射,銅印,墨綬。給五時朝服,納言幘,進賢兩梁
 冠,佩水蒼玉。尚書,給五時朝服,納言幘,進賢兩梁冠,佩水蒼玉。中書監令、秘書監,銅印,墨綟綬。給五時朝服,進賢兩梁冠。佩水蒼玉。



 光祿大夫、卿、尹、太子保、傅、大長秋、太子詹事,銀章,青綬。給五時朝服,進賢兩梁冠。佩水蒼玉。



 衛尉,則武冠。衛尉,江左不置。宋孝武孝建初始置,不檢晉服制,止以九卿皆文冠及進賢兩梁冠,非舊也。
 司隸校尉、武尉、左右衛、中堅、中壘、驍騎、遊擊、前軍、左軍、右軍、後軍、寧朔、建威、振威、奮威、揚威、廣威、建武、振武、奮武、揚武、廣武、左右積弩、彊弩諸將軍、監軍,銀章,青綬。給五時朝服,武冠,佩水蒼玉。領軍、護軍、城門五營校尉、東南西北中郎將,銀印,青綬。給五時朝服,武冠,佩水蒼玉。



 縣、鄉、亭侯,金印,紫綬。朝服,進賢三梁冠。



 鷹揚、折衝、輕車、揚烈、威遠、寧遠、虎威、材官、伏波、凌江諸
 將軍,銀章,青綬。給五時朝服,武冠。奮武護軍、安夷撫軍、護軍、軍州郡國都尉、奉車、駙馬、騎都尉、諸護軍將兵助郡都尉、水衡、典虞、牧官、典牧都尉、度支中郎將、校尉、都尉、司監都尉、材官校尉、王國中尉、宜和伊吾都尉、監淮南津都尉,銀印,青綬。五時朝服,武冠。



 州刺史,銅印,墨綬。給絳朝服,進賢兩梁冠。御史中丞、都水使者,銅印,墨綬。給五時朝服,進賢兩梁
 冠,佩水蒼玉。謁者僕射,銅印,墨綬。給四時朝服,高山冠,佩水蒼玉。諸軍司馬,銀章,青綬。朝服,武冠。



 給事中、黃門侍郎、散騎侍郎、太子中庶子、庶子,給五時朝服,武冠。中書侍郎,給五時朝服,進賢一梁冠。冗從僕射、太子衛率,銅印,墨綬。給五時朝服,武冠。



 虎賁中郎將、羽林監,銅印,墨綬。給四時朝服,武冠。其在
 陛列及備鹵簿,鶡尾,絳紗穀單衣。鶡鳥似雞,出上黨。為鳥彊猛,鬥不死不止。復著鶡尾。



 北軍中侯、殿中監,銅印,墨綬。給四時朝服,武冠。護匈奴中郎將、護羌夷戎蠻越烏丸西域戊己校尉,銅印,青綬。朝服,武冠。



 郡國太守、相、內史,銀章,青綬。朝服,進賢兩梁冠。江左止單衣幘。其加中二千石者,依卿、尹。牙門將,銀章,青綬。朝服,武冠。



 騎都督、守,銀印,青綬。朝服,武冠。



 尚書左右丞、秘書丞,銅印,黃綬。朝服,進賢一梁冠。尚書秘書郎、太子中舍人、洗馬、舍人,朝服,進賢一梁冠。黃沙治書侍御史,銀印,墨綬。朝服,法冠。侍御史,朝服,法冠。



 關內、關中名號侯,金印,紫綬。朝服,進賢兩梁冠。諸博士,給皂朝服,進賢兩梁冠,佩水蒼玉。公府長史、諸卿尹丞、諸縣署令秩千石者,銅印,墨綬。朝
 服,進賢兩梁冠。江左公府長史無朝服,縣令止單衣幘。宋後廢帝元徽四年,司徒右長史王儉議公府長史應服朝服,曰:「《春秋國語》云:『貌者情之華,服者心之文。』巖廊盛禮,衣冠為大。是故軍國異容,內外殊序。而自頃承用,每有乖違。



 府職掌人,教四方是則。臣居毗佐,志在當官,永言先典,載懷夕惕。按晉令,公府長史,官品第六,銅印,墨綬,朝服,進賢兩梁冠。掾、屬,官品第七,朝服,進賢一梁冠。晉官表注,亦與《令》同。而今長史、掾、屬,但著朱服而已,
 此則公違明文,積習成謬。謂宜依舊制,長史兩梁冠,掾、屬一梁冠,並同備朝服。中單韋幹,率由舊章。若所上蒙允,并請班司徒二府及諸儀同三府,通為永準。又尋舊事,司徒公府領步兵者,職僚悉同降朝不領兵者。主簿祭酒,中單韋幹並備,令史以下,唯著玄衣。今府既開公,謹遵此制。其或有署臺位者,玄服為宜。按《令》稱諸有兼官,皆從重官之例。尋內官為重,其署臺位者,悉宜著位之服,不在玄服之例。若署諸卿寺位兼府職者,雖三品,
 而卿寺為卑,則宜依公府玄衣之制。服章事重,禮儀所先,請臺詳服。」



 議曹郎中沈俁之議曰:「制珪象德,損替因時;裁服象功,施用隨代。車旗變於商、周,冠佩革於秦、漢,豈必殊代襲容,改尚沿物哉。夫邊貂假幸侍之首,賤幘登尊極之顏,一適時用,便隆後制。況朱裳以朝,緬傾百祀,韋幹不加,浩然惟舊。服為定章,事成永則。其儉之所秉,會非古訓。青素相因,代有損益,何事棄盛宋之興法,追往晉之頹典。變改空煩,謂不宜革。」儉又上議曰:「自頃
 服章多闕,有違前準。近議依令文,被報不宜改革,又稱左丞劉議,『按令文,凡有朝服,今多闕亡。然則文存服損,非唯鉉佐,用舍既久,即為舊章』。如下旨,伏尋皇宋受終,每因晉舊制,律令條章,同規在昔。若事有宜,必合懲改,則當上關詔書,下由朝議,縣諸日月,垂則後昆。豈得因外府之乖謬,以為盛宋之興典;用晉氏之律令,而謂其儀為頹法哉!順違從失,非所望於高議;申明舊典,何改革之可論。



 又左丞引令史之闕服,以為鉉佐之明比。夫
 名位不同,禮數異等,令史從省,或有權宜;達官簡略,為失彌重。又主簿、祭酒,備服於王庭,長史、掾、屬,硃衣以就列。於是倫比,自成矛盾。此而可忍,孰不可安!將引令以遵舊,臺據失以為例,研詳符旨,良所未譬。當官而行,何彊之有,制令昭然,守以無貳。」俁之又議:「雲火從物,沿損異儀,帝樂五殊,王禮三變,豈獨大宋造命,必咸仍於晉舊哉!



 夫宗社疑文,庭廟闕典,或上降制書,下協朝議,何乃鉉府佐屬裳黻,稍改白虎之詔,斷宣室之疇咨乎。又
 許令史之從省,咎達官之簡略。律茍可遵,固無辨於貴賤;規若必等,亦何關於權宜。一用一舍,彌增其滯。且佐非韋幹之職,吏本朝服之官,凡在班列,罔不如一,此蓋前令違而遂改,今制允而長用也。爵異服殊,寧會矛盾之譬;討論疑制,焉取彊弱之辨。府執既革之餘文,臺據永行之成典,良有期於無固,非所望於行迷。」參詳並同儉,議遂寢。



 諸軍長史、諸卿尹丞、獄丞、太子保傅詹事丞、郡國太守
 相內史、丞、長史、諸縣署令長相、關谷長、王公侯諸署令、長、司理、治書、公主家僕,銅印,墨綬。



 朝服,進賢一梁冠。江左太子保傅卿尹詹事丞,皁朝服。郡丞、縣令長,止單衣幘。



 公車司馬、太史、太醫、太官、御府、內省令、太子諸署令、僕、門大夫、陵令,銅印,墨綬。朝服,進賢一梁冠。太子率更、家令、僕,銅印,墨綬。給五時朝服,進賢兩梁冠。黃門諸署令、僕、長,銅印,墨綬。四時朝服,進賢一梁冠。



 黃門冗從僕射監、太子寺人監,銅印,墨綬。給四時朝服,武冠。



 公府司馬、諸軍城門五營校尉司馬、護匈奴中郎將護羌戎夷蠻越烏丸戊己校尉長史、司馬,銅印,墨綬。朝服,武冠。江左公府司馬無朝服,餘止單衣幘。廷尉正、監、平,銅印,墨綬。給皁零辟朝服,法冠。



 王郡公侯郎中令、大農,銅印,青綬。朝服,進賢兩梁冠。北軍中候丞,銅印,黃綬。朝服,進賢一梁冠。
 太子常從虎賁督、校督、司馬虎賁督,銅印,墨綬。朝服,武冠。殿中將軍,銀章,青綬。四時朝服,武冠。宋末不復給章綬。水衡、典虞、牧官、典牧、材官、州郡國都尉、司馬,銅印,墨綬。朝服,武冠。諸謁者,朝服,高山冠。門下中書通事舍人令史、門下主事令史,給四時朝服,武冠。



 尚書典事、都水使者參事、散騎集書中書尚書令史、門下散騎中書尚書令史、錄尚書中書監令僕省事史、秘書著作治書、主書、主璽、主譜令史、蘭臺殿中蘭臺謁者都水使者令史、書令史,朝服,進賢一梁冠。江左凡令史無朝服。



 節騎郎,朝服,武冠。其在陛列及備鹵簿,著鶡尾、絳紗縠單衣。



 殿中中郎將校尉、都尉、黃門中郎將校尉、殿中太醫校
 尉、都尉,銀印,青綬。



 四時朝服,武冠。



 關外侯,銀印,青綬。朝服,進賢兩梁冠。左右都候、閶闔司馬、城門候,銅印,墨綬朝服,武冠。王郡公侯中尉,銅印,墨綬。朝服,武冠。



 部曲督護、司馬史、部曲將,銅印。朝服,武冠。司馬史,假墨綬。



 太中中散諫議大夫、議郎、郎中、舍人,朝服,進賢一梁冠。秩千石者,兩梁。



 城門令史,朝服,武冠。江左凡令史無朝服。諸門僕射佐史、東宮門吏,皁零辟朝服。僕射東宮門吏,卻非冠。佐史,進賢冠。



 宮內游徼、亭長,皁零辟朝服,武冠。太醫校尉、都尉、總章協律中郎將校尉、都尉,銀印,青綬。朝服,武冠。小黃門,給四時朝服,武冠。黃門謁者,給四時朝服,進賢一梁冠。朝賀通謁時,著高
 山冠。



 黃門諸署史,給四時朝服,武冠。



 中黃門黃門諸署從官寺人,給四時科單衣,武冠。



 殿中司馬、及守陵者、殿中太醫司馬,銅印,墨綬。給四時朝服,武冠。



 太醫司馬,銅印。朝服,武冠。總章監鼓吹監司律司馬,銅印,墨綬。朝服。



 鼓吹監總章協律司馬,武冠。總章監司律司馬,進賢一梁冠。



 諸縣署丞、太子諸署丞、王公侯諸署及公主家丞,銅印,黃綬。朝服,進賢一梁冠。太醫丞,銅印。朝服,進賢一梁冠。黃門諸署丞,銅印,黃綬。給四時朝服,進賢一梁冠。黃門稱長、園監,銅印,黃綬。給四時朝服,武冠。



 諸縣尉、關谷塞護道尉,銅印,黃綬。朝服,武冠。江左止單衣幘。



 洛陽卿有秩,銅印,青綬。朝服,進賢一梁冠。



 宣威將軍以下至裨將軍,銅印。朝服,武冠。其以此官為刺史、郡守、若萬人司馬虎賁督以上、及司馬史者,皆假青綬。平虜武猛中郎將、校尉、都尉,銀印。



 朝服,武冠。其以此官為千人司馬虎賁督以上、及司馬史者,皆假青綬。別部司馬、軍假司馬,銀印。朝服,武冠。



 圖像都匠行水中郎將、校尉、都尉,銀印,青綬。朝服,武冠。若非以工伎巧能特加此官者,羽林長郎,佩武猛都尉
 以上印者,假青綬。別部司馬以下,假墨綬。



 朝服,武冠。其長郎壯士,武弁冠。在陛列及鹵簿,服絳縠單衣。



 陛下甲僕射主事吏將騎、廷上五牛旗假使虎賁,在陛列及備鹵簿,服錦文衣,武冠,鶡尾。陛長,假銅印,墨綬,旄頭。



 羽林在陛列及備鹵簿,服絳科單衣,上著韋畫要襦,假旄頭。



 舉輦跡禽前驅由基彊弩司馬、守陵虎賁,佩武猛都尉
 以上印者,假青綬。別部司馬以下,假墨綬。守陵虎賁,給絳科單衣,武冠。



 殿中冗從虎賁、殿中虎賁、及守陵者持鈒戟冗從虎賁,佩武猛都尉以下印者,假青綬。別部司馬以下,假墨綬。絳科單衣,武冠。



 持椎斧武騎虎賁、五騎傳詔虎賁、殿中羽林及守陵者太官尚食虎賁、稱飯宰人、諸官尚食虎賁,佩武猛都尉以上印者,假青綬。別部司馬以下,假墨綬。給絳蠙,武冠。
 其在陛列及備鹵簿,五騎虎賁,服錦文衣,鶡尾。宰人服離支衣。



 黃門鼓吹、及釘官僕射、黃門鼓吹史主事、諸官鼓吹、尚書廊下都坐門下守皞、殿中威儀騶、虎賁常直殿黃雲龍門者、門下左右部虎賁羽林騶、給傳事者諸導騶、門下中書守皞,給絳蠙,武冠。南書門下虎賁羽林騶、蘭臺五曹節藏射廊下守皞、威儀、發符騶、都水使者黃沙廊下守皞、謁者、錄事、威儀騶、河隄謁者騶、諸官謁者騶,絳
 蠙,武冠。給其衣服,自如故事。大誰士皁科單衣,樊噲冠。衛士墨布皞,卻敵冠。凡此前眾職,江左多不備,又多闕朝服。



 諸應給朝服佩玉,而不在京都者,給朝服;非護烏丸羌夷戎蠻諸校尉以上及刺史、西域戊己校尉,皆不給佩玉。其來朝會,權時假給,會罷輸還。凡應朝服者,而官不給,聽自具之。諸假印綬而官不給鞶囊者,得自具作。其但假印不假綬者,不得佩綬。



 鞶,古制也。漢代著鞶囊者,側在腰間。或謂之傍囊,或謂
 之綬囊。然則以此囊盛綬也。或盛或散,各有其時乎。



 朝服一具,冠幘各一,絳緋袍、皁緣中單衣領袖各一領,革帶袷褲各一,幹、襪各一量,簪導餉自副。四時朝服者,加絳絹黃緋青緋皁緋袍單衣各一領;五時朝服者,加給白絹袍單衣一領。



 諸受朝服,單衣七丈二尺,科單衣及皞五丈二尺,中衣絹五丈,緣皂一丈八尺,領袖練一匹一尺,絹七尺五寸。給褲練一丈四尺,縑二丈。襪布三尺。單衣及皞袷帶,縑
 各一段,長七尺。江左止給絹各有差。宋元嘉末,斷不復給,至今。山鹿、豽、柱豽白豽、施毛狐白領、黃豹、斑白鼲子、渠搜裘、步搖、八金奠、蔽結、多服蟬、明中、欋白,又諸織成衣帽、錦帳、純金銀器、雲母從廣一寸以上物者,皆為禁物。



 諸在官品令第二品以上,其非禁物,皆得服之。第三品以下,加不得服三金奠以上、蔽結、爵叉、假真珠翡翠校飾纓佩、雜采衣、杯文綺、齊繡黻、金適離、袿袍。第六品以下,加不得服金金奠、綾、錦、錦繡、七緣綺、貂豽裘、金叉環鉺、及以
 金校飾器物、張絳帳。第八品以下,加不得服羅、紈、綺、縠,雜色真文。騎士卒百工人,加不得服大絳紫襈、假結、真珠璫珥、犀、玳瑁、越疊、以銀飾器物、張帳、乘犢車,履色無過綠、青、白。奴婢衣食客,加不得服白幘、茜、絳、金黃銀叉、環、鈴、金適、鉺,履色無過純青。諸去官及薨卒不祿物故,家人所服,皆得從故官之例。諸王皆不得私作禁物,及罽碧校鞍,珠玉金銀錯刻鏤彫飾無用之物。



 天子坐漆床,居朱屋。史臣按《左傳》,丹桓宮之楹。何休注《
 公羊》,亦有硃屋以居。所從來久矣。漆床亦當是漢代舊儀,而《漢儀》不載。尋所以必硃必漆者,其理有可言焉。夫珍木嘉樹,其品非一,莫不植根深岨,致之未易。藉地廣之資,因人多之力,則役苦費深,為敝滋重。是以上古聖王,采椽不斫,斫之則懼刻桷彫楹,莫知其限也。哲人縣鑑微遠,杜漸防萌,知采椽不愜後代之心,不斫不為將來之用,故加朱施漆,以傳厥後。散木凡材,皆可入用。遠探幽旨,將在斯乎。



 殿屋之為員淵方井兼植荷華者,以厭火祥也。



 古者貴賤皆執笏,其有事則搢之於腰帶。所謂搢紳之士者,搢笏而垂紳帶也。



 紳垂三尺。笏者有事則書之,故常簪筆,今之白筆,是其遺象。三臺五省二品文官簪之;王公侯伯子男卿尹及武官不簪。加內侍位者,乃簪之。手板,則古笏矣。尚書令、僕射、尚書手板頭復有白筆,以紫皮裹之,名笏。朝服肩上有紫生袷囊,綴之朝服外,俗呼曰紫荷。或云漢代以盛奏事,負荷以行,未詳也。



 魏文帝黃初三年,詔賜漢太尉楊彪几杖,待以客禮。延請之日,使挾杖入朝。



 又令著鹿皮冠。彪辭讓,不聽。乃使服布單衣皮弁以見。《傅玄子》曰:「漢末王公名士,多委王服,以幅巾為雅。是以袁紹、崔鈞之徒,雖為將帥,皆著㡘巾。」



 魏武以天下凶荒,資財乏匱,擬古皮弁,裁縑帛以為𢂿,合乎簡易隨時之義,以色別其貴賤。本施軍飾,非為國容也。徐爰曰:「俗說𢂿本未有歧,荀文若巾之,行觸樹
 枝成歧,謂之為善,因而弗改。」通以為慶弔服。巾以葛為之,形如𢂿,而橫著之,古尊卑共服也。故漢末妖賊以黃為巾,時謂之「黃巾賊。」今國子太學生冠之,服單衣以為朝服,執一卷經以代手板。居士野人,皆服巾焉。



 徐爰曰:「帽名猶冠也。義取於蒙覆其首。其本纚也。古者有冠無幘,冠下有纚,以繒為之。後世施幘於冠,因裁纚為帽。自乘輿宴居,下至庶人無爵者,皆服之。」史臣案晉成帝咸和九年制,聽尚書八座丞郎、門下三省侍郎乘
 車白帢低幘出入掖門。又二宮直宮著烏紗𢂿。然則士人宴居,皆著帢矣。而江左時野人已著帽,士人亦往往而然,但其頂圓耳。後乃高其屋云。古者人君有朝服,有祭服,有宴服,有弔服。弔服皮弁疑衰,今以單衣黑幘為宴會服,拜陵亦如之。以單衣𢂿為弔服,修敬尊秩亦服之也。單衣,古之深衣也。今單衣裁製與深衣同,唯絹帶為異。



 深衣絹帽以居喪,單衣素帢以施吉。



 晉武帝泰始三年,詔太宰安平王孚服侍中之服,賜大司馬義陽王望袞冕之服。



 四年,又詔趙、樂安、燕王服散騎常侍之服。十年,賜彭城王袞冕之服。偽楚桓玄將篡,亦加安帝母弟太宰琅邪王袞冕服。宋興以來,王公貴臣加侍中、散騎常侍,乃得服貂璫也。



 宋孝武孝建元年,丞相南郡王義宣,二年,雍州刺史武昌王渾,又有異圖。世祖嫌侯王彊盛,欲加減削。其年十月己未,大司馬江夏王義恭、驃騎大將軍竟陵王誕表改
 革諸王車服制度,凡九條,表在《義恭傳》。上因諷有司更增廣條目。奏曰:「車服以庸,《虞書》茂典;名器慎假,《春秋》明誡。是以尚方所制,禁嚴漢律,諸侯竊服,雖親必罪。自頃以來,下僭彌盛。器服裝飾,樂舞音容,通于王公,達于眾庶。上下無辨,人志靡一。今表之所陳,實允禮度。九條之格,猶有未盡,謹共附益,凡二十四條。聽事不得南向坐,施帳并𢃕。蕃國官正冬不得跣登國殿,及夾侍國師傳令及油戟。公主王妃傳令,不得硃服。輿不得重杠。鄣扇
 不得雉尾;劍不得鹿盧形;槊毦不得孔雀白鷩;夾轂隊不得絳襖;平乘誕馬不得過二匹;胡伎不得彩衣。舞伎正冬著袿衣,不得莊而蔽花;正冬會不得鐸舞、杯柈舞。長褵伎、褷舒、丸劍、博山伎、緣大橦伎、升五案伎,自非正冬會奏舞曲,不得舞。諸妃主不得著袞帶。信幡,非臺省官悉用絳。郡縣內史相及封內官長,於其封君,既非在三,罷官則不復追敬,不合稱臣,正宜上下官敬而已。諸鎮常行,車前後不得過六隊,白直夾轂,不在其限。刀不得
 過銀銅為裝。諸王女封縣主、諸王子孫襲封王王之妃及封侯者夫人行,並不得鹵簿。諸王子繼體為王者,婚姻吉凶,悉依諸國公侯之禮,不得同皇弟皇子。車輿不得油幢,軺車不在其限。平乘舫皆平兩頭作露平形,不得擬像龍舟,悉不得硃油。帳鉤不得作五花及豎筍形。若先有器物者,悉輸送臺臧。書到後二十日期,若有竊玩犯禁者,及統司無舉糾,並臨時議罪。」詔可。



 車前五百者,卿行旅從,五百人為一旅。漢氏一統,故去
 其人,留其名也。



 宋孝武孝建二年十一月乙巳,有司奏:「侍中祭酒何偃議:『自今臨軒,乘輿法服,燾華蓋,登殿宜依廟齋以夾御,侍中、常侍夾扶上殿,及應為王公興,又夾扶,畢,還本位。』求詳議。」曹郎中徐爰參議:「宜如省所稱,以為永準。」詔可。



 孝建三年五月壬戌,有司奏:「案漢胡廣、蔡邕並云古者諸侯貳車九乘,秦滅六國,兼其車服,故王者大駕屬車八十一乘。尚書、御史乘之。最後一車,懸豹尾。



 法駕則三
 十六乘。檢晉江左逮至于今,乘輿出行,副車相承五乘。」尚書令建平王宏參議:「八十一乘,義兼九國,三十六乘無所準,並不出經典。自邕、廣傳說,又是從官所乘,非帝者副車正數。江左五乘,儉不中禮。案《周官》云:『上公九命,貳車九乘。侯伯七命,車七乘。子男五命,車五乘。』然則帝王十二乘。」詔可。



 大明元年九月丁未朔,有司奏:「未有皇太后出行副車定數,下禮官議正。」



 博士王燮之議:「《周禮》,后六服五路之
 數,悉與王同,則副車之制,不應獨異。



 又《記》云:『古者后立六宮、三夫人、九嬪、二十七世婦、八十一御妻,以聽天下之內治。』『天子立六官、三公、九卿、二十七大夫、八十一元士,以聽天下之外治。』鄭注云:『后象王立六宮而居之,亦正寢一,燕寢五。』推所立每與王同,禮無降亦明矣。皇太后既禮均至極,彌不應殊。謂並應同十二乘。」通關為允。詔可。



 大明四年正月戊辰,尚書左丞荀萬秋奏:「《籍田儀注》,『皇
 帝冠通天冠,朱珣,青介幘,衣青紗袍。侍中陪乘,奉車郎秉轡。』案《漢·輿服志》曰:『通天冠,乘輿常服也。』若斯豈可以常服降千畝邪?《禮記》曰:『昔者天子為藉千畝,冕而硃珣,躬秉耒耜。』鄭玄注《周官》司服曰:『六服同冕』,尊故也。時服雖變,冕制不改。又潘岳《藉田賦》云:『常伯陪乘,太僕秉轡。』推此,輿駕藉田,宜冠冕,璪十二旒,朱珣,黑介幘,衣青紗袍。常伯陪乘,太僕秉轡。宜改儀注,一遵二《禮》,以為定儀。」詔可。



 大明四年正月己卯,有司奏:「南郊親奉儀注,皇帝初著平天冠,火龍黼黻之服。還,變通天冠,絳紗袍。廟祠親奉,舊儀,皇帝初服與郊不異,而還變著黑介幘,單衣即事,乖體。謂宜同郊還,亦變著通天冠,絳紗袍。又舊儀乘金根車,今五路既備,依《禮》玉路以祀,亦宜改金根車為玉路。」詔可。



 大明六年八月壬戌,有司奏:「《漢儀注》『大駕鹵簿,公卿奉引,大將軍參乘,太僕卿御。法駕,侍中參乘,奉車郎御』。晉
 氏江左,大駕未立,故郊祀用法駕,宗廟以小駕。至於儀服,二駕不異。拜陵,御服單衣幘,百官陪從,硃衣而已,亦謂之小駕,名實乖舛。考尋前記,大駕上陵,北郊。周禮宗廟於昊天有降,宜以大駕郊祀,法駕祠廟,小駕上陵,如為從序。今改祠廟為法駕鹵簿,其軍幢多少,臨時配之。至尊乘玉路,以金路象路革路木路小輦輪御軺衣書等車為副。其餘並如常儀。」詔可。大明七年二月甲寅,輿駕巡南豫、兗二州,冕服,御玉路,
 辭二廟。



 改服通天冠,御木路,建大麾,備春搜之典。



 明帝太始四年五月甲戌,尚書令建安王休仁參議:「天子之子,與士齒讓,達於辟雍,無生而貴者也。既命而尊,禮同上公。周制五等,車服相涉,公降王者,一等而已。王以金路賜同姓諸侯,象及革木,以賜異姓侯伯,在朝卿士,亦準斯禮。



 按如此制,則東宮應乘金路。自晉武過江,禮儀疏舛,王公以下,車服卑雜;唯有東宮,禮秩崇異,上次辰極,下絕侯王。而皇太子乘石山安車,義不見經,事
 無所出。《禮》所謂金、玉路者,正以金玉飾輅諸末耳。左右前後,同以漆畫。秦改周輅,制為金根,通以金薄,周匝四面。漢、魏、二晉,因循莫改。逮于大明,始備五輅。金玉二制,並類金根,造次瞻睹,殆無差別。若錫之東儲,於禮嫌重,非所以崇峻陛級,表示等威。且《春秋》之義,降下以兩,臣子之義,宜從謙約。謂東宮車服,宜降天子二等,驂駕四馬,乘象輅,降龍碧旂九葉。進不斥尊,退不逼下,沿古酌時,於禮為衷。」詔可。



 泰始四年八月甲寅,詔曰:「車服之飾,象數是遵。故盛皇留範,列聖垂制。



 朕近改定五路,酌古代今,修成六服,沿時變禮。所施之事,各有條敘;便可付外,載之典章。朕以大冕純玉繅,玄衣黃裳,乘玉輅,郊祀天,宗祀明堂。又以法冕五彩繅,玄衣絳裳,乘金路,祀太廟,元正大會諸侯。又以飾冠冕四彩繅,紫衣紅裳,乘象輅,小會宴饗,餞送諸侯,臨軒會王公。又以繡冕三綵繅,朱衣裳,乘革路,征伐不賓,講武校獵。又以宏冕二彩繅,青衣裳,乘木輅,耕
 稼,饗國子。又以通天冠,硃紗袍,為聽政之服。」



 泰始六年正月戊辰,有司奏:「被敕皇太子正冬朝駕,合著袞冕九章衣不?」



 儀曹郎丘仲起議:「案《周禮》,公自袞冕以下。鄭注:『袞冕以至卿大夫之玄冕,皆其朝聘天子之服也。』伏尋古之上公,尚得服袞以朝。皇太子以儲副之尊,率土瞻仰。愚謂宜式遵盛典,服袞冕九旒以朝賀。」兼左丞陸澄議:「服冕以朝,實著經典。秦除六冕之制,至漢明帝始與諸儒還備古章。自魏、晉以來,宗廟行禮之外,不
 欲令臣下服袞冕,故位公者,每加侍官。今皇太子承乾作副,禮絕群後,宜遵聖王之盛典,革近代之陋制。臣等參議,依禮,皇太子元正朝賀,應服袞冕九章衣。



 以仲起議為允。撰載儀注。」詔可。



 後廢帝即位,尊所生陳貴妃為皇太妃,輿服一如晉孝武太妃故事,唯省五牛旗及赤旗。



\end{pinyinscope}