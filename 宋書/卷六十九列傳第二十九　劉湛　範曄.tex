\article{卷六十九列傳第二十九 劉湛 範曄}

\begin{pinyinscope}

 劉湛,字弘仁,南陽涅陽人也。祖耽,父柳,並晉左光祿大夫、開府儀同三司。



 湛出繼伯父淡,襲封安眾縣五等男。少有局力,不尚浮華。博涉史傳,諳前世舊典,弱年便有
 宰世情,常自比管夷吾、諸葛亮,不為文章,不喜談議。本州辟主簿,不就。除著作佐郎,又不拜。高祖以為太尉行參軍,賞遇甚厚。高祖領鎮西將軍、荊州刺史,以湛為功曹,仍補治中別駕從事史,復為太尉參軍,世子征虜西中郎主簿。父柳亡於江州,州府送故甚豐,一無所受,時論稱之。服終,除秘書丞,出為相國參軍。謝晦、王弘並稱其有器幹。



 高祖入受晉命,以第四子義康為冠軍將軍、豫州刺史,留鎮壽陽。以湛為長史、梁郡太守。義康弱年未
 親政,府州軍事悉委湛。府進號右將軍,仍隨府轉。義康以本號徙為南豫州,湛改領歷陽太守。為人剛嚴用法,奸吏犯贓百錢以上,皆殺之,自下莫不震肅。廬陵王義真出為車騎將軍、南豫州刺史,湛又為長史,太守如故。



 義真時居高祖憂,使帳下備膳,湛禁之,義真乃使左右索魚肉珍羞,於齋內別立廚帳。會湛入,因命臑酒炙車螯,湛正色曰:「公當今不宜有此設。」義真曰:「旦甚寒,一碗酒亦何傷!長史事同一家,望不為異。」酒既至,湛因起曰:「
 既不能以禮自處,又不能以禮處人。」



 景平元年,召入,拜尚書吏部郎,遷右衛將軍。出督廣、交二州諸軍事、建威將軍、平越中郎將、廣州刺史。嫡母憂去職。服闋,為侍中。撫軍將軍江夏王義恭鎮江陵,以湛為使持節、南蠻校尉、領撫軍長史,行府州事。時王弘輔政,而王華、王曇首任事居中,湛自謂才能不後之,不願外出;是行也,謂為弘等所斥,意甚不平,常曰:「二王若非代邸之舊,無以至此,可謂遭遇風雲。」



 湛負其志氣,常慕汲黯、崔琰為人,故
 名長子曰黯字長孺,第二子曰琰字季珪。



 琰於江陵病卒,湛求自送喪還都,義恭亦為之陳請。太祖答義恭曰:「吾亦得湛啟事,為之酸懷,乃不欲茍違所請。但汝弱年,新涉庶務,八州殷曠,專斷事重,疇諮委仗,不可不得其人,量算二三,未獲便相順許。今答湛啟,權停彼葬。頃朝臣零落相係,寄懷轉寡,湛實國器,吾乃欲引其令還,直以西夏任重,要且停此事耳。



 汝慶賞黜罰,豫關失得者,必宜悉相委寄。」



 義恭性甚狷隘,年又漸長,欲專政事,每
 為湛所裁,主佐之間,嫌隙遂構。太祖聞之,密遣使詰讓義恭,并使深加諧緝。義恭具陳湛無居下之禮,又自以年長,未得行意,雖奉詔旨,頗有怨言。上友于素篤,欲加酬順,乃詔之曰:「事至於此,甚為可嘆。當今乏才,委授已爾,宜盡相彌縫,取其可取,棄其可棄。汝疏云『泯然無際』,如此甚佳。彼多猜,不可令萬一覺也。汝年已長,漸更事物,且群情矚望,不以幼昧相期,何由故如十歲時,動止諮問。但當今所專,必是小事耳。亦恐量此輕重,未必盡
 得,彼之疑怨,兼或由此邪。」



 先是,王華既亡,曇首又卒,領軍將軍殷景仁以時賢零落,白太祖征湛。八年,召為太子詹事,加給事中、本州大中正,與景仁並被任遇。湛常云:「今世宰相何難,此政可當我南陽郡漢世功曹耳。」明年,景仁轉尚書僕射、領選、護軍將軍,湛代為領軍將軍。十二年,又領詹事。湛與景仁素款,又以其建議征之,甚相感說。



 及俱被時遇,猜隙漸生,以景仁專管內任,謂為間己。



 時彭城王義康專秉朝權,而湛昔為上佐,遂以舊
 情委心自結,欲因宰相之力以回主心,傾黜景仁,獨當時務。義康屢構之於太祖,其事不行。義康僚屬及湛諸附隸潛相約勒,無敢歷殷氏門者。湛黨劉敬文父成未悟其機,詣景仁求郡,敬文遽往謝湛曰:「老父悖耄,遂就殷鐵干祿。由敬文暗淺,上負生成,合門慚懼,無地自處。」敬文之奸諂無愧如此。



 義康擅勢專朝,威傾內外,湛愈推崇之,無復人臣之禮,上稍不能平。湛初入朝,委任甚重,日夕引接,恩禮綢繆。善論治道,并諳前世故事,敘致
 銓理,聽者忘疲。每入雲龍門,御者便解駕,左右及羽儀隨意分散,不夕不出,以此為常。及至晚節,驅煽義康,凌轢朝廷,上意雖內離,而接遇不改。上嘗謂所親曰:「劉班初自西還,吾與語,常看日早晚,慮其當去。比入,吾亦看日早晚,慮其不去。」



 湛小字班虎,故云班也。遷丹陽尹,金紫光祿大夫,加散騎常侍,詹事如故。



 十七年,所生母亡。時上與義康形跡既乖,釁難將結,湛亦知無復全地。及至丁艱,謂所親曰:「今年必敗。常日正賴口舌爭之,故得
 推遷耳。今既窮毒,無復此望,禍至其能久乎!」十月,詔曰:「劉湛階藉門蔭,少叨榮位,往佐歷陽,奸詖夙著。謝晦之難,潛使密告,求心即事,久宜誅屏。朕所以棄罪略瑕,庶收後效,寵秩優忝,踰越倫匹。而凶忍忌克,剛愎靡厭,無君之心,觸遇斯發。遂乃合黨連群,構扇同異,附下蔽上,專弄威權,薦子樹親,互為表裏,邪附者榮曜九族,乘理者推陷必至。旋觀奸慝,為日已久,猶欲弘納遵養,冀或悛革。自邇以來,凌縱滋甚,悖言懟容,罔所顧忌,險謀潛
 計,睥睨兩宮。豈唯彰暴國都,固亦達于四海。



 比年七曜違度,震蝕表災,侵陽之徵,事符幽顯。搢紳含憤,義夫興歎。昔齊、魯不綱,禍頃邦國;昭、宣電斷,漢祚方延。便收付廷尉,肅明刑典。」於獄伏誅,時年四十九。



 子黯,大將軍從事中郎。黯及二弟亮、儼並從誅。湛弟素,黃門侍郎,徙廣州。



 湛初被收,嘆曰:「便是亂邪。」仍又曰:「不言無我應亂,殺我自是亂法耳。」



 入獄見素,曰:「乃復及汝邪?相勸為惡,惡不可為;相勸為善,正見今日。如何!」



 湛生女輒殺之,為士流
 所怪。



 範曄,字蔚宗,順陽人,車騎將軍泰少子也。母如廁產之,額為磚所傷,故以磚為小字。出繼從伯弘之,襲封武興縣五等侯。少好學,博涉經史,善為文章,能隸書,曉音律。年十七,州辟主簿,不就。高祖相國掾,彭城王義康冠軍參軍,隨府轉右軍參軍,入補尚書外兵郎,出為荊州別駕從事史。尋召為祕書丞,父憂去職。



 服終,為征南大將軍檀道濟司馬,領新蔡太守。道濟北征,曄憚行,辭以腳
 疾,上不許,使由水道統載器仗部伍。軍還,為司徒從事中郎。傾之,遷尚書吏部郎。



 元嘉元年冬,彭城太妃薨,將葬,祖夕,僚故並集東府。曄弟廣淵,時為司徒祭酒,其日在直。曄與司徒左西屬王深宿廣淵許,夜中酣飲,開北牖聽挽歌為樂。



 義康大怒,左遷曄宣城太守。不得志,乃刪眾家《後漢書》為一家之作。在郡數年,遷長沙王義欣鎮軍長史,加寧朔將軍。兄皓為宜都太守,嫡母隨皓在官。十六年,母亡,報之以疾,曄不時奔赴;及行,又攜妓妾
 自隨,為御史中丞劉損所奏。太祖愛其才,不罪也。服闋,為始興王濬後軍長史,領南下邳太守。及浚為揚州,未親政事,悉以委曄。尋遷左衛將軍、太子詹事。



 曄長不滿七尺,肥黑,禿眉鬚。善彈琵琶,能為新聲。上欲聞之,屢諷以微旨,曄偽若不曉,終不肯為上彈。上嘗宴飲歡適,謂曄曰:「我欲歌,卿可彈。」曄乃奉旨。上歌既畢,曄亦止弦。



 初,魯國孔熙先博學有縱橫才志,文史星算,無不兼善。為員外散騎侍郎,不為時所知,久不得調。初熙先父默之
 為廣州刺史,以贓貨得罪下廷尉,大將軍彭城王義康保持之,故得免。及義康被黜,熙先密懷報效,欲要朝廷大臣,未知誰可動者,以曄意志不滿,欲引之。而熙先素不為曄所重,無因進說。曄外甥謝綜,雅為曄所知,熙先嘗經相識,乃傾身事綜,與之結厚。熙先藉嶺南遺財,家甚富足,始與綜諸弟共博,故為拙行,以物輸之。綜等諸年少,既屢得物,遂日夕往來,情意稍款。綜乃引熙先與曄為數,曄又與戲,熙先故為不敵,前後輸曄物甚多。曄
 既利其財寶,又愛其文藝。熙先素有詞辯,盡心事之,曄遂相與異常,申莫逆之好。始以微言動曄,曄不回,熙先乃極辭譬說。曄素有閨庭論議,朝野所知,故門胄雖華,而國家不與姻娶。熙先因以此激之曰:「丈人若謂朝廷相待厚者,何故不與丈人婚,為是門戶不得邪?人作犬豕相遇,而丈人欲為之死,不亦惑乎?」曄默然不答,其意乃定。



 時曄與沈演之並為上所知待,每被見多同。曄若先至,必待演之俱入;演之先至,嘗獨被引,曄又以此為
 怨。曄累經義康府佐,見待素厚。及宣城之授,意好乖離。綜為義康大將軍記室參軍,隨鎮豫章。綜還,申義康意於曄,求解晚隙,復敦往好。曄既有逆謀,欲探時旨,乃言於上曰:「臣歷觀前史二漢故事,諸蕃王政以訞詛幸災,便正大逆之罰。況義康奸心釁跡,彰著遐邇,而至今無恙,臣竊惑焉。



 且大梗常存,將重階亂,骨肉之際,人所難言。臣受恩深重,故冒犯披露。」上不納。



 熙先素善天文,云:「太祖必以非道晏駕,當由骨肉相殘。江州應出天子。」



 以
 為義康當之。綜父述亦為義康所遇,綜弟約又是義康女夫,故太祖使綜隨從南上,既為熙先所獎說,亦有酬報之心。廣州人周靈甫有家兵部曲,熙先以六十萬錢與之,使於廣州合兵。靈甫一去不反。大將軍府史仲承祖,義康舊所信念,屢銜命下都,亦潛結腹心,規有異志。聞熙先有誠,密相結納。丹陽尹徐湛之,素為義康所愛,雖為舅甥,恩過子弟,承祖因此結事湛之,告以密計。承祖南下,申義康意於蕭思話及曄,云:「本欲與蕭結婚,恨
 始意不果。與范本情不薄,中間相失,傍人為之耳。」



 有法略道人,先為義康所供養,粗被知待;又有王國寺法靜尼亦出入義康家內,皆感激舊恩,規相拯拔,並與熙先往來。使法略罷道,本姓孫,改名景玄,以為臧質寧遠參軍。熙先善於治病,兼能診脈。法靜尼妹夫許耀,領隊在臺,宿衛殿省。



 嘗有病,因法靜尼就熙先乞治,為合湯一劑,耀疾即損。耀自往酬謝,因成周旋。



 熙先以耀膽幹可施,深相待結,因告逆謀,耀許為內應。豫章胡遵世,籓之
 子也,與法略甚款,亦密相酬和。法靜尼南上,熙先遣婢採藻隨之,付以箋書,陳說圖讖。



 法靜還,義康餉熙先銅匕、銅鑷、袍段、棋奩等物。熙先慮事泄,鴆採藻殺之。湛之又謂曄等:「臧質見與異常,歲內當還,已報質,悉攜門生義故,其亦當解人此旨,故應得健兒數百。質與蕭思話款密,當仗要之,二人並受大將軍眷遇,必無異同。思話三州義故眾力,亦不減質。郡中文武,及合諸處偵邏,亦當不減千人。不憂兵力不足,但當勿失機耳。」乃略相署
 置,湛之為撫軍將軍、揚州刺史,曄中軍將軍、南徐州刺史,熙先左衛將軍,其餘皆有選擬。凡素所不善及不附義康者,又有別簿,並入死目。熙先使弟休先先為檄文曰:夫休否相乘,道無恒泰,狂狡肆逆,明哲是殛。故小白有一匡之勳,重耳有翼戴之德。自景平肇始,皇室多故,大行皇帝天誕英姿,聰明睿哲,拔自籓國,嗣位統天,憂勞萬機,垂心庶務,是以邦內安逸,四海同風。而比年以來,奸豎亂政,刑罰乖淫,陰陽違舛,致使釁起蕭墻,危禍
 萃集。賊臣趙伯符積怨含毒,遂縱奸凶,肆兵犯蹕,禍流儲宰,崇樹非類,傾墜皇基。罪百浞、犬壹,過十玄、莽,開闢以來,未聞斯比。率土叩心,華夷泣血,咸懷亡身之誠,同思糜軀之報。



 湛之、曄與行中領軍蕭思話、行護軍將軍臧質、行左衛將軍孔熙先、建威將軍孔休先,忠貫白日,誠著幽顯,義痛其心,事傷其目,投命奮戈,萬殞莫顧,即日斬伯符首,及其黨與。雖豺狼即戮,王道惟新,而普天無主,群萌莫系。彭城王體自高祖,聖明在躬,德格天地,勳
 溢區宇,世路威夷,勿用南服,龍潛鳳棲,于茲六稔,蒼生饑德,億兆渴化,豈唯東征有《鴟鴞》之歌,陜西有勿翦之思哉!靈祗告徵祥之應,讖記表帝者之符,上答天心,下愜民望,正位辰極,非王而誰?



 今遣行護軍將軍臧質等,齎皇帝璽綬,星馳奉迎。百官備禮,駱驛繼進,並命群帥,鎮戍有常。若干撓義徒,有犯無貸。昔年使反,湛之奉賜手敕,逆誡禍亂,預睹斯萌,令宣示朝賢,共拯危溺,無斷謀事,失於後機,遂使聖躬濫酷,大變奄集,哀恨崩裂,撫
 心摧哽,不知何地,可以厝身。輒督厲尪頓,死而後已。



 熙先以既為大事,宜須義康意旨,曄乃作義康與湛之書,宣示同黨曰:吾凡人短才,生長富貴,任情用己,有過不聞,與物無恒,喜怒違實,致使小人多怨,士類不歸。禍敗已成,猶不覺悟,退加尋省,方知自招,刻肌刻骨,何所復補。然至於盡心奉上,誠貫幽顯,拳拳謹慎,惟恐不及,乃可恃寵驕盈,實不敢故為期罔也。豈苞藏逆心,以招灰滅,所以推誠自信,不復防護異同,率意信心,不顧萬物
 議論,遂致讒巧潛構,眾惡歸集。甲奸險好利,負吾事深;乙凶愚不齒,扇長無賴;丙、丁趨走小子,唯知諂進,伺求長短,共造虛說,致令禍陷骨肉,誅戮無辜。凡在過釁,竟有何征,而刑罰所加,同之元惡,傷和枉理,感徹天地。



 吾雖幽逼日苦,命在漏刻,義慨之士,時有音信。每知天文人事,及外間物情,土崩瓦解,必在朝夕。是為釁起群賢,濫延國家,夙夜憤踴,心復交戰。朝之君子及士庶白黑懷義秉理者,寧可不識時運之會,而坐待橫流邪。除君
 側之惡,非唯一代,況此等狂亂罪骫,終古所無,加之翦戮,易於摧朽邪。可以吾意宣示眾賢,若能同心奮發,族裂逆黨,豈非功均創業,重造宋室乎!但兵凶戰危,或致侵濫,若有一豪犯順,誅及九族。處分之要,委之群賢,皆當謹奉朝廷,動止聞啟。往日嫌怨,一時豁然,然後吾當謝罪北闕,就戮有司。茍安社稷,暝目無恨。勉之,勉之!



 二十二年九月,征北將軍衡陽王義季、右將軍南平王鑠出鎮,上於武帳岡祖道,曄等期以其日為亂,而差互不
 得發。於十一月,徐湛之上表曰:「臣與范曄,本無素舊,中忝門下,與之鄰省,屢來見就,故漸成周旋。比年以來,意態轉見,傾動險忌,富貴情深,自謂任遇未高,遂生怨望。非唯攻伐朝士,譏謗聖時,乃上議朝廷,下及籓輔,驅扇同異,恣口肆心,如此之事,已具上簡。近員外散騎侍郎孔熙先忽令大將軍府吏仲承祖騰曄及謝綜等意,慾收合不逞,規有所建。以臣昔蒙義康接盼,又去歲群小為臣妄生風塵,謂必嫌懼,深見勸誘。兼云人情樂亂,機
 不可失,讖緯天文,並有徵驗。曄尋自來,復具陳此,并說臣論議轉惡,全身為難。即以啟聞,被敕使相酬引,究其情狀。於是悉出檄書、選事、及同惡人名、手墨翰跡,謹封上呈,凶悖之甚,古今罕比。由臣暗於交士,聞此逆謀,臨啟震惶,荒情無措。」



 詔曰:「湛之表如此,良可駭惋。曄素無行檢,少負瑕釁,但以才藝可施,故收其所長,頻加榮爵,遂參清顯。而險利之性,有過谿壑,不識恩遇,猶懷怨憤。每存容養,冀能悛革,不謂同惡相濟,狂悖至此。便可收
 掩,依法窮詰。」



 其夜,先呼曄及朝臣集華林東閣,止於客省。先已於外收綜及熙先兄弟,並皆款服。于時上在延賢堂,遣使問曄曰:「以卿觕有文翰,故相任擢,名爵期懷,於例非少。亦知卿意難厭滿,正是無理怨望,驅扇朋黨而已,云何乃有異謀?」曄倉卒怖懼,不即首款。上重遣問曰:「卿與謝綜、徐湛之、孔熙先謀逆,並已答款,猶尚未死,徵據見存,何不依實。」曄對曰:「今宗室磐石,蕃嶽張跱,設使竊發僥幸,方鎮便來討伐,幾何而不誅夷。且臣位任
 過重,一階兩級,自然必至,如何以滅族易此。古人云:『左手據天下之圖,右手刎其喉,愚夫不為。』臣雖泥下,朝廷許其觕有所及,以理而察,臣不容有此。」上復遣問曰:「熙先近在華林門外,寧欲面辨之乎?」曄辭窮,乃曰:「熙先茍誣引臣,臣當如何!」熙先聞曄不服,笑謂殿中將軍沈邵之曰:「凡諸處分,符檄書疏,皆範曄所造及治定。云何於今方作如此抵蹋邪!」上示以墨迹,曄乃具陳本末,曰:「久欲上聞,逆謀未著。又冀其事消弭,故推遷至今。負國罪
 重,分甘誅戮。」



 其夜,上使尚書僕射何尚之視之,問曰:「卿事何得至此?」曄曰:「君謂是何?」尚之曰:「卿自應解。」曄曰:「外人傳庾尚書見憎,計與之無惡。謀遂之事,聞孔熙先說此,輕其小兒,不以經意。今忽受責,方覺為罪。君方以道佐世,使天下無冤。弟就死之後,猶望君照此心也。」明日,仗士送曄付廷尉,入獄,問徐丹陽所在,然後知為湛之所發。熙先望風吐款,辭氣不橈,上奇其才,遣人慰勞之曰:「以卿之才,而滯於集書省,理應有異志。此乃我負卿
 也。」又詰責前吏部尚書何尚之曰:「使孔熙先年將三十作散騎郎,那不作賊。」熙先於獄中上書曰:「囚小人猖狂,識無遠概,徒手旬意氣之小感,不料逆順之大方。與第二弟休先首為奸謀,干犯國憲,捴膾脯醢,無補尤戾。陛下大明含弘,量苞天海,錄其一介之節,猥垂優逮之詔。恩非望始,沒有遺榮,終古以來,未有斯比。夫盜馬絕纓之臣,懷璧投書之士,其行至賤,其過至微,由識不世之恩,以盡軀命之報,卒能立功齊、魏,致勳秦、楚。囚雖身陷禍
 逆,名節俱喪,然少也慷慨,竊慕烈士之遺風。但墜崖之木,事絕升躋,覆盆之水,理乖收汲。方當身膏鈇鉞,詒誡方來,若使魂而有靈,結草無遠。然區區丹抱,不負夙心,貪及視息,少得申暢。自惟性愛群書,心解數術,智之所周,力之所至,莫不窮攬,究其幽微。考論既往,誠多審驗。謹略陳所知,條牒如故別狀,願且勿遺棄,存之中書。若囚死之後,或可追存,庶九泉之下,少塞釁責。」所陳並天文占候,讖上有骨肉相殘之禍,其言深切。



 曄在獄,與綜
 及熙先異處,乃稱疾求移考堂,欲近綜等。見聽,與綜等果得隔壁。遙問綜曰:「始被收時,疑誰所告?」綜云:「不知。」曄曰:「乃是徐童。」



 童,徐湛之小名仙童也。在獄為詩曰:「禍福本無兆,性命歸有極。必至定前期,誰能延一息。在生已可知,來緣心畫無識。好醜共一丘,何足異枉直。豈論東陵上,寧辨首山側。雖無嵇生琴,庶同夏侯色。寄言生存子,此路行復即。」曄本意謂入獄便死,而上窮治其獄,遂經二旬,曄更有生望。獄吏因戲之曰:「外傳詹事或當長
 系。」曄聞之驚喜,綜、熙先笑之曰:「詹事當前共疇昔事時,無不攘袂瞋目。



 及在西池射堂上,躍馬顧盼,自以為一世之雄。而今擾攘紛紜,畏死乃爾。設令今時賜以性命,人臣圖主,何顏可以生存?」曄謂衛獄將曰:「惜哉!薶如此人。」



 將曰:「不忠之人,亦何足惜。」曄曰:「大將言是也。」



 將出市,曄最在前,於獄門顧謂綜曰:「今日次第,當以位邪?」綜曰:「賊帥為先。」在道語笑,初無暫止。至市,問綜曰:「時欲至未?」綜曰:「勢不復久。」曄既食,又苦勸綜,綜曰:「此異病篤,何事彊飯。」
 曄家人悉至市,監刑職司問:「須相見不?」曄問綜曰:「家人以來,幸得相見,將不暫別。」綜曰:「別與不別,亦何所存。來必當號泣,正足亂人意。」曄曰:「號泣何關人,向見道邊親故相瞻望,亦殊勝不見。吾意故欲相見。」於是呼前。曄妻先下撫其子,回罵曄曰:「君不為百歲阿家,不感天子恩遇,身死固不足塞罪,奈何枉殺子孫。」



 曄幹笑云罪至而已。曄所生母泣曰:「主上念汝無極,汝曾不能感恩,又不念我老,今日奈何?」仍以手擊曄頸及頰,曄顏色不怍。妻
 云:「罪人,阿家莫念。」妹及妓妾來別,曄悲涕流漣,綜曰:「舅殊不同夏侯色。」曄收淚而止。綜母以子弟自蹈逆亂,獨不出視。曄語綜曰:「姊今不來,勝人多也。」曄轉醉,子藹亦醉,取地土及果皮以擲曄,呼曄為別駕數十聲。曄問曰:「汝恚我邪?」藹曰:「今日何緣復恚,但父子同死,不能不悲耳。」曄常謂死者神滅,欲著《無鬼論》;至是與徐湛之書,云「當相訟地下」。其謬亂如此。又語人:「寄語何僕射,天下決無佛鬼。若有靈,自當相報。」收曄家,樂器服玩,並皆珍麗,
 妓妾亦盛飾,母住止單陋,唯有一廚盛樵薪,弟子冬無被,叔父單布衣。曄及子藹、遙、叔蔞、孔熙先及弟休先、景先、思先、熙先子桂甫、桂甫子白民、謝綜及弟約、仲承祖、許耀,諸所連及,並伏誅。曄時年四十八。曄兄弟子父已亡者及謝綜弟緯,徙廣州。藹子魯連,吳興昭公主外孫,請全生命,亦得遠徙,世祖即位得還。



 曄性精微有思致,觸類多善,衣裳器服,莫不增損制度,世人皆法學之。撰《和香方》,其序之曰:「麝本多忌,過分必害;沈實易和,盈斤
 無傷。零藿虛燥,詹唐粘濕。甘松、蘇合、安息、鬱金、捺多、和羅之屬,並被珍於外國,無取於中土。又棗膏昏鈍,甲煎淺俗」,非唯無助於馨烈,乃當彌增於尤疾也。」此序所言,悉以比類朝士:「麝本多忌」,比庾炳之;「零藿虛燥」,比何尚之;「詹唐粘濕」,比沈演之;「棗膏昏鈍」,比羊玄保;「甲煎淺俗」,比徐湛之;「甘松、蘇合」,比慧琳道人;「沈實易和」,以自比也。曄獄中與諸甥姪書以自序曰:吾狂釁覆滅,豈復可言,汝等皆當以罪人棄之。然平生行己任懷,猶應可尋。



 至
 於能不,意中所解,汝等或不悉知。吾少懶學問,晚成人,年三十許,政始有向耳。自爾以來,轉為心化,推老將至者,亦當未已也。往往有微解,言乃不能自盡。



 為性不尋注書,心氣惡,小苦思,便憒悶;口機又不調利,以此無談功。至於所通解處,皆自得之於胸懷耳。文章轉進,但才少思難,所以每於操筆,其所成篇,殆無全稱者。常恥作文士。文患其事盡於形,情急於藻,義牽其旨,韻移其意。雖時有能者,大較多不免此累,政可類工巧圖繢,竟無
 得也。常謂情志所托,故當以意為主,以文傳意。以意為主,則其旨必見;以文傳意,則其詞不流。然後抽其芬芳,振其金石耳。此中情性旨趣,千條百品,屈曲有成理。自謂頗識其數,嘗為人言,多不能賞,意或異故也。



 性別宮商,識清濁,斯自然也。觀古今文人,多不全了此處,縱有會此者,不必從根本中來。言之皆有實證,非為空談。年少中,謝莊最有其分,手筆差易,文不拘韻故也。吾思乃無定方,特能濟難適輕重,所稟之分,猶當未盡。但多公
 家之言,少於事外遠致,以此為恨,亦由無意於文名故也。



 本未關史書,政恒覺其不可解耳。既造《後漢》,轉得統緒,詳觀古今著述及評論,殆少可意者。班氏最有高名,既任情無例,不可甲乙辨。後贊於理近無所得,唯志可推耳。博贍不可及之,整理未必愧也。吾雜傳論,皆有精意深旨,既有裁味,故約其詞句。至於《循吏》以下及《六夷》諸序論,筆勢縱放,實天下之奇作。其中合者,往往不減《過秦》篇。嘗共比方班氏所作,非但不愧之而已。欲遍作
 諸志,前漢所有者悉令備。雖事不必多,且使見文得盡。又欲因事就卷內發論,以正一代得失,意復未果。贊自是吾文之傑思,殆無一字空設,奇變不窮,同合異體,乃自不知所以稱之。此書行,故應有賞音者。紀、傳例為舉其大略耳,諸細意甚多。自古體大而思精,未有此也。恐世人不能盡之,多貴古賤今,所以稱情狂言耳。



 吾於音樂,聽功不及自揮,但所精非雅聲,為可恨。然至於一絕處,亦復何異邪。其中體趣,言之不盡,弦外之意,虛
 響之音,不知所從而來。雖少許處,而旨態無極。亦嘗以授人,士庶中未有一毫似者。此永不傳矣。吾書雖小小有意,筆勢不快,餘竟不成就,每愧此名。



 曄《自序》並實,故存之。藹幼而整潔,衣服竟歲未嘗有塵點。死時年二十。



 曄少時,兄晏常云:「此兒進利,終破門戶。」終如晏言。



 史臣曰:古之人云:「利令智昏。」甚矣,利害之相傾。劉湛識用才能,實苞經國之略,豈不知移弟為臣,則君臣之道用,變兄成主,則兄弟之義殊乎。而義康數懷奸計,茍相
 崇說,與夫推長戟而犯魏闕,亦何以異哉!



\end{pinyinscope}