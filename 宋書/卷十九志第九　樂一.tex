\article{卷十九志第九 樂一}

\begin{pinyinscope}

 《易》曰:「先王作樂崇德,殷薦之上帝,以配祖考。」自黃帝至于三代,名稱不同。周衰凋缺,又為鄭衛所亂。魏文侯雖好古,然猶昏睡於古樂。於是淫聲熾而雅音廢矣。及秦
 焚典籍,《樂經》用亡。漢興,樂家有制氏,但能記其鏗鏘鼓舞,而不能言其義。周存六代之樂,至秦唯餘《韶》、《武》而已。始皇改周舞曰《五行》,漢高祖改《韶舞》曰《文始》,以示不相襲也。又造《武德舞》,舞人悉執干戚,以象天下樂己行武以除亂也。故高祖廟奏《武德》、《文始》、《五行》之舞。周又有《房中之樂》,秦改曰《壽人》。其聲,楚聲也,漢高好之;孝惠改曰《安世》。高祖又作《昭容樂》、《禮容樂》。《昭容》生於《武德》,《禮容》生於《文始》、《五行》也。漢初,又有《嘉至樂》,叔孫通因秦樂人制
 宗廟迎神之樂也。文帝又自造《四時舞》,以明天下之安和。蓋樂先王之樂者,明有法也;樂己所自作者,明有制也。孝景采《武德舞》作《昭德舞》,薦之太宗之廟。孝宣採《昭德舞》為《盛德舞》,薦之世宗之廟。漢諸帝奏《文始》、《四時》、《五行》之舞焉。



 武帝時,河間獻王與毛生等共採《周官》及諸子言樂事者,以著《樂記》,獻八佾之舞,與制氏不相殊。其內史中丞王定傳之,以授常山王禹。禹,成帝時為謁者,數言其義,獻記二十四卷。劉向校書,得二十三篇,然竟
 不用也。至明帝初,東平憲王蒼總定公卿之議,曰:「宗廟宜各奏樂,不應相襲,所以明功德也。承《文始》、《五行》、《武德》為《大武》之舞。」又制舞哥一章,薦之光武之廟。



 漢末大亂,眾樂淪缺。魏武平荊州,獲杜夔,善八音,常為漢雅樂郎,尤悉樂事,於是以為軍謀祭酒,使創定雅樂。時又有鄧靜、尹商,善訓雅樂,哥師尹胡能哥宗廟郊祀之曲,舞師馮肅、服養曉知先代諸舞,夔悉總領之。遠考經籍,近采故事,魏復先代古樂,自夔始也。而左延年等,妙善鄭聲,
 惟夔好古存正焉。



 文帝黃初二年,改漢《巴渝舞》曰《昭武舞》,改宗廟《安世樂》曰《正世樂》,《嘉至樂》曰《迎靈樂》,《武德樂》曰《武頌樂》,《昭容樂》曰《昭業樂》,《雲翹舞》曰《鳳翔舞》,《育命舞》曰《靈應舞》,《武德舞》曰《武頌舞》,《文始武舞》曰《大韶舞》,《五行舞》曰《大武舞》。其眾哥詩,多即前代之舊;唯魏國初建,使王粲改作登哥及《安世》、《巴渝》詩而已。



 明帝太和初,詔曰;「禮樂之作,所以類物表庸而不忘其本者也。凡音樂以舞為主,自黃帝《雲門》以下,至於周《大武》,皆太廟舞名也。然
 則其所司之官,皆曰太樂,所以總領諸物,不可以一物名。武皇帝廟樂未稱,其議定廟樂及舞,舞者所執,綴兆之制,聲哥之詩,務令詳備。樂官自如故為太樂。」太樂,漢舊名,後漢依讖改太予樂官,至是改復舊。於是公卿奏曰:「臣聞德盛而化隆者,則樂舞足以象其形容,音聲足以發其哥詠。故薦之郊廟,而鬼神享其和;用之朝廷,則君臣樂其度。使四海之內,遍知至德之盛,而光輝日新者,禮樂之謂也。故先王殷薦上帝,以配祖考,蓋當其時
 而制之矣。周之末世,上去唐、虞幾二千年,《韶箾》、《南》、《龠》、《武》、《象》之樂,風聲遺烈,皆可得而論也。由斯言之,禮樂之事,弗可以已。今太祖武皇帝樂,宜曰《武始之樂》。武,神武也;武,又跡也。言神武之始,又王跡所起也。高祖文皇帝樂,宜曰《咸熙之舞》。咸,皆也;熙,興也。言應受命之運,天下由之皆興也。至於群臣述德論功,建定烈祖之稱,而未制樂舞,非所以昭德紀功。夫哥以詠德,舞以象事。於文,文武為斌,兼秉文武,聖德所以章明也。臣等謹制樂舞名《
 章斌之舞》。昔《簫韶》九奏,親於虞帝之庭,《武》、《象》、《大武》,亦振於文、武之阼。特以顯其德教,著其成功,天下被服其光輝,習詠其風聲者也。自漢高祖、文帝各逮其時,而為《武德》、《四時》之舞,上考前代制作之宜,以當今成業之美,播揚弘烈,莫盛於《章斌》焉。《樂志》曰:『鐘磬干戚,所以祭先王之廟,又所以獻酬酳酢也。在宗廟之中,君臣莫不致敬;族長之中,長幼無不從和。』故仲尼答賓牟賈之問曰:『周道四達,禮樂交通。』《傳》云:『魯有禘樂,賓祭用之。』此皆祭禮
 大享,通用盛樂之明文也。今有事於天地宗廟,則此三舞宜並以為薦享;及臨朝大享,亦宜舞之。然後乃合古制事神訓民之道,關於萬世,其義益明。又臣等思惟,三舞宜有總名,可名《大鈞之樂》。鈞,平也。言大魏三世同功,以至隆平也。於名為美,於義為當。」



 尚書奏:「宜如所上。」帝初不許制《章斌之樂》;三請,乃許之。



 於是尚書又奏:「祀圓丘以下,《武始舞》者,平冕,黑介幘,玄衣裳,白領袖,絳領袖中衣,絳合幅褲,絳釭,黑韋鞮。《咸熙舞》者,冠委貌,其餘服
 如前。



 《章斌舞》者,與《武始》、《咸熙》舞者同服。奏於朝庭,則《武始舞》者,武冠,赤介幘,生絳袍單衣,絳領袖,皂領袖中衣,虎文畫合幅褲,白布釭,黑韋鞮。



 《咸熙舞》者,進賢冠,黑介幘,生黃袍單衣,白合幅褲,其餘服如前。」奏可。



 史臣案,《武始》、《咸熙》二舞,冠制不同,而云《章斌》與《武始》、《咸熙》同服,不知服何冠也?



 侍中繆襲又奏:「《安世哥》本漢時哥名。今詩哥非往詩之文,則宜變改。案《周禮》注云:《安世樂》,猶周《房中之樂》也。是以往昔議者,以《房中》哥后妃之德,所以風
 天下,正夫婦,宜改《安世》之名曰《正始之樂》。自魏國初建,故侍中王粲所作登哥《安世詩》,專以思詠神靈及說神靈鑒享之意。襲後又依哥省讀漢《安世哥》詠,亦說『高張四縣,神來燕享,嘉薦令儀,永受厥福』。無有《二南》后妃風化天下之言。今思惟往者謂《房中》為后妃之哥者,恐失其意。方祭祀娛神,登堂哥先祖功德,下堂哥詠燕享,無事哥后妃之化也。自宜依其事以名其樂哥,改《安世哥》曰《享神哥》。」奏可。案文帝已改《安世》為《正始》,而襲至是又
 改《安世》為《享神》,未詳其義。王粲所造《安世詩》,今亡。襲又奏曰:「文昭皇后廟,置四縣之樂,當銘顯其均奏次第,依太祖廟之名,號曰昭廟之具樂。」尚書奏曰:「禮,婦人繼夫之爵,同牢配食者,樂不異文。昭皇后今雖別廟,至於宮縣樂器音均,宜如襲議。」奏可。



 散騎常侍王肅議曰:「王者各以其禮制事天地,今說者據《周官》單文為經國大體,懼其局而不知弘也。漢武帝東巡封禪還,祠太一於甘泉,祭后土于汾陰,皆盡用其樂。言盡用者,為盡用宮縣
 之樂也。天地之性貴質者,蓋謂其器之不文爾,不謂庶物當復減之也。禮,天子宮縣,舞八佾。今祀圓丘方澤,宜以天子制,設宮縣之樂,八佾之舞。」衛臻、繆襲、左延年等咸同肅議。奏可。



 肅又議曰:「說者以為周家祀天,唯舞《雲門》;祭地,唯舞《咸池》;宗廟,唯舞《大武》,似失其義矣。周禮賓客皆作備樂。《左傳》:『王子頹享五大夫,樂及遍舞。』六代之樂也。然則一會之日,具作六代樂矣。天地宗廟,事之大者,賓客燕會,比之為細。《王制》曰:『庶羞不踰牲,燕衣不踰
 祭服。』可以燕樂而踰天地宗廟之樂乎?《周官》:『以六律、六呂、五聲、八音、六舞大合樂,以致鬼神,以和邦國,以諧萬民,以安賓客,以說遠人。』夫六律、六呂、五聲、八音,皆一時而作之,至於六舞獨分擘而用之,所以不厭人心也。又《周官》:『韎師掌教韎樂,祭祀則帥其屬而舞之,大享亦如之。』韎,東夷之樂也。又:『鞮鞻氏掌四夷之樂與其聲哥,祭祀則吹而哥之,燕亦如之。』四夷之樂,乃入宗廟;先代之典,獨不得用。大享及燕日如之者,明古今夷、夏之樂,皆
 主之於宗廟,而後播及其餘也。夫作先王樂者,貴能包而用之;納四夷之樂者,美德廣之所及也。高皇帝、太皇帝、太祖、高祖、文昭廟,皆宜兼用先代及《武始》、《太鈞》之舞。」有司奏:「宜如肅議。」奏可。肅私造宗廟詩頌十二篇,不被哥。晉武帝泰始二年,改制郊廟哥,其樂舞亦仍舊也。



 漢光武平隴、蜀,增廣郊祀,高皇帝配食,樂奏《青陽》、《硃明》、《西皓》、《玄冥》、《雲翹》、《育命》之舞。北郊及祀明堂,並奏樂如南郊。迎時氣五郊:春哥《青陽》,夏哥《朱明》,並舞《雲翹》之舞;秋哥《西
 皓》,冬哥《玄冥》,並舞《育命》之舞;季夏哥《朱明》,兼舞二舞。章帝元和二年,宗廟樂,故事,食舉有《鹿鳴》、《承元氣》二曲。三年,自作詩四篇,一曰《思齊皇姚》,二曰《六騏驎》,三曰《竭肅雍》,四曰《陟叱根》。合前六曲,以為宗廟食舉。加宗廟食舉《重來》、《上陵》二曲,合八曲為上陵食舉。減宗廟食舉《承元氣》一曲,加《惟天之命》、《天之歷數》二曲,合七曲為殿中御食飯舉。又漢太樂食舉十三曲:一曰《鹿鳴》,二曰《重來》,三曰《初造》,四曰《俠安》,五曰《歸來》,六曰《遠期》,七曰《有所思》,八
 曰《明星》,九曰《清涼》,十曰《涉大海》,十一曰《大置酒》,十二曰《承元氣》,十三曰《海淡淡》。魏氏及晉荀勖、傅玄並為哥辭。魏時以《遠期》、《承元氣》、《海淡淡》三曲多不通利,省之。魏雅樂四曲:一曰《鹿鳴》,後改曰《於赫》,詠武帝;二曰《騶虞》,後改曰《巍巍》,詠文帝;三曰《伐檀》,後省除;四曰《文王》,後改曰《洋洋》,詠明帝。《騶虞》、《伐檀》、《文王》並左延年改其聲。正旦大會,太尉奉璧,群后行禮,東廂雅樂郎作者是也。今謂之行禮曲,姑洗廂所奏。按《鹿鳴》本以宴樂為體,無當於朝享,
 往時之失也。



 晉武泰始五年,尚書奏使太僕傅玄、中書監荀勖、黃門侍郎張華各造正旦行禮及王公上壽酒食舉樂哥詩。詔又使中書郎成公綏亦作。張華表曰:「按魏上壽食舉詩及漢氏所施用,其文句長短不齊,未皆合古。蓋以依詠弦節,本有因循,而識樂知音,足以制聲,度曲法用,率非凡近所能改。二代三京,襲而不變,雖詩章詞異,興廢隨時,至其韶逗曲折,皆繫於舊,有由然也。是以一皆因就,不敢有所改易。」



 荀勖則曰:「魏氏哥詩,或
 二言,或三言,或四言,或五言,與古詩不類。」以問司律中郎將陳頎,頎曰:「被之金石,未必皆當。」故勖造晉哥,皆為四言,唯王公上壽酒一篇為三言五言,此則華、勖所明異旨也。九年,荀勖遂典知樂事,使郭瓊、宋識等造《正德》、《大豫》之舞,而勖及傅玄、張華又各造此舞哥詩。勖作新律笛十二枚,散騎常侍阮咸譏新律聲高,高近哀思,不合中和。勖以其異己,出咸為始平相。晉又改魏《昭武舞》曰《宣武舞》,《羽龠舞》曰《宣文舞》。咸寧元年,詔定祖宗之號,
 而廟樂同用《正德》、《大豫》之舞。



 至江左初立宗廟,尚書下太常祭祀所用樂名,太常賀循答云:「魏氏增損漢樂,以為一代之禮,未審大晉樂名所以為異。遭離喪亂,舊典不存,然此諸樂,皆和之以鐘律,文之以五聲,詠之於哥詞,陳之於舞列,宮縣在下,琴瑟在堂,八音迭奏,雅樂並作,登哥下管,各有常詠,周人之舊也。自漢氏以來,依放此禮,自造新詩而已。舊京荒廢,今既散亡,音韻曲折,又無識者,則於今難以意言。」于時以無雅樂器及伶人,省
 太樂并鼓吹令。是後頗得登哥,食舉之樂,猶有未備。明帝太寧末,又詔阮孚等增益之。成帝咸和中,乃復置太樂官,鳩習遺逸,而尚未有金石也。



 初,荀勖既以新律造二舞,又更修正鐘磬,事未竟而勖薨。惠帝元康三年,詔其子黃門侍郎籓修定金石,以施郊廟。尋值喪亂,遺聲舊制,莫有記者。庾亮為荊州,與謝尚共為朝廷修雅樂,亮尋薨。庾翼、桓溫專事軍旅,樂器在庫,遂至朽壞焉。晉氏之亂也,樂人悉沒戎虜。及胡亡,鄴下樂人,頗有來者。
 謝尚時為尚書僕射,因之以具鐘磬。太元中,破符堅,又獲樂工楊蜀等,閑練舊樂,於是四廂金石始備焉。宋文帝元嘉九年,太樂令鐘宗之更調金石。十四年,治書令史奚縱又改之。



 語在《律歷志》。晉世曹毗、王珣等亦增造宗廟哥詩,然郊祀遂不設樂。何承天曰:「世咸傳吳朝無雅樂。案孫皓迎父喪明陵,唯云倡伎晝夜不息,則無金石登哥可知矣。」承天曰:「或云今之《神絃》,孫氏以為宗廟登哥也。」史臣案陸機《孫權誄》「《肆夏》在廟,《雲翹》承□」,機不容
 虛設此言。又韋昭孫休世上《鼓吹鐃哥》十二曲表曰:「當付樂官善哥者習哥。」然則吳朝非無樂官,善哥者乃能以哥辭被絲管,寧容止以《神絃》為廟樂而已乎?



 宋武帝永初元年七月,有司奏:「皇朝肇建,廟祀應設雅樂,太常鄭鮮之等八十八人各撰立新哥。黃門侍郎王韶之所撰哥辭七首,並合施用。」詔可。十二月,有司又奏:「依舊正旦設樂,參詳屬三省改太樂諸哥舞詩。黃門侍郎王韶之立三十二章,合用教試,日近,宜逆誦習。輒申攝施行。」
 詔可。又改《正德舞》曰《前舞》,《大豫舞》曰《後舞》。元嘉十八年九月,有司奏:「二郊宜奏登哥。」又議宗廟舞事,錄尚書江夏王義恭等十二人立議同,未及列奏,值軍興,事寢。二十二年,南郊,始設登哥,詔御史中丞顏延之造哥詩,廟舞猶闕。



 孝建二年九月甲午,有司奏:「前殿中曹郎荀萬秋議:按禮,祭天地有樂者,為降神也。故《易》曰:『雷出地奮豫。先王以作樂崇德,殷薦之上帝,以配祖考。』《周官》曰:『作樂於圓丘之上,天神皆降。作樂於方澤之中,地祇皆出。』
 又曰:『乃奏黃鐘,哥大呂,舞《雲門》,以祀天神。乃奏太簇,哥應鐘,舞《咸池》,以祀地祇。』由斯而言,以樂祭天地,其來尚矣。今郊享闕樂,竊以為疑。《祭統》曰:『夫祭有三重焉,獻之屬莫重於祼,聲莫重於升哥,舞莫重於《武宿夜》,此周道也。』至於秦奏《五行》,魏舞《咸熙》,皆以用享。爰逮晉氏,太始之初,傅玄作晉郊廟哥詩三十二篇。元康中,荀籓受詔成父勖業,金石四縣,用之郊廟。



 是則相承郊廟有樂之證也。今廟祠登哥雖奏,而象舞未陳,懼闕備禮。夫聖王
 經世,異代同風,雖損益或殊,降殺迭運,未嘗不執古御今,同規合矩。方茲休明在辰,文物大備,禮儀遺逸,罔不具舉,而況出祇降神,輟樂於郊祭,昭德舞功,有闕於廟享。謂郊廟宜設備樂。」



 於是使內外博議。驃騎大將軍竟陵王誕等五十一人並同萬秋議。尚書左僕射建平王宏議以為:「聖王之德雖同,創制之禮或異,樂不相沿,禮無因襲。自寶命開基,皇符在運,業富前王,風通振古,朝儀國章,並循先代。自後晉東遷,日不暇給,雖大典略備,
 遺闕尚多。至於樂號廟禮,未該往正。今帝德再昌,大孝御宇,宜討定禮本,以昭來葉。尋舜樂稱《韶》,漢改《文始》,周樂《大武》,秦革《五行》。眷夫祖有功而宗有德,故漢高祖廟樂稱《武德》,太宗廟樂曰《昭德》。



 魏制《武始》舞武廟,制《咸熙》舞文廟。則祖宗之廟,別有樂名。晉氏之樂,《正德》、《大豫》,及宋不更名,直為《前》《後》二舞,依據昔代,義舛事乖。



 今宜釐改權稱,以《凱容》為《韶舞》,《宣烈》為《武舞》。祖宗廟樂,總以德為名。若廟非不毀,則樂無別稱,猶漢高、文、武,咸有嘉號,
 惠、景二主,樂無餘名。章皇太后廟,依諸儒議,唯奏文樂。何休、杜預、范寧注『初獻六羽』,並不言佾者,佾則乾在其中,明婦人無武事也。郊祀之樂,無復別名,仍同宗廟而已。



 尋諸《漢志》,《永至》等樂,各有義況,宜仍舊不改。爰及東晉,太祝唯送神而不迎神。近議者或云廟以居神,恒如在也,不應有迎送之事,意以為並乖其衷。立廟居靈,四時致享,以申孝思之情。夫神升降無常,何必恆安所處?故《祭義》云:『樂以迎來,哀以送往。』鄭注云:『迎來而樂,樂親
 之來;送往而哀,哀其享否,不可知也。』《尚書》曰『祖考來格』。又《詩》云:『神保遹歸。』注曰:『歸於天地也。』此並言神有去來,則有送迎明矣。即周《肆夏》之名,備迎送之樂。



 古以尸象神,故《儀禮》祝有迎尸送尸,近代雖無尸,豈可闕迎送之禮?又傅玄有迎神送神哥辭,明江左不迎,非舊典也。」



 散騎常侍、丹陽尹建城縣開國侯顏竣議以為:「德業殊稱,則干羽異容,時無沿制,故物有損益。至於禮失道愆,稱習忘反,中興釐運,視聽所革,先代繆章,宜見刊正。
 郊之有樂,蓋生《周易》、《周官》,歷代著議,莫不援準。夫『掃地而祭,器用陶匏』,唯質與誠,以章天德,文物之備,理固不然。《周官》曰:『國有故,則旅上帝及四望。』又曰:『四圭有邸,以祀天旅上帝。兩圭有邸,以祀地旅四望。』四望非地,則知上帝非天。《孝經》云:『郊祀后稷以配天,宗祀文王於明堂,以配上帝。』則《豫》之作樂,非郊天也。大司樂職,『奏黃鐘,哥大呂,舞《雲門》,以祀天神』。鄭注:『天神,五帝及日月星辰也。』王者以夏正月祀其所受命之帝於南郊,則二至之祀,
 又非天地。考之眾經,郊祀有樂,未見明證。宗廟之禮,事炳載籍。爰自漢元,迄乎有晉,雖時或更制,大抵相因,為不襲名號而已。今樂曲淪滅,知音世希,改作之事,臣聞其語。《正德》、《大豫》,禮容具存,宜殊其徽號,飾而用之。以《正德》為《宣化》之舞,《大豫》為《興和》之舞,庶足以光表世烈,悅被後昆。前漢祖宗,廟處各異,主名既革,舞號亦殊。今七廟合食,庭殿共所,舞蹈之容,不得廟有別制。後漢東平王蒼已議之矣。



 又王肅、韓祗以王者德廣無外,六代四
 夷之舞,金石絲竹之樂,宜備奏宗廟。愚謂蒼、肅、祗議,合於典禮,適於當今。」



 左僕射建平王宏又議:「竣據《周禮》、《孝經》,天與上帝,連文重出,故謂上帝非天,則《易》之作樂,非為祭天也。按《易》稱『先王以作樂崇德,殷薦之上帝,以配祖考』。《尚書》云:『肆類于上帝。』《春秋傳》曰:『告昊天上帝。』凡上帝之言,無非天也。天尊不可以一稱,故或謂昊天,或謂上帝,或謂昊天上帝,不得以天有數稱,便謂上帝非天。徐邈推《周禮》『國有故,則旅上帝』,以知禮天,旅上帝,同是
 祭天。言禮天者,謂常祀也;旅上帝者,有故而祭也。



 《孝經》稱『嚴父莫大於配天』,故云『郊祀后稷以配天,宗祀文王於明堂,以配上帝』。既天為議,則上帝猶天益明也。不欲使二天文同,故變上帝爾。《周禮》祀天之言再見,故鄭注以前天神為五帝,後冬至所祭為昊天。竣又云『二至之祀,又非天地』。未知天地竟應以何時致享?《記》云:『掃地而祭,器用陶匏。』旨明所用質素,無害以樂降神。萬秋謂郊宜有樂,事有典據。竣又云『東平王蒼以為前漢諸祖別
 廟,是以祖宗之廟可得各有舞樂。至於袷祭始祖之廟,則專用始祖之舞。



 故謂後漢諸祖,共廟同庭,雖有祖宗,不宜入別舞』。此誠一家之意,而未統適時之變也。後漢從儉,故諸祖共廟,猶以異室存別廟之禮。晉氏以來,登哥誦美,諸室繼作。至於祖宗樂舞,何猶不可迭奏。茍所詠者殊,雖復共庭,亦非嫌也。魏三祖各有舞樂,豈復是異廟邪?」眾議並同宏:「祠南郊迎神,奏《肆夏》。皇帝初登壇,奏登哥。初獻,奏《凱容》、《宣烈》之舞。送神,奏《肆夏》。祠廟迎神,
 奏《肆夏》。皇帝入廟門,奏《永至》。皇帝詣東壁,奏登哥。初獻,奏《凱容》、《宣烈之舞》。終獻,奏《永安》。送神奏《肆夏》。」詔可。



 孝建二年十月辛未,有司又奏:「郊廟舞樂,皇帝親奉,初登壇及入廟詣東壁,並奏登哥,不及三公行事。」左僕射建平王宏重參議:「公卿行事,亦宜奏登哥。」



 有司又奏:「元會及二廟齋祠,登哥依舊並於殿庭設作。尋廟祠,依新儀注,登哥人上殿,弦管在下;今元會,登哥人亦上殿,弦管在下。」並詔可。文帝章太后廟未有樂章,孝武大明中使尚
 書左丞殷淡造新哥,明帝又自造昭太后宣太后哥詩。



 後漢正月旦,天子臨德陽殿受朝賀,舍利從西方來,戲於殿前,激水化成比目魚,跳躍嗽水,作霧翳日;畢,又化成黃龍,長八九丈,出水游戲,炫耀日光。以兩大絲繩繫兩柱頭,相去數丈,兩倡女對舞,行於繩上,相逢切肩而不傾。



 魏晉訖江左,猶有《夏育扛鼎》、《巨象行乳》、《神龜抃舞》、《北負靈岳》、《桂樹白雪》、《畫地成川》之樂焉。



 晉成帝咸康七年,散騎侍郎顧臻表曰:「臣聞聖王制樂,贊揚治道,養以仁義,防其邪淫,上享宗廟,下訓黎民,體五行之正音,協八風以陶氣。以宮聲正方而好義,角聲堅齊而率禮,弦哥鐘鼓金石之作備矣。故通神至化,有率舞之感;移風改俗,致和樂之極。末世之伎,設禮外之觀,逆行連倒,頭足入筥之屬,皮膚外剝,肝心內摧。敦彼行葦,猶謂勿踐,矧伊生民,而不惻愴。加以四海朝覲,言觀帝庭,耳聆《雅》《頌》之聲,目睹威儀之序,足以蹋天,頭以
 履地,反兩儀之順,傷彞倫之大。方今夷狄對岸,外御為急,兵食七升,忘身赴難,過泰之戲,日稟五斗。



 方掃神州,經略中甸,若此之事,不可示遠。宜下太常,纂備雅樂,《簫韶》九成,惟新於盛運;功德頌聲,永著于來葉。此乃《詩》所以『燕及皇天,克昌厥後』者也。雜伎而傷人者,皆宜除之。流簡儉之德,邁康哉之詠,清風既行,民應如草,此之謂也。愚管之誠,唯垂采察。」於是除《高絙》、《紫鹿》、《跂行》、《鱉食》及《齊王捲衣》、《笮兒》等樂。又減其稟。其後復《高絙》、《紫鹿》焉。



 宋文帝元嘉十三年,司徒彭城王義康於東府正會,依舊給伎。總章工馮大列:「相承給諸王伎十四種,其舞伎三十六人。」太常傅隆以為:「未詳此人數所由。



 唯杜預注《左傳》佾舞云諸侯六六三十六人,常以為非。夫舞者,所以節八音者也。



 八音克諧,然後成樂。故必以八八為列,自天子至士,降殺以兩,兩者,減其二列爾。預以為一列又減二人,至士止餘四人,豈復成樂。按服虔注《傳》云:『天子八八,諸侯六八,大夫四八,士二八。』其義甚允。今諸王
 不復舞佾,其總章舞伎,即古之女樂也。殿庭八八,諸王則應六八,理例坦然。又《春秋》,鄭伯納晉悼公女樂二八,晉以一八賜魏絳,此樂以八人為列之證也。若如議者,唯天子八,則鄭應納晉二六,晉應賜絳一六也。自天子至士,其文物典章,尊卑差級,莫不以兩,未有諸侯既降二列,又列輒減二人,近降太半,非唯八音不具,於兩義亦乖,杜氏之謬可見矣。國典事大,宜令詳正。」事不施行。



 民之生,莫有知其始也。含靈抱智,以生天地之間。夫喜
 怒哀樂之情,好得惡失之性,不學而能,不知所以然而然者也。怒則爭鬥,喜則詠哥。夫哥者,固樂之始也。詠哥不足,乃手之舞之,足之蹈之,然則舞又哥之次也。詠哥舞蹈,所以宣其喜心,喜而無節,則流淫莫反。故聖人以五聲和其性,以八音節其流,而謂之樂,故能移風易俗,平心正體焉。昔有娥氏有二女,居九成之臺。天帝使燕夜往,二女覆以玉筐,既而發視之,燕遺二卵,五色,北飛不反。二女作哥,始為北音。禹省南土,嵞山之女令其妾
 候禹於嵞山之陽,女乃作哥,始為南音。夏后孔甲田於東陽萯山,天大風晦冥,迷入民室。主人方乳,或曰:「后來是良日也,必大吉。」



 或曰:「不勝之子,必有殃。」后乃取以歸,曰:「以為餘子,誰敢殃之?」後析,斧破斷其足。孔甲曰:「鳴呼!有命矣。」乃作《破斧》之哥,始為東音。



 周昭王南征,殞於漢中。王右辛餘靡長且多力,振王北濟,周公乃封之西翟,徙宅西河,追思故處作哥,始為西音。此蓋四方之哥也。



 黃帝、帝堯之世,王化下洽,民樂無事,故因擊壤之歡,
 慶雲之瑞,民因以作哥。其後《風》衰《雅》缺,而妖淫靡漫之聲起。



 周衰,有秦青者,善謳,而薛談學謳於秦青,未窮青之伎而辭歸。青餞之於郊,乃撫節悲歌,聲震林木,響遏行雲。薛談遂留不去,以卒其業。又有韓娥者,東之齊,至雍門,匱糧,乃鬻哥假食。既而去,餘響繞梁,三日不絕。左右謂其人不去也。過逆旅,逆旅人辱之,韓娥因曼聲哀哭,一里老幼,悲愁垂涕相對,三日不食。



 遽而追之,韓娥還,復為曼聲長哥,一里老幼,喜躍抃舞,不能自禁,忘向
 之悲也。



 乃厚賂遣之。故雍門之人善哥哭,效韓娥之遺聲。衛人王豹處淇川,善謳,河西之民皆化之。齊人綿駒居高唐,善哥,齊之右地,亦傳其業。前漢有虞公者,善哥,能令梁上塵起。若斯之類,並徒哥也。《爾雅》曰:「徒哥曰謠。」



 凡樂章古詞,今之存者,並漢世街陌謠謳,《江南可采蓮》、《烏生》、《十五子》、《白頭吟》之屬是也。吳哥雜曲,並出江東,晉、宋以來,稍有增廣。



 《子夜哥》者,有女子名子夜,造此聲。晉孝武太元中,琅邪
 王軻之家有鬼哥《子夜》。殷允為豫章時,豫章僑人庾僧虔家亦有鬼哥《子夜》。殷允為豫章,亦是太元中,則子夜是此時以前人也。《鳳將雛哥》者,舊曲也。應琚《百一詩》云:「為作《陌上桑》,反言《鳳將雛》。」然則《鳳將雛》其來久矣,將由訛變以至於此乎?



 《前溪哥》者,晉車騎將軍沈玩所制。



 《阿子》及《歡聞哥》者,晉穆帝升平初,哥畢輒呼「阿子!汝聞不?」語在《五行志》。後人演其聲,以為二曲。《
 團扇哥》者,晉中書令王氏與嫂婢有情,愛好甚篤,嫂捶撻婢過苦,婢素善哥,而氏好捉白團扇,故制此哥。《督護哥》者,彭城內史徐逵之為魯軌所殺,宋高祖使府內直督護丁旿收斂殯埋之。逵之妻,高祖長女也,呼旿至閣下,自問斂送之事,每問,輒歎息曰:「丁督護!」其聲哀切,後人因其聲,廣其曲焉。《懊憹哥》者,晉隆安初,民間訛謠之曲。語在《五行志》。



 宋少帝更制新哥,太祖常謂之《中朝曲》。《
 六變》諸曲,皆因事制哥。《長史變》者,司徒左長史王廞臨敗所制。《讀曲哥》者,民間為彭城王義康所作也。其哥云「死罪劉領軍,誤殺劉第四」是也。凡此諸曲,始皆徒哥,既而被之弦管。又有因弦管金石,造哥以被之,魏世三調哥詞之類是也。



 古者天子聽政,使公卿大夫獻詩,耆艾修之,而後王斟酌焉。秦、漢闕采詩之官,哥詠多因前代,與時事既不相
 應,且無以垂示後昆。漢武帝雖頗造新哥,然不以光揚祖考、崇述正德為先,但多詠祭祀見事及其祥瑞而已。商周《雅頌》之體闕焉。



 《鞞舞》,未詳所起,然漢代已施於燕享矣。傅毅、張衡所賦,皆其事也。曹植《鞞舞哥序》曰:「漢靈帝《西園故事》,有李堅者,能《鞞舞》。遭亂,西隨段煨。先帝聞其舊有技,召之。堅既中廢,兼古曲多謬誤,異代之文,未必相襲,故依前曲改作新哥五篇,不敢充之黃門,近以成下國之陋樂焉。」晉《
 鞞舞哥》亦五篇,又《鐸舞哥》一篇,《幡舞哥》一篇,《鼓舞伎》六曲,並陳於元會。今《幡》、《鼓》哥詞猶存,舞並闕。《鞞舞》,即今之《鞞扇舞》也。又云晉初有《杯槃舞》、《公莫舞》。史臣按:杯盤,今之《齊世寧》也。張衡《舞賦》云:「歷七槃而縱躡。」王粲《七釋》云:「七槃陳於廣庭。」近世文士顏延之云:「遞間關於槃扇。」鮑昭云:「七槃起長袖。」皆以七槃為舞也。《搜神記》云:「晉太康中,天下為《晉世寧舞》,矜手以接杯盤反覆之。」此則漢世唯有盤舞,而晉加之以杯,反覆之也。



 《
 公莫舞》,今之巾舞也。相傳云項莊劍舞,項伯以袖隔之,使不得害漢高祖。



 且語莊云:「公莫。」古人相呼曰「公」,云莫害漢王也。今之用巾,蓋像項伯衣袖之遺式。按《琴操》有《公莫渡河曲》,然則其聲所從來已久,欲云項伯,非也。



 江左初,又有《拂舞》。舊云《拂舞》,吳舞。檢其哥,非吳詞也,皆陳於殿庭。揚泓《拂舞序》曰:「自到江南,見《白符舞》,或言《白鳧鳩舞》,云有此來數十年。察其詞旨,乃是吳人患孫皓虐政,思屬晉也。」又有《白珝舞》,按舞詞有巾袍之言;珝本
 吳地所出,宜是吳舞也。晉《俳歌》又云:「皎皎白緒,節節為雙。」吳音呼緒為珝,疑白珝即白緒。



 《鞞舞》,故二八,桓玄將即真,太樂遣眾伎,尚書殿中郎袁明子啟增滿八佾,相承不復革。宋明帝自改舞曲哥詞,並詔近臣虞龢並作。又有西、傖、羌、胡諸雜舞。隨王誕在襄陽,造《襄陽樂》;南平穆王為豫州,造《壽陽樂》;荊州刺史沈攸之又造《西烏飛哥曲》,並列於樂官。哥詞多淫哇不典正。



 前世樂飲,酒酣,必起自舞。《詩》云「屢舞仙仙」是也。宴樂必
 舞,但不宜屢爾。譏在屢舞,不譏舞也。漢武帝樂飲,長沙定王舞又是也。魏、晉已來,尤重以舞相屬。所屬者代起舞,猶若飲酒以杯相屬也。謝安舞以屬桓嗣是也。近世以來,此風絕矣。



 孝武大明中,以《鞞》、《拂》、雜舞合之鐘石,施於殿庭。順帝昇明二年,尚書令王僧虔上表言之,並論三調哥曰:「臣聞《風》、《雅》之作,由來尚矣。



 大者系乎興衰,其次者著於率舞。在於心而木石感,鏗鏘奏而國俗移。故鄭相出郊,辯聲知戚;延陵入聘,觀樂知風。是則音不妄
 啟,曲豈徒奏。哥倡既設,休戚已征,清濁是均,山琴自應。斯乃天地之靈和,升降之明節。今帝道四達,禮樂交通,誠非寡陋所敢裁酌。伏以三古缺聞,六代潛響,舞詠與日月偕湮,精靈與風雲俱滅。



 追余操而長懷,撫遺器而太息,此則然矣。夫鐘縣之器,以雅為用,凱容之制,八佾為體。故羽龠擊拊,以相諧應,季氏獲誚,將在於此。今總章舊佾二八之流,袿服既殊,曲律亦異,推今校古,皎然可知。又哥鐘一肆,克諧女樂,以哥為稱,非雅器也。大明
 中,即以宮縣合和《鞞》、《拂》,節數雖會,慮乖雅體。將來知音,或譏聖世。若謂鐘舞已諧,不欲廢罷,別立哥鐘,以調羽佾,止於別宴,不關朝享,四縣所奏,謹依雅則,斯則舊樂前典,不墜於地。臣昔已制哥磬,猶在樂官,具以副鐘,配成一部,即義沿理,如或可安。又今之《清商》,實由銅雀,魏氏三祖,風流可懷,京、洛相高,江左彌重。諒以金縣干戚,事絕於斯。而情變聽改,稍復零落,十數年間,亡者將半。自頃家競新哇,人尚謠俗,務在噍危,不顧律紀,流宕無
 涯,未知所極,排斥典正,崇長煩淫。士有等差,無故不可以去禮;樂有攸序,長幼不可以共聞。故喧醜之制,日盛於廛里;風味之韻,獨盡於衣冠。夫川震社亡,同災異戒,哀思靡漫,異世齊歡。咎徵不殊,而欣畏並用,竊所未譬也。方今塵靜畿中,波恬海外,《雅》《頌》得所,實在茲辰。臣以為宜命典司,務勤課習,緝理舊聲,迭相開曉,凡所遺漏,悉使補拾。曲全者祿厚,藝敏者位優。利以動之,則人思自勸;風以靡之,可不訓自革。反本還源,庶可跂踵。」詔曰:「
 僧虔表如此。夫鐘鼓既陳,《雅》《頌》斯辨,所以惠感人祇,化動翔泳。頃自金龠弛韻,羽佾未凝,正俗移風,良在茲日。昔阮咸清識,王度昭奇,樂緒增修,異世同功矣。



 便可付外遵詳。」



 樂器凡八音:曰金,曰石,曰土,曰革,曰絲,曰木,曰匏,曰竹。



 八音一曰金。金,鐘也,褲也,錞也,鐲也,鐃也。,鐸也。鐘者,《世本》云「黃帝工人垂所造。」《爾雅》云「大鐘曰鏞」。《書》曰「笙鏞以間」是也。



 中者曰剽,剽音瓢。小者曰棧,棧音盞,晉江左初
 所得棧鐘是也。縣鐘磬者曰筍虡,橫曰筍,從曰虡。蔡邕曰:「寫鳥獸之形,大聲有力者以為鐘虡,清聲無力者以為磬虛,擊其所縣,知由其虡鳴焉。」褲如鐘而大。史臣案:前代有大鐘,若周之無射,非一,皆謂之鐘;褲之言,近代無聞焉。



 筼,筼于也。圓如碓頭,大上小下,今民間猶時有其器。《周禮》,「以金筼和鼓」。



 鐲,鉦也。形如小鐘,軍行鳴之,以為鼓節。《周禮》,「以金鐲節
 鼓」。



 鐃,如鈴而無舌,有柄,執而鳴之。《周禮》,「以金鐃止鼓」。漢《鼓吹曲》曰鐃哥。



 鐸,大鈴也。《周禮》,「以金鐸通鼓」。



 八音二曰石。石,磬也。《世本》云叔所造,不知叔何代人。《爾雅》曰:「形似犁簹,以玉為之。」大曰綍。綍音囂。



 八音三曰土。土,塤也。《世本》云,暴新公所造,亦不知何代人也。周畿內有暴國,豈其時人乎?燒土為之,大如鵝卵,銳
 上平底,形似稱錘,六孔。《爾雅》云,大者曰祇,祇音叫。「小者如雞子」。



 八音四曰革。革,鼓也,鞉也,節也。大曰鼓,小曰朄,又曰應。應劭《風俗通》曰:「不知誰所造。」以桴擊之曰鼓,以手搖之曰鞉。鼓及鞉之八面者曰雷鼓、雷鞉;六面者曰靈鼓、靈鞉;四面者曰路鼓、路鞉。《周禮》:「以雷鼓祀天神,以靈鼓鼓社祭,以路鼓致鬼享。」鼓長八尺者曰鼖鼓,以鼓軍事。長丈二尺者曰鼛鼓,凡守備及役事則鼓之。今世謂之下鼜。
 鼜,《周禮》音戚,今世音切豉反。



 長六尺六寸者曰晉鼓,金奏則鼓之。應鼓在大鼓側,《詩》云「應朄懸鼓」是也。



 小鼓有柄曰鞀。大鞀謂之鞞。《月令》「仲夏修鞀、鞞」。是也。然則鞀、鞞即鞉類也。又有鼉鼓焉。



 節,不知誰所造。傅玄《節賦》云:「黃鐘唱哥,《九韶》興舞。口非節不詠,手非節不拊。」此則所從來亦遠矣。



 八音五曰絲。絲,琴、瑟也,築也,箏也,琵琶、空侯也。



 琴,馬融《笛賦》云:「宓羲造琴。」《世本》云:「神農所造。」《爾雅》「大琴曰離」,二
 十絃。今無其器。齊桓曰號鐘,楚莊曰繞梁,相如曰燋尾,伯喈曰綠綺,事出傅玄《琴賦》。世云燋尾是伯喈琴,伯喈傳亦云爾。以傅氏言之,則非伯喈也。



 瑟,馬融《笛賦》云「神農造瑟。」世本,「宓羲所造」。《爾雅》云:「瑟二十七弦者曰灑。」今無其器。築,不知誰所造。史籍唯云高漸離善擊築。



 箏,秦聲也。傅玄《箏賦序》曰:「世以為蒙恬所造。今觀其體合法度,節究哀樂,乃仁智之器,豈亡國之臣所能關思
 哉?」《風俗通》則曰:「築身而瑟絃。」



 不知誰所改作也。



 琵琶,傅玄《琵琶賦》曰:「漢遣烏孫公主嫁昆彌,念其行道思慕,故使工人裁箏、築,為馬上之樂。欲從方俗語,故名曰琵琶,取其易傳於外國也。」《風俗通》云:「以手琵琶,因以為名。」杜摯云:「長城之役,弦鞀而鼓之。」並未詳孰實。其器不列四廂。



 空侯,初名坎侯。漢武帝賽滅南越,祠太一后土用樂,令樂人侯暉依琴作坎侯,言其坎坎應節奏也。侯者,因工人姓爾。後言空,音訛也。古施郊廟雅樂,近世
 來專用於楚聲。宋孝武帝大明中,吳興沈懷遠被徙廣州,造繞梁,其器與空侯相似。



 懷遠後亡,其器亦絕。



 八音六曰木。木,柷也,敔也。並不知誰所造。《樂記》曰:「聖人作為控、楬、塤、篪。」所起亦遠矣。柷如漆筒,方二尺四寸,深尺八寸,中有椎柄,連底挏之,令左右擊敔。,狀如伏虎,背上有二十七鉏鋙。以竹長尺名曰止,橫擽之,以節樂終也。



 八音七曰匏。匏,笙也,竽也。笙,隨所造,不知何代人。列管
 匏內,施簧管端。宮管在中央,三十六簧曰竽;宮管在左傍,十九簧至十三簧曰笙。其它皆相似也。竽今亡。「大笙謂之巢,小者謂之和」。其笙中之簧,女媧所造也。《詩》傳云:「吹笙則簧鼓矣。」蓋笙中之簧也。《爾雅》曰:「笙十九簧者曰巢。」漢章帝時,零陵文學奚景於舜祠得笙,白玉管。後世易之以竹乎。



 八音八曰竹。竹,律也,呂也,簫也,管也,篪也,龠也,笛也。律呂在《律歷志》。



 簫,《世本》云:「舜所造。」《爾雅》曰:「編二十三管,尺四寸者曰言;十六管長尺二寸者筊。」筊者爻。凡簫一名籟。前世有洞簫,其器今亡。蔡邕曰:「簫,編竹有底。」然則邕時無洞簫矣。



 管,《爾雅》曰:「長尺,圍寸,併漆之,有底。」大者曰簥。簥音驕;中者曰篞;小者曰篎,篎音妙。古者以玉為管,舜時西王母獻白玉琯是也。《月令》:「均琴、瑟、管、簫。」蔡邕章句曰:「管者,形長尺,圍寸,有孔無底。」其器今亡。



 篪,《世本》云:「暴新公所造。」舊志云,一曰管。史臣案:非也。雖
 不知暴新公何代人,而非舜前人明矣。舜時西王母獻管,則是已有其器,新公安得造篪乎?《爾雅》曰:「篪,大者尺四寸,圍三寸,曰沂。」沂音銀,一名翹。「小者尺二寸」。今有胡篪,出於胡吹,非雅器也。



 籥,不知誰所造。《周禮》有籥師,掌教國子秋冬吹籥。今《凱容》、《宣烈》舞所執羽籥是也。蓋《詩》所云「左手執籥,右手秉翟」者也。《爾雅》云:「籥如笛,三孔而短小。」《廣雅》云,七孔。大者曰產,中者曰仲,小者曰箹。箹音握。



 笛,案馬融《長笛賦》,此器起近世,出於羌中,京房備其五音。又稱丘仲工其事,不言仲所造。《風俗通》則曰:「丘仲造笛,武帝時人。」其後更有羌笛爾。



 三說不同,未詳孰實。



 絪,杜摯《笳賦》云:「李伯陽入西戎所造。」漢舊注曰:「箛,號曰吹鞭。



 《晉先蠶注》:「車駕住,吹小箛發,吹大箛。」箛即絪也。又有胡笳。漢舊《箏笛錄》有其曲,不記所出本末。



 鼓吹,蓋短簫鐃哥。蔡邕曰:「軍樂也,黃帝岐伯所作,以揚德建武,勸士諷敵也」《周官》曰:「師有功則愷樂。」《左傳》曰,晉
 文公勝楚,「振旅,凱而入」。《司馬法》曰:「得意則愷樂愷哥。」雍門周說孟嘗君,「鼓吹于不測之淵」。



 說者云,鼓自一物,吹自竽、籟之屬,非簫、鼓合奏,別為一樂之名也。然則短簫鐃哥,此時未名鼓吹矣。應劭漢《鹵簿圖》,唯有騎執箛。箛即笳,不云鼓吹,而漢世有黃門鼓吹。漢享宴食舉樂十三曲,與魏世鼓吹長簫同。長簫短簫,《伎錄》並云,絲竹合作,執節者哥。又《建初錄》云,《務成》、《黃爵》、《玄雲》、《遠期》,皆騎吹曲,非鼓吹曲。此則列於殿庭者為鼓吹,今之從行鼓吹
 為騎吹,二曲異也。又孫權觀魏武軍,作鼓吹而還,此又應是今之鼓吹。魏、晉世,又假諸將帥及牙門曲蓋鼓吹,斯則其時謂之鼓吹矣。魏、晉世給鼓吹甚輕,牙門督將五校,悉有鼓吹。晉江左初,臨川太守謝摛每寢,輒夢聞鼓吹。有人為其占之曰:「君不得生鼓吹,當得死鼓吹爾。」摛擊杜韜戰沒,追贈長水校尉,葬給鼓吹焉。謝尚為江夏太守,詣安西將軍庾翼於武昌咨事,翼與尚射,曰:「卿若破的,當以鼓吹相賞。」尚射破的,便以其副鼓吹給之。
 今則甚重矣。



 角,書記所不載。或云出羌胡,以驚中國馬;或云出吳越。舊志云:「古樂有籟、缶。」今並無。史臣按:《爾雅》,籟自是簫之一名耳。《詩》云:「坎其擊缶。」毛傳曰:「盎謂之缶。」



 築城相杵者,出自梁孝王。孝王築睢陽城,方十二里,造倡聲,以小鼓為節,築者下杵以和之。後世謂此聲為《睢陽曲》,至今傳之。



 魏、晉之世,有孫氏善弘舊曲,宋識善擊節倡和,陳左善
 清哥,列和善吹笛,郝索善彈箏,硃生善琵琶,尤發新聲。傅玄著書曰:「人若欽所聞而忽所見,不亦惑乎!設此六人生於上世,越古今而無儷,何但夔、牙同契哉!」案此說,則自茲以後,皆孫、硃等之遺則也。



\end{pinyinscope}