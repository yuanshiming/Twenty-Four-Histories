\article{卷七十四列傳第三十四 臧質 魯爽 沈攸之}

\begin{pinyinscope}

 臧質,字含文,東莞莒人。父熹,字義和,武敬皇后弟也。與兄燾並好經籍。



 隆安初,兵革屢起,熹乃習騎射,志在立功。嘗至溧陽,溧陽令阮崇與熹共獵,值虎突圍,獵徒並
 奔散,熹直前射之,應弦而倒。高祖入京城,熹族子穆斬桓脩。進至京邑,桓玄奔走,高祖使熹入宮收圖書器物,封閉府庫。有金飾樂器,高祖問熹:「卿得無欲此乎?」熹正色曰:「皇上幽逼,播越非所。將軍首建大義,劬勞王家。



 雖復不肖,無情於樂。」高祖笑曰:「聊以戲卿爾。」行參高祖鎮軍事,員外散騎侍郎,重參鎮軍軍事,領東海太守。以建義功封始興縣五等侯。又參高祖車騎、中軍軍事。高祖將征廣固,議者多不同。熹從容言曰:「公若凌威北境,拯其
 塗炭,寧一六合,未為無期。」高祖曰:「卿言是也。」及行,熹求從,不許,以為建威將軍、臨海太守。郡經兵寇,百不存一,熹綏緝綱紀,招聚流散,歸之者千餘家。



 孫季高海道襲廣州,路由臨海,熹資給發遣,得以無乏。徵拜散騎常侍,母憂去職。



 頃之,討劉毅,起為寧朔將軍,從征。事平,高祖遣朱齡石統大眾伐蜀,命熹奇兵出中水,以本號領建平、巴東二郡太守。蜀主譙縱遣大將譙撫之萬餘人屯牛脾,又遣譙小茍重兵塞打鼻。熹至牛脾,撫之戰敗退
 走,追斬之。小茍聞撫之死,即便奔散。成都既平,熹遇疾。義熙九年,卒於蜀郡牛脾縣,時年三十九。追贈光祿勳。



 質少好鷹犬,善蒱博意錢之戲。長六尺七寸,出面露口,禿頂拳髮。年未二十,高祖以為世子中軍行參軍。永初元年,為員外散騎侍郎,從班例也。母憂去職。服闋,為江夏王義恭撫軍,以輕薄無檢,為太祖所知,徙為給事中。會稽宣長公主每為之言,乃出為建平太守,甚得蠻楚心。南蠻校尉劉湛還朝,稱為良守。遷寧遠將軍、歷陽太
 守。仍遷竟陵、江夏內史,復為建武將軍、巴東、建平二郡太守,吏民便之。



 質年始出三十,屢居名郡,涉獵史籍,尺牘便敏,既有氣幹,好言兵權。太祖謂可大任,欲以為益州事,未行,徵為使持節、都督徐兗二州諸軍事、寧遠將軍、徐兗二州刺史。在鎮奢費,爵命無章,為有司所糾,遇赦。與范曄、徐湛之等厚善,曄謀反,量質必與之同,會事發,復為建威將軍、義興太守。元嘉二十六年,太祖謁京陵,質朝丹徒,與何勖、檀和之並功臣子,時共上禮。太祖
 設燕盡歡,賜布千匹。



 二十七年春,遷南譙王義宣司馬、寧朔將軍、南平內史。未之職,會索虜大帥拓跋燾圍汝南,汝南戍主陳憲固守告急。太祖遣質輕往壽陽,即統彼軍,與安蠻司馬劉康祖等救憲。虜退走,因使質伐汝南西境刀壁等山蠻,大破之,獲萬餘口,遷太子左衛率。坐前伐蠻,枉殺隊主嚴祖,又納面首生口,不以送臺,免官。是時上大舉北討,質白衣與驃騎司馬王方回等率軍出許、洛,安北司馬王玄謨攻滑臺,不拔,質請乘驛代
 將,太祖不許。



 虜侵徐、豫,拓跋燾率大眾數十萬遂向彭城,以質為輔國將軍、假節、置佐,率萬人北救。始至盱眙,燾已過淮,冗從僕射胡崇之領質府司馬,崇之副太子積弩將軍毛熙祚亦受統於質。盱眙城東有高山,質慮虜據之,使崇之、澄之二軍營於山上,質營城南。虜攻崇之、澄之二營,崇之等力戰不敵,眾散,並為虜所殺。虜又攻熙祚,熙祚所領悉北府精兵,幢主李灌率厲將士,殺賊甚多。隊主周胤之、外監楊方生又率射賊,賊垂退,會
 熙祚被創死,軍遂散亂。其日質案兵不敢救,故二營一時覆沒。



 初,仇池之平也,以崇之為龍驤將軍、北秦州刺史,宋百頃,行至濁水,為索虜所克,舉軍敗散;崇之及將佐以下,皆為虜所執,後得叛還,至是又為虜所敗焉。



 熙祚,司州刺史脩之兄子也。崇之、熙祚並贈正員郎;澄之事在祖父燾傳。



 三營既敗,其夕質軍亦奔散,棄輜重器甲,單七百人投盱眙。盱眙太守沈璞完為守戰之備,城內有實力三千,質大喜,因共守。虜初南出,後無資糧,唯
 以百姓為命。及過淮,食平越、石鱉二屯穀,至是抄掠無所,人馬饑困,聞盱眙有積粟,欲以為歸路之資。既破崇之等,一攻城不拔,便引眾南向。城內增脩守備,莫不完嚴。二十八年正月初,燾自廣陵北返,便悉力攻盱眙,就質求酒,質封溲便與之。



 燾怒甚,築長圍,一夜便合,開攻道,趣城東北,運東山土石填之。虜又恐城內水路遁走,乃引大船,欲於君山作浮橋,以絕淮道。城內乘艦逆戰,大破之。明旦,賊更方舫為桁,桁上各嚴兵自衛。城內更
 擊不能禁,遂於軍山立桁,水陸路並斷。



 燾與質書曰:「吾今所遣鬥兵,盡非我國人,城東北是丁零與胡,南是三秦氐、羌。設使丁零死者,正可減常山、趙郡賊;胡死,正減并州賊;氐、羌死,正減關中賊。卿若殺丁零、胡,無不利。」質答書曰:「省示,具悉姦懷。爾自恃四腳,屢犯國疆,諸如此事,不可具說。王玄謨退於東,梁坦散於西,爾謂何以不聞童謠言邪:『虜馬飲江水,佛狸死卯年。』此期未至,以二軍開飲江之徑爾,冥期使然,非復人事。寡人受命相滅,
 期之白登,師行未遠,爾自送死,豈容復令生全,饗有桑乾哉!但爾往攻此城,假令寡人不能殺爾,爾由我而死。爾若有幸,得為亂兵所殺。爾若不幸,則生相剿縛,載以一驢,直送都市。我本不圖全,若天地無靈,力屈於爾,齏之粉之,屠之裂之,如此未足謝本朝。爾識智及眾力,豈能勝苻堅邪!



 頃年展爾陸梁者,是爾未飲江,太歲未卯年故爾。斛蘭昔深入彭城,值少日雨,隻馬不返,爾豈憶邪?即時春雨已降,四方大眾,始就雲集,爾但安意攻城
 莫走。糧食闕乏者告之,當出廩相飴。得所送劍刀,欲令我揮之爾身邪!甚苦,人附反,各自努力,無煩多云。」是時虜中童謠曰:「軺車北來如穿雉,不意虜馬飲江水。虜主北歸石濟死,虜欲渡江天不徙。」故質答引之。燾大怒,乃作鐵床,於其上施鐵鑱,云破城得質,當坐之此上。質又與虜眾書曰:「示詔虜中諸士庶:貍伐見與書如別,等正朔之民,何為力自取如此。大丈夫豈可不知轉禍為福邪!今寫臺格如別書,自思之。」時購斬燾封開國縣侯,食邑
 一萬戶,賜布絹各萬匹。



 虜以鉤車鉤垣樓,城內繫以驅絙,數百人叫喚引之,車不能退。既夜,以木桶盛人,懸出城外,截其鉤獲之。明日,又以衝車攻城,城土堅密,每至,頹落不過數升。虜乃肉薄登城,分番相代,墜而復升,莫有退者,殺傷萬計,虜死者與城平。



 又射殺高梁王。如此三旬,死者過半。燾聞彭城斷其歸路,京邑遣水軍自海入淮,且疾疫死者甚眾。二月二日,乃解圍遁走。上嘉質功,以為使持節、監雍、梁、南北秦四州諸軍事、冠軍將
 軍、寧蠻校尉、雍州刺史,封開國子,食邑五百戶。明年,太祖又北伐,使質率所統見力向潼關,質頓兵近郊,不肯時發,獨遣司馬柳元景屯兵境上,不時進軍。質又顧戀嬖妾,棄營單馬還城,散用臺庫見錢六七百萬,為有司所糾,上不問也。



 元凶弒立,以質為丹陽尹,加征虜將軍。質家遣門生師顗報質,具太祖崩問。



 質疏顗所言,馳告司空義宣,又遣州祭酒從事田穎起銜命報世祖,率眾五千,馳下討逆,自陽口進江陵義宣。質諸子在都邑,聞
 質舉義,並逃亡。劭欲相慰悅,乃下書曰:「臧敦等無因自駭,急便竄逸,迷昧過甚,良可怪歎。質國戚勳臣,忠誠篤亮,方當顯位,贊翼京輦,而子弟波迸,傷其乃懷。可遣宣譬令還,咸復本位。」



 劭尋錄得敦,使大將軍義恭行訓杖三十,厚給賜之。義宣得質報,即日舉兵,馳信報世祖,板進質號征北將軍。質徑赴尋陽,與世祖同下。



 世祖至新亭即位,以質為都督江州諸軍事、車騎將軍、開府儀同三司、江州刺史,加散騎常侍,持節如故。使質率所領自
 白下步上,直至廣莫門,門者不守。薛安都、程天祚等亦自南掖門入,與質同會太極殿,生禽元凶。仍使質留守朝堂,甲仗百人自防。封始興郡公,食邑三千戶。之鎮,舫千餘乘,部伍前後百餘里,六平乘並施龍子幡。



 時世祖自攬威柄,而質以少主遇之,是事專行,多所求欲。及至尋陽,刑政慶賞,不復諮稟朝廷。盆口、鉤圻米,輒散用之,臺符屢加檢詰,質漸猜懼。自謂人才足為一世英傑,始聞國禍,便有異圖,以義宣凡暗,易可制勒,欲外相推奉,
 以成其志。及至江陵,便致拜稱名。質於義宣雖為兄弟,而年大近十歲,義宣驚曰:「君何意拜弟?」質曰:「事中宜然。」時義宣已推崇世祖,故其計不行。質每慮事泄,及至新亭,又拜江夏王義恭,義恭愕然,問質所以。質曰:「天下屯危,禮異常日,前在荊州,亦拜司空。」會義宣有憾於世祖,事在《義宣傳》。質因此密信說誘,陳朝廷得失。又謂:「震主之威,不可持久,主相勢均,事不兩立。今專據閫外,地勝兵強,持疑不決,則後機致禍。」質女為義宣子採妻,謂質
 無復異同,納其說。且義宣腹心將佐蔡超民之徒,咸有富貴之情,願義宣得,欲倚質威名,以成其業,又勸獎義宣。義宣時未受丞相,質子敦為黃門侍郎,奉詔敦勸,道經尋陽,質令敦具更譬說,并言世祖短長,義宣乃意定。馳報豫州刺史魯爽,期孝建元年秋同舉。爽失旨,即便起兵。遣人至京邑報弟瑜,瑜席卷奔叛。瑜弟弘為質府佐,世祖遣報質,質於是執臺使,狼狽舉兵。上表曰:臣聞執藥隨親,非情謬於甘苦;揮斤斬毒,豈忘痛於肌膚。蓋
 以先疑後順,忠焉必往;忍小存大,雖愛必從。丞相臣義宣,育哲台鉉,拊聲聯服,定主勤王之業,勳越乎齊、晉;宗戚懿親之寄,望崇於魯、衛。而惡直醜正,實繁有黨,或染凶作偽,疾害元功;或藉勞挾寵,乘威縱戾。自知愆深釁重,必貽剿戮,乃成紫毀朱,交間忠輔。崇樹私徒,招聚群惡,念舊愛老,無一而存,豈不由凶醜相扇,志肆讒惑。陛下垂慈狎達,不稍惟疑,遂令負扆席圖,蔽於流議,投杼市虎,成於十夫。



 鑒古揆今,實懷危逼,故投袂樊、葉,立節
 於本朝;揮戈晉陽,務清于君側。臣誠庸懦,奉教前朝,雖恧《緇衣》好賢之美,敢希《巷伯》惡惡之情,固已藉風聽而宵憤,撫短策而馳念。況乃宏命爰格,誠係宗社,今奉旨前邁,星言啟行。



 臣本凡瑣,少無遠概,因緣際會,遂班槐鼎,素望既盈,愜心實足,豈應徼功非冀,更希異寵,直以蔓草難除,去惡宜速,是以無顧夷險,慮不及身。仰恃天眷,察亮丹款,茍血誠不照,甘心罪戮。



 伏願陛下先鑒元輔匪躬茂節,末錄庸瑣奉國微誠,不遂淟涊之情,以失
 四海之望,昭戮馬劍,顯肆市朝,則結旌向國,全鋒凱歸,九流凝序,三光並耀,斯則仰說宗廟,俯愜兆民。裁表感慨,涕言無已。



 加魯弘輔國將軍,下戍大雷。馳報義宣,義宣遣諮議參軍劉諶之萬人就弘。世祖遣撫軍將軍柳元景統豫州刺史王玄謨等水軍,屯梁山洲內,兩岸築偃月壘,水陸待之。殿中將軍沈靈賜領百舸,破其前軍於南陵,生禽軍主徐慶安、軍副王僧,質至梁山,亦夾陣兩岸。元景檄書宣告曰:夫革道應運,基命之洪符;嗣業
 興邦,紹歷之明算。自非瑞積神衷,德充民極,孰能升臨寶位,景屬天居。大宋啟期,理高中世,皇根帝葉,永流無疆。夷陂遞來,遘茲凶難,國禍冤深,人綱鬱滅。主上聖略聰武,孝感通神,義變草木,哀動精緯,躬幸南郢,親掃大逆,道援橫流,德模靈造,三光重照,七廟載興。



 臧質少負疵釁,衣冠不齒,昧利誣天,著於觸事。受任述職,不以宣效為心;專方蒞民,惟以侵剝為務。官自賄至,族以貨傾。是以康周陀覆命屠宗,冤達蒼昊;郭伯、西門遺出自皂
 隸,寵越州朝。往蒞東守,鬻爵三千。率卒西討,竊俘取黜。



 荷恩彭、泗,貪虐以逞,坑戮邊氓,忽若草芥,傾渴倉庾,割沒軍糧。作牧漢南,公盜府蓄,矯易文簿,專行欺妄。及受命北伐,憚役緩期,師出有辰,顧懷私愛,匹馬棄眾,宵行獨返,遂復攜嬪擁姬,淫宴軍幕。孔、范之變,顯於逆辭。凡此諸釁,皆彰著於憲簡,振曝於觀聽。



 去歲義舉,雖豫誠款,而淹留西楚,私相崇戴,奉書致命,形於心迹。新亭之捷,大難已夷,凶命假存,懸在晷刻,廣莫之軍,曾無遺矢,
 重關自開,偽眾已潰,質猶復盤桓衢巷,後騎陳師。勞不足甄,定於朝議,而虛張功伐,扇動怨辭,自謂斯舉,勳莫己若。初踐殿守,忘犬馬之情,奔趣帑藏,頓傾天府。山海弘量,苞荒藏疾,錄其一介之心,掩其不逞之釁。遂爵首元等,職班盛級,優榮溢寵,莫與為疇。自恣醜薄,罔知涯涘,幹謁陳聞,曾無紀極,請樂窮太子之英,求器盡官府之選。徐司空匪躬王室,遭罹凶禍,質與之少長,親交兼常,曾無撫孤之仁,惟聞陵侮之酷,尺田寸寶,靡有孑遺。
 及受命南徂,臨路滋甚,逼奪妻嬪,略市金帛,怨動京邑,醜聞都鄙。棄逐舊故,委蔑忠勤,魯尚期、尹周之徒,心腹所倚,泣訴於御筵;袁同、連子敬之疇,爪牙所杖,一逝而不反。雖上旨頻煩,屢求勞牒,質但稱伐在己,不逮僚隸,託咎朝廷,歸罪有司,國士解心,有識莫附。何文敬趨走廝養,天性愚狡,質迷其姦諂,寘懷委仗,遂外擅威刑,內遊房室。質生與釁俱,不可詳究,擢發數罪,曾何足言!



 丞相威重位尊,任居分陜,宗國倚賴,實兼恆情,而不及謙
 沖之塗,弗見逆順之訓,蔽同郤至,理乖范燮。遂乃遠忽世祀,近受欺搆,杖納姦疏,還謀社稷。日者宴安上流,坐觀成敗,示遣疲卒,眾裁三千,戎馬不供,軍糧靡獻。皇朝直以親秩之重,酬寵兼極,近漸別子,禮越常均,茍識無所守,功弗由己,必為義不全,終於敗德。今茲放命,恨心於本,推諸昔歲,迹是誠非矣。且家國夷險,情事異常,豫是臣子,孰不星赴,而玩寇忘哀,曾無奔拽。面蕃十稔,惠政蔑聞,重贓深掠,縱慾已甚,姬妾百房,尼僧千計,敗道
 傷俗,悖亂人神,民怨盈塗,國謗彌歲。又賊劭未禽,凶威猶彊,將毀其私墳,戮其諸子,圖成駭機,垂賴義舉,捷期云速,不日告平,釋怨毒之心,解倒懸之急,論恩敘德,造育為重。援人自助,棄人快讒,怙亂疑功,未聞其比。



 僕以不肖,過蒙榮私,荷佩升越,光絕倫伍。家本北邊,志存慷慨,常甘投生,以殉艱棘,惟恩思難,激氣衝襟,故以眺三湘而永慨,望九江而遐憤。若使身死國康,誓在殞命,況仰稟聖略,俯鞠義徒,萬全之形,愚夫所照。夫薛竟陵控
 率突騎,陸道步馳。檀右衛、申右率、垣游擊整勒銳師,飛輪構路。王豫州方舟繕甲,久已前驅。僕訓卒利兵,凌波電進。沈鎮軍、蕭安南接舳連旌,首尾風合。驃騎竟陵王懿親令譽,問望攸歸,大司馬江夏王道略明遠,徽猷茂世,並旄鉞臨塗,雲驅齊引。



 群兵競邁,秘駕徐啟。八鑾搖響,五牛舒旆。千乘雷動,萬舳雲回。騰威發號,星流漢轉。以上臨下,易於轉圓。加以三謀協從,七緯告慶,幽顯同心,昭然易睹。



 諸君或世荷恩幸,或身聞教義,當知君臣
 大節,誓不可犯,冠屨至誨,難用倒設。履安奉順,聲泰事全,孰與附逆居危,身害名醜,慈親垂白受戮,弱子嬰孩就誅。所以有詔遲回,未震雷霆者,正為諸君身拘寇手,或懷乃心。吉凶由人,無謂為遠,今而不變,後悔何及。授檄之日,心馳賊庭。



 義宣亦相次係至。江夏王與義宣書曰:「昔桓玄借兵於仲堪,有似今日。」義宣由此與質相疑。質進計曰:「今以萬人取南州,則梁山中絕,萬人綴玄謨,必不敢動。質浮舟外江,直向石頭,此上略也。」義宣將從
 之,腹心劉諶之曰:「質求前馳,此志難測。不如盡銳攻梁山,事剋然後長驅,萬安之計也。」質遣將尹周之攻胡子反、柳叔政於西壘,時子反渡東岸就玄謨計事,聞賊至,馳歸。周之攻壘甚急,劉季之水軍殊死戰,賊勢盛,求救於玄謨。玄謨不遣,崔勳之固爭,乃遣勳之救之。比至,城已陷,勛之戰死,季之收眾而退。子反、叔政奔還東岸,玄謨斬子反軍副李文仲。



 質欲仍攻東城,義宣黨顏樂之說義宣曰:「質若復拔東城,則大功盡歸之矣。



 宜遣麾下
 自行。」義宣遣劉諶之就質,陳軍城南。玄謨留羸弱守城,悉精兵出戰,薛安都騎軍前出,垣護之督諸將繼之。戰良久,賊陣小拔,騎得入。劉季之、宗越又陷其西北,眾軍乘之,乃大潰。因風放火,船艦悉見焚燒,延及西岸。質求義宣欲一計事,密已出走矣。質不知所為,亦走,眾悉降散。質至尋陽,焚燒府舍,載妓妾西奔。使所寵何文敬領兵居前,至西陽。西陽太守魯方平,質之黨也,至是懷貳,誑文敬曰:「傳詔宣敕,唯捕元惡一人,餘並無所問。」文敬
 棄眾而走。



 質先以妹夫羊沖為武昌郡,質往投之。既至,沖已為郡丞胡庇之所殺。無所歸,乃入南湖逃竄,無食,摘蓮啖之。追兵至,窘急,以荷覆頭,自沈於水,出鼻。軍主鄭俱兒望見,射之中心,兵刃亂至,腸胃纏縈水草,隊主裘應斬質首,傳京都,時年五十五。錄尚書江夏王臣義恭、左僕射臣宏等奏曰:「臧質底棄下才,而藉遇深重,窮愚悖常,構煽凶逆,變至滔天,志圖泯夏,違恩叛德,罪過恒科。梟首之憲,有國通典,懲戾思永,去惡宜深。臣等參
 議,須辜日限意,使依漢王莽事例,漆其頭首,藏于武庫。庶為鑒戒,昭示將來。」詔可。



 質初下,義宣以質子敦為征虜將軍、雍州刺史。質留子敞為監軍,將敦自隨,至是並為武昌郡所執送。敦官至黃門郎。敦弟敷,司徒屬。敷弟敞,太子洗馬。敞弟斁,敦子仲璋,質之二子二孫未有名,同誅。



 質之起兵也,豫章太守任薈之、臨川內史劉懷之、鄱陽太守杜仲儒並為盡力,發遣郡丁,并送糧運,伏誅。任薈之,字處茂,樂安人也。歷世祖、南平王鑠撫軍右軍
 司馬、長史行事。太祖稱之曰:「望雖不足,才能有餘。」杜仲儒,杜驥兄子也。豫章望蔡子相孫沖之起義拒質,質遣將郭會膚、史山夫討之,為沖之所破。世祖發詔,以為尚書都官曹郎中。沖之,太原中都人,晉秘書監盛曾孫也。官至右軍將軍,巴東太守。後事在《劉琬傳》。沈靈賜以破質前軍於南陵功,封南平縣男,食邑三百戶。贈崔勳之通直郎。大司馬參軍劉天賜亦梁山戰亡,追贈給事中。



 魯爽,小名女生,扶風郿人也。祖宗之,字彥仁,晉孝武太
 元末,自鄉里出襄陽,歷官至南郡太守。義熙元年起義,襲偽雍州刺史桓蔚,進向江陵。以功為輔國將軍、雍州刺史,封霄城縣侯,食邑千五百戶。桓謙、荀林逼江陵,宗之率眾馳赴,事在《臨川烈武王道規傳》。進號平北將軍。高祖討劉毅,與宗之同會江陵,進號鎮北將軍,封南陽郡公,食邑二千五百戶。子軌,一名象齒,爽之父也。便弓馬,筋力絕人,為竟陵太守。宗之自以非高祖舊隸,屢建大功,有自疑之心。會司馬休之見討,猜懼,遂與休之北
 奔。善於撫御,士民皆為盡力,衛送出境,盡室入羌,頃之病卒。高祖定長安,軌為寧南將軍、荊州刺史、襄陽公,鎮長社。世祖鎮襄陽,軌遣親人程整奉書,規欲歸順,自拔致誠,以昔殺劉康祖、徐湛之父,故不歸。太祖累遣招納,許以為司州刺史。



 爽少有武藝,虜主拓跋燾知之,常置左右。元嘉二十六年,軌死,爽為寧南將軍、荊州刺史、襄陽公,鎮長社。幼染殊俗,無復華風。粗中使酒,數有過失,燾將誅之。爽有七弟秀,小字天念,頗有意略,才力過爽。
 燾以充宿衛,甚知待之。



 偽高梁王阿叔泥為芮芮所圍甚急,使秀往救,燾自率大眾繼其後。燾未及至,秀已擊破之,拔阿叔泥而反。燾壯其功,以為中書郎,封廣陵侯。或告燾,鄴民欲據城反,復遣檢察,并燒石虎殘宮殿。秀常乘驛往反,是時病還遲,為燾所詰讓,秀復恐懼。燾尋南寇,因從渡河。



 先是,程天祚為虜所沒,燾引置左右,與秀囗寬,勸令歸降,秀納之。天祚,廣平人,為殿中將軍,有武力。元嘉二十七年,助戍彭城,會世祖遣府劉泰之輕
 軍襲虜於汝陽,天祚督戰,戰敗被創,為虜所獲。天祚妙善針術,燾深加愛賞,或與同輿,常不離於側,封為南安公。燾北還蕃,天祚因其沈醉,偽若受使督切後軍者,所至輕罰。天祚為燾所愛,群虜並畏之,莫敢問,因得逃歸,後為山陽太守。太宗初,與四方同反,事在《薛安都傳》。



 燾始南行,遣爽隨永昌王庫仁真向燾陽,與弟瑜共破劉祖於尉武,仍至瓜步,始得與秀定歸南之謀。燾還至湖陸,爽等請曰:「奴與南有仇,每兵來,常慮禍及墳墓,乞共
 迎喪,還葬國都。」虜群下於其主稱奴,猶中國稱臣也。燾許之。長社戍虜有六七百人,爽譎之曰:「南更有軍,可遣三百騎往界上參聽。」騎去,爽率腹心夜擊餘虜,盡殺之,馳入虎牢。



 爽唯第三弟在北,餘家屬悉自隨,率部曲及願從合千餘家奔汝南。遣秀從許昌還壽陽,奉辭於南平王鑠曰:「爽、秀得罪晉朝,負釁三世,生長絕域,遠身胡虜,兄弟闔門,淪點偽授,殞命不可,還國無因。近係南雲,傾屬東日,蓋猶痿人思步,盲者願明。嵩、霍咫尺,江、河匪
 遠,夷庚壅塞,隔同天地,痛心疾首,書慨宵悲。



 虜主猖狂,豺豕其志,虐遍華、戎,怨結幽顯。自盱眙旋軍,亡殪過半,昏酣沈湎,恣性肆身。爽、秀等因民之憤,藉將旅之願,齊契義奮,梟馘醜徒,馮恃皇威,肅清逋穢,牢、洛諸城,指期克定。規以涓塵,微雪夙負,方當束骸北闕,待戮司寇,懦節未申,伏心邊表。明大王殿下以睿茂居蕃,文武兼姿,遠邇欽傾,承風聞德,願垂援拯,以慰虔望。老弱百口,先遣歸庇。逼逼丹心,仰希懷遠。謹遣同義潁川聶元初奉
 詞陳聞。」鑠馳驛以聞,上大說,下詔曰:「偽寧南將軍魯爽、中書郎魯秀,志乾列到,忠誠久著,撫茲福先,闔門效款,招集義銳,梟剪獯醜,肅定邊城,獻馘象魏。雖宣孟之去翟歸晉,頹當之出胡入漢,方之此日,曾何足云。朕實嘉之,宜即授任,逞其忠略。爽可督司州、陳留、東郡、濟陰、濮陽五郡諸軍事、征虜將軍、司州刺史。秀可輔國將軍、滎陽、潁川二郡太守。其諸子弟及同契士庶,委征虜府以時申言,詳加酬敘。」爽至汝南,加督豫州之義陽、宋安二
 郡軍事,領義陽內史,將軍、刺史如故。秀參右將軍南平王鑠軍事、汝陰內史,將軍如故。餘弟侄並授官爵,賞賜資給甚厚。爽北鎮義陽。北來部曲凡六千八百八十三人,是歲二十八年也。虜毀其墳墓。



 明年四月入朝,時燾已死,上更謀經略。五月,遣爽、秀、程天祚等率步騎并荊州軍甲士四萬,出許、洛。八月,虜長社戍主永平公禿髮幡乃同棄城走。進向大索戍,戍主偽豫州刺史跋僕蘭曰:「爽勇而無防,我今出城,必輕來據之,設伏檀山,必可
 禽也。」爽果夜進,秀諫不止,馳往繼之。比曉,虜騎夾發,賴秀縱兵力戰,虜乃退還虎牢。爽因進攻之,本期舟師入河,斷其水門。王玄謨攻確磝不拔,敗退,水軍不至,爽亦收眾南還。轉鬥數百里,至曲彊,虜候其饑疲,盡銳來攻,爽身自奮擊,虜乃退走。



 三十年,元凶弒逆,南譙王義宣起兵入討,爽即受命,率部曲至襄陽,與雍州刺史臧質俱詣江陵。義宣進爽號平北將軍,領巴陵太守,度支校尉,本官如故。留爽停江陵,事平,以爽為使持節、督豫、司、
 雍、秦、并五州諸軍事、左將軍、豫州刺史。爽至壽陽,便曲意賓客,爵命士人,蓄仗聚馬,如寇將至。元凶之為逆也,秀在京師,謂秀曰:「我為卿誅徐湛之矣,方相委任。」以為右軍將軍,配精兵五千,使攻新亭壘。將戰,秀命打退軍鼓,因此歸順。世祖即位,以為左軍將軍,出督司州豫州之新蔡、汝南、汝陽、潁川、義陽、弋陽六郡諸軍事、輔國將軍、司州刺史,領汝南太守。



 爽與義宣及質相結已久,義宣亦欲資其勇力,情契甚至。孝建元年二月,義宣報爽,
 秋當同舉。爽狂酒乖謬,即日便起兵,馳信報弟瑜,將家奔叛,皆得西歸。



 爽使其眾載黃標,稱建平元年,竊造法服,登壇自號。疑長史韋處穆、中兵參軍楊元駒、治中庾騰之不與己同,殺之。義宣、質聞爽已處分,便狼狽反,進爽號征北將軍。爽於是送所造輿服詣江陵,版義宣及臧質等並起。征北府戶曹版文曰:「丞相劉補天子,名義宣,車騎臧今補丞相,名質,平西朱今補車騎,名修之,皆版到奉行。」義宣駭愕。爽所送法物,並留竟陵縣不聽進。



 爽直出歷陽,自采石濟軍,與質水陸俱下。爽遣弟瑜守蒙蘢,歷陽太守張幼緒請擊瑜,世祖配以兵力。遣左軍將薛安都步騎為前驅,別遣水軍入淵,分路並會。



 安都進次大峴,爽已立營。世祖以賊彊壘固,未可輕拔,使量宜進止。幼緒便引軍退還,下獄。更遣驍騎將軍垣護之代幼緒據歷陽。鎮軍將軍沈慶之系安都進軍,與爽相遇於小峴。爽親自前,將戰,而飲酒過醉,安都刺爽倒馬,左右范雙斬首,傳送京都。瑜亦為部下所斬送,進平壽
 陽,子弟並伏誅。



 義宣初舉兵,召秀加節,進號征虜將軍,當繼諶之俱下。雍州刺史朱脩之起兵奉順,更遣秀擊脩之。王玄謨聞之,喜曰:「魯秀不來,臧質易與耳。」秀至襄陽,大敗而反。會益州刺史劉秀之遣軍襲江陵,秀擊破之。義宣還江陵,秀與共北走,眾叛且盡。秀向城,上射之,中箭,赴水死,軍人宗敬叔、康僧念斬首,傳京邑。



 贈韋處穆、楊元駒給事中,庾騰之員外散騎侍郎。爽初南歸,秀以爽武人,不閑吏職,白太祖請處穆為長史以輔爽,太
 祖以補司馬,後轉長史云。



 沈攸之,字仲達,吳興武康人,司空慶之從父兄子也。父叔仁,為衡陽王義季征西長史,兼行參軍,領隊,又隨義季鎮彭城,度征北府。攸之少孤貧,元嘉二十七年,索虜南寇,發三吳民丁,攸之亦被發。既至京都,詣領軍將軍劉遵考,求補白丁隊主。遵考謂之曰:「君形陋,不堪隊主。」因隨慶之征討。二十九年,征西陽蠻,始補隊主。巴口建義,南中郎府板長史,兼行參軍。新亭之戰,身被重創,事寧,
 為太尉行參軍,封平洛縣五等侯。隨府轉大司馬行參軍。晉世京邑二岸,揚州舊置都部從事,分掌二縣非違,永初以後罷省,孝建三年,復置其職。攸之掌北岸,會稽孔璨掌南岸,後又罷。攸之遷員外散騎侍郎。又隨慶之徵廣陵,屢有功,被箭破骨。世祖以其善戰,配以仇池步槊。事平,當加厚賞,為慶之所抑,遷太子旅賁中郎,攸之甚恨之。七年,遭母憂,葬畢,起為龍驤將軍、武康令。



 前廢帝景和元年,除豫章王子尚車騎中兵參軍,直閣,與宗
 越、譚金等並為廢帝所寵,誅戮群公,攸之等皆為之用命。封東興縣侯,食邑五百戶。尋遷右軍將軍,增邑百戶。太宗即位,以例削封。宗越、譚金等謀反,攸之復召入直閣,除東海太守。未拜,會四方反叛,南賊已次近道,以攸之為寧朔將軍、尋陽太守,率軍據虎檻。時王玄謨為大統,未發。前鋒有五軍在虎檻,五軍後又絡驛繼至,每夜各立姓號,不相稟受。攸之謂軍吏曰:「今眾軍姓號不同,若有耕夫漁父,夜相呵叱,便致駭亂,取敗之道也。」乃就
 一軍請號,眾咸從之。殷孝祖為前鋒都督,而大失人情,攸之內撫將士,外諧群帥,眾並倚賴之。時南賊前鋒鐘沖之、薛常寶等屯據赭圻,殷孝祖率眾軍攻之,為流矢所中死,軍主范潛率五百人投賊,人情震駭,並謂攸之宜代孝祖為統。時建安王休仁屯虎檻,總統眾軍,聞孝祖死,遣寧朔將軍江方興、龍驤將軍劉靈遺各率三千人赴赭圻。攸之以為孝祖既死,賊有乘勝之心,明日若不更攻,則示之以弱。方興名位相亞,必不為己下,軍政
 不一,致敗之由。乃率諸軍主詣方興,謂之曰:「四方並反,國家所保,無復百里之地。唯有殷孝祖為朝廷所委賴,鋒鏑裁交,輿尸而反,文武喪氣,朝野危心。事之濟否,唯在明旦一戰,戰若不捷,則大事去矣。詰朝之事,諸人咸謂吾應統之,自卜懦薄,幹略不辦及卿,今輒相推為統。但當相與戮力爾。」方興甚悅。攸之既出,諸軍主並尤之,攸之曰:「卿忘廉、藺、寇、賈之事邪?吾本以濟國活家,豈計彼此之升降。且我能下彼,彼必不能下我,共濟艱難,豈可
 自厝同異!」明旦進戰,自寅訖午,大破賊於赭圻城外,追奔至姥山,分遣水軍乘勢進討;又破其水軍,拔胡白二城。



 尋假攸之節,進號輔國將軍,代孝祖督前鋒諸軍事。薛常寶在赭圻食盡,南賊大帥劉胡屯濃湖,以囊盛米繫流查及船腹,陽覆船,順風流下,以餉赭圻。攸之疑其有異,遣人取船及流查,大得囊米。攸之從子懷寶,為賊將帥,在赭圻,遣親人楊公贊齎密書招誘攸之,攸之斬公贊,封懷寶書呈太宗。尋克赭圻,遷使持節、督雍、梁、南
 北秦四州郢州之竟陵諸軍事、冠軍將軍、領寧蠻校尉、雍州刺史。



 袁顗復率大眾來入鵲尾,相持既久,軍主張興世越鵲尾上據錢溪,劉胡自攻之。



 攸之率諸將攻濃湖,顗遣人傳唱錢溪已平,眾並懼。攸之曰:「不然。若錢溪實敗,萬人中應有逃亡得還者。必是彼戰失利,唱空聲以惑眾耳。」勒軍中不得妄動。錢溪信尋至,果大破賊。攸之悉以錢溪所送胡軍耳鼻示之,顗駭懼,急追胡還。攸之諸軍悉力進攻,多所斬獲,日暮引歸。鵲尾食盡,遣千人
 往南陵迎米,為臺軍所破,燒其資實,胡於是棄眾而奔,顗亦叛走。赭圻、濃湖之平也,賊軍委棄資財,珍貨殷積,諸軍各競收斂,以彊弱為少多。唯攸之、張興世約勒所部,不犯秋毫,諸將以此多之。攸之進平尋陽,徙臨郢州諸軍事、前將軍、郢州刺史,持節如故。不拜,遷中領軍,封貞陽縣公,食邑二千戶。



 時四方皆已平定,徐州刺史薛安都據彭城請降,上雖相酬許,而辭旨簡略。攸之前將軍,置佐吏,假節,與鎮軍將軍張永以重兵征安都。安都
 懼,要引索虜;索虜引大眾援之。攸之等米船在呂梁,又遣軍主王穆之上民口;穆之為虜攻覆米船,又破運車於武原,攸之等引退,為虜所乘,又值寒雪,士眾墮指十二三。留長水校尉王玄載守下邳,積射將軍沈韶守宿豫,睢陵、淮陽亦置戍,攸之還淮陰。免官,以公領職。復求進討,上不聽,入朝面陳,又不許,復歸淮陰。三年六月,自率運送米下邳,並鑿四周深塹,遣龍驤將軍垣護之領民口還淮陰。



 時軍主陳顯達當領千兵守下邳,攸之留
 待顯達至,虜遣清泗間人詐告攸之云:「安都欲降,求軍迎接。」攸之副吳喜納其說,咸謂宜遣千人參之,既而來者轉多,喜所執彌固。攸之乃集來者告之,語曰:「薛徐州早宜還朝,今能爾,深副本望。



 但遣子弟一人來,便當遣大軍相接。君諸人既有志心,若能與薛子弟俱來者,皆即假君以本鄉縣,唯意所欲;如其不爾,無為空勞往還。」自此一去不反。



 其年秋,太宗復令攸之進圍彭城。攸之以清泗既乾,糧運不繼,固執以為非宜,往反者七。上大
 怒,詔攸之曰:「卿春中求伐彭城,吾恐軍士疲勞,且去冬奔散,人心未宜復用,不許卿所啟。今便不肯為吾行邪?卿若不行,便可使吳喜獨去。」



 攸之懼,乃奉旨進軍。行至遲墟,上悔,追軍令反。攸之還至下邳,而陳顯達於睢口為虜所破,龍驤將軍姜產之、司徒參軍高遵世戰沒。虜追攸之甚急,因交戰,被槊創,會暮,引軍入顯達壘,夕眾散,八月十八日也。攸之棄眾南奔。



 初,吳興丘幼弼、丘隆先、沈誕、沈榮守、吳陸道量,並以文記之才隨攸之,及張
 永北討,永一奔,攸之再敗,幼弼等並皆陷沒。攸之之還淮陰,以為持節、假冠軍將軍、行南兗州刺史。追贈姜產之左軍將軍,高遵世屯騎校尉。



 四年,徵攸之為吳興太守,辭不拜。乃除左衛將軍,領太子中庶子。五年,出為持節、監郢州諸軍、郢州刺史。為政刻暴,或鞭士大夫,上佐以下有忤意,輒面加詈辱。將吏一人亡叛,同籍符伍充代者十餘人。而曉達吏事,自彊不息,士民畏憚,人莫敢欺。聞有虎,輒自圍捕,往無不得,一日或得兩三。若逼暮
 不獲禽,則宿昔圍守,須曉自出。賦斂嚴苦,徵發無度,繕治船舸,營造器甲。自至夏口,便有異圖。六年,進監豫州之西陽、司州之義陽二郡軍事,進號鎮軍將軍。



 泰豫元年,太宗崩,攸之與蔡興宗在外蕃,同豫顧命,進號安西將軍,加散騎常侍,給鼓吹一部。未拜,會巴西民李承明反,執太守張澹,蜀土騷擾。時荊州刺史建平王景素被徵,新除荊州刺史蔡興宗未之鎮,乃遣攸之權行荊州事。攸之既至,會承明已平,乃以攸之都督荊、湘、雍、益、梁、
 寧、南北秦八州諸軍事、鎮西將軍、荊州刺史,持節、常侍如故。至荊州,政治如在夏口,營造舟甲,常如敵至。



 時幼主在位,群公當朝,攸之漸懷不臣之跡,朝廷制度,無所遵奉。



 江州刺史桂陽王休範密有異志,以微旨動攸之,使道士陳公昭作天公書一函,題云「沈丞相」,送付攸之門者;攸之不開書,推得公昭,送之朝廷。後廢帝元徽二年,休範舉兵襲京邑,攸之謂僚佐曰:「桂陽今反朝廷,必聲云與攸之同。若不顛沛勤王,必增朝野之惑。」於是遣
 軍主孫同、沈懷奧興軍馳下,受郢州刺史晉熙王燮節度。同等始過夏口,會休範平,還。進攸之號征西大將軍、開府儀同三司,固讓開府。



 攸之自擅閫外,朝廷疑憚之,累欲徵入,慮不受命,乃止。群公稱皇太后令,遣中使問攸之曰:「久勞于外,宜還京輦,然任寄之重,換代殊為未易,還止之宜,一以相委。」欲以觀察其意。攸之答曰:「荷國重恩,名器至此,自惟凡陋,本無廊廟之姿。至如戍防一蕃,撲討蠻、蜒,可彊充斯任。雖自上如此,豈敢厝心去留,
 歸還之事,伏聽朝旨。」朝廷逾懾憚,徵議遂息。四年,建平王景素據京城反,攸之復應朝廷;景素尋平。



 初元嘉中,巴東、建平二郡,軍府富實,與江夏、竟陵、武陵並為名郡。世祖於江夏置郢州,郡罷軍府,竟陵、武陵亦並殘壞,巴東、建平為峽中蠻所破,至是民人流散,存者無幾。其年春,攸之遣軍入峽討蠻帥田五郡等。及景素反,攸之急追峽中軍,巴東太守劉攘兵、建平太守劉道欣並疑攸之自有異志,阻兵斷峽,不聽軍下。時攘兵元子天賜為
 荊州西曹,攸之遣天賜譬說之,令其解甲,一無所問。攘兵見天賜,知景素實反,乃釋甲謝愆,攸之待之如故,後以攘兵為府司馬。劉道欣堅守建平,攘兵譬說不回,乃與伐蠻軍攻之,破建平,斬道欣。



 臺直閣高道慶家在江陵,攸之初至州,道慶時在家,牒其親戚十餘人,求州從事西曹,攸之為用三人。道慶大怒,自入州取教,毀之而去。及還都,不詣攸之別。



 道慶至都,云:「攸之聚眾繕甲,姦逆不久。」楊運長等常相疑畏,乃與道慶密遣刺客,齎廢
 帝手詔,以金餅賜攸之州府佐吏,進其階級。時有象三頭至江陵城北數里,攸之自出格殺之,忽有流矢集攸之馬障泥,其後刺客事發。



 廢帝既殞,順帝即位,進攸之號東騎大將軍、開府儀同三司,加班劍二十人。



 遣攸之長子司徒左長史元琰齎廢帝刳剒之具以示攸之。元琰既至江陵,攸之便有異志,腹心議有不同,故其事不果。其年十一月,乃發兵反叛。攸之素蓄士馬,資用豐積,至是戰士十萬,鐵馬二千。遣使要雍州刺史張敬兒、梁
 州刺史范伯年、司州刺史姚道和、湘州行事庾佩玉、巴陵內史王文和等。敬兒、文和斬其使,馳表以聞;伯年、道和、佩玉懷兩端,密相應和。



 十二月十二日,攸之遣其輔國將軍、中兵參軍、督前鋒軍事孫同,率寧朔將軍中兵參軍武寶、龍驤將軍騎兵參軍朱君拔、寧朔將軍沈慧真、龍驤將軍中兵參軍王道起;又遣司馬、冠軍將軍劉攘兵,率寧朔將軍外兵參軍公孫方平、龍驤將軍騎兵參軍朱靈寶、龍驤將軍騎兵參軍沈僧敬、龍驤將軍高
 茂;又遣輔國將軍中兵參軍王靈秀、輔國將軍中兵參軍丁珍東,率寧朔將軍中兵參軍王珍之、寧朔將軍外兵參軍楊景穆,相繼俱下。攸之自率輔國將軍錄事參軍兼司馬武茂宗、輔國將軍中兵參軍沈韶、寧朔將軍中兵參軍皇甫賢、寧朔將軍中兵參軍胡欽之、龍驤將軍中兵參軍東門道順,閏十二月四日至夏口。攸之將發江陵,使沙門釋僧桀筮之,曰:「不至京邑,當自郢州回還。」意甚不悅。初,江津有雲氣,狀如塵霧,從西北來,正蓋軍
 上。至沌口,云:「當問訊安西,暫泊黃金浦。」既登岸,郢城出軍擊之。攸之聞齊王世子據盆口,震懾不敢下,因攻郢城。時齊王輔政,遣眾軍西討。尚書符征西府曰:尊冠賤屨,君臣之位,奉順忌逆,成敗斯兆,未有憑陵我郊圻,侵軼我河縣,而不焚師殪甲,靡旗亂轍者也。沈攸之少長庸賤,擢自閻伍,邀百戰之運,乘一捷之功,鐫山裂地,腰金拖紫,窮貴於國,極富於家。擁旄蕃伯,便無北面之禮;受督志屏,即有專征之釁。橘柚不薦,璆罝罕入,箕賦深
 斂,毒被南郢,枉繩矯墨,害著西荊,饕餮其心,溪壑其性,從始至終,沿壯得老。今遂驅迫妖黨,繕集尪卒,結釁外城,送死中甸,是而可忍,孰不要懷!



 今遣新除使持節督郢州之義陽諸軍事平西將軍郢州刺史聞喜縣開國侯黃回、員外散騎常侍冠軍驍騎將軍南臨淮太守重安縣開國子軍主王敬則、輔國將軍屯騎校尉長壽縣開國男王宜與、輔國將軍南高平太守軍主陳承叔、輔國將軍左軍將軍南濮陽太守葛陽縣開國男軍主彭
 文之、龍驤將軍驃騎行參軍軍主召宰,精甲二萬,前鋒雲騰。又遣散騎常侍領游擊將軍湘南縣開國男新除使持節督湘州諸軍事征虜將軍湘州刺史軍主呂安國、屯騎校尉寧朔將軍崔慧景、輔國將軍軍主任候伯、輔國將軍驍騎將軍軍主蕭順之、輔國將軍游擊將軍軍主垣崇祖、寧朔將軍虎賁中郎將軍主尹略、屯騎校尉南城令曹虎頭,舳艫二萬,駱驛繼邁。又遣輔國將軍後軍將軍右軍中兵參軍事軍主茍元賓、寧朔將軍撫
 軍中兵參軍事軍主郭文孝、龍驤將軍撫軍中兵參軍事軍主程隱雋,輕艓一萬,截其津要。新除持節督廣交越寧湘州之廣興諸軍事領平越中郎將征虜將軍廣州刺史統馬軍主沌陽縣開國子周盤龍、輔國將軍後軍統馬軍主張文憘、龍驤將軍軍主薛道淵、冠軍將軍游擊將軍并州刺史南清河太守太原公軍主王敕勤、龍驤將軍射聲校尉王洪範、龍驤將軍冗從僕射軍主成置等,鐵馬五千,龍驤後陳。凡此諸帥,莫不勇力動天,
 勁志駕日,接衝拔距,鷹瞵鶚視,顧盼則前後風生,喑嗚則左右電起,以此攻城,何城不克,以此赴敵,何陳能堅。然後鑾戎薄臨,龍虎百萬,六軍齊軌,五輅舒旆,丹檻發照,素甲生波,樓煩白羽,投鞍成岳,漁陽墨騎,浴鐵為群,芝艾同焚,悔將何及。



 符到之日,幸加三省。其鋒陳營壁之主,驅逼寇手之人,若有投命軍門,一無所問。或能因罪立績,終不爾欺,斬裾射玦,唯功是與。能斬送攸之首,封三千戶縣公,賜布絹各五千匹。信如河海,皎然無貳。
 飛火軍攝文書,千里驛行。齊王出頓新亭,馳檄數攸之罪惡,曰:夫彎弓射天,未見能至;揮戈擊地,多力安施。何則?逆順之勢定殊,禍福之驗易原也。是以違乎天者,鬼神不能使其成;會乎人者,聖哲不能令其毀。故劉濞賴七國連兵之勢,隗囂恃跨河據隴之資,毋丘儉伐其踰海越島之功,諸葛誕矜其待士愛民之德,彼四子者,皆當世雄傑,以犯順取禍,覆窟傾巢,為豎子笑。況乎行陳凡才,斗筲小器,而懷問鼎之志,敢構無君之逆哉!



 逆賊
 沈攸之,出自萊畝,寂寥累世,故司空沈公以從父宗廕,愛之若子,卵翼吹噓,得升官秩。廢帝昏悖,猜畏柱臣,攸之貪競乘機,凶忍趨利,躬行反噬,請銜誅旨。又攸之與譚金、童太壹等並受寵任,朝為牙爪,同功共體,世號三侯,當時親暱,情過管、鮑。遭仰革運,凶黨懼戮,攸之狡猾用數,圖全賣禍,既殺從父,又害良朋。雖呂布販君,酈寄賣友,方之斯人,未足為酷。此其不信不義,言詐翻覆,諸夏之所未有,夷狄之所不為也。泰始開闢,網漏吞舟,略其
 凶險,取其搏噬,故得階亂獲全,因禍保福。攸之空淺,躁而無謀,濃湖崩挫,本非己力;及北伐彭泗,望賊宵奔;重討下邳,一鼓而遁;再鄙王師,又應肆法。先帝英聖,量深河海,宥其回谿之敗,冀收曲崤之捷,故得推遷幸會,頓升崇顯,內端戎禁,外臨方牧。



 聖靈鼎湖,遠頒顧命,託寄崇深,義感金石。而攸之始奉國諱,喜見于容,普天同哀,己以為慶。此其樂禍幸災,大逆之罪一也。



 又攸之累登蕃兵,自郢遷荊,晉熙殿下以皇弟代鎮,地尊望重,攸之
 肆情陵侮,斷割候迎,料擇士馬,簡算器甲,精器銳士,並取自隨,郢城所留,十不遺一,專擅略虜,罔顧國典。此其苞藏禍志,不恭不虔,大逆之罪二也。



 又攸之踐荊以來,恒用姦數,既欲發兵,宜有因假,遂乃蹙迫群蠻,騷擾山谷,揚聲討伐,盡戶發上,蟻聚郭邑,伺國盛衰,從來積年,永不解甲。遂使四野百縣,路無男人;耕田載租,皆驅女弱。自古酷虐,未聞有此。其侮蔑朝廷,大逆之罪三也。



 去昔桂陽奇兵囗起,京師內DS,宗廟阽危。攸之任居上流,
 兵彊地廣,救援顛沛,實宜悉力。國家倒懸,方思身慮,威遣弱卒三千,並皆羸老,使就郢州,稟受節度,欲令判否之日,委罪晉熙。何其平日輈張,實輕周、邵,爾時恭謹,虛重皇戚。此其伏慝藏詐,持疑兩端,大逆之罪四也。



 又攸之累據方州,跋扈滋甚,招誘輕狡,往者咸納;羈絆行侶,過境必留。仕子窮困,不得歸其鄉;商人畢命,無由還其土。叛亡入境,輒加擁護;逋逃出界,必遣窮追。此其大逆之罪五也。



 又攸之自任專恣,恃行慘酷,視吏若仇,遇民
 如草。峻太半之賦,暴參夷之刑。



 鞭捶國士,全用虜法;一人逃亡,闔宗補代。毒遍嬰孩,虐加斑白。獄囚恆滿,市血常流。男不得耕,女不得織。奔馳道路,號哭動天。皇朝赦令,初不遵奉,欲殺欲擊,故曠蕩之澤,長隔彼州。此其無君陵上,大逆之罪六也。



 蒼梧狂凶,釁深桀、紂,猜貳外蕃,鴞目西顧。留其長息元琰,以為交質;父子分張,彌積年稔。賴社稷靈長,獨夫遄戮,攸之豫稟心靈,宜同歡幸。遂迷惑顛倒,深相嗟惜。舉言哀桀,揚聲吠堯。此其不辨是
 非,罔識善惡,違情背理,大逆之罪七也。



 廢昏立明,先代盛典,交、廣先到,梁、秦蚤及,而攸之密邇內畿,川塗弗遠,驛書至止,晏若不聞,末遣章表,奄積旬朔。防風後至,夏典所誅,此其大逆之罪八也。



 升明肇歷,恩深澤遠,申其父子之情,矜其骨肉之恩,馳遣元琰,銜使西歸,並加崇授,寵貴重疊。元琰達西,便應反命,攸之得此集聚,蒙誰之恩?不荷盛德,反生仇釁,此其大逆之罪九也。



 攸之以谿壑之性,含梟鴆之腸,直置天壤,已稱醜穢。況乃舉兵
 內侮,逞肆姦回,斯實惡熟罪成之辰,決癰潰疽之日。幕府過荷朝寄,義百常憤,董司元戎,龔行天罰。今皇上聖明,將相仁厚,約法三章,輕刑緩賦,年登歲阜,家給人足,上有惠和之澤,下無樂亂之心。攸之不識天時,妄圖姦逆,舉無名之師,驅怨仇之黨。



 是以朝野審其易取,含識判其成禽。熊羆厲爪,蓄攫裂之心;虎豹摩牙,起吞噬之憤。鼓怒則冰原激電,奮發則霜野奔雷,以此定亂,豈移晷刻。雖復眾徒梗陸,舉郡阻川,何足以抗沸海之濤,當
 燒山之焰。



 彼土士民,罹毒日久,逃竄無路,常所憫然。今復相逼,起接鋒刃,交戰之日,蘭艾難分。土崩倒戈,宜為蚤計,無使一人迷昧,而九族就禍也。弘宥之典,有如皎日。



 攸之盡銳攻郢州,行事柳世隆隨宜距應,屢摧破之。攸之與武陵王贊箋曰:「江陵一總八州,地居形勝,鎮撫之重,宜以上歸。本欲仰移節蓋,改臨荊部,所以未具上聞者,欲待至止,面自咨申。不圖重關擊柝,覲接莫由。若使匡朝之誠,終蔽於聖察,襲遠之舉,近擁於郢都,則無
 以謝烈士之心,何用塞義夫之志,便不犯關陵漢,期一接奉。若夫斬蛟陷石之卒,裂骼卷鐵之將,煙騰飆迅,容或驚動左右,茍不獲已,敢不先布下情。」又曰:「下官位重分陜,富兼金穴,子弟勝衣,爵命已及,親黨辨菽,抽序便加,耳倦絃歌,口厭粱肉,布衣若此,復欲何求?豈不知俯眉茍安,保養餘齒,何為不計百口,甘冒危難。誠感歷朝之遇,欲報之於皇家爾。昧理之徒,謂下官懷無厭之願,既貫誠於白日,不復明心於殿下。若使天必喪道,忠節
 不立,政復闔門碎滅,百死無恨。但高祖王業艱難,太祖劬勞日昃,卜世不盡七百之期,宗社已成他人之有。家國之事,未審於聖心何如?」



 攸之遣中兵參軍公孫方平馬步三千向武昌,太守臧渙棄郡投西陽太守王毓,奔于盆口,方平因據西陽。建寧太守張謨率二守千人攻之,方平破走。攸之攻郢城久不決,眾心離沮。昇明二年正月十九日夜,劉攘兵燒營入降郢城,眾於是離散,不可復制。將曉,攸之斬劉天賜,率大眾過江,至魯山,諸軍
 因此散走。還向江陵,未百餘里,聞城已為雍州刺史張敬兒所據,無所歸,乃與第三子中書侍郎文和至華容界,為封人所斬送。



 攸之初下,留元琰守江陵,張敬兒剋城,元琰逃走。第五子幼和、幼和弟靈和、元琰子法先、懿子囗囗、文和子法徵、幼和子法茂,並為敬兒所禽,伏誅。初,文和尚齊王女義興憲公主,公主早薨,有二女,至是齊王迎還第內。今皇帝即位,聽攸之及諸子喪還葬墓。攸之第二子懿,太子洗馬,先攸之卒。攸之弟登之,新安太守,
 去職在家,為吳興太守沈文季所收斬。登之弟雍之,鄱陽太守,先攸之卒。詔以雍之孫僧照為義興公主後。雍之與攸之異生,諸弟中最和謹,尤見親愛。攸之性儉吝,子弟不得妄用財物,唯恣雍之所須,輒取齋中服飾,分與親舊,以此為常。



 雍之弟榮之,尚書庫部郎,亦先攸之卒。



 攸之晚好讀書,手不釋卷,《史》、《漢》事多所諳憶,常歎曰:「早知窮達有命,恨不十年讀書。」及攻郢城,夜遇風浪,米船沉沒,倉曹參軍崔靈鳳女幼適柳世隆子,攸之正色
 謂曰:「當今軍糧要急,而卿不以在意,將由與城內婚姻邪?」



 靈鳳答曰:「樂廣有言,下官豈以五男易一女。」攸之歡然意解。



 初,攸之招集才力之士,隨郡人雙泰真有幹力,召不肯來。後泰真至江陵賣買,有以告攸之者,攸之因留之,補隊副,厚加料理。泰真無停志,少日叛走,攸之遣二十人被甲追之,逐討甚急。泰真殺數人,餘者不敢近。欲過家將母去,事迫不獲,單身走入蠻;追者既失之,錄其母而去。泰真既失母,乃出自歸,攸之不罪,曰:「此孝子
 也。」賜錢一萬,轉補隊主,其矯情任算皆如此。



 初,攸之賤時,與吳郡孫超之、全景文共乘小船出京都,三人共上引埭,有一人止而相之曰:「君三人皆當至方伯。」攸之曰:「豈有三人俱有此相?」相者曰:「骨法如此,若有不驗,便是相書誤耳。」其後攸之為郢、荊二州,超之廣州,景文豫州刺史。攸之初至郢州,有順流之志。府主簿宗儼之勸攻郢城,功曹臧寅以為:「攻守勢異,非旬日所拔,若不時舉,挫銳損威。今順流長驅,計日可捷,既傾根本,則郢城豈
 能自固。」攸之不從,既敗,諸將帥皆奔散,惟寅曰:「我委質事人,豈可茍免。我之不負公,猶公之不負朝廷也。」乃投水死。寅,字士若,東莞莒人也。



 先是,攸之在郢州,州從事輒與府錄事鞭,攸之免從事官,而更鞭錄事五十。



 謂人曰:「州官鞭府職,誠非體要,由小人凌侮士大夫。」倉曹參軍事邊榮為府錄事所辱,攸之自為榮鞭殺錄事。攸之自江陵下,以榮為留府司馬,守城。張敬兒將至,人或說之使詣敬兒降,榮曰:「受沈公厚恩,共如此大事,一朝緩
 急,便改易本心,不能行也。」城敗,見敬兒,敬兒問曰:「邊公何不早來?」榮曰:「沈公見留守城,而委城求活,所不忍也。本不蘄生,何須見問。」敬兒曰:「死何難得。」



 命斬之,歡笑而去,容無異色。泰山程邕之者,素依隨榮,至是抱持榮曰:「與邊公周遊,不忍見邊公前死,乞見殺。」兵不得行戮,以告敬兒,敬兒曰:「求死甚易,何為不許。」先殺邕之,然後及榮。三軍莫不垂泣,曰:「奈何一日殺二義士。」



 比之臧洪及陳容。榮,金城人也。



 廢帝之殞也,攸之欲起兵,問其知星
 人葛珂之。珂之曰:「自古起兵,皆候太白。太白見則成,伏則敗。昔桂陽以太白伏時舉兵,一戰授首,此近世明驗。今蕭公廢昏立明,政值太白伏時,此與天合也。且太白尋出東方,東方利用兵,西方不利。」故攸之止不反。及後舉兵,珂之又曰:「今歲星守南斗,其國不可伐。」攸之不從。凡同逆丁珍東、孫同、裴茂仲、武、宗儼之並伏誅。攸之表檄文疏,皆儼之詞也。臧渙詣盆城自歸,今皇帝命斬之。餘同惡或為亂軍所殺,或遇赦得原。



 史臣曰:臧質雖貪虐夙樹,問望多闕,奉義治流,本無吞噬之志也。徒欲以幼君弱政,期之於世祖,據有中流,嗣桓、庾之業。既主異穆、哀,臣皆代黨,雖禮秩外厚,而疑防內深,功高位重,終非自安之地,至於陵天犯順,其出於此乎!攸之伺隙西郢,年逾十載,擅命專威,無君已積。及天厭宋道,鼎運將離,不識代德之紀,獨迷樂推之數,公休既覆其族,攸之亦屠厥身。夫以釁亂自終,固異代如
 一也。



\end{pinyinscope}