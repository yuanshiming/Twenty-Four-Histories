\article{卷五十二列傳第十二 庾悅 王誕 謝景仁 弟述 袁湛 弟豹 褚叔度}

\begin{pinyinscope}

 庾悅,字仲豫,潁川𨻳陵人也。曾祖亮,晉太尉。祖羲,吳國內史。父準,西中郎將、豫州刺史。悅少為衛將軍琅邪王
 行參軍、司馬,徙主簿,轉右長史。桓玄輔政,領豫州,以悅為別駕從事史,遷驍騎將軍。玄篡位,徙中書侍郎。高祖定京邑,武陵王遵承制,以悅為寧遠將軍、安遠護軍、武陵內史。以病去職。鎮軍府版咨議參軍,轉車騎從事中郎。劉毅請為撫軍司馬,不就。遷車騎中軍司馬。從征廣固,竭其誠力。



 盧循偪京都,以為督江州豫州之西陽新蔡汝南潁川司州之恆農揚州之松滋六郡諸軍事、建威將軍、江州刺史,從東道出鄱陽。循遣將英糾千餘人斷五畝嶠,悅
 破之,進據豫章,絕循糧援。



 初,毅家在京口,貧約過常,嘗與鄉曲士大夫往東堂共射。時悅為司徒右長史,暫至京,要府州僚佐共出東堂。毅已先至,遣與悅相聞,曰:「身久躓頓,營一游集甚難。君如意人,無處不可為適,豈能以此堂見讓。」悅素豪,徑前,不答毅語。



 眾人並避之,唯毅留射如故。悅廚饌甚盛,不以及毅。毅既不去,悅甚不歡,俄頃亦退。毅又相聞曰:「身今年未得子鵝,豈能以殘炙見惠。」悅又不答。



 盧循平後,毅求都督江州,以江州內地,
 治民為職,不宜置軍府,上表陳之曰:「臣聞天以盈虛為道,治以損益為義。時否而政不革,民凋而事不損,則無以救急病於已危,拯塗炭於將絕。自頃戎車屢駕,干戈溢境,江州以一隅之地,當逆順之衝,力弱民慢,而器運所繼。自桓玄以來,驅蹙殘毀,至乃男不被養,女無對匹,逃亡去就,不避幽深,自非財單力竭,無以至此。若不曲心矜理,有所改移,則靡遺之歎,奄焉必及。臣謬荷增統,傷慨兼懷。夫設官分職,軍國殊用,牧民以息務為大,武
 略以濟事為先。今兼而領之,蓋出於權事,因藉既久,遂為常則。江州在腹心之中,憑接揚、豫籓屏所倚,實為重復。昔胡寇縱逸,朔馬臨江,抗禦之宜,蓋出權計。以溫嶠明達,事由一己,猶覺其弊,論之備悉。今江右區區,戶不盈數十萬,地不踰數千里,而統司鱗次,未獲減息,大而言之,足為國恥。況乃地在無軍,而軍府猶置,文武將佐,資費非一,豈所謂經國大情,揚湯去火者哉。其州郡邊江,民戶遼落,加以郵亭險闊,畏阻風波,轉輸往還,常有
 淹廢;又非所謂因其所利,以濟其弊者也。愚謂宜解軍府,移治豫章,處十郡之中,厲簡惠之政,比及數年,可有生氣。且屬縣凋散,亦有所存,而役調送迎,不得休止,亦謂應隨宜並減,以簡眾費。刺史庾悅,自臨州部,甚有恤民之誠,但綱維不革,自非綱目所理。



 尋陽接蠻,宜有防遏,可即州府千兵,以助郡戍。」於是解悅都督、將軍官,以刺史移鎮豫章。毅以親將趙恢領千兵守尋陽,建威府文武三千悉入毅府,符攝嚴峻,數相挫辱。悅不得志,疽
 發背,到豫章少日卒,進年三十八。追贈征虜將軍。以廣固之功,追封新陽縣五等男。



 王誕,字茂世,琅邪臨沂人,太保弘從兄也。祖恬,中軍將軍。父混,太常。



 誕少有才藻,晉孝武帝崩,從叔尚書令珣為哀策文,久而未就,謂誕曰:「猶少序節物一句。」因出本示誕。誕攬筆便益之,接其秋冬代變後云:「霜繁廣除,風回高殿。」珣嗟歎清拔,因而用之。襲爵雉鄉侯,拜秘書郎,琅邪王文學,中軍功曹。



 隆安四年,會稽王世子元顯開
 後軍府,又以誕補功曹。尋除尚書吏部郎,仍為後軍長史,領廬江太守,加鎮蠻護軍。轉龍驤將軍、琅邪內史,長史如故。誕結事元顯嬖人張法順,故為元顯所寵。元顯納妾,誕為之親迎。隨府轉驃騎長史,將軍、內史如故。元顯討桓玄,欲悉誅桓氏,誕固陳修等與玄志趣不同,由此得免。修,誕甥也。及玄得志,誕將見誅,修為之陳請;又言修等得免之由,乃徙誕廣州。



 盧循據廣州。以誕為其平南府長史,甚賓禮之。誕久客思歸,乃說循曰:「下官流
 遠在此,被蒙殊眷,士感知己,實思報答。本非戎旅,在此無用。素為劉鎮軍所識,情味不淺,若得北歸,必蒙任寄,公私際會,思報厚恩,愈於停此,空移歲月。」循甚然之。時廣州刺史吳隱之亦為循所拘留,誕又曰:「將軍今留吳公,公私非計。孫伯符豈不欲留華子魚,但以一境不容二君耳。」於是誕及隱之並得還。



 除員外散騎常侍,未拜,高祖請為太尉咨議參軍,轉長史。盡心歸奉,日夜不懈,高祖甚委仗之。北伐廣固,領齊郡太守。盧循自蔡洲南
 走,劉毅固求追討,高祖持疑未決,誕密白曰:「公既平廣固,復滅盧循,則功蓋終古,勳無與二,如此大威,豈可餘人分之。毅與公同起布衣,一時相推耳。今既已喪敗,不宜復使立功。」



 高祖從其說。七年,以誕為吳國內史。母憂去職。高祖征劉毅,起為輔國將軍,誕固辭軍號,墨絰從行。時諸葛長民行太尉留府事,心不自安,高祖甚慮之。毅既平,誕求先下,高祖曰:「長民似有自疑心,卿詎宜便去。」誕曰:「長民知我蒙公垂眄,今輕身單下,必當以為無
 虞,乃可以少安其意。」高祖笑曰:「卿勇過賁、育矣。」於是先還。九年,卒,時年三十九。以南北從征,追封作唐縣五等侯。子詡,宋世子舍人,早卒。



 謝景仁,陳郡陽夏人,衛將軍晦從叔父也。名與高祖同諱,故稱字。祖據,太傅安第二弟。父允,宣城內史。景仁幼時與安相及,為安所知。始為前軍行參軍、輔國參軍事。會稽王世子元顯嬖人張法順,權傾一時,內外無不造門者,唯景仁不至。年三十,方為著作佐郎。桓玄誅元顯,見
 景仁,甚知之,謂四坐曰:「司馬庶人父子云何不敗,遂令謝景仁三十方作著作佐郎。」玄為太尉,以補行參軍,府轉大將軍,仍參軍事。玄建楚臺,以補黃門侍郎。及篡位,領驍騎將軍。景仁博聞強識,善敘前言往行,玄每與之言,不倦也。玄出行,殷仲文、卞範之之徒,皆騎馬散從,而使景仁陪輦。



 高祖為桓修撫軍中兵參軍,嘗詣景仁咨事,景仁與語悅之,因留高祖共食。食未辦,而景仁為玄所召。玄性促急,俄頃之間,騎詔續至。高祖屢求去,景仁不
 許,曰:「主上見待,要應有方。我欲與客共食,豈當不得待。」竟安坐飽食,然後應召。高祖甚感之,常謂景仁是太傅安孫。及平京邑,入鎮石頭,景仁與百僚同見高祖,高祖目之曰:「此名公孫也。」謂景仁曰:「承制府須記室參軍,今當相屈。」



 以為大將軍武陵王遵記室參軍,仍為從事中郎,遷司徒左長史。出為高祖鎮軍司馬,領晉陵太守,復為車騎司馬。



 義熙五年,高祖以內難既寧,思弘外略,將伐鮮卑;朝議皆謂不可。劉毅時鎮姑孰,固止高祖,以為:「
 苻堅侵境,謝太傅猶不自行。宰相遠出,傾動根本。」



 景仁獨曰:「公建桓、文之烈,應天人之心,匡復皇祚,芟夷姦逆,雖業高振古,而德刑未孚,宜推亡固存,廣樹威略。鮮卑密邇疆甸,屢犯邊垂,伐罪弔民,於是乎在。平定之後,養銳息徒,然後觀兵洛汭,修復園寢,豈有坐長寇虜,縱敵貽患者哉!」高祖納之。及北伐,大司馬琅邪王,天子母弟,屬當儲副,高祖深以根本為憂,轉景仁為大司馬左司馬,專總府任,右衛將軍,加給事中,又遷吏部尚書。



 時從兄
 混為左僕射,依制不得相臨,高祖啟依僕射王彪之、尚書王劭前例,不解職。



 坐選吏部令史邢安泰為都令史、平原太守,二官共除,安泰以令史職拜謁陵廟,為御史中丞鄭鮮之所糾,白衣領職。八年,遷領軍將軍。十一年,轉右僕射,仍轉左僕射。



 景仁性矜嚴整潔,居宇靜麗,每唾,轉唾左右人衣;事畢,即聽一日浣濯。每欲唾,左右爭來受。高祖雅相重,申以婚姻,廬陵王義真妃,景仁女也。十二年,卒,時年四十七。追贈金紫光祿大夫,加散騎常
 侍。葬日,高祖親臨,哭之甚慟。



 與驃騎將軍道憐書曰:「謝景仁殞逝,悲痛摧割,不能自勝。汝聞問惋愕,亦不可堪。其器體淹中,情寄實重,方欲與之共康時務,一旦至此,痛惜兼深。往矣柰何!



 當復柰何!」



 子恂,鄱陽太守。恂子稚,善吹笙。官至西陽太守。



 景仁弟純,字景懋,初為劉毅豫州別駕。毅鎮江陵,以為衛軍長史、南平相。



 王鎮惡率軍襲毅,已至城下,時毅疾病,佐吏皆入參承。純參承畢,已出,聞兵至,馳還入府。左右引車欲還外解,純叱之曰:「我
 人吏也,逃欲何之!」乃入。及毅兵敗眾散,時已暗夜,司馬毛修之謂純曰:「君但隨僕。」純不從,扶兩人出,火光中為人所殺。純孫沈,太宗泰始初,為巴陵王休若衛軍錄事參軍、山陰令,坐事誅。



 述字景先,少有志行,隨兄純在江陵。純遇害,述奉純喪還都。行至西塞,值暴風,純喪舫流漂,不知所在,述乘小船尋求之。經純妻庾舫過,庾遣人謂述曰:「喪舫存沒,已應有在,風波如此,豈可小船所冒?小郎去必無及,寧可存亡俱盡邪?」述號泣答曰:「若安全
 至岸,當須營理。如其已致意外,述亦無心獨存。」



 因冒浪而進,見純喪幾沒,述號叫呼天,幸而獲免,咸以為精誠所致也。高祖聞而嘉之,及臨豫州,諷中正以述為主簿,甚被知器。景仁愛其第三弟甝而憎述,嘗設饌請高祖,希命甝豫坐,而高祖召述。述知非景仁夙意,又慮高祖命之,請急不從。



 高祖馳遣呼述,須至乃懽。及景仁有疾,述盡心營視,湯藥飲食,必嘗而後進,不解帶、不盥櫛者累旬,景仁深懷感愧。



 轉太尉參軍,從征司馬休之,封吉陽
 縣五等侯。世子征虜參軍,轉主簿,宋臺尚書祠部郎,世子中軍主簿,轉太子中舍人,出補長沙內史,有惠政。元嘉二年,徵拜中書侍郎。明年,出為武陵太守,彭城王義康驃騎長史,領南郡太守。先是,述從兄曜為義康長史,喪官,述代之。太祖與義康書曰:「今以謝述代曜。其才應詳練,著於歷職,故以佐汝。汝始親庶務,而任重事殷,宜寄懷群賢,以盡弼諧之美,想自得之,不俟吾言也。」義康入相,述又為司徒左長史,轉左衛將軍。蒞官清約,私無
 宅舍。義康遇之甚厚。尚書僕射殷景仁、領軍將軍劉湛並與述為異常之交。美風姿,善舉止,湛每謂人曰:「我見謝道兒,未嘗足。」道兒,述小字也。



 雍州刺史張邵以黷貨下廷尉,將致大辟,述上表陳邵先朝舊勳,宜蒙優貸,太祖手詔酬納焉。述語子綜曰:「主上矜邵夙誠,將加曲恕,吾所啟謬會,故特見酬納耳。若此疏迹宣布,則為侵奪主恩,不可之大者也。」使綜對前焚之。太祖後謂邵曰:「卿之獲免,謝述有力焉」。



 述有心虛疾,性理時或乖謬。除吳
 郡太守,以疾不之官。病差,補吳興太守。



 在郡清省,為吏民所懷。十二年,卒,時年四十六。喪還京師,未至數十里,殷景仁、劉湛同乘迎赴,望船流涕。十七年,劉湛誅,義康外鎮,將行,歎曰:「謝述唯勸吾退,劉湛唯勸吾進,今述亡而湛存,吾所以得罪也。」太祖亦曰:「謝述若存,義康必不至此。」



 三子:綜、約、緯。綜有才藝,善隸書,為太子中舍人,與舅范曄謀反,伏誅。



 約亦坐死。緯尚太祖第五女長城公主,素為約所憎,免死,徙廣州。孝建中,還京師。方雅有父
 風。太宗泰始中,至正員郎中。



 袁湛,字士深,陳郡陽夏人也。祖耽,晉歷陽太守。父質,瑯邪內史,並知名。



 湛少為從外祖謝安所知,以其兄子玄之女妻之。初為衛軍行參軍,員外散騎,通直正員郎,中軍功曹,桓玄太尉參軍事。入為中書黃門侍郎,出補桓修撫軍長史。



 義旗建,高祖以為鎮軍咨義參軍。明年,轉尚書吏部郎,司徒左長史,侍中。



 以從征功,封晉寧縣五等男。出為高祖太尉長史,遷左民尚書,徙掌吏部。出為
 吳興太守,秩中二千石,蒞政和理,為吏民所稱。入補中書令,又出為吳國內史,秩中二千石。義熙十二年,轉尚書右僕射、本州大中正。時高祖北伐,湛兼太尉,與兼司空、散騎常侍、尚書範泰奉九命禮物,拜授高祖。高祖沖讓,湛等隨軍至洛陽,住柏谷塢。泰議受使未畢,不拜晉帝陵,湛獨至五陵致敬,時人美之。



 初,陳郡謝重,王胡之外孫,於諸舅禮敬多闕。重子絢,湛之甥也,嘗於公座陵湛;湛正色謂曰:「汝便是兩世無《渭陽》之情。」絢有愧色。十
 四年,卒官,時年四十。追贈左光祿大夫,加散騎常侍。太祖即位,以后父,追贈侍中、以左光祿大夫、開府儀同三司。謚曰敬公。世祖大明三年,幸籍田,行經湛墓。下詔曰:「故侍中、左光祿大夫、開府儀同三司晉寧敬公,外氏尊戚,素風簡正,歲紀稍積,墳塋浸遠。朕近巡覽千畝,遙瞻松隧,緬惟徽塵,感慕增結。可遣使祭,少申永懷。」



 又增守墓五戶。



 子淳,淳子桓卒。湛弟豹,字士蔚,亦為謝安所知,好學博聞,多覽典籍。初為著作佐郎,衛軍桓謙記室參
 軍。大將軍武陵王遵承制,復為記室參軍。其年,丹陽尹孟昶以為建威司馬。歲餘,轉司徒左西屬,遷劉毅撫軍咨議參軍,領記室。毅時建議大田,豹上議曰:國因民以為本,民資食以為天,修其業則教興,崇其本則末理,實為治之要道,致化之所階也。不敦其本,則末業滋章;饑寒交湊,則廉恥不立。當今接篡偽之末,值凶荒之餘,爭源既開,彫薄彌啟,榮利蕩其正性,賦斂罄其所資,良疇無側趾之耦,比屋有困餧之患,中間多故,日不暇給。自
 卷甲卻馬,甫一二年,積弊之黎,難用克振,實仁懷之所矜恤,明教之所爰發也。



 然斯業不修,有自來矣。司牧之官,莫或為務,俗吏庸近,猶秉常科,依勸督之故典,迷民情之屢變。譬猶修隄以防川,忘淵丘之改易;膠柱於昔弦,忽宮商之乖調。徒有考課之條,而無毫分之益。不悟清流在於澄源,止輪由乎高閾,患生於本,治之於末故也。夫設位以崇賢,疏爵以命士,上量能以審官,不取人於浮譽,則比周道息,游者言歸;游子既歸,則南畝闢矣。
 分職以任務,置吏以周役,職不以無任立,吏必以非用省,冗散者廢,則萊荒墾矣。器以應用,商以通財,剿靡麗之巧,棄難得之貨,則彫偽者賤,穀稼重矣。耕耨勤悴,力殷收寡,工商逸豫,用淺利深,增賈販之稅,薄疇畝之賦,則末技抑而田畯喜矣。居位無義從之徒,在野靡兼并之黨,給賜非可恩致,力役不入私門,則游食者反本,肆勤自勸;游食省而肆勤眾,則東作繁矣。密勿者甄異,怠慢者顯罰,明勸課之令,峻糾違之官,則懶惰無所容,力
 田有所望;力者欣而惰者懼,則穡人勸矣。凡此數事,亦務田之端趣也。蒞之以清心,鎮之以無欲,勖之以無倦,翼之以廉謹,舍日計之小成,期遠致於莫歲,則澆薄自淳,心化有漸矣。



 豹善言雅俗,每商較古今,兼以誦詠,聽者忘疲。



 尋轉撫軍司馬,遷御史中丞。鄱陽縣侯孟懷玉上母檀氏拜國太夫人,有司奏許。



 豹以為婦人從夫之爵,懷玉父大司農綽見居列卿,妻不宜從子,奏免尚書右僕射劉柳、左丞徐羨之、郎何邵之官,詔並贖論。孟昶
 卒,豹代為丹陽尹。義熙七年,坐使徙上錢,降為太尉咨議參軍,仍轉長史。從討劉毅。高祖遣益州刺史朱齡石伐蜀,使豹為檄文,曰:夫順德者昌,逆德者亡,失仁與義,難以求安,馮阻負釁,鮮克有成。詳觀自古,隆替有數,故成都不世祀,華陽無興國。日者王室多故,夷羿遘紛,波振塵駭,覃及遐裔。蕞爾譙縱,編戶黔首,同惡相求,是崇是長,肆反噬於州相,播毒害於民黎,俾我西服,隔閡皇澤。自義風電靡,天光反輝,昭晢舊物,煙煴區宇。



 以庶務
 草創,未遑九伐,自爾以來,奄延十載。而野心不革,伺隙乘間,招聚逋叛,共相封殖,侵擾我蠻獠,搖蕩我疆垂。我是以有治洲之役,醜類盡殪,匹馬無遺,桓謙折首,譙福鳥逝,奔伏窠穴,引頸待戮。



 當今北狄露晞,南寇埃掃,朝風載韙,庶績其凝,康哉之歌日熙,比屋之隆可詠。孤職是經略,思一九有,眷彼禹跡,願言載懷,奉命西行,途戾荊、郢,瞻望巴、漢,憤慨交深。清江源於濫觴,澄氛昆于井絡,誅叛柔遠,今也其時。即命河間太守蒯恩、下邳太守
 劉鐘,精勇二萬,直指成都。龍驤將軍臧熹,戎卒二萬,進自墊江。益州刺史朱齡石,舟師三萬,電曜外水。分遣輔國將軍索懇,率漢中之眾,濟自劍道。振威將軍朱客子,提寧州之銳,渡瀘而入。神兵四臨,天綱宏掩,衡翼千里,金鼓萬張,組甲貝胄,景煥波屬,華夷百濮,雲會霧臻,以此攻戰,誰與為敵!況又奉義而行,以順而動者哉!



 今三陜之隘,在我境內,非有岑彭荊門之險。彌入其阻,平衢四達,實無鄧艾綿竹之艱。山川之形,抑非曩日,攻守難
 易,居然百倍。當全蜀之強,士民之富,子陽不能自安於庸、僰,劉禪不敢竄命於南中,荊邯折謀,伯約挫銳。故知成敗有數,非可智延,此皆益土前事,當今元龜也。盛如盧循,彊如容超,陵威南海,跨制北岱,樓船萬艘,掩江蓋汜,鐵馬千群,充原塞隰。然廣固之攻,陸無完雉;左里之戰,水靡全舟。或顯戮京畿,或傳首萬里。故知逆順有勢,難以力抗,斯又目前殷鑒,深切著明者也。



 梁益人士,咸明王化,雖驅迫一時,本非奧主。從之淫虐,日月增播,刑
 殺非罪,死以澤量。而待命寇讐之戮,㩻䧢豺狼之吻,豈不遡誠南凱,延首東雲,普天有來蘇之幸,而一方懷後予之怨。王者之師,以仁為本,舍逆取順,爰自三驅,齊斧所加,縱身而已。其有衿甲反接,自投軍門者,一無所問。士子百姓,列肆安堵,審擇吉凶,自求多祐。大信之明,皦若朝日,如其迷復姦邪,守愚不改,火燎孟諸,芝艾同爛,河決金隄,淵丘同體,雖欲悔之,亦將何及!



 九年,卒官。時年四十一。次年,以參伐蜀之謀,追封南昌縣五等子。



 子
 洵,元嘉中,歷顯官,廬陵王紹為南中郎將、江州刺史,年少未親政,洵為長史、尋陽太守,行府州事。元嘉末,為吳郡太守。元凶弒立,加洵建威將軍,置佐史。會安東將軍隨王誕起義,檄洵為前鋒,加輔國將軍。事平,頃之卒,追贈征虜將軍,謚曰貞子。長子顗,別有傳。少子覬,好學善屬文,有清譽於世。官至司徒從事中郎、武陵內史,蚤卒。洵弟濯,揚州秀才,蚤卒。濯弟淑,濯子粲,並有別傳。



 褚叔度,河南陽翟人也。曾祖裒,晉太傅。祖歆,秘書監。父
 爽,金紫光祿大夫。長兄秀之,字長倩,歷大司馬琅邪王從事中郎,黃門侍郎、高祖鎮西長史。秀之妹,恭帝后也,雖晉氏姻戚,而盡心於高祖。遷侍中,出補大司馬右司馬。恭帝即位,為祠部尚書、本州大中正。高祖受命,徙為太常。元嘉元年,卒官,時年四十七。



 秀之弟淡之,字仲源,亦歷顯官,為高祖車騎從事中郎,尚書吏部郎,廷尉卿,左衛將軍。高祖受命,為侍中。淡之兄弟並盡忠事高祖,恭帝每生男,輒令方便殺焉,或誘賂內人,或密加毒害,
 前後非一。及恭帝遜位,居秣陵宮,常懼見禍,與褚后共止一室,慮有鴆毒,自煮食於床前。高祖將殺之,不欲遣人入內,令淡之兄弟視褚后,褚后出別室相見,兵人乃踰垣而入,進藥於恭帝。帝不肯飲,曰:「佛教自殺者,不得復人身。」乃以被掩殺之。後會稽郡缺,朝議欲用蔡廓,高祖曰:「彼自是蔡家佳兒,何關人事,可用佛。」佛,淡之小字也。乃以淡之為會稽太守。



 景平二年,富陽縣孫氏聚合門宗,謀為逆亂,其支黨在永興縣,潛相影響。永興令羊恂
 覺其姦謀,以告淡之;淡之不信,乃以誣人之罪,收縣職局。於是孫法亮號冠軍大將軍,與孫道慶等攻沒縣邑,即用富陽令顧粲為令,加輔國將軍。遣偽建威將軍孫道仲、孫公喜、法殺攻永興。永興民灟恭期初與賊同,後反善就羊恂,率吏民拒戰,力少退敗。賊用縣人許祖為令,恂逃伏江唐山中,尋復為賊所得,使還行縣事。賊遂磐據,更相樹立,遙以鄮令司馬文寅為征西大將軍,孫道仲為征西長史,孫道覆為左司馬,與公喜、法殺等建
 旗鳴鼓,直攻山陰。



 淡之自假凌江將軍,以山陰令陸邵領司馬,加振武將軍,前員外散騎常侍王茂之為長史,前國子博士孔欣、前員外散騎常侍謝芩之並參軍事,召行參軍七十餘人。



 前鎮西咨議參軍孔寧子、左光祿大夫孔季恭子山士在艱中,皆起為將軍。遣隊主陳願、郡議曹掾虞道納二軍過浦陽江。願等戰敗,賊遂摧鋒而前,去城二十餘里。淡之遣陸邵督帶戟公石綝、廣武將軍陸允以水軍拒之,又別遣行參軍水屬恭期率步軍與
 邵合力。淡之率所領出次近郊。恭期等與賊戰於柯亭,大破之,賊走還永興。遣偽寧朔將軍孫倫領五百人攻錢唐,與縣戍軍建武將軍戰於琦,倫敗走還富陽。倫因反善,殺法步帥等十餘人,送首京都。詔遣殿中員外將軍徐卓領千人,右將軍彭城王義康遣龍驤將軍丘顯率眾五百東討,司空徐羨之版揚州主簿沈嗣之為富陽令領五百人,於吳興道東出,並未至而賊平。吳郡太守江夷輕行之職,停吳一宿,進至富陽,分別善惡,執送
 願徙賊餘黨數百家於彭城、壽陽、青州諸處。二年,淡之卒,時年四十五。謚曰質子。



 叔度名與高祖同,故以字行。初為太宰琅邪王參軍,高祖車騎參軍事,司徒左西屬,中軍咨議參軍,署中兵,加建威將軍。從伐鮮卑,盡其誠力。盧循攻查浦,叔度力戰有功。循南走,高祖版行廣州刺史,仍除都督交廣二州諸軍事、建威將軍、領平越中郎將、廣州刺史。桓玄族人開山聚眾,謀掩廣州,事覺,叔度悉平之。義熙八年,盧循餘黨劉敬道窘迫,詣交州歸
 降。交州刺史杜慧度以事言統府,叔度以敬道等路窮請命,事非款誠,報使誅之。慧度不加防錄,敬道招集亡命,攻破九真,殺太守杜章民,慧度討平之。叔度輒貶慧度號為奮揚將軍,惡不先上,為有司所糾,詔原之。



 高祖征劉毅,叔度遣三千人過嶠,荊州平乃還。在任四年,廣營賄貨,家財豐積,坐免官,禁錮終身。還至都,凡諸舊及有一面之款,無不厚加贈遺。尋除太尉咨議參軍、相國右司馬。高祖受命,為右衛將軍。高祖以其名家,而能竭
 盡心力,甚嘉之。乃下詔曰:「夫賞不遺勤,則勞臣增勸;爵必疇庸,故在功咸達。叔度南北征討,常管戎要,西夏不虔,誠著嶺表,可封番禺縣男,食邑四百戶。」尋加散騎常侍。永初三年,出為使持節、監雍梁南北秦四州荊州之南陽竟陵順陽義陽新野隨六郡諸軍事、征虜將軍、雍州刺史,領寧蠻校尉、襄陽義成太守。在任每以清簡致稱。景平二年,卒,時年四十四。



 子恬之嗣,官至南琅邪太守。恬之卒,子昭嗣。昭卒,子瑄嗣。齊受禪,國除。



 叔度第二子寂
 之,著作佐郎,早卒。子曖,尚太祖第六女琅邪貞長公主,太宰參軍,亦早卒。



 秀之弟湛之,字休玄,尚高祖第七女始安哀公主,拜駙馬都尉、著作郎。哀公主薨,復尚高祖第五女吳郡宣公主。諸尚公主者,並用世胄,不必皆有才能。湛之謹實有意乾,故為太祖所知。歷顯位,揚武將軍、南彭城沛二郡太守,太子中庶子,司徒左長史,侍中,左衛將軍,左民尚書,丹陽尹。元凶弒逆,以為吏部尚書,復出為輔國將軍、丹陽尹,統石頭戍事。



 世祖入伐,劭自
 攻新亭壘,使湛之率水師俱進。湛之因攜二息淵、澄輕船南奔。



 淵有一男始生,為劭所殺。世祖即位,以為尚書右僕射。孝建元年,為中書令,丹陽尹。坐南郡王義宣諸子逃藏郡牴,建康令王興之、江寧令沈道源下獄,湛之免官楚錮。其年,復為散騎常侍、左衛將軍,俄遷侍中,左衛如故。以久疾,拜散騎常侍、光祿大夫,加金章紫綬。頃之,復為丹陽尹,光祿如故。尋為尚書左僕射。以南奔賜爵都鄉侯。大明四年,卒,時年五十。追贈侍中、特進、驃騎
 將軍,給鼓吹一部,左僕射如故。謚曰敬侯。



 子淵庶生,宣公主以淵有才,表為嫡嗣。淵,升明末為司空。



 史臣曰:高祖雖累葉江南,楚言未變,雅道風流,無聞焉爾。凡此諸子,並前代名家,莫不望塵請職,負羈先路,將由庇民之道邪。



\end{pinyinscope}