\article{卷八十一列傳第四十一 劉秀之 顧琛 顧覬之}

\begin{pinyinscope}

 劉秀之,字道寶,東莞莒人,司徒劉穆之從兄子也,世居京口。祖爽,尚書都官郎,山陰令。父仲道,高祖克京城,以補建武參軍,與孟昶留守,事定,以為餘姚令,卒官。



 秀子
 少孤貧,有志操。十許歲時,與諸兒戲於前渚,忽有大蛇來,勢甚猛,莫不顛沛驚呼,秀之獨不動,眾並異焉。東海何承天雅相知器,以女妻之。兄欽之為朱齡石右軍參軍,隨齡石敗沒,秀之哀戚,不歡宴者十年。景平二年,除駙馬都尉、奉朝請。家貧,求為廣陵郡丞。仍除撫軍江夏王義恭、平北彭城王義康行參軍,出為無錫、陽羨、烏程令,並著能名。



 元嘉十六年,遷建康令,除尚書中兵郎,重除建康。性纖密,善糾摘微隱,政甚有聲。吏部尚書沈演
 之每稱之於太祖。世祖鎮襄陽,以為撫軍錄事參軍、襄陽令。



 襄陽有六門堰,良田數千頃,堰久決壞,公私廢業。世祖遣秀之修復,雍部由是大豐。改領廣平太守。二十五年,除督梁、南北秦三州諸軍事、寧遠將軍、西戎校尉、梁、南秦二州刺史。時漢川飢儉,境內騷然,秀之善於為政,躬自儉約。先是,漢川悉以絹為貨,秀之限令用錢,百姓至今受其利。



 二十七年,大舉北伐,遣輔國將軍楊文德、巴西、梓潼二郡太守劉弘宗受秀之節度,震蕩汧、隴。秀
 之遣建武將軍錫千秋二千人向子午谷南口,府司馬竺宗之三千人向駱谷南口,威遠將軍梁尋千人向斜谷南口。氐賊楊高為寇,秀之討之,斬高兄弟。元凶弒逆,秀之聞問,即日起兵,求率眾赴襄陽,司空南譙王義宣不許。事寧,遷使持節、督益寧二州諸軍事、寧朔將軍、益州刺史。折留俸祿二百八十萬,付梁州鎮庫,此外蕭然。梁、益二州土境豐富,前後刺史,莫不營聚蓄,多者致萬金。所攜賓僚,並京邑貧士,出為郡縣,皆以茍得自資。秀
 之為治整肅,以身率下,遠近安悅焉。



 南譙王義宣據荊州為逆,遣參軍王曜徵兵於秀之,秀之即日斬曜戒嚴。遣中兵參軍韋山松萬人襲江陵,出峽。竺超民遣將席天生逆之,山松一戰,即梟其首。進至江陵,為魯爽所敗,山松見殺。其年,進號征虜將軍,改督為監,持節、刺史如故,以起義功,封康樂縣侯,食邑六百戶。明年,遷監郢州諸軍事、郢州刺史,將軍如故。未就。



 大明元年,徵為右衛將軍。明年,遷丹陽尹。先是,秀之從叔穆之為丹陽,與子
 弟於事上飲宴,秀之亦與焉。事柱有一穿,穆之謂子弟及秀之曰:「汝等試以栗遙擲此柱,若能入穿,後必得此郡。」穆之諸子並不能中,唯秀之獨入焉。時賒市百姓物,不還錢,市道嗟怨,秀之以為非宜,陳之甚切,雖納其言,竟不從用。



 廣陵王誕為逆,秀之入守東城。其年,遷尚書右僕射。四年,改定制令,疑民殺長史科,議者謂值赦宜加徙送,秀之以為:「律文雖不顯民殺官長之旨,若值赦但止徙送,便與悠悠殺人曾無一異。民敬官長,比
 之父母,行害之身,雖遇赦,謂宜長付尚方,窮其天命,家口令補兵。」從之。明年,領太子右衛率。



 五年,雍州刺史海陵王休茂反,為土人所誅,遣秀之以本官慰勞,分別善惡。



 事畢還都,出為使持節、散騎常侍、都督雍、梁、南北秦四州、郢州之竟陵、隨二郡諸軍事、安北將軍、寧蠻校尉、雍州刺史。上車駕幸新亭,視秀之發引,將徵為左僕射,事未行,八年卒,時年六十八。上甚痛惜之,詔曰:「秀之識局明遠,才應通暢,誠著蕃朝,績宣累嶽。往歲逆臣交構,
 首義萬里,及職司端尹,贊戎兩宮,嘉謀征譽,實彰朝野。漢南法繁民嗛,屬佇良牧,故暫輟心膂,外弘風規,出未踰期,德庇西服。詳考古烈,旅觀終始,淳心忠概,無以尚茲。方式亮皇猷,入衛根本,奄至薨逝,震慟于朕心。生榮之典,未窮寵數,哀終之禮,宜盡崇飾。兼履謙守約,封社弗廣,興言悼往,益增痛恨。可贈侍中、司空,持節、都督、刺史、校尉如故,并增封邑為千戶。謚為忠成公。」秀之野率無風采,而心力堅正。上以其蒞官清潔,家無餘財,賜錢
 二十萬,布三百匹。



 子景遠嗣,官至前軍將軍。景遠卒,子俊,齊受禪,國除。秀之弟粹之,晉陵太守。



 顧琛,字弘瑋,吳郡吳人也。曾祖和,晉司空。祖履之,父惔,並為司徒左西掾。



 琛謹確不尚浮華,起家州從事,駙馬都尉,奉朝請。少帝景平中,太皇太后崩,除大匠丞。彭城王義康右軍驃騎參軍,晉陵令,司徒參軍,尚書庫部郎,本邑中正。



 元嘉七年,太祖遣到彥之經略河南,大敗,悉委棄兵甲,武庫為之空虛。後太祖宴會,有荒外歸化人
 在坐,上問琛:「庫中仗猶有幾許?」琛詭答:「有十萬人仗。」



 舊武庫仗秘不言多少,上既發問,追悔失言,及琛詭對,上甚喜。



 尚書寺門有制,八座以下門生隨入者各有差,不得雜以人士。琛以宗人顧碩頭寄尚書張茂度門名,而與碩頭同席坐。明年,坐遣出,免中正。凡尚書官,大罪則免,小罪則遣出。遣出者,百日無代人,聽還本職。琛仍為彭城王義康所請,補司徒錄事參軍,山陰令,復為司徒錄事,遷少府。十五年,出為義興太守。初,義康請琛入府,
 欲委以腹心,琛不能承事劉湛,故尋見斥外。十九年,徙東陽太守,欲使琛防守大將軍彭城王義康,固辭忤旨,廢黜還家積年。



 二十七年,索虜南至瓜步,權假琛建威將軍。尋除東海王禕冠軍司馬,行會稽郡事。隨王誕代禕,復為誕安東司馬。元凶弒立,分會稽五郡置會州,以誕為刺史,即以琛為會稽太守,加五品將軍,置將佐。誕起義,加冠軍將軍。事平,遷吳興太守。孝建元年,徵為五兵尚書。未拜,復為寧朔將軍、吳郡太守。以起義功,封永新
 縣五等侯。大明元年,吳縣令張闓坐居母喪無禮,下廷尉。錢唐令沈文秀判劾違謬,應坐被彈。琛宣言於眾:「闓被劾之始,屢相申明。」又云:「當啟文秀留縣。」



 世祖聞之大怒,謂琛賣惡歸上,免官。琛母老,仍停家。



 琛及前西陽太守張牧,並司空竟陵王誕故佐,誕待琛等素厚。三年,誕據廣陵反,遣客陸延稔齎書板琛為征南將軍,牧為安東將軍,琛子前尚書郎寶素為諮議參軍,寶素弟前司空參軍寶先為從事中郎,牧兄前吳郡丞濟為冠軍將
 軍,從弟前司空主簿晏為諮議參軍。



 時世祖以琛素結事誕,或有異志,遣使就吳郡太守王曇生誅琛父子。會延稔先至,琛等即執斬之,遣二子送延稔首啟世祖曰:「劉誕猖狂,遂構釁逆,凡在含齒,莫不駭惋,臣等預荷國恩,特百常憤。忽以今月二十四日中獲賊誕疏,欲見邀誘。



 臣即共執錄偽使,并得誕與撫軍長史沈懷文、揚州別駕孔道存、撫軍中兵參軍孔璪、前司兵參軍孔桓之、前司空主簿張晏書,具列本郡太守王曇生。臣即日便
 應星馳歸骨輦轂,臣母年老,身在侍養,輒遣息寶素、寶先束骸詣闕。」世祖所遣誅琛使其日亦至,僅而獲免。上嘉之,召琛出,以為西陽王子尚撫軍司馬,牧為撫軍中兵參軍。琛母孔氏,時年百餘歲。晉安帝隆安初,琅邪王廞於吳中為亂,以女為貞烈將軍,悉以女人為官屬,以孔氏為司馬。及孫恩亂後,東土飢荒,人相食,孔氏散家糧以賑邑里,得活者甚眾,生子皆以孔為名焉。



 琛仍為吳興太守。明年,坐郡民多翦錢及盜鑄,免官。六年,起為
 大司農,都官尚書,新安王子鸞北中郎司馬、東海太守、行南徐州事,隨府轉撫軍司馬,太守如故。前廢帝即位,復為吳郡太守。太宗泰始初,與四方同反,兵敗,奉母奔會稽。



 臺軍既至,歸降。寶素與琛相失,自殺。琛尋丁母憂,服闋,起為員外常侍、中散大夫。後廢帝元徽三年,卒,時年八十六。



 寶先大明中為尚書水部郎。先是,琛為左丞荀萬秋所劾,及寶先為郎,萬秋猶在職,自陳不拜。世祖詔曰:「敕違糾慢,憲司之職,若理有不公,自當更有釐正。



 而自頃刻無輕重,輒致私絕。此風難長,主者嚴為其科。寶先蓋依附世準,不足問。」



 先是,宋世江東貴達者,會稽孔季恭,季恭子靈符,吳興丘淵之及琛,吳音不變。淵之字思玄,吳興烏程人也。太祖從高祖北伐,留彭城,為冠軍將軍、徐州刺史,淵之為長史。太祖即位,以舊恩歷顯官,侍中,都官尚書,吳郡太守。卒於太常,追贈光祿大夫。



 顧覬之,字偉仁,吳郡吳人也。高祖謙,字公讓,晉平原內史陸機姊夫。祖崇,大司農。父黃老,司徒左西掾。覬之初
 為郡主簿。謝晦為荊州,以為南蠻功曹,仍為晦衛軍參軍。晦愛其雅素,深相知待。王弘辟為揚州主簿,仍為弘衛軍參軍,鹽官令,衡陽王義季右軍主簿,尚書都官郎,護軍司馬。時大將軍彭城王義康秉權,殷、劉之隙已著,覬之不欲與殷景仁久接事,乃辭腳疾自免歸。在家每夜常於床上行腳,家人竊異之,而莫曉其意。後義康徙廢,朝廷多以異同受禍。復為東遷、山陰令。山陰民戶三萬,海內劇邑,前後官長,晝夜不得休,事猶不舉。覬之理
 繁以約,縣用無事,晝日垂簾,門階閑寂。自宋世為山陰,務簡而績修,莫能尚也。還為揚州治中從事史,廣陵王誕、廬陵王紹北中郎左司馬,揚州別駕從事史,尚書吏部郎。嘗於太祖坐論江左人物,言及顧榮,袁淑謂覬之曰:「卿南人怯懦,豈辦作賊。」覬之正色曰:「卿乃復以忠義笑人!」淑有愧色。



 元凶弒立,朝士無不移任,唯覬之不徙官。世祖即位,遷御史中丞。孝建元年,出為義陽王昶東中郎長史、寧朔將軍、行會稽郡事。尋徵為右衛將軍,領本
 邑中正。



 明年,出為湘州刺史,善於蒞民,治甚有績。大明元年,徵守度支尚書,領本州中正。二年,轉吏部尚書。四年,致仕,不許。



 時沛郡相縣唐賜往比村朱起母彭家飲酒還,因得病,吐蠱蟲十餘枚。臨死語妻張,死後刳腹出病。後張手自破視,五藏悉糜碎。郡縣以張忍行刳剖,賜子副又不禁駐,事起赦前,法不能決。律傷死人,四歲刑;妻傷夫,五歲刑;子不孝父母,棄市,並非科例。三公郎劉勰議:「賜妻痛往遵言,兒識謝及理,考事原心,非存忍害,
 謂宜哀矜。」覬之議曰:「法移路尸,猶為不道,況在妻子,而忍行凡人所不行。不宜曲通小情,當以大理為斷,謂副為不孝,張同不道。」詔如覬之議。加左軍將軍,出為吳郡太守。



 八年,復為吏部尚書,加給事中,未拜,欲以為會稽,不果。還為吳郡太守。



 幸臣戴法興權傾人主,而覬之未嘗降意。左光祿大夫蔡興宗與覬之善,嫌其風節過峻。覬之曰:「辛毗有云:孫、劉不過使吾不為三公耳!」及世祖晏駕,法興遂以覬之為光祿大夫,加金章紫綬。



 太宗泰
 始初,四方同反,覬之家尋陽,尋陽王子房加以位號,覬之不受,曰:「禮年六十不服戎,以其筋力衰謝,非復軍旅之日,況年將八十,殘生無幾,守盡家門,不敢聞命。」孔覬等不能奪。時普天叛逆,莫或自免,唯覬之心迹清全,獨無所與。太宗甚嘉之,東土既平,以為左將軍、吳郡太守,加散騎常侍。泰始二年,復為湘州刺史,常侍、將軍如故。三年卒,時年七十六。追贈鎮軍將軍,常侍、刺史如故。謚曰簡子。



 覬之家門雍睦,為州鄉所重。五子:約、緝、綽、縝、緄。綽
 私財甚豐,鄉里士庶多負其責,覬之每禁之,不能止。及後為吳郡,誘綽曰:「我常不許汝出責,定思貧薄亦不可居。民間與汝交關有幾許不盡,及我在郡,為汝督之。將來豈可得。



 凡諸券書皆何在?」綽大喜,悉出諸文券一大廚與覬之,覬之悉焚燒,宣語遠近:「負三郎責,皆不須還,凡券書悉燒之矣。」綽懊歎彌日。



 覬之常謂秉命有定分,非智力所移,唯應恭己守道,信天任運,而闇者不達,妄求僥倖,徒虧雅道,無關得喪。乃以其意命弟子愿著《
 定命論》,其辭曰:仲尼云:「道之將行,命也;道之將廢,命也。」丘明又稱:「天之所支不可壞,天之所壞不可支。」卜商亦曰:「死生有命,富貴在天。」孟軻則以不遇魯侯為辭。斯則運命奇偶,生數離合,有自來矣。馬遷、劉向、揚雄、班固之徒,著書立言,咸以為首,世之論者,多有不同。嘗試申之曰:夫生之資氣,清濁異原;命之稟數,盈虛乖致。是以心貌詭貿,性運舛殊,故有邪正昏明之差,修夭榮枯之序,皆理定於萬古之前,事徵於千代之外,沖神寂鑒,一以
 貫之。至乃卜相末技,巫史賤術,猶能豫題興亡,逆表成敗。禍福指期,識照不能徙;吉凶素著,威衛不能防。若夏氓宅生於帝宮,豈蠲殘傷之祟;漢臣衍貨於天府,寧免喂斃之魂。且又善惡之理雖詳,而禍福之驗常昧;逆順之體誠分,而吉兇之效常隱。智絡天地,猶罹沈牖之災;明照日月,必嬰深匡之難。增信積德,離患於長飢;席義枕仁,徼禍於促算。何則?理運茍其必至,聖明其猶病諸。況乃蕞跡流惑之徒,投心顓蒙之域,而欲役慮以揣利
 害,策情以算窮通,其為重傷,豈不惑甚。是以通人君子,閑泰其神,沖緩其度,不矯俗以延聲,不依世以期榮。審乎無假,自求多福,榮辱修夭,夫何為哉!



 問曰:夫《書》稱惠迪貽吉,《易》載履信逢祐,前哲餘議,亦以將迎有會,淪塞無兆,宣攝有方,夭閼無命。善游銷魂於深梁,工騎燼生於曠野,明珠招駭於暗至,蟠木取悅於先容。是以罕、樂以陽施長世;景、惠以陰德遐紀。彭、竇以繕衛延命;盈、忌以荒湎促齡。陳、張稱台鼎之崇;嚴、辛衍宰司之盛。若乃
 遊惡蹈凶,處逆踐禍,宣昭史策,易以研正。至如神仙所序,天竺所書,事雖難征,理未易詰,留滯傾光,思聞通裁。



 對曰:子可謂扶繩而辨,循刻而議。若乃宣攝有方,豈非吉運所屬;將迎有會,實亦凶數自挻。若夫陽施陰德,長世遐年,揆厥所原,孰往非命。研復來旨,仇校往說,起予惟商,未識所異。資生稟運,參差萬殊,逆順吉凶,理數不一。原夫餐椒非養生之術,咀劍豈衛性之經。命之所延,人肉其骨,而含嚼膏粱,時或嬰患。



 深澗乖徼寵之津,空
 谷絕探榮之轍,運之所集,物稊其枯,而俯仰竿牘,終然離沮。



 爾乃蹻、跖橫行;曾、原窘步。湯、周延世,詡、邑絕緒。吉凶徵應,糾纆若茲。



 畢萬保軀,宓賤喪領,梁野之言,豈不或妄。穀南、魯北,甘此促生;彭翁、竇叟,將以何術。晉平、趙敬,淫放已該;漢主、魏相,奚獨傷夭。同異若斯,是非孰正。



 至如雷濱凝分,挫志遠圖;棘津陰拱,振功高世。樊生沖矯,鐫旌善之文,華子高抗,銘懲非之策,皆士衡所云「同川而異歸」者也。殊塗均致,實繁有征。即理易推,在言可
 略。昔兩都全盛,六合殷昌,霧集貴寵之閭,雲動權豪之術,鈞貿貽談,豈唯陳、張而已。觀夫二子,才未越眾,而此以藉榮揮價,彼獨擯景淪聲,通否之運,斷可知矣。嚴、辛不安時任命,而委罪亮直,亦地脈之徒歟。若神仙所序,顯明修習,齊彊燕平,厥驗未著,李覃董芬,其效安在。喬、松之侶,雲飛天居,夷、列之徒,風行水息,良由理數懸挺,實乃鐘茲景命。天竺遺文,星華方策,因造前定,果報指期,貧豪莫差,修夭無爽,有允瑣辭,無愆鄙說,統而言之,
 孰往非命。



 冥期前定,各從所歸,善惡無所矯其趨,愚智焉能殊其理。若乃得議其工,失嗤其拙,操之則慄,舍之則悲,斯固染情於近累,豈不貽誚於通識。



 問曰:清論光心,英辯溢目,求諸鄙懷,良有未盡。若動止皆運,險易自天,理定前期,靡非闇至。玉門犁丘,睿識弗免。豈非聖愚齊致,仁虐同功。昏明之用,將何施而可?



 對曰:夫聖人懷虛以涵育,凝明以洞照。惟虛也,故無往而不通;惟明也,故無來而不燭。涸海流金,弗染溫涼之岨;嚴兵猛兕,無
 累爪刃之災。忘生而生愈全,遺神而神彌暢。若玉門犁丘,蓋同跡於人,故同人有患,然而均心於天,亦均天無害。大賢則體備形器,慮盡藏假,靜默以居否,深拱以違礥,皆數在清全,故鐘茲妙識。是以稟仲尼之道,不在奔車之上;資伯夷之運,不處覆舟之下。若乃越難趨險,逡巡弗獲,履危踐機,黽勉從事,愚之所司,聖亦何為。及中下之流,馳心妄動,是非舛幹,倚伏移貿,故北宮意逆而功順,東門心晦而迹明;宣應遺筮而逢吉,張松協數而
 遘禍。且智防有紀,患累無方。爾乃猘狗逐而華子奔,腐鼠遺而虞氏滅;匣猿逸而林木殘,櫝珠亡而池水竭。凡厥條流,曲難詳備,搖形役思,其效安征。



 豈若澡雪靈府,洗練神宅,据道為心,依德為慮,使迹窮則義斯暢,身泰則理兼通,豈不美哉!何必遺此而取彼。



 問曰:夫建極開化,樹聲貽則,典防之興,由來尚矣。必乃幽符懸兆,冥數指期,善惡前征,是非素定,名教之道,不亦幾乎息哉!



 對曰:天生蒸民,樹之物則,教義所稟,豈非冥數。何則?形氣
 之具,必有待而存;顓蒙之倫,豈無因而立。必假纖紈以安生,藉梁豢以延祀,資信禮以繕性,秉廉義以劾情。聖人聰明深懿,履道測化,通體天地,同情日月,仰觀俯察,撫運裁風。於是乎昭日星之紀,正霜雨之度,張雲霞之明,衍風露之渥,浮舟翼滯,騰駕振幽。又乃甄理三才,辨綜五德,弘鋪七體之端,宣昭八經之緒。是以時雍在運,群方自通,抱德煬和,全真保性。故信食相資,代為脣齒;富教相假,遞成輔車。



 今弛棄纖紈,損絕梁豢,必云徼生
 委命,豈不已曉其迷。至乎湮斥廉義,屏黜信禮,責以祈存推數,遂乃未辨其惑;連類若斯,乖妄滋甚。然則教義之道,生運所資,寵辱榮枯,常由此作。斯固命中之一物,非所以為難也。



 問曰:循復前旨,既以理命縣兆,生數冥期。研覆後文,又云依杖名教,帥循訓範。若藉數任天,則放情蕩思;拘訓馴範,則防慮檢喪。函矢殊用,矛戈異適,雙美之談,豈能兩遂。



 對曰:夫性運乖舛,心貌詭殊,請布末懷,略言其要。若乃吉命所鐘,縱情蹈道,訓性而順,因
 心則靈。凶數所挻,率由踐逆,聞言不信,長惡無悛。此愚智不移,聲訓所遺者也。其有見善如不及,從諫如順流,是則命待教全,運須化立。譬以良醫之室,病者所存,至如澄神清魂,平心實氣,無妄之痾,勿藥有喜,所謂縱情蹈道,無假隱括。若膏肓之疾,長桑不治,體府之病,陽慶弗理,此則率由踐逆,自絕調御。至乃趙儲之命宜永,須扁鵲而後全,齊后之數必延,待文摯而後濟。亦猶運鐘循獎,彞範所興,善惡無主,唯運所集而異。膏梁方丈,沈
 疾弗顧;瑤碧盈尺,阽危弗存。夫靜躁之容,造次必於是;曲直之性,顛沛不可移。是以夷、惠均聖而異方;遵、竦齊通而殊事。雖復鉗桎羿、帟,思服巢、許之情;捶勒曾、史,言膺蹻、跖之慮。不然之事,斷可知也。必幽符鉆仰,冥數修習,雖存陵惰,其可得乎!故運屬波流,勢無防慮,命徼山立,理無放情。用殊函矢,雙美奚躓;談異矛戈,兩濟何傷。



 問曰:夫君臣恩深,師資義固,所以霑榮塗施,提飾荷聲。故刳心流腸,捐生以亢節;火妻灰子,霾名以償義。若幽
 期天兆,則明揚可遺;冥數自賓,則感效宜絕。豈其然乎?



 對曰:論之所明,原本以為理,難之所疑,即末以為用。蓋陰閉之巧不傳,萌漸之調長絕。故知妄言賞理,古人所難。吾所謂命,固以綿絡古今,彌貫終始,爰及君臣父子,師友夫妻,皆天數冥合,神運玄至。逮乎睽愛離會,既命之所甄,昏爽順戾,亦運之所漸。爾乃松柳異質,薺荼殊性,故疾風知勁草,嚴霜識貞木,何異忠孝之質,資行夙昭。至於刻志酬生,題誠復施,殉節投命,馴義忘己。亦由
 石雖可毀,堅不可銷,丹雖可磨,赤不可滅。因斯而言,君臣師資,既幽期自賓,心力感效,亦冥數天兆。夫獨何怪哉!



 願字子恭,父淵之,散騎侍郎。愿好學,有文辭於世。大明中,舉秀才,對策稱旨,擢為著作佐郎,太子舍人。早卒。



 史臣曰:孝建啟基,西楚放命,難連淮、濟,勢盛江服。朱修之著節漢南,劉秀之推鋒萬里,並誠載艱一,忠惟帝念。而踰峴之鋒,戰有獨克,出硤之師,舟無隻反。雖霜霰並時,而計功則異也。及定終之命,等數相懸,蓋由義結蕃
 朝,故恩有厚薄。雖故舊不遺,聞之前訓,隆名爽實,亦無取焉!



\end{pinyinscope}