\article{卷六十一列傳第二十一 武三王}

\begin{pinyinscope}

 武帝七男:張夫人生少帝,孫脩華生廬陵孝獻王義真,胡婕妤生文皇帝,王脩容生彭城王義康,袁美人生江夏文獻王義恭,孫美人生南郡王義宣,呂美人生衡陽
 文王義季。義康、義宣別有傳。



 廬陵孝獻王義真,美儀貌,神情秀徹。初封桂陽縣公,食邑千戶。年十二,從北征大軍進長安,留守栢谷塢,除員外散騎常侍,不拜。及關中平定,高祖議欲東還,而諸將行役既久,咸有歸願,止留偏將,不足鎮固人心,乃以義真行都督雍、涼、秦三州之河東、平陽、河北三郡諸軍事、安西將軍、領護西戎校尉、雍州刺史。



 太尉諮議參軍京兆王脩為長史,委以關中之任。高祖將還,三秦父老詣門流涕訴曰:「殘民不沾王化,於
 今百年矣。始睹衣冠,方仰聖澤。長安十陵,是公家墳墓,咸陽宮殿數千間,是公家屋宅,舍此欲何之?」高祖為之愍然,慰譬曰:「受命朝廷,不得擅留。感諸君戀本之意,今留第二兒,令文武賢才共鎮此境。」臨還,自執義真手以授王脩,令脩執其子孝孫手以授高祖。義真尋除正,加節,又進督并東秦二州、司州之東安定、新平二郡諸軍事,領東秦州刺史。時隴上流人,多在關中,望因大威,復得歸本。及置東秦州,父老知無復經略隴右、固關中之
 意,咸共歎息。



 而佛佛虜寇逼交至。



 沈田子既殺王鎮惡,王脩又殺田子。義真年少,賜與左右不節,脩常裁減之,左右並怨。因是白義真曰:「鎮惡欲反,故田子殺之。脩今殺田子,是又欲反也。」



 義真乃使左右劉乞等殺脩。脩字叔治,京兆灞城人也。初南渡見桓玄,玄知之,謂曰:「君平世吏部郎才。」脩既死,人情離駭,無相統一。



 高祖遣將軍朱齡石替義真鎮關中,使義真輕兵疾歸。諸將競斂財貨,多載子女,方軌徐行,虜追騎且至。建威將軍傅弘之
 曰:「公處分亟進,恐虜追擊人也。今多將輜重,一日行不過十里;虜騎追至,何以待之?宜棄車輕行,乃可以免。」不從。



 賊追兵果至,騎數萬匹。輔國將軍蒯恩斷後,不能禁;至青泥,後軍大敗,諸將及府功曹王賜悉被俘虜。義真在前,故得與數百人奔散。日暮,虜不復窮追。義真與左右相失,獨逃草中。中兵參軍段宏單騎追尋,緣道叫喚,義真識其聲,出就之,曰:「君非段中兵邪?身在此。」宏大喜,負之而歸。義真謂宏曰:「今日之事,誠無算略。然丈夫不
 經此,何以知艱難。」



 初,高祖聞青泥敗,未得義真審問,有前至者訪之,並云「暗夜奔敗,無以知存亡」。高祖怒甚,克日北伐,謝晦諫不從。及得宏啟事,知義真已免,乃止。



 義真尋都督司、雍、秦、并、涼五州諸軍、建威將軍、司州刺史,持節如故。



 以段宏為義真諮議參軍,尋遷宋臺黃門郎,領太子右衛率。宏,鮮卑人也,為慕容超尚書左僕射、徐州刺史。高祖伐廣固,歸降。太祖元嘉中,為征虜將軍、青冀二州刺史。追贈左將軍。時義真將鎮洛陽,而河南蕭
 條,未及脩理,改除揚州刺史,鎮石頭。



 永初元年,封廬陵王,食邑三千戶,移鎮東城。高祖始踐阼,義真意色不悅,侍讀博士蔡茂之問其故,義真曰:「安不忘危,休泰何可恃。」明年,遷司徒。高祖不豫,以為使持節、侍中、都督南豫、豫、雍、司、秦、并六州諸軍事、車騎將軍、開府儀同三司、南豫州刺史,出鎮歷陽。未之任而高祖崩。



 義真聰明愛文義,而輕動無德業。與陳郡謝靈運、瑯邪顏延之、慧琳道人並周旋異常,云得志之日,以靈運、延之為宰相,慧琳
 為西豫州都督。徐羨之等嫌義真與靈運、延之暱狎過甚,故使范晏從容戒之。義真曰:「靈運空疏,延之隘薄,魏文帝云鮮能以名節自立者。但性情所得,未能忘言於悟賞,故與之遊耳。」將之鎮,列部伍於東府前,既有國哀,義真所乘舫單素,不及母孫脩儀所乘者。義真與靈運、延之、慧琳等共視部伍,因宴舫內,使左右剔母舫函道以施己舫,而取其勝者。及至歷陽,多所求索;羨之等每裁量不盡與,深怨執政,表求還都。而少帝失德,羨之等
 密謀廢立,則次第應在義真,以義真輕吵,不任主社稷,因其與少帝不協,乃奏廢之,曰:臣聞二叔不咸,難結隆周,淮南悖縱,禍興盛漢,莫不義以斷恩,情為法屈。



 二代之事,殷鑒無遠,仁厚之主,行之不疑。故共叔不斷,幾傾鄭國;劉英容養,釁廣難深。前事之不忘,後王之成鑒也。



 案車騎將軍義真,凶忍之性,爰自稚弱,咸陽之酷,醜聲遠播。先朝猶以年在紈綺,冀能改厲,天屬之愛,想聞革心。自聖體不豫,以及大漸,臣庶憂惶,內外屏氣。而縱博
 酣酒,日夜無輟,肆口縱言,多行無禮。先帝貽厥之謀,圖慮經固,親敕陛下,面詔臣等,若遂不悛,必加放黜;至言苦厲,猶在紙翰。而自茲迄今,日月增甚,至乃委棄籓屏,志還京邑,潛懷異圖,希幸非冀,轉聚甲卒,徵召車馬。



 陵墳未幹,情事猶昨,遂蔑棄遺旨,顯違成規,整棹浮舟,以示歸志,肆心專己,無復諮承。聖恩低徊,深垂隱忍,屢遣中使,苦相敦釋。而親對散騎侍郎邢安泰、廣武將軍茅仲思,縱其悖罵,訕主謗朝,此久播於遠近,暴於人聽。



 臣
 聞原火不撲,蔓草難除;青青不伐,終致尋斧。況憂深患著,社稷慮切。請一遵晉朝武陵舊典,使顧懷之旨,不墜於武廟;全宥之德,獲申於暱親。仰尋感慟,臨啟悲咽。



 乃廢義真為庶人,徙新安郡。前吉陽令堂邑張約之上疏諫曰:臣聞仁義之在天下,若中原之有菽;理感之被萬物,故不系於貴賤。是以考叔反悔誓於及泉,壺關復冤魂於湖邑。當斯之時,豈無尊卿賢輔,或以事迫心違,或以道壅謀屈,何嘗不願聞善於輿隸,藥石於阿氏哉!臣雖
 草
 芥,備充黔首,少不量力,頗高殉義之風,謂蹈善於朝聞,愈徒生於白首。用敢干禁忘戮,披敘丹愚。



 伏惟高祖武皇帝誕茲神武,撫運龍興,仰清天步,則齊德有虞,俯廓九州,則侔功大夏,故虔順天人,享有萬國。雖靈祚脩長,聖躬弗永,陛下繼明紹統,遐邇一心,籓王哲茂,四維寧謐,傾耳康哉之詠,企踵升平之風。



 竊念廬陵王少蒙先皇優慈之遇,長受陛下睦愛之恩。故在心必言,所懷必亮,容犯臣子之道,致招驕恣之愆。至於天姿夙成,實
 有卓然之美。宜在容養,錄善掩瑕,訓盡義方,進退以漸。今猥加剝辱,幽徙遠郡,上傷陛下棠棣之篤,下令遠近恇然失圖,士庶杜口,人為身計。臣伏思大宋之興,雖協應符緯,而開基造次,根條未繁。宜廣樹籓戚,敦睦以道,使兄弟之美,比輝魯、衛;龜策告同,祚均七百,豈不善哉!



 陛下富於春秋,慮未重復,忽安危之遠算,肆不忍於一朝。特願留神允思,重加詢采。上考前代興亡之由,中存武皇締構之業,下顧蒼生顒顒之望,時開曲宥,反王都
 邑。選保傅於舊老,求四友於髦俊,引誘情性,導達聰明。凡人在苦,皆能自厲,況王質朗心聰,易加訓範。且中賢之人,未能無過;過貴自改,罪願自新。



 以武皇之愛子,陛下之懿弟,豈可以其一眚,長致淪棄哉!謹昧死詣闕,伏地以聞。



 惟願丹誠,一經天聽,退就斧金矍,無愧地下矣。



 書奏,以約之為梁州府參軍,尋又見殺。景平二年六月癸未,羨之等遣使殺義真於徙所,時年十八。元嘉元年八月,詔曰:「前廬陵王靈柩在遠,國封墮替,感惟拱慟,情若
 貫割。王體自至極,地戚屬尊,豈可令情禮永淪,終始無寄。可追復先封,特遣奉迎,并孫脩華、謝妃一時俱還。言增摧哽。」三年正月,誅徐羨之、傅亮等。是日詔曰:「故廬陵王含章履正,英哲自然,道心內昭,徽風遐被。遭時多難,志匡權逼,天未悔禍,運鐘屯險,群凶肆醜,專竊國柄,禍心潛構,釁生不圖。朕每永念讎恥,含痛內結,遵養姦慝,情禮未申。今王道既亨,政刑始判,宣昭國體,於是乎在。可追崇侍中、大將軍,王如故。為慰冤魂,少申悲憤。」又詔
 曰:「乃者權臣陵縱,兆亂基禍,故吉陽令張約之抗疏矢言,至誠慷慨,遂事屈群醜,殞命遐疆,志節不申,感焉兼至。昔關老奏書,見紀漢策,閻纂獻規,荷榮晉代。考其忠概,參跡前蹤,宜加旌顯,式揚義烈。可贈以一郡,賜錢十萬,布百匹。」



 義真無子,太祖以第五子紹字休胤為嗣。元嘉九年,襲封廬陵王。少而寬雅,太祖甚愛之。二十年,出為南中郎將、江州刺史,時年十二。二十二年,入朝,加棨戟,進都督江州、豫州之西陽、晉熙、新蔡三郡諸軍事。在
 任七年,改授左將軍、南徐州刺史,給鼓吹一部。未之鎮,仍遷揚州刺史,將軍如故。索虜至瓜步,紹從太子鎮石頭。二十九年,疾患解職。其年薨,時年二十一。遺令斂以時服,素棺周身,太祖從之。追贈散騎常侍、鎮軍將軍、開府儀同三司,刺史如故。



 無子,南平王鑠第三子敬先為嗣。本名敬秀,既出繼而紹妃褚秀之孫女,故改焉。景和二年,為前廢帝所害。追贈中書侍郎,謚曰恭王。無子,太宗泰始元年,以世祖第二十一子晉熙王子輿字孝文
 為紹嗣,封廬陵王。為輔國將軍、南高平、臨淮二郡太守,並未拜,為太宗所殺。三年,更以桂陽王休範第二子德嗣紹。為建威將軍、淮陵、南彭城二郡太守。後廢帝元徽二年,與休範俱伏誅。國復絕。三年,復以臨澧忠侯襲第三子皓字淵華繼紹。為給事中。順帝昇明元年,薨,謚曰元王。



 又無子,國除。



 江夏文獻王義恭,幼而明穎,姿顏美麗,高祖特所鐘愛,諸子莫及也。飲食寢臥,常不離於側。高祖為性儉約,諸
 子食不過五盞盤,而義恭愛寵異常,求須果食,日中無算,得未嘗啖,悉以乞與傍人。廬陵諸王未嘗敢求,求亦不得。



 景平二年,監南豫、豫、司、雍、秦、并、六州諸軍事、冠軍將軍、南豫州刺史,代廬陵王義真鎮歷陽,時年十二。元嘉元年,封江夏王,食邑五千戶。加使持節,進號撫軍將軍,給鼓吹一部。三年,監南徐、兗二州、揚州之晉陵諸軍事、徐州刺史,持節、將軍如故。進監為都督,未之任。太祖征謝晦,義恭還鎮京口。六年,改授散騎常侍、都督荊、湘、雍、
 益、梁、寧南北秦八州諸軍事、荊州刺史,持節、將軍如故。義恭涉獵文義,而驕奢不節,既出鎮,太祖與書誡之曰:汝以弱冠,便親方任。天下艱難,家國事重,雖曰守成,實亦未易。隆替安危,在吾曹耳,豈可不感尋王業,大懼負荷。今既分張,言集無日,無由復得動相規誨,宜深自砥礪,思而後行。開布誠心,厝懷平當,親禮國士,友接佳流,識別賢愚,鑒察邪正,然後能盡君子之心,收小人之力。



 汝神意爽悟,有日新之美,而進德脩業,未有可稱,吾所
 以恨之而不能已已者也。汝性褊急,袁太妃亦說如此。性之所滯,其欲必行,意所不在,從物回改,此最弊事。宜應慨然立志,念自裁抑。何至丈夫方欲贊世成名而無斷者哉!今粗疏十數事,汝別時可省也。遠大者豈可具言,細碎復非筆可盡。



 禮賢下士,聖人垂訓;驕侈矜尚,先哲所去。豁達大度,漢祖之德;猜忌褊急,魏武之累。《漢書》稱衛青云:「大將軍遇士大夫以禮,與小人有恩。」西門、安于,矯性齊美;關羽、張飛,任偏同弊。行己舉事,深宜鑒此。



 若事異今日,嗣子幼蒙,司徒便當周公之事,汝不可不盡祗順之理。茍有所懷,密自書陳。若形迹之間,深宜慎護。至於爾時安危,天下決汝二人耳,勿忘吾言。



 今既進袁太妃供給,計足充諸用,此外一不須復有求取,近亦具白此意。唯脫應大餉致,而當時遇有所乏,汝自可少多供奉耳。汝一月日自用不可過三十萬,若能省此,益美。



 西楚殷曠,常宜早起,接對賓侶,勿使留滯。判急務訖,然後可入問訊,既睹顏色,審起居,便應即出,不須久停,
 以廢庶事也。下日及夜,自有餘閑。



 府舍住止,園池堂觀,略所諳究,計當無須改作。司徒亦云爾。若脫於左右之宜,須小小回易,當以始至一治為限,不煩紛紜,日求新異。



 凡訊獄多決,當時難可逆慮,此實為難,汝復不習,殊當未有次第。訊前一二日,取訊簿密與劉湛輩共詳,大不同也。至訊日,虛懷博盡,慎無以喜怒加人。能擇善者而從之,美自歸己。不可專意自決,以矜獨斷之明也。萬一如此,必有大吝,非唯訊獄,君子用心,自不應爾。刑獄不
 可壅滯,一月可再訊。



 凡事皆應慎密,亦宜豫敕左右,人有至誠,所陳不可漏泄,以負忠信之款也。



 古人言「君不密則失臣,臣不密則失身」。或相讒構,勿輕信受,每有此事,當善察之。



 名器深宜慎惜,不可妄以假人。暱近爵賜,尤應裁量。吾於左右雖為少恩,如聞外論,不以為非也。以貴陵物物不服,以威加人人不厭,此易達事耳。



 聲樂嬉游,不宜令過,蒱酒漁獵,一切勿為。供用奉身,皆有節度;奇服異器,不宜興長。汝嬪侍左右,已有數人,既始至
 西,未可匆匆復有所納。



 又誡之曰:宜數引見佐史,非唯臣主自應相見。不數,則彼我不親。不親則無因得盡人;人不盡,復何由知其眾事。廣引視聽,既益開博,於言事者,又差有地也。



 九年,徵為都督南兗、徐、兗、青、冀、幽六州、豫州之梁郡諸軍事、征北將軍、開府儀同三司、南兗州刺史,鎮廣陵。時詔內外百官舉才,義恭上表曰:臣聞雲和備樂,則繁會克諧,驊騮驂服,則致遠斯效。陛下順簡夤化,文明在躬,玉衡既正,泰階載一,而猶發慮英髦,垂
 情仄陋,幽谷空同,顯著揚歷。是以潛虯聳鱗,佇利見之期;翔鳳弭翼,應來儀之感。



 竊見南陽宗炳,操履閑遠,思業真純,砥節丘園,息賓盛世,貧約而苦,內無改情,軒冕屢招,確爾不拔。若以蒲帛之聘,感以大倫之美,庶投竿釋褐,翻然來儀,必能毗燮九官,宣贊百揆。



 尚書金部郎臣徐森之,臣府中直兵參軍事臣王天寶,並局力允濟,忠諒款誠。



 往年逆臣叛逸,華陽失守,森之全境寧民,績章危棘。前者經略伊、瀍,元戎喪旅,天寶北勤河朔,東據
 營丘,勳勇既昭,心事兼竭。雖蒙褒敘,未盡才宜,並可授以邊籓,展其志力。



 交趾遼邈,累喪籓將,政刑每闕,撫蒞惟艱。南中夐遠,風謠迥隔,蠻獠狡竊,邊氓荼炭,實須練實,以綏其難。謂森之可交州刺史,天寶可寧州刺吏,庶足威懷荒表,肅清遐服。昔魏戊之賢,功存薦士;趙武之明,事彰管庫。臣識愧前良,理謝先哲,率舉所知,仰酬採訪,退懼瞽言,無足甄獎。



 十六年,進位司空。明年,大將軍彭城王義康有罪出籓,徵義恭為侍中、都督揚、南徐、兗
 三州諸軍事、司徒、錄尚書,領太子太傅,持節如故,給班劍二十人,置仗加兵。明年,解督南兗。二十一年,進太尉,領司徒,餘如故。義恭既小心恭慎,且戒義康之失,雖為總錄,奉行文書而已,故太祖安之。相府年給錢二千萬,它物倍此,而義恭性奢,用常不足,太祖又別給錢年千萬。二十六年,領國子祭酒。



 時有獻五百里馬者,以賜義恭。



 二十七年春,索虜寇豫州,太祖因此欲開定河、洛。其秋,以義恭總統群帥,出鎮彭城,解國子祭酒。虜遂深入,徑
 至瓜步,義恭與世祖閉彭城自守。二十八年春,虜退走,自彭城北過,義恭震懼不敢追。其日,民有告:「虜驅廣陵民萬餘口,夕應宿安王陂,去城數十里。今追之,可悉得。」諸將並請,義恭又禁不許。經宿,太祖遣驛至,使悉力急追。義恭乃遣鎮軍司馬檀和之向蕭城。虜先已聞知,乃盡殺所驅廣陵民,輕騎引去。初,虜深入,上慮義恭不能固彭城,備加誡敕。義恭答曰:「臣未能臨瀚海,濟居延,庶免劉仲奔逃之恥。」及虜至,義恭果走,賴眾議得停,事在《
 張暢傳》。降義恭號驃騎將軍、開府儀同三司,餘悉如故。



 魯郡孔子舊庭有柏樹二十四株,經歷漢、晉,其大連抱。有二株先折倒,士人崇敬,莫之敢犯,義恭悉遣人伐取,父老莫不歎息。又以本官領南兗州刺史,增督南兗、豫、徐、兗、青、冀、司、雍、秦、幽、并十一州諸軍事,并前十三州,移鎮盱眙。脩治館宇,擬制東城。



 二十九年冬,還朝,上以御所乘蒼鷹船上迎之。遭太妃憂,改授大將軍、都督揚、南徐二州諸軍事、南徐州刺史,持節、侍中、錄尚書、太子太
 傅如故。還鎮東府。辭侍中,未拜。值元凶肆逆,其日劭召義恭。先是,詔召太子及諸王,各有常人,慮有詐妄致害者。至是義恭求常所遣傳詔,劭遣之而後入。義恭請罷兵,凡府內兵仗,並送還臺。進位太保,進督會州諸軍事,服侍中服,又領大宗師。



 世祖入討,劭疑義恭有異志,使入住尚書下省,分諸子並住神虎門外侍中下省。



 劭聞世祖已次近路,欲悉力逆之,決戰中道。義恭慮世祖船乘陋小,劭豕突中流,容能為患,乃進說曰:「割棄南岸,柵
 斷石頭,此先朝舊法;以逸待勞,不憂不破也。」劭從之。世祖前鋒至新亭,劭挾義恭出戰,恒錄在左右,故不能自拔。戰敗,使義恭於東堂簡將。義恭先使人具船於東冶渚,因單馬南奔。始濟淮,追騎已至北岸,僅然得免。劭大怒,遣始興王濬就西省殺義恭十二子。



 世祖時在新林浦,義恭既至,上表勸世祖即位,曰:「臣聞治亂無兆,倚伏相因,乾靈降禍,二凶極逆,深酷巨痛,終古未有。陛下忠孝自天,赫然電發,投袂泣血,四海順軌,是以諸侯雲赴,
 數均八百;義奮之旅,其會如林。神祚明德,有所底止,而沖居或躍,未登天祚,非所以嚴重宗社,紹延七百。昔張武抗辭,代王順請;耿純陳款,光武正位。況今罪逆無親,惡盈釁滿,阻兵安忍,戮善崇姦,履地戴天,畢命俄頃;宜早定尊號,以固社稷。景平之季,實惟樂推,王室之亂,天命有在,故抱拜兆於壓璧,赤龍表於霄征。伏惟大明無私,遠存家國七廟之靈,近哀黔首荼炭之切,時陟帝祚,永慰群心。臣負釁嬰罰,偷生人壤,幸及寬政,待罪有司,
 敢以漏刻視息,披露肝膽。」世祖即祚,授使持節、侍中、都督揚、南徐二州諸軍事、太尉、錄尚書六條事、南徐、徐二州刺史,給鼓吹一部,班劍二十人;又假黃鉞。事寧,進位太傅,領大司馬,增班劍為三十人。以在籓所服玉環大綬賜之。增封二千戶。



 上不欲致禮太傅,諷有司奏曰:「聖旨謙光,尊師重道,欲致拜太傅,斯誠弘茲遠風,敦闡盛則。然周之師保,實稱三吏,晉因於魏,特加其禮。帝道嚴極,既有常尊,考之史載,未見茲典。故卞壺、孫楚並謂人
 君無降尊之義。遠稽聖典,近即群心,臣等參議謂不應有加拜之禮。」詔曰:「暗薄纂統,實憑師範,思盡虔恭,以承道訓。所奏稽諸往代,謂無拜禮,據文既明,便從所執。」世祖立太子,東宮文案,使先經義恭。



 孝建元年,南郡王義宣、臧質、魯爽等反,加黃鉞,白直百人入六門。事平,以臧質七百里馬賜義恭,又增封二千戶。世祖以義宣亂逆,由於彊盛,至是欲削弱王侯。義恭希旨,乃上表省錄尚書,曰:「臣聞天地設位,三極同序,皇王化則,九官咸事。時
 亮之績,昭於《虞典》;論道之風,宣於周載。台輔之設,坐調陰陽,元、凱之置,起釐百揆。所以欒針矢言,侵官是誡;陳平抗辭,匪職罔答。漢承秦後,庶僚稍改。爵因時變,任與世移,總錄之制,本非舊體,列代相沿,茲仍未革。



 今皇家中造,事遵前文,宜憲章先代,證文古則,停省條錄,以依昔典。使物競思存,人懷勤壹,則名實靡愆,庸節必紀。臣謬典國重,虛荷崇位,興替宜知,敢不輸盡。」上從其議。又與驃騎大將軍竟陵王誕奏曰:「臣聞佾懸有數,等級異
 儀,佩笏有制,卑高殊序。斯蓋上哲之洪謨,範世之明訓。而時至彌流,物無不弊,僭侈由俗,軌度非古。晉代東徙,舊法淪落,侯牧典章,稍與事廣,名實一差,難以卒變,章服崇濫,多歷年所。今樞機更造,皇風載新,耗弊未充,百用思約,宜備品式之律,以定損厭之條。臣等地居枝暱,位參台輔,遵正之首,請以爵先;致貶之端,宜從戚始。輒因暇日,共參愚懷,應加省易,謹陳九事。雖懼匪衷,庶竭微款。伏願陛下聽覽之餘,薄垂昭納,則上下相安,表裏
 和穆矣。」詔付外詳。有司奏曰:車服以庸,《虞書》茂典;名器慎假,《春秋》明誡。是以尚方所制,漢有嚴律,諸侯竊服,雖親必罪。降于頃世,下僭滋極。器服裝飾,樂舞音容,通於王公,達于眾庶。上下無辨,民志靡壹。義恭所陳,實允禮度。九條之格,猶有未盡,謹共附益,凡二十四條:聽事不得南向坐,施帳并沓。籓國官,正冬不得跣登國殿,及夾侍國師傳令及油戟;公主王妃傳令,不得朱服;輿不得重㭎;鄣扇不得雉尾;劍不得鹿盧形;槊眊不得孔雀白氅;
 夾轂隊不得絳襖;平乘誕馬不得過二匹;胡伎不得綵衣;舞伎正冬著褂衣,不得裝面;冬會不得鐸舞、杯柈舞;長蹺、透狹、舒丸劍、博山、緣大橦、升五案,自非正冬會奏舞曲,不得舞;諸妃主不得著緄帶;信幡非臺省官悉用絳;郡縣內史相及封內官長,於其封君,既非在三,罷官則不復追敬,不合稱臣,宜止下官而已;諸鎮常行,車前後不得過六隊,白直夾轂,不在其限。刀不得過銀銅為飾;諸王女封縣主,諸王子孫襲封之王妃及封侯者夫人
 行,並不得鹵簿;諸王子繼體為王者,婚葬吉凶,悉依諸國公侯之禮,不得同皇弟皇子。車非軺車,不得油幢;平乘船皆下兩頭作露平形,不得擬象龍舟,悉不得朱油;帳鉤不得作五花及豎筍形。



 詔可。



 是歲十一月,還鎮京口。二年春,進督東、南兗二州。其冬,徵為揚州刺史,餘如故。加入朝不趨,贊拜不名,劍履上殿,固辭殊禮。又解持節、都督并侍中。



 義恭撰《要記》五卷,起前漢訖晉太元,表上之,詔付秘閣。時西陽王子尚有盛寵,義恭解揚州以
 避之,乃進位太宰,領司徒。義恭常慮為世祖所疑,及海陵王休茂於襄陽為亂,乃上表曰:古先哲王,莫不廣植周親,以屏帝宇,諸侯受爵,亦願永固邦家。至有管蔡、梁燕,致禍周、漢,上乖顯授之恩,下亡血食之業。夫善積慶深,宜享長久,而歷代侯王,甚乎匹庶。豈異姓皆賢,宗室悉不賢。由生於深宮,不睹稼穡,左右近習,未值田蘇,富貴驕奢,自然而至,聚毛折軸,遂乃危禍。漢之諸王,並置傅相,猶不得禁逆;七國連謀,實由彊盛。晉氏列封,正足
 成永嘉之禍。尾大不掉,終古同疾,不有更張,則其源莫救。



 日者庶人恃親,殆傾王業。去歲西寇藉寵,幾敗皇基。不圖襄楚,復生今釁,良以地勝兵勇,獎成凶惡。前事之不忘,後事之明兆。陛下大明紹祚,垂法萬葉。



 臣年衰意塞,無所知解。忝皇族耆長,慚慨內深,思表管見,裨崇萬一。竊謂諸王貴重,不應居邊,至於華州優地,時可暫出。既以有州,不須置府。若位登三事,止乎長史掾屬。若宜鎮御,別差扞城大將。若情樂沖虛,不宜逼以戎事。若舍
 文好武,尤宜禁塞。僚佐文學,足充話言,遊梁之徒,一皆勿許。文武從鎮,以時休止,妻子室累,不煩自隨。百僚脩詣,宜遵晉令,悉須宣令齊到,備列賓主之則。衡泌之士,亦無煩干候貴王。器甲於私,為用蓋寡,自金銀裝刀劍戰具之服,皆應輸送還本。曲突徙薪,防之有素,庶善者無懼,惡者止姦。



 時世祖嚴暴,義恭慮不見容,乃卑辭曲意,盡禮祗奉,且便辯善附會,俯仰承接,皆有容儀。每有符瑞,輒獻上賦頌,陳詠美德。大明元年,有三脊茅生石
 頭西岸,累表勸封禪,上大悅。三年,省兵佐,加領中書監,以崇藝、昭武、永化三營合四百三十七戶給府;更增吏僮千七百人,合為二千九百人。六年,解司徒府太宰府依舊辟召。又年給三千匹布。七年,從巡,兼尚書令,解中書監。八年閏月,又領太尉。其月,世祖崩,遺詔:「義恭解尚書令,加中書監;柳元景領尚書令,入住城內。事無巨細,悉關二公;大事與沈慶之參決,若有軍旅,可為總統。尚書中事委顏師伯。外監所統委王玄謨。」



 前廢帝即位,詔
 曰:「總錄之典,著自前代。孝建始年,雖暫并省,而因革有宜,理存濟務。朕煢獨在躬,未涉政道,百揆庶務,允歸尊德。太宰江夏王義恭新除中書監、太尉,地居宗重,受遺阿衡,實深憑倚,用康庶績,可錄尚書事,本官監、太宰、王如故;侍中、驃騎大將軍、南兗州刺史、巴東郡開國公、新除尚書令元景,同稟顧誓,翼輔皇家,贊業宣風,繄公是賴。可即本號開府儀同三司,領兵置佐,一依舊準,領丹陽尹、侍中、領公如故。」又增義恭班劍為四十人,更申殊
 禮之命。固辭殊禮。



 義恭性嗜不恆,日時移變,自始至終,屢遷第宅。與人遊款,意好亦多不終。



 而奢侈無度,不愛財寶,左右親幸者,一日乞與,或至一二百萬;小有忤意,輒追奪之。大明時,資供豐厚,而用常不足,賒市百姓物,無錢可還,民有通辭求錢者,輒題後作「原」字。善騎馬,解音律,游行或三五百里,世祖恣其所之。東至吳郡,登虎丘山,又登無錫縣烏山以望太湖。大明中撰國史,世祖自為義恭作傳。及永光中,雖任宰輔,而承事近臣戴法
 興等,常若不及。



 前廢帝狂悖無道,義恭、元景等謀欲廢立。永光元年八月,廢帝率羽林兵於第害之,并其四子,時年五十三。斷析義恭支體,分裂腸胃,挑取眼精,以蜜漬之,以為鬼目精。



 太宗定亂,令書曰:「故中書監、太宰、領太尉、錄尚書事江夏王道性淵深,睿鑒通遠,樹聲列籓,宣風鉉德,位隆姬輔,任屬負圖,勤勞國家,方熙託付之重,盡心毗導,永融雍穆之化。而凶醜忌威,奄加冤害,夷戮有暴,殯穸無聞,憤達幽明,痛貫朝野。朕蒙險在難,含哀
 莫申,幸賴宗祏之靈,克纂祈天之祚,仰惟勳戚,震慟于厥心。昔梁王徵庸,警蹕備禮;東平好善,黃屋在廷。況公德猷弘懋,彞典未殊者哉!可追崇使持節、侍中、都督中外諸軍事、丞相、領太尉,中書監、錄尚書事、王如故。給九旒鸞輅,虎賁班劍百人,前後部羽葆、鼓吹,轀輬車。」



 泰始三年,又下詔曰:「皇基崇建,《屯》、《剝》維難,弘啟熙載,底績忠果,故從饗世祀,勒勳宗彞。世祖寧亂定業,實資翼亮。故使持節、侍中、都督中外諸軍事、丞相、領太尉、中書監、錄
 尚書事江夏文獻王義恭,故使持節、侍中、都督南豫、江豫、三州軍事、太尉、南豫州刺史巴東郡開國忠烈公元景,故侍中、司空始興郡開國襄公慶之,故持節、征西將軍、雍州刺史洮陽縣開國肅侯愨,或體道沖玄,燮化康世,或盡誠致效,庚難龕逆,宜式遵國典,陪祭廟庭。」



 義恭長子朗,字元明,出繼少帝,封南豐縣王,食邑千戶。為湘州刺史、持節、侍中,領射聲校尉。為元凶所殺。世祖即位,追贈前將軍、江州刺史。孝建元年,以宗室祗長子歆繼
 封。祗伏誅,歆還本。泰始三年,更以宗室韞第二子銑繼封。為秘書郎,與韞俱死。順帝昇明二年,復以宗室琨子績繼封。三年,薨。會齊受禪,國除。



 朗弟睿,字元秀,太子舍人。為元凶所害。追贈侍中,謚宣世子。大明二年,追封安隆王。以第四皇子子綏字寶孫繼封,食邑二千戶。追謚睿曰宣王。以子綏為都督郢州諸軍事、冠軍將軍、郢州刺史;進號後軍將軍,加持節。太宗泰始元年,進號征南將軍,改封江夏王,食邑五千戶。改睿為江夏宣王。子綏
 未受命,與晉安王子勛同逆,賜死。七年,太宗以第八子躋字仲升,繼義恭為孫,封江夏王,食邑五千戶。後廢帝即位,督會稽、東陽、新安、臨海、永嘉五郡諸軍事、東中郎將、會稽太守,進號左將軍。齊受禪,降為沙陽縣公,食邑一千五百戶。謀反,賜死。



 睿弟韶,字元和,封新吳縣侯,官至步兵校尉。追贈中書侍郎,謚曰烈侯。韶弟坦,字元度,平都懷侯。坦弟元諒,江安愍侯。元諒弟元粹,興平悼侯。坦、元諒、元粹並追贈散騎侍郎。元粹弟元仁、元方、元旒、
 元淑、元胤與朗等凡十二人,並為元凶所殺。元胤弟伯禽,孝建三年生。義恭諸子既遇害,為朝廷所哀,至是世祖名之曰伯禽,以擬魯公伯禽,周公旦之子也。官至輔國將軍、湘州刺史。又為前廢帝所殺。謚曰哀世子。又追贈江夏王,改謚曰愍。伯禽弟仲容,封永脩縣侯。為寧朔將軍、臨淮、濟陽二郡太守。仲容弟叔子,封永陽縣侯。叔子弟叔寶,及仲容、叔子,並為前廢帝所殺。謚仲容、叔子並曰殤侯。



 衡陽文王義季,幼而夷簡,無鄙近之累。太祖為荊州,高祖使隨往江陵,由是特為太祖所愛。元嘉元年,封衡陽王,食邑五千戶。五年,為征虜將軍。八年,領石頭戍事。九年,遷使持節、都督南徐州諸軍事、右將軍、南徐州刺史。十六年,代臨川王義慶都督荊、湘、雍、益、梁、寧、南北秦八州諸軍事、安西將軍、荊州刺史,持節如故,給鼓吹一部。先是,義慶在任,值巴蜀亂擾,師旅應接,府庫空虛,義季躬行節儉,畜財省用,數年間,還復充實。隊主續豐母老
 家貧,無以充養,遂斷不食肉。義季哀其志,給豐母月白米二斛,錢一千,并制豐啖肉。義季素拙書,上聽使餘人書啟事,唯自署名而已。二十年,加散騎常侍,進號征西大將軍,領南蠻校尉。



 義季素嗜酒,自彭城王義康廢後,遂為長夜之飲,略少醒日。太祖累加詰責,義季引愆陳謝。上詔報之曰:「誰能無過,改之為貴耳。此非唯傷事業,亦自損性命,世中比比,皆汝所諳。近長沙兄弟,皆緣此致故。將軍蘇徽,耽酒成疾,旦夕待盡,吾試禁斷,并給藥
 膳,至今能立。此自是可節之物,但嗜者不能立志裁割耳。



 晉元帝人主,尚能感王導之諫,終身不復飲酒。汝既有美尚,加以吾意殷勤,何至不能慨然深自勉厲,乃復須嚴相割裁,坐諸紜紜,然後少止者。幸可不至此,一門無此酣法,汝於何得之?臨書歎塞。」義季雖奉此旨,酣縱如初,遂以成疾。上又詔之曰:「汝飲積食少,而素羸多風,常慮至此,今果委頓。縱不能以家國為懷,近不復顧性命之重,可歎可恨,豈復一條。本望能以理自厲,未欲相
 苦耳。今遣孫道胤就楊佛等令晨夕視汝,并進止湯食,可開懷虛受,慎勿隱避。吾飽嘗見人斷酒,無它慊吸,蓋是當時甘嗜罔己之意耳。今者憂怛,政在性命,未暇及美業,復何為吾煎毒至此邪!」義季終不改,以至於終。



 二十一年,為都督南兗、徐、青、冀、幽六州諸軍事、征北大將軍、開府儀同三司、南兗州刺史,持節、常侍如故。登舟之日,帷帳器服,諸應隨刺史者,悉留之,荊楚以為美談。二十二年,進督豫州之梁郡。遷徐州刺史,持節、常侍、都督
 如故。明年,索虜侵逼,北境擾動,義季懲義康禍難,不欲以功勤自業,無它經略,唯飲酒而已。太祖又詔之曰:「杜驥、申怙,倉卒之際,尚以弱甲瑣卒,徼寇作援。



 彼為元統,士馬桓桓,既不懷奮發,連被意旨,猶復逡巡。豈唯大乖應赴之宜,實孤百姓之望。且匈奴輕漢,將自此而始。賊初起逸,未知指趨,故且裝束,兼存觀察耳。少日勢漸可見,便應大有經略,何合安然,遂不敢動。遣軍政欲乘際會,拯危急,以申威援,本無驅馳平原方幅爭鋒理。又山
 路易憑,何以畏首尾迥弱。若謂事理政應如此者,進大鎮,聚甲兵,徒為煩耳。」



 二十四年,義季病篤,上遣中書令徐湛之省疾,召還京師。未及發,薨於彭城,時年三十三。太尉江夏王義恭表解職迎喪,不許。上遣東海王禕北迎義季喪。追贈侍中、司空,持節、都督、刺史如故。



 子恭王嶷,字子岐嗣。中書侍郎,太子中庶子。世祖大明七年,薨,追贈冠軍將軍、豫州刺史。子伯道嗣。順帝升明三年,薨。其年,齊受禪,國除。



 史臣曰:戒懼乎其所不睹,恐畏乎其所不聞,在於慎所忽也。江夏王,高祖寵子,位居上相,大明之世,親典冠朝。屈體降情,盤闢於軒檻之上,明其為卑約亦已至矣。得使虐朝暴主,顧無猜色,歷載逾十,以尊戚自保。及在永光,幼主南面,公旦之重,屬有所歸。自謂踐冰之慮已除,泰山之安可恃,曾未雲幾,而磔體分肌。



 古人以隱微致戒,斯為篤矣。



\end{pinyinscope}