\article{卷三十九志第二十九 百官上}

\begin{pinyinscope}

 太
 宰,一人。周武王時,周公旦始居之,掌邦治,為六卿之首。秦、漢、魏不常置。晉初依《周禮》,備置三公。三公之職,太師居首,景帝名師,故置太宰以代之。太宰,蓋古之太師
 也。殷紂之時,箕子為太師。周武王時,太公為太師。周成王時,周公為太師;周公薨,畢公代之。漢西京初不置,平帝始復置太師官,而孔光居焉。漢東京又廢。獻帝初,董卓為太師,卓誅又廢。魏世不置。晉既因太師而置太宰,以安平王孚居焉。



 太傅,一人。周成王時,畢公為太傅。漢高後元年,初用王陵。



 太保,一人。殷太甲時,伊尹為太保。周武王時,召公為太
 保。漢平帝元始元年,始用王舜。後漢至魏不置,晉初復置焉。自太師至太保,是為三公。論道經邦,燮理陰陽,無其人則闕,所以訓護人主,導以德義者。



 相國,一人。漢高帝十一年始置,以蕭何居之,罷丞相;何薨,曹參代之;參薨,罷。魏齊王以晉景帝為相國。晉惠帝時趙王倫,愍帝時南陽王保,安帝時宋高祖,順帝時齊王,並為相國。自魏、晉以來,非復人臣之位矣。



 丞相,一人。殷湯以伊尹為右相,仲虺為左相。秦悼武王
 二年,始置丞相官。



 丞,奉;相,助也。悼武王子昭襄王始以樗里疾為丞相,後又置左右丞相。漢高帝初,置一丞相;十一年,更名相國。孝惠、高后置左右丞相,文帝二年,復置一丞相。哀帝元壽二年,更名大司徒。漢東京不復置。至獻帝建安十三年,復置丞相。



 魏世及晉初又廢。惠帝世,趙王倫篡位,以梁王肜為丞相。永興元年,以成都王穎為丞相。愍帝建興元年,以琅邪王睿為左丞相,南陽王保為右丞相;三年,以保為相國,睿為丞相。元帝永昌
 元年,以王敦為丞相,轉司徒荀組為太尉,以司徒官屬并丞相為留府,敦不受。成帝世,以王導為丞相,罷司徒府以為丞相府,導薨,罷丞相,復為司徒府。宋世祖初,以南郡王義宣為丞相,而司徒府如故。



 太尉,一人。自上安下曰尉。掌兵事,郊祀掌亞獻,大喪則告謚南郊。堯時舜為太尉官,漢因之。武帝建元二年省。光武建武二十七年,罷大司馬,置太尉以代之。靈帝末,以劉虞為大司馬,而太尉如故。



 司徒,一人。掌民事,郊祀掌省牲視濯,大喪安梓宮。少昊氏以鳥名官,而祝鳩氏為司徒。堯時舜為司徒。舜攝帝位,命契為司徒。契玄孫之孫曰微,亦為夏司徒。周時司徒為地官,掌邦教。漢西京初不置。哀帝元壽二年,罷丞相,置大司徒。



 光武建武二十七年,去「大」。



 司空,一人。掌水土事,郊祀掌掃除陳樂器,大喪掌將校復土。舜攝帝位,以禹為司空。契之子曰冥,亦為夏司空。殷湯以咎單為司空。周時司空為冬官,掌邦事。漢西京
 初不置。成帝綏和元年,更名御史大夫為大司空;哀帝建平二年,復為御史大夫;元壽二年,復為大司空;光武建武二十七年,去「大」字。獻帝建安十三年,又罷司空,置御史大夫。御史大夫郗慮免,不復補。魏初,又置司空。



 大司馬,一人。掌武事。司,主也;馬,武也。堯時棄為後稷,兼掌司馬。周時司馬為夏官,掌邦政。項籍以曹無咎、周殷並為大司馬。漢初不置,武帝元狩四年,初置大司馬。始直云司馬,議者以漢有軍候千人司馬官,故加大。及置司
 空,又以縣道官有獄司空,又加大。王莽居攝,以漢無小司徒,而定司馬、司徒、司空之號並加大。光武建武二十七年,省大司馬,以太尉代之。魏文帝黃初二年,復置大司馬,以曹仁居之,而太尉如故。



 大將軍,一人。凡將軍皆掌征伐。周制,王立六軍。晉獻公作二軍,公將上軍。



 將軍之名,起於此也。楚懷王遣三將入關,宋義為上將。漢高帝以韓信為大將軍。



 漢西京以大司馬冠之。漢東京大將軍自為官,位在三司上。魏明
 帝青龍三年,晉宣帝自大將軍為太尉,然則大將軍在三司下矣。其後又在三司上。晉景帝為大將軍,而景帝叔父孚為太尉,奏改大將軍在太尉下,後還復舊。



 晉武帝踐阼,安平王孚為太宰,鄭沖為太傅,王祥為太保,義陽王望為太尉,何曾為司徒,荀顗為司空,石苞為大司馬,陳騫為大將軍,凡八公同時並置,唯無丞相焉。



 有蒼頭字宜祿。至漢,丞相府每有所關白,到閣輒傳呼「宜祿」,以此為常。



 丞相置三長史。丞相有疾,御史大夫率百僚三
 旦問起居,及瘳,詔遣尚書令若光祿大夫賜養牛,上尊酒。漢景帝三公病,遣中黃門問病。魏、晉則黃門郎,尤重者或侍中也。魏武為丞相以來,置左右二長史而已。漢東京太傅府置掾、屬十人,御屬一人,令史十二人,不知皆何曹也。自太尉至大將軍、驃騎、車騎、衛將軍,皆有長史一人,將軍又各置司馬一人,太傅不置長史也。



 太尉府置掾、屬二十四人,西曹主府吏署用事,東曹主二千石長吏遷除事,戶曹主民戶祠祀農桑事,奏曹主奏議
 事,辭曹主辭訟事,法曹主郵驛科程事,尉曹主卒徒轉運事,賊曹主盜賊事,決曹主罪法事,兵曹主兵事,金曹主貨幣鹽鐵事,倉曹主倉穀事,黃閣主簿省錄眾事。御屬一人,令史二十二人。御屬主為公御,令史則有閣下、記室、門下令史,其餘史闕。案掾、屬二十四人,自東西曹凡十二曹,然則曹各置掾、屬一人,合二十四人也。



 司徒置掾、屬三十一人,御屬一人,令史三十五人。司空置掾二十九人,御屬一人,令史三十一人。司空別有道橋掾。
 其餘張減之號,史闕不可得知也。



 漢東京大將軍、驃騎將軍從事中郎二人,掾、屬二十九人,御屬一人,令史三十人。騎、衛將軍從事中郎二人,掾、屬二十人,御屬一人,令史二十四人。兵曹掾史主兵事,稟假掾史主稟假,又置外刺姦主罪法。其領兵外討,則營有五部,部有校尉一人,軍司馬一人;部下有曲,曲有軍候一人;曲下有屯,屯有屯長一人。



 若不置校尉,則部但有軍司馬一人。又有軍假司馬、軍假候,其別營者則為別部司馬。其餘將
 軍置以征伐者,府無員職,亦有部曲司馬、軍候以領兵焉。案大將軍以下掾屬與三府張減,史闕不可得知。置令史、御屬者,則是同三府也。其云掾史者,則是有掾而無屬,又無令史、御屬,不同三府也。



 魏初公府職僚,史不備書。及晉景帝為大將軍,置掾十人,西曹、東曹、戶曹、倉曹、賊曹、金曹、水曹、兵曹、騎兵各一人,則無屬矣。魏元帝咸熙中,晉文帝為相國,相國府置中衛將軍、驍騎將軍、左右長史、司馬、從事中郎四人,主薄四人,舍人十九人,
 參軍二十二人,參戰十一人,掾、屬三十三人。東曹掾、屬各一人;西曹屬一人,戶曹掾一人,屬二人;賊曹掾一人,屬二人;金曹掾、屬各一人;兵曹掾、屬各一人,騎兵掾二人,屬一人;車曹掾、屬各一人;鎧曹掾、屬各一人;水曹掾、屬各一人,集曹掾、屬各一人,法曹掾、屬各一人,奏曹掾、屬各一人,倉曹屬二人,戎曹屬一人,馬曹屬一人,媒曹屬一人,合為三十三人。散屬九人,凡四十二人。



 晉初,凡位從公以上,置長史、西閣、東閣祭酒、西曹、東曹掾、戶曹、
 倉曹、賊曹屬各一人;加兵者又置司馬、從事中郎、主簿、記室督各一人,舍人四人;為持節都督者,置參軍六人。安平獻王孚為太宰,增掾、屬為十人,兵、鎧、士、營軍、刺姦五曹皆置屬,并前為十人也。楊駿為太傅,增祭酒為四人,掾、屬為二十人,兵曹分為左、右、法、金、田、集、水、戎、車、馬十曹,皆置屬,則為二十人。趙王倫為相國,置左右長史、司馬、從事中郎四人,參軍二十人,主簿、記室督、祭酒各四人,掾、屬四十人,東西曹又置屬,其餘十八曹皆置掾,則四
 十人矣。



 凡諸曹皆置御屬、令史、學幹,御屬職錄事也。江左以來,諸公置長史、倉曹掾、戶曹屬、東西閣祭酒各一人,主簿、舍人二人,御屬二人,令史無定員。領兵者置司馬一人,從事中郎二人,參軍無定員;加崇者置左右長史、司馬、從事中郎四人,掾、屬四人,則倉曹增置屬,戶曹置掾,江左加崇,極於此也。



 長史、司馬、舍人,秦官。從事中郎、掾、屬、主簿、令史,前漢官,陳湯為大將軍王鳳從事中郎是也。御屬、參軍,後漢官,孫堅為車騎參軍事是也。本
 於府主無敬,晉世太原孫楚為大司馬石苞參軍,輕慢苞,始制施敬。祭酒,晉官也,漢吳王濞為劉氏祭酒。夫祭祀以酒為本,長者主之,故以祭酒為稱。漢之侍中、魏之散騎常侍高功者,並為祭酒焉。公府祭酒,蓋因其名也。長史、從事中郎主吏,司馬主將,主簿、祭酒、舍人主閣內事,參軍、掾、屬、令史主諸曹事。司徒若無公,唯省舍人,其府常置,其職僚異於餘府。有左右長史、左西曹掾、屬各一人,餘則同矣。餘府有公則置,無則省。晉元帝為鎮東
 大將軍及丞相,置從事中郎,無定員,分掌諸曹,有錄事中郎、度支中郎、三兵中郎。其參軍則有諮議參軍二人,主諷議事,晉江左初置,因軍諮祭酒也。宋高祖為相,止置諮議參軍,無定員。今諸曹則有錄事、記室、戶曹、倉曹、中直兵、外兵、騎兵、長流賊曹、刑獄賊曹、城局賊曹、法曹、田曹、水曹、鎧曹、車曹、士曹、集、右戶、墨曹,凡十八曹參軍。參軍不署曹者,無定員。江左初,晉元帝鎮東丞相府有錄事、記室、東曹、西曹、度支、戶曹、法曹、金曹、倉曹、理曹、中兵、外兵、
 騎兵、典兵、兵曹、賊曹、運曹、禁防、典賓、鎧曹、田曹、士曹、騎士、車曹參軍。其東曹、西曹、度支、金曹、理曹、典兵、兵曹、賊曹、運曹、禁防、典賓、騎士、車曹凡十三曹,今闕所餘十二曹也。其後又有直兵、長流、刑獄、城局、水曹、右戶、墨曹七曹。高祖為相,合中兵、直兵置一參軍,曹則猶二也。今小府不置長流參軍者,置禁防參軍。



 蜀丞相諸葛亮府有行參軍,晉太傅司馬越府又有行參軍、兼行參軍,後漸加長兼字。



 除拜則為參軍事,府板則為行參軍。晉末以來,
 參軍事、行參軍又各有除板。板行參軍下則長兼行參軍。參軍督護,江左置。本皆領營,有部曲,今則無矣。公府長史、司馬,秩千石;從事中郎,六百石;東西曹掾,四百石;他掾三百石;屬二百石。



 特進,前漢世所置,前後二漢及魏、晉以為加官,從本官車服,無吏卒。晉惠帝元康中定位令在諸公下,驃騎將軍上。



 驃騎將軍,一人。漢武帝元狩二年,始用霍去病為驃騎
 將軍。漢西京制,大將軍、驃騎將軍位次丞相。



 車騎將軍,一人。漢文帝元年,始用薄昭為車騎將軍。魚豢曰:「魏世車騎為都督,儀與四征同。若不為都督,雖持節屬四征者,與前後左右雜號將軍同。其或散還從文官之例,則位次三司。」晉、宋車騎、衛不復為四征所督也。



 衛將軍,一人。漢文帝元年,始用宋昌為衛將軍。三號位亞三司。漢章帝建初三年,始使車騎將軍馬防班同三司,班同三司自此始也。漢末奮威將軍,晉江右伏波、輔
 國將軍,並加大而儀同三司。江左以來,將軍則中、鎮、撫、四鎮以上或加大,餘官則左右光祿大夫以上並得儀同三司,自此以下不得也。



 持節都督,無定員。前漢遣使,始有持節。光武建武初,征伐四方,始權時置督軍御史,事竟罷。建安中,魏武帝為相,始遣大將軍督軍。二十一年,征孫權還,夏侯惇督二十六軍是也。魏文帝黃初二年,始置都督諸州軍事,或領刺史。三年,上軍大將軍曹真都督中外諸軍事,假黃
 鉞,則總統外內諸軍矣。明帝太和四年,晉宣帝征蜀,加號大都督。高貴公正元二年,晉文帝都督中外諸軍,尋加大都督。晉世則都督諸軍為上,監諸軍次之,督諸軍為下。使持節為上,持節次之,假節為下。



 使持節得殺二千石以下;持節殺無官位人,若軍事得與使持節同;假節唯軍事得殺犯軍令者。晉江左以來,都督中外尤重,唯王導居之。宋氏人臣則無也。江夏王義恭假黃鉞。假黃鉞,則專戮節將,非人臣常器矣。



 征東將軍,一人。漢獻帝初平三年,馬騰居之。征南將軍,一人。漢光武建武中,岑彭居之。征西將軍,一人。漢光武建武中,馮異居之。征北將軍,一人。魚豢曰:「四征,魏武帝置,秩二千石。黃初中,位次三公。漢舊諸征與偏裨雜號同。」



 鎮東將軍,一人。後漢末,魏武帝居之。鎮南將軍,一人。後漢末,劉表居之。



 鎮西將軍,一人。後漢初平三年,韓遂居之。
 鎮北將軍,一人。



 中軍將軍,一人。漢武帝以公孫敖為之,時為雜號。鎮軍將軍,一人。魏以陳群為之。撫軍將軍,一人。魏以司馬宣王為之。中、鎮、撫三號比四鎮。



 安東將軍,一人。後漢末,陶謙為之。安南將軍,一人。安西將軍,一人。後漢末,段煨為之。
 安北將軍,一人。魚豢曰:「鎮北、四安,魏黃初、太和中置。」



 平東將軍,一人。平南將軍,一人。平西將軍,一人。平北將軍,一人。四平,魏世置。



 左將軍、右將軍、前將軍、
 後將軍。左將軍以下,周末官,秦、漢並因之,光武建武七年省,魏以來復置。



 征虜將軍,漢光武建武中,始以祭遵居之。冠軍將軍,楚懷王以宋義為卿子冠軍。冠軍之名,自此始也。魏正始中,以文欽為冠軍將軍、揚州刺史。輔國將軍,漢獻帝以伏完居之。宋太宗泰始四年,改為輔師;後廢帝元徽二年,復故。
 龍驤將軍,晉武帝始以王浚居之。



 東中郎將,漢靈帝以董卓居之。南中郎將,漢獻帝建安中,以臨淄侯曹植居之。



 西中郎將。北中郎將,漢建安中,以焉陵侯曹彰居之。凡四中郎將,何承天云,並後漢置。



 建威將軍,漢光武建武中,以耿弇為建威大將軍。振威將軍,後漢初,宋登為之。
 奮威將軍,前漢世,任千秋為之。揚威將軍,魏置。廣威將軍,魏置。建武將軍,魏置。振武將軍,前漢末,王況為之。奮武將軍,後漢末,呂布為之。揚武將軍,光武建武中,以馬成為之。廣武將軍,晉江左置。



 鷹揚將軍,漢建安中,魏武以曹洪居之。折衝將軍,漢建安中,魏武以樂進居之。輕車將軍,漢武帝以公孫賀為之。揚烈將軍,建安中,以假公孫淵。寧遠將軍,晉江左置。材官將軍,漢武帝以李息為之。伏波將軍,漢武帝征南越,始置此號,以路博德為之。



 凌江將軍,魏置。自凌江以下,則有宣威、明威、驤
 威、厲威、威厲、威寇、威虜、威戎、威武、武烈、武毅、武奮、綏遠、綏邊、綏戎、討寇、討虜、討難、討夷、蕩寇、蕩虜、蕩難、蕩逆、殄寇、殄虜、殄難、掃夷、掃寇、掃虜、掃難、掃逆、厲武、厲鋒、虎威、虎牙、廣野、橫野、偏將軍、裨將軍,凡四十號。其威虜,漢光武以馮俊居之。虎牙,以蓋延居之,為虎牙大將軍。橫野,以耿純居之。



 蕩寇,漢建安中,滿寵居之。虎威,于禁居之。其餘或是後漢及魏所置,今則或置或不。自左右前後將軍以下至此四十號,唯四中郎將各一人,餘皆無定員。自車
 騎以下為刺史又都督及儀同三司者,置官如領兵;但云都督不儀同三司者,不置從事中郎,置功曹一人,主吏,在主簿上,漢末官也。漢東京司隸有功曹從事史,如諸州治中,因其名也。功曹參軍一人,主佐囗囗記室下,戶曹上。監以下不置諮議、記室,餘則同矣。宋太宗已來,皇子、皇弟雖非都督,亦置記室參軍。小號將軍為大郡邊守置佐吏者,又置長史,餘則同也。



 太常,一人。舜攝帝位,命伯夷作秩宗,掌三禮,即其任也。
 周時曰宗伯,是為春官,掌邦禮。秦改曰奉常,漢因之。景帝中六年,更名曰太常。應劭曰:「欲令國家盛大常存,故稱太常。」前漢常以列侯忠孝敬慎者居之,後漢不必列侯也。



 博士,班固云,秦官。史臣案,六國時往往有博士,掌通古今。漢武建元五年,初置《五經》博士。宣、成之世,《五經》家法稍增,經置博士一人。至東京凡十四人。《易》,施、孟、梁丘、京氏;《尚書》,歐陽、大小夏侯;《詩》,齊、魯、韓;《禮》,大小戴;《春秋》,嚴、顏:
 各一博士。而聰明有威重者一人為祭酒。魏及晉西朝置十九人,江左初減為九人,皆不知掌何經。元帝末,增《儀禮》、《春秋公羊》博士各一人,合為十一人。後又增為十六人,不復分掌《五經》,而謂之太學博士也。秩六百石。



 國子祭酒一人,國子博士一人,國子助教十人。《周易》、《尚書》、《毛詩》、《禮記》、《周官》、《儀禮》、《春秋左氏傳》、《公羊》、《穀梁》各為一經,《論語》、《孝經》為一經,合十經,助
 教分掌。國子,周舊名,周有師氏之職,即今國子祭酒也。晉初復置國子學,以教生徒,而隸屬太學焉。晉初助教十五人,江左以來,損其員。自宋世若不置學,則助教唯置一人,而祭酒、博士常置也。



 太廟令,一人。丞一人。並前漢置。西京曰長,東京曰令。領齋郎二十四人。



 明堂令,一人。丞一人。丞,漢東京初置;令,宋世祖大明中置。



 太
 祝令,一人。丞一人。掌祭祀讀祝迎送神。太祝,周舊官也。漢西京置太祝令、丞,武帝太初元年,更名曰廟祀。漢東京改曰太祝。



 太史令,一人,丞一人。掌三辰時日祥瑞妖災,歲終則奏新歷。太史,三代舊官,周世掌建邦之六典,正歲年,以序事頒朔于邦國。又有馮相氏,掌天文次序;保章氏,掌天文。今之太史,則并周之太史、馮相、保章三職也。漢西京曰太史令。



 漢東京有二丞,其一在靈臺。



 太樂令,一人。丞一人。掌凡諸樂事。周時為大司樂。漢西京曰太樂令。漢東京曰大予樂令。魏復為太樂令。



 陵令,每陵各一人,漢舊官也。



 乘黃令,一人。掌乘輿車及安車諸馬。魏世置。自博士至乘黃令,並屬太常。



 光祿勛,一人。丞一人。光,明也;祿,爵也;勳,功也。秦曰郎中令,漢因之。漢武太初元年,更名光祿勳。掌三署郎,郎執戟衛宮展門戶。光祿勳居禁中如御史,有獄在殿門外,
 謂之光祿外部。光祿勳郊祀掌三獻。魏、晉以來,光祿勳不復居禁中,又無復三署郎,唯外宮朝會,則以名到焉。二臺奏劾,則符光祿加禁止,解禁止亦如之。禁止,身不得入殿省,光祿主殿門故也。宮殿門戶,至今猶屬。晉哀帝興寧二年,省光祿勳,并司徒。孝武寧康元年,復置。漢東京三署郎有行應四科者,歲舉茂才二人,四行二人,及三署郎罷省,光祿勳猶依舊舉四行,衣冠子弟充之。三署者,五官署、左署、右署也,各置中郎將以司之。郡舉
 孝廉以補三署郎,年五十以上,屬五官,其次分在左右署。凡有中郎、議郎、侍郎、郎中四等,無員,多至萬人。



 左光祿大夫,右光祿大夫,二大夫,晉初置。光祿大夫,秦時為中大夫,漢武太初元年,更名光祿大夫;晉初又置左右光祿大夫,而光祿大夫如故。光祿大夫銀章青綬,其重者加金章紫綬,則謂之金紫光祿大夫。舊秩比二千石。



 中散大夫,王莽所置,後漢因之。前漢大夫皆無員,掌論
 議。後漢光祿大夫三人,中大夫二十人,中散大夫三十人。魏以來復無員。自左光祿大夫以下,養老疾,無職事。中散,六百石。



 衛尉,一人。丞二人。掌宮門屯兵,秦官也。漢景初,改為中大夫令。後元年,復為衛尉。晉江右掌冶鑄,領冶令三十九,戶五千三百五十。冶皆在江北,而江南唯有梅根及冶塘二冶,皆屬揚州,不屬衛尉。衛尉,江左不置,宋世祖孝建元年復置。舊一丞,世祖增置一丞。



 廷尉,一人。丞一人。掌刑辟。凡獄必質之朝廷,與眾共之之義。兵獄同制,故曰廷尉。舜攝帝位,咎繇作士,即其任也。周時大司寇為秋官,掌邦刑。秦為廷尉。漢景帝中六年,更名大理。武帝建元四年,復為廷尉。哀帝元壽二年,復為大理。漢東京初,復為廷尉。



 廷尉正,一人。廷尉監,一人。正、監並秦官。本有左右監,漢光武省右,猶云左監;魏、晉以來,直云監。
 廷尉評,一人。漢宣帝地節三年,初置左右評。漢光武省右,猶云左評。魏、晉以來,直云評。正、監、評並以下官禮敬廷尉卿。正、監秩千石,評六百石。廷尉律博士,一人。魏武初建魏國置。



 大司農,一人。丞一人。掌九穀六畜之供膳羞者。舜攝帝位,命棄為后稷,即其任也。周則為太府,秦治粟內史;漢景帝後元年,更名大農令;武帝太初元年,更名曰大司農。晉哀帝末,省並都水,孝武世復置。漢世丞二人,魏以
 來一人。



 太倉令,一人。丞一人。秦官也。晉江左以來,又有東倉、石頭倉丞各一人。



 導官令,一人。丞一人。掌舂御米。漢東亦置。導,擇也。擇米令精也。司馬相如《封禪書》云,導一莖六穗於庖。



 籍田令,一人。丞一人。掌耕宗廟社稷之田,於周為甸師。漢文帝初立籍田,置令、丞各一人。漢東京及魏並不置。晉武泰始十年復置。江左省,宋太祖元嘉中又置。自太
 倉至籍田令,並屬司農。



 少府,一人。丞一人。掌中服御之物。秦官也,漢因之。掌禁錢以給私養,故曰少府。晉哀帝末,省并丹陽尹。孝武世復置。



 左尚方令、丞各一人。右尚方令、丞各一人。並掌造軍器。秦官也,漢因之。



 於周則為玉府。晉江右有中尚方、左尚方、右尚方,江左以來,唯一尚方。宋高祖踐阼,以相府作部配臺,謂之左尚方,而
 本署謂之右尚方焉。又以相府細作配臺,即其名置令一人,丞二人,隸門下。世祖大明中,改曰御府,置令一人,丞一人。



 御府,二漢世典官婢作褻衣服補浣之事,魏、晉猶置其職,江左乃省焉。後廢帝初,省御府,置中署,隸右尚方。漢東京太僕屬官有考工令,主兵器弓弩刀鎧之屬,成則傳執金吾入武庫,及主織綬諸雜工。尚方令唯主作御刀綬劍諸玩好器物而已。然則考工令如今尚方,尚方令如今中署矣。



 東冶令,一人。丞一人。南冶令,一人。丞一人。漢有鐵官,晉署令,掌工徒鼓鑄,隸衛尉。江左以來,省衛尉,度隸少府。宋世雖置衛尉,冶隸少府如故。江南諸郡縣有鐵者或署冶令,或署丞,多是吳所置。



 平準令,一人。丞一人。掌染,秦官也,漢因之。漢隸司農,不知何世隸少府。



 宋順帝即位,避帝諱,改曰染署。



 將作大匠,一人。丞一人。掌土木之役。秦世置將作少府,
 漢因之。景帝中六年,更名將作大匠。光武中元二年省,以謁者領之。章帝建初元年復置。晉氏以來,有事則置,無則省。



 大鴻臚,掌贊導拜授諸王。秦世為典客,漢景帝中六年,更名大行令;武帝太初元年,更名大鴻臚。鴻,大也;臚,陳也。晉江左初省。有事則權置,事畢即省。



 太僕,掌輿馬。周穆王所置,秦因之。《周官》則校人掌馬,巾車掌車,及置太僕,兼其任也。晉江左或署或省,宋以來
 不置。郊祀則權置太僕執轡,事畢即省。



 太后三卿,各一人。應氏《漢官》曰:「衛尉、少府,秦官;太僕,漢成帝置。



 皆隨太后宮為號,在正卿上,無太后乃闕。」魏改漢制,在九卿下。晉復舊,在同號卿上。



 大長秋,皇后卿也。有后則置,無則省。秦時為將行,漢景帝中六年,更名大長秋。韋曜曰:「長秋者,以皇后陰官,秋者陰之始,取其終而長,欲其久也。」



 自太常至長秋,皆置功曹、主簿、五官。漢東京諸郡有五官掾,因其名也。漢制
 卿尹秩皆中二千石,丞一千石。



 尚書,古官也。舜攝帝位,命龍作納言,即其任也。《周官》司會,鄭玄云,若今尚書矣。秦世少府遣吏四人在殿中主發書,故謂之尚書。尚猶主也。漢初有尚冠、尚衣、尚食、尚浴、尚席、尚書,謂之六尚。戰國時已有尚冠、尚衣之屬矣。



 秦時有尚書令、尚書僕射、尚書丞。至漢初並隸少府,漢東亦猶文屬焉。古者重武官,以善射者掌事,故曰僕射。僕射者,僕役於射事也。秦世有左右曹諸吏,官無職事,
 將軍大夫以下皆得加此官。漢武帝世,使左右曹諸吏分平尚書奏事。昭帝即位,霍光領尚書事;成帝初,王鳳錄尚書事。漢東京每帝即位,輒置太傅,錄尚書事,薨輒省。晉康帝世,何充讓錄表曰:「咸康中,分置三錄,王導錄其一,荀崧、陸曄各錄六條事。」然則似有二十四條,若止有二十條,則荀、陸各錄六條,導又何所司乎?若導總錄,荀、陸分掌,則不得復云導錄其一也。其後每置二錄,輒云各掌六條事,又是止有十二條也。十二條者,不知悉何
 條。晉江右有四錄,則四人參錄也。江右張華、江左庾亮並經關尚書七條,則亦不知皆何事也。後何充解錄,又參關尚書。錄尚書職無不總。王肅注《尚書》「納于大麓」曰:「堯納舜於尊顯之官,大錄萬機之政也。」凡重號將軍刺史,皆得命曹授用,唯不得施除及加節。



 宋世祖孝建中,不欲威權外假,省錄。大明末復置。此後或置或省。漢獻帝建安四年,以執金吾榮郃為尚書左僕射,衛臻為右僕射。二僕射分置,自此始也。漢成帝建始四年,初置尚
 書,員四人,增丞亦為四人。曹尚書其一曰常侍曹,主公卿事;其二曰二千石曹,主郡國二千石事;其三曰民曹,主吏民上書事;其四曰客曹,主外國夷狄事。光武分二千石曹為二,又分客曹為南主客曹、北主客曹,改常侍曹為吏曹,凡六尚書。減二丞,唯置左右二丞而已。應劭《漢官》云:「尚書令、左丞,總領綱紀,無所不統。僕射、右丞,掌稟假錢穀。三公尚書二人,掌天下歲盡集課;吏曹掌選舉、齋祠;二千石曹掌水、火、盜賊、詞訟、罪法;客曹掌羌、胡
 朝會,法駕出,護駕;民曹掌繕治、功作、鹽池、苑囿。吏曹任要,多得超遷。」則漢末曹名及職司又與光武時異也。魏世有吏部、左民、客曹、五兵、度支五曹尚書。晉初有吏部、三公、客曹、駕部、屯田、度支六曹尚書。武帝咸寧二年,省駕部尚書,四年又置。太康中,有吏部、殿中、五兵、田曹、度支、左民六尚書。惠帝世,又有右民尚書。尚書止於六曹,不知此時省何曹也。江左則有祠部、吏部、左民、度支、五兵,合不五曹尚書。宋高祖初,又增都官尚書。若有右僕
 射,則不置祠部尚書。世祖大明二年,置二吏部尚書,而省五兵尚書,後還置一吏部尚書。順帝昇明元年,又置五兵尚書。



 尚書令,任總機衡;僕射、尚書,分領諸曹。左僕射領殿中、主客二曹;吏部尚書領吏部、刪定、三公、比部四曹;祠部尚書領祠部、儀曹二曹;度支尚書領度支、金部、倉部、起部四曹。左民尚書領左民、駕部二曹;都官尚書領都官、水部、庫部、功論四曹;五兵尚書領中兵、外兵二曹。昔有
 騎兵、別兵、都兵,故謂之五兵也。五尚書、二僕射、一令,謂之八坐。若營宗廟宮室,則置起部尚書,事畢省。



 漢成帝之置四尚書也,無置郎之文。《漢儀》,尚書郎四人,一人主匈奴單于營部,一人主羌夷吏民,一人主戶口墾田,一人主財帛委輸。匈奴單于,宣帝之世,保塞內附;成帝世,單于還北庭矣。一郎主匈奴單于營部,則置郎疑是光武時,所主匈奴,是南單于也。《漢官》云,置郎三十六人,不知是何帝增員。然則一尚書則領六郎也。主作文書,
 起立事草。初為郎中,滿歲則為侍郎。尚書寺居建禮門內。



 尚書郎入直,官供青縑白綾被,或以綿緤為之。給帷帳、氈褥、通中枕,太官供食物,湯官供餅餌及五孰果實之屬,給尚書伯使一人,女侍二人,皆選端正妖麗,執香爐,護衣服,奏事明光殿。殿以胡粉塗壁,畫古賢烈士。以丹朱色地,謂之丹墀。



 尚書郎口含雞舌香,以其奏事答對,欲使氣息芬芳也。奏事則與黃門侍郎對揖。黃門侍郎稱已聞,乃出。天子所服五時衣以賜尚書令僕,而丞、
 郎月賜赤管大筆一雙,隃麋墨一丸。魏世有殿中、吏部、駕部、金部、虞曹、比部、南主客、祠部、度支、庫部、農部、水部、儀曹、三公、倉部、民曹、二千石、中兵、外兵、別兵、都兵、考功、定科,凡二十三郎。青龍二年有軍事,尚書令陳矯奏置都官、騎兵二曹郎,合為二十五曹。晉西朝則直事、殿中、祠部、儀曹、吏部、三公、比部、金部、倉部、度支、都官、二千石、左民、右民、虞曹、屯田、起部、水部、左主客、右主客、駕部、車部、庫部、左中兵、右中兵、左外兵、右外兵、別兵、都兵、騎兵、
 左士右士、北主客、南主客為三十四曹郎;後又置運曹,凡三十五曹。晉江左初,無直事、右民、屯田、車部、別兵、都兵、騎兵、左士、右士、運曹十曹郎,而主客、中外兵各置一郎而已,所餘十七曹也。康、穆以來,又無虞曹、二千石二郎,猶有殿中、祠部、吏部、儀曹、三公、比部、金部、倉部、度支、都官、左民、起部、水部、主客、駕部、庫部、中兵、外兵十八曹郎。後又省主客、起部、水部,餘十五曹。宋高祖初,加置騎兵、主客、起部、水部四曹郎,合為十九曹。太祖元嘉十年,又省儀曹、主客、
 比部、騎兵四曹郎。十一年,又並置。十八年,增刪定曹郎,次在左民曹上,蓋魏世之定科郎也。三十年,又置功論郎,次都官之下,在刪定之上。太宗世,省騎兵。今凡二十曹郎。以三公、比部主法制。度支主算。支,派也;度,景也。都官主軍事刑獄。其餘曹所掌,各如其名。



 漢制,公卿御史中丞以下,遇尚書令、僕、丞、郎,皆闢車豫相回避,臺官過,乃得去。今尚書官上朝及下,禁斷行人,猶其制也。漢又制,丞、郎見尚書,呼曰明時。郎見二丞,呼曰左君、右君。郎以
 下則有都令史、令史、書令史、書吏乾。



 漢東京尚書令史十八人,晉初正令史百二十人,書令史百三十人。自晉至今,或減或益,難以定言。《漢儀》有丞相令史。令史,蓋前漢官也。晉西朝有尚書都令史硃誕,則都令史其來久矣。分曹所掌如尚書也。



 晉西朝八坐丞郎,朝晡詣都坐朝,江左唯旦朝而已。八坐丞郎初拜,並集都坐,交禮。遷,又解交。漢舊制也。今唯八坐解交,丞郎不復解交也。尚書令千石,僕射尚書六百石,丞郎四百石。



 武庫令,一人。掌軍器,秦官。至二漢,屬執金吾。晉初罷執金吾,至今隸尚書庫部。



 車府令,一人。丞一人。秦官也。二漢、魏、晉並隸太僕。太僕既省,隸尚書駕部。



 上林令,一人。丞一人。漢西京上林中有八丞、十二尉、十池監。丞、尉屬水衡都尉。池監隸少府。漢東京曰上林苑令及丞各一人,隸少府。晉江左闕。宋世祖大明三年復置,隸尚書殿中曹及少府。



 材官將軍,一人。司馬一人。主工匠土木之事。漢左右校令,其任也。魏右校又置材官校尉,主天下材木事。晉江左改材官校尉曰材官將軍,又罷左校令。今材官隸尚書起部及領軍。



 侍中,四人。掌奏事,直侍左右,應對獻替。法駕出,則正直一人負璽陪乘。



 殿內門下眾事皆掌之。周公戒成王《立政》之篇所云「常伯」,即其任也。侍中本秦丞相史也,使五人往來殿內東廂奏事,故謂之侍中。漢西京無員,多至
 數十人,入侍禁中,分掌乘輿服物,下至褻器虎子之屬。武帝世,孔安國為侍中,以其儒者,特聽掌御唾壺,朝廷榮之。久次者為僕射。漢東京又屬少府,猶無員。掌侍左右,贊導眾事,顧問應答。法駕出,則多識者一人負傳國璽,操斬白蛇劍,參乘;餘皆騎,在乘輿車後。光武世,改僕射為祭酒焉。漢世,與中官俱止禁中。武帝時,侍中莽何羅挾刃謀逆,由是侍中出禁外,有事乃入,事畢即出。王莽秉政,侍中復入,與中官共止。章帝元和中,侍中郭舉
 與後宮通,拔佩刀驚御,舉伏誅,侍中由是復出外。魏、晉以來,置四人,別加官不主數。秩比二千石。



\end{pinyinscope}