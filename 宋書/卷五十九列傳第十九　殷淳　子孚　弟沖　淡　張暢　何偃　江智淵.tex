\article{卷五十九列傳第十九 殷淳 子孚 弟沖 淡 張暢 何偃 江智淵}

\begin{pinyinscope}

 殷淳,字粹
 遠,陳郡長平人也。曾祖融,祖允,並晉太常。父穆,以和謹致稱,歷顯官,自五兵尚書為高祖相國左長
 史。及受禪,轉散騎常侍,國子祭酒,復為五兵尚書,吳郡太守。太祖即位,為金紫光祿大夫,領竟陵王師,遷護軍,又遷特進、右光祿大夫,領始興王師。元嘉十五年卒官,時年六十,謚曰元子。



 淳少好學,有美名。少帝景平初,為祕書郎,衡陽王文學,祕書丞,中書黃門侍郎。淳居黃門為清切,下直應留下省,以父老特聽還家。高簡寡慾,早有清尚,愛好文義,未嘗違捨。在秘書閣撰《四部書目》凡四十卷,行於世。元嘉十一年卒,時年三十二,朝廷痛惜
 之。



 子孚,有父風。世祖大明末,為始興相。官至尚書吏部郎,順帝撫軍長史。



 淳弟沖,字希遠,歷中書黃門郎,坐議事不當免。復為太子中庶子,尚書吏部郎,御史中丞,有司直之稱。出為吳興太守,入為度支尚書。元凶妃即淳女,而沖在東宮為劭所知遇;劭弒立,以為侍中、護軍,遷司隸校尉。沖有學義文辭,劭使為尚書符,罪狀世祖,亦為劭盡力。世祖剋京邑,賜死。



 沖弟淡,字夷遠,亦歷黃門吏部郎,太子中庶子,領步兵校尉。大明世,以文章見知,
 為當時才士。



 張暢,字少微,吳郡吳人,吳興太守邵兄子也。父禕,少有孝行,歷宦州府,為瑯邪王國郎中令。從琅邪王至洛。還京都,高祖封藥酒一罌付禕,使密加鴆毒。



 禕受命,既還,於道自飲而卒。



 暢少與從兄敷、演、敬齊名,為後進之秀。起家為太守徐佩之主簿,佩之被誅,暢馳出奔赴,制服盡哀,為論者所美。弟牧嘗為猘犬所傷,醫云宜食蝦蟆膾,牧甚難之,暢含笑先嘗,牧因此乃食,創亦即愈。州辟
 從事,衡陽王義季征虜行參軍,彭城王義康平北主簿,司徒祭酒,尚書主客郎。未拜,又除度支左民郎,江夏王義恭征北記室參軍、晉安太守。又為義季安西記室參軍、南義陽太守,臨川王義慶衛軍從事中郎,揚州治中別駕從事史,太子中庶子。



 世祖鎮彭城,暢為安北長史、沛郡太守。元嘉二十七年,索虜托跋燾南侵,太尉江夏王義恭總統諸軍,出鎮彭、泗。時燾親率大眾,已至蕭城,去彭城十數里。



 彭城眾力雖多,而軍食不足,義恭欲棄
 彭城南歸,計議彌日不定。時歷城眾少食多,安北中兵參軍沈慶之建議,欲以車營為函箱陣,精兵為外翼。奉二王及妃媛直趨歷城;分兵配護軍蕭思話留守。太尉長史何勖不同,欲席卷奔鬱洲,自海道還京都。



 義恭去意已判,唯二議未決,更集群僚謀之。眾咸惶擾,莫有異議。暢曰:「若歷城、鬱洲有可致之理,下官敢不高贊。今城內乏食,百姓咸有走情,但以關扃嚴固,欲去莫從耳。若一旦動腳,則各自散走,欲至所在,何由可得。今軍食雖
 寡,朝夕猶未窘罄,量其欲盡,臨時更為諸宜,豈有舍萬安之術,而就危亡之道。若此計必用,下官請以頸血汗公馬蹄!」世祖既聞暢議,謂義恭曰:「阿父既為總統,去留非所敢干。道民忝為城主,而損威延寇,其為愧恧,亦已深矣。委鎮奔逃,實無顏復奉朝廷,期與此城共其存沒,張長史言不可異也。」暢言既堅,世祖又贊成其議,義恭乃止。



 時太祖遣員外散騎侍郎徐爰乘驛至彭城取米穀定最,爰既去,城內遣騎送之。



 燾聞知,即遣數百騎急
 追,爰已過淮,僅得免。初爰去,城內聞虜遣追,慮爰見禽,失米最,慮知城內食少,義恭憂懼無計,猶欲奔走。爰既免,其日虜大眾亦至彭城。



 燾始至,仍登城南亞父冢,於戲馬臺立氈屋。先是,燾未至,世祖遣將馬文恭向蕭城,為虜所破,文恭走得免,隊主蒯應見執。至小市門曰:「魏主致意安北,遠來疲乏,若有甘蔗及酒,可見分。」時防城隊主梁法念答曰:「當為啟聞。」應乃自陳蕭城之敗。又問應:「虜主自來不?」曰:「來。」問:「今何在?」應舉手指西南。又曰:「士
 馬多少?」答云:「四十餘萬。」法念以燾語白世祖,世祖遣人答曰:「知行路多乏,今付酒二器,甘蔗百挺。聞彼有駱駝,可遣送。」



 明旦,燾又自上戲馬臺,復遣使至小市門曰:「魏主致意安北,安北可暫出門,欲與安北相見。我亦不攻此城,安北何勞苦將士在城上。又騾、驢、駱駝,是北國所出,今遣送,并致雜物。」又語小市門隊主曰:「既有餉物,君可移度南門受之。」



 燾送駱駝、騾、馬及貂裘、雜飲食,既至南門,門先閉,請龠未出。暢於城上視之,虜使問:「是張長
 史邪?」暢曰:「君何得見識?」虜使答云:「君聲名遠聞,足使我知。」暢因問虜使姓,答云:「我是鮮卑,無姓。且道亦不可。」暢又問:「君居何任?」答云:「鮮卑官位不同,不可輒道,然亦足與君相敵耳。」虜使復問:「何為匆匆杜門絕橋?」暢答曰:「二王以魏主營壘未立,將士疲勞,此精甲十萬,人思致命,恐輕相凌踐,故且閉城耳。待彼休息士馬,然後共治戰場,剋日交戲。」虜使曰:「君當以法令裁物,何用發橋,復何足以十萬誇人。我亦有良馬逸足,若雲騎四集,亦可以
 相拒。」暢曰:「侯王設嶮,何但法令而已邪。我若誇君,當言百萬。所以言十萬者,政二王左右素所畜養者耳。此城內有數州士庶,二徒營伍,猶所未論。我本鬥智,不鬥馬足。且冀之北土,馬之所生,君復何以逸足見誇邪!」虜使曰:「不爾。城守,君之所長;野戰,我之所長。我之恃馬,猶如君之恃城耳。」城內有具思者,嘗在北國,義恭遣視之,思識是虜尚書李孝伯。思因問:「李尚書,若行途有勞。」孝伯曰:「此事應相與共知。」思答:「緣共知,所以有勞。」孝伯曰:「感
 君至意。」



 既開門,暢屏卻人仗,出對孝伯,并進餉物。虜使云:「貂裘與太尉,駱駝、騾與安北,蒲陶酒雜飲,叔侄共嘗。」燾又乞酒并甘橘。暢宣世祖問:「致意魏主,知欲相見,常遲面寫。但受命本朝,過蒙籓任,人臣無境外之交,恨不暫悉。且城守備防,邊鎮之常,但悅以使之,故勞而無怨耳。太尉、鎮軍得所送物,魏主意,知復須甘橘,今並付如別。太尉以北土寒鄉,皮褲褶脫是所須,今致魏主。螺杯、雜粽,南土所珍,鎮軍今以相致。」此信未去,燾復遣使令
 孝伯傳語曰:「魏主有詔語太尉、安北,近以騎至,車兩在後,今端坐無為,有博具可見借。」暢曰:「博具當為申啟。但向語二王,已非遜辭,且有詔之言,政可施於彼國,何得稱之於此。」孝伯曰:「詔之與語,朕之與我,並有何異。」暢曰:「若辭以通,可如來談;既言有所施,則貴賤有等。向所稱詔,非所敢聞。」孝伯又曰:「太尉、安北是人臣與非?」暢曰:「是也。」孝伯曰:「鄰國之君,何為不稱詔於鄰國之臣?」



 暢曰:「君之此稱,尚不可聞於中華,況在諸王之貴,而猶曰鄰國
 之君邪。」孝伯曰:「魏主言太尉、鎮軍並皆年少,分闊南信,殊當憂邑。若欲遣信者,當為護送;脫須騎者,亦當以馬送之。」暢曰:「此方間路甚多,使命日夕往來,不復以此勞魏主。」孝伯曰:「亦知有水路,似為白賊所斷。」暢曰:「君著白衣,故稱白賊邪?」孝伯大笑曰:「今之白賊,亦不異黃巾、赤眉。」暢曰:「黃巾、赤眉,似不在江南。」孝伯曰:「雖不在江南,亦不在青、徐也。」暢曰:「今者青、徐,實為有賊,但非白賊耳。」虜使云:「向借博具,何故不出?」暢曰:「二王貴遠,啟聞難徹。」孝
 伯曰:「周公握髮吐哺,二王何獨貴遠?」暢曰:「握髮吐飡,本施中國耳。」孝伯曰:「賓有禮,主則擇之。」暢曰:「昨見眾賓至門,未為有禮。」



 俄頃送博具出,因以與之。



 燾又遣人云:「魏主致意安北,程天祚一介常人,誠知非宋朝之美,近於汝陽身被九創,落在殿外,我手牽而出之。凡人骨肉分張,並思集聚,輒已語之,但其弟苦辭。今令與來使相見。」程天福謂使人曰:「兄受命汝陽,不能死節,各在一國,何煩相見。」燾又送氈各一領,鹽各九種,並胡豉:「凡此諸鹽,
 各有所宜。



 白鹽是魏主自所食。黑鹽治腹脹氣懣,細刮取六銖,以酒服之。胡鹽治目痛。柔鹽不食,治馬脊創。赤鹽、駁鹽、臭鹽、馬齒鹽四種,並不中食。胡豉亦中啖。黃甘幸彼所豐,可更見分。」又云:「魏主致意太尉、安北,何不遣人來至我間。彼此之情,雖不可盡,要須見我小大,知我老少,觀我為人。若諸佐不可遣,亦可使僮幹來。」暢又宣旨答曰:「魏主形狀才力,久為來往所見。李尚書親自銜命,不患彼此不盡,故不復遺使信。」又云:「魏主恨向所送
 馬,殊不稱意。安北若須大馬,當更送之,脫須蜀馬,亦有佳者。」暢曰:「安北不乏良駟,送自彼意,非此所求。」



 義恭餉燾炬燭十挺,世祖亦致錦一匹,曰:「知更須黃甘,誠非所吝。但送不足周彼一軍,向給魏主,未應便乏,故不復重付。」燾復求甘蔗、安石榴,暢曰:「石榴出自鄴下,亦當非彼所乏。」孝伯又曰:「君南土膏粱,何為著屩。君而著此,使將士云何?」暢曰:「膏粱之言,誠為多愧。但以不武,受命統軍,戎陣之間,不容緩服。」孝伯又曰:「長史,我是中州人,久處
 北國,自隔華風,相去步武,不得致盡,邊皆是北人聽我語者,長史當深得我。」孝伯又曰:「永昌王,魏主從弟,自復常鎮長安,今領精騎八萬,直造淮南,壽春久閉門自固,不敢相禦。向送劉康祖頭,彼之所見。王玄謨甚是所悉,亦是常才耳。南國何意作如此任使,以致奔敗。自入此境七百餘里,主人竟不能一相拒逆。鄒山之險,君家所憑,前鋒始得接手,崔邪利便藏入穴,我間諸將倒曳腳而出之,魏主賜其生命,今從在此。復何以輕脫遣馬文
 恭至蕭縣,使望風退撓邪。君家民人甚相忿怨,云清平之時,賦我租帛,至有急難,不能相拯。」暢曰:「知永昌已過淮南,康祖為其所破,比有信使,無此消息。王玄謨南土偏將,不謂為才,但以人為前驅引導耳。大軍未至而河冰向合,玄謨量宜反旆,未為失機,但因夜回師,致戎馬小亂耳。我家玄謨斗城,陳憲小將,魏主傾國,累旬不剋。胡盛之偏裨小帥,眾無一旅,始濟融水,魏國君臣奔迸,僅得免脫,滑臺之師,無所多愧。鄒山小戍,雖有微險,河
 畔之民,多是新附,始慕聖化,姦盜未息,亦使崔邪利撫之而已,今沒虜手,何損於國。魏主自以十萬師而制一崔邪利,方復足言邪。聞蕭、相百姓,並依山險,聊遣馬文恭以十隊示之耳。文恭謂前以三隊出,還走後,大營嵇玄敬以百騎至留城,魏軍奔敗。輕敵致此,亦非所衄。王境人民,列居河畔,二國交兵,當互加撫養,而魏師入境,肆行殘虐,事生意外,由彼無道。官不負民,民何怨人。知入境土,百無相拒,此自上由太尉神算,次在鎮軍聖略。
 經國之要,雖不豫聞,然用兵有機,間亦不容相語。」孝伯曰:「魏主當不圍此城,自率眾軍,直造瓜步。南事若辦,彭城不待圍;若不捷,彭城亦非所須也。我今當南飲江湖,以療渴耳。」暢曰:「去留之事,自適彼懷。



 若虜馬遂得飲江,便為無復天道。各應反命,遲復更悉。」暢便回還,孝伯追曰:「長史深自愛敬,相去步武,恨不執手。」暢因復謂曰:「善將愛,冀蕩定有期,相見無遠。君若得還宋朝,今為相識之始。」孝伯曰:「待此未期。」燾又遣就二王借箜篌、琵琶、箏、
 笛等器及棋子,義恭答曰:「受任戎行,不齎樂具。在此燕會,政使鎮府命妓,有弦百條,是江南之美,今以相致。」世祖曰:「任居方岳,初不此經慮,且樂人常器,又觀前來諸王贈別,有此琵琶,今以相與。棋子亦付。」



 孝伯言辭辯贍,亦北土之美也。暢隨宜應答,吐屬如流,音韻詳雅,風儀華潤,孝伯及左右人並相視歎息。



 虜尋攻彭城南門,并放火,暢躬自前戰,身先士卒。及燾自瓜步北走,經彭城下過,遣人語城內:「食盡且去,須麥熟更來。」義恭大懼,閉
 門不敢追。虜期又至,議欲芟麥剪苗,移民堡聚,眾論並不同,復更會議。鎮軍錄事參軍王孝孫獨曰:「虜不能復來,既自可保,如其更至,此議亦不可立。百姓閉在內城,饑饉日久,方春之月,野採自資,一入堡聚,餓死立至。民知必死,何可制邪?虜若必來,芟麥無晚。」四坐默然,莫之敢對。暢曰:「孝孫之議,實有可尋。」鎮軍府典簽董元嗣侍世祖側,進曰:「王錄事議不可奪,實如來論。」別駕王子夏因曰:「此論誠然。」暢斂板白世祖曰:「下官欲命孝孫彈子
 夏。」世祖曰:「王別駕有何事邪?」



 暢曰:「芟麥移民,可謂大議,一方安危,事係於此。子夏親為州端,曾無同異,及聞元嗣之言,則懽笑酬答,阿意左右,何以事君。」子夏大慚,元嗣亦有慚色。



 義恭之議遂寢。太祖聞暢屢有正議,甚嘉之。世祖猶停彭城,召暢先反,并使履行盱眙城,欲立大鎮。時虜聲云當出襄陽,故以暢為南譙王義宣司空長史、南郡太守。



 又欲暢代劉興祖為青州及彭城都督,並不果。



 三十年,元凶弒逆,義宣發哀之日,即便舉兵,暢為
 元佐,居僚首,哀容俯仰,蔭映當時。舉哀畢,改服,著黃韋褲褶,出射堂簡人,音姿容止,莫不矚目,見之者皆願為盡命。事平,徵為吏部尚書,夷道縣侯,食邑千戶。義宣既有異圖,蔡超等以暢民望,勸義宣留之,乃解南蠻校尉以授暢,加冠軍將軍,領丞相長史。暢遣門生荀僧寶下都,因顏竣陳義宣釁狀。僧寶有私貨停巴陵,不時下,會義宣起兵,津徑斷絕,僧寶遂不得去。義宣將為逆,遣嬖人翟靈寶謂暢:「朝廷簡練舟甲,意在西討,今欲發兵自
 衛。」暢曰:「必無此理,請以死保之。」靈寶知暢不回,勸義宣殺以徇眾。即遣召暢,止于東齋,彌日不與相見,賴司馬竺超民保持,故獲全免。既而進號撫軍,別立軍部,以收民望。暢雖署文檄,而飲酒常醉,不省文書。



 隨義宣東下,梁山戰敗,義宣奔走,暢於兵亂自歸,為軍人所掠,衣服都盡。值右將軍王玄謨乘輿出營,暢已得敗衣,排玄謨上輿,玄謨意甚不悅,諸將欲殺之,隊主張世營救得免。送京師,下廷尉,削爵土,配左右尚方。尋見原。復起為都
 官尚書,轉侍中,代子淹領太子右衛率。



 孝建二年,出為會稽太守。大明元年,卒官,時年五十。顏竣表世祖:「張暢遂不救疾。東南之秀,蚤樹風範,聞問悽愴,深切常懷。」謚曰宣子。暢愛弟子輯,臨終遺命與輯合墳。



 子浩,官至義陽王昶征北諮議參軍。浩弟淹,世祖南中郎主簿。世祖即立,為黃門郎,封廣晉縣子,食邑五百戶。太子右衛率,東陽太守。逼郡吏燒臂照佛,民有罪使禮佛,動至數千拜。免官禁錮。起為光祿勳,臨川內史。太宗泰始初,與晉
 安王子勛同逆,率眾至鄱陽,軍敗見殺。



 暢弟悅,亦有美稱。歷中書吏部郎,侍中,臨海王子頊前軍長史、南郡太守。



 晉安王子勛建偽號於尋陽,召為吏部尚書,與鄧琬共輔偽政。事敗,殺琬歸降,事在《琬傳》。復為太子庶子,仍除巴陵王休若衛軍長史、襄陽太守。四年,即代休若為雍州刺史、寧遠將軍。復為休若征西長史、南郡太守。六年,太宗於巴郡置三巴校尉,以悅補之,加持節、輔師將軍,領巴郡太守。未拜,卒。



 何偃,字仲弘,廬江灊人,司空尚之中子也。州辟議曹從事,舉秀才,除中軍參軍,臨川王義慶平西府主簿。召為太子洗馬,不拜。元嘉十九年,為丹陽丞,除廬陵王友,太子中舍人,中書郎,太子中庶子。時義陽王昶任東官,使偃行義陽國事。



 二十九年,太祖欲更北伐,訪之群臣,偃議曰:「內幹胡法宗宣詔,逮問北伐。



 伏計賊審有殘禍,犬羊易亂,殲殄非難,誠如天旨。今雖廟算無遺,而士未精習。



 緣邊鎮戍,充實者寡,邊民流散,多未附業。控引所資,取
 給根本。虧根本以殉邊患,宜動必萬剋。無慮往歲挫傷,續以內釁,侮亡取亂,誠為沛然。然淮、泗數州,實亦彫耗,流傭未歸,創痍未起。且攻守不等,客主形異,薄之則勢艱,圍之則曠日,進退之間,姦虞互起。竊謂當今之弊易衄,方來之寇不深,宜含垢藏疾,以齊天道。」遷始興王濬征北長史、南東海太守。



 元凶弒立,以偃為侍中,掌詔誥。時尚之為司空、尚書令,偃居門下,父子並處權要,時為寒心;而尚之及偃善攝機宜,曲得時譽。會世祖即位,
 任遇無改,除大司馬長史,遷侍中,領太子中庶子。時責百官讜言,偃以為:「宜重農恤本,并官省事,考課以知能否,增俸以除吏姦。責成良守,久於其職。都督刺史,宜別其任。」



 改領驍騎將軍,親遇隆密,有加舊臣。轉吏部尚書。尚之去選未五載,偃復襲其迹,世以為榮。侍中顏竣至是始貴,與偃俱在門下,以文義賞會,相得甚歡。竣自謂任遇隆密,宜居重大,而位次與偃等未殊,意稍不悅。及偃代竣領選,竣愈憤懣,與偃遂有隙。竣時勢傾朝野,偃
 不自安,遂發心悸病,意慮乖僻,上表解職,告醫不仕。世祖遇偃既深,備加治療,名醫上藥,隨所宜須,乃得瘥。時上長女山陰公主愛傾一時,配偃子戢。素好談玄,注《莊子·消搖篇》傳於世。



 大明二年,卒官,時年四十六。世祖與顏竣詔曰:「何偃遂成異世,美志長往。



 與之周旋,重以姻媾,臨哭傷怨,良不能已。往矣如何!宜贈散騎常侍、金紫光祿大夫,本官如故。」謚曰靖子。子戢,昇明末,為相國左長史。



 江智淵,濟陽考城人,湘州刺史夷弟子。父僧安,太子中庶子。智淵初為著作郎,江夏王義恭太尉行參軍,太子太傅主簿,隨王誕後軍參軍。世父夷有盛名,夷子湛又有清譽,父子並貴達,智淵父少無名問,湛禮敬甚簡,智淵常以為恨,自非節歲,不入湛門。及為隨王誕佐,在襄陽,誕待之甚厚。時諮議參軍謝莊、府主簿沈懷文並與智淵友善。懷文每稱之曰:「人所應有盡有,人所應無盡無者,其江智淵乎!」元嘉末,除尚書庫部郎。時高流官序,
 不為臺郎,智淵門孤援寡,獨有此選,意甚不說,固辭不肯拜。竟陵王誕復版為驃騎參軍,轉主簿,隨府轉司空主簿、記室參軍,領南濮陽太守,遷從事中郎。誕將為逆,智淵悟其機,請假先反。誕事發,即除中書侍郎。



 智淵愛好文雅,詞采清贍,世祖深相知待,恩禮冠朝。上燕私甚數,多命群臣五三人游集,智淵常為其首。同侶末及前,輒獨蒙引進,智淵每以越眾為慚,未嘗有喜色。每從游幸,與群僚相隨,見傳詔馳來,知當呼己,聳動愧恧,形於容貌,
 論者以此多之。



 遷驍騎將軍,尚書吏部郎。上每酣宴,輒詬辱群臣,并使自相嘲訐,以為歡笑。



 智淵素方退,漸不會旨。嘗使以王僧朗嘲戲其子景文,智淵正色曰:「恐不宜有此戲。」上怒曰:「江僧安癡人,癡人自相惜。」智淵伏席流涕,由此恩寵大衰,出為新安王子鸞北中郎長史、南東海太守,加拜寧朔將軍,行南徐州事。初,上寵姬宣貴妃殷氏卒,使群臣議謚,智淵上議曰「懷」。上以不盡嘉號,甚銜之。後車駕幸南山,乘馬至殷氏墓,群臣皆騎從,上
 以馬鞭指墓石柱謂智淵曰:「此上不容有懷字!」智淵益惶懼。大明七年,以憂卒,時年四十六。



 子季筠,太子洗馬,早卒。後廢帝即位,以后父,追贈金紫光祿大夫。季筠妻王,平望鄉君。



 智淵兄子概,早孤,養之如子。概歷黃門吏部郎,侍中,武陵王北中郎長史、南東海太守,行南徐州事。後廢帝元徽中,卒。



 史臣曰:夫將帥者,御眾之名;士卒者,一夫之用。坐談兵機,制勝千里,安在乎蒙楯前驅,履腸涉血而已哉!山濤
 之稱羊祜曰:「大將雖不須筋力,軍中猶宜強健。」以此為言,則叔子之幹力弱矣。杜預文士儒生,身不能穿札,射未嘗跨馬,一朝統大眾二十餘萬,為平吳都督。王戎把臂入林,亦受專征之寄。何必山西猛士,六郡良家,然後可受脤於朝堂,荷推轂之重。及虜兵深入,徐服心匡震,非張暢正言,則彭、汴危矣。豈其身捍飛鏑,手折雲沖,方足使窮堞假命,危城載安乎?仁者之有勇,非為臆說。



\end{pinyinscope}