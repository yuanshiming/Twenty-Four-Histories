\article{卷六十八列傳第二十八 武二王}

\begin{pinyinscope}

 彭城王義康南郡王義宣彭城王義康,年
 十二,
 宋臺除督豫、司、雍、並四州諸軍事、冠軍將軍、豫州刺史。時高祖自壽陽被徵入輔,留義康代鎮壽陽。又領司州刺史,進督徐州之鐘離、荊州之義陽諸軍事。永初元年,封彭城王,食邑三千戶,進號右將軍。二年,徙監南豫、豫、司、雍、並五州諸軍事、南豫州刺史,
 將軍如故。三年,遷使持節、都督南徐、兗二州揚州之晉陵諸軍事、南徐州刺史,將軍如故。



 太祖即位,增邑二千戶,進號驃騎將軍,加散騎常侍,給鼓吹一部。尋加開府儀同三司。元嘉三年,改授都督荊、湘、雍、梁、益、寧、南北秦八州諸軍事、荊州刺史,給班劍三十人,持節、常侍、將軍如故。義康少而聰察,及居方任,職事修理。六年,司徒王
 弘表義康宜還入輔,徵侍中、都
 督揚、南徐、兗三州諸軍事、
 司徒、
 錄尚書事,領平北將軍、南徐州刺史,持節如故。二府並置佐領兵,與王弘共輔朝政。弘既多疾,且每事推謙,自是內外眾務,一斷之義康。太子詹事劉湛有經國才,義康昔在豫州,湛為長史,既素經情款,至是意委特隆,人物雅俗,舉動事宜,莫不咨訪之。故前後在籓,多有
 善政,為遠近所稱。九年,弘薨,又領揚州刺史。其年,太妃薨,解侍中,辭班劍。十二年,又領太子太傅,復加侍中、班劍。



 義康性好吏職,銳意文案,糾剔是非,莫不精盡。既專總朝權,事決自己,生殺大事,以錄命斷之。凡所陳奏,入無不可,方伯以下,並委義康授用,由是朝野輻湊,勢傾天下。義康亦自強不息,無有懈倦。府門每旦常有數百乘車,雖復位卑人微,皆被引接。又聰識過人,一聞必記,常所暫遇,終生不忘,稠人廣席,每標所憶以示聰明,人
 物益以此推服之。愛惜官爵,未嘗以階級私人,凡朝士有才用者,皆引入己府,無施及忤旨,即度為臺官。自下樂為竭力,不敢欺負。太祖有虛勞疾,寢頓積年,每意有所想,便覺心中痛裂,屬纊者相係。義康醫藥,盡心衛奉,湯藥飲食,非口所嘗不進;或連夕不寐,彌日不解衣;內外眾事,皆專決施行。十六年,進位大將軍,領司徒,辟召掾屬。



 義康素無術學,闇於大體,自謂兄弟至親,不復存君臣形跡,率心徑行,曾無猜防。私置僮部六千餘人,不以
 言臺。四方獻饋,皆以上品薦義康,而以次者供御。



 上嘗冬月啖甘,歎其形味並劣,義康在坐曰:「今年甘殊有佳者。」遣人還東府取甘,大供御者三寸。尚書僕射殷景仁為太祖所寵,與太子詹事劉湛素善,而意好晚衰。湛常欲因宰輔之權以傾之,景仁為太祖所保持,義康屢言不見用,湛愈僨。南陽劉斌,湛之宗也,有涉俗才用,為義康所知,自司徒右長史擢為左長史。從事中郎琅邪王履、主簿沛郡劉敬文、祭酒魯郡孔胤秀,並以傾側自入,
 見太祖疾篤,皆謂宜立長君。上疾嘗危殆,使義康具顧命詔。義康還省,流涕以告湛及殷景仁,湛曰:「天下艱難,詎是幼主所御。」義康、景仁並不答,而胤秀等輒就尚書議曹索晉咸康末立康帝舊事,義康不知也。及太祖疾豫,微聞之。而斌等既為義康所寵,又威權盡在宰相,常欲傾移朝廷,使神器有歸。遂結為朋黨,伺察省禁,若有盡忠奉國,不與己同志者,必構造愆釁,加以罪黜。每採拾景仁短長,或虛造異同以告湛。自是主相之勢分,內外
 之難結矣。



 義康欲以斌為丹陽尹,言次啟太祖,陳其家貧。上覺其旨,義康言未卒,上曰:「以為吳郡。」後會稽太守羊玄保求還,義康又欲以斌代之,又啟太祖曰:「羊玄保欲還,不審以誰為會稽?」上時未有所屬,倉卒曰:「我已用王鴻。」自十六年秋,不復幸東府。上以嫌隙既成,將致大禍。十七年十月,乃收劉湛付廷尉,伏誅。



 又誅斌及大將軍錄事參軍劉敬文、賊曹參軍孔邵秀、中兵參軍邢懷明、主簿孔胤秀、丹陽丞孔文秀、司空從事中郎司馬亮、
 烏程令盛曇泰等。徙尚書庫部郎何默子、餘姚令韓景之、永興令顏遙之、湛弟黃門侍郎素、斌弟給事中溫於廣州,王履廢於家。



 胤秀始以書記見任,漸預機密,文秀、邵秀,皆其兄也。司馬亮,孔氏中表,並由胤秀而進。懷明、曇泰為義康所遇。默子、景之、遙之,劉湛黨也。



 其日刺義康入宿,留止中書省,其夕分收湛等。青州刺史杜驥勒兵殿內,以備非常。遣人宣旨告以湛等罪釁,義康上表遜位曰:「臣幼荷國靈,爵遇逾等。陛下推恩睦親,以隆棠棣,
 愛忘其鄙,寵授遂崇,任總內外,位兼台輔。不能正身率下,以肅庶僚,暱近失所,漸不自覺,致令毀譽違實,賞罰謬加,由臣才弱任重,以及傾撓。今雖罪人即戮,王猷載靜,養釁貽垢,實由於臣。鞠躬慄悚,若墮谿壑,有何心顏,而安斯寵,輒解所職,待罪私第。」改授都督江州諸軍事、江州刺史,持節、侍中、將軍如故,出鎮豫章。



 停省十餘日,桂陽侯義融、新喻侯義宗、祕書監徐湛之往來慰視。於省奉辭,便下渚。上唯對之慟哭,餘無所言。上又遣沙門
 釋慧琳視之,義康曰:「弟子有還理不?」慧琳曰:「恨公不讀數百卷書。」征虜司馬蕭斌,昔為義康所暱,劉斌等害其寵,讒斥之。乃以斌為諮議參軍,領豫章太守,事無大小,皆以委之。司徒主簿謝綜,素為義康所狎,以為記室參軍,左右愛念者,並聽隨從至豫章。辭州,見許,增督廣、交二州、湘州之始興諸軍事。資奉優厚,信賜相係,朝廷大事,皆報示之。義康未敗,東府聽事前井水忽湧溢,野雉江鷗並飛入所住齋前。



 龍驤參軍巴東扶令育詣闕上
 表曰:蓋聞哲王不逆切旨之諫,以博聞為道;人臣不忌殲夷之罰,以盡言為忠。是故周昌極諫,馮唐面折,孝惠所以克固儲嗣,魏尚所以復任雲中。彼二臣豈好逆主乾時,犯顏違色者哉!又爰盎之諫孝文曰:「淮南王若道遇疾死,則陛下有殺弟之名,奈何?」文帝不用,追悔無及。臣草莽微臣,竊不自揆,敢抱葵藿傾陽之心,仰慕《周易》匪躬之志,故不遠六千里,願言命侶,謹貢丹愚,希垂察納。



 伏惟陛下躬執大象,首出萬物,王化咸通,三才必理,闢
 天人之路,開大道之門,搜殊逸于巖穴,招奇英於側陋,窮谷無白駒之倡,喬岳無遺寶之嗟,豈特羅飛翮于垂天,網沈鱗於溟海。況於彭城王義康,先朝之愛子,陛下之次弟哉!一旦黜削,遠送南服,恩絕於內,形隔於遠,躬離明主,身放聖世,草萊黔首,皆為陛下痛之。



 臣追惟景平、元嘉之釁,幾於危殆,三公託以興廢之宜,密懷不臣之計,台輔伺隙於京甸,強楚窺窬於上流,或瑩惡而窺國或顯逆而陵主,有生之所惴恐,神只之所忿忌也。賴宗社靈長,廟算
 流遠,灑滌塵埃,殲馘醜類,氛霧時靖,四門載清。



 當爾之時,義康豈不預參皇謀,均此休否哉。且陛下舊楚形勝,非親勿居,遂以驃騎之號,任以籓夏之重,撫政南郢,綏民遏寇,播皇宋之澤,以洽幽荒。陛下之潤,被之九有,豈直南荊之民沾渥而已焉。遂召之以宰輔,又寄之以和味,既居三事,又牧徐、揚,所以幽顯齊歡,人神同忭。莫不言陛下授之為得,義康受之為是也。



 今如何信疑貌之似,闕兄弟之恩乎?若有迷謬之愆,可責之罪,正可數之
 以善惡,導之以義方。且廬陵王往事,足以知今,此乃陛下前車之殷鑒,後乘之靈龜也。夫曾子之不殺,忠臣之篤譬;二告而猶織,仁王之令範。故《詩》云「無信人之言,人實不信」。又云兄弟雖鬩,不廢親也。《尚書》曰:「克明俊德,以親九族。」



 九族既睦,可以親百姓,兄弟安可棄乎!



 臣伏願陛下上尋往代黜廢之禍,下惟近者讒言之釁。廬陵王既申冤魂於后土,彭城王亦弭疑愆於宋京,豈徒皇代當今之計,蓋乃良史萬代之美也。且諂諛難辨,是非易
 黷,福始禍先,古人所畏。故愛身之士,自為己計,莫不結舌杜口,孰肯冒忌乾主哉!臣以頑昧,獨獻微管,所以勤勤懇懇,必訴丹誠者,實恐義康年窮命盡,奄忽于南,遂令陛下有棄弟之責。臣雖微賤,竊為陛下羞之。況書言記事,史豈能屈典謨而諱哉。脫如臣慮,陛下恨之何益。揚子雲曰:「獲福之大,莫先於和穆;遘禍之深,莫過於內難。」每服斯言,以為警戒。矧今睹王室大事,豈得韜筆默爾而已哉。臣將恐天下風靡,離間是懼,遂令宇內遷觀,
 民庶革心,欲致康哉,實為難也。



 陛下徒云惡枝之宜伐,豈悟伐柯之傷樹,乃往古之所悲,當今所宜改也。陛下若蕩以平聽,屏此猜情,垂訊芻蕘之謀,曲察狂瞽之計,一發非意之詔,逮訪博古之士,速召義康返於京甸,兄弟協和,君臣緝穆,息宇內之譏,絕多言之路,如是則四海之望塞,讒說之道消矣。何必司徒公、揚州牧,然後可以安彭城王哉!若臣所啟違憲,於國為非,請即伏誅,以謝陛下。雖復分形赴鑊,煮體烹屍,始願所甘,豈不幸甚!



 表奏,即收付建康獄,賜死。



 會稽長公主,於兄弟為長,太祖至所親敬。義康南上後,久之,上嘗就主宴集甚懽,主起再拜稽顙,悲不自勝。上不曉其意,自起扶之。主曰:「車子歲暮,必不為陛下所容,今特請其生命。」因慟哭。上流涕,舉手指蔣山曰:「必無此慮。



 若違今誓,便負初寧陵。」即封所飲酒賜義康,並書曰:「會稽姊飲宴憶弟,所餘酒今封送。」車子,義康小字也。



 二十二年,太子詹事范曄等謀反,事逮義康,事在《曄傳》。有司上曰:「義康昔擅國權,恣心
 凌上,結朋樹黨,苞納凶邪。重釁彰著,事合明罰。特遭陛下仁愛深至,敦惜周親,封社不削,爵寵無貶。四海之心,朝野之議,咸謂皇德雖厚,實撓典刑。而義康曾不思此大造之德,自出南服,詭飾情貌,外示知懼,內實不悛。



 窮好極欲,干請無度。聖慈含弘,每不折舊,矜釋屢加,恩疇已往。而陰敦行李,方啟交通之謀,潛資左右,以要死士之命。崎嶇伺隙,不忘窺窬。時猶隱忍,罰止僕侍。狂疾之性,永不懲革,兇心遂成,悖謀仍構。遠投群醜,千里相結,
 再議宗社,重窺鼎祚。賴陛下至誠感神,宋歷方永,故姦事昭露,罪人斯得。周公上聖,不辭同氣之刑;漢文仁明,無隱從兄之惡。況義康釁深二叔,謀過淮南,背親反道,自棄天地。臣等參議,請下有司削義康王爵,收付廷尉法獄治罪。」詔特宥大辟。



 於是免義康及子泉陵侯允、女始寧、豐城、益陽、興平四縣主為庶人,絕屬籍,徙付安成郡。以寧朔將軍沈邵為安成公相,領兵防守。義康在安成讀書,見淮南厲王長事,廢書歎曰:「前代乃有此,我得
 罪為宜也。」



 二十四年,豫章胡誕世、前吳平令袁惲等謀反,襲殺豫章太守桓隆、南昌令諸葛智之,聚眾據郡,復欲奉戴義康。太尉錄尚書江夏王義恭等奏曰:「投畀之言,義著《雅》篇,流殛之教,事在《書》典。庶人義康負釁深重,罪不容戮。聖仁不忍,屢加遲回,宥其大辟,賜遷近甸,斯乃至愛發天,超邈終古。曾不遇愆甘引,而讒言同眾,佷悖徼幸,每形辭色,內宣家人,外動民聽,不逞之族,因以生心。



 胡誕世假竊名號,構成凶逆。杜漸除微,古今所務,
 況禍機驟發,庸可忽乎!臣等參議,宜徙廣州遠郡,放之邊表,庶有防絕。」奏可,仍以安成公相沈邵為廣州事。



 未行,值邵病卒,索虜來寇瓜步,天下擾動。上慮異志者或奉義康為亂,世祖時鎮彭城,累啟宜為之所,太子及尚書左僕射何尚之並以為言。二十八年正月,遣中書舍人嚴龍齎藥賜死。義康不肯服藥,曰:「佛教自殺不復得人身,便隨宜見處分。」



 乃以被掩殺之,時年四十三,以侯禮葬安成。



 六子:允、肱、珣、昭、方、曇辯。允初封泉陵縣侯,食
 邑七百戶。昭、方并早夭。允等留安成,元凶得志,遣殺之。



 世祖大明四年,義康女玉秀等露板辭曰:「父凶滅無狀,孤負天明,存荷優養,沒蒙加禮,明罰羽山,未足敕法。烏鳥微心,昧死上訴,乞反葬舊塋,糜骨鄉壤。」



 詔聽,并加資給。前廢帝永光元年,太宰江夏王義恭表曰:「臣聞忝祖遠支,猶或慮親,降霍省序,義重令戚。故嚴道疾終,嗣啟方宇,阜陵愆屏,身膋晚恩。竊惟故庶人劉義康昔昧姦回,自貽非命,沈魂漏籍,垂誡來典。運革三朝,歲盈三紀,
 天地改朔,日月再升,陶形賦氣,咸蒙更始。義康妻息漂沒,早違盛化,眾女孤弱,永淪黔首。即情原釁,本非己招,感事哀煢,俯增傷咽。敢緣陛下聖化融泰,春澤覃被,慈育群生,仁被泉草。實希洗宥,還齒帝宗,則施及陳荄,榮施朽壤。臣特憑國私,冒以誠表,塵觸靈威,伏紙悲悸。」詔曰:「太宰表如此,公緣情追遠,覽以憎慨。昔淮、楚推恩,胙流支胤,抑法弘親,古今成準。使以公表付外,依旨奉行。故泉陵侯允橫罹凶虐,可特為置後。」太宗泰始四年,復
 絕屬籍,還為庶人。



 南郡王義宣,生而舌短,澀於言論。元嘉元年,年十二,封竟陵王,食邑五千戶。仍拜右將軍,鎮石頭。七年,遷使持節、都督徐、兗、青、冀、幽五州諸軍事、徐州刺史,將軍如故。猶戍石頭。八年,又改都督南兗、兗州刺史,當鎮山陽,未行。明年,遷中書監,進號中軍將軍,加散騎常侍,給鼓吹一部。時竟陵群蠻充斥,役刻民散,改封南譙王,又領石頭戍事。十三年,出都督江州、豫州之西陽、晉熙、新蔡三
 郡諸軍事、鎮南將軍、江州刺史。



 初,高祖以荊州上流形勝,地廣兵彊,遺詔諸子次第居之。謝晦平後,以授彭城王義康。義康入相,次江夏王義恭。又以臨川王義慶宗室令望,且臨川武烈王有大功於社稷,義慶又居之。其後應在義宣。上以義宣人才素短,不堪居上流。十六年,以衡陽王義季代義慶,而以義宣代義季為南徐州刺史,都督南徐州軍事、征北將軍,持節如故。加散騎常侍。而會稽公主每以為言,上遲回久之。二十一年,乃以義
 宣都督荊、雍、益、梁、寧、南北秦七州諸軍事、車騎將軍、荊州刺史,持節、常侍如故。先賜中詔曰:「師護以在西久,比表求還,出內左右,自是經國常理,亦何必其應於一往。今欲聽許,以汝代之。護雖無殊績,潔己節用,通懷期物,不恣群下。此信未易,非唯聲著西土,朝野以為美談。在彼已有次第,為士庶所安,論者乃謂未議遷之,今之回換,更在欲為汝耳。汝與護年時一輩,各有其美,物議亦互有少劣。若今向事脫一減之者,既於西夏交有巨礙,
 遷代之譏,必歸責於吾矣。



 復當為護怨,非但一誚而已也。如此則公私俱損,為不可不先共善詳。此事亦易勉耳,無為使人動生評論也。」師護,義季小字也。



 義宣至鎮,勤自課厲,政事修理。白皙,美須眉,長七尺五寸,腰帶十圍,多畜嬪媵,後房千餘,尼媼數百,男女三十人。崇飾綺麗,費用殷廣。進位司空,改侍中,領南蠻校尉。二十七年,索虜南侵,義宣慮寇至,欲奔上明。及虜退,太祖詔之曰:「善修民務,不須營潛逃計也。」



 三十年,遷司徒、中軍將軍、
 揚州刺史,侍中如故。未及就徵,值元凶弒立,以義宣為中書監、太尉,領司徒、侍中如故。義宣聞之,即時起兵,徵聚甲卒,傳檄近遠。會世祖入討,義宣遣參軍徐遺寶率眾三千,助為前鋒。世祖即位,以義宣為中書監,都督揚、豫二州、刺史,加羽葆、鼓吹,給班劍四十人,持節、侍中如故。改封南郡王,食邑萬戶。進謚義宣所生為獻太妃,封次子宜陽侯愷為南譙王,食邑千戶。義宣固辭內任,及愷王爵。於是改授都督荊、湘、雍、益、梁、寧、南北秦八州諸
 軍事、荊、湘二州刺史,持節、侍中、丞相如故。降愷為宜陽縣王。義宣將佐以下,並加賞秩。長史張暢,事在本傳。諮議參軍蔡超專掌書記並參謀,除尚書吏部郎,仍為丞相諮議參軍、南郡內史,封汝南縣侯,食邑千戶。司馬竺超民為黃門侍郎,仍除丞相司馬、南平內史。其餘各有差。



 義宣在鎮十年,兵彊財富,既首創大義,威名著天下,凡所求欲,無不必從。



 朝廷所下制度,意所不同者,一不遵承。嘗獻世祖酒,先自酌飲,封送所餘,其不識大體如
 此。初,臧質陰有異志,以義宣凡弱,易可傾移,欲假手為亂,以成其姦。



 自襄陽往江陵見義宣,便盡禮,事在《質傳》。及至江州,每密信說義宣,以為「有大才,負大功,挾震主之威,自古鮮有全者,宜在人前,蚤有處分。且萬姓莫不係心於公,整眾入朝,內外孰不欣戴。不爾,一旦受禍,悔無所及。」義宣陰納質言。而世祖閨庭無禮,與義宣諸女淫亂,義宣因此發怒,密治舟甲,克孝建元年秋冬舉兵。報豫州刺史魯爽、兗州刺史徐遺寶使同。爽狂酒失旨,
 其年正月便反。



 遣府戶曹送版,以義宣補天子,并送天子羽儀;遺寶亦勒兵向彭城。義宣及質狼狽起兵。二月二十六日,加都督中外諸軍事,置左右長史、司馬,使僚佐悉稱名。遣傳奉表曰:臣聞博陸毗漢,獲疑宣後;昌國翼燕,見猜惠王。常謂異姓震主,嫌隙易構;葭莩淳戚,昭亮可期。臣雖庸懦,少希忠謹。值巨逆滔天,忘家殉國,雖歷算有歸,微績不樹,竭誠盡愚,貫之幽顯。而微疑莫監,積毀日聞;投杼之聲,紛紜溢聽。



 諒緣奸臣交亂,成是貝
 錦。夫澆俗之季,少貞節之臣;冰霜競至,靡後彫之木。並寢處凶世,甘榮偽朝,皆纓冕之所棄,投畀之所取。至乃位超昔寵,任參大政,惡直醜勛,妄生邪說,疑惑明主,誣罔視聽。又南從郡僚,勞不足紀,橫叨天功,以為己力,同弊相扇,圖傾宗社。臧質去歲忠節,勳高古賢;魯爽協同大義,志契金石,此等猜毀,必欲禍陷。昔汲黯尚存,劉安寢志;孔父既逝,華督縱逆。臣雖不武,績著艱難,復肆讒狡,規見誘召。宗祀之危,綴旒非所。



 臣託體皇基,連暉日
 月,王室顛墜,咎在微躬,敢忘抵鼠之忌,甘受犯墉之責。



 輒徵召甲卒,分命眾籓,使忠勤申憤,義夫效力,戮此凶醜,謝愆闕廷,則進不負七廟之靈,退無愧二朝之遇。臨表感愧,辭不自宣。



 上詔答曰:皇帝敬問。朕以不天,招罹屯難,家國阽危,剪焉將及。所以身先八百,雪清冤恥,遠憑高算,共濟艱難。遂登寡暗,嗣奉洪祀,尊戚酬勛,實表心事,粃政闕職,所願匡拯。而嘉言蔑聞,末德先著,勤王之績未終,毀冕之圖已及。臧質嶮躁無行,見棄人倫,以
 此不識,志在問鼎,凶意將逞,先借附從,扇誘欺熾,成此亂階。如使群逆並濟,眾邪競逐,將恐瞻烏之命,未識所止,構怨連禍,孰知其極。



 公明有不照,背本崇姦,迷暱讒丑,還謀社稷,雖履霜有日,喧議糾紛。朕以至道無私,杜遏疑議,信理推誠,暴於遐邇。不虞物變難籌,醜言遂驗,是用悼心失圖,忽忘寢食。



 今便親御六師,廣命群牧,告靈誓眾,直造柴桑,梟轘元惡,以謝天下。然後警蹕清江,鳴鑾郢路,投戈襲袞,面稟規勖。有宋不造,家禍仍纏,昔
 歲事寧,方承遠訓,冀以虛薄,永弭厥艱。豈謂曾未期稔,復睹斯釁,二祖之業,將墜于淵,仰瞻鴻基,但深感慟。



 太傅江夏王義恭又與義宣書曰:頃聞之道路云,二魯背叛,致之有由,謂不然之言,絕於智者之耳。忽見來表,將興晉陽之甲,驚愕駭惋,未譬所由。若主幼臣強,政移塚宰,或時昏下縱,在上畏逼,然後賢籓忠構,睹難赴機。未聞聖主御世,百辟順軌,稱兵於言興之初,扶危於既安之日。以此取濟,竊為大弟憂之。



 昔歲二凶構逆,四海同
 奮。弟協宣忠孝,奉戴明主,元功盛德,既已昭著;皇朝欽嘉,又亦優渥。丞相位極人臣,江左罕授,一門兩王,舉世希有。表倍推誠,彰於見事,出納之宜,唯意所欲。裒升進益,方省後命,一旦棄之,可謂運也。



 吾等荷先帝慈育,得及人群,思報厚恩,昊天罔極,竭力盡誠,猶懼無補。奈何妄聽邪說,輕造禍難。國靡流言,遽歸愆於二叔;世無晁錯,仍襲轍於七籓。棄漢蒼之令範,遵齊冏之敗跡。



 往時仲堪假兵靈寶,旋害其族;孝伯授之劉牢,忠誠逝踵。皆
 曩代之成事,當今之殷鑒也。臧質少無美行,弟所具悉,憑恃末戚,並有微勤,承乏推遷,遂超倫伍,藉西楚彊力,圖濟其私。凶謀若果,恐非復池中物。魯宗父子,世為國冤,太祖方弘遐略,故爽等均雍齒之封。令據有五州,虎兕出於匣,是須為劉淵耳。徐遺寶是垣護之婦弟,前因護之歸於吾,苦求北出,不樂遠西。近磐桓湖陸,示遣劉雍,其意見可。雍是徐沖舅,適有密信,誓倒戈。自虜侵境以來,公私彫弊,安以撫之,庶可寧靜,弟復隨而擾亂,吾
 恐邊鄙皆為禾黍。宜遠尋高祖創業艱難,近念家國比者禍釁,時息兵戈,共安社稷。責躬謝過,誅除險佞,追保前勛,傳美竹帛。昔梁孝悔罪,景帝垂恩,阜、質改過,肅宗降澤。忠焉之誨,聊希往言;禍福之機,明者是察。



 主上神武英斷,群策如林,忠臣發憤,虎士投袂,雄騎布野,舳艫蓋川。吾以不才,忝權節鉞,總督群帥,首戒戎先,指晨電舉,式清南服。所以積行緩期,冀弟不遠而悟。如其遂溺姦說者,天實為之。臨書慨懣,不識次第。



 義宣移檄諸州
 郡,加進號位。遣參軍劉諶之、尹周之等率軍下就臧質。雍州刺史硃修之起兵奉順。義宣二月十一日率眾十萬發自江津,舳艫數百里。是日大風,船垂覆沒,僅得入中夏口。以第八子慆為輔國將軍,留鎮江陵。遣魯秀、硃曇韶萬餘人北討朱修之。秀初至江陵,見義宣,既出,拊膺曰:「阿兄誤人事,乃與癡人共作賊,今年敗矣!」義宣至尋陽,與質俱下,質為前鋒。至鵲頭,聞徐遺寶敗,魯爽於小峴授首,相視失色。世祖使鎮北大將軍沈慶之送爽
 首示義宣,並與書:「僕荷任一方,而釁生所統。近聊率輕師,指往翦撲,軍鋒裁交,賊爽授首。公情契異常,或欲相見,及其可識,指送相呈。」義宣、質並駭懼。



 上先遣豫州刺史王玄謨舟師頓梁山洲內,東西兩岸為卻月城,營柵甚固。義宣屢與玄謨書,要令降。玄謨書報曰:頻奉二誨,伏對戰駭。先在彭、泗,聞諸將皆云必有今日之事,以鄙意量,謂無此理。去年九月,故遣參軍先僧瑗修書表心,并密陳入相之計,欲使周旦之美,復見於今。豈意理數
 難推,果至於此。昔因幸會,蒙國士之顧,思報厚德,甘起泉壤,豈謂一旦事與願違。公崇長姦回,自放西服,信邪細之說,忘大節之重,溺流狡之志,滅君親之恩,狎玩極寵,越希非覬,祖宗世祀,自圖顛覆,瞑目行事,未有如斯之甚者也,乃復枉覃書檄,遠示見招。此則丹心微款,未亮於高鑒,赤誠幽志,虛感於平日,環念周回,始悟知己之為難也。



 公但念提職在昔,不思善教有本,徒見徐、魯去就,未知仗義有人,豈不惜哉!



 有臣則欲其忠,誘人而
 導諸逆,君子忠恕,其如是乎?茍不忠恕,則擇木之翰,有所不集矣。夫挑妾者愛其易,求妻則敬其難。若承命如響,將焉用之。原轂存輿,無禮必及,竊恐荊郢之士,已當潛貳其懷,非皇都陋臣,秉義不徙。公雖心迷跡往,猶願勉建良圖。抑撫軍忠壯慷慨,亮誠有素,新亭之勳,莫與為等,而妄信姦虛,坐相貶謗,不亦惑哉!



 幸承人乏,夙誡前驅,精甲已次近路;鎮軍駱驛繼發,太傅、驃騎嗣董元戎;乘輿親御六師,威靈遐振。人百其氣,慕義如林,舟騎
 雲回,赫弈千里。輒屬鞬秉銳,與執事周旋,授命當仁,理無所讓。夫君道既盡,民禮亦絕,執筆裁答,感慨交懷。



 撫軍柳元景據姑孰為大統,偏帥鄭琨、武念戍南浦。質徑入梁山,去玄謨一里許結營,義宣屯蕪湖。五月十九日,西南風猛,質乘風順流攻玄謨西壘,冗從僕射胡子友等戰失利,棄壘渡就玄謨。質又遣將龐法起數千兵從洲外趨南浦,仍使自後掩玄謨。與琨、念相遇,法起戰大敗,赴水死略盡。二十一日,義宣至梁山,質上出軍東岸
 攻玄謨。玄謨分遣游擊將軍垣護之、竟陵太守薛安都等出壘奮擊,大敗質軍,軍人一時投水。護之等因風縱火,焚其舟乘,風勢猛盛,煙焰覆江。義宣時屯西岸,延火燒營殆盡。諸將乘風火之勢,縱兵攻之,眾一時奔潰。



 義宣與質相失,各單舸迸走,東人士庶並歸順,西人與義宣相隨者,船舸猶有百餘。女先適臧質子,過尋陽,入城取女,載以西奔。至江夏,聞巴陵有軍,被抄斷,回入徑口,步向江陵。眾散且盡,左右唯十許人,腳痛不復能行,就
 民僦露車自載。無復食,緣道求告。至江陵郭外,遣人報竺超民,超民具羽儀兵眾迎之。時外猶自如舊,帶甲尚萬餘人。義宣既入城,仍出聽事見客,左右翟靈寶誡使撫慰眾賓,以「臧質違指授之宜,用致失利,今治兵繕甲,更為後圖;昔漢高百敗,終成大業」。而義宣忘靈寶之言,誤云「項羽千敗」,眾咸掩口而笑。魯秀、竺超民等猶為之爪牙,欲收合餘燼,更圖一決,而義宣惛墊無復神守,入內不復出。左右腹心,相率奔叛。魯秀北走,義宣不復自
 立,欲隨秀去,乃於內戎服,幰囊盛糧,帶佩刀,攜息慆及所愛妾五人,皆著男子服相隨。城內擾亂,白刃交橫,義宣大懼落馬,仍便步地,超民送城外,更以馬與之,超民因還守城。義宣冀及秀,望諸將送北入虜。即失秀所在,未出郭,將士逃散盡,唯餘慆及五妾兩黃門而已。夜還向城,入南郡空廨,無床,席地至旦。遣黃門報超民,超民遣故車一乘,載送刺姦。義宣送止獄戶,坐地歎曰:「臧質老奴誤我。」始與五妾俱入獄,五妾尋被遣出,義宣號泣
 語獄吏曰:「常日非苦,今日分別始是苦。」



 大司馬江夏王義恭諸公王八座與荊州刺史朱修之書曰:「義宣反道叛恩,自陷極逆。大義滅親,古今同準。無將之誅,猶或囚殺,況醜文悖志,宣灼遐邇,鋒指絳闕,兵纏近郊,釁逼憂深,臣主旰食。賴朝略震明,祖宗靈慶,罪人斯得,七廟弗隳。司刑定罰,典辟攸在。而皇慈逮下,愍其愚迷,抑法申情,屢奏不省,人神悚遑,省心震惕。義宣自絕於天,理無容受。社稷之慮,臣子責深。便宜專行大戮,以紓國難。但
 加諸斧鉞,有傷聖仁,示以弘恩,使自為所,上全天德,下一洪憲。



 臨書悲慨,不復多云。」書未達,修之至江陵,已於獄盡焉。時年四十。世祖聽還葬。



 義宣子悽、愷、恢、憬、惔、心矣、惇、慆、伯實、業、悉達、法導、僧喜、慧正、慧知、明彌虜、妙覺、寶明凡十八人;愷、恢、惔、惇並於江寧墓所賜死,心矣、悉達早卒,餘並與義宣俱為硃脩之所殺。蔡超及諮議參軍顏樂之、徐壽之等諸同惡,並伏誅。超,濟陽考城人。父茂之,侍廬陵王義真讀書,官至彭城王義康驃騎從事中郎,
 始興太守。超少有才學,初為兗州主簿,時令百官舉才,超與前始寧令同郡江淳之、前征南參軍會稽賀道養並為興安侯義賓所表薦。竺超民,青州刺史竺夔子也。



 恢,字景度,既嫡長,少而辯慧,義宣甚愛重之。年十一,拜南譙王世子,除給事中。義宣為荊州,常停都邑。太祖欲令還西,乃以為河東太守,加寧朔將軍。



 頃之,徵為黃門侍郎。元兇弒立,恢為侍中。義宣起義,劭收恢及弟愷、心炎、心妻、憬、心矣繫于外,散騎郎沈煥防守之。煥密有歸順意,謂
 恢等曰:「禍福與諸郎同之,願勿憂。」及臧質自白下上趨廣莫門,劭令煥殺恢等。煥乃解其桎梏,率所領數十人與恢等向廣莫門欲出。門者拒之,煥曰:「臧公已至,凶人走矣。此司空諸郎,並能為諸君得富貴,非徒免禍而已,勿相留。」亦值質至,因以得出。恢至新亭,即除侍中。俄遷侍中、散騎常侍、西中郎將、湘州刺史。義宣並領湘州,轉恢侍中,領衛尉。晉氏過江,不置城門校尉及衛尉官,世祖欲重城禁,故復置衛尉卿。衛尉之置,自恢始也。轉右
 衛將軍,侍中如故。義宣舉兵反,恢與兄弟姊妹一時逃亡。恢藏江寧民陳銑家,有告之者,錄付廷尉。恢子善藏,與恢俱死。



 愷,字景穆,生而養於宮內,寵均皇子。十歲,封宜陽縣侯。仍為建威將軍、南彭城、沛二郡太守。遷步兵校尉,轉黃門侍郎,太子中庶子,領長水校尉。元凶以愷為散騎常侍。世祖以為秘書監。未拜,遷輔國將軍、南彭城、下邳二郡太守。



 其年,轉五兵尚書,進爵為王。義宣反問至,愷於尚書寺內,著婦人衣,乘問訊車,投臨汝公蓋
 詡。詡於妻室內為地窟藏之,事覺,收付廷尉,詡伏誅。心矣封臨武縣侯,年十八卒,謚曰悼侯。悽封湘南縣侯。憬封祁陽縣侯。



 徐遺寶,字石俊,高平金鄉人。初以新亭戰功,為輔國將軍、衛軍司馬、河東太守,不之官。遷兗州刺史,將軍如故,戍湖陸。封益陽縣侯,食邑二千五百戶。



 義宣既叛,遣使以遺寶為征虜將軍、徐州刺史,率軍出瓜步。遺寶遣長史劉雍之襲彭城,寧朔司馬明胤擊破之。更遣高平太守王玄楷與雍之復逼彭城。時徐州刺史蕭
 思話未之鎮,因詔安北司馬夏侯祖權率五百人馳往助胤,既至,擊玄楷斬之,雍之還湖陸。遺寶復遣使人檀休祖應玄楷,聞敗,亦潰散。遺寶棄城奔魯爽,爽敗,逃東海郡界,土人斬送之,傳首京邑。



 夏侯祖權,譙人也。以功封祁陽縣子,食邑四百戶。大明中,為建武將軍、兗州刺史,卒官。謚曰烈子。



 史臣曰:襄陽龐公謂劉表曰:「若使周公與管、蔡處茅屋之下,食藜藿之羹,豈有若斯之難。」夫天倫由子,共氣分
 形,寵愛之分雖同,富貴之情則異也。追味尚長之言,以為太息。



\end{pinyinscope}