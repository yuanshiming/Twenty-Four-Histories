\article{卷六十七列傳第二十七 謝靈運}

\begin{pinyinscope}

 謝靈運,陳郡陽夏人也。祖玄,晉車騎將軍。父瑍,生而不慧,為秘書郎,蚤亡。靈運幼便穎悟,玄甚異之,謂親知曰:「我乃生瑍,瑍那得生靈運!」



 靈運少好學,博覽群書,文章
 之美,江左莫逮。從叔混特知愛之,襲封康樂公,食邑三千戶。以國公例,除員外散騎侍郎,不就。為瑯邪王大司馬行參軍。性奢豪,車服鮮麗,衣裳器物,多改舊制,世共宗之,咸稱謝康樂也。撫軍將軍劉毅鎮姑孰,以為記室參軍。毅鎮江陵,又以為衛軍從事中郎。毅伏誅,高祖版為太尉參軍,入為祕書丞,坐事免。



 高祖伐長安,驃騎將軍道憐居守,版為咨議參軍,轉中書侍郎,又為世子中軍咨議,黃門侍郎。奉使慰勞高祖於彭城,作《撰征賦》。其
 序曰:蓋聞昏明殊位,貞晦異道,雖景度回革,亂多治寡,是故升平難於恒運,剝喪易以橫流。皇晉囗囗河汾,來遷吳楚,數歷九世,年踰十紀,西秦無一援之望,東周有三辱之憤,可謂積禍纏釁,固以久矣。況乃陵塋幽翳,情敬莫遂,日月推薄,帝心彌遠。慶靈將升,時來不爽,相國宋公,得一居貞,回乾運軸,內匡寰表,外清遐陬。每以區宇未統,側席盈慮。值天祚攸興,昧弱授機,龜筮元謀,符瑞景征。



 於是仰祗俯協,順天從兆,興止戈之師,躬暫勞
 之討。以義熙十有二年五月丁酉,敬戒九伐,申命六軍,治兵于京畿,次師于汳上。靈檣千艘,雷輜萬乘,羽騎盈途,飛旍蔽日。別命群帥,誨謨惠策,法奇於《三略》,義祕於《六韜》。所以鉤棘未曜,殞前禽於金墉,威弧始彀,走鈒隼於滑臺。曾不踰月,二方獻捷。宏功懋德,獨絕古今。天子感《東山》之劬勞,慶格天之光大,明發興於鑒寐,使臣遵于原隰。



 余攝官承乏,謬充殊役,《皇華》愧於先《雅》,靡盬顇於征人。以仲冬就行,分春反命。塗經九守,路踰千里。沿
 江亂淮,溯薄泗、汳,詳觀城邑,周覽丘墳,眷言古跡,其懷已多。昔皇祖作蕃,受命淮、徐,道固苞桑,勳由仁積。年月多歷,市朝已改,永為洪業,纏懷清歷。於是采訪故老,尋履往跡,而遠感深慨,痛心殞涕。遂寫集聞見,作賦《撰征》,俾事運遷謝,託此不朽。其詞曰:系烈山之洪緒,承火正之明光。立熙載於唐后,申贊事於周王。疇庸命而順位,錫寶圭以徹疆。歷尚代而平顯,降中葉以繁昌。業服道而德徽,風行世而化揚。投前蹤
 以永冀,省輶質以遠傷。睽謀始於蓍蔡,違用舍於行藏。



 庇常善之罔棄,憑曲成之不遺。昭在幽而偕煦,賞彌久而愈私。顧晚草之薄弱,仰青春之葳蕤。引蔓穎於松上,擢纖枝於蘭逵。施隆貸而有渥,報涓塵而無期。歡太階之休明,穆皇道之緝熙。



 惟王建國,辨方定隅,內外既正,華夷有殊。惟昔《小雅》,逮于班書,戎蠻孔熾,是殛是誅。所以宣王用棘於獫狁,高帝方事於匈奴。然侵鎬至涇,自塞及平。



 窺郊伺鄙,囗囗囗囗慕攜王之矯虔,階喪亂之
 未寧。竊彊秦之三輔,陷隆周之兩京。



 雄崤、澠以制險,據繞霤而作扃。家永懷於故壤,國願言於先塋。俟太平之曠期,屬應運之聖明。坤寄通於四瀆,乾假照於三辰。水潤土以顯比,火炎天而同人。惟上相之睿哲,當草昧而經綸。總九流以貞觀,協五才而平分。時來之機,悟先於介石,納隍之誡,一援於生民。龜筮允臧,人鬼同情。順天行誅,司典詳刑。樹牙選徒,秉鉞抗旍。弧矢罄楚孝之心智,戈棘單吳子之精靈。



 迅三翼以魚麗,襄兩服以鴈逝。
 陣未列於都甸,威已振於秦、薊。灑嚴霜於渭城,被和風於洛汭。就終古以比猷,考墳冊而莫契。昔西怨於東徂,今北伐而南悲。



 豈朝野之恆情,動萬乘之幽思。歌零雨於《豳風》,興《採薇》於周詩。慶金墉之凱定,眷戎車之遷時。佇千里而感遠,涉弦望而懷期。詔微臣以勞問,奉王命於河湄。夕飲餞以俶裝,旦出宿而言辭。歲既晏而繁慮,日將邁而戀乖。闕敬恭於桑梓,謝履長於庭階。冒沈雲之晻藹,迎素雪之紛霏。凌結湍而凝清,風矜籟以揚哀。
 情在本而易阜,物雖末而難懷。眷餘勤以就路,苦憂來其城頹。



 爾乃經雉門,啟浮梁,眺鐘巖,越查塘。覽永嘉之紊維,尋建武之緝綱。于時內慢神器,外侮戎狄。君子橫流,庶萌分析。主晉有祀,福祿來格。明兩降覽,三七辭厄。元誕德以膺緯,肇回光於陽宅。明思服於下武,興繼代以消逆。簡文因心以秉道,故沖用而刑廢。孝武舍己以杖賢,亦寧外而治內。觀日化而就損,庶雍熙之可對。閔隆安之致寇,傷龜玉之毀碎。漏妖凶於滄洲,纏釁難而
 盈紀。時焉依於晉、鄭,國有蹙於百里。賴英謨之經營,弘兼濟以忘己。主寰內而緩虞,澄海外以漬滓。至如昏祲蔽景,鼎祚傾基。《黍離》有歎,《鴻鴈》無期。瞻天命之貞符,秉順動而履機。率駿民之思效,普邦國而同歸。蕩積霾之穢氛,啟披陰之光暉。反平陵之杳藹,復七廟之依稀。務役簡而農勸,每勞賞而忠甄。燮時雍於祖宗,囗囗囗囗囗囗。掃逋醜於漢渚,滌僭逆於岷山。羈巢處於西木,引鼻飲於源淵。惠要襋而思韙,援冠弁而來虔。



 視冶城而北
 屬,懷文獻之收揚。匪元首之康哉,執股肱之惟良。譬觀曲而識節,似綴組以成章。業彌纏而彌微,事愈有而莫傷。次石頭之雙岸,究孫氏之初基。幸漢庶之漏網,憑江介以抗維。初鵲起於富春,果鯨躍於川湄。匝三世而國盛,歷五偽而宗夷。察成敗之相仍,猶脣亡而齒寒。載十二而謂紀,豈蜀滅而吳安。眾咸昧於謀兆,羊獨悟於理端。請廣武以誨情,樹襄陽以作蕃。拾建業其如遺,沿萬里而誰難。疾魯荒之詖辭,惡京陵之譖言。責當朝之憚
 貶,對曩籍而興歎。



 敦怙寵而判違,敵既勍而國圮。彼問鼎而何階,必先賊於君子。原性分之異託,雖殊塗而歸美。或卷舒以愚智,或治亂其如矢。謝昧迹而託規,卒安身以全里。周顯節而犯逆,抱正情而喪己。



 薄四望而尤眄,歎王路之中鯁。蠢于越之妖燼,敢凌蹈於五嶺。崩雙嶽於中流,擬凶威於荊郢。隱雷霆於帝坐,飛芒鏃於宮省。于時朝有遷都之議,人無守死之志。



 師旅痛於久勤,城墉闕於素備。安危勢在不侔,眾寡形於見事。於赫淵
 謀,研其神策。緩轡待機,追奔躡迹。遇雷池而振曜,次彭蠡而殲滌。穆京甸以清晏,撤多壘而寧役。



 造白石之祠壇,懟二豎之無君。踐掖庭以幽辱,凌祧社而火焚。愍文康之罪己,嘉忠武之立勳。道有屈於災蝕,功無謝於如仁。訊落星之饗旅,索舊棲於吳餘。迹階戺而不見,橫榛卉以荒除。彼生成之樂辰,亦猶今之在餘。慨齊吟於爽鳩,悲唐歌於《山樞》。



 弔偽孫於徐首,率君臣以奉疆。時運師以伐罪,偏投書於武王。迄西北之落紐,乏東南以振
 綱。誠鉅平之先覺,實中興之後祥。據左史之攸征,胡影跡之可量。過江乘而責始,知遇雄之無謀。厭紫微之宏凱,甘陵波而遠遊。越雲夢而南溯,臨浙河而東浮。彀連弩於川上,候蛟龍於中流。



 爰薄方與,乃屆歐陽。入夫江都之域,次乎廣陵之鄉。易千里之曼曼,溯江流之湯湯。洊赤圻以經復,越二門而起漲。眷北路以興思,看東山而怡目。林叢薄,路逶迤,石參差,山盤曲。水激瀨而駿奔,日映石而知旭。審兼照之無偏,怨歸流之難濯。羨輕魵
 之涵泳,觀翔鷗之落啄。在飛沈其順從,顧微躬而緬邈。



 於是抑懷蕩慮,揚搉易難。利涉以吉,天險以艱。于敵伊阻,在國期便。勾踐行霸於瑯邪,夫差爭長於黃川。葛相發歎而思正,曹后愧心於千魂。登高堞以詳覽,知吳濞之衰盛。戒東南之逆氣,成劉后之駴聖。藉鹽鐵之殷阜,臨淮楚之剽輕。盛几杖而弭心,怒抵局而遂爭。忿爰盎之扶禍,惜徒傷於家令。匪條侯之忠毅,將七國之陵正。褒漢籓之治民,並訪賢以招明。侯文辨其誰在,曰鄒陽
 與枚生。據忠辭於吳朝,執義說於梁庭。敷高才於兔園,雖正言而免刑。闕里既已千載,深儒流於末學。欽仲舒之睟容,遵縫掖於前躅。對園囿而不窺,下帷幕而論屬。相端、非之兩驕,遭弘、偃之雙慝。恨有道之無時,步險塗以側足。



 聞宣武之大閱,反師旅於此廛。自皇運之都東,始昌業以濟難。抗素旄於秦嶺,揚朱旗於巴川。懼帝系之墜緒,故黜昏而崇賢。嘉收功以垂世,嗟在嗣而覆趫。德非陟而繼宰,釁踰禹其必顛。



 造步丘而長想,欽太傅
 之遺武。思嘉遁之餘風,紹素履之落緒。民志應而願稅,國屯難而思撫。譬乘舟之待楫,象提釣之假縷。總出入於和就,兼仁用於默語。弘九流以拮四維,復先陵而清舊宇。卻西州之成功,指東山之歸予。惜圖南之啟運,恨鵬翼之未舉。



 發津潭而迥邁,逗白馬以憩舲。貫射陽而望邗溝,濟通淮而薄甬城。城坡陀兮淮驚波,平原遠兮路交過。面艽野兮悲橋梓,溯急流兮苦磧沙。夐千里而無山,緬百谷而有居。被宿莽以迷徑,睹生煙而知墟。囗囗
 囗囗囗囗,謂信美其可娛。身少長於樂土,實長歎於荒餘。囗囗囗囗具瘁,值歲寒之窮節。視層雲之崔巍,聆悲飆之掩屑。彌晝夜以滯淫,怨凝陰之方結。望新晴於落日,起明光於躋月。眷轉蓬之辭根,悼朔鴈之赴越。披微物而疚情,此思心其可悅。問徭役其幾時,駭閱景於興沒。感曰歸於《采薇》,予來思於雨雪。豈初征之懼對,冀鸛鳴之在垤。



 囗囗囗囗踰宿,騖吾楫於邳鄉。奚車正以事夏,虺左相以輔湯。綿三代而享邑,廁踐土之一匡。嗟仲幾之
 寵侮,遂舍存以徵亡。喜薛宰之善對,美士彌之能綱。升曲垣之逶迤,訪淮陰之所都。原入跨之達恥,俟遭時以遠圖。舍西楚以擇木,迨南漢以定謨。亂孟津而魏滅,攀井陘而趙徂。播靈威於齊橫,振餘猛於龍且。觀讓通而告犬希,曷始智而終愚。



 迄沂上而停枻,登高圯而不進。石幽期而知賢,張揣景而示信。本文成之素心,要王子於雲仞。豈無累於清霄,直有概於貞吝。始熙績於武關,卒敷功於皇胤。處夷險以解挫,弘憂虞以時順。矜若華之
 翳晷,哀飛驂之落駿。傷粒食而興念,眷逸翮而思振。



 戾臣山而東顧,美相公之前代。嗟殘虜之將糜,熾餘猋於海濟。驅鮐稚於淮曲,暴鰥孤於泗澨。託末命囗囗雲,冀靈武之北閱。惟授首之在晨,當盛暑而選徒。肅嚴威以振響,漸溫澤而沾腴。既雲撤於朐城,遂席卷於齊都。曩四關其奚阻,道一變而是孚。



 傷炎季之崩弛,長逆布以滔天。假父子以詐愛,借兄弟以偽恩。相魏武以譎狂,宄謨奮於東籓。桴未噪於東郭,身已馘於樓門。審貢牧
 於前說,證所作於舊徐。聆泗川之浮磬,玩夷水之蠙珠。草漸苞於熾壤,桐孤干於嶧隅。慨禹迹於尚世,惠遺文於《夏書》。



 紛征邁之淹留,彌懷古於舊章。商伯文於故服,咸徵名於彭、殤。眺靈壁之曾峰,投呂縣之迅梁。想蹈水之行歌,雖齊汩其何傷。啟仲尼之嘉問,告性命以依方。



 豈茍然於迂論,聆寓言於達莊。



 於是濫石橋,登戲臺。策馬釣渚,息轡城隅。永感四山,零淚雙渠。怨物華之推驛,慨舟壑之遞遷。謂徂歲之悠闊,結幽思之方根。感皇祖
 之徽德,爰識沖而量淵。降俊明以鏡鑒,回風猷以昭宣。道既底於國難,惠有覃於黎元。士頌歌於政教,民謠詠於渥恩。兼《採芑》之致美,協《漢廣》之發言。彊虎氐之搏翼,灟雲網於所禁。驅黔萌以蘊崇,取園陵而湮沈。錫殘落於河西,序淪胥於漢陰。攻方城而折扃,擾譙潁其誰任。世闕才而貽亂,時得賢而興治。救祖考之邦壤,在幽人而枉志。



 體飛書之遠情,悟犒師之通識。迨明達之高覽,契古今而同事。拔淵謨於潛機,騁神鋒於雲旆。驅斥澤
 而風靡,蹙坑谷而鳥竄。中華免夫左衽,江表此焉緩帶。既克黜於肥六,又作鎮於彭沛。晏皇塗於國內,震天威於河外。掃東齊而已寧,指西崤而將泰。值秉均而代謝,實大業之興廢。心無忝於樂生,事有像於燕惠。抱明哲之不伐,奉宏勳而是稅。捐七州以爰來,歸五湖以投袂。屈盛績於平生,申遠期於暮歲。



 訪曩載於宋鄙,採《陽秋》於魯經。晉申好於東吳,鄭憑威於南荊。故反師於曹門,將以塞於夷庚。納五叛以長寇,伐三邑以侵彭。美西鉏
 之忠辭,快韓厥之奇兵。追項王之故臺,跡霸楚之遺端。挺宏志於總角,奮英勢於弱冠。氣蓋天而倒日,力拔山而傾湍。始飆起於勾越,中電激於衡關。興偏慮於攸吝,忘即易於所難。忌陳錦而莫照,思反鄉而有歎。且夫殺義害嬰,而心戛豐疑,緤賢不策,失位誰持。



 迨理屈而愈閉,方怨天而懷悲。對駿騅以發憤,傷虞姝於末詞。陟亞父之故營,諒謀始之非託。遭衰嬴之崩綱,值威炎之結絡。迄皓首於阜陵,猶謬覺於然諾。視一人於三傑,豈在己
 之庸弱。置豐沛而不舉,故自同於俎鑊。



 發汴口而游歷,迄西山而弭轡。觀終古之幽憤,懷元王之沖粹。丁戰國之權爭,方恬心於道肆。學浮丘以就德,友三儒以成類。潔流始於初源,累仁基於前美。撥楚族之休烈,傳芳素於來祀。彊見譽於清虛,德致稱於千里。或避寵以辭姻,或遺榮而不仕。政直言以安身,駿絕才以喪己。驅信道之成終,表昧世之虧始。悟介焉之已差,則不俟於終日。既防萌於未著,雖念德其何益。



 爾乃孟陬發節,雷隱蟄
 驚。散葉荑柯,芳FM飾萌。麥萋萋於旄丘,柳依依於高城。相雎鳩之集河,觀鳴鹿之食蘋。沂泗遠兮清川急,秋冬近兮緒風襲。風流蕙兮水增瀾,訴愁衿兮鑒戚顏。愁盈根而蕰際,戚發條而成端。嗟我行之彌日,待征邁而言旋。荷慶雲之優渥,周雙七於此年。陶逸豫於京甸,違險難於行川。轉歸弦而眷戀,望脩檣而流漣。願關鄴之遄清,遲華鑾之凱旋。穆淳風於六合,溥洪澤於八埏。頒賢愚於大小,順規矩於方圓。固四民之獲所,宜稅稷於萊
 田。苦邯鄲之難步,庶行迷之易痊。長守朴以終稔,亦拙者之政焉。



 仍除宋國黃門侍郎,遷相國從事中郎,世子左衛率。坐輒殺門生,免官。高祖受命,降公爵為侯,食邑五百戶。起為散騎常侍,轉太子左衛率。靈運為性褊激,多愆禮度,朝廷唯以文義處之,不以應實相許。自謂才能宜參權要,既不見知,常懷憤憤。廬陵王義真少好文籍,與靈運情款異常。少帝即位,權在大臣,靈運構扇異同,非毀執政,司徒徐羨之等患之,出為永嘉太守。郡有
 名山水,靈運素所愛好,出守既不得志,遂肆意游遨,遍歷諸縣,動踰旬朔,民間聽訟,不復關懷。所至輒為詩詠,以致其意焉。在郡一周,稱疾去職,從弟晦、曜、弘微等並與書止之,不從。



 靈運父祖並葬始寧縣,並有故宅及墅,遂移籍會稽,修營別業,傍山帶江,盡幽居之美。與隱士王弘之、孔淳之等縱放為娛,有終焉之志。每有一詩至都邑,貴賤莫不競寫,宿昔之間,士庶皆遍,遠近欽慕,名動京師。作《山居賦》並自注,以言其事。曰:古巢居穴處曰
 巖棲,棟宇居山曰山居,在林野曰丘園,在郊郭曰城傍,四者不同,可以理推。言心也,黃屋實不殊於汾陽;即事也,山居良有異乎市廛。抱疾就閑,順從性情,敢率所樂,而以作賦。揚子雲云:「詩人之賦麗以則。」文體宜兼,以成其美。今所賦既非京都宮觀遊獵聲色之盛,而敘山野草木水石穀稼之事,才乏昔人,心放俗外,詠於文則可勉而就之,求麗邈以遠矣。覽者廢張、左之艷辭,尋臺、皓之深意,去飾取素,儻值其心耳。意實言表,而書不盡,遺
 跡索意,託之有賞。其辭曰:謝子臥疾山頂,覽古人遺書,與其意合,悠然而笑曰:夫道可重,故物為輕;理宜存,故事斯忘。古今不能革,質文咸其常。合宮非縉雲之館,衢室豈放勳之堂。


邁深心於鼎湖,送高情於汾陽。嗟文成之卻粒,願追松以遠遊。嘉陶硃之鼓棹,乃語種以免憂。判身名之有辨,權榮素其無留。孰如牽犬之路既寡,聽鶴之途何由哉!
 \gezhu{
  理以相得為適,古人遺書,與其意合,所以為笑。孫權亦謂周瑜「公瑾與孤意合」。夫能重道則輕物,存理則忘事,古今質文可謂不同,而此處不異。縉雲、放勛不以天居為所樂,故合宮、衢室,皆}


\gezhu{
  非淹留,鼎湖、汾陽,乃是所居。囗文成、張良,卻粒棄人間事,從赤松子遊。陶硃、范蠡,臨去之際,亦語文種云云。謂二賢既權榮素,故身名有判也。牽犬,李斯之歎;聽鶴,陸機領成都眾大敗後,云「思聞華亭鶴唳,不可復得」。}
 若夫巢穴以風露貽患,則《大壯》以棟宇袪弊;宮室以瑤璇致美,則白賁以丘園殊世。惟上囗於巖壑,幸兼善而罔滯。雖非市朝而寒暑均和,雖是築構而飭朴兩逝。
 \gezhu{
  《易》云,上古穴居野處,後世聖人易之以宮室,上棟下宇,以蔽風雨,蓋取諸《大壯》。璇堂自是素,故曰白賁最是上爻也。此堂世異矣。謂巖壑道深於丘園,而不為巢穴,斯免囗囗得寒暑之適,雖是築構,無妨非朝市云云。}



 昔仲長願言,流水高山;應璩作書,邙阜洛川。勢有偏側,地闕周
 員。銅陵之奧,卓氏充釽摫之端;金谷之麗,石子致音徽之觀。徒形域之薈蔚,惜事異於棲盤。


至若鳳、叢二臺,雲夢、青丘,漳渠、淇園,橘林、長洲,雖千乘之珍苑,孰嘉遁之所遊。且山川之未備,亦何議於兼求。
 \gezhu{
  仲長子云:「欲使居有良田廣宅,在高山流川之畔。溝池自環,竹木周布,場囿在前,果園在後。」應璩與程文信書云:「故求道田,在關之西,南臨洛水,北據邙山,托崇岫以為宅,因茂林以為蔭。」謂二家山居,不得周員之美。揚雄《蜀都賦》云:「銅陵衍。」卓王孫採山鑄銅,故《漢書·貨殖傳》云:「卓氏之臨邛,公擅山川。」揚雄《方言》:「梁、益之間裁木為器曰釽,裂帛為衣曰摫。」金谷,石季倫之別廬,在河南界,有山川林木池沼水碓。其鎮下邳時,過遊賦詩,一代盛集。謂二地雖珍麗,然制作非棲盤之意}


\gezhu{
  也。鳳臺,秦穆公時秦女所居,以致簫史。叢臺,趙之崇館。張衡謂趙築叢臺於前,楚建章華於後。楚之雲夢,大中囗居《長飲賦》:楚靈王遊雲夢之中,息於荊臺之上。前方淮之水,左洞庭之波,右顧彭蠡之濤,南望巫山之阿,遂造章華之臺。亦見諸史。淮南青丘,齊之海外,皆獵所。司馬相如云:「秋田乎青丘,徬徨乎海外。」漳渠,史起為魏文侯所起,溉水之所。淇園,衛之竹園,在淇水之澳,《詩》人所載。橘林,蜀之園林,揚子云《蜀都賦》亦云橘林。左太沖謂戶有橘柚之園。長洲,吳之苑囿,左亦謂長洲之茂苑,因江海洲渚以為苑囿囗。囗囗囗囗囗囗囗囗故囗表此園之珍靜。千乘宴嬉之所,非囗囗憩止之囗,且山川亦不能兼茂,隨地勢所遇耳。}
 覽明達之撫運,乘機緘而理默。指歲暮而歸休,詠宏徽於刊勒。狹三閭之喪江,矜望諸之去國。選自然之神麗,盡高棲之意得。
 \gezhu{
  余祖車騎建大功淮、肥,江左得免橫流之禍。後及太傅既薨,建圖已輟,於是便求解駕東歸,以避君側之亂。廢興隱顯,當是賢達之心,故選神麗之所,以申高棲之意。經始山川,實基於此。}



 仰前哲之遺訓,俯性情之所便。奉微軀以宴息,保自事以乘閑。愧班生之夙悟,慚尚子之晚研。年與疾而偕來,志乘拙而俱旋。謝平生於知遊,棲清曠於山川。


\gezhu{
  謂經始此山,遺訓於後也。性情各有所便,山居是其宜也。《易》云:「向晦入宴息。」莊周云:「自事其心。」此二是其所處。班嗣本不染世,故曰夙悟;尚平未能去累,故曰晚研。想遲二人,更以年衰疾至。志寡求拙曰乘,并可山居。曰與知遊別,故曰謝平生;就山川,故曰棲清曠。}



 其居也,左湖右江,往渚還汀。面山背阜,東阻西傾。抱含吸
 吐,款跨紆縈。


綿聯邪亙,側直齊平。
 \gezhu{
  枚乘曰:「左江右湖,其樂無有。」此吳客說楚公子之詞。當謂江都之野,彼雖有江湖而乏山巖,此憶江湖左右與之同,而山嶽形勢,池城所無也。往渚還汀,謂四面有水;面山背阜,亦謂東西有山,便是四水之裏也。抱含吐吸,謂中央復有川。款跨紆縈,謂邊背相連帶。迂回處謂之邪亙,平正處謂之側直。}


近東則上田、下湖,西溪、南谷,石堟、石滂,閔硎、黃竹。決飛泉於百仞,森高簿於千麓。寫長源於遠江,派深毖於近瀆。
 \gezhu{
  上田在下湖之水囗,名為田口。下湖在田之下下處,並有名山川。西谿、南谷分流,谷鄣水畎入田口。西谿水出始寧縣西谷鄣,是近山之最高峰者,西谿便是囗之背。入西谿之裏,得石堟,以石為阻,故謂為堟。石滂在西谿之東,從縣南入九里,兩面峻峭數十丈,水自上飛下。比至外谿,封墱
  十數里,皆飛流迅激,左右巖壁綠竹。閔硎,在石滂之東溪,逶迤下注良田。黃竹與其連,南界莆中也。}


近南則會以雙流,縈以三洲。表裏回游,離合山川。崿崩飛於東峭,槃傍薄於西阡。拂青林而激波,揮白沙而生漣。
 \gezhu{
  雙流,謂剡江及小江,此二水同會於山南,便合流注下。三洲在二水之口,排沙積岸,成此洲漲。表裏合,是其貌狀也。崿者,謂回江岑,在其山居之南界,有石跳出,將崩江中,行者莫不駭心慄。槃者,是縣故治之所,在江之囗囗用盤石竟渚,並帶青林而連白沙也。}


近西則楊、賓接峰,唐皇連縱。室、壁帶溪,曾、孤臨江。竹緣浦以被綠,石照澗而映紅。月隱山而成陰,木鳴柯以起風。
 \gezhu{
  楊中、元賓,並小江之近處,與山相接也。唐皇便從北出。室,石室,在小江口
  南岸。壁,小江北岸。並在楊中之下。壁高四十丈,色赤,故曰照澗而映紅。曾山之西,孤山之南,王子所經始,並臨江,皆被以綠竹。山高月隱,便謂為陰;鳥集柯嗚,便謂為風也。}


近北則二巫結湖,兩軿通沼。橫、石判盡,休、周分表。引修堤之逶迤,吐泉流之浩溔。山𡼠下而回澤,瀨石上而開道。
 \gezhu{
  大小巫湖,中隔一山。外軿周回,在圻西北。邊浦出江,並是美處。義熙中,王穆之居大巫湖,經始處所猶在。兩軿皆長溪,外幹出山之後四五里許,裏軿亦隔一山,出新堟。橫山,野舍之北面。常石,野舍之西北。巫湖舊唐,故曰修堤。長溪甚遠,故曰泉流。常石𡼠囗囗囗囗故曰下𡼠而回澤。里軿漫石數里,水從上過,故曰瀨石上而開道。休山東北,周里山在休之南,並是北邊。}


遠東則天台、桐柏,方石、太平,二韭、四明,五奧、三菁。表神
 異於緯牒,驗感應於慶靈。凌石橋之莓苔,越楢溪之紆縈。
 \gezhu{
  天台、桐柏,七縣餘地,南帶海。二韭、四明、五奧,皆相連接,奇地所無,高於五岳,便是海中三山之流。韭以菜為名。四明、方石,四面自然開窗也。五奧者,曇濟道人、蔡氏、郗氏、謝氏、陳氏各有一奧,皆相掎角,並是奇地。三菁,太平之北。太平,天台之始。方石,直上萬丈,下有長谿,亦是縉雲之流云。此諸山並見圖緯,神仙所居。往來要徑石橋,過楢谿,人跡之艱,不復過此也。}


遠南則松箴、棲雞,唐嵫、漫石。崪、嵊對嶺,釐、孟分隔。入極浦而邅回,迷不知其所適。上嶔崎而蒙籠,下深沉而澆激。
 \gezhu{
  棲雞,在保口之上,別浦入其中,周回甚深,四山之裏。松箴在棲雞之上,緣江。唐嵫入太平水路,上有瀑布數百丈。漫石在唐嵫下,郗景興經始精舍,亦是名山之流。崪、嵊與分界,去
  山八十里,故曰遠南。前嶺鳥道,正當五十里高,左右所無,就下地形高,乃當不稱。遠望釐山甚奇,謂白爍尖者最高,下有良田,王敬弘經始精舍。曇濟道人住孟山,名曰孟埭,芋薯之矰田。清溪秀竹,回開巨石,有趣之極。此中多諸浦澗,傍依茂林,迷不知所通,嶔崎深沉,處處皆然,不但一處。}


遠西則
 \gezhu{
  下闕。}
 遠北則長江永歸,巨海延納。昆漲緬曠,島嶼綢沓。山縱橫以布護,水迴沉而縈浥。信荒極之綿眇,究風波之睽合。
 \gezhu{
  江從山北流,窮上虞界,謂之三江口,便是大海。老子謂海為百谷王,以其善處下也。海人謂孤山為昆。薄洲有山,謂之島嶼,即
  洲也。漲者,沙始起將欲成嶼,縱橫無常,於一處迴沉相縈擾也。大荒東極,故為荒極。風波不恒,為睽合也。}


徒觀其南術之囗囗囗生𡼠囗囗成衍囗岸測深,相渚知淺。洪濤滿則曾石沒,清瀾減則沉沙顯。及風興濤作,水勢奔壯。于歲春秋,在月朔望。湯湯驚波,滔滔駭浪。電激雷崩,飛流灑漾。凌絕壁而起岑,橫中流而連薄。始迅轉而騰天,終倒底而見壑。此楚貳心醉於吳客,河靈懷慚於海若。
 \gezhu{
  南術是其臨江舊宅,門前對江,三轉曾山,路窮四江,對岸西面常石。此二山之間,西南角岸孤山,此二山皆是狹處,故曰生𡼠。勇門以南上便大閬,故曰成衍。岸高測深,渚下知淺也。江中有孤石沉沙,
  隨水增減,春秋朔望,是其盛時。故枚乘云,楚太子有疾,吳客問之,舉秋濤之美,得以瘳病。太子,國之儲貳,故曰楚貳。河靈,河伯居河,所謂河靈。懼於海若,事見莊周《秋水篇》。}


爾其舊居,曩宅今園,枌囗囗槿尚援,基井具存。曲術周乎前後,直陌矗其東西。豈伊臨谿而傍沼,乃抱阜而帶山。考封域之靈異,實茲境之最然。葺駢梁於巖麓,棲孤棟於江源。敞南戶以對遠嶺,闢東窗以矚近田。田連岡而盈疇,嶺枕水而通阡。
 \gezhu{
  葺室在宅里山之東麓。東窗矚田,兼見江山之美。三間故謂之駢梁。門前一棟,枕幾上,存江之嶺,南對江上遠嶺。此二館屬望,殆無優劣也。}


阡陌縱橫,塍埒交經。導渠引流,脈散溝
 并。蔚蔚豐秫,苾苾香秔。送夏蚤秀,迎秋晚成。兼有陵陸,麻麥粟菽。候時覘節,遞藝遞孰。供粒食與漿飲,謝工商與衡牧。生何待於多資,理取足於滿腹。
 \gezhu{
  許由云:「偃鼠飲河,不過滿腹。」謂人生食足,則歡有餘,何待多須邪!工商衡牧,似多須者,若少私寡欲,充命則足。}



 但非田無以立耳。



 自園之田,自田之湖。泛濫川上,緬邈水區。浚潭澗而窈窕,除菰洲之紆餘。


毖溫泉於春流,馳寒波而秋徂。風生浪於蘭渚,日倒景於椒塗。飛漸榭於中沚,取水月之歡娛。旦延陰而物清,夕棲芬而氣敷。顧情交之永絕,覬雲客之
 暫如。
 \gezhu{
  此皆湖中之美,但患言不盡意,萬不寫一耳。諸澗出源入湖,故曰濬潭澗。澗長是以窈窕。除菰以作洲,言所以紆餘也。}


水草則萍藻蕰菼,雚蒲芹蓀,蒹菰蘋蘩,蕝荇菱蓮。雖備物之偕美,獨扶渠之華鮮。播綠葉之鬱茂,含紅敷之繽翻。怨清香之難留,矜盛容之易闌。必充給而後搴,豈蕙草之空殘。卷《叩弦》之逸曲,感《江南》之哀嘆。秦箏倡而溯游往,《唐上》奏而舊愛還。
 \gezhu{
  搴出《離騷》。《叩弦》是《採菱歌》。《江南》是《相和曲》,云江南采蓮。秦箏倡《蒹茄篇》,《唐上》奏《蒲生》詩,皆感物致賦。}



 魚藻蘋繁荇亦有詩人之詠,不復具敘。



 《本草》所載,山澤不一。雷、桐是別,和、緩是悉。參核六根,五華九
 實。


二冬並稱而殊性,三建異形而同出。水香送秋而擢茜,林蘭近雪而揚猗。卷柏萬代而不殞,伏苓千歲而方知。映紅葩於綠蒂,茂素蕤於紫枝。既住年而增靈,亦驅妖而斥疵。
 \gezhu{
  《本草》所出藥處,於今不復依,隨土所生耳。此境出藥甚多,雷公、桐君,古之采藥。醫緩,古之良工,故曰別悉。參核者,雙核桃杏仁也。六根者,茍七根、五茄根、葛根、野葛根、囗囗根也。五華者,堇華、芫華、檖華、菊華、旋覆華也。九實者,連前實、槐實、柏實、兔絲實、女貞實、蛇床實、蔓荊實、蓼實、囗囗也。二冬者,天門、麥門冬。三建者,附子、天雄、烏頭。水香,蘭草。}



 林蘭,支子。卷柏、伏苓,並皆仙物。凡此眾藥,事悉見於《神農》。


其竹則二箭殊葉,四苦齊味。水石別谷,巨細各彙。既修竦而便娟,
 亦蕭森而蓊蔚。露夕沾而心妻陰,風朝振而清氣。捎玄雲以拂杪,臨碧潭而挺翠。蔑上林與淇澳,驗東南之所遺。企山陽之游踐,遲鸞鷖之棲托。憶昆園之悲調,慨伶倫之哀籥。衛女行而思歸詠,楚客放而防露作。
 \gezhu{
  二箭,一者苦箭,大葉;一者笄箭,細葉。四苦,青苦、白苦、紫苦、黃苦。水竹,依水生,甚細密,吳中以為宅援。石竹,本科叢大,以充屋榱,巨者竿挺之屬,細者無箐之流也。修竦、便娟、蕭森、蓊蔚,皆竹貌也。上林,關中之禁苑,淇澳,衛地之竹園,方此皆不如。東南會稽之竹箭,唯此地最富焉。山陽,竹林之游;鸞鷖,棲食之所。崑山之竹任為笛,黃帝時,伶倫斬其厚均者吹之,為黃鐘之宮。衛女思歸,作《竹竿》之詩,楚人放逐,東方朔感江潭而作《七諫》。}


其木則松柏檀
 櫟,囗囗桐榆。檿柘穀棟,楸梓檉樗。剛柔性異,貞脆質殊。卑高沃塉,各隨所如。乾合抱以隱岑,杪千仞而排虛。凌岡上而喬竦,蔭澗下而扶疏。沿長谷以傾柯,攢積石以插衢。華映水而增光,氣結風而回敷。當嚴勁而蔥倩,承和煦而芬腴。送墜葉於秋晏,遲含萼於春初。
 \gezhu{
  皆木之類,選其美者載之。山脊曰岡。岡上澗下,長谷積石,各隨其方。《離騷》云:「青春受謝。白曰昭只。」}



 《詩》云「萼不𩋾𩋾」也。


植物既載,動類亦繁。飛泳騁透,胡可根源。觀貌相音,備列山川。寒燠順節,隨宜匪敦。
 \gezhu{
  草、木、竹,植物。魚、鳥、獸、動物。獸有數種,有騰者,有走者。}



 走者騁,騰者透。謂種
 類既繁,不可根源,但觀其貌狀,相其音聲,則知山川之好。



 興節隨宜,自然之數,非可敦戒也。


魚則魷鱧鮒鱮,鱒鯇鰱扁,魴鮪魦鱖,鱨鯉鯔鱣。輯采雜色,錦爛雲鮮。唼藻戲浪,汎苻流淵。或鼓鰓而湍躍,或掉尾而波旋。鱸鮆乘時以入浦,鰔𩷰沿瀨以出泉。
 \gezhu{
  魷音優。鱧音禮。鮒音附。鱮音敘。鱒音寸袞反。鯇音皖。鰱音連。扁音毖仙反。魴音房。鮪音磐。魦音沙。鱖音居綴反。鱨音上羊反。鯔音比之反。鱣音竹屳反。皆《說文》、《字林》音。《詩》云:「錦衾有爛。」故云錦爛。鱸鮆乘時魚。鱤音感。𩷰音迅。皆出谿中石上,恆以為玩。}


鳥則鵾鴻鶂鵠,鶖鷺鴇𪃥。雞鵲繡質,鶷雊綬章。晨鳧朝集,時鷮山梁。海鳥違風,朔禽避涼。荑生歸北,霜降客南。接
 響雲漢,侶宿江潭。聆清哇以下聽,載王子而上參。薄回涉以弁翰,映明壑而自耽。
 \gezhu{
  鵾音昆。鴻音洪。鶂音溢。《左傳》云:「六鶂退飛」,字如此。鵠音下竺反。鶖音秋。鷺音路。鴇音䳰。𪃥音相。唐公之馬,與此鳥色同,故謂為𪃥,音相。雞鵲鶷雊,見張茂先《博物志》。}



 鸐音翟,亦雉之美者,此四鳥並美采質。鳧音符,野鴨也,常待晨而飛。鷮音已消反,長尾雉也。《論語》云:「山梁雌雉,時哉時哉!」海鳥爰居,臧文仲不知其鳥,以為神也。事見《左傳》。朔禽,雁也,寒月轉往衡陽。《禮記》,霜始降,鴈來賓。歲莫云,鴈北向。政是陽初生時,荑生歸北,霜降客南。山雞映水自玩其羽儀者。


山上則猨𤟤貍貛,犴獌猰犬盈。山下則熊羆豺虎,羱鹿麕麖。擲飛枝於窮崖,踔空絕於深硎。蹲谷底而長嘯,攀木杪而哀鳴。
 \gezhu{
  猨音袁。𤟤音
  魂。貍音力之反。}



 貛音火丸反。犴音五懸反。獌音曼,似貛而長,狼之屬,一曰貙。猰音安黠反。犬盈音弋生反,貍之黃黑者,一曰似犬分。豺音在皆反。羱音元,野羊大角。麕音鬼氏反。麖音京,能踔擲。虎長嘯,猿哀鳴,鳴聲可玩。



 緡綸不投,置羅不披。磻弋靡用,蹄筌誰施。鑒虎狼之有仁,傷遂欲之無崖。


顧弱齡而涉道,悟好生之咸宜。率所由以及物,諒不遠之在斯。撫鷗䱔而悅豫,杜機心於林池。
 \gezhu{
  八種皆是魚獵之具。自少不殺,至乎白首,故在山中,而此歡永廢。莊周云,虎狼仁獸,豈不父子相親。世云虎狼暴虐者,政以其如禽獸,而人物不自悟其毒害,而言虎狼可疾之甚,茍其遂欲,豈復崖限。自弱齡奉法,故得免殺生之事。茍此悟萬物好生之理。《易》云:「不遠復,無只悔。」庶乘此得以入道。}



 莊周云,海人有機心,鷗鳥舞而不下。今無害彼之
 心,各說豫於林池也。



 敬承聖誥,恭窺前經。山野昭曠,聚落膻腥。故大慈之弘誓,拯群物之淪傾。


豈寓地而空言,必有貨以善成。欽鹿野之華苑,羨靈鷲之名山。企堅固之貞林,希庵羅之芳園。雖粹容之緬邈,謂哀音之恒存。建招提於幽峰,冀振錫之息肩。庶鐙王之贈席,想香積之惠餐。事在微而思通,理匪絕而可溫。
 \gezhu{
  賈誼《弔屈》云:「恭承嘉惠。」敬承,亦此之流。聚落是墟邑,謂歌哭諍訟,有諸喧嘩,不及山野為僧居止也。經教欲令在山中,皆有成文。老子云:「善貸且善成。」此道惠物也。}



 鹿苑,說《四真諦》處。靈鷲山,說《般若法華》處。堅固林,說泥洹處。庵羅園,說不思議處。今旁林藝園制苑,仿佛在昔,
 依然托想,雖粹容緬邈,哀音若存也。



 招提,謂僧不能常住者,可持作坐處也。所謂息肩。鐙王、香積,事出《維摩經》。



 《論語》云:「溫故知新。」理既不絕,更宜復溫,則可待為己之日用也。


爰初經略,杖策孤征。入澗水涉,登嶺山行。陵頂不息,窮泉不停。櫛風沐雨,犯露乘星。研其淺思,罄其短規。非龜非筮,擇良選奇。翦榛開徑,尋石覓崖。四山周回,雙流逶迤。面南嶺,建經臺;倚北阜,築講堂。傍危峰,立禪室;臨浚流,列僧房。對百年之高木,納萬代之芬芳。抱終古之泉源,美膏液之清長。謝麗塔於郊郭,殊世間於城傍。欣見素以抱樸,果甘露於道
 場。
 \gezhu{
  云初經略,躬自履行,備諸苦辛也。罄其淺短,無假於龜筮,貧者既不以麗為美,所以即安茅茨而已。是以謝郊郭而殊城傍。然清虛寂寞,實是得道之所也。}


苦節之僧,明發懷抱。事紹人徒,心通世表。是遊是憩,倚石構草。寒暑有移,至業莫矯。觀三世以其夢,撫六度以取道。乘恬知以寂泊,含和理之窈窕。指東山以冥期,實西方之潛兆。雖一日以千載,猶恨相遇之不早。
 \gezhu{
  謂曇隆、法流二法師也。二公辭恩愛,棄妻子,輕舉入山,外緣都絕,魚肉不入口,糞掃必在體,物見之絕歎,而法師處之夷然。詩人西發不勝造道者,其亦如此。往石門瀑布中路高棲之游,昔告離之始。期生東山,沒存西方。相遇之欣,實以一日為千載,猶慨恨不早。}



 賤物重己,棄世
 希靈。駭彼促年,愛是長生。冀浮丘之誘接,望安期之招迎。


甘松桂之苦味,夷皮褐以頹形。羨蟬蛻之匪日,撫雲蜺其若驚。陵名山而屢憩,過巖室而披情。雖未階於至道,且緬絕於世纓。指松菌而興言,良未齊於殤彭。
 \gezhu{
  此一章敘仙學者雖未及佛道之高,然出於世表矣。浮丘公是王子喬師,安期先生是馬明生師,二事出《列仙傳》。《洞直經》云:「今學仙者亦明師以自發悟,故不辭苦味頹形也。」莊周云:「和以天倪。」倪者,崖也。數經歷名山,遇餘巖室,披露其情性,且獲長生。方之松菌殤彭,邈然有間也。}


山作水役,不以一牧。資待各徒,隨節競逐。陟嶺刊木,除榛伐竹。抽筍自篁,擿箬于谷。楊
 勝所拮,秋冬籥獲。野有蔓草,獵涉蘡薁。亦醖山清,介爾景福。苦以術成,甘以手審熟。慕椹高林,剝芨巖椒。掘茜陽崖,擿手鮮陰摽。晝見搴茅,宵見索綯。芟菰翦蒲,以薦以茭。既坭既埏,品收不一。其灰其炭,咸各有律。六月採蜜,八月樸栗。備物為繁,略載靡悉。
 \gezhu{
  此一章謂山水採拾諸事也。然漁獵之事皆不載。楊,楊桃也。山間謂之木子。籥音覆,字出《字林》。《詩》人云:「六月食鬱及薁。」獵涉字出《爾雅》。術,術酒,味苦。手審,手審酒,味甘,並至美,兼以療病。手審治癰核,術治痰冷。椹音甚,味似菰菜而勝,刊木而作之,謂之慕。芨音及,採以為紙。茜音倩,採以為渫。手鮮音鮮,採以為飲。採蜜樸果,各隨其月也。}


若乃南北兩居,水通陸阻。觀
 風瞻雲,方知厥所。
 \gezhu{
  兩居謂南北兩處,各有居止。峰崿阻絕,水道通耳。觀風瞻雲,然後方知其處所。}
 南山則夾渠二田,周嶺三苑。九泉別澗,五谷異巘。群峰參差出其間,連岫復陸成其阪。眾流溉灌以環近,諸堤擁抑以接遠。遠堤兼陌,近流開湍。凌阜泛波,水往步還。還回往匝,枉渚員巒。呈美表趣,胡可勝單。抗北頂以葺館,殷南峰以啟軒。羅曾崖於戶裏,列鏡瀾於窗前。因丹霞以赬楣,附碧雲以翠椽。視奔星之俯馳,顧□□之未牽。鵾鴻翻翥而莫及,何但燕雀之翩翾。氿泉傍出,潺
 湲於東簷;桀壁對歭,硿礲於西霤。修竹葳蕤以翳薈,灌木森沉以蒙茂。蘿曼延以攀援,花芬薰而媚秀。日月投光於柯間,風露披清於㟪岫。夏涼寒燠,隨時取適。階基回互,橑欞乘隔。此焉卜寢,玩水弄石。邇即回眺,終歲罔斁。傷美物之遂化,怨浮齡之如借。眇遁逸於人群,長寄心於雲霓。
 \gezhu{
  南山是開創卜居之處也。從江樓步路,跨越山嶺,綿亙田野,或升或降,當三里許。塗路所經見也,則喬木茂竹,緣畛彌阜,橫波疏石,側道飛流,以為寓目之美觀。及至所居之處,自西山開道,迄於東山,二里有餘。南悉連嶺疊鄣,青翠相接,雲煙霄路,殆無倪際。從徑入谷,凡有三口。方壁西南石門世□南□池東
  南,皆別載其事。緣路初入,行於竹徑,半路闊,以竹渠澗。既入東南傍山渠,展轉幽奇,異處同美。路北東西路,因山為鄣。正北狹處,踐湖為池。南山相對,皆有崖巖。東北枕壑,下則清川如鏡,傾柯盤石,被隩映渚。西巖帶林,去潭可二十丈許,葺基構宇,在巖林之中,水衛石階,開窗對山,仰眺曾峰,俯鏡浚壑。去巖半嶺,復有一樓。迴望周眺,既得遠趣,還顧西館,望對窗戶。緣崖下者,密竹蒙徑,從北直南,悉是竹園。東西百丈,南北百五十五丈。北倚近峰,南眺遠嶺,四山周回,溪澗交過,水石林竹之美,巖岫隈曲之好,備盡之矣。刊翦開築,此焉居處,細趣密玩,非可具記,故較言大勢耳。越山列其表側傍緬□□為異觀也。}


因以小湖,鄰於其隈。眾流所湊,萬泉所回。氿濫異形,首毖終肥。別有山水,路邈緬歸。
 \gezhu{
  氿濫、肥毖,皆是泉名,事見於《詩》。云此萬泉所湊,各有形勢。}



 求歸其路,乃界北山。
 棧道傾虧,蹬閣連卷。復有水徑,繚繞回圓。瀰瀰平湖,泓泓澄淵。孤岸竦秀,長洲芊綿。既瞻既眺,曠矣悠然。及其二川合流,異源同口。


赴隘入險,俱會山首。瀨排沙以積丘,峰倚渚以起阜。石傾瀾而捎巖,木映波而結藪。徑南漘以橫前,轉北崖而掩後。隱叢灌故悉晨暮,託星宿以知左右。
 \gezhu{
  往反經過,自非巖澗,便是水徑,洲島相對,皆有趣也。}


山川澗石,州岸草木。既標異於前章,亦列同於後牘。山匪砠而是岵,川有清而無濁。石傍林而插巖,泉協澗而下谷。淵轉渚而散芳,
 岸靡沙而映竹。草迎冬而結葩,樹凌霜而振綠。向陽則在寒而納煦,面陰則當暑而含雪。連岡則積嶺以隱嶙,舉峰則群竦以𡽱嶭。浮泉飛流以寫空,沈波潛溢於洞穴。凡此皆異所而咸善,殊節而俱悅。
 \gezhu{
  土山載石曰砠,山有林曰岵。此章謂山川眾美,亦不必有,故總敘其最。居山之後事,亦皆有尋求也。}


春秋有待,朝夕須資。既耕以飯,亦桑貿衣。藝菜當肴,採藥救頹。自外何事,順性靡違。法音晨聽,放生夕歸。研書賞理,敷文奏懷。凡厥意謂,揚較以揮。且列于言,誡特此推。
 \gezhu{
  謂寒待綿纊,暑待絺綌,朝夕餐飲,設此諸
  業以待之。藥以療疾,又在其外,事之相推,自不得不然。至於聽講放生,研書敷文,皆其所好。}



 韓非有《揚較》,班固亦云「揚較古今」,其義一也。左思曰:「為左右揚較而陳之。」


北山二園,南山三苑。百果備列,乍近乍遠。羅行布株,迎早候晚。猗蔚溪澗,森疏崖巘。杏壇、㮈園,橘林、栗圃。桃李多品,梨棗殊所。枇杷林檎,帶谷映渚。椹梅流芬於回巒,椑柿被實於長浦。
 \gezhu{
  莊周云:「漁父見孔子杏壇之上。」}



 《維摩詰經》㮈樹園。揚雄《蜀都賦》云橘林。左太沖亦云:「戶有橘柚之園。」



 桃李所殖甚多,棗梨事出北河、濟之間,淮、潁諸處,故云殊所也。


畦町所藝,含蕊藉芳,蓼蕺祼薺,葑菲蘇姜。綠葵眷節以懷露,白薤感時而負霜。寒蔥摽倩以
 陵陰,春藿吐苕以近陽。
 \gezhu{
  葑菲見《詩·柏舟》中。管子曰:「北伐山戎,得寒蔥。」庾闡云,寒蔥挺園。灌蔬自供,不待外求者也。}



 弱質難恒,頹齡易喪。撫鬢生悲,視顏自傷。承清府之有術,冀在衰之可壯。


尋名山之奇藥,越靈波而憩轅。採石上之地黃,摘竹下之天門。摭曾嶺之細辛,拔幽澗之溪蓀。訪鐘乳於洞穴,訊丹陽於紅泉。
 \gezhu{
  此皆駐年之藥,即近山之所出,有採拾,欲以消病也。}



 安居二時,冬夏三月。遠僧有來,近眾無闕。法鼓朗響,頌偈清發。散華霏蕤,流香飛越。析曠劫之微言,說像法之遺旨。乘此心之一豪,濟彼生之萬理。
 啟善趣於南倡,歸清暢於北機。非獨愜於予情,諒僉感於君子。山中兮清寂,群紛兮自絕。



 周聽兮匪多,得理兮俱悅。寒風兮搔屑,面陽兮常熱。炎光兮隆熾,對陰兮霜雪。


愒曾臺兮陟雲根,坐澗下兮越風穴。在茲城而諧賞,傳古今之不滅。
 \gezhu{
  眾僧冬夏二時坐,謂之安居,輒九十日。眾遠近聚萃,法鼓、頌偈、華、香四種,是齋講之事。}



 析說是齋講之議。乘此之心,可濟彼之生。南倡者都講,北機者法師。山中靜寂,實是講說之處。兼有林木,可隨寒暑,恒得清和,以為適也。



 好生之篤,以我而觀。懼命之盡,吝景之懽。分一往之仁心,拔萬族之險難。


招驚魂於殆化,收
 危形於將闌。漾水性於江流,吸雲物於天端。睹騰翰之頏頡,視鼓鰓之往還。馳騁者儻能狂愈,猜害者或可理攀。
 \gezhu{
  雲物皆好生,但以我而觀,便可知彼之情。吝景懼命,是好生事也。能放生者,但有一往之仁心,便可拔萬族之險難。水性雲物,各尋其生。老子云,馳騁田獵,令人心發狂。猜害者恒以忍害為心,見放生之理,或可得悟也。}



 哲人不存,懷抱誰質。糟粕猶在,啟縢剖帙。見柱下之經二,睹濠上之篇七。


承未散之全樸,救已頹於道術。嗟夫!六藝以宣聖教,九流以判賢徒。國史以載前紀,家傳以申世模。篇章以陳美刺,論難以核有無。兵技醫日,龜
 莢筮夢之法,風角塚宅,算數律歷之書。或平生之所流覽,並於今而棄諸。驗前識之喪道,抱一德而不渝。
 \gezhu{
  莊周云「輪扁語齊桓公,公之所讀書,聖人之糟粕。」縢者,《金縢》之流也。柱下,老子。濠上,莊子。二、七,是篇數也。云此二書,最有理,過此以往,皆是聖人之教,獨往者所棄。}


伊昔齠齔,實愛斯文。援紙握管,會性通神。詩以言志,賦以敷陳。箴銘誄頌,咸各有倫。爰暨山棲,彌歷年紀。幸多暇日,自求諸己。研精靜慮,貞觀厥美。懷秋成章,含笑奏理。
 \gezhu{
  謂少好文章,及山棲以來,別緣既闌,尋慮文詠,以盡暇日之適。便可得通神會性,以永終朝。}



 若乃乘攝持之告,評養達之篇。畏絕跡
 之不遠,懼行地之多艱。均上皇之自昔,忌下衰之在旃。投吾心於高人,落賓名於聖賢。廣滅景於崆峒,許遁音於箕山。愚假駒以表谷,涓隱巖以搴芳。□□□□□□□□□□□□□□□□□□萊庇蒙以織畚。皓棲商而頤志,卿寢茂而敷詞。□□□□□□,鄭別谷而永逝。梁去霸而之會,□□□□□□。高居唐而胥宇,臺依崖而穴墀。咸自得以窮年,眇貞思於所遺。


\gezhu{
  老子云:「善攝生者。」莊子云,謂之不善持生。又云,養生有無崖,達生者不務生之所無,奈何。絕跡,上皇,下衰,賓名,義亦皆出莊周。廣成子在崆峒
  之上,黃帝之師也。許由隱於箕山,堯以天下讓而不取。愚公居于駒阜,齊桓公逐鹿入山,見之。涓子隱于宕山,好餌術,告伯陽《琴心》三篇。庚桑楚得老子之道,居㟪礨之山。楚狂接輿,楚王聞其賢,使使者聘之,於是遂游諸名山,在蜀峨眉山上。徐無鬼巖棲,魏侯勞之,問:「先生苦山林矣,乃肯見寡人。」無鬼問:「君絀嗜欲,屏好惡,則耳目察矣。」常采芋栗。老萊子耕於蒙山之陽,著書十五篇,言道家之事,織畚為業。四皓避秦亂,入商洛深山,漢祖召不能出。司馬長卿高才,而處世不樂預公卿大事,(缺)遂與弟子別於山阿,終身不反。梁伯鸞隱霸陵山中,耕織以自娛,後復入會稽山。臺孝威居武安山下,依崖為土室,采藥自給。高文通居西唐山,從容自娛也。}


暨其窈窕幽深,寂漠虛遠。事與情乖,理與形反。既耳目之靡端,豈足跡之所踐。蘊終古於
 三季,俟通明於五眼。權近慮以停筆,抑淺知而絕簡。
 \gezhu{
  謂此既非人跡所求,更待三明五通,然後可踐履耳。故停筆絕簡,不復多云,冀夫賞音悟夫此旨也。}



 太祖登祚,誅徐羨之等,徵為秘書監,再召不起,上使光祿大夫範泰與靈運書敦獎之,乃出就職。使整理祕閣書,補足闕文。以晉氏一代,自始至終,竟無一家之史,令靈運撰《晉書》,粗立條流;書竟不就。尋遷侍中,日夕引見,賞遇甚厚。



 靈運詩書皆兼獨絕,每文竟,手自寫之,文帝稱為二寶。既自以名輩,才能應參時政,初被召,便以此自許;既至,文帝唯以文義見接,每侍上
 宴,談賞而已。王曇首、王華、殷景仁等,名位素不踰之,並見任遇,靈運意不平,多稱疾不朝直。穿池植援,種竹樹堇,驅課公役,無復期度。出郭游行或一日百六七十里,經旬不歸,既無表聞,又不請急。上不慾傷大臣,諷旨令自解。靈運乃上表陳疾,上賜假東歸。



 將行,上書勸伐河北,曰:自中原喪亂,百有餘年,流離寇戎,湮沒殊類。先帝聰明神武,哀濟群生,將欲蕩定趙魏,大同文軌,使久凋反於正化,偏俗歸於華風。運謝事乖,理違願絕,仰德抱
 悲,恨存生盡。況陵塋未幾,兇虜伺隙,預在有識,誰不憤歎。而景平執事,並非其才,且遘紛京師,豈慮托付。遂使孤城窮陷,莫肯極。忠烈囚朔漠,綿河三千,翻為寇有。晚遣鎮戍,皆先朝之所開拓,一旦淪亡,此國恥宜雪,被於近事者也。又北境自染逆虜,窮苦備罹,徵調賦斂,靡有止已,所求不獲,輒致誅殞,身禍家破,闔門比屋,此亦仁者所為傷心者也。



 咸云西虜舍末,遠師隴外,東虜乘虛,呼可掩襲。西軍既反,得據關中,長圍咸陽,還路已絕,雖
 遣救援,停住河東,遂乃遠討大城,欲為首尾。而西寇深山重阻,根本自固,徒棄巢窟,未足相拯。師老於外,國虛於內,時來之會,莫復過此。



 觀兵耀威,實在茲日。若相持未已,或生事變,忽值新起之眾,則異於今,茍乖其時,難為經略,雖兵食倍多,則萬全無必矣。又歷觀前代,類以兼弱為本,古今聖德,未之或殊。豈不以天時人事,理數相得,興亡之度,定期居然。故古人云:「既見天殃,又見人災,乃可以謀。」昔魏氏之彊,平定荊、冀,乃乘袁、劉之弱;晉
 世之盛,拓開吳、蜀,亦因葛、陸之衰。此皆前世成事,著於史策者也。自羌平之後,天下亦謂虜當俱滅,長驅滑臺,席卷下城,奪氣喪魄,指日就盡。



 但長安違律,潼關失守,用緩天誅,假延歲月,日來至今,十有二載,是謂一紀,曩有前言。況五胡代數齊世,虜期餘命,盡於來年。自相攻伐,兩取其困,卞莊之形,驗之今役。仰望聖澤,有若渴飢,注心南雲,為日已久。來蘇之冀,實歸聖明,此而弗乘,後則未兆。即日府藏,誠無兼儲,然凡造大事,待國富兵彊,
 不必乘會,於我為易,貴在得時。器械既充,眾力粗足,方於前後,乃當有優。常議損益,久證冀州口數,百萬有餘,田賦之沃,著自《貢》典,先才經創,基趾猶存,澄流引源,桑麻蔽野,彊富之實,昭然可知。為國長久之計,孰若一往之費邪!



 或懲關西之敗,而謂河北難守。二境形勢,表裏不同,關西雜居,種類不一,昔在前漢,屯軍霸上,通火甘泉。況乃遠戍之軍,值新故交代之際者乎!河北悉是舊戶,差無雜人,連嶺判阻,三關作隘。若遊騎長驅,則沙漠
 風靡;若嚴兵守塞,則冀方山固。昔隴西傷破,晁錯興言;匈奴慢侮,賈誼憤歎。方於今日,皆為賒矣。



 晉武中主耳,值孫晧虐亂,天祚其德,亦由鉅平奉策,荀、賈折謀,故能業崇當年,區宇一統。況今陛下聰明聖哲,天下歸仁,文德與武功並震,霜威共素風俱舉,協以宰輔賢明,諸王美令,岳牧宣烈,虎臣盈朝,而天或遠命,亦何敵不滅,矧伊頑虜,假日而已哉。伏惟深機志務,久定神謨。臣卑賤側陋,竄景巖穴,實仰希太平之道,傾睹岱宗之封,雖乏
 相如之筆,庶免史談之憤,以此謝病京師,萬無恨矣。久欲上陳,懼在觸置,蒙賜恩假,暫違禁省,消渴十年,常慮朝露,抱此愚志,昧死以聞。



 靈運以疾東歸,而遊娛宴集,以夜續晝,復為御史中丞傅隆所奏,坐以免官。



 是歲,元嘉五年。靈連既東還,與族弟惠連、東海何長瑜、潁川荀雍、泰山羊璿之,以文章賞會,共為山澤之游,時人謂之四友。惠連幼有才悟,而輕薄不為父方明所知。靈運去永嘉還始寧,時方明為會稽郡。靈運嘗自始寧至會稽
 造方明,過視惠連,大相知賞。時長瑜教惠連讀書,亦在郡內,靈運又以為絕倫,謂方明曰:「阿連才悟如此,而尊作常兒遇之。何長瑜當今仲宣,而飴以下客之食。尊既不能禮賢,宜以長瑜還靈運。」靈運載之而去。



 荀雍,字道雍,官至員外散騎郎。璿之,字曜璠,臨川內史,為司空竟陵王誕所遇,誕敗坐誅。長瑜文才之美,亞於惠連,雍、璿之不及也。臨川王義慶招集文士,長瑜自國侍郎至平西記室參軍。嘗於江陵寄書與宗人何勖,以韻語序義
 慶州府僚佐云:「陸展染鬢髮,欲以媚側室。青青不解久,星星行復出。」如此者五六句,而輕薄少年遂演而廣之,凡厥人士,並為題目,皆加劇言苦句,其文流行。義慶大怒,白太祖除為廣州所統曾城令。及義慶薨,朝士詣第敘哀,何勖謂袁淑曰:「長瑜便可還也。」淑曰:「國新喪宗英,未宜便以流人為念。」廬陵王紹鎮尋陽,以長瑜為南中郎行參軍,掌書記之任。行至板橋,遇暴風溺死。



 靈運因父祖之資,生業甚厚。奴僮既眾,義故門生數百,鑿山浚湖,
 功役無已。



 尋山陟嶺,必造幽峻,巖嶂千重,莫不備盡。登躡常著木履,上山則去前齒,下山去其後齒。嘗自始寧南山伐木開徑,直至臨海,從者數百人。臨海太守王琇驚駭,謂為山賊,徐知是靈運乃安。又要琇更進,琇不肯,靈運贈琇詩曰:「邦君難地險,旅客易山行。」在會稽亦多徒眾,驚動縣邑。太守孟顗事佛精懇,而為靈運所輕,嘗謂顗曰:「得道應須慧業文人,生天當在靈運前,成佛必在靈運後。」顗深恨此言。



 會稽東郭有回踵湖,靈運求決
 以為田,太祖令州郡履行。此湖去郭近,水物所出,百姓惜之,顗堅執不與。靈運既不得回踵,又求始寧岯崲湖為田,顗又固執。



 靈運謂顗非存利民,正慮決湖多害生命,言論毀傷之,與顗遂構仇隙。因靈運橫恣,百姓驚擾,乃表其異志,發兵自防,露板上言。靈運馳出京都,詣闕上表曰:「臣自抱疾歸山,于今三載,居非郊郭,事乖人間,幽棲窮巖,外緣兩絕,守分養命,庶畢餘年。忽以去月二十八日得會稽太守臣顗二十七日疏云:『比日異論噂
 𠴲,此雖相了,百姓不許寂默,今微為其防。』披疏駭惋,不解所由,便星言奔馳,歸骨陛下。及經山陰,防衛彰赫,彭排馬槍,斷截衢巷,偵邏縱橫,戈甲竟道。不知微臣罪為何事。及見顗,雖曰見亮,而裝防如此,唯有罔懼。臣昔忝近侍,豫蒙天恩,若其罪迹炳明,文字有證,非但顯戮司敗,以正國典,普天之下,自無容身之地。今虛聲為罪,何酷如之。夫自古讒謗,聖賢不免,然致謗之來,要有由趣。或輕死重氣,結黨聚群,或勇冠鄉邦,劍客馳逐。未聞俎
 豆之學,欲為逆節之罪;山棲之士,而構陵上之釁。今影跡無端,假謗空設,終古之酷,未之或有。匪吝其生,實悲其痛。誠復內省不疚,而抱理莫申。是以牽曳疾病,束骸歸款。仰憑陛下天鑒曲臨,則死之日,猶生之年也。臣憂怖彌日,羸疾發動,尸存恍惚,不知所陳。」



 太祖知其見誣,不罪也。不欲使東歸,以為臨川內史,賜秩中二千石。在郡遊放,不異永嘉,為有司所糾。司徒遣使隨州從事鄭望生收靈運,靈運執錄望生,興兵叛逸,遂有逆志。為詩
 曰:「韓亡子房奮,秦帝魯連恥。本自江海人,忠義感君子。」追討禽之,送廷尉治罪。廷尉奏靈運率部眾反叛,論正斬刑。上愛其才,欲免官而已。彭城王義康堅執謂不宜恕,乃詔曰:「靈運罪釁累仍,誠合盡法。但謝玄勳參微管,宜宥及後嗣,可降死一等,徙付廣州。」



 其後,秦郡府將宗齊受至塗口,行達桃墟村,見有七人下路亂語,疑非常人,還告郡縣,遣兵隨齊受掩討,遂共格戰,悉禽付獄。其一人姓趙名欽,山陽縣人,云:「同村薛道雙先與謝康樂
 共事,以去九月初,道雙因同村成國報欽云:『先作臨川郡、犯事徙送廣州謝,給錢令買弓箭刀楯等物,使道雙要合鄉里健兒,於三江口篡取謝。若得志,如意之後,功勞是同。』遂合部黨要謝,不及。既還飢饉,緣路為劫盜。」有司又奏依法收治,太祖詔於廣州行棄市刑。臨死作詩曰:「龔勝無餘生,李業有終盡。嵇公理既迫,霍生命亦殞。悽悽凌霜葉,網網衝風菌。邂逅竟幾何,修短非所愍。送心自覺前,斯痛久已忍。恨我君子志,不獲巖上泯。」詩所
 稱龔勝、李業,猶前詩子房、魯連之意也。時元嘉十年,年四十九。所著文章傳於世。子鳳,蚤卒。



 史臣曰:民稟天地之靈,含五常之德,剛柔迭用,喜慍分情。夫志動於中,則歌詠外發。六義所因,四始攸繫,升降謳謠,紛披風什。雖虞夏以前,遺文不睹,稟氣懷靈,理無或異。然則歌詠所興,宜自生民始也。周室既衰,風流彌著,屈平、宋玉,導清源於前,賈誼、相如,振芳塵於後,英辭潤金石,高義薄雲天。自茲以降,情志愈廣。王褒、劉向、揚、
 班、崔、蔡之徒,異軌同奔,遞相師祖。雖清辭麗曲,時發乎篇,而蕪音累氣,固亦多矣。若夫平子艷發,文以情變,絕唱高蹤,久無嗣響。至於建安,曹氏基命,二祖陳王,咸蓄盛藻,甫乃以情緯文,以文被質。



 自漢至魏,四百餘年,辭人才子,文體三變。相如巧為形似之言,班固長於情理之說,子建、仲宣以氣質為體,並標能擅美,獨映當時。是以一世之士,各相慕習,原其飆流所始,莫不同祖《風》、《騷》。徒以賞好異情,故意製相詭。降及元康,潘、陸特秀,律異
 班、賈,體變曹、王,縟旨星稠,繁文綺合。綴平臺之逸響,採南皮之高韻,遺風餘烈,事極江右。有晉中興,玄風獨振,為學窮於柱下,博物止乎七篇,馳騁文辭,義單乎此。自建武暨乎義熙,歷載將百,雖綴響聯辭,波屬雲委,莫不寄言上德,托意玄珠,遒麗之辭,無聞焉爾。仲文始革孫、許之風,叔源大變太元之氣。爰逮宋氏,顏、謝騰聲。靈運之興會標舉,延年之體裁明密,並方軌前秀,垂範後昆。若夫敷衽論心,商榷前藻,工拙之數,如有可言。夫五色
 相宣,八音協暢,由乎玄黃律呂,各適物宜。欲使宮羽相變,低昂互節,若前有浮聲,則後須切響。一簡之內,音韻盡殊;兩句之中,輕重悉異。妙達此旨,始可言文。至於先士茂製,諷高歷賞,子建函京之作,仲宣霸岸之篇,子荊零雨之章,正長朔風之句,並直舉胸情,非傍詩史,正以音律調韻,取高前式。自《騷》人以來,而此秘未睹。至於高言妙句,音韻天成,皆暗與理合,匪由思至。張、蔡、曹、王,曾無先覺,潘、陸、謝、顏,去之彌遠。世之知音者,有以得之,知此
 言之非謬。如曰不然,請待來哲。



\end{pinyinscope}