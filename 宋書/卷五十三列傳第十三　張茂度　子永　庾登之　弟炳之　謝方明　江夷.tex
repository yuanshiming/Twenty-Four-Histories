\article{卷五十三列傳第十三 張茂度 子永 庾登之 弟炳之 謝方明 江夷}

\begin{pinyinscope}

 張茂度,吳郡吳人,張良後也。名與高祖諱同,故稱字。良七世孫為長沙太守,始遷於吳。高祖嘉,曾祖澄,晉光祿大夫。祖彭祖,廣州刺史。父敞,侍中、尚書、吳國內史。



 茂度
 郡上計吏,主簿,功曹,州命從事史,並不就。除琅邪王衛軍參軍,員外散騎侍郎,尚書度支郎,父憂不拜。服闋,為何無忌鎮南參軍。頃之,出補晉安太守,盧循為寇,覆沒江州,茂度及建安太守孫蚪之並受其符書,供其調役。循走,俱坐免官。復以為始興相,郡經賊寇,廨宇焚燒,民物凋散,百不存一。茂度創立城寺,弔死撫傷,收集離散,民戶漸復。在郡一周,徵為太尉參軍,尋轉主簿、揚州治中從事史。高祖西伐劉毅,茂度居守。留州事悉委之。軍
 還,遷中書侍郎。出為司馬休之平西司馬、河南太守。高祖將討休之,茂度聞知,乘輕船逃下,逢高祖於中路,以為錄事參軍,太守如故。江陵平,驃騎將軍道憐為荊州,茂度仍為咨議參軍,太守如故。還為揚州別駕從事史。高祖北伐關洛,復任留州事。出為使持節、督廣交二州諸軍事、建武將軍、平越中郎將、廣州刺史。綏靜百越,嶺外安之。以疾求還,復為道憐司馬。丁繼母憂,服闋,除廷尉,轉尚書吏部郎。



 太祖元嘉元年,出為使持節、督益寧
 二州梁州之巴西梓潼宕渠南漢中秦州之懷寧安固六郡諸軍事、冠軍將軍、益州刺史。三年,太祖討荊州刺史謝晦,詔益州遣軍襲江陵,晦已平而軍始至白帝。茂度與晦素善,議者疑其出軍遲留,時茂度弟邵為湘州刺史,起兵應大駕,上以邵誠節,故不加罪,被代還京師。七年,起為廷尉,加奉車都尉,領本州中正。入為五兵尚書,徙太常。以腳疾出為義興太守,加秩中二千石。上從容謂茂度曰:「勿復以西蜀介懷。」對曰:「臣若不遭陛下之
 明,墓木拱矣。」頃之,解職還家。徵為都官尚書,加散騎常侍,固辭以疾。就拜光祿大夫,加金章紫綬。



 茂度內足於財,自絕人事,經始本縣之華山以為居止,優游野澤,如此者七年。



 十八年,除會稽太守。素有吏能,在郡縣,職事甚理。明年,卒官,時年六十七。



 謚曰恭子。



 茂度同郡陸仲元者,晉太尉玩曾孫也。以事用見知,歷清資,吏部郎,右衛將軍,侍中,吳郡太守。自玩洎仲元,四世為侍中,時人方之金、張二族。弟子真,元嘉十年,為海陵太守。中書舍
 人狄當為太祖所信委,家在海陵,死還葬,橋路毀壞,不通喪車,縣求發民脩治,子真不許。司徒彭城王義康聞而善之,召為國子博士,司徒左西掾,州治中,臨海東陽太守。



 茂度子演,太子中舍人;演弟鏡,新安太守,皆有盛名,並早卒。鏡弟永。永字景雲,初為郡主簿,州從事,轉司徒士曹參軍,出補餘姚令,入為尚書中兵郎。



 先是,尚書中條制繁雜,元嘉十八年,欲加治撰,徙永為刪定郎,掌其任。二十二年,除
 建康令,所居皆有稱績。又除廣陵王誕北中郎錄事參軍。永涉獵書史,能為文章,善隸書,曉音律,騎射雜藝,觸類兼善,又有巧思,益為太祖所知。紙及墨皆自營造,上每得永表啟,輒執玩咨嗟,自嘆供御者了不及也。二十三年,造華林園、玄武湖,並使永監統。凡諸制置,皆受則於永。徙為江夏王義恭太尉中兵參軍、越騎校尉、振武將軍、廣陵南沛二郡太守。二十八年,又除江夏王義恭驃騎中兵參軍,沛郡如故。



 永既有才能,所在每盡心力,太
 祖謂堪為將。二十九年,以永督冀州青州之濟南樂安太原三郡諸軍事、揚威將軍、冀州刺史,督王玄謨、申坦等諸將,經略河南。



 攻確磝城,累旬不能拔。其年八月七日夜,虜開門燒樓及攻車,士卒燒死及為虜所殺甚眾,永即夜撤圍退軍,不報告諸將,眾軍驚擾,為虜所乘,死敗塗地;永及申坦並為統府撫軍將軍蕭思話所收,繫於歷城獄。太祖以屢征無功,諸將不可任,責永等與思話詔曰:「虜既乘利,方向盛冬,若脫敢送死,兄弟父子,
 自共當之耳。



 言及增憤,可以示張永、申坦。」又與江夏王義恭書曰:「早知諸將輩如此,恨不以白刃驅之,今者悔何所及!」



 三十年,元凶弒立,起永督青州徐州之東安東莞二郡諸軍事、輔國將軍、青州刺史。司空南譙王義宣起義,又板永為督冀州青州之濟南樂安太原三郡諸軍事、輔國將軍、冀州刺史。永遣司馬崔勳之、中兵參軍劉則二軍馳赴國難。時蕭思話在彭城,義宣慮二人不相諧緝,與思話書,勸與永坦懷。又使永從兄長史張暢與永
 書曰:「近有都信,具汝刑網之原,可謂雖在縲紲,而腹心無愧矣。蕭公平厚,先無嫌隙,見汝翰跡,言不相傷,何其滔滔稱人意邪!當今世故艱迫,義旗雲起,方藉群賢,共康時難。當遠慕廉、藺在公之德,近效平、勃忘私之美,忽此蒂芥,剋申舊情。



 公亦命蕭示以疏達,兼令相報,共遵此旨。」事平,召為江夏王義恭大司馬從事中郎,領中兵。



 時使百僚獻讜言,永以為宜立諫官,開不諱之路,講師旅,示安不忘危。世祖孝建元年,臧質反,遣永輔武昌王
 渾鎮京口。其年,出為揚州別駕從事史。明年,召入為尚書左丞。時將士休假,年開三番,紛紜道路。永建議曰:「臣聞開兵從稼,前王以之兼隙,耕戰遞勞,先代以之經遠。當今化寧萬里,文同九服,捐金走驥,於焉自始。伏見將士休假,多蒙三番,程會既促,裝赴在早。故一歲之間,四馳遙路,或失遽春耜,或違要秋登,致使公替常儲,家闕舊粟,考定利害,宜加詳改。



 愚謂交代之限,以一年為制,使徵士之念,勞未及積;遊農之望,收功歲成。斯則王度
 無騫,民業斯植矣。」從之。



 大明元年,遷黃門侍郎,尋領虎賁中郎將、本郡中正。三年,遷廷尉。上謂之曰:「卿既與釋之同姓,欲使天下須無冤民。」加寧朔將軍、尚書吏部郎、司徒右長史、尋陽王子房冠軍長史。四年,立明堂,永以本官兼將作大匠。事畢,遷太子右衛率。七年,為宣貴妃殷氏立廟,復兼將作大匠。轉右衛將軍。其年,世祖南巡,自宣城候道東入,使永循行水路。是歲旱,塗逕不通,上大怒,免。時上寵子新安王子鸞為南徐州刺史,割吳郡
 度屬徐州。八年,起永為別駕從事史。其年,召為御史中丞。前廢帝永光元年,出為吳興太守,遷度支尚書。



 太宗即位,除吏部尚書。未拜,會四方反叛,復以為吳興太守,加冠軍將軍。



 假節。未拜,以將軍假節,徙為吳郡太守,率軍東討。又為散騎常侍、太子詹事。



 未拜,遷使持節、監青冀幽并四州諸軍事、前將軍,青冀二州刺史,統諸將討徐州刺史薛安都,累戰剋捷,破薛索兒等,事在《安都傳》。又遷散騎常侍、鎮軍將軍、太子詹事,權領徐州刺史。又
 都督徐、兗、青、冀四州諸軍事,又為使持節、都督南兗徐二州諸軍事、南兗州刺史,常侍、將軍如故。時薛安都據彭城請降,而誠心不款,太宗遣永與沈攸之以重兵迎之,加督前鋒軍事,進軍彭城。安都招引索虜之兵既至,士卒離散,永狼狽引軍還,為虜所追,大敗。復值寒雪,士卒離散,永腳指斷落,僅以身免,失其第四子。



 三年,徙都督會稽東陽臨海永嘉新安五郡諸軍事、會稽太守,將軍如故。以北討失律,固求自貶,降號左將軍。永痛悼所
 失之子,有兼常哀,服制雖除,猶立靈座,飲食衣服,待之如生。每出行,常別具名車好馬,號曰侍從,有事輒語左右報郎君。以破薛索兒功,封孝昌縣侯,食邑千戶。在會稽,賓客有謝方童等,坐贓下獄死,永又降號冠軍將軍。四年,遷使持節、督雍梁南北秦四州郢州之竟陵隨二郡諸軍事、右將軍、雍州刺史。未拜,停為太子詹事,加散騎常侍、本州大中正。六年,又加護軍將軍,領石頭戍事;給鼓吹一部。七年,遷金紫光祿大夫,尋復領護軍。後廢帝
 即位,進右光祿大夫,加侍中,領安成王師,加親信二十人。又領本州中正,出為吳郡太守,秩中二千石,侍中、右光祿如故。元徽二年,遷使持節、都督南兗徐青冀益五州諸軍事、征北將軍、南兗州刺史,侍中如故。



 永少便驅馳,志在宣力,年雖已老,志氣未衰,優游閑任,意甚不樂,及有此授,喜悅非常,即日命駕還都。未之鎮,值桂陽王休範作亂,永率所領出屯白下。



 休範至新亭,大桁不守,前鋒遂攻南掖門。永遣人覘賊,既返,唱云:「臺城陷矣。」



 永
 眾於此潰散,永亦棄軍奔走,還先所住南苑。以永舊臣不加罪,止免官削爵,永亦愧歎發病。三年,卒,時年六十六。順帝昇明二年,追贈侍中、右光祿大夫。子瑰,昇明末,達官。永弟辯,太宗亦見任遇,歷尚書吏部郎,廣州刺史,大司農。



 辯弟岱,昇明末,吏部尚書。



 庾登之,字元龍,潁川鄢陵人也。曾祖冰,晉司空。祖蘊,廣州刺史。父廓,東陽太守。登之少以彊濟自立,初為晉會稽王道子太傅參軍。義旗初,又為高祖鎮軍參軍。以預
 討桓玄功,封曲江縣五等男。參大司馬瑯邪王軍事,豫州別駕從事史,大司馬主簿,司徒左西曹屬。登之雖不涉學,善於世事,王弘、謝晦、江夷之徒,皆相知友。轉太尉主簿。義熙十二年,高祖北伐,登之擊節驅馳,退告劉穆之,以母老求郡。于是士庶咸憚遠役,而登之二三其心,高祖大怒,除吏名。大軍發後,乃以補鎮蠻護軍、西陽太守。入為太子庶子,尚書左丞。出為新安太守。



 謝晦為撫軍將軍、荊州刺史,請為長史、南郡太守,仍為衛軍長史,
 太守如故。



 登之與晦俱曹氏婿,名位本同,一旦為之佐,意甚不愜。到箋,唯云「即日恭到,」



 初無感謝之言。每入覲見,備持箱囊几席之屬,一物不具不坐。晦常優容之。晦拒王師,欲使登之留守,登之不許,語在《晦傳》。晦敗,登之以無任免罪,禁錮還家。



 元嘉五年,起為衡陽王義季征虜長史。義季年少,未親政,眾事一以委之。尋加南東海太守。入為司徒右長史,尚書吏部郎,司徒左長史,南東海太守。府公彭城王義康專覽政事,不欲自下厝懷,
 而登之性剛,每陳己意,義康甚不悅,出為吳郡太守。州郡相臨,執意無改,因其蒞任贓貨,以事免官。弟炳之時為臨川內史,登之隨弟之郡,優游自適。俄而除豫章太守,便道之官。登之初至臨川,吏民咸相輕侮,豫章與臨川接境,郡又華大,儀迓光赫,士人並驚歎焉。十八年,遷江州刺史。疾篤,徵為中護軍。未拜。二十年,卒,時年六十二。即以為贈。



 子沖遠,太宗鎮姑孰,為衛軍長史,卒於豫章太守,追贈侍中。
 炳之,字仲文,初為秘書、太子舍人,劉粹征北長史、廣平太守。兄登之為謝晦長史,炳之往省之。



 晦時位高權重,朝士莫不加敬,炳之獨與抗禮,時論健之。為尚書度支郎,不拜。



 出補錢塘令,治民有績。轉彭城王義康驃騎主簿,未就,徙為丹陽丞。炳之既未到府,疑於府公禮敬,下禮官博議。中書侍郎裴松之議曰:「案《春秋》桓八年,祭公逆王后于紀。《公羊傳》曰:『女在國稱女,此其稱王后何?王者無外,其辭成矣。』推此而言,則炳之為吏之道,定於受
 命之日矣,其辭已成,在官無外,名器既正,則禮亦從之。且今宰牧之官,拜不之職,未接之民,必有其敬者,以既受王命,則成君民之義故也。吏之被敕,猶除者受拜,民不以未見闕其被禮,吏安可以未到廢其節乎?愚懷所見,宜執吏禮。」從之。遷司徒左西屬。左將軍竟陵王義宣未親府板炳之為咨議參軍,眾務悉委焉。後將軍長沙王義欣鎮壽陽,炳之為長史、南梁郡太守,轉鎮國長史,太守如故。出為臨川內史。後將軍始興王浚鎮湘州,以
 炳之為司馬,領長沙內史。浚不之任,除南太山太守,司馬如故。



 于時領軍將軍劉湛協附大將軍彭城王義康,而與僕射殷景仁有隙,凡朝士遊殷氏者,不得入劉氏之門,獨炳之遊二人之間,密盡忠於朝廷。景仁稱疾不朝見者歷年,太祖常令炳之銜命去來,湛不疑也。義康出籓,湛伏誅,以炳之為尚書吏部郎,與右衛將軍沈演之俱參機密。頃之,轉侍中,本州大中正。遷吏部尚書,領義陽王師。內外歸附,勢傾朝野。



 炳之為人彊急而不耐
 煩,賓客幹訴非理者,忿詈形於辭色。素無術學,不為眾望所推。性好潔,士大夫造之者,去未出戶,輒令人拭席洗床。時陳郡殷沖亦好凈,小史非凈浴新衣,不得近左右。士大夫小不整潔,每容接之。炳之好潔反是,沖每以此譏焉。領選既不緝眾論,又頗通貨賄。炳之請急還家,吏部令史錢泰、主客令史周伯齊出炳之宅咨事。泰能彈琵琶,伯齊善歌,炳之因留停宿。尚書舊制,令史咨事,不得宿停外,雖有八座命,亦不許。為有司所奏。上於炳之素
 厚,將恕之,召問尚書右僕射何尚之,尚之具陳炳之得失。又密奏曰:「夫為國為家,何嘗不謹用前典,今茍欲通一人,慮非哲王御世之長術。炳之所行,非曖昧而已。臣所聞既非一旦,又往往眼見,事如丘山,彰彰若此,遂縱而不糾,不知復何以為治。晉武不曰明主,斷鬲令事,遂能奮發,華暠見待不輕,廢錮累年,後起,止作城門校尉耳。若言炳之有誠於國,未知的是何事?政當云與殷景仁不失其舊,與劉湛亦復不疏。且景仁當時事意,豈復
 可蔑,朝士兩邊相推,亦復何限,縱有微誠,復何足掩其惡。今賈充勳烈,晉之重臣,雖事業不勝,不聞有大罪,諸臣進說,便遠出之。



 陛下聖睿,反更遲遲於此。炳之身上之釁,既自藉藉,交結朋黨,構扇是非,實足亂俗傷風。諸惡紛紜,過於范曄,所少賊一事耳。伏願深加三思,試以諸聲傳,普訪諸可顧問者。群下見陛下顧遇既重,恐不敢苦相侵傷;顧問之日,宜布嫌責之旨。



 若不如此,亦當不辯有所得失。臣蠢,既有所啟,要欲盡其心,如無可納,
 伏願宥其觸忤之罪。」



 時炳之自理:「不諳臺制,令史並言停外非嫌。」太祖以炳之信受失所,小事不足傷大臣。尚之又陳曰:「炳之呼二令史出宿,令史咨都令史駱宰,宰云不通,吏部曹亦咸知不可,令史具向炳之說不得停之意,炳之了不聽納。此非為不解,直是茍相留耳。由外悉知此,而誣於信受,群情豈了,陛下不假為之辭。雖是令史,出乃遠虧朝典,又不得謂之小事。謝晦望實,非今者之疇,一事錯誤,免侍中官。



 王珣時賢小失,桓胤春搜
 之謬,皆白衣領職。況公犯憲制者邪?不審可有同王、桓白衣例不?於任使無損,兼可得以為肅戒。孔萬祀居左丞之局,不念相當,語駱宰云:『炳之貴要,異他尚書身,政可得無言耳。』又云:『不癡不聾,不成姑公。』敢作此言,亦為異也。」



 太祖猶優游之,使尚之更陳其意。尚之乃備言炳之愆過,曰:「尚書舊有增置乾二十人,以元、凱丞郎幹之假疾病,炳之常取十人私使,詢處乾闕,不得時補。



 近得王師,猶不遣還,臣令人語之,『先取人使,意常未安,今既
 有手力,不宜復留。』得臣此信,方復遣耳。大都為人好率懷行事,有諸紜紜,不悉可曉。臣思張遼之言,關羽雖兄弟,曹公父子,豈得不言。觀今人憂國實寡,臣復結舌,日月之明,或有所蔽。然不知臣者,豈不謂臣有爭競之迹,追以悵悵。臣與炳之周旋,俱被恩接,不宜復生厚薄。太尉昨與臣言,說炳之有諸不可,非唯一條,遠近相崇畏,震動四海,凡短人辦得致此,更復可嘉。虞秀之門生事之,累味珍肴,未嘗有乏,其外別貢,豈可具詳。炳之門中
 不問大小,誅求張幼緒,幼緒轉無以堪命。炳之先與劉德願殊惡,德願自持琵琶甚精麗。遺之,便復款然。市令盛馥進數百口材助營宅,恐人知,作虛買券。劉道錫驟有所輸,傾南俸之半。劉雍自謂得其力助,事之如父,夏中送甘庶,若新發於州。國吏運載樵荻,無輟於道。諸見人有物,鮮或不求。聞劉遵考有材;便乞材,見好燭盤,便復乞之。選用不平,不可一二。太尉又云,炳之都無共事之體,凡所選舉,悉是其意,政令太尉知耳。論虞秀之作
 黃門,太尉不正答和,故得停。太尉近與炳之疏,欲用德原兒作州西曹,炳之乃啟用為主簿,即語德願,德願謝太尉。前後漏泄志恩,亦復何極,縱不加罪,故宜出之。士庶忿疾之,非直項羽楚歌而已也。自從裴、劉刑罰以來,諸將陳力百倍,今日事實好惡可問。若赫然發憤,顯明法憲,陛下便可閑臥紫闥,無復一事也。」



 太祖欲出炳之為丹陽,又以問尚之。尚之答曰:「臣既乏賈生應對之才,又謝汲公犯顏之直,至於侍坐仰酬,每不能盡。昨出伏
 復深思,祇有愚滯,今之事跡,異口同音,便是彰著,政未測得物之數耳。可為蹈罪負恩,無所復少。且居官失和,未有此比。陛下遲遲舊恩,未忍窮法,為弘之大,莫復過此。方復有尹京赫赫之授,恐悉心奉國之人,於此而息;貪狼恣意者,歲月滋甚。非但虧點王化,乃治亂所由。



 如臣所聞天下論議,炳之常塵累日月,未見一豪增輝。今曲阿在水南,恩寵無異,而協首郡之榮,乃更成其形勢,便是老王雅也。古人云:『無賞罰,雖堯、舜不能為治也。』陛
 下豈可坐損皇家之重,迷一凡人。事若復在可否之間,亦不敢茍陳穴管。今之枉直,明白灼然,而睿王令王,反更不悟,令賈誼、劉向重生,豈不慷慨流涕於聖世邪!臣昔啟范曄,當時亦懼犯觸之尤,茍是愚懷所挹,政自不能不舒達,所謂雖九死而不悔者也。謂炳之且外出,若能脩改,在職著稱,還亦不難,則可得少明國典,粗酬四海之誚。今愆釁如山,榮任不損,炳之若復有彰大之罪,誰復敢以聞述。且自非殊勳異績,亦何足塞今日之尤。歷
 觀古今,未有眾過藉藉,受貨數百萬,更得高官厚祿如今者也。臣每念聖化中有此事,未嘗不痛心疾首。設令臣等數人縱橫狼藉復如此,不審當復云何處之。近啟賈充遠鎮,今亦何足分,外出恐是策之良者。臣知陛下不能採臣言,故是臣不能盡己之愚至耳。今蒙恩榮者不少,臣何為獨懇懇於斯,實是尊主樂治之意。伏願試更垂察」



 又曰:「臣見劉伯寵大慷慨炳之所行,云有人送張幼緒,幼緒語人,吾雖得一縣,負三十萬錢,庾沖遠乃
 當送至新林,見縛束,猶示得解手。荀萬秋嘗詣炳之,值一客姓夏侯,主人問『有好牛不?』云:『無。』問『有好馬不?』又云:『無。



 政有佳驢耳。』炳之便答:『甚是所欲。』客出門,遂與相聞索之。劉道錫云是炳之所舉,就道錫索嫁女具及祠器,乃當百萬數。猶謂不然。選令史章龍向臣說,亦歎其受納之過,言『實得嫁女具,銅爐四人舉乃勝,細葛鬥帳等物,不可稱數。』在尚書中,令奴酤酃酒,利其百十,亦是立臺閣所無,不審少簡聖聽不?恐仰傷日月之明,臣竊為
 之歎息。」



 太祖乃可有司之奏,免炳之官。是歲,元嘉二十五年也。二十七年,卒於家,時年六十三。太祖錄其宿誠,追復本官。二子季遠、弘遠。



 謝方明,陳郡陽夏人,尚書僕射景仁從祖弟也。祖鐵,永嘉太守。父沖,中書侍郎。家在會稽,謝病歸,除黃門侍郎,不就。為孫恩所殺,追贈散騎常侍。



 方明隨伯父吳興太守邈在郡,孫恩寇會稽,東土諸郡皆響應,吳興民胡桀、郜驃破東遷縣,方明勸邈避之,不從,賊至被害,方明逃
 竄遂免。初,邈舅子長樂馮嗣之及北方學士馮翊仇玄達,俱往吳興投邈,並舍之郡學,禮待甚簡。二人並忿慍,遂與恩通謀。恩嘗為嗣之等從者,夜入郡,見邈眾,遁,不悟。本欲於吳興起兵,事趣不果,乃遷於會稽。及郜等攻郡,嗣之、玄達並豫其謀。劉牢之、謝琰等討恩,恩走入海,嗣之等不得同去,方更聚合。方明結邈門生義故得百餘人,掩討嗣之等,悉禽而手刃之。



 于時荒亂之後,吉凶禮廢。方明合門遇禍,資產無遺,而營舉凶事,盡其力用;
 數月之間,葬送並畢,平世備禮,無以加也。頃之,孫恩重沒會稽,謝琰見害。恩購求方明甚急。方明於上虞載母妹奔東陽,由黃蘗嶠出鄱陽,附載還都,寄居國子學。流離險阨,屯苦備經,而貞立之操,在約無改。元興元年,桓玄克京邑,丹陽尹卞範之勢傾朝野,欲以女嫁方明,使尚書吏部郎王騰譬說備至,方明終不回。桓玄聞而賞之,即除著作佐郎,補司徒王謐主簿。



 從兄景仁舉為高祖中兵主簿。方明事思忠益,知無不為。高祖謂之曰:「愧
 未有瓜衍之賞,且當與卿共豫章國祿。」屢加賞賜。方明嚴恪,善自居遇,雖處闇室,未嘗有惰容。無他伎能,自然有雅韻。從兄混有重名,唯歲節朝宗而已。丹陽尹劉穆之權重當時,朝野輻輳,不與穆之相識者,唯有混、方明、郗僧施、蔡廓四人而已;穆之甚以為恨。方明、廓後往造之,大悅,白高祖曰:「謝方明可謂名家駒。



 直置便自是台鼎人,無論復有才用。」頃之,轉從事中郎,仍為左將軍道憐長史、高祖命府內眾事,皆咨決之。隨府轉中軍長史。
 尋更加晉陵太守,復為驃騎長史、南郡相,委任如初。



 嘗年終,江陵縣獄囚事無輕重,悉散聽歸家,使過正三日還到。罪應入重者有二十餘人,綱紀以下,莫不疑懼。時晉陵郡送故主簿弘季盛、徐壽之並隨在西,固諫以為:「昔人雖有其事,或是記籍過言。且當今民情偽薄,不可以古義相許。」



 方明不納,一時遣之。囚及父兄皆驚喜涕泣,以為就死無恨。至期,有重罪二人不還,方明不聽討捕。其一人醉不能歸,逮二日乃反;餘一囚十日不至,五
 官朱千期請見欲白討之,方明知為囚事,使左右謝五官不須入,囚自當反。囚逡巡墟里,不能自歸,鄉村責讓之,率領將送,遂竟無逃亡者。遠近咸歎服焉。遭母憂,去職。



 服闋,為宋臺尚書吏部郎。



 高祖受命,遷侍中。永初三年,出為丹陽尹,有能名。轉會稽太守。江東民戶殷盛,風俗峻刻,強弱相陵,姦吏蜂起,符書一下,文攝相續。又罪及比伍,動相連坐,一人犯吏,則一村廢業,邑里驚擾,狗吠達旦。方明深達治體,不拘文法,闊略苛細,務存綱領。
 州臺符攝,即時宣下,緩民期會,展其辦舉;郡縣監司,不得妄出,貴族豪士,莫敢犯禁,除比伍之坐,判久繫之獄。前後征伐,每兵運不充,悉發倩士庶;事既寧息,皆使還本。而屬所刻害,或即以補吏。守宰不明,與奪乖舛,人事不至,必被抑塞。方明簡汰精當,各慎所宜,雖服役十載,亦一朝從理,東土至今稱詠之。性尤愛惜,未嘗有所是非,承代前人,不易其政。有必宜改者,則以漸移變,使無迹可尋。元嘉三年,卒官,年四十七。



 子惠連,幼而聰敏,年十歲,能屬文,族兄靈運深相知賞,事在《靈運傳》。



 本州辟主簿,不就。惠連先愛會稽郡吏杜德靈,及居父憂,贈以五言詩十餘首,文行於世。坐被徙廢塞,不豫榮伍。尚書僕射殷景仁愛其才,因言次白太祖:「臣小兒時,便見世中有此文,而論者云是謝惠連,其實非也。」太祖曰:「若如此,便應通之。」元嘉七年,方為司徒彭城王義康法曹參軍。是時義康治東府城,城塹中得古塚,為之改葬,使惠連為祭文,留信待成,其文甚美。又
 為《雪賦》,亦以高麗見奇。文章並傳於世。十年,卒,時年二十七。既早亡,且輕薄多尤累,故官位不顯。無子。弟惠宣,竟陵王誕司徒從事中郎,臨川內史。



 江夷,字茂遠,濟陽考城人也。祖霖彡,晉護軍將軍。父敳,驃騎咨議參軍。



 夷少自藻厲,為後進之美。州辟主簿,不就。桓玄篡位,以為豫章王文學。義旗建,高祖板為鎮軍行參軍,尋參大司馬瑯邪王軍事,轉以公事免。頃之,復補主簿。豫討桓玄功,封南郡州陵縣五等侯。孟昶建威府司
 馬,中書侍郎,中軍太尉從事中郎,征西大將軍道規長史、南郡太守,尋轉太尉咨議參軍,領錄事,遷長史,入為侍中,大司馬,從府公北伐,拜洛陽園陵,進至潼關。還領寧遠將軍、琅邪內史、本州大中正。高祖命大司馬府、瑯邪國事,一以委焉。



 宋臺初建,為五兵尚書。高祖受命,轉掌度支。出為義興太守,加秩中二千石,以疾去職。尋拜吏部尚書,為吳郡太守。營陽王於吳縣見害,夷臨哭盡禮。又以兄疾去官。復為丹陽尹,吏部尚書,加散騎常
 侍,遷右僕射。夷美風儀,善舉止,歷任以和簡著稱。出為湘州刺史,加散騎常侍,未之職,病卒,時年四十八。遺命薄斂蔬奠,務存儉約。追贈前將軍,本官如故。子湛,別有傳。



 史臣曰:為國之道,食不如信,立人之要,先質後文。士君子當以體正為基,蹈義為本,然後飾以藝能,文以禮樂,茍或難備,不若文不足而質有餘也。是以小心翼翼,可祗事於上帝,嗇夫喋喋,終不離於虎圈。江夷、謝方明、謝
 弘微、王惠、王球,學義之美,未足以成名,而貞心雅體,廷臣所罕及。《詩》云:「溫溫恭人,惟德之基,」信矣!



\end{pinyinscope}