\article{卷三十四志第二十四 五行五}

\begin{pinyinscope}

 《五行傳》
 曰:「治宮室,飾臺榭,內淫亂,犯親戚,侮父兄,則稼穡不成。」



 謂土失其性而為災也。又曰:「思心不睿,是謂不聖。厥咎瞀,厥罰恆風,厥極兇短折。時則有脂夜之妖,時
 則有華孽,時則有牛禍,時則有心腹之痾,時則有黃眚、黃祥,時則有金木水火沴土。」班固曰:「不言『惟』而獨曰『時則有』者,非一沖氣所沴,明其異大也。」華孽,劉歆傳以為蠃蟲之孽,謂螟屬也。



 稼穡不成:吳孫皓時,嘗歲無水旱,苗稼豐美,而實不成,百姓以饑,皞境皆然,連歲不已。吳人以為傷露,非也。按劉向《春秋說》曰:「水旱當書,不書水旱而曰大無麥禾者,土氣不養,稼穡
 不成。」此其義也。皓初遷都武昌,尋遷建業,又起新館,綴飾珠玉,壯麗過甚,破壞諸宮,增修苑囿,犯暑妨農,官民疲怠。《月令》,「季夏不可以興土功」。皓皆冒之。此治宮室飾臺榭之罰,與《春秋》魯莊公三築臺同應也。班固曰:「無水旱之災,而草木百穀不熟,皆為稼穡不成。」



 晉穆帝永和十年,三麥不登,至關西亦然。自去秋至是夏,無水旱,無麥者,如劉向說也。又俗云,「多苗而不實為傷」,又其義也。



 恒風:
 魏齊王正始九年十一月,大風數十日,發屋折樹;十二月戊子晦,尤甚,動太極東閣。魏齊王嘉平元年正月壬辰朔,西北大風,發屋折木,昏塵蔽天。按管輅說此為時刑,大風,執政之憂也。是時曹爽區瞀目專,驕僭過度,天戒數見,終不改革。此思心不睿,恒風之罰也。後踰旬而爽等滅。京房《易傳》曰:「眾逆同志,至德乃潛,厥異風。其風也,行不解,物不長,雨小而傷。政悖德隱,茲謂亂。厥風先風不雨,大風暴起,發屋折木。
 守義不進,茲謂眊。厥風與雲俱起,折五穀莖。



 臣易上政,茲謂不順。厥風大飆發屋。賦斂不理,茲謂禍。厥風絕經紀,止即溫,溫即蟲。侯專封,茲謂不統。厥風疾而樹不搖,穀不成。闢不思道利,茲謂無澤。



 厥風不搖木,旱無雲,傷禾。公常於利,茲謂亂。厥風微而溫,生蟲蝗,害五穀。



 棄正作淫,茲謂惑。厥風溫,螟蟲起,害有益人之物。侯不朝,茲謂叛。厥風無恒,地變赤,雨殺人。」



 吳孫權太元元年八月朔,大風,江海涌溢,平地水深八
 尺,拔高陵樹二株,石碑蹉動,吳城兩門飛落。按華核對,役繁賦重,區瞀不睿之罰也。明年,權薨。



 吳孫亮建興元年十二月丙申,大風震電。是歲,魏遣大眾三道來攻,諸葛恪破其東興軍,二軍亦退。明年,恪又攻新城,喪眾大半,還伏誅。



 吳孫休永安元年十一月甲午,風四轉五,復蒙霧連日。是時孫綝一門五侯,權傾吳主,風霧之災,與漢五侯、丁、傅同應也。十二月丁卯夜,又大風,發木揚沙。



 明日,綝誅。



 晉武帝泰始五年五月辛卯朔,廣平大風折木。晉武帝咸寧元年五月,下邳、廣陵大風,壞千餘家,折樹木。咸寧元年五月甲申,廣陵、司吾、下邳大風折木。咸寧三年八月,河間大風折木。



 晉武帝太康二年五月,濟南大風,折木傷麥。太康二年六月,高平大風折木,發壞邸閣四十餘區。太康八年六月,郡國八大風。
 太康九年正月,京都風雹,發屋拔木。後二年,宮車晏駕。



 晉惠帝元康四年六月,大風雨拔樹。元康五年四月庚寅夜,暴風,城東渠波浪;七月,下邳大風,壞廬舍;九月,雁門、新興、太原、上黨災風傷稼。明年,氐、羌反叛,大兵西討。元康九年六月,飆風吹賈謐朝服,飛數百丈。明年,謐誅。元康九年十一月甲子朔,京都連大風,發屋折木。十二月,太子廢。



 晉惠帝永康元年二月,大風拔木。三月,愍懷被害。己卯,喪柩發許還洛,是日,大風雷電,幃蓋飛裂。永康元年四月,張華第舍飆風折木,飛繒軸六七。是月,華遇害。永康元年十一月戊子朔,大風從西北來,折木飛石。明年正月,趙王倫篡位。



 晉惠帝永興元年正月癸酉,趙王倫祠太廟,災風暴起,塵沙四合。其年四月,倫伏辜。



 晉元帝永昌元年七月丙寅,大風拔木,屋瓦皆飛。永昌元年八月,暴風壞屋,拔御道柳樹百餘株。其風縱橫無常,若風自八方來者。十一月,宮車晏駕。



 晉成帝咸康四年三月壬辰,成都大風,發屋折木。四月,李壽襲殺李期。



 晉康帝建元元年七月庚申,晉陵、吳郡災風。



 晉穆帝升平元年八月丁未,策立皇后何氏。是日疾風。升平五年正月戊戌朔,疾風。



 晉海西公太和六年二月,大風迅急。



 晉孝武帝寧康元年三月戊申朔,暴風迅起,從丑上來,須臾轉從子上來,飛沙揚礫。晉孝武帝太元元年二月乙丑朔,暴風折木。太元二年閏三月甲子朔,暴風疾雨俱至,發屋折木。太元二年六月,長安大風拔苻堅宮中樹。其後堅再南伐,身戮國亡。太元四年八月乙未,暴風。
 太元十二年正月壬午夜,暴風。太元十二年七月甲辰,大風拔木。太元十七年六月乙未,大風折木。



 晉安帝元興二年二月甲辰,大風雨,大航門屋瓦飛落。明年,桓玄篡位,由此門入。元興三年正月,桓玄游大航南,飄風飛其䡟輗至。三月,玄敗。元興三年五月,江陵大風折木。是月,桓玄敗於崢嶸洲,
 身亦屠裂。元興三年十一月丁酉,大風,江陵多死者。



 晉安帝義熙四年十一月辛卯朔,西北疾風起。義熙五年閏十月丁亥,大風發屋。



 明年,盧循至蔡洲。義熙六年五月壬申,大風拔北郊樹,樹幾百年也。琅邪、揚州二射堂倒壞。是日,盧循大艦漂沒。甲戌,又風,發屋折木。是冬,王師南討。義熙十年四月己丑朔,大風拔木。
 義熙十年六月辛亥,大風拔木。明年,西討司馬休之。



 宋少帝景平二年正月癸亥朔旦,暴風發殿庭,會席翻揚數十丈。五月,帝廢。



 文帝元嘉二十六年二月庚申,壽陽驟雨,有回風雲霧,廣三十許步,從南來,至城西回散滅。當其沖者,室屋樹木摧倒。元嘉二十九年三月,大風,拔木飛瓦。



 元嘉三十年正月,大風拔木,雨凍殺牛馬,雷電晦冥。二
 月,宮車晏駕。



 孝武帝大明七年,風吹初寧陵隧口左標折。鐘山通天臺新成,飛倒,散落山澗。



 明年閏五月,帝崩。



 前廢帝永光元年正月乙未朔,京邑大風。



 明帝泰始二年三月丙申,京邑大風。泰始二年四月甲子,京邑大風。泰始二年五月丁未,京邑大風。泰始二年五月己酉,京邑大風。
 泰始二年九月乙巳,京邑大風。



 後廢帝元徽二年七月甲子,京邑大風。元徽三年三月丁卯,京邑大風。元徽三年六月甲戌,京邑大風。元徽四年十一月辛卯,京邑大風。元徽五年三月庚寅,京邑大風,發屋折木。元徽五年六月甲寅,京邑大風。



 夜妖:
 魏高貴鄉公正元二年閏正月戊戌,大風晦暝,行者皆頓伏。近夜妖也。劉向曰:「正晝而暝,陰為陽,臣制君也。」時晉景王討毋丘儉,是日始發。



 魏元帝景元三年十月,京都大震,晝晦。此夜妖也。班固曰:「夜妖者,雲風並起而杳冥,故與常風同象也。」劉向《春秋說》云:「天戒若曰,勿使大夫世官,將令專事,冥晦。明年,魯季友卒,果世官而公室卑矣。」魏見此妖,晉有天下之應
 也。



 晉孝武帝太元十三年十二月乙未,大風晦暝。其後帝崩,而諸侯違命,干戈內侮,權奪於元顯,禍成於桓玄。是其應也。



 蠃蟲之孽晉孝武咸寧元年七月,郡國螟;九月,青州又螟。咸寧元年七月,郡國有青蟲食禾稼。咸寧四年,司、冀、兗、豫、荊、揚郡國皆螟。



 晉武帝太康四年,會稽彭蜞及蟹皆化為鼠,甚眾,覆野,
 大食稻為災。太康九年八月,郡國二十四螟,螟說與蝗同。是時帝聽讒訴。太康九年九月,蟲傷稼。



 晉惠帝元康二年九月,帶方、含資、提奚、南新、長岑、海冥、列口蟲食禾葉蕩盡。



 晉惠帝永寧元年七月,梁、益、涼三州螟。是時齊王冏秉政。貪苛之應也。
 永寧元年十月,南安、巴西、江陽、太原、新興、北海青蟲食禾葉,甚者十傷五六。



 永寧元年十二月,郡國八螟。



 牛禍:晉武帝太康九年,幽州塞北有死牛頭語。近牛禍也。是時帝多疾病,深以後事為念,而託付不以至公,思心瞀亂之應也。師曠曰:「怨昚動於民,則有非言之物而言。」又其義也。



 晉惠帝太安中,江夏張騁所乘牛言曰:「天下方亂,乘我何之!」騁懼而還,犬又言曰:「歸何蚤也。」尋後牛又人立而行。騁使善卜者卦之。謂曰:「天下將有兵亂,為禍非止一家。」其年張昌反,先略江夏,騁為將帥。於是五州殘亂,騁亦族滅。京房《易妖》曰:「牛能言,如其言占吉凶。」《易萌氣樞》曰:「人君不好士,走馬被文繡,犬狼食人食,則有六畜妖言。」時天子諸侯不以惠下為務,又其應也。



 晉愍帝建武元年,曲阿門牛生犢,一體兩頭。



 元帝太興元年,武昌太守王諒牛生子,兩頭八足,兩尾共一腹。三年後死。又有牛生一足三尾,皆生而死。按司馬彪說,兩頭者,政在私門,上下無別之象也。



 京房《易傳》曰:「足多者,所任邪也。足少者,下不勝任也。」其後皆有此應。



 晉元帝太興四年十二月,郊牛死。按劉向說《春秋》郊牛死曰,宣公區瞀昏亂,故天不饗其祀。元帝中興之業,實王導之謀也。劉隗探會主意,以得親幸,導見疏外。



 此區
 瞀不睿之禍也。



 晉成帝咸和二年五月,護軍牛生犢,兩頭六足。是冬,蘇峻作亂。咸和七年,九德民袁榮家牛產犢,兩頭八足,二尾共身。京房《易傳》:「殺無罪,則牛生妖。」



 桓玄之國在荊州,詣刺史殷仲堪,行至鶴穴,逢一老公,驅青牛,形色瑰異。



 桓玄即以所乘牛易取。乘至零陵涇溪,駿駛非常,因息駕飲牛。牛徑入江水不出。



 玄遣人覘
 守,經日無所見。



 宋文帝元嘉三年,司徒徐羨之大兒喬之行欲入廣莫門。牛徑將入廷尉寺,左右禁捉不能禁。入方得出。明日被收。元嘉二十九年,晉陵送牛,角生右脅,長八尺。



 明年二月,東宮為禍。



 孝武帝大明三年,廣州刺史費淹獻三角水牛。



 黃眚黃祥:
 蜀劉備章武二年,東伐。二月,自秭歸進屯夷道。六月,秭歸有黃氣見,長十餘里,廣數十丈。後踰旬,備為陸遜所破。近黃祥也。



 魏齊王正始中,中山王周南為襄邑長。有鼠從穴出,語曰:「王周南,爾以某日死。」南不應。鼠還穴。後至期,更冠幘皂衣出,語曰:「周南,汝日中當死。」



 又不應。鼠復入,斯須更出,語如向日。適欲日中,鼠入復出,出復入,轉更數語如前。日適中,鼠曰:「周南,汝不應我,復何道。」言絕,顛蹶而死,
 即失衣冠。



 取視,俱如常鼠。案班固說,此黃祥也。是時曹爽秉政,競為比周,故鼠作變也。



 宋孝武大明七年春,太湖邊忽多鼠。其年夏,水至,悉變成鯉魚。民人一日取,轉得三五十斛。明年,大飢。



 晉元帝太興四年八月,黃霧四塞,埃氣蔽天。案楊宣對,近土氣,亂之祥也。



 晉元帝永昌二年正月癸巳,黃霧四塞。



 晉穆帝永和七年三月,涼州大風拔木,黃霧下塵。是時
 張重華納譖,出謝艾為酒泉太守,而所任非其人。至九年死,嗣子見弒。是其應也。京房《易傳》曰:「聞善不予,茲謂不知。厥異黃,厥咎聾,厥災不嗣。黃者,有黃濁氣四塞天下,蔽賢絕道,故災至絕世也。」



 晉安帝元興元年十月丙申朔,黃霧昏濁,不雨。



 宋文帝元嘉十八年秋七月,天有黃光,洞照于地。太子率更令何承天謂之榮光,太平之詳,上表稱慶。



 地震:
 吳孫權黃武四年,江東地連震。是時權受魏爵命,為大將軍、吳王,改元專制,不修臣跡。京房《易傳》曰:「臣事雖正,專必震。」董仲舒、劉向並云「臣下彊盛,將動而為害」之應也。



 魏明帝青龍二年十一月,京都地震,從東來,隱隱有聲,屋瓦搖。魏明帝景初元年六月戊申,京都地震。是秋,吳將朱然圍江夏,荊州刺史胡質擊退之。又公孫淵自立為燕王,改年,置百官。明年,討平之。



 吳孫權嘉禾六年五月,江東地震。赤烏二年正月,地又再震。是時呂壹專政,步騭上疏曰:「伏聞校事,吹毛求瑕,趣欲陷人,成其威福,無罪無辜,橫受重刑,雖有大臣,不見信任。如此,天地焉得無變。故嘉禾六年、赤烏二年,地連震動,臣下專政之應也。冀所以警悟人主,可不深思其意哉!」壹後卒敗。



 魏齊王正始二年十一月,南安郡地地震。正始三年七月甲申,南安郡地震;十二月,魏郡地震。
 正始六年二月丁卯,南安郡地震。是時曹爽專政,遷太后于永寧宮,太后與帝相泣而別。連年地震,是其應也。



 吳孫權赤烏十一年二月,江東地仍震。是時權聽讒,尋黜朱據,廢太子。



 蜀劉禪炎興元年,蜀地震。時宦人黃皓專權。按司馬彪說,奄宦無陽施,猶婦人也。此皓見任之應,與漢和帝時同事也。是冬,蜀亡。



 晉武帝泰始五年四月辛酉,地震。是年冬,新平氐、羌叛。
 明年,孫皓大遺眾入渦口。叛虜寇秦、涼,刺史胡烈、蘇愉並為所害。泰始七年六月丙申,地震。武帝世,始於賈充,終於楊駿,阿黨昧利,茍專權寵,終喪天下,由是也。末年所任轉敝,故亦一年六震,是其應也。裴叔則曰:「晉德所以不比隆堯、舜者,以有賈充諸人在朝。」



 晉武帝咸寧二年八月庚辰,河南、河東、平陽地震。咸寧四年六月丁未,陰平、廣武地震;甲子,陰平、廣武地
 又震。



 晉武帝太康二年二月庚申,淮南、丹陽地震。太康五年正月壬辰,地震。太康六年七月己丑,地震。太康七年七月,南安、犍為地震;八月,京兆地震。太康八年五月壬子,建安地震;七月,陰平地震;八月,丹陽地震。太康九年正月,會稽、丹陽、吳興地震;四月辛酉,長沙、南
 海等郡國八地震;七月至于八月,地又四震,其三有聲如雷。太康十年十二月己亥,丹陽地震。



 晉武帝太始元年,地震。



 晉惠帝元康元年十二月辛酉,京都地震。元康四年二月,蜀郡山崩殺人;上谷、上庸、遼東地震。五月壬子,壽春山崩,洪水出,城壞,地墜方三十丈,水出殺人。



 六月,壽春大雷震,山崩地坼,家人陷死,上庸郡亦如之。八月,上
 谷地震,水出,殺百餘人。居庸地裂,廣三十六丈,長八十四丈,水出,大饑。上庸四處山崩地陷,廣三十丈,長百三十丈,水出殺人。十月,京都地震;十一月,滎陽、襄城、汝陰、梁國、南陽地皆震;十二月,京都又震。是時賈后亂朝,據權專制,終至禍敗之應也。漢鄧太后攝政時,郡國地震。李固以為:「地,陰也,法當安靜。今乃越陰之職,專陽之政,故應以震。」此同事也。京房《易傳》曰:「無德專祿,茲謂不順。



 厥震動,丘陵涌水出。」又曰:「小人剝廬,厥妖山崩。茲謂陰
 乘陽,弱勝彊。」



 又曰:「陰背陽,則地裂。父子分離,夷、羌叛去。」元康五年五月丁丑,地震;六月,金城地震。元康六年正月丁丑,地震。元康八年正月丙辰,地震。



 晉惠帝太安元年十月,地震。是時齊王冏專政。太安二年十二月丙辰,地震。



 是時長沙王專政。



 晉孝懷帝永嘉三年十月,荊、湘二州地震。時司馬越專政。
 永嘉四年四月,兗州地震。



 晉愍帝建興二年四月甲辰,地震。是時幼主在上,權傾於下,四方雲擾,兵亂不息。建興三年六月丁卯,長安地震。



 晉元帝太興元年四月,西平地震,涌水出;十二月,廬陵、豫章、武昌、西陵地震,山崩。干寶曰:「王敦陵上之應。」太興二年五月癸丑,祁山地震,山崩殺人。是時相國南陽王保在祁山稱晉王,不終之象也。
 太興三年四月庚寅,丹陽、吳郡、晉陵地震。其年,南平郡山崩,出雄黃數千斤。



 晉成帝咸和二年三月,益州地震;四月己未,豫章地震。是年,蘇峻作亂。咸和九年三月丁酉,會稽地震。是時政在臣下。



 晉穆帝永和元年六月癸亥,地震。是時嗣主幼沖,母后稱制,政在臣下,所以連年地震。永和二年十月,地震。
 永和三年正月丙辰,地震。永和四年十月己未,地震。永和五年正月庚寅,地震。永和九年八月丁酉,京都地震,有聲如雷。永和十年正月丁酉,地震,有聲如雷,雞雉鳴呴。永和十一年四月乙酉,地震;五月丁未,地震。



 晉穆帝升平五年八月,涼州地震。



 晉哀帝隆和元年四月甲戌,地震。是時政在將相,人主
 南面而已。隆和元年四月丁丑,涼州地震,浩亹山崩。張天錫降亡之象也。隆和二年二月庚寅,江陵地震。



 是時桓溫專政。



 晉海西太和元年二月,涼州地震水涌。



 晉簡文帝咸安二年十月辛未,安成地震。



 晉孝武帝寧康元年十月辛未,地震。是時嗣主幼沖,政在將相。
 寧康二年七月甲午,涼州地震山崩。



 晉孝武帝太元二年閏月壬午,地震;五月丁丑,地震。太元十一年六月己卯,地震。是後緣河諸將,連歲兵役。太元十五年三月己酉朔夜,地震。太元十七年六月癸卯,地震;十二月己未,地又震。是時群小弄權,天下側目。太元十八年正月癸亥朔,地震;二月乙未,地震。



 晉安帝隆安四年九月癸酉,地震。是時幼主沖昧,政在
 臣下。晉安帝義熙四年正月壬子夜,地震有聲;十月癸亥,地震。義熙五年正月戊戌夜,尋陽地震,有聲如雷。明年,盧循下。義熙八年,自正月至四月,南康、廬陵地四震。明年,王旅西討荊、益。



 宋文帝元嘉七年四月丙辰,地震。時遣軍經略司、兗。
 元嘉十二年四月丙辰,京邑地震。元嘉十五年七月辛未,地震。元嘉十六年,地震。



 孝武帝大明二年四月辛丑,地震。大明六年七月甲申,地震,有聲自河北來,魯郡山搖地動,彭城城女牆四百八十丈墜落,屋室傾倒,兗州地裂泉涌,二年不已。



 其後虜主死,兗州刺史夏侯祖權卒。



 明帝泰始二年四月,地震。
 泰始四年七月己酉,東北有聲如雷,地震。明帝泰豫元年閏七月甲申,東北有聲如雷,地震。



 後廢帝元徽二年四月戊申,地震。元徽五年五月戊申,地震。七月,帝殞。



 宋文帝元嘉二十五年,青州城南地,遠望見地中如水有影,人馬百物皆見影中,積年乃滅。



 山崩地陷裂:吳孫權赤烏十三年八月,丹陽、句容及故鄣、寧國諸山
 崩,鴻水溢。按劉向說,「山,陽,郡也;水,陰,民也。天戒若曰,君道崩壞,百姓將失其所也。」與《春秋》梁山崩,漢齊、楚眾山發水同事也。「夫三代命祀,祭不越望,吉凶禍福,不是過也」。吳雖帝,其實列國,災發丹陽,其天意矣。國主山川,山崩川竭,亡之徵也。後二年而權薨,薨二十六年而吳亡。



 魏元帝咸熙二年二月,太行山崩。此魏亡之徵也。其冬,晉有天下。



 晉武帝太始三年三月戊子,太行山崩。
 太始四年七月,泰山崩,墜三里。此晉之咎徵也。至帝晏駕,而祿去王室,懷、愍淪胥於北,元帝中興於南,是其應也。



 京房《易傳》曰:「自上下者為崩,厥應泰山之石顛而下,聖王受命,人君虜。」



 晉武帝太康五年丙午,宣帝廟地陷。太康六年三月,南安新興縣山崩,涌水出。



 太康七年七月,朱提之大瀘山崩,震壞郡舍;陰平之仇池崖隕。
 太康八年七月,大雨。殿前地陷,方五尺,深數丈。



 晉惠帝元康四年五月壬子,地陷,方三十丈,殺人。史闕其處。元康四年八月,居庸地裂,廣三十丈,長百三十丈,水出殺人。



 晉孝懷帝永嘉元年三月,洛陽東北步廣里地陷。永嘉三年八月乙亥,鄄城城無故自壞七十餘丈,司馬越惡之,遷于濮陽。此見沴之異也。越卒陵上,終亦受禍。



 永嘉三年七月戊辰,當陽地裂三所,所廣三丈,長二百餘步。京房《易傳》曰:「地坼裂者,臣下分離,不肯相從也。」其後司馬越、茍晞交惡,四方牧伯莫不離散,王室遂亡。永嘉三年十月,宜都夷道山崩。永嘉四年四月,湘東酃黑石山崩。



 晉元帝太興四年八月,常山崩,水出,滹沱盈溢,大木傾拔。



 晉成帝咸和四年十月,柴桑廬山西北崖崩。十二月,劉
 胤為郭默所殺。



 晉惠帝元康九年六月夜,暴雷雨。賈謐齋屋柱陷入地,壓謐床帳。此木沴土,土失其性,不能載也。明年,謐誅。晉惠帝光熙元年五月,范陽地然,可以爨。此火沴土也。是時禮樂征伐自諸侯出。



 晉安帝義熙八年三月壬寅,山陰有聲如雷,地陷深廣各四尺。義熙十年五月戊寅,西明門地穿,涌水出,毀門扇及限。
 此水沴土也。



 《五行傳》曰:「皇之不極,是謂不建。厥咎眊,厥罰恒陰,厥極弱。時則有射妖,時則有龍蛇之孽,時則有馬禍,時則有下人伐上之痾,時則有日月亂行,星辰逆行。」



 常陰吳孫亮太平三年,自八月沈陰不雨,四十餘日。是時將誅孫綝,謀泄。九月戊午,綝以兵圍宮,廢亮為會稽王。此常陰之罰也。



 吳孫皓寶鼎元年十二月,太史奏久陰不雨,將有陰謀。皓深驚懼。時陸凱等謀因其謁廟廢之。及出,留平領兵前驅,凱語平,平不許,是以不果。皓既肆虐,群下多懷異圖,終至降亡。



 宋後廢帝元徽三年四月,連陰不雨。元徽三年八月,多陰。後二年,廢帝殞。



 射妖:蜀車騎將軍鄧芝征涪陵,見玄猿緣山,手射中之。猿拔
 其箭,卷木葉塞其創。



 芝曰:「嘻!吾違物之性,其將死矣。」俄而卒。此射妖也。一曰猿母抱子,芝射中之,子為拔箭,取木葉塞創。芝歎息,投弓水中,自知當死矣。



 晉恭帝之為琅邪王時,好奇戲,嘗閉一馬於門內,令人射之,欲觀幾箭而死。



 左右有諫者,曰:「馬,國姓也,而今射之,不祥甚矣。」於是乃止,而馬已被十許箭矣。此蓋射妖也。俄而桓玄篡位。



 龍蛇之孽:
 魏明帝青龍元年正月甲申,青龍見郟之摩陂井中。凡瑞興非時,則為妖孽,況困於井,非嘉祥矣。魏以改年,非也。晉武不賀,是也。干寶曰:「自明帝終魏世,青龍黃龍見者,皆其主廢興之應也。魏,土運;青,木色也,而不勝於金。黃得位,青失位之象也。青龍多見者,君德國運內相剋伐也。故高貴鄉公卒敗于兵。案劉向說:『龍貴象,而困井中,諸侯將有幽執之禍也。』魏世龍莫不在井,此居上者逼制之應。高貴鄉公著《潛龍詩》,即此旨也。」



 魏高貴鄉公正元元年冬十月戊戌,黃龍見于鄴井中。魏高貴鄉公甘露元年正月辛丑,青龍見軹縣井中;六月乙丑,青龍見元城縣界井中。甘露二年二月,青龍見溫縣井中。甘露三年,黃龍青龍仍見頓丘、冠軍、陽夏縣界井中。



 景元三年二月,青龍見軹縣井中。



 吳孫皓天冊中,龍乳於長沙民家,啖雞雛。京房《易妖》曰:「龍乳人家,王者為庶人。」其後皓降。



 晉武帝咸寧二年六月丙申,白龍二見于九原井中。晉武帝太康五年正月癸卯,二龍見于武庫井中。帝見龍,有喜色,百僚將賀。劉毅獨表曰:「昔龍漦夏庭,禍發周室;龍見鄭門,子產不賀。」帝答曰:「朕德政未修,未有以膺受嘉祥。」遂不賀也。孫盛曰:「龍,水物也,何與於人,子產言之當矣。但非其所處,實為妖災。夫龍以飛翔顯見為美,則潛伏幽處,非休祥也。漢惠帝二年,兩龍見蘭陵井中,本志以為其後趙王幽死之象也。武庫者,帝王威御之
 器所寶藏也,室宇邃密,非龍所處。後七年,蕃王相害,二十八年,果有二胡僭竊神器。勒、虎二逆皆字曰龍,此之表異,為有證矣。」史臣案龍為休瑞,而屈於井中,前史言之已祥。但兆幽微,非可臆斷,故《五行》、《符瑞》兩存之。



 晉愍帝建興二年十一月,桴罕羌妓產一龍子,色似錦文,嘗就母乳,遙見神光,少得就視。



 晉武帝咸寧中,司徒府有二大蛇,長十許丈,居聽事平橑上,數年而人不知,但怪府中數失小兒及豬犬之屬。
 後一蛇夜出,傷於刃,不能去,乃覺之。發徒攻擊,移時乃死。夫司徒五教之府,此皇極不建,故蛇孽見之。漢靈帝時,蛇見御座,楊賜以為帝溺於色之應也。魏氏宮人猥多,晉又過之,宴游是湎,此其孽也。《詩》云:「惟虺惟蛇,女子之祥。」



 晉惠帝元康五年三月癸巳,臨菑有大蛇長十餘丈,負二小蛇,入城北門,徑從市入漢城陽景王祠中不見。天戒若曰,齊方有劉章定傾之功,若不厲節忠慎,又將蹈章
 失職奪功之辱也。齊王冏不悟,雖建興復之功,而以驕陵取禍。負二小蛇出朝市,皆有象類也。



 晉明帝太寧初,武昌有大蛇,常居故神祠空樹中,每出頭從人受食。京房《易妖》曰:「蛇見於邑,不出三年,有大兵。國有大憂。」其後討滅王敦及其黨與。



 馬禍:晉武帝太熙元年,遼東有馬生角,在兩耳下,長三寸。按劉向說,此兵象也。



 及帝晏駕之後,王室毒於兵禍,是其
 應也。京房《易傳》曰:「臣易上政厥妖馬生角。」又有「天子親伐,馬生角」。《呂氏春秋》曰:「人君失道,馬有生角。」



 晉惠帝元康元年十二月,皇太子將釋奠,太傅趙王倫驂乘,至南城門,馬止,力士推之不能動。倫入軺車,乃進。此馬禍也。天戒若曰,倫不知義方,終為亂逆,非傅導行禮之人。倫不悟,故亡。元康九年十一月戊寅冬,有牝騮馬驚奔至廷尉訊堂,悲鳴而死。是殆愍懷冤死之象也。見廷尉訊堂,又天意
 乎!



 晉孝懷帝永嘉六年二月,神馬鳴南城門。



 晉元帝大興二年,丹陽郡吏濮陽楊演馬生駒,兩頭自頸前別,生而死。按司馬彪說,政在私門,二頭之象也。是後王敦陵上。



 晉成帝咸康八年五月甲戌,有馬色赤如血,自宣陽門直走入于殿前,盤旋走出,尋逐莫知所在。己卯,帝不豫,六月崩。此馬禍,又赤祥也。張重華在涼州,將誅其西河
 相張祚,祚廄馬數十匹,同時悉皆無後尾。



 晉安帝隆安四年十月,梁州有馬生角,刺史郭銓送示都督桓玄。案劉向說,馬不當生角,由玄不當舉兵向上也。睹災不悟,故至夷滅。



 人痾:魏文帝黃初初,清河宋士宗母化為鱉,入水。



 魏明帝太和三年,曹休部曲兵奚農女死復生。時人有開周世塚,得殉葬女子,數日而有氣,數月而能語。郭太
 后愛養之。又太原民發塚破棺,棺中有一生婦人,問其本事,不知也。視其墓木,可三十歲。案京房《易傳》,至陰為陽,下人為上,晉宣王起之象也。漢平帝、獻帝並有此異,占以為王莽、曹操之徵。公孫淵炊,有小兒蒸死甑中,其後夷滅。



 吳孫亮建興二年,諸葛恪將征淮南,有孝子著衰衣入其閣。詰問,答曰:「不自覺入也。」時中外守備,亦悉不見。眾皆異之。及還,果見殺。恪已被害,妻在室,使婢沃盥,聞婢
 血曈。又眼目視瞻非常,妻問其故,婢蹶然躍起,頭至棟,攘臂切齒曰:「諸葛公乃為峻所殺。」



 吳孫休永安四年,安吳民陳焦死七日,復穿塚出。干寶曰:「此與漢宣帝同事。



 烏程侯皓承廢故之家,得位之祥也。」



 吳孫皓寶鼎元年,丹陽宣騫母,年八十,因浴化為黿。兄弟閉戶衛之,掘堂上作大坎,實水其中。黿入坎戲一二日,恒延頸外望,伺戶小開,便輪轉自躍,入于遠潭,遂不
 復還。與漢靈帝時黃氏母事同,吳亡之象也。



 魏元帝咸熙二年八月,襄武縣言有大人見,長三丈餘,跡長三尺二寸,髮白,著黃巾黃單衣,柱杖,呼民王始語曰:「今當太平。」尋晉代魏。



 晉武帝泰始五年,元城人年七十,生角。案《漢志》說,殆趙王倫篡亂之象也。



 晉武帝咸寧二年二月,琅邪人顏畿病死,棺斂已久,家人咸夢畿謂己曰:「我當復生,可急開棺。」遂出之。漸能飲
 食屈申視瞻,不能行語也。二年復死。其後劉淵、石勒遂亡晉室。



 晉惠帝元康中,安豐有女子周世寧,年八歲,漸化為男,至十七八,而氣性成。



 此劉淵、石勒蕩覆晉室之妖也。漢哀帝、獻帝時並有此異,皆有易代之兆。京房《傳》曰:「女子化為丈夫,茲謂陰昌,賤人為王。丈夫化為女子,茲謂陰勝陽,厥咎亡。」



 晉惠帝永寧初,齊王冏唱義兵,誅除亂逆,乘輿反正。忽
 有婦人詣大司馬門求寄產。門者詰之,婦人曰:「我截齊便去耳。」是時齊王冏匡復王室,天下歸功。



 識者為其惡之。後果斬戮。永寧元年十二月甲子,有白頭公入齊王冏大司馬府,大呼有大兵起,不出甲子旬,冏殺之。明年十二月戊辰,冏敗,即甲子旬也。



 晉惠帝太安元年四月癸酉,有人自雲龍門入殿前,北面再拜曰:「我當作中書監。」即收斬之。干寶曰:「夫禁庭,尊
 秘之處。今賤人徑入,而門衛不覺者,宮室將虛,而下人踰上之妖也。」是後帝北遷鄴,又西遷長安,盜賊蹈籍宮闕,遂亡天下。



 晉惠帝世,梁國女子許嫁,已受禮娉,尋而其夫戍長安,經年不歸。女家更以適人,女不樂行,其父母逼強,不得已而去,尋得病亡。後其夫還,問女所在,其家具說之。其夫徑至女墓,不勝哀情,便發塚開棺,女遂活,因與俱歸。後婿聞之,詣官爭之,所在不能決。秘書郎王導議曰:「此
 是非常事,不得以常理斷之,宜還前夫。」朝廷從其議。



 晉惠帝世,杜錫家葬,而婢誤不得出。後十餘年,開塚祔葬,而婢尚生。其始如瞑,有頃漸覺。問之,自謂當一再宿耳。初婢之埋,年十五六,及開冢更生,猶十五六也。嫁之有子。晉惠帝光熙元年,會稽謝真生子,大頭有鬢,兩蹠反向上,有男女兩體。生便作丈夫聲,經日死。



 晉惠、懷之世,京、洛有兼男女體,亦能兩用人道,而性尤
 淫。案此亂氣之所生也。自咸寧、太康之後,男寵大興,甚於女色,士大夫莫不尚之,天下皆相放效,或有至夫婦離絕,怨曠妒忌者。故男女氣亂,而妖形作也。



 元帝太興初,又有女子陰在腹上,在揚州,性亦淫。京房《易妖》曰:「人生子,陰在首,天下大亂;在腹,天下有事;在背,天下無後。」



 晉孝懷帝永嘉元年,吳郡吳縣萬祥婢生子,鳥頭,兩足馬蹄,一手無毛,黃色,大如枕。



 晉愍帝建興四年,新蔡縣吏任僑妻胡,年二十五,產二女,相向,腹心合同,自胸以上,齊以下,各分。此蓋天下未一之妖也。時內史呂會上言:「案《瑞應圖》,異根同體謂之連理,異苗同穎謂之嘉禾。草木之異,猶以為瑞,今二人同心,《易》稱『二人同心,其利斷金』。嘉徵顯見,生於陜東之國,斯蓋四海同心之瑞,不勝喜踴,謹畫圖以上。」時有識者哂之。



 晉中興初,有女子,其陰在腹,當齊下。自中國來江東,性
 甚淫,而不產。京房《易妖》曰:「人生子,陰在首,天下大亂;在腹,天下有事;在背,天下無後。」



 晉元帝太興三年十二月,尚書騶謝平妻生女,墮地濞濞有聲,須臾便死。鼻目皆在頂上,面處如項,口有齒,都連為一,胸如鱉,手足爪如鳥爪,皆下句。京房《易妖》曰:「人生他物,非人所見者,皆為天下大兵。」後二年,有石頭之敗。



 晉明帝太寧二年七月,丹陽江寧侯紀妻死,三日復生。



 晉成帝咸康四年十一月辛丑,有何一人詣南止車門自列為聖人所使。錄付光錄外部檢問,是東海郯縣呂暢,辭語落漠,髡鞭三百,遣。咸康五年四月,下邳民王和僑居暨陽。息女可,年二十,自云:「上天來還,得征瑞印綬,當母天下。」晉陵太守以為妖,收付獄。至十一月,有人持柘杖,絳衣,詣止車門口,列為聖人使,求見天子。門候受辭,列姓呂名錫。云王和女可,右足下有七星,星皆有毛,長七寸,天今命可為天下
 母。奏聞,即伏誅。并下晉陵誅可。



 晉康帝建元二年十月,衛將軍營督過望所領兵陳瀆女壹,有文在足,曰「天下之母」。灸之逾明,京都喧嘩。有司收系以聞,俄自建康縣獄亡去。



 石虎末,大武殿前所圖賢聖人像人頭,忽悉縮入肩中。



 晉孝武帝寧康初,南郡州陵女人唐氏,漸化為丈夫。



 晉安帝義熙七年,無錫人趙朱,年八歲,一旦暴長八尺,髭鬚蔚然,三日而死。



 義熙中,東陽人黃氏生女不養,埋之。數日於土中啼,取養遂活。義熙末,豫章吳平人有二陽道,重累生。



 晉恭帝元熙元年,建安人陽道無頭正平,本下作女人形體。



 宋文帝元嘉十七年,劉斌為吳郡。婁縣有一女,忽夜乘風雨,怳忽至郡城內。



 自覺去家正炊頃,衣不沾濡。曉在門上求通,言:「我天使也。」斌令前,因曰:「府君宜起迎我,當
 大富貴。不爾,必有凶禍。」斌問所以來,亦不自知也。謂是狂人,以付獄,符其家迎之。數日乃得去。後二十日許,斌誅。



 孝武帝大明中,張暢為會稽郡,妾懷孕,兒於腹中啼,聲聞於外。暢尋死。大明末,荊州武寧縣人楊始歡妻,於腹中生女兒。此兒至今猶存。



 明帝泰豫元年正月,巨人見太子西池水上,跡長三尺
 餘。



 後廢帝元徽中,南東莞徐坦妻懷孕,兒在腹中有聲。元徽中,暨陽縣女人於黃山穴中得二卵,如斗大,剖視有人形。



 魏文帝黃初四年三月,宛、許大疫,死者萬數。



 魏明帝青龍二年四月,大疫。青龍三年正月,京都大疫。



 吳孫權赤烏五年,大疫。
 吳孫亮建興二年四月,諸葛恪圍新城。大疫,死者太半。吳孫皓鳳凰二年,疫。



 晉武帝泰始十年,大疫。吳土亦同。晉武帝咸寧元年十一月,大疫,京都死者十萬人。晉武帝太康三年春,疫。



 晉惠帝元康二年十一月,大疫。元康七年五月,秦、雍二州疾疫。



 晉孝懷帝永嘉四年五月,秦、雍州饑疫至秋。
 永嘉六年,大疫。



 晉元帝永昌元年十一月,大疫,死者十二三;河朔亦同。



 晉成帝咸和五年五月,大饑且疫。



 晉穆帝永和九年五月,大疫。



 晉海西太和四年冬,大疫。



 晉孝武帝太元五年五月,自冬大疫,至于此夏,多絕戶者。



 晉安帝義熙元年十月,大疫,發赤班乃愈。
 義熙七年春,大疫。



 宋文帝元嘉四年五月,京都疾疫。



 孝武帝大明元年四月,京邑疾疫。大明四年四月,京邑疾疫。



 日蝕:魏文帝黃初二年六月戊辰晦,日有蝕之。有司奏免太尉。詔曰:「災異之作,以譴元首,而歸過股肱,豈禹、湯罪己之義乎?其令百官各虔厥職。後有天地眚,勿復劾三公。」
 黃初三年正月丙寅朔,日有蝕之;十一月庚申晦,又日有蝕之。黃初五年十一月戊申晦,日有蝕之。後二年,宮車晏駕。



 魏明帝太和初,太史令許芝奏日應蝕,與太尉於靈臺祈禳。帝詔曰:「蓋聞人主政有不得,則天懼之以災異,所以譴告使得自修也。故日月薄蝕,明治道有不當者。朕即位以來,既不能光明先帝聖德,而施化有不合於皇神,故上天有以寤之。



 宜勵政自修,以報于神明。天之於
 人,猶父之於子,未有父欲責其子,而可獻盛饌以求免也。今外欲遣上公與太史令具禳祠,於義未聞也。群公卿士,其各勉修厥職。


有可以補朕不逮者,各封上之。」魏明帝太和五年十一月戊戌晦,日有蝕之。太和六年正月戊辰朔,日有蝕之。
 \gezhu{
  見《吳歷》}
 。



 魏明帝青龍元年閏月庚寅朔,日有蝕之。



 魏齊王正始元年七月戊申朔,日有蝕之。《紀》無。正始三年四月戊戌朔,日有蝕之。《紀》無。
 正始六年四月壬子,日有蝕之;十月戊寅朔,又日有蝕之。正始八年二月庚午朔,日有蝕之。是時曹爽專政,丁謐、鄧颺等轉改法度。會有日蝕變,詔群臣問得失。蔣濟上疏曰:「昔大舜佐治,戒在比周;周公輔政,慎於其朋。



 齊侯問災,晏子對以布惠;魯君問異,臧孫答以緩役。塞變應天,乃實人事。」濟旨譬甚切,而君臣不悟,終至敗亡矣。正始九年正月乙未朔,日有蝕之。



 魏齊王嘉平元年二月己未,日有蝕之。



 魏高貴鄉公甘露四年七月戊子朔,日有蝕之。甘露五年正月乙酉朔,日有蝕之。



 按谷永說,正朝,尊者惡之。京房占曰:「日蝕乙酉,君弱臣強。司馬將兵,反徵其王。」五月,有成濟之變。



 魏元帝景元二年五月丁未朔,日有蝕之。景元三年三月己亥朔,日有蝕之。



 晉武帝泰始二年七月丙午晦,日有蝕之。
 泰始七年五月庚辰,日有蝕之。泰始八年十月辛未朔,日有蝕之。泰始九年四月戊辰朔,日有蝕之。泰始十年三月癸亥,日有蝕之。



 晉武帝咸寧元年七月甲申晦,日有蝕之。咸寧三年正月丙子朔,日有蝕之。



 晉武帝太康四年三月辛丑朔,日有蝕之。太康六年八月丙戌朔,日有蝕之。
 太康七年正月甲寅朔,日有蝕之。乙亥,詔曰:「比年災異屢發,邦之不臧,實在朕躬。震蝕之異,其咎安在?將何施行,以濟其愆?」太尉亮、司徒舒、司空瓘遜位,弗許。太康八年正月戊申朔,日有蝕之。太康九年六月庚子朔,日有蝕之。後二年,宮車晏駕。



 晉惠帝元康九年十月甲子朔,日有蝕之。晉惠帝永康元年四月辛卯朔,日有蝕之。晉惠帝永寧元年閏三月丙戌朔,日有蝕之。
 晉惠帝光熙元年正月戊子朔,日有蝕之。尊者惡之。七月乙酉朔,又日有蝕之既。占曰:「日蝕盡,不出三月,國有凶。」十一月,宮車晏駕。十二月壬午朔,又日有蝕之。



 晉孝懷帝永嘉元年十一月戊申,日有蝕之。永嘉二年正月丙午朔,日有蝕之。



 永嘉六年二月壬子朔,日有蝕之。明年,帝崩于平陽。



 晉愍帝建興四年六月丁巳朔,日有蝕之。十一月,帝為劉曜所虜。十二月乙卯朔,又日有蝕之。明年,帝崩于平
 陽。



 晉元帝太興元年四月丁丑朔,日有蝕之。



 晉明帝太寧三年十一月癸巳朔,日有蝕之。



 晉成帝咸和二年五月甲申朔,日有蝕之。晉成帝咸康元年十月乙未朔,日有蝕之。咸康七年二月甲子朔,日有蝕之。咸康八年正月乙未朔,日有蝕之。正朝,尊者惡之。六月,宮車晏駕。



 晉穆帝永和七年正月丁酉朔,日有蝕之。永和十二年十月癸巳朔,日有蝕之。



 晉穆帝升平四年八月辛丑朔,日有蝕之,不盡如鉤。明年,宮車晏駕。



 晉哀帝隆和元年十二月戊午朔,日有蝕之。



 晉海西公太和三年三月丁巳朔,日有蝕之。太和五年七月癸酉朔,日有蝕之。



 明年,廢為海西公。



 晉孝武帝寧康三年十月癸酉朔,日有蝕之。
 晉孝武帝太元四年閏月己酉朔,日有蝕之。太元六年六月庚子朔,日有蝕之。太元九年十月辛亥朔,日有蝕之。太元十七年五月丁卯朔,日有蝕之。太元二十年三月庚辰朔,日有蝕之。明年,宮車晏駕。海西時有此變。又曰,臣有蔽主明者。



 晉惠帝永興元年十一月,黑氣分日。晉惠帝光熙元年五月癸已,日散,光流如血,所照皆赤。
 甲午,又如之。占曰:「君道失明。」



 晉孝懷帝永嘉元年十一月乙亥,黃黑氣掩日,所炤皆黃。案《河圖占》曰:「日薄也」。其說曰:「凡日蝕皆於晦朔,有不於晦朔者,為日薄。雖非日月同宿,時陰氣盛,掩薄日光也。占類蝕。」永喜二年二月癸卯,白虹貫日,青黃暈五重。



 占曰:「白虹貫日,近臣不亂,則諸侯有兵,破亡其地。」明年,司馬越殺繆播等,暴蔑人主。五年,胡破京都,帝遂見虜。一說王者
 有兵周之象。永嘉五年三月庚申,日散,光如血,下流,所照皆赤。日中有若飛燕鳥者。



 晉愍帝建武元年正月庚子,白虹彌天,三日並照,日有重暈,左右兩珥。占曰:「白虹,兵氣也。三、四、五、六日俱出並爭,天下兵作,王立亦如其數。」又曰:「三日並出,不過三旬,諸侯爭為帝。」



 晉安帝隆安四年六月庚辰朔,日有蝕之。
 晉安帝元興二年四月癸巳朔,日有蝕之。晉安帝義熙三年七月戊戌朔,日有蝕之。義熙十年九月己巳朔,日有蝕之;七月辛亥晦,日有蝕之。義熙十三年正月甲戌朔,日有蝕之。明年,宮車晏駕。



 晉恭帝元熙元年十一月丁亥朔,日有蝕之。



 宋少帝景平二年二月癸巳朔,日有蝕之。



 文帝元嘉四年六月癸卯朔,日有蝕之。
 元嘉六年五月壬辰朔,日有蝕之。十一月己丑朔,又日有蝕之,不盡如鉤,蝕時星見,晡方沒,河北地暗。元嘉十二年正月乙未朔,日有蝕之。元嘉十七年四月戊午朔,日有蝕之。元嘉十九年七月甲戌晦,日有蝕之。元嘉二十三年六月癸未朔,日有蝕之。元嘉三十年七月辛丑朔,日有蝕之,既,星辰畢見。



 孝武帝孝建元年七月丙戌朔,日有蝕之,既,列宿粲然。
 孝武帝大明五年九月甲寅朔,日有蝕之。



 明帝泰始四年八月丙子朔,日有蝕之;十月癸酉,又日有蝕之。泰始五年十月丁卯朔,日有蝕之。



 後廢帝元徽元年十二月癸卯朔,日有蝕之。順帝昇明二年九月乙巳朔,日有蝕之。



 升明三年三月癸卯朔,日有蝕之。



 吳孫權赤烏十一年二月,白虹貫日,時地又頻震。權發
 詔,深戒懼天眚。



 晉武帝泰始五年七月甲寅,日暈再重,白虹貫之。晉武帝太康元年正月己丑朔,五色氣冠日,自卯至酉。占曰:「君道失明。丑主斗、牛,斗、牛為吳地。」是時孫皓淫暴,四月降。



 晉惠帝元康九年正月,日中有若飛燕者,數月乃消。王隱以為愍懷廢死之徵也。



 晉惠帝永康元年十月乙未,日鬥,黃霧四塞。占曰:「不及
 三年,下有拔城大戰。」



 晉惠帝永寧元年九月甲申,日有黑子。按京房占:「黑者,陰也。臣不掩君惡,令下見百姓惡君。」日重暈,天下有立王。暈而珥,天下有立侯。故陳卓曰:「當有大慶,天下其參分乎?」三月而江東改元朔,胡亦改元朔,跨曹、劉疆宇。於是兵連積世。



 晉元帝太興四年三月癸亥,日有黑子。辛亥,帝親錄訊囚徒。
 晉元帝永昌元年十月辛卯,日有黑子。



 晉明帝太寧元年正月己丑朔,日暈無光;癸巳,黃霧四塞。占曰:「君道失明,臣有陰謀。」是時王敦陵上,卒伏其辜。



 晉成帝咸康元年七月,白虹貫日。咸康八年正月壬申,日中有黑子。丙子,乃滅。



 晉海西公太和四年四月戊辰,日暈厚密,白虹貫日中。太和六年三月辛未,白虹貫日,日暈五重。十一月,桓溫廢帝。張重華在涼州,日暴赤如火,中有三足烏,形見分
 明,數旦乃止。



 晉安帝元興元年二月甲子,日暈,白虹貫日。明年,桓玄篡位。晉安帝義熙元年五月庚午,日有採珥。義熙十一年,日在東井,有白虹十餘丈,在南干日。依司馬彪說,則災在分野,羌亡之象也。



 晉恭帝元熙二年正月壬辰,日暈,東西有直珥各一丈,白氣貫之交匝。



 晉孝懷帝永嘉五年三月丙申夜,月蝕既;丁酉夜,又蝕既。占曰:「月蝕既盡,夫人憂。」又曰:「其國貴人死。」



 安帝義熙九年十二月辛卯朔旦,月猶見東方。按占,謂之「側匿。」



 宋文帝元嘉二十九年十一月己卯朔,日始出,色赤如血,外生牙,塊壘不員。



 明年二月,宮車晏駕。



 孝武帝大明七年十一月,日始出四五丈,色赤如血,未沒四五丈,亦如之。至于八年春,凡三,謂日死。閏五月,帝
 崩。



 後廢帝元徽三年三月乙亥,日未沒數丈,日色紫赤無光。元徽五年三月庚寅,日暈五重,又重生二直,一抱一背。



 文帝元嘉中,有兩白虹見宣陽門外。



 後廢帝元徽二年八月壬子夜,白虹見。元徽四年正月己酉,白虹貫日。



 後帝升明元年九月乙未夜,白虹見東方。



\end{pinyinscope}