\article{卷十一志第一 志序 歷上}

\begin{pinyinscope}

 左史
 記言,右史記事,事則《春秋》是也,言則《尚書》是也。至於楚《書》、鄭《志》、晉《乘》、楚《杌》之篇,皆所以昭述前史,俾不泯於後。司馬遷制一家之言,始區別名題。至乎禮儀刑政,
 有所不盡;乃於紀傳之外,創立八書。片文隻事,鴻纖備舉。班氏因之,靡違前式,網羅一代,條流遂廣。《律歷》、《禮樂》,其名不變,以《天官》為《天文》,改《封禪》為《郊祀》,易《貨殖》、《平準》之稱,革《河渠》、《溝洫》之名;綴孫卿之辭,以述《刑法》;采孟軻之書,用序《食貨》。劉向《鴻範》,始自《春秋》;劉歆《七略》,儒墨異部,朱贛博採風謠,尤為詳洽。固並因仍,以為三志。而《禮樂》疏簡,所漏者多,典章事數,百不記一。《天文》雖為該舉,而不言天形,致使三天之說,紛然莫辨。是故蔡邕於朔
 方上書,謂宜載述者也。



 漢興,接秦坑儒之後,典墳殘缺,耆生碩老,常以亡逸為慮。劉歆《七略》,固之《藝文》,蓋為此也。河自龍門東注,橫被中國,每漂決所漸,寄重災深,堤築之功,勞役天下。且關、洛高塏,地少川源,是故鎬、酆、潦、潏,咸入禮典。



 漳、滏、鄭、白之饒,溝渠沾溉之利,皆民命所祖,國以為天,《溝洫》立志,亦其宜也。世殊事改,於今可得而略。竊以班氏《律歷》,前事已詳,自楊偉改創《景初》,而《魏書》闕志。及元嘉重造新法,大明博議回改。自魏至宋,宜
 入今書。



 班固《禮樂》、《郊祀》,馬彪《祭祀》、《禮儀》,蔡邕《朝會》,董巴《輿服》,並各立志。夫禮之所苞,其用非一,郊祭朝饗,匪云別事,旗章服物,非禮而何?今總而裁之,同謂《禮志》。《刑法》、《食貨》,前說已該,隨流派別,附之紀傳。《樂經》殘缺,其來已遠。班氏所述,止抄舉《樂記》;馬彪《後書》,又不備續。至於八音眾器,並不見書,雖略見《世本》,所闕猶眾。爰及《雅》《鄭》,謳謠之節,一皆屏落,曾無概見。郊廟樂章,每隨世改,雅聲舊典,咸有遺文。又案今鼓吹鐃歌,雖有章曲,樂人傳習,
 口相師祖,所務者聲,不先訓以義。今樂府鐃歌,校漢、魏舊曲,曲名時同,文字永異,尋文求義,無一可了。



 不知今之鐃章,何代曲也。今《志》自郊廟以下,凡諸樂章,非淫哇之辭,並皆詳載。



 《天文》、《五行》,自馬彪以後,無復記錄。何書自黃初之始,徐志肇義熙之元。今以魏接漢,式遵何氏。然則自漢高帝五年之首冬,暨宋順帝昇明二年之孟夏,二辰六沴,甲子無差。聖帝哲王,咸有瑞命之紀。蓋所以神明寶位,幽贊禎符,欲使逐鹿弭謀,窺覬不作,握河
 括地,綠文赤字之書,言之詳矣。爰逮道至天而甘露下,德洞地而醴泉出,金芝玄秬之祥,朱草白烏之瑞,斯固不可誣也。若夫衰世德爽,而嘉應不息,斯固天道茫昧,難以數推。亦由明主居上,而震蝕之災不弭;百靈咸順,而懸象之應獨違。今立《符瑞志》,以補前史之闕。



 地理參差,事難該辨,魏晉以來,遷徙百計,一郡分為四五,一縣割成兩三,或昨屬荊、豫,今隸司、兗,朝為零、桂之士,夕為廬、九之民。去來紛擾,無暫止息,版籍為之渾淆,職方所
 不能記。自戎狄內侮,有晉東遷,中土遺氓,播徙江外,幽、并、冀、雍、兗、豫、青、徐之境,幽淪寇逆。自扶莫而裹足奉首,免身於荊、越者,百郡千城,流寓比室。人佇鴻鴈之歌,士蓄懷本之念,莫不各樹邦邑,思復舊井。既而民單戶約,不可獨建,故魏邦而有韓邑,齊縣而有趙民。且省置交加,日回月徙,寄寓遷流,迄無定託,邦名邑號,難或詳書。大宋受命,重啟邊隙,淮北五州,翦為寇境,其或奔亡播遷,復立郡縣,斯則元嘉、泰始,同名異實。今以班固、馬
 彪二志,晉、宋《起居》,凡諸記註,悉加推討,隨條辨析,使悉該詳。



 百官置省,備有前說,尋源討流,於事為易。元嘉中,東海何承天受詔纂《宋書》,其志十五篇,以續馬彪《漢志》,其證引該博者,即而因之,亦由班固、馬遷共為一家者也。其有漏闕,及何氏後事,備加搜采,隨就補綴焉。淵流浩漫,非孤學所盡;足蹇途遙,豈短策能運。雖斟酌前史,備睹姘嗤,而愛嗜異情,取捨殊意,每含豪握簡,杼軸忘餐,終亦不足與班、左並馳,董、南齊轡。庶為後之君子,削
 稿而已焉。



 黃帝使伶倫自大夏之西,阮隃之陰,取竹之嶰谷生,其竅厚均者,斷兩節間而吹之,以為黃鐘之宮。制十二管,以聽鳳鳴,以定律呂。夫聲有清濁,故協以宮商;形有長短,故檢以丈尺;器有大小,故定以斛斗;質有累重,故平以鈞石。故《虞書》曰:「乃同律、度、量、衡。」然則律呂、宮商之所由生也。夫樂有器有文,有情有官。鐘鼓干戚,樂之器也;屈伸舒疾,樂之文也;「論倫無患,樂之情也;欣喜歡愛,樂
 之官也。」「是以君子反情以和志,廣樂以成教,故能情深而文明,氣盛而化神,和順積中,而英華發外。」故曰:「樂者,心之動也;聲者,樂之象也。」《周禮》曰:「乃奏黃鐘,歌大呂,舞《雲門》,以祀天神。乃奏太蔟,歌應鐘,舞《咸池》,以祭地祇。」四望山川先祖,各有其樂。又曰:「圜鐘為宮,黃鐘為徵,姑洗為羽,雷鼓雷鞀,孤竹之管,雲和之琴瑟,《雲門》之舞,冬日至,於地上之圜丘奏之。若樂六變,則天神皆降,可得而禮矣。」地祇人鬼,禮亦如之。



 其可以感物興化,若此之
 深也。



 「道始於一,一生二,二生三,三三而九。故黃鐘之數六,分而為雌雄十二鐘。



 鐘以三成,故置一而三之,凡積分十七萬七千一百四十七,為黃鐘之實。故黃鐘位子,主十一月,下生林鐘。林鐘之數五十四,主六月,上生太蔟。太蔟之數七十二,主正月,下生南呂。南呂之數四十八,主八月,上生姑洗。姑洗之數六十四,主三月,下生應鐘。應鐘之數四十三,主十月,上生蕤賓。蕤賓之數五十七,主五月,上生大呂。大呂之數七十六,主十二月,下生夷
 則。夷則之數五十,主七月,上生夾鐘。夾鐘之數六十七,主二月,下生無射。無射之數四十五,主九月,上生中呂。


中呂之數六十,主四月,極不生。
 \gezhu{
  極不生,鐘律不能復相生。}
 宮生徵,徵生商,商生羽,羽生角,角生姑洗,姑洗生應鐘,不比於正音,故為和。
 \gezhu{
  姑洗三月,應鐘十月,不與正音比效為和。和,徙聲也。}
 應鐘生蕤賓,蕤賓不比於正音,故為繆。
 \gezhu{
  繆,音相干也。周律故有繆、和,為武王伐紂七音也。}
 日冬至,音比林鐘浸以濁;日夏至,音比黃鐘浸以清,以十二月律應二十四時。甲子,中呂之徵也;丙子,夾鐘之羽也;戊子,
 黃鐘之宮也;庚子,無射之商也;壬子,夷則之角也。」


「古人為度量輕重,皆生乎天道。黃鐘之律長九寸,物以三生,三三九,三九二十七,故幅廣二尺七寸,古之制也。音以八相生,故人長八尺,尋自倍,故八尺而為尋。有形即有聲,音之數五,以五乘八,五八四十尺為匹。匹者,中人之度也,一匹為制。秋分而禾票定,
 \gezhu{
  票,禾穗芒也。}
 票定而禾孰。律之數十二,故十二票而當一粟,十粟而當一寸。律以當辰,音以當日。日之數十,故十寸而為尺,十尺為丈。其以
 為重,十二粟而當一分,十二分而當一銖,十二銖而當半兩。



 衡有左右,因而倍之,故二十四銖而當一兩。天有四時,以成一歲,因而四之,四四十六,故十六兩而一斤。三月而一時,三十日一月,故三十斤為一鈞。四時而一歲,故四鈞而一石。」「其為音也,一律而生五音,十二律而為六十音;因而六之,六六三十六,故三百六十音以當一歲之日。故律歷之數,天地之道也。下生者倍,以三除之;上生者四,以三除之。」


揚子雲曰:「聲生於日,
 \gezhu{
  謂甲己為角,乙
  庚為商,丙辛為徵,丁壬為羽,戊癸為宮。}
 律生於辰,
 \gezhu{
  謂子為黃鐘,醜為大呂之屬。}
 聲以情質,
 \gezhu{
  質,正也。}



 各以其行本情為正也。



 夫陰陽和則景至,律氣應則灰除。是故天子常以冬夏至御前殿,合八能之士,陳八音,聽樂均,度晷景,候鐘律,權土炭,效陰陽。冬至陽氣應,則樂均清,景長極,黃鐘通,土炭輕而衡仰。夏至陰氣應,則樂均濁,景短極,蕤賓通,土炭重而衡低。進退
 於先後五日之中,八能各以候狀聞。太史令封上。效則和,否則占。



 候氣之法,為室三重,戶閉,塗釁周密,布緹幔。室中以木為案,每律各一,內庳外高,從其方位,加律其上。以葭莩灰布其內端,案歷而候之。氣至者灰動,其為氣動者其灰散,人及風所動者,其灰聚。殿中候,用玉律十二。唯二至乃候靈臺,用竹律六十。取弘農宜陽縣金門山竹為管,河內葭莩為灰。



 三代陵遲,音律失度。漢興,北平侯張蒼始定律歷。孝武之世,置協律之官。



 元帝時,郎中京房知五音六十律之數,受學於小黃令焦延
 壽。其下生、上生,終於中呂,而十二律畢矣。中呂上生執始,執始下生去滅,終於南事,而六十律畢矣。



 夫十二律之變至於六十,猶八卦之變至於六十四也。宓羲作《易》,紀陽氣之初,以為律法。建日冬至之聲,以黃鐘為宮,太蔟為商,姑洗為角,林鐘為徵,南呂為羽,應鐘為變宮,蕤賓為變徵。此聲氣之元,五音之正也。故各統一日。其餘以次運行,當日者各自為宮,而商角徵羽以類從焉。《禮運篇》曰:「五聲、六律、十二管還相為宮。」此之謂也。以六十
 律分一期之日,黃鐘自冬至始,及冬至而復,陰陽寒煖風雨之占於是生焉。房又曰:「竹聲不可以度調,故作準以定數。準之狀如瑟,長丈而十三弦,隱間九尺,以應黃鐘之律九寸;中央一弦,下有畫分寸,以為六十律清濁之節。」房言律詳,其術施行於史官,候部用之。《續漢志》具載其律準度數。



 漢章帝元和元年,待詔候鐘律殷肜上言:「官無曉六十律以準調音者,故待詔嚴嵩具以準法教子男宣,願召
 宣補學官,主調樂器。」詔曰:「嵩子學審曉律,別其族,協其聲者,審試。不得依托父學,以聾為聰。聲微妙,獨非莫知,獨是莫曉,以律錯吹,能知命十二律不失一,乃為能傳嵩學耳。」試宣十二律,其二中,其四不中,其六不知何律,宣遂罷;自此律家莫能為準。靈帝熹平六年,東觀召典律者太子舍人張光等問準意,光等不知。歸閱舊藏,乃得其器,形制如房書,猶不能定其弦緩急。音不可書以曉人,知之者欲教而無從,心達者體知而無師,故史官
 能辨清濁者遂絕。其可以相傳者,唯候氣而已。


\gezhu{
  表略}



 論曰:律呂相生,皆三分而損益之。先儒推十二律,從子至亥,每三之,凡十七萬七千一百四十七,而三約之,是為上生。故《漢志》云:三分損一,下生林鐘,三分益一,上生太蔟。無射既上生中呂,則中呂又當上生黃鐘,然後五聲、六律、十二管還相為宮。今上生不及黃鐘實二千三百
 八十四,九約實一千九百六十八為一分,此則不周九分寸之律一分有奇,豈得還為宮乎?凡三分益一為上生,三分損一為下生,此其大略,猶周天
 鬥分四分之一耳。京房不思此意,比十二律微有所增,方引而伸之,
 中呂上生
 執始,執始下生去滅,至於南事,為六十律,竟復不合,彌益其疏。班氏所志,未能通律呂本源,徒訓角為
 觸,徵為祉,陽氣施種於黃鐘,如斯之屬,空煩其文,而為辭費。又推九六,欲符劉歆三統之數,假託非類,以飾其說,皆孟堅之妄矣。



 蔡邕從朔方上書,云《前漢志》但載十律,不及六十。六律尺寸相生,司馬彪皆已志之。漢末,亡失雅樂。黃初中,鑄工柴玉巧有意思,形器之中,多所造作。



 協律都尉杜夔令玉鑄鐘,其聲清濁,多不如法。數毀改作,玉甚厭之,謂夔清濁任意。更相訴白於魏王。魏王取玉所鑄鐘,雜錯
 更試,然後知夔為精,於是罪玉及諸子,皆為養馬主。



 晉泰始十年,中書監荀勖、中書令張華,出御府銅竹律二十五具,部太樂郎劉秀等校試,其三具與杜夔及左延年律法同,其二十二具,視其銘題尺寸,是笛律也。



 問協律中郎將列和,辭:「昔魏明帝時,令和承受笛聲,以作此律,欲使學者別居一坊,歌詠講習,依此律調。至於都合樂時,但識其尺寸之名,則絲竹歌詠,皆得均合。歌聲濁者,用長笛長律;歌聲清者,用短笛短律。凡絃歌調張清
 濁之制,不依笛尺寸名之,則不可知也。」



 勖等奏:「昔先王之作樂也,以振風蕩俗,饗神佐賢,必協律呂之和,以節八音之中。是故郊祀朝宴,用之有制,歌奏分敘,清濁有宜。故曰『五聲十二律,還相為宮。』此經傳記籍可得而知者也。如和對辭,笛之長短,無所象則,率意而作,不由曲度。考以正律,皆不相應,吹其聲均,多不諧合。又辭:『先師傳笛,別其清濁,直以長短,工人裁制,舊不依律。』是為作笛無法。而知寫笛造律,又令琴瑟歌詠,從之為正,非所
 以稽古先哲,垂憲於後者也。謹條牒諸律,問和意狀如左。



 及依典制,用十二律造笛像十二枚,聲均調和,器用便利。講肄彈擊,必合律呂,況乎宴饗萬國,奏之廟堂者哉!雖伶、夔曠遠,至音難精,猶宜刑古昔,以求厥衷,合於經禮,於制為詳。若可施用,請更部笛工,選竹造作,太樂、樂府施行。平議諸杜夔、左延年律可皆留。其御府笛正聲下徵各一具,皆銘題作者姓名。其餘無所施用,還付御府毀。」奏可。



 勖又問和:「作笛為可依十二律作十二笛,
 令一孔依一律,然後乃以為樂不?」



 和辭:「太樂東廂長笛正聲已長四尺二寸,令當復取其下徵之聲;於法,聲濁者笛當長,計其尺寸,乃五尺有餘,和昔日作之,不可吹也。又笛諸孔,雖不校試,意謂不能得一孔輒應一律也。」案太樂,四尺二寸笛正聲均應蕤賓,以十二律還相為宮,推法下徵之孔,當應律大呂。大呂笛長二尺六寸有奇,不得長五尺餘。令太樂郎劉秀、鄧昊等依律作大呂笛以示和。又吹七律,一孔一校,聲皆相應。然後令郝生
 鼓箏,宋同吹笛,以為《雜引》、《相和》諸曲。和乃辭曰:「自和父祖漢世以來,笛家相傳,不知此法,而令調均與律相應,實非所及也。」郝生、魯基、種整、朱夏,皆與和同。



 又問和:「笛有六孔,及其體中之空為七。和為能盡名其宮商角徵不?孔調與不調,以何檢知?」和辭:「先師相傳,吹笛但以作曲相語,為某曲當舉某指,初不知七孔盡應何聲也。若當作笛,其仰尚方笛工,依案舊像訖,但吹取鳴者,初不復校其諸孔調與不調也。」案《周禮》調樂金石,有一定之
 聲,是故造鐘磬者,先依律調之,然後施於廂懸。作樂之時,諸音皆受鐘磬之均,即為悉應律也。至於饗宴殿堂之上,無廂懸鐘磬,以笛有一定調,故諸絃歌皆從笛為正。是為笛猶鐘磬,宜必合於律呂。如和所對,直以意造,率短一寸,七孔聲均,不知其皆應何律?調與不調,無以檢正。唯取竹之鳴者,為無法制。輒令部郎劉秀、鄧昊、魏邵等與笛工參共作笛。工人造其形,律者定其聲,然後器象有制,音均和協。



 又問和:「若不知律呂之義,作樂
 音均高下清濁之調,當以何名之?」和辭:「每合樂時,隨歌者聲之清濁,用笛有長短。假令聲濁者用三尺二笛,因名曰此三尺二調也。聲清者用二尺九笛,因名曰此二尺九調也。漢、魏相傳,施行皆然。」



 案《周禮》奏六樂,乃奏黃鐘;歌大呂,乃奏太蔟,歌應鐘,皆以律呂之義,紀歌奏清濁。而和所稱以二尺三尺為名,雖漢、魏用之,俗而不典。部郎劉秀、鄧昊等以律作笛,三尺二寸者,應無射之律,若宜用長笛,執樂者曰:「請奏無射。」



 《周語》曰:「無射所以宣
 布哲人之令德,示民軌儀也。」二尺八寸四分四釐應黃鐘之律,若宜用短笛,執樂者曰:「請奏黃鐘。」《周語》曰:「黃鐘所以宣養六氣九德也。」是則歌奏之義,當合經禮,考之古典,於制為雅。



 《書》曰:「予欲聞六律五聲八音,在治忽始。」《周禮》載六律六同。《禮記》又曰:「五聲十二律,還相為宮。」劉歆、班固纂《律歷志》,亦紀十二律。


唯京房始創六十律,至章帝時,其法已亡;蔡邕雖追古作,其言亦曰:「今無能為者。」依案古典及今音家所用六十律者,無施於樂。謹依
 典記,以五聲十二律還相為宮之法,制十二笛象,記注圖側,如別。省圖,不如視笛之了,故復重作蕤賓伏孔笛。其制云:黃鐘之笛,正聲應黃鐘,下徵應林鐘,長二尺八寸四分四釐有奇。
 \gezhu{
  《周語》曰:「黃鐘所以宣養六氣九德也。」主聲調法,以黃鐘為宮,則姑洗為角。翕笛之聲應姑洗,故以四角之長為黃鐘之笛也。其宮聲正而不倍。故曰正聲。}
 正聲調法,黃鐘為宮,
 \gezhu{
  第一孔。}
 應鐘為變宮,
 \gezhu{
  第二孔。}
 南呂為羽,
 \gezhu{
  第三孔。}
 林鐘為徵,
 \gezhu{
  第四孔。}
 蕤賓為變徵,
 \gezhu{
  第五附孔。}
 姑洗為角,
 \gezhu{
  笛體中聲。}
 太蔟為商。
 \gezhu{
  笛後出孔也。商聲濁於角,當在角下,而角聲以在體中,故上其商孔,令在宮上,清於宮也。然則宮商正也,餘聲皆倍也。是
  故從宮以下,孔轉下轉濁也。}



 此章說笛孔上下次第之名也。下章說律呂相生,笛之制也。


\gezhu{
  以蕤賓律從變宮下度之,盡律為孔,則得變徵之聲。十二笛之制,各以其宮為主。}



 相生之法,或倍或半,其便事用,例皆一者也。


黃鐘為羽,
 \gezhu{
  非正也。}
 太蔟為變宮。
 \gezhu{
  非正也。清角之調,唯宮商及徵,與律相應,餘四聲非正者皆濁,一律哨吹令清,假而用之,其例一也。}


凡笛體用
 角律,其長者八之,
 \gezhu{
  蕤賓、林鐘也。}
 短者四之,
 \gezhu{
  其餘十笛,皆四角也。}
 空中實容,長者十六,
 \gezhu{
  短笛竹宜受八律之黍也。若長短大小不合於此,或器用不便聲均法度之齊等也。然笛竹率上大下小,不能均齊,必不得已,取其聲均合。}
 三宮
 \gezhu{
  一曰正聲,二曰下徵,三曰清角。}
 二十一變也。
 \gezhu{
  宮有七聲,錯綜用之,故二十一變也。諸笛例皆一也。}
 伏孔四,所以便事用也。
 \gezhu{
  一曰正角,出於商上者也。二曰倍角,近笛下者也。三曰變宮,近於宮孔,倍令下者也。四曰變徵,遠於徵孔,倍令高者也,或倍或半,或四分一,取則於琴徵也。四者皆不作其孔而取其度,以應進退上下之法,所以協聲均,便事用也。其本孔隱而不見,故曰伏孔。}


大呂之笛:正聲應大呂,下徵應夷則,長二尺六寸六分
 三釐有奇。
 \gezhu{
  《周語》曰:「元間大呂,助宣物也。」}


太蔟之笛:正聲應太蔟,下徵應南呂,長二尺五寸三分一厘有奇。
 \gezhu{
  《周語》曰:「太蔟所以金奏,贊陽出滯也。」}


夾鐘之笛:正聲應夾鐘,下徵應無射,長二尺四寸。
 \gezhu{
  《周語》曰:「二間夾鐘,出四隙之細也。」}


姑洗之笛:正聲應姑洗,下徵應應鐘,長二尺二寸三分三釐有奇。
 \gezhu{
  《周語》曰:「三間中呂,宣中氣也。」}


蕤賓之笛,正聲應蕤賓,下徵應大呂,長三尺九寸九分
 五厘有奇。
 \gezhu{
  《周語》曰:「蕤賓所以安靜神人,獻酬交酢。」變宮近孔,故倍半令下,便於用也。林鐘亦如之。}


林鐘之笛:正聲應林鐘,下徵應太蔟,長三尺七寸九分二厘有奇。
 \gezhu{
  《周語》曰:「四間林鐘,和展百事,俾莫不任肅純恪。」}


夷則之笛:正聲應夷則,下徵應夾鐘,長三尺六寸。
 \gezhu{
  《周語》曰:「夷則所以詠歌九州,平民無貳也。」變宮之法,亦如蕤賓,體用四角,故四分益一也。}


南呂之笛,正聲應南呂,下徵應姑洗,長三尺三寸七分。
 \gezhu{
  《周語》曰:「五間南呂,贊陽秀也。」}


無射之笛:正聲應無射,下徵應中呂,長三尺二寸。
 \gezhu{
  《周語》曰:「無
  射所以宣布哲人之令德,示民軌儀也。」}


應鐘之笛:正聲應應鐘,下徵應蕤賓,長五尺九寸九分六厘有奇。
 \gezhu{
  《周語》曰:「六間應鐘,均利器用,俾應復也。」}



 勖又以魏杜夔所制律呂,檢校太樂、總章、鼓吹八音,與律乖錯。始知後漢至魏,尺度漸長於古四分有餘。夔依為律呂,故致失韻。乃部佐著作郎劉恭依《周禮》更積黍起度,以鑄新律。既成,募求古器,得周時玉律,比之不差毫釐。又漢世故鐘,以律命之,不叩而自應。初,勖行道,逢
 趙郡商人縣鐸於牛,其聲甚韻。至是搜得此鐸,以調律呂焉。



 晉武帝以勖律與周、漢器合,乃施用之。散騎侍郎阮咸譏其聲高,非興國之音。



 咸亡後,掘地得古銅尺,果長勖尺四分,時人咸服其妙。元康中,裴頠以為醫方民命之急,而稱兩不與古同,為害特重,宜因此改治權衡。不見省。


黃鐘箱笛,晉時三尺八寸。元嘉九年,太樂令鐘宗之減
 為三尺七寸。十四年,治書令吏奚縱又減五分,為三尺六寸五分。
 \gezhu{
  列和云:「東箱長笛四尺二寸也。」}



 太蔟箱笛,晉時三尺七寸,宗之減為三尺三寸七分,縱又減一寸一分,為三尺二寸六分。姑洗箱笛,晉時三尺五寸,宗之減為二尺九寸七分,縱又減五分,為二尺九寸二分。蕤賓箱笛,晉時二尺九寸,宗之減為二尺六寸,縱又減二分,為二尺五寸八分。



\end{pinyinscope}