\article{卷七十八列傳第三十八 蕭思話 劉延孫}

\begin{pinyinscope}

 蕭思
 話,南蘭陵人,孝懿皇后弟子也。父源之,字君流,歷中書黃門郎,徐、兗二州刺史,冠軍將軍、南琅邪太守。永初元年卒,追贈前將軍。



 思話年十許歲,未知書,以博誕
 遊遨為事,好騎屋棟,打細腰鼓,侵暴鄰曲,莫不患毒之。自此折節,數年中,遂有令譽。好書史,善彈琴,能騎射。高祖一見,便以國器許之。年十八,除琅邪王大司馬行參軍,轉相國參軍,父憂去職。服闋,拜羽林監,領石頭戍事,襲爵封陽縣侯,轉宣威將軍、彭城、沛二郡太守。涉獵書傳,頗能隸書,解音律,便弓馬。元嘉元年,謝晦為荊州,欲請為司馬,思話拒之。



 五年,遷中書侍郎,仍督青州、徐州之東莞諸軍事、振武將軍、青州刺史,時年二十七。亡命
 司馬朗之、元之、可之兄弟,聚黨於東莞發干縣,謀為寇亂。思話遣北海太守蕭汪之討斬之,餘黨悉平。八年,除竟陵王義宣左軍司馬、南沛郡太守。



 未及就徵,索虜南寇,檀道濟北伐,既而回師,思話懼虜大至,乃棄鎮奔平昌。思話先使參軍劉振之戍下邳,聞思話奔,亦委城走。虜定不至,而東陽積聚,已為百姓所焚,由是徵下廷尉,仍繫尚方。初在青州,常所用銅斗,覆在藥廚下,得二死雀,思話曰:「斗覆而雙雀殞,其不祥乎!」既而被系。



 九年,仇
 池大饑,益、梁州豐稔,梁州刺史甄法護在任失和,氐帥楊難當因此寇漢中。乃自徒中起思話督梁、南秦二州諸軍事、橫野將軍、梁、南秦二州刺史。



 既行,聞法護已委鎮北奔西城,遣司馬、建威將軍、南漢中太守蕭諱五百人前進;又遣西戎長史蕭汪之係之。諱緣路收合士眾,得精兵千人。十年正月,進據磝頭。



 難當焚掠漢中,引眾西還,留其輔國將軍、梁秦二州刺史趙溫守梁州,魏興太守薛健據黃金。諱進屯磝頭,遣陰平太守蕭坦赴黃
 金,薛健副姜寶據鐵城,鐵城與黃金相對,去一里,斫樹塞道。坦進攻二戍,拔之。二月,趙溫又率薛健及其寧朔將軍、馮翼太守蒲早子來攻坦營,坦奮擊,大破之。坦被創,賊退保西水。諱司馬錫文祖進據黃金,蕭汪之步騎五百相繼而至。平西將軍臨川王義慶遣龍驤將軍裴方明三千人赴,諱等進黃金,早子、健等退保下桃。思話先遣行參軍王靈濟率偏軍出洋川,因向南城。偽陵江將軍趙英堅守險,靈濟擊破之,生禽英。南城空虛,因資
 無所,復引軍還與諱合。



 三月,諱率眾軍進據峨公固。難當遣其子和率趙溫、蒲早子及左衛將軍呂平、寧朔將軍司馬飛龍,步騎萬餘,跨漢津結柴,其間立浮橋,悉力攻諱,合圍數十重,短兵接戰,弓矢無復用。賊悉衣犀革,戈矛所不能加。諱乃截槊長數尺,以大斧椎之,一槊輒貫十餘賊。賊不能當,因大敗,燒柴奔走,退據大桃。閏月,諱及方明臺軍至,龍驤將軍楊平興、幢主殿中將軍梁坦直入角弩追之,賊又敗走,殺傷虜獲甚多。漢中平,悉
 收沒地,置戍葭萌水。



 先是,桓玄篡晉,以桓希為梁州。布敗走,氐楊盛據有漢中,刺史范元之、傅歆悉治魏興,唯得魏興、上庸、新城三郡。其後索邈為刺史,乃治南城。為賊所焚燒不可固,即思話遷鎮南域,加節,進號寧朔將軍,徵諱為太子屯騎校尉。法護,中山無極人,過江寓居南郡。弟法崇,元嘉十年,自少府為益州刺史。法護委鎮之罪,統府所收,於獄賜死。太祖以法崇受任一方,令獄官言法護病卒。太祖使思話上平定漢中本末,下之
 史官。



 十四年,遷使持節、臨川王義慶平西長史、南蠻校尉。太祖賜以弓琴,手敕曰:「丈人頃何所作?事務之暇,故以琴書為娛耳,所得不曰義邪!眷想常不忘情,想亦同之。前得此琴,云是舊物,亦有名京邑,今以相借。因是戴顒意於彈撫,響韻殊勝,直爾嘉也。并往桑弓一張,材理乃快,先所常用,既久廢射,又多病,略不能制之,便成老公,令人歎息。良材美器,宜在盡用之地,丈人真無所與讓也。」



 十六年,衡陽王義季代義慶,又除安西長史,餘如
 故。十九年,徵為侍中,領前軍將軍,未就徵,復先職。明年,遷持節、監雍州、梁、南北秦四州、荊州之南陽、竟陵、順陽、襄陽、新野、隨六郡諸軍事、寧蠻校尉、雍州刺史、襄陽太守。



 二十二年,除侍中,領太子右率。二十四年,改領左衛將軍。嘗從太祖登鐘山北嶺,中道有磐石清泉,上使於石上彈琴,因賜以銀鐘酒,謂曰:「相賞有松石間意。」



 又領南徐州大中正。明年,復監雍、梁、南北秦四州、荊州之竟陵、隨二郡諸軍事、右將軍、寧蠻校尉、雍州刺史如故。



 二十
 六年,徵為吏部尚書。詔思話曰:「沈尚書暴病不救,其體業貞審,立朝盡公,年時尚可,方相委任,奄忽不永,痛惋特深。銓管要機,通塞所寄,丈人才用體國,二三惟允。」思話以去州無復事力,倩府軍身九人,太祖戲之曰:「丈人終不為田父於里閭,何應無人使邪?」未拜,二十七年,遷護軍將軍。



 是年春,虜攻懸瓠,太祖將大舉北討,朝士僉同,莫或異議。思話固諫,不從。



 乃領精甲三千,助鎮彭城。虜退,即代世祖為持節、監徐、兗、青冀四州、豫州之梁郡
 諸軍事、撫軍將軍、兗徐二州刺史。



 二十九年,統揚武將軍、冀州刺史張永眾軍圍確磝。初,鎮軍諮議參軍申坦與王玄謨圍滑臺,不克,免官。青州刺史蕭斌板坦行建威將軍、濟南平原二郡太守,守歷城,令任仲仁又為坦副,並前鋒入河。五月,發沿口,永司馬崔訓、建武將軍齊郡太守胡景世率青州軍來會。七月,思話及眾軍並至確磝,治三攻道。太祖遣員外散騎侍郎徐爰宣旨督戰。張永、胡景世當東攻道,申坦、任仲仁西攻道,崔訓南攻
 道。賊夜地道出,燒崔訓樓及蟆車,又燒胡景世樓及攻具,尋又毀崔訓攻道,城不可拔。思話馳來,退師。攻城凡十八日,解圍還歷下。崔訓以樓見燒,又不能固攻道,被誅於確磝;永、坦並繫獄。詔曰:「得撫軍將軍思話啟事,確磝不拔,士卒疲勞,且班師清濟,更圖進討。此鎮山川嚴阻,控臨河朔,形勝之要,擅名自古,宜除其授,以允望實。思話可解徐州為冀州,餘如故。彭城文武,復量分配,即鎮歷城。」尋為江夏王義恭所奏,免官。



 元凶弒立,以為使持
 節、監徐、青兗、冀四州、豫州之梁郡諸軍事、徐兗二州刺史,將軍如故。思話即率部曲還彭城,起義以應世祖。遣使奉箋曰:「下官近在歷下,始奉國諱,所承使人,不知闊狹,既還在路,漸有所聞,猶謂人倫無容有此,私懷感慨,未敢在言。奉被今教,果出慮表,重增哀惋,不能自勝。此實天地所不覆載,人神所不容忍,率土民氓,莫不憤咽,況下官蒙荷榮渥,義兼常志。此月五日,被驛使追命騎還朝,切齒拊心,輒已鐘疾,雖百口在都,一非所顧。正欲
 遣啟受規略,會奉今旨,悲懼兼情。伏承司徒英圖電發,殿下神武霜斷,臧質忠義並到,不謀同時,仗順沿流,席卷江甸,前驅風邁,已應在近。下官復練始集,遣輔國將軍申坦、龍驤將軍梁坦二軍,分配精甲五千,申坦為統,便以即日水陸齊下。下官悉率文武,駱驛繼發。憑威策懦,勢同振朽,開泰有期,悲欣交集。」世祖至新亭,坦亦進克京口。



 上即位,徵為散騎常侍、尚書左僕射,固辭,不受拜。改為中書令、丹陽尹,常侍如故。時京邑多有劫掠,二
 旬中十七發,引咎陳遜,不許。明年,出為使持節、都督徐兗、青、冀、幽五州、豫州之梁郡諸軍事、安北將軍、徐州刺史,加鼓吹一部。未行而江州刺史臧質反,復以為使持節、都督江州、豫州之西陽、晉熙、新蔡三郡諸軍事、江州刺史。事平,分荊、江、豫三州置郢州,復都督郢湘二州諸軍事、鎮西將軍、郢州刺史,持節、常侍如故,鎮夏口。



 孝建二年卒,時年五十。追贈征西將軍、開府儀同三司,持節、常侍、都督、刺史如故,謚曰穆侯。思話宗戚令望,蚤見任
 待,凡歷州十二,杖節監都督九焉。



 所至雖無皦皦清節,亦無穢黷之累。愛才好士,人多歸之。



 長子惠開嗣,別有傳。次子惠明,亦有世譽,歷黃門郎,御史中丞,司徒左長史,吳興太守。後廢帝元徽末,卒官。第四子惠基,順帝昇明末,為侍中。



 源之從父弟摹之,丹陽尹,追贈征虜將軍。子斌,亦為太祖所遇。彭城王義康鎮豫章,以為大將軍諮議參軍、豫章太守。歷南蠻校尉,侍中,輔國將軍、青冀二州刺史。



 元嘉二十七年,統王玄謨等眾軍北伐。斌遣
 將軍崔猛攻虜青州刺史張淮之於樂安,淮之棄城走。先是,猛與斌參軍傅融分取樂安及確磝,樂安水道不通,先并定確磝,至是又克樂安。既而攻圍滑臺,不拔。斌追還歷下,事在《王玄謨傳》。二十八年,亡命司馬順則詐稱晉室近屬,自號齊王,聚眾據梁鄒城。又有沙門自稱司馬百年,號安定王,亡命秦凱之、祖元明等各據村屯以應順則。初,梁鄒戍主、宣威將軍、樂安、渤海二郡太守崔勳之出州,故順則因虛竊據。勳之司馬曹敬會拒戰
 不敵,出走。斌即遣勳之率行建威將軍濟南、平原二郡太守申坦、長流參軍羅文昌等諸軍討順則,攻之不克。勳之等始謂城內出於逼附,軍至即應奔逃,而並為賊堅守,殺傷官軍甚多。斌又遣府司馬、建武將軍、齊郡太守龐秀之總諸軍。祖元明又據安丘城,斌更遣振武將軍劉武之及軍主劉回精兵千人,討司馬百年,斬之。順則既失據,眾稍離阻。文昌遣道連偽投賊,賊信納之,潛以官賞格示眾,城內賊黨李繼叔等並有歸順心。道連
 謀泄,為賊所殺,繼叔踰城出降,賊黨於是大離。乃四面進攻,衝車所衝,輒三五丈崩落。時南門樓上擲下一級,并垂繩釣取外人,外人上,賊並放仗,云向已斬順則,所投首是也。秦凱之走河北。斌坐滑臺退敗,免官。久之,復起為南平王鑠右軍長史。其後事跡在《二凶傳》。



 斌弟簡,歷位長沙內史。廣陵王誕為廣州,未之鎮,以簡為安南諮議參軍、南海太守,行府州事。東海王禕代誕,簡仍為前軍諮議,太守如故。世祖入討元凶,遣輔國將軍、南海
 太守劉琬討簡,固守經時,城陷伏誅。斌、簡諸子並誅滅。



 龐秀之,河南人也。以斌故吏,賊劭甚加信委,以為遊擊將軍。奔世祖於新亭。



 時劭諸將未有降者,唯秀之先至,事平,以為梁州刺史。秀之子弟為劭所殺者將十人,而酣燕不廢,坐免官。後又為徐州刺史,太子右衛率。孝建元年,卒,追贈本官,加散騎常侍。子彌之,順帝昇明末,廣興公相。秀之弟況之,太宗世,亦為始興相。



 劉延孫,彭城呂人,雍州刺史道產子也。初為徐州主簿,
 舉秀才,彭城王義康司徒行參軍,尚書都官郎,為錢唐令,世祖撫軍、廣陵王誕北中郎中兵參軍、南清河太守。世祖為徐州,補治中從事史。時索虜圍縣瓠,分軍送所掠民口在汝陽,太祖詔世祖遣軍襲之,議者舉延孫為元帥,固辭無將用,舉劉泰之自代。泰之既行,太祖大怒,免延孫官。為世祖鎮軍北中郎中兵參軍,南中郎諮議參軍,領錄事。世祖伐逆,府缺上佐,轉補長史、尋陽太守,行留府事。



 世祖即位,以為侍中,領前軍將軍。下詔曰:「朕
 藉群能之力,雪莫大之恥,以眇眇之身,託於王公之上,思所以策勳樹良,永寧世烈。新除侍中、領前軍將軍延孫率懷忠敏,器局沈正,協贊義初,誠力俱盡。左衛將軍竣立志開亮,理思清要,茂策忠謨,經綸惟始,俾積基更造,咸有勤焉。宜顯授龜社,大啟邦家。延孫可封東昌縣侯,竣建城縣侯,食邑各二千戶。」其年,侍中改領衛尉。



 孝建元年,遷丹陽尹。臧質反叛,上深以東土為憂,出為冠軍將軍、吳興太守,置佐史。事平,徵為尚書右僕射,領徐
 州大中正。遣至江陵,分判枉直,行其誅賞。



 三年,又出為南兗州刺史,加散騎常侍。仍徙為使持節、監雍、梁、南北秦四州、郢州之竟陵、隨二郡諸軍事、鎮軍將軍、寧蠻校尉、雍州刺史,以疾不行。留為侍中、護軍,又領徐州大中正。素有勞患,其年增篤,詔遣黃門侍郎宣旨問疾。



 大明元年,除金紫光祿大夫,領太子詹事,中正如故。其年,又出為鎮軍將軍、南徐州刺史。先是,高祖遺詔,京口要地,去都邑密邇,自非宗室近戚,不得居之。



 延孫與帝室雖
 同是彭城人,別屬呂縣。劉氏居彭城縣者,又分為三里,帝室居綏輿里,左將軍劉懷肅居安上里,豫州刺史劉懷武居叢亭里,及呂縣凡四劉。雖同出楚元王,由來不序昭穆。延孫於帝室本非同宗,不應有此授。時司空竟陵王誕為徐州,上深相畏忌,不欲使居京口,遷之於廣陵。廣陵與京口對岸,欲使腹心為徐州,據京口以防誕,故以南徐授延孫,而與之合族,使諸王序親。



 三年,南兗州刺史竟陵王誕有罪,不受徵,延孫馳遣中兵參軍杜
 幼文率兵起討。



 既至,誕已閉城自守,乃還。誕遣使劉公泰齎書要之,延孫斬公泰,送首京邑。復遣幼文率軍渡江,受沈慶之節度。其年,進號車騎將軍,加散騎常侍,給鼓吹一部。



 五年,詔延孫曰:「舊京樹親,由來常準。卿前出所有別議,今此防久弭,當以還授小兒。」徵延孫為侍中、尚書左僕射,領護軍將軍。延孫疾病,不任拜起,上使於五城受封版,乘船自青溪至平昌門,仍入尚書下舍。又欲以代朱修之為荊州,事未行,明年,卒,時年五十二。上
 甚惜之,下詔曰:「故侍中尚書左僕射、領護軍將軍東昌縣開國侯延孫,風局簡正,體識沈明,綢繆心膂,自蕃升朝,契闊唯舊,幾將二紀。靈業中圮,則首贊宏圖;義令既舉,則任均蕭、寇。器允棟幹,勳實佐時。及累司馬兩官,出內尹牧,惠政茂課,著自民聽,忠謨令節,簡乎朕心。方燮和台階,永毗國道,奄至薨殞,震慟兼深。考終定典,宜盡哀敬。可贈司徒,給班劍二十人,侍中、僕射、侯如故。」有司奏謚忠穆,詔為文穆。又詔曰:「故司徒文穆公延孫,居身
 寡約,家素貧虛,每念清美,良深淒嘆。葬送資調,固當闕乏,可賜錢三十萬,米千斛。」



 子質嗣,太宗泰始中,有罪,國除。延孫弟延熙,義興太守,在《孔覬傳》。



 史臣曰:延孫接款蕃日,固出顏、袁矣。風飆局力,又無等級可言,而隆名盛寵,必擇而後授,何哉?良以休運甫開,沈疾方被,雖宿恩內積,而安私外簡。夫侮因事狎,敬由近疏,疏必相思,狎必相厭,厭思一殊,榮禮自隔,遂得為一世宗臣,蓋由此也。子曰:「事君數,斯疏矣。」然乎!然乎!



\end{pinyinscope}