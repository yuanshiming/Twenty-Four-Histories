\article{卷二十志第十 樂二}

\begin{pinyinscope}

 蔡邕論敘漢
 樂曰:一曰郊廟神靈,二曰天子享宴,三曰大射闢雍,四曰短簫鐃歌。



 晉郊祀歌五篇,傅玄造:天命有晉,穆穆明明。我其夙夜,祗事上靈。常於時假,迄用有成。於薦玄牡,進夕其牲。崇德作樂,神祗是聽。



 右祠天地五郊夕牲歌一篇。



 宣文烝哉,曰靖四方。永言保之,夙夜匪康。光天之命,上帝是皇。嘉樂殷薦,靈祚景詳。神祗隆假,享福無疆。



 右祠天地五郊迎送神歌一篇。



 天祚有晉,其命惟新。受終於魏,奄有兆民。燕及皇天,懷柔百神。不顯遺烈,之德之純。享其玄牡,式用肇禋。神祇來格,福祿是臻。



 時邁其猶,昊天子之。祐享有晉,兆民戴之。畏天之威,敬授民時。不顯不承,於猶繹思。皇極斯建,庶績咸熙。庶幾夙夜,惟晉之祺。



 宣文惟後,克配彼天。撫寧四海,保有康年。於乎緝熙,肆用靖民。爰立曲制,爰修禮紀。作民之極,莫匪資始。克昌厥後,永
 言保之。



 右饗天地五郊歌三篇。



 前所作
 天地郊明堂歌五篇,傅玄造:皇
 矣有晉,時
 邁其
 德。受終於天,光濟萬國。萬國既光,神定厥祥。虔於郊祀,祗事上皇。祗事上皇,百祿是臻。巍巍祖考,克配彼天。嘉牲匪歆,德馨惟饗。受天之祚,神和四暢。



 右天地郊明堂夕牲歌。



 於赫大晉,膺天景祥。二帝邁德,宣茲重光。我皇受命,奄有萬方。郊祀配享,禮樂孔章。神祇嘉饗,祖考是皇。克昌厥後,保祚無疆。



 右天地郊明堂降神歌。



 整泰壇,祀皇神。精氣感,百靈賓。蘊硃火,燎芳薪。紫煙游,冠青雲。神之體,靡象形。曠無方,幽以清。神之來,光景照。聽無聞,視無兆。神之至,舉歆歆。靈爽協,動餘心。
 神之坐,同歡娛。澤雲翔,化風舒。嘉樂奏,文中聲。八音諧,神是聽。咸潔齋,並芬芳。烹牷牲,享玉觴。神說饗,歆禋祀。祐大晉,降繁祉。胙京邑,行四海。保天年,窮地紀。



 右天郊饗神歌。



 整泰行,俟皇祗。眾神感,群靈儀。陰祀設,吉禮施。夜將極,時未移。祗之體,無形象。潛泰幽,洞忽荒。祗之出,渼若有。靈無遠,
 天下母。祗之來,遺光景。照若存,終冥冥。祗之至,舉欣欣。舞象德,歌成文。祗之坐,同歡豫。澤雨施,化雲布。樂八變,聲教敷。物咸享,祗是娛。齋既潔,侍者肅。玉觴進,咸穆穆。饗嘉慶,歆德馨。胙有晉,暨群生。溢九壤,格天庭。保萬壽,延億齡。



 右地郊饗神歌。



 經始明堂,享祀匪懈。於皇烈考,光配上帝。
 赫赫上帝,既高既崇。聖考是配,明德顯融。率土敬職,萬方來祭。常于時假,保胙永世。



 右明堂饗神歌。



 宋南郊雅樂登歌三篇,顏延之造:夤威寶命,嚴恭帝祖。表海炳岱,系唐胄楚。靈鑑濬文,民屬睿武。奄受敷錫,宅中拓宇。亙地稱皇,罄天作主。月竁來賓,日際奉土。開元首正,禮交樂舉。六曲聯事,九官列序。
 有牷在滌,有潔在俎。以薦王衷,以答神祜。



 右天地郊夕牲歌。



 維聖饗帝,維孝饗親。皇乎備矣,有事上春。禮行宗祀,敬達郊禋。金枝中樹,廣樂四陳。陟配在京,降德在民。奔精照夜,高燎煬晨。陰明浮爍,沈崇深淪。告成大報,受釐元神。月御按節,星驅扶輪。遙興遠駕,耀耀振振。



 右天地郊迎送神歌。



 營泰畤,定天衷。思心睿,謀筮從。建表蕝,設郊宮。田燭置,爟火通。歷元旬,律首吉。飾紫壇,坎列室。中星兆,六宗秩。乾宇晏,地區謐。大孝昭,祭禮供。牲日展,盛自躬。具陳器,備禮容。形舞綴,被歌鐘。望帝閽,聳神蹕。靈之來,辰光溢。潔粢酌,娛太一。明輝夜,華晢日。稞既始,獻又終。煙薌鬯,報清穹。饗宋德,胙王功。休命永,福履充。



 右天地饗神歌。



 宋明堂歌,謝莊造:地紐謐,乾樞回。華蓋動,紫微開。旌弊日,車若雲。駕六氣,乘芃縕。曄帝京,煇天邑。聖祖降,五靈集。構瑤戺,聳珠簾。漢拂幌,月棲簷。舞綴暢,鐘石融。駐飛景,鬱行風。懋粢盛,潔牲牷。百禮肅,群司虔。皇德遠,大孝昌。貫九幽,洞三光。神之安,解玉鑾。
 景福至,萬宇歡。


右迎神歌詩。
 \gezhu{
  依漢郊祀迎神,三言,四句一轉韻}
 。



 雍臺辨朔,澤宮練辰。潔火夕照,明水朝陳。六瑚賁室,八羽華庭。昭事先聖,懷濡上靈。《肆夏》式敬,升歌發德。永固鴻基,以綏萬國。



 右登歌詞。舊四言。



 維天為大,維聖祖是則。辰居萬宇,綴旒下國。內靈八輔,外光四瀛。蒿宮仰蓋,日館希旌。
 復殿留景,重簷結風。刮楹接緯,達響承虹。設業設虛,在王庭。



 肇禋祀,克配乎靈。我將我享,維孟之春。以孝以敬,以立我烝民。



 右歌太祖文皇帝詞。依《周頌》體。



 參映夕,駟照晨。靈乘震,司青春。鴈將向,桐始蕤。柔風舞,暄光遲。萌動達,萬品新。潤無際,澤無垠。



 右歌青帝詞。三言,依木數。



 龍精初見大火中。硃光北至圭景同。帝位在《離》實司衡。水雨方降木槿榮。



 庶物盛長咸殷阜。恩覃四溟被九有。



 右歌赤帝辭。七言,依火數。



 履建宅中宇,司繩御四方。裁化遍寒燠,布政周炎涼。景麗條可結,霜明冰可折。凱風扇硃辰,白雲流素節。分至乘結晷,啟閉集恒度。帝運緝萬有,皇靈澄國步。



 右歌黃帝辭。五言,依土數。



 百川如鏡,天地爽且明。雲沖氣舉,德盛在素精。木葉初下,洞庭始揚波。夜光徹地,翻霜照懸河。庶類收成,歲功行欲寧。浹地奉渥,罄宇承秋靈。



 右歌白帝辭。九言,依金數。



 歲既晏,日方馳。靈乘坎,德司規。玄雲合,晦鳥路。白雲繁,亙天涯。雷在地,時未光。飭國典,閉關梁。四節遍,萬物殿。福九域,祚八鄉。晨晷促,夕漏延。
 大陰極,微陽宣。鵲將巢,冰已解。氣濡水,風動泉。



 右歌黑帝辭。六言,依水數。



 蘊禮容,餘樂度。靈方留,景欲暮。開九重,肅五達。鳳參差,龍已秣。雲既動,河既梁。萬里照,四空香。神之車,歸清都。旋庭寂,玉殿虛。睿化凝,孝風熾。顧靈心,結皇思。


右送神歌辭。
 \gezhu{
  漢郊祀送神,亦三言}
 。右天郊饗神歌。



 魏《俞兒舞歌》四篇,王粲造:漢初建國家,匡九州。蠻荊震服,五刃三革休。安不忘備武樂脩。宴我賓師,敬用御天,永樂無憂。子孫受百福,常與松喬遊。蒸庶德,莫不咸歡柔。



 右《矛俞新福歌》。



 材官選士,劍弩錯陳。應桴蹈節,俯仰若神。綏我武烈,篤我淳仁。自東自西,莫不來賓。



 右《弩俞新福歌》。



 我功既定,庶士咸綏。樂陳我廣庭,式宴賓與師。昭文德,宣武威。平九有,撫民黎。荷天寵,延壽尸。千載莫我違。



 右《安臺新福歌》曲。



 神武用師士素厲。仁恩廣覆,猛節橫逝。自古立功,莫我弘大。桓桓征四國,爰及海裔。漢國保長慶,垂祚延萬世。



 右《行辭新福歌》曲。



 晉《宣武舞歌》四篇,傅玄造:《惟聖皇篇》《矛俞》第一:惟聖皇,德巍巍,光四海。禮樂猶形影,文武為表裏,乃作《巴俞》。肆舞士,劍弩齊列,戈矛為之始。進退疾鷹鷂,龍戰而豹起。如亂不可亂,動作順其理,離合有統紀。



 《
 短兵篇》《劍俞》第二:劍為短兵,其勢險危。疾踰飛電,回旋應規。武節齊聲,或合或離。電發星騖,若景若差。兵法攸象,軍容是儀。



 《軍鎮篇》《弩俞》第三:弩為遠兵軍之鎮,其發有機。體難動,往必速,重而不遲。銳精分褲,射遠中微。《弩俞》之樂,一何奇!變多姿,退若激,進若飛。
 五聲協,八音諧。宣武象,贊天威。



 《窮武篇》《安臺行亂》第四:窮武者喪,何但敗北。柔弱亡戰,國家亦廢。秦始徐偃,既已作戒前世。先王鑒其機,脩文整武藝。文武足相濟,然後得光大。



 亂曰:高則亢,滿則盈。亢必危,盈必傾。去危傾,守以平。沖則久,濁能清。



 混文武,順天經。



 晉《宣文舞歌》二篇,傅玄造:《
 羽龠舞歌》:羲皇之初,天地開元。網罟禽獸,群黎以安。神農教耕,創業誠難。民得粒食,澹然無所患。黃帝始征伐,萬品造其端。軍駕無常居,是曰軒轅。



 軒轅既勤止,堯舜匪荒寧。夏禹治水,湯武又用兵。孰能保安逸,坐致太平?聖皇邁乾乾,天下興頌聲,穆穆且明明。惟聖皇,道化彰。澄四海,
 清三光。萬機理,庶事康。潛龍升,儀鳳翔。風雨時,物繁昌。卻走馬,降瑞祥。揚仄陋,簡忠良。



 百祿是荷,眉壽無疆。



 《羽鐸舞歌》:昔在渾成時,兩儀尚未分。陽升垂清景,陰降興浮雲。中和含氛氳,萬物各異群。人倫得其序,眾生樂聖君。三統繼五行,然後有質文。皇王殊運代,治亂亦繽紛。
 伊大晉,德兼往古。越犧農,邈舜禹。參天地,陵三五。禮唐周,樂《韶武》。豈唯《簫韶》六代具舉。澤霑地境,化充天宇。聖明臨朝,元凱作輔,普天同樂胥。浩浩元氣,遐哉太清。五行流邁,日月代征。隨時變化,庶物乃成。聖皇繼天,光濟群生。化之以道,萬國咸寧。受茲介福,延于億齡。



 晉宗廟歌十一篇,傅玄造:
 我夕我牲,猗歟敬止。嘉豢孔時,供茲享祀。神鑒厥誠,博碩斯歆。神考降饗,以虞孝孫之心。



 右祠廟夕牲歌。



 嗚呼悠哉!日鑒在茲。以時享祀,神明降之。神明斯降,既祐饗之。祚我無疆,受天之祜。赫赫太上,巍巍聖祖。明明烈考,丕承繼序。



 右祠廟迎送神歌。



 經始宗廟,神明戾止。申錫無疆,祗承享祀。
 假哉皇祖,綏予孫子。燕及後昆,錫茲繁祉。



 右祠征西將軍登歌。



 嘉樂肆庭,薦祀在堂。皇皇宗廟,乃祖先皇。濟濟辟公,相予烝嘗。享祀不忒,降福穰穰。



 右祠豫章府君登歌。



 於邈先后,實司于天。顯矣皇祖,帝祉肇臻。本支克昌,資始開元。惠我無疆,享祚永年。



 右祠潁川府君登歌。



 於惟曾皇,顯顯令德。高明清亮,匪競柔克。保乂命祜,基命惟則。篤生聖祖,光濟四國。



 右祠京兆府君登歌。



 於鑠皇祖,聖德欽明。勤施四方,夙夜敬止。載敷文教,載揚武烈。匡定社稷,龔行天罰。經始大業,造創帝基。畏天之命,於時保之。



 右祠宣皇帝登歌。



 執競景皇,克明克哲。旁作穆穆,惟祗惟畏。
 纂宣之緒,耆定厥功。登此雋乂,糾彼群凶。業業在位,帝既勤止。維天之命,於穆不已。



 右祠景皇帝登歌。



 於皇時晉,允文文皇。聰明睿智,聖敬神武。萬機莫綜,皇斯清之。虎兕放命,皇斯平之。柔遠能邇,簡授英賢。創業垂統,勳格皇天。



 右祠文皇帝登歌。



 曰晉是常,享祀時序。宗廟致敬,禮樂具舉。
 惟其來祭,普天率土。犧樽既奠,清酤既載。亦有和羹,薦羞斯備。蒸蒸永慕,感時興思。登歌奏舞,神樂其和。祖考來格,祐我邦家。敷天之下,罔不休嘉。



 肅肅在位,濟濟臣工。四海來格,禮儀有容。鐘鼓振,管絃理。舞開元,歌永始。神胥樂兮。肅肅在位,臣工濟濟。小大咸敬,上下有禮。理管絃,振鼓鐘。舞象德,
 歌詠功。神胥樂兮。肅肅在位,有來雍雍。穆穆天子,相惟辟公。禮有儀,樂有則。舞象功,歌詠德。神胥樂兮。右祠廟饗神歌二篇。



 晉江左宗廟歌十三篇,曹毗造十一首,王珣造二首:歌高祖宣皇帝,曹毗造:於赫高祖,德協靈符。應運撥亂,厘整天衢。勳格宇宙,化動八區。肅以典刑,陶以玄珠。神石吐瑞,靈芝自敷。肇基天命,道均唐虞。



 歌世宗景皇帝:景皇承運,纂隆洪緒。皇維重抗,天暉再舉。蠢矣二寇,擾我揚楚。乃整元戎,以膏齊斧。亹亹神算,赫赫王旅。鯨鯢既平,功冠帝宇。



 歌太祖文皇帝:太祖齊聖,王猷誕融。仁教四塞,天基累崇。皇室多難,嚴清紫宮。威厲秋霜,惠過春風。平蜀夷楚,以文以戎。奄有參墟,聲流無窮。



 歌世祖武皇帝:於穆武皇,允龔欽明。應期登禪,龍飛紫庭。百揆時序,聽斷以情。殊域既賓,偽吳亦平。晨流甘露,宵映朗星。野有擊壤,路垂頌聲。



 歌中宗元皇帝:運屯百六,天羅解貫。元皇勃興,網籠江漢。仰齊七政,俯平禍亂。化若風行,澤猶雨散。淪光更耀,金輝復煥。德冠千載,蔚有餘粲。



 歌肅祖明皇帝:明明肅祖,闡弘帝胙。英風夙發,清暉載路。姦逆縱忒,罔式皇度。躬振硃旗,遂豁天步。宏猷淵塞,高羅雲布。品物咸寧,洪基永固。



 歌顯宗成皇帝:於休顯宗,道澤玄播。式宣德音,暢物以和。邁德蹈仁,匪禮弗過。敷以純風,濯以清波。連理映阜,鳴鳳棲柯。同規放勛,義蓋山河。



 歌康皇帝:康皇穆穆,仰嗣洪德。為而不宰,雅音四塞。閑邪以誠,鎮物以默。威靜區宇,道宣邦國。



 歌孝宗穆皇帝:孝宗夙哲,休音允臧。如彼晨離,耀景扶桑。垂訓華幄,流潤八荒。幽贊玄妙,爰該曲章。西平僭蜀,北靜舊疆。高猷遠暢,朝有遺芳。



 歌哀皇帝:
 於穆哀皇,聖心虛遠。雅好玄古,大庭是踐。道尚無為,治存易簡。化若風行,民猶草偃。雖曰登遐,徽音彌闡。愔愔《雲》《韶》,盡美盡善。



 歌太宗簡文皇帝,王珣造:皇矣簡文,於昭于天。靈明若神,周淡如淵。沖應其來,實與其遷。娓娓心化,日用不言。易而有親,簡而可傳。觀流彌遠,求本愈玄。



 歌烈宗孝武皇帝,王珣造:
 天鑒有晉,欽哉烈宗。同規文考,玄默允龔。威而不猛,約而能通。神鉦一震,九域來同。道積淮海,《雅》《頌》自東。氣陶淳露,化協時雍。



 四時祠祀歌,曹毗造:肅肅清廟,巍巍聖功。萬國來賓,禮儀有容。鐘鼓振,金石熙。宣兆祚,武開基。神斯樂兮。理管絃,有來斯和。說功德,吐清歌。神斯樂兮。洋洋玄化,潤被九壤。
 民無不悅,道無不往。禮有儀,樂有式。詠九功,永無極。神斯樂兮。



 宋宗廟登歌八篇,王韶之造:綿綿遐緒,昭明載融。漢德未遠,堯有遺風。於穆皇祖,永世克隆。本枝惟慶,貽厥靡窮。



 右祠北平府君登歌。



 乃立清廟,清廟肅肅。乃備禮容,禮容穆穆。顯允皇祖,昭是嗣服。錫茲繁祉,聿懷多福。



 右祠相國掾府君登歌。



 四縣既序,簫管既舉。堂獻六瑚,庭舞八羽。先王有典,克禋皇祖。丕顯洪烈,永介休祜。



 右祠開封府君登歌。



 鐘鼓喤喤,威儀將將。溫恭禮樂,敬享曾皇。邁德垂仁,係軌重光。天命純嘏,惠我無疆。



 右祠武原府君登歌。



 鑠矣皇祖,帝度其心。永言配命,播茲徽音。
 思我茂猷,如玉如金。駿奔在陛,是鑑是歆。



 右祠東安府君登歌。



 烝哉孝皇,齊聖廣淵。發祥誕慶,景胙自天。德敷金石,道被管絃。有命既集,徽風永宣。



 右祠孝皇帝登歌。



 惟天有命,眷求上哲。赫矣聖武,撫運桓撥。功並敷土,道均汝墳。止戈曰武,經緯稱文。鳥龍失紀,云火代名。受終改物,作我宋京。
 至道惟王,大業有劭。降德兆民,升歌清廟。



 右祠高祖武皇帝登歌。



 奕奕寢廟,奉璋在庭。笙籥既列,犧象既盈。黍稷匪芳,明祀惟馨。樂具禮充,潔羞薦誠。神之格思,介以休禎。濟濟群辟,永觀厥成。


右祠七廟享神登歌。
 \gezhu{
  並以歌章太后篇。}



 世祖孝武皇帝歌,謝莊造:帝錫二祖,長世多祜。於穆睿考,襲聖承矩。
 玄極弛馭,乾紐墜緒。闢我皇維,締我宋宇。刊定四海,肇構神京。復禮輯樂,散馬墮城。澤牣九有,化浮八瀛。慶雲承掖,甘露飛甍。肅肅清廟,徽徽宮。舞蹈象德,笙磬陳風。黍稷非盛,明德惟崇。神其歆止,降福無窮。



 宣皇太后廟歌:稟祥月輝,毓德軒光。嗣徽媯,思媚周姜。母臨萬宇,訓藹紫房。硃絃玉龠,式載瓊芳。


晉四廂樂歌三首,傅玄造:天鑒有晉,世祚聖皇。時齊七政,朝此萬方。
 \gezhu{
  其一}


鐘鼓斯震,九賓備禮。正位在朝,穆穆濟濟。
 \gezhu{
  其二}


煌煌三辰,實麗於天。君后是象,威儀孔虔。
 \gezhu{
  其三}


率禮無愆,莫匪邁德。儀刑聖皇,萬邦惟則。
 \gezhu{
  其四}



 右《天鑒》四章,章四句。正旦大會行禮歌。



 於赫明明,聖德龍興。三朝獻酒,萬壽是膺。敷佑四方,如日之升。自天降祚,元吉有征。



 右《於赫》一章,八句。上壽酒歌。


天命大晉,載育群生。於穆上德,隨時化成。
 \gezhu{
  其一}


自祖配命,皇皇后辟。繼天創業,宣文之績。
 \gezhu{
  其二}


丕顯宣文,先知稼穡。克恭克儉,足教足食。
 \gezhu{
  其三}


既教食之,弘濟艱難。上帝是祐,下民所安。
 \gezhu{
  其四}


天祐聖皇,萬邦來賀。雖安勿安,乾乾匪暇。
 \gezhu{
  其五}


乃正丘郊,乃定塚社。暠暠作宗,光宅天下。
 \gezhu{
  其六}


惟敬朝饗,爰奏食舉。盡禮供御,嘉樂有序。
 \gezhu{
  其
  七}


樹羽設業,笙鏞以間。琴瑟齊列,亦有篪塤。
 \gezhu{
  其八}


喤喤鼓鐘,鎗槍磬管。八音克諧,載夷載簡。
 \gezhu{
  其九}


既夷既簡,其大不禦。風化潛興,如雲如雨。
 \gezhu{
  其十}


如雲之覆,如雨之潤。聲教所暨,無思不順。
 \gezhu{
  其十一}


教以化之,樂以和之。和而養之,時惟邕熙。
 \gezhu{
  其十二}


禮慎其儀,樂節其聲。於鑠皇繇,既和且平。
 \gezhu{
  其十三}



 右《天命》十三章,章四句。食舉東西廂歌。



 晉《正德大豫》二舞歌二篇,傅玄造:
 天命有晉,光濟萬國。穆穆聖皇,文武惟則。在天斯正,在地成德。載韜政刑,載崇禮教。我敷玄化,臻于中道。右《正德舞歌》。



 於鑠皇晉,配天受命。熙帝之光,世德惟聖。嘉樂《大豫》,保祐萬姓。淵兮不竭,沖而用之。先天弗違,虔奏天時。



 右《大豫舞歌》。



 晉四廂樂歌十七篇,荀勖造:正旦大會行禮歌四篇:於皇元首,群生資始。履端大享,敬御繁祉。肆覲群后,爰及卿士。欽順則元,允也天子。


《于皇》一章,八句。
 \gezhu{
  當《於赫》}



 明明天子,臨下有赫。四表宅心,惠浹荒貊。柔遠能邇,孔淑不逆。來格祁祁,邦家是若。


《明明》一章,八句。
 \gezhu{
  當《巍巍》}
 。



 光光邦國,天篤其祜。丕顯哲命,顧柔三祖。世德作求,奄有九土。思我皇度,彞倫攸序。


《邦國》一章,八句。
 \gezhu{
  當《洋洋》}
 。



 惟祖惟宗,高朗緝熙。對越在天,駿惠在茲。聿求厥成,我皇崇之。式固其猶,往敬用治。


《祖宗》一章,八句。
 \gezhu{
  當《鹿鳴》}



 正旦大會王公上壽酒歌一篇踐元辰,延顯融。獻羽觴,
 祈令終。我皇壽而隆,我皇茂而嵩。本枝奮百世,休祚鐘聖躬。


《踐元辰》一章,八句。
 \gezhu{
  當《羽觴行》}



 食舉樂東西廂歌十二篇:煌煌七耀,重明交暢。我有嘉賓,是應是貺。邦政既圖,接以大饗。人之好我,式遵德讓。


《煌煌》一章,八句。
 \gezhu{
  當《鹿鳴》}



 賓之初筵,藹藹濟濟。既朝乃宴,以洽百禮。
 頒以位敘,或廷或陛。登儐台叟,亦有兄弟。胥子陪僚,憲茲度楷。觀頤養正,降福孔偕。


《賓之初筵》一章,十二句。
 \gezhu{
  當《於穆》}



 昔我三後,大業是維。今我聖皇,焜耀前暉。奕世重規,明照九畿。思輯用光,時罔有違。陟禹之跡,莫不來威。天被顯祿,福履是綏。


《三后》一章,十二句。
 \gezhu{
  當《昭昭》}



 赫矣太祖,克廣明德。廓開宇宙,正世立則。
 變化不經,民無瑕慝。創業垂統,兆我晉國。


《赫矣》一章,八句。
 \gezhu{
  當《華華》}



 烈文伯考,時惟帝景。夷險平亂,威而不猛。御衡不迷,皇塗煥炳。七德咸宣,其寧惟永。


《烈文》一章,八句。
 \gezhu{
  當《朝宴》}



 猗歟盛歟,先皇聖文。則天作孚,大哉為君。慎徽五典,帝載是勤。文武發揮,茂建嘉勳。修己濟治,民用寧殷。懷遠燭幽,玄教氛氳。
 善世不伐,服事參分。德博化隆,道冒無垠。


《猗歟》一章,十六句。
 \gezhu{
  當《盛德》}



 隆化洋洋,帝命溥將。登我晉道,越惟聖皇。龍飛革運,臨燾八荒。睿哲欽明,配蹤虞唐。封建厥福,駿發其祥。三朝習吉,終然允臧。其臧惟何,總彼萬方。元侯列辟,四嶽蕃王。時見世享,率茲有常。旅揖在庭,嘉客在堂。宋衛既臻,陳留山陽。我有賓使,觀國之光。
 貢賢納計,獻璧奉璋。保祐命之,申錫無疆。


《隆化》一章,二十八句。
 \gezhu{
  當《綏萬邦》}



 振鷺于飛,鴻漸其翼。京邑穆穆,四方是式。無競惟人,王綱允敕。君子來朝,言觀其極。


《振鷺》一章,八句。
 \gezhu{
  當《朝朝》}



 翼翼大君,民之攸暨。信理天工,惠康不匱。將遠不仁,訓以淳粹。幽明有倫,俊乂在位。九族既睦,庶邦順比。開元布憲,四海鱗萃。
 協時正統,殊塗同致。厚德載物,靈心隆貴。敷奏讜言,納以無諱。樹之典象,誨之義類。上教如風,下應如卉。一人有慶,群萌以遂。我后宴喜,令聞不墜。


《翼翼》一章,二十六句。
 \gezhu{
  當《順天》}



 既宴既喜,翕是萬邦。禮儀卒度,物有其容。晢晢庭燎,喤喤鼓鐘。笙磬詠德,萬舞象功。八音克諧,俗易化從。其和如樂,庶品時邕。


《
 既宴》一章,十二句。
 \gezhu{
  當《陟天庭》}



 時邕份份,六合同塵。往我祖宣,威靜殊鄰。首定荊楚,遂平燕秦。娓娓文皇,邁德流仁。爰造草昧,應乾順民。靈瑞告符,休徵饗震。天地弗違,以和神人。既戡庸蜀,吳會是賓。肅慎率職,楛矢來陳。韓惛進樂,均協清《鈞》。西旅獻獒,扶南效珍。蠻裔重譯,玄齒文身。我皇撫之,景命惟新。


《
 時邕》一章,二十六句。
 \gezhu{
  當《參兩儀》}



 愔愔嘉會,有聞無聲。清酤既奠,籩豆既馨。禮充樂備,《簫韶》九成。愷樂飲酒,酣而不盈。率土歡豫,邦國以寧。王猷允塞,萬載無傾。



 《嘉會》一章,十二句。



 晉《正德》《大豫》二舞歌二篇,荀勖造:人文垂則,盛德有容。聲以依詠,舞以象功。干戚發揮,節以笙鏞。羽龠雲會,翊宣令蹤。
 敷美盡善,允協時邕。煥炳其章,光乎萬邦。萬邦洋洋,承我晉道。配天作享,元命有造。上化如風,民應如草。穆穆斌斌,形于綴兆。文武旁作,慶流四表。無競維烈,永世是紹。



 右《正德舞歌》。



 豫順以動,大哉惟時。時邁其仁,世載邕熙。兆我區夏,宣文是基。大業惟新,我皇隆之。重光累曜,欽明文思。迄用有成,惟晉之祺。
 穆穆聖皇,受命既固。品物咸寧,芳烈雲布。文教旁通,篤以淳素。玄化洽暢,被之暇豫。作樂崇德,同美《韶》、《濩》。浚邈幽遐,式遵王度。



 右《大豫舞歌》。



 晉四廂樂歌十六篇,張華造:稱元慶,奉壽觴。后皇延遐祚,安樂撫萬方。



 右王公上壽詩一章。



 明明在上,丕顯厥繇。翼翼三壽,蕃后惟休。群生漸德,六合承流。



 三正元辰,朝慶鱗萃。華夏奉職貢,八荒覲殊類。黻冕充廣庭,鳴玉盈朝位。



 濟濟朝位,言觀其光。儀序既以時,禮文渙以彰。思皇享多祜,嘉樂永無央。



 九賓在庭,臚贊既通。升瑞奠贄,乃侯乃公。穆穆天尊,隆禮動容。
 履端承元吉,介福御萬邦。



 朝享上,下咸雍。崇多儀,繁禮容。舞盛德,歌九功。揚芳烈,播休蹤。皇化洽,洞幽明。懷柔百神,輯祥禎。潛龍躍,雕虎仁。儀鳳鳥,屆游麟。枯蠹榮,竭泉流。菌芝茂,枳棘柔。和氣應,休徵滋。協靈符,彰帝期。綏宇宙,萬國和。昊天成命,賚皇家,
 賚皇家。



 世資聖哲,三后在天,啟鴻烈。啟鴻烈,隆王基。率土謳吟,欣戴于時。恒文示象,代氣著期。



 太始開元,龍升在位。四庾同風,燮寧殊類。五韙來備,嘉生以遂。



 凝庶績,臻太康。申繁祉,胤無疆。本枝百世,繼緒不忘。繼緒不忘,休有烈光。
 永言配命,惟晉之祥。



 聖明統世,篤皇仁。廣大配天地,順動若陶鈞。玄化參自然,至德通神明。清風暢八極,流澤被無垠。



 於皇時晉,奕奕齊聖。惟天降嘏,神祇保定。弘濟區夏,允集大命。有命既集,光帝猷。大明重耀,鑑六幽。聲教洋溢,惠滂流。
 惠滂流,移風俗。多士盈朝,賢俊比屋。敦世心,斫彫反素樸。反素樸,懷庶方。干戚舞階庭,疏狄說遐荒。扶南假重譯,肅慎襲衣裳。雲覆雨施,德洽無疆。旁作穆穆,仁化翔。



 朝元日,賓王庭。承宸極,當盛明。衍和樂,竭祗誠。仰嘉惠,懷德馨。游淳風,泳淑清。
 協億兆,同歡榮。建皇極,統天位。運陰陽,御六氣。殷群生,成性類。王道浹,治功成。人倫序,俗化清。虔明祀,祗三靈。崇禮樂,式儀刑。



 慶元吉,宴三朝。播金石,詠泠簫。奏《九夏》,舞《雲》《韶》。邁德音,流英聲。八珣一,六合寧。六合寧,承聖明。王澤洽,道登隆。綏函夏,總華戎。



 齊德教,混殊風。混殊風,康萬國。
 崇夷簡,尚敦德。弘王度,表遐則。



 右食舉東西廂樂詩十一章。



 於赫皇祖,迪哲齊聖。經緯大業,基天之命。克開洪緒,誕篤天慶。旁濟彞倫,仰齊七政。



 烈烈景皇,克明克聰,靜封略,定勳功。成民立政,儀刑萬邦。式固崇軌,光紹前蹤。



 允文烈考,浚哲應期。參德天地,比功四時。大亨以正,庶績咸熙。肇啟晉宇,遂登皇基。



 明明我后,玄德通神。受終正位,協應天人。容民厚下,育物流仁。躋我王道,暉光日新。



 右雅樂正旦大會行禮詩四章。



 晉《正德》、《大豫》二舞歌二篇,張華造:《正德舞》歌詩:曰皇上天,玄鑒惟光。神器周回,五德代章。祚命於晉,世有哲王。弘濟區夏,甄陶萬方。大明垂曜,旁燭無疆。蚩蚩庶類,風德永康。
 皇道惟清,禮樂斯經。金石在縣,萬舞在庭。象容表慶,協律被聲。軼《武》超《濩》,取節六英。同進退讓,化漸無形。太和宣洽,通于幽冥。



 《大豫舞》歌詩:惟天之命,符運有歸。赫赫大晉,三後重暉。繼明紹世,光撫九圍。我皇紹期,遂在璿璣。群生屬命,奄有庶邦。慎徽五典,玄教遐通。萬方同軌,率土咸雍。爰制《大豫》,宣德舞功。
 淳化既穆,王道協隆。仁及草木,惠加昆蟲。億兆夷人,說仰皇風。丕顯大業,永世彌崇。



 晉四廂歌十六篇,成公綏造:上壽酒,樂未央。大晉應天慶,皇帝永無疆。



 右詩一章,王公上壽酒所用。



 穆穆天子,光臨萬國。多士盈朝,莫匪俊德。流化罔極,王猷允塞。嘉會置酒,嘉賓充庭。
 羽旄耀辰極,鐘鼓振泰清。百辟朝三朝,彧彧明儀刑。濟濟鏘鏘,金振玉聲。



 禮樂具,宴嘉賓。眉壽祚聖皇,景福惟日新。群后戾止,有來雍雍。獻酬納贄,崇此禮容。豐肴萬俎,旨酒千鐘。嘉樂盡樂宴,福祿咸攸同。



 樂哉!天下安寧。道化行,風俗清。《簫韶》作,詠九成。年豐穰,世泰平。



 至治哉!樂無窮。
 元首聰明,股肱忠。澍豐澤,揚清風。



 嘉瑞出,靈應彰。麒麟見,鳳皇翔。醴泉涌,流中唐。嘉禾生,穗盈箱。降繁祉,祚聖皇。承天位,統萬國。受命應期,授聖德。四世重光,宣開洪業,景克昌,文欽明,德彌彰。肇啟晉邦,流祚無疆。



 泰始建元,鳳皇龍興。龍興伊何,享祚萬乘。奄有八荒,化育黎蒸。圖書煥炳,金石有征。
 德光大,道熙隆。被四表,格皇穹。奕奕萬嗣,明明顯融。高朗令終。



 保茲永祚,與天比崇。



 聖皇君四海,順人應天期。三葉合重光,泰始開洪基。明耀參日月,功化侔四時。宇宙清且泰,黎庶咸雍熙。善哉雍熙。



 惟天降命,翼仁祐聖。於穆三皇,載德彌盛。總齊璇璣,光統七政。百揆時序,化若神聖。四海同風,興至仁。濟民育物,擬陶鈞。
 擬陶鈞,垂惠潤。皇皇群賢,峨峨英俊。德化宣,芬芳播來胤。播來胤,垂後昆。



 清廟何穆穆,皇極闢四門。皇極闢四門,萬機無不綜。娓娓翼翼,樂不及荒,饑不遑食。大禮既行,樂無極。



 登崑崙,上增城。乘飛龍,升泰清。冠日月,佩五星。揚虹霓,建彗旌。披慶雲,廕繁榮。覽八極,游天庭。順天地,和陰陽。序四氣,
 耀三光。張帝網,正皇綱。播仁風,流惠康。邁洪化,振靈威。懷萬方,納九夷。朝閶闔,宴紫微。



 建五旗,羅鐘虡。列四縣,奏《韶》《武》。鏗金石,揚旌羽。縱八佾,巴渝舞。詠《雅》《頌》,和律呂。於胥樂,樂聖主。



 化蕩蕩,清風泄。總英雄,御俊傑。開宇宙,掃四裔。光緝熙,美聖哲。超百代,揚休烈,流景祚,顯萬世。



 皇皇顯祖,翼世佐時。寧濟六合,受命應期。神武鷹揚,大化咸熙。廓開皇衢,用成帝基。



 光光景皇,無競維烈。匡時拯俗,休功蓋世。宇宙既康,九域有截。天命降鑒,啟祚明哲。



 穆穆烈考,克明克俊。實天生德,誕膺靈運。肇建帝業,開國有晉。載德奕世,垂慶洪胤。



 明明聖帝,龍飛在天。與靈合契,通德幽玄。仰化清雲,俯育重淵。受靈之祐,於萬斯年。



 右雅樂正旦大會行禮詩十五章。



 宋四廂樂歌五篇,王韶之造:於鑠我皇,禮仁包元。齊明日月,比量乾坤。陶甄百王,稽則黃軒。訏謨定命,辰告四蕃。



 將將蕃后,翼翼群僚。盛服待晨,明發來朝。饗以八珍,樂以九《韶》。仰祗天顏,厥猷孔昭。



 法章既設,初筵長舒。濟濟列闢,端委皇除。飲和無盈。威儀有餘。溫恭在位,敬終如初。



 九功既歌,六代惟時。被德在樂,宣道以詩。穆矣太和,品物咸熙。慶積自遠,告成在茲。


右《肆夏》樂歌四章。
 \gezhu{
  客入,於四廂振作《於鑠曲》。皇帝當陽,四廂振作《將將曲》。皇帝入變服,四廂振作《於鑠》、《將將》二曲。又黃鐘、太簇二廂作《法章》、《九功》二曲。}



 大哉皇宋,長發其祥,纂系在漢,統源伊唐。德之克明,休有烈光。配天作極,辰居四方。



 皇矣我后,聖德通靈。有命自天,誕授休禎。龍飛紫極,造我宋京。光宅宇宙,赫赫明明。


右大會行禮歌二章。
 \gezhu{
  姑洗廂作。}



 獻壽爵,慶聖皇。靈祚窮二儀,休明等三光。


右王公上壽歌一章。
 \gezhu{
  黃鐘廂作。}



 明明大宋,緝熙皇道。則天垂化,光定天保。天保既定,肆覲萬方。禮繁樂富,穆穆皇皇。



 沔彼流水,朝宗天池。洋洋貢職,抑抑威儀。既習威儀,亦閑禮容。一人有則,作孚萬邦。



 烝哉我皇,固天誕聖。履端惟始,對越休慶。如天斯久,如日斯盛。介茲景福,永固駿命。



 右殿前登歌三章,別有金石。



 晨羲載耀,萬物咸睹。嘉慶三朝,禮樂備舉。元正肇始,典章暉明。萬方畢來賀,華裔充皇庭。多士盈九位,俯仰觀玉聲。恂恂俯仰,載爛其輝。鼓鐘震天區,禮容塞皇闈。思樂窮休慶,
 福履同所歸。



 五玉既獻,三帛是薦。爾公爾侯,鳴玉華殿。皇皇聖后,降禮南面。元首納嘉禮,萬邦同歡願。休哉!君臣嘉燕。建五旗,列四縣。樂有文,禮無倦。融皇風,窮一變。



 體至和,感陰陽。德無不柔,繁休祥。瑞徽璧,
 應嘉鐘。舞靈鳳,躍潛龍。景星見,甘露墜。木連理,禾同穗。玄化洽,仁澤敷。極禎瑞,窮靈符。



 懷荒裔,綏齊民。荷天祐,靡不賓。靡不賓,長世弘盛。昭明有融,繁嘉慶。



 繁嘉慶,熙帝載。合氣成和,蒼生欣戴。三靈協瑞,惟新皇代。



 王道四達,流仁布德。窮理詠乾元,
 垂訓順帝則。靈化侔四時,幽誠通玄默。



 德澤被八珣,乾寧軌萬國。



 皇猷緝,咸熙泰。禮儀煥帝庭,要荒服遐外。被髮襲纓冕,左衣任回衿帶。天覆地載,流澤汪惛。聲教布濩,德光大。



 開元辰,畢來王。奉貢職,朝后皇。鳴珩佩,觀典章。樂王度,說徽芳。陶盛化,游太康。
 丕昭明,永克昌。



 惟永初,德丕顯。齊七政,敷五典。彞倫序,洪化闡。王澤流,太平始。樹聲教,明皇紀。和靈祇,恭明祀。衍景祚,膺嘉祉。



 禮有容,樂有儀。金石陳,牙羽施。邁《武》《濩》,均《咸池》。歌《南風》,舞德稱。文武煥,頌聲興。



 王道純,德彌淑。寧八表,康九服。道禮讓,移風俗。移風俗,永克融。歌盛美,告成功。
 詠徽烈,邈無窮。


右食舉歌十章。
 \gezhu{
  黃鐘、太簇二廂更作。黃鐘作《晨羲》、《體至和》、《王道》、《開元辰》、《禮有容》五曲。太簇作《五玉》、《懷荒裔》、《皇猷緝》、《惟永初》、《王道純》五曲。}



 宋《前舞》《後舞》歌二篇,王韶之造:於赫景明,天監是臨。樂來伊陽,禮作惟陰。歌自德富,儛由功深。庭列宮縣,陛羅瑟琴。翿籥繁會,笙磬諧音。《簫韶》雖古,九成在今。道志和聲,德音孔宣。



 光我帝基,協靈配乾。
 儀刑六合,化穆自然。如彼雲漢,為章於天。熙熙萬類,陶和當年。擊轅中《韶》,永世弗騫。


右《前舞歌》一章。
 \gezhu{
  晉《正德之舞》,蕤賓廂作。}



 假樂聖后,實天誕德。積美自中,王猷四塞。龍飛在天,儀刑萬國。欽明惟神,臨朝淵默。不言之化,品物咸德。告成于天,銘勳是勒。翼翼厥猶,娓娓其仁。順天創制,因定和神。海外有截,九圍無塵。冕旒司契,垂拱臨民。
 乃舞《大豫》,欽若天人。純嘏孔休,萬載彌新。


右《後舞歌》一章。
 \gezhu{
  晉《大豫之舞》,蕤賓廂作。}


章廟樂舞歌祠。
 \gezhu{
  雜歌悉同用太廟詞,唯三后別撰。}
 殷淡造:賓出入奏《肅成樂》歌詞二章:彞承孝曲,恭事嚴聖。浹天奉贐,罄壤齊慶。司儀具序,羽容夙彰。芬枝颺烈,黼構周張。助寶奠軒,酎珍充庭。璆縣凝會,涓硃佇聲。先期選禮,肅若有承。祗對靈祉,皇慶昭膺。



 尊事威儀,暉容昭敘。迅恭神明,梁盛牲俎。肅肅嚴宮,藹藹崇基。皇靈降祉,百祗具司。戒誠望夜,端列承朝。依微昭旦,物色輕霄。鴻慶遐鬯,嘉薦令芳。翊帝明德,永祚流光。



 牲出入奏《引牲樂》歌詞:維誠潔饗,維孝奠靈。敬芬黍稷,敬滌犧牲。騂繭在豢,載溢載豐。以承宗祀,以肅皇衷。蕭芳四舉,華火周傳。神監孔昭,嘉是柔牷。



 薦豆呈毛血奏《嘉薦樂》歌詞:肇禋戒祀,禮容咸舉。六典飾文,九司昭序。牲柔既昭,儀剛既陳。恭滌惟清,敬事惟神。加籩再御,兼俎重薦。節動軒越,聲流金縣。奕奕幄,娓娓嚴闈。潔誠夕鑑,端服晨暉。聖靈戾止,翊我皇則。上綏四宇,下洋萬國。永言孝饗,孝饗有容。儐僚贊列,肅肅雍雍。



 右夕牲歌詞。



 迎神奏《韶夏樂》歌詞:宮黝黝,復殿微微。璇除肅炤,釭璧彤輝。黼帟神凝,玉堂嚴馨。圓火夕耀,方水朝清。金枝委樹,翠鐙佇縣。渟波澄宿,華漢浮天。恭事既夙,虔心有慕。仰降皇靈,俯寧依祚。



 皇帝入廟北門奏《永至樂》歌詞:皇明鬯矣,孝容以昭。鑾華羽迾,拂漢涵滈。申申嘉夜,翊翊休朝。行金景送,步玉風《韶》。
 師承祀則,肅對禋祧。



 太祝稞地奏登歌樂詞二章:帝容承祀,練時涓日。九重徹關,四靈賓室。肅倡函音,庶旄委佾。休靈告饗,嘉薦尚芬。玉瑚飾列,桂簋昭陳。具司選禮,翼翼振振。



 稞崇祀典,酎恭孝時。禮無爽物,信靡媿詞。精華孚鬯,誠監昭通。升歌翊節,下管調風。皇心履變,敬明尊親。大哉孝德,至矣交神。



 章皇太后神室奏《章德凱容》之樂舞歌詞:幽瑞浚靈,表彰嬪聖。翊載徽文,敷光崇慶。上緯纏祥,中維飾詠。永屬輝猷,聯昌景命。



 昭皇太后神室奏《昭德凱容》之樂舞歌詞,明帝造:表靈纏象,纘儀緯風。膺華丹耀,登瑞紫穹。訓形霄宇,武彰宸宮。騰芬金會,寫德聲容。



 宣皇太后神室奏《宣德凱容》之樂舞歌詞,
 明帝造:天樞凝耀,地紐儷輝。聯光騰世,炳慶翔機。薰藹中宇,景纏上微。玉頌鏤德,金龠傳徽。



 皇帝還東壁受福酒奏《嘉時》之樂舞詞:禮薦洽,福時昌。皇聖膺嘉祐,帝業凝休祥。居極乘景運,宅德瑞中王。澄明臨四表,精華延八鄉。洞海周聲惠,徹宇麗乾光。靈慶纏世祉,鴻烈永無疆。



 送神奏《昭夏》之樂舞歌詞二章:大孝備,盛禮豐。神安留,嘉樂充。旋駕聳,汎青穹。延八虛,闢四空。藹流景,肅行風。



 昭融教,緝風度。戀皇靈,結深慕。解羽縣,輟華樹。背璇除,端玉輅。流汪惛,慶國步。



 皇帝詣便殿奏《休成》之樂歌詞:釃醴具登,嘉俎咸薦。饗洽誠陳,禮周樂遍。祝詞罷稞,序容輟縣。蹕動端庭,鑾回嚴殿。
 神儀駐景,華漢亭虛。八靈案衛,三祗解途。翠蓋耀澄,璟奕凝宸。玉鑣息節,金輅懷音。式誠達孝,底心肅感。追憑皇鑒,思承淵範。神錫懋祉,四緯昭明。仰福帝徽,俯齊庶生。



\end{pinyinscope}