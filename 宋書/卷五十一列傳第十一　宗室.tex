\article{卷五十一列傳第十一 宗室}

\begin{pinyinscope}

 長沙景王道
 憐臨川烈武王道規營浦侯遵
 考
 長沙景王道憐,高祖中弟也。初為國子學生。謝琰為徐州,命為從事史。高祖克京城,進平京邑,道憐常留家侍慰太后。桓玄走,大將軍武陵王遵承制,除員外散騎侍
 郎。尋遷建威將軍、南彭城內史。



 時北青州刺史劉該反,引索虜為援,清河、陽平二郡太守孫全聚眾應之。義熙元年,索虜托跋開遣偽豫州刺史索度真、大將軍斛斯蘭寇徐州,攻相縣,執鉅鹿太守賀申,進圍寧朔將軍羊穆之於彭城;穆之告急,道憐率眾救之。軍次陵柵,斬全。



 進至彭城,真、蘭退走。道憐率寧遠將軍孟龍符、龍驤將軍孔隆及穆之等追,真、蘭走奔相城;又追躡至光水溝,斬劉該,虜眾見殺及赴水死略盡。



 高祖鎮京口,進道憐號龍驤將軍,
 又領堂邑太守,戍石頭。明年,加使持節、監征蜀諸軍事,率冠軍將軍劉敬宣等伐譙縱,而文處茂、溫祚據險不得進,故不果行。以義勛封新興縣五等侯。四年,代諸葛長民為并州刺史、義昌太守,將軍、內史如故。猶戍石頭。



 時鮮卑侵逼,自彭城以南,民皆保聚,山陽、淮陰諸戍,並不復立。道憐請據彭城,以漸修創,朝議以彭城縣遠,使鎮山陽。進號征虜將軍、督淮北軍郡事、北東海太守,并州刺史、義昌太守如故。以破索度真功,封新渝縣男,食
 邑五百戶。



 從高祖征廣固,常為軍鋒。及城陷,慕容超將親兵突圍走,道憐所部獲之。加使持節,進號左將軍。七年,解并州,加北徐州刺史,移鎮彭城。八年,高祖伐劉毅,徵為都督袞青二州晉陵京口淮南諸郡軍事、兗青州刺史,持節、將軍、太守如故,還鎮京口。九年,甲仗五十人入殿。以廣固功,改封竟陵縣公,食邑千戶。減先封戶邑之半,以賜次子義宗。十年,進號中軍將軍,加散騎常侍,給鼓吹一部。明年,討司馬休之,道憐監留府事,甲仗
 百人入殿。江陵平,以為都督荊湘益秦寧梁雍七州諸軍事、驃騎將軍、開府儀同三司、鎮護南蠻校尉、荊州刺史,持節,常侍如故。



 北府文武悉配之。道憐素無才能,言音甚楚,舉止施為,多諸鄙拙。高祖雖遣將軍佐輔之,而貪縱過甚,畜聚財貨,常若不足,去鎮之日,府庫為之空虛。



 高祖平定三秦,方思外略,徵道憐還為侍中、都督徐兗青三州揚州之晉陵諸軍事、守尚書令、徐袞二州刺史,持節、將軍如故。元熙元年,解尚書令,進位司空,出鎮京
 口。高祖受命,進位太尉,封長沙王,食邑五千戶,持節、侍中、都督、刺史如故。永初二年朝正,入住殿省。先是,盧陵王義真為揚州刺史,太后謂上曰:「道憐汝布衣兄弟,故宜為揚州。」上曰:「寄奴於道憐豈有所惜。揚州根本所寄,事務至多,非道憐所了。」太后曰:「道憐年出五十,豈當不如汝十歲兒邪?」上曰:「車士雖為刺史,事無大小,悉由寄奴。道憐年長,不親其事,於聽望不足。」



 太后乃無言。車士,義真小字也。



 三年春,高祖不豫,加班劍三十人。時道憐
 入朝,留司馬陸仲元居守,刁逵子彌為亡命,率數十人入京城,仲元擊斬之。先是,府中陳㹠告彌有異謀,至是賜錢二十萬,除縣令。五月,宮車晏駕,道憐疾患不堪臨喪。六月,薨,年五十五。追贈太傅,持節、侍中、都督、刺史如故。祭禮依晉太宰安平王故事,鸞輅九旒,黃屋左纛,轀輬挽歌二部,前後部羽葆、鼓吹,虎賁班劍百人。



 太祖元嘉九年,詔曰:「古者明王經國,司勳有典,平章以馭德刑,班瑞以疇功烈,銘徽庸於鼎彞,配祫祀於清廟。是以從
 饗先王,義存商誥,祭於大蒸,禮著周典。自漢迄晉,世崇其文,王猷既昭,幽顯咸秩。先皇經緯天地,撥亂受終,駿命爰集,光宅區宇。雖聖明淵運,三靈允協,抑亦股肱翼亮之勤,祈父宣力之效。



 故使持節、侍中、都督南徐兗二州揚州之晉陵京口諸軍事、太傅、南徐兗二州刺史長沙景王,故侍中、大司馬臨川烈武王,故司徒南康文宣公穆之,侍中、衛將軍、開府儀同三司、錄尚書事、揚州刺史華容縣開國公弘,使持節、散騎常侍、都督江州豫州
 西陽新蔡晉熙四郡軍事、征南大將軍、開府儀同三司、江州刺史永修縣開國公道濟,故左將軍、青州刺史龍陽縣開國侯鎮惡,或履道廣流,秉德沖邈,或雅量高劭,風鑒明遠,或識唯知正,才略開邁,咸文德以熙帝載,武功以隆景業,固以侔蹤姬旦,方軌伊、邵者矣。朕以寡德,纂戎鴻緒,每惟道勛,思遵令典,而大常未銘,從祀尚闕,鑒寐欽屬,永言深懷。便宜敬是前式,憲茲嘉禮,勒功天府,配祭廟庭,俾示徽章,垂美長世,茂績遠猷,永傳不朽。」



 道
 憐六子:義欣嗣、義慶、義融、義宗、義賓、義綦。



 義欣,為員外散騎侍郎,不拜。歷中領軍,征虜將軍,青州刺史、魏郡太守,將軍如故,戍石頭。元嘉元年,進號後將軍,加散騎常侍。三年,以本號為南兗州刺史。七年,到彥之率大眾入河,義欣進彭城,為眾軍聲援。彥之退敗,青、齊搔擾,將佐慮寇大至,勸義欣委鎮還都,義欣堅志不動。遷使持節、監豫司雍并四州諸軍事、豫州刺史,將軍如故。給鼓吹一部。鎮壽陽。



 于時土境荒毀,人民雕散,城郭頹敗,盜賊公
 行。義欣綱維補緝,隨宜經理,劫盜所經,立討誅之制。境內畏服,道不拾遺,城府庫藏,並皆完實,遂為盛籓彊鎮。時淮西、江北長吏,悉敘勞人武夫,多無政術。義欣陳之曰:「江淮左右,土瘠民疏,頃年以來,薦飢相襲,百城彫弊,於今為甚。綏牧之宜,必俟良吏。勞人武夫,不經政術,統內官長,多非才授。東南殷實,猶或簡能,況賓接荒垂,而可輯柔頓闕。願敕選部,必使任得其人,庶得不勞而治。」芍陂良田萬餘頃,堤堨久壞,秋夏常苦旱。義欣遣咨議
 參軍殷肅循行修理。有舊溝引渒水入陂,不治積久,樹木榛塞。肅伐木開榛,水得通注,旱患由是得除。十年,進號鎮軍將軍,進監為都督。十一年夏,入朝,太祖厚加恩禮。十六年,薨,時年三十六。追贈散騎常侍、征西將軍、開府儀同三司,持節、都督、刺史如故。謚曰成王。



 子悼王瑾,字彥瑜,官至太子屯騎校尉。三十年,為元凶所殺。世祖即位,追贈散騎常侍。子粲早夭,粲弟纂,字元績嗣,官至步兵校尉。順帝昇明二年薨,會齊受禪,國除。



 瑾弟祗,字
 彥期,大明中為中書郎。太宰江夏王義恭領中書監,服親不得相臨,表求解職。世祖詔曰:「昔二王兩謝,俱至崇禮,自今三臺五省,悉同此例。」太宗初,為南兗州刺史、都官尚書,謀應晉安王子勛為逆,伏誅。



 祗弟楷,祕書郎,為元凶所殺,追贈通直郎。楷弟瞻,晉安太守,與子勛同逆,伏誅。瞻弟韞,字彥文,步兵校尉,宣城太守。子勛為亂,大眾屯據鵲尾,攻逼宣城。于時四方牧守,莫不同逆,唯韞棄郡赴朝廷;太宗嘉其誠,以為黃門郎,太子中庶子,侍
 中,加荊、湘州,南兗州刺史,吳興太守。侍中,領左軍將軍。又改領驍騎將軍,撫軍將軍,雍州刺史。侍中,領右衛將軍。改領左衛將軍、散騎常侍、中領軍。昇明元年,謀反伏誅。韞人才凡鄙,以有宣城之勳,特為太宗所寵。在湘州及雍州,使善畫者圖其出行鹵簿羽儀,常自披玩。嘗以此圖示征西將軍蔡興宗,興宗戲之,陽若不解畫者,指韞形像問曰:「此何人而在輿上?」韞曰:「此正是我。」其庸鄙如此。



 韞弟弼,武昌太守,亦與子勛同逆,伏誅。



 弟鑒,員外
 散騎侍郎,蚤卒。



 監弟勰,字彥和,侍中,吳興太守,後廢帝元徽元年卒。



 勰弟顥,字彥明,侍中、左衛將軍,冠軍將軍、吳興太守,未拜,元徽四年卒,追贈右將軍。



 顥弟述,東陽太守,黃門郎,與從弟秉同逆,事敗走白山,追禽伏誅。



 義欣弟義慶,出繼臨川烈武王道規。



 義慶弟義融,永初元年,封桂陽縣侯,食邑千戶。凡王子為侯者,食邑皆千戶。



 義融歷侍中,左衛將軍,太子中庶子,五兵尚書,領軍。有質幹,善於用短楯。元嘉十八年,卒,追贈車騎將軍,謚曰恭
 侯。



 子孝侯顗嗣,官至太子翊軍校尉,為元凶所殺。世祖即位,追贈散騎常侍。無子,弟襲以子晃繼封。昇明二年,與員外散騎侍郎安成戢仁祖、荒人王武連、羽林副彭元俊等謀反,國除。



 襲字茂德,太子舍人,安成太守。晉安王子勛為逆,襲據郡距之,子勛遣軍攻圍不能下。太宗嘉之,以為郢州刺史,封建陵縣侯,食邑五百戶。建陵縣屬蒼梧郡,以道遠,改封臨澧縣侯。始六年,卒於中護軍。追贈護軍將軍,加散騎常侍,謚曰忠侯。襲亦庸鄙,在郢
 州,暑月露軍上聽事,綱紀正伏閣,怪之,訪問,乃知是襲。子旻嗣,昇明三年,改封東昌縣侯,與兄晃俱伏誅。



 襲弟彪,秘書郎;弟寔,太子舍人,並蚤卒。寔弟爽,海陵太守。



 義融弟義宗,幼為高祖所愛,字曰伯奴,賜爵新渝縣男。永初元年,進爵為侯,歷黃門侍郎,太子左衛率。元嘉八年,坐門生杜德靈放橫打人,還弟內藏,義宗隱蔽之,免官。德靈雅有姿色,為義宗所愛寵,本會稽郡吏。謝方明為郡,方明子惠連愛幸之,為之賦詩十餘首,《乘流遵歸渚》篇
 是也。又為侍中、太子詹事,加散騎常侍、征虜將軍、南兗州刺史。二十一年,卒,追贈散騎常侍、平北將軍,謚曰惠侯。愛士樂施,兼好文籍,世以此稱之。



 子懷侯玠嗣,瑯邪、秦郡太守。為元凶所殺,追贈散騎常侍。無子,弟秉以子承繼封。



 秉字彥節,初為著作郎,歷羽林監,越騎校尉,中書、黃門侍郎。太宗泰始初,為侍中,頻徙左衛將軍,丹陽尹,太子詹事,吏部尚書。時宗室雖多,材能甚寡。



 秉少自砥束,甚得朝野之譽,故為太宗所委。五年,出為前將軍、
 淮南宣城二郡太守,不拜,還復本任。復為侍中,守秘書監,領太子詹事。未拜,遷使持節、都督南徐徐兗豫青冀六州諸軍事、後將軍、南徐州刺史,加散騎常侍。後廢帝即位,改都督郢州豫州之西陽司州之義陽二郡諸軍事、郢州刺史,持節、常侍如故。未拜,留為尚書左僕射,參選。元徽元年,領吏部,加兵五百人。尋領衛尉,辭不拜。桂陽王休範為逆,中領軍劉勔出守石頭,秉權兼領軍將軍,所給加兵,自隨入殿。二年,加散騎常侍、丹陽尹,解吏
 部。封當陽縣侯,食邑千戶。與齊王、袁粲、褚淵分日入直決機事。四年,遷中書令,加撫軍將軍,常侍、尹如故。順帝即位,轉尚書令、中領軍,將軍如故。



 時齊王輔政,四海屬心,秉知鼎命有在,密懷異圖。袁粲鎮石頭,不識天命,沈攸之舉兵反,齊王入屯朝堂,粲潛與秉及諸大將黃回等謀欲作亂。本期夜會石頭,旦乃舉兵。秉素恇怯騷動,擾不自安,再食甫後,便自丹陽郡車載婦女,盡室奔石頭,部曲數百,赫奕滿道。既至見粲,粲驚曰:「何遽便來,事今
 敗矣!」秉曰:「今得見公,萬死亦何恨。」從弟中領軍韞,直在省內,與直閣將軍卜伯興謀,其夜共攻齊王。會秉去事覺,齊王夜使驍騎將軍王敬則收韞。韞已戒嚴,敬則率壯士直前,韞左右皆披靡,因殺之,伯興亦伏誅。粲敗,秉踰城出走,於額簷湖見擒,與二子承、俁並死。秉時年四十五。秉妻蕭氏,思話女也。元徽中,朝廷危殆,妻常懼禍敗,每謂秉曰:「君富貴已足,故應為兒子作計。年垂五十,殘生何足吝邪!」



 秉不能從。



 秉弟謨,奉朝請。謨弟遐,字彥
 道,亦奉朝請、員外散騎侍朗。與嫡母殷養女雲敷私通,殷每禁之。殷暴病卒,未大殮,口鼻流血,疑遐潛加毒害,為有司所糾。



 世祖徙之始安郡,永光中,得還。太宗世,歷黃門侍郎,都官尚書,吳郡太守。兄秉既死,齊王遣誅之。遐人才甚凡,自諱名,常對賓客曰:「孝武無道,枉我殺母。」



 其頑騃若此。秉當權,遐累求方伯,秉曰:「我在,用汝作州,於聽望不足。」遐曰:「富貴時則云不可相關,從坐之日,為得免不?」至是果死焉。



 義宗弟義賓,元嘉二年,封新野縣
 侯。六年,以新野荒敝,改封興安縣侯。黃門郎,秘書監,左衛將軍,位至輔國將軍、徐州刺史。二十五年,卒,追贈後將軍,謚曰肅侯。子惠侯綜嗣。卒。子憲嗣。昇明二年,齊受禪,國除。綜弟琨,晉平太守。



 義賓弟義綦,元嘉六年,封營道縣侯。凡鄙無識知,每為始興王濬兄弟所戲弄。



 浚嘗謂義綦曰:「陸士衡詩云:『營道無烈心。』其何意苦阿父如此?」義綦曰:「下官初不識,何忽見苦。」其庸塞可笑類若此。歷右衛將軍,湘州刺史。孝建二年,卒,贈平南將軍,謚曰
 僖侯。子長猷嗣,官至步兵校尉。昇平三年,卒。齊受禪,國除。



 臨川烈武王道規,字道則,高祖少弟也。少倜儻有大志,高祖奇之,與謀誅桓玄。時桓弘鎮廣陵,以為征虜中兵參軍。高祖克京城,道規亦以其日與劉毅、孟昶共斬弘,收眾濟江。進平京邑,玄敗走。晉大將軍武陵王遵承制,以道規為振武將軍、義昌太守。



 與劉毅、何無忌追玄。玄西走江陵,留郭銓、何澹之等固守盆口,義軍既至,賊列艦
 距之。澹之空設羽儀旗幟於一舫,而別在它船,無忌欲攻羽儀所在,眾悉不同,曰:「澹之必不在此舫,雖得無益也。」無忌曰:「澹之不在此舫,固不須言也。既不在此,則戰士必弱,我以勁兵攻之,必可禽也。禽之之日,彼必以為失其軍主,我徒咸謂已得賊帥,我勇而彼懼,懼而薄之,破之必矣。」道規喜曰:「此名計也。」因往彼攻之,即禽此舫。因鼓噪倡曰:「已斬何澹之!」賊徒及義軍並以為然。因縱兵,賊眾奔敗,即克盆口,進平尋陽。因復馳進,遇玄於崢
 嶸洲。道規等兵不滿萬人,而玄戰士數萬,眾並憚之,欲退還尋陽。道規曰:「不可。彼眾我寡,強弱異勢。今若畏懦不進,必為所乘,雖至尋陽,豈能自固。玄雖竊名雄豪,內實恇怯,加已經奔敗,眾無固心。決機兩陣,將雄者克。昔光武昆陽之戰,曹操官渡之師,皆以少制多,共所聞也。今雖才謝古人,豈可先為之弱!」因麾眾而進,毅等從之,大破玄軍。郭銓與玄單舸走,江陵不復能守,欲入蜀,為馮遷所斬。



 義軍遇風不進,桓謙、桓振復據江陵,毅留巴
 陵,道規與無忌俱進攻桓謐於馬頭,桓蔚於寵洲,皆破之。無忌欲乘勝直造江陵,道規曰:「兵法屈申有時,不可茍進。諸桓世居西楚,群小皆為竭力,振勇冠三軍,難與爭勝。且可頓兵養銳,徐以計策縻之,不憂不克也。」無忌不從,果為振所敗。乃退還尋陽,繕治舟甲,復進軍夏口。偽鎮軍將軍馮該戍夏口東岸,揚武將軍孟山圖據魯城,輔國將軍桓仙客守偃月壘。於是毅攻魯城,道規、無忌攻偃月,並克之,生禽仙客、山圖。其夕,該遁走,進平巴陵。
 謙、振遣使求割荊、江二州,奉歸晉帝,不許。會南陽太守魯宗之起義攻襄陽,偽雍州刺史桓蔚走江陵。宗之進至紀南,振自往距之,使桓謙留守。時毅、道規已次馬頭,馳往襲,謙奔走,即日克江陵城。振大破宗之而歸,聞城已陷,亦走。無忌翼衛天子還京師,道規留夏口。江陵之平也,道規推毅為元功,無忌為次功,自居其末。進號輔國將軍、督淮北諸軍事、并州刺史,義昌太守如故。



 時荊州、湘、江、豫猶多桓氏餘燼,往往屯結。復以本官進督江
 州之武昌、荊州之江夏隨郡義陽綏安、豫州之西陽汝南潁川新蔡九郡諸軍事,隨宜剪撲,皆悉平之。以義勛封華容縣公,食邑三千戶。遷使持節、都督荊寧秦梁雍六州司州之河南諸軍事、領護南蠻校尉、荊州刺史,將軍如故。辭南蠻以授殷叔文。叔文被誅,乃復還領。善於為治,刑政明理,士民莫不畏而愛之。劉敬宣征蜀不克,道規以督統降為建威將軍。



 盧循寇逼京邑,道規遣司馬王鎮之及揚武將軍檀道濟、廣武將軍到彥之等赴
 援朝廷,至尋陽,為賊黨荀林所破。循即以林為南蠻校尉,分兵配之。使乘勝伐江陵,揚聲云徐道覆已克京邑。而桓謙自長安入蜀,譙縱以謙為荊州刺史,厚加資給,與其大將譙道福俱寇江陵,正與林會。林屯江津,謙軍枝江,二寇交逼,分絕都邑之間。荊楚既桓氏義舊,並懷異心。道規乃會將士,告之曰:「桓謙今在近畿,聞者頗有去就之計。吾東來文武,足以濟事。若欲去者,本不相禁。」因夜開城門,達曉不閉,眾咸憚服,莫有去者。雍州刺史
 魯宗之率眾數千自襄陽來赴。或謂宗之未可測,道規乃單馬迎之,宗之感悅。眾議欲使檀道濟、到彥之與宗之共擊,道規曰:「盧循擁隔中流,扇張同異,桓謙、荀林更相首尾。人懷危懼,莫有固心,成敗之機,在此一舉。非吾自行,其事不決。」乃使宗之居守,委以腹心,率諸軍攻謙。



 諸將佐皆固諫曰:「今遠出討謙,其勝難必。荀林近在江津,伺人動靜。若來攻城,宗之未必能固,脫有差跌,大事去矣。」道規曰:「諸君不識兵機耳。荀林愚豎,無它奇計。以
 吾去未遠,必不敢向城。吾今取謙,往至便克,沈疑之間,已自還反。



 謙敗則林破膽,豈暇得來?。且宗之獨守,何為不支數日。」解南蠻校尉印以授咨議參軍劉遵。馳往攻謙,水陸齊進。謙大敗,單舸走,欲下就林,追斬之。還至浦口,林又奔散。劉遵率軍追林,至巴陵,斬之。



 初,謙至枝江,江陵士庶皆與謙書,言城內虛實,咸欲謀為內應。至是參軍曹仲宗檢得之,道規悉焚不視,眾於是大安。進號征西將軍。先是,桓歆子道兒逃于江西,出擊義陽郡,與
 盧循相連接,循使蔡猛助之。道規遣參軍劉基破道兒於大薄,臨陳斬猛。



 徐道覆率眾三萬,奄至破塚,魯宗之已還襄陽,追召不及,人情大震。或傳循已平京師,遣道覆上為刺史,江漢士庶感焚書之恩,無復貳志。道規使劉遵為游軍,自距道覆於豫章口。前驅失利,道規壯氣愈厲,激揚三軍;遵自外橫擊,大破之。



 斬首萬餘級,赴水死者殆盡,道覆單舸走還盆口。初使遵為游軍,眾咸云:「今彊敵在前,唯患眾少,不應割削見力,置無用之地。」及
 破道覆,果得游軍之力,眾乃服焉。



 遵字慧明,臨淮海西人,道規從母兄蕭氏舅也。官至右將軍、宣城內史、淮南太守。義熙十年,卒,追贈撫軍將軍。追封監利縣侯,食邑七百戶。



 道規進號征西大將軍、開府儀同三司,加散騎常侍,固辭。俄而寢疾,改授都督豫江二州揚州之宣城淮南盧江歷陽安豐堂邑六郡諸軍事、豫州刺史,持節、常侍、將軍如故。以疾不拜。八年閏月,薨于京師,時年四十三。,追贈侍中、司徒,加班劍二十人。謚曰烈武公。平
 桓謙功,進封南郡公,邑五千戶。高祖受命,贈大司馬,追封臨川王,食邑如先。



 道規無子,以長沙景王第二子義慶為嗣。初,太祖少為道規所養,高祖命紹焉,咸以禮無二繼,太祖還本,而定義慶為後。義慶為荊州,廟主當隨往江陵,太祖詔曰:「褒崇道勛,經國之盛典;尊親追遠,因心之所隆。故侍中、大司馬臨川烈武王,體道欽明,至德淵邈,睿哲自天,孝友光備。爰始協規,則翼贊景業;陵威致討,則克剪梟鯨。逮妖逆交侵,方難孔棘,勢踰累綦,人無固志。王
 神謨獨運,靈武宏發,輯寧內外,誅覆群凶,固已化被江漢,勳高微管,遠猷侔於二南,英雄邁於兩獻者矣。朕幼蒙殊愛,德廕特隆,豐恩慈訓,義深情戚,永惟仁範,感慕纏懷。



 今當擁移寢祏,初祀西夏,思崇嘉禮,式備徽章,庶以昭宣風度,允副幽顯。其追崇丞相,加殊禮,鸞輅九旒,黃屋左纛,給節鉞、前後部羽葆、鼓吹、虎賁班劍百人,侍中如故。」及長沙太妃檀氏、臨川太妃曹氏後薨,祭皆給鸞輅九旒,黃屋左纛,紵輬車,挽歌一部,前後部羽葆、鼓
 吹,虎賁班劍百人。



 義慶幼為高祖所知,常曰:「此吾家豐城也。」年十三,襲封南郡公。除給事,不拜。義熙十二年,從伐長安,還拜輔國將軍、北青州刺史,未之任,徙督豫州諸軍事、豫州刺史,復督淮北諸軍事,豫州刺史、將軍並如故。永初元年,襲封臨川王。徵為侍中。元嘉元年,轉散騎常侍,秘書監,徙度支尚書,遷丹陽尹,加輔國將軍、常侍並如故。



 時有民黃初妻趙殺子婦,遇赦應徙送避孫仇。義慶曰:「案《周禮》,父母之仇,避之海外,雖遇市朝,鬥不反兵。
 蓋以莫大之冤,理不可奪,含戚枕戈,義許必報。至於親戚為戮,骨肉相殘,故道乖常憲,記無定準,求之法外,裁以人情。



 且禮有過失之宥,律無仇祖之文。況趙之縱暴,本由於酒,論心即實,事盡荒耄。



 豈得以荒耄之王母,等行路之深仇。臣謂此孫忍愧銜悲,不違子義,共天同域,無虧孝道。」



 六年,加尚書左僕射。八年,太白星犯右執法,義慶懼有災禍,乞求外鎮。太祖詔譬之曰:「玄象茫昧,既難可了。且史家諸占,各有異同,兵星王時,有所干犯,乃主當誅。以此
 言之,益無懼也。鄭僕射亡後,左執法嘗有變,王光祿至今平安。日蝕三朝,天下之至忌,晉孝武初有此異,彼庸主耳,猶竟無他。天道輔仁福善,謂不足橫生憂懼。兄與後軍,各受內外之任,本以維城,表裏經之,盛衰此懷,實有由來之事。設若天必降災,寧可千里逃避邪?既非遠者之事,又不知吉凶定所;若在都則有不測,去此必保利貞者,豈敢茍違天邪?」義慶固求解僕射,乃許之,加中書令,進號前將軍,常侍、尹如故。在京尹九年,出為使持
 節、都督荊雍益寧梁南北秦七州諸軍事、平西將軍、荊州刺史。荊州居上流之重,地廣兵彊,資實兵甲,居朝廷之半,故高祖使諸子居之。義慶以宗室令美,故特有此授。性謙虛,始至及去鎮,迎送物並不受。



 十二年,普使內外群官舉士,義慶上表曰:「詔書疇咨群司,延及連牧,旌賢仄陋,拔善幽遐。伏惟陛下惠哲光宣,經緯明遠,皇階藻曜,風猷日昇,而猶詢衢室之令典,遵明臺之睿訓,降淵慮於管庫,紆聖思乎版築,故以道邈往載,德高前王。臣
 敢竭虛暗,祗承明旨。伏見前臨沮令新野庾實,秉真履約,愛敬淳深。昔在母憂,毀瘠過禮;今罹父疚,泣血有聞。行成閨庭,孝著鄰黨,足以敦化率民,齊教軌俗。前征奉朝請武陵龔祈,恬和平簡,貞潔純素,潛居研志,耽情墳籍,亦足鎮息頹競,獎勖浮動。處士南郡師覺,才學明敏,操介清修,業均井渫,志固冰霜。



 臣往年辟為州祭酒,未污其慮。若朝命遠暨,玉帛遐臻,異人間出,何遠之有。」



 義慶留心撫物,州統內官長親老,不隨在官舍者,年聽遣
 五吏餉家。先是,王弘為江州,亦有此制。在州八年,為西土所安。撰《徐州先賢傳》十卷,奏上之。又擬班固《典引》為《典敘》,以述皇代之美。十六年,改授散騎常侍、都督江州豫州之西陽晉熙新蔡三郡諸軍事、衛將軍、江州刺史,持節如故。十七年,即本號都督南兗徐兗青冀幽六州諸軍事、南兗州刺史。尋加開府儀同三司。



 為性簡素,寡嗜欲,愛好文義,文詞雖不多,然足為宗室之表。受任歷籓,無浮淫之過,唯晚節奉養沙門,頗致費損。少善騎乘,及
 長以世路艱難,不復跨馬。



 招聚文學之士,近遠必至。太尉袁淑,文冠當時;義慶在江州,請為衛軍咨議參軍。



 其餘吳郡陸展、東海何長瑜、鮑照,等,並為辭章之美,引為佐史國臣。太祖與義慶書,常加意斟酌。



 鮑照,字明遠,文辭贍逸,嘗為古樂府,文甚遒麗。元嘉中,河、濟俱清,當時以為美瑞,照為《河清頌》,其序甚工。其辭曰:臣聞善談天者,必徵象於人;工言古者,先考績於今。鴻、犧以降,遐哉邈乎,鏤山岳,彫篆素,昭德垂勛,可謂多矣。而史編唐堯
 之功,載「格于上下,」樂登文王之操,稱「於昭于天」。素狐玄玉,聿彰符命,朴牛大螾,爰定祥歷,魚鳥動色,禾雉興讓,皆物不盈眥,而美溢金石。詩人於是不作,頌聲為之而寢,庸非惑歟。



 自我皇宋之承天命也,仰符應龍之精,俯協河龜之靈,君圖帝寶,粲爛瑰英,固業光曩代,事華前德矣。聖上天飛踐極,迄茲二十四載。道化周流,玄澤汪水歲。



 地平天成,上下含熙;文同軌通,表裏禔福。耀德中區,黎庶知讓;觀英遐表,夷貉懷惠。恤勤秩禮,罷露臺之
 金;紓國振民,傾鉅橋之粟。約違迫脅,奢去泰甚。



 燕無留飲,畋不盤樂。物色異人,優游據正。顯不失心,幽無怨氣。精炤日月,事洞天情。故不勞杖斧之臣,號令不嚴而自肅;無辱鳳舉之使,靈怪不召而自彰。萬里神行,飆塵不起。農商野廬,邊城偃柝。冀馬南金,填委內府;馴象西爵,充羅外囿。阿紈綦組之饒,衣覆宗國;漁鹽杞梓之利,傍贍荒遐。士民殷富,五陵既有慚德;宮宇宏麗,三川莫之能比。閭閈有盈,歌吹無絕。朱輪疊轍,華冕重肩。豈徒世無
 窮人,民獲休息,朝呼韓、罷酤鐵而已哉!是以嘉祥累仍,福應尤盛:青丘之狐,丹穴之鳥,棲阿閣,遊禁園。金芝九莖,木禾六刃,秀銅池,發膏畝。宜以協調律呂,謁薦郊廟,煙霏霧集,不可勝紀。然而聖上猶昧旦夙興,若有望而未至,閎規遠圖,如有追而莫及,神明之貺,推而弗居也。是以琬碑鏐檢,盛典蕪而不治;朝神省方,大化抑而未許。崇文協律之士,蘊儛頌於外;坐朝陪宴之臣,懷揄揚於內,三靈佇眷,九壤注心,既有日矣。



 歲宮乾維,月躔蒼
 陸,長河巨濟,異源同清,澄波萬壑,潔瀾千里。斯誠曠世偉觀,昭啟皇明者也。語曰:「影從表,瑞從德。」此其效焉。宣尼稱「鳳鳥不至,河不出圖。」《傳》曰:「俟河之清,人壽幾何!」皆傷不可見也。然則古人所不見者,今殫見之矣。孟軻曰:「千載一聖,是旦暮也。」豈不大哉。夫四皇六帝,樹聲長世,大寶也。澤浸群生,國富刑清,鴻德也。制禮裁樂,惇風遷俗,文教也。



 誅華逋羯,束顙絳闕,武功也。鳴鳥躍魚,滌穢河渠,至祥也。大寶鴻德,文教武功,其崇如此;幽明協贊,
 民祇與能,厥應如彼。唯天為大,堯實則之;皇哉唐哉,疇與為讓。抑又聞之,勢之所覃者淺,則美之所傳者近;道之所感者深,則慶之所流者遠。是以豐功韙命,潤色縢策,盛德形容,藻被歌頌。察之上代,則奚斯、吉甫之徒,鳴玉鑾於前;視之中古,則相如、王褒之屬,施金羈於後。絕景揚光,清埃繼路,班固稱漢成之世,奏御者千有餘篇,文章之盛,與三代同風。由是言之,斯乃臣子舊職,國家通義,不可輟也。臣雖不敏,寧不勉乎。



 世祖以照為中書
 舍人。上好為文章,自謂物莫能及,照悟其旨,為文多鄙言累句,當時咸謂照才盡,實不然也。臨海王子頊為荊州,照為前軍參軍,掌書記之任。



 子頊敗,為亂兵所殺。



 義慶在廣陵,有疾,而白虹貫城,野麇入府,心甚惡之,固陳求還。太祖許解州,以本號還朝。二十一年,薨於京邑,時年四十二。追贈侍中、司空,謚曰康王。



 子哀王燁字景舒嗣,官至通直郎,為元凶所殺。追贈散騎常侍。子綽,字子流嗣,官至步兵校尉。昇明三年反,伏誅,國除。綽弟綰,早
 卒。燁弟衍,太子舍人。



 衍弟鏡,宣城太守。鏡弟穎,前將軍。穎弟倩,南新蔡太守。



 遵考,高祖族弟也。曾祖淳,皇曾祖武原令混之弟,官至正員郎。祖巖,海西令。父涓子,彭城內史。



 遵考始為將軍振武參軍,預討盧循,封鄉侯。自建威將軍、彭城內史隨高祖北伐。時高祖諸子並弱,宗室唯有遵考。長安平定,以督并州司州之北河東北平陽北雍州之新平安定五郡諸軍事、輔國將軍、并州刺史,領河東太守,鎮蒲阪。
 關中失守,南還,除游擊將軍,遷冠軍將軍。晉帝遜位居秣陵宮,遵考領兵防衛。



 高祖初即大位,下推恩之詔,曰:「遵考服屬之親,國戚未遠,宗室無多,宜蒙寵爵。可封營浦縣侯,食邑五百戶。」以本號為彭城、沛二郡太守。景平元年,遷右衛將軍。元嘉二年,出為征虜將軍、淮南太守。明年,轉使持節,領護軍,入直殿省。出為使持節、督雍梁南北秦四州荊州之南陽竟陵順陽襄陽新野隨六郡諸軍事、征虜將軍、寧蠻校尉、雍州刺史,襄陽新野二郡太守。
 遵考為政嚴暴,聚斂無節。五年,為有司所糾,上不問,赦還都。七年,除太子右衛率,加給事中。明年,督南徐兗州之江北淮南諸軍事、征虜將軍、南兗州刺史,領廣陵太守。又徵為侍中,領後軍將軍,徙太常。九年,遷右衛將軍,加散騎常侍。十二年,坐厲疾不待對,免常侍,以侯領右衛。明年,復本官。十五年,又領徐州大中正、太子中庶子,本官如故。其年,監徐兗二州豫州之梁郡諸軍事、前將軍、徐兗二州刺史。未之鎮,留為侍中,領左衛將軍。明年,
 出為使持節、監豫司雍并四州南豫州之梁郡弋陽馬頭荊州之義陽四郡諸軍事、前將軍、豫州刺史,領南梁郡太守。二十一年,坐統內旱,百姓饑,詔加賑給,而遵考不奉符旨,免官。起為散騎常侍、五兵尚書,遷吳興太守,秩中二千石。二十五年,徵為領軍。二十七年,索虜南至瓜步,率軍出江上,假節蓋。三十年,復出為使持節,監豫州刺史。元凶弒立,進號安西將軍,遣外監徐安期、仰捷祖防守之。遵考斬安期等,起義兵應南譙王義宣,義宣
 加遵考鎮西將軍。夏侯獻率眾至瓜步承候世祖,又坐免官。



 孝建元年,魯爽、臧質反,起為征虜將軍,率眾屯臨沂縣,仍除吳興太守。明年,徵為湘州刺史,未行,遷尚書左僕射。三年,轉丹陽尹,加散騎常侍。復為尚書右僕射,領太子右衛率。明年,又除領軍將軍,加散騎常侍。五年,復遷尚書右僕射、金紫光祿大夫,常侍如故。明年,轉左僕射,常侍如故。又領徐州刺史、大中正、崇憲太僕。前廢帝即位,遷特進、右光祿大夫,常侍、太僕如故。景和元年,
 出督南豫州諸軍事、安西將軍、南豫州刺史。太宗即位,以為侍中、特進、右光祿大夫,領崇憲太僕,給親侍三十人。崇憲太后崩,太僕解,餘如故。泰始五年,賜几杖,大官四時賜珍味,疾病太醫給藥,固辭几杖。後廢帝即位,進左光祿大夫,餘如故。元徽元年卒,時年八十二。追贈左光祿大夫、開府儀同三司,侍中如故。



 謚曰元公。遵考無才能,直以宗室不遠,故歷朝顯遇。年老有疾失明。



 子澄之,順帝昇明末貴達。澄之弟琨之,為竟陵王誕司空主
 簿。誕作亂,以為中兵參軍,不就,縶繫數十日,終不受,乃殺之。追贈黃門郎。詔吏部尚書謝莊為之誄。



 遵考從弟思考,亦被遇。歷朝官,極清顯,為豫章、會稽太守,益、徐州刺史,凡經十郡三州。泰始元年,卒於散騎常侍、金紫光祿大夫,時年七十五。追贈特進,常侍,光祿如故。



 史臣曰:餘妖內侮,偏眾西臨,荀、桓交逼,荊楚之勢危矣。必使上略未盡,一算或遺,則城壞壓境,上流之難方結。敵資三分有二之形,北向而爭天下,則我全勝之道,或
 未可知。烈武王覽群才,揚盛策,一舉磔勍寇,非曰天時,抑亦人謀也。降年不永,遂不得與大業始終,惜矣
 哉!



\end{pinyinscope}