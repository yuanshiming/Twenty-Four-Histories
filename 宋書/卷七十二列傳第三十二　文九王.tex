\article{卷七十二列傳第三十二 文九王}

\begin{pinyinscope}

 文帝
 十
 九男:元皇后生劭,潘淑妃生浚,路淑媛生孝武帝,吳淑儀生南平王鑠,高脩儀生廬陵昭王紹,殷脩華生竟陵王誕,曹婕妤生建平宣簡王宏,陳脩容生東海
 王禕,謝容華生晉熙王昶,江脩儀生武昌王渾,沈婕妤生明帝,楊脩儀生建安王休仁,邢美人生晉平王休祐,蔡美人生海陵王休茂,董美人生鄱陽哀王休業,顏美人生臨慶沖王休倩,陳美人生新野懷王夷父,荀美人生桂陽王休範,羅美人生巴陵哀王休若。劭、浚、誕、禕、渾、休茂、休範別有傳。紹出繼廬陵孝獻王義真。



 南平穆王鑠,字休玄,文帝第四子也。元嘉十七年,都督湘州諸軍事、冠軍將軍、湘州刺史,不之鎮,領石頭戍事。
 二十二年,遷使持節、都督南豫、豫、司、雍、秦、並六州諸軍事、南豫州刺史。時太祖方事外略,乃罷南豫並壽陽,即以鑠為豫州刺史,尋領安蠻校尉,給鼓吹一部。二十六年,進號平西將軍,讓不拜。



 索虜大帥托跋燾南侵陳、潁,遂圍汝南懸瓠城。行汝南太守陳憲保城自固,賊晝夜攻圍之,憲且守且戰,矢石無時不交。虜多作高樓,施弩以射城內,飛矢雨下,城中負戶以汲。又毀佛浮圖,取金像以為大鉤,施之衝車端,以牽樓堞。城內有一沙門,頗
 有機思,輒設奇以應之。賊多作蝦蟆車以填塹,肉薄攻城。憲督厲將士,固女墻而戰。賊之死者,屍與城等,遂登屍以陵城,短兵相接;憲銳氣愈奮,戰士無不一當百,殺傷萬計,汝水為之不流。相拒四十餘日,鑠遣安蠻司馬劉康祖與寧朔將軍臧質救之,虜燒攻具走。



 二十七年,大舉北伐,諸蕃並出師。鑠遣中兵參軍胡盛之出汝南、上蔡,向長社,長社戍主魯爽委城奔走。即克長社,遣幢主王陽兒、張略等進據小索。偽豫州刺史僕蘭於大索
 率步騎二千攻陽兒,陽兒擊大破之。到坦之等進向大索,勞楊氏鄭德玄、張和各起義以應坦之,僕蘭奔虎牢。會王陽兒等至,即據大索,因向虎牢,鑠又遣安蠻司馬劉康祖繼坦之。虜永昌王宜勤仁庫真救虎牢,坦之敗走。虜乘勝徑進,於尉氏津逢康祖,康祖戰敗見殺。賊進脅壽陽,因東過與燾會於江上。



 二十八年夏,虜荊州刺史魯爽及弟秀等,率部曲詣鑠歸順。其年七月,鑠所生吳淑儀薨,鑠歸京師,葬畢,還攝本任。時江夏王義恭領
 南兗州刺史,鎮盱眙。丁母憂,還京師。上以兗土彫荒,罷南兗併南徐州,當別置淮南都督住盱眙,開創屯田,應接遠近,欲以授鑠。既而改授散騎常侍、撫軍將軍,領兵戍石頭。



 元凶弒立,以為中軍將軍,護軍、常侍如故。世祖入討,劭屯兵京邑,使鑠巡行撫勞。劭還立南兗,以鑠為使持節、都督南兗、徐、兗、青、冀、幽六州諸軍事、征北將軍、開府儀同三司、南兗州刺史,常侍如故。柳元景至新亭,劭親自攻之,挾鑠自隨。江夏王義恭南奔,使鑠守東府,以
 腹心防之。進授侍中、驃騎將軍、錄尚書事,餘如故。劭迎蔣侯神於宮內,疏世祖年諱,厭祝祈請,假授位號,使鑠造策文。及義軍入宮,鑠與濬俱歸世祖,濬即伏法,上迎鑠入營。當時倉卒失國璽,事寧,更鑄給之。進侍中、司空,領兵置佐,以國哀未闋,讓侍中。



 鑠素不推事世祖,又為元凶所任,上乃以藥內食中毒殺之,時年二十三,追贈侍中、司徒。三子:敬猷、敬淵、敬先。敬猷嗣,官至黃門郎。敬淵初封南安縣侯,官至後軍將軍。敬先繼廬陵王紹。前
 廢帝景和末,召鑠妃江氏入宮,使左右於前逼迫之,江氏不受命。謂曰:「若不從,當殺汝三子。」江氏猶不肯。於是遣使於第殺敬猷、敬淵、敬先,鞭江氏一百。其夕廢帝亦殞。太宗即位,追贈敬猷侍中,謚曰懷王。追贈敬淵黃門侍郎,謚曰悼侯。改封孝武帝第十八子臨賀王子產字孝仁為南平王,繼鑠後,未拜,被殺。泰始五年,立晉平王休祐第七子宣曜為南平王繼鑠。



 休祐死,宣曜被廢還本。後廢帝元徽元年,立衡陽恭王嶷第二子伯玉為南
 平王繼鑠,後官至給事中。升明二年,謀反誅,國除。



 建平宣簡王宏,字休度,文帝第七子也。早喪母。元嘉二十一年,年十一,封建平王,食邑二千戶。少而閑素,篤好文籍。太祖寵愛殊常,為立第於雞籠山,盡山水之美。建平國職,高他國一階。二十四年,為中護軍,領石頭戍事。出為征虜將軍、江州刺史。二十八年,徵為中書令,領驍騎將軍。元凶弒立,以宏為左將軍、丹陽尹。又以為散騎常侍、鎮軍將軍、江州刺史。世祖入討,劭錄宏殿內。世祖
 先嘗以一手板與宏,宏遣左右親信周法道齎手板詣世祖。事平,以為尚書左僕射,使奉迎太后,還加中軍將軍,中書監,僕射如故。臧質為逆,宏以仗士五十人入六門。



 為人謙儉周慎,禮賢接士,明曉政事,上甚信仗之。時普責百官讜言,宏議曰:臣聞建國之道咸殊,興王之政不一。至於開諫致寧,防口取禍,固前王同軌,後主共則。秦、殷之敗,語戮刺亡;周、漢之盛,謗升箴顯。陛下以至德神臨,垂精思治,進儒禮而崇寬教,哀獄法而黜嚴刑,表
 忠行而舉貞節,辟處士而求賢異,修廢官而出滯賞,撤天膳而重農食,禁貴遊而弛榷酤,通山澤而易關梁,固已海內仰道,天下知德。今復開不諱之塗,獎直辭之路,四海希風,普天幸甚。舉蒙採問,敢不悉心,謹條鄙見,置陳如左。辭理違謬,伏用震讋。



 夫用兵之道,自古所慎。頃干戈未戢,戰備宜修,而卒不素練,兵非夙習。且戎衛之職,多非其才,或以資厚素加,或以祿薄帶帖,或寵由權門,恩自私假,既無將領,虛尸榮祿。至於邊城舉燧,羽驛
 交馳,而望其擐甲推鋒,立功閫外,譬緣木求魚,不可得矣。常謂臨難命師,皆出倉卒,驅烏合之眾,隸造次之主,貌疏情乖,有若胡、越,豈能使其同力,拔危濟難!故奔北相望,覆敗繼有。



 今欲改選將校,皆得其人,分臺見將,各以配給,領、護二軍,為其總統。令撫養士卒,使恩信先加,農隙校獵,以習其事,三令五申,以齊其心,使動止應規,進退中律,然後畜銳觀釁,因時而動,摧敵陷堅,折衝于外。孫子曰:「視卒如赤子,故可與之共死。」所以張弮效爭
 先之心,吮癰致必盡之命,豈不由恩著者士輕其生,令明者卒畢其力。考心跡事,如或有在,妄陳膚知,追懼乖謬。



 轉尚書令,加散騎常侍,將軍如故;給鼓吹一部,尋進號衛將軍,中書監、尚書令如故。



 宏少而多病,大明二年疾動,求解尚書令,以本號開府儀同三司,加散騎常侍,中書監如故。未拜,其年薨,時年二十五。追贈侍中、司徒,中書監如故,給班劍二十人。上痛悼甚至,每朔望輒出臨靈,自為墓志銘并序。與東揚州刺史顏竣詔曰:「宏夙
 情業尚,素心令績,雖年未及壯,願言兼申。謂天道可倚,輔仁無妄,雖寢患淹時,慮不至禍。豈圖祐善虛設,一旦永謝,驚惋摧慟,五內交殞。平生未遠,舉目如昨,而賞對遊娛,緬同千載,哀酷纏綿,實增痛切。卿情均休戚,重以周旋,乖拆少時,奄成今古,聞問傷惋,當何可言。」五年,益諸弟國各千戶,先薨者不在其例,唯宏追益。



 子景素,少愛文義,有父風。大明四年,為寧朔將軍、南濟陰太守,徙歷陽、南譙二郡太守,將軍如故。中書侍郎,不拜。監南豫、
 豫二州諸軍事、輔國將軍、南豫州刺史,又不拜。太宗初,太子中庶子,領步兵校尉,太子左衛率,加給事中,冠軍將軍、南兗州刺史,丹陽尹,吳興太守,使持節、監湘州諸軍事、湘州刺史,將軍並如故。進號左將軍。泰始六年,都督荊、湘、雍、益、梁、寧、南北秦八州諸軍事、左將軍、荊州刺史,持節如故。徵為散騎常侍、後將軍、太常,未拜。授使持節、都督南徐、南兗、兗、徐、青、冀六州諸軍事、鎮軍將軍、南徐州刺史。



 桂陽王休範為逆,景素雖纂集兵眾,以赴朝
 廷為名,而陰懷兩端。及事平,進號鎮北將軍。齊王為南兗州,景素解都督。



 時太祖諸子盡殂,眾孫唯景素為長,建安王休祐諸子並廢徙,無在朝者。景素好文章書籍,招集才義之士,傾身禮接,以收名譽。由是朝野翕然,莫不屬意焉。



 而後廢帝狂凶失道,內外皆謂景素宜當神器,唯廢帝所生陳氏親戚疾忌之。而楊運長、阮佃夫並太宗舊隸,貪幼少以久其權,慮景素立,不見容於長主,深相忌憚。



 元徽三年,景素防閣將軍王季符失景素旨,
 怨恨,因單騎奔京邑,告運長、佃夫云「景素欲反」。運長等便欲遣軍討之,齊王及衛將軍袁粲以下並保持之,謂為不然也。景素亦馳遣世子延齡還都,具自申理。運長等乃徙季符於梁州,又奪景素征北將軍、開府儀同三司。



 自是廢帝狂悖日甚,朝野並屬心景素,陳氏及運長等彌相猜疑。景素因此稍為自防之計,與司馬廬江何季穆、錄事參軍陳郡殷濔、記室參軍濟陽蔡履、中兵參軍略陽垣慶延、左右賀文超等謀之。以參軍沈顒、毋丘
 文子、左暄、州西曹王潭等為爪牙。季穆薦從弟豫之為參軍。景素遣豫之、潭、文超等去來京邑,多與金帛,要結才力之士。由是冠軍將軍黃回、游擊將軍高道慶、輔國將軍曹欣之、前軍韓道清、長水校尉郭蘭之、羽林監垣祗祖,並皆響附,其餘武人失職不得志者,莫不歸之。



 時廢帝單馬獨出,遊走郊野,曹欣之謀據石頭,韓道清、郭蘭之欲說齊王使同,若不回者圖之。候廢帝出行,因眾作難,事克奉景素。景素每禁駐之,未欲匆匆舉動。運長
 密遣傖人周天賜偽投景素,勸為異計;景素知為運長所遣,即斬之,遣司馬孫謙送首還臺。



 元徽四年七月,垣祗祖率數百人奔景素,云京邑已潰亂,勸令速入。景素信之,即便舉兵,負戈至者數千人。運長等常疑景素有異志,及聞祗祖叛走,便纂嚴備辦。



 齊王出屯玄武湖,冠軍將軍任農夫、黃回、左軍將軍李安民各領步軍,右軍將軍張保率水軍,并北討。冠軍將軍、南豫州刺史段佛榮為都統,其餘眾軍相繼進。冠軍將軍齊王世子鎮東
 府城。齊王知黃回有異圖,故使安民、佛榮俱行以防之。



 景素欲斷據竹里,以拒臺軍。垣慶延、祗祖、沈顒等曰:「今天時旱熱,臺軍遠來疲困,引之使至,以逸待勞,可一戰而克也。」殷濔等固爭不能。農夫等既至,放火燒市邑,而垣慶延等各相顧望,並無鬥志。景素本乏威略,恇擾不知所為。時張保水軍泊西渚,景素左右勇士數十人,並荊楚快手,自相要結,擊水軍,應時摧陷,斬張保,而諸將不相應赴,復為臺軍所破。臺軍既薄城池,顒先眾叛走,
 垣祗祖次之,其餘諸軍相係奔敗。左暄驍果有膽力,欲為景素盡節,而所配兵力甚弱,猶力戰不退,於萬歲樓下橫射臺軍,不能禁,然後退散。右衛殿中將軍張倪奴、前軍將軍周盤龍攻陷京城,倪奴禽景素斬之,時年二十五,即葬京口。垣慶延、祗祖、左暄、賀文超並伏誅;殷彌、蔡履徙梁州;何季穆先遷官,故不及禍;其餘皆逃亡,值赦得免。



 景素即敗,曹欣之反告韓道清、郭蘭之之謀,道清等並誅。黃回、高道慶等,齊王撫之如舊。景素子延齡
 及二少子,並從誅。其年冬,封長沙成王義欣子勰第三子恬為秭歸縣侯,食邑千戶,繼宏後。順帝升明二年,卒,國除。張倪奴以禽景素功,封築陽縣侯,食邑千戶。



 景素敗後,故記室參軍王螭、故主簿何昌禹並上書訟景素之冤。齊受禪,建元初,故景素秀才劉璡又上書曰:臣聞曾子孝於其親而沈乎水,介生忠於其主而焚於火,何則?仁也不必可依,信也不必可恃。昔者墨翟議雲梯於荊臺之下,宋人逐之;夷叔為衛軍隱難於晉,公子殪之;
 李牧北逝強胡之旗,南拒全秦之卒;趙王不圖其功,賜以利劍;陳蕃白首固義,忘生事主;漢靈不明其忠,卒被刑戮。彼數子者,皆身栖青雲之上,而困於泥塵之裏,誠以危行不容於衰世,孤立聚尤於眾人,加讒諂蛆蠱其中,謗隙蜂飛而至故也。臣聞浸潤之行,骨肉離絕,疑似一至,君臣易心,此中山所以歔欷奏樂,孟博所以慷慨囊頭者也。臣每惟故舉將宋建平王之禍,悲徹骨髓,氣凝霜霰。今璇鼎啟運,人神改物,生罪尚宥,死冤必申。臣
 誠不忍王之負謗而不雪,故敢明言其理。



 臣聞孝悌為志者,不以犯上,曾子不逆薪而爨,知其不為暴也;秦仁獲麑,知其可為傅也。臣聞王之事獻太妃也,朝夕不違養,甘苦不見色。帳下進珍饌,太妃未食,王投箸輟飯。太妃起居有不安,王傍行蓬髮。臣聞求忠臣者於孝子之門,安有孝如王而不忠者乎?其可明一也。



 當泰始、元徽中,王公貴人無謁景寧陵者,王獨抗情而行,不以趨時舍義,出鎮入朝,必俯拜陵所。王尚不棄先君,豈背今君
 乎?其可明二也。



 王博聞而容眾,與諫而愛士,與人言呴呴若有傷。聞人之善,譽而進之;見人之惡,掩而誨之。李蔚之,蓬廬之寒素也,王枉駕而訊之;何季穆等,宣簡王之舊也,王提挈以升之。王虛己以厚天下之士,尚不欲傷一人之心,何乃親戚圖相菹膾乎?其可明三也。



 臣昔以法曹參軍,奉訊於聽朝之末。王每斷獄,降聲辭,和顏色,以待士女之訟。時見夏伯以童子縲縶,王愴然改貌,用不加刑。徐州嘗歲饑,王散秩粟俸帛,以斷民之乏。蠲
 理冤疑,咸息徭務,所在皆有愛於民。臣聞善人,國之紀也。安有仁於民庶,而虐其宗國者乎?其可明四也。



 王修身潔行,言無近雜,內去聲酌之娛,外無田弋之好。每所臨踐,不加穿築,直衛不繁,第宅無改。荊州高齋,刻楹柏構,王廢而不處。昔朝廷欲賜王東陵甲第,又辭而不當。兩宮所遺珍玩,塵於笥篋。無它嬖私,不耽內寵,姬嬙數人,皆詔令所賜。王身食不踰一肉,器用瓦素,時有獻鏤玉器,王顧謂何昌宇曰:「我持此安所用哉?」乃謝而反之。
 王恭己蹈義若此。其可明五也。



 王之在荊州也,時獻太妃初薨,宋明帝新棄天下,京畿諸王又相繼非命,王乃徵入為太常,楚下人士並勸勿下,王謂:「為臣而距先皇之命,不忠;為子不奉親之窀穸,不孝。」於是棄西州之重,而匍伏北闕。王若志欲倔彊,便應高枕江漢,何為屈折而受制於人乎?其可明六也。



 王名高海內,義重太山,耆幼懷仁,士庶慕德。故從昏者忌明,同枉者毀正,搦弦為鉤,張一作百,行坐欬嚏,皆生風塵。會王季符負罪流謗,
 事會讒人之心,權醜相扇,鴟梟奮翼。王雖遘愍離凶,而誠分彌款,散情中孚,揮斥滿素。虞玩之銜使歸旋,世子入質京邑,續解徐州,請身東第,後求會稽,降階外撫。虞玩、殷煥實為詮譯,誠心殷勤,備留聖聽。王若侜張跋扈,何事若斯?其可明七也。



 自是以後,日同殊論,蒼梧之衰德既彰,群小之姦慝彌廣,下盈其毒,上不可依。時長王並見誅鋤,公卿如蹈虎尾,眾人翕翕,莫不注仰於王。廂閣諸人,同謀異志,王心不從利,忠不背本,執周天賜而
 斬之,以距王宜與等,遣司馬孫謙歸款朝廷。王若欲擬非覬,寧當如此乎?其可明八也。



 又是年五月以後,道路皆謂阮佃夫等欲潛圖宮禁,因兵北襲,而黃回、高道慶等傳構其事,武人獎亂,更相恐脅。至六月而京師征賦車徒,將講眾北壘,都鄙疑駭,僉言釁作。垣祗祖因民情囂蕩,揚聲北奔,紿辭惑眾,窮亂極禍。會州人自都還,說:「掖門已閉,殊不知臺中安不?」王既素籍異論,謂為信然,收率疲弱,志在投散,冰炭在懷,但恐遲後。何圖兵以順
 出,翻為逆動乎?夫往來之人,喧嘩幻惑,皆出輦轂,非從徐州起也。且臺以六月晦夜無何呼北兵已至,皆登陴抽刃,而硃方七月朔猶緩帶從容,其晚聞京都變亂,始乃鳩兵簡甲耳。王豈先造禍哉!其可明九也。



 王聞京室有難,坐不安,食不甘,言及太后,未嘗不交巾掩泣。又臨危之際,撫楹而歎曰:「吾恐三才於斯絕矣。」茲豈不誠在本朝,以天下為憂乎?自非深忠遠概,孰能身滅之不恤,獨眷眷國家安危哉?其可明十也。



 夫王起兵之日,止在
 匡救昏難,放殛姦盜,非它故也。請較言之。當時君臣之道,治亂云何?楊運長、阮佃夫為有罪邪?為無罪邪?若其無罪,何故為戮?若其有罪,討之何辜?王豈不知君親之無將乎?顧以救火之家,豈遑先白丈人,非不恭也,徒以運屬陵喪,智力無所用之,蹉跌傾覆,此乃時也,豈謂反乎?果然今日王亡,明日宋亡,王何負於社稷,何愧于天下哉!



 臣聞武王克商,未及下車,而封王子之墓;漢高定天下,過大梁,躡燕、代,脩信陵之祀,存望諸之裔;晉世受
 命,亦追王凌之冤,而詔其孫為郎。夫比干,殷辛之罪人也;無忌,魏之疑臣也;樂毅,燕之逃將也;彥雲,齊之賊而晉害也。適逢聖明之君,革運創制,昭功誠,蕩嫌怨,清議以天下之善也。或殊世而相明,故四賢咸濟其令問,三后馳光於萬葉,君子榮其輝,小人服其義。



 今陛下尊英雄之高軌,振逸世之奇聲,何至仍衰世之異議,以掩賢人之名哉!



 若王之中外不明,終始慆德,臣懼方今之人,不復為善矣。且世之興衰,何代無有,今齊苗裔萬世之
 後,其能無污隆乎?茍前良可廢,何以勸後之能者。伏願上同周、漢、西晉之如彼,下為來胤垂範之如此。儻能降明詔,箋枉道,使往王得洗謗議,拯冥魂,賜以王禮反葬,則民之從義,猶若回風之卷草也。臣聞鸛鳴皋垤,則降陰吐雨;騰蛇聳躍,而沈雲鬱冥。但傷臣言輕落毛,身如橫芥,神高聽邈,終焉莫省,直欲內不負心,庶將來知王之意耳。



 又不省。至今上即位,乃下詔曰:「宋建平王劉景素,名父之子,少敦清尚。



 雖末路失圖,而原心有本。年流
 運改,宜弘優澤,可聽以王禮還葬舊墓。」



 晉熙王昶,字休道,文帝第九子也。元嘉二十二年,年十歲,封義陽王,食邑二千戶。二十七年,為輔國將軍、南彭城、下邳二郡太守。元凶弒立,加散騎常侍。



 世祖踐祚,遷太常,出為東中郎將、會稽太守,尋監會稽、東陽、臨海、永嘉、新安五郡諸軍事。孝建元年,立東揚州,拜昶為刺史,東中郎將如故,進號後將軍。



 大明元年,徵為秘書監,領驍騎將軍,加散騎常侍,遷中軍將軍、南彭城、下邳二郡
 太守。又出為都督江州、郢州之西陽、豫州之新蔡、晉熙三郡諸軍事、前將軍、江州刺史。三年,徵為護軍將軍,給鼓吹一部,增邑千戶。轉中書令,中軍將軍,尋以本號開府儀同三司,加散騎常侍,太常。從世祖南巡,坐斥皇太后龍舟,免開府,尋又以加授。前廢帝即位,出為使持節、都督徐、兗、南兗、青、冀、幽六州、豫州之梁郡諸軍事、征北將軍、徐州刺史,加散騎常侍,開府如故。



 昶輕吵褊急,不能祗事世祖,大明中常被嫌責;民間喧然,常云昶當有
 異志。



 永光、景和中,此聲轉甚。廢帝既誅群公,彌縱狂悖,常語左右曰:「我即大位來,遂未嘗戒嚴,使人邑邑。」江夏王義恭誅後,昶表入朝,遣典簽蘧法生銜使。帝謂法生曰:「義陽與太宰謀反,我正欲討之,今知求還,甚善。」又屢詰問法生:「義陽謀反,何故不啟?」法生懼禍,叛走還彭城。帝因此北討,親率眾過江。法生既至,昶即聚眾起兵。統內諸郡,並不受命,斬昶使。將佐文武,悉懷異心。昶知其不捷,乃夜與數十騎開門北奔索虜,棄母妻,唯攜愛妾一
 人,作丈夫服,亦騎馬自隨。昶家還都,二妾各生一子。時太宗已即位,名長者曰思遠,小者曰懷遠,尋並卒。追封懷遠為池陽縣侯,食邑千戶。



 泰始六年,以第六皇子燮字仲綏繼昶,改昶封為晉熙王。燮襲爵,食邑三千戶。



 太宗既以燮繼昶,乃下詔曰:「夫虎狼護子,猴猨負孫,毒性薄情,亦有仁愛,故識念氣類,尚均群品,況在人倫,可忘天屬。晉熙太妃謝氏,沈刻無親,物理罕比,征北公雖孝道無替,而遭此不慈,自少及長,闕恩鞠之囗,乃至休否
 莫關,寒溫不訪,晨昏屏塞,定省靡因。事無違忤,動致誚責,毒句發口,人所難聞,加惡備苦,過於仇隙,遂事憤於宗姻,義傷於行路。公故妃郗氏,婦禮無違,逢此嚴酷,遂以憂卒,用夭盛年。又謝氏食則豐珍,衣則文麗,奉己之餘,播覃群下;而諸孫纊不溫體,食不充饑,付於姆妳之手,縱以任軍之路。遇其所生,棄若糞土,繿縷比於重囚,窮困過於下使。誠皇規方遠,沙塞將一,公修短不諱,亦難豫圖。兼妾女累弱,一第領主,防閑之道,人理斯急。朕
 所以詔第六子燮奉公為胤,欲以毗整一門,為公繼紹。但謝氏待骨肉至親,尚相棄蔑,況以義合,免苦為難。患萌防漸,危機須斷,便可還其本家,削絕蕃秩。」先是,改謝氏為射氏。



 時主幼時艱,宗室寡弱。元徽元年,燮年四歲,以為使持節、監郢州、豫州之西陽、司州之義陽二郡諸軍事、征虜將軍、郢州刺史,以黃門郎王奐為長史,總府州之任。明年,太尉、江州刺史桂陽王休範舉兵逼朝廷,燮遣中兵參軍馮景祖襲尋陽,休範留中兵參軍毛惠
 連、州別駕程罕之居守,開門詣景祖降。進燮號安西將軍,加督江州諸軍事,復昶所生謝氏為晉熙國太妃。四年,又進燮鎮西將軍,加鼓吹一部。



 順帝即位,徵為使持節、都督揚、南徐二州諸軍事、撫軍將軍、揚州刺史。先是,齊世子為燮安西長史,行府州事,時亦被徵為左衛將軍,與燮俱下。會荊州刺史沈攸之舉兵反,世子因奉燮鎮尋陽之盆城,據中流,為內外形援。攸之平,燮還京邑。齊王為南徐州,燮解督南徐,進督南豫、江州諸軍事,進
 號中軍將軍、開府儀同三司,遷司徒。齊受禪,解司徒,降封陰安縣侯,食邑千五百戶。謀反,賜死。



 始安王休仁,文帝第十二子也。元嘉二十九年,年十歲,立為建安王,食邑二千戶。孝建三年,為秘書監,領步兵校尉。尋都督南兗、徐二州諸軍事、冠軍將軍、南兗州刺史。大明元年,入為侍中,領右軍將軍。四年,出為湘州刺史,加散騎常侍,加號平南將軍。八年,遷使持節、督江州、南豫州之晉熙、新蔡、郢州之西陽三郡諸軍事、安南將
 軍、江州刺史。未拜,徙為散騎常侍、太常,又不拜。仍為護軍將軍,常侍如故。



 前廢帝永光元年,遷領軍將軍。常侍如故。景和元年,又遷使持節、都督雍、梁、南北秦四州諸軍事、安西將軍、寧蠻校尉、雍州刺史,未之任,留為散騎常侍、護軍將軍,又加特進、左光祿大夫,給鼓吹一部。



 時廢帝狂悖無道,誅害群公,忌憚諸父,並囚之殿內,毆捶凌曳,無復人理。



 休仁及太宗、山陽王休祐,形體並肥壯,帝乃以竹籠盛而稱之,以太宗尤肥,號為「豬王」,號休仁為「
 殺王」,休祐為「賊王」。以三王年長,尤所畏憚,故常錄以自近,不離左右。東海王禕凡劣,號為「驢王」,桂陽王休範、巴陵王休若年少,故並得從容。嘗以木槽盛飯,內諸雜食,攪令和合,掘地為坑阱,實之以泥水,裸太宗內坑中,和槽食置前,令太宗以口就槽中食,用之為歡笑。欲害太宗及休仁、休祐前後以十數,休仁多計數,每以笑調佞諛悅之,故得推遷。常於休仁前使左右淫逼休仁所生楊太妃,左右並不得已順命,以至右衛將軍劉道隆,道
 隆歡以奉旨,盡諸醜狀。時廷尉劉矇妾孕,臨月,迎入後宮,冀其生男,欲立為太子。太宗嘗忤旨,帝怒,乃裸之,縛其手腳,以杖貫手腳內,使人擔付太官,曰:「即日屠豬。」



 休仁笑謂帝曰:「豬今日未應死。」帝問其故,休仁曰:「待皇太子生,殺豬取其肝肺。」帝意乃解,曰:「且付廷尉。」一宿出之。



 帝將南遊荊、湘二州,明旦欲殺諸父便發。其夕,太宗克定禍難,殞帝於華林園。休仁即日推崇太宗,便執臣禮。明旦,休仁出住東府。時南平,廬陵敬猷兄弟,為廢帝所
 害,猶未殯殮,休仁、休祐同載臨之,開帷歡笑,奏鼓吹往反,時人咸非焉。



 先是,廢帝進休仁為驃騎大將軍、開府儀同三司,常侍如故。未拜,太宗令書以為使持節、侍中、都督揚、南徐二州諸軍事、司徒、尚書令、揚州刺史,加班劍二十人,給三望十五乘。時劉道隆為護軍,休仁請求解職,曰:「臣不得與此人同朝。」上乃賜道隆死。尋諸方逆命,休仁都督征討諸軍事,增班劍三十人。出據虎檻,進據赭圻。尋領太子太傅,總統諸軍,隨宜應接。中流平定,
 休仁之力也。初行,與蘇侯神結為兄弟,以求神助。及事平,太宗與休仁書曰:「此段殊得蘇侯兄弟力。」增休仁邑四千戶,固辭,乃受千戶。上流雖平,薛安都據彭城,招引索虜,復都督北討諸軍事,又增邑三千戶,不受。時豫州刺史殷琰據壽陽,未平。晉平王休祐先督征討諸軍事,休祐出領江陵,休仁代督西討諸軍事。泰始五年,進都督豫、司二州。



 休仁年與太宗鄰亞,俱好文籍,素相愛友。及廢帝世,同經危難,太宗又資其權譎之力。泰始初,四
 方逆命,兵至近畿,休仁親當矢石,大勳克建,任總百揆,親寄甚隆。朝野四方,莫不輻輳。上漸不悅。休仁悟其旨,其冬,表解揚州,見許。



 六年,進位太尉,領司徒,固讓,又加漆輪車、劍履。



 太宗末年,多忌諱,猜害稍甚,休仁轉不自安。及殺晉平王休祐,憂懼彌切。



 其年,上疾篤,與楊運長等為身後之計,慮諸弟彊盛,太子幼弱,將來不安。運長又慮帝宴駕後,休仁一旦居周公之地,其輩不得秉權,彌贊成之。上疾嘗暴甚,內外莫不屬意於休仁,主書以
 下,皆往東府休仁所親信,豫自結納,其或直不得出者,皆恐懼。上既宿懷此意,至是又聞物情向之,乃召休仁入見。既而又謂曰:「夕可停尚書下省宿,明可早來。」其夜,遣人齎藥賜休仁死,時年三十九。



 上寢疾久,內外隔絕,慮人情有同異,自力乘輿出端門。休仁死後,乃詔曰:「夫無將之誅,諒惟通典,知咎自引,實有偏介。劉休仁地屬密親,位居台重,朕友寄特深,寵秩兼茂。不能弘贊國猷,裨宣政道,而自處相任,妄生猜嫌,側納群小之說,內懷
 不逞之志,晦景蔽跡,無事陽愚。因近疾患沉篤,內外憂悚,休仁規逼禁兵,謀為亂逆。朕曲推天倫,未忍明法,申詔誥礪,辨核事原。休仁慚恩懼罪,遽自引決。追尋悲痛,情不自勝,思屈法科,以申矜悼。可宥其二子,并全封爵。



 但家國多虞,釁起台輔,永尋既往,感慨追深。」



 有司奏曰:「臣聞明罰無親,情屈於司綱,國典有經,威申於義滅。是以梁、趙之誅,跣出稱過,來言之罰,克入致動。謹案劉休仁苞蓄禍跡,事蔽於天明,竄匿沉姦,情宣於民聽。自以
 屬居戚近,早延恩睦,異禮殊義,望越常均。往歲授鉞南討,本非才命,啟行濃湖,特以親攝,仰遵廟略,俯藉眾效,屬承泰運,竊附成勳,而亟叨天功,多自臧伐。既聖明御宇,躬覽萬機,百司有紀,官方無越,而休仁矜勳怙貴,自謂應總朝權,遂妄生疑難,深自猜外。故司空晉平刺王休祐,少無令業,長滋貪暴,蒞任陜荊,毒流西夏,編戶嗟散,列邑雕虛,聖澤含弘,未明正憲。亟與休仁論其愆迹,辭意既密,不宜傳廣,遂飾容旨,反相勸激。休祐以休仁
 位居朝右,任遇優崇,必能為己力援,故深相黨結。休祐於是輸金薦寶,承顏接意,造膝之間,必論朝政,遂無日不俱行,無時不同宿,聲酣聚集,密語清閑。休仁含姦扇惑,善於計數,說休祐使外託專慎之法,密行貪詐之心,謂朝廷不覺,人莫之悟。休祐遂乃外積怨懼,內協禍心,既得贊激,凶慝轉熾,與休仁共為姦謀,潛伺機隙,圖造釁變,規肆凶狡。休仁致殞倉卒,實維天誅,而晉平國太妃妾邢不能追慚子惡,上感曲恩,更懷不逞,巫蠱咒詛。
 休仁因聖躬不和,猥謀姦逆,滅道反常,莫斯為甚,殛肆朝市,庶申國刑,而法網未加,自引厥命。天慈矜厚,減法崇恩,賜全二息,及其爵封,斯誠弘風曠德,貫絕通古,然非所以棄惡流釁,懲懼亂臣者也。臣等參議,謂宜追降休仁為庶人,絕其屬籍,見息悉徙遠郡。休祐愆謀始露,亦宜裁黜,徙削之科,一同舊準。收邢付獄,依法窮治。」詔曰:「邢匹婦狂愚,不足與計。休仁知釁自引,情有追傷,可特為降始安縣王,食邑千戶,并停伯融等流徙,聽襲封
 爵。伯猷先紹江夏國,令還本,賜爵鄉侯。」



 上既殺休仁,慮人情驚動,與諸方鎮及諸大臣詔曰:休仁致殞,卿未具悉,事之始末,今疏以相示。休祐貪恣非政,法網之所不容。



 昔漢梁孝王、淮南厲王無它釁悖,正以越漢制度耳。況休祐吞嚼聚斂,為西數州之蝗,取與鄙虐,無復人情。屢得王景文、褚淵、沈攸之等啟,陳其罪惡,轉不可容。



 吾篤兄弟之恩,不欲致之以法,且每恨大明兄弟情薄,親見休祐屯苦之時,始得寬寧,彌不忍問。所以改授徐州,
 冀其去朝廷近,必應能自悛革。及拜徐州,未及之任,便徵動萬端,暴濁愈甚,既每為民蠹,不可復全。



 休仁身粗有知解,兼為宰相;又吾與其兄弟情暱,特復異常,頗與休仁論休祐釁狀。休祐以休仁為吾所親,必應知吾意;又云休仁言對,能為損益。遂多與財賂,深相結事,乃寢必同宿,行必共車。休仁性軟,易感說,遂成繾綣,共為一家,是吾所吐密言,一時倒寫。



 吾與休仁,少小異常,唯虛心信之,初不措疑。雖爾猶慮清閑之時,非意脫有聞者。
 吾近向休祐推情,戒訓嚴切,休祐更不復致疑。休祐死後,吾將其內外左右,問以情狀,方知言語漏泄並具之由,彌日懊惋,心神萎孰。休仁又說休祐云:「汝但作佞,此法自足安。我常秉許為家,從來頗得此力。但試用,看有驗不?」休祐從之,於是大有獻奉,言多乖實,積惡既不可恕。



 自休祐殞亡之始,休仁款曲共知。休仁既無罪釁,主相本若一體,吾之推意,初無有間。休祐貪愚,為天下所疾,致殞之本,為民除患,兄弟無復多人,彌應思弔不咸,
 益相親信。休祐平生,狼抗無賴,吾慮休仁往哭,或生祟禍。且吾爾日本辦仗往哭,晚定不行。吾所以為設方便,呼入在省。而休仁得吾召入,大自驚疑,遂入辭楊太妃,顏色狀意,甚與常異。既至省,楊太妃驟遣監子去來參察。從此日生嫌懼,而吾之推情,初不疑覺。從休祐死後,吾再幸休仁第,飲啖極日,排閣入內,初無猜防,休仁坐生嫌畏。



 一日,吾春中多期射雉,每休仁清閑,多往雉場中,或敕使陪輦,及不行日,多不見之。每值宵,休仁輒語
 左右云:「我已復得今一日。」及在房內見諸妓妾,恒語:「我去不知朝夕見底,若一旦死去作鬼,亦不取汝,取汝正足亂人耳。」休祐死時,日已三晡,吾射雉,始從雉場出,休仁從騎在右,伏野中,吾遣人召之,稱云:「腹痛,不堪騎馬。」爾時諸王車皆停在朱雀門裡,日既暝,不暇遠呼車,吾衣書車近在離門裏,敕呼來,下油幢絡,擬以載之。吾由來諳悉其體有冷患,聞腹痛,知必是冷,乃敕太醫上省送供御高梁姜飲以賜之。休仁得飲,忽大驚,告左右稱:「
 敗今日了。」左右答曰:「此飲是御師名封題。」休仁乃令左右先飲竟,猶不甚信,乃僶俯噬之,裁進一合許。妄生嫌貳,事事如是。由來十日五日,一就問太妃。自休祐死後,每吾詔,必先至楊太妃問,如分別狀。休仁由來自營府國興生文書,二月中,史承祖齎文書呈之,忽語承祖云:「我得成許那,何煩將來。」



 吾虛心如舊,不復見信,既懷不安,大自嫌恐,惟以情理,不容復有善心。



 休仁既經南討,與宿衛將帥經習狎共事相識者,布滿外內。常日出入,
 於廂下經過,與諸相識將帥,都不交言。及吾前者積日失適,休仁出入殿省,諸衛主帥裁相悉者,無不和顏厚相撫勞。爾時吾既甚惡,意不欲見外人,悠悠所傳,互言差劇。



 休仁規欲聞知方便,使曇度道人及勞彥遠屢求啟,闞覘吾起居。及其所啟,皆非急事,吾意亦不厝疑。吾與休仁,親情實異,年少以來,恒相追隨,情向大趣,亦往往多同,難否之日,每共契闊。休仁南討為都統,既有勳績,狀之於心,亦何極已。



 但休仁於吾,望既不輕,小人無
 知,亦多挾背向,既生猜貳,不復自寧。夫禍難之由,皆意所不悟,如其意趣,人莫能測,事不獲已,反覆思惟,不得不有近日處分。



 夫於兄弟之情,不能無厚薄。休祐之亡,雖復悼念,猶可以理割遣;及休仁之殞,悲愍特深,千念不能已已,舉言傷心。事之細碎,既不可曲載詔文,恐物不必即解,兼欲存其兒子,不欲窮法。為詔之辭,不得不云有兵謀,非事實也。故相報卿知。



 上與休仁素厚,至於相害,慮在後嗣不安。休仁既死,痛悼甚至,謂人曰:「我與
 建安年時相鄰,少便狎從。景和、泰始之間,勳誠實重。事計交切,不得不相除。痛念之至,不能自已。今有一事不如與諸侯共說,歡適之方,於今盡矣。」



 因流涕不自勝。



 子伯融,妃殷氏所生。殷氏,吳興太守沖女也。範陽祖翻有醫術,姿貌又美,殷氏有疾,翻入視脈,說之,遂通好。事泄,遣還家賜死。伯融歷南豫州刺史,琅邪、臨淮二郡太守,寧朔將軍,廣州刺史,不之職。廢徙丹楊縣。後廢帝元徽元年,還京邑,襲封始興王。弟伯猷,初出繼江夏愍王伯
 禽,封江夏王,邑二千戶。休仁死後還本,與伯融俱徙丹楊縣。後廢帝元徽元年,賜爵都鄉侯。建平王景素為逆,楊運長等畏忌宗室,稱詔賜伯融等死。伯融時年十九,伯猷年十一。



 晉平剌王休祐,文帝第十三子也。孝建三年,年十一,封山陽王,食邑二千戶。



 大明元年,為散騎常侍,領長水校尉,尋遷東揚州刺史。未拜,徙湘州刺史,加號征虜將軍。四年,還為秘書監,領右軍將軍,增邑千戶。遷侍中,又遷
 左中郎將,都官尚書;又為秘書監,領驍騎將軍。出為使持節、都督豫、司二州、南豫州之梁郡諸軍事、右將軍、豫州刺史。景和元年,入朝,進號鎮西大將軍,仍遷散騎常侍、鎮軍大將軍、開府儀同三司。



 太宗定亂,以為使持節、都督荊、湘、雍、益、梁、寧、南北秦八州諸軍事、驃騎大將軍、荊州刺史,開府、常侍如故。又改都督江、郢、雍、湘五州、江州刺史;又改都督江南豫司州、南豫州刺史,改都督豫、江、司三州、豫州刺史。時豫州刺史殷琰據壽陽反叛,休
 祐出鎮歷陽,督劉勔等討琰,琰未平,勔築長圍守之。



 休祐復徙都督荊、湘、雍、益、梁、寧、南北秦八州諸軍事、荊州刺史,持節、常侍、將軍、開府並如故,增封二千戶,受五百戶。以山陽荒敝,改封晉平王。



 休祐素無才能,彊梁自用,大明之世,年尚少,未得自專,至是貪淫,好財色。



 在荊州,裒刻所在,多營財貨。以短錢一百賦民,田登,就求白米一斛,米粒皆令徹白,若有破折者,悉刪簡不受。民間糴此米,一升一百。至時又不受米,評米責錢。凡諸求利,皆
 悉如此,百姓嗷然,不復堪命。泰始六年,徵為都督南徐、南兗、徐、兗、青、冀六州諸軍事、南徐州刺史,加侍中,持節、將軍如故。上以休祐貪虐不可蒞民,留之京邑,遣上佐行府州事。



 休祐狠戾彊梁,前後忤上非一。在荊州時,左右苑景達善彈棋,上召之,休祐留不遣。上怒,詰責之曰:「汝剛戾如此,豈為下之義!」積不能平。且慮休祐將來難制,欲方便除之。七年二月,車駕於巖山射雉,有一雉不肯入場,日暮將反,令休祐射之。語云:「不得雉,勿歸。」休祐
 時從在黃麾內,左右從者並在部伍後,休祐便馳去,上遣左右數人隨之。上既還,前驅清道,休祐人從悉分散,不復相得,上因遣壽寂之等諸將追之。日已欲暗,與休祐相及,逼令墜馬。休祐素勇壯有氣力,奮拳左右排擊,莫得近。有一人後引陰,因頓地,即共毆拉殺之。乃遣人馳白上,行唱:「驃騎落馬。」上曰:「驃騎體大,落馬殊不易。」即遣御醫絡驛相係。頃之,休祐左右人至,久已絕。去車腳,輿以還第,時年二十七。追贈司空,持節、侍中、都督、刺史
 如故,給班劍二十人,三望車一乘。



 時巴陵王休若在江陵,其日即馳信報休若曰:「吾與驃騎南山射雉,驃騎馬驚,與直閣夏文秀馬相丱,文秀墮地,驃騎失鞚,馬驚,觸松樹墮地,落刑中,時頓悶,不識人,故馳報弟。」其年五月,追免休祐為庶人。



 長子仕薈,早卒。次子宣翊為世子,為寧朔將軍、湘州刺史,未拜,免廢。次士弘,繼鄱陽哀王休業。襲封,被廢還本。次宣彥,封原豐縣侯,為寧朔將軍、彭城太守,未拜,免廢。次宣諒。次宣曜,出繼南平穆王鑠封,
 被廢還本。次宣景,次宣梵,次宣覺,次宣受,次宣則,次宣直,次宣季,凡十三子,並徙晉平郡。太宗尋病,見休祐為祟,乃遣前中書舍人劉休至晉平撫慰宣翊等,上遂崩。後廢帝元徽元年,聽宣翊等還都。順帝升明三年,謀反,並賜死。



 鄱陽哀王休業,文帝第十五子也。孝建二年,年十一,封鄱陽王,食邑二千戶。



 三年,薨,追贈太常。大明六年,以山陽王休祐次子士弘嗣封。被廢還本,國除。



 臨慶沖王休倩,文帝第十六子也。孝建元年,年九歲,疾篤,封東平王,食邑二千戶,未拜,薨。



 大明七年,立第二十七皇子子嗣為東平王,紹休倩後。太宗泰始二年還本,國絕。六年,以第五皇子智井為東平王,繼休倩,未拜,薨。其年,追改休倩為臨慶王,以臨賀郡為臨慶國,立第八皇子躋為臨慶王,食邑二千戶,繼休倩後。明年,還本國。休倩,太祖所愛,故前後屢加紹門嗣。



 新野懷王夷父,文帝第十七子也。元嘉二十九年,薨,時
 年六歲。太宗泰始五年,追加封謚。



 巴陵哀王休若,文帝第十九子也。孝建三年,年九歲,封巴陵王,食邑二千戶。



 大明二年,為冠軍將軍、南琅邪、臨淮二郡太守,徙南彭城、下邳二郡太守,將軍如故。四年,出為都督徐州諸軍事、徐州刺史,將軍如故,增督豫州之梁郡,增邑千戶。明年,徵為散騎常侍、左右郎將、吳興太守。復徵為散騎常侍、太常。未拜,前廢帝永光元年,遷左衛將軍。



 太宗泰始元年,遷散騎常侍、中書令,領衛尉。未拜,
 復為左衛將軍,常侍、衛尉如故。又未拜,出為使持節、都督會稽、東陽、永嘉、臨海、新安五郡諸軍事、領安東將軍、會稽太守,率眾東討。進督吳、吳興、晉陵三郡。尋加散騎常侍,進號衛將軍,給鼓吹一部。又進督晉安、囗囗二郡諸軍事。二年,遷梁、雍、南北秦四州、荊州之竟陵、隨二郡諸軍事、寧蠻校尉、雍州刺史,持節、常侍、將軍如故,增邑二千戶,受三百戶。



 前在會稽,錄事參軍陳郡、謝沈以諂佞事休若,多受賄賂。時內外戒嚴,普著褲褶,沈居母喪。
 被起,聲樂酣飲,不異吉人,衣冠既無殊異,並不知沈居喪,嘗自稱孤子,眾乃駭愕。休若坐與沈褻黷,致有姦私,降號鎮西將軍。又進衛將軍。



 典簽夏寶期事休若無禮,繫獄,啟太宗殺之,慮不被許,啟未報,輒於獄行刑,信反果錮送,而寶期已死。上大怒,與休若書曰:「孝建、大明中,汝敢行此邪?」



 休若母加杖三百,降號左將軍,貶使持節都督為監,行雍州刺史,使寧蠻校尉,削封五百戶。四年,遷使持節、都督湘州諸軍事、行湘州刺史,將軍如故。六年,荊
 州刺史晉平王休祐入,以休若監荊州事,進號征南將軍、湘州刺史。仍為都督荊、湘、雍、益、梁、寧、南北秦八州諸軍事、征西將軍、荊州刺史,持節如故。尋加散騎常侍,又進號征西大將軍、開府儀同三司。



 七年,晉平王休祐被殺,建安王休仁見疑。京邑訛言休若有至貴之表,太宗以言報之,休若內甚憂懼。會被徵,代休祐為都督南徐、南兗、徐、兗、青、冀六州諸軍事、征北大將軍、南徐州刺史,持節、常侍、開府如故。休若腹心將佐咸謂還朝必有大
 禍,中兵參軍京兆王敬先固陳不宜入,勸割據荊楚以距朝廷,休若偽許之。



 敬先既出,執錄,馳使白太宗,敬先坐誅死。休若至京口,建安王休仁又見害,益懷危慮。上以休若和善,能諧緝物情,慮將來傾幼主,欲遣使殺之。慮不奉詔,徵入朝,又恐猜駭,乃偽遷休若為都督江郢、司、廣、交、豫州之西陽、新蔡、晉熙、湘州之始興四郡諸軍事、車騎大將軍、江州刺史,持節、常侍、開府如故。徵還召拜,手書殷勤,使赴七月七日,即於第賜死,時年二十四。
 贈侍中、司空,持節、都督、刺史如故,給班劍二十人,三望車一乘。



 休若既死,上與驃騎大將軍桂陽王休範書曰:外間有一師,姓徐名紹之,狀如狂病,自云為塗步郎所使。去三月中,忽云:「神語道巴陵王應作天子,汝使巴陵王密知之。」於是師便訪覓休若左右人,不能得。東宮典書姓何者相識,數去來,師解神語,東宮典書具道神語,東宮典書答云:「我識巴陵間一左右,當為汝向道。」數日,東宮典書復來語師云:「我已為汝語巴陵左右,道因達
 巴陵,巴陵具知,云莫聲但聽。」



 又頃者史官奏天文占候,頗雲休若應挾異端。神道芒昧,乃不可全信,然前後相準,略亦不無仿佛。且帖肆間,自大明以來有「若好」之謠,于今未止。詔若百重章句,皆配以美辭美事,諸不逞之徒,咸云必是休若。休若且知道路有異音,里巷有「若好」之謠,在西已奇懼,致王敬先吐猖狂之言。近休祐、休仁被誅,休若彌不自安,又左右多是不相當負罪之徒,恒說以道路之言叩動之,相與唱云:「萬民之心,屬在休若」,
 感激其意。



 尋休若從來心迹,殊有可嫌。劉亮問高次祖,汝一應識此人,當給休若。休若在東縱恣群下無本末,還朝被貶,爵位小退,次祖被亮使歸,過問訊,大泣,語次祖云:「我東行是一段功,在郡橫為群小輩過失,大被貶降,我實憤怨,不解劉輔國何意不作。」次祖答云:「劉輔國蒙朝廷生成之恩,豈容有此理。」推此已是有奇意。吾使諸王在蕃,正令優游而已,本不以武事,而休若在西,廣召弓馬健兒,都不啟聞。又戾道明等,昔親為賊,罪應萬
 死,休若至西,大信遇之,乃潛將往不啟京。吾知汝意謂休若處奉因事事何如,心跡既不復可測,因其還朝在第與書,事事詰誚於內,許密自引分,狀如暴疾致故,差得於其名位及見子悉得全也。休若既是汝弟,使其狼心得申者,汝得守冶城邊作太尉公邪?非但事關計,亦於汝甚切,汝可密白荀太妃令知。



 廬江王禕,昔在西州,故上云冶城邊也。休若子沖始襲封。順帝昇明三年,薨。



 會齊受禪,國除。



 史臣曰:《詩》云「不自我先,不自我後。」古人畏亂世也。太宗晚途,疑隙內成,尋斧所加,先自至戚。晉剌以獷暴摧軀,巴哀由和良鴆體,保身之路,未知攸適。昔之戒子,慎勿為善,將遠有以
 乎!



\end{pinyinscope}