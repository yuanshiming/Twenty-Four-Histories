\article{卷二十八志第十八 符瑞中}

\begin{pinyinscope}

 麒麟者,仁獸也。牡曰麒,牝曰
 麟。不刳胎剖卵則至。麕身而牛尾,狼項而一角,黃色而馬足。含仁而戴義,音中鐘呂,步中規矩,不踐生蟲,不折生草,不食不義,不飲洿池,不入坑阱,不行羅網。明王動靜有儀則見。牡鳴曰「逝聖」,牝鳴曰「歸和」,春鳴曰「扶幼」,夏鳴曰「養綏」。



 漢武帝元狩元年十月,行幸雍,祠五畤,獲白麟。漢武帝太始二年三月,獲白麟。漢章帝元和二年以來,至章和元年,凡三年,麒麟五十一見郡國。漢安帝延光三年七月,麒麟見潁川陽翟。延光三年八月戊子,麒麟見潁川陽翟。延光四年正月壬午,麒麟見東郡濮陽。漢獻帝延康元年,麒麟十見郡國。



 吳孫權赤烏元年八
 月,武昌言麒麟見。又白麟見建業。晉武帝泰始元年十二月,麒麟見南郡枝江。晉武帝咸寧五年二月甲午,白麟見平原鬲縣。咸寧五年九月甲午,麒麟見河南陽城。晉武帝太康元年四月,白
 麟見頓丘。晉愍帝建興二年九月丙戌,麒麟見襄平,州刺史崔毖以聞。晉元帝太興元年正月戊子,麒麟見豫章。晉成帝咸和八年五月己巳,麒麟見遼東。



 鳳凰者,仁鳥也。不刳胎剖卵則至。或翔或集。雄曰鳳,雌曰凰。蛇頭燕頷,龜背鱉腹,鶴頸雞喙,鴻前魚尾,青首駢翼,鷺立而鴛鴦思。首戴德而背負仁,項荷義而膺抱信,足履正而尾系武。小音中鐘,大音中鼓。延頸奮翼,五光備舉。興八風,降時雨,食有節,飲
 有儀,往有文,來有嘉,游必擇地,飲不妄下。其鳴,雄曰「節節」,雌曰「足足」。晨鳴曰「發明」,晝鳴曰「上朔」,夕鳴曰「歸昌」,昏鳴曰「固常」,夜鳴曰「保長」。其樂也,徘徘徊徊,雍雍喈喈。唯鳳皇為能究萬物,通天祉,象百狀,達王道,率五音,成九德,備文武,正下國。故得鳳之象,一則過之,二則翔之,三則集之,四則春秋居之,五則終身居之。



 漢昭帝始元三年十月,鳳皇集東海,遣使祠其處。漢宣帝本始元年五月,鳳皇集膠東。本始四年五月,鳳皇集北海。漢宣帝地節二年
 四月,鳳皇集魯,群鳥從之。



 漢宣帝元康元年三月,鳳皇集泰山、陳留。元康四年,南郡獲威鳳。漢宣帝神雀二年二月,鳳皇集京師,群鳥從之以萬數。神雀四年春,鳳皇集京師。神雀四年十月,鳳皇十一集杜陵。神雀四
 年十二月,鳳皇集上林。



 漢宣帝甘露三年二月,鳳皇集新蔡,群鳥四面行列,皆向鳳皇立,以萬數。



 漢光武建武十七年十月,鳳皇五,高八九尺,毛羽五採,集潁川郡,群鳥並從行列,蓋
 地
 數
 頃,留十七日乃去。



 漢章帝元和二年以來,至章和元年,凡三年,鳳皇百三十九見郡國。



 漢安帝延光三年二月,車駕東巡。其月戊子,鳳皇集濟南臺縣丞霍收舍樹上,賜臺長嶷帛十五匹,收二十匹,
 尉半之,吏卒人三匹;鳳皇所過亭部,無出今年田租;賜男子爵人二級。延光三年十月壬午,鳳皇集京兆新豐西界槐樹。漢桓帝建和元年十一月,鳳皇見濟陰己氏。漢靈帝光和四年秋,五色大鳥見新城,群鳥隨之。



 民皆謂之鳳皇。漢獻帝延康元年八月,石邑縣言鳳皇集。又郡國十三言鳳皇見。



 吳孫權黃武五年七月,蒼梧言鳳皇見。孫權黃龍元年四月,夏口、武昌並言鳳皇見。吳孫亮建興二年十一月,大鳥五見於春申。吳孫皓建衡四年正月,西苑言鳳皇集。



 晉武帝泰始元年十二月,鳳皇見上黨高都。泰始元年十二月,鳳皇二見河南山陽。泰始元年十二月,鳳皇三見馮翊下邽。晉穆帝升平四年二月辛亥,鳳皇將九子見鄖鄉之豐
 城。十二月甲子,又見豐城,眾鳥隨從。升平五年四月己未,鳳皇集沔北,至於辛酉。百姓聚觀之。



 宋武帝永初元年七月戊戌,鳳皇見會稽山陰。文帝元嘉十四年三月丙申,大鳥二集秣陵民王顗園中李樹上,大如孔雀,頭足小高,毛羽鮮明,文採五色,聲音諧從,眾鳥如山雞者隨之,如行三十步頃,東南飛去。揚州刺史彭城王義康以聞。改鳥所集永昌里曰鳳皇
 里。孝武帝孝建元年正月庚申,鳳皇見丹徒篸賢亭,雙鵠為引,眾鳥陪從。征虜將軍武昌王渾以聞。



 神鳥者,赤神之精也,知音聲清濁和調者也。雖赤色而備五采,雞身,鳴中五音,肅肅雍雍。喜則鳴舞,樂處幽隱。風俗從則至。



 漢宣帝五鳳三年三月辛丑,神鳥集長樂宮東闕樹上,又飛下地,五采炳發,留十餘刻。漢章帝元和中,神鳥見郡國。



 黃龍者,四龍之長也。不漉池而漁,德至淵泉,則黃龍游於池。能高能下,能細能大,能幽能冥,能短能長,乍存乍亡。赤龍、《河圖》者,地之符也。王者德至淵泉,則河出《龍圖》。



 漢惠帝二年正月癸酉,兩龍見蘭陵人家井中。漢文帝十五年春,黃龍見成紀。



 漢宣帝甘露元年四月,黃龍見新豐。



 漢成帝鴻嘉元年冬,黃龍見
 真定。漢成帝永始二年二月癸未,黃龍見東萊。漢光武建武十二年六月,黃龍見東阿。



 漢章帝元和二年以來,至章和元年,凡三年,黃龍四十四見郡國。元和中,青龍見郡國。元和中,白龍見郡國。



 漢安帝延光元年八月辛卯,黃龍見九真。延光三年九月辛亥,黃龍見濟南歷城。



 延光三年十二月乙未,黃龍見琅邪諸縣。延光四年正月壬午,黃龍二見東郡濮陽。



 漢桓帝建和元年二月,黃龍見沛國譙。漢桓帝元嘉二年八月,黃龍見濟陰句陽,又見金城允街。漢桓帝永光元年八月,黃龍見巴郡。漢獻帝延康元年三月,黃龍見譙。又郡國十三言黃龍見。
 魏明帝青龍元年正月甲申,青龍見郟之摩陂井。帝親與群臣共觀之,既而詔書工圖寫,龍潛而不見。魏明帝景初元年二月壬辰,山茌縣言黃龍見。



 魏少帝正元元年十月戊戌,黃龍見鄴井中。魏少帝甘露元年正月辛丑,青龍見軹縣井中凡二。甘露元年六月,青龍見元城縣界井中。甘露二年二月,青龍見溫縣井中。甘露三年八月甲戌,黃龍、青龍仍見頓丘、冠軍、陽夏縣
 井中。甘露四年正月,黃龍二見寧陵縣井中。



 魏元帝景元元年十二月甲申,黃龍見莘縣井中。景元三年二月,青龍見軹縣井中。



 劉備未即位前,黃龍見武陽赤水,九日乃去。



 吳孫權黃武元年三月,鄱陽言黃龍見。吳孫權黃龍元年四月,夏口、武昌並言黃龍見;權因此改元。作黃龍牙,常在軍中,進退視其所向,命胡綜為賦。



 吳孫權赤烏五年三月,海鹽縣言黃龍見縣井中二。赤烏十一年,雲陽言黃龍見。



 黃龍二又見武陵吳壽,光色炫耀。



 吳孫休永安四年九月,布山言白龍見。永安五年七月,始新言黃龍見。永安六年四月,泉陵言黃龍見。



 晉武帝泰始元年十二月,青龍二見濟陰定陶。泰始元年十二月,青龍見魏郡湯陰。
 泰始元年十二月,黃龍見河南洛陽洛濱。泰始元年十二月,白龍二見太原祁。



 泰始二年七月壬午,黃龍見巴西閬中。泰始三年四月戊午,有司奏:「張掖太守焦勝言,氐池縣大柳谷口青龍見。」



 晉武帝咸寧二年六月丙申,白龍二見於新興九原居民井中。咸寧二年十月庚午,黃龍二見於漢嘉靈關。
 咸寧二年十一月癸巳,白龍二見須度支部。咸寧五年十一月甲寅,青龍見京兆霸城。



 晉武帝太康元年八月,白龍三見於永昌。太康三年閏四月己丑,白龍二見濟南歷城。太康五年正月癸卯,青龍二見武庫井中,帝親往觀之。太康六年九月,白龍見京兆陰槃。太康九年十二月戊申,青龍一見魯國公丘居民井中。晉惠帝元康七年三月己酉朔,成皋縣獄有龍昇天。



 宋武帝永初元年七月,青龍見義興陽羨。永初元年八月,青龍二見南郡江陵。



 文帝元嘉十三年九月己酉,會稽郡西南向曉,忽大光明,有青龍騰躍凌雲,久而後滅。吳興諸處並以其日同見光景。揚州刺史彭城王義康以聞。元嘉二十一年十月己丑,永嘉永寧見黃龍自雲而下,太守臧藝以聞。元嘉二十五年五月丁丑,黑龍見玄武湖北,苑丞王世
 宗以聞。元嘉二十五年五月戊戌,黑龍見玄武湖東北隈,揚州野吏張立之以聞。元嘉二十五年八月辛亥,黃龍見會稽,太守孟顗以聞。元嘉二十五年,廣陵有龍自湖水中升天,百姓皆見。



 孝武帝孝建二年七月癸丑,黃龍見石頭城外水濱,中護軍湘東王彧以聞。孝建三年五月己未,龍見臨川郡,江州刺史東海王禕
 以聞。孝武大明元年五月癸亥,黑龍見晉陵占石村。改村為津里。



 靈龜者,神龜也。王者德澤湛清,漁獵山川從時則出。五色鮮明,三百歲游於蕖葉之上,三千歲常游於卷耳之上。知存亡,明於吉兇。禹卑宮室,靈龜見。玄龜書者,天符也。王者德至淵泉,則雒出龜書。



 魏文帝初,神龜出於靈池。



 吳孫權時,靈龜出會稽章安。



 魏元帝咸熙二年二月甲辰,朐䏰縣獲靈龜以獻。



 晉長沙王乂坐同產兄楚王瑋事,徙封常山,後還復國。在常山穿井,入地四丈,得白玉方三四尺。玉下有大石,其中有龜長二尺餘,時人以為復國之祥。



 宋文帝元嘉十九年四月戊申,白龜見吳興餘杭,太守文道恩以獻。元嘉二十年四月辛卯,白龜見吳興餘杭,揚州刺史始
 興王濬以聞。元嘉二十四年十月甲午,揚州刺史始興王濬獲白龜以獻。



 孝武帝大明三年三月戊子,毛龜見宣城廣德,太守張辨以獻。大明四年六月壬寅,車駕幸籍田,白龜見於千畝,尚書右僕射劉秀之以獻。大明七年八月乙未,毛龜見新安王子鸞第,獲以獻。



 明帝泰始二年八月丙辰朔,四眼龜見會稽,會稽太守巴陵王休若以獻。泰始二年八月丙寅,六眼龜見東陽長山,文如爻卦,太守劉勰以獻。泰始六年九月己巳,八眼龜見吳興故鄣,太守褚淵以獻。明帝泰豫元年十月壬戌,義興陽羨縣獲毛龜,太守王蘊以獻。


龍馬者,仁馬也,河水之精。高八尺五寸,長頸有翼,傍有垂毛,鳴聲九哀
 \gezhu{
  一作音}
 。騰黃者,神馬也,其色黃。王者德御四方則出。白馬硃鬣,王者任賢良則見。澤馬者,王者勞來百姓則至。夏馬颻,黑身白鬣尾,殷馬駱,白身黑鬣尾,周馬騂,赤身黑鬣尾。



 漢章帝元和中,神馬見郡國。



 晉懷帝永嘉六年二月壬子,神馬鳴南城門。



 晉孝武帝太元十四年六月甲申朔,寧州刺史費統上
 言:「所統晉寧之滇池縣,舊有河水,周回二百餘里。六月二十八日辛亥,神馬二匹,一白一黑,忽出於河中,去岸百步。縣民董聰見之。」



 白象者,人君自養有節則至。



 宋文帝元嘉元年十二月丙辰,白象見零陵洮陽。元嘉六年三月丁亥,白象見安成安復,江州刺史南譙王義宣以聞。



 漢武帝元狩二年三月,南越獻馴象。



 白狐,王者仁智則至。



 晉成帝咸康八年七月,燕王慕容皝上言白狢見國內。



 赤熊,佞人遠,姦猾息,則入國。



 宋文帝元嘉二十年十二月,白熊見新安歙縣,太守到元度以獻。



 九尾狐,文王得之,東夷歸焉。



 漢章帝元和中,九尾狐見郡國。



 魏文帝黃初元年十一月甲午,九尾狐見鄄城,又見譙。



 白鹿,王者明惠及下則至。



 漢章帝建初七年十月,車駕西巡,得白鹿於臨平觀。漢章帝元和中,白鹿見郡國。



 漢安帝延光三年六月辛未,白鹿見右扶風雍。延光三年七月,白鹿見左馮翊。



 漢桓帝永興元年二月,白鹿見張掖。



 魏文帝黃初元年,郡國十九言白鹿及白麋見。



 晉武帝泰始八年十月,白鹿見扶風雍,州刺史嚴詢獲
 以獻。晉武帝太康元年三月,白鹿見零陵泉陵。太康元年五月甲辰,白鹿見天水西縣,太守劉辛獲以獻。太康三年七月壬子,白鹿見零陵,零陵令蔣微獲以獻。



 晉惠帝元康元年九月乙酉,白鹿見交趾武寧。



 晉愍帝建武元年五月戊子,白鹿見高山縣。



 晉元帝太興三年正月,白鹿二見豫章。
 太興三年四月,白鹿見晉陵延陵。晉元帝永昌元年九月,白鹿見江乘縣。



 晉成帝咸和四年五月甲子,白鹿見零陵洮陽,獲以獻。咸和四年七月壬寅,長沙郡邏吏黃光於南郡道遇白鹿,驅之不去,直來就光,追尋光三百餘步。光遂抱取,遣吏李堅奉獻。咸和九年八月己未,白鹿見長沙臨湘。晉成帝咸康二年七月,白鹿見豫章望蔡,太守桓景獲
 以獻。



 晉孝武太元十六年三月癸酉,白鹿見豫章望蔡,獲以獻。太元十八年五月辛酉,白鹿見江乘,江乘令田熙之獲以獻。太元二十年九月丁丑,白鹿見巴陵清水山,荊州刺史殷仲堪以獻。



 晉安帝隆安五年十一月,白鹿見長沙,荊州刺史桓玄
 以聞。



 宋文帝元嘉五年七月丙戌,白鹿見東莞莒縣岣峨山,太守劉玄以聞。元嘉九年正月,白鹿見南譙譙縣,豫州刺史長沙王義欣以獻。元嘉十四年,白鹿見文鄉。元嘉十七年五月甲午,白鹿見南汝陰宋縣,太守文道恩以獻。
 元嘉二十年八月,白鹿見譙郡蘄縣,太守鄧琬以獻。元嘉二十二年二月,白鹿見建康縣,揚州刺史始興王濬以聞。元嘉二十二年二月辛未,白鹿見南康灨縣,南康相劉興祖以獻。元嘉二十三年二月戊戌,白鹿見交州,交州刺史檀和之以獻。元嘉二十三年六月丙辰,白鹿見彭城彭城縣,征北將
 軍衡陽王義季獲以獻。元嘉二十七年二月壬辰朔,白鹿見濟陰,徐州刺史武陵王駿以聞。元嘉二十九年八月癸酉,白鹿見鄱陽,南中郎將武陵王駿以獻。元嘉三十年十一月壬午,白鹿見南琅邪,南琅邪太守王僧虔以獻。元嘉三十年十一月癸亥,白鹿見武建郡,雍州刺史朱
 脩之以獻。



 孝武帝孝建三年三月庚子,白鹿見臨川西豐縣。孝武帝大明元年四月甲申,白鹿見南平。大明二年四月己丑,白鹿見桂陽郴縣,湘州刺史山陽王休祐以獻。大明三年正月癸巳,白鹿見南琅邪江乘,南徐州刺史劉延孫以獻。大明三年三月辛卯,白鹿見廣陵新市,太守柳光宗以
 聞。大明五年五月丙寅,白鹿見南東海丹徒,南徐州刺史劉延孫以獻。大明八年六月甲子,白鹿見衡陽郡,湘州刺史江夏王世子伯禽以獻。



 明帝泰始二年二月乙亥,白鹿見宣城,宣城太守劉韞以聞。泰始五年二月己亥,白鹿見長沙,湘州刺史劉韞以獻。
 泰始六年十二月乙未,白鹿見梁州,梁州刺史杜幼文以聞。



 後廢帝元徽三年二月甲子,白鹿見鬱洲,青冀二州刺史、西海太守劉善明以獻。


三角獸,先王法度修則至。
 \gezhu{
  闕}


一角獸,天下平一則至。
 \gezhu{
  闕}


六足獸,王者謀及眾庶則至。
 \gezhu{
  闕}


比肩獸,王者德及矜寡則至。
 \gezhu{
  闕}


獬豸知曲直,獄訟平則至。
 \gezhu{
  闕}



 白虎,王者不暴虐,則白虎仁,不害物。



 漢宣帝元康四年,南郡獲白虎。



 漢章帝元和二年以來,至章和元年,凡三年,白虎二十九見郡國。



 漢安帝延光三年八月戊子,白虎二見潁川陽翟。



 漢獻帝延康元年四月丁巳,饒安縣言白虎見。又郡國二十七言白虎見。



 吳孫權赤烏六年正月,新都言白虎見。赤烏十一年五月,鄱陽言白虎仁。



 晉武帝泰始元年十二月,白虎見河南陽翟。泰始元年十二月,白虎見弘農陸渾。



 泰始二年正月己亥,白虎見遼東樂浪。泰始二年正月辛丑,白虎見天水西。



 晉武帝咸寧三年二月乙丑,白虎見沛國。晉武帝太康元年八月,白虎見永昌南罕。
 太康四年七月丙辰,白虎見建平北井。太康十年十月丁酉,白虎見犍為。



 晉成帝咸和八年五月己已,白虎見新昌縣。晉簡文帝咸安二年三月,白虎見豫章南昌縣西鄉石馬山前。



 晉孝武太元十四年十一月辛亥,白虎見豫章郡。太元十九年二月,行鞏令劉啟期言白虎頻見。太元十九年二月,行溫令趙邳言白虎頻見。



 晉安帝隆安五年十一月,襄陽言騶虞見於新野。



 宋武帝永初元年八月癸巳,白虎見枝江。



 少帝景平元年十月,白虎見桂陽耒陽。



 文帝元嘉十九年十月,白虎見弋陽、期思二縣,南豫州刺史武陵王駿以聞。元嘉二十五年二月己亥,白虎見武昌,武昌太守蔡興宗以聞。元嘉二十五年十一月丁丑,白虎見蜀郡二,赤虎導前,
 益州刺史陸徽以聞。元嘉二十六年四月戊戌,白虎見南琅邪半陽山,二虎隨從,太守王僧達以聞。



 孝武孝建三年三月壬子,白虎見臨川西豐。


白狼,宣王得之而犬戎服。
 \gezhu{
  闕}



 白麞,王者刑罰理則至。



 晉武帝咸寧元年四月丙戌、乙卯,白麞見琅邪,趙王倫
 以獻。咸寧三年七月壬辰,白麞見魏郡。晉武帝太康三年八月,白麞見梁國蒙,梁相解隆獲以獻。太康五年九月己酉,白麞見義陽。太康七年五月戊辰,白麞見汲郡。



 晉成帝咸和九年五月癸酉,白麞見吳國吳縣,內史虞潭獲以獻。



 晉穆帝永和元年八月,白麞見吳國吳縣西界包山,獲
 以獻。永和八年十二月,白麞見丹陽永世,永世令徐該獲以獻。永和十二年十一月庚午,白麞見梁郡,梁郡太守劉遂獲以獻。



 晉安帝隆安五年十一月,白麞見荊州,荊州刺史桓玄以聞。



 宋少帝景平元年五月癸未,白麞見義興陽羨,太守王準之獲以獻。
 景平二年六月,白麞見南郡江陽,太守王華獻之太祖。太祖時入奉大統,以為休祥。



 文帝元嘉五年四月乙巳,白麞見汝陽武津,太守鄭據獲以獻。元嘉十二年正月,白麞見東萊黃縣,青、冀州刺史王方回以獻。元嘉十九年五月,山陽張休宗獲白麞,南兗州刺史臨川王義慶以獻。
 元嘉二十年八月,白麞見江夏安陸,內史劉思考以獻。



 元嘉二十五年二月己丑,白麞見淮南,太守王休獲以獻。元嘉二十五年四月戊午,白麞見南瑯邪,太守王遠獲以獻。元嘉二十五年五月辛未朔,華林園白麞生二子皆白,園丞梅道念以聞。元嘉二十六年五月丙戌,白麞見馬頭,豫州刺史南平
 王鑠以獻。元嘉二十七年正月己丑,白麞見濟陰,徐州刺史武陵王駿以聞。元嘉二十七年四月癸丑,華林園白麞生一白子,園丞梅道念以聞。元嘉二十九年六月壬戌,白麞見晉陵暨陽,南徐州刺史始興王濬以獻。



 孝武帝孝建三年六月癸巳,白麞見廣陵,南兗州以獻。
 孝武帝大明元年七月丁丑,白麞見東萊曲城縣,獲以獻。大明二年正月壬戌,白麞見山陽,山陽內史程天祚以獻。大明二年二月辛丑,白麞見濟北,濟北太守殷孝祖以獻。大明五年九月己巳,白麞見南陽,雍州刺史永嘉王子仁以獻。
 大明六年四月戊辰,白麞見營陽,湘州刺史建安王休仁以獻。大明七年正月庚寅,白麞見南陽,荊州刺史臨海王子頊以獻。大明七年六月己巳,白麞見武陵臨沅,太守劉衍以獻。大明七年九月癸未,白麞見南陽,雍州刺史劉秀之以獻。



 明帝泰始三年五月癸酉,白麞見南東海丹徒,南徐州
 刺史桂陽王休範以獻。泰始三年五月乙卯,白麞見北海都昌,青州刺史沈文秀以獻。泰始五年正月癸卯,白麞見汝陰樓煩,豫州刺史劉勔以獻。明帝泰豫元年十月壬戌,白麞見義興國山,太守王蘊以獻。



 後廢帝元徽元年正月甲午,白麞見海陵寧海,寧海太
 守孫嗣之以獻。



 文帝元嘉二十三年五月甲寅,東宮隊白從陳超獲黑麞於肥如縣,皇太子以獻。



 元嘉二十三年十月辛巳,東宮將魏榮獲青麞於秣陵。元嘉十年十二月,營城縣民成公會之於廣陵高郵界獲白麞麂以獻。



 孝武帝大明元年二月己亥,白麂見會稽諸暨縣,獲以獻。


銀麂,刑罰得共,民不為非則至。
 \gezhu{
  闕}


赤兔,王者德盛則至。
 \gezhu{
  闕}


比翼鳥,王者德及高遠則至。
 \gezhu{
  闕}
 。



 赤雀,周文王時銜丹書來至。



 晉愍帝建興三年四月癸酉,赤雀見平州府舍。



 宋文帝元嘉二十年五月,赤雀集南平郡府,內史臧綽以聞。



 孝武帝孝建元年五月己亥,臨沂縣魯尚斯軍人於城
 上獲赤雀,太傅假黃鉞江夏王義恭以獻。


福草者,宗廟肅,則生宗廟之中。
 \gezhu{
  闕}



 蒼烏者,賢君脩行孝慈於萬姓,不好殺生則來。



 宋孝武帝大明元年五月丁丑,蒼鳥見襄陽縣。大明二年四月甲申,蒼鳥見襄陽,雍州刺史王玄謨以獻。



 甘露,王者德至大,和氣盛,則降。柏受甘露,王者耆老見敬,則柏受甘露。



 竹受甘露,王者尊賢愛老,不失細微,則竹葦受甘露。



 漢宣帝元康元年三月,甘露降未央宮。漢宣帝神雀二年二月,甘露降京師。神雀四年春,甘露降京師。



 漢宣帝五鳳二年正月,甘露降京師。



 漢成帝元延四年三月,甘露降京師。



 漢光武建武中元元年五月,郡國上甘露降。



 漢明帝永平十七年正月戊子夜,帝夢見光武帝、光烈
 皇后,夢中喜覺,悲不能寐。明旦上陵,百官、胡客悉會。太常丞上言,其日陵樹葉有甘露。帝令百官采甘露。帝自伏御床,視太后莊器奩中物,流涕,敕易奩中脂澤之具。永平十七年春,甘露仍降京師。



 漢章帝元和中,甘露降郡國。



 漢安帝延光三年四月丙戌,甘露下沛國豐。延光三年七月,甘露下左馮翊頻陽。



 漢桓帝延熹三年四月,甘露降上郡。
 漢桓帝永康元年八月,甘露降巴郡。



 魏文帝初,郡國三十七言甘露降。魏少帝甘露元年五月,鄴及上洛並言甘露降。



 魏元帝咸熙二年四月,南深澤縣言甘露降。



 吳孫權黃武前,建業言甘露降。黃武二年五月,曲阿言甘露降。



 吳孫權嘉禾五年三月,武昌言甘露降於禮賓殿。吳孫權赤烏二年三月,零陵言甘露降。赤烏九年四月,武昌言甘露降。
 吳孫皓甘露元年四月,蔣陵言甘露降。



 晉武帝泰始十年四月乙亥,甘露降西河離石。晉武帝咸寧元年四月丙戌,甘露降張掖。咸寧元年五月戊午,甘露降清河繹幕。咸寧元年九月,甘露降太原晉陽。



 咸寧二年五月戊子,甘露降玄菟郡治。咸寧三年六月戊申,甘露降巴郡南充國。
 晉武帝太康五年三月乙卯,甘露降東宮。太康七年四月,甘露降京兆杜陵。太康七年五月,甘露降魏郡鄴。



 晉惠帝元康四年五月,甘露降樂陵郡。



 晉愍帝建興元年六月,甘露降西平縣。建興三年八月己未,甘露降新昌縣。晉愍帝建武元年六月丁丑,甘露降壽春。



 晉元帝太興三年四月,甘露降琅邪費。



 晉明帝泰寧二年正月,巴郡言甘露降。



 晉成帝咸和四年四月,甘露降武昌郡閣前柳樹,太守詡以聞。咸和六年三月,甘露降寧州城內北園榛桃樹,刺史以聞。咸和七年四月癸巳,甘露降京邑,揚州刺史王導以聞。咸和八年四月癸卯,甘露降廬江襄安縣蔣胄家。咸和八年四月癸卯,甘露降宣城宛陵縣之須里。
 咸和九年四月甲寅,甘露降吳國錢唐縣右鄉康巷之柳樹。



 咸和九年十二月丙辰,甘露降建平陵。咸和九年十二月丁巳,甘露降武平陵。晉成帝咸康元年四月癸卯,甘露降西堂桃樹。咸康二年三月甲戌,甘露降鬱林城內。咸康二年四月,甘露降西堂,又降尚書都坐桃樹,又降會稽永興縣,眾官畢賀。戊午,甘露降會稽山陰縣,又降
 吳興武康縣。庚申,又降武康。咸康三年四月戊午,甘露降殿後桃李樹。五月,甘露降義興陽羨縣柞樹,東西十四步,南北十五步。咸康七年四月丙子,甘露降彭城王紘第內,眾官畢賀。



 晉穆帝永和元年三月,甘露降廬江郡內桃李樹,太守永以聞。永和五年十一月,太常劉邵上崇平陵令王昂即日奉行陵內,甘露降於玄宮前殿。
 永和五年十二月己酉,甘露降丹陽湖熟縣西界劉敷墓松樹,縣令王恬以聞,眾官畢賀。



 晉簡文帝咸安二年正月,甘露降隨郡灄陽縣界桑木,沾凝十餘里中。



 晉孝武帝太元十二年八月,甘露降寧州界內,刺史費統以聞。太元十五年閏月,甘露降永平陵。太元十六年十一月庚午,甘露降句陽縣。
 太元十七年二月,甘露降南海番禺縣楊樹。



 晉安帝元興二年十月,甘露降武昌王成基家竹。元興三年三月己卯,甘露降丹徒。元興三年四月己酉,甘露降蘭臺。



 宋武帝永初元年九月庚辰,甘露降丹徒峴山。永初元年十月庚午,甘露降興寧、永寧二陵,彌冠百餘里。



 文帝元嘉三年閏正月己丑,甘露降吳興烏程,太守王
 韶之以聞。元嘉四年五月辛巳,甘露降齊郡西安臨朐城。元嘉四年十一月辛未朔,甘露降初寧陵。元嘉四年十一月己丑,甘露降南海熙安,廣州刺史江桓以聞。元嘉八年五月,甘露降南海番禺。元嘉九年十一月壬子,甘露降初寧陵。元嘉十一年八月甲辰,甘露降費縣之沙里,琅邪太守
 呂綽以聞。元嘉十三年二月丁卯,甘露降上明巴山。元嘉十三年二月,甘露降吳興武康董道益家園樹。元嘉十三年三月甲午,甘露降初寧陵。元嘉十六年三月己卯,甘露降廣州城北門楊樹,刺史陸徽以聞。元嘉十七年四月丁丑,甘露降廣陵永福里梁昌季家樹,南兗州刺史江夏王義恭以聞。
 元嘉十七年,甘露降高平金鄉富民村方三十里中。徐州刺史趙伯符以聞。元嘉十七年十一月乙酉,甘露降樂游苑。元嘉十八年五月甲申,甘露降丹陽秣陵衛將軍臨川王義慶園,揚州刺史始興王濬以聞。元嘉十八年六月,甘露降廣陵廣陵孟玉秀家樹,南兗州刺史臨川王義慶以聞。元嘉十九年五月丁卯,甘露降建康司徒參軍督護顧
 俊之宅竹柳。元嘉十九年五月乙亥,甘露降馬頭濟陽宋慶之園樹,太守荀預以聞。元嘉二十一年,甘露降益州府內梨李樹,刺史庾俊之以聞。元嘉二十一年四月,甘露頻降樂游苑。元嘉二十一年四月,甘露降彭城綏輿里,徐州刺史臧質以聞。
 元嘉二十一年四月,甘露降義陽平陽,太守龐秀之以聞。元嘉二十二年十一月辛巳,甘露降南郡江陵方城里,荊州刺史南譙王義宣以聞。元嘉二十二年十二月丁酉,甘露降長寧陵,陵令包誕以聞。



 元嘉二十三年二月丁未,甘露降樂遊苑,苑丞張寶以聞。
 元嘉二十三年九月丙子,甘露降長寧陵,陵令華林以聞。元嘉二十三年十二月庚子,甘露降襄陽郡治,雍州刺史武陵王駿以聞。元嘉二十三年十二月辛丑,甘露頻降樂遊苑,苑丞何道之以聞。



 元嘉二十四年二月己亥、庚子,甘露頻降景陽山,山監張績以聞。
 元嘉二十四年二月己亥、癸卯、三月丙辰,甘露頻降景陽山,華林園丞陳襲祖以聞。元嘉二十四年三月甲寅,甘露降尋陽松滋,江州刺史廬陵王紹以聞。元嘉二十四年四月癸未,甘露降尋陽松滋;丙申,又降江州城內桐樹;丁酉,又降城北數里之中,江州刺史廬陵王紹以聞。元嘉二十四年七月乙卯,甘露降京師,揚州刺史始興
 王濬以聞。元嘉二十四年七月,甘露降襄城治下無量寺,雍州刺史武陵王駿以聞。元嘉二十四年十月甲午,甘露降魏興郡內,太守韋寧民以聞。元嘉二十三年至二十四年十二月,甘露頻降,狀如細雪,京都及郡國處處皆然,不可稱紀。元嘉二十五年十一月庚辰,甘露降南郡,荊州刺史南
 譙王義宣以聞。元嘉二十五年十一月乙未,甘露降丹陽秣陵巖山。元嘉二十六年三月壬午,甘露降景陽山,華林園丞梅道念以聞。元嘉二十六年三月庚寅、癸巳,甘露頻降武昌,江州刺史廬陵王紹以聞。元嘉二十六年四月甲辰、丙午、戊申,甘露頻降豫章南昌,太守劉思考以聞。
 元嘉二十六年七月,甘露降南郡江陵,荊州刺史南譙王義宣以聞。元嘉二十七年四月乙卯、丙辰、丁巳,甘露頻降豫章南昌。戊午午時,天氣清明,有彩霧映覆郡邑,甘露又自雲降。太守劉思考以聞。元嘉二十七年五月甲戌,甘露降東海丹徒,南徐州刺史始興王濬以聞。



 元嘉二十八年二月戊辰,甘露降鐘山延賢寺,揚州刺
 史廬陵王紹以聞。元嘉二十八年二月壬午,甘露降徽音殿前果樹。元嘉二十八年二月,甘露降合歡殿後香花諸草。



 孝武帝孝建元年三月丙辰,甘露降華林園。孝建二年三月己酉,甘露降丹陽秣陵中里路與之墓樹。孝建二年三月辛亥,甘露降長寧陵松樹。孝建二年三月,甘露降襄陽民家梨樹。
 孝建二年三月戊午,甘露降丹陽秣陵尚書謝莊園竹林,莊以聞。



 孝武帝大明元年四月癸卯,甘露降華林園桐樹。大明三年三月己卯,甘露降樂遊苑梅樹。大明三年三月戊子,甘露降宣城郡舍,太守張辯以聞。大明四年正月壬辰,甘露降初寧陵松樹。大明四年二月丙申,甘露降長寧陵松樹。大明四年二月乙巳,甘露降丹陽秣陵龍山,丹陽尹孔
 靈符以聞。大明五年四月辛亥,甘露降吳興安吉,太守歷陽王子頊以聞。大明五年四月乙卯,甘露降吳興烏程,太守歷陽王子頊以聞。大明六年二月戊午,甘露降建康靈耀寺及諸苑園,及秣陵龍山,至於婁湖。



 是日,又降句容、江寧二縣。大明七年三月丙申,甘露降尋陽松滋,太守劉矇以聞。



 大明七年四月己未,甘露降荊州城內,刺史臨海王子頊以聞。大明七年十二月辛丑朔,甘露降吳興烏程,令茍卞之以聞。



 明帝泰始二年四月己亥,甘露降上林苑,苑令徐承道以獻。泰始二年四月庚申,甘露降華林園,園令臧延之以獻。泰始二年五月己未,甘露降丹陽秣陵縣舍齋前竹,丹
 陽尹王景文以獻。泰始三年十一月庚申,甘露降晉陵,晉陵太守王蘊以聞。泰始三年十一月癸亥,甘露降南東海丹徒建岡,徐州刺史桂陽王休範以聞。泰始三年十二月壬午,甘露降崇寧陵,揚州刺史建安王休仁以聞。



 後廢帝元徽四年十一月乙巳,甘露降吳興烏程,太守
 蕭惠明以聞。



 順帝升明二年十二月,甘露降建康禁中里。升明二年十一月,甘露降南東海武進彭山,太守謝朏以聞。升明二年十一月,甘露降吳興長城卞山,太守王奐以聞。


威香者,王者禮備則常生。
 \gezhu{
  闕}



\end{pinyinscope}