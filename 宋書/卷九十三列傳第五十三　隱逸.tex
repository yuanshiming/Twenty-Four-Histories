\article{卷九十三列傳第五十三 隱逸}

\begin{pinyinscope}

 《易》曰:「天地閉,賢人隱。」又
 曰:「遁世無悶。」又曰:「高尚其事。」



 又曰:「幽人貞吉。」《論語》「作者七人」,表以逸民之稱。又曰:「子路遇荷丈人,孔子曰:隱者也。」又曰:「賢者避地,其次避言。」
 又曰:「虞仲,夷逸,隱居放言。」品目參差,稱謂非一,請試言之:夫隱之為言,迹不外見,道不可知之謂也。若夫千載寂寥,聖人不出,則大賢自晦,降夷凡品。止於全身遠害,非必穴處巖栖,雖藏往得二,鄰亞宗極,而舉世莫窺,萬物不睹。若此人者,豈肯洗耳潁濱,皦皦然顯出俗之志乎!遁世避言,即賢人也。夫何適非世,而有避世之因,固知義惟晦道,非曰藏身。至於巢父之名,即是見稱之號,號曰裘公,由有可傳之迹。此蓋荷之隱,而非賢人之
 隱也。賢人之隱,義深於自晦,荷之隱,事止於違人。論跡既殊,原心亦異也。身與運閉,無可知之情,雞黍宿賓,示高世之美。運閉故隱,為隱之跡不見;違人故隱,用致隱者之目。身隱故稱隱者,道隱故曰賢人。或曰:「隱者之異乎隱,既聞其說,賢者之同於賢,未知所異?」應之曰:「隱身之於晦道,名同而義殊,賢人之於賢者,事窮於亞聖,以此為言,如或可辨。



 若乃高尚之與作者,三避之與幽人,及逸民隱居,皆獨往之稱,雖復漢陰之氏不傳,河上
 之名不顯,莫不激貪厲俗,秉自異之姿,猶負揭日月,鳴建鼓而趨也。」陳郡袁淑集古來無名高士,以為《真隱傳》,格以斯談,去真遠矣。賢人在世,事不可誣,今為《隱逸篇》,虛置賢隱之位,其餘夷心俗表者,蓋逸而非隱云。



 戴顒,字仲若,譙郡銍人也。父逵,兄勃,並隱遁有高名。顒年十六,遭父憂,幾於毀滅,因此長抱羸患。以父不仕,復脩其業。父善琴書,顒並傳之,凡諸音律,皆能揮手。會稽剡縣多名山,故世居剡下。顒及兄勃,並受琴於父。父沒,
 所傳之聲,不忍復奏,各造新弄,勃五部,顒十五部。顒又制長弄一部,並傳於世。中書令王綏常攜賓客造之,勃等方進豆粥,綏曰:「聞卿善琴,試欲一聽。」不答,綏恨而去。



 桐廬縣又多名山,兄弟復共游之,因留居止。勃疾患,醫藥不給。顒謂勃曰:「顒隨兄得閑,非有心於默語。兄今疾篤,無可營療,顒當干祿以自濟耳。」乃告時求海虞令,事垂行而勃卒,乃止。桐廬僻遠,難以養疾,乃出居吳下。吳下士人共為築室,聚石引水,植林開澗,少時繁密,有若
 自然。乃述莊周大旨,著《逍遙論》,注《禮記·中庸》篇。三吳將守及郡內衣冠要其同遊野澤,堪行便往,不為矯介,眾論以此多之。



 高祖命為太尉行參軍,瑯邪王司馬屬,並不就。宋國初建,令曰:「前太尉參軍戴顒、辟士韋玄,秉操幽遁,守志不渝,宜加旌引,以弘止退。並可散騎侍郎,在通直。」不起。太祖元嘉二年,詔曰:「新除通直散騎侍郎戴顒、太子舍人宗炳,並志託丘園,自求衡蓽,恬靜之操,久而不渝。顒可國子博士,炳可通直散騎侍郎。」



 東宮初建,
 又徵太子中庶子。十五年,徵散騎常侍,並不就。



 衡陽王義季鎮京口,長史張邵與顒姻通,迎來止黃鵠山。山北有竹林精舍,林澗甚美。顒憩于此澗,義季亟從之遊,顒服其野服,不改常度。為義季鼓琴,並新聲變曲,其三調《遊絃》、《廣陵》、《止息》之流,皆與世異。太祖每欲見之,嘗謂黃門侍郎張敷曰:「吾東巡之日,當晏戴公山也。」以其好音,長給正聲伎一部。顒合《何嘗》、《白鵠》二聲,以為一調,號為清曠。自漢世始有佛像,形制未工,逵特善其事,顒亦參
 焉。宋世子鑄丈六銅像於瓦官寺,既成,面恨瘦,工人不能治,乃迎顒看之。顒曰:「非面瘦,乃臂胛肥耳。」既錯減臂胛,瘦患即除,無不歎服焉。



 十八年,卒,時年六十四。無子。景陽山成,顒已亡矣。上歎曰:「恨不得使戴顒觀之。」



 宗炳,字少文,南陽涅陽人也。祖承,宜都太守。父繇之,湘鄉令。母同郡師氏,聰辯有學義,教授諸子。炳居喪過禮,為鄉閭所稱。刺史殷仲堪、桓玄並辟主簿,舉秀才,不就。高祖誅劉毅,領荊州,問毅府咨議參軍申永曰:「今日何
 施而可?」永曰:「除其宿釁,倍其惠澤,貫敘門次,顯擢才能,如此而已。」高祖納之,辟炳為主簿,不起。問其故,答曰:「棲丘飲谷,三十餘年。」高祖善其對。



 妙善琴書,精於言理,每遊山水,往輒忘歸。征西長史王敬弘每從之,未嘗不彌日也。乃下入廬山,就釋慧遠考尋文義。兄臧為南平太守,逼與俱還,乃於江陵三湖立宅,閑居無事。高祖召為太尉參軍,不就。二兄蚤卒,孤累甚多,家貧無以相贍,頗營稼穡。高祖數致餼賚,其後子弟從祿,乃悉不復受。



 高
 祖開府辟召,下書曰:「吾忝大寵,思延賢彥,而《兔置》潛處,《考槃》未臻,側席丘園,良增虛佇。南陽宗炳、鴈門周續之,並植操幽棲,無悶巾褐,可下辟召,以禮屈之。」於是並辟太尉掾,皆不起。宋受禪,徵為太子舍人;元嘉初,又徵通直郎;東宮建,徵為太子中舍人,庶子,並不應。妻羅氏,亦有高情,與炳協趣。羅氏沒,炳哀之過甚,既而輟哭尋理,悲情頓釋。謂沙門釋慧堅曰:「死生不分,未易可達,三復至教,方能遣哀。」衡陽王義季在荊州,親至炳室,與之歡
 宴,命為咨議參軍,不起。



 好山水,愛遠遊,西陟荊、巫,南登衡、岳,因而結宇衡山,欲懷尚平之志。



 有疾還江陵,嘆曰:「老疾俱至,名山恐難遍睹,唯當澄懷觀道,臥以遊之。」凡所遊履,皆圖之於室,謂人曰:「撫琴動操,欲令眾山皆響。」古有《金石弄》,為諸桓所重,桓氏亡,其聲遂絕,惟炳傳焉。太祖遣樂師楊觀就炳受之。



 炳外弟師覺授亦有素業,以琴書自娛。臨川王義慶辟為祭酒,主簿,並不就,乃表薦之,會病卒。元嘉二十年,炳卒,時年六十九。衡陽王義
 季與司徒江夏王義恭書曰:「宗居士不救所病,其清履肥素,終始可嘉,為之惻愴,不能已已。」子朔,南譙王義宣車騎參軍。次綺,江夏王義恭司空主簿。次昭,郢州治中。次說,正員郎。



 周續之,字道祖,鴈門廣武人也。其先過江居豫章建昌縣。續之年八歲喪母,哀戚過於成人,奉兄如事父。豫章太守范寧於郡立學,招集生徒,遠方至者甚眾。



 續之年十二,詣寧受業。居學數年,通《五經》并《緯候》,名冠同門,號
 曰「顏子」。既而閑居讀《老》、《易》,入廬山事沙門釋慧遠。時彭城劉遺民遁迹廬山,陶淵明亦不應徵命,謂之「尋陽三隱。」以為身不可遣,餘累宜絕,遂終身不娶妻,布衣蔬食。



 劉毅鎮姑孰,命為撫軍參軍,徵太學博士,並不就。江州刺史每相招請,續之不尚節峻,頗從之游。常以嵇康《高士傳》得出處之美,因為之注。高祖之北討,世子居守,迎續之館於安樂寺,延入講禮,月餘,復還山。江州刺史劉柳薦之高祖,曰:「臣聞恢耀和肆,必在兼城之寶;翼亮崇本,宜
 紆高世之逸。是以渭濱佐周,聖德廣運,商洛匡漢,英業乃昌。伏惟明公道邁振古,應天繼期,遊外暢於冥內,體遠形於應近,雖汾陽之舉,輟駕於時艱;明揚之旨,潛感於穹谷矣。竊見處士雁門周續之,清真貞素,思學鉤深,弱冠獨往,心無近事,性之所遣;榮華與饑寒俱落,情之所慕,巖澤與琴書共遠。加以仁心內發,義懷外亮,留愛昆卉,誠著桃李。



 若升之宰府,必鼎味斯和;濯纓儒官,亦王猷遐緝。臧文不知,失在降賢;言偃得人,功由升士。願
 照其丹款,不以人廢言。」俄而辟為太尉掾,不就。



 高祖北伐,還鎮彭城,遣使迎之,禮賜甚厚。每稱之曰:「心無偏吝,真高士也。」尋復南還。高祖踐阼,復召之,乃盡室俱下。上為開館東郭外,招集生徒。



 乘輿降幸,並見諸生,問續之《禮記》「傲不可長」、「與我九齡」、「射於矍圃」



 三義,辨析精奧,稱為該通。續之素患風痺,不復堪講,乃移病鐘山。景平元年卒,時年四十七。通《毛詩》六義及《禮論》、《公羊傳》,皆傳於世。無子。兄子景遠有續之風,太宗泰始中,為晉安內史,
 未之郡,卒。



 王弘之,字方平,琅邪臨沂人,宣訓衛尉鎮之弟也。少孤貧,為外祖征士何準所撫育。從叔獻之及太原王恭,並貴重之。晉安帝隆安中,為琅邪王中軍參軍,遷司徒主簿。家貧,而性好山水,求為烏程令,尋以病歸。桓玄輔晉,桓謙以為衛軍參軍。時琅邪殷仲文還姑孰,祖送傾朝,謙要弘之同行,答曰:「凡祖離送別,必在有情,下官與殷風馬不接,無緣扈從。」謙貴其言。每隨兄鎮之之安成郡,
 弘之解職同行,荊州刺史桓偉請為南蠻長史。



 義熙初,何無忌又請為右軍司馬。高祖命為徐州治中從事史,除員外散騎常侍,並不就。家在會稽上虞。從兄敬弘為吏部尚書,奏曰:「聖明司契,載德惟新,垂鑑仄微,表揚隱介,默語仰風,荒遐傾首。前員外散騎常侍琅邪王弘之,恬漠丘園,放心居逸。前衛將軍參軍武昌郭希林,素履純潔,嗣徽前武。並擊壤聖朝,未蒙表飾,宜加旌聘,賁于丘園,以彰止遜之美,以祛動求之累。臣愚謂弘之可太
 子庶子,希林可著作郎。」即徵弘之為庶子,不就。太祖即位,敬弘為左僕射,又陳:「弘之高行表於初筮,苦節彰於暮年。今內外晏然,當修太平之化,宜招空谷,以敦沖退之美。」元嘉四年,徵為通直散騎常侍,又不就。敬弘嘗解貂裘與之,即著以采藥。



 性好釣,上虞江有一處名三石頭,弘之常垂綸於此。經過者不識之,或問:「漁師得魚賣不?」弘之曰:「亦自不得,得亦不賣。」日夕載魚入上虞郭,經親故門,各以一兩頭置門內而去。始寧汰川有佳山水,
 弘之又依巖築室。謝靈運、顏延之並相欽重,靈運與廬陵王義真箋曰:「會境既豐山水,是以江左嘉遁,並多居之。但季世慕榮,幽棲者寡,或復才為時求,弗獲從志。至若王弘之拂衣歸耕,踰歷三紀;孔淳之隱約窮岫,自始迄今;阮萬齡辭事就閑,纂成先業;浙河之外,棲遲山澤,如斯而已。既遠同羲、唐,亦激貪厲競。殿下愛素好古,常若布衣,每憶昔聞,虛想巖穴,若遣一介,有以相存,真可謂千載盛美也。」



 弘之四年卒,時年六十三。顏延之欲為
 作誄,書與弘之子曇生曰:「君家高世之節,有識歸重,豫染豪翰,所應載述。況僕託慕末風,竊以敘德為事,但恨短筆不足書美。」誄竟不就。曇生好文義,以謙和見稱。歷顯位,吏部尚書,太常卿。



 大明末,為吳興太守。太宗初,四方同逆,戰敗奔會稽,歸降被宥,終於中散大夫。



 阮萬齡,陳留尉氏人也。祖思曠,左光祿大夫。父寧,黃門侍郎。萬齡少知名,自通直郎為孟昶建威長史。時袁豹、江夷相係為昶司馬,時人謂昶府有三素望。萬齡家在
 會稽剡縣,頗有素情。永初末,自侍中解職東歸,徵為秘書監,加給事中,不就。尋除左民尚書,復起應命,遷太常,出為湘州刺史,在州無政績。還為東陽太守,又被免。復為散騎常侍、金紫光祿大夫。元嘉二十五年卒,時年七十二。



 孔淳之,字彥深,魯郡魯人也。祖惔,尚書祠部郎。父粲,祕書監征,不就。



 淳之少有高尚,愛好墳籍,為太原王恭所稱。居會稽剡縣,性好山水,每有所遊,必窮其幽峻,或旬
 日忘歸。當遊山,遇沙門釋法崇,因留共止,遂停三載。法崇嘆曰:「緬想人外,三十年矣,今乃公傾蓋于茲,不覺老之將至也。」及淳之還反,不告以姓。除著作佐郎,太尉參軍,並不就。



 居喪至孝,廬于墓側。服闋,與徵士戴顒、王弘之及王敬弘等共為人外之遊。



 敬弘以女適淳之子尚。會稽太守謝方明苦要入郡,終不肯往。茅室蓬戶,庭草蕪徑,唯床上有數卷書。元嘉初,復徵為散騎侍郎,乃逃于上虞縣界,家人莫知所之。弟默之為廣州刺史,出都與
 別。司徒王弘要淳之集冶城,即日命駕東歸,遂不顧也。



 元嘉七年,卒,時年五十九。默之儒學,注《穀梁春秋》。默之子熙先,事在《范曄傳》。



 劉凝之,字志安,小名長年,南郡枝江人也。父期公,衡陽太守。兄盛公,高尚不仕。凝之慕老萊、嚴子陵為人,推家財與弟及兄子,立屋於野外,非其力不食,州里重其德行。州三禮辟西曹主簿,舉秀才,不就。妻梁州刺史郭銓女也,遣送豐麗,凝之悉散之親屬。妻亦能不慕榮華,與
 凝之共安儉苦。夫妻共乘薄笨車,出市買易,周用之外,輒以施人。為村里所誣,一年三輸公調,求輒與之。有人嘗認其所著屐,笑曰:「僕著之已敗,今家中覓新者備君也。」此人後田中得所失屐,送還之,不肯復取。



 元嘉初,徵為祕書郎,不就。臨川王義慶、衡陽王義季鎮江陵,並遣使存問。



 凝之答書頓首稱僕,不脩民禮,人或譏焉。凝之曰:「昔老萊向楚王稱僕,嚴陵亦抗禮光武,未聞巢、許稱臣堯、舜。」時戴顒與衡陽王義季書,亦稱僕。荊州年饑,義
 季慮凝之喂斃,餉錢十萬。凝之大喜,將錢至市門,觀有饑色者,悉分與之,俄頃立盡。性好山水,一旦攜妻子泛江湖,隱居衡山之陽。登高嶺,絕人跡,為小屋居之,采藥服食,妻子皆從其志。元嘉二十五年,卒,時年五十九。



 龔祈,字孟道,武陵漢壽人也。從祖玄之,父黎民,並不應徵辟。祈年十四,鄉黨舉為州迎西曹,不行。謝晦臨州,命為主簿;彭城王義康舉秀才,除奉朝請;臨川王義慶平西參軍,皆不就。風姿端雅,容止可觀,中書郎范述見而
 嘆曰:「此荊楚仙人也。」衡陽王義季臨荊州,發教以祈及劉凝之、師覺授不應徵召,辟其三子。祈又徵太子舍人,不起。時或賦詩,言不及世事。元嘉十七年,卒,時年四十二。



 翟法賜,尋陽柴桑人也。曾祖湯,湯子莊,莊子矯,並高尚不仕,逃避征辟。



 矯生法賜。少守家業,立屋於廬山頂,喪親後,便不復還家。不食五穀,以獸皮結草為衣,雖鄉親中表,莫得見也。州辟主簿,舉秀才,右參軍,著作佐郎,員
 外散騎侍郎,並不就。後家人至石室尋求,因復遠徙,違避徵聘,遁跡幽深。尋陽太守鄧文子表曰:「奉詔書徵郡民新除著作佐郎南陽翟法賜,補員外散騎侍郎。法賜隱跡廬山,于今四世,棲身幽巖,人罕見者。如當逼以王憲,束以嚴科,馳山獵草,以期禽獲,慮致顛殞,有傷盛化。」乃止。後卒於巖石之間,不知年月。



 陶潛,字淵明,或云淵明,字元亮,尋陽柴桑人也,曾祖侃,晉大司馬。潛少有高趣,嘗著《五柳先生傳》以自況,曰:先
 生不知何許人,不詳姓字,宅邊有五柳樹,因以為號焉。閑靜少言,不慕榮利。好讀書,不求甚解,每有會意,欣然忘食。性嗜酒,而家貧不能恒得。親舊知其如此,或置酒招之。造飲輒盡,期在必醉,既醉而退,曾不吝情去留。環堵蕭然,不蔽風日,短褐穿結,簞瓢屢空,晏如也。嘗著文章自娛,頗示己志,忘懷得失,以此自終。



 其自序如此,時人謂之實錄。親老家貧,起為州祭酒,不堪吏職,少日,自解歸。州召主簿,不就。躬耕自資,遂抱羸疾,復為鎮軍、建
 威參軍。謂親朋曰:「聊欲弦歌,以為三徑之資,可乎?」執事者聞之,以為彭澤令。公田悉令吏種秫稻。妻子固請種粳,乃使二頃五十畝種秫,五十畝種粳。郡遣督郵至,縣吏白應束帶見之。潛嘆曰:「我不能為五斗米折腰向鄉里小人。」即日解印綬去職。賦《歸去來》,其詞曰:歸去來兮,園田荒蕪胡不歸。既自以心為形役,奚惆悵而獨悲。悟已往之不諫,知來者之可追。實迷途其未遠,覺今是而昨非。舟遙遙以輕颺,風飄飄而吹衣。問征夫以前路,恨
 晨光之希微。



 乃瞻衡宇,載欣載奔。僮僕歡迎,稚子候門。三徑就荒,松菊猶存。攜幼入室,有酒停尊。引壺觴而自酌,盼庭柯以怡顏。倚南窗而寄傲,審容膝之易安。園日涉而成趣,門雖設而常關。策扶老以流心妻,時矯首而遐觀,雲無心以出岫,鳥倦飛而知還。景翳翳其將入,撫孤松以盤桓。



 歸去來兮,請息交而絕遊,世與我以相遺,復駕言兮焉求。說親戚之情話,樂琴書以消憂。農人告余以上春,將有事於西疇。或命巾車,或棹扁舟。既窈窕以
 窮壑,亦崎嶇而經丘。木欣欣以向榮,泉涓涓而始流。善萬物之得時,感吾生之行休。



 已矣乎,寓形宇內復幾時,奚不委心任去留,胡為遑遑欲何之。富貴非吾願,帝鄉不可期。懷良辰以孤往,或植杖而耘耔。登東皋以舒嘯,臨清流而賦詩。聊乘化以歸盡,樂夫天命復奚疑。



 義熙末,徵著作佐郎,不就。江州刺史王弘欲識之,不能致也。潛嘗往廬山,弘令潛故人龐通之齎酒具於半道栗里要之。潛有腳疾,使一門生二兒輿籃輿,既至,欣然便共
 飲酌,俄頃弘至,亦無忤也。先是,顏延之為劉柳後軍功曹,在尋陽,與潛情款。後為始安郡,經過,日日造潛,每往必酣飲致醉。臨去,留二萬錢與潛,潛悉送酒家,稍就取酒。嘗九月九日無酒,出宅邊菊叢中坐久,值弘送酒至,即便就酌,醉而後歸。潛不解音聲,而畜素琴一張,無絃,每有酒適,輒撫弄以寄其意。



 貴賤造之者,有酒輒設,潛若先醉,便語客:「我醉欲眠,卿可去。」其真率如此。



 郡將候潛值其酒熟,取頭上葛巾漉酒,畢,還復著之。



 潛弱年薄
 官,不潔去就之跡。自以曾祖晉世宰輔,恥復屈身後代,自高祖王業漸隆,不復肯仕。所著文章,皆題其年月,義熙以前,則書晉氏年號;自永初以來,唯云甲子而已。與子書以言其志,并為訓戒曰:天地賦命,有往必終,自古賢聖,誰能獨免。子夏言曰:「死生有命,富貴在天。」四友之人,親受音旨,發斯談者,豈非窮達不可妄求,壽夭永無外請故邪。



 吾年過五十,而窮苦荼毒,家貧弊,東西遊走。性剛才拙,與物多忤,自量為己,必貽俗患,僶俛辭世,使
 汝幼而饑寒耳。常感孺仲賢妻之言,敗絮自擁,何慚兒子。



 此既一事矣。但恨鄰靡二仲,室無萊婦,抱茲苦心,良獨罔罔。



 少年來好書,偶愛閑靜,開卷有得,便欣然忘食。見樹木交蔭,時鳥變聲,亦復歡爾有喜。嘗言五六月北窗下臥,遇涼風暫至,自謂是羲皇上人。意淺識陋,日月遂往,緬求在昔,眇然如何。疾患以來,漸就衰損,親舊不遺,每以藥石見救,自恐大分將有限也。恨汝輩稚小,家貧無役,柴水之勞,何時可免,念之在心,若何可言。然雖
 不同生,當思四海皆弟兄之義。鮑叔、敬仲,分財無猜;歸生、伍舉,班荊道舊,遂能以敗為成,因喪立功。他人尚爾,況共父之人哉!潁川韓元長,漢末名士,身處卿佐,八十而終,兄弟同居,至於沒齒。濟北氾稚春,晉時操行人也,七世同財,家人無怨色。《詩》云:「高山仰止,景行行止。」汝其慎哉!吾復何言。



 又為《命子詩》以貽之曰:悠悠我祖,爰自陶唐。邈為虞賓,歷世垂光。御龍勤夏,豕韋翼商。穆穆司徒,厥族以昌。紛紜戰國,漠漠衰周。鳳隱于林,幽人在丘。
 逸虯撓雲,奔鯨駭流。天集有漢,眷予愍侯。於赫愍侯,運當攀龍。撫劍夙邁,顯茲武功。參誓山河,啟土開封。亹亹丞相,允迪前蹤。渾渾長源,蔚蔚洪柯。群川載導,眾條載羅。時有默語,運固隆汙。在我中晉,業融長沙。桓桓長沙,伊勳伊德。天子疇我,專征南國。



 功遂辭歸,臨寵不惑。孰謂斯心,而可近得。肅矣我祖,慎終如始。直方二臺,惠和千里。於皇仁考,淡焉虛止。寄跡夙運,冥茲慍喜。嗟餘寡陋,瞻望靡及。顧慚華鬢,負景只立。三千之罪,無後其急。
 我誠念哉,呱聞爾泣。卜云嘉日,占爾良時。名爾曰儼,字爾求思。溫恭朝夕,念茲在茲。尚想孔伋,庶其企而。厲夜生子,遽而求火。凡百有心,奚待于我。既見其生,實欲其可。人亦有言,斯情無假。日居月諸,漸免于孩。福不虛至,禍亦易來。夙興夜寐,願爾斯才。爾之不才,亦已焉哉。



 潛元嘉四年卒,時年六十三。



 宗彧之,字叔粲,南陽涅陽人,炳從父弟也。蚤孤,事兄恭謹,家貧好學,雖文義不逮炳,而真澹過之。州辟主簿,舉
 秀才,不就。公私餼遺,一無所受。高祖受禪,徵著作佐郎,不至。元嘉初,大使陸子真觀采風俗,三詣彧之,每辭疾不見也。告人曰:「我布衣草萊之人,少長壟畝,何枉軒冕之客。」子真還,表薦之,徵員外散騎侍郎,又不就。元嘉八年,卒,時年五十。



 沈道虔,吳興武康人也。少仁愛,好《老》、《易》,居縣北石山下。孫恩亂後饑荒,縣令庾肅之迎出縣南廢頭里,為立小宅,臨溪,有山水之玩。時復還石山精廬,與諸孤兄子共
 釜庾之資,困不改節。受琴於戴逵,王敬弘深敬之。郡州府凡十二命,皆不就。



 有人竊其園萊者,還見之,乃自逃隱,待竊者取足去後乃出。人拔其屋後筍,令人止之,曰:「惜此筍欲令成林,更有佳者相與。」乃令人買大筍送與之。盜者慚不取,道虔使置其門內而還。常以捃拾自資,同捃者爭穟,道虔諫之不止,悉以其所得與之,爭者愧恧。後每爭,輒云:「勿令居士知。」冬月無復衣,戴顒聞而迎之,為作衣服,并與錢一萬。既還,分身上衣及錢,悉供諸
 兄弟子無衣者。鄉里年少,相率受學。道虔常無食,無以立學徒。武康令孔欣之厚相資給,受業者咸得有成。太祖聞之,遣使存問,賜錢三萬,米二百斛,悉以嫁娶孤兄子。征員外散騎侍郎,不就。累世事佛,推父祖舊宅為寺。至四月八日,每請像。請像之日,輒舉家感慟焉。道虔年老,菜食,恆無經日之資,而琴書為樂,孜孜不倦。太祖敕郡縣令,隨時資給。元嘉二十六年,卒,時年八十二。子慧鋒,脩父業,辟從事,皆不就。



 郭希林,武昌武昌人也。曾祖翻,晉世高尚不仕。希林少守家業,徵州主簿,秀才,衛軍參軍,並不就。元嘉初,吏部尚書王敬弘舉王弘之為太子庶子,希林為著作佐郎。後又徵員外散騎侍郎,並不就。十年,卒,時年四十七。子蒙,亦隱居不仕。泰始中,郢州刺史蔡興宗辟為主簿,不就。



 雷次宗,字仲倫,豫章南昌人也。少入廬山,事沙門釋慧遠,篤志好學,尤明《三禮》、《毛詩》,隱退不交世務。本州辟從事,員外散騎侍郎征,並不就。與子侄書以言所守,曰:夫
 生之脩短,咸有定分,定分之外,不可以智力求,但當於所稟之中,順而勿率耳。吾少嬰羸患,事鐘養疾,為性好閑,志棲物表,故雖在童稚之年,已懷遠迹之意。暨于弱冠,遂託業廬山,逮事釋和尚。于時師友淵源,務訓弘道,外慕等夷,內懷悱發,於是洗氣神明,玩心墳典,勉志勤躬,夜以繼日。爰有山水之好,悟言之歡,實足以通理輔性,成夫亹亹之業,樂以忘憂,不知朝日之晏矣。自游道餐風,二十餘載,淵匠既傾,良朋凋索,續以釁逆違天,備
 嘗荼蓼,疇昔誠願,頓盡一朝,心慮荒散,情意衰損,故遂與汝曹歸耕壟畔,山居谷飲,人理久絕。



 日月不處,忽復十年,犬馬之齒,已踰知命。崦嵫將迫,前塗幾何,實遠想尚子五岳之舉,近謝居室瑣瑣之勤。及今耄未至惛,衰不及頓,尚可厲志於所期,縱心於所託,棲誠來生之津梁,專氣莫年之攝養,玩歲日於良辰,偷餘樂於將除,在心所期,盡於此矣。汝等年各成長,冠娶已畢,脩惜衡泌,吾復何憂。但顧守全所志,以保令終耳。自今以往,家事
 大小,一勿見關,子平之言,可以為法。



 元嘉十五年,征次宗至京師,開館於雞籠山,聚徒教授,置生百餘人。會稽朱膺之、潁川庾蔚之並以儒學,監總諸生。時國子學未立,上留心藝術,使丹陽尹何尚之立玄學,太子率更令何承天立史學,司徒參軍謝元立文學,凡四學並建。車駕數幸次宗學館,資給甚厚。又除給事中,不就。久之,還廬山,公卿以下,並設祖道。



 二十五年,詔曰:「前新除給事中雷次宗,篤尚希古,經行明脩,自絕招命,守志隱約。宜
 加升引,以旌退素。可散騎侍郎。」後又徵詣京邑,為築室於鐘山西巖下,謂之招隱館,使為皇太子諸王講《喪服》經。次宗不入公門,乃使自華林東門入延賢堂就業。二十五年,卒於鐘山,時年六十三。太祖與江夏王義恭書道次宗亡,義恭答曰:「雷次宗不救所疾,甚可痛念。其幽棲窮藪,自賓聖朝,克己復禮,始終若一。伏惟天慈弘被,亦垂矜愍。」子肅之,頗傳其業,官至豫章郡丞。



 朱百年,會稽山陰人也。祖愷之,晉右衛將軍。父濤,揚州
 主簿。百年少有高情,親亡服闋,攜妻孔氏入會稽南山,以伐樵採箬為業。每以樵箬置道頭,輒為行人所取,明旦亦復如此。人稍怪之,積久方知是朱隱士所賣,須者隨其所堪多少,留錢取樵箬而去。或遇寒雪,樵箬不售,無以自資,輒自搒船送妻還孔氏,天晴復迎之。有時出山陰為妻買繒彩三五尺,好飲酒,遇醉或失之。頗能言理,時為詩詠,往往有高勝之言。郡命功曹,州辟從事,舉秀才,並不就。隱跡避人,唯與同縣孔覬友善。覬亦嗜酒,
 相得輒酣,對飲盡懽。百年家素貧,母以冬月亡,衣並無絮,自此不衣綿帛。嘗寒時就覬宿,衣悉夾布,飲酒醉眠,覬以臥具覆之,百年不覺也。



 既覺,引臥具去體,謂覬曰:「綿定奇溫。」因流涕悲慟,覬亦為之傷感。



 除太子舍人,不就。顏竣為東揚州,發教餉百年穀五百斛,不受。時山陰又有寒人姚吟,亦有高趣,為衣冠所重。義陽王昶臨州,辟為文學從事,不起。竣餉吟米二百斛,吟亦辭之。百年孝建元年卒山中,時年八十七。蔡興宗為會稽太守,餉百
 年妻米百斛,百年妻遣婢詣郡門奉辭固讓,時人美之,以比梁鴻妻。



 王素,字休業,琅邪臨沂人也。高祖翹之,晉光祿大夫。素少有志行,家貧母老。初為廬陵國侍郎,母憂去職。服闋,廬陵王紹為江州,親舊勸素修完舊居,素不答,乃輕身往東陽,隱居不仕,頗營田園之資,得以自立。愛好文義,不以人俗累懷。世祖即位,欲搜揚隱退,下詔曰:「濟世成務,咸達隱微,軌俗興讓,必表清節。朕昧旦求善,思惇薄
 風,琅邪王素、會稽朱百年,并廉約貞遠,與物無競,自足皋畝,志在不移。宜加褒引,以光難進。並可太子舍子。」大明中,太宰江夏王義恭開府辟召,辟素為倉曹屬;太宗泰始六年,又召為太子中舍人,並不就。素既屢被徵辟,聲譽甚高。山中有蚿蟲,聲清長,聽之使人不厭,而其形甚醜,素乃為《蚿賦》以自況。七年,卒,時年五十四。



 時又有宋平劉睦之、汝南州韶、吳郡褚伯玉,亦隱身求志。睦之居交州,除武平太守,不拜。韶字伯和,黃門侍郎文孫也。
 築室湖孰之方山,徵員外散騎侍郎,征北行參軍,不起。伯玉居剡縣瀑布山三十餘載,揚州辟議曹從事,不就。



 關康之,字伯愉,河東楊人。世居京口,寓屬南平昌。少而篤學,姿狀豐偉。



 下邳趙繹以文義見稱,康之與之友善。特進顏延之見而知之。晉陵顧悅之難王弼《易》義四十餘條,康之申王難顧,遠有情理。又為《毛詩義》,經籍疑滯,多所論釋。嘗就沙門支僧納學,妙盡其能。竟陵王義宣自京口遷鎮江陵,要康之同行,距不應命。元嘉中,太祖
 聞康之有學義,除武昌國中軍將軍,蠲除租稅。江夏王義恭、廣陵王誕臨南徐州,辟為從事、西曹,並不就。棄絕人事,守志閑居。弟雙之為臧質車騎參軍,與質俱下,至赭圻病卒,瘞於水濱。康之其春得疾困篤,小差,牽以迎喪,因得虛勞病,寢頓二十餘年。時有閑日,輒臥論文義。世祖即位,遣大使陸子真巡行天下,使反,薦康之「業履恒貞,操勖清固,行信閭黨,譽延邦邑,棲志希古,操不可渝,宜加徵聘,以潔風軌。」不見省。太宗泰始初,與平原明
 僧紹俱徵為通直郎,又辭以疾。順帝昇明元年,卒,時年六十三。



 史臣曰:夫獨往之人,皆稟偏介之性,不能摧志屈道,借譽期通。若使值見信之主,逢時來之運,豈其放情江海,取逸丘樊。蓋不得已而然故也。且巖壑閑遠,水石清華,雖復崇門八襲,高城萬雉,莫不蓄壤開泉,仿佛林澤。故知松山桂渚,非止素玩,碧澗清潭,翻成麗矚。掛冠東都,夫何難之有哉!



\end{pinyinscope}