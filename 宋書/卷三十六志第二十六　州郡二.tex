\article{卷三十六志第二十六 州郡二}

\begin{pinyinscope}

 南豫
 州豫州江州青州冀州司州南豫州刺史,晉江左胡寇強盛,豫部殲覆,元帝永昌元年,刺史祖約始自譙城退還壽春。
 成帝咸和四年,僑立豫州,庾亮為刺史,治蕪湖。咸康四年,毛寶為刺史,治邾城。六年,荊州刺史庾翼鎮武昌,領豫州。八年,庾懌為刺史,又鎮蕪湖。



 穆帝永和元年,刺史趙胤鎮牛渚。二年,刺史謝尚鎮蕪湖;四年,進壽春;九年,尚又鎮歷陽;十一年,進馬頭。升平元年,刺史謝奕戍譙。哀帝隆和元年,刺史袁真自譙退守壽春。簡文
 咸安元年,刺〔史桓熙戍歷陽。孝武寧康元年,刺〕史桓沖戍姑孰。太元十年,刺史硃序戍馬頭。十二年,刺史桓石虔戍歷陽。安帝義熙二年,刺史劉毅戍姑孰。宋武帝欲開拓河南,綏定豫土,九年,割揚州大江以西、大雷以北,悉屬豫州,豫基址因此而立。十三年,刺史劉義慶鎮壽陽。永初三年,分淮東為南豫州,治歷陽;淮西為豫州。文帝元嘉七年〔合二豫州為一,十六年又分,二十二年又合,考武大明三年〕,又分。五年,割揚州之淮南、宣城又屬焉。徙治姑孰。明帝泰始二年又合,而以淮南、宣城還揚州。九月
 又分,還治歷陽。三年五月,又
 合。四年,以揚
 州之淮南、宣城為南豫州,治宣城,五年罷。時自淮以西,悉沒寇矣。七年,復分歷陽、淮陰、南譙、南兗州之臨江立南豫
 州。
 泰豫元年,以南汝陰度屬豫州,豫州之廬江度屬南豫州。按淮東自永初至于大明,便為南豫,雖乍有離合,而分立居多。爰自泰始甫失淮西,復於淮東分立兩豫。今南豫以
 淮東為境,不復于此更列二州,覽者按此以淮東為境,推尋便自得泰始兩豫分域也。徐志領郡十三,縣六十一,戶三萬七千六百二,口二十一萬九千五百。今領郡十九,縣九十一。去京都水一百六十。



 歷陽太守,晉惠帝永興元年,分淮南立,屬揚州,安帝割屬豫州。《永初郡國》唯有歷陽、烏江、龍亢三縣,何、徐又有酂、雍丘二縣。今領縣五,戶三千一百五十六,口一萬九千四百七十。



 歷陽令,漢舊縣,屬九江。



 烏江令,二漢無,《晉書》有烏江,《太康地志》屬準南。



 龍亢令,漢舊名,屬沛郡,《晉太康地志》屬譙。江左流寓立。



 雍丘令,漢舊名,屬陳留。流寓立,先屬泰山郡,文帝元嘉八年度。



 酂令,漢屬沛,《晉太康地志》屬譙。流寓立,文帝元嘉八年度。


南譙太守
 \gezhu{
  譙郡別見}
 ,晉孝武太元中,於淮南僑立郡縣,後割地志咸實土。



 《郡國》又有酂縣,何、徐無。今領縣六,戶四千四百三十二,口二萬二千三百五十八。去州水五百四十,陸一百七十;去京都水七百,陸五百。



 山桑令,前漢屬沛,後漢屬汝南,《晉太康地志》屬譙。



 譙令,漢屬沛,《晉太康地志》屬譙。



 銍令,漢屬沛,《晉太康地志》屬譙。



 扶陽令,前漢屬沛,後漢、《晉太康地志》並無。


蘄令。
 \gezhu{
  別見}



 城父令,前漢屬沛,後漢屬汝南,《晉太康地志》屬譙。



 廬江太守,漢文帝十六年,分淮南國立。光武建武十三年,又省六安國以併焉。



 領縣三,戶一千九百九,口一萬一千九百九十七。去州水二千七百二十,陸四百七十;去京都水一千一百,陸六百三十一。



 灊令,漢舊縣。



 舒令,漢舊縣。



 始新令,《永初郡國》、何並無,徐有始新左縣,明帝泰始三年立。


南汝陰太守
 \gezhu{
  汝陰郡別見}
 ,江左立。領縣五,戶二千七百一,口一萬九千五百八十五。去州陸三百;去京都水一千,陸五百三十。


汝陰令
 \gezhu{
  別見}
 ,所治即二漢、晉合肥縣,後省。



 慎令,漢屬汝南,《晉太康地志》屬汝陰。


宋令。
 \gezhu{
  別見}



 陽夏令,前漢屬淮陽,後漢屬陳。《晉太康地志》陳令屬梁,無復此縣。又晉地志,惠帝永康中復立。《永初郡國》、何並屬南梁,徐志屬此。


安陽令
 \gezhu{
  別見}
 ,《永初郡國》、何並屬南梁,徐屬此。


南梁太守
 \gezhu{
  梁郡別見}
 ,晉孝武太元中,僑立於淮南,安帝始有淮南故地,屬徐州。武帝永初二年,還南豫,孝武大明六年廢屬西豫,改名淮南,八年復舊。


《永初郡國》又有虞、陽夏、安豐三縣
 \gezhu{
  並別見}
 。何、徐無安豐;又有義昌而並無寧陵
 縣。今領縣九,戶六千二百一十二,口四萬二千七百五十四。去州水一千八百,陸五百;去京都水一千七百,陸七百。



 睢陽令,漢舊名。孝武大明六年,改名壽春,八年復舊。前廢帝永光有義寧、寧昌二縣併睢陽。所治即二漢、晉壽春縣,後省。


蒙令。
 \gezhu{
  別見}



 虞令,漢舊名。



 穀熟令,漢舊名。



 陳令,前漢屬淮陽,後漢屬陳,《晉太康地志》屬梁。



 義寧長,何無,徐有,宋末又立。



 新汲令,漢舊名,屬潁川。



 崇義令,《永初郡國》羌人始立。


寧陵
 \gezhu{
  別見}
 ,徐志後所立。



 晉熙太守,晉安帝分廬江立。領縣五,戶一千五百二十一,口七千四百九十七。



 去州陸八百,無水;去京都水一
 千二百,無陸。



 懷寧令,晉安帝立。



 新冶令,晉安帝立。



 陰安令,漢舊名,屬魏郡,《晉太康地志》屬頓丘。



 南樓煩令,《永初郡國》、何、徐志無。



 太湖左縣長,文帝元嘉二十五年,以豫部蠻民立太湖、呂亭二縣,屬晉熙,後省,明帝太始二年復立。


弋陽太守,本縣名,屬汝南,魏文帝分立。領縣六,戶三千
 二百七十五,口二萬四千二百六十二。去州陸一千一百,去京都水
 \gezhu{
  闕}
 。



 期思令,漢舊縣。



 弋陽令,漢舊縣。



 安豐令,舊郡,晉安帝並為縣。



 樂安令,新立。



 茹由令,新立。



 安豐太守,魏文帝分廬江立。江左僑立,晉安帝省為縣,
 屬弋陽,宋末復立。



 安豐令,《前漢地理志》無,後漢屬廬江。


松滋令。
 \gezhu{
  別見}


汝南太守。
 \gezhu{
  別見}


上蔡侯相。
 \gezhu{
  別見}


平輿令。
 \gezhu{
  別見}


北新息令。
 \gezhu{
  別見}


真陽令。
 \gezhu{
  別
  見}


安城令。
 \gezhu{
  別見}


南新息令。
 \gezhu{
  別見}


臨汝令,漢舊名。
 \gezhu{
  別見}


陽安令。
 \gezhu{
  別見}


西平令。
 \gezhu{
  別見}


瞿陽令。
 \gezhu{
  別見}


安陽令。
 \gezhu{
  別見}


新蔡太守。
 \gezhu{
  別見}


鮦陽令。
 \gezhu{
  別見}


固始令。
 \gezhu{
  別見}


新蔡令。
 \gezhu{
  別見}


東苞信令。
 \gezhu{
  別見}



 西苞信令,徐志南豫唯一苞信,疑是後僑立所分。


東郡太守
 \gezhu{
  別見}
 ,《永初郡國》無萇平、父陽而有扶溝
 \gezhu{
  別見}
 ;何無陽夏、扶溝,徐無陽夏。


項城令。
 \gezhu{
  別見}


西華令。
 \gezhu{
  別見}


陽夏令。
 \gezhu{
  別見}


萇平令。
 \gezhu{
  別見}


穀陽令。
 \gezhu{
  別見}


南頓太守
 \gezhu{
  別見}
 ,帖治陳郡。


南頓令。
 \gezhu{
  別見}


和城令。
 \gezhu{
  別見}


潁川太守。
 \gezhu{
  別見}


邵陵令。
 \gezhu{
  別見}


臨潁令。
 \gezhu{
  別見}


曲陽令。
 \gezhu{
  別見}



 西汝陰太守,《永初郡國》、何、徐並無此郡。


汝陰令。
 \gezhu{
  別見}


安城令。
 \gezhu{
  別見}


樓煩令。
 \gezhu{
  別見}


宋令。
 \gezhu{
  別見}


汝陽太守。
 \gezhu{
  別見}


汝陽令。
 \gezhu{
  別見}



 武津令。


陳留太守
 \gezhu{
  別見}
 ,《永初郡國》無浚儀、封丘,而有酸棗,何、徐無封丘、尉氏。


浚儀令。
 \gezhu{
  別見}


小黃令。
 \gezhu{
  別見}


雍丘令。
 \gezhu{
  別見}


白馬令。
 \gezhu{
  別見}


襄邑令。
 \gezhu{
  別見}



 封丘令,漢舊名。


尉氏令。
 \gezhu{
  別見}


南陳左郡太守,少帝景平中省此郡,以宋民度屬南梁、汝陰郡,而《永初郡國》無,未詳。孝建二年以蠻戶復立。分赤官左縣為蓼城左縣。領縣二。樂
 \gezhu{
  疑}
 大明八年,省郡,即名為縣,屬陳左縣。



 邊城左郡太守,文帝元嘉二十五年,以豫部蠻民立茹由、樂安、光城、雩婁、史水、開化、邊城七縣,屬弋陽郡。徐志有邊城郡,領雩婁、史水、開化、邊城兩縣。大明八年,復省為縣,屬弋陽,後復立。領縣四,戶四百一十七,口二千四百七十九。



 雩婁令,二漢屬廬江,《晉太康地志》云屬安豐。



 開化令。



 史水令。



 邊城令。



 光城左郡太守,《永初郡國》、何、徐並無。按《起居注》,大明八年,省光城左郡為縣,屬弋陽,疑是大明中分弋陽所立。八年復省,後復立。



 樂安令。



 茹由令。



 光城令。此三縣,徐志屬弋陽。



 豫州刺史,後漢治譙,魏治汝南安成,晉平吳後治陳國,
 晉江左所治,已列於前。《永初郡國》、何、徐寄治睢陽,而郡縣在淮西。徐又有邊城,別見南豫州。



 何又有初安、綏城二郡,初安領新懷、懷德二縣,綏城領安昌、招遠二縣,並云新立。徐無,則是徐志前省也。領郡十,縣四十三,戶二萬二千九百一十九,口一十五萬八百三十九。



 汝南太守,漢高帝立。領縣十一,戶一萬一千二百九十一,口八萬九千三百四十九。去州水一千,陸七百;去京都水三千,陸一千五百。



 上蔡令,漢舊縣。



 平樂令,漢舊縣。



 北新息令,漢舊縣。



 慎陽令,漢舊縣。《永初郡國》及徐並作真陽。



 安成令,漢舊縣。



 南新息令,漢舊縣。



 朗陵令,漢舊縣。



 陽安令,漢舊縣。



 西平令,漢舊縣。



 瞿陽令,漢舊縣,作灈陽。



 安陽令,漢舊縣。晉武太康元年,改為南安陽。



 新蔡太守,晉惠帝分汝陰立,今帖治汝南。領縣四,戶二千七百七十四,口一萬九千八百八十。去州陸六百;去京都水二千五百,陸一千四百。



 鮦陽令,漢舊縣。晉成帝咸康二年,省併新蔡,後又
 立。



 固始令,故名寢丘之地也。漢光武更名。晉成帝咸康二年,併新蔡,後又立。



 新蔡令,漢舊縣。



 苞信令,前漢無,後漢屬汝南,《晉太康地志》屬汝陰。後漢《郡國》、《晉太康地志》並作「褒」。



 譙郡太守,何志故屬沛,魏明帝分立。按王粲詩:「既入譙郡界,曠然消人憂。」



 粲是建安中亡,非明帝時立明矣。《永初郡國》無長垣縣。今領縣六,戶一千四百二十四,口七千
 四百四。去州陸道三百五十;去京都水二千,陸一千二百。



 蒙令,漢舊縣,屬沛。



 蘄令,漢舊縣,屬沛。



 寧陵令,前漢屬陳留,後漢、《晉太康地志》屬梁。



 魏令,故魏郡,流寓配屬。



 襄邑令。



 長垣令,漢舊縣,屬陳留。《永初郡國》無。何故屬陳留,
 徐新配。



 梁郡太守,秦碭郡,漢高更名。孝武大明元年度徐州,二年還豫。領縣二,戶九百六十八,口五千五百。去州陸一百六十;去京都水九百。



 下邑令,漢舊縣。何云魏立,非也。



 碭令,漢舊縣。



 陳郡太守,漢高立為淮陽國,章帝元和三年更名。晉初併,梁王肜薨,還為陳。


《永初郡國》有扶溝
 \gezhu{
  前漢屬淮陽,後漢、《晉
  太康地志》屬陳留。}
 、陽夏
 \gezhu{
  別見}
 ,而無穀陽、長平。領縣四,戶六百九十三,口四千一百一十三。去州陸七百六十;去京都水一千四百五十。



 項城令,漢舊縣,屬汝南,《晉太康地志》屬陳郡。



 西華令,漢舊縣,屬汝南,晉初省,惠帝永康元年復立,屬潁川。江左度此。



 穀陽令,本苦縣,前漢舊淮陽,後漢屬陳,《晉太康地志》屬梁,成帝咸康三年更名
 長平令,前漢屬汝南,後漢屬陳,《晉太康地志》屬潁川。



 南潁太守,故屬汝南,晉惠帝分立。領縣二,戶五百二十六,口二千三百六十五。去州七百六十;去京都陸一千四百五十。



 南頓令,漢舊縣,何故屬汝陽,晉武帝改屬汝南。按《晉太康地志》、王隱《地道》無汝陽郡。



 和城令,何江左立。



 那個傻瓜愛過你潁川太守,秦立。魏分潁川為襄城郡,晉成帝咸康二年,省襄城還併潁川。


《永初郡國》又有許昌
 \gezhu{
  本名許,漢舊縣。魏曰許昌}
 、新汲
 \gezhu{
  別見}
 、𨻳陵、長社、潁陰、陽翟
 \gezhu{
  四縣並漢舊縣。陽翟,魏、晉屬河南。}
 六縣,而無曲陽。領縣三,戶六百四十九,口三千五百七十九。去州一千;去京都陸一千八百。



 邵陵令,漢舊縣,屬汝南,《晉太康地志》屬潁川。



 臨潁令,漢舊縣。



 曲陽令,前漢屬東海,後漢屬下邳,《晉太康地志》無。



 汝陽太守,《晉太康地志》、王隱《地道》無此郡,應是江左分汝南立。晉成帝咸康三年,省併汝南,後又立。領縣二,戶九百四十一,口四千四百九十五。去州二百;去京都陸一千四百,水三千五百。



 汝陽令,漢舊縣,屬汝南。何故屬汝陰,晉武改屬汝南。按晉武分汝南為汝陰,何所言非也。



 武津令,何不注置立。



 汝陰太守,晉武帝分汝南立,成帝咸康二年,省併新蔡,
 後復立。領縣四,戶二千七百四十九,口一萬四千三百三十五。



 汝陰令,漢舊縣。



 宋令,前漢名新郪。章帝建初四年,徙宋公國於此,改曰宋。



 宋城令,漢舊縣。



 樓煩令,漢舊縣,屬雁門。流寓配屬。


陳留太守,漢武帝元狩元年立,屬兗州,中原亂廢。晉成
 帝咸康四年復立,《永初郡國》屬兗州,何、徐屬豫州。《永初郡國》無浚儀,有酸棗
 \gezhu{
  別見}
 。今領縣四,戶百九十六,口二千四百一十三。寄治譙郡長垣縣界。



 浚儀令,漢舊名。



 小黃令,漢舊名。



 白馬令,漢屬東郡,《晉太康地志》屬濮陽。



 雍丘令,漢舊名。



 江州刺史,晉惠帝元康元年,分揚州之豫章、鄱陽、廬陵、
 臨川、南康、建安、晉安,荊州之武昌、桂陽、安成十郡為江州。初治豫章,成帝咸康六年,移治尋陽;庾翼又治豫章,尋還尋陽。領郡九,縣六十五,戶五萬二千三十三,口二十七萬七千一百四十七。去京都水一千四百。



 尋陽太守,尋陽本縣名,因水名縣,水南注江。二漢屬廬江,吳立蘄春郡,尋陽縣屬焉。晉武帝太康元年,省蘄春郡,以尋陽屬武昌,改蘄春之安豐為高陵及邾縣,皆屬武昌。二年,以武昌之尋陽復屬廬江郡。惠帝永興元年,
 分廬江、武昌立尋陽郡。尋陽縣後省。領縣三,戶二千七百二十,口一萬六千八。



 柴桑男相,二漢屬豫章,晉屬武昌。郡既立,治此。鼓澤子相,漢、《晉太康地志》屬豫章,立尋陽郡後,割度。



 松滋伯相,前漢屬廬江,後漢無,《晉太康地志》屬安豐。安豐縣名,前漢無,後漢屬廬江,晉武帝立為安豐郡。江左流民寓尋陽,僑立安豐、松滋二郡,遙隸
 揚州,安帝省為松滋縣。尋陽又有弘農縣流寓。文帝元嘉十八年,省併松滋。


豫章太守,漢高帝立,本屬揚州。《永初郡國》有海昏
 \gezhu{
  漢舊縣}
 ,何志無。



 今領縣十二,戶一萬六千一百三十九,口一十二萬二千五百七十三。去州水六百,陸三百五十;去京都水一千九百,陸二千一百。



 南昌侯相,漢舊縣。



 新淦侯相,漢舊縣。



 豐城侯相,吳立曰富城,晉武帝太康元年更名。



 建城侯相,漢舊縣。



 望蔡子相,漢靈帝中平中,汝南上蔡民分徙此地,立縣名曰上蔡;晉武帝太康元年更名。



 吳平侯相,漢靈帝中平中立漢平,吳更名。



 永脩男相,漢靈帝中平中立。



 建昌公相,漢和帝永元十六年,分海昏立。



 豫寧侯相,漢獻帝建安中立,吳曰西安,晉武帝太
 康元年更名。



 康樂侯相,吳孫權黃武中立,曰陽樂,晉武帝太康元年更名。



 新吳令,漢靈帝中平中立。



 艾侯相,漢舊縣。


鄱陽太守,漢獻帝建安十五年,孫權分豫章立,治鄱陽縣;赤烏八年,徙治吳芮故城。《永初郡國》有歷陵縣
 \gezhu{
  漢舊縣}
 ,何志無。領縣六,戶三千二百四十二,口一萬九百五十。
 去州水四百四十;去京都水一千八百四十,陸二千六十。



 廣晉令,吳立曰廣昌,晉武帝太康元年更名。



 鄱陽侯相,漢舊縣。



 餘干令,漢舊縣。



 上饒男相,吳立。《太康地志》有,王隱《地道》無。



 葛陽令,吳立。



 樂安男相,吳立。



 臨川內史,吳孫亮太平二年,分豫章東部都尉立。領縣九,戶八千九百八十三,口六萬四千八百五。去州水一千一百,陸一千二十;去京都水二千八百三十,陸三千。



 臨汝侯相,漢和帝永元八年立。



 西豐侯相,吳立曰西平,晉武帝太康元年更名。



 新建侯相,吳立。



 永城男相,吳立。



 宜黃侯相,吳立。



 南城男相,漢舊縣,晉武帝太康元年,更曰新南城,江左復舊。



 南豐令,吳立。



 東興侯相,吳立。



 安浦男相,吳立。



 廬陵太守,廬陵本縣名,屬豫章,漢獻帝興平元年,孫策分豫章立。領縣九,戶四千四百五十五,口三萬一千二百七十一。去州水二千,陸一千六百;去京都水三千六
 百。



 石陽子相,前漢無,後漢有。



 西昌侯相,吳立。



 東昌子相,吳立。



 吉陽男相,吳立。



 己丘男相,吳立。



 興平侯相,吳立。



 陽豐男相,吳曰陽城,晉武帝太康元年更名。



 高昌男相,吳立。



 遂興男相,吳立曰新興,晉武帝太康元年更名。《永初郡國》無比縣,何、徐並有。



 安成太守,孫皓寶鼎二年,分豫章、廬陵、長沙立。《晉太康地志》屬荊州。



 領縣七,戶六千一百一十六,口五萬三百二十三。去州水三千三百,陸三千六百;去京都水三千七百,無陸。



 平都子相,前漢曰安平,後漢更名,屬豫章。



 新喻侯相,吳立。



 宜陽子相,漢舊縣,本名宜春,屬豫章,晉孝武改名。



 永新男相,吳立。



 安復侯相,漢舊縣,本名安成,晉武帝太康元年更名,屬長沙。



 萍鄉侯相,吳立。



 廣興侯相,《晉太康地志》有此縣,何云江左立,非也。



 南康公相,晉武帝太康三年,以廬陵南部都尉立。領縣
 七,戶四千四百九十三,口三萬四千六百八十四。去州水三千七百四十;去京都水三千八十。



 贛侯相,漢舊縣,屬豫章。



 寧都子相,吳立曰楊都,晉武帝太康元年更名。



 雩都侯相,漢舊縣,屬豫章。



 平固侯相,吳立曰平陽,晉武帝太康元年更名。



 南康公相,吳立曰安南,晉武帝太康元年更名。



 陂陽男相,吳立曰揭陽,晉武帝太康五年,以西康
 揭陽移治故陂陽縣,改曰陂縣,然則陂陽先已為縣矣。後漢《郡國》無,疑是吳所立而改曰揭陽也。



 南野伯相,漢舊縣,屬豫章。



 虔化男相,孝武大明五年,以虔化屯立。



 南新蔡太守,江左立。領縣四,戶一千七百三十,口八千八百四十八。去州水二百;去京都水一千三百七十,陸一千八百八十。


苞信令
 \gezhu{
  別見}
 ,本作褒信,《永初郡國》作苞信。



 慎令,漢舊名,本屬汝南。


宋令
 \gezhu{
  別見}
 ,徐志云宋樂,後復舊。



 陽唐左縣令,孝武大明八年立。



 建安太守,本閩越,秦立為閩中郡。漢武帝世,閩越反,滅之,徙其民於江、淮間,虛其地。後有遁逃山谷者頗出,立為冶縣,屬會稽。司馬彪云,章安是故冶,然則臨海亦冶地也。張勃《吳錄》云:「閩越王冶鑄地,故曰安閩王冶。此不
 應偏以受名,蓋句踐冶鑄之所,故謂之冶乎?閩中有山名湛,疑湛山之爐鑄劍為湛爐也。」


後分冶地為會稽東、南二部都尉。東部,臨海是也;南部,建安是也。吳孫休永安三年,分南部立為建安郡。領縣七。
 \gezhu{
  疑}
 戶三千四十二,口一萬七千六百八十六。



 去州水二千三百八十;去京都水三千四十,並無陸。



 吳興子相,漢末立曰漢興,吳更名。



 將樂子相,《晉太康地志》有。



 邵武子相,吳立曰昭武,晉武帝更名。



 建陽男相,《晉太康地志》有。



 綏成男相,《永初郡國》、何、徐並有;何、徐不注置立。



 沙村長,《永初郡國》、何、徐並有;何、徐不注置立。



 晉安太守,晉武帝太康三年,分建安立。領縣五,戶二千八百四十三,口一萬九千八百三十八。去州水三千九百九十;去京都水三千五百八十。



 侯官囗相,前漢無,後漢曰東侯官,屬會稽。



 原豐令,晉武帝太康三年,省建安典船校尉立。



 晉安男相,吳立曰東安,晉武帝更名。



 羅江男相,吳立,屬臨海。晉武帝立晉安郡,度屬。



 溫麻令,晉武帝太康四年,以溫麻船屯立。《永初郡國》無,何、徐並有。



 青州刺史,治臨淄。江左僑立,治廣陵。安帝義熙五年,平廣固,北青州刺史治東陽城,而僑立南青州如故。後省南青州,而北青州直曰青州。孝武孝建二年,移治歷城。
 大明八年,還治東陽。明帝失淮北,於鬱洲僑立青州,立齊、北海、西海郡。舊州領郡九,縣四十六,戶四萬五百四,口四十萬二千七百二十九。去京都陸二千。



 齊郡太守,秦立。領縣七,戶七千三百四十六,口萬四千八百八十九。



 臨淄令,漢舊縣。



 西安令,漢舊縣。



 安平令,六國時其地曰安平,二漢、魏、晉曰東安平。
 前漢屬淄川,後漢屬北海,魏度屬齊。



 般陽令,前漢屬濟南,後漢、《晉太康地志》屬齊。



 廣饒令,漢舊縣。



 昌國令,漢舊縣。



 益都令,魏立。


濟南太守,漢文帝十六年,分齊立。晉世濟岷郡,云魏平蜀,徙蜀豪將家於濟、河,故立此郡。安帝義熙中土斷,并濟南。案《晉太康地志》無濟岷郡。《永初郡國》濟南又有祝
 阿
 \gezhu{
  二漢屬平原,《晉太康地志》無。}
 、於陵縣
 \gezhu{
  漢舊縣}
 ,而無朝陽、平陵二縣。領縣六,戶五千五十六,口三萬八千一百七十五。去州陸四百;去京都二千四百。



 歷城令,漢舊縣。



 朝陽令,前漢曰朝陽,後漢、晉曰東朝陽。二漢屬濟南,《晉太康地志》屬樂安。



 著令,漢舊縣。



 土鼓令,漢舊縣,晉無。



 逢陵令,二漢、晉無,《永初郡國》、何、徐有。



 平陵令,漢舊縣,至晉並曰東平陵。



 樂安太守,漢高立,名千乘,和帝永元七年更名。領縣三,戶二千二百五十九,口一萬四千九百九十一。去州陸一百八十;去京都陸一千八百。



 千乘令,漢舊縣。



 臨濟令,前漢曰狄,安帝永初二年更名。



 博昌令,漢舊名。


高密太守,漢文帝分齊為膠西,宣帝本始元年,更名高密。光武建武十三年,併北海,晉惠帝又分城陽立
 \gezhu{
  城陽郡,前漢有,後漢無,魏復分北海立。}
 ;宋孝武併北海。領縣六,戶二千三百四,口一萬三千八百二。去州陸二百;去京都陸一千六百。



 黔陬令,前漢屬琅邪,後漢屬東萊,《晉太康地志》屬城陽。



 淳于令,二漢屬北海,《晉太康地志》屬城陽。



 高密令,前漢屬高密,後漢屬北海,《晉太康地志》屬
 城陽。



 夷安令,前漢屬高密,後漢屬北海,《晉太康地志》屬城陽。



 營陵令,二漢屬北海,《晉太康地志》屬城陽。



 昌安令,漢安帝延光元年立,屬高密,後漢屬北海,《晉太康地志》屬城陽。



 平昌太守,故屬城陽,魏文帝分城陽立,後省,晉惠帝又立。領縣五,戶二千二百七十,口一萬五千五十。去州陸
 二百;去京都陸千七百。



 安丘令,二漢屬北海,《晉太康地志》屬琅邪。



 平昌令,前漢屬琅邪,後漢屬北海,《晉太康地志》屬城陽。



 東武令,二漢屬琅邪,《晉太康地志》屬東莞。



 琅邪令,二漢屬琅邪,《晉太康地志》無。



 硃虛令,前漢屬琅邪,安帝永初元年屬北海,《晉太康地志》屬城陽。



 北海太守,漢景帝中二年立。領縣六,戶三千九百六十八,口三萬五千九百九十五。寄治州下。



 都昌令,漢舊縣。寄治州下,餘依本治。



 膠東令,本膠東國,後漢、《晉太康地志》屬北海。



 劇令,二漢屬北海,《晉太康地志》屬琅邪。



 即墨令,前漢屬膠東,後漢、《晉太康地志》屬北海。



 下密令,前漢屬膠東,後漢、《晉太康地志》屬北海。



 平壽令,漢舊縣。



 東萊太守,漢高帝立。領縣七,戶一萬一百三十一,口七萬五千一百四十九。



 去州陸五百;去京都二千一百。



 曲城令,漢舊縣。



 掖令,漢舊縣。



 手弦令,漢舊縣。



 盧鄉令,漢舊縣。



 牟平令,漢舊縣。



 當利令,漢舊縣。



 黃令,漢舊縣。



 太原太守,秦立,屬并州。文帝元嘉十年,割濟南、太山立。領縣三,戶二千七百五十七,口二萬四千六百九十四。去州陸五百;去京都一千八百。



 山茌令,漢舊縣,屬泰山。孝武孝建元年,度濟北。



 太原令,晉安帝義熙中土斷立,屬泰山。


祝阿令。
 \gezhu{
  別見}



 長廣太守,本長廣縣,前漢屬琅邪,後漢屬東萊,《晉太
 康地志》云故屬東萊。



 《起居注》,咸寧三年,以齊東部縣為長廣郡。領縣四,戶二千九百六十六,口二萬二十三。去州五百;去京都一千九百五十。



 不其令,前漢屬琅邪,後漢屬東萊,《晉太康地志》屬長廣。



 長廣令,前漢屬琅邪,後漢屬東萊,《晉太康地志》屬長廣。



 昌陽令,晉惠帝元康八年,分長廣縣立。



 挺令,前漢屬膠東,後漢屬北海,《晉太康地志》屬長廣。



 冀州刺史,江左立南冀州,後省。義熙中更立,治青州,又省。文帝元嘉九年,又分青州立,治歷城,割土置郡縣。領郡九,縣五十,戶三萬八千七十六,口一十八萬一千一。去京都陸二千四百。


廣川太守,本縣名,屬信都,《地理志》不言始立。景帝二年,以為廣川國,宣帝甘露三年復。明帝更名樂安,安帝延
 光中,改曰安平;晉武帝太康五年,又改為長樂。廣川縣,前漢屬信都,後漢屬清河,魏屬勃海,晉還清河。何志,廣川江左所立。又有蓚縣
 \gezhu{
  前漢屬信都,後漢、晉屬勃海。}
 ,而無廣川。孝武大明元年,省廣川之棗強
 \gezhu{
  前漢屬清河,後漢、晉江左無。}
 、勃海之浮陽、高城
 \gezhu{
  並漢舊縣}
 ,立廣川縣,非舊廣川縣也。屬廣川郡。領縣四,戶三千二百五十,口二萬三千六百一十四。去州陸一百六十;去京都陸一千九百八十。


廣川令。
 \gezhu{
  巳前
  見}
 。



 中水令,前漢屬涿,後漢、《晉太康地志》屬河間。孝武大明七年,自河間割度。



 武強令,何江左立。



 索盧令,何江左立。



 平原太守,漢高帝立。舊屬青州,魏、晉屬冀州。領縣八,戶五千九百一十三,口二萬九千二百六十七。



 廣宗令,前漢無,後漢屬鉅鹿;《晉太康地志》屬安平;《永初郡國》、何無;孝武大明元年復立。



 平原令,漢舊縣。



 鬲令,漢舊縣。



 安德令,漢舊縣。



 平昌令,漢舊縣。後漢無。《晉太康地志》曰西平昌。



 般縣令,漢舊縣。



 茌平令,前漢屬東郡,後漢屬濟北,《晉太康地志》屬平原。



 高唐令,漢舊縣。


清河太守,漢立,桓帝建和二年,改曰甘陵,魏復舊。何有重合縣
 \gezhu{
  別見}
 。



 領縣七,戶三千七百九十四,口二萬九千二百七十四。去州一百一十;去京都陸一千八百。



 清河令,二漢無,《晉太康地志》有。



 武城令,漢舊縣,並曰東武城。



 繹幕令,漢舊縣。



 貝丘令,漢舊縣。



 零令,漢舊縣,作靈。



 鄃令,漢舊縣。



 安次令,前漢舊縣,屬勃海,後漢屬廣陽,《晉太康地志》屬燕國。



 樂陵太守,晉武帝分平原立。舊屬青州,今來屬。領縣五,戶三千一百三,口一萬六千六百六十一。去州一百四十;去京都陸一千八百。



 樂陵令,漢舊縣,故屬平原。



 陽信令,二漢屬勃海,《晉太康地志》屬樂陵。


新樂令。
 \gezhu{
  別見}



 厭次令,前漢曰富平,明帝更名,屬平原,《晉太康地志》屬樂陵。



 涇沃令,前漢屬千乘,後漢無。何云魏立,當是魏復立也。《晉太康地志》屬樂陵。



 魏郡太守,漢高帝立。二漢屬冀州,魏、晉屬司隸,江左屢省置;宋孝武又僑立,何無。領縣八,戶六千四百五,口三萬三千六百八十二。



 魏令,漢舊縣。



 安陽令,《晉太康地志》有。



 聊城令,漢屬東郡,晉屬平原。



 博平安,漢屬東郡,晉屬平原。



 肥鄉令,《晉太康地志》屬廣平。



 蠡吾令,前漢屬涿,後漢屬中山,《晉太康地志》屬高陽。孝武始立,屬高陽,大明七年度此。


頓丘令
 \gezhu{
  別見}
 ,文帝元嘉二十八年,流民歸順,孝武孝
 建二年立。



 臨邑令,漢屬東郡,晉屬濟北。孝武孝建二年,與頓丘同立。



 河間太守,漢文帝二年,分趙立。江左屢省置,宋孝武又僑立,何無。領縣六,戶二千七百八十一,口一萬七千七百七。



 樂城令,漢舊縣。



 城平令,前漢屬勃海,後漢、《晉太康地志》屬河間。



 武垣令,前漢屬涿,後漢、《晉太康地志》屬河間。



 章武令,二漢屬勃海,《晉太康地志》屬章武。江左立,屬廣川,孝武大明七年度此。



 南皮令,漢舊縣,屬勃海。孝武始立,屬勃海,大明七年度此。



 阜城令,前漢勃海有阜城縣,《續漢》安平有阜城縣,注云「故昌城」。漢信都有昌城,未詳孰是。


頓丘太守
 \gezhu{
  別見}
 ,江左屢省置,孝武又僑立,何無。領縣四,戶
 一千二百三十八,口三千八百五十一。


頓丘令。
 \gezhu{
  別見}



 衛國令,《晉太康地志》有。



 肥陽令,何志以前無。



 陰安令,二漢屬魏。魏屬陽平,晉屬頓丘。



 高陽太守,高陽,前漢縣名,屬涿,後漢屬河間。晉武帝泰始元年,分涿為范陽,又屬焉。後又分范陽為高陽。江左屢省置,孝武又僑立,何無。領縣五,戶二千二百九十七,
 口一萬四千七百二十五。



 安平令,前漢屬涿,後漢屬安平,《晉太康地志》屬博陵。



 饒陽令,前漢屬涿,《續漢》安平有饒陽縣,注云「故名饒,屬涿。」按《地理》,涿唯有饒陽縣,無饒縣。



 鄴令,漢舊縣,屬魏郡。江左避愍帝諱,改曰臨漳。孝武始立,屬魏郡,大明七年度此。


高陽令。
 \gezhu{
  已見}



 新城令,前漢屬中山,後漢屬涿,《晉太康地志》屬高陽,並曰北新城。



 勃海太守,漢高帝立,屬幽州;後漢、晉屬冀州。江左省置,孝武又僑立,何無。領縣三,戶一千九百五,口萬二千一百六十六。



 長樂令,晉之長樂郡也。疑是江左省為縣,至是又立。


蓚令。
 \gezhu{
  別見。何志屬廣川。徐志屬此。}



 重合令,漢舊縣。


司州刺史,漢之司隸校尉也。晉江左以來,淪沒戎寇,雖永和、太元王化暫及,太和、隆安還復湮陷。牧司之任,示舉大綱而已。縣邑戶口,不可具知。武帝北平關、洛,河南底定,置司州刺史,治虎牢,領河南
 \gezhu{
  漢舊郡}
 、滎陽
 \gezhu{
  晉武帝泰始元年,分河南立。}
 、弘農
 \gezhu{
  漢舊郡}
 實土三郡。河南領洛陽、河南、鞏、緱氏、新城、梁
 \gezhu{
  並漢舊縣}
 、河陰
 \gezhu{
  《晉太康地志》有}
 、陸渾
 \gezhu{
  漢舊縣,屬弘農,《晉太康地志》屬河南。}
 、東垣
 \gezhu{
  二漢、《晉太康地志》、何有垣縣。}
 、新安
 \gezhu{
  二漢屬弘農,《晉太康地志》屬河東。}
 、西東垣
 \gezhu{
  新立}
 凡十一
 縣。滎陽領京、密、滎陽、卷、陽武、苑陵、中牟、開封、成皋
 \gezhu{
  並漢舊縣。屬河南。}
 凡九縣。


弘農領弘農、陜、宜陽、黽池、盧氏
 \gezhu{
  並漢舊縣}
 、曲陽
 \gezhu{
  前漢屬東海,後漢屬下邳,《晉太康地志》無。}
 凡七縣。三郡合二十七縣,一萬六千三百六戶。又有河內
 \gezhu{
  漢舊郡}
 、東京兆
 \gezhu{
  京兆別見雍州,東京兆新立。}
 二僑郡。河內寄治河南,領溫、野王、軹、河陽、沁水、山陽、懷、平皋、
 \gezhu{
  並漢舊名。}
 、朝歌
 \gezhu{
  二漢屬河內,《晉太康地志》屬汲郡。晉武太康元年始立。}
 凡十縣。東京兆寄治滎陽,領長安
 \gezhu{
  漢舊縣}
 、萬年
 \gezhu{
  別見}
 、新豐
 \gezhu{
  別見}
 、藍田
 \gezhu{
  別見}
 、蒲阪
 \gezhu{
  二漢、《晉太康地志》屬河東。}
 凡六縣。合十六縣,一千九百九十二戶。少
 帝景平初,司州復沒北虜。文帝元嘉末,僑立於汝南,尋亦省廢。明帝復於南豫州之義陽郡立司州,漸成實土焉。領郡四,縣二十,去京都水二千七百,陸一千七百。



 義陽太守,魏文帝立,後省,晉武帝又立。《太康地志》、《永初郡國》、何志並屬荊州,徐則南豫也。明帝泰始五年,度郢州,後廢帝元徽四年,屬司州。領縣七。戶八千三十二,口四萬一千五百九十七。



 平陽侯相,前漢無,後漢屬江夏曰平春,《晉太康地
 志》屬義陽,晉孝武改。



 鄳令,二漢屬江夏,《晉太康地志》屬義陽,並作鄳,音盲。《永初郡國》、何並作鄳。



 鐘武令,前漢屬江夏,後漢、《晉太康地志》無,《永初郡國》屬義陽。



 寶城令,孝武孝建三年,分鄳立。



 義陽令,《晉太康地志》有,後省。孝武孝建三年,分平陽立。



 平春令,孝武孝建三年,分平陽立。



 環水長,《永初郡國》、何、徐並無。明帝泰始三年,度屬宋安郡,後省宋安,還此。宋安,本縣名,孝武大明八年,省義陽郡所統東隨二左郡立為宋安縣,屬義陽。明帝立為郡。



 隨陽太守,晉武帝分南陽義陽立義陽國,太康年,又分義陽為隨國,屬荊州。



 孝武孝建元年度屬郢,前廢帝永光元年度屬雍;明帝泰始五年還屬郢,改為隨陽;後廢
 帝元徽四年,度屬司州。徐志又有革音縣,今無。領縣四,戶四千六百。去京都三千四百八十。



 隨陽子相,漢隨縣屬南陽,《晉太康地志》屬義陽。後隨國與郡俱改。



 永陽男相,徐志有。


關西令
 \gezhu{
  別見荊州,作厥西。}
 ,宋末新立。



 西平林令,宋末新立。



 安陸太守,孝武孝建元年,分江夏立,屬郢州;後廢帝元
 徽四年度司州。徐志有安蠻縣,《永初郡國》、何並無,當是何志後所立。尋為郡,孝武大明八年,省為縣,屬安陸;明帝泰始初,又立為左郡,宋末又省。領縣二,戶六千四十三,口二萬五千八十四。去京都水二千三百。



 安陸公相,漢舊縣,屬江夏。江夏又有曲陵縣,本名石陽,吳立。《晉起居注》,太康元年,改江夏石陽曰曲陵;明帝泰始六年,併安陸。


南汝南太守。
 \gezhu{
  汝南郡別見}



 平輿令。



 北新息令。



 真陽令。



 安城令。



 南新息令。


安陽令。
 \gezhu{
  並別見}



 臨汝令,新
 立。



\end{pinyinscope}