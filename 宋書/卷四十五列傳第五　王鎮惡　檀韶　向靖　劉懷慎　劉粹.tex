\article{卷四十五列傳第五 王鎮惡 檀韶 向靖 劉懷慎 劉粹}

\begin{pinyinscope}

 王鎮惡,北海劇人也。祖猛,字景略,苻堅僭號關中,猛為將相,有文武才,北土重之。父休,為河東太守。鎮惡以五月五日生,家人以俗忌,欲令出繼疏宗。



 猛見奇之,曰:「此非常兒,昔孟嘗君惡月生而相齊,是兒亦將興吾門矣!」故名之為鎮
 惡。年十三而苻氏敗亡,關中擾亂,流寓崤、澠之間。嘗寄食澠池人李方家,方善遇之。謂方曰:「若遭遇英雄主,要取萬戶侯,當厚相報。」方答曰:「君丞相孫,人才如此,何患不富貴。至時願見用為本縣令,足矣。」後隨叔父曜歸晉,客居荊州。頗讀諸子兵書,論軍國大事,騎乘非所長,
 關
 弓亦甚弱,而意略縱橫,果決能斷。



 廣固之役,或薦鎮惡於高祖,時鎮惡為天門臨澧令,即遣召之。
 既至與語,甚異焉,因留宿。明旦謂諸佐曰:「鎮惡,王猛之孫,所謂將門有將也。」即以為青州治中從事史,行參中軍太尉軍事,署前部賊曹。拒盧循於查浦,屢戰有功,封博陸縣五等子。高祖謀討劉毅,鎮惡曰:「公若有事西楚,請賜給百舸為前驅。」義熙八年,劉毅有疾,求遣從弟兗州刺史籓為副貳,高祖偽許之。九月,大軍西討,轉鎮惡參軍事,加振武將軍。高祖至姑孰,遣鎮惡率龍驤將軍蒯恩百舸前發,其月二十九日也。戒之曰:「若賊知吾上,
 比軍至,亦當少日耳。政當岸上作軍,未辦便下船也。卿至彼,深加籌量,可擊,便燒其船艦,且浮舸水側,以待吾至。慰勞百姓,宣揚詔旨并赦文、及吾與衛軍府文武書。罪止一人,其餘一無所問。若賊都不知消息,未有備防,可襲便襲。今去,但云劉兗州上。」鎮惡受命,便晝夜兼行,於鵲洲、尋陽、河口、巴陵守風凡四日;十月二十二日,至豫章口,去江陵城二十里。



 自鎮惡進路,揚聲劉兗州上,毅謂為信然,不知見襲。鎮惡自豫章口捨船步上,蒯恩
 軍在前,鎮惡次之。舸留一二人,對舸岸上豎六七旗,下輒安一鼓。語所留人:「計我將至城,便長嚴,令後有大軍狀。」又分隊在後,令燒江津船艦。鎮惡徑前襲城,語前軍:「若有問者,但云劉兗州至。」津戍及百姓皆言劉籓實上,晏然不疑。



 未至城五六里,逢毅要將朱顯之,與十許騎,步從者數十,欲出江津。問是何人,答云:「劉兗州至。」顯之馳前問籓在所,答云:「在後。」顯之既見軍不見籓,而見軍人擔彭排戰具,望見江津船艦已被燒,煙焰張天,而鼓
 嚴之聲甚盛,知非籓上,便躍馬馳去告毅:「外有大軍,似從下上,垂已至城,江津船悉被火燒矣。」



 行令閉諸城門。鎮惡亦馳進,軍人緣城得入,門猶未及下關,因得開大城東門。大城內,毅凡有八隊,帶甲千餘,已得戒嚴。蒯恩入東門,便北回擊射堂,前攻金城東門。鎮惡入東門,便直擊金城西門。軍分攻金城南門,毅金城,內東從舊將,猶有六隊千餘人,西將及能細直吏快手,復有二千餘人。食時就鬥,至中晡,西人退散及歸降略盡。鎮惡入城,便因風放火,
 燒大城南門及東門。又遣人以詔及赦文并高祖手書凡三函示毅,毅皆燒不視。金城內亦未信高祖自來。有王桓者,家住江陵,昔手斬桓謙,為高祖所賞拔,常在左右。求還西迎家,至是率十餘人助鎮惡戰。下晡間,於金城東門北三十步鑿城作一穴,桓便先眾入穴,鎮惡自後繼之,隨者稍多,因短兵接戰。鎮惡軍人與毅東來將士,或有是父兄子弟中表親親者,鎮惡令且鬥且共語,眾並知高祖自來,人情離懈。一更許,聽事前陣散潰,斬毅勇
 將趙蔡。毅左右兵猶閉東西閣拒戰,鎮惡慮暗夜自相傷犯,乃引軍出,繞金城,開其南面,以為退路。毅慮南有伏兵,三更中,率左右三百許人開北門突出。初,毅常所乘馬在城外不得入,倉卒無馬,毅便就子肅民取馬,肅民不與。朱顯之謂曰:「人取汝父,而惜馬不與,汝今自走,欲何之?」奪馬以授毅。初出,政值鎮惡軍,衝之不得去;回衝蒯恩軍,軍人鬥已一日,疲倦,毅得從大城東門出奔牛牧佛寺,自縊死。鎮惡身被五箭,射鎮惡手所執槊,於
 手中破折。江陵平後二十日,大軍方至。



 署中兵,出為安遠護軍、武陵內史。以討劉毅功,封漢壽縣子,食邑五百戶。



 蠻帥向博抵根據阮頭,屢為凶暴,鎮惡討平之。初行,告刺史司馬休之,求遣軍以為聲援,休之遣其將朱襄領眾助鎮惡。會高祖西討休之,鎮惡乃告諸將曰:「百姓皆知官軍已上,硃襄等復是一賊,表裏受敵,吾事敗矣。」乃率軍夜下,江水迅急,倏忽行數百里,直據都尉治。既至,乃以竹籠盛石,堙塞水道。襄軍下,夾岸擊之,斬襄首,
 殺千餘人。鎮惡性貪,既破襄,因停軍抄掠諸蠻,不時反。及至江陵,休之已平,高祖怒,不時見之。鎮惡笑曰:「但令我一見公,無憂矣。」高祖尋登城喚鎮惡,鎮惡為人彊辯,有口機,隨宜酬應,高祖乃釋。休之及魯宗之奔襄陽,鎮惡統蒯恩諸軍水路追之,休之等奔羌,鎮惡追躡,盡境而還。除游擊將軍。



 十二年,高祖將北伐,轉鎮惡為咨議參軍,行龍驤將軍,領前鋒。將發,前將軍劉穆之見鎮惡於積弩堂,謂之曰:「公愍此遺黎,志蕩逋逆。昔晉文王委
 伐蜀於鄧艾,今亦委卿以關中,想勉建大功,勿孤此授。」鎮惡曰:「不克咸陽,誓不復濟江而還也!」鎮惡入賊境,戰無不捷,邵陵、許昌,望風奔散;破虎牢及柏谷塢,斬賊帥趙玄。軍次洛陽,偽陳留公姚洸歸順。進次澠池,造故人李方家,升堂見母,厚加酬賚,即版授方為澠池令。遣司馬毛德祖攻偽弘農太守尹雅於蠡城,生擒之。



 仍行弘農太守。方軌長驅,徑據潼關。偽大將軍姚紹率大眾拒嶮,深溝高壘以自固。



 鎮惡懸軍遠入,轉輸不充,與賊相
 持久,將士乏食,乃親到弘農督上民租,百姓競送義粟,軍食復振。



 初,高祖與鎮惡等期,若剋洛陽,須大軍至,未可輕前。既而鎮惡等徑向潼關,為紹所拒不得進,而軍又乏食,馳告高祖,求遣糧援。時高祖沿河,索虜屯據河岸,軍不得前。高祖呼所遣人開舫北戶,指河上虜示之曰:「我語令勿進,而輕佻深入。



 岸上如此,何由得遣軍?」鎮惡既得義租,紹又病死,偽撫軍姚贊代紹守險,眾力猶盛。高祖至湖城,贊引退。



 大軍次潼關,謀進取之計,鎮惡
 請率水軍自河入渭。偽鎮北將軍姚強屯兵涇上,鎮惡遣毛德祖擊破之,直至渭橋。鎮惡所乘皆蒙衝小艦,行船者悉在艦內,羌見艦溯渭而進,艦外不見有乘行船人,北土素無舟楫,莫不驚惋,咸謂為神。鎮惡既至,令將士食畢,便棄船登岸。渭水流急,倏忽間,諸艦悉逐流去。時姚泓屯軍在長安城下,猶數萬人。鎮惡撫慰士卒曰:「卿諸人並家在江南,此是長安城北門外,去家萬里,而舫乘衣糧,並已逐流去,豈復有求生之計邪!唯宜死戰,
 可以立大功,不然,則無遺類矣。」乃身先士卒,眾亦知無復退路,莫不騰踴爭先。泓眾一時奔潰,即陷長安城。泓挺身逃走,明日,率妻子歸降。城內夷、晉六萬餘戶,鎮惡宣揚國恩,撫尉初附,號令嚴肅,百姓安堵。



 高祖將至,鎮惡於灞上奉迎。高祖勞之曰:「成吾霸業者,真卿也。」鎮惡再拜謝曰:「此明公之威,諸將之力,鎮惡何功之有焉!」高祖笑曰:「卿欲學馮異也。」是時關中豐全,倉庫殷積,鎮惡極意收斂,子女玉帛,不可勝計。高祖以其功大,不問也。
 進號征虜將軍。時有白高祖以鎮惡既克長安,藏姚泓偽輦,為有異志。高祖密遣人覘輦所在,泓輦飾以金銀,鎮惡悉剔取,而棄輦於垣側。高祖聞之,乃安。



 高祖留第二子桂陽公義真為安西將軍、雍秦二州刺史,鎮長安。鎮惡以本號領安西司馬、馮翊太守,委以扞禦之任。時西虜佛佛彊盛,姚興世侵擾北邊,破軍殺將非一。高祖既至長安,佛佛畏憚不敢動。及大軍東還,便寇逼北地。義真遣中兵參軍沈田子距之。虜甚盛,田子屯劉回堡,
 遣使還報鎮惡。鎮惡對田子使,謂長史王脩曰:「公以十歲兒付吾等,當各思竭力,而擁兵不進,寇虜何由得平!」使還,具說鎮惡言,田子素與鎮惡不協,至是益激怒。二人常有相圖志,彼此每相防疑。



 鎮惡率軍出北地,為田子所殺,事在《序傳》。時年四十六。田子又於鎮惡營內,殺鎮惡兄基、弟鴻、遵、淵及從弟昭、朗、弘,凡七人。是歲,十四年正月十五日也。



 高祖表曰:「故安西司馬、征虜將軍王鎮惡,志節亮直,機略明舉。自策名州府,屢著誠績。荊南
 遘釁,勢據上流,難興彊蕃,憂兼內侮。鎮惡輕舟先邁,神兵電臨,旰食之虞,一朝霧散。及王師西伐,有事中原,長驅洛陽,肅清湖、陜。入渭之捷,指麾無前,遂廓定咸陽,俘執偽后,克成之效,莫與為疇,實捍城所寄,國之方召也。近北虜遊魂,寇掠渭北,統率眾軍,曜威撲討。賊既還奔,還次涇上,故龍驤將軍沈田子忽發狂易,奄加刃害,忠勳未究,受禍不圖,痛惜兼至,惋悼無已,伏惟聖懷,為之傷惻。田子狂悖,即已備憲。鎮惡誠著艱難,勳參前烈,殊
 績未酬,宜蒙追寵,願敕有司,議其褒贈。」於是追贈左將軍、青州刺史。高祖受命,追封龍陽縣侯,食邑千五百戶,謚曰壯侯。配食高祖廟廷。



 子靈福嗣,位至南平王鑠右軍咨議參軍。靈福卒,子述祖嗣。述祖卒,子睿嗣。



 齊受禪,國除。



 鎮惡弟康,留關中,及高祖北伐,鎮惡為前鋒,康逃匿田舍。鎮惡次潼關,康將家奔之,高祖板為彭城公前將軍行參軍。鎮惡被害,康逃藏得免,攜家出洛陽,到彭城,歸高祖。即以康為相國行參軍。求還洛陽視母,尋值
 關、陜不守,康與長安徙民張旰醜、劉雲等唱集義徒,得百許人,驅率邑郭僑戶七百餘家,共保金墉城,為守戰之備。時有一人邵平,率部曲及并州乞活一千餘戶屯城南,迎亡命司馬文榮為主。又有亡命司馬道恭自東垣率三千人屯城西,亡命司馬順明五千人屯陵雲臺。



 順明遣刺殺文榮,平復推順明為主。又有司馬楚之屯柏谷塢,索虜野阪戍主黑弰公遊騎在芒上,攻偪交至,康堅守六旬。



 宋臺建,除康寧朔將軍、河東太守。遣龍驤
 將軍姜囗率軍救之,諸亡命並各奔散。高祖嘉康節,封西平縣男,食邑三百戶,進號龍驤將軍。迎康家還京邑。勸課農桑,百姓甚親賴之。永初元年卒金墉,時年四十九,葬於偃師城西。追贈輔國將軍。無子,以兄河西太守基子天祐嗣。當太祖元嘉二十七年,隨劉康祖伐索虜敗沒,子懷祖嗣。



 檀韶,字令孫,高平金鄉人也,世居京口。初辟本州從事,西曹主簿,輔國司馬。高祖建義,韶及弟祗、道濟等從平
 京城,行參高祖建武將軍事。都邑既平,為鎮軍參軍,加寧遠將軍、東海太守,進號建武將軍,遷龍驤將軍、秦郡太守,北陳留內史。以平桓玄功,封巴丘縣侯,食邑五百戶;復參車騎將軍事,加龍驤將軍,遷驍騎將軍,中軍咨議參軍,加寧朔將軍。



 從征廣固,率向彌、胡籓等五十人攻臨朐城,克之。及圍廣固,慕容超夜燒樓當韶圍分,降號橫野將軍。城陷之日,韶率所領先登,領北琅邪太守,進號寧朔將軍、琅邪內史。從討盧循於左里,又有戰功,并論
 廣固功,更封宜陽縣侯,食邑七百戶,降先封一等為伯,減戶之半二百五十戶,賜祗子臻。坐六門內乘輿,白衣領職。義熙七年,號輔國將軍。八年,丁母憂,起為冠軍將軍。明年,復為琅邪內史,淮南太守,將軍如故。鎮姑孰。尋進號左將軍,領本州大中正。十二年,遷督江州豫州之西陽新蔡二郡諸軍事、江州刺史,將軍如故。有罪,免官。



 高祖受命,以佐命功,增八百戶,并前千五百戶。韶嗜酒貪橫,所蒞無績,上嘉其合門從義,弟道濟又有大功,故
 特見寵授。永初二年,卒於京邑,時年五十六。



 追贈安南將軍,加散騎常侍。子緒嗣。緒卒,無子,國除。祗子臻。臻卒,子遐嗣,齊受禪,國除。祗、弟道濟並別有傳。



 向靖,字奉仁,小字彌,河內山陽人也。名與高祖祖諱同,改稱小字。世居京口,與高祖少舊。從平京城,參建武軍事。進平京邑,板參鎮軍軍事,加寧遠將軍。



 京邑雖平,而群寇互起,彌與劉籓、孟龍符征破桓歆、桓石康、石綏於白茅,攻壽陽剋之。義熙三年,遷建武將軍、秦郡太守,北陳留
 內史,戍堂邑。以平京城功,封山陽縣五等侯。從征鮮卑,大戰於臨朐,累月不決。彌與檀韶等分軍自間道攻臨朐城。彌擐甲先登,即時潰陷,斬其牙旗,賊遂奔走。攻拔廣固,彌又先登。盧循屯據蔡洲,以親黨阮賜為豫州刺史,攻逼姑孰。彌率譙國內史趙恢討之。時輔國將軍毛脩之戍姑孰,告急續至,彌兼行進討,破賜,收其輜重。除中軍咨議參軍,將軍如故。盧循退走,高祖南征,彌為前鋒,於南陵、雷池、左里三戰,並大捷。軍還,除太尉咨議參
 軍、下邳太守,將軍如故。



 八年,轉遊擊將軍,尋督馬頭淮西諸郡軍事、龍驤將軍、鎮蠻護軍、安豐汝陰二郡太守、梁國內史,戍壽陽。以平廣固、盧循功,封安南縣男,食邑五百戶。十年,遷冠軍將軍、高陽內史,臨淮太守,領石頭戍事。高祖西伐司馬休之,以彌為吳興太守,將軍如故。明年,高祖北伐,彌以本號侍從,留戍確磝,進屯石門、柏谷。遷督北青州諸軍事、北青州刺史,將軍如故。高祖受命,以佐命功,封曲江縣侯,食邑千戶。遷太子左衛率,加
 散騎常侍。二年,卒官,時年五十九。追贈前將軍。彌治身儉約,不營室宇,無園田商貨之業,時人稱之。



 子植嗣,多過失,不受母訓,奪爵。更以植次弟楨紹封,又坐何殺人,國除。



 植弟柳,字玄季,有學義才能,立身方雅,無所推先,諸盛流並容之。太尉袁淑、司空徐湛之、東揚州刺史顏竣皆與友善。歷始興王濬征北中兵參軍,始興內史,南康相。臧質為逆,召柳至尋陽,與之俱下。質敗歸降,下獄死。



 彌弟劭,永初中,為宣城太守。劭弟子亮,以私忿殺彌妻
 施氏,託云奴客所殺,劭輒於墓所殺亮及彌妾並奴婢七八人,匿不聞官,為有司所奏,詔無所問。元嘉初,卒於義興太守。



 劉懷慎,彭城人,左將軍懷肅弟也。少謹慎質直。始參高祖鎮軍將軍事,振威將軍、彭城內史。從征鮮卑,每戰必身先士卒,及克廣固,懷慎率所領先登。從高祖距盧循於石頭,屢戰克捷,加輔國將軍。義熙八年,以本號監北徐州諸軍事,鎮彭城,尋加徐州刺史。為政嚴猛,境內
 震肅。九年,亡命王靈秀為寇,討平之。十一年,進北中郎將。以平廣固、盧循功,封南城縣男,食邑五百戶。十三年,高祖北伐,以為中領軍、征虜將軍,衛輦轂。坐府中相殺,免官。雖名位轉優,而恭恪愈至,每所之造位任不踰己者,皆束帶門外下車,其謹退類如此。



 宋臺立,召為五兵尚書,仍督江北淮南諸軍、前將軍、南晉州刺史。復徵為度支尚書,加散騎常侍。高祖遷都壽春,留懷慎督北徐兗青淮北諸軍事、中軍將軍、徐州刺史。以亡命入廣陵
 城,降號征虜將軍。永初元年,以佐命功,進爵為侯,增邑千戶。進號平北將軍。徵為五兵尚書,加散騎常侍,光祿大夫。景平元年,遷護軍將軍,常侍如故。特賜班於宗族,家無餘財。二年卒,時年六十一。追贈撫軍,謚曰肅侯。



 子德願嗣。世祖大明初,為游擊將軍,領石頭戍事。坐受賈客韓佛智貨,下獄,奪爵土。後復為秦郡太守。德願性粗率,為世祖所狎侮。上寵姬殷貴妃薨,葬畢,數與群臣至殷墓。謂德願曰:「卿哭貴妃若悲,當加厚賞。」德願應聲便
 號慟,撫膺擗踴,涕泗交流。上甚悅,以為豫州刺史。又令醫術人羊志哭殷氏,志亦嗚咽。



 他日有問志:「卿那得此副急淚?」志時新喪愛姬,答曰:「我爾日自哭亡妾耳。」



 志滑稽,善為諧謔,上亦愛狎之。德願善御車,嘗立兩柱,使其中劣通車軸,乃於百餘步上振轡長驅,未至數尺,打牛奔從柱間直過,其精如此。世祖聞其能,為之乘畫輪車,幸太宰江夏王義恭第。德願岸著籠冠,短朱衣,執轡進止,甚有容狀。



 永光中,為廷尉,與柳元景厚善。元景敗,下
 獄誅。



 懷慎庶長子榮祖,少好騎射,為高帝所知。及盧循攻逼,時賊乘小艦,入淮拔柵。武帝宣令三軍,不得輒射賊。榮祖不勝憤怒,冒禁射之,所中應弦而倒,帝益奇焉。以戰功參太尉軍事。從討司馬休之,彭城內史徐達之敗沒,諸將意沮,榮祖請戰愈厲,高祖乃解所著鎧以授之。榮祖率所領陷陣,身被數創,會賊破走。加振威將軍,尋參世子征虜軍事,領遂成令。高祖北伐,轉鎮西中兵參軍,寧遠將軍。



 水軍入河,與朱超石大破索虜於半城,
 又攻劉度壘克之。高祖大饗戰士,謂榮祖曰:「卿以寡克眾,攻無堅城,雖古名將,何以過此。」轉為太尉中兵參軍,加建威將軍。既破長安,姚泓女婿徐眾率其餘眾連營叛走,榮祖與檀道濟等攻營破之,斬首擒馘,不可稱計。十四年,除彭城內史,又補相國參軍。其年,遣榮祖還都,為世子中兵參軍。



 永初元年,除越騎校尉,尋轉右軍將軍。索虜南寇,司州刺史毛德祖陷沒,榮祖時居父艱,起為輔國將軍。追論半城之功,賜爵都鄉侯。榮祖為人輕
 財貴義,善撫將士,然性偏險褊隘,頗失士君子之心。領軍將軍謝晦深接待之,廢立之際,要榮祖,固辭獲免。及晦出鎮荊楚,欲請為南蠻校尉,榮祖又固止之。其年冬卒。德願弟興祖,青州刺史。



 懷慎弟懷默,冠軍將軍、江夏內史,太中大夫。懷默子道球,巴東、建平二郡太守。道球弟孫登,武陵內史。孫登子亮,世祖大明中,為武康令。時境內多盜鑄錢,亮掩討無不禽,所殺以千數。太宗泰始初,為巴陵王休若鎮東中兵參軍,北伐南討,功冠諸將,
 封順陽縣侯,食邑六百戶,歷黃門郎,梁、益二州刺史。在任廉儉,不營財貨,所餘公祿,悉以還官。太宗嘉之,下詔褒美。亮在梁州,忽服食修道,欲致長生。迎武當山道士孫道胤,令合仙藥。至益州,泰豫元年藥始成,而未出火毒。孫不聽亮服,亮苦欲服,平旦開城門取井華水服,至食鼓後,心動如刺,中間便絕。後人逢見,乘白馬,將數十人,出關西行,共語分明,此乃道家所謂尸解者也。追贈冠軍將軍,謚曰剛侯。



 孫登弟道隆,元嘉二十二年,為廬
 江太守。世祖舉義,棄郡來奔,以補南中郎參軍事,加龍驤將軍。時世祖分麾下以為三幢,道隆與中兵參軍王謙之、馬文恭各領其一。大明中,歷黃門侍郎,徐、青、冀三州刺史。前廢帝景和中,以為右衛將軍,永昌縣侯,食邑五百戶,委以腹心之任。泰始初,為太守盡力,遷左衛將軍,中護軍。尋賜死,事在《建安王休仁傳》。



 王謙之,字休光,琅邪臨沂人,晉司州刺史胡之曾孫也。世祖初,歷驍騎將軍,御史中丞,吳興太守。以南下之功,封石陽縣子,食邑
 五百戶。大明三年卒,贈前將軍,謚曰肅。子應之嗣。大明末,為衡陽內史。晉安王子勛反,應之起義拒湘州行事何慧文,為慧文所殺,事在《鄧琬傳》,追贈侍中。應之弟雲之,順帝昇明中貴達。



 馬文恭,扶風人也。亦以功封泉陵縣子,食邑五百戶。世祖即位,為遊擊將軍。



 頃之,卒。



 劉粹,字道沖,沛郡蕭人也。祖恢,持節、監河中軍事,征虜將軍。粹家在京口。少有志幹,初為州從事。高祖克京城,參建武軍事。從平京邑,轉參鎮軍事,尋加建武將軍、沛
 郡太守;又領下邳太守,復為車騎中軍參軍。從征廣固,戰功居多。以建義功,封西安縣五等侯。軍還,轉中軍咨議參軍。盧循逼京邑,京口任重,太祖時年四歲,高祖使粹奉太祖鎮京城。轉游擊將軍。遷建威將軍、江夏相。



 衛將軍毅,粹族兄也,粹盡心高祖,不與毅同。高祖欲謀毅,眾並疑粹在夏口,高祖愈信之。及大軍至,粹竭其誠力。事平,封灄陽縣男,食邑五百戶。母憂去職。



 俄而高祖討司馬休之,起粹為寧朔將軍、竟陵太守,統水軍入河。明年,進
 號輔國將軍,遷相國右司馬、侍中、中軍司馬、冠軍將軍,遷左衛將軍。永初元年,以佐命功,改封建安縣侯,食邑千戶。二年,以役使監吏,免官。尋督江北淮南郡事、征虜將軍、廣陵太守。三年,以本號督豫司雍并四州南豫州之梁郡弋陽馬頭三郡諸軍事、豫州刺史,領梁郡太守,鎮壽陽,治有政績。少帝景平二年,譙郡流離六十餘家叛沒虜,趙炅、秦剛等六家悔倍還投陳留襄邑縣,頓謀等村,粹遣將苑縱夫討叛戶不及,因誅殺謀等三十家,
 男丁一百三十七人,女弱一百六十二口,收付作部。



 粹坐貶號為寧朔將軍。時索虜南寇,粹遣將軍李德元襲許昌,殺偽潁川太守庾龍,於是陳留人董邈自稱小黃盟主,斬偽征虜將軍、廣州刺史司馬世賢,傳首京都。



 太祖即位,遷使持節、督雍梁南北秦四州荊州之南陽竟陵順陽襄陽新野隨六郡諸軍事、征虜將軍、領寧蠻校尉、雍州刺史、襄陽新野二郡太守。在任簡役愛民,罷諸沙門二千餘人,以補府史。元嘉三年討謝晦,遣粹弟車
 騎從事中郎道濟、龍驤將軍沈敞之就粹,自陸道向江陵。粹以道濟行竟陵內史,與敞之及南陽太守沈道興步騎至沙橋,為晦司馬周超所敗,士眾傷死者過半,降號寧朔將軍。初,晦與粹厚善,以粹子曠之為參軍。粹受命南討,一無所顧,太祖以此嘉之。晦遣送曠之還粹,亦不害也。明年,粹卒,時年五十三。追贈安北將軍,持節、本官如故。



 曠之嗣,官至晉熙太守。曠之卒,子琛嗣。琛卒,無子,國除。琛弟亮,順帝升明末,尚書駕部郎。粹庶長子懷
 之,為臨川內史,與臧質同逆,伏誅。



 粹弟道濟,尚書起部郎,王弘車騎從事中郎,江夏王義恭撫軍司馬,河東太守,仍遷振武將軍、益州刺史。長史費謙、別駕張熙、參軍楊德年等,並聚斂興利,而道濟委任之,傷政害民,民皆怨毒。太祖聞之,與道濟詔,戒之曰:「聞卿在任,未盡清省,又頗為殖貨,若萬一有此,必宜改之。比傳人情不甚緝諧,當以法御下,深思自警,以副本望。」道濟雖奉此旨,政化如初。



 有司馬飛龍者,自稱晉之宗室,晉末走仇池。元
 嘉九年,聞道濟綏撫失和,遂自仇池入綿竹,崩動群小,得千餘人,破巴興縣,殺令王貞之。進攻陰平,陰平太守沈法興焚城遁走。道濟遣軍擊飛龍,斬之。初,道濟以五城人帛氐奴、梁顯為參軍督護,費謙固執不與。遠方商人多至蜀土資貨,或有直數百萬者,謙等限布絲綿各不得過五十斤,馬無善惡,限蜀錢二萬。府又立冶,一斷民私鼓鑄,而貴賣鐵器,商旅吁嗟,百姓咸欲為亂。氐奴既懷恚忿,因聚黨為盜賊。其年七月,道濟遣羅習為五
 城令,氐奴等謀曰:「羅令是使君腹心,而卿猶有作賊盜不止者,一旦發露,則為禍不測。宜結要誓,共相禁檢。」乃殺牛盟誓。俄而氐奴及趙廣等唱曰:「官禁殺牛,而村中公違法禁,脫使羅令白使君,疑吾徒更欲作賊,則無餘類矣。」因詐言司馬殿下猶在陽泉山中,若能共建大事,則功名可立,不然,立滅不久。眾既樂亂,因相率從之,得數千人,復向廣漢。道濟遣參軍程展會、治中李抗之五百人擊之,並為所殺。賊於是徑向涪城,巴西人唐頻聚
 眾應之,寧遠將軍、巴西梓潼二郡太守王懷業再遣軍拒之,戰敗失利。懷業及司馬、南漢中太守韋處伯並棄城走。



 涪陵太守阮惠、江陽太守社玄起、遂寧太守馮遷聞涪城不守,並委郡出奔。蜀土僑舊,翕然並反。道濟惶懼,乃免吳兵三十六營以為平民,分立宋興、宋寧二郡,又招集商賈及免道俗奴僮,東西勝兵可有四千人。賊眾數萬屯城西及城北,道濟嬰城自守。



 趙廣本以譎詐聚兵,頓兵城下,不見飛龍,各欲分散。廣懼,乃將三千人
 及羽儀,詐其眾云迎飛龍。至陽泉寺中,謂道人程道養曰:「但自言是飛龍,則坐享富貴;若不從,即日便斬頭。」道養惶怖許諾。道養,枹罕人也,廣改名為龍興,號為蜀王、車騎大將軍、益梁二州牧,建號泰始元年,備置百官。以道養弟道助為驃騎將軍、長沙王,鎮涪城。廣自號鎮軍,帛氐奴征虜將軍,梁顯鎮北將軍,同黨大帥張寧秦州刺史,嚴遐前將軍。奉道養還成都,眾十餘萬,四面圍城。就道濟索費謙、張熙,曰:「但送此人來,我等自不復作賊。」



 道濟遣中兵參軍裴方明、任浪之各將千餘人出西門戰,皆失利。十一月,方明等復出戰,破賊營,焚其積聚。賊黨江陽人楊孟子領千餘人,屯城南。道濟參軍梁俊之統南樓,屢與孟子交言,因投書曉以禍福,要使入城。孟子許諾,入見道濟;道濟大喜,即板為主簿,遣子為任,克期討賊。趙廣知其謀,孟子懼,將所領奔晉原。晉原太守文仲興拾合得二千餘人,與孟子并力自固。廣遣同黨袁玄子攻晉原,為仲興所殺。廣又遣帛氐奴攻之,連戰,
 仲興軍敗,及孟子並死。



 方明復出東門,破賊三營,斬首數百級。賊雖敗,已復還合。方明復偽出北門,仍回軍擊城東大營,殺千餘人,斬偽僕射蔡滔。時天大霧,方明等復揚聲出東門,而潛自北門出攻城北城西諸營,賊眾大潰,於是奔散。道養收合得七千人還廣漢,趙廣以別卒五千餘人還涪城。



 初,別駕張熙說道濟令糶太倉穀,賊以九月末圍城,至十二月末,廩糧便盡。



 方明將二千人出城求食,為賊所敗,匹馬獨還。賊因追之,眾復大集。
 方明夜於城西縋上,道濟為設食,饐不能飧,唯泣涕而已。道濟時有疾已篤,自力慰勉之曰:「卿非大丈夫,小敗何苦。賊勢既衰,臺兵垂至,但令卿還,何憂於賊。」即減左右數十人配之。賊城外云:「方明已死,可來取喪。」城中大恐。道濟夜列炬火,方明自出,眾見之乃安。道濟悉出財物於北射堂,令方明募人。時城中或傳道濟已亡,莫有至者。梁俊之說道濟曰:「將軍氣息綿綿,而外論互有同異。今軍師屢敗,妖寇未殄,若一旦不虞,則危禍立至。宜
 稱小損,聽左右給使暫出,不然敗矣。」



 道濟從之,即喚左右三十餘人,告之曰:「吾疾久,汝等扶侍疲勞。今既小損,各聽歸家休息,喚復還。」給使既出,其父兄皆問:「使君亡來幾日?」子弟皆言:「君漸差,誰言亡者!」傳相告語,城內乃安,由是應募者一日千餘人。十年正月,賊眾大至,攻逼成都。道濟卒,梁俊之與方明等,及其故舊門生數人,共埋尸於後齋。使書與道濟相似者為教命,酬答簽疏,不異常日,故雖母妻,不知也。



 二月,道養於毀金橋升壇郊
 天,方就柴燎,方明將三千人出擊之。賊列陣營前死戰,日夕乃大敗。臨陣斬偽征虜將軍趙石之等八百餘級,道養等退保廣漢。是月,平西將軍臨川王義慶,以揚武將軍、巴東太守周籍之即本號督巴西梓潼宕渠遂寧巴郡五郡諸軍事、巴西梓潼二郡太守,率平西參軍費淡、龍驤將軍羅猛二千人援成都。



 廣等屯據廣漢,分守郫川,連營百數,處處屯結。籍之與方明及費淡等攻郫,克之。



 廣等退據郡城,傍竹自固。羅猛率隊主王盱等并
 力追討。張尋自涪城率眾二萬來助廣等,方明、淡斬竹開徑邀之,戰敗,退還郫縣。廣等又移營屯箭竿橋,方明等破其六營,乘勝追奔,徑至廣漢。廣等走還涪及五城。四月十日,發道濟喪。五月,方明進軍向涪城。張尋、唐頻渡水拒戰,方明擊破之,生擒偽驃騎將軍、雍秦二州刺史司馬龍伸,斬之。龍伸,道助也。州吏嚴道度斬嚴遐首,廣等並奔散,涪、蜀皆平。俄而張尋攻破陰平,復與道養合。帛氐奴攻廣漢,費淡督將軍種松等與戰,斬其梁州刺
 史杜承等百餘級。九月,益州刺史甄法崇至成都,誅費謙之,道濟喪及方明等並東反。道養等領二千餘家逃於郪山,其餘群賊,亦各擁戶藏竄,出為寇盜不絕。



 十三年六月,太祖遣寧朔將軍蕭汪之統軍討之。軍次郪口,帛氐奴斬偽衛將軍司馬飛燕歸降。汪之擊破道養,道養還入郪山。十四年四月,趙廣、張尋、梁顯各率部曲歸降,偽輔軍將軍王道恩斬道養,送首,餘黨悉平。遷趙廣、張尋等於京師。



 十六年,廣、尋復與國山令司馬敬琳謀
 反,伏誅。



 先是,道濟振武司馬、蜀郡太守任薈之雖不任軍事,事寧,以為正員郎。裴方明虎賁中郎將,仍為義慶平西中兵參軍、龍驤將軍、河東太守。費淡,太子屯騎校尉。周籍之後為益州刺史。



 粹族弟損,字子騫,衛將軍毅從父弟也。父鎮之,字仲德,以毅貴,歷顯位,閑居京口,未嘗應召。常謂毅:「汝必破我家。」毅甚憚之,每還京,未嘗敢以羽儀人從入鎮之門。左光祿大夫征,不就。元嘉二年,年九十餘,卒於家。損,元嘉中歷職義興太守。東土殘饑,
 太祖遣揚州治中沈演之東入賑恤,以損綏撫有方,稱為良守。官至吳郡太守,追贈太常。



 史臣曰:帝王受命,自非以功靜亂,以德濟民,則其道莫由也。自三代以來,醇風稍薄,成功濟務,尊出權道,雖復負扆南面,比號軒、犧,莫不自謝王風,率由霸德。高祖崛起布衣,非藉民譽,義無曹公英傑之響,又闕晉氏輔魏之基,一旦驅烏合,不崇朝而制國命,功雖有餘,而德未足也。是故王謐以內懼流奔,王綏以外侮成釁,若非樹
 奇功於難立,震大威於四海,則不能承配天之業,一異同之心。



 義熙以後,大功仍建,自桓溫旍旆所臨,莫不獻珍受朔。及金墉請吏,元勛將舉,九命之禮既行,代終之符已及。方復觀兵函、渭,用師天險,獨克之舉,振古難稱。



 若使閉門反政,置兵散地,後敗責其前功,一眚虧其盛業,豈復得以黃屋硃戶,為衰晉之貞臣乎?及其靈威薄震,重關莫守,故知英算所苞,先勝而後戰也。王鎮惡推鋒直指,前無強陳,為宋方叔,壯矣哉!



\end{pinyinscope}