\article{卷三本紀第三 武帝下}

\begin{pinyinscope}

 永初元年夏六月丁卯,設壇於南郊,即皇帝位,柴燎告天。策曰:皇帝臣諱,敢用玄牡,昭告後天后帝。晉帝以卜世告終,歷數有歸,欽若景運,以命于諱。夫樹君宰世,天
 下為公,德充帝王,樂推攸集。越俶唐、虞,降暨漢、魏,靡不以上哲格文祖,元勳陟帝位,故能大拯黔首,垂訓無窮。晉自東遷,四維不振,宰輔憑依,為日已久。難棘隆安,禍成元興,遂至帝主遷播,宗禮堙滅。諱雖地非齊、晉,眾無一旅,仰憤時難,俯悼橫流,投袂一援,則皇祀克復。及危而能持,顛而能扶,姦宄具殲,僭偽必滅。誠興廢有期,否終有數。至於大造晉室,撥亂濟民,因藉時來,實尸其重。加以殊俗慕義,重譯來庭,正朔所暨,咸服聲教。



 至乃三
 靈垂象,山川告祥,人神協祉,歲月滋著。是以群公卿士,億兆夷人,僉曰皇靈降鑒於上,晉朝款誠於下,天命不可以久淹,宸極不可以暫曠。遂逼群議,恭茲大禮。猥以寡德,託於兆民之上,雖仰畏天威,略是小節,顧深永懷,祗懼若霣。



 敬簡元辰,升壇受禪,告類上帝,用酬萬國之情。克隆天保,永祚于有宋。惟明靈是饗。



 禮畢,備法駕幸建康宮,臨太極前殿。詔曰:「夫世代迭興,承天統極。雖遭遇異途,因革殊事,若乃功濟區宇,道振生民,興廢所階,異世一
 揆。朕以寡薄,屬當艱運,藉否終之期,因士民之力,用獲拯溺,匡世揆亂,安國寧民,業未半古,功參曩烈。晉氏以多難仍遘,歷運已移,欽若前王,憲章令軌,用集大命于朕躬。



 惟德匪嗣,辭不獲申,遂祗順三靈,饗茲景祚,燔柴于南郊,受終于文祖。猥當與能之期,爰集樂推之運,嘉祚肇開,隆慶惟始,思俾休嘉,惠茲兆庶。其大赦天下。



 改晉元熙二年為永初元年。賜民爵二級。鰥寡孤獨不能自存者,人穀五斛。逋租宿債勿復收。其有犯鄉
 論清議、贓汙淫盜,一皆蕩滌洗除,與之更始。長徒之身,特皆原遣。亡官失爵,禁錮奪勞,一依舊準。」



 封晉帝為零陵王,全食一郡。載天子旌旗,乘五時副車,行晉正朔,郊祀天地禮樂制度,皆用晉典。上書不為表,答表勿稱詔。追尊皇考為孝穆皇帝,皇妣為穆皇后,尊王太后為皇太后。詔曰:「夫微禹之感,歎深後昆,盛德必祀,道隆百世。



 晉氏封爵,咸隨運改,至於德參微管,勳濟蒼生,愛人懷樹,猶或勿翦,雖在異代,義無泯絕。降殺之儀,一依前典。
 可降始興公封始興縣公,廬陵公封柴桑縣公,各千戶;始安公封荔浦縣侯,長沙公封醴陵縣侯,康樂公可即封縣侯,各五百戶:以奉晉故丞相王導、太傅謝安、大將軍溫嶠、大司馬陶侃、車騎將軍謝玄之祀。其宣力義熙,豫同艱難者,一仍本秩,無所減降。」封晉臨川王司馬寶為西豐縣侯,食邑千戶。



 庚午,以司空道憐為太尉,封長沙王。追封司徒道規為臨川王。尚書僕射徐羨之加鎮軍將軍,右衛將軍謝晦為中領軍,宋國領軍檀道濟為
 護軍將軍,中領軍劉義欣為青州刺史。立南郡公義慶為臨川王。又詔曰:「夫銘功紀勞,有國之要典,慎終追舊,在心之所隆。自大業創基,十有七載,世路迍邅,戎車歲動,自東徂西,靡有寧日。實賴將帥竭心,文武盡效;寧內拓外,迄用有成。威靈遠著,寇逆消蕩,遂當揖讓之禮,猥饗天人之祚。念功簡勞,無忘鑒寐,凡厥誠勤,宜同國慶。其酬賞復除之科,以時論舉。戰亡之身,厚加復贈。」乙亥,立桂陽公義真為廬陵王,彭城公義隆為宜都王,第四
 皇子義康為彭城王。



 丁丑,詔曰:「古之王者,巡狩省方,躬覽民物,搜揚幽隱,拯災恤患,用能風澤遐被,遠至邇安。朕以寡暗,道謝前哲,因受終之期,託兆庶之上,鑒寐屬慮,思求民瘼。才弱事艱,若無津濟,夕惕永念,心馳遐域。可遣大使分行四方,旌賢舉善,問所疾苦。其有獄訟虧濫,政刑乖愆,傷化擾治,未允民聽者,皆當具以事聞。萬事之宜,無失厥中。暢朝遷乃眷之旨,宣下民壅隔之情。」戊寅,詔曰:「百官事殷俸薄,祿不代耕。雖國儲未豐,要令
 公私周濟。諸供納昔減半者,可悉復舊。六軍見祿粗可,不在此例。其餘官僚,或自本俸素少者,亦疇量增之。」乙卯,改晉《泰始歷》為《永初歷》。



 秋七月丁亥,原放劫賊餘口沒在臺府者,諸徙家並聽還本土。又運舟材及運船,不復下諸郡輸出,悉委都水別量。臺府所須,皆別遣主帥與民和市,即時裨直,不復責租民求辦。又停廢虜車牛,不得以官威假借。又以市稅繁苦,優量減降。從征關、洛,殞身戰場,幽沒不反者,贍賜其家。己丑,陳留王曹虔嗣
 薨。辛卯,復置五校三將官,增殿中將軍員二十人,餘在員外。戊戌,後將軍、雍州刺史趙倫之進號安北將軍;征虜將軍、北徐州刺史劉懷慎進號平北將軍;征西大將軍、開府儀同三司楊盛進號車騎大將軍。甲辰,鎮西將軍李歆進號征西將軍,平西將軍乞佛熾盤進號安西大將軍,征東將軍高句驪王高璉進號征東大將軍,鎮東將軍百濟王扶餘映進號鎮東大將軍。置東宮冗從僕射、旅賁中郎將官。戊申,遷神主於太廟,車駕親奉。壬
 子,詔曰:「往者軍國務殷,事有權制,劫科峻重,施之一時。今王道維新,政和法簡,可一除之,還遵舊條。反叛淫盜三犯補冶士,本謂一事三犯,終無悛革。



 主者頃多并數眾事,合而為三,甚違立制之旨,普更申明。」



 八月戊午,西中郎將、荊州刺史宜都王諱進號鎮西將軍。辛酉,開亡叛赦,限內首出,蠲租布二年。先有資狀、黃籍猶存者,聽復本注。諸舊郡縣以北為名者,悉除;寓方於南者,聽以南為號。又制有無故自殘傷者補冶士,實由政刑煩苛,
 民不堪命,可除此條。罷青州並兗州。戊辰,詔曰:「彭、沛、下邳三郡,首事所基,情義繾綣,事由情獎,古今所同。彭城桑梓本鄉,加隆攸在,優復之制,宜同豐、沛。其沛郡、下邳可復租布三十年。」辛未,追謚妃臧氏為敬皇后。癸酉,立王太子為皇太子。乙亥,詔曰:「朕承歷受終,猥饗天命。荷積善之祚,藉士民之力,率由令範。先后祗嚴宣訓,七廟肇建,情敬無違。加以儲宮備禮,皇基彌固,國慶家禮,爰集旬日,豈予一人,獨荷茲慶。其見刑罪無輕重,可悉原
 赦。限百日,以今為始。先因軍事所發奴僮,各還本主;若死亡及勳勞破免,亦依限還直。」



 閏月壬午朔,詔曰:「晉世帝后及籓王諸陵守衛,宜便置格。其名賢先哲,見優前代,或立德著節,或寧亂庇民,墳塋未遠,並宜洒掃。主者具條以聞。」丁酉,特進、左光祿大夫孔季恭加開府儀同三司。辛丑,詔曰:「主者處案雖多所諮詳,若眾官命議,宜令明審。自頃或總稱參詳,於文漫略。自今有厝意者,皆當指名其人;所見不同,依舊繼啟。」又詔曰:「諸處冬使,或
 遣或不,事役宜省,今可悉停。唯元正大慶,不在其例。郡縣遣冬使詣州及都督府,亦停之。」九月壬子朔,置東宮殿中將軍十人,員外二十人。壬申,置都官尚書。冬十月辛卯,改晉所用王肅祥禫二十六月儀,依鄭玄二十七月而後除。十二月辛巳朔,車駕臨延賢堂聽訟。



 二年春正月辛酉,車駕祠南郊,大赦天下。丙寅,斷金銀塗。以揚州刺史廬陵王義真為司徒,以尚書僕射、鎮軍將軍徐羨之為尚書令、揚州刺史。丙子,南康揭陽蠻反,
 郡縣討破之。己卯,禁喪事用銅釘。罷會稽郡府。二月己丑,車駕幸延賢堂策試諸州郡秀才、孝廉。揚州秀才顧練、豫州秀才殷朗所對稱旨,並以為著作佐郎。戊申,制中二千石加公田一頃。三月乙丑,初限荊州府置將不得過二千人,吏不得過一萬人;州置將不得過五百人,吏不得過五千人。兵士不在此限。夏四月己卯朔,詔曰:「淫祠惑民費財,前典所絕,可並下在所除諸房廟。其先賢及以勳德立祠者,不在此例。」戊申,車駕於華林園聽
 訟。己亥,以左衛將軍王仲德為冀州刺史。五月己酉,置東宮屯騎、步兵、翊軍三校尉官。甲戌,車駕又幸華林園聽訟。



 六月壬寅,詔曰:「杖罰雖有舊科,然職務殷碎,推坐相尋。若皆有其實,則體所不堪;文行而已,又非設罰之意。可籌量觕為中否之格。」車駕又於華林園聽訟。



 甲辰,制諸署敕吏四品以下,又府署所得輒罰者,聽統府寺行四十杖。秋七月己巳,地震。八月壬辰,車駕又於華林園聽訟。九月己丑,零陵王薨。車駕三朝率百僚舉哀於
 朝堂,一依魏明帝服山陽公故事。太尉持節監護,葬以晉禮。冬十月丁酉,詔曰:「兵制峻重,務在得宜。役身死叛,輒考傍親,流遷彌廣,未見其極。遂令冠帶之倫,淪陷非所。宜革以弘泰,去其密科。自今犯罪充兵合舉戶從役者,便付營押領。其有戶統及謫止一身者,不得復侵濫服親,以相連染。」己亥,以涼州胡帥大沮渠蒙遜為鎮軍大將軍、開府儀同三司、涼州刺史。癸卯,車駕於延賢堂聽訟。



 以員外散騎常侍應襲為寧州刺史。



 三年春正月甲辰朔,詔刑罰無輕重,悉皆原降。壬子,以前冀州刺史王仲德為徐州刺史。癸丑,以尚書令、揚州刺史徐羨之為司空、錄尚書事,刺史如故。撫軍將軍、江州刺史王弘進號衛將軍、開府儀同三司,太子詹事傅亮為尚書僕射,中領軍謝晦為領軍將軍。乙卯,以輔國將軍毛德祖為司州刺史。乙丑,詔曰:「古之建國,教學為先,弘風訓世,莫尚於此;發蒙啟滯,咸必由之。故爰自盛王,迄于近代,莫不敦崇學藝,修建庠序。自昔多故,戎馬在
 郊,旌旗卷舒,日不暇給。遂令學校荒廢,講誦蔑聞,軍旅日陳,俎豆藏器,訓誘之風,將墜于地。後生大懼於墻面,故老竊歎於子衿。此《國風》所以永思,《小雅》所以懷古。今王略遠屆,華域載清,仰風之士,日月以冀。便宜博延胄子,陶獎童蒙,選備儒官,弘振國學。



 主者考詳舊典,以時施行。」二月丁丑,詔曰:「豫州南臨江滸,北接河、洛,民荒境曠,轉輸艱遠,撫蒞之宜,各有其便。淮西諸郡,可立為豫州;自淮以東,為南豫州。」以豫州刺史彭城王義康為南
 豫州刺史,征虜將軍劉粹為豫州刺史。又分荊州十郡還立湘州,左衛將軍張紀為湘州刺史。戊寅,以徐州之梁,還屬豫州。三月,上不豫。太尉長沙王道憐、司空徐羨之、尚書僕射傅亮、領軍將軍謝晦、護軍將軍檀道濟並入侍醫藥。群臣請祈禱神祇,上不許,唯使侍中謝方明以疾告廟而已。



 丁未,以司徒廬陵王義真為車騎將軍、開府儀同三司、南豫州刺史。上疾瘳,己未,大赦天下。時秦雍流戶悉南入梁州。庚申,送珝絹萬匹,荊、雍州運米,
 委州刺史隨宜賦給。辛酉,亡命刁彌攻京城,得入,太尉留府司馬陸仲元討斬之。夏四月乙亥,封仇池公楊盛為武都王,平南將軍楊撫進號安南將軍。丁亥,以車騎司馬徐琰為兗州刺史。庚寅,左光祿大夫、開府儀同三司孔季恭薨。五月,上疾甚,召太子誡之曰:「檀道濟雖有幹略,而無遠志,非如兄韶有難御之氣也。徐羨之、傅亮當無異圖。謝晦數從征伐,頗識機變,若有同異,必此人也。小卻,可以會稽、江州處之。」又為手詔曰:「朝廷不須復
 有別府,宰相帶揚州,可置甲士千人。若大臣中任要,宜有爪牙以備不祥人者,可以臺見隊給之。有征討悉配以臺見軍隊,行還復舊。後世若有幼主,朝事一委宰相,母后不煩臨朝。仗既不許入臺殿門,要重人可詳給班劍。」癸亥,上崩于西殿,時年六十。秋七月己酉,葬丹陽建康縣蔣山初寧陵。



 上清簡寡慾,嚴整有法度,未嘗視珠玉輿馬之飾,後庭無紈綺絲竹之音。寧州嘗獻虎魄枕,光色甚麗。時將北征,以虎魄治金創,上大悅,命搗碎分
 付諸將。平關中,得姚興從女,有盛寵,以之廢事。謝晦諫,即時遣出。財帛皆在外府,內無私藏。宋臺既建,有司奏東西堂施局腳床、銀塗釘,上不許;使用直腳床,釘用鐵。



 諸主出適,遣送不過二十萬,無錦繡金玉。內外奉禁,莫不節儉。性尤簡易,常著連齒木履,好出神虎門逍遙,左右從者不過十餘人。時徐羨之住西州,嘗思羨之,便步出西掖門;羽儀絡繹追隨,已出西明門矣。諸子旦問起居,入皞,脫公服,止著裙帽,如家人之禮。孝武大明中,壞
 上所居陰室,於其處起玉燭殿,與群臣觀之。



 床頭有土鄣,壁上掛葛燈籠、麻繩拂。侍中袁鳷盛稱上儉素之德。孝武不答,獨曰:「田舍公得此,以為過矣。」故能光有天下,克成大業者焉。



 史臣曰:漢氏載祀四百,比胙隆周,雖復四海橫潰,而民繫劉氏,惵惵黔首,未有遷奉之心。魏武直以兵威服眾,故能坐移天歷;鼎運雖改,而民未忘漢。及魏室衰孤,怨非結下。晉籍宰輔之柄,因皇族之微,世擅重權,用基王
 業。至於宋祖受命,義越前模。晉自社廟南遷,祿去王室,朝權國命,遞歸台輔。君道雖存,主威久謝。桓溫雄才蓋世,勳高一時,移鼎之業已成,天人之望將改。自斯以後,晉道彌昏,道子開其禍端,元顯成其末釁,桓玄藉運乘時,加以先父之業,因基革命,人無異心。高祖地非桓、文,眾無一旅,曾不浹旬,夷凶翦暴,祀晉配天,不失舊物,誅內清外,功格區宇。至於鐘石變聲,柴天改物,民已去晉,異於延康之初,功實靜亂,又殊咸熙之末。所以恭皇高
 遜,殆均釋負。若夫樂推所歸,謳歌所集,魏、晉採其名,高
 祖收其實矣。盛哉!



\end{pinyinscope}