\article{卷二十三志第十三 天文一}

\begin{pinyinscope}

 言天者有三家,一曰宣夜,二曰蓋天,三曰渾天,而天之正體,經無前說,馬《書》、班《志》,又闕其文。漢靈帝議郎蔡邕於朔方上書曰:「論天體者三家,宣夜之學,絕無師法。《周
 髀》術數具存,考驗天狀,多所違失。惟渾天僅得其情,今史官所用候臺銅儀,則其法也。立八尺圓體,而具天地之形,以正黃道;占察發斂,以行日月,以步五緯,精微深妙,百世不易之道也。官有其器而無本書,前志亦闕而不論。本欲寢伏儀下,思惟微意,按度成數,以著篇章。罪惡無狀,投畀有北,灰滅雨絕,勢路無由。宜問群臣,下及巖穴,知渾天之意者,使述其義。」時閹官用事,邕議不行。



 漢末吳人陸績善天文,始推渾天意。王蕃者,盧江人,吳時
 為中常侍,善數術,傳劉洪《乾象歷》。依《乾象法》而制渾儀,立論考度曰:前儒舊說,天地之體,狀如鳥卵,天包地外,猶殼之裹黃也。周旋無端,其形渾渾然,故曰渾天也。周天三百六十五度五百八十九分度之百四十五,半露地上,半在地下。其二端謂之南極、北極。北極出地三十六度,南極入地亦三十六度,兩極相去一百八十二度半強。繞北極徑七十二度,常見不隱,謂之上規;繞南極七十二度,常隱不見,謂之下規。赤道帶天之紘,去兩極
 各九十一度少強。黃道,日之所行也。半在赤道外,半在赤道內,與赤道東交於角五少弱,西交於奎十四少強。



 其出赤道外極遠者,去赤道二十四度,斗二十一度是也。其入赤道內極遠者,亦二十四度,井二十五度是也。



 日南至在斗二十一度,去極百一十五度少強是也。日最南,去極最遠,故景最長。黃道斗二十一度,出辰入申,故日亦出辰入申。日晝行地上百四十六度強,故日短;夜行地下二百一十九度少弱,故夜長。自南至之後,日
 去極稍近,故景稍短。



 日晝行地上度稍多,故日稍長;夜行地下度稍少,故夜稍短。日所在度稍北,故日稍北,以至於夏至,日在井二十五度,去極六十七度少強,是日最北,去極最近,景最短。黃道井二十五度,出寅入戌,故日亦出寅入戌。日晝行地上二百一十九度少弱,故日長;夜行地下百四十六度強,故夜短。自夏至之後,日去極稍遠,故景稍長。日晝行地上度稍少,故日稍短;夜行地下度稍多,故夜稍長。日所在度稍南,故日出入稍南,
 以至於南至而復初焉。斗二十一,井二十五,南北相覺四十八度。



 春分日,在奎十四少強;秋分日,在角五少弱,此黃赤二道之交中也。去極俱九十一度少強,南北處斗二十一井二十五之中,故景居二至長短之中。奎十四,角五,出卯入酉,故日亦出卯入酉。日晝行地上,夜行地下,俱百八十二度半強。故日見之漏五十刻,不見之漏五十刻,謂之晝夜同。夫天之晝夜,以日出入為分;人之晝夜,以昏明為限。日未出二刻半而明,日已入二刻半
 而昏,故損夜五刻以益晝,是以春秋分之漏晝五十五刻。



 三光之行,不必有常,術家以算求之,各有同異,故諸家歷法參差不齊。《洛書甄耀度》、《春秋考異郵》皆云周天一百七萬一千里,一度為二千九百三十二里七十一步二尺七寸四分四百八十七分分之三百六十二。陸績云:天東西南北徑三十五萬七千里,此言周三徑一也。考之徑一不啻周三,率周百四十二而徑四十五,則天徑三十三萬九千四百一里一百二十二步三尺二
 寸一分七十一分分之九。



 《周禮》:「日至之景,尺有五寸,謂之地中。」鄭眾說:「土圭之長,尺有五寸。以夏至之日,立八尺之表,其景與土圭等,謂之地中,今潁川陽城地也。」



 鄭玄云:「凡日景於地千里而差一寸,景尺有五寸者,南戴日下萬五千里也。」以此推之,日當去其下地八萬里矣。日邪射陽城,則天徑之半也。天體圓如彈丸,地處天之半,而陽城為中,則日春秋冬夏,昏明晝夜,去陽城皆等,無盈縮矣。故知從日邪射陽城為天徑之半也。



 以句股
 法言之,傍萬五千里,句也;立八萬里,股也;從日邪射陽城,弦也。



 以句股求弦法入之,得八萬一千三百九十四里三十步五尺三寸六分,天徑之半,而地上去天之數也。倍之,得十六萬二千七百八十八里六十一步四尺七寸二分,天徑之數也。以周率乘之,徑率約之,得五十一萬三千六百八十七里六十八步一尺八寸二分,周天之數也。減《甄耀度》、《考異郵》五十五萬七千三百一十二里有奇。



 一度凡千四百六里百二十四步六寸四分
 十萬七千五百六十五分分之萬九千三十九,減舊度千五百二十五里二百五十六步三尺三寸二十一萬五千一百三十分分之十六萬七百三十分。黃赤二道,相與交錯,其間相去二十四度。以兩儀推之,二道俱三百六十五度有奇,是以知天體圓如彈丸。而陸績造渾象,其形如鳥卵,然則黃道應長於赤道矣。績云天東西南北徑三十五萬七千里,然則績亦以天形正圓也。而渾象為鳥卵,則為自相違背。



 古舊渾象以二分為一度,
 凡周七尺三寸半分。張衡更制,以四分為一度,凡周一丈四尺六寸。蕃以古制局小,星辰稠穊;衡器傷大,難可轉移。更制渾象,以三分為一度,凡周天一丈九寸五分四分分之三也。



 御史中丞何承天論渾象體曰:「詳尋前說,因觀渾儀,研求其意,有以悟天形正圓,而水周其下。言四方者,東暘谷,日之所出,西至濛汜,日之所入。莊子又云:『北溟之魚,化而為鳥,將徙於南溟。』斯亦古之遺記,四方皆水證也。四方皆水,謂之四海。凡五行相生,水生於金,
 是故百川發源,皆自山出,由高趣下,歸注於海。日為陽精,光耀炎熾,一夜入水,所經燋竭,百川歸注,足於補復,故旱不為減,浸不為益。徑天之數,蕃說近之。」



 太中大夫徐爰曰:「渾儀之制,未詳厥始。王蕃言:『《虞書》稱「在璇璣玉衡,以齊七政」。則今渾天儀日月五星是也。鄭玄說:「動運為機,持正為衡,皆以玉為之。視其行度,觀受禪是非也。」渾儀,羲和氏之舊器,歷代相傳,謂之機衡,其所由來,有原統矣。而斯器設在候臺,史官禁密,學者寡得聞見;穿
 鑿之徒,不解機衡之意,見有七政之言,因以為北斗七星,構造虛文,託之讖緯,史遷、班固,猶尚惑之。鄭玄有贍雅高遠之才,沈靜精妙之思,超然獨見,改正其說,聖人復出,不易斯言矣。』蕃之所云如此。夫候審七曜,當以運行為體,設器擬象,焉得定其盈縮,推斯而言,未為通論。設使唐、虞之世,已有渾儀,涉歷三代,以為定準,後世聿遵,孰敢非革。而三天之儀,紛然莫辯,至揚雄方難蓋通渾。張衡為太史令,乃鑄銅制範。衡傳云:『其作渾天儀,考
 步陰陽,最為詳密。』故知自衡以前,未有斯儀矣。蕃又云:『渾天遭秦之亂,師徒喪絕,而失其文,惟渾天儀尚在候臺。』案既非舜之璇玉,又不載今儀所造,以緯書為穿鑿,鄭玄為博實,偏信無據,未可承用。夫璇玉,貴美之名;機衡,詳細之目。所以先儒以為北斗七星,天綱運轉,聖人仰觀俯察,以審時變焉。」



 史臣案:設器象,定其恒度,合之則吉,失之則兇,以之占察,有何不可。渾文廢絕,故有宣、蓋之論,其術並疏,故後人莫述。揚雄《法言》云:「或人問渾
 天於雄。雄曰:『落下閎營之,鮮于妄人度之,耿中丞象之,幾乎莫之違也』。」



 若問天形定體,渾儀疏密,則雄應以渾儀答之,而舉此三人以對者,則知此三人制造渾儀,以圖晷緯。問者蓋渾儀之疏密,非問渾儀之淺深也。以此而推,則西漢長安已有其器矣。將由喪亂亡失,故衡復鑄之乎?王蕃又記古渾儀尺度並張衡改制之文,則知斯器非衡始造,明矣。衡所造渾儀,傳至魏、晉,中華覆敗,沈沒戎虜;績、蕃舊器,亦不復存。晉安帝義熙十四年,
 高祖平長安,得衡舊器,儀狀雖舉,不綴經星七曜。



 文帝元嘉十三年,詔太史令錢樂之更鑄渾儀,徑六尺八分少,周一丈八尺二寸六分少,地在天內,立黃赤二道,南北二極規二十八宿,北斗極星,五分為一度,置日月五星於黃道之上,置立漏刻,以水轉儀,昏明中星,與天相應。十七年,又作小渾天,徑二尺二寸,周六尺六寸,以分為一度,安二十八宿中外宮,以白黑珠及黃三色為三家星,日月五星,悉居黃道。



 蓋天之術,云出周公旦訪之
 殷商,蓋假託之說也。其書號曰周髀。髀者,表也,周天之數也。其術云:「天如覆蓋,地如覆盆,地中高而四隤,日月隨天轉運,隱地之高,以為晝夜也。天地相去凡八萬里,天地之中,高於外衡六萬里;地上之高,高於天之外衡二萬里也。」或問蓋天於揚雄。揚雄曰:「蓋哉!蓋哉!」難其八事。



 鄭玄又難其二事。為蓋天之學者,不能通也。劉向《五紀》說,《夏歷》以為列宿日月皆西移,列宿疾而日次之,月最遲。故日與列宿昏俱入西方;後九十一日,是宿在北
 方;又九十一日,是宿在東方;九十一日,在南方。此明日行遲於列宿也。



 月生三日,日入而月見西方;至十五日,日入而月見東方;將晦,日未出,乃見東方。以此明月行之遲於日,而皆西行也。向難之以《鴻範傳》曰:「晦而月見西方,謂之朓。朓,疾也。朔而月見東方,謂之側匿。側匿,遲不敢進也。星辰西行,史官謂之逆行。」此三說,《夏歷》皆違之,跡其意,好異者之所作也。



 晉成帝咸康中,會稽虞喜造《安天論》,以為「天高窮於無窮,地深測於不測。



 地有居
 靜之體,天有常安之形。論其大體,當相覆冒,方則俱方,圓則俱圓,無方圓不同之義也。」喜族祖河間太守聳又立《穹天論》云:「天形穹隆,當如雞子幕,其際周接四海之表,浮乎元氣之上。」而吳太常姚信造《昕天論》曰:「嘗覽《漢書》云:冬至日在牽牛,去極遠;夏至日在東井,去極近。欲以推日之長短,信以太極處二十八宿之中央,雖有遠近,不能相倍。」今《昕天》之說,以為「冬至極低,而天運近南。故日去人遠,而斗去人近;北天氣至,故冰寒也。夏至極起,而天
 運近北,而斗去人遠,日去人近,南天氣至,故炎熱也。極之立時,日行地中淺,故夜短;天去地高,故晝長也。極之低時,日行地中深,故夜長;天去地下淺,故晝短也。然則天行寒依於渾,夏依於蓋也。」按此說應作「軒昂」之「軒」,而作「昕」,所未詳也。凡三說,皆好異之談,失之遠矣。凡天文經星,常宿中外宮,前史已詳。今惟記魏文帝黃初以來星變為《天文志》,以續司馬彪云。



 魏文帝黃初三年九月甲辰,客星見太微左掖門內。占
 曰:「客星出太微,國有兵喪。」十月,孫權叛命,帝自南征,前驅臨江,破其將呂範等。是後累有征役。



 七年五月,文帝崩。



 黃初四年三月癸卯,月犯心大星。十二月丙子,月又犯心大星。占曰:「心為天王,王者惡之。」七年五月,文帝崩。黃初四年六月甲申,太白晝見。五年十一月辛卯,太白又晝見。案劉向《五紀論》曰:「太白少陰,弱,不得專行,故以己未為界,不得經天而行。經天則晝見,其占為兵,為喪,
 為不臣,為更王。強國弱,小國強。」是時,孫權受魏爵號,而稱兵距守。七年五月,文帝崩。八月,吳遂圍江夏,寇襄陽,魏江夏太守文聘固守得全。大將軍司馬懿救襄陽,斬吳將張霸。



 黃初四年十一月,月暈北斗。占曰:「有大喪,赦天下。」七年五月,文帝崩,明帝即位,大赦天下。黃初五年十月,歲星入太微,逆行積百三十九日乃出。占曰:「五星入太微,從右入三十日以上,人主有大憂。」一
 曰:「有赦至。」七年五月,文帝崩,明帝即位,大赦天下。



 黃初六年五月十六日壬戌,熒惑入太微,至二十六日壬申,與歲星相及,俱犯右執法;至二十七日癸酉,乃出。占曰:「從右入三十日以上,人主有大憂。」又「日月五星犯左右執法,大臣有憂。」一曰:「執法者誅。金火尤甚。」十一月,皇子東武陽王鑒薨。七年正月,驃騎將軍曹洪免為庶人。四月,征南大將軍夏侯尚薨。五月,文帝崩。《蜀記》稱:「明帝問黃權曰:『天下鼎立,何地為正?』對曰:『當驗天文。往熒
 惑守心,而文皇帝崩,吳、蜀無事,此其微也。」案三國史,並無熒惑守心之文,宜是入太微。黃初六年十月乙未,有星孛于少微,歷軒轅。案占,孛、彗異狀,其殃一也。為兵喪除舊布新之象,餘災不盡,為旱凶飢暴疾。長大見久災深;短小見速災淺。是時帝軍廣陵,辛丑,親御甲胄,跨馬觀兵。明年五月,文帝崩。



 魏明帝太和四年十一月壬戌,太白犯歲星。占曰:「太白犯五星,有大兵;犯列宿,為小兵。」五年三月,諸葛亮以大眾寇天
 水,遣大將軍司馬懿距退之。太和五年五月,熒惑犯房。占曰:「房四星,股肱臣將相位也。月五星犯守之,將相有憂。」七月,車騎將軍張郃追諸葛亮,為其所害。十二月,太尉華歆薨。太和五年十一月乙酉,月犯軒轅大星。占曰:「女主憂。」十二月甲辰,月犯鎮星。占曰:「女主當之。」六年三月乙亥,月又犯軒轅大星。青龍二年十一月乙丑,月又犯鎮星。三年正月,太后郭氏崩。



 太和六年十一月丙寅,太白晝見南斗,遂歷八十餘日恒見。占曰:「吳有兵。」



 明年,孫權遣張彌等將兵萬人,錫授公孫淵為燕王。淵斬彌等,虜其眾。太和六年十一月丙寅,有星孛于翼,近太微上將星。占曰:「為兵喪。」甘氏曰:「孛彗所當之國,是受其殃。」翼又楚分,孫權封略也。明年,權有遼東之敗。權又向合肥新城,遣全琮征六安,皆不克而還。又明年,諸葛亮入秦川,據渭南,司馬懿距之。



 孫權遣陸議、諸葛瑾等屯江夏口,孫韶、
 張承等向廣陵淮陽,權以大眾圍新城以應亮。於是帝自東征,權及諸將乃退。太和六年十二月,陳王植薨。青龍元年夏,北海王蕤薨。三年正月,太后郭氏崩。



 明帝青龍二年二月己未,太白犯熒惑。占曰:「大兵起,有大戰。」是年四月,諸葛亮據渭南,吳亦起兵應之,魏東西奔命。九月,亮卒,軍退,將帥分爭,為魏所破。案占,太白所犯在南,南國敗,在北,北國敗,此宜在熒惑南也。青龍二年三月辛卯,月犯輿鬼。輿鬼主斬殺。占曰:「民多
 病,國有憂,又有大臣憂。」是年夏,大疫;冬,又大病,至三年春乃止。正月,太后郭氏崩。四年五月,司徒董昭薨。青龍二年五月丁亥,太白晝見,積三十餘日。以晷度推之,非秦、魏,則楚也。是時諸葛亮據渭南,司馬懿與相持。孫權寇合肥,又遣陸議、孫韶等入淮、沔,帝親東征。蜀本秦地,則為秦、晉及楚兵悉起應占。青龍二年七月己巳,月犯楗閉。



 占曰:「天子崩,又為火災。」三年七月,崇華殿災。景初三年正月,明帝崩。
 青龍二年十月戊寅,月犯太白。占曰:「人君死,又為兵。」景初元年七月,公孫淵叛。二年正月,遣司馬懿討之。三年正月,明帝崩。



 蜀後主建興十二年,諸葛亮帥大眾伐魏,屯于渭南,有長星赤而芒角,自東北,西南流投亮營,三投再還,往大還小。占曰:「兩軍相當,有大流星來走軍上及墜軍中者,皆破敗之徵也。」九月,亮卒于軍,焚營而退。群帥交惡,多相誅殘。



 魏明帝青龍三年六月丁未,鎮星犯井鉞。四年閏四月乙巳,復犯。戊戌,太白又犯。占曰:「凡月五星犯井鉞,悉為兵起。」一曰:「斧鉞用,大臣誅。」景初元年,公孫淵叛,司馬懿討滅之。青龍三年七月己丑,鎮星犯東井。四年三月癸卯,在參,又還犯之。占曰:「鎮星入井,大人憂。行近距為行陰,其占大水,五穀不成。」景初元年夏,大水,傷五穀。九月,皇后毛氏崩。三年正月,明帝崩。
 青龍三年十月壬申,太白晝見在尾,歷二百餘日恒見。占曰:「尾為燕,燕臣強,有兵。」



 青龍四年三月己巳,太白與月俱加丙,晝見。月犯太白。景初元年七月辛卯,太白又晝見,積二百八十餘日。占悉同上。是時公孫淵自立為燕王,署置百官,發兵距守,遣司馬懿討滅之。青龍三年十二月戊辰,月犯鉤鈐。占曰:「王者憂。」景初三年正月,明帝崩。



 青龍四年五月壬寅,太白犯畢左股第一星。占曰:「畢為邊兵,又主刑罰。」



 九月,涼州塞外胡阿畢師侵犯諸國,西域校尉張就討之,斬首捕虜萬許人。青龍四年七月甲寅,太白犯軒轅大星。占曰:「女主憂。」景初元年,皇后毛氏崩。青龍四年十月甲申,有星孛于大辰,長三尺。乙酉,又孛于東方。十一月己亥,彗星見,犯宦者天紀星。占曰:「大辰為天王,天下有喪。」劉向《五紀論》曰:「《春秋》星孛于東方,不
 言宿者,不加宿也。」宦者在天市為中外有兵,天紀為地震。孛彗主兵喪。景初元年六月,地震。九月,吳將朱然圍江夏,荊州刺史胡質擊走之。皇后毛氏崩。二年正月,討公孫淵。三年正月,明帝崩。



 魏明帝景初元年二月乙酉,月犯房第二星。占曰:「將相有憂。」七月,司徒陳矯薨。二年四月,司徒韓暨薨。景初元年十月丁未,月犯熒惑。占曰:「貴人死。」



 二年四月,司徒韓暨薨。八月,公孫淵滅。



 景初二年二月癸丑,月犯心距星,又犯中央大星。五月己亥,又犯心距星及中央大星。閏月癸丑,月又犯心、中央大星。按占,「大星為天王,前為太子,後為皇子。犯大星,王者惡之。犯前星,太子有憂。犯後星,庶子有憂。」三年正月,帝崩,太子立,卒見廢為齊王。正始四年,秦王詢薨。景初二年八月彗星見張,長三尺,逆西行,四十一日滅。占曰:「為兵喪。張,周分野,洛邑惡之。」其十月,斬公孫淵。明年正月,明帝崩。
 景初二年十月甲午,月犯箕。占曰:「軍將死。」



 正始元年四月,車騎將軍黃權薨。景初二年,司馬懿圍公孫淵於襄平。八月丙寅夜,有大流星長數十丈,色白有芒鬣,從首山北流墜襄平城東南。占曰:「圍城而有流星來走城上及墜城中者破。」又曰:「星墜,當其下有戰場。」又曰:「凡星所墜,國易姓。」九月,淵突圍,走至星墜所被斬,屠城坑其眾。景初二年十月癸巳,客星見危,逆行在離宮北,騰蛇南。
 甲辰,犯宗星。己酉,滅。占曰:「客星所出有兵喪。虛危為宗廟,又為墳墓。客星近離宮,則宮中將有大喪,就先君於宗廟,皆王者崩殞之象也。」三年正月,明帝崩。正始二年五月,吳將朱然圍樊城,司馬懿率眾距卻之。



 魏齊王正始元年四月戊午,月犯昴東頭第一星。其年十月庚寅,月又犯昴北頭第四星。占曰:「犯昴,胡不安。」二年六月,鮮卑阿妙兒等寇西方,敦煌太守王延斬之,并二千餘級。三年,又斬鮮卑大帥及千餘級。
 正始元年十月乙酉,彗星見西方,在尾,長三丈,拂牽牛,犯太白。十一月甲子,進犯羽林。占曰:「尾為燕,又為吳,牛亦吳、越之分。太白為上將,羽林中軍兵。吳、越有兵喪,中軍兵動。」



 二年五月,吳將全琮寇芍陂,朱然圍樊城,諸葛瑾入沮中。吳太子登卒。六月,司馬懿討諸葛恪於皖。恪焚積聚,棄城走。三年,太尉滿寵薨。



 正始二年九月癸酉,月犯輿鬼西北星。西北星主金。三年二月丁未,又犯西南星。西南星主布帛。占曰:「有錢令。」
 一曰:「大臣憂。」三年三月,太尉滿寵薨。四年正月,帝加元服,賜群臣錢各有差。



 正始四年十月、十一月,月再犯井鉞。是月,司馬懿討諸葛恪,恪棄城走。五年三月,曹爽征蜀。正始五年十一月癸巳,鎮星犯亢距星。占曰:「諸侯有失國者。」



 嘉平元年,曹爽兄弟誅。



 正始六年八月戊午,彗星見七星,長二尺,色白,進至張,積二十三日滅。七年十一月癸亥,又見軫,長一尺,積百
 五十六日滅。九年三月,又見昴,長六尺,色青白,芒西南指。七月,又見翼,長二尺,進至軫,積四十二日滅。按占,「七星、張,周分野,翼、軫為楚,昴為趙、魏,彗所以除舊布新,主兵喪也。」嘉平元年,司馬懿誅曹爽兄弟及其黨與,皆夷族,京師嚴兵,實始翦魏。三年,誅楚王彪,又襲王凌於淮南。淮南,東楚也。幽魏諸王于鄴。



 正始七年七月丁丑,月犯左角。占曰:「天下有兵,將軍死。」九年正月辛亥,月犯亢南星。占曰:「兵起。」一曰:「軍將死。」七
 月乙亥,熒惑犯畢距星。占曰:「有邊兵。」一曰:「刑罰用。」嘉平元年,曹爽等誅。三年,王凌等又誅。



 正始九年七月癸丑,鎮星犯楗閉。占曰:「王者不宜出宮下殿。」明年,車駕謁陵,司馬懿奏誅曹爽等,天子野宿,於是失勢。



 魏齊王嘉平元年六月壬戌,太白犯東井距星。二年三月己未,又犯。占曰:「國失政,大臣為亂。」四月辛巳,太白犯輿鬼。占曰:「大臣誅。」一曰:「兵起。」三年七月,王凌與楚王彪
 有謀,皆伏誅。人主遂卑。



 吳主孫權赤烏十三年五月,日北至,熒惑逆行入南斗。七月,犯魁第二星而東。



 《漢晉春秋》云逆行。按占,熒惑入南斗,三月,吳王死。一曰:「熒惑逆行,其地有死君。」太元二年權薨,是其應也。故國志書於吳而不書於魏也。是時,王凌謀立楚王彪,謂斗中有星,當有暴貴者,以問知星人浩詳。詳疑有故,欲說其意,不言吳有死喪,而言淮南楚分,吳、楚同占,當有王者興,故凌計遂定。



 魏齊王嘉平二年十月丙申,月犯輿鬼。占曰:「國有憂。」一曰:「大臣憂。」



 三年四月戊寅,月犯東井。占曰:「軍將死。」一曰:「國有憂。」五月,王凌、楚王彪等誅。七月,皇后甄氏崩。



 嘉平三年五月甲寅,月犯距星。占曰:「將軍死。」一曰:「為兵。」是月,王凌誅。四年三月,吳將朱然、硃異為寇,鎮東將軍諸葛誕破走之。嘉平三年七月己巳,月犯輿鬼。九月乙己,又犯。四年十一月丁未,又犯鬼積尸。五年七月丙午,月又犯鬼西北
 星。占曰:「國有憂。」正元元年,李豐等誅,皇后張氏廢。九月,帝廢為齊王。齊王嘉平三年十月癸未,熒惑犯亢南星。占曰:「大臣有亂。」正元元年二月,李豐等謀亂誅。嘉平三年十一月癸未,有星孛于營室,西行積九十日滅。



 占曰:「有兵喪。室為後宮,後宮且有亂。」四年二月丁酉,彗星見西方,在胃,長五六丈,色白,芒南指貫參,積二十日滅。五年十一月,彗星又見軫,長五丈,在太微左執法
 西,東南指,積百九十日滅。按占,「胃,兗州之分,參白虎主兵,太微天子廷,執法為執政,孛彗為兵,除舊布新之象。」正元元年二月,李豐、豐弟兗州刺史翼、后父光祿大夫張緝等謀亂,皆誅,皇后亦廢。九月,帝廢為齊王,高貴鄉公代立。



 嘉平五年六月庚辰,月犯箕。占曰:「軍將死。」正元元年正月,鎮東將軍母丘儉反,兵敗死。嘉平五年六月戊午,太白犯角。占曰:「群臣謀不成。」正元
 元年,李豐等謀泄,悉誅。嘉平五年七月,月犯井鉞。正元元年二月,李豐等誅。蜀將姜維攻隴西,車騎將軍郭淮討破之。嘉平五年十一月癸酉,月犯東井距星。占曰:「軍將死。」至六年正月,鎮東將軍豫州刺史毋丘儉、前將軍揚州刺史文欽反,被誅。



 魏高貴鄉公正元元年十一月,有白氣出斗側,廣數丈,長竟天。王肅曰:「蚩尤之旂也,東南其有亂乎!」二年正月,
 毋丘儉等據淮南以叛,大將軍司馬師討平之。案占,「蚩尤旗見,王者征伐四方。」自後又征淮南,西平巴蜀。是歲,吳主孫亮五鳳元年,斗牛,吳、越分。案占:「有兵喪,除舊布新之象也。」太平三年,孫綝盛兵圍宮,廢亮為會稽王,孫休代立,是其應也。故國志又書於吳。由是淮南江東同揚州地,故于時變見吳、楚之分。則魏之淮南,多與吳同災,是以毋丘儉以孛為己應,遂起兵而敗,又其應也。後三年,即魏甘露二年,諸葛誕又反淮南,吳遣硃異救之。
 及城陷,誕眾吳兵死沒各數萬人,猶前長星之應也。高貴鄉公正元二年二月戊午,熒惑犯東井北轅西頭第一星。占曰:「群臣有家坐罪者。」甘露元年,諸葛誕族滅。



 吳孫亮太平元年九月壬辰,太白犯南斗,《吳志》所書也。占曰:「太白犯斗,國有兵,大臣有反者。」其明年,諸葛誕反。又明年,孫琳廢亮,吳、魏並有兵事也。



 魏高貴鄉公甘露元年七月乙卯,熒惑犯井鉞。壬戌,月又犯鉞星。二年八月壬子,歲星犯井鉞。九月庚寅,歲星
 又逆行乘鉞星。三年,諸葛誕夷滅。甘露元年八月辛亥,月犯箕。占曰:「軍將死。」九月丁巳,月犯東井。占曰:「軍將死。」



 二年,諸葛誕誅。



 甘露二年六月己酉,月犯心中央大星。景元元年五月,高貴鄉公敗。甘露二年十月丙寅,太白犯亢距星。占曰:「廷臣為亂,人君憂。」景元元年,有成濟之變。



 甘露二年十一月,彗星見角,色白。占曰:「彗見兩角間,色
 白者,軍起不戰,邦有大喪。」景元元年,高貴鄉公帥左右兵襲晉文王,未交戰,為成濟所害。



 甘露三年三月庚子,太白犯東井。占曰:「國失政,大臣為亂。」是夜,歲星又犯東井。占曰:「兵起。」至景元元年,高貴鄉公敗。甘露三年八月壬辰,歲星犯輿鬼質星。占曰:「斧質用,大臣誅。」甘露四年四月甲申,歲星又犯輿鬼東南星。占曰:「鬼東南星主兵。木入鬼,大臣誅。」景元元年,高貴鄉公敗,
 殺尚書王經。



 甘露四年十月丁丑,客星見太微中,轉東南行,歷軫宿,積七日滅。占曰:「客星出太微,有兵喪。」景元元年,高貴鄉公被害。



 魏陳留王景元元年二月,月犯建星。案占,「月五星犯建星,大臣相譖」。是後鐘會、鄧艾破蜀,會譖艾,遂皆夷滅。



 景元二年四月,熒惑入太微,犯右執法。占曰:「人主有大憂。」又曰:「大臣憂。」後四年,鄧艾、鐘會皆夷滅。五年,帝遜位。



 景元三年十一月壬寅,彗星見亢,色白,長五寸,轉北行,積四十五日滅。占為兵喪。一曰:「彗見亢,天子失德。」四年,鐘會、鄧艾伐蜀克之。會、艾反亂,皆誅。魏遜天下。



 景元四年六月,大流星二,並如斗,見西方,分流南北,光照隆隆有聲。案占,流星為貴使,大者使大。是年,鐘、鄧克蜀,二星蓋二帥之象。二帥相背,又分流南北之應。鐘會既叛,三軍憤怒,隆隆有聲,兵將怒之徵也。景元四年十月,歲星守房。占曰:「將相有憂。」一曰:「有大赦。」
 明年正月,太尉鄧艾、司徒鐘會並誅滅,特赦益土。咸熙二年秋,又大赦。



 陳留王咸熙二年五月,彗星見王良,長丈餘,色白,東南指,積十二日滅。占曰:「王良,天子御駟,彗星掃之,禪代之表,除舊布新之象。白色為喪。王良在東壁宿,又并州之分也。」八月,晉文王薨。十二月,帝遜位于晉。



 晉武帝泰始四年正月丙戌,彗星見軫,青白色,西北行,又轉東行。占曰:「為兵喪。軫又楚分也。」三月,皇太后王氏崩。
 十月,吳將施績寇江夏,萬彧寇襄陽,後將軍田璋、荊州刺史胡烈等破卻之。泰始四年七月,星隕如雨,皆西流。



 占曰:「星隕為民叛,西流,吳民歸晉之象也。」二年,吳夏口督孫秀率部曲二千餘人來降。



 泰始五年九月,有星孛於紫宮,占如上。紫宮,天子內宮。十年,武元楊皇后崩。



 泰始十年十二月,有星孛于軫。占曰:「天下兵起。軫又楚
 分也。」咸寧二年六月,星孛于氐。占曰:「天子失德易政。氐又兗州分。」七月,星孛大角。大角為帝坐。八月,星孛太微,至翼、北斗、三臺。占曰:「太微天子廷,大人惡之。」



 一曰:「有徙王。翼又楚分也。」「北斗主殺罰,三臺為三公。」三年,星孛于胃。



 胃,徐州分。四月,星孛女御。女御為後宮。五月,又孛于東方。七月,星孛紫宮。



 占曰:「天下易主。」五年三月,星孛于柳。占曰:「外臣陵主。柳又三河分也。



 大角、太微、紫宮、女御,並為王者。」明年吳亡,是其應也。孛主兵喪,征吳之役,三
 河、徐、兗之兵悉出,交戰於吳、楚之地。吳丞相都督以下,梟戮十數,偏裨行陣之徒,馘斬萬計,皆其徵也。《春秋》星孛北方,則齊、魯、晉、鄭、陳、宋、莒之君,並受殺亂之禍。星孛東方,則楚滅陳,三家、田氏分篡齊、晉。漢文帝末,星孛西方,後吳、楚七國誅滅。案泰始末至太康初,災異數見,而晉氏隆盛,吳實滅,天變在吳可知矣。昔漢三年,星孛大角,項籍以亡,漢氏無事,此項氏主命故也。吳、晉之時,天下橫分,大角孛而吳亡,是與項氏同事。後學皆以咸寧
 災為晉室,非也。



 晉武帝咸寧四年四月,蚩尤旗見。案《星傳》,蚩尤旗類彗,而後曲象旗。漢武帝時見,長竟天。獻帝時又見,長十餘丈,皆長星也。魏高貴時則為白氣。案校眾記,是歲無長星,宜又是異氣。後二年,傾三方伐吳,是其應。至武帝崩,天下兵又起,遂亡諸夏。咸寧四年九月,太白當見不見。占曰:「是謂失舍,不有破軍,必有死王之墓。又有亡國。」是時羊祜表求伐吳,上許
 之。五年十一月,兵出,太白始夕見西方。太康元年三月,大破吳軍,孫皓面縛請死,吳國遂亡。



 晉武帝太康二年八月,有星孛于張。占曰:「為兵喪。」周分野,災在洛邑。



 十一月,星孛軒轅。占曰:「後宮當之。」四年三月戊申,星孛于西南。四年三月癸丑,齊王攸薨。四月戊寅,任城王陵薨。五月己亥,琅邪王伷薨。十一月戊午,新都王該薨。



 太康八年三月,熒惑守心。占曰:「王者惡之。」太熙元年四
 月己酉,武帝崩。



 太康八年九月,星孛於南斗,長數十丈,十餘日滅。占曰:「斗主爵祿,國有大憂。」



 一曰:「孛於斗,王者疾病,臣誅其父,天下易政,大亂兵起。」太熙元年四月,客星在紫宮。占曰:「為兵喪。」太康末,武帝耽宴游,多疾病。是月乙酉,帝崩。



 永平元年,賈後誅楊駿及其黨與,皆夷三族;楊太后亦見殺。是年,又誅汝南王亮、太保衛瓘、楚王瑋,王室兵喪之
 應。



\end{pinyinscope}