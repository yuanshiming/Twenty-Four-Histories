\article{卷七十七列傳第三十七 柳元景 顏師伯 沈慶之}

\begin{pinyinscope}

 柳元景,字孝仁,河東解人也。曾祖卓,自本郡遷於襄陽,官至汝南太守。祖恬,西河太守。父憑,馮翊太守。元景少便弓馬,數隨父伐蠻,以勇稱。寡言有器質。荊州刺史謝
 晦聞其名,要之,未及往而晦敗。雍州刺史劉道產深愛其能,元景時居父憂,未得加命。會荊州刺史江夏王義恭召之,道產謂曰:「久規相屈。今貴王有召,難輒相留,乖意以為惘惘。」服闋,補江夏王國中軍將軍,遷殿中將軍。



 復為義恭司空行參軍,隨府轉司徒太尉城局參軍,太祖見又嘉之。



 先是,劉道產在雍州有惠化,遠蠻悉歸懷,皆出緣沔為村落,戶口殷盛。及道產死,群蠻大為寇暴。世祖西鎮襄陽,義恭以元景為將帥,即以為廣威將軍、
 隨郡太守。既至,而蠻斷驛道,欲來攻郡。郡內少糧,器杖又乏,元景設方略,得六七百人,分五百人屯驛道。或曰:「蠻將逼城,不宜分眾。」元景曰:「蠻聞郡遣重戍,豈悟城內兵少。且表裏合攻,於計為長。」會蠻垂至,乃使驛道為備,潛出其後,戒曰:「火舉馳進。」前後俱發,蠻眾驚擾,投鄖水死者千餘人,斬獲數百,郡境肅然,無復寇抄。朱脩之討蠻,元景又與之俱,後又副沈慶之徵鄖山,進克太陽。除世祖安北府中兵參軍。



 隨王誕鎮襄陽,為後軍中兵參
 軍。及朝廷大舉北討,使諸鎮各出軍。二十七年八月,誕遣振威將軍尹顯祖出貲谷,奮武將軍魯方平、建武將軍薛安都、略陽太守龐法起入盧氏,廣威將軍田義仁入魯陽,加元景建威將軍,總統群帥。後軍外兵參軍龐季明年已七十三,秦之冠族,羌人多附之,求入長安,招懷關、陜。乃自貲谷入盧氏,盧氏人趙難納之,弘農強門先有內附意,故委季明投之。十月,魯方平、薛安都、龐法起進次白亭,時元景猶未發。法起率方平、安都諸軍前
 入,自脩陽亭出熊耳山。季明進達高門木城,值永昌王入弘農,乃回,還盧氏,據險自固。頃之,招盧氏少年進入宜陽茍公谷,以扇動義心。元景以其月率軍繼進。閏月,法起、安都、方平諸軍入盧氏,斬縣令李封,以趙難為盧氏令,加奮武將軍。難驅率義徒,以為眾軍鄉導。法起等度鐵嶺山,次開方口,季明出自木城,與法起相會。元景大軍次臼口,以前鋒深入,懸軍無繼,馳遣尹顯祖入盧氏,以為軍援。元景以軍食不足,難可曠日相持,乃束馬
 懸車,引軍上百丈崖,出溫谷,以入盧氏。



 法起諸軍進次方伯堆,去弘農城五里。賊遣兵二千餘人覘候,法起縱兵夾射之,賊騎退走。諸軍造攻具,進兵城下,偽弘農太守李初古拔嬰城自固,法起、安都、方平諸軍鼓噪以陵城,季明、趙難並率義徒相繼而進,衝車四臨,數道俱攻,士皆殊死戰,莫不奮勇爭先。時初古拔父子據南門,督其處距戰,弘農人之在城內者三千餘人,於北樓豎白幡,或射無金箭。安都軍副譚金、薛係孝率眾先登,生禽
 李初古拔父子二人,魯方平入南門,生禽偽郡丞,百姓皆安堵。



 元景引軍度熊耳山,安都頓軍弘農,法起進據潼關,季明率方平、趙難軍向陜西七里谷。殿中將軍鄧盛、幢主劉驂亂使人入荒田,招宜陽人劉寬糾率合義徒二千餘人,共攻金門鄔,屠之。殺戍主李買得,古拔子也,為虜永昌王長史,勇冠戎類。



 永昌聞其死,若失左右手。誕又遣長流行參軍姚範領三千人向弘農,受元景節度。



 十一月,元景率眾至弘農,營於開方口。仍以元景
 為弘農太守,置吏佐。



 初,安都留住弘農,而諸軍已進陜,元景既到,謂安都曰:「無為坐守空城,而令龐公深入,此非計也。宜急進軍,可與顯祖並兵就之。吾須督租畢,尋後引也。」



 眾並造陜下,即入郭城,列營於城內以逼之,並大造攻具。賊城臨河為固,恃險自守,季明、安都、方平、顯祖、趙難諸軍,頻三攻未拔。虜洛州刺史地河公張是連提眾二萬,度崤來救,安都、方平各列陣城南以待之,顯祖勒精卒以為後柱。季明率高明、宜陽義兵當南門而陣,
 趙難領盧氏樂從少年,與季明為掎角。賊兵大合,輕騎挑戰。安都瞋目橫矛,單騎突陣,四向奮擊,左右皆辟易不能當,殺傷不可勝數,於是眾軍並鼓噪俱前,士皆殊死戰。虜初縱突騎,眾軍患之。安都怒甚,乃脫兜鍪,解所帶鎧,唯著絳衲兩當衫,馬亦去具裝,馳奔以入賊陣,猛氣咆勃,所向無前,當其鋒者,無不應刃而倒。賊忿之,夾射不能中,如是者數四,每一入,眾無不披靡。



 初,元景令將魯元保守函谷關,賊眾既盛,元保不能自固,乃率所
 領作函箱陣,多列旗幟,緣險而還。正會安都諸軍與賊交戰,虜三郎將見元保軍從山下,以為元景大眾至,日且暮,賊於是奔退,騎多得入城。



 賊之將至也,方平遣驛騎告元景,時諸軍糧盡,各餘數日食。元景方督義租,並上驢馬,以為運糧之計。而方平信至,元景遣軍副柳元怙簡步騎二千,以赴陜急,卷甲兼行,一宿而至。詰朝,賊眾又出,列陣於城外。方平諸軍並成列,安都並領馬軍,方平悉勒步卒,左右掎角之,餘諸義軍並於城西南列陳。
 方平謂安都曰:「今勍敵在前,堅城在後,是吾取死之日。卿若不進,我當斬卿;我若不進,卿當斬我也。」安都曰:「善,卿言是也。我豈惜身命乎!」遂合戰。時元怙方至,悉偃旗鼓,士馬皆銜枚,潛師伏甲而進,賊未之覺也。方平等方與虜交鋒,而元怙勒眾從城南門函道直出,北向結陳,旌旗甚盛,彭噪而前,出賊不意,虜眾大駭。元怙與幢主宗越,率手下猛騎,以沖賊陳,一軍皆馳之。安都、方平等督諸軍一時齊奮,士卒無不用命。安都不堪其憤,橫矛
 直前,出入賊陳,殺傷者甚多,流血凝肘,矛折,易之復入。軍副譚金率騎從而奔之。自詰旦而戰,至於日昃,虜眾大潰,斬張是提,又斬三千餘級,投河赴塹死者甚眾,面縛軍門者二千餘人。



 元景輕騎晨至,虜兵之面縛者多河內人,元景詰之曰:「汝等怨王澤不浹,請命無所,今並為虜盡力,便是本無善心。順附者存拯,從惡者誅滅,欲知王師正如此爾。」皆曰:「虐虜見驅,後出赤族,以騎蹙步,未戰先死,此親將軍所見,非敢背中國也。」諸將欲盡殺
 之,元景以為不可,曰:「今王旗北掃,當令仁聲先路。」



 乃悉釋而遣之,家在關裏者,符守關諸軍聽出,皆稱萬歲而去。誕以崤、陜既定,其地宜撫,以弘農劉寬虯行東弘農太守。給元景鼓吹一部。



 法起率眾次於潼關,先是,建義將軍華山太守劉槐糾合義兵攻關城,拔之,力少不固。頃之,又集眾以應王師,法起次潼關,槐亦至。賊關城戍主婁須望旗奔潰,虜眾溺於河者甚眾。法起與槐即據潼關。虜蒲城鎮主遣偽帥何難於封陵堆列三營以擬
 法起。法起長驅入關,行王、檀故壘。虜謂直向長安,何難率眾欲濟河以截軍後,法起回軍臨河,縱兵射之,賊退散。關中諸義徒並處處鋒起,四山羌、胡咸皆請奮。



 誕又遣揚武將軍康元撫領二千人出上洛,受元景節度,援方平於函谷。元景去,賊眾向關。時軍中食盡,元景回據白楊嶺,賊定未至,更下山進弘農,入湖關口,虜蒲阪戍主沃州刺史杜道生率眾二萬至閿鄉水,去湖關一百二十里。元景募精勇一千人,夜斫賊營,迷失道,天曉而
 反。道生率手下驍銳縱兵射之,鋒刃既交,虜又奔散。



 時北討諸軍王玄謨等敗退,虜遂深入。太祖以元景不宜獨進,且令班師。元景乃率諸將自湖關度白楊嶺,出于長洲,安都斷後,宗越副之。法起自潼關向商城,與元景會;季明亦從胡谷南歸,並有功而入,士馬旌旗甚盛。誕登城望之,以鞍下馬迎元景。除寧朔將軍、京兆、廣平二郡太守,於樊城立府舍,率所領居之,統行北蠻事。龐季明為定蠻長,薛安都為後軍行參軍,魯方平為寧蠻參
 軍。臧質為雍州,除元景為冠軍司馬、襄陽太守,將軍如故。魯爽向虎牢,復使元景率安都等北出至關城,關城棄戍走,即據之。元景至洪關,欲進與安都濟河攻杜道生於蒲阪,會爽退,復還。再出北討,威信著於境外。又使率所領進西陽,會伐五水蠻。



 世祖入討元凶,以為諮議參軍,領中兵,加冠軍將軍,太守如故。配萬人為前鋒,宗愨、薛安都等十三軍皆隸焉。元景與朝士書曰:「國禍冤深,凶人肆逆,民神崩憤,若無天地。南中郎親率義師,剪討
 元惡,司徒、臧冠軍並同大舉,舳艫千里,購賞之利備之。元景不武,忝任行間,總勒精勇,先鋒道路,勢乘上流,眾兼百倍。諸賢弈世忠義,身為國良,皆受遇先朝,荷榮日久,而拘逼寇廷,莫由申效,想聞今問,悲慶兼常。大行屆道,廓清惟始,企遲面對,展雪哀情。」



 時義軍船率小陋,慮水戰不敵,至蕪湖,元景大喜,倍道兼行,聞石頭出戰艦,乃於江寧步上,於板橋立柵以自固。進據陰山,遣薛安都率馬軍至南岸,元景潛至新亭,依山建壘,東西據險。
 世祖復遣龍驤將軍、行參軍程天祚率眾赴之。天祚又於東南據高丘,屯寨柵。凡歸順來奔者,皆勸元景速進,元景曰:「不然。理順難恃,同惡相濟,輕進無防,實啟寇心。當倚我之不可勝,豈幸寇之不攻哉!」元景壘營未立,為龍驤將軍詹叔兒覘知之,勸劭出戰,不許。經日,乃水陸出軍,劭自登朱雀門督戰。軍至瓦官寺,與義軍游邏相逢,游邏退走,賊遂薄壘。劭以元景壘塹未立,可得平地決戰,既至,柴柵已堅,倉卒無攻具,便使肉薄攻之。元景
 宿令軍中曰:「鼓繁氣易衰,叫數力易竭。但各銜枚疾戰,一聽吾營鼓音。」賊步將魯秀、王羅漢、劉簡之、騎將常伯與等及其士卒,皆殊死戰。劉簡之先攻西南,頻得燒草舫,略渡人。程天祚柴未立,亦為所摧。王羅漢等攻壘北門,賊艦亦至。元景水陸受敵,意氣彌彊,麾下勇士悉遣出戰,左右唯留數人宣傳。分軍助程天祚,天祚還得固柴,因此破賊。元景察賊衰竭,乃命開壘,鼓噪以奔之,賊眾大潰,透淮死者甚多。劭更率餘眾自來攻壘,復大破
 之,其所殺傷,過於前戰。劭手斬退者不能禁,奔還宮,僅以身免,蕭斌被創。簡之收兵而止,陳猶未散。元景復出薄之,乃走,競投死馬澗,澗為之滿,斬簡之及軍主姚叔藝、王江寶、朱明智、諸葛邈之等,水軍主褚湛之、副劉道存並來歸順。



 上至新亭即位,以元景為侍中,領左衛將軍,轉使持節、監雍、梁、南北秦四州、荊州之竟陵、隨二郡諸軍事、前將軍、寧蠻校尉、雍州刺史。上在巴口,問元景:「事平,何所欲?」對曰:「若有過恩,願還鄉里。」故有此授。初,臧
 質起義,以南譙王義宣暗弱易制,欲相推奉,潛報元景,使率所領西還。元景即以質書呈世祖,語其使曰:「臧冠軍當是未知殿下義舉爾。方應伐逆,不容西還。」質以此恨之。及元景為雍州刺史,質慮其為荊、江後患,建議爪牙不宜遠出。上重違其言,更以元景為護軍將軍,領石頭戍事,不拜。徙領軍將軍,加散騎常侍,曲江縣公,食邑三千戶。



 孝建元年正月,魯爽反,遣左衛將軍王玄謨討之,加元景撫軍,假節置佐,後玄謨。復以為都督雍、梁、南
 北秦四州、荊州之竟陵、隨二郡諸軍事、撫軍將軍、領寧蠻校尉、雍州刺史,持節如故。臧質、義宣並反,玄謨南據梁山,夾江為壘,垣護之、薛安都渡據歷陽,元景出屯采石。玄謨聞賊盛,遣司馬管法濟求益兵,上使元景進屯姑孰。元景使將武念前進,質遣將龐法起襲姑孰,值念至,擊破之,法起單船走。質攻陷玄謨西壘,玄謨使垣護之告元景曰:「今餘東岸萬人,賊軍數倍,強弱不敵,謂宜還就節下協力當之。」元景謂護之曰:「師有常刑,不可先
 退。賊眾雖多,猜而不整,今當卷甲赴之。」護之曰:「逆徒皆云南州有三萬人,而麾下裁十分之一,若往造賊,虛實立見,則賊氣成矣。」元景納其言,悉遣精兵助玄謨,以羸弱居守。所遣軍多張旗幟,梁山望之如數萬人,皆曰:「京師兵悉至。」於是克捷。



 上遣丹陽尹顏竣宣旨慰勞,與沈慶之俱以本號開府儀同三司,封晉安郡公,邑如故。固讓開府儀同,復為領軍、太子詹事,加侍中。尋轉驃騎將軍、本州大中正,領軍、侍中如故。大明二年,復加開府儀
 同三司,又固讓。明年,遷尚書令,太子詹事、侍中、中正如故。以封在嶺南,秋輸艱遠,改封巴東郡公。五年,又命左光祿大夫、開府儀同三司,侍中、令、中正如故。又讓開府,乃與沈慶之俱依晉密陵侯鄭袤不受司空故事,事在《慶之傳》。六年,進司空,侍中、令、中正如故,又固讓,乃授侍中、驃騎將軍、南兗州刺史,留衛京師。世祖晏駕,與太宰江夏王義恭、尚書僕射顏師伯並受遺詔輔幼主。遷尚書令,領丹陽尹,侍中、將軍如故,給班劍二十人,固辭班
 劍。



 元景起自將帥,及當朝理務,雖非所長,而有弘雅之美。時在朝勳要,多事產業,唯元景獨無所營。南岸有數十畝菜園,守園人賣得錢二萬送還宅,元景曰:「我立此園種菜,以供家中啖爾。乃復賣菜以取錢,奪百姓之利邪!」以錢乞守園人。



 世祖嚴暴異常,元景雖荷寵遇,恒慮及禍。太宰江夏王義恭及諸大臣,莫不重足屏氣,未嘗敢私往來。世祖崩,義恭、元景等並相謂曰:「今日始免橫死。」義恭與義陽等諸王,元景與顏師伯等,常相馳逐,聲
 樂酣酒,以夜繼晝。



 前廢帝少有凶德,內不能平,殺戴法興後,悖情轉露。義恭、元景等憂懼無計,乃與師伯等謀廢帝立義恭,日夜聚謀,而持疑不能速決。永光年夏,元景遷使持節、督南豫之宣城諸軍事、即本號開府儀同三司、南豫州刺史,侍中、令如故。未拜,發覺,帝親率宿衛兵自出討之。先稱詔召元景,左右奔告兵刃非常,元景知禍至,整朝服,乘車應召。出門逢弟車騎司馬叔仁,戎服率左右壯士數十人欲拒命,元景苦禁之。既出巷,軍士
 大至,下車受戮,容色恬然,時年六十。



 長子慶宗,有幹力,而情性不倫,世祖使元景送還襄陽,於道中賜死。次子嗣宗,豫章王子尚車騎從事中郎。嗣宗弟紹宗、茂宗、孝宗、文宗、仲宗、成宗、秀宗。叔仁弟衛軍諮議參軍僧珍等諸弟姪在京邑及襄陽從死者數十人。元景少子承宗,及嗣宗子纂,並在孕獲全。太宗即位,令曰:「故侍中、尚書令、驃騎大將軍、巴東郡開國公、新除開府儀同三司、南豫州刺史元景,風度弘簡,體局深沈,正義亮時,恭素範
 物。幽明道盡,則首贊孝圖,盛運開歷,則毗燮皇化。方任孚漢輔,業懋殷衡,而蜂豺肆濫,顯加禍毒,冤動勳烈,悲深朝貫。朕承七廟之靈,纂臨寶業,情典既申,痛悼彌軫,宜崇賁徽冊,以旌忠懿。可追贈使持節、都督南豫、江二州諸軍事、太尉、侍中、刺史、國公如故。給班劍三十人,羽葆、鼓吹一部,謚曰忠烈公。」



 叔仁為梁州刺史,黃門郎。以破臧質功,封宜陽侯,食邑八百戶。元景從兄元怙,大明末,代叔仁為梁州,與晉安王子勛同逆,事敗,歸降。元景
 從父弟先宗,大明初,為竟陵王誕司空參軍,誕作亂,殺之,追贈黃門侍郎。元景從祖弟光世,先留鄉里,索虜以為折沖將軍、河北太守,封西陵男。光世姊夫偽司徒崔浩,虜之相也。元嘉二十七年,虜主拓跋燾南寇汝、潁,浩密有異圖,光世要河北義士為浩應。浩謀泄被誅,河東大姓坐連謀夷滅者甚眾,光世南奔得免。太祖以為振武將軍。



 前廢帝景和中,左將軍,直閣。太宗定亂,光世參謀,以為右衛將軍,封開國縣侯,食邑千戶。既而四方反
 叛,同閣宗越、譚金又誅,光世乃北奔薛安都,安都使守下邳城。及安都招引索虜,光世率眾歸降,太宗宥之,以為順陽太守。子欣慰謀反,光世賜死。



 顏師伯,字長淵,琅邪臨沂人,東揚州刺史竣族兄也。父邵,剛正有局力,為謝晦所知。晦為領軍,以為司馬,廢立之際,與之參謀。晦鎮江陵,請為諮議參軍,領錄事,軍府之務悉委焉。邵慮晦將有禍,求為竟陵太守,未及之郡,值晦見討,晦與邵謀起兵距朝廷,邵飲藥死。



 師伯少孤
 貧,涉獵書傳,頗解聲樂。劉道產為雍州,以為輔國行參軍。弟師仲,妻臧質女也。質為徐州,辟師伯為主簿。衡陽王義季代質為徐州,質薦師伯於義季,義季即命為征西行參軍。興安侯義賓代義季,世祖代義賓,仍為輔國、安北行參軍。



 王景文時為諮議參軍,愛其諧敏,進之世祖。師伯因求杖節,乃以為徐州主簿。善於附會,大被知遇。及去鎮,師伯以主簿送故。世祖鎮尋陽,啟太祖請為南中郎府主簿。太祖不許,謂典簽曰:「中郎府主簿那得
 用顏師伯。」世祖啟為長流正佐,太祖又曰:「朝廷不能除之,郎可自板,亦不宜署長流。」世祖乃板為參軍事,署刑獄。及入討元凶,轉主簿。



 世祖踐阼,以為黃門侍郎,隨王誕驃騎長史、南郡太守。改為驃騎大將軍長史、南濮陽太守,御史中丞。臧質反,出為寧遠將軍、東陽太守,領兵置佐,以備東道。



 事寧,復為黃門侍郎,領步兵校尉,改領前軍將軍,徙御史中丞,遷侍中。上以伐逆寧亂,事資群謀,大明元年,下詔曰:「昔歲國難方結,疑懦者眾,故散騎
 常侍、太子右率龐秀之履嶮能貞,首暢義節,用使狡狀先聞,軍備夙固,醜逆時殄,頗有力焉。追念厥誠,無忘於懷。侍中祭酒顏師伯、侍中領射聲校尉袁愍孫、豫章太守王謙之、太子前中庶子領右衛率張淹,爰始入討,預參義謀,契闊大難,宜蒙殊報。



 秀之可封樂安縣伯,食邑六百戶,師伯平都縣子,愍孫興平縣子,謙之石陽縣子,淹廣晉縣子,食邑各五百戶。」



 師伯遷右衛將軍,母憂去職。二年,起為持節、督青冀二州、徐州之東安、東莞、兗州
 之濟北三郡諸軍事、輔國將軍、青冀二州刺史。其年,索虜拓跋濬遣偽散騎常侍、鎮西將軍天水公拾賁敕文率眾寇清口,清口戍主振威將軍傅乾愛率前員外將軍周盤龍等擊大破之。世祖遣虎賁主龐孟虯、積射將軍殷孝祖等赴討,受師伯節度。師伯遣中兵參軍茍思達與孟虯合力。行達沙溝,虜窟環公、五軍公等馬步數萬,迎軍拒戰。孟虯等奮擊盡日,孟虯手斬五軍公,虜於是大奔。孝祖又斬窟環公,赴水死者千計。虜又遣河南公、
 黑水公、濟州公、青州刺史張懷之等屯據濟岸,師伯又遣中兵參軍江方興就傅乾愛擊破之,斬河南公樹蘭等。虜別帥它門又遣萬餘人攻清口戍城,乾愛、方興出城拒戰,即斬它門,餘眾奔走。虜天水公又率二萬人復來逼城,乾愛等出戰,又破之,追奔至赤龍門,殺賊甚眾。上嘉其功,詔曰:「虜驅率犬羊,規暴邊塞,輔國將軍、青冀二州刺史師伯宣略命師,合變應機,濟戍奮怒,一月四捷,支軍異部,騁勇齊效,頻梟名王,大殲群醜。朕用嘉嘆,
 良深於懷。可遣使慰勞,并符輔國府詳考功最,以時言上。」



 茍思達、龐孟虯等又追虜至杜梁,虜眾多,四面俱合,平南參軍童太一及茍思達等並單騎出盪,應手披靡。孟虯等繼至,虜乃散走,透河死者甚多。既而虜更合眾大至,孟虯等又破之。世祖又遣司空參軍卜天生助師伯。張懷之據縻溝城,師伯遣天生等破之,懷之出城逆戰,天生率軍主劉懷珍、白衣客朱士義、殿中將軍孟繼祖等擊之。懷之敗走入城,僅以身免。繼祖於陣遇害,追
 贈郡守。又虜隴西王等屯據申城,背濟向河,三面險固,天生又率眾攻之,朱士義等貫甲先登,賊赴河死者無算,即日陷城。虜天水公又攻樂安城,建威將軍、平原樂安二郡太守分武都與卜天生等拒擊,大破之,虜乃奔退,追戰克捷,直至清口。虜攻圍傅乾愛,乾愛隨方拒對,孝祖等既至,虜徹圍遁走。師伯進號征虜將軍。



 三年,竟陵王誕反,師伯遣長史嵇玄敬率五千人赴難。四年,徵為侍中,領右軍將軍,親幸隆密,群臣莫二。遷吏部尚書,右
 軍如故。上不欲威柄在人,親覽庶務,前後領選者,唯奉行文書,師伯專情獨斷,奏無不可。遷侍中,領右衛將軍。



 七年,補尚書右僕射。時分置二選,陳郡謝莊、琅邪王曇生並為吏部尚書。師伯子舉周旋寒人張奇為公車令,上以奇資品不當,使兼市買丞,以蔡道惠代之。令史潘道棲、褚道惠、顏禕之、元從夫、任澹之、石道兒、黃難、周公選等抑道惠敕,使奇先到公車,不施行奇兼市買丞事。師伯坐以子預職,莊、曇生免官,道栖、道惠棄市。禕之等
 六人鞭杖一百。師伯尋領太子中庶子,雖被黜挫,受任如初。



 世祖臨崩,師伯受遺詔輔幼主,尚書中事,專以委之。廢帝即位,復還即真,領衛尉。師伯居權日久,天下輻輳,游其門者,爵位莫不踰分。多納貨賄,家產豐積,伎妾聲樂,盡天下之選,園池第宅,冠絕當時,驕奢淫恣,為衣冠所嫉。又遷尚書僕射,領丹陽尹。廢帝欲親朝政,發詔轉師伯為左僕射,加散騎常侍,以吏部尚書王景文為右僕射。奪其京尹,又分臺任,師伯至是始懼。尋與太
 宰江夏王義恭、柳元景同誅,時年四十七。六子並幼,皆見殺。



 弟師仲,中書郎,晉陵太守。師叔,司徒主簿,南康相。太宗即位,詔曰:「故散騎常侍、僕射、領丹陽尹、平都縣子師伯,昔逢代運,豫班榮賞。遭罹厄會,隕命淫刑,宗嗣殄絕,良用矜悼。但其心瀆貨,宜貶贈典,可紹封社,以慰冤魂。



 謚曰荒子。」師仲子幹繼封。齊受禪,國除。



 沈慶之,字弘先,吳興武康人也。兄敞之,為趙倫之征虜參軍、監南陽郡,擊蠻有功,遂即真。



 慶之少有志力。孫恩
 之亂也,遣人寇武康,慶之未冠,隨鄉族擊之,由是以勇聞。荒擾之後,鄉邑流散,慶之躬耕壟畝,勤苦自立。年三十,未知名,往襄陽省兄,倫之見而賞之。倫之子伯符時為竟陵太守,倫之命伯符版為寧遠中兵參軍。竟陵蠻屢為寇,慶之為設規略,每擊破之,伯符由此致將帥之稱。伯符去郡,又別討西陵蠻,不與慶之相隨,無功而反。



 永初二年,慶之除殿中員外將軍,又隨伯符隸到彥之北伐。伯符病歸,仍隸檀道濟。道濟還白太祖,稱慶之忠
 謹曉兵,上使領隊防東掖門,稍得引接,出入禁省。



 出戍錢唐新城,及還,領淮陵太守。領軍將軍劉湛知之,欲相引接,謂之曰:「卿在省年月久,比當相論。」慶之正色曰:「下官在省十年,自應得轉,不復以此仰累。」尋轉正員將軍。及湛被收之夕,上開門召慶之,慶之戎服履襪縛褲入。上見而驚曰:「卿何意乃爾急裝?」慶之曰:「夜半喚隊主,不容緩服。」遣收吳郡太守劉斌,殺之。遷始興王浚後軍行參軍,員外散騎侍郎。



 元嘉十九年,雍州刺史劉道
 產卒,群蠻大動,征西司馬朱修之討蠻失利,以慶之為建威將軍,率眾助脩之。修之失律下獄,慶之專軍進討,大破緣沔諸蠻,禽生口七千人。進征湖陽,又獲萬餘口。遷廣陵王誕北中郎中兵參軍,領南東平太守,又為世祖撫軍中兵參軍。世祖以本號為雍州,隨府西上。時蠻寇大甚,水陸梗礙,世祖停大隄不得進。分軍遣慶之掩討,大破之,降者二萬口。世祖至鎮,而驛道蠻反殺深式,還慶之又討之。王玄謨領荊州,王方回領臺軍並會,平
 定諸山,獲七萬餘口。鄖山蠻最彊盛,魯宗之屢討不能克,慶之剪定之,禽三萬餘口。還京師,復為廣陵王誕北中郎中兵參軍,加建威將軍、南濟陰太守。



 雍州蠻又為寇,慶之以將軍、太守復與隨王誕入沔。既至襄陽,率後軍中兵參軍柳元景、隨郡太守宗愨、振威將軍劉顒、司空參軍魯尚期、安北參軍顧彬、馬文恭、左軍中兵參軍蕭景嗣、前青州別駕崔目連、安蠻參軍劉雍之、奮威將軍王景式等二萬餘人伐沔北諸山蠻,宗愨自新安道
 入太洪山,元景從均水據五水嶺,文恭出蔡陽口取赤係鄔,景式由延山下向赤圻阪,目連、尚期諸軍八道俱進,慶之取五渠,頓破鄔以為眾軍節度。前後伐蠻,皆山下安營以迫之,故蠻得據山為阻,於矢石有用,以是屢無功。慶之乃會諸軍於茹丘山下,謂眾曰:「今若緣山列旆以攻之,則士馬必損。去歲蠻田大稔,積穀重巖,未有饑弊,卒難禽剪。今令諸軍各率所領以營於山上,出其不意,諸蠻必恐,恐而乘之,可不戰而獲也。」於是諸軍並
 斬山開道,不與蠻戰,鼓噪上山,衝其腹心,先據險要,諸蠻震擾,因其懼而圍之,莫不奔潰。自冬至春,因糧蠻谷。



 頃之,南新郡蠻帥田彥生率部曲十封六千餘人反叛,攻圍郡城,慶之遣元景率五千人赴之。軍未至,郡已被破,焚燒城內倉儲及廨舍蕩盡,并驅略降戶,屯據白楊山。元景追之至山下,眾軍悉集,圍山數重。宗愨率其所領先登,眾軍齊力急攻,大破,威震諸山,群蠻皆稽顙。慶之患頭風,好著狐皮帽,群蠻惡之,號曰「蒼頭公」。每見慶
 之軍,輒畏懼曰:「蒼頭公已復來矣!」慶之引軍自茹丘山出檢城,大破諸山,斬首三千級,虜生蠻二萬八千餘口,降蠻二萬五千口,牛馬七百餘頭,米粟九萬餘斛。隨王誕築納降、受俘二城於白楚。



 慶之復率眾軍討幸諸山犬羊蠻,緣險築重城,施門櫓,甚峻。山多木石,積以為壘。立部曲,建旌旗,樹長帥,鐵馬成群。慶之連營山中,開門相通。又命諸軍各穿池於營內,朝夕不外汲,兼以防蠻之火。頃之風甚,蠻夜下山,人提一炬以燒營。營內多幔
 屋及草庵,火至輒以池水灌滅,諸軍多出弓弩夾射之,蠻散走。慶之令諸軍斬山開道攻之,而山高路險,暑雨方盛,乃置東岡、蜀山、宜民、西柴、黃徼、上夌六戍而還。蠻被圍守日久,並饑乏,自後稍出歸降。慶之前後所獲蠻,並移京邑,以為營戶。



 二十七年,遷太子步兵校尉。其年,太祖將北討,慶之諫曰:「馬步不敵,為日已久矣。請舍遠事,且以檀、到言之。道濟再行無功,彥之失利而返。今料王玄謨等未踰兩將,六軍之盛,不過往時。將恐重辱王
 師,難以得志。」上曰:「小醜竊據,河南修復,王師再屈,自別有以;亦由道濟養寇自資,彥之中塗疾動。虜所恃唯馬,夏水浩汗,河水流通,泛舟北指,則確磝必走,滑臺小戍,易可覆拔。克此二戍,館穀弔民,虎牢、洛陽,自然不固。比及冬間,城守相接,虜馬過河,便成禽也。」慶之又固陳不可。丹陽尹徐湛之、吏部尚書江湛並在坐,上使湛之等難慶之。慶之曰:「治國譬如治家,耕當問奴,織當訪婢。陛下今欲伐國,而與白面書生輩謀之,事何由濟!」上大笑。



 及北討,慶之副玄謨向確磝,戍主棄城走。玄謨圍滑臺,慶之與蕭斌留確磝,仍領斌輔國司馬。玄謨攻滑臺,積旬不拔。虜主拓跋燾率大眾南向,斌遣慶之率五千人救玄謨。慶之曰:「玄謨兵疲眾老,虜寇已逼,各軍營萬人,乃可進耳;少軍輕往,必無益也。」斌固遣令去,會玄謨退,斌將斬之,慶之固諫乃止。太祖後問:「何故諫斌殺玄謨?」對曰:「諸將奔退,莫不懼罪,自歸而死,將至逃散。且大兵至,未宜自弱,故以攻為便耳。」



 蕭斌以前驅敗績,欲死固
 確磝。慶之曰:「夫深入寇境,規求所欲,退敗如此,何可久住。今青、冀虛弱,而坐守窮城,若虜眾東過,青東非國家有也。確磝孤絕,復作朱修之滑臺耳。」會詔使至,不許退,諸將並謂宜留,斌復問計於慶之。慶之曰:「閫外之事,將所得專,詔從遠來,事勢已異。節下有一範增而不能用,空議何施。」斌及坐者並笑曰:「沈公乃更學問。」慶之厲聲曰:「眾人雖見古今,不如下官耳學也。」玄謨自以退敗,求戍確磝,斌乃還歷城,申坦、垣護之共據清口。



 慶之乘驛
 馳歸,未至,上驛詔止之,使還救玄謨。會虜已至彭城,不得向北,太尉江夏王義恭留領府中兵參軍。拓跋燾至卯山,義恭遣慶之率三千拒之,慶之以為虜眾彊,往必見禽,不肯行。太祖後謂之曰:「河上處分,皆合事宜,惟恨不棄確磝耳。卿在左右久,偏解我意,正復違詔濟事,亦無嫌也。」



 二十七年,使慶之自彭城徙流民數千家於瓜步,征北參軍程天祚徙江西流民於南州,亦如之。二十九年,復更北伐,慶之固諫不從,以立議不同,不使北
 出。是時亡命司馬黑石、廬江叛吏夏侯方進在西陽五水,誑動群蠻,自淮、汝至於江沔,咸罹其患。十月,遣慶之督諸將討之,詔豫、荊、雍並遣軍,受慶之節度。三十年正月,世祖出次五洲,總統群帥,慶之從巴水出至五洲,諮受軍略。會世祖典簽董元嗣自京師還,陳元凶弒逆,世祖遣慶之還山引諸軍。慶之謂腹心曰:「蕭斌婦人不足數,其餘將帥,並是所悉,皆易與耳。東宮同惡不過三十人,此外屈逼,必不為用力。今輔順討逆,不憂不濟也。」
 眾軍既集,假慶之征虜將軍、武昌內史,領府司馬。世祖還至尋陽,慶之及柳元景等並以天下無主,勸世祖即大位,不許。賊劭遣慶之門生錢無忌齎書說慶之解甲,慶之執無忌白世祖。



 世祖踐阼,以慶之為領軍將軍,加散騎常侍,尋出為使持節、督南兗、豫、徐、兗四州諸軍事、鎮軍將軍、南兗州刺史,常侍如故,鎮盱眙。上伐逆定亂,思將帥之功,下詔曰:「朕以不天,有生罔二,泣血千里,志復深逆,鞠旅伐罪,義氣雲踴,群帥仗節,指難如歸。故曾
 未積旬,宗社載穆,遂以眇身,猥纂大統。永念茂庸,思崇徽錫。新除使持節、散騎常侍、都督南兗、豫、徐、兗四州諸軍事、鎮軍將軍、南兗州刺史沈慶之,新除散騎常侍、領軍將軍柳元景,新除散騎常侍、右衛將軍宗愨,督兗州諸軍事、輔國將軍、兗州刺史徐遺寶,寧朔將軍、始興太守沈法系,驃騎諮議參軍顧彬之,或盡誠謀初,宣綜戎略;或受命元帥,一戰寧亂;或稟奇軍統,協規效捷,偏師奉律,勢振東南。皆忠國忘身,義高前烈,功載民聽,誠簡
 朕心。定賞策勳,茲焉攸在,宜列土開邑,永蕃皇家。慶之可封南昌縣公,元景曲江縣公,並食邑三千戶。愨洮陽縣侯,食邑二千戶。遺寶益陽縣侯,食邑一千五百戶。法系平固縣侯,彬之陽新縣侯,並食邑千戶。」又特臨軒召拜。又使慶之自盱眙還鎮廣陵。



 孝建元年正月,魯爽反,上遣左衛將軍王玄謨討之,軍溯淮向壽陽,總統諸將。



 尋聞荊、江二州並反,徵慶之入朝,率所領屯武帳崗,甲仗五十人入六門。魯爽先遣弟瑜進據蒙蘢,歷陽太守
 張幼緒率軍討瑜,值爽至,眾散而反。乃遣慶之濟江討爽。爽聞慶之至,連營稍退,自留斷後。慶之與薛安都等進與爽戰,安都臨陣斬爽。



 進慶之號鎮北大將軍,進督青、冀、幽三州,給鼓吹一部。前軍破賊,轉位等後至追躡一階。尋與柳元景俱開府儀同三司,辭。改封始興郡公,戶邑如故。



 慶之以年滿七十,固請辭事,上嘉其意,許之。以為侍中、左光祿大夫、開府儀同三司,又固讓,上不許。表疏數十上,又面陳曰:「張良名賢,漢高猶許其退;臣有
 何用,必為聖朝所須。」乃至稽顙自陳,言輒泣涕。上不能奪,聽以郡公罷就第,月給錢十萬,米百斛,衛史五十人。大明元年,又申前命,復固辭。



 三年,司空竟陵王誕據廣陵反,復以慶之為使持節、都督南兗、徐、兗三州諸軍事、車騎大將軍、開府儀同三司、南兗州刺史,率眾討之。至歐陽,誕遣客慶之宗人沈道愍齎書說慶之,餉以玉鈽刀,慶之遣道愍反,數以罪惡。慶之至城下,誕登樓謂之曰:「沈君白首之年,何為來?」慶之曰:「朝廷以君狂愚,不足
 勞少壯,故使僕來耳!」上慮誕北奔,使慶之斷其走路。慶之移營白土,去城十八里。夕進新亭,誕果出走,不得去,還城,事在《誕傳》。



 慶之進營洛橋西,焚其東門,值雨不克。慶之兄子僧榮,時為兗州刺史,鎮瑕丘,遣子懷明率數百騎詣受慶之節度。慶之塞漸,造攻道,立行樓土山,并諸攻具。



 時夏雨,不得攻城,上使御史中丞庾徽之奏免慶之官以激之,詔無所問。誕餉慶之食,提挈者百餘人,出自北門,慶之不問,悉焚之。誕於城上授函表,倩慶之
 為送,慶之曰:「我奉詔討賊,不得為汝送表。汝必欲歸死朝廷,自應開門遣使,吾為汝送護之。」每攻城,輒身先士卒。上戒之曰:「卿為統任,當令處分有方,何蒙楯城下,身受矢石邪。脫有傷挫,為損不少。」自四月至于七月,乃屠城斬誕。進慶之司空,又固讓。於是與柳元景並依晉密陵侯鄭袤故事,朝會慶之位次司空,元景在從公之上,給恤吏五十人,門施行馬。



 四年,西陽五水蠻復為寇,慶之以郡公統諸軍討之,攻戰經年,皆悉平定,獲生口數
 萬人。居清明門外,有宅四所,室宇甚麗。又有園舍在婁湖,慶之一夜攜子孫徙居之,以宅還官。悉移親戚中表於婁湖,列門同閈焉。廣開田園之業,每指地示人曰:「錢盡在此中。」身享大國,家素富厚,產業累萬金,奴僮千計。再獻錢千萬,穀萬斛。以始興優近,求改封南海郡,不許。妓妾數十人,並美容工藝。慶之優游無事,盡意歡愉,非朝賀不出門。每從游幸及校獵,據鞍陵厲,不異少壯。



 太子妃上世祖金鏤匕箸及杅杓,上以賜慶之,曰:「卿辛
 勤匪殊,歡宴宜等,且觴酌之賜,宜以大夫為先也。」上嘗歡飲,普令群臣賦詩,慶之手不知書,眼不識字,上逼令作詩,慶之曰:「臣不知書,請口授師伯。」上即令顏師伯執筆,慶之口授之曰:「微命值多幸,得逢時運昌。朽老筋力盡,徒步還南崗。辭榮此聖世,何媿張子房。」上甚悅,眾坐稱其辭意之美。



 世祖晏駕,慶之與柳元景等並受顧命,遺詔若有大軍旅及征討,悉使委慶之。



 前廢帝即位,加慶之几杖,給三望車一乘。慶之每朝賀,常乘豬鼻無
 憲車,左右從者不過三五人。騎馬履行園田,政一人視馬而已。每農桑劇月,或時無人,遇之者不知三公也。及加三望車,謂人曰:「我每遊履田園,有人時與馬成三,無人則與馬成二。今乘此車,安所之乎。」及賜几杖,並固讓。



 廢帝狂悖無道,眾並勸慶之廢立,及柳元景等連謀,以告慶之。慶之與江夏王義恭素不厚,發其事,帝誅義恭、元景等,以慶之為侍中、太尉,封次子中書郎文季建安縣侯,食邑千戶。義陽王昶反,慶之從帝度江,總統眾軍。少
 子文耀,年十餘歲,善騎射,帝愛之。又封永陽縣侯,食邑千戶。帝凶暴日甚,慶之猶盡言諫爭,帝意稍不說。及誅何邁,慮慶之不同,量其必至,乃閉清溪諸橋以絕之。慶之果往,不得度而還。帝乃遣慶之從子攸之齎藥賜慶之死,時年八十。是年初,慶之夢有人以兩匹絹與之,謂曰:「此絹足度。」謂人曰:「老子今年不免。兩匹,八十尺也。



 足度,無盈餘矣。」及死,賜與甚厚,追贈侍中,太尉如故,給鸞輅轀輬車,前後羽葆、鼓吹,謚曰忠武公。未及葬,帝敗。太
 宗即位,追贈侍中、司空,謚曰襄公。



 長子文叔,歷中書黃門郎,景和末,為侍中。慶之之死也,不肯飲藥,攸之以被掩殺之。文叔密取藥藏錄。或勸文叔逃避,文叔見帝斷截江夏王義恭支體,慮奔亡之日,帝怒,容致義恭之變,乃飲藥自殺。子祕書郎昭明,亦自縊死。泰始七年,改封蒼梧郡公。元年,還復先封。時改始興為廣興,昭明子曇亮,襲廣興郡公。齊受禪,國除。



 慶之弟劭之,元嘉中,為廬陵王紹南中郎行參軍,討建安、揭陽諸賊,病卒。



 兄子僧
 榮,敞之之子也。孝建初,為安成相。荊、江反叛,發兵拒臧質,質遣其安成相臧眇之討僧榮,擊破之。大明中,為兗州刺史。景和中,徵為黃門郎,未還,卒。子懷明,太宗泰始初,居父憂,起為建威將軍,東征南討有功,封吳興縣子,食邑四百戶。歷位黃門侍郎,再為南兗州刺史。元徽初,丁母艱,去職。桂陽王休範為逆,起為冠軍將軍,統水軍防固石頭,朱雀失守,懷明委軍奔走,頃之憂卒。



 慶之從弟法系,字體先,亦有將用。初為趙伯符將佐,後隨慶之
 征五水蠻。世祖伐逆,以為南中郎參軍,加寧朔將軍,領三千人前發,與柳元景旦至新亭。元景居中營,宗愨居西營,法系居東營。東營據崗,賊攻元景,法系臨射之,所殺甚眾。



 法系塹外樹悉伐之令倒,賊劭來攻,緣樹以進,彭棑多開隙,選善射手,的發無不中,死者交橫。事平,以為寧朔將軍、始興太守,討蕭簡於廣州。聞臺軍將至,簡誑其眾曰:「臺軍是賊劭所遣。」並信之。前征北參軍顧邁被賊徙在城內,善天文,云「荊、江有大兵。」城內由此固守。
 初,世祖先遣鄧琬圍簡,唯治一攻道,法系至,曰:「宜四面並攻,若守一道,何時可拔」琬慮功不在己,不從。法系曰:「更相申五十日。」日盡又不克,乃從之。八道俱攻,一日即拔,斬蕭簡,廣州平。



 封庫藏付鄧琬而還。官至驍騎將軍、尋陽太守,新安王子鸞北中郎司馬。



 劭之子文秀,別有傳。慶之群從姻戚,由之在列位者數十人。



 史臣曰:張釋之云,用法一偏,天下獄皆隨輕重。縣衡於上,四海共稟其平,法亂於朝,民無所措手足。師伯藉寵
 代臣,勢震朝野,傾意廝臺,情以貨結,自選部至於局曹,莫不從風而靡。曲徇私請,因停詔敕,天震霣怒,僕者相望,師伯任用無改,而王、謝免職。君子謂是舉也,豈徒失政刑而已哉!



\end{pinyinscope}