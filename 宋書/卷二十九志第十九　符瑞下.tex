\article{卷二十九志第十九 符瑞下}

\begin{pinyinscope}

 嘉禾,
 五穀之長,王者德盛,則二苗共秀。於周德,三苗共穗;於商德,同本異穟;於夏德,異本同秀。



 漢宣帝元康四年,嘉穀玄稷,降于郡國。



 漢章帝元和中,嘉禾生郡國。



 漢安帝延光二年六月,嘉禾生九真,百五十六本,七百六十八穗。



 漢桓帝建和二年四月,嘉禾生大司農帑。漢桓帝永康元年八月,嘉禾生魏郡。



 魏文帝黃初元年,郡國三言嘉禾生。



 吳孫權黃龍三年十月,會稽南始平言嘉禾生。孫權赤烏七年秋,宛陵言嘉禾生。



 晉武帝泰始八年十月,瀘水胡王彭護獻嘉禾。晉武帝太康四年十二月,嘉禾生扶風雍。太康五年七月,嘉禾生豫章南昌。太康八年閏三月,嘉禾生東夷校尉園。



 太康八年九月,嘉禾生東萊掖。



 晉愍帝建興元年八月癸亥,嘉禾生襄平縣,一莖七穗。建興二年六月,嘉禾生平州治,三實同蒂。建興三年七月,嘉禾生襄平縣,異體同蒂。



 宋文帝元嘉二年十月,嘉禾生潁川陽翟,太守垣苗以聞。元嘉九年三月,嘉禾生義陽,豫州刺史長沙王義欣以獻。元嘉十年八月,嘉禾生汝南苞信,豫州刺史長沙王義欣以獻。元嘉十一年八月,嘉禾一莖九穗生北汝陰,太守王玄謨以獻。
 元嘉二十年六月,嘉禾一莖九穗生上庸新安,梁州刺史劉道以獻。元嘉二十一年,嘉禾生新野鄧縣,雍州刺史蕭思話以獻。元嘉二十二年六月,嘉禾生籍田,一莖九穗。



 元嘉二十二年七月癸酉,嘉禾生平虜陵,徐州刺史臧質以獻。元嘉二十二年九月,嘉禾生太尉府田,太尉江夏王義
 恭以聞。元嘉二十二年九月,嘉禾生揚州東耕田,刺史始興王濬以聞。元嘉二十二年,嘉禾生華林園,百六十穗,園丞陳襲祖以聞。



 元嘉二十二年,嘉禾生潁川陽白,豫州刺史趙伯符以獻。元嘉二十三年七月乙丑,嘉禾旅生藉田,藉田令褚熙
 伯以聞。元嘉二十三年七月庚午,嘉禾生丹陽椒唐里,揚州刺史始興王濬以聞。元嘉二十三年七月庚辰,嘉禾生醴湖屯,屯主王世宗以聞。



 元嘉二十三年八月己酉,嘉禾生華林園,園丞陳襲祖以聞。元嘉二十三年九月庚申,嘉禾生沛郡蕭,征北大將軍
 衡陽王義季以聞。元嘉二十三年,嘉禾生江夏汝南,荊州刺史南譙王義宣以聞。



 元嘉二十四年七月乙卯,嘉禾旅生華林園及景陽山,園丞梅道念以聞。太尉江夏王義恭上表曰:臣聞居高聽卑,上帝之功;天且弗違,聖王之德。故能影響二儀,甄陶萬有。



 鑒觀今古,採驗圖緯,未有道闕化虧,而禎物著明者也。自皇運受終,辰曜交和,是以卉木表靈,山淵效
 寶。伏惟陛下體《乾》統極,休符襲逮。若乃鳳儀西郊,龍見東邑,海酋獻改緇之羽,河祗開俟清之源。三代象德,不能過也。有幽必闡,無遠弗屆,重譯歲至,休瑞月臻。前者躬藉南畝,嘉穀仍植,神明之應,在斯尤盛。



 四海既穆,五民樂業,思述汾陽,經始靈囿。蘭林甫樹,嘉露頻流,板築初就,祥穗如積。太平之符,於是乎在。臣以寡立,承乏槐鉉,沐浴芳津,預睹冥慶,不勝抃舞之情。謹上《嘉禾甘露頌》一篇,不足稱揚美烈,追用悚汗。其頌曰:二象攸分,三
 靈樂主。齊應合從,在今猶古。天道誰親,唯仁斯輔。皇功帝績,理冠區宇。四民均極,我后體茲。惟機惟神,敬昭文思。九族既睦,萬邦允釐。德以位敘,道致雍熙。於穆不已,顯允東儲。生知夙睿,嶽茂淵虛。因心則哲,令問弘敷。繼徽下武,儷景辰居。軒制合宮,漢興未央。矧伊聖朝,九有已康。率由舊典,思燭前王。乃造陵霄,遂作景陽。有藹景陽,天淵之涘。清暑爽立,雲堂特起。



 植類斯育,動類斯止。極望江波,遍對岳峙。化德惟達,休瑞惟懋。誕降嘉種,呈
 祥初構。甘露春凝,禎穟秋秀。于今匪烈,嗣歲仍富。昔在放勳,歷莢數朝。降及重華,倚扇清庖。鑠矣皇慶,比物競昭。倫彼典策,被此風謠。資臣六蔽,任兼兩司。既恧仲袞,又慚鄭緇。豈忘衡泌,樂道明時。敢述休祉,愧闕令辭。


中領軍吉陽縣侯沈演之奏上《嘉禾頌》曰:「煥炳禎圖,昭晰瑞典。運傾方閟,時亨始顯。綈狀既章,鳥文斯辨。於皇聖辟,承物紀遠。明兩辰麗,昌輝天衍。
 \gezhu{
  其一}


理妙位崇,事神業盛。淵渥德澤,虛寂道政。協化安心,調樂移性。玉衡從
 體,瑤光得正。巨星垂採,景雲立慶。
 \gezhu{
  其二}


極仁所被,罔幽不攘。至和所感,靡況弗彰。鴛出丹穴,鸚起西湘。白鹿踰海,素鳥越江。結響穹陰,儀形鐘陽。
 \gezhu{
  其三}


治人奉天,乃勤乃格。黛耒俶載,高廩已積。嘉禾重穋,甘露流液。擢秀辰畦,揚穎角澤。離穟合豪,榮區廕斥。
 \gezhu{
  其四}


盈箱征殷,貫桑表周。今我大宋,靈貺綢繆。帝終捴謙,繹思勿休。躬薦宗廟,溫恭率由。降福以誠,孝享虔羞。
 \gezhu{
  其五}


頒趾推功,登徽睿詔。恩覃隱顯,賞延荒徼。河溓海夷,山華岳耀。憬琛夐贐,兼澤委
 效。日表地外,改服請教。
 \gezhu{
  其六}


茂對盛時,綏萬屢豐。厭厭歸素,秩秩大同。上藏諸用,下知所從。仰式王度,俯歌《南風》。鴻名稱首,永保無窮。
 \gezhu{
  其七}
 。」



 元嘉二十四年八月乙巳,嘉禾生魚城內晉陵,南徐州刺史廣陵王誕以聞。元嘉二十五年六月壬寅,嘉禾旅生華林園,十株七百穗,園丞梅道念以聞。元嘉二十五年六月壬子,嘉禾生藉田,藉田令褚熙伯以
 獻。元嘉二十五年七月壬辰,嘉禾生北海,青、冀二州刺史杜坦以獻。元嘉二十五年八月丙午,嘉禾生太尉江夏王義恭果園,江夏國典書令陳穎以聞。元嘉二十五年八月壬子,嘉禾生建康化義里,令丘珍孫以獻。元嘉二十五年八月癸丑,嘉禾生華林園,園丞梅道念
 以獻。元嘉二十五年十一月,嘉禾生巴東,荊州刺史南譙王義宣以聞。元嘉二十六年五月癸酉,嘉禾生建康禁中里,揚州刺史始興王濬以獻。元嘉二十六年六月甲寅,嘉禾生藉田,藉田令褚熙伯以獻。元嘉二十六年七月,嘉禾生巴東朐䏰,荊州刺史南譙
 王義宣以獻。



 元嘉二十七年十月己丑,嘉禾生北海,青州刺史杜坦以聞。元嘉二十八年七月戊戌,嘉禾生廣陵邵伯埭,兗州刺史江夏王義恭以聞。



 孝武帝孝建二年六月癸巳,嘉禾二株生江夏王義恭東田。孝建二年九月己丑朔,嘉禾異畝同穎生齊郡廣饒縣。
 孝建三年七月庚午,嘉禾生吳興武康。



 孝武帝大明元年五月戊午,嘉禾一株五莖生清暑殿鴟尾中。大明元年八月甲申,嘉禾生青州,異根同穗。大明三年九月乙亥,嘉禾生北海都昌縣,青州刺史顏師伯以聞。大明六年八月辛未,嘉禾生樂陵,青、冀二州刺史劉道隆以聞。



 明帝泰始二年七月己酉,嘉禾生會稽永興,太守巴陵王休若以獻。



 漢章帝元和中,嘉麥生郡國。



 晉武帝太康十年六月,嘉麥生扶風郡,一莖四穗。是歲收三倍。



 宋文帝元嘉二十三年,醴湖屯生嘉粟,一莖九穗,屯主王世宗以聞。元嘉二十五年六月壬子,嘉黍生藉田,藉田令褚熙伯
 以獻。



 吳孫權黃龍三年,由拳野稻生,改由拳為禾興。吳孫亮五鳳元年,交址稗草化為稻。



 宋文帝元嘉二十三年,吳郡嘉興鹽官縣野稻自生三十許種,揚州刺史始興王濬以聞。元嘉二十八年七月癸卯,尋陽柴桑菽粟旅生,彌漫原野,江州刺史建平王宏以聞。



 漢章帝元和中,嘉瓜生郡國。



 漢安帝元初三年三月,東平陵有瓜異處共生,八瓜同蒂。



 漢桓帝建和二年七月,河東有嘉瓜,兩體共蒂。



 晉武帝太康三年六月,嘉瓜異體同蒂,生河南洛陽輔國大將軍王濬園。晉武帝太康元年十二月戊子,嘉瓠生寧州,寧州刺史費統以聞。



 宋文帝元嘉二十五年四月戊辰,嘉瓠生京邑新園,園
 丞徐道興以獻。



 孝武帝大明五年五月,嘉瓜生建康蔣陵里,丹陽尹王僧朗以獻。



 明帝太始二年八月戊午,嘉瓜生南豫州,南豫州刺史山陽王休祐以獻。



 文帝元嘉七年七月乙酉,建康頟簷湖二蓮一蒂。元嘉十六年七月壬申,華林池雙蓮同幹。元嘉十年七月己丑,華林天淵池芙蓉異花同蒂。
 元嘉十九年八月壬子,揚州後池二蓮合華,刺史始興王濬以獻。元嘉二十年五月,廬陵郡池芙蓉二花一蒂,太守王淵以聞。元嘉二十年六月壬寅,華林天淵池芙蓉二花一蒂,園丞陳襲祖以聞。



 元嘉二十年夏,永嘉郡後池芙蓉二花一蒂,太守臧藝以聞。
 元嘉二十年七月,吳興郡後池芙蓉二花一蒂,太守孔山士以聞。元嘉二十年,揚州後池芙蓉二花一蒂,刺史始興王濬以獻。元嘉二十一年六月丙午,華林園天淵池二蓮同乾,園丞陳襲祖以聞。元嘉二十二年四月,樂遊苑池二蓮同乾,苑丞梅道念以聞。
 元嘉二十二年七月,東宮玄圃園池二蓮同乾,內監殿守舍人宮勇民以聞。元嘉二十三年六月壬寅,華林天淵池芙蓉二花一蒂,園丞陳襲祖以聞。元嘉二十三年六月辛丑,太子西池二蓮共乾,池統胡永祖以聞。元嘉二十三年八月己酉,魚邑三周池二蓮同乾,園丞徐道興以聞。



 孝武帝孝建二年六月庚寅,玄武湖二蓮同幹。孝武帝大明五年,藉田芙蓉二花同蒂,大司農蕭邃以獻。



 明帝太始二年八月丙辰,五城澳池二蓮同乾,都水使者羅僧愍以獻。太始二年八月己未,豫州刺史山陽王休祐獻蓮,二花一蒂。太始五年六月甲子,嘉蓮生湖熟,南臺侍御史竺曾
 度以聞。太始六年六月壬子,嘉蓮生東宮玄圃池,皇太子以聞。



 晉武帝太始二年六月壬申,嘉柰一蒂十實,生酒泉。太始七年六月己亥,東宮玄圃池芙蓉二花一蒂,皇太子以獻。



 晉成帝咸和六年,鎮西將軍庾亮獻嘉橘,一蒂十二實。



 晉安帝隆安三年,武陵臨沅獻安石榴,一蒂六實。



 雲有五色,太平之應也,曰慶雲。若云非雲,若煙非煙,五
 色紛縕,謂之慶雲。



 漢宣帝神爵四年春,齋戒之莫,神光顯著。薦鬯之夕,神光交錯,或降于天,或登于地,或從四方,來集於壇上。



 漢章帝元和三年正月,車駕北巡,以太牢祠北岳山,見黃白氣。



 宋孝武帝大明元年五月壬子,紫氣從景陽樓上層出,狀如煙,回薄良久。



 明帝泰始二年三月丙午,黃紫雲從景陽樓出,隨風回,
 久乃消,華林園令臧延之以聞。泰始二年六月己卯,日入後,有黃白赤白氣東西竟天,光明潤澤,久乃消。



 泰始四年十一月辛未,崇寧陵令上書言,自大明八年至今四年二月,宣太后陵明堂前後數有光及五色雲,又芳香四滿,又五采雲在松下,狀如車蓋。泰始七年四月戊申夜,京邑崇虛館堂前有黃氣,狀如寶蓋,高十許丈,漸有五色,道士陸脩靜以聞。



 白兔,王者敬耆老則見。



 漢光武建武十三年九月,南越獻白兔。



 章帝元和中,白兔見郡國。



 魏文帝黃初中,郡國十九言白兔見。



 晉武泰始五年七月己亥,白兔見北海即墨,即墨長獲以獻。晉武帝咸寧二年十月癸亥,白兔二見河南陽翟,陽翟令華衍獲以獻。
 咸寧四年六月,白兔見天水。晉武帝太康二年八月壬子,白兔見彭城。太康二年十月,白兔見趙國平鄉,趙王倫獲以獻。太康四年十一月癸未,白兔見北地富平。太康八年十二月庚戌,白兔見陳留酸棗,關內侯成公忠獲以獻。



 晉穆帝永和十二年九月甲申,白兔見鄱陽,太守王耆之以獻,并上頌一篇。
 晉穆帝升平三年十二月庚申,北中郎將郗曇獻白兔。



 晉海西公太和九年四月,陽穀獻白兔。



 晉孝武帝太元十五年三月,白兔見淮南壽陽。



 晉安帝義熙二年四月,無錫獻白兔。義熙二年四月,壽陽獻白兔。



 宋文帝元嘉六年九月,長廣昌陽淳于邈獲白兔,青州刺史蕭思話以獻。元嘉八年閏六月丁亥,司徒府白從伊生於淮南繁昌
 獲白兔以獻。元嘉十三年七月甲戌,濟南朝陽王道獲白兔,青州刺史段宏以獻。元嘉十四年正月丙申,白兔見山陽縣,山陽太守劉懷之以獻。元嘉十五年七月壬申,山陽師齊獲白兔,南兗州刺史江夏王義恭以獻。元嘉二十二年三月,白兔見東萊當利,青州刺史杜冀
 以聞。元嘉二十四年七月丁巳,白兔見兗州,刺史徐瓊以聞。元嘉二十四年七月己酉,白兔見東莞,太守趙球以獻。元嘉二十七年二月壬辰,白兔見竟陵,荊州刺史南譙王義宣以獻。元嘉二十七年六月丙午,白兔見南汝陰,豫州刺史南平王鑠以獻。



 孝武帝孝建二年正月庚戌,白兔見淮南,太守申坦以聞。
 孝建三年閏三月乙丑,白兔見平原,獲以獻。孝武大明元年六月庚子,白兔見即墨,獲以獻。大明六年八月辛未,白兔見北海,青、冀二州刺史劉道隆以獻。大明六年六月乙丑,白兔見,青、冀二州刺史劉道隆以獻。


斗殞精,王者孝行溢則見。
 \gezhu{
  闕}
 。



 赤烏,周武王時銜穀至,兵不血刃而殷服。



 漢章帝元和中,赤烏見郡國。



 吳孫權赤烏元年,有赤烏集於殿前。吳孫休永安三年三月,西陵言赤烏見。



 晉元帝永昌二年正月,赤烏見暨陽。



 宋武帝永初二年二月,赤烏六見北海都昌。



 孝武帝大明五年六月戊子,赤烏見蜀郡,益州刺史劉思考以獻。



 白燕者,師曠時,銜丹書來至。



 漢章帝元和中,白燕見郡國。



 晉惠帝元康元年七月,白燕二見酒泉郤福,太守索靖以聞。



 宋文帝元嘉元年七月壬戌,白燕集齊郡城,游翔庭宇,經九日乃去,眾燕隨從無數。元嘉十四年,白燕集荊州府門,刺史臨川王義慶以聞。元嘉十八年六月,白燕產丹徒縣,南徐州刺史南譙王
 義宣以聞。元嘉二十年五月,白燕集南平郡府內,內史臧綽以聞。元嘉二十一年,白燕見廣陵,南兗州刺史廣陵王誕以獻。元嘉二十四年五月辛未,白燕集司徒府西園,太尉江夏王義恭以聞。元嘉二十五年八月壬子,白燕見廣陵城,南兗州刺史徐湛之以聞。
 元嘉二十六年五月戊寅,白燕產衡陽王墓亭,郎中令朱曠之獲以聞。元嘉二十七年五月甲戌,白燕產京口,南徐州刺史始興王濬以聞。元嘉二十七年六月壬辰,白燕見秣陵,丹陽尹徐湛之以獻。



 孝武帝大明二年五月乙巳,白燕產南郡江陵民家,荊州刺史硃脩之以獻。
 大明二年五月甲子,白燕二產山陽縣舍,南兗州刺史竟陵王誕以獻。大明二年六月甲戌,白燕產吳郡城內,太守王翼之以獻。大明三年五月甲申,白燕產武陵臨沅民家,郢州刺史孔靈符以聞。大明四年六月乙卯,白燕見平昌,青州刺史劉道隆以
 獻。



 明帝泰始二年六月,白燕見零陵,獲以獻。


金車,王者至孝則出。
 \gezhu{
  闕}



 三足烏,王者慈孝天地則至。



 漢章帝元和中,三足烏見郡國。


象車者,山之精也。王者德澤流洽四境則出。
 \gezhu{
  闕}



 白烏,王者宗廟肅敬則至。



 漢桓帝永壽元年四月,白烏見齊國。



 晉武帝咸寧五年七月戊辰,白烏見濟南隰陰,太守獲以
 獻。晉武帝太康元年五月庚午,白烏見襄城。太康十年五月丁丑,白烏見京兆長安。



 晉惠帝元康元年四月,白烏見河南成皋,縣令劉機獲以聞。元康元年五月戊戌,白烏見梁國睢陽。元康元年七月辛丑,白烏見陳留,獲以獻。元康四年十月,白烏見鄱陽。



 晉明帝泰寧二年十一月,白烏見京都。泰寧三年三月,白烏見吳郡海虞,獲以獻,群官畢賀。



 晉孝武帝太元十一年八月乙酉,白烏集江州寺庭,群烏翔衛。太元二十一年五月癸卯,白烏見吳國,獲以獻。



 宋武帝永初二年六月丁酉,白烏見吳郡婁縣,太守孟顗以獻。



 文帝元嘉二年十一月丙辰,白烏見山陽,太守阮寶以
 聞。元嘉三年三月甲戌,丹陽湖熟薛爽之獲白烏以獻。元嘉十一年六月乙巳,吳郡海鹽王說獲白烏,揚州刺史彭城王義康以獻。元嘉十三年三月戊辰,義興陽羨令獲白烏,太守劉禎以獻。元嘉十九年五月,海陵王文秀獲白烏,南兗州刺史臨川王義慶以獻。
 元嘉十九年十月,白烏產晉陵暨陽僑民彭城劉原秀宅樹,原秀以聞。元嘉二十年七月,彭城劉原秀又獲白烏以獻。元嘉二十四年八月乙巳,白烏見晉陵,南徐州刺史廣陵王誕以獻。



 孝武帝大明元年四月甲申,白烏見南郡江陵。



 明帝泰始二年六月丁巳,白烏見吳郡海鹽,太守顧覬之以獻。
 泰始二年九月壬寅,白烏見吳興烏程,太守卻顒以獻。



 白雀者,王者爵祿均則至。



 漢章帝元和初,白雀見郡國。



 魏文帝初,郡國十九言白雀見。



 晉武帝咸寧元年,白雀見梁國,梁王肜獲以獻。晉武帝太康二年六月丁卯,白雀二見河內南陽,太守阮侃獲以獻。太康二年六月,白雀二見河南,河南尹向雄獲以獻。
 太康七年七月庚午,白雀見豫章。太康八年八月,白雀見河南洛陽。太康十年五月丁亥,白雀見宣光北門,華林園令孫邵獲以獻。



 晉愍帝建武元年四月,尚書僕射刁協獻白雀於晉王。



 晉孝武帝太元十六年十二月,白雀見南海增城縣民吳比屋。



 晉安帝隆安五年十一月,白雀見宜都。



 晉安帝元興三年六月丙申,白雀見豫章新淦,獲以獻。



 宋文帝元嘉元年七月己巳,白雀見齊郡昌國。元嘉四年七月乙酉,白雀見北海劇。元嘉八年五月辛丑,白雀集左衛府。元嘉十一年五月丁丑,齊郡西安宗顯獲白雀,青州刺史段宏以獻。元嘉十四年五月甲午,白雀集費縣員外散騎侍郎顏敬家,獲以獻。
 元嘉十四年,白雀二見荊州府客館。元嘉十五年五月辛未,白雀集建康都亭里,揚州刺史彭城王義康以聞。元嘉十五年六月,白雀見建康定陰里,彭城王義康以獻。元嘉十五年八月,白雀見西陽,江州刺史南譙王義宣以獻。元嘉十七年五月壬寅,白雀二集荊州後園,刺史衡陽
 王義季以聞。元嘉十八年七月,吳郡鹽官于玄獲白雀,太守劉禎以獻。元嘉二十年五月乙卯,秣陵衛猗之獲白雀,丹陽尹徐湛之以獻。元嘉二十二年四月丙子,白雀見東安郡,徐州刺史臧質以獻。元嘉二十二年閏五月丙午,白雀見華林園,員外散騎
 侍郎長沙王瑾獲以獻。元嘉二十二年六月庚申,南彭城蕃縣時佛護獲白雀以獻。元嘉二十四年四月,白雀產吳郡鹽官民家,太守劉禎以獻。元嘉二十四年六月己亥,白雀五集長沙廟,長沙王瑾以聞。元嘉二十五年五月丁丑,白雀二見京都,材官吏黃公
 歡、軍人丁田夫各獲以獻。元嘉二十七年六月乙卯,白雀見濟南郡,薛榮以獻。元嘉二十八年八月己巳,崇義軍人獲白雀一隻,太子左率王錫以獻。元嘉二十九年四月癸丑,白雀見會稽山陰,太守東海王禕獲以獻。



 孝武帝孝建元年五月己亥,臨沂縣魯尚期於城上得白雀,太傅假黃鉞江夏王義恭以獻。
 孝建二年六月丙子,左衛軍獲白雀以獻。孝建三年閏三月辛酉,黃門侍郎庾徽之家獲白雀以獻。孝建三年五月丁卯,白雀見建康,獲以獻。



 孝武帝大明元年四月戊申,白雀見尋陽。大明元年五月甲寅,白雀二見渤海,獲以獻。大明元年五月甲子,白雀見建康,獲以獻。大明元年六月丁亥,白雀見零陵祁陽,獲以獻。
 大明元年七月辛亥,白雀見南陽宛,獲以獻。大明二年五月丁未,白雀見建康,揚州刺史西陽王子尚以獻。大明二年六月丁亥,白雀見河東定襄縣,荊州刺史硃脩之以聞。大明三年四月庚戌,白雀見秣陵,丹陽尹劉秀之以獻。大明三年五月壬午,太宰府崇藝軍人獲白雀,太宰江夏王義恭以獻。
 大明四年五月辛巳,白雀見廣陵,侍中顏師伯以獻。大明五年四月庚戌,白雀見晉陵,太守沈文叔以獻。



 大明五年五月癸未,白雀二見尋陽,江州刺史桂陽王休範以獻。大明五年五月癸未,白雀二見濟南,青州刺史劉道隆以獻。大明五年十月,白雀見太原,青州刺史劉道隆以獻。大明六年八月辛巳,白雀見齊郡,青、冀二州刺史劉道
 隆以獻。大明七年四月乙未,白雀集廬陵王第,廬陵王敬先以獻。大明七年四月乙丑,白雀見歷陽,太守建平王景素以獻。大明七年五月辛未,白雀見汝陰,豫州刺史垣護之以獻。大明七年六月,白雀見寶城,南豫州刺史尋陽王子房
 以獻。大明七年十月丁卯,白雀見建康,丹陽尹永嘉王子仁以獻。大明七年十一月,車駕南巡,肄水師於梁山,中江,白雀二集華蓋。



 前廢帝永光元年四月乙亥,白雀見會稽,東揚州刺史尋陽王子房以獻。



 永光元年六月丙子,白雀見彭城,徐州刺史義陽王昶
 以聞。



 明帝泰始二年七月戊子,白雀見虎檻洲,都督征討諸軍建安王休仁以聞。泰始六年七月壬午,白雀二見廬陵吉陽,內史江孜以聞。



 明帝泰豫元年六月辛丑,白雀見廣州,刺史孫超以獻。



 後廢帝元徽五年四月己巳,白雀二見尋陽柴桑,江州刺史邵陵王友以獻。



 孝武帝大明六年三月丙午,青雀見華林園。



 明帝泰始二年九月庚寅,青雀見京城內,南徐州刺史桂陽王休範以獻。


玉馬,王者精明,尊賢者則出。
 \gezhu{
  闕}


根車者,德及山陵則出。
 \gezhu{
  闕}



 白鳩,成湯時來至。



 魏文帝黃初初,郡國十九言白鳩見。



 吳孫權赤烏十二年八月癸丑,白鳩見章安。



 晉武帝泰始八年五月甲辰,白鳩二集太廟南門,議郎董胄獲以獻。



 晉武帝太康二年七月,白鳩見太僕寺。太康四年十二月,白鳩見安定臨涇。太康十年正月乙亥,白鳩見河南新城。



 宋文帝元嘉十八年八月庚午,會稽山陰商世寶獲白鳩,眼足並赤,揚州刺史始興王濬以獻。太子率更令何承天上表曰:謹考尋先典,稽之前志,王德所覃,物以應
 顯。是以玄扈之鳳,昭帝軒之鴻烈,酆宮之雀,徵姬文之徽祚。伏惟陛下重光嗣服,永言祖武,洽惠和於地絡,燭皇明於天區。故能九服混心,萬邦含受,員神降祥,方祗薦裕,休珍雜沓,景瑞畢臻。


去七月上旬,時在昧旦,黃暉洞照,宇宙開朗,徽風協律,甘液灑津。雖朱晃瑰瑋於運衡,榮光圖靈於河紀,蔑以尚茲。臣不量卑懵,竊慕擊壤有作,相杵成謳。近又豫白鳩之觀,目玩奇偉,心歡盛烈。謹獻頌一篇。野思古拙,意及庸陋,不足以發揮清英,敷
 贊幽旨,瞻前顧後,亦各其志。謹冒以聞。其《白鳩頌》曰:三極協情,五靈會性。理感冥符,道實玄聖。於赫有皇,光天配命。朝景升躔,八維同映。休祥載臻,榮光播慶。宇宙照爛,日月光華。陶山練澤,是生柔嘉。回龍表粹,離穗合柯。翩翩者鳩,亦皎其暉。理翮台領,揚鮮帝畿。匪仁莫集
 \gezhu{
  闕四字}
 匪德莫歸。暮從儀鳳,棲閣陰闈。烝哉明后,昧旦乾乾。惟德之崇,其峻如山。



 惟澤之瞻,其潤如淵。禮樂四達,頌聲遐宣。窮髮納貢,九譯導言。伊昔唐萌,愛逢慶祚。餘生既辰,而年
 之暮。提心命耋,式歌王度。晨晞永風,夕漱甘露。思樂靈臺,不遐有固。



 元嘉二十四年九月,白鳩又見。庚戌,中領軍沈演之上表曰:臣聞貞裕之美,介於盛王,休瑞之臻,罔違哲后。故鳴鳳表垂衣之化,翔鷦徵解網之仁。陛下道德嗣基,聖明纘世,教清鳥紀,治昌雲官,禮漸同川,澤浹硃徼。


天嘉明懿,民樂薰風,星辰以之炳煥,日月以之光華。神圖祗緯,盈觀閟序,白質黑章,充牣靈囿。應感之符畢臻,而因
 心之祥未屬。以素鳩自遠,毨翰歸飛,資性閑淑,羽貌鮮麗,既聞之先說,又親睹嘉祥,不勝藻抃,上頌一首。辭不稽典,分乏採章,愧不足式昭皇慶,崇贊盛美,蓋率輿誦,備之篇末。其頌曰:有哲其儀,時惟皓鳩。性勰五教,名編素丘。殷歷方昌,婉翹來遊。漢錄克韡,爰降爰休。
 \gezhu{
  其一}
 於顯盛宋,睿慶遐傳。聖皇在上,道照鴻軒。稱施既平,孝思永言。人和於地,神豫于天。
 \gezhu{
  其二}
 禮樂孔秩,靈物咸昭。白雀集苞,丹鳳棲郊。


文騶儷跡,嘉穎擢苗。灼灼縞羽,從化馴朝。
 \gezhu{
  其三}
 豈伊赴林,必周之栩。豈伊歸義,必商之所。惟德是依,惟仁是處。育景陽嶽,濯姿帝宇。
 \gezhu{
  其四}
 刑歷頒興,理感迭通。雉飛越常,鷺起西雍。烝然戾止,實兼斯容。壹茲民聽,穆是王風。


\gezhu{
  其五}
 。


玉羊,師曠時來至。
 \gezhu{
  闕}


玉雞,王者至孝則至。
 \gezhu{
  闕}


璧流離,王者不隱過則至。
 \gezhu{
  闕}


玉英,五常並修則見。
 \gezhu{
  闕}


玄圭,水泉流通,四海會同則出。
 \gezhu{
  闕}



 漢桓帝永興二年四月,光祿勳府吏舍,夜壁下有青氣,得玉鉤、玦各一。鉤長七寸三分,玦周五寸四分,身中皆雕鏤。



 晉懷帝永嘉六年二月壬子,玉龜出灞水。



 晉愍帝建興二年十月,大將軍劉琨掘地得玉璽,使參軍郎碩奉之歸于京師。建興二年十二月,涼州刺史張實遣使獻行璽一紐,封
 送璽使關內侯。晉愍帝建武元年三月己酉,丹陽江寧民虞由墾土得白麒麟璽一紐,文曰「長壽萬年」。獻晉王。



 晉成帝咸康八年九月,廬江舂穀縣留珪夜見門內有光,取得玉鼎一枚,外圍四寸。豫州刺史路永以獻。著作郎曹毗上《玉鼎頌》。



 晉安帝義熙十二年六月,左衛兵陳陽於東府前淮水中得玉璽一
 枚。



 宋孝武帝大明元年五月戊寅,江乘縣民朱伯地中得玉璧,徑五寸八分,以獻。



 大明四年二月乙巳,徐州刺史劉道隆於汴水得白玉戟,以獻。



 明帝泰始五年十月庚辰,郢州獲玄璧,廣八寸五分,安西將軍蔡興宗以獻。



 後廢帝元徽四年十一月乙巳,吳興烏程餘山道人慧獲蒼玉璧,太守蕭惠開以獻。



 金勝,國平盜賊,四夷賓服則出。



 晉穆帝永和元年二月,舂穀民得金勝一枚,長五寸,狀如織勝。明年,桓溫平蜀。永和元年三月,廬江太守路永上言,於舂穀城北,見水岸邊有紫赤光,取得金狀如印,遣主簿李邁表送。



 吳孫皓天璽元年,吳郡言掘地得銀一,長尺,廣三分,刻上有年月字。


丹甑五穀豐熟則出。
 \gezhu{
  闕}



 白魚,武王度孟津,中流入于王舟。



 宋明帝太始二年十月己巳,幸華林天淵池,白魚躍入御舟。



 漢章帝元和三年正月,車駕北巡,以太牢具祠北岳,有神魚躍出十數。


金人,王者有盛德則游後池。
 \gezhu{
  闕}



 木連理,王者德澤純洽,八方合為一,則生。



 漢章帝元和中,木連理生郡國。



 安帝元初三年正月丁丑,東平陵樹連理。漢安帝延光三年七月,左馮翊衙有木連理。延光三年七月,潁川定陵有木連理。



 漢桓帝建和二年七月,河東有木連理。



 吳孫權黃武四年六月,皖口言有木連理。



 魏文帝黃初初,郡國二言木連理。



 晉武帝泰始元年十二月,木連理生遼東方城。泰始二年八月,木連理生河南成皋。
 泰始八年正月,木連理生東平范。泰始八年五月甲辰,木連理生東平壽張。泰始八年十月,木連理生建寧。



 晉武帝咸寧元年正月,木連理生汝陰南頓。咸寧二年四月,木連理生清河靈。



 咸寧二年六月,木連理生燕國。咸寧三年七月壬辰,木連理生始平鄠。咸寧四年八月,木連理生陳留長垣。
 咸寧五年,木連理生義陽。咸寧五年,木連理生樂安臨濟。



 晉武帝太康元年正月,木連理生涪陵永平。太康元年四月,木連理生頓丘。太康元年五月,木連理二生濟陰乘氏,沛國。太康元年七月,木連理生馮翊粟邑。太康二年正月,木連理生滎陽密。太康二年十月,木連理十三生南安羱道。
 太康三年四月,木連理生琅邪華。太康三年六月,木連理生廣陵海西。太康四年正月,木連理生馮翊臨晉,蜀郡成都。太康四年十二月,木連理生扶風。太康七年三月,木連理生河南新安。太康七年六月,木連理生始興中宿,南鄉范陽。太康八年四月,木連理生廬陵東昌。太康八年九月,木連理生東萊盧鄉。
 太康九年九月,木連理生陳留浚儀。太康十年十一月,木連理生鄱陽寔陽。



 晉武帝太熙元年二月,木連理生河南梁。



 晉惠帝元康元年五月,木連理三生成都臨邛。元康元年七月辛丑,梁國內史任式上言,武平界有柞櫟二樹,合為一體,連理。



 晉愍帝建興二年三月庚辰,木連理生硃提。建興二年三月,木連理二生益州雙柏。
 建興二年六月,木連理生襄平。



 晉愍帝建武元年閏月乙丑,木連理生嵩山。建武元年八月甲午,木連理生汝陰。



 建武元年十一月,木連理生武昌,大將軍王敦以聞晉王。建武元年十一月癸酉,木連理生汝陰,太守以聞。



 晉元帝太興元年七月戊辰,木連理生武昌,大將軍王敦以聞。太興三年十一月,木連理生零陵永昌。



 晉成帝咸和八年五月己巳,木連理生昌黎咸和。咸康三年三月庚戌,木連理生平州世子府治故園中。咸康七年十二月,吳國內史王恬上言,木連理生吳縣沙裏。



 晉穆帝永和五年二月癸丑,臨海太守藍田侯述言郡界木連理。



 晉孝武帝寧康三年六月辛卯,江寧縣建興里僑民留康家樹,異木連理。
 晉孝武帝太元十一年四月壬申,琅邪費有榆木,異根連理,相去四尺九寸。太元十八年十月戊午,臨川東興令惠欣之言,縣東南溪傍有白銀樹、芳靈樹、李樹,並連理。太元十九年正月丁亥,華林園延賢堂西北李樹連理。太元二十一年正月丙子,木連理生南康寧都縣社後。



 晉安帝隆安三年十一月,木連理生汝陽,太守垣苗以聞。
 元興元年正月,木連理生泰山武陽。



 宋文帝元嘉八年四月乙亥,東莞莒縣松樹連理,太守劉玄以聞。元嘉八年八月,木連理生東安新泰縣。元嘉九年六月,木連理生營陽冷道,太守展禽以聞。元嘉十二年二月丁卯,南郡江陵庾和園甘樹連理,荊州刺史臨川王義慶以獻。元嘉十二年三月,馬頭濟陽柞樹連理,豫州刺史長沙
 王義欣以聞。元嘉十四年二月,宮內螽斯堂前梨樹連理,豫州刺史長沙王義欣以聞。元嘉十四年,南郡江陵光禕之園甘李二連理。元嘉十五年二月,太子家令劉征園中林檎樹連理,徵以聞。元嘉十七年七月,武昌崇讓鄉程僧愛家候風木連理,江州刺史臨川王義慶以聞。
 元嘉十七年十月,尋陽弘農祐幾湖芙蓉連理,臨川王義慶以聞。元嘉十八年十二月,木連理生歷陽劉成之家,南豫州刺史武陵王駿以聞。元嘉二十年七月,盱眙考城縣柞樹二株連理,南兗州刺史臨川王義慶以聞。元嘉二十年八月,木連理生汝陰,豫州刺史劉遵考以聞。



 元嘉二十一年,木連理生歷陽烏江,南豫州刺史武陵王駿以聞。元嘉二十一年,木連理生晉陵無錫,南徐州刺史南譙王義宣以聞。元嘉二十二年七月辛巳,南頓櫟連理,豫州刺史趙伯符以聞。元嘉二十二年九月,木連理生建康,建康令張永以聞。



 元嘉二十二年,木連理生武昌,江州刺史廬陵王紹以
 聞。元嘉二十三年二月辛亥,木連理生南陰柔縣,太守以聞。元嘉二十三年,木連理生淮南當塗,揚州刺史始興王濬以聞。元嘉二十四年二月壬午,臨川王第梨樹連理,臨川王燁以聞。元嘉二十四年七月壬子,晉陵無錫穀櫟樹連理,南徐
 州刺史廣陵王誕以聞。元嘉二十四年七月乙卯,木連理生會稽諸暨,揚州刺史始興王濬以聞。會稽太守羊玄保上改連理所生處康亭村為「木連理」。元嘉二十四年七月乙卯,臨川王第梨樹連理,臨川王燁以聞。元嘉二十五年四月戊辰,木連理生晉陵,南徐州刺史廣陵王誕以聞。
 元嘉二十八十正月戊子,木連理生尋陽柴桑,又生州城內,江州刺史建平王宏以聞。元嘉二十九年十月丁未,木連理生南琅邪,太守劉成以聞。



 孝武帝孝建二年三月己酉,木連理生南郡江陵,荊州刺史硃脩之以聞。孝建三年五月,木連理生北海都昌,冀州刺史垣護之以聞。
 孝建三年七月癸未,木連理生歷陽,歷陽太守袁敳以聞。



 孝武帝大明元年正月乙亥,木連理生高平。大明元年二月壬寅,華林園雙橘樹連理。大明元年九月乙丑,華林園梨樹連理。大明元年十月丁丑朔,木連理生豫章南昌。大明二年四月辛丑,木連理生汝南,豫州刺史宗愨以聞。
 大明三年九月甲午,木連理生丹陽秣陵,材官將軍范悅時以聞。大明四年三月丁亥,木連理生華林園曜靈殿北。大明四年四月壬子,木連理生華林園日觀臺北。大明四年六月戊戌,木連理生會稽山陰,揚州刺史西陽王子尚以聞。大明五年閏九月,木連理生邊城,豫州刺史垣護之以聞。
 大明五年十二月戊寅,淮南松木連理,豫州刺史尋陽王子房以聞。



 大明六年二月乙丑,木連理生晉陵,南徐州刺史新安王子鸞以聞。大明六年四月戊辰,木連理生營陽,湘州刺史建安王休仁以聞。大明六年八月乙丑,木連理生彭城城內,徐州刺史王玄謨以聞。
 大明七年正月己酉,珊瑚連理生鬱林,安始太守劉勔以聞。



 明帝泰始二年七月,木連理生丹陽秣陵。泰始四年三月庚戌,太子西池冬生樹連理,園丞周犬禽猗以獻。泰始六年四月丙午,木連理生會稽永興,太守蔡興宗以聞。泰始六年十二月壬辰,木連理生豫章南昌,太守劉愔
 之以聞。泰始七年二月戊寅,木連理生吳郡錢唐,太守王延之以聞。



 昇明二年,木連理生豫州界內,刺史劉懷珍以聞。


比目魚,王者德及幽隱則見。
 \gezhu{
  闕}


珊瑚鉤,王者恭信則見。
 \gezhu{
  闕}



 芝草,王者慈仁則生。食之令人度世。



 漢武帝元封二年,甘泉宮內產芝,九莖連葉。



 漢宣帝元康四年,金芝九莖,產于函德殿銅池中。



 漢明帝永平十七年春,芝生前殿。



 漢桓帝建和元年四月,芝草生中黃藏府。



 宋從帝升明二年,宣城山中生紫芝一株,在所獲以獻。



 明月珠,王者不盡介鱗之物則出。



 漢高后景帝時,會稽人硃仲獻三寸四寸珠。



 漢章帝元和中,郡國獻明珠。



 巨鬯,三禺之禾,一稃二米,王者宗廟脩則出。



 黃帝時,南夷乘白鹿來獻鬯。



 漢章帝元和中,秬秠生郡國。



 華平,其枝正平,王者有德則生。德剛則仰,德弱則低。



 漢章帝元和中,華平生郡國。


平露,如蓋,以察四方之政。其國不平,則隨方而傾。
 \gezhu{
  闕}


蓂莢,一名歷莢,夾階而生,一日生一葉,從朔而生,望而止,十六日,日落一葉;若月小,則一葉萎而不落。堯時生
 階。
 \gezhu{
  闕}


甫,一名倚扇,狀如蓬,大枝葉小,根根如絲,轉而成風,殺蠅。堯時生於廚。
 \gezhu{
  闕}



 硃草,草之精也,世有聖人之德則生。



 漢光武建武中元元年五月,京師有赤草生水涯。



 漢章帝元和中,硃草生郡國。



 魏文帝初,朱草生文昌殿側。



 宋文帝元嘉十一年,朱草生蜀郡郫縣王之家,益州刺史甄法崇以聞。


景星,大星也。狀如半月,於晦朔助月為明。
 \gezhu{
  闕}


賓連闊達,生於房室,王者御后妃有節則生。
 \gezhu{
  闕}


渠溲,禹時來獻裘。
 \gezhu{
  闕}


浪井,不鑿自成,王者清靜則應。
 \gezhu{
  闕}


西王母,舜時來獻白環白琯。
 \gezhu{
  闕}


越常,周公時來獻白雉、象牙。
 \gezhu{
  闕}



 漢平帝元始元年正月,越常重譯獻白雉一,黑雉二,詔二公薦宗廟。



 漢光武建武十三年九月,南越獻白雉。



 漢章帝元和中,白雉見郡國。



 漢桓帝永康元年十一月,白雉見西河。



 漢獻帝延康元年四月丁巳,饒安縣言白雉見;又郡國十九言白雉見。



 晉武帝咸寧元年四月丁巳,白雉見安豐松滋。咸寧元年十二月丙午,白雉見梁國睢陽,梁王肜獲以
 獻。咸寧三年十一月,白雉見渤海饒安,相阮溫獲以獻。



 晉武帝太康元年庚戌,白雉見中山。



 晉愍帝建興三年十二月戊午,白雉見襄平。建興三年十二月戊午,白雉見。



 安帝義熙七年五月,白雉見豫章南昌。



 宋文帝元嘉五年五月庚辰,白雉見東莞莒縣,太守劉玄以聞。元嘉十六年二月,白雉見陳郡,豫州刺史長沙王義欣
 以獻。元嘉十八年二月癸亥,白雉見南汝陰宋縣,太守文道恩以獻。元嘉二十年六月,白雉見高平方興縣,徐州刺史臧質以獻。元嘉二十六年三月戊寅,白雉見東安、沛郡各一,徐、兗二州刺史武陵王獲以獻。



 孝武帝大明二年三月己巳,白雉雌雄各一見海陵,南
 兗州刺史竟陵王誕以獻。



 大明五年十二月,白雉見秦郡,南兗州刺史晉安王子勛以獻。大明八年二月丁卯,白雉見南郡江陵,荊州刺史臨海王子頊以獻。



 前廢帝永光元年正月丙午,白雉見渤海,青州刺史王玄謨以獻。永光元年三月甲午朔,白雉見新蔡,豫州刺史劉德願
 以獻。



 黃銀紫玉,王者不藏金玉,則黃銀紫玉光見深山。



 宋明帝泰始二年八月,於赭圻城南得紫玉一段,圍三尺二寸,長一尺,厚七尺。



 太宗攻為二爵,以獻武、文二廟。


玉女,天賜妾也。《禮含文嘉》曰:「禹卑宮室,盡力溝洫,百穀用成,神龍女降。」
 \gezhu{
  闕}


地珠,王者不以財為寶則生珠。
 \gezhu{
  闕}


天鹿者,純靈之獸也。五色光耀洞明,王者道備則至。
 \gezhu{
  闕}


角端者,日行萬八千里,又曉四夷之語,明君聖主在位,明達方外幽遠之事,則奉書而至。
 \gezhu{
  闕}


周印者,神獸之名也,星宿之變化。王者德盛則至。
 \gezhu{
  闕}


飛菟者,神馬之名也,日行三萬里。禹治水勤勞歷年,救民之害,天應其德而至。
 \gezhu{
  闕}


澤獸,黃帝時巡狩至於東濱,澤獸出,能言,達知萬物之精,以戒於民,為時除害。賢君明德幽遠則來。
 \gezhu{
  闕}


者,幽隱之獸也,有明王在位則來,為時闢除災害。
 \gezhu{
  闕}


要巉者,神馬也,與飛菟同,亦各隨其方而至,以明君德也。
 \gezhu{
  闕}


同心鳥,王者德及遐方,四夷合同則至。
 \gezhu{
  闕}


趹槍蹄者,后土之獸,自能言語。王者仁孝於國則來。禹治水而至。
 \gezhu{
  闕}


紫達,王者仁義行則見。
 \gezhu{
  闕}


小鳥生大鳥,王者土地開闢則至。
 \gezhu{
  闕}


河精者,人頭魚身,師曠時所受讖也。
 \gezhu{
  闕}


延嬉,王者孝道行則至。
 \gezhu{
  闕}


大貝,王者不貪財寶則出。
 \gezhu{
  闕}


威蕤,王者禮備則生於殿前。
 \gezhu{
  闕}



 醴泉,水之精也,甘美。王者脩理則出。



 漢光武建武中元元年五月,醴泉出京師及郡國。飲醴泉者,痼病皆愈;獨眇者蹇者不差。



 魏文帝初,郡國二言醴泉出。



 宋文帝元嘉十二年,衡陽湘鄉醴泉出縣庭,荊州刺史
 臨川王義慶以聞。



 孝武帝孝建三年九月甲戌,細仗隊省井泉春夏深不盈尺,忽至一丈,有五色,水清澄,醴味,汲引不窮。



 孝武帝大明二年三月壬子,北汝陰樓煩平地出醴泉,豫州刺史宗愨以聞。



 明帝泰豫元年四月乙酉,會稽山陰思義醴泉出,太守蔡興宗以聞。


日月揚光,日者,人君象也。人君不假臣下之權,則日月
 揚光明。
 \gezhu{
  闕}



 芝英者,王者親近耆老,養有道,則生。



 漢章帝元和中,芝英生郡國。


碧石者,玩好之物棄則至。
 \gezhu{
  闕}


玉甕者,不汲而滿,王者清廉則出。
 \gezhu{
  闕}


山車者,山藏之精也。不藏金玉,山澤以時,通山海之饒,以給天下,則山成其車。
 \gezhu{
  闕}


雞駭犀,王者賤難得之物則出。
 \gezhu{
  闕}


陵出黑丹,王者脩至孝則出。
 \gezhu{
  闕}



 神鼎者,質文之精也。知吉知凶,能重能輕,不炊而沸,五味自生,王者盛德則出。



 漢武帝元鼎元年五月五日,得鼎汾水上。


漢明帝永平六年二月,廬江太守獻寶鼎。出王雒山。
 \gezhu{
  雒或作雄}
 。



 漢章帝建初七年十月,車駕西巡至槐里,右扶風禁上美陽得銅器於岐山,似酒尊。詔在道晨夕以為百官熱
 酒。



 漢和帝永元元年,竇憲征匈奴,於漠北酒泉得仲山甫鼎,容五斗。



 吳孫權赤烏十二年六月戊戌,寶鼎出臨平湖。又出東部酃縣。



 吳孫皓寶鼎元年八月,在所言得大鼎。



 晉愍帝建興二年十二月,晉陵武進縣民陳龍在田中得銅鐸五枚。



 晉成帝咸和元年十月辛卯,宣城舂穀縣山岸崩,獲石鼎重二斤,受斛餘。晉成帝咸康五年,豫章南昌民掘地得銅鐘四枚,太守褚裒以獻。



 晉穆帝升平五年二月乙未,南掖門有馬足陷地,得銅鐘一枚。



 宋文帝元嘉十三年四月辛丑,武昌縣章山水側自開出神鼎,江州刺史南譙王義宣以獻。
 元嘉十九年九月戊申,廣陵肥如石梁澗中出石鐘九口,大小行次,引列南向,南兗州刺史臨川王義慶以獻。元嘉二十一年十二月,新陽獲古鼎於水側,有篆書四十二字,雍州刺史蕭思話以獻。元嘉二十二年,豫章豫寧縣出銅鐘,江州刺史廣陵王紹以獻。



 孝武帝孝建三年四月丁亥,臨川宜黃縣民田中得銅鐘七口,內史傅徽以獻。
 孝建三年四月甲辰,晉陵延陵得古鐘六口,徐州刺史竟陵王誕以獻。孝武帝大明七年六月,江夏蒲圻獲銅路鼓,四面獨足,郢州刺史安陸王子綏以獻。



 明帝泰始四年二月丙申,豫章望蔡獲古銅鐘,高一尺七寸,圍二尺八寸,太守張辯以獻。泰始五年五月壬戌,豫章南昌獲古銅鼎,容斛七斗,江州刺史王景文以獻。
 泰始七年六月甲寅,義陽郡獲銅鼎,受一斛,並蓋並隱起鏤,豫州刺史段佛榮以獻。



 從帝升明二年九月,建寧萬歲山澗中得銅鐘,長二尺一寸,豫州刺史劉懷珍以獻。



 漢宣帝元康二年夏,神雀集雍。元康三年春,神雀集泰山。宣帝元康三年春,五色雀以萬數,飛過屬縣。元康四年三月,神雀五采以萬數,飛過集長樂、未央、北
 宮、高寢、甘泉泰畤殿。元康四年,神雀仍集。漢宣帝五鳳三年正月,神雀集京師。



 漢明帝永平十七年春,神雀五色集京師。



 漢章帝元和中,神雀見郡國。



 宋文帝元嘉二十二年,白鵲見新野鄧縣,雍州刺史蕭思話以聞。元嘉二十六年五月癸酉,白鵲見建康崇孝里,揚州刺史始興王濬以獻。



 孝武帝大明七年三月辛巳,白鵲見汝南安陽,太守申令孫以獻。



 晉惠帝永嘉元年五月,白鼠見東宮,皇太子獲以獻。



 宋明帝泰始三年二月壬寅,白鼠見樂安,青州刺史沈文秀以獻。



 漢昭帝始元元年二月,黃鵠下建章宮太液池中。



 漢章帝元和二年二月,車駕東巡,柴祭岱宗。禮畢,黃鵠三十從西南來,經祠壇上東北過。



 漢武帝太初三年二月五日,行幸東海,獲赤雁。



 魏文帝初,鑊中生赤魚。



 孫權時,神雀巢朱雀門。



 孫皓天璽元年,臨海郡吏伍曜在海水際得石樹,高三尺餘,枝莖紫色,詰屈傾靡,有光采。《山海經》所載玉碧樹之類也。



 晉武帝泰始二年六月壬申,白鴿見酒泉延壽,延壽長王音以獻。



 晉成帝咸和九年五月癸酉,白鵝見吳國錢塘,內史虞潭以獻。



 安帝義熙元年,南康雩都嵩山有金雞,青黃色,飛集巖間。



 宋文帝元嘉二十二年,湘州刺史南平王鑠獻赤鸚鵡。



 孝武帝大明三年正月丙申,媻皇國獻赤白鸚鵡各一。



 宋文帝元嘉二十四年十月甲午,揚州刺史始興王濬獻白鸚鵡。



 孝武帝大明五年正月丙子,交州刺史垣閎獻白孔雀。



 明帝泰始三年五月乙亥,白鴝鵒見京兆,雍州刺史巴陵王休若以獻。



 漢桓帝延熹九年四月,濟陰、東郡、濟北、平原河水清。



 宋文帝元嘉二十四年二月戊戌,河、濟俱清,龍驤將軍、青冀二州刺史杜坦以聞。文帝元嘉二十五年五月,征北長史、廣陵太守範邈上言:「所領輿縣,前有大浦,控引潮流,水常淤濁。自比以來,
 源流清潔,纖鱗呈形。古老相傳,以為休瑞。」



 孝武帝孝建三年九月,濟、河清,冀州刺史垣護之以聞。孝武帝大明五年九月庚戌,河、濟俱清,平原太守申纂以聞。



 明帝泰始元年二月丙寅,揚、淮水清潔有異於常,州治中從事史張緒以聞。



 漢光武建武初,野繭、穀充給百姓。其後耕蠶稍廣,二事
 漸息。



 吳孫權黃龍三年夏,野蠶繭大如卵。



 宋文帝元嘉十六年,宣城宛陵廣野蠶成繭,大如雉卵,彌漫林谷,年年轉盛。



 孝武帝大明三年五月癸巳,宣城宛陵縣石亭山生野蠶,三百餘里,太守張辯以聞。孝武帝大明三年十一月己巳,肅慎氏獻楛矢石砮,高麗國譯而至。大明五年正月戊午元日,花雪降殿庭。時右衛將軍謝
 莊下殿,雪集衣。還白,上以為瑞。於是公卿並作花雪詩。史臣按《詩》云:「先集為霰。」《韓詩》曰:「霰,英也。」



 花葉謂之英。《離騷》云:「秋菊之落英。」左思云「落英飄濆」是也。然則霰為花雪矣。草木花多五出,花雪獨六出。



 明帝泰始二年五月甲寅,赭中獲石柏長三尺二寸,廣三尺五寸,揚州刺史建安
 漢和帝在位十七年,郡國言瑞應八十餘品,帝讓而不宣。



\end{pinyinscope}