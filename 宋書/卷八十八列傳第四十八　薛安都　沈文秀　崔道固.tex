\article{卷八十八列傳第四十八 薛安都 沈文秀 崔道固}

\begin{pinyinscope}

 薛安都,河東汾陰人也。世為彊族,同姓有三千家。父廣,為宗豪,高祖定關、河,以為上黨太守。安都少以勇聞,身長七尺八寸,便弓馬。索虜使助秦州刺史北賀汨擊反
 胡白龍子,滅之。由是為偽雍、秦二州都統,州各有刺史,都統總其事。



 元嘉二十一年,索虜主拓跋燾擊芮芮大敗,安都與宗人薛永宗起義,永宗營汾曲,安都襲得弘農。會北地人蓋吳起兵,遂連衡相應。燾自率眾擊永宗,滅其族,進擊蓋吳。安都料眾寡不敵,率壯士辛靈度等,棄弘農歸國。太祖延見之,求北還構扇河、陜,招聚義眾。上許之,給錦百匹,雜繒三百匹。復襲弘農,虜已增戍,城不可克,蓋吳又死,乃退還上洛。世祖鎮襄陽,板為揚武
 將軍、北弘農太守。虜漸彊盛,安都乃歸襄陽。從叔沈亦同歸國,官至綏遠將軍、新野太守。



 二十七年,隨王誕版安都為建武將軍,隨柳元景向關、陜,率步騎居前,所向克捷,事在元景傳。軍還,誕版為後軍行參軍。二十九年,除始興王濬征北行參軍,加建武將軍。魯爽向虎牢,安都復隨元景北出,即據關城,期俱濟河取蒲阪。會爽退,安都復率所領隨元景引還。仍伐西陽五水蠻。



 世祖伐逆,轉參軍事,加寧朔將軍,領馬軍,與柳元景俱發。四月十
 四日,至朱雀航,橫矛瞋目,叱賊將皇甫安民等曰:「賊弒君父,何心事之!」世祖踐阼,除右軍將軍。五月四日,率所領騎為前鋒,直入殿庭。賊尚有數百人,一時奔散。



 以功封南鄉縣男,食邑五百戶。安都從征關、陜,至臼口,夢仰頭視天,正見天門開,謂左右曰:「汝見天門開不?」至是歎曰:「夢天開,乃中興之象邪!」



 從弟道生,亦以軍功為大司馬參軍。犯罪,為秣陵令庾淑之所鞭。安都大怒,乃乘馬從數十人,令左右執槊,欲往殺淑之。行至朱雀航,逢柳
 元景。元景遙問:「薛公何處去?」安都躍馬至車後曰:「小子庾淑之鞭我從弟,今詣往刺殺之。」



 元景慮其不可駐,乃紿之曰:「小子無宜適,卿往與手,甚快。」安都既回馬,復追呼之:「別宜與卿有所論。」令下馬入車。既入車,因責讓之曰:「卿從弟服章言論,與寒細不異,雖復人士,庾淑之亦何由得知?且人身犯罪,理應加罰,卿為朝廷勳臣,宜崇奉法憲,云何放恣,輒欲於都邑殺人?非唯科律所不容,主上亦無辭以相宥。」因載之俱歸,安都乃止。其年,以憚
 直免官。



 孝建元年,復除左軍將軍。二月,魯爽反叛,遣安都及冗從僕射胡子反、龍驤將軍宗越率步騎據歷陽。爽遣將鄭德玄戍大峴,德玄使前鋒楊胡與輕兵向歷陽。安都遣宗越及歷陽太守程天祚逆擊破之,斬胡與及其軍副。德玄復使其司馬梁嚴屯峴東,安都幢主周文恭晨往偵候,因而襲之,悉禽;賊未敢進。世祖詔安都留三百人守歷陽,渡還採石,遷輔國將軍、竟陵內史。四月,魯爽使弟瑜率三千人出小峴,爽尋以大眾阻大峴。
 又遣安都步騎八千度江,與歷陽太守張幼緒等討爽。安都軍副建武將軍譚金率數十騎挑戰,斬其偏帥。幼緒恇怯,輒引軍退還,安都復還歷陽。



 臧質久不至,世祖復遣沈慶之濟江督統諸軍。爽軍食少,引退,慶之使安都率輕騎追之;四月丙戌,及爽於小峴,爽自與腹心壯騎繼後。譚金先薄之,不能入,安都望見爽,便躍馬大呼,直往刺之,應手而倒,左右范雙斬爽首。爽累世梟猛,生習戰陳,咸云萬人敵。安都單騎直入,斬之而反,時人皆
 雲關羽之斬顏良,不是過也。



 進爵為侯,增邑五百戶,并前千戶。



 時王玄謨距南郡王義宣、臧質於梁山,安都復領騎為支軍。賊有水步營在蕪湖,安都遣將呂興壽率數十騎襲之,賊眾驚亂,斬首及赴水死者甚眾。義宣遣將劉湛之及質攻玄謨,玄謨命眾軍擊之,使安都引騎出賊陣右。譚金三歷賊陳,乘其隙縱騎突之,諸將係進。是朝,賊馬軍發蕪湖,欲來會戰,望安都騎甚盛,隱山不敢出。



 賊陣東南猶堅,安都橫擊陷之,賊遂大潰。安都隊主
 劉元儒於艦中斬湛之首。轉太子左衛率。大明元年,虜向無鹽,東平太守劉胡出戰失利。二月,遣安都領馬軍北討,東陽太守沈法系水軍向彭城,並受徐州刺史申坦節度。上戒之曰:「賊若可及,便盡力殄之。若度已回,可過河耀威而反。」時虜已去,坦求回軍討任榛,見許。



 安都當向左城,左城去滑臺二百餘里,安都以去虜鎮近,軍少不宜分行。至東坊城,遇任榛三騎,討擒其一,餘兩騎得走。任榛聞知,皆得逃散。時天旱,水泉多竭,人馬疲困,不
 能遠追。安都、法系並白衣領職,坦繫尚方。任榛大抵在任城界,積世逋叛所聚,所在皆棘榛深密,難為用師,故能久自保藏,屢為民患。安都明年復職,改封武昌縣侯,加散騎常侍。七年,又加征虜將軍,為太子左衛率十年,終世祖世不轉。



 前廢帝即位,遷右衛將軍,加給事中。永光元年,出為使持節、督兗州諸軍事、前將軍、兗州刺史。景和元年,代義陽王昶督徐州豫州之梁郡諸軍事、平北將軍、徐州刺史。太宗即位,進號安北將軍,給鼓吹一
 部。安都不受命,舉兵同晉安王子勛。初,安都從子索兒,前廢帝景和中,為前軍將軍,直閣,從誅諸公,封武安縣男,食邑三百戶。太宗即位,以為左將軍,直閣如故。安都將為逆,遣密信報之,又遣數百人至瓜步迎接。時右衛將軍柳光世亦與安都通謀。



 泰始二年正月,索兒、光世並在省,安都信催令速去,二人俱自省逃出,攜安都諸子及家累,席卷北奔。青州刺史沈文秀、冀州刺史崔道固並皆同反。文秀遣劉彌之、張靈慶、崔僧FM三軍,道固
 遣子景征、傅靈越領眾,並應安都。彌之等南出下邳,靈越自泰山道向彭城。時濟陰太守申闡據睢陵城起義,索兒率靈越等攻之。



 安都使同黨裴祖隆守下邳城,彌之等至下邳,改計歸順,因進軍攻祖隆,僧FM不同,率所領歸安都。索兒聞彌之有異志,舍睢陵馳赴下邳,彌之等未戰潰散,並為索兒所執,見殺。



 時太宗以申令孫為徐州,代安都。令孫進據淮陽,密有反志,遣人告索兒曰:「欲相從順,而百口在都。可進軍見攻,若戰敗被執,家人
 可得免禍。」索兒乃遣靈越向淮陽,令孫出城,為相距之形,既而奔散,北投索兒。索兒使令孫說闡令降,闡既降,索兒執闡及令孫,並殺之。索兒因引軍渡淮,軍糧不給,掠奔百姓穀食。



 太宗遣齊王率前將軍張永、寧朔將軍垣山寶、王寬、員外散騎侍郎張寘震、蕭順之、龍驤將軍張季和、黃文玉等諸軍北討。其年五月,軍次平原,索兒等率馬步五千,列陳距戰,擊大破之。索兒又虜掠民穀,固守石梁,齊王又率鎮北參軍趙曇之、呂湛之擊之。索
 兒軍無資實,所資野掠,既見攻逼,無以自守,於是奔散;又追破之於葛家白鵠。索兒走向樂平縣界,為申令孫子孝叔所斬。安都子道智、大將范雙走向合肥,詣南汝陰太守裴季降。



 時武衛將軍王廣之領軍隸劉勔,攻殷琰於壽陽。傅靈越奔逃,為廣之軍人所生禽,厲聲曰:「我傅靈越也。汝得賊何不即殺。」生送詣勔,勔躬自慰勞,詰其叛逆。對曰:「九州唱義,豈獨在我。」勔又問:「四方阻逆,無戰不禽,主上皆加以曠蕩,即其才用。卿何不早歸天闕,
 乃逃命草間乎?」靈越答曰:「薛公舉兵淮北,威震天下,不能專任智勇,委付子侄,致敗之由,實在於此。然事之始末,備皆參豫,人生歸於一死,實無面求活。」勔壯其意,送還京師。太宗欲加原宥,靈越辭對如一,終不回改,乃殺之。靈越,清河人也。時輔國將軍、山陽內史程天祚據郡同安都,攻圍彌時,然後歸順。



 子勛平定,安都遣別駕從事史畢眾愛、下邳太守王煥等奉啟書詣太宗歸款,曰:「臣庸隸荒萌,偷生上國,過蒙世祖孝武皇帝過常之恩,
 犬馬有心,實感恩遇。是以晉安始唱,投誠孤往,不期生榮,實存死報。今天命大歸,群迷改屬,輒率領所部,束骸待誅,違拒之罪,伏聽湯鑊。」索兒之死也,安都使柳光世守下邳,至是亦率所領歸降。太宗以四方已平,欲示威於淮外,遣張永、沈攸之以重軍迎之。安都謂既已歸順,不應遣重兵,懼不免罪,乃遣信要引索虜。三年正月,索虜遣博陵公尉遲茍人、城陽公孔伯恭二萬騎救之。永等引退,安都開門納虜,虜即授安都徐州刺史、河東公。
 四年三月,召還桑乾。五年,死於虜中,時年六十。



 初,安都起兵,長史蘭陵儼密欲圖之,見殺。安都未向桑乾,前軍將軍裴祖隆謀殺茍人,舉彭城歸順,事洩,見誅。員外散騎侍郎孫耿之擊索兒戰死,及劉彌之、張靈慶皆戰敗見殺,並為太宗所哀,追贈儼光祿勳,祖隆寧朔將軍、兗州刺史,耿之羽林監,彌之輔國將軍、青州刺史,靈慶寧朔將軍、冀州刺史。



 安都子伯令、環龍,亡命梁、雍二州之間。三年,率亡命數千人襲廣平,執太守劉冥虯,攻順陽,
 克之,略有義成、扶風,置立守宰。雍州刺史巴陵王休若遣南陽太守張敬兒、新野太守劉攘兵擊破之,並禽。先是,東安、東莞二郡太守張讜守團城,在彭城東北。始同安都,未亦歸順,太宗以為東徐州刺史,復為虜所沒。



 沈文秀,字仲遠,吳興武康人,司空慶之弟子也。父劭之,南中郎行參軍。文秀初為郡主簿,功曹史,慶之貴後,文秀起家為東海王禕撫軍行參軍;又度義陽王昶東中郎府,東遷錢唐令,西陽王子尚撫軍參軍,武康令,尚書
 庫部郎,本邑中正,建康令。坐為尋陽王鞭殺私奴,免官,加杖一百;尋復官。前廢帝即位,為建安王休仁安南錄事參軍,射聲校尉。



 景和元年,遷督青州之東莞東安二郡諸軍事、建威將軍、青州刺史。時帝狂悖無道,內外憂危,文秀將之鎮,部曲出屯白下,說慶之曰:「主上狂暴如此,土崩將至,而一門受其寵任,萬物皆謂與之同心。且此人性情無常,猜忌特甚,將來之禍,事又難測。今因此眾力,圖之易於反掌,千載一時,萬不可失。」慶之不從。



 文
 秀固請非一,言輒流涕,終不回。文秀既行,慶之果為帝所殺。慶之死後,帝遣直閣江方興領兵誅文秀,方興未至,太宗已定亂,馳驛駐之。方興既至,為文秀所執。尋見釋,遣還京師。



 時晉安王子勛據尋陽反叛,六師外討,徵兵於文秀。文秀遣劉彌之、張靈慶、崔僧FM三軍赴朝廷。時徐州刺史薛安都已同子勛,遣使報文秀,以四方齊舉,勸令同逆,文秀即令彌之等回應安都。彌之等尋歸順,事在《安都傳》。彌之青州彊姓,門族甚多,諸宗從相合
 率奔北海,據城以拒文秀。平原、樂安二郡太守王玄默據琅邪,清河、廣川二郡太守王玄邈據盤陽城,高陽、勃海二郡太守劉乘民據臨濟城,並起義。文秀司馬房文慶謀應之,為文秀所殺。文秀遣軍主解彥士攻北海陷之,乘民從弟伯宗合率鄉兵,復克北海,因率所領向青州所治東陽城。文秀拒之,伯宗戰敗被創,弟天愛扶持將去,伯宗曰:「丈夫當死戰場,以身殉國,安能歸死兒女手中乎!弟可速去,無為兩亡。」乃見殺,追贈龍驤將軍、長
 廣太守。



 太宗遣青州刺史明僧皓、東莞東安二郡太守李靈謙率軍伐文秀。玄邈、乘民、僧皓等並進軍攻城,每戰輒為文秀所破,離而復合,如此者十餘。泰始二年八月,尋陽平定,太宗遣尚書度支郎崔元孫慰勞諸義軍,隨僧皓戰敗見殺,追贈寧朔將軍、冀州刺史。上遣文秀弟文炳詔文秀曰:「皇帝前問督青州徐州之東莞東安二郡諸軍事、建威將軍、青州刺史,朕去歲撥亂,功振普天,於卿一門,特有殊澤,卿得延命至今,誰之力邪?何故
 背國負恩,遠同逆豎。今天下已定,四方寧一,卿獨守窮城,何所歸奉?且卿百口在都,兼有墳墓,想情非木石,猶或顧懷。故指遣文炳具相宣示。凡諸逆郎,親為戎首,一不加罪,文炳所具。卿獨何人,而能自立。便可速率部曲,同到軍門,別詔有司,一無所問。如其不爾,國有常刑,非惟戮及弟息,亦當夷卿墳壟,既以謝齊土百姓,亦以勞將士之心。故有今詔。」三年二月,文秀歸命請罪,即安本任。



 先是,冀州刺史崔道固亦據歷城同逆,為土人起義
 所攻,與文秀俱遣信引虜;虜遣將慕輿白曜率大眾援之,文秀已受朝命,乃乘虜無備,縱兵掩擊,殺傷甚多。



 虜乃進軍圍城,文秀善於撫御,將士咸為盡力,每與虜戰,輒摧破之,掩擊營寨,往無不捷。太宗進文秀號輔國將軍。其年八月,虜蜀郡公拔式等馬步數萬人入西郭,直至城下。文秀使輔國將軍垣諶擊破之。九月,又逼城東。十月,進攻南郭。文秀使員外散騎侍郎黃彌之等邀擊,斬獲數千。四年,又進文秀號右將軍,封新城縣侯,食邑
 五百戶。虜青州刺史王隆顯於安丘縣又為軍主高崇仁所破,死者數百人。虜圍青州積久,太宗所遣救兵並不敢進,乃以文秀弟徵北中兵參軍文靜為輔國將軍,統高密、北海、平昌、長廣、東萊五郡軍事,從海道救青州。文靜至東萊之不其城,為虜所斷遏,不得進,因保城自守,又為虜所攻,屢戰輒剋,太宗加其東青州刺史。



 四年,不其城為虜所陷,文靜見殺。



 文秀被圍三載,外無援軍,士卒為之用命,無離叛者,日夜戰鬥,甲胄生蟣虱。



 五年正
 月二十四日,遂為虜所陷。城敗之日,解釋戎衣,緩服靜坐,命左右取所持節。虜既入,兵刃交至,問曰:「青州刺史沈文秀何在?」文秀厲聲曰:「身是。」



 因執之,牽出聽事前,剝取衣服。時白曜在城西南角樓,裸縛文秀至曜前,執之者令拜。文秀曰:「各二國大臣,無相拜之禮。」曜命還其衣,為設酒食,鎖送桑乾。



 其餘為亂兵所殺,死者甚眾。太宗先遣尚書功論郎何如真選青州文武,亦為虜所殺。



 文秀在桑乾凡十九年,齊之永明四年,病死,時年六十一。



 崔道固,清河人也。世祖世,以幹用見知,歷太子屯騎校尉,左軍將軍。大明三年,出為齊、北海二郡太守。民焦恭破古冢,得玉鎧,道固檢得,獻之,執繫恭。



 入為新安王子鸞北中郎諮議參軍,永嘉王子仁左軍司馬。景和元年,出為寧朔將軍、冀州刺史,鎮歷城。泰始二年,進號輔國將軍,又進號征虜將軍。時徐州刺史薛安都同逆,上即還道固本號為徐州代之。道固不受命,遣子景微、軍主傅靈越率眾赴安都。既而為土人起義所攻,屢戰失利,
 閉門自守。會四方平定,上遣使宣慰,道固奉詔歸順。先是與沈文秀共引虜,虜既至,固守距之,因被圍逼。虜每進,輒為道固所摧。三年,以為都督冀青兗幽並五州諸軍事、前將軍、冀州刺史,加節,又進號平北將軍。其年,為虜所陷,被送桑乾,死於虜中。



 史臣曰:《春秋》列國大夫得罪,皆先致其邑而後去,唯邾、莒三臣,書以叛人之目,蓋重地也。安都勤王之略,義闕於籓屏,以地外奔,罪同於三叛。《詩》云:「誰生厲階,至今為
 梗。」其此之謂乎?



\end{pinyinscope}