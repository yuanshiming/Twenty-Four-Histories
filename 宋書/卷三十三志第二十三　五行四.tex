\article{卷三十三志第二十三 五行四}

\begin{pinyinscope}

 《五行傳》曰:「
 簡宗廟,不禱祠,廢祭祀,逆天時,則水不潤下。」謂水失其性而為災也。又曰:「聽之不聰,是謂不謀。厥咎急,厥罰恒寒,厥極貧。時則有鼓妖,時則有魚孽,時則有豕禍,時
 則有耳痾,時則有黑眚、黑祥。惟火沴水。」



 魚孽,劉歆傳以為介蟲之孽,謂蝗屬也。



 水不潤下:魏文帝黃初四年六月,大雨霖,伊、洛溢至津陽城門,漂數千家,流殺人。初,帝即位,自鄴遷洛,營造宮室,而不起宗廟,太祖神主猶在鄴。嘗於建始殿饗祭如家人之禮,終黃初不復還鄴,而圓丘、方澤、南北郊、社、稷等神位,未有定所。



 此簡宗廟,廢祭祀之罰也。京房《易傳》曰:「顓事者
 知,誅罰絕理,厥災水。其水也,雨殺人已隕霜,大風天黃。饑而不損,茲謂泰。厥災水殺人。避遏有德,茲謂狂。厥災水,水流殺人也;已水則地生蟲。歸獄不解,茲謂追非。厥水寒殺人。



 追誅不解,茲謂不理。厥水五穀不收。大敗不解,茲謂皆陰。厥水流入國邑,隕霜殺穀。」



 吳孫權赤烏八年夏,茶陵縣鴻水溢出,流漂二百餘家;十三年秋,丹陽故鄣等縣又鴻水溢。案權稱帝三十年,竟不於建業創七廟,但有父堅一廟,遠在長沙,而郊禋
 禮禮闕。嘉禾初,群臣奏宜效祀,又弗許。末年雖一南郊,而北郊遂無聞焉。



 且三江、五湖、衡、霍、會稽,皆吳、楚之望,亦不見秩,反禮羅陽妖神,以求福助。天意若曰,權簡宗廟,不禱祠,廢祭祀,示此罰,欲其感悟也。



 太元元年,又有大風涌水之異。是冬,權南郊,疑是鑒咎徵乎?還而寢疾。明年四月,薨。一曰,權時信納譖訴,雖陸議勳重,子和儲貳,猶不得其終。與漢安帝聽讒、免楊震、廢太子同事也。且赤烏中無年不用兵,百姓愁怨。八年
 秋,將軍馬茂等又圖逆云。



 魏明帝景初元年九月,淫雨過常,冀、兗、徐、豫四州水出,沒溺殺人,漂失財產。帝自初即位,便淫奢極欲,多占幼女,或奪士妻,崇飾宮室,妨害農戰,觸情恣欲,至是彌甚。號令逆時,饑不損役,此水不潤下之應也。



 吳孫亮五鳳元年夏,大水。亮即位四年,乃立權廟;又終吳世,不上祖宗之號,不修嚴父之禮,昭穆之數有闕。亮及休、皓又並廢二郊,不秩群神。此簡宗廟,不祭祀之罰
 也。又是時,孫峻專政,陰勝陽之應乎。



 吳孫休永安四年五月,大雨,水泉涌溢。昔歲作浦里塘,功費無數,而田不可成,士卒死叛,或自賊殺,百姓愁怨,陰氣盛也。休又專任張布,退盛沖等,吳人賊之之應也。吳孫休永安五年八月壬午,大雨震電,水泉湧溢。



 晉武帝泰始四年九月,青、徐、兗、豫四州大水。七年六月,大雨霖,河、洛、伊、沁皆溢,殺二百餘人。帝即尊位,不加三后祖宗之號。泰始二年,又除明堂南郊五帝坐,同稱昊
 天上帝,一位而已。又省先后配地之禮。此簡宗廟,廢祭祀之罰,與漢成帝同事。一曰,昔歲及此年,藥蘭泥、白虎文秦涼殺刺史胡烈、牽弘,遣田璋討泥。又司馬望以大眾次淮北禦孫皓。內外兵役,西州饑亂,百姓愁怨,陰氣盛也。咸寧初,始上祖宗號,太熙初,還復五帝位。



 晉武帝咸寧元年九月,徐州水。二年七月癸亥,河南魏郡暴水,殺百餘人;八月,荊州郡國五大水。去年采擇良家子女,露面入殿,帝親簡閱,務在姿色,不訪德行。有蔽
 匿者,以不敬論。搢紳愁怨,天下非之。陰盛之應也。咸寧三年六月,益、梁二州郡國八暴水,殺三百餘人;七月,荊州大水;九月,始平郡大水;十月,青、徐、兗、豫、荊、益、梁七州又水。是時賈充等用事日盛,而正人疏外者多。



 咸寧四年七月,司、冀、兗、豫、荊、揚郡國二十大水。



 晉武帝太康二年六月,泰山、江夏大水。泰山流三百家,殺六千餘人;江夏亦殺人。是時平吳後,王濬為元功,而詆劾妄加;荀、賈為無謀,而並蒙重賞。收吳姬五千,納之
 後宮。此其應也。太康四年七月,司、豫、徐、兗、荊、揚郡國二十大水,傷秋稼,壞屋室,有死者。太康六年三月,青、涼、幽、冀郡國十五大水。



 太康七年九月,西方安定等郡國八大水。太康八年六月,郡國八大水。晉惠帝元康二年,有水災。元康五年五月,潁川、淮南大水;六月,城陽、東莞大水殺
 人;荊、揚、徐、兗、豫五州又大水。是時帝即位已五載,猶未郊祀,烝嘗亦多不身親近。



 簡宗廟,廢祭祀之罰也。班固曰:「王者即位,必郊祀天地,望秩山川。若乃不敬鬼神,政令違逆,則霧水暴至,百川逆溢,壞鄉邑,溺人民,水不潤下也。」元康六年五月,荊、揚二州大水。按董仲舒說,水者,陰氣盛也。是時賈后亂朝,寵樹賈、郭。女主專政之應也。元康八年五月,金墉城井水溢。漢成帝時有此妖,班固
 以為王莽之象。及趙倫篡位,即此應也。倫廢帝於此城,井溢所在,又天意乎!元康八年九月,荊、揚、徐、兗、冀五州大水。是時賈后暴戾滋甚,韓謐驕猜彌扇,卒害太子,旋亦禍滅。元康九年四月,宮中井水沸溢。



 晉惠帝永寧元年七月,南陽、東海大水。是時,齊王冏秉政專恣。陰盛之應。



 晉惠帝太安元年七月,兗、豫、徐、冀四州水。時將相力政,
 無尊主心。



 晉孝懷帝永嘉四年四月,江東大水。是時,王導等潛懷翼戴之計。陰氣盛也。



 晉元帝太興三年六月,大水。是時王敦內懷不臣,傲很作威,後終夷滅。大興四年七月,大水。明年有石頭之敗。



 晉元帝永昌二年五月,荊州及丹陽、宣城、吳興、壽春大水。



 晉明帝太寧元年五月,丹陽、宣城、吳興、壽陽大水。是時王敦疾害忠良,威權震主,尋亦誅滅。



 晉成帝咸和元年五月,大水。是時嗣主幼沖,母后稱制,庾亮以元舅民望,決事禁中。陰勝陽也。咸和二年五月戊子,京都大水。是冬,蘇峻稱兵,都邑塗炭。



 咸和四年七月,丹陽、宣城、吳興、會稽大水。是冬,郭默作亂,荊、豫共討之,半歲乃定。
 咸和七年五月,大水。是時帝未親務,政在大臣。陰勝陽也。



 晉成帝咸康元年八月,長沙、武陵大水。是年三月,石虎掠騎至歷陽,四月,圍襄陽。於是加王導大司馬,集徒旅;又使趙胤、路永、劉仕、王允之、陳光五將軍,各帥眾戍衛。百姓愁怨。陰氣盛也。



 晉穆帝永和四年五月,大水。是時幼主沖弱,母后臨朝;又將相大臣,各爭權政。與咸和初同事也。
 永和五年五月,大水。永和六年五月,大水。永和七年七月甲辰夜,濤水入石頭,死者數百人。去年,殷浩以私忿廢蔡謨,遐邇非之。又幼主在上,而殷、桓交惡,選徒聚甲,各崇私權。陰勝陽之應也。一說濤入石頭,江右以為兵占。是後殷浩、桓溫、謝尚、荀羨連年征伐。



 晉穆帝升平二年五月,大水。是時桓溫權制朝廷,征伐是專。
 升平五年四月,大水。



 晉海西太和六年六月,京都大水,平地數尺,侵及太廟。朱雀大航纜斷,三艘流入大江。丹陽、晉陵、吳國、吳興、臨海五郡又大水,稻稼蕩沒,黎庶饑饉。初四年,桓溫北伐敗績,十喪其九;五年,又征淮南,踰歲乃克。百姓愁怨之應也。



 晉簡文帝咸安元年十二月壬午,濤水入石頭。明年,妖賊盧竦率其屬數百人入殿,略取武庫三庫甲仗,游擊
 將軍毛安之討滅之。



 晉孝武帝太元三年六月,大水。是時孝武幼弱,政在將相。太元五年,大水。



 去年氐賊攻沒襄陽,又向廣陵。於是逼徙江、淮民悉令南渡,三州失業,道饉相望。



 謝玄雖破句難等,自後征戍不已。百姓愁怨之應也。太元六年六月,荊、江揚三州大水。太元十年夏,大水。初八年,破苻堅,自後有事中州,役無
 已歲。兵民愁怨之應也。太元十三年十二月,濤水入石頭。明年,丁零、鮮卑寇擾司、兗鎮戍,西、北疲於奔命。太元十五年七月,兗州大水。是時緣河紛爭,征戍勤悴。太元十七年六月甲寅,濤水入石頭,毀大航,漂船舫,有死者;京口西浦,亦濤入殺人。永嘉郡潮水涌起,近海四縣人民多死。後四年帝崩,而王恭再攻京師。京師亦發大眾以禦之。
 太元十九年七月,荊州、彭城大水傷稼。太元二十年,荊州、彭城大水。太元二十一年五月癸卯,大水。是時政事多弊,兆庶非之。



 晉安帝隆安三年五月,荊州大水。去年,殷仲堪舉兵向京都;是年春,又殺郗恢。陰盛作威之應也。仲堪尋亦敗亡。隆安五年五月,大水。是時司馬元顯作威陵上,又桓玄
 擅西夏,孫恩亂東國。陰勝陽之應也。



 晉安帝元興二年十二月,桓玄篡位。其明年二月庚寅夜,濤水入石頭。是時貢使商旅,方舟萬計,漂敗流斷,骸胔相望。江右雖有濤變,未有若斯之甚。三月,義軍克京都,玄敗走,遂夷滅。元興三年二月己丑朔夜,濤水入石頭,漂沒殺人,大航流敗。



 晉安帝義熙元年十二月己未,濤水入石頭。
 義熙二年十二月己未夜,濤水入石頭。明年,駱球父環潛結桓胤、殷仲文等謀作亂,劉稚亦謀反,凡所誅滅數十家。



 義熙三年五月丙午,大水。義熙四年十二月戊寅,濤水入石頭。明年,王旅北討鮮卑。義熙六年五月丁巳,大水。乙丑,盧循至蔡洲。義熙八年六月,大水。
 義熙九年五月辛巳,大水。義熙十年五月丁丑,大水;戊寅,西明門地穿涌水出,毀門扉及限;七月乙丑,淮北災風大水殺人。義熙十一年七月丙戌,大水,淹漬太廟,百官赴救。明年,王旅北討關、河。



 宋文帝元嘉五年六月,京邑大水。七年,右將軍到彥之率師入河。元嘉十一年五月,京邑大水。十三年,司空檀道濟誅。
 元嘉十二年六月,丹陽、淮南、吳、吳興、義興五郡大水,京邑乘船。元嘉十八年五月,江水泛溢,沒居民,害苗稼。明年,右軍將軍裴方明率雍、梁之眾伐仇池。元嘉十九年、二十年,東諸郡大水。元嘉二十九年五月,京邑大水。



 孝武帝孝建元年八月,會稽大水,平地八尺。後二年,虜寇青、冀州,遣羽林軍卒討伐。
 孝武帝大明元年五月,吳興、義興大水。大明四年八月,雍州大水。大明四年,南徐、南兗州大水。



 後廢帝元徽元年六月,壽陽大水。



 順帝升明元年七月,雍州大水,甚於關羽樊城時。升明二年二月,於潛翼異山一夕五十二處水出,流漂居民。七月丙午朔,濤水入石頭,居民皆漂沒。



 恆寒:
 庶徵之恒寒,劉歆以為「大雨雪、及未當雨雪而雨雪、及大雨雹、隕霜殺菽草,皆常寒之罰也」。京房《易傳》曰:「有德遭險,茲謂逆命。厥異寒。誅罰過深,當燠而寒,盡六日,亦為雹。害正不誅,茲謂養賊。寒七十二日,殺飛禽。道人始去,茲謂傷。其寒物無霜而死,涌水出。戰不量敵,茲謂辱命。其寒雖雨物不茂。」



 吳孫權嘉禾三年九月朔,隕霜傷穀。按劉向說:「誅罰不由君出,在臣下之象也」。是時校事呂壹專作威福,與漢
 元帝時石顯用事隕霜同應。班固書九月二日,陳壽言朔,皆明未可以傷穀也。壹後亦伏誅。京房《易傳》曰:「興兵妄誅,茲謂亡法。厥災霜,夏殺五穀,冬殺麥。誅不原情,茲謂不仁。其霜夏先大雷風,冬先雨,乃隕霜,有芒角。賢聖遭害,其霜附木不下地。佞人依刑,茲謂私賊。其霜在草根土隙間。不教而誅,茲謂虐。其霜反在草下。」



 嘉禾四年七月,雨雹,又隕霜。案劉向說:「雹者陰脅陽」。是時呂壹作威用事,詆毀重臣,排陷無辜。自太子登以下,
 咸患毒之,而壹反獲封侯寵異。與《春秋》公子遂專任,雨雹同應也。漢安帝信讒,多殺無辜,亦雨雹。董仲舒曰「凡雹皆為有所脅,行專一之政」故也。



 吳孫權赤烏四年正月,大雪,平地深三尺,鳥獸死者太半。是年夏,全琮等四將軍攻略淮南、襄陽,戰死者千餘人。其後權以讒邪,數責讓陸議,議憤恚致卒。



 與漢景、武大雪同事也。赤烏十一年四月,雨雹。是時權聽讒,將危太子。其後朱
 據、屈晃以迕意黜辱,陳象以忠諫族誅,而太子終廢。此有德遭險,誅罰過深之應也。



 晉武帝泰始六年冬,大雪。泰始七年十二月,大雪。明年。有步闡、楊肇之敗,死傷甚眾。泰始九年四月辛未,隕霜。是時賈充親黨比周用事。與魯定公、漢元帝時隕霜同應也。



 晉武帝咸寧三年八月,平原、安平、上黨、秦郡霜害三豆。
 咸寧三年八月,河間暴風寒冰,郡國五隕霜傷穀。是後大舉征吳,馬隆又帥精勇討涼州。咸寧五年五月丁亥,鉅鹿、魏郡雨雹傷禾、麥;辛卯,鴈門雨雹傷秋稼#咸寧五年六月庚戌,汲郡、廣平、陳留、滎陽雨雹;丙辰,又雨雹,損傷秋麥千三百餘頃,壞屋百三十餘間;癸亥,安定雨雹;七月丙申,魏郡又雨雹;閏月壬子,新興又雨雹;八月庚子,河東、弘農又雨雹,兼傷秋稼三豆。



 晉武帝太康元年三月,河東、高平霜雹,傷桑、麥;四月,河南、河內、河東、魏郡、弘農雨雹,傷麥、豆;五月,東平、平陽、上黨、雁門、濟南雨雹,傷禾、麥、三豆。太康元年四月庚午,畿內縣二及東平范陽縣雨雹;癸酉,畿內縣五又雨雹。是時王浚有大功,而權戚互加陷抑,帝從容不斷。陰脅陽之應也。太康二年二月辛酉,殞霜于濟南、琅邪,傷麥;壬申,瑯邪雨雪傷麥;三月甲午,河東隕霜害桑。
 太康二年五月丙戌,城陽、章武、琅邪傷麥;庚寅,河東、樂安、東平、濟陰、弘農、濮陽、齊國、頓丘、魏郡、河內、汲郡、上黨雨雹,傷禾稼。太康二年六月,郡國十六雨雹。太康三年十二月,大雪。太康五年七月乙卯,中山、東平雨雹,傷秋稼。太康五年七月甲辰,中山雨雹;九月,南安大雪折木。太康六年二月,東海霜傷桑、麥。
 太康六年三月戊辰,齊郡臨菑、長廣不其等四縣,樂安梁鄒等八縣,琅邪臨沂等八縣,河間易城等六縣,高陽北新城等四縣,隕霜傷桑、麥。太康六年六月,滎陽、汲郡、雁門雨雹。太康八年四月,齊國、天水二郡隕霜;十二月,大雪。太康九年正月,京都大風雨雹,發屋拔木;四月,隴西隕霜。太康十年四月,郡國八隕霜。



 晉惠帝元康二年八月,沛及湯陰雨雹。元康三年四月,滎陽雨雹;弘農湖、華陰又雨雹,深三尺。是時賈后凶淫專恣,與《春秋》魯桓夫人同事。陰氣盛也。元康五年六月,東海雨雹,深五寸;十二月,丹陽雨雹。元康五年十二月,丹陽建業大雪。元康六年三月,東海隕霜殺桑、麥。元康七年五月,魯國雨雹;七月,秦、雍二州隕霜殺稼。元康九年三月旬有八日,河南、滎陽、潁川隕霜傷禾;五
 月,雨雹。是時賈后凶躁滋甚,是冬遂廢愍懷。



 晉惠帝永寧元年七月,襄城雨雹。是時齊王冏專政。十月,襄城、河南、高平、平陽風雹,折木傷稼。晉惠帝光熙元年閏八月甲申朔,霰雪。劉向曰:「盛陽雨水湯熱,陰氣脅之,則轉而為雹。盛陰雨雪凝滯,陽氣薄之,則散而為霰。」今雪非其時,此聽不聰之應也。



 晉孝懷帝永嘉元年十二月冬,雪平地三尺。永嘉七年十月庚午,大雪。



 晉愍帝建興元年十一月戊午,會稽大雨震電。己巳夜,赤氣曜於西北。是夕,大雨震電。庚午,大雪。案劉向說,「雷以二月出,八月入」。此月雷電者,陽不閉藏也。既發泄而明日便大雪,皆失節之異也。是時劉載僭號平陽,李雄稱制於蜀,九州幅裂,西京孤微。為君失時之象。



 晉元帝太興二年三月丁未,成都風雹殺人。太興三年三月,海鹽郡雨雹。是時王敦陵上。



 晉元帝永昌二年十二月,幽、冀、并三州大雪。



 晉明帝太寧元年十二月,幽、冀、并州大雪。太寧二年四月庚子,京都大雨雹,燕雀死。太寧三年三月丁丑,雨雹;癸巳,隕霜;四月,大雨雹。是年帝崩,尋有蘇峻之亂。



 晉成帝咸和六年三月癸未,雨雹。是時帝幼弱,政在大臣。咸和九年八月,成都雪。其日李雄死。晉成帝咸康二年正月丁巳,皇后見于太廟。其夕雨雹。



 晉康帝建元元年八月,大雪。是時政在將相,陰氣盛也。與《春秋》魯昭公時季孫宿專政同事。劉向曰:「凡雨,陰也,雪又雨之陰也。出非其時,迫近象也。」



 晉穆帝永和三年八月,冀方大雪,人馬多凍死。永和五年六月,臨漳暴風震霆,雨雹大如升。永和十年五月,涼州雪。明年八月,桴罕護軍張瓘帥宗混等攻滅張祚,更立張曜靈弟玄靚。京房《易傳》曰:「夏雨雪,戒臣為亂。」
 永和十一年四月壬申朔,雪;十二月戊午,雷;己未,雷。是時帝幼,母后稱制,政在大臣。晉穆帝升平二年正月,大雪。



 晉孝武帝太元二年四月己酉,雨雹;十二月,大雪。是時帝幼弱,政在將相。



 太元十二年四月己丑,雨雹。是時有事中州,兵役連歲。太元二十年五月癸卯,上虞雨雹。太元二十一年四月丁亥,雨雹。是時張夫人專幸,及帝
 暴崩,兆庶尤之。



 太元二十一年十二月,連雪二十三日。是時嗣主幼沖,塚宰專政。



 晉安帝隆安二年三月乙卯,雨雹。是秋,王恭、殷仲堪入伐,終皆誅。晉安帝元興二年十二月,酷寒過甚。是時桓玄篡位,政事煩苛,是其應也。晉氏失在舒緩,玄則反之。劉向曰:「周衰無寒歲,秦滅無燠年。」此之謂也。
 元興三年正月甲申,霰雪,又雷。雷霰不應同日,失節之應也。二月,義兵起,玄敗。元興三年四月丙午,江陵雨雹。是時安帝蒙塵。



 晉安帝義熙元年四月壬申,雨雹。是時四方未一,鉦鼓日戒。義熙五年三月己亥,雪深數寸。義熙五年五月癸巳,溧陽雨雹;九月己丑,廣陵雨雹。明年,盧循至蔡洲。
 義熙五年九月己丑,廣陵雨雹。義熙六年正月丙寅,雪,又雷。義熙六年五月壬申,雨雹。義熙八年四月辛未朔,雨雹;六月癸亥,雨雹,大風發屋。是秋,誅劉籓等。義熙十年四月辛卯,雨雹。



 宋文帝元嘉九年春,京都雨雹,溧陽、盱眙尤甚,傷牛馬,殺禽獸。
 元嘉十八年三月,雨雹。二十五虜寇青州。元嘉二十五年正月,積雪冰寒。元嘉二十九年五月,盱眙雨雹,大如雞卵。三十年,國家禍亂,兵革大起。



 孝武帝大明元年十二月庚寅,大雪,平地二尺餘。明年,虜侵冀州,遣羽林軍北討。



 明帝泰始五年四月壬辰,京邑雨雹。



 後廢帝元徽三年五月乙卯,京邑雨雹。



 雷震:魏明帝景初中,洛陽城東橋、洛水浮橋桓楹,同日三處俱震;尋又震西城上候風木飛烏。時勞役大起,帝尋晏駕。



 吳孫權赤烏八年夏,震宮門柱;又擊南津大橋桓楹。



 孫亮建興元年十二月朔,大風震電;是月又雷雨。義同前說。亮終廢。



 晉武帝太康六年十二月甲申朔,淮南郡震電。
 太康七年十二月己亥,毗陵雷電,南沙司鹽都尉戴亮以聞。太康十年十二月癸卯,廬江、建安雷電大雨。



 晉惠帝永康元年六月癸卯,震崇陽陵標西南五百步,標破為七十片。是時賈后陷害鼎輔,寵樹私戚。與漢桓帝時震憲陵寢同事也。后終誅滅。晉惠帝永興二年十月丁丑,雷電。



 晉懷帝永嘉四年十月,震電。



 晉元帝永昌二年七月丙子朔,雷震太極殿柱。永昌二年十一月,會稽、吳郡雨震電。



 晉明帝太寧元年七月丙子朔,震太極殿柱。



 晉成帝咸和元年十月己巳,會稽郡大雨震電。咸和三年六月辛卯,臨海大雷,破郡府內小屋柱十枚,殺人。咸和三年九月二日立冬,會稽震電。咸和四年十二月,吳郡、會稽震電。
 咸和四年十二月,丹陽震電。



 晉穆帝永和七年十月壬午,雷雨、震電。晉穆帝升平元年十一月庚戌,雷;乙丑,又雷。升平五年十月庚午,雷發東南。



 晉孝武帝太元五年六月甲寅,雷震含章殿四柱。太元五年十二月,雷聲在南方。



 太元十四年七月甲寅,震宣陽門西柱。



 晉安帝隆安二年九月壬辰,雨雷。
 晉安帝元興三年,永安皇后至自巴陵。將設儀導入宮,天雷,震人馬各一俱殪。晉安帝義熙四年十一月辛卯朔,西北疾風;癸丑,雷。義熙五年六月丙寅,震太廟,破東鴟尾,徹壁柱。義熙六年正月丙寅,雷;丁卯,又雪。義熙六年十二月壬辰,大雷。義熙九年十一月甲戌,雷;乙亥,又雷。



 宋文帝元嘉四年十一月癸丑,雷。
 元嘉五年六月丙寅,震太廟,破東鴟尾,徹壁柱。元嘉六年正月丙寅,雷且雪。元嘉七年十月丙子,雷。元嘉八年十二月庚辰,雷。元嘉九年十一月甲戌,雷且雪。元嘉十四年,震初寧陵口標,四破至地。十七年,廢大將軍彭城王義康。骨肉相害,自此始也。



 前廢帝景和元年九月甲午,雷震。
 明帝泰始二年九月辛巳,雷震。泰始四年十月辛卯,雷震。泰始四年十一月癸卯朔,雷震。泰始五年十一月乙巳,雷震。泰始六年十一月庚午,雷。



 後廢帝元徽三年九月戊戌,雷。元徽三年九月丁未,雷。元徽三年九月戊午,雷震。
 元徽三年十月辛未,雷;甲戌,又雷。



 從帝升明三年二月二十四日丙申,震建陽門。



 鼓妖:晉惠帝元康九年三月,有聲若牛,出許昌城。十二月,廢太子,幽于許宮。按《春秋》晉文公柩有聲如牛,劉向以為鼓妖。其說曰:「聲如此,怒象也。將有急怒之謀,以生兵甲之禍。」此其類也。明年,賈后遣黃門孫慮殺太子,擊以藥杵,聲聞于外。



 蘇峻在歷陽,外營將軍鼓自鳴,如人弄鼓者。峻手自斫之,曰:「我鄉土時有此,則城空矣。」俄而作亂夷滅。此聽不聰之罰,鼓妖先作也。石虎末,洛陽城西北九里石牛在青石趺上,忽鳴喚,聲聞四十里。虎遣人打落兩耳及尾,鐵釘釘四腳。



 晉孝武太元十五年三月己酉朔,東北有聲如雷。案劉向說以為:「雷當託於雲,猶君託於臣。」無雲而雷,此君不恤下,下民將叛之象也。及帝崩而天下漸亂,孫恩、桓玄
 交陵京邑。



 吳興長城縣夏架山有石鼓,長丈餘,面徑三尺所,下有盤石為足,鳴則聲如金鼓,三吳有兵。晉安帝隆安中大鳴,後有孫靈秀之亂。



 魚孽:魏齊王嘉平四年五月,有二魚集于武庫屋上。此魚孽也。王肅曰:「魚生于淵,而亢於屋,介鱗之物,失其所也。邊將其殆有棄甲之變乎。」後果有東關之敗。干寶又以為
 高貴鄉公兵禍之應。二說皆與班固旨同。



 晉武帝太康中,有鯉魚二見武庫屋上。干寶曰:「武庫兵府,魚有鱗甲,亦兵類也。魚既極陰,屋上太陽,魚見屋上,象至陰以兵革之禍干太陽也。」至惠帝初,誅楊駿,廢太后,矢交館閣。元康末,賈后謗殺太子,尋亦誅廢。十年間,母后之難再興,是其應也。自是禍亂構矣。京房《易妖》曰:「魚去水,飛入道路,兵且作。」



 蝗蟲:
 魏文帝黃初三年七月,冀州大蝗,民饑。案蔡邕說:「蝗者,在上貪苛之所致也。」是時孫權歸從,帝因其有西陵之役,舉大眾襲之,權遂背叛。



 晉武帝泰始十年六月,蝗。是時荀、賈任政,疾害公直。



 晉孝懷帝永嘉四年五月,大蝗,自幽、并、司、冀至于秦、雍,草木牛馬毛鬣皆盡。是時天下兵亂,漁獵生民,存亡所繫,唯司馬越、茍晞而已,而競為暴刻,經略無章。



 晉愍帝建興四年六月,大蝗。去歲胡寇頻攻北地、馮翊,
 暐允等悉眾禦之。是時又禦劉曜,為曜所破,西京遂潰。



 晉元帝太興元年六月,蘭陵合鄉蝗,害禾稼。乙未,東莞蝗蟲縱廣三百里,害苗稼。太興元年七月,東海、彭城、下邳、臨淮四郡蝗蟲害禾、豆。太興元年八月,冀、青、徐三州蝗食生草盡,至于二年。是時中州淪喪,暴亂滋甚。太興二年五月,淮陵、臨淮、淮南、安豐、廬江諸郡蝗食秋麥。
 太興三年五月癸丑,徐州及揚州江西諸郡蝗,吳民多餓死。去年,王敦并領荊州,苛暴之釁,自此興矣。又是年初,徐州刺史蔡豹帥眾伐周撫。



 晉孝武帝太元十五年八月,兗州蝗。是時丁零寇兗、豫,鮮卑逼河南,征戍不已。太元十六年五月,飛蝗從南來,集堂邑縣界,害苗稼。是年春,發取江州兵營甲士二千人家口六七千人,配護軍及東宮,後尋散亡殆盡;又邊將連有徵殺。



 豕禍:吳孫皓寶鼎元年,野豕入右司馬丁奉營。此豕禍也。後奉見遣攻穀陽,無功反,皓怒,斬其導軍。及舉大眾北出,奉及萬彧等相謂曰:「若至華里,不得不各自還也。」此謀泄,奉時雖已死,皓追討穀陽事,殺其子溫,家屬皆遠徙。豕禍之應也。



 龔遂曰:「山野之獸,來入宮室,宮室將空。」又其象也。



 晉孝懷帝永嘉中,壽春城內有豕生兩頭而不活。周馥
 取而觀之。時通數者竊謂曰:「夫豕,北方之畜,胡、狄象也。兩頭者,無上也。生而死,不遂也。天意若曰,勿生專利之謀,將自致傾覆也。」周馥不悟,遂欲迎天子,令諸侯,俄為元帝所敗。是其應也。石勒亦尋渡淮,百姓死者十八九。



 晉愍帝建武元年,有豕生八足。聽不聰之罰也。京房《易傳》曰:「凡妖作,各象其類。足多者,所任邪也。」是後有劉隗之變。



 晉成帝咸和六年六月,錢塘民家豭豕生兩子,皆人面,
 如胡人狀,其身猶豕。



 京房《易妖》曰:「豕生人頭豕身者,邑且亂亡。」此豭豕而產,異之甚者也。



 晉孝武帝太元十年四月,京都有豕,一頭二身八足。十三年,京都民家豕產子,一頭二身八足。並與建武同妖也。是後宰相沈酗,不恤朝政,近習用事,漸亂國綱,至於大壞也。



 黑眚黑祥:晉孝懷帝永嘉五年十二月,黑氣四塞。近黑祥也。



 宋文帝元嘉二十六年三月,幸京口。有黑氣暴起,占有兵。明年,虜南寇至瓜步,飲馬于江。



 火沴水:晉武帝太康五年六月,任城、魯國池水皆赤如血。案劉向說,近火沴水也。聽之不聰之罰也。京房《易傳》曰:「淫於色,賢人潛,國家危,厥異水流赤。」



 晉穆帝升平三年二月,涼州城東池中有火;四年四月,姑臧澤水中又有火。此火沴水之妖也。明年,張天錫殺
 中護軍張邕。邕,執政臣也。



 晉安帝元興二年十月,錢塘臨平湖水赤。桓玄諷吳郡使言開除,以為己瑞。俄而玄敗。



\end{pinyinscope}