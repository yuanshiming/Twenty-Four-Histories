\article{卷七十一列傳第三十一 徐湛之 江湛 王僧綽}

\begin{pinyinscope}

 徐湛之,
 字孝源,東海郯
 人。司徒羨之兄孫,吳郡太守佩之弟子也。祖欽之,秘書監。父逵之,尚高祖長女會稽公主,為振威將軍、彭城、沛二郡太守。高祖諸子並幼,以逵
 之姻戚,將大任之,欲先令立功。及討司馬休之,使統軍為前鋒,配以精兵利器,事克,當即授荊州。休之遣魯宗之子軌擊破之,於陣見害。追贈中書侍郎。



 湛之幼孤,為高祖所愛,常與江夏王義恭寢食不離於側。永初三年,詔曰:「永興公主一門嫡長,早罹辛苦。外孫湛之,特所鐘愛。且致節之胤,情實兼常。



 可封枝江縣侯,食邑五百戶。」年數歲,與弟淳之共車行,牛奔車壞,左右馳來赴之。湛之先令取弟,眾咸嘆其幼而有識。及長,頗涉大義,善自
 位待。事祖母及母,並以孝謹聞。



 元嘉二年,除著作佐郎,員外散騎侍郎,並不就。六年,東宮始建,起家補太子洗馬,轉國子博士,遷奮威將軍、南彭城、沛二郡太守,徙黃門侍郎。祖母年老,辭以朝直,不拜。復授二郡,加輔國將軍,遷秘書監,領右軍將軍,轉侍中,加驍騎將軍。復為秘書監,加散騎常侍,驍騎如故。



 會稽公主身居長嫡,為太祖所禮,家事大小,必咨而後行。西征謝晦,使公主留止臺內,總攝六宮。忽有不得意,輒號哭,上甚憚之。初,高祖
 微時,貧陋過甚,嘗自往新洲伐荻,有納布衫襖等衣,皆敬皇后手自作;高祖既貴,以此衣付公主,曰:「後世若有驕奢不節者,可以此衣示之。」湛之為大將軍彭城王義康所愛,與劉湛等頗相附協。及劉湛得罪,事連湛之,太祖大怒,將致大辟。湛之憂懼無計,以告公主。公主即日入宮,既見太祖,因號哭下床,不復施臣妾之禮。以錦囊盛高祖納衣,擲地以示上曰:「汝家本貧賤,此是我母為汝父作此納衣。今日有一頓飽食,便欲殘害我兒子!」上亦
 號哭,湛之由此得全也。遷中護軍,未拜,又遷太子詹事,尋加侍中。



 湛之善於尺牘,音辭流暢。貴戚豪家,產業甚厚。室宇園池,貴遊莫及。伎樂之妙,冠絕一時。門生千餘人,皆三吳富人之子,姿質端妍,衣服鮮麗。每出入行遊,途巷盈滿,泥雨日,悉以後車載之。太祖嫌其侈縱,每以為言。時安成公何勖,無忌之子也,臨汝公孟靈休,昶之子也,並各奢豪,與湛之共以肴膳、器服、車馬相尚。京邑為之語曰:「安成食,臨汝飾。」湛之二事之美,兼於何、孟。勖
 官至侍中,追謚荒公。靈休善彈棋,官至祕書監。



 湛之遷冠軍將軍、丹陽尹,進號征虜將軍,加散騎常侍,以公主憂不拜。過葬,復授前職,湛之表啟固辭,又詣廷尉受罪;上詔獄官勿得受,然後就命。固辭常侍,許之。二十二年,范曄等謀逆,湛之始與之同,後發其事,所陳多不盡,為曄等款辭所連,乃詣廷尉歸罪,上慰遣令還郡。湛之上表曰:賊臣范曄、孔熙先等,連結謀逆,法靜尼宣分往還,與大將軍臣義康共相脣齒,備於鞠對。伏尋仲承祖始
 達熙先等意,便極言姦狀。而臣兒女近情,不識大體,上聞之初,不務指斥,紙翰所載,尤復漫略者,實以凶計既表,逆事歸露;又仰緣聖慈,不欲窮盡,故言勢依違,未敢縷陳。情旨無隱,已昭天鑒。及群凶收禽,各有所列,曄等口辭,多見誣謗;承祖醜言,紛紜特甚。乃云臣與義康宿有密契,在省之言,期以為定,潛通姦意,報示天文。末云熙先縣指必同,以誑於曄,或以智勇見稱,或以愚懦為目。既美其信懷可履,復駭其動止必啟。凡諸詭妄,還自
 違伐,多舉事端,不究源統,齎傳之信,無有主名,所征之人,又已死沒,首尾乖互,自為矛楯。即臣誘引之辭,以為始謀之證,銜臣糾告,並見怨咎,縱肆狂言,必規禍陷。伏自探省,亦復有由。



 昔義康南出之始,敕臣入相伴慰,晨夕覲對,經踰旬日。逆圖成謀,雖無顯然,懟容異意,頗形言旨。遺臣利刃,期以際會,臣苦相諫譬,深加拒塞。以為怨憤所至,不足為慮,便以關啟,懼成虛妄,思量反覆,實經愚心,非為納受,曲相蔽匿。



 又令申情范曄,釋中間之
 憾,致懷蕭思話,恨婚意未申,謂此僥幸,亦不宣達。



 陛下敦惜天倫,彰於四海,籓禁優簡,親理咸通;又昔蒙眷顧,不容自絕,音翰信命,時相往來。或言少意多,旨深文淺,辭色之間,往往難測。臣每懼異聞,皆略而不答。惟心無邪悖,故不稍以自嫌。心婁心婁丹實,具如此啟。至於法靜所傳,及熙先等謀,知實不早,見關之日,便即以聞。雖晨光幽燭,曲昭窮款,裁以正義,無所逃刑。束骸北闕,請罪司寇,乾施含宥,未加治考,中旨頻降,制使還往,仰荷恩私,
 哀惶失守。



 臣殃積罪深,丁罹酷罰,久應屏棄,永謝人理。況奸謀所染,忠孝頓闕,智防愚淺,暗於禍萌,士類未明其心,群庶謂之同惡,朝野側目,眾議沸騰,專信仇隙之辭,不復稍相申體。臣雖駑下,情非木石。豈不知醜點難嬰,伏劍為易。而靦然視息,忍此餘生,實非茍吝微命,假延漏刻。誠以負戾灰滅,貽惡方來,貪及視息,少自披訴;冀幽誠丹款,儻或昭然,雖復身膏草土,九泉無恨。顯居官次,垢穢朝班,厚顏何地,可以自處。乞蒙隳放,伏待鈇
 金質。



 上優詔不許。二十四年,服闋,轉中書令,領太子詹事。出為前軍將軍、南兗州刺史。善於為政,威惠並行。廣陵城舊有高樓,湛之更加修整,南望鐘山。城北有陂澤,水物豐盛。湛之更起風亭、月觀,吹臺、琴室,果竹繁茂,花藥成行,招集文士,盡游玩之適,一時之盛也。時有沙門釋惠休,善屬文,辭采綺艷,湛之與之甚厚。世祖命使還俗。本姓湯,位至揚州從事史。二十六年,復入為丹陽尹,領太子詹事,將軍如故。二十七年,索虜至瓜步,湛之領兵
 置佐,與皇太子分守石頭。



 二十八年春,魯爽兄弟率部曲歸順,爽等,魯軌子也。湛之以為廟算遠圖,特所獎納,不敢茍申私怨。乞屏居田里,不許。



 轉尚書僕射,領護軍將軍。時尚書令何尚之以湛之國戚,任遇隆重,欲以朝政推之。凡諸辭訴,一不料省。湛之亦以《職官記》及令文,尚書令敷奏出內,事無不總,令缺則僕射總任。又以事歸尚之,互相推委。御史中丞袁淑並奏免官,詔曰:「令僕治務所寄,不共求體當,而互相推委,糾之是也。然故事
 殘舛,所以致茲疑執,特無所問,時詳正之。」乃使湛之與尚之並受辭訴。尚之雖為令,而朝事悉歸湛之。



 初,劉湛伏誅,殷景仁卒,太祖委任沈演之、庾炳之、范曄等,後又有江湛、何瑀之。曄誅,炳之免,演之、瑀之並卒,至是江湛為吏部尚書,與湛之並居權要,世謂之江、徐焉。



 上每有疾,湛之輒入侍醫藥。二凶巫蠱事發,上欲廢劭,賜浚死。而世祖不見寵,故累出外蕃,不得停京輦。南平王鑠、建平王宏並為上所愛,而鑠妃即湛妹,勸上立之。元嘉末,
 徵鑠自壽陽入朝,既至,又失旨,欲立宏,嫌其非次,是以議久不決。與湛之屏人共言論,或連日累夕。每夜常使湛之自秉燭,繞壁檢行,慮有竊聽者。劭入弒之旦,其夕,上與湛之屏人語,至曉猶未滅燭。湛之驚起趣北戶,未及開,見害。時年四十四。世祖即位,追贈司空,加散騎常侍,本官如故,謚曰忠烈公。又詔曰:「徐湛之、江湛、王僧綽門戶荼酷,遺孤流寓,言念既往,感痛兼深。可令歸居本宅,厚加恤賜。」於是三家長給廩。



 三子:聿之、謙之,為元凶
 所殺。恒之嗣侯,尚太祖第十五女南陽公主,蚤卒,無子。聿之子孝嗣紹封,齊受禪,國除。



 江湛,字徽淵,濟陽考城人,湘州刺史夷子也。居喪以孝聞。愛好文義,喜彈棋鼓琴,兼明算術。初為著作佐郎,遷彭城王義康司徒行參軍,南譙王義宣左軍功曹。復為義康司徒主簿,太子中舍人。司空檀道濟為子求湛妹婚,不許。義康有命,又不從。時人重其立志。義康欲引與日夕,湛固求外出,乃以為武陵內史,還為司徒從事中
 郎,遷太子中庶子,尚書吏部郎。隨王誕為北中郎將、南徐州刺史,以湛為長史、南東海太守,政事委之。



 元嘉二十五年,徵為侍中,任以機密,領本州大中正,遷左衛將軍。時改選學職,以太尉江夏王義恭領國子祭酒,湛及侍中何攸之領博士。二十七年,轉吏部尚書。家甚貧約,不營財利,餉饋盈門,一無所受,無兼衣餘食。嘗為上所召,值浣衣,稱疾經日,衣成然後赴。牛餓,馭人求草,湛良久曰:「可與飲。」在選職,頗有刻核之譏,而公平無私,不受
 請謁,論者以此稱焉。



 上大舉北代,舉朝為不可,唯湛贊成之。索虜至瓜步,領軍將軍劉遵考率軍出江上,以湛兼領軍,軍事處分,一以委焉。虜遣使求婚,上召太子劭以下集議,眾並謂宜許,湛曰:「戎狄無信,許之無益。」劭怒,謂湛曰:「今三王在厄,詎宜茍執異議。」聲色甚厲。坐散俱出,劭使班劍及左右推之,殆將側倒。劭又謂上曰:「北伐敗辱,數州淪破,獨有斬江湛,可以謝天下。」上曰:「北伐自我意,江湛但不異耳。」劭後燕集,未嘗命湛。常謂上曰:「
 江湛佞人,不宜親也。」上乃為劭長子偉之娉湛第三女,欲以和之。



 上將廢劭,使湛具詔草。劭之入弒也,湛直上省,聞叫噪之聲,乃匿傍小屋中。



 劭遣收之,舍吏紿云:「不在此。」兵士即殺舍吏,乃得湛。湛據窗受害,意色不撓。時年四十六。湛五子恁、恕、憼、愻、法壽,皆見殺。初,湛家數見怪異,未敗少日,所眠床忽有數升血。世祖即位,追贈左光祿大夫、開府儀同三司,加散騎常侍,本官如故,謚曰忠簡公。長子恁,尚太祖第九女淮陽長公主,為著作佐
 郎。



 王僧綽,琅邪臨沂人,左光祿大夫曇首子也。幼有大成之度,弱年眾以國器許之。好學有理思,練悉朝典。年十三,太祖引見,下拜便流涕哽咽,上亦悲不自勝。



 襲封豫章縣侯,尚太祖長女東陽獻公主。初為江夏王義恭司徒參軍,轉始興王文學,秘書丞,司徒左長史,太子中庶子。元嘉二十六年,徙尚書吏部郎,參掌大選。究識流品,諳悉人物,拔才舉能,咸得其分。二十八年,遷侍中,任以
 機密。僧綽沈深有局度,不以才能高人。先是,父曇首與王華並為太祖所任,華子嗣人才既劣,位遇亦輕。僧綽嘗謂中書侍郎蔡興宗曰:「弟名位應與新建齊,超至今日,蓋由姻戚所致也。」新建者,嗣之封也。及為侍中,時年二十九。始興王濬嘗問其年,僧綽自嫌蚤達,逡巡良久乃答,其謙虛自退若此。



 元嘉末,太祖頗以後事為念,以其年少,方欲大相付託,朝政小大,皆與參焉。



 從兄徽,清介士也,懼其太盛,勸令損抑。僧綽乃求吳郡及廣州,上
 並不許。會二凶巫蠱事泄,上獨先召僧綽具言之。及將廢立,使尋求前朝舊典。劭於東宮夜饗將士,僧綽密以啟聞,上又令撰漢魏以來廢諸王故事。撰畢,送與江湛、徐湛之。湛之欲立隨王誕,江湛欲立南平王鑠,太祖欲立建平王宏,議久不決。延妃即湛之女,鑠妃即湛妹。太祖謂僧綽曰:「諸人各為身計,便無與國家同憂者。」僧綽曰:「建立之事,仰由聖懷。臣謂唯宜速斷,不可稽緩。當斷不斷,反受其亂。願以義割恩,略小不忍。不爾,便應坦懷
 如初,無煩疑論。淮南云:『以石投水,吳越之善沒取之。』事機雖密,易致宣廣,不可使難生慮表,取笑千載。」上曰:「卿可謂能斷大事。此事重,不可不殷勤三思。且庶人始亡,人將謂我無復慈愛之道。」



 僧綽曰:「臣恐千載之後,言陛下唯能裁弟,不能裁兒。」上默然。江湛同侍坐,出閣,謂僧綽曰:「卿向言,將不大傷切直。」僧綽曰:「弟亦恨君不直。」



 及劭弒逆,江湛在尚書上省,聞變,歎曰:「不用僧綽言,以至於此。」劭既立,轉為吏部尚書,委以事任,事在《二凶傳》。頃
 之,劭料檢太祖巾箱及江湛家書疏,得僧綽所啟饗士并廢諸王事,乃收害焉,時年三十一。因此陷北第諸王侯,以為與僧綽有異志,并殺僧綽門客太學博士賈匪之、奉朝請司馬文穎、建平國常侍司馬仲秀等。世祖即位,追贈散騎常侍、金紫光祿大夫,謚曰愍侯。



 初,太社西空地一區,吳時丁奉宅,孫晧流徙其家。江左初為周顗、蘇峻宅,其後為袁悅宅,又為章武王司馬秀宅,皆以凶終。後給臧燾,亦頗遇喪禍,故世稱為凶地。僧綽常以正達
 自居,謂宅無吉凶,請以為第。始就造築,未及居而敗。



 子儉嗣,昇明末,為齊國尚書右僕射。



 史臣曰:甚矣,宋氏之家難也,仇釁所鐘,親地兼極,雖復傾天滅道,跡非嫌路,而災隙內兆,邪蠱外興,天性既離,愛敬同盡,探雀請熊,非無前釁,猜防之道,有未足乎。世祖弱年輕躁,夙無朝寵,累任邊外,未嘗居中。當璧之重,將由愛立,臣主回疑,事無蚤斷。若使守器以長,命不待賢,則密禍自銷,危機可免。



 聖哲之訓,豈欺我哉!昔山濤
 舉羊祜為太子太傅,蓋欲以後事委之,而羊公短世。



 僧綽綢繆主心,將任以國重,而宮車晏駕。二臣並以道德謙沖,名高兩代。胙未中年,功謝成日,惜矣哉!



\end{pinyinscope}