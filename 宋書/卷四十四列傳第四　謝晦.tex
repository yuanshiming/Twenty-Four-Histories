\article{卷四十四列傳第四 謝晦}

\begin{pinyinscope}

 謝晦,字宣
 明,陳郡陽夏人也。祖朗,東陽太守。父重,會稽王道子驃騎長史。



 兄絢,高祖鎮軍長史,蚤卒。晦初為孟昶建威府中兵參軍。昶死,高祖問劉穆之:「孟昶參佐,誰
 堪入我府?」穆之舉晦,即命為太尉參軍。高祖嘗訊囚,其旦刑獄參軍有疾,札晦代之,於車中一鑒訊牒,催促便下。相府多事,獄繁殷積,晦隨問酬辯,曾無違謬。高祖奇之,即日署刑獄賊曹,轉豫州治中從事。義熙八年,土斷僑流郡縣,使晦分判揚、豫民戶,以平允見稱。入為太尉主簿,從征司馬休之。時徐逵之戰敗見殺,高祖怒,將自被甲登岸,諸將諫,不從,怒愈甚。晦前抱持高祖,高祖曰:「我斬卿!」晦曰:「天下可無晦,不可無公,晦死何有!」會胡籓已
 得登岸,賊退走,乃止。



 晦美風姿,善言笑,眉目分明,鬢發如點漆。涉獵文義,朗贍多通,高祖深加愛賞,群僚莫及。從征關、洛,內外要任悉委之。劉穆之遣使陳事,晦往往措異同,穆之怒曰:「公復有還時否?」高祖欲以為從事中郎,以訪穆之,堅執不與。終穆之世,不遷。穆之喪問至,高祖哭之甚慟。晦時正直,喜甚,自入閣內參審穆之死問。其日教出,轉晦從事中郎。



 宋臺初建,為右衛將軍,尋加侍中。高祖受命,於石頭登壇,備法駕入宮。晦領游軍
 為警備,遷中領軍,侍中如故。以佐命功,封武昌縣公,食邑二千戶。二年,坐行璽封鎮西司馬、南郡太守王華大封,而誤封北海太守球,版免晦侍中。尋轉領軍將軍、散騎常侍,依晉中軍羊祜故事,入直殿省,總統宿衛。三月,高祖不豫,給班劍二十人,與徐羨之、傅亮、檀道濟並侍醫藥。少帝即位,加領中書令,與羨之,亮共輔朝政。少帝既廢,司空徐羨之錄詔命,以晦行都督荊湘雍益寧南北秦七州諸軍事、撫軍將軍、領護南蠻校尉、荊州刺史,
 欲令居外為援,慮太祖至或別用人,故遽有此授。精兵舊將,悉以配之,器仗軍資甚盛。太祖即位,加使持節,依本位除授。晦慮不得去,甚憂惶,及發新亭,顧望石頭城,喜曰:「今得脫矣。」



 尋進號衛將軍,加散騎常侍,進封建平郡公,食邑四千戶,固讓進封。又給鼓吹一部。



 初為荊州,甚有自矜之色,將之鎮,詣從叔光祿大夫澹別。澹問晦年,晦答曰:「三十五。」澹笑曰:「昔荀中郎年二十七為北府都督,卿比之,已為老矣。」晦有愧色。至江陵,深結侍中王
 華,冀以免禍。二女當配彭城王義康、新野侯義賓。



 元嘉二年,遣妻曹及長子世休送女還京邑。先是景平中,索虜為寇,覆沒河南。至是上欲誅羨之等,并討晦。聲言北伐,又言拜京陵,治裝舟艦。傅亮與晦書曰:「薄伐河朔,事猶未已,朝野之慮,憂懼者多。」又言:「朝士多諫北征,上當遣外監萬幼宗往相咨訪。」時朝廷處分異常,其謀頗泄。三年正月,晦弟黃門侍郎㬭馳使告晦,晦猶謂不然,呼咨議參軍何承天,示以亮書,曰:「計幼宗一二日必至,傅
 公慮我好事,故先遣此書。」承天曰:「外間所聞,咸謂西討已定,幼宗豈有上理。」晦尚謂虛妄,使承天豫立答詔啟草,伐虜宜須明年。江夏內史程道惠得尋陽人書,言:「朝廷將有大處分,其事已審。」使其輔國府中兵參軍樂冏封以示晦。晦又謂承天曰:「幼宗尚未至,若復二三日無消息,便是不復來邪?」承天答曰:「詔使本無來理,如程所說,其事已判,豈容復疑。」



 晦欲焚南蠻兵籍,率見力決戰。士人多勸發兵,乃立幡戒嚴,謂司馬庾登之曰:「今當自
 下,欲屈卿以三千人守城,備禦劉粹。」登之曰:「下官親老在都,又素無旅,情計二三,不敢受此旨。」晦仍問諸佐:「戰士三千,足守城不?」南蠻司馬周超對曰:「非徒守城而已,若有外寇,可以立勳。」登之乃曰:「超必能辦,下官請解司馬、南郡以授。」即於坐命超為司馬、建威將軍、南義陽太守,轉登之為長史,南郡如故。



 太祖誅羨之等及晦子新除祕書郎世休,收㬭、㬭子世平、兄子著作佐郎紹等。樂冏又遣使告晦:「徐、傅二公及㬭等並已誅。」晦先舉羨之、
 亮哀,次發子弟凶問。既而自出射堂,配衣軍旅。數從高祖征討,備睹經略,至是指麾處分,莫不曲盡其宜。二三日中,四遠投集,得精兵三萬人。乃奉表曰:臣階緣幸會,蒙武皇帝殊常之眷,外聞政事,內謀帷幄,經綸夷險,毗贊王業,預佐命之勳,膺河山之賞。及先帝不豫,導揚末命,臣與故司徒臣羨之、左光祿大夫臣亮、征北將軍臣道濟等,並升御床,跪受遺詔,載貽話言,託以後事。臣雖凡淺,感恩自厲,送往事居,誠貫幽顯。逮營陽失德,自絕
 宗廟,朝野岌岌,憂及禍難,忠謀協契,徇國忘己,援登聖朝,惟新皇祚。陛下馳傳乘流,曾不惟疑,臨朝殷勤,增崇封爵。此則臣等赤心已亮於天鑒,遠近萬邦咸達於聖旨。若臣等志欲專權,不顧國典,便當協翼幼主,孤背天日,豈復虛館七旬,仰望鸞旗者哉?故廬陵王於營陽之世,屢被猜嫌,積怨犯上,自貽非命。天祚明德,屬當昌運,不有所廢,將何以興?成人之美,《春秋》之高義;立帝清館,臣節之所司。耿弇不以賊遺君父,臣亦何負於宋室邪?
 況釁結鬩牆,禍成畏逼,天下耳目,豈伊可誣!



 臣忝居蕃任,乃誠匪懈,為政小大,必先啟聞。糾剔群蠻,清夷境內,分留弟侄,並侍殿省。陛下聿遵先志,申以婚姻,童稚之目,猥荷齒召,薦女遷子,合門相送。事君之道,義盡於斯。臣羨之總錄百揆,翼亮三世,年耆乞退,屢抗表疏,優旨綢繆,未垂順許。臣亮管司喉舌,恪虔夙夜,恭謹一心,守死善道。此皆皇宋之宗臣,社稷之鎮衛,而讒人傾覆,妄生國釁,天威震怒,加以極刑,并及臣門,則被孥戮。雖未
 知臣道濟問,推理即事,不容獨存。先帝顧託元臣翼命之佐,剿於佞邪之手,忠貞匪躬之輔,不免夷滅之誅。陛下春秋方富,始覽萬機,民之情偽,未能鑒悉。王弘兄弟,輕躁昧進;王華猜忌忍害,規弄威權,先除執政,以逞其欲。



 天下之人,知與不知,孰不為之痛心憤怨者哉!



 臣等見任先帝,垂二十載,小心謹慎,無纖介之愆,伏事甫爾,而嬰若斯之罪。



 若非先帝謬於知人,則為陛下未察愚款。臣去歲末使反,得朝士及殿省諸將書,並言嫌隙已
 成,必有今日之事。臣推誠仰期,罔有二心,不圖姦回潛遘,理順難恃,忠賢隕朝,愚臣見襲,到彥之、蕭欣等在近路。昔白公稱亂,諸梁嬰胄,惡人在朝,趙鞅入伐。臣義均休戚,任居分陜,豈可顛而不扶,以負先帝遺旨!輒率將士,繕治舟甲,須其自送,投袂撲討。若天祚大宋,卜世靈長,義師克振,中流清蕩,便當浮舟東下,戮此三豎,申理冤恥,謝罪闕庭,雖伏鑕赴鑊,無恨於心。伏願陛下遠尋永初託付之旨,近存元嘉奉戴之誠,則微臣丹款,猶有
 可察。臨表哽慨,言不自盡。



 太祖時已戒嚴,諸軍相次進路。尚書符荊州曰:禍福無門,逆順有數,天道微於影響,人事鑒於前圖,未有蹈義而福不延,從惡而禍不至也。故智計之士,審敗以立功,守正之臣,臨難以全節。徐羨之、傅亮、謝晦,安忍鴆殺,獲罪於天,名教所極,政刑所取,已遠暴四海,宣於聖詔。羨之父子、亮及晦息,電斷之初,並即大憲。復王室之仇,攄義夫之憤,國典澄明,人神感悅。三姓同罪,既擒其二,晦之室屬,縲仆獄戶,茍幽明所
 怨,孤根易拔,以順討逆,雖厚必崩。然歸死難圖,獸困則噬,是以爰整其旅,用為過防。京師之眾,天下雲集,士練兵精,大號響震。



 使持節、中領軍佷山縣開國侯到彥之率羽林選士果勁二萬,雲旍首路,組甲曜川。使持節、散騎常侍、都督南徐兗之江北淮南青州徐州之淮陽下邳琅邪東莞七郡諸軍事、征北將軍、南兗州刺史、永脩縣開國公檀道濟統勁銳武卒三萬,戈船蔽江,星言繼發,千帆俱舉,萬棹遄征。散騎常侍、驍騎將軍段宏鐵馬
 二千,風驅電擊,步自竟陵,直至鄢郢。又命征虜將軍、雍州刺史劉粹控河陰之師,衝其巢窟。湘州刺史張邵提湘川之眾,直據要害。巴、蜀杜荊門之險,秦、梁絕丹圻之逕,雲網四合,走伏路盡。然後鑾輿效駕,六軍鵬翔,警蹕前驅,五牛整旆。雖以英布之氣,彭寵之資,登陴無名,授兵誰御?加以西土之人,咸沐皇澤,東吳將士,懷本首丘,必不自陷罪人之黨,橫為亂亡之役。置軍則魚潰,嬰城則鳥散,其勢然矣。聖上殷勤哀愍,其罪由晦,士民何辜。
 是用一分前麾,宣示朝旨。符到,其即共收擒晦身,輕舟護送。若已猖蹶,先事阻衛,宜翻然背亂,相率歸朝。頃大刑所加,洪恩曠洽,傅亮三息,特蒙全宥,晦同產以下,羨之諸姪,咸無所染。況彼府州文武,並列王職,荷國榮任,身雖在外,乃心辰極。夫轉禍貴速,後機則凶,遂使王師臨郊,雷電皆至,噬臍之恨,亦將何及。



 時益州刺史蕭摹之、巴西太守劉道產被徵還,始至江陵,晦並繫縶,沒其財貨,以充軍資。竟陵內史殷道鸞未之郡,以為咨議參
 軍。以弟遁為冠軍、竟陵內史,總留任;兄子世猷為建威將軍、南平太守。劉粹若至,周超能破之者,即以為龍驤將軍、雍州刺史。晦率眾二萬,發自江陵,舟艦列自江津至於破塚,旍旂相照,蔽奪日光。晦乃嘆曰:「恨不得以此為勤王之師!」自領湘州刺史,以張邵為輔國將軍,邵不受命。晦檄京邑曰:王室多故,禍難薦臻。營陽失德,自絕宗廟。廬陵王構鬩有本,屢被猜嫌,且居喪失禮,遐邇所具,積怨犯上,自貽非道。群后釋位,爰登聖明,亂之未乂,職
 有所係。按車騎大將軍王弘、侍中王曇首,謬蒙時私,叨竊權要。弘於永初之始,實荷不世之恩,元嘉之讓,自謂任遇浮淺,進誣先皇委誠之寄,退長嫌隙異同之端。



 曇首往因使下,訪以今上起居,不能光揚令德,彰於朝聽,其言多誣,故不具說。



 王華賊亡之餘,賞擢之次,先帝常見訪逮,庶有一分可取,而華稟性凶猜,多所忍害。曩者縱人入城,託疾辭事,此都士庶,咸所聞知。以其所啟及上手答示宗叔獻,又令宣告徐、傅二公。及周糾使下,又
 令見咨,云:「欲自攬政事,求離任還都,并令曇首具述此意。」又惠觀道人說,外人告華及到彥之謀反,不謂無之。城內東將,數日之內,操戈相待。華說數為秋當所譖,常不自安。凡此諸事,豈有忠誠冥契若此者邪?自以父亡道側,情事異人,外絕酒醴,而宵飲是恣。靦貌囗囗囗囗囗囗凡厥士庶,誰不側目。又常歎宰相頓有數人,是何憤憤,規總威權,不顧國典。



 保祐皇家者,罹屠戮之誅;效勤社稷者,致殲夷之禍。搢紳之徒,孰不慷慨!遂矯違詔
 旨,遣到彥之、蕭欣之輕舟見襲。即日監利左尉露檄眾軍已至揚子。



 雖以不武,忝荷蕃任,國家艱難,悲憤兼集。若使小人得志,君子道消,凡百有殄瘁之哀,蒼生深橫流之懼。輒糾勒義徒,繕治舟甲,舳艫亙川,駟介蔽野,武夫鷙勇,人百其誠。今遣南蠻司馬寧遠將軍庾登之統參軍事建武將軍建平太守安泰、宣威將軍昭弘宗、參軍事宣威將軍王紹之等,精銳一萬,前鋒致討。南蠻參軍、振武將軍魏像統參軍事、宣威將軍陳珍虎旅二千,
 參軍事、建威將軍、新興太守賀愔甲卒三千,相係取道。南蠻參軍、振威將軍郭卓鐵騎二千,水步齊舉。大軍三萬,駱驛電邁。行冠軍將軍竟陵內史河東太守謝遁、建威將軍南平太守謝世猷驍勇一萬,留守江陵。分命參軍、長寧太守竇應期步騎五千,直出義陽。司馬、建威將軍、行南義陽太守周超之統軍司馬、振武將軍胡崇之精悍一萬,北出高陽,長兼行參軍、寧遠將軍朱澹之步騎五千,西出雁塞,同討劉粹,並趨襄陽。奇兵尚速,指景
 齊奮。



 諸賢並同國恩,情兼義烈,今誠志士忘身之日,義夫著績之秋,見機而動,望風而不待勖。



 晦至江口,到彥之已到彭城洲。庾登之據巴陵,畏懦不敢進。會霖雨連日,參軍劉和之曰:「彼此共有雨耳,檀征北尋至,東軍方彊,唯宜速戰。」登之心匡怯,使小將陳祐作大囊,貯茅數千斛,縣於帆檣,云可以焚艦,用火宜須晴,以緩戰期。



 晦然之,遂停軍十五日。乃攻蕭欣於彭城洲,中兵參軍孔延秀率三千人進戰,甚力。



 欣於陳後擁楯自衛,又委軍還
 船,於是大敗。延秀又攻洲口柵,陷之,彥之退保隱圻。



 晦又上表曰:臣聞凶邪敗國,先代成患;讒豎亂朝,異世齊禍。故趙高矯逼,秦氏用傾;董卓階亂,漢祚伊覆。雖哲王宰世,大明照臨,未能使其漸弗興,茲害不作。姦臣王弘等竊弄威權,興造禍亂,遂與弟華內外影響,同惡相成,忌害忠賢,圖希非望。



 故司徒臣羨之、左光祿大夫臣亮橫被酷害,并及臣門。雖未知征北將軍臣道濟存亡,不容獨免。遂遣蕭欣、到彥之等輕舟見襲,姦偽之甚,一至
 於斯。羨之及亮,或宿德元臣,姻婭皇極,或任總文武,位班三事,道濟職惟上將,扞城是司,皆受遇先朝,棟梁一代。臣昔因時幸,過蒙先眷,內聞政事,外經戎旅,與羨之、亮等同被齒盼。既經啟王基,協濟大業,爰自權輿,暨于揖讓,誠策雖微,仍見紀錄,並蒙丹書之誓,各受山河之賞,欲使與宋升降,傳之無窮。及聖體不預,穆卜無吉,召臣等四人,同升御床,顧命領遺,委以家國。仰奉成旨,俯竭股肱,忠貞不效,期之以死。但營陽悖德,自絕於天,社
 稷之危,憂在託付,不有所廢,將焉以興。乃遠稽殷、漢,用升聖德。



 陛下順流乘傳,不聽張武之疑,入邸龍飛,非俟宋昌之議,斯乃主臣相信,天人合契,九五當陽,化形四海。羨之及亮,內贊皇猷,臣與道濟,分翰于外,普天之下,孰曰不宜。遂蒙寵授,來鎮此方,分留弟侄,以侍臺省。到任以來,首尾三載,雖形在遠外,心係本朝,事無大小,動皆咨啟,八州之政,罔一專輒,尊上之心,足貫幽顯。陛下遠述先旨,申以婚姻,大息世休,復蒙引召,是以去年送
 女遣兒,闔家俱下,血誠如此,未知所愧。而凶狡無端,妄生釁禍,羨之內誅,臣受外伐,顧省諸懷,不識何罪?天聽遐邈,陳訴靡由。弘等既蒙寵任,得侍左右,自謂勢擅狐鼠,理隔熏掘。又以陛下富於春秋,始覽政事,欲馮陵恩幸,窺望國權,親從磐歭,規自封殖,不除臣等,罔得專權,所以交結讒慝,成是亂階。又惟弘等所構,當以營陽為言,廬陵為罪。又以臣等位高功同,內外膠固。陛下信其厚貌,忘厥左道,三至下機,能不暫惑。



 伏自尋省,廢昏立
 明,事非為己。廬陵之事,不由傍人,內積蕭牆之釁,外行叔段之罰,既制之有主,臣何預焉。然廬陵為性輕險,悌順不足,武皇臨崩,亦有口詔,比雖發自營陽,實非國禍。至於羨之、亮等,周旋同體,心腹內外,政欲戮力皇家,盡忠報主。若令臣等頗欲執權,不專為國,初廢營陽,陛下在遠,武皇之子,尚有童幼,擁以號令,誰敢非之。而溯流三千,虛館三月,奉迎鑾駕,以遵下武,血心若斯,易為可鑒。



 且臣等奉事先朝,十有七年,並居顯要,世稱恭謹,不
 圖一旦致茲釁罰。夫周公大賢,尚有流言之謗,伯奇至孝,不免譖訴之禍。慈父非無情於仁子,明君豈有志於貞臣。姦遘所移,勢回山岳,況乃精誠微淺,而望求信者哉!《詩》不云乎:「讒人罔極,交亂四國。愷悌君子,無信讒言。」陛下躬覽篇籍,研覈是非,釁兆之萌,宜應深察。臣竊懼王室小有皇甫之患,大有閻樂之禍,夙夜殷憂,若無首領。



 夫周道浸微,桓、文稱伐,君側亂國,趙鞅入誅。況今凶禍滔天,辰極危逼,台輔孥戮,岳牧傾陷。臣才非絳侯,安
 漢是職,人愧博陸,廁奉遺旨。國難既深,家痛亦切。輒簡徒繕甲,軍次巴陵,蕭欣窘懾,望風奔迸。臣誠短劣,在國忘身,仰憑社稷之靈,俯厲義勇之氣,將長驅電掃,直入石頭,梟翦元凶,誅夷首惡,弔二公之冤魂,寫私門之禍痛。然後分歸司寇,甘赴鼎鑊,雖死之日,猶生之年。



 伏惟陛下德合乾元,道侔玄極,鑒凶禍之無端,察貞亮之有本,回日月之照,發霜電之威,梟四凶於廟庭,懸三監於絳闕,申二台之匪辜,明兩蕃之無罪,上謝祖宗,下告百
 姓,遣一乘之使,賜咫尺之書,臣便勒眾旋旗,還保所任。須次近路,尋復表聞。



 初,晦與徐羨之、傅亮謀為自全之計,晦據上流,而檀道濟鎮廣陵,各有彊兵,以制持朝廷;羨之、亮於中秉權,可得持久。及太祖將行誅,王華之徒咸云:「道濟不可信。」太祖曰:「道濟止於脅從,本非事主。殺害之事,又所不關。吾召而問之,必異。」於是詔道濟入朝,授之以眾,委之西討。晦聞羨之等死,謂道濟必不獨全,及聞率眾來上,惶懼無計。



 道濟既至,與彥之軍合,牽艦
 緣岸。晦始見艦數不多,輕之,不即出戰。至晚,因風帆上,前後連咽,西人離阻,無復鬥心。臺軍至忌置洲尾,列艦過江,晦大軍一時潰散。晦夜出,投巴陵,得小船還江陵。初,雍州刺史劉粹遣弟竟陵太守道濟與臺軍主沈敞之襲江陵,至沙橋,周超率萬餘人與戰,大破之。俄而晦敗問至。晦至江陵,無它處分,唯愧謝周超而已。超其夜舍軍單舸詣到彥之降。眾散略盡,乃攜其弟遁、兄子世基等七騎北走。遁肥壯不能騎馬,晦每待之,行不得速。
 至安陸延頭,為戍主光順之所執。順之,晦故吏也。檻送京師,於路作《悲人道》,其詞曰:悲人道兮,悲人道之實難。哀人道之多險,傷人道之寡安。懿華宗之冠胄,固清流而遠源。樹文德於庭戶,立操學於衡門。應積善之餘祐,當履福之所延。何小子之凶放,實招禍而作愆。值革變之大運,遭一顧於聖皇。參謀猷於創物,贊帝制於宏綱。出治戎於禁衛,入關言於帷房。分河山之珪組,繼文武之龜章。稟顧命於西殿,受遺寄於御床。伊懦劣其無節,
 實懷此而不忘。荷隆遇於先主,欲報之於後王。憂託付之無效,懼愧言於存亡。謂繼體其嗣業,能增輝於前光。居遏密之未幾,越禮度而湎荒。普天壤而殞氣,必社稷之淪喪。矧吾儕之體國,實啟處而匪遑。藉億兆之一志,固昏極而明彰。諒主尊而民晏,信卜祚之無疆。國既危而重構,家已衰而載昌。獲扶顧而休否,冀世道之方康。



 朝褒功以疏爵,祗命服於西蕃。奏簫管之嘈𡂐,擁朱旄之赫煌。臨八方以作鎮,響文武之桓桓。厲薄弱以為政,
 實忘食於日旰。豈申甫之敢慕,庶惟宋之屏翰。



 甫逾歷其三稔,實周回其未再。豈有慮於內囗囗囗囗其云裁。痛夾輔之二宰,並加辟而靡貸。哀弱息之從禍,悲發中而心痗。



 伊荊漢之良彥,逮文武之子民。見忠貞而弗亮,睹理屈而莫申。皆義概而同憤,咸荷戈而競臻。浮舳艫之弈弈,陳車騎之轔轔。觀人和與師整,謂茲兵其誰陳。庶亡魂之雪怨,反涇、渭於彞倫。齊輕舟於江曲,殄銳敵其皆湮。勒陸徒於白水,寇無反於隻輪。氣有捷而益
 壯,威既肅而彌振。嗟時哉之不與,迕風雨以踰旬。我謀戰而不克,彼繼奔其躡塵。乏智勇之奇正,忽孟明而是遵。茍成敗其有數,豈怨天而尤人。恨矢石之未竭,遂摧師而覆陳。誠得喪之所遭,固當之其無吝。痛同懷之弱子,橫遭罹之殃釁。智未窮而事傾,力未極而莫振。誓同盡於鋒鏑,我怯力而愆信。愍弟侄之何辜,實吾咎之所嬰。謂九夷其可處,思致免以全生。嗟性命之難遂,乃窘糸世於邊亭。亦何忤於天地,備艱危而是丁。



 我聞之於昔
 誥,功彌高而身蹙。霍芒刺而幸免,卒傾宗而滅族。周嘆貴於獄吏,終下蕃而靡鞠。雖明德之大賢,亦不免於殘戮,懷今憚而忍人,忘向惠而莫復。績無賞而震主,將何方以自牧。非砏石之圓照,孰違禍以取福,著殷鑑於自古,豈獨歎於季叔。能安親而揚名,諒見稱於先哲。保歸全而終孝,傷在餘而皆缺。辱歷世之平素,忽盛滿而傾滅。惟烝嘗與灑掃,痛一朝而永絕。問其誰而為之,實孤人之險戾。罪有踰於丘山,雖萬死其何雪。



 羈角偃兮衡
 閭,親朋交兮平義。雖履尚兮不一,隆分好兮情寄。俱憚耕兮從祿,睹世道兮艱詖。規志局兮功名,每謂之兮為易。今定謚兮闔棺,慚明智兮昔議。雖待盡兮為恥,嗟厚顏兮靡置。長揖兮數子,謝爾兮明智。百齡兮浮促,終焉兮斟克。



 臥盡兮斧斤,理命兮同得。世安彼兮非此,豈曉分兮辨惑。御莊生之達言,請承風以為則。



 周超既降,到彥之以參府事,劉粹遣參軍沈敞之告彥之沙橋之敗,事由周超,彥之乃執之。先系爵等,猶未即戮,於是與晦、
 遁、兄子世基、世猷及同黨孔延秀、周超、賀愔、竇應期、蔣虔、嚴千斯等並伏誅。世基,絢之子也,有才氣。臨死為連句詩曰:「偉哉橫海鱗,壯矣垂天翼。一旦失風水,翻為螻蟻食。」晦續之曰:「功遂侔昔人,保退無智力。既涉太行險,斯路信難陟。」晦死時,年三十七。



 庾登之、殷道鸞、何承天並皆原免。



 初,河東人商玄石為晦參軍,晦為逆,玄石密欲推西人庾田夫及到彥之從弟為主,田夫等不敢許。知玄石獨謀不立,遂為晦領幢。事既平,恨本心
 之不遂,投水死。太祖嘉之,以其子懷福為衡陽王義季右軍參軍督護。晦走,左右皆棄之,唯有延陵蓋追隨不舍。太祖嘉之,後以蓋為長沙王義欣鎮軍功曹督護。



 史臣曰:謝晦坐璽封違謬,遂免侍中,斯有以見高祖之識治,宰臣之稱職也。



 夫孥戮所施,事行重釁,左黜或用,義止輕愆。輕愆,物之所輕;重釁,人之所重。



 故斧鉞希行於世,徽簡日用於朝,雖貴臣細故,不以任隆弛法,至乎下肅上尊,用此道也。自太祖臨務,茲典稍違,網以疏行,
 法為恩息,妨德害美,抑此之由。降及大明,傾詖愈甚,自非訐竊深私,陵犯密諱,則左降之科,不行於權戚。若有身觸盛旨,釁非國刑,免書裁至,吊客固望其門矣。由是律無恆條,上多弛行,綱維不舉,而網目隨之。所以吉人防著在微,慎大由小,蓋為此云。



\end{pinyinscope}