\article{卷七十五列傳第三十五 王僧達 顏竣}

\begin{pinyinscope}

 王僧達,瑯邪臨沂人,太保弘少子。兄錫,質訥乏風采。太祖聞僧達蚤慧,召見於德陽殿,問其書學及家事,應對閑敏,上甚知之,妻以臨川王義慶女。



 少好學,善屬文。年
 未二十,以為始興王浚後軍參軍,遷太子舍人。坐屬疾,於楊列橋觀鬥鴨,為有司所糾,原不問。性好鷹犬,與閭里少年相馳逐,又躬自屠牛。義慶聞如此,令周旋沙門慧觀造而觀之。僧達陳書滿席,與論文義,慧觀酬答不暇,深相稱美。與錫不協,訴家貧,求郡,太祖欲以為秦郡,吏部郎庾炳之曰:「王弘子既不宜作秦郡,僧達亦不堪蒞民。」乃止。尋遷太子洗馬,母憂去職。兄錫罷臨海郡還,送故及奉祿百萬以上,僧達一夕令奴輦取,無復所餘。
 服闋,為宣城太守。性好游獵,而山郡無事,僧達肆意馳騁,或三五日不歸,受辭訟多在獵所。



 民或相逢不識,問府君所在,僧達曰:「近在後。」元嘉二十八年春,索虜寇逼,都邑危懼,僧達求入衛京師,見許。賊退,又除宣城太守,頃之,徙任義興。



 三十年,元凶弒立,世祖入討,普檄諸州郡;又符郡發兵,僧達未知所從。客說之曰:「方今釁逆滔天,古今未有,為君計,莫若承義師之檄,移告傍郡,使工言之士,明示禍福,茍在有心,誰不響應,此策上也。如其
 不能,可躬率向義之徒,詳擇水陸之便,致身南歸,亦其次也。」僧達乃自候道南奔,逢世祖於鵲頭,即命為長史,加征虜將軍。初,世祖發尋陽,沈慶之謂人曰:「王僧達必來赴義。」人問其所以,慶之曰:「虜馬飲江,王出赴難,見在先帝前,議論開張,執意明決,以此言之,其至必也。」



 上即位,以為尚書右僕射,尋出為使持節、南蠻校尉,加征虜將軍。時南郡王義宣求留江陵,南蠻不解,不成行。仍補護軍將軍。僧達自負才地,謂當時莫及。



 上初踐阼,即居
 端右,一二年間,便望宰相。及為護軍,不得志,乃啟求徐州,曰:臣衰索餘生,逢辰藉業,先帝追念功臣,眷及遺賤,飾短捐陋,布策稠采,從官委褐,十有一載。早憑慶泰,脫親盛明,而有志於學,無獨見之敏,有務在身,無偏鑒之識,固不足建言世治,備辨時宜。竊以天恩不可終報,尸素難可久處,故猖狂蕪謬,每陳所懷。



 陛下孝誠發衷,義順動物,自龍飛以來,實應九服同歡,三光再朗。而臣假視巷里,借聽民謠,黎氓囗囗,未締其感,遠近風議,不
 獲稍進,臣所用夙宵疾首,寤寐疚心者也。臣取之前載,譬之於今。當漢文之時,可謂藉已成之業,據既安之運,重以布衣菲食,憂勤治道,而賈誼披露乃誠,猶有歎哭之諫。況今承顛沛,萬機惟始,恩未及普,信未遑周。臣又聞前達有言,天下,重器也,一安不可卒危,一危亦不可卒安。陛下神思淵通,亦當鑒之聖慮。



 竊謂當今之務,惟在萬有為己,家國同憂,允彼庶心,從民之欲。民有咨瘼之聲,君表納隍之志。下有愆弊之苦,上無侈豫之情。又應
 官酌其才,爵疇其望,與失不賞,寧失不刑。至若樞任重司,籓扞要鎮,治亂攸寄,動靜所歸,百度惟新,或可因而弗革,事在適宜,無或定其出處。天下多才,在所用之。



 臣非惟寄觀世路,謬識其難,即之於身,詳見其弊。何者?臣雖得免牆面,書不入於學伍,行無愆戾,自無近於才能,直以廕託門世,夙列榮齒。且近雖奔迸江路,歸命南闕,竟何功效,可以書賞。而頻出內寵,陛下綢繆數旬之中,累發明詔。



 自非才略有素,聲實相任,豈可聞而弗驚,履
 而無懼。固宜退省身分,識恩之厚,不知報答,當在何期。夫見危致命,死而後已,皆殷勤前誥,重其忘生。臣感先聖格言,思在必效之地,使生獲其志,死得其所。如使臣享厚祿,居重榮,衣狐坐熊,而無事於世者,固所不能安也。



 今四夷猶警,國未忘戰,辮髮凶詭,尤宜裁防。間者天兵未獲,已肆其輕漢之心,恐戎狄貪惏,猶懷匪遜。脫以神州暫擾,中夏兵饑,容或遊魂塞內,重窺邊壘。



 且高秋在節,胡馬興威,宜圖其易,蚤為之所。臣每一日三省,志
 在報效,遠近小大,顧其所安,受效偏方,得司者則慮之所辦,情有不疑。若首統軍政,董勒天兵,既才所不周,實誠亦非願。陛下矜諒已厚,願復曲體此心。護軍之任,臣不敢處,彭城軍府,即時過立。且臣本在驅馳,非希崇顯,輕智小號,足以自安。願垂鑒恕,特賜申獎,則內外榮荷,存沒銘分。



 上不許。僧達三啟固陳,上甚不說。以為征虜將軍、吳郡太守。期歲五遷,僧達彌不得意。吳郭西臺寺多富沙門,僧達求須不稱意,乃遣主簿顧曠率門義劫
 寺內沙門竺法瑤,得數百萬。荊、江反叛,加僧達置佐領兵,臺符聽置千人,而輒立三十隊,隊八十人。又立宅於吳,多役公力。坐免官。



 初,僧達為太子洗馬,在東宮,愛念軍人朱靈寶,及出為宣城,靈寶已長,僧達詐列死亡,寄宣城左永之籍,注以為己子,改名元序,啟太祖以為武陵國典衛令,又以補竟陵國典書令,建平國中軍將軍。孝建元年春,事發,又加禁錮。上表陳謝云:「不能因依左右,傾意權貴。」上愈怒。僧達族子確年少,美姿容,僧達與之
 私款。確叔父休為永嘉太守,當將確之郡,僧達欲逼留之,確知其意,避不復往。



 僧達大怒,潛於所住屋後作大坑,欲誘確來別,因殺而埋之。從弟僧虔知其謀,禁呵乃止。御史中丞劉瑀奏請收治,上不許。



 孝建三年,除太常,意尤不悅。頃之,上表解職,曰:臣自審庸短,少闕宦情,兼宿抱重疾,年月稍甚,生平素念,願閑衡廬。先朝追遠之恩,早見榮齒。曩者以親貧須養,黽勉從祿,解褐後府,十有餘旬。俄遷舍人,殆不朝直。實無緣坐閱宸寵,尸爵家
 庭,情計二三,屢經聞啟,終獲允亮,賜反初服。還私未用,又擢為洗馬,意旨優隆,其令且拜,許有郡缺,當務處置。會琅邪遷改,即蒙敕往反神翰,慈誘殷勤,令裝成即自隨。靈寶往年淪覆長溪,因彼散失,仰感沉恩,俯銘浮寵。臣釁積禍并,仍丁艱罰,聊及視息,即蒙逮問,具啟以奉營情事,負舉猥多。賜蒞宣城,極其窮躓。仲春移任,方冬便值虜南侵。臣忝同肺腑,情為義動,苦求還都,侍衛輦轂。至止之日,戎旗已搴。在郡雖淺,而貪得分了,方拂農
 衣,還事耕牧,宣城民庶,詣闕見請。爾時敕亡從兄僧綽宣見留之旨。暗疾寡任,野心素積,仍附啟苦乞且旋任。還務未期,亡兄臣錫奄見棄背,啟解奔赴,賜帶郡還都,曾未淹積,復除義興。



 臣自天飛海泳,豈假鱗翼,徒思橫施,與日而深。自處官以來,未嘗有涓毫之積,羸疾暗疚,又無人一諾。而性狎林水,偏愛禽魚,議其所託,動乖治要。故收崖斂分,無忘俄頃,實由有待難供,上裝未立,東郡奉輕,西陜祿重。具陳蘄懇,備執初願,氣置江、湘遠郡,
 一二年中,庶反耕之日,糧藥有寄。即蒙亮許,當賜矜擢。



 遭逢厄運,天地崩離,世蒙聖朝門情之顧,及在臣身,復荷殊識,義雖君臣,恩猶父子。臣誠庸蔽,心過草木,奉諱之日,不覺捐身。單軀弱嗣,千里共氣,繼罹凶塗,動臨危盡,生微朝露,不察如絲,信順所扶,得獲全濟,再見天地,重睹三光。于時兄子僧亮等幽窘醜逆,盡室獄戶,山川險阻,吉凶路塞,悠遠之思,誰能勿勞。嘗膽濡足,是其公願,分心掛腹,實亦私苦。



 幸屬聖武,克復大業,宇宙廓清,
 四表靖晏。臣父子叔侄,同獲泰辰,造情追尋,歸骨之本,欲以死明心,誤有餘辰;情願已展,避逆向順,終古常節,智力無效,有何勳庸,而頻煩恩榮,動踰分次。但忽病之日,不敢固辭,故吞訴於鵲渚,飲愧於新亭。及元凶既殄,人神獲乂,端右之授,即具陳請。天慈優渥,每越常倫,南蠻、護軍,旬月私授。臣三省非分,必致孤負,居常輕任,尚懼網墨,況參要內職,承寵外畿,其取覆折,不假識見。故披誠啟訴,表疏相屬,或乞輕高就卑,或願以閑易要,言
 誓致苦,播於辭牘,誠知固陋,當觸明科。去歲往年,累犯刑禁,理無申可,罪有恒典,虛穢朝序,慚累家業,臣甘其終,物議其盡。陛下棄其身瑕,矜其貴戚,迂略法憲,曲相全養。臣一至之感,口此何忘。利伊恩升,加以今位,當時震驚,收足失所,本忘閑情,不敢聞命。內慮於己,外訪於親,以為天地之仁,施不期報,再造之恩,不可妄屬。故洗拂灰壤,登沐膏露,上處聖澤,下更生辰,合芳離蛻,遐邇改觀。但偷榮託幸,忽移此歲,自見妨長,轉不可寧,宜其
 沈放,志事俱盡。



 伏願陛下承太始之德,加成物之恩,及臣狂蔽未至,得於榮次自引,聖朝厚終始之惠,孤臣保不泯之澤。夫讓功為高,臣無功而讓;專素為美,臣榮採已積。以是求退,誠亦可愍。又妻子為居,更無餘累,婢僕十餘,粗有田入,歲時是課,足繼朝昏。兼比日眩瞀更甚,風虛漸劇,湊理合閉,榮衛惛底,心氣忡弱,神志衰散,念此根疵,不支歲月。公私誠願,宜蒙諒許,乞徇餘辰,以終瑣運。白水皎日,不足為譬,願垂矜鑒,哀申此請。



 僧達文
 旨抑揚,詔付門下。侍中何偃以其詞不遜,啟付南臺,又坐免官。頃之,除江夏王義恭太傅長史、臨淮太守,又徙太宰長史,太守如故。大明元年,遷左衛將軍,領太子中庶子。以歸順功,封寧陵縣五等侯。二年,遷中書令。



 先是,南彭城蕃縣民高闍、沙門釋曇標、道方等共相誑惑,自言有鬼神龍鳳之瑞,常聞簫鼓音,與秣陵民藍宏期等謀為亂。又要結殿中將軍苗允、員外散騎侍郎嚴欣之、司空參軍闞千纂、太宰府將程農、王恬等,謀克二年八
 月一日夜起兵攻宮門,晨掩太宰江夏王義恭,分兵襲殺諸大臣,以闍為天子。事發覺,凡黨與死者數十人。



 僧達屢經狂逆,上以其終無悛心,因高闍事陷之,下詔曰:「王僧達餘慶所鐘,早登榮觀,輕險無行,暴於世談。值國道中艱,盡室願效,甄其薄誠,貰其鴻慝,爵遍外內,身窮榮寵。曾無在泮,食椹懷音,乃協規西楚,志擾東區,公行剽掠,顯奪凶黨,倚結群惡,誣亂視聽。朕每容隱,思加蕩雪,曾無犬馬感恩之志,而炎火成燎原之勢,涓流兆江
 河之形,遂脣齒高闍,契規蘇寶,搜詳妖圖,覘察象緯。



 逮賊長臨梟,餘黨就鞫,咸布辭獄牒,宣言虛市,猶欲隱忍,法為情屈。小醜紛紜,人扇方甚,矯構風塵,志希非覬,固已達諸公卿,彰於朝野。朕焉得輕宗社之重,行匹夫之仁。殛山誅邪,聖典所同,戮諷翦律,漢法攸尚。便可收付延尉,肅正刑書。故太保華容文昭公弘契闊歷朝,綢繆眷遇,豈容忘茲勳德,忽其世祀,門爵國姻,一不貶絕。」於獄賜死,時年三十六。



 子道琰,徙新安郡。前廢帝即位,得
 還京邑。後廢帝元徽中,為廬陵國內史,未至郡,卒。蘇寶者,名寶生,本寒門,有文義之美。元嘉中立國子學,為《毛詩》助教,為太祖所知,官至南臺侍御史,江寧令。坐知高闍反不即啟聞,與闍共伏誅。


顏竣,字士遜,琅邪臨沂人,光祿大夫延之子也。太祖問延之:「卿諸子誰有卿風?」對曰:「竣得臣筆,測得臣文,
 \gezhu{
  大}
 得臣義,躍得臣酒。」



 竣初為太學博士,太子舍人,出為世祖撫軍主簿,甚被愛遇,竣亦盡心補益。



 元嘉中,上不欲諸
 王各立朋黨,將召竣補尚書郎。吏部尚書江湛以為竣在府有稱,不宜回改,上乃止。遂隨府轉安北、鎮軍、北中郎府主簿。二十八年,虜自彭城北歸,復求互市,竣議曰:「愚以為與虜和親無益,已然之明效。何以言其然?夷狄之欲侵暴,正苦力之不足耳。未嘗拘制信義,用輟其謀。昔年江上之役,乃是和親之所招。歷稔交聘,遂求國婚,朝廷羈縻之義,依違不絕,既積歲月,漸不可誣,獸心無厭,重以忿怒,故至於深入。幸今因兵交之後,華、戎隔判,
 若言互市,則復開曩敝之萌。議者不過言互市之利在得馬,今棄此所重,得彼下駟,千匹以上,尚不足言,況所得之數,裁不十百邪。一相交關,卒難閉絕。寇負力玩勝,驕黠已甚,雖云互市,實覘國情,多贍其求,則桀慠罔已,通而為節,則必生邊虞。不如塞其端漸,杜其觖望,內修德化,外經邊事,保境以觀其釁,於是為長。」



 初,沙門釋僧含粗有學義,謂竣曰:「貧道粗見讖記,當有真人應符,名稱次第,屬在殿下。」竣在彭城嘗向親人敘之,言遂宣布,
 聞於太祖。時元凶巫蠱事已發,故上不加推治。世祖鎮尋陽,遷南中郎記室參軍。三十年春,以父延之致仕,固求解職,不許。賜假未發,而太祖崩問至,世祖舉兵入討。轉諮議參軍,領錄事,任總外內,并造檄書。世祖發尋陽,便有疾,領錄事自沈慶之以下,並不堪相見,唯竣出入臥內,斷決軍機。時世祖屢經危篤,不任咨稟,凡厥眾事,竣皆專斷施行。



 世祖踐阼,以為侍中,俄遷左衛將軍,加散騎常侍,辭常侍,見許。封建城縣侯,食邑二千戶。



 孝建
 元年,轉吏部尚書,領驍騎將軍。留心選舉,自強不息,任遇既隆,奏無不可。其後謝莊代竣領選,意多不行。竣容貌嚴毅,莊風姿甚美,賓客喧訴,常歡笑答之。時人為之語曰:「顏竣嗔而與人官,謝莊笑而不與人官。」



 南郡王義宣、臧質等反,以竣普領軍。義宣、質諸子藏匿建康、秣陵、湖熟、江寧縣界,世祖大怒,免丹陽尹褚湛之官,收四縣官長,以竣為丹陽尹,加散騎常侍。先是,竣未有子,而大司馬江夏王義恭諸子為元凶所殺,至是並各產男,上
 自為制名,名義恭子為伯禽,以比魯公伯禽,周公旦之子也;名竣子為辟彊,以比漢侍中張良之子。



 先是,元嘉中,鑄四銖錢,輪郭形制,與五銖同,用費損,無利,故百姓不盜鑄。及世祖即位,又鑄孝建四銖。三年,尚書右丞徐爰議曰:「貴貨利民,載自五政,開鑄流圜,法成九府,民富國實,教立化光。及時移俗易,則通變適用,是以周、漢俶遷,隨世輕重。降及後代,財豐用足,因條前寶,無復改創。年歷既遠,喪亂屢經,堙焚剪毀,日月銷減,貨薄民貧,公
 私俱困,不有革造,將至大乏。謂應式遵古典,收銅繕鑄,納贖刊刑,著在往策,今宜以銅贖刑,隨罰為品。」詔可。



 鑄錢形式薄小,輪廓不成。於是民間盜鑄者雲起,雜以鉛錫,並不牢固。又剪鑿古錢,以取其銅,錢轉薄小,稍違官式。雖重制嚴刑,民吏官長坐死免者相係,而盜鑄彌甚,百物踴貴,民人患苦之。乃立品格,薄小無輪郭者,悉加禁斷。



 始興郡公沈慶之立議曰:「昔秦幣過重,高祖是患,普令民鑄,改造榆莢,而貨輕物重,又復乖時。太宗放鑄,
 賈誼致譏,誠以采山術存,銅多利重,耕戰之器,曩時所用,四民競造,為害或多。而孝文弗納,民鑄遂行,故能朽貫盈府,天下殷富。況今耕戰不用,采鑄廢久,熔冶所資,多因成器,功艱利薄,絕吳、鄧之資,農民不習,無釋耒之患。方今中興開運,聖化惟新,雖復偃甲銷戈,而倉庫未實,公私所乏,唯錢而已。愚謂宜聽民鑄錢,郡縣開置錢署,樂鑄之家,皆居署內,平其雜式,去其雜偽,官斂輪郭,藏之以為永寶。去春所禁新品,一時施用,今鑄悉依此
 格。萬稅三千,嚴檢盜鑄,並禁剪鑿。數年之間,公私豐贍,銅盡事息,姦偽自止。且禁鑄則銅轉成器,開鑄則器化為財,翦華利用,於事為益。」



 上下其事公卿,太宰江夏王義恭議曰:「伏見沈慶之議,『聽民私鑄,樂鑄之室,皆入署居。平其準式,去其雜偽』。愚謂百姓不樂與官相關,由來甚久。又多是人士,蓋不願入署。凡盜鑄為利,利在偽雜,偽雜既禁,樂入必寡。云『斂取輪郭,藏為永寶』。愚謂上之所貴,下必從之,百姓聞官斂輪郭,輪郭之價百倍,大小
 對易,誰肯為之。彊制使換,則狀似逼奪。又『去春所禁新品,一時施用』。愚謂此條在可開許。又云『今鑄宜依此格,萬稅三千』。又云『嚴檢盜鑄,不得更造』。



 愚謂禁制之設,非惟一旦,昧利犯憲,群庶常情,不患制輕,患在冒犯。今入署必萬輸三千,私鑄無十三之稅,逐利犯禁,居然不斷。又云『銅盡事息,姦偽自禁』。



 愚謂赤縣內銅,非可卒盡,比及銅盡,姦偽已積。又云『禁鑄則銅轉成器,開鑄則器化為財』。然頃所患,患於形式不均,加以剪鑿,囗鉛錫眾訴
 越耳。若止於盜鑄銅者,亦無須苦禁。」



 竣議曰:「泉貨利用,近古所同,輕重之議,定於漢世,魏、晉以降,未之能改。誠以物貨既均,改之偽生故也。世代漸久,弊運頓至,因革之道,宜有其術。



 今云開署放鑄,誠所欣同。但慮採山事絕,器用日耗,銅既轉少,器亦彌貴。設器直一千,則鑄之減半,為之無利,雖令不行。又云『去春所禁,一時施用』。是欲使天下豐財。若細物必行,而不從公鑄,利己既深,情偽無極,私鑄剪鑿,盡不可禁。五銖半兩之屬,不盈一年,
 必至於盡。財貨未贍,大錢已竭,數歲之間,悉為塵土,豈可令取弊之道,基於皇代。今百姓之貨,雖為轉少,而市井之民,未有嗟怨,此新禁初行,品式未一,須臾自止,不足以垂聖慮。唯府藏空匱,實為重憂。



 今縱行細錢,官無益賦之理,百姓雖贍,無解官乏。唯簡費去華,設在節儉,求贍之道,莫此為貴。然錢有定限,而消失無方;剪鑄雖息,終致窮盡者。亡應官開取銅之署,絕器用之塗,定其品式,日月漸鑄,歲久之後,不為世益耳。」



 時議者又以銅
 轉難得,欲鑄二銖錢。竣又議曰:「議者將為官藏空虛,宜更改鑄,天下銅少,宜減錢式,以救交弊,賑國紓民。愚以為不然。今鑄二銖,恣行新細,於官無解於乏,而民姦巧大興,天下之貨,將靡碎至盡。空立嚴禁,而利深難絕,不過一二年間,其弊不可復救。其甚不可一也。今熔鑄有頓得一二億理,縱復得此,必待彌年。歲暮稅登,財幣暫革,日用之費,不贍數月。雖權征助,何解乏邪?徒使姦民意騁,而貽厥愆謀。此又甚不可二也。民徵大錢之改,兼
 畏近日新禁,市井之間,必生喧擾。遠利未聞,切患猥及,富商得志,貧民困窘。此又甚不可三也。若使交益深重,尚不可行,況又未見其利,而眾弊如此,失算當時,取誚百代乎!」



 前廢帝即位,鑄二銖錢,形式轉細。官錢每出,民間即模效之,而大小厚薄,皆不及也。無輪郭,不磨鑢,如今之剪鑿者,謂之耒子。景和元年,沈慶之啟通私鑄,由是錢貨亂敗,一千錢長不盈三寸,大小稱此,謂之鵝眼錢。劣於此者,謂之綖環錢。入水不沉,隨手破碎,市井不
 復料數,十萬錢不盈一掬,斗米一萬,商貨不行。太宗初,唯禁鵝眼、綖環,其餘皆通用。復禁民鑄,官署亦廢工,尋復並斷,唯用古錢。



 竣自散騎常侍、丹陽尹,加中書令,丹陽尹如故。表讓中書令曰:「虛竊國靈,坐招禁要,聞命慚惶,形魂震越。臣東州凡鄙,生微於時,長自閭閻,不窺官轍,門無富貴,志絕華伍。直以委身壟畝,飢寒交切,先朝陶均庶品,不遺愚賤,得免耕稅之勤,廁仕進之末。陛下盛德居蕃,總攬英異,越以不才,超塵清軌,奉躬歷稔,勞
 效莫書,仰恃曲成之仁,畢願守宰之秩。豈期天地中闋,殷憂啟聖,倚附興運,擢景神塗,雲飛海泳,冠絕倫等,曾未三期,殊命八萃。詳料賞典,則臣不應科;瞻言勤良,則臣與侔貴。方欲訴款皇朝,降階盛序,微已國言,少徹身謗,而制書猥下,爵樹彌隆。臣小人也,不及遠謀,寵利之來,何能居約,徒以上瀆天明,下汨彞議,災謫之興,懼必在邇。今之過授,以先微身,茍曰非據,危辱將及,十手所指,諭等膏肓,所以寤寐兢遽,維縈苦疾者也。伏願陛下
 察其丹誠,矜其疾願,絕會收恩,以全愚分,則造化之施,方茲為薄。」見許。時歲旱民饑,竣上言禁餳一月,息米近萬斛。復代謝莊為吏部尚書,領太子左衛率,未拜,丁憂。起為右將軍,丹陽尹如故。



 竣藉蕃朝之舊,極陳得失。上自即吉之後,多所興造,竣諫爭懇切,無所回避,上意甚不說,多不見從。竣自謂才足乾時,恩舊莫比,當贊務居中,永執朝政,而所陳多不被納,疑上欲疏之,乃求外出,以占時旨。大明元年,以為東揚州刺史,將軍如故。所求
 既許,便憂懼無計。至州,又丁母艱,不許去職,聽送喪還都,恩待猶厚,竣彌不自安。每對親故,頗懷怨憤,又言朝事違謬,人主得失。及王僧達被誅,謂為竣所讒構,臨死陳竣前後忿懟,每恨言不見從。僧達所言,頗有相符據。



 上乃使御史中丞庾徽之奏之曰:臣聞人臣之奉主,毀家光國,竭情無私;若乃無禮陵人,怙富卑上,是以王叔作戒,子晰為戮。未有背本塞原,好利忘義,而得自容盛世,溷亂清流者也。右將軍、東揚州刺史建城縣開國侯
 顏竣,因附風雲,謬蒙翼長,天地更造,拔以非次。



 聖朝親攬,萬務一歸,而窺覘國柄,潛圖秉執。受任選曹,驅扇滋甚;出尹京輦,形勢彌放。傳詔犯憲,舊須啟聞,而竣以通訴忤己,輒加鞭辱,罔顧威靈,莫此為甚。嚴詔屢發,當官責效,竣權恣不行,怨懟彌起,懷挾姦數,苞藏陰慝。預聞中旨,罔不宣露,罰則委上,恩必歸己,荷遇之門,即加謗辱,受譴之室,曲相哀撫。



 翻戾朝紀,狡惑視聽,肋懼上宰,激動閭閻。末上慮聞,內懷猜懼,偽請東牧,以卜天旨。既
 獲出蕃,怨詈方肆,反脣腹誹,方之已輕。且時有啟奏,必協姦私,宣示親朋,動作群小。



 前冬母亡,詔賜還葬,事畢不去,盤桓經時。方構間勳貴,造立同異。又表示危懼,深營身觀,曲訪大臣,慮不全立,遂以己被斥外,國道將顛,釁積懷抱,惡窮辭色。兼行闕於家,早負世議,逮身居崇寵,奉兼萬金,榮以夸親,祿不充養。



 宿憾母弟,恃貴輒戮,天倫怨毒,親交震駭。凡所蒞任,皆闕政刑,輒開丹陽庫物,貸借吏下。多假資禮,解為門生,充朝滿野,殆將千計。
 驕放自下,妨公害私,取監解見錢,以供帳下。賓旅酣歌,不異平日,街談道說,非復風聲。



 竣代都文吏,特荷天私,棄瑕錄用,豫參要重,勞無汗馬,賞班河、山,出內寵靈,踰越倫伍。山川之性,日月彌滋,溪壑之心,在盈彌奢,虎冠狼貪,未足為譬。今皇明開耀,品物咸亨,傷俗點化,實唯害焉,宜加顯戮,以彰盛化。請以見事免竣所居官,下太常削爵土,須事御收付廷尉法獄罪。



 上未欲便加大戮,且止免官。竣頻啟謝罪,并乞性命。上愈怒,詔答曰:「憲司
 所奏,非宿昔所以相期。卿受榮遇,故當極此,訕訐怨憤,已孤本望,乃復過煩思慮,懼不自全,豈為下事上誠節之至邪!」及竟陵王誕為逆,因此陷之。召御史中丞庾徽之於前為奏,奏成,詔曰:「竣孤負恩養,乃可至此。於獄賜死,妻息宥之以遠。」子辟彊徙送交州,又於道殺之。竣文集行於世。



 史臣曰:世祖弱歲監蕃,涵道未廣,披胸解帶,義止賓僚。及運鐘傾陂,身危慮切,擢膽抽肝,猶患言未盡也。至於
 馮玉負扆,威行萬物,欲有必從,事無暫失。



 既而憂歡異日,甘苦變心,主挾今情,臣追昔款,宋昌之報,上賞已行;同舟之慮,下望愈結。嫌怨既萌,誅責自起。竣之取釁於世,蓋由此乎?為人臣者,若能事主而捐其私,立功而忘其報,雖求顛陷,不可得也。



\end{pinyinscope}