\article{卷一百列傳第六十 自序}

\begin{pinyinscope}

 昔少暤金天氏有裔子曰昧,為玄冥師,生允格、臺駘。臺駘能業其官,宣汾、洮,障大澤以處太原,帝顓頊嘉之,封諸汾川。其後四國,沈、姒、蓐、黃。沈子國,今汝南平輿沈亭
 是也。春秋之時,列於盟會。定公四年,諸侯會召陵伐楚,沈子不會,晉使蔡伐沈,滅之,以沈子嘉歸。其後,因國為氏。自茲以降,譜諜罔存。



 秦末有沈逞,徵丞相,不就。漢初逞曾孫保,封竹邑侯。保子遵,自本國遷居九江之壽春,官至齊王太傅、敷德侯。遵子達,驃騎將軍。達子乾,尚書令。乾子弘,南陽太守。弘子勖,河內守。勖子奮,御史中丞。奮子恪,將作大匠。恪子謙,尚書、關內侯。謙子靖,濟陰太守。靖子戎,字威卿,仕州為從事,說降劇賊尹良,漢光武
 嘉其功,封為海昏縣侯,辭不受。因避地徙居會稽烏程縣之餘不鄉,遂世家焉。順帝永建元年,分會稽為吳郡,復為吳郡人。靈帝初平五年,分烏程、餘杭為永安縣,吳孫皓寶鼎二年,分吳郡為吳興郡,復為郡人,雖邦邑屢改,而築室不遷。



 晉武帝平吳後,太康二年,改永安為武康縣,史臣七世祖延始居縣東鄉之博陸里餘烏村。王父從官京師,義熙十一年,高祖賜館于建康都亭里之運巷。



 戎子酆,字聖通,零陵太守,致黃龍芝草之瑞。第二
 子滸,字仲高,安平相。



 少子景,河間相,演之、慶之、曇慶、懷文其後也。滸子鸞,字建光,少有高名,州舉茂才,公府辟州別駕從事史。時廣陵太守陸稠,鸞之舅也,以義烈政績,顯名漢朝,復以女妻鸞。年二十三,早卒。子直,字伯平,州舉茂才,亦有清名,年二十八卒。



 子儀,字仲則,少有至行,兄瑜十歲,儀九歲而父亡,居喪過禮,毀瘠過於成人。外祖會稽盛孝章,漢末名士也,深加憂傷,每擾慰之,曰:「汝並黃中沖爽,終成奇器,何為逾制,自取殄滅邪!」三年禮
 畢,殆至滅性,故兄弟並以孝著。瑜早卒。儀篤學有雄才,以儒素自業。時海內大亂,兵革並起,經術道弛,士少全行,而儀淳深隱默,守道不移,風操貞整,不妄交納,唯與族子仲山、叔山及吳郡陸公紀友善。州郡禮請,二府交辟,公車徵,並不屈,以壽終。



 子憲,字元禮,左中郎、新都都尉、定陽侯,才志顯於吳朝。子矯,字仲桓,以節氣立名,仕為立武校尉、偏將軍,封列侯,建威將軍、新都太守。孫皓時,有將帥之稱。吳平後,為鬱林、長沙太守,並不就。太康
 末卒。子陵,字景高,太傅東海王越辟為從事。元帝之為鎮東將軍,命參軍事。徐馥作亂,殺吳興太守袁琇,陵討平之。子延,字思長,桓溫安西參軍、潁川太守。子賀,字子寧,桓沖南中郎參軍,圍袁真於壽陽,遇疾卒。



 子警,字世明,惇篤有行業,學通《左氏春秋》。家世富殖,財產累千金,仕郡主簿,後將軍謝安命為參軍,甚相敬重。警內足於財,為東南豪士,無仕進意,謝病歸。安固留不止,乃謂警曰:「沈參軍,卿有獨善之志,不亦高乎!」警曰:「使君以道御
 物,前所以懷德而至,既無用佐時,故遂飲啄之願爾。」還家積載,以素業自娛。前將軍、青兗二州刺史王恭鎮京口,與警有舊好,復引為參軍,手書殷勤,苦相招致,不得已而應之,尋復謝職。



 子穆,夫字彥和,少好學,亦通《左氏春秋》。王恭命為前軍主簿,與警書曰:「足下既執不拔之志,高臥東南,故屈賢子共事,非以吏職嬰之也。」初,錢唐人杜子恭通靈有道術,東土豪家及京邑貴望,並事之為弟子,執在三之敬。警累世事道,亦敬事子恭。子恭死,
 門徒孫泰、泰弟子恩傳其業,警復事之。隆安三年,恩於會稽作亂,自稱征東將軍,三吳皆響應。穆夫時在會稽,恩以為前部參軍、振武將軍、餘姚令。其年十二月二十八日,恩為劉牢之所破,輔國將軍高素於山陰回踵埭執穆夫及偽吳郡太守陸瑰之、吳興太守丘尪,並見害,函首送京邑,事見《隆安故事》。先是,宗人沈預素無士行,為警所疾,至是警聞穆夫預亂,逃藏將免矣,預以告官,警及穆夫、弟仲夫、任夫、預夫、佩夫並遇害;唯穆夫子淵
 子、雲子、田子、林子、虔子獲全。



 淵子,字敬深,少有志節,隨高祖克京城,封繁畤縣五等侯。參鎮軍、車騎中軍事,又為道規輔國、征西參軍,領寧蜀太守。與劉基共斬蔡猛於大簿,還為太尉參軍,從征司馬休之,與徐逵之同沒。時年三十五。



 子正,字元直,淹詳有器度,美風姿,善容止,好老、莊之學。弱冠,州辟從事。宗人光祿大夫演之稱之曰:「此宗中千里駒也。」出為始寧、烏傷、婁令,母憂去職。服闋,為隨王誕後軍安南行參軍。誕鎮會稽,復參安東軍
 事。元嘉三十年,元凶弒立,分江東為會州,以誕為刺史。誕將受命,正說司馬顧琛曰:「國家此禍,開闢未聞,今以江東義銳之眾,為天下倡始,若馳一介,四方詎不響應。以此雪朝庭冤恥,大明臣子之節,豈可北面凶逆,使殿下受其偽寵。」琛曰:「江東忘戰日久,士不習兵。雖云逆順不同,然彊弱又異,當須四方有義舉者,然後應之,不為晚也。」正曰:「天下若有無父之國,則可矣。茍其不爾,寧可自安仇恥,而責義於餘方。今正以弒逆冤醜,義不同戴,
 舉兵之日,豈求必全耶!馮衍有言,大漢之貴臣,將不如荊、齊之賤士乎!況殿下義兼臣子,事實家國者哉。」琛乃與正俱入說誕,誕猶預未決。會尋陽義兵起,世祖使至,誕乃加正寧朔將軍,領軍繼劉季之。



 誕入為驃騎大將軍,正為中兵參軍,遷長水校尉。孝建元年,移青州鎮歷城,臨淄地空,除寧朔將軍、齊北海二郡太守,委以全齊之任。未拜,二年卒,時年四十三。



 正生好樂,厚自奉養,既終之後,家無餘財。



 淵子弟雲子,元嘉中,為晉安太守。雲
 子子煥,字士蔚,少為駙馬都尉、奉朝請。元凶之入弒也,煥時兼中庶子,直坊,逼從入臺。劭既自立,以為羽林監,辭不拜,拜員外散騎侍郎,使防南譙王義宣諸子,事在《義宣傳》。仍除丞相行參軍,員外散騎侍郎,南昌令,有能名。晉平王休祐驃騎中兵記室參軍,同僚皆以諂進,煥獨不。頃之,記室參軍周敬祖等為太宗所責得罪,轉煥諮議參軍。後廢帝元徽中,以為寧遠將軍、交州刺史,未至鎮,病卒,時年四十五。



 田子,字敬光,雲子弟也。從高祖
 克京城,進平京邑,參鎮軍軍事,封營道縣五等侯。義熙五年,高祖北伐鮮卑。田子領偏師,與龍驤將軍孟龍符為前鋒。慕容超屯臨朐以距大軍,龍符戰沒,田子力戰破之。及盧循逼京邑,高祖遣田子與建威將軍孫季高由海道襲廣州,加振武將軍。循黨徐道覆還保始興,田子復與右將軍劉籓同共攻討。循尋還廣州圍季高,田子慮季高孤危,謂籓曰:「廣州城雖險固,本是賊之巢穴。今循還圍之,或有內變。且季高眾力寡弱,不能持久。若
 使賊還據此,凶勢復振。下官與季高同履艱難,汎滄海,於萬死之中,克平廣州,豈可坐視危逼,不相拯救。」於是率軍南還,比至,賊已收其散卒,還圍廣州。季高單守危迫,聞田子忽至,大喜。田子乃背水結陳,身率先士卒,一戰破之。於是推鋒追討,又破循於蒼梧、鬱林、寧浦。還至廣州,而季高病死。既兵荒之後,山賊競出,攻沒城郭,殺害長吏。田子隨宜討伐,旬日平殄。刺史褚升度至,乃還京師。除太尉參軍、振武將軍、淮陵內史,賜爵都鄉侯。復
 參世子征虜軍事,將軍、內史如故。八年,從討劉毅。十一年,復從討司馬休之,領別軍,與征虜將軍趙倫之,參征虜軍事、振武將軍、扶風太守。



 十二年,高祖北伐,田子與順陽太守傅弘之各領別軍,從武關入,屯據青泥。



 姚泓欲自禦大軍,慮田子襲其後,欲先平田子,然後傾國東出。乃率步軍數萬,奄至清泥。田子本為疑兵,所領裁數百,欲擊之。傅弘之曰:「彼眾我寡,難可與敵。」



 田子曰:「師貴用奇,不必在眾。」弘之猶固執,田子曰:「眾寡相傾,勢不兩
 立。



 若使賊圍既固,人情喪沮,事便去矣。及其未整,薄之必克,所謂先人有奪人之志也。」便獨率所領鼓而進。合圍數重,田子撫慰士卒曰:「諸君捐親戚,棄墳墓,出矢石之間,正希今日耳。封侯之業,其在此乎!」乃棄糧毀舍,躬勒士卒,前後奮擊,所向摧陷。所領江東勇士,便習短兵,鼓噪奔之,賊眾一時潰散,所殺萬餘人,得泓偽乘輿服御。高祖表言曰:「參征虜軍事、振武將軍、扶風太守沈田子,率領勁銳,背城電激,身先士卒,勇冠戎陳,奮寡對眾,
 所向必摧,自辰及未,斬馘千數。泓喪旗棄眾,奔還霸西,咸陽空盡,義徒四合,清蕩餘燼,勢在跂踵。」



 天子慰勞高祖曰:「逋寇阻隘,晏安假日,舉斧函谷,規延王誅,群師勤王,將離寒暑。公躬秉鈇鉞,棱威首塗,戎略載脂,則郊壘疊卷,崤陜甫踐,則潼塞開扃。



 姚泓窘逼,棄城送死,藍田偏師,覆之霸川,甲首成林,俘獲蔽野,偽首奔迸,華、戎雲集,積紀逋寇,旦夕夷殄。」長安既平,高祖燕于文昌殿,舉酒賜田子曰:「咸陽之平,卿之功也。」即以咸陽相賞。田子
 謝曰:「咸陽之平,此實聖略所振,武臣效節,田子何力之有。」即授咸陽、始平二郡太守。大軍既還,桂陽公義真留鎮長安,以田子為安西中兵參軍、龍驤將軍、始平太守。時佛佛來寇,田子與安西司馬王鎮惡俱出北地禦之。初,高祖將還,田子及傅弘之等並以鎮惡家在關中,不可保信,屢言之高祖。高祖曰:「今留卿文武將士精兵萬人。彼若欲為不善,正足自滅耳。勿復多言。」及俱出北地,論者謂鎮惡欲盡殺諸南人,以數千人送義真南還,因據
 關中反叛。田子與弘之謀,矯高祖令誅之,併力破佛佛,安關中,然後南還謝罪。田子宗人沈敬仁驍果有勇力,田子於弘之營內請鎮惡計事,使敬仁於坐殺之,率左右數十人自歸義真。長史王修收殺田子於長安稿倉門外,是歲,義熙十四年正月十五日也。時年三十六。田子初以功應封,因此事寢。高祖表天子,以田子卒發狂易,不深罪也。無子,弟林子以第二子亮為後。



 亮,字道明,清操好學,善屬文。未弱冠,州辟從事。會稽太守孟顗在
 郡不法,亮糾劾免官,又言災異,轉西曹主簿。時三吳水淹,穀貴民饑,刺史彭城王義康使立議以救民急,亮議以:「東土災荒,民凋穀踴,富民蓄米,日成其價。宜班下所在,隱其虛實,令積蓄之家,聽留一年儲,餘皆勒使糶貨,為制平價,此所謂常道行於百世,權宜用於一時也。又緣淮歲豐,邑富地穰,麥既已登,黍粟行就,可析其估賦,仍就交市,三吳饑民,即以貸給,使強壯轉運,以贍老弱。且酒有喉脣之利,而非餐餌所資,尤宜禁斷,以息遊費。」
 即並施行。



 世祖出鎮歷陽,行參征虜軍事。民有盜發塚者,罪所近村民,與符伍遭劫不赴救同坐。亮議曰:尋發塚之情,事止竊盜,徒以侵亡犯死,故同之嚴科。夫穿掘之侶,必銜枚以晦其迹;劫掠之黨,必讙呼以威其事。故赴凶赫者易,應潛密者難。且山原為無人之鄉,丘壟非恒途所踐,至於防救,不得比之村郭。督實效名,理與劫異,則符伍之坐,居宜降矣。又結罰之科,雖有同符伍之限,而無遠近之斷。夫塚無村界,當以比近坐之。若不域
 之以界,則數步之內,與十里之外,便應同罹其責。防民之禁,不可頓去,止非之憲,宜當其律。愚謂相去百步同赴告不時者,一歲刑,自此以外,差不及罰。



 又啟太祖陳府事曰:「伏見西府兵士,或年幾八十,而猶伏隸;或年始七歲,而已從役。衰耗之體,氣用湮微,兒弱之軀,肌膚未實,而使伏勤昏稚,騖苦傾晚,於理既薄,為益實輕。書制休老以六十為限,役少以十五為制,若力不周務,故當粗存優減。」詔曰:「前已令卿兄改革,尋值遷回,竟是不施
 行耶,今更敕西府也。」



 時營創城府,功課嚴促,亮又陳之曰:「經始城宇,莫非造創,基築既廣,夫課又嚴,不計其勞,茍務其速,以歲月之事,求不日之成。比見役人未明上作,閉鼓乃休,呈課既多,理有不逮。至於息日,拘備關限,方涉暑雨,多有死病,頃日所承,亦頗有逃逸。竊惟此既內籓,事殊外鎮,撫蒞之宜,無繫早晚。若得少寬其工課,稍均其優劇,徒隸既苦,易以悅加,考其卒功,廢闕無幾。臣聞不居其職,不謀其事,庖割有主,尸不越樽,豈臣疏
 小,所當預議。但臣泳恩歲厚,服義累世,茍是所懷,忘其常體。」詔答曰:「啟之甚佳。此亦由來常患,比屢敕之,猶復如此,甚為無理。近復令孟休宣旨,想當不同,卿比可密觀其優劇也。」始興王濬臨揚州,復為主簿、秣陵令,善擿奸伏,有非必禽。太祖稱其能,入為尚書都官郎。



 襄陽地接邊關,江左來未有皇子重鎮。元嘉二十二年,世祖出為撫軍將軍、雍州刺史。天子甚留心,以舊宛比接二關,咫尺崤、陜,蓋襄陽之北扞,且表裏彊蠻,盤帶疆場,以亮
 為南陽太守,加揚武將軍。邊蠻畏服,皆納賦調,有數村狡猾,亮悉誅之。遣吏巡行諸縣,孤寡老疾不能自存者,皆就蠲養,耆年老齒,歲時有餼。



 時儒學崇建,亮開置庠序,訓授生徒。民多發塚,并婚嫁違法,皆嚴為條禁。郡界有古時石堨,蕪廢歲久,亮簽世祖修治之,曰:「施生興業,首教農畝,立民崇政,訓本播穡,故能殷邦康俗,禮節用成。頃北洛侵蕪,南宛彫毀,獫狁肆凶。犬夷充疆,遠肅烽驛,近虞郊閈,遂使沃衍弗井,巨防莫修,窘力輟耕,闕於
 分地,凶荒無待,流冗及今。禮化孚內,威禁清外,斯實去盜修畎,昭農緒稼之時,弘圖廣務,拓土祈年之日。殿下降心育物,振民復古,且方提封榛棘,綏入殊荒。竊見郡境有舊石堨,區野腴潤,實為神皋,而蕪決稍積,久廢其利,凡管所見,謂宜創立。昔文翁守官,起沃成產,偉連撫民,開奧增業,惠昭二邦,庸列兩漢。雖效政圖功,不見所絕,聯事惟忝,憂同職同。」囗囗囗囗囗囗囗囗囗囗囗囗囗囗囗囗又修治馬人陂,民獲其利。在任四年,遷南譙
 王義宣司空中兵參軍。詔曰:「陜西心膂須才,故授卿此職。」隨王誕鎮襄陽,復為後軍中兵,領義成太守。亮蒞官清約,為太祖所嘉,賜以車馬服玩,前後累積。每遠方貢獻絕國勳器,輒班賚焉。又賜書二千卷。



 二十七年,卒官,時年四十七。所著詩、賦、頌、贊、三言、誄、哀辭、祭告請雨文、樂府、挽歌、連珠、教記、白事、箋、表、簽、議一百八十九首。



 林子,字敬士,田子弟也。少有大度,年數歲,隨王父在京口。王恭見而奇之,曰:「此兒王子師之流也。」與眾人共見遺
 寶,咸爭趨之,林子直去不顧。年十三,遇家禍,時雖逃竄,而哀號晝夜不絕聲。王母謂之曰:「汝當忍死彊視,何為空自殄絕。」林子曰:「家門酷橫,無復假日之心,直以至仇未復,故且茍存爾。」一門既陷妖黨,兄弟並應從誅,逃伏草澤,常慮及禍,而沈預家甚彊富,志相陷滅。



 林子與諸兄晝藏夜出,即貨所居宅,營墓葬父祖諸叔,凡六喪,儉而有禮。時生業已盡,老弱甚多,東土饑荒,易子而食,外迫國網,內畏彊仇,沈伏山草,無所投厝。時孫恩屢出會
 稽,諸將東討者相續,劉牢之、高素之放縱其下,虜暴縱橫,獨高祖軍政嚴明,無所侵犯。林子乃自歸曰:「妖賊擾亂,僕一門悉被驅逼,父祖諸叔,同罹禍難,猶復偷生天壤者,正以仇讎未復,親老漂寄爾。今日見將軍伐惡旌善,是有道之師,謹率老弱,歸罪請命。」因流涕哽咽,三軍為之感動。高祖甚奇之,謂曰:「君既是國家罪人,彊讎又在鄉里,唯當見隨還京,可得無恙。」乃載以別船,遂盡室移京口,高祖分宅給焉。博覽眾書,留心文義,從高祖克
 京城,進平都邑。時年十八,身長七尺五寸。沈預慮林子為害,常被甲持戈。至是林子與兄田子還東報讎。五月夏節日至,預正大集會,子弟盈堂,林子兄弟挺身直入,斬預首,男女無長幼悉屠之,以預首祭父、祖墓。仍為本郡所命,毅又板為冠軍參軍,並不就。林子以家門荼蓼,無復仕心,高祖敦逼,至彌年不起。及高祖為揚州,辟為從事,謂曰:「卿何由遂得不仕。頃年相申,欲令萬物見卿此心爾。」固辭不得已,然後就職,領建熙令,封資中縣五
 等侯,時年二十一。



 義熙五年,從伐鮮卑,行參鎮軍軍事。大軍於臨朐交戰,賊遣虎班突騎馳軍後,林子率精勇東西奮擊,皆大破之。慕容超退守廣固,復與劉敬宣攻其西隅。廣固既平,而盧循奄至。初,循之下也,廣固未拔,循潛遣使結林子及宗人叔長。林子即密白高祖,叔長不以聞,反以循旨動林子。叔長素驍果,高祖以超未平,隱之,還至廣固,乃誅叔長。謂林子曰:「昔魏武在官渡,汝、兗之士,多懷貳心,唯李通獨斷大義,古今一也。」循至蔡
 洲,貴遊之徒,皆議還徙,唯林子請移家京邑,高祖怪而問之,對曰:「耿純盡室從戎,李典舉宗居魏。林子雖才非古人,實受恩深重。」高祖稱善久之。



 林子時領別軍於石頭,屢戰摧寇。循每戰無功,乃偽揚聲當悉眾於白石步上,而設伏於南岸,故大軍初起白石,留林子與徐赤將斷拒查浦。林子乃進計曰:「此言妖詐,未必有實,宜深為之防。」高祖曰:「石頭城險,且淮柵甚固,留卿在後,足以守之。」大軍既去,賊果上,赤特將擊之。林子曰:「賊聲往白石,
 而屢來挑戰,其情狀可知矣。賊養銳待期,而吾眾不盈二旅,難以有功。今距守此險,足以自固。若賊偽計不立,大軍尋反,君何患焉?」赤特曰:「今賊悉眾向白石,留者必皆羸老,以銳卒擊之,無不破也。」便鼓噪而出,賊伏兵齊發,赤特軍果敗,棄軍奔北岸;林子率軍收赤特散兵,進戰,摧破之。徐道覆乃更上銳卒,沿塘數里。



 林子策之曰:「賊沿塘結陣,戰者不過一隊。今我據其津而阨其要,彼雖銳師數里,不敢過而東必也。」於是乃斷塘而鬥。久之,
 會硃齡石救至,與林子并勢,賊乃散走。大軍至自白石,殺赤特以殉,以林子參中軍軍事。



 從征劉毅,轉參太尉軍事。十一年,復從討司馬休之。高祖每征討,林子輒摧鋒居前,雖有營部,至於宵夕,輒敕還內侍。賊黨郭亮之招集蠻眾,屯據武陵,武陵太守王鎮惡出奔,林子率軍討之,斬亮之於七里澗,納鎮惡。武陵既平,復討魯軌於石城,軌棄眾奔襄陽,復追躡之。襄陽既定,權留守江陵。十二年,高祖領平北將軍,林子以太尉參軍,復參平北
 軍事。其冬,高祖伐羌,復參征西軍事,悉署三府中兵,加建武將軍,統軍為前鋒,從汴入河。



 時襄邑降人董神虎有義兵千餘人,高祖欲綏懷初附,即板為太尉參軍,加揚武將軍,領兵從戎。林子率神虎攻倉垣,剋之,神虎伐其功,徑還襄邑。林子軍次襄邑,即殺神虎而撫其眾。時偽建威將軍、河北太守薛帛先據解縣,林子至,馳往襲之,帛棄軍奔關中,林子收其兵糧。偽并州刺史、河東太守尹昭據蒲阪,林子於陜城與冠軍檀道濟同攻蒲阪,
 龍驤王鎮惡攻潼關。姚泓聞大軍至,遣偽東平公姚紹爭據潼關。林子謂道濟曰:「今蒲阪城堅池深,不可旬日而剋,攻之則士卒傷,守之則引日久,不如棄之,還援潼關。且潼關天阻,所謂形勝之地,鎮惡孤軍,勢危力屈。若使姚紹據之,則難圖也。及其未至,當并力爭之。若潼關事捷,尹昭可不戰而服。」道濟從之。既至,紹舉關右之眾,設重圍圍林子及道濟、鎮惡等。



 時懸師深入,糧輸艱遠,三軍疑阻,莫有固志。道濟議欲渡河避其鋒,或欲棄捐
 輜重,還赴高祖。林子按劍曰:「相公勤王,志清六合,許、洛已平,關右將定,事之濟否,所係前鋒。今舍已捷之形,棄垂成之業,大軍尚遠,賊眾方盛,雖欲求還,豈可復得。下官受命前驅,誓在盡命,今日之事,自為將軍辦之。然二三君子,或同業艱難,或荷恩罔極,以此退撓,亦何以見相公旗鼓耶!」塞井焚舍,示無全志,率麾下數百人犯其西北。紹眾小靡,乘其亂而薄之,紹乃大潰,俘虜以千數,悉獲紹器械資實。時諸將破賊,皆多其首級,而林子獻
 捷書至,每以實聞,高祖問其故,林子曰:「夫王者之師,本有征無戰,豈可復增張虛獲,以自夸誕。國淵以事實見賞,魏尚以盈級受罰,此亦前事之師表,後乘之良轍也。」高祖曰:「乃所望於卿也。」



 初,紹退走,還保定城,留偽武衛將軍姚鸞精兵守險。林子銜枚夜襲,即屠其城,劓鸞而坑其眾。高祖賜書曰:「頻再破賊,慶快無譬。既屢摧破,想不復久爾。」



 紹復遣撫軍將軍姚贊將兵屯河上,絕水道。贊壘塹未立,林子邀擊,連破之,贊輕騎得脫,眾皆奔敗。
 紹又遣長史領軍將軍姚伯子、寧朔將軍安鸞、護軍姚默騾、平遠將軍河東太守唐小方率眾三萬,屯據九泉,憑河固險,以絕糧援。高祖以通津阻要,兵糧所急,復遣林子爭據河源。林子率太尉行參軍嚴綱、竺靈秀卷甲進討,累戰,大破之,即斬伯子、默騾、小方三級,所俘馘及驢馬器械甚多。所虜獲三千餘人,悉以還紹,使知王師之弘。兵糧兼儲,三軍鼓行而西矣。或曰:「彼去國遠鬥,其鋒不可當。」林子白高祖曰:「姚紹氣蓋關右,而力以勢屈,
 外兵屢敗,衰亡協兆,但恐兇命先盡,不得以釁齊斧爾。」尋紹忽死,可謂天誅。於是贊統後事,鳩集餘眾,復襲林子。林子率師禦之,旗鼓未交,一時披潰,贊輕騎遁走。既連戰皆捷,士馬旌旗甚盛,高祖賜書勸勉,并致縑帛肴漿。



 高祖至閿鄉,姚泓掃境內之民,屯兵堯柳。時田子自武關北入,屯軍藍田,泓自率大眾攻之。高祖慮眾寡不敵,遣林子步自秦嶺,以相接援。比至,泓已摧破,兄弟復共追討,泓乃舉眾奔霸西。田子欲窮追,進取長安,林子
 止之,曰:「往取長安,如指掌爾。復克賊城,便為獨平一國,不賞之功也。」田子乃止。復參相國事,總任如前。林子威聲遠聞,三輔震動,關中豪右,望風請附。西州人李焉等並求立功,孫妲羌雜夷及姚泓親屬,盡相率歸林子。高祖以林子綏略有方,頻賜書褒美,并令深慰納之。長安既平,殘羌十餘萬口,西奔隴上,林子追討至寡婦水,轉鬥達于槐里,克之,俘獲萬計。



 大軍東歸,林子領水軍於石門,以為聲援。還至郡,高祖器其才智,不使出也。



 故出
 仕以來,便管軍要,自非戎軍所指,未嘗外典焉。後太祖出鎮荊州,議以林子及謝晦為蕃佐,高祖曰:「吾不可頓無二人,林子行則晦不宜出。」乃以林子為西郎中兵參軍,領新興太守。林子思議弘深,有所陳畫,高祖未嘗不稱善。大軍還至彭城,林子以行役既久,士有歸心,深陳事宜,并言:「聖王所以戒慎祗肅,非以崇威立武,實乃經國長民,宜廣建蕃屏,崇嚴宿衛。」高祖深相訓納。俄而謝翼謀反,高祖歎曰:「林子之見,何其明也。」太祖進號鎮
 西,隨府轉,加建威將軍、河東太守。時高祖以二虜侵擾,復欲親戎,林子固諫,高祖答曰:「吾輒當不復自行。」



 高祖踐阼,以佐命功,封漢壽縣伯,食邑六百戶,固讓,不許。傅亮與林子書曰:「班爵疇勳,歷代常典,封賞之發,簡自帝心。主上委寄之懷,實參休否,誠心所期,同國榮戚,政復是卿諸人共弘建內外爾。足下雖存挹退,豈得獨為君子邪!」



 除府諮議參軍,將軍、太守如故。尋召暫下,以中兵局事副錄事參軍王華。上以林子清公勤儉,賞賜重疊,
 皆散於親故。家無餘財,未嘗問生產之事,中表孤貧悉歸焉。遭母憂,還東葬,乘輿躬幸,信使相望。葬畢,詔曰:「軍國多務,內外須才,前鎮西諮議、建威將軍、河東太守沈林子,不得遂其情事,可起輔國將軍。」林子固辭,不許,賜墨詔,朔望不復還朝,每軍國大事,輒詢問焉。時領軍將軍謝晦任當國政,晦每疾寧,輒攝林子代之。林子居喪至孝,高祖深相憂愍。頃之有疾,上以林子孝性,不欲使哭泣減損,逼與入省,日夕撫慰。敕諸公曰:「其至性過人,
 卿等數慰視之。」小差乃出。上尋不豫,被敕入侍醫藥,會疾動還外。



 永初三年,薨,時年四十六。群公知上深相矜重,恐以實啟,必有損慟,每見呼問,輒答疾病還家,或有中旨,亦假為其答。高祖尋崩,竟不知也。賜東園秘器,朝服一具,衣一襲,錢二十萬,布二百匹。詔曰:「故輔國將軍沈林子,器懷真審,忠績允著,才志未遂,傷悼在懷。可追贈征虜將軍。」有司率常典也。元嘉二十五年,謚曰懷伯。



 林子簡泰廉靖,不交接世務,義讓之美,著於閨門,雖在
 戎旅,語不及軍事。



 所著詩、賦、贊、三言、箴、祭文、樂府、表、箋、書記、白事、啟事、論、老子一百二十一首。太祖後讀林子集,歎息曰:「此人作公,應繼王太保。」子邵嗣。


劭,字道輝,美風姿,涉獵文史。襲爵,駙馬都尉、奉朝請。太祖以舊恩召見,入拜,便流涕,太祖亦悲不自勝。會彊弩將軍缺,上詔錄尚書彭城王義康曰:「沈邵人身不惡,吾與林子周旋異常,可以補選。」
 \gezhu{
  事見宋文帝中詔}
 於是拜彊弩將軍。出為鐘離太守,在郡有惠政,夾淮人民慕其化,遠近莫不投集。郡
 先無市,時江夏王義恭為南兗州,啟太祖置立焉
 \gezhu{
  事見宋文帝中詔}
 。義恭又啟太祖曰:「盱眙太守劉顯真求自解說,邵往蒞任有績,彰於民聽,若重授盱眙,足為良二千石。」


上不許,曰:「其願還經年,方復作此流遷,必當大罔罔也。」
 \gezhu{
  事見宋文帝中詔}
 。


上敕州辟邵弟亮,邵以從弟正蚤孤,乞移恩於正,上嘉而許之。在任六年,入為衡陽王義季右軍中兵參軍。始興王濬初開後軍府,又為中兵。義季在江陵,安西府中兵久缺,啟太祖求人,上答曰:「稱意才難得。沈邵
 雖未經軍事,既是腹心,作鐘離郡,及在後軍府,房中甚修理,或欲遣之。」其事不果
 \gezhu{
  事見宋文帝中詔}
 。入為通直郎。


時上多行幸,還或侵夜,邵啟事陳論,即為簡出。前後密陳政要,上皆納用之,深相寵待,晨夕兼侍,每出游,或敕同輦。時車駕祀南郊,特詔邵兼侍中負璽,代真官陪乘。大將軍彭城王義康出鎮豫章,申謨為中兵參軍,掌城防之任,廬陵王紹為江州,以邵為南中郎府錄事參軍,行府州事,事未行,會謨丁艱,邵代謨為大將軍中兵,加寧朔
 將軍
 \gezhu{
  事見宋文帝中詔}
 。邵南行,上遂相任委,不復選代,仍兼錄事,領城局。後義康被廢,邵改為廬陵王紹南中郎參軍,將軍如故。義康徙安成,邵復以本號為安成相。在郡以寬和恩信,為南土所懷。郡民王孚有學業,志行見稱州里,邵蒞任未幾,而孚卒,邵贈以孝廉,板教曰:「前文學主簿王孚,行潔業淳,棄華息競,志學修道,老而彌篤。方授右職,不幸暴亡,可假孝廉檄,薦以特牲。



 緬想延陵,以遂本懷。」邵慰恤孤老,勸課農桑,前後累蒙賞賜。邵疾病,使
 命累續,遣御醫上藥,異味遠珍,金帛衣裘,相望不絕。元嘉二十六年,卒,時年四十三。上甚相痛悼。



 子侃嗣,官至山陽王休祐驃騎中兵參軍、南沛郡太守。侃卒,子整應襲爵,齊受禪,國除。



 璞,字道真,林子少子也。童孺時,神意閑審,有異於眾。太祖問林子:「聞君小兒器質不凡,甚欲相識。」林子令璞進見,太祖奇璞應對,謂林子曰:「此非常兒。」年十許歲,智度便有大成之姿,好學不倦,善屬文,時有憶識之功。尤練究萬事,經耳過目,人莫能欺之。居家
 精理,姻族資賴。弱冠,吳興太守王韶之再命,不就。張邵臨郡,又命為主簿,除南平王左常侍。太祖引見,謂曰:「吾昔以弱年出蕃,卿家以親要見輔,今日之授,意在不薄。王家之事,一以相委,勿以國官乖清塗為罔罔也。」



 元嘉十七年,始興王浚為揚州刺史,寵愛殊異,以為主簿。時順陽范曄為長史,行州事。曄性頗疏,太祖召璞謂曰:「神畿之政,既不易理。浚以弱年臨州,萬物皆屬耳目,賞罰得失,特宜詳慎。范曄性疏,必多不同。卿腹心所寄,當密
 以在意。



 彼雖行事,其實委卿也。」璞以任遇既深,乃夙夜匪懈,其有所懷,輒以密啟,每至施行,必從中出。曄正謂聖明留察,故深更恭慎,而莫見其際也。在職八年,神州大治,民無謗黷,璞有力焉。



 二十二年,范曄坐事誅,于時浚雖曰親覽,州事一以付璞。太祖從容謂始興王曰:「沈璞奉時無纖介之失,在家有孝友之稱,學優才贍,文義可觀,而沈深守靜,不求名譽,甚佳。汝但應委之以事,乃宜引與晤對。」浚既素加賞遇,又敬奉此旨。



 璞嘗作《舊宮
 賦》,久而未畢,浚與璞疏曰:「卿常有速藻,《舊宮》何其淹耶?


想行就爾。」璞因事陳答,辭義可觀。浚重教曰:「卿沈思淹日,向聊相敦問,還白斐然,遂兼紙翰。昔曹植有言,下筆成章,良謂逸才贍藻,誇其辭說,以今況之,方知其信。執省躊躇,三復不已。吾遠慚楚元,門盈申、白之賓,近愧梁孝,庭列枚、馬之客,欣恧交至,諒唯深矣。薄因末牘,以代一面。」又與主簿顧邁、孔道存書曰:「沈璞淹思踰歲,卿研慮數旬,瑰麗之美,信同在昔。向聊問之,而遠答累翰,辭藻
 艷逸,致慰良多。既欣股肱備此髦楚,還慚予躬無德而稱。復裁少字,宣志於璞,聊因尺紙,使卿等具知厥心。」
 \gezhu{
  此書真本猶存}
 。浚年既長,璞固求辭事,上雖聽許,而意甚不悅。以璞為浚始興國大農,尋除秣陵令。


時天下殷實,四方輻輳,京邑二縣,號為難治。璞以清嚴制下,端平待物,奸吏斂手,猾民知懼。其閭里少年,博徒酒客,或財利爭鬥,妄相誣引,前後不能判者,璞皆知其名姓,及巧詐緣由,探擿是非,各標證據,或辨甲有以知乙,或驗東而西事自
 顯,莫不厭伏,有如神明。以疾去職。太祖厚加存問,賞賜甚厚。浚出為南徐州,謂璞曰:「浚既出蕃,卿故當臥而護之。」與浚詔曰:「沈璞累年主簿,又經國卿,雖未嘗為行佐,今故當正參軍耶。若爾,正當署餘曹,兼房任,不爾便宜行佐正署中兵,恐於選體如不多耳。」
 \gezhu{
  事見宋文帝中詔}
 乃為正佐。



 俄遷宣威將軍、盱眙太守。時王師北伐,彭、汴無虞。璞以彊寇對陣,事未可測,郡首淮隅,道當衝要,乃修城壘,浚重隍,聚材石,積鹽米,為不可勝之算。



 眾咸不同,朝旨
 亦謂為過。俄而賊大越逸,索虜大帥託跋燾自率步騎數十萬,陵踐六州,京邑為之騷懼,百守千城,莫不奔駭。腹心勸璞還京師,璞曰:「若賊大眾,不盼小城,故無所懼。若肉薄來攻,則成禽也。諸軍何嘗見數十萬人聚在一處,而不敗者。昆陽、合淝,前事之明驗。此是吾報國之秋,諸軍封侯之日。」眾既見璞神色不異,老幼在焉,人情乃定。收集得二千精手,謂諸將曰:「足矣。但恐賊不過爾。」賊既濟淮,諸軍將帥毛遐祚、胡崇之、臧澄之等,為虜所覆,
 無不殄盡,唯輔國將軍臧質挺身走,收散卒千餘人來向城。眾謂璞曰:「若不攻則無所事眾,若其來也,城中止可容見力爾,地狹人多,鮮不為患。且敵眾我寡,人所共知,雖云攻守不同,故當粗量彊弱,知難而退,亦用兵之要。若以今眾法能退敵完城者,則全功不在我,若宜避賊歸都,會資舟楫,則更相蹂踐,正足為患。今閉門勿受,不亦可乎!」璞歎曰:「不然。賊不能登城,為諸君保之。舟楫之計,固已久息。



 賊之殘害,古今之未有,屠剝之刑,眾所共
 見,其中有福者,不過得驅還北國作奴婢爾。彼雖烏合,寧不憚此耶!所謂『同舟而濟,胡、越不患異心』也。今人多則退速,人少則退遲,吾寧欲專功緩賊乎!」乃命開門納質。質見城隍阻固,人情輯和,鮭米豐盛,器械山積,大喜,眾皆稱萬歲。及賊至,四面蟻集攻城,璞與質隨宜應拒,攻守三旬,殄其太半,燾乃遁走。有議欲追之者,璞曰:「今兵士不多,又非素附,雖固守有餘未可以言戰也。但可整舟艫,示若欲渡岸者,以速其走計,不須實行。」咸以為
 然。



 臧質以璞城主,使自上露板。璞性謙虛,推功於質。既不自上,質露板亦不及焉。太祖嘉璞功效,遣中使深相褒美。太祖又別詔曰:「近者險急,老弱殊當憂迫耶。念卿爾時,難為心想。百姓流轉已還,此遣部運尋至,委卿量所贍濟也。」始興王浚亦與璞書曰:「狡虜狂凶,自送近服,偽將即斃,酋長傷殘,實天威所喪,卿諸人忠勇之效也。吾式遏無素,致境蕪民瘠,負乘之愧,允當其責。近乞退謝愆,不蒙垂許,故以報卿。」宣城太守王僧達書與璞曰:「
 足下何如,想館舍正安,士馬無恙。離析有時,音旨無日,憂詠沈吟,增其勞望。間者獯獫扈橫,掠剝邊鄙,郵販絕塵,坰介靡達,瞻江盼淮,眇然千里。吾聞涇陽梗棘,伊滑薦遁,鳥集絃絕,患深自古。承知乃昔寇苦城境,勝胄朝餐,伍甲宵舍,烽鼓交警,羽鏑驟合。而足下砥兵礪伍,總厲豪彥,師請一奮,氓無貳情。遂能固孤城,覆嚴對,陷死地,覿生光,古之田、孫,何以尚茲。商驛始通,粗知梗概,崇贊膽智,嘉賀文猛,甚善甚善。吾近以戎暴橫斥,規效情
 命,收龜落簪,星舍京里,既獲遄至,胡馬卷跡,支離霑德,復繼前緒,《行葦》之懽,實協初慮。但乖塗重隔,顧增慨涕,比恒疾臥,憂委兼疊,裁書送想,無斁久懷。」


征還,淮南太守,賞賜豐厚,日夕宴見。朝士有言璞功者,上曰:「臧質姻戚,又年位在前,盱眙元功,當以歸之。沈璞每以謙自牧,唯恐賞之居前,此士燮之意也。」時中書郎缺,尚書令何尚之領吏部,舉璞及謝莊、陸展,事不行。
 \gezhu{
  事見文帝中詔。凡中詔今悉在臺,猶法書典書也。}



 三十年,元凶弒立,璞乃號泣曰:「一門蒙殊常
 之恩,而逢若斯之運,悠悠上天,此何人哉!」日夜憂歎,以至動疾。會二凶逼令送老弱還都,璞性篤孝,尋聞尊老應幽執,輒哽咽不自勝,疾遂增篤,不堪遠迎,世祖義軍至界首,方得致身。



 先是,琅邪顏竣欲與璞交,不酬其意,竣以致恨。及世祖將至都,方有讒說以璞奉迎之晚,橫罹世難,時年三十八。所著賦、頌、贊、祭文、誄、七、弔、四五言詩、箋、表,皆遇亂零失,今所餘詩筆雜文凡二十首。璞有子曰囗。



 伯玉,字德潤,虔子子也。溫恭有行業,能為文章。
 少除世祖武陵國侍郎,轉右常侍,南中郎行參軍,自國入府,以文義見知,文章多見世祖集。世祖踐阼,除員外散騎郎,不拜。左衛顏竣請為司馬。出補句容令,在縣有能名。復為江夏王義恭太宰行參軍,與奉朝請謝超宗、何法盛校書東宮,復為餘姚令,還為衛尉丞。世祖舊臣故佐,普皆升顯,伯玉自守私門,朔望未嘗問訊。顏師伯、戴法興等並有蕃邸之舊,一不造問,由是官次不進。上以伯玉容狀似畫圖仲尼像,常呼為孔丘。舊制,車駕出
 行,衛尉丞直門,常戎服。張永謂伯玉曰:「此職乖卿志。」王景文亦與伯玉有舊,常陪輦出,指伯玉白上:「孔丘奇形容。」上於是特聽伯玉直門服玄衣。出為晉安王子勛前軍行參軍,侍子勛讀書。隨府轉鎮軍行佐。



 前廢帝時,王景文領選,謂子勛典簽沈光祖曰:「鄧琬一旦為長史行事,沈伯玉先帝在蕃囗佐,今猶不改,民生定不應佳。」戴法興聞景文此言,乃轉伯玉為參軍事。子勛初起兵,轉府功曹。及即偽位,以為中書侍郎。初,伯玉為衛尉丞,太
 宗為衛尉,共事甚美。及子勛敗,伯玉下獄,見原,猶以在南無誠,被責,除南臺御史,尋轉武陵國詹事,又轉大農,母老解職。貧薄理盡,閑臥一室,自非弔省親舊,不嘗出門。司徒袁粲、司空褚淵深相知賞,選為永世令,轉在永興,皆有能名。



 後廢帝元徽三年,卒,時年五十七。伯玉性至孝,奉親有聞,未嘗妄取於人,有物輒散之知故。溫雅有風味,和而能辨,與人共事,皆為深交。



 弟仲玉,泰始末,為寧朔長史、蜀郡太守。益州刺史劉亮卒,仲玉行府州事。



 巴西李承明為亂,仲玉遣司馬王天生討平之。廢帝詔以為安成王撫軍中兵參軍,加建威將軍。沈攸之請為征西諮議,未拜,卒。



 史臣年十三而孤,少頗好學,雖棄日無功,而伏膺不改。常以晉氏一代,竟無全書,年二十許,便有撰述之意。泰始初,征西將軍蔡興宗為啟明帝,有敕賜許,自此迄今,年逾二十,所撰之書,凡一百二十卷。條流雖舉,而採掇未周,永明初,遇盜失第五帙。建元四年未終,被敕撰國史。永明二年,又朅奏兼著作郎,撰次起
 居注。自茲王役,無暇搜撰。五年春,又被敕撰《宋書》。六年二月畢功,表上之,曰:臣約言:臣聞大禹刊木,事炳虞書,西伯戡黎,功煥商典。伏惟皇基積峻,帝烈弘深,樹德往朝,立勳前代,若不觀風唐世,無以見帝媯之美,自非睹亂秦餘,何用知漢祖之業。是以掌言未記,爰動天情,曲詔史官,追述大典。臣實庸妄,文史多闕,以茲不才。對揚盛旨,是用夕惕載懷,忘其寢食者也。



 臣約頓首死罪:竊惟宋氏南面,承歷統天,雖世窮八主,年減百載,而兵車
 亟動,國道屢屯,垂文簡牘,事數繁廣。若夫英主啟基,名臣建績,拯世夷難之功,配天光宅之運,亦足以勒銘鐘鼎,昭被方策。及虐后暴朝,前王罕二,國釁家禍,曠古未書,又可以式規萬葉,作鑒于後。



 宋故著作郎何承天始撰《宋書》,草立紀傳,止於武帝功臣,篇牘未廣。其所撰志,唯《天文》,《律歷》,自此外,悉委奉朝請山謙之。謙之,孝建初,又被詔撰述,尋值病亡,仍使南臺侍御史蘇寶生續造諸傳,元嘉名臣,皆其所撰。寶生被誅,大明中,又命著作
 郎徐爰踵成前作。爰因何、蘇所述,勒為一史,起自義熙之初,訖于大明之末。至於臧質、魯爽、王僧達諸傳,又皆孝武所造。自永光以來,至於禪讓,十餘年內,闕而不續,一代典文,始末未舉。且事屬當時,多非實錄,又立傳之方,取舍乖衷,進由時旨,退傍世情,垂之方來,難以取信。臣以謹更創立,製成新史,始自義熙肇號,終於升明三年。桓玄、譙縱、盧循、馬、魯之徒,身為晉賊,非關後代。吳隱、謝混、郗僧施,義止前朝,不宜濫入宋典。劉毅、何無忌、魏
 詠之、檀恁之、孟昶、諸葛長民,志在興復,情非造宋,今並刊除,歸之晉籍。



 臣遠愧南、董,近謝遷、固,以閭閻小才,述一代盛典,屬辭比事,望古慚良,鞠躬跼蹐,靦汗亡厝。本紀列傳,繕寫已畢,合志表七十卷,臣今謹奏呈。所撰諸志,須成續上。謹條目錄,詣省拜表奉書以聞。臣約誠惶誠恐,頓首頓首!死罪死罪!



\end{pinyinscope}