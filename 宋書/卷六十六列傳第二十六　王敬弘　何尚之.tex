\article{卷六十六列傳第二十六 王敬弘 何尚之}

\begin{pinyinscope}

 王敬弘,琅邪臨沂人也。與高祖諱同,故稱字。曾祖暠,晉驃騎將軍。祖胡之,司州刺史。父茂之,晉陵太守。敬弘少有清尚,起家本國左常侍,衛軍參軍。性恬靜,樂山水,為天
 門太守。敬弘妻,桓玄姊也。敬弘之郡,玄時為荊州,遣信要令過。敬弘至巴陵,謂人曰:「靈寶見要,正當欲與其姊集聚耳,我不能為桓氏贅婿。」



 乃遣別船送妻往江陵。妻在桓氏,彌年不迎。山郡無事,恣其游適,累日不回,意甚好之。轉桓偉安西長史、南平太守。去官,居作唐縣界。玄輔政及篡位,屢召不下。



 高祖以為車騎從事中郎,徐州治中從事史,征西將軍道規咨議參軍。時府主簿宗協亦有高趣,道規並以事外相期。嘗共酣飲致醉,敬弘因
 醉失禮,為外司所白,道規即更引還,重申初宴。召為中書侍郎,始攜家累自作唐還京邑。久之,轉黃門侍郎,不拜。仍除太尉從事中郎,出為吳興太守。舊居餘杭縣,悅是舉也。尋徵為侍中。高祖西討司馬休之,敬弘奉使慰勞,通事令史潘尚於道疾病,敬弘單船送還都,存亡不測,有司奏免官,詔可。未及釋朝服,值赦復官。宋國初建,為度支尚書,遷太常。



 高祖受命,補宣訓衛尉,加散騎常侍。永初三年,轉吏部尚書,常侍如故。敬弘每被除召,即
 便祗奉,既到宜退,旋復解官,高祖嘉其志,不茍違也。復除廬陵王師,加散騎常侍,自陳無德,不可師範令王,固讓不拜。又除秘書監,金紫光祿大夫,加散騎常侍,本州中正,又不就。太祖即位,又以為散騎常侍、金紫光祿大夫,領江夏王師。



 元嘉三年,為尚書僕射。關署文案,初不省讀。嘗豫聽訟,上問以疑獄,敬弘不對。上變色,問左右:「何故不以訊牒副僕射?」敬弘曰:「臣乃得訊牒讀之,政自不解。」上甚不悅。六年,遷尚書令,敬弘固讓,表求還東,上
 不能奪。改授侍中、特進、左光祿大夫,給親信二十人。讓侍中、特進,求減親信之半,不許。



 及東歸,車駕幸冶亭餞送。



 十二年,徵為太子少傅。敬弘詣京師上表曰:「伏見詔書,以臣為太子少傅,承命震惶,喜懼交悸。臣抱疾東荒,志絕榮觀,不悟聖恩,猥復加寵。東宮之重,四海瞻望,非臣薄德,所可居之。今內外英秀,應選者多,且板築之下,豈無高逸,而近私愚朽,污辱清朝。嗚呼微臣,永非復大之一物矣。所以牽曳闕下者,實瞻望聖顏,貪《系》表之旨。
 臣如此而歸,夕死無恨。」詔不許。表疏屢上,終以不拜。



 東歸,上時不豫,自力見焉。



 十六年,以為左光祿大夫、開府儀同三司,侍中如故,又詣京師上表曰:「臣比自啟聞,謂誠心已達,天鑒玄邈,未蒙在宥,不敢宴處,牽曳載馳。臣聞君子行道,忘其為身,三復斯言,若可庶勉,顧惜昏耄,志與願違。禮年七十,老而傳家,家道猶然,況於在國。伏願陛下矜臣西夕,愍臣一至,特回聖恩,賜反其所,則天道下濟,愚心盡矣。」竟不拜,東歸。二十三年,重申前命,又
 表曰:「臣躬耕南澧,不求聞達。先帝拔臣於蠻荊之域,賜以國士之遇。陛下嗣徽,特蒙眷齒,由是感激,委質聖朝。雖懷犬馬之誠,遂無塵露之益。年向九十,生理殆盡,永絕天光,淪沒丘壑。謹冒奉表,傷心久之。」



 明年,薨於餘杭之舍亭山,時年八十八。追贈本官。順帝昇明二年詔曰:「夫塗秘蘭幽,貞芳載越,徽猷沈遠,懋禮彌昭。故侍中、左光祿大夫、開府儀同三司敬弘,神韻沖簡,識宇標峻,德敷象魏,道藹丘園。高挹榮冕,凝心塵外,清光粹範,振俗淳
 風。兼以累朝延賞,聲華在詠,而嘉篆闕文,猷策韜裹,尚想遙芬,興懷寢寤。便可詳定輝謚,式旌追典。」於是謚為文貞公。



 敬弘形狀短小,而坐起端方,桓玄謂之「彈棋八勢」。所居舍亭山,林澗環周,備登臨之美,時人謂之王東山。太祖嘗問為政得失,敬弘對曰:「天下有道,庶人不議。」上高其言。左右常使二老婢,戴五絳五辮,著青紋褲襦,飾以朱粉。女適尚書僕射何尚之弟述之,敬弘嘗往何氏看女,值尚之不在,寄齋中臥。俄頃,尚之還,敬弘使二
 卑守閣不聽尚之入,云「正熱,不堪相見,君可且去」。尚之於是移於它室。子恢之被召為祕書郎,敬弘為求奉朝請,與恢之書曰:「秘書有限,故有競。朝請無限,故無競。吾欲使汝處於不競之地。」太祖嘉而許之。敬弘見兒孫歲中不過一再相見,見輒克日。恢之嘗請假還東定省,敬弘克日見之,至日輒不果,假日將盡,恢之乞求奉辭,敬弘呼前,既至閣,復不見。恢之於閣外拜辭,流涕而去。



 恢之至新安太守,中大夫。恢之弟瓚之,世祖大明中,吏部
 尚書,金紫光祿大夫,謚曰貞子。瓚之弟昇之,都官尚書。昇之子延之,昇明末,為尚書左僕射,江州刺史。



 何尚之,字彥德,廬江灊人也。曾祖準,高尚不應徵辟。祖恢,南康太守。父叔度,恭謹有行業,姨適沛郡劉璩,與叔度母情愛甚篤,叔度母蚤卒,奉姨有若所生。姨亡,朔望必往致哀,并設祭奠,食並珍新,躬自臨視。若朔望應有公事,則先遣送祭,皆手自料簡,流涕對之。公事畢,即往致哀,以此為常,至三年服竟。



 義熙五年,吳興武康縣民
 王延祖為劫,父睦以告官。新制,凡劫身斬刑,家人棄市。睦既自告,於法有疑。時叔度為尚書,議曰:「設法止姦,本於情理,非謂一人為劫,闔門應刑。所以罪及同產,欲開其相告,以出為惡之身。睦父子之至,容可悉共逃亡,而割其天屬,還相縛送,螫毒在手,解腕求全,於情可愍,理亦宜宥。使凶人不容於家,逃刑無所,乃大絕根源也。睦既糾送,則餘人無應復告,並全之。」後為金紫光祿大夫,吳郡太守,加秩中二千石。太保王弘稱其清身潔己。



 元嘉
 八年,卒。



 尚之少時頗輕薄,好摴蒱,既長折節蹈道,以操立見稱。為陳郡謝混所知,與之遊處。家貧,起為臨津令。高祖領征南將軍,補府主簿。從征長安,以公事免,還都。因患勞疾積年,飲婦人乳,乃得差。以從征之勞,賜爵都鄉侯。少帝即位,為廬陵王義真車騎咨議參軍。義真與司徒徐羨之、尚書令傅亮等不協,每有不平之言,尚之諫戒,不納。義真被廢,入為中書侍郎。太祖即位,出為臨川內史,入為黃門侍郎,尚書吏部郎,左衛將軍,父憂去
 職。服闋,復為左衛,領太子中庶子。



 尚之雅好文義,從容賞會,甚為太祖所知。十二年,遷侍中,中庶子如故。尋改領遊擊將軍。



 十三年,彭城王義康欲以司徒左長史劉斌為丹陽尹,上不許。乃以尚之為尹,立宅南郭外,置玄學,聚生徒。東海徐秀、廬江何曇、黃回、潁川荀子華、太原孫宗昌、王延秀、魯郡孔惠宣,並慕道來遊,謂之南學。女適劉湛子黯,而湛與尚之意好不篤。湛欲領丹陽,乃徙尚之為祠部尚書,領國子祭酒。尚之甚不平。湛誅,遷吏
 部尚書。時左衛將軍范曄任參機密,尚之察其意趣異常,白太祖宜出為廣州,若在內釁成,不得不加以鈇鉞,屢誅大臣,有虧皇化。上曰:「始誅劉湛等,方欲超昇後進。曄事跡未彰,便豫相黜斥,萬方將謂卿等不能容才,以我為信受讒說。



 但使共知如此,不憂致大變也。」曄後謀反伏誅,上嘉其先見。國子學建,領國子祭酒。又領建平王師,乃徙中書令,中護軍。



 二十三年,遷尚書右僕射,加散騎常侍。是歲造玄武湖,上欲於湖中立方丈、蓬萊、瀛
 洲三神山,尚之固諫乃止。時又造華林園,並盛暑役人工,尚之又諫,宜加休息,上不許,曰:「小人常自暴背,此不足為勞。」時上行幸,還多侵夕,尚之又表諫曰:「萬乘宜重,尊不可輕,此聖心所鑒,豈假臣啟。輿駕比出,還多冒夜,群情傾側,實有未寧。清道而動,帝王成則,古今深誡,安不忘危。若值汲黯、辛毗,必將犯顏切諫,但臣等碌碌,每存順默耳。伏願少採愚誠,思垂省察,不以人廢,適可以慰四海之望。」亦優詔納之。



 先是,患貨重,鑄四銖錢,民間
 頗盜鑄,多翦鑿古錢以取銅,上患之。二十四年,錄尚書江夏王義恭建議,以一大錢當兩,以防翦鑿,議者多同。尚之議曰:「伏鑒明命,欲改錢制,不勞採鑄,其利自倍,實救弊之弘算,增貨之良術。求之管淺,猶有未譬。夫泉貝之興,以估貨為本,事存交易,豈假數多。數少則幣輕,數多則物重,多少雖異,濟用不殊。況復以一當兩,徒崇虛價者邪!凡創制改法,宜從民情,未有違眾矯物而可久也。泉布廢興,囗囗驟議,前代赤仄白金,俄而罷息,六貨憒亂,
 民泣於市。良由事不畫一,難用遵行,自非急病權時,宜守久長之業。煩政曲雜,致遠常泥。且貨偏則民病,故先王立井田以一之,使富不淫侈,貧不過匱。雖茲法久廢,不可頓施,要宜而近,粗相放擬。若今制遂行,富人貲貨自倍,貧者彌增其困,懼非所以欲均之意。又錢之形式,大小多品,直云大錢,則未知其格。若止於四銖五銖,則文皆古篆,既非下走所識,加或漫滅,尤難分明,公私交亂,爭訟必起,此最是其深疑者也。命旨兼慮翦鑿日多,
 以至消盡;鄙意復謂殆無此嫌。民巧雖密,要有蹤跡,且用錢貨銅,事可尋檢,直由屬所怠縱,糾察不精,致使立制以來,發覺者寡。今雖有懸金之名,竟無酬與之實,若申明舊科,禽獲即報,畏法希賞,不日自定矣。愚者之議,智者擇焉,猥參訪逮,敢不輸盡。」



 吏部尚書庾炳之、侍中太子左衛率蕭思話、中護軍趙伯符、御史中丞何承天、太常郗敬叔並同尚之議。中領軍沈演之以為:「龜貝行於上古,泉刀興自有周,皆所以阜財通利,實國富民者
 也。歷代雖遠,資用彌便,但採鑄久廢,兼喪亂累仍,糜散湮滅,何可勝計。晉遷江南,疆境未廓,或土習其風,錢不普用,其數本少,為患尚輕。今王略開廣,聲教遐暨,金鏹所布,爰逮荒服,昔所不及,悉已流行之矣。用彌曠而貨愈狹,加復競竊翦鑿,銷毀滋繁,刑禁雖重,姦避方密,遂使歲月增貴,貧室日劇,暋作肆力之氓,徒勤不足以贍。誠由貨貴物賤,常調未革,弗思釐改,為弊轉深,斯實親教之良時,通變之嘉會。愚謂若以大錢當兩,則國傳難
 朽之寶,家贏一倍之利,不俟加憲,巧源自絕,施一令而眾美兼,無興造之費,莫盛於茲矣。」上從演之議,遂以一錢當兩,行之經時,公私非便,乃罷。



 二十五年,遷左僕射,領汝陰王師,常侍如故。二十八年,轉尚書令,領太子詹事。二十九年,致仕,於方山著《退居賦》以明所守,而議者咸謂尚之不能固志。



 太子左衛率袁淑與尚之書曰:「昨遣修問,承丈人已晦志山田,雖曰年禮宜遵,亦事難斯貴,俾疏、班、邴、魏,通美於前策,龔、貢、山、衛,淪慚乎曩篇。規迨休
 告,雪滌素懷,冀尋幽之歡,畢囗玄之適。但淑逸操偏迥,野性瞢滯,果茲沖寂,必沈樂忘歸。然而已議塗聞者,謂丈人徽明未耗,譽業方籍,儻能屈事康道,降節殉務,舍南瀕之操,淑此行永決矣。望眷有積,約日無誤。」尚之宅在南澗寺側,故書云「南瀕」,《毛詩》所謂「于以採蘋,南澗之瀕」也。詔書敦勸,上又與江夏王義恭詔曰:「今朝賢無多,且羊、孟尚不得告謝,尚之任遇有殊,便未宜申許邪。」義恭答曰:「尚之清忠貞固,歷事唯允,雖年在懸車,而體獨
 充壯,未相申許,下情所同。」尚之復攝職。羊即羊玄保,孟即孟顗,字彥重,本昌安丘人。兄昶貴盛,顗不就徵辟。昶死後,起家為東陽太守,遂歷吳郡、會稽、丹陽三郡,侍中,僕射,太子詹事,復為會稽太守,卒官,贈左光祿大夫。子劭,尚太祖第十六女南郡公主,女適彭城王義康、巴陵哀王休若。



 尚之既還任事,上待之愈隆。是時復遣軍北伐,資給戎旅,悉以委之。元凶弒立,進位司空,領尚書令。時三方興義,將佐家在都邑,劭悉欲誅之,尚之誘說百端,
 並得免。世祖即位,復為尚書令,領吏部,遷侍中、左光祿大夫,領護軍將軍。



 尋辭護軍,加特進。復以本官領尚書令。丞相南郡王義宣、車騎將軍臧質反,義宣司馬竺超民、臧質長史陸展兄弟并應從誅,尚之上言曰:「刑罰得失,治亂所由,聖賢留心,不可不慎。竺超民為賊既遁走,一夫可禽,若反覆昧利,即當取之,非唯免愆,亦可要不義之賞,而超民曾無此意,微足觀過知仁。且為官保全城府,謹守庫藏,端坐待縛。今戮及兄弟,與向始末無論
 者復成何異。陸展盡質復灼然,便同之巨逆,於事為重。臣豫蒙顧待,自殊凡隸,茍有所懷,不敢自默。」超民坐者由此得原。



 時欲分荊州置郢州,議其所居。江夏王義恭以為宜在巴陵,尚之議曰:「夏口在荊、江之中,正對沔口,通接雍、梁,實為津要,由來舊鎮,根基不易。今分取江夏、武陵、天門、竟陵、隨五郡為一州,鎮在夏口,既有見城,浦大容舫。竟陵出道取荊州,雖水路,與去江夏不異,諸郡至夏口皆從流,並為利便。湘州所領十一郡,其巴陵邊
 帶長江,去夏口密邇,既分湘中,乃更成大,亦可割巴陵屬新州,於事為允。」上從其議,荊、揚二州,戶口半天下,江左以來,揚州根本,委荊以閫外,至是並分,欲以削臣下之權,而荊、揚並因此虛耗。尚之建言復合二州,上不許。



 大明二年,以為左光祿、開府儀同三司,侍中如故。尚之在家常著鹿皮帽,及拜開府,天子臨軒,百僚陪位,沈慶之於殿廷戲之曰:「今日何不著鹿皮冠?」慶之累辭爵命,朝廷敦勸甚篤,尚之謂曰:「主上虛懷側席,詎宜固辭。」慶之
 曰:「沈公不效何公,去而復還也。」尚之有愧色。愛尚文義,老而不休,與太常顏延之論議往反,傳於世。立身簡約,車服率素,妻亡不娶,又無姬妾。秉衡當朝,畏遠權柄,親戚故舊,一無薦舉,既以致怨,亦以此見稱。復以本官領中書令。四年,疾篤,詔遣侍中沈懷文、黃門侍郎王釗問疾。薨于位,時年七十九。追贈司空,侍中、中書令如故。謚曰簡穆公。子偃,別有傳。



 尚之弟悠之,義興太守,侍中,太常。與琅邪王徽相善。悠之卒,徽與偃書曰:「吾與義興,直
 恨相知之晚,每惟君子知我。若夫嘉我小善,矜餘不能,唯賢叔耳。」



 悠之弟愉之,新安太守。愉之弟翌之,都官尚書。悠之子顒之,尚太祖第四女臨海惠公主。太宗世,官至通直常侍。



 史臣曰:江左以來,樹根本於揚越,任推轂於荊楚。揚土自廬、蠡以北,臨海而極大江;荊部則包括湘、沅,跨巫山而掩鄧塞。民戶境域,過半於天下。晉世幼主在位,政歸輔臣,荊、揚司牧,事同二陜。宋室受命,權不能移,二州之
 重,咸歸密戚。是以義宣藉西楚強富,因十載之基,嫌隙既樹,遂規問鼎。而建郢分揚,矯枉過直,籓城既剖,盜實人單,閫外之寄,於斯而盡。若長君南面,威刑自出,至親在外,事不患強。若運經盛衰,時艱主弱,雖近臣懷禍,止有外憚,呂宗不競,實由齊、楚,興喪之源,於斯尤著。尚之言並合,可謂識治也矣!



\end{pinyinscope}