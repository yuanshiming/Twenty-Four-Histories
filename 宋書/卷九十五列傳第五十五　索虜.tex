\article{卷九十五列傳第五十五 索虜}

\begin{pinyinscope}

 索頭虜姓托跋氏,其先
 漢將
 李陵
 後也。陵降匈奴,有數百千種,各立名號,索頭亦其一也。晉初,索頭種有部落數萬家在雲中。惠帝末,並州刺史東嬴公司馬騰於晉陽
 為匈奴所圍,索頭單于猗馳遣軍助騰。懷帝永嘉三年,馳弟盧率部落自雲中入鴈門,就并州刺史劉琨求樓煩等五縣,琨不能制,且欲倚盧為援,乃上言:「盧兄馳有救騰之功,舊勛宜錄,請移五縣民於新興,以其地處之。」琨又表封盧為代郡公。愍帝初,又進盧為代王,增食常山郡。其後盧國內大亂,盧死,子又幼弱,部落分散。盧孫什翼鞬勇壯,眾復附之,號上洛公,北有沙漠,南據陰山,眾數十萬。其後為苻堅所破,執還長安,後聽北歸。鞬死,
 子開字涉珪代立。



 先是,鮮卑慕容垂僭號中山。晉孝武太元二十一年,垂死,開率十萬騎圍中山。



 明年四月,克之,遂王有中州,自稱曰魏,號年天賜。元年,治代郡桑乾縣之平城。



 立學官,置尚書曹。開頗有學問,曉天文。其俗以四月祠天,六月末率大眾至陰山,謂之卻霜。陰山去平城六百里,深遠饒樹木,霜雪未嘗釋,蓋欲以暖氣卻寒也。死則潛埋,無墳壟處所,至於葬送,皆虛設棺柩,立冢槨,生時車馬器用皆燒之以送亡者。開暴虐好殺,民
 不堪命。先是,有神巫誡開當有暴禍,唯誅清河殺萬民,乃可以免。開乃滅清河一郡,常手自殺人,欲令其數滿萬。或乘小輦,手自執劍擊簷輦人腦,一人死,一人代,每一行,死者數十。夜恒變易寢處,人莫得知,唯愛妾名萬人知其處。萬人與開子清河王私通,慮事覺,欲殺開,令萬人為內應。夜伺開獨處,殺之。開臨死,曰:「清河、萬人之言,乃汝等也。」是歲,安帝義熙五年。



 開次子齊王嗣字木末,執清河王,對之號哭,曰:「人生所重者父,云何反逆。」



 逼
 令自殺,嗣代立,謚開道武皇帝。



 十三年,高祖西伐長安,嗣先娶姚興女,乃遣十萬騎屯結河北以救之,大為高祖所破,事在朱超石等傳。於是遣使求和,自是使命歲通。高祖遣殿中將軍沈範、索季孫報使,反命已至河,未濟,嗣聞高祖崩問,追執範等,絕和親。太祖即位,方遣範等歸。



 永初三年十月,嗣自率眾至方城,遣鄭兵將軍揚州刺史山陽公達奚斤、吳兵將軍廣州刺史蒼梧公公孫表、尚書滑稽,領步騎二萬餘人,於滑臺西南東燕縣
 界石濟南渡,輜重弱累自隨。滑臺戍主、寧遠將軍、東郡太守王景度馳告冠軍將軍、司州刺史毛德祖,戍虎牢,遣司馬翟廣率參軍龐諮、上黨太守劉談之等步騎三千拒之。



 軍次卷縣土樓,虜徙營滑臺城東二里,造攻具,日往脅城。德祖以滑臺戍人少,使翟廣募軍中壯士,遣寧遠將軍劉芳之率領,助景度守。芳之將八十餘人,突得入城。



 德祖又遣討虜將軍、弘農太守竇應明領五百人,建武將軍竇霸領二百五十人,並以水軍相繼發,咸
 受翟廣節度。



 初,亡命司馬楚之等常藏竄陳留郡界,虜既南渡,馳相要結,驅扇疆場,大為民患。德祖遣長社令王法政率五百人據邵陵,將劉憐領二百騎至雍丘以防之。楚之於白馬縣襲憐,為憐所破。會臺送軍資至,憐往迎之,而酸棗民王玉知憐南,馳以告虜;虜將滑稽領千乘襲倉垣,兵吏悉踰城散走。陳留太守嚴慢為虜所獲,虜即用王玉為陳留太守,給兵守倉垣。十一月,虜悉力攻滑臺城,城東北崩壞,王景度出奔;景度司馬陽瓚
 堅守不動,眾潰,抗節不降,為虜所殺。竇應明擊虜輜重於石濟,破之,殺賊五百餘人,斬其戍主囗連內頭、張索兒等。應明自石濟赴滑臺,聞城已沒,遂進屯尹卯,竇霸馳就翟廣。虜既克滑臺,并力向廣等,力不敵,引退,轉鬥而前,二日一夜,裁行十許里。虜步軍續至,廣等矢盡力竭,大敗,廣、霸、談之等各單身迸還。



 虜乘勝遂至虎牢,德祖出步騎欲擊之,虜退屯土樓,又退還滑臺。長安、魏昌、藍田三縣民居在虎牢下,德祖皆使入城。虜別遣黑槊
 公率三千人至河陽,欲南渡取金墉。德祖遣振威將軍、河陰令竇晃五百人戍小壘,緱氏令王瑜四百人據監倉,鞏令臣琛五百人固小平,參軍督護張季五百人屯牛蘭,又遣將領馬隊,與洛陽令楊毅合二百騎,緣河上下,隨機赴接。十二月,虜置守於洛川小壘,德祖遣翟廣馳往擊之,虜退走。廣安立守防,修治城塢,復還虎牢。豫州刺史劉粹遣治中高道瑾領步騎五百據項,又遣司馬徐瓊繼之,臺遣將輔伯遣、姚珍、杜坦、梁靈宰等水步
 諸軍續進。徐州刺史王仲德率軍次湖陸。黑槊公遣長史將千人逼竇晃、楊毅,晃等逆擊,禽之,生獲二百人。其後鄭兵將軍五千騎掩襲晃等,黑槊渡與并力,四面攻壘,晃等力少眾散,晃、毅皆被重創。虜將安平公鵝青二軍七千人南渡,於確磝東下,至泗瀆口,去尹卯百許里。兗州刺史徐琰委軍鎮走,於是泰山諸郡並失守。



 鄭兵與公孫表及宋兵將軍、交州刺史交址侯普幾萬五千騎,復向虎牢,於城東南五里結營,分步騎自成皋開向
 虎牢外郭西門。德祖逆擊,殺傷百餘人,虜退還保營。鎮北將軍檀道濟率水軍北救,車騎將軍廬陵王義真遣龍驤將軍沈叔狸三千人就豫州刺史劉粹,量宜赴援。少帝景平元年正月,鄭兵分軍向洛,攻小壘,小壘守將竇晃拒戰,陷沒,河南太守王涓之棄金墉出奔。自虜分軍向洛,德祖每戰輒破之。



 嗣自率大眾至鄴。鄭兵既剋金墉,復還虎牢,德祖於城內穴城,入七丈,二道,出城外,又分作六道,出虜陣後。募敢死之士四百人,參軍范道
 基率二百人為前驅,參軍郭王符、劉規等以二百人為後係,出賊圍外,掩襲其後。虜陣擾亂,斬首數百級,焚燒攻具。虜雖退散,隨復更合。



 虜又遣楚兵將軍徐州刺史安平公涉歸幡能健、越兵將軍青州刺史臨菑侯薛道千、陳兵將軍淮州刺史壽張子張模東擊青州,所向城邑皆奔走。冠軍將軍、青州刺史竺夔鎮東陽城,聞虜將至,斂眾固守。龍驤將軍、濟南太守垣苗率二府郡文武奔就夔。



 夔與將士盟誓,居民不入城者,使移就山阻,燒
 除禾稼,令虜至無所資。虜眾向青州,前後濟河凡六萬騎。三月,三萬騎前追脅。城內文武一千五百人,而半是羌蠻流雜,人情駭懼。竺夔夜遣司馬車宗領五百人出城掩擊,虜眾披退。間二日,虜步騎悉至,繞城四圍,列陣十餘里。至晡退還安水結營,去城二十里,大治攻具,日日分步騎常來逼城。夔夜使殿中將軍竺宗之、參軍賈元龍等領百人,於楊水口兩岸設伏。虜將阿伏斤領三百人晨渡水,兩岸伏發,虜騎四迸,殺傷數十人,梟阿伏
 斤首。虜又進營水南,去城西北四里。



 嗣自鄴遣兵益虎牢,增圍急攻,鄭兵於虎牢率步騎三千,攻潁川太守李元德於許昌。車騎參軍王玄謨領千人,助元德守,與元德俱散敗。虜即用潁川人庾龍為潁川太守,領騎五百,并發民丁以戍城。德祖出軍擊公孫表,大戰,從朝至晡,殺虜數百。會鄭兵軍從許昌還,合圍,德祖大敗,失甲士千餘人,退還固城。嗣又於鄴遣萬餘人從白沙口過河,於濮陽城南寒泉築壘。朝議以:「項城去虜不遠,非輕軍
 所抗,使劉粹召高道瑾還壽陽。若沈叔狸已進,亦宜且追。」粹以虜攻虎牢,未復南向,若便攝軍舍項城,則淮西諸郡,無所憑依。沈叔狸已頓肥口,又不宜便退。



 時李元德率散卒二百人至項,劉粹使助高道瑾戍守,請宥其奔敗之罪,朝議並許之。



 檀道濟至彭城,以青、司二州並急,而所領不多,不足分赴,青州道近,竺夔兵弱,先救青州。竺夔遣人出城作東西南塹,虜於城北三百餘步鑿長圍。夔遣參軍閭茂等領善射五十人,依墻射虜,虜騎數百
 馳來圍墻,墻內納射,固墻死戰。虜下馬步進,短兵接,城上弓弩俱發,虜乃披散。虜遂填外塹,引高樓四所,蝦蟆車二十乘,置長圍內。夔先鑿城北作三地道,令通外塹,復鑿裏塹,內去城二丈作子塹,遣三百餘人出地道,欲燒虜攻具。時回風轉焰,火不得燃,虜兵矢橫下,士卒多傷,斂眾還入。虜填三塹盡平,唯餘子塹,蝦蟆車所不及。虜以橦攻城,夔募人力,於城上係大磨石堆之;又出於子塹中,用大麻絙張骨骨,攻車近城,從地道中多人力
 挽令折。虜復於城南掘長圍,進攻逾急。夔能持重,垣苗有膽幹,故能堅守移時。



 然被攻日久,城轉毀壞,戰士多死傷,餘眾困乏,旦暮且陷,檀道濟、王仲德兼行赴之。



 劉粹遣李元德襲許昌,庾龍奔迸,將宋晃追躡,斬龍首。元德因留綏撫,并上租糧。虜悅勃大肥率三千餘騎,破高平郡所統高平、方與、任城、金鄉、亢父等五縣,殺略二千餘家,殺其男子,驅虜女弱。兗州刺史鄭順之戍湖陸,以兵卒不敢出。



 冠軍將軍申宣戍彭城,去高平二百餘里,
 懼虜至,移郭外居民,並諸營署,悉入小城。



 嗣又遣并州刺史伊樓拔助鄭兵攻虎牢,填塞兩塹。德祖隨方抗拒,頗殺虜,而將士稍零落。四月壬申,虜聞道濟將至,焚燒器械,棄青州走。竺夔上言東陽城被攻毀壞,不可守,移鎮長廣之不其城。夔以固守功,進號前將軍,封建陵縣男,食邑四百戶。夔字祖季,東莞人也。官至金紫光祿大夫。



 嗣率大眾至虎牢,停三日,自督攻城,不能下,回軍向洛陽,留三千人益鄭兵。



 停洛數日,渡河北歸。虜安平
 公等諸軍從青州退還,徑趨滑臺;檀道濟、王仲德步軍乏糧,追虜不及。道濟於泰山分遣仲德向尹卯,道濟停軍湖陸。仲德未至尹卯,聞虜已遠,還就道濟,共裝治水軍。虜安平公諸軍就滑臺,西就鄭兵,共攻虎牢。



 虎牢被圍二百日,無日不戰,德祖勁兵戰死殆盡,而虜增兵轉多。虜撞外城,德祖於內更築三重,仍舊為四,賊撞三城已毀,德祖唯保一城,晝夜相拒,將士眼皆生創,死者太半。德祖恩德素結,眾無離心。德祖昔在北,與虜將公孫表有舊,表
 有權略,德祖患之,乃與交通音問,密遣人說鄭兵,云表與之連謀,每答表書,輒多所治定。表以書示鄭兵,鄭兵倍疑之,言於嗣,誅表。虜眾盛,檀道濟諸救軍並不敢進。劉粹據項城,沈叔狸屯高橋。



 二十一日,虜作地道偷城內井,井深四十丈,山勢峻峭,不可得防。至其月二十三日,人馬渴乏饑疫,體皆乾燥,被創者不復出血。虜因急攻,遂剋虎牢。自德祖及翟廣、竇霸,凡諸將佐及郡守在城內者,皆見囚執,唯上黨太守劉談之、參軍范道基將二
 百人突圍南還。城將潰,將士欲扶德祖出奔,德祖曰:「我與此城并命,義不使此城亡而身在也。」嗣重其固守之節,勒眾軍生致之,故得不死。司空徐羨之、尚書傅亮、領軍將軍謝晦表曰:「去年逆虜縱肆,陵暴河南,司州刺史臣德祖竭誠盡力,抗對強寇,孤城獨守,將涉期年,救師淹緩,舉城淪沒,聖懷垂悼,遠近嗟傷。陛下殷憂諒闇,委政自下,臣等謀猷淺蔽,託付無成,遂令致節之臣,抱忠傾覆,將士殲辱,王略虧挫,上墜先規,下貽國恥。稽之朝
 典,無所辭責。雖有司撓筆,未加準繩,豈宜尸祿,昧安殊寵,乞蒙屏固,以申國法。」不許。



 德祖,滎陽南武陽人也。晉末自鄉里南歸。初為冠軍參軍、輔國將軍,道規為荊州,德祖為之將佐。復為高祖太尉參軍。高祖北伐,以為王鎮惡龍驤司馬;加建武將軍。為鎮惡前鋒,斬賊寧朔將軍趙玄石於柏谷,破弘農太守尹雅於梨城,又破賊大帥姚難於涇水,斬其鎮北將軍姚強。鎮惡剋立大功,蓋德祖之力也。長安平定,以為龍驤將軍、扶風太守,仍遷
 秦州刺史,將軍如故。時佛佛虜為寇,復以德祖為王鎮惡征虜司馬,尋復為桂陽公義真安西參軍、南安太守,將軍如故。復徙馮翊太守。



 高祖東還,以德祖督司州之河東平陽二郡諸軍、輔國將軍、河東太守,代並州刺史劉遵考戍蒲阪。長安不守,合部曲還彭城,除世子中兵參軍,將軍如故。又除督司州之河東平陽河北雍州之京兆豫州之潁川兗州之陳留九郡軍事、滎陽太守,將軍如故,又加京兆太守。高祖踐阼,進號冠軍。論前後功,
 封觀陽縣男,食邑四百戶。又除督司雍并三州豫州之潁川兗州之陳留諸軍事、司州刺史,將軍如故。太祖元嘉六年,死於虜中,時年六十五。世祖大明元年,以德祖弟子熙祚第二息詡之紹德祖封。



 虜既克虎牢,留兵居守,餘眾悉北歸。少帝曰:「故寧遠司馬、濮陽太守陽瓚,滑臺之逼,厲誠固守,投命均節,在危無撓,古之忠烈,無以加之。可追贈給事中,并存恤遺孤,以慰存亡。」尚書令傅亮議瓚家在彭城,宜即以入臺絹一百匹,粟三百斛賜
 給。文士顏延之為誄焉。龍驤將軍兗州刺史徐琰、東郡太守王景度並坐失守,鉗髡居作,琰五歲,景度四歲。



 時宣威將軍、潁川太守李元德戍許昌,仍除滎陽太守,督二郡軍事。其年十一月,虜遣軍并招集亡命,攻逼許昌城,以土人劉遠為滎陽太守。李元德欲出戰,兵仗少,至夜,悉排女牆散潰,元德復奔還項城。虜又圍汝陽,太守王公度將十餘騎突圍奔項城。虜又破邵陵縣,殘害二千餘家,盡殺其男丁,驅略婦女一萬二千口。



 劉粹遣將
 姚聳夫率軍助守項城,又遣司馬徐瓊五百人繼之。虜掘破許昌城,又毀壞鐘離城,以立疆界而還。



 嗣死,謚曰明元皇帝,子燾字佛貍代立。母杜氏,冀州人,入其宮內,生燾。



 燾年十五六,不為嗣所知,遇之如僕隸。嗣初立慕容氏女為后,又娶姚興女,並無子,故燾得立。壯健有筋力,勇於戰鬥,忍虐好殺,夷、宋畏之。攻城臨敵,皆親貫甲胄。元嘉五年,使大將吐伐斤西伐長安,生禽赫連昌於安定,封昌為公,以妹妻之。昌弟赫連定在隴上,吐
 伐斤乘勝以騎三萬討定;定設伏於隴山彈箏谷破之,斬吐伐斤,盡坑其眾。定率眾東還,後剋長安,燾又自攻不剋,乃分軍戍大城而還。



 燾常使昌侍左右,常共單馬逐鹿,深入山澗。昌素有勇名,諸將咸謂昌不可親,壽曰:「天命有在,亦何所懼。」親遇如初。復攻長安,克之,定西走,為吐谷渾慕璝所禽。



 赫連氏有名衛臣者,種落在朔方塞外,部落千餘戶。朔方以西,西至上郡,東西千餘里,漢世徙謫民居之,土地良沃。苻堅時,衛臣入塞寄田,春來
 秋去。堅雲中護軍賈雍掠其田者,獲生口馬牛羊,堅悉以還之,衛臣感恩,遂稱臣入居塞內,其後漸強盛。衛臣死,子佛佛驍猛有謀算,遠近雜種皆附之。姚興與相抗,興覆軍喪眾,前後非一,關中為之傷殘。高祖入長安,佛佛震懾不敢動。高祖東還,即入寇北地。安西將軍義真之歸也,佛佛遣子昌破之青泥,俘囚諸將帥,遂有關中,自稱尊號,號年曰真興元年。



 京兆人韋玄隱居養志,有高名,姚興備禮徵,不起;高祖辟為相國掾,宋臺通直郎,
 又並不就。佛佛召為太子庶子,玄應命。佛佛大怒,曰:「姚興及劉公相徵召,並不起,我有命即至,當以我殊類,不可理其故耶!」殺之。元嘉二年,佛佛死,昌立,至是為燾所兼。燾西定隴右,東滅黃龍,海東諸國,並遣朝貢。



 太祖踐阼,便有志北略。七年三月,詔曰:「河南,中國多故,湮沒非所,遺黎荼炭,每用矜懷。今民和年豐,方隅無事,宜時經理,以固疆場。可簡甲卒五萬,給右將軍到彥之,統安北將軍王仲德、兗州刺史竺靈秀舟師入河;驍騎將軍段宏
 精騎八千,直指虎牢;豫州刺史劉德武勁勇一萬,以相掎角;後將軍長沙王義欣可權假節,率見力三萬,監征討諸軍事。便速備辦,月內悉發。」先遣殿中將軍田奇銜命告燾:「河南舊是宋土,中為彼所侵,今當修復舊境,不關河北。」燾大怒,謂奇曰:「我生頭髮未燥,便聞河南是我家地,此豈可得河南。必進軍,今權當斂戍相避,須冬行地凈,河冰合,自更取之。」



 後將軍長沙王義欣出鎮彭城,總統群帥,告司、兗二州曰:夫王者之兵,以義德相濟,非
 徒疆理土地,恢廣經略,將以大庇蒼生,保全黎庶。是以蒙踐霜雪,踰歷險難,匡國寧民,肅清四表。昔我高祖武皇帝,誕膺明命,爰造區夏,內夷篡逆,外寧寇亂,靈武紛紜,雷動風舉,響斬龍堆,聲浮雲、朔,陵天振地,拔山蕩海。於是華域肅清,謳歌允集,王綱帝典,煥哉惟文,太和煙煴,流澤洋溢。中葉諒暗,委政塚宰,黠虜乘釁,侵侮上國。遂令司、兗良民,復蹈非所,周、鄭遺黎,重隔王化。



 聖皇踐阼,重光開朗,明哲柔遠,以隆中興,遐夷慕義,雲騰波湧。
 方將蹈德履信,被藝襲文,增修業統,作規于後,勤施洽於三方,惠和雍於北狄。夫養魚者除其猵獺,育禽者去其豺狼,故智士研其慮,勇夫厲其節,嘉謀動蒼天,精氣貫辰緯。莫府忝任,稟承廟算,剪爪明衣,誓不顧命,提吳、楚之勁卒,總八州之銳士,紅旗絳天,素甲奪日,虎步中原,龍超河渚。興雲散雨,慰大旱之思;弔民伐罪,積後己之情。師以順動,何征而不克,況乎遵養耆昧,綏復境土而已哉!



 昔淮、泗初開,狡徒縱逸,王旅入關,群豎飆扇,襄
 邑之戰,素旗授首,半城之役,伏尸蔽野,支解體分,羽翼摧挫。加以構難西虜,結怨黃龍,控弦熸滅,首尾逼畏,蜂屯蟻聚,假息旦夕,豈復能超蹈長河,以當堂堂之陳哉!夫順從貴速,歸德惡晚,賞褒先附,威加後服。是以秦、趙羈旅,披棒委誠,施紱乘軒,剖符州郡。慕容、姚泓,恃強作禍,提挈萬里,卒嬰鈇鉞。皆目前之誠驗,往世之所知也。



 聖上明發愛恤,以道懷二州士民,若能審決安危,翻然革面,率其支黨,歸投軍門者,當表言天臺,隨才敘用。如
 其迷心不悛,竄首巢穴,長圍既周,臨衝四至,雖欲壺漿厥篚,其可得乎?幸加三思,詳擇利害。



 彥之進軍,虜悉斂河南一戍歸河北。太祖以前征虜司馬、南廣平太守尹沖為督司雍并三州豫州之潁川兗州之陳留二郡諸軍事、奮威將軍、司州刺史,戍虎牢。十一月,虜大眾南渡河,彥之敗退,洛陽、滑臺、虎牢諸城並為虜所沒。尹沖及司馬滎陽太守崔模抗節不降,投塹死。沖字子順,天水冀人也。先為姚興吏部郎,與興子廣平公弼結黨,欲傾
 興太子泓;泓立,沖與弟弘俱逃叛南歸。至是追贈前將軍。



 太祖與江夏王義恭書曰:「尹沖誠節志概,繼蹤古烈,以為傷惋,不能已已。」



 上以滑臺戰守彌時,遂至陷沒,乃作詩曰:逆虜亂疆埸,邊將嬰寇仇。堅城效貞節,攻戰無暫休。覆沈不可拾,離機難復收。勢謝歸塗單,於焉見幽囚。烈烈制邑守,舍命蹈前修。忠臣表年暮,貞柯見嚴秋。楚莊投袂起,終然報強仇。去病辭高館,卒獲舒國憂。戎事諒未殄,民患焉得瘳。撫劍懷感激,志氣若雲浮。願想
 凌扶搖,弭旆拂中州。爪牙申威靈,帷幄騁良籌。華裔混殊風,率土浹王猷。惆悵懼遷逝,北顧涕交流。



 其後,燾又遣使通好,并求婚姻,太祖每依違之。十七年,燾號太平真君元年。



 十九年,虜鎮東將軍武昌王宜勒庫莫提移書益、梁二州,往伐仇池,侵其附屬,而移書越詣徐州曰:我大魏之興,德配二儀,與造化並立。夏、殷以前,功業尚矣,周、秦以來,赫赫堂堂,垂耀先代。逮我烈祖,重之聖明,應運龍飛,廓清燕、趙。聖朝承王業之資,奮神武之略,遠
 定三秦,西及蔥嶺,東平遼碣,海隅服從,北暨鐘山,萬國納貢,威風所扇,想彼朝野,備聞威德。往者劉、石、苻、姚,遞據三郡,司馬琅邪,保守揚、越,綿綿連連,綿歷年紀。數窮運改,宋氏受終,仍晉之舊,遠通聘享。故我朝庭解甲,息心東南之略,是為不欲違先故之大信也。而彼方君臣,苞藏禍心,屢為邊寇。去庚午年,密結赫連,侵我牢、洛,致師徒喪敗,舉軍囚俘。



 我朝庭仁弘,不窮人之非,不遂人之過,與彼交和,前好無改。昔南秦王楊玄識達天運,於
 大化未及之前,度越赫連,遠歸忠款。玄既即世,弟難當忠節愈固,上請納女,連婚宸極,任土貢珍,自比內郡,漢南白雉,登俎御羞,朝庭嘉之,授以專征之任。不圖彼朝計疆場之小疵,不相關移,竊興師旅,亡我賓屬。難當將其妻子,及其同義,告敗關下。聖朝憮然,顧謂群臣曰:「彼之違信背和,與牢、洛為三,一之為甚,其可再乎。是若可忍,孰不可忍!」是以分命吾等磬聲之臣,助難當報復。



 使持節、侍中、都督雍秦二州諸軍事、安西將軍、建興公吐
 奚愛弼,率南秦王楊難當自祁山南出,直衝建安,令南秦自遣信臣,招集舊戶。使持節、侍中、都督雍梁益三州諸軍事、安西將軍、開府儀同三司、淮陰公皮豹子,員外散騎常侍、平南將軍、南益州刺史、建德公庫拔阿浴河引出斜谷,阨白馬之險。散騎常侍、安南將軍、雍州刺史、南平公娥後延出自駱谷,直截漢水。冠軍將軍、南蠻校尉、荊州刺史、建平公宗𦋎,使持節、員外散騎常侍、冠軍將軍、梁州刺史、順陽公劉買德,平遠將軍、永安侯若
 干內亦千出自子午,東襲梁、漢。使持節、侍中、都督荊梁南雍三州諸軍事、領護南蠻校尉、征南大將軍、開府儀同三司、荊州刺史故晉譙王司馬文思,寧遠將軍、荊州刺史、襄陽公魯軌南趨荊州。使持節、都督洛豫州及河內諸軍事、鎮南大將軍、開府儀同三司、淮南王直勒它大翰為其後繼。使持節、侍中、都督梁益寧三州諸軍事、領護西戎校尉、鎮西大將軍、開府儀同三司、揚州刺史晉琅邪王司馬楚之南趣壽春。使持節、侍中、都督揚豫
 兗徐四州諸軍事、征南將軍、徐兗二州刺史、東安公刁雍東趣廣陵,南至京口。使持節、侍中、都督青、兗、徐三州諸軍事、征東將軍、青徐二州刺史、東海公故晉元顯子司馬天助直趣濟南。十道並進,連營五千,步騎百萬,隱隱桓桓。以此屠城,何城不潰,以此奮擊,何堅不摧!邵陵、踐土,區區齊、晉,尚能克勝強楚,以致一匡,況大魏以沙漠之突騎,兼咸、夏之勁卒哉!



 若眾軍就臨,將令南海北汎,江湖南溢,高岸墊為浦澤,深谷積為丘陵。晉餘黎民,
 將雲集霧聚,仇池之師,τ區山谷之中,何能自固。彼之所謂肆忿於目前之小得,以至於敗亡之大失也。昔信陵君濟窮鳩之危,義士歸之,故我朝廷欲救難當投命之誠,為此舉動。既而愛惜前好,猶復沈吟,多殺生生,在之一亡十,仁者之所不為。吾等別愛後自馳檄相譬書。若攝兵還反,復南秦之國,則諸軍同罷,好穆如初;若距我義言,很愎遂往,敗國亡身,必成噬齊之悔。望所列上彼朝,惠以報告。



 徐州答移曰:知以楊難當投命告敗,比
 之窮鳩,欲動眾以相存拯。救危恤難,有國者之所用心。雖然,移書之言,亦已過矣。何者?楊氏先世以來,受晉爵號,修職守籓,為我西服。十載之中,再造逆亂,號年建義,猖狂妄作,為臣不忠,宜加誅討。又知難當稱臣彼國,宜是顧畏首尾,兩屬求全。果是純臣,服事於魏,何宜與人和親,而聽臣下縱逸。



 昔景平之末,國祚中微,彼乘我內難,侵我司、兗,是以七年治兵,義在經略,三帥涉河,秋豪不犯。但崇此信誓,不負約言耳。彼伺我軍,仍相掩襲,俘
 我甲土,翦我邊民,是彼有兩曲,我有二直也。司馬楚、文思亡命竄伏,魯軌、刁雍實為蠆尾,而擁其逋逃,開其疆場。元顯無子,焉得天助,謬稱假託,何足以云。又譏竊興師旅,不相關移,若如來言,又非所受。黃龍國王受我正朔,且渠茂虔父子歸款,彼皆殘滅俘馘,豈有先言。況仇池奉晉十世,事宋三葉,九伐所加,何傷於彼。僕聞師曲為老,義作亂雄,言貴稱情,不在夸大。移書本詣梁、益,而謬來鄙府,大人不遠,幸無過談。



 二十年,燾以國授其太
 子,下書曰:「朕承祖宗重光之緒,思闡洪基,恢隆萬世。自經營天下,平暴除逆,掃清不順,武功既昭,而文教未闡,非所以崇太平之治也。今者域內安逸,百姓富昌,軍國異容,宜定制度,為萬世之法。夫陰陽有往復,四時有代序,授子任賢,安全相附,所以休息疲勞,式固長久,成其祿福,古今不易之典也。諸朕功臣,勤勞日久,皆當致仕歸第,雍容高爵,頤神養壽,朝請隨時,饗宴朕前,論道陳謀而已,不須復親有司苦劇之職。其令皇太子嗣理萬
 機,總統百揆,更舉賢良,以被列職,皆取後進明能,廣啟選才之路,擇人授任而黜陟之。故孔子曰:『後生可畏,焉知來者之不如今。』主者明為科制,宣敕施行。」



 於是王公以下上書太子皆稱臣,首尾與表同,唯用白紙為異。是歲,燾伐芮芮虜,大敗而還,死者十六七。不聽死家發哀,犯者誅之。



 二十三年,虜安南平南府又移書兗州,以南國僑置州,不依城土,多濫北境名號,又欲遊獵具區。兗州答移曰:夫皇極肇建,實膺神明之符,生民初載,實稟
 沖和之氣。故司牧之功,宣於上代,仁義之道,興自諸華。在昔有晉,混一區宇,九譯承風,遐戎嚮附。永嘉失御,天網圮裂,石、容、苻、姚,遞乘非據,或棲息趙、魏,或保聚邠、岐。我皇宋屬當歸歷,受終晉氏,北臨河、濟,西盡咸、汧,弔民代罪,流澤五都。魏爾時祗德悔禍,思用和輯,交通使命,以祗天衷。來移所謂分疆畫境,其志久定者也。俄而不恆其信,虞我國憂,侵牢及洛,至于清濟。往歲入河,且欲綏理舊城,是以頓兵南澨,秋毫無犯。軍師不能奉遵廟
 算,保有成功,回旆之日,重失司、兗。



 來移云:「不因土立州,招引亡命。」夫古有分土,而無分民,德之休明,四方繦負。昔周道方隆,靈臺初構,民之附化,八十萬家。彼不思弘善政,而恐人之棄己,縱威肆虐,老弱無遺。詳觀今古,略聽輿誦,未有窮凶以延期,安忍而懷眾者也。若必宜因土立州,則彼立徐、揚,豈有其地?



 往年貴主獻書云:「強者為雄。」斯則棄德任力,逆行倒施,有一於此,何以能振。復加欲「游獵具區,觀化南國」。今治道方融,遠人必至,開館
 飾邸,則有司存。來歲元辰,天人協慶,鸞旗省方,東巡稽嶺。若欲邀恩,宜赴茲會,懷德貴蚤,無或後期。又稱:馳獵積年,野無飛伏。」此邦解網舍前,矜蜫育飀,七澤八藪,禽獸豐碩,虞候搜算,義非所吝。三代肆覲,其典雖缺,呼韓入漢,厥儀猶全,饋餼之秩,每存豐厚。



 先是,虜中謠言:「滅虜者吳也。」燾甚惡之。二十三年,北地瀘水人蓋吳,年二十九,於杏城天台舉兵反虜,諸戎夷普並響應,有眾十餘萬。燾聞吳反,惡其名,累遣軍擊之,輒敗。吳上表歸順,
 曰:自靈祚南遷,禍纏神土,二京失統,豹狼縱毒,蒼元蹈犬噬之悲,舊都哀荼蓼之痛。臣以庸鄙,杖義因機,乘寇虜天亡之期,藉二州思奮之憤,故創迹天台,爰暨咸、雍。義風一鼓,率士響同,威聲既張,士卒效勇,師不崇朝,群狡震裂,殄逆鱗於函關,掃凶迹於秦土,非仰協宋靈,俯允群願,焉能若斯者哉!



 今平城遺虐,連兵大壇,東西狼顧,威形莫接,長安孤危,河、洛不戍,平陽二孽,世連土宇,擁率部落,控弦五萬,東屯潼塞,任質軍門。私署安西將
 軍常山白廣平練甲高平,進師汧、隴。北漠護軍結駟連騎,提戈載驅。胡蘭洛生等部曲數千,擬擊偽鎮,闔境顒顒,仰望皇澤。伏願陛下給一旅之眾,北臨河、陜,賜臣威儀,兼給戎械,進可以厭捍凶寇,覆其巢窟,退可以宣國威武,鎮御舊京。使中都有鳴鸞之響,荒餘懷來蘇之德。謹遣使人趙綰馳表丹誠。



 燾遣軍屢敗,乃自率大眾攻之。吳又上表曰:臣聞天無二日,地無二主。昔中都失統,九域分崩,群凶丘列於天邑,飛鴞鴟目於四海。先皇慈
 懷內發,愍及戎荒,翦偽羌於長安,雪黎民之荼炭,政教既被,民始寧蘇。天未忘難,禍亂仍起,獫狁侏張,侵暴中國,使長安為豺狼之墟,鄴、洛為蜂蛇之藪,縱毒生民,虐流兆庶,士女能言,莫不歎憤。傾首東望,仰希拯接,咸同旱苗之待天澤,赤子之望慈親。



 臣仰恩天時,以義伐暴,輒東西結連,南北樹黨,五州同盟,迭相要契。仰馮威靈,千里雲集,冀廓除棒莽,以待王師,義夫始臻,莫不瓦解。虜主二月四日傾資倒庫,與臣連營,接刃交鋒,無日不
 戰,獲賊過半,伏屍蔽野。伏願特遣偏師,賜垂拯接。若天威既震,足使姦虜潰亡,遺民小大,咸蒙生造。



 太祖詔曰:「北地蓋吳,起眾秦川,華戎響附,奮其義勇,頻煩克捷,屢遣表疏,遠效忠款,志梟逆虜,以立勳績。宜加爵號,褒獎乃誠,可以為使持節、都督關隴諸軍事、安西將軍、雍州刺史、北地郡公。使雍、梁遣軍界上,以相援接。」



 燾攻吳大小數十戰,不能剋。太祖遣使送雍、秦二州所統郡及金紫以下諸將印合一百二十一紐與吳,使隨宜假授。屠
 各反叛,吳自攻之,為流矢所中,死。吳弟吾生率餘眾入木面山,皆尋破散。其年,太原民顏白鹿私行入荒,為虜所錄,相州刺史欲殺之,白鹿詐云「青州刺史杜驥使其歸誠」。相州刺史送白鹿至桑乾,燾喜曰:「我外家也。」使其司徒崔浩作書與驥,使司徒祭酒王琦齎書隨白鹿南歸。遣從弟高梁王以重軍延驥,入太原界,攻冀州刺史申恬於歷城,恬擊破之。杜驥遣其寧朔府司馬夏侯祖歡、中兵參軍吉淵馳往赴援,虜破略太原,得四千餘口,
 牛六千餘頭。尋又寇兗、青、冀三州,遂及清東,殺略甚眾。太祖思弘經略,詔群臣曰:吾少覽篇籍,頗愛文義,遊玄玩採,未能息卷。自纓紼世務,情兼家國,徒存日昃,終有慚德。而區宇未一,師饉代有,永言斯瘼,彌干其慮。加疲疾稍增,志隨時往,屬思之功,與事而廢。殘虐遊魂,齊民塗炭,乃眷北顧,無忘弘拯。思總群謀,掃清逋逆,感慨之來,遂成短韻。卿等體國情深,亦當義篤其懷也。詩曰:季父鑒禍先,辛生識機始。崇替非無征,興廢要有以。自昔
 淪中畿,倏焉盈百祀。



 不睹南雲陰,但見胡風起。亂極治必形,塗泰由積否。方欲滌遺氛,矧乃穢邊鄙。



 眷言悼斯民,納隍良在己。逝將振宏羅,一麾同文軌。時乎豈再來?河清難久俟。



 駘駟安局步,騏驥志千里。梁傅畜義心,伊相抱深恥。賞契將誰寄,要之二三子。



 無令齊晉朝,取愧鄒魯士。



 時疆場埸多相侵盜。二十五年,虜寧南將軍、豫州刺史北井侯若庫辰樹蘭移書豫州曰:僕以不德,荷國榮寵,受任邊州,經理民物,宣播政化,鷹揚萬里,雖盡
 節奉命,未能令上化下布,而下情上達也。比者以來,邊民擾動,互有反逆,無復為害,自取誅夷。死亡之餘,雉菟逃竄,南入宋界,聚合逆黨,頻為寇掠,殺害良民,略取資財,大為民患。此之界局,與彼通連,兩民之居,煙火相接,來往不絕,情偽繁興。是以南奸北入,北姦南叛,以類推之,日月彌甚。奸宄之人,數得侵盜之利,雖加重法,不可禁止。僕常申令境局,料其奸源,而彼國牧守,縱不禁御,是以遂至滋蔓,寇擾疆場。譬猶蚤虱疥癬,雖為小痾,令
 人終歲不安。



 當今上國和通,南北好合,唯邊境民庶,要約不明。自古列國,封疆有畔,各自禁斷,無復相侵,如是可以保之長久,垂之永世。故上表臺閣,馳書明曉,自今以後,魏、宋二境,宜使人跡不過。自非聘使行人,無得南北。邊境之民,煙火相望,雞狗之聲相聞,至老死不相往來,不亦善乎!又能此亡彼歸,彼亡此致,則自我國家所望於仁者之邦也。



 右將軍、豫州刺史南平王鑠答移曰:知以邊氓擾動,多有叛逆,欲杜絕奸宄,兩息民患;又欲
 迭送奔亡,禁其來往。



 申告嘉貺,實獲厥心。但彼和好以來,矢言每缺,侵軼之弊,屢違義舉,任情背畔,專肆暴略,豈唯竊犯王黎,乃害及行使。頃誅討蠻髦,事止畿服,或有狐奔鼠竄,逃首北境,而輒便苞納,待之若舊,資其糧仗,縱為寇賊。往歲擅興戎旅,禍加孩耄,罔顧善鄰之約,不惟疆域之限。來示所云,彼並行之,雖豐辭盈觀,即事違實,興嫌長亂,實彼之由,反以為言,將違躬厚之義。



 疆場之民,有自來矣,且相期有素,本不介懷。若於本欲消
 奸弭暴,永存匪石,宜先謹封守,斥遣諸亡,驚蹄逸鏃,不妄入境,則邊城之下,外戶不閉。王制嚴明,豈當獨負來信。若亡命奔越,侵盜彼民,期固刑之所取,無勞遠及。自荷閫外,思闡皇猷,每申敕守宰,務敦義讓。往誠未布,能不愧怍,當重約示,以副至懷。



 二十七年,燾自率步騎十萬寇汝南。初,燾欲為邊寇,聲云獵於梁川。太祖慮其侵犯淮、泗,乃敕邊戍:「小寇至,則堅守拒之;大眾來,則拔民戶歸壽陽。」



 諸戍偵候不明,虜奄來入境,宣威將軍陳南
 頓二郡太守鄭緄、綏遠將軍汝南潁川二郡太守郭道隱並棄城奔走。虜掠抄淮西六郡,殺戮甚多。攻圍懸瓠城,城內戰士不滿千人。先是,汝南、新蔡二郡太守徐遵之去郡,南平王鑠時鎮壽陽,遣左軍行參軍陳憲行郡事。憲嬰城固守,燾盡銳以攻之,憲自登郭城督戰。起樓臨城,飛矢雨集,沖車攻破南城,憲於內更築扞城,立柵以補之。虜肉薄攻城,死者甚眾,憲將士死傷亦過半。燾唯恐壽陽有救兵,不以彭城為慮。



 燾遣從弟永昌王庫
 仁真步騎萬餘,將所略六郡口,北屯汝陽。時世祖鎮彭城,太祖遣隊主吳香爐乘驛敕世祖,遣千騎,齎三日糧襲之。世祖發百里內馬,得千五百匹。眾議舉別駕劉延孫為元帥,延孫辭不肯行,舉參軍劉泰之自代。世祖以問司馬王玄謨、長史張暢,暢等並贊成之。乃分為五軍,以泰之為元帥,與安北騎兵行參軍垣謙之、田曹行參軍臧肇之、集曹行參軍尹定、武陵國左常侍杜幼文五人,各領其一。謙之領泰之軍嗣殿中將軍程天祚督戰,
 至譙城,更簡閱人馬,得精騎千一百匹,直向汝陽。虜不意奇兵從北來,大營在汝陽北,去城三里許。泰之等至,虜都不覺,馳入襲之,殺三千餘人,燒其輜重。營內有數區氈屋,屋中皆有帳,器仗甚精,食具皆是金銀,帳內諸大主帥,悉殺之。諸亡口悉得東走,大呼云:「官軍痛與手。」虜眾一時奔散,因追之,行已經日,人馬疲倦,引還汝南。城內有虜一幢,馬步可五百,登城望知泰之無後繼,又有別帥鉅鹿公餘嵩自虎牢至,因引出擊泰之。泰之軍
 未食,旦戰已疲勞,結陣未及定,垣謙之先退,因是驚亂,棄仗奔走。



 行迷道趨溵水,水深岸高,人馬悉走水爭渡,泰之獨不去,曰:「喪敗如此,何面復還。」下馬坐地,為虜所殺。肇之溺水死,天祚為虜所執,謙之、定、幼文及將士免者九百餘人,馬至者四百匹。世祖降安北之號為鎮軍將軍,玄謨、延孫免官,暢免所領沛郡,謙之伏誅,定、幼文付尚方。



 燾初聞汝陽敗,又傳彭城有係軍,大懼,謂其眾曰:「但聞淮南遣軍,乃復有奇兵出。今年將墮人計中。」即
 燒攻具,欲走。會泰之死問續至,乃停壽陽。遣劉康祖救懸瓠,燾亦遣任城公拒康祖,與戰破之,斬任城。燾攻城四十二日不拔,死者甚多,任城又死,康祖救軍漸進,乃委罪大將,多所斬戮,倍道奔走。太祖嘉憲固守,詔曰:「右軍行參軍、行汝南新蔡二郡軍事陳憲,盡力捍禦,全城摧寇,忠敢之效,宜加顯擢,可龍驤將軍、汝南新蔡二郡太守。」又以布萬匹委憲分賜汝南城內文武吏民戰守勤勞者。



 燾雖不克懸瓠,而虜掠甚多,南師屢無功,為燾
 所輕侮。與太祖書曰:彼前使間諜,詃略奸人,竊聞朱修之、申謨,近復得胡崇之,敗軍之將,國有常刑,乃皆用為方州,虞我之隙,以自慰慶。得我普鐘蔡一豎子,何所損益,無異得我舉國之民,厚加奉養。禽我卑將衛拔,非其身,各便鎖腰苦役以辱之。觀此所行,足知彼之大趣,辨校以來,非一朝一夕也。



 頃關中蓋吳反逆,扇動隴右氐、羌,彼復使人就而誘勸之。丈夫遺以弓矢,婦人遺以環釧,是曹正欲譎誑取賂,豈有遠相順從。為大丈夫之法,
 何不自來取之,而以貨詃引誘我邊民,募往者復除七年,是賞奸人也。我今來至此土,所得多少,孰與彼前後得我民戶邪。彼今若欲保全社稷,存劉氏血食者,當割江以北輸之,攝守南度,如此釋江南使彼居之。不然,可善敕方鎮、刺史、守宰,嚴供張之具,來秋當往取揚州,大勢已至,終不相縱。頃者往索真珠璫,略不相與,今所馘截髑髏,可當幾詐珠璫也。



 彼往日北通芮芮,西結赫連、蒙遜、吐谷渾,東連馮弘、高麗。凡此數國,我皆滅之。以此
 而觀,彼豈能獨立!芮芮吳提以死,其子菟害真襲其凶迹,以今年二月復死。我今北征,先除有足之寇。彼若不從命,來秋當復往取。以彼無足,故不先致討。諸方已定,不復相釋。



 我往之日,彼作何方計,為塹城自守,為築垣以自鄣也。彼土小雨,水便迫掖,彼能水中射我也。我顯然往取揚州,不若彼翳行竊步也。彼來偵諜,我已禽之放還,其人目所盡見,委曲善問之。彼前使裴方明取仇池,既得,疾其勇功,不能容。有臣如此,尚殺之,烏得與我校
 邪!彼非敵也。彼常願欲共我一過交戰,我亦不癡,復不是苻堅。何時與彼交戰,晝則遣騎圍繞,夜則離彼百里宿去,彼人民好,降我者驅來,不好者盡刺殺之。近有穀米,我都啖盡,彼軍復欲食啖何物,能過十日邪?



 彼吳人正有斫營伎,我亦知彼情,離彼百里止宿,雖彼軍三里安邏,使首尾相次,募人裁五十里,天自明去,此募人頭何得不輸我也。彼謂我攻城日,當掘塹圍守,欲出來斫營,我亦不近城圍彼,止築堤引水,灌城取之。彼揚州城
 南北門有兩江水,此二水引用,自可如人意也。知彼公時舊臣,都已殺盡,彼臣若在,年幾雖老,猶有智策,今已殺盡,豈不天資我也。取彼亦不須我兵刃,此有能祝婆羅門,使鬼縛彼送來也。



 此後復求通和,聞太祖有北伐意,又與書曰:「彼此和好,居民連接,為日已久,而彼無厭,誘我邊民,其有往者,復之七年。去春南巡,因省我民,即使驅還。



 自天地啟闢已來,爭天下者,非唯我二人而已。今聞彼自來,設能至中山及桑乾川,隨意而行,來亦不迎,
 去亦不送。若厭其區宇者,可來平城居,我往揚州住,且可博其土地。傖人謂換易為博。彼年已五十,未嘗出戶,雖自力而來,如三歲嬰兒,復何知我鮮卑常馬背中領上生活。更無餘物可以相與,今送獵白鹿馬十二匹并氈藥等物。彼來馬力不足,可乘之。道里來遠,或不服水土,藥自可療。」其年,大舉北討,下詔曰:虜近雖摧挫,獸心靡革,驅逼遺氓,復規竊暴。比得河朔秦雍華戎表疏,歸訴困棘,跂望綏拯,潛相糾結,以候王師。并陳芮芮此春
 因其來掠,掩襲巢窟,種落畜牧,所亡太半,連歲相持,于今未解。又猜虐互發,親黨誅殘,根本危敝,自相殘殄。芮芮間使適至,所說並符,遠輸誠款,誓為犄角。遐邇注情,既宜赴獎,且水雨豐澍,舟楫流通,經略之會,實在茲日。



 可遣寧朔將軍王玄謨率太子步兵校尉沈慶之、鎮軍諮議參軍申坦等,戈船一萬,前驅入河。使持節、督青冀幽三州徐州之東安東莞二郡諸軍事、輔國將軍、青冀二州刺史霄城侯蕭斌,推三齊之鋒,為之統帥。持節、都
 督徐兗青冀幽五州豫州之梁郡諸軍事、鎮軍將軍、徐兗二州刺史武陵王駿,總四州之眾,水陸並驅。太子左衛率始興縣五等侯臧質勒東宮禁兵,統驍騎將軍安復縣開國侯王方回、建武將軍安蠻司馬新康縣開國男劉康祖、右軍參軍事梁坦步騎十萬,徑造許、洛。使持節、督豫司雍秦并五州諸軍事、右將軍、豫州刺史、領安蠻校尉南平王鑠悉荊、河之師,方軌繼進。東西齊舉,宜有董一,使持節、侍中、都督揚南徐二州諸軍事、太尉、領
 司徒、錄尚書、太子太傅、國子祭酒江夏王義恭,德望兼崇,風略遐被,即可三府文武,並被以中儀精卒,出次徐方,為眾軍節度。別府司空府使所督諸鎮,各遣虎旅,數道爭先。督梁南北秦三州諸軍事、綏遠將軍、西戎校尉、梁南北秦三州刺史秀之,統輔國將軍楊文德、宣威將軍巴西梓潼二郡太守劉弘宗,連旗深入,震蕩汧、隴。護軍將軍、封陽縣開國侯蕭思話,部龍驤將軍杜坦、寧遠將軍竟陵太守南城縣開國侯劉德願,籍荊雍之勁,攬
 群師之銳,宜由武關,稜威震水彥。指授之宜,委司空義宣議量。



 是歲,軍旅大起,王公妃主及朝士牧守,各獻金帛等物,以助國用,下及富室小民,亦有獻私財至數十萬者。又以兵力不足,尚書左僕射何尚之參議發南兗州三五民丁,父祖伯叔兄弟仕州居職從事、及仕北徐兗為皇弟皇子從事、庶姓主簿、諸皇弟皇子府參軍督護國三令以上相府舍者,不在發例,其餘悉倩暫行征。符到十日裝束,緣江五郡集廣陵,緣淮三郡集盱眙。又募
 天下弩手,不問所從,若有馬步眾藝武力之士應科者,皆加厚賞。有司又奏軍用不充,揚、南徐、兗、江四州富有之民,家資滿五十萬,僧尼滿二十萬者,並四分換一,過此率討,事息即還。



 歷城建武府司馬申元吉率馬步囗餘人向確磝,取泗瀆口。虜確磝戍主、濟州刺史王買德憑城拒戰,元吉破之。買德棄城走,獲奴婢一百四十口,馬二百餘匹,驢騾二百,牛羊各千餘頭,氈七百領,粗細車三百五十乘,地倉四十二所,粟五十餘萬斛,城內居
 民私儲又二十萬斛,虜田五穀三百頃,鐵三萬斤,大小鐵器九千餘口,餘器仗雜物稱此。



 玄謨攻滑臺不克,燾自率大眾渡河,玄謨敗走。燾從弟永昌王庫仁真發關西兵趨汝、潁,從弟高梁王阿斗泥青州道,燾自確磝,並南出。諸鎮悉斂民保城。其十一月至鄒山,鄒山戍主、宣威將軍、魯陽平二郡太守崔耶利敗沒。燾登鄒山,見秦始皇刻石,使人排倒之。遣楚王樹洛真、南康侯杜道雋進軍清西,至蕭城;步尼公進軍清東,至留城。世祖遣
 參軍馬文恭至蕭城,江夏王義恭遣軍主嵇玄敬至留城,並為覘候。蕭城虜偃旗旌,文恭斥候不明,卒與相遇,乃捨汴趣南山;東至山而虜圍合,文恭戰敗,僅以身免。玄敬亦與留城虜相值,幢主華欽繼其後,虜望玄敬後有軍,引去,趨苞橋。至,欲渡清西,沛縣民燒苞橋,夜於林中擊鼓。虜謂官軍大至,爭渡苞水,水深,溺死殆半。



 先是,燾遣員外散騎侍郎王老壽乘驛就太祖乞黃甘,太祖餉甘十簿、甘蔗千挺。



 并就求馬,曰:「自頃歲成民阜,朝野
 無虞,春末當東巡吳、會,以盡游豫。臨滄海,探禹穴,陟姑蘇之臺,搜長洲之苑,舟楫雖盛,寡於良駟,想能惠以逸足,令及此行。」老壽反命,未出境,虜兵深入,乃錄還。



 虜又破尉武戍,執戍主左軍長兼行參軍王羅漢。先是,南平王鑠以三百人配羅漢出戍,而尉武東北有小壘,因據之。或曰:「賊盛不足自固,南依卑林,寇至易以免。」羅漢以受命來此,不可輒去。是日虜攻之,矢盡力屈,遂沒。虜法,獲生將,付其三郎大帥,連鎖鎖頸後。羅漢夜斷三郎頭,
 抱鎖亡走,得入盱眙城。永昌王破劉康祖於尉武,引眾向壽陽,自青岡屯孫叔敖冢,脅壽陽城,又焚掠馬頭、鐘離。南平王鑠保城固守。



 燾自彭城南出,十二月,於盱眙渡淮,破胡崇之等軍。留尚書韓元興數千人守盱眙,自率大眾南向,中書郎魯秀出廣陵,高梁王阿斗泥出山陽,永昌王於壽陽出橫江。凡所經過,莫不殘害。壽至瓜步,壞民屋宇,及伐蒹葦,於滁口造箄筏,聲欲渡江。太祖大具水軍,為防禦之備。



 初,領軍將軍劉遵考率軍向彭
 城,至小澗,虜已斷道,召還,與左軍將軍尹弘守橫江,少府劉興祖守白下,建威將軍、黃門侍郎蕭元邕守裨洲,羽林左監孟宗嗣守新洲上,建武將軍泰容守新洲下,征北中兵參軍事向柳守貴洲,司馬到元度守蒜山,諮議參軍沈曇慶守北固,尚書褚湛之先行京陵,仍守西津,徐州從事史蕭尚之守練壁,征北參軍管法祖守譙山,徐州從事武仲河守博落,尚書左丞劉伯龍守採石。



 尋遷建武將軍、淮南太守,仍總守事。遊邏上接于湖,下
 至蔡洲,陳艦列營,周亙江畔,自採石至于暨陽,六七百里,船艦蓋江,旗甲星燭。皇太子出戍石頭城,前將軍徐湛之守石頭倉城,都水使者樂詢、尚書水部郎劉淵之並以裝治失旨,付建康。乘輿數幸石頭及莫府山,觀望形勢。購能斬佛狸伐頭者,封八千戶開國縣公,賞布絹各萬匹,金銀各百斤;斬其子及弟、偽相、大軍主,封四百戶開國縣侯,布絹各五千匹;自此以下各有差。又募人齎冶葛酒置空村中,欲以毒虜,竟不能傷。



 燾鑿瓜步山
 為盤道,於其頂設氈屋。燾不飲河南水,以駱駝負河北水自隨,一駱駝負三十斗。遣使餉太祖駱駝名馬,求和請婚。上遣奉朝請田奇餉以珍羞異味。



 燾得黃甘,即啖之,并大進酃酒,左右有耳語者,疑食中有毒,燾不答,以手指天,而以孫兒示奇曰:「至此非唯欲為功名,實是貪結姻援,若能酬酢,自今不復相犯秋毫。」又求嫁女與世祖。二十八年正月朔,燾會于山上,并及土人。會竟,掠民戶,燒邑屋而去。虜初緣江舉烽火,尹弘曰:「六夷如此必
 走。」正月二日,果退。



 初,太祖聞虜寇逆,焚燒廣陵城府船乘,使廣陵、南沛二郡太守劉懷之率人民一時渡江。虜以海陵多陂澤,不敢往。山陽太守蕭僧珍亦斂居民及流奔百姓,悉入城。臺送糧仗給盱眙,賊逼,分留山陽。又有數萬人攻具,當往滑臺,亦留付郡。



 城內垂萬家,戰士五千餘人。有白米陂,去郡數里,僧珍逆下諸處水,注令滿,須賊至,決以灌之。虜既至,不敢停,引去。自廣陵還,因攻盱眙,盡銳攻城,三十日不能剋,乃燒攻具退走。燾凡
 破南兗、徐、兗、豫、青、冀六州,殺略不可稱計,而其士馬死傷過半,國人並尤之。



 是歲,燾病死,謚為太武皇帝。初,燾有六子,長子晃,字天真,為太子。次曰晉王,燾所住屠蘇為疾雷擊,屠蘇倒,見厭殆死,左右皆號泣,晉王不悲,燾怒賜死。次曰秦王烏弈肝,與晃對掌國事,晃疾之,訴其貪暴,燾鞭之二百,遣鎮桴罕。次曰燕王。次曰吳王,名可博真。次曰楚王,名樹洛真。燾至汝南瓜步,晃私遣取諸營,鹵獲甚眾。燾歸聞知,大加搜檢。晃懼,謀殺燾,燾乃詐
 死,使其近習召晃迎喪,於道執之,及國,罩以鐵籠,尋殺之。以烏弈肝有武用,以為太子。會燾死,使嬖人宗愛立博真為後,宗愛、博真恐為弈肝所危,矯殺之而自立,號年承平。博真懦弱,不為國人所附,晃子浚字烏雷直勤,素為燾所愛,燕王謂國人曰:「博真非正,不宜立,直勤嫡孫,應立耳。」乃殺博真及宗愛,而立浚為主,號年為正平。



 先是,虜寧南將軍魯爽兄弟率眾歸順。二十九年,太祖更遣張永、王玄謨及爽等北伐,青州刺史劉興祖建議
 伐河北,曰:「河南阻饑,野無所掠,脫意外固守,非旬月可拔,稽留大眾,轉輸方勞。伐罪弔民,事存急速,今偽帥始死,兼逼暑時,國內猜擾,不暇遠赴,關內之眾,裁足自守。愚謂宜長驅中山,據其關要。冀州已北,民人尚豐,兼麥已向熟,資因為易。向義之徒,必應響赴,若中州震動,黃河以南,自當消潰。臣城守之外,可有二千人,今更發三千兵,假別駕崔勳之振威將軍,領所發隊,并二州望族,從蓋柳津直衝中山。申坦率歷城之眾,可有二千,駱驛
 俱進。較略二軍,可七千許人,既入其心腹,調租發車,以充軍用。若前驅乘勝,張永及河南眾軍,便宜一時濟河,使聲實兼舉。愚計謬允,宜並建司牧,撫柔初附。



 定州刺史取大嶺,冀州刺史向井陘,並州刺史屯雁門,幽州刺史塞軍都,相州刺史備大行,因事指麾,隨宜加授。畏威欣寵,人百其懷,濟河之日,請大統版假。常忿將率憚於深遠,勳之等慷慨之誠,誓必死效。若能成功,清一可待;若不克捷,不為大傷。並催促裝束,伏聽敕旨。」上意止存
 河南,不納。玄謨攻確磝,不克退還。



 世祖即位,索虜求互市,江夏王義恭、竟陵王誕、建平王宏、何尚之、何偃以為宜許;柳元景、王玄謨、顏竣、謝莊、檀和之、褚湛之以為不宜許。時遂通之。



 大明二年,虜寇青州,為刺史顏師伯所破,退走。前廢帝永光元年,濬死,謚文成皇帝。子弘之字第豆胤代立。景和中,北討徐州刺史義陽王昶,昶單騎奔虜。太宗泰始初,江州刺史晉安王子勛為逆,四方反,徐州刺史薛安都、青州刺史沈文秀、冀州刺史歷城鎮
 主崔道固等,亦各舉兵。虜謀欲納昶,下書曰:《易》稱「利用行師」,《書》云「恭行天罰」,必觀時而後施,因機而後舉。



 故夏伐有扈,四海以平,晉定吳會,萬方以壹。今宗室衰微,凶難洊起,國有殺君之逆,邦罹崩離之難,起自蕭墻,釁流合境。偽使持節、散騎常侍、都督徐南北兗青冀幽七州豫州之梁郡諸軍事、征北將軍、儀同三司、徐州刺史義陽王昶,踵微子之蹤,蹈項伯之跡,知機體運,歸款闕庭,朕錫以顯爵,班同親舊。昶弟湘東王進不能扶危定傾,
 退不能降身高謝,阻兵安忍,篡位自立,既無闔閭靜亂之功,而有無知悖禮之變,怠棄三正,慢易天常,覆敗之徵既兆,危亡之應已著。偽江州刺史晉安王復稱大號,自立一隅。荊郢二州刺史安陸臨海王劉子綏、子頊大擅威令,不相祗伏。徐州刺史彭城鎮主薛安都、青州刺史沈文秀、冀州刺史歷城鎮主崔道固等,皆彼之要籓,懼及禍難,擁眾獨據,各無定主。仰觀天象,俯察人謀,六軍燮伐之期,率土同軌之日。



 朕承休烈,屬當泰運,思播
 靈武,廓寧九服,豈可得臨萬乘之機,遘時來之遇,而不討其讎逆,振其艱患哉!今可分命諸軍,以行九伐。使持節征東大將軍安定王直勤伐伏玄、侍中尚書左僕射安西大將軍平北公直勤美晨、散騎常侍殿中尚書平北將軍山陽公呂羅漢,領隴右之眾五萬,沿漢而東,直指襄陽。使持節征南大將軍勃海王直勤天賜、侍中尚書令安東大將軍始平王直勤渴言侯、散騎常侍殿中尚書令安西將軍西陽王直勤蓋戶千,領幽、冀之眾七
 萬,濱海而南,直指東陽。使持節征南將軍京兆王直勤子、侍中司徒安南大將軍新建王獨孤侯尼須、散騎常侍西平公韓道人,領江、雍之眾八萬,出洛陽,直至壽陽。使持節征南大將軍宜陽王直勤新成、侍中太尉征東大將軍直勤駕頭拔、羽直征東將軍北平公拔敦及義陽王劉昶,領定、相之眾十萬,出濟、兗,直造彭城,與諸軍剋期同到,會于秣陵。納昶反國,定其社稷,使荊、陽沾德義之風,江、漢被來蘇之惠。邊疆將吏,不得因宋衰亂,有
 所侵損,以傷我國家存救之義。主者明宣所部,咸使聞知,稱朕意焉。



 既而晉安王子勛事平,太宗遣張永、沈攸之北討,薛安都大懼,遣使引虜。虜遣萬騎救之,永、攸之敗退;虜攻青、冀二州,並剋,執沈文秀、崔道固。又下書:朕承天序,臨御兆民,思闡皇風,以隆治道。而荊吳僭傲,跨歭一方,天降其殃,以罰有罪,篡戮發於蕭墻,毒害嬰於群庶。徐州刺史薛安都、司州刺史常珍奇,深體逆順,歸誠獻款。遭難已久,饑饉薦臻,或以糊口之功,私力竊盜;
 或不識王命,藏竄山藪;或為囚徒,先被執繫,元元之命,甚可哀愍。其曲赦淮北三州之民,自天安二年正月三十日壬寅昧爽以前,諸犯死罪以下,繫囚見徒,一切原遣。唯子殺父母,孫殺祖父母,弟殺兄,妻殺夫,奴殺主,不從赦例。若亡命山澤,百日不首,復其初罪。



 今陽春之初,東作方興,三州之民,各安其業,以就農桑。有饑窮不自存,通其市糶之路,鎮統之主,勤加慰納,遵用輕典,以蒞新化。若綏導失中,令民逃亡,加罪無縱。其普宣下,咸使
 聞知朕意焉。



 此後虜復和親,信餉歲至,朝庭亦厚相報答。泰豫元年,虜狹石鎮主白虎公、安陽鎮主莫索公、貞陽鎮主鵝落生、襄陽王桓天生等,引山蠻馬步二萬餘人,攻圍義陽縣義陽戍。司州刺史王贍遣從弟司空行參軍思遠、撫軍行參軍王叔瑜擊大破之,虜退走。



 自索虜破慕容,蠻馬二萬餘人攻圍義陽,據有中國,而芮芮虜有其故地,蓋漢世匈奴之北庭也。芮芮一號大檀,又號檀檀,亦匈奴別種。自西路通京師,三萬餘里。僭稱大
 號,部眾殷強,歲時遣使詣京師,與中國亢禮,西域諸國焉耆、鄯善、龜茲、姑墨東道諸國,並役屬之。無城郭,逐水草畜牧,以氈帳為居,隨所遷徙。



 其土地深山則當夏積雪,平地則極望數千里,野無青草。地氣寒涼,馬牛齕枯啖雪,自然肥健。國政疏簡,不識文書,刻木以記事,其後漸知書契,至今頗有學者。去北海千餘里,與丁零相接。常南擊索虜,世為仇讎,故朝庭每羈縻之。其東有槃盤國、趙昌國,渡流沙萬里,又有粟特國。太祖世,並奉表貢
 獻。粟特大明中遣使獻生獅子、火浣布、汗血馬,道中遇寇,失之。



 史臣曰:久矣,匈奴之與中國並也。自漢氏以前,綿跨年世,紛梗外區,驚震中宇。周無上算,漢收下策。魏代分離,種落遷散,數十年間,外郡無風塵之警,邊城早開晚閉,胡馬不敢南臨。至于晉始,奸黠漸著,密邇畿封,窺候疆場,俘民略畜者,無歲月而闕焉。元康以後,《風雅》雕喪,五胡遞襲,翦覆諸華。及涉珪以鐵馬長驅,席卷趙、魏,負其
 眾力,遂與上國爭衡矣。



 高祖宏圖盛略,欲以苞括宇宙為念,逮于懸旗清洛,飲馬長涇,北狄恤銳挫鋒,閉重嶮而自固。于時戎車外動,王命相屬,裳冕委蛇,軺軒繼路,舊老懷思古之情,行人或為之殞涕。自是關、河響動,表裏寧壹。宮車甫晏,戎心外駭,覆我牢、滑,翦我伊、瀍,是以太祖忿之,開定司、兗,而兵無勝略,棄師隕眾,委甲橫原,捐州亙水,荊、吳銳卒,逸氣未攄,偏城孤將,銜冤就虜,遂蹙境延寇,僅保清東。



 自是兵摧勢弱,邊隙稍廣,壯騎陵
 突,鳴鏑日至,芻牧年傷,禾麥歲犯。小則囚虜吏民,大則俘執長守,羽書繼塗,奔命相屬,青、徐、兗、冀之間蕭然矣。而自木末以來,並有賢才狡算,妙識兵權,深通戰術,屬鞬凌厲,氣冠百夫,故能威服華甸,志雄群虜。至於狸伐篡偽,彌煽凶威,英圖武略,事駕前古,雖冒頓之鷙勇,檀石之驍強,不能及也。遂西吞河右,東舉龍碣,總括戎荒,地兼萬里。雖裂土分區,不及魏、晉,而華氓戎落,眾力兼倍。至乃連騎百萬,南向而斥神華,胡旆映江,穹帳遵渚,
 京邑荷簷,士女喧惶。天子內鎮群心,外御群寇,役竭民徭,費殫府實,舉天下以攘之,而力猶未足也。既而虜縱歸師,殲累邦邑,剪我淮州,俘我江縣,喋喋黔首,跼高天,蹐厚地,而無所控告。強者為轉屍,弱者為繫虜,自江、淮至于清、濟,戶口數十萬,自免湖澤者,百不一焉。村井空荒,無復鳴雞吠犬。



 時歲唯暮春,桑麥始茂,故老遺氓,還號舊落,桓山之響,未足稱哀。六州蕩然,無復餘蔓殘構,至於乳燕赴時,銜泥靡託,一枝之間,連窠十數,春雨裁
 至,增巢已傾。雖事舛吳宮,而殲亡匪異,甚矣哉,覆敗之至於此也。



 太祖懲禍未深,復興外略,頓兵堅城,棄甲河上,是我有再敗,敵有三勝也。



 自此以後,通互市,納和親,而侵疆軼戍,于歲連屬。逮泰始構紛,邊將外叛,致夷引寇,亡我四州。高祖劬勞日昃,思一區宇,旍旗卷舒,僅而後克。後主守文,刑德不樹,一舉而棄司、兗,再舉而喪徐方,華服蕭條,鞠為茂草,豈直天時,抑由人事。夫地勢有便習,用兵有短長。胡負駿足,而平原悉車騎之地;南習
 水斗,江湖固舟楫之鄉。代馬胡駒,出自冀北;梗柟豫章,植乎中土,蓋天地所以分區域也。若謂氈裘之民,可以決勝於荊、越,必不可矣;而曰樓船之夫,可以爭鋒於燕、冀,豈或可乎!虞詡所謂「走不逐飛」,蓋以我徒而彼騎也。因此而推勝負,殆可以一言蔽之。



\end{pinyinscope}