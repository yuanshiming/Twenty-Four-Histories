\article{卷二十一志第十一 樂三}

\begin{pinyinscope}

 《但歌》四曲,出自漢世。無弦節,作伎,最先一人倡,三人和。魏武帝尤好之。時有宋容華者,清徹好聲,善唱此曲,當時特妙。自晉以來,不復傳,遂絕。



 《相和》,漢舊歌也。絲竹更相和,執節者歌。本
 一部,魏明帝分為二,更遞夜宿。



 本十七曲,硃生、宋識、列和等復合之為十三曲。



 《相和》《駕六龍》、《氣出倡》,武帝詞:駕六龍乘風而行,行四海外。路下之八邦,歷登高山,臨溪谷,乘雲而行,行四海外,東到泰山。仙人玉女,下來翱游,驂駕六龍,飲玉漿,河水盡,不東流。



 解愁腹,飲玉漿。奉持行,東到蓬萊山。上至天之門。玉闕下,引見得
 入,赤松相對,四面顧望,視正焜煌。開王心正興,其氣百道至,傳告無窮。閉其口,但當愛氣,壽萬年。東到海,與天連。神仙之道,出窈入冥。常當專之,心恬心詹無所愒欲,閉門坐自守,天與期
 氣。願得神之人,乘駕雲車,驂駕白鹿,上到天之門,來賜神之藥。跪受之,敬神齊。當
 如
 此,
 道自來。



 華陰山,自以為大,高百丈,浮雲為之蓋。仙人欲來,出隨風,列之雨。吹我洞簫鼓瑟琴,何誾誾,酒與歌戲。今日相樂誠為樂,玉女起,起儛移數時。鼓吹一何嘈嘈,從西北來時,
 仙道多駕煙,乘雲駕龍,鬱何蓩々。遨游八極,乃到昆侖之山,西王母側。神仙金止玉亭,來者為誰?赤松王喬,乃德旋之門。樂共飲食到黃昏,多駕合坐,萬歲長宜子孫。



 游君山,甚為真,磪磈砟硌,爾自為神。乃到王母臺,金階玉為堂,芝草生殿旁。東西廂,客滿堂。主人當行觴,坐者長壽遽何央。長樂甫始宜孫子,常願主人增年,與天相守。



 《厥初生》、《精列》,武帝詞:
 厥初生,造化之陶物,莫不有終期。莫不有終期,聖賢不能免,何為懷此憂。



 願螭龍之駕,思想昆侖居。思想昆侖居,見期於迂怪,志意在蓬萊。志意在蓬萊,周孔聖徂落,會稽以墳丘。會稽以墳丘,陶陶誰能度,君子以弗憂。年之暮,奈何,過時時來微。



 《江南可採蓮》、《江南》,古祠:江南可採蓮,蓮葉何田田。魚戲蓮葉間,魚戲蓮葉東,魚戲蓮葉西,魚戲蓮葉南,魚戲蓮葉北。



 《
 天地間》、《度關山》,武帝詞:天地間,人為貴。立君牧民,為之軌則。車轍馬迹,經緯四極。絀陟幽明,黎庶繁息。於鑠賢聖,總統邦域,封建五爵,井田刑獄。有燔丹書,無普赦贖。皋陶《甫刑》,何有失職。嗟哉後世,改制易律。勞民為君,役賦其力。舜漆食器,畔者十國;不及唐堯,採椽不斫。世歎伯夷,欲以厲俗。侈惡之大,儉為恭德。許由推讓,豈有訟曲。兼愛尚同,疏者為戚。



 《東光乎》、《東光乎》,古詞:
 東光乎!倉梧何不乎!倉梧多腐粟,無益諸軍糧。諸軍游蕩子,蚤行多悲傷。



 《登山有遠望》、《十五》,文帝詞:登山而遠望,谿谷多所有。楩柟千餘尺,眾草之盛茂。華葉耀人目。五色難可紀。雉雊山雞鳴,虎嘯谷風起。號羆當我道,狂顧動牙齒。



 《惟漢二十二世》、《薤露》,武帝詞:惟漢二十二世,所任誠不良。沐猴而冠帶,智小而謀強。
 猶豫不敢斷,因狩執君王。白虹為貫日,己亦先受殃。賊臣持國柄,殺主滅宇京。蕩覆帝基業,宗廟以燔喪。播越西遷移,號泣而且行。瞻彼洛城郭,微子為哀傷。



 《關東有義士》、《蒿里行》,武帝詞:關東有義士,興兵討群凶。初期會孟津,乃心在咸陽。軍合力不齊,躊躇而雁行。勢利使人爭,嗣還自相戕。淮南弟稱號,刻璽於北方。鎧甲生蟣虱,萬姓以死亡。白骨露於野,千里無雞鳴。生民百遺一,念之絕人腸。



 《
 對酒歌太平時》、《對酒》,武帝詞:對酒歌,太平時,吏不呼門。王者賢且明,宰相股肱皆忠良,咸禮讓,民無所爭訟。三年耕有九年儲,倉穀滿盈,斑白不負戴。雨澤如此,五穀用成。卻走馬以糞其土田。爵公侯伯子男,咸愛其民,以黜陟幽明,子養有若父與兄。犯禮法,輕重隨其刑。路無拾遺之私,囹圄空虛,冬節不斷人。耄耋皆得以壽終,恩德廣及草木昆蟲。



 《雞鳴高樹顛》、《雞鳴》,古詞:
 雞鳴高樹顛,狗吠深宮中。蕩子何所之,天下方太平。刑法非有貸,柔協正亂名。黃金為君門,璧玉為軒闌堂。上有雙尊酒,作使邯鄲倡。劉玉碧青甓,後出郭門王。舍後有方池,池中雙鴛鴦。鴛鴦七十二,羅列自成行。鳴聲何啾啾,聞我殿東廂。兄弟四五人,皆為侍中郎。五日一時來,觀者滿道傍。黃金絡馬頭,何煌煌。桃生露井上,李樹生桃傍,蟲來齧桃根,李樹代桃僵。樹木身相代,兄弟還相忘!



 《
 烏生八九子》、《烏生》,古詞:烏生八九子,端坐秦氏桂樹間。唶我秦氏,家有游遨蕩子,工用睢陽強蘇合彈。



 左手持強彈,兩丸出入烏東西。唶我一丸即發中烏身,烏死魂魄飛揚上天。阿母生烏子時,乃在南山巖石間。唶我人民安知烏子處,蹊徑窈窕安從通。白鹿乃在上林西苑中,射工尚復得白鹿脯哺。唶我黃鵠摩天極高飛,後宮尚復得烹煮之。鯉魚乃在洛水深淵中,釣鉤尚得鯉魚口。唶我人民生各各有
 壽命,死生何須復道前後。



 《平陵東》、《平陵》,古詞:平陵東,松柏桐,不知何人劫義公。劫義公在高堂下,交錢百萬兩走馬。兩走馬,亦誠難,顧見追吏心中惻。心中惻,血出漉,歸告我家賣黃犢。


《棄故鄉》
 \gezhu{
  亦在瑟調《東西門行》}
 《陌上桑》,文帝詞:棄故鄉,離室宅,遠從軍旅萬里客。披荊棘,求阡陌,側足獨窘步,路局笮。



 虎豹嗥動,雞驚,禽失群,鳴相索。登南山,
 奈何蹈槃石,樹木叢生鬱差錯。寢蒿草,蔭松柏,涕泣雨面霑枕席。伴旅單,稍稍日零落,惆悵竊自憐,相痛惜。



 《今有人》、《陌上桑》,《楚詞》鈔:今有人,山之阿,被服薜荔帶女蘿。既含睇,又宜笑,子戀慕予善窈窕。乘赤豹,從文貍,辛夷車駕結桂旗。被石蘭,帶杜衡,折芳拔荃遺所思。處幽室,終不見,天路險艱獨後來。表獨立,山之上,雲何容容而在下。杳冥冥,羌晝晦,東風飄濆神靈雨。風瑟瑟,木搜搜,思念公子徒以憂。



 《
 駕虹霓》、《陌上桑》,武帝詞:駕虹霓,乘赤雲,登彼九疑歷玉門。濟天漢,至昆侖,見西王母,謁東君。交赤松,及羨門,受要祕道愛精神。食芝英,飲醴泉,柱杖桂枝佩秋蘭。絕人事,游渾元,若疾風游歘飄飄。景未移,行數千,壽如南山不忘愆。



 清商三調歌詩,荀勖撰,舊詞施用者,平調。


《周西》、《短歌行》,武帝詞
 \gezhu{
  六
  解}
 :周西伯昌,懷此聖德,參分天下,而有其二。脩奉貢獻,臣節不墜。崇侯讒之,是以拘繫。
 \gezhu{
  一解}
 後見赦原,賜之斧鉞,得使征伐。為仲尼所稱,達及德行,猶奉事殷,論敘其美。
 \gezhu{
  二解}
 齊桓之功,為霸之首,九合諸侯,一匡天下。一匡天下,不以兵車。正而不譎,其德傳稱。
 \gezhu{
  三解}
 孔子所歎,并稱夷吾,民受其恩。


賜與廟胙,命無下拜。小白不敢爾,天威在顏咫尺。
 \gezhu{
  四解}
 晉文亦霸,躬奉天王。


受賜珪瓚、秬鬯雕弓、盧弓、矢千,虎賁三百人。
 \gezhu{
  五解}
 威服諸侯,師之者尊,八方聞之,名亞
 齊桓。河陽之會,詐稱周王,是以其名紛葩。
 \gezhu{
  六解}


《秋風》、《燕歌行》,文帝詞
 \gezhu{
  七解}
 :秋風蕭瑟天氣涼,草木搖落露為霜。
 \gezhu{
  一解}
 群燕辭歸鵠南翔,念君客遊多思腸。
 \gezhu{
  二解}
 慊慊思歸戀故鄉,君何淹留寄它方。
 \gezhu{
  三解}
 賤妾煢煢守空房,憂來思君不敢忘。
 \gezhu{
  四解}
 不覺淚下霑衣裳,援瑟鳴弦發清商。
 \gezhu{
  五解}
 短歌微吟不能長,明月皎皎照我床。
 \gezhu{
  六解}
 星漢西流夜未央,牽年織女遙相望,爾獨何辜限河梁。
 \gezhu{
  七
  解}


《仰瞻》、《短歌行》,文帝詞
 \gezhu{
  六解}
 :仰瞻帷幕,俯察幾筵。其物如故,其人不存。
 \gezhu{
  一解}
 神靈倏忽,棄我遐遷。


靡瞻靡恃,泣涕連連。
 \gezhu{
  二解}
 呦呦游鹿,銜草鳴麑。翩翩飛鳥,挾子巢棲。
 \gezhu{
  三解}
 我獨孤煢,懷此百離。憂心孔疚,莫我能知。
 \gezhu{
  四解}
 人亦有言,憂令人老。


嗟我白髮,生一何早。
 \gezhu{
  五解}
 長吟永歎,懷我聖考。曰仁者壽,胡不是保。
 \gezhu{
  六解}


《別日》、《燕歌行》,文帝詞
 \gezhu{
  六解}
 :別日何易會日難,山川悠遠路漫漫。
 \gezhu{
  一解}
 鬱陶思君未敢
 言,寄書浮雲往不還。
 \gezhu{
  二解}
 涕零雨面毀形顏,誰能懷憂獨不歎。
 \gezhu{
  三解}
 耿耿伏枕不能眠,披衣出戶步東西。
 \gezhu{
  四解}
 展詩清歌聊自寬,樂往哀來摧心肝。悲風清厲秋氣寒,羅帷徐動經秦軒。
 \gezhu{
  五解}
 仰戴星月觀雲間,飛鳥晨鳴,聲氣可憐,留連顧懷不自存。


\gezhu{
  六解}


《對酒》、《短歌行》,武帝詞
 \gezhu{
  六解}
 :對酒當歌,人生幾何!譬如朝露,去日苦多。
 \gezhu{
  一解}
 慨當以慷,憂思難忘。


以何解愁,唯有「杜康」。
 \gezhu{
  二解}
 青青子衿,悠悠我心。
 但為君故,沈吟至今。


\gezhu{
  三解}
 明明如月,何時可掇。憂從中來,不可斷絕。
 \gezhu{
  四解}
 呦呦鹿鳴,食野之蘋。我有嘉賓,鼓瑟吹笙。
 \gezhu{
  五解}
 山不厭高,水不厭深。周公吐哺,天下歸心。


\gezhu{
  六解}



 清調《晨上》、《秋胡行》,武帝詞:晨上散關山,此道當何難!晨上散關山,此道當何難!牛頓不起,車墮谷間。


坐盤石之上,彈五弦之琴,作為清角韻,意中述煩。歌以言志,晨上散關山。
 \gezhu{
  一解}
 有何三老公,卒
 來在我傍。有何三老公,卒來在我傍。員掩被裘,似非恒人。


謂卿云何,困苦以自怨,徨徨所欲,來到此間。歌以言志,有何三老公。
 \gezhu{
  二解}


我居昆侖山,所謂者真人。我居昆侖山,所謂者真人。道深有可得。名山歷觀,遨游八極。枕石漱流飲泉。沈吟不決,遂上升天。歌以言志,我居昆侖山。
 \gezhu{
  三解}



 去去不可追,長恨相牽攀。去去不可追,長恨相牽攀。夜夜安得寐,惆悵以自憐。


正而不譎,辭賦依因。經傳所過,西來所傳。歌以言志,去去不可追。
 \gezhu{
  四解。又本:晨‖上‖散‖關‖山‖,
  此‖道‖當‖何‖難。有‖何‖三‖老‖公,卒‖來‖在‖我‖傍‖。我‖居‖我‖昆‖侖‖山‖,所‖謂‖真‖人‖,去‖不‖可‖追‖,長‖相‖牽‖攀‖。}


《北上》、《苦寒行》,武帝詞
 \gezhu{
  六解}
 :北上太‖行‖山‖,艱‖哉‖何‖巍‖巍‖。羊腸阪詰屈,車輪為之摧。
 \gezhu{
  一解}
 樹木何蕭‖瑟‖,北‖風‖聲‖正‖悲‖。熊羆對我蹲,虎豹夾道啼。
 \gezhu{
  二解}


谿谷少‖人‖民‖,雪‖落‖何‖霏‖霏‖。延頸長歎息,遠行多所懷。
 \gezhu{
  三解}


我心何‖佛‖鬱‖,思‖欲‖一‖東‖歸‖。水深橋
 梁絕,中道正裴回。
 \gezhu{
  四解}


迷惑失‖徑‖路‖,暝‖無‖所‖宿‖棲‖。行行日以遠,人馬同時飢。
 \gezhu{
  五解}


擔‖囊‖行‖取‖薪‖,斧‖冰‖持‖作‖糜‖。悲彼東山詩,悠悠使我哀。
 \gezhu{
  六解}


《願登》、《秋胡行》,武帝詞
 \gezhu{
  五解}
 :願‖登‖泰‖華‖山‖,神‖人‖共‖遠‖游‖。經歷昆侖山,到蓬萊。飄濆八極,與神人俱。思得神藥,萬歲為期。歌以言志,願登泰華山。
 \gezhu{
  一解}
 天‖地‖何‖長‖久
 ‖,人‖道‖居‖之‖短‖。世言伯陽,殊不知老,赤松王喬,亦云得道。得之未聞,庶以壽考。歌以言志,天地何長久!
 \gezhu{
  二解}
 明‖明‖日‖月‖光‖,何‖所‖不‖光‖昭‖。二儀合聖化,貴者獨人不。萬國率土,莫非王臣。


仁義為名,禮樂為榮。歌以言志,明明日月光。
 \gezhu{
  三解}
 四‖時‖更‖逝‖去‖,晝‖夜‖以‖成‖歲‖。大人先天,而天弗違。不戚年往,世憂不治。存亡有命,慮之為蚩。歌以言志,四時更逝去。
 \gezhu{
  四解}
 戚‖戚‖欲‖何‖念‖,歡‖笑
 ‖意‖所‖之‖。盛壯智惠,殊不再來。愛時進趣,將以惠誰。泛泛放逸,亦同何為。


歌以言志,戚戚欲何念?
 \gezhu{
  五解}


《上謁》、《董桃行》,古詞
 \gezhu{
  五解}
 :吾欲上謁從高山,山頭危嶮大難。遙望五嶽端,黃金為闕,班璘。但見芝草,葉落紛紛。
 \gezhu{
  一解}
 百鳥集,來如煙。山獸紛綸,麟辟邪其端。鵾雞聲鳴,但見山獸援戲相拘攀。
 \gezhu{
  二解}
 小復前行玉堂,未心懷流還。傳教出門來,門外人何求?


所言欲從聖道,求一得命延。
 \gezhu{
  三解}
 教敕凡吏受言,采取神藥
 若木端。白兔長跪搗藥蝦蟆丸,奉上陛下一玉柈,服此藥可得即仙。
 \gezhu{
  四解}
 服爾神藥,無不歡喜。


陛下長生老壽,四面肅肅稽首,天神擁護左右,陛下長與天相保守。
 \gezhu{
  五解}


《蒲生》、《塘上行》,武帝詞
 \gezhu{
  五解}
 :蒲‖生‖我‖池‖中‖,其葉何離離。傍能行儀儀,莫能縷自知。眾口鑠黃金,使君生別離。
 \gezhu{
  一解}
 念‖君‖去‖我‖時‖,獨愁常苦悲。想見君顏色,感結傷心脾。今悉夜夜愁不寐。
 \gezhu{
  二解}
 莫‖用‖豪‖賢‖故‖,棄捐素所愛;
 莫用魚肉貴,棄捐蔥與薤;莫用麻枲賤,棄捐菅與蒯。
 \gezhu{
  三解}
 倍‖恩‖者‖苦‖栝‖,蹶船常苦沒。教君安息定,慎莫致倉卒。念與君一共離別,亦當何時共坐復相對。


\gezhu{
  四解}
 出‖亦‖復‖苦‖愁‖,入亦復苦愁。邊地多悲風,樹木何蕭蕭。今日樂相樂,延年壽千秋。
 \gezhu{
  五解}


《悠悠》、《苦寒行》,明帝詞
 \gezhu{
  五解}
 :悠‖悠‖發‖洛‖都‖,茾‖我‖征‖東‖行‖。征行彌二旬,屯吹龍陂城。
 \gezhu{
  一解}
 顧觀故‖壘‖處‖,皇‖祖
 ‖之‖所‖營‖。屋室若平昔,棟宇無邪傾。
 \gezhu{
  二解}
 奈何‖我‖皇‖祖‖,潛‖德‖隱‖聖‖形‖。雖沒而不朽,書貴垂休名。
 \gezhu{
  三解}
 光光我‖皇‖祖‖,軒‖耀‖同‖其‖榮‖。遺化布四海,八表以肅清。
 \gezhu{
  四解}
 雖有吳‖蜀‖寇‖,春‖秋‖足‖耀‖兵。徒悲我皇祖,不永享百齡。賦詩以寫懷,伏軾淚霑纓。
 \gezhu{
  五解}


瑟調《朝日》、《善哉行》,文帝詞。
 \gezhu{
  五解}
 :朝日樂相樂,酣飲不知醉。悲弦激新聲,長笛吐清氣。
 \gezhu{
  一解}
 弦歌感人腸,四坐皆歡說。寥寥高堂上,涼風入我室。
 \gezhu{
  二解}
 持滿如不盈,有得者能卒。君子多苦心,所愁不但一。
 \gezhu{
  三解}
 慊慊下白屋,吐握不可失。眾賓飽滿歸,主人苦不悉。
 \gezhu{
  四解}
 比翼翔雲漢,羅者安所羈。沖靜得自然,榮華何足為。
 \gezhu{
  五解}


《上山》、《善哉行》,文帝詞
 \gezhu{
  六解}
 :上山采薇,薄莫苦饑。溪谷多風,霜露沾衣。
 \gezhu{
  一解}
 野雉群雊,猿猴相追。


還望故鄉,鬱何壘壘,
 \gezhu{
  二解}
 高山有崖,林木有支。
 憂來無方,人莫之知。
 \gezhu{
  三解}
 人生若寄,多憂何為。今我不樂,歲月其馳。
 \gezhu{
  四解}
 湯湯川流,中有行舟。


隨波轉薄,有似客游。
 \gezhu{
  五解}
 策我良馬,被我輕裘。載馳載驅,聊以忘憂。
 \gezhu{
  六解}


《朝游》、《善哉行》,文帝詞
 \gezhu{
  五解}
 :朝游高臺觀,夕宴華池陰。大酋奉甘醪,狩人獻嘉禽。
 \gezhu{
  一解}
 齊倡發東舞,秦箏奏西音。有客從南來,為我彈清琴。
 \gezhu{
  二解}
 五音紛繁會,拊者激微吟。淫魚乘波聽,踴躍自浮沉。
 \gezhu{
  三解}
 飛鳥翻翔舞,悲鳴集北林。樂極哀情來,憀亮摧肝心。
 \gezhu{
  四解} 清角豈不妙,德薄所不任。大哉子野言,弭弦且自禁。
 \gezhu{
  五解}


《古公》、《善哉行》,武帝詞
 \gezhu{
  七解}
 :古公亶甫,積德垂仁。思弘一道,哲王於幽。
 \gezhu{
  一解}
 太伯仲雍,王德之仁。


行施百世,斷髮文身。
 \gezhu{
  二解}
 伯夷叔齊,古之遺賢。讓國不用,餓殂首山。
 \gezhu{
  三解}
 智哉山甫,相彼宣王。何用杜伯,累我聖賢。
 \gezhu{
  四解}
 齊桓之霸,賴得仲父。


後任豎刁,蟲流出戶。
 \gezhu{
  五解}
 晏子平仲,積德兼仁。與世沈德,未必思命。
 \gezhu{
  六解}
 仲尼之世,王國為君。隨制飲酒,揚波使官。
 \gezhu{
  七解}


《
 自惜》、《善哉行》,武帝詞
 \gezhu{
  六解}
 :自惜身薄祜,夙賤罹孤苦。既無三徙教,不聞過庭語。
 \gezhu{
  一解}
 其窮如抽裂,自以思所怙。雖懷一介志,是時其能與。
 \gezhu{
  二解}
 守窮者貧賤,惋歎淚如雨。泣涕於悲夫,乞活安能睹。
 \gezhu{
  三解}
 我願於天窮,琅邪傾側左。雖欲竭忠誠,欣公歸其楚。
 \gezhu{
  四解}
 快人曰為歎,抱情不得敘。顯行天教人,誰知莫不緒。
 \gezhu{
  五解}
 我願何時隨,此歎亦難處。今我將何照於光耀,釋銜不如雨。
 \gezhu{
  六解}


《
 我徂》、《善哉行》,明帝詞
 \gezhu{
  八解}
 :我徂我征,伐彼蠻虜。練師簡卒,爰正其旅。
 \gezhu{
  一解}
 輕舟竟川,初鴻依浦。


桓桓猛毅,如羆如虎。
 \gezhu{
  二解}
 發砲若雷,吐氣成雨。旄旍指麾,進退應矩。
 \gezhu{
  三解}
 百馬齊轡,御由造父。休休六軍,咸同斯武。
 \gezhu{
  四解}
 兼塗星邁,亮茲行阻。


行行日遠,西背京許。
 \gezhu{
  五解}
 游弗淹旬,遂屆揚土。奔寇震懼,莫敢當御。
 \gezhu{
  六解}
 虎臣列將,怫鬱充怒。淮泗肅清,奮揚微所。
 \gezhu{
  七解}
 運德耀威,惟鎮惟撫。


反旆言歸,告入皇祖。
 \gezhu{
  八解}


《
 赫赫》、《善哉行》,明帝詞
 \gezhu{
  四解}
 :赫赫大魏,王師徂征。冒暑討亂,振耀威靈。
 \gezhu{
  一解}
 汎舟黃河,隨波潺湲。


通渠回越,行路綿綿。
 \gezhu{
  二解}
 采旄蔽日,旗旒翳天。淫魚瀺灂,游戲深淵。
 \gezhu{
  三解}
 唯塘泊,從如流。不為單,握揚楚。心惆悵,歌《采薇》。心綿綿,在淮肥。


願君速捷蚤旋歸。
 \gezhu{
  四解}


《來日》、《善哉行》,古詞
 \gezhu{
  六解}
 :來日大難,口燥脣乾。今日相樂,皆當喜歡。
 \gezhu{
  一解}
 經歷名山,芝草翻翻。


仙人王喬,奉藥一丸。
 \gezhu{
  二解}
 自惜袖短,內手知寒。
 慚無靈輒,以報趙宣。
 \gezhu{
  三解}
 月沒參橫,北斗闌干。親交在門,饑不及餐。
 \gezhu{
  四解}
 歡日尚少,戚日苦多。


以何忘憂,彈箏酒歌。
 \gezhu{
  五解}
 淮南八公,要道不煩。參駕六龍,游戲雲端。
 \gezhu{
  六解}


大曲《東門》、《東門行》,古詞
 \gezhu{
  四解}
 :出東門,不願歸;來入門,悵欲悲。盎中無斗儲,還視桁上無縣衣。
 \gezhu{
  一解}


拔劍出門去,兒女牽衣啼。它家但願富貴,賤妾與君共食甫糜。
 \gezhu{
  二解}
 共鋪糜,上用倉浪天故,下為黃口小
 兒。今時清廉,難犯教言,君復自愛莫為非。
 \gezhu{
  三解}


今時清廉,難犯教言,君復自愛莫為非。行!吾去為遲,平慎行,望吾歸。
 \gezhu{
  四解}


《西山》、《折楊柳行》,文帝詞
 \gezhu{
  四解}
 :西山一何高,高高殊無極。上有兩仙僮,不飲亦不食。與我一丸藥,光耀有五色。
 \gezhu{
  一解}
 服藥四五日,身體生羽翼。輕舉乘浮雲,倏忽行萬億。流覽觀四海,芒芒非所識。
 \gezhu{
  二解}
 彭祖稱七百,悠悠安可原。老聃適西戎,于今竟不還。王喬
 假虛詞,赤松垂空言。
 \gezhu{
  三解}
 達人識真偽,愚夫好妄傳。追念往古事,憒憒千萬端。百家多迂怪,聖道我所觀。
 \gezhu{
  四解}


《羅敷》、《艷歌羅敷行》,古詞
 \gezhu{
  三解}
 :日出東南隅,照我秦氏樓。秦氏有好女,自名為羅敷。羅敷喜蠶桑,采桑城南隅。青絲為籠係,桂枝為籠鉤。頭上倭墮發,耳中明月珠。緗綺為下群,紫綺為上襦。行者見羅敷,下擔捋髭須。少年見羅敷,脫帽著帩頭。耕者忘其犁,鋤者忘其鋤。來歸相怒怨,但坐觀羅敷。
 \gezhu{
  一解}
 使君從南
 來,五馬立歭躇。使君遣吏往,問是誰家姝?秦氏有好女,自名為羅敷。羅敷年幾何?二十尚不足,十五頗有餘。



 使君謝羅敷,寧可共載不?羅敷前置詞,使君一何愚!使君自有婦,羅敷自有夫。


\gezhu{
  二解}
 東方千餘騎,夫婿居上頭。何用識夫婿?白馬從驪駒。青絲繫馬尾,黃金絡馬頭。腰中鹿盧劍,可直千萬餘。十五府小史,二十朝大夫,三十侍中郎,四十專城居。為人潔白晳,鬑々頗有須。盈盈公府步,冉冉府中趨。坐中數千人,皆言夫婿殊。
 \gezhu{
  三解}
 \gezhu{
  前有艷詞曲,後有趨。}


《
 西門》、《西門行》,古詞
 \gezhu{
  六解}
 :出西門,步念之。今日不作樂,當待何時。
 \gezhu{
  一解}
 夫為樂,為樂當及時。何能坐愁怫鬱,當復來茲。
 \gezhu{
  二解}
 飲醇酒,炙肥牛。請呼心所歡,可用解愁憂。


\gezhu{
  三解}
 人生不滿百,常懷千歲憂。晝短而夜長,何不秉燭游。
 \gezhu{
  四解}
 自‖非‖仙‖人‖王‖子‖喬‖,計‖會‖壽‖命‖難‖‖與‖期‖。
 \gezhu{
  五解}
 人壽非金石,年命安可期;貪財愛惜費,但為後世嗤。
 \gezhu{
  六解}
 。
 \gezhu{
  一本「燭游」後「行去之,如雲除,弊車羸馬為自推」,無「自非」以下四十八字。}


《
 默默》、《折楊柳行》,古詞
 \gezhu{
  四解}
 :默默施行違,厥罰隨事來,末喜殺龍逢,桀放於鳴條。
 \gezhu{
  一解}
 祖伊言不用,紂頭縣白旄。指鹿用為馬,胡亥以喪軀。
 \gezhu{
  二解}
 夫差臨命絕,乃云負子胥。戎王納女樂,以亡其由餘。璧馬禍及虢,二國俱為墟。
 \gezhu{
  三解}
 三夫成市虎,慈母投杼趨。卞和之刖足,接予歸草廬。
 \gezhu{
  四解}


《園桃》、《煌煌京洛行》,文帝詞
 \gezhu{
  五解}
 :夭夭園桃,無子空長。虛美難假,偏輪不行。
 \gezhu{
  一解}
 淮陰五刑,
 鳥得弓藏。


保身全名,獨有子房。大憤不收,褒衣無帶;多言寡誠,祗令事敗。
 \gezhu{
  二解}
 蘇秦之說,六國以亡。傾側賣主,車裂固當。賢矣陳軫,忠而有謀,楚懷不從,禍卒不救。
 \gezhu{
  三解}
 禍夫吳起,智小謀大,西河何健,伏尸何劣。
 \gezhu{
  四解}
 嗟彼郭生,古之雅人,智矣燕昭,可謂得臣。峨峨仲連,齊之高士;北辭千金,東蹈滄海。
 \gezhu{
  五解}


《白鵠》、《艷歌何嘗》
 \gezhu{
  一曰《飛鵠行》}
 ,古詞
 \gezhu{
  四解}
 :飛來雙白鵠,乃從西北來。十十五五,羅列成行。
 \gezhu{
  一解}
 妻卒
 被病,行不能相隨。五里一反顧,六里一裴回。
 \gezhu{
  二解}
 吾欲銜汝去,口噤不能開;吾欲負汝去,毛羽何摧頹。
 \gezhu{
  三解}
 樂哉新相知,憂來生別離。躇躊顧群侶,淚下不自知。
 \gezhu{
  四解}
 念與君離別,氣結不能言。各各重自愛,道遠歸還難。妾當守空房,閉門下重關。若生當相見,亡者會黃泉。今日樂相樂,延年萬歲期。
 \gezhu{
  「念與」下為趨曲,前有艷。}


《碣石》、《步出夏門行》,武帝詞
 \gezhu{
  四解}
 :雲行雨步,超越九江之皋,臨觀異同。心意懷游豫,不知
 當復何從。經過至我碣石,心惆悵我東海。
 \gezhu{
  《雲行》至此為艷。}
 東臨碣石,以觀滄海。水何淡淡,山島竦峙。樹木叢生,百草豐茂。秋風蕭瑟,洪濤湧起。日月之行,若出其中;星漢粲爛,若出其裏。幸甚至哉!歌以詠志。
 \gezhu{
  《觀滄海》,一解。}



 孟冬十月,北風裴回。天氣肅清,繁霜霏霏。鵾雞晨鳴,鴻雁南飛,鷙鳥潛藏,熊羆窟棲。錢褲停置,農收積場。逆旅整設,以通賈商。幸甚至哉!歌以詠志。


\gezhu{
  《冬十月》,二解。}



 鄉土不同,河朔隆寒。流澌浮漂,舟船行難。錐不入地,蘴裛深奧。水竭不流,冰
 堅可蹈。士隱者貧,勇俠輕非。心常歎怨,戚戚多悲。幸甚至哉!歌以詠志。


\gezhu{
  《河朔寒》,三解。}



 神龜雖壽,猶有竟時;騰蛇乘霧,終為土灰。驥老伏櫪,志在千里;烈士暮年,壯心不已。盈縮之期,不但在天;養怡之福,可得永年。幸甚至哉!歌以詠志。


\gezhu{
  《神龜雖壽》,四解。}


《何嘗》、《艷歌何嘗行》,古詞
 \gezhu{
  五解}
 :何嘗快獨無憂?但當飲醇酒,炙肥牛。
 \gezhu{
  一解}
 長兄為二千石,中兄被貂裘。


\gezhu{
  二解}
 小弟雖無官爵,鞍馬反反,往來王侯長
 者遊。
 \gezhu{
  三解}
 但當在王侯殿上,快獨摴蒲六博,對坐彈棋。
 \gezhu{
  四解}
 男兒居世,各當努力;蹙迫日暮,殊不久留。
 \gezhu{
  五解}
 少小相觸抵,寒苦常相隨,忿恚安足諍,吾中道與卿共別離。約身奉事君,禮節不可虧。上慚滄浪之天,下顧黃口小兒。奈何復老心皇皇,獨悲誰能知!


\gezhu{
  「少小」下為趨曲,前為艷。}


《置酒》、《野田黃雀行》,
 \gezhu{
  《空侯引》亦用此曲。}
 東阿王詞
 \gezhu{
  四解}
 :置酒高殿上,親交從我游。中廚辦豐膳,烹羊宰肥牛。秦箏何慷慨,齊瑟和且柔。
 \gezhu{
  一解}
 陽阿奏奇舞,京洛出名謳。樂
 飲過三爵,緩帶傾庶羞,主稱千金壽,賓奉萬年酬。
 \gezhu{
  二解}
 久要不可忘,薄終義所尤。謙謙君子德,磬折欲何求。盛時不再來,百年忽我遒。
 \gezhu{
  三解}
 驚風飄白日,光景馳西流。生存華屋處,零落歸山丘。先民誰不死,知命復何憂!
 \gezhu{
  四解}


《為樂》、《滿歌行》,古詞:
 \gezhu{
  四解}
 :為樂未幾時,遭世險𡾟,逢此百離;伶丁荼毒,愁懣難支。遙望辰極,天曉月移。憂來闐心,誰當我知。
 \gezhu{
  一解}
 戚戚多思慮,耿耿不寧。禍福無形,唯念古人,遜位躬耕。遂我所願,
 以茲自寧。自鄙山棲,守此一榮。
 \gezhu{
  二解}
 莫秋冽風起。


西蹈滄海,心不能安。攬衣起瞻夜,北斗闌幹。星漢照我,去去自無它。奉事二親,勞心可言。
 \gezhu{
  三解}
 窮達天所為,智者不愁,多為少憂。安貧樂正道,師彼莊周。


遺名者貴,子熙同戲。往者二賢,名垂千秋。
 \gezhu{
  四解}
 飲酒歌舞,不樂何須!善哉照觀日月,日月馳驅。轗軻世間,何有何無!貪財惜費,此一何愚!命如鑿石見火,居世竟能幾時?但當歡樂自娛,盡心極所熙怡。安善養君德性,百年保此期頤。


\gezhu{
  「飲酒」下為趨。}


《
 夏門》、《步出夏門行》,
 \gezhu{
  一曰《隴西行》}
 明帝詞
 \gezhu{
  二解}
 :步出夏門,東登首陽山。嗟哉夷叔,仲尼稱賢。君子退讓,小人爭先;惟斯二子,于今稱傳。林鐘受謝,節改時遷。日月不居,誰得久存。善哉殊復善,弦歌樂情。
 \gezhu{
  一解}
 商風夕起,悲彼秋蟬,變形易色,隨風東西。乃眷西顧,雲霧相連,丹霞蔽日,采虹帶天。弱水潺潺,落葉翩翩,孤禽失群,悲鳴其間。善哉殊復善,悲鳴在其間。
 \gezhu{
  二解}
 朝游清泠,日莫嗟歸。
 \gezhu{
  「朝游」上為艷。}
 蹙迫日莫,烏鵲南飛。繞樹三匝,何枝可依。卒逢
 風雨,樹折枝摧。雄來驚雌,雌獨愁棲。夜失群侶,悲鳴裴回。芃芃荊棘,葛生綿綿。感彼風人,惆悵自憐。月盈則沖,華不再繁;古來之說,嗟哉一言。
 \gezhu{
  「蹙迫」下為趨。}


《王者布大化》、《櫂歌行》,明帝詞
 \gezhu{
  五解}
 :王者布大化,配乾稽后祗。陽育則陰殺,晷景應度移。
 \gezhu{
  一解}
 文德以時振,武功伐不隨。重華儛干戚,有苗服從媯。
 \gezhu{
  二解}
 蠢爾吳蜀虜,馮江棲山阻。哀哀王士民,瞻仰靡依怙。
 \gezhu{
  三解}
 皇上悼愍斯,宿昔奮天怒。發我許昌宮,列舟于長浦。
 \gezhu{
  四解} 翌日乘波揚,棹歌悲且涼。大常拂白日,旗幟紛設張。
 \gezhu{
  五解}
 將抗旄與鉞,耀威於彼方。伐罪以弔民,清我東南疆。
 \gezhu{
  「將抗」下為趨。}


《洛陽行》、《雁門太守行》,古詞
 \gezhu{
  八解}
 :孝和帝在時,洛陽令王君,本自益州廣漢民,少行宦,學通五經論。
 \gezhu{
  一解}


明知法令,歷世衣冠。從溫補洛陽令,治行致賢,擁護百姓,子養萬民。
 \gezhu{
  二解}


外行猛政,內懷慈仁。文武備具,料民富貧,移惡子姓名,五篇著里端。
 \gezhu{
  三解}
 ,傷殺人,比
 伍同罪對門。禁鎦矛八尺,捕輕薄少年,加笞決罪,詣馬市論。
 \gezhu{
  四解}


無妄發賦,念在理冤,敕吏正獄,不得苛煩。財用錢三十,買繩禮竿。
 \gezhu{
  五解}
 賢哉賢哉!我縣王君。臣吏衣冠,奉事皇帝。功曹主簿,皆得其人。
 \gezhu{
  六解}
 臨部居職,不敢行恩。清身苦體,夙夜勞勤。治有能名,遠近所聞。
 \gezhu{
  七解}
 天年不遂,蚤就奄昏。為君作祠,安陽亭西。欲令後世,莫不稱傳。
 \gezhu{
  八解}


《白頭吟》、與《棹歌》同調,古詞
 \gezhu{
  五解}
 :晴如山上云,皎若雲間月。聞君有兩意,故來相決絕。
 \gezhu{
  一解} 平生共城中,何嘗斗酒會。今日斗酒會,明旦溝水頭。蹀踥御溝上,溝水東西流。
 \gezhu{
  二解}
 郭東亦有樵,郭西亦有樵。兩樵相推與,無親為誰驕?
 \gezhu{
  三解}
 淒淒重悽淒,嫁娶亦不啼;願得一心人,白頭不相離。
 \gezhu{
  四解}
 竹竿何嫋嫋,魚尾何離簁,男兒欲相知,何用錢刀為?如五馬啖萁,川上高士嬉。今日相對樂,延年萬歲期。
 \gezhu{
  五解}


\gezhu{
  一本云:詞曰上有「紫羅咄咄奈何」。}


楚調怨詩《明月》,東阿王詞
 \gezhu{
  七解}
 :明月照高樓,流光正裴回。上有愁思婦,悲歎有餘哀。
 \gezhu{
  一解}
 借問歎者誰?


自云客子妻。夫行踰十載,賤妾常獨棲。
 \gezhu{
  二解}
 念君過於渴,思君劇於饑。君為高山柏,妾為濁水泥。
 \gezhu{
  三解}
 北風行蕭蕭,烈烈入吾耳。心中念故人,淚墮不能止。
 \gezhu{
  四解}
 沉浮各異路,會合當何諧?願作東北風,吹我入君懷。
 \gezhu{
  五解}
 君懷常不開,賤妾當何依。恩情中道絕,流止任東西。
 \gezhu{
  六解}
 我欲竟此曲,此曲悲且長。今日樂相樂,別後莫相忘!
 \gezhu{
  七解}



\end{pinyinscope}