\article{卷四本紀第四 少帝}

\begin{pinyinscope}

 少帝諱義符,小字車兵,武帝長子也,母曰張夫人。晉義熙二年,生於京口。



 武帝晚無男,及帝生,甚悅。年十歲,拜豫章公世子。帝有旅力,善騎射,解音律。



 宋臺建,拜宋世
 子。元熙元年,進為宋太子。武帝受禪,立為皇太子。永初三年五月癸亥,武帝崩,是日,太子即皇帝位。大赦;尊皇太后曰太皇太后。六月壬申,以尚書僕射傅亮為中書監,司空徐羨之、領軍將軍謝晦及亮輔政。戊子,太尉長沙王道憐薨。秋九月丁未,有司奏武皇帝配南郊,武敬皇后配北郊。冬十一月戊午,有星孛於營室。十二月庚戌,魏軍克滑臺。



 明年春正月己亥朔,大赦,改元為景平元年。文武進位二等。辛巳,祀南郊。



 虜將達奚仰破金墉,
 進圍虎牢。毛德祖擊虜敗之,虜退而復合。拓跋木末又遣安平公涉歸寇青州。癸卯,河南郡失守。乙卯,有星孛于東壁。二月丁丑,太皇太后崩。



 沮渠蒙遜、吐谷渾阿豺並遣使朝貢。庚辰,爵蒙遜為大將軍,封河西王。以阿豺為安西將軍、沙州刺史,封澆河公。辛未,富陽人孫法光反,寇山陰,會稽太守褚淡之遣山陰令陸劭討敗之。三月壬寅,孝懿皇后祔葬於興寧陵。是月,高麗國遣使朝貢。甲子,豫州刺史劉粹遣軍襲許昌,殺虜潁川太守庾龍。
 乙丑,虜騎寇高平。初,虜自河北之敗,請修和親;及聞高祖崩,因復侵擾,河、洛之地騷然矣。夏四月,檀道濟北征,次臨朐,焚虜攻具。乙未,魏軍克虎牢,執司州刺史毛德祖以歸。秋七月癸酉,尊所生張夫人為皇太后。丁丑,以旱,詔赦五歲刑以下罪人。冬十月己未,有星孛於氐,指尾,貫攝提,向大角,仲月在危,季月掃天倉而後滅。是歲,魏主拓跋嗣薨,子燾立。十二月丙寅,省寧州之江陽、犍為、安上三郡,合為宋昌郡。



 二年春正月癸巳朔,日有蝕之。廢南豫州刺史廬陵王義真為庶人,徙新安郡。



 乙未,以皇弟義恭為冠軍將軍,南豫州刺史。乙巳,大風,天有五色雲,占者以為有兵。高麗國遣使貢獻。執政使使者誅義真於新安。夏五月,江州刺史檀道濟、揚州刺史王弘入朝。帝居處所為多過失。乙酉,皇太后令曰:王室不造,天禍未悔,先帝創業弗永,棄世登遐。義符長嗣,屬當天位,不謂窮凶極悖,一至於此。大行在殯,宇內哀惶,幸災肆於悖詞,喜容表於在
 戚。至乃徵召樂府,鳩集伶官,優倡管弘,靡不備奏,珍羞甘膳,有加平日。採擇媵御,產子就宮,鋋然無怍,醜聲四達。及懿后崩背,重加天罰,親與左右執紼歌呼,推排梓宮,抃掌笑謔,殿省備聞。加復日夜媟狎,群小慢戲,興造千計,費用萬端,帑藏空虛,人力殫盡。刑罰苛虐,幽囚日增。居帝王之位,好阜隸之役;處萬乘之尊,悅廝養之事。親執鞭撲,毆擊無辜,以為笑樂。穿池築觀,朝成暮毀;徵發工匠,疲極兆民。遠近歎嗟,人神怨怒。社稷將墜,豈可
 復嗣守洪業,君臨萬邦。今廢為營陽王,一依漢昌邑、晉海西故事。奉迎鎮西將軍宜都王義隆入纂皇統。



 始徐羨之、傅亮將廢帝,諷王弘、檀道濟求赴國訃。弘等來朝,使中書舍人邢安泰、潘盛為內應。是旦,道濟、謝晦領兵居前,羨之等隨後,因東掖門開,入自雲龍門。盛等先戒宿衛,莫有禦者。時帝於華林園為列肆,親自酤賣。又開瀆聚土,以象破岡埭,與左右引船唱呼,以為歡樂。夕游天泉池,即龍舟而寢。其朝未興,兵士進,殺二侍者於帝
 側,傷帝指。扶出東皞,就收璽紱,群臣拜辭,送於東宮,遂幽於吳郡。是日,赦死罪以下。太后令奉還璽紱,檀道濟入守朝堂。六月癸丑,徐羨之等使中書舍人邢安泰弒帝於金昌亭。帝有勇力,不即受制,突走出昌門,追以門關踣之,致殞。時年十九。



\end{pinyinscope}