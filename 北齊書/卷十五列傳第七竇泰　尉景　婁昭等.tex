\article{卷十五列傳第七竇泰 尉景 婁昭等}

\begin{pinyinscope}

 竇
 泰,字世寧,大安捍殊人也。本出清河觀津,曾祖羅,魏統萬鎮將,因居北邊。父樂,魏末破六韓拔陵為亂,與鎮將楊鈞固守,遇害。泰貴,追贈司徒。初,泰母夢風雷暴起,若有雨狀,出庭觀之,見電光奪目,駛雨霑灑,寤而驚汗,遂有娠。期而不產,大懼。有巫曰:「渡河湔裙,產子必易。」便向水所。忽見一人,曰:「當生貴子,可徙而南。」泰母從之。俄
 而生泰。及長,善騎射,有勇略。泰父兄戰歿於鎮,泰身負骸骨歸爾朱榮。以從討邢杲功,賜爵廣阿子。神武之為晉州,請泰為鎮城都督,參謀軍事。累遷侍中、京畿大都督,尋領御史中尉。泰以勳戚居臺,雖無多糾舉,而百僚畏懼。



 天平三年,神武西討,令泰自潼關入。四年,泰至小關,為周文帝所襲,眾盡沒,泰自殺。初,泰將發鄴,鄴有惠化尼謠云:「竇行臺,去不回。」未行之前,夜三更,忽有朱衣冠幘數千人入臺,云「收竇中尉」,宿直兵吏皆驚,其人入
 數屋,俄頃而去。旦視關鍵不異,方知非人。皆知其必敗。贈大司馬、太尉、錄尚書事,謚曰武貞。泰妻,武明婁后妹也。泰雖以親見待,而功名自建。齊受禪,祭告其墓。



 皇建初,配享神武廟庭。子孝敬嗣。位儀同三司。



 尉景,字士真,善無人也。秦、漢置尉候官,其先有居此職者,因以氏焉。景性溫厚,頗有俠氣。魏孝昌中,北鎮反,景與神武入杜洛周軍中,仍共歸爾朱榮。



 以軍功封博野縣伯。後從神武起兵信都。韓陵之戰,唯景所統失利。神
 武入洛,留景鎮鄴。尋進封為公。景妻常山君,神武之姊也。以勳戚,每有軍事,與厙狄干常被委重,而不能忘懷財利,神武每嫌責之。轉冀州刺史,又大納賄,發夫獵,死者三百人。厙狄干與景在神武坐,請作御史中尉。神武曰:「何意下求卑官?」乾曰:「欲捉尉景。」神武大笑,令優者石董桶戲之。董桶剝景衣,曰:「公剝百姓,董桶何為不剝公!」神武誡景曰:「可以無貪也。」景曰:「與爾計生活孰多,我止人上取,爾割天子調。」神武笑不答。改長樂郡公。歷位太
 保、太傅,坐匿亡人見禁止。使崔暹謂文襄曰:「語阿惠兒,富貴欲殺我耶!」神武聞之泣,詣闕曰:「臣非尉景,無以至今日。」三請,帝乃許之。於是黜為驃騎大將軍、開府儀同三司。神武造之,景恚臥不動,叫曰:「殺我時趣耶!」常山君謂神武曰:「老人去死近,何忍煎迫至此。」又曰:「我為爾汲水胝生。」因出其掌。神武撫景,為之屈膝。先是,景有果下馬,文襄求之,景不與,曰:「土相扶為墻,人相扶為王,一馬亦不得畜而索也。」神武對景及常山君責文襄而杖之。
 常山君泣救之。景曰:「小兒慣去,放使作心腹,何須乾啼濕哭不聽打耶!」尋授青州刺史,操行頗改,百姓安之。征授大司馬。遇疾,薨於州。贈太師、尚書令。齊受禪,以景元勳,詔祭告其墓。皇建初,配享神武廟庭,追封長樂王。



 子粲,少歷顯職,性粗武。天保初,封厙狄乾等為王,粲以父不預王爵,大恚恨,十餘日閉門不朝。帝怪,遣使就宅問之。隔門謂使者曰:「天子不封粲父為王,粲不如死。」使云:「須開門受敕。」粲遂彎弓隔門射使者。使者以狀聞之,文宣
 使段韶諭旨。粲見韶,唯撫膺大哭,不答一言。文宣親詣其宅慰之,方復朝請。尋追封景長樂王。粲襲爵。位司徒、太傅,薨。子世辯嗣。周師將入鄴,令辯出千餘騎覘候,出滏口,登高阜西望,遙見群烏飛起,謂是西軍旗幟,即馳還,比至紫陌橋,不敢回顧。隋開皇中,卒於浙州刺史。



 婁昭,字菩薩,代郡平城人也,武明皇后之母弟也。祖父提,雄傑有識度,家僮千數,牛馬以谷量。性好周給,士多歸附之。魏太武時,以功封真定侯。父內干,有武力,未仕
 而卒。昭貴,魏朝贈司徒。齊受禪,追封太原王。昭方雅正直,有大度深謀,腰帶八尺,弓馬冠世。神武少親重之。昭亦早識人,恒曲盡禮敬。數隨神武獵,每致請不宜乘危歷階。神武將出信都,昭贊成大策,即以為中軍大都督。從破爾朱兆於廣阿,封安喜縣伯,改濟北公,又徙濮陽郡公,授領軍將軍。魏孝武將貳於神武,昭以疾辭還晉陽。從神武入洛,兗州刺史樊子鵠反,以昭為東道大都督討之。子鵠既死,諸將勸昭盡捕誅其黨。昭曰:「此州無
 狀,橫被殘賊,其君是怨,其人何罪。」遂皆捨焉。後轉大司馬,仍領軍。遷司徒,出為定州刺史。昭好酒,晚得偏風,雖愈,猶不能處劇務,在州事委僚屬,昭舉其大綱而已。薨於州。贈假黃鉞、太師、太尉,謚曰武。齊受禪,詔祭告其墓,封太原王。皇建初,配享神武廟庭。長子仲達嗣,改封濮陽王。



 次子定遠,少歷顯職,外戚中偏為武成愛狎。別封臨淮郡王。武成大漸,與趙郡王等同受顧命,位司空。趙郡王之奏黜和士開,定遠與其謀,遂納士開賄賂,成趙
 郡之禍,其貪鄙如此。尋除瀛州刺史。初,定遠弟季略,穆提婆求其伎妾,定遠不許。因高思好作亂,提婆令臨淮國郎中令告定遠陰與思好通。後主令開府段暢率三千騎掩之,令侍御史趙秀通至州,以贓貨事劾定遠。定遠疑有變,遂縊而死。



 昭兄子睿。睿字佛仁,父拔,魏南部尚書。睿幼孤,被叔父昭所養。為神武帳內都督,封掖縣子,累遷光州刺史。在任貪縱,深為文襄所責。後改封九門縣公。



 齊受禪,得除領軍將軍,別封安定侯。睿無他器
 幹,以外戚貴幸,縱情財色。為瀛州刺史,聚斂無厭。皇建初,封東安王。大寧元年,進位司空。平高歸彥於冀州,還拜司徒。河清三年,濫殺人,為尚書左丞宋仲羨彈奏,經赦乃免。尋為太尉,以軍功進大司馬。武成至河陽,仍遣總偏師赴懸瓠。睿在豫境留停百餘日,專行非法,詔免官,以王還第。尋除太尉,薨。贈大司馬。子子彥嗣,位開府儀同三司。



 厙狄干,善無人也。曾祖越豆眷,魏道武時以功割善無
 之西臘汙山地方百里以處之,後率部北遷,因家朔方。乾梗直少言,有武藝。魏正光初,除掃逆黨,授將軍,宿衛於內。以家在寒鄉,不宜毒暑,冬得入京師,夏歸鄉里。孝昌元年,北邊擾亂,奔雲中,為刺史費穆送于爾朱榮。以軍主隨榮入洛。後從神武起兵,破四胡於韓陵,封廣平縣公,尋進郡公。河陰之役,諸將大捷,唯干兵退。神武以其舊功,竟不責黜。尋轉太保、太傅。及高仲密以武牢叛,神武討之,以乾為大都督前驅。



 乾上道不過家,見侯
 景不遑食,景使騎追饋之。時文帝自將兵至洛陽,軍容甚盛。



 諸將未欲南度,乾決計濟河。神武大兵繼至,遂大破之。還為定州刺史。不閑吏事,事多擾煩,然清約自居,不為吏人所患。遷太師。天保初,以乾元勳佐命,封章武郡王,轉太宰。乾尚神武妹樂陵長公主,以親地見待。自預勤王,常總大眾,威望之重,為諸將所伏,而最為嚴猛。會詣京師,魏譙王元孝友於公門言戲過度,諸公無能面折者,干正色責之,孝友大慚,時人稱之。薨,贈假黃鉞,
 太宰,給轀輬車,謚曰景烈。乾不知書,署名為乾字,逆上畫之,時人謂之穿錐。又有武將王周者,署名先為「吉」而後成其外,二人至子孫始並知書。乾,皇建初配享神武廟庭。子敬伏,位儀同三司,卒。子士文嗣。



 士文性孤真,雖鄰里至親,莫與通狎。在齊,襲封章武郡王,位領軍將軍。周武帝平齊,山東衣冠多來迎,唯士文閉門自守。帝奇之,授開府儀同三司,隨州刺史。隋文受禪,加上開府,封湖陂縣子。尋拜貝州刺史。性清苦,不受公料,家無餘財。
 其子嘗啖官廚餅,士文枷之於獄累日,杖之二百,步送還京,僮隸無敢出門。



 所買鹽菜,必於外境。凡有出入,皆封署其門,親故絕跡,慶弔不通。法令嚴肅,吏人貼服,道不拾遺。凡有細過,必深文陷害之。嘗入朝,遇上賜公卿入左藏,任取多少。人皆極重,士文獨口銜絹一匹,兩手各持一匹。上問其故,士文曰:「臣口手俱足,餘無所須。」上異之,別齎遺之。士文至州,發摘姦吏,尺布斗粟之贓,無所寬貸,得千人奏之,悉配防嶺南。親戚相送,哭聲遍於
 州境。至嶺南,遇瘴厲死者十八九,於是父母妻子唯哭士文。士文聞之,令人捕搦,捶楚盈前,而哭者彌甚。司馬京兆韋焜、清河令河東趙達十人並苛刻,唯長史有惠政,時人語曰:「刺史羅剎政,司馬蝮蛇瞋,長史含笑判,清河生吃人。」上聞,歎曰:「士文暴過猛獸。」竟坐免。未幾為雍州長史,謂人曰:「我向法深,不能窺候要貴,無乃必死此官。」及下車,執法嚴正,不避貴戚,賓客莫敢至門。人多怨望。士文從妹為齊氏嬪,有色,齊滅後,賜薛公長孫覽。覽
 妻鄭氏妒,譖之文獻后,后令覽離絕。士文恥之,不與相見。後應州刺史唐君明居母憂,娉以為妻,由是君明、士文並為御史所劾。士文性剛,在獄數日,憤恚而死,家無餘財,有三子,朝夕不繼,親賓無贍之者。



 韓軌,字百年,太安狄那人也。少有志操,性深沉,喜怒不形於色。神武鎮晉州,引為鎮城都督。及起兵於信都,軌贊成大策。從破爾朱兆於廣阿,又從韓陵陣,封平昌縣侯。仍督中軍,從破爾朱兆於赤谼嶺。再遷泰州刺史。甚
 得邊和。神武巡泰州,欲以軌還,仍賜城人戶別絹布兩匹。州人田昭等七千戶皆辭不受,唯乞留軌。



 神武嘉歎,乃留焉。頻以軍功,進封安德郡公。遷瀛州刺史,在州聚斂,為御史糾劾,削除官爵。未幾,復其安德郡公。歷位中書令、司徒。齊受禪,封安德郡王。



 軌妹為神武所納,生上黨王渙,復以勳庸,歷登台鉉。常以謙恭自處,不以富貴驕人。後拜大司馬,從文宣征蠕蠕,在軍暴疾薨。贈假黃鉞,太宰、太師,謚曰肅武。



 皇建初,配饗文襄廟庭。



 子晉明
 嗣。天統中,改封東萊王。晉明有俠氣,諸勳貴子孫中最留心學問。好酒誕縱,招引賓客,一席之費,動至萬錢,猶恨儉率。朝庭處之貴要之地,必以疾辭。告人云:「廢人飲美酒、封名勝,安能作刀筆吏返披故紙乎?」武平末,除尚書左僕射,百餘日便謝病解官。



 潘樂,字相貴,廣寧石門人也。本廣宗大族,魏世分鎮北邊,因家焉。父永,有技藝,襲爵廣宗男。樂初生,有一雀止其母左肩,占者咸言富貴之徵,因名相貴,後始為字。及
 長,寬厚有膽略。初歸葛榮,授京兆王,時年十九。榮敗,隨爾朱榮,為別將討元顥,以功封敷城縣男。齊神武出牧晉州,引樂為鎮城都將。從破爾朱兆於廣阿,進爵廣宗縣伯。累以軍功拜東雍州刺史。神武嘗議欲廢州,樂以東雍地帶山河,境連胡、蜀,形勝之會,不可棄也,遂如故。後破周師於河陰,議欲追之,令追者在西,不願者東,唯樂與劉豐居西。神武善之,以眾議不同而止。改封金門郡公。文宣嗣事,鎮河陽,破西將楊檦等。時帝以懷州刺
 史平鑒等所築城深入敵境,欲棄之,樂以軹關要害,必須防固,乃更修理,增置兵將,而還鎮河陽,拜司空。



 齊受禪,樂進璽綬。進封河東郡王,遷司徒。周文東至崤、陜,遣其行臺侯莫陳崇自齊子嶺趣軹關,儀同楊檦從鼓鐘道出建州,陷孤公戍。詔樂總大眾禦之。樂晝夜兼行,至長子,遣儀同韓永興從建州西趣崇,崇遂遁。又為南道大都督,討侯景。



 樂發石鱉,南度百餘里,至梁涇州。涇州舊在石梁,侯景改為懷州,樂獲其地,仍立涇州,又克安
 州。除瀛州刺史,仍略淮、漢。天保六年,薨於懸瓠。贈假黃鉞,太師、大司馬、尚書令。



 子子晃嗣。諸將子弟,率多驕縱,子晃沉密謹愨,以清凈自居。尚公主,拜駙馬都尉。武平末,為幽州道行臺右僕射、幽州刺史。周師將入鄴,子晃率突騎數萬赴援。至博陵,知鄴城不守,詣冀州降。周授上開府。隋大業初卒。



\end{pinyinscope}