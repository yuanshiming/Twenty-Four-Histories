\article{卷二十一列傳第十三}

\begin{pinyinscope}

 高乾弟慎弟昂
 弟季式封隆之子子繪從子孝琬孝琰高乾,字乾邕,渤海蓚人也。父翼,字次同,豪俠有風神,為州里所宗敬。孝昌末,葛榮作亂
 于燕、趙,朝廷以翼山東豪右,即家拜渤海太守。至郡未幾,賊徒愈盛,翼部率合境,徙居河、濟之間。魏因置東冀州,以翼為刺史,加鎮東將軍、樂城縣侯。及爾朱兆弒莊帝,翼保境自守。謂諸子曰:「主憂臣辱,主辱臣死,今社稷阽危,人神憤怨,破家報國,在此時也。爾朱兄弟,性甚猜忌,忌則多害,汝等宜早圖之。先人有奪人之心,時不可失也。」事未輯而卒。中興初,贈使持節、侍中、太保、錄尚
 書
 事、冀
 定瀛相殷幽六州諸軍事、冀州刺史,謚曰文宣。



 乾性明悟,俊偉有知略,美音容,
 進止都雅。少時輕俠,數犯公法,長而修改,輕財重義,多所交結。魏領軍元叉,權重當世,以意氣相得,接乾甚厚。起家拜員外散騎侍郎,領直後,轉太尉士曹、司徒中兵,遷員外。魏孝莊之居藩也,乾潛相託附。及爾朱榮入洛,乾東奔於翼。莊帝立,遙除龍驤將軍、通直散騎常侍。乾兄弟本有從橫志,見榮殺害人士,謂天下遂亂,乃率河北流人反於河、濟之間,受葛榮官爵,屢敗齊州士馬。莊帝尋遣右僕射元羅巡撫三齊,乾兄弟相率出降。
 朝廷以乾為給事黃門侍郎。爾朱榮以乾前罪,不應復居近要,莊帝聽乾解官歸鄉里。於是招納驍勇,以射獵自娛。榮死,乾馳赴洛陽,莊帝見之,大喜。時爾朱徒黨擁兵在外,莊帝以乾為金紫光祿大夫、河北大使,令招集鄉閭為表裏形援。乾垂涕奉詔,弟昂拔劍起舞,請以死自效。



 俄而爾朱兆入洛,尋遣其監軍孫白鷂百餘騎至冀州,託言普徵民馬,欲待乾兄弟送馬,因收之。乾既宿有報復之心,而白鷂忽至,知將見圖,乃先機定策,潛勒
 壯士,襲據州城,傳檄州郡,殺白鷂,執刺史元仲宗。推封隆之權行州事,為莊帝舉哀,三軍縞素。乾昇壇誓眾,辭氣激揚,涕淚交下,將士莫不哀憤。北受幽州刺史劉靈助節度,共為影響。俄而靈助被殺。屬高祖出山東,揚聲來討,眾情莫不惶懼。乾謂其徒曰:「吾聞高晉州雄略蓋世,其志不居人下。且爾朱無道,殺主虐民,正是英雄效義之會也。今日之來,必有深計,吾當輕馬奉迎,密參意旨,諸君但勿憂懼,聽我一行。」乾乃將十數騎於關口迎
 謁。乾既曉達時機,閑習世事,言辭慷慨,雅合深旨,高祖大加賞重,仍同帳寢宿。時高祖雖內有遠圖,而外跡未見,爾朱羽生為殷州刺史,高祖密遣李元忠舉兵逼其城,令乾率眾偽往救之。乾遂輕騎入見羽生,與指畫軍計。羽生與乾俱出,因擒之,遂平殷州。又共定策推立中興主,拜乾侍中、司空。先是信都草創,軍國權輿,乾遭喪不得終制。及武帝立,天下初定,乾乃表請解職,行三年之禮。詔聽解侍中,司空如故,封長樂郡公,邑一千戶。



 乾
 雖求退,不謂便見從許。既去內侍,朝廷罕所關知,居常怏怏。



 武帝將貳於高祖,望乾為己用,會於華林園,宴罷,獨留乾,謂之曰:「司空奕世忠良,今日復建殊效,相與雖則君臣,實亦義同兄弟,宜共立盟約以敦情契。」



 殷勤逼之。乾對曰:「臣世奉朝廷,遇荷殊寵,以身許國,何敢有貳。」乾雖有此對,然非其本心。事出倉卒,又不謂武帝便有異圖,遂不固辭,而不啟高祖。及武帝置部曲,乾乃私謂所親曰:「主上不親勳賢,而招集群豎。數遣元士弼、王思
 政往來關西,與賀拔岳計議。又出賀拔勝為荊州刺史,外示疏忌,實欲樹黨,令其兄弟相近,冀據有西方。禍難將作,必及於我。」乃密啟高祖。高祖召乾詣并州,面論時事,乾因勸高祖以受魏禪。高祖以袖掩其口曰:「勿妄言。今啟司空復為侍中,門下之事,一以相委。」高祖屢啟,詔書竟不施行。



 乾以頻請不遂,知變難將起,密啟高祖,求為徐州,乃除使持節、都督三徐諸軍事、開府儀同三司、徐州刺史。指期將發,而帝知乾泄漏前事,乃詔高祖云:「
 曾與乾邕私有盟約,今復反覆兩端。」高祖便取乾前後數啟論時事者,遣使封送武帝。帝召乾邕示之,禁於門下省,對高祖使人責乾前後之失。乾曰:「臣以身奉國,義盡忠貞,陛下既立異圖,而乃云臣反覆。以匹夫加諸罪,尚或難免,況人主推惡,復何逃命。欲加之罪,其無辭乎?功大身危,自古然也。若死而有知,庶無負莊帝。」遂賜死,時年三十七。乾臨死,神色不變,見者莫不歎惜焉。時武衛將軍元整監刑,謂乾曰:「頗有書及家人乎?」乾曰:「吾兄
 弟分張,各在異處,今日之事,想無全者,兒子既小,未有所識,亦恐巢傾卵破,夫欲何言。」後高祖討斛斯椿等,次盟津,謂乾弟昂曰:「若早用司空之策,豈有今日之舉也!」天平初,贈使持節、都督冀定滄瀛幽齊徐青光兗十州軍事、太師、錄尚書事、冀州刺史,謚曰文昭。長子繼叔襲祖樂城縣侯,令第二子呂兒襲乾爵。



 乾弟慎,字仲密,頗涉文史,與兄弟志尚不同,偏為父所愛。魏中興初,除滄州刺史、東南道行臺尚書。太昌初,遷光州刺史,加驃騎
 大將軍、儀同三司。時天下初定,聽慎以本鄉部曲數千人自隨。慎為政嚴酷,又縱左右,吏民苦之。兄乾死,密棄州將歸高祖,武帝敕青州斷其歸路。慎間行至晉陽,高祖以為大行臺左丞,轉尚書,當官無所迴避,時咸畏憚之。自義旗之後,安州民恃其邊險,不賓王化,尋以慎為行臺僕射,率眾討平之。天平末,拜侍中,加開府。元象初,出為兗州刺史。



 尋徵為御史中尉,選用御史,多其親戚鄉閭,不稱朝望,世宗奏令改選焉。慎前妻吏部郎中崔
 暹妹,為慎所棄。暹時為世宗委任,慎謂其構己,性既狷急,積懷憤恨,因是罕有糾劾,多所縱舍。高祖嫌責之,彌不自安。出為北豫州刺史,遂據武牢降西魏。慎先入關。周文帝率眾東出,高祖破之於邙山。慎妻子將西度,於路盡禽之。



 高祖以其勳家,啟慎一房配沒而已。



 昂,字敖曹,乾第三弟。幼稚時,便有壯氣。長而俶儻,膽力過人,龍眉豹頸,姿體雄異。其父為求嚴師,令加捶撻。昂不遵師訓,專事馳騁,每言男兒當橫行天下,自取富貴,
 誰能端坐讀書作老博士也。與兄乾數為劫掠,州縣莫能窮治。招聚劍客,家資傾盡,鄉閭畏之,無敢違迕。父翼常謂人曰:「此兒不滅我族,當大吾門,不直為州豪也。」



 建義初,兄弟共舉兵,既而奉旨散眾,仍除通直散騎侍郎,封武城縣伯,邑五百戶。乾解官歸,與昂俱在鄉里,陰養壯士。爾朱榮聞而惡之,密令刺史元仲宗誘執昂,送於晉陽。永安末,榮入洛,以昂自隨,禁於駝牛署。既而榮死,魏莊帝既引見勞勉之。時爾朱既隆還逼宮闕,帝親臨
 大夏門指麾處分。昂既免縲紲,被甲橫戈,志凌勁敵。乃與其從子長命等推鋒徑進,所向披靡。帝及觀者莫不壯之。既除直閣將軍,賜帛千匹。昂以寇難尚繁,非一夫所濟,乃請還本鄉,招集部曲。仍除通直常侍,加平北將軍。所在義勇,競來投赴。尋值京師不守,遂與父兄據信都起義。殷州刺史爾朱羽生潛軍來襲,奄至城下,昂不暇擐甲,將十餘騎馳之,羽生退走,人情遂定。後廢帝立,除使持節、冀州刺史以終其身。仍為大都督,率眾從
 高祖破爾朱兆於廣阿。及平鄴,別率所部領黎陽。又隨高祖討爾朱兆於韓陵,昂自領鄉人部曲王桃湯、東方老、呼延族等三千人。高祖曰:「高都督純將漢兒,恐不濟事,今當割鮮卑兵千餘人共相參雜,於意如何?」昂對曰:「敖曹所將部曲練習已久,前後戰鬥,不減鮮卑,今若雜之,情不相合,勝則爭功,退則推罪,願自領漢軍,不煩更配。」高祖然之。及戰,高祖不利,軍小卻,兆等方乘之。高嶽、韓匈奴等以五百騎衝其前,斛律敦收散卒躡其後,昂
 與蔡俊以千騎自栗園出,橫擊兆軍。



 兆眾由是大敗。是日微昂等,高祖幾殆。



 太昌初,始之冀州。尋加侍中、開府,進爵為侯,邑七百戶。兄乾被殺,乃將十餘騎奔晉陽,歸於高祖。及斛斯椿釁起,高祖南討,令昂為前驅。武帝西遁,昂率五百騎倍道兼行,至於崤陜,不及而還。尋行豫州刺史,仍討三荊諸州不附者,並平之。天平初,除侍中、司空公。昂以兄乾薨於此位,固辭不拜,轉司徒公。時高祖方有事關隴,以昂為西南道大都督,徑趣商洛。山道
 峻隘,已為寇所守險,昂轉鬥而進,莫有當其鋒者。遂攻剋上洛,獲西魏洛州刺史泉企,並將帥數十人。會竇泰失利,召昂班師。時昂為流矢所中,創甚,顧謂左右曰:「吾以身許國,死無恨矣,所可歎息者,不見季式作刺史耳。」高祖聞之,既馳驛啟季式為濟州刺史。



 昂還,復為軍司大都督,統七十六都督,與行臺侯景治兵於武牢。御史中尉劉貴時亦率眾在北豫州,與昂小有忿爭,昂怒,鳴鼓會兵而攻之。侯景與冀州刺史萬俟受洛干救解乃
 止。其俠氣凌物如此。于時鮮卑共輕中華朝士,唯憚服於昂。高祖每申令三軍,常鮮卑語,昂若在列,則為華言。昂嘗詣相府,掌門者不納,昂怒,引弓射之。高祖知而不責。



 元象元年,進封京兆郡公,邑一千戶。與侯景等同攻獨孤如願於金墉城,周文帝率眾救之。戰於邙陰,昂所部失利,左右分散,單馬東出,欲趣河陽南城,門閉不得入,遂為西軍所害,時年四十八。贈使持節侍中、都督冀定滄瀛殷五州諸軍事、太師、大司馬、太尉公、錄尚書
 事、冀州刺史,謚忠武。子突騎嗣,早卒。世宗復召昂諸子,親簡其第三子道豁嗣。皇建初,追封昂永昌王。道豁襲,武平末,開府儀同三司。入周,授儀同大將軍。開皇中,卒於黃州刺史。



 季式,字子通,乾第四弟也。亦有膽氣。中興初,拜鎮遠將軍、正員郎,遷衛將軍、金紫光祿大夫,尋加散騎常侍,領主衣都統。太昌初,除尚食典御。天平中,出為濟州刺史。山東舊賊劉盤陀、史明曜等攻劫道路,剽掠村邑,齊、兗、
 青、徐四州患之,歷政不能討。季式至,皆破滅之。尋有濮陽民杜靈椿等攻城剽野,聚眾將萬人,季式遣騎三百,一戰擒之。又陽平路叔文徒黨緒顯等立營柵為亂,季式討平之。又有群賊破南河郡,季式遣兵臨之,應時斬戮。自茲以後,遠近清晏。季式兄弟貴盛,並有勳於時,自領部曲千餘人,馬八百匹,戈甲器仗皆備,故凡追督賊盜,多致克捷。有客嘗謂季式曰:「濮陽、陽平乃是畿內,既不奉命,又不侵境,而有何急,遣私軍遠戰?萬一失脫,豈
 不招罪?」季式曰:「君言何不忠之甚也!



 我與國義同安危,豈有見賊不討之理?且賊知臺軍卒不能來,又不疑外州有救,未備之間,破之必矣。兵尚神速,何得後機,若以獲罪,吾亦無限。」



 元象中,西寇大至。高祖親率三軍以禦之,陣於邙北,師徒大敗,河中流尸相繼,敗兵首尾不絕。人情騷動,謂世事艱難。所親部曲請季式曰:「今日形勢,大事去矣,可將腹心二百騎奔梁,既得避禍,不失富貴。何為坐受死也?」季式曰:「吾兄弟受國厚恩,與高王共定
 天下,一旦傾危,亡去不義。若社稷顛覆,當背城死戰,安能區區偷生茍活!」是役也,司徒歿焉。



 入為散騎常侍。興和中,行晉州事。解州,仍鎮永安戍。高慎以武牢叛,遣信報季式。季式得書驚懼,既狼狽奔告高祖。高祖嘉其至誠,待之如舊。武定中,除侍中,尋加冀州大中正,時世宗先為此任,啟以迴授。為都督,從清河公岳破蕭明於寒山,敗侯景於渦陽。還,除衛尉卿。復為都督,從清河公攻王思政於潁川,拔之。以前後功加儀同三司。天保初,封
 乘氏縣子。仍為都督,隨司徒潘樂征討江、淮之間。為私使樂人於邊境交易,還京,坐被禁止,尋而赦之。四年夏,發疽卒,年三十八。贈侍中、使持節、都督滄冀州諸軍事、開府儀同三司、冀州刺史,謚曰恭穆。



 季式豪率好酒,又恃舉家勳功,不拘檢節。與光州刺史李元忠生平遊款,在濟州夜飲,憶元忠,開城門,令左右乘驛持一壺酒往光州勸元忠。朝廷知而容之。兄慎叛後,少時解職。黃門郎司馬消難,左僕射子如之子,又是高祖之婿,勢
 盛當時。



 因退食暇,尋季式與之酣飲。留宿旦日,重門並閉,關籥不通。消難固請云:「我是黃門郎,天子侍臣,豈有不參朝之理?且已一宿不歸,家君必當大怪。今若又留我狂飲,我得罪無辭,恐君亦不免譴責。」季式曰:「君自稱黃門郎,又言畏家君怪,欲以地勢脅我邪?高季式死自有處,實不畏此。」消難拜謝請出,終不見許。



 酒至,不肯飲。季式云:「我留君盡興,君是何人,不為我痛飲。」命左右索車輪括消難頸,又索一輪自括頸,仍命酒引滿相勸。消難不
 得已,欣笑而從之,方乃俱脫車輪,更留一宿。是時失消難兩宿,莫知所在,內外驚異。及消難出,方具言之。



 世宗在京輔政,白魏帝賜消難美酒數石,珍羞十輿,并令朝士與季式親狎者,就季式宅宴集。其被優遇如此。



 翼長兄子永樂、次兄子延伯,並和厚有長者稱,俱從翼舉義。永樂官至衛將軍、右光祿大夫、冀州大中正,出為博陵太守,以民事不濟,自殺。贈使持節、督滄冀二州諸軍事、儀同三司、冀州刺史。子長命,本自賤出,年二十餘始被
 收舉。猛暴好殺,然亦果於戰鬥。初於大夏門拒爾朱世隆,以功累遷左光祿大夫。高祖遙授長命雍州刺史,封沮陽鄉男,一百戶。尋進封鄢陵縣伯,增二百戶,武定中,隨儀同劉豐討侯景,為景所殺。贈冀州刺史。延伯歷中散大夫、安州刺史,封萬年縣男,邑二百戶。天保初,加征西將軍,進爵為子。卒,贈太府少卿。



 自昂初以豪俠立名,為之羽翼者,呼延族、劉貴珍、劉長狄、東方老、劉士榮、成五彪、韓願生、劉桃棒;隨其建義者,李希光、劉叔宗、劉孟
 和。並仕宦顯達。



 孟和名協,浮陽饒安人也。孟和少好弓馬,率性豪俠。幽州刺史劉靈助之起兵也,孟和亦聚眾附昂兄弟,昂遙應之。及靈助敗,昂乃據冀州,孟和為其致力。會高祖起義冀州,以孟和為都督。中興初,拜通直常侍。二年,除安東將軍。尋加征東將軍、金紫光祿。以建義勳,賜爵長廣縣伯。天平中,衛將軍、上黨內史,罷郡,除大丞相司馬。武定元年,坐事死。



 叔宗字元纂,樂陵平昌人。和謹,頗有學業,舉秀才。稍遷滄州治中。永安中,加鎮
 遠將軍、諫議大夫。兄海寶,少輕俠,然為州里所愛。昂之起義也,海寶率鄉閭襲滄州以應昂,昂以海寶權行滄州事。前范陽太守刁整心附爾朱,遣弟子安壽襲殺海寶。叔宗仍歸於昂。中興初,高祖除前將軍、廷尉少卿。太昌初,加鎮軍將軍、光祿大夫。天平初,除車騎將軍、左光祿大夫。二年卒。贈使持節、儀同、定州刺史。



 老字安德,鬲人。家世寒微。身長七尺,膂力過人。少麤獷無賴,結輕險之徒共為賊盜,鄉里患之。魏末兵起,遂與昂為部曲。義
 旗建,仍從征討,以軍功除殿中將軍。累遷平遠將軍。除魯陽太守。後除南益州刺史,領宜陽太守,賜爵長樂子。



 老頻為二郡,出入數年,境接群蠻,又鄰西敵,至於攻城野戰,率先士卒,屢以少制眾,西人憚之。顯祖受禪,別封陽平縣伯,遷南兗州刺史。後與蕭軌等渡江,戰沒。



 希光,渤海蓚人也。父紹,魏長廣太守,希光隨高乾起義信都。中興初,除安南將軍、安德郡守。後為世祖開府長史。武定末,從高嶽平潁川,封義寧縣開國侯,歷潁、梁、南兗三
 州刺史。天保中,揚州刺史,與蕭軌等渡江,戰沒。贈開府儀同三司、西兗州刺史。子子令,尚書外兵郎中。武平末,通直常侍。隋開皇中,卒於易州刺史。希光族弟子貢,以與義旗之功,官至吏部郎,後為兗州刺史。坐貪暴為世宗所殺。



 顯祖責陳武廢蕭明,命儀同蕭軌率希光、東方老、裴英起、王敬寶步騎數萬伐之。以七年三月渡江,襲剋石頭城。五將名位相侔,英起以侍中為軍司,蕭軌與希光並為都督,軍中抗禮,不相服御,競說謀略,動必乖
 張。頓軍丹陽城下,值霖雨五十餘日,及戰,兵器並不堪施用,故致敗亡。將帥俱死,士卒得還者十二三,所沒器械軍資不可勝紀。蕭軌、王寶事行,史闕其傳。



 裴英起,河東人。其先晉末渡淮,寓居淮南之壽陽縣。祖彥先,隨薛安都入魏,官至趙郡守。父約,渤海相。英起聰慧滑稽,好劇談,不拘儀檢,仁魏至定州長史。



 世宗引為行臺左丞。天保中,都官尚書,兼侍中,及戰沒,贈開府、尚書左僕射。



 封隆之,字祖裔,小名皮,渤海之蓚人也。父回,魏司空。隆
 之性寬和,有度量。弱冠州郡主簿,起家奉朝請,領直後。汝南王悅開府,為中兵參軍。



 初,延昌中,道人法慶作亂冀方,自號「大乘」,眾五萬餘。遣大都督元遙及隆之擒獲法慶,賜爵武城子。俄兼司徒主簿、河南尹丞。時青、齊二州士民反叛,隆之奉使慰諭,咸即降款。永安中,撫軍府長史。爾朱兆等屯據晉陽,魏朝以河內要衝,除隆之龍驤將軍、河內太守,尋加持節、後將軍、假平北將軍、當郡都督。



 未及到郡,屬爾朱兆入洛,莊帝幽崩。隆之以父
 遇害,常懷報雪,因此遂持節東歸,圖為義舉。時高乾告隆之曰:「爾朱暴逆,禍加至尊,弟與兄並荷先帝殊常之眷,豈可不出身為主,以報仇恥乎?」隆之對曰:「國恥家怨,痛入骨髓,乘機而動,今實其時。」遂與乾等定計,夜襲州城,克之。乾等以隆之素為鄉里所信,乃推為刺史。隆之盡心慰撫,人情感悅。



 尋高祖自晉陽東出,隆之遣子子繪奉迎於滏口,高祖甚嘉之。既至信都,集諸州郡督將僚吏等議曰:「逆胡爾朱兆窮凶極虐,天地之所不容,人
 神之所捐棄,今所在蜂起,此天亡之時也。欲與諸君剪除凶羯,其計安在?」隆之對曰:「爾朱暴虐,天亡斯至,神怒民怨,眾叛親離,雖握重兵,其彊易弱。而大王乃心王室,首唱義旗,天下之人,孰不歸仰?願大王勿疑。」中興初,拜左光祿大夫、吏部尚書。



 爾朱兆等軍於廣阿,十月,高祖與戰,大破之。乃遣隆之持節為北道大使。高祖將擊爾朱兆等軍於韓陵,留隆之鎮鄴城。爾朱兆等走,以隆之行冀州事,仍領降俘三萬餘人,分置諸州。



 尋徵為侍中。時
 高祖自洛還師於鄴。隆之將赴都,因過謁見,啟高祖曰:「斛斯椿、賀拔勝、賈顯智等往事爾朱,中復乖阻,及討仲遠,又與之同,猜忍之人,志欲無限。又叱烈延慶侯念賢皆在京師,王授以名位,此等必構禍隙。」高祖經宿乃謂隆之曰:「侍中昨言,實是深慮。」尋封安德郡公,邑二千戶,進位儀同三司。



 于時朝議以爾朱榮佐命前朝,宜配食明帝廟庭。隆之議曰:「榮為人臣,親行殺逆,安有害人之母,與子對饗?考古詢今,未見其義。」從之。詔隆之參議麟
 趾閣,以定新制。又贈其妻祖氏范陽郡君。隆之表以先爵富城子及武城子轉授弟子孝琬等,朝廷嘉而從之。後為斛斯椿等構之於魏帝,逃歸鄉里。高祖知其被誣,召赴晉陽。



 魏帝尋以本官征之,隆之固辭不赴。仍以隆之行并州刺史。魏清河王亶為大司馬。



 長史。



 天平初,復入為侍中,預遷都之議。魏靜帝詔為侍講,除吏部尚書,加侍中,以本官行冀州事。陽平民路紹遵聚眾反,自號行臺,破定州博陵郡,虜太守高永樂,南侵冀州。隆之令
 所部長樂太守高景等擊破之,生擒紹遵,送於晉陽。元象初,除冀州刺史。尋加開府。時初召募勇果,都督孛八、高法雄、封子元等不願遠戍,聚眾為亂。隆之率州軍破平之。興和元年,復徵為侍中。隆之素得鄉里人情,頻為本州,留心撫字,吏民追思,立碑頌德。轉行梁州事,又行濟州事,徵拜尚書右僕射。



 武定初北豫州刺史高仲密將叛,遣使陰通消息於冀州豪望,使為內應。輕薄之徒,頗相扇動,詔隆之馳驛慰撫,遂得安靜。世宗密書與隆
 之云:「仲密枝黨同惡向西者,宜悉收其家累,以懲將來。」隆之以為恩旨既行,理無追改,今若收治,示民不信,脫或驚擾,所虧處大。乃啟高祖,事遂得停。



 隆之自義旗始建,首參經略,奇謀妙算,密以啟聞,手書削稿,罕知於外。高祖嘉其忠謹,每多從之。復以本官行濟州事,轉齊州刺史,武定三年卒官,年六十一。詔遣主書監神貴就弔,賻物五百段,贈使持節、都督滄瀛二州諸軍事、驃騎大將軍、瀛州刺史、司徒公。高祖以隆之勳舊,追榮未盡,復啟
 贈使持節、都督冀瀛滄齊濟五州諸軍事、冀州刺史、太保,餘如故,謚曰宣懿。高祖後至冀州境,次於交津。追憶隆之,顧謂冀州行事司馬子如曰:「封公積德履仁,體通性達,自出納軍國,垂二十年,契闊艱虞,始終如一。以其忠信可憑,方以後事託之。何期報善無徵,奄從物化,言念忠賢,良可痛惜。」為之流涕。令參軍仲羨以太牢就祭焉。



 長子早亡。第二子子繪嗣。



 子繪,字仲藻,小名搔,性和理,有器局。釋褐秘書郎中。爾
 朱兆之害魏莊帝也,與父隆之舉義信都,奉使詣高祖。至信都,召署開府主簿,仍典書記。中興元年,轉大丞相主簿,加伏波將軍。從高祖征爾朱兆。及平中山,軍還,除通直常侍、左將軍,領中書舍人。母憂解職,尋復本任。太昌中,從高祖定并、汾、肆數州,平爾朱兆及山胡等,加征南將軍、金紫光祿大夫,魏武帝末,斛斯椿等佞倖用事,父隆之以猜忌,懼難潛歸鄉里,子繪亦棄官俱還。孝靜初,兼給事黃門侍郎,與太常卿李元忠等並持節出使,
 觀省風俗,問人疾苦。還,赴晉陽,從高祖征夏州。二年,除衛將軍、平陽太守,尋加散騎常侍。晉州北界霍太山,舊號千里徑者,山阪高峻,每大軍往來,士馬勞苦,子繪啟高祖,請於舊徑東谷別開一路。高祖從之,仍令子繪領汾、晉二州夫修治,旬日而就。高祖親總六軍,路經新道,嘉其省便,賜穀二百斛。後大軍討復東雍,平柴壁及喬山、紫谷、絳蜀等,子繪恒以太守前驅慰勞,徵兵運糧,軍士無乏。興和初,自郡徵補大行臺吏部郎中。



 武定元年,高
 仲密以武牢西叛,周文帝擁眾東侵,高祖於邙山破之,乘勝長驅,遂至潼關。或諫不可窮兵極武者,高祖總命群僚議其進止。子繪言曰:「賊帥才非人雄,偷竊名號,遂敢驅率亡叛,遂死伊瀍,天道禍淫,一朝瓦解。雖僅以身免,而魂膽俱喪。混一車書,正在今日,天與不取,反得其咎。時難遇而易失。昔魏祖之平漢中,不乘勝而取巴蜀,失在遲疑,悔無及已。伏願大王不以為疑。」高祖深然之。但以時既盛暑,方為後圖,遂命班師。



 三年,父喪去職。四
 年,高祖西討,起為大都督,領冀州兵赴鄴,從高祖自滏口西趣晉州,會大軍於玉壁。復以子繪為大行臺吏部郎中。及高祖病篤,師還晉陽,引入內室,面受密旨,銜命山東,安撫州郡。高祖崩,秘未發喪,世宗以子繪為渤海太守,令馳驛赴任。世宗親執其手曰:「誠知此郡未允勳望,但時事未安,須卿鎮撫。且衣錦晝遊,古人所貴,善加經略,綏靜海隅,不勞學習常太守向州參也。」



 仍聽收集部曲一千人。後進秩一等,加驃騎將軍。天保二年,除太
 尉長史。三年,頻以本官再行南青州事。四年,坐事免。六年,行南兗州事,尋除持節、海州刺史,不行。



 七年,改授合州刺史。到州未幾,值蕭軌、裴英起等江東敗沒,行臺司馬恭發歷陽,徑還壽春,疆埸大駭。兼在州器械,隨軍略盡,城隍樓雉,虧壞者多,子繪乃造城隍樓雉,繕治軍器,守禦所須畢備,人情漸安。尋敕於州營造船艦,子繪為大使,總監之。陳武帝曾遣其護軍將軍徐度等率輕舟從柵口歷東關入巢湖,徑襲合肥,規燒船舫。以夜一
 更潛寇城下,子繪率將士格戰,陳人奔退。



 九年,轉鄭州刺史。子繪曉達政事,長於綏撫,歷宰州郡,所在安之。徵為司徒左長史,行魏尹事。乾明初,轉大司農,尋正除魏尹。皇建中,加驃騎大將軍。



 大寧二年,遷都官尚書。高歸彥作逆,召子繪入見昭陽殿。帝親詔子繪曰:「冀州密邇京甸,歸彥敢肆凶悖。已敕大司馬、平原王段孝先總勒重兵,乘機電發;司空、東安王婁睿督率諸軍,絡繹繼進。卿世載名德,恩洽彼州,故遣參贊軍事,隨便慰撫,宜善
 加謀略,以稱所寄。」即以其日馳傳赴軍。子繪祖父世為本州,百姓素所歸附。既至,巡城諭以禍福,民吏降款,日夜相繼,賊中動靜,小大必知。賊平,仍敕子繪權行州事。



 尋徵還,敕與群官議定律令,加儀同三司。後突厥入逼晉陽,詔子繪行懷州事,乘驛之任。還為七兵尚書,轉祠部尚書。河清三年,暴疾卒,年五十。世祖深歎惜之。贈使持節、瀛冀二州軍事、冀州刺史、開府儀同、尚書右僕射,謚曰簡。子寶蓋嗣。武平末,通直常侍。



 子繪弟子繡,武平
 中,渤海太守、霍州刺史。陳將吳明徹侵略淮南,子繡城陷,被送揚州。齊亡後,逃歸。隋開皇初,終於通州刺史。子繡外貌儒雅,而俠氣難忤。



 司空婁定遠,子繡兄之婿也,為瀛州刺史。子繡在渤海,定遠過之,對妻及諸女宴集,言戲微有褻慢,子繡大怒,鳴鼓集眾將攻之。俄頃,兵至數千,馬將千匹。定遠免冠拜謝,久乃釋之。



 隆之弟延之,字祖業。少明辯,有世用。起家員外郎。中興初,除中堅將軍。



 高祖以為大行臺左光祿大夫,封郟城縣子。行渤海
 郡事。以都督從婁昭討樊子鵠,事平,除青州刺史。延之好財利,在州多所受納。後行晉州事,高祖沙苑失利還,延之棄州北走。高祖大怒,同罪人皆死,以隆之故,獨得免。興和二年卒,年五十四。贈使持節、都督冀殷瀛三州諸軍事、驃騎大將軍、尚書左僕射、司徒、冀州刺史,謚曰文恭。子孝纂嗣。



 隆之弟子孝琬,字子蒨,父祖曹,魏冀州平北府長史。以隆之佐命之功,贈雍州刺史、殿中尚書。孝琬七歲而孤,獨為隆之所鞠養,慈愛甚篤。年十六,本
 州辟主簿。魏永熙二年,隆之啟以父爵富城子授焉。三年,釋褐開府參軍事。天平中,輕車將軍、司徒主簿。武定中,為顯祖開府主簿,遷從事中郎將,領東宮洗馬。天保二年卒,時年三十六,帝聞而歎惜焉。贈左將軍、太府少卿。孝琬性恬靜,頗好文詠。太子少師邢邵、七兵尚書王昕並先達高才,與孝琬年位懸隔,晚相逢遇,分好遂深。孝琬靈櫬言歸,二人送於郊外,悲哭悽慟,有感路人。



 孝琬弟孝琰,字士光。少修飾學尚,有風儀。年十六,辟州主
 簿,釋褐秘書郎。



 天保元年,為太子舍人,出入東宮,甚有令望。丁母憂,解任,除晉州法曹參軍。



 尋徵還,復除太子舍人。乾明初,為中書舍人。皇建初,司空掾、秘書丞、散騎常侍,聘陳使主,已發道途,遙授中書侍郎。還,坐事除名。天統三年,除并省吏部郎中、南陽王友,赴晉陽典機密。



 和士開母喪,託附咸往奔哭。鄴中富商丁鄒、儼興等並為義孝,有一士人,亦哭在限,孝琰入弔,出謂人曰:「嚴興之南,丁鄒之北,有一朝士,號叫甚哀。」



 聞者傳之。士
 開知而大怒。其後會黃門郎李懷奏南陽王綽專恣,士開因譖之曰:「孝琰從綽出外,乘其副馬,捨離部伍,別行戲話。」時孝琰女為范陽王妃,為禮事因假入辭,帝遂決馬鞭百餘。放出,又遣高阿那肱重決五十,幾致於死。還京,在集書省上下,從是沉廢。士開死後,為通直散騎常侍。後與周朝通好,趙彥深奏之,詔以為聘周使副。祖珽輔政,又奏令入文林館,撰《御覽》。孝琰文筆不高,但以風流自立,善於談謔,威儀閑雅,容止進退,人皆慕之。嘗謂
 祖珽云:「公是衣冠宰相,異於餘人。」近習聞之,大以為恨。



 尋以本官為尚書左丞,其所彈射,多承意旨。時有道人曇獻者,為皇太后所幸,賞賜隆厚,車服過度。又乞為沙門統,後主意不許,但太后欲之,遂得居任,然後主常憾焉。因有僧尼以他事訴競者,辭引曇獻。上令有司推劾。孝琰案其受納貨賄,致於極法,因搜索其家,大獲珍異,悉以沒官。由是正授左丞,仍令奏門下事。性頗簡傲,不諧時俗,恩遇漸高,彌自矜誕,舉動舒遲,無所降屈,識者
 鄙之。與崔季舒等以正諫同死,時年五十一。子開府行參軍君確、君靜等二人徙北邊,少子君嚴、君贊下蠶室。南安之敗,君確二人皆坐死。



 史臣曰:高、封二公,無一人尺土之資,奮臂而起河朔,將致勤王之舉,以雪莊帝仇,不亦壯哉!既剋本藩,成其讓德,異夫韓馥懾袁紹之威。然力謝時雄,才非命世,是以奉迎麾掞,用葉本圖。高祖因之,遂成霸業。重以昂之膽力,氣冠萬物,韓陵之下,風飛電擊。然則齊氏元功,一
 門而已。但以非潁川元從,異豐沛故人,腹心之寄,有所未允。露其啟疏,假手天誅,枉濫之極,莫過於此。子繪才幹可稱,克荷堂構,弈世載德,斯為美焉。



 贊曰:烈烈文昭,雄圖斯契。灼灼忠武,英資冠世。門下之酷,進退惟穀。黃河之濱,蹈義亡身。封公矯矯,共濟時屯。比承明德,暉光日新。



\end{pinyinscope}