\article{卷十八列傳第十孫騰 高隆之 司馬子如}

\begin{pinyinscope}

 孫騰,字龍雀,咸陽石安人也。祖通,仕沮渠氏為中書舍
 人,沮渠滅,入魏,因居北邊。及騰貴,魏朝贈通使持節、侍中、都督雍華岐幽四州諸軍事、驃騎大將軍、司徒公、尚書左僕射、雍州刺史,贈騰父機使持節、侍中、都督冀定滄瀛殷五州諸軍事、太尉公、尚書令、冀州刺史。



 騰少而質直,明解吏事。魏正光中,北方擾亂,騰間關危險,得達秀容。屬爾朱榮建義,騰隋榮入洛,例除冗從僕射。尋為高祖都督府長史,從高祖東征邢杲。



 師次齊城,有撫宜鎮軍人謀逆,將害督帥。騰知之,密啟高祖。俄頃事發,高
 祖以有備,擒破之。高祖之為晉州,騰為長史,加後將軍,封石安縣伯。高祖自晉陽出滏口,行至襄垣,爾朱兆率眾追。高祖與兆宴飲於水湄,誓為兄弟,各還本營。明旦,兆復招高祖,高祖欲安其意,將赴之,臨上馬,騰牽衣止之。兆乃隔水肆罵,馳還晉陽。高祖遂東。及起義信都,騰以誠款,常預謀策。騰以朝廷隔絕,號令無所歸,不權有所立,則眾將沮散,苦請於高祖。高祖從之,遂立中興主。除侍中,尋加使持節、六州流民大都督、北道大行臺。高
 祖進軍於鄴,初留段榮守信都,尋遣榮鎮中山,仍令騰居守。及平鄴,授相州刺史,改封咸陽郡公,增邑通前一千三百戶,入為侍中。時魏京兆王愉女平原公主寡居,騰欲尚之,公主不許。侍中封隆之無婦,公主欲之,騰妒隆之,遂相間構。高祖啟免騰官,請除外任,俄而復之。



 騰以高祖腹心,入居門下,與斛斯椿同掌機密。椿既生異端,漸至乖謬。騰深見猜忌,慮禍及己,遂潛將十餘騎馳赴晉陽。高祖入討斛斯椿,留騰行并州事,又使騰為冀
 相殷定滄瀛幽安八州行臺僕射、行冀州事,復行相州事。天平初,入為尚書左僕射,內外之事,騰咸知之,兼司空、尚書令。時西魏遣將寇南兗,詔騰為南道行臺,率諸將討之。騰性尪怯,無威略,失利而還。又除司徒。初北境亂離,亡一女,及貴,遠加推訪,終不得,疑其為人婢賤。及為司徒,奴婢訴良者,不研虛實,率皆免之,願免千人,冀得其女。時高祖入朝,左右有言之者,高祖大怒,解其司徒。武定中,使於青州,括浮逃戶口,遷太保。初,博陵崔孝
 芬取貧家子賈氏以為養女,孝芬死,其妻元更適鄭伯猷,攜賈於鄭氏。賈有姿色,騰納之,始以為妾。其妻袁氏死,騰以賈有子,正以為妻,詔封丹陽郡君,復請以袁氏爵回授其女。



 違禮肆情,多此類也。



 騰早依附高祖,契闊艱危,勤力恭謹,深見信待。及高祖置之魏朝,寄以心腹,遂志氣驕盈,與奪由己,求納財賄,不知紀極。生官死贈,非貨不行,餚藏銀器,盜為家物,親狎小人,專為聚斂。在鄴與高岳、高隆之、司馬子如號為四貴,非法專恣,騰為
 甚焉。高祖屢加譴讓,終不悛改,朝野深非笑之。武定六年四月薨,時年六十八。贈使持節、都督冀定等五州諸軍事、冀州刺史、太師、開府、錄尚書事,謚曰文。天保初,以騰佐命,詔祭告其墓。皇建中,配享高祖廟庭。子鳳珍嗣。鳳珍庸常,武平中,卒於開府儀同三司。



 高隆之,字延興,本姓徐氏,云出自高平金鄉。父幹,魏白水郡守,為姑婿高氏所養,因從其姓。隆之貴,魏朝贈司徒公、雍州刺史。隆之後有參議之功,高祖命為從弟,仍
 云渤海蓚人。



 隆之身長八尺,美鬚髯,深沉有志氣。魏汝南王悅為司州牧,以為戶曹從事。



 建義初,釋褐員外散騎常侍,與行臺于暉出討羊侃於太山,暉引隆之為行臺郎中,又除給事中。與高祖深自結託。高祖之臨晉州,引為治中,行平陽郡事。從高祖起義山東,以為大行臺右丞。魏中興初,除御史中尉,領尚食典御。從高祖平鄴,行相州事。從破四胡於韓陵,太昌初,除驃騎大將軍、儀同三司。西魏文帝曾與隆之因酒忿競,文帝坐以黜免。
 高祖責隆之不能協和,乃啟出為北道行臺,轉并州刺史,封平原郡公,邑一千七百戶。隆之請減戶七百,并求降己四階讓兄騰,並加優詔許之,仍以騰為滄州刺史。高祖之討斛斯椿,以隆之為大行臺尚書。及大司馬、清河王亶承制,拜隆之侍中、尚書右僕射,領御史中尉。廣費人工,大營寺塔,為高祖所責。



 天平初,丁母艱解任,尋詔起為并州刺史,入為尚書右僕射。時初給民田,貴勢皆占良美,貧弱咸受瘠薄。隆之啟高祖,悉更反易,乃
 得均平。又領營構大將軍,京邑制造,莫不由之。增築南城,周迴二十五里。以漳水近於帝城,起長隄以防汎溢之患。又鑿渠引漳水周流城郭,造治水碾磑,並有利於時。魏自孝昌已後,天下多難,刺史太守皆為當部都督,雖無兵事,皆立佐僚,所在頗為煩擾。隆之表請自非實在邊要,見有兵馬者,悉皆斷之。又朝貴多假常侍以取貂蟬之飾,隆之自表解侍中,并陳諸假侍中服用者,請亦罷之。詔皆如表。自軍國多事,冒名竊官者不可勝數,隆
 之奏請檢括,獲五萬餘人,而群小喧囂,隆之懼而止。詔監起居事,進位司徒公。



 武定中,為河北括戶大使。追還,授領軍將軍、錄尚書事,尋兼侍中。續出行青州事。追還,拜太子太師、兼尚書左僕射、吏部尚書,遷太保。時世宗作宰,風俗肅清,隆之時有受納,世宗於尚書省大加責辱。齊受禪,進爵為王。尋以本官錄尚書事,領大宗正卿,監國史。隆之性小巧,至於公家羽儀、百戲、服制時有改易,不循典故,時論非之。於射堋上立三像人為壯勇之
 勢。顯祖曾至東山,因射,謂隆之曰:「射堋上可作猛獸,以存古義,何為置人?終日射人,朕所不取。」隆之無以對。



 初,世宗委任兼右僕射崔暹、黃門郎崔季舒等,及世宗崩,隆之啟顯祖並欲害之,不許。顯祖以隆之舊齒,委以政事,季舒等仍以前隙,乃譖云:「隆之每見訴訟者,輒加哀矜之意,以示非己能裁。」顯祖以其受任既重,知有冤狀,便宜申滌,何得委過要名,非大臣義。天保五年,禁止尚書省。隆之曾與元昶宴飲,酒酣,語昶曰:「與王交遊,當生
 死不相背。」人有密言之者。又帝未登庸之日,隆之意常侮帝。帝將受魏禪,大臣咸言未可,隆之又在其中。帝深銜之。因此,遂大發怒,令壯士築百餘下。放出,渴將飲水,人止之,隆之曰:「今日何在!」遂飲之。因從駕,死於路中,年六十一。贈冀定瀛滄幽五州諸軍事、大將軍、太尉、太保、冀州刺史、陽夏王。竟不得謚。



 隆之雖不涉學,而欽尚文雅,縉紳名流,必存禮接。寡姊為尼,事之如母,訓督諸子,必先文義。世甚以此稱之。顯祖末年,既多猜害,追忿隆
 之,誅其子德樞等十餘人,並投漳水。又發隆之塚,出其尸,葬已積年,其貌不改,斬截骸骨,亦棄於漳流,遂絕嗣。乾明中,詔其兄子子遠為隆之後,襲爵陽夏王,還其財產。初,隆之見信高祖,性多陰毒,睚眥之忿,無不報焉。儀同三司崔孝芬以結婚姻不果,太府卿任集同知營構,頗相乖異,瀛州刺史元晏請託不遂,前後構成其罪,並誅害之。終至家門殄滅,論者謂有報應焉。



 司馬子如,字遵業,河內溫人也。八世祖模,晉司空、南陽
 王。模世子保,晉亂出奔涼州,因家焉。魏平姑臧,徙居於雲中,其自序云爾。父興龍,魏魯陽太守。



 子如少機警,有口辯。好交遊豪傑,與高祖相結託,分義甚深。孝昌中,北州淪陷,子如攜家口南奔肆州,為爾朱榮所禮遇,假以中軍。榮之向洛也,以子如為司馬、持節、假平南將軍,監前軍。次高都,榮以建興險阻,往來衝要,有後顧之憂,以子如行建興太守、當郡都督。永安初,封平遙縣子,邑三百戶,仍為大行臺郎中。榮以子如明辯,能說時事,數遣
 奉使詣闕,多稱旨,孝莊亦接待焉。葛榮之亂,相州孤危,榮遣子如間行入鄴,助加防守。葛榮平,進爵為侯。元顥入洛,人情離阻,以子如曾守鄴城,頗有恩信,乃令行相州事。顥平,徵為金紫光祿大夫。



 爾朱榮之誅,子如知有變,自宮內突出,至榮宅,棄家隨榮妻子與爾朱世隆等走出京城。世隆便欲還北,子如曰:「事貴應機,兵不厭詐,天下恟恟,唯彊是視,於此際會,不可以弱示人。若必走北,即恐變故隨起,不如分兵守河橋,迴軍向京,出其不
 意,或可離潰。假不如心,猶足示有餘力,使天下觀聽,懼我威彊。」於是世隆還逼京城。魏長廣王立,兼尚書右僕射。前廢帝以為侍中、驃騎大將軍、儀同三司,進爵陽平郡公,邑一千七百戶。固讓儀同不受。高祖起義信都,世隆等知子如與高祖有舊,疑慮,出為南岐州刺史。子如憤恨,泣涕自陳,而不獲免。高祖入洛,子如遣使啟賀,仍敘平生舊恩。尋追赴京,以為大行臺尚書,朝夕左右,參知軍國。天平初,除左僕射,與侍中高岳、侍中孫騰、右僕
 射高隆之等共知朝政,甚見信重。高祖鎮晉陽,子如時往謁見,待之甚厚,並坐同食,從旦達暮,及其當還,高祖及武明后俱有賚遺,率以為常。



 子如性既豪爽,兼恃舊恩,簿領之務,與奪任情,公然受納,無所顧憚。興和中,以為北道行臺,巡檢諸州,守令已下,委其黜陟。子如至定州,斬深澤縣令;至冀州,斬東光縣令。皆稽留時漏,致之極刑。若言有進退,少不合意,便令武士頓曳,白刃臨項。士庶惶懼,不知所為。轉尚書令。子如義旗之始,身不參
 預,直以高祖故舊,遂當委重,意氣甚高,聚斂不息。時世宗入輔朝政,內稍嫌之,尋以贓賄為御史中尉崔暹所劾,禁止於尚書省。詔免其大罪,削官爵。未幾,起行冀州事。子如能自厲改,甚有聲譽,發摘姦偽,僚吏畏伏之。轉行并州事。詔復官爵,別封野王縣男,邑二百戶。



 齊受禪,以有翼贊之功,別封須昌縣公,尋除司空。子如性滑稽,不治檢裁,言戲穢褻,識者非之。而事姊有禮,撫諸兄子慈篤,當時名士並加欽愛,世以此稱之。然素無鯁正,不
 能平心處物。世宗時,中尉崔暹、黃門郎崔季舒俱被任用。世宗崩,暹等赴晉陽。子如乃啟顯祖,言其罪惡,仍勸誅之。其後子如以馬度關,為有司所奏。顯祖引子如數讓之曰:「崔暹、季舒事朕先世,有何大罪,卿令我殺之?」



 因此免官。久之,猶以先帝之舊,拜太尉。尋以疾薨,時年六十四。贈使持節、都督冀定瀛滄懷五州諸軍事、太師、太尉、懷州刺史,贈物一千段,謚曰文明。



 子消難嗣。尚高祖女,以主婿、貴公子,頻歷中書、黃門郎、光祿少卿。出為北
 豫州刺史,鎮武牢。消難博涉史傳,有風神,然不能廉潔,在州為御史所劾。又於公主情好不睦,公主譖訴之,懼罪,遂招延鄰敵,走關西。



 子如兄纂,先卒,子如貴,贈岳州刺史。纂長子世雲,輕險無行,累遷衛將軍、潁州刺史。世雲本無勳業,直以子如故,頻歷州郡。恃叔之勢,所在聚斂,仍肆姦穢。將見推治,內懷驚懼,侯景反,遂舉州從之。時世雲母弟在鄴,便傾心附景,無復顧望。諸將圍景於潁川,世雲臨城遙對諸將,言甚不遜。世宗猶以子如恩
 舊,免其諸弟死罪,徙於北邊。侯景於渦陽敗後,世雲復有異志,為景所殺。



 世雲弟膺之,字仲慶。少好學,美風儀。天平中,子如貴盛,膺之自尚書郎歷中書、黃門郎。子如別封須昌縣公,迴授膺之。膺之家富於財,厚自封殖。王元景、邢子才之流以夙素重之。以其疏簡傲物,竟天保世,淪滯不齒。乾明中,王晞白肅宗,除衛尉少卿。河清末,光祿大夫。患泄利,積年不起,至武平中,猶不堪朝謁,就家拜儀同三司。好讀《太玄經》,注揚雄《蜀都賦》。每云:「
 我欲與揚子雲周旋。」齊亡歲,以利疾終,時年七十一。



 膺之弟子瑞,天保中為定州長史,遷吏部郎中。舉清勤平約。遷司徒左長史,兼廷尉卿,以平直稱。乾明初,領御史中丞,正色舉察,為朝廷所許。以疾去職,就拜祠部尚書。卒,贈瀛州刺史,謚曰文節。



 子瑞弟幼之,清貞有素行,少歷顯位。隋開皇中,卒於眉州刺史。子瑞妻,令萱之妹,及令萱得寵於後主,重贈子瑞懷州剌史,諸子亦並居顯職。同遊,武平末給事黃門侍郎。同回,太府卿。同憲,通直
 常侍。然同遊終為嘉吏,隋開皇中尚書民部侍郎,卒於遂州刺史。



 史臣曰:高祖以晉陽戎馬之地,霸圖攸屬,治兵訓旅,遙制朝權,京臺機務,委寄深遠。孫騰等俱不能清貞守道,以治亂為懷,厚斂貨財,填彼溪壑。昔蕭何之鎮關中,荀彧之居許下,不亦異於是乎!賴世宗入輔,責以驕縱,厚遇崔暹,奮其霜簡,不然則君子屬厭,豈易間焉。孫騰牽裾之誠,有足稱美。隆之勞其志力,經始鄴京,又並是潛
 德僚寀,早申任遇,崇其名器,未失朝序。子如徒以少相親重,情深暱狎,義非草昧,恩結寵私,勛德莫聞,坐致臺輔。猶子之愛,訓以義方,膺之風素可重,幼之清簡自立,有足稱也。



 贊曰:閎、散胥附,蕭、曹扶翼。齊運勃興,孫、高陳力。黷貨無厭,多慚袞職。司馬滑稽,巧言令色。



\end{pinyinscope}