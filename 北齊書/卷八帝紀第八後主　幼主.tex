\article{卷八帝紀第八後主 幼主}

\begin{pinyinscope}

 後主諱緯,字仁綱,武成皇帝之長子也。母曰胡皇后,夢於海上坐玉盆,日入裙下,遂有娠。天保七年五月五日,
 生帝於并州邸。帝少美容儀,武成特所愛寵,拜王世子。及武成入纂大業,大寧二年正月丙戌,立為皇太子。河清四年,武成禪位於帝。



 天統元年夏四月丙子,皇帝即位於晉陽宮,大赦,改河清四年為天統。丁丑,以太保賀拔仁為太師,太尉侯莫陳相為太保,司空、馮翊王潤為司徒,錄尚書事、趙郡王睿為司空,尚書左僕射、河間王孝琬為尚書令。戊寅,以瀛州刺史尉粲為太尉,斛律光為大將軍,東安王婁睿
 為太尉,尚書右僕射趙彥深為左僕射。六月壬戌,彗星出文昌東北,其大如手,後稍長,乃至丈餘,百日乃滅。己巳,太上皇帝詔兼散騎常侍王季高使於陳。秋七月乙未,太上皇帝詔增置都水使者一人。冬十一月癸未,太上皇帝至自晉陽。己丑,太上皇帝詔改「太祖獻武皇帝」為「神武皇帝」,廟號「高祖」,「獻明皇后」為「武明皇后」;其「文宣」謚號委有司議定。十二月庚戌,太上皇帝狩於北郊。壬子,狩於南郊。乙卯,狩於西郊。壬戌,太上皇帝幸晉陽。丁
 卯,帝至自晉陽。庚午,有司奏改「高祖文宣皇帝」為「威宗景烈皇帝。」



 是歲,高麗、契丹、靺鞨並遣使朝貢。河南大疫。



 二年丙戌春正月辛卯,祀圓丘。癸巳,袷祭於太廟,詔降罪人各有差。丙申,以吏部尚書尉瑾為尚書右僕射。庚子,行幸晉陽。二月庚戌,太上皇帝至自晉陽。



 壬子,陳人來聘。三月乙巳,太上皇帝詔以三臺施興聖寺。以旱故,降禁囚。夏四月,陳文帝殂。五月乙酉,以兼尚書左僕射、武興王普為尚書令。己亥,封太上皇帝子儼為東平王,
 仁弘為齊安王,仁堅為北平王,仁英為高平王,仁光為淮南王。



 六月,太上皇帝詔兼散騎常侍韋道儒聘於陳。秋八月,太上皇帝幸晉陽。冬十月乙卯,以太保侯莫陳相為太傅,大司馬、任城王湝為太保,太尉婁睿為大司馬,徙馮翊王潤為太尉,開府儀同三司韓祖念為司徒。十一月,大雨雪,盜竊太廟御服。十二月乙丑,陳人來聘。是歲,殺河間王孝琬。突厥、靺鞨國並遣使朝貢。於周為天和元年。



 三年春正月壬辰,太上皇帝至自晉陽。乙未,大雪,平地二尺。戊戌,太上皇帝詔京官執事散官三品已上各舉三人,五品已上各舉二人;稱事七品已上及殿中侍御史、尚書都檢校御史、主書及門下錄事各舉一人。鄴宮九龍殿災,延燒西廊。二月壬寅朔,帝加元服,大赦,九州職人各進四級,內外百官普進二級。夏四月癸丑,太上皇帝詔兼散騎常侍司馬幼之使於陳。五月甲午,太上皇帝詔以領軍大將軍、東平王儼為尚書令。乙未,大風
 晝晦,發屋拔樹。六月己未,太上皇帝詔封皇子仁幾為西河王,仁約為樂浪王,仁儉為潁川王,仁雅為安樂王,仁統為丹陽王,仁謙為東海王。閏六月辛巳,左丞相斛律金薨。壬午,太上皇帝詔尚書令、東平王儼錄尚書事,以尚書左僕射趙彥深為尚書令,并省尚書左僕射婁定遠為尚書左僕射,中書監徐之才為右僕射。秋八月辛未,太上皇帝詔以太保、任城王湝為太師,太尉、馮翊王潤為大司馬,太宰段韶為左丞相,太師賀拔仁為右
 丞相,太傅侯莫陳相為太宰,大司馬婁睿為太傅,大將軍斛律光為太保,司徒韓祖念為大將軍,司空、趙郡王睿為太尉,尚書令、東平王儼為司徒。九月己酉,太上皇帝詔:「諸寺署所綰雜保戶姓高者,天保之初雖有優敕,權假力用未免者,今可悉蠲雜戶,任屬郡縣,一準平人。」丁巳,太上皇帝幸晉陽。是秋,山東大水,人饑,僵尸滿道。冬十月,突厥、大莫婁、室韋、百濟、靺鞨等國各遣使朝貢。十一月丙午,以晉陽大明殿成故,大赦,文武百官進二
 級,免并州居城、太原一郡來年租賦。癸未,太上皇帝至自晉陽。



 十二月己巳,太上皇帝詔以故左丞相、趙郡王琛配饗神武廟庭。



 四年正月,詔以故清河王岳、河東王潘相樂十人並配饗神武廟庭。癸亥,太上皇帝詔兼散騎常侍鄭大護使於陳。三月乙巳,太上皇帝詔以司徒、東平王儼為大將軍,南陽王綽為司徒,開府儀同三司徐顯秀為司空,開府儀同三司、廣寧王孝珩為尚書令。夏四月辛未,鄴宮
 昭陽殿災,及宣光、瑤華等殿。辛巳,太上皇帝幸晉陽。



 五月癸卯,以尚書右僕射胡長仁為左僕射,中書監和士開為右僕射。壬戌,太上皇帝至自晉陽。自正月不雨至於是月。六月甲子朔,大雨。甲申,大風,拔木折樹。



 是月,彗星見于東井。秋九月丙申,周人來通和,太上皇帝詔侍中斛斯文略報聘于周。冬十月辛巳,以尚書令、廣寧王孝珩為錄尚書,左僕射胡長仁為尚書令,右僕射和士開為左僕射,中書監唐邕為右僕射。十一月壬辰,太上
 皇帝詔兼散騎常侍李翥使於陳。是月,陳安成王頊廢其主伯宗而自立。十二月辛未,太上皇帝崩。丙子,大赦,九州職人普加四級,內外百官並加兩級。戊寅,上太上皇后尊號為皇太后。



 甲申,詔細作之務及所在百工悉罷之。又詔掖庭、晉陽、中山官人等及鄴下、并州太官官口二處,其年六十已上及有癃患者,仰所司簡放。庚寅,詔天保七年已來諸家緣坐配流者,所在令還。是歲,契丹、靺鞨國並遣使朝貢。



 五年春正月辛亥,詔以金鳳等三臺未入寺者施大興聖寺。是月,殺定州刺史、博陵王濟。二月乙丑,詔應宮刑者普免刑為官口。又詔禁網捕鷹鷂及畜養籠放之物。



 癸酉,大莫婁國遣使朝貢。己丑,改東平王儼為琅邪王。詔侍中叱列長叉使於周。



 是月,殺太尉、趙郡王睿。三月丁酉,以司空徐顯秀為太尉,并省尚書令婁定遠為司空。是月,行幸晉陽。夏四月甲子,詔以并州尚書省為大基聖寺,晉祠為大崇皇寺。乙丑,車駕至自晉陽。秋七月
 己丑,詔降罪人各有差。戊申,詔使巡省河北諸州無雨處,境內偏旱者優免租調。冬十月壬戌,詔禁造酒。十一月辛丑,詔以太保斛律光為太傅,大司馬、馮翊王潤為太保,大將軍、琅邪王儼為大司馬。十二月庚午,以開府儀同三司、蘭陵王長恭為尚書令。庚辰,以中書監魏收為尚書右僕射。



 武平元年春正月乙酉朔,改元。太師、并州刺史、東安王婁睿薨。戊申,詔兼散騎常侍裴獻之聘于陳。二月癸亥,
 以百濟王餘昌為使持節、侍中、驃騎大將軍、帶方郡公,王如故。己巳,以太傅、咸陽王斛律光為右丞相,并州刺史、右丞相、安定王賀拔仁為錄尚書事,冀州刺史、任城王湝為太師。丙子,降死罪已下囚。閏月戊戌,隸尚書事、安定王賀拔仁薨。三月辛酉,以開府儀同三司徐之才為尚書左僕射。夏六月乙酉,以廣寧王孝珩為司空。甲辰,以皇子恒生故,大赦,內外百官普進二級,九州職人普進四級。己酉,詔以開府儀同三司唐邕為尚書右僕
 射。秋七月癸丑,封孝昭皇帝子彥基為城陽王,彥康為定陵王,彥忠為梁郡王。甲寅,以尚書令、蘭陵王長恭為錄尚書事,中領軍和士開為尚書令。癸亥,靺鞨國遣使朝貢。



 癸酉,以華山王凝為太傅。八月辛卯,行幸晉陽。九月乙巳,立皇子恒為皇太子。



 冬十月辛巳,以司空、廣寧王孝珩為司徒,以上洛王思宗為司空,封蕭莊為梁王。



 戊子,曲降并州死罪已下囚。己丑,復改威宗景烈皇帝謚號為「顯祖文宣皇帝。」



 十二月丁亥,車駕至自晉陽。詔
 右丞相斛律光出晉州道,修城戍。



 二年春正月丁巳,詔兼散騎常侍劉環俊使於陳。戊寅,以百濟王餘昌為使持節、都督、東青州刺史。二月壬寅,以錄尚書事、蘭陵王長恭為太尉,并省錄尚書事趙彥深為司空,尚書令和士開錄尚書事,左僕射徐之才為尚書令,右僕射唐邕為左僕射,吏部尚書馮子琮為右僕射。夏四月壬午,以大司馬、琅邪王儼為太保。甲午,陳遣使連和,謀伐周,朝議弗許。六月,段韶攻周汾州,剋之,
 獲刺史楊敷。秋七月庚午,太保、琅邪王儼矯詔殺錄尚書事和土開於南臺。即日誅領軍大將軍厙狄伏連、書侍御史王子宣等,尚書右僕射馮子琮賜死殿中。八月己亥,行幸晉陽。九月辛亥,以太師、任城王湝為太宰,馮翊王潤為太師。己未,左丞相、平原王段韶薨。



 戊午,曲降并州界內死罪已下各有差。庚午,殺太保、琅邪王儼。壬申,陳人來聘。



 冬十月,罷京畿府入領軍府。己亥,車駕至自晉陽。十一月庚戌,詔侍中赫連子悅使於周。丙寅,以
 徐州行臺、廣寧王孝珩錄尚書事。庚午,以錄尚書事、廣寧王孝珩為司徒。癸酉,以右丞相斛律光為左丞相。



 三年春正月己巳,祀南郊。辛亥,追贈故琅邪王儼為楚王。二月己卯,以衛菩薩為太尉。辛巳,以并省吏部尚書元海為尚書右僕射。庚寅,以左僕射唐邕為尚書令,侍中祖珽為左僕射。是月,敕撰《玄洲苑御覽》,後改名《聖壽堂御覽》。三月辛酉,詔文武官五品已上各舉一人。是月,周誅冢宰宇文護。夏四月,周人來聘。



 秋七月戊辰,誅
 左丞相、咸陽王斛律光及其弟幽州行臺、荊山公豐樂。八月庚寅,廢皇后斛律氏為庶人。以太宰、任城王湝為右丞相,太師、馮翊王潤為太尉,蘭陵王長恭為大司馬,廣寧王孝珩為大將軍,安德王延宗為司徒。使領軍封輔相聘于周。



 戊子,拜右昭儀胡氏為皇后。己丑,以司州牧、北平王仁堅為尚書令,特進許季良為左僕射,彭城王寶德為右僕射。癸巳,行幸晉陽。是月,《聖壽堂禦覽》成,敕付史閣。後改為《修文殿御覽》。九月,陳人來聘。冬十
 月,降死罪已下囚。甲午,拜弘德夫人穆氏為左皇后,大赦。十二月辛丑,廢皇后胡氏為庶人。是歲,新羅、百濟、勿吉、突厥並遣使朝貢。於周為建德元年。



 四年春正月戊寅,以并省尚書令高阿那肱為錄尚書事。庚辰,詔兼散騎常侍崔象使於陳。是月,鄴都、并州並有狐媚,多截人髮。二月乙巳,拜左皇后穆氏為皇后。丙午,置文林館。乙卯,以尚書令、北平王仁堅為錄尚書事。丁巳,行幸晉陽。



 是月,周人來聘。三月辛未,盜入信州,殺
 刺史和士休,南兗州刺史鮮于世榮討平之。庚辰,車駕至晉陽。夏四月戊午,以大司馬、蘭陵王長恭為太保,大將軍、定州刺史、南陽王綽為大司馬,太尉衛菩薩為大將軍,司徒、安德王延宗為太尉,司空、武興王普為司徒,開府儀同三司、宜陽王趙彥深為司空。癸丑,祈皇祠壇壝蕝之內忽有車軌之轍,按驗傍無人跡,不知車所從來。乙卯,詔以為大慶,班告天下。



 己未,周人來聘。五月丙子,詔史官更撰《魏書》。癸巳,以領軍穆提婆為尚
 書左僕射,以侍中、中書監段孝言為右僕射。是月,開府儀同三司尉破胡、長孫洪略等與陳將吳明徹戰於呂梁南,大敗,破胡走以免,洪略戰沒,遂陷秦、涇二州。明徹進陷和、合二州。是月,殺太保、蘭陵王長恭。六月,明徹進軍圍壽陽。壬子,幸南苑,從官暍死者六十人。以錄尚書事高阿那肱為司徒。丙辰,詔開府王師羅使於周。九月,校獵于鄴東。冬十月,陳將吳明徹陷壽陽。辛丑,殺侍中崔季舒、張彫虎,散騎常侍劉逖、封孝琰,黃門侍郎裴澤、
 郭遵。癸卯,行幸晉陽。十二月戊寅,以司徒高阿那肱為右丞相。是歲,高麗、靺鞨並遣使朝貢,突厥使來求婚。



 五年春正月乙丑,置左右娥英各一人。二月乙未,車駕至自晉陽。朔州行臺、南安王思好反。辛丑,行幸晉陽。尚書令唐邕等大破思好,思好投水死,焚其屍,并其妻李氏。丁未,車駕至自晉陽。甲寅,以尚書令唐邕為錄尚書事。夏五月,大旱,晉陽得死魃,長二尺,面頂各二目。帝聞之,便刻木為其形以獻。庚午,大赦。



 丁亥,陳人寇淮北。秋
 八月癸卯,行幸晉陽。甲辰,以高勱為尚書右僕射。是歲,殺南陽王綽。



 六年春三月乙亥,車駕至自晉陽。丁丑,烹妖賊鄭子饒於都市。是月,周人來聘。夏四月庚子,以中書監陽休之為尚書右僕射。癸卯,靺鞨遣使朝貢。秋七月甲戌,行幸晉陽。八月丁酉,冀、定、趙、幽、滄、瀛六州大水。是月,周師入洛川,屯芒山,攻逼洛城,縱火船焚浮橋,河橋絕。閏月己丑,遣右丞相高阿那肱自晉陽禦之,師次河陽,周師夜
 遁。庚辰,以司空趙彥深為司徒,斛律阿列羅為司空。辛巳,以軍國資用不足,稅關市、舟車、山澤、鹽鐵、店肆,輕重各有差,開酒禁。



 七年春正月壬辰,詔去秋已來,水潦人饑不自立者,所在付大寺及諸富戶濟其性命。甲寅,大赦。乙卯,車駕至自晉陽。二月辛酉,括雜戶女年二十已下十四已上未嫁悉集省,隱匿者家長處死刑。二月丙寅,風從西北起,發屋拔樹,五日乃止。



 夏六月戊申朔,日有食之。庚申,司
 徒趙彥深薨。秋七月丁丑,大雨霖。是月,以水澇遣使巡撫流亡人戶。八月丁卯,行幸晉陽。雉集於御坐,獲之,有司不敢以聞。



 詔營邯鄲宮。冬十月丙辰,帝大狩於祁連池。周師攻晉州。癸亥,帝還晉陽。甲子,出兵,大集晉祠。庚午,帝發晉陽。癸酉,帝列陣而行,上雞棲原,與周齊王憲相對,至夜不戰,周師斂陣而退。十一月,周武帝退還長安,留偏師守晉州。高阿那肱等圍晉州城。戊寅,帝至圍所。十二月戊申,周武帝來救晉州,庚戌,戰于城南,我軍大
 敗。帝棄軍先還。癸丑,入晉陽,憂懼不知所之。甲寅,大赦。帝謂朝臣曰:「周師甚盛,若何?」群臣咸曰:「天命未改,一得一失,自古皆然。宜停百賦,安慰朝野,收拾遺兵,背城死戰,以存社稷。」帝意猶豫,欲向北朔州。乃留安德王延宗、廣寧王孝珩等守晉陽。若晉陽不守,即欲奔突厥。群臣皆曰不可,帝不從其言。開府儀同三司賀拔伏恩、封輔相、慕容鐘葵等宿衛近臣三十餘人西奔周師。



 乙卯,詔募兵,遣安德王廷宗為左,廣寧王孝珩為右。延宗入見,
 帝告欲向北朔州。



 延宗泣諫,不從。帝密遣王康德與中人齊紹等送皇太后、皇太子於北朔州。丙辰,帝幸城南軍,勞將士,其夜欲遁,諸將不從。丁巳,大赦,改武平七年為隆化元年。



 其日,穆提婆降周。詔除安德王延宗為相國,委以備禦,延宗流涕受命。帝乃夜斬五龍門而出,欲走突厥,從官多散。領軍梅勝郎叩馬諫,乃回之鄴。時唯高阿那肱等十餘騎,廣寧王孝珩、襄城王彥道續至,得數十人同行。戊午,延宗從眾議即皇帝位於晉陽,改隆
 化為德昌元年。



 庚申,帝入鄴。幸酉,延宗與周師戰於晉陽,大敗,為周師所虜。帝遣募人,重加官賞,雖有此言,而竟不出物。廣寧王孝珩奏請出宮人及珍寶班賜將士,帝不悅。斛律孝卿居中受委,帶甲以處分,請帝親勞,為帝撰辭,且曰宜慷慨流涕,感激人心。帝既出臨眾,將令之,不復記所受言,遂大笑,左右亦群咍,將士莫不解體。於是自大丞相已下太宰、三師、大司馬、大將軍、三公等官並增員而授,或三或四,不可勝數。甲子,皇太后從北
 道至。引文武一品已上入朱華門,賜酒食,給紙筆,問以禦周之方。群臣各異議,帝莫知所從。又引高元海、宋士素、盧思道、李德林等,欲議禪位皇太子。先是望氣者言,當有革易,終是依天統故事,授位幼主。



 幼主名恒,帝之長子也。母曰穆皇后,武平元年六月生於鄴。其年十月,立為皇太子。隆化二年春正月乙亥,即皇帝位,時八歲,改元為承光元年,大赦,尊皇太后為太皇太后,帝為太上皇帝,后為太上皇后。於是黃門侍郎
 顏之推、中書侍郎薛道衡、侍中陳德信等勸太上皇帝往河外募兵,更為經略,若不濟,南投陳國,從之。丁丑,太皇太后、太上皇后自鄴先趣濟州。周師漸逼,癸未,幼主又自鄴東走。



 己丑,周師至紫陌橋。癸巳,燒城西門。太上皇將百餘騎東走。乙亥,渡河入濟州。



 其日,幼主禪位於大丞相、任城王湝,令侍中斛律孝卿送禪文及璽紱於瀛州,孝卿乃以之歸周。又為任城王詔,尊太上皇為無上皇,幼主為守國天王。留太皇太后濟州,遣高阿那肱留
 守。太上皇并皇后攜幼主走青州,韓長鸞、鄧顒等數十人從。太上皇既至青州,即為入陳之計。而高阿那肱召周軍,約生致齊主,而屢使人告言,賊軍在遠,已令人燒斷橋路。太上所以停緩。周軍奄至青州,太上窘急,將遜於陳,置金囊於鞍後,與長鸞、淑妃等十數騎至青州南鄧村,為周將尉遲綱所獲。送鄴,周武帝與抗賓主禮,并太后、幼主、諸王俱送長安,封帝溫國公。至建德七年,誣與宜州刺史穆提婆謀反,及延宗等數十人無少長咸
 賜死,神武子孫所存者一二而已。



 至大象末,陽休之、陳德信等啟大丞相隋公,請收葬,聽之,葬長安北原洪瀆川。



 帝幼而令善,及長,頗學綴文,置文林館,引諸文士焉。而言語澀吶,無志度,不喜見朝士。自非寵私暱狎,未嘗交語,性懦不堪,人視者,即有忿責。其奏事者,雖三公令錄莫得仰視,皆略陳大旨,驚走而出。每災異寇盜水旱,亦不貶損,唯諸處設齋,以此為脩德。雅信巫覡,解禱無方。初,琅邪王舉兵,人告者誤云厙狄伏連反,帝曰:「此必
 仁威也。」又斛律光死後,諸武官舉高思好堪大將軍,帝曰:「思好喜反。」皆如所言。遂自以策無遺算,乃益驕縱。盛為無愁之曲,帝自彈胡琵琶而唱之,侍和之者以百數。人間謂之無愁天子。嘗出見群厲,盡殺之,或剝人面皮而視之。任陸令萱、和士開、高阿那肱、穆提婆、韓長鸞等宰制天下,陳德信、鄧長顒、何洪珍參預機權。各引親黨,超居非次,官由財進,獄以賄成,其所以亂政害人,難以備載。諸宮奴婢、閹人、商人、胡戶、雜戶、歌舞人、見鬼人濫
 得富貴者將萬數,庶姓封王者百數,不復可紀。開府千餘,儀同無數。領軍一時二十,連判文書,各作依字,不具姓名,莫知誰也。諸貴寵祖禰追贈官,歲一進,位極乃止。宮掖婢皆封郡君,宮女寶衣玉食者五百餘人,一裙直萬匹,鏡臺直千金,競為變巧,朝衣夕弊。承武成之奢麗,以為帝王當然。乃更增益宮苑,造偃武脩文臺,其嬪嬙諸宮中起鏡殿、寶殿、瑇瑁殿,丹青彫刻,妙極當時。又於晉陽起十二院,壯麗逾於鄴下。所愛不恒,數毀而又復。
 夜則以火照作,寒則以湯為泥,百工困窮,無時休息。鑿晉陽西山為大佛像,一夜然油萬盆,光照宮內。又為胡昭儀起大慈寺,未成,改為穆皇后大寶林寺,窮極工巧,運石填泉,勞費億計,人牛死者不可勝紀。



 御馬則藉以氈罽,食物有十餘種,將合牝牡,則設青廬,具牢饌而親觀之。狗則飼以粱肉。馬及鷹犬乃有儀同、郡君之號,故有赤彪儀同、逍遙郡君、凌霄郡君,高思好書所謂「駮龍、逍遙」者也。犬於馬上設褥以抱之,鬥雞亦號開府,犬馬
 雞鷹多食縣幹。鷹之入養者,稍割犬肉以飼之,至數日乃死。又於華林園立貧窮村舍,帝自弊衣為乞食兒。又為窮兒之市,躬自交易。嘗築西鄙諸城,使人衣黑衣為羌兵,鼓噪凌之,親率內參臨拒,或實彎弓射人。自晉陽東巡,單馬馳騖,衣解髮散而歸。



 又好不急之務,曾一夜索歇,及旦得三升。特愛非時之物,取求火急,皆須朝征夕辦,當勢者因之,貸一而責十焉。賦斂日重,徭役日繁,人力既殫,幣藏空竭。



 乃賜諸佞幸賣官。或
 得郡兩三,或得縣六七,各分州郡,下逮鄉官亦多降中旨,故有敕用州主簿,敕用郡功曹。於是州縣職司多出富商大賈,競為貪縱,人不聊生。



 爰自鄴都及諸州郡,所在征稅,百端俱起。凡此諸役,皆漸於武成,至帝而增廣焉。



 然未嘗有帷薄淫穢,唯此事頗優於武成云。初,河清末,武成夢大胃攻破鄴城,故索境內膏以絕之。識者以後主名聲與胃相協,亡齊徵也。又婦人皆剪剔以著假髻,而危邪之狀如飛鳥,至於南面,則髻心正西。始自宮內為之,被
 於四遠,天意若曰元首剪落,危側當走西也。又為刀子者刃皆狹細,名曰盡勢。遊童戲者好以兩手持繩,拂地而卻上跳,且唱曰「高末」,高未之言,蓋高氏運祚之末也。然則亂亡之數蓋有兆云。



 論曰:武成風度高爽,經算弘長,文武之官,俱盡其力,有帝王之量矣。但愛狎庸豎,委以朝權,帷薄之間,淫侈過度,滅亡之兆,其在斯乎?玄象告變,傳位元子,名號雖殊,政猶己出,跡有虛飾,事非憲典,聰明臨下,何易可誣。又
 河南、河間、樂陵等諸王,或以時嫌,或以猜忌,皆無罪而殞,非所謂知命任天道之義也。



 後主以中庸之姿,懷易染之性,永言先訓,教匪義方。始自襁褓,至于傳位,隔以正人,閉其善道。養德所履,異乎春誦夏弦;過庭所聞,莫非不軌不物。輔之以中宮妳媼,屬之以麗色淫聲,縱韝紲之娛,恣朋淫之好。語曰「從惡若崩」,蓋言其易。武平在御,彌見淪胥,罕接朝士,不親政事,一日萬機,委諸凶族。內侍帷幄,外吐絲綸,威厲風霜,志迴天日,虐人害物,
 搏噬無厭,賣獄鬻官,溪壑難滿。重以名將貽禍,忠臣顯戮,始見浸弱之萌,俄觀土崩之勢,周武因機,遂混區夏,悲夫!蓋桀、紂罪人,其亡也忽焉,自然之理矣。



 鄭文貞公魏徵總而論之曰:神武以雄傑之姿,始基霸業;文襄以英明之略,伐叛柔遠。于時喪君有君,師出以律。河陰之役,摧宇文如反掌;渦陽之戰,掃侯景如拉枯。故能氣攝西鄰,威加南服,王室是賴,東夏宅心。文宣因累世之資,膺樂推之會,地居當璧,遂遷魏鼎。懷譎詭非
 常之才,運屈奇不測之智,網羅俊乂,明察臨下,文武名臣,盡其力用。親戎出塞,命將臨江,定單于於龍城,納長君於梁國,外內充實,疆埸無警,胡騎息其南侵,秦人不敢東顧。既而荒淫敗德,罔念作狂,為善未能亡身,餘殃足以傳後。得以壽終,幸也,胤嗣不永,宜哉。孝昭地逼身危,逆取順守,外敷文教,內蘊雄圖,將以牢籠區域,奄一函夏,享齡不永,勣用無成。若或天假之年,足使秦、吳旰食。武成即位,雅道陵遲,昭、襄之風,漼焉已墜。洎乎後主,
 外內崩離,眾潰於平陽,身離於青土。天道深遠,或未易談,吉凶由人,抑可揚榷。觀夫有齊全盛,控帶遐阻,西苞汾、晉,南極江、淮,東盡海隅,北漸沙漠,六國之地,我獲其五,九州之境,彼分其四。料甲兵之眾寡,校帑藏之虛實,折衝千里之將,帷幄六奇之士,比二方之優劣,無等級以寄言。然其太行、長城之固自若也,江淮、汾晉之險不移也,帑藏輸稅之賦未虧也,士庶甲兵之眾不缺也;然而前王用之而有餘,後主守之而不足,其故何哉?前王
 之御時也,沐雨櫛風,拯其溺而救其焚,信賞必罰,安而利之,既與共其存亡,故得同其生死。



 後主則不然,以人從欲,損物益己。彫墻峻宇,甘酒嗜音,廛肆遍於宮園,禽色荒於外內,俾晝作夜,罔水行舟,所欲必成,所求必得。既不軌不物,又暗於聽受,忠信不聞,萋斐必入,視人如草芥,從惡如順流。佞閹處當軸之權,婢媼擅回天之力,賣官鬻獄,亂政淫刑,刳剒被於忠良,祿位加於犬馬,讒邪並進,法令多聞,持瓢者非止百人,搖樹者不唯一手,
 於是土崩瓦解,眾叛親離,顧瞻周道,咸有西歸之志,方更盛其宮觀,窮極荒淫,謂黔首之可誣,指白日以自保。馳倒戈之旅,抗前歌之師,五世崇基,一舉而滅,豈非鐫金石者難為功,摧枯朽者易為力歟?抑又聞之:皇天無親,唯德是輔;天時不如地利,地利不如人和。齊自河清之後,逮于武平之末,土木之功不息,嬪嬙之選無已,征稅盡,人力殫,物產無以給其求,江海不能贍其欲。所謂火既熾矣,更負薪以足之,數既窮矣,又為惡以促之,欲
 求大廈不燔,延期過歷,不亦難乎!由此言之,齊氏之敗亡,蓋亦由人,匪唯天道也。



\end{pinyinscope}