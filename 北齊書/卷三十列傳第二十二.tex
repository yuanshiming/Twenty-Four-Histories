\article{卷三十列傳第二十二}

\begin{pinyinscope}

 崔暹子
 達鷿高德政崔昂崔暹,字季倫,博陵安平人,漢尚書寔之後也,世為北州
 著姓。父穆,州主簿。



 暹少為書生,避地渤海,依高乾,以妹妻乾弟慎。慎後臨光州,啟暹為長史。趙郡公琛鎮定州,辟為開府諮議。隨琛往晉陽,高祖與語說之,以兼丞相長史。高祖舉兵將入洛,留暹佐琛知後事。謂之曰:「丈夫相知,豈在新舊。軍戎事重,留守任切,家弟年少,未閑事宜,凡百後事,一以相屬。」握手殷勤,至於三四。後遷左丞、吏部郎,主議《麟趾格》。



 暹親遇日隆,好薦人士。言邢邵宜任府僚,兼任機密,世宗因以征邵,甚見親重。言論之際,
 邵遂毀暹。世宗不悅,謂暹曰:「卿說子才之長,子才專言卿短,此癡人也。」暹曰:「子才言暹短,暹說子才長,皆是實事,不為嫌也。」高慎之叛,與暹有隙,高祖欲殺之,世宗救免。武定初,遷御史中尉,選畢義雲、盧潛、宋欽道、李愔、崔瞻、杜蕤、嵇曄、酈伯偉、崔子武、李廣皆為御史,世稱其知人。



 世宗欲遐暹威勢,諸公在坐,令暹高視徐步,兩人掣裾而入,世宗分庭對揖。



 暹不讓席而坐,觴再行,便辭退。世宗曰:「下官薄有蔬食,願
 公少留。」暹曰:「適受敕在臺檢校。」遂不待食而去,世宗降階送之。旬日後,世宗與諸公出之東山,遇暹於道,前驅為赤棒所擊,世宗回馬避之。



 暹前後表彈尚書令司馬子如及尚書元羨、雍州刺史慕容獻,又彈太師咸陽王坦、并州刺史可朱渾道元,罪狀極筆,並免官。其餘死黜者甚眾。高祖書與鄴下諸貴曰:「崔暹昔事家弟為定州長史,後吾兒開府諮議,及遷左丞吏部郎,吾未知其能也。



 始居憲臺,乃爾糾劾。咸陽王、司馬令並是吾對門
 布衣之舊,尊貴親暱,無過二人,同時獲罪,吾不能救,諸君其慎之。」高祖如京師,群官迎於紫陌。高祖握暹手而勞之曰:「往前朝廷豈無法官,而天下貪婪,莫肯糾劾。中尉盡心為國,不避豪強,遂使遠邇肅清,群公奉法。衝鋒陷陣,大有其人,當官正色,今始見之。今榮華富貴,直是中尉自取,高歡父子,無以相報。」賜暹良馬,使騎之以從,且行且語。



 暹下拜,馬驚走,高祖為擁之而授轡。魏帝宴於華林園,謂高祖曰:「自頃朝貴、牧守令長、所在百司
 多有貪暴,侵削下人。朝廷之中有用心公平,直言彈劾,不避親戚者,王可勸酒。」高祖降階,跪而言曰:「唯御史中尉崔暹一人。謹奉明旨,敢以酒勸,并臣所射賜物千匹,乞回賜之。」帝曰:「崔中尉為法,道俗齊整。」



 暹謝曰:「此自陛下風化所加,大將軍臣澄勸獎之力。」世宗退謂暹曰:「我尚畏羨,何況餘人。」由是威名日盛,內外莫不畏服。



 高祖崩,未發喪,世宗以暹為度支尚書,兼僕射,委以心腹之寄。暹憂國如家,以天下為己任。世宗車服過度,誅戮變
 常,言談進止,或有虧失,暹每厲色極言,世宗亦為之止。有囚數百,世宗盡欲誅之,每催文帳。暹故緩之,不以時進,世宗意釋,竟以獲免。



 自出身從官,常日晏乃歸。侵曉則與兄弟問母之起居,暮則嘗食視寢,然後至外齋對親賓。一生不問家事。魏、梁通和,要貴皆遣人隨聘使交易,暹惟寄求佛經。



 梁武帝聞之,為繕寫,以幡花寶蓋贊唄送至館焉。然而好大言,調戲無節。密令沙門明藏著《佛性論》而署己名,傳諸江表。子達拏年十三,暹命儒者
 權會教其說《周易》兩字,乃集朝貴名流,令達拏升高座開講。趙郡睦仲讓陽屈服之,暹喜,擢為司徒中郎。鄴下為之語曰:「講義兩行得中郎。」此皆暹之短也。



 顯祖初嗣霸業,司馬子如等挾舊怨,言暹罪重,謂宜罰之。高隆之亦言宜寬政網,去苛察法官,黜崔暹,則得遠近人意。顯祖從之。及踐祚,譖毀之者猶不息。



 帝乃令都督陳山提等搜暹家,甚貧匱,唯得高祖、世宗與暹書千餘紙,多論軍中大事。帝嗟賞之。仍不免眾口,乃流暹於馬城,晝
 則負土供役,夜則置地牢。歲餘,奴告暹謀反,鎖赴晉陽,無實,釋而勞之。尋遷太常卿。帝謂群臣曰:「崔太常清正,天下無雙,卿等不及。」



 初,世宗欲以妹嫁暹子,而會世宗崩,遂寢。至是,群臣宴於宣光殿,貴戚之子多在焉。顯祖歷與之語,於坐上親作書與暹曰:「賢子達拏,甚有才學。亡兄女樂安主,魏帝外甥,內外敬待,勝朕諸妹,思成大兄宿志。」乃以主降達拏。天保末,為右僕射。帝謂左右曰:「崔暹諫我飲酒過多,然我飲何所妨?」常山王私謂
 暹曰:「至尊或多醉,太后尚不能致言,吾兄弟杜口,僕射獨能犯顏,內外深相感愧。」十年,暹以疾卒,帝撫靈而哭。贈開府。



 達拏溫良清謹,有識學,少歷職為司農卿。入周,謀反伏誅。天保時,顯祖嘗問樂安公主:「達拏於汝何似?」答曰:「甚相敬重,唯阿家憎兒。」顯祖召達拏母入內,殺之,投屍漳水。齊滅,達拏殺主以復仇。



 高德政,字士貞,渤海蓚人。父顥,魏滄州刺史。德政幼而敏慧,有風神儀表。



 顯祖引為開府參軍,知管記事,甚相
 親狎。高祖又擢為相府掾,委以腹心。遷黃門侍郎。世宗嗣業,如晉陽,顯祖在京居守,令德政參掌機密,彌見親重。世宗暴崩,事出倉卒,群情草草。勳將等以纘戎事重,勸帝早赴晉陽。帝亦回遑不能自決,夜中召楊愔、杜弼、崔季舒及德政等,始定策焉。以楊愔居守。



 德政與帝舊相暱愛,言無不盡。散騎常侍徐之才、館客宋景業先為天文圖讖之學,又陳山提家客楊子術有所援引,並因德政勸顯祖行禪代之事。德政又披心固請。



 帝乃手書
 與楊愔,具論諸人勸進意。德政恐愔猶豫不決,自請馳驛赴京,託以餘事,唯與楊愔言,愔方相應和。德政還未至,帝便發晉陽,至平城都,召諸勳將入,告以禪讓之事。諸將等忽聞,皆愕然,莫敢答者。時杜弼為長史,密啟顯祖云:「關西是國家勁敵,若今受魏禪,恐其稱義兵挾天子而東向,王將何以待之?」顯祖入,召弼入與徐之才相告。之才云:「今與王爭天下者,彼意亦欲為帝,譬如逐兔滿市,一人得之,眾心皆定。今若先受魏禪,關西自應息
 心。縱欲屈強,止當逐我稱帝。



 必宜知機先覺,無容後以學人。」弼無以答。帝已遣馳驛向鄴,書與太尉高岳、尚書令高隆之、領軍婁睿、侍中張亮、黃門趙彥深、楊愔等。岳等馳傳至高陽驛,帝使約曰:「知諸貴等意,不須來。」唯楊愔見,高岳等並還。帝以眾人意未協,又先得太后旨云:「汝父如龍,汝兄如虎,尚以人臣終,汝何容欲行舜、禹事?此亦非汝意,正是高德政教汝。」又說者以為昔周武王再駕盟津,然始革命,於是乃旋晉陽。自是居常不悅。徐
 之才、宋景業等每言卜筮雜占陰陽緯侯,必宜五月應天順人,德政亦勸不已。仍白帝追魏收。收至,令撰禪讓詔冊、九錫建臺及勸進文表。



 至五月初,帝發晉陽。德政又錄在鄴諸事條進於帝,帝令陳山提馳驛齎事條并密書與楊愔。大略令撰儀注,防察魏室諸王。山提以五月至鄴,楊愔即召太常卿邢邵、七兵尚書崔鷿、度支尚書陸操、詹事王昕、黃門侍郎陽休之、中書侍郎裴讓之等議撰儀注。六日,要魏太傅咸陽王坦等總集,引人
 北宮,留于東齋,受禪後,乃放還宅。帝初發至亭前,所乘馬忽倒,意甚惡之,大以沉吟。至平城都,便不復肯進。德政、徐之才苦請帝曰:「山提先去,若為形容,恐其漏泄不果。」即命司馬子如、杜弼馳驛續入,觀察物情。七日,子如等至鄴,眾人以事勢已決,無敢異言。



 八日,楊愔書中旨,以魏襄城王旭并司空公潘相樂、侍中張亮、黃門趙彥深入通奏事。魏孝靜在昭陽殿引見。旭云:「五行遞運,有始有終,齊王聖德欽明,萬方歸仰,臣等昧死聞奏,願陛
 下則堯禪舜。」魏帝便斂容曰:「此事推挹已久,謹當遜避。」又道:「若爾,須作詔。」中書侍郎崔劼奏云:「詔已作訖。」即付楊愔進於魏靜帝。凡有十餘條,悉書。魏靜云:「安置朕何所,復若為去?」楊愔對:「在北城別有館宇,還備法駕,依常仗衛而去。」魏靜帝於是下御坐,就東廊,口詠范蔚宗《後漢書贊》云:「獻生不辰,身播國屯,終我四百,永作虞賓。」所司尋奏請發。魏靜帝曰:「人念遺簪弊屨,欲與六宮別,可乎?」乃入與夫人嬪御以下訣別,莫不歔欷掩涕。嬪趙國
 李氏口誦陳思王詩云:「王其愛玉體,俱享黃髮期。」



 魏靜帝登車出萬春門,直長趙道德在車中陪侍,百官在門外拜辭。遂入北城下司馬子如南宅。帝至城南頓所。受禪之日,除德政為侍中,尋封藍田公。七年,遷尚書右僕射,兼侍中,食渤海郡幹。德政與尚書令楊愔綱紀政事,多有弘益。



 顯祖末年,縱酒酣醉,所為不法,德政屢進忠言。後召德政飲,不從,又進言於前,諫曰:「陛下道我尋休,今乃甚於既往,其若社稷何,其若太后何!」帝不悅。又
 謂左右云:「高德政恒以精神凌逼人。」德政甚懼,乃稱疾屏居佛寺,兼學坐禪,為退身之計。帝謂楊愔曰:「我大憂德政,其病何似?」愔以禪代之際,因德政言情切至,方致誠款,常內忌之。由是答云:「陛下若用作冀州刺史,病即自差。」帝從之,德政見除書而起。帝大怒,召德政謂之曰:「聞爾病,我為爾針。」



 親以刀子刺之,血流霑地。又使曳下,斬去其趾。劉桃枝捉刀不敢下。帝起臨階砌,切責桃枝曰:「爾頭即墮地!」因索大刀自帶,欲下階。桃枝乃斬足之
 三指。帝怒不解,禁德政於門下,其夜開城門,以氈輿送還家。旦日,德政妻出寶物滿四床,欲以寄人。帝奄至其宅,見而怒曰:「我府藏猶無此物!「詰其所從得,皆諸元賂之也。遂曳出斬之。時妻出拜,又斬之,并其子祭酒伯堅。德政死後,顯祖謂群臣曰:「高德政常言宜用漢人,除鮮卑,此即合死。又教我誅諸元,我今殺之,為諸元報仇也。」帝後悔,贈太保,嫡孫王臣襲焉。



 崔昂,字懷遠,博陵安平人也。祖挺,魏州刺史。昂年七
 歲而孤,伯父吏部尚書孝芬嘗謂所親曰:「此兒終當遠至,是吾家千里駒也。」昂性端直少華,沉深有志略,堅實難傾動。少好章句,頗綜文詞。世宗廣開幕府,引為記室參軍,委以腹心之任。世宗入輔朝政,召為開府長史。時勳將親族兵客在都下放縱,多行不軌,孫騰、司馬子如之門尤劇。昂受世宗密旨,以法繩之,未幾之間,內外齊肅。遷尚書左丞,其年,又兼度支尚書。左丞之兼尚書,近代未有,唯昂獨為冠首,朝野榮之。



 武定六年,甘露降於
 宮闕,文武官僚同賀顯陽殿。魏帝問僕射崔暹、尚書楊愔等曰:「自古甘露之瑞,漢、魏多少,可各言往代所降之處,德化感致所由。」次問昂,昂曰:「案《符瑞圖》,王者德致於天,則甘露降。吉凶兩門,不由符瑞,故桑雉為戒,實啟中興,小鳥孕大,未聞福感。所願陛下雖休勿休。」帝為斂容曰:「朕既無德,何以當此。」



 齊受禪,遷散騎常侍,兼太府卿、大司農卿。二寺所掌,世號繁劇,昂校理有術,下無姦偽,經手歷目,知無不為,朝廷歎其至公。又奏上
 橫市妄費事三百一十四條,詔下,依啟狀速議以聞。其年,與太子少師邢邵議定國初禮,仍封華陽縣男。



 又詔刪定律令,損益禮樂,令尚書右僕射薛琡等四十三人在領軍府議定。又敕昂云:若諸人不相遵納,卿可依事啟聞。」昂奉敕笑曰:「正合生平之願。」昂素勤慎,奉敕之後,彌自警勖,部分科條,校正今古,所增損十有七八。轉廷尉卿。昂本性清嚴,凡見黷貨輩,疾之若仇,以是治獄文深,世論不以平恕相許。



 顯祖幸東山,百官預宴,升射堂。
 帝召昂於御坐前,謂曰:「舊人多出為州,我欲以臺閣中相付,當用卿為令僕,勿望刺史。卿六十外當與卿本州,中間州不可得也。」後九卿以上陪集東宮,帝指昂及尉瑾、司馬子瑞謂太子曰:「此是國家柱石,汝宜記之。」未幾,復侍宴金鳳臺,帝歷數諸人,咸有罪負,至昂曰:「崔昂直臣,魏收才士,婦兄妹夫,俱省罪過。」天保十年,策拜儀同燕子獻,百司陪列,昂在行中。帝特召昂至御所,曰:「歷思群臣可綱紀省闥者,唯冀卿一人。」即日除為兼右僕射。
 數日後,昂因入奏事,帝謂尚書令楊愔曰:「昨不與崔昂正者,言其太速,欲明年真之。終是除正,何事早晚,可除正僕射。」明日,即拜為真。楊愔曰:「昨不與崔昂正者,言其太速,欲明年真之。終是除正,何事早晚,可除正僕射。」明日,即拜為真。楊愔少時與昂不平,顯祖崩後,遂免昂僕射,除儀同三司。後坐事除名。卒祠部尚書。



 昂有風調才識,舊立堅正剛直之名。然好探揣上意,感激時主,或列陰私罪失,深為顯祖所知賞,發言獎護,人莫之能毀。議曹律令,京畿密獄,及朝廷之大事多委之。尚嚴猛,好行鞭撻,雖苦楚萬端,對之自若。前者崔暹、
 季舒為之親援,後乃高德政是其中表,常有挾恃,意色矜高,以此不為名流所服。子液嗣。



\end{pinyinscope}