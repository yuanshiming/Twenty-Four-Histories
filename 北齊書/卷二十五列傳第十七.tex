\article{卷二十五列傳第十七}

\begin{pinyinscope}

 張纂張亮張耀趙
 起徐遠王峻王紘張纂,字徽纂,代郡平城人也。父烈,桑乾太守。纂初事爾朱榮,又為爾朱兆都督長史。為兆使於高祖,遂被顧識。高祖舉義山東,劉誕據相州拒守,時纂亦在其中。高祖攻而拔之,以纂參丞相軍事。



 纂性便僻,左右出內,稍見親待,仍補行臺
 郎中。高祖啟減國封,分賞文武,纂隨例封壽張伯。魏武帝末,高祖赴洛,以趙郡公琛為行臺,守晉陽,以纂為右丞。



 轉相府功曹參軍
 事,
 除右光祿大夫。使於茹茹,以銜命稱旨。歷中外、丞相二府從事中郎。邙山之役,大獲俘虜,高祖令纂部送京師,魏帝賜絹五百匹,封武安縣伯。



 復為高祖行臺右丞,從征玉壁。大軍將還山東,行達晉州,忽值寒雨,士卒饑凍,至有死者。州以邊禁不聽入城。於時纂為別使,遇見,輒令開門內之,分寄民家,給其火食,多所全濟。高祖聞而善之。



 纂事高祖
 二十餘歲,傳通教令,甚見親賞。世宗嗣位,侯景作亂潁川,招引西魏。以纂為南道行臺,與諸將率討之。還,除瀛州刺史。會世宗入為太子少傅。後與平原王段孝先、行臺尚書辛術等攻圍東楚,仍拔廣陵、涇州數城,斬賊帥東方白額。授儀同三司,監築長城大使,領步騎數千鎮防北境。還,遷護軍將軍,尋卒。



 張亮,字伯德,西河隰城人也。少有乾用。初事爾朱兆,拜平遠將軍。以功封隰城縣伯,邑五百戶。高祖討兆於晉
 陽,兆奔秀容。兆左右皆密通誠款,唯亮獨無啟疏。及兆敗,竄於窮山,令亮及倉頭陳山提斬己首以降,皆不忍,兆乃自縊於樹。



 伯德伏屍而哭。高祖嘉歎之。授丞相府參軍事,漸見親待,委以書記之任。天平中,為世宗行臺郎中,典七兵事。雖為臺郎,而常在高祖左右。遷行臺右丞。



 高仲密之叛也,與大司馬斛律金守河陽。周文帝於上流放火船燒河橋。亮乃備小艇百餘艘,皆載長鎖,鎖頭施釘。火船將至,即馳小艇,以釘釘之,引鎖向岸,火船
 不得及橋。橋之獲全,亮之計也。



 武定初,拜太中大夫。薛琡嘗夢亮於山上持絲,以告亮,且占之曰:「山上絲,幽字也。君其為幽州乎?」數月,亮出為幽州刺史。屬侯景叛,除平南將軍、梁州刺史。尋加都督揚、潁等十一州諸軍事,兼行臺殿中尚書,轉都督二豫、揚、潁等八州軍事、征西大將軍、豫州刺史、尚書右僕射、西南道行臺。攻梁江夏、潁陽等七城,皆下之。



 亮性質直,勤力彊濟,深為高祖、世宗所信,委以腹心之任。然少風格,好財利,久在左右,不
 能廉潔,及歷諸州,咸有黷貨之聞。武定末,徵拜侍中、汾州大中正。天保初,授光祿勳,加驃騎大將軍、儀同三司,別封安定縣男,轉中領軍。



 尋卒於位,贈司空公。



 張耀,字靈光,上谷昌平人也。父鳳,晉州長史。耀少而貞謹,頗曉史職。解褐給事中,轉司徒水曹行參軍。義旗建,高祖擢為中軍大都督韓軌府長史。及軌除瀛、冀二州刺史,又以耀為軌諮議參軍。後為御史所劾,州府僚佐及軌左右以贓罪掛網者百有餘人,唯耀清白獨免。
 徵為丞相府倉曹。



 顯祖嗣事,遷相府掾。天保初,賜爵都亭鄉男,攝倉、庫二曹事。諸有賜給,常使耀典之。轉秘書丞,遷尚書右丞。顯祖曾因近出,令耀居守。帝夜還,耀不時開門,勒兵嚴備。帝駐蹕門外久之,催迫甚急。耀以夜深,真偽難辯,須火至面識,門乃可開,於是獨出見帝。帝笑曰:「卿欲學郅君章也?」乃使耀前開門,然後入,深嗟賞之,賜以錦采。出為南青州刺史,未之任。肅宗輔政,累遷秘書監。



 耀歷事累世,奉職恪勤,咸見親待,未嘗有過。每
 得祿賜,散之宗族。性節儉率素,車服飲食,取給而已。好讀《春秋》,月一遍,時人比之賈梁道。趙彥深嘗謂耀曰:「君研尋《左氏》,豈求服虔、杜預之紕繆邪?」耀曰:「何為其然乎?



 《左氏》之書,備敘言事,惡者可以自戒,善者可以庶幾。故厲己溫習,非欲詆訶古人之得失也。」天統元年,世祖臨朝,耀奏事,遇暴疾,仆於御前。帝下座臨視,呼數聲不應。帝泣曰:「豈失我良臣也!」旬日卒,時年六十三。詔稱耀忠貞平直,溫恭廉慎。贈開府儀同三司、尚書右僕射、燕州
 刺史,謚曰貞簡。



 趙起,字興洛,廣平人也。父達,幽州錄事參軍。起性沉謹有乾用。義旗建,高祖以段榮為定州刺史,以起為榮典簽,除奉車都尉。天平中,徵為相府騎曹,累加中散大夫。世宗嗣事,出為建州刺史,累遷侍中。起,高祖世頻為相府騎兵二局,典知兵馬十有餘年。至顯祖即阼之後,起罷州還闕,雖歷位九卿、侍中,常以本官監兵馬,出內驅使,居腹心之寄,與二張相亞。出為西兗州刺史,糾劾禁
 止,歲餘,以無驗獲免。河清二年,徵還晉陽。三年,又加祠部尚書、開府。天統初,轉太常卿,食琅邪郡幹。二年,除滄州刺史,加六州都督。武平中,卒於官。



 徐遠,字彥遐,廣寧石門人也。其先出自廣平。曾祖定,為雲中軍將、平朔戍主,因家於朔。遠少習吏事,郡辟功曹。未幾,與太守率戶赴義旗,署防城都督,除癭陶縣令。高祖以遠閒習書計,命為丞相騎兵參軍事,常征伐,克濟軍務,深為高祖所知。累歷巨鹿、陳留二郡太守。天保初,
 為御史所劾,遇赦免,沉廢二年。



 顯祖以遠勳舊,特用為領軍府長史,累遷東徐州刺史,入為太中大夫。河清初,加衛將軍。二年,除使持節、都督東楚州諸軍事、東楚州刺史。天統二年,授儀同三司、衛尉。四年,加開府、右光祿大夫。武平初卒。



 遠為治慕寬和,有恩惠。至東楚,其年冬,邑郭大火,城民亡產業,遠躬自赴救,對之流涕,仍為經營,皆得安立。長子世榮,中書舍人、黃門侍郎。



 王峻,字巒嵩,靈丘人也。明悟有幹略。高祖以為相府墨
 曹參軍,坐事去官。



 久之,顯祖為儀同開府,引為城局參軍。累遷恒州大中正、世宗相府外兵參軍。隨諸軍平淮陰,賜爵北平縣男。除營州刺史。營州地接邊城,賊數為民患。峻至州,遠設斥候,廣置疑兵,每有賊發,常出其不意要擊之,賊不敢發,合境獲安。先是刺史陸士茂詐殺室韋八百餘人,因此朝貢遂絕。至是,峻分命將士,要其行路,室韋果至,大破之,虜其首帥而還。因厚加恩禮,放遣之。室韋遂獻誠款,朝貢不絕,峻有力焉。初,茹茹主庵
 羅辰率其餘黨東徙,峻度其必來,預為之備。未幾,庵羅辰到,頓軍城西。峻乃設奇伏大破之,獲其名王郁久閭豆拔提等數十人,送於京師。



 庵羅辰於此遁走。帝甚嘉之。遷秘書監。



 廢帝即位,除洛州刺史、河陽道行臺左丞。皇建中,詔於洛州西界掘長塹三百里,置城戍以防間諜。河清元年,徵拜祠部尚書。詔詣晉陽檢校兵馬,俄而還鄴,轉太僕卿。及車駕巡幸,常與吏部尚書尉瑾輔皇太子、諸親王同知後事。仍賜食梁郡幹,遷侍中,除都官
 尚書。及周師寇逼,詔峻以本官與東安王婁睿、武興王普等自鄴率眾赴河陽禦之。車駕幸洛陽,以懸瓠為周人所據,復詔峻為南道行臺,與婁睿率軍南討。未至,周師棄城走,仍使慰輯永、郢二州。四年春,還京師。坐違格私度禁物并盜截軍糧,有司依格處斬,家口配沒。特詔決鞭一百,除名配甲坊,蠲其家口。會赦免,停廢私門。天統二年,授驃騎大將軍、儀同三司,尋加開府。武平初,除侍中。四年卒。贈司空公。



 王紘,字師羅,太安狄那人也,為小部酋帥。父基,頗讀書,有智略。初從葛榮反,榮授基濟北王、寧州刺史。後葛榮破,而基據城不下,爾朱榮遣使喻之,然後始降。榮後以為府從事中郎,令率眾鎮磨川。榮死,紇豆陵步藩虜基歸河西,後逃歸爾朱兆。高祖平兆,以基為都督,除義寧太守。基先於葛榮軍與周文帝相知,及文帝據有關中,高祖遣基與長史侯景同使於周文帝,文帝留基不遣。基後逃歸,除冀州長史,後行肆州事。元象初,累遷南益
 州、北豫州刺史。所歷皆好聚斂,然性和直,吏民不甚患之。興和四年冬為奴所害,時年六十五。贈征東將軍、吏部尚書、定州刺史。



 紘少好弓馬,善騎射,頗愛文學。性機敏,應對便捷。年十三,見揚州刺史太原郭元貞,元貞撫其背曰:「汝讀何書?」對曰:「誦《孝經》。」曰:「《孝經》云何?」曰:「在上不驕,為下不亂。」元貞曰:「吾作刺史,豈其驕乎?」紘曰:「公雖不驕,君子防未萌,亦願留意。」元貞稱善。年十五,隨父在北豫州,行臺侯景與人論掩衣法為當左為當右。尚書敬
 顯俊曰:「孔子云:『微管仲,吾其被髮左衽矣』以此言之,右衽為是。」紘進曰:「國家龍飛朔野,雄步中原,五帝異儀,三王殊制,掩衣左右,何足是非。」景奇其早慧,賜以名馬。



 興和中,世宗召為庫直,除奉朝請。世宗暴崩,紘冒刃捍禦,以忠節賜爵平春縣男,賚帛七百段、綾錦五十匹、錢三萬並金帶駿馬,仍除晉陽令。天保初,加寧遠將軍,頗為顯祖所知待。帝嘗與左右飲酒,曰:「快哉大樂。」紘對曰:「亦有大樂,亦有大苦。」帝曰:「何為大苦?」紘曰:「長夜荒飲不寤,
 亡國破家,身死名滅,所謂大苦。」帝默然。後責紘曰:「爾與紇奚舍樂同事我兄,舍樂死,爾何為不死?」紘曰:「君亡臣死,自是常節,但賊豎力薄斫輕,故臣不死。」帝使燕子獻反縛紘,長廣王捉頭,帝手刃將下,紘曰:「楊遵彥、崔季舒逃走避難,位至僕射、尚書,冒死效命之士,反見屠戮,曠古未有此事。」帝投刃於地曰:「王師羅不得殺。」遂舍之。



 乾明元年,昭帝作相,補中外府功曹參軍事。皇建元年,進爵義陽縣子。河清三年,與諸將征突厥,加驃騎大將軍。
 天統元年,除給事黃門侍郎,加射聲校尉,四遷散騎常侍。武平初,開府儀同三司。紘上言:「突厥與宇文男來女往,必當相與影響,南北寇邊。宜選九州勁勇彊弩,多據要險之地。伏願陛下哀忠念舊,愛孤恤寡,矜愚嘉善,舍過記功,敦骨肉之情,廣寬仁之路,思堯、舜之風,慕禹、湯之德,克己復禮,以成美化,天下幸甚。」



 五年,陳人寇淮南,詔令群官共議禦捍。封輔相請出討擊。紘曰:「官軍頻經失利,人情騷動,若復興兵極武,出頓江淮,恐北狄西寇,
 乘我之弊,傾國而來,則世事去矣。莫若薄賦省徭,息民養士,使朝廷協睦,遐邇歸心,征之以仁義,鼓之以道德,天下皆當肅清,豈直偽陳而已!」高阿那肱謂眾人曰:「從王武衛者南席。」眾皆同焉。尋兼侍中,聘於周。使還即正,未幾而卒。紘好著述,作《鑒誡》二十四篇,頗有文義。



 史臣曰:張纂等並趨事霸朝,申其功用,皆有齊之良臣也。伯德之慟哭伏屍,靈光之拒關駐蹕,有古人風焉。



 贊曰:纂、亮、耀、起,徐遠、紘、峻,奉日高昇,凌風遠振。樹死拒
 關,終明信順。



\end{pinyinscope}