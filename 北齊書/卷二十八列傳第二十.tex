\article{卷二十八列傳第二十}

\begin{pinyinscope}

 元坦
 元斌元孝友元暉業
 元弼元韶元坦,祖魏獻文皇帝,咸陽王禧第七子。禧誅後,兄翼、樹等五人相繼南奔,故坦得承襲,改封敷城王。永安初,復本封咸陽郡王,累遷侍中。莊帝從容謂曰:「王才非荀、蔡,中歲屢遷,當由少長朕家,故有超授。」初,禧死後,諸子貧乏,坦兄弟為彭城王勰所收養,故有此言。



 孝武初,其兄樹見禽。坦見樹既長且賢,慮其代己,密勸朝廷以法除
 之。樹知之,泣渭坦曰:「我往因家難,不能死亡,寄食江湖,受其爵命。今者之來,非由義至,求活而已,豈望榮華。汝何肆其猜忌,忘在原之義,腰背雖偉,善無可稱。」



 坦作色而去。樹死,竟不臨哭。



 坦歷司徒、太尉、太傅,加侍中、太師、錄尚書事、宗正、司州牧。雖祿厚位尊,貪求滋甚,賣獄鬻官,不知紀極。為御史劾奏免官,以王歸第。尋起為特進,出為冀州刺史,專復聚斂。每百姓納賦,除正稅外,別先責絹五匹,然後為受。性好畋漁,無日不出,秋冬獵雉兔,
 春夏捕魚蟹,鷹犬常數百頭。自言寧三日不食,不能一日不獵。入為太傅。齊天保初準例降爵,封新豐縣公,除特進、開府儀同三司。坐子世寶與通直散騎侍郎彭貴平因酒醉誹謗,妄說圖讖,有司奏當死,詔並宥之。坦配北營州,死配所。



 元斌,字善集,祖魏獻文皇帝。父高陽王雍,從孝莊於河陰遇害。斌少襲祖爵,歷位侍中、尚書左僕射。斌美儀貌,性寬和,居官重慎,頗為齊文襄愛賞。齊天保初,準例降
 爵,為高陽縣公,拜右光祿大夫。二年,從文宣討契丹還,至白狼河,以罪賜死。



 元孝友,祖魏太武皇帝。兄臨淮王彧無子,令孝友襲爵。累遷滄州刺史,為政溫和,好行小惠,不能清白,而無所侵犯,百姓亦以此便之。魏靜帝宴文襄於華林,孝友因醉自譽,又云:「陛下許賜臣能。」帝笑曰:「朕恒聞王自道清。」文襄曰:「臨淮王奉旨舍罪。」於是君臣俱笑而不罪。



 孝友明於政理,嘗奏表曰:令制:百家為黨族,二十家為閭,五
 家為比鄰。百家之內,有帥二十五人,徵發皆免,苦樂不均。羊少狼多,復有蠶食。此之為弊久矣。京邑諸坊,或七八百家唯一里正、二史,庶事無闕,而況外州乎?請依舊置三正之名不改,而百家為四閭,閭二比。計族少十二丁,得十二匹貲絹。略計見管之戶應二萬餘族,一歲出貲絹二十四萬匹。十五丁為一番兵,計得一萬六千兵。此富國安人之道也。



 古諸侯娶九女,士一妻一妾。《晉令》:諸王置妾八人;郡君、侯,妾六人。



 《官品令》:第一第二品有
 四妾,第三第四有三妾,第五第六有二妾,第七第八有一妾。所以陰教聿修,繼嗣有廣。廣繼嗣孝也,修陰教禮也。而聖朝忽棄此數,由來漸久,將相多尚公主,王侯娶后族,故無妾媵,習以為常。婦人不幸,生逢今世,舉朝既是無妾,天下殆皆一妻。設令人彊志廣娶,則家道離索,身事迍邅,內外親知,共相嗤怪。凡今之人,通無準節。父母嫁女,則教以妒,姑姊逢迎,必相勸以忌。以制夫為婦德,以能妒為女工。自云不受人欺,畏他笑我。王公猶自
 一心,已下何敢二意。夫妒忌之心生,則妻妾之禮廢,妻妾之禮廢,則女淫之兆興,斯臣之所以毒恨者也。請以王公第一品娶八,通妻以備九女,稱事。二品備七,三品四品備五,五品六品則一妻二妾。限以一周,悉令充數。若不充數,及待妾非禮,使妻妒加捶撻,免所居官。其妻無子而不娶妾,斯則自絕,無以血食祖父,請科不孝之罪,離遣其妻。



 臣之赤心,義唯家國,欲使吉凶無不合禮,貴賤各有其宜,省人帥以出兵丁,立倉儲以豐穀食,設
 賞格以擒姦盜,行典令以示朝章,庶使足食足兵,人信之矣。



 又冒申妻妾之數,正欲使王侯將相功臣子弟,苗胤滿朝,傳祚無窮。此臣之志也。



 詔付有司,議奏不同。



 孝友又言:「今人生為皂隸,葬擬王侯,存沒異途,無復節制。崇壯丘隴,盛飾祭儀,鄰里相榮,稱為至孝。又夫婦之始,王化所先,共食合瓢,足以成禮。而今之富者彌奢,同牢之設,甚於祭盤,累魚成山,山有林木,林木之上,鸞鳳斯存。



 徒有煩勞,終成委棄。仰惟天意,其或不然。請自茲以
 後,若婚葬過禮者,以違旨論。官司不加糾劾,即與同罪。」



 孝友在尹積年,以法自守,甚著聲稱,然性無骨鯁,善事權勢,為正直者所譏。



 齊天保初,準例降爵,封臨淮縣公,拜光祿大夫。二年冬,被詔入晉陽宮,出與元暉業同被害。



 元
 暉業,字紹遠,魏景穆皇帝之玄孫。少險薄,多與寇盜交通。長乃變節,涉子史,亦頗屬文,而慷慨有志節。歷位司空、太尉,加特進,領中書監,錄尚書事。



 文襄嘗問之曰:「
 此何所披覽?」對曰:「數尋伊、霍之傳,不讀曹、馬之書。」



 暉業以時運漸謝,不復圖全,唯事飲啖,一日一羊,三日一犢。又嘗賦詩云:「昔居王道泰,濟濟富群英;今逢世路阻,狐兔鬱縱橫。」齊初,降封美陽縣公,開府儀同三司、特進。暉業之在晉陽也,無所交通,居常閑暇,乃撰魏藩王家世,號為《辯宗錄》,四十卷,行於世。位望隆重,又以性氣不倫,每被忌。天保二年,從駕至晉陽,於宮門外罵元韶曰:「爾不及一老嫗,背負璽與人,何不打碎之。我出此言,即
 知死也,然爾亦詎得幾時!」文宣聞而殺之,亦斬臨淮公孝友。孝友臨刑,驚惶失措,暉業神色自若。仍鑿冰沉其屍。暉業弟昭業,頗有學問,位諫議大夫。莊帝幸洛南,昭業立於閶闔門外叩馬諫,帝避之而過,後勞勉之。位給事黃門侍郎、衛將軍,右光祿大夫,卒。謚曰文侯。



 元弼,字輔宗,魏司空暉之子。性剛正,有文學。位中散大夫。以世嫡應襲先爵。為季父尚書僕射麗因於氏親寵,遂
 奪弼王爵,橫授同母兄子誕,於是弼絕棄人事,托疾還私第。宣武中為侍中,弼上表固讓。入嵩山,以穴為室,布衣蔬食,卒。



 建元元年,子暉業訴,復王爵。永安三年,追贈尚書令、司徒公,謚曰
 文獻。初,弼嘗夢人謂之曰:「君身不得傳世封,其紹先爵者,君長子紹遠也。」弼覺,即告暉業,終如其言。



 元韶,字世胄,魏孝莊之侄。避爾朱之難,匿於嵩山。性好學,美容儀。初,爾朱榮將入洛,父劭恐,以韶寄所親滎陽太守鄭仲明。仲明尋為城人所殺,韶因亂與乳母相失,遂與仲明兄子僧副避難。路中為賊逼,僧副恐不免,因令韶下馬。僧副謂客曰:「窮鳥投人,尚或矜愍,況諸王如何棄乎?」僧副舉刃逼之,客乃退。



 韶逢一老母姓程,哀之,隱於私家十餘日,莊帝訪而獲焉,襲封彭城王。齊神武帝
 以孝武帝后配之。魏室奇寶,多隨后入韶家。有二玉缽相盛,可轉而不可出;馬瑙榼容三升,玉縫之。皆稱西域鬼作也。歷位太尉、侍中、錄尚書、司州牧,進
 太傅。



 齊天保元年,降爵為縣公。



 韶性行溫裕,以高氏婿,頗膺時寵。能自謙退,臨人有惠政。好儒學,禮致才彥。愛林泉,修第宅,華而不侈。文宣帝剃韶須髯,加以粉黛,衣婦人服以自隨曰:「我以彭城為嬪御。」譏元氏微弱,比之婦女。



 十年,太史奏云:「今年當除舊布新。」文宣謂韶曰:「漢光武何故中興?」



 韶
 曰:「為誅諸劉不盡。」於是乃誅諸元以厭之。遂以五月誅元世哲、景武等二十
 五家,餘十九家並禁止之。韶幽於京畿地牢,絕食,啖衣袖而死。及七月,大誅元氏,自昭成已下並無遺焉。或父祖為王,或身常貴顯,或兄弟強壯,皆斬東市。其嬰兒投於空中,承之以槊。前後死者凡七百二十一人,悉投尸漳水,剖魚多得爪甲,都下為之久不食魚。



 贊曰:元氏蕃熾,馮茲慶靈。道隨終運,命偶淫刑。



\end{pinyinscope}