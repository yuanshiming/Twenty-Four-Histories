\article{卷十六列傳第八段榮(子韶)}

\begin{pinyinscope}

 段榮,字子茂,姑臧武威人也。祖信,仕沮渠氏,後入魏,以豪族徙北邊,仍家於五原郡。父連,安北府司馬。榮少好歷術,專意星象。正光初,語人曰:「《易》云『觀於天文以察時
 變』,又曰『天垂象,見吉凶』,今觀玄象,察人事,不及十年,當有亂矣。」或問曰:「起於何處,當可避乎?」榮曰:「構亂之源,此地為始,恐天下因此橫流,無所避也。」未幾,果如言。榮遇亂,與鄉舊攜妻子,南趣平城。屬杜洛周為亂,榮與高祖謀誅之,事不捷,共奔爾朱榮。後高祖建義山東,榮贊成大策。為行臺右丞,西北道慰喻大使,巡方曉喻,所在下之。高祖南討鄴,留榮鎮信都,仍授鎮北將軍,定州刺史。時攻鄴未克,所須軍資,榮轉輸無闕。



 高祖入洛,論功封
 姑臧縣侯,邑八百戶。轉授瀛州刺史。榮妻,皇后姊也,榮恐高祖招私親之議,固推諸將,竟不之州。尋行相州事,後為濟州刺史。天平三年,轉行泰州事。榮性溫和,所歷皆推仁恕,民吏愛之。初,高祖將圖關右,與榮密謀,榮盛稱未可。及渭曲失利,高祖悔之,曰:「吾不用段榮之言,以至於此。」四年,除山東大行臺、大都督,甚得物情。元象元年,授儀同三司。二年五月卒,年六十二。贈使持節、定冀滄瀛四州諸軍事、定州刺史、太尉、尚書左僕射,謚曰昭
 景。



 皇建初,配饗高祖廟庭。二年,重贈大司馬、尚書令、武威王。長子韶嗣。



 韶,字孝先,小名鐵伐。少工騎射,有將領才略。高祖以武明皇后姊子,盡器愛之,常置左右,以為心腹。建義初,領親信都督。中興元年,從高祖拒爾朱兆,戰於廣阿。高祖謂韶曰:「彼眾我寡,其若之何?」韶曰:「所謂眾者,得眾人之死;強者,得天下之心。爾朱狂狡,行路所見,裂冠毀冕,拔本塞源,邙山之會,搢紳何罪,兼殺主立君,不脫旬朔,天下思亂,十室而九。王躬昭德義,除君側之
 惡,何往而不克哉!」高祖曰:「吾雖以順討逆,奉辭伐罪,但弱小在強大之間,恐無天命,卿不聞之也?」答曰:「韶聞小能敵大,小道大淫,皇天無親,唯德是輔,爾朱外賊天下,內失善人,知者不為謀,勇者不為斗,不肖失職,賢者取之,復何疑也。」遂與兆戰,兆軍潰。攻劉誕於鄴。及韓陵之戰,韶督率所部,先鋒陷陣。尋從高祖出晉陽,追爾朱兆於赤谼嶺,平之。以軍功封下洛縣男。又從襲取夏州,擒斛律彌娥突,加龍驤將軍、諫議大夫,累遷武衛將軍。後
 恩賜父榮姑臧縣侯,其下洛縣男啟讓繼母弟寧安。



 興和四年,從高祖御周文帝於邙山。高祖身在行間,為西魏將賀拔勝所識,率銳來逼。韶從傍馳馬引弓反射,一箭斃其前驅,追騎懾憚,莫敢前者。西軍退,賜馬并金,進爵為公。



 武定四年,從征玉壁。時高祖不豫,攻城未下,召集諸將,共論進止之宜。謂大司馬斛律金、司徒韓軌、左衛將軍劉豐等曰:「吾每與段孝先論兵,殊有英略,若使比來用其謀,亦可無今日之勞矣。吾患勢危篤,恐或不
 虞,欲委孝先以鄴下之事,何如?」金等曰:「知臣莫若君,實無出孝先。」仍謂韶曰:「吾昔與卿父冒涉險艱,同獎王室,建此大功。今病疾如此,殆將不濟,宜善相翼佐,克茲負荷。」



 即令韶從顯祖鎮鄴,召世宗赴軍。高祖疾甚,顧命世宗曰:「段孝先忠亮仁厚,智勇兼備,親戚之中,唯有此子,軍旅大事,宜共籌之。」五年春,高祖崩於晉陽,秘不發喪。俄而侯景構亂,世宗還鄴,韶留守晉陽。世宗還,賜女樂十數人,金十斤,繒帛稱是,封長樂郡公。世宗征潁川,韶
 留鎮晉陽。別封真定縣男,行并州刺史。顯祖受禪,別封朝陵縣,又封霸城縣,加位特進。啟求歸朝陵公,乞封繼母梁氏為郡君。顯祖嘉之,別以梁氏為安定郡君。又以霸城縣侯讓其繼母弟孝言。論者美之。



 天保三年,為冀州刺史、六州大都督,有惠政,得吏民之心。四年十二月,梁將東方白額潛至宿預,招誘邊民,殺害長吏,淮、泗擾動。五年二月,詔徵韶討之。



 既至,會梁將嚴超達等軍逼涇州;又陳武帝率眾將攻廣陵,刺史王敬寶遣使告急;
 復有尹思令率眾萬餘人謀襲盱貽。三軍咸懼。韶謂諸將曰:「自梁氏喪亂,國無定主,人懷去就,強者從之。霸先等智小謀大,政令未一,外託同德,內有離心,諸君不足憂,吾揣之熟悉矣。」乃留儀同敬顯俊、堯難宗等圍守宿預,自將步騎數千人倍道赴涇州。途出盱眙,思令不虞大軍卒至,望旗奔北。進與超達合戰,大破之,盡獲其舟艦器械。謂諸將士曰:「吳人輕躁,本無大謀,今破超達,霸先必走。」



 即回赴廣陵。陳武帝果遁去。追至楊子柵,望揚州
 城乃還,大獲其軍資器物,旋師宿預。六月,韶遣辯士喻白額禍福,白額於是開門請盟。韶與行臺辛術等議,且為受盟。盟訖,度白額終不為用,因執而斬之,并其諸弟等並傳首京師。江淮帖然,民皆安輯。顯祖嘉其功,詔賞吳口七十人,封平原郡王。清河王岳之克郢州,執司徒陸法和,韶亦豫行,築層城,於新蔡立郭默戍而還。皇建元年,領太子太師。大寧二年,除并州刺史。高歸彥作亂冀州,詔與東安王婁睿率眾討平之,遷太傅,賜女樂十
 人,并歸彥果園一千畝。仍蒞并州,為政舉大綱,不存小察,甚得民和。



 十二月,周武帝遣將率羌夷與突厥合眾逼晉陽,世祖自鄴倍道兼行赴救。突厥從北結陣而前,東距汾河,西被風谷。時事既倉卒,兵馬未整,世祖見如此,亦欲避之而東。尋納河間王孝琬之請,令趙郡王盡護諸將。時大雪之後,周人以步卒為前鋒,從西山而下,去城二里。諸將咸欲逆擊之。詔曰:「步人氣勢自有限,今積雪既厚,逆戰非便,不如陣以待之。彼勞我逸,破之必
 矣。」既而交戰,大破之,敵前鋒盡殪,無復孑遺,自餘通宵奔遁。仍令韶率騎追之,出塞不及而還。世祖嘉其功,別封懷州武德郡公,進位太師。



 周冢宰宇文護母閻氏先配中山宮,護聞閻尚存,乃因邊境移書,請還其母,并通鄰好。時突厥屢犯邊,韶軍於塞下。世祖遣黃門徐世榮乘傳齎周書問韶。韶以周人反覆,本無信義,比晉陽之役,其事可知。護外託為相,其實王也,既為母請和,不遣一介之使申其情理,乃據移書即送其母,恐示之弱。如
 臣管見,且外許之,待後放之未晚。不聽。遂遣使以禮將送。護既得母,仍遣將尉遲迥等襲洛陽。詔遣蘭陵王長恭、大將軍斛律光率眾擊之,軍於邙山之下,逗留未進。世祖召謂曰:「今欲遣王赴洛陽之圍,但突厥在此,復須鎮禦,王謂如何?」韶曰:「北虜侵邊,事等疥癬,今西羌窺逼,便是膏肓之病,請奉詔南行。」世祖曰:「朕意亦爾。」乃令韶督精騎一千,發自晉陽。五日便濟河,與大將共量進止。韶旦將帳下二百騎與諸軍共登邙阪,聊觀周軍
 形勢。至大和谷,便值周軍,即遣馳告諸營,追集兵馬。



 仍與諸將結陣以待之。韶為左軍,蘭陵王為中軍,斛律光為右軍,與周人相對。韶遙謂周人曰:「汝宇文護幸得其母,不能懷恩報德,今日之來,竟何意也?」周人曰:「天遣我來,有何可問。」韶曰:「天道賞善罰惡,當遣汝送死來耳。」周軍仍以步人在前,上山逆戰。韶以彼徒我騎,且卻且引,待其力弊,乃遣下馬擊之。



 短兵始交,周人大潰。其中軍所當者,亦一時瓦解,投墜溪谷而死者甚眾。洛城之圍,亦
 即奔遁,盡棄營幕,從邙山至穀水三十里中,軍資器物彌滿川澤。車駕幸洛陽,親勞將士,於河陰置酒高會,策勳命賞,除太宰,封靈武縣公。天統三年,除左丞相,永昌郡公,食滄州幹。



 武平二年正月,出晉州道,到定隴,築威敵、平寇二城而還。二月,周師來寇,遣韶與右丞相斛律光、太尉蘭陵王長恭同往捍禦。以三月暮行達西境。有柏谷城者,乃敵之絕險,石城千仞,諸將莫肯攻圍。韶曰:「汾北、河東,勢為國家之有,若不去柏谷,事同痼疾。計彼
 援兵,會在南道,今斷其要路,救不能來。且城勢雖高,其中甚狹,火弩射之,一旦可盡。」諸將稱善,遂鳴鼓而攻之。城潰,獲儀同薛敬禮,大斬獲首虜,仍城華谷,置戍而還。封廣平郡公。



 是月,周又遣將寇邊。右丞相斛律光先率師出討,韶亦請行。五月,攻服秦城。



 周人於姚襄城南更起城鎮,東接定陽,又作深塹,斷絕行道。韶乃密抽壯士,從北襲之。又遣人潛渡河,告姚襄城中,令內外相應。渡者千有餘人,周人始覺。於是合戰,大破之,獲其儀同若
 干顯寶等。諸將咸欲攻其新城,韶曰:「此城一面阻河,三面地險,不可攻,就令得之,一城地耳。不如更作一城壅其路,破服秦,併力以圖定陽,計之長者。」將士咸以為然。六月,徙圍定陽,其城主開府儀同楊範固守不下。韶登山望城勢,乃縱兵急攻之。七月,屠其外城,大斬獲首級。時韶病在軍中,以子城未克,謂蘭陵王長恭曰:「此城三面重澗險阻,並無走路,唯恐東南一處耳。賊若突圍,必從此出,但簡精兵專守,自是成擒。」長恭乃令壯士千餘人
 設伏於東南澗口。其夜果如所策,賊遂出城,伏兵擊之,大潰,範等面縛,盡獲其眾。



 韶疾甚,先軍還。以功別封樂陵郡公。竟以疾薨。上舉哀東堂,贈物千段、溫明秘器、轀輬車,軍校之士陳衛送至平恩墓所,發卒起塚。贈假黃鉞、使持節、都督朔并定趙冀滄齊兗梁洛晉建十二州諸軍事,相國、太尉、錄尚書事、朔州刺史,謚曰忠武。



 韶出總軍旅,入參帷幄,功既居高,重以婚媾,望傾朝野。長於計略,善於御眾,得將士之心,臨敵之日,人人爭奮。又雅
 性溫慎,有宰相之風。教訓子弟,閨門雍肅,事後母以孝聞,齊世勳貴之家罕有及者。然僻於好色,雖居要重,微服間行。有皇甫氏,魏黃門郎元瑀之妻,弟謹謀逆,皇甫氏因沒官。韶美其容質,上啟固請,世宗重違其意,因以賜之。尤嗇於財,雖親戚故舊略無施與。其子深尚公主,并省丞郎在家佐事十餘日,事畢辭還,人唯賜一盃酒。長子懿嗣。



 懿,字德猷,有資儀,頗解音樂,又善騎射。天保初,尚潁川長公主。累遷行臺右僕射,兼殿中尚書,出除
 兗州刺史。卒。子寶鼎嗣。尚中山長公主,武平末,儀同三司。隋開皇中,開府儀同三司、驃騎大將軍,大業初,卒於饒州刺史。



 韶第二子深,字德深。美容貌,寬謹有父風。天保中,受父封姑臧縣公。大寧初,拜通直散騎侍郎。二年,詔尚永昌公主,未婚,主卒。河清三年,又詔尚東安公主。以父頻著大勳,累遷侍中、將軍、源州大中正,食趙郡幹。韶病篤,詔封深濟北王,以慰其意。武平末,徐州行臺左僕射、徐州刺史。入周,拜大將軍,郡公,坐事死。



 韶第三子
 德舉,武平末,儀同三司。周建德七年,在鄴城與高元海謀逆,誅。



 韶第四子德衡,武平末,開府儀同三司,隆化時,濟州刺史。入周,授儀同大將軍。



 韶第七子德堪,武平中,儀同三司。隋大業初,汴州刺史,卒於汝南郡守。



 榮第二子孝言,少警發有風儀。魏武定末,起家司徒參軍事。齊受禪,其兄韶以別封霸城縣侯授之。累遷儀同三司、度支尚書、清都尹。孝言本以勳戚緒餘,致位通顯,至此便驕奢放逸,無所畏憚。曾夜行,過其賓客宋孝王家宿,
 喚坊民防援,不時應赴,遂拷殺之。又與諸淫婦密遊,為其夫覺,復恃官勢,拷掠而殞。時苑內須果木,科民間及僧寺備輸,悉分向其私宅種植。又殿內及園中須石,差車牛從漳河運載,復分車迴取。事悉聞徹,出為海州刺史。尋以其兄故,徵拜都官尚書,食陽城郡幹,仍加開府。遷太常卿,除齊州刺史,以贓賄為御史所劾。屬世祖崩,遇赦免。拜太常卿,轉食河南郡幹,遷吏部尚書。



 祖珽執政,將廢趙彥深,引孝言為助。除兼侍中,入內省,典機密,
 尋即正,仍吏部尚書。孝言既無深鑒,又待物不平,抽擢之徒,非賄則舊。有將作丞崔成,忽於眾中抗言曰:「尚書天下尚書,豈獨段家尚書也!」孝言無辭以答,惟厲色遣下而已。尋除中書監,加特進。又託韓長鸞,共構祖珽之短。及祖出後,孝言除尚書右僕射,仍掌選舉,恣情用捨,請謁大行。敕濬京城北隍,孝言監作,儀同三司崔士順、將作大匠元士將、太府少卿酈孝裕、尚書左民郎中薛叔昭、司州治中崔龍子、清都尹丞李道隆、鄴縣令尉長
 卿、臨漳令崔象、成安令高子徹等並在孝言部下。



 典作日,別置酒高會,諸人膝行跪伏,稱觴上壽,或自陳屈滯,更請轉官,孝言意色揚揚,以為己任,皆隨事報答,許有另授。富商大賈多被銓擢,所進用人士,咸是麤險放縱之流。尋遷尚書左僕射,特進、侍中如故。



 孝言富貴豪侈,尤好女色。後娶婁定遠妾董氏,大耽愛之,為此內外不和,更相糾列,坐爭免官,徙光州。隆化敗後,有敕追還。孝言雖黷貨無厭,恣情酒色,然舉止風流,招致名士,美景
 良辰,未嘗虛棄,賦詩奏伎,畢盡歡洽。雖草萊之士,粗閑文藝,多引入賓館,與同興賞,其貧躓者亦時有乞遺。世論復以此多之。齊亡入周,授開府儀同大將軍,後加上開府。



 史臣曰:段榮以姻戚之重,遇時來之會,功伐之地,亦足稱焉。韶光輔七君,克隆門業,每出當閫外,或任以留臺,以猜忌之朝,終其眉壽。屬亭候多警,為有齊上將,豈其然乎?當以志謝矜功,名不逾實,不以威權御物,不以智
 數要時,欲求覆餗,其可得也?語曰「率性之謂道」,此其效歟?



 贊曰:榮發其原,韶大其門。位因功顯,望以德尊。



\end{pinyinscope}