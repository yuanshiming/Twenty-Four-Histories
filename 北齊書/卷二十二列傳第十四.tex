\article{卷二十二列傳第十四}

\begin{pinyinscope}

 李元忠盧文偉李義深李元忠,趙郡柏人人也。曾祖靈,魏定州刺史、巨鹿公。祖恢,鎮西將軍。父顯甫,安州刺史。元忠少厲志操,居喪以孝聞。襲爵平棘子。魏清河王懌為司空,闢為士曹參軍。遷太尉,復啟為長流參軍。懌後為太傅,
 尋被詔為營構明堂大都督,又引為主簿。元忠粗覽史書及陰陽數術,解
 鼓箏,兼好射彈,有巧思。遭母憂,去任。未幾,相州刺史、安樂王鑒請為府司馬,元忠以艱憂,固辭不就。



 初,元忠以母老多患,乃專心醫藥,研習積年,遂善於方技。性仁恕,見有疾者,不問貴賤,皆為救療。家素富實,其家人在鄉,多有舉貸求利,元忠每焚契免責。鄉人甚敬重之。魏孝明時,盜賊蜂起,清河有五百人西戍,還經南趙郡,以路梗共投元忠。奉絹千匹,元忠唯受一匹,殺五羊以食之,遣奴為導,曰:「若逢賊,但道李元忠遣送。」奴如其言,賊皆舍避。



 永安初,就拜南趙郡太守,以好酒,無政績。值洛陽傾覆,莊帝幽崩,元忠棄官還家,潛圖義舉。會高祖率眾東出,便自往奉迎。乘露車,載素箏濁酒以見高祖,因
 進從橫之策,備陳誠款,深見嘉納。時刺
 史爾朱羽生阻兵據州,元忠先聚眾於西山,仍與大軍相合,擒斬羽生。即令行殷州事。中興初,除中
 軍將軍、衛尉卿。二年,轉太常卿、殷州大中正。後以從兄瑾年長,以中正讓之。尋加征南將軍。武帝將納后,即高祖之長女也,詔元忠與尚書令元羅致娉於晉陽。高祖每於宴席論敘舊事,因撫掌欣笑云:「此人逼我起兵。」賜白馬一匹。元忠戲謂高祖曰:「若不與侍中,當更覓建義處。」高祖答曰:「建義處不慮無,止畏如此老翁不可遇耳。」



 元忠曰:「止為此翁難遇,所以不去。」因捋高祖須而大笑。高祖亦悉其雅意,深
 相嘉重。後高祖奉送皇后,仍田於晉澤,元忠馬倒被傷,當時殞絕,久而方蘇。高祖親自撫視。其年,封晉陽縣伯,邑五百戶。後以微譴失官。時朝廷離貳,義旗多見猜阻。
 斛斯椿等以元忠淡於榮利,又不以世事經懷,故不在嫌嫉之地。尋兼中書令。



 天平初,復為太常。後加驃騎將軍。四年,除使持節、光州刺史。時州境災儉,人皆菜色,元忠表求賑貸,俟秋徵收。被報,聽用萬石。元忠以為萬石給人,計
 一家不過升斗而已,徒有虛名,不救其弊,遂出十五萬石以賑之。事訖表陳,朝廷嘉而不責。興和末,拜侍中。



 元忠雖居要任,初不以物務干懷,唯以聲酒自娛,大率常醉,家事大小,了不關心。園庭之內,羅種果藥,親朋尋詣,必留連宴賞。每挾彈攜壺,敖遊里閈,遇會飲酌,蕭然自得。常布言於執事云:「年漸遲暮,志力已衰,久忝名官,以妨賢路。若朝廷厚恩,未便放棄者,乞在閑冗,以養餘年。」武定元年,除東徐州刺史,固辭不拜。乃除驃騎大將軍、
 儀同三司。曾貢世宗蒲桃一盤。世宗報以百練縑,遺其書曰:「儀同位亞台鉉,識懷貞素,出藩入侍,備經要重。而猶家無擔石,室若懸磬,豈輕財重義,奉時愛己故也。久相嘉尚,嗟詠無極,恒思標賞,有意無由。



 忽辱蒲桃,良深佩帶。聊用絹百匹,以酬清德也。」其見重如此。孫騰、司馬子如嘗共詣元忠,見其坐樹下,擁被對壺,庭室蕪曠。謂二公曰:「不意今日披藜藿也。」



 因呼妻出,衣不曳地。二公相顧歎息而去,大餉米絹衣服,元忠受而散之。三年,
 復以本官領衛尉卿。其年卒於位,年六十。詔贈縑布五百匹,使持節、督定冀殷幽四州諸軍事、大將軍、司徒、定州刺史,謚曰敬惠。初,元忠將仕,夢手執炬火入其父墓,中夜驚起,甚惡之。旦告其受業師,占云:「大吉,此謂光照先人,終致貴達矣。」子搔嗣。



 搔,字德況,少聰敏,有才藝,音律博弈之屬,多所通解。曾采諸聲,別造一器,號曰八糸玄,時人稱其思理。起家司徒行參軍。累遷河內太守,百姓安之。入為尚書儀曹郎。天保八年卒。



 元忠族弟密,字希
 邕,平棘人也。祖伯膺,魏東郡太守,贈幽州刺史。父煥,治書侍御史、河內太守,贈青州刺史。密少有節操,屬爾朱兆殺逆,乃陰結豪右,與渤海高昂為報復之計。屬高祖出山東,密以兵從舉義,遙授并州刺史,封容城縣侯,邑四百戶。爾朱兆至廣阿,高祖令密募殷、定二州兵五千人鎮黃沙、井陘二道。



 及兆韓陵敗還晉陽,隨軍平兆。高祖乃以薛循義行并州事,授密建州刺史。又除襄州刺史。在州十餘年,甚得安邊之術,威信聞於外境。高祖頻
 降手書勞問,并賜口馬。侯景外叛,誘密執之,授以官爵。景敗歸朝,朝廷以密從景非元心,不之罪也。



 天保初,以舊功授散騎常侍,復本爵縣侯,卒。贈殿中尚書、濟州刺史。密性方直,有行檢。因母患積年,得名醫治療,不愈,乃精習經方,洞曉針藥,母疾得除。當世皆服其明解,由是亦以醫術知名。魏末行護軍司馬、武邑太守。天保初,司空長史。大寧、武平中,清河、廣平二郡守,銀青光祿大夫。齊亡後卒。子道謙,武平中,侍御史。道謙弟道貞,南青州
 司馬,為逆賊邢杲所殺。贈北徐州刺史。



 元忠宗人愍,字魔憐,形貌魁傑,見異於時。少有大志,年四十,猶不仕州郡,唯招致姦俠,以為徒侶。孝昌之末,天下兵起,愍潛居林慮山,觀候時變。賊帥鮮于脩禮、毛普賢作亂,詔遣大都督長孫稚討之。稚素聞愍名,召兼帳內統軍。軍達呼沱,賊來逆戰,稚軍為賊所敗,愍遂歸家。安樂王元鑒為北道大行臺,至鄴,以賊眾盛彊,未得前。遣使徵愍,表授武騎常侍、假節、別將,鎮鄴城東郭。葛榮之圍信都,餘黨
 南抄,陽平以北,皆為賊有。鑒命愍為前驅,別討之,頗有斬獲。及鑒謀逆,愍乃詐患暴風,鑒信之,因此得免。未幾,大都督源子邕屯安陽,大都裴衍屯鄴城,西討鑒。愍棄家口奔子邕,仍被徵赴洛,除奉車都尉,持節鎮汁河。汁河在鄴之西北,重山之中,並、相二州交境。以葛榮南逼,故用愍鎮之。榮遣其叔樂陵王葛萇率精騎一萬擊愍,愍據險拒戰,萇不得前。爾朱榮至東關,愍乃見榮。



 榮欲分賊勢,遣愍別道向襄國,襲賊署廣州刺史田怙軍。
 愍未至襄國,已擒葛榮。



 即表授愍建忠將軍。分廣平易陽、襄國,南趙郡之中丘三縣為易陽郡,以愍為太守。



 賜爵襄國侯。



 永安末,假平北將軍、持節、當郡大都督,遷樂平太守。未之郡,洛京傾覆,愍率所部西保石門山。潛與幽州刺史劉靈助及高昂兄弟、安州刺史盧曹等同契義舉。



 助敗,愍遂入石門。高祖建義,以書招愍,愍奉書,擁眾數千人以赴高祖,高祖親迎之。除使持節、征南將軍、都督相州諸軍事、相州刺史,兼尚書西南道行臺、當州
 都督。令愍率本眾西還舊鎮,高祖親送之。愍至鄉,據馬鞍山,依險為壘,徵糧集兵,以為聲勢。爾朱兆出井陘,高祖破兆於廣阿。愍統其本眾,屯故城以備爾朱兆。相州既平,命愍還鄴,除西南道行臺都官尚書,復屯故城。爾朱兆等將至,高祖征愍參守鄴城。



 太昌初,除太府卿。後出為南荊州刺史、當州大都督。此州自孝昌以來,舊路斷絕,前後刺史皆從間道始得達州。愍勒部曲數千人,徑向懸瓠,從北陽復舊道,且戰且前三百餘里,所經之
 處,即立郵亭,蠻左大服。梁遣其南司州刺史任思祖、隨郡太守桓和等率馬步三萬,兼發邊蠻,圍逼下溠戍。愍躬自討擊,破之。詔加車騎將軍。愍於州內開立陂梁,溉稻千餘頃,公私賴之。轉行東荊州,仍除驃騎將軍、東荊州刺史、當州大都督,加散騎常侍。天平二年卒。贈使持節、定殷二州軍事、儀同、定州刺史。



 元忠族叔景遺,少雄武有膽力,好結聚亡命,共為劫盜,鄉里每患之。永安末,其兄南巨鹿太守無為以贓罪為御史糾劾,禁於州獄。
 景遺率左右十餘騎,詐稱臺使,徑入州城,劫無為而出之。州軍追討,竟不能制。由是以俠聞。及高祖舉義於信都,景遺赴於軍門。高祖素聞其名,接之甚厚。命與元忠舉兵於西山,仍與大軍俱會,擒刺史爾朱羽生。以功除龍驤將軍,昌平縣公,邑八百戶。爾朱兆來伐,又力戰有功,除使持節、大都督、左將軍。太昌初,進爵昌平郡公,增邑三百戶,加車騎將軍。天平初,出為潁州刺史。未幾,為前潁川太守元洪威所襲殺。贈侍中、殷滄二州軍事、大
 將軍、開府、殷州刺史。子伽林襲。



 盧文偉,字休族,范陽涿人也。為北州冠族。父敞,出後伯假。文偉少孤,有志尚,頗涉經史,篤於交遊,少為鄉閭所敬。州辟主簿。年三十八,始舉秀才。除本州平北府長流參軍,說刺史裴俊按舊迹修督亢陂,溉田萬餘頃,民賴其利,修立之功,多以委文偉。文偉既善於營理,兼展私力,家素貧儉,因此致富。孝昌中,詔兼尚書郎中,時行臺常景啟留為行臺郎中。及北方將亂,文偉積稻穀於范陽
 城,時經荒儉,多所賑贍,彌為鄉里所歸。尋為杜洛周所虜。洛周敗,復入葛榮,榮敗,歸家。時韓樓據薊城,文偉率鄉閭屯守范陽,與樓相抗。乃以文偉行范陽郡事。防守二年,與士卒同勞苦,分散家財,拯救貧乏,莫不人人感說。爾朱榮遣將侯深討樓,平之,文偉以功封大夏縣男,邑二百戶,除范陽太守。深乃留鎮范陽。及榮誅,文偉知深難信,乃誘之出獵,閉門拒之。深失據,遂赴中山。



 莊帝崩,文偉與幽州刺史劉靈助同謀起義。靈助克瀛州,留
 文偉行事,自率兵赴定州,為爾朱榮將侯深所敗,文偉棄州,走還本郡,仍與高乾邕兄弟共相影響。



 屬高祖至信都,文偉遣子懷道奉啟陳誠,高祖嘉納之。中興初,除安東將軍、安州刺史。時安州未賓,仍居帥任,行幽州事,加鎮軍、正刺史。時安州剌史盧曹亦從靈助舉兵,助敗,因據幽州降爾朱兆,兆仍以為刺史,據城不下。文偉不得入州,即於郡所為州治。太昌初,遷安州刺史,累加散騎常侍。天平末,高祖以文偉行東雍州事,轉行青州事。



 文偉性輕財,愛賓客,善於撫接,好行小惠,是以所在頗得人情,雖有受納,吏民不甚苦之。經紀生資,常若不足,致財積聚,承候寵要,餉遺不絕。興和三年卒於州,年六十。贈使持節、侍中、都督定瀛殷三州軍事、司徒、尚書左僕射、定州刺史,謚曰孝威。



 子恭道,性溫良,頗有文學。州辟主簿。李崇北征,以為開府墨曹參軍。自文偉據范陽,屢經寇難,恭道常助父防守。七兵尚書郭秀素與恭道交款,及任事,每稱薦之,高祖亦聞其名。天平初,特除龍
 驤將軍、范陽太守。在郡有德惠。先文偉卒。贈使持節、都督幽平二州軍事、幽州刺史、度支尚書,謚曰定。



 子詢祖,襲祖爵大夏男。有術學,文章華靡,為後生之俊。舉秀才入京。李祖勳嘗宴文士,顯祖使小黃門敕祖勳曰:「茹茹既破,何故無賀表?」使者佇立待之。



 諸賓皆為表,詢祖俄頃便成。後朝廷大遷除,同日催拜。詢祖立於東止車門外,為二十餘人作表,文不加點,辭理可觀。



 詢祖初襲爵封大夏男,有宿德朝士謂之曰:「大夏初成。」應聲答曰:「且
 得燕雀相賀。」天保末,以職出為築長城子使。自負其才,內懷鬱怏,遂毀容服如賤役者,以見楊愔。愔曰:「故舊皆有所縻,唯大夏未加處分。」詢祖厲聲曰:「是誰之咎!」既至役所,作《築長城賦》,其略曰:「板則紫柏,杵則木瓜,何斯材而斯用也?草則離離靡靡,緣崗而殖,但使十步而有一芳,餘亦何辭間於荊棘。」



 邢邵曾戲曰:「卿少年才學富盛,戴角者無上齒,恐卿不壽。」對曰:「詢祖初聞此言,實懷恐懼,見丈人蒼蒼在鬢,差以自安。」邵甚重其敏贍。既有口
 辯,好臧否人物,嘗語人曰:「我昨東方未明,過和氏門外,已見二陸兩源,森然與槐柳齊列。」蓋謂彥師、仁惠與文宗、那延也,邢邵盛譽盧思道,以詢祖為不及。詢祖曰:「見未能高飛者借其羽毛,知逸勢沖天者剪其翅翮。」謗毀日至,素論皆薄其為人。



 長廣太守邢子廣目二盧云:「詢祖有規檢禰衡,思道無冰棱文舉。」後頗折節。歷太子舍人、司徒記室,卒官。有文集十卷,皆致遺逸。嘗為趙郡王妃鄭氏製挽歌詞,其一篇云:「君王盛海內,伉儷盡
 寰中。女儀掩鄭國,嬪容映趙宮。春艷桃花水,秋度桂枝風。遂使叢臺夜,明月滿床空。」



 恭道弟懷道,性輕率好酒,頗有慕尚,以守范陽勳,出身員外散騎侍郎。文偉遣奉啟詣高祖。中興初,加平西將軍、光祿大夫。元象初,行臺薛琡表行平州事,征赴霸府。興和中,行汾州事。懷道家預義舉,高祖親待之,出為烏蘇鎮城都督,卒官。



 懷道弟宗道,性粗率,重任俠。歷尚書郎、通直散騎常侍,後行南營州刺史。



 嘗於晉陽置酒,賓遊滿坐。中書舍人馬士達
 目其彈箜篌女妓云:「手甚纖素。」宗道即以此婢遺士達,士達固辭,宗道便命家人將解其腕,士達不得已而受之。將赴營州,於督亢陂大集鄉人,殺牛聚會。有一舊門生酒醉,言辭之間,微有疏失,宗道遂令沉之於水。後坐酷濫除名。



 文偉族人勇,字季禮,父璧,魏下邳太守。勇初從兄景裕俱在學,其叔同稱之曰:「白頭必以文通,季禮當以武達,興吾門在二子也。」幽州反者僕骨那以勇為本郡范陽王,時年十八。後葛榮作亂,又以勇為燕王。



 義
 旗之起也,盧文偉召之,不應。爾朱滅後,乃赴晉陽。高祖署勇丞相主簿。



 屬山西霜儉,運山東鄉租輸,皆令載實,違者治罪,令勇典其事。琅邪公主虛僦千餘車,勇繩劾之。公主訴於高祖,而勇守法不屈。高祖謂郭秀曰:「盧勇懍懍有不可犯之色,真公直人也,方當委之大事,豈直納租而已。」遷汝北太守,行陜州事,轉行洛州事。



 元象元年,官軍圍廣州,數旬未拔。行臺侯景聞西魏救兵將至,集諸將議之。



 勇進觀形勢,於是率百騎,各籠一匹馬。至
 大隗山,知魏將李景和率軍將至。勇多置幡旗於樹頭,分騎為十隊,鳴角直前,擒西魏儀同程華,斬儀同王征蠻,驅馬三百匹,通夜而還。廣州守將駱超以城降,高祖令勇行廣州事。



 以功授儀同三司、陽州刺史,鎮宜陽。叛民韓木蘭、陳忻等常為邊患,勇大破之。啟求入朝,高祖賜勇書曰:「吾委卿陽州,唯安枕高臥,無西南之慮矣。但依朝廷所委,表啟宜停。卿之妻子任在州住,當使漢兒之中無在卿前者。」武定二年卒,年三十二。勇有馬五百
 匹,繕造甲仗六車,遺啟盡獻之朝廷。賻物之外,別賜布絹四千匹。贈司空、冀州刺史,謚曰武貞侯。



 李義深,趙郡高邑人也。祖真,魏中書侍郎。父紹宗,殷州別駕。義深學涉經史,有當世才用。解褐濟州征東府功曹參軍,累加龍驤將軍。義旗初,歸高祖於信都,以為大行臺郎中。中興初,除平南將軍、鴻臚少卿。義深見爾朱兆兵盛,遂叛高祖奔之。兆平,高祖恕其罪,以為大丞相府記室參軍。累遷左光祿大夫、相府司馬,所經稱職。轉
 并州長史。時刺史可朱渾道元不親細務,民事多委義深,甚濟機速。復為大丞相司馬。武定中,除齊州剌史,好財利,多所受納。天保初,行鄭州事,轉行梁州事,尋除散騎常侍,為陽夏太守。段業告其在州聚斂,被禁止,送梁州窮治,未竟。三年,遇疾,卒於禁所,年五十七。



 子騊駼,有才辯,尚書郎、鄴縣令,武平初,兼通直散騎常侍。聘陳,為陳人所稱。後為壽陽道行臺左丞,與王琳等同陷。周末逃歸。開皇初,永安太守。卒於絳州長史。



 子正藻,明敏有
 才幹。武平末,儀同開府行參軍、判集書省事。以父騊餘沒陳,正藻便謝病解職,憂思毀瘠,居處飲食若在喪之禮,人士稱之。隋開皇中,歷尚書工部員外郎、盩厔縣令。卒於宜州長史。



 騊駼弟文師,中書舍人、齊郡太守。



 義深兄弟七人,多有學尚。第二弟同軌,以儒學知名。第六弟,稚廉別有傳。



 義深族弟神威。曾祖融,魏中書侍郎,神威幼有風裁,傳其家業,禮學粗通義訓。又好音樂,撰集《樂書》,近於百卷。魏武之末,尚書左丞。天保初卒。贈信州刺
 史。



 史臣曰:元忠本自素流,有聞教義,人倫之譽,未以縱橫許之。屬莊帝幽崩,群胡矯擅,士之有志力者皆望勤王之師。及高祖東轅,事與心會,一遇雄姿,遂瀝肝膽,以石投水,豈徒然哉?既享功名,終知止足,進退之道,有可觀焉。文偉望重地華,早有志尚,間關夷險之際,終遇英雄之主,雖禮秩未弘,亦為佐命之一。



 詢祖詞情艷發,早著聲名,負其才地,肆情矜矯,京華人士,莫不畏其舌端。任
 遇未聞,弱年夭逝,若得終介眉壽,通塞未可量焉。



 贊曰:晉陽、大夏,抱質懷文。蹈仁履義,咸會風雲。盧嬰貨殖,李厭囂氛。



 始終之操,清濁斯分。義深參贊,有謝忠勤。



\end{pinyinscope}