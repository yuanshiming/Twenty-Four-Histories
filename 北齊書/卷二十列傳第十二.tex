\article{卷二十列傳第十二}

\begin{pinyinscope}

 張
 瓊斛律羌舉堯雄宋顯王則慕容紹宗薛循義叱列平步大汗薩慕容儼張瓊,字連德,代人也。少壯健,有武用。魏世自盪寇將軍
 為朔州征虜府外兵參軍,隨葛榮為亂。榮敗,爾朱榮以為都督。討元顥有功,除汲郡太守。建明初,為東道尉勞大使,封行唐縣子,邑三百戶。轉太尉長史。出為河內太守,除濟州刺史。爾朱兆敗,歸高祖,遷汾州刺史。天平中,高祖襲克夏州,以為尉勞大使,仍留鎮之。尋為周文帝所陷,卒。贈使持節、燕恒雲朔四州諸軍事、大將軍、司徒公、恆州刺史。有二子。長忻,次遵業。



 忻,普泰中為都督,隨爾朱世隆。以功尚魏平陽公主,除駙馬都尉、大將軍、開
 府儀同三司、建州刺史、南鄭縣伯。瓊常憂其太盛,每語親識曰:「凡人官爵,莫若處中,忻位秩太高,深為憂慮。」而忻豪險放縱,遂與公主情好不協,尋為武帝所害,時稱瓊之先見。



 遵業,討元顥有功,封固安縣開國子,除寧遠將軍、雲州大中正。天平中,除清河太守,尋加安西將軍、建州刺史。武定中,隨儀同劉豐討侯景,為景所擒。景敗,殺遵業於渦陽。喪還,世宗親自臨弔,贈并肆幽安四州軍事、開府儀同三司、并州刺史。



 斛律羌舉,太安人也。世為部落酋長。父謹,魏龍驤將軍、武川鎮將。羌舉少驍果,有膽力。永安中,從爾朱兆入洛,有戰功,深為兆所愛遇,恒從征伐。高祖破兆,方始歸誠。高祖以其忠於所事,亦加嗟賞。天平中,除大都督,令率步騎三千導眾軍西襲夏州,克之。後從高祖西討,大軍濟河,集諸將議進趣之計。羌舉曰:「黑獺聚兇黨,強弱可知,若欲固守,無糧援可恃。今揣其情,已同困獸,若不與其戰,而逕趣咸陽,咸陽空虛,可不戰而克。拔其根本,彼
 無所歸,則黑獺之首懸於軍門矣。」諸將議有異同,遂戰於渭曲,大軍敗績。



 天平末,潁川人張儉聚眾反叛,西通關右,羌舉隨都督侯景、高昂等討破之。



 元象中,除清州刺史,封密縣侯。興和初,高祖以為中軍大都督,尋轉東夏州刺史。



 時高祖欲招懷遠夷,令羌舉使於阿至羅,宣揚威德,前後稱旨,甚被知賞。卒於州,時年三十六。高祖深悼惜之。贈并恒二州軍事、恒州刺史。



 子孝卿,少聰敏幾悟,有風檢,頻歷顯職。武平末,侍中、開府儀同三司,封
 義寧王,知內省事,典外兵、騎兵機密。是時,朝綱日亂,政由群豎。自趙彥深死,朝貴典機密者,唯孝卿一人差居雅道,不至貪穢。後主至齊州,以孝卿為尚書令。



 又以中書侍郎薛道衡為侍中,封北海王。二人勸後主作承光主詔,禪位任城王,令孝卿齎詔策及傳國璽往瀛州。孝卿便詣鄴城,歸於周武帝,仍從入長安,授納言上士。隋開皇中,位太府卿,卒於民部尚書。



 代人劉世清,祖拔,魏燕州刺史;父巍,金紫光祿大夫。世清武平末侍中、開府
 儀同三司,任遇與孝卿相亞。情性甚整,周慎謹密,在孝卿之右。能通四夷語,為當時第一。後主命世清作突厥語翻《涅槃經》,以遺突厥可汗,敕中書侍郎李德林為其序。世清隋開皇中卒於開府、親衛驃騎將軍。



 堯雄,字休武,上黨長子人也。祖暄,魏司農卿。父榮,員外侍郎,雄少驍果,善騎射,輕財重氣,為時輩所重。永安中,拜宣威將軍、給事中、持節慰勞恒燕朔三州大使。仍為都督,從叱列延討劉靈助,平之,拜鎮東將軍、燕州刺史,
 封城平縣伯,邑五百戶。



 義旗初建,雄隨爾朱兆敗於廣阿,遂率所部據定州以歸高祖。時雄從兄傑,爾朱兆用為滄州刺史,至瀛州,知兆敗,亦遣使歸降。高祖以其兄弟俱有誠款,便留傑行瀛州事,尋以雄為車騎大將軍、瀛州刺史以代傑,進爵為公,增邑五百戶。于時禁網疏闊,官司相與聚斂,唯雄義然後取,復能接下以寬恩,甚為吏民所懷附。



 魏武帝入關,雄為大都督,隨高昂破賀拔勝於穰城。周旋征討三荊,仍除二豫、揚、郢四州都督、
 豫州刺史。元洪威據潁州叛,民趙繼宗殺潁川太守邵招,據樂口,自稱豫州刺史,北應洪威。雄率眾討之,繼宗敗走。民因雄之出,遂推城人王長為刺史,據州引西魏。雄復與行臺侯景討平之。梁將李洪芝、王當伯襲破平鄉城,侵擾州境。雄設伏要擊,生擒洪芝、當伯等,俘獲甚眾。梁司州刺史陳慶之復率眾逼州城,雄出與戰,所向披靡,身被二創,壯氣益厲,慶之敗,棄輜重走。後慶之復圍南荊州,雄曰:「白茍堆,梁之北面重鎮,因其空虛,攻之
 必克,彼若聞難,荊圍自解,此所謂機不可失也。」遂率眾攻之,慶之果棄荊州來。未至,雄陷其城,擒梁鎮將茍元廣,兵二千人。梁以元慶和為魏王,侵擾南城。雄率眾討之,大破慶和於南頓。尋與行臺侯景破梁楚城。豫州民上書,更乞雄為刺史,復行豫州事。



 潁州長史賀若徽執刺史田迅,據州降西魏,詔雄與廣州刺史趙育、揚州刺史是雲寶等各總當州士馬,隨行臺任延敬并勢攻之。西魏遣其將怡鋒率眾援之,延敬等與戰失利。育、寶各還
 本州,據城降敵。雄收集散卒,保大梁。周文帝因延敬之敗,遣其右丞韋孝寬等攻豫州。雄都督郭丞伯、程多寶等舉豫州降敵,執刺史馮邕并家屬及部下妻子數千口,欲送之長安。至樂口,雄外兵參軍王恒伽、都督赫連俊等數十騎從大梁邀之,斬多寶,拔雄等家口還大梁。西魏以丞伯為潁川太守,雄仍與行臺侯景討之。雄別攻破樂口,擒丞伯。進討懸瓠,逐西魏刺史趙繼宗、韋孝寬等。



 復以雄行豫州事。西魏以是雲寶為揚州刺史,
 據項城;義州刺史韓顯據南頓。雄復率眾攻之,一日拔其二城,擒顯及長史丘岳,寶遁走,獲其妻妾將吏二千人,皆傳送京師。加驃騎大將軍。仍隨侯景平魯陽,除豫州刺史。



 雄雖武將,而性質寬厚,治民頗有誠信,為政去煩碎,舉大綱而已。撫養兵民,得其力用。在邊十年,屢有功績,豫人於今懷之。又愛人物,多所施與,賓客往來,禮遺甚厚,亦以此見稱。興和三年,徵還京師,尋領司、冀、瀛、定、齊、青、膠、兗、殷、滄十州士卒十萬人,巡行西南,分守險
 要。四年,卒於鄴,時年四十四。



 贈使持節、都督青徐膠三州軍事、大將軍、司徒公、徐州刺史,謚武恭。子師嗣。



 雄弟奮,字彥舉。解褐宣威將軍、給事中,轉中堅將軍、金紫光祿大夫,賜爵安夷縣子。從高祖平鄴,破爾朱兆等,進爵為伯。出為南汾州刺史,胡夷畏憚之。



 西魏行臺薛崇禮舉眾攻奮,與戰,大破之,崇禮兄弟乞降,送於相府。轉奮驃騎將軍、左光祿大夫、潁州刺史,卒。贈兗豫梁三州諸軍事、司空、兗州刺史。



 雄從父兄傑,字壽。性輕率,嗜酒,頗
 有武用。歷給事中、羽林監。從高祖破紇豆陵步藩有功,除鎮東將軍。封樂城縣伯,邑百戶。出為滄州刺史。屬義兵起,歸高祖。從平鄴及破爾朱兆,進爵為侯。後為都督,率眾隨樊子鵠討元樹於譙城,平之。仍除南兗州,多所取受,然性果決,吏民畏之。尋加行兗州事。元象初,拜車騎大將軍、儀同三司,進爵為公。出為磨城鎮大都督,轉安州刺史,卒於州。贈使持節、滄瀛二州諸軍事、尚書右僕射,滄州刺史,謚曰(闕)。



 宋顯,字仲華,敦煌效穀人也。性果敢,有乾用。初事爾朱榮為軍主,擢為長流參軍。永安中,除前軍、襄垣太守,轉榮府記室參軍。從平元顥,加平東將軍。



 榮死,世隆等向洛,復以顯為襄垣太守。普泰初,遷使持節、征北將軍、晉州刺史。



 後歸高祖,以為行臺右丞。樊子鵠據兗州反,前西兗州刺史乙瑗、譙郡太守辛景威屯據五梁,以應子鵠。高祖以顯行西兗州事,率眾討破之,斬瑗,景威遁走。拜西兗州刺史。時梁州刺史鹿永吉據州外叛,西魏遣
 博陵王元約、趙郡王元景神率眾迎接。顯勒當州士馬邀破之,斬約等,仍與左衛將軍斛律平共會大梁。拜儀同三司。



 在州多所受納,然勇決有氣幹,檢御左右,咸能得其心力。及河陰之戰,深入赴敵,遂沒于行陣。贈司空公。



 顯從祖弟繪,少勤學,多所博覽,好撰述,魏時,張緬《晉書》未入國,繪依準裴松之注《國志》體,注王隱及《中興書》。又撰《中朝多士傳》十卷,《姓系譜錄》五十篇。以諸家年歷不同,多有紕繆,乃刊正異同,撰《年譜錄》,未成,河清五年
 並遭水漂失。繪雖博聞彊記,而天性恍惚,晚又遇風疾,言論遲緩。及失所撰之書,乃撫膺慟哭曰:「可謂天喪予也!」天統中卒。



 王則,字元軌,自云太原人也。少驍果,有武藝。初隨叔父魏廣平內史老生征討,每有戰功。老生為朝廷所知,則頗有力。初以軍功除給事中,賜爵白水子。後從元天穆討邢杲,輕騎深入,為杲所擒。元顥入洛,則與老生俱降顥,顥疑老生,遂殺之。則奔廣州刺史鄭先護,與同拒
 顥,顥敗,遷征虜將軍,出為東徐州防城都督。



 爾朱榮之死也,東徐州刺史斛斯椿其枝黨,內懷憂怖,時梁立魏汝南王悅為魏主,資其士馬,送境上,椿遂翻城降悅,則與蘭陵太守李義擊其偏師,破之。魏因以則行北徐州事,後隸爾朱仲遠仲遠敗,始歸高祖。仍加征南將軍、金紫光祿大夫。



 初隨荊州刺史賀拔勝,後從行臺侯景,周旋征討,屢有功績。天平初,行荊州事,都督三荊、二襄、南雍六州軍事,荊州刺史。則有威武,邊人畏服之。渭曲之
 役,則為西師圍逼,遂棄城奔梁。梁尋放還,高祖怒而不責。元象初,除洛州刺史。則性貪婪,在州取受非法,舊京諸像,毀以鑄錢,于時世號河陽錢,皆出其家。武定中,復隨侯景西討。景於潁川作逆,時則鎮柏崖戍,世宗以則有武用,徵為徐州刺史。景既南附,梁遣貞陽侯蕭明率大眾向徐州,以為影響,堰泗水灌州城。則固守歷時,而取受狼藉,鎖送晉陽,世宗恕其罪。武定七年春,卒,時年四十八。贈青齊二州軍事、司空、青州刺史,謚曰烈懿。



 則
 弟敬寶,少歷顯位。後為東廣州刺史,與蕭軌等攻建業,不剋,沒焉。



 慕容紹宗,慕容晃第四子太原王恪後也,曾祖騰,歸魏,遂居於代。祖都,岐州刺史。父遠,恆州刺史。紹宗容貌恢毅,少言語,深沉有膽略。爾朱榮即其從舅子也。值北邊撓亂,紹宗攜家屬詣晉陽以歸榮,榮深待之。及榮稱兵入洛,私告紹宗曰:「洛中人士繁盛,驕侈成俗,若不加除剪,恐難制馭。吾欲因百官出迎,仍悉誅之,爾謂可不?」紹
 宗對曰:「太后臨朝,淫虐無道,天下憤惋,共所棄之。



 公既身控神兵,心執忠義,忽欲殲夷多士,謂非長策,深願三思。」榮不從。後以軍功封索盧縣子。尋進爵為侯。從高祖破羊侃,又與元天穆平邢杲,累遷并州刺史。



 紇豆陵步藩逼晉陽,爾朱兆擊之,累為步藩所破,欲以晉州征高祖共圖步藩。



 紹宗諫曰:「今天下擾擾,人懷覬覦。正是智士用策之秋。高晉州才雄氣猛,英略蓋世,譬諸蛟龍,安可借以雲雨!」兆怒曰:「我與晉州推誠相待,何忽輒相猜
 阻,橫生此言!」便禁止紹宗,數日方釋。遂割鮮卑隸高祖。高祖共討步藩,滅之。及高祖舉義信都,兆以紹宗為長史,又命為行臺,率軍壺關,以抗高祖。及廣阿、韓陵之敗,兆乃撫膺自咎,謂紹宗曰:「比用卿言,今豈至此!」



 兆之敗於韓陵也,士卒多奔,兆懼,將欲潛遁。紹宗建旗鳴角,招集義徒,軍容既振,與兆徐而上馬。後高祖從鄴討兆於晉陽,兆窘急,走赤谼嶺,自縊而死。



 紹宗行到烏突城,見高祖追至,遂攜榮妻子及兆餘眾自歸。高祖仍加恩禮,
 所有官爵並如故,軍謀兵略,時參預焉。



 天平初,遷都鄴,庶事未周,乃令紹宗與高隆之共知府庫圖籍諸事。二年,宜陽民李延孫聚眾反,乃以紹宗為西南道軍司,率都督厙狄安盛等討破之。軍還,行揚州刺史,尋行青州刺史。丞相府記室孫搴屬紹宗以兄為州主簿,紹宗不用。搴譖之於高祖,云:「慕容紹宗嘗登廣固城長歎,謂其所親云『大丈夫有復先業理不』。」



 由是徵還。元象初,西魏將獨孤如願據洛州,梁、潁之間,寇盜鋒起。高祖命紹宗
 率兵赴武牢,與行臺劉貴等平之。進爵為公,除度支尚書。後為晉州刺史、西道大行臺,還朝,遷御史中尉。屬梁人劉烏黑入寇徐方,令紹宗率兵討擊之,大破,因除徐州刺史。烏黑收其散眾,復為侵竊,紹宗密誘其徒黨,數月間,遂執烏黑殺之。



 侯景反叛,命紹宗為東南道行臺,加開府,轉封燕郡公,與韓軌等詣瑕丘,以圖進趣。梁武帝遣其兄子貞陽侯淵明等率眾十萬,頓軍寒山,與侯景掎角,擁泗水灌彭城。仍詔紹宗為行臺,節度三徐、二兗
 州軍事,與大都督高岳等出討,大破之,擒淵明及其將帥等,俘虜其眾。乃回軍討侯景於渦陽。于時景軍甚眾,前後諸將往者莫不為其所輕。及聞紹宗與岳將至,深有懼色,謂其屬曰:「岳所部兵精,紹宗舊將,宜共慎之。」於是與景接戰,諸將持疑,無肯先者,紹宗麾兵徑進,諸將從之,因而大捷,景遂奔遁。軍還,別封永樂縣子。初,高祖末,命世宗云:「侯景若反,以慕容紹宗當之。」至是,竟立功效。



 西魏遣其大將王思政入據潁州,又以紹宗為南道
 行臺,與太尉高岳、儀同劉豐等率軍圍擊,堰洧水以灌之。時紹宗頻有凶夢,意每惡之。乃私謂左右曰:「吾自年二十已還,恆有蒜髮,昨來蒜髮忽然自盡。以理推之,蒜者算也,吾算將盡乎?」



 未幾,與豐臨堰,見北有塵氣,乃入艦同坐。暴風從東北來,遠近晦冥,舟纜斷,飄艦徑向敵城。紹宗自度不免,遂投水而死,時年四十九。三軍將士莫不悲惋,朝廷嗟傷。贈使持節、二青二兗齊濟光七州軍事、尚書令、太尉、青州刺史,謚曰景惠。除其長子士肅
 為散騎常侍。尋以謀反伏誅。朝廷以紹宗功,罪止士肅身。皇建初,配饗世宗廟庭。士肅弟建中,襲紹宗爵。武平末,儀同三司。隋開皇中,大將軍、疊州總管。



 薛循義,字公讓,河東汾陰人也。曾祖紹,魏七兵尚書、太子太保。祖壽仁,河東河北二郡守、秦州刺史、汾陰公。父寶集,定陽太守。



 循義少而姦俠,輕財重氣,招召膏猾,時有急難相奔投者,多能容匿之。魏咸陽王為司州牧,用為法曹從事。魏北海王顥鎮徐州,引為墨曹參軍。正光
 末,天下兵起,顥為征西將軍,都督華、豳、東秦諸軍事,兼左僕射、西道行臺,以循義為統軍。時有詔,能募得三千人別將。於是循義還河東,仍歷平陽、弘農諸郡,合得七千餘人,即假安北將軍、西道別將。俄而東西二夏、南北兩華及豳州等反叛,顥進討之。循義率所部,頗有功。絳蜀賊陳雙熾等聚汾曲,詔循義為大都督,與行臺長孫稚共討之。循義以雙熾是其鄉人,遂輕詣壘下,曉以利害,熾等遂降。拜循義龍門鎮將。



 後循義宗人鳳
 賢等作亂,圍鎮城。循義亦以天下紛擾,規自縱擅,遂與鳳賢聚眾為逆,自號黃鉞大將軍。詔都督宗正珍孫討之。軍未至,循義慚悔,乃遣其帳下孫懷彥奉表自陳,乞一大將招慰。魏孝明遣西北大行臺胡元吉奉詔曉喻,循義降。



 鳳賢等猶據險屯結,長孫稚軍於弘農,珍孫軍靈橋,未能進。循義與其從叔善樂、從弟嘉族等各率義勇為攻取之勢,與鳳賢書示其禍福。鳳賢降,拜鳳賢龍驤將軍、假節、稷山鎮將,夏陽縣子、邑三百戶。封循義
 汾陰縣侯,邑八百戶。



 爾朱榮以循義豪猾反覆,錄送晉陽,與高昂等並見拘防。榮赴洛,以循義等自隨,置於駝牛署。榮死,魏孝莊以循義為弘農、河北、河東、正平四郡大都督。時高祖為晉州刺史,見循義,待之甚厚。及爾朱兆立魏長廣王為主,除循義右將軍、陜州刺史,假安南將軍。魏前廢帝初,以循義為持節、後將軍、南汾州刺史。



 高祖起義信都,破四胡於韓陵,遣征循義,從至晉陽,以循義行并州事。又從高祖平爾朱兆。武帝之入關也,高
 祖奉迎臨潼關,以循義為關右行臺,自龍門濟河。



 西魏北華州刺史薛崇禮屯楊氏壁,循義以書招之,崇禮率萬餘人降。樊子鵠之據兗州,循義從大司馬婁昭破平之。天平中,除衛將軍、南中郎將,帶汲郡太守、頓丘、淮陽、東郡、黎陽五郡都督。遷東徐州。元象初,拜儀同。沙苑之役,從諸軍退。



 還,行晉州事。封祖業棄城走,循義追至洪洞,說祖業還守,而祖業不從。循義還據晉州,安集固守。西魏儀同長孫子彥逼城下,循義開門伏甲以待之,
 子彥不測虛實,於是遁去。高祖甚嘉之,就拜晉州刺史、南汾、東雍、陜四州行臺,賞帛千匹。



 循義在州,擒西魏所署正平太守段榮顯。招降胡酋胡垂黎等部落數千口,表置五城郡以安處之。高仲密之叛,以循義為西南道行臺,為掎角聲勢,不行。尋除齊州刺史,以黔貨除名。追其前守晉州功,復其官爵,仍拜衛尉卿。時山胡侵亂晉州,遣循義追討,破之。進爵正平郡公,加開府。世宗以高祖遺旨,減封二百戶,別封循義為平鄉男。天保初,除護
 軍,別封藍田縣公,又拜太子太保。五年七月卒,時年七十七。贈晉太華三州諸軍事、司空、晉州刺史,贈物三百段。子文殊嗣。



 循義從弟嘉族,性亦豪爽。釋褐員外散騎侍郎,稍遷正平太守。屬高祖在信都,嘉族聞而赴義。從平四胡於韓陵,除華州刺史。及賀拔岳拒命,令嘉族置騎河上,以禦大軍。嘉族遂棄其乘馬,浮河而度,歸於高祖。由是拜揚州刺史,卒於官。子震,字文雄。天平初,受旨鎮守龍門,陷於西魏。元象中,方得逃還。高祖嘉其至誠,
 除廣州刺史。後從慕容紹宗討侯景,以功別封膚施縣男。天保四年,從討山胡,破茹茹,並有功績,累遷譙州刺史。



 循義從子元穎,父光熾,東雍州刺史、太常卿。元穎廉謹有信義,起家永安王參軍。行秀容縣事,有清名。累轉定州別駕,舉清平勤幹,除漁陽太守。



 叱列平,字殺鬼,代郡西部人也,世為酋帥。平有容貌,美須髯,善騎射。襲第一領民酋長,臨江伯。孝昌末,拔陵反叛,茹茹餘眾入寇馬邑,平以統軍屬,有戰功,補別將。後
 牧子作亂,劉胡崙、斛律可那律俱時構逆,以平為都督,討定胡崙等。魏孝莊初,除武衛將軍。隨爾朱榮破葛榮,平元顥,遷中軍都督、右衛將軍,封癭陶縣伯,邑七百戶。榮死,平與榮妻及爾朱世隆等北走。長廣王曄立,授右衛將軍,加京畿大都督。



 時爾朱氏凌僭,平常慮危禍,會高祖起義,平遂歸誠。從平鄴,破四胡於韓陵。



 仲遠既走,以平為東郡大行臺。軍還,從高祖平爾朱兆。復從領軍婁昭討樊子鵠平之。授使持節、華州刺史。高仲密之叛,
 平從高祖破周文帝於邙山。武定初,除廓州刺史。五年,加儀同三司,鎮河陽。八年,進爵為侯。天保初,授兗州刺史,尋加開府,別封臨洮縣子。三年,與諸將南討江淮,克陽平郡。陳人攻圍廣陵,詔平統河南諸軍赴援,陳人退,乃還。五年夏,卒於州,時年五十一。贈瀛滄幽三州軍事、瀛州刺史、中書監,謚曰莊惠。子孝中嗣。



 弟長乂。武平末,侍中、開府儀同三司,封新寧王。隋開皇中,上柱國,卒於涇州長史。雖無他伎,前在官以清乾著稱。



 步大汗薩,太安狄那人也。曾祖榮,仕魏歷金門、化正二郡太守。父居,龍驤將軍、領民別將。正光末,六鎮反亂,薩乃將家避難南下,奔爾朱榮於秀容。後從榮入洛,以軍功除揚武軍帳內統軍,賜爵江夏子。從平葛榮,累前後功,加鎮南將軍。榮死後,從爾朱兆入洛,補帳內大都督,從兆拒戰於韓陵。兆敗,薩以所部降。



 高祖以為第三領民酋長,累遷秦州鎮城都督、北雍州刺史。天平中,轉東壽陽三泉都督。元象中,行燕州,累遷臨川領民大都督,
 賜爵長廣伯。時茹茹寇鈔,屢為邊害,高祖撫納之,遣薩將命。還,拜儀同三司。出為五城大都督,鎮河陽。又加車騎大將軍、開府,進封行唐縣公,減勃海三百戶以增其封。仍授晉州刺史,別封安陵縣男,邑二百戶,加驃騎大將軍。齊受禪,改封義陽郡公。



 慕容儼,字恃德,清都成安人,慕容廆之後也。父叱頭,魏南頓太守,身長一丈,腰帶九尺。武平初,追贈開府儀同三司、尚書左僕射、持節、都督滄恒二州軍事、恒州刺史。



 儼容貌出群,衣冠甚偉,不好讀書,頗學兵法,工騎射。正光中,魏河間王元琛率眾救壽春,辟儼左廂軍主,以戰功賞帛五十匹。軍次西硤石,因解渦陽之圍,平倉陵城、荊山戍。梁遣將鄭僧等要戰,儼擊之,斬其將蕭喬,梁人奔遁。又襲破王神念等軍,擒二百餘人,神念僅以身免。三年,梁遣將攻東豫州,大都督元寶掌討之。儼為別將。鄭海珍與戰,斬其軍主朱僧珍、軍副秦太。又擊賊王茍於陽夏,平之。



 孝昌中,爾朱榮入洛,授儼京畿南面都督。
 永安中,西荊州為梁將曹義宗所圍,儼應募赴之。時北育太守宋帶劍謀叛,儼乃輕騎出其不意,直至城下,語云:「大軍已到,太守何不迎?」帶劍造次惶恐不知所為,便出迎,儼即執之,一郡遂定。



 又破梁將馬元達、蔡天起、柳白嘉等,累有功。除強弩將軍。與梁將王玄真、董當門等戰,並破之,解穰城圍,剋復南陽、新鄉。轉積射將軍,持節、豫州防城大都督。



 爾朱敗,與豫州刺史李恩歸高祖。以勳累遷安東將軍、高梁太守,轉五城太守、東雍州刺史。
 沙苑之敗,西魏荊州刺史郭鸞率眾攻儼,拒守二百餘日,晝夜力戰,大破鸞軍,追斬三百餘級,又擒西魏刺史郭他。時諸州多有翻陷,唯儼獲全。進號鎮南將軍。武定三年,率師解襄州圍。頻使茹茹。又從攻玉壁,賜帛七百匹並衣帽等。五年,鎮河橋五城。侯景叛,儼擊陳郡賊,獲景麾下庫狄曷賴及偽暑太守鄭道合、兗州刺史王彥夏、行臺狄暢等,擒斬百餘級。旋軍項城,又擒景偽署刺史辛光及蔡遵,并其部下二千人。六年,除譙州刺史,屢
 有戰功,多所降附。七年,又除膠州刺史。



 天保初,除開府儀同三司。六年,梁司徒陸法和、儀同宋蒨等率其部下以郢州城內附。時清河王岳帥師江上,乃集諸軍議曰:「城在江外,人情尚梗,必須才略兼濟,忠勇過人,可受此寄耳。」眾咸共推儼。岳以為然,遂遣鎮郢城。始入,便為梁大都督侯瑱、任約率水陸軍奄至城下。儼隨方禦備,瑱等不能剋。又於上流鸚鵡洲上造荻洪竟數里,以塞船路。人信阻絕,城守孤懸,眾情危懼,儼導以忠義,又悅以
 安之。城中先有神祠一所,俗號城隍神,公私每有祈禱。於是順士卒之心,乃相率祈請,冀獲冥祐。須臾,衝風欻起,驚濤涌激,漂斷荻洪。約復以鐵鎖連治,防禦彌切。儼還共祈請,風浪夜驚,復以斷絕,如此者再三。城人大喜,以為神功。



 瑱移軍於城北,造柵置營,焚燒坊郭,產業皆盡。約將戰士萬餘人,各持攻具,於城南置營壘,南北合勢。儼乃率步騎出城奮擊,大破之,擒五百餘人。先是郢城卑下,兼土疏頹壞,儼更修繕城雉,多作大樓。又造船
 艦,水陸備具,工無暫闕。蕭循又率眾五萬,與瑱、約合軍,夜來攻擊。儼與將士力戰終夕,至明,約等乃退。



 追斬瑱驍將張白石首,瑱以千金贖之,不與。夏五月,瑱、約等又相與并力,悉眾攻圍。城中食少,糧運阻絕,無以為計,唯煮槐楮、桑葉并紵根、水萍、葛、艾等草及靴、皮帶、觔角等物而食之。人有死者,即取其肉,火別分啖,唯留骸骨。儼猶申令將士,信賞必罰,分甘同苦,死生以之。自正月至於六月,人無異志。後蕭方智立,遣使請和。顯祖以城在
 江表,據守非便,有詔還之。儼望帝,悲不自勝。



 帝呼令至前,執其手,持儼鬚鬢,脫帽看髮,歎息久之。謂儼曰:「觀卿容貌,朕不復相識,自古忠烈,豈能過此!」儼對曰:「臣恃陛下威靈,得申愚節,不屈豎子,重奉聖顏。今雖夕死,沒而無恨。」帝嗟稱不已。除趙州刺史,進伯為公,賜帛一千匹、錢十萬。



 九年,又討賊有功,賜帛一百匹、錢十萬。十年,詔除揚州行臺,與王貴顯、侯子監將兵衛送蕭莊。築郭默、若邪二城。與陳新蔡太守魯悉達戰大蛇洞,破走之。



 又
 監蕭莊、王琳軍,與陳將侯瑱、侯安都戰於蕪湖,敗歸。皇建初,別封成陽郡公。



 天統二年,除特進。四年十月,又別封猗氏縣公,并賜金銀酒鐘各一枚、胡馬一匹。



 五年四月,進爵為義安王。武平元年,出為光州刺史。儼少任俠,交通輕薄,遨遊京洛間。及從征討,每立功效,經略雖非所長,而有將帥之節。所歷諸州,雖不能清白守道,亦不貪殘。卒,贈司徒、尚書令。子子顒,給事黃門侍郎。



 爾朱將帥,義旗建後歸順立功者,武威牒舍樂、代郡范舍樂亦
 致通顯。



 牒舍樂,少從爾朱榮為軍主、統軍,後西河領民都督。爾朱兆敗,率眾歸高祖,拜鎮西將軍、金紫光祿大夫。以都督隸侯景,破賀拔勝於穰城。又與諸將討平青、兗、荊三州,拜鎮西將軍、營州刺史。天保初,封漢中郡公。後因戰沒於關中。



 范舍樂,有武藝,筋力絕人。魏末,從崔暹、李崇等征討有功,授統軍。後入爾朱榮軍中,頻有戰功,授都督。後隨爾朱兆破步藩於梁都。高祖義旗舉,棄兆歸信都。從高祖破兆於廣阿、韓陵,並有功,賜爵平舒
 男。每從征役,多有剋捷。除相府左廂大都督。尋出為東雍州刺史。世宗嗣事,封平舒縣侯,拜儀同。天保中,進位開府。



 又有代人庫狄伏連,字仲山,少以武幹事爾朱榮,至直閣將軍。後從高祖建義,賜爵蛇丘男。世宗輔政,遷武衛將軍。天保初,儀同三司。四年,除鄭州刺史,尋加開府。伏連質朴,勤於公事。直衛官闕,曉夕不離帝所,以此見知。鄙吝愚狠,無治民政術。及居州任,專事聚斂。性又嚴酷,不識士流。開府參軍多是衣冠士族,伏連加以捶
 撻,逼遣築墻。武平中,封宜都郡王,除領軍大將軍。尋與郎琊王儼殺和士開,伏誅。伏連家口有百數,盛夏之日,料以倉料二升,不給鹽菜,常有饑色。冬至之日,親表稱賀,妻為設豆餅。伏連問此豆因何而得,妻對向於食馬豆中分減充用,伏連大怒,典馬、掌食之人並加杖罰,積年賜物,藏在別庫,遣侍婢一人專掌管籥。每入庫檢閱,必語妻子云:「此是官物,不得輒用。」至是薄錄,并歸天府。



 史臣曰:高祖霸業始基,招集英勇。張瓊等雖識非先覺,而運屬時來,驅馳戎旅,日不暇給,義宣御侮,契協寵圖,臨敵制勝,有足稱也。慕容紹宗兵機武略,在世見推。昔事爾朱,固執忠義,不用範增之言,終見烏江之禍。侯景狼戾,固非後主之臣,末命諸言,實表知人之鑒。寒山、渦水,往若摧枯,算盡數奇,逢斯厄運,悲夫!



 贊曰:霸圖立肇,王業是因。偉哉諸將,實曰功臣。永懷耿、賈,無累清塵。



\end{pinyinscope}