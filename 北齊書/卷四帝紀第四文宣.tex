\article{卷四帝紀第四文宣}

\begin{pinyinscope}

 顯祖文宣皇帝,諱洋,字子進,高祖第二子,世宗之母弟。后初孕,每夜有赤光照室,后私嘗怪之。初,高祖之歸爾朱榮,時經危亂,家徒壁立,后與親姻相對,共憂寒餒。帝
 時尚未能言,欻然應曰「得活」,太后及左右大驚而不敢言。及長,黑色,大頰兌下,鱗身重踝。不好戲弄,深沉有大度。晉陽曾有沙門,乍愚乍智,時人不測,呼為阿禿師。帝曾與諸童共見之,歷問祿位,至帝,舉手再三指天而已,口無所言。見者異之。高祖嘗試觀諸子意識,各使治亂絲,帝獨抽刀斬之,曰:「亂者須斬。」高祖是之。又各配兵四出,而使甲騎偽攻之。世宗等怖撓,帝乃勒眾與彭樂敵,樂免胄言情,猶擒之以獻。後從世宗行過遼陽山,獨見天
 門開,餘人無見者。內雖明敏,貌若不足,世宗每嗤之,云:「此人亦得富貴,相法亦何由可解。」唯高祖異之,謂薛琡曰:「此兒意識過吾。」幼時師事范陽盧景裕,默識過人,景裕不能測也。天平二年,授散騎常侍、驃騎大將軍、儀同三司、左光祿大夫、太原郡開國公。武定元年,加侍中。二年,轉尚書左僕射、領軍將軍。五年,授尚書令、中書監、京畿大都督。



 武定七年八月,世宗遇害,事出倉卒,內外震駭。帝神色不變,指麾部分,自臠斬群賊而漆其頭,徐宣
 言曰:「奴反,大將軍被傷,無大苦也。」當時內外莫不驚異焉。乃赴晉陽,親總庶政,務從寬厚,事有不便者咸蠲省焉。冬十月癸未朔,以咸陽王坦為太傅,潘相樂為司空。十一月戊午,吐谷渾國遣使朝貢。梁齊州刺史茅靈斌、德州刺史劉領隊、南豫州刺史皇甫慎等並以州內屬。十二月己酉,以并州刺史彭樂為司徒,太保賀拔仁為并州刺史。



 八年春正月庚申,梁楚州刺史宋安顧以州內屬。辛酉,魏帝為世宗舉哀於東堂。



 梁定州刺史田聰
 能、洪州刺史張顯等以州內屬。戊辰,魏詔進帝位使持節、丞相、都督中外諸軍事、錄尚書事、大行臺、齊郡王,食邑一萬戶。甲戌,地豆于國遣使朝貢。三月辛酉,又進封齊王,食冀州之渤海長樂安德武邑、瀛州之河間五郡,邑十萬戶。自居晉陽,寢室夜有光如晝。既為王,夢人以筆點己額。旦以告館客王曇哲曰:「吾其退乎?」曇哲再拜賀曰:「王上加點,便成主字,乃當進也。」夏五月辛亥,帝如鄴。甲寅,進相國,總百揆,封冀州之渤海長樂安德武邑、
 瀛州之河間高陽章武、定州之中山常山博陵十郡,邑二十萬戶,加九錫,殊禮,齊王如故。



 魏帝遣兼太尉彭城王韶、司空潘相樂冊命曰:於戲!敬聽朕命:夫惟天為大,列晷宿而垂象;謂地蓋厚,疏川岳以阜物。所以四時代序,萬類駢羅,庶品得性,群形不夭。然則皇王統歷,深視高居,拱默垂衣,寄成師相,此則夏伯、殷尹竭其股肱,周成、漢昭無為而治。頃者天下多難,國命如旒,則我建國之業將墜於地。齊獻武王奮迅風雲,大濟艱危,爰翼朕
 躬,國為再造,經營庶土,以至勤憂。及文襄承構,愈廣前業,康邦夷難,道格穹蒼。王縱德應期,千齡一出,惟幾惟深,乃神乃聖,大崇霸德,實廣相猷。雖冥功妙實,藐絕言象,標聲示跡,典禮宜宣。今申後命,其敬虛受。



 王摶風初舉,建旟上地,庇民立政,時雨滂流,下識廉恥,仁加水陸,移風易俗,自齊變魯,此王之功也。仍攝天臺,總參戎律,策出若神,威行朔土,引弓竄跡,松塞無煙,此又王之功也。逮光統前緒,持衡匡合,華戎混一,風海調夷,日月光
 華,天地清晏,聲接響隨,無思不偃,此又王之功也。逖矣炎方,逋違正朔,懷文曜武,授略申規,淮楚連城,漼然桑落,此又王之功也。關、峴衿帶,跨躡蕭條,腸胃之地,岳立鴟跱,偏師纔指,渙同冰散,此又王之功也。晉熙之所,險薄江雷,迥隔聲教,迷方未改,命將鞠旅,覆其巢穴,威略風騰,傾懾南海,此又王之功也。群蠻跋扈,世絕南疆,搖蕩邊垂,亟為塵梗,懷德畏威,向風請順,傾陬盡落,其至如雲,此又王之功也。胡人別種,延蔓山谷,酋渠萬族,廣
 袤千里,憑險不恭,恣其桀黠,有樂淳風,相攜叩款,粟帛之調,王府充積,此又王之功也。



 茫茫涉海,世敵諸華,風行鳥逝,倏來忽往,既飲醇醪,附同膠漆,毛裘委仞,奇獸銜尾,此又王之功也。秦川尚阻,作我仇讎,爰挹椒蘭,飛書請好,天動其衷,辭卑禮厚,區宇乂寧,遐邇畢至,此又王之功也。江陰告禍,民無適歸,蕭宗子弟,尚相投庇,如鳥還山,猶川赴海,荊江十部,俄而獻割,乘此會也,將混朱方,此又王之功也。天平地成,率土咸茂,禎符顯見,史
 不停筆,既連百木,兼呈九尾,素過秦雀,蒼比周烏,此又王之功也。搜揚管庫,衣冠獲序,禮云樂云,銷沉俱振,輕徭徹賦,矜獄寬刑,大信外彰,深仁遠洽,此又王之功也。王有安日下之大勳,加以表光明之盛德,宣贊洪猷,以左右朕言。昔旦、奭外分,毛、畢入佐,出內之任,王宜總之。



 人謀鬼謀,兩儀協契,錫命之行,義申公道。以王踐律蹈禮,軌物蒼生,圓首安志,率心歸道,是以錫王大路、戎路各一,玄牡二駟。王深重民天,唯本是務,衣食之用,榮辱
 所由,是用錫王袞冕之服,赤舄副焉。王深廣惠和,易調風化,神祗且格,功德可象,是用錫王軒懸之樂,六佾之舞。王風聲振赫,九域咸綏,遠人率俾,奔走委贐,是用錫王朱戶以居。王求賢選眾,草萊以盡,陳力就列,罔非其人,是用錫王納陛以登。王英圖猛概,抑揚千品,毅然之節,肅是非違,是用錫王武賁之士三百人。王興亡所繫,制極幽顯,糾行天討,罪人咸得,是用錫王鈇鉞各一。王鷹揚豹變,實扶下土,狼顧鴟張,罔不彈射,是用錫王彤
 弓一、彤矢百、盧弓十、盧矢千。王孝悌之至,通於神明,率民興行,感達區宇,是用錫王秬鬯一卣,珪瓚副焉。往欽哉。其祗順往冊,保弼皇家,用終爾休德,對揚我太祖之顯命。



 魏帝以天人之望有歸,丙辰,下詔曰:三才剖判,百王代興,治天靜地,和神敬鬼,庇民造物,咸自靈符,非一人之大寶,實有道之神器。昔我宗祖應運,奄一區宇,歷聖重光,暨於九葉。德之不嗣,仍離屯圮,盜名字者遍於九服,擅制命者非止三公,主殺朝危,人神靡繫,天下之
 大,將非魏有。賴齊獻武王奮揚靈武,克剪多難,重懸日月,更綴參辰,廟以掃除,國由再造,鴻勳巨業,無德而稱。逮文襄承構,世業逾廣,邇安遠服,海內晏如,國命已康,生生得性。迄相國齊王,緯文經武,統茲大業,盡睿窮幾,研深測化,思隨冥運,智與神行,恩比春天,威同夏日,坦至心於萬物,被大道於八方,故百僚師師,朝無秕政,網疏澤洽,率土歸心。外盡江淮,風靡屈膝,辟地懷人,百城奔走,關隴慕義而請好,瀚漠仰德而致誠。伊所謂命世
 應期,實撫千載。禎符雜遝,異物同途,謳頌填委,殊方一致,代終之迹斯表,人靈之契已合,天道不遠,我不獨知。朕入纂鴻休,將承世祀,籍援立之厚,延宗社之算,靜言大運,欣於避賢,遠惟唐、虞禪代之典,近想魏、晉揖讓之風,其可昧興替之禮,稽神祇之望?今便遜於別宮,歸帝位於齊國,推聖與能,眇符前軌。主者宣布天下,以時施行。



 又使兼太尉彭城王韶、兼司空敬顯俊奉冊曰:咨爾相國齊王:夫氣分形化,物繫君長,皇王遞興,人非一姓。
 昔放勳馭世,沉璧屬子;重華握歷,持衡擁璇。所以英賢茂實,昭晰千古,豈盛衰有運,興廢在時,知命不得不授,畏天不可不受。是故漢劉告否,當塗順民,曹歷不永,金行納禪,此皆重規襲矩,率由舊章者也。



 我祖宗光宅,混一萬宇。迄於正光之末,奸孽乘權,厥政多僻,九域離盪。永安運窮,人靈殄瘁,群逆滔天,割裂四海,國土臣民,行非魏有。齊獻武王應期授手,鳳舉龍驤,舉廢極以立天,扶傾柱而鎮地,剪滅黎毒,匡我墜歷,有大德於魏室,被
 博利於蒼生。及文襄繼軌,誕光前業,內剿凶權,外摧侵叛,遐邇肅晏,功格上玄。王神祇協德,舟梁一世,體文昭武,追變窮微。自舉跡藩旟,頌歌總集,入統機衡,風猷弘遠。及大承世業,扶國昌家,相德日躋,霸風愈邈,威靈斯暢,則荒遠奔馳,聲略所播,而鄰敵順款。以富有之資,運英特之氣,顧眄之間,無思不服。圖諜潛蘊,千祀彰明,嘉禎幽秘,一朝紛委,以表代德之期,用啟興邦之跡,蒼蒼在上,照臨不遠。朕以虛昧,猶未逡巡,靜言愧之,坐而待
 旦。且時來運往,媯舜不暇以當陽,世革命改,伯禹不容於北面,況於寡薄,而可踟躕。是以仰協穹昊,俯從百姓,敬以帝位式授於王。天祿永終,大命格矣。於戲!其祗承歷數,允執其中,對揚天休,斯年千萬,豈不盛歟!



 又致璽書於帝,遣兼太保彭城王韶、兼司空敬顯俊奉皇帝璽綬,禪代之禮一依唐虞、漢魏故事。又尚書令高隆之率百僚勸進。戊午,乃即皇帝位於南郊,升壇柴燎告天曰:皇帝臣洋敢用玄牡昭告於皇皇后帝:否泰相沿,廢興
 迭用,至道無親,應運斯輔。上覽唐、虞,下稽魏、晉,莫不先天揖讓,考歷終歸。魏氏多難,年將三十,孝昌已後,內外去之。世道橫流,蒼生塗炭。賴我獻武,拯其將溺,三建元首,再立宗祧,掃絕群凶,芟夷奸宄。德被黔黎,勳光宇宙。文襄嗣武,克構鴻基,功浹寰宇,威陵海外,窮髮懷音,西寇納款,青丘保候,丹穴來庭,扶翼危機,重匡頹運,是則有大造於魏室也。



 魏帝以卜世告終,上靈厭德,欽若昊天,允歸大命,以禪於臣洋。夫四海至公,天下為一,總民
 宰世,樹之以君,既川岳啟符,人神效祉,群公卿士,八方兆庶,僉曰皇極乃顧於上,魏朝推進於下,天位不可以暫虛。遂逼群議,恭膺大典。猥以寡薄,託於兆民之上,雖天威在顏,咫尺無遠,循躬自省,實懷祗惕。敬簡元辰,升壇受禪,肆類上帝,以答萬國之心,永隆嘉祉,保祐有齊,以被於無窮之祚。



 是日,京師獲赤雀,獻於南郊。事畢,還宮,御太極前殿。詔曰:「無德而稱,代刑以禮,不言而信,先春後秋。故知惻隱之化,天人一揆,弘宥之道,今古同風。



 朕以虛薄,功業無紀。昔先獻武王值魏世不造,九鼎行出,乃驅御侯伯,大號燕、趙,拯厥顛墜,俾亡則存。文襄王外挺武功,內資明德,纂戎先業,闢土服遠。年踰二紀,世歷兩都,獄訟有適,謳歌斯在。故魏帝俯遵歷數,爰念褰裳,遠取唐、虞,終同脫屣。實幽憂未已,志在陽城,而群公卿士,誠守愈切,遂屬代終,居於民上,如涉深水,有眷終朝。始發晉陽,九尾呈瑞,外壇告天,赤雀效祉。惟爾文武不貳心之臣,股肱爪牙之將,左右先王,克隆大業,永言
 誠節,共斯休祉。思與億兆,同始茲日,其大赦天下。改武定八年為天保元年。其百官進階,男子賜爵,鰥寡六疾,義夫節婦,旌賞各有差。」



 己未,詔封魏帝為中山王,食邑萬戶;上書不稱臣,答不稱詔,載天子旌旗,行魏正朔,乘五時副車;封王諸子為縣公,邑一千戶;奉絹萬匹,錢千萬,粟二萬石,奴婢二百人,水碾一具,田百頃,園一所。詔追尊皇祖文穆王為文穆皇帝,妣為文穆皇后,皇考獻武王為獻武皇帝,皇兄文襄王為文襄皇帝,祖宗之稱,
 付外速議以聞。辛酉,尊王太后為皇太后。乙丑,詔降魏朝封爵各有差。其信都從義及宣力霸朝者,及西來人并武定六年以來南來投化者,不在降限。辛未,遣大使於四方,觀察風俗,問民疾苦,嚴勒長吏,厲以廉平,興利除害,務存安靜。若法有不便於時,政有未盡於事者,具條得失,還以聞奏。甲戌,遷神主於太廟。



 六月己卯,高麗遣使朝貢。辛巳,詔曰:「頃者風俗流宕,浮競日滋,家有吉凶,務求勝異。婚姻喪葬之費,車服飲食之華,動竭歲資,
 以營日富。又奴僕帶金玉,婢妾衣羅綺,始以創出為奇,後以過前為麗,上下貴賤,無復等差。今運屬惟新,思蠲往弊,反朴還淳,納民軌物。可量事具立條式,使儉而獲中。」又詔封崇聖侯邑一百戶,以奉孔子之祀,并下魯郡以時脩治廟宇,務盡褒崇之至。詔分遣使人致祭於五岳四瀆,其堯祠舜廟,下及孔父、老君等載於祀典者,咸秩罔遺。詔曰:「冀州之渤海、長樂二郡,先帝始封之國,義旗初起之地。并州之太原、青州之齊郡,霸業所在,王命
 是基。君子有作,貴不忘本,思申恩洽,蠲復田租。齊郡、渤海可並復一年,長樂復二年,太原復三年。」



 詔故太傅孫騰、故太保尉景、故大司馬婁昭、故司徒高昂、故尚書左僕射慕容紹宗、故領軍萬俟干、故定州刺史段榮、故御史中尉劉貴、故御史中尉竇泰、故殷州刺史劉豐、故濟州刺史蔡俊等並左右先帝,經贊皇基,或不幸早徂,或殞身王事,可遣使者就墓致祭,并撫問妻子,慰逮存亡。又詔封宗室高岳為清河王,高隆之為平原王,高歸彥
 為平秦王,高思宗為上洛王,高長弼為廣武王,高普為武興王,高子瑗為平昌王,高顯國為襄樂王,高睿為趙郡王,高孝緒為脩城王。又詔封功臣厙狄乾為章武王,斛律金為咸陽王,賀拔仁為安定王,韓軌為安德王,可朱渾道元為扶風王,彭樂為陳留王,潘相樂為河東王。癸未,詔封諸弟青州刺史浚為永安王,尚書左僕射淹為平陽王,定州刺史浟為彭城王,儀同三司演為常山王,冀州刺史渙為上黨王,儀同三司水肓為襄城王,儀同
 三司湛為長廣王,湝為任城王,湜為高陽王,濟為博陵王,凝為新平王,潤為馮翊王,洽為漢陽王。



 丁亥,詔立王子殷為皇太子,王后李氏為皇后。庚寅,詔以太師厙狄乾為太宰,司徒彭樂為太尉,司空潘相樂為司徒,開府儀同三司司馬子如為司空。辛卯,以前太尉、清河王岳為使持節、驃騎大將軍、司州牧。壬辰,詔曰:「自今已後,諸有文啟論事并陳要密,有司悉為奏聞。」己亥,以皇太子初入東宮,赦畿內及并州死罪已下,餘州死降,徒流已
 下一皆原免。



 秋七月辛亥,詔尊文襄妃元氏為文襄皇后,宮曰靜德。又詔封文襄皇帝子孝琬為河間王,孝瑜為河南王。乙卯,以尚書令、平原王隆之錄尚書事,尚書左僕射、平陽王淹為尚書令。又詔曰:「古人鹿皮為衣,書囊成帳,有懷盛德,風流可想。



 其魏御府所有珍奇雜彩常所不給人者,徒為蓄積,命宜悉出,送內後園,以供七日宴賜。」



 八月,詔郡國脩立黌序,廣延髦俊,敦述儒風。其國子學生亦仰依舊銓補,服膺師說,研習《禮經》。往者文
 襄皇帝所運蔡邕石經五十二枚,即宜移置學館,依次脩立。又詔曰:「有能直言正諫,不避罪辜,謇謇若朱雲,諤諤若周舍,開朕意,沃朕心,弼于一人,利兼百姓者,必當寵以榮祿,待以不次。」又曰:「諸牧民之官,仰專意農桑,勤心勸課,廣收天地之利,以備水旱之災。」庚寅,詔曰:「朕以虛寡,嗣弘王業,思所以贊揚盛績,播之萬古。雖史官執筆,有聞無墜,猶恐緒言遺美,時或未書。在位王公文武大小,降及民庶,爰至僧徒,或親奉音旨,或承傳傍說,凡
 可載之文籍,悉宜條錄封上。」甲午,詔曰:「魏世議定《麟趾格》,遂為通制,官司施用,猶未盡善。可令群官更加論究。適治之方,先盡要切。引綱理目,必使無遺。」



 九月癸丑,以散騎常侍、車騎將軍、領東夷校尉、遼東郡開國公、高麗王成為使持節、侍中、驃騎大將軍、領護東夷校尉,王、公如故。詔梁侍中、使持節、假黃鉞、都督中外諸軍事、大將軍、承制、邵陵王蕭綸為梁王。庚午,帝如晉陽,拜辭山陵。是日皇太子入居涼風堂,監總國事。



 冬十月己卯,備法
 駕,御金輅,入晉陽宮,朝皇太后於內殿。辛巳,曲赦并州太原郡晉陽縣及相國府四獄囚。癸未,茹茹國遣使朝貢。乙酉,以特進元韶為尚書左僕射,并州刺史段韶為尚書右僕射。丙戌,吐谷渾國遣使朝貢。壬辰,罷相國府,留騎兵、外兵曹,各立一省,別掌機密。十一月,周文帝率眾至陜城,分騎北渡,至建州。甲寅,梁湘東王蕭繹遣使朝貢。丙寅,帝親戎出次城東。周文帝聞帝軍容嚴盛,歎曰:「高歡不死矣。」遂退師。庚午,還宮。十二月丁丑,茹茹、庫
 莫奚國並遣使朝貢。辛丑,帝至自晉陽。



 二年春正月丁未,梁湘東王蕭繹遣使朝貢。辛亥,有事於圓丘,以神武皇帝配。



 癸亥,親耕籍田於東郊。乙酉,前黃門侍郎元世寶、通直散騎侍郎彭貴平謀逆,免死配邊。有事於太廟。甲戌,帝汎舟於城東。二月壬辰,太尉彭樂謀反,伏誅。壬寅,茹茹國遣使朝貢。三月丙午,襄城王水肓薨。己未,詔梁承制湘東王繹為梁使持節、假黃鉞、相國,建梁臺,總百揆,承制。梁交州刺史李景盛、梁州刺史
 馬嵩仁、義州刺史夏侯珍洽、新州刺史李漢等並率州內附。庚申,司空司馬子如坐事免。



 夏四月壬辰,梁王蕭繹遣使朝貢。閏月乙丑,室韋國遣使朝貢。五月丙戌,合州刺史斛斯顯攻克梁歷陽鎮。丁亥,高麗國遣使朝貢。是月,侯景廢梁簡文,立蕭棟為主。六月庚午,以前司空司馬子如為太尉。七月壬申,茹茹遣使朝貢。癸酉,行臺郎邢景遠破梁龍安戍,獲鎮城李洛文。己卯,改顯陽殿為昭陽殿。九月壬申,詔免諸伎作、屯、牧、雜色役隸之徒
 為白戶。癸巳,帝如趙、定二州,因如晉陽。冬十月戊申,起宣光、建始、嘉福、仁壽諸殿。庚申,蕭繹遣使朝貢。丁卯,文襄皇帝神主入於廟。十一月,侯景廢梁主,僭即偽位於建鄴,自稱曰漢。十二月,中山王殂。



 三年春正月丙申,帝親討庫莫奚於代郡,大破之,獲雜畜十餘萬,分賚將士各有差。以奚口付山東為民。二月,茹茹主阿那瑰為突厥虜所破,瑰自殺,其太子庵羅辰及瑰從弟登注俟利發、注子庫提並擁眾來奔。茹茹餘
 眾立注次子鐵伐為主。辛丑,契丹遣使朝貢。三月戊子,以司州牧清河王岳為使持節、南道大都督,司徒潘相樂為使持節、東南道大都督,及行臺辛術率眾南伐。癸巳,詔進梁王蕭繹為梁主。



 夏四月壬申,東南道行臺辛術於廣陵送傳國璽。甲申,以吏部尚書楊愔為尚書右僕射。丙申,室韋國遣使朝貢。六月乙亥,清河王岳等班師。丁未,帝至自晉陽。乙卯,帝如晉陽。九月辛卯,帝自并州幸離石。冬十月乙未,至黃櫨嶺,仍起長城,北至社乾
 戍四百餘里,立三十六戍。十一月辛巳,梁王蕭繹即帝位於江陵,是為元帝,遣使朝貢。十二月壬子,帝還宮。戊午,帝如晉陽。



 四年春正月丙子,山胡圍離石。戊寅,帝討之,未至,胡已逃竄,因巡三堆戍,大狩而歸。戊寅,庫莫奚遣使朝貢。己丑,改鑄新錢,文曰「常平五銖」。二月,送茹茹主鐵伐父登注及子庫提還北。鐵伐尋為契丹所殺,國人復立登注為主,仍為其大人阿富提等所殺,國人復立庫提為主。夏
 四月戊戌,帝還宮。戊午,西南有大聲如雷。五月庚午,帝校獵於林慮山。戊子,還宮。九月,契丹犯塞。壬午,帝北巡冀、定、幽、安,仍北討契丹。冬十月丁酉,帝至平州,遂從西道趣長塹。詔司徒潘相樂率精騎五千自東道趣青山。辛丑,至白狼城。壬寅,經昌黎城。復詔安德王韓軌率精騎四千東趣,斷契丹走路。癸卯,至陽師水,倍道兼行,掩襲契丹。甲辰,帝親踰山嶺,為士卒先,指麾奮擊,大破之,虜獲十萬餘口、雜畜數十萬頭。



 樂又於青山大破契丹
 別部。所虜生口皆分置諸州。是行也,帝露頭袒膊,晝夜不息,行千餘里,唯食肉飲水,壯氣彌厲。丁未,至營州。丁巳,登碣石山,臨滄海。十一月己未,帝自平州,遂如晉陽。閏月壬寅,梁帝遣使來聘。十二月己未,突厥復攻茹茹,茹茹舉國南奔。癸亥,帝自晉陽北討突厥,迎納茹茹。乃廢其主庫提,立阿那瑰子庵羅辰為主,置之馬邑川,給其稟餼繒帛。親追突厥於朔州,突厥請降,許之而還。於是貢獻相繼。



 五年春正月癸巳,帝討山胡,從離石道。遣太師、咸陽王斛律金從顯州道,常山王演從晉州道,掎角夾攻,大破之,斬首數萬,獲雜畜十餘萬,遂平石樓。石樓絕險,自魏世所不能至。於是遠近山胡莫不懾服。是月周文帝廢西魏主,立齊王廓,是為恭帝。三月,茹茹庵羅辰叛,帝親討,大破之,辰父子北遁。太保賀拔仁坐違節度除名。夏四月,茹茹寇肆州。丁巳,帝自晉陽討之,至恒州黃瓜堆,虜騎走。



 時大軍已還,帝率麾下千餘騎,遇茹茹別部
 數萬,四面圍逼。帝神色自若,指畫形勢,虜眾披靡,遂縱兵潰圍而出。虜乃退走,追擊之,伏尸二十里,獲庵羅辰妻子及生口三萬餘人。五月丁亥,地豆干、契丹等國並遣使朝貢。丁未,北討茹茹,大破之。六月,茹茹率部眾東徙,將南侵。帝率輕騎於金山下邀擊之,茹茹聞而遠遁。



 秋七月戊子,肅慎遣使朝貢。壬辰,降罪人。庚戌,帝至自北伐。八月丁巳,突厥遣使朝貢。庚子,以司州牧、清河王岳為太保,司空尉粲為司徒,太子太師侯莫陳相為司
 空,尚書令、平陽王淹錄尚書事,常山王演為尚書令,中書令、上黨王渙為尚書左僕射。乙亥,儀同三司元旭以罪賜死。丁丑,帝幸晉陽。己卯,開府儀同三司、錄尚書事、平原王高隆之薨。是月,詔常山王演、上黨王渙、清河王岳、平原王段韶等率眾於洛陽西南築伐惡城、新城、嚴城、河南城。九月,帝親自臨幸,欲以致周師。周師不出,乃如晉陽。冬十月,西魏伐梁元帝於江陵。詔清河王岳、河東王潘相樂、平原王段韶等率眾救之,未至而江陵陷,
 梁元帝為西魏將于謹所殺。



 梁將王僧辯在建康,共推晉安王蕭方智為太宰、都督中外諸軍,承制置百官。十二月庚申,帝北巡至達速嶺,覽山川險要,將起長城。



 六年春正月壬寅,清河王岳以眾軍渡江,克夏首。送梁郢州刺史陸法和。詔以梁散騎常侍、貞陽侯蕭明為梁主,遣尚書左僕射、上黨王渙率眾送之。二月甲子,以陸法和為使持節、都督荊雍江巴梁益湘萬交廣十州諸軍事、太尉公、大都督、西南道大行臺,梁鎮北將軍、侍中、
 荊州刺史宋蒨為使持節、驃騎大將軍、郢州刺史。



 甲戌,上黨王渙克譙郡。三月丙戌,上黨王渙克東關,斬梁將裴之橫,俘斬數千。



 丙申,帝至自晉陽。封世宗二子孝珩為廣寧王,延宗為安德王。戊戌,帝臨昭陽殿聽獄決訟。夏四月庚申,帝如晉陽。丁卯,儀同蕭軌克梁晉熙城,以為江州。戊寅,突厥遣使朝貢。梁反人李山花自號天子,逼魯山城。五月乙酉,鎮城李仲侃擊斬之。



 庚寅,帝至自晉陽。蕭明入于建鄴。丁未,茹茹遣使朝貢。六月壬子,詔
 曰:「梁國遘禍,主喪臣離,逷彼炎方,盡生荊棘。興亡繼絕,義在於我,納以長君,拯其危弊,比送梁主,已入金陵。藩禮既修,分義方篤。越鳥之思,豈忘南枝,凡是梁民,宜聽反國,以禮發遣。」丁卯,帝如晉陽。壬申,親討茹茹。甲戌,諸軍大會於祁連池。乙亥,出塞,至厙狄谷,百餘里內無水泉,六軍渴乏,俄而大雨。戊寅,梁主蕭明遣其子章、兼侍中袁泌、兼散騎常侍楊裕奉表朝貢。秋七月己卯,帝頓白道,留輜重,親率輕騎五千追茹茹。壬午,及於懷朔鎮。
 帝躬當矢石,頻大破之,遂至沃野,獲其俟利藹焉力婁阿帝、吐頭發郁久閭狀延等,并口二萬餘,牛羊數十萬頭。茹茹俟利郁久閭李家提率部人數百降。壬辰,帝還晉陽。九月乙卯,帝至自晉陽。冬十月,梁將陳霸先襲王僧辯,殺之,廢蕭明,復立蕭方智為主。辛亥,帝如晉陽。十一月丙戌,高麗遣使朝貢。梁秦州刺史徐嗣輝、南豫州刺史任約等襲據石頭城,並以州內附。壬辰,大都督蕭軌率眾至江,遣都督柳達摩等渡江鎮石頭。



 東南道行
 臺趙彥深獲秦郡等五城,戶二萬餘,所在安輯之。己亥,太保、司州牧、清河王岳薨。是月,柳達摩為霸先攻逼,以石頭降。十二月戊申,庫莫奚遣使朝貢。



 是年,發夫一百八十萬人築長城,自幽州北夏口至恒州九百餘里。



 七年春正月甲辰,帝至自晉陽。於鄴城西馬射,大集眾庶而觀之。二月辛未,詔常山王演等於涼風堂讀尚書奏按,論定得失,帝親決之。三月丁酉,大都督蕭軌等率眾濟江。夏四月乙丑,儀同婁睿率眾討魯陽蠻,大破之。
 丁卯,詔造金華殿。



 五月丙申,漢陽王洽薨。是月,帝以肉為斷慈,遂不得食。六月乙卯,蕭軌等與梁師戰於鐘山之西,遇霖雨,失利,軌及都督李希光、王敬寶、東方老、軍司裴英起並沒,土卒散還者十二三。乙丑,梁湘州刺史王琳獻馴象。是年,脩廣三臺宮殿。



 秋七月己亥,大赦天下。八月庚申,帝如晉陽。九月甲辰,庫莫奚遣使朝貢。冬十月丙戌,契丹遣使朝貢。是月,發山東寡婦二千六百人以配軍士,有夫而濫奪者五分之一。是月,周文帝殂。
 十一月壬子,詔曰:昆山作鎮,厥號神州;瀛海為池,是稱赤縣。蒸民乃粒,司牧存焉。王者之制,沿革迭起,方割成災,肇分十二,水土既平,還復九州。道或繁簡,義在通時,殷因於夏,元所改作。然則日月纏於天次,王公國於地野,皆所以上葉玄儀,下符川嶽。逮于秦政,鞭撻區宇,罷侯置守,天下為家。洎兩漢承基,曹、馬屬統,其間損益,難以勝言。魏自孝昌之季,數鐘澆否,祿去公室,政出多門,衣冠道盡,黔首塗炭。銅馬、鐵脛之徒。黑山、青犢之侶,梟
 張晉、趙,豕突燕、秦,綱紀從茲而頹,彞章因此而紊。是使豪家大族,鳩率鄉部,託跡勤王,規自署置。或外家公主,女謁內成,昧利納財,啟立州郡。離大合小,本逐時宜,部竹分符,蓋不獲已,牧守令長,虛增其數,求功錄實,諒足為煩,損害公私,為弊殊久,既乖為政之禮,徒有驅羊之費。自爾因循,未遑刪改。朕寅膺寶歷,恭臨八荒,建國經野,務存簡易。將欲鎮躁歸靜,反薄還淳,茍失其中,理從刊正。傍觀舊史,逖聽前言,周曰成、康,漢稱文、景,編戶之
 多,古今為最。而丁口滅於疇日,守令倍於昔辰,非所以馭俗調風,示民軌物。且五嶺內賓,三江乃化,拓土開疆,利窮南海。但要荒之所,舊多浮偽,百室之邑,便立州名,三戶之民,空張郡目。譬諸木犬,猶彼泥龍,循名督實,事歸烏有。今所併省,一依別制。



 於是併省三州、一百五十三郡、五百八十九縣、二鎮二十六戍。又制刺史令盡行兼,不給乾物。十二月,西魏相宇文覺受魏禪。先是,自西河總秦戍築長城東至於海,前後所築東西凡三千餘
 里,率十里一戍,其要害置州鎮,凡二十五所。



 八年春三月,大熱,人或暍死。夏四月庚午,詔諸取蝦蟹蜆蛤之類,悉令停斷,唯聽捕魚。乙酉,詔公私鷹鷂俱亦禁絕。以太師、咸陽王斛律金為右丞相,前大將軍、扶風王可朱渾道元為太傅,開府儀同三司賀拔仁為太保,尚書令、常山王演為司空、錄尚書事,長廣王湛為尚書令,尚書右僕射楊愔為尚書左僕射,以并省尚書右僕射崔暹為尚書右僕射,上黨王渙錄尚書事。是月,帝在
 城東馬射,敕京師婦女悉赴觀,不赴者罪以軍法,七日乃止。五月辛酉,冀州民劉向於京師謀逆,黨與皆伏誅。、秋八月己巳,庫莫奚遣使朝貢。庚辰,詔丘、郊、禘、祫、時祀,皆仰市取,少牢不得剖割,有司監視,必令豐備;農社先蠶,酒肉而已;雩、禖、風、雨、司民、司祿、靈星、雜祀,果餅酒脯。唯當務盡誠敬,義同如在。自夏至九月,河北六州、河南十二州、畿內八郡大蝗。是月,飛至京師,蔽日,聲如風雨。甲辰,詔今年遭蝗之處免租。是月,周冢宰宇文護殺其
 主閔帝而立帝弟毓,是為明帝。冬十月乙亥,陳霸先弒其主方智自立,是為陳武帝,遣使稱藩朝貢。是年,於長城內築重城,自庫洛拔而東至於塢紇戍,凡四百餘里。



 九年春二月丁亥,降罪人。己丑,詔限仲冬一月燎野,不得他時行火,損昆蟲草木。三月丁酉,帝至自晉陽。夏四月辛巳,大赦。是夏,大旱。帝以祈雨不應,毀西門豹祠,掘其塚。山東大蝗,差夫役捕而坑之。是月,北豫州刺史司馬消難以城叛,入於周。五月辛丑,尚書令、長廣王湛錄
 尚書事,驃騎大將軍、平秦王歸彥為尚書左僕射。甲辰,以前尚書左僕射楊愔為尚書令。六月乙丑,帝自晉陽北巡。



 己巳,至祁連池。戊寅,還晉陽。秋七月辛丑,給京畿老人劉奴等九百四十三人版職及杖帽各有差。戊申,詔趙、燕、瀛、定、南營五州及司州廣平、清河二郡去年螽澇損田,兼春夏少雨,苗稼薄者,免今年租賦。八月乙丑,至自晉陽。甲戌,帝如晉陽。是月,陳江州刺史沈泰以三千人內附。先是,發丁匠三十餘萬營三臺於鄴下,因其
 舊基而高博之,大起宮室及游豫園。至是,三臺成,改銅爵曰金鳳,金獸曰聖應,冰井曰崇光。十一月甲午,帝至自晉陽,登三臺,御乾象殿,朝宴群臣,並命賦詩。以新宮成,丁酉,大赦,內外文武並進一大階。丁巳,梁湘州刺史王琳遣使請立蕭莊為梁主,仍以江州內屬,令莊居之。十二月癸酉,詔梁王蕭莊為梁主,進居九派。戊寅,以太傅可朱渾道元為太師,司徒尉粲為太尉,冀州刺史段韶為司空,錄尚書事、常山王演為大司馬,錄尚書事、長
 廣王湛為司徒。是月,起大莊嚴寺。是年,殺永安王浚、上黨王渙。



 十年春正月戊戌,以司空侯莫陳相為大將軍。甲寅,帝如遼陽甘露寺。乙卯,詔於麻城置衛州。二月丙戌,帝於甘露寺禪居深觀,唯軍國大政奏聞。三月戊戌,以侍中高德政為尚書右僕射。丙辰,帝至自遼陽。是月,梁主蕭莊至郢州,遣使朝貢。閏四月丁酉,以司州牧、彭城王浟為司空,侍中、高陽王湜為尚書右僕射。乙巳,以司空、彭
 城王浟兼太尉,封皇子紹廉為長樂郡王。五月癸未,誅始平公元世、東平公元景式等二十五家,特進元韶等十九家並令禁止。六月,陳武帝殂,兄子蒨立,是為文帝。秋八月戊戌,封皇子紹義為廣陽郡王,以尚書右僕射、河間王孝琬為尚書左僕射。癸卯,詔諸軍民或有父祖改姓冒入元氏,或假託攜認,忘稱姓元者,不問世數遠近,悉聽改復本姓。九月己巳,帝如晉陽。是月,使酈懷則、陸仁惠使於蕭莊。冬十月甲午,帝暴崩於晉陽宮德
 陽堂,時年三十一。遺詔:「凡諸凶事一依儉約。三年之喪,雖曰達禮,漢文革創,通行自昔,義有存焉,同之可也,喪月之斷限以三十六日。嗣主、百僚、內外遐邇奉制割情,悉從公除。」癸卯,發喪,斂於宣德殿。十一月辛未,梓宮還京師。十二月乙酉,殯於太極前殿。乾明元年二月丙申,葬於武寧陵,謚曰文宣皇帝,廟號威宗。武平初,又改為文宣,廟號顯祖。



 帝少有大度,志識沉敏,外柔內剛,果敢能斷。雅好吏事,測始知終,理劇處繁,終日不倦。初踐大
 位,留心政術,以法馭下,公道為先。或有違犯憲章,雖密戚舊勳,必無容舍,內外清靖,莫不祗肅。至於軍國幾策,獨決懷抱,規模宏遠,有人君大略。又以三方鼎跱,諸夷未賓,修繕甲兵,簡練士卒,左右宿衛置百保軍士。每臨行陣,親當矢石,鋒刃交接,唯恐前敵之不多,屢犯艱危,常致克捷。嘗於東山游宴,以關隴未平,投杯震怒,召魏收於御前,立為詔書,宣示遠近,將事西伐。是歲,周文帝殂,西人震恐,常為度隴之計。既征伐四克,威振戎夏,六
 七年後,以功業自矜,遂留連耽湎,肆行淫暴。或躬自鼓舞,歌謳不息,從旦通宵,以夜繼晝。或袒露形體,塗傅粉黛,散髮胡服,雜衣錦彩。拔刃張弓,遊於市肆,勳戚之第,朝夕臨幸。時乘馲駝牛驢,不施鞍勒,盛暑炎赫,隆冬酷寒,或日中暴身,去衣馳騁,從者不堪,帝居之自若。親戚貴臣,左右近習,侍從錯雜,無復差等。徵集淫嫗,分付從官,朝夕臨視,以為娛樂。凡諸殺害,多令支解,或焚之於火,或投之於河。沉酗既久,彌以狂惑,至於末年,每言見
 諸鬼物,亦云聞異音聲。



 情有蒂芥,必在誅戮,諸元宗室咸加屠剿,永安、上黨並致冤酷,高隆之、高德政、杜弼、王元景、李蒨之等皆以非罪加害。嘗在晉陽以槊戲刺都督尉子耀,應手即殞。



 又在三臺大光殿上,以鋸鋸都督穆嵩,遂至於死。又嘗幸開府暴顯家,有都督韓悊無罪,忽於眾中喚出斬之。自餘酷濫,不可勝紀。朝野慘憎,各懷怨毒。而素以嚴斷臨下,加之默識彊記,百僚戰慄,不敢為非,文武近臣,朝不謀夕。又多所營繕,百役繁興,舉
 國騷擾,公私勞弊。凡諸賞賚,無復節限,府藏之積,遂至空虛。自皇太后諸王及內外勳舊,愁懼危悚,計無所出。暨于末年,不能進食,唯數飲酒,曲蘗成災,因而致斃。



 論曰:高祖平定四胡,威權延世。遷鄴之後,雖主器有人,號令所加,政皆自出。顯祖因循鴻業,內外協從,自朝及野,群心屬望。東魏之地,舉世樂推,曾未期月,玄運集己。始則存心政事,風化肅然,數年之間,翕斯致治。其後縱酒肆欲,事極猖狂,昏邪殘暴,近世未有。饗國弗永,實由
 斯疾,胤嗣殄絕,固亦餘殃者也。



 贊曰:天保定位,受終攸屬。奄宅區夏,爰膺帝籙。勢葉謳歌,情毀龜玉。始存政術,聞斯德音。罔遵克念,乃肆其心。窮理殘虐,盡性荒淫。



\end{pinyinscope}