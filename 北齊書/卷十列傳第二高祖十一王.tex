\article{卷十列傳第二高祖十一王}

\begin{pinyinscope}

 永安簡平王浚平陽靖翼王淹彭城景思王浟
 上黨剛肅王渙襄城景王水肓任城王湝高陽康穆王湜博陵文簡王濟華山王凝馮翊王潤漢陽敬懷王洽
 神武皇帝十五男:武明婁皇后生文襄皇帝、文宣皇帝、孝昭皇帝、襄城景王淯、武成皇帝、博陵文簡王濟,王氏生永安簡平王浚,穆氏生平陽靖翼王淹,大爾朱氏生彭城景思王浟、華山王凝,韓氏生上黨剛王渙,小爾朱氏生任城王湝,游氏生高陽康穆王湜,鄭氏生馮翊王潤,馮氏生漢陽敬懷王洽。



 永安簡平王浚,字定樂,神武第三子也。初,神武納浚母,當月而有孕,及產浚,疑非己類,不甚愛之。而浚早慧,後
 更被寵。年八歲時,問於博士盧景裕曰:「『祭神如神在。』為有神邪,無神邪?」對曰;「有。」浚曰:「有神當云祭神神在,何煩『如』字?」景裕不能答。及長,嬉戲不節,曾以屬請受納,大見杖罰,拘禁府獄,既而見原。後稍折節,頗以讀書為務。元象中,封永安郡公。豪爽有氣力,善騎射,為文襄所愛。文宣性雌懦,每參文襄,有時涕出。浚常責帝左右,何因不為二兄拭鼻,由是見銜。累遷中書監、兼侍中。出為青州刺史,頗好畋獵,聰明矜恕,上下畏悅之。天保初,進爵為
 王。文宣末年多酒,浚謂親近曰:「二兄舊來不甚了了,自登祚已後,識解頓進。今因酒敗德,朝臣無敢諫者,大敵未滅,吾甚以為憂,欲乘驛至鄴面諫,不知用吾不。」人有知,密以白帝,又見銜。八年來朝,從幸東山。帝裸裎為樂,雜以婦女,又作狐掉尾戲。浚進言此非人主所宜。帝甚不悅。浚又於屏處召楊遵彥,譏其不諫。帝時不欲大臣與諸王交通,遵彥懼以奏。



 帝大怒曰:「小人由來難忍!」遂罷酒還宮。浚尋還州,又上書切諫。詔令征浚,浚懼禍,謝
 疾不至。上怒,馳驛收浚,老幼泣送者數千人。至,盛以鐵籠,與上黨王渙俱置北城地牢下,飲食溲穢共在一所。明年,帝親將左右臨穴歌謳,令浚和之。



 浚等惶怖且悲,不覺聲戰。帝為愴然,因泣,將赦之。長廣王湛先與浚不睦,進曰:「猛獸安可出穴。」帝默然。浚等聞之,呼長廣小字曰:「步落稽,皇天見汝!」



 左右聞者,莫不悲傷。浚與渙皆有雄略,為諸王所傾服,帝恐為害,乃自刺渙,又使壯土劉桃枝就籠亂刺。槊每下,浚、渙輒以手拉折之,號哭呼天。
 於是薪火亂投,燒殺之,填以石土。後出,皮髮皆盡,屍色如炭,天下為之痛心。



 後帝以其妃陸氏配儀同劉郁捷,舊帝蒼頭也,以軍功見用,時令郁捷害浚,故以配焉。後數日,帝以陸氏先無寵於浚,敕與離絕。乾明元年,贈太尉。無子,詔以彭城王浟第二子準嗣。



 平陽靖翼王淹,字子邃,神武第四子也。元象中,封平陽郡公,累遷尚書左僕射。天保初,進爵為王,歷位尚書令、開府儀同三司、司空、太尉。皇建初,為太傅,與彭城、河間
 王並給仗衛、羽林百人。大寧元年,遷太宰。性沉謹,以寬厚稱。



 河清三年,薨於晉陽,或云鴆終。還葬鄴,贈假黃鉞、太宰、錄尚書事。子德素嗣。



 彭城景思王浟,字子深,神武第五子也。元象二年,拜通直散騎常侍,封長樂郡公。博士韓毅教浟書,見浟筆迹未工,戲浟曰:「五郎書畫如此,忽為常侍開國,今日後宜更用心。」浟正色答曰:「昔甘羅幼為秦相,未聞能書。凡人唯論才具何如,豈必動誇筆迹。博士當今能者,何為不
 作三公?」時年蓋八歲矣。毅甚慚。



 武定六年,出為滄州刺史,為政嚴察,部內肅然。守令參佐,下及胥吏,行遊往來,皆自齎糧食。浟纖介知人間事。有隰沃縣主簿張達嘗詣州,夜投人舍,食雞羹,浟察知之。守令畢集,浟對眾曰:「食雞羹何不還價直也?」達即伏罪。合境號為神明。又有一人從幽州來,驢馱鹿脯。至滄州界,腳痛行遲,偶會一人為伴,遂盜驢及脯去。明旦,告州。浟乃令左右及府僚吏分市鹿脯,不限其價。其主見脯識之,推獲盜者。轉都
 督、定州刺史。時有人被盜黑牛,背上有白毛。長史韋道建謂中從事魏道勝曰:「使君在滄州日,擒姦如神,若捉得此賊,定神矣。」浟乃詐為上府市牛皮,倍酬價直,使牛主認之,因獲其盜。建等歎服。又有老母姓王,孤獨,種菜三畝,數被偷。浟乃令人密往書菜葉為字,明日市中看菜葉有字,獲賊。



 爾後境內無盜,政化為當時第一。天保初,封彭城王。四年,徵為侍中,人吏送別悲號。有老公數百人相率具饌曰:「自殿下至來五載,人不識吏,吏不欺
 人,百姓有識已來,始逢今化。殿下唯飲此鄉水,未食此鄉食,聊獻疏薄。」浟重其意,為食一口。七年,轉司州牧,選從事皆取文才士明剖斷者,當時稱為美選。州舊案五百餘,浟未期悉斷盡。別駕羊脩等恐犯權戚,乃詣閣諮陳。浟使告曰:「吾直道而行,何憚權戚,卿等當成人之美,反以權戚為言。」脩等慚悚而退。後加特進,兼司空、太尉,州牧如故。太妃薨,解任,尋詔復本官。俄拜司空,兼尚書令。濟南嗣位,除開府儀同三司、尚書令、領大宗正卿。皇
 建初,拜大司馬,兼尚書令,轉太保。武成入承大業,遷太師、錄尚書事。浟明練世務,果於斷決,事無大小,咸悉以情。趙郡李公統預高歸彥之逆,其母崔氏即御史中丞崔昂從父子,兼右僕射魏收之內妹也。依令,年出六十,例免入官。崔增年陳訴,所司以昂、收故,崔遂獲免。浟摘發其事,昂等以罪除名。



 自車駕巡幸,浟常留鄴。河清三年三月,群盜田子禮等數十人謀劫浟為主,詐稱使者,徑向浟第,至內室,稱敕牽浟上馬,臨以白刃,欲引向南
 殿。浟大呼不從,遂遇害,時年三十二,朝野痛惜焉。初浟未被劫前,其妃鄭氏夢人斬浟頭持去,惡之,數日而浟見殺。贈假黃鉞、太師、太尉、錄尚書事,給轀輬車。子寶德嗣,位開府,兼尚書左僕射。



 上黨剛肅王渙,字敬壽,神武第七子也。天姿雄傑,俶儻不群,雖在童幼,恒以將略自許。神武壯而愛之,曰:「此兒似我。」及長,力能扛鼎,材武絕倫。每謂左右曰:「人不可無學,但要不為博士耳。」故讀書頗知梗概,而不甚耽習。元
 象中,封平原郡公。文襄之遇賊,渙年尚幼,在西學,聞宮中嘩,驚曰:「大兄必遭難矣!」彎弓而出。武定末,除冀州刺史,在州有美政。天保初,封上黨王,歷中書令、尚書左僕射。與常山王演等築伐惡諸城。遂聚鄴下輕薄,凌犯郡縣,為法司所糾。文宣戮其左右數人,渙亦被譴。六年,率眾送梁王蕭明還江南,仍破東關,斬梁特進裴之橫等,威名甚盛。八年,錄尚書事。



 初,術士言亡高者黑衣,由是自神武後,每出行,不欲見沙門,為黑衣故也。



 是時文宣
 幸晉陽,以所忌問左右曰:「何物最黑?」對曰:「莫過漆。」帝以渙第七子為當之,乃使庫真都督破六韓伯昇之鄴征渙。渙至紫陌橋,殺伯升以逃,憑河而度,土人執以送帝。鐵籠盛之,與永安王浚同置地牢下。歲餘,與浚同見殺,時年二十六。以其妃李氏配馮文洛,是帝家舊奴,積勞位至刺史,帝令文洛等殺渙,故以其妻妻焉。



 至乾明元年,收二王餘骨葬之,贈司空,謚曰剛肅。有敕李氏還第。而文洛尚以故意,修飾詣李,李盛列左右,引文洛立於階
 下,數之曰:「遭難流離,以至大辱,志操寡薄,不能自盡,幸蒙恩詔,得反藩闈。汝是誰家孰奴,猶欲見侮!」於是杖之一百,流血灑地。渙無嫡子,庶長子寶嚴以河清二年襲爵,位金紫光祿大夫、開府儀同三司。



 襄城景王淯,神武第八子也。容貌甚美,弱年有器望。元象中,封章武郡公。



 天保初,封襄城郡王。二年春,薨。齊氏諸王選國臣府佐,多取富商群小、鷹犬少年,唯襄城、廣寧、蘭陵王等頗引文藝清識之士,當時以此稱之。乾明
 元年二月,贈假黃鋮、太師、太尉、錄尚書事。無子,詔以常山王演第二子亮嗣。



 亮字彥道,性恭孝,美風儀,好文學。為徐州刺史,坐奪商人財物免官。後主敗奔鄴,亮從焉,遷兼太尉、太傅。周師入鄴,亮於啟夏門拒守。諸軍皆不戰而敗,周軍於諸城門皆入,亮軍方退走。亮入太廟行馬內,慟哭拜辭,然後為周軍所執。



 入關,依例授儀同,分配遠邊,卒於龍州。



 任城王湝,神武第十子也,少明慧。天保初封。自孝昭、武
 成時,車駕還鄴,常令湝鎮晉陽,總并省事,歷司徒、太尉、并省錄尚書事。天統三年,拜太保、并州刺史,別封正平郡公。時有婦人臨汾水浣衣,有乘馬人換其新靴馳而去者,婦人持故靴,詣州言之。湝召城外諸嫗,以靴示之,紿曰:「有乘馬人在路被賊劫害,遺此靴焉,得無親屬乎?」一嫗撫膺哭曰:「兒昨著此靴向妻家。」如其語,捕獲之。時稱明察。武平初,遷太師、司州牧,出為冀州刺史,加太宰,遷右丞相、都督、青州刺史。湝頻牧大藩,雖不潔己,然寬
 恕為吏人所懷。五年,青州崔蔚波等夜襲州城,湝部分倉卒之際,咸得齊整,擊賊,大破之。拜左丞相,轉瀛州刺史。



 及後主奔鄴,加湝大丞相。



 及安德王稱尊號於晉陽,使劉子昂修啟於湝:「至尊出奔,宗廟既重,群公勸迫,權主號令,事寧終歸叔父。」湝曰:「我人臣,何容受此啟。」執子昂送鄴。



 帝至濟州,禪位於湝,啟竟不達。湝與廣寧王孝珩於冀州召募得四萬餘人,拒周軍。



 周齊王憲來伐,先遣送書并赦詔,湝並沉諸井。戰敗,湝、孝珩俱被擒。憲曰:「
 任城王何苦至此?」湝曰:「下官神武帝子,兄弟十五人,幸而獨存,逢宗社顛覆,今日得死,無愧墳陵。」憲壯之,歸其妻子。將至鄴城,湝馬上大哭,自投于地,流血滿面。至長安,尋與後主同死。



 妃盧氏,賜斛斯征,蓬首垢面,長齋不言笑。徵放之,乃為尼。隋開皇三年,表請文帝葬湝及五子於長安北原。



 高陽康穆王湜,神武第十一子也。天保元年封。十年,稍遷尚書令。以滑稽便辟,有寵於文宣,常在左右,行杖以撻
 諸王。太后深銜之。其妃父護軍長史張晏之嘗要道拜湜,湜不禮焉。帝問其故,對曰:「無官職漢,何須禮。」帝於是擢拜晏之為徐州刺史。文宣崩,兼司徒,導引梓宮,吹笛,云「至尊頗知臣不」,又擊胡鼓為樂。太后杖湜百餘,未幾薨。太后哭之哀,曰:「我恐其不成就,與杖,何期帶創死也!」乾明初,贈假黃鉞、太師、司徒、錄尚書事。子士義襲爵。



 博陵文簡王濟,神武第十二子也。天保元年封。濟嘗從文宣巡幸,在路忽憶太后,遂逃歸。帝怒,臨以白刃,因此
 驚恍。歷位太尉。河清初,出為定州刺史。天統五年,在州語人云:「計次第亦應到我。」後主聞之,陰使人殺之。贈假黃鉞、太尉、錄尚書事。子智襲爵。



 華山王凝,神武第十三子也。天保元年,封新平郡王;九年,改封安定;十年,封華山。歷位中書令、齊州刺史,就加太傅。薨於州,贈左丞相、太師、錄尚書。



 凝諸王中最為孱弱,妃王氏,太子洗馬王洽女也,與倉頭姦,凝知而不能限禁。後事發,王氏賜死,詔杖凝一百。其愚如此。



 馮翊王潤,字子澤,神武第十四子也。幼時,神武稱曰:「此吾家千里駒也。」



 天保初封。歷位東北道大行臺、右僕射、都督、定州刺史。潤美姿儀,年十四五,母鄭妃與之同寢,有穢雜之聲。及長,廉慎方雅,習於吏職,至摘發隱偽,姦吏無所匿其情。開府王迴洛與六州大都督獨孤枝侵竊官田,受納賄賂,潤按舉其事。二人表言,王出送臺使,登魏文舊壇,南望歎息,不測其意。武成使元文遙就州宣敕曰:「馮翊王少小謹慎,在州不為非法,朕信之熟矣。
 登高遠望,人之常情,鼠輩欲橫相間構,曲生眉目。」於是回洛決鞭二百,獨孤枝決杖一百。尋為尚書令,領太子少師,歷司徒、太尉、大司馬、司州牧、太保、河南道行臺、領錄尚書,別封文成郡公、太師、太宰,復為定州刺史。薨,贈假黃鉞、左丞相。子茂德嗣。



 漢陽敬懷王洽,字敬延,神武第十五子也。天保元年封。五年,薨,年十三。



 乾明元年,贈太保、司空。無子,以任城王第二子建德為後。



\end{pinyinscope}