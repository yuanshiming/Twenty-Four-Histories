\article{卷十四列傳第六}

\begin{pinyinscope}

 廣平公盛陽州公永樂弟長弼襄樂王顯國上洛王思宗子元海
 平秦王歸彥武興王普長樂太守靈山嗣子伏護廣平公盛,神武從叔祖也。寬厚有長者風。神武起兵於信都,以盛為中軍大都督,封廣平郡公。歷位司徒、太尉。天平三年,薨於位。贈假黃鉞,太尉、太師、錄尚書事。無子,以兄子子瑗嗣。天保初,改封平昌王,卒於魏尹。



 陽州公永樂,神武從祖兄子也。太昌初,封陽州縣伯,進
 爵為公。累遷北豫州刺史。河陰之戰,司徒高昂失利退。永樂守河陽南城,昂走趣城,西軍追者將至,永樂不開門,昂遂為西軍所擒。神武大怒,杖之二百。後罷豫州,家產不立。神武問其故,對曰:「裴監為長史,辛公正為別駕,受王委寄,斗酒隻雞不入。」神武乃以永樂為濟州,仍以監、公正為長史、別駕。謂永樂曰:「爾勿大貪,小小義取莫復畏。」永樂至州,監、公正諫不見聽,以狀啟神武。神武封啟以示永樂。然後知二人清直,並擢用之。永樂卒於州。
 贈太師、太尉、錄尚書事,謚曰武昭。無子,從兄思宗以第二子孝緒為後,襲爵。天保初,改封脩城郡王。



 永樂弟長弼,小名阿伽。性粗武,出入城市,好毆擊行路,時人皆呼為阿伽郎君。以宗室封廣武王。時有天恩道人,至兇暴,橫行閭肆,後入長弼黨,專以斗為事。文宣並收掩付獄,天恩黨十餘人皆棄市,長弼鞭一百。尋為南營州刺史,在州無故自驚走,叛亡入突厥,竟不知死所。



 襄樂王顯國,神武從祖弟也。無才伎,直以宗室謹厚,天
 保元年,封襄樂王,位右衛將軍。卒。



 上洛王思宗,神武從子也。性寬和,頗有武幹。天保初,封上洛郡王。歷位司空、太傅。薨於官。



 子元海,累遷散騎常侍。願處山林,脩行釋典。文宣許之。乃入林慮山,經二年,絕棄人事,志不能固,啟求歸。徵復本任,便縱酒肆情,廣納姬侍。又除領軍,器小志大,頗以智謀自許。皇建末,孝昭幸晉陽,武成居守,元海以散騎常侍留典機密。初孝昭之誅楊愔等,謂武成云:「事成,以爾為皇太弟。」及踐
 祚,乃使武成在鄴主兵,立子百年為皇太子,武成甚不平。先是,恒留濟南於鄴,除領軍厙狄伏連為幽州刺史,以斛律豐樂為領軍,以分武成之權。武成留伏連而不聽豐樂視事。



 乃與河南王孝瑜偽獵,謀於野,暗乃歸。先是童謠云:「中興寺內白鳧翁,四方側聽聲雍雍,道人聞之夜打鐘。」時丞相府在北城中,即舊中興寺也。鳧翁,謂雄雞,蓋指武成小字步落稽也。道人,濟南王小名。打鐘,言將被擊也。既而太史奏言北城有天子氣。昭帝以為
 濟南應之,乃使平秦王歸彥之鄴,迎濟南赴並州。武成先咨元海,并問自安之計。元海曰:「皇太后萬福,至尊孝性非常,殿下不須別慮。」



 武成曰:「豈我推誠之意耶?」元海乞還省一夜思之。武成即留元海後堂。元海達旦不眠,唯繞床徐步。夜漏未曙,武成遽出,曰:「神算如何?」答云:「夜中得三策,恐不堪用耳。」因說梁孝王懼誅入關事,請乘數騎入晉陽,先見太后求哀,後見主上,請去兵權,以死為限,求不干朝政,必保太山之安。此上策也。若不然,
 當具表,云「威權大盛,恐取謗眾口,請青、齊二州刺史。沉靜自居,必不招物議。



 此中策也。」更問下策曰:「發言即恐族誅。」因逼之,答曰:「濟南世嫡,主上假太后令而奪之。今集文武,示以此敕,執豐樂,斬歸彥,尊濟南,號令天下,以順討逆,此萬世一時也。」武成大悅,狐疑,竟未能用。乃使鄭道謙卜之,皆曰:「不利舉事,靜則吉。」又召曹魏祖,問之國事。對曰:「當有大凶。」又時有林慮令姓潘,知占候,密謂武成曰:「宮車當晏駕,殿下為天下主。」武成拘之於內以
 候之。又令巫覡卜之,多云不須舉兵,自有大慶。武成乃奉詔,令數百騎送濟南於晉陽。



 及孝昭崩,武成即位,除元海侍中、開府儀同三司、太子詹事。河清二年,元海為和士開所譖,被捶馬鞭六十。責云:「爾在鄴城,說我以弟反兄,幾許不義!



 鄴城兵馬抗并州,幾許無智!不義無智,若為可使?」出為兗州刺史。元海後妻,陸太姬甥也,故尋被追任使。武平中,與祖珽共執朝政。元海多以太姬密語告珽。



 珽求領軍,元海不可,珽乃以其所告報太姬。
 姬怒,出元海為鄭州刺史。鄴城將敗,徵為尚書令。周建德七年,於鄴城謀逆,伏誅。元海好亂樂禍,然詐仁慈,不飲灑啖肉。文宣天保末年敬信內法,乃至宗廟不血食,皆元海所謀。及為右僕射,又說後主禁屠宰,斷酤酒。然本心非靖,故終致覆敗。思宗弟思好。



 思好本浩氏子也,思宗養以為弟,遇之甚薄。少以騎射事文襄。及文宣受命,為左衛大將軍。本名思孝,天保五年,討蠕蠕,文宣悅其驍勇,謂曰:「爾擊賊如鶻入鴉群,宜
 思好事。」故改名焉。累遷尚書令、朔州道行臺、朔州刺史、開府、南安王,甚得邊朔人心。後主時,斫骨光弁奉使至州,思好迎之甚謹,光弁倨敖,思好因心銜恨。武平五年,遂舉兵反。與并州諸貴書曰:「主上少長深宮,未辨人之情偽,暱近凶狡,疏遠忠良。遂使刀鋸刑餘,貴溢軒階,商胡醜類,擅權帷幄,剝削生靈,劫掠朝市。闇於聽受,專行忍害。幽母深宮,無復人子之禮;二弟殘戮,頓絕孔懷之義。仍縱子立奪馬於東門,光弁擎鷹於西市,駮龍得儀
 同之號,逍遙受郡君之名,犬馬班位,榮冠軒冕。人不堪役,思長亂階。趙郡王睿實曰宗英,社稷惟寄,左丞相斛律明月,世為元輔,威著鄰國,無罪無辜,奄見誅殄。孤既忝預皇枝,實蒙殊獎,今便擁率義兵,指除君側之害。幸悉此懷,無致疑惑。」行臺郎王行思之辭也。



 思好至陽曲,自號大丞相,置百官,以行臺左丞王尚之為長史。武衛趙海在晉陽掌兵,時倉卒不暇奏,矯詔發兵拒之。軍士皆曰:「南安王來,我輩唯須唱萬歲奉迎耳。」帝聞變,使
 唐邕、莫多婁敬顯、劉桃枝、中領軍厙狄士文馳之晉陽,帝勒兵續進。思好軍敗,與行思投水而死。其麾下二千人,桃枝圍之,且殺且招,終不降以至盡。時帝在道,叱奴世安自晉陽送露布於平都,遇斛斯孝卿。孝卿誘使食,因馳詣行宮,叫已了。帝大懽,左右呼萬歲。良久,世安乃以狀自陳。帝曰:「告示何物事,乃得坐食。」於是賞孝卿而免世安罪。暴思好屍七日,然後屠剝焚之,烹尚之於鄴市,令內參射其妃於宮內,仍火焚殺之。思好反前五旬,
 有人告其謀反。



 韓長鸞女適思好子,故奏有人誣告諸貴,事相擾動,不殺無以息後,乃斬之。思好既誅,死者弟伏闕下訴求贈兄,長鸞不為通也。平秦王歸彥,字仁英,神武族弟也。



 父徽,魏末坐事當徙涼州,行至河、渭間,遇賊,以軍功得免流。因於河州積年。



 以解胡言,為西域大使,得胡師子來獻,以功得河東守。尋遂死焉。徽於神武舊恩甚篤。及神武平京洛,迎徽喪與穆同營葬。贈司徒,謚曰文宣。



 初,徽嘗過長安市,與婦
 人王氏私通而生歸彥,至是年已九歲。神武追見之,撫對悲喜。稍遷徐州刺史。歸彥少質朴,後更改節,放縱好聲色,朝夕酣歌。妻魏上黨王元天穆女也,貌不美而甚驕妒,數忿爭,密啟文宣求離,事寢不報。天保元年,封平秦王。嫡妃康及所生母王氏並為太妃。善事二母,以孝聞。徵為兼侍郎,稍被親寵。以討侯景功,別封長樂郡公,除領軍大將軍。領軍加大,自歸彥始也。



 文宣誅高德正,金寶財貨悉以賜之。乾明初,拜司徒,仍總知禁衛。



 初,濟南
 自晉陽之鄴,楊愔宣敕,留從駕五千兵於西中,陰備非常。至鄴數日,歸彥乃知之,由是陰怨楊、燕。楊、燕等欲去二王,問計於歸彥。歸彥詐喜,請共元海量之。元海亦口許心違,馳告長廣。長廣於是誅楊、燕等。孝昭將入雲龍門,都督成休寧列仗拒而不內,歸彥諭之,然後得入,進向柏閣、永巷亦如之。孝昭踐祚,以此彌見優重,每入常在平原王段韶上。以為司空,兼尚書令。齊制,宮內唯天子紗帽,臣下皆戎帽,特賜歸彥紗帽以寵之。



 孝昭崩,歸
 彥從晉陽迎武成於鄴。及武成即位,進位太傅,領司徒,常聽將私部曲三人帶刀入仗。從武成還都,諸貴戚等競要之,其所往處,一坐盡傾。歸彥既地居將相,志意盈滿,發言陵侮,旁若無人。議者以威權震主,必為禍亂。上亦尋其前翻覆之跡,漸忌之。高元海、畢義雲、高乾和等咸數言其短。上幸歸彥家,召魏收對御作詔草,欲加右丞相。收謂元海曰:「至尊以右丞相登位,今為歸彥威名太盛,故出之,豈可復加此號。」乃拜太宰、冀州刺史,即乾
 和繕寫。晝日,仍敕門司不聽輒入內。時歸彥在家縱酒,經宿不知,至明欲參,至門知之,大驚而退。



 及通名謝,敕令早發,別賜錢帛、鼓吹、醫藥,事事周備。又敕武職督將悉送至青陽宮,拜而退,莫敢共語。唯與趙郡王睿久語,時無聞者。



 至州,不自安,謀逆,欲待受調訖,班賜軍士,望車駕如晉陽,乘虛入鄴。為其郎中令呂思禮所告,詔平原王段韶襲之。歸彥舊於南境置私驛,聞軍將逼,報之,便嬰城拒守。先是,冀州長史宇文仲鸞、司馬李祖挹、別
 駕陳季璩、中從事房子弼、長樂郡守尉普興等疑歸彥有異,使連名密啟,歸彥追而獲之,遂收禁仲鸞等五人,仍並不從,皆殺之。軍已逼城,歸彥登城大叫云:「孝昭皇帝初崩,六軍百萬眾悉由臣手,投身向鄴迎陛下,當時不反,今日豈有異心?正恨高元海、畢義雲、高乾和誑惑聖上,疾忌忠良。但為殺此三人,即臨城自刎。」其後城破,單騎北走,至交津見獲,鎖送鄴。帝令趙郡王睿私問其故。歸彥曰:「使黃領小兒牽挽我,何可不反!」曰:「誰耶?」歸彥
 曰:「元海、乾和豈是朝廷老宿?如趙家老公時,又詎懷怨。」於是帝又使讓焉。對曰:「高元海受畢義雲宅,用作本州刺史,給後部鼓吹。臣為藩王、太宰,仍不得鼓吹。正殺元海、義雲而已。」上令都督劉桃枝牽入,歸彥猶作前語望活。帝命議其罪,皆云不可赦。乃載以露車,銜枚面縛,劉桃枝臨之以刃,擊鼓隨之,并子孫十五人皆棄市。贈仁州刺史。



 魏時山崩,得石角二,藏在武庫。文宣入庫,賜從臣兵器,特以二石角與歸彥。



 謂曰:「爾事常山不得反,事
 長廣得反,反時,將此角嚇漢。」歸彥額骨三道,著幘不安。文宣嘗見之,怒,使以馬鞭擊其額,血被面,曰:「爾反時當以此骨嚇漢。」



 其言反竟驗云。



 武興王普,字德廣,歸彥兄歸義之子也。性寬和有度量。九歲,歸彥自河州俱入洛,神武使與諸子同遊處。天保初,封武興郡王。武平二年,累遷司空。六年,為豫州道行臺、尚書令。後主奔鄴,就加太宰。周師逼,乃降。卒於長安。贈上開府、豫州刺史。



 長樂太守靈山,字景嵩,神武族弟也。從神武起兵信都,終於長樂太守。贈大將軍、司空,謚曰文宣。子懿,卒於武平鎮將,無子,文宣帝以靈山從父兄齊州刺史建國子伏護為靈山後。



 伏護,字臣援,粗有刀筆。天統初,累遷黃門侍郎。伏護歷事數朝,恆參機要,而性嗜酒,每多醉失,末路逾劇,乃至連日不食,事事酣酒,神識恍惚,遂以卒。



 贈兗州刺史。建國侯孫乂襲。乂少謹。武平末,給事黃門侍郎。隋開皇中,為太府少卿,坐事卒。



\end{pinyinscope}