\article{卷二十七列傳第十九}

\begin{pinyinscope}

 萬
 俟普子洛可朱渾元劉豐破六韓常
 金祚韋子粲萬俟普,字普撥,太平人,其先匈奴之別種也。雄果有武力。正光中,破六韓拔陵構逆,授普太尉。率部下降魏,授後將軍,第二領民酋長。高祖起義,普遠通誠款,高祖甚嘉之。斛斯椿逼帝西出,授司空、秦州刺史,據覆靺城。高祖平夏州,普乃率其部落來奔,高祖躬自迎接,授普河西公。累遷太尉、朔州刺史,卒。



 子洛,字受洛干。豪壯有武藝,騎射過人,為鄉閭所伏。拔陵反,隨父歸順,除顯武將軍。隨汆朱榮每有戰功,累遷汾州刺史、驃騎將軍。及起義信都,遠送誠款,高祖嘉其父子俱至,甚優其禮。除撫軍,兼靈州刺史。武帝入關,除左僕射。



 天平中,隨父東歸,封建昌郡公,再遷領軍將軍。與諸將圍獨孤如願於金墉,及河陰之戰,並有功。高祖以其父普尊老,特崇禮之,嘗親扶上馬。洛免冠稽首曰:「願出死力以報深恩。」及此役也,諸軍北渡橋,洛以一軍
 不動。謂西人曰:「萬俟受洛干在此,能來可來也!」西人畏而去。高祖以雄壯,名其所營地為回洛城。



 洛慷慨有氣節,勇銳冠時,當世推為名將。興和初卒。



 可朱渾元,字道元。自云遼東人。世為渠帥,魏時擁眾內附,曾祖護野肱終於懷朔鎮將,遂家焉。元寬仁有武略,少與高祖相知。北邊擾亂,遂將家屬赴定州,值鮮于修禮作亂,元擁眾屬焉。葛榮併修禮,復以元為梁王。遂奔爾朱榮,以為別將,隸天光征關中,以功為渭州刺史。



 侯
 莫陳悅之殺賀拔岳也,周文帝率岳所部還,共圖悅。元時助悅,悅走,元收其眾,入據秦州,為周攻圍,苦戰,結盟而罷。元既早被高祖知遇,兼其母兄在東,嘗有思歸之志,恒遣表疏,與高祖陰相往來。周文忌元智勇,知元懷貳,發兵攻之。



 元乃率所部發自渭州,西北渡烏蘭津。周文頻遣兵邀之,元戰必摧之。引軍歷河、源二州境,乃得東出。靈州刺史曹女婿劉豐與元深相交結,元因說豐以高祖英武非常,克成大業,豐自此便有委質
 之心,遂資遣元。元從靈州東北入雲州。高祖聞其來也,遣平陽守高嵩持金環一枚以賜元,并運資糧,遠遣候接。元至晉陽,引見執手,賜帛千匹並奴婢田宅。兄弟四人先在并州者,進官爵。元所部督將,皆賞以爵邑。封元縣公,除車騎大將軍。



 討西魏儀同金祚、皇甫智達於東雍,擒之。遷并州刺史。又與諸將征伐,頻有剋捷降下。天保初,封扶風王。頻從顯祖討山胡、茹茹,累有戰功。遷太師,薨。



 贈假黃鉞、太宰、錄尚書。元善於御眾,行軍用兵,務在
 持重,前後出征,未嘗負敗。及卒,朝廷深悼之。皇建初,配享世宗廟庭。



 劉豐,字豐生,普樂人也。有雄姿壯氣,果毅絕人,有口辯,好說兵事。破六韓拔陵之亂,豐以守城之功,除普樂太守。魏永安初,除靈州鎮城大都督。周文授以衛大將軍,豐不受,乃遣攻圍,不剋。豐遠慕高祖威德,乃率戶數萬來奔。高祖上豐為平西將軍、南汾州刺史。遂與諸將征討,平定寇亂。又從高祖破周文於河陰,豐功居多,高祖
 執手嗟賞。入為左衛將軍,出除殷州。



 王思政據長社,世宗命豐與清河王岳攻之。豐建水攻之策,遂遏洧水以灌之,水長,魚鱉皆游焉。九月至四月,城將陷。豐與行臺慕容紹宗見北有白氣同入船,忽有暴風從東北來,正晝昏暗,飛沙走礫,船纜忽絕,漂至城下。豐游水向土山,為浪所激,不時至,西人鉤之,並為敵人所害。豐壯勇善戰,為諸將所推。死之日,朝野駭惋。贈大司馬、司徒公、尚書令,謚曰忠。子曄嗣。



 破六韓常,字保年,附化人,匈奴單于之裔也。右谷蠡王潘六奚沒於魏,其子孫以潘六奚為氏,後人訛誤,以為破六韓。世領部落,其父孔雀,世襲酋長。孔雀少驍勇。時宗人拔陵為亂,以孔雀為大都督、司徒、平南王。孔雀率部下一萬人降於爾朱榮,詔加平北將軍、第一領民酋長,卒。常沉敏有膽略,善騎射,累遷平西將軍。高祖起義,常為附化守,與萬俟受洛干東歸,高祖嘉之,上為撫軍。與諸將征討,又從高祖攻擊諸寇,累遷車騎大將軍、開
 府,封平陽公,除洛州刺史。常啟世宗曰:「常自鎮河陽以來,頻出關口,太谷二道,北荊已北,洛州已南,所有要害,頗所知悉。而太谷南口去荊路踰一百,經赤工阪,是賊往還東西大道,中間曠絕一百五十里,賊之糧饟,唯經此路。愚謂於彼選形勝之處,營築城戍,安置士馬,截其遠還,自然不能更有行送。」世宗納其計,遣大司馬斛律金等築楊志、百家、呼延三鎮。常秩滿,還晉陽,拜太保、滄州刺史,卒。贈尚書令、司徒公、太傅、第一領民酋長,假王,
 謚曰忠武。



 金祚,字神敬,安定人也。性驍雄,尚氣任俠。魏正光中,隴右賊起,詔雍州刺史元猛討之,召募狼家,以為軍導,祚應選。以軍功累遷龍驤將軍、靈州刺史。



 高祖舉義,爾朱天光率關右之眾與仲遠等北抗義師。天光留祚東秦,總督三州,鎮靜二州。天光敗,歸高祖,除車騎大將軍。邙山之戰,以大都督從破西軍。祚除華州刺史,加開府儀同三司,別封臨濟縣子,卒。贈司空。



 韋子粲,字暉茂,京兆人。曾祖閬,魏咸陽守。父雋,都水使者。子粲仕郡功曹史,累遷為大行臺郎中,從爾朱天光平關右。孝武入關,以為南汾州刺史。神武命將出討,城陷,子弟俱破獲,送晉陽,蒙放免。以粲為並州長史,累遷豫州刺史,卒。初,子粲兄弟十三人,子侄親屬,闔門百口悉在西魏。以子粲陷城不能死難,多致誅滅,歸國獲存,唯與弟道諧二人而已。諧與粲俱入國。粲富貴之後,遂特棄道諧,令其異居,所得廩祿,略不相及,其不顧恩義
 如此。



\end{pinyinscope}