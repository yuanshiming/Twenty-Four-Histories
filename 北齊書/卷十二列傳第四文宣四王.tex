\article{卷十二列傳第四文宣四王}

\begin{pinyinscope}

 太
 原王紹德范陽王紹義西河王紹仁
 隴西王紹廉孝昭六王樂陵王百年始平王彥德城陽王彥基定陽王彥康汝陽王彥忠汝南王彥理
 武成十二王南陽王綽琅邪王儼齊安王廓北平王貞高平王仁英淮南王仁光西河王仁幾
 樂平王仁邕潁川王仁儉安陽王仁雅丹陽王仁直東海王仁謙文宣五男:李后生廢帝及太原王紹德,馮世婦生范陽王紹義,裴嬪生西河王紹仁,顏嬪生隴西王紹廉。



 太原王紹德,文宣第二子也。天保末,為開府儀同三司。
 武成因怒李后,罵紹德曰:「你父打我時,竟不來救!」以刀環築殺之,親以土埋之遊豫園。武平元年,詔以范陽王子辨才為後,襲太原王。



 范陽王紹義,文宣第三子也。初封廣陽,後封范陽。歷位侍中、清都尹。好與群小同飲,擅置內參,打殺博士任方榮。武成嘗杖之二百,送付昭信后,后又杖一百。及後主奔鄴,以紹義為尚書令、定州刺史。周武帝克并州,以封輔相為北朔州總管。此地齊以重鎮,諸勇士多聚焉。前
 長史趙穆、司馬王當萬等謀執輔相,迎任城王於瀛州。事不果,便迎紹義。紹義至馬邑。輔相及其屬韓阿各奴等數十人皆齊叛臣,自肆州以北城戍二百八十餘盡從輔相,及紹義至,皆反焉。紹義與靈州刺史袁洪猛引兵南出,欲取并州,至新興而肆州已為周守。前隊二儀同以所部降周。周兵擊顯州,執刺史陸瓊,又攻陷諸城。紹義還保北朔。周將宇文神舉軍逼馬邑,紹義遣杜明達拒之,兵大敗。紹義曰:「有死而已,不能降人。」遂奔突厥。
 眾三千家,令之曰:「欲還者任意。」於是哭拜別者太半。突厥他缽可汗謂文宣為英雄天子,以紹義重踝似之,甚見愛重,凡齊人在北者,悉隸紹義。高寶寧在營州,表上尊號,紹義遂即皇帝位,稱武平元年。以趙穆為天水王。他缽聞寶寧得平州,亦招諸部,各舉兵南向,云共立范陽王作齊帝,為其報仇。周武帝大集兵於雲陽,將親北伐,遇疾暴崩。紹義聞之,以為天贊己。盧昌斯據范陽,亦表迎紹義。俄而周將宇文神舉攻滅昌期。其日,紹義適
 至幽州,聞周總管出兵于外,欲乘虛取薊城,列天子旌旗,登燕昭王塚,乘高望遠,部分兵眾。神舉遣大將軍宇文恩將四千人馳救幽州,半為齊軍所殺。紹義聞范陽城陷,素服舉哀,迴軍入突厥。周人購之於他缽,又使賀若誼往說之。他缽猶不忍,遂偽與紹義獵於南境,使誼執之,流于蜀。紹義妃渤海封孝琬女,自突厥逃歸。紹義在蜀,遺妃書云:「夷狄無信,送吾於此。」



 竟死蜀中。



 西河王紹仁,文宣第四子也,天保末,為開府儀同三司。
 尋薨。



 隴西王紹廉,文宣第五子也。初封長樂,後改焉。性粗暴,嘗拔刀逐紹義,紹義走入廄,閉門拒之。紹義初為清都尹,未及理事,紹廉先往,喚囚悉出,率意決遣之。能飲酒,一舉數升,終以此薨。



 孝昭七男:元后生樂陵王百年,桑氏生襄城王亮,出後襄城景王,諸姬生汝南王彥理、始平王彥德、城陽王彥基、定陽王彥康、汝陽王彥忠。



 樂陵王百年,孝昭第二子也。孝昭初即位,在晉陽,群臣請建中宮及太子,帝謙未許,都下百僚又有請,乃稱太后令立為皇太子。帝臨崩,遺詔傳位於武成,并有手書,其末曰:「百年無罪,汝可以樂處置之,勿學前人。」大寧中,封樂陵王。



 河清三年五月,白虹圍日再重,又橫貫而不達。赤星見,帝以盆水承星影而蓋之,一夜盆自破。欲以百年厭之。會博陵人賈德胄教百年書,百年嘗作數「敕」字,德胄封以奏。帝乃發怒,使召百年。百年被召,自知不免,
 割帶玦留與妃斛律氏。見帝於玄都苑涼風堂,使百年書「敕」字,驗與德胄所奏相似,遣左右亂捶擊之,又令人曳百年繞堂且走且打,所過處血皆遍地。氣息將盡,曰:「乞命,願與阿叔作奴。」遂斬之,棄諸池,池水盡赤,於後園親看埋之。妃把玦哀號,不肯食,月餘亦死,玦猶在手,拳不可開,時年十四,其父光自擘之,乃開。後主時,改九院為二十七院,掘得一小屍,緋袍金帶,一髻一解,一足有靴。諸內參竊言,百年太子也,或言太原王紹德。詔以襄
 成王子白澤襲爵樂陵王。齊亡,入關,徙蜀死。



 汝南王彥理,武平初封王,位開府、清都尹。齊亡,入關,隨例授儀同大將軍,封縣子。女入太子宮,故得不死。隋開皇中,卒并州刺史。



 始平王彥德、城陽王彥基、定陽王彥康、汝陽王彥忠,與汝南同受封,並加儀同三司,後事闕。



 武成十三男:胡皇后生後主及琅邪王儼,李夫人生南陽王綽,後宮生齊安王廓、北平王貞、高平王仁英、淮南
 王仁光、西河王仁幾、樂平王仁邕、潁川王仁儉、安樂王仁雅、丹陽王仁直、東海王仁謙。



 南陽王綽,字仁通,武成長子也。以五月五日辰時生,至午時,後主乃生。武成以綽母李夫人非正嫡,故貶為第二,初名融,字君明,出後漢陽王。河清三年,改封南陽,別為漢陽置後。綽始十餘歲,留守晉陽。愛波斯狗,尉破胡諫之,欻然斫殺數狗,狼藉在地。破胡驚走,不敢復言。後為司徒、冀州刺史,好裸人,使踞為獸狀,縱犬噬而食之。左
 轉定州,汲井水為後池,在樓上彈人。好微行,遊獵無度,恣情彊暴,云學文宣伯為人。有婦人抱兒在路,走避入草,綽奪其兒飼波斯狗。



 婦人號哭,綽怒,又縱狗使食,狗不食,塗以兒血,乃食焉。後主聞之,詔鎖綽赴行在所。至而宥之。問在州何者最樂,對曰:「多取歇將蛆混,看極樂。」後主即夜索歇一斗,比曉得三二升,置諸浴斛,使人裸臥斛中,號叫宛轉。帝與綽臨觀,喜噱不已,謂綽曰:「如此樂事,何不早馳驛奏聞。」綽由是大為後主寵,拜大將軍,
 朝夕同戲。韓長鸞間之,除齊州刺史。將發,長鸞令綽親信誣告其反,奏云:「此犯國法,不可赦。」後主不忍顯戮,使寵胡何猥薩後園與綽相撲,搤殺之。



 瘞於興聖佛寺。經四百餘日乃大斂,顏色毛髮皆如生,俗云五月五日生者腦不壞。



 綽兄弟皆呼父為兄兄,嫡母為家家,乳母為姊姊,婦為妹妹。齊亡,妃鄭氏為周武帝所幸,請葬綽。敕所司葬於永平陵北。



 琅邪王儼,字仁威,武成第三子也。初封東平王,拜開府、
 侍中、中書監、京畿大都督、領軍大將軍、領御史中丞,遷司徒、尚書令、大將軍、錄尚書事、大司馬。魏氏舊制,中丞出,清道,與皇太子分路行,王公皆遙住車,去牛,頓軛於地,以待中丞過,其或遲違,則赤棒棒之。自都鄴後,此儀浸絕,武成欲雄寵儼,乃使一依舊制。初從北宮出,將上中丞,凡京畿步騎,領軍之官屬,中丞之威儀,司徒之鹵簿,莫不畢備。帝與胡后在華林園東門外張幕,隔青紗步障觀之。遣中貴驟馬趣仗,不得入,自言奉敕,赤棒
 應聲碎其鞍,馬驚人墜。帝大笑,以為善。更敕令駐車,傳語良久,觀者傾京邑。儼恒在宮中,坐含光殿以視事,諸父皆拜焉。帝幸并州,儼常居守,每送駕,或半路,或至晉陽,乃還。王師羅常從駕,後至,武成欲罪之,辭曰:「臣與第三子別,留連不覺晚。」武成憶儼,為之下泣,舍師羅不問。儼器服玩飾,皆與後主同,所須悉官給。於南宮嘗見新冰早李,還,怒曰:「尊兄已有,我何意無!」從是,後主先得新奇,屬官及工匠必獲罪。太上、胡后猶以為不足。儼常患
 喉,使醫下針,張目不瞬。又言於帝曰:「阿兄懦,何能率左右?」帝每稱曰:「此黠兒也,當有所成。」以後主為劣,有廢立意。武成崩,改封琅邪。儼以和土開、駱提婆等奢恣,盛修第宅,意甚不平,嘗謂曰:「君等所營宅早晚當就,何太遲也。」二人相謂曰:「琅邪王眼光奕奕,數步射人,向者暫對,不覺汗出,天子前奏事尚不然。」由是忌之。武平二年,出儼居北宮,五日一朝,不復得每日見太后。四月,詔除太保,餘官悉解,猶帶中丞,督京畿。以北城有武庫,欲移儼
 於外,然後奪其兵權。治書侍御史王子宜與儼左右開府高舍洛、中常侍劉辟疆說儼曰:「殿下被疏,正由士開間構,何可出北宮入百姓叢中也。」儼謂侍中馮子琮曰:「士開罪重,兒欲殺之。」子琮心欲廢帝而立儼,因贊成其事。儼乃令子宜表彈士開罪,請付禁推。子琮雜以他文書奏之,後主不審省而可之。儼誑領軍厙狄伏連曰:「奉敕令軍收士開。」伏連以咨子琮,且請覆奏。子琮曰:「琅邪王受敕,何須重奏。」伏連信之,伏五十人於神獸門外,
 詰旦,執士開送御史。儼使馮永洛就臺斬之。儼徒本意唯殺士開,及是,因逼儼曰:「事既然,不可中止。」



 儼遂率京畿軍士三千餘人屯千秋門。帝使劉桃枝將禁兵八十人召儼。桃枝遙拜,儼命反縛,將斬之,禁兵散走。帝又使馮子琮召儼,儼辭曰:「士開昔來實合萬死,謀廢至尊,剃家家頭使作阿尼,故擁兵馬欲坐著孫鳳珍宅上,臣為是矯詔誅之。尊兄若欲殺臣,不敢逃罪,若放臣,願遣姊姊來迎臣,臣即入見。」姊姊即陸令萱也,儼欲誘出殺之。
 令萱執刀帝後,聞之戰慄。又使韓長鸞召儼,儼將入,劉辟疆牽衣諫曰:「若不斬提婆母子,殿下無由得入。」廣寧、安德二王適從西來,欲助成其事,曰:「何不入?」辟疆曰:「人少。」安德王顧眾而言曰:「孝昭帝殺楊遵彥,止八十人,今乃數千,何言人少?」後主泣啟太后曰:「有緣更見家家,無緣永別。」



 乃急召斛律光,儼亦召之。光聞殺士開,撫掌大笑曰:「龍子作事,固自不似凡人。」



 入見後主於永巷。帝率宿衛者步騎四百,授甲將出戰。光曰:「小兒輩弄兵,與交
 手即亂。鄙諺云『奴見大家心死』,至尊宜自至千秋門,琅邪必不敢動。」皮景和亦以為然,後主從之。光步道,使人出曰:「大家來。」儼徒駭散。帝駐馬橋上,遙呼之,儼猶立不進。光就謂曰:「天子弟殺一漢,何所苦。」執其手,彊引以前。



 請帝曰:「琅邪王年少,腸肥腦滿,輕為舉措,長大自不復然,願寬其罪。」帝拔儼帶刀環亂築辮頭,良久乃釋之。收伏連及高舍洛、王子宜、劉辟疆、都督翟顯貴於後園,帝親射之而後斬,皆支解,暴之都街下。文武職吏盡欲
 殺之。光以皆勳貴子弟,恐人心不安,趙彥深亦云《春秋》責帥,於是罪之各有差。儼之未獲罪也,鄴北城有白馬佛塔,是石季龍為澄公所作,儼將修之。巫曰:「若動此浮圖,北城失主。」不從,破至第二級,得白蛇長數丈,回旋失之,數旬而敗。自是太后處儼於宮內,食必自嘗之。陸令萱說帝曰:「人稱琅邪王聰明雄勇,當今無敵,觀其相表,殆非人臣。自專殺以來,常懷恐懼,宜早為計。」何洪珍與和士開素善,亦請殺之。未決,以食輿密迎祖珽問之,珽
 稱周公誅管叔,季友鴆慶父,帝納其言。以儼之晉陽,使右衛大將軍趙元侃誘執儼。元侃曰:「臣昔事先帝,日見先帝愛王,今寧就死,不能行。」帝出元侃為豫州刺史。九月下旬,帝啟太后曰:「明旦欲與仁威出獵,須早出早還。」是夜四更,帝召儼,儼疑之。陸令萱曰:「兄兄喚,兒何不去?」儼出至永巷,劉桃枝反接其手。儼呼曰:「乞見家家、尊兄!」桃枝以袂塞其口,反袍蒙頭負出,至大明宮,鼻血滿面,立殺之,時年十四。不脫靴,裹以席,埋於室內。帝使啟太
 后,臨哭十餘聲,便擁入殿。明年三月,葬於鄴西,贈謚曰楚恭哀帝,以慰太后。有遺腹四男,生數月,皆幽死。以平陽王淹孫世俊嗣。



 儼妃,李祖欽女也,進為楚帝后,居宣則宮。齊亡,乃嫁焉。



 齊安王廓,字仁弘,武成第四子也。性長者,無過行。位特進、開府、儀同三司、定州刺史。



 北平王貞,字仁堅,武成第五子也。沉審寬恕。帝常曰:「此兒得我鳳毛。」



 位司州牧、京畿大都督,兼尚書令、錄尚書
 事。帝行幸,總留臺事。積年,後主以貞長大,漸忌之。阿那肱承旨,令馮士幹劾繫貞於獄,奪其留後權。



 高平王仁英,武成第六子也。舉止軒昂,精神無檢格。位定州刺史。



 淮南王仁光,武成第七子也。性躁且暴,位清都尹。次西河王仁幾,生而無骨,不自支持;次樂平王仁邕;次潁川王仁儉;次安樂王仁雅,從小有喑疾;次丹陽王仁直;次東海王仁謙。皆養於北宮。琅邪王死後,諸王守禁彌切。
 武平末年,仁邕已下始得出外,供給儉薄,取充而已。尋後主窮蹙,以廓為光州,貞為青州,仁英為冀州,仁儉為膠州,仁直為濟州刺史。自廓已下,多與後主死於長安。仁英以清狂,仁雅以喑疾,獲免,俱徙蜀。隋開皇中,追仁英,詔與蕭琮、陳叔寶修其本宗祭禮。未幾而卒。



 後主五男:穆皇后生幼主,諸姬生東平王恪,次善德,次買德,次質錢。胡太后以恪嗣琅邪王,尋夭折。齊滅,周武帝以任城已下大小三十王歸長安,皆有封爵。



 其後不
 從戮者散配西土,皆死邊。



 論曰:文襄諸子,咸有風骨,雖文雅之道,有謝間平,然武藝英姿,多堪禦侮。



 縱咸陽賜劍,覆敗有征,若使蘭陵獲全,未可量也,而終見誅翦,以至土崩,可為太息者矣。安德以時艱主暗,匿迹韜光,及平陽之陣,奮其忠勇,蓋以臨難見危,義深家國。德昌大舉,事迫群情,理至淪亡,無所歸命。廣寧請出後宮,竟不獲遂,非孝珩辭致有謝李同,自是後主心識去平原已遠。存亡事異,安可同年
 而說。武成殘忍奸穢,事極人倫。太原跡異猜嫌,情非釁逆,禍起昭信,遂及淫刑。嗟乎!欲求長世,未之有也。以孝昭德音,庶可慶流後嗣,百年之酷,蓋濟南之濫觴。其云「莫效前人」之言,可為傷嘆,各愛其子,豈其然乎?瑯邪雖無師傅之資,而早聞氣尚。士開淫亂,多歷歲年,一朝剿絕,慶集朝野,以之受斃,深可痛焉。然專戮之釁,未之或免,贈帝謚恭,矯枉過直,觀過知仁,不亦異於是乎?



\end{pinyinscope}