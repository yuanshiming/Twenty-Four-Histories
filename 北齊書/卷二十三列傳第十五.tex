\article{卷二十三列傳第十五魏蘭根 崔(子瞻)}

\begin{pinyinscope}

 魏蘭根,巨鹿下曲陽人也。父伯成,魏中山太守。蘭根身長八尺,儀貌奇偉,汎覽群書,誦《左氏傳》、《周易》,機警有識
 悟。起家北海王國侍郎,歷定州長流參軍。丁母憂,居喪有孝稱。將葬常山郡境,先有董卓祠,祠有柏樹。蘭根以卓凶逆無道,不應遺祠至今,乃伐柏以為郭材。人或勸之不伐,蘭根盡取之,了無疑懼。遭父喪,廬於墓側,負土成墳,憂毀殆於滅性。後為司空、司徒二府記室參軍,轉夏州平北府長史,入為司徒掾,出除本郡太守,並有當官之能。



 正光末,尚書令李崇為本郡都督,率眾討茹茹,以蘭根為長史。因說崇曰:「緣邊諸鎮,控攝長遠。昔時初
 置,地廣人稀,或徵發中原強宗子弟,或國之肺腑,寄以爪牙。中年以來,有司乖實,號曰府戶,役同廝養,官婚班齒,致失清流。而本宗舊類,各各榮顯,顧瞻彼此,理當憤怨。更張琴瑟,今也其時,靜境寧邊,事之大者。宜改鎮立州,分置郡縣,凡是府戶,悉免為民,入仕次敘,一準其舊,文武兼用,威恩並施。此計若行,國家庶無北顧之慮矣。」崇以秦聞,事寢不報。軍還,除冠軍將軍,轉司徒右長史,假節,行豫州事。



 孝昌初,轉岐州刺史。從行臺蕭寶寅討
 破宛川,俘其民人為奴婢,以美女十人賞蘭根。蘭根辭曰:「此縣界於彊虜,皇威未接,無所適從,故成背叛。今當寒者衣之,飢者食之,奈何將充僕隸乎?」盡以歸其父兄。部內麥多五穗,鄰州田鼠為災,犬牙不入岐境。屬秦隴反叛,蕭寶寅敗於涇州,高平虜賊逼岐州,州城民逼囚蘭根降賊。寶寅至雍州,收輯散亡,兵威復振,城民復斬賊刺史侯莫陳仲和,推蘭根復任。朝廷以蘭根得西土人心,加持節、假平西將軍、都督涇岐東秦南岐四州軍
 事,兼四州行臺尚書。尋入拜光祿大夫。



 孝昌末,河北流人南渡,以蘭根兼尚書,使齊、濟、二兗四州安撫,并置郡縣。



 河間邢杲反於青、兗之間,杲,蘭根之甥也,復詔蘭根銜命慰勞。杲不下,仍隨元天穆討之。還,除太府卿,辭不拜。轉安東將軍、中書令。



 莊帝之將誅爾朱榮也,蘭根聞其計,遂密告爾朱世隆。榮死,蘭根恐莊帝知之,憂懼不知所出。時應詔王道習見信於莊帝,蘭根乃托附之,求得在外立功。道習為啟聞,乃以蘭根為河北行臺,於定州
 率募鄉曲,欲防井陘。時爾朱榮將侯深自范陽趣中山,蘭根與戰,大敗,走依渤海高乾。屬乾兄弟舉義,因在其中。高祖至,以蘭根宿望,深禮遇之。中興初,加車騎大將軍、尚書右僕射。及高祖將入洛陽,遣蘭根先至京師。時廢立未決,令蘭根觀察魏前廢帝。帝神采高明,蘭根恐於後難測,遂與高乾兄弟及黃門崔鷿同心固請於高祖,言廢帝本是胡賊所推,今若仍立,於理不允。高祖不得已,遂立武帝。廢帝素有德業,而為蘭根等構毀,深為
 時論所非。



 太昌初,除儀同三司,尋加開府,封巨鹿縣侯,邑七百戶。啟授兄子同達。蘭根既預義勳,位居端揆,至是始敘復岐州勳,封永興縣侯,邑千戶。高乾之死,蘭根懼,去宅,避於寺。武帝大加譴責,蘭根憂怖,乃移病解僕射。天平初,以病篤上表求還鄉里。魏帝遣舍人石長宣就家勞問,猶以開府儀同,門施行馬,歸於本鄉。



 二年卒,時年六十一。贈冀定殷三州軍事、定州刺史、司徒公、侍中,謚曰文宣。



 蘭根雖以功名自立,然善附會,出處之際,
 多以計數為先,是以不為清論所許。



 長子相如,秘書郎中。以建義勳,尋加將軍。襲父爵,遷安東將軍、殷州別駕,入為侍御史。武定三年卒。次子敬仲。肅宗時,佐命功臣配享,而不及蘭根。敬仲表訴,帝以詔命既行,難於追改,擢敬仲為祠部郎中。卒於章武太守。



 蘭根族弟明朗,頗涉經史,粗有文性。累遷大司馬府法曹參軍,兼尚書金部郎中。元顥入洛陽,明朗為南道行臺郎中,為顥所擒。後棄顥逃還,除龍驤將軍、中散大夫,賜爵巨鹿侯。永安
 末,蘭根為河北行臺,引明朗為左丞。及蘭根中山之敗,俱歸高祖。中興初,拜撫軍將軍,出為安德太守。後轉衛將軍、右光祿大夫、定州大中正。武定初,為顯祖諮議參軍。出為平陽太守,為御史所劾,因被禁止。遇病卒。



 明朗從弟愷,少抗直有才辯。魏末,辟開府行參軍,稍遷尚書郎、齊州長史。



 天保中,聘陳使副。遷青州長史,固辭不就。楊愔以聞,顯祖大怒,謂愔云:「何物漢子,我與官不肯就!明日將過,我自共語。」是時顯祖已失德,朝廷皆為之懼,
 而愷情貌坦然。顯祖切責之,仍云:「死與長史孰優,任卿選一處。」愷答云:「能殺臣者是陛下,不受長史者是愚臣,伏聽明詔。」顯祖謂愔云:「何慮無人作官職,苦用此漢何為,放其還家,永不收採。」由是積年沉廢。後遇楊愔於路,微自披陳。楊答曰:「發詔授官,咸由聖旨,非選曹所悉,公不勞見訴。」愷應聲曰:「雖復零雨自天,終待雲興四嶽。公豈得言不知?」楊欣然曰:「此言極為簡要,更不須多語。」數日,除霍州刺史。在職有治方,為邊民悅服。大寧中,卒於
 膠州刺史。



 愷從子彥卿,魏大司農季景之子。武平中,兼通直散騎常侍,聘陳使副。



 彥卿弟淡,學識有詞藻。武平初,殿中御史,遷中書舍人,待詔文林館。隋開皇中,太子舍人、著作郎。撰《後魏書》九十二卷,甚得史體,時稱其善云。



 崔鷿,字長孺,清河東武城人也。父休,魏七兵尚書,贈僕射。鷿狀貌偉麗,善於容止,少有名望,為當時所知。初為魏世宗挽郎,釋褐太學博士。永安中,坐事免歸鄉里。高
 祖於信都起義,鷿歸焉。高祖見之,甚悅,以為諮議參軍。尋除給事黃門侍郎,遷將軍、右光祿大夫。高祖入洛,議定廢立。太僕綦俊盛稱普泰王賢明,可以為社稷主。鷿曰:「若其明聖,自可待我高王,徐登九五。既為逆胡所立,何得猶作天子。若從俊言,王師何名義舉?」由是中興、普泰皆廢,更立平陽王為帝。以建義功,封武城縣公,邑一千四百戶,進位車騎大將軍、左光祿大夫,仍領黃門郎。



 鷿居門下,恃預義旗,頗自矜縱。尋以貪汙為御史糾劾,
 因逃還鄉里,遇赦始出。高祖以鷿本預義旗,復其黃門。天平初,為侍讀,監典書。尋除徐州刺史,給廣宗部曲三百、清河部曲千人。鷿性豪慢,寵妾馮氏,假其威刑,恣情取受,風政不立。初鷿為常侍,求人修起居注。或曰:「魏收可。」鷿曰:「收輕薄徒耳。」



 更引祖鴻勛為之。既居樞要,又以盧元明代收為中書郎,由是收銜之。及收聘梁,過徐州,鷿備刺史鹵簿而送之,使人相聞魏曰:「勿怪儀衛多,稽古之力也。」收報曰:「白崔徐州,建義之勳,何稽古之有!」鷿
 自以門閥素高,特不平此言。收乘宿憾,故以挫之。罷州,除七兵尚書、清河邑中正。



 趙郡李渾嘗宴聚名輩,詩酒正歡嘩,鷿後到,一坐無復談話者。鄭伯獻歎曰:「身長八尺,面如刻畫,謦咳為洪鐘響,胸中貯千卷書,使人那得不畏服!」



 鷿每以籍地自矜,謂盧元明曰:「天下盛門,唯我與爾,博崔趙李,何事者哉!」



 崔暹聞而銜之。高祖葬後,鷿又竊言:「黃頷小兒堪當重任不?」暹外兄李慎以鷿言告暹。暹啟世宗,絕鷿朝謁。鷿要拜道左,世宗發怒曰:「黃
 頷小兒,何足拜也!」



 於是鎖鷿赴晉陽而訊之。鷿不伏,暹引邢子才為證,子才執無此言。鷿在禁,謂子才曰:「卿知我意屬太丘不?」子才出告鷿子瞻云:「尊公意正應欲結姻於陳元康。」



 瞻有女,乃許妻元康子,求其父。元康為言之於世宗曰:「崔鷿名望素重,不可以私處言語便以殺之。」世宗曰:「若免其性命,猶當徙之遐裔。」元康曰:「鷿若在邊,或將外叛。以英賢資寇敵,非所宜也。」世宗曰:「既有季珪之罪,還令輸作可乎?」元康曰:「嘗讀《崔琰傳》,追恨魏武
 不弘。鷿若在作所而殞,後世豈道公不殺也?」世宗曰:「然則奈何?」元康曰:「崔鷿合死,朝野莫不知之,公誠能以寬濟猛,特輕其罰,則仁德彌著,天下歸心。」乃舍之。鷿進謁奉謝,世宗猶怒曰:「我雖無堪,忝當大任,被卿名作黃頷小兒,金石可銷,此言難滅!」



 天保初,除侍中,監起居。以禪代之際,參掌儀禮,別封新豐縣男,邑二百戶,迴授第九弟約。鷿一門婚嫁,皆是衣冠之美,吉凶儀範,為當時所稱。婁太后為博陵王納鷿妹為妃,敕中使曰:「好作法用,
 勿使崔家笑人。」婚夕,顯祖舉酒祝曰:「新婦宜男,孝順富貴。」鷿奏曰:「孝順出自臣門,富貴恩由陛下。」



 五年,出為東兗州刺史,復攜馮氏之部。鷿尋遇偏風,而馮氏驕縱,受納狼藉,為御史所劾,與鷿俱召詣廷尉。尋有別敕,斬馮於都市。鷿以疾卒獄中,年六十一。



 鷿歷覽群書,兼有詞藻,自中興立後,迄於武帝,詔誥表檄,多鷿所為。然率性豪侈,溺於財色,諸弟之間,不能盡雍穆之美,世論以此譏之。鷿素與魏收不協,收既專典國史,鷿恐被惡言,
 乃悅之曰:「昔有班固,今則魏子。」收笑而憾不釋。



 子瞻嗣。



 瞻字彥通,聰明彊學,有文情,善容止,神采嶷然,言不妄發。年十五,刺史高昂召署主簿,清河公嶽辟開府西閣祭酒。崔暹為中尉,啟除御史,以才望見收,非其好也。高祖入朝,還晉陽,被召與北海王晞陪從,俱為諸子賓友。仍為相府中兵參軍,轉主簿。世宗崩,秘未發喪,顯祖命瞻兼相府司馬使鄴。魏孝靜帝以人日登雲龍門,其父鷿侍宴,又敕瞻令近御坐,亦有應詔詩,問邢邵等曰:「
 此詩何如其父?」咸云:「鷿博雅弘麗,瞻氣調清新,並詩人之冠。」宴罷,共嗟賞之,咸云:「今日之宴,併為崔瞻父子。」



 天保初,兼并省吏部郎中。尋丁憂,起為司徒屬。楊愔欲引瞻為中書侍郎。時盧思道直中書省,因問思道曰:「我此日多務,都不見崔瞻文藻,卿與其親通,理當相悉。」思道答曰:「崔瞻文詞之美,實有可稱,但舉世重其風流,所以才華見沒。」愔云:「此言有理。」便奏用之。事既施行。愔又曰:「昔裴瓚晉世為中書郎,神情高邁,每於禁門出入,宿衛
 者肅然動容。崔生堂堂之貌,亦當無愧裴子。」



 皇建元年,除給事黃門侍郎。與趙郡李概為莫逆之友。概將東還,瞻遺之書曰:「仗氣使酒,我之常弊,詆訶指切,在卿尤甚。足下告歸,吾於何聞過也?」瞻患氣,兼性遲重,雖居二省,竟不堪敷奏。加征虜將軍,除清河邑中正。肅宗踐祚,皇太子就傅受業,詔除太子中庶子,征赴晉陽。敕專在東宮,調護講讀,及進退禮度,皆歸委焉。太子納妃斛律氏,敕瞻與鴻臚崔劼撰定婚禮儀注。仍面受別旨曰:「雖有
 舊事,恐未盡善,可好定此儀,以為後式。」



 大寧元年,除衛尉少卿,尋兼散騎常侍,聘陳使主。瞻詞韻溫雅,南人大相欽服,乃言:「常侍前朝通好之日,何意不來?」其見重如此。還除太常少卿,加冠軍將軍,轉尚書吏部郎中。因患急十餘日。舊式,百日不上解官,吏部尚書尉瑾性褊急,以瞻舉指舒緩,曹務繁劇,遂附驛奏聞,因而被代。瞻遂免歸鄉里。天統末,加驃騎大將軍,就拜銀青光祿大夫。武平三年卒,時年五十四。贈使持節、都督濟州軍
 事、大理卿、刺史,謚曰文。



 瞻性簡傲,以才地自矜,所與周旋,皆一時名望。在御史臺,恒於宅中送食,備盡珍羞,別室獨餐,處之自若。有一河東人士姓裴,亦為御史,伺瞻食,便往造焉。瞻不與交言,又不命匕箸。裴坐觀瞻食罷而退。明日,裴自攜匕箸,恣情飲啖。



 瞻方謂裴云:「我初不喚君食,亦不共君語,君遂能不拘小節。昔劉毅在京口,冒請鵝炙,豈亦異於是乎?君定名士。」於是每與之同食。



 鷿昆季仲文,有學尚,魏高陽太守、清河內史。興和中,為
 丞相掾。沙苑之敗,仲文持馬尾以渡河,波中乍沒乍出。高祖望見曰:「崔掾也。」遽遣船赴接。既濟,勞之曰:「卿為親為君,不顧萬死,可謂家之孝子,國之忠臣。」加中軍將軍。天保初,拜散騎常侍、光祿大夫。七年卒,年六十。子偃,武平中,歷太子洗馬、尚書郎。偃弟儦,學識有才思,風調甚高。武平中,琅琊王大司馬中兵參軍。參定五禮,待詔文林館。隋仁壽中,卒於通直散騎常侍。叔仁,魏潁州刺史。子彥武,有識用,朝歌令。隋開皇初,魏州刺史。子侃,魏末
 兼通直常侍,聘梁使。子極,武平初太子僕,卒於武德郡守。子聿,魏東莞太守。子約,司空祭酒。



 鷿族叔景鳳,字鸞叔,鷿五世祖逞玄孫也。景鳳涉學,以醫術知名。魏尚藥典御,天保中譙州刺史。景鳳兄景哲,魏太中大夫、司徒長史。子國,字法峻,幼好學,泛覽經傳,多伎藝,尤工相術。天保初尚藥典御,乾明拜高陽郡太守、太子家令,武平假儀同三司,卒於鴻臚卿。法峻以武平六年從駕在晉陽,嘗語中書侍郎李德林云:「此日看高相王以下文武
 官人相表,俱盡其事,口不忍言。唯弟一人,更應富貴,當在他國,不在本朝,吾亦不及見也。」其精妙如此。



 鷿族子肇師,魏尚書僕射亮之孫也。父士太,諫議大夫。肇師少時疏放,長遂變節,更成謹厚。涉獵經史,頗有文思。襲父爵樂陵男。釋褐開府東閣祭酒,轉司空外兵參軍,遷大司馬府記室參軍。天平初,轉通直侍郎,為尉勞青州使。至齊州界,為土賊崔迦葉等所虜,欲逼與同事。肇師執節不動,諭以禍福,賊遂捨之。乃巡慰青部而還。元象中,
 數以中舍人接梁使。武定中,復兼中正員郎,送梁使徐州。



 還,敕修起居注。尋兼通直散騎常侍,聘梁副使。轉中書舍人。天保初,參定禪代禮儀,封襄城縣男,仍兼中書侍郎。二年卒,時年四十九。



 史臣曰:蘭根早有名行,為時論所稱;長孺才望之美,見重當世。並功參霸迹,位遇通顯,與李元忠、盧文偉蓋義旗之人物歟?魏之要幸附會,崔以門地驕很,雖有周公之美,猶以為累德,況未足喻其高下也。瞻詞韻溫雅,風
 神秀發,亦一時之領袖焉。



 贊曰:崔、魏才望,見重霸初。名教之跡,其猶病諸。彥通尚志,家風有餘。



\end{pinyinscope}