\article{卷六帝紀第六孝昭}

\begin{pinyinscope}

 孝昭皇帝演,字延安,神武皇帝第六子,文宣皇帝之母弟也。幼而英特,早有大成之量,武明皇太后早所愛重。魏元象元年,封常山郡公。及文襄執政,遣中書侍郎李
 同軌就霸府為諸弟師。帝所覽文籍,源其指歸而不好辭彩。每歎云:「雖盟津之師,左驂震而不衄。」以為能。遂篤志讀《漢書》,至《李陵傳》,恒壯其所為焉。聰敏過人,所與遊處,一知其家諱,終身未嘗誤犯。同軌病卒,又命開府長流參軍刁柔代之,性嚴褊,不適誘訓之宜,中被遣出。帝送出閣,慘然斂容,淚數行下,左右莫不歔欷。其敬業重舊也如此。



 天保初,進爵為王。五年,除并省尚書令。帝善斷割,長於文理,省內畏服。



 七年,從文宣還鄴。文宣
 以尚書奏事,多有異同,令帝與朝臣先論定得失,然後敷奏。帝長於政術,剖斷咸盡其理,文宣歎重之。八年,轉司空、錄尚書事。九年,除大司馬,仍錄尚書。時文宣溺於遊宴,帝憂憤表於神色。文宣覺之,謂帝曰:「但令汝在,我何為不縱樂?」帝唯啼泣拜伏,竟無所言。文宣亦大悲,抵盃於地曰:「汝以此嫌我,自今敢進酒者,斬之!」因取所御盃盡皆壞棄。後益沉湎,或入諸貴賤家角力批拉,不限貴賤。唯常山王至,內外肅然。帝又密撰事條,將諫,其友
 王晞以為不可。帝不從,因間極言,遂逢大怒。順成后本魏朝宗室,文宣欲帝離之,陰為帝廣求淑媛,望移其寵。帝雖承旨有納,而情義彌重。帝性頗嚴,尚書郎中剖斷有失,輒加捶楚,令史姦慝,便即考竟。文宣乃立帝於前,以刀環擬脅召被帝罰者,臨以白刃,求帝之短,咸無所陳,方見解釋。自是不許笞箠郎中。後賜帝魏時宮人,醒而忘之,謂帝擅取,遂以刀環亂築,因此致困。皇太后日夜啼泣,文宣不知所為。先是禁友王晞,乃捨之,令侍帝。
 帝月餘漸瘳,不敢復諫。



 及文宣崩,帝居禁中護喪事,幼主即位,乃即朝班。除太傅、錄尚書,朝政皆決於帝。月餘,乃居藩邸,自是詔敕多不關帝。客或言於帝曰:「鷙烏捨巢,必有探卵之患,今日之地,何宜屢出。」乾明元年,從廢帝赴鄴,居于領軍府。時楊愔、燕子獻、可朱渾天和、宋欽道、鄭子默等以帝威望既重,內懼權逼,請以帝為太師、司州牧、錄尚書事;長廣王湛為大司馬、錄并省尚書事,解京畿大都督。帝時以尊親而見猜斥,乃與長廣王期
 獵,謀之於野。三月甲戌,帝初上省,旦發領軍府,大風暴起,壞所御車幔,帝甚惡之。及至省,朝士咸集。坐定,酒數行,執尚書令楊愔、右僕射燕子獻、領軍可朱渾天和、侍中宋欽道等於坐。帝戎服與平原王段韶、平秦王高歸彥、領軍劉洪徽入自雲龍門,於中書省前遇散騎常侍鄭子默,又執之,同斬於御府之內。帝至東閤門,都督成休寧抽刃呵帝。帝令高歸彥喻之,休寧厲聲大呼不從。歸彥既為領軍,素為兵士所服,悉皆弛仗,休寧歎
 息而罷。帝入至昭陽殿,幼主、太皇太后、皇太后並出臨御坐。帝奏愔等罪,求伏專擅之辜。時庭中及兩廊下衛士二千餘人皆被甲待詔,武衛娥永樂武力絕綸,又被文宣重遇,撫刃思效。



 廢帝性吃訥,兼倉卒不知所言。太皇太后又為皇后誓,言帝無異志,唯去逼而已。



 高歸彥敕勞衛士解嚴,永樂乃內刀而泣。帝乃令歸彥引侍衛之士向華林園,以京畿軍入守門閣,斬娥永樂於園。詔以帝為大丞相、都督中外諸軍、錄尚書事,相府佐史
 進位一等。帝尋如晉陽,有詔軍國大政咸諮決焉。



 帝既當大位,知無不為,擇其令典,考綜名實,廢帝恭己以聽政。太皇太后尋下令廢少主,命帝統大業。皇建元年八月壬午,皇帝即位於晉陽宣德殿,大赦,改乾明元年為皇建。詔奉太皇太后還稱皇太后,皇太后稱文宣皇后,宮曰昭信。乙酉,詔自太祖創業已來,諸有佐命功臣子孫絕滅,國統不傳者,有司搜訪近親,以名聞,當量為立後;諸郡國老人各授版職,賜黃帽鳩杖。又
 詔謇正之士並聽進見陳事;軍人戰亡死王事者,以時申聞,當加榮贈;督將、朝士名望素高,位歷通顯,天保以來未蒙追贈者,亦皆錄奏;又以廷尉、中丞,執法所在,繩違按罪,不得舞文弄法;其官奴婢年六十已上免為庶人。戊子,以太傅、長廣王湛為右丞相,以太尉、平陽王淹為太傅,以尚書令、彭城王浟為大司馬。壬辰,詔分遣大使巡省四方,觀察風俗,問人疾苦,考求得失,搜訪賢良。甲午,詔曰:「昔武王克殷,先封兩代,漢、魏、二晉,無廢茲典。
 及元氏統歷,不率舊章。朕纂承大業,思弘古典,但二王三恪,舊說不同,可議定是非,列名條奏。其禮義體式亦仰議之。」又詔國子寺可備立官屬,依舊置生,請習經典,歲時考試。其文襄帝所運石經,宜即施列於學館。



 外州大學亦仰典司勤加督課。丙申,詔九州勳人有重封者,聽分授子弟,以廣骨肉之恩。九月壬申,詔議定三祖樂。冬十一月辛亥,立妃元氏為皇后,世子百年為皇太子。賜天下為父後者爵一級。癸丑,有司奏太祖獻武皇
 帝廟宜奏《武德》之樂,舞《昭烈》之舞;世宗文襄皇帝廟宜奏《文德》之樂,舞《宣政》之舞;顯祖文宣皇帝廟宜奏《文正》之樂,舞《光大》之舞。詔曰可。庚申,詔以故太師尉景、故太師竇泰、故太師太原王婁昭、故太宰章武王厙狄干、故太尉段榮、故太師萬俟普、故司徒蔡俊、故太師高乾、故司徒莫多婁貸文、故太保劉貴、故太保封祖裔、故廣州刺史王懷十二人配饗太祖廟庭,故太師清河王岳、故太宰安德王韓軌、故太宰扶風王可朱渾道元、故太師高
 昂、故大司馬劉豐、故太師萬俟受洛干、故太尉慕容紹宗七人配饗世宗廟庭,故太尉河東王潘相樂、故司空薛修義、故太傅破六韓常三人配饗顯祖廟庭。是月,帝親戎北討庫莫奚,出長城,虜奔遁,分兵致討,大獲牛馬,括總入晉陽宮。十二月丙午,車駕至晉陽。



 二年春正月辛亥,祀圓丘。壬子,禘於太廟。癸丑,詔降罪人各有差。二月丁丑,詔內外執事之官從五品已上及三府主簿錄事參軍、諸王文學、侍御史、廷尉三官、尚書
 郎中、中書舍人,每二年之內各舉一人。冬十月丙子,以尚書令、彭城王浟為太保,長樂王尉粲為太尉。己酉,野雉栖于前殿之庭。十一月甲辰,詔曰:「朕嬰此暴疾,奄忽無逮。今嗣子沖眇,未閑政術,社稷業重,理歸上德。右丞相、長廣王湛研機測化,體道居宗,人雄之望,海內瞻仰,同胞共氣,家國所憑,可遣尚書左僕射、趙郡王睿喻旨,徵王統茲大寶。其喪紀之禮一同漢文,三十六日悉從公除,山陵施用,務從儉約。」先是帝不豫而無闕聽覽,是
 月,崩於晉陽宮,時年二十七。大寧元年閏十二月癸卯,梓宮還鄴,上謚曰孝昭皇帝。庚午,葬於文靖陵。



 帝聰敏有識度,深沉能斷,不可窺測。身長八尺,腰帶十圍,儀望風表,迥然獨秀。自居臺省,留心政術,閑明簿領,吏所不逮。及正位宸居,彌所剋勵。輕徭薄賦,勤恤人隱。內無私寵,外收人物,雖后父位亦特進無別。日昃臨朝,務知人之善惡,每訪問左右,冀獲直言。曾問舍人裴澤在外議論得失。澤率爾對曰:「陛下聰明至公,自可遠侔古昔,而
 有識之士,咸言傷細,帝王之度,頗為未弘。」帝笑曰:「誠如卿言。朕初臨萬機,慮不周悉,故致爾耳。此事安可久行,恐後又嫌疏漏。」澤因被寵遇。其樂聞過也如此。趙郡王睿與厙狄顯安侍坐,帝曰:「須拔我同堂弟,顯安我親姑子,今序家人禮,除君臣之敬,可言我之不逮。」顯安曰:「陛下多妄言。」曰:「若何?」對曰:「陛下昔見文宣以馬鞭撻人,常以為非,而今行之,非妄言耶?」帝握其手謝之。又使直言。對曰:「陛下太細,天子乃更似吏。」帝曰:「朕甚知之,然無法
 來久,將整之以至無為耳。」又問王晞,晞答如顯安,皆從容受納。性至孝,太后不豫,出居南宮,帝行不正履,容色貶悴,衣不解帶,殆將四旬。殿去南宮五百餘步,雞鳴而去,辰時方還,來去徒行,不乘輿輦。太后所苦小增,便即寢伏閤外,食飲藥物盡皆躬親。太后常心痛不自堪忍,帝立侍帷前,以爪掐手心,血流出袖。友愛諸弟,無君臣之隔。雄斷有謀,于時國富兵強,將雪神武遺恨,意在頓駕平陽,為進取之策。遠圖不遂,惜哉!初,帝與濟南約不
 相害。及輿駕在晉陽,武成鎮鄴,望氣者云鄴城有天子氣。帝常恐濟南復興,乃密行鳩毒,濟南不從,乃扼而殺之。後頗愧悔。初苦內熱,頻進湯散。時有尚書令史姓趙,於鄴見文宣從楊愔、燕子獻等西行,言相與復仇。帝在晉陽宮,與毛夫人亦見焉。遂漸危篤。備禳厭之事,或煮油四灑,或持炬燒逐。諸厲方出殿梁,騎棟上,歌呼自若,了無懼容。時有天狗下,乃於其所講武以厭之。有兔驚馬,帝墜而絕肋。太后視疾,問濟南所在者三,帝不對。太
 后怒曰:「殺之耶?不用吾言,死其宜矣!」臨終之際,唯扶服床枕,叩頭求哀。遣使詔追長廣王入纂大統,手書云:「宜將吾妻子置一好處,勿學前人也。」



 論曰:神武平定四方,威權在己,遷鄴之後,雖主器有人,號令所加,政皆自出。文宣因循鴻業,內外葉從,自朝及野,群心屬望,東魏之地,舉國樂推,曾未期月,遂登宸極。始則存心政事,風化肅然,數年之間,朝野安出。其後縱酒肆欲,事極猖狂,昏邪殘暴,近代未有,饗國不永,實由
 斯疾。濟南繼業,大革其弊,風教粲然,搢紳稱幸。股肱輔弼,雖懷厥誠,既不能贊弘道德,和睦親懿,又不能遠慮防身,深謀衛主,應斷不斷,自取其咎。臣既誅夷,君壽廢辱,皆任非其器之所致爾。孝昭早居臺閣,故事通明,人吏之間,無所不委。文宣崩後,大革前弊。及臨尊極,留心更深,時人服其明而識其細也。情好稽古,率由禮度,將封先代之胤,且敦學校之風,徵召英賢,文武畢集。于時周氏朝政移於宰臣,主將相猜,不無危殆。乃眷關右,實
 懷兼並之志,經謀宏遠,實當代之明主,而降年不永,其故何哉?



 豈幽顯之間,實有報復,將齊之基宇止在於斯,帝欲大之,天不許也?



\end{pinyinscope}