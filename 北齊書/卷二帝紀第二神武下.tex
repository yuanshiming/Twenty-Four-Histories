\article{卷二帝紀第二神武下}

\begin{pinyinscope}

 天平元年正月壬辰,神武西伐費也頭虜紇豆陵伊利於河西,滅之,遷其部於河東。



 二月,永寧寺九層浮圖災。既而人有從東萊至,云及海上人咸見之於海中,俄而霧起乃滅。說者以為天意若曰:永寧見災,魏不寧矣;飛入東海,渤海應矣。



 魏帝既有異圖,時侍中封隆之與孫騰私言,隆之喪
 妻,魏帝欲妻以妹。騰亦未之信,心害隆之,泄其言於斛斯椿。椿以白魏帝。又孫騰帶仗入省,擅殺御史。並亡來奔。稱魏帝撾舍人梁續於前,光祿少卿元子乾攘臂擊之,謂騰曰:「語爾高王,元家兒拳正如此。」領軍婁昭辭疾歸晉陽。魏帝於是以斛斯椿兼領軍,分置督將及河南、關西諸刺史。華山王鷙在徐州,神武使邸珍奪其管籥。建州刺史韓賢、濟州刺史蔡俊皆神武同義,魏帝忌之。故省建州以去賢,使御史中尉綦俊察俊罪,以開府賈顯智為濟州。俊拒
 之,魏帝
 逾怒。



 五月下詔,云將征句吳,發河南諸州兵,增宿衛,守河橋。六月丁巳,魏帝密詔神武曰:「宇文黑獺自平破秦、隴,多求非分,脫有變詐,事資經略。但表啟
 未
 全背戾,進討事涉抃抃,遂召群臣,議其可否。僉言假稱南伐,內外戒嚴,一則防黑獺不虞,二則可威吳楚。」時魏帝將伐神武,神武部署
 將帥,慮疑,故有此詔。



 神武乃表曰:「荊州綰接蠻左,密邇畿服,關隴恃遠,將有逆圖。臣今潛勒兵馬三萬,擬從河東而渡;又遣恒州刺史厙狄干、瀛州刺史郭瓊、汾州刺史斛律金、前武衛將軍彭樂擬兵四萬,從其來違津渡;遣領軍將軍婁昭、相州刺史竇泰、前瀛州刺史堯雄、并州刺史高隆之擬兵五萬,以討荊州;遣冀州刺史尉景、前冀州刺史高敖曹、濟州刺史蔡俊、前侍中封隆之擬山東兵七萬、突騎五萬,以征江左。皆約所部,伏聽處分。」
 魏帝知覺其變,乃出神武表,命群官議之,欲止神武諸軍。神武乃集在州僚佐,令其博議,還以表聞。仍以信誓自明忠款曰:「臣為嬖佞所間,陛下一旦賜疑,今猖狂之罪,爾朱時討。臣若不盡誠竭節,敢負陛下,則使身受天殃,子孫殄絕。陛下若垂信赤心,使干戈不動,佞臣一二人願斟量廢出。」辛未,帝復錄在京文武議意以答神武,使舍人溫子昇草敕。子昇逡巡未敢作,帝據胡床,拔劍作色。子昇乃為敕曰:前持心血,遠以示王,深冀彼此共
 相體悉,而不良之徒坐生間貳。近孫騰倉卒向彼,致使聞者疑有異謀,故遣御史中尉綦俊具申朕懷。今得王啟,言誓懇惻,反覆思之,猶所未解。以朕眇身,遇王武略,不勞尺刃,坐為天子,所謂生我者父母,貴我者高王。今若無事背王。規相攻討,則使身及子孫,還如王誓。皇天后土,實聞此言。近慮宇文為亂,賀拔勝應之,故纂嚴欲與王俱為聲援。宇文今日使者相望,觀其所為,更無異迹。賀拔在南,開拓邊境,為國立功,念無可責。君若欲分
 討,何以為辭?東南不賓,為日已久,先朝已來,置之度外。今天下戶口減半,未宜窮兵極武。朕既闇昧,不知佞人是誰,可列其姓名,令朕知也。如聞厙狄乾語王云:「本欲取懦弱者為主,王無事立此長君,使其不可駕御,今但作十五日行,自可廢之,更立餘者。」如此議論,自是王間勳人,豈出佞臣之口?去歲封隆之背叛,今年孫騰逃走,不罪不送,誰不怪王!騰既為禍始,曾無愧懼,王若事君盡誠,何不斬送二首。王雖啟圖西去,而四道俱進,或欲
 南度洛陽,或欲東臨江左,言之者猶應自怪,聞之者寧能不疑?王若守誠不貳,晏然居北,在此雖有百萬之眾,終無圖彼之心。王脫信邪棄義,舉旗南指,縱無匹馬隻輪,猶欲奮空拳而爭死。朕本寡德,王已立之,百姓無知,或謂實可。若為他所圖,則彰朕之惡,假令還為王殺,幽辱齏粉,了無遺恨。何者?王既以德見推,以義見舉,一朝背德舍義,便是過有所歸。



 本望君臣一體,若合符契,不圖今日,分疏到此。古語云:「越人射我,笑而道之;吾兄射
 我,泣而道之。」朕既親王,情如兄弟,所以投筆拊膺,不覺歔欷。



 初,神武自京師將北,以為洛陽久經喪亂,王氣衰盡,雖有山河之固,土地褊狹,不如鄴,請遷都。魏帝曰:「高祖定鼎河洛,為永永之基,經營制度,至世宗乃畢。王既功在社稷,宜遵太和舊事。」神武奉詔,至是復謀焉。遣三千騎鎮建興,益河東及濟州兵,於白溝虜船不聽向洛,諸州和糴粟運入鄴城。魏帝又敕神武曰:「王若厭伏人情,杜絕物議,唯有歸河東之兵,罷建興之戍,送相州之
 粟,追濟州之軍,令蔡俊受代,使邸珍出徐,止戈散馬,各事家業。脫須糧廩,別遣轉輸,則讒人結舌,疑悔不生。王高枕太原,朕垂拱京洛,終不舉足渡河,以干戈相指。王若馬首南向,問鼎輕重,朕雖無武,欲止不能,必為社稷宗廟出萬死之策。決在於王,非朕能定,為山止簣,相為惜之。」魏帝時以任祥為兼尚書左僕射,加開府,祥棄官走至河北,據郡待神武。魏帝乃敕文武官北來者任去留,下詔罪狀神武,為北伐經營。神武亦勒馬宣告曰:「孤
 遇爾朱擅權,舉大義於四海,奉戴主上,義貫幽明。橫為斛斯椿讒構,以誠節為逆首。昔趙鞅興晉陽之甲,誅君側惡人,今者南邁,誅椿而已。」以高昂為前鋒,曰:「若用司空言,豈有今日之舉!」司馬子如答神武曰:「本欲立小者,正為此耳。」



 魏帝徵兵關右,召賀拔勝赴行在所,遣大行臺長孫承業、大都督潁川王斌之、斛斯椿共鎮武牢,汝陽王暹鎮石濟,行臺長孫子彥帥前恒農太守元洪略鎮陜,賈顯智率豫州刺史斛斯元壽伐蔡俊。神武使竇
 泰與左廂大都督莫多婁貸文逆顯智,韓賢逆暹。元壽軍降。泰、貸文與顯智遇於長壽津,顯智陰約降,引軍退。軍司元玄覺之,馳還。請益師。魏帝遣大都督侯幾紹赴之。戰於滑臺東,顯智以軍降,紹死之。



 七月,魏帝躬率大眾屯河橋。神武至河北十餘里,再遣口申誠款,魏帝不報。神武乃引軍渡河。魏帝問計於群臣,或云南依賀拔勝,或云西就關中,或云守洛口死戰。



 未決。而元斌之與斛斯椿爭權不睦,斌之棄椿徑還,紿帝云:「神武兵至。」即
 日,魏帝遜於長安。己酉,神武入洛陽,停於永寧寺。



 八月甲寅,召集百官,謂曰:「為臣奉主,匡救危亂,若處不諫爭,出不陪隨,緩則耽寵爭榮,急便逃竄,臣節安在?」遂收開府儀同三司叱列延慶、兼尚書左僕射辛雄、兼吏部尚書崔孝芬、都官尚書劉廞、兼度支尚書楊機、散騎常侍元士弼並殺之,誅其貳也。士弼籍沒家口。神武以萬機不可曠廢,乃與百僚議以清河王亶為大司馬,居尚書下舍而承制決事焉。王稱警蹕,神武醜之。神武尋至恒
 農,遂西剋潼關,執毛洪賓。進軍長城,龍門都督薛崇禮降。神武退舍河東,命行臺尚書長史薛瑜守潼關,大都督庫狄溫守封陵。於蒲津西岸築城,守華州,以薛紹宗為刺史,高昂行豫州事。神武自發晉陽,至此凡四十啟,魏帝皆不答。九月庚寅,神武還於洛陽,乃遣僧道榮奉表關中,又不答。乃集百僚四門耆老,議所推立。以為自孝昌喪亂,國統中絕,神主靡依,昭穆失序。永安以孝文為伯考,永熙遷孝明於夾室,業喪祚短,職此之由。遂議
 立清河王世子善見。議定,白清河王。王曰:「天子無父,茍使兒立,不惜餘生。」乃立之,是為孝靜帝。魏於是始分為二。



 神武以孝武既西,恐逼崤、陜,洛陽復在河外,接近梁境,如向晉陽,形勢不能相接,乃議遷鄴,護軍祖瑩贊焉。詔下三日,車駕便發,戶四十萬狼狽就道。神武留洛陽部分,事畢還晉陽。自是軍國政務,皆歸相府。先是童謠曰:「可憐青雀子,飛來鄴城裏,羽翮垂欲成,化作鸚鵡子。」好事者竊言,雀子謂魏帝清河王子,鸚鵡謂神武也。



 初,
 孝昌中,山胡劉螽升自稱天子,年號神嘉,居雲陽谷,西土歲被其寇,謂之胡荒。二年正月,西魏渭州刺史可朱渾道元擁眾內屬,神武迎納之。壬戌,神武襲擊劉螽升,大破之。己巳,魏帝褒詔,以神武為相國,假黃鉞,劍履上殿,入朝不趨。神武固辭。三月,神武欲以女妻螽升太子,候其不設備,辛酉,潛師襲之。



 其北部王斬螽升首以送。其眾復立其子南海王,神武進擊之,又獲南海王及其弟西海王、北海王、皇后公卿已下四百餘人,胡、魏五萬
 戶。壬申,神武朝于鄴。四月,神武請給遷人廩各有差。九月甲寅,神武以州郡縣官多乖法,請出使問人疾苦。



 三年正月甲子,神武帥厙狄乾等萬騎襲西魏夏州,身不火食,四日而至。縛槊為梯,夜入其城,禽其刺史費也頭斛拔俄彌突,因而用之。留都督張瓊以鎮守,遷其部落五千戶以歸。西魏靈州刺史曹泥與其婿涼州刺史劉豐遣使請內屬。周文圍泥,水灌其城,不沒者四尺。神武命阿至羅發騎三萬徑度靈州,繞出西軍後,獲馬五十
 匹,西師乃退。神武率騎迎泥、豐生,拔其遺戶五千以歸,復泥官爵。魏帝詔加神武九錫,固讓乃止。二月,神武令阿至羅逼西魏秦州刺史建忠王萬俟普撥,神武以眾應之。六月甲午,普撥與其子太宰受洛干、豳州刺史叱干寶樂、右衛將軍破六韓常及督將三百餘人擁部來降。八月丁亥,神武請均斗尺,班於天下。九月辛亥,汾州胡王迢觸、曹貳龍聚眾反,署立百官,年號平都。神武討平之。十二月丁丑,神武自晉陽西討,遣兼僕射行臺汝
 陽王暹、司徒高昂等趣上洛,大都督竇泰入自潼關。



 四年正月癸丑,竇泰軍敗自殺。神武次蒲津,以冰薄不得赴救,乃班師。高昂攻剋上洛。二月乙酉,神武以并、肆、汾、建、晉、東雍、南汾、泰、陜九州霜旱,人饑流散,請所在開倉賑給。六月壬申,神武如天池,獲瑞石,隱起成文曰「六王三川」。十月壬辰,神武西討,自蒲津濟,眾二十萬。周文軍於沙苑。神武以地阨少卻,西人鼓噪而進,軍大亂,棄器甲十有八萬,神武跨橐駝,候船以歸。



 元象元年三月
 辛酉,神武固請解丞相,魏帝許之。四月庚寅,神武朝于鄴,壬辰,還晉陽。請開酒禁,并賑恤宿衛武官。七月壬午,行臺侯景、司徒高昂圍西魏將獨孤信於金墉,西魏帝及周文並來赴救。大都督厙狄乾帥諸將前驅,神武總眾繼進。八月辛卯,戰於河陰,大破西魏軍,俘獲數萬。司徒高昂、大都督李猛、宋顯死之。西師之敗,獨孤信先入關,周文留其都督長孫子彥守金墉,遂燒營以遁。神武遣兵追奔,至崤,不及而還。初,神武知西師來侵,自晉陽
 帥眾馳赴,至孟津,未濟,而軍有勝負。既而神武渡河,子彥亦棄城走,神武遂毀金墉而還。十一月庚午,神武朝於京師。十二月壬辰,還晉陽。



 興和元年七月丁丑,魏帝進神武為相國、錄尚書事,固讓乃止。十一月乙丑,神武以新宮成,朝於鄴。魏帝與神武燕射,神武降階稱賀,又辭渤海王及都督中外諸軍事,詔不許。十二月戊戌,神武還晉陽。



 二年十二月,阿至羅別部遣使請降。神武帥眾迎之,出武州塞,不見,大獵而還。



 三年五月,神武巡北
 境,使使與蠕蠕通和。



 四年五月辛巳,神武朝鄴,請令百官每月面敷政事,明揚側陋,納諫屏邪,親理獄訟,褒黜勤怠;牧守有愆,節級相坐;椒掖之內,進御以序;後園鷹犬,悉皆棄之。六月甲辰,神武還晉陽。九月,神武西征。十月己亥,圍西魏儀同三司王思政於玉壁城,欲以致敵,西師不敢出。十一月癸未,神武以大雪士卒多死,乃班師。



 武定元年二月壬申,北豫州刺史高慎據武牢西叛。三月壬辰,周文率眾援高慎,圍河橋南城。戊申,神武大
 敗之於芒山,擒西魏督將已下四百餘人,俘斬六萬計。



 是時軍士有盜殺驢者,軍令應死,神武弗殺,將至并州決之。明日復戰,奔西軍,告神武所在。西師盡銳來攻,眾潰,神武失馬,赫連陽順下馬以授神武,與蒼頭馮文洛扶上俱走,從者步騎六七人。追騎至,親信都督尉興慶曰:「王去矣,興慶腰邊百箭,足殺百人。」神武勉之曰:「事濟,以爾為懷州,若死,則用爾子。」興慶曰:「兒小,願用兄。」許之。興慶斗,矢盡而死。西魏太師賀拔勝以十三騎逐神武,
 河州刺史劉豐射中其二。勝槊將中神武,段孝先橫射勝馬殪,遂免。豫、洛二州平。神武使劉豐追奔,拓地至弘農而還。七月,神武貽周文書,責以殺孝武之罪。八月辛未,魏帝詔神武為相國、錄尚書事、大行臺,餘如故,固辭乃止。是月,神武命於肆州北山築城,西自馬陵戍,東至士隥,四十日罷。十二月己卯,神武朝京師,庚辰,還晉陽。二年三月癸巳,神武巡行冀、定二州,因朝京師。以冬春亢旱,請蠲懸責,賑窮乏,宥死罪以下。又請授老人板
 職各有差。四月丙辰,神武還晉陽。十一月,神武討山胡,破平之,俘獲一萬餘戶口,分配諸州。



 三年正月甲午,開府儀同三司爾朱文暢、開府司馬任胄、都督鄭仲禮、中府主簿李世林、前開府參軍房子遠等謀賊神武,因十五日夜打簇,懷刃而入,其黨薛季孝以告,並伏誅。丁未,神武請於并州置晉陽宮,以處配口。三月乙未,神武朝鄴,丙午,還晉陽。十月丁卯,神武上言,幽、安、定三州北接奚、蠕蠕,請於險要修立城戍以防之,躬自臨覆,莫不嚴
 固。乙未,神武請釋芒山俘桎梏,配以民間寡婦。



 四年八月癸巳,神武將西伐,自鄴會兵於晉陽。殿中將軍曹魏祖曰:「不可。



 今八月西方王,以死氣逆生氣,為客不利,主人則可。兵果行,傷大將軍。」神武不從。自東、西魏構兵,鄴下每先有黃黑蟻陣鬥,占者以為黃者東魏戎衣色,黑者西魏戎衣色,人間以此候勝負。是時黃蟻盡死。九月,神武圍玉壁以挑西師,不敢應。西魏晉州刺史韋孝寬守玉壁,城中出鐵面,神武使元盜射之,每中其目。用李
 業興孤虛術,萃其北。北,天險也。乃起土山,鑿十道,又於東面鑿二十一道以攻之。城中無水,汲於汾。神武使移汾,一夜而畢。孝寬奪據土山,頓軍五旬,城不拔,死者七萬人,聚為一塚。有星墜於神武營,眾驢並鳴,士皆讋懼。神武有疾。



 十一月庚子,輿疾班師。庚戌,遣太原公洋鎮鄴。辛亥,徵世子澄至晉陽。有惡烏集亭樹,世子使斛律光射殺之。己卯,神武以無功,表解都督中外諸軍事,魏帝優詔許之。是時西魏言神武中弩,神武聞之,乃勉坐
 見諸貴,使斛律金作《敕勒歌》,神武自和之,哀感流涕。



 侯景素輕世子,嘗謂司馬子如曰:「王在,吾不敢有異,王無,吾不能與鮮卑小兒共事。」子如掩其口。至是,世子為神武書召景。景先與神武約:得書,書背微點,乃來。書至,無點,景不至。又聞神武疾,遂擁兵自固。神武謂世子曰:「我雖疾,爾面更有餘憂色,何也?」世子未對。又問曰:「豈非憂侯景叛耶?」



 曰:「然。神武曰:「景專制河南十四年矣,常有飛揚跋扈志,顧我能養,豈為汝駕御也!今四方未定,勿
 遽發哀。厙狄乾鮮卑老公,斛律金敕勒老公,並性遒直,終不負汝。可朱渾道元、劉豐生遠來投我,必無異心。賀拔焉過兒樸實無罪過。潘樂本作道人,心和厚,汝兄弟當得其力。韓軌少戇,宜寬借之。彭樂心腹難得,宜防護之。少堪敵侯景者唯有慕容紹宗,我故不貴之,留以與汝,宜深加殊禮,委以經略。」



 五年正月朔,日蝕,神武曰:「日蝕其為我耶,死亦何恨。」丙午,陳啟於魏帝。是日,崩於晉陽,時年五十二,祕不發喪。六月壬午,魏帝於東堂舉
 哀,三日,制緦衰。詔凶禮依漢大將軍霍光、東平王蒼故事;贈假黃鉞、使持節、相國、都督中外諸軍事、齊王璽紱,轀輬車、黃屋、左纛、前後羽葆、鼓吹、輕車、介士,兼備九錫殊禮,謚獻武王。八月甲申,葬於鄴西北漳水之西,魏帝臨送於紫陌。天保初,追崇為獻武帝,廟號太祖,陵曰義平。天統元年,改謚神武皇帝,廟號高祖。



 神武性深密高岸,終日儼然,人不能測。機權之際,變化若神。至於軍國大略,獨運懷抱,文武將吏,罕有預之。統馭軍眾,法令嚴
 肅,臨敵制勝,策出無方。聽斷昭察,不可欺犯。知人好士,全護勳舊。性周給,每有文教,常殷勤款悉,指事論心,不尚綺靡。擢人授任,在於得才,茍其所堪,乃至拔於廝養,有虛聲無實者,稀見任用。諸將出討,奉行方略,罔不克捷,違失指畫,多致奔亡。雅尚儉素,刀劍鞍勒無金玉之飾。少能劇飲,自當大任,不過三爵。居家如官。仁恕愛士。始,范陽盧景裕以明經稱,魯郡韓毅以工書顯,咸以謀逆見擒,並蒙恩置之第館,教授諸子。其文武之士盡
 節所事,見執獲而不罪者甚多。故遐邇歸心,皆思效力。至南威梁國,北懷蠕蠕,吐谷渾、阿至羅咸所招納,獲其力用,規略遠矣。



\end{pinyinscope}