\article{卷二十六列傳第十八}

\begin{pinyinscope}

 薛琡敬顯俊平鑒薛琡,字曇珍,河南人。其先代人,本姓叱干氏。父彪子,魏
 徐州刺史。琡形貌魁偉,少以幹用稱。為典客令,每引客見,儀望甚美。魏帝召而謂之曰:「卿風度峻整,姿貌秀異,後當升進,何以處官?」琡曰:「宗廟之禮,不敢不敬,朝廷之事,不敢不忠,自此以外,非庸臣所及。」正光中,行洛陽令,部內肅然。有犯法者,未加拷掠,直以辭理窮覈,多得其情。於是豪猾畏威,事務簡靜。時以久旱,京師見囚悉召集華林,理問冤滯,洛陽繫獄,唯有三人。魏孝明嘉之,賜縑百匹。



 遷吏部,尚書崔亮奏立停年之格,不簡人才,專
 問勞舊。琡上書,言:「黎元之命,繫於長吏,若得其人,則蘇息有地,任非其器,為患更深。若使選曹唯取年勞,不簡賢否,便義均行鴈,次若貫魚,執簿呼名,一吏足矣,數人而用,何謂銓衡?請不依此。」書奏不報。後因引見,復進諫曰:「共治天下,本屬百官。是以漢朝常令三公大臣舉賢良方正、有道直言之士,以為長吏,監撫黎元。自晉末以來,此風遂替。今四方初定,務在養民。臣請依漢氏更立四科,令三公貴臣各薦時賢,以補郡縣,明立條格,防其
 阿黨之端。」詔下公卿議之,事亦寢。



 元天穆討邢杲也,以琡為行臺尚書。時元顥已據酂城。天穆集文武議其所先。



 議者咸以杲眾甚盛,宜先經略。琡以為邢杲聚眾無名,雖彊猶賊;元顥皇室暱親,來稱義舉,此恐難測。杲鼠盜狗竊,非有遠志,宜先討顥。天穆以群情所欲,遂先討杲。杲降軍還,顥遂入洛。天穆謂琡曰:「不用君言,乃至於此。」



 天平初,高祖引為丞相長史。琡宿有能名,深被禮遇,軍國之事,多所聞知。



 琡亦推誠盡節,屢進忠讜。高祖大
 舉西伐,將度蒲津。琡諫曰:「西賊連年饑饉,無可食啖,故冒死來入陜州,欲取倉粟。今高司徒已圍陜城,粟不得出。但置兵諸道,勿與野戰,比及來年麥秋,人民盡應餓死,寶炬、黑獺,自然歸降。願王無渡河也。」侯景亦曰:「今者之舉,兵眾極大,萬一不捷,卒難收斂。不如分為二軍,相繼而進,前軍若勝,後軍合力,前軍若敗,後軍承之。」高祖皆不納,遂有沙苑之敗。累遷尚書僕射,卒。臨終敕其子斂以時服,踰月便葬,不聽干求贈官。自制喪車,不加彫
 飾,但用麻為流蘇,繩用網絡而已。明器等物並不令置。



 琡久在省闥,閑明簿領,當官剖斷,敏速如流。然天性險忌,情義不篤,外似方格,內實浮動。受納貨賄,曲法舞文,深情刻薄,多所傷害,士民畏惡之。魏東平王元匡妾張氏淫逸放恣,琡初與姦通,後納以為婦。惑其讒言,逐前妻于氏,不認其子,家內怨忿,競相告列,深為世所譏鄙。贈青州刺史。



 敬顯俊,字孝英,平陽人。少英俠有節操,交結豪傑。為羽
 林監。高祖臨晉州,俊因使謁見,與語說之,乃啟為別駕。及義舉,以俊為行臺倉部郎中。從攻鄴,令俊督造土山。城拔,又從平西胡。轉都官尚書,與諸將征討,累有功。又從高祖平寇難,破周文帝。敗侯景,平壽春,定淮南。又略地三江口,多築城戍。累除兗州刺史,卒。



 平鑒,字明達,燕郡薊人。父勝,安州刺史。鑒少聰敏,頗有志力。受學於徐遵明,不為章句,雖崇儒業,而有豪俠氣。孝昌末,盜賊蜂起,見天下將亂,乃之洛陽,與慕容儼騎
 馬為友。鑒性巧,夜則胡畫,以供衣食。謂其宗親曰:「運有污隆,亂極則治。并州戎馬之地,爾朱王命世之雄,杖義建旗,奉辭問罪,勞忠竭力,今也其時。」遂相率奔爾朱榮於晉陽,因陳靜亂安民之策。榮大奇之,即署參軍前鋒,從平鞏、密,每陣先登。除撫軍、襄州刺史。



 高祖起義信都,鑒自歸。高祖謂鑒曰:「日者皇綱中弛,公已早竭忠誠。今爾朱披猖,又能去逆從善。搖落之時,方識松筠。」即啟授征西。懷州刺史。



 鑒奏請於州西故軹道築城以防遏西
 寇,朝廷從之。尋而西魏來攻。是時新築之城,糧仗未集,舊來乏水,眾情大懼。南門內有一井,隨汲即竭。鑒乃具衣冠俯井而祝,至旦有井泉湧溢,合城取之。魏師敗還,以功進位開府儀同三司。



 時和士開以佞幸勢傾朝列,令人求鑒愛妾劉氏,鑒即送之。仍謂人曰:「老公失阿劉,與死何異。要自為身作計,不得不然。」由是除齊州刺史。鑒歷牧八州,再臨懷州,所在為吏所思,立碑頌德。入為都官尚書令。



\end{pinyinscope}