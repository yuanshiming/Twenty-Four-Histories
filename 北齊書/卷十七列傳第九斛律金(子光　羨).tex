\article{卷十七列傳第九斛律金(子光 羨)}

\begin{pinyinscope}

 斛律金,字阿六敦,朔州敕勒部人也。高祖倍侯利,以壯勇有名塞表,道武時率戶內附,賜爵孟都公。祖幡地斤,殿中尚書。父大那瑰,光祿大夫、第一領民酋長。天平
 中,金貴,贈司空公。



 金性敦直,善騎射,行兵用匈奴法,望塵識馬步多少,嗅地知軍度遠近。初為軍主,與懷朔鎮將楊鈞送茹茹主阿那瑰還北。瑰見金射獵,深歎其工。後瑰入寇高陸,金拒擊破之。正光末,破六韓拔陵構逆,金擁眾屬焉,陵假金王號。金度陵終敗滅,乃統所部萬戶詣雲州請降,即授第二領民酋長。稍引南出黃瓜堆,為杜洛周所破,部眾分散,金與兄平二人脫身歸爾朱榮。榮表金為別將,累遷都督。孝莊立,賜爵阜城縣男,加
 寧朔將軍、屯騎校尉。從破葛榮、元顯,頻有戰功,加鎮南大將軍。



 及爾朱兆等逆亂,高祖密懷匡復之計,金與婁昭、厙狄乾等贊成大謀,仍從舉義。高祖南攻鄴,留金守信都,領恒、雲、燕、朔、顯、蔚六州大都督,委以後事,別討李脩,破之,加右光祿大夫。會高祖於鄴,仍從平晉陽,追滅爾朱兆。太昌初,以金為汾州刺史、當州大都督,進爵為侯。從高祖破紇豆陵於河西。天平初,遷鄴,使金領步騎三萬鎮風陵以備西寇,軍罷,還晉陽。從高祖戰於沙苑,
 不利班師,因此東雍諸城復為西軍所據,遣金與尉景、厙狄乾等討復之。元象中,周文帝復大舉向河陽。高祖率眾討之,使金徑往太州,為掎角之勢。金到晉州,以軍退不行,仍與行臺薛修義共圍喬山之寇。俄而高祖至,仍共討平之,因從高祖攻下南絳、邵郡等數城。武定初,北豫州刺史高仲密據城西叛,周文帝入寇洛陽。高祖使金統劉豐、步大汗薩等步騎數萬守河陽城以拒之。高祖到,仍從破密。軍還,除大司馬,改封石城郡公,邑一
 千戶,轉第一領民酋長。三年,高祖出軍襲山胡,分為二道。以金為南道軍司,由黃櫨嶺出。高祖自出北道,度赤谼嶺,會金於烏突戍,合擊破之。



 軍還,出為冀州刺史。四年,詔金率眾從烏蘇道會高祖於晉州,仍從攻玉壁。軍還,高祖使金總督大眾,從歸晉陽。



 世宗嗣事,侯景據潁川降於西魏,詔遣金帥潘樂、薛孤延等固守河陽以備。西魏使其大都督李景和、若干寶領馬步數萬,欲從新城赴援侯景。金率眾停廣武以要之,景和等聞而退走。
 還為肆州刺史,仍率所部於宜陽築楊志、百家、呼延三戍,置守備而還。侯景之走南豫,西魏儀同三司王思政入據潁川。世宗遣高岳、慕容紹宗、劉豐等率眾圍之。復詔金督彭樂、可朱渾道元等出屯河陽,斷其奔救之路。又詔金率眾會攻潁川。事平,復使金率眾從崿阪送米宜陽。西魏九曲戍將馬紹隆據險要鬥,金破之。以功別封安平縣男。



 顯祖受禪,封咸陽郡王,刺史如故。其年冬,朝晉陽宮。金病,帝幸其宅臨視,賜以醫藥,中使不絕。病
 愈還州。三年,就除太師。帝征奚賊,金從帝行。軍還,帝幸肆州,與金宴射而去。四年,解州,以太師還晉陽。車駕復幸其第,六宮及諸王盡從,置酒作樂,極夜方罷。帝忻甚,詔金第二子豐樂為武衛大將軍,因謂金曰:「公元勳佐命,父子忠誠,朕當結以婚姻,永為蕃衛。」仍詔金孫武都尚義寧公主。



 成禮之日,帝從皇太后幸金宅,皇后、太子及諸王等皆從,其見親待如此。



 後以茹茹為突厥所破,種落分散,慮其犯塞,驚撓邊民,乃詔金率騎二萬屯白
 道以備之。而虜帥豆婆吐久備將三千餘戶密欲西過,候騎還告,金勒所部追擊,盡俘其眾。茹茹但缽將舉國西徙,金獲其候騎送之,并表陳虜可擊取之勢。顯祖於是率眾與金共討之於吐賴,獲二萬餘戶而還。進位右丞相,食齊州乾,遷左丞相。



 肅宗踐阼,納其孫女為皇太子妃。又詔金朝見,聽步挽車至階。世祖登極,禮遇彌重,又納其孫女為太子妃。金長子光大將軍,次子羨及孫武都並開府儀同三司,出鎮方岳,其餘子孫皆封侯貴
 達。一門一皇后、二太子妃、三公主,尊寵之盛,當時莫比。金嘗謂光曰:「我雖不讀書,聞古來外戚梁冀等無不傾滅。女若有寵,諸貴人妒;女若無寵,天子嫌之。我家直以立勳抱忠致富貴,豈可藉女也?」辭不獲免,常以為憂。天統三年薨,年八十。世祖舉哀西堂,後主又舉哀於晉陽宮。贈假黃鉞、使持節、都督朔定冀并瀛青齊滄幽肆晉汾十二州諸軍事、相國、太尉公、錄尚書、朔州刺史,酋長、王如故,贈錢百萬,謚曰武。子光嗣。



 光,字明月,少工騎射,以武藝知名。魏末,從金西征,周文帝長史莫孝暉時在行間,光馳馬射中之,因擒於陣,光時年十七。高祖嘉之,即擢為都督。世宗為世子,引為親信都督,稍遷征虜將軍,累加衛將軍。武定五年,封永樂縣子。嘗從世宗於洹橋校獵,見一大鳥,雲表飛颺,光引弓射之,正中其頸。此鳥形如車輪,旋轉而下,至地,乃大雕也。世宗取而觀之,深壯異焉。丞相屬邢子高見而歎曰:「此射雕手也。」當時傳號落雕都督。尋兼左衛將軍,進
 爵為伯。



 齊受禪,加開府儀同三司,別封西安縣子。天保三年,從征出塞,光先驅破敵,多斬首虜,并獲雜畜。還,除晉州刺史。東有周天柱、新安、牛頭三戍,招引亡叛,屢為寇竊。七年,光率步騎五千襲破之,又大破周儀同王敬俊等,獲口五百餘人,雜畜千餘頭而還。九年,又率眾取周絳川、白馬、澮交、翼城等四戍。除朔州刺史。



 十年,除特進、開府儀同三司。二月,率騎一萬討周開府曹回公,斬之。柏谷城主儀同薛禹生棄城奔遁,遂取文侯鎮,立戍
 置柵而還。乾明元年,除并州刺史。皇建元年,進爵巨鹿郡公。時樂陵王百年為皇太子,肅宗以光世載醇謹,兼著勳王室,納其長女為太子妃。大寧元年,除尚書右僕射,食中山郡幹。二年,除太子太保。



 河清二年四月,光率步騎二萬築勳掌城於軹關西,仍築長城二百里,置十三戍。三年正月,周遣將達奚成興等來寇平陽,詔光率步騎三萬禦之,興等聞而退走。光逐北,遂入其境,獲二千餘口而還。其年三月,遷司徒。四月,率騎北討突厥,獲馬
 千餘匹。是年冬,周武帝遣其柱國大司馬尉遲迥、齊國公宇文憲,柱國庸國公可叱雄等,眾稱十萬,寇洛陽。光率騎五萬馳往赴擊,戰於邙山,迥等大敗。光親射雄,殺之,斬捕首虜三千餘級,迥、憲僅而獲免,盡收其甲兵輜重,仍以死者積為京觀。



 世祖幸洛陽,策勳班賞,遷太尉,又封冠軍縣公。先是世祖命納光第二女為太子妃,天統元年,拜為皇后。其年,光轉大將軍。三年六月,父喪去官,其月,詔起光及其弟羨並復前任。秋,除太保,襲爵咸
 陽王,並襲第一領民酋長,別封武德郡公,徙食趙州乾,遷太傅。



 十二月,周遣將圍洛陽,壅絕糧道。武平元年正月,詔光率步騎三萬討之。軍次定隴,周將張掖公宇文桀、中州刺史梁士彥、開府司水大夫梁景興等又屯鹿盧交道,光擐甲執銳,身先士卒,鋒刃纔交,桀眾大潰,斬首二千餘級。直到宜陽,與周齊國公宇文憲、申國公手翕跋顯敬相對十旬。光置築統關、豐化二城,以通宜陽之路。軍還,行次安鄴,憲等眾號五萬,仍躡軍後,光從騎擊
 之,憲眾大潰,虜其開府宇文英、都督越勤世良、韓延等,又斬首三百餘級。憲仍令桀及其大將軍中部公梁洛都與景興、士彥等步騎三萬於鹿盧交塞斷要路。光與韓貴孫、呼延族、王顯等合擊,大破之,斬景興,獲馬千匹。詔加右丞相、并州刺史。其冬,光又率步騎五萬於玉壁築華谷、龍門二城,與憲、顯敬等相持,憲等不敢動。光乃進圍定陽,仍築南汾城。置州以逼之,夷夏萬餘戶並來內附。



 二年,率眾築平隴、衛壁、統戎等鎮戍十有三所。周
 柱國枹罕公普屯威、柱國韋孝寬等步騎萬餘,來逼平隴,與光戰於汾水之北,光大破之,俘斬千計。又封中山郡公,增邑一千戶。軍還,詔復令率步騎五萬出平陽道,攻姚襄、白亭城戍,皆克之,獲其城主儀同、大都督等九人,捕虜數千人。又別封長樂郡公。是月,周遣其柱國紇干廣略圍宜陽。光率步騎五萬赴之,大戰於城下,乃取周建安等四戍,捕虜千餘人而還。軍未至鄴,敕令便放兵散。光以為軍人多有勛功,未得慰勞,若即便散,恩澤
 不施,乃密通表請使宣旨,軍仍且進。朝廷發使遲留,軍還,將至紫陌,光仍駐營待使。帝聞光軍營已逼,心甚惡之,急令舍人追光入見,然後宣勞散兵。



 拜光左丞相,又別封清河郡公。



 光入,常在朝堂垂簾而坐。祖珽不知,乘馬過其前。光怒,謂人曰:「此人乃敢爾!」後珽在內省,言聲高慢,光適過,聞之,又怒。珽知光忿,而賂光從奴而問之曰:「相王瞋孝徵耶?」曰:「自公用事,相王每夜抱膝歎曰:『盲人入,國必破矣!』」穆提婆求娶光庶女,不許。帝賜提婆晉
 陽之田,光言於朝曰:「此田神武帝以來常種禾,飼馬數千匹,以擬寇難,今賜提婆,無乃闕軍務也?」由是祖、穆積怨。



 周將軍韋孝寬忌光英勇,乃作謠言,令間諜漏其文於鄴,曰「百升飛上天,明月照長安」,又曰「高山不推自崩,槲樹不扶自豎」。祖珽因續之曰:「盲眼老公背上下大斧,饒舌老母不得語。」令小兒歌之於路。提婆聞之,以告其母令萱。萱以饒舌斥己也,盲老公謂珽也,遂相與協謀,以謠言啟帝曰:「斛律累世大將,明月聲震關西,豐樂威
 行突厥,女為皇后,男尚公主,謠言甚可畏也。」帝以問韓長鸞,鸞以為不可,事寢。祖珽又見帝請間,唯何洪珍在側。帝曰:「前得公啟,即欲施行,長鸞以為無此理。」珽未對,洪珍進曰:「若本無意則可,既有此意而不決行,萬一泄露如何?」帝曰:「洪珍言是也。」猶豫未決。會丞相府佐封士讓密啟云:「光前西討還,敕令放兵散,光令軍逼帝京,將行不軌,事不果而止。家藏弩甲,奴僮千數,每遣使豐樂、武都處,陰謀往來。若不早圖,恐事不可測。」啟云「軍逼帝
 京」,會帝前所疑意,謂何洪珍云:「人心亦大聖,我前疑其欲反,果然。」帝性至怯懦,恐即變發,令洪珍馳召祖珽告之。又恐追光不從命。珽因云:「正爾召之,恐疑不肯入。宜遣使賜其一駿馬,語云『明日將往東山游觀,王可乘此馬同行』,光必來奉謝,因引入執之。」帝如其言。頃之,光至,引入涼風堂,劉桃枝自後拉而殺之,時年五十八。於是下詔稱光謀反,今已伏法,其餘家口並不須問。尋而發詔,盡滅其族。



 光性少言剛急,嚴於御下,治兵督眾,唯仗
 威刑。版築之役,鞭撻人士,頗稱其暴。自結髮從戎,未嘗失律,深為鄰敵所懾憚。罪既不彰,一旦屠滅,朝野痛惜之。周武帝聞光死,大喜,赦其境內。後入鄴,追贈上柱國、崇國公。指詔書曰:「此人若在,朕豈能至鄴!」



 光有四子。長子武都,歷位特進、太子太保、開府儀同三司、梁兗二州刺史。



 所在並無政績,唯事聚斂,侵漁百姓。光死,遣使於州斬之。次須達,中護軍、開府儀同三司,先光卒。次世雄,開府儀同三司。次恒伽,假儀同三司。並賜死。光小子鐘,年數
 歲,獲免。周朝襲封崇國公。隋開皇中卒於驃騎將軍。



 羨,字豐樂,少有機警,尤善射藝,高祖見而稱之。世宗擢為開府參軍事。遷征虜將軍、中散大夫,加安西將軍,進封大夏縣子,除通州刺史。顯祖受禮,進號征西,別封顯親縣伯。河清三年,轉使持節,都督幽、安、平、南、北營、東燕六州諸軍事,幽州刺史。其年秋,突厥眾十餘萬來寇州境,羨總率諸將禦之。突厥望見軍威甚整,遂不敢戰,即遣使求款。慮其有詐,且喻之曰:「爾輩此行,本非朝貢,見
 機始變,未是宿心。若有實誠,宜速歸巢穴,別遣使來。」於是退走。天統元年夏五月,突厥木汗遣使請朝獻,羨始以聞,自是朝貢歲時不絕,羨有力焉。詔加行臺僕射。羨以北虜屢犯邊,須備不虞,自庫堆戍東拒於海,隨山屈曲二千餘里,其間二百里中凡有險要,或斬山築城,或斷谷起障,并置立戍邏五十餘所。又導高梁水北合易京,東會於潞,因以灌田。邊儲歲積,轉漕用省,公私獲利焉。其年六月,丁父憂去官,與兄光並被起復任,還鎮燕薊。三
 年,加位特進。四年,遷行臺尚書令,別封高城縣侯。武平元年,加驃騎大將軍。時光子武都為兗州刺史。羨歷事數帝,以謹直見推,雖極榮寵,不自矜尚,至是以合門貴盛,深以為憂。乃上書推讓,乞解所職,優詔不許。其年秋,進爵荊山郡王。



 三年七月,光誅,敕使中領軍賀拔伏恩等十餘人驛捕之。遣領軍大將軍鮮于桃枝、洛州行臺僕射獨孤永業便發定州騎卒續進,仍以永業代羨。伏恩等既至,門者白使人衷甲馬汗,宜閉城門。羨曰:「
 敕使豈可疑拒?」出見之,伏恩把手,遂執之,死於長史廳事。臨終歎曰:「富貴如此,女為皇后,公主滿家,常使三百兵,何得不敗!」及其五子世達、世遷、世辨、世酋、伏護,餘年十五已下者宥之。羨未誅前,忽令其在州諸子自伏護以下五六人,鎖頸乘驢出城,合家皆泣送之至門,日晚而歸。吏民莫不驚異。行燕郡守馬嗣明,醫術之士,為羨所欽愛,乃竊問之,答曰:「須有禳厭。」數日而有此變。



 羨及光並少工騎射,其父每日令其出畋,還即較所獲禽
 獸。光所獲或少,必麗龜達腋。羨雖獲多,非要害之所。光常蒙賞,羨或被捶撻。人問其故,金答云:「明月必背上著箭,豐樂隨處即下手,其數雖多,去兄遠矣。」聞者咸服其言。



 金兄平,便弓馬,有乾用。魏景明中,釋褐殿中將軍,遷襄威將軍。正光末,六鎮擾亂,隸大將軍尉賓北討。軍敗,為賊所虜。後走奔其弟金於雲州,進號龍驤將軍。與金擁眾南出,至黃瓜堆,為杜洛周所破,部落離散。及歸爾朱榮,待之甚厚,以平襲父爵第一領民酋長。高祖起義,以
 都督從。稍遷平北將軍、顯州刺史,加鎮南將軍,封固安縣伯。尋進為侯,行肆州刺史。周文帝遣其右將軍李小光據梁州,平以偏師討擒之。出為燕州刺史。入兼左衛將軍,領眾一萬討北徐賊,破之,除濟州刺史。侯景度江,詔平為大都督,率青州刺史敬顯俊、左衛將軍厙狄伏連等略定壽陽、宿預三十餘城。事罷還州,加開府,進位驃騎大將軍,進爵為公。顯祖受禪,別封羨陽侯。行兗州刺史,以黷貨除名。後除開府儀同三司。廢帝即位,拜特
 進,食滄州樂陵郡幹。皇建初,封定陽郡公,拜護軍。後為青州刺史,卒。贈太尉。



 史臣曰:斛律金以高祖撥亂之始,翼成王業,忠款之至,成此大功,故能終享遐年,位高百辟。觀其盈滿之戒,動之微也,纔及後嗣,遂至誅夷,雖為威權之重,蓋符道家所忌。光以上將之子,有沈毅之姿,戰術兵權,暗同韜略,臨敵制勝,變化無方。自關、河分隔,年將四紀。以高祖霸王之期,屬宇文草創之日,出軍薄伐,屢挫兵鋒。而大寧
 以還,東鄰浸弱,關西前收巴蜀,又殄江陵,葉建瓴而用武,成并吞之壯氣。斛律治軍誓眾,式遏邊鄙,戰則前無完陣,攻則罕有全城,齊氏必致拘原之師,秦人無復啟關之策。而世亂才勝,詐以震主之威;主暗時艱,自毀藩籬之固。昔李牧之為趙將也,北翦胡寇,西卻秦軍,郭開譖之,牧死趙滅。其議誅光者,豈秦之反間歟,何同術而同亡也!內令諸將解體,外為強鄰報仇。嗚呼!後之君子,可為深戒。



 贊曰:赳赳咸陽,邦家之光。明月忠壯,仍世將相。聲振關右,勢高時望。迫此威名,易興讒謗。始自工言,終斯交喪。



\end{pinyinscope}