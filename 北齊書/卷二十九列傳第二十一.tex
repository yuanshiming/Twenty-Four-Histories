\article{卷二十九列傳第二十一}

\begin{pinyinscope}

 李渾子湛渾弟繪族子公緒李璵弟瑾族弟曉鄭述祖子元德李渾,字季初,趙郡柏人人也。曾祖靈,魏巨鹿公。父遵,魏
 冀州征東府司馬,京兆王愉冀州起逆,遇害。渾以父死王事,除給事中。時四方多難,乃謝病,求為青州征東府司馬。與河間邢邵、北海王昕俱奉老母、攜妻子同赴青、齊。未幾而爾朱榮入洛,衣冠殲盡。論者以為知機。永安初,除散騎常侍。



 普泰中,崔社客反於海岱,攻圍青州。詔渾為征東將軍、都官尚書,行臺赴援。



 而社客宿將多謀,諸城各自保,固壁清野。時議有異同。渾曰:「社客賊之根本,圍城復踰晦朔。烏合之眾,易可崩離。若簡練驍勇,銜枚
 夜襲,徑趣營下,出其不意,咄嗟之間,便可擒殄。如社客就擒,則諸郡可傳檄而定。何意冒熱攻城,疲損軍士。」諸將遲疑,渾乃決行。未明,達城下,賊徒驚散,生擒社客,斬首送洛。



 海隅清定。



 後除光祿大夫,兼常侍,聘使至梁。梁武謂之曰:「伯陽之後,久而彌盛,趙李人物,今實居多。常侍曾經將領,今復充使,文武不墜,良屬斯人。」使還,為東郡太守,以贓徵還。世宗使武士提以入,渾抗言曰:「將軍今日猶自禮賢耶?」



 世宗笑而捨之。



 天保初,除太子少
 保。時邢邵為少師,場愔為少傅,論者為榮。以參禪代儀注,賜爵涇陽縣男。刪定《麟趾格》。尋除海州刺史。土人反,共攻州城。城中多石,無井,常食海水。賊絕其路。城內先有一池,時旱久涸,一朝天雨,泉流涌溢。賊以為神,應時駭散。渾督勵將士,捕斬渠帥。渾妾郭氏在州干政納貨,坐免官。卒。



 子湛,字處元。涉獵文史,有家風。為太子舍人,兼常侍,聘陳使副。襲爵涇陽縣男。渾與弟繪、緯俱為聘梁使主,湛
 又為使副,是以趙郡人士,目為四使之門。



 繪,字敬文。年六歲,便自願入學,家人偶以年俗忌,約而弗許。伺其伯姊筆牘之間,而輒竊用,未幾遂通《急就章》。內外異之,以為非常兒也。及長,儀貌端偉,神情朗雋。河間邢晏,即繪舅也,與繪清言,歎其高遠。每稱曰:「若披雲霧,如對珠玉,宅相之寄,良在此甥。」齊王蕭寶夤引為主簿記室,專管表檄,待以賓友之禮。司徒高邕辟為從事中郎,徵至洛。時敕侍中西河王、秘書監常景選儒學十
 人緝撰五禮,繪與太原王又同掌軍禮。魏靜帝於顯陽殿講《孝經》、《禮記》,繪與從弟騫、裴伯茂、魏收、盧元明等俱為錄議。素長筆札,尤能傳受,緝綴詞議,簡舉可觀。天平初,世宗用為丞相司馬。每罷朝,文武總集,對揚王庭,常令繪先發言端,為群僚之首。音辭辯正,風儀都雅,聽者悚然。



 武定初,兼常侍,為聘梁使主。梁武帝問繪:「高相今在何處?」繪曰:「今在晉陽,肅遏邊寇。」梁武曰:「黑獺若為形容?高相作何經略?」繪曰:「黑獺遊魂關右,人神厭毒,連歲
 凶災,百姓懷土。丞相奇略不世,畜銳觀釁,攻昧取亡,勢必不遠。」梁武曰:「如卿言極佳。」與梁人汎言氏族,袁狎曰:「未若我本出自黃帝,姓在十四之限。」繪曰:「兄所出雖遠,當共車千秋分一字耳。」一坐大笑。前後行人,皆通啟求市,繪獨守清尚,梁人重其廉潔。



 使還,拜平南將軍、高陽內史。郡境舊有猛獸,民常患之。繪欲修檻,遂因鬥俱死。咸以為化感所致,皆請申上。繪曰:「猛獸因鬥而斃,自是偶然,貪此為功,人將窺我。」竟不聽。高祖東巡郡國,在
 瀛州城西駐馬久立,使慰之曰:「孤在晉,知山東守唯卿一人用意。及入境觀風,信如所聞。但善始令終,將位至不次。」河間守崔謀恃其弟暹勢,從繪乞麋角鴒羽。繪答書曰:「鴒有六翮,飛則沖天,麋有四足,走便入海。下官膚體疏懶,手足遲鈍,不能逐飛追走,遠事佞人。」是時世宗使暹選司徒長史,暹薦繪,既而不果,咸謂由此書。天保初,為司徒右長史。繪質性方重,未嘗趨事權勢,以此久而屈沉。卒。贈南青州刺史,謚曰景。



 公緒,字穆叔,渾族兄籍之子。性聰敏,博通經傳。魏末冀州司馬,屬疾去官。



 後以侍御史徵,不至,卒。



 公緒沉冥樂道,不關世務,故誓心不仕。尤善陰陽圖緯之學。嘗語人云:「吾每觀齊之分野,福德不多,國家世祚,終於四七。」及齊亡之歲,上距天保之元二十八年矣。公緒潛居自待,雅好著書,撰《典言》十卷,又撰《質疑》五卷,《喪服章句》一卷,《古今略記》二十卷,《玄子》五卷,《趙語》十三卷,並行於世。



 李璵,字道璠,隴西成紀人,涼武昭王暠之五世孫。父韶,
 並有重名於魏代。



 璵溫雅有識量。釋褐太尉行參軍,累遷司徒右長史。及遷都於鄴,留於後,監掌府藏,及撤運宮廟材木,以明乾見稱。累遷驃騎大將軍、東徐州刺史。解州還,遂稱老疾,不求仕。齊受禪,進璵兼前將軍,導從於圓丘行禮。璵意不願策名兩朝,雖以宿舊被徵,過事即絕朝請。天保四年卒。



 子詮、韞、誦。韞無行。誦以女妻穆提婆子懷廆,超遷臨漳令、儀同三司。韞與陸令萱女弟私通,令萱奏授太子舍人。



 弟瑾,字道瑜,名在魏書。才識
 之美,見稱當代。瑾六子,彥之、倩之、壽之、禮之、行之、凝之,並有器望。行之與兄弟深相友愛,又風素夷簡,為士友所稱。



 范陽盧思道是其舅子,嘗贈詩云:「水衡稱逸人,潘楊有世親,形骸預冠蓋,心思出風塵。」時人以為實錄。璵從弟曉,字仁略。魏太尉虔子。學涉有思理。釋褐員外侍郎。爾朱榮之害朝士,將行,曉衣冠為鼠所噬,遂不成行,得免河陰之難。及遷都鄴,曉便寓居清河,託從母兄崔悛宅。給良田三十頃,曉遂築室安居,訓勖子姪,無復宦
 情。武定末,以世道方泰,乃入都從仕。除頓丘守,卒。



 鄭述祖,字恭文,滎陽開封人。祖羲,魏中書令。父道昭,魏秘書監。述祖少聰敏,好屬文,有風檢,為先達所稱譽。釋褐司空行參軍。天保初,累遷太子少師、儀同三司、兗州刺史。時穆子容為巡省使,歎曰:「古人有言:『聞伯夷之風,貪夫廉,懦夫有立。』今於鄭兗州見之矣。」



 初,述祖父為光州,於城南小山起齋亭,刻石為記。述祖時年九歲。及為刺史,往尋舊迹,得一破石,有銘云:「中岳先生鄭道昭之
 白雲堂。」述祖對之嗚咽,悲動群僚。有人入市盜布,其父怒曰:「何忍欺人君!」執之以歸首,述祖特原之。



 自是之後,境內無盜。人歌之曰:「大鄭公,小鄭公,相去五十載,風教猶尚同。」



 述祖能鼓琴,自造《龍吟十弄》,云嘗夢人彈琴,寤而寫得。當時以為絕妙。



 所在好為山池,松竹交植。盛饌以待賓客,將迎不倦。未貴時,在鄉單馬出行,忽有騎者數百,見述祖皆下馬,曰「公在此」,行列而拜。述祖顧問從人,皆不見,心甚異之。未幾備征,終歷顯位。及病篤,乃自
 言之。且曰:「吾今老矣,一生富貴足矣,以清白之名遺子孫,死無所恨。」遂卒於州。述祖女為趙郡王睿妃。述祖常坐受王拜,命坐,王乃坐。妃薨後,王更娶鄭道蔭女。王坐受道蔭拜,王命坐,乃敢坐。王謂道蔭曰:「鄭尚書風德如此,又貴重宿舊,君不得譬之。」子元德,多藝術,官至琅邪守。



 元德從父弟元禮,字文規。少好學,愛文藻,有名望。世宗引為館客,歷太子舍人。崔昂妻,即元禮之姊也,魏收又
 昂之妹夫。昂嘗持元禮數篇詩示盧思道,乃謂思道云:「看元禮比來詩詠,亦當不減魏收?」答云:「未覺元禮賢於魏收,但知妹夫疏於婦弟。」元禮入周,卒於始州別駕。



\end{pinyinscope}