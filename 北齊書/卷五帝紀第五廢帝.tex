\article{卷五帝紀第五廢帝}

\begin{pinyinscope}

 廢帝殷,字正道,文宣帝之長子也,母曰李皇后。天保元年,立為皇太子,時年六歲。性敏慧。初學反語,於「跡」字下注云自反。時侍者未達其故,太子曰:「跡字足傍亦為跡,豈非自反耶?」常宴北宮,獨令河間王勿入。左右問其故,太子曰:「世宗遇賊處,河間王復何宜在此。」文宣每言太子
 得漢家性質,不似我,欲廢之,立太原王。初詔國子博士李寶鼎傅之,寶鼎卒,復詔國子博士邢峙侍講。



 太子雖富
 於春秋,而溫裕開朗,有人君之度,貫綜經業,省覽時政,甚有美名。七年冬,文宣召朝臣文學者及禮學官於宮宴會,令以經義相質,親自臨聽。太子手筆措問,在坐莫不嘆美。九年,文宣在晉陽,太子監國,集諸儒講《孝經》。令楊愔傳旨,謂國子助教許散愁曰:「先生在世何以自資?」對曰:「散愁自少以來,不登孌童之床,不入季女之室,服膺簡策,不知老之將至。平生素懷,若斯而已。」



 太子曰:「顏子縮屋稱貞,柳下
 嫗而不亂,未若此翁白首不娶者也。」乃賚絹百匹。



 後文宣登金鳳臺,召太子
 使手刃囚。太子惻然有難色,再三不斷其首。文宣怒,親以馬鞭撞太子三下,由是氣悸語吃,精神時復昏擾。



 十年十月,文宣崩。癸卯,太子即帝位於晉陽宣德殿,大赦,內外百官普加泛級,亡官失爵,聽復資品。庚戌,尊皇太后為太皇太后,皇后為皇太后。詔九州軍人七十已上授以板職,武官年六十已上及癃病不堪驅使者,並皆放免。土木營造金銅鐵諸雜作工,一切停罷。十一月乙卯,以右丞相、咸陽王斛律金為左丞相,以錄尚書事、常山王演為太傅,以司徒、長廣王湛為太尉,以司空段韶為司徒,以平陽王淹為司空,高陽王湜為尚書左僕射,河間王孝琬為司州牧,侍中燕子獻為右僕射。



 戊午,
 分命使者巡省四方,求政得失,省察風俗,問人
 疾苦。十二月戊戌,改封上黨王紹仁為漁陽王,廣陽王紹義為範陽王,長樂王紹廉為隴西王。是歲,周武成元年。



 乾明元年庚辰,春正月癸丑朔,改元。己末,詔寬徭賦。癸亥,高陽王湜薨。



 是月,車駕至自晉陽。二月己亥,以太傅、
 常山王演為太師、錄尚書事,以太尉、長廣王湛為大司馬、並省錄尚書事,以尚書左僕射、平秦王歸彥為司空,趙郡王睿為尚書左僕射。詔諸元良口配沒宮內及賜人者,並放免。甲辰,帝幸芳林園,親錄囚徒,死罪以下降免各有差。乙巳,太師、常山王演矯詔誅尚書
 令楊愔、尚書右僕射燕子獻、領軍大將軍可朱渾天和、侍中宋欽道、散騎常侍鄭子默。戊申,以常山
 王演為大丞相、都督中外諸軍、錄尚書事,以大司馬、長廣王湛為太傅、京畿大都督,以司徒段韶為大將軍,以前司空、平陽王淹為太尉,以司空、平秦王歸彥為司徒,彭城王浟為尚書令。又以高麗王世子湯為使持節、
 領東夷校尉、遼東郡公、高麗王。是月,王琳為陳所敗,蕭莊自拔至和州。三月甲寅,詔軍國事皆申晉陽,稟大丞相常山王規算。壬申,封文襄第二字孝珩為廣寧王,第三子長恭為蘭陵王。夏四月癸亥,詔河南、定、冀、趙、瀛、滄、南膠、光、青九州,往因螽水,頗傷時稼,遣使分途贍恤。是月,周明帝崩。五月壬子,以開府儀同三司劉洪徽為尚書右僕射。秋八月壬午,太皇太后令廢帝為濟南王,令食一郡,
 以大丞相、常山王演入纂大統。是日,王居別宮。皇建二年九月,殂於晉陽,年十七。



 帝聰慧夙成,寬厚仁智,天保間雅有令名。及承大位,楊愔、燕子獻、宋欽道等同輔。以常山王地親望重,內外畏服,加以文宣初崩之日,太后本欲立之,故愔等並懷猜忌。常山王憂悵,乃白太后誅其黨,時平秦王歸彥亦預謀焉。皇建二年秋,天文告變,歸彥慮有後害,仍白孝昭,以王當咎。乃遣歸彥馳驛
 至晉陽宮
 殺之。王薨後,孝昭不豫,見文宣為祟。孝昭深惡之,厭勝術備設而無益也。薨三旬而孝昭崩。大寧二年,葬於武寧之西北,謚
 閔悼王。初,文宣命邢邵制帝名殷,字正道,帝從而尤之曰:「殷家弟及,『正』字一止,吾身後兒不得也。」邵懼,請改焉。



 文宣不許曰:「天也。」因謂孝昭帝曰:「奪但奪,慎勿殺也。」



\end{pinyinscope}