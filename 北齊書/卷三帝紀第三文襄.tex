\article{卷三帝紀第三文襄}

\begin{pinyinscope}

 世宗文襄皇帝,諱澄,字子惠,神武長子也,母曰婁太后。生而岐嶷,神武異之。魏中興元年,立為渤海王世子。就杜詢講學,敏悟過人,詢甚歎服。二年,加侍中、開府儀同
 三司,尚孝靜帝妹馮翊長公主,時年十二,神情俊爽,便若成人。



 神武試問以時事得失,辨析無不中理,自是軍國籌策皆預之。



 天平元年,加使持節、尚書令、大行臺、并州刺史。三年,入輔朝政,加領左右、京畿大都督。時人雖聞器識,猶以少年期之,而機略嚴明,事無凝滯,於是朝野振肅。元象元年,攝吏部尚書。魏自崔亮以後。選人常以年勞為制,文襄乃釐改前式,銓擢唯在得人。又沙汰尚書郎,妙選人地以充之。至於才名之士,咸被薦擢,
 假有未居顯位者,皆致之門下,以為賓客,每山園游燕,必見招攜,執射賦詩,各盡其所長,以為娛適。興和二年,加大將軍,領中書監,仍攝吏部尚書。自正光已後,天下多事,在任群官,廉潔者寡。文襄乃奏吏部郎崔暹為御史中尉,糾劾權豪,無所縱捨,於是風俗更始,私枉路絕。乃榜於街衢,具論經國政術,仍開直言之路,有論事上書苦言切至者,皆優容之。



 武定四年十一月,神武西討,不豫,班師,文襄馳赴軍所,侍衛還晉陽。五年正月丙午,
 神武崩,秘不發喪。辛亥,司徒侯景據河南反,潁州刺史司馬世雲以城應之。景誘執豫州刺史高元成、襄州刺史李密、廣州刺史暴顯等。遣司空韓軌率眾討之。夏四月壬申,文襄朝於鄴。六月己巳,韓軌等自潁州班師。丁醜,文襄還晉陽,乃發喪,告喻文武,陳神武遺志。七月戊戌,魏帝詔以文襄為使持節、大丞相、都督中外諸軍、錄尚書事、大行臺、渤海王。文襄啟辭位,願停王爵。壬寅,魏帝詔太原公洋攝理軍國,遣中使敦喻。八月戊辰,文襄
 啟申神武遺令,請減國邑分封將督,各有差。辛未,朝鄴,固辭丞相。魏帝詔曰:「既朝野攸憑,安危所繫,不得令遂本懷,須有權奪,可復前大將軍,餘如故。」



 議者咸云侯景猶有北望之心,但信命不至耳。又景將蔡遵道北歸,稱景有悔過之心。王以為信然,謂可誘而致,乃遺景書曰:先王與司徒契闊夷險,孤子相依,偏所眷屬,義貫終始,情存歲寒。待為國士者乃立漆身之節,饋以一餐者便致扶輪之效,況其重於此乎?常以故舊之義,欲將子孫
 相託,方為秦晉之匹,共成劉范之親。況聞負杖行歌,便以狼顧反噬,不蹈忠臣之路,便陷叛人之地。力不足以自彊,勢不足以自保,率烏合之眾,為累卵之危。



 西取救於宇文,南請援於蕭氏,以狐疑之心,為首鼠之事。入秦則秦人不容,歸吳則吳人不信。當是不逞之人,曲為無端之說,遂懷市虎之疑,乃致投杼之惑。比來舉止,事已可見,人相疑誤,想自覺知。闔門大小,悉在司寇,意謂李氏未滅,猶言少卿可反。孤子無狀招禍,丁天酷罰,但禮
 由權奪,志在忘私,聊遣偏裨,前驅致討,南兗、揚州,應時剋復。即欲乘機,席卷縣瓠,屬以炎暑,欲為後圖,且令還師,待時更舉。今寒膠向折,白露將團,方憑國靈,龔行天罰。器械精新,士馬彊盛,內外感恩,上下戮力,三令五申,可赴湯火。使旗鼓相望,埃塵相接,勢如沃雪,事等注熒。夫明者去危就安,智者轉禍為福,寧人負我,不我負人,當開從善之途,使有改迷之路。若能卷甲來朝,垂橐還闕者,即當授豫州,必使終君身世。



 所部文武更不追攝,
 進得保其祿位,退則不喪功名。今王思政等皆孤軍偏將,遠來深入,然其性命在君掌握,脫能刺之,想有餘力。節相加授,永保疆埸。君門眷屬,可以無患,寵妻愛子,亦送相還,仍為通家,共成親好。君今不能東封函谷,南面稱孤,受制於人,威名頓盡。得地不欲自守,聚眾不以為彊,空使身有背叛之名,家有惡逆之禍,覆宗絕嗣,自貽伊戚。戴天履地,能無愧乎!孤子今日不應遣此,但見蔡遵道云:「司徒本無西歸之心,深有悔過之意」,未知此語
 為虛為實。吉凶之理,想自圖之。



 景報書曰:僕鄉曲布衣,本乖藝用,出身為國,綿歷二紀,犯危履難,豈避風霜,遂得富貴當年,榮華身世。一旦舉旗掞,援鼓枹,北面相抗者,何哉?實以畏懼危亡,恐招禍害故耳。往年之暮,尊王遘疾,神不祐善,祈禱莫瘳。遂使嬖倖弄權,心腹離貳,妻子在宅,無事見圍。及迴歸長社,希自陳狀,簡書未遣,斧鉞已臨。既旌旗相對,咫尺不遠,飛書每奏,冀申鄙情。而群帥恃雄,眇然弗顧,連戰推鋒,專欲屠滅。掘圍堰水,僅
 存三版,舉目相看,命縣漏刻。不忍死亡,出戰城下,拘秦送地,豈樂為之?禽獸惡死,人倫好生,僕實不辜,桓、莊何罪。且尊王平昔見與比肩,戮力同心,共獎帝室,雖復權勢參差,寒暑小異,丞相司徒,鴈行而已。福祿官榮,自是天爵,勞而後授,理不相干,欲求吞炭,何其謬也!然竊人之財,猶謂之盜。祿去公室,抑謂不取。今魏德雖衰,天命未改,拜恩私第,何足關言。賜嗤不能東封函谷,受制於人,當似教僕賢祭仲而褒季氏。無主之國,在禮未聞,動
 而不法,將何以訓?竊以分財養幼,事歸令終,舍宅存孤,誰云隙末?復言僕眾不足以自彊,身危如累卵。然億兆夷人,卒降十亂,紂之百克,終自無後,潁川之戰,即是殷監。輕重由人,非鼎在德,茍能忠信,雖弱必彊,殷憂啟聖,處危何苦。況今梁道邕熙,招攜以禮,被我虎文,縻之好爵,方欲苑五岳而池四海,掃氛穢以拯黎元。東羈甌越,西道汧隴,吳越悍勁,帶甲千群,秦兵冀馬,控弦十萬,大風一振,枯幹必摧,凝霜暫落,秋帶自殞,此而為弱,誰足
 稱雄?又見誣兩端,受疑二國,斟酌物情,一何太甚!昔陳平背楚,歸漢則彊,百里出虞,入秦斯霸。蓋昏明由主,用舍在人,奉禮而行,神其吐邪!書稱士馬精新,剋日齊舉,誇張形勢,必欲相滅。切以寒膠白露,節候乃同,秋風揚塵,馬首何異。徒知北方之力爭,未識西南之合從,茍欲狥意於前途,不覺坑阱在其側。去危就安,今歸正朔;轉禍為福,已脫網羅。彼當嗤僕之過迷,此亦笑君之晦昧。今引二邦,揚旌北討,熊虎齊奮,克復中原,荊、襄、廣、潁,已
 屬關右,項城、縣瓠,亦奉江南,幸自取之,何勞見援。然權變非一,理有萬途,為君計者,莫若割地兩和,三分鼎峙,燕、衛、趙、晉,足相俸祿,齊、曹、宋、魯,悉歸大梁。使僕得輸力南朝,北敦姻好,束帛自行,戎車不駕,僕立當世之功,君卒父禰之業,各保疆壘,聽享歲時,百姓乂寧,四人安堵。孰若驅農夫於壟畝,抗勁敵於三方,避干戈於首尾,當鋒鏑於心腹。縱太公為將,不能獲存,歸之高明,何以克濟。來書曰,妻子老幼悉在司寇,以此見要,庶其可反。當
 是見疑褊心,未識大趣。昔王陵附漢,母在不歸;太上囚楚,乞羹自若。矧伊妻子,而可介意。脫謂誅之有益,欲止不能,殺之無損,復加坑戮,家累在君,何關僕也。遵道所說,頗亦非虛,故重陳辭,更論款曲。昔與盟主,事等琴瑟,讒人間之,翻為仇敵。撫弦搦矢,不覺傷懷,裂帛還書,其何能述。



 王尋覽書,問誰為作。或曰:「其行臺郎王偉。」王曰:「偉才如此,何因不使我知?」王欲間景於梁,又與景書而謬其辭,云本使景陽叛,欲與圖西,西人知之,故景更與
 圖南為事。漏其書於梁,梁人亦不之信。



 壬申,東魏主與王獵於鄴東,馳逐如飛。監衛都督烏那羅受工伐從後呼曰:「天子莫走馬,大將軍怒。」王嘗侍飲,舉大觴曰:「臣澄勸陛下酒。」東魏主不悅曰:「自古無不亡之國,朕亦何用如此生!」王怒曰:「朕!朕!狗腳朕!」使崔季舒毆之三拳,奮衣而出。尋遣季舒入謝。東魏主賜季舒彩,季舒未敢即受,啟之於王,王使取一段。東魏主以四百匹與之,曰:「亦一段耳。」東魏主不堪憂辱,詠謝靈運詩曰:「韓亡子房奮,秦
 帝魯連恥。本自江海人,忠義感君子。」因流涕。



 三月辛亥,王南臨黎陽,濟於虎牢,自洛陽從太行而反晉陽。於路遺書百僚,以相戒勵。朝野承風,莫不震肅。又令朝臣牧宰各舉賢良及驍武膽略堪守邊城,務得其才,不拘職業。六月,王巡北邊城戍,賑賜有差。



 七月,王還晉陽。辛卯,王遇盜而殂,時年二十九。葬於峻成陵。齊受禪,追謚為文襄皇帝,廟號世宗。時有童謠曰:「百尺高竿摧折,水底燃燈燈滅。」識者以為王將殂之兆也。數日前,崔季舒無
 故於北宮門外諸貴之前誦鮑明遠詩曰:「將軍既下世,部曲亦罕存。」聲甚淒斷,淚不能已,見者莫不怪之。初,梁將蘭欽子京為東魏所虜,王命以配廚。欽請贖之,王不許。京再訴,王使監廚蒼頭薛豐洛杖之,曰:「更訴當殺爾。」京與其黨六人謀作亂。時王居北城東柏堂蒞政,以寵琅邪公主,欲其來往無所避忌,所有侍衛,皆出於外。太史啟言宰輔星甚微,變不出一月。王曰:「小人新杖之,故嚇我耳。」將欲受禪,與陳元康、崔季舒等屏斥左右,署擬
 百官。京將進食,王卻,謂諸人曰:「昨夜夢此奴斫我,宜殺卻。」京聞之,置刀於盤,冒言進食。王怒曰:「我未索食,爾何據來!」京揮刀曰:「來將殺汝!」王自投傷足,入於床下。賊黨去床,因而見殺。先是訛言曰:「軟脫帽,床底喘」,其言應矣。時太原公洋在城東雙堂,入而討賊,臠割京等,皆漆其頭。



 秘不發喪,徐出言曰:「奴反,大將軍被傷,無大苦也。」



\end{pinyinscope}