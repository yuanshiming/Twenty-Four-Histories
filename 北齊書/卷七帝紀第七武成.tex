\article{卷七帝紀第七武成}

\begin{pinyinscope}

 世祖武成皇帝,諱湛,神武皇帝第九子,孝昭皇帝之母弟也。儀表瑰傑,神武尤所鐘愛。神武方招懷荒遠,乃為帝聘蠕蠕太子庵羅辰女,號「鄰和公主」。帝時年八歲,冠
 服端嚴,神情閑遠,華戎歎異。元象中,封長廣郡公。天保初,進爵為王,拜尚書令,尋兼司徒,遷太尉。乾明初,楊愔等密相疏忌,以帝為大司馬,領并州刺史。帝既與孝昭謀誅諸執政,遷太傅、錄尚書事、領京畿大都督。皇建初,進位右丞相。孝昭幸晉陽,帝以懿親居守鄴,政事咸見委託。二年,孝昭崩,遺詔徵帝入統大位。及晉陽宮,發喪於崇德殿。皇太后令所司宣遺詔,左丞相斛律金率百僚敦勸,三奏,乃許之。



 大寧元年冬十一月癸丑,皇帝即位於南宮,大赦,改皇建二年為大寧。乙卯,以司徒、平秦王歸彥為太傅,以尚書右僕射、趙郡王睿為尚書令,以太尉尉粲為太保,以尚書令段韶為大司馬,以豐州刺史婁睿為司空,以太傅、平陽王淹為太宰,以太保、彭城王浟為太師、錄尚書事,以冀州刺史、博陵王濟為太尉,以中書監、任城王潛為尚書左僕射,以并州刺史斛律光為右僕射,封孝昭皇帝太子百年為樂陵郡王。庚申,詔大使巡行天下,求
 政善惡,問人疾苦,擢進賢良。是歲,周武帝保定元年。



 河清元年春正月乙亥,車駕至自晉陽。辛巳,祀南郊。壬午,享太廟。丙戌,立妃胡氏為皇后,子緯為皇太子。大赦,內外百官普加汎級,諸為父後者賜爵一級。



 己亥,以前定州刺史、馮翊王潤為尚書左僕射。詔斷屠殺以順春令。二月丁未,以太宰、平陽王淹為青州刺史、太傅、領司徒,以領軍大將軍、宗師、平秦王歸彥為太宰、冀州刺史。乙卯,以兼尚書令、任城王湝為司徒。詔散騎常侍崔瞻
 聘于陳。



 夏四月辛丑,皇太后婁氏崩。乙巳,青州刺史上言,今月庚寅河、濟清。以河、濟清,改大寧二年為河清,降罪人各有差。五月甲申,祔葬武明皇后於義平陵。己丑,以尚書右僕射斛律光為尚書令。秋七月,太宰、冀州刺史、平秦王歸彥據州反,詔大司馬段韶、司空婁睿討擒之。乙未,斬歸彥并其三子及黨與二十人於都市。丁酉,以大司馬段韶為太傅,以司空婁睿為司徒,以太傅、平陽王淹為太宰,以尚書令斛律光為司空,以太子太傅、
 趙郡王睿為尚書令,中書監、河間王孝琬為尚書左僕射。



 癸亥,行幸晉陽。陳人來聘。冬十一月丁丑,詔兼散騎常侍封孝琰使於陳。十二月丙辰,車駕至自晉陽。是歲,殺太原王紹德。



 二年春正月乙亥,帝詔臨朝堂策試秀才。以太子少傅魏收為兼尚書右僕射。己卯,兼右僕射魏收以阿縱除名。丁丑,以武明皇后配祭北郊。辛卯,帝臨都亭錄見囚,降在京罪人各有差。三月乙丑,詔司空斛律光督五營
 軍士築戍於軹關。壬申,室韋國遣使朝貢。丙戌,以兼尚書右僕射趙彥深為左僕射。夏四月,并、汾、晉、東雍、南汾五州蟲旱傷稼,遣使賑恤。戊午,陳人來聘。五月壬午,詔以城南雙堂閏位之苑,乃造大總持寺。六月乙巳,齊州言濟、河水口見八龍昇天。乙卯,詔兼散騎常侍崔子武使于陳。庚申,司州牧、河南王孝瑜薨。秋八月辛丑,詔以三臺宮為大興聖寺。冬十二月癸巳,陳人來聘。己酉,周將楊忠帥突厥阿史那木汗等二十餘萬人自恒州
 分為三道,殺掠吏人。是時,大雨雪連月,南北千餘里平地數尺,霜晝下,雨血於太原。戊午,帝至晉陽。己未,周軍逼并州,又遣大將軍達奚武帥眾數萬至東雍及晉州,與突厥相應。是歲,室韋、庫莫奚、靺羯、契丹並遣使朝貢。



 三年春正月庚申朔,周軍至城下而陳,戰於城西。周軍及突厥大敗,人畜死者相枕,數百里不絕。詔平原王段韶追出塞而還。三月辛酉,以律令班下,大赦。己巳,盜殺太師、彭城王浟。庚辰,以司空斛律光為司徒,以侍中、武
 興王普為尚書左僕射。甲申,以尚書令、馮翊王潤為司空。夏四月辛卯,詔兼散騎常侍皇甫亮使於陳。五月甲子,帝至自晉陽。壬午,以尚書令、趙郡王睿為錄尚書事,以前司徒婁睿為太尉。甲申,以太傅段韶為太師。丁亥,以太尉、任城王湝為大將軍。壬辰,行幸晉陽。六月庚子,大雨晝夜不息,至甲辰乃止。是月,晉陽訛言有鬼兵,百姓競擊銅鐵以捍之。殺樂陵王百年。歸宇文媼于周。秋九月乙丑,封皇子綽為南陽王,儼為東平王。是月,歸閻
 媼于周。陳人來聘。突厥寇幽州,入長城,虜掠而還。閏月乙未,詔遣十二使巡行水潦州,免其租調。乙巳,突厥寇幽州。周軍三道並出,使其將尉遲迥寇洛陽,楊檦入軹關,權景宣趣懸瓠。冬十一月甲午,迥等圍洛陽。



 戊戌,詔兼散騎常侍劉逖使於陳。甲辰,太尉婁睿大破周軍於軹關,擒楊檦。



 十二月乙卯,豫州刺史王士良以城降周將權景宣。丁巳,帝自晉陽南討。己未,太宰、平陽王淹薨。壬戌,太師段韶大破尉遲迥等,解洛陽圍。丁卯,帝至洛
 陽,免洛州經周軍處一年租賦,赦州城內死罪已下囚。己巳,以太師段韶為太宰,以司徒斛律光為太尉,并州刺史蘭陵王長恭為尚書令。壬申,帝至武牢,經滑臺,次於黎陽,所經減降罪人。丙子,車駕至自洛陽。是歲,高麗、靺羯、新羅並遣使朝貢。



 山東大水,饑死者不可勝計,詔發賑給,事竟不行。



 四年春正月癸卯,以大將軍、任城王湝為大司馬。辛未,幸晉陽。二月甲寅,詔以新羅國王金真興為使持節、東
 夷校尉、樂浪郡公、新羅王。壬申,以年穀不登,禁酤酒。己卯,詔減百官食稟各有差。三月戊子,詔給西兗、梁、滄、趙州,司州之東郡、陽平、清河、武都,冀州之長樂、渤海遭水潦之處貧下戶粟,各有差。家別斗升而已,又多不付。是月,彗星見;有物隕於殿庭,如赤漆鼓帶小鈴;殿上石自起,兩兩相對。又有神見於後園萬壽堂前山穴中,其體壯大,不辯其面,兩齒絕白,長出於唇,帝直宿嬪御已下七百人咸見焉。帝又夢之。夏四月戊午,大將軍、東安王婁
 睿坐事免。乙亥,陳人來聘。太史奏天文有變,其占當有易王。丙子,乃使太宰段韶兼太尉,持節奉皇帝璽綬傳位於皇太子,大赦,改元為天統元年,百官進級降罪各有差。又詔皇太子妃斛律氏為皇后。於是群公上尊號為太上皇帝,軍國大事咸以奏聞。始將傳政,使內參乘子尚乘驛送詔書於鄴。子尚出晉陽城,見人騎隨後,忽失之,尚未至鄴而其言已布矣。



 天統四年十二月辛未,太上皇帝崩於鄴宮乾壽堂,時年三十二,謚曰武成皇
 帝,廟號世祖。五年二月甲申,葬於永平陵。



\end{pinyinscope}