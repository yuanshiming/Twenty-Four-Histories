\article{卷十三列傳第五趙郡王琛 清河王岳}

\begin{pinyinscope}

 趙
 郡王琛,字永寶,高祖之弟也。少時便弓馬,有志氣。高祖既匡天下,中興初,授散騎常侍、鎮西將軍、金紫光祿
 大夫。既居禁衛,恭勤慎密,率先左右。太昌初,除車騎大將軍、左光祿大夫,封南趙郡公,食邑五千戶。尋拜驃騎大將軍、特進、開府儀同三司、散騎常侍。永熙二年,除使持節、都督定州刺史、六州大都督。琛推誠撫納,拔用人士,甚有聲譽。及斛斯椿等釁結。高祖將謀內討,以晉陽根本,召琛留掌後事,以為並、肆、汾大行臺僕射,領六州九酋長大都督,其相府政事琛悉決之。天平中,除御史中尉,正色糾彈,無所回避,遠近肅然。尋亂高祖後庭,高
 祖責罰之,因杖而斃,時年二十三。贈使持節、侍中、都督冀定滄瀛幽殷并肆雲朔十州諸軍事、驃騎大將軍、冀州刺史、太尉、尚書令,謚曰貞平。天統三年,又贈假黃鉞、左丞相、太師、錄尚書事、冀州刺史,進爵為王,配饗高祖廟庭。



 子睿嗣。



 睿小名須拔,生三旬而孤,聰慧夙成,特為高祖所愛,養於宮中,令游娘母之,恩同諸子。魏興和中,襲爵南趙郡公。至四歲,未嘗識母,其母則魏華陽公主也。



 有鄭氏者,
 睿母之從母姊妹之女,戲語睿曰:「汝是我姨兒,何因倒親游氏。」睿因問訪,遂精神不怡。高祖甚以為怪,疑其感疾,欲命醫看之。睿對曰:「兒無患苦,但聞有所生,欲得暫見。」高祖驚曰:「誰向汝道耶?」睿具陳本末。高祖命元夫人令就宮與睿相見,睿前跪拜,因抱頭大哭。高祖甚以悲傷。語平秦王曰:「此兒天生至孝,我兒子無有及者。」遂為休務一日。睿初讀《孝經》,至「資於事父」,輒流涕歔欷。十歲喪母,高祖親送睿至領軍府,為睿發喪,與聲殞絕,哀感
 左右,三日水漿不入口。高祖與武明婁皇后殷勤敦譬,方漸順旨。居喪盡禮,持佛像長齋,至於骨立,杖而後起。高祖令常山王共臥起,日夜說喻之。并敕左右不聽進水,雖絕清漱,午後輒不肯食。由是高祖食必喚睿同案。其見愍惜如此。高祖崩,哭泣嘔血。及壯,將為婚娶,而貌有戚容。世宗謂之曰:「我為爾娶鄭述祖女,門閥甚高,汝何所嫌而精神不樂?」睿對曰:「自痛孤遺,常深膝下之慕,方從婚冠,彌用感切。」言未卒,嗚咽不自勝。世宗為之憫
 默。勵己勤學,常夜久方罷。



 武定末,除太子庶子。顯祖受禪,進封爵為趙郡王,邑一千二百戶,遷散騎常侍。



 睿身長七尺,容儀甚偉,閑習吏職,有知人之鑒。二年,出為定州刺史,加撫軍將軍、六州大都督,時年十七。睿留心庶事,糾摘姦非,勸課農桑,接禮民俊,所部大治,稱為良牧。三年,加儀同三司。六年,詔睿領山東兵數萬監築長城。于時盛夏六月,睿在途中,屏除蓋扇,親與軍人同其勞苦。而定州先有冰室,每歲藏冰,長史宋欽道以睿冒
 犯暑熱,遂遣輿冰,倍道追送。正值日中停車,炎赫尤甚,人皆不堪,而送冰者至,咸謂得冰一時之要。睿乃對之歎息云:「三軍之人,皆飲溫水,吾以何義,獨進寒冰,非追名古將,實情所不忍。」遂至消液,竟不一嘗。



 兵人感悅,遐邇稱歎。先是,役徒罷作,任其自返。丁壯之輩,各自先歸;羸弱之徒,棄在山北,加以饑病,多致僵殞。睿於是親帥所部,與之俱還,配合州鄉,部分營伍,督帥監領,強弱相持,遇善水草,即為停頓,分有餘,贍不足,屯以全者十三
 四焉。



 七年,詔以本官都督滄瀛幽安平東燕六州諸軍事、滄州刺史。八年,徵睿赴鄴,仍除北朔州刺史,都督北燕、北蔚、北恒三州,及庫推以西黃河以東長城諸鎮諸軍事。睿慰撫新遷,量置烽戍,內防外禦,備有條法,大為兵民所安。有無水之處,禱而掘井,鍬鍤裁下,泉源湧出,至今號曰趙郡王泉。



 九年,車駕幸樓煩,睿朝於行宮,仍從還晉陽。時濟南以太子監國,因立大都督府,與尚書省分理眾事,仍開府置佐。顯祖特崇其選,乃除睿侍中、
 攝大都督府長史。睿後因侍宴,顯祖從容顧謂常山王演等曰:「由來亦有如此長史不?吾用此長史何如?」演對曰:「陛下垂心庶政,優賢禮物,須拔進居蟬珥之榮,退當委要之職,自昔以來,實未聞如此銓授。」帝曰:「吾於此亦自謂得宜。」十年,轉儀同三司、侍中、將軍、長史,王如故。尋加開府儀同三司、驃騎大將軍、太子太保。



 皇建初,行并州事。孝昭臨崩,預受顧託,奉迎世祖於鄴,以功拜尚書令,別封浮陽郡公,監太史,太子太傅,議律令。又以討北
 狄之功,封潁川郡公。復拜尚書令,攝大宗正卿。天統中,追贈睿父琛假黃鉞,母元氏贈趙郡王妃,謚曰貞昭,華陽長公主如故,有司備禮儀就墓拜授。時隆冬盛寒,睿跣步號哭,面皆破裂,嘔血數升。及還,不堪參謝,帝親就第看問。拜司空,攝錄尚書事。突厥嘗侵軼至并州,帝親御戎,六軍進止皆令取睿節度。以功復封宣城郡公。攝宗正卿,進拜太尉,監議五禮。睿久典朝政,清真自守,譽望日隆,漸被疏忌,乃撰古之忠臣義士,號曰《要言》,以致
 其意。



 世祖崩,葬後數日,睿與馮翊王潤、安德王延宗及元文遙奏後主云:「和士開不宜仍居內任。」并入奏太后,因出士開為兗州刺史。太后曰:「士開舊經驅使,欲留過百日。」睿正色不許。數日之內,太后數以為言。有中官要人知太后密旨,謂睿曰:「太后意既如此,殿下何宜苦違。」睿曰:「吾國家事重,死且不避,若貪生茍全,令國家擾攘,非吾志也。況受先皇遺旨,委寄不輕。今嗣主幼沖,豈可使邪臣在側。不守之以正,何面戴天。」遂重進言,詞理懇
 切。太后令酌酒賜睿。



 睿正色曰:「今論國家大事,非為卮酒!」言訖便出。其夜,睿方寢,見一人可長丈五,臂長丈餘,當門向床,以臂壓睿,良久,遂失所在。睿意甚惡之,便起坐獨歎曰:「大丈夫命運一朝至此!」恐為太后所殺,旦欲入朝,妻子咸諫止之。睿曰:「自古忠臣,皆不顧身命,社稷事重,吾當以死效之,豈容令一婦人傾危宗廟。且和士開何物豎子,如此縱橫,吾寧死事先皇,不忍見朝廷顛沛。」至殿門,又有人曰:「願殿下勿入,慮有危變。」睿曰:「吾上不
 負天,死亦無恨。」入見太后,太后復以為言,睿執之彌固。出至永巷,遇兵被執,送華林園,於雀離佛院令劉桃枝拉而殺之,時年三十六。大霧三日,朝野冤惜之。期年後,詔聽以王禮葬,竟無贈謚焉。



 子整信嗣。歷散騎常侍、儀同三司。好學有行檢,少年時因獵墜馬,傷腰腳,卒不能行起,終於長安。琛同母弟惠寶早亡,元象初,贈侍中、尚書令、都督四州諸軍事、青州刺史。天統三年,重贈十州都督,封陳留王,謚曰文恭,以清河王岳第十子敬文嗣。



 清河王岳,字洪略,高祖從父弟也。父翻,字飛雀,魏朝贈太尉,謚孝宣公。



 岳幼時孤貧,人未之知也,長而敦直,姿貌嶷然,沈深有器量。初,岳家於洛邑,高祖每奉使入洛,必止于岳舍。岳母山氏,嘗夜起,見高祖室中有光,密往覘之,乃無燈,即移高祖於別室,如前所見。怪其神異,詣卜者筮之,遇《乾》之《大有》,占之曰:「吉,《易》稱『飛龍在天,大人造也』,飛龍九五大人之卦,貴不可言。」



 山氏歸報高祖。後高祖起兵於信都,山氏聞之,大喜,謂岳曰:「赤光之瑞,今
 當驗矣,汝可間行從之,共圖大計。」岳遂往信都。高祖見之,大悅。



 中興初,除散騎常侍、鎮東將軍、金紫光祿大夫,領武衛將軍。高祖與四胡戰于韓陵,高祖將中軍,高昂將左軍,岳將右軍。中軍敗績,賊乘之,岳舉麾大呼,橫衝賊陣,高祖方得回師,表裏奮擊,因大破賊。以功除衛將軍、右光祿大夫,仍領武衛。太昌初,除車騎將軍、左光祿大夫,領左右衛,封清河郡公,食邑二千戶。



 母山氏,封為郡君,授女侍中,入侍皇后。時爾朱兆猶據并州,高祖將
 討之,令岳留鎮京師,遷驃騎大將軍、儀同三司。天平二年,除侍中、六州軍事都督,尋加開府。岳辟引時賢,以為僚屬,論者以為美。尋都監典書,復為侍學,除使持節、六州大都督、冀州大中正。俄拜京畿大都督,其六州事悉詣京畿。時高祖統務晉陽,岳與侍中孫騰等在京師輔政。元象二年,遭母憂去職。岳性至孝,盡力色養,母若有疾,衣不解帶,及遭喪,哀毀骨立,高祖深以憂之,每日遣人勞勉。尋起復本任。



 二年,除兼領軍將軍。興和初,世宗
 入總朝政,岳出為使持節、都督、冀州刺史,侍中、驃騎、開府儀同如故。三年,轉青州刺史。岳任權日久,素為朝野畏服,及出為藩,百姓望風讋憚。武定元年,除晉州刺史、西南道大都督,得綏邊之稱。時岳遇患,高祖令還並治療,疾瘳,復令赴職。



 及高祖崩,侯景叛,世宗征岳還並,共圖取景之計。而武帝乘間遣其貞陽侯明率眾於寒山,擁泗水灌彭城,與景為掎角聲援。岳總帥諸軍南討,與行臺慕容紹宗等擊明,大破之,臨陣擒明及其大將
 胡貴孫,自餘俘馘數萬。景乃擁眾於渦陽,與左衛將軍劉豐等相持。岳回軍追討,又破之,景單騎逃竄。六年,以功除侍中、太尉,餘如故,別封新昌縣子。又拜使持節、河南總管、大都督,統慕容紹宗、劉豐等討王思政於長社。思政嬰城自守,岳等引洧水灌城。紹宗、劉豐為思政所獲,關西出兵援思政,岳內外防禦,甚有謀算。城不沒者三板。會世宗親臨,數日城下,獲思政等。以功別封真定縣男,世宗以為己功,故賞典弗弘也。



 世宗崩,顯祖出撫
 晉陽,令岳以本官兼尚書左僕射,留鎮京師。天保初,進封清河郡王,尋除使持節、驃騎大將軍、開府儀同三司、宗師、司州牧。五年,加太保。梁蕭繹為周軍所逼,遣使告急,且請援。冬,詔岳為西南道大行臺,統司徒潘相樂等救江陵。六年正月,師次義陽,遇荊州陷,因略地南至郢州,獲梁州刺史司徒陸法和,仍克郢州。岳先送法和於京師,遣儀同慕容儼據郢城。朝廷知江陵陷,詔岳旋師。



 岳自討寒山、長社及出隨、陸,並有功績,威名彌重。而性
 華侈,尤悅酒色,歌姬舞女,陳鼎擊鐘,諸王皆不及也。初,高歸彥少孤,高祖令岳撫養,輕其年幼,情禮甚薄。歸彥內銜之而未嘗出口。及歸彥為領軍,大被寵遇,岳謂其德己,更倚賴之。歸彥密構其短。岳於城南起宅,聽事後開巷。歸彥奏帝曰:「清河造宅,僭擬帝宮,制為永巷,但唯無闕耳。」顯祖聞而惡之,漸以疏岳。仍屬顯祖召鄴下婦人薛氏入宮,而岳先嘗喚之至宅,由其姊也。帝懸薛氏姊而鋸殺之,讓岳以為姦民女。岳曰:「臣本欲取之,嫌其
 輕薄不用,非姦也。」帝益怒。六年十一月,使高歸彥就宅切責之。岳憂悸不知所為,數日而薨,故時論紛然,以為賜鴆也。朝野歎惜之。時年四十四。詔大鴻臚監護喪事,贈使持節、都督冀定滄瀛趙幽濟七州諸軍、太宰、太傅、定州刺史,假黃鉞,給轀輬車,賵物二千段,謚曰昭武。



 初,岳與高祖經綸天下,家有私兵,並畜戎器,儲甲千餘領。世宗之末,岳以四海無事,表求納之。世宗敦至親之重,推心相任,云:「叔屬居肺腑,職在維城,所有之甲,本資國
 用,叔何疑而納之。」文宣之世,亦頻請納,又固不許。及將薨,遺表謝恩,并請上甲于武庫,至此葬畢,方許納焉。皇建中,配享世宗廟庭。後歸彥反,世祖知其前譖,曰:「清河忠烈,盡力皇家,而歸彥毀之,間吾骨肉。」籍沒歸彥,以良賤百口賜岳家。後又思岳之功,重贈太師、太保,餘如故。子勱嗣。



 勱,字敬德,夙智早成,為顯祖所愛。年七歲,遣侍皇太子。後除青州刺史,拜日,顯祖戒之曰:「叔父前牧青州,甚有
 遺惠,故遣汝慰彼黎庶,宜好用心,無墜聲績。」勱流涕對曰:「臣以蒙幼,濫叨拔擢,雖竭庸短,懼忝先政。」帝曰:「汝既能有此言,吾不慮也。」尋追授武衛將軍、領軍、祠部尚書、開府儀同三司。



 以清河地在畿內,改封樂安王。轉侍中、尚書右僕射,出為朔州行臺僕射。



 後主晉州敗,太后從土門道還京師,敕勱統領兵馬,侍衛太后。時佞幸閽寺,猶行暴虐,民間雞豬,悉放鷹犬搏噬取之。勱收儀同三司茍子溢徇軍,欲行大戮。



 太后有令,然後釋之。劉文殊
 竊謂勱曰:「子溢之徒,言成禍福,何容如此,豈不慮後生毀謗耶?」勱攘袂語文殊曰:「自獻武皇帝以來,撫養士卒,委政親賢,用武行師,未有折衄。今西寇已次并州,達官多悉委叛,正坐此輩專政弄權,所以內外離心,衣冠解體。若得今日斬此卒,明日及誅,亦無所恨。王國家姻婭,須同疾惡,返為此言,豈所望乎!」太后還至鄴,周軍續至,人皆恟懼,無有鬥心,朝士出降,晝夜相屬。勱因奏後主曰:「今所翻叛,多是貴人,至於卒伍,猶未離貳。



 請追五品
 已上家屬,置之三臺,因協之曰:『若戰不捷,即退焚臺。』此曹顧惜妻子,必當死戰。且王師頻北,賊徒輕我,今背城一決,理必破之,此亦計之上者。」



 後主卒不能用。齊亡入周,依例授開府。隋朝歷楊、楚、光、洮四州刺史。開皇中卒。



 史臣曰:《易》稱:「天地盈虛,與時消息,況於人乎!」蓋以通塞有期,污隆適道。舉世思治,則顯仁以應之;小人道長,則儉德以避之。至若負博陸之圖,處藩屏之地,而欲迷邦違難,其可得乎。趙郡以跗萼之親,當顧命之重,高揖則
 宗社易危,去惡則人神俱泰。是用安夫一德,同此貞心,踐畏途而不疑,履危機而莫懼。以斯忠義,取斃凶慝。豈道光四海,不遇周成之明;將朝去三仁,終見殷墟之禍。不然則邦國殄瘁,何影響之速乎!清河屬經綸之會,自致青雲,出將入相,翊成鴻業,雖漢朝劉賈,魏室曹洪,俱未足論其高下。天保不辰,易生悔咎,固不可掩其風烈,適以彰顯祖之失德云。



 贊曰:趙郡英偉,風範凝正。天道無親,斯人斯命。赫赫清
 河,於以經國。末路小疵,非為敗德。



\end{pinyinscope}