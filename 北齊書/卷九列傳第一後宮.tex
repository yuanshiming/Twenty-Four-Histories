\article{卷九列傳第一後宮}

\begin{pinyinscope}

 神武明皇后婁氏,諱昭君,贈司徒內乾之女也。少明悟,強族多聘之,並不肯行。及見神武於城上執役,驚曰:「此真吾夫也。」乃使婢通意,又數致私財,使以聘己,父母不得已而許焉。神武既有澄清之志,傾產以結英豪,密謀秘策,後恆參預。及拜渤海王妃,閫闈之事悉決焉。



 後高明嚴斷,雅遵儉約,往來外舍,侍從不過十人。性寬厚,不妒忌,神武姬侍,咸加恩待。神武嘗將西討出師,後夜孿生一男一女,左右以危急,請追告神武。



 後弗聽曰:「王出統大兵,何得以我故輕離軍幕。死生命也,來復何為!」神武聞之,嗟嘆良久。沙
 苑敗後,侯景屢言請精騎二萬,必能取之。神武悅,以告於後。



 後曰:「若如其言,豈有還理,得獺失景,亦有何利。」乃止。神武逼於茹茹,欲娶其女而未決。後曰:「國家大計,願不疑也。」及茹茹公主至,後避正室處之。



 神武愧而拜謝焉,曰:「彼將有覺,願絕勿顧。」慈愛諸子,不異己出,躬自紡績,人賜一袍一褲。手縫戎服,以帥左右。弟昭,以功名自達,其餘親屬,未嘗為請爵位。每言有材當用,義不以私亂公。文襄嗣位,進為太妃。文宣將受魏禪,后固執不許,帝所以
 中止。天保初,尊為皇太后,宮曰宣訓。
 濟南即位,尊為太皇太后。



 尚書令楊愔等
 受遺詔輔政,疏忌諸王。太皇太
 后密與孝昭及諸大將定策誅之,下令廢立。孝昭即位,復為皇太后。孝昭帝崩,太后又下詔立武成帝。大寧二年春,太后寢疾,衣忽自舉,用巫媼言改姓石氏。四月辛丑,崩於北宮,時年六十二。五月甲申,合葬義平陵。



 太后凡孕六男二女,皆感夢:孕文襄則夢一斷龍;孕文宣則夢大龍,首尾屬天地,張口動目,勢狀驚人;孕孝昭則夢
 蠕龍於地;孕武成則夢龍浴於海;孕魏二后並夢月入懷;孕襄城、博陵二王夢鼠入衣下。后未崩,有童謠曰「九龍母死不作孝」。



 及后崩,武成不改服,緋袍如故。未幾,登三臺,置酒作樂。帝女進白袍,帝怒,投諸臺下。和士開請止樂,帝大怒,撻之。帝於昆季次實九,蓋其
 徵驗也。文襄敬皇后元氏,魏孝靜帝之姊也。孝武帝時,封馮翊公主而歸於文襄。容德兼美,曲盡和敬。初生河間王孝琬,時文襄為世子,三日而孝
 靜帝幸世子第,贈錦彩及布帛萬匹。世子辭,求通受諸貴禮遺,於是十屋皆滿。次生兩公主。文宣受禪,尊為文襄皇后,居靜德宮。及天保六年,文宣漸致昏狂,乃移居於高陽之宅,而取其府庫,曰:「吾兄昔姦我婦,我今須報。」乃淫於后。其高氏女婦無親疏,皆使左右亂交之於前。以葛為,令魏安德主騎上,使人推引
 之,又命胡人苦辱之。帝又自呈露,以示群下。武平中,后崩,祔葬義平陵。文宣皇后李氏,諱祖娥,趙郡李希宗女也。



 容德甚美。初為太原公夫人。及帝將建中宮,高隆之、高德正言漢婦人不可為天下母,宜更擇美配。楊愔固請依漢、魏故事,不
 改元妃。而德正猶固請廢后而立段昭儀,欲以結勳貴之援,帝竟不從而立后焉。帝好捶撻嬪御,乃至有殺戮者,唯后獨蒙禮敬。天保十年,改為可賀敦皇后。孝昭即位,降居昭信宮,號昭信皇后。武成踐祚,逼后淫亂,云:「若不許,我當殺爾兒。」后懼,從之。後有娠,太原王紹德至閣,不得
 見,慍曰:「兒豈不知耶,姊姊腹大,故不見兒。」后聞之,大慚,由是生女不舉。帝橫刀詬曰:「爾殺我女,我何不殺爾兒!」對后前築殺紹德。后大哭,帝愈怒,裸后亂撾撻
 之,號天不已。盛以絹囊,流血淋漉,投諸渠水,良久乃蘇,犢車載送妙勝尼寺。后性愛佛法,因此為尼。齊亡入關。隋時得還趙郡。



 孝昭皇后元氏,開府元蠻女也。初為常山王妃。天保末,賜姓步六孤。孝昭即位,立為皇后。帝崩,梓宮之鄴。始渡汾橋,武成聞后有奇藥,追索之不得,使閹人就車頓辱。降居順成宮。武成既殺樂陵王,元被閟隔,不得與家相知。宮闈內忽有飛語,帝令檢推,得后父兄書信,元蠻由
 是坐免官。后以齊亡入周氏宮中。隋文帝作相,放還山東。



 武成皇后胡氏,安定胡延之女。其母范陽盧道約女,初懷孕,有胡僧詣門曰:「此宅瓠蘆中有月」,既而生后。天保初,選為長廣王妃。產後主日,號鳴於產帳上。武成崩,尊為皇太后,陸媼及和士開密謀殺趙郡王睿,出婁定遠、高文遙為刺史。和、陸諂事太后,無所不至。初武成時,后與諸閹人褻狎。武成寵幸和士開,每與后握槊,因此與
 后姦通。自武成崩後,數出詣佛寺,又與沙門曇獻通。布金錢於獻席下,又挂寶裝胡床於獻屋壁,武成平生之所御也。乃置百僧於內殿,託以聽講,日夜與曇獻寢處。以獻為昭玄統。僧徒遙指太后以弄曇獻,乃至謂之為太上者。



 帝聞太后不謹而未之信,後朝太后,見二少尼,悅而召之,乃男子也,於是曇獻事亦發,皆伏法,并殺元、山、王三郡君,皆太后之所暱也。帝自晉陽奉太后還鄴,至紫陌,卒遇大風。舍人魏僧伽明風角,奏言即時當有
 暴逆事。帝詐云鄴中有急,彎弓緾槊,馳入南城,令鄧長顒幽太后北宮,仍有敕內外諸親一不得與太后相見。



 久之,帝復迎太后。太后初聞使者至,大驚,慮有不測。每太后設食,帝亦不敢嘗。



 周使元偉來聘,作《述行賦》,敘鄭莊公克段而遷姜氏,文雖不工,當時深以為愧。



 齊亡入周,恣行姦穢。隋開皇中殂。



 後主皇后斛律氏,左丞相光之女也。初為皇太子妃。後主受禪,立為皇后。武平三年正月生女,帝欲悅光,詐稱
 生男,為之大赦。光誅,后廢在別宮,後令為尼。



 齊滅,嫁為開府元仁妻。



 後主皇后胡氏,隴東王長仁女也。胡太后失母儀之道,深以為愧,欲求悅後主,故飾后於宮中,令帝見之。帝果悅,立為弘德夫人,進左昭儀,大被寵愛。斛律后廢,陸媼欲以穆夫人代之,太后不許。祖孝徵請立胡昭儀,遂登為皇后。陸媼既非勸立,又意在穆夫人,其後於太后前作色而言曰:「何物親姪女,作如此語言!」



 太后問有何言,
 曰:「不可道。」固問之,乃曰:「語大家云,太后行多非法,不可以訓。」太后大怒,喚后出,立剃其髮,送令還家。帝思之,每致物以通意。後與斛律廢后俱召入內,數日而鄴不守。後亦改嫁。



 後主皇后穆氏,名邪利,本斛律后從婢也。母名輕霄,本穆子倫婢也,轉入侍中宋欽道家,姦私而生后,莫知氏族,或云后即欽道女子也。小字黃花,後字舍利。



 欽道婦妒,黥輕霄面為「宋」字。欽道伏誅,黃花因此入宮,有幸於
 後主,宮內稱為舍利太監。女侍中陸太姬知其寵,養以為女,薦為弘德夫人。武平元年六月,生皇子恒。於時後主未有儲嗣,陸陰結待,以監撫之任不可無主,時皇后斛律氏,丞相光之女也,慮其懷恨,先令母養之,立為皇太子。陸以國姓之重,穆、陸相對,又奏賜姓穆氏。胡庶人之廢也,陸有助焉,胡遂立為皇后,大赦。初,有折衝將軍元正烈於鄴城東水中得璽以獻,文曰「天王后璽」,蓋石氏所作。詔書頒告,以為穆后之瑞焉。武成時,為胡后造
 真珠裙褲,所費不可稱計,被火所燒。後主既立穆皇后,復為營之。屬周武遭太后喪,詔侍中薛孤、康買等為弔使,又遣商胡齎錦綵三萬匹與使同往,欲市真珠為皇后造七寶車,周人不與交易,然而竟造焉。先是童謠曰:「黃華勢欲落,清觴滿盃酌。」言黃花不久也,後主自立穆后以後,昏飲無度,故云清觴滿盃酌。陸息駱提婆詔改姓為穆,陸太姬,皆以皇后故也。后既以陸為母,提婆為家,更不採輕霄。輕霄後自療面,欲求見,太后、陸媼使
 禁掌之,竟不得見。



\end{pinyinscope}