\article{卷十一列傳第三文襄六王}

\begin{pinyinscope}

 河
 南康舒王孝瑜廣寧王孝珩河間王孝琬蘭陵武王孝瓘安德王
 延宗漁陽王紹信文襄六男:文敬元皇后生河間王孝琬,宋氏生河南王孝瑜,王氏生廣寧王孝珩,蘭陵王長恭不得母氏姓,陳氏生安德王延宗,燕氏生漁陽王紹信。



 河南康舒王孝瑜,字正德,文襄長子也。初封河南郡公,齊受禪,進爵為王。



 歷位中書令、司州牧。初,孝瑜養於神
 武宮中,與武成同年相愛。將誅楊愔等,孝瑜預其謀。及武成即位,禮遇特隆。帝在晉陽,手敕之曰:「吾飲汾清二盃,勸汝於鄴酌兩盃。」其親愛如此。孝瑜容貌魁偉,精彩雄毅,謙慎寬厚,兼愛文學,讀書敏速,十行俱下,覆棋不失一道。初,文襄於鄴東起山池遊觀,時俗眩之。孝瑜遂於第作水堂、龍舟,植幡槊於舟上,數集諸弟宴射為樂。武成幸其第,見而悅之,故盛興後園之玩,於是貴賤慕斅,處處營造。



 武成常使和土開與胡后對坐握槊,孝瑜
 諫曰:「皇后天下之母,不可與臣下接手。」帝深納之。後又言趙郡王父死非命,不可親。由是睿及士開皆側目。士開密告其奢僭,睿又言山東唯聞河南王,不聞有陛下。帝由是忌之。爾朱御女名摩女,本事太后,孝瑜先與之通,後因太子婚夜,孝瑜竊與之言。武成大怒,頓飲其酒三十七盃。體至肥大,腰帶十圍。使婁子彥載以出,鴆之於車。至西華門,煩熱躁悶,投水而絕。贈太尉、錄尚書事。子弘節嗣。



 孝瑜母,魏吏部尚書宋弁孫也,本魏潁川王
 斌之妃,為文襄所納,生孝瑜,孝瑜還第,為太妃。孝瑜妃,盧正山女,武成胡后之內姊也。孝瑜薨後,宋太妃為盧妃所譖訴,武成殺之。



 廣寧王孝珩,文襄第二子也。歷位司州牧、尚書令、司空、司徒、錄尚書、大將軍、大司馬。孝珩愛賞人物,學涉經史,好綴文,有伎藝。嘗於廳事壁自畫一蒼鷹,見者皆以為真,又作朝士圖,亦當時之妙絕。



 後主自晉州敗奔鄴,詔王公議於含光殿。孝珩以大敵既深,事藉機變,宜任
 城王領幽州道兵入土門,揚聲趣并州;獨孤永業領洛州兵趣潼關,揚聲趣長安;臣請領京畿兵出滏口,鼓行逆戰。敵聞南北有兵,自然潰散。又請出宮人珍寶賜將士,帝不能用。承光即位,以孝珩為太宰。與呼延族、莫多婁敬顯、尉相願同謀,期正月五日,孝珩於千秋門斬高阿那肱,相願在內以禁兵應之,族與敬顯自遊豫園勒兵出。既而阿那肱從別宅取便路入宮,事不果。乃求出拒西軍,謂阿那肱、韓長鸞、陳德信等云:「朝廷不賜遣擊
 賊,豈不畏孝珩反耶?孝珩破宇文邕,遂至長安,反時何與國家事。以今日之急,猶作如此猜疑。」高、韓恐其變,出孝珩為滄州刺史。



 至州,以五千人會任城王於信都,共為匡復計。周齊王憲來伐,兵弱不能敵。怒曰:「由高阿那肱小人,吾道窮矣!」齊叛臣乞扶令和以槊剌孝珩墜馬,奴白澤以身扞之,孝珩猶傷數處,遂見虜。齊王憲問孝珩齊亡所由,孝珩自陳國難,辭淚俱下,俯仰有節。憲為之改容,親為洗創傅藥,禮遇甚厚。孝珩獨歎曰:「李穆叔
 言齊氏二十八年,今果然矣。自神武皇帝以外,吾諸父兄弟無一人得至四十者,命也。嗣君無獨見之明,宰相非柱石之寄,恨不得握兵符,受廟算,展我心力耳。」至長安,依例授開府、縣侯。後周武帝在雲陽,宴齊君臣,自彈胡琵琶,命孝珩吹笛。辭曰:「亡國之音,不足聽也。」固命之,舉笛裁至口,淚下嗚咽,武帝乃止。其年十月,疾甚,啟歸葬山東,從之。尋卒,令還葬鄴。



 河間王孝琬,文襄第三子也。天保元年封。天統中,累遷
 尚書令。初,突厥與周師入太原,武成將避之而東。孝琬叩馬諫,請委趙郡王部分之,必整齊,帝從其言。孝琬免胄將出,帝使追還。周軍退,拜并州刺史。



 孝琬以文襄世嫡,驕矜自負。河南王之死,諸王在宮內莫敢舉聲,唯孝琬大哭而出。又怨執政,為草人而射之。和士開與祖珽譖之,云:「草人擬聖躬也。又前突厥至州,孝琬脫兜鍪抵地,云『豈是老嫗,須著此』。此言屬大家也。」初,魏世謠言:「河南種穀河北生,白楊樹頭金雞鳴。」珽以說曰:「河南、河北,
 河間也。金雞鳴,孝琬將建金雞而大赦。」帝頗惑之。時孝琬得佛牙,置於第內,夜有神光。昭玄都法順請以奏聞,不從。帝聞,使搜之,得鎮庫槊幡數百。帝聞之,以為反。訊其諸姬,有陳氏者無寵,誣對曰「孝琬畫作陛下形哭之」,然實是文襄像,孝琬時時對之泣。帝怒,使武衛赫連輔玄倒鞭撾之。孝琬呼阿叔,帝怒曰:「誰是爾叔?敢喚我作叔!」孝琬曰:「神武皇帝嫡孫,文襄皇帝嫡子,魏孝靜皇帝外甥,何為不得喚作叔也?」帝愈怒,折其兩脛而死。
 瘞諸西山,帝崩後,乃改葬。子正禮嗣,幼聰穎,能誦《左氏春秋》。齊亡,遷綿州卒。



 蘭陵武王長恭,一名孝瓘,文襄第四子也。累遷并州刺史。突厥入晉陽,長恭盡力擊之。芒山之敗,長恭為中軍,率五百騎再入周軍,遂至金墉之下,被圍甚急,城上人弗識,長恭免胄示之面,乃下弩手救之,於是大捷。武士共歌謠之,為《蘭陵王入陣曲》是也。歷司州牧、青瀛二州,頗受財貨。後為太尉,與段韶討栢谷,又攻定陽。韶病,長
 恭總其眾。前後以戰功別封巨鹿、長樂、樂平、高陽等郡公。



 芒山之捷,後主謂長恭曰:「入陣太深,失利悔無所及。」對曰:「家事親切,不覺遂然。」帝嫌其稱家事,遂忌之。及在定陽,其屬尉相願謂曰:「王既受朝寄,何得如此貪殘?」長恭未答。相願曰:「豈不由芒山大捷,恐以威武見忌,欲自穢乎?」長恭曰:「然。」相願曰:「朝廷若忌王,於此犯便當行罰,求福反以速禍。」



 長恭泣下,前膝請以安身術。相願曰:「王前既有勳,今復告捷,威聲太重,宜屬疾在家,勿預事。」
 長恭然其言,未能退。及江淮寇擾,恐復為將,歎曰:「我去年面腫,今何不發。」自是有疾不療。武平四年五月,帝使徐之範飲以毒藥。長恭謂妃鄭氏曰:「我忠以事上,何辜於天,而遭鴆也!」妃曰:「何不求見天顏?」



 長恭曰:「天顏何由可見。」遂飲藥薨。贈太尉。



 長恭貌柔心壯,音容兼美。為將躬勤細事,每得甘美,雖一瓜數果,必與將士共之。初在瀛州,行參軍陽士深表列其贓,免官。及討定陽,士深在軍,恐禍及。



 長恭聞之曰:「吾本無此意。」乃求小失,杖士
 深二十以安之。嘗入朝而僕從盡散,唯有一人,長恭獨還,無所譴罰,武成賞其功,命賈護為買妾二十人,唯受其一。



 有千金責券,臨死日,盡燔之。



 安德王延宗,文襄第五子也。母陳氏,廣陽王妓也。延宗幼為文宣所養,年十二,猶騎置腹上,令溺己臍中,抱之曰:「可憐止有此一個。」問欲作何王,對曰:「欲作衝天王。」文宣問楊愔,愔曰:「天下無此郡名,願使安於德。」於是封安德焉。為定州刺史,於樓上大便,使人在下張口承之。以
 蒸豬糝和人糞以飼左右,有難色者鞭之。孝昭帝聞之,使趙道德就州杖之一百。道德以延宗受杖不謹,又加三十。又以囚試刀,驗其利鈍。驕縱多不法。武成使撻之,殺其暱近九人,從是深自改悔。蘭陵王芒山凱捷,自陳兵勢,諸兄弟咸壯之。延宗獨曰:「四兄非大丈夫,何不乘勝徑入?使延宗當此勢,關西豈得復存!」及蘭陵死,妃鄭氏以頸珠施佛。



 廣寧王使贖之。延宗手書以諫,而淚滿紙。河間死,延宗哭之淚亦甚。又為草人以像武成,鞭而訊
 之曰:「何故殺我兄!」奴告之,武成覆臥延宗於地,馬鞭撾之二百,幾死。後歷司徒、太尉。



 及平陽之役,後主自禦之,命延宗率右軍先戰,城下擒周開府宗挺。及大戰,延宗以麾下再入周軍,莫不披靡。諸軍敗,延宗獨全軍。後主將奔晉陽,延宗言:「大家但在營莫動,以兵馬付臣,臣能破之。」帝不納。及至并州又聞周軍已入雀鼠谷,乃以延宗為相國、并州刺史,總山西兵事。謂曰:「并州阿兄自取,兒今去也。」延宗曰:「陛下為社稷莫動,臣為陛下出死力
 戰。」駱提婆曰:「至尊計已成,王不得輒沮。」後主竟奔鄴。在并將率咸請曰:「王若不作天子,諸人實不能出死力。」延宗不得已,即皇帝位,下詔曰:「武平孱弱,政由宦豎,釁結蕭墻,盜起疆埸。斬關夜遁,莫知所之,則我高祖之業將墜於地。王公卿士,猥見推逼,今便祗承寶位。可大赦天下,改武平七年為德昌元年。」以晉昌王唐邕為宰輔,齊昌王莫多婁敬顯、沐陽王和阿于子、右衛大將軍段暢、武衛將軍相里僧伽、開府韓骨胡、侯莫陳洛州為爪牙。
 眾聞之,不召而至者,前後相屬。延宗容貌充壯,坐則仰,偃則伏,人笑之,乃赫然奮發。氣力絕異,馳騁行陣,勁捷若飛。傾覆府藏及後宮美女,以賜將士,籍沒內參千餘家。後主謂近臣曰:「我寧使周得并州,不欲安德得之。」左右曰:「理然。」延宗見士卒,皆親執手,陳辭自稱名,流涕嗚咽。



 眾皆爭為死,童兒女子亦乘屋攘袂,投磚石以御周軍。特進、開府那盧安生守太谷,以萬兵叛。周軍圍晉陽,望之如黑雲四合。延宗命莫多婁敬顯、韓骨胡拒城南,
 和阿于子、段暢拒城東。延宗親當周齊王於城北,奮大槊,往來督戰,所向無前。尚書令史沮山亦肥大多力,捉長刀步從,殺傷甚多。武衛蘭芙蓉、綦連延長皆死於陣。



 阿于子、段暢以千騎投周。周軍攻東門,際昏,遂入。進兵焚佛寺門屋,飛焰照天地。延宗與敬顯自門入,夾擊之,周軍大亂,爭門相填壓,齊人從後斫刺,死者二千餘人。周武帝左右略盡,自拔無路,承御上士張壽輒牽馬頭,賀拔佛恩以鞭拂其後,崎嶇僅得出。齊人奮擊,幾中
 焉。城東阨曲,佛恩及降者皮子信為之導,僅免,時四更也。延宗謂周武帝崩於亂兵,使於積屍中求長鬣者,不得。時齊人既勝,入坊飲酒,盡醉臥,延宗不復能整。周武帝出城,饑甚,欲為遁逸計。齊王憲及柱國王誼諫,以為去必不免。延宗叛將段暢亦盛言城內空虛。周武帝乃駐馬,鳴角收兵,俄頃復振。詰旦,還攻東門,克之,又入南門。延宗戰,力屈,走至城北,於人家見禽。周武帝自投下馬,執其手。延宗辭曰:「死人手何敢迫至尊。」帝曰:「兩國天
 子,有何怨惡,直為百姓來耳。勿怖,終不相害。」便復衣帽,禮之。先是,高都郡有山焉,絕壁臨水,忽有黑書見,云:「齊亡延宗。」洗視逾明。帝使人就寫,使者改亡為上。至是應焉。延宗敗前,在鄴廳事,見兩日相連置,以十二月十三日晡時受敕守并州,明日建瘭號,不間日而被圍,經宿,至食時而敗。年號德昌,好事者言其得二日云。既而周武帝問取鄴計,辭曰:「亡國大夫不可以圖存,此非臣所及。」彊問之,乃曰:「若任城王援鄴,臣不能知,若今主自守,
 陛下兵不血刃。」



 及至長安,周武與齊君臣飲酒,令後主起舞,延宗悲不自持。屢欲仰藥自裁,傅婢苦執諫而止。未幾,周武誣後主及延宗等,云遙應穆提婆反,使並賜死。皆自陳無之,延宗攘袂,泣而不言。皆以椒塞口而死。明年,李妃收殯之。



 後主之傳位於太子也,孫正言竊謂人曰:「我武定中為廣州士曹,聞襄城人曹普演有言,高王諸兒,阿保當為天子,至高德之承之,當滅。」阿保謂天保,德之謂德昌也,承之謂後主年號承光,其言竟信云。



 漁陽王紹信,文襄第六子也。歷特進、開府、中領軍、護軍、青州刺史。行過漁陽,與大富人鐘長命同床坐。太守鄭道蓋謁,長命欲起,紹信不聽,曰:「此何物小人,而主人公為起。」乃與長命結為義兄弟,妃與長命妻為姊妹,責其闔家幼長皆有贈賄,鐘氏因此遂貧。齊滅,死於長安。



\end{pinyinscope}