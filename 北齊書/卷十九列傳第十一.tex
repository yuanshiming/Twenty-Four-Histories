\article{卷十九列傳第十一}

\begin{pinyinscope}

 賀拔允蔡俊韓賢尉長命
 王懷劉貴任延敬莫多婁貸文高市貴厙狄迴洛厙狄盛薛孤延
 張保洛侯莫陳相賀拔允,字可泥,神武尖山人也。祖爾頭,父度拔,俱見魏史。允便弓馬,頗有膽略,與弟岳殺賊帥衛可肱,仍奔魏。廣陽王元深上允為積射將軍,持節防滏口。



 深敗,歸爾朱榮。允父子兄弟並以武藝知名,榮素聞之。見允,待之甚厚。建義初,除征東將軍、光祿大夫,封壽陽縣侯,邑七百戶。永安中,除征北將軍、蔚州刺史,進爵為公。魏
 長廣王立,改封燕郡公,兼侍中。使茹茹,還至晉陽,值高祖將出山東,允素知高祖非常人,早自結託。高祖以其北士之望,尤親禮之。遂與允出信都,參定大策。魏中興初,轉司徒,領尚書令。高祖入洛,進爵為王,轉太尉,加侍中。



 魏武帝之猜忌高祖也,以允弟岳深相委託,潛使來往。當時咸慮允為變。及岳死,武帝又委岳弟勝心腹之寄。高祖重其舊,久全護之。天平元年乃賜死,時年四十八,高祖親臨哭。贈定州刺史、五州軍事。



 允有三子,長
 子世文,次世樂,次難陀。興和末,高祖並召與諸子同學。武定中,敕居定州,賜其田宅。



 蔡俊,字景彥,廣寧石門人也。父普,北方擾亂,奔走五原,守戰有功。拜寧朔將軍,封安上縣男,邑二百戶。尋卒,贈輔國將軍、燕州剌史。



 俊豪爽有膽氣,高祖微時,深相親附。與遼西段長、太原龐蒼鷹俱有先知之鑒。



 長為魏懷朔鎮將,嘗見高祖,甚異之,謂高祖云:「君有康世之才,終不徒然也,請以子孫為託。」興和中,啟贈司空公。子寧,相
 府從事中郎,天保初,兼南中郎將。蒼鷹交遊豪俠,厚待賓旅,居於州城。高祖客其舍,初居處於蝸牛廬中,蒼鷹母數見廬上赤氣屬天。蒼鷹亦知高祖有霸王之量,每私加敬,割其宅半以奉高祖,由此遂蒙親識。高祖之牧晉州,引為兼治中從事史,行義寧郡事。及義旗建,蒼鷹乃棄家間行歸高祖,高祖以為兼行臺倉部郎中。卒於安州刺史。



 俊初為杜洛周所虜,時高祖亦在洛周軍中,高祖謀誅洛周,俊預其計。事泄,走奔葛榮,仍背葛歸
 爾朱榮。榮入洛,為平遠將軍、帳內別將。從破葛榮,除諫議大夫。又從平元顥,封烏洛縣男。隨高祖舉義,為都督。高祖平鄴,及破四胡於韓陵,俊並有戰功。太昌中,出為濟州剌史,為治嚴暴,又多受納,然亦明解有部分,吏民畏服之。性好賓客,頗稱施與。後胡遷等據兗州作逆,俊與齊州刺史尉景討平之。



 魏武帝貳於高祖,以濟州要重,欲令腹心據之。陰詔御史構俊罪狀,欲以汝陽王代俊,由是轉行兗州事。高祖以俊非罪,啟復其任。武帝不
 許,除賈顯智為刺史,率眾赴州。俊以防守嚴備,顯智憚之,至東郡,不敢前。



 天平中,為都督,隨領軍婁昭攻樊子鵠於兗州,又與行臺元子思討元慶和,俱平之。侯深反,復以俊為大都督,率眾討之,深敗走。又轉揚州刺史。天平三年秋,卒於州,時年四十二。贈持節、侍中、都督、冀州刺史、尚書令、司空公,謚曰威武。齊受禪,詔祭告其墓。皇建初,配享高祖廟庭。



 韓賢,字普賢,廣寧石門人也。壯健有武用。初隨葛榮作
 逆,榮破,隨例至并州,爾朱榮擢充左右。榮妻子北走,世隆等立魏長廣王曄為主,除賢鎮遠將軍、屯騎校尉。先是,世隆等攻建州及石城,賢並有戰功。爾朱度律用為帳內都督,封汾陽縣伯,邑四百戶。普泰初,除前將軍、廣州刺史。屬高祖起義,度律以賢素為高祖所知,恐其有變,遣使徵之。賢不願應召,乃密遣群蠻,多舉烽火,有如寇難將至。使者遂為啟,得停。賢仍潛遣使人通誠於高祖。高祖入洛,爾朱官爵例皆削除,以賢遠送誠款,令其
 復舊。太昌初,累遷中軍將軍、光祿大夫,出為建州刺史。武帝西入,轉行荊州事。



 天平初,為洛州刺史。民韓木蘭等率土民作逆,賢擊破之,親自按檢,欲收甲仗。有一賊窘迫,藏於死屍之間,見賢將至,忽起斫之,斷其脛而卒。賢雖武將,性和直,不甚貪暴,所歷雖無善政,不為吏民所苦。昔漢明帝時,西域以白馬負佛經送洛,因立白馬寺,其經函傳在此寺,形制淳朴,世以為古物,歷代藏寶。賢無故斫破之,未幾而死,論者或謂賢因此致禍。
 贈侍中、持節、定營安平四州軍事、大將軍、尚書令、司空公、定州刺史。子裔嗣。



 尉長命,太安狄那人也。父顯,魏鎮遠將軍、代郡太守。長命性和厚,有器識。



 扶陽之亂,寄居太原。及高祖將建大義,長命參計策,從高祖破四胡於韓陵,拜安南將軍。樊子鵠據兗州反,除東南道大都督,與諸軍討平之。轉鎮范陽城,就拜幽州剌史,督安、平二州事。州居北垂,土荒民散,長命雖多聚斂,然以恩撫民,少得安集。尋以疾去
 職。未幾,復徵拜車騎大將軍、都督西燕幽滄瀛四州諸軍事、幽州刺史。卒於州。贈以本官,加司空,謚曰武壯。



 子興敬,便弓馬,有武藝,高祖引為帳內都督。出為常山公府參軍事,賜爵集中縣伯。晉州民李小興群聚為賊,興敬隨司空韓軌討平之,進爵為侯。高祖攻周文帝於邙山,興敬因戰為流矢所中,卒。贈涇、岐、豳三州軍事,爵為公,謚曰閔莊。



 高祖哀惜之,親臨弔,賜其妻子祿如興敬存焉。子士林嗣。



 王懷,字懷周,不知何許人也。少好弓馬,頗有氣尚,值北邊喪亂,早從戎旅。



 韓樓反於幽州,懷知其無成,陰結所親,以中興初叛樓歸魏,拜征虜將軍、第一領民酋長、武周縣侯。高祖東出,懷率其部人三千餘家,隨高祖於冀州。義旗建,高祖以為大都督,從討爾朱兆於廣阿,破之,除安北將軍、蔚州刺史。又隨高祖攻鄴,克之,從破四胡於韓陵,進爵為侯。仍從入洛,拜車騎將軍,改封盧鄉縣侯。天平中,除使持節、廣州軍事。梁遣將湛僧珍、楊暕來
 寇,懷與行臺元晏擊項城,拔之,擒暕。又從高祖襲克西夏州。還,為大都督,鎮下館,除儀同三司。元象初,為大都督,與諸將西討,遇疾卒於建州。贈定幽恒肆四州諸軍事、刺史、司徒公、尚書僕射。懷以武藝勳誠為高祖所知,志力未申,論者惜其不遂。皇建初,配饗高祖廟庭。



 劉貴,秀容陽曲人也。父乾,魏世贈前將軍、肆州刺史。貴剛格有氣斷,歷爾朱榮府騎兵參軍。建義初,以預定策勳,封敷城縣伯,邑五百戶。除左將軍、太中大夫,尋進為
 公。榮性猛急,貴尤嚴峻,每見任使,多愜榮心,遂被信遇,位望日重,加撫軍將軍。永安三年,除涼州刺史。建明初,爾朱世隆專擅,以貴為征南將軍、金紫光祿、兼左僕射、西道行臺,使抗孝莊行臺元顯恭於正平。貴破顯恭,擒之,并大都督裴俊等,復除晉州刺史。普泰初,轉行汾州事。高祖起義,貴棄城歸高祖於鄴。太昌初,以本官除肆州刺史,轉行建州事。天平初,除陜州刺史。四年,除御史中尉、肆州大中正。其年,加行臺僕射,與侯景、高昂
 等討獨孤如願於洛陽。



 貴凡所經歷,莫不肆其威酷。修營城郭,督責切峻,非理殺害,視下如草芥。



 然以嚴斷濟務,有益機速。性峭直,攻訐無所迴避,故見賞於時。雖非佐命元功,然與高祖布衣之舊,特見親重。興和元年十一月卒。贈冀定并殷瀛五州軍事、太保、太尉公、錄尚書事、冀州刺史,謚曰忠武。齊受禪,詔祭告其墓。皇建中,配享高祖廟庭。長子元孫,員外郎、肆州中正,早卒。贈肆州刺史。次子洪徽嗣。武平末,假儀同三司,奏門下事。



 任延敬,廣寧人也。伯父桃,太和初為雲中軍將,延敬隨之,因家焉。延敬少和厚,有器度。初從葛榮為賊,榮署為王,甚見委任。榮敗,延敬擁所部先降,拜鎮遠將軍、廣寧太守,賜爵西河縣公。後隨高祖建義,中興初,累遷光祿大夫。太昌初,累轉尚書左僕射,進位開府儀同三司。延敬位望既重,能以寬和接物,人士稱之。及斛斯椿釁發,延敬棄家北走,至河北郡,因率土民據之,以待高祖。



 魏武帝入關,荊蠻不順,以延敬為持節南道大都督,
 討平之。天平初,復拜侍中。時范陽人盧仲延率河北流人反於陽夏,西兗州民田龍聚眾應之,以延敬為大都督、東道軍司,率都督元整、叱列陀等討之。尋為行臺僕射,除徐州刺史。時梁遣元慶和及其諸將寇邊,延敬破梁仁州刺史黃道始於北濟陰,又破梁俊於單父,俘斬萬人。又拜侍中。在州大有受納。然為政不殘,禮敬人士,不為民所疾苦。



 潁州長史賀若徽執刺史田迅據城降西魏,復令延敬率豫州刺史堯雄等討之。西魏遣其將
 怡鋒率眾來援,延敬等與戰失利,收還北豫,仍與行臺侯景、司徒高昂等相會,共攻潁川,拔之。元象元年秋,卒於鄴,時年四十五。贈使持節、太保、太尉公、錄尚書事、都督冀定瀛幽安五州諸軍事、冀州刺史。子胄嗣。



 胄輕俠,頗敏惠。少在高祖左右,天平中,擢為東郡太守。家本豐財,又多聚斂,動極豪華,賓客往來,將迎至厚。尋以贓污為有司所劾,高祖舍之。及解郡,高祖以為都督。興和末,高祖攻玉壁還,以晉州西南重要,留清河公岳為行臺
 鎮守,以胄隸之。胄飲酒遊縱,不勤防守,高祖責之。胄懼,遂潛遣使送款於周。為人糾列,窮治未得其實,高祖特免之,謂胄曰:「我推誠於物,謂卿必無此理。且黑獺降人,首尾相繼,卿之虛實,於後何患不知。」胄內不自安。是時,儀同爾朱文暢及參軍房子遠、鄭仲禮等並險薄無賴,胄厚與交結,乃陰圖殺逆。武定三年正月十五日,因高祖夜戲,謀將竊發。有人告之,令捕窮治,事皆得實。胄及子弟並誅。



 莫多婁貸文,太安狄那人也。驍果有膽氣。從高祖舉義。中興初,除伏波將軍、武賁中郎將、虞候大都督。從擊爾朱兆於廣阿,有功,加前將軍,封石城縣子,邑三百戶。又從破四胡於韓陵,進爵為侯。從平爾朱兆於赤谼嶺。兆窮迫自經,貸文獲其屍。遷左廂大都督。斛斯椿等釁起,魏武帝遣賈顯智據守石濟。高祖令貸文率精銳三萬,與竇泰等於定州相會,同趣石濟,擊走顯智。天平中,除晉州刺史。汾州胡賊為寇竊,高祖親討焉,以貸文為先
 鋒,每有戰功。還,賚奴婢三十人、牛馬各五十匹、布一千匹,仍為汾、陜、東雍、晉、泰五州大都督。後與太保尉景攻東雍、南汾二州,克之。元象初,除車騎大將軍、儀同、南道大都督,與行臺侯景攻獨孤如願於金墉城。周文帝軍出函谷,景與高昂議整旅厲卒,以待其至。貸文請率所部,擊其前鋒,景等固不許。貸文性勇而專,不肯受命,以輕騎一千軍前斥候,西過瀍澗,遇周軍,戰沒。贈并肆恒雲朔五州軍事、并州刺史、尚書右僕射、司徒公。



 子敬顯,彊
 直勤幹,少以武力見知。恒從斛律光征討,數有戰功。光每命敬顯前驅,安置營壘,夜中巡察,或達旦不睡。臨敵置陳,亦令敬顯部分將士,造次之間,行伍整肅。深為光所重。位至領軍將軍,恒檢校虞候事。武平中,車駕幸晉陽,每令敬顯督留臺兵馬,糾察盜賊,京師肅然。七年,從後主平陽,敗歸并州,與唐邕等推立安德王稱尊號。安德敗,文武群官皆投周軍,唯敬顯走還鄴。授司徒。周武帝平鄴城之明日,執敬顯,斬於閶闔門外,責其不留晉
 陽也。



 高市貴,善無人也。少有武用。孝昌初,恒州內部敕勒劉崙等聚眾反,市貴為都督,率眾討崙,一戰破之。累遷撫軍將軍、諫議大夫。及爾朱榮立魏莊帝,高貴預翼戴之勳,遷衛將軍、光祿大夫、秀容大都督、第一領民酋長,賜爵上洛縣伯。



 爾朱榮擊葛榮於滏口,以市貴為前鋒都督。榮平,除使持節、汾州刺史,尋為晉州刺史。紇豆陵步藩之侵亂并州也,高祖破之,市貴亦從行有功,除
 驃騎大將軍、儀同三司,封常山郡公,邑一千五百戶。高祖起義,市貴預其謀。及樊子鵠據州反,隨大都督婁昭討之。子鵠平,除西兗州刺史,不之州。天平初,復除晉州刺史。高祖尋以洪峒要險,遣市貴鎮之。高祖沙苑失利,晉州行事封祖業棄城而還,州民柴覽聚眾作逆。高祖命市貴討覽,覽奔柴壁,市貴破斬之。是時,東雍、南汾二州境多群賊,聚為盜,因市貴平覽,皆散歸復業。後秀容人五千戶叛應山胡,復以市貴為行臺,統諸軍討平之。
 元象中,從高祖破周文帝於邙山。重除晉州刺史、西道軍司,率眾擊懷州逆賊潘集。未至,遇疾道卒。贈并汾懷建東雍五州軍事、太尉公、并州刺史。子可那肱貴寵,封成皋王。敕令其第二子孔雀承襲。



 厙狄回洛,代人也。少有武力,儀貌魁偉。初事爾朱榮為統軍,預立莊帝,轉為別將,賜爵毋極伯。從破葛榮,轉都督。榮死,隸爾朱兆。高祖舉兵信都,迴洛擁眾歸義。從破四胡於韓陵,以軍功補都督,加後將軍、太中大夫,封順
 陽縣子,邑四百戶。遷右廂都督。從征山胡,先鋒斬級,除朔州刺史。破周文於河陽,轉授夏州刺史。邙山之役,力戰有功,增邑通前七百戶。世宗嗣事,從平潁川。天保初,除建州刺史。肅宗即位,封順陽郡王。大寧初,轉朔州刺史,食博陵郡幹。轉太子太師,遇疾卒。贈使持節、都督定瀛恒朔雲五州軍事、大將軍、太尉公、定州刺史,贈物一千段。



 厙狄盛,懷朔人也。性和柔,少有武用。初為高祖親信都
 督,除伏波將軍,每從征討。以功封行唐縣伯,復累加安北將軍,幽州刺史,加中軍將軍,為豫州鎮城都督。以勳舊進爵為公,世宗減封二百戶,以增其邑。除征西大將軍、開府儀同三司、朔州刺史。齊受禪,改封華陽縣公。又除北朔州刺史,以華陽封邑在遠,隨例割并州之石艾縣、肆州之平寇縣、原平之馬邑縣各數十戶,合二百戶為其食邑。未幾,例罷,拜特進,卒。贈使持節、都督朔瀛趙幽安五州諸軍事、太尉公、朔州刺史。



 薛孤延,代人也。少驍果,有武力。韓樓之反,延隨眾屬焉。後與王懷等密計討樓,為樓尉帥乙弗醜所覺,力戰破醜,遂相率歸。行臺劉貴表為都督,加征虜將軍,賜爵永固縣侯。後隸高祖為都督,仍從起義。破爾朱兆於廣阿,因從平鄴,以功進爵為公,轉大都督。從破四胡於韓陵,加金紫光祿大夫。從追爾朱兆於赤谼嶺,除第一領民酋長。孝靜立,拜顯州刺史,累加車騎將軍。天平四年,從高祖西伐。



 至蒲津,竇泰於河南失利,高祖班師,延殿後,
 且戰且行,一日斫折刀十五口。還,轉梁州刺史。從征玉壁,又轉恒州刺史。從破周文帝於邙山,進爵為縣公,邑一千戶。



 高祖嘗閱馬於北牧,道逢暴雨,大雷震地。前有浮圖一所,高祖令延視之。延乃馳馬按槊直前,未至三十步,雷火燒面,延喝殺,繞浮圖走,火遂滅。延還,眉鬢及馬鬃尾俱焦。高祖歎曰:「薛孤延乃能與霹靂鬥!」其勇決如此。



 又頻從高祖討破山胡,西攻玉壁。入為左衛將軍,改封平秦郡公。為左廂大都督,與諸軍將討潁州。延專
 監造土山,以酒醉為敵所襲據。潁州平,諸將還京師,宴於華林園。世宗啟魏帝,坐延於階下以辱之。後兼領軍將軍,出為滄州刺史,別封溫縣男,邑三百戶。齊受禪,別賜爵都昌縣公。性好酒,率多昏醉。而以勇決善戰,每大軍征討,常為前鋒,故與彭、劉、韓、潘同列。天保二年,為太子太保,轉太子太傅。八年,除肆州刺史,加開府儀同三司,食洛陽郡幹,尋改食河間郡幹。



 張保洛,代人也,自云本出南陽西鄂。家世好賓客,尚氣
 俠,頻為北土所知。



 保洛少率健,善弓馬。魏孝昌中,北鎮擾亂,保洛亦隨眾南下。葛榮僭逆,以保洛為領左右。榮敗,仍為爾朱榮統軍,累遷揚烈將軍、奉車都尉。後隸高祖為都督,從討步蕃。及高祖起義,保洛為帳內,從破爾朱兆於廣阿。尋遷右將軍、中散大夫,仍以帳內從高祖圍鄴城,既拔,除平南將軍、光祿大夫。從破爾朱兆等於韓陵,因隨高祖入洛,加安東將軍。後高祖啟減國邑,分授將士,保洛隨例封昌平縣薄家城鄉男一百戶。



 魏出帝不協於高祖,令儀同賈顯智率豫州刺史斛斯壽東趣濟州。高祖遣大都督竇泰濟自滑臺拒顯智,保洛隸泰前驅。事定,轉都督。從高祖襲夏州,克之。萬俟受洛干之降也,高祖遣保洛與諸將於路接援。元象初,除西夏州刺史、當州大都督,又以前後功,封安武縣伯,邑四百戶。轉行蔚州刺史。從高祖攻周文帝於邙山,圍玉壁,攻龍門。還,留鎮晉州。



 世宗即位,以保洛為左廂大都督。後出晉州,加征西將軍。王思政之援潁州,攻圍未克。
 世宗仍令保洛鎮楊志塢,使與陽州為掎角之勢。潁川平,尋除梁州刺史。



 顯祖受禪,仍為刺史,所在聚斂為務,民吏怨之。濟南初,出為滄州刺史,封敷城郡王。為在州聚斂,免官,削奪王爵。及卒,贈以前官,追復本封。子默言嗣。武平末,衛將軍。



 以帳內從高祖出山東,又有曲珍、段琛、牒舍樂、尉摽、乞伏貴和及弟令和、王康德,並以軍功至大官。



 曲珍字舍洛,西平酒泉人也。壯勇善騎射。以帳內從高祖晉州,仍起義,所在征討。武定末,封富平縣伯。
 天保初,食黎陽郡幹,除晉州刺史。武平初,遷豫州道行臺、尚書令、豫州刺史,卒,贈太尉。



 段琛字懷寶,代人也。少有武用。從高祖起義信都。天保中,兗州刺史。



 牒舍樂,武成開府儀同三司、營州刺史,封漢中郡公。戰歿關中。



 尉摽,代人也。大寧初,封海昌王。子相貴嗣。相貴,武平末晉州道行臺尚書僕射、晉州刺史。為行臺左丞侯子欽等密啟周武請帥,欽等為內應。周武自率眾至城下,欽等夜開城門引軍入,鎖相貴送長安。尋卒。弟相願,彊幹
 有膽略。武平末,領軍大將軍。自平陽至并州,及到鄴,每立計將殺高阿那肱,廢後主,立廣寧王,事竟不果。及廣寧被出,相願拔佩刀斫柱而歎曰:「大事去矣,知復何言!」



 貴和及令和兄弟,武平末,並開府儀同三司。令和,領軍將軍。并州未敗前,與領軍大將軍韓建業、武衛大將軍封輔相相繼投周軍。令和授柱國,封西河郡公。



 隋大業初,卒於秦州總管。建業、輔相,俱不知所從來。建業授上柱國,封郇國公,隋開皇中卒。輔相,上柱國,封郡公。周
 武平並州,即以為朔州總管。



 康德,代人也。歷數州刺史、并省尚書,封新蔡郡王。



 侯莫陳相,代人也。祖伏頹,魏第一領民酋長。父斛古提,朔州刺史、白水郡公。



 尋除蔚州刺史,仍為大行臺,節度西道諸軍事。又遷車騎將軍,顯州刺史。入除太僕卿。頃之,出為汾州刺史。別封安次縣男,又別封始平縣公。天保初,除太師,轉司空公,進爵為白水王,邑一千一百戶。累授太傅,進食建州乾,別封義寧郡公。武平二年四月,
 薨於州,年八十三。贈假黃鉞、使持節、督冀定瀛滄濟趙幽并朔恒十州軍事、右丞相、太宰、太尉公、朔州刺史。有二子。長子貴樂,尚公主,駙馬都尉。次子晉貴,武衛將軍、梁州刺史。隆化時,并州失守,晉貴遣使降周,授上大將軍,封信安縣公。



 史臣曰:高祖世居雲代,以英雄見知。後過爾朱,武功漸振,鄉邑故人,彌相推重。賀拔允以昆季乖離,處猜嫌之地,初以舊望矜護,而竟不獲令終,比於吳、蜀之安瑾、亮,
 方知器識之淺深也。劉貴、蔡俊有先見之明,霸業始基,義深匡贊,配饗清廟,豈徒然哉。韓賢等及聞義舉,競趣戎行,憑附末光,申其志力,化為公侯,固其宜矣。



 贊曰:帝鄉之親,世有其人。降靈雲朔,載挺良臣。功名之地,望古為鄰。



\end{pinyinscope}