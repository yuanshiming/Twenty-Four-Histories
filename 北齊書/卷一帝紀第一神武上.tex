\article{卷一帝紀第一神武上}

\begin{pinyinscope}

 齊高祖神武皇帝,姓高名歡,字賀六渾,渤海蓚人也。六世祖隱,晉玄菟太守。



 隱生慶,慶生泰,泰生湖,三世仕慕容氏。及慕容寶敗,國亂,湖率眾歸魏,為右將軍。湖生四
 子,第三子謐,仕魏,位至侍御史,坐法徙居懷朔鎮。謐生皇考樹,性通率,不事家業。住居白道南,數有赤光紫氣之異,鄰人以為怪,勸徙居以避之。



 皇考曰:「安知非吉?」居之自若。及神武生而皇妣韓氏殂,養於同產姊婿鎮獄隊尉景家。



 神武既累世北邊,故習其俗,遂同鮮卑。長而深沉有大度,輕財重士,為豪俠所宗。目有精光,長頭高顴,齒白如玉,少有人傑表。家貧,及聘武明皇后,始有馬,得給鎮為隊主。鎮將遼西段長常奇神武貌,謂曰:「君有
 康濟才,終不徒然。」



 便以子孫為託。及貴,追贈長司空,擢其子寧用之。神武自隊主轉為函使。嘗乘驛過建興,雲霧晝晦,雷聲隨之,半日乃絕,若有神應者。每行道路,往來無風塵之色。又嘗夢履眾星而行,覺而內喜。為函使六年,每至洛陽,給令史麻祥使。詳嘗以肉啖神武,神武性不立食,坐而進之。祥以為慢己,笞神武四十。及自洛陽還,傾產以結客。親故怪問之,答曰:「吾至洛陽,宿衛羽林相率焚領軍張彞宅,朝廷懼其亂而不問。為政若此,
 事可知也。財物豈可常守邪?」自是乃有澄清天下之志。



 與懷朔省事雲中司馬子如及秀容人劉貴、中山人賈顯智為奔走之友,懷朔戶曹史孫騰、外兵史侯景亦相友結。劉貴嘗得一白鷹,與神武及尉景、蔡俊、子如、賈顯智等獵於沃野。見一赤兔,每搏輒逸,遂至迴澤。澤中有茅屋,將奔入,有狗自屋中出,噬之,鷹兔俱死。神武怒,以鳴鏑射之,狗斃。屋中有二人出,持神武襟甚急。



 其母兩目盲,曳杖呵其二子曰:「何故觸大家!」出甕中酒,烹羊以
 飯客。因自言善暗相,遍捫諸人皆貴,而指麾俱由神武。又曰:「子如歷位,顯智不善終。」飯竟出,行數里還,更訪之,則本無人居,乃向非人也。由是諸人益加敬異。



 孝昌元年,柔玄鎮人杜洛周反於上谷,神武乃與同志從之。醜其行事,私與尉景、段榮、蔡俊圖之。不果而逃,為其騎所追。文襄及魏永熙后皆幼,武明后於牛上抱負之。文襄屢落牛,神武彎弓將射之以決去。后呼榮求救,賴榮遽下取之以免。



 遂奔葛榮,又亡歸爾朱榮於秀容。先是,劉
 貴事榮,盛言神武美,至是始得見,以憔悴故,未之奇也。貴乃為神武更衣,復求見焉。因隨榮之廄。廄有惡馬,榮命翦之。神武乃不加羈絆而翦,竟不蹄齧,已而起曰:「御惡人亦如此馬矣。」榮遂坐神武於床下,屏左右而訪時事。神武曰:「聞公有馬十二谷,色別為群,將此竟何用也?」榮曰:「但言爾意。」神武曰:「方今天子愚弱,太后淫亂,孽寵擅命,朝政不行。以明公雄武,乘時奮發,討鄭儼、徐紇而清帝側,霸業可舉鞭而成。此賀六渾之意也。」榮大悅,語
 自日中至夜半,乃出。自是每參軍謀。後從榮徙據并州,抵揚州邑人龐蒼鷹,止團焦中。每從外歸,主人遙聞行響動地。蒼鷹母數見團焦赤氣赫然屬天。又蒼鷹嘗夜欲入,有青衣人拔刀叱曰:「何故觸王!」言訖不見。



 始以為異,密覘之,唯見赤蛇蟠床上,乃益驚異。因殺牛分肉,厚以相奉。蒼鷹母求以神武為義子。及得志,以其宅為第,號為南宅。雖門巷開廣,堂宇崇麗,其本所住團焦,以石堊塗之,留而不毀,至文宣時,遂為宮。



 既而榮以神武為
 親信都督。于時魏明帝銜鄭儼、徐紇,逼靈太后,未敢制,私使榮舉兵內向。榮以神武為前鋒。至上黨,明帝又私詔停之。及帝暴崩,榮遂入洛,因將篡位。神武諫,恐不聽,請鑄像卜之,鑄不成,乃止。孝莊帝立,以定策勳,封銅鞮伯。及爾朱榮擊葛榮,令神武喻下賊別稱王者七人。後與行臺于暉破羊侃于泰山,尋與元天穆破邢杲于濟南。累遷第三鎮人酋長,常在榮帳內。榮嘗問左右曰:「一日無我,誰可主軍?」皆稱爾朱兆。曰:「此正可統三千騎以
 還,堪代我主眾者,唯賀六渾耳。」因誡兆曰:「爾非其匹,終當為其穿鼻。」乃以神武為晉州刺史。於是大聚斂,因劉貴貨榮下要人,盡得其意。時州庫角無故自鳴,神武異之,無幾而孝莊誅榮。



 及爾朱兆自晉陽將舉兵赴洛,召神武。神武使長史孫騰辭以絳蜀、汾胡欲反,不可委去。兆恨焉。騰復命,神武曰:「兆舉兵犯上,此大賊也,吾不能久事之。」



 自是始有圖兆計。及兆入洛,執莊帝以北,神武聞之,大驚。又使孫騰偽賀兆,因密覘孝莊所在,將劫
 以舉義,不果。乃以書喻之,言不宜執天子以受惡名於海內。



 兆不納,殺帝,而與爾朱世隆等立長廣王曄,改元建明。封神武為平陽郡公。及費也頭紇豆陵步藩入秀容,逼晉陽,兆徵神武。神武將往,賀拔焉過兒請緩行以弊之。



 神武乃往往逗遛,辭以河無橋不得渡。步藩軍盛,兆敗走。初,孝莊之誅爾朱榮,知其黨必有逆謀,乃密敕步藩令襲其後。步藩既敗兆等,以兵勢日盛,兆又請救於神武。神武內圖兆,復慮步藩後之難除,乃與兆悉力破
 之。藩死,深德神武,誓為兄弟。時世隆、度律、彥伯共執朝政,天光據關右,兆據并州,仲遠據東郡,各擁兵為暴,天下苦之。



 葛榮眾流入并、肆者二十餘萬,為契胡陵暴,皆不聊生,大小二十六反,誅夷者半,猶草竊不止。兆患之,問計於神武。神武曰:「六鎮反殘,不可盡殺,宜選王素腹心者私使統焉。若有犯者,直罪其帥,則所罪者寡。」兆曰:「善,誰可行也?」賀拔允時在坐,請神武。神武拳毆之,折其一齒,曰:「生平天柱時,奴輩伏處分如鷹犬,今日天下安
 置在王,而阿鞠泥敢誣下罔上,請殺之。」兆以神武為誠,遂以委焉。神武以兆醉,恐醒後或致疑貳,遂出,宣言受委統州鎮兵,可集汾東受令。乃建牙陽曲川,陳部分。有款軍門者,絳巾袍,自稱梗楊驛子,願廁左右。



 訪之,則以力聞,常於并州市搭殺人者,乃署為親信。兵士素惡兆而樂神武,於是莫不皆至。居無何,又使劉貴請兆,以并、肆頻歲霜旱,降戶掘黃鼠而食之,皆面無穀色,徒污人國土,請令就食山東,待溫飽而處分之。兆從其議。
 其長史慕容紹宗諫曰:「不可,今四方擾擾,人懷異望,況高公雄略,又握大兵,將不可為。」



 兆曰:「香火重誓,何所慮也。」紹宗曰:「親兄弟尚爾難信,何論香火!」時兆左右已受神武金,因譖紹宗與神武舊有隙,兆乃禁紹宗而催神武發。神武乃自晉陽出滏口。路逢爾朱榮妻北鄉長公主,自洛陽來,馬三百匹,盡奪易之。兆聞,乃釋紹宗而問焉。紹宗曰:「猶掌握中物也。」於是自追神武。至襄垣,會漳水暴長,橋壞。神武隔水拜曰:「所以借公主馬,非有他故,
 備山東盜耳。王受公主言,自來賜追,今渡河而死不辭,此眾便叛。」兆自陳無此意,因輕馬渡,與神武坐幕下,陳謝,遂授刀引頭,使神武斫己。神武大哭曰:「自天柱薨背,賀六渾更何所仰,願大家千萬歲,以申力用。今旁人構間至此,大家何忍復出此言!」兆投刀於地,遂刑白馬而盟,誓為兄弟,留宿夜飲。尉景伏壯士欲執兆,神武嚙臂止之曰:「今殺之,其黨必奔歸聚結。兵饑馬瘦,不可相支,若英雄崛起,則為害滋甚。不如且置之。兆雖勁捷,而兇
 狡無謀,不足圖也。」旦日,兆歸營,又召神武,神武將上馬詣之,孫騰牽衣,乃止。兆隔水肆罵,馳還晉陽。兆心腹念賢領降戶家累別為營,神武偽與之善,觀其佩刀,因取之以殺其從者,從者盡散。於是士眾咸悅,倍願附從。初,魏真君內學者奏言上黨有天子氣,云在壺關大王山。太武帝於是南巡以厭當之,累石為三封,斬其北鳳凰山,以毀其形。後上黨人居晉陽者,號上黨坊,神武實居之。及是行,舍大王山六旬而進。將出滏口,倍加約束,纖毫
 之物,不聽侵犯。將過麥地,神武輒步牽馬。遠近聞之,皆稱高儀同將兵整肅,歸心焉。遂前行,屯鄴,求糧相州刺史劉誕,誕不供。有軍營租米,神武自取之。



 魏普泰元年二月,神武自軍次信都,高乾、封隆之開門以待,遂據冀州。是月,爾朱度律廢元曄而立節閔帝,欲羈縻神武。三月,乃白節閔帝,封神武為渤海王,徵使入覲。神武辭。四月癸巳,又加授東道大行臺、第一鎮人酋長。龐蒼鷹自太原來奔,神武以為行臺郎,尋以為安州刺史。神武自
 向山東,養士繕甲,禁侵掠,百姓歸心。乃詐為書,言爾朱兆將以六鎮人配契胡為部曲,眾皆愁怨。又為并州符,徵兵討步落稽。發萬人,將遣之,孫騰、尉景為請留五日,如此者再。神武親送之郊,雪涕執別,人皆號慟,哭聲動地。神武乃喻之曰:「與爾俱失鄉客,義同一家,不意在上乃爾徵召。直向西已當死,後軍期又當死,配國人又當死,奈何!」眾曰:「唯有反耳!」神武曰:「反是急計,須推一人為主。」眾願奉神武。神武曰:「爾鄉里難制,不見葛榮乎?雖百
 萬眾,無刑法,終自灰滅。今以吾為主,當與前異,不得欺漢兒,不得犯軍令,生死任吾則可,不爾不能為,取笑天下。」眾皆頓顙,死生唯命。神武曰若不得已。明日,椎牛饗士,喻以討爾朱之意。封隆之進曰:「千載一時,普天幸甚。」神武曰:「討賊,大順也;拯時,大業也。吾雖不武,以死繼之,何敢讓焉!」



 六月庚子,建義於信都,尚未顯背爾朱氏。及李元忠與高乾平殷州,斬爾朱羽生首來謁,神武撫膺曰:「今日反決矣。」乃以元忠為殷州刺史。是時兵威既振,
 乃抗表罪狀爾朱氏。世隆等祕表不通。八月,爾朱兆攻陷殷州,李元忠來奔。孫騰以為朝廷隔絕,不權立天子,則眾望無所係。十月壬寅,奉章武王融子渤海太守朗為皇帝,年號中興,是為廢帝。時度律、仲遠軍次陽平,爾朱兆會之。神武用竇泰策,縱反間,度律、仲遠不戰而還。神武乃敗兆於廣阿。十一月,攻鄴,相州刺史劉誕嬰城固守。神武起土山,為地道,往往建大柱,一時焚之,城陷入地。麻祥時為湯陰令,神武呼之曰:「麻都!」祥慚而逃。永熙
 元年正月壬午,拔鄴城,據之。



 廢帝進神武大丞相、柱國大將軍、太師。是時青州建義,大都督崔靈珍、大都督耿翔皆遣使歸附。行汾州事劉貴棄城來降。閏三月,爾朱天光自長安、兆自并州、度律自洛陽、仲遠自東郡同會鄴,眾號二十萬,挾洹水而軍,節閔以長孫承業為大行臺總督焉。神武令封隆之守鄴,自出頓紫陌。時馬不滿二千,步兵不至三萬,眾寡不敵。乃於韓陵為圓陣,連牛驢以塞歸道,於是將士皆有死志,四面赴擊之。爾朱
 兆責神武以背己,神武曰:「本戮力者,共輔王室,今帝何在?」兆曰:「永安枉害天柱,我報仇耳。」神武曰:「我昔日親聞天柱計,汝在戶前立,豈得言不反邪?



 且以君殺臣,何報之有?今日義絕矣。」乃合戰,大敗之。爾朱兆對慕容紹宗叩心曰:「不用公言,以至於此!」將輕走。紹宗反旗鳴角,收聚散卒,成軍容而西上。



 高季式以七騎追奔,度野馬崗,與兆遇。高昂望之不見,哭曰:「喪吾弟矣!」夜久,季式還,血滿袖。斛斯椿倍道先據河橋。初,普泰元年十月,歲星、熒
 惑、鎮星、太白聚於觜,參色甚明。太史占云當有王者興。是時神武起於信都,至是而破兆等。四月,斛斯椿執天光、度律送洛陽。長孫承業遣都督賈顯智、張歡入洛陽,執世隆、彥伯斬之。兆奔并州。仲遠奔梁州,遂死焉。時凶蠹既除,朝廷慶悅。初,未戰之前月,章武人張紹夜中忽被數騎將踰城,至一大將軍前,敕紹為軍導向鄴,云佐受命者除殘賊。紹迴視之,兵不測,整疾無聲。將至鄴,乃放焉。及戰之日,爾朱氏軍人見陣外士馬四合,蓋神助
 也。



 既而神武至洛陽,廢節閔及中興主而立孝武。孝武既即位,授神武大丞相、天柱大將軍、太師、世襲定州刺史,增封并前十五萬戶。神武辭天柱,減戶五萬。壬辰,還鄴,魏帝餞於乾脯山,執手而別。



 七月壬寅,神武帥師北伐爾朱兆。封隆之言:「侍中斛斯椿、賀拔勝、賈顯智等往事爾朱,普皆反噬,今在京師,寵任,必構禍隙。」神武深以為然,乃歸天光、度律於京師,斬之。遂自滏口入。爾朱兆大掠晉陽,北保秀容。并州平。神武以晉陽四塞,乃建大
 丞相府而定居焉。爾朱兆既至秀容,分兵守險,出入寇抄。神武揚聲討之,師出止者數四,兆意怠。神武揣其歲首當宴會,遣竇泰以精騎馳之,一日一夜行三百里,神武以大軍繼之。二年正月,竇泰奄至爾朱兆庭。軍人因宴休惰,忽見泰軍,驚走。追破之於赤洪嶺。兆自縊,神武親臨厚葬之。慕容紹宗以爾朱榮妻子及餘眾自保烏突城,降,神武以義故,待之甚厚。



 神武之入洛也,爾朱仲遠部下都督橋寧、張子期自滑臺歸命,神武以其助亂,
 且數反覆,皆斬之。斛斯椿由是內不自安,乃與南陽王寶炬及武衛將軍元毗、魏光、王思政構神武於魏帝。舍人元士弼又奏神武受敕大不敬。故魏帝心貳於賀拔岳。初,孝明之時,洛下以兩拔相擊,謠言曰:「銅拔打鐵拔,元家世將末。」好事者以二拔謂拓拔、賀拔,言俱將衰敗之兆。時司空高乾密啟神武,言魏帝之貳,神武封呈。



 魏帝殺之,又遣東徐州刺史潘紹業密敕長樂太守龐蒼鷹令殺其弟昂。昂先聞其兄死,以槊刺柱,伏壯士執
 紹業於路,得敕書於袍領,來奔。神武抱其首,哭曰:「天子枉害司空!」遽使以白武幡勞其家屬。時乾次弟慎在光州,為政嚴猛,又從部下取納,魏帝使代之。慎聞難,將奔梁。其屬曰:「公家勳重,必不兄弟相及。」乃弊衣推鹿車歸渤海。逢使者,亦來奔。於是魏帝與神武隙矣。



 阿至羅虜正光以前常稱藩,自魏朝多事,皆叛。神武遣使招納,便附款。先是,詔以寇賊平,罷行臺。至是,以殊俗歸降,復授神武大行臺,隨機處分。神武常賚其粟帛,議者以為徒
 費無益,神武不從,撫慰如初。其酋帥吐陳等感恩,皆從指麾,救曹泥,取萬俟受洛干,大收其用。河西費也頭虜紇豆陵伊利居河池,恃險擁眾,神武遣長史侯景屢招不從。



\end{pinyinscope}