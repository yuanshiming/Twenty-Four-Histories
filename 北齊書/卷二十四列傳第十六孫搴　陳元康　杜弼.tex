\article{卷二十四列傳第十六孫搴 陳元康 杜弼}

\begin{pinyinscope}

 孫搴,字彥舉,樂安人也。少厲志勤學,自檢校御史再遷
 國子助教。太保崔光引修國史,頻歷行臺郎,以文才著稱。崔祖螭反,搴預焉,逃於王元景家,遇赦乃出。孫騰以宗情薦之,未被知也。會高祖西討,登風陵,命中外府司馬李義深、相府城局李士略共作檄文,二人皆辭,請以搴自代。高祖引搴入帳,自為吹火,催促之。搴援筆立成,其文甚美。高祖大悅,即署相府主簿,專典文筆。又能通鮮卑語,兼宣傳號令。當煩劇之任,大見賞重。賜妻韋氏,既士人子女,又兼色貌,時人榮之。尋除左光祿大夫,常
 領主簿。



 世宗初欲之鄴,總知朝政,高祖以其年少未許。搴為致言,乃果行。恃此自乞特進,世宗但加散騎常侍。時又大括燕、恒、雲、朔、顯、蔚、二夏州、高平、平涼之民以為軍士,逃隱者身及主人、三長、守令罪以大辟,沒入其家。於是所獲甚眾,搴之計也。



 搴學淺而行薄,邢邵嘗謂之曰:「更須讀書。」搴曰:「我精騎三千,足敵君羸卒數萬。」嘗服棘刺丸,李諧等調之曰:「卿棘刺應自足,何假外求。「坐者皆笑。司馬子如與高季式召搴飲酒,醉甚而卒,時年五
 十二。高祖親臨之。子如叩頭請罪,高祖曰:「折我右臂,仰覓好替還我。」子如舉魏收、季式舉陳元康,以繼搴焉。贈儀同三司、吏部尚書、青州刺史。



 陳元康,字長猷,廣宗人也。父終德,魏濟陰內史,終於鎮南將軍、金紫光祿大夫。元康貴,贈冀州刺史,謚曰貞。元康頗涉文史,機敏有乾用。魏正光五年,從尚書令李崇北伐,以軍功賜爵臨清縣男。普泰中,除主書,加威烈將軍。天平元年,修起居注。二年,遷司徒府記室參軍,尤為
 府公高昂所信。後出為瀛州開府司馬,加輔國將軍。所歷皆為稱職,高祖聞而徵焉。稍被任使,以為相府功曹參軍,內掌機密。



 高祖經綸大業,軍務煩廣,元康承受意旨,甚濟速用。性又柔謹,通解世事。



 高祖嘗怒世宗於內,親加毆蹋,極口罵之。出以告元康,元康諫曰:「王教訓世子,自有禮法,儀刑式瞻,豈宜至是。」言辭懇懇,至于流涕。高祖從此為之懲忿。時或恚撻,輒曰:「勿使元康知之。」其敬憚如此。高仲密之叛,高祖知其由崔暹故也,將殺暹。
 世宗匿而為之諫請。高祖曰:「我為舍其命,須與苦手。」世宗乃出暹而謂元康曰:「卿若使崔得杖,無相見也。」暹在廷,解衣將受罰,元康趨入,歷階而昇,且言曰:「王方以天下付大將軍,有一崔暹不能容忍耶?」高祖從而宥焉。世宗入輔京室,崔暹、崔季舒、崔昂等並被任使,張亮、張徽纂並高祖所待遇,然委任皆出元康之下。時人語曰:「三崔二張,不如一康。」魏尚書僕射范陽盧道虞女為右衛將軍郭瓊子婦,瓊以死罪沒官,高祖啟以賜元康為妻,
 元康乃棄故婦李氏,識者非之。元康便辟善事人,希顏候意,多有進舉,而不能平心處物,溺於財利,受納金帛,不可勝紀,放責交易,遍於州郡,為清論所譏。



 從高祖破周文帝於邙山,大會諸將,議進退之策。咸以為野無青草,人馬疲瘦,不可遠追。元康曰:「兩雄交戰,歲月已久,今得大捷,便是天授,時不可失,必須乘勝追之。」高祖曰:「若遇伏兵,孤何以濟?」元康曰:「王前涉沙苑還軍,彼尚無伏,今奔敗若此,何能遠謀。若舍而不追,必成後患。」高祖竟
 不從。以功封安平縣子,邑三百戶。尋除平南將軍、通直常侍,轉大行臺郎中,徙右丞。及高祖疾篤,謂世宗曰:「邙山之戰,不用元康之言,方貽汝患。以此為恨,死不瞑目。」



 高祖崩,秘不發喪,唯元康知之。



 世宗嗣事,又見任待。拜散騎常侍、中軍將軍,別封昌國縣公,邑一千戶。侯景反,世宗逼於諸將,欲殺崔暹以謝之,密語元康。元康諫曰:「今四海未清,綱紀已定,若以數將在外,茍悅其心,枉殺無辜,虧廢刑典,豈直上負天神,何以下安黎庶?晁錯前
 事,願公慎之。」世宗乃止。高岳討侯景未剋,世宗欲遣潘相樂副之。元康曰:「相樂緩於機變,不如慕容紹宗,且先王有命,稱其堪敵侯景,公但推赤心於此人,則侯景不足憂也。」是時紹宗在遠,世宗欲召見之,恐其驚叛。元康曰:「紹宗知元康特蒙顧待,新使人來餉金,以致其誠款。元康欲安其意,故受之而厚答其書。保無異也。」世宗乃任紹宗,遂以破景。賞元康金五十斤。王思政入潁城,諸將攻之不能拔,元康進計於世宗曰:「公匡輔朝政,未有
 殊功,雖敗侯景,本非外賊。今潁城將陷,原公因而乘之,足以取威定業。」世宗令元康馳驛觀之。復命曰:「必可拔。」世宗於是親征,既至而剋,賞元康金百鋌。



 初,魏朝授世宗相國、齊王,世宗頻讓不受。乃召諸將及元康等密議之,諸將皆勸世宗恭應朝命,元康以為未可。又謂魏收曰:「觀諸人語,專欲誤王。我向已啟王,受朝命,置官僚,元康叨忝或得黃門郎,但時事未可耳。」崔暹因間之,薦陸元規為大行臺郎,欲以分元康權也。元康既貪貨賄,世
 宗內漸嫌之,元康頗亦自懼。又欲用為中書令,以閑地處之,事未施行。



 屬世宗將受魏禪,元康與楊愔、崔季舒並在世宗坐,將大遷除朝士,共品藻之。



 世宗家蒼頭奴蘭固成先掌廚膳,甚被寵暱。先是,世宗杖之數十,其人性躁,又恃舊恩,遂大忿恚,與其同事阿改謀害世宗。阿改時事顯祖,常執刀隨從,云若聞東齋叫聲,即以加刃於顯祖。是日值魏帝初建東宮,群官拜表。事罷,顯祖出東止車門,別有所之,未還而難作。固成因進食,置刀於
 盤下而殺世宗。元康以身扞蔽,被刺傷重,至夜而終,時年四十三。楊愔狼狽走出,季舒逃匿於廁,庫直紇奚舍樂扞賊死。是時秘世宗凶問,故殯元康於宮中,託以出使南境,虛除中書令。明年,乃詔曰:「元康識超往哲,才極時英,千仞莫窺,萬頃難測。綜核戎政,彌綸霸道,草昧邵陵之謀,翼贊河陽之會,運籌定策,盡力盡心,進忠補過,亡家徇國,掃平逋寇,廓清荊楚,申、甫之在隆周,子房之處盛漢,曠世同規,殊年共美。大業未融,山隤奄及,悼傷
 既切,宜崇茂典。贈使持節、都督冀定瀛殷滄五州諸軍事、驃騎大將軍、司空公、冀州刺史,追封武邑縣一千戶,舊封並如故,謚曰文穆。賻物一千二百段。大鴻臚監喪事。凶禮所須,隨由公給。」元康母李氏,元康卒後,哀感發病而終,贈廣宗郡君,謚曰貞昭。



 元康子善藏,溫雅有鑒裁,武平末,假儀同三司、給事黃門侍郎。隋開皇中,尚書禮部侍郎。大業初,卒於彭城郡贊治。



 元康弟諶,官至大鴻臚。次季璩,巨鹿太守,轉冀州別駕。平秦王歸彥反,季
 璩守節不從,因而遇害。贈衛尉卿、趙州刺史。



 杜弼,字輔玄,中山曲陽人也,小字輔國。自序云,本京兆杜陵人,九世祖驁,晉散騎常侍,因使沒趙,遂家焉。祖彥衡,淮南太守。父慈度,繁畤令。弼幼聰敏,家貧無書,年十二,寄郡學受業,講授之祭,師每奇之。同郡甄琛為定州長史,簡試諸生,見而策問,義解閑明,應答如響,大為琛所歎異。其子寬與弼為友。州牧任城王澄聞而召問,深相嗟賞,許以王佐之才。澄、琛還洛,稱之於朝,丞相高陽
 王等多相招命。



 延昌中,以軍功起家,除廣武將軍、恒州征虜府墨曹參軍,典管記。弼長於筆札,每為時輩所推。孝昌初,除太學博士,帶廣陽王驃騎府法曹行參軍,行臺度支郎中。還,除光州曲城令。為政清靜,務盡仁恕,詞訟止息,遠近稱之。時天下多難,盜賊充斥,徵召兵役,途多亡叛,朝廷患之。乃令兵人所齎戎具,道別車載;又令縣令自送軍所。時光州發兵,弼送所部達北海郡,州兵一時散亡,唯弼所送不動。他境叛兵並來攻劫,欲與同
 去。弼率所領親兵格鬥,終莫肯從,遂得俱達軍所。



 軍司崔鐘以狀上聞。其得人心如此。普泰中,吏曹下訪守令尤異,弼已代還,東萊太守王昕以弼應訪。弼父在鄉,為賊所害,弼行喪六年。以常調除御史,加前將軍、太中大夫,領內正字。臺中彈奏,皆弼所為。諸御史出使所上文簿,委弼覆察,然後施行。



 遷中軍將軍、北豫州、驃騎大將軍府司馬。未之官,儀同竇泰總戎西伐,詔弼為泰監軍。及泰失利自殺,弼與其徒六人走還,陜州刺史劉貴鎖
 送晉陽。高祖詰之曰:「竇中尉此行,吾前具有法用,乃違吾語,自取敗亡。爾何由不一言諫爭也?」



 弼對曰:「刀筆小生,唯文墨薄技,便宜之事,議所不及。」高祖益怒。賴房謨諫而獲免。左遷下灌鎮司馬。



 元象初,高祖征弼為大丞相府法曹行參軍,署記室事,轉大行臺郎中,尋加鎮南將軍。高祖又引弼典掌機密,甚見信待。或有造次不及書教,直付空紙,即令宣讀。弼嘗承間密勸高祖受魏禪,高祖舉杖擊走之。相府法曹辛子炎諮事,云須取署,子
 炎讀「署」為「樹」。高祖大怒曰:「小人都不知避人家諱!」杖之於前。弼進曰:「《禮》,二名不偏諱,孔子言「徵」不言「在」,言「在」不言「徵」。子炎之罪,理或可恕。」高祖罵之曰:「眼看人瞋,乃復牽經引《禮》!」叱令出去。



 弼行十步許,呼還,子炎亦蒙釋宥。世子在京聞之,語楊愔曰:「王左右賴有此人方正,庶天下皆蒙其利,豈獨吾家也。」



 弼以文武在位,罕有廉潔,言之於高祖。高祖曰:「弼來,我語爾。天下濁亂,習俗已久。今督將家屬多在關西,黑獺常相招誘,人情去留未定。江
 東復有一吳兒老翁蕭衍者,專事衣冠禮樂,中原士大夫望之以為正朔所在。我若急作法網,不相饒借,恐督將盡投黑獺,士子悉奔蕭衍,則人物流散,何以為國?爾宜少待,吾不忘之。」及將有沙苑之役,弼又請先除內賊,卻討外寇。高祖問內賊是誰。弼曰:「諸勳貴掠奪萬民者皆是。」高祖不答,因令軍人皆張弓挾矢,舉刀按槊以夾道,使弼冒出其間,曰:「必無傷也。」弼戰慄汗流。高祖然後喻之曰:「箭雖注不射,刀雖舉不擊,槊雖按不刺,爾猶頓
 喪魂膽。諸勳人身觸鋒刃,百死一生,縱其貪鄙,所取處大,不可同之循常例也。」弼于時大恐,因頓顙謝曰:「愚癡無智,不識至理,今蒙開曉,始見聖達之心。」



 後從高祖破西魏於邙山,命為露布,弼手即書絹,曾不起草。以功賜爵定陽縣男,邑二百戶,加通直散騎常侍、中軍將軍。奉使詣闕,魏帝見之於九龍殿,曰:「朕始讀《莊子》,便值秦名,定是體道得真,玄同齊物。聞卿精學,聊有所問。



 經中佛性、法性為一為異?」弼對曰:「佛性法性,止是一理。」詔又問
 曰:「佛性既非法性,何得為一?」對曰:「性無不在,故不說二。」詔又問曰:「說者皆言法性寬,佛性狹,寬狹既別,非二如何?」弼又對曰:「在寬成寬,在狹成狹,若論性體,非寬非狹。」詔問曰:「既言成寬成狹,何得非寬非狹?若定是狹,亦不能成寬。」對曰:「以非寬狹,故能成寬狹,寬狹所成雖異,能成恒一。」上悅稱善。乃引入經書庫,賜《地持經》一部,帛一百匹。平陽公淹為并州刺史,高祖又命弼帶並州驃騎府長史。


弼性好名理,探味玄宗,自在軍旅,帶經從役。注
 老子《道德經》二卷,表上之曰:「臣聞乘風理弋,追逸羽於高雲;臨波命鉤,引沉鱗於大壑。茍得其道,為工其事,在物既爾,理亦固然。竊惟《道》、《德》二經,闡明幽極,旨冥動寂,用周凡聖。論行也清凈柔弱,語迹也成功致治。實眾流之江海,乃群藝之本根。臣少覽經書,偏所篤好,雖從役軍府,而不捨遊息。鉆味既久,斐
 \gezhu{
  文}
 如有所見,比之前注,微謂異於舊說。情發於中而彰諸外,輕以管窺,遂成穿鑿。無取於遊刃,有慚於運斤。不足破秋毫之論,何以
 解連環之結。本欲止於門內,貽厥童蒙,兼以近資愚鄙,私備忘闕。不悟姑射凝神,汾陽流照,蓋高之聽卑,邇言在察。春末奉旨,猥蒙垂誘,今上所注《老子》,謹冒封呈,并序如別。」詔答云:「李君遊神冥窅,獨觀恍惚,玄同造化,宗極群有。從中被外,周應可以裁成;自己及物,運行可以資用。隆家寧國,義屬斯文。卿才思優洽,業尚通遠,息棲儒門,馳騁玄肆,既啟專家之學,且暢釋老之言。戶列門張,途通徑達,理事兼申,能用俱表,彼賢所未悟,遺老所
 未聞,旨極精微,言窮深妙。朕有味二經,倦於舊說,歷覽新注,所得已多,嘉尚之來,良非一緒。已敕殺青編,藏之延閣。」又上一本於高祖,一本於世宗。



 武定中,遷衛尉卿。會梁遣貞陽侯蕭明等入寇彭城,大都督高岳、行臺慕容紹宗率諸軍討之,詔弼為軍司,攝臺左右。臨發,世宗賜胡馬一匹,語弼曰:「此廄中第二馬,孤恒自乘騎,今方遠別,聊以為贈。」又令陳政務之要可為鑒戒者,錄一兩條。弼請口陳曰:「天下大務,莫過賞罰二端,賞一人使天
 下人喜,罰一人使天下人服。但能二事得中,自然盡美。」世宗大悅曰:「言雖不多,於理甚要。」



 握手而別。破蕭明於寒山,別與領軍潘樂攻拔梁潼州,仍與岳等撫軍恤民,合境傾賴。



 六年四月八日,魏帝集名僧於顯陽殿講說佛理,弼與吏部尚書楊愔、中書令邢邵、秘書監魏收等並侍法筵。敕弼昇師子座,當眾敷演。昭玄都僧達及僧道順並緇林之英,問難鋒至,往復數十番,莫有能屈。帝曰:「此賢若生孔門,則何如也?」



 關中遣儀同王思政據潁
 州,太尉高岳等攻之。弼行潁州事,攝行臺左丞。時大軍在境,調輸多費,弼均其苦樂,公私兼舉,大為州民所稱。潁州之平也,世宗曰:「卿試論王思政所以被擒。」弼曰:「思政不察逆順之理,不識大小之形,不度強弱之勢,有此三蔽,宜其俘獲。」世宗曰:「古有逆取順守,大吳困於小越,弱燕能破強齊。卿之三義,何以自立?」弼曰:「王若順而不大,大而不強,強而不順,於義或偏,得如聖旨。今既兼備眾勝,鄙言可以還立。」世宗曰:「凡欲持論,宜有定指,那得
 廣包眾理,欲以多端自固?」弼曰:「大王威德,事兼眾美,義博故言博,非義外施言。」世宗曰:「若爾,何故周年不下,孤來即拔?」弼曰:「此蓋天意欲顯大王之功。」



 顯祖引為兼長史,加衛將軍,轉中書令,仍長史。進爵定陽縣侯,增邑通前五百戶。弼志在匡贊,知無不為。顯祖將受魏禪,自晉陽至平城都,命弼與司空司馬子如馳驛先入,觀察物情。踐祚之後,敕命左右箱入柏閣。以預定策之功,遷驃騎將軍、衛尉卿,別封長安縣伯。



 嘗與邢邵扈從東山,共
 論名理。邢以為人死還生,恐為蛇畫足。弼答曰:「蓋謂人死歸無,非有能生之力。然物之未生,本亦無也,無而能有,不以為疑,因前生後,何獨致怪?」邢云:「聖人設教,本由勸獎,故懼以將來,理望各遂其性。」



 弼曰:「聖人合德天地,齊信四時,言則為經,行則為法,而云以虛示物,以詭勸民,將同魚腹之書,有異鑿楹之誥,安能使北辰降光,龍宮韞櫝。就如所論,福果可以熔鑄性靈,弘獎風教,為益之大,莫極於斯。此既真教,何謂非實?」邢云:「死之言澌,精
 神盡也。」弼曰:「此所言澌,如射箭盡,手中盡也。《小雅》曰『無草不死』,《月令》又云『靡草死』,動植雖殊,亦此之類。無情之卉,尚得還生,含靈之物,何妨再造。若云草死猶有種在,則復人死亦有識。識種不見,謂以為無者。神之在形,亦非自矚,離朱之明不能睹。雖孟軻觀眸,賢愚可察;鐘生聽曲,山水呈狀。乃神之工,豈神之質。猶玉帛之非禮,鐘鼓之非樂,以此而推,義斯見矣。」邢云:「季札言無不之,亦言散盡,若復聚而為物,不得言無不之也。」



 弼曰:「骨肉下
 歸於土,魂氣則無不之,此乃形墜魂遊,往而非盡。如鳥出巢,如蛇出穴。由其尚有,故無所不之,若令無也,之將焉適?延陵有察微之識,知其不隨於形;仲尼發習禮之歎,美其斯與形別。若許以廓然,然則人皆季子。不謂高論,執此為無。」邢云:「神之在人,猶光之在燭,燭盡則光窮,人死則神滅。」弼曰:「舊學前儒,每有斯語,群疑眾惑,咸由此起。蓋辨之者未精,思之者不篤。竊有末見,可以覈諸。燭則因質生光,質大光亦大;人則神不係於形,形小神
 不小。故仲尼之智,必不短於長狄;孟德之雄,乃遠奇於崔琰。神之於形,亦猶君之有國。



 國實君之所統,君非國之所生。不與同生,孰云俱滅?」邢云:「捨此適彼,生生恒在。周、孔自應同莊周之鼓缶,和桑扈之循歌?」弼曰:「共陰而息,尚有將別之悲;窮轍以游,亦與中途之歎。況曰聯體同氣,化為異物,稱情之服,何害於聖。」



 邢云:「鷹化為鳩,鼠變為鴽,黃母為鱉,皆是生之類也。類化而相生,猶光去此燭,復然彼燭。」弼曰:「鷹未化為鳩,鳩則非有。鼠既二有,
 何可兩立。光去此燭,得燃彼燭,神去此形,亦託彼形,又何惑哉?」邢云:「欲使土化為人,木生眼鼻,造化神明,不應如此。」弼曰:「腐草為螢,老木為蠍,造化不能,誰其然也?」其後別與邢書云:「夫建言明理,宜出典證,而違孔背釋,獨為君子。若不師聖,物各有心,馬首欲東,誰其能禦?奚取於適衷,何貴於得一。逸韻雖高,管見未喻。」前後往復再三,邢邵理屈而止,文多不載。



 又以本官行鄭州事,未發,為家客告弼謀反,收下獄,案治無實,久乃見原。



 因此絕
 朝見。復坐第二子廷尉監臺卿斷獄稽遲,與寺官俱為郎中封靜哲所訟。事既上聞,顯祖發忿,遂徙弼臨海鎮。時楚州人東方白額謀反,南北響應,臨海鎮為賊師張綽、潘天合等所攻,弼率厲城人,終得全固。顯祖嘉之,敕行海州事,即所徙之州。在州奏通陵道並韓信故道。又於州東帶海而起長堰,外遏鹹潮,內引淡水。



 敕並依行。轉徐州刺史,未之任,又除膠州刺史。



 弼儒雅寬恕,尤曉史職。所在清潔,為吏民所懷。耽好玄理,老而愈篤。又注《
 莊子·惠施篇》、《易上下繫》,名《新注義苑》,並行於世。弼性質直,前在霸朝,多所匡正。及顯祖作相,致位僚首,初聞揖讓之議,猶有諫言。顯祖嘗問弼云:「治國當用何人?」對曰:「鮮卑車馬客,會須用中國人。」顯祖以為此言譏我。高德政居要,不能下之,乃於眾前面折云:「黃門在帝左右,何得聞善不驚,唯好減削抑挫!」德政深以為恨,數言其短。又令主書杜永珍密啟弼在長史日,受人請屬,大營婚嫁。顯祖內銜之。弼恃舊,仍有公事陳請。十年夏,上因飲
 酒,積其愆失,遂遣就州斬之,時年六十九。既而悔之,驛追不及。長子蕤、第四子光遠徙臨海鎮。次子臺卿,先徙東豫州。乾明初,並得還鄴。天統五年,追贈弼使持節、揚郢二州軍事、開府儀同三司、尚書右僕射、揚州刺史,謚曰文肅。



 蕤、臺卿,並有學業。臺卿文筆尤工,見稱當世。蕤字子美,武平中大理少卿,兼散騎常侍,聘陳使主。末年,吏部郎中。隋開皇中,終於開州刺史。臺卿字少山,歷中書、黃門侍郎,兼大著作、修國史。武平末,國子祭酒,領尚
 書左丞。周武帝平齊,命尚書左僕射陽休之以下知名朝士十八人隨駕入關,蕤兄弟並不預此名。臺卿後雖被徵,為其聾疾放歸。隋開皇中,徵為著作郎,歲餘以年老致事,詔許之。



 特優其禮,終身給祿,未幾而終。



 史臣曰:孫搴便藩左右,處文墨之地,入幕未久,情義已深。及倉卒致殞,高祖折我右臂,雖戎旌未卷,愛惜才子,不然何以成霸王之業。太史公云:「非死者難,處死者難。」「或重於太山,或輕於鴻毛。」斯其義也。元康以智能才幹,
 委質霸朝,綢繆帷幄,任寄為重。及難無茍免,忘生殉義,可謂得其地焉。楊愔自謂異行奇才,冠絕夷等,弒逆之際,趨而避之,是則非處死者難,死者亦難也。顯祖弱齡藏器,未有朝臣所知,及北宮之難,以年次推重,故受終之議,時未之許焉。



 杜弼識學甄明,發言讜正,禪代之際,先起異圖。王怒未息,卒蒙顯戮。直言多矣,能無及是者乎?



 贊曰:彥舉驅馳,才高行詖。元康忠勇,舍生存義。卬卬輔
 玄,思極談天,道亡時晦,身沒名全。






\end{pinyinscope}