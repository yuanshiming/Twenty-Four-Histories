\article{志第十三 地理四}

\begin{pinyinscope}

 雲南諸路行中書省,為路三十七、府二,屬府三,屬州五十四,屬縣四十七。其餘甸寨軍民等府不在此數。馬站七十四處人也看成機器。2。形而上學性。否認自然界的相互聯系和運,水站四處。



 雲南諸路道肅政廉訪司大德三年,罷雲南行御史臺,立肅政廉訪司。



 中慶路,上。唐姚州。閣羅鳳叛,取姚州,其子鳳伽異增築城曰柘東,六世孫券豐祐改曰善闡,歷五代迄宋,羈縻而已。元世祖征大理,凡收府八,善闡其一也。?轘模荝咳逴釁摺F淶囟療瞻猜分昜嶸劍髦撩宓刂黓鄘煩恰,凡三千九百里而遠;南至臨安路之鹿滄江,北至羅羅斯之大渡河,凡四千里而近。憲宗五年,立萬戶府十有九,分善闡為萬戶府四。至元七年,改為路。八年,分大理國三十七部為南北中三路,路設達魯花赤並總管。十三年,立雲南行中書省,初置郡縣,遂改善闡為中慶路。領司一、縣三、州四。州領八縣。本路軍民屯田二萬二千四百雙有奇。



 錄事司。



 縣三



 昆明,中。倚郭。唐置。元憲宗四年,分其地立千戶二。至元十二年,改善州,領縣。二十一年,州革,縣如故。其地有昆明池,五百餘里,夏潦必冒城郭。張立道為大理等處勸農使,求泉源所出,洩其水,得地萬餘頃,皆為良田云。富民,下。至元四年,立黎灢千戶。十二年,即黎灢立縣。宜良。下。唐匡州,即其地。蠻酋羅氏於此立城居之,名曰羅裒龍,乃今縣也。元憲宗六年,立太池千戶,隸嵩明萬戶。至元十三年,升宜良州,治太池縣。二十一年,州罷為縣,後廢太池來屬。



 州四



 嵩明州,下。州在中慶東北,治沙札臥城。烏蠻車氏所築,白蠻名為嵩明。昔漢人居之,後烏、白蠻強盛,漢人徙去,盟誓於此,因號嵩盟,今州南有土臺,盟會處也。漢人嘗立長州,築金城、阿葛二城。蒙氏興,改長州為嵩盟部,段氏因之。元憲宗六年,立嵩明萬戶。至元十二年,復改長州。十五年,升嵩明府。二十二年,降為州。領二縣:



 楊林,下。在州東南,治楊林城,乃雜蠻枳氏、車氏、斗氏、麼氏四種所居之地,城東門內有石如羊形,故又作羊。唐有羊林部落,即此地。元憲宗七年,立羊林千戶。至元十二年,改為縣。邵甸。下。在州西,治白邑村,無城郭,車蠻、斗蠻舊地,名為束甸,以束為邵,憲宗七年,立邵甸千戶。至元十二年,改為縣。



 晉寧州,下。唐晉寧縣,蒙氏、段氏皆為陽城堡部。元憲宗七年,立陽城堡萬戶。至元十二年,改晉寧州。領二縣:



 呈貢,下。西臨滇澤之濱,在路之南,州之北,其間相去六十里,有故城曰呈貢,世為些莫強宗邵蠻所居。元憲宗六年,立呈貢千戶。至元十二年,割詔營、切龍、呈貢、雌甸、塔羅、和羅忽六城及烏納山立呈貢縣。歸化。下。在州東北,呈貢縣南,西賓滇澤,地名大吳龍,昔吳氏所居,後為些莫徒蠻所有,世隸善闡。憲宗六年,分隸呈貢千戶。至元十二年,割大吳龍、安江、安水明立歸化縣。



 昆陽州,下。在滇池南,僰、獹雜夷所居,有城曰巨橋,今為州治。閣羅鳳叛唐,令曲旂蠻居之。段氏興,隸善闡。元憲宗並羅瑀等十二城,立巨橋萬戶。至元十三年,改昆陽州。領二縣:



 三泊,下。至元十三年,於那龍城立縣。易門。下。在州之西,治市坪村,世為烏蠻所居。段氏時,高智升治善闡,奄而有之。至元四年,立洟門千戶。十二年,改為縣。縣西有泉曰洟源。訛作易門。



 安寧州,下。唐初置安寧縣,隸昆州。閣羅鳳叛唐後,烏、白蠻遷居。蒙氏終,善闡酋孫氏為安寧城主,及袁氏、高氏互有其地。元憲宗七年,隸陽城堡萬戶。至元三年,立安寧千戶。十二年,改安寧州。領二縣:



 祿豐,下。在州西,治白村,其地瘴熱,非大酋所居,惟烏、雜蠻居之,遷徙不常。至元十三年,割安寧千戶之碌琫、化泥、驥琮籠三處立祿豐縣。因江中有石如甑,俗名碌琫,譯謂碌為石,琫為甑,訛為今名。羅次。下。在州北,治壓磨呂白村,本烏蠻羅部,地險俗悍。至元十二年,因羅部立羅次州,隸中慶路。二十四年,改州為縣。二十七年,隸安寧州。



 威楚開南等路,下。為雜蠻耕牧之地,夷名俄碌,歷代無郡邑,後爨酋威楚築城俄碌炎居之。唐時蒙舍詔閣羅鳳合六詔為一,侵俄碌,取和子城,今鎮南州是也。後閣羅鳳叛,於本境立郡縣,諸爨盡附。蒙氏立二都督、六節度,銀生節度即今路也。及段氏興,銀生隸姚州,又名當箸驗。及高升泰執大理國柄,封其侄子明量於威楚,築外城,號德江城,傳至其裔長壽。元憲宗三年征大理,平之。六年,立威楚萬戶。至元八年,改威楚路,置總管府。領縣二、州四。州領一縣。本路軍民屯田共七千一百雙。



 縣二



 威楚,下。倚郭。至元十五年,升威州,仍立富民、凈樂二縣。二十一年,降州為威楚縣,革二縣為鄉來屬。定遠。下。在路北,地名目直睒,雜蠻居之。諸葛孔明征南中,經此睒,後號為牟州。唐蒙氏遣爨蠻酋抬萼鎮牟州,築城曰耐籠。至高氏專大理國政,命雲南些莫徒酋夷羨徙民二百戶於黃蓬阱,其抬萼故城隸高氏。元憲宗四年,立牟州千戶,黃蓬阱為百戶。至元十二年,改為定遠州,黃蓬阱為南寧縣,後革縣為鄉,改州為縣,隸本路。



 州四



 鎮南州,下。州在路北,昔樸、落蠻所居。川名欠舍,中有城曰雞和。至唐時,蒙氏並六詔,征東蠻,取和子、雞和二城,置石鼓縣,又於沙卻置俗富郡。沙卻即今州治。至段氏封高明量為楚公,欠舍、沙卻皆隸之。元憲宗三年,其酋內附。七年,立欠舍千戶、石鼓百戶。至元二十二年,改欠舍千戶為鎮南州,立定邊、石鼓二縣。二十四年,革二縣為鄉,仍隸本州。



 南安州,下。州在路東南,山嶺稠疊,內一峰竦秀,林麓四周,其頂有泉。昔黑爨蠻祖瓦晟吳立柵居其上,子孫漸盛,不隸他部,至高氏封威楚方隸焉。憲宗立摩芻千戶,隸威楚萬戶。至元十二年,改千戶為南安州,隸本路。領一縣:



 廣通。下。縣在州之北,夷名為路睒,雜蠻居之。南詔閣羅鳳曾立路睒縣,至段氏封高明量於威楚,其後宜州酋些莫徒裔易裒等附之,至高長壽遂處於路睒,易裒去舊堡二十里,山上築城白龍戲新柵。憲宗七年,長壽內附,立路炎千戶。至元十二年改為廣通縣,隸南安州。



 開南州,下。州在路西南,其川分十二甸,昔樸、和泥二蠻所居也。莊王滇池,漢武開西南夷,諸葛孔明定益州,皆未嘗涉其境。至蒙氏興,立銀生府,後為金齒、白蠻所陷,移府治於威楚,開南遂為生蠻所據。自南詔至段氏,皆為徼外荒僻之地。元中統三年平之,以所部隸威楚萬戶。至元十二年,改為開南州。



 威遠州,下。州在開南州西南,其川有六,昔撲、和泥二蠻所居。至蒙氏興,開威楚為郡,而州境始通。其後金齒、白夷蠻酋阿只步等奪其地。中統三年征之,悉降。至元十二年,立開南州及威遠州,隸威楚路。



 武定路軍民府,下。唐隸姚州,在滇北,昔獹鹿等蠻居之。至段氏使烏蠻阿治納洟胒共龍城於共甸,又築城名曰易龍,其裔孫法瓦浸盛,以其遠祖羅婺為部名。元憲宗四年內附。七年,立為萬戶,隸威楚。至元八年,並仁德、於矢入本部為北路。十二年,割出二部,改本路為武定。領州二。州領四縣。本路屯田七百四十八雙。



 州二



 和曲州,下。州在路西南,蠻名叵簉甸,僰、犬鹿諸種蠻所居。地多漢塚,或謂漢人曾居。蒙氏時,白蠻據其地,至段氏以烏蠻阿歷刂並吞諸蠻聚落三十餘處,分兄弟子侄治之,皆隸羅婺部。元憲宗六年,改叵簉甸曰和曲。至元二十六年,升為州。領二縣:



 南甸,下。路治本縣,蠻曰瀼甸,又稱洟陬籠。至元二十六年改為縣。元謀。下。夷中舊名環州,元治五甸,至元十六年改為縣。



 祿勸州,下。州在路東北,甸名洪農碌券,雜蠻居之,無郡所。至元二十六年,立祿勸州。領二縣:



 易籠,下。易籠者,城名,在州北,地名倍場。縣境有二水,蠻語謂洟為水,籠為城,因此為名。昔羅婺部大酋居之,為群酋會集之所。至元二十六年,立縣。



 石舊。下。縣在州東,有四甸:曰掌鳩,曰法塊,曰抹捻,曰曲蔽。掌鳩甸有溪繞其三面,凡數十渡,故名,今訛名石舊。至元二十六年,立縣。



 鶴慶路軍民府,下。府治在麗江路東南,大理路東北,夷名其地曰鶴川、樣共。昔隸越析詔,漢、唐未建城邑。開元末,閣羅鳳合六詔為一,稱南詔,徙治羊苴城,地近龍尾、鶴柘,今府即其地也。大和中,蒙勸封祐於樣共立謀統郡。蒙氏後,經數姓如故。元憲宗三年內附,為鶴州。七年,立二千戶,仍稱謀統,隸大理上萬戶。至元十一年,罷謀統千戶,復為鶴州。二十年,為燕王分地,隸行省。二十三年,升為鶴慶府。領一縣:



 劍川。下。縣治在劍川湖西,夷雲羅魯城。按《唐史》南詔有六節度,劍川其一也。初蒙氏未合六詔時,有浪穹詔與南詔戰,不勝,遂保劍川,更稱劍浪。貞元中,南詔擊破之,奪劍、共諸川地,其酋徙居劍睒西北四百里,號劍羌。蒙氏終,至段氏,改劍川為義督瞼。憲宗四年內附。七年,立義督千戶。至元十一年,罷千戶,立劍川縣,隸鶴州。軍民屯田共二千餘雙。



 雲遠路軍民總管府,元貞二年置。



 徹里軍民總管府,大德中置。大德中,雲南省言:「大徹里地與八百媳婦犬牙相錯,勢均力敵。今大徹里胡念已降,小徹里復控扼地利,多相殺掠,胡念日與相拒,不得離,遣其弟胡倫入朝,指畫地形,乞別立徹里軍民宣撫司,擇通習蠻夷情狀者為之帥,招其來附,以為進取之地。」乃立徹里軍民總管府。



 廣南西路宣撫司。闕。



 麗江路軍民宣撫司,路因江為名,謂金沙江出沙金,故云。源出吐蕃界。今麗江即古麗水,兩漢至隋、唐皆為越巂郡西徼地,昔麼蠻、些蠻居之,遂為越析詔。二部皆烏蠻種,居鐵橋。貞元中,其地歸南詔。元憲宗三年,征大理,從金沙濟江,麼、些負固不服。四年春,平之,立茶罕章管民官。至元八年,立宣慰司。十三年,改為麗江路,立軍民總管府。二十二年,府罷,於通安、巨津之間立宣撫司。領府一、州七。州領一縣。



 府一



 北勝府,在麗江之東。唐南詔時,鐵橋西北有施蠻者,貞元中為異牟尋所破,遷其種居之,號劍羌,名其地曰成偈炎,又改名善巨郡。蒙氏終,段氏時,高智升使其孫高大惠鎮此郡。後隸大理。元憲宗三年,其酋高俊內附。至元十五年,立為施州。十七年,改為北勝州。二十年,升為府。



 州七



 順州,在麗江之東,俗名牛炎。昔順蠻種居劍、共川。唐貞元間,南詔異牟尋破之,徙居鐵橋、大婆、小婆、三探覽等川。其酋成斗族漸盛,自為一部,遷於牛炎。至十三世孫自瞠猶隸大理。元憲宗三年內附。至元十五年,改牛睒為順州。



 蒗蕖州,治羅共炎,在麗江之東,北勝、永寧南北之間,羅落、麼、些三種蠻世居之。憲宗三年,征大理。至元九年內附。十六年,改羅共睒為蒗蕖州。



 永寧州,昔名樓頭炎,接吐蕃東徼,地名答藍,麼、些蠻祖泥月烏逐出吐蕃,遂居此炎,世屬大理。憲宗三年,其三十一世孫和字內附。至元十六年,改為州。



 通安州,治在麗江之東,雪山之下。昔名三炎。僕繲蠻所居,其後麼、些蠻葉古乍奪而有之,世隸大理。憲宗三年,其二十三世孫麥良內附。中統四年,以麥良為察罕章管民官。至元九年,其子麥兀襲父職。十四年,改三睒為通安州。



 蘭州,在蘭滄水之東。漢永平中始通博南山道,渡蘭滄水,置博南縣。唐為盧鹿蠻部。至段氏時,置蘭溪郡,隸大理。元憲宗四年內附,隸茶罕章管民官。至元十二年,改蘭州。



 寶山州,在雪山之東,麗江西來,環帶三面。昔麼、些蠻居之。其先自樓頭徙居此,二十餘世。世祖征大理,自卞頭濟江,由羅邦至羅寺,圍大匱等寨,其酋內附,名其寨曰察罕忽魯罕。至元十四年,以大匱七處立寶山縣,十六年升為州。



 巨津州,昔名羅波九炎,北接三川、鐵橋,西鄰吐蕃。按《唐書》,南詔居鐵橋之南,西北與吐蕃接。今州境實大理西北陬要害地,麼、些大酋世居之。憲宗三年內附。至元十四年,於九炎立巨津州,蓋以鐵橋自昔為南詔、吐蕃交會之大津渡,故名。領一縣:



 臨西。下。縣在州之西北,乃大理極邊險僻之地,夷名羅裒間,居民皆麼、些二種蠻。至元十四年,立大理州縣,於羅裒間立臨西縣,以西臨吐蕃境故也,隸巨津州。



 東川路,下。至元二十八年立。



 茫部路軍民總管府。下。



 益良州。下。強州。下。



 孟傑路。自東川路以下闕。泰定三年,八百媳婦蠻請官守,置木安、孟傑二府於其地。



 普安路,下。治在盤町山陽,巴盤江東。古夜郎地。秦為黔中地,兩漢隸牂牁郡,蜀隸興古郡,隋立牂州。唐置西平州,後改興古郡為盤州。蒙氏叛唐,其地為南詔東鄙,東爨烏蠻七部落居之。其後爨酋阿宋逐諸蠻據其地,號於失部,世為酋長。元憲宗七年,其酋內附,命為於失萬戶。至元十三年,改普安路總管府。明年,更立招討司。十六年,改為宣撫司。二十二年,罷司為路。



 曲靖等路宣慰司軍民萬戶府,曲、靖二州在漢為夜郎味縣地。蜀分置興古郡。隋初為恭州、協州。唐置南寧州。東、西爨分烏、白蠻二種,自曲靖州西南昆川距龍和城,通謂之西爨白蠻。自彌鹿、升麻二川南至步頭,通謂之東爨烏蠻。貞觀中,以西爨歸王為南寧都督,襲殺東爨首領蓋聘。南詔閣羅鳳以兵脅西爨,徙之至龍和,皆殘於兵。東爨烏蠻復振,徙居西爨故地,世與南詔為婚,居故曲靖州。天寶末,征南詔,進次曲靖州,大敗,其地遂沒於蠻。元憲宗六年,立磨彌部萬戶。至元八年,改為中路。十三年,改曲靖路總管府。二十年,以隸皇太子。二十五年,升宣撫司。領縣一、州五。州領六縣。本路屯田四千四百八十雙,歲輸金三千五百五十兩、馬一百八十四。



 縣一



 南寧。下。倚郭。唐以爨歸王為南寧州都督,治石城。及閣羅鳳叛,州廢,蒙氏改石城郡。至段氏,烏蠻莫瀰部酋據石城。元憲宗三年內附。六年,立千戶,隸莫瀰部萬戶。至元十三年,升南寧州。二十二年,革為縣。



 州五



 陸涼州,下。即漢牂牁郡之平夷縣。南詔叛後,落溫部蠻世居之。憲宗三年內附,立落溫千戶,屬落蒙萬戶。至元十三年,改為陸涼州。領二縣:



 芳華,下。在州西。河納。下,在州南,治蔡村。



 越州,下。在路之南,其川名魯望,普麼部蠻世居之。憲宗四年內附。六年,立千戶,隸末迷萬戶。至元十二年,改越州,隸曲靖路。



 羅雄州,下。與溪洞蠻獠接壤,歷代未嘗置郡,夷名其地為塔敝納夷甸。俗傳盤瓠六男,其一曰蒙由丘,後裔有羅雄者居此甸。至其孫普恐,名其部曰羅雄。憲宗四年內附。七年,隸普摩千戶。至元十三年,割夜苴部為羅雄州,隸曲靖路。



 馬龍州,下。夷名曰撒匡。昔僰、剌居之,盤瓠裔納垢逐舊蠻而有其地。至羅苴內附,於本部立千戶。至元十三年,改為州,即舊馬龍城也。領一縣:



 通泉。下。在州西南,與嵩明州楊林縣接壤,納垢之孫易陬分居其地。元初為易龍百戶,隸馬龍千戶。至元十三年,改名通泉縣,隸馬龍州。



 沾益州,下。在本路之東北,據南盤江、北盤江之間。唐初置州,天寶末,沒於蠻,為僰、剌二種所居。後磨彌部奪之。元初其孫普垢歷刂內附。憲宗七年,以本部隸曲靖磨彌萬戶府。至元十三年,改沾益州。領三縣:



 交水,下。治易陬龍城。其先磨彌部酋蒙提居之,後大理國高護軍逐其子孫為私邑。憲宗五年內附。至元十三年,即其城立縣。石梁,下。系磨彌部,又名伍勒部。其酋世為巫,居石梁原山。至元十三年為縣。羅山。下。夷名落蒙山,乃磨彌部東境。



 澂江路,下。治在滇池東南。唐屬牂州,隸黔州都督府。開元中,降為羈縻州。今夷中名其地曰羅伽甸。初,麼、些蠻居之,後為僰蠻所奪。南詔蒙氏為河陽郡,至段氏,麼、些蠻之裔復居此甸,號羅伽部。元憲宗四年內附,六年以羅伽部為萬戶。至元三年,改萬戶為中路。十六年,升為澄江路。領縣三、州二。州領三縣。本路屯田四千一百雙。



 縣三



 河陽,下。內附後為千戶。至元十六年,為河陽州。二十六年,降為縣。江川,下。在抃江路南,星雲湖之北。蒙氏叛唐,使白蠻居之。至段氏,些麼徒蠻之裔居此城。更名步雄部。其後弄景內附,即本部為千戶。至元十三年,改千戶為江川州。二十年,降為縣。陽宗。下。在本路西北,明湖之南。昔麼、些蠻居之,號曰強宗部,其酋盧舍內附,立本部千戶。至元十三年,改為縣。



 州二



 新興州,下。漢新興縣。唐初隸牂州,後南詔叛,降為羈縻州。蒙氏為溫富州。段氏時麼、些蠻分居其地。內附後,立為千戶。至元十三年,改新興州,隸澄江路。領二縣:



 普舍,下。在州西北。昔有強宗部蠻之裔,長曰部傍,據普具龍城,次曰普舍,據普札龍城。二城之西有白城,漢人所築。二酋屢爭其地,莫能定。後普舍孫苴歷刂內附,立本部為千戶。十三年,改千戶為普舍縣,治普札龍城,隸新興州。研和。下。麼些徒蠻步雄居之,其孫龍鐘內附,立百戶。至元十三年,改為縣。



 路南州,下。州在本路之東,夷名路甸,有城曰撒呂,黑爨蠻之裔落蒙所築,子孫世居之,因名落蒙部。憲宗朝內附,即本部立萬戶。至元七年,並落蒙、羅伽、末迷三萬戶為中路。十三年,分中路為二路,改羅伽為澄江路,落蒙為路南州,隸澄江路。領一縣:



 邑市。下。至元十三年,即邑市、彌歪二城立邑市縣,彌沙等五城立彌沙縣。二十四年,並彌沙入本縣,隸路南州。



 普定路,本普里部,歸附後改普定府。至元二十七年,初斡羅思、呂國瑞入賄丞相桑哥及要束木等,請創羅甸宣慰司。至是,言招到羅甸國札哇並龍家、宋家、犵狫、貓人諸種蠻夷四萬六千六百戶。阿卜、阿牙者來朝,為曲靖路宣慰同知脫因及普安路官所阻。會雲南行省言:「羅甸即普里也,歸附後改普定府,印信具存,隸雲南省三十餘年,賦役如期。今所創羅甸宣慰安撫司,隸湖南省。斡羅思等擅以兵脅降普定土官矣資男、札哇、希古等,勒令同其入覲,邀功希賞。氣罷之,仍以其地隸雲南。」制可。大德七年,改為路。大德七年,中書省臣言:「蛇節、宋隆濟等作亂,普定知府容苴率眾效順。容苴沒,其妻適姑亦能宣力戎行,乞令襲其夫職。仍改普定為路,隸曲靖宣慰司,以適姑為本路總管,虎符。」



 仁德府,昔僰、剌蠻居之,無郡縣。其部曰仲扎溢源,後烏蠻之裔新丁奪而有之。至四世孫,因其祖名新丁,以為部號,語訛為仁地。憲宗五年內附。明年,立本部為仁地萬戶。至元初復叛,四年降之,仍為萬戶。十三年,改萬戶為仁德府。本府屯田五百六十雙。領縣二:



 為美,下。縣治在府北,地名溢浦適侶睒甸,即仁地故部。至元二十四年置縣。



 歸厚。下。縣治在府西,地名易浪湳龍,舊隸仁地部。至元二十四年,分立二縣,曰倘俸,曰為美。二十五年,改倘俸曰歸厚。



 羅羅蒙慶等處宣慰司都元帥府



 建昌路,下。本古越巂地,唐初設中都督府,治越巂。至德中,沒於吐蕃。貞元中復之。懿宗時,蒙詔立城曰建昌府,以烏、白二蠻實之。其後諸酋爭強,不能相下,分地為四,推段興為長。其裔浸強,遂並諸酋,自為府主,大理不能制。傳至阿宗,娶落蘭部建蒂女沙智。元憲宗朝,建蒂內附,以其婿阿宗守建昌。至元十二年,析其地置總管府五、州二十三,建昌其一路也,設羅羅宣慰司以總之。本路領縣一、州九。州領一縣。本路立軍民屯田。



 縣一



 中縣。縣治在住頭回甸,蓋越巂之東境也。所居烏蠻自別為沙麻部,以酋長所立處為中州。至元十年內附。十四年,仍為中州。二十二年,降為縣。隸建昌路。



 州九



 建安州,下。即總府所治。建蒂既平,分建昌府為萬戶二,又置千戶二。至元十五年,割建鄉城十四村及建蒂四村立寶安州。十七年,改本千戶為建安州。二十六年,革寶安州,以其鄉村來屬。



 永寧州,下。在建昌之東郭。唐時南詔立建昌郡,領建安、永寧二州。元至元九年,西平王平建蒂。十六年,分建昌為二州,在城曰建安,東郭曰永寧,俱隸建昌路。



 瀘州,下。州在路西,昔名沙城瞼,即諸葛武侯禽孟獲之地。有瀘水,深廣而多瘴,鮮有行者,冬夏常熱,其源可燅雞豚。至段氏時,於熱水甸立城。名洟籠,隸建昌。憲宗時,建蒂內附,復叛,至元九年平之。十五年,改洟籠為瀘州。



 禮州,下。州在路西北,瀘沽水東,所治曰籠麼城。南詔末,諸蠻相侵奪,至段氏興,並有其地。裔孫阿宗內附,復叛,至元九年平之,設千戶。十五年,改為禮州。領一縣:



 瀘沽。縣在州北。昔羅落蠻所居,至蒙氏霸諸部,以烏蠻酋守此城,後漸盛,自號曰落蘭部,或稱羅落。其裔蒲德遣其侄建蒂內附。建蒂繼叛,殺蒲德,自為酋長,並有諸部。至元九年平之,設千戶。十三年升萬戶,十五年改縣。



 里州,下。唐隸巂州都督。蒙詔時落蘭部小酋阿都之裔居此,因名阿都部。傳至納空,隨建蒂內附。中統三年叛。至元十年,其子耶吻效順,隸烏蒙。十八年,設千戶。二十二年,同烏蠻叛,奔羅羅斯。二十三年,升軍民總管府。二十六年,府罷為州,隸建昌路。



 闊州,下。州治密納甸。古無城邑,烏蒙所居。昔仲由蒙之裔孫名科居此,因以名為部號,後訛為闊。至三十七世孫僰羅內附。至元九年,設千戶。二十六年,改為州。



 邛部州,下。州在路東北,大渡河之南,越巂之東北。君長十數,筰都最大。唐立邛部縣,後沒於蠻。至宋歲貢名馬土物,封其酋為邛都王。今其地夷稱為邛部川,治烏弄城,昔麼、些蠻居之,後仲由蒙之裔奪其地。元憲宗時內附。中統五年,立邛部川安撫招討使,隸成都元帥府。至元十年,割屬羅羅斯宣慰司。二十一年,改為州。



 隆州,下。州在路之西南,與漢邛都縣接境,唐會川縣之西北。蒙氏改會川為會同邏,立五瞼,本州為邊府瞼。其後瞼主楊大蘭於瞼北塏上立城,分派而居,名曰大隆城,即今州治也。元至元十三年內附。十四年,設千戶。十七年,改隆州。



 姜州,下。姜者蠻名也。烏蠻仲牟由之裔阿壇絳始居閟畔部,其孫阿羅仕大理國主高泰,是時會川有城曰龍納,羅落蠻世居焉。阿羅挾高氏之勢,攻拔之,遂以祖名曰絳部。憲宗時,隨閟畔內附,因隸焉。至元八年,為落蘭部酋建蒂所破。九年平之,遂隸會川,後屬建昌。十五年,改為姜州。二十七年,復屬閟畔部,後又屬建昌。



 德昌路軍民府,下。漢邛都縣地,唐沒於南詔。路在建昌西南,所居蠻號屈部。元至元九年內附。十二年,立定昌路,以本部為昌州。二十三年,罷定昌路,並入德昌路,治本州葛魯城。領州四。本路立軍民屯田。



 昌州,下。路治本州。初,烏蠻阿屈之裔浸強,用祖名為屈部。其孫烏則,至元九年內附。十二年,改本部為州,兼領普濟、威龍,隸定昌路。二十三年,罷定昌路,並隸德昌。



 德州,下。在路之北。其地今名吾越甸,城曰亦苴龍,所居蠻苴郎,以遠祖名部曰赬綖。憲宗時內附。至元十二年,立千戶。十三年,改為德州,隸德平路。二十三年,改隸德昌。



 威龍州,下。州在路西南,夷名巴翠部,領小部三,一曰沙媧普宗,二曰烏雞泥祖,三曰媧諾龍菖蒲,皆獹魯蠻種也。至元十五年,合三部立威龍州,隸德昌。



 普濟州,下。州在路西北,夷名玕甸。昔為荒僻之地,獹魯蠻世居之,後屬屈部。至元九年,隨屈部內附。十五年,於玕甸立定昌路。二十三年,路革,改隸德昌。



 會川路,下。路在建昌南。唐移邛都於此。其地當征蠻之要沖,諸酋聽會之所,故名。天寶末,沒於南詔,立會川都督府,又號清寧郡。至段氏仍為會川府。元至元九年內附。十四年立會川路,治武安州。領州五。本路立軍民屯田。



 武安州,下。蠻稱龍泥城。至元十四年,立管民千戶。十七年,改為武安州。



 黎溪州,下。古無城邑,蠻雲黎彄,訛為今名。初,烏蠻與漢人雜處,及南詔閣羅鳳叛,徙白蠻守之。蒙氏終,羅羅逐去白蠻。段氏興,令羅羅蠻乞夷據其地。至元九年,其裔阿夷內附,改其部為黎溪州。



 永昌州,下。州在路北,治故歸依城,即古會川也。唐天寶末,沒於南詔,置會川都督。至蒙氏改會同府,置五瞼,徙張、王、李、趙、楊、周、高、段、何、蘇、龔、尹十二姓於此,以趙氏為府主,居今州城。趙氏弱,王氏據之。及段氏與高氏專政,逐王氏,以其子高政治會川。元憲宗三年,征大理,高氏逃去。九年,故酋王氏孫阿龍率眾內附。至元八年,以其男阿禾領會川。十四年,改管民千戶。十七年,立永昌州,隸會川路。



 會理州,下。州在會川府東南。唐時南詔屬會川節度,地名昔陀。有蠻名阿壇絳,亦仲由蒙之遺種。其裔羅於則,得昔陀地居之,取祖名曰絳部,後強盛,盡有四州之地,號蒙歪。元憲宗八年,其孫亦蘆內附,隸閟畔萬戶。至元四年,屬落蘭部。十三年,改隸會川路。十五年,置會理州,仍隸會川。二十七年,復屬閟■■畔部。



 麻龍州,下。麻龍者,城名也,地名棹羅能。烏蠻蒙次次之裔,祖居閟畔東川,後普恐遷苗臥龍,其孫阿麻內附。至元五年,為建蒂所並。十二年,屬會川。十四年,立管民千戶,隸會川路。十七年,立為州。二十七年,割屬閟畔部。



 柏興府,昔摩沙夷所居。漢為定笮縣,隸越巂郡。唐立昆明縣。天寶末沒於吐蕃。後復屬南詔,改香城郡。元至元十年,其鹽井摩沙酋羅羅將犬鹿鹿、茹庫內附。十四年,立鹽井管民千戶。十七年,改為閏鹽州,以犬鹿鹿部為普樂州,俱隸德平路。二十七年,並普樂、閏鹽二州為閏鹽縣,立柏興府,隸羅羅宣慰司。領縣二:



 閏鹽,下。倚郭。夷名為賀頭甸,以縣境有鹽井故名。金縣。下。縣在府北,夷名利寶揭勒。所居蠻因茹庫,乃漢越巂郡北境,與吐蕃接。至元十五年,立為金州,後降為縣,以縣境斛僰和山出金,故名焉。



 臨安廣西元江等處宣慰司兼管軍萬戶府



 臨安路,下。唐隸牂州,天寶末沒於南詔。蒙氏立都督府二,其一曰通海郡,段氏改為秀山郡,阿僰部蠻居之。元憲宗六年內附,以本部為萬戶。至元八年改為南路,十三年又改為臨安路。領縣二、千戶一、州三。州領二縣。宣慰司所領屯田六百雙,本路有司所管三千四百雙,爨僰軍千戶所管一千一百五十雙有奇。



 縣二



 河西,下。縣在杞麓湖之南,又名其地曰休臘。昔莊掞王其地。唐初於姚州之南置西宗州,領三縣,河西其一也。天寶後沒於蠻,為步雄部,後阿僰蠻易渠奪而居之。元憲宗六年內附。七年,即阿僰部立萬戶,休臘隸之。至元十三年,始為河西州,隸臨安路。二十六年,降為縣。蒙自。下。縣界南鄰交趾,西近建水州。縣境有山名自則,漢語訛為蒙自,上有故城。白夷所築,即今縣治,下臨巴甸。南詔時以趙氏鎮守,至段氏,阿僰蠻居之。憲宗六年內附,繼叛,七年平之,立千戶,隸阿僰萬戶。至元十三年,改阿僰萬戶為臨安路,以本千戶為縣。



 舍資千戶。蒙自縣之東,阿僰蠻所居地。昔名褒古,又曰部裊踵甸。傳至裔孫舍資,因以為名。內附後,隸蒙自千戶。至元十三年,改蒙自為縣,其地近交趾,遂以舍資為安南道防送軍千戶,隸臨安路。



 州三



 建水州,下。在本路之南,近接交趾,為雲南極邊。治故建水城,唐元和間蒙氏所築,古稱步頭,亦云巴甸。每秋夏溪水漲溢如海,夷謂海為惠,歷刂為大,故名惠歷刂,漢語曰建水,歷趙、楊、李、段數姓,皆仍舊名,些麼徒蠻所居。內附後,立千戶,隸阿僰萬戶。至元十三年,改建水州,隸臨安路。



 石平州,下。在路之西南,阿僰蠻據之,得石坪,聚為居邑,名曰石坪。至元七年,改邑為州,隸臨安路。



 寧州,下。在本路之東。唐置黎州,天寶末沒於蠻。地號浪曠,夷語謂旱龍也。步雄部蠻些麼徒據之,後屬爨蠻酋阿幾,以浪曠割與寧酋豆圭。元憲宗四年,寧酋內附。至元十三年,改為寧州,隸臨安路。舊領三縣:通海,翏峨,西沙。西沙在州東,寧部蠻世居之。其裔孫西沙築城於此,因名西沙籠。憲宗四年,其酋普提內附,就居此城為萬戶。至元十三年,立為西沙縣。二十六年,以隸寧州。至治二年,並入州。領二縣:



 通海,下。倚郭。元初立通海千戶,隸善闡萬戶。至元十三年,改通海縣,隸寧海府。二十七年,府革,直隸臨安路,今割隸寧州。翏峨。下。縣在河西縣之西,控扼山谷,北接滇池,亦屬滇國。昔鷿猊蠻居之,後阿僰酋逐鷿猊據其地。至其孫阿次內附,以其部立千戶。至元十三年,改為州,領邛洲、平甸二縣。二十六年,降為縣,並二縣為鄉,隸臨安路。今割隸寧州。



 廣西路,下。東爨烏蠻彌鹿等部所居。唐為羈縻州,隸黔州都督府。後師宗、彌勒二部浸盛,蒙氏、段氏莫能制。元憲宗七年,二部內附,隸落蒙萬戶。至元十二年,籍二部為軍,立廣西路。十八年,復為民。領州二。



 師宗州,下。在路之東南。昔爨蠻逐獠、僰等居之,其後師宗據匿弄甸,故名師宗部。至元十二年,立為千戶。十八年,復為民。二十七年,改為州。



 彌勒州,下。在路南。昔些莫徒蠻之裔彌勒得郭甸、巴甸、部籠而居之,故名其部曰彌勒。至元十二年,為千戶。十八年,復為民。二十七年,改為州。



 元江路,下。古西南夷地。今元江在梁州之西南,又當在黑水之西南也。阿僰諸部蠻自昔據之。憲宗四年內附,七年復叛,率諸部築城以拒命。至元十三年,遙立元江府以羈縻之。二十五年,命雲南王討平之,割羅盤、馬籠、步日、思麼、羅丑、羅陀、步騰、步竭、臺威、臺陽、設棲、你陀十二部於威遠,立元江路。



 步日部。在本路之西。蒙氏立此甸,徙白蠻鎮之,名步日瞼。



 馬籠部。因馬籠山立寨,在本路之北,所居蠻阿僰。元初立為千戶,屬寧州萬戶。至元十三年,改隸元江萬戶。二十五年,屬元江路。



 大理金齒等處宣慰司都元帥府



 大理路軍民總管府,上。本漢楪榆縣地。唐於昆明之梇棟川置姚州都督府,治楪榆洱河蠻。後蒙舍詔皮羅閣逐河蠻取太和城,至閣羅鳳號大蒙國。雲南先有六詔,至是請於朝,求合為一,從之。蒙舍在其南,故稱南詔。徙治太和城。至異牟尋又遷於喜郡史城,又徙居羊苴乖城,即今府治。改號大禮國。其後鄭、趙、楊三氏互相篡奪,至石晉時,段思平更號大理國。元憲宗三年收附。六年,立上下二萬戶。至元七年,並二萬戶為大理路。有點蒼山在大理城西,周廣四百里,為雲南形勝要害之地。城中有五花樓,唐大中十年,南詔王券豐佑所建。樓方五里,高百尺,上可容萬人。世祖征大理時,駐兵樓前。至元三年,嘗賜金重修焉。領司一、縣一、府二、州五。府領一縣,州領二縣。



 錄事司。憲宗七年,立中千戶,屬大理萬戶。至元十一年,罷千戶,立錄事司。十二年,升理州。二十一年,州罷,復立錄事司。



 縣一



 太和。倚郭。憲宗七年,於城內外立上中下三千戶。至元二十六年,即中千戶立錄事司,上下二千戶立縣。



 府二



 永昌府,唐時蒙氏據其地,歷段氏、高氏皆為永昌府。元憲宗七年,分永昌之永平立千戶。至元十一年,立永昌州。十五年升為府,隸大理路。領一縣:



 永平。下。縣在府東,鹿滄江之東,即漢博平縣。唐蒙氏改勝鄉郡,屬永昌。至元十一年,改永平縣,隸永昌府。



 騰沖府,在永昌之西,即越炎地。唐置羈縻郡。蒙氏九世孫異牟尋取越炎,逐諸蠻有其地,為軟化府。其後白蠻徙居之,改騰沖府。元憲宗三年,府酋高救內附。至元十一年,改藤越州,又立藤越縣。十四年,改騰沖府。二十五年,罷州縣,府如故。永昌、騰沖二府軍民屯田共二萬二千一百五雙。



 州五



 鄧川州,下。在本路北。夷有六詔,矰炎其一也。唐置矰川州,治大厘。蒙氏襲而奪之,後改德原城,隸大理。段氏因之。元憲宗三年內附。七年,立德原千戶,隸大理上萬戶。至元十一年,改德原城為鄧川州。領一縣:



 浪穹。下。本名彌茨,乃浪穹詔所居之地。唐初,其王鐸羅望與南詔戰,不勝,保劍川,更稱劍浪。貞元中,南詔破之,以浪穹、施浪、鄧睒總三浪為浪穹州。元憲宗七年內附,立浪穹千戶,隸大理上萬戶。至元十一年降為縣,隸鄧川州。



 蒙化州,下。本蒙舍城。唐置陽瓜州。天寶間,鳳伽異為州刺史。段氏為開南縣。元憲宗七年,以蒙舍立千戶,屬大理上萬戶。至元十一年,立蒙化府。十四年,升為路。二十年,降為州,復隸大理路。



 趙州,下。昔為羅落蠻所居地。蒙氏立國,有十瞼,趙川瞼其一也。夷語瞼若州。皮羅閣置趙郡,閣羅鳳改為州,段氏改天水郡。憲宗七年立趙瞼千戶,隸大理下萬戶。至元十一年改為州,又於白崖瞼立建寧縣,隸本州,即古勃弄地。二十五年縣革入州,隸大理路。



 姚州,下。唐於梇棟川置姚州都督府。天寶間,閣羅鳳叛,取姚州,附吐蕃。終段氏為姚州。元憲宗三年內附。七年,立統矢千戶、大姚堡千戶。至元十二年,罷統矢,立姚州,隸大理路。領一縣:



 大姚,下。唐置西濮州,後更名髳州,南接姚州,統縣四,一曰青蛉,即此地。夷名大姚堡,與梇棟川相接。元憲宗七年,立千戶,隸大理下萬戶。至元十一年,罷千戶立大姚縣,隸姚州。



 雲南州,下。唐以漢雲南縣置郡。蒙氏至段氏並為雲南州。元憲宗七年立千戶,隸大理下萬戶。至元十一年,立雲南州。



 蒙憐路軍民府。至元二十七年,從雲南行省請,以蒙憐甸為蒙憐路軍民總管府,蒙萊甸為蒙萊路軍民總管府。其餘闕。



 蒙萊路軍民府。闕。



 金齒等處宣撫司。其地在大理西南,蘭滄江界其東,與緬地接其西。土蠻凡八種:曰金齒,曰白夷,曰僰,曰峨昌,曰驃,曰繲,曰渠羅,曰比蘇。按《唐史》,茫施蠻本關南種,在永昌之南,樓居,無城郭。或漆齒,或金齒,故俗呼金齒蠻。自漢開西南夷後,未嘗與中國通。唐南詔蒙氏興,異牟尋破群蠻,盡虜其人以實其南東北,取其地,南至青石山緬界,悉屬大理。及段氏時,白夷諸蠻漸復故地,是後金齒諸蠻浸盛。元憲宗四年,平定大理,繼征白夷等蠻。中統初,金齒、白夷諸酋各遣子弟朝貢。二年,立安撫司以統之。至元八年,分金齒、白夷為東西兩路安撫使。十二年,改西路為建寧路,東路為鎮康路。十五年,改安撫為宣撫,立六路總管府。二十三年,罷兩路宣撫司,並入大理金齒等處宣撫司。



 柔遠路,在大理之西,永昌之南。其地曰潞江,曰普坪瞼,曰申瞼僰寨,曰烏摩坪。僰蠻即《通典》所謂黑爨也。中統初,僰酋阿八思入朝。至元十三年,與茫施、鎮康、鎮西、平緬、麓川俱立為路,隸宣撫司。



 茫施路,在柔遠路之南,瀘江之西。其地曰怒謀,曰大枯炎,曰小枯炎。即《唐史》所謂茫施蠻也。中統初內附。至元十三年,立為路,隸宣撫司。



 鎮康路,在柔遠路之南,蘭江之西。其地曰石炎,亦黑僰所居。中統初內附。至元十三年,立為路,隸宣撫司。



 鎮西路,在柔遠路正西,東隔麓川。其地曰於賴睒,曰渠瀾炎,白夷蠻居之。中統初內附,至元十三年立為路,隸宣撫司。



 平緬路,北近柔遠路。其地曰驃炎,曰羅必四莊,曰小沙摩弄,曰驃炎頭,白夷居之。中統初內附,至元十三年立為路,隸宣撫司。



 麓川路,在茫施路東。其地曰大布茫。曰炎頭附賽,曰炎中彈吉,曰炎尾福祿培,皆白夷所居。中統初內附,至元十三年立為路,隸宣撫司。



 南炎,在鎮西路西北。其地有阿賽炎、午真炎,白夷峨昌所居。元初內附,至元十五年隸宣撫司。金齒六路一睒,歲賦金銀各有差。



 烏撒烏蒙宣慰司,在本部巴的甸。烏撒者蠻名也。其部在中慶東北七百五十里,舊名巴凡兀姑,今曰巴的甸,自昔烏雜蠻居之。今所轄部六,曰烏撒部、阿頭部、易溪部、易娘部、烏蒙部、閟畔部。其東西又有芒布、阿晟二部。後烏蠻之裔折怒始強大,盡得其地,因取遠祖烏撒為部名。憲宗征大理,累招不降。至元十年始附。十三年,立烏撒路。十五年,為軍民總管府。二十一年,改軍民宣撫司。二十四年,升烏撒烏蒙宣慰司。



 木連路軍民府。以下闕。



 蒙光路軍民府。



 木邦路軍民府。



 孟定路軍民府。



 謀粘路軍民府。



 南甸軍民府。



 六難路甸軍民府。



 陋麻和管民官。



 雲龍甸軍民府。



 縹甸軍民府。



 二十四寨達魯花赤。



 孟隆路軍民府。



 木朵路軍民總管府。至元三十年,以金齒木朵甸戶口增殖,立下路總管府,其為長者給兩珠虎符。



 金齒孟定各甸軍民官。



 孟愛等甸軍民府。至元二十一年,金齒新附孟愛甸酋長遣其子來朝,即其地立軍民總管府。



 蒙兀路。



 通西軍民總管府。大德元年,蒙陽甸酋領緬吉納款,遣其弟阿不剌等赴闕進方物,且請歲貢銀千兩及置郡縣驛傳,遂立通西軍民府。



 木來軍民府。至元二十九年,雲南省言:「新附金齒適當忙兀禿兒迷失出征軍馬之沖,資其芻糧,擬立為木來路。」中書省奏置散府,以布伯為達魯花赤,用其土人馬列知府事。



\end{pinyinscope}