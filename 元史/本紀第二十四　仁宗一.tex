\article{本紀第二十四 仁宗一}

\begin{pinyinscope}

 仁宗聖文欽孝皇帝,諱愛育黎拔力八達,順宗次子,武宗之弟也。母曰興聖太后,弘吉剌氏。至元二十二年三月丙子生。



 大德九年冬十月,成宗不豫,中宮秉政,詔帝與太后出居懷州。十年冬十二月,至懷州,所過郡縣,供帳華侈,悉令撤去,嚴飭扈從毋擾於民,且諭僉事王毅察而言之,民皆感悅。



 十一年春正月,成宗崩,時武宗為懷寧王,總兵北邊,戊子,帝與太后聞哀奔赴。庚寅,至衛輝,經比干墓,顧左右曰:「紂內荒於色,毒痡四海,比干諫,紂刳其心,遂失天下。」令祠比干於墓,為後世勸。至漳河,值大風雪,田叟有以盂粥進者,近侍卻不受。帝曰:「昔漢光武嘗為寇兵所迫,食豆粥。大丈夫不備嘗艱阻,往往不知稼穡艱難,以致驕惰。」命取食之。賜叟綾一匹,慰遣之。行次邯鄲,諭縣官曰:「吾慮衛士不法,胥吏科斂,重為民困。」乃命王傅巡行察之。二月辛亥,至大都,與太后入內,哭盡哀,復出居舊邸,日朝夕入哭奠。左丞相阿忽臺等潛謀推皇后伯要真氏稱制,安西王阿難答輔之。時左丞相哈剌哈孫答剌罕稱疾,守宿掖門凡三月,密持其機,陽許之,夜遣人啟帝曰:「懷寧王遠,不能猝至,恐變生不測,當先事而發。」三月丙寅,帝率衛士入內,召阿忽臺等責以亂祖宗家法,命執之,鞫問辭服,戊辰,伏誅。諸王闊闊出、牙忽都等曰:「今罪人斯得,太子實世祖之孫,宜早正天位。」帝曰:「王何為出此言也!彼惡人潛結宮壼,構亂我家,故誅之,豈欲作威覬望神器耶?懷寧王吾兄也,正位為宜。」乃遣使迎武宗於北邊。五月乙丑,帝與太后會武宗於上都。甲申,武宗即位。六月癸巳,詔立帝為皇太子,受金寶。遣使四方,旁求經籍,識以玉刻印章,命近侍掌之。時有進《大學衍義》者,命詹事王約等節而譯之。帝曰:「治天下,此一書足矣。」因命與《圖象孝經》、《列女傳》並刊行,賜臣下。十一月戊寅,受玉冊,領中書省、樞密院。



 至大元年七月,帝諭詹事曲出曰:「汝舊事吾,其與同僚協議,務遵法度,凡世祖所未嘗行及典故所無者,慎勿行。」二年八月,立尚書省,詔太子兼尚書令,戒飭百官有司,振紀綱,重名器,夙夜以赴事功。詹事院臣啟金州獻瑟瑟洞,請遣使採之,帝曰:「所寶惟賢,瑟瑟何用焉?若此者,後勿復聞。」先是,近侍言賈人有售美珠者,帝曰:「吾服御雅不喜飾以珠璣,生民膏血,不可輕耗。汝等當廣進賢才,以恭儉愛人相規,不可以奢靡蠹財相導。」言者慚而退。淮東宣慰使撒都獻玉觀音、七寶帽頂、寶帶、寶鞍,卻之,戒諭如初。詹事王約啟事,二宦者侍側,帝問:「自古宦官壞人家國,有諸?」約對曰:「宦官善惡皆有之,但恐處置失宜耳。」帝然之。九月,河間等路獻嘉禾,有異畝同穎及一莖數穗者,命集賢學士趙孟頫繪圖,藏諸秘書。



 四年春正月庚辰,武宗崩。壬午,罷尚書省。以丞相脫虎脫、三寶奴,平章樂實,右丞保八,左丞忙哥帖木兒,參政王羆,變亂舊章,流毒百姓,命中書右丞相塔思不花、知樞密院事鐵木兒不花等參鞫。丙戌,脫虎脫、三寶奴、樂實、保八、王羆伏誅,忙哥帖木兒杖流海南。壬子,日赤如赭。罷城中都。召世祖朝諳知政務素有聲望老臣平章程鵬飛、董士選,太子少傅李謙,少保張驢,右丞陳天祥、尚文、劉正,左丞郝天挺,中丞董士珍,太子賓客蕭,參政劉敏中、王思廉、韓從益,侍御趙君信,謙訪使程鉅夫,杭州路達魯花赤阿合馬,給傳詣闕,同議庶務。甲午,宥阿附脫虎脫等左右司、六部官罪。乙未,禁百官役軍人營造及守護私第。丁酉,以雲南行中書省左丞相鐵木迭兒為中書右丞相,太子詹事完澤、集賢大學士李孟並平章政事。戊戌,以塔思不花及徽政院使沙沙並為御史大夫。己亥,改行尚書省為行中書省。庚子,減價糶京倉米,日千石,以賑貧民。停各處營造。罷廣武康里衛,追還印符、驛券、璽書,及其萬戶等官宣敕。辛丑,以塔失鐵木兒知樞密院事。壬寅,禁鷹坊馳驛擾民。敕中書,凡傳旨非親奉者勿行。以諸王朝會,普賜金三萬九千六百五十兩、銀百八十四萬九千五十兩、鈔二十二萬三千二百七十九錠、幣帛四十七萬二千四百八十八匹。



 二月,復玉宸樂院為儀鳳司,改延慶司為都功德使司。乙巳,命和林、江浙行省依前設左丞相,餘省唯置平章二員,遙授職事勿與。戊申,罷運江南所印佛經。辛亥,禁諸王、駙馬、權豪擅據山場,聽民樵採。罷阿老瓦丁買賣浙鹽,供中政食羊。禁宣政院違制度僧。甲寅,遣使檢核小雲石不花所獻河南荒田。司徒蕭珍以城中都徼功毒民,命追奪其符印,令百司禁錮之,還中都所占民田。罷江南行通政院、行宣政院。甲子,太陰犯填星。升典內司為典內院,秩從三品。命中書平章李孟領國子監學,諭之曰:「學校人材所自出,卿等宜數詣國學課試諸生,勉其德業。」敕:「諸王、駙馬戶在縉山、懷來、永興縣者,與民均服徭役。諸司擅奏除官者,毋給宣敕。」御史臺臣言:「白云宗總攝所統江南為僧之有發者,不養父母,避役損民,乞追收所受璽書銀印,勒還民籍。」從之。罷福建繡匠、河南魚課兩提舉司,省宣徽院參議、斷事官。丙寅,監察御史言:「比者尚書省臣蠹國亂政,已正典刑,其餘黨附之徒布在百司,亦須次第沙汰。今中書奏用孛羅鐵木兒為陜西平章、烏馬兒為江浙平章、闊里吉思為甘肅平章、塔失帖木兒為河南參政、萬僧為江浙參政,各人前任,皆受重贓,或挾勢害民,咸乞罷黜。」制曰:「可。」丁卯,命西番僧非奉璽書驛券及無西番宣慰司文牒者,勿輒至京師,仍戒黃河津吏驗問禁止。罷總統所及各處僧錄、僧正、都綱司,凡僧人訴訟,悉歸有司。罷仁虞院,復置鷹坊總管府。庚午,命廣西靜江、融州軍民官,鎮守三載無虞者,民官減一資,軍官升一階,著為令。思州軍民宣撫司招諭官唐銓以洞蠻楊正思等五人來朝,賜金帛有差。立淮安忠武王伯顏祠於杭州,仍給田以供祀事。是月,帝謂侍臣曰:「郡縣官有善有惡,其命臺官選正直之人為廉訪司官而體察之,果有廉能愛民者,不次擢用,則小人自知激厲矣。」旌表漳州長泰縣民王初應孝行。



 三月庚辰,召前樞密副使吳元珪,左丞拜降、兀伯都剌至京師,同諸老臣議事。丙戌,太陰犯太微上相。罷五臺行工部。己丑,命毋赦十惡大逆等罪。復典瑞院為典瑞監。庚寅,即皇帝位於大明殿,受諸王百官朝賀,詔曰:



 惟昔先帝,事皇太后,撫朕眇躬,孝友天至。由朕得托順考遺體,重以母弟之嫡,加有削平內難之功,於其踐阼未逾月,授以皇太子寶,領中書令、樞密使,百揆機務,聽所總裁,於今五年。先帝奄棄天下,勛戚元老咸謂大寶之承,既有成命,非與前聖賓天而始徵集宗親議所宜立者比,當稽周、漢、晉、唐故事,正位宸極。朕以國恤方新,誠有未忍,是用經時。今則上奉皇太后勉進之命,下徇諸王勸戴之勤,三月十八日,於大都大明殿即皇帝位。凡尚書省誤國之臣,先已伏誅,同惡之徒,亦已放殛,百司庶政,悉歸中書,命丞相鐵木迭兒、平章政事李道復等從新拯治。可大赦天下,敢以赦前事相告言者,罪以其罪。諸衙門及近侍人等,毋隔越中書奏事。諸上書陳言者,量加旌擢。其僥幸獻地土並山場、窯冶及中寶之人,並禁止之。諸王、駙馬經過州郡,不得非理需索,應和顧和買,隨即給價,毋困吾民。



 辛卯,禁民間制金箔、銷金、織金,以御史中丞李士英為中書左丞。壬辰,發京倉米,減價以糶賑貧民。丁酉,命月赤察兒依前太師,宣徽使鐵哥為太傅,集賢大學士曲出為太保。敕百司改升品級者,悉復至元舊制。己亥,增置左翼、右翼指揮各一員。寧夏路地震。是月,帝諭省臣曰:「卿等裒集中統、至元以來條章,擇曉法律老臣,斟酌重輕,折衷歸一,頒行天下,俾有司遵行,則抵罪者庶無冤抑。」又諭太府監臣曰:「財用足則可以養萬民,給軍旅,自今雖一繒之微,不言於朕,毋輒與人。」以陜西行尚書省左丞兀伯都剌為中書右丞;昭文館大學士察罕參知政事;中書平章政事、知樞密院事床兀兒,欽察親軍都指揮使脫火赤拔都兒,中書右丞相、知樞密院事鐵木兒不花,錄軍國重事、知樞密院事也速,知樞密院事兼山東河北蒙古軍都萬戶也先鐵木兒,遙授左丞相、仁虞院使也兒吉尼,太子詹事月魯鐵木兒,並知樞密院事。賜大都路民年九十者二千三百三十一人,人帛二匹;八十者八千三百三十一人,人帛一匹。



 夏四月壬寅,詔分汰宿衛士,漢人、高麗、南人冒入者,還其元籍。癸卯,翽星於回回司天臺。以即位,恩賜太師、太傅、太保,人金五十兩、銀三百五十兩、衣四襲。行省臣預朝會者,賞銀有差。丁未,以太子少保張驢為江浙平章,戒之曰:「以汝先朝舊人,故命汝往。民為邦本,無民何以為國?汝其上體朕心,下愛斯民。」戊申,以即位告天地於南郊。庚戌,拘收下番將校不典兵者虎符、銀牌。癸丑,詔:「路、府、州、縣官,三年為滿。」罷典醫監。甲寅,太陰犯亢,熒惑犯壘壁陣。丙辰,詔諭宣徽使亦列赤,諸蒙古民有貧乏者,發廩濟之。丁巳,罷中政院。戊午,以即位告於太廟。辛酉,敕:「國子監師儒之職有才德者,不拘品級,雖布衣亦選用。」癸亥,敕:「諸使臣非軍務急速者,毋給金字圓牌。」定四宿衛士歲賜鈔二十四萬二百五錠。罷中都留守司,復置隆興路總管府,凡創置司存悉罷之。乙丑,封知樞密院事鐵木兒不花為宣寧王,賜銀印。丁卯,詔曰:「我世祖皇帝,參酌古今,立中統、至元鈔法,天下流行,公私蒙利,五十年於茲矣。比者尚書省不究利病,輒意變更,既創至大銀鈔,又鑄大元、至大銅錢。鈔以倍數太多,輕重失宜;錢以鼓鑄弗給,新舊恣用;曾未再期,其弊滋甚。爰咨廷議,允協輿言,皆願變通,以復舊制。其罷資國院及各處泉貨監提舉司,買賣銅器,聽民自便。應尚書省已發各處至大鈔本及至大銅錢,截日封貯,民間行使者,赴行用庫倒換。」仍免大都、上都、隆興差稅三年。命中書省賑濟甘肅過川軍,罷僧、道、也裏可溫、答失蠻、頭陀、白云宗諸司。改封親王迭裡哥兒不花為湘寧王,賜金印,食湘鄉州、寧鄉縣六萬五千戶。拘還甘肅、陜西、遼陽省臣所佩虎符,禁鷹坊擾民,罷通政院,以其事歸兵部。增置尚書員外郎各一員,罷回回合的司屬。帝御便殿,李孟進曰:「陛下御極,物價頓減,方知聖人神化之速,敢以為賀。」帝蹙然曰:「卿等能盡力贊襄,使兆民乂安,庶幾天心克享,至於秋成,尚未敢必。今朕踐阼曾未逾月,寧有物價頓減之理?朕托卿甚重,茲言非所賴也。」孟愧謝。帝諭集賢學士忽都魯都兒迷失曰:「向召老臣十人,所言治政,汝其詳譯以進,仍諭中書悉心舉行。」南陽等處風、雹。



 五月壬申,以宦者鐵昔里為利用監卿。癸酉,八百媳婦蠻與大、小徹里蠻寇邊,命雲南王及右丞阿忽臺以兵討之。改封乳母夫壽國公楊德榮為雲國公。丙子,命翰林國史院纂修先帝實錄及累朝皇后、功臣列傳,俾百司悉上事跡。丁丑,禁毋以毒藥釀酒。庚辰,敕中書省裁省冗司,置高昌王傅,復度支院為監,罷泉府司、長信院、司禋監。辛巳,賜大長公主祥哥剌吉鈔一萬錠。壬午,制定翰林國史院承旨五員,學士、侍讀、侍講、直學士各二員。拘諸王、駙馬及有司驛券,自今遣使,悉從中書省給降。置祥和署,掌伶人。金齒諸國獻馴象。癸未,太陰犯氐。賜國師板的答鈔萬錠,以建寺於舊城。戊子,羅鬼蠻來獻方物。甲午,復太常禮儀院為太常寺。是月,禁民捕駕鵝。



 六月癸卯,敕宣政院:「凡西番軍務,必移文樞密院同議以聞。」吐蕃犯永福鎮,敕宣政院與樞密院遣兵討之。乙巳,命侍臣咨訪內外,才堪佐國者,悉以名聞。仍戒敕諸王,恪恭乃職。丙午,以內侍楊光祖為秘書卿,譚振宗為武備卿,關居仁為尚乘卿,並授弘文館學士。置湘寧王迭裡哥兒不花王傅。丁未,太陰犯太微東垣上相。己酉,詔存恤軍人。庚戌,太陰犯氐。壬子,敕甘肅省給過川軍牛種農器,令屯田。癸丑,復太府院為太府監,省上都兵馬指揮為五員。甲寅,封亦思丹為懷仁郡王,賜銀印。丁巳,敕翰林國史院春秋致祭太祖、太宗、睿宗御容,歲以為常。命和林行省右丞孛里、馬速忽經理稱海屯田。大同路宣寧縣民家產犢而死,頗類麒麟,車載以獻,左右曰:「古所謂瑞物也。」帝曰:「五穀豐熟,百姓安業,乃為瑞也。」己未,復置長信寺。封樞密使孛羅為澤國公。庚申,敕自今諸司白事,須殿中侍御史侍側。癸亥,賜晉王也孫鐵木兒鈔五千錠,幣、帛各二千匹;太尉不花金百兩。復雲州銀場提舉司,置儀鸞局,秩皆五品。甲子,請大行皇帝謚於南郊,上尊謚曰仁惠宣孝皇帝,廟號武宗。丙寅,拘收泉府司元給諸商販璽書。丁卯,罷只合赤八剌合孫所造上供酒。戊辰,敕諸王朝會後至者,如例給賜。己巳,衛王阿木哥入見,帝諭省臣曰:「朕與阿木哥同父而異母,朕不撫育,彼將誰賴?其賜鈔二萬錠,他勿援例。」帝覽《貞觀政要》,諭翰林侍講阿林鐵木兒曰:「此書有益於國家,其譯以國語刊行,俾蒙古、色目人誦習之。」濟寧、東平、歸德、高唐、徐、邳諸州水,給鈔賑之。河間、陜西諸縣水、旱傷稼,命有司賑之,仍免其今年租。諸王塔剌馬的遣使進馴象。



 秋七月辛未朔,拘還遼陽省官提調諸事圓符、璽書、驛券,裁減虎賁司職員,賜上都宿衛士貧乏者鈔十三萬九千錠。丁丑,鞏昌寧遠縣暴雨,山土流湧。敕內外軍官並覃官一等。癸未,甘州地震,大風,有聲如雷。以朝會,恩賜諸王禿滿金百五十兩、銀五千二百五十兩、幣帛三千匹。乙酉,賜湘寧王迭裡哥兒不花所部鈔三萬二千錠。癸巳,太陰掩畢。甲午,置經正監,掌蒙古軍牧地,秩正三品,官五員。丁酉,太陰犯鬼距星。己亥,詔諭省臣曰:「朕前戒近侍毋輒以文記傳旨中書,自今敢有犯者,不須奏聞,直捕其人付刑部究治。」敕御史臺臣選更事老成者為監察御史,超授中散大夫、典內院使孛叔榮祿大夫。是月,江陵屬縣水,民死者眾,太原、河間、真定、順德、彰德、大名、廣平等路,德、濮、恩、通等州霖雨傷稼,大寧等路隕霜,敕有司賑恤。



 閏七月辛丑,命國子祭酒劉賡詣曲阜,以太牢祠孔子。甲辰,車駕將還大都,太后以秋稼方盛,勿令鷹坊、駝人、衛士先往,庶免害稼擾民,敕禁止之。樞密院奏:「居庸關古道四十有三,軍吏防守之處僅十有三,舊置千戶,位輕責重,請置隆鎮萬戶府,俾嚴守備。」制曰:「可。」翽五星於司天臺。以故魯王刁斡八剌嫡子阿禮嘉世禮襲其封爵、分地。乙巳,以朝會,恩賜月赤察兒、床兀兒金二百兩、銀二千八百兩、幣帛有差。丙午,奉武宗神主祔於太廟。戊申,封李孟秦國公,命亦憐真乞剌思為司徒。己酉,吐番寇禮店、文州,命總帥亦憐真等討之。辛亥,以西僧藏不班八為國師,賜玉印。戊午,復置司禋監。己未,詔諭省臣曰:「國子學,世祖皇帝深所注意,如平章不忽木等皆蒙古人,而教以成才。朕今親定國子生額為三百人,仍增陪堂生二十人,通一經者,以次補伴讀,著為定式。」敕:「軍官七十致仕,始聽子弟承襲。其有未老即托疾引年,令幼弱子弟襲職者,除名不敘;其巧計求遷者,以違制論。」壬戌,命賑恤嶺北流民。上都立通政院,領蒙古諸驛,秩正二品。甲子,寧夏地震。乙丑,魯國大長公主祥哥剌吉進號皇姊大長公主。遣使招諭黑水、白水等蠻十二萬餘戶來降。丙寅,太陰犯軒轅。賜諸王阿不花等金二百兩、銀七百五十兩、鈔一萬三千六百三十錠、幣帛各有差。丁卯,完澤、李孟等言:「方今進用儒者,而老成日以凋謝,四方儒士成才者,請擢任國學、翰林、秘書、太常或儒學提舉等職,俾學者有所激勸。」帝曰:「卿言是也。自今勿限資級,果才而賢,雖白身亦用之。」敕直省舍人以其半給事殿庭,半聽中書差遣。禁醫人非選試及著籍者,毋行醫藥。大同宣寧縣雨雹,積五寸,苗稼盡殞。



 八月己巳朔,裁定京朝諸司員數,並依至元三十年舊額。楚王牙忽都所部乏食,給鈔萬錠,出粟五千石賑之。賜環衛圉人鈔三萬錠,以近侍曲列失為戶部尚書。甲戌,賜皇姊大長公主鈔萬錠。丙戌,安南世子陳日泬奉表以方物來貢。敕西番軍務隸宣政院。九月己亥朔,遙授左丞相不花進太尉。丙午,遙授湖廣平章、安南國王陳益稷入見,言:「臣自世祖朝來歸,妻子皆為國人所害,朝廷授以王爵,又賜漢陽田五百頃,俾自贍以終餘年。今臣年幾七十,而有司拘臣所授田,就食無所。」帝謂省臣曰:「安南國王慕義來歸,宜厚其賜,以懷遠人,其進勛爵、受田如故。」戊申,禁民彈射飛鳥、殺馬牛羊當乳者。禁衛士不得私衣侍宴服,及以質於人。庚戌,命樞密院閱各省軍馬。壬子,改元皇慶,詔曰:「朕賴天地祖宗之靈,纂承聖緒,永惟治古之隆,群生咸遂,國以乂寧。朕夙興夜寐,不敢怠遑,任賢使能,興滯補闕,庶其臻茲斂時五福,用敷錫厥庶民,朕之志也。逾年改元,厥有彞典,其以至大五年為皇慶元年。」都水監卿木八剌沙傳旨,給驛往取杭州所造龍舟,省臣諫曰:「陛下踐祚,誕告天下,凡非宣索,毋得擅進。誠取此舟,有乖前詔。」詔止之。復置中宮位下怯憐口諸色民匠打捕鷹坊都總管府,秩正三品。乙卯,太陰犯畢。丁巳,奉太后旨,以永平路歲入,除經費外,悉賜魯國大長公主。給雲南王老的部屬馬價一萬二千錠。丙寅,敕省部官,勿托以宿衛廢職。罷西番茶提舉司。是月,江陵路水漂民居,溺死十有八人。



 冬十月戊辰朔,有事於太廟。己巳,敕繪武宗御容,奉安大崇恩福元寺,月四上祭。辛未,賜大普慶寺金千兩,銀五千兩,鈔萬錠,西錦、彩段、紗、羅、布帛萬端,田八萬畝,邸舍四百間。丁丑,禁諸僧寺毋得冒侵民田。辛巳,罷宣政院理問僧人詞訟。以蘄縣萬戶府鎮慶元,紹興沿海萬戶府鎮處州,宿州萬戶府兼鎮臺州。戊子,省海道運糧萬戶為六員,千戶為七所。特授故太師月兒魯子木剌忽榮祿大夫、知樞密院事。辛卯,罷諸王斷事官,其蒙古人犯盜詐者,命所隸千戶鞫問。壬辰,詔收至大銀鈔。敕諸衛漢軍練習武事。置群牧監,秩正三品,掌興聖宮位下畜牧。癸巳,詔置汴梁、平江等處田賦提舉司,掌大承華普慶寺貲產。給雲南增戍軍鈔二萬五千錠。丙申,太白犯壘壁陣。



 十一月戊戌,封司徒買僧為趙國公。辛丑,命延安、鳳翔、安西軍屯田紅城者,還陜西屯田。敕:「商稅官盜稅課者,同職官贓罪。」立乖西府,以土官阿馬知府事,佩金符。李孟奏:「錢糧為國之本,世祖朝量入為出,恆務撙節,故倉庫充牣。今每歲支鈔六百餘萬錠,又土木營繕百餘處,計用數百萬錠,內降旨賞賜復用三百餘萬錠,北邊軍需又六七百萬錠;今帑藏見貯止十一萬餘錠,若此安能周給。自今不急浮費,宜悉停罷。」帝納其言,凡營繕悉罷之。辛亥,諸王不里牙屯等誣八不沙以不法,詔竄不里牙屯、禿干於河南,因忽乃於楊州,納里於湖廣,太那於江西,班出兀那於雲南。壬子,賑欽察衛糧五千七百五十三石。甲寅,太陰犯輿鬼。戊午,禁漢人、回回術者出入諸王、駙馬及大臣家。己未,以遼陽省平章政事合撒為中書平章政事。甲子,敕增置京城米肆十所,日平糶八百石以賑貧民。丙寅,加徽政使羅源為大司徒,賑諸軍糧七千六十石。



 十二月辛未,增置經正監官為八員。置尚牧所,秩五品,掌太官羊。癸酉,封宣政、會福院使暗普為秦國公。增置兵部侍郎、郎中各一員。庚辰,太白經天。復以陜西屯田軍三千隸紅城萬戶府。壬午,詔曰:「今歲不登,民何以堪?春蒐其勿令供億。」癸未,太白經天。甲申,太陰犯太微西垣上將。浙西水災,免漕江浙糧四分之一,存留賑濟;命江西、湖廣補運,輸京師。占城遣使奉表貢方物。庚寅,申禁漢人持弓矢兵器田獵。曲赦大都大闢囚一人,並流以下罪。辛卯,裁宗正府官為二十八員,遣官監視焚至大鈔。壬辰,太白經天。敕:「創設邊遠官員,俟到任方降敕牒。」乙未,命李孟整飭國子監學。中書省臣言:「世祖定立選法升降,以示激勸。今官未及考,或無故更代,或躐等進階,僭受國公、丞相等職,諸司已裁而復置者有之。今春以內降旨除官千餘人,其中欺偽,豈能悉知?壞亂選法,莫此為甚。」帝曰:「凡內降旨,一切勿行。」賜濟王朵列納印,以和林稅課建延慶寺。詔諭安南國世子陳日泬曰:「惟我祖宗,受天明命,撫有萬方,威德所加,柔遠能邇。乃者先皇帝龍馭上賓,朕以王侯臣民不釋之故,於至大四年三月十八日即皇帝位,遵逾年改元之制,以至大五年為皇慶元年。今遣禮部尚書乃馬臺等齎詔往諭,仍頒皇慶元年歷日一本。卿其敬授人時,益修臣職,毋替爾祖事大之誠,以副朕不忘柔遠之意。」



 皇慶元年春正月庚子,帝諭御史大夫塔思不花曰:「凡大臣不法,卿等劾奏毋避,朕自裁之。」癸卯,敕諸僧犯奸盜、詐偽、鬥訟,仍令有司專治之。甲辰,授太師、錄軍國重事、知樞密院事脫兒赤顏開府儀同三司,嗣淇陽王。戊申,改隆鎮萬戶府為隆鎮衛。庚戌,封知樞密院事醜漢為安遠王,出總北軍。壬子,敕軍不滿五千者,勿置萬戶。癸丑,太陰犯太微東垣上將。旌表廣州路番禺縣孝子陳韶孫。戊午,制諸王設王傅六員,銀印,其次設官四員。改封濟王朵列納為吳王,賜衛王阿木哥慶元路定海縣六萬五千戶,加崇福使也裏牙秦國公。己未,升崇祥監為崇祥院,秩正二品。壬戌,升翰林國史院秩從一品。帝諭省臣曰:「翰林、集賢儒臣,朕自選用,汝等毋輒擬進。人言御史臺任重,朕謂國史院尤重;御史臺是一時公論,國史院實萬世公論。」



 二月丁卯朔,徙大都路學所置周宣王石鼓於國子監。敕稱海屯內漢軍存恤二年。庚午,西北諸王也先不花遣使貢珠寶、皮幣、馬駝,賜鈔一萬三千六百錠。辛未,改安西路為奉元路,吉州路為吉安路。壬申,以霸州文安縣屯田水患,遣官疏決之。遣使賜西僧金五千兩、銀二萬五千兩、幣帛三萬九千九百匹。甲戌,制定封贈名爵等級,著為令。改和林省為嶺北省。丙子,給稱海屯田牛二千。賜晉王也孫鐵木兒南康路戶六萬五千,世祖諸皇子也先鐵木兒福州路福安縣、脫歡之子不答失里福州路寧德縣、忽都魯鐵木兒之子泉州路南安縣、愛牙赤之子邵武路光澤縣,戶並一萬三千六百有四,食其歲賦。己卯,置衛龍都元帥府,秩正二品,以古阿速衛隸之。八百媳婦來獻馴象二。壬午,太陰犯亢。封孛羅為永豐郡王。置德安府行用鈔庫,罷莊浪州唐兀千戶所。丙戌,省樞密斷事官為八員。庚寅,敕嶺北省賑給闕食流民。敕兩淮民種荒田者,如例輸稅。遣官同江西、江浙省整治茶、鹽法。賜韓國公主普達實憐鈔萬錠。詔勉勵學校,賑山東流民至河南境者。通、漷州饑,賑糧兩月。



 三月丁酉朔,熒惑犯東井。升給事中秩正三品,罷諸王、大臣私第營繕。戊戌,右丞相鐵木迭兒言:「自今左右司、六部官,有不盡心,初則論決,不悛,則黜而不敘。」制曰:「可。」省女直水達達萬戶府冗員。敕:「諸王脫脫所招戶,其未籍者,俾隸有司。」己亥,以生日為天壽節。庚子,加御史大夫火尼赤開府儀同三司。罷衛龍都元帥府。壬寅,太陰犯東井。敕歸德亳州,以憲宗所賜不憐吉帶地一千七十三頃還其子孫。丙子,敕:「北邊使者,非軍機毋給驛。」丁未,置內正司,秩正三品,卿、少卿、丞各一員。戊申,升典內院秩正二品。以前河南行省平章政事塔失海牙為御史大夫。改翰林國史院司直司為經歷司,置經歷、都事各一員。置五臺寺濟民局,秩從五品。賜安王完澤及其子金三百兩、銀一千二百五十兩、鈔三千五百錠,賜汴梁路上方寺地百頃。遼陽省增置灤陽、寬河驛。甲寅,西北諸王也先不花等遣使以橐駝、方物入貢。丙辰,封同知徽政院事不闌奚為趙國公。庚申,敕簡汰大明宮、興聖宮宿衛。甲子,給北軍幣帛二十萬匹,遣戶部尚書馬兒經理河南屯田。乙丑,命河南省建故丞相阿術祠堂,封諸王塔思不花為恩平王。



 夏四月丁卯,簡汰控鶴還本籍。以都水監隸大司農寺。置察罕腦兒捕盜司,秩從七品。庚午,命浙東都元帥鄭祐同江浙軍官教練水軍。辛未,給鈔萬錠修香山永安寺。趙王汝安郡告饑,賑糧八百石。升保定路萬戶府為上萬戶。癸酉,車駕幸上都。丙子,太白晝見。封鄄國大長公主忙哥臺為大長公主,賜金印。增也可扎魯忽赤為四十二員。壬午,熒惑犯輿鬼。敕皇子碩德八剌置四宿衛。敕:「僧人田除宋之舊有並世祖所賜外,餘悉輸租如制。」阿速衛指揮那懷等冒增衛軍六百名,盜支糧七千二百石、幣帛一千二百匹、鈔二百八錠,敕中書、樞密按治。封知樞密院事木剌忽為廣平王。癸未,熒惑犯積尸氣。庚寅,太白經天。大崇恩福元寺成,置隆禧院。龍興新建縣霖雨傷禾,彰德安陽縣蝗。



 五月丙申朔,以中書平章政事合散為中書左丞相,江浙行省平章張驢為中書平章政事,知樞密院事也先鐵木兒授開府儀同三司。壬寅,諸王脫忽思海迷失以農時出獵擾民,敕禁止之,自今十月方許出獵。改和林路為和寧路。賜諸王阿木哥鈔萬錠,速速迭兒、按麻思等各千錠。以蒙古驛隸通政院,置濮陽王脫脫木兒王傅官四員,給上都、灤陽驛馬三百匹。丁未,縉山縣行宮建涼殿。己酉,以西寧州田租、稅課賜大長公主忙古臺,賑宿衛士糧二萬石。升回回司天臺秩正四品。彰德、河南、隴西雹。六月乙丑朔,日有食之。丁卯,天雨毛。己巳,太陰犯天關。敕李孟博選中外才學之士任職翰林。給羊馬鈔價,濟嶺北、甘肅戍軍之貧者。壬申,減四川鹽額五千引。賜崇福寺河南官地百頃。丁亥,敕罷封贈,誡左右守法度,勤職業,忽妄僥幸加官。賜安遠王丑漢金百兩、銀五百兩、鈔千錠。鞏昌、河州等路饑,免常賦二分。



 秋七月辛丑,定內正司官為六員。禁諸王徑宣旨於各路。徙中都內帑、金銀器歸太府監。賜新店諸驛鈔三千八百錠,充使者餼廩。癸卯,詔獎勵御史臺。丙午,升大司農司秩從一品。帝諭司農曰:「農桑衣食之本,汝等舉諳知農事者用之。」敕諸王小薛部歸晉寧路襄垣縣民田。中書參政賈鈞以病請告,賜鈔三百錠,給安車還鄉。戊午,太陰犯東井。



 八月丁卯,敕探馬赤軍羊馬牛依舊制百稅其一。戊辰,太白犯軒轅。辛未,太陰犯填星。丁丑,罷司禋監。己卯,以吏部尚書許師敬為中書參知政事。庚辰,車駕至自上都。壬午,辰星犯右執法。置少府監,隸大都留守司。甲申,賜諸王闊闊出金束帶一、銀百五十兩、鈔二百錠。乙酉,太白犯右執法。辛卯,敕云南省右丞阿忽臺等,領蒙古軍從雲南王討八百媳婦蠻。濱州旱,民饑,出利津倉米二萬石,減價賑糶。寧國路涇縣水,賑糧二月。安南國王陳益稷來朝。九月丁酉,增江浙海漕糧二十萬石。戊戌,罷徵八百媳婦蠻、大、小徹里蠻,以璽書招諭之。辛丑,命司徒田忠良等詣真定玉華宮,祀睿宗御容。八百媳婦、大、小徹里蠻獻馴象及方物。甲辰,升參議中書省事阿卜海牙為參知政事。拘火者等所佩國公、司徒印。丁巳,太陰犯亢。壬戌,瓊州黎賊嘯聚,遣官招諭。



 冬十月甲子,有事於太廟。改隆興路為興和路,賜銀印。雲南行省右丞算只兒威有罪,國師搠思吉斡節兒奏請釋之,帝斥之曰:「僧人宜誦佛書,官事豈當與耶?」癸未,以中書參知政事察罕為中書平章政事,商議中書省事。丁亥,太陰犯平道。戊子,太陰犯亢。翰林學士承旨玉連赤不花等進《順宗》、《成宗》、《武宗實錄》。罷造船提舉司。辛卯,赦天下。賜李孟潞州田二十頃。



 十一月戊戌,調汀、漳畬軍代亳州等翼漢軍於本處屯田。己亥,太陰犯壘壁陣。甲辰,捕滄州群盜阿失答兒等,擒之,支解以徇。丙午,諭六部官毋逾越中書奏事。丙辰,封駙馬脫脫木兒為岐王。庚申,賜諸王寬徹、忽答迷失金百五十兩、銀一千五百兩、鈔三千錠、幣帛有差。占城國進犀象,緬國主遣其婿及雲南不農蠻酋長岑福來朝。



 十二月癸亥,中書平章政事李孟致仕,以樞密副使張珪為中書平章政事。癸酉,遣使分道決囚。壬申,晉王也孫鐵木兒所部告饑,賑鈔一萬五千錠。庚辰,知樞密院事答失蠻罷。省海道運糧萬戶一員,增副萬戶為四員。甲申,熒惑、填星、辰星聚鬥。鷹坊不花即列請往河南、湖廣括取孔雀、珍禽,敕以擾民,不允。丁亥,遣官祈雪於社稷、岳鎮、海瀆。省臣言:「中書職在總挈綱維,比者行省六部諸司應決不決者,往往作疑咨呈,以致文繁事弊。」詔體世祖立中書初意,定擬程式以聞,俾遵行之。敕回回合的如舊祈福,凡詞訟悉歸有司,仍拘還先降璽書。戊子,太陰犯熒惑。己丑,宗王女班丹給驛取江南田租,命拘還驛券。是月,諸王春丹叛。



 二年春正月甲午,以察罕腦兒等處宣慰使伯忽為御史大夫。辛丑,封前尚書右丞相乞臺普濟為安吉王。丙午,寧王闊闊出薨。丁未,以太府卿禿忽魯為中書右丞相。戊申,太陰犯三公。己未,置遼陽行省儒學提舉司。



 二月壬戌,改典內院為中政院,秩正一品。甲子,以皇后受冊寶,遣官祭告天地於南郊及太廟。丁丑,日赤如赭。己卯,免征益都饑民所貸官糧二十萬石。各寺修佛事日用羊九千四百四十,敕遵舊制,易以蔬食。命張珪綱領國子學。庚辰,冀寧路饑,禁釀酒。辛巳,詔以錢糧、造作、訴訟等事悉歸有司,以清中書之務。壬午,西北諸王也先不花進馬、駝、璞玉。丁亥,敕:「外任官應有公田而無者,皆以至元鈔給之。」以乖西府隸播州宣撫司。功德使亦憐真等以佛事奏釋重囚,不允。帝諭左右曰:「回回以寶玉鬻於官,朕思此物何足為寶,唯善人乃可為寶。善人用則百姓安,茲國家所宜寶也。」三月丙申,以御史中丞脫歡答剌罕為御史大夫。庚子,熒惑犯壘壁陣。以晉寧、大同、大寧、四川、鞏昌、甘肅饑,禁酒。丙午,冊立皇後弘吉剌氏,詔天下。丁未,彗出東井。壬子,禿忽魯言:「臣等職專燮理,去秋至春亢旱,民間乏食,而又隕霜雨沙,天文示變,皆由不能宣上恩澤,致茲災異,乞黜臣等以當天心。」帝曰:「事豈關汝輩耶?其勿復言。」御史中丞郝天挺上疏論時政,帝嘉納之。賜西僧搠思吉斡節兒鈔萬錠。丙辰,以皇后受冊寶,遣官恭謝太廟。以亢旱既久,帝於宮中焚香默禱,遣官分禱諸祠,甘雨大注。詔敦諭勸課農桑。



 夏四月甲子,翽星於司天臺。癸酉,賜壽寧公主橐駝三十六。乙亥,車駕幸上都。丙子,高麗王辭位,以其世子王燾為征東行中書省左丞相、上柱國,封高麗國王。辛巳,加御史大夫伯忽開府儀同三司、太傅。壬午,置中瑞司,秩正四品。甲申,詔遴選賢士,纂修國史。乙酉,御史臺臣言:「富人夤緣特旨,濫受官爵。徽政、宣徽用人,率多罪廢之流。近侍托為貧乏,互奏恩賞。西僧以作佛事之故,累釋重囚。外任之官,身犯刑憲,輒營求內旨以免罪。諸王、駙馬、寺觀、臣僚土田每歲征租,亦極為擾民。請悉革其弊。」制曰:「可。」詔罷不急之役。真定、保定、河間、大寧路饑,並免今年田租十之三,仍禁釀酒。安南國遣使來貢方物。



 五月辛丑,升中書右丞兀伯都剌為平章政事,左丞八剌脫因為右丞,參知政事阿卜海牙為左丞,參議中書省事禿魯花鐵木兒為參知政事,順德、冀寧路饑,辰州水,賑以米、鈔,仍禁釀酒。檀州及獲鹿縣蝻。六月己未朔,京師地震。癸亥,禿忽魯等以災異乞賜放黜,不允。丙寅,京師地震。辛未,以參知政事許思敬綱領國子學。乙亥,詔諭僧俗辯訟,有司及主僧同問,續置土田,如例輸稅。丙子,賜諸王按灰金五十兩、銀七百五十兩、金束帶一、幣帛各四十匹。己卯,河東廉訪使趙簡言:「請選方正博洽之士,任翰林侍讀、侍講學士,講明治道,以廣聖聽。」從之。御史臺臣言:「比年廉訪司多不悉心奉職,宜令監察御史檢核名實而黜陟之。廣海及雲南、甘肅地遠,遷調者憚弗肯往,乞今後加一等官之。」制曰:「可。」壬午,命監察御史檢察監學官,考其殿最。癸未,命委官簡汰衛士。甲申,建崇文閣於國子監。給馬萬匹與豳王南忽裏等軍士之貧乏者。以宋儒周敦頤、程顥、顥弟頤、張載、邵雍、司馬光、硃熹、張栻、呂祖謙及故中書左丞許衡從祀孔子廟廷。上都民饑,出米五千石減價賑糶。河決陳、亳、睢州、開封、陳留縣,沒民田廬。



 秋七月己丑朔,歲星犯東井。辛卯,太白晝見。癸巳,以作佛事,釋囚徒二十九人。賜宣寧王鐵木兒不花幣帛百二十匹,安遠王、亦思丹等各百匹。保定、真定、河間民流不止,命所在有司給糧兩月,仍悉免今年差稅,諸被災地並弛山澤之禁,獵者毋入其境。甲午,置榷茶批驗所並茶由局官。乙未,太白晝見。庚子,立長秋寺,掌武宗皇后宮政,秩三品。敕衛王阿木哥歲賜外,給鈔萬錠。賜駙馬脫鐵木兒金百五十兩、銀七百五十兩、鈔二千錠、幣帛五十匹。辛丑,復立四川等處儒學提舉司。壬寅,京師地震。免大寧路今歲鹽課。丁未,賜諸王火羅思迷、脫歡、南忽裏、駙馬忙兀帶金二百兩、銀一千二百兩、鈔一千六百錠、幣帛各有差。己酉,改淮東淮西道宣慰司為淮東宣慰司,以淮西三路隸河南省。敕守令勸課農桑,勤者升遷,怠者黜降,著為令。丙辰,太白晝見。丁巳,太白經天。雲州蒙古軍乏食,戶給米一石。興國屬縣蝻,發米賑之。



 八月戊午朔,太白晝見。揚州路崇明州大風,海潮泛溢,漂沒民居。壬戌,歲星犯東井。丁卯,車駕至自上都。庚午,以侍御史薛居敬為中書參知政事。壬午,太陰犯輿鬼。九月,以相兒加思巴為帝師。癸巳,以宣徽院使完澤知樞密院事。戊申,封脫歡為安定王,賜金印。敕鎮江路建銀山寺,勿徙寺傍塋塚。京師大旱,帝問弭災之道,翰林學士程鉅夫舉湯禱桑林事,帝獎諭之。



 冬十月己卯,敕中書省議行科舉。封不答失里為安德王。辛未,徙昆山州治於太倉,昌平縣治於新店。癸未,以遼陽路之懿州隸遼陽行省。復置蒙陰縣,隸莒州。乙酉,旌表高州民蕭乂妻趙氏貞節,免其家科差。



 十一月壬寅,敕漢人、南人、高麗人宿衛,分司上都,勿給弓矢。甲辰,行科舉。詔天下以皇慶三年八月,天下郡縣興其賢者、能者,充貢有司,次年二月,會試京師,中選者親試於廷,賜及第出身有差。帝謂侍臣曰:「朕所願者,安百姓以圖至治,然匪用儒士,何以致此。設科取士,庶幾得真儒之用,而治道可興也。」



 十二月辛酉,可里馬丁上所編《萬年歷》。發米五千石,賑阿只吉部之貧乏者。海都、都哇屬戶內附,敕所在給衣糧。丙子,定百官致仕資格。甲申,詔飭海道漕運萬戶府。京師以久旱,民多疾疫,帝曰:「此皆朕之責也,赤子何罪!」明日,大雪。以嘉定州、德化縣民災,發粟賑之。



\end{pinyinscope}