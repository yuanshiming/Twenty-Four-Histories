\article{列傳第七}

\begin{pinyinscope}

 ○察罕亦力撒合立智理威



 察罕,初名益德,唐兀烏密氏。父曲也怯律,為夏臣。其妾方懷察罕,不容於嫡母,以配掌羊群者及裏木。察罕稍長,其母以告,且曰:「嫡母已有弟矣。」察罕武勇過人,幼牧羊於野,植杖於地,脫帽置杖端,跪拜歌舞。太祖出獵,見而問之。察罕對曰:「獨行則帽在上而尊,二人行則年長者尊,今獨行,故致敬於帽。且聞有大官至,先習禮儀耳。」帝異之,乃挈以歸,語光獻皇后曰:「今日出獵得佳兒,可善視之。」命給事內廷。及長,賜姓蒙古,妻以宮人弘吉剌氏。嘗行困,脫靴藉草而寢。鴞鳴其旁,心惡之,擲靴擊之,有蛇自靴中墜。歸,以其事聞。帝曰:「是禽人所惡者,在爾則為喜神,宜戒子孫勿殺其類。」



 從帝略云中、桑乾。金將定薛擁重兵守野狐嶺,帝遣察罕覘虛實,還言彼馬足輕動,不足畏也。帝命鼓行而前,遂破其軍。圍白樓七日,拔之,以功為御帳前首千戶。從帝征西域孛哈里、薛迷思乾二城。回回國主札剌丁拒守鐵門關,兵不得進。察罕先驅開道,斬其將,餘眾悉降。又從攻西夏,破肅州。師次甘州,察罕父曲也怯律居守城中,察罕射書招之,且求見其弟。時弟年十三,命登城於高處見之。且遣使諭城中,使早降。其副阿綽等三十六人合謀,殺曲也怯律父子,並殺使者,並力拒守。城破,帝欲盡坑之,察罕言百姓無辜,止罪三十六人。進攻靈州,夏人以十萬眾赴援,帝親與戰,大敗之。還次六盤,夏主堅守中興,帝遣察罕入城,諭以禍福。眾方議降,會帝崩,諸將擒夏主殺之,復議屠中興,察罕力諫止之,馳入,安集遺民。



 太宗即位,從略河南。北還清水答蘭答八之地,賜馬三百、珠衣、金帶、鞍勒。皇子闊出、忽都禿伐宋,命察罕為斥候。又從親王口溫不花南伐,歲乙未,克棗陽及光化軍。未幾,召口溫不花赴行在,以全軍付察罕。丁酉,復與口溫不花進克光州。戊戌,授馬步軍都元帥,率諸翼軍攻拔天長縣及滁、壽、泗等州。定宗即位,賜黑貂裘一、鑌刀十,命拓江淮地。



 憲宗即位,召見,累賜金五十兩、珠衣一、金綺二匹,以都元帥兼領尚書省事,賜汴梁、歸德、河南、懷、孟、曹、濮、太原三千餘戶為食邑,及諸處草地,合一萬四千五百餘頃,戶二萬餘。未幾,復召,賜金四百五十兩、金綺、弓矢等物。乙卯卒。贈推忠開濟翊運功臣、開府儀同三司、上柱國,追封河南王,謚武宣。子十人,長木花裏。



 木花裏事憲宗,直宿衛,從攻釣魚山,以功授四斡耳朵怯憐口千戶,賜金幣及黃金馬鞍勒。世祖即位,賜金五十兩、珠二串。至元四年,攻宋,自江陵略地回,至安陽灘,宋兵扼其歸路,木花里奮擊敗之。都元帥阿術墜馬,宋軍追及之,木花裏挾之上馬鏖戰,退宋兵,由是得免。特賜銀二百五十兩,佩金虎符。為蒙古軍萬戶。復攻襄樊有功,卒於軍。贈推誠宣力功臣、榮祿大夫、平章政事、柱國,追封梁國公,謚武毅。從孫亦力撒合。



 亦力撒合,祖曲也怯祖。太祖時,得召見,屬皇子察哈臺,為扎魯火赤。父阿波古,事諸王阿魯忽,居西域。至元十年,擇貴族子備宿衛,召亦力撒合至闕下,以為速古兒赤,掌服御事,甚見親幸,有大政時以訪之,稱之曰秀才而不名。嘗奉使河西還,奏諸王只必帖木兒用官太濫,帝嘉之。擢河東提刑按察使,逐平陽路達魯花赤泰不花。召還,賜黃金百兩、銀五百兩,以旌其直。進南臺中丞。帝出內中寶刀賜之曰:「以鎮外臺。」時丞相阿合馬之子忽辛為江浙行省平章政事,恃勢貪穢,亦力撒合發其奸,得贓鈔八十一萬錠,奏而誅之。並劾江淮釋教總攝楊輦真加諸不法事,諸道竦動。二十一年,改北京宣慰使。諸王乃顏鎮遼東,亦力撒合察其有異志,必反,密請備之。二十三年,罷宣慰司,立遼陽行省,以亦力撒合為參知政事。已而乃顏果反,帝自將征之。時諸軍皆會,亦力撒合掌運糧儲,軍供無乏。東方平,帝嘉其先見,且餉運有勞,加左丞。二十七年,命尚諸王算吉女,親為資裝以送之,並贈玉帶一。改四川行省左丞。二十九年,再賜玉帶一。元貞元年,成宗即位,入朝,卒。弟立智理威。



 立智理威,為裕宗東宮必闍赤,典文書。至元十八年,蜀初定,帝閔其地久受兵,百姓傷殘,擇近臣撫安之,以立智理威為嘉定路達魯花赤。時方以闢田、均賦、弭盜、息訟諸事課守令,立智理威奉詔甚謹,民安之,使者交薦其能。會盜起雲南,號數十萬,聲言欲寇成都。立智理威馳入告急,言辭懇切,繼以泣涕。大臣疑其不然,帝曰:「雲南朕所經理,未可忽也。」乃推食以勞之。又語立智理威曰:「南人生長亂離,豈不厭兵畏禍耶?御之乖方,保之不以其道,故為亂耳。其歸以朕意告諸將,叛則討之,服則舍之,毋多殺以傷生意,則人必定矣。」立智理威至蜀,宣布上旨。俄召為泉府卿,後遷刑部尚書。有小吏誣告漕臣劉獻盜倉粟,宰相桑哥方事聚斂,眾阿其意,鍛煉枉服。立智理威曰:「刑部天下持平,今輦轂之下,漕臣以冤死,何以正四方乎?」即以實聞。以是忤丞相,出為江東道宣慰使。在官務興學,諸生有俊秀者,拔而用之。為政嚴明,豪民猾吏,縮手不敢犯,然亦無所刑戮而治。元貞二年,遷四川行省參知政事。蜀有婦人殺夫者,系治數十人,加以箠楚,卒不得其實,立智理威至,盡按得之。大德三年,以參知政事為湖南宣慰使,繼改荊湖。荊湖多弊政,而公田為甚。部內實無田,隨民所輸租取之,戶無大小,皆出公田租,雖水旱不免。立智理威問民所不便凡十數事上於朝,而言公田尤切。朝議遣使理之。會有詔,凡官無公田者,始隨俸給之,民力少蘇。七年,再遷四川行省參知政事。八年,進左丞。雲南王入朝,所在以驛騎縱獵。立智理威曰:「驛騎所以傳命令,事非有急,且不得馳,況獵乎!」王憚,為之止獵。蜀人饑,親勸分以賑之,所活甚眾。有死無葬者,則以己錢買地使葬。且修寬政以撫其民,部內以治。十年,入朝,帝以白金對衣錫之,加資德大夫、湖廣行省左丞。湖廣歲織幣上供,以省臣領工作,遣使買絲他郡,多為奸利,工官又為刻剝,故匠戶日貧,造幣益惡。立智理威不遣使,令工視賈人有藏絲者擇買之,工不告病,歲省費數萬貫。他郡推用之,皆便。至大三年,以疾卒於官,年五十七。初贈資德大夫、陜西行省右丞、上護軍、寧夏郡公,謚忠惠。再贈推誠亮節崇德贊治功臣、榮祿大夫、中書平章政事、柱國、秦國公。子二人:長買訥,翰林學士承旨;次韓嘉訥,御史大夫。孫達理麻,內府宰相。



 ○札八兒火者



 札八兒火者,賽夷人。賽夷,西域部之族長也,因以為氏。火者,其官稱也。札八兒長身美髯,方瞳廣顙,雄勇善騎射。初謁太祖於軍中,一見異之。太祖與克烈汪罕有隙。一夕,汪罕潛兵來,倉卒不為備,眾軍大潰。太祖遽引去,從行者僅十九人,札八兒與焉。至班硃尼河,餱糧俱盡,荒遠無所得食。會一野馬北來,諸王哈札兒射之,殪。遂刳革為釜,出火於石,汲河火煮而啖之。太祖舉手仰天而誓曰:「使我克定大業,當與諸人同甘苦,茍渝此言,有如河水。」將士莫不感泣。汪罕既滅,西域諸部次第亦平。乃遣札八兒使金,金不為禮而歸。金人恃居庸之塞,冶鐵錮關門,布鐵蒺藜百餘里,守以精銳。札八兒既還報,太祖遂進師,距關百里不能前,召札八兒問計。對曰:「從此而北,黑樹林中有間道,騎行可一人,臣向嘗過之。若勒兵銜枚以出,終夕可至。」太祖乃令札八兒輕騎前導。日暮入谷,黎明,諸軍已在平地,疾趨南口,金鼓之聲若自天下,金人猶睡未知也。比驚起,已莫能支吾,鋒鏑所及,流血被野。關既破,中都大震。已而金人遷汴。太祖覽中都山川形勢,顧謂左右近臣曰:「朕之所以至此者,札八兒之功為多。」又謂札八兒曰:「汝引弓射之,隨箭所落,悉畀汝為己地。」乘輿北歸,留札八兒與諸將守中都。授黃河以北鐵門以南天下都達魯花赤,賜養老一百戶,並四王府為居第。



 札八兒每戰,被重甲舞槊,陷陣馳突如飛。嘗乘橐駝以戰,眾莫能當。有丘真人者,有道之士也,隱居昆侖山中。太祖聞其名,命札八兒往聘之。丘語札八兒曰:「我嘗識公。」札八兒曰:「我亦嘗見真人。」他日偶坐,問札八兒曰:「公欲極一身貴顯乎?欲子孫蕃衍乎?」札八兒曰:「百歲之後,富貴何在?子孫無恙,以承宗祀足矣。」丘曰:「聞命矣。」後果如所願云,卒年一百一十八。贈推忠佐命功臣、太傅、開府儀同三司、上柱國,追封涼國公,謚武定。二子:阿里罕,明裏察。



 阿里罕蚤從札八兒出入行陣,勇而善謀。憲宗伐蜀,為天下質子兵馬都元帥。生哈只,終湖南宣慰使,贈推誠保德功臣、金紫光祿大夫、司徒,追封涼國公,謚安惠。生陜西行省平章政事養安、太府監丞阿思蘭、太僕寺丞補孛。養安生阿葩實,太僕寺卿。



 明裏察,贈開府儀同三司、上柱國,追封涼國公,謚康懿。生戶部尚書亦不剌金、陜西行省參知政事哈剌。



 ○術赤臺



 術赤臺,兀魯兀臺氏。其先剌真八都,以材武雄諸部。生子曰兀魯兀臺,曰忙兀,與扎剌兒、弘吉剌、亦乞列思等五人。當開創之先,協贊大業。厥後太祖即位,命其子孫各因其名為氏,號五投下。朔方既定,舉六十五人為千夫長,兀魯兀臺之孫曰術赤臺,其一也。術赤臺有膽略,善騎射,勇冠一時。初,怯列王可汗之子鮮昆有智勇,諸部畏之。怯列亦哈剌哈真沙陀等帥眾來侵,兵戰不利。近臣忽因答兒等馳告太祖曰:「事急矣,群下忠勇無逾術赤臺者,宜急遣之拒敵。」從之。術赤臺承命,單騎陷陣,射殺鮮昆,降其大將失列門等,遂並有怯列之地。乃蠻、滅兒乞臺合兵來侵,諸部有陰附之者,不虞太祖領兵卒至,諸部潰去,乘勝敗之,術赤臺俘其主扎哈堅普及二女以歸,諸部悉平,與扎哈堅普盟而歸之。未幾,乃蠻復叛,術赤臺以計襲扎哈堅普,殺之,遂平其國。術赤臺始從征怯列亦,自罕哈啟行,歷班真海子,間關萬里,每遇戰陣,必為先鋒。帝嘗諭之曰:「朕之望汝,如高山前日影也。」賜嬪御亦八哈別吉、引者思百,俾統兀魯兀四千人,世世無替。



 子怯臺,材武過人,自太宗及世祖,歷事四朝,以勞封德清郡王,賜金印。丙申,賜德州戶二萬為食邑。至元十八年,增食邑二萬一千戶,肇慶路、連州、德州洎屬邑俱隸焉。怯臺薨,子端真拔都兒襲爵為郡王。太宗時與亦剌哈臺戰,勝,帝即以亦剌哈臺妻賜之。



 世祖之征阿里不哥也,怯臺子哈答與忽都忽跪而自獻於前曰:「臣父祖幸在先朝,當軍旅征伐之寄,屢立戰功。今王師北征,臣等幸少壯,願如父祖以力戰自效。」既得請,於是戰於石木溫都之地。諸王哈丹、駙馬臘真與兀魯、忙兀居右,諸王塔察兒及太丑臺居左,合必赤將中軍。兵始交,獲其將合丹斬之,外剌之軍遂敗衄。又戰於失烈延塔兀之地,當帝前混戰,至日晡勝之。帝賜以黃金,將佐吏卒行賞各有差。李鋋叛,帝遣哈必赤及兀里羊哈臺闊闊出往討之,哈答與兀魯納兒臺亦在行。鋋平,與有功焉。



 哈答子脫歡,亦嘗從諸王徹徹都討只兒火臺,獲之。又嘗破失烈吉、要不忽兒於野孫漢連。及徵乃顏,脫歡弟慶童亦在軍,雖病,猶力戰。



 怯臺二子:曰端真,曰哈答。哈答三子:曰脫歡,曰亦鄰只班,曰慶童。脫歡二子:曰塔失帖木兒,曰朵來。塔失帖木兒一子,曰匣剌不花。自怯臺而下凡九人,皆封郡王云。



 ○鎮海



 鎮海,怯烈臺氏。初以軍伍長從太祖同飲班硃尼河水。與諸王百官大會兀難河,上太祖尊號曰成吉思皇帝。歲庚午,從太祖征乃蠻有功,賜良馬一。壬申,從攻曲出諸國,賜珍珠旗,佩金虎符,為闍里必。從攻塔塔兒、欽察、唐兀、只溫、契丹、女直、河西諸國,所俘生口萬計,悉以上獻,賜御用服器白金等物。命屯田於阿魯歡,立鎮海城戍守之。壬申,從太祖謀定漢地,師次隆興,與金將忽察虎戰,矢中臆間,裹瘡而出者復數四,軍聲為之大振。既破燕,太祖命於城中環射四箭,凡箭所至圓池邸舍之處,悉以賜之。尋拜中書右丞相。己丑,太宗即位,扈從至西京,攻河中、河南、鈞州。癸巳,攻蔡州。以功賜恩州一千戶。先是,收天下童男童女及工匠,置局弘州。既而得西域織金綺紋工三百餘戶,及汴京織毛褐工三百戶,皆分隸弘州,命鎮海世掌焉。定宗即位,以鎮海為先朝舊臣,仍拜中書右丞相。薨,年八十四。



 子十人,勃古思繼食其封邑。從世祖征花馬大理,率兵千人,結浮橋於金沙江以濟師。中統初,論功授益都等路宣撫使,賜金虎符、玉帶。三年,改東平路副達魯花赤,討平叛寇。尋遷濟南等路宣慰。至元二年,遷南京路達魯花赤。四年,討平蘄縣叛民。以病乞謝事,特授保定路達魯花赤,賜錢一萬貫,歸老於家,卒年八十一。



 ○肖乃臺



 肖乃臺,禿伯怯烈氏,以忠勇侍太祖。時木華黎、博兒術既立為左右萬戶,帝從容謂肖乃臺曰:「汝願屬誰麾下為我宣力?」對曰:「願屬木華黎。」即日命佩金符,領蒙古軍,從太師國王為先鋒。兵至河北,史天澤之父率老幼數千詣軍門降。國王承制,授天澤兄天倪河北西路都元帥,領真定。乙酉,天澤送母還白霫,副帥武仙殺天倪,以真定叛。經歷王縉追天澤至燕,請攝主帥。遣監軍李伯祐詣國王軍前言狀,且請援兵。國王命肖乃臺率精甲三千,與天澤合兵進圍中山。仙遣其將葛鐵槍來援,肖乃臺撤圍迎之,遇諸新樂,奮擊敗之。會日暮,阻水為營。肖乃臺料其氣索,必宵遁,乘勝復進擊,大敗之,擒鐵槍。中山守將亦宵遁,遂克中山,取無極,拔趙州。仙棄真定,奔西山抱犢寨。肖乃臺與天澤入城,撫定其民。未幾,仙潛結水軍為內應,夜開南門納仙,復據其城。肖乃臺倉卒以步兵七十逾城,奔槁城。遲明,部曲稍來集,兵威復振,襲取真定,仙棄城遁。將士怒民之反覆,驅萬人出,將屠之。肖乃臺曰:「金氏慕國威信,傒我來蘇,此民為賊所驅脅,有何罪焉?若不勝一朝之忿,非惟自屈其力,且堅他城不降之心。」乃皆釋之。初,仙之叛也,其弟質國王軍中,聞之遁去。肖乃臺遣弟撒寒追及於紫荊關,斬之,俘其妻子而還。乃整兵前進,下太原,略太行,拔長勝寨,斬仙守將盧治中,圍仙於雙門寨,仙遁去。引兵出太行山東,遇宋將彭義斌,與戰,敗之,追至火炎山,破其營,擒義斌斬之。至大名,守將蘇元帥以城降,遂引兵臨東平,敗安撫王立剛於陽谷,圍東平。立剛走漣水,金守將棄城遁,他將邀擊敗之,遂定東平。又與蒙古不花徇河北、懷、孟、衛,從國王定益都。壬辰,度河,略汴京,徇睢州,遇金將完顏慶山奴,與戰,敗之,追斬慶山奴。金主入蔡,諸軍圍之。肖乃臺、史天澤攻城北面,汝水阻其前,結筏潛渡,血戰連日。金亡,朝廷以肖乃臺功多,命並將史氏三萬戶軍以圖南征,賜東平戶三百,俾食其賦,命嚴實為治第宅,分撥牧馬草地,日膳供二羊及衣糧等。以老病卒於東平,歸葬漠北。子七人,抹兀答兒、兀魯臺知名。



 抹兀答兒,歲戊戌,從國王忽林赤行省於襄陽,略地兩淮。己未,從渡江,攻鄂州,以功賞銀五十兩。中統元年,追阿蘭答兒、渾都海,預有戰功。二年,從北征,敗阿里不哥於失木禿之地。三年,又與李璮戰,有功。國王忽林赤上其功,奉旨賞銀五十兩,授提舉本投下諸色匠戶達魯花赤。卒。子四人,火你赤,江南行臺御史大夫。



 兀魯臺,中統三年,從石高山奉旨拘集探馬赤軍,授本軍千戶。至元八年,授武略將軍,佩銀符。十年,攻樊城有功,換金符,武德將軍。十一年,渡江有功,賞銀三百兩,改武節將軍。十二年四月,軍至建安,卒於軍。



 子脫落合察兒襲職,從參政阿剌罕攻獨松關有功,升宣武將軍。尋命管領侍衛軍。樞密院錄其渡江以來累次戰功,十八年,升懷遠大將軍。二十年,江西行省命討武寧叛賊董琦,平之,改授虎符、江州萬戶府達魯花赤。二十四年,移鎮潮州,值賊張文惠、羅半天等嘯聚江西,行樞密院檄討之,領兵破賊寨,斬賊首羅大老、李尊長等,獲其偽銀印三。卒於軍。



 ○吾也而



 吾也而,珊竹氏,狀貌甚偉,腰大十圍。父曰圖魯華察,以武勇稱。太祖五年,吾也而與折不那演克金東京,有功。九年,從太師木華黎取北京,領兵為先驅,下之。捷聞,授金紫光祿大夫、北京總管都元帥。留撫其人,綏懷有方,自京以南,相繼來降。時金將撻魯,以惠州漁河口為隘,有眾數萬,圖復北疆。吾也而以銳兵千人擊摧其鋒,殺數千人,獲其旗鼓羊馬,斬撻魯於軍中。有趙守玉者,據興州,吾也而討平之。十一年,張致以錦州叛,又攻破之。木華黎大喜,以馬十匹、甲五事賞其功。十二年,興州監軍重兒以兵叛,吾也而往征之,賊軍射殺所乘馬,軍士憤怒,奮戈沖擊,大破賊軍。十五年,從征山東,大戰東平,馳赴陷陣,生挾二將以還。木華黎壯之,以功上聞。十六年,從徵延安,矢中右股,力戰破之。俄又取葭、鄜二州,擒金梟將張鐵槍以獻。十七年,克鳳翔及所屬州郡。十八年,從帝親征河西,明年下之。詔賜吾也而馬五匹、甲一事。二十年,從木華黎圍益都。越二年,下三十餘城。太宗元年,入覲。命與撒裡答火兒赤征遼東,下之。三年,又與撒裡答征高麗,下受、開、龍、宣、泰、葭等十餘城。高麗懼,請和。吾也而諭之曰:「若能以子為質,當休兵。」十三年,遣其子綧從吾也而來朝。帝大悅,厚加賜予,俾充北京東京廣寧蓋州平州泰州開元府七路征行兵馬都元帥,佩虎符。憲宗元年,召問東夷事,對曰:「臣雖老,倘藉威靈,指麾三軍,敵國猶可克,況東夷小醜乎!」帝壯其言,問飲酒幾何,對曰:「唯所賜。」時有一駙馬都尉在側,素以酒稱,命與之角飲。帝大笑,賜錦衣名馬。俄謝病歸。七年,復來朝,帝憫其老,謂曰:「自太祖時效勞至今者,獨卿無愆。」賜賚甚厚,以都元帥授其中子阿海。八年秋九月辛亥夜中,星隕帳前,光數丈,有聲。吾也而曰:「吾死矣。」明日卒。年九十六。



 子四人,霅禮最有名,太宗時授北京等路達魯花赤。至元七年,改授昭勇大將軍、河間路總管。



 ○曷思麥里



 曷思麥里,西域谷則斡兒朵人。初為西遼闊兒罕近侍,後為穀則斡兒朵所屬可散八思哈長官。太祖西征,曷思麥里率可散等城酋長迎降,大將哲伯以聞。帝命曷思麥里從哲伯為先鋒,攻乃蠻,克之,斬其主曲出律。哲伯令曷思麥裡持曲出律首往徇其地,若可失哈兒、押兒牽、斡端諸城,皆望風降附。又從征你沙不兒城,諭下之。帝親征至薛迷思乾,與其主扎剌丁合戰於月戀揭赤之地,敗之。追襲扎剌丁等於阿剌黑城,戰於禿馬溫山,又敗之。追至憨顏城西寨,又敗之。扎剌丁逃入於海。曷思麥裡收其珍寶以還。取玉兒谷、德痕兩城。繼而憨顏城亦下。帝遣使趣哲伯疾馳以討欽察。命曷思麥裏招諭曲兒忒、失兒灣沙等城,悉降。至谷兒只部及阿速部,以兵拒敵,皆戰敗而降。又招降黑林城,進擊斡羅思於鐵兒山,克之,獲其國主密只思臘,哲伯命曷思麥里獻諸術赤太子,誅之。尋徵康里,至孛子八里城,與其主霍脫思罕戰,又敗其軍,進至欽察亦平之。軍還,哲伯卒。會帝親征河西,曷思麥裡持所獲珍寶及七寶傘迎見於阿剌思不剌思,帝顧群臣曰:「哲伯常稱曷思麥里之功,其軀幹雖小,而聲聞甚大。」就以所進金寶,命隨其力所勝,悉賜之。仍命與薛徹兀兒為必闍赤。未幾,曷思麥裏奏,往者嘗招安到士卒留亦八里城,宜令扈從征河西,許之,命常居左右。至也吉里海牙,又討平失的兒威。從太祖征汴,至懷孟,令領奧魯事。帝由白坡渡黃河,會睿宗兵攻金將合達,敗之,回駐金蓮川。壬辰,授懷孟州達魯花赤,佩金符。癸巳,金將強元帥圍懷州,曷思麥里率其眾及昔里吉思、鎖剌海等力戰,金兵退。又遣蒲察寒奴、乞失烈札魯招諭金總帥範真率其麾下軍民萬餘人來降。己亥六月,帝以曷思麥里從軍西域,宣力居多,命其長子捏只必襲為懷孟達魯花赤,次子密里吉襲為必闍赤,令曷思麥里為扎魯火赤,歸西域。大帥察罕、行省帖木迭兒奏留之,帝允其請。庚子,進懷孟河南二十八處都達魯花赤,所隸州郡不從命者,制令籍其家。乙卯五月卒。



 子密里吉復為懷孟達魯花赤。中統三年,從攻淮西,與宋戰死。



\end{pinyinscope}