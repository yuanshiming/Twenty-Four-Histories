\article{本紀第四十三 順帝六}

\begin{pinyinscope}

 十三年春正月庚午朔,用帝師請,釋放在京罪囚。以中書添設平章政事哈麻為平章政事,參知政事悟良哈臺為右丞,參知政事烏古孫良楨為左丞。詔印造中統元寶交鈔一百九十萬錠、至元鈔一十萬錠。辛未,命悟良哈臺、烏古孫良楨兼大司農卿,給分司農司印。西自西山,南至保定、河間,北至檀、順州,東至遷民鎮,凡系官地及元管各處屯田,悉從分司農司立法佃種,合用工價、牛具、農器、穀種、召募農夫諸費,給鈔五百萬錠,以供其用。旌表真定路槁城縣董氏婦貞節。壬申,命陜西行省平章政事卜答失里為總兵官。癸酉,享於太廟。以皇第二子育於太尉眾家奴家,賜眾家奴及乳母鈔各一千錠。甲戌,重建穆清閣。乙亥,命中書右丞禿禿以兵討商州賊。丙子,方國珍復降。以司農司舊署賜哈麻。庚辰,中書省臣言:「近立分司農司,宜於江浙、淮東等處召募能種水田及修築圍堰之人各一千名為農師,教民播種。宜降空名添設職事敕牒一十二道,遣使齎往其地,有能募農民一百名者授正九品,二百名者正八品,三百名者從七品,即書填流官職名給之,就令管領所募農夫,不出四月十五日,俱至田所,期年為滿,即放還家。其所募農夫,每名給鈔十錠。」從之。以杜秉彞為中書參知政事。乙酉,太陰犯太微垣。丙戌,以武衛所管鹽臺屯田八百頃,除軍見種外,荒閑之地,盡付分司農司。答失八都魯克復襄陽、樊城有功,升四川行省右丞,賜金系腰一。庚寅,知樞密院事老章克復南陽唐州,賜金一錠、銀一十錠、鈔一千錠、幣帛各五十匹。戊戌,熒惑、太白、辰星聚於奎宿。



 二月丁未,祭先農。己酉,太陰犯軒轅。庚戌,太白犯熒惑。壬子,太陰犯太微垣。甲寅,中書省臣言徐州民願建廟宇,生祠右丞相脫脫,從之,詔仍立脫脫《平徐勛德碑》。壬戌,以宣政院使篤憐帖木兒知經筵事,中書右丞悟良哈臺、左丞烏古孫良楨、參知政事杜秉彞並同知經筵事。



 三月己卯,命脫脫領大司農司。甲申,詔修大承天護聖寺,賜鈔二萬錠。丁亥,命脫脫以太師開府,提調太史院、回回、漢兒司天監。己丑,以各衙門系官田地並宗仁等衛屯田地土,並付分司農司播種。是月,會州、定西、靜寧、莊浪等州地震。命江浙行省左丞帖裏帖木兒、江南行臺侍御史左答納失裏招諭方國珍。



 夏四月戊戌朔,命南北兵馬司各分官一員,就領通州、漷州、直沽等處巡捕官兵,往來巡邏,給分司印,一同署事,半載一更。特命烏古孫良楨得用軍器。庚子,以禮部所轄掌薪司並地土給付分司農司。以甘肅行省平章政事鎖南班為永昌宣慰使,總永昌軍馬,仍給平章政事俸。先是,永昌愚魯罷等為亂,鎖南班討平之,至是復起,故有是命。辛丑,太白犯井宿。乙巳,時享太廟。己酉,詔取勘徐州、汝寧、南陽、鄧州等處荒田並戶絕籍沒入官者。立司牧署,掌分司農司耕牛。又立玉田屯署。降徐州路為武安州,以所轄縣屬歸德府,其滕州、嶧州仍屬益都路。辛亥,太陰犯房宿。是月,車駕時巡上都。



 五月己巳,命東安州、武清、大興、宛平三縣正官添給河防職名,從都水監官巡視渾河堤岸,或有損壞,即修理之。辛未,江西行省左丞相亦憐真班、江浙行省左丞老老引兵取道自信州,元帥韓邦彥、哈迷取道由徽州、浮梁,同復饒州,蘄、黃等賊聞風皆奔潰。癸酉,以太尉阿剌吉為嶺北行省左丞相。知行樞密院事伯家奴封武國公,與諸王孛羅帖木兒同出軍。甲戌,行樞密院添設僉院二員。乙亥,太陰犯歲星。乙未,泰州白駒場亭民張士誠及其弟士德、士信為亂,陷泰州及興化縣,遂陷高郵,據之,僭國號大周,自稱誠王,建元天祐。六月丙申朔,立詹事院,設詹事三員、同知二員、副詹事二員、丞二員。命四川行省平章政事玉樞虎兒吐華便宜行事。丁酉,立皇子愛猷識理達臘為皇太子、中書令、樞密使,授以金寶,告祭天地、宗廟。命右丞相脫脫兼詹事。乙亥,詔征西都元帥汪只南發本處精銳勇敢軍一千人從征討,以千戶二員、百戶一十員領之。庚子,知樞密院事失剌把都總河南軍,平章政事答失八都魯總四川軍,自襄陽分道而下,克復安陸府。辛丑,罷宮傅府,以所掌錢帛歸詹事院。癸卯,詔以敕牒二十道、鈔五萬錠,給付淮南行省平章政事達識帖睦邇,於淮南、淮北等處召募壯丁,並總領漢軍、蒙古守禦淮安。遼東搠羊哈及乾帖困、術赤術等五十六名吾者野人以皮貨來降,給搠羊哈等三人銀牌一面,管領吾者野人。甲辰,以立皇太子詔天下,大赦。己酉,亦都護高昌王月魯帖木兒薨於南陽軍中,命其子桑哥襲亦都護高昌王爵。辛亥,親王完者禿泰州陣亡,八禿亳州陣亡,各賻鈔五百錠。命前河西廉訪副使也先不花為淮西添設宣慰副使,討泰州。丙辰,詔皇太子位下立儀衛司,設指揮二員,給二珠金牌,副指揮二員,一珠金牌。賜吳王搠思監金二錠、銀五錠、鈔二千錠、幣帛各九匹。以資政院所轄左、右都威衛屬詹事院。是月,命淮南行省平章政事達識帖睦邇便宜行事。詔淮南行省平章政事福壽討興化。是夏,薊州大水。



 秋七月丁卯,泉州天雨白絲,海潮日三至。時享太廟。戊辰,太白晝見。宦官至一品二品者,依常例給俸祿。壬申,湖廣行省參知政事阿魯輝復武昌及漢陽府。癸酉,詔詹事院自行銓注本院屬官。壬辰,親王只兒哈忽薨於海寧軍中,以其子寶童繼襲王爵。



 八月癸卯,親王闊兒吉思、帖木兒獻馬。辛亥,賜脫脫東泥河田一十二頃。親王只兒哈郎討捕金山賊,薨於軍中,命其子禿魯帖木兒入備宿衛。庚申,命不花帖木兒襲封文濟王。是月,車駕還自上都。資政院使脫火赤以兵復江州路。以四川行省平章政事玉樞虎兒吐華、右丞完者不花守鎮中興路。左遷平章政事咬住為淮西元帥,供給烏撒軍,進討蘄、黃。九月乙丑朔,日有食之。乙亥,以怯薛官廣平王咬咬征討慢功,削其王爵,降為河南行省平章政事。己丑,廣寧王渾都帖木兒薨,賻鈔一千錠。建皇太子鹿頂殿於聖安殿西。歪剌歹桑哥失里獻馬一百匹,賜金系腰一、幣帛各九。庚寅,太陰犯熒惑。辛卯,扎你別之地獻大撒哈剌、察亦兒、米西兒刀、弓、鎖子甲及青、白西馬各二匹,賜鈔二萬錠。壬辰,太白經天,熒惑犯左執法。南臺御史大夫納麟以老疾辭職,從之,命太尉如故。



 冬十月丁酉,享於太廟。庚子,太白經天。癸卯,以江浙行省參知政事買住丁升本省右丞,提調明年海運。甲辰,歲星犯氐宿。丁未,廣西元帥甄崇福復道州,誅賊將周伯顏。庚戌,從帖裏帖木兒、左答納失里之請,授方國珍徽州路治中,國璋廣德路治中,國瑛信州路治中,督遣之任,國珍疑懼,不受命。立水軍都萬戶府於昆山州,以浙東宣慰使納麟哈剌為正萬戶,宣慰使董搏霄為副萬戶。庚申,賜皇太子妃鈔十萬錠。壬戌,賜皇太子五愛馬怯薛丹二百五十人鈔各一百一十錠。癸亥,太白犯亢宿。是月,撤世祖所立氈殿,改建殿宇。



 十一月壬申,太陰犯壘壁陣。乙酉,立典藏庫,貯皇太子錢帛。丁亥,江西右丞火你赤以兵平富州、臨江,遂引兵復瑞州。是月,立義兵千戶、水軍千戶所於江西,事平,願還為民者聽。



 十二月丁酉,太白犯東咸。己亥,寧王旭滅該還大斡耳朵思,賜金系腰一、鈔一千錠。庚子,熒惑入氐宿。癸卯,脫脫請以趙完普家產田地賜知樞密院事桑哥失里。庚戌,京城天無雲而雷鳴,少頃,有火見於東南。淮慶路及河南府西北有聲如擊鼓者數四,已而雷聲震地。癸丑,以西安王阿剌忒納失里為豫王;弟答兒麻討南陽賊有功,以西安王印與之,命鎮寵吉兒之地。丁巳,太陰犯心宿。西寧王牙罕沙鎮四川,還沙州,賜鈔一千錠。是月,大同路疫,死者大半。江浙行省平章政事卜顏帖木兒、南臺御史中丞蠻子海牙及四川行省參知政事哈臨禿、左丞桑禿失里、西寧王牙罕沙,合軍討徐壽輝於蘄水,敗之,壽輝遁走,獲其偽官四百餘人。陜西行省平章政事孛羅、四川行省右丞答失八都魯復均、房等州,詔孛羅等守之,答失八都魯討東正陽。是歲,自六月不雨至於八月。造清寧殿前山子、月宮諸殿宇,以宦官留守也先帖木兒、留守同知也速迭兒及都水少監陳阿木哥等董其役。哈麻及禿魯帖木兒等陰進西天僧於帝,行房中運氣之術,號演揲兒法,又進西番僧善秘密法,帝皆習之。



 十四年春正月甲子朔,汴梁城東汴河冰,皆成五色花草如繪畫,三日方解。乙丑,熒惑犯歲星。丁卯,太白犯建星。辛未,享於太廟。壬申,命帖木兒不花襲封廣寧王,賜鈔一千錠。癸酉,熒惑犯房宿。立遼陽等處漕運庸田使司,屬分司農司。丁丑,帝謂脫脫曰:「朕嘗作朵思哥兒好事,迎白傘蓋游皇城,實為天下生靈之故。今命剌麻選僧一百八人,仍作朵思哥兒好事,凡所用物,官自給之,毋擾於民。」丙戌,以答兒麻監藏遙授陜西行省平章政事,實授行宣政院使,整治西番人民。是月,命桑哥失里、哈臨禿守中興。答失八都魯復峽州。



 二月戊戌,祭社稷。乙卯,命中書平章政事搠思監提調規運總管府。戊午,太白犯壘壁陣。己未,以湖廣行省平章政事茍兒為淮南行省平章政事,以兵攻高郵。是月,以呂思誠為湖廣行省左丞。命湖廣行省右丞伯顏普化、江南行臺中丞蠻子海牙、江浙行省平章政事卜顏帖木兒、參知政事阿里溫沙,會合湖廣行省平章政事也先帖木兒討沿江賊。立鎮江水軍萬戶府,命江浙行省右丞佛家閭領之。詔河南、淮南兩省並立義兵萬戶府。建清河大壽元忠國寺,以江浙廢寺田歸之。



 三月癸亥朔,日有食之。己巳,廷試進士六十二人,賜薛朝晤、牛繼志進士及第,餘授官出身有差。壬申,以皇太子行幸,和買駝馬。甲戌,命親王速哥帖木兒以兵討宿州賊。丙子,潁州陷。是月,中書定擬義兵立功者權任軍職,事平授以民職,從之。命四川行省右丞答失八都魯升本省平章政事兼知行樞密院事,總荊、襄諸軍,從宜調遣。詔和買馬於北邊以供軍用,凡有馬之家,十匹內和買二匹,每匹給鈔一十錠。



 夏四月癸巳朔,汾州介休縣地震,泉湧。以武祺參議中書省事。是月,車駕時巡上都。江西、湖廣大饑,民疫癘者甚眾。御史臺臣糾言江浙行省左丞帖裏帖木兒等罪。先是,帖裏帖木兒與江南行臺侍御史左答納失里奉旨招諭方國珍,報國珍已降,乞立巡防千戶所,朝廷授以五品流官,令納其船,散遣徒眾,國珍不從,擁船一千三百餘艘,仍據海道,阻絕糧運,以故歸罪二人。以江浙行省參知政事阿兒溫沙升本省右丞,浙東宣慰使恩寧普為江浙行省參知政事,皆總兵討方國珍。發陜西軍討河南賊,給鈔令自備鞍馬軍器,合二萬五千人,馬七千五百匹,永昌、鞏昌沿邊人匠雜戶亦在遣中。造過街塔於蘆溝橋,命有司給物色人匠,以御史大夫也先不花督之。復立應昌、全寧二路。先是,有詔罷之,以撥屬魯王馬某沙王傅府,至是有司以為不便,復之。詔復起永昌、鞏昌、喃巴、臨洮等處軍。命各衛軍人修白浮、甕山等處堤堰。



 五月甲子,安豐、正陽賊圍廬州。是月,詔修砌北巡所經色澤嶺、黑石頭河西沿山道路,創建龍門等處石橋。皇太子徙居宸德殿,命有司修葺之。立南陽、鄧州等處毛胡蘆義兵萬戶府,募土人為軍,免其差役,令討賊自效。因其鄉人自相團結,號毛胡蘆,故以名之。詔以玉樞虎兒吐華募兵萬人下蜀江,代答失八都魯守中興、荊門;命答失八都魯以兵赴汝寧。升湖廣行省參知政事阿兒灰為右丞,討廬州。募寧夏善射者及各處回回、術忽殷富者赴京師從軍。復發禿卜軍萬人,命太傅阿剌吉領之。命荊王答兒麻失裡代闊端阿合鎮河西,討西番賊。六月辛卯朔,薊州雨雹。高郵張士誠寇揚州。丙申,達識帖睦邇以兵討張士誠,敗績,諸軍皆潰。詔江浙行省參知政事佛家閭會達識帖睦邇,復進兵討之。甲辰,太陰入斗宿。己酉,盱眙縣陷。庚戌,陷泗州,官軍潰。



 秋七月甲子,潞州襄垣縣大風拔木偃禾。乙丑,太陰犯角宿。壬申,詔免大都、上都、興和三路今年稅糧。命刑部尚書阿魯於汝寧州等處募兵討泗州。壬午,太陰犯昴宿。是月,汾州孝義縣地震。



 八月,冀寧路榆次縣桃李花。車駕還自上都。九月己未朔,賜親王撒蠻答失金二錠、銀二十錠、鈔一萬錠、幣帛表裏各三百匹。創設奧剌赤二十名,仍給衣糧草料。庚申,以湖廣行省左丞呂思誠復為中書左丞。辛酉,以知樞密院事月闊察兒為中書平章政事。詔脫脫以太師、中書右丞相,總制諸王各愛馬、諸省各翼軍馬,董督總兵、領兵大小官將,出征高郵。甲子,封高麗國王脫脫不花為沈王。丁卯,普顏忽都皇后母歿,賻鈔三百錠。立寧宗影堂。戊子,免河南蒙古軍人雜泛差役。是月,賜穆清閣工匠皮衣各一領。蓋海青鷹房。禁河南、淮南酒。階州西番賊起,遣兵擊之。方國珍拘執元帥也忒迷失、黃巖州達魯花赤宋伯顏不花、知州趙宜浩,以俟詔命。



 冬十月甲午,享於太廟。戊戌,詔答失八都魯及泰不花等會軍討安豐。甲辰,詔加號海神為輔國護聖庇民廣濟福惠明著天妃。壬子,太陰犯太微垣。



 十一月丙寅,敕:「中書省、樞密院、御史臺,凡奏事先啟皇太子。」詔:「江浙應有諸王、公主、后妃、寺觀、官員撥賜田糧,及江淮財賦、稻田、營田各提舉司糧,盡數赴倉,聽候海運,以備軍儲,價錢依本處十月時估給之。」丁卯,脫脫領大兵至高郵,辛未,戰於高郵城外,大敗賊眾。丙子,太陰犯鬼宿。癸未,賜親王喃答失金鍍銀印。乙酉,脫脫遣兵平六合縣。是月,答失八都魯復苗軍所據鄭、均、許三州。皇太子修佛事,釋京師死罪以下囚。



 十二辛卯,絳州北方有紅氣如火蔽天。丙申,以中書平章政事定住為左丞相;宣政院使哈麻、永昌宣慰鎖南班並為中書平章政事,進階光祿大夫。監察御史袁賽因不花等劾奏:「脫脫出師三月,略無寸功,傾國家之財以為己用,半朝廷之官以為自隨。又其弟也先帖木兒,庸材鄙器,玷污清臺,綱紀之政不修,貪淫之心益著。」章三上,詔令也先帖木兒出都門聽旨,以宣徽使汪家奴為御史大夫。丁酉,詔以脫脫老師費財,已逾三月,坐視寇盜,恬不為意,削脫脫官爵,安置淮安路,弟御史大夫也先帖木兒安置寧夏路。以河南行省平章政事泰不花為本省左丞相,中書平章政事月闊察兒加太尉,集賢大學士雪雪知樞密院事,一同總兵,總領諸處征進軍馬,並在軍諸王、駙馬、省、院、臺官及大小出軍官員,其滅里、卜亦失你山、哈八兒禿、哈怯來等拔都兒、雲都赤、禿兒怯里兀、孛可、西番軍人、各愛馬朵憐赤、高麗、回回民義丁壯等軍人,並聽總兵官節制。詔:「被災殘破之處,令有司賑恤,仍蠲租稅三年。賜高年帛。」罷庸田、茶運、寶泉等司。戊戌,以定住領經筵事,中政院使桑哥失里為中書添設右丞。己亥,太陰掩昴宿。庚子,以桑哥失里同知經筵事。冀國公禿魯加太尉,進階金紫光祿大夫。癸卯,命哈麻提調經正監、都水監、會同館,知經筵事,就帶元降虎符。甲辰,以桑哥失裡提調宣文閣;哈麻兼大司農,呂思誠兼司農卿,提調農務。己酉,紹興路地震。是月,命織造世祖御容。詔威順王寬徹普化還鎮湖廣。先是以賊據湖廣,命奪其王印,至是寬徹普化討賊累立功,故詔還其印,仍守舊鎮。命甘肅右丞嵬的討捕西番賊。答失八都魯復河陰、鞏縣。徭賊自耒陽寇衡州,萬戶許脫因死之。是歲,詔諭:「民間私租太重,以十分為率普減二分,永為定例。」降鈔十萬錠賞江西守城官吏軍民。京師大饑,加以疫癘,民有父子相食者。帝於內苑造龍船,委內官供奉少監塔思不花監工。帝自制其樣,船首尾長一百二十尺,廣二十尺,前瓦簾棚、穿廊、兩暖閣,後吾殿樓子,龍身並殿宇用五彩金妝,前有兩爪。上用水手二十四人,身衣紫衫,金荔枝帶,四帶頭巾,於船兩旁下各執篙一。自後宮至前宮山下海子內,往來游戲,行時,其龍首眼口爪尾皆動。又自制宮漏,約高六七尺,廣半之,造木為匱,陰藏諸壺其中,運水上下。匱上設西方三聖殿,匱腰立玉女捧時刻籌,時至,輒浮水而上。左右列二金甲神,一懸鐘,一懸鉦,夜則神人自能按更而擊,無分毫差。當鐘鉦之鳴,獅鳳在側者皆翔舞。匱之西東有日月宮,飛仙六人立宮前,遇子午時,飛仙自能耦進,度仙橋,達三聖殿,已而復退立如前。其精巧絕出,人謂前代所鮮有。時帝怠於政事,荒於游宴,以宮女三聖奴、妙樂奴、文殊奴等一十六人按舞,名為十六天魔,首垂發數辮,戴象牙佛冠,身被纓絡、大紅綃金長短裙、金雜襖、雲肩、合袖天衣、綬帶鞋襪,各執加巴剌般之器,內一人執鈴杵奏樂。又宮女一十一人,練槌髻,勒帕,常服,或用唐帽、窄衫,所奏樂用龍笛、頭管、小鼓、箏、緌、琵琶、笙、胡琴、響板、拍板。以宦者長安迭不花管領,遇宮中贊佛,則按舞奏樂。宮官受秘密戒者得入,餘不得預。



\end{pinyinscope}