\article{志第二十 禮樂三}

\begin{pinyinscope}

 ○郊祀樂章



 成宗大德六年,合祭天地五方帝樂章:



 降神,奏《乾寧之曲》,六成:



 圜鐘宮三成



 惟皇上帝,監德昭明。祀考承天,治底隆平。孝思維則,禋祀薦誠。神其降格,萬福來並。



 黃鐘角一成詞同前。



 太簇徵一成詞同前。



 姑洗羽一成詞同前。



 初獻盥洗,奏《肅寧之曲》:



 黃鐘宮



 明水在下,鐘鼓既奏。有孚顒若,陟降左右。闢公處之,多士稞將。吉蠲以祭,上帝其饗。



 初獻升降,奏《肅寧之曲》:



 大呂宮



 禋祀孔肅,盥薦初升。攝齊恭敬,以薦惟馨。肅雝多士,來格百靈。降福受厘,萬世其承。



 奠玉幣,奏:



 大呂宮



 宗祀配饗,肇舉明禋。嘉玉既設,量幣斯陳。惟德格天,惟誠感神。於萬斯年,休命用申。



 迎俎,奏《豐寧之曲》:



 黃鐘宮



 有碩斯俎,有滌斯牲。鑾刀屢奏,血膋載升。禮崇繭慄,氣達尚腥。上帝臨止,享於克誠。



 酌獻,奏《嘉寧之曲》:



 大呂宮



 崇崇泰畤,穆穆昊穹。神之格思,祇蚃斯通。犧尊載列,黃流在中。酒既和止,萬福攸同。



 亞獻,奏《咸寧之曲》:



 黃鐘宮



 六成既闋,三獻雲終。神其醉止,穆穆雍雍。和風慶雲,賁我郊宮。受茲祉福,億載無窮。



 終獻詞同前。



 徹籩豆,奏《豐寧之曲》:



 大呂宮



 禋禮既備,神具宴娭。籩豆有楚,廢徹不遲。多士駿奔,樂且有儀。乃錫純嘏,永佐丕基。



 送神奏:



 圜鐘宮



 殷祀既畢,靈馭載旋。禮洽和應,降福自天。動植咸若,陰陽不愆。明明天子,億萬斯年。



 望燎奏:



 黃鐘宮



 享申百禮,慶洽百靈。奠玉高壇,燔柴廣庭。祥光達曙,燦若景星。神之降福,萬國咸寧。



 大德九年以後,定擬親祀樂章:



 皇帝入中壝:



 黃鐘宮



 赫赫有臨,洋洋在上。克配皇祖,於穆來饗。肇此大禋,乾文弘朗。被袞圜丘,巍巍玄象。



 皇帝盥洗:



 黃鐘宮



 翼翼孝思,明德洽禮。功格玄穹,有光帝始。著我精誠,潔茲薦洗。幣玉攸奠,永集嘉祉。



 皇帝升壇:降同。



 大呂宮



 天行惟健,盛德御天。日月龍章,筍虡宮縣。槁鞂尚明,禮璧蒼圜。神之格思,香升燔煙。



 降神,奏《天成之曲》:



 圜鐘宮三成



 烝哉皇元,丕承帝眷。報本貴誠,於郊殷薦。槁鞂載陳,雲門六變。神之格思,來處來燕。



 黃鐘角一成



 大簇徵一成



 姑洗羽一成詞並同前。



 初獻盥洗,奏《隆成之曲》:



 黃鐘宮



 肇禋南郊,百神受職。齊潔惟先,匪馨於稷。乃沃乃盥,祠壇是陟。上帝監觀,其儀不忒。



 初獻升壇,降同。奏《隆成之曲》:



 大呂宮



 於穆圜壇,陽郊奠位。孔惠孔時,吉蠲為饎。降登祗若,百禮既至。願言居歆,允集熙事。



 奠玉幣,正配位同。奏《欽成之曲》:



 黃鐘宮



 謂天蓋高,至誠則格。克祀克禋,駿奔百闢。制幣斯陳,植以蒼璧。神其降康,俾我來益。



 司徒捧俎,奏《寧成之曲》:



 黃鐘宮



 我牲既潔,我俎斯實。笙鏞克諧,籩豆有飶。神來宴娭,歆茲明德。永錫繁禧,如幾如式。



 昊天上帝位酌獻,奏《明成之曲》:



 黃鐘宮



 於昭昊天,臨下有赫。陶匏薦誠,聲聞在德。酌言獻之,上靈是格。降福孔偕,時萬時億。



 皇地祗位酌獻:



 大呂宮



 至哉坤元,與天同德。函育群生,玄功莫測。合饗圜壇,舊典時式。申錫無疆,聿寧皇國。



 太祖位酌獻:



 黃鐘宮



 禮大報本,郊定天位。皇皇神祖,反始克配。至德難名,玄功宏濟。帝典式敷,率育攸塈。



 皇帝飲福:



 大呂宮



 特牲享誠,備物循質。上帝居歆,百神受職。皇武昭宣,孝祀芬苾。萬福攸同,下民陰騭。



 皇帝出入小次:



 黃鐘宮



 惟天為大,惟帝饗帝。以配祖考,肅贊靈祉。定極崇功,永我昭事。升中於天,象物畢至。



 文舞退,武舞進,奏《和成之曲》:



 黃鐘宮



 羽籥既竣,載揚玉戚。一弛一張,匪舒匪棘。八音克諧,萬舞有奕。永觀厥成,純嘏是錫。



 亞終獻,奏《和成之曲》:



 黃鐘宮



 有嚴郊禋,恭陳幣玉。大糦是承,載祗載肅。上帝居歆,馨香既飫。惠我無疆,介以景福。



 徹籩豆,奏《寧成之曲》:



 大呂宮



 三獻攸終,六樂斯遍。既右享之,徹其有踐。洋洋在上,默默靈眷。明禋告成,於皇錫羨。



 送神,奏《天成之曲》:



 圜鐘宮



 神之來歆,如在左右。神保聿歸,靈斿先後。恢恢上圓,無聲無臭。日監孔昭,思皇多祐。



 望燎,奏《隆成之曲》:



 黃鐘宮



 熙事備成,禮文鬱鬱。紫煙聿升,靈光下燭。神人樂康,永膺戩穀。祚我丕平,景命有僕。



 皇帝出中壝:



 黃鐘宮



 泰壇承光,寥廓玄曖。暢我揚明,饗儀惟大。九服敬宣,聲教無外。皇拜天祐,照臨斯屆。



 宗廟樂章



 世祖中統四年至至元三年,七室樂章:《太常集禮槁》云,此系卷牘所載。



 太祖第一室:



 天垂靈顧,地獻中方。帝力所拓,神武莫當。陽溪昧谷,咸服要荒。昭孝明禋,神祖皇皇。



 太宗第二室:



 和林勝域,天邑地宮。四方賓貢,南北來同。百司分置,胄教肇崇。潤色祖業,德仰神宗。



 睿宗第三室:



 珍符默授,疇昔自天。爰生聖武,寶祚開先。霓旌回狩,龍駕游仙。追遠如生,皇慕顒然。



 皇伯考術赤第四室:



 威武鷹揚,塚位克當。從龍遠拓,千萬里疆。誕總虎旅,駐壓西方。航海梯山,東西來王。



 皇伯考察合帶第五室:



 雄武軍威,滋多歷年。深謀遠略,協贊惟專。流沙西域,餞日東邊。百國畏服,英聲赫然。



 定宗第六室:



 三朝承休,恭己優游。欽繩祖武,其德聿修。帝ME錫壽,德澤期周。蠲食喜惟薌,祈饗於幽。



 憲宗第七室:



 龍躍潛居,風雲會通。知民病苦,軫念宸衷。夔門之旅,繼志圖功。俎豆敬祭,華儀孔隆。



 至元四年至十七年,八室樂章:《太常集禮》云,周馭所藏《儀注》所錄舞節同。



 迎神,奏《來成之曲》,九成:



 黃鐘宮三成



 齊明盛服,翼翼靈眷。禮備多儀,樂成九變。烝烝孝心,若聞且見。肸蚃端臨,來寧來燕。



 大呂角二成詞同黃鐘。



 大簇徵二成詞同黃鐘。



 應鐘羽二成詞同黃鐘。



 初獻盥洗,奏《肅成之曲》:再詣盥洗同。至大以後,名《順成之曲》,詞律同。



 無射宮



 天德維何,如水之清。維水內耀,配彼天明。以滌以濯,犧象光晶。孝思維則,式薦忱誠。



 初獻升殿,登歌樂奏《肅成之曲》:降同。



 夾鐘宮



 祀事有嚴,太官有侐。陟降靡違,禮容翼翼。籩豆旅陳,鐘磬翕繹。於昭吉蠲,神保是格。



 司徒捧俎,奏《嘉成之曲》:別本所錄親祀樂章詞同。



 無射宮



 色純體全,三犧五牲。鸞刀屢奏,毛炰唆羹。神具厭飫,聽我磬聲。居歆有永,胡考之寧。



 烈祖第一室,奏《開成之曲》:



 無射宮



 於皇烈祖,積厚流長。大勛未集,燮伐用張。篤生聖嗣,奄有多方。錫我景福,萬世無疆。



 太祖第二室,奏《武成之曲》:



 無射宮



 天扶昌運,混一中華。爰有真人,奮起龍沙。祭天開宇,亙海為家。肇修禋祀,萬世無涯。



 太宗第三室,奏《文成之曲》:



 無射宮



 纂成前烈,底定丕圖。禮文簡省,禁網寬疏。還風太古,躋世華胥。三靈順協,四海無虞。



 皇伯考術赤第四室,奏《弼成之曲》:



 無射宮



 神支挺秀,右壤疏封。創業艱難,相我祖宗。敘親伊邇,論功亦崇。春秋祭祀,萬世攸同。



 皇伯考察合帶第五室,奏《協成之曲》:



 無射宮



 玉牒期親,神支懿屬。論德疏封,展親分玉。相我祖宗,風櫛雨沐。昔同其勞,今共茲福。



 睿宗第六室,奏《明成之曲》:



 無射宮



 神祖創業,爰著戎衣。聖考撫軍,代行天威。河南底定,江北來歸。貽謀翼子,奕葉重輝。



 定宗第七室,奏《熙成之曲》:



 無射宮



 嗣承丕祚,累洽重熙。堂構既定,垂拱無為。邊庭閑暇,田里安綏。歆茲禋祀,萬世攸宜。



 憲宗第八室,奏《威成之曲》:



 無射宮



 羲馭未出,螢爝騰光。大明麗天,群陰披攘。百神受職,四海寧康。愔愔靈韶,德音不忘。



 文舞退,武舞進,奏《和成之曲》:別本所錄親祀樂章詞同。



 無射宮



 天生五材,孰能去兵。恢張弘業,我祖天聲。干戈曲盤,濯濯厥靈。於赫七德,展也大成。



 亞獻行禮,奏《順成之曲》:終獻詞律同。



 無射宮



 幽通神明,所重精禋。清宮肅肅,百禮具陳。九韶克諧,八佾兟兟。靈光昭答,天休日申。



 徹籩豆,登歌樂奏《豐成之曲》:



 夾鐘宮



 豆籩苾芬,金石鏘鏗。禮終三獻,樂奏九成。有嚴執事,進徹無聲。神保聿歸,萬福來寧。



 送神,奏《來成之曲》:或作《保成》。



 黃鐘宮



 神主在室,神靈在天。禮成樂閟,神返幽玄。降福冥冥,百順無愆。於皇孝思,於萬斯年。



 至元十八年冬十月,世祖皇后祔廟酌獻樂章:《太常集禮》云,卷牘所載。



 黃鐘宮



 徽柔懿哲,溫默靖恭。範儀宮閫,任姒同風。敷天寧謐,內助多功。淑德祔廟,萬世昌隆。



 親祀禘祫樂章:未詳年月。《太常集禮》云,別本所錄。以時考之,疑至元三年以前擬用,詳見《制樂始末》。



 皇帝入門,宮縣奏《順成之曲》:



 無射宮



 熙熙雍雍,六合大同。維皇有造,典禮會通。金奏王夏,祗款神宮。感格如響,嘉氣來叢。



 皇帝升殿,奏《順成之曲》:



 夾鐘宮



 皇明燭幽,沿時制作。宗廟之威,降登時若。趨以採茨,聲容有恪。曰藝曰文,監茲衎樂。



 皇帝詣罍洗,宮縣奏《順成之曲》:《太常集禮》云,至元四年用此曲,名曰《肅成》。至大以後用此,詞律同。



 無射宮



 酌彼行潦,維挹其清。潔齊以祀,祀事昭明。肅肅闢公,沃盥乃升。神之至止,歆於克誠。



 皇帝詣酌尊所,宮縣奏《順成之曲》:



 無射宮



 靈庭愔愔,乃神攸依。文為在禮,載斟匪祈。皇皇穆穆,玉佩聲希。列侯百闢,濟濟宣威。



 迎神,宮縣奏《思成之曲》。至元四年,名《來成之曲》,詞律同。



 司徒捧俎,宮縣奏《嘉成之曲》。至元四年,詞律同。



 酌獻始祖,宮縣奏《慶成之曲》:



 無射宮



 啟運流光,幅員既長。敬恭祀事,鬱鬯芬薌。德以舞象,功以歌揚。式歌且舞,神享是皇。



 諸廟奏《熙成》、《昌成》、《鴻成》、《樂成》、《康成》、《明成》等曲。詞闕。



 文舞退,武舞進,宮縣奏《肅成之曲》。至元四年,名《和成之曲》,詞律同。



 亞終獻,宮縣奏《肅成之曲》。至元四年,名《順成之曲》,詞律同。



 皇帝飲福,登歌奏《成之曲》:



 夾鐘宮



 誠通恩降,靈慈昭宣。左右明命,六合大全。啐飲椒馨,純嘏如川。皇人壽谷,億萬斯年。



 徹豆,登歌奏《豐成之曲》:



 夾鐘宮



 三獻九成,禮畢樂闋。於豆於登,於焉靖徹。多士密勿,樂且有儀。能事脫穎,孔惠孔時。



 送神,奏《保成之曲》:



 黃鐘宮



 雲車之來,不疾而速。風馭言還,闃其恍惚。神心之欣,孝孫之祿。燕翼無疆,景命有僕。



 武宗至大以後,親祀攝樂章:《太常集禮》云,孔思逮本所錄。



 皇帝入門,奏《順成之曲》。別本,親祀禘祫樂章,詞律同。



 皇帝盥洗,奏《順成之曲》。至元四年,名《肅成之曲》,詞律同。



 皇帝升殿,登歌樂奏《順成之曲》。別本,親祀樂章,詞律同。



 皇帝出入小次,奏《昌寧之曲》:《太常集禮》云,此金曲,思逮取之。詳見《制樂始末》。



 無射宮



 於皇神宮,象天清明。肅肅來止,相維公卿。威儀孔彰,君子攸寧。神之休之,綏我思成。



 迎神,奏《思成之曲》:至元四年,名《來成之曲》,詞律同。



 黃鐘宮三成



 齊明盛服,翼翼靈眷。禮備多儀,樂成九變。烝烝孝心,若聞且見。肸蚃端臨,來寧來燕。



 大呂角二成



 太簇征二成



 應鐘羽二成詞並同上。



 初獻盥洗,奏《肅成之曲》。別本,親祀樂章,名《順成之曲》,詞律同。



 初獻升殿,降同。登歌樂奏《肅寧之曲》。至元四年,名《肅成之曲》,詞律同。



 司徒捧俎,奏《嘉成之曲》。至元四年,曲名詞律同。



 太祖第一室,奏《開成之曲》。至元四年,名《武成之曲》,詞同。



 睿宗第二室,奏《武成之曲》。至元四年,名《明成之曲》,詞同。



 世祖第三室,奏《混成之曲》:



 無射宮



 於昭皇祖,體健乘乾。龍飛應運,盛德光前。神功耆定,澤被垓埏。詒厥孫謀,何千萬年。



 裕宗第四室,奏《昭成之曲》:



 無射宮



 天啟深仁,須世而昌。追惟顯考,敢後光揚。徽儀肇舉,禮備音鏘。皇靈鑒止,降無疆。



 順宗第六室,奏《慶成之曲》:



 無射宮



 龍潛於淵,德昭於天。承休基命,光被紘埏。洋洋如臨,籩豆牲牷。惟明惟馨,皇祚綿延。



 成宗第七室,奏《守成之曲》:



 無射宮



 天開神聖,繼世清寧,澤深仁溥,樂協《韶英》。宗枝嘉會,氣和惟馨。繁禧來格,永被皇靈。



 武宗第八室,奏《威成之曲》:



 無射宮



 紹天鴻業,繼世隆平。惠孚中國,威靖邊庭。厥功惟茂,清廟妥靈。歆茲明祀,福祿來成。



 仁宗第九室,奏《歆成之曲》:



 無射宮



 紹隆前緒,運啟文明。深仁及物,至孝躬行。惟皇建極,盛德難名。居歆萬祀,福祿崇成。



 英宗第十室,奏《獻成之曲》:



 無射宮



 神聖繼作,式是憲章。誕興禮樂,躬事烝嘗。翼翼清廟,燁有耿光。於千萬年,世仰明良。



 皇帝飲福,登歌樂奏《厘成之曲》:



 夾鐘宮



 穆穆天子,禋祀太宮。禮成樂備,敬徹誠通。神胥樂止,錫之醇醲。天子萬世,福祿無窮。



 文舞退,武舞進,奏《肅成孔本作《肅寧》。之曲》。至元四年,名《和成之曲》,詞律同。



 亞終獻行禮,宮縣奏《肅成之曲》。至元四年,名《順成之曲》,詞律同。



 徹籩豆,登歌樂奏《豐寧之曲》。至元四年,名《豐成之曲》,詞律同。



 送神,奏《保成之曲》。至元四年,名《來成之曲》,詞律同。



 皇帝出廟廷,奏《昌寧之曲》:



 無射宮



 緝熙維清,吉蠲致誠。上儀具舉,明德薦馨。已事而竣,歡通三靈。先祖是皇,來燕來寧。



 文宗天歷三年,明宗祔廟酌獻,奏《永成之曲》:



 無射宮



 猗那皇明,世纘神武。敬天弗違,時潛時旅。龍旗在塗,言受率土。不遐有臨,永錫多嘏。



 社稷樂章



 降神,奏《鎮寧之曲》:



 林鐘宮二成



 以社以方,國有彞典。大哉元德,基祚綿遠。農功萬世,於焉報本。顯相默祐,降監壇墠。



 太簇角二成



 錫民地利,厥功甚溥。昭代典禮,清聲律呂。穀旦於差,洋洋來下。相此有年,根本日固。



 姑洗徵二成



 平厥水土,百穀用成。長扶景運,宜歆德馨。五祀為大,千古舉行。感通肸蚃,登歌鎮寧。



 南宮羽二成



 幣齊虔修,粢盛告備。倉庾坻京,繄誰之賜。崇壇致恭,幽光孔邇。享於精誠,休祥畢至。



 初獻盥洗,奏《肅寧之曲》:



 太簇宮



 禮備樂陳,辰良日吉。挹彼樽罍,馨哉黍稷。濯溉揭虔,維巾及LV。萬年嚴祀,蹌蹌受職。



 初獻升壇,奏《肅寧之曲》:降同。



 應鐘宮



 春祈秋報,古今彞章。民天是資,神靈用彰。功崇禮嚴,人阜時康。雍雍為儀,燔芬苾香。



 正配位奠玉幣,奏《億寧之曲》:



 太簇宮



 地祇向德,稽古美報。幣帛斯陳,圭璋式繅。載烈載燔,肴羞致告。雨暘時若,丕圖永保。



 司徒捧俎,奏《豐寧之曲》:



 太簇宮



 我稼既同,群黎遍德。我祀如何,牲牷孔碩。有翼有嚴,隨方布色。報功求福,其儀不忒。



 正位酌獻,奏《保寧之曲》:



 太簇宮



 異世同德,於皇聖昭。降茲嘉祥,衛我大寶。生乃烝民,侔德覆燾。厥作祼將,有相之道。



 配位酌獻,奏《保寧之曲》:



 太簇宮



 以御田祖,皇家秩祀。有民人焉,盍究本始。惟敘惟修,誰實介止。酒旨且多,盛德宜配。



 亞終獻,奏《咸寧之曲》:



 太簇宮



 以引以翼,來處來燕。豆籩牲牢,有楚有踐。庸答神休,神亦錫羨。土穀是依,成此酬獻。



 徹豆,奏《豐寧之曲》:



 應鐘宮



 文治修明,相成田功。功為特殊,儀為特隆。終如其初,誠則能通。明神毋忘,時和歲豐。



 送神,奏《鎮寧之曲》:



 林鐘宮



 不屋受陽,國所崇敬。以興來歲,苞秀堅穎。雲軿莫駐,神其諦聽。景命有僕,與國同永。



 望瘞位,奏《肅寧之曲》:



 太簇宮



 雅奏肅寧,繁厘降格。篚厥玄黃,丹誠烜赫。肇祀以歸,瞻言咫尺。萬年攸介,丕承帝德。



 先農樂章



 降神,奏《鎮寧之曲》:



 林鐘宮二成



 民生斯世,食為之天。恭惟大聖,盡心於田。仲春劭農,明祀吉蠲。馨香感神,用祈豐年。



 太簇角二成



 耕種務農,振古如茲。爰粒烝庶,功德茂垂。降嘉奏艱,國家攸宜。所依惟神,庸潔明粢。



 姑洗徵二成



 俶載平疇,農功肇敏。千耦耕耘,同徂隰畛。田祖丕靈,為仁至盡。豐歲穰穰,延洪有引。



 南呂羽二成



 群黎力耕,及茲方春。維時東作,篤我農人。我黍既華,我稷宜新。由天降康,永賴明神。



 初獻盥洗,奏《肅寧之曲》:



 太簇宮



 泂酌行潦,真足為薦。奉茲潔清,神在乎前。分作甘霖,沾溉芳甸。慎於其初,誠意攸見。



 初獻升壇,奏《肅寧之曲》:



 應鐘宮



 有椒其馨,維多且旨。式慎爾儀,降登庭止。黍稷稻粱,民無渴饑。神嗜飲食,永綏嘉祉。



 正配位奠玉幣,奏《億寧之曲》:



 太簇宮



 奉幣維恭,前陳嘉玉。聿昭盛儀,肅雝純如。南畝深耕,麻麥禾菽。用祈三登,膺受多福。



 司徒捧俎,奏《豐寧之曲》:



 太簇宮



 奉牲孔嘉,登俎豐備。地官駿奔,趨進光輝。肥碩蕃孳,歆此誠意。有年斯今,均被神賜。



 正位酌獻,奏《保寧之曲》:



 太簇宮



 寶壇巍煌,神應如響。備腯咸有,牲體苾芳。洋洋如在,降格來享。秉誠罔怠,群生瞻仰。



 配位酌獻,奏《保寧之曲》:



 太簇宮



 酒清斯香,牲碩斯大。具列觴俎,精意先會。民命維食,稗莠毋害。我倉萬億,神明攸介。



 亞終獻,奏《咸寧之曲》:



 太簇宮



 至誠攸感,蕣蚃潛通。百穀嘉種,爰降時豐。祈年孔夙,稼穡為重。俯歆醴齊,載揚歌頌。



 徹豆,奏《豐寧之曲》:



 應鐘宮



 有來雍雍,存誠敢匱。廢徹不遲,靈神攸嗜。孔惠孔時,三農是宜。眉壽萬歲,穀成丕乂。



 送神,奏《鎮寧之曲》:



 林鐘宮


\gezhu{
  君火}
 蒿淒愴,萬靈來唉。靈神具醉,聿言旋歸。歲豐時和,風雨應期。皇圖萬年,永膺洪禧。



 望瘞位,奏《肅寧之曲》:



 太簇宮



 禮成文備,歆受清祀。加牲兼幣,陳玉如儀。靈馭言旋,面陰昭瘞。集茲嘉祥,常致豐歲。



 宣聖樂章



 迎神,奏《凝安之曲》:



 黃鐘宮三成



 大哉宣聖,道尊德崇。維持王化,斯文是宗。典祀有常,精純並隆。神其來格,於昭盛容。



 大呂角二成



 生而知之,有教無私。成均之祀,威儀孔時。惟茲初丁,潔我盛粢。永言其道,萬世之師。



 太簇徵二成



 巍巍堂堂,其道如天。清明之象,應物而然。時維上丁,備物薦誠。維新禮典,樂諧中聲。



 應鐘羽二成



 聖王生知,闡乃儒規。《詩》《書》文教,萬世昭垂。良日惟丁,靈承丕爽。揭此精虔,神其來享。



 初獻盥洗,奏《同安之曲》:



 姑洗宮



 右文興化,憲古師經。明祀有典,吉日惟丁。豐犧在俎,雅奏在庭。周回陟降,福祉是膺。



 初獻升殿,奏《同安之曲》:降同。



 南呂宮



 誕興斯文,經天緯地。功加於民,實千萬世。笙鏞和鳴,粢盛豐備。肅肅降登,歆茲秩祀。



 奠幣,奏《明安之曲》:



 南呂宮



 自生民來,誰底其盛。惟王神明,度越前聖。粢幣具成,禮容斯稱。黍稷惟馨,惟神之聽。



 捧俎,奏《豐安之曲》:



 姑洗宮



 道同乎天,人倫之至。有享無窮,其興萬世。既潔斯牲,粢明醑旨。不懈以忱,神之來塈。



 大成至聖文宣王位酌獻,奏《成安之曲》:



 南呂宮



 大哉聖王,實天生德。作樂以崇,時祀無斁。清酤惟馨,嘉牲孔碩。薦羞神明,庶幾昭格。



 兗國復聖公位酌獻,奏《成安之曲》:



 南呂宮



 庶幾屢空,淵源深矣。亞聖宣猷,百世宜祀。吉蠲斯辰,昭陳尊簋。旨酒欣欣,神其來止。



 郕國宗聖公酌獻,奏《成安之曲》:



 南呂宮



 心傳忠恕,一以貫之。爰述《大學》,萬工訓彞。惠我光明,尊聞行知。繼聖迪後,是享是宜。



 沂國述聖公酌獻,奏《成安之曲》:



 南呂宮



 公傳自曾,孟傳自公。有嫡緒承,允得其宗。提綱開蘊,乃作《中庸》。侑於元聖,億載是崇。



 鄒國亞聖公酌獻,奏《成安之曲》:



 南呂宮



 道之由興,於皇宣聖。維公之傳,人知趨正。與饗在堂,情文斯稱。萬年承休,假哉天命。



 亞獻,奏《文安之曲》:終獻同。



 姑洗宮



 百王宗師,生民物軌。瞻之洋洋,神其寧止。酌彼金罍,惟清且旨。登獻惟三,於嘻成禮。



 飲福受胙。與盥洗同,惟國學釋奠親祀用之,攝事則不用,外路州縣並皆用之。



 徹豆,奏《娛安之曲》:



 南呂宮



 犧象在前,豆籩在列。以享以薦,既芬既潔。禮成樂備,人和神悅。祭則受福,率尊無越。



 送神,奏《凝安之曲》:



 黃鐘宮



 有嚴學宮,四方來崇。恪恭祀事,威儀雍雍。歆茲惟馨,飆馭回復。明禋斯畢,咸膺百福。



 望瘞。與盥洗同。



 右釋奠樂章,皆舊曲。元朝嘗擬撰易,而未及用,今並附於此。



 迎神,奏《文明之曲》:



 天縱之聖,集厥大成。立言垂教,萬世準程。廟庭孔碩,尊俎既盈。神之格思,景福來並。



 盥洗,奏《昭明之曲》:



 神既寧止,有孚顒若。罍洗在庭,載盥載濯。匪惟潔修,亦新厥德。對越在茲,敬恭惟則。



 升殿,奏《景明之曲》:降同。



 大哉聖功,薄海內外。禮隆秩宗,光垂昭代。陟降在庭,攝齊委佩。莫不肅雝,洋洋如在。



 奠幣,奏《德明之曲》:



 圭袞尊崇,佩紳列侑。籩豆有楚,業具和奏。式陳量幣,駿奔左右。天睠斯文,繄神之佑。



 文宣王酌獻,奏《誠明之曲》:



 惟聖監格,享於克誠。有樂在縣,有碩斯牲。奉醴以告,嘉薦惟馨。綏以多福,永底隆平。



 兗國公酌獻,奏《誠明之曲》:



 潛心好學,不違如愚。用舍行藏,乃與聖俱。千載景行,企厥步趨。廟食作配,祀典弗渝。



 成國公酌獻。闕。



 沂國公酌獻。闕。



 鄒國公酌獻,奏《誠明之曲》:



 洙泗之傳,學窮性命。力距楊墨,以承三聖。遭時之季,孰識其正。高風仰止,莫不肅敬。



 亞獻,奏《靈明之曲》:終獻同。



 廟成奕奕,祭祀孔時。三爵具舉,是饗是宜。於昭聖訓,示我民彞。紀德報功,配於兩儀。



 送神,奏《慶明之曲》:



 禮成樂備,靈馭其旋。濟濟多士,不懈益虔。文教茲首,儒風是宣。佑我闕。



\end{pinyinscope}