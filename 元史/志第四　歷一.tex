\article{志第四 歷一}

\begin{pinyinscope}

 夫明時治歷,自黃帝、堯、舜與三代之盛王,莫不重之,其文備見於傳記矣。雖去古既遠,其法不詳作者的文章、著作所表現出來的一個共同傾向。主要派別有,然原其要,不過隨時考驗,以合於天而已。漢劉歆作《三統歷》,始立積年日法,以為推步之準。後世因之,歷唐而宋,其更元改法者,凡數十家,豈故相為乖異哉?蓋天有不齊之運,而歷為一定之法,所以既久而不能不差,既差則不可不改也。



 元初承用金《大明歷》,庚辰歲,太祖西征,五月望,月蝕不效;二月、五月朔,微月見於西南。中書令耶律楚材以《大明歷》後天,乃損節氣之分,減周天之秒,去交終之率,治月轉之餘,課兩曜之後先,調五行之出沒,以正《大明歷》之失。且以中元庚午歲,國兵南伐,而天下略定,推上元庚午歲天正十一月壬戌朔,子正冬至,日月合璧,五星聯珠,同會虛宿六度,以應太祖受命之符。又以西域、中原地里殊遠,創為裏差以增損之,雖東西萬里,不復差忒。遂題其名曰《西征庚午元歷》,表上之,然不果頒用。



 至元四年,西域札馬魯丁撰進《萬年歷》,世祖稍頒行之。十三年,平宋,遂詔前中書左丞許衡、太子贊善王恂、都水少監郭守敬改治新歷。衡等以為金雖改歷,止以宋《紀元歷》微加增益,實未嘗測驗於天,乃與南北日官陳鼎臣、鄧元麟、毛鵬翼、劉巨淵、王素、岳鉉、高敬等參考累代歷法,復測候日月星辰消息運行之變,參別同異,酌取中數,以為歷本。十七年冬至,歷成,詔賜名曰《授時歷》。十八年,頒行天下。二十年,詔太子諭德李謙為《歷議》,發明新歷順天求合之微,考證前代人為附會之失,誠可以貽之永久,自古及今,其推驗之精,蓋未有出於此者也。今衡、恂、守敬等所撰《歷經》及謙《歷議》故存,皆可考據,是用具著於篇。惟《萬年歷》不復傳,而《庚午元歷》雖未嘗頒用,其為書猶在,因附著於後,使來者有考焉。作《歷志》。



 授時歷議上



 驗氣



 天道運行,如環無端,治歷者必就陰消陽息之際,以為立法之始。陰陽消息之機,何從而見之?惟候其日晷進退,則其機將無所遁。候之之法,不過植表測景,以究其氣至之始。智作能述,前代諸人為法略備,茍能精思密索,心與理會,則前人述作之外,未必無所增益。



 舊法擇地平衍,設水準繩墨,植表其中,以度其中晷。然表短促,尺寸之下所為分秒太、半、少之數,未易分別。表長,則分寸稍長,所不便者,景虛而淡,難得實景。前人欲就虛景之中考求真實,或設望筒,或置小表,或以木為規,皆取表端日光下徹圭面。今以銅為表,高三十六尺,端挾以二龍,舉一橫梁,下至圭面,共四十尺,是為八尺之表五。圭表刻為尺寸,舊寸一,今申而為五,厘毫差易分。別創為景符,以取實景。其制以銅葉,博二寸,長加博之二,中穿一竅,若針芥然,以方跂為趺,一端設為機軸,令可開闔,耆其一端,使其勢斜倚,北高南下,往來遷就於虛景之中,竅達日光,僅如米許,隱然見橫梁於其中。舊法以表端測晷,所得者日體上邊之景,今以橫梁取之,實得中景,不容有毫末之差。



 地中八尺表景,冬至長一丈三尺有奇,夏至尺有五寸。今京師長表,冬至之景七丈九尺八寸有奇,在八尺表則一丈五尺九寸六分;夏至之景一丈一尺七寸有奇,在八尺表則二尺三寸四分。雖晷景長短所在不同,而其景長為冬至,景短為夏至,則一也。惟是氣至時刻考求不易,蓋至日氣正,則一歲氣節從而正矣。劉宋祖沖之嘗取至前後二十三四日間晷景,折取其中,定為冬至,且以日差比課,推定時刻。宋皇祐間,周琮則取立冬、立春二日之景,以為去至既遠,日差頗多,易為推考。《紀元》以後諸歷,為法加詳,大抵不出沖之之法。新歷積日累月,實測中晷,自遠日以及近日,取前後日率相埒者,參考同異,初非偏取一二日之景,以取數多者為定,實減《大明歷》一十九刻二十分。仍以累歲實測中晷日差分寸,定擬二至時刻於後。



 推至元十四年丁丑歲冬至



 其年十一月十四日己亥,景長七丈九尺四寸八分五厘五毫;至二十一日丙午,景長七丈九尺五寸四分一厘;二十二日丁未,景長七丈九尺四寸五分五厘。以己亥、丁未二日之景相校,餘三分五毫為晷差,進二位;以丙午、丁未二日之景相校,餘八分六厘為法;除之,得三十五刻;用減相距日八百刻,餘七百六十五刻;折取其中,加半日刻,共為四百三十二刻半;百約為日,得四日;餘以十二乘之,百約為時,得三時,滿五十又作一時,共得四時;餘以十二收之,得三刻;命初起距日己亥算外,得癸卯日辰初三刻為丁丑歲冬至。此取至前後四日景。



 十一月初九日甲午,景七丈八尺六寸三分五厘五毫;至二十六日辛亥,景七丈八尺七寸九分三厘五毫;二十七日壬子,景七丈八尺五寸五分。以甲午、壬子景相減,復以辛亥、壬子景相減,準前法求之,亦得癸卯日辰初三刻。至二十八日癸丑,景七丈八尺三寸四厘五毫,用壬子、癸丑二日之景與甲午景,準前法求之,亦合。此取至前後八九日景。



 十一月丙戌朔,景七丈五尺九寸八分六厘五毫;二日丁亥,景七丈六尺三寸七分七厘;至十二月初六日庚申,景七丈五尺八寸五分一厘。準前法求之,亦在辰初三刻。此取至前後一十七日景。



 十月二十一日丙子,景七丈九寸七分一厘;至十二月十六日庚午,景七丈七寸六分;十七日辛未,景七丈一寸五分六厘五毫。準前法求之,亦得辰初三刻。此取至前後二十七日景。



 六月初五日癸亥,景一丈三尺八分;距十五年五月癸未朔,景一丈三尺三分八厘五毫;初二日甲申,景一丈二尺九寸二分五毫。準前法求之,亦合。此取至前後一百六十日景。



 推十五年戊寅歲夏至



 五月十九日辛丑,景一丈一尺七分七厘五毫;距二十八日庚戌,景一丈一尺七寸八分;二十九日辛亥,景一丈一尺八寸五厘五毫。用辛丑、庚戌二日之景相減,餘二厘五毫,進二位為實;復用庚戌、辛亥景相減,餘二分五厘五毫為法;除之,得九刻,用減相距日九百刻,餘八百九十一刻;半之,加半日刻,百約,得四日;餘以十二乘之,百約,得十一時;餘以十二收為刻,得三刻;命初起距日辛丑算外,得乙巳日亥正三刻夏至。此取至前後四日景。



 十四年十二月十五日己巳,景七丈一尺三寸四分三厘;距十五年十一月初二日辛巳,景七丈七寸五分九厘五毫;初三日壬午,景七丈一尺四寸六厘。用己巳、壬午景相減,以辛巳、壬午景相減除之,亦合。此用至前後一百五十六日景。



 十四年十二月十二日丙寅,景七丈二尺九寸七分二厘五毫;十三日丁卯,景七丈二尺四寸五分四厘五毫;十四日戊辰,景七丈一尺九寸九厘;距十五年十一月初四日癸未,景七丈一尺九寸五分七厘五毫;初五日甲申,景七丈二尺五寸五厘;初六日乙酉,景七丈三尺三分三厘五毫。前後互取,所得時刻皆合。此取至前後一百五十八九日景。



 十四年十二月初七日辛酉,景七丈五尺四寸一分七厘;初八日壬戌,景七丈四尺九寸五分九厘五毫;初九日癸亥,景七丈四尺四寸八分六厘;距十五年十一月初九日戊子,景七丈四尺五寸二分五毫;初十日己丑,景七丈五尺三厘五毫;十一日庚寅,景七丈五尺四寸四分九厘五毫。以壬戌、己丑景相減為實,以辛酉、壬戌景相減為法,除之;或以壬戌、癸亥景相減,或以戊子、己丑景相減,若己丑、庚寅景相減,推前法求之,皆合。此取至前後一百六十三四日景。



 推十五年戊寅歲冬至



 其年十一月十九日戊戌,景七丈八尺三寸一分八厘五毫;距閏十一月初九日戊午,景七丈八尺三寸六分三厘五毫;初十日己未,景七丈八尺八分二厘五毫。用戊戌、戊午二日景相減,餘四分五厘為晷差,進二位,以戊午、己未景相減,餘二寸八分一厘為法,除之,得一十六刻,加相距日二千刻,半之,加半日刻,百約,得十日;餘以十二乘之,百約為時,滿五十又進一時,共得七時;餘以十二收為刻;命初起距日己亥算外,得戊申日未初三刻為戊寅歲冬至。此取至前後十日景。



 十一月十二日辛卯,景七丈五尺八寸八分一厘五毫;十三日壬辰,景七丈六尺三寸一厘五毫;閏十一月十五日甲子,景七丈六尺三寸六分六厘五毫;十六日乙丑,景七丈五尺九寸五分三厘;十七日丙寅,景七丈五尺五寸四厘五毫。用壬辰、甲子景相減為實,以辛卯、壬辰景相減為法,除之,亦得戊申日未初三刻。或用甲子、乙丑景相減,推之,亦合。若用辛卯、乙丑景相減為實,用乙丑、丙寅景相減,除之,並同。此取至前後十六七日景。



 十一月初八日丁亥,景七丈四尺三分七厘五毫;閏十一月二十日己巳,景七丈四尺一寸二分;二十一日庚午,景七丈三尺六寸一分四厘五毫。用丁亥、己巳景相減為實,以己巳、庚午景相減,除之,亦同。此取至前後二十一日景。



 六月二十六日戊寅,景一丈四尺四寸五分二厘五毫;二十七日己卯,景一丈四尺六寸三分八厘;至十六年四月二日戊寅,景一丈四尺四寸八分一厘。以二戊寅景相減,用後戊寅、己卯景相減,推之,亦同。此取至前後一百五十日景。



 五月二十八日庚戌,景一丈一尺七寸八分;至十六年四月二十九日乙巳,景一丈一尺八寸六分三厘;三十日丙午,景一丈一尺七寸八分三厘。用庚戌、丙午景相減,以乙巳、丙午景相減,推之,亦同。此取至前後百七十八日景。



 推十六年己卯歲夏至



 四月十九日乙未,景一丈二尺三寸六分九厘五毫;二十日丙申,景一丈二尺二寸九分三厘五毫;至五月十九日乙丑,景一丈二尺二寸六分四厘。以丙申、乙丑景相減,餘二分九厘五毫為晷差,進二位;以乙未、丙申景相減,得七分六厘為法;除之,得三十八刻;加相距日二千九百刻,半之,加半日刻,百約,得十五日;餘以十二乘之,百約,得二時;餘以十二收之,得二刻;命初起距日丙申算外,得辛亥日寅正二刻為夏至。此取至前後十五日景。



 三月二十一日戊辰,景一丈六尺三寸九分五毫;六月十六日壬辰,景一丈六尺九分九厘五毫;十七日癸巳,景一丈六尺三寸一分一厘。用戊辰、癸巳景相減,以壬辰、癸巳景相減,準前法推之,亦合。此取至前後四十二日景。



 三月初二日己酉,景二丈一尺三寸五厘;至七月初七日壬子,景二丈一尺一寸九分五厘五毫;初八日癸丑,景二丈一尺四寸八分六厘五毫。用己酉、壬子景相減,以壬子、癸丑景相減,如前法推之,亦合。此取至前後六十一二日景。



 三月戊申朔,景二丈一尺六寸一分一厘;至七月初八日癸丑,景二丈一尺四寸八分六厘五毫;初九日甲寅,景二丈一尺九寸一分五厘五毫。用戊申、癸丑景相減,以癸丑、甲寅景相減,準前法推之,亦同。此取至前後六十二三日景。



 二月十八日乙未,景二丈六尺三分四厘五毫;至七月二十一日丙寅,景二丈五尺八寸九分九厘;二十二日丁卯,景二丈六尺二寸五分九厘。用乙未、丙寅景相減,以丙寅、丁卯景相減,如前法推之,亦同。此取至前後七十五六日景。



 二月三日庚辰,景三丈二尺一寸九分五厘五毫;至八月初五日庚辰,景三丈一尺五寸九分六厘五毫;初六日辛巳,景三丈二尺二分六厘五毫。用前庚辰與辛巳景相減,以後庚辰、辛巳景相減,如前推之,亦同。此取至前後九十日景。



 正月十九日丁卯,景三丈八尺五寸一厘五毫;至八月十八日癸巳,景三丈七尺八寸二分三厘;十九日甲午,景三丈八尺三寸一分五毫。用丁卯、甲午景相減,以癸巳、甲午景相校,如前推之,亦同。此取至前後一百三四日景。



 推十六年己卯歲冬至



 十月二十四日戊戌,景七丈六尺七寸四分;至十一月二十五日己巳,景七丈六尺五寸八分;二十六日庚午,景七丈六尺一寸四分二厘五毫。用戊戌、己巳景相減,餘一寸六分為晷差,進二位;以己巳、庚午景相減,餘四寸三分七厘五毫為法;除之,得三十六刻;以相減距日三千一百刻,餘三千六十四刻;半之,加五十刻,百約,得一十五日;餘以十二乘之,百約為時,滿五十,又進一時,共得十時;餘以十二收之為刻,得二刻;命初起距日戊戌算外,得癸丑日戌初二刻冬至。此取至前後十五六日景。



 十月十八日壬辰,景七丈四尺五分二厘五毫;十九日癸巳,景七丈四尺五寸四分五厘;二十日甲午,景七丈五尺二分五厘;至十一月二十八日壬申,景七丈五尺三寸二分;二十九日癸酉,景七丈四尺八寸五分二厘五毫;十二月甲戌朔,景七丈四尺三寸六分五厘;初二日乙亥,景七丈三尺八寸七分一厘五毫。用甲午、癸酉景相減,癸巳、甲午景相減,如前推之,亦同。若以壬申、癸酉景相減為法,推之亦同。此取至前後十八九日景。



 若用癸巳與甲戌景相減,以壬辰、癸巳景相減,推之,或癸巳、甲午景相減,推之,或用甲戌、癸酉景相減,推之,或甲戌、乙亥景相減,推之,或以壬辰、乙亥景相減,用壬辰、癸巳景相減,推之並同。此取至前後二十日景。



 十月十六日庚寅,景七丈三尺一分五厘;十二月初三日丙子,景七丈三尺三寸二分;初四日丁丑,景七丈二尺八寸四分二厘五毫。用庚寅、丁丑景相減,以丙子、丁丑景相減,推之亦同。此取至前後二十三日景。



 十月十四日戊子,景七丈一尺九寸二分二厘五毫;十五日己丑,景七丈二尺四寸六分九厘;十二月初五日戊寅,景七丈二尺二寸七分二厘五毫。用己丑、戊寅景相減,以戊子、己丑景相減,推之,或用己丑、庚寅相減,推之亦同。此取至前後二十四日景。



 十月初七日辛巳,景六丈七尺七寸四分五厘;初八日壬午,景六丈八尺三寸七分二厘五毫;初九日癸未,景六丈八尺九寸七分七厘五毫;十二月十二日乙酉,景六丈八尺一寸四分五厘。用壬午、乙酉景相減,以辛巳、壬午相減,推之,壬午、癸未景相減,推之亦同。此取至前後三十一二日景。



 十月乙亥朔,景六丈三尺八寸七分;十二月十八日辛卯,景六丈四尺二寸九分七厘五毫;十九日壬辰,景六丈三尺六寸二分五厘。用乙亥、壬辰景相減,以辛卯、壬辰景相減,推之亦同。此取至前後三十八日景。



 九月二十二日丙寅,景五丈七尺八寸二分五厘;十二月二十八日辛丑,景五丈七尺五寸八分;二十九日壬寅,景五丈六尺九寸一分五厘。用丙寅、辛丑景相減,以辛丑、壬寅景相減,推之亦同。此取至前後四十七八日景。



 九月二十日甲子,景五丈六尺四寸九分二厘五毫;至十二月二十九日壬寅,景五丈六尺九寸一分五厘;至十七年正月癸卯朔,景五丈六尺二寸五分。用甲子、癸卯相減,壬寅、癸卯景相減,推之亦同。此取至前後五十日景。



 右以累年推測到冬夏二至時刻為準,定擬至元十八年辛巳歲前冬至,當在己未日夜半後六刻,即丑初一刻。



 歲餘歲差



 周天之度,周歲之日,皆三百六十有五。全策之外,又有奇分,大率皆四分之一。自今歲冬至距來歲冬至,歷三百六十五日,而日行一周,凡四周,歷千四百六十,則餘一日,析而四之,則四分之一也。然天之分常有餘,歲之分常不足,其數有不能齊者,惟其所差至微,前人初未覺知。迨漢末劉洪,始覺冬至後天,謂歲周餘分太強,乃作《乾象歷》,減歲餘分二千五百為二千四百六十二。至晉虞喜,宋何承天、祖沖之,謂歲當有差,因立歲差之法。其法損歲餘,益天周,使歲餘浸弱,天周浸強,強弱相減,因得日躔歲退之差。歲餘、天周,二者實相為用,歲差由斯而立,日躔由斯而得,一或損益失當,詎能與天葉哉?



 今自劉宋大明壬寅以來,凡測景驗氣得冬至時刻真數者有六,取相距積日時刻,以相距之年除之,各得其時所用歲餘。復自大明壬寅距至元戊寅積日時刻,以相距之年除之,得每歲三百六十五日二十四分二十五秒,比《大明歷》減去一十一秒,定為方今所用歲餘。餘七十五秒,用益所謂四分之一,共為三百六十五度二十五分七十五秒,定為天周。餘分強弱相減,餘一分五十秒,用除全度,得六十六年有奇,日卻一度,以六十六年除全度,適得一分五十秒,定為歲差。



 復以《堯典》中星考之,其時冬至日在女、虛之交。及考之前史,漢元和二年,冬至日在斗二十一度;晉太元九年,退在斗十七度;宋元嘉十年,在斗十四度末;梁大同十年,在斗十二度;隋開皇十八年,猶在斗十二度;唐開元十二年,在斗九度半;今退在箕十度。取其距今之年、距今之度較之,多者七十餘年,少者不下五十年,輒差一度。宋慶元間,改《統天歷》,取大衍歲差率八十二年及開元所距之差五十五年,折取其中,得六十七年,為日卻行一度之差。施之今日,質諸天道,實為密近。



 然古今歷法,合於今必不能通於古,密於古必不能驗於今。今《授時歷》,以之考古,則增歲餘而損歲差;以之推來,則增歲差而損歲餘;上推春秋以來冬至,往往皆合;下求方來,可以永久而無弊;非止密於今日而已。仍以《大衍》等六歷,考驗春秋以來冬至疏密,凡四十九事,具列如後。



 ○冬至刻



 《大衍》《宣明》《紀元》《天》《大明》《授時》



 獻公十五年戊寅歲,正月甲寅朔旦冬至。



 丙辰22乙卯88丁巳33乙卯2丁巳35甲寅99



 僖公五年丙寅歲,正月辛亥朔旦冬至。



 辛亥94辛亥66壬子74辛亥27壬子89辛亥14)



 昭公二十年己卯歲,正月己丑朔旦冬至。



 己丑45己丑2庚寅25戊子92庚寅29戊子83



 宋元嘉十二年乙亥歲,十一月十五日戊辰景長。



 戊辰35戊辰32戊辰39戊辰51戊辰41戊辰47



 元嘉十三年丙子歲,十一月二十六日甲戌景長。



 癸酉59癸酉57癸酉63癸酉75癸酉65癸酉71



 元嘉十五年戊寅歲,十一月十八日甲申景長。



 甲申8甲申6甲申十2甲申24甲申14甲申19



 元嘉十六年己卯歲,十月二十九日己丑景長。



 己丑33己丑3己丑37己丑48己丑37己丑44



 元嘉十七年庚辰歲,十一月初十日甲午景長。



 甲午57甲午55甲午61甲午72甲午63甲午68



 元嘉十八年辛巳歲,十一月二十一日己亥景長。



 己亥82己亥79己亥85己亥97己亥87己亥93



 元嘉十九年壬午歲,十一月初三日乙巳景長。



 乙巳6乙巳4乙巳十乙巳21乙巳11乙巳17



 大明五年辛丑歲,十一月乙酉冬至。



 甲申7甲申68甲申73甲申89甲申74甲申79



 陳天嘉六年乙酉歲,十一月庚寅景長。



 庚寅12庚寅13庚寅5庚寅24庚寅8庚寅17



 光大二年戊子歲,十一月乙巳景長。



 乙巳8乙巳86乙巳79乙巳97乙巳81乙巳9



 太建四年壬辰歲,十一月二十九日丁卯景長。



 丙寅83丙寅78丙寅77丙寅95丙寅98丙寅87



 太建六年甲午歲,十一月二十日丁丑景長。



 丁丑32丁丑33丁丑25丁丑43丁丑27丁丑36



 太建九年丁酉歲,十一月二十三日壬辰景長。



 癸巳4癸巳6壬辰99癸巳16癸巳空癸巳8



 太建十年戊戌歲,十一月五日戊戌景長。



 戊戌3戊戌3戊戌23戊戌4戊戌24戊戌33



 隋開皇四年甲辰歲,十一月十一日己巳景長。



 己巳77己巳78己巳69己巳86己巳71己巳86



 開皇五年乙巳歲,十一月二十二日乙亥景長。



 乙亥1乙亥2甲戌92乙亥11甲戌55乙亥10



 開皇六年丙午歲,十一月三日庚辰景長。



 庚辰25庚辰26庚辰18庚辰34庚辰19庚辰34



 開皇七年丁未歲,十一月十四日乙酉景長。



 乙酉5乙酉51乙酉42乙酉59乙酉44乙酉59



 開皇十一年辛亥歲,十一月二十八日丙午景長。



 丙午48丙午49丙午43丙午57丙午41丙午56



 開皇十四年甲寅歲,十一月辛酉朔旦冬至。



 壬戌21壬戌22壬戌13壬戌3壬戌14壬戌29



 唐貞觀十八年甲辰歲,十一月乙酉景長。



 甲申43甲申45甲申31甲申5甲申32甲申44



 貞觀二十三年己酉歲,十一月辛亥景長。



 庚戌65庚戌68庚戌53庚戌72庚戌54庚戌66



 龍朔二年壬戌歲,十一月四日己未至戊午景長。



 戊午83戊午86戊午69戊午88戊午71戊午82



 儀鳳元年丙子歲,十一月壬申景長。



 壬申25壬申28壬申10壬申28壬申12壬申22



 永淳元年壬午歲,十一月癸卯景長。



 癸卯72癸卯75癸卯57癸卯76癸卯58癸卯68



 開元十年壬戌歲,十一月癸酉景長。



 癸酉49癸酉54癸酉31癸酉5癸酉32癸酉46



 開元十一年癸亥歲,十一月戊寅景長。



 戊寅74戊寅77戊寅55戊寅74戊寅56戊寅7



 開元十二年甲子歲,十一月癸未冬至。



 癸未98甲申3癸未8癸未99癸未81癸未95



 宋景德四年丁未歲,十一月戊辰日南至。



 戊辰15戊辰26丁卯74丁卯82丁卯74丁卯8



 皇祐二年庚寅歲,十一月三十日癸丑景長。



 癸丑65癸丑79癸丑22癸丑25癸丑22癸丑23



 元豐六年癸亥歲,十一月丙午景長。



 丙午73丙午85丙午26丙午27丙午26丙午26



 元豐七年甲子歲,十一月辛亥景長。



 辛亥97壬子10辛亥5辛亥51辛亥5辛亥51



 元祐三年戊辰歲,十一月壬申景長。



 壬申94癸酉8壬申48壬申48壬申48壬申48



 元祐四年己巳歲,十一月丁丑景長。



 戊寅19戊寅32丁丑72丁丑72丁丑72丁丑72



 元祐五年庚午歲,十一月壬午冬至。



 癸未44癸未56壬午96壬午97壬午96壬午96



 元祐七年壬申歲,十一月癸巳冬至。



 癸巳92甲午5癸巳45癸巳45癸巳45癸巳45



 元符元年戊寅歲,十一月甲子冬至。



 乙丑39乙丑52甲子91甲子91甲子91甲子91



 崇寧三年甲申歲,十一月丙申冬至。



 丙申86丙申99丙申37丙申36丙申37丙申37



 紹熙二年辛亥歲,十一月壬申冬至。



 癸酉12癸酉27壬申57壬申47壬申57壬申46



 慶元三年丁巳歲,十一月癸卯日南至。



 甲辰59甲辰74甲辰3癸卯92甲辰3癸卯92



 嘉泰三年癸亥歲,十一月甲戌日南至。



 丙子5丙子21乙亥49乙亥37乙亥49乙亥37



 嘉定五年壬申歲,十一月壬戌日南至。



 癸亥25癸亥41壬戌69壬戌56壬戌68壬戌56



 紹定三年庚寅歲,十一月丙申日南至。



 丁酉65丁酉83丁酉7丙申63丁酉7丙申92



 淳祐十年庚戌歲,十一月辛巳日南至。



 壬午94壬午71辛巳96辛巳77辛巳94辛巳78



 本朝至元十七年庚辰歲,十一月己未夜半後六刻冬至。



 己未87庚申5己未25己未4己未24己未6



 右自春秋獻公以來,凡二千一百六十餘年,用《大衍》、《宣明》、《紀元》、《統天》、《大明》、《授時》六歷推算冬至,凡四十九事。《大衍歷》合者三十二,不合者十七;《宣明歷》合者二十六,不合者二十三;《紀元歷》合者三十五,不合者十四;《統天歷》合者三十八,不合者十一;《大明歷》合者三十四,不合者十五;《授時歷》合者三十九,不合者十事。



 今按獻公十五年戊寅歲正月甲寅朔旦冬至,《授時歷》得甲寅,《統天歷》得乙卯,後天一日;至僖公五年丙寅歲正月辛亥朔旦冬至,《授時》、《統天》皆得辛亥,與天合;下至昭公二十年己卯歲正月己丑朔旦冬至,《授時》、《統天》皆得戊子,並先一日,若曲變其法以從之,則獻公、僖公皆不合矣。以此知《春秋》所書昭公冬至,乃日度失行之驗。一也。《大衍歷》考古冬至,謂劉宋元嘉十三年丙子歲十一月甲戌日南至,《大衍》與《皇極》、《麟德》三歷皆得癸酉,各先一日,乃日度失行,非三歷之差。今以《授時歷》考之,亦得癸酉。二也。大明五年辛丑歲十一月乙酉冬至,諸歷皆得甲申,殆亦日度之差。三也。陳太建四年壬辰歲十一月丁卯景長,《大衍》、《授時》皆得丙寅,是先一日;太建九年丁酉歲十一月壬辰景長,《大衍》、《授時》皆得癸巳,是後一日;一失之先,一失之後,若合於壬辰,則差於丁酉,合於丁酉,則差於壬辰,亦日度失行之驗。五也。開皇十一年辛亥歲十一月丙午景長,《大衍》、《統天》、《授時》皆得丙午,與天合;至開皇十四年甲寅歲十一月辛酉冬至,而《大衍》、《統天》、《授時》皆得壬戌,若合於辛亥,則失於甲寅,合於甲寅,則失於辛亥,其開皇十四年甲寅歲冬至,亦日度失行。六也。唐貞觀十八年甲辰歲十一月乙酉景長,諸歷得甲申,貞觀二十三年己酉歲十一月辛亥景長,諸歷皆得庚戌,《大衍歷議》以永淳、開元冬至推之,知前二冬至乃史官依時歷以書,必非候景所得,所以不合,今以《授時歷》考之亦然。八也。自前宋以來,測景驗氣者凡十七事,其景德丁未歲戊辰日南至,《統天》、《授時》皆得丁卯,是先一日;嘉泰癸亥歲甲戌日南至,《統天》、《授時》皆得乙亥,是後一日;一失之先,一失之後,若曲變其數以從景德,則其餘十六事多後天,從嘉泰,則其餘十六事多先天,亦日度失行之驗。十也。



 前十事皆《授時歷》所不合,以此理推之,非不合矣,蓋類其同則知其中,辨其異則知其變。今於冬至略其日度失行及史官依時歷書之者凡十事,則《授時歷》三十九事皆中,《統天歷》與今歷不合者僅有獻公一事,《大衍歷》推獻公冬至後天二日,《大明》後天三日,《授時歷》與天合。下推至元庚辰冬至,《大衍》後天八十一刻,《大明》後天一十九刻,《統天歷》先天一刻,《授時歷》與天合。以前代諸歷校之,《授時》為密,庶幾千歲之日至,可坐而致雲。



 古今歷參校疏密



 《授時歷》與古歷相校,疏密自見,蓋上能合於數百載之前,則下可行之永久,此前人定說。古稱善治歷者,若宋何承天,隋劉焯,唐傅仁均、僧一行之流,最為傑出。今以其歷與至元庚辰冬至氣應相校,未有不舛戾者,而以新歷上推往古,無不吻合,則其疏密從可知已。



 宋文帝元嘉十九年壬午歲十一月乙巳日十一刻冬至,距本朝至元十七年庚辰歲,計八百三十八年。其年十一月,氣應己未六刻冬至,《元嘉歷》推之,得辛酉,後《授時》二日,《授時》上考元嘉壬午歲冬至,得乙巳,與元嘉合。



 隋大業三年丁卯歲十一月庚午日五十二刻冬至,距至元十七年庚辰歲,計六百七十三年。《皇極歷》推之,得庚申冬至,後《授時》一日;《授時》上考大業丁卯歲冬至,得庚午,與《皇極》合。



 唐武德元年戊寅歲十一月戊辰日六十四刻冬至,距至元十七年庚辰歲,計六百六十二年。《戊寅歷》推之,得庚申冬至,後《授時》一日;《授時歷》上考武德戊寅歲,得戊辰冬至,與《戊寅歷》合。



 開元十五年丁卯歲十一月己亥日七十二刻冬至,距至元十七年庚辰歲,計五百五十三年。《大衍歷》推之,得己未冬至,後《授時》八十一刻;《授時歷》上考開元丁卯歲,得己亥冬至,與《大衍歷》合,先四刻。



 長慶元年辛丑歲十一月壬子日七十六刻冬至,距至元十七年庚辰歲,計四百五十九年。《宣明歷》推之,得庚申冬至,後《授時》一日;《授時歷》上考長慶辛丑歲,得壬子冬至,與《宣明歷》合。



 宋太平興國五年庚辰歲十一月丙午日六十三刻冬至,距至元十七年庚辰歲,計三百年。《乾元歷》推之,得庚申冬至,後《授時》一日;《授時歷》上考太平興國庚辰歲,得丙午冬至,與《乾元》合。



 咸平三年庚子歲十一月辛卯日五十三刻冬至,距至元十七年庚辰歲,計二百八十年。《儀天歷》推之,得庚申冬至,後《授時》一日;《授時》上考咸平庚子歲,得辛卯冬至,與《儀天》合。



 崇寧四年乙酉歲十一月辛丑日六十二刻冬至,距至元十七年庚辰歲,計一百七十五年。《紀元歷》推之,得己未日冬至,後《授時》十九刻;《授時歷》上考崇寧乙酉歲,得辛丑日冬至,與《紀元歷》合,先二刻。



 金大定十九年己亥歲十一月己巳日六十四刻冬至,距至元十七年庚辰歲,計一百一年。《大明歷》推之,得己未冬至,後《授時》一十九刻;《授時歷》上考大定己亥歲,己巳冬至,與《大明歷》合,先九刻。《大明》冬至蓋測驗未密故也。



 慶元四年戊午歲十一月己酉日一十七刻冬至,距至元十七年庚辰歲,計八十二年。《統天歷》推之,得己未冬至,先《授時》一刻;《授時歷》上考慶元戊午歲,得己酉日冬至,與《統天歷》合。



 周天列宿度



 列宿著於天,為舍二十有八,為度三百六十五有奇。非日躔無以校其度,非列舍無以紀其度,周天之度,因二者以得之。天體渾圓,當二極南北之中,絡以赤道,日月五星之行,常出入於比。天左旋,日月五星溯而右轉,昔人歷象日月星辰,謂此也。然列舍相距度數,歷代所測不同,非微有動移,則前人所測或有未密。古用窺管,今新制渾儀,測用二線,所測度數分秒與前代不同者,今列於左。



 表略



 日躔



 日之麗天,縣象最著,大明一生,列宿俱熄。古人欲測躔度所在,必以昏旦夜半中星衡考其所距,從考其所當;然昏旦夜半時刻未易得真,時刻一差,則所距、所當,不容無舛。晉姜岌首以月食沖檢,知日度所在;《紀元歷》復以太白志其相距遠近,於昏後明前驗定星度,因得日躔。今用至元丁丑四月癸酉望月食既,推求得冬至日躔赤道箕宿十度,黃道九度有奇。仍自其年正月至己卯歲終,三年之間,日測太陰所離宿次及歲星、太白相距度,定驗參考,共得一百三十四事,皆躔箕宿,適與月食所沖允合。以金趙知微所修《大明歷法》推之,冬至猶躔鬥初度三十六分六十四秒,比新測實差七十六分六十四秒。



 日行盈縮



 日月之行,有冬有夏,言日月行度,冬夏各不同也。人徒知日行一度,一歲一周天,曾不知盈縮損益,四序有不同者。北齊張子信積候合蝕加時,覺日行有入氣差,然損益未得其正。趙道嚴復準晷景長短,定日行進退,更造盈縮以求虧食。至劉焯立躔度,與四序升降,雖損益不同,後代祖述用之。



 夫陰陽往來,馴積而變,冬至日行一度強,出赤道二十四度弱,自此日軌漸北,積八十八日九十一分,當春分前三日,交在赤道,實行九十一度三十一分而適平。自後其盈日損,復行九十三日七十一分,當夏至之日,入赤道內二十四度弱,實行九十一度三十一分,日行一度弱,向之盈分盡損而無餘。自此日軌漸南,積九十三日七十一分,當秋分後三日,交在赤道,實行九十一度三十一分而復平。自後其縮日損,行八十八日九十一分,出赤道外二十四度弱,實行九十一度三十一分,復當冬至,向之縮分盡損而無餘。盈縮均有損益,初為益,末為損。自冬至以及春分,春分以及夏至,日躔自北陸轉而西,西而南,於盈為益,益極而損,損至於無餘而縮。自夏至以及秋分,秋分以及冬至,日躔自南陸轉而東,東而北,於縮為益,益極而損,損至於無餘而復盈。盈初縮末,俱八十八日九十一分而行一象;縮初盈末,俱九十三日七十一分而行一象;盈縮極差,皆二度四十分。由實測晷景而得,仍以算術推考,與所測允合。



 月行遲疾



 古歷謂月平行十三度十九分度之七。漢耿壽昌以為日月行至牽牛、東井,日過度,月行十五度,至婁、角,始平行,赤道使然。賈逵以為今合朔、弦、望、月食加時,所以不中者,蓋不知月行遲疾意。李梵、蘇統皆以月行當有遲疾,不必在牽牛、東井、婁、角之間,乃由行道有遠近出入所生。劉洪作《乾象歷》,精思二十餘年,始悟其理,列為差率,以囿進退損益之數。後之作歷者,咸因之。至唐一行,考九道委蛇曲折之數,得月行疾徐之理。



 先儒謂月與五星,皆近日而疾,遠日而遲。歷家立法,以入轉一周之日,為遲疾二歷,各立初末二限,初為益,末為損。在疾初遲末,其行度率過於平行;遲初疾末,率不及於平行。自入轉初日行十四度半強,從是漸殺,歷七日,適及平行度,謂之疾初限,其積度比平行餘五度四十二分。自是其疾日損,又歷七日,行十二度微強,向之益者盡損而無餘,謂之疾末限。自是復行遲度,又歷七日,適及平行度,謂之遲初限,其積度比平行不及五度四十二分。自此其遲日損,行度漸增,又歷七日,復行十四度半強,向之益者亦損而無餘,謂之遲末限。入轉一周,實二十七日五十五刻四十六分,遲疾極差皆五度四十二分。舊歷日為一限,皆用二十八限。今定驗得轉分進退時各不同,今分日為十二,共三百三十六限,半之為半周限,析而四之為象限。



 白道交周



 當二極南北之中,橫絡天體以紀宿度者,赤道也。出入赤道,為日行之軌者,黃道也。所謂白道,與黃道交貫,月行之所由也。古人隨方立名,分為八行,與黃道而九,究而言之,其實一也。惟其隨交遷徙,變動不居,故強以方色名之。



 月道出入日道,兩相交值,當朔則日為月所掩,當望則月為日所沖,故皆有食。然涉交有遠近,食分有深淺,皆可以數推之。所謂交周者,月道出入日道一周之日也。日道距赤道之遠,為度二十有四。月道出入日道,不逾六度;其距赤道也,遠不過三十度,近不下十八度。出黃道外為陽,入黃道內為陰,陰陽一周,分為四象。月當黃道為正交,出黃道外六度為半交,復當黃道為中交,入黃道內六度為半交,是為四象。象別七日,各行九十一度,四象周歷,是謂一交之終,以日計之,得二十七日二十一刻二十二分二十四秒。每一交,退天一度二百分度之九十三,凡二百四十九交,退天一周有奇,終而復始。正交在春正,半交出黃道外六度,在赤道內十八度。正交在秋正,半交出黃道外六度,在赤道外三十度。中交在春正,半交入黃道內六度,在赤道內三十度。中交在秋正,半交入黃道內六度,在赤道外十八度。月道與赤道正交,距春秋二正黃赤道正交宿度,東西不及十四度三分度之二。夏至在陰歷內,冬至在陽歷外,月道與赤道所差者多;夏至在陽歷外,冬至在陰歷內,月道與赤道所差者少。蓋白道二交,有斜有直,陰陽二歷,有內有外,直者密而狹,斜者疏而闊,其差亦從而異。今立象置法求之,差數多者不過三度五十分,少者不下一度三十分,是為月道與赤道多少之差。



 晝夜刻



 日出為晝,日入為夜,晝夜一周,共為百刻。以十二辰分之,每辰得八刻三分刻之一。無間南北,所在皆同。晝短則夜長,夜短則晝長,此自然之理也。春秋二分,日當赤道出入,晝夜正等,各五十刻。自春分以及夏至,日入赤道內,去極浸近,夜短而晝長。自秋分以及冬至,日出赤道外,去極浸遠,晝短而夜長。以地中揆之,長不過六十刻,短不過四十刻。地中以南,夏至去日出入之所為遠,其長有不及六十刻者;冬至去日出入之所為近,其短有不止四十刻者。地中以北,夏至去日出入之所為近,其長有不止六十刻者;冬至去日出入之所為遠,其短有不及四十刻者。今京師冬至日出辰初二刻,日入申正二刻,故晝刻三十八,夜刻六十二;夏至日出寅正二刻,日入戌初二刻,故晝刻六十二,夜刻三十八。蓋地有南北,極有高下,日出入有早晏,所以不同耳。今《授時歷》晝夜刻,一以京師為正,其各所實測北極高下,具見《天文志》。



\end{pinyinscope}