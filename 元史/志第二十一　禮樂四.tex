\article{志第二十一 禮樂四}

\begin{pinyinscope}

 ○郊祀樂舞



 降神文舞,崇德之舞。《乾寧之曲》六成。



 圜鐘宮三成。始聽三鼓,一聲鐘,一聲鼓,凡三作,後仿此。一鼓稍前,開手立;二鼓合手,退後;三鼓相顧蹲。三鼓畢,間聲作。二聲鐘,一聲鼓。一鼓稍前,舞蹈;二鼓舉左手,收,左揖;三鼓舉右手,收,右揖;四鼓高呈手;五鼓兩兩相向蹲;六鼓稍前,開手立;七鼓退後,俯伏;八鼓舉左手,收,左揖;九鼓舉右手,收,右揖;十鼓稍前,開手立;十一鼓合手,退後,躬身;十二鼓伏,興,仰視;十三鼓舞蹈,相向立;十四鼓復位,交籥,正蹲;十五鼓躬身,受。終聽三鼓。止。



 黃鐘角一成。始聽三鼓。一鼓稍前,舞蹈;二鼓合手,退後;三鼓相顧蹲。三鼓畢,間聲作。一鼓稍前,舞蹈;二鼓高呈手;三鼓兩兩相向蹲;四鼓舉左手,收,左揖;五鼓舉右手,收,右揖;六鼓稍前,開手;七鼓復位,正揖;八鼓兩兩相向,交籥,正蹲;九鼓復位立;十鼓稍前,開手立;十一鼓合手,退後,躬身;十二鼓伏,興,仰視;十三鼓舉左手,收,開手,正蹲;十四鼓舉右手,收,開手,正蹲;十五鼓躬身,受。終聽三鼓。止。



 太簇徵一成。始聽三鼓,一鼓稍前,開手立;二鼓合手,退後;三鼓相顧蹲。三鼓畢,間聲作,一鼓稍前,舞蹈;二鼓復位,躬身;三鼓高呈手;四鼓舉左手,收,左揖;五鼓舉右手,收,右揖;六鼓兩兩相向,交籥,正蹲;七鼓復位,躬身;八鼓舞蹈,相向立;九鼓復位,俯伏;十鼓舉左手,收,左揖;十一鼓舉右手,收,右揖;十二鼓伏,興,仰視;十三鼓舞蹈,相向立;十四鼓復位,交籥,正蹲;十五鼓躬身,受。終聽三鼓。止。



 姑洗羽一成。始聽三鼓,一鼓稍前,開手立;二鼓合手,退後;三鼓相顧蹲。三鼓畢,間聲作,一鼓稍前,舞蹈;二鼓復位,正揖;三鼓高呈手;四鼓推左手,收,左揖;五鼓推右手,收,右揖;六鼓兩兩相向,交籥,正蹲;七鼓復位,俯伏;八鼓舞蹈,相向立;九鼓復位,躬身;十鼓伏,興,仰視;十一鼓舉左手,收,左揖;十二鼓舉右手,收,右揖;十三鼓舞蹈,相向立;十四鼓復位,交籥,正蹲;十五鼓躬身,受。終聽三鼓。止。



 昊天上帝位酌獻文舞,崇德之舞。《明成之曲》,黃鐘宮一成。始聽三鼓,一鼓稍前,開手立;二鼓合手,退後;三鼓相顧蹲。三鼓畢,間聲作,一鼓稍前,舞蹈,相向立;二鼓復位,相顧蹲;三鼓復位,開手立;四鼓合手,正揖;五鼓舉左手,收,左揖;六鼓舉右手,收,右揖;七鼓兩兩相向,交籥,正蹲;八鼓復位,正揖;九鼓稍前,開手立;十鼓退後,俯伏;十一鼓稍前,開手立;十二鼓推左手,收;十三鼓推右手,收;十四鼓三叩頭,拜舞;十五鼓躬身,受。終聽三鼓。止。



 皇地祇酌獻,大呂宮一成。始聽三鼓,一鼓稍前,開手立;二鼓合手,退後;三鼓相顧蹲。三鼓畢,間聲作,一鼓稍前,舞蹈,相向立;二鼓復位,正揖;三鼓舉左手,收,左揖;四鼓舉右手,收,右揖;五鼓高呈手;六鼓兩兩相向,交籥,正蹲;七鼓復位,俯伏;八鼓舞蹈,相向立;九鼓復位,躬身;十鼓交籥,正蹲;十一鼓兩兩相向,開手,正蹲;十二鼓伏,興,仰視;十三鼓舞蹈,相向立;十四鼓三叩頭,拜舞;十五鼓躬身,受。終聽三鼓。止。



 太祖位酌獻,黃鐘宮一成。始聽三鼓,一鼓稍前,開手立;二鼓合手,退後;三鼓相顧蹲。三鼓畢,間聲作。一鼓稍前,舞蹈;二鼓復位,正揖;三鼓舉左手,收,左揖;四鼓舉右手,收,右揖;五鼓高呈手;六鼓兩兩相向,交籥,正蹲;七鼓復位,俯伏;八鼓舞蹈,相向立;九鼓復位,躬身;十鼓交籥,正蹲;十一鼓兩兩相向,開手,正蹲;十二鼓伏,興,仰視;十三鼓合手,正揖;十四鼓叩頭,拜舞;十五鼓躬身,受。終聽三鼓。止。



 亞獻、酌獻武舞,定功之舞。黃鐘宮一成。始聽三鼓,一鼓稍前,開手立;二鼓合手,退後,按腰立;三鼓相顧蹲。三鼓畢,間聲作,一鼓稍前,左右揚干戚;二鼓退後,相顧蹲;三鼓舉左手,收;四鼓舉右手,收;五鼓左右揚干戚,相向立;六鼓復位,相顧蹲;七鼓呈幹戚;八鼓復位,按腰立;九鼓刺干戚;十鼓復位,推左手,收;十一鼓推右手,收;十二鼓稍前,開手立;十三鼓左右揚干戚;十四鼓復位,按腰,相顧蹲;十五鼓躬身,受。終聽三鼓。止。



 終獻武舞,黃鐘宮一成。始聽三鼓,一鼓稍前,開手立;二鼓合手,退後,按腰立;三鼓相顧蹲。三鼓畢,間聲作,一鼓稍前,左右揚干戚;二鼓退後,高呈手;三鼓復位,相顧蹲;四鼓左右揚干戚,相向立;五鼓復位,舉左手,收;六鼓舉右手,收;七鼓面向西,開手,正蹲;八鼓呈幹戚;九鼓復位,按腰立;十鼓刺干戚;十一鼓兩兩相向立;十二鼓復位,左右揚干戚;十三鼓退後,相顧蹲;十四鼓三叩頭,拜舞;十五鼓躬身,受。終聽三鼓。止。



 宗廟樂舞



 世祖至元三年,八室時享,文舞武定文綏之舞。降神,《來成之曲》九成。黃鐘宮三成。始聽三鼓,一鼓稍前,開手立;二鼓退後,合手;三鼓相顧蹲。三鼓畢,間聲作,一鼓稍前,舞蹈,次合手而立;二鼓正面高呈手,住;三鼓退後,收手蹲;四鼓正面躬身,興身立;五鼓推左手,右相顧,左揖;六鼓皆推右手,左相顧;七鼓稍前,正面開手立;八鼓舉左手,右相顧,左揖;九鼓舉右手,左相顧,右揖;十鼓稍退後,俯身而立;十一鼓稍前,開手立;十二鼓合手,退後,相顧蹲;十三鼓稍前,舞蹈;十四鼓退後,合手,相顧蹲;十五鼓正面躬身,受。終聽三鼓。止。



 大呂角二成。始聽三鼓,一鼓稍前,開手立;二鼓退後,合手;三鼓相顧蹲。三鼓畢,間聲作,一鼓稍進前,舞蹈,合手立;二鼓舉左手,住,收右足;三鼓舉右手,住,收左足;四鼓兩兩相向而立;五鼓稍前,高呈手,住;六鼓舞蹈,退後立;七鼓稍前,開手立;八鼓合手,退後蹲;九鼓正面歸佾立;十鼓推左手,收右足,推右手,收左足;十一鼓舉左手,收右足,舉右手,收左足;十二鼓稍進前,正面仰視;十三鼓稍退後,相顧蹲;十四鼓合手,俯身立;十五鼓正面躬身,受。終聽三鼓。止。



 太簇徵二成。始聽三鼓。一鼓稍前,開手立;二鼓退後,合手;三鼓相顧蹲。三鼓畢,間聲作,一鼓稍進前,舞蹈,次合手立;二鼓俯身而正面揖;三鼓稍進前,高呈手立;四鼓收手,正面蹲;五鼓舉左手,住,收右足;六鼓舉右手,收左足,收手;七鼓兩兩相向而立;八鼓稍前,高仰視;九鼓稍退,收手蹲;十鼓舉左手,住而蹲;十一鼓舉右手,收手而蹲;十二鼓正面歸佾,舞蹈;十三鼓俯身,正揖;十四鼓交籥翟,相顧蹲;十五鼓正面躬身,受。終聽三鼓。止。



 應鐘羽二成。始聽三鼓,一鼓稍前,開手立;二鼓退後,合手;三鼓相顧蹲。三鼓畢,間聲作,一鼓稍進前,舞蹈,次合手立;二鼓兩兩相向立;三鼓舉左手,收右足,左揖;四鼓舉右手,收左足,右揖;五鼓歸佾,正面立;六鼓稍進前,高呈手,住;七鼓收手,稍退,相顧蹲;八鼓兩兩相向立;九鼓稍前,開手蹲;十鼓退後,合手對揖;十一鼓正面歸佾立;十二鼓稍進前,舞蹈,次合手立;十三鼓垂左手而右足應;十四鼓垂右手而左足應;十五鼓正面躬身,受。終聽三鼓。止。



 烈祖第一室文舞,《開成之曲》,無射宮一成。始聽三鼓,一鼓稍前,開手立;二鼓稍退,合手;三鼓相顧蹲。三鼓畢,間聲作,一鼓稍進前,舞蹈,合手立;二鼓稍退,俯身,開手立;三鼓垂左手,住,收右足;四鼓垂右手,住,收左足;五鼓左側身相顧,左揖;六鼓右側身相顧,右揖;七鼓正面躬身,興身立;八鼓兩兩相向,合手立;九鼓相顧高呈手,住;十鼓收手,舞蹈;十一鼓舞左而收手立;十二鼓舞右而收手立;十三鼓揚左手,相顧蹲;十四鼓揚右手,相顧蹲;十五鼓稍前,正面躬身,受。終聽三鼓。止。



 太祖第二室文舞,《武成之曲》,無射宮一成。始聽三鼓,一鼓稍前,開手立;二鼓退後,合手;三鼓相顧蹲。三鼓畢,間聲作,一鼓稍前,舞蹈,次合手立;二鼓正面高呈手,住;三鼓兩兩相向而對揖;四鼓正面歸佾,舞蹈,次合手立;五鼓稍前,開手蹲,收手立;六鼓稍退,合手蹲,收手立;七鼓舉左手而左揖;八鼓舉右手而右揖;九鼓推左手住而正蹲;十鼓推右手正蹲;十一鼓開手執籥翟,正面俯視;十二鼓垂左手,收右足;十三鼓垂右手,收左足;十四鼓稍前,正面仰視而立;十五鼓稍前,正面躬身,受。終聽三鼓。止。



 太宗第三室文舞,《文成之曲》,無射宮一成。始聽三鼓。一鼓稍前,開手立;二鼓退後,合手;三鼓相顧蹲。三鼓畢,間聲作。一鼓稍進前,舞蹈;二鼓兩相向而高呈手立;三鼓稍前,開手立,相顧蹲;四鼓退後,合手立,相顧蹲;五鼓垂左手而右足應;六鼓垂右手而左足應;七鼓推左手,住,左揖;八鼓推右手,住,右揖;九鼓稍前,仰視,正揖;十鼓舉左手,住,收右足;十一鼓舉右手,住,收左足;十二鼓稍前,舞蹈;十三鼓稍前,開手而相顧立;十四鼓退後,合手立;十五鼓稍前,正面躬身,受。終聽三鼓。止。



 皇伯考術赤第四室文舞,《弼成之曲》,無射宮一成。始聽三鼓,一鼓稍前,開手立;二鼓退後,合手;三鼓相顧蹲。三鼓畢,間聲作,一鼓稍進前,舞蹈;二鼓合手,俯身相顧蹲;三鼓正面高呈手,住;四鼓稍前,舞蹈,次合手立;五鼓垂左手,右相顧,收手立;六鼓垂右手,左相顧,收手立;七鼓稍前,高仰視,收手,正面立;八鼓再退,高執籥翟,相顧蹲;九鼓舞蹈,次合手而立;十鼓舉左手,住,收右足;十一鼓舉右手,住,收左足;十二鼓稍前,開手立,收手蹲;十三鼓稍前,退後,合手立;十四鼓俯身,合手而立;十五鼓稍前,正面躬身,受。終聽三鼓。止。



 皇伯考察合帶第五室文舞,《協成之曲》,無射宮一成。始聽三鼓,一鼓稍前,開手立;二鼓退後,合手;三鼓相顧蹲。三鼓畢,間聲作,一鼓稍進前,舞蹈,次合手立;二鼓開手,相顧蹲;三鼓合手,相顧蹲;四鼓稍前,高呈手,住;五鼓舉左手,右相顧,左揖;六鼓舉右手,左相顧,右揖;七鼓推左手,住,收右足;八鼓推右手,住,收左足;九鼓稍前,舞蹈,次合手立;十鼓開手,正蹲,收,合手立;十一鼓稍前,正面仰視立;十二鼓交籥翟,相顧蹲;十三鼓各盡舉左手而住;十四鼓各盡舉右手,收手立;十五鼓稍前,正面躬身,受。終聽三鼓。止。



 睿宗第六室文舞,《明成之曲》,無射宮一成。始聽三鼓,一鼓稍前,開手立;二鼓退後,合手;二鼓相顧蹲。三鼓畢,間聲作,一鼓稍前,舞蹈;二鼓稍前,開手立;三鼓退後,合手立;四鼓垂左手,相顧蹲;五鼓垂右手,相顧蹲;六鼓稍前,正面仰視立;七鼓舞左手,住,收右足,收手;八鼓舞右手,住,收左足,收手;九鼓兩相向,合手而立;十鼓推左手,推右手;十一鼓皆舉左右手;十二鼓正面高呈手,立;十三鼓退後,合手,俯身;十四鼓開手,高呈籥翟,相顧蹲;十五鼓正面稍前,躬身,受。終聽三鼓。止。



 定宗第七室文舞,《熙成之曲》,無射宮一成。始聽三鼓,一鼓稍前,開手立;二鼓退後,合手;三鼓相顧蹲。三鼓畢,間聲作,一鼓稍前,舞蹈;二鼓兩相向,高呈手立;三鼓垂左手而右足應;四鼓垂右手而左足應;五鼓稍前,開手立,相顧蹲;六鼓退後,合手立,相顧蹲;七鼓舉左手,住,收右足;八鼓舉右手,住,收左足;九鼓推左手,左揖;十鼓推右手,右揖;十一鼓稍前,舞蹈;十二鼓退後,正揖;十三鼓稍前,開手相顧立;十四鼓退後,合手立;十五鼓稍前,正面躬身,受。終聽三鼓。止。



 憲宗第八室文舞,《威成之曲》,無射宮一成。始聽三鼓,一鼓稍前,開手;二鼓退後,合手;三鼓相顧蹲。三鼓畢,間聲作,一鼓進前,舞蹈,次合手立;二鼓高呈手,住;三鼓舉左手,右顧;四鼓舉右手,左顧;五鼓推左手,右揖;六鼓推右手,左揖;七鼓兩相向,交籥翟,立;八鼓正面歸佾,合手立;九鼓稍前,舞蹈,收手立;十鼓退後,正揖;十一鼓俯身,正面揖;十二鼓高仰視;十三鼓垂左手;十四鼓垂右手;十五鼓正面躬身,受。終聽三鼓。止。



 亞獻武舞,內平外成之舞。《順成之曲》,無射宮一成。始聽三鼓,一鼓側身開手,二鼓合手,三鼓相顧蹲。三鼓畢,間聲作,一鼓皆稍進前,舞蹈,次按腰立;二鼓按腰,相顧蹲;三鼓左右揚干戚,收手按腰;右以象滅王罕。四鼓稍退,舞蹈,按腰立;五鼓兩兩相向,按腰立;六鼓歸佾,開手,蹲;七鼓面西,收手按腰立;八鼓側身擊幹戚,收,立;右以象破西夏。九鼓正面歸佾,躬身,次興身立;十鼓稍進前,舞蹈,次按腰立;十一鼓左右推手,次按腰立;十二鼓跪左膝,疊手,呈幹戚,住;右以象克金國。十三鼓收手,按腰,興身立;十四鼓兩相向而相顧,蹲;十五鼓正面躬身,受。終聽三鼓。止。



 終獻武舞,《順成之曲》,無射宮一成。始聽三鼓,一鼓側身,開手立;二鼓合手,按腰;三鼓相顧蹲。三鼓畢,間聲作,一鼓稍進前,舞蹈,次按腰立;二鼓開手,正面蹲,收手按腰;三鼓面西,舞蹈,次按腰立;四鼓面南,左右揚干戚,收手按腰;五鼓側身擊幹戚,收手按腰,立;右以象收西域、定河南。六鼓兩兩相向立;七鼓歸佾,正面開手,蹲,收手按腰;八鼓東西相向,躬身,受;右以象收西蜀、平南詔。九鼓歸佾,舞蹈,退後,次按腰立;十鼓推左右手,躬身,次興身立;十一鼓進前,舞蹈,次按腰立;右以象臣高麗、服交趾。十二鼓兩兩相向,按腰蹲;十三鼓歸佾,左右揚手,按腰立;十四鼓正面開手,俯視;十五鼓收手按腰,躬身,受。終聽三鼓。止。



 泰定十室樂舞



 迎神文舞,《思成之曲》。



 黃鐘宮三成。始聽三鼓,一鼓稍前,開手立;二鼓合手,退後;三鼓相顧蹲。三鼓畢,間聲作,一鼓稍前,舞蹈;二鼓高呈手;三鼓舉左手,收,左揖;四鼓舉右手,收,右揖;五鼓退後,相顧蹲;六鼓兩兩相向立;七鼓復位,俯伏;八鼓舉左手,開手,正蹲;九鼓舉右手,開手,正蹲;十鼓稍前,開手立;十一鼓合手,退後,躬身;十二鼓伏,興,仰視;十三鼓舞蹈,相向立;十四鼓復位,交籥,正蹲;十五鼓躬身,受。終聽三鼓。止。



 大呂角二成。始聽三鼓,一鼓稍前,舞蹈;二鼓合手,退後;三鼓相顧蹲。三鼓畢,間聲作,一鼓稍前,舞蹈;二鼓舉左手,收,左揖;三鼓舉右手,收,右揖;四鼓高呈手;五鼓兩兩相顧蹲;六鼓稍前,開手立;七鼓復位,正揖;八鼓兩兩相向,交籥,正蹲;九鼓復位,正揖;十鼓舉左手,收,左揖;十一鼓舉右手,收,右揖;十二鼓伏,興,仰視;十三鼓舞蹈,相向立;十四鼓復位,立;十五鼓躬身,受。終聽三鼓。止。



 太簇徵二成。始聽三鼓,一鼓稍前,開手立;二鼓合手,退後;三鼓相顧蹲。三鼓畢,間聲作,一鼓稍前,舞蹈;二鼓復位,躬身;三鼓高呈手;四鼓兩兩相向,交籥,正蹲;五鼓復位立;六鼓舞蹈,相向立;七鼓舉左手,收,左揖;八鼓舉右手,收,右揖;九鼓稍前,舞蹈;十鼓退後,俯伏;十一鼓稍前,開手立;十二鼓推左手,收;十三鼓推右手,收;十四鼓三叩頭,拜舞;十五鼓躬身,受。終聽三鼓。止。



 應鐘羽二成。始聽三鼓,一鼓稍前,開手立;二鼓合手,退後;三鼓相顧蹲。三鼓畢,間聲作,一鼓稍前,舞蹈;二鼓復位,正揖;三鼓高呈手;四鼓稍前,開手立;五鼓退後,躬身;六鼓推左手,收;七鼓推右手,收;八鼓舞蹈,相向立;九鼓復位,躬身;十鼓交籥,正蹲;十一鼓兩兩相向,開手,正蹲;十二鼓舉左手,收,左揖;十三鼓舉右手,收,右揖;十四鼓三叩頭,拜舞;十五鼓躬身,受。終聽三鼓。止。



 初獻、酌獻太祖第一室文舞,《開成之曲》,無射宮一成。始聽三鼓,一鼓稍前,開手立;二鼓合手,退;三鼓相顧蹲。三鼓畢,間聲作,一鼓稍前,舞蹈,相向立;二鼓復位,正揖;三鼓推左手,收;四鼓推右手,收;五鼓三叩頭,拜舞;六鼓兩兩相向,交籥,正蹲;七鼓復位立;八鼓稍前,舞蹈;九鼓復位,俯伏;十鼓高呈手,正揖;十一鼓兩兩相向蹲;十二鼓復位,開手立;十三鼓合手,正揖;十四鼓伏,興,仰視;十五鼓躬身,受。終聽三鼓。止。



 睿宗第二室文舞,《武成之曲》,無射宮一成。始聽三鼓。一鼓稍前,開手立;二鼓合手,退後;三鼓相顧蹲。三鼓畢,間聲作,一鼓稍前,舞蹈;二鼓復位,正揖;三鼓高呈手;四鼓稍前,開手立;五鼓退後,躬身;六鼓舉左手,收,左揖;七鼓舉右手,收,右揖;八鼓舞蹈,相向立;九鼓復位立;十鼓推左手,收;十一鼓推右手,收;十二鼓伏,興,仰視;十三鼓兩兩相向蹲;十四鼓復位,交籥,正蹲;十五鼓躬身,受。終聽三鼓。止。



 世祖第三室文舞,《混成之曲》,無射宮一成。始聽三鼓,一鼓稍前,開手立;二鼓合手,退後;三鼓相顧蹲。三鼓畢,間聲作,一鼓稍前,舞蹈;二鼓高呈手;三鼓交籥,正蹲;四鼓兩兩相向,開手,正蹲;五鼓伏,興,仰視;六鼓舉左手,收,左揖;七鼓舉右手,收,右揖;八鼓退後,躬身;九鼓稍前,開手立;十鼓舉左手,收,左揖;十一鼓舉右手,收,右揖;十二鼓高呈手,正揖;十三鼓舞蹈,相顧蹲;十四鼓三叩頭,拜舞;十五鼓躬身,受。終聽三鼓。止。



 裕宗第四室文舞,《昭成之曲》,無射宮一成。始聽三鼓,一鼓稍前,開手立;二鼓合手,退後;三鼓相顧蹲。三鼓畢,間聲作,一鼓稍前,舞蹈;二鼓退後,高呈手;三鼓舉左手,收,左揖;四鼓舉右手,收,右揖;五鼓稍前,開手立;六鼓退後,躬身;七鼓兩兩相向,交籥,正蹲;八鼓伏,興,仰視;九鼓推左手,收,左揖;十鼓推右手,收,右揖;十一鼓稍前,舞蹈;十二鼓退後,相顧蹲;十三鼓高呈手;十四鼓三叩頭,拜舞;十五鼓躬身,受。終聽三鼓。止。



 顯宗第五室文舞,《德成之曲》,無射宮一成。始聽三鼓,一鼓稍前,開手立;二鼓合手,退後;三鼓相顧蹲。三鼓畢,間聲作,一鼓稍前,舞蹈,相向立;二鼓復位,正揖;三鼓舉左手,收;四鼓舉右手,收;五鼓伏,興,仰視;六鼓兩兩相向立;七鼓復位,交籥,正蹲;八鼓退後,躬身;九鼓稍前,開手立;十鼓舉左手,收,左揖;十一鼓舉右手,收,右揖;十二鼓高呈手;十三鼓復位,正蹲;十四鼓三叩頭,拜舞;十五鼓躬身,受。終聽三鼓。止。



 順宗第六室文舞,《慶成之曲》,無射宮一成。始聽三鼓,一鼓稍前,開手立;二鼓合手,退後;三鼓相顧蹲。三鼓畢,間聲作。一鼓稍前,舞蹈;二鼓復位,相顧蹲;三鼓稍前,開手立;四鼓合手,正揖;五鼓舉左手,收,左揖;六鼓舉右手,收,右揖;七鼓兩兩相向,交籥,正蹲;八鼓復位立;九鼓稍前,開手立;十鼓伏,興,仰視;十一鼓舉左手,收,相顧蹲;十二鼓舉右手,收,相顧蹲;十三鼓高呈手,正揖;十四鼓三叩頭,拜舞;十五鼓躬身,受。終聽三鼓。止。



 成宗第七室文舞,《守成之曲》,無射宮一成。始聽三鼓,一鼓稍前,開手立;二鼓合手,退後;三鼓相顧蹲。三鼓畢,間聲作,一鼓稍前,舞蹈;二鼓退後,躬身;三鼓舉左手,收,左揖;四鼓舉右手,收,右揖;五鼓伏,興,仰視;六鼓兩兩相向,交籥,正蹲;七鼓復位,正揖;八鼓高呈手;九鼓舉左手,收,左揖;十鼓舉右手,收,右揖;十一鼓開手立;十二鼓合手,正揖;十三鼓稍前,舞蹈;十四鼓三叩頭,拜舞;十五鼓躬身,受。終聽三鼓。止。



 武宗第八室文舞,《威成之曲》,無射宮一成。始聽三鼓,一鼓稍前,開手立;二鼓合手,退後;三鼓相顧蹲。三鼓畢,間聲作,一鼓稍前,舞蹈;二鼓退後,正揖;三鼓高呈手;四鼓稍前,開手立;五鼓退後,躬身;六鼓舉左手,收,左揖;七鼓舉右手,收,右揖;八鼓舞蹈,相向立;九鼓復位立;十鼓舉左手,收,左揖;十一鼓舉右手,收,右揖;十二鼓伏,興,仰視;十三鼓兩兩相向立;十四鼓復位,交籥,正蹲;十五鼓躬身,受。終聽三鼓。止。



 仁宗第九室文舞,《歆成之曲》,無射宮一成。始聽三鼓,一鼓稍前,開手立;二鼓合手,退後;三鼓相顧蹲。三鼓畢,間聲作,一鼓稍前,舞蹈,相向立;二鼓復位,正揖;三鼓高呈手;四鼓推左手,收;五鼓推右手,收;六鼓稍前,開手立;七鼓退後,躬身;八鼓兩兩相向立;九鼓復位,交籥,正蹲;十鼓舉左手,收,左揖;十一鼓舉右手,收,右揖;十二鼓稍前,舞蹈;十三鼓復位,正揖;十四鼓伏,興,仰視;十五鼓躬身,受。終聽三鼓。止。



 英宗第十室文舞,《獻成之曲》,無射宮一成。始聽三鼓,一鼓稍前,開手立;二鼓合手,退後;三鼓相顧蹲。三鼓畢,間聲作,一鼓稍前,舞蹈,相向立;二鼓舉左手,收,左揖;三鼓舉右手,收,右揖;四鼓高呈手;五鼓伏,興,仰視;六鼓兩兩相向蹲;七鼓退後,俯伏;八鼓復位,交籥,正蹲;九鼓稍前,開手立;十鼓復位,躬身;十一鼓稍前,舞蹈;十二鼓復位,正揖;十三鼓舞蹈,兩兩相向立;十四鼓三叩頭,拜舞;十五鼓躬身,受。終聽三鼓。止。



 亞獻武舞,《肅寧之曲》,無射宮一成。始聽三鼓,一鼓稍前,開手立;二鼓合手,退後,按腰立;三鼓相顧蹲。三鼓畢,間聲作,一鼓稍前,左右揚干戚;二鼓退後,相顧蹲;三鼓高呈手;四鼓左右揚干戚;五鼓呈幹戚;六鼓復位,按腰立;七鼓刺干戚;八鼓兩兩相向,開手,正蹲;九鼓復位,舉左手,收;十鼓舉右手,收;十一鼓稍前,開手立;十二鼓退後,按腰立;十三鼓左右揚干戚,相向立;十四鼓復位,按腰,相顧蹲;十五鼓躬身,受。終聽三鼓。止。



 終獻武舞,《肅寧之曲》,無射宮一成。始聽三鼓,一鼓稍前,開手立;二鼓合手,退後,按腰立;三鼓相顧蹲。三鼓畢,間聲作,一鼓稍前,左右揚干戚;二鼓退後,高呈手;三鼓舉左手,收;四鼓舉右手,收;五鼓面向西,開手,正蹲;六鼓復位,左右揚干戚;七鼓躬身,受;八鼓呈幹戚;九鼓復位,按腰立;十鼓刺干戚;十一鼓兩兩相向立;十二鼓復位,按腰立;十三鼓退後,相顧蹲;十四鼓三叩頭,拜舞;十五鼓躬身,受。終聽三鼓。止。



 天歷三年新制樂舞。明宗酌獻武舞,《永成之曲》,無射宮一成。始聽三鼓,一鼓合手稍前,開手立;二鼓退後立;三鼓相顧蹲。三鼓畢,間聲作,一鼓向前,舞蹈,相向立;二鼓復位,三叩頭,拜舞;三鼓兩兩開手,正蹲;四鼓復位,俯伏;五鼓交籥,正蹲;六鼓伏,興,仰視;七鼓躬身;八鼓稍前,開手立;九鼓復位,正揖,高呈手;十鼓舉左手,收,左揖;十一鼓舉右手,收,右揖;十二鼓正揖;十三鼓兩兩交籥,相揖;十四鼓復位;十五鼓躬身,受。終聽三鼓。止。



\end{pinyinscope}