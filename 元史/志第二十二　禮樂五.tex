\article{志第二十二 禮樂五}

\begin{pinyinscope}

 ○樂服



 樂正副四人,舒腳襆頭,紫羅公服,烏角帶,木笏,皁靴。



 照燭二人,服同前,無笏。



 樂師二人,服緋,冠、笏同前。



 運譜二人,服綠,冠、笏同前。



 舞師二人,舒腳襆頭,黃羅繡抹額,紫服,金銅荔枝帶,皁靴,各執仗。仗,牙仗也。



 執旌二人,平冕,前後各九旒五就,青生色鸞袍,黃綾帶,黃絹褲,白絹襪,赤革履。平冕鸞袍,皆仿金制,惟冕之旒數不同,詳見後至元二年博士議。



 執纛二人,青羅巾,餘同執旌。



 樂工,介幘冠,緋羅生色鸞袍,黃綾帶,皁靴。冠以皮為之,黑油如熊耳,亦金制也。



 歌工,服同樂工。



 執麾,服同上,惟加平巾幘。狀若籠金幘,以革為之。



 舞人,青羅生色義花鸞袍,緣以皁綾,平冕冠。冠前後有旒,青白硝石珠相間。



 執器二十人,服同樂工,綠油母追冠,革為之,一名武弁。加紅抹額。



 至元二年閏五月,大樂署言,堂上下樂舞官員及樂工,合用衣服冠冕靴履等物,乞行制造。太常寺下博士議定:樂正副四人、樂師二人、照燭二人、運譜二人,皆服紫羅公服,皁紗襆頭舒腳,紅鞓角帶,木笏,皁靴。引舞色長四人,紫羅公服,皁紗襆頭展腳,黃羅繡南花抹額,金銅帶,皁靴。樂工二百四十有六人,緋繡義花鸞袍,縣黃插口,介幘冠,紫羅帶,全黃羅抹帶,黃絹夾褲,白綾襪,硃履。金太常寺掌故張珍所著《疊代世範》載金制:舞人服黑衫,皆四襖,有黃插口,左右垂之,黃綾抹帶,其衫以綢為之,胸背二答、兩肩二答,前後和一答,皆彩色,繡二鸞盤飛之狀,綴之於衫。冠以平冕,亦有天板、口圈,天門納言以紫絹標背,銅裹邊圈,前後各五旒,以青白硝石珠相間。《大備集》所載,二舞人皁繡義花鸞衫,縣紫插口,黃綾抹帶,硃履,平冕。其冠有口圈,亦有天門納言系帶,口圈高一尺許,天板長二尺,闊一尺,前微高後低,里外紫絹糊,銅楞道妝釘,無旒。執器二十人,緋繡義花鸞袍,縣黃插口,綠油革冠,黃羅抹帶,黃絹夾褲,白綾襪,硃履。旌纛四人,青繡義花鸞袍,縣紫插口,平冕冠二,青包巾二,黃羅抹帶,黃絹夾褲,白綾襪,硃履。七月,中書吏部再準太常博士議定,行下所司製造。三年九月服成,緋鸞袍二百六十有七,青鸞袍一百三十二,黃絹褲一百五十二,紫羅公服一十四,黃綾帶三百九十七,介幘冠二百四十有四,平冕冠百三十,簪全,木笏十有六,襆頭十有四,平巾幘二,綠油革冠二十,荔枝銅帶四,角帶十,皁靴二百六十對,硃履百五十對。



 宣聖廟樂工,黑漆冠三十五,綠羅生色胸背花袍三十五,皁靴三十五對,黃絹囊三十五,黃絹夾袱三十五。



 大樂職掌



 大樂署,令一人,丞一人,掌郊社、宗廟之樂。凡樂,郊社、宗廟則用宮縣,工二百六十有一人;社稷,則用登歌,工五十有一人;二樂用工三百一十有二人,代事故者五十人。前祭之月,召工習樂及舞。祀前一日,宿縣於庭中。東方西方設十二鎛鐘,各依辰位。編鐘處其左,編磬處其右。黃鐘之鐘起子位,在通街之西。蕤賓之鐘居午位,在通街之東。每辰三虡,謂之一肆,十有二辰,凡三十六虡。樹建鞞應於四隅,左柷右敔,設縣中之北。歌工次之,三十二人,重行相向而坐。巢笙次之,簫次之,竽次之,籥次之,篪次之,塤次之,長笛又次之。夾街之左右,瑟翼柷敔之東西,在前行。路鼓、路鞀次之。郊祀則雷鼓、雷鞀。閏餘匏在簫之東,七星匏在西,九曜匏次之。一弦琴列路鼓之東西,東一,西二。三弦、五弦、七弦、九弦次之。晉鼓一,處縣中之東南,以節樂。一弦琴三,五弦以下皆六。凡坐者,高以杌,地以氈。立四表於橫街之南,少東。設舞位於縣北。文郎左執籥,右秉翟;武郎左執干,右執戚;皆六十有四人。享日,與工人先入就位。舞師二人,執纛二人,引文舞分立於表南。武舞及執器者,俟立於宮縣之左右。器鞀二,雙鐸二,單鐸二,鐃二,錞二,二錞用六人。鉦二,相鼓二,雅鼓二,凡二十人。文舞退,舞師二人、執旌二人,引武舞進,立其處,文舞還立於縣側。又設登歌樂於殿之前楹,殿陛之旁,設樂床二,樂工列於上。搏拊二,歌工六,柷一,敔一,在門內,相向而坐。鐘一虡,在前楹之東。一弦、三弦、五弦、七弦、九弦琴五,次之。瑟二,在其東,笛一、籥一、篪一在琴之南,巢笙、和笙各二次之。塤一,在笛之南。閏餘匏、排簫各一,次之,皆西上。磬一虡,在前楹之西。一弦、三弦、五弦、七弦、九弦琴五,次之。塤一,在笛之南。七星匏、九曜匏、排簫各一,次之,皆東上。凡宗廟之樂九成,舞九變。黃鐘之宮,三成,三變。大呂之角,二成,二變。太簇之徵,二成,二變。應鐘之羽,二成,二變。圜丘之樂六成,舞六變。夾鐘之宮,三成,三變。黃鐘之角,一成,一變。太簇之徵,一成,一變。姑洗之羽,一成,一變。社稷之樂八成:林鐘之宮二成,太簇之角二成,姑洗之徵二成,南呂之羽二成。凡有事於宗廟,大樂令位於殿楹之東,西向;丞位於縣北,通街之東,西向;以肅樂舞。



 協律郎二人,掌和律呂,以合陰陽之聲。陽律六:黃鐘子,太簇寅,姑洗辰,蕤賓午,夷則申,無射戌。陰呂六:大呂丑,夾鐘卯,仲呂巳,林鐘未,南呂酉,應鐘亥。文之以宮、商、角、征、羽、變宮、變徵,播之以金、石、絲、竹、匏、土、革、木。凡律管之數九,九九相乘,八十一以為宮;三分去一,五十四以為徵;三分益一,七十二以為商;三分去一,四十八以為羽;三分益一,六十四以為角。如黃鐘為宮,則林鐘為征,太簇為商,南呂為羽,姑洗為角,應鐘為變宮,蕤賓為變徵,是為七聲十二律,還相為宮,為八十四調。凡大祭祀皆法服,一人立於殿楹之西,東向;一人立於縣北通街之西,東向;以節樂。堂上者主登歌,堂下者主宮縣。凡樂作,則跪,俯伏,舉麾以興,工鼓柷以奏;樂止則偃麾,工戛敔而樂止。今執麾者代執之,協律郎特拜而已。



 樂正二人,副二人,掌肄樂舞、展樂器、正樂位。凡祭,二人立於殿內,二人立於縣間,以節樂。殿內者視獻者奠獻用樂作止之節,以笏示照燭,照燭舉偃以示堂下。若作登歌,則以笏示柷敔而已。縣間者視堂上照燭。及引初獻,照燭動,亦以笏示柷敔。



 樂師一人,運譜一人,掌以樂教工人。凡祭,立於縣間,皆北上,相向而立。



 舞師四人,皆執梃,梃,牙仗也。執纛二人,執旌二人,祭則前舉以為舞容。舞人從南表向第一表,為一成,則一變;從第二至第三,為二成;從第三至北第四表,為三成;舞人各轉身南向於北表之北,還從第一至第二,為四成;從第二至第三,為五成;從第三至南第一表,為六成;若八變者,更從南北向第二,為七成;又從第二至第三,為八成;若九變者,又從第三至北第一,為九變。



 執麾一人,從協律郎以麾舉偃而節樂。



 照燭二人,掌執籠燭而節樂。凡樂作止,皆舉偃其籠燭。一人立於堂上門東,視殿內獻官禮節,麾燭以示縣間。一人立於堂下縣間,俟三獻入導初獻至位,立於其左。初獻行,皆前導,亞、終則否。凡殿下禮節,則麾其燭以示上下。初獻詣盥洗位,乃偃其燭,止亦如之。俟初獻動為節,宮縣樂作,詣盥洗位,洗拭瓚訖,樂止。詣階,登歌樂作,升自東階,至殿門,樂止,乃立於陛側以俟。晨稞訖,初獻出殿,登歌樂作,至版位,樂止。司徒迎饌至橫街,轉身北向,宮縣樂作,司徒奉俎至各室遍奠訖,樂止。酌獻,初獻詣盥洗位,宮縣樂作,詣爵洗位,洗拭爵訖,樂止。出笏,登歌樂作,升自東階,至殿門,樂止。初獻至酒尊所,酌訖,宮縣樂作,詣神位前,祭酒訖,拜、興、讀祝,樂止。讀訖,樂作,再拜訖,樂止。次詣每室,作止如初。每室各奏本室樂曲,俱獻畢,還至殿門,登歌樂作,降自東階,至版位,樂止。文舞退,武舞進,宮縣樂作,舞者立定,樂止。亞獻行禮,無節步之樂,至酒尊所,酌酒訖,出笏,宮縣樂作,詣神位前,奠獻畢,樂止。次詣每室,作止如初。俱畢,還至版位,皆無樂。終獻樂作,同亞獻,助奠以下升殿,奠馬湩,至神位,蒙古巫祝致詞訖,宮縣樂作,同司徒進饌之曲,禮畢,樂止。出殿,登歌樂作,各復位,樂止。太祝徹籩豆,登歌樂作,卒徹,樂止。奉禮贊拜,眾官皆再拜訖,送神,宮縣樂作,一成而止。



 宴樂之器



 興隆笙,制以楠木,形如夾屏,上銳而面平,縷金雕鏤枇杷、寶相、孔雀、竹木、雲氣,兩旁側立花板,居背三之一,中為虛櫃,如笙之匏。上豎紫竹管九十,管端實以木蓮苞。櫃外出小橛十五,上豎小管,管端實以銅杏葉。下有座,獅象繞之,座上櫃前立花板一,雕鏤如背,板間出二皮風口,用則設硃漆小架於座前,系風囊於風口,囊面如琵琶,硃漆雜花,有柄,一人挼小管,一人鼓風囊,則簧自隨調而鳴。中統間,回回國所進。以竹為簧,有聲而無律。玉宸樂院判官鄭秀乃考音律,分定清濁,增改如今制。其在殿上者,盾頭兩旁立刻木孔雀二,飾以真孔雀羽,中設機。每奏,工三人,一人鼓風囊,一人按律,一人運動其機,則孔雀飛舞應節。



 殿庭笙十,延祐間增制,不用孔雀。



 琵琶,制以木,曲首,長頸,四軫,頸有品,闊面,四弦,面飾雜花。



 箏,如瑟,兩頭微垂,有柱,十三弦。



 火不思,制如琵琶,直頸,無品,有小槽,圓腹如半瓶榼,以皮為面,四弦,皮絣同一孤柱。



 胡琴,制如火不思,卷頸,龍首,二弦,用弓捩之,弓之弦以馬尾。



 方響,制以鐵,十六枚,懸於磬虡,小角槌二。廷中設,下施小交足幾,黃羅銷金衣。



 龍笛,制如笛,七孔,橫吹之,管首制龍頭,銜同心結帶。



 頭管,制以竹為管,卷蘆葉為首,竅七。



 笙,制以匏為底,列管於上,管十三,簧如之。



 箜篌,制以木,闊腹,腹下施橫木,而加軫二十四,柱頭及首並加鳳喙。



 雲璈,制以銅,為小鑼十三,同一木架,下有長柄,左手持,而右手以小槌擊之。



 簫,制如笛,五孔。



 戲竹,制如籈,長二尺餘,上系流蘇香囊,執而偃之,以止樂。



 鼓,制以木為匡,冒以革,硃漆雜花,面繪復身龍,長竿二。廷中設,則有大木架,又有擊撾高座。



 杖鼓,制以木為匡,細腰,以皮冒之,上施五彩繡帶,右擊以杖,左拍以手。



 札鼓,制如杖鼓而小,左持而右擊之。



 和鼓,制如大鼓而小,左持而右擊之。



 闉,制如箏而七弦,有柱,用竹軋之。



 羌笛,制如笛而長,三孔。



 拍板,制以木為板,以繩聯之。



 水盞,制以銅,凡十有二,擊以鐵箸。



 樂隊



 樂音王隊:元旦用之。引隊大樂禮官二員,冠展角襆頭,紫袍,塗金帶,執笏。次執戲竹二人,同前服。次樂工八人,冠花襆頭。紫窄衫,銅束帶。龍笛三,杖鼓三,金鞚小鼓一,板一,奏《萬年歡》之曲。從東階升,至御前,以次而西,折繞而南,北向立。後隊進,皆仿此。次二隊,婦女十人,冠展角襆頭,紫袍,隨樂聲進至御前,分左右相向立。次婦女一人,冠唐帽,黃袍,進北向立定,樂止,念致語畢,樂作,奏《長春柳》之曲。次三隊,男子三人,戴紅發青面具,雜彩衣,次一人,冠唐帽,綠襴袍,角帶,舞蹈而進,立於前隊之右。次四隊,男子一人,戴孔雀明王像面具,披金甲,執叉,從者二人,戴毗沙神像面具,紅袍,執斧。次五隊,男子五人,冠五梁冠,戴龍王面具,繡氅,執圭,與前隊同進,北向立。次六隊,男子五人,為飛天夜叉之像,舞蹈以進。次七隊,樂工八人,冠霸王冠,青面具,錦繡衣,龍笛三,觱慄三,杖鼓二,與前大樂合奏《吉利牙》之曲。次八隊,婦女二十人,冠廣翠冠,銷金綠衣,執牡丹花,舞唱前曲,與樂聲相和,進至御前,北向,列為九重,重四人,曲終,再起,與後隊相和。次九隊,婦女二十人,冠金梳翠花鈿,繡衣,執花鞚稍子鼓,舞唱前曲,與前隊相和。次十隊,婦女八人,花髻,服銷金桃紅衣,搖日月金鞚稍子鼓,舞唱同前。次男子五人,作五方菩薩梵像,搖日月鼓。次一人,作樂音王菩薩梵像,執花鞚稍子鼓,齊聲舞前曲一闋,樂止。次婦女三人,歌《新水令》、《沽美酒》、《太平令》之曲終,念口號畢,舞唱相和,以次而出。



 壽星隊:天壽節用之。引隊禮官樂工大樂冠服,並同樂音王隊。次二隊,婦女十人,冠唐巾,服銷金紫衣,銅束帶。次婦女一人,冠平天冠,服繡鶴氅,方心曲領,執圭,以次進至御前,立定,樂止,念致語畢,樂作,奏《長春柳》之曲。次三隊,男子三人,冠服舞蹈,並同樂音王隊。次四隊,男子一人,冠金漆弁冠,服緋袍,塗金帶,執笏;從者二人,錦帽,繡衣,執金字福祿牌。次五隊,男子一人,冠卷雲冠,青面具,綠袍,塗金帶,分執梅、竹、松、椿、石,同前隊而進,北向立。次六隊,男子五人,為烏鴉之像,作飛舞之態,進立於前隊之左,樂止。次七隊,樂工十有二人,冠雲頭冠,銷金緋袍,白裙,龍笛三,觱慄三,札鼓三,和鼓一,板一,與前大樂合奏《山荊子》帶《妖神急》之曲。次八隊,婦女二十人,冠鳳翹冠,翠花鈿,服寬袖衣,加雲肩、霞綬、玉佩,各執寶蓋,舞唱前曲。次九隊,婦女三十人,冠玉女冠,翠花鈿,服黃銷金寬袖衣,加雲肩、霞綬、玉佩,各執棕毛日月扇,舞唱前曲,與前隊相和。次十隊,婦女八人,服雜彩衣,被槲葉、魚鼓、簡子。次男子八人,冠束發冠,金掩心甲,銷金緋袍,執戟。次為龜鶴之像各一。次男子五人,冠黑紗帽,服繡鶴氅,硃履,策龍頭濆杖,齊舞唱前曲一闋,樂止。次婦女三人,歌《新水令》、《沽美酒》、《太平令》之曲終,念口號畢,舞唱相和,以次而出。



 禮樂隊:朝會用之。引隊禮官樂工大樂冠服,並同樂音王隊。次二隊,婦女十人,冠黑漆弁冠,服青素袍,方心曲領,白裙,束帶,執圭;次婦女一人,冠九龍冠,服繡紅袍,玉束帶,進至御前,立定,樂止,念致語畢,樂作,奏《長春柳》之曲。次三隊,男子三人,冠服舞蹈同樂音王隊。次四隊,男子三人,皆冠卷雲冠,服黃袍,塗金帶,執圭。次五隊,男子五人,皆冠三龍冠,服紅袍,各執劈正金斧,同前隊而進,北向立。次六隊,童子五人,三髻,素衣,各執香花,舞蹈而進,樂止。次七隊,樂工八人,皆冠束發冠,服錦衣白袍,龍笛三,觱慄三,杖鼓二,與前大樂合奏《新水令》、《水仙子》之曲。次八隊,婦女二十人,冠籠巾,服紫袍,金帶,執笏,歌《新水令》之曲,與樂聲相和,進至御前,分為四行,北向立,鞠躬拜,興,舞蹈,叩頭,山呼,就拜,再拜,畢,復趁聲歌《水仙子》之曲一闋,再歌《青山口》之曲,與後隊相和。次九隊,婦女二十人,冠車髻冠,服銷金藍衣,云肩,佩綬,執孔雀幢,舞唱與前隊相和。次十隊,婦女八人,冠翠花唐巾,服錦繡衣,執寶蓋,舞唱前曲。次男子八人,冠鳳翅兜牟,披金甲,執金戟。次男子一人,冠平天冠,服繡鶴氅,執圭,齊舞唱前曲一闋,樂止。次婦女三人,歌《新水令》、《沽美酒》、《太平令》之曲終,念口號畢,舞唱相和,以次而出。



 說法隊:引隊禮官樂工大樂冠服,並同樂音王隊。次二隊,婦女十人,冠僧伽帽,服紫禪衣,皁絳;次婦女一人,服錦袈裟,餘如前,持數珠,進至御前,北向立定,樂止,念致語畢,樂作,奏《長春柳》之曲。次三隊,男子三人,冠、服、舞蹈,並同樂音王隊。次四隊,男子一人,冠隱士冠,服白紗道袍,皁絳,執麈拂;從者二人,冠黃包巾,服錦繡衣,執令字旗。次五隊,男子五人,冠金冠,披金甲,錦袍,執戟,同前隊而進,北向立。次六隊,男子五人,為金翅雕之像,舞蹈而進,樂止。次七隊,樂工十有六人,冠五福冠,服錦繡衣,龍笛六,觱慄六,杖鼓四,與前大樂合奏《金字西番經》之曲。次八隊,婦女二十人,冠珠子菩薩冠,服銷金黃衣,纓絡,佩綬,執金浮屠白傘蓋,舞唱前曲,與樂聲相和,進至御前,分為五重,重四人,曲終,再起,與後隊相和。次九隊,婦女二十人,冠金翠菩薩冠,服銷金紅衣,執寶蓋,舞唱與前隊相和。次十隊,婦女八人,冠青螺髻冠,服白銷金衣,執金蓮花。次男子八人,披金甲,為八金剛像。次一人,為文殊像,執如意;一人為普賢像,執西番蓮花;一人為如來像;齊舞唱前曲一闋,樂止。次婦女三人,歌《新水令》、《沽美酒》、《太平令》之曲終,念口號畢,舞唱相和,以次而出。



 凡吉禮,郊祀、享太廟、告謚,見《祭祀志》。軍禮,見《兵志》。喪禮五服,見《刑法志》。水旱賑恤,見《食貨志》。內外導從,見《儀衛志》。



\end{pinyinscope}