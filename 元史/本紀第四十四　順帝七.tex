\article{本紀第四十四 順帝七}

\begin{pinyinscope}

 十五年春正月戊午朔,以中書平章政事搠思監提調留守司,宣徽使黑廝為中書平章政事,河南行省左丞許有壬為集賢大學士,遼陽行省左丞奇伯顏不花升本省平章政事。壬戌,以宣政院副使忻都為太子詹事。癸亥,享於太廟。甲子,親王禿堅帖木兒歿於軍中,賜鈔五百錠。江西行省平章政事道童加大司徒。戊辰,太陰犯五車。辛未,太陰犯鬼宿。大斡耳朵儒學教授鄭咺建言:「蒙古乃國家本族,宜教之以禮,而猶循本俗,不行三年之喪,又收繼庶母、叔嬸、兄嫂,恐貽笑後世,必宜改革,繩以禮法。」不報。丙子,上都饑,賑糶米二萬石。丁丑,徐壽輝偽將倪文俊復陷沔陽府,威順王寬徹普化令王子報恩奴等同湖南元帥阿思藍水陸並進討之。至漢川,水淺,文俊用火筏燒船,報恩奴遇害。庚辰,復設仁虞、雲需、尚供三總管府。丙戌,大同路饑,出糧一萬石減價糶之。是月,詔以湖廣行省平章政事乞剌班慢功,削其官爵,令從軍自效。詔安置脫脫於亦集乃路,收所賜田土。命河南行省參知政事洪丑驢守禦河南,陜西行省參知政事述律朵兒只守禦潼關,宗王扎牙失裡守禦興元,陜西行省參知政事阿魯溫沙守禦商州,通政院使朵來守禦山東。詔豫王阿剌忒納失里與陜西行省平章政事搠思監從宜商議軍事。



 閏月壬寅,以各衛軍人屯田京畿,人給鈔五錠,以是日入役,日支鈔二兩五錢,仍給牛、種、農器,命司農司令本管萬戶督其勤惰。丙午,太陰犯心宿。丙辰,太白經天。



 是月,上都路饑,詔嚴酒禁。命河南行省參知政事塔失帖木爾領元管陜西軍馬,守禦河南。二月己未,劉福通等自碭山夾河迎韓林兒至,立為皇帝,又號小明王,建都亳州,國號宋,改元龍鳳。以其母楊氏為皇太后,杜遵道、盛文鬱為丞相,羅文素、劉福通為平章,劉六知樞密院事;拆鹿邑縣太清宮材建宮闕,遵道等各遣子入侍。遵道得寵專權,劉福通疾之,命甲士撾殺遵道,福通遂為丞相,後稱太保。丙寅,以中書平章政事黑廝、左丞許有壬並知經筵事。戊辰,命太傅、御史大夫汪家奴為中書右丞相,中書平章政事定住為左丞相,詔天下。庚午,以河南行省平章政事咬咬為遼陽行省左丞相。壬申,立淮東等處宣慰使司都元帥府於天長縣,統濠、泗義兵萬戶府並洪澤等處義兵,聽富民願出丁壯義兵五千名者為萬戶,五百名者為千戶,一百名者為百戶,仍降宣敕牌面。丙子,以達識帖睦邇為中書平章政事,提調留守司;平章政事黑廝兼大司農。是月,命刑部尚書董銓等與江西行省平章政事火你赤專任征討之務,便宜從事,遣使先降曲赦,諭以禍福,如能出降,釋其本罪,執迷不悛,克日進討。三月庚寅,太陰犯五車。癸巳,徐壽輝兵陷襄陽路。甲午,命汪家奴攝太尉,持節授皇太子愛猷識理達臘玉冊,錫以冕服九旒,祗謁太廟。丙申,太陰犯房宿。辛丑,以監察御史言,安置脫脫於雲南鎮西路,也先帖木兒於四川碉門,脫脫長男哈剌章安置肅州,次男三寶奴安置蘭州,仍籍其家產。己酉,命知樞密院事眾家奴知經筵事,知樞密院事捏兀失該提調內史府。癸丑,太白經天。



 夏四月壬戌,中書省臣言:「江南因盜賊阻隔,所在闕官,宜遣人與各省及行臺官以廣東、廣西、海北、海南三品以下通行遷調,五品以下先行照會之任,江浙行省三年一次遷調,福建等處闕官亦依前例。」從之。命彰德等處分樞密院添設同知、副使、都事各一員。癸亥,以中書平章政事達識帖睦邇知經筵事,命樞密院添設僉院一員、判官二員,直沽分樞密院添設副使一員、都事一員。以御史中丞扎撒兀孫同知經筵事。乙丑,以中書右丞臧卜、左丞烏古孫良楨分省彰德。辛未,命御史中丞伯家奴同知經筵事,中書參議成遵兼經筵官。癸酉,以左丞相定住為右丞相,平章政事哈麻為左丞相,太子詹事桑哥失里為中書平章政事,雪雪為御史大夫。丁丑,加知樞密院事眾家奴太傅。辛巳,親王脫脫薨,賜鈔二百錠。是月,車駕時巡上都。詔翰林待制烏馬兒、集賢待制孫捴招安高郵張士誠,仍齎宣命、印信、牌面,與鎮南王孛羅不花及淮南行省、廉訪司等官商議給付之。御史臺劾奏中書左丞呂思誠,罷之。詔四川等處立宣化鎮南軍民府,改四川忠孝軍民府為忠孝軍民安撫司;罷盤順府,改立盤順軍民安撫司;罷四川羊毋甲洞、臭南王洞長官司,改立忠義軍民安撫司,立汴梁等處義兵萬戶府。



 五月壬辰,復襄陽路。監察御史也裏忽都等劾奏河南行省左丞相太不花慢功虐民,詔削其官職,仍令率領火赤溫,從總兵官、平章政事答失八都魯征進,答失八都魯管領太不花一應軍馬。庚戌,倪文俊自沔陽陷中興路,元帥朵兒只斑死之。是月,命淮南行省平章政事咬住、淮東廉訪使王也先迭兒撫諭高郵。六月丙辰,命御史大夫雪雪提調端本堂。癸亥,太白經天。丁卯,監察御史哈林禿劾奏脫脫之師集賢大學士吳直方及其參軍黑漢、長史火里赤等並宜追奪,從之。監察御史歪哥等辯明中書左丞呂思誠,給付元追所授宣命、玉帶。戊辰,命中書平章政事搠思監兼大司農,桑哥失裡知經筵事。己巳,靖安王闊不花薨,無後,命其侄襲封靖安王。癸酉,以四川行省平章政事答失八都魯為河南行省平章政事。乙亥,命將作院判官烏馬兒招安濠、泗等處,章佩監丞普顏帖木兒招安沔陽等處。諸王倒吾沒於軍中,賻鈔二百錠。丁丑,保德州地震。己卯,陜西行省平章政事禿禿加答剌罕。庚辰,徵徽州隱士鄭玉為翰林待制,不至。江浙省臣言:「至正十五年稅課等鈔,內除詔書已免稅糧等鈔,較之年例,海運糧並所支鈔不敷,乞減海運,以蘇民力。」戶部定擬本年稅糧,除免之外,其寺觀並撥賜田糧,十月開倉,盡行拘收;其不敷糧,撥至元折中統鈔一百五十萬錠,於產米處糴一百五十萬石,貯瀕河之倉,以聽撥運,從之。癸未,中書參知政事實理門言:「舊立蒙古國子監,專教四怯薛並各愛馬官員子弟,今宜諭之,依先例入學,俾嚴為訓誨。」從之。是月,大明皇帝起兵,自和州渡江,取太平路。自紅巾妖寇倡亂之後,南北郡縣多陷沒,故大明從而取之。荊州大水。命湖廣行省平章政事阿魯灰領軍,與淮南行省平章政事蠻子海牙、淮西道宣慰使完者不花以兵攻和州等處。命郡王只兒敢伯、湖廣行省右丞卜蘭奚攻討河南。以湖廣行省平章政事咬住為總兵官,領本省軍馬並江州楊完者、黃州李勝等軍,守禦湖廣。江浙行省參知政事納麟哈剌統領水軍萬戶等軍,會本省平章政事定定,進攻常州、鎮江等處。命將作院判官烏馬兒、利用監丞八十奴招諭濠、泗,淮南行省左丞相太平助之;章佩監丞普顏帖木兒、翰林修撰烈瞻招諭沔陽,四川行省平章政事玉樞虎兒吐華等助之。以怯薛丹潑皮等六十名從江南行御史臺大夫福壽守禦集慶路。國王朵兒只薨於揚州軍中,命郡王只兒敢伯管領其所部軍馬。



 秋七月辛卯,享於太廟。壬寅,倪文俊復陷武昌、漢陽等處。是月,命親王失里門以兵守曹州,山東宣慰馬某火者以兵分府沂州、莒州等處。命知樞密院事答兒麻監藏及四川行省左丞沙剌班、湖南同知宣慰使劉答兒麻失里,以兵屯中興,招諭諸處,有不降者,與親王禿魯及玉樞虎兒吐華討之。命湖廣行省平章政事桑哥、亦禿渾及禿禿守御襄陽,參知政事哈林禿及王塔失帖木爾守禦沔陽,如賊徒不降,即進兵討之。升臺州海道巡防千戶所為海道防禦運糧萬戶府。



 八月庚申,命南陽等處義兵萬戶府召募毛胡蘆義兵萬人,進攻南陽。戊辰,以中書平章政事達識帖睦邇為江浙行省左丞相,便宜行事,賜鈔一千錠。甲戌,以大宗正府扎魯忽赤迭裏迷失為甘肅行省平章政事。戊寅,太白經天。雲南死可伐等降,令其子莽三以方物來貢,乃立平緬宣撫司。四川向思勝降,以安定州改立安定軍民安撫司。是月,車駕還自上都。詔淮南行省左丞相太平統淮南諸軍討所陷郡邑,仍命湖廣行省平章政事阿魯灰以所部苗軍聽其節制。立吾者野人乞列迷等處諸軍萬戶府於哈兒分之地。命親王寬徹班守興元,永昌宣慰使完者帖木兒討西番賊。以淮南行省平章政事蠻子海牙與同知樞密院事絆住馬等,自蕪湖至鎮江南岸守御,同阿魯灰所部軍馬協力衛護江南行臺。命答失八都魯從便調度湖廣行省左丞卜蘭奚所領苗軍,江浙行省平章政事卜顏帖木兒守禦蘄、黃、蘭溪等處。九月癸未,命搠思監提調武衛。以知嶺北行樞密院事紐的該為中書平章政事。乙酉,立分海道防禦運糧萬戶府於平江路。己丑,太白犯太微垣。辛卯,命秘書卿答蘭提調別吉太后影堂祭祀,知樞密院事野仙帖木兒提調世祖影堂祭祀,宣政院使蠻子提調裕宗、英宗影堂祭祀。己亥,倪文俊圍岳州路。壬子,命桑哥失裡提調宣文閣,呂思誠知經筵事,集賢大學士許有壬兼太子諭德。是月,移置脫脫於阿輕乞之地,命答失八都魯移軍住陳留。



 冬十月丁巳,立淮南行樞密院於揚州。己未,太陰犯壘壁陣。甲子,命兵、工二部尚書撒八兒、王安童,以金銀牌一百六十五面,給淮東宣慰使司等處義兵官員。命哈麻領大司農司。帝謂右丞相定住等曰:「敬天地,尊祖宗,重事也。近年以來,闕於舉行,當選吉日,朕將親祀郊廟,務盡誠敬,不必繁文,卿等其議典禮,從其簡者行之。」遂命右丞斡欒、左丞呂思誠領其事。以中書右丞拜住為平章政事。庚午,以襲封衍聖公孔克堅同知太常禮儀院事,以克堅子希學為襲封衍聖公。癸酉,太陰犯軒轅。哈麻奏言:「郊祀之禮,以太祖配。皇帝出宮,至郊祀所,便服乘馬,不設內外儀仗、教坊隊子,齋戒七日,內散齋四日於別殿,致齋三日,二日於大明殿西幄殿,一日在南郊祀所。」丙子,以郊祀,命皇太子愛猷識理達臘祭告太廟。己卯,以翰林學士承旨慶童為淮南行省平章政事。立黃河水軍萬戶府於小清口。



 十一月甲申,熒惑犯氐宿。庚寅,填星犯井宿。壬辰,親祀上帝於南郊,以皇太子愛猷識理達臘為亞獻,攝太尉、右丞相定住為終獻。甲午,以太不花為湖廣行省左丞相,總兵招捕湖廣、沔陽等處,湖廣、荊襄諸軍悉聽節制,給還元追奪河南行省丞相宣命,仍給以功賞宣敕、金銀牌面。戊戌,介休縣桃杏花。己亥,太陰犯鬼宿。戊申,右丞相定住以病辭職,命以太保就第治病。庚戌,賊陷饒州路。辛亥,賜高麗國王伯顏帖木兒為親仁輔義宣忠奉國彰惠靖遠功臣。是月,答失八都魯攻夾河賊,大破之。賊陷懷慶,命河南行省右丞不花討之。以湖廣歸州改隸四川行省。



 十二月壬子朔,熒惑犯房宿。給湖廣行省分省印。丁巳,命中書參知政事月倫失不花、陳敬伯分省彰德。癸亥,立忠義、忠勤萬戶府於宿州、武安州。己巳,以諸郡軍儲供餉繁浩,命戶部印造明年鈔本六百萬錠給之。壬申,以平章政事帖裏帖木兒、右丞斡欒並知經筵事,參議丁好禮兼經筵官。乙亥,以天下兵起,下詔罪己,大赦天下。是月,答失八都魯大敗劉福通等於太康,遂圍亳州,偽宋主遁於安豐。立興元等處宣慰使司都元帥府於興元路。是歲,薊州雨血。詔:「凡有水田之處,設大兵農司,招集人夫,有警乘機進討,無事栽植播種。」詔浚大內河道,以宦官同知留守野先帖木兒董其役。野先帖木兒言:「自十一年以來,天下多事,不宜興作。」帝怒,命往使高麗,改命宦官答失蠻董之。以中書平章政事拜住分省濟寧,設四部。是歲,察罕帖木兒與賊戰於河南北,屢有功,除中書刑部侍郎。



 十六年春正月壬午,改福建宣慰使司都元帥府為福建行中書省。戊子,親享太廟。命中書平章政事帖裏帖木兒提調國子監。己丑,太陰犯昴宿。丁酉,太保定住以病辭職,太尉、太宗正府扎魯忽赤月闊察兒以出軍中傷辭職,皆不允。乙亥,詔命太尉阿吉剌開府設官屬。乙巳,以遼陽行省左丞相咬咬為太子詹事,翰林學士承旨朵列帖木兒同知詹事院事。丙子,以知樞密院事實理門兼大府監卿。戊申,雲南土官阿蘆降,遣侄腮斡以方物來貢。庚戌,左丞相哈麻罷。辛亥,御史大夫雪雪亦罷,以搠思監為御史大夫。復以定住為右丞相。是月,薊州地震。倪文俊建偽都於漢陽,迎徐壽輝據之。



 二月癸酉,禿魯帖木兒辭職,不允。搠思監糾言哈麻及其弟雪雪等罪惡,帝曰:「哈麻兄弟雖有罪,然侍朕日久,與朕弟懿璘質班皇帝實同乳,且緩其罰,令之出征自效。」甲寅,命右丞相定住依前太保,中書一切機務,悉聽總裁,詔天下。丙辰,以鎮南王孛羅不花自兵興以來率怯薛丹討賊,累立戰功,賜鈔一萬錠。定住及平章政事桑哥失裡等復奏哈麻兄弟罪惡,遂命貶哈麻惠州安置,雪雪肇州安置,尋杖殺之。壬戌,詹事伯撒里辭職。乙丑,禁銷毀、販賣銅錢。丙寅,命翰林國史院、太常禮儀院定擬皇后奇氏三代功臣謚號、王爵。甲戌,命六部、大司農司、集賢翰林國史兩院、太常禮儀院、秘書、崇文、國子、都水監、侍儀司等正官,各舉才堪守令者一人,不拘蒙古、色目、漢、南人,從中書省斟酌用之,或任內害民受贓者,舉官量事輕重降職。命蠻蠻為靖安王,賜金印,置王傅等官。己卯,命集賢直學士楊俊民致祭曲阜孔子廟,仍葺其廟宇。詔諭:「山東鹽法,軍民毋得沮壞。」賜定住篤憐赤、怯薛丹三十名,給衣糧、馬匹、草料。是月,高郵張士誠陷平江路,據之,改平江路為隆平府,遂陷湖州、松江、常州。



 月辛巳,復立酒課提舉司。命中書平章政事帖裏帖木兒、參知政事成遵等議鈔法。壬午,徐壽輝復寇襄陽。癸未,臺臣言:「系官牧馬草地,俱為權豪所占。今後除規運總管府見種外,餘盡取勘,令大司農召募耕墾,歲收租課以資國用。」從之。丁亥,以今秋出師,詔和買馬六萬匹。戊子,命宣讓王帖木兒不花、威順王寬徹普化以兵鎮遏懷慶路,各賜金一錠、銀五錠、幣帛九匹、鈔二千錠。庚寅,大明兵取集慶路,江南行臺御史大夫福壽死之。丙申,倪文俊陷常德路,總兵官俺都剌遁。命搠思監提調承徽寺。丁酉,立行樞密院於杭州。命江浙行省左丞相達識帖睦邇兼知行樞密院事,節制諸軍,省、院等官並聽調遣,凡賞功、罰罪、招降、討逆,許以便宜行事。大明兵取鎮江路。戊申,方國珍復降,以為海道運糧漕運萬戶,兼防御海道運糧萬戶。其兄方國璋為衢州路總管,兼防御海道事。是月,有兩日相蕩。



 夏四月辛亥,以搠思監為中書左丞相。丙辰,以資正院使普化為御史大夫。丁巳,命左丞相搠思監領經筵事,中書平章政事悟良哈臺、御史大夫普化並知經筵事。庚申,以河南行省左丞卜蘭奚為湖廣行省平章政事,答失八都魯加金紫光祿大夫。丙寅,命阿因班太子與陜西行省官同討均、房、南陽。遼陽行省平章政事奇伯顏不花加大司徒。丁卯,以陜西行臺御史大夫朵朵為陜西行省左丞相,大司農咬咬為遼陽行省左丞相。以知樞密院事實理門分院濟寧,翰林學士承旨脫脫同知詹事院事。壬申,命豫王阿剌忒納失里與陜西行省官商議軍機,從宜攻討。己卯,命悟良哈臺兼太子諭德。是月,車駕時巡上都。



 五月壬辰,太白犯鬼宿,癸巳,亦如之。甲午,太陰入斗宿。丙申,倪文俊陷澧州路。丁酉,太陰犯壘壁陣。乙巳,賊寇辰州,守將和尚以鄉兵擊敗之。六月甲寅,江浙行省平章政事三旦八、參知政事楊完者以兵守嘉興路,御張士誠。乙丑,大明兵取廣德路。



 秋七月癸未,以翰林學士禿魯帖木兒為侍御史。丁酉,太陰犯壘壁陣。是月,張士誠遣兵陷杭州,江浙行省平章政事左答納失里戰死,丞相達識帖木邇遁,楊完者及萬戶普賢奴擊敗之,復其城。



 八月丙辰,奉元路判官王淵等以義兵復商州,升淵同知關商襄鄧等處宣慰司事。己未,賊侵河南府路,參知政事洪丑驢以兵敗之。丁卯,太陰犯昴宿。庚午,倪文俊陷衡州路,元帥甄崇福戰死。甲戌,彗星見張宿,色青白,彗指西南,長尺餘,至十二月戊午始滅。是月,車駕還自上都。黃河決,山東大水。九月庚辰,汝、潁賊李武、崔德等破潼關,參知政事述律傑戰死。壬午,豫王阿剌忒納失里、同知樞密院事定住引兵復潼關,河南行省平章政事伯家奴以兵守之。丙申,潼關復陷,伯家奴兵潰,豫王阿剌忒納失里復以兵取之,李武、崔德敗走。戊戌,賊陷陜州及虢州。詔以太尉納麟復為江南行臺御史大夫,遷行臺治紹興。是月,察罕帖木兒復陜州及虢州,復襲敗賊兵於平陸、安邑,以功由兵部尚書升僉河北行樞密院事。



 冬十月丁未,大名路有星如火,從東南流,芒尾如曳篲,墮地有聲,火焰蓬勃,久之乃息,化為石,青黑色,光瑩,形如狗頭,其斷處如新割者,命藏於庫。壬辰,太陰犯井宿。是月,詔罷太尉也先帖木兒。



 十一月丙戌,以老的沙、答里麻失裡並為詹事。丁亥,流星大如酒杯,色青白,尾跡約長五尺餘,光明燭地,起自東北,東南行,沒於近濁,有聲如雷。壬辰,太陰犯井宿。



 是月,河南陷,河南廉訪副使俺普遁。置河南廉訪司於沂州,又於沂州設分樞密院,以兵馬指揮使司隸之。



 十二月,倪文俊陷岳州路,殺威順王子歹帖木兒。湖廣參知政事也先帖木兒與左江義兵萬戶鄧祖勝合兵復衡州。是歲,詔:「沿海州縣為賊所殘掠者,免田租三年。賜高年帛。」河南行省左丞相太不花駐軍於南陽嵩、汝等州,叛民皆降,軍勢大振。陜西行臺監察御史李尚絅上《關中形勝急論》,凡十有二事。命大司農司屯種雄、霸二州以給京師,號京糧。



\end{pinyinscope}