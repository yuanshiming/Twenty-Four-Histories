\article{志第二十七下 祭祀六}

\begin{pinyinscope}

 ○至正親祀南郊



 至正三年十月十七日,親祀昊天上帝於圜丘,以太祖皇帝配享,如舊行儀制。右丞相脫脫為亞獻官,太尉、樞密知院阿魯禿為終獻官,御史大夫伯撒里為攝司徒,樞密知院汪家奴為大禮使,中書平章也先帖木兒、鐵木兒達識二人為侍中,御史大夫也先帖木兒、中書右丞太平二人為門下侍郎,宣徽使達世帖睦爾、太常同知李好文二人為禮儀使,宣徽院使也先帖木兒執劈正斧,其餘侍祀官依等第定擬。



 前期八月初七日,太常禮儀院移關禮部,具呈都省,會集翰林、集賢、禮部等官,講究典禮。九月內,承奉班都知孫玉鉉具錄《親祀南郊儀注》云:致齋日停奏刑殺文字,應享執事官員蒞誓於中書省。享前一日質明,所司備法駕儀仗暨侍享官分左右敘立於崇天門外,太僕卿控御馬立於大明門外,侍儀官、導駕官各具公服,備擎執,立於致齋殿前。通事舍人二員引門下侍郎、侍中入殿相向立。侍中跪奏請皇帝中嚴,就拜興,退出。少頃,引侍中跪奏外辦,就拜興。皇帝出致齋殿,侍中跪奏請皇帝升輿,侍儀官、導駕官引擎執前導,巡輦路至大明殿西陛下。侍中跪奏請皇帝降輿升殿,就拜興。皇帝入殿,即御座。舍人引執事等官,敘於殿午陛下,相向立。通班舍人贊起居,引班鞠躬平身。舍人引門下侍郎、侍中入殿至御座前,門下侍郎、侍中相向立。侍中跪奏請皇帝降殿升輿,就拜興。侍儀官前導,至大明殿門外,侍中跪奏請皇帝升輿,就拜興。至大明門外,侍中跪奏請皇帝降輿乘馬,門下侍郎跪奏請車駕進發,就拜興,動稱警蹕。至崇天門外,門下侍郎跪奏請車駕少駐,敕眾官上馬,就拜興。侍中承旨,退稱曰「制可」,門下侍郎退傳制,敕眾官上馬,贊者承傳,敕眾官於欞星門外上馬。少頃,門下侍郎跪奏請車駕進發,就拜興,動稱警蹕。華蓋傘扇儀仗百官左右前導,教坊樂鼓吹不作。至郊壇南欞星門外,門下侍郎跪奏請皇帝權停,敕眾官下馬。侍中傳制,敕眾官下馬,自卑而尊與儀仗倒卷而北,左右駐立。駕至內欞星門,侍中跪奏請皇帝降馬,步入欞星門,由右偏門入。稍西,侍中跪奏請皇帝升輿,就拜興。侍儀官暨導駕官引擎執前導,至大次殿門前,侍中跪奏請皇帝降輿,入就大次殿,就拜興。皇帝入就大次,簾降,宿衛如式。侍中入跪奏,敕眾官各退齋次,就拜興。通事舍人承旨,敕眾官各還齋次。尚食進膳訖,禮儀使以祝冊奏御署訖,奉出,郊祀令受而奠於坫。



 其享日醜時二刻,侍儀官備擎執,同導駕官列於大次殿前。通事舍人引侍中、門下侍郎入大次殿。侍中跪奏請皇帝中嚴,服袞冕,就拜興,退。少頃,舍人再拜引侍中跪版奏外辦,就拜興,退出。禮儀使入跪奏皇帝行禮,就拜興。簾卷出大次,侍儀官備擎執,同導駕官前導。皇帝至西壝門,侍儀官、導駕官擎執止於壝門外,近侍官、代禮官皆後從入。殿中監跪進大圭,禮儀使跪請皇帝執大圭,皇帝入行禮,禮節一如舊制。行禮畢,侍儀官備擎執,同導駕官前導,皇帝還至大次。通事舍人引侍中入跪奏,請皇帝解嚴,釋袞冕。停五刻頃,尚食進膳如儀。所司備法駕儀仗,同侍享等官分左右,敘立於郊南欞星門外,以北為上。舍人引侍中入跪奏,請皇帝中嚴,就拜興,退。少頃,再引侍中跪版奏外辦,就拜興。皇帝出大次,侍中跪奏請皇帝升輿,侍儀官備擎執,同導駕官前導,至欞星門外,太僕卿進御馬,侍中跪奏請皇帝降輿乘馬,就拜興。門下侍郎跪奏請車駕進發,就拜興,動稱警蹕。至欞星門外,門下侍郎跪請皇帝少駐,敕眾官上馬,就拜興。侍中承旨退稱曰「制可」,門下侍郎傳制,敕眾官上馬,贊者承傳,敕眾官上馬。少頃,門下侍郎跪奏請車駕進發,就拜興。侍儀官備擎執,同導駕官前導,動稱警蹕,華蓋儀仗傘扇眾官左右前導,教坊樂鼓吹皆作。至麗正門裏石橋北,舍人引門下侍郎下馬,跪奏請皇帝權停,敕眾官下馬,贊者承傳,敕眾官下馬,舍人引眾官分左右,先入紅門內,倒卷而北駐立。引甲馬軍士於麗正門內石橋大北駐立,依次倒卷至欞星門外,左右相向立。仗立於欞星門內,倒卷亦如之。門下侍郎跪奏請車駕進發,侍儀官備擎執,導駕官導由崇天門入,至大明門外。引侍中跪奏請皇帝降馬升輿,就拜興。至大明殿,引眾官相向立於殿陛下。俟皇帝入殿升座,侍中跪奏請皇帝解嚴,敕眾官皆退,通事舍人承旨敕眾官皆退,郊祀禮成。



 至正親祀太廟



 至元六年六月,監察御史呈:「嘗聞《五行傳曰》,簡宗廟,廢祭祀,則水不潤下。近年雨澤愆期,四方多旱,而歲減祀事,變更成憲,原其所致,恐有感召。欽惟國家四海乂安,百有餘年,列聖相承,典禮具備,莫不以孝治天下。古者宗廟四時之祭,皆天子親享,莫敢使有司攝也。蓋天子之職,莫大於禮,禮莫大於孝,孝莫大於祭。世祖皇帝自新都城,首建太廟,可謂知所本矣。《春秋》之法,國君即位,逾年改元,必行告廟之禮。伏自陛下即位以來,於今七年,未嘗躬詣太廟,似為闕典。方今政化更新,並遵舊制,告廟之典,理宜新享。」時帝在上都,臺臣以聞,奉旨若曰:「俟到大都,親自祭也。」



 九月二十七日,中書省奏以十月初四日皇帝親祀太廟,制曰:「可。」前期,告示以太師、右丞相馬扎兒臺為亞獻官,樞密知院阿魯禿為終獻官,知院潑皮、翰林承旨老章為助奠官,大司農愛牙赤為七祀獻官,侍中二人,門下侍郎二人,大禮使一人,執劈正斧一人,禮儀使四人,餘各如故事。有司具儀注云:享前一日質明,所司備法駕於崇天門外,侍儀官引擎執,同導駕官具公服,於致齋殿前左右分班侍立。承奉舍人引門下侍郎、侍中人殿門下,侍郎相向立,侍中跪奏臣某等官請皇帝中嚴,就拜興,退出。少頃,引侍中版奏外辦,就拜興,退。皇帝出齋室,侍中跪奏請皇帝升輿,巡輦路,由正門至大明殿酉陛下。侍中跪奏請皇帝降輿升殿,就拜興,引皇帝即御座。執事官於午陛下起居訖,舍人引侍中、門下侍郎入殿,至御榻前,門下侍郎相向立。侍中跪奏請皇帝降殿升輿,就拜興,導至大明殿外。侍中跪奏請皇帝升輿,就拜興。至大明門外,太僕卿進御馬。侍中跪奏請皇帝降輿乘馬訖,門下侍郎跪奏請車駕進發,就拜興,進發時稱警蹕。至崇天門外,門下侍郎跪奏請車駕少駐,敕眾官上馬,就拜興。侍中承旨退稱曰「制可」,贊者承傳,敕眾官上馬。少頃,門下侍郎跪奏請車駕進發,就拜興,進發時稱警蹕。導至太廟外紅門內,門下侍郎跪奏請車駕權停,敕眾官下馬,就拜興。贊者承傳,敕眾官下馬。門下侍郎跪奏請車駕進發,至石橋南,侍中跪奏請皇帝下馬,步入神門,就拜興。皇帝下馬,侍儀官同導駕官前導,皇帝步入神門稍西,侍中跪奏請皇帝升輿,就拜興。至大次殿門前,侍中跪奏請皇帝降輿,入就大次,就拜興。簾降,宿衛如式。侍中入跪奏,敕眾官各還齋次,承旨贊者承傳,敕眾官各還齋次,俟行禮時至丑時二刻頃,侍儀官備擎執,同導駕官於大次殿門前,舍人引侍中、門下侍郎入大次座前,侍中跪奏請皇帝中嚴,服袞冕,就拜興,退。少頃,再引侍中跪奏外辦,就拜興,退。禮儀使跪奏請皇帝行禮,侍儀官同導駕官導引皇帝至西神門,擎執侍儀官同導駕官止。行禮畢,皇帝由西神門出,侍儀官備擎執,同導駕官引導皇帝還至大次。舍人引侍中入跪奏,請皇帝解嚴,釋袞冕。尚食進膳如式畢,侍中跪版奏外辦,就拜興,退。導皇帝出大次,侍中跪奏請皇帝升輿,就拜興。侍儀官同導駕官前導,至神門外,太僕卿進御馬,侍中跪奏請皇帝降輿乘馬,就拜興。乘馬訖,門下侍郎跪奏請車駕進發,就拜興,退,進發時稱警蹕。至欞星門外,門下侍郎跪奏請車駕少駐,敕眾官上馬,就拜興。侍中承旨退稱曰「制可」,贊者承傳,敕眾官上馬。少頃,門下侍郎跪奏請車駕進發,就拜興,進發時稱警蹕,教坊樂振作。至麗正門裏石橋北,引門下侍郎跪奏請車駕權停,敕眾官下馬,就拜興。贊者承傳,敕眾官下馬。門下侍郎跪奏請車駕進發,侍儀官引擎執,同導駕官前導,執事官後從,皇帝由紅門裡輦路至大明門外。侍中跪奏請皇帝降馬乘輿,就拜興。侍儀官擎執,同導駕官導至大明殿,諸執事殿下相向立。俟皇帝入殿升座,侍中跪奏敕眾官皆退,贊者承傳,敕眾官皆退。



 三皇廟祭祀禮樂



 至正九年,御史臺以江西湖東道肅政廉訪使文殊訥所言具呈中書。其言曰:「三皇開天立極,功被萬世。京師每歲春秋祀事,命太醫官主祭,揆禮未稱。請如國子學、宣聖廟春秋釋奠,上遣中書省臣代祀,一切儀禮仿其制。」中書付禮部集禮官議之。是年十月二十四日,平章政事太不花、定住等以聞,制曰「可」。於是命太常定儀式,工部範祭器,江浙行省制雅樂器。復命太常博士定樂曲名,翰林國史院撰樂章十有六曲。明年,祭器、樂器俱備,以醫籍百四十有八戶充廟戶禮樂生。御藥院大使盧亨素習音律,受命教樂工四十有二人,各執其技,乃季秋九月九日蕆事。宣徽供禮饌,光祿勛供內醖,太府供金帛,廣源庫供薌炬,大興府尹供犧牲、制幣、粢盛、肴核。中書奏擬三獻官以次定,諸執事並以清望充。前一日,內降御香,三獻官以下公服備大樂儀仗迎香,至開天殿庋置。退習明日祭儀,習畢就廟齋宿。京朝文武百司與祭官如之,各以禮助祭。翰林詞臣具祝文,曰「皇帝敬遣某官某致祭」。



 樂章前卷祀社稷樂章,俱在禮樂類中,今附於此。



 降神,奏《咸成之曲》:



 黃鐘宮三成



 於皇三聖,神化無方。繼天立極,垂憲百王。聿崇明祀,率由舊章。靈兮來下,休有烈光。



 降神,奏《賓成之曲》:



 大呂角二成



 帝德在人,日用不知。神之在天,矧可度思。辰良日吉,蕆事有儀。感以至誠,尚右享之。



 降神,奏《顧成之曲》:



 太簇徵二成



 大道之行,肇自古先。功烈所加,何千萬年。是尊是奉,執事孔虔。神哉沛兮,泠風馺然。



 降神,奏《臨成之曲》:



 應鐘羽二成



 雅奏告成,神斯降格。妥安有位,清廟奕奕。肸蚃潛通,豐融烜赫。我其承之,百世無斁。



 初獻盥洗,奏《蠲成之曲》:



 姑洗宮



 靈斿戾止,式燕以寧。吉蠲致享,惟寅惟清。挹彼注茲,沃盥而升。有孚顒若,交於神明。



 初獻升殿,奏《恭成之曲》:



 南呂宮



 齋明盛服,恪恭命祀。洋洋在上,匪遠具邇。左右周旋,陟降庭止。式禮莫愆,用介多祉。



 奠幣,奏《祗成之曲》:



 南呂宮



 駿奔在列,品物咸備。禮嚴載見,式陳量幣。惟茲篚實,肅將忱意。靈兮安留,成我熙事。



 初獻降殿。與升殿同。



 捧俎,奏《闕成之曲》:



 姑洗宮



 我祀如何,有牲在滌。既全且潔,為俎孔碩。以將以享,其儀不忒。神其迪嘗,純嘏是錫。



 初獻盥洗。與前同。



 初獻升殿。與前同。



 大皞伏羲氏位酌獻,奏《闕成之曲》:



 南呂宮



 五德之首,巍巍聖神。八卦有作,誕開我人。物無能稱,玄酒在尊。歆監在茲,惟德是親。



 炎帝神農氏位酌獻,奏《闕成之曲》:



 南呂宮



 耒耜之利,人賴以生。鼓腹含哺,帝力難名。欲報之德,黍稷非馨。眷言顧之,享於克誠。



 黃帝有熊氏位酌獻,奏《闕成之曲》:



 南呂宮



 為衣為裳,法乾效坤。三辰順序,萬國來賓。典祀有常,多儀具陳。純精鬯達,匪藉彌文。



 配位酌獻,奏《闕成之曲》



 南呂宮



 三聖儼臨,孰侑其食。惟爾有神,同功合德。丕擁靈休,留娛嘉席。歷世昭配,永永無極。



 初獻降殿。與前同。



 亞獻,奏《闕成之曲》:終獻同。



 姑洗宮



 緩節安歌,載升貳觴。禮成三終,申薦令芳。凡百有職,罔敢怠遑。神具醉止,欣欣樂康。



 徹豆,奏《闕成之曲》:



 南呂宮



 籩豆有踐,殷薦亶時。禮文疏洽,廢徹不遲。慎終如始,進退無違。神其祚我,綏以繁。



 送神,奏《闕成之曲》:



 黃鐘宮



 夜如何其,明星煌煌。靈逝弗留,飆舉雲翔。瞻望靡及,德音不忘。庶回景貺,發為禎祥。



 望瘞,奏《闕成之曲》:



 姑洗宮



 工祝致告,禮備樂終。加牲兼幣,訖珣愈恭。精神斯罄,惠澤無窮。儲休錫美,萬福來崇。



 顏子考妣封謚



 至順元年冬十一月望,曲阜兗國復聖公新廟落成。元統二年,改封顏子考曲阜侯為杞國公,謚文裕;妣齊姜氏為杞國夫人,謚端獻;夫人戴氏兗國夫人,謚貞素。又割益都鄒縣牧地三十頃,徵其歲入,以給常祀。



 宋五賢從祀



 至正十九年十一月,江浙行省據杭州路申備本路經歷司呈,準提控案牘兼照磨承發架閣胡瑜牒,嘗謂:



 文治興隆,宜舉行於曠典;儒先褒美,期激勵於將來。凡在聞知,詎容緘默。蓋國家化民成俗,莫先於學校;而學校之設,必崇先聖先師之祀者,所以報功而示勸也。我朝崇儒重道之意,度越前古。既已加封先聖大成之號,又追崇宋儒周敦頤等封爵,俾從祀廟庭,報功示勸之道,可謂至矣。然有司討論未盡,尚遺先儒楊時等五人,未列從祀,遂使盛明之世,猶有闕典。惟故宋龍圖閣直學士、謚文靖、龜山先生楊時,親得程門道統之傳,排王氏經義之謬,南渡後,硃、張、呂氏之學,其源委脈絡,皆出於時者也。故宋處士、延平先生李侗,傅河洛之學,以授硃熹,凡《集注》所引師說,即其講論之旨也。故宋中書舍人、謚文定胡安國,聞道伊洛,志在《春秋》,纂為《集傳》,羽翼正經,明天理而扶世教,有功於聖人之門者也。故宋處士、贈太師榮國公、謚文正、九峰先生蔡沈,從學硃子,親承指授,著《書集傳》,發明先儒之所未及,深有功於聖經者也。故宋翰林學士、參知政事、謚文忠、西山先生真德秀,博學窮經,踐履篤實。當時立偽學之禁,以錮善類,德秀晚出,獨以斯文為己任,講習躬行,黨禁解而正學明。此五人者,學問接道統之傳,著述發儒先之秘,其功甚大。況科舉取士,已將胡安國《春秋》、蔡沈《尚書集傳》表章而尊用之,真德秀《大學衍義》亦備經筵講讀,是皆有補於國家之治道者矣。各人出處,詳見《宋史》本傳,俱應追錫名爵,從祀先聖廟廷,可以敦厚儒風,激勸後學。如蒙備呈上司,申達朝省,命禮官討論典禮,如周敦頤等例,聞奏施行,以補闕典,吾道幸甚。



 本省以其言具咨中書省,仍遣胡瑜赴都投呈。至正二十一年七月,中書判送禮部,行移翰林、集賢、太常三院會議,俱準所言,回呈中書省。二十二年八月,奏準送禮部定擬五先生封爵謚號。俱贈太師。楊時追封吳國公,李侗追封越國公,胡安國追封楚國公,蔡沈追封建國公,真德秀追封福國公。各給詞頭宣命,遣官齎往福建行省,訪問各人子孫給付。如無子孫者,於其故所居鄉里郡縣學,或書院祠堂內,安置施行。



 硃熹加封齊國父追謚獻靖



 至正二十二年十二月,追謚硃熹父為獻靖,其制詞云:「考德而論時,灼見風儀之俊;觀子而知父,迨聞《詩》、《禮》之傳。久閟幽堂,丕昭公論。故宋左承議郎、守尚書吏部員外郎、兼史館校勘、累贈通議大夫硃松,仕不躁進,德合中行。溯鄒魯之淵源,式開來學;開圖書之蘊奧,妙契玄機。奏對雖忤於權奸,嗣續篤生於賢哲。化民成俗,著書滿家。既繼志述事之光前,何節惠易名之孔後。才高弗展,嗟沉滯於下僚;道大莫容,竟昌明於永世。神靈不昧,休命其承。可謚獻靖。」



 其改封熹為齊國公,制詞云:「聖賢之蘊載諸經,義理實明於先正;風節之厲垂諸世,褒崇豈間於異時。不有巨儒,孰膺寵數?故宋華文閣待制、累贈寶謨閣直學士、太師、追封徽國公、謚文硃熹,挺生異質,蚤擢科名。試用於郡縣,而善政孔多;回翔於館閣,而直言無隱。權奸屢挫,志慮不回。著書立言,嘉乃簡編之富;愛君憂國,負其經濟之長。正學久達於中原,渙號申行於仁廟。詢諸僉議,宜易故封。國啟營丘,爰錫太公之境土;壤鄰洙泗,尚觀尼父之宮墻。緬想英風,載欽親命。可追封齊國公,餘並如故。」



 國俗舊禮



 每歲,太廟四祭,用司禋監官一員,名蒙古巫祝。當省牲時,法服,同三獻官升殿,詣室戶告腯,還至牲所,以國語呼累朝帝後名諱而告之。明旦,三獻禮畢,獻官、御史、太常卿、博士復升殿,分詣各室,蒙古博兒赤跪割牲,太僕卿以硃漆盂奉馬乳酌奠,巫祝以國語告神訖,太祝奉祝幣詣燎位,獻官以下復版位載拜,禮畢。



 每歲,駕幸上都,以八月二十四日祭祀,謂之灑馬妳子。用馬一,羯羊八,彩段練絹各九匹,以白羊毛纏若穗者九,貂鼠皮三,命蒙古巫覡及蒙古、漢人秀才達官四員領其事,再拜告天,又呼太祖成吉思御名而祝之,曰:「托天皇帝福廕,年年祭賽者。」禮畢,掌祭官四員,各以祭幣表裏一與之;餘幣及祭物,則凡與祭者共分之。



 每歲,九月內及十二月十六日以後,於燒飯院中,用馬一,羊三,馬湩,酒醴,紅織金幣及里絹各三匹,命蒙古達官一員,偕蒙古巫覡,掘地為坎以燎肉,仍以酒醴、馬湩雜燒之。巫覡以國語呼累朝御名而祭焉。



 每歲,十二月下旬,擇日,於西鎮國寺內墻下,灑掃平地,太府監供彩幣,中尚監供細氈針線,武備寺供弓箭環刀,束稈草為人形一,為狗一,剪雜色彩段為之腸胃,選達官世家之貴重者交射之。非別速、札剌爾、乃蠻、忙古、臺列班、塔達、珊竹、雪泥等氏族,不得與列。射至糜爛,以羊酒祭之。祭畢,帝後及太子嬪妃並射者,各解所服衣,俾蒙古巫覡祝贊之。祝贊畢,遂以與之,名曰脫災。國俗謂之射草狗。



 每歲,十二月十六日以後,選日,用白黑羊毛為線,帝後及太子,自頂至手足,皆用羊毛線纏系之,坐於寢殿。蒙古巫覡念咒語,奉銀槽貯火,置米糠於其中,沃以酥油,以其煙薰帝之身,斷所系毛線,納諸槽內。又以紅帛長數寸,帝手裂碎之,唾之者三,並投火中。即解所服衣帽付巫覡,謂之脫舊災、迎新福云。



 凡后妃妊身,將及月辰,則移居於外氈帳房。若生皇子孫,則錫百官以金銀彩段,謂之撒答海。及彌月,復還內寢。其帳房則以頒賜近臣云。



 凡帝後有疾危殆,度不可愈,亦移居外氈帳房。有不諱,則就殯殮其中。葬後,每日用羊二次燒飯以為祭,至四十九日而後已。其帳房亦以賜近臣云。



 凡宮車晏駕,棺用香楠木,中分為二,刳肖人形,其廣狹長短,僅足容身而已。殮用貂皮襖、皮帽,其靴襪、系腰、盒缽,俱用白粉皮為之。殉以金壺瓶二,盞一,碗碟匙箸各一。殮訖,用黃金為箍四條以束之。輿車用白氈青緣納失失為簾,覆棺亦以納失失為之。前行,用蒙古巫媼一人,衣新衣,騎馬,牽馬一匹,以黃金飾鞍轡,籠以納失失,謂之金靈馬。日三次,用羊奠祭。至所葬陵地,其開穴所起之土成塊,依次排列之。棺既下,復依次掩覆之。其有剩土,則遠置他所,送葬官三員,居五里外。日一次燒飯致祭,三年然後返。



 世祖至元七年,以帝師八思巴之言,於大明殿御座上置白傘蓋一,頂用素段,泥金書梵字於其上,謂鎮伏邪魔獲安國剎。自後每歲二月十五日,於大明殿啟建白傘蓋佛事,用諸色儀仗社直,迎引傘蓋,周游皇城內外,雲與眾生祓除不祥,導迎福祉。歲正月十五日,宣政院同中書省奏,請先期中書奉旨移文樞密院,八衛撥傘鼓手一百二十人,殿後軍甲馬五百人,抬舁監壇漢關羽神轎軍及雜用五百人。宣政院所轄官寺三百六十所,掌供應佛像、壇面、幢幡、寶蓋、車鼓、頭旗三百六十壇,每壇擎執抬舁二十六人,鈸鼓僧一十二人。大都路掌供各色金門大社一百二十隊,教坊司云和署掌大樂鼓、板杖鼓、篳篥、龍笛、琵琶、箏、緌七色,凡四百人。興和署掌妓女雜扮隊戲一百五十人,祥和署掌雜把戲男女一百五十人,儀鳳司掌漢人、回回、河西三色細樂,每色各三隊,凡三百二十四人。凡執役者,皆官給鎧甲袍服器仗,俱以鮮麗整齊為尚,珠玉金繡,裝束奇巧,首尾排列三十餘里。都城士女,閭閻聚觀。禮部官點視諸色隊仗,刑部官巡綽喧鬧,樞密院官分守城門,而中書省官一員總督視之。先二日,於西鎮國寺迎太子游四門,舁高塑像,具儀仗入城。十四日,帝師率梵僧五百人,於大明殿內建佛事。至十五日,恭請傘蓋於御座,奉置寶輿,諸儀衛隊仗列於殿前,諸色社直暨諸壇面列於崇天門外,迎引出宮。至慶壽寺,具素食,食罷起行,從西宮門外垣海子南岸,入厚載紅門,由東華門過延春門而西。帝及后妃公主,於玉德殿門外,搭金脊吾殿彩樓而觀覽焉。及諸隊仗社直送金傘還宮,復恭置御榻上。帝師僧眾作佛事,至十六日罷散。歲以為常,謂之游皇城。或有因事而輟,尋復舉行。夏六月中,上京亦如之。



\end{pinyinscope}