\article{志第三十九 百官五}

\begin{pinyinscope}

 儲政院,秩正二品。至元十九年,立詹事院,備左右輔翼皇太子之任,置左、右詹事各一員的天人合一之論。闡發子思之學,繼承孔子之說,成思孟學,副詹事、詹事丞、院判各二員,吏屬六十有二人,別置宮臣賓客二員,左右諭德、左右贊善各一員,校書郎二員,中庶子、中允各一員。三十一年,太子裕宗既薨,乃以院之錢糧選法工役,悉歸太后位下,改為徽政院以掌之。大德九年,復立詹事院,尋罷。十一年,更置詹事院,秩從一品,設官十二員。至大四年罷。延祐四年復立,七年罷。泰定元年,罷徽政院,改立詹事如前。天歷元年,改詹事院為儲慶使司。二年罷,復立詹事院。未幾,改儲政院,院使六員,正二品;同知二員,正三品;僉院二員,從三品;同僉二員,正四品;院判二員,正五品;司議二員,從五品;長史二員,正六品;照磨二員,管勾二員,俱正八品;掾史一十二人,譯史四人,回回掾史二人,通事、知印各二人,宣使十人,典吏六人。其屬附見:



 家令司,秩三品,家令、家丞各二員,典簿二員,照磨一員,掌太子飲膳供帳倉庫。至元二十年置。三十一年,改內宰司,隸徽政。大德十一年復立,秩升從二。至大四年罷。延祐四年復立,秩正三品。七年罷。泰定元年,復以內宰司為家令司。天歷元年罷,未幾復立。二年又罷。



 典幄署,掌太子供帳,令、丞各二員,書史、書吏各二人。



 府正司,秩從三品,掌鞍轡弓矢等物。至元二十年置。府正、府丞各二員,典簿二員,照磨一員。三十一年,改官正司。大德十一年,復為府正司。至大四年罷。延祐四年復立,七年罷。泰定元年復立。天歷二年,增府正、府丞各二員,尋罷。



 資武庫,掌軍器,提點一員,大使一員。



 驥用庫,掌鞍轡,提點一員,大使一員。



 延慶司,秩正三品,掌修建佛事,使二員,同知一員,副使、典簿各二員,照磨一員。至元二十一年始立,隸詹事院。三十一年,隸徽政院。大德十一年,立詹事院,別立延慶司,秩仍正三品,置卿、丞等員。泰定元年,改隸詹事院。天歷元年罷,二年復立,增丞二員。



 典用監,卿四員,太監二員,少監二員,丞二員,經歷、知事各一員,照磨一員,掌供須、文成、藏珍三庫,內府供給段匹寶貨等物。至大元年立。天歷二年,設官如故,以三庫隸內宰司。



 典醫監,秩正三品,領東宮太醫,修合供進藥餌。至元十九年,置典醫署,秩從五品。三十一年,改掌醫署,尋罷。大德十一年,復立典醫監。至大四年罷,泰定四年,復立署。天歷二年,改典醫監,秩正三品。置達魯花赤二員,卿三員,太監二員,少監二員,丞二員,經歷、知事各一員,吏屬凡十八人。其屬司一、局二。



 廣濟提舉司,達魯花赤一員,提舉,同提舉、副提舉各一員,掌修合藥餌,以施貧民。



 行典藥局,達魯花赤一員,大使、副使各二員,掌供奉東宮藥餌。



 典藥局,達魯花赤一員,大使、副使、直長各二員,掌修制東宮藥餌。



 典牧監,秩正三品,卿二員,太監二員,少監二員,丞二員,經歷、知事各一員,照磨一員,吏屬凡十六人,掌孳畜之事。天歷二年始置。



 儲膳司,秩正三品,卿四員,少卿二員,丞二員,主事二員,照磨一員,令史六人,譯史、通事、知印各二人,奏差六人,典吏四人,掌皇太子飲膳之事。天歷二年立。



 典寶監,秩正三品,卿、太監、少監、丞各二員,經歷、知事各一員,吏屬八人。至元十九年,立典寶署,從五品。二十年,升正五品。三十一年罷。大德十一年立監,秩正三品。至大四年罷。延祐四年復立,七年罷。泰定元年復置。天歷元年罷,二年復置。



 以上俱系詹事院司屬。



 掌謁司,秩正三品,司卿四員,少卿四員,丞二員,典簿二員,典書九人,奏差二人,知印、譯史、通事各一人。至元三十一年,改典寶署為掌謁司,秩從五品,設官如之。元貞元年,升四品,設官四員。大德十一年,升正三品。至治三年罷。



 甄用監,秩正三品,卿三員,太監、少監、丞各二員,經歷、知事、照磨各一員,掌供須、文成、藏珍三庫出納之事。至大元年設,至治三年罷。



 延福司,秩正三品,令、丞各四員,典簿二員,照磨一員,掌供帳及扈從蓋造之人。大德十一年置,後並入群牧監。



 章慶使司,秩正三品,司使四員,同知、副使、司丞各二員,經歷、都事各二員,照磨、管勾各一員。至大三年立,至治三年罷。



 奉徽庫,秩從五品,提點、大使各二員,副使四員,庫子六人,掌內府供給。至治三年罷,並入文成等庫。



 壽和署,秩正五品,署令四員,署丞六員。舊隸儀鳳司,皇慶元年,改隸徽政院,遂為章慶使司之屬。至治二年罷。



 上都掌設署,秩正五品,署令五員,署丞二員。至大四年立,至治三年罷。



 掌醫監,秩正五品,領監官一員、達魯花赤一員、卿四員、太卿五員、太監五員、少監六員、丞二員。至元三十一年,改典醫為掌醫署,秩五品。至大元年升監,設已上官員。至治三年罷。



 修合司藥正司,秩從五品,達魯花赤一員,副使、直長各二員,掌藥六人,掌修合御用藥餌。至治三年罷。



 行篋司藥局,秩從五品,達魯花赤一員,使、副使各二員,掌供奉御用藥餌。至治三年罷。



 廣濟提舉司,秩從五品,達魯花赤、提舉、同提舉、副提舉各一員,掌修合藥餌,以濟貧民。



 群牧監,秩正二品,掌中宮位下孳畜,卿三員,太卿、少卿、監丞各二員。至大四年立,至治三年罷。



 掌儀署,秩正五品,令、丞各二員,掌戶口房舍等。至元二十年立,隸詹事院。三十一年,改隸徽政院。泰定元年,改典設署。



 上都掌儀署,秩五品,令、丞各一員,掌戶口房舍等。大德十一年立,至治三年罷。



 江西財賦提舉司,秩從五品,達魯花赤一員,提舉、同提舉、副提舉各一員,掌事產戶口錢糧造作等事。至元二十七年立,至治二年罷。



 織染局,局使、副使、局副各一員,相副官一員。



 桑落娥眉洲管民提領所,提領、副提領各一員。



 封郭等洲管民提領所,提領、同提領、副提領各一員。



 龍興打捕提領所,提領、副提領各一員。



 鄂州等處民戶水陸事產提舉司,達魯花赤一員,提舉、同提舉、副提舉各一員,掌太子位下江南園囿地土莊宅人戶。至元二十一年立,隸詹事,後改隸徽政。至治三年罷。



 瑞州上高縣戶計長官司,秩從五品,達魯花赤一員,長官、副長官各一員,領本處戶八千。後隸徽政院,至治三年罷。



 以上俱系徽政院司屬。



 左都威衛使司,秩正三品,使三員,副使二員,僉事二員,經歷、知事、照磨各一員。至元十六年,以侍衛親軍一萬戶撥屬東宮,立侍衛都指揮使司。三十一年,改隆福宮左都威衛使司,隸中宮。至大三年,選造作軍士八百人,立千戶所一、百戶翼八以領之,而分局造作。延祐二年,置教授二。至治三年,罷軍匠千戶所。



 鎮撫所,鎮撫二員,都目一員。



 行軍千戶所,千戶二員,副千戶二員,知事、彈壓各一員,百戶二十員。



 屯田左右千戶二所,千戶二員,都目一員,彈壓一員,百戶每所二十員。



 弩軍千戶所,千戶二員,都目一員,彈壓一員。



 資食倉,大使一員,副使一員。



 右都威衛使司,秩正三品,衛使三員,副使二員,僉事二員,經歷、知事、照磨各一員。中統三年,以世祖五投下探馬赤立總管府,秩四品,設總管一員。二十一年,撥屬東宮。二十二年,改蒙古侍衛親軍都指揮使司,秩正三品。三十一年,改隆福宮右都威衛使司,秩仍舊。延祐二年,置儒學教授一員。四年,增蒙古字教授一員。其屬附見:



 鎮撫司,鎮撫二員,都目一員。



 行軍千戶凡五所,秩正四品,千戶五員,副千戶五員,知事五員,百戶五十員,彈壓五員。



 屯田千戶所,秩正五品,千戶二員,彈壓一員,百戶七員,都目一人。



 廣貯倉,秩從九品,大使一員,副使一員,攢典一人。



 衛候直都指揮使司,秩正四品。至元二十年,以控鶴一百三十五人,隸府正司。三十年,隸家令司。三十一年,增控鶴六十五人,立衛候司以領之,兼掌東宮儀從金銀器物,置衛候一員,副衛候二員,及儀從庫百戶。大德十一年,復增懷孟從行控鶴二百人,升都指揮使司,秩正四品。延祐元年,升正三品。七年,降正四品。至治三年罷。四年,以控鶴六百三十人,歸中宮位下。泰定四年,復立司,秩仍正四品。達魯花赤二員,佩三珠虎符;都指揮使二員,佩三珠虎符;副指揮使二員,佩雙珠虎符;知事一員,提控案牘一員,令史四人,譯史、通事各一人,奏差二人。其屬附見:



 百戶所凡六,秩從七品,每所置百戶二員。



 儀從庫,秩從七品,大使二員,副使一員。



 內宰司,秩三品。至元三十一年,既立徽政院,改家令為內宰司。泰定元年,復為家令司。天歷元年罷,未幾復立。二年罷,復改內宰司。內宰六員,司丞四員,典簿二員,照磨一員,令史十有二人,譯史、知印、通事各二人,奏差六人,典吏四人。其屬附見:



 典膳署,秩五品,令二員,丞二員,書吏二員,倉赤三十五人,掌內府飲膳之事。至元十九年始立,隸家令司。三十一年,改掌膳,隸內宰。泰定元年,復改為典膳。



 洪濟鎮,提領三員,掌辦納雁只,隸典膳署。



 柴炭局,秩從七品,提領一員,大使一員,副使一員。至元二十年,以東宮位下民一百戶燒炭二月,軍一百人採薪二月,供內府歲用,立局以主其出納,設官三員,俱受詹事院札。大德十一年,隸徽政院。



 藏珍、文成、供須三庫,秩俱從五品,各設提點二員、大使二員、副使二員,分掌金銀珠玉寶貨、段匹絲綿、皮氈鞍轡等物。國初,詹事出納之事,未有官署印信,至元二十七年分為三庫,各設官六員,及庫子有差。



 提舉備用庫,秩從五品,達魯花赤一員,提舉一員,大使一員,提控案牘一員,掌出納田賦財賦、差發課程、一切錢糧規運等事。至元二十年置。二十二年,設達魯花赤及首領官。



 嘉醖局,秩五品。至元十七年,立掌飲局。大德十一年,改掌飲司,秩升正四品。延祐六年,降掌飲司為局。至治三年罷。泰定四年復立。天歷二年,改嘉醖局,提點二員,大使二員,副使二員,書史一員,書吏四人。



 西山煤窯場,提領一員,大使一員,副使二員,俱受徽政院札。至元二十四年置。領馬安山大峪寺石灰煤窯辦課,奉皇太后位下。



 保定等路打捕提領所,秩從七品,提領四員,典史一員。至元十一年,收集人戶為打捕戶計,及招到管絲銀差發稅糧等戶,立提領所。



 廣平彰德課麥提領所,秩從七品。至元三十年,以二路渡江時駐蹕之地,召民種佃,遂立所,置官統之。



 廣惠庫,大使一員,副使一員。至元三十年,以鈔本五千錠立庫,放典收息,納於備用庫。



 豐裕倉,秩從七品,掌收貯中宮位下糯米。至治二年,設提領等官。三年罷。天歷二年,立儲政院,復給印,置監支納一員、倉使一員、攢典二人。



 備物庫,秩從七品,掌東宮造作顏料及雜器等物。至元二十五年置,隸詹事院。大德元年給印。十一年,置官四員。至治三年罷,泰定三年復立。大使二員,副使二員,庫子二人,攢典二人。



 管領怯憐口諸色民匠都總管府,秩正三品,達魯花赤一員,總管一員,並正三品;同知一員,正四品;副總管二員,正五品;經歷一員,從七品;知事一員,從八品;提控案牘、照磨、管勾各一員,令史十人,知印二人,通事一人,譯史二人,奏差六人,典史四人,領怯憐口人匠造作等事。至大三年,立總管府。至治三年罷。天歷元年復立,隸儲政院。其屬附見:



 管領大都怯憐口諸色人匠提舉司,秩正五品,達魯花赤一員,提舉一員,同提舉、副提舉各一員,首領官一員,司吏四人,部役二人。



 管領上都怯紞口諸色人匠提舉司,秩正五品,達魯花赤一員,提舉一員,同提舉、副提舉各一員,首領官一員,司吏四人,部役二人。



 典制局,秩從七品,大使、副使各一員,直長二員。



 典設署,秩從五品,令、丞各四員,書史一員,書吏四人,掌內府術剌赤二百二十戶。至元二十年置。三十一年,改掌儀署,隸內宰司。泰定元年,復為典設。天歷二年,隸本府。



 雜造人匠提舉司,秩從四品,達魯花赤一員,提舉一員,同提舉、副提舉各一員,都目一員,司吏二人,部役二人。至元八年置。初隸繕珍司,至大三年改隸章慶司。章慶罷,凡造作之事悉歸之。天歷二年,隸本府。



 雜造局,秩正九品,院長一員,直長一員,管勾一員。



 隨路諸色人匠都總管府,秩正三品。中統五年,命招集析居放良還俗僧道等戶,習諸色匠藝,立管領怯憐口總管府,以司其造作,秩正四品。至元九年,升正三品。大德十一年,改繕珍司。延祐六年,升徽儀使司,秩正二品。七年,仍為繕珍司,官屬如舊。至治三年,復改都總管府。達魯花赤一員,總管二員,並正三品;同知一員,正五品;副總管二員,從五品;經歷、知事、照磨、提控案牘各一員,令史四人,譯史一人,奏差二人,典吏一人。其屬附見:



 上都諸色民匠提舉司,秩從五品,提舉一員,同提舉、副提舉、吏目各一員。至元十九年立。至大元年,增達魯花赤一員。至治三年,省增置之員,設官如舊。



 金銀器盒局,秩從八品,大使一員,副使一員。至元七年置。



 染局,秩正八品,大使一員,副使一員。至元七年置。



 雜造局,正八品,大使、副使各一員。至元七年置。



 泥瓦局,大使、副使各一員。至元七年置。



 鐵局,大使一員,副使一員。至元七年置。



 上都葫蘆局,大使一員,副使一員。至元七年置。



 器物局,副使一員。中統五年置。



 砑金局,大使一員。至元二十年置。



 鞍子局,大使一員。至元七年置。



 雲州管納色提領所,提領一員,掌納色人戶。至元七年置。



 大都等路諸色人匠提舉司,秩從五品,提舉、同提舉、副提舉各一員。至元十六年置。其屬附見:



 雙線局,提領一員,副使一員。至元十八年置,受詹事院札。



 大小木局,大使一員,副使一員,直長一員。至元十八年置,受詹事院札。元貞元年,並領皇后位下木局。



 盒缽局,大使一員,副使一員,直長一員。至元七年立,受府札。



 管納色提領一員,受府札,管銅局、筋局、鎖兒局、妝釘局、雕木局。至元三十年置。



 成制提舉司,秩從五品,達魯花赤一員,提舉一員,同提舉、副提舉各一員,吏目一員,司吏四人,部役二人,掌縫制之事。至元二十九年置,設官四員,受院札。大德二年,升提舉司。至治三年罷,泰定四年復置。



 上都、大都貂鼠軟皮等局提領所,提領二員。至元九年置,受府札。二十七年,給從七品印,改受省札。大德十一年,給從六品印,改受敕牒。至治三年,仍改受省札。其屬附見:



 大都軟皮局,使一員,副使一員。至元十三年置。



 斜皮局,局使一員,副使一員。至元十三年置。



 上都軟皮局,局使一員,副使一員。至無十三年置。



 牛皮局,大使一員。至元十三年置。



 金絲子局,大使一員,副使一員,直長一員。至元十二年置,掌金絲子匠造作之事。



 畫油局,大使一員,副使一員,直長一員。至元二十年置,受詹事院札。



 氈局,提領一員,大使一員,副使一員,直長一員。至元十三年,收集人戶為氈匠。二十六年,始立局。



 材木庫,大使、副使各二員。至元十六年置,掌造作材木。



 瑪瑙玉局,大使、副使各一員,直長二員。至元十四年置。



 大都奧魯提領所,提領一員,掌理人匠詞訟。至元十八年置,受詹事院札。



 上都奧魯提領所,提領一員,同提領一員,掌理人匠詞訟。至元十八年置,受詹事院札。



 上都異樣毛子局,大使一員,副使一員。至元二十年置,受詹事院札。



 上都氈局,大使一員,副使一員,直長一員。至元二十年置,受詹事院劄。



 上都斜皮等局,大使一員,副使一員。至元二十年置,受詹事院札。



 蔚州定安等處山場採木提領所,秩正八品,提領一員,大使一員,副使二員。至元十二年置。



 上都隆興等路雜造鞍子局,提領一員,大使一員,直長二員。至元二十二年置,受詹事院札。



 真定路冀州雜造局,大使一員,副使一員,掌造作之事。至元十九年置。



 珠翠局,大使、副使各一員,直長一員。至元三十年置。



 管領大都等路打捕鷹房胭粉人戶總管府,秩正四品。至元十四年,打捕鷹房達魯花赤,招集平灤散逸人戶。二十九年,立總管府。大德十一年,撥隸皇太后位下。延祐六年,升正四品,置達魯花赤一員,總管一員,首領官一員,令史四人,譯史一人,奏差二人。



 管領本投下大都等路怯憐口民匠總管府,國初招集怯憐口哈赤民匠一千一百餘戶,中統元年,立總管府。二年,給六品印,掌戶口錢帛差發等事。至元九年,撥隸安西王位下。皇慶元年,又屬公主皇后位下。延祐元年,改隸章慶司。天歷二年,又改隸儲政院。達魯花赤一員,總管一員,俱受御寶聖旨;同知一員,副總管一員,俱受安西王令旨;知事一員,令史二人。其屬附見:



 織染提舉司,秩正七品,掌織造段匹。提舉一員,受安西王令旨;同提舉一員,本府擬人;副提舉一員,都目一員,俱受安西王傅札;司吏一人。



 管民提領所,凡三,大都路兼奉聖州提領六員,曹州提領二員,河間路提領三員,受本府札。



 管地提領所,凡二,奉聖州提領三員,東安州提領三員,受本府札。



 管領諸路怯憐口民匠都總管府,秩正三品。至元七年,招集析居從良還俗僧道,編籍人戶為怯憐口,立總管府以領之。十四年,以所隸戶口善造作,屬中宮。十六年,立織染、雜造二局以司造作,立提領所以司徭役。二十五年,改升正三品。延祐六年,改繕用司,仍三品。七年,復改府。達魯花赤一員,總管一員,並正三品;同知二員,正五品;副總管二員,從五品;經歷、知事、提控案牘兼照磨各一員,令史五人,譯史一人。其屬附見:



 各處管民提領所,秩正七品。



 河間,益都,保定,冀寧,晉寧,大名,濟寧,衛輝,宣德。



 以上九所,提領、副提領各一員,相副官二員,典史一人,司吏二人。



 汴梁,曹州,大同,開元,大寧,上都,濟南,真定。



 以上八所,提領、副提領、相副官各一員,典史一人,司吏一人。



 大都,歸德,鄂漢。



 以上三所,提領、同提領、副提領各一員,相副官一員,大都增一員,典史、司吏各一人。



 織染局,秩正七品,大使、副使、相副官各一員,典史、司吏各一人。



 雜造局,秩正七品,大使、副使、相副官各一員,典史、司吏各一人。



 弘州衣錦院,秩正七品,大使、副使、直長各一員,典史、司吏各一人。



 豐州毛子局,秩正七品,大使、副使各一員,典史、司吏各一人。



 縉山毛子旋匠局,秩正七品,大使一員,典史、司吏各一人。



 徐邳提舉司,秩正五品,提舉、同提舉、副提舉各一員,吏目、司吏各一人。



 廣備庫,大使、副使各一員,俱受院札。



 汴梁等路管民總管府,秩正三品,達魯花赤、總管、同知、府判各一員,經歷、知事、提控案牘各一員。國初,立息州總管府,領歸附六千三百餘戶。元貞元年,又並壽潁歸附民戶二千四百餘戶,改汴梁等路管民總管府,掌各屯佃戶差發子粒,隸徽政院。泰定元年,改隸詹事院,後隸儲政院,其屬庫一,提領所八,管佃提領十二。



 常盈庫,大使、副使各一員。



 提領所:



 新降戶,真陽,新蔡,息州,汝寧,陳州,汴梁,鄭州,真定。



 以上八所,每所提領各一員,副提領、相副官有差。



 管佃提領:



 汝陽五里岡,許州郾城縣,青龍宋岡,陳州項城商水等屯,分山曲堰,許州臨潁屯,許州襄城屯,汝陽金鄉屯,潁豐堰,遂平橫山屯,上蔡浮召屯,汝陽縣煙亭屯。



 以上十有二處,各設提領二員。



 江淮等處財賦都總管府,秩正三品。達魯花赤、總管各一員,並正三品;同知一員,正五品;副總管二員,從五品;經歷、知事、照磨兼提控案牘各一員,令史十五人,奏差十五人,譯史一人,典吏三人。至元十六年,以宋謝太后、福王所獻事產,及賈似道地土、劉堅等田,立總管府以治之。大德四年罷,命有司掌其賦。天歷二年復立,其賦復歸焉。



 儲用庫,提領、大使、副使各一員。



 杭州織染局,大使、副使、相副官各一員。



 揚州等處財賦提舉司,達魯花赤、提舉、同提舉、副提舉各一員,提控案牘、都目各一員。其屬附見:



 安慶等處河泊所,提領、大使、副使各一員。



 建康等處財賦提舉司,達魯花赤、提舉、同提舉、副提舉各一員,提控案牘、都目各一員。



 建康織染局,大使、副使、相副官各一員。



 黃池織染局,大使、副使、相副官各一員。



 建康等處三湖河泊所,提領、大使、副使、相副官各一員。



 池州等處河泊所,提領、大使、副使各一員。



 平江等處財賦提舉司,達魯花赤、提舉、同提舉、副提舉各一員,提控案牘、都目各一員。



 杭州等處財賦提舉司,設官同上。



 陜西等處管領毛子匠提舉司,達魯花赤、提舉各一員。國初,收集織造毛子人匠。至元三年,置官二員,皆世襲。



 昭功萬戶都總使司,秩正三品。都總使二員,正三品;同知一員,從三品;副使二員,正四品;經歷、知事、照磨各一員,令史六人,譯史六人,知印二人,怯里馬赤二人,奏差六人,典吏四人。至順二年立,凡文宗潛邸扈從之臣,皆領於是府。其屬則宮相、膳工等司。



 宮相都總管府,秩正三品,達魯花赤二員,都總管一員,副達魯花赤二員,同知二員,副總管二員,經歷、知事、提控案牘承發架閣各一員。至順二年,罷宮相府並鶴馭司,改怯憐口錢糧總管府為本府。



 織染雜造人匠都總管府,秩正三品,達魯花赤一員,總管一員,同知一員,副總管二員,經歷、知事、提控案牘、照磨各一員。至元二十年,為管領織染段匹匠人設總管府。元貞二年,以營繕浩繁,事務冗滯,升為都總管府,隸徽政院。天歷元年,改隸儲慶使司。三年,改屬宮相。



 織染局,秩從七品,大使一員,副使一員。至元二十三年,改織染提舉司為局。



 綾錦局,秩從七品,大使一員,副使一員。至元八年置。九年,以招收析居放良還俗僧道為工匠,二百八十有二戶,教習織造之事,遂定置以上官。



 紋錦局,秩從七品,大使一員,副使一員。國初,以招收漏籍人戶,各管教習立局,領送納絲銀物料織造段匹。至元八年,設長官。十二年,以諸人匠賜東宮。十三年,罷長官,設以上官掌之。



 中山局,秩從七品,大使一員,副使一員。國初,以招收隨路漏籍不當差人戶,立局管領,教習織造。至元十二年,以賜東宮,遂定置局官如上。



 真定局,秩從七品,大使一員。國初,招收戶計。中統元年置,掌織染造作。至正十六年,以賜東宮,設官悉如舊。



 弘州、蕁麻林納失失局,秩從七品,二局各設大使一員、副使一員。至元十五年,招收析居放良等戶,教習人匠織造納失失,於弘州、蕁麻林二處置局。十六年,並為一局。三十一年,徽政院以兩局相去一百餘里,管辦非便,後為二局。



 大名織染雜造兩提舉司,秩正六品。至元二十一年置,掌大名路民戶內織造人匠一千五百四十有奇,各置提舉、同提舉、副提舉一員。三十年,增置雜造達魯花赤一員。



 供用庫,秩從九品,大使、副使各一員,受徽政院札。國初,為綾錦總庫。至元二十一年,改為供用庫。



 管領諸路打捕鷹房納綿等戶總管府,秩正三品,達魯花赤、都總管、同知、治中、府判各一員,經歷、知事、提控案牘各一員。掌人匠一萬三千有奇,歲辦稅糧皮貨,採捕野物鷹鷂,以供內府。至元十二年,賜東宮位下,遂以真定所立總管府移置大都,隸詹事。十六年,合並所管之戶,置都總管以總治之。三十一年,詹事院罷,隸徽政。至大四年,隸崇祥院。延祐六年,又隸詹事。天歷元年,隸儲慶使司。至順元年,改屬宮相府。



 管領上都等處打捕鷹房納綿等戶大使司,大使、副使各一員。



 管領順德等處打捕鷹房納綿等戶提領所,達魯花赤、提領、副提領各一員。



 管領冀寧等處打捕鷹房納綿等戶提領所,提領、副提領各一員。



 管領大都左右巡院等處打捕鷹房納綿等戶提領所,提領、副提領各一員。



 管領固安等處打捕鷹房納綿等戶提領所,提領、副提領各一員。



 管領中山等處打捕鷹房納綿等戶提領所,提領、副提領各一員。



 管領濟南等處打捕鷹房納綿等戶提領所,提領、副提領各一員。



 管領德州等處打捕鷹房納綿等戶提領所,提領、副提領各一員。



 管領益都等處打捕鷹房納綿等戶提領所,提領、副提領各一員。



 管領大同等處打捕鷹房納綿等戶提領所,提領、副提領各一員。



 管領濟寧等處打捕鷹房納綿等戶提領所,提領、副提領各一員。



 管領興和等處打捕鷹房納綿等戶提領所,提領、副提領各一員。



 管領晉寧等處打捕鷹房納綿等戶提領所,提領、副提領各一員。



 管領順州稻田提領所,提領、副提領各一員。



 管領懷慶稻田提領所,提領一員。



 管領檀州等處打捕鷹房納綿等戶提領所,提領、副提領各一員。



 管領大寧等處打捕鷹房納綿等戶提領所,提領、副提領各一員。



 管領薊州等處打捕鷹房納綿等戶提領所,提領、副提領各一員。



 管領真定等處打捕鷹房納綿等戶提領所,設官同上。



 管領趙州等處打捕鷹房納綿等戶提領所,設官同上。



 管領保定等處打捕鷹房納綿等戶提領所,設官同上。



 管領冀州等處打捕鷹房納綿等戶提領所,設官同上。



 管領汴梁等處打捕鷹房納綿等戶提領所,設官同上。



 廣衍庫,大使一員。



 管領滑山炭場所,大使一員。



 繕工司,秩正三品,卿二員,少卿二員,丞二員,經歷、知事、照磨兼提控案牘、管勾承發架閣各一員,令史四人,譯史二人,知印二人,怯里馬赤一人,典吏三人,掌人匠營造之事。天歷二年置。其屬附見:



 金玉珠翠提舉司,達魯花赤、提舉、同提舉、副提舉各一員,吏目一員,司吏四人。



 大都織染提舉司,提舉二員,同提舉、副提舉各一員,吏目一員,司吏四人。



 大都雜造提舉司,達魯花赤、提舉、同提舉、副提舉各一員,吏目一員,司吏四人。



 富昌庫,大使一員,副使一員,庫子二人,攢典一人。



 內史府,秩正二品。內史九員,正二品;中尉六員,正三品;司馬四員,正四品;諮議二員,從五品;記室二員,從六品;照磨兼管勾承發架閣庫,從八品;掾史八人,譯史四人,知印、通事各二人,宣使五人,典吏二人。至元二十九年,封晉王於太祖四斡耳朵之地,改王傅為內史,秩從二,置官十四員。延祐五年,升正二品,給印,分司京師,並分置官屬。



 延慶司,秩正三品,掌王府祈禳之事。使三員,正三品;同知二員,正四品;典簿一員,從七品;令史二人,譯史、知印、通事各一人,奏差二人。至元二十七年置。



 斷事官,秩正三品,理王府詞訟之事。斷事官一十六員,正三品;經歷、知事各一員,令史三人。



 典軍司,秩從七品,掌控鶴百二十有六人,典軍二員,副使二員。大德四年置。



 隨路諸色民匠打捕鷹房都總管府,秩正二品,總四斡耳朵位下戶計民匠造作之事。達魯花赤二員,都總管一員,同知一員,副總管二員,經歷、知事、提控案牘各一員,令史四人,奏差二人。至元二十四年置。官吏不入常調,凡斡耳朵之事,復置四總管以分掌之。



 管領保定等路阿哈探馬兒諸色人匠總管府,秩從二品,掌太祖大斡耳朵一切事務。達魯花赤、總管、同知、副總管各一員,知事一員,吏二人。至元十七年置。



 管領曹州東平等路民匠提舉司,秩從五品,達魯花赤、提舉、同提舉、副提舉各一員。至元十七年置。



 管領大都納綿提舉司,秩從六品,達魯花赤、提舉、副提舉各一員。至元十七年置。



 管領上都大都奉聖州長官司,秩從六品,管領出征軍五十有一戶,達魯花赤、長官各一員。至元十七年置。



 管領保定織染局,秩從六品,管匠一百有一戶,達魯花赤、提舉、同提舉、副提舉各一員。至元十七年置。



 管領豐州捏只局,頭目一員,掌織造花毯。至元十七年置。



 管領打捕鷹房民匠達魯花赤總管府,秩正四品,掌二皇后斡耳朵位下歲賜財物造作等事,達魯花赤、總管、同知、副總管、知事各一員,吏二人。至元二十一年置。



 管領口子迤北長官司,秩從五品,掌領戶計二百有六,達魯花赤、長官、副長官各一員。至元二十一年置。



 管領隨路諸色民匠達魯花赤等官,秩正五品,統民匠一千五百二十有五戶,達魯花赤、總管、同知、副總管各一員。至元二十一年置。



 管領隨路打捕納綿民匠長官司,秩從五品,掌民匠一百七十有九戶,達魯花赤、長官各一員。至元二十一年置。



 管領大都民匠提舉司,秩正七品,掌民匠二百有二戶,提舉、同提舉、副提舉各一員。至元二十一年置。



 管領涿州成錦局人匠提舉司,秩從五品,領匠一百有二戶,達魯花赤、提舉、同提舉、副提舉各一員。至元二十一年置。



 管領河間民匠提舉司,秩從四品,掌民匠二百一十戶,達魯花赤、提舉、同提舉、副提舉各一員。至元二十一年置。



 管領河間滄州等處長官司,秩正五品,領戶計五百四十有八,達魯花赤、長官、副長官各一員。至元二十一年置。



 管領河間臨邑等處軍民長官司,秩正七品,掌軍民二百有二戶,達魯花赤、長官、副長官各一員。至元二十一年置。



 管領隨路諸色民匠打捕鷹房等戶總管府,秩從四品,掌太祖斡耳朵四季行營一切事務,達魯花赤、總管、同知、副總管、知事各一員,司吏二人。大德二年置。



 管領涿州等處民匠異錦局,秩正五品,掌民匠一百五十戶,達魯花赤、提舉、同提舉、副提舉各一員。大德二年置。



 管領上用織染局,秩從七品,掌工匠七十有八戶,提舉、同提舉、副提舉各一員。大德二年置。



 管領上都大都曲米等長官司,秩從七品,領民匠七十有九戶,達魯花赤、長官、副長官各一員。大德二年置。



 管領彰德等處長官司,秩從七品,掌民一百一十有七戶,達魯花赤、長官、副長官各一員。大德二年置。



 管領上都大都等處長官司,秩從五品,掌民二百六十有一戶,達魯花赤、長官、副長官各一員。大德二年置。



 管領泰安等處長官司,秩正七品,掌民一百有一戶,達魯花赤、長官、副長官各一員。大德二年置。



 管領曹州等處長官司,秩從五品,管民一百有五戶,達魯花赤、長官、副長官各一員。大德二年置。



 管領隨路打捕鷹房諸色民匠怯憐口總管府,秩從三品。掌太祖四皇后位下四季行營並歲賜造作之事,達魯花赤、總管、同知、副總管各一員,經歷、知事、提控案牘兼照磨各一員,司吏二人。延祐五年置。



 管領大都上都打捕鷹房納米面提舉司,秩從五品,統領一百九十有五戶,達魯花赤、提舉各一員。延祐五年置。



 管領大都涿州織染提舉司,秩從七品,掌領九十有六戶,達魯花赤、提舉各一員。延祐五年置。



 管領河間路清州人匠提舉司,秩從五品,掌戶計二百三十有四戶,達魯花赤、提舉各一員。延祐五年置。



 管領隨路打捕鷹房諸色民匠總管府,秩正四品,掌北安王位下歲賜錢糧之事,達魯花赤、總管、同知、副總管、知事各一員。至元二十二年置。



 管領大都等處納綿提舉司,秩正七品,掌納綿戶計七百有三戶,達魯花赤、提舉、副提舉各一員。至元二十二年置。



 管領大都等處金玉民匠稻田提舉司,秩從五品,掌納綿人匠五百二十有一戶,達魯花赤、提舉、副提舉各一員。至元二十二年置。



 管領大都蘇州等處打捕提舉司,秩從五品。掌打捕戶及民匠六百餘戶。達魯花赤、提舉、副提舉各一員。至元二十二年置。



 雜造局,秩正六品,達魯花赤一員,提舉、同提舉、副提舉各一員。至元十六年置。



 怯憐口諸色民匠達魯花赤並管領上都納綿提舉司,秩正五品,掌迭只斡耳朵位下怯憐口諸色民匠及歲賜錢糧等事,達魯花赤、長官、同知、副長官各一員,提控案牘一員。



 上都人匠提領所,秩從七品,達魯花赤、提領、同提領、副提領各一員。至元二十四年置。



 上都大都提領所,秩從七品,掌本位下怯憐口等事,達魯花赤、大使、副使各一員。至元二十七年置。



 歸德長官司,秩從六品,達魯花赤、長官、副長官各一員。至治三年置。



 管領上都大都諸色人匠納綿戶提舉司,秩從五品。掌斡耳朵位下歲賜等事。達魯花赤、提舉、同提舉各一員。至元十七年置。



 致用庫,秩從七品,提領、大使各一員,副使二員。至元二十七年置。



 提領司,秩從八品,提領三員,副提領一員。至元十一年置。



 上都人匠局,秩從七品,達魯花赤二員,副使二員。至元二十七年置。



 諸王傅官,寬徹不花太子至齊王位下,凡四十五王,每位下各設王傅、傅尉、司馬三員。傅尉,唯寬徹不花、也不干、斡羅溫孫三王有之。自此以下,皆稱府尉,別於王傅之下,司馬之上。而三員並設,又多寡不同,或少至一員,或多至三員者。齊王則又獨設王傅一員。



 都護府,秩從二品,掌領舊州城及畏吾兒之居漢地者,有詞訟則聽之。大都護四員,從二品;同知二員,從三品;副都護二員,從四品;經歷一員,從六品;都事一員,從七品;照磨兼承發架閣庫管勾一員,正八品;令史四人,譯史二人,通事、知印各一人,宣使四人,典吏二人。至元十一年,初置畏吾兒斷事官,秩三品。十七年,改領北庭都護府,秩二品,置官十二員。二十年,改大理寺,秩正三品。二十二年,復為大都護,品秩如舊。延祐三年,升正二品。七年,復從二品,定官制如上。



 崇福司,秩二品,掌領馬兒哈昔列班也裏可溫十字寺祭享等事。司使四員,從二品;同知二員,從三品;副使二員,從四品;司丞二員,從五品;經歷一員,從六品;都事一員,從七品;照磨一員,正八品;令史二人,譯史、通事、知印各一人,宣使二人。至元二十六年置。延祐二年,改為院,置領院事一員,省並天下也裏可溫掌教司七十二所,悉以其事歸之。七年,復為司,後定置已上官員。



\end{pinyinscope}