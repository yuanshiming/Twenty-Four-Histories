\article{列傳第十九}

\begin{pinyinscope}

 ○杭忽思



 杭忽思,阿速氏,主阿速國。太宗兵至其境,杭忽思率眾來降,賜名拔都兒,錫以金符,命領其土民。尋奉旨選阿速軍千人,及其長子阿塔赤扈駕親征。既還,阿塔赤入直宿衛。杭忽思還國,道遇敵人,戰歿,敕其妻外麻思領兵守其國。外麻思躬擐甲胄,平叛亂,後以次子按法普代之。



 阿塔赤從憲宗征西川軍於釣魚山,與宋兵戰有功,帝親飲以酒,賞以白金。阿里不哥叛,從也裏可徵之。至寧夏,與阿藍答兒、渾都海戰,率先赴敵,矢中其腹,不懼。世祖聞而嘉之,賞以白金,召入宿衛。中統二年,扈駕親征阿里不哥,追至失木里禿之地,以功復賞白金。三年,從征李璮,平之。至元五年,奉旨同不答臺領兵南征,攻破金剛臺。六年,從攻安慶府,戰有功。七年,從下五河口。十一年,從下沿江諸郡,戍鎮巢,民不堪命,宋降將洪福以計乘醉而殺之。世祖憫其死,賜其家白金五百兩、鈔三千五百貫,並鎮巢降民一千五百三十九戶,且命其子伯答兒襲千戶,佩金符。



 時失烈吉叛,詔伯答兒領阿速軍一千往征之,與甕吉剌只兒瓦臺軍戰于押里,復與藥木忽兒軍戰於禿剌及斡魯歡之地。十五年春,至伯牙之地,與赤憐軍合戰。五月,駐兵呵剌牙,與外剌臺、寬赤哥思等軍合戰。其大將塔思不花樹木為柵,積石為城,以拒大軍。伯答兒督勇士先登,拔之,伯答兒矢中右股,別吉裏迷失以其功聞,賞白金。二十年,授虎符、定遠大將軍、後衛親軍都指揮使,兼領阿速軍,充阿速拔都達魯花赤。二十二年,徵別失八里,軍於亦里渾察罕兒之地,與禿呵、不早麻軍戰,有功。二十六年,徵杭海,敵勢甚盛,大軍乏食,其母乃咬真輸己帑及畜牧等給軍食。世祖聞而嘉之,賜予甚厚。大德四年,伯答兒卒。



 長子斡羅思,由宿衛仕至隆鎮衛都指揮使。次子福定,襲職,官懷遠大將軍,尋改右阿速衛達魯花赤,兼管後衛軍。至大四年,兄都丹充右阿速衛都指揮使;福定復職後衛,升樞密同僉,命領軍一千守遷民鎮,尋授定遠大將軍、僉樞密院事、後衛親軍都指揮使,提調右衛阿速達魯花赤。二年,進資善大夫、同知樞密院事。後至元間,進知樞密院事。



 ○步魯合答



 步魯合答,蒙古弘吉剌氏。祖按主奴,太宗時率蒙古軍千人從諸王察合臺征河西,至山丹。攻下定、會、階、文諸州,以功為元帥,佩金符,駐軍漢陽禮店,戍守西和、階、文南界及西蕃邊境。換金虎符,真除元帥。父車里,襲職。從都元帥紐璘攻成都,宋將劉整以重兵守雲頂山,車里擊敗之,進圍其城。整遣裨校出戰,敗走,追至簡州斬之,殺三百餘人,遂拔其城。攻重慶,車裏將兵千人為先鋒,渡馬湖江,敗宋兵於馬老山,俘獲百餘人。戊午,諸軍還屯灰山,宋兵夜來劫營,車里擊敗之,斬首三百級。世祖即位,賜金符,為奧魯元帥,又改征行元帥。至元二年,車里以老疾不任事,諸王阿只吉命步魯合答代領其軍。至元八年,制授管軍千戶,佩金符。宋將昝萬壽攻成都,僉省嚴忠範遣步魯合答將兵七百人御之於沙坎。



 以下缺



\end{pinyinscope}