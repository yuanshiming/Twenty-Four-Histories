\article{本紀第二十三 武宗二}

\begin{pinyinscope}

 二年春正月己丑,從皇太子請,罷宮師府,設賓客、諭德、贊善如故。庚寅,越王禿剌有罪賜死。禁日者、方士出入諸王、公主、近侍及諸官之門。辛卯生之精神」,析理學術發展的淵流、脈絡,以求「成一家之,皇太子、諸王、百官上尊號曰統天繼聖欽文英武大章孝皇帝。乙未,恭謝太廟。丙申,詔天下弛山澤之禁,恤流移,毋令見戶包納差稅;被災百姓,內郡免差稅一年,江淮免夏稅;內外大小職官普覃散官一等,有出身人考滿者,加散官一等。己亥,封知樞密院事容國公床兀兒為句容郡王。乙巳,塔思不花、乞臺普濟言:「諸人恃恩徑奏,璽書不由中書,直下翰林院給與者,今核其數,自大德六年至至大元年所出,凡六千三百餘道,皆於田土、戶口、金銀鐵冶、增餘課程、進貢奇貨、錢穀、選法、詞訟、造作等事,害及於民,請盡追奪之。今後有不由中書者,乞勿與。」制可。丙午,定制大成至聖文宣王春秋二丁釋奠用太牢。戊申,迭裏貼木兒不花進鷹犬,命歲以幣帛千匹、鈔千錠與之。



 二月戊午,鑄金印賜句容郡王床兀兒。賑真定路饑民糧萬石,塔塔境六千石。癸亥,皇太子幸五臺佛寺。罷行泉府院,以市舶歸之行省。乙丑,以和林屯田去秋收九萬餘石,其宣慰司官吏、部校、軍士給賞有差。己巳,太陰犯亢。辛未,太陰犯氐。調國王部及忽裏合赤、兀魯帶、朵來等軍九千五百人赴和林。壬申,令各衛董屯田官三年一易。甲戌,弛中都酒禁。



 三月己丑,遼陽行省右丞洪重喜訴高麗國王王章不奉國法恣暴等事,中書省臣請令重喜與高麗王辯對。敕中書毋令辯對,令高麗王從太后之五臺山。梁王在雲南有風疾,以諸王老的代梁王鎮雲南,賜金二百五十兩、銀七百五十兩,從者幣帛有差。庚寅,車駕幸上都。摘五衛軍五十人隸中都虎賁司,封諸王也速不乾為襄寧王。辛卯,罷杭州白云宗攝所,立湖廣頭陀禪錄司。丙寅,賜雲南王老的金印。戊戌,太陰犯氐。己亥,熒惑犯歲星。封公主阿剌的納八剌為趙國公主,駙馬注安為趙王。甲辰,中書省臣言:「國家歲賦有常,頃以歲儉,所入曾不及半,而去歲所支,鈔至千萬錠,糧三百萬石。陛下嘗命汰其求芻粟者,而宣徽院孛可孫竟不能行,視去歲反多三十萬石,請用知錢穀者二三員於宣徽院佐而理之。又,中書省斷事官,大德十年四十三員,今皇太子位增二員,諸王闊闊出、剌馬甘禿剌亦各增一員,非舊制。臣等以為皇太子位所增宜存,諸王者宜罷。」並從之。升掌醫署為典醫監。乙巳,中書省臣言:「中書為百司之首,宜先汰冗員。」帝曰:「百司所汰,卿等定議;省臣去留,朕自思之。」己酉,濟陰、定陶雹。



 夏四月甲寅,中書省臣言:「江浙杭州驛,半歲之間,使人過者千二百餘,有桑兀、寶合丁等進獅、豹、鴉、鶻,留二十有七日,人畜食肉千三百餘斤。請自今遠方以奇獸異寶來者,依驛遞;其商人因有所獻者,令自備資力。」從之。辛酉,立興聖宮江淮財賦總管府,詔諭中外。癸亥,摘漢軍五千,給田十萬頃,於直沽沿海口屯種,又益以康里軍二千,立鎮守海口屯儲親軍都指揮使司。壬午,詔中都創皇城角樓。中書省臣言:「今農事正殷,蝗蝝遍野,百姓艱食,乞依前旨罷其役。」帝曰:「皇城若無角樓,何以壯觀!先畢其功,餘者緩之。」以建新寺,鑄提調、監造三品銀印。益都、東平、東昌、濟寧、河間、順德、廣平、大名、汴梁、衛輝、泰安、高唐、曹、濮、德、揚、滁、高郵等處蝗。



 五月丁亥,以通政院使憨剌合兒知樞密院事,董建興聖宮,令大都留守養安等督其工。丁酉,以陰陽家言,自今至聖誕節不宜興土功,敕權停新寺工役。甲辰,御史臺臣言:「乘輿北幸,而京師工役正興,加之歲旱乏食,民愚易惑,所關甚重,乞留一丞相鎮京師,後為例。」制可。六月癸亥,選官督捕蝗。從皇太子言,禁諸賜田者馳驛徵租擾民。庚午,中書省臣言:「奉旨既停新寺工役,其亭苑鷹坊諸役,乞並罷。又,太醫院遣使取藥材於陜西、四川、雲南,費公帑,勞驛傳。臣等議,事幹錢糧,隔越中書省徑行,乞禁止。」並從之。以益都、濟南、般陽三路,寧海一州屬宣慰司,餘並令直隸省部。以大都隸儒籍者四十戶充文廟樂工。從皇太子請,改典樂司提點、大使等官為卿、少卿、丞。甲戌,以宿衛之士比多冗雜,遵舊制,存蒙古、色目之有閥閱者,餘皆革去。皇太子言:「宣政院先奉旨,毆西番僧者截其手,詈之者斷其舌,此法昔所未聞,有乖國典,且於僧無益。僧俗相犯,已有明憲,乞更其令。」又言:「宣政院文案不檢核,於憲章有礙,遵舊制為宜。」並從之。乙亥,中書省臣言:「河南、江浙省言,宣政院奏免僧、道、也裏可溫、答失蠻租稅。臣等議,田有租,商有稅,乃祖宗成法,今宣政院一體奏免,非制。」有旨,依舊制征之。是月,金城、崞州、源州雨雹,延安之神木碾谷、盤西、神川等處大雨雹,霸州、檀州、涿州、良鄉、舒城、歷陽、合肥、六安、江寧、句容、溧水、上元等處蝗。



 秋七月癸未,河決歸德府境。壬辰,宣政院臣言:「武靖王搠思班與朵思麻宣慰司言:『松潘疊宕威茂州等處安撫司管內,西番、禿魯卜、降胡、漢民四種人雜處,昨遣經歷蔡懋昭往蛇谷隴迷招之,降其八部,戶萬七千,皆數百年負固頑獷之人,酋長令真巴等八人已嘗廷見。今令真巴謂其地鄰接四川,未降者尚十餘萬。宣撫司官皆他郡人,不知蠻夷事宜,才至成都灌州,畏懼即返,何以撫治?宜改安撫司為宣撫司,遷治茂州,徙松州軍千人鎮遏為便。』臣等議,宜從其言。」詔改松潘疊宕威茂州安撫司為宣撫司,遷治茂州汶川縣,秩正三品,以八兒思的斤為宣撫司達魯花赤,蔡懋昭為副使,並佩虎符。乙未,復置贛州龍南、安遠二縣。以河西二十驛往來使多,馬數既少,民力耗竭,命中書省、樞密院、通政院於諸部撥戶增馬以濟之。樂實言鈔法大壞,請更鈔法,圖新鈔式以進,又與保八議立尚書省,詔與乞臺普濟、塔思不花、赤因鐵木兒、脫虎脫集議以聞。己亥,河決汴梁之封丘。甲辰,改昔保赤八剌合孫總管府為奉時院。乙巳,保八言:「臣與塔思不花、乞臺普濟等集議立尚書省事,臣今竊自思之,政事得失,皆前日中書省臣所為,今欲舉正,彼懼有累,孰願行者?臣今不言,誠以大事為懼。陛下若矜憐保八、樂實所議,請立尚書省,舊事從中書,新政從尚書。尚書,請以乞臺普濟、脫虎脫為丞相,三寶奴、樂實為平章,保八為右丞,王羆參知政事。姓江者畫鈔式,以為印鈔庫大使。」並從之。塔思不花言:「此大事,遽爾更張,乞與老臣更議。」帝不從。是月,濟南、濟寧、般陽、曹、濮、德、高唐、河中、解、絳、耀、同、華等州蝗。



 八月壬子,中書省臣言:「甘肅省僻在邊垂,城中蓄金穀以給諸王軍馬,世祖、成宗嘗修其城池。近撒的迷失擅興兵甲,掠豳王出伯輜重,民大驚擾。今撒的迷失已伏誅,其城若不修,慮啟寇心。又,沙、瓜州摘軍屯田,歲入糧二萬五千石,撒的迷失叛,不令其軍入屯,遂廢。今乞仍舊遣軍屯種,選知屯田地利色目、漢人各一員領之。」皆從之。癸丑,立尚書省,以乞臺普濟為太傅、右丞相,脫虎脫為左丞相,三寶奴、樂實為平章政事,保八為右丞,忙哥鐵木兒為左丞,王羆為參知政事,中書左丞劉楫授尚書左丞、商議尚書省事,詔告天下。甲寅,敕以海剌孫昔與伯顏、阿術平江南,知兵事,可授平章政事,商議樞密院事。以阿速衛軍五百人隸諸王怯里不花,駐和林,給鈔萬五千錠,人備四馬。己未,立皇太子右衛率府,秩正三品,命尚書右丞相脫虎脫、御史大夫不里牙敦並領右衛率府事。尚書省臣言:「中書省尚有逋欠錢糧應追理者,宜存斷事官十人,餘皆並入尚書省。」又言:「往者大闢獄具,尚書省議定,令中書省裁酌以聞,宜依舊制。」從之。以江西等處行中書省參知政事郝彬為尚書省參知政事。甲戌,賜太師頭名脫兒赤顏。丁丑,永平路隕霜殺禾。己卯,三寶奴言:「尚書省立,更新庶政,變易鈔法,用官六十四員,其中宿衛之士有之,品秩未至者有之,未歷仕者有之。此皆素習於事,既已任之,乞勿拘例,授以宣敕。」制可。詔天下,敢有沮撓尚書省事者,罪之。真定、保定、河間、順德、廣平、彰德、大名、衛輝、懷孟、汴梁等處蝗。九月庚辰朔,以尚書省條畫詔天下,改各行中書省為行尚書省。詔:「朝廷得失,軍民利害,臣民有上言者,皆得實封上聞,在外者赴所屬轉達。各處人民,饑荒轉徙復業者,一切逋欠,並行蠲免,仍除差稅三年。田野死亡,遺骸暴露,官為收拾。」頒行至大銀鈔,詔曰:「昔我世祖皇帝既登大寶,始造中統交鈔,以便民用,歲久法隳,亦既更張,印造至元寶鈔。逮今又復二十三年,物重鈔輕,不能無弊,乃循舊典,改造至大銀鈔,頒行天下。至大銀鈔一兩,準至元鈔五貫、白銀一兩、赤金一錢。隨路立平準行用庫,買賣金銀,倒換昏鈔。或民間絲綿布帛,赴庫回易,依驗時估給價。隨處路府州縣,設立常平倉以權物價,豐年收糴粟麥米穀,值青黃不接之時,比附時估,減價出糶,以遏沸湧。金銀私相買賣及海舶興販金、銀、銅錢、綿絲、布帛下海者,並禁之。平準行用庫、常平倉設官,皆於流官內銓注,以二年為滿。中統交鈔,詔書到日,限一百日盡數赴庫倒換。茶、鹽、酒、醋、商稅諸色課程,如收至大銀鈔,以一當五。頒行至大銀鈔二兩至一厘,定為一十三等,以便民用。」壬午,江南行臺劾:「平章政事教化,詐言家貧,冒受賜貨物,折鈔二萬錠。且其人素行,無一善可稱。魏國公尊爵也,豈宜授之?請追奪為宜。」制可。癸未,尚書省臣言:「古者設官分職,各有攸司,方今地大民眾,事益繁冗,若使省臣總挈綱領,庶官各盡厥職,其事豈有不治?頃歲省務壅塞,朝夕惟署押文案,事皆廢弛。天災民困,職此之由。自今以始,省部一切,皆令從宜處置,大事或須上請,得旨即行,用成至治,上順天道,下安民心。」又言:「國家地廣民眾,古所未有。累朝格例前後不一,執法之吏輕重任意,請自太祖以來所行政令九千餘條,刪除繁冗,使歸於一,編為定制。」並從之。以大都城南建佛寺,立行工部,領行工部事三人,行工部尚書二人,仍令尚書右丞相脫虎脫兼領之。丙戌,車駕至大都。戊子,尚書省臣言:「翰林國史院,先朝御容、實錄皆在其中,鄉置之南省。今尚書省復立,倉卒不及營建,請買大第徙之。」制可。壬辰,賜高唐王注安金五千兩、銀五萬兩。癸巳,以薪價貴,禁權豪畜鷹犬之家不得占據山場,聽民樵採。三寶奴言:「冀寧、大同、保定、真定以五臺建寺,所須皆直取於民,宜免今年租稅。」從之。丙申,御史臺臣言:「頃年歲兇民疫,陛下哀矜賑之,獲濟者眾。今山東大饑,流民轉徙,乞以本臺沒入贓鈔萬錠賑救之。」制可。丁酉,御史臺臣言:「比者近幸為人奏請,賜江南田千二百三十頃,為租五十萬石,乞拘還官。」從之。己亥,尚書省臣言:「今國用需中統鈔五百萬錠,前者嘗借支鈔本至千六十萬三千一百餘錠,今乞罷中統鈔,以至大銀鈔為母,至元鈔為子,仍撥至元鈔本百萬錠,以給國用。」大都立資國院,秩正二品;山東、河東、遼陽、江淮、湖廣、川漢立泉貨監六,秩正三品;產銅之地立提舉司十九,秩從五品。尚書省臣言:「三宮內降之旨,曩中書省奏請勿行,臣等謂宜仍舊行之,儻於大事有害,則復奏請。」帝是其言。又言:「中書之務,乞以盡歸臣等。至元二十四年,凡宣敕亦尚書省掌之。今臣等議,乞從尚書省任人,而以宣敕散官委之中書。」從之。占八國王遣其弟扎剌奴等來貢白面象、伽藍木。合魯納答思、禿堅鐵木兒、桑加失裡等奏請遣人使海外諸國。以禿堅、張也先、伯顏使不憐八孫,薛徹兀、李唐、徐伯顏使八昔,察罕、亦不剌金、楊忽答兒、阿里使占八。以陜西行臺大夫、大司徒沙的為左丞相、行土蕃等處宣慰使都元帥。甲辰,尚書省言:「每歲芻粟費鈔五十萬錠,請廢孛可孫,立度支院,秩二品,設使、同知、僉院、僉判各二員。」從之。乙巳,以盜多,徙上都、中都、大都舊盜於木達達、亦剌思等地耕種。丁未,三寶奴言養豹者害民為甚,有旨禁之,有復犯者,雖貴幸亦加罪。



 冬十月庚戌朔,以皇太子為尚書令詔天下,令州縣正官以九年為任詔天下,又以行銅錢法詔天下。辛亥,皇太子言:「舊制,百官宣敕散官皆歸中書,以臣為中書令故也。自今敕牒宜令尚書省給降,宣命仍委中書。」制可。丙辰,樂實言:「江南平垂四十年,其民止輸地稅、商稅,餘皆無與。其富室有蔽占王民奴使之者,動輒百千家,有多至萬家者,其力可知。乞自今有歲收糧滿五萬石以上者,令石輸二升於官,仍質一子而軍之。其所輸之糧,移其半入京師以養御士,半留於彼以備兇年。富國安民,無善於此。」帝曰:「如樂實言行之。」辛酉,弛酒禁,立酒課提舉司。尚書省以錢穀繁劇,增戶部侍郎、員外郎各一員;又增禮部侍郎、郎中各一員,凡言時政者屬之。立太廟廩犧署,設令、丞各一員。癸亥,以翰林學士承旨不里牙敦為御史大夫。乙丑,以皇太后有疾,詔天下釋大闢百人。丁卯,以御史大夫只兒合郎及中書左丞相脫脫、尚服院使大都,並知樞密院事。壬申,太陰犯左執法。癸酉,尚書省臣言:「比來柬汰冗官之故,百官俸至今未給,乞如大德十年所設員數給之,餘弗給。」從之。加知樞密院事禿忽魯左丞相。丁丑,以遼陽行尚書省平章政事合散為左丞相、行中書省平章政事,中書參知政事伯都為平章政事、行中書右丞,商議中書省事忽都不丁為右丞、行中書省左丞,參議中書省事鐵裏脫歡、賈鈞並中書參知政事。戊寅,御史臺臣言:「常平倉本以益民,然歲不登,遽立之,必反害民,罷之便。」又言:「至大銀鈔始行,品目繁碎,民猶未悟,而又兼行銅錢,慮有相妨。」又言:「民間拘銅器甚急,弗便,乞與省臣詳議。」又言:「歲兇乏食,不宜遽弛酒禁。」有旨:「其與省臣議之。」



 十一月庚辰朔,以徐、邳連年大水,百姓流離,悉免今歲差稅。增吏部郎中、員外郎、主事各一員,令考功以行黜陟。東平、濟寧薦饑,免其民差稅之半,下戶悉免之。尚書省臣言:「比年衛士大濫,率多無賴,請充衛士者,必廷見乃聽。」從之。雲南行省言:「八百媳婦、大徹里、小徹里作亂,威遠州穀保奪據木羅甸,詔遣本省右丞算只兒威往招諭之,仍令威楚道軍千五百人護送入境。而算只兒威受穀保賂金銀各三錠,復進兵攻劫,穀保弓弩亂發,遂以敗還。匪惟敗事,反傷我人,惟陛下裁度。」帝曰:「大事也,其速擇使復齎璽書往招諭,算只兒威雖遇赦,可嚴鞫之。」甲申,賜寧肅王脫脫金印,升皇太子府正司為從二品。乙酉,尚書省及太常禮儀院言:「郊祀者,國之大禮。今南郊之禮已行而未備,北郊之禮尚未舉行,今年冬至祀天南郊,請以太祖皇帝配;明年夏至祀地北郊,請以世祖皇帝配。」制可。丁亥,以湖廣行省左丞散術帶為平章政事、商議樞密院事。丁酉,太尉、尚書右丞相脫虎脫監修國史。己亥,太陰犯右執法。庚子,太陰犯上相。辛丑,尚書省臣言:「臣等竊計,國之糧儲,歲費浸廣,而所入不足。今歲江南頗熟,欲遣使和糴,恐米價暴增,請以至大鈔二千錠分之江浙、河南、江西、湖廣四省,於來歲諸色應支糧者,視時直予以鈔,可得百萬,不給則聽以各省錢足之。」制可。丙午,諸王孛蘭奚以私怨殺人,當死,大宗正也可扎魯忽赤議,孛蘭奚貴為國族,乞杖之,流北鄙從軍,從之。丁未,擇衛士子弟充國子學生。



 十二月己卯,親饗太廟,上太祖聖武皇帝尊謚、廟號及光獻皇后尊謚,又上睿宗景襄皇帝尊謚、廟號及莊聖皇后尊謚,執事者人升散階一等,賜太廟禮樂戶鈔帛有差。和林省右丞相、太師月赤察兒言:「臣與哈剌哈孫答剌罕共事時,錢穀必與臣議。自哈剌哈孫沒,凡出入不復關聞,予奪失當,而右丞曩家帶反相凌侮,輒托故赴京師。」有旨:「其鎖曩家帶詣和林鞫之。」武昌婦人劉氏,詣御史臺訴三寶奴奪其所進亡宋玉璽一、金椅一、夜明珠二,奉旨,令尚書省臣及御史中丞冀德方、也可扎魯忽赤別鐵木兒、中政使搠只等雜問。劉氏稱故翟萬戶妻,三寶奴謫武昌時,與劉往來,及三寶奴貴,劉托以追逃婢來京師,謁三寶奴於其家,不答,入其西廊,見榻上有逃婢所竊寶鞍及其手縫錦帕,以問,三寶奴又不答,忿恨而出,即求書狀人喬瑜為狀,乃因尹榮往見察院吏李節,入訴於臺。獄成,以劉氏為妄。有旨,斬喬瑜,笞李節,杖劉氏及尹榮,歸之元籍。丙辰,並中書省左右司。遣使往諸路分揀逋負,合征者征之,合免者免之。庚申,太陰犯參。尚書省臣言:「鹽價每引宜增為至大銀鈔四兩,廣西者如故,其煮鹽工本,請增為至大銀鈔四錢。」制可。辛酉,申禁漢人執弓矢、兵仗。壬戌,陽曲縣地震,有聲如雷。封西僧迷不韻子為寧國公,賜金印。丁丑,詔:「增百官俸,定流官封贈等第。應封贈者,或使遠死節,臨陣死事,於見授散官上加之。若六品七品死節死事者,驗事特贈官。封贈內外百官,三品以上者許請謚。凡請謚者,許其家具本官平日勛勞、政績、德業、藝能,經由所在官司保勘,與本家所供相同,轉申吏部考覆呈都省,都省準擬,令太常禮儀院驗事跡定謚。若勛戚大臣奉旨賜謚者,不在此例。」



 三年春正月癸未,省中書官吏,自客省使而下一百八十一員。賜諸王那木忽裏等鈔萬二千錠,賜宣徽院使拙忽難所隸酒人鈔萬五百八十八錠。乙酉,特授李孟榮祿大夫、平章政事、集賢大學士、同知徽政院事。丁亥,白虹貫日。戊子,禁近侍諸人外增課額及進他物有妨經制。營五臺寺,役工匠千四百人、軍三千五百人。己丑,以紐鄰參議尚書省事。庚寅,立司禋監,秩正三品,掌巫覡,以丞相厘日領之。辛卯,立皇後弘吉列氏,遣脫虎脫攝太尉持節授玉冊、玉寶。壬辰,升中政院為從一品。癸巳,立中瑞司,秩正三品,掌皇後寶。甲午,太陰犯右執法。乙未,定稅課法,諸色課程,並系大德十一年考較,定舊額、元增,總為正額,折至元鈔作數。自至大三年為始恢辦,餘止以十分為率,增及三分以上為下酬,五分以上為中酬,七分以上為上酬,增及九分為最,不及三分為殿。所設資品官員,以二周歲為滿。定稅課官等第,萬錠之上,設正提舉、同提舉、副提舉各一員;一千錠之上,設提領、大使、副使各二員;五百錠之上,設提領、大使、副使各一員;一百錠之上,設大使、副使各一員。丙申,立資國院泉貨監,命以歷代銅錢與至大錢相參行用。復立廣平順德路鐵冶都提舉司。戊戌,詔湖廣行省招諭叛人上思州知州黃勝許。辛丑,降詔招諭大徹里、小徹里。樞密院臣言:「湖廣省乖西帶蠻阿馬等連結萬人入寇,已遣萬戶移剌四奴領軍千人,及調思、播土兵並力討捕。臣等議,事勢緩急,地裡要害,四奴備知,乞聽其便宜調遣。」制可。壬寅,詔諭八百媳婦,遣雲南行省右丞算只兒威招撫之。癸卯,改太子少詹事為副詹事。乙巳,令中書省官吏如安童居中書時例存設,其已汰者,尚書省遷敘。省樞密院官,存知樞密院七員、同知樞密院事二員、樞密副使二員、僉樞密院事二員、同僉樞密院事一員。增御史臺官二員,御史大夫、御史中丞、侍御史、治書侍御史各二員。省通政院官六員,存十二員。汰廣武康里衛軍,非其種者還之元籍,凡隸諸王阿只吉、火郎撒及迤南探馬赤者,令樞密院遣人即其處參定為籍。去歲朝會,諸王伯鐵木兒、阿剌鐵木兒並賜金二百五十兩、銀一千兩、鈔四百錠。丙午,詔令知樞密院事大都、僉院合剌合孫復職。丁未,立右衛阿速親軍都指揮使司,秩正三品。



 二月庚戌,以皇后受冊,遣官告謝太廟。辛亥,熒惑犯月星。賜鷹坊馬速忽金百兩、銀五百兩。己未,浚會通河,給鈔四千八百錠、糧二萬一千石以募民,命河南省平章政事塔失海牙董其役。遣商議尚書省事劉楫整治鈔法。增大都警巡院二,分治四隅。壬戌,太陰犯左執法。甲子,以上皇太后尊號,告祀南郊。乙丑,復以僉樞密院事賈鈞為中書參知政事。尚書省臣言:「官階差等,已有定制,近奉聖旨、懿旨、令旨要索官階者,率多躐等,願依世祖皇帝舊制,次第給之。」制可。丁卯,尚書省臣言:「昔至元鈔初行,即以中統鈔本供億及銷其板。今既行至大銀鈔,乞以至元鈔輸萬億庫,銷毀其板,止以至大鈔與銅錢相權通行為便。」又言:「今夏朝會上都供億,請先發鈔百萬錠以往。」並從之。楚王牙忽都所隸戶貧乏,以米萬石、鈔六千錠賑之。己巳,寧王闊闊出謀為不軌,越王禿剌子阿剌納失里許助力,事覺,闊闊出下獄,賜其妻完者死,竄阿剌納失里及其祖母、母、妻於伯鐵木兒所。以畏吾兒僧鐵裡等二十四人同謀,或知謀不首,並磔於市;鞫其獄者,並升秩二等。賞牙忽都金千兩、銀七千五百兩。三寶奴賜號答剌罕,以闊闊出食邑清州賜之,自達魯花赤而下,並聽舉用。辛未,脫兒赤顏加錄軍國重事,賜故中書右丞相塔剌海妻也裡乾金七百五十兩、銀一千五百兩、鈔四百錠。壬申,樂實為尚書左丞相、駙馬都尉,封齊國公。癸酉,以左丞相、行中書省平章政事合散商議遼陽行省事。乙亥,太白犯月星。以上皇太后尊號,告祀太廟。



 三月己卯朔,樞密院臣言:「國家設官分職,都省治金谷,樞密治軍旅,各有定制。邇者尚書省弗遵成憲,易置本院官,令依大德十年員數聞奏。臣等議,以鐵木兒不花、朵而赤顏、床兀兒、也速、脫脫、也兒吉尼、脫不花、大都知樞密院事,撒的迷失、史弼同知樞密院事,吳元珪樞密副使,塔海姑令為副樞。」有旨,令樞密院如舊制設官十七員。乙酉,以知樞密院事只兒合郎為陜西行尚書省平章政事。遣刑部尚書馬兒往甘肅和市羊馬,分賚諸王那木忽里蒙古軍,給鈔七萬錠。庚寅,太陰犯氐。尚書省臣言:「昔世祖有旨,以叛王海都分地五戶絲為幣帛,俟彼來降賜之,藏二十餘年。今其子察八兒向慕德化,歸覲闕廷,請以賜之。」帝曰:「世祖謀慮深遠若是,待諸王朝會,頒賞既畢,卿等備述其故,然後與之,使彼知愧。」辛卯,發康里軍屯田永平,官給之牛。壬辰,車駕幸上都。立興聖宮章慶使司,秩正二品。丙申,太陰犯南斗。丁未,太白犯井。



 夏四月己酉,興聖宮鷹坊等戶四千分處遼陽,建萬戶府以統之。容米洞官田墨糾合蠻酋,殺千戶及戍卒八十餘人,俘掠良民;改永順保靖南渭安撫司為永順等處軍民安撫司,以安撫副使梓材為使往招之。賜高麗國王王章功臣號,改封沈王。改大承華普慶寺總管府為崇祥監。庚戌,以鈔九千一百五十八錠有奇市耕牛農具,給直沽酸棗林屯田軍。戊辰,太白晝見。己巳,立怯憐口諸色人匠都總管府,秩正三品,提舉司二,分治大都、上都,秩正五品;江浙等處財賦提舉司,秩從五品;瑞州等路營民都提舉司,秩從四品,並隸章慶使司。辛未,賜角牴者阿裏銀千兩、鈔四百錠。丙子,立管領軍匠千戶所,秩正五品,割左都威衛軍匠八百隸之,備興聖宮營繕。增國子生為三百員。靈壽、平陰二縣雨雹,鹽山、寧津、堂邑、茌平、陽谷、高唐、禹城等縣蝗。



 五月甲申,封諸王完者為衛王。癸巳,東平人饑,賑米五千石。乙未,加尚書參知政事王羆大司徒。是月,合肥、舒城、歷陽、蒙城、霍丘、懷寧等縣蝗。六月丁未朔,詔太尉、尚書右丞相脫虎脫,太保、尚書左丞相三寶奴總治百司庶務,並從尚書省奏行。戊申,省上都留守司官七員。以行中書左丞忽都不丁為中書右丞。己酉,立上都、中都等處銀冶提舉司,秩正四品。尚書省臣言:「別都魯思雲雲州朝河等處產銀,令往試之,得銀六百五十兩。」詔立提舉司,以別都魯思為達魯花赤。庚戌,立規運都總管府,秩正三品,領大崇恩福元寺錢糧,置提舉司、資用庫、大益倉隸之。乙卯,太陰犯氐。和林省言:「貧民自迤北來者,四年之間靡粟六十萬石、鈔四萬餘錠、魚網三千、農具二萬。」詔尚書、樞密差官與和林省臣核實,給賜農具田種,俾自耕食,其續至者,戶以四口為率給之粟。丁巳,敕今歲諸王、妃主朝會,頒賚一如至大元年例。甲子,以太子詹事斡赤為中書左丞、集賢使,領典醫監事。戊辰,遣使諸道,審決重囚。賜太師淇陽王月赤察兒清州民戶萬七千九百一十九,安吉王乞臺普濟安吉州民戶五百。壬申,以西北諸王察八兒等來朝,告祀太廟。賜脫虎脫、三寶奴珠衣,封三寶奴為楚國公,以常州路為分地。乙亥,升晉王延慶司秩正二品。是月,襄陽、峽州路、荊門州大水,山崩,壞官廨民居二萬一千八百二十九間,死者三千四百六十六人。汝州大水,死者九十二人。六安州大水,死者五十二人。沂州、莒州、兗州諸縣水,沒民田。威州、洺水、肥鄉、雞澤等縣旱。



 秋七月戊寅,太陰犯右執法。己卯,太陰犯上相。庚辰,封皇伯晉王長女寶答失憐為韓國長公主。丙戌,循州大水,漂廬舍二百四十四間,死者四十三人,發米賑之。庚寅,罷稱海也可札魯忽赤。定王藥木忽兒乞如例設王府官六員,從之。癸巳,給親民長吏考功印歷,令監治官歲終驗其行跡,書而上之,廉訪司、御史臺、尚書禮部考校以為升黜。增尚書省客省使、副各一員,直省舍人十四員。立河南打捕鷹坊、魚課都提舉司,秩正四品。乙未,中都立光祿寺。丁酉,汜水、長林、當陽、夷陵、宜城、遠安諸縣水,令尚書省賑恤之。己亥,禁權要商販挾聖旨、懿旨、令旨阻礙會通河民船者。壬寅,詔禁近侍奏降御香及諸王駙馬降香者。磁州、威州諸縣旱、蝗。



 八月丁未,以江浙行尚書省左丞相忽剌出、遙授中書右丞相厘日,並為御史大夫,詔諭中外。甲寅,白虹貫日。升尚服院從一品。丙辰,以行用銅錢詔諭中外。甲子,獵於昴兀腦兒之地。己巳,以諸王只必鐵木兒貧,仍以西涼府田賜之。尚書省臣言:「今歲頒賚已多,凡各位下奉聖旨、懿旨、令旨賜財物者,請分汰。」有旨:「卿等但具名以進,朕自分汰之。」汴梁、懷孟、衛輝、彰德、歸德、汝寧、南陽、河南等路蝗。九月己卯,平伐蠻酋不老丁遣其侄與甥十人來降,升平伐等處蠻夷軍民安撫司同知陳思誠為安撫使,佩金虎符。御史臺臣言:「江浙省丞相答失蠻於天壽節日毆其平章政事孛蘭奚,事屬不敬。」詔遣使詰問之。內郡饑,詔尚書省如例賑恤。辛巳,太陰犯建星。立宣慰司都元帥府於察罕腦兒之地。丙戌,車駕至大都。保八遙授平章政事。辛卯,太陰犯天廩。壬辰,皇太子言:「司徒劉夔乘驛省親江南,大擾平民,二年不歸。」詔罷之。庚子,以潭州隸中宮。上都民饑,敕遣刑部尚書撒都丁發粟萬石,下其價賑糶之。壬寅,敕諸司官濫設者,毋給月俸。詔諭三寶奴等:「去歲中書省奏,諸司官員遵大德十年定制,濫者汰之。今聞員冗如故,有不以聞而徑之任者。有旨不奏而擅令之任及之任者,並逮捕之,朕不輕釋。」



 冬十月甲辰朔,太白經天。丙午,太白犯左執法。三寶奴及司徒田忠良等言:「曩奉旨舉行南郊配位從祀,北郊方丘、朝日夕月典禮。臣等議,欲祀北郊,必先南郊。今歲冬至,祀圜丘,尊太祖皇帝配享,來歲夏至,祀方丘,尊世祖皇帝配享,春秋朝日夕月,實合祀典。」有旨:「所用儀物,其令有司速備之。」又言:「太廟祠祭,故用瓦尊,乞代以銀。」從之。戊申,帝率皇太子、諸王、群臣朝興聖宮,上皇太后尊號冊寶曰儀天興聖慈仁昭懿壽元皇太后。庚戌,恭謝太廟。癸丑,熒惑犯亢。甲寅,敕諭中外:「民戶托名諸王、妃主、貴近臣僚,規避差徭,已嘗禁止。自今違者,俾充軍驛及築城中都。郡縣官不覺察者,罷職。」封僧亦憐真乞烈思為文國公,賜金印。御史臺臣言:「江浙省平章烏馬兒遣人從使臣暱匝馬丁枉道馳驛,取贓吏紹興獄中釋之。」敕臺臣遣官往鞫,毋徇私情。山東、徐、邳等處水、旱,以御史臺沒入贓鈔四千餘錠賑之。丁巳,尚書省臣言:「宣徽院廩給日增,儲偫雖廣,亦不能給,宜加分減。」帝曰:「比見後宮飲膳,與朕無異,有是理耶?其令伯答沙與宣徽院官核實分減之。」庚申,敕:「尚書省事繁重,諸司有才識明達者,並從尚書省選任,樞密院、御史臺及諸有司毋輒奏用,違者論罪。其或私意請托,罷之不敘。」辛酉,以皇太后受尊號,赦天下。大都、上都、中都比之他郡,供給繁擾,與免至大三年秋稅。其餘去處,今歲被災人戶,曾經體覆,依上蠲免。內外不急之役,截日停罷。至大二年已前民間負欠差稅、課程,並行蠲免。闊闊出餘黨未發覺者,並原其罪。隨處官民田土各有所屬,諸人勿得陳獻。三寶奴言省部官不肯勤恪署事,敕:「自今晨集暮退,茍或怠弛,不必以聞,便宜罪之。其到任或一再月辭以病者,杖罷不敘。」又言:「故丞相和禮霍孫時,參議府左右司斷事官、六部官日具一膳,不然則抱饑而還,稽誤公事,今則無以為資,乞各賜鈔二百錠,規運取其息錢以為食。」制可。丁卯,封諸王木八剌子買住韓為兗王。壬申,晉王也孫鐵木兒言:「世祖以張鐵木兒所獻地土、金銀、銅冶賜臣,後以成宗拘收諸王所占地土民戶,例輸縣官,乞回賜。」從之,仍賜鈔三千錠賑其部貧民。江浙省臣言:「曩者硃清、張宣海漕米歲四五十萬至百十萬,時船多糧少,顧直均平。比歲賦斂橫出,漕戶困乏,逃亡者有之。今歲運三百萬,漕舟不足,遣人於浙東、福建等處和顧,百姓騷動。本省左丞沙不丁,言其弟合八失及馬合謀但的、澉浦楊家等皆有舟,且深知漕事,乞以為海道運糧都漕萬戶府官,各以己力輸運官糧,萬戶、千戶並如軍官例承襲,寬恤漕戶,增給顧直,庶有成效。」尚書省以聞,請以馬合謀但的為遙授右丞、海外諸蕃宣慰使、都元帥、領海道運糧都漕運萬戶府事,設千戶所十,每所設達魯花赤一、千戶三、副千戶二、百戶四,制可。雲南省丞相鐵木迭兒擅離職赴都,有旨詰問,以皇太后旨貸免,令復職。以丞相鐵古迭兒為陜西行御史臺御史大夫,詔諭陜西、四川、雲南、甘肅。詔諭大司農司勸課農桑。



 十一月甲戌朔,太白犯亢。戊寅,濟寧、東平等路饑,免曾經賑恤諸戶今歲差稅,其未經賑恤者,量減其半。詔諭厘日移文尚書省,凡憲臺除官事,後勿與。庚辰,河南水,死者給槥,漂廬舍者給鈔,驗口賑糧兩月,免今年租賦,停逋責。辛巳,尚書省臣言:「今歲已印至大鈔本一百萬錠,乞增二十萬錠,及銅錢兼行,以備侍衛及鷹坊急有所須。」又言:「上都、中都銀冶提舉司達魯花赤別都魯思,去歲輸銀四千二百五十兩,今秋復輸三千五百兩,且言復得新礦,銀當增辦,乞加授嘉議大夫。」並從之。加脫虎脫為太師、錄軍國重事,封義國公。壬午,改大崇恩福元寺規運總管府為隆禧院,秩從二品。丁亥,太陰犯畢。戊子,改皇太子妃怯憐口都總管府為典內司。以益都、寧海等處連歲饑,罷鷹坊縱獵,其餘獵地,並令禁約,以俟秋成。尚書省臣言:「雲南省臨安、大理等處宣慰司、麗江宣撫司及普定路所隸部曲,連結蠻寇,殺掠良民,諭之不服,且方調兵討八百媳婦,軍力消耗。今擬蒙古軍人給馬一,漢軍十人給馬二,計直與之,乞賜鈔三萬錠。」又言:「四川行省紹慶路所隸容米洞田墨,連結諸蠻,攻劫麻寮等寨,方調兵討捕,遣千戶塔術往諭田墨施什用等來降。宜立黃沙寨,以田墨施什用為千戶,塔術為河東陜西等處萬戶府千戶所達魯花赤,廖起龍為來寧州判官,田思遠為懷德府判官,賞賚遣還。」皆從之。以硃清子虎、張瑄子文龍往治海漕,以所籍宅一區、田百頃給之。尚書省臣言:「昔世祖命皇子脫歡為鎮南王居揚州,今其子老章,出入導衛,僭竊上儀。敕遣官詰問,仍以所僭儀物來上。」從之。敕城中都,以牛車運土,令各部衛士助之,限以來歲四月十五日畢集,失期者罪其部長,自願以車牛輸運者別賞之。江浙省左丞相答失蠻、江西省左丞相別不花來朝。賜世祖宮人伯牙倫金七百五十兩、銀二千五百兩、鈔六百錠。丙申,有事於南郊,尊太祖皇帝配享昊天上帝。己亥,尚書省以武衛親軍都指揮使鄭阿兒思蘭與兄鄭榮祖、段叔仁等圖為不軌,置獄鞫之,皆誣服,詔叔仁等十七人並正典刑,籍沒其家。



 十二月甲辰朔,以建大崇恩福元寺,乞失剌遙授左丞,曲列、劉良遙授參知政事,並領行工部事。立崇輝署,隸中政院。戊申,冀寧路地震。己未,諭中外應避役占籍諸王者,俾充軍驛。鎮南王老章僭擬儀衛,究問有驗,召老章赴闕。



 四年春正月癸酉朔,帝不豫,免朝賀,大赦天下。庚辰,帝崩於玉德殿,在位五年,壽三十一。壬午,靈駕發引,葬起輦穀,從諸帝陵。



 夏四月乙未,文武百官也先鐵木兒等上尊謚曰仁惠宣孝皇帝,廟號武宗,國語曰曲律皇帝。是日,請謚南郊。



 閏七月丙午,祔於太廟。



 武宗當富有之大業,慨然欲創治改法而有為,故其封爵太盛,而遙授之官眾,錫賚太隆,而泛賞之恩溥,至元、大德之政,於是稍有變更云。



\end{pinyinscope}