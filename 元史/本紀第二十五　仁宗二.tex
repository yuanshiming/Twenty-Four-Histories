\article{本紀第二十五 仁宗二}

\begin{pinyinscope}

 延祐元年春正月丁亥,授中書右丞劉正平章政事、商議中書省事。丙申,除四川酒禁。興元、鳳翔、涇州、邠州歲荒,禁酒。庚子,敕各省平章為首者及漢人省臣一員洛克承認有「反時觀念和上帝的存在。體現了作者全部哲,專意訪求遺逸,茍得其人,先以名聞,而後致之。以江浙行中書省左丞高昉為中書參知政事。丁未,詔改元延祐。釋天下流以下罪囚,免上都、大都差稅二年,其餘被災曾經賑濟人戶,免差稅一年。庚戌,中書省臣禿忽魯等以災變乞罷免,不允。



 二月庚申,立印經提舉司。戊辰,大寧路地震。癸酉,熒惑犯東井。甲戌,以侍御史趙世延為中書參知政事。詔免蒙古地差稅二年,商賈勿免。己卯,給鈔六千三百錠,賑濟良鄉諸驛。壬午,以合散為中書右丞相、監修國史。癸未,以中書參政高昉為集賢學士。



 三月壬辰,太陰掩熒惑。賜諸王塔失蒙古鈔千錠、衣二襲。戊戌,真定、保定、河間民饑,給糧兩月。己亥,白暈亙天,連環貫日。癸卯,暹國王遣其臣愛耽入貢。改南劍路曰延平,劍浦縣曰南平。乙巳,以僧人作佛事,擇釋獄囚,命中書審察。丙午,封阿魯禿為趙王。戊申,車駕幸上都。己酉,敕:「奸民宮其子為閹宦,謀避徭役者,罪之。」辛亥,命參知政事趙世延綱領國子學。癸丑,中書平章政事察罕致仕。晉寧民侯喜兒昆弟五人,並坐法當死,帝嘆曰:「彼一家不幸而有是事,其擇情輕者一人杖之,俾養父母,毋絕其祀。」



 閏三月甲寅朔,敕減樞密知院冗員。辛酉,太陰犯輿鬼。罷咒僧月給俸。遣人視大都至上都駐蹕之地,有侵民田者,計畝給直。丙寅,太陰犯太微東垣。丁丑,畿內及諸衛屯軍饑,賑鈔七千五百錠。汴梁、濟寧、東昌等路,隴州、開州,青城、齊東、渭源、東明、長垣等縣,隕霜殺桑果禾苗,歸州告饑,出糧減價賑糶。馬八兒國主昔剌木丁遣其臣愛思丁貢方物。



 夏四月甲申朔,大寧路地震,有聲如雷。丁亥,敕儲稱海、五河屯田粟,以備賑濟。太常寺臣請立北郊,不允。升延慶寺秩正二品。西番諸驛貧乏,給鈔萬錠。曲魯部畜牧斃耗,賑鈔八百七十三錠。己丑,廢真陽、含光二縣,入英德州。壬辰,諸王脫脫薨,以月思別襲位。己酉,敕:「郡縣官勤職者,加賜幣帛。」以鐵木迭兒錄軍國重事,監修國史。立回回國子監。帝以《資治通鑒》載前代興亡治亂,命集賢學士忽都魯都兒迷失及李孟擇其切要者譯寫以進。武昌路饑,命發米減價賑糶。



 五月甲寅朔,賜營王也先鐵木兒鈔萬錠。戊午,辰星犯輿鬼。丁卯,賜李孟孝感縣地二十八頃。禁諸王支屬徑取分地租賦擾民。敕嶺北行省瘞陣沒遺骼。乙亥,賑怯魯連地貧乏者米三千石。丁丑,徙滄州治於長蘆鎮。戊寅,京兆為故儒臣許衡立魯齋書院,降璽書旌之。庚辰,盧陽、麻陽二縣以土賊作耗,蠲其地稅賦。營王也先鐵木兒支屬貧乏,賑糧兩月。武陵縣霖雨,水溢,溺死居民,漂沒廬舍禾稼,潭州、漢陽、思州民饑,並發廩減價糶賑之。膚施縣大風、雹,損禾並傷人畜。六月戊子,敕:「內侍今後止授中官,勿畀文階。」置雲南行省儒學提舉司。封河南省丞相卜憐吉帶為河南王。壬辰,增置畿內州縣同知、主簿各一員。諸王察八兒屬戶匱乏,給糧一歲,仍俾屯田以自贍。發軍增墾河南芍陂等處屯田。乙未,熒惑犯右執法。戊申,增置兩浙鹽運司判官一員。甲辰,拘河西僧免輸租賦璽書。敕:「諸王、戚里入覲者,趁夏時芻牧至上都,毋輒入京師,有事則遣使奏稟。」衡州、郴州、興國、永州路、耒陽州饑,發廩減價賑糶。宣平、仁壽、白登縣雹損稼,傷人畜。



 秋七月乙卯,答即乃所部匱乏,戶給糧二石。庚午,命中書省臣議復封贈,賜晉王也孫鐵木兒部鈔十萬錠。詔開下番市舶之禁。賜衛王阿木哥等鈔七千錠。乙亥,會福院越制奏旨除官,敕自今舉人,聽中書可否以聞。申飭私鹽之禁。沅陵、盧溪二縣水,武清縣渾河堤決,淹沒民田,發廩賑之。



 八月戊子,車駕至大都。癸卯,升太常寺為太常禮儀院,秩正二品。丁未,冀寧、汴梁及武安、涉縣地震,壞官民廬舍,武安死者十四人,涉縣三百二十六人。臺州、岳州、武岡、常德、道州等路水,發廩減價賑糶。九月壬戌,改提點教坊司事為大使。己巳,復以鐵木迭兒為右丞相,合散為左丞相。罷陜西諸道行御史臺,降儀鳳卿為儀鳳大使。肇慶、武昌、建德、建康、南康、江州、袁州、建昌、贛州、杭州、撫州、安豐等路水,發廩減價賑糶。



 冬十月癸巳,升潁州萬戶府為中萬戶府。乙未,敕:「吏人轉官,止從七品,在選者降等注授。」申飭內侍及諸司隔越中書奏請之禁。敕:「下番商販須江浙省給牒以往,歸則徵稅如制,私往者沒其物。」遣官括淮民所佃閑田不輸稅者。丙申,復甘肅屯田,置沙瓜等處屯儲總管萬戶府,秩正三品。乙巳,置恩平王塔思不花傅二人。庚戌,辰星犯東咸。監察御史言:「乞命樞密院設法教練士卒,應軍官襲職者,試以武事而後任之。」制曰:「可。」遣張驢經理江南田糧。



 十一月壬子,升司天臺為司天監,秩正三品,賜銀印。乙卯,改大同侍衛親軍都指揮使司為中都威衛使司。置保安軍於麻陽縣以御徭蠻。戊辰,以通政院使蕭拜住為中書右丞。辛未,以翰林學士承旨答失蠻知樞密院事。癸酉,敕:「吏人賊行者黥其面。」大寧路地震,有聲如雷。戊寅,鐵木迭兒言:「比者僚屬及六部諸臣,皆晚至早退,政務廢弛。今後有如此者,視其輕重杖責之。臣或自惰,亦令諸人陳奏。」帝曰:「如更不悛,則罷不敘。」以前中書右丞相禿忽魯知樞密院事。靜安路饑,發糧賑之。詔檢核浙西、江東、江西田稅。



 十二月壬午,汴梁、南陽、歸德、汝寧、淮安水,敕禁釀酒,量加賑恤。癸未,賑諸王鐵木兒不花部米五千石,禿滿部二千石。辛卯,禁諸王、駙馬、權勢之人增價鬻鹽。壬辰,詔定官員士庶衣服車輿制度。甲午,太陰犯輿鬼。己亥,敕中書省定議孔子五十三代孫當襲封衍聖公者以名聞。庚子,遣官浚楊州、淮安等處運河,以翰林學士承旨李孟復為中書平章政事。癸卯,太陰犯房。甲辰,太陰犯天江。乙巳,敕經界諸衛屯田。沔陽、歸德、汝寧、安豐等處饑,發米賑之。



 二年春正月乙卯,歲星犯輿鬼。戊午,懷孟、衛輝等處饑,發米賑之。己未,太白晝見。癸亥,太陰犯軒轅。丙寅,霖雨壞渾河堤堰,沒民田,發卒補之。禁民煉鐵。發卒浚漷州漕河。丁卯,太陰犯進賢。戊辰,晉寧等處民饑,給鈔賑之。己巳,置大聖壽萬安寺都總管府,秩正三品。庚午,立行用庫於江陰州。敕以江南行臺贓罰鈔賑恤饑民。乙亥,詔遣宣撫使分十二道問民疾苦,黜陟官吏,並給銀印。命中書省臣分領庶務。禁南人典質妻子販買為驅。御史臺臣言:「比年地震水旱,民流盜起,皆風憲顧忌失於糾察,宰臣燮理有所未至,或近侍蒙蔽,賞罰失當,或獄有冤濫,賦役繁重,以致乖和。宜與老成共議所由。」詔明言其事當行者以聞。諸王脫列鐵木兒部闕食,以鈔七千五百錠給之。益都、般陽、晉寧民饑,給鈔、米賑之。



 二月己卯朔,會試進士。戊子,太白晝見。癸巳,太白經天。甲午,詔禁民轉鬻養子。丙申,賜諸王納忽答兒金五十兩、銀二百五十兩、鈔五百錠。庚子,詔以公哥羅古羅思監藏班藏卜為帝師,賜玉印,仍詔天下。壬寅,雲南王老的來朝。辰、沅洞蠻吳千道為寇,敕調兵捕之。乙巳,賜諸王月魯鐵木兒鈔萬錠。丙午,太白經天。是月,晉寧、宣德等處饑,給米、鈔賑之。真州揚子縣火,發米減價賑糶。



 三月乙卯,廷試進士,賜護都沓兒、張起巖等五十六人及第、出身有差。丙辰,太陰色赤如赭。庚午,帝率諸王、百官奉玉冊、玉寶,加上皇太后尊號,詔天下蠲逋欠稅課。丁丑,以中書平章張驢為江浙行省平章政事。



 夏四月戊寅朔,日有食之。辛巳,賜進士恩榮宴於翰林院。癸巳,敕亦思丹等部出征軍,有後期及逃還者,並斬以徇。甲午,諭晉王也孫鐵木兒,以先朝所賜惠州銀礦洞歸還有司。庚子,太陰犯壘壁陣。辛丑,賜會試下第舉人七十以上從七流官致仕,六十以上府、州教授,餘並授山長、學正,後勿援例。敕諸王分地仍以流官為達魯花赤,各位所闢為副達魯花赤。命李孟等類集累朝條格,俟成書,聞奏頒行。立規運提點所,秩五品,置官四員;廣貯庫,秩七品,置官三員;並隸壽福院。乙巳,車駕幸上都。宣徽院以供尚膳,遣人獵於歸德,敕以其擾民,特罷之。加授特進上卿、玄教大宗師張留孫開府儀同三司。丙午,封諸王察八兒為汝寧王。潭州、江州、建昌、沅州饑,發廩賑糶。



 五月戊申朔,改給各道廉訪司銀印,復立陜西諸道行御史臺。貴赤張小廝等招戶六千,勒還民籍。御史中丞王毅乞歸養親,不許。庚申,賜公主燕海牙鈔千錠。辛酉,太陰犯天江。乙丑,秦州成紀縣山移。是夜,疾風電雹,北山南移至夕河川,次日再移,平地突出土阜,高者二三丈,陷沒民居。敕遣官核驗賑恤。庚午,太白晝見。立海西、遼東鷹坊萬戶府,隸中政院。壬申,諸王撒都失里薨。甲戌,日赤如赭。加授宦者中尚卿續元暉昭文館大學士。乙亥,日赤如赭。是月,發粟三百石,賑諸王按鐵木兒等部貧民。奉元、龍興、吉安、南康、臨江、袁州、撫州、江州、建昌、贛州、南安、梅州、辰州、興國、潭州、岳州、常德、武昌等路,南豐州、澧州等處饑,並發廩賑糶。六月辛巳,察罕腦兒諸驛乏食,給糧賑之。甲申,太白晝見,是夜太陰犯平道。乙未,徙陜西肅政廉訪司於鳳翔。戊戌,豳王南忽裏等部困乏,給鈔俾買馬羊以濟之。河決鄭州。己亥,置汝寧王察八兒王傅官。辛丑,以濟寧、益都亢旱,汰省宿衛士芻粟。癸卯,太白犯東井。丙午,辰星犯輿鬼。緬國主遣其子脫剌合等來貢方物。



 秋七月庚戌,增興和路治中一員。戊申,賜宣寧王鐵木兒不花及其二弟鈔萬錠,並玉具、鞍勒、幣帛。壬子,增尚舍寺官六員為八員,雲需總管府增同知二員。癸丑,復賜晉王也孫鐵木兒惠州銀鐵洞。甲寅,置諸王斡羅溫孫王傅官四員,復陳州商水鎮為南頓縣,省兩淮屯田總管府官四員,並提領所入提舉司,改只合赤八剌合孫總管府為尚供府。乙卯,贛州土賊蔡五九聚眾作亂,敕遣兵捕之。敕阿速衛戶貧乏者,給牛、種、耕具,於連怯烈地屯田。甲子,江南湖廣道奉使溫迪罕言:「廉訪司公田多取民租,宜復舊制。」從之。乙丑,升崇福院秩正二品。癸酉,賜衛王阿木哥鈔萬錠。命鐵木迭兒總宣政院事,詔諭中外。是月,畿內大雨,漷州、昌平、香河、寶坻等縣水,沒民田廬;潭州、全州、永州路、茶陵州霖雨,江漲,沒田稼,出米減價賑糶。



 八月丙戌,贛州賊蔡五九陷汀州寧化縣,僭稱王號,詔遣江浙行省平章張驢等率兵討之。己丑,車駕至自上都。乙未,臺臣言:「蔡五九之變,皆由暱匝馬丁經理田糧,與郡縣橫加酷暴,逼抑至此。新豐一縣,撤民廬千九百區,夷墓揚骨,虛張頃畝,流毒居民,乞罷經理及冒括田租。」制曰:「可。」庚子,改遼陽省泰州為泰寧府。壬寅,增國子生百員,歲貢伴讀四員。詔江浙行省印《農桑輯要》萬部,頒降有司遵守勸課。旌表貴州達魯花赤相元孫妻脫脫真死節,仍俾樹碑任所。九月丁未,張驢以括田逼死九人,敕吏部尚書王居仁等鞫之。己酉,太陰犯房。甲寅,日色如赭。辛酉,太白犯左執法。壬戌,蔡五九眾潰伏誅,餘黨悉平,敕賞軍士討捕功,並官其死事者子孫。己巳,徙典尤倉於赤斤之地。賜諸王別鐵木兒永昌路及西涼州田租。



 冬十月丙子朔,客星見太微垣。丁丑,封脫火赤為威寧郡王,賜金印,忽兒赤鐵木兒不花為趙國公。庚辰,以淮西廉訪使郭貫為中書參知政事。壬午,有事於太廟。給雲南廉訪司公田。乙未,升同知樞密院事鐵木兒脫知樞密院事。授白云宗主沈明仁榮祿大夫、司空。丁酉,加授鐵木迭兒太師。癸卯,八百媳婦蠻遣使獻馴象二,賜以幣帛。



 十一月丙午,客星變為彗,犯紫微垣,歷軫至壁十五宿,明年二月庚寅乃滅。辛未,以星變赦天下,減免各路差稅有差。甲戌,封和世束為周王,賜金印。左丞相合散等言:「彗星之異,由臣等不才所致,願避賢路。」帝曰:「此朕之愆,豈卿等所致?其復乃職,茍政有過差,勿憚於改。凡可以安百姓者,當悉言之,庶上下交修,天變可弭也。」十二月戊寅,賜雲南行省參政汪長安虎符,預軍政。庚寅,增置平江路行用庫。癸巳,給鈔買羊馬,賑北邊諸軍。命省臣定擬封贈通例,俾高下適宜以聞。旌表汀州寧化縣民賴祿孫孝行。



 三年春正月乙巳,漢陽路饑,出米賑之。特授昔寶赤八剌合孫達魯花赤脫歡金紫光祿大夫、太尉,仍給印。丙午,封前中書左丞相忽魯答兒壽國公,增置晉王部斷事官四員,都水太監二員,省卿一員。以真定、保定薦饑,禁畋獵。改直沽為海津鎮。辛酉,升同知樞密院事買閭知院事。壬戌,賜上都開元寺江浙田二百頃,華嚴寺百頃,賜趙王阿魯禿部鈔二萬錠。



 二月丁丑,調海口屯儲漢軍千人,隸臨清運糧萬戶府,以供轉漕,給鈔二千錠。戊寅,命湖廣行省諭安南,歸占城國主。置安遠王丑漢王傅。河間、濟南、濱棣等處饑,給糧兩月。



 三月辛亥,特授高麗王世子王暠開府儀同三司、沈王,加授將作院使呂天麟大司徒。甲寅,敕蕭拜住及陜西、四川省臣各一員護送周王之云南,置周王常侍府,秩正二品,設常侍七員,中尉四員,諮議、記室各二員。置打捕鷹坊民匠總管府,設官六員,斷事官八員;延福司、飲膳署官各六員;並隸周王常侍府。辛酉,升太史院秩正二品。癸亥,車駕幸上都。壬申,鷹坊孛羅等擾民於大同,敕拘還所奉璽書。禁天下春時畋獵。



 夏四月癸酉朔,賜皇姊大長公主鈔五千錠、幣帛二百匹。河南流民群聚渡江,所過擾害,命行臺、廉訪司以見貯贓鈔賑之。橫州徭蠻為寇,命湖廣省發兵討捕。壬午,諭中書省,歲給衛王阿木哥鈔萬錠。敕衛輝、昌平守臣修殷比干、唐狄仁傑祠,歲時致祭。戊子,升印經提舉司為廣福監。己丑,升會福院秩正二品。癸巳,賜安遠王丑漢金各五百雨、鈔千錠、幣帛二十匹。己亥,增置周王斷事官二員。以淮東廉訪司僉事苗好謙善課民農桑,賜衣一襲。庚子,以上都留守憨剌合兒知樞密院事,升殊祥院秩正二品。命中書省與御史臺、翰林、集賢院集議封贈通制,著為令。遼陽蓋州及南豐州饑,發倉賑之。



 五月甲辰至戊申,日赤如赭。辛亥,以江西行省右丞相斡赤為大司徒。庚申,以大都留守伯鐵木兒為中書平章政事,升中書右丞蕭拜住為平章政事,左丞阿卜海牙為右丞,參政郭貫為左丞,參議不花為參知政事。庚午,置甘肅儒學提舉司、遼陽金銀鐵冶提舉司,秩並從五品。賜諸王迭裡哥兒不花等金三百五十兩、銀一千二百兩、鈔三千二百錠、幣帛有差。潭、永、寶慶、桂陽、澧、道、袁等路饑,發米賑糶。六月乙亥,制封孟軻父為邾國公,母為邾國宣獻夫人。改諸王公臣分地郡邑同知、縣丞為副達魯花赤,中、下縣及錄事司增置副達魯花赤一員。丙子,融、賓、柳州徭蠻叛,命湖廣行省遣官督兵捕之。丁丑,敕:「大闢罪,臨刑敢有橫加刲割者,以重罪論。凡鞫囚,非強盜毋加酷刑。」戊寅,吳王朵列納等部乏食,賑糧兩月。己卯,詔諭百司各勤其職,毋隳廢大政。甲申,給安遠王丑漢分樞密院印。丁亥,封床兀兒為句容郡王。丁酉,賜周王從衛鈔四十萬錠。河決汴梁,沒民居,遼陽之蓋州饑,並發糧賑之。



 秋七月壬子,命御史大夫伯忽、脫歡答剌罕拯治臺綱,仍降詔宣諭中外。乙卯,封玉龍鐵木兒為保恩王,賜金印。辛酉,賜普慶寺益都田百七十頃。丙寅,復以燕鐵木兒知樞密院事。庚午,發高麗、女直、漢軍千五百人,於濱州、遼河、慶雲、趙州屯田。



 八月癸酉,以兵部尚書乞塔為中書參知政事。己卯,車駕至自上都。戊戌,置織佛像工匠提調所,秩七品,設官二員。九月辛丑,復五條河屯田,以中書左丞郭貫為集賢大學士,集賢大學士王毅為中書左丞。庚戌,割上都宣德府奉聖州懷來、縉山二縣隸大都路,改縉山縣為龍慶州,帝生是縣,特命改焉。癸丑,太白晝見。己未,冀寧、晉寧路地震。丙寅,太白經天。



 冬十月辛未,以江南行臺侍御史高昉為中書參知政事。壬申,有事於太廟。調四川軍二千人、雲南軍三千人烏蒙等處屯田,置總管萬戶府,秩正三品,設官四員,隸雲南省。壬午,河南路地震。甲申,太白犯鬥。庚寅,敕五臺靈鷲寺置鐵冶提舉司。乙未,賜豳王南忽里部鈔四萬錠。丁酉,修甘州城。申禁民有父在者,不得私貸人錢及鬻墓木。甘州、肅州等路饑,免田租。



 十一月壬寅,命監察御史監治嶺北鉤校錢糧,半歲更代。大萬寧寺住持僧米普云濟以所佩國公印移文有司,紊亂官政,敕禁止之。乙巳,增集寧、砂井、凈州路同知、府判、提控、案牘各一員。乙卯,改舊運糧提舉司為大都陸運提舉司,新運糧提舉司為京畿運糧提舉司,澧州路安撫司為安定軍民府。



 十二月庚午,以知樞密院事禿忽魯為陜西行省左丞相。壬午,授嗣漢三十九代天師張嗣成太玄輔化體仁應道大真人,主領三山符籙,掌江南道教事。丁亥,立皇子碩德八剌為皇太子,兼中書令、樞密使,授以金寶,告天地宗廟。升同知樞密院事床兀兒知樞密院事。諸王按灰部乏食,給米三千一百八十六石濟之。



\end{pinyinscope}