\article{志第五十二 刑法三}

\begin{pinyinscope}

 ○食貨



 諸犯私鹽者,杖七十,徒二年,財產一半沒官,於沒物內一半付告人充賞。鹽貨犯界者,減私鹽罪一等。提點官禁治不嚴,初犯笞四十,再犯杖八十,本司官與總管府官一同歸斷,三犯聞奏定罪。如監臨官及灶戶私賣鹽者,同私鹽法。諸偽造鹽引者斬,家產付告人充賞。失覺察者,鄰佑不首告,杖一百。商賈販鹽,到處不呈引發賣,及鹽引數外夾帶,鹽引不相隨,並同私鹽法。鹽已賣,五日內不赴司縣批納引目,杖六十,徒一年,因而轉用者同賣私鹽法。犯私鹽及犯罪斷後,發鹽場充鹽夫,帶鐐居役,役滿放還。諸給散煎鹽灶戶工本,官吏通同克減者,計贓論罪。諸大都南北兩城關廂,設立鹽局,官為發賣,其餘州縣鄉村並聽鹽商興販。諸賣鹽局官、煎鹽灶戶、販鹽客旅行鋪之家,輒插和灰土硝鹼者,笞五十七。諸蒙古人私煮鹽者,依常法。諸犯私鹽,會赦,家產未入官者,革撥。諸私鹽再犯,加等斷徒如初犯,三犯杖斷同再犯,流遠,婦人免徒,其博易諸物,不論巨細,科全罪。諸轉買私鹽食用者,笞五十七,不用斷沒之令。諸捕獲私鹽,止理見發之家,勿聽攀指平民。有榷貨,無犯人,以榷貨解官;無榷貨,有犯人,勿問。諸巡捕私鹽,非承告報明白,不得輒入人家搜檢。諸犯私鹽,被獲拒捕者,斷罪流遠,因而傷人者處死。諸巡鹽軍官,輒受財脫放鹽徒者,以枉法計贓論罪,奪所佩符及所受命,罷職不敘。



 諸茶法,客旅納課買茶,隨處驗引發賣畢,三日內不赴所在官司批納引目者,杖六十;因而轉用,或改抹字號,或增添夾帶斤重,及引不隨茶者,並同私茶法。但犯私茶,杖七十,茶一半沒官,一半付告人充賞,應捕人同。若茶園磨戶犯者,及運茶船主知情夾帶,同罪。有司禁治不嚴,致有私茶生發,罪及官吏。茶過批驗去處不批驗者,杖七十。其偽造茶引者斬,家產付告人充賞。諸私茶,非私自入山採者,不從斷沒法。



 諸產金之地,有司歲征金課,正官監視人戶,自執權衡,兩平收受。其有巧立名色,廣取用錢,及多稱金數,克除火耗,為民害者,從監察御史廉訪司糾之。



 諸出銅之地,民間敢私煉者禁之。



 諸鐵法,無引私販者,比私鹽減一等,杖六十,錢沒官,內一半折價付告人充賞。偽造鐵引者,同偽造省部印信論罪,官給賞鈔二錠付告人。監臨正官禁治私鐵不嚴,致有私鐵生發者,初犯笞三十,再犯加一等,三犯別議黜降。客旅赴冶支鐵引後,不批月日出給,引鐵不相隨,引外夾帶,鐵沒官。鐵已賣,十日內不赴有司批納引目,笞四十;因而轉用,同私鐵法。凡私鐵農器鍋釜刀鐮斧杖及破壞生熟鐵器,不在禁限。江南鐵貨及生熟鐵器,不得於淮、漢以北販賣,違者以私鐵論。



 諸衛輝等處販賣私竹者,竹及價錢並沒官,首告得實者,於沒官物約量給賞。犯界私賣者,減私竹罪一等。若民間住宅內外並闌檻竹不成畝,本主自用外貨賣者,依例抽分。有司禁治不嚴者罪之,仍於解由內開寫。



 諸私造唆魯麻酒者,同私酒法,杖七十,徒二年,財產一半沒官,有首告者,於沒官物內一半給賞。諸蒙古、漢軍輒醖造私酒醋曲者,依常法。諸犯禁飲私酒者,笞三十七。諸犯界酒,十瓶以下,罰中統鈔一十兩,笞二十,七十瓶以上,罰鈔四十兩,笞四十七,酒給元主。酒雖多,罰止五十兩,罪止六十。



 諸匿稅者,物貨一半沒官,於沒官物內一半付告人充賞,但犯笞五十,入門不吊引,同匿稅法。諸辦課官,估物收稅而輒抽分本色者,禁之。其監臨官吏輒於稅課務求索什物者,以盜官物論,取與同坐。諸辦課官所掌應稅之物,並三十分中取一,輒冒估直,多收稅錢,別立名色,巧取分例,及不應收稅而收稅者,各以其罪罪之,廉訪司常加體察。諸在城及鄉村有市集之處,課稅有常法。其在城稅務官吏,輒於鄉村妄執經過商賈匿稅者,禁之。諸辦課官,侵用增餘稅課者,以不枉法贓論罪。諸職官,印契不納稅錢者,計應納稅錢,以不枉法論。



 諸市舶金銀銅錢鐵貨、男女人口、絲綿段匹、銷金綾羅、米糧軍器等,不得私販下海,違者舶商、船主、綱首、事頭、火長各杖一百七,船物沒官,有首告者,以沒官物內一半充賞,廉訪司常加糾察。諸市舶司於回帆物內,三十分抽稅一分,輒以非理受財者,計贓,以枉法論。諸舶商、大船給公驗,小船給公憑,每大船一,帶柴水船、八櫓船各一,驗憑隨船而行。或有驗無憑,及數外夾帶,即同私販,犯人杖一百七,船物並沒官,內一半付告人充賞。公驗內批寫物貨不實,及轉變滲洩作弊,同漏舶法,杖一百七,財物沒官;舶司官吏容隱,斷罪不敘。諸番國遣使奉貢,仍具貢物,報市舶司稱驗,若有夾帶,不與抽分者,以漏舶論。諸海門鎮守軍官,輒與番邦回舶頭目等人,通情滲洩舶貨者,杖一百七,除名不敘。諸中賣寶貨,耗蠹國財者,禁之。諸雲南行使贌法,官司商賈輒以他贌入境者,禁之。



 大惡



 諸大臣謀危社稷者誅。諸無故議論謀逆,為倡者處死,和者流。諸潛謀反亂者處死,宅主及兩鄰知而不首者同罪,內能悔過自首者免罪給賞,不應捕人首告者官之。諸謀反已有反狀,為首及同情者凌遲處死,為從者處死,知情不首者減為從一等流遠,並沒入其家。其相須連坐者,各以其罪罪之。諸父謀反,子異籍不坐。諸謀反事覺,捕治得實,行省不得擅行誅殺,結案待報。諸匿反叛不首者,處死。諸妖言惑眾,嘯聚為亂,為首及同謀者處死,沒入其家;為所誘惑相連而起者,杖一百七。諸假托神異,狂謀犯上者,處死。諸亂言犯上者處死,仍沒其家。諸指斥乘輿者,非特恩,必坐之。諸妄撰詞曲,誣人以犯上惡言者,處死。諸職官輒指斥詔旨亂言者,雖會赦,仍除名不敘。



 諸子孫弒其祖父母、父母者,凌遲處死,因風狂者處死。諸醉後毆其父母,父母無他子,告乞免死養老者,杖一百七,居役百日。諸子弒其繼母者,與嫡母同。諸部內有犯惡逆,而鄰佑、社長知而不首,有司承告而不問,皆罪之。諸子弒其父母,雖瘐死獄中,仍支解其尸以徇。諸毆傷祖父母、父母者,處死。諸謀殺已改嫁祖母者,仍以惡逆論。諸挾仇毆死義父,及殺傷幸獲生免者,皆處死。諸圖財殺傷義母者,處死。諸為人子孫,或因貧困,或信巫覡說誘,發掘祖宗墳墓,盜其財物,賣其塋地者,驗輕重斷罪:移棄尸骸,不為祭祀者,同惡逆結案。買者知情,減犯人罪二等,價錢沒官;不知情,臨事詳審,有司仍不得出給賣墳地公據。諸為人子孫,為首同他盜發掘祖宗墳墓,盜取財物者,以惡逆論,雖遇大赦原免,仍刺字徙遠方屯種。諸婦毆舅姑者,處死。諸因奸毆死其夫及其舅姑者,凌遲處死。諸弟殺其兄者,處死。諸父子同謀殺其兄,欲圖其財而收其嫂者,父子並凌遲處死。諸兄因爭,毆其弟,弟還毆其兄,邂逅致死,會赦,仍以故殺論。諸嫂叔爭,殺死其嫂者,處死。諸因爭虐殺其兄者,雖死仍戮其尸。諸因爭移怒,戳傷其兄者,於市曹杖一百七,流遠。諸挾仇毆死其伯叔母者,處死。諸因爭兄弟同謀毆死諸父者,皆處死。諸挾仇故殺其從父,偶獲生免者,罪與已死同。諸妻因爭殺其夫者,處死。諸婦人問醫人買毒藥殺其夫者,醫人同處死。諸妻殺傷其夫,幸獲生免者,同殺死論。諸婿因醉殺其婦翁,偶獲生免者,罪與已死同。



 諸奴殺傷本主者,處死。諸奴詬詈其主不遜者,杖一百七,居役二年,役滿日歸其主。諸奴故殺其主者,凌遲處死。諸奴毆死主婿者,處死。



 諸挾仇殺傷人一家,俱獲生免者,與已死同。其同謀悔過不至者,減等論。諸以奸盡殺其母黨一家者,凌遲處死。諸兄挾仇,與子同謀殺其弟一家者,皆處死。



 諸支解人,煮以為食者,以不道論,雖瘐死,仍徵燒埋銀給苦主。諸魘魅大臣者,處死。諸妻魘魅其夫,子魘魅其父,會大赦者,子流遠,妻從其夫嫁賣。諸造蠱毒中人者,處死。諸採生人支解以祭鬼者,凌遲處死,仍沒其家產。其同居家口,雖不知情,並徙遠方。已行而不曾殺人者,比強盜不曾傷人、不得財,杖一百七,徒三年。謀而未行者,九十七,徒二年半。其應死之人,能自首,或捕獲同罪者,給犯人家產,應捕者減半。



 奸非



 諸和奸者,杖七十七;有夫者,八十七。誘奸婦逃者,加一等,男女罪同,婦人去衣受刑。未成者,減四等。強奸有夫婦人者死,無夫者杖一百七,未成者減一等,婦人不坐。其媒合及容止者,各減奸罪三等,止理見發之家,私和者減四等。諸指奸不坐。諸無夫婦人有孕,稱與某人奸,即同指奸,罪止本婦。諸宿衛士與宮女奸者,出軍。諸翁欺奸男婦,已成者處死,未成者杖一百七,男婦歸宗。和奸者皆處死。男婦虛執翁奸已成,有司已加翁拷掠,男婦招虛者,處死;虛執翁奸未成,已加翁拷掠,男婦招虛者,杖一百七,發付夫家從其嫁賣。婦告或翁告同。若男婦告翁強奸已成,卻問得翁欲欺奸未成,男婦妄告重事,笞三十七,歸宗。諸欺奸義男婦,杖一百七,欺奸不成,杖八十七,婦並不坐。婦及其夫異居當差,雖會赦,仍異居。諸男婦與奸夫謀誣翁欺奸,買休出離者,杖一百七,從夫嫁賣,奸夫減一等,買休錢沒官。諸與弟妻奸者,各杖一百七,奸夫流遠,奸婦從夫所欲。諸嫂寡守志,叔強奸者,杖九十七。諸與同居侄婦奸,各杖一百七,有官者除名。諸強奸侄婦未成者,杖一百七。諸與兄弟之女奸,皆處死;與從兄弟之女奸,減一等;與族兄弟之女奸,減二等。諸居父母喪欺奸父妾者,各杖九十七,婦人歸宗。諸奸私再犯者,罪加二等,婦人聽其夫嫁賣。諸因奸偷遞家財,止以奸論。諸雇人之妻為妾,年滿而歸,雇主復與通,即以奸論。因又與殺其夫者,皆處死。諸子犯奸,父出首,仍坐之,諸奸不理首原。諸奸生男女,男隨父,女隨母。諸僧尼道士女冠犯奸,斷後並勒還俗,諸強奸人幼女者處死,雖和同強,女不坐。凡稱幼女,止十歲以下。諸年老奸人幼女,杖一百七,不聽贖。諸十五歲未成丁男,和奸十歲以下女,雖和同強,減死,杖一百七,女不坐。諸強奸十歲以上女者,杖一百七。諸強奸妻前夫男婦未成,及強奸妻前夫女已成,並杖一百七,妻離之。諸三男強奸一婦者,皆處死,婦人不坐。



 諸職官犯奸者,如常律,仍除名,但有祿人犯者同。諸職官求奸未成者,笞五十七,解見任,雜職敘。諸職官因謔部民妻,致其夫棄妻者,杖六十七,罷職,降二等雜職敘,記過。諸職官強奸部民妻未成,杖一百七,除名不敘。諸職官因奸,買部民妾,奸非奸所捕獲,止以買部民妾論,笞三十七,解職別敘。諸監臨官與所監臨囚人妻奸者,杖九十七,除名。諸職官與倡優之妻奸,因娶為妾者,杖七十七,罷職不敘。諸監臨令人奸污所部寡婦者,杖八十七,除名。諸蠻夷官擅以籍沒婦人為妻者,杖八十七,罷職記過,婦人笞四十七。



 諸主奸奴妻者,不坐。諸奴有女,已許嫁為良人妻,即為良人,其主輒欺奸者,杖一百七,其妻縱之者,笞五十七,其女夫家仍願為婚者,減元議財錢之半,不願者,追還元下聘財,令父收管,為良改嫁。諸奴奸主女者,處死。諸以傔從與命婦奸,以命婦從奸夫逃者,皆處死。諸強奸主妻者,處死。諸奴與主妾奸者,各杖九十七。諸良民竊奴婢生子,子隨母還主,奴竊良民生子,子隨母為良,仍異籍當差。諸奴婢相奸,笞四十七。



 諸夫受財,縱妻為倡者,夫及奸婦、奸夫各杖八十七,離之。若夫受財,勒妻妾為倡者,妻量情論罪。諸和奸,同謀以財買休,卻娶為妻者,各杖九十七,奸婦歸其夫。諸夫妻不睦,夫以威虐,逼其妻指與人奸者,杖七十七,妻不坐,離之。諸婿誣妻父與女奸者,杖九十七,妻離之。諸夫指奸而棄其妻,所指奸夫輒停妻而娶之者,兩離之。



 諸奸夫奸婦同謀殺其夫者,皆處死,仍於奸夫家屬徵燒埋銀。諸因奸殺其本夫,奸婦不知情,以減死論。諸妻與人奸,同謀藥死其夫,偶獲生免者,罪與已死同,依例結案。諸婦人為首,與眾奸夫同謀,親殺其夫者,凌遲處死,奸夫同謀者如常法。諸夫獲妻奸,妻拒捕,殺之無罪。諸與無夫婦奸,約為妻,卻毆死正妻者,處死。諸與奸婦同謀藥死其正妻者,皆處死。諸妻妾與人奸,夫於奸所殺其奸夫及其妻妾,及為人妻殺其強奸之夫,並不坐。若於奸所殺其奸夫,而妻妾獲免,殺其妻妾,而奸夫獲免者,杖一百七。諸奸夫殺死奸婦者,與故殺常人同。諸求奸不從,毆死其婦,以強盜持仗殺人論。諸兩奸夫與一奸婦皆有宿約,其先至者因鬥殺其後至者,以故殺論。



 盜賊



 諸盜賊共盜者,並贓論,仍以造意之人為首,隨從者各減一等。或二罪以上俱發,從其重得論之。諸竊盜初犯,刺左臂,謂已得財者。再犯刺右臂,三犯刺項。強盜初犯刺項,並充警跡人,官司以法拘檢關防之。其蒙古人有犯,及婦人犯者,不在刺字之例。諸評盜贓者,皆以至元鈔為則,除正贓外,仍追倍贓。其有未獲賊人,及雖獲無可追償,並於有者名下追徵。諸犯徒者,徒一年,杖六十七;一年半,杖七十七;二年,杖八十七;二年半,杖九十七;三年,杖一百七。皆先決訖,然後發遣合屬,帶鐐居役。應配役人,隨有金銀銅鐵洞冶、屯田、堤岸、橋道一切等處就作,令人監視,日計工程,滿日放還,充警跡人。諸盜未發而自首者,原其罪;能捕獲同伴者,仍依例給賞。其於事主有所損傷,及準首再犯,不在原免之例。諸杖罪以下,府州追勘明白,即聽斷決。徒罪,總管府決配,仍申合乾上司照驗。流罪以上,須牒廉訪司官,審覆無冤,方得結案,依例待報。其徒伴有未獲,追會有不完者,如復審既定,贓驗明白,理無可疑,亦聽依上歸結。



 諸強盜持仗但傷人者,雖不得財,皆死。不曾傷人,不得財,徒二年半;但得財,徒三年;至二十貫,為首者死,餘人流遠。不持仗傷人者,惟造意及下手者死。不曾傷人,不得財徒一年半,十貫以下徒二年;每十貫加一等,至四十貫,為首者死,餘人各徒三年。若因盜而奸,同傷人之坐,其同行人止依本法,謀而未行者,於不得財罪上,各減一等坐之。



 諸竊盜始謀而未行者,笞四十七;已行而不得財者,五十七;得財十貫以下,六十七;至二十貫,七十七。每二十貫加一等,一百貫,徒一年,每一百貫加一等,罪止徒三年。諸盜庫藏錢物者,比常盜加一等,贓滿至五百貫以上者流。



 諸盜駝馬牛驢騾,一陪九。盜駱駝者,初犯為首九十七,徒二年半,為從八十七,徒二年;再犯加等;三犯不分首從,一百七,出軍。盜馬者,初犯為首八十七,徒二年,為從七十七,徒一年半;再犯加等,罪止一百七,出軍。盜牛者,初犯為首七十七,徒一年半,為從六十七,徒一年;再犯加等,罪止一百七,出軍。盜驢騾者,初犯為首六十七,徒一年,為從五十七,刺放;再犯加等,罪止徒三年。盜羊豬者,初犯為首五十七,刺放,為從四十七,刺放;再犯加等,罪止徒三年。盜系官駝馬牛者,比常盜加一等。



 諸劇賊既款附得官,復以捕賊為由,虐取民財者,計贓論罪,流遠。諸強盜再犯,仍刺。



 諸強盜殺傷事主,不分首從,皆處死。諸強奪人財,以強盜論。諸以藥迷瞀人,取其財者,以強盜論。諸白晝持仗,剽掠得財,毆傷事主;若得財,不曾傷事主,並以強盜論。諸官民行船,遭風著淺,輒有搶虜財物者,比同強盜科斷。若會赦,仍不與真盜同論,徵贓免罪。諸強盜出外國,其邊臣執以來獻者,賜金帛以旌之。諸盜乘輿服御器物者,不分首從,皆處死。知情領賣,克除價錢者,減一等。



 諸盜官錢,追徵未盡,到官禁系既久,實無可折償者,除之。諸守庫軍,但盜庫中財物者,處死,會赦者仍刺之。諸內藏典守,輒盜庫中財物者,處死。諸造鈔庫工匠,私藏合毀之鈔出庫者,杖一百七。監臨失關防者,笞三十七。諸盜印鈔庫鈔者,處死。諸檢昏鈔行人,盜取昏鈔,為監臨搜獲,不得財者,以盜庫藏錢物不得財加等論,杖七十七。諸燒鈔庫合乾檢鈔行人,輒盜昏鈔出庫分使者,刺斷。諸盜局院官物,雖贓不滿貫,仍加等,杖七十七,刺字。諸工匠已關出庫物料,成造及額餘外,不曾還官,因盜出局者,斷罪,免刺。諸盜已到倉官糧,而未離倉事覺者,以不得財論,免刺。諸盜官員符節,比常盜加一等,計贓坐罪。諸盜官府文卷作故紙變賣者,杖七十七,同竊盜,刺字;買卷人笞四十七。



 諸圖財謀故殺人多者,凌遲處死,仍驗各賊所殺人數,於家屬均徵燒埋銀。諸圖財陷溺人於死,幸獲生免者,罪與已死同。諸圖財殺死他人奴婢,即以圖財殺人論。諸奴盜主財而逃,送其逃者,輒殺其奴而取其財,即以強盜殺人論。



 諸發塚,已開塚者同竊盜,開棺郭者同強盜,毀尸骸者同傷人,仍於犯人家屬徵燒埋銀。諸挾仇發塚,盜棄其尸者,處死。諸發塚得財不傷尸,杖一百七,刺配。諸盜發諸王駙馬墳寢者,不分首從,皆處死。看守禁地人,杖一百七,三分家產,一分沒官,同看守人杖六十七。



 諸事主殺死盜者,不坐。諸寅夜潛入人家,被毆傷而死者,勿論。



 諸於迥野盜伐人材木者,免刺,計贓科斷。諸被脅眾上盜,至盜所,復逃去,不以為從論。諸竊盜贓不滿貫,斷罪,免刺。諸子為盜,父殺之,不坐。諸為盜,初經刺斷,再犯奸私,止以奸為坐,不以為盜再犯論。諸奴婢數為盜,應識過於門者,其主不知情,不得輒書於其主之門。諸被誘脅上盜,不曾分贓,而容隱不首者,杖六十七,免刺。諸先盜親屬財,免刺,再盜他人財,止作初犯論。諸先犯誘奸婦人在逃,後犯竊盜,二事俱發,以誘奸為重,杖從奸,刺從盜。諸瘖啞為盜,不論瘖啞。諸詐稱搜稅,攔頭剽奪行李財物者,以盜論,刺斷,充警跡人。諸盜米糧,非因饑饉者,仍刺斷。諸盜塔廟神像服飾,無人看守者,斷罪,免刺。諸事主及盜私相休和者,同罪;所盜錢物頭匹、倍贓等,沒官。諸竊盜應徒,若有祖父母、父母年老,無兼丁侍養者,刺斷免徒;再犯而親尚存者,候親終日,發遣居役。諸女直人為盜,刺斷同漢人。諸年饑民窮,見物而盜,計贓斷罪,免刺配及徵倍贓。諸竊盜,一歲之中頻犯者,從一重,論刺斷。諸為盜為所得贓與人博不勝,失所得贓,事覺追正贓,仍坐博者罪。諸父以子同盜,子年未出幼,不曾分贓,免罪。諸年饑,迫其子若婿同持仗行劫,子若婿減死一等,坐免刺,充警跡人。諸父為人誘為盜,疾不能往,命其子從之,而分其贓者,父減為從一等,免刺,子以為從論。諸兄逼未成丁弟同上盜,減為從一等論,仍罰贖。諸兄弟同盜,罪皆至死,父母老而乏養者,內以一人情罪可逭者,免死養親。諸兄弟同盜,皆刺。諸父子兄弟頻同上盜,從凡盜首從論。諸父子兄弟同為強盜者,皆處死。諸夫謀為強盜,妻不諫,反從之盜者,減為從一等論罪。



 諸親屬相盜,謂本服緦麻以上親,及大功以上共為婚姻之家,犯盜止坐其罪,並不在刺字、倍贓、再犯之限。其別居尊長於卑幼家竊盜,若強盜及卑幼於尊長家行竊盜者,緦麻小功減凡人一等,大功減二等,期親減三等,強盜者準凡盜論,殺傷者各依故殺傷法。若同居卑幼將人盜己家財物者,五十貫以下,笞二十七,每五十貫加一等,罪止五十七,他人依常盜減一等。諸姑表侄盜姑夫財,同親屬相盜論。諸女在室,喪其父,不能自存,有祖父母而不之恤,因盜祖父母錢者,不坐。諸弟為首強劫從兄財,即以強盜論。諸嘗過房他人子孫以為子孫,輒盜所過房之家財物者,即以親屬相盜論。



 諸奴盜主財,應流遠,而主求免者聽。諸奴盜主財,斷罪,免刺。諸盜雇主財者,免刺,不追倍贓。盜先雇主財者,同常盜論。諸佃客盜地主財,同常盜論。諸同主奴相盜,斷罪,免刺配,不追倍贓。諸盜同受雇人財,不以同居論。諸賃屋與房主同居,而盜房主財者,與常盜論。諸盜同本財者,笞五十七,不以真盜計贓論。



 諸巡捕軍兵因自為盜者,比常盜加一等論罪;若自相覺察,告捕到官,或曾共為盜,首獲同伴者,免罪給賞。諸軍人為盜,刺斷,免充警跡人,仍追賞錢給告者。諸守庫藏軍人,輒為首誘引外人偷盜官物,但經二次三次入庫為盜,又提鈴把門軍人,受贓縱賊者,皆處死。為從者杖一百七,刺字流遠。諸見役軍人在逃,因為竊盜得財,杖一百七,仍刺字,杖從逃軍,刺從盜。諸軍人在路奪人財物,又迫逐人致死非命者,為首杖一百七,為從七十七,徵燒埋銀給苦主。



 諸婦人為盜,斷罪,免刺配及充警跡人,免征倍贓,再犯並坐其夫。諸婦人寡居與人奸,盜舅姑財與奸夫,令娶己為妻者,奸非奸所捕獲,止以同居卑幼盜尊長財為坐,笞五十七,歸宗,奸夫杖六十七。



 諸偽僧竊取佛像腹中裝者,以盜論。諸僧道為盜,同常盜,刺斷,徵倍贓,還俗充警跡人。諸僧道盜其親師祖、師父及同師兄弟財者,免刺,不追倍贓,斷罪還俗。



 諸幼小為盜,事發長大,以幼小論。未老疾為盜,事發老疾,以老疾論。其所當罪,聽贖,仍免刺配,諸犯罪亦如之。諸年未出幼,再犯竊盜者,仍免刺贖罪,發充警跡人。諸竊盜年幼者為首,年長者為從,為首仍聽贖免刺配,為從依常律。諸掏摸人身上錢物者,初犯、再犯、三犯,刺斷徒流,並同竊盜法,仍以赦後為坐。諸以七十二局欺誘良家子弟、富商大賈,博塞錢物者,以竊盜論,計贓斷配。諸夜發同舟橐中裝,取其財者,與竊盜真犯同論。



 諸略賣良人為奴婢者,略賣一人,杖一百七,流遠;二人以上,處死;為妻妾子孫者,一百七,徒三年;因而殺傷人者,同強盜法。若略而未賣者,減一等,和誘者又各減一等,及和同相賣為奴婢者,各一百七。略誘奴婢,貨賣為奴婢者,各減誘略良人罪一等;為妻妾子孫者,七十七,徒一年半;知情娶買及藏匿受錢者,各遞減犯人罪一等。假以過房乞養為名,因而貨賣奴婢者,九十七,引領牙保知情,減二等,價沒官,人給親。如無元買契券,有司輒給公據者,及承告不即追捕者,並笞四十七。關津主司知而受財縱放者,減犯人罪三等,除名不敘,失檢察者笞二十七。如能告獲者,略人每人給賞三十貫,和誘每人二十貫,以至元鈔為則,於犯人名下追徵,無財者征及知情安主,牙保應捕人減半。其事未發而自首者,若同黨能悔過自首,擒獲其徒黨者,並原其罪,仍給賞之半。再犯及因略傷人者,不在首原之例。諸婦人誘賣良人,罪應徒者,免徒。諸職官誘略良人為奴,革後不首,仍除名不敘,所誘略人給親。



 諸兄盜牛,脅其弟同宰殺者,弟不坐。諸白晝剽奪驛馬,為首者處死,為從減一等流遠。諸盜親屬馬牛,事未覺自首,顧償價,不從,既送官,仍以自首論免刺。諸強盜行劫,為主所逐,分散奔走,為首者殺傷鄰人,為從者不知,不以殺傷事主不分首從論,為首者處死,為從者杖一百七,刺配。諸竊盜棄財拒捕,毆傷事主者,杖一百七,免刺。諸為盜先竊後強,會赦,其下手殺傷事主者,不赦,餘仍刺而釋之。諸盜賊分贓不均,從賊欲首,為首賊所殺者,仍以謀故殺人論。諸盜賊聞赦,故殺捕盜之人者,不赦。



 諸藏匿強竊盜賊,有主謀糾合,指引上盜,分受贓物者,身雖不行,合以為首論。若未行盜,及行盜之後,知情藏匿之家,各減強竊從賊一等科斷,免刺,其已經斷,怙終不改者,與從賊同。諸謀欲圖人所質之田,輒遣人強劫贖田之價者,主謀、下手一體刺斷,其卑幼為尊長驅役者免刺。



 諸盜賊應徵正贓及燒埋銀,貧無以備,令其折庸。凡折庸,視各處庸價而會之。庸滿發元籍,充警跡人。婦人日準男子工價三分之二,官錢役於旁近之處,私錢役於事主之家。諸盜賊得財,用於酒肆倡優之家,不知情,止於本盜追徵。其所盜即官錢,雖不知情,於所用之家追徵。若用買貨物,還其貨物,徵元贓。諸奴婢盜人牛馬,既斷罪,其贓無可徵者,以其人給物主,其主願贖者聽。諸盜官錢,追徵未盡,到官禁系既久,實無可折償者,除之。諸系官人口盜人牛馬,免征倍贓。諸盜賊正贓已徵給主,倍贓無可追理者,免征。諸盜賊正贓,或曲質於人,典主不知情,而歸其贓,仍徵還元價。諸遐荒盜賊,盜駝馬牛驢羊,倍贓無可徵者,就發配役出軍。



 諸盜先犯後發,與後犯先發罪同者,勿論。諸先犯強盜刺斷,再犯竊盜,止依再犯竊盜刺配。諸出軍賊徒在逃,初犯杖六十七,再犯加二等,罪止一百七,仍發元流所出軍。



 諸強竊盜充警跡人者,五年不犯,除其籍。其能告發,及捕獲強盜一名,減二年,二名比五年,竊盜一名減一半,應除籍之外,所獲多者,依常人獲盜理賞,不及數者,給憑通理。籍既除,再犯,終身拘籍之。凡警跡人緝捕之外,有司毋差遣出入,妨其理生。諸警跡人,有不告知鄰佑輒離家經宿,及游惰不事生產作業者,有司究之,鄰佑有失覺察者,亦罪之。諸警跡人受命捕盜,既獲其盜,卻挾恨殺其盜而取其財,不以平人殺有罪賊人論。諸色目人犯盜,免刺科斷,發本管官司設法拘檢,限內改過者,除其籍。無本管官司發付者,從有司收充警跡人。



 諸為盜經刺,自除其字,再犯非理者,補刺。五年不再犯,已除籍者,不補刺,年未滿者仍補刺。諸盜賊赦前擅去所刺字,不再犯,赦後不補刺。諸應刺左右臂,而臂有雕青者,隨上下空歇之處刺之。諸犯竊盜已經刺臂,卻徧文其身,覆蓋元刺,再犯竊盜,於手背刺之。諸累犯竊盜,左右項臂刺遍,而再犯者,於項上空處刺之。



 諸子盜父首、弟盜兄首、婿盜翁首,並同自首者免罪。諸奴盜主首者,斷罪免刺,不征倍贓,仍付其主為奴。諸脅從上盜,而不受贓者,止以不首之罪罪之,杖六十七,不刺。諸為盜悔過,以所盜贓還主者免罪。諸為盜得財者,聞有涉疑根捕,卻以贓還主者,減二等論罪,免徒刺及倍贓。諸竊盜因事主盤詰,而自首服,其贓未還主者,計贓減二等論罪,刺字。諸盜贓,為首者自首,免罪,為從不首仍全科。諸無服之親,相首為盜,止科其罪,免刺配倍贓。諸竊盜悔過,以贓還主不盡,其餘贓猶及刺罪者,仍刺之。



\end{pinyinscope}