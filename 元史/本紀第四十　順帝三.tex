\article{本紀第四十 順帝三}

\begin{pinyinscope}

 五年春正月癸亥,禁濫予僧人名爵。庚午,太陰犯井宿。乙亥,熒惑犯天江。濮州鄄城、範縣饑,賑鈔二千一百八十錠。冀寧路交城等縣饑民問題、民族問題、戰略和策略、黨、工作作風等基本問題。,賑米七千石。桓州饑,賑鈔二千錠。雲需府饑,賑鈔五千錠。開平縣饑,賑米兩月。興和寶昌等處饑,賑鈔萬五千錠。



 二月庚寅,信州雨土。甲午,太陰犯昴宿。戊戌,祭社稷。庚子,免廣海添辦鹽課萬五千引,止辦元額。壬寅,太陰犯靈臺。



 三月辛酉,八魯剌思千戶所民被災,遣太禧宗禋院斷事官塔海發米賑之。戊辰,灤河住冬怯憐口民饑,每戶賑糧一石、鈔二十兩。



 夏四月辛卯,革興州興安縣。癸巳,立伯顏南口過街塔二碑。乙未,加封孝女曹娥為慧感靈孝昭順純懿夫人。壬寅,太陰犯日星及房宿。己酉,申漢人、南人、高麗人不得執軍器、弓矢之禁。是月,車駕時巡上都。



 五月己未朔,晃火兒不剌、賽禿不剌、紐阿迭列孫、三卜剌等處六愛馬大風雪,民饑,發米賑之。庚午,太陰犯心宿。壬申,太陰犯鬥宿。丙子,太白犯昴宿。丙戌,加封瀏陽州道吾山龍神崇惠昭應靈顯廣濟侯。六月壬寅,月食。甲辰,熒惑退入南斗。庚戌,汀州路長汀縣大水,平地深可三丈餘,沒民廬八百家,壞民田二百頃,戶賑鈔半錠,死者一錠。乙卯,達達民饑,賑糧三月。是月,沂、莒二州民饑,發糧賑糶之。



 秋七月辛酉、壬戌,熒惑犯南斗。甲子,熒惑犯南斗,太陰犯房宿。甲戌,太白經天。丙子,開上都、興和等處酒禁。丁丑,封皇姊月魯公主為昌國大長公主。戊寅,太白經天。詔:「諸王位下官毋入常選。」甲申,常州宜興山水出,勢高二丈,壞民廬。乙酉,太白經天。丙戌,太白復經天。



 八月丁亥,車駕至自上都。戊子,太白經天。祭社稷。己丑,太白復經天。庚寅,宗王脫歡脫木爾各愛馬人民饑,以鈔三萬四千九百錠賑之。宗王脫憐渾禿各愛馬人民饑,以鈔萬一千三百五十七錠賑之。太白經天。辛卯,太白復經天。甲午,太陰犯鬥宿。丁酉,太白犯軒轅。戊戌、己亥,太白經天。壬寅至甲辰,太白復經天。乙巳,太陰犯昴宿。九月丁巳,沈陽饑,民食木皮,賑糶米一千石。戊午,太白經天。己未,太白復經天。



 冬十月辛卯,享於太廟。壬辰,禁倡優盛服,許男子裹青巾,婦女服紫衣,不許戴笠、乘馬。甲午,詔命伯顏為大丞相,加元德上輔功臣之號,賜七寶玉書龍虎金符。己亥,熒惑犯壘壁陣。是月,衡州饑,賑糶米五千石。遼陽饑,賑米五百石。文登、牟平二縣饑,賑糶米一萬石。



 十一月丁巳,熒惑犯壘壁陣。禁宰殺。戊辰,開封杞縣人範孟反,偽傳帝旨,殺河南行省平章政事月祿帖木兒、左丞劫烈、廉訪使完者不花等,已而捕誅之。癸酉,瑞州路新昌州雨木冰,至明年二月始解。是月,八番順元等處饑,賑鈔二萬二十錠。



 十二月辛卯,復立都水庸田使司於平江。先是嘗置而罷,至是復立。甲午,太陰犯昴宿。癸酉,熒惑犯外屏。是歲,敕賜曲阜宣聖廟碑。工部梁上出芝草,一本七莖。袁州饑,賑糶米五千石。膠、密、莒、濰等州饑,賑鈔二萬錠。



 六年春正月丁卯,太陰犯鬼宿。甲戌,立司禋監,奉太祖、太宗、睿宗三朝御容於石佛寺。乙亥,太陰犯房宿。戊寅,追封闊兒吉思宣誠戡難翊運致美功臣、太師、開府儀同三司、上柱國,追封晉寧王,謚忠襄。是月,察忽、察罕腦兒等處馬災,賑鈔六千八百五十八錠。邳州饑,賑米兩月。



 二月甲申朔,詔權止今年印鈔。戊子,祭社稷。己丑,太陰犯昴宿。丙申,太陰犯太微垣。己亥,黜中書大丞相伯顏為河南行省左丞相,詔曰:「朕踐位以來,命伯顏為太師、秦王、中書大丞相,而伯顏不能安分,專權自恣,欺朕年幼,輕視太皇太后及朕弟燕帖古思,變亂祖宗成憲,虐害天下。加以極刑,允合輿論。朕念先朝之故,尚存憫恤,今命伯顏出為河南行省左丞相。所有元領諸衛親軍並怯薛丹人等,詔書到時,即許散還。」以太保馬札兒臺為太師、中書右丞相,太尉塔失海牙為太傅,知樞密院事塔馬赤為太保,御史大夫脫脫為知樞密院事,汪家奴為中書平章政事,嶺北行省平章政事也先帖木兒為御史大夫。增設京城米鋪,從便賑糶。壬寅,詔:「除知樞密院事脫脫之外,諸王侯不得懸帶弓箭、環刀輒入內府。」癸卯,太陰犯心宿。乙巳,罷各處船戶提舉、廣東採珠提舉二司。丁未,太陰犯羅堰。立延徽寺,以奉寧宗祀事。罷司禋監。罷通州、河西務等處抽分按利房,大都東里山查提領所。戊申,熒惑犯月星。己酉,彗星如房星大,色白,狀如粉絮,尾跡約長五寸餘。彗指西南,漸向西北行。是月,福寧州大水,溺死人民。京畿五州十一縣水,每戶賑米兩月。



 三月甲寅,漳州義士陳君用襲殺反賊李志甫,授君用同知漳州路總管府事。乙卯,益都、般陽等處饑,賑之。丙辰,赦漳、潮二州民為李志甫、劉虎仔脅從之罪,褒贈軍將死事者。丁巳,大斡耳朵思風雪為災,馬多死,以鈔八萬錠賑之。癸亥,四怯薛役戶饑,賑米一千石、鈔二千錠。成宗潛邸四怯薛戶饑,賑米二百石、鈔二百錠。以知樞密院事脫脫、御史大夫別兒怯不花、知樞密院事牙不花知經筵事,中書參議阿魯佛住兼經筵官。太陰犯軒轅。丁卯,詔賜江南行臺御史中丞史惟良、御史中丞耿煥、山東廉訪使張友諒、中書參知政事許有壬上尊、束帛。庚午,太陰犯房宿。辛未,詔徙伯顏於南恩州陽春縣安置。壬申,太陰犯南斗。丁丑,以治書侍御史達識帖睦邇為奎章閣大學士,翰林直學士揭傒斯為奎章閣供奉學士。戊寅,太白犯月星。辛巳,彗星見,自二月己酉至三月庚辰,凡三十二日。是月,淮安路山陽縣饑,賑鈔二午五百錠,給糧兩月。順德路邢臺縣饑,賑鈔三千錠。



 夏四月己丑,享於太廟。庚寅,詔大天源延聖寺立明宗神御殿碑。以同知樞密院事鐵木兒塔識為中書右丞。丙午,詔封馬札兒臺為忠王及加答剌罕之號,馬札兒臺辭。



 五月癸丑,禁民間藏軍器。乙卯,監察御史普魯臺言:「右丞相馬札兒臺辭答剌罕及王爵名號,宜示天下,以勸廉讓。」從之。己未,詔以黨兀巴太子擒賊阿答理胡,歿於王事,追封涼王,謚忠烈。漳州龍巖尉黃佐才獲李志甫餘黨鄭子箕,佐才因與賊戰,妻子四十餘口皆遇害,以佐才為龍巖縣尹。丁卯,太陰犯鬥宿。辛未,降鈔萬錠,給守衛宮闕內外門禁唐兀,左、右阿速,貴赤,阿兒渾,欽察等衛軍。丙子,車駕時巡上都。置月祭各影堂香於大明殿,遇行禮時,令省臣就殿迎香祭之。以宦者伯不花為長寧寺卿。是月,濟南饑,賑鈔萬錠。六月丙申,詔撤文宗廟主,徙太皇太后不答失里東安州安置,放太子燕帖古思於高麗,其略曰:



 昔我皇祖武宗皇帝升遐之後,祖母太皇太后惑於憸慝,俾皇考明宗皇帝出封雲南。英宗遇害,正統浸偏,我皇考以武宗之嫡,逃居朔漠,宗王大臣同心翊戴,肇啟大事,於時以地近,先迎文宗,暫總機務。繼知天理人倫之攸當,假讓位之名,以寶璽來上,皇考推誠不疑,即授以皇太子寶。文宗稔惡不悛,當躬迓之際,乃與其臣月魯不花、也裏牙、明裏董阿等謀為不軌,使我皇考飲恨上賓。歸而再御宸極,思欲自解於天下,乃謂夫何數日之間,宮車弗駕。海內聞之,靡不切齒。又私圖傳子,乃構邪言,嫁禍於八不沙皇后,謂朕非明宗之子,遂俾出居遐陬。祖宗大業,幾於不繼。內懷愧慊,則殺也裏牙以杜口。上天不佑,隨降殞罰。叔嬸不答失里,怙其勢焰,不立明考之塚嗣,而立孺稚之弟懿璘質班,奄復不年,諸王大臣以賢以長,扶朕踐位。國之大政,屬不自遂者,詎能枚舉。每念治必本於盡孝,事莫先於正名,賴天之靈。權奸屏黜,盡孝正名,不容復緩,永惟鞠育罔極之恩,忍忘不共戴天之義。既往之罪,不可勝誅,其命太常徹去脫脫木兒在廟之主。不答失裏本朕之嬸,乃陰構奸臣,弗體朕意,僭膺太皇太后之號,跡其閨門之禍,離間骨肉,罪惡尤重,揆之大義,削去鴻名,徙東安州安置。燕帖古思昔雖幼沖,理難同處,朕終不陷於覆轍,專務殘酷,惟放諸高麗,當時賊臣月魯不花、也裏牙已死,其以明裏董阿等明正典刑。



 監察御史崔敬言燕帖古思不宜放逐,不報。己亥,秦州成紀縣山崩地坼。癸卯,太白晝見。己酉,太白復晝見。辛亥,太白晝見,夜犯歲星。是月,濟南路歷城縣饑,賑鈔二千五百錠。



 秋七月甲寅,太白晝見。詔封微子為仁靖公,箕子為仁獻公,比干加封為仁顯忠烈公。乙卯,奉元路盩厔縣河水溢,漂流人民。丁巳,太白晝見。戊午,以星文示異,地道失寧,蝗旱相仍,頒罪己詔於天下。享於太廟。己未,以亦憐真班為御史大夫。庚申,太陰犯心宿。壬戌至癸亥,太白晝見。甲子,太陰犯羅堰。乙丑至丙寅,太白復晝見。丁卯,燕帖古思薨,詔以鈔一百錠備物祭之。癸酉,太白晝見。戊寅,命翰林學士承旨腆哈、奎章閣學士巙巙等刪修《大元通制》。庚辰,達達之地大風雪,羊馬皆死,賑軍士鈔一百萬錠;並遣使賑怯烈干十三站,每站一千錠。是月,禁色目人勿妻其叔母。



 八月壬午,以也先帖木兒為御史大夫。戊子,祭社稷。是月,車駕至自上都。



 九月辛亥,明裏董阿伏誅。癸丑,加封漢張飛武義忠顯英烈靈惠助順王。辛酉,太陰犯虛梁。丙寅,詔:「今後有罪者,毋籍其妻女以配人。」丁卯,太陰犯昴宿,熒惑犯歲星。甲戌,太陰犯軒轅。



 冬十月甲申,奉玉冊、玉寶尊皇考為順天立道睿文智武大聖孝皇帝,親祼太室。庚寅,奉符、長清、元城、清平四縣饑,詔遣制國用司官驗而賑之。辛卯,各愛馬人不許與常選。壬辰,立曹南王阿剌罕、淮安王伯顏、河南王阿術祠堂。丁酉,太白入南斗。己亥,太白犯鬥宿。壬寅,馬札兒臺辭右丞相職,仍為太師。以脫脫為中書右丞相,宗正札魯忽赤鐵木兒不花為中書左丞相。是月,河南府宜陽等縣大水,漂沒民廬,溺死者眾,人給殯葬鈔一錠,仍賑義倉糧兩月。



 十一月甲寅,監察御史世圖爾言,宜禁答失蠻、回回、主吾人等叔伯為婚姻。乙卯,太陰犯虛梁。以親祼大禮慶成,御大明殿受群臣朝。戊午,熒惑犯氐宿。甲子,月食,辰星犯東咸。辛未,以孔克堅襲封衍聖公。戊寅,辰星犯天罡。是月,處州、婺州饑,以常平、義倉糧賑之。



 十二月,復科舉取士制。國子監積分生員,三年一次,依科舉例入會試,中者取一十八名。癸未,太陰犯虛梁。乙酉,太陰犯土公。丁亥,熒惑犯鉤鈐。戊子,罷天歷以後增設太禧宗禋等院及奎章閣。乙未,熒惑犯東咸。戊戌,太陰犯明堂。是月,東平路民饑,賑之。寶慶路大雪,深四尺五寸。



 至正元年春正月己酉朔,改元,詔曰:



 朕惟帝王之道,德莫大於克孝,治莫大於得賢。朕早歷多難,入紹大統,仰思祖宗付托之重,戰兢惕勵,於茲八年。慨念皇考,久勞於外,甫即大命,四海觖望,夙夜追慕,不忘於懷。乃以至元六年十月初四日,奉玉冊、玉寶,追上皇考曰順天立道睿文智武大聖孝皇帝,被服袞冕,祼於太室,式展孝誠。十有一月六日,勉徇大禮慶成之請,御大明殿受群臣朝。爰自去春,疇咨於眾,以知樞密院事馬札兒臺為太師、右丞相,以正百官,以親萬民。尋即控辭,養疾私第,再三諭旨,勉令就位,自春徂秋,其請益固。朕憫其勞日久,察其至誠,不忍煩之以政,俾解機務,仍為太師。而知樞密院事脫脫,早歲輔朕,克著忠貞,乃命為中書右丞相;宗正札魯忽赤帖木兒不花,嘗歷政府,嘉績著聞,為中書左丞相,並錄軍國重事。夫三公論道,以輔予德,二相總政,以弼予治,其以至元七年為至正元年,與天下更始。



 甲寅,熒惑犯天江。丁巳,享於太廟。庚申,太陰犯井宿。癸亥,詔天壽節禁屠宰六日。辛未,太陰犯心宿。癸酉,太陰犯鬥宿。甲戌,太白晝見,凡四日。是月,命脫脫領經筵事。命永明寺寫金字經一藏。免天下稅糧五分。湖南諸路饑,賑糶米十八萬九千七十六石。



 二月戊寅,祭社稷。己卯,太白晝見。庚辰,太白復晝見。辛巳,立廣福庫,罷藏珍等庫。乙酉,濟南濱州沾化等縣饑,以鈔五萬三千錠賑之。丙戌,太白晝見。癸巳,太陰犯明堂。乙未,加封皇姊不答昔你明惠貞懿大長公主。是月,大都寶坻縣饑,賑米兩月。河間莫州、滄州等處饑,賑鈔三萬五千錠。晉州饒陽、阜平、安喜、靈壽四縣饑,賑鈔二萬錠。印造至元鈔九十九萬錠、中統鈔一萬錠。



 三月庚戌,罷兩淮屯田手號打捕軍役,令屬本所領之。癸丑,命屯儲御軍於河南芍陂、洪澤、德安三處屯種。甲寅,給還帖木兒不花宣讓王印,鎮淮西。己未,汴梁地震。大都路涿州範陽、房山饑,賑鈔四千錠。丙子,以行省平章政事燕帖木兒就佩虎符,提調屯田。是月,般陽路長山等縣饑,賑鈔萬錠。彰德路安陽等縣饑,賑鈔萬五千錠。



 夏四月丁丑,道州土賊蔣丙等反,破江華縣,掠明遠縣。戊寅,彰德有赤風自西北起,晝晦如夜。甲申,享於太廟。丁亥,臨賀縣民被徭寇鈔掠,發義倉糧賑之。庚寅,帝幸護聖寺。命中書右丞鐵木兒塔識為平章政事,阿魯為右丞,許有壬為左丞。癸巳,立富平庫,隸資正院。復立衛候司。丁酉,以兩浙水災,免歲辦餘鹽三萬引。己亥,立吏部司績官。庚子,復封太師馬札兒臺為忠王。罷漷州河西務。彰德饑,賑鈔萬五千錠。是月,車駕時巡上都。



 五月戊申,以崇文監屬翰林國史院。己未,罷河西務行用庫。壬戌,月食。是月,賑阿剌忽等處被災之民三千九百一十三戶,給鈔二萬一千七百五錠。



 閏五月丁丑,改封徽州土神汪華為昭忠廣仁武烈靈顯王。甲午,賞賜扈從明宗諸王官屬八百七人金、銀、鈔、幣各有差。壬寅,詔刻宣文、至正二寶。



 六月戊午,禁高麗及諸處民以親子為宦者,因避賦役。戊辰,改舊奎章閣為宣文閣。庚午,太陰犯井宿。是月,揚州路崇明、通、泰等州,海潮湧溢,溺死一千六百餘人,賑鈔萬一千八百二十錠。



 秋七月己卯,享於太廟。乙酉,太陰犯填星。庚寅,太陰犯雲雨。



 八月戊申,祭社稷。是月,車駕至自上都。九月庚辰,太陰犯建星。壬午,賜文臣燕於拱辰堂。己丑,冀寧路嘉禾生,異畝同穎。壬辰,太陰犯鉞星,又犯井宿。壬寅,許有壬進講明仁殿,帝悅,賜酒宣文閣中,仍賜貂裘、金織文幣。



 冬十月丁未,享於太廟。己酉,封阿沙不花順寧王,昔寶赤寒食順國公。甲寅,中書省臣奏:「海運不給,宜令江浙行省於中政院財賦府撥賜諸人寺觀田糧,總運二百六十萬石。」從之。乙卯,歲星犯氐宿。丁巳,太陰犯月星。戊午,月食既。



 十一月丙子,道州路賊何仁甫等反。戊寅,彰德屬縣各添設縣尉一員。庚辰,分吏部、禮部、兵部、刑部為二庫,戶部、工部為二庫,各設管勾一員。己亥,太陰犯東井。庚子,太陰犯天江。徭賊寇邊,詔湖廣行省平章政事鞏卜班總兵討平之,定賞有差。



 十二月乙卯,詔:「民年八十以上,蒙古人賜繒帛二表裏,其餘州縣,旌以高年耆德之名,免其家雜役。」丁巳,太白犯壘壁陣。己未,立四川安岳縣。增設嘉興等處鹽倉。壬戌,雲南車裏寒賽、刀等反,詔雲南行省平章政事脫脫木兒討平之。癸亥,以在庫至元、中統鈔二百八十二萬二千四百八十八錠可支二年,住造明年鈔本。詔革王伯顏察兒等所獻檀、景等處產金地土。山東、燕南強盜縱橫,至三百餘處,選官捕之。復立拱儀局。己巳,以翰林學士承旨張起巖知經筵事。是月,復立司禋監。加封真定路滹沱河神為昭佑靈源侯。



 二年春正月丁丑,享於太廟。丙戌,開京師金口河,深五十尺,廣一百五十尺,役夫一十萬。戊子,太陰犯明堂。癸巳,遣翰林學士三保等代祀五岳四瀆。甲午,熒惑犯月星。是月,大同饑,人相食,運京師糧賑之。順寧保安饑,賑鈔一萬錠。廣平磁、威州饑,賑鈔五萬錠。降咸平府為縣;升懿州為路,以大寧路所轄興中、義州屬懿州。



 二月壬寅,頒《農桑輯要》。戊申,祭社稷。乙卯,李沙的偽造御寶聖旨,稱樞密院都事,伏誅。己巳,織造明宗御容。是月,彰德路安陽、臨漳等縣饑,賑鈔二萬錠。大同路渾源州饑,以鈔六萬二千錠、糧二萬石兼賑之。大名路饑,以鈔萬二千錠賑之。河間路饑,以鈔五萬錠賑之。



 三月戊寅,親試進士七十八人,賜拜住、陳祖仁及第,其餘出身有差。辛巳,冀寧路饑,賑糶米三萬石。戊子,太陰犯房宿。是月,順德路平鄉縣饑,賑鈔萬五千錠。衛輝路饑,賑鈔萬五千錠。杭州路火災,給鈔萬錠賑之。



 夏四月辛丑,冀寧路平晉縣地震,聲鳴如雷,裂地尺餘,民居皆傾。乙巳,享於太廟。己酉,罷雲南蒙慶宣慰司。庚申,太陰犯羅堰。是月,車駕時巡上都。



 五月甲申,太白經天。丁亥,以江浙行省平章政事只而瓦臺為河南行省平章政事。東平雨雹如馬首。六月戊申,命江浙撥賜僧道田還官徵糧,以備軍儲。壬子,濟南山崩,水湧。乙丑,支邦牙宣慰司。是月,汾水大溢。



 秋七月庚午,惠州路羅浮山崩。辛未,享於太廟。乙未,太陰掩太白。丁酉,太白晝見。己亥,慶遠路莫八聚眾反,攻陷南丹、左右兩江等處,命脫脫赤顏討平之。立司獄司於上都,比大都兵馬司。是月,拂郎國貢異馬,長一丈一尺三寸,高六尺四寸,身純黑,後二蹄皆白。



 八月庚子朔,日有食之。癸卯,罷上都事產提舉司。丙午,太白晝見。戊申,祭社稷。是月,冀寧路饑,賑糶米萬五千石。九月己巳,詔遣湖廣行省平章政事鞏卜班領河南、江浙、湖廣諸軍討道州賊,平之,復平嵠峒堡寨二百餘處。辛未,車駕至自上都。丁丑,太陰犯羅堰。京城強賊四起。戊子,太陰犯井宿。是月,歸德府睢陽縣因黃河為患,民饑,賑糶米萬三千五百石。



 冬十月己亥朔,日有食之。癸卯,太陰犯建星。陜西行省平章政事朵朵辭職侍親,不允。丁未,享於太廟。甲寅,太陰犯天關。壬戌,詔遣官致祭孔子於曲阜。罷織染提舉司。甲子,杭州、嘉興、紹興、溫州、臺州等路各立檢校批驗鹽引所,權免兩浙額鹽十萬引,福建餘鹽三萬引。



 十一月甲申,詔免雲南明年差稅。辛卯,歲星、熒惑、太白聚於尾宿。



 十二月壬寅,申服色之禁。丙午,命中書右丞太平、樞密副使姚庸、御史中丞張起巖知經筵事。己酉,京師地震。辛亥,封晃火帖木兒之子徹里帖木兒為撫寧王。丙辰,賜雲南行省參知政事不老三珠虎符,以兵討死可伐。癸亥,阿魯、禿滿等以謀害宰臣,圖為叛逆,伏誅。



\end{pinyinscope}