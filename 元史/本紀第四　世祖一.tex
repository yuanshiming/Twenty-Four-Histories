\article{本紀第四 世祖一}

\begin{pinyinscope}

 世祖聖德神功文武皇帝,諱忽必烈,睿宗皇帝第四子。母莊聖太后,怯烈氏。以乙亥歲八月乙卯生。及長,仁明英睿,事太后至孝,尤善撫下。納弘吉剌氏為妃。



 歲甲辰,帝在潛邸,思大有為於天下,延籓府舊臣及四方文學之士,問以治道。



 歲辛亥,六月,憲宗即位,同母弟惟帝最長且賢,故憲宗盡屬以漠南漢地軍國庶事,遂南駐瓜忽都之地。



 邢州有兩答剌罕言於帝曰:「邢吾分地也,受封之初,民萬餘戶,今日減月削,才五七百戶耳,宜選良吏撫循之。」帝從其言,承制以脫兀脫及張耕為邢州安撫使,劉肅為商榷使,邢乃大治。



 歲壬子,帝駐桓、撫間。憲宗令斷事官牙魯瓦赤與不只兒等總天下財賦於燕,視事一日,殺二十八人。其一人盜馬者,杖而釋之矣,偶有獻環刀者,遂追還所杖者,手試刀斬之。帝責之曰:「凡死罪,必詳讞而後行刑,今一日殺二十八人,必多非辜。既杖復斬,此何刑也?」不只兒錯愕不能對。太宗朝立軍儲所於新衛,以收山東、河北丁糧,後惟計直取銀帛,軍行則以資之。帝請於憲宗,設官築五倉于河上,始令民入粟。宋遣兵攻虢之盧氏、河南之永寧、衛之八柳渡,帝言之憲宗,立經略司於汴,以忙哥、史天澤、楊惟中、趙璧為使,陳紀、楊果為參議,俾屯田唐、鄧等州,授之兵、牛,敵至則御,敵去則耕,仍置屯田萬戶於鄧,完城以備之。夏六月,入覲憲宗於曲先惱兒之地,奉命帥師征雲南。秋七月丙午,祃牙西行。



 歲癸丑,受京兆分地。諸將皆築第京兆,豪侈相尚,帝即分遣,使戍興元諸州。又奏割河東解州鹽池以供軍食,立從宜府於京兆,屯田鳳翔,募民受鹽入粟,轉漕嘉陵。夏,遣王府尚書姚樞立京兆宣撫司,以孛蘭及楊惟中為使,關隴大治。又立交鈔提舉司,印鈔以佐經用。秋八月,師次臨洮。遣玉律術、王君候、王鑒諭大理,不果行。九月壬寅,師次忒剌,分三道以進。大將兀良合帶率西道兵,由晏當路;諸王抄合、也只烈帥東道兵,由白蠻;帝由中道。乙巳,至滿陀城,留輜重。冬十月丙午,過大渡河,又經行山谷二千餘里,至金沙江,乘革囊及筏以渡。摩娑蠻主迎降,其地在大理北四百餘里。十一月辛卯,復遣玉律術等使大理。丁酉,師至白蠻打郭寨,其主將出降,其侄堅壁拒守,攻拔殺之,不及其民。庚子,次三甸。辛丑,白蠻送款。十二月丙辰,軍薄大理城。初,大理主段氏微弱,國事皆決於高祥、高和兄弟,是夜,祥率眾遁去,命大將也古及拔突兒追之。帝既入大理,曰:「城破而我使不出,計必死矣。」己未,西道兵亦至,命姚樞等搜訪圖籍,乃得三使尸。既瘞,命樞為文祭之。辛酉,南出龍首城,次趙瞼。癸亥,獲高祥,斬於姚州。留大將兀良合帶戍守,以劉時中為宣撫使,與段氏同安輯大理,遂班師。



 歲甲寅,夏五月庚子,駐六盤山。六月,以廉希憲為關西道宣撫使,姚樞為勸農使。秋八月,至自大理,駐桓、撫間,復立撫州。冬,駐瓜忽都之地。



 歲乙卯,春,復駐桓、撫間。冬,駐奉聖州北。



 歲丙辰,春三月,命僧子聰卜地於桓州東、灤水北,城開平府,經營宮室。冬,駐於合剌八剌合孫之地。憲宗命益懷州為分地。



 歲丁巳,春,憲宗命阿藍答兒、劉太平會計京兆、河南財賦,大加鉤考,其貧不能輸者,帝為代償之。冬十二月,入覲於也可迭烈孫之地,議分道攻宋,以明年為期。



 歲戊午,冬十一月戊申,祃牙於開平東北,是日啟行。



 歲己未,春二月,會諸王於邢州。夏五月,駐小濮州。征東平宋子貞、李昶,訪問得失。秋七月甲寅,次汝南,命大將拔都兒等前行,備糧漢上,戒諸將毋妄殺。命楊惟中、郝經宣撫江淮,必闍赤孫貞督軍須蔡州。有軍士犯法者,貞縛致有司,白於帝,命戮以徇,諸軍凜然,無敢犯令者。八月丙戌,渡淮。辛卯,入大勝關,宋戍兵皆遁。壬辰,次黃陂。甲午,遣廉希憲招臺山寨,比至,千戶董文炳等已破之。時淮民被俘者眾,悉縱之。庚子,先鋒茶忽得宋沿江制置司榜來上,有云:「今夏諜者聞北兵會議,取黃陂民船系筏,由陽邏堡以渡,會於鄂州。」帝曰:「此事前所未有,願如其言。」辛丑,師次江北。九月壬寅朔,親王穆哥自合州釣魚山遣使以憲宗兇問來告,且請北歸以系天下之望。帝曰:「吾奉命南來,豈可無功遽還?「甲辰,登香爐山,俯瞰大江,江北曰武湖,湖之東曰陽邏堡,其南岸即滸黃洲。宋以大舟扼江渡,帝遣兵奪二大舟,是夜,遣木魯花赤、張文謙等具舟楫。乙巳遲明,至江岸,風雨晦冥,諸將皆以為未可渡,帝不從,遂申敕將帥揚旗伐鼓,三道並進,天為開霽。與宋師接戰者三,殺獲甚眾,徑達南岸。軍士有擅入民家者,以軍法從事。凡所俘獲,悉縱之。丁未,遣王沖道、李宗傑、訾郊招諭鄂城,比至東門,矢下如雨,沖道墜馬,為敵所獲,宗傑、郊奔還。帝駐滸黃洲。己酉,抵鄂,屯兵教場。庚戌,圍鄂。壬子,登城東北壓雲亭,立望樓,高可五丈,望見城中出兵,趣兵迎擊,生擒二人,云:「賈似道率兵救鄂,事起倉卒,皆非精銳。」遂命官取逃民棄糧,聚之軍中,為攻取計。戊午,順天萬戶張柔兵至。大將拔突兒等以舟師趨岳州,遇宋將呂文德自重慶來,拔都兒等迎戰,文德乘夜入鄂城,守愈堅。冬十月辛未朔,移駐烏龜山。甲戌,拔突兒還自嶽。十一月丙辰,移駐牛頭山。兀良合帶略地諸蠻,由交趾歷邕、桂,抵潭州,聞帝在鄂,遣使來告。時先朝諸臣阿藍答兒、渾都海、脫火思、脫里赤等謀立阿里不哥。阿里不哥者,睿宗第七子,帝之弟也。於是阿藍答兒發兵於漠北諸部,脫里赤括兵於漠南諸州,而阿藍答兒乘傳調兵,去開平僅百餘里。皇后聞之,使人謂之曰:「發兵大事,太祖皇帝曾孫真金在此,何故不令知之?」阿藍答兒不能答。繼又聞脫里赤亦至燕,後即遣脫歡、愛莫干馳至軍前密報,請速還。丁卯,發牛頭山,聲言趣臨安,留大將拔突兒等帥諸軍圍鄂。閏月庚午朔,還駐青山磯。辛未,臨江岸,遣張文謙還諭諸將曰:「遲六日,當去鄂退保滸黃洲。」命文謙發降民二萬北歸。宋賈似道遣宋京請和,命趙璧等語之曰:「汝以生靈之故來請和好,其意甚善,然我奉命南征,豈能中止?果有事大之心,當請於朝。」是日,大軍北還。己丑,至燕。脫里赤方括民兵,民甚苦之,帝詰其由,托以憲宗臨終之命。帝察其包藏禍心,所集兵皆縱之,人心大悅。是冬,駐燕京近郊。



 中統元年春三月戊辰朔,車駕至開平。親王合丹、阿只吉率西道諸王,塔察兒、也先哥、忽剌忽兒、爪都率東道諸王,皆來會,與諸大臣勸進。帝三讓,諸王大臣固請。辛卯,帝即皇帝位,以祃祃、趙璧、董文炳為燕京路宣慰使。陜西宣撫使廉希憲言:「高麗國王嘗遣其世子倎入覲,會憲宗將兵攻宋,倎留三年不遣。今聞其父已死,若立倎,遣歸國,彼必懷德於我,是不煩兵而得一國也。」帝是其言,改館倎,以兵衛送之,仍赦其境內。夏四月戊戌朔,立中書省,以王文統為平章政事,張文謙為左丞。以八春、廉希憲、商挺為陜西四川等路宣撫使,趙良弼參議司事,粘合南合、張啟元為西京等處宣撫使。己亥,詔諭高麗國王王倎,仍歸所俘民及其逃戶,禁邊將勿擅掠。辛丑,以即位詔天下。詔曰:



 朕惟祖宗肇造區宇,奄有四方,武功迭興,文治多缺,五十餘年於此矣。蓋時有先後,事有緩急,天下大業,非一聖一朝所能兼備也。先皇帝即位之初,風飛雷厲,將大有為。憂國愛民之心雖切於己,尊賢使能之道未得其人。方董夔門之師,遽遺鼎湖之泣。豈期遺恨,竟勿克終。



 肆予沖人,渡江之後,蓋將深入焉,乃聞國中重以僉軍之擾,黎民驚駭,若不能一朝居者。予為此懼,驛騎馳歸。目前之急雖紓,境外之兵未戢。乃會群議,以集良規。不意宗盟,輒先推戴。左右萬里,名王巨臣,不召而來者有之,不謀而同者皆是,咸謂國家之大統不可久曠,神人之重寄不可暫虛。求之今日,太祖嫡孫之中,先皇母弟之列,以賢以長,止予一人。雖在征伐之間,每存仁愛之念,博施濟眾,實可為天下主。天驏道助順,人謨與能。祖訓傳國大典,於是乎在,孰敢不從。朕峻辭固讓,至於再三,祈懇益堅,誓以死請。於是俯徇輿情,勉登大寶。自惟寡昧,屬時多艱,若涉淵冰,罔知攸濟。爰當臨御之始,宜新弘遠之規。祖述變通,正在今日。務施實德,不尚虛文。雖承平未易遽臻,而饑渴所當先務。嗚呼!歷數攸歸,欽應上天之命;勛親斯托,敢忘烈祖之規?建極體元,與民更始。朕所不逮,更賴我遠近宗族、中外文武,同心協力,獻可替否之助也。誕告多方,體予至意!



 丁未,以翰林侍讀學士郝經為國信使,翰林待制何源、禮部郎中劉人傑副之,使於宋。丙辰,收輯中外官吏宣札牌面。遣帖木兒、李舜欽等行部,考課各路諸色工匠。置急遞鋪。乙丑,徵諸道兵六千五百人赴京師宿衛。置互市於漣水軍,禁私商不得越境,犯者死。是月,阿里不哥僭號於和林城西按坦河。召賈居貞、張儆、王煥、完顏愈乘傳赴闕。五月戊辰朔,詔燕貼木兒、忙古帶節度黃河以西諸軍。丙戌,建元中統,詔曰:



 祖宗以神武定四方,淳德御群下。朝廷草創,未遑潤色之文;政事變通,漸有綱維之目。朕獲纘舊服,載擴丕圖,稽列聖之洪規,講前代之定制。建元表歲,示人君萬世之傳;紀時書王,見天下一家之義。法《春秋》之正始,體大《易》之乾元。炳煥皇猷,權輿治道。可自庚申年五月十九日,建元為中統元年。惟即位體元之始,必立經陳紀為先。故內立都省,以總宏綱;外設總司,以平庶政。仍以興利除害之事、補偏救弊之方,隨詔以頌。於戲!秉籙握樞,必因時而建號;施仁發政,期與物以更新。敷宣懇惻之辭,表著憂勞之意。凡在臣庶,體予至懷!



 詔安撫壽春府軍民。甲午,以阿里不哥反,詔赦天下。乙未,立十路宣撫司:以賽典赤、李德輝為燕京路宣撫使,徐世隆副之;宋子貞為益都濟南等路宣撫使,王磐副之;河南路經略使史天澤為河南宣撫使;楊果為北京等路宣撫使,趙昞副之;張德輝為平陽太原路宣撫使,謝瑄副之;孛魯海牙、劉肅並為真定路宣撫使;姚樞為東平路宣撫使,張肅副之;中書左丞張文謙為大名彰德等路宣撫使,游顯副之;粘合南合為西京路宣撫使,崔巨濟副之;廉希憲為京兆等路宣撫使。以汪惟正為鞏昌等處便宜都總帥,虎闌箕為鞏昌路元帥。詔諭成都路侍郎張威安撫元、忠、綿、資、邛、彭等州,西川、潼川、隆慶、順慶等府及各處山寨歸附官吏,皆給宣命、金符有差。詔平陽、京兆兩路宣撫司僉兵七千人,於延安等處守隘,以萬戶鄭鼎、昔剌忙古帶領之,貧不能應役者,官為資給。徵諸路兵三萬駐燕京近地,命諸路市馬萬匹送開平府。以總帥汪良臣統陜西漢軍於沿河守隘。立望雲驛,非軍事毋得輒入。熒惑入南斗,留五十餘日。



 六月戊戌,詔燕京、西京、北京三路宣撫司運米十萬石,輸開平府及撫州、沙井、凈州、魚兒濼,以備軍儲。以李璮為江淮大都督。劉太平等謀反,事覺伏誅,並誅乞帶不花於東川,明裏火者於西川。渾都海反。乙巳,李璮言:「獲宋諜者,言賈似道調兵,聲言攻漣州,遣人覘之,見許浦江口及射陽湖兵船二千艘,宜繕理城塹以備。」罷阿藍帶兒所簽解鹽戶軍百人。壬子,詔陜西四川宣撫司八春節制諸軍。乙卯,詔東平路萬戶嚴忠濟等發精兵一萬五千人赴開平。乙丑,以石長不為大理國總管,佩虎符。詔十路宣撫司造戰襖、裘、帽,各以萬計,輸開平。是月,召真定劉鬱,邢州郝子明,彰德胡祗遹,燕京馮渭、王光益、楊恕、李彥通、趙和之,東平韓文獻、張昉等,乘傳赴闕。高麗國王王倎遣其子永安公僖、判司宰事韓即來賀即位,以國王封冊、王印及虎符賜之。



 秋七月戊辰,敕燕京、北京、西京、真定、平陽、大名、東平、益都等路宣撫司,造羊裘、皮帽、褲、靴,皆以萬計,輸開平。己巳,以萬戶史天澤扈從先帝有功,賜銀萬五千兩。遣靈州種田民還京兆。庚午,賜山東行省大都督李璮金符二十、銀符五,俾給所部有功將士。癸酉,以燕京路宣慰使祃祃行中書省事,燕京路宣慰使趙璧平章政事,張啟元參知政事,王鶚翰林學士承旨兼修國史,河南路宣撫使史天澤兼江淮諸翼軍馬經略使。丙子,詔中書省給諸王塔察兒益都、平州封邑歲賦、金帛,並以諸王白虎、襲剌門所屬民戶、人匠、歲賦給之。詔造中統元寶交鈔。立互市於潁州、漣水、光化軍。北京路都元帥阿海乞免所部軍士征徭,從之。宋兵攻邊城,詔遣太尹、怯列、忙古帶率所部,合兵擊之。下詔褒賞行省大都督李璮。帝自將討阿里不哥。敕劉天麟規措中都析津驛傳馬。



 八月丙午,授中書左丞、行大名等路宣撫使張文謙虎符。丁未,詔都元帥紐璘所過毋擅捶掠官吏。己酉,立秦蜀行中書省,以京兆等路宣撫使廉希憲為中書省右丞,行省事。宋兵臨漣州,李璮乞諸道援兵。癸丑,賜必庠赤塔剌渾銀二千五百兩。李璮乞遣將益兵,渡淮攻宋,以方遣使修好,不從。癸亥,澤州、潞州旱,民饑,敕賑之。



 九月丁卯,帝在轉都兒哥之地,以阿里不哥遺命,下詔諭中外。乙亥,李璮復請攻宋,復諭止之。壬午,初置拱衛儀仗。是月,阿藍答兒率兵至西涼府,與渾都海軍合,詔諸王合丹、合必赤與總帥汪良臣等率師討之。丙戌,大敗其軍於姑臧,斬阿藍答兒及渾都海,西土悉平。



 冬十月丁未,李璮言宋兵復軍於漣州。癸丑,初行中統寶鈔。戊午,車駕駐昔光之地,命給官錢,雇在京橐駝,運米萬石,輸行在所。



 十一月戊子,發常平倉賑益都、濟南、濱棣饑民。



 十二月丙申,以禮部郎中孟甲、禮部員外郎李文俊使安南、大理。乙巳,李璮上將士功,命璮以益都官銀賞之。帝至自和林,駐蹕燕京近郊。始制祭享太廟祭器、法服。以梵僧八合思八為帝師,授以玉印,統釋教。立仙音院,復改為玉宸院,括樂工。立儀鳳司,又立符寶局及御酒庫、群牧所。升衛輝為總管府。賜親王穆哥銀二千五百兩;諸王按只帶、忽剌忽兒、合丹、忽剌出、勝納合兒銀各五千兩,文綺帛各三百匹,金素半之;諸王塔察、阿術魯鈔各五十九錠有奇,綿五千九十八斤,絹五千九十八匹,文綺三百匹,金素半之;海都銀八百三十三兩,文綺五十匹,金素半之;睹兒赤、也不幹銀八百五十兩;兀魯忽帶銀五千兩,文綺三百匹,金素半之;只必帖木兒銀八百三十三兩;爪都、伯木兒銀五千兩,文綺三百匹,金素半之;都魯、牙忽銀八百三十三兩,特賜綿五十斤;阿只吉銀五千兩,文綺三百,金素半之;先朝皇后怗古倫銀二千五百兩,羅絨等折寶鈔二十三錠有奇;皇后斡者思銀二千五百兩;兀魯忽乃妃子銀五千兩。自是歲以為常。



 二年春正月辛未夜,東北赤氣照人,大如席。乙酉,宋兵圍漣州。己丑,李璮率將士迎戰,敗之,賜詔獎諭,給金銀符以賞將士。庚寅,璮擅發兵修益都城塹。



 二月丁酉,太陰掩昴。己亥,宋兵攻漣水,命阿術等帥兵赴之。丙午,車駕幸開平。詔減免民間差發,罷守隘諸軍。秦蜀行省借民錢給軍,以今年稅賦償之。免平陽、太原軍站戶重科租稅。丁未,詔行中書省平章祃祃及王文統等率各路宣撫使赴闕。丁巳,李璮破宋兵於沙湖堰。



 三月壬戌朔,日有食之。夏四月丙午,詔軍中所俘儒士聽贖為民。辛亥,遣弓工往教鄯闡人為弓。乙卯,詔十路宣撫使量免民間課程。命宣撫司官勸農桑,抑游惰,禮高年,問民疾苦,舉文學才識可以從政及茂才異等,列名上聞,以聽擢用;其職官污濫及民不孝悌者,量輕重議罰。辛酉,詔太康弩軍二千八百人戍蔡州。以禮部郎中劉芳使大理等國。



 五月乙丑,禁使臣毋入民家,令止頓析津驛。遣崔明道、李全義為詳問官,詣宋淮東制司,訪問國信使郝經等所在,仍以稽留信使、侵擾疆場詰之。庚辰,敕使臣及軍士所過城邑,官給廩餼,毋擾於民。丁亥,申嚴沿邊軍民越境私商之禁。唐慶子政臣入見,詔復其家。弛諸路山澤之禁。禁私殺馬牛。申嚴越境私商,販馬匹者罪死。以河南經略宣撫使史天澤為中書右丞相,河南軍民並聽節制。詔成都路置惠民藥局。遣王祐於西川等路採訪醫、儒、僧、道。



 六月癸巳,括漏籍老幼等戶,協濟編戶賦稅。丙申,賜新附人王顯忠、王誼等衣物有差。李璮遣人獻漣水捷。罷諸路拘收孛蘭奚。禁諸王擅遣使招民及徵私錢。戊戌,太陰犯角。詔諭十路宣撫司並管民官,定鹽酒稅課等法。癸卯,以嚴忠範為東平路行軍萬戶兼管民總管,仍諭東平路達魯花赤等官並聽節制。詔定中外官所乘馬數各有差。乙巳,賑火少里驛戶之乏食者。賞欽察所部將校有功者銀二千五百兩及幣帛有差。己酉,命竇默仍翰林侍講學士。默與王鶚面論王文統不宜在相位,薦許衡代之,帝不懌而罷。辛亥,轉懿州米萬石賑親王塔察兒所部饑民。賜親王合丹所部軍幣帛九百匹、布千九百匹。乙卯,敕平陽路安邑縣蒲萄酒自今毋貢。詔:「宣聖廟及管內書院,有司歲時致祭,月朔釋奠,禁諸官員使臣軍馬,毋得侵擾褻瀆,違者加罪。」丙辰,以汪良臣同簽鞏昌路便宜都總帥,凡軍民官並聽良臣節制。丁巳,敕諸路造人馬甲及鐵裝具萬二千,輸開平。戊午,詔毋收衛輝、懷孟賦稅,以償其所借芻粟。庚申,宋瀘州安撫使劉整舉城降,以整行夔府路中書省兼安撫使,佩虎符。仍諭都元帥紐璘等使存恤其民。賜故金翰林修撰魏璠謚靖肅。秦蜀行省言青居山都元帥欽察等所部將校有功,詔降虎符一、金符五、銀符五十七,令行省銓定職名給之。城臨洮。升真定鼓城縣為晉州,以鼓城、安平、武強、饒陽隸焉。賜僧子聰懷孟、邢州田各五十頃。罷金、銀、銅、鐵、丹粉、錫碌坑冶所役民夫及河南舞陽姜戶、藤花戶,還之州縣。賜大理國主段實虎符,優詔撫諭之。命李璮領益都路鹽課。出工局繡女,聽其婚嫁。懷孟廣濟渠提舉王允中、大使楊端仁鑿沁河渠成,溉田四百六十餘所。高麗國王倎更名禃,遣其世子愖奉表來朝,命宿衛將軍孛里察、禮部郎中高逸民持詔往諭,仍以玉帶賜之。以不花為中書右丞相,耶律鑄為中書左丞相,張啟元為中書右丞。授管領崇慶府、黎、雅、威、茂、邛、灌七處軍民小太尉虎符。



 秋七月辛酉朔,立軍儲都轉運使司,以馬月合乃為使,周鍇為副使。癸亥,初立翰林國史院。王鶚請修遼、金二史,又言:「唐太宗置弘文館,宋太宗設內外學士院。今宜除拜學士院官,作養人才。乞以右丞相史天澤監修國史,左丞相耶律鑄、平章政事王文統監修《遼》、《金史》,仍採訪遺事。」並從之。賑和林饑民。賞鞏昌路總帥汪惟正將校斬渾都海功銀二千五百兩、馬價銀四千九百兩。諸王昌童招河南漏籍戶五百,命付之有司。命總管王青制神臂弓、柱子弓。諭河南管軍官於近城地量存牧場,余聽民耕。巴思答兒乞於高麗鴨綠江西立互市,從之。乙丑,遣使持香幣祀岳瀆。丁丑,渡江新附民留屯蔡州者,徙居懷孟,貸其種食。以萬家奴為安撫高麗軍民達魯花赤,賜虎符。庚辰,西京、宣德隕霜殺稼。辛巳,詔許衡即其家教懷孟生徒。命西京宣撫司造船備西夏漕運。壬午,遣納速剌丁、孟甲等使安南。乙酉,以牛驛雨雪,道途泥濘,改立水驛。己丑,命煉師王道婦於真定築道觀,賜名玉華。諭將士舉兵攻宋,詔曰:「朕即位之後,深以戢兵為念,故年前遣使於宋以通和好。宋人不務遠圖,伺我小隙,反啟邊釁,東剽西掠,曾無寧日。朕今春還宮,諸大臣皆以舉兵南伐為請,朕重以兩國生靈之故,猶待信使還歸,庶有悛心,以成和議,留而不至者,今又半載矣。往來之禮遽絕,侵擾之暴不已。彼嘗以衣冠禮樂之國自居,理當如是乎?曲直之分,灼然可見。今遣王道貞往諭。卿等當整爾士卒,礪爾戈矛,矯爾弓矢,約會諸將,秋高馬肥,水陸分道而進,以為問罪之舉。尚賴宗廟社稷之靈,其克有勛。卿等當宣布朕心,明諭將士,各當自勉,毋替朕命。」鄂州青山磯、滸黃洲所招新民遷至江北者,設官領之。敕懷孟牧地聽民耕墾。



 八月壬辰,賜故金補闕李大節謚貞肅。丁酉,命開平守臣釋奠於宣聖廟。戊戌,以燕京等路宣撫使賽典赤為平章政事,敕以賀天爵為金齒等國安撫使,忽林伯副之,仍招諭使安其民。己亥,諭武衛軍都指揮使李伯祐汰本軍疲老者,選精銳代之,給海青銀符一,有奏,馳驛以聞。辛丑,以宣撫使粘合南合為中書右丞,闊闊為中書左丞,賈文備為開元女直水達達等處宣撫使,賜虎符。以宋降將王青為總管,教武衛軍習射。乙巳,禁以俘掠婦女為娼。丙午,太白犯歲星。以許衡為國子祭酒。丁未,以姚樞為大司農,竇默仍翰林侍講學士。先是,以樞為太子太師,衡為太子太傅,默為太子太保,樞等以不敢當師傅禮,皆辭不拜,故復有是命。初立勸農司,以陳邃、崔斌、成仲寬、粘合從中為濱棣、平陽、濟南、河間勸農使,李士勉、陳天錫、陳膺武、忙古帶為邢洺、河南、東平、涿州勸農使。己酉,命大名等路宣撫使歲給翰林侍講學士竇默、太醫副使王安仁衣糧,賜田以為永業。甲寅,賞董文炳所將渡江及北征有功者二十二人,銀各五十兩。封順天等路萬戶張柔為安肅公,濟南路萬戶張榮為濟南公。陜西四川行省乞就決邊方重刑,不允。詔陜西四川行省存恤歸附軍民。詔:「自今使臣有矯稱上命者,有司不得聽受。諸王、后妃、公主、駙馬非聞奏,不許擅取官物。」賜慶壽寺、海雲寺陸地五百頃。敕西京運糧於沙井,北京運糧於魚兒泊。立檀州驛。頒斗斛權衡。賑桓州饑民。賜諸王塔察兒金千兩、銀五千兩、幣三百匹。給阿石寒甲價銀千二百兩。核實新增戶口,措置諸路轉輸法。命劉整招懷夔府、嘉定等處民戶。宋私商七十五人入宿州,議置於法,詔宥之,還其貨,聽榷場貿易。仍檄宋邊將還北人之留南者。



 九月庚申朔,詔以忽突花宅為中書省署。奉遷祖宗神主於聖安寺。癸亥,邢州安撫使張耕告老,詔以其子鵬翼代之。武衛親軍都指揮使李伯祐、董文炳言:「武衛軍疲老者,乞補換,仍存恤其家。」從之。丙寅,詔以粘合南合行中興府中書省。戊辰,大司農姚樞請以儒人楊庸教孔、顏、孟三氏子孫,東平府詳議官王鏞兼充禮樂提舉。詔以庸為教授,以鏞特兼太常少卿。辛未,以清、滄鹽課銀償往歲所貸民錢給公費者。置和糴所於開平,以戶部郎中宋紹祖為提舉和糴官。丙子,諭諸王、駙馬,凡民間詞訟無得私自斷決,皆聽朝廷處置。河南民王四妻靳氏一產三男,命有司量給贍養。敕今歲田租輸沿河近倉,官為轉漕,不可勞民。癸未,以甘肅等處新罹兵革,民務農安業者為戍兵所擾,遣阿沙、焦端義往撫治之。以海青銀符二、金符十給中書省,量軍國事情緩急,付乘驛者佩之。以開元路隸北京宣撫司。真定路官民所貸官錢,貧不能償,詔免之。王鶚請於各路選委博學老儒一人,提舉本路學校,特詔立諸路提舉學校官,以王萬慶、敬鉉等三十人充之。敕燕京、順天等路續制人甲五千、馬甲及鐵裝具各二千。



 冬十月庚寅朔,詔鳳翔府種田戶隸平陽兵籍,毋令出征,務耕屯以給軍餉。辛卯,陜西四川行省上言:「軍務急速,若待奏報,恐失事機。」詔與都元帥紐璘會議行之。遣道士訾洞春代祀東海廣德王廟。壬辰,敕火兒赤、奴懷率所部略地淮西。丁酉,敕愛亦伯等及陜西宣撫司校核不魯歡、阿藍塔兒所貸官銀。庚子,以右丞張啟元行中書省於平陽、太原等路。括西京兩路官民,有壯馬皆從軍,令宣德州楊庭訓統之,有力者自備甲仗,無力者官與供給。兩路奧魯官並在家軍人,凡有馬者並付新軍劉總管統領。昂吉所管西夏軍,並豐州、蕁麻林、夏水阿剌渾皆備鞍馬甲仗,及孛魯歡所管兵,凡徒行者市馬給之,並令從軍,違者以失誤軍期論。修燕京舊城。命平章政事趙璧、左三部尚書怯烈門率蒙古、漢軍駐燕京近郊、太行一帶,東至平灤,西控關陜,應有險阻,於附近民內選諳武事者,修立堡寨守御。以河南屯田萬戶史權為江漢大都督,依舊戍守。又選銳卒三千付史樞管領,於燕京近郊屯駐。壬寅,命亳州張柔、歸德邸浹、睢州王文乾、水軍解成、張榮實、東平嚴忠嗣、濟南張宏七萬戶,以所部兵來會。罷東平會計前任官侵用財賦。甲辰,宋兵攻瀘州,劉整擊敗之。詔賞整銀五千兩,幣帛二千匹。失裡答、劉元振守御有功,各賞銀五百兩,將士銀萬兩、幣帛千匹。乙巳,詔指揮副使鄭江將千人赴開平,指揮使董文炳率善射者千人由魚兒泊赴行在所,指揮使李伯祐率餘兵屯潮河川。壬子,詔霍木海、乞帶等自得勝口至中都預備糧餉芻粟。丙辰,詔平章政事塔察兒率軍士萬人,由古北口西便道赴行在所。



 十一月壬戌,大兵與阿里不哥遇於昔木土腦兒之地,諸王合丹等斬其將合丹火兒赤及其兵三千人,塔察兒與合必赤等復分兵奮擊,大破之,追北五十餘里。帝親率諸軍以躡其後,其部將阿脫等降,阿里不哥北遁。庚午,太陰犯昴。壬申,詔免今年賦稅。癸酉,駐蹕帖買和來之地。以尚書怯烈門、平章趙璧兼大都督,率諸軍從塔察兒北上。分蒙古軍為二,怯烈門從麥肖出居庸口,駐宣德德興府;訥懷從阿忽帶出古北口,駐興州。帝親將諸萬戶漢軍及武衛軍,由檀、順州駐潮河川。敕官給芻糧,毋擾居民。罷十路宣撫司,止存開元路。命諸路市馬二萬五千餘匹,授蒙古軍之無馬者。丁丑,徵諸路宣撫司官赴中都。移蹕於速木合打之地。詔漢軍屯懷來、縉山。鷹坊阿里沙及阿散兄弟二人以擅離扈從伏誅。



 十二月庚寅,詔封皇子真金為燕王,領中書省事。辛卯,熒惑犯房。壬辰,熒惑犯鉤鈐。癸巳,以昌、撫、蓋利泊等處薦罹兵革,免今歲租賦。甲午,師還,詔撤所在戍兵,放民間新簽軍。命太常少卿王鏞教習大樂。壬寅,以隆寒命諸王合必赤所部軍士無行帳者,聽舍民居。命陜蜀行中書省給綏德州等處屯田牛、種、農具。初立宮殿府,秩正四品,專職營繕。立尚食局、尚藥局。初設控鶴五百四人,以劉德為軍使領之。立異樣局達魯花赤,掌御用織造,秩正三品,給銀印。賜諸王金銀幣帛如歲例。是歲,天下戶一百四十一萬八千四百九十有九,斷死罪四十六人。



\end{pinyinscope}