\article{本紀第十五 世祖十二}

\begin{pinyinscope}

 二十五年春正月,日烜復走入海,鎮南王以諸軍追之,不及,引兵還交趾城。命烏馬兒將水兵迎張文虎等糧船,又發兵攻其諸寨,破之。己丑,詔江淮省管內並聽忙兀帶節制。庚寅,祭日於司天臺。賜諸王火你赤銀五百兩、珠一索、錦衣一襲,玉都銀千兩、珠一索、錦衣一襲。辛卯,尚書省臣言:「初以行省置丞相與內省無別,罷之。今江淮平章政事忙兀帶所統,地廣事繁,乞依前為丞相。」詔以忙兀帶為右丞相。以蘄、黃二州、壽昌軍隸湖廣省。毀中統鈔板。乙未,賞征東功:從乘輿,將吏升散官二階,軍士鈔人三錠;從皇孫,將吏升散官一階,軍士鈔人二錠;死者給其家十錠。凡為鈔四萬一千四百二十五錠。丁酉,遣使代祀岳瀆、東海、后土。戊戌,大赦。敕弛遼陽漁獵之禁,惟毋殺孕獸。壬寅,高麗遣使來貢方物。賀州賊七百餘人焚掠封州諸郡,循州賊萬餘人掠梅州。癸卯,海都犯邊。敕駙馬昌吉,諸王也只烈,察乞兒、合丹兩千戶,皆發兵從諸王術伯北征。賜諸王亦憐真部曲鈔三萬錠。掌吉舉兵叛,諸王拜答罕遣將追之,至八立渾,不及而還。甲辰,也速不花謀叛,逮捕至京師,誅之。乙巳,太陰犯角。蠻洞十八族饑餓,死者二百餘人,以鈔千五百錠有奇市米賑之。丙午,畋於近郊。以平江鹽兵屯田於淮東、西。杭、蘇二州連歲大水,賑其尤貧者。戊申,太陰犯房。己酉,詔中興、西涼無得沮壞河渠,兩淮、兩浙無得沮壞歲課。發海運米十萬石,賑遼陽省軍民之饑者。辛亥,省器盒局入諸路金玉人匠總管府。癸丑,詔:「行大司農司、各道勸農屯田司,巡行勸課,舉察勤惰,歲具府、州、縣勸農官實跡,以為殿最,路經歷官、縣尹以下並聽裁決。或怙勢作威侵官害農者,從提刑按察司究治。」募民能耕江南曠土及公田者,免其差役三年,其輸租免三分之一。江淮行省言:「兩淮土曠民寡,兼並之家皆不輸稅。又,管內七十餘城,止屯田兩所,宜增置淮東、西兩道勸農營田司,督使耕之。」制曰:「可。」



 二月丁巳,改濟州漕運司為都漕運司,並領濟之南北漕,京畿都漕運司惟治京畿。鎮南王引兵還萬劫。烏馬兒迎張文虎等糧船不至,諸將以糧盡師老,宜全師而還,鎮南王從之。戊午,命李庭整漢兵五千東征。賜葉李平江、嘉興田四頃。庚申,司徒撒里蠻等進讀《祖宗實錄》,帝曰:「太宗事則然,睿宗少有可易者,定宗固日不暇給,憲宗汝獨不能憶之耶?猶當詢諸知者。」征大都南諸路所放扈從馬赴京,官給芻粟價,令自糴之,無擾諸縣民。遼陽、武平等處饑,除今年租稅及歲課貂皮。浚滄州鹽運渠。辛酉,忙兀帶、忽都忽言其軍三年薦饑,賜米五百石。壬戌,省遼東海西道提刑按察司入北京,江南湖北道提刑按察司入荊南。敕江淮勿捕天鵝,弛魚濼禁。丙寅,賜雲南王塗金駝鈕印。改南京路為汴梁路,北京路為武平路,西京路為大同路,東京路為遼陽路,中興路為寧夏府路。改江西茶運司為都轉運使司,並榷酒醋稅。改河渠提舉司為轉運司。江淮總攝楊璉真加言以宋宮室為塔一,為寺五,已成,詔以水陸地百五十頃養之。詔征葛洪山隱士劉彥深。甲戌,蓋州旱,民饑,蠲其租四千七百石。己卯,以高麗國王王睶復為征東行尚書省左丞相。豪、懿州饑,以米十五萬石賑之。禁遼陽酒。京師水,發官米,下其價糶貧民。以江南站戶貧富不均,命有司料簡,合戶稅至七十石當馬一匹,並免雜徭;獨戶稅逾七十石願入站者聽。合戶稅不得過十戶,獨戶稅無上百石。辛巳,以杭州西湖為放生池。壬午,鎮南王命烏馬兒、樊楫將水兵先還,程鵬飛、塔出將兵護送之。以御史臺監察御史、提刑按察司多不舉職,降詔申飭之。命皇孫云南王也先鐵木兒帥兵鎮大理等處。



 三月丙戌,諸王昌童部曲饑,給糧三月。丁亥,熒惑犯太微東垣上相。戊子,太陰犯畢。車駕還宮。淞江民曹夢炎願歲以米萬石輸官,乞免他徭,且求官職。桑哥以為請,遙授浙東道宣慰副使。改曲靖路總管府為宣撫司。庚寅,大駕幸上都。改闌遺所為闌遺監,升正四品。敕遼陽省亦乞列思、吾魯兀、札剌兒探馬赤自懿州東征。李庭遙授尚書左丞,食其祿,將漢兵以行。江淮行省忙兀帶言:「宜除軍官更調法,死事者增散官,病故者降一等。」帝曰:「父兄雖死事,子弟不勝任者,安可用之?茍賢矣,則病故者亦不可降也。」辛卯,以六衛漢兵千二百、新附軍四百、屯田兵四百造尚書省。鎮南王以諸軍還。張文虎糧船遇賊兵船三十艘,文虎擊之,所殺略相當。費拱辰、徐慶以風不得進,皆至瓊州。凡亡士卒二百二十人、船十一艘、糧萬四千三百石有奇。癸巳,賜諸王術伯銀五萬兩、幣帛各一萬匹,兀魯臺、爪忽兒銀五千兩、幣帛各一百。甲午,禁捕鹿羔。鎮南王次內傍關,賊兵大集以遏歸師,鎮南王遂由單巳縣趣盝州,間道以出。乙未,以往歲北邊大風雪,拔突古倫所部牛馬多死,賜米千石。丁酉,駐蹕野狐嶺,命阿束、塔不帶總京師城守諸軍。己亥,太陰掩角。壬寅,禮部言:「會同館蕃夷使者時至,宜令有司仿古《職貢圖》,繪而為圖,及詢其風俗、土產、去國里程,籍而錄之,實一代之盛事。」從之。鎮南王次思明州,命愛魯引兵還雲南,奧魯赤以諸軍北還。日烜遣使來謝,進金人代己罪。乙巳,詔江西管內並聽行尚書省節制。戊申,改山東轉運使司為都轉運使司,兼濟南路酒稅醋課。己酉,徐、邳屯田及靈壁、睢寧二屯雨雹如雞卵,害麥。甲寅,循州賊萬餘人寇漳浦,泉州賊二千人寇長泰、汀、贛,畬賊千餘人寇龍溪,皆討平之。



 夏四月丙辰,萊縣、蒲臺旱饑,出米下其直賑之。戊午,太陰犯井。庚申,以武岡、寶慶二路薦經寇亂,免今年酒稅課及前歲逋租。辛酉,從行泉府司沙不丁、烏馬兒請,置鎮撫司、海船千戶所、市舶提舉司。省平陽投下總管府入平陽路,雜造提舉司入雜造總管府。桑哥言:「自至元丙子置應昌和糴所,其間必多盜詐,宜加鉤考。扈從之臣,種地極多,宜依軍站例,除四頃之外,驗畝征租。」並從之。癸亥,渾河決,發軍築堤捍之。乙丑,廣東賊董賢舉等七人皆稱大老,聚眾反,剽掠吉、贛、瑞、撫、龍興、南安、韶、雄、汀諸郡,連歲擊之不能平,江西行樞密院副使月的米失請益兵,江西行省平章忽都鐵木兒亦以地廣兵寡為言,詔江淮省分萬戶一軍詣江西,俟賊平還翼。戊辰,浚怯烈河以溉口溫腦兒黃土山民田。庚午,立弘吉剌站。癸酉,尚書省臣言:「近以江淮饑,命行省賑之,吏與富民因結為奸,多不及於貧者。今杭、蘇、湖、秀四州復大水,民鬻妻女易食,請輟上供米二十萬石,審其貧者賑之。」帝是其言。甲戌,萬安寺成,佛像及窗壁皆金飾之,凡費金五百四十兩有奇、水銀二百四十斤。遼陽省新附軍逃還各衛者,令助造尚書省,仍命分道招集之。增立直沽海運米倉。命征交趾諸軍還家休息一歲。敕緬中行省,比到緬中,一稟云南王節制。庚辰,安南國王陳日烜遣中大夫陳克用來貢方物。賜諸王小薛金百兩、銀萬兩、鈔千錠及幣帛有差。辛巳,賜諸王阿赤吉金二百兩、銀二萬二千五百兩、鈔九千錠及紗羅絹布有差。命甘肅行省發新附軍三百人屯田亦集乃,陜西省督鞏昌兵五千人屯田六盤山。癸未,雲南省右丞愛魯上言:「自發中慶,經羅羅、白衣入交趾,往返三十八戰,斬首不可勝計,將士自都元帥以下獲功者四百七十四人。」甲申,詔皇孫撫諸軍討叛王火魯火孫、合丹禿魯幹。



 五月丙戌,敕武平路括馬千匹。戊子,諸王察合子闊闊帶叛,床兀兒執之以來。己丑,汴梁大霖雨,河決襄邑,漂麥禾。以左右怯薛衛士及漢軍五千三百人從皇孫北征。甲午,發五衛漢兵五千人北征。乙未,桑哥言:「中統鈔行垂三十年,省官皆不知其數,今已更用至元鈔,宜差官分道置局鉤考中統鈔本。」從之。丙申,賜諸王八八金百兩、銀萬兩、金素段五百、紗羅絹布等四千五百。兀馬兒來獻璞玉。丁酉,平江水,免所負酒課。減米價,賑京師。改雲南烏撒宣撫司為宣慰司,兼管軍萬戶府。戊戌,復蘆臺、越支、三叉沽三鹽使司。王家奴、火魯忽帶、察罕復舉兵反。己亥,雲南行省言:「金沙江西通安等五城,宜依舊隸察罕章宣撫司,金沙江東永寧等處五城宜廢,以北勝施州為北勝府。」從之。壬寅,渾天儀成。運米十五萬石詣懿州餉軍及賑饑民。乙巳,罷興州採蜜提舉司。營上都城內倉。丁未,奉安神主於太廟。戊申,太白犯畢。賜拔都不倫金百五十兩、銀萬五千兩及幣帛紗羅等萬匹。辛亥,盂州烏河川雨雹五寸,大者如拳。癸丑,詔湖廣省管內並聽平章政事禿滿、要束木節制。遷四川省治重慶,復遷宣慰司於成都。高麗遣使來貢方物。詔四川管內並聽行尚書省節制。河決汴梁,太康、通許、杞三縣,陳、潁二州皆被害。



 六月甲寅,以新府軍修尚食局。庚申,賑諸王答兒伯部曲之饑者及桂陽路饑民。辛酉,禁上都、桓州、應昌、隆興酒。壬戌,賜諸王術伯金銀皆二百五十兩、幣帛紗羅萬匹。乙丑,詔蒙古人總漢軍,閱習水戰。丁卯,又賜諸王術伯銀二萬五千兩、幣帛紗羅萬匹。復立咸平至建州四驛。以延安屯田總管府復隸安西省。戊辰,海都將暗伯、著暖以兵犯業里乾腦兒,管軍元帥阿裏帶戰卻之。壬申,睢陽霖雨,河溢害稼,免其租千六十石有奇。命諸王怯憐口及扈從臣,轉米以饋將士之從皇孫者。太醫院、光祿寺、儀鳳寺、侍儀司、拱衛司,皆毋隸宣徽院,罷教坊司入拱衛司。癸酉,詔加封南海明著天妃為廣祐明著天妃。甲戌,太白犯井。改西南番總管府為永寧路。乙亥,以考城、陳留、通許、杞、太康五縣大水及河溢沒民田,蠲其租萬五千二百石。丙子,給兵五十人衛浙西宣慰使史弼,使任治盜之責。丁丑,太陰犯歲星。發兵千五百人詣漢北浚井。癸未,處州賊柳世英寇青田、麗水等縣,浙東道宣慰副使史耀討平之。資國、富昌等一十六屯雨水、蝗害稼。



 秋七月甲申朔,復葺興、靈二州倉,始命昔寶赤、合剌赤、貴由赤、左右衛士轉米輸之,委省官督運,以備賑給。丙戌,真定、汴梁路蝗。運大同、太原諸倉米至新城,為邊地之儲。以南安、瑞、贛三路連歲盜起,民多失業,免逋稅萬二千六百石有奇。弛寧夏酒禁。發大同路粟賑流民。保定路霖雨害稼,蠲今歲田租。改儲偫所為提舉司。敕征交趾兵官還家休息一歲。壬辰,遣必闍赤以鈔五千錠往應昌和糴軍儲。改會同館為四賓庫。戊戌,駐蹕許泥百牙之地。同知江西行樞密院事月的迷失上言:「近以盜起廣東,分江西、江淮、福建三省兵萬人令臣將之討賊。臣願萬人內得蒙古軍三百,並臣所籍降戶萬人,置萬戶府,以撒木合兒為達魯花赤,佩虎符。」詔許之。以沐川等五寨割隸嘉定者,還隸馬湖蠻部總管府。己亥,熒惑犯氐。庚子,太白犯鬼。膠州連歲大水,民採橡而食,命減價糶米以賑之。霸、漷二州霖雨害稼,免其今年田租。乙巳,太陰掩畢。諸王也真部曲饑,分五千戶就食濟南。保定路唐縣野蠶繭絲可為帛。壬子,命斡端戍兵三百一十人屯田。命六衛造軍器。



 八月癸丑,諸王也真言:「臣近將濟寧投下蒙古軍東征,其家皆乏食,願賜濟南路歲賦銀,使易米而食。」詔遼陽省給米萬石賑之。丙辰,熒惑犯房。袁之萍鄉縣進嘉禾。詔安童以本部怯薛蒙古軍三百人北征。己未,太白犯軒轅大星。辛酉,免江州學田租。癸亥,尚書省成。壬申,安西省管內大饑,蠲其田租二萬一千五百石有奇,仍貸粟賑之。癸酉,以河間等路鹽運司兼管順德、廣平、綦陽三鐵冶。丙子,發米三千石賑滅吉兒帶所部饑民。趙、晉、冀三州蝗。丁丑,嘉祥、魚臺、金鄉三縣霖雨害稼,蠲其租五千石。庚辰,車駕次孛羅海腦兒。以咸平薦經兵亂,發沈州倉賑之。分萬億庫為寶源、賦源、綺源、廣源四庫。



 九月癸未朔,熒惑犯天江。大駕次野狐嶺。甘州旱饑,免逋稅四千四百石。丙戌,置汀、梅二州驛。己丑,獻、莫二州霖雨害稼,免田租八百餘石。壬辰,大駕至大都。乙未,罷檀州淘金戶。都哇犯邊。庚子,太陰犯畢。鬼國、建都皆遣使來貢方物。從桑哥請,營五庫禁中以貯幣帛。癸卯,熒惑犯南斗。命忽都忽民戶履地輸稅。尚書省臣言:「自立尚書省,凡倉庫諸司無不鉤考,宜置徵理司,秩正三品,專治合追財穀,以甘肅等處行尚書省參政禿烈羊呵、簽省吳誠並為征理使。」從之。升寶鈔總庫、永盈庫並為從五品。改八作司為提舉八作司,秩正六品。增元寶、永豐及八作司官吏俸。庚戌,太醫院新編《本草》成。



 冬十月己未,享於太廟。庚申,從桑哥請,以省、院、臺官十二人理算江淮、江西、福建、四川、甘肅、安西六省錢穀,給兵使以為衛。烏思藏宣慰使軟奴汪術嘗賑其管內兵站饑戶,桑哥請賞之,賜銀二千五百兩。甲子,置虎賁司,復改為武衛司。丙寅,賜瀛國公趙鳷鈔百錠。以甘州轉運司隸都省。湖廣省言:「左、右江口溪洞蠻獠,置四總管府,統州、縣,洞百六十,而所調官畏憚瘴癘,多不敢赴,請以漢人為達魯花赤,軍官為民職,雜土人用之。」就擬夾谷三合等七十四人以聞,從之。大同民李伯祥、蘇永福八人,以謀逆伏誅。庚午,海都犯邊。桑哥請明年海道漕運江南米須及百萬石。又言:「安山至臨清,為渠二百六十五里。若開浚之,為工三百萬,當用鈔三萬錠、米四萬石、鹽五萬斤。其陸運夫萬三千戶復罷為民,其賦入及芻粟之估為鈔二萬八千錠,費略相當,然渠成亦萬世之利。請以今冬備糧費,來春浚之。」制可。丙子,始造鐵羅圈甲。瀛國公趙鳷學佛法於土番。己卯,也不乾入寇,不都馬失引兵奮擊之。塔不帶反,忽剌忽、阿塔海等戰卻之。詔免儒戶雜徭。尚書省臣請令集賢院諸司,分道鉤考江南郡學田所入羨餘,貯之集賢院,以給多才藝者,從之。給倉官俸。高麗遣使來貢方物。



 十一月壬午,鞏昌路薦饑,免田租之半,仍以鈔三千錠賑其貧者。以忽撒馬丁為管領甘肅陜西等處屯田等戶達魯花赤,督斡端、可失合兒工匠千五十戶屯田。丁亥,金齒遣使貢方物。以山東東西道提刑按察使何榮祖為中書省參知政事。修國子監以居胄子。禁有分地臣私役富室為柴米戶及賦外雜徭。柳州民黃德清叛,潮州民蔡猛等拒殺官軍,並伏誅。庚寅,床哥里合引兵犯建州,殺三百餘人,咸平大震。辛卯,兀良合饑民多殍死,給三月糧。壬辰,罷建昌路屯田總管府。癸巳,賜諸王也裡乾金五十兩、銀五千兩、鈔千錠、幣帛紗羅等二千匹。也速帶兒、牙林海剌孫執捏坤、忽都答兒兩叛王以歸。甲午,北兵犯邊。詔福建省管內並聽行尚書省節制。丙申,合迷里民饑,種不入土,命愛牙赤以屯田餘糧給之。己亥,命李思衍為禮部侍郎,充國信使,以萬奴為兵部郎中副之,同使安南,詔諭陳日烜親身入朝,否則必再加兵。大都民史吉等請立桑哥德政碑,從之。辛丑,馬八兒國遣使來朝。帖列滅入寇。甲辰,以鞏昌便宜都總帥府統五十餘城兵民事繁,改為宣慰使司,兼便宜都總帥府。改釋教總制院為宣政院,秩從一品,印用三臺,以尚書右丞相桑哥兼宣政使。庚戌,益咸平府戍兵三百。



 十二月乙卯,賜按答兒禿等金千二百五十兩、銀十二萬五千兩、鈔二萬五千錠、幣帛布氎布二萬三千六百六十六匹。命上都募人運米萬石赴和林,應昌府運米三萬石給弘吉剌軍。丁巳,海都兵犯邊,拔都也孫脫迎擊,死之。先是,安童將兵臨邊,為失里吉所執,一軍皆沒。至是八鄰來歸,從者凡三百九十人,賜鈔萬二千五百一十三錠。辛酉,太陰犯畢。癸亥,置大都等路打捕民匠等戶總管府。甲子,太陰犯井。辛未,桑哥言:「有分地之臣,例以貧乏為辭,希覬賜與。財非天墜地出,皆取於民,茍不慎其出入,恐國用不足。」帝曰:「自今不當給者汝即畫之,當給者宜覆奏,朕自處之。」甲戌,太陰犯亢,熒惑犯壘壁陣。安西王阿難答來告兵士饑,且闕橐駝,詔給米六千石及橐駝百。乙亥,湖頭賊張治囝掠泉州,免泉州今歲田租。丙子,也速不花以昔列門叛。甘肅行省官約諸王八八、拜答罕、駙馬昌吉,合兵討之,皆自縛請罪。獨昔列門以其屬西走,追至朵郎不帶之地,邀而獲之,以歸於京師。庚辰,六衛屯田饑,給更休三千人六十日糧。高麗國王遣使來貢方物。賜諸王愛牙合赤等金千兩、銀一萬八千三百六十兩、絲萬兩、綿八萬三千二百兩、金素幣一千二百匹、絹五千九十八匹。賜皇子愛牙赤部曲等羊馬鈔二十九萬百四十七錠、馬二萬六千九百一十四、羊十萬二百一十、駝八、牛九百。賙諸王貧乏者,鈔二十一萬六百錠、馬六千七百二十五、羊一萬二千八百五十七、牛四十。賜妻子家貲沒於寇者,鈔三萬二千八百八十錠、馬牛百,償以羊馬諸物供軍者,鈔千六百七十四錠、馬四千三百二十五、羊三萬四千百九十九、駝七十二、牛三十。賞自寇中拔歸者,鈔四千七十八錠。因雨雹、河溢害稼,除民租二萬二千八百石。命亦思麻等七百餘人作佛事坐靜於玉塔殿、寢殿、萬壽山、護國仁王等寺凡五十四會,天師張宗演設醮三日。以光祿寺直隸都省。置醴源倉,分太倉之麴米藥物隸焉。以滄州之軍營城為滄溟縣,以施州之清江縣隸夔路總管府。罷安和署。大司農言耕曠地三千五百七十頃,立學校二萬四千四百餘所,積義糧三十一萬五千五百餘石。斷死罪九十五人。



 二十六年春正月丙戌,地震。詔江淮省忙兀帶與不魯迷失海牙及月的迷失合兵進討群盜之未平者。己丑,發兵塞沙陀間鐵烈兒河。辛卯,拔都不倫言其民千一百五十八戶貧乏,賜銀十萬五千一百五十兩。徙江州都轉運使司治龍興。沙不丁上市舶司歲輸珠四百斤、金三千四百兩,詔貯之以待貧乏者。合丹入寇。戊戌,以荊湖占城省左丞唐兀帶副按的忽都合為蒙古都萬戶,統兵會江淮、福建二省及月的迷失兵,討盜於江西。蠲漳、汀二州田租。辛丑,遣使代祀岳瀆、后土、東南海。立武衛親軍都指揮使司,以侍衛軍六千、屯田軍三千、江南鎮守軍一千,合兵一萬隸焉。太陰犯氐。壬寅,海船萬戶府言:「山東宣慰使樂實所運江南米,陸負至淮安,易閘者七,然後入海,歲止二十萬石。若由江陰入江至直沽倉,民無陸負之苦,且米石省運估八貫有奇。乞罷膠萊海道運糧萬戶府,而以漕事責臣,當歲運三十萬石。」詔許之。癸卯,高麗遣使來貢方物。賊鐘明亮寇贛州,掠寧都,據秀嶺,詔發江淮省及鄰郡戍兵五千,遷江西省參政管如德為左丞,使將兵往討。畬民丘大老集眾千人寇長泰縣,福州達魯花赤脫歡同漳州路總管高傑討平之。甲辰,復立光祿寺。戊申,徙廣州按察司於韶州。以荊南按察司所統遼遠,割三路入淮西,二路入江西。立咸平至聶延驛十五所。廢甘州路宣課提舉司入寧夏都轉運使司。遣參知政事張守智、翰林直學士李天英使高麗,督助征日本糧。



 二月辛亥朔,詔籍江南戶口,凡北方諸色人寓居者亦就籍之。浚滄州御河。癸丑,愛牙合赤請以所部軍屯田咸平、懿州,以省糧餉。己未,發和林糧千石賑諸王火你赤部曲。置延禧司,秩正三品。壬戌,合木裡饑,命甘肅省發米千石賑之。癸亥,詔立崇福司,為從二品。徙江淮省治杭州,改浙西道宣慰司為淮東道宣慰司,治揚州。丙寅,尚書省臣言:「行泉府所統海船萬五千艘,以新附人駕之,緩急殊不可用。宜招集乃顏及勝納合兒流散戶為軍,自泉州至杭州立海站十五,站置船五艘、水軍二百,專運番夷貢物及商販奇貨,且防御海道為便。」從之。命福建行省拜降、江西行院月的迷失、江淮行省忙兀帶,合兵擊賊江西。大都路總管府判官蕭儀嘗為桑哥掾,坐受贓事覺,帝貸其死,欲徙為淘金。桑哥以儀嘗鉤考萬億庫,有追錢之能,足贖其死,宜解職杖遣之,帝曲從之。丁卯,幸上都。以中書右丞相伯顏知樞密院事,將北邊諸軍。成都管軍萬戶劉德祿上言,願以兵五千人招降八番蠻夷,因以進取交趾。樞密院請立元帥府,以藥剌罕及德祿並為都元帥,分四川軍萬人隸之,帝從之。以伯答兒為中書平章政事。紹興大水,免未輸田租。合丹兵寇胡魯口,開元路治中兀顏牙兀格戰連日,破之。己巳,立左右翼屯田萬戶府,秩從三品。玉呂魯奏,江南盜賊凡四百餘處,宜選將討之。帝曰:「月的迷失屢以捷聞,忙兀帶已往,卿無以為慮。」皇孫甘不剌所部軍乏食,發大同路榷場糧賑之。甲戌,命鞏昌便宜都總帥汪惟和將所部軍萬人北征,令過闕受命。乙亥,省屯田六署為營田提舉司。



 三月庚辰朔,日有食之。臺州賊楊鎮龍聚眾寧海,僭稱大興國,寇東陽、義烏,浙東大震。諸王甕吉帶時謫婺州,帥兵討之。立雲南屯田,以供軍儲。桑哥言:「省部成案皆財穀事,當令監察御史即省部稽照,書姓名於卷末,仍命侍御史堅童視之,失則連坐。」從之。安西饑,減估糶米二萬石。甘州饑,發鈔萬錠賑之。己丑,賜陜西屯田總管府農器種粒。癸巳,東流縣獻芝。甲午,太陰犯亢。乙未,鑄渾天儀成。癸巳,金齒人塞完以其民二十萬一千戶有奇來歸,仍進象三。



 夏四月己酉,復立營田司於寧夏府。遼陽省管內饑,貸高麗米六萬石以賑之。壬子,孛羅帶上別十八里招集戶數,令甘肅省賑之。癸丑,命塔海發忽都不花等所部軍,屯狗站北以禦寇。寶慶路饑,下其估糶米萬一千石。丙辰,命甘肅行省給合的所部饑者粟。丁巳,遣宮驗視諸王按灰貧民,給以糧。戊午,禁江南民挾弓矢,犯者籍而為兵。置江西福建打捕鷹坊總管府,福建轉運司及管軍總管言其非宜,詔罷之。省江淮屯田打捕提舉司七所,存者徐邳、海州、揚州、兩淮、淮安、高郵、昭信、安豐、鎮巢、蘄黃、魚網、石湫,猶十二所。甲子,池州貴池縣民王勉進紫芝十二本。戊辰,安南國王陳日烜遣其中大夫陳克用等來貢方物。己巳,乞兒乞思戶居和林,驗其貧者賑之。庚午,沙河決,發民築堤以障之。癸酉,以高麗國多產銀,遣工即其地,發旁近民冶以輸官。以萊蕪鐵冶提舉司隸山東鹽運司。甲戌,以御史大夫玉呂魯為太傅,加開府儀同三司,簽江西等處行尚書省事。召江淮行省參知政事忻都赴闕,以戶部尚書王巨濟專理算江淮省,左丞相忙兀帶總之。置浙東、江東、江西、湖廣、福建木綿提舉司,責民歲輸木綿十萬匹,以都提舉司總之。罷皇孫按攤不花所設斷事官也先,仍收其印。尚書省臣言:「鞏昌便宜都總帥府已升為宣慰使司,乞以舊兼府事別立散府,調官分治。」從之。立諸王愛牙赤投下人匠提舉司於益都。並省雲南大理、中慶等路州縣。丁丑,升市令司為從五品。改大都路甲匠總管府為軍器人匠都總管府。尚書省臣言:「乃顏以反誅,其人戶月給米萬七千五百二十三石,父母妻子俱在北方,恐生它志,請徙置江南,充沙不丁所請海船水軍。」從之。



 五月庚辰,發武衛親軍千人浚河西務至通州漕渠。癸未,移諸王小薛饑民就食汴梁,發大同、宣德等路民築倉于昴兀剌。壬辰,太白犯鬼。軟奴王術私以金銀器皿給諸王出伯、合班等,且供饋有勞,命有司如數償之,復賞銀五萬兩、幣帛各二千匹。丙申,詔:「季陽、益都、淄萊三萬戶軍久戍廣東,疫死者眾,其令二年一更。」賊鐘明亮率眾萬八千五百七十三人來降,江淮、福建、江西三省所抽軍各還本翼。行御史臺復徙於揚州,浙西提刑按察司徙蘇州。以參知政事忻都為尚書左丞,中書參知政事何榮祖為參知政事,參議尚書省事張天祐為中書參知政事。己亥,設回回國子學。升利用監為從三品。遼陽路饑,免往歲未輸田租。尚書省臣言:「括大同、平陽、太原無籍民及人奴為良戶,略見成效。益都、濟南諸道,亦宜如之。」詔以農時民不可擾,俟秋冬行之。罷永盈庫,以所貯上供幣帛入太府監及萬億庫。辛丑,御河溢入會通渠,漂東昌民廬舍。以莊浪路去甘肅省遠,改隸安西省。省流江縣入渠州。泰安寺屯田大水,免今歲租。青山貓蠻以不莫臺、卑包等三十三寨相繼內附。



 六月戊申朔,發侍衛軍二千人浚口溫腦兒河渠。己酉,鞏昌汪惟和言:「近括漢人兵器,臣管內已禁絕,自今臣凡用兵器,乞取之安西官庫。」帝曰:「汝家不與它漢人比,弓矢不汝禁也,任汝執之。」辛亥,詔以雲南行省地遠,州縣官多闕,六品以下,許本省選闢以聞。桂陽路寇亂水旱,下其估糶米八千七百二十石以賑之。己未,西番進黑豹。庚申,諸王乃蠻帶敗合丹兵於托吾兒河。丙寅,要忽兒犯邊。辛巳,詔遣尚書省斷事官禿烈羊呵理算雲南,復立雲南提刑按察司。月的迷失請以降賊鐘明亮為循州知州,宋士賢為梅州判官,丘應祥等十八人為縣尹、巡尉,帝不允,令明亮、應祥並赴都。大都增設倒鈔庫三所。遼陽等路饑,免今歲差賦。移八八部曲饑者就食甘州。海都犯邊,和林宣慰使怯伯、同知乃滿帶、副使八黑鐵兒皆反應之。合剌赤饑,出粟四千三百二十八石有奇以賑之。甲戌,西南夷中下爛土等處洞長忽帶等以洞三百、寨百一十來歸,得戶三千餘。乙亥,金剛奴寇折連怯兒。立江淮等處財賦總管府,掌所籍宋謝太后貲產,隸中宮。丁丑,汲縣民硃良進紫芝。濟寧、東平、汴梁、濟南、棣州、順德、平灤、真定霖雨害稼,免田租十萬五千七百四十九石。



 秋七月戊寅朔,海都兵犯邊,帝親征。尚珍署屯田大水,從征者給其家。己卯,駙馬爪忽兒部曲饑,賑之。辛巳,兩淮屯田雨雹害稼,蠲今歲田租。雨壞都城,發兵、民各萬人完之。開安山渠成,河渠官禮部尚書張孔孫、兵部郎中李處選、員外郎馬之貞言:「開魏博之渠,通江淮之運,古所未有。」詔賜名會通河,置提舉司,職河渠事。甲申,四川山齊蠻民四寨五百五十戶內附。丙戌,命百官市馬助邊。敕以禿魯花及侍衛兵百人為桑哥導從。丁亥,發至元鈔萬錠,市馬於燕南、山東、河南、太原、平陽、保定、河間、平灤。戊子,太白經天四十五日。庚寅,黃兀兒月良等驛乏食,以鈔賑之。辛卯,太陰犯牛。詔遣牙牙住僧詣江南搜訪術藝之士。發和林所屯乞兒乞思等軍北征。癸巳,平灤屯田霖雨損稼。甲午,御河溢。東平、濟寧、東昌、益都、真定、廣平、歸德、汴梁、懷孟蝗。乙未,太陰犯歲星。丁酉,命遼陽行省益兵戍咸平、懿州。戊戌,誅信州叛賊鮑惠日等三十三人。左丞李庭等北征。辛丑,發侍衛親軍萬人赴上都。河間大水害稼。壬寅,賦百官家,制戰襖。癸卯,沙河溢。鐵燈捍堤決。



 八月壬子,霸州大水,民乏食,下其估糶直沽倉米五千石。乙卯,郴之宜章縣為廣東寇所掠,免今歲田租。辛酉,大都路霖雨害稼,免今歲租賦,仍減價糶諸路倉糧。壬戌,漷州饑,發河西務米二千石,減其價賑糶之。癸亥,諸王鐵失、孛羅帶所部皆饑,敕上都留守司、遼陽省發粟賑之。甲子,月的迷失以鐘明亮貢物來獻。辛未,歲星晝見。癸酉,以八番羅甸宣慰使司隸四川省。臺、婺二州饑,免今歲田租。甲戌,詔兩淮、兩浙都轉運使司及江西榷茶都轉運司諸人,毋得沮辦課。改四川金竹寨為金竹府。徙浙東道提刑按察司治婺州,河東山西道提刑按察司治太原,宣慰司治大同。



 九月戊寅,歲星犯井。己卯,置高麗國儒學提舉司,從五品。丙戌,罷濟州泗汶漕運使司。丁亥,罷斡端宣慰使元帥府。癸巳,以京師糴貴,禁有司拘顧商車。乙未,太陰犯畢。丙申,熒惑犯太微西垣上將。增浙東道宣慰使一員。江淮省平章沙不丁言:「提調錢穀,積怨於眾,乞如要束木例,撥戍兵三百人為衛。」從之。平灤、昌國等屯田霖雨害稼。甲辰,以保定、新城、定興屯田糧賑其戶饑貧者。乙巳,詔福建省及諸司毋沮擾魏天祐銀課。



 冬十月癸丑,營田提舉司水害稼。太陰犯牛宿距星。甲寅,熒惑犯右執法。以駝運大都米五百石有奇給皇子北安王等部曲。乙卯,以八番、羅甸隸湖廣省。丙辰,禁內外百官受人饋酒食者,沒其家貲之半。甲子,享於太廟。己巳,赤那主里合花山城置站一所。癸酉,尚書省臣言:「沙不丁以便宜增置浙東二鹽司,合浙東、西舊所立者為七,乞官知鹽法者五十六人。」從之。平灤水害稼。以平灤、河間、保定等路饑,弛河泊之禁。



 閏十月戊寅,車駕還大都。尚書省臣言:「南北鹽均以四百斤為引,今權豪家多取至七百斤,莫若先貯鹽於席,來則授之為便。」從之。庚辰,桑哥言:「初改至元鈔,欲盡收中統鈔,故令天下鹽課以中統、至元鈔相半輸官。今中統鈔尚未可急斂,宜令稅賦並輸至元鈔,商販有中統料鈔,聽易至元鈔以行,然後中統鈔可盡。」從之。月的迷失以首賊丘應祥、董賢舉歸於京師。癸未,命遼陽行省給諸王乃蠻帶民戶乏食者。乙酉,命自今所授宣敕並付尚書省。通州河西務饑,民有鬻子、去之他州者,發米賑之。丙戌,西南夷生番心嵒等八族計千二百六十戶內附。廣東賊鐘明亮復反,以眾萬人寇梅州,江羅等以八千人寇漳州,又韶、雄諸賊二十餘處皆舉兵應之,聲勢張甚。詔月的迷失復與福建、江西省合兵討之,且諭旨月的迷失:「鐘明亮既降,朕令汝遣之赴闕,而汝玩常不發,致有是變。自今降賊,其即遣之。」丁亥,安南國王陳日烜遣使來貢方物。左、右衛屯田新附軍以大水傷稼乏食,發米萬四百石賑之。辰星犯房。己丑,太陰犯畢,熒惑犯進賢。庚寅,江西宣慰使胡頤孫援沙不丁例,請至元鈔千錠為行泉府司,歲輸珍異物為息,從之,以胡頤孫遙授行尚書省參政、泉府太卿、行泉府司事。詔籍江南及四川戶口。丙申,寶坻屯田大水害稼。河南宣慰司請給管內河間、真定等路流民六十日糧,遣還其土,從之。婺州賊葉萬五以眾萬人寇武義縣,殺千戶一人,江淮省平章不鄰吉帶將兵討之。遣使鉤考大同錢穀及區別給糧人戶。庚子,取石泗濱為磬,以補宮縣之樂。辛丑,羅斛、女人二國遣使來貢方物。癸卯,禁殺羔羊。浙西宣慰使史弼請討浙東賊,以為浙東道宣慰使,位合剌帶上。甲辰,武平路饑,發常平倉米萬五千石。賑保定等屯田戶饑,給九十日糧。檀州饑民劉德成犯獵禁,詔釋之。湖廣省臣言:「近招降贛州賊胡海等,令將其眾屯田自給,今過耕時,不恤之,恐生變。」命贛州路發米千八百九十石賑之。丙午,緬國遣委馬剌菩提班的等來貢方物。



 十一月丙午朔,回回、昔寶赤百八十六戶居汴梁者,申命宣慰司給其田。丁未,禁江南、北權要之家毋沮鹽法。戊申,敕尚書省發倉賑大都饑民。壬子,漳州賊陳機察等八千人寇龍巖,執千戶張武義,與楓林賊合。福建行省兵大破之,陳機察、丘大老、張順等以其黨降。行省請斬之以警眾,事下樞密院議。範文虎曰:「賊固當斬,然既降乃殺之,何以示信?宜並遣赴闕。」從之。癸丑,建寧賊黃華弟福,結陸廣、馬勝復謀亂,事覺,皆論誅。甲寅,瓜、沙二州城壞,詔發軍民修完之。丙辰,罷阿你哥所領採石提舉司。發米五百八十七石給昔寶赤五百七十八人之乏食者。丁巳,平灤、昌國屯戶饑,賑米千六百五十六石。改播州為播南路。丁卯,詔山東東路毋得沮淘金。賑文安縣饑民。陜西鳳翔屯田大水。戊辰,太陰犯亢。己巳,發米千石賑平灤饑民。改平恩鎮為丘縣。武平路饑,免今歲田租。桓州等驛饑,以鈔給之。



 十二月丁丑,蠡州饑,發義倉糧賑之。戊寅,罷平州望都、榛子二驛,放其戶為民。辛巳,詔括天下馬,一品、二品官許乘五匹,三品三匹,四品、五品二匹,六品以下皆一匹。平灤大水傷稼,免其租。小薛坐與合丹禿魯乾通謀叛,伏誅。紹興路總管府判官白絜矩言:「宋趙氏族人散居江南,百姓敬之不衰,久而非便,宜悉徙京師。」桑哥以聞,請擢絜矩為尚書省舍人,從之。給玉呂魯所招集戶五百人九十日糧。徙甕吉剌民戶貧乏者就食六盤。乙酉,命四川蒙古都萬戶也速帶選所部軍萬人西征。太白犯南斗。丁亥,封皇子闊闊出為寧遠王。河間、保定二路饑,發義倉糧賑之,仍免今歲田租。木鄰站經亂乏食,給九十日糧。命回回司天臺祭熒惑。庚寅,禿木合之地霜殺稼,禿魯花之地饑,給九十日糧。甲午,以官軍萬戶汪惟能為征西都元帥,將所部軍入漠,其先戍漠兵無令還翼。乙未,蠲大名、清豐逋租八百四十石,命甘肅行省賑千戶也先所部人戶之饑者,給鈔賑黃兀兒月良站人戶。庚子,武平饑,以糧二萬三千六百石賑之。伯顏遣使來言邊民乏食,詔賜網罟,使取魚自給。拔都昔剌所部阿速戶饑,出粟七千四百七十石賑之。癸卯,發麥賑廣濟署饑民。是歲,馬八兒國進花驢二,寧州民張世安進嘉禾一本。詔天下梵寺所貯《藏經》,集僧看誦,仍給所費,俾為歲例。幸大聖壽萬安寺,置旃檀佛像,命帝師及西僧作佛事坐靜二十會。免災傷田租:真定三萬五千石,濟寧二千一百五十四石,東平一百四十七石,大名九百二十五石,汴梁萬三千九十七石,冠州二十七石。賜諸王、公主、駙馬如歲例,為金二千兩、銀二十五萬二千六百三十兩、鈔一十一萬二百九十錠、幣十二萬二千八百匹。斷死罪五十九人。



\end{pinyinscope}