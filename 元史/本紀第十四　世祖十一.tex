\article{本紀第十四 世祖十一}

\begin{pinyinscope}

 二十三年春正月戊辰朔,以皇太子故罷朝賀。禁齎金銀銅錢越海互市。甲戌,帝以日本孤遠島夷,重困民力,罷征日本,召阿八赤赴闕,仍散所顧民船。以江南廢寺田土為人占據者,悉付總統楊璉真加修寺。己卯,立羅不、怯臺、闍鄽斡端等驛。呂文煥以江淮行省右丞告老,許之,任其子為宣慰使。庚辰,馬八國遣使進銅盾。壬午,太陰犯軒轅太民。遣使代祀岳瀆東海。癸未,罷鞏昌二十四城拘榷所,以其事入有司。發鈔五千錠糴糧於沙、靜、隆興。從桑哥請,命楊璉真加遣宋宗戚謝儀孫、全允堅、趙沂、趙太一入質。甲申,忽都魯言:「所部屯田新軍二百人,鑿河渠于亦集乃之地,役久功大,乞以傍近民、西僧餘戶助其力。」從之。憨答孫遣使言:「軍士疲乏者八百餘人,乞賑贍,宜於朵魯朵海處驗其虛實。」帝曰:「比遣人往,事已緩矣,其使贍之。」丁亥,焚陰陽偽書《顯明歷》。辛卯,命阿里海牙等議征安南事宜。癸巳,升福州長溪縣為福寧州,以福安、寧德二縣隸之。丙申,以新附軍千人屯田合思罕關東曠地,官給農具牛種。丁酉,畋於近郊。降敘州為縣,隸蠻夷宣撫司。詔禁沮擾鹽課。設諸路推官以審刑獄,上路二員,下路一員。升龍興武寧縣為寧州,以分寧隸之。



 二月己亥,敕中外,凡漢民持鐵尺、手撾及杖之藏刃者,悉輸於官。辛丑,遣使以鈔五千錠賑諸王小薛所部饑民。甲辰,以雪雪的斤為緬中行省左丞相,阿臺董阿參知政事,兀的迷失簽行中書省事。以阿里海牙仍安南行中書省左丞相,奧魯赤平章政事,都元帥烏馬兒、亦裏迷失、阿里、昝順、樊楫並參知政事。遣使諭皇子也先鐵木兒,調合剌章軍千人或二三千,付阿里海牙從征交趾,仍具將士姓名以聞。乙巳,廷議以東北諸王所部雜居其間,宣慰司望輕,罷山北遼東道、開元等路宣慰司,立東京等處行中書省,以闊闊你敦為左丞相,遼東道宣慰使塔出右丞,同簽樞密院事楊仁風、宣慰使亦而撒合並參知政事。敕中書省:「大府監所儲金銀,循先朝例分賜諸王。」復立大司農司,專掌農桑。升宣徽院正二品,降鎮巢府為巢州。丁未,用御史臺臣言,立按察司巡行郡縣法,除使二員留司,副使以下每歲二月分蒞按治,十月還司。丙午,太陰犯井。戊申,樞密院奏:「前遣蒙古軍萬人屯田,所獲除歲費之外可糶鈔三千錠,乞分廩諸翼軍士之貧者。」帝悅,令從便行之。調京師新附軍二千立營屯田。癸丑,復置隰州大寧縣。丁巳,命湖廣行省造征交趾海船三百,期以八月會欽、廉州。戊午,並江南行樞密院四處入行省。命荊湖占城行省將江浙、湖廣、江西三行省兵六萬人伐交趾。荊湖行省平章奧魯赤以征交趾事宜請入覲,詔乘傳赴闕。集賢直學士程文海言:「省院諸司皆以南人參用,惟御史臺按察司無之。江南風俗,南人所諳,宜參用之便。」帝以語玉速鐵木兒,對曰:「當擇賢者以聞。」帝曰:「汝漢人用事者,豈皆賢邪?」江南諸路學田昔皆隸官,詔復給本學,以便教養。封陳益稷為安南王,陳秀嵈為輔義公,仍下詔諭安南吏民。復立岳、鄂、常德、潭州、靜江榷茶提舉司。癸亥,太史院上《授時歷經》、《歷議》,敕藏於翰林國史院。甲子,復以平原郡公趙與芮江南田隸東宮,立甘州行中書省。丙寅,以編地裡書,召曲阜教授陳儼、京兆蕭、蜀人虞應龍,唯應龍赴京師。



 三月己巳,御史臺臣言:「近奉旨按察司參用南人,非臣等所知,宜令侍御史、行御史臺事程文海與行臺官博採公潔知名之士,具以名聞。」帝命齎詔以往。太陰犯婁。浚治中興路河渠。省雲和署入教坊司。辛未,降梅、循為下州。甲戌,雄、霸二州及保定諸縣水泛溢,冒官民田,發軍民築河堤御之。乙亥,以麥術丁仍中書右丞,與郭佑並領錢穀,楊居寬典銓選。立欽察衛親軍都指揮使司。賜諸王脫忽帖木兒羊二萬。丙子,大駕幸上都。詔行御史臺按察司以八月巡行郡縣。中書省臣言:「阿合馬時諸王駙馬往來餉給之費,悉取於萬億庫,後徵百官俸入以償,最非便。」詔在籍者除之勿征。以榷茶提舉李起南為江西榷茶轉運使。起南嘗言:「江南茶每引價三貫六百文,今宜增每引五貫。」事下中書議,因令起南為運使,置達魯花赤處其上。丁丑,徙東京行中書省於咸平府。癸巳,歲星犯壘壁陣。以臨江路為北安王分邑。



 夏四月庚子,中書省臣請立汴梁行中書省及燕南、河東、山東宣慰司。有旨:「南京戶寡盜息,不必置省,其宣慰司如所請。濟南乃勝納合兒分地,太原乃阿只吉分地,其令各衛委官一人同治之。」敕免雲南從征交趾蒙古軍屯田租。立烏蒙站。江南諸路財賦並隸中書省。雲南省平章納速剌丁上便宜數事:一曰弛道路之禁,通民來往;二曰禁負販之徒,毋令從征;三曰罷丹當站賦民金為飲食之費;四曰聽民伐木貿易;五曰戒使臣勿擾民居,立急遞鋪以省驛騎。詔議行之。辛丑,陜西行省言:「延安置屯田鷹坊總管府,其火失不花軍逃散者,皆入屯田,今復供秦王阿難答所部阿黑答思飼馬及輸他賦。」有旨皆罷之,其不悛者罪當死。甲辰,行御史臺自杭州徙建康。以山南、淮東、淮西三道按察司隸內臺。增置行臺色目御史員數。丁未,江東宣慰司進芝一本。庚戌,制謚法。壬子,樞密院納速剌丁言:「前所統漸丁軍五千人往征打馬國,其力已疲,今諸王復籍此軍征緬,宜取進止。」帝曰:「茍事力未損,即遣之。」仍諭納速剌丁分阿剌章、蒙古軍千人,以能臣將之,赴交趾助皇子脫歡。己未,遣要束木勾考荊湖行省錢穀。中書擬要束木平章政事,脫脫忽參知政事。有旨:「要束木小人,事朕方五年,授一理算官足矣。脫脫忽人奴之奴,令史、宣使才也。讀卿等所進擬,令人恥之。其以朕意諭安童。」以漢民就食江南者多,又從官南方者秩滿多不還,遣使盡徙北還。仍設脫脫禾孫於黃河、江、淮諸津渡,凡漢民非齎公文適南者止之,為商者聽。中書省臣言:「比奉旨,凡為盜者毋釋。今竊鈔數貫及佩刀微物,與童幼竊物者,悉令配役。臣等議,一犯者杖釋,再犯依法配役為宜。」帝曰:「朕以漢人徇私,用《泰和律》處事,致盜賊滋眾,故有是言。人命至重,今後非詳讞者,勿輒殺人。」



 五月丁卯朔,樞密院臣言:「臣等與玉速帖木兒議別十八里軍事,凡軍行並聽伯顏節制,其留務委孛欒帶及諸王阿只吉官屬統之為宜。」從之。己巳,熒惑犯太微西垣上將。荊湖行省阿里海牙上言:「要束木在鄂省勾考,豈無貪賄?臣亦請勾考之。」詔遣參知政事禿魯罕、樞密院判李道、治書侍御史陳天祥偕行。甲戌,汴梁旱。徙江東按察司於宣州。庚辰,歲星犯壘壁陣。乙酉,熒惑犯太微右執法。敕遣耽羅戍兵四百人還家。庚寅,廣平等路蠶災。辛卯,霸州、漷州蝻生。安南國遣使來貢方物。癸巳,京畿旱。



 六月丙申朔,太白犯御女。辛丑,中書省臣言:「禿魯罕來奏,前要束木、阿里海牙互請鉤考,今阿里海牙雖已死,事之是非,當令暴白。」帝曰:「卿言良是,其連引諸人,近者即彼追逮,遠者宜以上聞。此事自要束木所發,當依其言究行之。」乙巳,以立大司農司詔諭中外。皇孫鐵木兒不花駐營亦奚不薛,其糧餉仰於西川,遠且不便,徙駐重慶府。詔以大司農司所定《農桑輯要》書頒諸路。命雲南、陜西二行省籍定建都稅賦。戊申,括諸路馬。凡色目人有馬者三取其二,漢民悉入官,敢匿與互市者罪之。辛亥,以亦馬剌丹忒忽裏使交趾。癸丑,湖廣行省線哥言:「今用兵交趾,分本省戍兵二萬八千七百人,期以七月悉會靜江,今已發精銳啟行,餘萬七千八百人,皆羸病、屯田等軍,不可用。」敕今歲姑罷之。丁巳,設陜西等路諸站總管府,從三品。庚申,甘肅新招貧民百一十八戶,敕廩給之。敕路、府、州、縣捕盜者持弓矢,各路十副,府、州七副,縣五副。以薛闍乾為中書省平章政事。辛酉,封楊邦憲妻田氏為永安郡夫人,領播州安撫司事。遣鎮西平緬等路招討使怯烈招諭緬國。廣元路閬中麥秀兩歧。高麗國遣使來貢。



 秋七月丙寅朔,遣必剌蠻等使瓜哇。己巳,用中書省臣言,以江南隸官之田多為強豪所據,立營田總管府,其所據田仍履畝計之。復尚醖監為光祿寺,罷遼陽等處行中書省,復北京、咸平等三道宣慰司。給鐵古思合敦貧民幣帛各二千、布千匹。庚午,江淮行省忙兀帶言:「今置省杭州,兩淮、江東諸路財賦軍實,皆南輸又復北上,不便。揚州地控江海,宜置省,宿重兵鎮之,且轉輸無往返之勞,行省徙揚州便。」從之。立淮南洪澤、芍陂兩處屯田。壬申,平陽饑民就食鄰郡者,所在發倉賑之。置中尚監。右丞拜答兒將兵討阿蒙,並其妻子禽之,皆伏誅。丁丑,斡脫吉思部民饑,遣就食北京,其不行者發米賑之。以雄、易二州復隸保定。給和林軍儲,自京師輸米萬石,發鈔即其地糴米萬石。辛巳,八都兒饑民六百戶駐八剌忽思之地,給米千石賑之。壬午,總制院使桑哥具省臣姓名以上,帝曰:「右丞相安童,右丞麥術丁,參知政事郭佑、楊居寬,並仍前職。以鐵木兒為左丞,其左丞相甕吉剌帶、平章政事阿必失合、都忽魯皆別議。」仍諭中書選可代者以聞。給金齒國使臣圓符。癸巳,銓定省、院、臺、部官,詔諭中外:「中書省,除中書令外,左、右丞相並一員,平章政事二員,左、右丞並一員,參知政事二員;行中書省,平章政事二員,左、右丞並一員,參知政事、簽行省事並二員;樞密院,除樞密院使外,同知樞密院事一員,樞密院副使、簽樞密院事並二員,樞密院判一員;御史臺,御史大夫一員,中丞、侍御史、治書侍御史並二員;行臺同;六部,尚書、侍郎、郎中、員外郎並二員。其餘諸衙門,並委中書省斟酌裁減。」



 八月丙申,發鈔二萬九千錠、鹽五萬引,市米賑諸王阿只吉所部饑民。己亥,敕樞密院遣侍衛軍千人扈從北征。平陽路歲比不登,免貧民稅賦。罷淮東、蘄黃宣慰司,以黃、蘄、壽昌隸湖廣行省,安慶、六安、光州隸淮西宣慰司。招集宋鹽軍。以市舶司隸泉府司。乙卯,太白犯軒轅右角。辛酉,婺州永康縣民陳巽四等謀反,伏誅。甘州饑,禁酒。罷德平、定昌二路,置德昌軍民總管府。



 九月乙丑朔,馬八兒、須門那、僧急里、南無力、馬蘭丹、那旺、丁呵兒、來來、急闌亦帶、蘇木都剌十國,各遣子弟上表來覲,仍貢方物。以太廟雨壞,遣甕吉剌帶致告,奉安神主別殿。甲申,太陰犯天關。行辰,高麗遣使獻日本俘。是月,南部縣生嘉禾,一莖九穗。芝產於蒼溪縣。



 冬十月甲午朔,太白犯右執法。以南康路隸江西行省,徙浙西按察司治杭州。罷諸道提刑按察司判官,行御史臺監察御史及按察司官,雖漢人並毋禁弓矢。襄邑縣尹張為治有績,鄒平縣達魯花赤回回能捕盜理財,進秩有差。丁酉,享於太廟。戊戌,太陰犯建星。己亥,車駕至自上都。壬寅,太白犯左執法。遣兵千人戍畏吾境。乙巳,賜合迷裏貧民及合剌和州民牛種,給鈔萬六千二百錠當其價,合迷里民加賜幣帛並千匹。己酉,遣塔塔兒帶、楊兀魯帶以兵萬人、船千艘征骨嵬。中書省具宣徽、大司農、大都、上都留守司存減員數以聞,帝曰:「在禁近者朕自沙汰,餘從卿等議之。」辛亥,太陰犯東井。河決開封、祥符、陳留、杞、太康、通許、鄢陵、扶溝、洧川、尉氏、陽武、延津、中牟、原武、眭州十五處,調南京民夫二十萬四千三百二十三人,分築堤防。癸丑,諭江南各省所統軍官教練水軍。遣侍衛新附兵千人屯田別十八里,置元帥府即其地總之。甲寅,太白犯進賢。以征緬功,調招討使張萬為征緬副都元帥,也先鐵木兒征緬招討司達魯花赤,千戶張成徵緬招討使,並虎符,敕造戰船,將兵六千人以征緬,俾禿滿帶為都元帥總之。乙卯,給皇子脫歡馬四千匹,部曲人三匹。庚申,濟寧路進芝二莖。壬戌,改河間鹽運司為都轉運使司。徙戍甘州新附軍千人屯田中興,千人屯田亦里黑。高麗遣使來獻日本俘十六人。馬法國進鞍勒、氈甲。興化路仙游縣蟲傷禾。



 十一月乙丑,中書省臣言:「硃清等海道運糧,以四歲計之,總百一萬石,斗斛耗折願如數以償,風浪覆舟請免其征。」從之。遂以昭勇大將軍、沿海招討使張瑄,明威將軍、管軍萬戶兼管海道運糧船硃清,並為海道運糧萬戶,仍佩虎符。敕禽獸字孕時無畋獵。戊辰,太白犯亢。遣蒙古千戶曲出等總新附軍四百人,屯田別十八里。己巳,改思明等四州並為路。以阿八赤為征交趾行省右丞。丙子,以涿、易二州良鄉、寶坻縣饑,免今年租,給糧三月。平灤、太原、汴梁水旱為災,免民租二萬五千六百石有奇。改廣東轉運市舶提舉司為鹽課市舶提舉司。丁丑,命塔義兒、忽難使阿兒渾。戊寅,遣使閱實宣寧縣饑民,周給之。己卯,太陰犯井。辛巳,歲星犯壘壁陣。



 十二月乙未,遼東開元饑,賑糧三月。戊戌,太白犯東咸。癸卯,要束木籍阿里海牙家貲,運致京師。賜諸王術伯所部軍五千人銀萬五千兩、鈔三千錠,探馬赤二千人羊七萬口。丙午,置燕南、河東、山東三道宣慰司。罷大有署。丁未,太陰犯井。乙卯,諸道宣慰司,在內地者設官四員,江南者六員。以阿里海牙所芘逃民無主者千人屯田。遣中書省斷事官禿不申復鉤考湖廣行省錢穀。復置泉州市舶提舉司。大都饑,發官米低其價糶貧民。丙辰,遣蒲昌赤貧民墾甘肅閑田,官給牛、種、農具。賜安南國王陳益稷羊馬鈔百錠。丁巳,太陰犯氐。戊午,翰林承旨撒里蠻言:「國史院纂修太祖累朝實錄,請以畏吾字繙繹,俟奏讀然後纂定。」從之。諸路分置六道勸農司。庚申,置尚珍署於濟寧等路,秩從五品。是歲,以亦攝思憐真為帝師。賜皇子奧魯赤、脫歡、諸王術伯、也不乾等羊馬鈔一十五萬一千九百二十三錠,馬七千二百九十匹,羊三萬六千二百六十九口,幣帛、毳段、木綿三千二百八十八匹,貂裘十四。又賜皇子脫歡所部憐牙思不花等及欠州諸局工匠鈔五萬六千一百三十九錠一十二兩。命西僧遞作佛事於萬壽山、玉塔殿、萬安寺,凡三十會。大司農司上諸路學校凡二萬一百六十六所,儲義糧九萬五百三十五石,植桑棗雜果諸樹二千三百九萬四千六百七十二株。斷死刑百一十四人。



 二十四年春正月乙丑,復云南石梁縣。戊辰,以修築柳林河堤南軍三千,浚河西務漕渠。皇子奧魯赤部曲饑,命大同路給六十日糧。免唐兀衛河西地元籍徭賦。壬申,御正殿受諸王百官朝賀。癸酉,俱藍國遣使不六溫乃等來朝。甲戌,太陰犯東井。乙酉,太陰犯房。丙戌,以參政程鵬飛為中書右丞,阿里為中書左丞。丁亥,以不顏里海牙為參知政事。發新附軍千人從阿八赤討安南。弛女直、水達達地弓矢之禁。復改江浙省為江淮行省。戊子,以鈔萬錠賑斡端貧民。西邊歲饑民困,賜絹萬匹。庚寅,遣使代祀岳瀆、后土、東海。辛卯,以淮東、淮西、山南三道按察司隸行御史臺。立上林署,秩從七品。詔發江淮、江西、湖廣三省蒙古、漢券軍及雲南兵,及海外四州黎兵,命海道運糧萬戶張文虎等運糧十七萬石,分道以討交趾。置徵交趾行尚書省,奧魯赤平章政事,烏馬兒、樊楫參知政事,總之,並受鎮南王節制。



 二月壬辰朔,遣使持香幣詣龍虎、閤皁、三茅設醮,召天師張宗演赴闕。癸巳,雍古部民饑,發米四千石賑之,不足,復給六千石米價。甲午,畋於近郊。乙未,以麥術丁為平章政事。真定路饑,發沿河倉粟減價糶之。以真定所牧官馬四萬餘匹分牧他郡。禁畏吾地禽獸孕孳時畋獵。庚子,太陰犯天關。辛丑,太陰犯東井。甲辰,升江淮行大司農司事秩二品,設勸農營田司六,秩四品,使副各二員,隸行大司農司。以範文虎為中書右丞,商議樞密院事。壬子,封駙馬昌吉為寧濮郡王。設都總管府以總皇子北安王民匠、斡端大小財賦。中書省臣言:「自正旦至二月中旬費鈔五十萬錠,臣等兼總財賦,自今侍臣奏請賜賚,乞令臣等預議。」帝曰:「此朕所當慮。」仍諭玉速鐵木兒、月赤徹兒知之。丙辰,馬八兒國貢方物。戊午,敕諸王闍里鐵木兒節制諸軍。以趙與芮子孟桂襲平原郡公。乃顏遣使征東道兵,闍諭里鐵木兒毋輒發。



 閏月癸亥,太陰犯辰星。以女直、水達達部連歲饑荒,移粟賑之,仍盡免今年公賦及減所輸皮布之半。以宋畬軍將校授管民官,散之郡邑。敕春秋二仲月上丙日祀堯帝祠。西京等處管課官馬合謀自言歲以西京、平陽、太原課程額外羨錢市馬駝千輸官,而實盜官錢市之,按問有跡,伏誅。乙丑,畋於近郊。召麥術丁、鐵木兒、楊居寬等與集賢大學士阿魯渾撒里及葉李、程文海、趙孟頫論鈔法。麥術丁言:「自制國用使司改尚書省,頗有成效,今仍分兩省為便。」詔從之,各設官六員。其尚書,以桑哥、鐵木兒平章政事,阿魯渾撒里右丞,葉李左丞,馬紹參知政事,餘一員議選回回人充;中書,宜設丞相二員、平章政事二員、參知政事二員。省隴右河西道提刑按察司,分置鞏昌者入甘州,設官五員;以鞏昌改隸京兆提刑按察司,設官六員;省太原提刑按察司,分置西京者入太原。辛未,以復置尚書省詔天下。除行省與中書議行,餘並聽尚書省從便以聞。設國子監,立國學監官:祭酒一員,司業二員,監丞一員,學官博士二員,助教四員,生員百二十人,蒙古、漢人各半,官給紙札、飲食,仍隸集賢院。設江南各道儒學提舉司。甲申,太陰犯牽牛。車駕還宮。乙酉,改淄萊路為般陽路,置錄事司。大都饑,免今歲銀俸鈔,諸路半征之。罷江南竹木柴薪及岸例魚牙諸課,停不給之務。敕行省宣慰司勿濫舉官吏,受除官延引歲月不即之任者,追所受宣敕。鎮南王脫歡徙鎮南京。改福建市舶都漕運司為都轉運鹽使司。範文虎改尚書右丞,商議樞密院事。改行中書省為行尚書省,六部為尚書六部,以吏部尚書忻都為尚書省參知政事。庚寅,大駕幸上都。札魯忽赤合剌合孫等言:「去歲審囚官所錄囚數,南京、濟南兩路應死者已一百九十人,若總校諸路,為數必多,宜留札魯忽赤數人分道行刑。」帝曰:「囚非群羊,豈可遽殺耶?宜悉配隸淘金。」



 三月甲午,更造至元寶鈔頒行天下,中統鈔通行如故。以至元寶鈔一貫文當中統交鈔五貫文,子母相權,要在新者無冗,舊者無廢。凡歲賜、周乏、餉軍,皆以中統鈔為準。禁無籍自效軍擾民,仍籍充軍。丙申,太陰犯東井。乙卯,幸涼陘。遼東饑,弛太子河捕魚禁。丙辰,馬八兒國遣使進奇獸一,類騾而巨,毛黑白間錯,名阿塔必即。降重慶路定遠州為縣。命都水監開汶、泗水以達京師。汴梁河水泛溢,役夫七千修完故堤。



 夏四月癸酉,太陰犯氐。甲戌,太陰犯房。甲申,忻都奏發新鈔十一萬六百錠、銀千五百九十三錠、金百兩,付江南各省與民互市。是月,諸王乃顏反。



 五月己亥,遣也先傳旨諭北京等處宣慰司,凡隸乃顏所部者禁其往來,毋令乘馬持弓矢。庚子,以不魯合罕總探馬赤軍三千人出征。移濟南宣慰司治益都,燕南按察司治大名,南京按察司治南陽,太原按察司治西京,復立豐州亦剌真站。壬寅,以御史臺吏王良弼等誹訕尚書省政事,誅良弼,籍其家,餘皆斷罪。用桑哥言,置上海、福州兩萬戶府,以維制沙不丁、烏馬兒等海運船。戶、工兩部各增尚書二員。授高麗王睶行尚書省平章政事。罷諸路站脫脫禾孫。括江南諸路匠戶。沙不丁言:「江南各省南官多,每省宜用一二人。」帝曰:「除陳巖、呂師夔、管如德、範文虎四人,餘從卿議。」帝自將征乃顏,發上都。括江南僧道馬匹。詔範文虎將衛軍五百鎮平灤,以欽察為親軍都指揮使,也速帶兒、右衛僉事王通副之。甲辰,免北京今歲絲銀,仍以軍旅經行,給鈔三千錠賑之。壬子,高麗王睶請益兵徵乃顏,以五百人赴之。



 六月庚申朔,百官以職守不得從征乃顏,願獻馬以給衛士。壬戌,至撒兒都魯之地。乃顏黨塔不帶率所部六萬逼行在而陣,遣前軍敗之。乙丑,敕遼陽省督運軍儲。壬申,發諸衛軍萬人、蒙古軍千人戍豪、懿州。諸王失都兒所部鐵哥率其黨取咸平府,渡遼,欲劫取豪、懿州,守臣以乏軍求援,敕以北京戍軍千人赴之。括平灤路馬。北京饑,免絲銀、租稅。乙亥,霸州益津縣霖雨傷稼。以陜西涇、邠、乾及安西屬縣閑田立屯田總管府,置官屬,秩三品。車駕駐干大利斡魯脫之地,獲乃顏輜重千餘,仍禁秋毫無犯。



 秋七月癸巳,乃顏黨失都兒犯咸平,宣慰塔出從皇子愛牙赤,合兵出沈州進討,宣慰亦兒撒合分兵趣懿州,其黨悉平。丁酉,弘州匠官以犬兔毛制如西錦者以獻,授匠官知弘州。戊戌,太陰犯南斗。樞密院奏:「簽征緬行省事合撒兒海牙言,比至緬國,諭其王赴闕,彼言鄰番數叛,未易即行,擬遣阿難答剌奉表齎土貢入覲。」辛丑,太陰犯牽牛。壬寅,熒惑犯輿鬼。庚戌,雲南行省愛魯言,金齒酋打奔等兄弟求內附,且乞入覲。壬子,太陰犯司怪。癸丑,日暈連環,白虹貫之。罷乃顏所署益都、平灤,也不乾河間分地達魯花赤,及勝納合兒濟南分地所署官。移北京道按察司置豪州,免東京等處軍民徭賦。升福建鹽運使司,依兩淮等例,為都轉運使司。以中興府隸甘州行省,以河西愛牙赤所部屯田軍同沙州居民修城河西瓜、沙等處。立闍鄽屯田。



 八月癸亥,太白犯亢。浚州進瑞麥,一莖九穗。乙丑,車駕還上都。以李海剌孫為征緬行省參政,將新附軍五千、探馬赤軍一千以行,仍調四川、湖廣行省軍五千赴之。召能通白夷、金齒道路者張成及前占城軍總管劉全,並為招討使,佩虎符,從征。以脫滿答兒為都元帥,將四川省兵五千赴緬省,仍令其省駐緬近地,以俟進止。置江南四省交鈔提舉司。己巳,謫從叛諸王赴江南諸省從軍自效。諭鎮南王脫歡,禁戢從征諸王及省官奧魯赤等,毋縱軍士焚掠,毋以交趾小國而易之。癸酉,朵兒朵海獲叛王阿赤思,赦之。亦集乃路屯田總管忽都魯請疏浚管內河渠,從之。丙子,填星南犯壘壁陣。己卯,太陰犯天關。辛巳,太陰犯東井。甲申,太白犯房。丁亥,沈州饑,又經乃顏叛兵蹂踐,免其今歲絲銀、租賦。以北京伐木三千戶屯田平灤。立豐贍、昌國、濟民三署,秩五品,設達魯花赤、令、丞、直長各一員。女人國貢海人。置河西務馬站。



 九月辛卯,東京義靜、麟、威遠、婆娑等處大霖雨,江水溢,沒民田;大定、金源、高州、武平、興中等處霜雹傷稼。丁酉,熒惑犯長垣。己亥,湖廣省臣言:「海南瓊州路安撫使陳仲達、南寧軍總管謝有奎、延欄總管符庇成,以其私船百二十艘、黎兵千七百餘人,助征交趾。」詔以仲達仍為安撫使,佩虎符,有奎、庇成亦仍為沿海管軍總管,佩金符。庚子,太白犯天江。給諸王八八所部窮乏者鈔萬一千錠。禁市毒藥者。以西京、平灤路饑,禁酒。乙巳,太陰犯畢。以米二萬石、羊萬口給阿沙所統唐兀軍。丁未,安南國遣其中大夫阮文彥、通侍大夫黎仲謙貢方物。戊申,咸平、懿州、北京以乃顏叛,民廢耕作,又霜雹為災,告饑,詔以海運糧五萬石賑之。辛亥,熒惑犯太微西垣上將。壬子,太白犯南斗。禁沮撓江南茶課。高麗王王睶來朝。



 冬十月戊午朔,日有食之。壬戌,太陰犯牽牛大星。甲子,享於太廟。桑哥請賜葉李、馬紹、不忽木、高翥等鈔,詔賜李鈔百五十錠,不忽木、紹、翥各百錠。又言:「中書省舊在大內前,阿合馬移置於北,請仍舊為宜。」從之。癸酉,江西行院月的迷失言:「廣東窮邊險遠,江西、福建諸寇出沒之窟,乞於江南諸省分軍一萬益臣。」詔江西忽都帖木兒以軍五千付之。丙子,誅郭佑、楊居寬。戊寅,桑哥言:「北安王相府無印,而安西王相獨有印,實非事例,乞收之。諸王勝納合兒印文曰『皇侄貴宗之寶』,寶非人臣所宜用,因其分地改為『濟南王印』為宜。」皆從之。從總帥汪惟和言,分所部戍四川軍五千人屯田六盤。乙酉,熒惑犯左執法。立陜西寶鈔提舉司。羅北甸土官火者、阿禾及維摩合剌孫之子並內附。丙戌,範文虎言:「豪、懿、東京等處,人心未安,宜立省以撫綏之。」詔立遼陽等處行尚書省,以薛闍干、闍裏帖木兒並行尚書省平章政事,洪茶丘右丞,亦兒撒合左丞,楊仁風、阿老瓦丁並參知政事。



 十一月壬辰,太白犯壘壁陣,月暈金、土二星。雲南省右丞愛魯兵次交趾木兀門,其將昭文王以四萬人守之,愛魯擊破之,獲其將黎石、何英。弛太原、保德河魚禁。以桑哥為金紫光祿大夫、尚書右丞相,兼統制院使,領功德使司事。從桑哥請,以平章帖木兒代其位,右丞阿剌渾撒裡升平章政事,葉李升右丞,參知政事馬紹升左丞。升集賢院秩正二品。丙申,熒惑犯太微東垣上相。丁酉,桑哥言:「先是皇子忙哥剌封安西王,統河西、土番、四川諸處,置王相府,後封秦王,綰二金印。今嗣王安難答仍襲安西王印,弟按攤不花別用秦王印,其下復以王傅印行,一籓而二王,恐於制非宜。」詔以阿難答嗣為安西王,仍置王傅,而上秦王印,按攤不花所署王傅罷之。戊戌,以別十八里漢軍及新附軍五百人屯田合迷玉速曲之地。己亥,鎮南王次思明,程鵬飛與奧魯赤等從鎮南王分道並進,阿八赤以萬人為前鋒。庚子,太白晝見。大都路水,賜今年田租十二萬九千一百八十石。辛丑,烏馬兒、樊楫及程鵬飛等遂趣交趾,所向克捷。改衛尉院為太僕寺,秩三品,仍隸宣徽,以月赤徹兒、禿禿合領之。丙午,鎮南王次界河,交趾發兵拒守,前鋒皆擊破之。己酉,詔議弭盜。桑哥、玉速帖木兒言:「江南歸附十年,盜賊迄今未靖者,宜降旨立限招捕,而以安集責州縣之吏,其不能者黜之。」葉李言:「臣在漳州十年,詳知其事,大抵軍官嗜利與賊通者,尤難弭息。宜令各處鎮守軍官,例以三年轉徙,庶革斯弊。」帝皆從其議,詔行之。封駙馬帖木兒濟寧郡王。壬子,以江西行省平章忽都帖木兒督捕廣東等處盜賊。甲寅,命京畿、濟寧兩漕運司分掌漕事。鎮南王次萬劫,諸軍畢會。獲福建首賊張治囝,其黨皆平。諭江南四省招捕盜賊。丙辰,熒惑犯進賢。



 十二月癸亥,立尚乘寺。順元宣慰使禿魯古言,金竹寨主搔驢等以所部百二十五寨內附。甲子,皇子北安王置王傅,凡軍需及本位諸事並以王傅領之。丙寅,太陰犯畢,太白晝見。丁卯,減揚州省歲額米十五萬石,以鹽引五十萬易糧。免浙西魚課三千錠,聽民自漁。發河西、甘肅等處富民千人往闍鄽地,與漢軍、新附軍雜居耕植。從安西王阿難答請,設本位諸匠都總管府。升萬億庫官秩四品。癸酉,鎮南王次茅羅港,攻浮山寨,破之。諸王薛徹都等所駐之地,雨土七晝夜,羊畜死不可勝計,以鈔暨幣帛綿布雜給之,其直計鈔萬四百六十七錠。丁丑,以硃清、張瑄海漕有勞,遙授宣慰使。乙酉,鎮南王以諸軍渡富良江,次交趾城下,敗其守兵,日烜與其子棄城走敢喃堡。是歲,命西僧監臧宛卜卜思哥等作佛事坐靜於大殿、寢殿、萬壽山、五臺山等寺,凡三十三會。斷天下死刑百二十一人。浙西諸路水,免今年田租十之二;西京、北京、隆興、平灤、南陽、懷孟等路風雹害稼;保定、太原、河間、般陽、順德、南京、真定、河南等路霖雨害稼,太原尤甚,屋壞壓死者眾;平陽春旱,二麥枯死,秋種不入土;鞏昌雨雹,虸方為災。分賜皇子、諸王、駙馬、怯薛帶等羊馬鈔,總二十五萬三千五百餘錠,又賜諸王、怯薛帶等軍人,馬一萬二千二百、羊二萬二千六百、駝百餘。賑貧乏者合剌忽答等鈔四萬八千二百五十錠。



\end{pinyinscope}