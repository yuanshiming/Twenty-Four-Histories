\article{列傳第四}

\begin{pinyinscope}

 ○別里古臺



 宗王別里古臺者,烈祖之第五子,太祖之季弟也。天性純厚,明敏多智略,不喜華飾,軀幹魁偉,勇力絕人。幼從太祖平諸部落,掌從馬。國法:常以腹心遇敗則牽從馬。其子孫最多,居處近太祖行在所,南接按只臺營地。嘗從太祖宴諸部族,或潛圖害別里古臺,以刀斫其臂,傷甚。帝大怒,欲索而誅之。別里古臺曰:「今將舉大事於天下,其可以臣故而生釁隙哉!且臣雖傷甚,幸不至死,請勿治。」帝尤賢之。當創業之初,徵取諸國,王未嘗不在軍中,摧鋒陷陣,不避艱險。帝嘗曰:「有別里古臺之力,哈撒兒之射,此朕之所以取天下也。」其見稱如此。嘗立為國相,又長扎魯火赤,別授之印。賜以蒙古百姓三千戶,及廣寧路、恩州二城戶一萬一千六百三,以為分地;又以斡難、怯魯連之地建營以居。江南平,加賜信州路及鉛山州二城戶一萬八千。王薨。子曰罕禿忽,曰也速不花,曰口溫不花。



 罕禿忽,性剛猛,知兵。從憲宗征伐,多立戰功,及攻釣魚山而還,道由河南,招來流亡百餘戶,悉以入籍。罕禿忽子曰霍歷極,以疾廢,不能軍,世祖俾居於恩,以統其籓人。至大三年,霍歷極薨,子塔出嗣。塔出性溫厚,謙恭好學,通經史,能撫恤其民云。



 也速不花子曰爪都,中統三年,始以推戴功,封廣寧王。至元十三年,賜銀印。



 口溫不花,領兵河南,屢建大功,子曰滅里吉臺、甕吉剌臺。



 ○術赤



 術赤者,太祖長子也。國初,以親王分封西北。其地極遠,去京師數萬里,驛騎急行二百餘日,方達京師,以故其地郡邑風俗皆莫得而詳焉。術赤薨,子拔都嗣。拔都薨,弟撒裡答嗣。撒裡答薨,弟忙哥帖木兒嗣。忙哥帖木兒薨,弟脫脫忙哥嗣。脫脫忙哥薨,弟脫脫嗣。脫脫薨,弟伯忽嗣。伯忽薨,弟月即別嗣。至元二年,月即別遣使來求分地歲賜,以賑給軍站,京師元無所領府治。三年,中書請置總管府,給正三品印。至大元年,月即別薨,子札尼別嗣。其位下舊賜平陽、晉州、永州分地,歲賦中統鈔二千四百錠,自至元五年己卯歲始給之。



 ○禿剌



 禿剌,太祖次子察合臺四世孫也。少以勇力聞。大德十一年春,成宗崩,左丞相阿忽臺等潛謀立安西王阿難答,而推皇后伯岳吾氏稱制,中外洶洶。仁宗歸自懷孟,引禿剌入內,縛阿忽臺等以出,誅之,大事遂定。武宗即位,第功,封越王,錫金印,以紹興路為其分地。禿剌居常怏怏,有怨望意。至大元年秋,武宗幸涼亭,將御舟,禿剌前止之。帝曰:「爾何如?朕欲登舟。」禿剌曰:「人有常言:一箭中麋,毋曰自能;百兔未得,未可遽止。」此蓋國俗儕輩相靳之語,而禿剌言之,武宗由是銜焉。既而大宴萬歲山,禿剌醉起,解其腰帶擲諸地,嗔目謂帝曰:「爾與我者,止此爾!」帝益疑其有異志。二年春,命楚王牙忽都、丞相脫脫、平章赤因鐵木兒鞫之,辭服,遂伏誅。



 子西安王阿剌忒納失里,天歷初以推戴功,進封豫王。



 ○牙忽都



 牙忽都,祖父撥綽,睿宗庶子也。撥綽之母曰馬一實,乃馬真氏。撥綽驍勇善騎射,憲宗命將大軍,北征欽察有功,賜號拔都。歲丁巳,分土諸侯王,賜蠡州三千三百四十七戶,為其食邑。撥綽娶察渾滅兒乞氏,生薛必烈傑兒。薛必烈傑兒娶弘吉剌氏,生牙忽都。



 牙忽都年十三,世祖命襲其祖父統軍。至元十二年,從北安王北征。十三年,失列吉叛,遣人誘脅之,牙忽都不從,事王益忠謹。八魯渾拔都兒、粘闓與海都通,相率引去,王遣牙忽都將兵追之,擒八魯渾等以獻。未幾,失列吉、約木忽兒、脫帖木兒等反,以兵攻王。脫帖木兒生致牙忽都,使失列吉拘系之。牙忽都與王親臣那臺等謀逃歸,事覺,那臺等被殺,復系牙忽都,困辱備至。十四年,兀魯兀臺、伯顏帥師討叛,失列吉、約木忽兒迎戰,牙忽都潛結赤斤帖木兒、禿禿哈亂其陣。失列吉軍亂,因得脫走。見帝,須發盡白。帝閔之,賞賚甚厚。至元十八年,加封耒陽州五千三百四十七戶。



 二十一年,命與禿禿哈同討海都,牙忽都先進,邏得諜人,知其虛實,直前沖敵陣,破其精兵,海都敗走,得所俘掠軍民而還。朵兒朵哈上其功,詔賜鈔幣、鎧甲、弓矢。其後北安王駐帖木兒河。乃顏、也不堅有異圖,也不堅引兵趨怯綠憐河大帳。王遣闊闊出、禿禿哈率眾追之。那懷之民擾攘不知所從。牙忽都將三百騎,進至阿赤怯地。會王帳下遜篤思部兵逃去,牙忽都諭之使還。時怯必禿忽兒霍臺誘蒙古軍二萬從乃顏,牙忽都知之,夜襲其河上軍,突入帳中,遇忽都滅兒堅,幾獲之,間道逸去。二十七年,海都入寇。時朵兒哈方居守大帳,詔遣牙忽都同力備御。軍未戰而潰,牙忽都妻帑輜重駐不思哈剌嶺上,悉為藥木忽兒、明理帖木兒所掠。牙忽都與其子脫列帖木兒相失,獨與十三騎奔還。世祖撫慰嘉嘆,賜爵鎮遠王,塗金銀印,以弘吉剌氏女賜之,資裝特厚。復命納里忽、徹徹不花往錫命其部屬同時被剽掠者,以故相桑哥家財分賜之,仍各賜白金五十兩、珠子一酒扈,鈔幣稱是。又命牙忽都居北安王第二帳。王薨,帝命掌大帳,固辭。成宗立,命牙忽都常侍左右。武宗撫兵漠北,請以子脫列帖木兒從。大德五年,海都、篤哇合軍入寇,脫列帖木兒將兵千人擁護,先後力戰,功多,在軍十年。



 成宗崩,安西王阿難答與明理帖木兒窺望神器。牙忽都曰:「世祖皇帝之嫡孫在,神器所當屬。安西,籓王也,入繼非制。」武宗即位,以其父子勞效忠勤,益厚遇之,進封楚王,賜金印,置王傅,以駙馬都尉都剌哈之女弟弘吉烈氏為楚王妃,又以叛王察八兒親屬賜之。脫列帖木兒襲封鎮遠王。



 至大三年,察八兒來歸,宗親皆會。牙忽都進曰:「太祖皇帝削平四方,惟南土未定,列聖嗣位,未遑統一。世祖皇帝混一四海,顧惟宗室諸王,弗克同堂而燕。今陛下洪福齊天,拔都罕之裔,首已附順,叛王察八兒舉族來歸,人民境土,悉為一家。地大物眾,有可恃者焉,有不可恃者焉。昔我太祖有訓,世祖誦之,臣與有聞,治亂國者,宜以法齊之,所以辨上下,定民志。今請有以氏整飭之,則人將有所勸懲,惟陛下鑒之。」帝嘉納其言。



 牙忽都薨,仁宗命脫列帖木兒嗣楚王。延祐中,明宗西出,脫列帖木兒坐累,徙西番,沒入其家貲之半。明宗即位,制曰:「脫列帖木兒何罪,其轉徙籍沒,豈不以我故耶。其復故號,人民貲帑悉歸之。」脫列帖木兒薨,子八都兒立。八都兒薨,有子三人:曰燕帖木兒,曰速哥帖木兒,曰朵羅不花。燕帖木兒嗣,時年十有二,妃弘吉剌氏,哈只兒駙馬之女孫,速哥失里皇后之從妹也。



 ○寬徹普化



 寬徹普化,世祖之孫,鎮南王脫歡子也。泰定三年,封威順王,鎮武昌,賜金印,撥付怯薛丹五百名,又自募至一千名。設王傳官屬。湖廣行省供億錢糧衣裝,歲支米三萬石,錢三萬二千錠,又日給王子諸妃飲膳。文宗天歷初,賜寬徹普化金銀各五十兩、幣三十匹,仍鎮湖廣,而寬徹普化縱怯薛等官侵奪民利,民頗患苦之。至元五年,太師伯顏矯制召赴京,貶之。及脫脫為相,始明其無辜,命復還鎮。至正二年,湖北廉訪司糾言,寬徹普化恃以宗室,恣行不法。不報。



 十一年,徐壽輝為亂,起蘄、黃,寬徹普化與其子別帖木兒、答帖木兒引兵至金剛臺,壽輝部將倪文俊敗之,執別帖木兒。十二年,壽輝偽將鄒普勝陷武昌,寬徹普化與湖廣行省平章和尚棄城走,詔追奪寬徹普化印,而誅和尚。十三年,湖廣行省參知政事阿魯輝克復武昌及漢陽。寬徹普化復率領王子並本部怯薛丹,屢討賊立功。十四年,詔寬徹普化復鎮武昌,還其印。十六年,命寬徹普化與宣讓王帖木兒不花以兵鎮遏懷慶,各賜黃金一錠、白金五錠、幣帛九匹、鈔二十錠。未幾,復還武昌,命其子報恩奴、接待奴、佛家奴以大船四十餘隻水陸並進,至沔陽攻徐壽輝偽將倪文俊,且載妃妾以行。兵至漢川縣雞鳴汊,水淺船閣,不能行,文俊以火筏盡焚其船,接待奴、佛家奴皆遇害,而報恩奴自死,妃妾皆陷,寬徹普化走陜西。二十五年,侯伯顏答失奉寬徹普化自雲南經蜀轉戰而去,至成州,欲之京師,李思齊以取蜀為名,扼不令行,俾屯田於成州以沒。



 其子曰和尚者,封義王,侍從順帝左右,多著勞效,帝出入常與俱。至正二十四年,孛羅帖木兒稱兵犯闕,遂為中書右丞相,總握國柄,恣為淫虐。和尚心忿其無君,數為帝言之。受密旨,與儒士徐士本謀,交結勇士上都馬、金那海、伯顏達兒、帖古思不花、火你忽都、洪寶寶、黃哈剌八禿、龍從雲,陰圖刺孛羅帖木兒。帝期以事濟,放鴿鈴為號,徐士本掌之。明年七月,孛羅帖木兒入奏事,行至延春閣李樹下,伯顏達兒自眾中奮出,斫孛羅帖木兒,中其腦,上都馬等兢前斫死之。詳見《孛羅帖木兒傳》。二十八年,順帝將北奔,詔淮王帖木兒不花監國,而以和尚佐之,及京城將破,即先遁,不知所之。



 ○帖木兒不花



 帖木兒不花,世祖孫,鎮南王脫歡第四子也。初,世祖第九子脫歡以討安南無成功,終身不許見,遂封鎮南王,出鎮揚州。脫歡薨,子老章襲封鎮南王。老章薨,弟脫不花襲封鎮南王。脫不花薨,子孛羅不花幼,帖木兒不花乃嗣為鎮南王。文宗天歷初,賜帖木兒不花黃金五十兩、白金五十兩、幣三十匹。二年,孛羅不花已長,帖木兒不花請以其位復還孛羅不花。朝廷以其讓而不居也,改封宣讓王,賜金印,移鎮於廬州。順帝至元元年,撥廬州、饒州牧地一百頃賜之。二年,賜市宅錢四千錠,命其王府官凡班次列於有司之右。五年,伯彥擅權,矯制貶帖木兒不花及威順王寬徹普化。至脫脫為相,始言於帝,明此兩王者皆無辜,詔令復還鎮。至正十二年,廬州境內賊起,淮西廉訪使陳思謙言於帖木兒不花曰:「王以帝室之胄,鎮撫淮甸,豈宜坐視。且府中官屬及怯薛丹人等數甚多,必有可使摧鋒陷陣者,惟王圖之。」帖木兒不花大悟其言,曰:「此吾責也。」即命以所部兵及諸王乞塔歹等,分道擊賊,擒其渠帥,廬州境內皆平。帝聞之,賜金帶、銀鈔,以賞其功。十六年,命帖木兒不花與寬徹普化以兵鎮遏懷慶路,賜金銀各一錠、幣帛九匹、鈔二十錠。既而汝、潁之寇南渡淮,帖木兒不花復以便宜,調芍陂屯軍拒之。及廬州不守,乃挈身北歸,留京師。二十七年,進封淮王,賜金印,設王傅等官。二十八年,大明兵逼京師,順帝北奔,詔以帖木兒不花監國,而拜慶童中書左丞相輔之。俄而城破,帖木兒不花死之,年八十三。



\end{pinyinscope}