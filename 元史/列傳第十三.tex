\article{列傳第十三}

\begin{pinyinscope}

 ○安童



 安童,木華黎四世孫,霸突魯長子也。中統初,世祖追錄元勛,召入長宿衛,年方十三,位在百僚上。母弘吉剌氏,昭睿皇后之姊,通籍禁中。世祖一日見之,問及安童,對曰:「安童雖幼,公輔器也。」世祖曰:「何以知之?」對曰:「每退朝必與老成人語,未嘗狎一年少,是以知之。」世祖悅。四年,執阿里不哥黨千餘,將置之法,安童侍側,帝語之曰:「朕欲置此屬於死地,何如?」對曰:「人各為其主,陛下甫定大難,遽以私憾殺人,將何以懷服未附。」帝驚曰:「卿年少,何從得老成語?此言正與朕意合。」由是深重之。



 至元二年秋八月,拜光祿大夫、中書右丞相,增食邑至四千戶。辭曰:「今三方雖定,江南未附,臣以年少,謬膺重任,恐四方有輕朝廷心。」帝動容,有間曰:「朕思之熟矣,無以逾卿。」冬十月,召許衡至,傳旨令衡入省議事,衡以疾辭。安童即親候其館,與語良久,既還,念之不釋者累日。三年,帝諭衡曰:「安童尚幼,未更事,善輔導之。汝有嘉謨,當先告之以達朕,朕將擇焉。」衡對曰:「安童聰敏,且有執守,告以古人所言,悉能領解,臣不敢不盡心。但慮中有人間之,則難行,外用勢力納人其中,則難行。臣入省之日淺,所見如此。」四年三月,安童奏:「內外官須用老成人,宜令儒臣姚樞等入省議事。」帝曰:「此輩雖閑,猶當優養,其令入省議事。」



 五年,廷臣密議立尚書省,以阿合馬領之,乃先奏以安童宜位三公。事下諸儒議,商挺倡言曰:「安童,國之柱石,若為三公,是崇以虛名而實奪之權也,甚不可。」眾曰然,事遂罷。七年四月,奏曰:「臣近言:『尚書省、樞密院各令奏事,並如常制,其大政令,從臣等議定,然後上聞。』既得旨矣,今尚書一切徑奏,似違前旨。」帝曰:「豈阿合馬以朕頗信用之,故爾專權耶?不與卿議,非是。」敕如前旨。



 八年,陜西省臣也速迭兒建言,比因饑饉,盜賊滋橫,若不顯戮一二,無以示懲。敕中書詳議,安童奏曰:「強、竊均死,恐非所宜,罪至死者,宜仍舊待報。」從之。十年春三月,奏以玉冊玉寶上皇後弘吉剌氏,以玉冊金寶立燕王為皇太子,兼中書令,判樞密院事。冬十月,帝諭安童及伯顏等曰:「近史天澤、姚樞纂定《新格》,朕已親覽,皆可行之典,汝等亦當一一留心參考,豈無一二可增減者。」各令紀錄促議行之。時天下待報死囚五十人,安童奏其中十三人因鬥毆殺人,餘無可疑。於是詔以所奏十三人免死從軍。十一年,奏阿合馬蠹國害民數事;又奏各部與大都路官多非才,乞加黜汰。從之。



 十二年七月,詔以行中書省樞密院事,從太子北平王出鎮極邊,在邊十年。二十一年三月,從王歸,待罪闕下,帝即召見慰勞之,頓首謝曰:「臣奉使無狀,有累聖德。」遂留寢殿,語至四鼓乃出。冬十一月,和禮霍孫罷,復拜中書右丞相,加金紫光祿大夫。二十二年,右丞盧世榮敗,詔與諸儒條其所用人及所為事,悉罷之。二十三年夏,中書奏擬漕司諸官姓名,帝曰:「如平章、右丞等,朕當親擇,餘皆卿等職也。」安童奏曰:「比聞聖意欲倚近侍為耳目,臣猥承任使,若所行非法,從其舉奏,罪之輕重,惟陛下裁處。今近臣乃伺隙援引非類,曰某居某官、某居某職,以所署奏目付中書施行。臣謂銓選之法,自有定制,其尤無事例者,臣常廢格不行,慮其黨有短臣者,幸陛下詳察。」帝曰:「卿言是也。今後若此者勿行,其妄奏者,即入言之。」奏徵前吏部尚書李昶,不起;復奏賜田十頃。



 二十四年,宗王乃顏叛,世祖親討平之。宗室詿誤者,命安童按問,多所平反。嘗退朝,自左掖門出,諸免罪者爭迎謝,或執轡扶上馬,安童毅然不顧。有乘間言於帝曰:「諸王雖有罪,皆帝室近親也,丞相雖尊,人臣也,何悖慢如此!」帝良久曰:「汝等小人,豈知安童所為?特辱之使改過耳。」是歲,復立尚書省,安童切諫曰:「臣力不能回天,乞不用桑哥,別相賢者,猶或不至虐民誤國。」不聽。二十五年,見天下大權盡歸尚書,屢求退,不許。二十八年,罷相,仍領宿衛事。三十年春正月,以疾薨於京師樂安里第,年四十九。雨木冰三日。世祖震悼曰:「人言丞相病,朕固弗信,果喪予良弼。」詔大臣監護喪事。大德七年,成宗制贈推忠同德翊運功臣、太師、開府儀同三司、上柱國、東平忠憲王。碑曰《開國元勛命世大臣之碑》。子兀都帶。



 兀都帶器度宏遠,世祖時襲長宿衛。父安童歿,凡賵賻之物,一無所受,以素車樸馬歸葬只蘭禿先塋。事母以孝聞。成宗即位,拜銀青榮祿大夫、大司徒,領太常寺事。為請謚南郊,攝太尉,奉冊上尊號、廟號、皇后尊號。常侍掖庭,贊畫大政,帝及中宮咸以家人禮待之。大德六年正月薨,年三十一。至大二年,制贈輸誠保德翊運功臣、太師、開府儀同三司、上柱國、東平王,謚忠簡。子拜住,自有傳。



 ○廉希憲



 廉希憲,字善用,布魯海牙子也。幼魁偉,舉止異凡兒。九歲,家奴四人盜五馬逃去,既獲,時於法當死,父怒,將付有司,希憲泣諫止之,俱得免死。又嘗侍母居中山,有二奴醉出惡言,希憲曰:「是以我為幼也。」即送系府獄,杖之。皆奇其有識。世祖為皇弟,希憲年十九,得入侍,見其容止議論,恩寵殊絕。希憲篤好經史,手不釋卷。一日,方讀《孟子》,聞召,急懷以進。世祖問其說,遂以性善義利仁暴之旨為對,世祖嘉之,目曰廉孟子,由是知名。嘗與近臣校射世祖前,希憲腰插三矢,有欲取以射者,希憲曰:「汝以我為不能耶?但吾弓力稍弱耳。」左右授以勁弓,三發連中。眾驚服曰:「真文武材也。」



 歲甲寅,世祖以京兆分地命希憲為宣撫使。京兆控制隴蜀,諸王貴籓分布左右,民雜羌戎,尤號難治。希憲講求民病,抑強扶弱。暇日從名儒若許衡、姚樞輩諮訪治道,首請用衡提舉京兆學校,教育人材,為根本計。國制,為士者無隸奴籍,京兆多豪強,廢令不行。希憲至,悉令著籍為儒。有民妻與卜者厭詛其夫,殺之,獄成,僚佐皆言方大旱,卜者宜減死,希憲議當伏法,已而大雨立應。



 初,世祖受命憲宗,經理河南關右,居數歲,讒者謂王府人多專擅不法,至是,命阿藍答兒、劉太平檢核所部,用酷吏分領其事,大開告訐。希憲曰:「宣撫司事由己出,有罪固當獨任,僚屬何預。」及事竟,卒無獲罪者。己未,憲宗駐蹕合州,世祖渡江取鄂州,命希憲入籍府庫。希憲引儒生百餘,拜伏軍門,因言:「今王師渡江,凡軍中俘獲士人,宜官購遣還,以廣異恩。」世祖嘉納之。還者五百餘人。



 憲宗崩,訃音至,希憲啟曰:「殿下太祖嫡孫,先皇母弟,前征雲南,克期撫定,及今南伐,率先渡江,天道可知。且殿下收召才傑,悉從人望,子惠黎庶,率土歸心。今先皇奄棄萬國,神器無主,願速還京,正大位以安天下。」世祖然之,且命希憲先行,審察事變。對曰:「劉太平、霍魯海在關右,渾都海在六盤,征南諸軍散處秦蜀,太平要結諸將,其性險詐,素畏殿下英武,倘倚關中形勝,設有異謀,漸不可制,宜遣趙良弼往覘人情事宜。」從之。阿里不哥構亂北邊,遣脫忽思發兵河朔,大肆兇暴。真定名士李盤嘗奉莊聖太后命侍阿里不哥講讀,脫忽思怒盤不附己,械之,希憲訪盤於獄,言於世祖而釋之。世祖命希憲賜膳於宗王塔察兒,希憲即以己意白王,宜首建翊戴之謀,王然之,許以身任其事。歸啟其言,世祖曰:「若此重事,卿何不懼之甚耶!」庚申,至開平,宗室諸王勸進,謙讓未允,希憲復以天時人事進言。且曰:「阿里不哥於殿下為母弟,居守朔方,專制有年,或覬望神器,事不可測,宜早定大計。」世祖然之。明日即位,建元中統。希憲上言:「高麗王子倎久留京師,今聞其父死,宜立為王,遣還國,以恩結之。」又言:「鄂兵未還,宜遣使與宋講好,敕諸軍北歸。」帝皆從之。



 趙良弼還自關右,奏劉太平、霍魯海反狀,皆如希憲言。初分漢地為十道,乃並京兆、四川為一道,以希憲為宣撫使。太平、霍魯海聞之,乘驛急入京兆,密謀為變。後三日,希憲至,宣布詔旨,遣使安諭六盤。未幾,斷事官闊闊出遣使來告:渾都海已反,殺所遣使者朵羅臺,遣人諭其黨密里火者於成都、乞臺不花於青居,使各以兵來援,又多與蒙古軍奧魯官兀奴忽等金帛,盡起新軍,且約太平、霍魯海同日俱發。希憲得報,召僚屬謂曰:「上新即位,責任吾等,正為今日。不早為之計,殆將無及。」遣萬戶劉黑馬、京兆治中高鵬霄、華州尹史廣,掩捕太平、霍魯海及其黨,獲之,盡得其奸謀,悉置於獄。復遣劉黑馬誅密里火者,總帥汪惟正誅乞臺不花,具以驛聞。時關中無兵備,命汪惟良將秦、鞏諸軍進六盤,惟良以未得上旨為辭,希憲即解所佩虎符銀印授之曰:「此皆身承密旨,君但辦吾事,制符已飛奏矣。」又付銀一萬五千兩,以充功賞,出庫幣制軍衣。惟良感激,遂行。又發蜀卒更戍及在家餘丁,推節制諸軍蒙古官八春將之,謂之曰:「君所將之眾,未經訓練,六盤兵精,勿與爭鋒,但張聲勢,使不得東,則大事濟矣。」會有詔赦至,希憲命絞太平等於獄,尸於通衢,方出迎詔,人心遂安。乃遣使自劾停赦行刑、徵調諸軍、擅以惟良為帥等罪,帝深善之。曰:「《經》所謂行權,此其是也。」別賜金虎符,使節制諸軍,且詔曰:「朕委卿以方面之權,事當從宜,毋拘常制,坐失事機。」



 西川將紐鄰奧魯官將舉兵應渾都海,八春獲之,系其黨五十餘人於乾州獄,送二人至京兆,請並殺之。二人自分必死,希憲謂海僚佐曰:「渾都海不能乘勢東來,保無他慮。今眾志未一,猶懷反側,彼軍見其將校執囚,或別生心,為害不細。今因其懼死,並加寬釋,使之感恩效力,就發此軍餘丁,往隸八春,上策也。」初,八春既執諸校,其軍疑懼,駭亂四出,莫可禁遏,及知諸校獲全,紐鄰奧魯官得釋,大喜過望。切諭其屬出兵效力,人人感悅,八春亦釋然開悟,果得精騎數千,將與俱西。詔以希憲為中書右丞,行秦蜀省事。渾都海聞京兆有備,遂西渡河,趨甘州,阿藍答兒復自和林提兵與之合,分結隴、蜀諸將,又使紐鄰兄宿敦為書招紐鄰。於是成都帥百家奴,興元忙古臺,青居汪惟正、欽察,俱遣使言,人心危疑,事不可測。希憲遣使深諭戒之,兩川諸將素憚希憲威名,按堵從命。渾都海、阿藍答兒合軍而東,諸將失利,河右大震,西土親王執畢帖木兒輜重皆空,就食秦雍。朝議欲棄兩川,退守興元,希憲力言不可,乃止。會親王合丹及汪惟良、八春等合兵復戰西涼,大敗之,俘斬略盡,得二叛首以送,梟之京兆市。事聞,帝大嘉之曰:「希憲真男子也。」進拜平章政事,賜宅一區。時希憲年三十矣。



 希憲奏:四川降民,皆散處山谷,宜申敕軍吏,禁止俘掠,違者,千戶以下與犯人同罪。又禁諸人無販易生口。由是四川遂安,降者益眾。又罷解鹽戶所摘軍,及京兆諸處無籍戶之戍靈州屯田者,以寬民力。欽察獲宋臣張炳震、王政二人,俱以母老,願賜矜放,希憲皆遣之還。因為書與宋四川制置餘玠,諭以天道人事,玠得書,愧感自守,不敢復輕動。鞏昌帥府言,鎮戎州有謀為叛者,連引四百餘人,希憲詳推之,惟誅首惡五人。宋將劉整以瀘州降,盡系前歸宋者數百人待報。希憲奏釋之,且致書宰臣,待整以恩,當得其死力。整後首建取襄陽之策,果立勛效。宋將家屬之在北者,希憲歲給其糧,仕於宋者,子弟得越界省其親,人皆感之。



 李璮反山東,事連王文統,平章趙璧素忌希憲勛名,因言文統由張易、希憲薦引,遂至大用,且關中形勝之地,希憲得民習,有商挺、趙良弼為之輔,此事宜關聖慮。帝曰:「希憲自幼事朕,朕知其心,挺、良弼皆正士,何慮焉。」蜀降人費正寅以私怨譖希憲因李璮叛,亦修城治兵,潛畜異志。帝因惑之,命中書右丞南合代希憲行省,且覆視所告事,卒無實狀。詔希憲還京師。陛見,言曰:「方關陜叛亂,川蜀未寧,事急星火,臣隨宜行事,不謀佐貳,如寅所言,罪止在臣,臣請逮系有司。」帝撫御床曰:「當時之言,天知之,朕知之,卿果何罪!」慰諭良久。進拜中書平章政事。一日夜半,召希憲入禁中,從容道籓邸時事,因及趙璧所言。希憲曰:「昔攻鄂時,賈似道作木柵環城,一夕而成,陛下顧扈從諸臣曰『吾安得如似道者用之』。劉秉忠、張易進曰『山東王文統,才智士也,今為李璮幕僚』。詔問臣,臣對『亦聞之,實未嘗識其人也』。」帝曰:「朕亦記此。」



 希憲在中書,振舉綱維,綜劾名實,汰逐冗濫,裁抑僥幸,興利除害,事無不便,當時翕然稱治,典章文物,粲然可考。又建言:「國家自開創已來,凡納土及始命之臣,咸令世守,至今將六十年,子孫皆奴視部下,都邑長吏,皆其皁隸僮使,前古所無,宜更張之,使考課黜陟。」始議行遷轉法。



 至元元年,丁母憂,率親族行古喪禮,勺飲不入口者三日,慟則嘔血,不能起,寢臥草土,廬於墓傍。宰執以憂制未定,欲極力起之,相與詣廬,聞號痛聲,竟不忍言。未幾,有詔奪情起復,希憲雖不敢違旨,然出則素服從事,入必縗絰。及喪父,亦如之。



 奸臣阿合馬領左右部,專總財賦,會其黨相攻擊,帝命中書推覆,眾畏其權,莫敢問。希憲窮治其事,以狀聞,杖阿合馬,罷所領歸有司。帝諭希憲曰:「吏廢法而貪,民失業而逃,工不給用,財不贍費,先朝患此久矣。自卿等為相,朕無此憂。」對曰:「陛下聖猶堯、舜,臣等未能以皋陶、稷、契之道贊輔治化,以致太平,懷愧多矣。今日小治,未足多也。」因論及魏徵,對曰:「忠臣良臣,何代無之,顧人主用不用爾。」有內侍傳旨入朝堂,言某事當爾,希憲曰:「此閹宦預政之漸,不可啟也。」遂入奏,杖之。



 言者訟丞相史天澤親黨布列中外,威權日盛,漸不可制。詔罷天澤政事,使待鞫問。希憲進曰:「天澤事陛下久,知天澤深者,無如陛下。始自潛籓,多經任使,將兵牧民,悉有治效。陛下知其可付大事,用為輔相。小人一旦有言,陛下當熟察其心跡,果有肆橫不臣者乎?今日信臣,故臣得預此旨,他日有訟臣者,臣亦遭疑矣。臣等備員政府,陛下之疑信若此,何敢自保。天澤既罷,亦當罷臣。」帝良久曰:「卿且退,朕思之。」明日,帝召希憲諭曰:「昨思之,天澤無對訟者。」事遂解。又有訟四川帥欽察者,帝敕中書急遣使誅之。明日,希憲覆奏,帝怒曰:「尚爾遲回耶!」對曰:「飲察大帥,以一小人言被誅,民心必駭,收系至此,與訟者廷對,然後明其罪於天下為宜。」詔遣能者按問。其後事竟無實,欽察得免。



 希憲每奏議帝前,論事激切,無少回惜。帝曰:「卿昔事朕王府,多所容受,今為天子臣,乃爾木強耶?」希憲對曰:「王府事輕,天下事重,一或面從,天下將受其害,臣非不自愛也。」方士請煉大丹,敕中書給所需,希憲具以秦、漢故事奏,且曰:「堯、舜得壽,不因大丹也。」帝曰:「然。」遂卻之。時方尊禮國師,帝命希憲受戒,對曰:「臣受孔子戒矣。」帝曰:「孔子亦有戒耶?」對曰:「為臣當忠,為子當孝,孔子之戒,如是而已。」



 五年,始建御史臺,繼設各道提刑按察司。時阿合馬專總財利,乃曰:「庶務責成諸路,錢穀付之轉運,今繩治之如此,事何由辦?」希憲曰:「立臺察,古制也,內則彈劾奸邪,外則察視非常,訪求民瘼,裨益國政,無大於此。若去之,使上下專恣貪暴,事豈可集耶!」阿合馬不能對。



 七年,詔釋京師系囚。西域人匿贊馬丁,用事先朝,資累巨萬,為怨家所告,系大都獄,既釋之矣,時希憲在告,實不預其事。是秋,車駕還自上都,怨家訴於帝,希憲取堂判補署之,曰:「天威莫測,豈可幸其獨不署以茍免耶!」希憲入見,以詔書為言,帝曰:「詔釋囚耳,豈有詔釋匿贊馬丁耶?」對曰:「不釋匿贊馬丁,臣等亦未聞有此詔。」帝怒曰:「汝等號稱讀書,臨事乃爾,宜得何罪?」對曰:「臣等忝為宰相,有罪當罷退。」帝曰:「但從汝言。」即與左丞相耶律鑄同罷。一日,帝問侍臣,希憲居家何為,侍臣以讀書對。帝曰:「讀書固朕所教,然讀之而不肯用,多讀何為。」意責其罷政而不復求進也。阿合馬因讒之曰:「希憲日與妻子宴樂爾。」帝變色曰:「希憲清貧,何從宴設!」希憲嘗有疾,帝遣醫三人診視,醫言須用沙糖作飲。時最艱得,家人求於外,阿合馬與之二斤,且致密意。希憲卻之曰:「使此物果能活人,吾終不以奸人所與求活也。」帝聞而遣賜之。



 嗣國王頭輦哥行省鎮遼陽,有言其擾民不便者。十一年,詔起希憲為北京行省平章政事。將行,肩輿入辭,賜坐,帝曰:「昔在先朝,卿深識事機,每以帝道啟朕。及鄂漢班師,屢陳天命,朕心不忘,丞相卿實宜為,顧退托耳。遼霫戶不下數萬,諸王、國婿分地所在,彼皆素知卿能,故命卿往鎮,體朕此意。」遼東多親王,使者傳令旨,官吏立聽,希憲至,始革正之。有西域人自稱駙馬,營於城外,系富民,誣其祖父嘗貸息錢,索償甚急,民訴之行省,希憲命收捕之。其人怒,乘馬入省堂,坐榻上,希憲命捽下跪,而問之曰:「法無私獄,汝何人,敢擅系民?」令械系之。其人惶懼求哀,國王亦為之請,乃稍寬,令待對,舉營夜遁。俄詔國王歸國,希憲獨行省事。朝廷降鈔買馬六千五百,希憲遣買於東州,得羨余馬千三百。希憲曰:「上之則若自衒。」即與他郡之不及者,以其直還官。長公主及國婿入朝,縱獵郊原,擾民為甚,希憲面諭國婿,欲入奏之。國婿驚愕,入語公主,公主出,飲希憲酒曰:「從者擾民,吾不知也。請以鈔萬五千貫還斂民之直,幸勿遣使者。」自是貴人過者,皆莫敢縱。



 十二年,右丞阿里海牙下江陵,圖地形上於朝,請命重臣開大府鎮之。帝急召希憲還,使行省荊南,賜坐,諭曰:「荊南入我版籍,欲使新附者感恩、未來者向化,宋知我朝有臣如此,亦足以降其心。南土卑濕,於卿非宜,今以大事付托,度卿不辭。」賜田以養居者,馬五十以給從者。希憲曰:「臣每懼才識淺近,不能勝負大任,何敢辭疾。然敢辭新賜。」復有詔,令希憲承制授三品以下官。希憲冒暑疾驅以進。至鎮,阿里海牙率其屬郊迎,望拜塵中,荊人大駭。即日禁剽奪,通商販,興利除害,兵民按堵。首錄宋故宣撫、制置二司幕僚能任事者,以備採訪,仍擇二十餘人,隨材授職。左右難之,希憲曰:「今皆國家臣子也,何用致疑。」時宋故官禮謁大府,必廣致珍玩,希憲拒之,且語之曰:「汝等身仍故官,或不次遷擢,當念聖恩,盡力報效。今所饋者,若皆己物,我取之為非義;一或系官,事同盜竊;若斂於民,不為無罪。宜戒慎之。」皆感激謝去。令凡俘獲之人,敢殺者,以故殺平民論。為軍士所虜,病而棄之者,許人收養;病愈,故主不得復有。立契券質賣妻子者,重其罪,仍沒入其直。先時,江陵城外蓄水捍禦,希憲命決之,得良田數萬畝,以為貧民之業。發沙市倉粟之不入官籍者二十萬斛,以賑公安之饑。大綱既舉,乃曰:「教不可緩也。」遂大興學,選教官,置經籍,旦日親詣講舍,以厲諸生。西南溪洞,及思、播田、楊二氏,重慶制置趙定應,俱越境請降。事聞,帝曰:「先朝非用兵不可得地,今希憲能令數千百里外越境納土,其治化可見也。」關吏得江陵人私書,不敢發,上之,樞密臣發之帝前,其中有曰:「歸附之初,人不聊生。皇帝遣廉相出鎮荊南,豈惟人漸德化,昆蟲草木,咸被澤矣。」帝曰:「希憲不嗜殺人,故能爾也。」



 希憲疾久不愈,十四年春,近臣董文忠言:「江陵濕熱,如希憲病何?」即召希憲還,江陵民號泣遮道留之不得,相與畫像建祠。希憲還,囊橐蕭然,琴書自隨而已。帝知其貧,特賜白金五千兩、鈔萬貫。五月,至上都,太常卿田忠良來問疾,希憲謂曰:「上都聖上龍飛之地,天下視為根本。近聞龍岡遺火,延燒民居,此常事耳,慎勿令妄談地理者惑動上意。」未幾,果有數輩以徙置都邑事奏,樞密副使張易、中書左丞張文謙與廷辨,力言不可,帝不悅。明日,召忠良質其事,忠良以希憲語對,帝曰:「希憲病甚,猶慮及此耶?」其議遂止。詔徵揚州名醫王仲明視希憲疾,既至,希憲服其藥,能杖而起,帝喜謂希憲曰:「卿得良醫,疾向愈矣。」對曰:「醫持善藥以療臣疾,茍能戒慎,則誠如聖諭;設或肆惰,良醫何益。」蓋以醫諷諫也。



 會議立門下省,帝曰:「侍中非希憲不可。」遣中使諭旨曰:「鞍馬之任,不以勞卿,坐而論道,時至省中,事有必須執奏,肩輿以入可也。」希憲附奏曰:「臣疾何足恤。輸忠效力,生平所願。」皇太子亦遣人諭旨曰:「上命卿領門下省,無憚群小,吾為卿除之。」竟為阿合馬所沮。



 十六年春,賜鈔萬貫,詔復入中書,希憲稱疾篤。皇太子遣侍臣問疾,因問治道,希憲曰:「君天下在用人,用君子則治,用小人則亂。臣病雖劇,委之於天。所甚憂者,大奸專政,群小阿附,誤國害民,病之大者。殿下宜開聖意,急為屏除,不然,日就沉彖,不可藥矣。」戒其子曰:「丈夫見義勇為,禍福無預於己,謂皋、夔、稷、契、伊、傅、周、召為不可及,是自棄也。天下事茍無牽制,三代可復也。」又曰:「汝讀《狄梁公傳》乎?梁公有大節,為不肖子所墜,汝輩宜慎之!」



 十七年十一月十九夜,有大星隕於正寢之旁,流光照地,久之方滅。是夕,希憲卒,年五十。大德八年,贈忠清粹德功臣、太傅、開府儀同三司,追封魏國公,謚文正。加贈推忠佐理翊運功臣、太師、開府儀同三司、上柱國、恆陽王,謚如故。



 子六人:孚,僉遼陽等處行中書事;恪,臺州路總管;恂,中書平章政事;忱,邵武路總管;恆,御史中丞;惇,江西等處行中書省參知政事。從弟希賢。



 希賢字達甫,一名中都海牙。伯父布魯海牙嘗曰:「是兒剛果,當大吾家。」年二十餘,與從兄希憲同侍世祖,出入禁中,小心慎密。至元初,北部王拘殺使者,世祖選使往諭之,廷臣推希賢。至則布上意,辭旨條暢,王悔謝,為設宴,贈貂裘一襲、白金一笏。還奏,帝喜,賜以御膳。尋進中議大夫、兵部尚書。左丞相伯顏伐宋,既渡江,至元十二年春,授希賢禮部尚書,佩金虎符,與工部侍郎嚴中範、秘書丞柴紫芝持國書使宋。三月丙戌,至廣德軍獨松關,守關者不知為使,襲而殺之。張濡以為己功,受賞,知廣德軍。明年宋亡,獲張濡殺之,詔遣使護希賢喪歸,後復籍濡家貲付其家。希賢死時,年二十九。



\end{pinyinscope}