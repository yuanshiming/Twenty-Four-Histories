\article{志第四十 百官六}

\begin{pinyinscope}

 大都留守司,秩正二品,掌守衛宮闕都城,調度本路供億諸務,兼理營繕內府諸邸、都宮原廟、尚方車服、殿廡供帳、內苑花木墨子墨家思想的著作總匯。舊題為戰國墨翟撰,實為其,及行幸湯沐宴游之所,門禁關鑰啟閉之事。留守五員,正二品;同知二員,正三品;副留守二員,正四品;判官二員,正五品;經歷一員,從六品;都事二員,從七品;管勾承發架閣庫一員,正八品;照磨兼覆料官一員,部役官兼壕寨一員,令史十八人,宣使十七人,典吏五人,知印二人,蒙古必闍赤三人,回回令史一人,通事一人。至元十九年,罷宮殿府行工部,置大都留守司,兼本路都總管,知少府監事。二十一年,別置大都路都總管府治民事,並少府監歸留守司。皇慶元年,別置少府監。延祐七年,罷少府監,復以留守兼監事。其屬附見:



 修內司,秩從五品,領十四局人匠四百五十戶,掌修建宮殿及大都造作等事,提點一員,大使一員,副使一員,直長五員,吏目一員,照磨一員,部役七員,司吏六人。中統二年置。至元中,增工匠,計一千二百七十有二戶。其屬附見:



 大木局,提領七員,管勾三員,掌殿閣營繕之事。中統二年置。



 小木局,提領二員,同提領一員,副提領三員,管勾二員,提控四員。中統四年置。



 泥廈局,提領八員,管勾二員。中統四年置。



 車局,提領二員,管勾一員。中統五年置。



 妝釘局,提領二員,同提領二員。中統四年置。



 銅局,提領一員,同提領一員,管勾一員。中統四年置。以上六局,秩從八品。



 竹作局,提領二員,提控一員。中統四年置。



 繩局,提領二員。中統五年始置。



 祗應司,秩從五品,掌內府諸王邸第異巧工作,修禳應辦寺觀營繕,領工匠七百戶。大使一員,從五品;副使一員,正七品;直長三員,正八品;吏目一員,司吏二人。國初,建兩京殿宇,始置司以備工役。其屬附見:



 油漆局,提領五員,同提領、副提領各一員,掌兩都宮殿髹漆之工。中統元年置。



 畫局,提領五員,管勾一員,掌諸殿宇藻繪之工。中統元年置。



 銷金局,提領一員,管勾二員,掌諸殿宇裝鋈之工。中統四年置。



 裱褙局,提領一員,掌諸殿宇裝潢之工。中統二年置。



 燒紅局,提領二員,掌諸宮殿所用心紅顏料。至元元年置。



 器物局,秩從五品,掌內府宮殿、京城門戶、寺觀公廨營繕,及御用各位下鞍轡、忽哥轎子、帳房車輛、金寶器物,凡精巧之藝,雜作匠戶,無不隸焉。大使一員,從五品;副使一員,正七品;直長二員,正八品;吏目一員,司吏二人。中統四年,始立御用器物局,受省札。至元七年,改為器物局,秩如上。其屬附見:



 鐵局,提領三員,管勾三員,提控一人,掌諸殿宇輕細鐵工。中統四年置。



 鐵局,提領三員,管勾三員,提控一人,掌諸殿宇輕細鐵工。中統四年置。



 減鐵局,管勾一員,提控二人,掌造御用及諸宮邸系腰。中統四年置。



 盒缽局,提領二員,掌制御用系腰。中統四年置。



 成鞍局,提領三員,掌造御用鞍轡、象轎。中統四年置。



 羊山鞍局,提領一員,提控一員。掌造常課鞍轡諸物。至元十八年置。



 網局,提領二員,管勾一員,掌成造宮殿網扇之工。中統四年置。



 刀子局,提控二員,掌造御用及諸宮邸寶貝佩刀之工。中統四年置。



 旋局,提領二員,掌造御用異樣木植器物之工。中統四年置。



 銀局,提領一員,掌造御用金銀器盒系腰諸物。中統四年置。



 轎子局,提領一員,掌造御用異樣木植鞍子諸物。中統四年置。



 採石局,秩從七品,大使、副使各一員,掌夫匠營造內府殿宇寺觀橋閘石材之役。至元四年,置石局總管。十一年,撥採石之夫二千餘戶,常任工役,置大都等處採石提舉司。二十六年罷,立採石局。



 山場,提領一員,管勾五員。至元四年置。



 大都城門尉,秩正六品,尉二員,副尉一員,掌門禁啟閉管鑰之事。至元二十年置,以四怯薛八剌哈赤為之。二十四年,復以六衛親軍參掌。凡十有一門:曰麗正,曰文明,曰順承,曰平則,曰和義,曰肅清,曰安貞,曰健德,曰光熙,曰崇仁,曰齊化。每門設官如上。



 犀象牙局,秩從六品,大使、副使、直長各一員,司吏一人,掌兩都宮殿營繕犀象龍床卓器系腰等事。中統四年置,設官一員。至元五年,增副使一員,管匠戶一百有五十。其屬附見:



 雕木局,提領一員,掌宮殿香閣營繕之事。至元十一年置。



 牙局,提領一員,管勾一員,掌宮殿象牙龍床之工。至元十一年置。



 大都四窯場,秩從六品,提領、大使、副使各一員,領匠夫三百餘戶,營造素白琉璃磚瓦,隸少府監。至元十三年置。其屬三:



 南窯埸,大使、副使各一員。中統四年置。



 西窯埸,大使、副使各一員。至元四年置。



 琉璃局,大使、副使各一員。中統四年置。



 凡山採木提舉司,秩從五品,掌採伐車輛等雜作木植,及造只孫系腰刀把諸物。達魯花赤、提舉各一員,並從五品;同提舉一員,正七品;副提舉一員,正八品;吏目一員,司吏六人。至元十四年置。



 上都採山提領所,秩從八品,提領、副提領、提控各一員。至元九年,以採伐材木,煉石為灰,徵發夫匠一百六十三戶,遂置官以統之。



 凡山宛平等處管夫匠所,提領二員,同提領二員,管領催車材戶提領一員。至元十五年置。



 器備庫,秩從五品,提點一員,從五品;大使一員,從六品;副使二員,正七品;直長四員,正八品,掌殿閣金銀寶器二千餘事。至元二十七年置。



 甸皮局,秩正七品,大使一員,管匠三十餘戶。至元七年置。十四年,始定品秩。二十一年,改隸留守司。歲辦熟造紅甸羊皮二千有奇。



 上林署,秩從七品,署令、署丞各一員,直長一員,掌宮苑栽植花卉,供進蔬果,種苜蓿以飼駝馬,備煤炭以給營繕。至元二十四年置。



 養種園,提領二員,掌西山淘煤,羊山燒造黑白木炭,以供修建之用。中統三年置。



 花園,管勾二員,掌花卉果木。至元二十四年置。



 苜蓿園,提領三員,掌種苜蓿,以飼馬駝膳羊。



 儀鸞局,秩正五品,掌殿庭燈燭張設之事,及殿閣浴室門戶鎖鑰,苑中龍舟,圈檻珍異禽獸,給用內府諸宮太廟等處祭祀庭燎,縫制簾帷,灑掃掖庭,領燭剌赤、水手、樂人、禁蛇人等二百三十餘戶。輪直怯薛大使四員,正五品;副使二員,從六品;直長二員,正八品;都目一員,書吏二人,庫子一人。至元十一年置局,秩正七品。二十三年,升正五品。至大四年,仁宗御西宮,又別立儀鸞局,設置亦同。延祐七年,增大使二員,以宦者為之。領四提領所:



 燭剌赤,提領八員,提控四員。



 水手,提領二員。



 針工,提領一員。



 蠟燭局,提領一員。



 木場,提領一員,大使一員,副使一員,掌受給營造宮殿材木。至元四年,置南東二木場。十七年,並為一場。



 大都路管領諸色人匠提舉司,秩從五品,掌大都諸色匠戶理斷昏田詞訟等事。提舉一員,從五品;同提舉一員,正七品;副提舉一員,正八品;吏目一人,司吏二人。中統四年,置人匠奧魯總管府,秩從四品。至元十二年,改提舉司。十五年,兼管採石人戶,秩如舊。



 真定路、東平路管匠官,秩從七品,每路大使一員,副使一員。中統四年置。



 保定路、宣德府管匠官,秩從七品。保定大使一員,副使一員,管匠官一員;宣德二員。中統四年置。



 大名路管匠官,秩從七品,大使一員,管匠官三員。中統四年置。



 晉寧、冀寧、大同、河間四路管匠官,秩從七品,每路大使、副使各一員。中統四年置。



 收支庫,秩正九品,掌受給營繕,提點一員,大使一員,副使二員,直長二員,庫子二人。至元四年置。



 諸色庫,秩從八品,掌修內材木,及江南征索異樣木植,並應辦官寺齋事,大使一員,副使一員,司庫二人。至大四年置。



 太廟收支諸物庫,秩從八品,大使、副使各一員,司庫四人。至治二年,以營治太廟始置。



 南寺、北寺收支諸物二庫,秩從七品,提領、大使各一員,副使二員,司庫之屬凡十人。至治元年,以建壽安山寺始置。



 廣誼司,秩正三品。司令二員,正三品;同知二員,正四品;副使二員,正五品;判官二員,正六品;經歷、知事各二員,照磨一員。總和顧和買、營繕織造工役、供億物色之務。至元十四年,改覆實司辨驗官,兼提舉市令司。大德五年,又分大都路總管府官屬,置供需府。至順二年罷之,立廣誼司。



 武備寺,秩正三品,掌繕治戎器,兼典受給。卿四員,正三品;同判六員,從三品;少卿四員,從四品;丞四員,從五品;經歷、知事各一員,照磨兼提控案牘一員,承發架閣庫管勾一員,辨驗弓官二員,辨驗筋角翎毛等官二員,令史十有三人。至元五年,始立軍器監,秩四品。十九年,升正三品。二十年,立衛尉院。改軍器監為武備監,秩正四品,隸衛尉院。二十一年,改監為寺,與衛尉並立。大德十一年,升為院。至大四年,復為寺,設官如舊。其所轄屬官,則自為選擇其匠戶之能者任之。



 壽武庫,秩從五品,提點二員,從五品;大使二員,正六品;副使四員,正七品;庫子一十人。至元十年,以衣甲庫改置。



 利器庫,秩從五品,提點三員,大使二員,副使三員,秩品同壽武庫,庫子一十人。至元五年,始立軍器庫。十年,通掌隨路軍器,改利器庫。



 廣勝庫,秩從五品,掌平陽、太原等處歲造兵器,以給北邊征戍軍需。達魯花赤一員,大使、副使各一員,庫子一人。



 大同路軍器人匠提舉司,秩從五品,達魯花赤一員,提舉一員,並從五品;同提舉一員,正七品;副提舉一員,正八品。其屬:豐州甲局,院長一員。應州甲局,院長一員。平地縣甲局,院長一員。山陰縣甲局,院長一員。白登縣甲局,頭目一人。豐州弓局,使一員。賽甫丁弓局,頭目一人。



 平陽路軍器人匠提舉司,秩正六品,達魯花赤一員,提舉、同提舉、副提舉各一員。其屬:本路投下雜造局,大使一員,副使一員。絳州甲局,大使一員。



 太原路軍器人匠局,秩正七品,達魯花赤一員,局使一員,副使一員,吏目一員。



 保定軍器人匠提舉司,秩從六品,達魯花赤、提舉、同提舉、副提舉各一員。其屬:河間甲局,院長一員。祈州安平縣甲局,院長一員。陵州箭局,頭目一人。



 真定路軍器人匠提舉司,秩從六品,達魯花赤、提舉、同提舉、副提舉各一員。其屬:冀州甲局,院長一人。



 懷孟河南等路軍器人匠局,秩正七品,局使、局副各一員。其屬:懷孟路弓局,院長一員。汴梁路軍器局,秩正七品,局使、局副各一員。其屬:常課弓局,院長一員。常課甲局,院長一員。



 益都濟南箭局,秩正七品,局使一員。



 彰德路軍器人匠局,秩正七品,大使一員,副使一員。



 大名軍器局,秩正七品,大使、副使各一員。



 上都甲匠提舉司,秩從五品,提舉、同提舉、副提舉各一員。其屬:興州白局子甲局,院長一員。興州千戶寨甲局,院長一員。松州五指崖甲局,院長一員。松州勝安甲局,院長一員。



 遼河等處諸色人匠提舉司,秩從五品,達魯花赤、提舉、同提舉各一員。其屬:遼蓋弓局,大使、副使各一員。蓋州甲局,局使一員。



 上都雜造局,秩正七品,大使、副使各一員。



 奉聖州軍器局,秩從七品,大使、副使各一員。



 蔚州軍器人匠提舉司,秩正六品,達魯花赤、提舉、同提舉、副提舉各一員。



 宣德府軍器人匠提舉司,秩正六品,達魯花赤、提舉、同提舉、副提舉各一員。



 廣平路甲局,院長一員。



 東平等路軍器人匠提舉司,秩從五品,達魯花赤、提舉、同提舉、副提舉各一員。



 通州甲匠提舉司,秩正六品,達魯花赤、提舉、同提舉、副提舉各一員。



 蘇州甲匠提舉司,秩正五品,達魯花赤、提舉、同提舉、副提舉各一員。



 欠州武器局,秩從五品,大使、副使各一員。



 大都甲匠提舉司,秩正六品,達魯花赤、提舉、同提舉、副提舉各一員。



 大都箭局,秩從七品,大使、副使各一員。



 大寧路軍器人匠提舉司,秩從六品,達魯花赤、提舉、同提舉、副提舉各一員。



 豐州雜造局,秩正六品,達魯花赤、大使、副使各一員。



 歸德府軍器局,院長一員。



 汝寧府軍器局,院長一員。



 陳州軍器局,院長一員。



 許州軍器局,秩從七品,大使、副使各一員。



 咸平府軍器人匠局,秩從七品,達魯花赤、大使、副使各一員。



 大都弓匠提舉司,秩正五品,達魯花赤、提舉、同提舉、副提舉各一員。其屬:雙搭弓局,大使、副使各一員。成吉里弓局,大使、副使各一員。通州弓局,院長一員。



 大都弦局,大使、副使各一員。至元三十年,改提舉司置局。



 隆興路軍器人匠局,達魯花赤、大使、副使各一員。至元三十年置。



 平灤路軍器人匠局,大使、副使各一員。至元三十年置。



 大都雜造局,提領二員。元貞二年置。



 太僕寺,秩從二品,掌阿塔思馬匹,受給造作鞍轡之事。中統四年,設群牧所。至元十六年,改尚牧監。十九年,又改太僕院。二十年,改衛尉院。二十四年,罷院,立太僕寺。又別置尚乘寺以管鞍轡,而本寺止管阿塔思馬匹。二十五年,隸中書,置提調官二員。大德十一年,復改太僕院。至大四年,仍為寺。卿二員,從二品;少卿二員,從四品;丞二員,從五品;經歷、知事、照磨、管勾各一員,令史七人,譯史、知印、通事各二人,奏差四人,回回令史一人,典吏二人。



 尚乘寺,秩正三品,掌上御鞍轡輿輦。阿塔思群牧騸馬驢騾,及領隨路局院鞍轡等造作,收支行省歲造鞍轡,理四怯薛阿塔赤詞訟,起取南北遠方馬匹等事。卿四員,正三品;少卿二員,從四品;丞二員,從五品;經歷、知事、照磨、管勾各一員,令史六人,譯史二人,知印二人,通事二人,奏差五人,典吏二人。至元二十四年,罷衛尉院,始設尚乘寺,領資乘庫。大德十一年,升為院,秩從二品。至大四年,復為寺。延祐七年,降從三品。



 資乘庫,秩從五品,提點四員,從五品;大使三員,正六品;副使四員,正七品;庫子四人。掌收支鞍轡等物。至元十三年置。二十年,隸衛尉。二十四年,隸尚乘寺。



 長信寺,秩正三品,領大斡耳朵怯憐口諸事。卿四員,正三品;少卿二員,從四品;寺丞二員,從五品;經歷、知事各一員,令史六人,譯史、知印各二人,通事一人,奏差四人。大德五年置。至大元年,改升為院。四年,仍為寺,卿五員,增少卿一員,以宦者為之。延祐七年,省寺卿、少卿各一員,定置如上。



 怯憐口諸色人匠提舉司,秩從五品,領大都、上都二鐵局並怯憐口人匠,以材木鐵炭皮貨諸色,備斡耳朵各枝房帳之需。達魯花赤一員,提舉、同提舉、副提舉各一員,吏目一人,司吏四人。至元二十五年置。



 大都鐵局,秩從五品,掌斡耳朵上下往來造作妝釘房車,大使一員,副使一員,直長一員。至元十二年置。



 上都鐵局,大使一員,副使一員。至元十六年置。掌職如前。



 長秋寺,秩正三品,掌武宗五斡耳朵戶口錢糧營繕諸事。寺卿五員,正三品;少卿二員,從四品;寺丞二員,從五品;經歷、知事各一員,令史六人,譯史、知印各二人,通事一人,奏差四人。皇慶二年置。其屬二:



 怯憐口諸色人匠提舉司,秩從五品,掌正宮造作之役,達魯花赤一員,同提舉、副提舉各一員,吏目一人,司吏四人。至大元年,斡耳朵三位下撥到人匠五百九十三戶,始置提舉司,隸中政院,後屬長信寺。



 怯憐口諸色人匠提舉司,秩從五品,掌領武宗軍上北來人匠,達魯花赤一員,提舉一員,同提舉、副提舉各一員,吏目一人,司吏二人。至大元年置。



 承徽寺,秩正三品,掌答兒麻失里皇后位下錢糧營繕等事。寺卿五員,正三品;少卿二員,從四品;寺丞二員,從五品;經歷、知事各一員,令史六人,譯史、知印各二人,通事一人,奏差四人。至治元年置。其屬二:



 怯憐口諸色人匠提舉司二,秩正五品,各設達魯花赤一員,提舉、同提舉、副提舉各一員,吏目一人,司吏三人。至治三年置。



 長寧寺,秩正三品,掌英宗速哥八剌皇后位下戶口錢糧營繕等事。寺卿六員,正三品;少卿二員,從四品;寺丞二員,從五品;經歷、知事各一員,吏屬令史六人,譯史、知印各二人,怯里馬赤一人,奏差四人。至治三年置。



 長慶寺,秩正三品,掌成宗斡耳朵及常歲管辦禾失房子、行幸怯薛臺人等衣糧之事。寺卿六員,少卿二員,寺丞二員,品秩同長寧寺;經歷、知事各一員,令史六人,譯史、知印各二人,怯里馬赤一人,奏差四人。泰定元年置。



 寧徽寺,秩正三品,隸八不沙皇后位下。寺卿六員,少卿四員,丞二員,品秩同長慶寺;經歷、知事各一員。天歷二年置。



 太府監,秩正三品,領左、右藏等庫,掌錢帛出納之數。太卿六員,正三品;太監六員,從三品;少監五員,從四品;丞五員,正五品;經歷、知事、照磨各一員,令史八人,譯史三人,通事、知印各一人,奏差四人。中統四年置。至元四年,為宣徽太府監,凡內府藏庫悉隸焉。八年,升正三品。大德九年,改為院,秩從二品,院判參用宦者。至大四年,復為監,定置如上。



 內藏庫,秩從五品,掌出納御用諸王段匹納失失紗羅絨錦南綿香貨諸物。提點四員,從五品;大使二員,正六品;副使二員,正七品。至元二年,置署上都。十九年,始署大都,以宦者領之。復有行內藏,二十八年省之,止存內藏及左右二庫。



 右藏,提點四員,大使二員,副使二員,品秩同上,掌收支金銀寶鈔、只孫段匹、水晶瑪瑙玉璞諸物。至元十九年置。



 左藏,提點四員,大使二員,副使二員,品秩同上,掌收支常課和買紗羅布絹絲綿絨錦木綿鋪陳衣服諸物。至元十九年置。



 度支監,秩正三品,掌給馬駝芻粟。卿三員,正三品;太監二員,從三品;少監三員,從四品;監丞二員,從五品;經歷二員,知事一員,提控案牘一員,照磨兼管勾一員,令史十四人,譯史四人,通事、知印三人,奏差四人,典吏五人。國初,置孛可孫。至元八年,以重臣領之。十三年,省孛可孫,以宣徽兼其任。至大二年,改立度支院。四年,改為監。



 利用監,秩正三品,掌出納皮貨衣物之事。監卿八員,正三品;太監五員,從三品;少監五員,從四品;監丞四員,正五品;經歷、知事、照磨、管勾各一員,令史八人,譯史二人,通事、知印各一人,奏差六人,典吏三人。至元十年置。二十年罷,二十六年復置。大德十一年,改為院。至大四年,復為監。



 資用庫,秩從五品,提點二員,從五品;大使三員,正六品;副使五員,正七品;庫子五人。至元二年置,隸太府。十年,隸利用。



 怯憐口皮局人匠提舉司,秩正五品,提舉二員,同提舉一員,提控案牘一員。中統元年置局。至元六年,改提舉司。



 雜造雙線局,秩從八品,造內府皮貨鷹帽等物,大使、副使、直長、典史各一員。



 熟皮局,掌每歲熟造野獸皮貨等物,大使、副使、直長各一員。至元二十年置。



 軟皮局,掌內府細色銀鼠野獸諸色皮貨,大使、副使、直長各一員。至元二十五年置。



 斜皮局,掌每歲熟造內府各色野馬皮胯,副使二員。至元二十年置。



 貂鼠局提舉司,秩從五品,提舉一員,同提舉、副提舉各一員。至元二十年置。



 貂鼠局,副使二員,直長一員。至元十九年立。



 染局,副使一員,直長一員,管勾一員,掌每歲變染皮貨。至元二十年始置。



 熟皮局,秩從七品,大使一員,副使一員,典史一人,司吏一人。至元六年置。



 中尚監,秩正三品,掌大斡耳朵位下怯憐口諸務,及領資成庫氈作,供內府陳設帳房簾幕車輿雨衣之用。監卿八員,正三品;太監二員,從三品;少監二員,從四品;監丞二員,正五品;經歷、知事、照磨各一員,令史七人,譯史三人,通事二人,知印二人,奏差五人。至元十五年,置尚用監。二十年罷。二十四年,改置中尚監。三十年,分置兩都灤河三庫怯憐口雜造等九司局而總領之。至大元年,升為院。四年,復為監,參用宦者三人。



 資成庫,秩從五品,掌造氈貨。提點三員,從五品;大使三員,正六品;副使三員,正七品。至元二年置,隸太府。二十三年,始歸於監。



 章佩監,秩正三品,掌宦者速古兒赤所收御服寶帶。監卿五員,正三品;太監四員,從三品;少監二員,從四品;監丞二員,正五品;經歷、知事、照磨各一員,令史七人,譯史二人,通事二人,奏差四人。至元二十二年置。至大元年,升為院,秩從二品。四年,復為監,定置如上。



 御帶庫,秩從五品,掌系腰偏束等帶並絳環諸物,供奉御用,以備賜予。提點三員,大使三員,副使二員,品秩同資成。至元二十八年置,俱以中官為之。元貞二年,增二員,兼署上都之事。



 異珍庫,秩從五品,掌御用珍寶、后妃公主首飾寶貝。提點三員,大使三員,副使二員,品秩同上。至元二十八年置。



 經正監,秩正三品,掌營盤納缽及標撥投下草地,有詞訟則治之。太卿一員,正三品;太監二員,從三品;少監二員,從四品;監丞二員,正五品;經歷、知事各一員,令史八人,譯史四人。至大四年置。監卿、太監、少監並奴都赤為之,監丞流官為之。



 都水監,秩從三品,掌治河渠並堤防水利橋梁閘堰之事。都水監二員,從三品;少監一員,正五品;監丞二員,正六品;經歷、知事各一員,令史十人,蒙古必闍赤一人,回回令史一人,通事、知印各一人,奏差十人,壕寨十六人,典吏二人。至元二十八年置。二十九年,領河道提舉司。大德六年,升正三品。延祐七年,仍從三品。



 大都河道提舉司,秩從五品,提舉一員,從五品;同提舉一員,從六品;副提舉一員,從七品。



 秘書監,秩正三品,掌歷代圖籍並陰陽禁書。卿四員,正三品;太監二員,從三品;少監二員,從四品;監丞二員,從五品;典簿一員,從七品;令史三人,知印、奏差各二人,譯史、通事各一人,典書二人,典吏一人。屬官:著作郎二員,從六品;著作佐郎二員,正七品;秘書郎二員,正七品;校書郎二員,正八品;辨驗書畫直長一員,正八品。至元九年置。其監丞皆用大臣奏薦,選世家名臣子弟為之。大德九年,升正三品,給銀印。延祐元年,定置卿四員,參用宦者二人。



 司天監,秩正四品,掌凡歷象之事。提點一員,正四品;司天監三員,正四品;少監五員,正五品;丞四員,正六品;知事一員,令史二人,譯史一人,通事兼知印一人。屬官:提學二員,教授二員,並從九品;學正二員,天文科管勾二員,算歷科管勾二員,三式科管勾二員,測驗科管勾二員,漏刻科管勾二員,並從九品;陰陽管勾一員,押宿官二員,司辰官八員,天文生七十五人。中統元年,因金人舊制,立司天臺,設官屬。至元八年,以上都承應闕官,增置行司天監。十五年,別置太史院,與臺並立,頒歷之政歸院,學校之設隸臺。二十三年,置行監。二十七年,又立行少監。皇慶元年,升正四品。延祐元年,特升正三品。七年,仍正四品。



 回回司天監,秩正四品,掌觀象衍歷。提點一員,司天監三員,少監二員,監丞二員,品秩同上;知事一員,令史二員,通事兼知印一人,奏差一人。屬官:教授一員,天文科管勾一員,算歷科管勾一員,三式科管勾一員,測驗科管勾一員,漏刻科管勾一員,陰陽人一十八人。世祖在潛邸時,有旨征回回為星學者,札馬剌丁等以其藝進,未有官署。至元八年,始置司天臺,秩從五品。十七年,置行監。皇慶元年,改為監,秩正四品。延祐元年,升正三品,置司天監。二年,命秘書卿提調監事。四年,復正四品。



 上都留守司兼本路都總管府,品秩職掌如大都留守司,而兼治民事。車駕還大都,則領上都諸倉庫之事。留守六員,正二品;同知二員,正三品;副留守二員,正四品;判官二員,正五品;經歷二員,都事四員,照磨兼管勾一員,令史四十四人,譯史六人,回回令史三人,通事、知印各二人,宣使一十二人。國初,置開平府。中統四年,改上都路總管府。至元三年,又給留守司印。十九年,並為上都留守司兼本路都總管府。其屬附見:



 修內司,秩從五品,掌營修內府之事。大使一員,從五品;副使三員,正七品;直長三員,正八品。至元八年置。



 祗應司,秩從五品,掌妝鑾油染裱褙之事。大使一員,從五品;副使二員,正七品;直長三員,正八品。



 器物局,秩從五品,掌造鐵器,內府營造釘線之事,大使一員,副使一員,直長二員。



 儀鸞局,秩正五品,大使二員,副使三員,直長二員。至大四年,罷典設署,改置為局。



 兵馬司,秩正四品,指揮使三員,副指揮使二員,知事一員,提控案牘一員,司吏八人。至元二十九年置。



 警巡院,秩正六品,達魯花赤一員,警巡使一員,副使二員,判官二員,司吏八人。



 開平縣,秩正六品,達魯花赤一員,尹一員,丞一員,主簿一員,尉一員,典史一員,司吏八人。



 平盈庫,大使一員,副使一員。至元三十年置。



 萬盈庫,達魯花赤、監支納、大使、副使各一員。中統初置。



 廣積倉,達魯花赤、監支納、大使、副使各一員。中統初,置永盈倉。大德間,改為廣積倉。



 萬億庫,秩正五品,達魯花赤一員,提舉一員,同提舉、副提舉各一員,提控案牘一員,司吏六人,譯史一人。至元二十三年置。



 行用庫,提點一員,大使一員,副使一員。



 稅課提舉司,秩正五品,提舉二員,同提舉、副提舉、提控案牘各一員。元貞元年置。



 八作司,品秩職掌,悉與大都左右八作司同,達魯花赤一員,提領、大使、副使各一員。至元十七年置。



 餼廩司,掌諸王駙馬使客飲食,大使一員,副使一員。至元二年,置上都應辦所。延祐五年,改為餼廩司。



 尚供總管府,秩正三品,掌守護東涼亭行宮,及游獵供需之事。達魯花赤一員,總管一員,並正三品;同知一員,從四品;副總管一員,從五品;判官一員,正六品;經歷、知事、提控案牘各一員,令史、譯史、知印、奏差有差。至元十三年,置只哈赤八剌哈孫達魯花赤。延祐二年,改總管府。其屬附見。



 香河等處巡檢司,巡檢一員,司吏一人。



 景運倉,秩從五品,提點一員,從五品;大使一員,正六品;副使一員,正七品。至元二十一年置。



 法物庫,秩從九品,大使、副使各一員。至元二十九年置。



 雲需總管府,秩正三品,掌守護察罕腦兒行宮,及行營供辦之事。達魯花赤一員,總管一員,並正三品;同知一員,從四品;副總管一員,從五品;判官一員,正六品;經歷一員,知事一員,提控案牘一員。延祐二年置。



 大都路都總管府,秩正三品,達魯花赤二員,都總管一員,副達魯花赤二員,同知二員,治中二員,判官二員,推官二員,經歷二員,知事二員,提控案牘四員,照磨兼管勾一員,令史九十有五人,譯史二人,回回令史一人,通事、知印各二人,奏差二十一人。國初,為燕京路,總管大興府。中統五年,稱中都路。至元九年,改號大都。二十一年,始專置大都路總管府,秩從三品,置都達魯花赤、都總管等官。二十七年,升為都總管府,進秩正三品,領府一、州十有一。凡本府官吏,唯達魯花赤一員及總管、推官專治路政,其餘皆分任供需之事,故又號曰供需府焉。其屬附見:



 大都路兵馬都指揮使司,凡二,秩正四品,掌京城盜賊奸偽鞫捕之事,都指揮使二員,副指揮使五員,知事一員,提控案牘一員,吏十四人。至元九年,改千戶所為兵馬司,隸大都路。而刑部尚書一員提調司事,凡刑名則隸宗正,且為宗正之屬。二十九年,置都指揮使等官,其後因之。一置司於北城,一置司於南城。



 司獄司,凡三,秩正八品,司獄一員,獄丞一員,獄典二人,掌囚系獄具之事。一置於大都路,一置於北城兵馬司,通領南城兵馬司獄事。皇慶元年,以兩司異禁,遂分置一司於南城。



 左、右警巡二院,秩正六品,達魯花赤各一員,使各一員,副使、判官各三員,典史各三人,司吏各二十五人。至元六年置。領民事及供需,視大都路。大德五年,分置供需院,以副使、判官、典史各一員主之。



 大都警巡院,品職分置如左、右院,達魯花赤一員,使一員,副使二員,判官二員,典史二員,司吏二十人。大德九年置,以治都城之南。



 大都路提舉學校所,秩正六品。提舉一員,教授二員,學正二員,學錄一員。至元二十四年,既立國學,以故孔子廟為京學,而提舉學事者,仍以國子祭酒系銜。



 管領諸路打捕鷹房總管府,秩正三品,達魯花赤一員,總管一員,副達魯花赤一員,同知一員,副總管一員,經歷、知事各一員。至元十七年置。



 宛平縣,秩正六品,達魯花赤一員,尹一員,丞三員,主簿三員,尉一員,典史三員,司吏二十六人。至元十一年置,治大都麗正門以西。



 大興縣,秩正六品,達魯花赤一員,尹一員,丞一員,主簿二員,尉一員,典史三員,司吏一十五人。至元十一年置,治大都麗正門以東。



 東關廂巡檢司,秩從九品,巡檢三員,司吏一人,掌巡捕盜賊奸宄之事。至元二十一年置。



 西北、南關廂兩巡檢司,設置並同上。



\end{pinyinscope}