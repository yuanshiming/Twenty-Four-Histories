\article{志第三十七 百官三}

\begin{pinyinscope}

 大宗正府,秩從一品。國初未有官制,首置斷事官,曰札魯忽赤,會決庶務。凡諸王駙馬投下蒙古、色目人等,應犯一切公事,及漢人奸盜詐偽、蠱毒厭魅、誘掠逃驅、輕重罪囚,及邊遠出征官吏、每歲從駕分司上都存留住冬諸事,悉掌之。至元二年,置十員。三年,置八員。九年,降從一品銀印,止理蒙古公事。以諸王為府長,餘悉御位下及諸王之有國封者。又有怯薛人員,奉旨署事,別無頒受宣命。十四年,置十四員。十五年,置十三員。二十一年,置二十一員。二十二年,增至三十四員。二十八年,增至四十六員。大德四年,省五員。十一年,四十一員。皇慶元年,省二員,以漢人刑名歸刑部。泰定元年,復命兼理,置札魯忽赤四十二員,令史改為掾史。致和元年,以上都、大都所屬蒙古人並怯薛軍站色目與漢人相犯者,歸宗正府處斷,其餘路府州縣漢人、蒙古、色目詞訟,悉歸有司刑部掌管。正官札魯忽赤四十二員,從一品;郎中二員,從五品;員外郎二員,從六品;都事二員,從七品;承發架閣庫管勾一員,從八品;掾史十人,蒙古必闍赤十三人,通事、知印各三人,宣使十人,蒙古書寫一人,典吏三人,庫子一人,醫人一人,司獄二員。



 大司農司,秩正二品,凡農桑、水利、學校、饑荒之事,悉掌之。至元七年始立,置官五員。十四年罷,以按察司兼領勸農事。十八年,改立農政院,置官六員。二十年,又改立務農司,秩從三品,置達魯花赤一員、務農使一員、同知二員。是年,又改司農寺,達魯花赤一員,司農卿二員,司丞一員。二十三年,仍為大司農司,秩仍正二品。大德元年,增領大司農事一員。皇慶二年,升從一品,增大司農一員。定置大司農四員,從一品;大司農卿二員,正二品;少卿二員,從二品;大司農丞二員,從三品;經歷一員,從五品;都事二員,從七品;架閣庫管勾一員,照磨一員,並正八品;掾史十二人,蒙古必闍赤二人,回回掾史一人,知印二人,通事一人,宣使一人,典吏五人。



 籍田署,秩從六品,掌耕種籍田,以奉宗廟祭祀。至元七年始立,隸大司農。十四年,罷司農,隸太常寺。二十三年,復立大司農司,仍隸焉。署令一員,從六品;署丞一員,從七品;司吏一人。



 供膳司,秩從五品。掌供給應需,貨買百色生料,並桑哥籍入貲產。至元二十二年始置,隸司農。置達魯花赤一員,提點一員,並從五品;司令一員,正六品;丞一員,正七品;吏一人。



 輔用庫,秩正九品。掌規運息錢,以給供需。大使一員,副使一員。



 興中州等處油戶提領所,秩從九品。提領一員,大使一員,副使一員。歲辦油十萬斤,以供內庖。至元二十九年始置。



 蔚州面戶提領所,提領一員,副使一員。掌辦白面蔥菜,以給應辦,歲計十餘萬斤。



 右屬供膳。



 永平屯田總管府,秩從三品。達魯花赤一員,總管一員,同知一員,知事一員,司吏四人。至元二十四年,始立於永平路南馬城縣,以北京採木三千人隸之。所轄昌國、濟民、豐贍三署,各置署令一員、署丞一員、直長一人、吏目二人、吏二人。



 翰林兼國史院,秩正二品。中統初,以王鶚為翰林學士承旨,未立官署。至元元年始置,秩正三品。六年,置承旨三員、學士二員、侍讀學士二員、侍講學士二員、直學士二員。八年,升從二品。十四年,增承旨一員。十六年,增侍讀學士一員。十七年,增承旨二員。二十年,省並集賢院為翰林國史集賢院。二十一年,增學士二員。二十二年,復分立集賢院。二十三年,增侍講學士一員。二十六年,置官吏五員,掌管教習亦思替非文字。二十七年,增承旨一員。大德九年,升正二品,改典簿為司直,置都事一員。至大元年,置承旨九員。皇慶元年,升從一品,改司直為經歷。延祐元年,別置回回國子監學,以掌亦思替非官屬歸之。五年,置承旨八員。後定置承旨六員,從一品;學士二員,正二品;侍讀學士二員,從二品;侍講學士二員,從二品;直學士二員,從三品。屬官:待制五員,正五品;修撰三員,從六品;應奉翰林文字五員,從七品;編修官十員,正八品;檢閱四員,正八品;典籍二員,正八品;經歷一員,從五品;都事一員,從七品;掾史四人,譯史、通事、知印各二人,蒙古書寫五人,書寫十人,接手書寫十人,典吏三人,典書二人。



 蒙古翰林院,秩從二品,掌譯寫一切文字,及頒降璽書,並用蒙古新字,仍各以其國字副之。至元八年,始立新字學士於國史院。十二年,別立翰林院,置承旨一員、直學士一員、待制二員、修撰一員、應奉四員、寫聖旨必闍赤十有一人、令史一人、知印一人。十八年,增承旨一員、學士三員,省漢兒令史,置蒙古必闍赤四人。二十九年,增承旨一員、侍讀學士一員、知印一人。三十年,增管勾一員。大德五年,升正二品。九年,置司直一員、都事一員。皇慶元年,改升從一品,設官二十有八,吏屬二十有四。延祐二年,改司直為經歷。後定置承旨七員、學士二員、侍讀學士二員、侍講學士二員、直學士二員、待制四員、修撰二員、應奉五員、經歷一員、都事一員,品秩並同翰林國史院。承發架閣庫管勾一員,正九品;必闍赤一十四人,掾史三人,通事一人,譯史一人,知印二人,書寫一人,典吏三人。



 蒙古國子監,秩從三品。至元十四年始立,置司業一員。二十九年,準漢人國學例,置祭酒、司業、監丞。延祐四年,升正三品。七年,復降為從三品。後定置祭酒一員,從三品;司業二員,正五品;監丞一員,正六品;令史一人,必闍赤一人,知印一人。



 蒙古國子學,秩正七品,博士二員,助教二員,教授二員,學正、學錄各二員,掌教習諸生。於隨朝百官、怯薛臺、蒙古、漢兒官員家,選子弟俊秀者入學。至元八年,置官五員。後以每歲從駕上都,教習事繁,設官員少,增學正二員、學錄二員。三十一年,增助教一員、典給一人。後定置博士二員,正七品;助教二員,教授二員,並正八品;學正、學錄各二員,典書一人,典給一人。



 內八府宰相,掌諸王朝覲儐介之事。遇有詔令,則與蒙古翰林院官同譯寫而潤色之。謂之宰相云者,其貴似侍中,其近似門下,故特寵之以是名。雖有是名,而無授受宣命,品秩則視二品焉。大德九年,以滅怯禿等八人為之。天歷元年,為內八府宰之職,故附見於此云。



 集賢院,秩從二品,掌提調學校、徵求隱逸、召集賢良,凡國子監、玄門道教、陰陽祭祀、占卜祭遁之事,悉隸焉。國初,集賢與翰林國史院同一官署。至元二十二年,分置兩院,置大學士三員、學士一員、直學士二員、典簿一員、吏屬七人。二十四年,增置學士一員、侍讀學士一員、待制一員。尋升正二品,置院使一員,正二品;大學士二員,從二品;學士三員,從二品;侍讀學士一員,從三品;侍講學士一員,從三品;直學士二員,從四品;司直一員,從五品;待制一員,正五品。二十五年,增都事一員,從七品;修撰一員,從六品。元貞元年,增院使一員。大德十一年,升從一品,置院使六員、經歷二員。至大四年,省院使六員。皇慶二年,省漢人經歷一員。後定置大學士五員,從一品;學士二員,正二品;侍讀學士二員,侍講學士二員,並從二品;直學士二員,從三品;經歷一員,從五品;都事二員,從七品;待制一員,正五品;修撰一員,從六品;兼管勾承發架閣庫一員,正八品;掾史六人,譯史、知印各二人,通事一人,宣使七人,典吏三人。



 國子監。至元初,以許衡為集賢館大學士、國子祭酒,教國子與蒙古大姓四怯薛人員。選七品以上朝官子孫為國子生,隨朝三品以上官得舉凡民之俊秀者入學,為陪堂生伴讀。至元二十四年,始置監祭酒一員,從三品,司業二員,正五品,掌國之教令,皆德尊望重者為之。監丞一員,正六品,專領監務。典簿一員,令史二人,譯史、知印、典吏各一人。



 國子學,秩正七品。置博士二員,掌教授生徒、考較儒人著述、教官所業文字。助教四員,分教各齋生員。大德八年,為分職上都,增置助教二員、學正二員、學錄二員,督習課業。典給一員,掌生員膳食。至元二十四年,定置生員額二百人、伴讀二十人。至大四年,生員三百人。延祐二年,增置生員一百人、伴讀二十人。



 興文署,秩從六品。署令一員,以翰林修撰兼之。署丞一員,以翰林應奉兼之。至治二年罷,置典簿一員,從七品,掌提調諸生飲膳,與凡文牘簿書之事。仍置典吏一人。



 宣政院,秩從一品,掌釋教僧徒及吐蕃之境而隸治之。遇吐蕃有事,則為分院往鎮,亦別有印。如大征伐,則會樞府議。其用人則自為選。其為選則軍民通攝,僧俗並用。至元初,立總制院,而領以國師。二十五年,因唐制吐蕃來朝見於宣政殿之故,更名宣政院。置院使二員、同知二員、副使二員、參議二員、經歷二員、都事四員、管勾一員、照磨一員。二十六年,置斷事官四員。二十八年,增僉院、同僉各一員。元貞元年,增院判一員。大德四年,罷斷事官。至大初,省院使一員。至治三年,置院使六員。天歷二年,罷功德使司歸宣政,定置院使一十員,從一品;同知二員,正二品;副使二員,從二品;僉院二員,正三品;同僉三員,正四品;院判三員,正五品;參議二員,正五品;經歷二員,從五品;都事三員,從七品;照磨一員,管勾一員,並正八品;掾史十五人,蒙古必闍赤二人,回回掾史二人,怯里馬赤四人,知印二人,宣使十五人,典吏有差。



 斷事官四員,從三品,經歷、知事各一員,令史五人,知印、奏差、譯史、通事各一人。至元二十五年始置。



 客省使,秩從五品,大使二員,副使一員。至元二十五年置。



 大都規運提點所,秩正四品,達魯花赤一員,提點一員,大使一員,副使一員。至元二十八年置。



 上都規運提點所,秩正四品,達魯花赤一員,提點一員,大使一員,副使一員,知事一員。至元二十八年置。



 大都提舉資善庫,秩從五品,達魯花赤一員,提舉一員,同提舉一員,副提舉一員,掌錢帛之事。至元二十六年置。



 上都利貞庫,秩從七品,提領一員,副使一員,掌飲膳好事金銀諸物。元貞元年置。



 大濟倉,監支納一員,大使一員。



 興教寺,管房提領一員。



 吐蕃等處宣慰司都元帥府,秩從二品,宣慰使五員,經歷二員,都事二員,照磨一員,捕盜官二員,儒學教授一員,鎮撫二員。其屬二:



 脫思麻路軍民萬戶府,秩正三品,達魯花赤一員,萬戶一員,副達魯花赤一員,副萬戶一員,經歷一員,知事一員,鎮撫一員。



 西夏中興河州等處軍民總管府,秩正三品,達魯花赤一員,總管一員,同知一員,治中一員,府判一員,經歷一員,知事一員。屬官:稅務提領,寧河縣官,寧河脫脫禾孫五員,寧河弓甲匠達魯花赤。



 洮州元帥府,秩從三品,達魯花赤一員,元帥二員,知事一員。



 十八族元帥府,秩從三品,達魯花赤一員,元帥一員,同知一員,知事一員。



 積石州元帥府,達魯花赤一員,元帥一員,同知一員,知事一員,脫脫禾孫一員。



 禮店文州蒙古漢軍西番軍民元帥府,秩正三品,達魯花赤一員,元帥一員,同知一員,經歷、知事各一員,鎮撫二員,蒙古奧魯官一員,蒙古奧魯相副官一員。



 禮店文州蒙古漢軍奧魯軍民千戶所,秩從五品,達魯花赤一員,千戶一員,副千戶一員,總把五員,百戶八員。



 禮店文州蒙古漢軍西番軍民上千戶所,秩正四品,達魯花赤一員,千戶一員,百戶一員,新附千戶二員。



 禮店階州西水蒙古漢軍西番軍民總把二員。



 吐蕃等處招討使司,秩正三品,招討使二員,知事一員,鎮撫一員。其屬附:



 脫思麻探馬軍四萬戶府,秩正三品,萬戶五員,千戶八員,經歷一員,鎮撫一員。



 脫思麻路新附軍千戶所,秩從五品,達魯花赤一員,千戶一員,副千戶一員。



 文扶州西路南路底牙等處萬戶府,秩從三品,達魯花赤一員,萬戶二員。



 鳳翔等處千戶所,秩從五品,達魯花赤一員,千戶一員,百戶二員。



 慶陽寧環等處管軍總把一員。



 文州課程倉糧官一員。



 岷州十八族周回捕盜官二員。



 常陽貼城阿不籠等處萬戶府,秩從三品,達魯花赤一員,千戶一員。



 階文扶州等處番漢軍上千戶所,秩正五品,達魯花赤一員,千戶二員。



 貴德州,達魯花赤、知州各一員,同知、州判各一員,脫脫禾孫一員,捕盜官一員。



 必呈萬戶府,達魯花赤二員,萬戶四員。



 松潘宕疊威茂州等處軍民安撫使司,秩正三品,達魯花赤一員,安撫使一員,同知一員,僉事一員,經歷、知事、照磨各一員,鎮撫一員。威州保寧縣,茂州汶山縣、汶川縣皆隸焉。



 靜州茶上必里溪安鄉等二十六族軍民千戶所,達魯花赤一員,千戶一員。



 龍木頭都留等二十二族軍民千戶所,達魯花赤一員,千戶一員。



 岳希蓬蘿卜村等處二十二族軍民千戶所,達魯花赤一員,千戶一員。



 折藏萬戶府,達魯花赤一員,萬戶一員。



 吐蕃等路宣慰使司都元帥府,宣慰使四員,同知二員,副使一員,經歷、都事各二員,捕盜官三員,鎮撫二員。



 朵甘思田地裡管軍民都元帥府,都元帥一員,經歷一員,鎮撫一員。



 剌馬兒剛等處招討使司,達魯花赤一員,招討使一員,經歷一員。



 奔不田地裡招討使司,招討使一員,經歷一員,鎮撫一員。



 奔不兒亦思剛百姓,達魯花赤一員。



 碉門魚通黎雅長河西寧遠等處軍民安撫使司,秩正三品,達魯花赤一員,安撫使一員,同知一員,副使一員,僉事一員,經歷、知事、照磨各一員,鎮撫二員。



 六番招討使司,達魯花赤一員,招討使一員,經歷一員,知事一員。雅州嚴道縣、名山縣隸之。



 天全招討使司,達魯花赤一員,招討一員,經歷、知事各一員。



 魚通路萬戶府,達魯花赤一員,萬戶一員,經歷、知事各一員。黎州隸之。



 碉門魚通等處管軍守鎮萬戶府,達魯花赤一員,萬戶二員,經歷、知事各一員,鎮撫二員,千戶八員,百戶二十員,彈壓四員。



 長河西管軍萬戶府,達魯花赤一員,萬戶二員。



 長河西里管軍招討使司,招討使二員,經歷一員。



 朵甘思招討使一員。



 朵甘思哈答李唐魚通等處錢糧總管府,達魯花赤一員,總管一員,副總管一員,答剌答脫脫禾孫一員,哈裏脫脫禾孫一員,朵甘思甕吉剌滅吉思千戶一員。



 亦思馬兒甘萬戶府,達魯花赤一員,萬戶二員。



 烏思藏納里速古魯孫等三路宣慰使司都元帥府,宣慰使五員,同知二員,副使一員,經歷一員,鎮撫一員,捕盜司官一員。其屬附見:



 納里速古兒孫元帥二員。



 烏思藏管蒙古軍都元帥二員。



 擔裡管軍招討使一員。



 烏思藏等處轉運一員。



 沙魯思地裡管民萬戶一員。



 搽里八田地裡管民萬戶一員。



 烏思藏田地裡管民萬戶一員。



 速兒麻加瓦田地裡管民官一員。



 撒剌田地裡管民官一員。



 出蜜萬戶一員。



 禋籠答剌萬戶一員。



 思答籠剌萬戶一員。



 伯木古魯萬戶一員。



 湯卜赤八千戶四員。



 加麻瓦萬戶一員。



 札由瓦萬戶一員。



 牙里不藏思八萬戶府,達魯花赤一員,萬戶一員,千戶一員,擔裏脫脫禾孫一員。



 迷兒軍萬戶府,達魯花赤一員,萬戶一員,初厚江八千戶一員,卜兒八官一員。



 宣徽院,秩正三品,掌供玉食。凡稻粱牲牢酒醴蔬果庶品之物,燕享宗戚賓客之事,及諸王宿衛、怯憐口糧食,蒙古萬戶、千戶合納差發,系官抽分,牧養孳畜,歲支芻草粟菽,羊馬價直,收受闌遺等事,與尚食、尚藥、尚醖三局,皆隸焉。所轄內外司屬,用人則自為選。至元十五年置院使一員,同知、同僉各二員,主事二員,照磨一員。二十年,升從二品,增院使一員,置經歷二員、典簿三員。二十三年,升正二品,置院判二員,省典簿,置都事三員。三十一年,院使四員。大德二年,增同知二員。三年,升從一品。四年,置副使二員。皇慶元年,增院使三員,始定怯薛丹一萬人,本院掌其給授。後定置院使六員,從一品;同知二員,正二品;副使二員,從二品;僉院二員,正三品;同僉二員,正四品;院判二員,正五品;經歷二員,從五品;都事三員,從七品;照磨一員,承發架閣庫一員,並正八品;掾史二十人,蒙古必闍赤六人,回回掾史二人,怯里馬赤二人,知印二人,典吏六人,蒙古書寫二人。其屬附見:



 光祿寺,秩正三品,掌起運米曲諸事,領尚飲、尚醖局,沿路酒坊,各路布種事。至元十五年,罷都提點,置寺,設卿一員、少卿三員、主事一員、照磨一員、管勾一員。二十年,改尚醖監,正四品。二十三年,復為光祿寺,卿二員,少卿、丞各一員。二十四年,增少卿一員。二十五年,撥隸省部。三十一年,復隸宣徽。延祐七年,降從三品。後復正三品。定置卿四員,正三品;少卿二員,從四品;丞二員,從五品;主事二員,從七品;令史八人,譯史、知印各二人,通事一人,奏差二十四人,典吏三人,蒙古書寫一人。



 大都尚飲局,秩從六品。中統四年始置,設大使、副使各一員,俱帶金符,掌醖造上用細酒。至元十二年,增副使二員。十五年,升從五品,置提點一員。後定置提點一員,從五品;大使一員,正六品;副使二員,正七品。



 上都尚飲局,秩正五品。皇慶中始置,提點一員,大使、副使各一員,品秩同上。



 大都尚醖局,秩從六品,掌醖造諸王百官酒醴。中統四年,立御酒庫,設金符宣差。至元十一年,始設提點。十六年,改尚醖局,從五品。置提點一員,從五品;大使一員,正六品;副使二員,正七品;直長一員,正八品。



 上都尚醖局,秩從五品。至元二十九年始置,設提點一員,大使一員,副使、直長各一員,品秩同上。



 大都醴源倉,秩從六品,掌受香莎蘇門等酒材糯米,鄉貢曲藥,以供上醖及歲賜諸王百官者。至元二十五年始置,設提舉一員,從六品;大使一員,從七品;副使一員,正八品。



 上都醴源倉,秩從九品,掌受大都轉輸米麴,並醖造車駕臨幸次舍供給之酒。至元二十五年始置,設大使一員,直長一員。



 尚珍署,秩從五品。掌收濟寧等處田土子粒,以供酒材。至元十三年始立。十五年,罷入有司。二十三年復置。設達魯花赤一員,令一員,並從五品;丞二員,正七品;吏目二員。



 安豐懷遠等處稻田提領所,秩從九品,掌稻田布種,歲收子粒,轉輸醴源倉。定置提領二員。



 尚舍寺,秩正四品,掌行在帷幕帳房陳設之事,牧養駱駝,供進愛蘭乳酪。至元三十一年始置監。至大元年,改為寺,升正三品。四年,仍為監,尋復為寺。延祐三年,復降為正四品。定置太監二員、少監二員、監丞二員、知事二員。



 諸物庫,秩從七品,掌出納。大德四年置,設提領一員、大使一員、副使一員。



 闌遺監,秩正四品,掌不闌奚人口、頭匹諸物。至元二十年,初立闌遺所,秩九品。二十五年,改為監,正四品。二十八年,升正三品。至大四年,復正四品,尋復正三品。延祐七年,復為正四品。定置太監一員,正四品;少監二員,正五品;監丞二員,正六品;知事一員,從八品;提控案牘一員,從九品;令史五人,譯史一人,知印兼通事一人,奏差五人。



 尚食局,秩從五品,掌供御膳,及出納油面酥蜜諸物。至元二年置提點,領進納百色生料。二十年,省並尚藥局為尚食局,別置生料庫。本局定置提點一員,從五品;大使一員,正六品;副使二員,正七品;直長一員,正八品。



 大都生料庫,秩從五品。至元十一年,置生料野物庫,隸尚食局。二十年,別置庫,擬內藏庫例,置提點二員,從五品;大使二員,正六品;副使三員,正七品。



 上都生料庫,秩從五品,掌受弘州、大同虎賁、司農等歲辦油面,大都起運諸物,供奉內府,放支宮人宦者飲膳。提點一員,大使一員,副使二員,品秩同上;直長一員,正八品。



 大都大倉、上都大倉,秩正六品,掌內府支持米豆,及酒材米曲藥物。至元五年初立,設官三員,俱受制國用使司劄付。十二年,改立提舉大倉,設官三員,隸宣徽。二十五年,升正六品。定置二倉各設提舉一員,正六品;大使一員,從六品;副使一員,從七品。



 大都、上都柴炭局各一,至元十二年置,秩從六品。十六年,改提舉司,升五品。大德八年,仍為局,降正七品。置達魯花赤各一員,正七品;大都大使一員,上都大使三員,各正七品;副使各二員,正八品;直長各一人,掌葦場;典吏各一人。



 尚牧所,秩從五品。至大四年始置,設提舉二員,從五品;同提舉一員,從六品;副提舉一員,從七品;吏目一員。



 沙糖局,秩從五品,掌沙糖、蜂蜜煎造,及方貢果木。至元十三年始置,秩從六品。十七年,置提點一員。十九年,升從五品,置達魯花赤一員,從五品;提點一員,從五品;大使一員,正六品;副使一員,正七品。



 永備倉,秩從五品。至元十四年始置,給從九品印,掌受兩都倉庫起運省部計置油面諸物,及雲需府所辦羊物,以備車駕行幸膳羞。二十四年,升從五品,置提點一員,從五品;大使一員,正六品;副使一員,正七品。



 豐儲倉,秩從九品,大使一員,掌出納車駕行幸支持膳羞。



 淮東淮西屯田打捕總管府,秩正三品,掌獻田歲入,以供內府,及湖泊山場漁獵,以供內膳。至元十四年,始立總管府,並管連海高郵河泊提舉司、沂州等處提舉司事。十六年,置揚州鷹房打捕達魯花赤總管府。二十二年,省並為淮東淮西屯田打捕總管府。二十五年,以兩淮新附手號軍千戶所隸本府,及分置提舉司一十處。定置達魯花赤一員,正三品;總管一員,正三品;同知一員,正五品;府判一員,正六品;經歷一員,從七品;知事一員,從八品;提控案牘一員,從九品;司吏六人。



 淮安州屯田打捕提舉司,高郵屯田打捕提舉司,招泗屯田打捕提舉司,安東海州屯田打捕提舉司,揚州通泰屯田打捕提舉司,安豐廬州等處打捕提舉司,鎮巢等處打捕提舉司,塔山徐邳沂州等處山場屯田提舉司,凡九處,秩俱從五品。每司各設達魯花赤一員,提舉一員,並從五品;同提舉一員,從六品;副提舉一員,從七品;吏目二人。



 抽分場提領所,凡十處:曰柴墟東西口,曰海州新壩,曰北砂太倉,曰安河桃源,曰大湖東西口,曰時堡興化,曰高郵寶應,曰汶湖等處,曰雲山白水,曰安東州。每所各設提領一員、同提領一員、副提領一員,俱受宣徽院劄付。



 滿浦倉,秩正八品,掌收受各處子粒米面等物,以待轉輸京師。至元二十五年始置,設大使一員,正八品;副使一員,正九品。



 圓米棋子局、軟皮局,各置提領一員、同提領一員、副提領一員,俱受宣徽院劄付。



 手號軍人打捕千戶所,秩從四品,管軍人打捕野物皮貨。至元二十五年始置,設達魯花赤一員、上千戶一員、上副千戶一員、彈壓一員。



 上百戶七所,各置百戶二員。



 鐘離縣,定遠縣,真揚州,安慶,安豐,招泗,和州。



 下百戶二所,各置百戶一員。



 璉海,懷遠軍。



 龍慶栽種提舉司,秩從五品,管領縉山歲輸粱米,並易州、龍門、凈邊官園瓜果桃梨等物,以奉上供。至元十七年,始置提舉司。延祐七年,縉山改為龍慶州,因以名之。定置達魯花赤一員,提舉一員,並從五品;同提舉一員,從六品;副提舉一員,從七品。



 弘州種田提舉司,秩正六品,掌輸納麥面之事,以供內府。定置達魯花赤一員,提舉一員,並正六品;同提舉一員,正七品;副提舉一員,正八品;直長一員。



 豐閏署,秩從五品,掌歲入芻粟,以給飼養駝馬之事。定置達魯花赤一員,令一員,並從五品;丞一員,從六品;直長一員,正八品。



 常湖等處茶園都提舉司,秩正四品,掌常、湖二路茶園戶二萬三千有奇,採摘茶芽,以貢內府。至元十三年置司,統提領所凡十有三處。十六年,升都提舉司。又別置平江等處榷茶提舉司,掌歲貢御茶。二十四年,罷平江提舉司,並掌其職。定置達魯花赤一員,提舉一員,俱從五品;同提舉一員,從六品;副提舉一員,從七品;提控案牘一員,都目一員。



 提領所七處,每所各設正、同、副提領各一員,俱受宣徽院劄付,掌九品印。



 烏程,武康德清,長興,安吉,歸安,湖汶,宜興。



 建寧北苑武夷茶場提領所,提領一員,受宣徽院劄,掌歲貢茶芽。直隸宣徽。



 太禧宗禋院,秩從一品,掌神御殿朔望歲時諱忌日辰禋享禮典。天歷元年,罷會福、殊祥二院,改置太禧院以總制之。初,院官秩正二品,升從一品,置參議二員,改令史為掾史。二年,改太禧宗禋院,置院使六員,增副使二員,立諸總管府為之屬。凡錢糧之出納,營繕之作輟,悉統之。定置院使都典制神御殿事六員,同知兼佐儀神御殿事二員,副使兼奉贊神御殿事二員,僉院兼祗承神御殿事二員,同僉兼肅治神御殿事二員,院判供應神御殿事二員,參議二員,經歷二員,都事二員,管勾、照磨各一員,掾史二十人,譯史四人,知印二人,怯里馬赤二人,宣使一十五人,斷事官四員,客省使大使、副使各二員。



 隆禧總管府,秩正三品。至大元年,建立南鎮國寺,初立規運提點所。二年,改為規運都總管府。三年,升為隆禧院。天歷元年,罷會福、殊祥二院,以隆禧、殊祥並立殊祥總管府,尋又改為隆禧總管府。定置達魯花赤一員,總管一員,副達魯花赤一員,同知一員,治中一員,判官一員,經歷一員,知事、照磨各一員,令史六人,譯史、知印各一人,怯里馬赤一人,奏差四人。



 福元營繕司,秩正五品。達魯花赤一員,司令一員,大使一員,副使一員,吏目一人,司吏一人。天歷元年,以南鎮國寺所立怯憐口事產提舉司,改為崇恩福元提點所。三年,又改為福元營繕司。



 普安智全營繕司,秩五品。達魯花赤一員,司令一員,大使、副使各一員,吏目一人,司吏一人。天歷元年,以太玉山普安寺、大智全寺兩規運提點所並為一,置提點二員。三年,又改為營繕司。



 祐國營繕都司,秩正四品。達魯花赤一員,司令一員,大使、副使各一員,知事一員,提控案牘一員。天歷元年,初置萬聖祐國營繕提點所。三年,改為營繕都司。



 平松等處福元田賦提舉司,秩五品。置達魯花赤一員,提舉一員,同提舉、副提舉各一員。



 田賦提舉司,秩五品。置提舉一員、同提舉一員、副提舉一員。



 資用庫,提領一員,大使一員。



 萬聖庫,提領一員,大使一員,副使一員。



 會福總管府,秩正三品。至元十一年,建大護國仁王寺及昭應宮,始置財用規運所,秩正四品。十六年,改規運所為總管府。至大元年,改都總管府,從二品。尋升會福院,置院使五員。延祐三年,升正二品。天歷元年,改為會福總管府,正三品。定置達魯花赤一員,總管一員,同知一員,治中一員,府判一員,經歷、知事、提控案牘各一員,令史八人,譯史、通事、知印各一人,奏差四人。



 仁王營繕司,正五品。至元八年,立護國仁王寺鎮遏提舉司。十九年,改鎮遏所。二十八年,並三提領所為諸色人匠提領所。天歷元年,改為鎮遏民匠提領所。三年,改為仁王營繕司。置達魯花赤一員,司令一員,大使一員,副使一員。



 襄陽營田提舉司,秩從五品。初置襄陽等處水陸地土人戶提領所,設官四員。大德元年,改提舉司。天歷二年,仍為襄陽營田提舉司。定置達魯花赤一員,提舉一員,同提舉一員、副提舉一員。



 江淮等處營田提舉司,秩從五品。至元二十七年始置。達魯花赤一員,提舉一員,同提舉一員,副提舉一員。



 大都等路民佃提領所,至元二十九年,以武清等一十處,並立大都水陸地土種田人民提領所。十五年,又設隨路管民都提領所。天歷元年,並為大都等路民佃提領所。定置提領一員,大使、副使各一員。



 會福財用所,秩從七品。掌大護國仁王寺糧草諸物。至元十七年,始立財用庫。二十六年,立盈益倉。天歷元年,並財用、盈益為所。提領一員,大使一員,副使二員。



 崇祥總管府,秩正三品。至大元年,立大承華普慶寺都總管府。二年,改延禧監,尋改崇祥監。四年,升為崇祥院,秩正二品。泰定四年,復改為大承華普慶寺總管府。天歷元年,改為崇祥總管府。定置達魯花赤一員,總管一員,副達魯花赤一員,同知、治中、府判各一員,經歷、知事、提控案牘兼照磨各一員,令史六人,譯史、知印各一人,怯里馬赤一人,奏差四人。



 永福營繕司,秩正五品。延祐三年,以起建新寺,始置營繕提點所。天歷元年,改為永福營繕提點所。三年,改營繕司。設達魯花赤一員,司令一員,大使一員,副使一員,都目一員。



 昭孝營繕司,秩正五品。天歷元年,立壽安山規運提點所。三年,改昭孝營繕司。定置達魯花赤一員,司令一員,大使、副使各一員。



 普慶營繕司,秩正五品。天歷元年,始置普慶營繕提點所。三年,改為營繕司。定置達魯花赤一員,司令一員,大使、副使各一員。



 崇祥財用所,至大二年,始置諸物庫。四年,置普贍倉。天歷二年,並諸物庫、普贍倉,改為崇祥財用所。定置官,提領一員,大使、副使各一員。



 永福財用所,掌出納顏料諸物。延祐三年,始置諸物庫,又置永積倉。天歷二年,以諸物庫、永積倉並改置為所,設提領、大使、副使各一員。



 鎮江稻田提舉司,達魯花赤、提舉、同提舉、副提舉各一員。



 汴梁稻田提舉司,達魯花赤、提舉、同提舉、副提舉各一員。



 平江等處田賦提舉司,達魯花赤、提舉、同提舉、副提舉各一員。



 冀寧提領所,提領二員。



 隆祥使司,秩正三品。天歷二年,中宮建大承天護聖寺,立隆祥總管府,設官八員。至順二年,升為隆祥使司,秩從二品。置官:司使四員,同知、副使、司丞各二員,經歷一員,都事二員,照磨兼架閣一員,令史十人,譯史、通事、知印各二人,宣使十人,典吏六人。



 普明營繕都司,秩正四品。天歷元年,創大龍興普明寺於海南,置規運提點所,設官六員。二年,撥隸隆祥總管府。三年,改為都司,品秩仍舊,以掌營造出納錢糧之事。定置達魯花赤、司令、大使、副使各一員,知事一員,提控案牘一員。



 集慶萬壽營繕都司,秩正四品。天歷二年,建龍翔、萬壽兩寺於建康,立龍翔萬壽營繕提點所,為隆祥總管府屬。三年,改為營繕都司,秩仍舊,以掌營造錢糧之事。定置達魯花赤、司令、大使、副使各一員,知事、提控案牘各一員。



 元興營繕都司,秩正四品。掌營造錢糧之事。天歷元年,始置大元興規運提點所,置官五員。三年,改都司,置達魯花赤二員,司令、大使、副使各一員,知事、提控案牘各一員。



 宣農提舉司,秩從五品,達魯花赤、提舉、同提舉、副提舉各一員,掌徵收田賦子粒之事。天歷二年,以大都等處田賦提舉司隸隆祥總管府。三年,改提舉司。



 護聖營繕司,秩正五品,達魯花赤、司令、大使、副使各一員,掌營造工匠、寺僧衣糧、收征房課之事。天歷二年,始立大承天護聖營繕提點所。三年,改為司。



 平江善農提舉司,秩從五品,達魯花赤、提舉、同提舉、副提舉各一員,天歷二年,立田賦提舉司,設官四員。三年,改為善農提舉司。



 善盈庫,天歷二年,隸隆祥總管府,置提領一員,大使、副使各一員,掌金銀錢糧之事。



 荊襄等處濟農香戶提舉司,秩正五品。天歷三年,以荊襄提舉司所領河南、湖廣田土為大承天護聖寺常住,改為荊襄濟農香戶提舉司,隸隆祥總管府,置達魯花赤、司令、提舉、同提舉、副提舉各一員。



 龍慶州等處田賦提領所,秩九品,提領、副提領各一員。天歷二年置,掌龍慶州所有土田歲賦。



 平江集慶崇禧田賦提領所,提領、同提領、副提領各一員。天歷三年始置。



 集慶崇禧財用所,大使、副使各一員。天歷三年始置。



 壽福總管府,掌祭供錢糧之事,秩正三品。至大四年,因建大聖壽萬安寺,置萬安規運提點所,秩正五品。延祐二年,升都總管府,秩正三品。尋升為壽福院,正二品。天歷元年,改立總管府,仍正三品。定置官:達魯花赤、總管、副達魯花赤、同知、治中、府判各一員,經歷、知事、案牘照磨各一員,令史六人,知印、通事、譯史各一人,奏差四人,典吏二人。



 萬安營繕司,秩正五品。天歷三年,以萬安規運提點所既廢,復立萬安營繕司,定置達魯花赤、司令、大使、副使、都目各一人。



 萬寧營繕司,秩正四品。大德十年,始置萬寧規運提點所。天歷元年,改營繕司,定置達魯花赤、司令、大使、副使、都目各一員。



 收支庫,提領一員,大使一員。



 延聖營繕司,秩正五品。初立天源營繕提點所,天歷三年,改營繕司。定置達魯花赤、司令、大使、副使、都目各一員。



 諸物庫,提領一員,大使一員。



\end{pinyinscope}