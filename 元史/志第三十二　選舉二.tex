\article{志第三十二 選舉二}

\begin{pinyinscope}

 ○銓法上



 凡怯薛出身:元初用左右宿衛為心膂爪牙,故四怯薛子孫世為宿衛之長,使得自舉其屬。諸怯薛歲久被遇,常加顯擢,惟長官薦用,則有定制。至元二十年議:「久侍禁闥、門地崇高者,初受朝命散官,減職事一等,否則量減二等。」至大四年,詔蒙古人降一等,色目人降二等,漢人降三等。



 凡臺憲選用:大德元年,省議:「臺官舊無選法,俱於民職選取,後互相保選,省、臺各為一選。宜令臺官,幕官聽自選擇,惟廉訪司官,則省、臺共選。若臺官於省部選人,則與省官共議之;省官於臺憲選人,亦與臺官共議之。」至元八年,定監察御史任滿,在職無異政,元系七品以下者例加一等,六品以上者升擢。其有不顧權勢,彈劾非違,及利國便民者,別議升除。或有不稱者,斟酌銓注。



 凡選舉守令:至元八年,詔以戶口增、田野闢、詞訟簡、盜賊息、賦役均五事備者,為上選。九年,以五事備者為上選,升一等。四事備者,減一資。三事有成者為中選,依常例遷轉。四事不備者,添一資。五事俱不舉者,黜降一等。二十三年,詔:「勸課農桑,克勤奉職者,以次升獎。其怠於事者,笞罷之。」二十八年,詔:「路府州縣,除達魯花赤外,長官並宜選用漢人素有聲望,及勛臣故家,並儒吏出身,資品相應者,佐貳官遴選色目、漢人參用,庶期於政平訟理,民安盜息,而五事備矣。」



 凡進用武官:至元十五年,詔:「軍官有功而升職者,舊以其子弟襲職,陣亡者許令承襲,若罷去者,以有功者代之。」十七年,詔:「渡江總把、百戶有功升遷者,總把依千戶降等承襲,百戶無遞降職名,則從其本等。」十九年,奏擬:「萬戶、千戶、百戶物故,視其子孫堪承襲者,依例承襲外,都元帥、招討使、總管、總把,視其子孫堪承襲者,止令管其元軍。元帥、招討子孫為萬戶,總管子孫為千戶,總把子孫為百戶,給元佩金銀符。病故者降等,惟陣亡者本等承襲。」二十一年,詔:「萬戶、千戶、百戶分上中下三等,定立絳格,通行遷轉。以三年為滿,理算資考,升加品級。若年老病故者,令其子弟依例廕敘。」是年,以舊制父子相繼,管領元軍,不設蒙古軍官,故定立資考,三年為滿,通行遷轉。後各翼大小軍官俱設蒙古軍官,又兼調遣征進,俱已離翼,難與民官一體遷轉廕敘,合將萬戶、千戶、鎮撫自奏準日為始,以三年為滿,通行遷轉。百戶以下,不拘此例。凡軍官征戰有功過者,驗實跡升降。又定蒙古奧魯官,大翼萬戶下設奧魯總管府,從四品。小翼萬戶下設奧魯官,從五品。各千戶奧魯,亦設奧魯官,受院劄。各千戶奧魯,不及一千戶者,或二百戶、三百戶,以遠就近,以小就大,合並為千戶翼奧魯官,受院劄。若干礙投下,難以合並,宜再議之。又定首領官受敕牒,元帥、招討司經歷、知事,就充萬戶府經歷、知事,換降敕牒,如元翼該革,別與遷除。若王令旨、並行省劄付、樞密院劄付經歷,充中、下萬戶府知事。行省諸司劄付,充提領案牘,並各翼萬戶自設經歷、知事,一例俱作提控案牘,受院劄。又議:「隨朝各衛千戶鎮撫所提控案牘,已擬受院劄,外任千戶鎮撫所提控案牘,合從行省許準,受萬戶府付身。」二十四年,詔:「諸求襲其父兄之職者,宜察其人而用之。凡舊臣勛閥及有戰功者,其子弟當先任以小職,若果有能,則大用之。」二十五年,軍官陣亡者,本等承襲,病故者,降二等。雖陣亡,其子弟無能,勿用。雖病故,其子弟果能,不必降等,於本等用之。大德四年,以上都虎賁司並武衛內萬戶、千戶、百戶達魯花赤亡歿,而無奏準承襲定例,似為偏負。今後各翼達魯花赤亡歿,宜察其子弟有能者用之,無能則止。五年,詔:「軍官有不赴任者,有患病因事不行者,有已赴任、被差委而出、公事已辦為私事稱故不回者,今後宜限以六月。越限者以他人代之,期年後以他職授之。」十一年,詔:「色目鎮撫已歿,其子有能,依例用之。子幼,則取其兄弟之子有能者用之,俟其子長,即以其職還之。」至大二年,議:「各衛翼首領官,至經歷以上,不得升除,似與官軍一體,其子孫乃不得承襲。今後年逾七十,而散官至正從四品者,宜正從五品軍官內任用。」四年,詔:「軍官有故,令其嫡長子,亡歿,令嫡長孫為之。嫡長孫亡歿,則令嫡長孫之嫡長子為之。若嫡長俱無,則以其兄弟之子相應者為之。」



 太禧院。天歷元年,罷會福、殊祥二院而立之,秩正二品。其所轄諸司,則從其擢用。



 宣徽院。皇慶二年,省臣奏:「其所轄倉庫、屯田官員,半由都省,半由本院用之。」奉旨,宜俱從省臣用之。



 中政院。至大四年言:「諸司錢糧選法,悉令中書省掌之,可更選人任用,移文中書,給降宣敕。」延祐七年,院臣啟:「皇后位下中政院用人,奉懿旨,依樞密院、御史臺等例行之。」



 直省舍人,內則侍相臣之興居,外則傳省闥之命令,選宿衛及勛臣子弟為之。又擇其高等二人,專掌奏事。至元二十五年,省臣奏:「其充是職者,俾受宣命。」大德八年,擬歷六十月者,始令從政。



 凡禮儀諸職:有太常寺檢討,至元十三年,擬歷一百月,除從八品。有御史臺殿中司知班,十五年,擬歷九十月,除正八品。有通事舍人,二十年,議:「從本司選已入流品職官為之,考滿驗應得資品,升一等遷用。未入流官人員,擬充侍儀舍人,受中書省劄,一考除從九品。」三十年,議:「於二品、三品官子內選用,不限廕敘,兩考從七品遷敘。」有侍儀舍人,三十年,議:「於四品、五品官子內選用,不限廕敘,一考從九品。」大德三年,議:「有闕,宜令侍儀司於到部正從九品流官內選用,仍受省劄,三十月為滿,依朝官內升轉,如不敷,於應得府州儒學教授內選用,歷一考,正九品敘。」有禮直管勾,大德三年,省選合用到部人員,俱從太常寺舉保,非常選除充者,任回止於本衙門敘用。有郊壇庫藏都監二人,至大三年,議:「受省劄者歷一考之上,受部劄者歷兩考之上,再歷本院屬官一任,擬於從九品內敘。」天歷二年,擬在朝文翰衙門,於國子生員內舉充。



 至元九年,部議:「巡檢流外職任,擬三十月為一考,任回於從九品遷敘。」二十年,議:「巡檢六十月,升從九品。」大德七年,議:「各處所委巡檢,自立格月日為始,已歷兩考之上者,循舊例九十月出職;不及兩考者,須歷一百二十月,方許出職遷轉。」十年,省奏:「奉旨腹裏巡檢,任回及考者,止於巡檢內注授。所歷未及者,於錢穀官內定奪,通理巡檢月日。各處行省所設巡檢,考滿者,咨省定奪;未及考滿者,行省於錢穀官等職內委用,通理月日,依舊升轉;不及一考,如系告廕並提控案牘例應轉充者,於雜職內委用,考滿各理本等月日,依例升轉。」



 腹裏諸路行用鈔庫,至元十九年,部擬:「州縣民官內選充,系八品、九品人員,三十月為滿,任回驗元資品,減一資歷,通理遷敘。庫使,受都省劄付,任滿從優遷敘。庫副,受本路劄付,二十月為滿,於本處上戶內公選交替。陜西、四川、西夏中興等路提舉司鈔庫,俱系行省管領,合就令依上選擬庫官,移文都省,給降敕牒札付。」省議:「除鈔庫使副咨各省選擬外,提領省部選注。」腹裏官員,二十六年,定選充倉庫等官,擬於應得資品上升一等,通理月日升轉。江南官員,若曾腹里歷仕,前資相應依例升轉。遷去江淮歷仕人員,所歷月日一考之上者,除一考準為根腳,餘有月日,後任通理;不及考者,添一資。若選充倉庫等官,擬於應得資品上,例升一等,任回依上於腹裏升轉。接連官員選充倉庫等官,應本地面從七品者,準算腹裏從七資品。歷過一考者,為始理算月日,後任通理;一考之上,餘有月日,後任通理;不及考者,添一資升轉。福建、兩廣官員選充倉庫等官,應得本地面從七品者,準算江南從七資品。歷過一考者,為始理算月日;一考之上,餘有月日,後任通理;不及考者,添一資升轉。元系流官,任回,止於流官內任用;雜職者,雜職內遷敘。萬億庫、寶鈔總庫、八作司,以一年滿代,錢物甚多,未易交割,宜以二年為滿,少者以一年為滿。上都稅務官,止依上例遷轉。都省所轄去處,二周歲為滿者:各處都轉運使司官、司屬官、首領官,各處都漕運使司官、首領官,諸路寶鈔都提舉司官,腹裏、江南隨路平準行用庫官,印造寶鈔庫官,鐵冶提舉司官、首領官,採金提舉司官、首領官,銀場提舉司官、首領官,新舊運糧提舉司官、首領官,都提舉萬億庫、八作司、寶鈔總庫首領官。一周歲為滿者:泉府司所轄富藏庫官,廩給司、四賓庫、薄斂庫官,大都稅課提舉司官、首領官,酒課提舉司官、首領官,提舉太倉官、首領官,提舉醴源倉官、首領官,大都省倉官,河倉官,通州等處倉官、應受省部劄付管錢穀院務雜職等官,大都平準行用庫官,燒鈔四庫官,抄紙坊官,幣源庫官。行省所轄去處,二周歲為滿者:各處都轉運使司官、司屬官、首領官,各處都漕運使司官、首領官,行諸路寶鈔都提舉司官,腹裏、江南隨路平準行用庫官,甘州、寧夏府等處都轉運使司官,市舶提舉司官、首領官,榷茶提舉司官、首領官。一周歲為滿者:行泉府司所轄阜通庫官,各處行省收支錢帛諸物庫官。三十年,部議:「凡內外平準行用庫官,提領從七品,大使從八品,副使從九品。若流官內選充者,任回減一資升轉。雜職人員,止理本等月日。」元貞二年,部議:「凡倉官有闕,於到選相應職官,並諸衙門有出身令譯史、通事、知印、宣使、奏差兩考之上人內選用,依驗難易收糧多寡升等,任回於應去地方遷敘。通州、河西務、李二寺等倉官,於應德資品上升一等,任滿,交割別無短少,減一資通理。在都並城外倉分,收糧五萬石之上倉官,於應得資品上升一等,任滿,交割別無短少,依例遷敘;收糧一萬石之上倉官,止依應得品級除授,任滿,交割別無短少,減一資通理。」大德元年,省擬:「大都萬億四庫、富寧庫、寶鈔總庫、上都萬億庫官,止依合德資品選注,須二周歲滿日,別無短少,擬同隨朝例升一等。」二年,省議:「上都、應昌倉官,比同萬億庫官例,二周歲為滿,於應得資品上擬升一等。」六年,部議:「在都平準行用庫官,擬合與外路一體二周歲為滿,元系流官內選充者,任回減一資升轉。萬億四庫知事例升一等,提控案牘減資遷轉。和林、昔寶赤八剌哈孫、孔古烈倉改立從五品提舉司。提舉一員,從五品,同提舉一員,從六品,副提舉一員,從七品,周歲為滿,於到選人內選充,應得資品上擬升二等,任回遷用,所歷月日通理。甘、肅二路,每處設監支納一員,正六品,倉使一員,從六品,倉副一員,正七品,二周歲為滿,於到選人內銓注,入倉先升一等,任滿交割,別無短少,又升一等。受給庫提領,從九品,使、副受省劄,攢典、合干人各設二名。」七年,部擬:「大都路永豐庫提領從七,大使從八,副使從九,於到選相應人內銓注。江西省英德路、河西務兩處,設立平準行用庫,擬合設官員,系從七以下人員,依例銓注。英德路平準行用庫,提領一員,從七,大使一員,從八,副使一員,從九品。河西務行用庫,大使一員,從八品,副使一員,吏部劄。甘肅行省豐備庫,提領一員,從七品,大使一員,正八品,於到選迤西資品人內升等銓注。大同倉官,擬二周歲交代,永盈倉例升一等,其餘六倉,任回擬減一資升轉。」八年,部議:「湖廣行省所轄散府司吏充倉官,依河南行省散府司吏充倉官,比總管府司吏取充者,降等定奪。」至大二年,部呈:「凡平準行用庫設官二員,常平倉設官三員,於流官內銓注,以二年為滿,依例減資。」四年,部議:「上都兩倉,二周歲為滿,於應得資品上升一等,歷過月日,今後比例通理。」皇慶元年,部議:「上都平盈庫,二周歲為滿,減一資升轉。」延祐四年,部議:「江浙行省各路見役司吏,已及兩考,選充倉官,五萬石之上,比同考滿出身充典史,一考升吏目。五萬石之下者,於典史添一考,依例遷敘。湖廣行省倉官,如系路吏及兩考,選充倉官一界,同考滿出身充典史,一考升吏目,遷敘庫官,周歲準理本等月日,考滿依例升轉。」



 凡稅務官升轉:至元二十一年,省議:「應敘辦課官分三等:一百錠之上,設提領一員、使一員。五十錠之上,設務使一員。五十錠之下,設都監一員。十錠以下,從各路差人管辦。都監歷三界,升務使,一周歲為滿,月日不及者通理。務使歷三界,升提領。提領歷三界,受省劄錢穀官,再歷三界,始於資品錢穀官並雜職任用。各處就差相副官,增及兩酬者,聽各處官司再差。增及三酬以上及後界又增者,申部定奪。」二十九年,省判所辦諸課增虧分數,升降人員。增六分升二等,增三分升一等。其增不及分數,比全無增者,到選量與從優。虧兌一分,降一等。三十年,省擬:「提領二年為滿,省部於流官內銓注,一萬錠之上擬從六品,五千錠之上擬正七品,二千錠之上擬從七品,一千錠之上正八品,五百錠之上從八品。大使、副使俱周歲交代,大使從行省吏部於解由合敘相應人內遷調,副使從各路於本處系籍近上戶內公選。」至大三年,詔定立辦課例。一百錠之下院務官分為三等:五十錠之上為上等,設提領一員,受省劄,大使一員,受部劄;二十錠之上為中等,設大使、副使各一員;二十錠之下為下等,設都監、同監各一員,俱受部劄,並以一年為滿,齊界交代。都監、同監四界升副使,又四界升大使,又三界升提領,又三界入資品錢穀官並雜職內遷用。行省差設人員,各添兩界升轉,仍自立界以後為始,理算月日,並於有升轉出身人員內定奪,不許濫用白身。議得例前部劄,提領於大使內銓注,都監、同監本等擬注,止依歷一十二界。至大三年例後,創入錢穀人員,及正從六品七品取廕子孫,亦依先例升轉,不須添界外,其餘雜進之人,依今次定例遷用,通歷一十四界,依上例升轉。



 至元九年,部議:「凡總府續置提控案牘,多系入仕年深,似比巡檢例同考滿轉入從九。緣從九系銓注巡檢闕,提領案牘吏員文資出職,難應捕捉,兼從九員多闕少,本等人員不敷銓注。凡升轉資考,從九三任升從八,正九兩任升從八,巡檢提領案牘等考滿轉入從九,從九再歷三考升從八,通理一百二十月升。巡檢依已擬,提領案牘權擬六十月正九,再歷兩任,通理一百二十月升從九,較之升轉資考,即比巡檢庶員闕易就。都、吏目,擬吏目一考,轉充都目,一考,轉充提領案牘,考滿依上轉入流品。都、吏目應升無闕,止注本等職名,驗理升轉。」二十年,部擬:「提控案牘九十月升九品。」二十五年,部擬:「各路司吏實歷六十月,吏目兩考升都目,歷一考升提控案牘,兩考升正九。若依路司吏九十月,吏目歷一考與都目,餘皆依上升轉。」省議:「江南提控案牘,除各路司吏比附腹里路司吏至元二十五年呈準定例遷除,其餘已行直補,並自行踏逐歷案牘兩考者,再添資遷除。」三十年,省準:「提控案牘補注巡檢,升轉資品,不相爭懸,如已歷提控案牘月日者,任回止於提控案牘內遷敘。」三十一年,省議:「都目、巡檢員闕,雖不相就,若不從宜調用,似涉壅滯,下部先盡到選巡檢,餘闕準告銓注,任回各理本等月日。」大德二年,省準:「京城內外省倉典吏,例於大都路州司吏、縣典史內勾補,二周歲轉升吏目。除行省所轄外,腹裏下州並雜職等衙門,計設吏目一百餘處,其籍記未注者,以次銓注,俱擬三十月為滿,任回本等內不次銓注。」三年,部擬:「提控案牘、都吏目有三周歲、二周歲、一周歲為滿者,俱以三十月為滿。」八年,省準:「和林兵馬司掌管案牘人等,比依下州,合設吏目一員,於籍記吏目外發補,任回從九品遷用,添一資升轉。司吏量擬四名,從本司選補通吏業者,六十月,提控案牘內任用。」九年,部呈:「都、吏目已於典史內銓注,宜將籍記案牘驗歷仕,以遠就近,於吏目闕內參注,各理本等月日。」十一年,江浙省臣言:「各路提控案牘改受敕牒,不見通例。」部照:「江北提控案牘,皆自府州司縣轉充路吏,請俸九十月方得吏目,一考升都目,都目一考,升提控案牘,兩考正九品,通理二百一十月入流,其行省所委者,九十月與九品。今議行省委用例革提控案牘,合於散府諸州案牘、都吏目並雜職錢穀官內,行省依例銓注,通理月日升轉。之後行省所設提控案牘、都吏目,合依江北由司縣府州轉充路吏,通理月日,考滿方許入流。」



 凡選取宣使奏差:至元十九年,部擬:「六部奏差額設數目,每一十名內,令各部選取四名,九十月與從九品,餘外合設數目,俱於到部巡檢、提領案牘、都吏目內選取,候考滿日,驗下項資品銓注。」省準:「解由到部,關會完備人員內選取。應人吏目,選充奏差,三考與從九品。吏目一考應入都目人員,選充奏差,兩考與從九品。都目一考應入提領案牘人員,選充奏差,一考與從九品。巡檢、提領案牘一考,選充奏差,一考與正九品。」二十六年,省準:「上都留守司兼本路都總管府典吏出身,歷九十月,比通政院例,合轉補本司宣使,考滿依例定奪。」二十九年,省議:「行省、行院宣使於正從九品有解由職官內選取,如是不敷,於各道宣慰司一考之上奏差、本衙門三考典吏內選取。不敷,於各道廉訪司三考奏差內並本衙門三考典史內選取,仍須色目、漢人相參選取。自行踏逐者,亦須相應人員,考滿例降一等,須歷九十月,方許出職。內外諸衙門宣使,以色目、漢人相參,九十月為滿,自行踏逐者降一等。凡內外諸衙門宣使、通事、知印、奏差,都省宣使有闕,於臺、院等衙門一考之上宣使、並有解由正從八品職官內選補,如系都省直選人員,不拘此例,仍須色目、漢人相參選取。自行踏逐者,考滿例降一等,須歷九十月,方許出職。樞密院宣使,正從九品職官內選取,仍須色目、漢人相參選用。自行踏逐者,亦須相應人員,考滿例降一等,須歷九十月,方許出職。御史臺宣使,正從九品職官內選取。自行踏逐者,考滿例降一等,須歷九十月,方許出職。宣政院宣使,選補同。宣慰司奏差,於本衙門三考典吏內選取。自行踏逐者,考滿降等敘,須色目、漢人參用,歷九十月,方許出職。山東運司奏差,九十月,於近下錢穀官內任用。大都運司,一體定奪。」七年,省準:「鞏昌等處便宜都總帥府令史人等,已擬依各道宣慰司令史人等一體出身,自行踏逐者降等敘,有闕於本司三考典吏內選取。」八年,部呈:「各寺監保本處典吏補奏差,若元系請俸典吏、本把人等補充者,考滿同自行踏逐者,降等敘。」九年,擬宣徽院典吏九十月補宣使,並所轄寺監令史。十年,省擬:「中政院宣使於本衙門三考之上典吏及正從九品職官內選用,以色目、漢人相參,自行踏逐者降等。」十一年,省擬:「燕南廉訪司奏差,州吏內選補,考滿於都目內遷用。」延祐三年,省議:「各衙門典吏,須歷九十月,方許轉補奏差。」



 凡匠官:至元九年,工部驗各管戶數,二千戶之上至一百戶之上,隨路管匠官品級。省議:「除在都總提舉司去處,依準所擬。東平雜造提舉司並隨路織染提舉司,二千戶之上,提舉正五品,同提舉從六品,副提舉從七品。一千戶之上,提舉從五品,同提舉正七品,副提舉正八品。五百戶之上至一千戶之下,提舉正六品,同提舉從七品,副提舉從八品。三百戶之上,大使正七品,副使正八品。一百戶之上,大使從七品,副使從八品。一百戶之下,院長一員,同院務,例不入流品,量給食錢。凡一百戶之下管匠官資品,受上司劄付者,依已擬充院長。已受宣牌充局使者,比附一百戶之上局使資品遞降,量作正九資品。」二十二年,凡選取升轉匠官資格,元定品給員數,提舉司二千戶之上者,無之。一千戶之上,提舉從五品,同提舉正七品,副提舉正八品。五百戶之上、一千戶之下,提舉正六品,同提舉從七品,副提舉從八品。使副,三百戶之上,局使正七品,副使正八品。一百戶之上,局使從七品,副使從八品。一百戶之下,院長一員,比同務院,例不入流品。工部議:「三百戶之上局副從八,一百戶之上局副正九,遇有闕,於一百戶之下院長內選充。院長一百二十月升正九,正九兩考升從八,從八三考、正八兩考,俱升從七。如正八有闕,別無資品相應人員,於已授從八匠官內選注,通歷九十月,升從七;從七三考升正七,正七兩考升從六;從六三考、正六兩考,俱升從五。如所轄司屬無從六,名闕,如已歷正七兩考,擬升加從六散官,止於正七匠官內遷轉,九十月升從五。如正六匠官有闕,於已授從六散官人員內選注,通歷九十月升從五。從五三考擬升正五,別無正五匠官,名闕,升加正五散官,止於從五匠官內遷轉。如歷仕年深,至日斟酌定奪。至元十二年以前受宣敕省劄人員,依管民官例,擬準已受資品。十三年以後受宣敕省劄人員,若有超升越等者,驗實歷俸月定擬,合得資品上例存一等遷用。管匠官遇有闕員去處,如無資品相應之人,擬於雜職資品相應到選人內銓用。凡中原、江淮匠官,正從五品子從九品匠官內廕敘,六品、七品子於院長內敘用。以匠官無從九,名闕,擬正從五品子應廕者,於正九匠官內銓注,任回,理算從九月日。」二十三年,詔:「管匠官,其造作有好惡虧少,勿令遷轉。」二十四年,部言:「管匠衙門首領官,宜於本衙門內選委知會造作相應人員區用,勿令遷轉,合依舊例,從本部於常選內選差相應人員掌管案牘,任滿交代遷敘。」元貞元年,準湖廣行省所擬:「三千戶之上提舉司從五品,提舉從五品,同提舉正七品,副提舉正八品。二千戶之上提舉司正六品,提舉正六品,同提舉從七品,副提舉從八品。一千戶之上局,局使正七品,副使正八品。五百戶之上局,局使從七品,副使正九品。五百戶之下,院長一員。」



 凡諸王分地與所受湯沐邑,得自舉其人,以名聞朝廷,而後授其職。至元二年,詔以各投下總管府長官不遷外,其所屬州縣長官,於本投下分到城邑內遷轉。四年,省劄:「應給印官員,若受宣命及諸王令旨、或投下官員批劄、省府樞密院制府左右部劄付者,驗戶給印。」五年,詔:「凡投下官,必須有蒙古人員。」六年,以隨路見任並各投下創差達魯花赤內,多女直、契丹、漢人,除回回、畏吾兒、乃蠻、唐兀同蒙古例許敘用,其餘擬合革罷,曾歷仕者,於管民官內敘用。十九年,詔:「各投下長官,宜依例三年一次遷轉。」省臣奏:「江南諸王分地長官,已令如例遷轉,其間若有兼管軍鎮守為達魯花赤者,一體代之,似為不宜。合令於投下長官之上署字,一同蒞事。」二十年,議:「諸王各投下千戶,於江南分地已於長官內委用,其州縣長官,亦令如之,似為相宜。」二十三年,諸王、駙馬並百官保送人員,若曾仕者,驗資歷於州縣內相間用,如無歷仕,從本投下自用。三十年,各投下州縣長官,三年一次給由互相遷轉,如無可遷轉,依例給由申呈省部,仍牒廉訪司體訪。大德元年,諸投下達魯花赤從七以下者,依例類選。十年,議:「各投下官員,非奉省部明文,毋得擅自離職。」皇慶二年,詔:「各投下分地城邑長官,其常選所用者,居眾人之上,投下所委者為添設,其常選內路府州及各縣內減一員。」三年,以中下縣主簿、錄事司錄判掌錢糧捕盜等事,不宜減去,並增置副達魯花赤一員。四年,凡投下郡邑,令自置達魯花赤,其為副者罷之。各投下有闕用人,自於其投下內選用,不許冒用常選內人。



 凡壕寨官:至元十九年,省部擬:「都水監並入本部,其壕寨官比依各部奏差出身。」大德二年,擬考滿除從九品。



 凡入粟補官:天歷三年,河南、陜西等處民饑,省臣議:「江南、陜西、河南等處富實之家願納粟補官者,驗糧數等第,從納粟人運至被災處所,隨即出給勘合硃鈔,實授茶鹽流官,咨申省部除授。凡錢穀官隸行省者行省銓注,腹裏省者吏部注擬,考滿依例升轉。其願折納價鈔者,並以中統鈔為則。江南三省每石四十兩,陜西省每石八十兩,河南並腹裏每石六十兩。其實授茶鹽流官,如不願仕而讓封父母者聽。陜西省:一千五百石之上,從七品。一千石之上,正八品。五百石之上,從八品。三百石之上,正九品。二百石之上,從九品。一百石之上,上等錢穀官。八十石之上,中等錢穀官。五十石之上,下等錢穀官。三十石之上,旌表門閭。河南並腹裏:二千石之上,從七品。一千五百石之上,正八品。一千石之上,從八品。五百石之上,正九品。三百石之上,從九品。二百石之上,上等錢穀官。一百五十石之上,中等錢穀官。一百石之上,下等錢穀官。江南三省:一萬石之上,正七品。五千石之上,從七品。三千石之上,正八品。二千石之上,從八品。一千石之上,正九品。五百石之上,從九品。三百石之上,上等錢穀官。二百五十石之上,中等錢穀官。二百石之上,下等錢穀官。凡先嘗入粟遙授虛名者,今再入粟,則依驗糧數,照依資品,令實授茶鹽流官。陜西省:一千石之上,從七品。六百六十石之上,正八品。三百三十石之上,從八品。二百石之上,正九品。一百三十石之上,從九品。河南並腹裏:一千三百石之上,從七品。一千石之上,正八品。六百六十石之上,從八品。三百三十石之上,正九品。二百石之上,從九品。江南三省:六千六百六十石之上,正七品。三千三百三十石之上,從七品。二千石之上,正八品。一千三百三十石之上,從八品。六百六十石之上,正九品。三百三十石之上,從九品。先嘗入粟實授茶鹽流官者,今再入粟,則依驗糧數,加等升職。陜西省:七百五十石之上,五百石之上,二百五十石之上,一百五十石之上,一百石之上。河南並腹裏:一千石之上,七百五十石之上,五百石之上,二百五十石之上,一百五十石之上。僧道能以自己衣缽濟饑民者,三百石之上,六字師號,都省出給。二百石之上,四字師號;一百石之上,二字師號;俱禮部出給。四川省所轄地分富實民戶,有能入粟赴江陵者,依河南省入粟補官例行之。其糧合用之時,從長處置。江浙、江西、湖廣三省已糶官糧,見在價鈔於此差人赴河南省別與收貯,合用之時,從長處置。」



 凡獲盜賞官:大德五年,詔:「獲強盜五人,與一官。捕盜官及應捕人,本境失盜而獲他境盜者,聽功過相補。獲強盜過五人,捕盜官減一資,至十五人升一等,應捕人與一官,不在論賞之列。」



 凡控鶴傘子:至元二十二年,擬:「控鶴受省劄,保充御前傘子者,除充拱衛直都指揮使司鈐轄,官進義副尉。」二十八年,控鶴提控受敕進義副尉,管控鶴百戶,及一考,擬元除散官從八,職事正九,於從八內遷注。元貞元年,控鶴提控奉旨充速古兒赤一年,受省劄充御前傘子,歷三百三十二月,詔於從六品內遷用。大德六年,控鶴百戶,部議於巡檢內任用。其離役百戶人等擬從八品,傘子從七品。延祐三年,控鶴百戶歷兩考之上,擬於正九品遷用。



 凡玉典赤:至元二十七年,定擬歷三十月至九十月者,並與縣達魯花赤、進義副尉。一百月以上者,官敦武校尉。至大二年,令玉典赤權於州判、縣丞內銓注。三年,令依舊例,九十月除從七下縣達魯花赤,任回添一資。



 凡蠻夷官:議:「播州宣撫司保蠻夷地分副長官,系遠方蠻夷,不拘常調之職,合準所保。其蠻夷地分,雖不拘常調之處,而所保之人,多有泛濫。今後除襲替土官外,急闕久任者,依例以相應人舉用,不許預報,違者罪及所由官司。」



\end{pinyinscope}