\article{志第二十四 祭祀二}

\begin{pinyinscope}

 ○郊祀下



 儀注之節,其目有十:



 一曰齋戒。祀前七日,皇帝散齋四日於別殿,致齋三日,其二日於大明殿,一日於大次,有司停奏刑罰文字。致齋前一日,尚舍監設御幄於大明殿西序,東向。致齋之日質明,諸衛勒所部屯門列仗。晝漏上水一刻,通事舍人引侍享執事文武四品以上官,俱公服詣別殿奉迎。晝漏上水二刻,侍中版奏請中嚴,皇帝服通天冠、絳紗袍。晝漏上水三刻,侍中版奏外辦,皇帝結佩出別殿,乘輿華蓋傘扇侍衛如常儀,奉引至大明殿御幄,東向坐,侍臣夾侍如常。一刻頃,侍中前跪奏「臣某言,請降就齋」,俯伏興。皇帝降座入室,解嚴。侍享執事官各還本司,宿衛者如常。凡侍祠官受誓戒於中書省,散齋四日,致齋三日。守壝門兵衛與大樂工人,俱清齋一宿。光祿卿以陽燧取明火供爨,以方諸取明水實尊。



 二曰告配。祀前二日,攝太尉與太常禮儀院官恭詣太廟,以一獻禮奏告太祖法天啟運聖武皇帝之室。寅刻,太尉以下公服自南神門東偏門入,至橫街南,北向立定。奉禮郎贊曰「拜」,禮直官承傳曰「鞠躬」,曰「拜」,曰「興」,曰「拜」,曰「興」,曰「平立」。又贊曰「各就位」。禮直官詣太尉前曰「請詣盥洗位」,引太尉至盥洗位,曰「盥手」,曰「帨手」,曰「詣爵洗位」,曰「滌爵」,曰「拭爵」,曰「請詣酒尊所」,曰「酌酒」,曰「請詣神座前」,曰「北向立」,曰「稍前」,曰「搢笏」,曰「跪」,曰「上香」,曰「再上香」,曰「三上香」,曰「授幣」,曰「奠幣」,曰「執爵」,曰「祭酒」,曰「祭酒」,曰「三祭酒」。祭酒於沙池訖,曰「讀祝」。舉祝官搢笏,跪對舉祝版。讀祝官跪讀祝文畢,舉祝官奠祝版於案,執笏興,讀祝官俯伏興。禮直官贊曰「出笏」,曰「俯伏興」,曰「拜」,曰「興」,曰「拜」,曰「興」,曰「平立」,曰「復位」。司尊彞、良醖令從降復位,北向立。奉禮郎贊曰「拜」,禮直官承傳再拜畢,太祝捧祝幣降自太階,詣望瘞位。太尉以下俱詣坎位焚瘞訖,自南神門東偏門以次出。



 三曰車駕出宮。祀前一日,所司備儀從內外仗,侍祠官兩行序立於崇天門外,太僕卿控御馬立於大明門外,諸侍臣及導駕官二十有四人,俱於齋殿前左右分班立俟。通事舍人引侍中,奏請中嚴,俯伏興。皇帝服通天冠、絳紗袍。少頃,侍中版奏外辦,皇帝出齋室,即御座。群臣起居訖,尚輦進輿,侍中奏請皇帝升輿,華蓋傘扇侍衛如常儀。導駕官導至大明門外,侍中進當輿前,跪奏請降輿乘馬,導駕官分左右步導。門下侍郎跪奏請進發,俯伏興,前稱警蹕。至崇天門外,門下侍郎奏請權停,敕眾官上馬,侍中承旨稱「制可」,門下侍郎傳制稱「眾官上馬」,贊者承傳「眾官出欞星門外上馬」。門下侍郎奏請進發,前稱警蹕。華蓋傘扇儀仗與眾官分左右前引,教坊樂鼓吹不作。至郊壇南欞星門外,侍中傳制「眾官下馬」,贊者承傳「眾官下馬」。下馬訖,自卑而尊,與儀仗倒卷而北,兩行駐立。駕至欞星門,侍中奏請皇帝降馬,步入欞星門,由西偏門稍西。侍中奏請升輿。尚輦奉輿,華蓋傘扇如常儀。導駕官前導皇帝乘輿至大次前,侍中奏請降輿。皇帝降輿入就次,簾降,侍衛如式。通事舍人承旨,敕眾官各還齋次。尚食進膳訖,禮儀使以祝冊奏請御署訖,奉出,郊祀令受之,各奠於坫。



 四曰陳設。祀前三日,尚舍監陳大次於外壝西門之道北,南向。設小次於內壝西門之外道南,東向。設黃道裀褥,自大次至於小次,版位及壇上皆設之。所司設兵衛,各具器服,守衛壝門,每門兵官二員。外垣東西南欞星門外,設蹕街清路諸軍,諸軍旗服各隨其方之色。去壇二百步,禁止行人。祀前一日,郊祀令率其屬掃除壇之上下。太樂令率其屬設登歌樂於壇上,稍南,北向;設宮縣二舞,位於壇南內壝南門之外,如式。奉禮郎設御版位於小次之前,東向;設御飲福位於壇上,午陛之西,亞終獻飲福位於午陛之東,皆北向。又設亞終獻、助奠、門下侍郎以下版位壇下御版位之後,稍南東向,異位重行,以北為上。又設司徒太常卿以下位於其東,相對北上,皆如常儀。又分設糾儀御史位於其東西二壝門之外,相向而立。又設御盥洗、爵洗位於內壝南門之內道西,北向。又設亞終獻、盥洗、爵洗位於內壝南門之外道西,北向。又設省牲饌等位,如常儀。未後二刻,郊祀令同太史令俱公服,升設昊天上帝位於壇上北方,南向,席以槁秸,加神席褥座。又設配位於壇上西方,東向,席以蒲越,加神席褥座。禮神蒼璧置於繅藉,青幣設於篚,正位之幣加燎玉,置尊所。俟告潔畢,權徹畢。祀日醜前重設。執事者實柴於燎壇,及設籩豆、簠簋、尊罍匏爵、俎坫等事,如常儀。



 五曰省牲器。祀前一日未後二刻,郊祀令率其屬又掃除壇之上下,司尊罍、奉禮郎率祠祭局以祭器入設於位。郊祀令率執事者以禮神之玉,置於神位前。未後三刻,廩犧令與諸太祝、祝史以牲就位,禮直官分引太常卿、光祿卿丞、監祭、監禮官、太官令丞等詣省牲位,立定。禮直官引太常卿、監祭、監禮由東壝北偏門入,自卯陛升壇,視滌濯。司尊罍跪舉冪曰「潔」。告潔畢,俱復位。禮直官稍前曰「請省牲」。太常卿稍前,省牲畢,退復位。次引廩犧令巡牲一匝,西向折身曰「充」。告充畢,復位。諸太祝俱巡牲一匝,復位。上一員出班,西向折身曰「腯」。告腯畢,復位。禮直官引太常卿、光祿卿丞、太官令丞、監祭、監禮詣省饌位,東西相向立。禮直官請太常卿省饌畢,退還齋所。廩牲令與諸太祝、祝史以次牽牲詣廚,授太官令。次引光祿卿、監祭、監禮等詣廚,省鼎鑊,視滌溉畢,還齋所。晡後一刻,太官令率宰人以鸞刀割牲,祝史各取血及左耳毛實於豆,仍取牲首貯於盤,用馬首。



 俱置於饌殿,遂烹牲。刑部尚書蒞之,監實水納烹之事。



 六曰習儀。祀前一日未後三刻,獻官諸執事各服其服,習儀於外壝西南隙地。其陳設、樂架、禮器等物,並如行事之儀。



 七曰奠玉幣。祀日醜前五刻,太常卿設燭於神座,太史令、郊祀令各服其服,升設昊天上帝及配位神座,執事者陳玉幣於篚,置尊所。禮部尚書設祝冊於案。光祿卿率其屬,入實籩豆、簠簋、尊罍如式。祝史以牲首盤設於壇,大樂令率工人二舞入就位。禮直官分引監祭禮、郊祀令及諸執事官、齋郎入就位。禮直官引監祭禮按視壇之上下,退復位。奉禮贊再拜。禮直官承傳,監祭禮以下皆再拜訖,又贊各就位。太官令率齋郎出詣饌殿,俟於門外;禮直官分引攝太尉及司徒等官入就位;符寶郎奉寶陳於宮縣之側,隨地之宜。太尉之將入也,禮直官引博士,博士引禮儀使,對立於大次前。侍中板奏請中嚴,皇帝服大裘袞冕。侍中奏外辦,禮儀使跪奏禮儀使臣某請皇帝行禮,俯伏興。凡奏二人皆跪,一人贊之。簾卷出次,禮儀使前導,華蓋傘扇如常儀。至西壝門外,殿中監進大圭,禮儀使奏請執大圭,皇帝執圭。華蓋傘扇停於門外。近侍官與大禮使皆後從皇帝入門,宮縣樂作。請就小次,釋圭,樂止。禮儀使以下分立左右。少頃,禮儀使奏有司謹具,請行事。降神樂作,《天成之曲》六成。太常卿率祝史捧馬首,詣燎壇升煙訖,復位。禮儀使跪奏請就板位,俯伏興。皇帝出次,請執大圭,至位東向立,再拜。皇帝再拜,奉禮贊眾官皆再拜訖,奉玉幣官跪取玉幣於篚,立於尊所。禮儀使奏請行事。遂前導,宮縣樂作,由南壝西偏門入,詣盥洗位,北向立,樂止。搢大圭,盥手。奉匜官奉匜沃水,奉盤官奉盤承水,執巾官奉巾以進。盥帨手訖,執大圭,樂作,至午陛,樂止。升階,登歌樂作,至壇上,樂止。宮縣《欽成之樂》作,殿中監進鎮圭,殿中監二員,一員執大圭,一員執鎮圭。禮儀使奏請搢大圭,執鎮圭,請詣昊天上帝神位前,北向立。內侍先設繅席於地,禮儀使奏請跪奠鎮圭於繅席。奉玉幣官加玉於幣以授侍中,侍中西向跪進,禮儀使奏請奠玉幣。皇帝受奠訖,禮儀使奏請執大圭,俯伏興,少退再拜。皇帝再拜興,平立。內侍取鎮圭授殿中監,又取繅藉置配位前。禮儀使前導,請詣太祖皇帝神位前,西向立,奠鎮圭及幣,並如上儀,樂止。禮儀使前導,請還版位。登歌樂作,降階,樂止。宮縣樂作,殿中監取鎮圭、繅藉以授有司。皇帝至版位,東向立,樂止。請還小次,釋大圭。祝史奉毛血豆。升自午陛,以進正位,升自卯陛,以進配位。太祝各迎奠於神座前,俱退立尊所。



 八曰進饌。皇帝奠玉幣還位,祝史取毛血豆以降,禮直官引司徒、太官令率齋郎奉饌入自正門,升殿如常儀。禮儀使跪奏請行禮,俯伏興。皇帝出次,宮縣樂作。請執大圭,前導由正門西偏門入,詣盥洗位,北向立,樂止。搢圭盥手如前儀。執圭,詣爵洗位,北向立,搢圭。奉爵官跪取匏爵於篚,以授侍中,侍中以進皇帝,受爵。執罍官酌水洗爵,執巾官授巾拭爵訖,侍中受之,以授捧爵官。執圭,樂作,至午陛,樂止;升階,登歌樂作,至壇上,樂止。詣正位酒尊所,東向立,搢圭。捧爵官進爵,皇帝受爵。司尊者舉冪,侍中贊酌太尊之泛齊。以爵授捧爵官,執圭。宮縣樂作,奏《明成之曲》。請詣昊天上帝神座前北向立,搢圭跪,三上香,侍中以爵跪進皇帝。執爵,三祭酒,以爵授侍中。太官丞注馬湩於爵,以授侍中,侍中跪進皇帝。執爵,亦三祭之,今有蒲萄酒與尚醖馬湩各祭一爵,為三爵。以爵授侍中,執圭,俯伏興,少退立。讀祝,舉祝官搢笏跪舉祝冊,讀祝官西向跪讀祝文,讀訖,俯伏興。舉祝官奠祝於案,奏請再拜。皇帝再拜興,平立。請詣配位酒尊所,西向立。司尊者舉冪,侍中贊酌著尊之泛齊。以爵授捧爵官,執圭。請詣太祖皇帝神位前西向立。宮縣樂作。侍中贊搢圭跪、三上香、三祭酒及馬湩訖,贊執圭,俯伏興,少退立。舉祝官舉祝,讀祝官北向跪讀祝文,讀訖,俯伏興。奠祝版訖,奏請再拜。皇帝再拜興,平立。樂止。請詣飲福位北向立,登歌樂作。太祝各以爵酌上尊福酒,合置一爵以授侍中,侍中西向以進。禮儀使奏請再拜,皇帝再拜興。奏請搢圭、跪受爵。祭酒啐酒以爵授侍中,侍中再以溫酒跪進。禮儀使奏請受爵。皇帝飲福酒訖,侍中受虛爵興,以授太祝。太祝又減神前胙肉加於俎,以授司徒。司徒以俎西向跪進皇帝,受以授左右。奏請執圭,俯伏興,平立,少退。奏請再拜,皇帝再拜訖,樂止。禮儀使前導,還版位。登歌樂作,降自午陛,樂止。宮縣樂作,至位,東向立,樂止。請還小次,至次釋圭。文舞退,武舞進,宮縣樂作,奏《和成之曲》,樂止。禮直官引亞終獻官升自卯陛,行禮如常儀,惟不讀祝,皆飲福而無胙俎。降自卯陛,復位。禮直官贊太祝徹籩豆。登歌樂作,奏《寧成之曲》,卒徹,樂止。奉禮贊賜胙,眾官再拜,在位者皆再拜。禮儀使奏請詣版位,出次執圭,至位東向立,再拜。皇帝再拜。奉禮贊曰「再拜」,贊者承傳「在位者皆再拜」。送神樂作,《天成之曲》一成,止。禮儀使奏禮畢,遂前導皇帝還大次。宮縣樂作,出門樂止,至大次釋圭。



 九曰望燎。皇帝既還大次,禮直官引攝太尉以下監祭禮詣望燎位,太祝各捧篚詣神位前,進取燔玉、祝幣、牲俎並黍稷、飯籩、爵酒,各由其陛降詣燎壇,以祝幣、饌物置柴上,禮直官贊「可燎半柴」,又贊「禮畢」,攝太尉以下皆出。禮直官引監祭禮、祝史、太祝以下從壇南,北向立定,奉禮贊曰「再拜」,監祭禮以下皆再拜訖,遂出。



 十曰車駕還宮。皇帝既還大次,侍中奏請解嚴。皇帝釋袞冕,停大次。五刻頃,所司備法駕,序立於欞星門外,以北為上。侍中版奏請中嚴,皇帝改服通天冠、絳紗袍。少頃,侍中版奏外辦,皇帝出次升輿,導駕官前導,華蓋傘扇如常儀。至欞星門外,太僕卿進御馬如式。侍中前奏請皇帝降輿乘馬訖,太僕卿執御,門下侍郎奏請車駕進發,俯伏興退。車駕動,稱警蹕。至欞星門外,門下侍郎跪奏曰:「請權停,敕眾官上馬。」侍中承旨曰「制可」,門下侍郎傳制,贊者承傳。眾官上馬畢,導駕官及華蓋傘扇分左右前導。門下侍郎跪請車駕進發,俯伏興。車駕動,稱警蹕。教坊樂鼓吹振作。駕至崇天門欞星門外,門下侍郎跪奏曰「請權停,敕眾官下馬」,侍中承旨曰「制可」,門下侍郎俯仗興,退傳制,贊者承傳。眾官下馬畢,左右前引入內,與儀仗倒卷而北駐立。駕入崇天門至大明門外,降馬升輿以入。駕既入,通事舍人承旨敕眾官皆退,宿衛官率衛士宿衛如式。



 攝祀之儀,其目有九:



 一曰齋戒。祀前五日質明,奉禮郎率儀鸞局,設獻官諸執事版位於中書省。獻官諸執事位俱藉以席,仍加紫綾褥。初獻攝太尉設位於前堂階上,稍西,東南向。監察御史二位,一位在甬道上,西稍北,東向;一位在甬道上,東稍北,西向。監禮博士二位,各次御史,以北為上。次亞獻官、終獻官、攝司徒位於其南。次助奠官,次太常太卿、太常卿、光祿卿,次太史令、禮部尚書、刑部尚書,次奉璧官、奉幣官、讀祝官、太常少卿、拱衛直都指揮使,次太常丞、光祿丞、太官令、良醖令、司尊罍,次廩犧令、舉祝官、奉爵官,次太官丞、盥洗官、爵洗官、巾篚官,次翦燭官,次與祭官。其禮直官分直於左右,東西相向。西設版位四列,皆北向,以東為上:郊祀令、太樂令、太祝、祝史,次齋郎。東設版位四列,皆北向,以西為上:郊祀丞、太樂丞、協律郎、奉禮郎,次齋郎、司天生。禮直官引獻官諸執事各就位。獻官諸執事俱公服,五品以上就服其服,六品以下皆借紫服。禮直局管勾進立於太尉之右,宣讀誓文曰:「某年某月某日,祀昊天上帝於圜丘,各揚其職,其或不敬,國有常刑。」散齋三日宿於正寢,致齋二日於祀所。散齋日治事如故,不吊喪問疾,不作樂,不判署刑殺文字,不決罰罪人,不與穢惡事。致齋日惟祀事得行,其餘悉禁。凡與祀之官已齋而闕者,通攝行事。讀畢,稍前唱曰「七品以下官先退」,復贊曰「對拜」,太尉與餘官皆再拜乃退。凡與祭者,致齋之宿,官給酒饌。守壝門兵衛及太樂工人,皆清齋一宿。



 二曰告配。祀前二日,初獻官與太常禮儀院官恭詣太廟,奏告太祖皇帝本室,即還齋次。



 三曰迎香。祝祀前二日,翰林學士赴禮部書寫祝文,太常禮儀院官亦會焉。書畢於公廨嚴潔安置。祀前一日質明,獻官以下諸執事皆公服,禮部尚書率其屬捧祝版,同太常禮儀院官俱詣闕廷,以祝版授太尉,進請御署訖,同香酒迎出崇天門外。香置於輿,祝置香案,御酒置輦樓,俱用金復覆之。太尉以下官比上馬,清道官率京官行於儀衛之先,兵馬司巡兵執矛幟夾道次之,金鼓又次之,京尹儀從左右成列前導,諸執事官東西二班行於儀仗之外,次儀鳳司奏樂,禮部官點視成列,太常禮儀院官導於香輿之前,然後控鶴舁輿案行,太尉等官從行至祀所。輿案由南欞星門入,諸執事官由左右偏門入,奉安御香、祝版於香殿。



 四曰陳設。祀前三日,樞密院設兵衛各具器服守衛壝門,每門兵官二員,及外垣東西南欞星門外,設蹕街清路諸軍,諸軍旗服,各隨其方色。去壇二百步,禁止行人。祀前一日,郊祀令率其屬掃除壇上下。大樂令率其屬設登歌樂於壇上,稍南,北向。編磬一虡在西,編鐘一虡在東。擊鐘磬者,皆有坐杌。大樂令位在鐘虡東,西向。協律郎位在磬虡西,東向。執麾者立於後。柷一,在鐘虡北,稍東。敔一,在磬虡北,稍西。搏拊二,一在柷北,一在敔北。歌工八人,分列於午陛左右,東西相向坐,以北為上,凡坐者皆藉以席加氈。琴一弦、三弦、五弦、七弦、九弦者各二,瑟四,籥二,篪二,笛二,簫二,巢笙四,和笙四,閏餘匏一,九曜匏一,七星匏一,塤二,各分立於午陛東西樂榻上。琴瑟者分列於北,皆北向坐;匏竹者分立於琴瑟之後,為二列重行,皆北向相對為首。又設圜宮懸樂於壇南,內壝南門之外。東方西方,編磬起北,編鐘次之。南方北方,編磬起西,編鐘次之。又設十二鎛鐘於編懸之間,各依辰位。每辰編磬在左,編鐘在右,謂之一肆。每面三辰,共九架,四面三十六架。設晉鼓於懸內通街之東,稍南,北向。置雷鼓、單鞀、雙鞀各二柄於北懸之內,通街之左右,植四楹雷鼓於四隅,皆左鼙右應。北懸之內,歌工四列。內二列在通街之東,二列於通街之西。每列八人,共三十二人,東西相向立,以北為上。柷一在東,敔一在西,皆在歌工之南。大樂丞位在北懸之外,通街之東,西向。協律郎位於通街之西,東向。執麾者立於後,舉節樂正立於東,副正立於西,並在歌工之北。樂師二員,對立於歌工之南。運譜二人,對立於樂師之南。照燭二人,對立於運譜之南,祀日分立於壇之上下,掌樂作樂止之標準。琴二十七,設於東西懸內:一弦者三,東一,西二,俱為第一列;三弦、五弦、七弦、九弦者各六,東西各四列,每列三人,皆北向坐。瑟十二,東西各六,共為列,在琴之後坐。巢笙十、簫十、閏餘匏一在東,七星匏一、九曜匏一,皆在竽笙之側。竽笙十、籥十、篪十、塤八、笛十,每色為一列,各分立於通街之東西,皆北向,又設文舞位於北懸之前,植四表於通街之東,舞位行綴之間。導文舞執衙仗舞師二員,執旌二人,分立於舞者行綴之外。舞者八佾,每佾八人,共六十四人,左手執籥,右手秉翟,各分四佾,立於通街之東西,皆北向。又設武舞,俟立位於東西縣外。導武舞執衙仗舞師二員,執纛二人,執器二十人,內單鞀二、單鐸二、雙鐸二、金鐃二、鉦二、金錞二,執扃者四人,扶錞二、相鼓二、雅鼓二,分立於東西縣外。舞者如文舞之數,左手執干,右手執戚,各分四佾,立於執器之外。俟文舞自外退,則武舞自內進,就立文舞之位,惟執器者分立於舞人之外。文舞亦退於武舞俟立之位。太史令、郊祀令各公服,率其屬升設昊天上帝神座於壇上,北方,南向;席以槁秸,加褥座,置璧於繅藉,設幣於篚,置酌尊所。皇地祇神座,壇上稍東,北方,南向;席以槁秸,加褥座,置玉於繅藉,設幣於篚,置酌尊所。配位神座,壇上東方,西向;席以蒲越,加褥座,置璧於繅藉,設幣於篚,置酌尊所。設五方五帝、日、月、天皇大帝、北極等九位,在壇之第一等;席以莞,各設玉幣於神座前。設內官五十四位於圜壇第二等,設中官一百五十九位於圜壇第三等,設外官一百六位於內壝內,設眾星三百六十位於內壝外,席皆以莞,各設青幣於神座之首,皆內向。候告潔畢,權徹第一等玉幣,至祀日醜前重設。執事者實柴於燎壇,仍設葦炬於東西。執炬者東西各二人,皆紫服。奉禮郎率儀鸞局,設獻官以下及諸執事官版位,設三獻官版位於內壝西門之外道南,東向,以北為上。次助奠位稍卻,次第一等至第三等分獻官,第四等、第五等分奠官,次郊祀令、太官令、良醖令、廩犧令、司尊罍,次郊祀丞、讀祝官、舉祝官、奉璧官、奉幣官、奉爵官、太祝、盥洗官、爵洗官、巾篚官、祝史,次齋郎,位於其後。每等異位重行,俱東向,北上。攝司徒位於內壝東門之外道南,與亞獻相對。次太常禮儀使、光祿卿、同知太常禮儀院事、太史令、分獻分奠官、僉太常禮儀院事、拱衛直都指揮使、太常禮儀院同僉院判、光祿丞,位於其南,皆西向,北上。監察御史二位,一位在內壝西門之外道北,東向;一位在內壝東門之外道北,西向。博士二位,各次御史,以北為上。設奉禮郎位於壇上稍南,午陛之東,西向;司尊罍位於尊所,北向。又設望燎位於燎壇之北,南向。設牲榜於外壝東門之外,稍南,西向;太祝、祝史位於牲後,俱西向。設省牲位於牲北,太常禮儀使、光祿卿、太官令、光祿丞、太官丞位於其北,太官令以下位皆少卻。監祭、監禮位在太常禮儀使之西,稍卻,南向。廩犧令位於牲西南,北向。又設省饌位於牲位之北,饌殿之南。太常禮儀使、光祿卿丞、太官令丞位在東,西向;監祭、監禮位在西,東向;俱北上。祠祭局設正配三位,各左十有二籩,右十有二豆,俱為四行。登三,鉶三,簠、簋各二,在籩豆間。登居神前,鉶又居前,簠左、簋右,居鉶前,皆藉以席。設牲首俎一,居中;牛羊豕俎七,次之。香案一,沙池、爵坫各一,居俎前。祝案一,設於神座之右。又設天地二位各太尊二、著尊二、犧尊二、山罍二於壇上東南,俱北向,西上。又設配位著尊二、犧尊二、象尊二、山罍二,在二尊所之東,皆有坫,加勺冪,惟玄酒有冪無勺,以北為上。馬湩三器,各設於尊所之首,加冪勺。又設玉幣篚二於尊所西,以北為上。又設正位象尊二、壺尊二、山罍四於壇下午陛之西。又設地祇尊罍,與正位同,於午陛之東,皆北向,西上。又設配位犧尊二、壺尊二、山罍四在酉陛之北,東向,北上,皆有坫、冪,不加勺,設而不酌。又設第一等九位各左八籩,右八豆,登一,在籩豆間,簠、簋各一,在登前,俎一,爵、坫各一,在簠、簋前。每位太尊二、著尊二,於神之左,皆有坫,加勺、冪,沙池、玉幣篚各一。又設第二等諸神每位籩二,豆二,簠、簋各一,登一,俎一,於神座前。每陛間象尊二,爵、坫、沙池、幣篚各一,於神中央之座首。又設第三等諸神,每位籩、豆、簠、簋各一,俎一,於神座前。每陛間設壺尊一,爵尊二,爵、坫、沙池、幣篚各一,於神中央之座首。又設內壝內諸神,每位籩、豆各一,簠、簋各一,於神座前。每道間概尊二,爵、坫、沙池、幣篚各一,於神中央之座首。又設內壝外眾星三百六十位,每位籩、豆、簠、簋、俎各一,於神座前。每道間散尊二,爵、坫、沙池、幣篚各一,於神中央之座前。自第一等以下,皆用匏爵洗滌訖,置於坫上。又設正配位各籩一,豆一,簠一,簋一,俎四,及毛血豆各一,牲首盤一。並第一等神位,每位俎二,於饌殿內。又設盥洗、爵洗於壇下,卯階之東,北向,罍在洗東加勺,篚在洗西南肆,實以巾,爵洗之篚實以匏,爵加坫。又設第一等分獻官盥洗、爵洗位,第二等以下分獻官盥洗位,各於陛道之左,罍在洗左,篚在洗右,俱內向。凡司尊罍篚位,各於其後。



 五曰省牲器,見親祀儀。



 六曰習儀,見親祀儀。



 七曰奠玉幣。祀日醜前五刻,太常卿率其屬,設椽燭於神座四隅,仍明壇上下燭、內外凡燎。太史令、郊祀令各服其服升,設昊天上帝神座,槁秸、席褥如前。執事者陳玉幣於篚,置於尊所。禮部尚書設祝版於案。光祿卿率其屬入實籩、豆、簠、簋。籩四行,以右為上。第一行魚鱐在前,糗餌、粉糍次之;第二行乾棗在前,乾尞形鹽次之;第三行鹿脯在前,榛實、乾桃次之;第四行菱在前,芡、慄次之。豆四行,以左為上。第一行芹菹在前,筍菹、葵菹次之。第二行菁菹在前,韭菹、厓食次之。第三行魚醢在前,兔醢、豚拍次之。第四行鹿MZ在前,醓醢、糝食次之。簠實以稻、粱,簋實以黍、稷,登實以太羹。良醖令率其屬入實尊、罍。太尊實以泛齊,著尊醴齊,犧尊盎齊,象尊醍齊,壺尊沈齊;山罍為下尊,實以玄酒;其酒、齊皆以尚醴酒代之。太官丞設革囊馬湩於尊所。祠祭局以銀盒貯香,同瓦鼎設於案。司香官一員立於壇上。祝史以牲首盤,設於壇上。獻官以下執事官,各服其服,就次所,會於齊班幕。拱衛直都指揮使率控鶴,各服其服,擎執儀仗,分立於外壝內東西,諸執事位之後;拱衛使亦就位。大樂令率工人二舞,自南壝東偏門以次入,就壇上下位。奉禮郎先入就位。禮直官分引監祭御史、監禮博士、郊祀令、太官令、良醖令、廩犧令、司尊罍、太官丞、讀祝官、舉祝官、奉玉幣官、太祝、祝史、奉爵官、盥爵洗官、巾篚官、齋郎,自南壝東偏門入,就位。禮直官引監祭、監禮,按視壇之上下祭器,糾察不如儀者。及其按視也,太祝先徹去蓋冪,按視訖,禮直官引監祭、監禮退復位。奉禮郎贊「再拜」,禮直官承傳曰「拜」,監祭禮以下皆再拜。奉禮郎贊曰「各就位」,太官令率齋郎以次出詣饌殿,俟立於南壝門外。禮直官分引三獻官、司徒、助奠官、太常禮儀院使、光祿卿、太史令、太常禮儀院同知僉院、同僉院判、光祿丞,自南壝東偏門,經樂縣內入就位。禮直官進太尉之左,贊曰「有司謹具,請行事」,退復位。宮縣樂作降神《天成之曲》六成,內圜鐘宮三成,黃鐘角、太簇征、姑洗羽各一成。文舞《崇德之舞》。初樂作,協律郎跪,俯伏舉麾興,工鼓柷,偃麾,戛敔而樂止。凡樂作、樂止,皆仿此。禮直官引太常禮儀院使率祝史,自卯陛升壇,奉牲首降自午陛,由南壝正門經宮縣內,詣燎壇北,南向立。祝史奉牲首升自南陛,置於戶內柴上。東西執炬者以火燎柴,升煙燔牲首訖,禮直官引太常禮儀院使祝史捧盤血,詣坎位瘞之。禮直官引太常禮儀院使、祝史,各復位。奉禮郎贊「再拜」,禮直官承傳曰「拜」,太尉以下皆再拜訖,其先拜者不拜。執事者取玉幣於篚,立於尊所。禮直官引太尉詣盥洗位,宮縣樂奏黃鐘宮《隆成之曲》,至位北向立,樂止。搢笏、盥手、帨手訖,執笏詣壇,升自午陛,登歌樂作大呂宮《隆成之曲》,至壇上,樂止。詣正位神座前,北向立,宮縣樂奏黃鐘宮《欽成之曲》,搢笏跪,三上香。執事者加璧於幣,西向跪,以授太尉,太尉受玉幣奠於正位神座前,執笏,俯伏興,少退立,再拜訖,樂止。次詣皇地祇位,奠獻如上儀。次詣配位神主前,奠幣如上儀。降自午陛,登歌樂作如升壇之曲,至位樂止。祝史奉毛血豆,入自南壝門詣壇,升自午陛。諸太祝迎取於壇上,俱跪奠於神座前,執笏,俯伏興,退立於尊所。



 至大三年大祀,奠玉幣儀與前少異,今存之,以備互考。祀日醜前五刻,設壇上及第一等神位,陳其玉幣及明燭,實籩、豆、尊、罍。樂工各入就位畢,奉禮郎先入就位。禮直官分引分獻官、監祭御史、監禮博士、諸執事、太祝、祝史、齋郎,入自中壝東偏門,當壇南重行西上,北向立定。奉禮郎贊曰「再拜」,分獻官以下皆再拜訖,奉禮贊曰「各就位」。禮直官引子丑寅卯辰巳陛道分獻官,詣版位,西向立,北上;午未申酉戌亥陛道分獻官,詣版位,東向立,北上。禮直官分引監祭禮點視陳設,按視壇之上下,糾察不如儀者,退復位。太史令率齋郎出俟。禮直官引三獻官並助奠等官入就位,東向立,司徒西向立。禮直官贊曰「有司謹具,請行事」,降神六成樂止。太常禮儀使率祝史二員,捧馬首詣燎壇,升煙訖,復位。奉禮郎贊曰「再拜,三獻」,司徒等皆再拜訖,奉禮郎贊曰「諸執事者各就位」,立定。禮直官請初獻官詣盥洗位,樂作,至位,樂止。盥畢詣壇,樂作,升自卯陛,至壇,樂止。詣正位神座前,北向立,樂作,搢笏跪,太祝加玉於幣,西向跪以授初獻,初獻受玉幣奠訖,執笏俯伏興,再拜訖,樂止。次詣配位神座前立,樂作,奠玉幣如上儀,樂止。降自卯陛,樂作,復位,樂止。初獻將奠正位之幣,禮直官分引第一等分獻官詣盥洗位,盥畢,執笏各由其陛升,詣各神位前,搢笏跪,太祝以玉幣授分獻官,奠訖,俯伏興,再拜訖,還位。初第一等分獻官將升,禮直官分引第二等內壝內、內壝外分獻官盥畢,盥洗官俱從至酌尊所立定,各由其陛道詣各神首位前奠,並如上儀。退立酌尊所,伺候終獻酌奠,詣各神首位前酌奠。祝史奉正位毛血豆由午陛升,配位毛血豆由卯陛升,太祝迎於壇上,進奠於正配位神座前,太祝與祝史俱退於尊所。



 八曰進熟。太尉既升奠玉幣,太官令丞率進饌齋郎詣廚,以牲體設於盤,馬牛羊豕鹿各五盤,宰割體段,並用國禮。各對舉以行至饌殿,俟光祿卿出實籩、豆、簠、簋。籩以粉糍,豆以糝食,簠以粱,簋以稷。齋郎上四員,奉籩、豆、簠、簋者前行,舉盤者次之。各奉正配位之饌,以序立於南壝門之外,俟禮直官引司徒出詣饌殿,齋郎各奉以序從司徒入自南壝正門。配位之饌,入自偏門。宮縣樂奏黃鐘宮《寧成之曲》,至壇下,俟祝史進徹毛血豆訖,降自卯陛以出。司徒引齋郎奉正位饌詣壇,升自午陛,太史令丞率齋郎奉配位及第一等之饌,升自卯陛,立定。奉禮贊諸太祝迎饌,諸太祝迎於壇陛之間,齋郎各跪奠於神座前。設籩於糗餌之前,豆於醯醢之前,簠於稻前,簋於黍前。又奠牲體盤於俎上,齋郎出笏,俯伏興,退立定,樂止。禮直官引司徒降自卯陛,太官令率齋郎從司徒亦降自卯陛,各復位。其第二等至內壝外之饌,有司陳設。禮直官贊,太祝搢笏,立茅苴於沙池,出笏,俯伏興,退立於本位。禮直官引太尉詣盥洗位,宮縣樂作,奏黃鐘宮《隆成之曲》,至位北向立,樂止。搢笏、盥手、帨手訖,出笏詣爵洗位,北向立。搢笏,執事者奉匏爵以授太尉,太尉洗爵、拭爵訖,以爵授執事者。太尉出笏,詣壇,升自午陛,一作卯陛。登歌樂作,奏黃鐘宮《明成之曲》,至壇上,樂止。詣酌尊所,西向立,搢笏,執事者以爵授太尉,太尉執爵,司尊罍舉冪,良醖令酌太尊之泛齊,凡舉冪、酌酒,皆跪。以爵授執事者。太尉出笏,詣正位神座前,北向立,宮縣樂作,奏黃鐘宮《明成之曲》,文舞《崇德之舞》。太尉搢笏跪,三上香。執事者以爵授太尉,太尉執爵三祭酒於茅苴,以爵授執事者,執事者奉爵退,詣尊所。太官丞傾馬湩於爵,跪授太尉,亦三祭於茅苴,復以爵授執事者,執事者受虛爵以興。太尉出笏,俯伏興,少退,北向立,樂止。舉祝官搢笏跪,對舉祝版,讀祝官搢笏跪,讀祝文。讀訖,舉祝官奠版於案,出笏興,讀祝官出笏,俯伏興,宮縣樂奏如前曲。舉祝、讀祝官俱先詣皇地祇位前,北向立。太尉再拜訖,樂止。次詣皇地祇位,並如上儀,惟樂奏大呂宮。次詣配位,並如上儀,惟樂奏黃鐘宮。降自午陛,一作卯陛。登歌樂作如前降神之曲,至位,樂止。讀祝、舉祝官降自卯陛,復位。文舞退,武舞進,宮縣樂作,奏黃鐘宮《和成之曲》,立定,樂止。禮直官引亞獻官詣盥洗位,北向立。搢笏、盥手、帨手訖,出笏詣爵洗位,北向立。搢笏、執爵、洗爵、拭爵,以爵授執事者。出笏詣壇,升自卯陛,至壇上酌尊所,東向一作西向。立。搢笏授爵執爵,司尊罍舉冪,良醖令酌著尊之醴齊,以爵授執事者。出笏,詣正位神座前,北向立。宮縣樂奏黃鐘宮《熙成之曲》,武舞《定功之舞》。搢笏跪,三上香,授爵執爵,三祭酒於茅苴,復祭馬湩如前儀,以爵授執事者。出笏,俯伏興,少退立,再拜訖,次詣皇地祇位、配位,並如上儀訖,樂止,降自卯陛,復位。禮直官引終獻官詣盥洗位,盥手、帨手訖,詣爵洗位,授爵執爵,洗爵拭爵,以爵授執事者。出笏,升自卯陛,至酌尊所,搢笏授爵執爵,良醖令酌犧尊之盎齊,以爵授執事者。出笏,詣正位神座前,北向立。宮縣樂作,奏黃鐘宮《熙成之曲》,武舞《定功之舞》。上香、祭酒、馬湩,並如亞獻之儀,降自卯陛。初終獻將升壇時,禮直官分引第一等分獻官詣盥洗位,搢笏、盥手、帨手、滌爵、拭爵訖,以爵授執事者。出笏,各由其陛詣酌尊所,搢笏,執事者以爵授分獻官,執爵,酌太尊之泛齊,以爵授執事者。各詣諸神位前,搢笏跪,三上香、三祭酒訖,出笏,俯伏興,少退,再拜興,降復位。第一等分獻官將升壇時,禮直官引第二等、第三等、內壝內、內壝外眾星位分獻官,各詣盥洗位,搢笏、盥手、帨手,酌奠如上儀訖,禮直官各引獻官復位,諸執事者皆退復位。禮直官贊太祝徹籩豆。登歌樂作大呂宮《寧成之曲》,太祝跪以籩豆各一少移故處,卒徹,出笏,俯伏興,樂止。奉禮郎贊曰「賜胙」,眾官再拜,禮直官承傳曰「拜」,在位者皆再拜,平,立定。送神宮縣樂作,奏圜鐘宮《天成之曲》一成止。



 九曰望燎。禮直官引太尉,亞獻助奠一員,太常禮儀院使,監祭、監禮各一員等,詣望燎位。又引司徒,終獻助奠、監祭、監禮各一員,及太常禮儀院使等官,詣望瘞位。樂作,奏黃鐘宮《隆成之曲》,至位,南向立,樂止。上下諸執事各執篚進神座前,取燔玉及幣祝版。日月已上,齋郎以俎載牲體黍稷,各由其陛降,南行,經宮縣樂,出東,詣燎壇。升自南陛,以玉幣、祝版、饌食致於柴上戶內。諸執事又以內官以下之禮幣,皆從燎。禮直官贊曰「可燎」,東西執炬者以炬燎火半柴。執事者亦以地祇之玉幣、祝版、牲體、黍稷詣瘞坎。焚瘞畢,禮直官引太尉以下官以次由南壝東偏門出,禮直官引監祭、監禮、奉玉幣官、太祝、祝史、齋郎俱復壇南,北向立。奉禮郎贊曰「再拜」,禮直官承傳曰「拜」,監祭、監禮以下皆再拜訖,各退出。太樂令率工人二舞以次出。禮直官引太尉以下諸執事官至齊班幕前立,禮直官贊曰「禮畢」,眾官員揖畢,各退於次。太尉等官、太常禮儀院使、監祭、監禮展視胙肉酒醴,奉進闕庭,餘官各退。



 祭告三獻儀,大德十一年所定。告前三日,三獻官、諸執事官具公服赴中書省受誓戒。前一日未正二刻,省牲器。告日質明,三獻官以下諸執事官各具法服。禮直官引監祭禮以下諸執事官,先入就位,立定。監祭禮點視陳設畢,復位,立定。太官令率齋郎出,禮直官引三獻司徒、太常禮儀院使、光祿卿入就位,立定。禮直官贊曰「有司謹具,請行事」,降神樂作六成止。太常禮儀院使燔牲首,復位,立定。奉禮贊三獻以下皆再拜,就位。禮直官引初獻詣盥洗位,盥手訖,升壇詣昊天上帝位前,北向立。搢笏跪,三上香,奠玉幣,出笏,俯伏興,再拜訖,降復位。禮直官引初獻詣盥洗位,盥手訖,詣爵洗位,洗拭爵訖,詣酒尊所,酌酒訖,請詣昊天上帝神位前,北向,搢笏跪,三上香,執爵三祭酒於茅苴,出笏,俯伏興,俟讀祝訖,再拜,平立。請詣皇地祇酒尊所,酌獻並如上儀,俱畢,復位。禮直官引亞獻,並如初獻之儀,惟不讀祝,降復位。禮直官引終獻,並如亞獻之儀,降復位。奉禮贊「賜胙」,眾官再拜,在位者皆再拜。禮直官引三獻司徒、太常卿、光祿卿、監祭、監禮等官請詣望燎位,南向立定,俟燎玉幣祝版。禮直官贊「可燎」,禮畢。



 祭告一獻儀,至元十二年所定。告前二日,郊祀令掃除壇壝內外,翰林國史院學士撰寫祝文。前一日,告官等各公服捧祝版,進請御署訖,同禦香上尊酒如常儀,迎至祠所齋宿。告日質明前三刻,禮直官引郊祀令率其屬詣壇,鋪筵陳設如儀。禮直官二員引告官等各具紫服,以次就位,東向立定。禮直官稍前曰「有司謹具,請行事」,贊者曰「鞠躬」,曰「拜」,曰「興」,曰「拜」,曰「興」,曰「平身」。禮直官先引執事官各就位,次詣告官前曰「請詣盥爵洗位」。至位,北向立,曰「搢笏」,曰「盥手」,曰「帨手」,曰「洗爵」,曰「拭爵」,曰「出笏」,曰「詣酒尊所」,曰「搢笏」,曰「執爵」,曰「司尊者舉冪」,曰「酌酒」。良醖令酌酒,曰「以爵授執事者」,告官以爵授執事者。曰「出笏」,曰「詣昊天上帝、皇地祇神位前,北向立」,曰「稍前」,曰「搢笏」,曰「跪」,曰「上香」,曰「上香」,曰「三上香」,曰「祭酒」,曰「祭酒」,曰「三祭酒」,曰「以爵授捧爵官」,曰「出笏」,曰「俯伏興」,曰「舉祝官跪」,曰「舉祝」,曰「讀祝官跪」,曰「讀祝」。讀訖,曰「舉祝官奠祝版於案」,曰「俯伏興」。告官再拜,曰「鞠躬」,曰「拜」,曰「興」,曰「拜」,曰「興」,曰「平身」,引告官以下降復位。禮直官贊曰「再拜」,曰「鞠躬」,曰「拜」,曰「興」,曰「拜」,曰「興」,曰「平身」,曰「詣望燎位」,燔祝版半燎,告官以下皆退。其瘞之坎於祭所壬地,方深足以容物。



\end{pinyinscope}