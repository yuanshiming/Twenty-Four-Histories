\article{本紀第二十九 泰定帝一}

\begin{pinyinscope}

 泰定皇帝,諱也孫鐵木兒,顯宗甘麻剌之長子,裕宗之嫡孫也。初,世祖以第四子那木罕為北安王,鎮北邊。北安王薨,顯宗以長孫封晉王代之,統領太祖四大斡耳朵及軍馬、達達國土。至元十三年十月二十九日,帝生於晉邸。大德六年,晉王薨,帝襲封,是為嗣晉王,仍鎮北邊。成宗、武宗、仁宗之立,咸與翊戴之謀,有盟書焉。王府內史倒剌沙得幸於帝,常偵伺朝廷事機,以其子哈散事丞相拜住,且入宿衛。久之,哈散歸,言御史大夫鐵失與拜住意相忤,欲傾害之。至治三年三月,宣徽使探忒來王邸,為倒剌沙言:「主上將不容於晉王,汝盍思之。」於是倒剌沙與探忒深相要結。



 八月二日,晉王獵於禿剌之地,鐵失密遣斡羅思來告曰:「我與哈散、也先鐵木兒、失禿兒謀已定,事成,推立王為皇帝。」又命斡羅思以其事告倒剌沙,且言:「汝與馬速忽知之,勿令旭邁傑得聞也。」於是王命囚斡羅思,遣別烈迷失等赴上都,以逆謀告,未至。癸亥,英宗南還,駐蹕南坡。是夕,鐵失等矯殺拜住,英宗遂遇弒於幄殿。諸王按梯不花及也先鐵木兒奉皇帝璽綬,北迎帝於鎮所。九月癸巳,即皇帝位於龍居河,大赦天下。詔曰:



 薛禪皇帝可憐見嫡孫、裕宗皇帝長子、我仁慈甘麻剌爺爺根底,封授晉王,統領成吉思皇帝四個大斡耳朵,及軍馬、達達國土都付來。依著薛禪皇帝聖旨,小心謹慎,但凡軍馬人民的不揀甚麼勾當里,遵守正道行來的上頭,數年之間,百姓得安業。在後,完澤篤皇帝教我繼承位次,大斡耳朵裡委付了來。已委付了的大營盤看守著,扶立了兩個哥哥曲律皇帝、普顏篤皇帝,侄碩德八剌皇帝。我累朝皇帝根底,不謀異心,不圖位次,依本分與國家出氣力行來;諸王哥哥兄弟每,眾百姓每,也都理會的也者。今我的侄皇帝生天了也麼道,迤南諸王大臣、軍士的諸王駙馬臣僚、達達百姓每,眾人商量著:大位次不宜久虛,惟我是薛禪皇帝嫡派,裕宗皇帝長孫,大位次里合坐地的體例有,其餘爭立的哥哥兄弟也無有;這般,晏駕其間,比及整治以來,人心難測,宜安撫百姓,使天下人心得寧,早就這裡即位提說上頭,從著眾人的心,九月初四日,於成吉思皇帝的大斡耳朵裡,大位次裡坐了也。交眾百姓每心安的上頭,赦書行有。



 是日,以知樞密院事淇陽王也先鐵木兒為中書右丞相,諸王月魯鐵木兒襲封安西王。甲午,以內史倒剌沙為中書平章政事,乃馬臺為中書右丞,鐵失知樞密院事,馬思忽同知樞密院事,孛羅為宣徽院使,旭邁傑為宣政院使。乙未,大理護子羅蠻為寇。以樞密副使阿散為御史中丞,內史善僧為中書左丞。丁酉,以完澤知樞密院事,禿滿同僉樞密院事。戊戌,以撒的迷失知樞密院事,章臺同知樞密院事。己亥,敕諭百司:「凡銓授官,遵世祖舊制,惟樞密院、御史臺、宣政院、宣徽院得自奏聞,餘悉由中書。」辛丑,以馬某沙知樞密院事,失禿兒為大司農。召諸王官屬流徙遠地及還元籍者二十四人還京師。是歲,大寧蒙古大千戶部風雪斃畜牧,賑米十五萬石。南康、漳州二路水,淮安、揚州屬縣饑,賑之。



 冬十月癸亥,修佛事於大明殿。甲子,遣使至大都,以即位告天地、宗廟、社稷,誅逆賊也先鐵木兒、完者、鎖南、禿滿等於行在所。以旭邁傑為中書右丞相,陜西行中書左丞相禿忽魯、通政院使紐澤並為御史大夫,速速為御史中丞。遣旭邁傑、紐澤誅逆賊鐵失、失禿兒、赤斤鐵木兒、脫火赤、章臺等於大都,並戮其子孫,籍入家產。己巳,太白犯亢。戊辰,召亦都護高昌王鐵木兒補化。壬申,以內史按答出為太師、知樞密院事。丙子,太白犯氐。詔百司遵守世祖成憲。癸未,以旭邁傑兼阿速衛達魯花赤。丙戌,以江浙行省平章政事兀伯都剌為中書平章政事。八番順元及靜江、大理、威楚諸路徭兵為寇,敕湖廣、雲南二省招諭之。揚州江都縣火,雲南王、西平王二部衛士饑,皆賑之。



 十一月己丑朔,熒惑犯亢。車駕次於中都,修佛事於昆剛殿。庚寅,太白犯鉤鈐。丙申,次於祖媯。乙未,太白犯東咸。辛丑,車駕至大都。壬寅,熒惑犯氐。諸王怯別遣使來朝。丁未,御大明殿,受諸王、百官朝賀。庚戌,詔百司朝夕視事毋怠。辛亥,御史中丞董守庸,坐黨鐵失免官。壬子,敕營繕不急者罷之。癸丑,遣使詣曲阜,以太牢祀孔子。敕會福院奉北安王那木罕像於高良河寺,祭遁甲五福神。甲寅,諸王怯別遣使來朝。乙卯,翽星於司天監。丙辰,御史中丞速速坐貪淫免官。丁巳,廣州路新會縣民汜長弟作亂,廣東副元帥烏馬兒率兵捕之。雲南開南州大阿哀、阿三木、臺龍買六千餘人寇哀卜白鹽井。詔:「凡有罪自首者,原其罪。」袁州路宜春縣、鎮江路丹徒縣饑,賑糶米四萬九千石。沅州黔陽縣饑,芍陂屯田旱,並賑之。



 十二月己未,御史臺經歷朵兒只班、御史撒兒塔罕、兀都蠻、郭也先忽都,並坐黨鐵失免官。御史言:「曩者鐵木迭兒專政,誣殺楊朵兒只、蕭拜住、賀伯顏、觀音保、鎖咬兒哈的迷失,黥竄李謙亨、成珪,罷免王毅、高昉、張志弼,天下咸知其冤,請昭雪之。」詔存者召還錄用,死者贈官有差。授諸王薛徹干以其父故金印。庚申,以宦者剛答里為中政院使。壬戌,賜潛邸衛士鈔,人六十錠。浚鎮江路漕河及練湖,役丁萬三千五百人。給諸王八剌失里印。戊辰,請皇考、皇妣謚於南郊,皇考晉王曰光聖仁孝皇帝,廟號顯宗,皇妣晉王妃曰宣懿淑聖皇后。己巳,辰星犯壘壁陣。庚午,以即位,大賚后妃、諸王、百官,金七百餘錠、銀三萬三千錠,錢及幣帛稱是。遣使祀海神天妃。盜入太廟,竊仁宗及莊懿慈聖皇后金主。辛未,熒惑犯房。壬申,作仁宗主,仍督有司捕盜。翽星於司天監。癸酉,德慶路瀧水縣徭劉寅等降。甲戌,命道士吳全節修醮事。乙亥,征東夷民奉獸皮來附。太常院臣言:「世祖以來,太廟歲惟一享,先帝始復古制,一歲四祭,請裁擇之。」帝曰:「祭祀,盛事也,朕何敢簡其禮。」命仍四祭。監察御史脫脫、趙成慶等言:「鐵木迭兒在先朝,包藏禍心,離間親籓,誅戮大臣,使先帝孤立,卒罹大禍。其子鎖南,親與逆謀,久逭天憲,乞正其罪,以快元元之心。月魯、禿禿哈、速敦皆鐵失之黨,不宜寬宥。」遂並伏誅。丙子,命嶺北守邊諸王徹徹禿,月修佛事,以卻寇兵。己卯,命僧作佛事於大內以厭雷。增諸王薛徹干、駙馬哈伯等歲賜金、銀、幣、帛有差。辛巳,熒惑犯東咸。壬午,諸王月思別遣怯烈來朝,賜以金、幣。癸未,廣西右江來安路總管岑世興遣其弟世元入貢。流諸王月魯鐵木兒於雲南,按梯不花於海南,曲呂不花於奴兒乾,孛羅及兀魯思不花於海島,並坐與鐵失等逆謀。乙酉,雲南車里於孟為寇,詔招諭之。諭百司借名器,各遵世祖定制。丙戌,旭邁傑言:「近也先鐵木兒之變,諸王買奴逃赴潛邸,願效死力,且言不除元兇,則陛下美名不著,天下後世何從而知。上契聖衷,嘗蒙獎諭。今臣等議,宗戚之中,能自拔逆黨,盡忠朝廷者,惟有買奴,請加封賞,以示激勸。」遂以泰寧縣五千戶封買奴為泰寧王。知樞密院事、大司徒闊徹伯授開府儀同三司,以前太師拜忽商議軍國重事。丁亥,議賞討逆功,賜旭邁傑金十錠、銀三十錠、鈔七千錠,倒剌沙為中書左丞相,知樞密院事馬某沙、御史大夫紐澤、宣政院使鎖禿並加授光祿大夫,仍賜金、銀、鈔有差。塑馬哈吃剌佛像於延春閣之徽清亭。下詔改元,詔曰:「朕荷天鴻禧,嗣大歷服,側躬圖治,夙夜祗畏,惟祖訓是遵,乃開歲甲子,景運伊始,思與天下更新。稽諸典禮,逾年改元,可以明年為泰定元年。」免大都、興和差稅三年,八番、思、播、兩廣洞寨差稅一年,江淮創科包銀三年,四川、雲南、甘肅秋糧三分,河南、陜西、遼陽絲鈔三分。除虛增田稅,免斡脫逋錢,賑恤雲南、廣海、八番等處戍軍。求直言,賜高年帛,禁獻山場湖泊之利。定吏員出身者秩正四品。以追尊皇考、皇妣,詔天下。雲南花腳蠻為寇,詔招諭之。平江嘉定州饑,遼陽答陽失蠻、闊闊部風、雹,並賑之。澧州、歸州饑,賑糶米二萬石。是歲,夏,諸衛屯田及大都、河間、保定、濟南、濟寧五路屬縣霖雨傷稼。秋,忻州定襄縣及忠翊侍衛屯田所營田、象食屯田所隕霜殺禾。土番岷州春疫,夏旱。西番寇鞏昌府。



 泰定元年春正月乙未,以乃馬臺為平章政事,善僧為右丞。敕諸王哈剌還本部,召江西行省平章政事也兒吉你赴闕。己亥,以誅逆臣也先鐵木兒等詔天下。辛丑,諸王、大臣請立皇太子。賜諸王徹徹禿金一錠、銀六十錠、幣帛各百匹,塔思不花金一錠、銀四十錠、幣帛二百匹,阿忽鐵木兒等金銀各有差。壬寅,以故丞相拜住子答兒麻失里為宗仁衛親軍都指揮使,徹里哈為左右衛阿速親軍都指揮使。命僧諷西番經於光天殿。甲辰,敕譯《列聖制詔》及《大元通制》,刊本賜百官。丁未,以稱海屯田萬戶府達魯花赤帖陳假嶺北行中書省參知政事,近侍忽都帖木兒假禮部尚書,使西域諸王不賽因部。戊申,八番生蠻韋光正等及楊、黃五種人,以其戶二萬七千來附,請歲輸布二千五百匹,置長官司以撫之。己酉,命諸王遠徙者悉還其部。召親王圖帖睦爾于瓊州,阿木哥於大同。定怯薛臺歲給鈔,人八十錠。甲寅,賜諸王太平、忽剌臺、別失帖木兒等金印。敕高麗王還國,仍歸其印。糶米二十萬石,賑京師貧民。丙辰,賜故監察御史觀音保、鎖咬兒哈的迷失妻、子鈔各千錠。賜司徒道住印。敕封解州鹽池神曰靈富公。廣德、信州、岳州、惠州、南恩州民饑,發粟賑之。



 二月丁巳朔,作顯宗影堂。己未,修西番佛事於壽安山寺,曰星吉思吃剌,曰闊兒魯弗卜,曰水朵兒麻,曰颯間卜裡喃家,經僧四十人,三年乃罷。庚申,監察御史傅巖起、李嘉賓言:「遼王脫脫乘國有隙,誅屠骨肉,其惡已彰,恐懷疑貳,如令歸籓,譬之縱虎出柙。請廢之,別立近族以襲其位。」不報。甲子,作佛事,命僧百八人及倡優百戲,導帝師游京城。庚午,選守令、推官。舊制,臺憲歲舉守令、推官二人,有罪連坐,至是言其不便,復命中書於常選擇人用之。壬申,請上大行皇帝謚於南郊,曰睿聖文孝皇帝,廟號英宗。甲戌,江浙行省左丞趙簡,請開經筵及擇師傅,令太子及諸王大臣子孫受學,遂命平章政事張珪、翰林學士承旨忽都魯都兒迷失、學士吳澄、集賢直學士鄧文原,以《帝範》、《資治通鑒》、《大學衍義》、《貞觀政要》等書進講,復敕右丞相也先鐵木兒領之。諸王怯別、孛羅各遣使來貢。高昌王亦都護帖木兒補化遣使進蒲萄酒。丁丑,監察御史宋本、趙成慶、李嘉賓言:「盜竊太廟神主,由太常守衛不謹,請罪之。」不報。戊寅,御史李嘉賓劾逆黨左阿速衛指揮使脫帖木兒,罷之。癸未,宣諭也裏可溫各如教具戒。加封廣德路祠山神張真君曰普濟,寧國路廣惠王曰福祐。紹興、慶元、延安、岳州、潮州五路及鎮遠府、河州、集州饑,發粟賑之。



 三月丁亥朔,罷徽政院,立詹事院,以太傅朵臺、宣徽使禿滿迭兒、桓國公拾得驢、太尉丑驢答剌罕,並為太子詹事;中書參知政事王居仁為太子副詹事,以同知宣政院事楊廷玉為中書參知政事,罷大同路黃華嶺及崇慶屯田。賜壽寧公主金十錠、銀五十錠、鈔二萬錠。乙未,以江西行省平章政事也兒吉你知樞密院事。置定王薛徹干總管府。給蒙古流民糧、鈔,遣還所部,敕擅徙者斬,藏匿者杖之。賜諸王徹徹禿永福縣戶萬三千六百為食邑,仍置王傅。戊戌,廷試進士,賜八剌、張益等八十四人及第、出身有差;會試下第者,亦賜教官有差。中書省臣請禁橫奏賞賚及逾越奏事者,從之。庚子,欽察罷為陜西行臺御史大夫。以四川行中書省平章政事囊加臺兼宣政院使,往征西番寇參卜郎。癸卯,命中書平章政事乃馬臺攝祭南郊,知樞密院事闊徹伯攝祭太廟,以冊皇后、皇太子告。丙午,御大明殿,冊八八罕氏為皇后,皇子阿速吉八為皇太子。己酉,以皇子八的麻亦兒間卜嗣封晉王。泰寧王買奴卒,以其子亦憐真朵兒赤嗣。遣湘寧王八剌失裡出鎮察罕腦兒,罷宣慰司,立王傅府。以知樞密院事也兒吉你為雲南行省右丞相。召流人還京師。庚戌,月直延民真只海、阿答罕來獻大珠。監察御史宋本、李嘉賓、傅巖起言:「太尉、司徒、司空,三公之職,濫假僧人,及會福、殊祥二院,並辱名爵,請罷之。」不報。癸丑,諸王不賽因遣使朝貢。臨洮狄道縣,冀寧石州、離石、寧鄉縣旱,饑,賑米兩月。廣西橫州徭寇永淳縣。



 夏四月戊午,廉恂罷為集賢大學士,食其祿終身。賜乳母李氏鈔千錠,賜征參卜郎軍千人鈔四萬七千錠。太尉不花、平章政事即烈,坐矯制以寡婦古哈強配撒梯,被鞫,詔以世祖舊臣,原其罪。己未,以珠字詔賜帝師所居撒思加部。庚申,詔整飭御史臺。作昭獻元聖皇后御容殿於普慶寺。辛酉,命昌王八剌失里往鎮阿難答昔所居地。親王圖帖睦爾至自潭州,及王禪,皆賜車帳、駝馬。癸亥,以國言上英宗廟號曰格堅皇帝。修佛事於壽昌殿。甲子,車駕幸上都。以諸王寬徹不花、失剌,平章政事兀伯都剌,右丞善僧等居守。以嶺北行中書省左丞潑皮為中書左丞,江南行臺中丞朵朵為中書參知政事,馬剌罷為太史院使,罷衛士四百人還宗仁衛。賜北庭的撒兒兀魯軍羊馬。諸王不賽因遣使來貢。發兵民築渾河堤。丙寅,賜昌王八剌失里牛馬橐駝。稅僧、道邸舍積貨。丁卯,遣諸王捏古伯等還和林。封八剌失裏繼母買的為皇妹昌國大長公主,給銀印。以忽咱某丁為哈贊忽咱,主西域戶籍。辛未,月食既。癸酉,以太子詹事禿滿迭兒為中書平章政事。甲戌,命咒師作佛事厭雷。庚辰,以風烈、月食、地震,手詔戒飭百官。辛巳,太廟新殿成。木憐撒兒蠻部及北邊蒙古戶饑,賑糧、鈔有差。江陵路屬縣饑,雲南中慶、昆明屯田水。



 五月丁亥,監察御史董鵬南、劉潛、邊笥、慕完、沙班以災異上言:「平章乃蠻臺、宣徽院使帖木兒不花、詹事禿滿答兒黨附逆徒,身虧臣節,太常守廟不謹,遼王擅殺宗親,不花、即里矯制亂法,皆蒙寬宥,甚為失刑,乞定其罪,以銷天變。」不允。己丑,帝諭倒剌沙曰:「朕即位以來,無一人能執成法為朕言者。知而不言則不忠,且陷人於罪。繼自今,凡有所知,宜悉以聞,使朕明知法度,斷不敢自縱。非獨朕身,天下一切政務,能守法以行,則眾皆乂安,反是,則天下罹於憂苦。」又曰:「凡事防之於小則易,救之於大則難,爾其以朕言明告於眾,俾知所慎。」壬辰,御史臺臣禿忽魯、紐澤以御史言:「災異屢見,宰相宜避位以應天變,可否仰自聖裁。顧惟臣等為陛下耳目,有徇私違法者,不能糾察,慢官失守,宜先退避以授賢能。」帝曰:「御史所言,其失在朕,卿等何必遽爾!」禿忽魯又言:「臣已老病,恐誤大事,乞先退。」於是中書省臣兀伯都剌、張珪、楊廷玉皆抗疏乞罷。丞相旭邁傑、倒剌沙言:「比者災異,陛下以憂天下為心,反躬自責,謹遵祖宗聖訓,修德慎行,敕臣等各勤乃職,手詔至大都,居守省臣皆引罪自劾。臣等為左右相,才下識昏,當國大任,無所襄贊,以致災昆,罪在臣等,所當退黜,諸臣何罪。」帝曰:「卿若皆辭避而去,國家大事,朕孰與圖之?宜各相諭,以勉乃職。」戊戌,遷列聖神主於太廟新殿。辛丑,循州徭寇長樂縣。甲辰,赦上都囚笞罪以下者。丙午,太白犯鬼。侍御史高奎上書,請求直言,辨邪正,明賞罰,帝善其言,賜以銀幣。丁未,太白犯鬼積尸氣。己酉,賓州民方二等為寇,有司捕擒之。癸丑,命司天監翽星。中書平章政事禿滿迭兒、領宣徽使詹事丞回回,請如裕宗故事,擇名儒輔太子,敕中書省臣訪求以聞。袁州火,龍慶、延安、吉安、杭州、大都諸路屬縣水,民饑,賑糧有差。六月乙卯朔,遣諸王闊闊出鎮畏兀,賜金、銀、鈔千計。戊午,雲南蒙化州高蘭神場寨主照明羅九等寇威楚。庚申,張珪自大都至,以守臣集議事言:「逆黨未討,奸惡未除,忠憤未雪,冤枉未理,政令不信,賞罰不公,賦役不均,財用不節,請裁擇之。」不允。諸王阿木哥薨,賻鈔千錠。諸王寬徹、亦里吉赤來朝。賜駙馬鐵木兒等部鈔一萬三千錠,北邊戍兵鈔萬六千八十錠。賑蒙古饑民,遣還所部。延安路饑,禁酒。癸亥,作禮拜寺於上都及大同路,給鈔四萬錠。丙寅,遣使招諭參卜郎。遣闊闊出等詣高麗,取女子三十人。廣西左右兩江黃勝許、岑世興乞遣其子弟朝貢,許之。丁卯,大幄殿成,作鎮雷坐靜佛寺。庚午,置海剌禿屯田總管府。辛未,修黑牙蠻答哥佛事於水晶殿。癸酉,帝受佛戒於帝師。己卯,諸王怯別等遣其宗親鐵木兒不花等,奉馴豹、西馬來朝貢。詔:「疏決系囚,存恤軍士,免天下和買雜役三年,蜒戶差稅一年。百官四品以下,普覃散官一等,三品遞進一階。遠仕瘴地,身故不得歸葬,妻子流落者,有司資給遣還,仍著為令。」雲南大理路你囊為寇。大都,真定晉州、深州,奉元諸路及甘肅河渠營田等處,雨傷稼,賑糧二月。大司農屯田、諸衛屯田、彰德、汴梁等路雨傷稼,順德、大名、河間、東平等二十一郡蝗,晉寧、鞏昌、常德、龍興等處饑,皆發粟賑之。大同渾源河,真定滹沱河,陜西渭水、黑水,渠州江水皆溢,並漂民廬舍。宣德府、鞏昌路及八番金石番等處雨雹。河間、晉寧、涇州、揚州、壽春等路,湖廣、河南諸屯田皆旱。



 秋七月丙戌,思州平茶楊大車、酉陽州冉世昌寇小石耶、凱江等寨,調兵捕之。諸王阿馬薨,賻鈔五千錠。賜雲南王王禪鈔二千錠,諸王阿都赤鈔三千錠。作楠木殿。招諭船領、義寧、靈川等處徭。庚寅,遣使代祀岳瀆。丙申,以諸王薛徹禿襲統其父完者所部,仍給故印。己亥,賑蒙古流民,給鈔二十九萬錠,遣還,仍禁毋擅離所部,違者斬。庚子,諸王伯顏帖木兒出鎮闊連東部,阿剌忒納失裡出鎮沙州,各賜鈔三千錠。撒忒迷失率衛士佐太師按塔出行邊,賜鈔千錠。癸卯,罷廣州、福建等處採珠蜒戶為民,仍免差稅一年。丙午,以畏兀字譯西番經。丁未,翽星於上都司天監。以山東鹽運司判官馬合謨為吏部尚書,佩虎符,翰林修撰揚宗瑞為禮部郎中,佩金符,奉即位詔往諭安南。置長慶寺,以宦者阿亦伯為寺卿。罷中瑞司。中書省臣言:「東宮衛士,先朝止三千人,今增至萬七千,請命詹事院汰去,仍依舊制。」從之。戊申,以籍入鐵木迭兒及子班丹、觀音奴貲產給還其家。奉元路朝邑縣、曹州楚丘縣、大名路開州濮陽縣河溢,大都路固安州清河溢,順德路任縣沙、灃、洺水溢,真定、廣平、廬州等十一郡雨傷稼,龍慶州雨雹大如雞子,平地深三尺,定州屯河溢、山崩,免河渠營田租。大都、鞏昌、延安、冀寧、龍興等處饑,賑糶有差。廣西慶遠徭酋潘父絹等率眾來降,署為簿、尉等官有差。加封溫州故平陽侯曰英烈侯。



 八月甲寅,徹徹兒、火兒火思之地五千貧乏,賑糧二月。乙卯,敕以刑獄復隸宗正府,依世祖舊制,刑部勿與。丙辰,享太廟。丁巳,賜諸王八里臺、黃頭鈔各千五百錠。禁言赦前事。庚申,市牝馬萬匹取湩酒。賑帖列幹、木倫等驛戶糧、鈔有差。辛亥,遣翰林學士承旨斡赤祀太祖、太宗、睿宗御容於普慶寺。賜親王圖帖睦爾鈔三千錠。庚午,作中宮金脊殿。辛未,繪帝師八思巴像十一頒各行省,俾塑祀之。敕武官坐罪制授者以聞,敕授者從行省處決。以金泉館酒課賜公主壽寧。丁丑,罷浚玉泉山河役。車駕至大都。癸未,敕樞密役軍凡三百人以上奏聞。詔諭云南大車里、小車里。秦州成紀縣大雨,山崩,水溢,壅土至來穀河成丘阜。汴梁、濟南屬縣雨水傷稼,賑之。延安、冀寧、杭州、潭州等十二郡及諸王哈伯等部饑,賑糧有差。九月乙酉,封也速不堅為荊王,賜金印。以宣德府復隸上都留守司。辛卯,罷哈思的結魯思伴卜總統所,更置臨洮總管府。賜潛邸衛士鈔萬錠。丙申,葺太祖神御殿。乙巳,昭獻元聖皇后忌日修佛事飯僧萬萬人,敕存恤武衛軍一年。癸丑,以籍入阿散家貲給其子脫列。改邕州為南寧路。岑世興遣其弟興元來朝貢。奉元路長安縣大雨,灃水溢,延安路洛水溢,濮州館陶縣及諸衛屯田水,建昌、紹興二路饑,賑糧有差。



 冬十月乙卯,秦州成紀縣趙氏婦一產三男。成都嘉穀生一莖九穗。丁巳,監察御史王士元請早諭教太子,帝嘉納之。戊午,享太廟。立壽福總管府,秩正三品,典累朝神御殿祭祀及錢穀事,降大天源延聖寺總管府為提點所以隸之。庚申,命左、右相日直禁中,有事則赴中書。丙寅,太白犯鬥。己巳,太白入斗,太陰犯填星。雲南車里蠻為寇,遣斡耳朵奉詔招諭之。其酋塞賽子尼而雁、構木子刁零出降。庚午,太白犯鬥。壬申,安南國世子陳日爌遣其臣莫節夫等來朝貢。真州珠金沙河,松江府、吳江州諸河淤塞,詔所在有司傭民丁浚之。丙子,命帝師作佛事於延春閣。丁丑,緬國王子吾者那等爭立,歲貢不入,命雲南行省諭之。徙封雲南王王禪為梁王,食邑益陽州六萬五千戶,仍以其子帖木兒不花襲封雲南王。封親王圖帖睦爾為懷王,食邑瑞州六萬五千戶,增歲賜幣帛千匹,並賜金印。壬午,熒惑犯壘壁陣。肇慶徭黃寶才等降。延安路饑,發義倉粟賑之,仍給鈔四千錠。廣東道及武昌路江夏縣饑,賑糶有差。河南廉訪使買奴坐多征公田租免官。以魯國大長公主女適懷王。



 十一月己丑,命道士修醮事。癸巳,遣兵部員外郎宋本,吏部員外郎鄭立、阿魯灰,工部主事張成,太史院都事費著,分調閩海、兩廣、四川、雲南選。諸王不賽因言其臣出班有功,請官之,以出班為開府儀同三司、翊國公,給銀印、金符。賜諸王散術臺、也速速兒鈔各千五百錠,斡耳朵罕鈔千二百錠,魯賓鈔千五百錠。甲午,禜星於回回司天監。己亥,以術溫臺知樞密院事。辛丑,造金寶蓋,飾以七寶,貯佛舍利。甲辰,作歇山鹿頂樓於上都。丁未,釋笞四十七以下囚及輕罪流人,給鈔二千錠散與貧者。印明年鈔本至元鈔四十萬錠、中統鈔十萬錠。己酉,詔免也裏可溫、答失蠻差役。庚戌,招諭融州徭般領、大、小木龍等百七十五團。河間路饑,賑糧二月。汴梁、信州、泉州、南安、贛州等路饑,賑糶有差。嘉定路龍興縣饑,賑糧一月。大都、上都、興和等路十三驛饑,賑鈔八千五百錠。



 十二月癸丑朔,以岑世興為懷遠大將軍,遙授沿邊溪洞軍民安撫使,佩虎符,仍來安路總管;黃勝許為懷遠大將軍,遙授沿邊溪洞軍民安撫使,佩虎符,致仕,其子志熟襲為上思州知州。降詔宣諭,仍各賜幣帛二。乙卯,雲南徭阿吾及歪鬧為寇,行省督兵捕之。庚申,同州地震,有聲如雷。癸亥,鹽官州海水溢,屢壞堤障,侵城郭,遣使祀海神,仍與有司視形勢所便,還請疊石為塘,詔曰:「築塘是重勞吾民也,其增石囤捍禦,庶天其相之。」乙丑,給蒙古子女孳畜。丙寅,命翰林國史院修纂《英宗》、《顯宗實錄》。敕:「內外百官凡行朝賀等禮,雨雪免朝服。」庚午,熒惑犯外屏。辛未,新作棕殿成。諸王鎖思的薨,賻鈔五百錠。乙亥,太白經天。曲赦重囚三十八人,以為三宮祈福。夔路容米洞蠻田先什用等九洞為寇,四川行省遣使諭降五洞,餘發兵捕之。陜西行省以兵討階州土蕃。察罕腦兒千戶部饑,賑糧一月。延安路雹災,賑糧一月。溫州路樂清縣鹽場水,民饑,發義倉粟賑之。兩浙及江東諸郡水、旱,壞田六萬四千三百餘頃。



 二年春正月丙戌,辰星犯天雞。乙未,以畿甸不登,罷春畋。禁後妃、諸王、駙馬毋通星術之士,非司天官不得妄言禍福。敕:「御史臺選舉,與中書合議以聞。」中書省臣言:「江南民貧僧富,諸寺觀田土,非宋舊置並累朝所賜者,請仍舊制與民均役。」從之,以籍八思吉思地賜故監察御史觀音寶、鎖咬兒哈的迷失妻子,各十頃。戊戌,造象輦。參卜郎來降,賜其酋班術兒銀、鈔、幣、帛。辛丑,懷王圖帖睦爾出居於建康。壬寅,太白犯建星。甲辰,奉安顯宗像於永福寺,給祭田百頃。廣西山獠為寇,命所在有司捕之。江浙行省平章政事脫歡答剌罕升為左丞相。諸王怯別遣使貢方物,賜鈔四萬錠。戊申,以乞剌失思八班藏卜為土蕃等路宣慰使都元帥,兼管長河西、奔不兒亦思剛、察沙加兒、朵甘思、朵思麻等管軍達魯花赤,與其屬往鎮撫參卜郎。庚戌,詔諭宰臣曰:「向者卓兒罕察苦魯及山後皆地震,內郡大小民饑。朕自即位以來,惟太祖開創之艱,世祖混一之盛,期與人民共享安樂,常懷祗懼,災沴之至,莫測其由。豈朕思慮有所不及而事或僭差,天故以此示儆?卿等其與諸司集議便民之事,其思自死罪始,議定以聞。朕將肆赦,以詔天下。」肇慶、鞏昌、延安、贛州、南安、英德、新州、梅州等處饑,賑糶有差。



 閏月壬子朔,詔赦天下,除江淮創科包銀,免被災地差稅一年。庚申,修野狐嶺、色澤、桑乾嶺道。乙丑,命整治屯田。河南行省左丞姚煒請禁屯田吏蠶食屯戶,及勿務羨增以廢裕民之意,不報。丁卯,中書省臣言:「國用不足,請罷不急之費。」從之。置惠遠倉、永需庫於海剌禿總管府。己巳,修滹沱河堰。壬申,罷永興銀場,聽民採煉,以十分之二輸官。罷松江都水庸田使司,命州縣正官領之,仍加兼知渠堰事。癸酉,作棕毛殿。丙子,浙西道廉訪司言:「四方代祀之使,棄公營私,多不誠潔,以是神不歆格,請慎擇之。」山南廉訪使帖木哥請削降鐵失所用驟升官。戊寅,諸王忽塔梯迷失等來朝,賜金、銀、鈔、帛有差。己卯,河間、真定、保定、瑞州四路饑,禁釀酒。階州土蕃為寇,鞏昌總帥府調兵御之。站八兒監藏叛於兀敦。保定路饑,賑鈔四萬錠、糧萬五千石。雄州歸信諸縣大雨,河溢,被災者萬一千六百五十戶,賑鈔三萬錠。南賓州、棣州等處水,民饑,賑糧二萬石,死者給鈔以葬。五花城宿滅禿、拙只乾、麻兀三驛饑,賑糧二千石。衡州衡陽縣民饑,瑞州蒙山銀場丁饑,賑粟有差。山東廉訪使許師敬請頒族葬制,禁用陰陽相地邪說。



 二月甲申,祭先農。丙戌,頒《道經》於天下名山宮觀。丁亥,平伐苗酋的娘率其戶十萬來降,土官三百六十人請朝。湖廣行省請汰其眾還部,令的娘等四十六人入覲,從之。己丑,加嗣漢三十九代天師張嗣成太玄輔化體仁應道大真人。庚寅,熒惑、歲星、填星聚於畢。辛卯,賑安定王朵兒只班部軍糧三月。瓜哇國遣其臣昔剌僧迦里也奉表及方物來朝貢。廣西徭潘寶陷柳城縣。丁酉,翽星於回回司天監。己亥,命西僧作燒壇佛事於延華閣。封阿裏迷失為和國公、張珪為蔡國公,仍知經筵事。以中書右丞善僧為平章政事,參知政事潑皮為右丞;御史大夫禿忽魯加太保,仍御史大夫。庚子,姚煒以河水屢決,請立行都水監於汴梁,仿古法備捍,仍命瀕河州縣正官皆兼知河防事,從之。丙午,造玉御床。戊申,命道士祭五福太一神。庚戌,通、漷二州饑,發粟賑糶。薊州、寶坻縣、慶元路象山諸縣饑,賑糧二月。甘州蒙古驛戶饑,賑糧三月。大都、鳳翔、寶慶、衡州、潭州、全州諸路饑,賑糶有差。三月癸丑,修曹州濟陰縣河堤,役民丁一萬八千五百人。甲寅,禁捕天鵝。丁巳,賜諸王帖木兒不花等鈔有差。辛酉,咸平府清河、寇河合流,失故道,壞堤堰,敕蒙古軍千人及民丁修之。乙丑,車駕幸上都。諸王搠思班部戰士四百人征參卜郎有功,人賞鈔四千錠。乙亥,安南國世子陳日爌遣使貢方物。荊門州旱,漷州、薊州、鳳州、延安、歸德等處民及山東蒙古軍饑,賑糧、鈔有差。肇慶、富州、惠州、袁州、江州諸路及南恩州、梅州饑,賑糶有差。



 夏四月丁亥,作吾殿。癸巳,和市牝馬有駒者萬匹。敕宿衛駝馬散牧民間者,歸官廄飼之。丁酉,濮州鄄城縣言城西堯塚上有佛寺,請徙之,不報。辛丑,加公主壽寧為皇姊大長公主。禁山東諸路酒。丙午,僰夷及蒐雁遮殺雲南行省所遣諭蠻使者,敕追捕之。丁未,封后父火裏兀察兒為威靖王。戊申,以許師敬為中書左丞;中政使馮亨為中書參知政事,仍中政使。奉元路白水縣雹。鞏昌路伏羌縣大雨,山崩。鎮江、寧國、瑞州、桂州、南安、寧海、南豐、潭州、涿州等處饑,賑糧五萬餘石。隴西、漢中、秦州饑,賑鈔三萬錠。



 五月壬子,車裏陶剌孟及大阿哀蠻兵萬人乘象寇陷朵剌等十四寨,木邦路蠻八廟率僰夷萬人寇陷倒八漢寨,督邊將嚴備之。癸丑,龍牙門蠻遣使奉表貢方物。辛未,罷京師官鬻鹽肆十五。改河間鹽運司為大都河間等路都轉運鹽使司。遣察乃使於周王和世束。癸酉,融州否泉洞、吉龍洞、洞村山、黑江諸徭為寇,廣西元帥府發兵討之。丙子,旭邁傑等以國用不足,請減廄馬,汰衛士,及節諸王濫賜,從之。賜潛邸怯憐口千人鈔三萬錠。浙西諸郡霖雨,江湖水溢,命江浙行省及都水庸田司興役疏洩之。置諫議書院於昌平縣,祀唐劉蕡。大都路檀州大水,平地深丈有五尺,汴梁路十五縣河溢,江陵路江溢,洮州、臨洮府雨雹,潭州、興國屬縣旱,彰德路蝗,龍興、平江等十二郡饑,賑糶米三十二萬五千餘石。鞏昌路臨洮府饑,賑鈔五萬五千錠。六月己卯朔,皇子生,命巫祓除於宮。葺萬歲山殿。靜江徭為寇,遣廣西宣慰司發兵捕之。辛巳,柳州徭為寇,戍兵討斬之。癸未,潯州平甫縣徭為寇,達魯花赤都堅、都監姚泰亨死之。甲申,改封嘉王晃火帖木兒為並王。丙戌,填星犯井鉞星。丙申,中書參知政事左塔不臺言:「大臣兼領軍務,前古所無。鐵失以御史大夫,也先帖木兒以知樞密院事,皆領衛兵,如虎而翼,故成逆謀。今軍衛之職,乞勿以大臣領之,庶勛舊之家得以保全。」從之,仍賜幣帛以旌其直。丁酉,靜江義寧縣及慶遠安撫司蠻徭為寇,敕守將捕之。息州民趙丑廝、郭菩薩,妖言彌勒佛當有天下,有司以聞,命宗正府、刑部、樞密院、御史臺及河南行省官雜鞫之。辛丑,柳州馬平縣徭為寇,湖廣行省督所屬追捕之。丙午,填星犯井。丁未,立都水庸田使司,浚吳、松二江。敕營造毋役五衛軍士,止以武衛、虎賁二衛給之。開南州阿只弄、哀培蠻兵為寇,命雲南行省督所屬兵捕之。通州三河縣大雨,水丈餘,潼川府綿江、中江水溢入城郭,冀寧路汾河溢。秦州秦安山移。新州路旱,濟南、河間、東昌等九郡蝗,奉元、衛輝路及永平屯田豐贍、昌國、濟民等署雨傷稼,蠲其租。濟寧、興元、寧夏、南康、歸州等十二郡饑,賑糶米七萬餘石。鎮西武靖王部及遼陽水達達路饑,賑糧一月。



 秋七月戊申朔,大、小車里蠻來獻馴象。己酉,賜諸王燕大等金、鈔有差。庚戌,遣阿失伯祀宅神於北部行幄。甲寅,遣使奉詔分諭徭蠻,鎮康路土官你囊、謀粘路土官賽丘羅出降;木邦路土官八廟既降復叛。翽星於上都司天監。紐澤、許師敬編類《帝訓》成,請於經筵進講,仍俾皇太子觀覽,有旨譯其書以進。丙辰,享太廟。播州蠻黎平愛等集群夷為寇,湖廣行省請兵討之,不許,詔播州宣撫使楊也裏不花招諭之。戊午,遣使代祀龍虎、武當二山。己未,置車里軍民總管府,以土人寒賽為總管,佩金虎符。中書省臣言:「往歲征徭,廉訪司劾其濫殺,今凡出師,請廉訪司官一員蒞軍糾正。」從之。庚申,以宮人二賜籓王怯別。癸亥,修大乾元寺。以許師敬及郎中買驢兼經筵官。廣西諸徭寇城邑,遣湖廣行省左丞乞住、兵部尚書李大成、中書舍人買驢將兵二萬二千人討之,仍以諸王斡耳朵罕監其軍。海北徭酋盤吉祥寇陽春縣,命江西行省督兵捕之。庚午,以國用不足,罷書金字《藏經》。威楚、大理諸蠻為寇,雲南行省請出師,不允,遣亦剌馬丹等使大理,普顏實立等使威楚,招諭之。思州洞蠻楊銀千等來獻方物。封駙馬孛羅帖木兒、知樞密院事火沙並為郡王。辛未,立河南行都水監。申禁漢人藏執兵仗,有軍籍者,出征則給之,還,復歸於官。壬申,御史臺臣言:「廉訪司蒞軍,非世祖舊制。賈胡鬻寶,西僧修佛事,所費不支,於國無益,並宜除罷。」從之。敕太傅朵臺、太保禿忽魯日至禁中集議國事。徭蠻潘寶寇鐔津、義寧、來賓諸縣,命廣西守將捕之。慶遠溪洞民饑,發米二萬五百石,平價糶之。敕山東州縣收養流民遺棄子女。延安、鄜州、綏德、鞏昌等處雨雹,般陽新城縣蝗,宗仁衛屯田隕霜殺禾,睢州河決,順德、汴梁、德安、汝寧諸路旱,免其租。梅州、饒州、鎮江、邠州諸路饑,賑糶米三萬餘石。八月戊子,修上都香殿。辛卯,雲南白夷寇雲龍州。癸巳,歲星犯天樽。辛丑,遣使代祀岳瀆名山大川。敕:「諸王私入京者,勿供其所用;諸部曲宿衛私入京者,罪之。」命度支監汰阿塔赤所掌駝馬,於外郡飼之。大都路檀州、鞏昌府靜寧縣、延安路安塞縣雨雹,衛輝路汲縣河溢。南恩州、瓊州饑,賑糧一月。臨江路、歸德府饑,賑糧二月。衡州、建昌、岳州饑,賑糶米一萬三千石。



 九月戊申朔,分天下為十八道,遣使宣撫。詔曰:「朕祗承洪業,夙夜惟寅,凡所以圖治者,悉遵祖宗成憲。曩屢詔中外百司,宣布德澤,蠲賦詳刑,賑恤貧民,思與黎元共享有生之樂。尚慮有司未體朕意,庶政或闕,惠澤未洽,承宣者失於撫綏,司憲者怠於糾察,俾吾民重困,朕甚憫焉。今遣奉使宣撫,分行諸道,按問官吏不法,詢民疾苦,審理冤滯,凡可以興利除害,從宜舉行。有罪者,四品以上停職申請,五品以下就便處決。其有政績尤異,暨晦跡丘園,才堪輔治者,具以名聞。」以湖廣行省參知政事馬合某、河東宣慰使李處恭之兩浙江東道,江東道廉訪使朵列禿、太史院使齊履謙之江西福建道,都功德使舉林伯、荊湖宣慰使蒙弼之江南湖廣道,禮部尚書李家奴、工部尚書硃蕡之河南江北道,同知樞密院事阿吉剌、御史中丞曹立之燕南山東道,太子詹事別帖木兒、宣徽院判韓讓之河東陜西道,吏部尚書納哈出、董訥之山北遼東道,陜西鹽運使眾家奴、中書斷事官韓庭茂之云南省,湖南宣慰使寒食、冀寧路總管劉文之甘肅省,山東宣慰使禿思帖木兒、陜西行省左丞廉惇之四川省,翰林侍講學士帖木兒不花、秘書卿吳秉道之京畿道。以郡縣饑,詔運粟十五萬石貯瀕河諸倉,以備賑救,仍敕有司治義倉。禁大都、順德、衛輝等十郡釀酒。募富民入粟拜官,二千石從七品,千石正八品,五百石從八品,三百石正九品,不願仕者旌其門。諸王斡即遣使貢金浮圖。己酉,海運江南糧百七十萬石至京師。庚戌,復尚乘寺、光祿寺為正三品,給銀印。癸丑,車駕至大都。遣使祀海神天妃。甲寅,禁饑民結扁簷社,傷人者杖一百,著為令。乙卯,享太廟。己未,岑世興上言,自明不反,請置蒙古、漢人監貳官,詔優從之。壬戌,諸王牙即貢馬。丁丑,浚河間陳玉帶河。廣西徭寇賓州。禮部員外郎元永貞言:「鐵失弒逆,皆由鐵木迭兒始禍,請明其罪,仍錄付史館,以為人臣之戒。」漢中道文州霖雨,山崩。檀州雨雹,開元路三河溢,瓊州、南安、德慶諸路饑,賑糧、鈔有差。



 冬十月戊寅朔,張珪歸保定上塚,以病辭祿,不允。岑世興及子鐵木兒率眾寇上林等州,命撫諭之。壬午,禁成都路釀酒。癸未,以倒剌沙為御史大夫。丁亥,享太廟。己丑,賜恩平王塔思不花部鈔五千錠。壬辰,熒惑犯氐。癸巳,填星退犯井。播州凱黎苗率諸寨苗、獠為寇。乙未,皇后亦憐真八剌受佛戒於帝師。丁酉,廣西獠酋何童降,請防邊自效,從之。乙巳,寧遠知州添插言,安南國土官押那攻掠其木末諸寨,請治之,敕安南世子諭押那歸其俘。丙辰,寧夏路、曹州屬縣水,霸州、衢州路饑,賑糧二月。



 十一月戊申,周王和世束遣使以豹來獻。改長寧軍為州。庚戌,旭邁傑以歲饑請罷皇后上都營繕,從之。紐澤以病乞罷,不允。丙辰,郭菩薩等伏誅,杖流其黨。丁巳,幸大承華普慶寺,祀昭獻元聖皇后於影堂,賜僧鈔千錠。岑世興結八番蠻班光金等合兵攻石頭等寨,敕調兵御之。八番宣慰司官失備坐罪。戊午,填星退犯井宿鉞星。己未,詔整飭臺綱。庚申,倭舶來互市。廣西道宣慰使獲徭酋潘寶下獄,其弟潘見遂寇柳州,命湖廣行省左丞乞住捕之。壬戌,敕軍民官廕襲者,由本貫圖宗支,申請銓授。丙寅,倒剌沙復為中書左丞相,加開府儀同三司、錄軍國重事。丁卯,罷蒙山銀冶提舉司,命瑞州路領之。壬申,賜諸王不賽因鈔二萬錠、帛百匹。諸王斡耳朵罕遣使以追捕廣西徭寇上聞,帝曰:「朕自即位,累詔天下憫恤黎元,惟廣徭屢叛,殺掠良民,故命斡耳朵罕等討之。今聞迎降者甚眾,宜更以恩撫之。若果不悛,嚴兵追捕。」京師饑,賑糶米四十萬石。內郡饑,賑鈔十萬錠、米五萬石。河間諸郡流民就食通、漷二州,命有司存恤之。杭州路火,賑貧民糧一月。常德路水,民饑,賑糧萬一千六百石。



 十二月戊寅,以塔失帖木兒為中書右丞相。癸未,加塔失帖木兒開府儀同三司、上柱國、錄軍國重事、監修國史,封薊國公。諸王不賽因遣使貢珠,賜鈔二萬錠。乙酉,帝復受佛戒於帝師。熒惑犯天江,辰星犯建星。丁亥,修鹿頂殿。鎮南王脫不花薨,遣中書平章政事乃馬歹攝鎮其地。中書省臣言山東、陜西、湖廣地接戎夷,請議選宗室往鎮,從之。申禁圖讖,私藏不獻者罪之。癸巳,京師多盜,塔失帖木兒請處決重囚,增調邏卒,仍立捕盜賞格,從之。甲午,太白犯壘壁陣。召張珪於保定。丁酉,加紐澤知樞密院事,與馬某沙並開府儀同三司。弛瑞州路酒禁。左丞乞住、諸王斡耳朵罕征徭賊,敗之。元江路土官普山為寇,命戍兵捕之。壬寅,大寧路鳳翔府饑,禁釀酒。右丞趙簡請行區田法於內地,以宋董煟所編《救荒活民書》頒州縣。濟南、延川二路饑,賑鈔三千五百錠。惠州、杭州等處饑,賑糶有差。是歲,陜西府雨雹,御河水溢。以故翰林學士不花、中政使普顏篤、指揮使卜顏忽里為鐵失等所系死,贈功臣號及階勛爵謚。



\end{pinyinscope}