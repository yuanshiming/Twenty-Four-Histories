\article{志第四十七 兵二}

\begin{pinyinscope}

 ○宿衛



 宿衛者,天子之禁兵也。元制,宿衛諸軍在內,而鎮戍諸軍在外,內外相維,以制輕重之勢,亦一代之良法哉。方太祖時,以木華黎、赤老溫、博爾忽、博爾術為四怯薛,領怯薛歹分番宿衛。及世祖時,又設五衛,以象五方,始有侍衛親軍之屬,置都指揮使以領之。而其後增置改易,於是禁兵之設,殆不止於前矣。夫屬枿鞬,列宮禁,宿衛之事也,而其用非一端。用之於大朝會,則謂之圍宿軍;用之於大祭祀,則謂之儀仗軍;車駕巡幸用之,則曰扈從軍;守護天子之帑藏,則曰看守軍;或夜以之警非常,則為巡邏軍;或歲漕至京師用之以彈壓,則為鎮遏軍。今總之為宿衛,而以餘者附見焉。



 四怯薛:太祖功臣博爾忽、博爾術、木華黎、赤老溫,時號掇裏班曲律,猶言四傑也,太祖命其世領怯薛之長。怯薛者,猶言番直宿衛也。凡宿衛,每三日而一更。申、酉、戌日,博爾忽領之,為第一怯薛,即也可怯薛。博爾忽早絕,太祖命以別速部代之,而非四傑功臣之類,故太祖以自名領之。其雲也可者,言天子自領之故也。亥、子、丑日,博爾術領之,為第二怯薛。寅、卯、辰日,木華黎領之,為第三怯薛。巳、午、未日,赤老溫領之,為第四怯薛。赤老溫後絕,其後怯薛常以右丞相領之。



 凡怯薛長之子孫,或由天子所親信,或由宰相所薦舉,或以其次序所當為,即襲其職,以掌環衛。雖其官卑勿論也,及年勞既久,則遂擢為一品官。而四怯薛之長,天子或又命大臣以總之,然不常設也。其它預怯薛之職而居禁近者,分冠服、弓矢、食飲、文史、車馬、廬帳、府庫、醫藥、卜祝之事,悉世守之。雖以才能受任,使服官政,貴盛之極,然一日歸至內庭,則執其事如故,至於子孫無改,非甚親信,不得預也。



 其怯薛執事之名:則主弓矢、鷹隼之事者,曰火兒赤、昔寶赤、怯憐赤。書寫聖旨,曰扎里赤。為天子主文史者,曰必闍赤。親烹飪以奉上飲食者,曰博爾赤。侍上帶刀及弓矢者,曰雲都赤、闊端赤。司閽者,曰八剌哈赤。掌酒者,曰答剌赤。典車馬者,曰兀剌赤、莫倫赤。掌內府尚供衣服者,曰速古兒赤。牧駱駝者,曰帖麥赤。牧羊者,曰火你赤。捕盜者,曰忽剌罕赤。奏樂者,曰虎兒赤。又名忠勇之士,曰霸都魯。勇敢無敵之士,曰拔突。其名類蓋不一,然皆天子左右服勞侍從執事之人,其分番更直,亦如四怯薛之制,而領於怯薛之長。



 若夫宿衛之士,則謂之怯薛歹,亦以三日分番入衛。其初名數甚簡,後累增為萬四千人。揆之古制,猶天子之禁軍。是故無事則各執其事,以備宿衛禁庭;有事則惟天子之所指使。比之樞密各衛諸軍,於是為尤親信者也。



 然四怯薛歹,自太祖以後,累朝所御斡耳朵,其宿衛未嘗廢。是故一朝有一朝之怯薛,總而計之,其數滋多,每歲所賜鈔幣,動以億萬計,國家大費每敝於此焉。



 右衛:中統三年,以侍衛親軍都指揮使董文炳兼山東東路經略使,共領武衛軍事。命益都行省大都督撒吉思驗壬子年已定民籍,及照李鋋總籍軍數,每千戶內選練習軍士二人充侍衛軍,並海州、東海、漣州三處之軍屬焉。至元元年,改武衛為侍衛親軍,分左右翼,置都指揮使。八年,改立左、右、中三衛,掌宿衛扈從,兼屯田,國有大事,則調度之。



 左衛、中衛:並至元八年侍衛親軍改立。



 前衛:至元十六年,以侍衛親軍創置前、後二衛,掌宿衛扈從,兼營屯田,國有大事,則調度之,置都指揮使。



 後衛:亦至元十六年置。



 武衛:至元二十五年,尚書省奏,那海那的以漢軍一萬人,如上都所立虎賁司,營屯田,修城隍。二十六年,樞密院官暗伯奏,以六衛六千人,塔剌海孛可所掌大都屯田三千人,及近路迤南萬戶府一千人,總一萬人,立武衛親軍都指揮使司,掌修治城隍及京師內外工役之事。



 左都威衛:至元十六年,世祖以新取到侍衛親軍一萬戶,屬之東宮,立侍衛親軍都指揮使司。三十一年,復以屬皇太后,改隆福宮左都威衛使司。至大三年,選其軍之善造作者八百人,立千戶所一及百戶翼八以掌之,而分局造作。皇慶元年,以王平章舊所領軍一千人,立屯田。至治三年,罷匠軍千戶所。



 右都威衛:國初,木華黎奉太祖命,收扎剌兒、兀魯、忙兀、納海四投下,以按察兒、孛羅、笑乃鷿、不里海拔都兒、闊闊不花五人領探馬赤軍。既平金,隨處鎮守。中統三年,世祖以五投下探馬赤立蒙古探馬赤總管府。至元十六年,罷其軍,各於本投下應役。十九年,仍令充軍。二十一年,樞密院奏,以五投下探馬赤軍俱屬之東宮,復置官屬如舊。二十二年,改蒙古侍衛親軍指揮使司。三十一年,改隆福宮右都威衛使司。



 唐兀衛:至元十八年,阿沙、阿束言:「今年春,奉命總領河西軍三千人,但其所帶虎符金牌者甚眾,征伐之重,若無官署,何以防閑之。」樞密院以聞,遂立唐兀衛親軍都指揮使司以總之。



 貴赤衛:至元二十四年立。



 西域親軍:元貞元年,依貴赤、唐兀二衛例,始立西域親軍都指揮使司。



 衛候直都指揮使司:至元元年,裕宗招集控鶴一百三十五人。三十一年,徽政院增控鶴六十五人,立衛候司以領之,且掌儀從金銀器物。元貞元年,皇太后復以晉王校尉一百人隸焉。大德十一年,益以懷孟從行控鶴二百人,升衛候直都指揮使司。至大元年,復增控鶴百人,總六百人,設百戶所六,以為其屬。至治三年罷之。四年,以控鶴六百三十人,歸於皇后位下,後復置立。



 右阿速衛:至元九年,初立阿速拔都達魯花赤,後招集阿速正軍三千餘名,復選阿速揭只揭了溫怯薛丹軍七百人,扈從車駕,掌宿衛城禁,兼營潮河、蘇沽兩川屯田,並供給軍儲。二十三年,為阿速軍南攻鎮巢,殘傷者眾,遂以鎮巢七百戶屬之,並前軍總為一萬戶,隸前後二衛。至大二年,始改立右衛阿速親軍都指揮使司。



 左阿速衛:亦至大二年改立。



 隆鎮衛:睿宗在潛邸,嘗於居庸關立南、北口屯軍,徼巡盜賊,各設千戶所。至元二十五年,以南、北口上千戶所總領之。至大四年,改千戶所為萬戶府,分欽察、唐兀、貴赤、西域、左右阿速諸衛軍三千人,並南、北口、太和嶺舊隘漢軍六百九十三人,屯駐東西四十三處,立十千戶所,置隆鎮上萬戶府以統之。皇慶元年,始改為隆鎮衛親軍都指揮使司。延祐二年,又以哈兒魯軍千戶所隸焉。至治元年,置蒙古、漢軍籍。



 左衛率府:至大元年,命以中衛兵萬人立衛率府,屬之東宮。時仁宗為皇太子,曰:「世祖立五衛,象五方也,其制猶中書之六部,殆不可易。」遂命江南行省萬戶府,選漢軍之精銳者一萬人,為東宮衛兵,立衛率府。延祐四年,改為中翊府,未幾復改為御臨親軍都指揮使司,又以御臨非古典,改為羽林。六年,英宗立為皇太子,復以隸東宮,仍為左衛率府。



 右衛率府:延祐五年,以詹事禿滿迭兒所管速怯那兒萬戶府,及迤東、女直兩萬戶府,右翼屯田萬戶府兵,合為右衛率府,隸皇太子位下。



 康禮衛:武宗至大三年,定康禮軍籍。凡康禮氏之非者,皆別而黜之,驗其實,始得入籍。及諸侯王阿只吉、火郎撒所領探馬赤,屬康禮氏者,令樞密院康禮衛遣人乘傳,往置籍焉。



 忠翊侍衛:至元二十九年,始立屯田府。大德十一年,增軍數,立為大同等處侍衛親軍都指揮使司。至大四年四月,皇太后修五臺寺,遂移屬徽政院,並以京兆軍三千人增入。延祐元年,改中都威衛使司,仍隸徽政院。至治元年,始改為忠翊侍衛親軍都指揮使司。



 宗仁衛:至治二年,右丞相拜住奏:「先脫別鐵木叛時,沒入亦乞列思人一百戶,與今所收蒙古子女三千戶,清州徹匠二千戶,合為行軍五千,請立宗仁衛以統之。」於是命右丞相拜住總衛事,給降虎符牌面,如右衛率府,又置行軍千戶所隸焉。



 右欽察衛:至元二十三年,依河西等衛例,立欽察衛。至治二年,分為左右兩衛。天歷二年,以本衛屬大都督府。



 左欽察衛:亦至治二年立。始至元中立衛時,設行軍千戶十有九所,屯田三所。大德中,置只兒哈郎、鐵哥納兩千戶所。至大元年,復設四千戶所。至是始分為左右二衛,亦屬大都督府。



 龍翊侍衛:天歷元年十二月,立龍翊衛親軍都指揮使司,以左欽察衛唐吉失等九千戶隸焉。



 虎賁親軍都指揮使司。



 左翊蒙古侍衛親軍都指揮使司。



 右翊蒙古侍衛親軍都指揮使司。



 宣忠斡羅思扈衛親軍都指揮使司。



 威武阿速衛親軍都指揮使司。



 東路蒙古侍衛親軍都指揮使司。



 女直侍衛親軍萬戶府。



 高麗女直漢軍萬戶府管女直侍衛親軍萬戶府。



 鎮守海口侍衛親軍屯儲都指揮使司。



 宣鎮侍衛。



 世祖中統元年四月,諭隨路管軍萬戶,有舊從萬戶三哥西征軍人,悉遣至京師充防城軍:忙古鷿軍三百一十九人,嚴萬戶軍一千三百四十五人,濟南路軍一百四十人,脫赤剌軍一百四十九人,查剌軍一百四十五人,馬總管軍一百四十四人。



 三年十月,諭益都大小管軍官及軍人等:「先李鋋懷逆,蒙蔽朝廷恩命,驅駕爾等以為己惠,爾等雖有效過功勞,殊無聞報,一旦泯絕,此非爾等不忠之愆,實李鋋懷逆之罪也。今侍衛親軍都指揮使董文炳來奏其詳,言爾等各有願為朝廷出力之語,此復見爾等存忠之久也。今命董文炳仍為山東東路經略使,收集爾等,直隸朝廷,充武衛軍近侍勾當。比及應職,且當守把南邊,堤防外隙,庶內境軍民各得安業。爾等宜益盡心,以圖勛效。」



 至元二年十二月,增侍衛親軍一萬人,內選女直軍三千,高麗軍三千,阿海三千,益都路一千。每千人置千戶一員,百人置百戶一員,以領之。仍選丁力壯銳者,以應役焉。



 三年五月,帝謂樞密臣曰:「侍衛親軍,非朕命不得發充夫役。修瓊華島士卒,即日放還。」



 四年七月,諭東京等路宣撫司,命於所管戶內,以十等為率,於從上第三等戶,簽選侍衛親軍一千八百名。若第三等戶不敷,於第二等戶內簽補。仍定立千戶、百戶、牌子頭,並其家屬同來,赴中都應役。



 十四年五月,以蒙古軍與漢軍相參,備都城內外及萬壽山宿衛,仍以也速不花領圍宿事。



 十五年五月,總管胡翔請還侍衛軍。先是,宿州蘄縣等萬戶府士卒百人,有旨俾充侍衛軍,後從僉省嚴忠範征西川,既而嘉定、重慶、夔府皆下,忠範回軍,留西道。翔上言,從之。九月,以總管張子良所匿軍二千二百三十二人,充侍衛軍士。



 十六年四月,選揚州省新附軍二萬人,充侍衛親軍,並其妻子,遷赴京師。



 二十四年十月,總帥汪惟和選麾下銳卒一千人,請擇昆弟中一人統之,以備侍衛,從之。



 成宗元貞四年八月,詔:「蒙古侍衛所管探馬赤軍人子弟,投充諸王位下身役者,悉遵世祖成憲,發還元役充軍。」



 大德六年二月,調蒙古侍衛等軍一萬人,往官山住夏。



 仁宗延祐六年九月,知樞密院事塔失鐵木兒言:「諸漢人不得點圍宿軍士,圖籍系軍數者,雖御史亦不得預知,此國制也。比者,領圍宿官言,中書命司計李處恭巡視守倉庫軍卒,有曠役者則罪之,以懲其後,使無怠而已。而李司計擅取軍數,士卒,在法為過。臣等議,宜自中書與樞密遣人案之,驗實以聞。」制可。七年六月,以紅城中都威衛系掌軍務之司,屬徽政院不便,命遵舊制,俾樞密總之。



 圍宿軍



 世祖至元二十六年七月,命大都侍衛軍內,復起一萬人赴上都,以備圍宿。



 成宗元貞二年十月,樞密院臣言:「昔大朝會時,皇城外皆無墻垣,故用軍環繞,以備圍宿。今墻垣已成,南北西三畔皆可置軍,獨御酒庫西,地窄不能容。臣等與丞相完澤議,各城門以蒙古軍列衛,及於周橋南置戍樓,以警昏旦。」從之。



 武宗至大四年正月,省臣等傳皇太子命,以大朝會調蒙古、漢軍三萬人備圍宿,仍遣使發山東、河北、河南、淮北諸路軍至京師。復命都府、左右翼、右都威衛整器仗車騎。六月,以諸侯王、駙馬等來朝,命發各衛色目、漢軍八百二十六人至上京,復命指揮使也乾不花領之。



 仁宗皇慶元年六月,命衛率府軍士備圍宿,守隆福宮內外禁門。十一月,樞密院臣言:「皇太后有旨,禁掖門可嚴守衛。臣等議,增置百戶一員,及於欽察、貴赤、西域、唐兀、阿速等衛調軍士九十人,增守諸掖門,復命千戶一員,帥領百戶一員,備巡邏。」從之。延祐三年十月以諸侯王來朝,命圍宿軍士六千人增至一萬人;復命也了乾、禿魯分左右部領其事。十一月,詔圍宿軍士,除舊有者,更增色目軍萬人,以備禁衛。十二月,樞密院臣言:「圍宿軍士不及數,其已發各衛者,地遠至不能如期,可遷刈葦草及青塔寺工役軍先備守衛。其各衛還家軍士,亦發二萬五千人,令備車馬器械,俱會京師。」制可。六年閏八月,命知樞密院事眾嘉領圍宿,發五衛軍代羽林軍士,仍以千戶二員、百戶十員,擇士卒精銳者二百人屬之。



 英宗至治元年正月,帝詣石佛寺,以其墻垣疏壞,命副樞術溫臺、僉院阿散領圍宿士卒,以備巡邏。八月,東內皇城建宿衛屋二十五楹,命五衛內摘軍二百五十人居之,以備禁衛。



 文宗天歷二年二月,樞密院臣言:「去歲嘗奉旨,依先制調軍守把圍宿,此時各翼軍人,皆隨處出征,亦有潰散者,故不及依次調遣,止於右翼侍衛及右都威衛內,發軍一千一百二十六名以備圍宿。今歲車駕行幸,臣等議於河南、山東兩都府內,起遣未差軍士一千三名,以備扈從。」制可。五月,樞密臣又言:「比奉令旨,放散軍人。臣等議,常制以三月一日放散,六月一日赴限,今放散既遲,可令於八月一日赴限。」從之。



 儀仗軍



 世祖至元十二年十二月,上尊號、受冊,告祭天地宗廟,調左、右、中三衛軍五十人為蹕街清路軍。



 武宗至大二年十二月,上尊號,百官行朝賀禮,樞密院調軍一千人備儀仗。三年十月,上皇太后尊號,行冊寶禮,用內外儀仗軍數,及防護五色甲馬軍二百人。四年二月,合祭天地、太廟、社稷,用蹕街清道及守內外壝門軍一百八十人,命以圍宿軍為之,事畢還役。七月,以奉迎武宗玉冊祔廟,用清路蹕街軍一百五十人,管軍千戶、百戶各一員。九月,以祭享太廟,用蹕街清路軍一百五十人,千戶、百戶各一員。



 仁宗皇慶元年三月,天壽節行禮,用內外儀仗軍一千人。



 英宗至治元年十一月,命有司選控鶴衛士,及色目、漢軍以備鹵簿儀仗。十二月,定鹵簿隊仗,用軍士二千三百三十人,萬戶、千戶、百戶四十五員。仍議用軍士一千九百五十人,萬戶、千戶、百戶五十九員,以備儀仗。



 致和元年六月,以享太廟,用蹕街清路軍一百名,看糝盆軍一百名,管軍官千戶、百戶各一員。九月,行大禮,用擎執儀仗蒙古、漢軍一千名。



 文宗天歷元年十一月,親祭太廟,內外用儀仗並五色甲馬軍一千六百五十名,仍命指揮青山及洪副使攝折沖都尉提調。二年,正旦行禮,用儀仗軍一千人。享太廟,用蹕街清路軍一百名,看守糝盆軍一百名,管軍千戶、百戶各一員。天壽節行禮,用儀仗軍一千名。皇后冊寶擎執儀仗,用軍一千二百名,軍官四員。



 扈從軍



 世祖至元十七年三月,發忙古鷿、抄兒赤所領河西軍士,及阿魯黑麾下二百人,入備扈從。



 武宗至大二年,太后將幸五臺,徽政院官請調軍扈從。省臣議:「昔大太后嘗幸五臺,於住夏探馬赤及漢軍內,各起扈從軍三百人,今遵故事。」從之。十一月,樞密院臣言:「去歲六衛漢軍內,以諸處興建工役,故用六千軍士於上都。臣等議,來歲車駕行幸,復令騎卒六千人,備車馬器仗,與步卒二千人扈從。」制可。



 看守軍



 世祖至元二十五年十一月,以軍守都城外倉。初,大都城內倉敖有軍守之,城外豐閏、豐實、廣貯、通濟四倉無守者。至是收糧頗多,丞相桑哥以為言,乃依都城內倉例,每倉發軍五人守之。十二月,中書省臣言:「樞密院公廨後,有倉貯糧,乞調軍五人看守。」從之。



 成宗大德四年二月,調軍五百人,於新浚河內看閘。



 武宗至大四年六月,帝御大安閣,樞密院官奏:「嘗奉旨,令各門置軍守備。臣等議,探馬赤軍士去其所戍地遠,卒莫能至,擬發阿速、唐兀等軍,參漢軍用之,各門置五十人。」制可。



 仁宗延祐元年閏三月,隆禧院官言:「初,世祖影殿,有軍士守之。今武宗御容於大崇恩福元寺安置,宜依例調軍守衛。」從之。三年二月,嶺北省乞軍守衛倉庫,命於丑漢所屬萬戶三千探馬赤軍內,摘軍三百人與之。



 英宗至治元年,增守太廟墻垣軍。初,以衛士軍人共守圍宿,故止用蒙古軍四百人,至是以衛士守內墻垣,其外需止用軍士,乃增至八百人,復命僉院哈散、院判阿剌鐵木兒領之。四月,敕搠思吉斡節兒八哈失寺內,常令軍士五人守衛。



 巡邏軍



 仁宗皇慶元年三月,丞相鐵木迭兒奏:「每歲既幸上京,於各宿衛中留衛士三百七十人,以備巡邏,今歲多盜賊,宜增百人,以嚴守御。」制可。仍命樞密與中書分領之。延祐七年五月,詔留守司及虎賁司官,親率眾於夜巡邏。



 鎮遏軍



 仁宗延祐元年閏三月,樞密院官奏:「中書省言,江浙春運糧八十三萬六千二百六十石,取日開洋,前來直沽,請預差軍人鎮遏。」詔依年例,調軍一千名,命右衛副都指揮使伯顏往鎮遏之。三年四月,海運至直沽,樞密院官奏:「今歲軍數不敷,乞調軍士五百人巡鎮。」從之。七年四月,調海運鎮遏軍一千人,如舊制。



 鎮戍



 元初以武功定天下,四方鎮戍之兵亦重矣。然自其始而觀之,則太祖、太宗相繼以有西域、中原,而攻取之際,屯兵蓋無定向,其制殆不可考也。世祖之時,海宇混一,然後命宗王將兵鎮邊徼襟喉之地,而河洛、山東據天下腹心,則以蒙古、探馬赤軍列大府以屯之。淮、江以南,地盡南海,則名籓列郡,又各以漢軍及新附等軍戍焉。皆世祖宏規遠略,與二三大臣之所共議,達兵機之要,審地理之宜,而足以貽謀於後世者也。故其後江南三行省,嘗以遷調戍兵為言,當時莫敢有變其法者,誠以祖宗成憲,不易於變更也。然卒之承平既久,將驕卒惰,軍政不修,而天下之勢遂至於不可為,夫豈其制之不善哉,蓋法久必弊,古今之勢然也。今故著其調兵屯守之制,而列之為鎮戍焉。



 世祖中統元年五月,詔漢軍萬戶,各於本管新舊軍內摘發軍人,備衣甲器仗,差官領赴燕京近地屯駐:萬戶史天澤一萬四百三十五人,張馬哥二百四十人,解成一千七百六十人,叱剌四百六十六人,斜良拔都八百九十六人,扶溝馬軍奴一百二十九人,內黃鐵木兒一百四十四人,趙奴懷四十一人,鄢陵勝都古六十五人。十一月,命右三部尚書怯烈門、平章政事趙璧領蒙古、漢軍,於燕京近地屯駐;平章塔察兒領武衛軍一萬人,屯駐北山;漢軍、質子軍及簽到民間諸投下軍,於西京、宣德屯駐。復命怯烈門為大都督,管領諸軍勾當,分達達軍為兩路,一赴宣德、德興,一赴興州。其諸萬戶漢軍,則令赴潮河屯守。後復以興州達達軍合入德興、宣德,命漢軍各萬戶悉赴懷來、縉山川中屯駐。



 三年十月,詔田德實所管固安質子軍九百十六戶,及平灤州劉不里剌所管質子軍四百戶,還元管地面屯駐。



 至元七年,以金州軍八百隸東川統軍司,還成都,忽朗吉軍戍東川。十一年正月,以忙古帶等新舊軍一萬一千人戍建都。調襄陽府生券軍六百人、熟券軍四百人,由京兆府鎮戍鴨池,命金州招討使欽察部領之。十二月,調西川王安撫、楊總帥軍與火尼赤相合,與醜漢、黃兀剌同鎮守合答之城。



 十二年二月,詔以東川新得城寨,逼近夔府,恐南兵來侵,發鞏昌路補簽軍三千人戍之。三月,海州丁安撫等來降,選五州丁壯四千人,守海州、東海。



 十三年十月,命別速鷿、忽別列八都兒二人為都元帥,領蒙古軍二千人、河西軍一千人,守斡端城。



 十五年三月,分揚州行省兵於隆興府。初,置行省,分兵諸路調遣,江西省軍為最少,至是以南廣地闊,阻山溪之險,命鐵木兒不花領兵一萬人赴之,合元帥塔出軍,以備戰守。四月,詔以伯顏、阿術所調河南新簽軍三千人,還守廬州。六月,命荊湖北道宣慰使塔海調遣夔府諸軍士。七月,詔以塔海征夔軍之還戍者,及揚州、江西舟師,悉付水軍萬戶張榮實將之,守禦江中。八月,命江南諸路戍卒,散歸各所屬萬戶屯戍。初,渡江所得城池,發各萬戶部曲士卒以戍之,久而亡命死傷者眾,續至者多不著行伍,至是縱還各營,以備屯戍。安西王相府言:「川蜀既平,城邑山寨洞穴凡八十三所,其渠州禮義城等處凡三十三所,宜以兵鎮守,餘悉撤去。」從之。九月,詔發東京、北京軍四百人,往戍應昌府,其應昌舊戍士卒,悉令散歸。十一月,定軍民異屬之制,及蒙古軍屯戍之地。先是,以李鋋叛,分軍民為二,而異其屬,後因平江南,軍官始兼民職,遂因之。凡以千戶守一郡,則率其麾下從之,百戶亦然,不便。至是,令軍民各異屬,如初制。士卒以萬戶為率,擇可屯之地屯之,諸蒙古軍士,散處南北及還各奧魯者,亦皆收聚。令四萬戶所領之眾屯河北,阿術二萬戶屯河南,以備調遣,餘丁定其版籍,編入行伍,俾各有所屬,遇征伐則遣之。



 十六年二月,命萬戶孛術魯敬,領其麾下舊有士卒守湖州。先是,以唐、鄧、均三州士卒二百八十八人屬敬麾下,後遷戍江陵府,至是還之。四月,定上都戍卒用本路元籍軍士。國制,郡邑鎮戍士卒,皆更相易置,故每歲以他郡兵戍上都,軍士罷於轉輸。至是,以上都民充軍者四千人,每歲令備鎮戍,罷他郡戍兵。六月,碉門、魚通及黎、雅諸處民戶,不奉國法,議以兵戍其地。發新附軍五百人、蒙古軍一百人、漢軍四百人,往鎮戍之。七月,以西川蒙古軍七千人、新附軍三千人,付皇子安西王。命闍里鐵木兒以戍杭州軍六百九十人赴京師,調兩淮招討小廝蒙古軍,及自北方回探馬赤軍代之。八月,調江南新附軍五千駐太原,五千駐大名,五千駐衛州。又發探馬赤軍一萬人,及夔府招討張萬之新附軍,俾四川西道宣慰使也罕的斤將之,戍斡端。



 十七年正月,詔以他令不罕守建都,布吉鷿守長河西之地,無令遷易。三月,同知浙東道宣慰司事張鐸言:「江南鎮戍軍官不便,請以時更易置之。」國制,既平江南,以兵戍列城,其長軍之官,皆世守不易,故多與富民樹黨,因奪民田宅居室,蠹有司政事,為害滋甚。鐸上言,以為皆不遷易之弊,請更其制,限以歲月遷調之。庶使初附之民,得以安業也。五月,命樞密院調兵六百人,守居庸關南、北口。七月,敕更代廣州鎮戍士卒。初以丞相伯顏等麾下合必赤軍二千五百人,從元帥張弘範徵廣王,因留戍焉。歲久皆貧困,多死亡者。至是,命更代之。復以揚州行省四萬戶蒙古軍,更戍潭州。十月,發砲卒千人入甘州,備戰守。十二月,八番羅甸宣慰司請增戍卒。先是,以三千人戍八番,後徵亦奚不薛,分摘其半。至是師還,宣慰司復請益兵,以備戰守,從之。



 十八年正月,命萬戶張珪率麾下往就潭州,還其祖父所領亳州士卒,並統之。二月,以合必赤軍三千戍揚州。十月,高麗王並行省皆言,金州、合浦、固城、全羅州等處,沿海上下,與日本正當沖要,宜設立鎮邊萬戶府屯鎮,從之。十一月,詔以征東留後軍,分鎮慶元、上海、澉浦三處上船海口。



 十九年二月,命唐兀鷿於沿江州郡,視便宜置軍鎮戍,及諭鄂州、揚州、隆興、泉州等四省,議用兵戍列城。徙浙東宣慰司於溫州,分軍戍守江南,自歸州以及江陰至三海口,凡二十八所。四月,調揚州合必赤軍三千人鎮泉州。又潭州行省以臨川鎮地接占城及未附黎洞,請立總管府,一同鎮戍,從之。七月,以隆興、西京軍士代上都戍卒,還西川。先是,上都屯戍士卒,其奧魯皆在西川,而戍西川者,多隆興、西京軍士,每歲轉餉,不勝勞費,至是更之。



 二十年八月,留蒙古軍千人戍揚州,餘悉縱還。揚州所有蒙古士卒九千人,行省請以三分為率,留一分鎮戍。史塔剌渾曰:「蒙古士卒悍勇,孰敢當,留一千人足矣。」從之。十月,發乾討虜軍千人,增戍福建行省。先是,福建行省以其地險,常有盜負固為亂,兵少不足戰守,請增蒙古、漢軍千人。樞密院議以劉萬奴所領乾討虜軍益之。



 二十一年四月,詔潭州蒙古軍依揚州例,留一千人,餘悉放還諸奧魯。十月,增兵鎮守金齒國,以其地民戶剛狠,舊嘗以漢軍、新附軍三千人戍守,今再調探馬赤、蒙古軍二千人,令藥剌海率赴之。



 二十二年二月,詔改江淮、江西元帥招討司為上、中、下三萬戶府,蒙古、漢人、新附諸軍,相參作三十七翼。上萬戶:宿州、蘄縣、真定、沂郯、益都、高郵、沿海,七翼。中萬戶:棗陽、十字路、邳州、鄧州、杭州、懷州、孟州、真州,八翼。下萬戶,常州、鎮江、潁州、廬州、亳州、安慶、江陰水軍、益都新軍、湖州、淮安、壽春、揚州、泰州、弩手、保甲、處州、上都新軍、黃州、安豐、松江、鎮江水軍、建康,二十二翼。每翼設達魯花赤、萬戶、副萬戶各一人,以隸所在行院。



 二十四年五月,調各衛諸色軍士五百人於平灤,以備鎮戍。十月,詔以廣東系邊徼之地,山險人稀,兼江西、福建賊徒聚集,不時越境作亂,發江西行省忽都鐵木兒麾下軍五千人,往鎮守之。



 二十五年二月,調揚州省軍赴鄂州,代鎮戍士卒。三月,詔黃州、蘄州、壽昌諸軍還隸江淮省。始三處舊置鎮守軍,以近鄂州省,嘗分隸領之,至是軍官以為言,遂仍其舊。遼陽行省言,懿州地接賊境,請益兵鎮戍,從之。四月,調江淮行省全翼一下萬戶軍,移鎮江西省。從皇子脫歡士卒及劉二拔都麾下一萬人,皆散歸各營。十一月,增軍戍咸平府,以察忽、亦兒思合言其地實邊徼,請益兵鎮守,以備不虞故也。



 二十六年二月,命萬戶劉得祿以軍五千人,鎮守八番。



 二十七年六月,調各行省軍於江西,以備鎮戍,俟盜賊平息,而後縱還。九月,以元帥那懷麾下軍四百人守文州。調江淮省下萬戶府軍於福建鎮戍。十一月,江淮行省言:「先是丞相伯顏及元帥阿術、阿塔海等守行省時,各路置軍鎮戍,視地之輕重,而為之多寡,厥後忙古鷿代之,悉更其法,易置將吏士卒,殊失其宜。今福建盜賊已平,惟浙東一道,地極邊惡,賊所巢穴,請復還三萬戶以鎮守之。合剌帶一軍戍沿海、明、臺,亦怯烈一軍戍溫、處,札忽帶一軍戍紹興、婺州。其寧國、徽州初用士兵,後皆與賊通,今盡遷之江北,更調高郵、泰州兩萬戶漢軍戍之。揚州、建康、鎮江三城,跨據大江,人民繁會,置七萬戶府。杭州行省諸司府庫所在,置四萬戶府。水戰之法,舊止十所,今擇瀕海沿江要害二十二所,分後閱習,伺察諸盜。錢塘控扼海口,舊置戰艦二十艘,今增置戰艦百艘,海船二十艘。」樞密院以聞,悉從之。



 二十八年二月,調江淮省探馬赤軍及漢軍二千人,於脫歡太子側近揚州屯駐。



 二十九年,以咸平府、東京所屯新附軍五百人,增戍女直地。



 三十年正月,詔西征探馬赤軍八千人,分留一千或二千,餘令放還。皇子奧魯赤、大王術伯言,切恐軍散釁生,宜留四千,還四千,從之。五月,命思播黃平、鎮遠拘刷亡宋避役手號軍人,以增鎮守。七月,調四川行院新附軍一千人,戍松山。



 成宗元貞元年七月,樞密院官奏:「劉二拔都兒言,初鄂州省安置軍馬之時,南面止是潭州等處,後得廣西海外四州、八番洞蠻等地,疆界闊遠,闕少戍軍,復增四萬人。今將元屬本省四翼萬戶軍分出,軍力減少。臣等謂劉二拔都兒之言有理,雖然江南平定之時,沿江安置軍馬,伯顏、阿術、阿塔海、阿里海牙、阿剌罕等,俱系元經攻取之人,又與近臣月兒魯、孛羅等樞密院官同議安置者。乞命通軍事、知地理之人,同議增減安置,庶後無弊。」從之。



 二年五月,江浙行省言:「近以鎮守建康、太平保定萬戶府全翼軍馬七千二百一十二名,調屬湖廣省,乞分兩淮戍兵,於本省沿海鎮遏。」樞密院官議:「沿江軍馬,系伯顏、阿術安置,勿令改動,止於本省元管千戶、百戶軍內,發兵鎮守之。」制可。九月,詔以兩廣海外四州城池戍兵,歲一更代,往來勞苦。給俸錢,選良醫,往治其疾病者。命三二年一更代之。



 三年二月,調揚州翼鄧新萬戶府全翼軍馬,分屯蘄、黃。



 大德元年三月,陜西平章政事脫烈伯領總帥府軍三千人,收捕西番回,詔留總帥軍百人及階州舊軍、禿思馬軍各二百人守階州,餘軍還元翼。湖廣省請以保定翼萬人,移鎮郴州,樞密院官議:「此翼乃張柔所領征伐舊軍,宜遷入鄂州省屯駐,別調兵守之。」七月,招收亡宋左右兩江土軍千人,從思明上思等處都元帥昔剌不花言也。十一月,河南行省言:「前揚州立江淮行省,江陵立荊湖行省,各統軍馬,上下鎮遏。後江淮省移於杭州,荊湖省遷於鄂州,黃河之南,大江迤北,汴梁古郡設立河南江北行省,通管江淮、荊湖兩省元有地面。近來並入軍馬,通行管領,所屬之地,大江最為緊要,兩淮地險人頑,宋亡之後,始來歸順。當時沿江一帶,斟酌緩急,安置定三十一翼軍馬鎮遏,後遷調十二翼前去江南,餘有一十九翼,於內調發,止存元額十分中一二。況兩淮、荊襄自古隘要之地,歸附至今,雖即寧靜,宜慮未然。乞照沿江元置軍馬,遷調江南翼分,並各省所占本省軍人,發還元翼,仍前鎮遏。」省院官議,以為沿江安置三十一翼軍馬之說,本院無此簿書,問之河南省官孛魯歡,其省亦無樞密院文卷,內但稱至元十九年,伯顏、玉速鐵木兒等共擬其地安置三萬二千軍,後增二千,總三萬四千,今悉令各省差占及逃亡事故者還充役足矣。又孛魯歡言,去年伯顏點視河南省見有軍五萬二百之上,又若還其占役事故軍人,則共有七八萬人。此數之外,脫歡太子位下有一千探馬赤、一千漢軍,阿剌八赤等哈剌魯亦在其地,設有非常,皆可調用。據各省占役,總計軍官、軍人一萬三千八百八十一名,軍官二百九名,軍人一萬三千六百七十二名,內漢軍五千五百八十名,新附軍八千二十八名,蒙古軍六十四名。江浙省占役軍官、軍人四千九百五十七名,湖廣省占役軍官、軍人七千六百三名,福建省占役軍官、軍人一千二百七十二名,江西省出征收捕未回新附軍四十九名,悉令還役。」江浙省亦言:「河南行省見占本省軍人八千八百三十三名,亦宜遣還鎮遏。」有旨,兩省各差官赴闕辨議。



 二年正月,樞密院臣言:「阿剌鷿、脫忽思所領漢人、女直、高麗等軍二千一百三十六名內,有稱海對陣者,有久戍四五年者,物力消乏,乞於六衛軍內分一千二百人,大同屯田軍八百人,徹里臺軍二百人,總二千二百人往代之。」制可。三月,詔各省合並鎮守軍,福建所置者合為五十三所,江浙所置者合為二百二十七所,江西元立屯軍鎮守二百二十六所,減去一百六十二所,存六十四所。



 三年三月,沅州賊人嘯聚,命以毗陽萬戶府鎮守辰州,鎮巢萬戶府鎮守沅州、靖州,上均萬戶府鎮守常德、澧州。



 五年三月,詔河南省占役江浙省軍一萬一千四百七十二名,除洪澤、芍陂屯田外,餘令發還元翼。



 七年四月,調碉門四川軍一千人,鎮守羅羅斯。



 八年二月,以江南海口軍少,調蘄縣王萬戶翼漢軍一百人、寧萬戶翼漢軍一百人、新附軍三百人守慶元,自乃顏來者蒙古軍三百人守定海。



 武宗至大二年七月,樞密院臣言:「去年日本商船焚掠慶元,官軍不能敵。江浙省言,請以慶元、臺州沿海萬戶府新附軍往陸路鎮守,以蘄縣、宿州兩萬戶府陸路漢軍移就沿海屯鎮。臣等議,自世祖時,伯顏、阿術等相地之勢,制事之宜,然後安置軍馬,豈可輕動。前行省忙古鷿等亦言,以水陸軍互換遷調,世祖有訓曰:『忙古鷿得非狂醉而發此言!以水路之兵習陸路之伎,驅步騎之士而從風水之役,難成易敗,於事何補。』今欲御備奸宄,莫若從宜於水路沿海萬戶府新附軍三分取一,與陸路蘄縣萬戶府漢軍相參鎮守。」從之。



 四年十月,以江浙省嘗言:「兩浙沿海瀕江隘口,地接諸蕃,海寇出沒,兼收附江南之後,三十餘年,承平日久,將驕卒惰,帥領不得其人,軍馬安置不當,乞斟酌沖要去處,遷調鎮遏。」樞密院官議:「慶元與日本相接,且為倭商焚毀,宜如所請,其餘遷調軍馬,事關機務,別議行之。」十二月,雲南八百媳婦、大、小徹里等作耗,調四川省蒙古、漢軍四千人,命萬戶囊加鷿部領,赴雲南鎮守。其四川省言:「本省地方,東南控接荊湖,西北襟連秦隴,阻山帶江,密邇蕃蠻,素號天險,古稱極邊重地,乞於存恤歇役六年軍內,調二千人往。」從之。



 仁宗皇慶元年十一月,詔江西省瘴地內諸路鎮守軍,各移近地屯駐。



 延祐四年四月,河南行省言:「本省地方寬廣,關系非輕,所屬萬戶府俱於臨江沿淮上下鎮守方面,相離省府,近者千里之上,遠者二千餘里,不測調度,卒難相應。況汴梁系國家腹心之地,設立行省,別無親臨軍馬,較之江浙、江西、湖廣、陜西、四川等處,俱有隨省軍馬,惟本省未蒙撥付。」樞密院以聞,命於山東河北蒙古軍、河南淮北蒙古軍兩都萬戶府,調軍一千人與之。十一月,陜西都萬戶府言:「碉門探馬赤軍一百五十名,鎮守多年,乞放還元翼。」樞密院臣議:「彼中亦系要地,不宜放還,止令於元翼起遣一百五十名,三年一更鎮守。元調四川各翼漢軍一千名,鎮守碉門、黎、雅,亦令一體更代。」



 泰定四年三月,陜西行省嘗言:「奉元建立行省、行臺,別無軍府,唯有蒙古軍都萬戶府,遠在鳳翔置司,相離三百五十餘里,緩急難用。乞移都萬戶府於奉元置司,軍民兩便。」及後陜西都萬戶府言:「自大德三年命移司酌中安置,經今三十餘年,鳳翔離大都、土番、甘肅俱各三千里,地面酌中,不移為便。」樞密議:「陜西舊例,未嘗提調軍馬,況鳳翔置司三十餘年,不宜移動。」制可。十二月,河南行省言:「所轄之地,東連淮、海,南限大江,北抵黃河,西接關陜,洞蠻草賊出沒,與民為害。本省軍馬俱在瀕海沿江安置,遠者二千,近者一千餘里,乞以砲手、弩軍兩翼,移於汴梁,並各萬戶府摘軍五千名,設萬戶府隨省鎮遏。」樞密院議:「自至元十九年,世祖命知地理省院官共議,於瀕海沿江六十三處安置軍馬。時汴梁未嘗置軍,揚州沖要重地,置五翼軍馬並砲手、弩軍。今親王脫歡太子鎮遏揚州,提調四省軍馬,此軍不宜更動。設若河南省果用軍,則不塔剌吉所管四萬戶蒙古軍內,三萬戶在黃河之南、河南省之西,一萬戶在河南省之南,脫別臺所管五萬戶蒙古軍俱在黃河之北、河南省東北,阿剌鐵木兒、安童等兩侍衛蒙古軍在河南省之北,共十一衛翼蒙古軍馬,俱在河南省周圍屯駐。又本省所轄一十九翼軍馬,俱在河南省之南,沿江置列。果用兵,即馳奏於諸軍馬內調發。」從之。



 阿剌鐵木兒、安童等兩侍衛蒙古軍在河南省之北,共十一衛翼蒙古軍馬,俱在河南省周圍屯駐。又本省所轄一十九翼軍馬,俱在河南省之南,沿江置列。果用兵,即馳奏於諸軍馬內調發。」從之。



\end{pinyinscope}