\article{列傳第十七}

\begin{pinyinscope}

 ○徹里



 徹里,燕只吉臺氏。曾祖太赤,為馬步軍都元帥,從太祖定中原,以功封徐、邳二州,因家於徐。徹里幼孤,母蒲察氏教以讀書。至元十八年,世祖召見,應對詳雅,悅之,俾常侍左右,民間事時有所咨訪。從征東北邊還,因言大軍所過,民不勝煩擾,寒餓且死,宜加賑給,帝從之,乃賜邊民穀帛牛馬有差,賴以存活者眾。擢利用監。二十三年,奉使江南,省風俗,訪遺逸。時行省理財方急,賣所在學田以價輸官。徹里曰:「學田所以供祭禮、育人才也,安可鬻?」遽止之。還朝以聞,帝嘉納焉。



 二十四年,分中書為尚書省。桑哥為相,引用黨與,鉤考天下錢糧,凡昔權臣阿合馬積年負逋,舉以中書失徵奏,誅二參政。行省乘風,督責尤峻。主無所償,則責及親戚,或逮系鄰黨,械禁榜掠。民不勝其苦,自裁及死獄者以百數,中外騷動。廷臣顧忌,皆莫敢言。徹里乃於帝前具陳桑哥奸貪誤國害民狀,辭語激烈。帝怒,謂其毀詆大臣,失禮體,命左右批其頰。徹里辯愈力,且曰:「臣與桑哥無仇,所以力數其罪而不顧身者,正為國家計耳。茍畏聖怒而不復言,則奸臣何由而除,民害何由而息!且使陛下有拒諫之名,臣竊懼焉。」於是帝大悟,即命帥羽林三百人往籍其家,得珍寶如內藏之半。桑哥既誅,諸枉系者始得釋。復奉旨往江南,籍桑哥姻黨江浙省臣烏馬兒、蔑列、忻都、王濟,湖廣省臣要束木等,皆棄市,天下大快之。徹里往來,凡四道徐,皆過門不入。



 進拜御史中丞,俄升福建行省平章政事,賜黃金五十兩、白金五千兩。汀、漳劇盜歐狗久不平,遂引兵征之,號令嚴肅,所過秋毫無犯。有降者,則勞以酒食而慰遣之,曰:「吾意汝豈反者耶,良由官吏污暴所致。今既來歸,即為平民,吾安忍罪汝。其返汝耕桑,安汝田里,毋恐。」他柵聞之,悉款附。未幾,歐狗為其黨縛致於軍,梟首以徇,脅從者不戮一人,汀、漳平。三十一年,帝不豫,徹里馳還京師,侍醫藥。帝崩,與諸王大臣共定策,迎立成宗。



 大德元年,拜江南諸道行臺御史大夫。一日,召都事賈鈞謂曰:「國家置御史臺,所以肅清庶官、美風俗、興教化也。乃者御史不存大體,按巡以苛為明,徵贓以多為功,至有迫子證父、弟證兄、奴訐主者。傷風敗教,莫茲為甚。君為我語諸御史,毋庸效尤為也。」帝聞而善之,改江浙行省平章政事。江浙稅糧甲天下,平江、嘉興、湖州三郡當江浙什六七,而其地極下,水鐘為震澤。震澤之注,由吳松江入海。歲久,江淤塞,豪民利之,封土為田,水道淤塞,由是浸淫泛溢,敗諸郡禾稼。朝廷命行省疏導之,發卒數萬人,徹里董其役,凡四閱月畢工。



 九年,召入為中書平章政事。十月,以疾薨,年四十七。薨之日,家資不滿二百緡,人服其廉。贈推忠守正佐理功臣、太傅、開府儀同三司、上柱國,追封徐國公,謚忠肅。至治二年,加贈宣忠同德弼亮功臣、太師、開府儀同三司、上柱國,追封武寧王,謚正憲。子朵兒只,江浙行省左丞。



 ○不忽木



 不忽木,一名時用,字用臣,世為康里部大人。康里,即漢高車國也。祖海藍伯,嘗事克烈王可汗。王可汗滅,即棄家從數千騎望西北馳去,太祖遣使招之,答曰:「昔與帝同事王可汗,今王可汗既亡,不忍改所事。」遂去,莫知所之。子十人,皆為太祖所虜,燕真最幼,年方六歲,太祖以賜莊聖皇后。後憐而育之,遣侍世祖於籓邸。長從征伐,有功。世祖威名日盛,憲宗將伐宋,命以居守。燕真曰:「主上素有疑志,今乘輿遠涉危難之地,殿下以皇弟獨處安全,可乎?」世祖然之,因請從南征。憲宗喜,即分兵命趨鄂州,而自將攻蜀之釣魚山,令阿里不哥居守。憲宗崩,燕真統世祖留部,覺阿里不哥有異志,奉皇后稍引而南,與世祖會於上都。



 世祖即位,燕真未及大用而卒,官止衛率。不忽木其仲子也,資稟英特,進止詳雅,世祖奇之,命給事裕宗東宮,師事太子贊善王恂。恂從北征,乃受學於國子祭酒許衡。日記數千言,衡每稱之,以為有公輔器。世祖嘗欲觀國子所書字,不忽木年十六,獨書《貞觀政要》數十事以進,帝知其寓規諫意,嘉嘆久之。衡纂歷代帝王名謚、統系、歲年,為書授諸生,不忽木讀數過即成誦,帝召試,不遺一字。至元十三年,與同舍生堅童、太答、禿魯等上疏曰:



 臣等聞之,《學記》曰:「君子如欲化民成俗,其必由學乎!」「玉不琢不成器,人不學不知道。」故古之王者,建國君民,教學為先。蓋自堯、舜、禹、湯、文、武之世,莫不有學,故其治隆於上,俗美於下,而為後世所法。降至漢朝,亦建學校,詔諸生課試補官。魏道武帝起自北方,既定中原,增置生員三千,儒學以興。此歷代皆有學校之證也。



 臣等今復取平南之君建置學校者,為陛下陳之。晉武帝嘗平吳矣,始起國子學。隋文帝嘗滅陳矣,俾國子寺不隸太常。唐高祖嘗滅梁矣,詔諸州縣及鄉並令置學。及至太宗,數幸國學,增築學舍至千二百間,國學、太學、四門學亦增生員,其書、算各置博士,乃至高麗、百濟、新羅、高昌、吐蕃諸國酋長亦遣子弟入學,國學之內至八千餘人。高宗因之,遂令國子監領六學:一曰國子學,二曰太學,三曰四門學,四曰律學,五曰書學,六曰算學,各置生徒有差,皆承高祖之意也。然晉之平吳得戶五十二萬而已,隋之滅陳得郡縣五百而已,唐之滅梁得戶六十餘萬而已,而其崇重學校已如此。況我堂堂大國,奄有江嶺之地,計亡宋之戶不下千萬,此陛下神功,自古未有,而非晉、隋、唐之所敢比也。然學校之政,尚未全舉,臣竊惜之。



 臣等向被聖恩,俾習儒學。欽惟聖意,豈不以諸色人仕宦者常多,蒙古人仕宦者尚少,而欲臣等曉識世務,以任陛下之使令乎?然以學制未定,朋從數少。譬猶責嘉禾於數苗,求良驥於數馬,臣等恐其不易得也。為今之計,如欲人材眾多,通習漢法,必如古昔遍立學校然後可。若曰未暇,宜且於大都弘闡國學。擇蒙古人年十五以下、十歲以上質美者百人,百官子弟與凡民俊秀者百人,俾廩給各有定制。選德業充備足為師表者,充司業、博士、助教而教育之。使其教必本於人倫,明乎物理,為之講解經傳,授以修身、齊家、治國、平天下之道。其下復立數科,如小學、律、書、算之類。每科設置教授,各令以本業訓導。小學科則令讀誦經書,教以應對進退事長之節;律科則專令通曉吏事;書科則專令曉習字畫;算科則專令熟閑算數。或一藝通然後改授,或一日之間更次為之。俾國子學官總領其事,常加點勘,務要俱通,仍以義理為主。有餘力者聽令學作文字。日月歲時,隨其利鈍,各責所就功課,程其勤惰而賞罰之。勤者則升之上舍,惰者則降之下舍,待其改過則復升之。假日則聽令學射,自非假日,無故不令出學。數年以後,上舍生學業有成就者,乃聽學官保舉,蒙古人若何品級,諸色人若何仕進。其未成就者,且令依舊學習,俟其可以從政,然後歲聽學官舉其賢者、能者,使之依例入仕。其終不可教者,三年聽令出學。凡學政因革、生員增減,若得不時奏聞,則學無弊政,而天下之材亦皆觀感而興起矣。然後續立郡縣之學,求以化民成俗,無不可者。



 臣等愚幼,見於書、聞於師者如此。未敢必其可行,伏望聖慈下臣此章,令諸老先與左丞王贊善等,商議條奏施行,臣等不勝至願。



 書奏,帝覽之喜。



 十四年,授利用少監。十五年,出為燕南河北道提刑按察副使。帝遣通事脫虎脫護送西僧往作佛事,還過真定,箠驛吏幾死,訴之按察使,不敢問。不忽木受其狀,以僧下獄。脫虎脫直欲出僧,辭氣倔強,不忽木令去其冠庭下,責以不職。脫虎脫逃歸以聞,帝曰:「不忽木素剛正,必爾輩犯法故也。」繼而燕南奏至,帝曰:「我固知之。」十九年,升提刑按察使。有訟凈州守臣盜官物者,凈州本隸河東,特命不忽木往按之,歸報稱旨,賜白金千兩、鈔五千貫。



 二十一年,召參議中書省事。時榷茶轉運使盧世榮阿附宣政使桑哥,言能用己,則國賦可十倍於舊。帝以問不忽木,對曰:「自昔聚斂之臣,如桑弘羊、宇文融之徒,操利術以惑時君,始者莫不謂之忠,及其罪稔惡著,國與民俱困,雖悔何及。臣願陛下無納其說。」帝不聽,以世榮為右丞,不忽木遂辭參議不拜。二十二年,世榮以罪被誅,帝曰:「朕殊愧卿。」擢吏部尚書。時方籍沒阿合馬家,其奴張散札兒等罪當死,繆言阿合馬家貲隱寄者多,如盡得之,可資國用。遂鉤考捕系,連及無辜,京師騷動。帝頗疑之,命丞相安童集六部長貳官詢問其事,不忽木曰:「是奴為阿合馬心腹爪牙,死有餘罪。為此言者,蓋欲茍延歲月,徼幸不死爾。豈可復受其誑,嫁禍善良耶?急誅此徒,則怨謗自息。」丞相以其言入奏,帝悟,命不忽木鞫之,具得其實,散札兒等伏誅,其捕系者盡釋之。



 二十三年,改工部尚書。九月,遷刑部。河東按察使阿合馬,以貲財諂媚權貴,貨錢於官,約償羊馬,至則抑取部民所產以輸。事覺,遣使按治,皆不伏,及不忽木往,始得其不法百餘事。會大同民饑,不忽木以便宜發倉廩賑之。阿合馬所善幸臣奏不忽木擅發軍儲,又鍛煉阿合馬使自誣服。帝曰:「使行發粟以活吾民,乃其職也,何罪之有。」命移其獄至京師審視,阿合馬竟伏誅。吐土哈求欽察之為人奴者增益其軍,而多取編民。中書僉省王遇驗其籍改正之。吐土哈遂奏遇有不臣語。帝怒,欲斬之,不忽木諫曰:「遇始令以欽察之人奴為兵,未聞以編民也。萬一他衛皆仿此,戶口耗矣。若誅遇,後人豈肯為陛下盡職乎?」帝意解,遇得不死。



 二十四年,桑哥奏立尚書省,誣殺參政楊居寬、郭佑。不忽木爭之不得,桑哥深忌之,嘗指不忽木謂其妻曰:「他日籍我家者此人也。」因其退食,責以不坐曹理務,欲加之罪,遂以疾免。車駕還自上都,其弟野禮審班侍坐輦中,帝曰:「汝兄必以某日來迎。」不忽木果以是日至。帝見其臒甚,問其祿幾何,左右對以滿病假者例不給,帝念其貧,命盡給之。



 二十七年,拜翰林學士承旨、知制誥兼修國史。二十八年春,帝獵柳林,徹里等劾奏桑哥罪狀,帝召問不忽木,具以實對。帝大驚,乃決意誅之。罷尚書省,復以六部歸於中書,欲用不忽木為丞相,固辭,帝曰:「朕過聽桑哥,致天下不安,今雖悔之,已無及矣。朕識卿幼時,使卿從學,政欲備今日之用,勿多讓也。」不忽木曰:「朝廷勛舊,齒爵居臣右者尚多,今不次用臣,無以服眾。」帝曰:「然則孰可?」對曰:「太子詹事完澤可。向者籍沒阿合馬家,其賂遺近臣,皆有簿籍,唯無完澤名;又嘗言桑哥為相,必敗國事,今果如其言,是以知其可也。」帝曰:「然非卿無以任吾事。」乃拜完澤右丞相,不忽木平章政事。



 上都留守木八剌沙言改按察司置廉訪司不便,宜罷去,乃求憲臣贓罪以動上聽。帝以責中丞崔彧,彧謝病不知。不忽木面斥彧不直言,因歷陳不可罷之說,帝意乃釋。王師征交趾失利,復謀大舉,不忽木曰:「島夷詭詐,天威臨之,寧不震懼,獸窮則噬,勢使之然。今其子日燇襲位,若遣一介之使,諭以禍福,彼能悔過自新,則不煩兵而下矣。如或不悛,加兵未晚。」帝從之。於是交趾感懼,遣其偽昭明王等詣闕謝罪,盡獻前六歲所當貢物。帝喜曰:「卿一言之力也。」即以其半賜之,不忽木辭曰:「此陛下神武不殺所致,臣何功焉。」惟受沉水假山、象牙鎮紙、水晶筆格而已。麥術丁請復立尚書省,專領右三部,不忽木庭責之曰:「阿合馬、桑哥相繼誤國,身誅家沒,前鑒未遠,奈何又欲效之乎!」事遂寢。或勸征流求,及賦江南包銀,皆諫止之。桑哥黨人納速剌丁等既誅,帝以忻都長於理財,欲釋不殺。不忽木力爭之,不從。日中凡七奏,卒正其罪。釋氏請以金銀幣帛祠其神,帝難之。不忽木曰:「彼佛以去貪為寶。」遂弗與。或言京師蒙古人宜與漢人間處,以制不虞。不忽木曰:「新民乍遷,猶未寧居,若復紛更,必致失業。此蓋奸人欲擅貨易之利,交結近幸,借為納忠之說耳。」乃圖寫國中貴人第宅已與民居犬牙相制之狀上之而止。有譖完澤徇私者,帝以問不忽木。對曰:「完澤與臣俱待罪中書,設或如所言,豈得專行。臣等雖愚陋,然備位宰輔,人或發其陰短,宜使面質,明示責降,若內懷猜疑,非人主至公之道也。」言者果屈,帝怒,命左右批其頰而出之。是日苦寒,解所御黑貂裘以賜。帝每顧侍臣,稱塞咥旃之能,不忽木從容問其故,帝曰:「彼事憲宗,常陰資朕財用,卿父所知。卿時未生,誠不知也。」不忽木曰:「是所謂為人臣懷二心者。今有以內府財物私結親王,陛下以為若何?」帝急揮以手曰:「卿止,朕失言。」



 三十年,有星孛於帝座。帝憂之,夜召入禁中,問所以銷天變之道,奏曰:「風雨自天而至,人則棟宇以待之;江河為地之限,人則舟楫以通之。天地有所不能者,人則為之,此人所以與天地參也。且父母怒,人子不敢疾怨,惟起敬起孝。故《易·震》之象曰『君子以恐懼修省』,《詩》曰『敬天之怒』,又曰『遇災而懼』。三代聖王,克謹天戒,鮮不有終。漢文之世,同日山崩者二十有九,日食地震頻歲有之,善用此道,天亦悔禍,海內乂安。此前代之龜鑒也,臣願陛下法之。」因誦文帝《日食求言詔》。帝悚然曰:「此言深合朕意,可復誦之。」遂詳論款陳,夜至四鼓,明日進膳,帝以盤珍賜之。



 三十年,帝不豫,故事,非國人勛舊不得入臥內。不忽木以謹厚,日視醫藥,未嘗去左右。帝大漸,與御史大夫月魯那顏、太傅伯顏並受遺詔,留禁中。丞相完澤至,不得入,伺月魯那顏、伯顏出,問曰:「我年位俱在不忽木上,國有大議而不預,何耶?」伯顏嘆息曰:「使丞相有不忽木識慮,何至使吾屬如是之勞哉!」完澤不能對,入言於太后。太后召三人問之,月魯那顏曰:「臣受顧命,太后但觀臣等為之。臣若誤國,即日伏誅,宗社大事,非宮中所當預知也。」太后然其言,遂定大策。其後發引、升祔、請謚南郊,皆不忽木領之。



 成宗即位,執政皆迎於上都之北。丞相常獨入,不忽木至數日乃得見,帝問知之,慰勞之曰:「卿先朝腹心,顧朕寡昧,惟朝夕啟沃,以匡朕不逮,庶無負先帝付托之重也。」成宗躬攬庶政,聽斷明果,廷議大事多採不忽木之言。太后亦以不忽木先朝舊臣,禮貌甚至。



 河東守臣獻嘉禾,大臣欲奏以為瑞。不忽木語之曰:「汝部內所產盡然耶,惟此數莖耶?」曰:「惟此數莖爾。」不忽木曰:「若如此,既無益於民,又何足為瑞。」遂罷遣之。西僧為佛事,請釋罪人祈福,謂之禿魯麻。豪民犯法者,皆賄賂之以求免。有殺主、殺夫者,西僧請被以帝後御服,乘黃犢出宮門釋之,云可得福。不忽木曰:「人倫者,王政之本,風化之基,豈可容其亂法如是!」帝責丞相曰:「朕戒汝無使不忽木知,今聞其言,朕甚愧之。」使人謂不忽木曰:「卿且休矣!朕今從卿言,然自是以為故事。」有奴告主者,主被誅,詔即以其主所居官與之。不忽木言:「若此必大壞天下之風俗,使人情愈薄,無復上下之分矣。」帝悟,為追廢前命。執政奏以為陜西行省平章政事,太后謂帝曰:「不忽木朝廷正人,先皇帝所付托,豈可出之於外耶!」帝復留之。竟以與同列多異議,稱疾不出。元貞二年春,召至便殿曰:「朕知卿疾之故,以卿不能從人,人亦不能從卿也。欲以段貞代卿,如何?」不忽木曰:「貞實勝於臣。」乃拜昭文館大學士、平章軍國重事。辭曰:「是職也,國朝惟史天澤嘗為之,臣何功敢當此。」制去「重」字。



 大德二年,御史中丞崔彧卒,特命行中丞事。三年,兼領侍儀司事。有因父官受賄賂,御史必欲歸罪其父,不忽木曰:「風紀之司,以宣政化、勵風俗為先,若使子證父,何以興孝!」樞密臣受人玉帶,徵贓不敘,御史言罰太輕,不忽木曰:「禮,大臣貪墨,惟曰簠簋不飾,若加笞辱,非刑不上大夫之意。」人稱其平恕。四年,病復作,帝遣醫治之,不效,乃附奏曰:「臣孱庸無取,叨承眷渥,大限有終,永辭昭代。」引觴滿飲而卒,年四十六。帝聞之驚悼,士大夫皆哭失聲。



 家素貧,躬自爨汲,妻織紝以養母。後因使還,則母已死,號慟嘔血,幾不起。平居服儒素,不尚華飾。祿賜有餘,即散施親舊。明於知人,多所薦拔,丞相哈剌哈孫答剌罕亦其所薦也。其學先躬行而後文藝。居則簡默,及帝前論事,吐辭洪暢,引義正大,以天下之重自任,知無不言。世祖嘗語之曰:「太祖有言,人主理天下,如右手持物,必資左手承之,然後能固。卿實朕之左手也。」每侍燕間,必陳說古今治要,世祖每拊髀嘆曰:「恨卿生晚,不得早聞此言,然亦吾子孫之福。」臨崩,以白璧遺之,曰:「他日持此以見朕也。」武宗時,贈純誠佐理功臣、太傅、開府儀同三司、上柱國、魯國公,謚文貞。



 子回回,陜西行省平章政事;巙巙,由江浙行省平章政事入為翰林學士承旨。



 ○完澤



 完澤,土別燕氏。祖土薛,從太祖起朔方,平諸部。太宗伐金,命太弟睿宗由陜右進師,以擊其不備,土薛為先鋒,遂去武休關,越漢江,略方城而北,破金兵於陽翟。金亡,從攻興元、閬、利諸州,拜都元帥。取宋成都,斬其將陳隆之,賜食邑六百戶。父糸泉真,宿衛禁中,掌御膳。中統初,從世祖北征。四年,拜中書右丞相,與諸儒臣論定朝制。



 完澤以大臣子選為裕宗王府僚屬。裕宗為皇太子,署詹事長。入參謀議,出掌環衛,小心慎密,太子甚器重之。一日會燕宗室,指完澤語眾曰:「親善遠惡,君之急務。善人如完澤者,群臣中豈易得哉!」自是常典東宮衛兵。裕宗薨,成宗以皇孫撫軍北方,完澤兩從入北。至元二十八年,桑哥伏誅,世祖咨問廷臣,特拜中書右丞相。完澤入相,革桑哥弊政,請自中統初積歲逋負之錢粟,悉蠲免之,民賴其惠。三十一年,世祖崩,完澤受遺詔,合宗戚大臣之議,啟皇太后,迎成宗即位,詔諭中外,罷征安南之師,建議加上祖宗尊謚廟號,致養皇太后,示天下為人子之禮。元貞以來,朝廷恪守成憲,詔書屢下散財發粟,不惜巨萬,以頒賜百姓,當時以賢相稱之。大德四年,加太傅、錄軍國重事。位望益崇,成宗倚任之意益重,而能處之以安靜,不急於功利,故吏民守職樂業,世稱賢相云。七年薨,年五十八,追封興元王,謚忠憲。



 ○阿魯渾薩理



 阿魯渾薩理,畏兀人。祖阿臺薩理,當太祖定西域還時,因從至燕。會畏兀國王亦都護請於朝,盡歸其民,詔許之,遂復西還。精佛氏學。生乞臺薩理,襲先業,通經、律、論。業既成,師名之曰萬全。至元十二年,入為釋教都總統,拜正議大夫、同知總制院事,加資德大夫、統制使。年七十卒。子三人:長曰畏吾兒薩理,累官資德大夫、中書右丞、行泉府太卿;季曰島瓦赤薩理;阿魯渾薩理,其中子也,以父字為全氏,幼聰慧,受業於國師八哈思巴,既通其學,且解諸國語。世祖聞其材,俾習中國之學,於是經、史、百家及陰陽、歷數、圖緯、方技之說皆通習之。後事裕宗,入宿衛,深見器重。



 至元二十年,有西域僧自言能知天象,譯者皆莫能通其說。帝問左右,誰可使者。侍臣脫烈對曰:「阿魯渾薩理可。」即召與論難,僧大屈服。帝悅,令宿衛內朝。會有江南人言宋宗室反者,命遣使捕至闕下。使已發,阿魯渾薩理趣入諫曰:「言者必妄,使不可遣。」帝曰:「卿何以言之?」對曰:「若果反,郡縣何以不知?言者不由郡縣,而言之闕庭,必其仇也。且江南初定,民疑未附,一旦以小民浮言輒捕之,恐人人自危,徒中言者之計。」帝悟,立召使者還,俾械系言者下郡治之,言者立伏,果以嘗貸錢不從誣之。帝曰:「非卿言,幾誤,但恨用卿晚耳。」自是命日侍左右。



 二十一年,擢朝列大夫、左侍儀奉御。遂勸帝治天下必用儒術,宜招致山澤道藝之士,以備任使。帝嘉納之,遣使求賢,置集賢館以待之。秋九月,命領館事,阿魯渾薩理曰:「陛下初置集賢以待士,宜擇重望大臣領之,以新觀聽。」請以司徒撒里蠻領其事,帝從之。仍以阿魯渾薩理為中順大夫、集賢館學士,兼太史院事,仍兼左侍儀奉御。士之應詔者,盡命館穀之,凡飲食供帳,車服之盛,皆喜過望。其弗稱旨者,亦請加賚而遣之。有官於宣徽者,欲陰敗其事,故盛陳所給廩餼於內前,冀帝見之。帝果過而問焉,對曰:「此一士之日給也。」帝怒曰:「汝欲使朕見而損之乎?十倍此以待天下士,猶恐不至,況欲損之,誰肯至者。」阿魯渾薩理又言於帝曰:「國學人材之本,立國子監,置博士弟子員,宜優其廩餼,使學者日盛。」從之。二十二年夏六月,遷嘉議大夫。二十三年,進集賢大學士、中奉大夫。



 二十四年春,立尚書省,桑哥用事,詔阿魯渾薩理與同視事,固辭,不許,授資德大夫、尚書右丞,繼拜榮祿大夫、平章政事。桑哥為政暴橫,且進其黨與。阿魯渾薩理數切諍之,久與乖剌,惟以廉正自持。桑哥奏立征理司,理天下逋欠,使者相望於道,所在囹圄皆滿,道路側目,無敢言者。會地震北京,阿魯渾薩理請罷征理司,以塞天變。詔下之日,百姓相慶。未幾,桑哥敗,以連坐,亦籍其產。帝問:「桑哥為政如此,卿何故無一言?」對曰:「臣未嘗不言,顧言不用耳。陛下方信任桑哥甚,彼所忌獨臣,臣數言不行,若抱柴救火,只益其暴,不若彌縫其間,使無傷國家大本,陛下久必自悟也。」帝亦以為然,且曰:「吾甚愧卿。」桑哥臨刑,吏猶以阿魯渾薩理為問,桑哥曰:「我惟不用其言,故至於敗,彼何與焉。」帝益信其無罪,詔還所籍財產,仍遣張九思賜以金帛,辭不受。



 二十八年秋,乞罷政事,並免太史院使,詔以為集賢大學士。司天劉監丞言,阿魯渾薩理在太史院時,數言國家災祥事,大不敬,請下吏治。帝大怒,以為誹謗大臣,當抵罪。阿魯渾薩理頓首謝曰:「臣不佞,賴陛下天地含容之德,雖萬死莫報。然欲致言者罪,臣恐自是無為陛下言事者。」力爭之,乃得釋。帝曰:「卿真長者。」後雖罷政,或通夕召入論事,知無不言。三十年,復領太史院事。明年,帝崩,成宗在邊,裕宗太后命為書趣成宗入正大位,又命率翰林、集賢、禮官備禮冊命。明年春,加守司徒、集賢院使,領太史院事。初,裕宗即世,世祖欲定皇太子,未知所立,以問阿魯渾薩理,即以成宗為封,且言成宗仁孝恭儉,宜立,於是大計乃決,成宗及裕宗皇后皆莫之知也。數召阿魯渾薩理不往,成宗撫軍北邊,帝遣阿魯渾薩理奉皇太子寶於成宗,乃一至其邸。及即位,語阿魯渾薩理曰:「朕在潛邸,誰不願事朕者,惟卿雖召不至,今乃知卿真得大臣體。」自是召對不名,賜坐視諸侯王等。嘗語左右曰:「若全平章者,真全材也,於今殆無其比。」大德三年,復拜中書平章政事。十一年,薨,年六十有三。延祐四年,贈推忠佐理翊亮功臣、太師、開府儀同三司、上柱國,追封趙國公,謚文定。



 子三人:長嶽柱;次久著,終翰林侍讀學士;次買住,蚤卒。嶽柱自有傳。阿臺薩理贈保德功臣、銀青榮祿大夫、司徒、柱國,追封趙國公,謚端願;乞臺薩理累贈純誠守正功臣、太保、儀同三司、上柱國,追封趙國公,謚通敏。



 岳柱字止所,一字兼山。自幼容止端嚴,性穎悟,有遠識。方八歲,觀畫師何澄畫《陶母剪發圖》,嶽柱指陶母手中金釧詰之曰:「金釧可易酒,何用剪發為也?」何大驚,即異之。既長就學,日記千言。年十八,從丞相答失蠻備宿衛,出入禁中,如老成人。至大元年,授集賢學士,階正議大夫,即以薦賢舉能為事。皇慶元年,升中奉大夫、湖南道宣慰使。日接見儒生,詢求民瘼。延祐三年,進資善大夫、隆禧院使。七年,授太史院使。英宗視其進止整暇,顧謂參政速速曰:「全院使真故家令子也。」泰定元年,改太常禮儀院使。四年,授禮部尚書,領會同館事,俄授江西等處行中書省參知政事。天歷元年,進榮祿大夫、集賢大學士。



 至順二年,除江西等處行中書省平章政事。時有誣告富民負永寧王官帑錢八百餘錠者,中書遣使諸路征之。使至江西,嶽柱曰:「事涉誣罔,不可奉命。」僚佐重違宰臣意,嶽柱曰:「民惟邦本,傷本以斂怨,亦非宰相福也。」令使者以此意復命。時燕帖木兒為丞相,聞其言,感悟,命刑部詰治,得誣罔狀,罪誣告者若干人。宰相以奏,帝嘉之,特賜幣帛及上尊酒。桂陽州民張思進等,嘯聚二千餘眾,州縣不能治,廣東宣慰司請發兵捕之。嶽柱曰:「有司不能撫綏邊民,乃欲僥幸興兵,以為民害耶?不可。」宰執皆失色,憲司亦以興兵不便為言,嶽柱終持不可,遣千戶王英往問狀。英直抵賊巢,諭以禍福,賊曰:「致我為非者,兩巡檢司耳,我等何敢有異心哉!」諭其眾,皆使復業,一方以寧。三年,遷河南江北等處行中書省平章政事。旋以軍事至揚州,得疾,明年十二月,端坐而卒,年五十三。



 岳柱天資孝友,母弟久住早卒,喪之盡哀。尤嗜經史,自天文、醫藥之書,無不究極。度量弘擴,有欺之者,恬不為意。或問之,則曰:「彼自欺也,我何與焉。」母郜氏亦常稱之曰:「吾子古人也。」



 子四人:長普達,同僉行宣政院事;次安僧,為久住後,章佩監丞;次仁壽,中憲大夫、長秋寺卿。



\end{pinyinscope}