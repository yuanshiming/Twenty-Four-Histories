\article{列傳第六}

\begin{pinyinscope}

 ○木華黎



 木華黎,札剌兒氏,世居阿難水東。父孔溫窟哇,以戚里故,在太祖麾下,從平篾里吉,徵乃蠻部,數立功。後乃蠻又叛,太祖與六騎走,中道乏食,擒水際橐駝殺之,燔以啖太祖。追騎垂及,而太祖馬斃,五騎相顧駭愕,孔溫窟哇以所乘馬濟太祖,身當追騎,死之。太祖獲免。有子五人,木華黎其第三子也。生時有白氣出帳中。神巫異之,曰:「此非常兒也。」及長,沉毅多智略,猿臂善射,挽弓二石強。與博爾術、博爾忽、赤老溫事太祖,俱以忠勇稱,號掇裏班曲律,猶華言四傑也。



 太祖軍嘗失利,會大雪,失牙帳所在,夜臥草澤中。木華黎與博爾術張裘氈,立雪中,障蔽太祖,達旦竟不移足。一日,太祖從三十餘騎行溪谷間,顧謂曰:「此中或遇寇,當奈何?」對曰:「請以身當之。」既而,寇果自林間突出,矢下如雨。木華黎引弓射之,三發中三人。其酋呼曰:「爾為誰?」曰:「木華黎也。」徐解馬鞍持之,捍衛太祖以出,寇遂引去。克烈王可汗與乃蠻部讎戰,求援於太祖。太祖遣木華黎及博爾術等救之,盡殺乃蠻之眾於按臺之下,獲甲仗、馬牛而還。既而王可汗謀襲太祖,其下拔臺知之,密告太祖。太祖遣木華黎選精騎夜斫其營,王可汗走死,諸部大人聞風款附。歲丙寅,太祖即皇帝位,首命木華黎、博爾術為左右萬戶。從容謂曰:「國內平定,汝等之力居多。我與汝猶車之有轅,身之有臂也。汝等切宜體此,勿替初心。」



 金之降者,皆言其主璟殺戮宗親,荒淫日恣。帝曰:「朕出師有名矣。」辛未,從伐金,薄宣德,遂克德興。壬申,攻雲中、九原諸郡,拔之,進圍撫州。金兵號四十萬,陣野狐嶺北。木華黎曰:「彼眾我寡,弗致死力戰,未易破也。」率敢死士,策馬橫戈,大呼陷陣,帝麾諸軍並進,大敗金兵,追至澮河,殭尸百里。癸酉,攻居庸關,壁堅,不得入,遣別將闍別統兵趨紫荊口,金左監軍高琪引兵來拒,不戰而潰,遂拔涿州。因分兵攻下益都、濱、棣諸城,遂次霸州,史天倪、蕭勃迭率眾來降,並奏為萬戶。甲戌,從圍燕,金主請和,北還。命統諸軍征遼東,次高州,盧琮、金樸以城降。乙亥,裨將蕭也先以計平定東京。進攻北京,金守將銀青率眾二十萬拒花道逆戰,敗之,斬首八萬餘級。城中食盡,契丹軍斬關來降,進軍逼之,其下殺銀青,推寅答虎為帥,遂舉城降。木華黎怒其降緩,欲坑之,蕭也先曰:「北京為遼西重鎮,既降而坑之,後豈有降者乎?」從之。奏寅答虎留守北京,以吾也而權兵馬都元帥鎮之。遣高德玉、劉蒲速窩兒招諭興中府,同知兀裡卜不從,殺蒲速窩兒,德玉走免。未幾,吏民殺兀裡卜,推土豪石天應為帥,舉城降,奏為興中尹、兵馬都提控。



 錦州張鯨聚眾十餘萬,殺節度使,稱臨海郡王,至是來降。詔木華黎以鯨總北京十提控兵,從掇忽闌南征未附州郡。木華黎密察鯨有反側意,請以蕭也先監其軍。至平州,鯨稱疾逗留,復謀遁去,監軍蕭也先執送行在,誅之。鯨弟致憤其兄被誅,據錦州叛,略平、灤、瑞、利、義、懿、廣寧等州。木華黎率蒙古不花等軍數萬討之,州郡多殺致所署長吏降。進逼紅羅山,主將杜秀降,奏為錦州節度使。丙子,致陷興中府。七月,進兵臨興中。先遣吾也而等攻溜石山,諭之曰:「今若急攻,賊必遣兵來援,我斷其歸路,致可擒也。」又遣蒙古不花屯永德縣東候之。致果遣鯨子東平將騎兵八千、步卒三萬,援溜石。蒙古不花引兵趨之,馳報,木華黎夜半引兵疾馳,遇於神水縣東,夾擊之。分麾下兵之半,下馬步戰。選善射者數千,令曰:「賊步兵無甲,疾射之!」乃麾騎兵齊進,大敗之,斬東平及士卒萬二千八百餘級。拔開義縣,進圍錦州。致遣張太平、高益出戰,又敗之,斬首三千餘級,溺死者不可勝數。圍守月餘,致憤將校不戮力,殺敗將二十餘人。高益懼,縛致出降,伏誅。廣寧劉琰、懿州田和尚降,木華黎曰:「此叛寇,存之無以懲後。」除工匠優伶外,悉屠之。拔蘇、復、海三州,斬完顏眾家奴。咸平宣撫薄鮮等率眾十餘萬,遁入海島。



 丁丑八月,詔封太師、國王、都行省承制行事,賜誓券、黃金印曰:「子孫傳國,世世不絕。」分弘吉剌、亦乞烈思、兀魯兀、忙兀等十軍,及吾也而契丹、蕃、漢等軍,並屬麾下。且諭曰:「太行之北,朕自經略,太行以南,卿其勉之。」賜大駕所建九斿大旗,仍諭諸將曰:「木華黎建此旗以出號令,如朕親臨也。」乃建行省於雲、燕,以圖中原,遂自燕南攻遂城及蠡州諸城,拔之。冬,破大名府,遂東定益都、淄、登、萊、濰、密等州。戊寅,自西京由太和嶺入河東,攻太原、忻、代、澤、潞、汾、霍等州,悉降之。遂徇平陽,金守臣棄城遁,以前鋒拓拔按察兒統蒙古軍鎮之拒金兵,以義州監軍李廷植之弟守忠權河東南路帥府事。己卯,以蕭特末兒等出雲、朔,攻降岢嵐火山軍。以穀裏夾打為元帥達魯花赤,攻拔石、隰州,擊絳州,克之。庚辰,復由燕徇趙,至滿城。武仙舉真定來降。權知河北西路兵馬事史天倪進言曰:「今中原粗定,而所過猶縱兵抄掠,非王者吊民之意也。」木華黎曰:「善。」下令禁無剽掠,所獲老稚,悉遣還田里,軍中肅然,吏民大悅。兵至滏陽,金邢州節度使武貴迎降,進攻天平寨,破之。遣蒙古不花分兵略定河北衛、懷、孟州,入濟南。嚴實籍所隸相、魏、磁、名、恩、博、滑、浚等州戶三十萬,詣軍門降。



 時金兵屯黃陵岡,號二十萬,遣步兵二萬襲濟南。木華黎以輕兵五百擊走之。遂會大軍,薄黃陵岡。金兵陣河南岸,示以死戰。木華黎曰:「此不可用長兵,當以短兵取勝。」令騎下馬,引滿齊發,亦下馬督戰,果大敗之,溺死者眾。進攻楚丘。楚丘城小而固,四面皆水,令諸軍以草木填塹,直抵城下。嚴實率所部先登,拔之。攻下單州,圍東平,以實權山東西路行省,戒之曰:「東平糧盡,必棄城走,汝伺其去,即入城安輯之,勿苦郡縣,以敗事也。」留梭魯忽禿以蒙古軍三千屯守之。辛巳四月,東平糧盡,金行省忙古奔汴,梭魯忽禿邀擊之,斬七千餘級,忙古引數百騎遁去。實入城,建行省,撫其民。先是,郡王帶孫攻洺不下,至是遣石天應拔之。五月,還軍野狐嶺。宋漣水忠義統轄石珪來降,以為濟、兗、單三州都總管,予繡衣玉帶,勞之曰:「汝不憚跋涉數千里,慕義而來,尋當列奏,賜汝高爵,爾其勉之。」京東安撫使張琳皆來降,以琳行山東東路益都滄景濱棣等州都元帥府事。鄭遵亦以棗鄉、蓨縣降,升為元州,以遵為節度使,行元帥府事。



 秋八月,從駐青塚,監國公主遣使來勞,大饗將士,由東勝渡河,西夏國李王請以兵五萬屬焉。冬十月,復由雲中歷太和寨,入葭州,金將王公佐遁,以石天應權行臺兵馬都元帥。進取綏德,破馬蹄寨,距延安三十里止舍。金行省完顏合達出兵三萬陣於城東,蒙古不花以騎三千覘之,馳報曰:「彼見吾兵少,有輕敵心,明日合戰,當佯敗,可以伏兵取勝也。」從之。夜半以大軍銜枚齊進,伏於城東十五里兩谷間。明日,蒙古不花進兵,望見金兵,即棄鼓旗走。金兵果追之,伏發,鼓聲震天地,萬矢齊下,金兵大敗,斬七千級,獲馬八百。合達走保延安,圍之旬日,不下,乃南徇洛川,克鄜州。



 北京權帥石天應擒送金驍將張鐵槍,木華黎責其不降,厲聲答曰:「我受金朝厚恩二十餘年,今事至此,有死而已!」木華黎義之,欲解其縛,諸將怒其不屈,竟殺之。遂降坊州,大饗士卒。聞金復取隰州,以軒成為經略使,於是復由丹州渡河圍隰,克之。留合丑統蒙古軍鎮石、隰間,以田雄權元帥府事。壬午秋七月,令蒙古不花引兵出秦隴,以張聲勢。視山川險夷,大兵道雲中,攻下孟州四蹄寨,遷其民於州。拔晉陽義和寨,進克三清巖,入霍邑山堡,遷其人於趙城縣。薄青龍堡,金平陽公胡天作拒守,裨將薄察定住、監軍王和開壁降,遷天作於平陽。



 八月,有星晝見,隱士喬靜真曰:「今觀天象,未可征進。」木華黎曰:「主上命我平定中原,今河北雖平,而河南、秦、鞏未下,若因天象而不進兵,天下何時而定耶?且違君命,得為忠乎!」冬十月,過晉至絳,拔榮州胡瓶堡,所至望風歸附,河中久為金有,至是復來歸。木華黎召石天應謂曰:「蒲為河東要害,我擇守者,非君不可。」乃以天應權河東南北路陜右關西行臺,平陽李守忠、太原攸哈剌拔都、隰州田雄,並受節制。命天應造浮梁,以濟歸師,乃渡河拔同州,下蒲城,徑趨長安。金京兆行省完顏合達擁兵二十萬固守,不下。乃分麾下兀胡乃、太不花兵六千屯守之。遣按赤將兵三千斷潼關,遂西擊鳳翔。月餘不下,謂諸將曰:「吾奉命專征,不數年取遼西、遼東、山東、河北,不勞餘力;前攻天平、延安,今攻鳳翔皆不下,豈吾命將盡耶!」乃駐兵渭水南,遣蒙古不花南越牛嶺關,徇宋鳳州而還。



 時中條山賊侯七等聚眾十餘萬,伺大兵既西,謀襲河中。石天應遣別將吳權府引兵五百夜出東門,伏兩谷間,戒之曰:「候賊過半,急擊之,我出其前,爾攻其後,可克也。」吳權府醉酒失期,天應戰死。城陷,賊燒毀廬舍,殺掠人民,還走中條。先鋒元帥按察兒邀擊,敗之,斬數萬級,侯七復遁去。木華黎以天應子斡可襲領其眾。癸未春,師還,浮梁未成,顧諸將曰:「橋未畢工,安可坐待乎!」復攻下河西堡寨十餘。三月,渡河,還聞喜縣,疾篤,召其弟帶孫曰:「我為國家助成大業,擐甲執銳垂四十年,東征西討,無復遺恨,第恨汴京未下耳,汝其勉之!」薨,年五十四。厥後太祖親攻鳳翔,謂諸將曰:「使木華黎在,朕不親至此矣。」至治元年,詔封孔溫窟哇推忠效節保大佐運功臣、太師、開府儀同三司、上柱國、魯國王,謚忠宣;木華黎體仁開國輔世佐命功臣、太師、開府儀同三司、上柱國、魯國王,謚忠武。子孛魯嗣。



 孛魯,沈毅魁傑,寬厚愛人,通諸國語,善騎射,年二十七,入朝行在所。時太祖在西域,夏國主李王陰結外援,蓄異圖,密詔孛魯討之。甲申秋九月,攻銀州,克之,斬首數萬級,獲生口馬駝牛羊數十萬,俘監府塔海,命都元帥蒙古不華將兵守其要害而還。乙酉春,復朝行在所。同知真定府事武仙叛,殺都元帥史天倪,脅居民遁於雙門寨。仙弟質於軍中,挈家逃歸,遣撒寒追及於紫荊關,斬之,命天倪弟天澤代領帥府事。丙戌夏,詔封功臣戶口為食邑,曰十投下,孛魯居其首。



 宋將李全陷益都,執元帥張琳送楚州。秋九月,郡王帶孫率兵圍全於益都。冬十二月,孛魯引兵入齊,先遣李喜孫招諭全,全欲降,部將田世榮等不從,殺喜孫。丁亥春三月,全突圍欲走,邀擊大敗之,斬首七千餘級,自相蹂踐溺死不可勝計。夏四月,城中食盡,全降。諸將皆曰:「全勢窮出降,非心服也,今若不誅,後必為患。」孛魯曰:「不然,誅一人易耳。山東未降者尚多,全素得人心,殺之不足以立威,徒失民望。」表聞,詔孛魯便宜處之。乃以全為山東淮南楚州行省,鄭衍德、田世榮副之,郡縣聞風款附,山東悉平。



 時滕州尚為金守,諸將或言炎暑未可進攻,孛魯曰:「主上親督大軍,平定西域數年,未聞當暑不戰,我等安敢自逸乎!」遂促進兵。金兵出戰,敗之,斬三千餘級,其餘老幼開門出降,以州屬石天祿。俾先鋒元帥蕭乃臺統蒙古軍屯濰、兗,課課不花以兵三千屯濰、沂、莒,以備宋。千戶按札統大軍駐河北,備金。九月,師還,至燕,獵於昌平,民持牛酒以獻,卻之。及還,賜館人銀數百兩。聞太祖崩,趨赴北庭,哀毀遘疾。戊子夏五月薨,年三十二。至治元年,詔封純誠開濟保德輔運功臣、太師、開府儀同三司、上柱國、魯國王,謚忠定。



 子七人:長塔思,次速渾察,次霸都魯,次伯亦難,次野蔑干,次野不干,次阿里乞失。



 塔思,一名查剌溫,幼與常兒異,英才大略,綽有祖風。木華黎常曰:「成吾志必此兒也。」及長,每語必先忠孝,曰:「大丈夫受天子厚恩,當效死行陣間,以圖報稱,安能委靡茍且目前,以隳先世勛業哉!」年十八襲爵,遂至雲中。庚寅秋九月,叛將武仙圍潞州,太宗命塔思救之,仙聞之,退軍十餘里。大兵未至,塔思率十餘騎覘賊形勢,仙恐有伏,不敢犯。塔思曰:「日暮矣,待明旦擊之。」是夜五鼓,金將移剌蒲瓦來襲,我師與戰不利,退守沁南。賊還攻潞州,城陷,主將任存死之。冬十月,帝親征,遣萬戶因只吉臺與塔思復取潞州,仙夜遁,邀擊之,斬首七千餘級,以任存侄代領其眾。十一月,帝攻鳳翔,命塔思守潼關以備金兵。河中自石天應死,復為金有。辛卯,帝親攻拔之,金元帥完顏火燎遁,塔思追斬之。壬辰春,睿宗與金兵相拒於汝、漢間,金步騎二十萬,帝命塔思與親王按赤臺、口溫不花合軍先進渡河,以為聲援。至三峰山,與睿宗兵合。金兵成列,將戰,會大雪,分兵四出,塔思冒矢石先挫其鋒,諸軍繼進,大敗金兵,擒移剌蒲瓦。完顏合達單騎走鈞州,追斬之,遂拔鈞州。三月,帝北還,詔塔思與忽都虎統兵,略定河南,諸郡皆降,惟汴京、歸德、蔡州未下。塔思遣使請曰:「臣之祖父,佐興大業,累著勛伐。臣襲世爵,曾無寸效,去歲復失利上黨,罪當萬死,願分攻汴城一隅,以報陛下。」帝壯其言,命卜之,不利,乃止。癸巳秋九月,從定宗於潛邸,東征,擒金咸平宣撫顏萬奴於遼東。萬奴自乙亥歲率眾保東海,至是平之。



 甲午秋七月,朝行在所。時諸王大會,帝顧塔思曰:「先皇帝肇開大業,垂四十年。今中原、西夏、高麗、回鶻諸國皆已臣附,惟東南一隅,尚阻聲教。朕欲躬行天討,卿等以為何如?」群臣未對,塔思對曰:「臣家累世受恩,圖報萬一,正在今日。臣雖駑鈍,願仗天威,掃清淮、浙,何勞大駕親臨不測之地哉!」帝悅曰:「塔思雖年少,英風美績,簡在朕心,終能成我家大事矣。」賜黃金甲、玻璃帶及良弓二十,命與王子曲出總軍南征。乙未冬,拔棗陽。曲出別徇襄、鄧,塔思引兵攻郢。郢瀕漢江,城堅兵精,且多戰艦。塔思命造木筏,遣汶上達魯花赤劉拔都兒將死士五百,乘筏進擊。引騎兵沿岸迎射,大破之,溺死者過半,餘皆走郢,壁堅,不能下,俘生口、馬牛數萬而還。丙申冬十月,復出鄧州,遂至蘄、黃。蘄州遣使獻金帛、牛酒犒師,請曰:「宋小國也,進貢大朝有年矣。惟王以生靈為念。」乃舍之。遂進拔符鎮、六安縣焦家寨。丁酉秋九月,由八柳渡河,入汴京。守臣劉甫置酒大慶殿。塔思曰:「此故金主所居,我人臣也,不可處此。」遂宴於甫家。冬十月,復與口溫不花攻光州,主將黃舜卿降。口溫不花別略黃州。塔思攻大蘇山,斬首數千級,獲生口、牛馬以千數。戊戌春正月,至安慶府,官民皆遁於江東。至北峽關,宋汪統制率兵三千降,遷之尉氏。三月,朝行在所。秋九月,帝宴群臣於行宮,塔思大醉。帝語侍臣曰:「塔思神已逝矣,其能久乎!」冬十二月,還雲中。己亥春三月,薨,年二十八。



 子碩篤兒幼,弟速渾察襲。碩篤兒既長,詔別賜民三千戶為食邑,得建國王旗幟,降五品印一、七品印二,付其家臣,置官屬如列侯故事。碩篤兒薨,子忽都華襲。忽都華薨,子忽都帖木兒襲。忽都帖木兒薨,子寶哥襲。寶哥薨,子道童襲。



 速渾察,性嚴厲,賞罰明信,人莫敢犯。與兄塔思從太宗攻鳳翔有功。將兵抵潼關,與金人戰屢捷。既滅金,皇子闊出攻宋棗陽,入郢,速渾察皆與焉。歲己亥,塔思薨,速渾察襲爵,即上京之西阿兒查禿置營,總中都行省蒙古、漢軍。凡他行省監鎮事,必先白之,定其可否,而後上聞。帝嘗遣使至,見其威容凜然,倜儻有奇氣,所部軍士紀綱整肅,還朝以告。帝曰:「真木華黎家兒也。」他國使有至者,每見皆倉皇失次,不能措辭,必慰撫良久,始得盡其所欲言。左右或諫曰:「諸王百司既莫敢越,而復示之以威,使人怖畏,盍少加寬恕以待之。」速渾察曰:「爾言誠是也,然時有不同,寬猛各有所宜施。天下初附,民心未安,萬一守者自縱,事變忽起,悔之晚矣。」尋薨。延祐三年,贈宣忠同德翊運功臣、太師、開府儀同三司、上柱國,追封為東平郡王,謚忠宣。



 子四人:曰忽林池,襲王爵;曰乃燕;曰相威;曰撒蠻。相威自有傳。



 乃燕,性謙和,好學,以賢能稱。速渾察既薨,憲宗擇於諸子,命乃燕襲爵。乃燕力辭曰:「臣有兄忽林池當襲。」帝曰:「朕知之,然柔弱不能勝。」忽林池亦固讓,乃燕頓首涕泣力辭,不得命,既而曰:「若然,則王爵必不敢受,願代臣兄行軍國之事。」於是忽林池襲為國王,事無巨細,必與乃燕謀議,剖決精當,無所擁滯。世祖在潛籓,常與論事。乃燕敷陳大義,又明習典故。世祖謂左右曰:「乃燕後必可大用。」因號之曰薛禪,猶華言大賢也。乃燕雖居顯要,而小心謹畏,每誨群從子弟曰:「先世從太祖皇帝出入矢石間,被堅執銳,斬將搴旗,勤勞四十餘年,遂成功名。以故一家蒙恩深厚,可謂極矣。慎勿驕惰,以墮先王之名,爾曹戒之。」病卒。世祖聞之,為之悲悼。至正八年,贈中奉大夫、遼陽等處行中書省參知政事、護軍,追封魯郡公。子二人:曰碩德,曰伯顏察兒。



 碩德,通敏有幹才。世祖即位,入宿衛,典朝儀,後同知通政院事。嘗言遼東斡拙、吉烈滅二種民數為寇,宜遣近臣諭之。帝方難其人,僉曰:「惟碩德元勛世胄,可使。」帝深然之,以問碩德,對曰:「先臣從太祖皇帝定天下,不辭險艱,以立勛業。陛下不以臣年少愚戇,願請行。」帝大喜,賜御衣,錫燕以行。碩德至,集諸萬戶陳兵沖要,詰其渠魁誅之。脅從者皆降。帝大悅,賞賚有差。後從征乃顏及使西域,屢建殊勛。卒,贈推忠宣惠寧遠功臣,謚忠敏,加贈資善大夫、嶺北等處行中書省右丞、上護軍,追封魯郡公。



 霸突魯,從世祖征伐,為先鋒元帥,累立戰功。世祖在潛邸,從容語霸突魯曰:「今天下稍定,我欲勸主上駐驛回鶻,以休兵息民,何如?」對曰:「幽燕之地,龍蟠虎踞,形勢雄偉,南控江淮,北連朔漠。且天子必居中以受四方朝覲。大王果欲經營天下,駐驛之所,非燕不可。」世祖憮然曰:「非卿言,我幾失之。」己未秋,命霸突魯率諸軍由蔡伐宋,且移檄諭宋沿邊諸將,遂與世祖兵合而南,五戰皆捷,遂渡大江,傅於鄂。會憲宗崩於蜀,阿里不哥構亂和林,世祖北還,留霸突魯總軍務,以待命。世祖至開平,即位,還定都於燕。嘗曰:「朕居此以臨天下,霸突魯之力也。」師還,中統二年卒於軍。大德八年,追贈推誠宣力翊衛功臣、太師、開府儀同三司、上柱國、東平王,謚武靖。夫人帖木倫,昭睿順聖皇后同母兄也。



 子四人:長安童,次定童,次霸都虎臺;他姬子曰和童,襲國王。安童別有傳。



 塔塔兒臺,孔溫窟哇第三子帶孫郡王之後。父曰忙哥,從憲宗征伐,累立戰功。歲己未,攻合州,會憲宗崩,命塔塔兒臺護靈駕赴北。會阿里不哥叛,拘留數日,逃歸,追騎執以北還,將殺之,親王阿速臺、玉龍塔思曰:「塔塔兒臺乃太師國王之裔,不可殺也。」遂獲免。至元元年,從阿速臺來歸,世祖嘉之,授懷遠大將軍,佩金虎符,世襲東平達魯花赤。命宿衛士四十人,給驛送之官所。蒞官一紀,鎮靜不擾,鄆人賴之以安。卒年四十二,子四人。



 只必,幼嗜讀書,習翰墨。至元十四年監東平,官少中大夫,多善政,以清白稱。嘗出家藏書二千餘卷,置東平廟學,使學徒講肄之。尋授嘉議大夫、江南湖北道提刑按察使,改浙西。大德四年入覲,賜金段十匹。明年春卒,年五十一。子三人,皆早喪。自只必除按察使,弟禿不申嗣其職。



 禿不申,性淳靖,喜怒不形,知民疾苦,而能以善道之。旱嘗致禱,即雨。歲饑,請於朝,發廩以賑之。睦同僚,興學校。加太中大夫。士民刻石,紀其政績雲。卒年五十一。子五人:長不老赤,次塔實脫因,次阿魯灰,次完者不花,次留住馬。皆以次嗣為東平達魯花赤。



 脫脫,祖嗣國王速渾察,沈深有智略。嘗奉命征討,所向克捷。父撒蠻,幼穎異,自襁褓時,世祖撫育之若子。嘗挾之南征,同舟濟大江,慮其有失,系之御榻。及長,常侍左右。帝嘗詔之曰:「男女異路,古制也,況掖庭乎。禮不可不肅,汝其司之。」既而近臣孛羅銜命遽出,行失其次。撒蠻怒其違禮,執而囚之別室。帝怪其久不至,詢知其故,命釋其罪。撒蠻因進曰:「令自陛下出,陛下乃自違之,何以責臣下乎?」帝曰:「卿言誠是也。」由是有意大任之。會以疾卒,不果,年僅一十有七。脫脫幼既失怙,其母孛羅海篤意教之,孜孜若恐不及。稍長,直宿衛,世祖復親誨導,尤以嗜酒為戒。既冠,儀觀甚偉。喜與儒士語,每聞一善言善行,若獲拱璧,終身識之不忘。至元二十四年,從征乃顏。帝駐驛於山巔,旌旗蔽野。鼓未作,候者報有隙可乘,脫脫即擐甲率家奴數十人疾馳擊之。眾皆披靡不敢前。帝望見之,大加嗟賞,遣使者勞之,且召還曰:「卿勿輕進,此寇易擒也。」視其刀已折,馬已中箭矣。帝顧謂近臣曰:「撒蠻不幸早死,脫脫幼,朕撫而教之,常恐其不立,今能如此,撒蠻可謂有子矣。」遂親解佩刀及所乘馬賜之。由是深加器重,得預聞機密之事。其後哈丹復為亂,成宗時在潛邸,督師往征之。脫脫引眾率先躍馬蹙之,其眾大潰。脫脫馬陷於淖泥中,哈丹兵復進挑戰,脫脫弟阿老瓦丁奮戈沖擊,遂大敗之。成宗即位,其寵顧為尤篤,常侍禁闥,出入惟謹,退語家人曰:「我昔親承先帝訓,飭令毋嗜飲,今未能絕也。豈有為人知過而不能改者乎!自今以往,家人有以酒至吾前者,即痛懲之。」帝聞之,喜曰:「扎剌兒臺如脫脫者無幾,今能剛制於酒,真可大用矣。」即拜資德大夫、上都留守、通政院使、虎賁衛親軍都指揮使,政令嚴肅,克修其職。三年,朝議以江浙行省地大人眾,非世臣有重望者,不足以鎮之。進拜榮祿大夫、江浙等處行中書省平章政事,有旨,命中書祖道都門外以餞之。始至,嚴飭左右,毋預公家事,且戒其掾屬曰:「僕從有私囑者,慎勿聽。若軍民諸事,有關於利害者,則言之。當言而不言,爾之責也;言而不聽,我之咎也。」聞者為之悚慄。時硃清、張瑄以海運之故,致位參知政事,恃其勢位,多行不法,恐事覺,以黃金五十兩、珠三囊賂脫脫,求蔽其罪。脫脫大怒,系之有司,遣使者以聞。帝喜曰:「脫脫我家老臣之子孫,其志固宜與眾人殊。」賜內府黃金五十兩,命回使寵賚之。有豪民白晝殺人者,脫脫立命有司按法誅之,自是豪猾屏息,民賴以安。帝以浙民相安之久,未及召還,大德十一年,卒於位,年四十四。子朵兒只,別有傳。



 ○博爾術



 博爾術,阿兒剌氏。始祖孛端察兒,以才武雄朔方。父納忽阿兒闌,與烈祖神元皇帝接境,敦睦鄰好。博爾術志意沉雄,善戰知兵,事太祖於潛邸,共履艱危,義均同氣,征伐四出,無往不從。時諸部未寧,博爾術每警夜,帝寢必安枕。寓直於內,語及政要,或至達旦。君臣之契,猶魚水也。初,要兒斤部卒盜牧馬,博爾術與往追之,時年十三,知眾寡不敵,乃出奇從旁夾擊之,盜舍所掠去。即戰於大赤兀里,兩軍相接,下令殊死戰,跬步勿退。博爾術系馬於腰,跽而引滿,分寸不離故處,太祖嘉其勇膽。又嘗潰圍於怯列,太祖失馬,博爾術擁帝累騎餌馳,頓止中野。會天雨雪,失牙帳所在,臥草澤中,與木華黎張氈裘以蔽帝,同夕植立,足跡不移,及旦,雪深數尺,遂免於難。篾里期之戰,亦以風雪迷陣,再入敵中,求太祖不見,急趨輜重,則帝已還臥憩車中,聞博爾術至,曰:「聰天贊我也。」丙寅歲,太祖即皇帝位,君臣之分益密,嘗從容謂博爾術及木華黎曰:「今國內平定,多汝等之力,我之與汝,猶車之有轅,身之有臂,汝等宜體此勿替。」遂以博爾術及木華黎為左右萬戶,各以其屬翊衛,位在諸將上。皇鬃察哈歹出鎮西域,有旨從博爾術受教,博爾術教以人生經涉險阻,必獲善地,她過無輕舍止。太祖謂皇子曰:「朕之教汝,亦不逾是。」未幾,賜廣平路戶一亡七千三百有奇為分地。以老病薨,太祖痛悼之。大德五年,贈推忠協謀佐運功臣、太師、開府儀同三司,追封廣平王,謚武忠。



 子孛欒臺,襲爵萬戶,贈推誠宣力保順功臣、太師、開府儀同三司,追封圭平王,謚忠定。孫玉昔帖木兒。



 玉昔帖木兒,世祖時嘗寵以不名,賜號月呂魯那演,猶華言能官也。弱冠襲爵,統按臺部眾,器量宏達,莫測其際。世祖聞其賢,驛召赴闕,見其風骨龐乎,解御服銀貂賜之。時重太官內膳之選,特命領其事。侍宴內殿,玉昔帖木兒起性酒,詔諸王妃皆為答禮。至元十二年,拜御史大夫。時江南既定,益封功臣後,遂賜全州清湘縣戶為分地。其在中臺,務振宏綱,弗親細故。興利之臣欲援金就制,並憲司入漕府;當政者又請以郡府之吏,互照憲司檢底。玉昔帖木兒曰:「風憲所以戢奸,若是,有傷監臨之體。」其議乃沮。遇事廷辯,吐辭鯁直,誓祖每為之霽威。



 至元二十四年,宗王乃顏叛東鄙,世祖躬行天討,命總戎者先之。世祖至幫道,玉昔帖木兒已退敵,殭尸覆野,數旬之間,三戰三捷,獲乃顏以獻。詔選誠輿橐駝百蹄勞之。謝曰:「天威所臨,猶風偃草,臣何力之有。」世祖還,留籲昔帖木兒剿其餘黨,乃執其酋金家奴以獻,戮其同惡數人於軍前。明年,乃顏之遺孽哈丹禿魯干復叛,再命出師,兩與之遇,皆敗之,追及兩河,其眾大衄,遂遁。時已盛冬,聲言俟春方進,乃倍道兼行過黑龍江,搗其巢穴,殺戮殆盡。哈丹禿魯幹莫知所終,夷其城,撫其民而還。詔賜內府七寶冠帶以旌之,加太傅、開府儀同三司。申命御邊杭海。二十九年,加錄軍國重事、知樞密院事。宗王帥臣咸稟命焉。特賜步輦入內。位望之崇,廷臣無出其右。



 三十年,成宗以皇孫撫軍北邊,玉昔帖木兒輔行,請授皇孫以儲闈舊璽,詔從之。三十一年,世祖崩,皇孫南還。宗室諸王會於上都。定策之際,玉昔帖木兒起謂晉王甘麻剌曰:「宮車晏駕,已逾三月,神器不可久虛,宗祧不可乏主。疇昔儲闈符璽既有所歸,王為宗盟之長,奚俟而不言。」甘麻剌遽曰:「皇帝踐祚,願北面事之。」於是宗親大臣合辭勸進,玉昔帖木兒復坐,曰:「大事已定,吾死且無憾。」皇孫遂即位。進秩太師,賜以尚方玉帶寶服,還鎮北邊。元貞元年冬,議邊事入朝,兩宮錫宴,如家人禮。賜其妻禿忽魯宴服,及他珍寶。十一月,以疾薨。大德五年,詔贈宣忠同德弼亮功臣,依前太師、開府儀同三司、錄軍國重事、御史大夫,追封廣平王,謚曰貞憲。



 子三人:木剌忽,仍襲爵為萬戶;次脫憐;次脫脫哈,為御史大夫。



 ○博爾忽



 博爾忽,許兀慎氏,事太祖為第一千戶,歿於敵。子脫歡襲職,從憲宗四征不庭,有拓地功。子失里門,鎮徼外,從征六詔等城,亦歿於兵。



 子月赤察兒,性仁厚勤儉,事母以孝聞。資貌英偉,望之如神。世祖雅聞其賢,且閔其父之死,年十六,召見。帝見其容止端重,奏對詳明,喜而謂曰:「失烈門有子矣。」即命領四怯薛太官。至元十七年,長一怯薛。明年詔曰:「月赤察兒秉心忠實,執事敬慎,知無不言,言無不盡,曉暢朝章,言輒稱旨,不可以其年少而弗升其官。可代糸泉真為宣徽使。」



 二十六年,帝討叛者於杭海,眾皆陣,月赤察兒奏曰:「丞相安童、伯顏,御史大夫月呂祿,皆已受命征戰,三人者臣不可以後之。今勍賊逆命,敢御天戈,惟陛下憐臣,使臣一戰。」帝曰:「乃祖博爾忽,佐我太祖,無征不在,無戰不克,其功大矣。卿以為安童輩與爾家同功一體,各立戰功,自恥不逮。然親屬橐鞬,恭衛朝夕,爾功非小,何必身踐行伍,手事斬馘,乃快爾心耶!」



 二十七年,桑哥既立尚書省,殺異己者,箝天下口,以刑爵為貨,既而紀綱大紊。尚書平章政事也速答兒,太官屬也,潛以其事白月赤察兒,請奏劾之。桑哥伏誅,帝曰:「月赤察兒口伐大奸,發其蒙蔽。」乃以沒入桑哥黃金四百兩、白金三千五百兩,及水田、水磑、別墅賞其清強。桑哥既敗,帝以湖廣行省西連番洞諸蠻,南接交趾島夷,延袤數千里,其間土沃人稠,畬丁、溪子善驚好鬥,思得賢方伯往撫安之。月赤察兒舉哈剌哈孫答剌罕以為行省平章政事,凡八年,威德交孚,洽於海外;入為丞相,天下稱賢。世以月赤察兒為知人。二十八年,都水使者請鑿渠西導白浮諸水,經都城中,東入潞河,則江淮之舟既達廣濟渠,可直泊於都城之匯。帝亟欲其成,又不欲役其細民,敕四怯薛人及諸府人專其役,度其高深,畫地分賦之,刻日使畢工。月赤察兒率其屬,著役者服,操畚鍤,即所賦以倡。趨者雲集,依刻而渠成,賜名曰通惠河,公私便之。帝語近臣曰:「是渠非月赤察兒身率眾手,成不速也。」成宗即位,制曰:「月赤察兒盡其誠力,深其謀議,抒忠於國,流惠於人,可加開府儀同三司、太保、錄軍國重事、樞密、宣徽使。」大德四年,拜太師。



 初,金山南北,叛王海都、篤娃據之,不奉正朔垂五十年,時入為寇。嘗命親王統左右部宗王諸帥,屯列大軍,備其沖突。五年,朝議北師少怠,紀律不嚴,命月赤察兒副晉王以督之。是年,海都、篤娃入寇。大軍分為五隊,月赤察兒將其一。鋒既交,頗不利。月赤察兒怒,被甲持矛,身先陷陣,一軍隨之,出敵之背,五軍合擊,大敗之。海都、篤娃遁去,月赤察兒亦罷兵歸鎮。厥後篤娃來請臣附。時武宗亦在軍,月赤察兒遣使詣武宗及諸王將帥議曰:「篤娃請降,為我大利,固當待命於上。然往返再閱月,必失事機。事機一失,為國大患,人民困於轉輸,將士疲於討伐,無有已時矣。篤娃之妻,我弟馬兀合剌之妹也,宜遣使報之,許其臣附。」眾議皆以為允。既遣,始以事聞,帝曰:「月赤察兒深識機宜。」既而馬兀合剌復命,由是叛人稍稍來歸。



 十年冬,叛王滅里鐵木兒等屯於金山,武宗帥師出其不意,先逾金山,月赤察兒以諸軍繼往,壓之以威,啖之以利,滅里鐵木兒乃降。其部人驚潰,月赤察兒遣禿滿鐵木兒、察忽將萬人深入,其部人亦降。察八兒者,海都長子也,海都死,嗣領其眾,至是掩取其部人,凡兩部十餘萬口。至大元年,月赤察兒遣使奏曰:「諸王禿苦滅本懷攜貳,而察八兒游兵近境,叛黨素無悛心,倘合謀致死,則垂成之功顧為國患。臣以為昔者篤娃先眾請和,雖死,宜遣使安撫其子款徹,使不我異。又諸部既已歸明,我之牧地不足,宜處諸降人於金山之陽,吾軍屯田金山之北,軍食既饒,又成重戍,就彼有謀,吾已搗其腹心矣。」奏入,帝曰:「是謀甚善,卿宜移軍阿答罕三撒海地。」月赤察兒既移軍,察八兒、禿苦滅果欲奔款徹,不見納,去留無所,遂相率來降,於是北邊始寧。



 帝詔月赤察兒曰:「卿之先世,佐我祖宗,常為大將,攻城戰野,功烈甚著。卿乃國之元老,宣忠底績,靖謐中外。朕入繼大統,卿之謀猷居多。今立和林等處行中書省,以卿為右丞相,依前太師、錄軍國重事,特封淇陽王,佩黃金印。宗籓將領,實瞻卿麾進退。其益懋乃德,悉乃心力,毋替所服。」四年,月赤察兒入朝,帝宴於大明殿,眷禮優渥。尋以疾薨於第。詔贈宣忠安遠佐運弼亮功臣,謚忠武。



 塔察兒,一名奔盞,居官山。伯祖父博爾忽,從太祖起朔方,直宿衛為火兒赤。火兒赤者,佩橐鞬侍左右者也。由是子孫世其職。博爾忽從太祖平諸國,宣力為多,當時與木華黎等俱以功號四傑。塔察兒,其從孫也,驍勇善戰,幼直宿衛。太祖平燕,睿宗監國,聞燕京盜賊恣意殘殺,直指富庶之家,載運其物,有司不能禁。乃遣塔察兒、耶律楚材窮治其黨,誅首惡十有六人,由是巨盜屏跡。太宗伐金,塔察兒從師,授行省兵馬都元帥,分宿衛與諸王軍士俾統之,下河東諸州郡,濟河,破潼關,取陜洛。辛卯,從圍河中府,拔之。壬辰,從渡白坡。時睿宗已自西和州入興元,由武關出唐、鄧,太宗以睿宗與金兵相持久,乃遣使約期,會兵合進。即詔發諸軍至鈞州,連日大雪,睿宗與金人戰於三峰山,大破之。詔塔察兒等進圍汴城。金主即以兄子曹王訛可為質,太宗與睿宗還河北。塔察兒復與金兵戰於南薰門。癸巳,金主遷蔡州,塔察兒復帥師圍蔡。甲午,滅金,遂留鎮撫中原,分兵屯大河之上,以遏宋兵。丙申,破宋光、息諸州,事聞於朝,以息州軍民三千戶賜之。戊戌卒。



 子別里虎鷿,嗣為火兒赤。憲宗即位,歲壬子,襲父職,總管四萬戶蒙古、漢軍,攻宋兩淮,悉定邊地。戊午,會師圍宋襄陽,逼樊城,力戰死之。



 次曰宋都鷿,至元七年,賜金虎符,襲蒙古軍萬戶。八年,悉兵再攻襄陽,圍樊城,進戰鄂、岳、漢陽、江陵、歸、峽諸州,皆有功。十二年,加昭毅大將軍,受詔為隆興出征都元帥,與李恆等長驅,而宋人莫當其鋒,戰勝攻取,望風迎降,盡平江西十一城,又徇嶺南、廣東。宋亡,還師,未及論功卒。



\end{pinyinscope}