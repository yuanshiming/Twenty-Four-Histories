\article{志第五十一 刑法二}

\begin{pinyinscope}

 ○職制下



 諸職官戶在軍籍,管軍官輒追逮其身者,禁之。諸中外大小軍官,不能以法撫循軍人而又害之者,從監察御史廉訪司糾察之;行省官及宣慰司元帥府官無故以軍官自衛者,亦如之。諸軍官不法,各處憲司就問之,樞府不得委官同問。諸管軍官,輒以所佩金銀符充典質者,笞五十七,降散官一等,受質者減二等。諸軍官犯贓,應罷職殿降者,上所佩符,再敘日給之。諸軍官役使軍人,萬戶八名,千戶減萬戶之半,彈壓減千戶之半,過是數者坐罪。諸軍官驅役軍人,致死非命者,量事斷罪並罷職,徵燒埋銀給苦主。諸管軍官擅放正軍,及分受雇役錢者,以枉法論,除名不敘。諸管軍官吏克除軍人衣糧鹽菜錢,並全未給散,會赦,克除已招者追給,未招者免征,未給散者給散。其私役軍人官牛,帶種官地,並管民官占種官地,所收子粒,已招者追沒,未招者免征。諸軍官役其出征軍人家屬,又借之錢而多取息者,並坐之。諸軍官輒縱軍人誣民以罪,赫取錢物而分贓自厚者,計贓科罪,除名不敘。諸民間失火,鎮守軍官坐視不救,而反縱軍剽掠者,從臺憲官糾之。諸軍官輒斷民訟者,禁之,違者罪之。諸軍官挾仇犯分,輒持刃欲殺連帥者,杖六十七,解職別敘。



 諸投下官吏受贓,與常選官同論。諸投下雜職犯贓罪者罷之,不以常調殿降論。諸投下妄稱上旨,影占民站,除其徭役,故縱為民害者,杖七十七,沒其家財之半,所占民杖一百七,還元籍。諸王傅文卷,監察御史考閱,與有司同。諸位下置財賦營田等司,歲終則會;會畢,從廉訪司考閱之。諸投下輕重囚徒,並從廉訪司審錄。諸籓邸事務,大者奏裁,小者移中書,擅以教令行者,禁之。



 諸倉庾官吏與府州司縣官吏人等,以百姓合納稅糧,通同攬納,接受折價飛鈔者,十石以上,各刺面,杖一百七;十石以下,九十七;官吏除名不敘。退閑官吏、豪勢富戶、行鋪人等違犯者,十石之上,杖九十七;十石之下,八十七。其部糧官吏知情分受,五十七,除名不敘。有失覺察者,監臨部糧官吏,二十七;府州總部糧官吏,一十七。若能捕獲犯人者,與免本罪。若倉官人吏等盜糶官糧,與攬納飛鈔同論。知情糴買,十石以上,杖一百七;七石之下,九十七。其漕運官吏有失覺察者,驗糧數多寡治罪。其盜糶糧價,結攬飛鈔,追徵沒官,正糧於倉官,並結攬糴買人均徵還官。諸倉庫官吏人等盜所主守錢糧,一貫以下,決五十七,至十貫杖六十七,每二十貫加一等;一百二十貫,徒一年,每三十貫加半年;二百四十貫,徒三年;三百貫處死。計贓以至元鈔為則,諸物以當時價估折計之。諸倉庫官、知庫子、攢典、斗腳人等,侵盜移易官物,匿不舉發者,與犯人同罪;失覺察者,減犯人罪四等。諸倉庫錢糧出納,所設首領官及提舉監支納以下攢典合干人以上,互相覺察,若有違法短少,一體均陪,任內收支錢糧,正收倒除皆完,方許給由。諸典守鈔庫官,已倒昏鈔,不用退印,笞五十七,解見任。提調官失計點,笞一十七,並記過名。諸鈔庫官,輒以自己昏鈔詭名倒換者,笞三十七,記過。諸平準行用庫倒換昏鈔,多取工墨錢,庫官知而不曾分贓者,減一等,並解職別敘。主謀又受贓者,以枉法論,除名不敘。諸白紙坊典守官,私受桑楮皮折價者,計贓以枉法論,除名不敘,仍追贓,收買本色還官。諸京倉受糧,部官董之,外倉收糧,州縣長官董之。收不如法致腐敗者,按治官通究之。諸倉官委任親屬為家丁,致盜糶官糧者,笞五十七,解職殿敘;同僚相容隱,四十七,解職。諸倉官輒翻釘官斛,多收民租,主謀者笞五十七,同僚初不知情,既知而不能改正者,三十七,並解職別敘。諸京師每日散糶官米,人止一斗,權豪勢要及有祿之家,輒糴買者,笞二十七,追中統鈔二十五貫,付告人充賞。諸官局造作典守,輒克除材料者,計贓以枉法論,除名不敘。



 諸運司辦課官,取受事發,辦課畢日追問;受代離職者,就問之。諸鹽場官勘問人致死者,從轉運司差官攝其職,發犯人歸有司。諸稅務官,輒以民到務文契,枉作匿稅,私其罰錢者,以枉法論,除名不敘。諸財賦總管淘金提舉司存,雖有護持制書,事應糾劾者,監察御史廉訪司準法行之。諸守庫藏軍官,夜不直宿,致有盜者,笞三十七,還職。捕盜不獲者,圍宿軍官軍人追陪所失物貨,俟獲盜徵贓給還。若遇強劫,軍官軍人力所不及者,不在追斷之限。諸雜造局院,輒與諸人帶造軍器者,禁之。諸兩浙財賦府隸徽政者,掌治錢穀造作,歲終報成,以次年正月至於二月,從廉訪司稽其文書,違者糾之。



 諸有司橋梁不修,道途不治,雖修治而不牢強者,按治及監臨官究治之。諸有司不以時修築堤防,霖雨既降,水潦並至,漂民廬舍,溺民妻子,為民害者,本郡官吏各罰俸一月,縣官各笞二十七,典史各一十七,並記過名。



 諸漕運官,輒拘括水陸舟車,阻滯商旅者,禁之。諸漕運官,輒受贓,縱水手人等以稻糠盜換官糧者,以枉法計贓論罪,除名不敘。諸海道都漕運萬戶府所轄千戶已下有罪,萬戶問之;萬戶有罪,行省問之。徇情者,監察御史廉訪司察之,漕事畢,然後廉訪司考其案牘。諸海道運糧船戶,盜糶官糧,詐稱遭風覆沒者,計贓刺斷,雖會赦,仍刺之。



 諸使臣行李,脫脫禾孫及驛吏輒敢搜檢者,禁之。諸使臣行橐過重,壓損驛馬,而脫脫禾孫與使臣交贈為好,不以法稱盤者,笞二十七,記過。諸急遞鋪,輒開所遞實封文書,妄入無名文字者,笞五十七。諸急遞鋪,每上下半月,府州判官縣主簿親臨檢視,所遞文字但有稽違、磨擦、沉匿,鋪司鋪兵即驗事重輕論罪,各路正官一員總之,廉訪司察之。其有弗職,親臨官初犯笞一十七,再犯加一等,三犯呈省別議,總提調官減親臨官一等。每季具申上司,有無稽違,仍於各官任滿日,解由開寫,而黜陟之。諸使臣輒騎懷駒馬者,取與各笞五十七,及以車易馬者,俱坐之。諸公主下嫁,迎送往還,並不得由傳置。諸使臣在城,輒騎占驛馬者禁之,違者罪之。諸驛使在道,奪回馬易所乘馬,馳至死者,償其直。若以私事故選良馬馳至死者,笞二十七,仍償其直。諸使臣多取分例,笞一十七,追所多還官,記過。使還人員,除軍情急務外,日不過三驛,驛官仍於關文標寫起止程期,違者各笞二十七,再犯罷役。諸乘驛使臣,或枉道營私,橫索祗待,或訪舊逸游,餓損馬乘,並申聞斷治。諸使臣枉道馳驛者,笞五十七;脫脫禾孫擅依隨給驛者,依例科罰。諸驛使詐改公牒,多起馬者,杖八十七;其部押官馬,輒夾帶私馬,多取草料者,並沒入其私馬。諸朝廷軍情大事,奉旨遣使者,佩以金字圓符給驛,其餘小事,止用御寶聖旨。諸王公主駙馬亦為軍情急務遣使者,佩以銀字圓符給驛,其餘止用御寶聖旨。若濫給者,從臺憲官糾察之。諸高麗使臣,所帶徒從,來則俱來,去則俱去,輒留中路郡邑買賣者,禁之;易馬出界者,禁之。諸出使官員,所至輒受官吏筵宴,及官吏輒相邀請,並從風憲糾察。諸使臣所過州縣,無故不得入城。有故入城者,止於公館安宿,輒宿於官民之家者,從風憲糾之。諸遣使開讀詔書,所過州郡就便開讀者聽,非所經由而輒往者禁之。若本宗事須親往者,不在此限。諸使臣所至之處,有親戚故舊,禮應追往者聽。諸受命出使還,匿給驛文字符節及錫貢之物,久不進者,杖六十七,記過。諸進表使臣,五日外不還職,托故稽留,他有營者,止所給驛,籍其姓名,罷黜之。諸出使郡國,使事之外,毋有所與,有必須上聞者,實封以聞。諸銜命出使,輒將有司刑囚審斷者,罪之。諸奉使循行郡縣,有告廉訪司官不法者,若其人嘗為風憲所黜罷,則與監察御史雜問之,余聽專問。諸官吏公差,輒受人贐行禮物者,隨事論罪,官還職,吏發鄰道貼補。



 諸捕盜,境內若失過盜賊,卻獲他境盜賊,許令功過相補。如獲他境強盜,或偽造寶鈔二起,各準境內強盜一起,無強者準竊盜二起。如獲竊盜,準亦如之。如境內無失,但獲強竊盜賊,依例理賞。若應捕之人,及事主等告指捕獲者,不賞。諸捕盜官,不得差遣,違者臺憲官糾之。諸捕盜官,任內失過盜賊,除獲別境盜準折外,三限不獲,強盜三起,竊盜五起,各笞一十七;強盜五起,竊盜十起,各笞二十七;強盜十起,竊盜十五起,各笞三十七。鎮守軍官一體捕限者同罪,親民提控捕盜官,減罪二等。其限內獲賊及半者免罪,若諸人獲盜應賞者,賞之。諸南北兵馬司,職在巡警非違,捕逐盜賊,輒理民訟者,禁之。諸南北兵馬司,罪囚八十七以下,決遣;應刺配者,就刺配之。諸各路在城錄事錄判,分番巡捕,若有失盜,止坐巡捕官。諸職官非應捕之人,告獲反賊者,升二等用。諸告獲強盜,每名官給賞錢至元鈔五十貫,竊盜二十五貫,新獲者倍之,獲強盜至五人與一官。諸捕獲弒逆兇徒,比獲強盜給賞。諸隨處鎮守軍官軍人,親獲強竊盜賊者,減半給賞。諸都城失盜,一年不獲者,勒巡軍陪償所盜財物,其敢差占巡軍者禁之。諸捕盜官捕獲強竊盜,不即牒發,淹禁死亡者,杖七十七,罷職。諸盜牛馬,悔過放還者,以竊盜已行不得財論,不征倍贓賞錢;有司輒以常盜刺斷者,以刑名違錯科罰。諸捕盜官,輒受人遞至匿名文字,枉勘平人為盜,致囚死獄中者,杖九十七,罷職不敘;正問官六十七,降先職二等敘;首領官笞四十七,注邊遠一任;承吏杖六十七,罷役不敘;主意寫匿名文書者,杖一百七,流遠;遞送匿名文書者,減二等;受命主事遞送者,減三等。諸捕盜官搜捕逆賊,輒將平人審問蹤跡,乘怒毆之,邂逅致死者,杖六十七,解職別敘,記過,徵燒埋銀給苦主。諸捕盜官受財故縱賊囚者,與犯人同罪,已敗獲者,徒杖並減一等。諸父有罪,不坐其子;兄有罪,不坐其弟。



 諸大宗正府理斷人命重事,必以漢字立案牘,以公文稱憲臺,然後監察御史審覆之。諸有司非法用刑者,重罪之。已殺之人,輒臠割其肉而去者禁之,違者重罪之。諸鞫獄不能正其心,和其氣,感之以誠,動之以情,推之以理,輒施以大披掛及王侍郎繩索,並法外慘酷之刑者,悉禁止之。諸鞫問罪囚,除朝省委問大獄外,不得寅夜問事,廉訪司察之。諸各路推官專掌推鞫刑獲,平反冤滯,董理州縣刑名之事,其餘庶務,毋有所與,按治官歲錄其殿最,秩滿則上其事而黜陟之。凡推官若受差不聞上司,輒離職者,亦坐罪。諸處斷重囚,雖叛逆,必令臺憲審錄,而後斬於市曹。諸內外囚禁,從各路正官及監察御史廉訪司以時審錄,輕者斷遣,重者結案,其有冤滯,就糾察之。諸正蒙古人,除犯死罪,監禁依常法,有司毋得拷掠,仍日給飲食。犯真奸盜者,解束帶佩囊,散收。餘犯輕重者,以理對證,有司勿執拘之,逃逸者監收。諸奏決天下囚,值上怒,勿輒奏。上欲有所誅,必遲回一二日,乃覆奏。諸有司因公依理決罰,邂逅身死者,不坐。諸累過不悛,年七十以上,應罰贖者,仍減等科決。諸犯罪,二罪俱發,以重者論,罪等從一。若一罪先發,已經論決,餘罪後發,其輕若等,勿論;重者,更論之,通計前罪,以充後數。諸職官輒以微故,乘怒不取招詞,斷決人邂逅致死,又誘苦主焚瘞其尸者,笞五十七,解職別敘,記過。諸鞫獄輒以私怨暴怒,去衣鞭背者,禁之。諸鞫問囚徒,重事須加拷訊者,長貳僚佐會議立案,然後行之,違者重加其罪。諸弓兵祗候獄卒,輒毆死罪囚者,為首杖一百七,為從減一等,均徵燒埋銀給苦主,其枉死應徵倍贓者,免征。諸有司輒改禁無罪之人者,正官並笞一十七,記過。無招枉禁,致自縊而死者,笞三十七,期年後敘。諸有司輒將無辜枉禁,瘐死者,解職,降先品一等敘。諸有司承告被盜,輒將警跡人,非理枉勘身死,卻獲正賊者,正問官笞五十七,解職,期年後,降先職一等敘;首領官及承吏,各五十七,罷役不敘;均徵燒埋銀給苦主,通記過名。諸有司受財故縱正賊,誣執非罪,非法拷訊,連逮妻子,銜冤赴獄,事未曉白,身已就死,正官杖一百七,除名,佐官八十七,降二等雜職敘,仍均徵燒埋銀。諸有司故入人罪,若未決者及囚自死者,以所入罪減一等論,入人全罪,以全罪論,若未決放,仍以減等論。諸故出人之罪,應全科而未決放者,從減等論,仍記過。諸失入人之罪者,減三等,失出人罪者減五等,未決放者又減一等,並記過。諸有司失出人死罪者,笞五十七,解職,期年後降先品一等敘,記過,正犯人追禁結案。諸有司輒將革前雜犯,承問斷遣者,以故入論。諸監臨挾仇,違法枉斷所監臨職官者,抵罪不敘。諸審囚官強愎自用,輒將蒙古人刺字者,杖七十七,除名,將已刺字去之。諸為盜,並從有司歸問,各投下輒擅斷遣者,坐罪。諸鬥毆殺人,無輕重,並結案上省部詳讞。有司輒任情擅斷者,笞五十七,解職,其年後,降先品一等敘。諸禁囚因械梏不嚴,致反獄者,直日押獄杖九十七,獄卒各七十七,司獄及提牢官皆坐罪,百日內全獲者不坐。諸罪在大惡,官吏受贓縱令私和者,罷之。諸司獲受財,縱犯奸囚人,在禁疏枷飲酒者,以枉法科罪,除名。



 諸流囚,強盜持杖不曾傷人,但得財,若得財至二十貫,為從;不持仗,不曾傷人,得財四十貫,為從;及竊盜,割車剜房,傷事主,為從;不曾傷事主,但曾得財;不曾得財,內有舊賊;初犯怯烈司盜駝馬牛,為從;略賣良人為奴婢一人;詐雕都省、行省印;套畫省官押字,動支錢糧,乾礙選法;或妄造妖言犯上:並杖一百七,流奴兒幹。初犯盜駝馬牛,為首;及盜財三百貫以上;盜財十貫以下,經斷再犯;發塚開棺傷尸,內應流者;挑剜裨湊寶鈔,以真作偽,再犯;知情買使偽鈔,三犯:並杖一百七,發肇州屯種。諸犯罪流遠逃歸,再獲,仍流。若中路遭亂而逃,不再犯,及已老病並會赦者,釋之。諸流囚居役,非遇元正、寒食、重午等節,並勿給假。諸配役囚徒,遇閏月,通理之。諸應徒流,未行,會赦者釋之;已行未至,會赦者亦釋之。諸囚徒配役,役所停罷者,會赦,免放。諸有罪,奉旨流遠,雖會赦,非奏請不得放還。諸徒罪,晝則帶鐐居役,夜則入囚牢房。其流罪發各處屯種者,止令監臨關防屯種。諸流遠囚徒,惟女直、高麗二族流湖廣,餘並流奴兒幹及取海青之地。諸徒罪,無配役之所者,發鹽司居役。諸主守失囚者,減囚罪三等,長押流囚官中路失囚者,視提牢官減主守罪四等,既斷還職。諸大小刑獄應監系之人,並送司獄司,分輕重監收。諸掌刑獄,輒縱囚徒在禁飲博,及帶刀刃紙筆陰陽文字入禁者,罪之。



 諸獄具,枷長五尺以上,六尺以下,闊一尺四寸以上,一尺六寸以下,死罪重二十五斤,徒流二十斤,杖罪一十五斤,皆以干木為之,長闊輕重各刻志其上。杻長一尺六寸以上,二尺以下,橫三寸,厚一寸。鎖長八尺以上,一丈二尺以下,鐐連環重三斤。笞大頭徑二分七厘,小頭徑一分七厘,罪五十七以下用之。杖大頭徑三分二厘,小頭徑二分二厘,罪六十七以上用之。訊杖大頭徑四分五厘,小頭徑三分五厘,長三尺五寸,並刊削節目,無令筋膠諸物裝釘。應決者,並用小頭,其決笞及杖者,臀受;拷詞者,臀若股分受,務令均停。



 諸郡縣佐貳及幕官,每月分番提牢,三日一親臨點視,其有枉禁及淹延者,即舉問。月終則具囚數牒次官,其在上都囚禁,從留守司提之。諸南北兵馬司,每月分番提牢,仍令提控案牘兼掌囚禁。諸鹽運司監收鹽徒,每月佐貳官分番董視,與有司同。



 諸內郡官仕雲南者,有罪依常律;土官有罪,罰而不廢。諸左右兩江所部土官,輒興兵相仇殺者,坐以叛逆之罪。其有妄相告言者,以其罪罪之。有司受財妄聽者,以枉法論。諸土官有能愛撫軍民,境內寧謐者,三年一次,保勘升官。其有勛勞,及應升賞承襲,文字至帥府,輒非理疏駁,故為難阻者,罷之。



 祭令



 諸國家有事於效廟,凡獻官及百執事之人,受誓戒之後,散齊宿於正寢,致齊於祀所。散齊日治事如故,不吊喪問疾,不作樂,不判署刑殺文字,不決罰罪人,不與穢惡事。致齊日惟祀事得行,餘悉禁之。諸嶽鎮名山,國家之所秩祀,小民輒僭禮犯義,以祈禱褻瀆者,禁之。諸五岳、四瀆、五鎮,國家秩禮有常,諸王公主駙馬輒遣人降香致祭者,禁之。



 諸郡縣宣聖廟,凡官員使臣軍馬,輒敢館穀於內,有司輒敢聽訟宴飲於內,工官輒敢營造於內,並行禁之。諸書院同。諸每月朔望,郡縣長吏率其參佐僚屬,詣孔子廟拜謁禮畢,從學官升堂講說。其鄉村市鎮,亦擇有學問德行可為師長者,於農隙之時,以教導民。其有視為迂緩而不務者,糾之。



 學規



 諸蒙古、漢人國子監學官任內,驗其教養出格生員多寡,以為升遷。博士教授有闕,從監察御史舉之,其不稱職者黜之,坐及元舉之官。諸國子生悖慢師長、及行禮失儀、言行不謹、講誦不熟、功課不辦、無故廢學、有故不告輒出、告假違限、執事失誤、忿戾鬥爭,並委正、錄糾舉。除悖慢師長別議,餘者初犯戒諭,再犯、三犯約量責罰。其廚人、僕夫、門子,常切在學,供給使令,違者就便決責。諸國學居首善之地,六館諸生,以次升齋,毋或躐等。其有未應升而求升,及曾犯學規者,輕者降之,重者黜之。其教之不以道者,監察御史糾之。諸國子監私試積分生員,其有不事課業,及一切違戾規矩,初犯罰一分,再犯罰二分,三犯除名。已補高等生員,其有違戾規矩,初犯殿試一年,再犯除名,並從學正、錄糾舉。正、錄知見不糾舉者,從本監議罰。在學生員,歲終實歷坐齋不滿半周歲者,並除名。除月假外,其餘告假,不用準算,學正、錄歲終通行考較。漢人生員,三年不能通一經,及不肯篤勤者,勒令出學。諸奎章閣授經郎生員,每月朔望上弦下弦,給假四日;當入宿衛者,給假三日;餘有故須請假者,於授經郎稟說,附歷給假。無故不入學,第一次罰當日會食,第二次於師席前罰拜及當日會食,第三次於學士院及師席前罰拜及當日會食,三次不改,奏聞懲戒黜退。



 諸隨路學校,計其錢糧多寡,養育生徒,提調正官時一詣學督視,必使課講有程,訓迪有法,賞勤罰惰,作成人材,其學政不舉者究之。諸教官在任,侵資錢糧,荒廢廟宇,教養無實,行止不臧,有忝師席,從廉訪司糾之;任滿,有司輒朦朧給由者究之。諸贍學田土,學官職吏或賣熟為荒,減額收租,或受財縱令豪右占佃,陷沒兼並,及巧名冒支者,提調官究之。諸貧寒老病之士,必為眾所尊敬者,保申本路體覆無異,下本學養贍,仍移廉訪司察之;但有冒濫,從提調官改正。諸各處學校,為講習作養之地,有司輒侵借其錢糧者,禁之。教官不稱職者,廉訪司糾之。諸在任及已代教官,輒攜家入學,褻瀆居止者,從廉訪司糾之。



 諸各路醫學大小生員,不令坐齋肄業,有名無實,及在學而訓誨無法,課講鹵莽,茍應故事者,教授、正、錄、提調官罰俸有差。諸醫人於十三科內,不能精通一科者,不得行醫。太醫院不精加考試,輒以私妄舉充隨朝太醫及內外郡縣醫官,內外郡縣醫學不依法考試,輒縱人行醫者,並從監察御史廉訪司察之。



 軍律



 諸軍官離職、屯軍離營、行軍離其部伍者,皆有罪。諸軍官不得擅離部署。赴闕言事,有必合言者,實封附遞以聞。諸隨處軍馬,有久遠營屯,或時暫經過,並從官給糧食,輒妨擾農民、阻滯客旅者,禁之。諸臨陣先退者,處死。諸統軍捕逐寇盜,分守要害,約相為聲援,稽留失期,致殺死將士,仍不即追襲者,處死,雖會赦,罷職不敘。諸軍民官,鎮守邊陲,帥兵擊賊,紀律無統,變易號令,背約失期,形分勢路,致令破軍殺將,或未戰逃歸,或棄城退走,復能建招徠之功者,減其罪,無功者,各以其罪罪之。諸防戍軍人於屯所逃者,杖一百七,再犯者處死。若科定出征,逃匿者,斬以徇。諸軍戶貧乏已經存恤而復逃者,杖八十七,發遣當軍。隱藏者減二等,兩鄰知而不首者,又減隱藏罪二等。諸軍戶告乏求替者,從有司覆實之,其詐妄者,廉訪司憲之。諸各衛扈從漢軍,每戶選練習壯丁一人常充,仍於貼戶內選兩人輪番供役,其有故必合替換者,自萬戶至於百戶,相視所換之可用,然後用之。百戶、千戶、萬戶私換者,驗名數多寡,論罪解降。諸管軍官吏,受錢代替軍空名者,驗入己錢數,以枉法科罪除名。令兄弟子侄驅丁代替者,驗名數多寡,論罪解降。諸軍馬征伐,虜掠良民,兇徒射利,略賣人口,或自賊殺,或以病亡棄尸道路、暴骸溝壑者,嚴行禁止。



 戶婚



 諸匠戶子女,使男習工事,女習黹繡,其輒敢拘刷者,禁之。諸系官當差人戶,非奉朝省文字,輒投充諸王及各投下給使者,論罪。諸僧道還俗,兄弟析居,奴放為良,未入於籍者,應諸王諸子公主駙馬毋拘藏之。民有敢隱藏者,罪之。諸庶民有妄以漏籍戶及土田於諸王公主駙馬呈獻者,論罪;諸投下輒濫收者,亦罪之。諸官吏占人戶供給私用者,治罪。



 諸有司治賦斂急,致貧民鬻男女為輸者,追還所鬻男女,而正有司罪,價勿償。諸生女溺死者,沒其家財之半以勞軍。首者為奴,即以為良。有司失舉者,罪之。諸民戶流移,所在有司起遣復業,輒以闌遺人收之者,禁之。諸鰥寡孤獨,老弱殘疾,窮而無告者,於養濟院收養。應收養而不收養,不應收養而收養者,罪其守宰,按治官常糾察之。諸被災流民,有司招諭復業。其年深不能復業及失所在者,蠲其賦。輒抑民包納者,從臺憲官糾之。諸年穀不熟,人民轉徙,所至既經賑濟,復聚黨持仗,剽劫財物,毆傷平民者,除孤老殘疾不能自贍,任便居住,有司依前存養,其餘有子弟者,驗其家口,計程遠近,支與行糧,次第押還元籍,沿路復為民害者,從所在有司斷遣。



 諸蒙古、回回、契丹、女直、漢人軍前所俘人口,留家者為奴婢,居外附籍者即為良民,已居外復認為奴婢者,沒入其家財。諸收捕叛亂軍人,掠取生口,並從按治官及軍民官一同審閱,實為賊黨妻屬者,給公據付之,無公據者,以掠良民之罪罪之。諸群盜降附,以所劫掠男女充收捕官饋獻者,勿受,仍還為民。無親屬可收系者,使男女相配,聽為民。其留賊所者,悉縱之。諸收到被掠婦人,忘其鄉里,並無親屬可歸者,有司與之嫁聘,所得聘財,與資妝束。諸軍民官輒隱藏降附人民,不令復業者,罪之。諸籍沒人口,元主私典賣者,追收入官,徵價還主。諸投下官員,招占已籍系官民匠戶計者,沒其家財,所占戶歸本籍。諸投下所籍戶,令出五戶絲,餘悉勿與。其有橫斂於民,從臺憲究之。



 諸願棄俗出家為僧道,若本戶丁多,差役不闕,及有兄弟足以侍養父母者,於本籍有司陳請,保勘申路,給據簪剃,違者斷罪歸俗。諸河西僧人有妻子者,當差發、稅糧、鋪馬、次舍與庶民同。其無妻子者,蠲除之。諸父母在,分財異居,父母困乏,不共子職,及同宗有服之親,鰥寡孤獨,老弱殘疾,不能自存,寄食養濟院,不行收養者,重議其罪。親疾亦貧不能給者,許養濟院收錄。



 諸典賣田宅,從有司給據立契,買主賣主隨時赴有司推收稅糧。若買主權豪,官吏阿徇,不即過割,止令賣主納稅,或為分派別戶包納,或為立詭名,但受分文之贓,笞五十七,仍於買主名下,驗元價追徵,以半沒官,半付告者。首領官及所掌吏,斷罪罷役。諸典賣田宅,須從尊長書押,給據立帳,歷問有服房親及鄰人典主,不願交易者,限十日批退,違限不批退者,笞一十七。願者限十五日議價,立契成交,違限不酬價者,笞二十七。任便交易,親鄰典主故相邀阻,需求書字錢物者,笞二十七。業主虛張高價,不相由問成交者,笞三十七,仍聽親鄰典主百日收贖,限外不得爭訴。業主欺昧,故不交業者,笞四十七。親鄰典主在他所者,百里之外,不在由問之限。若違例事覺,有司不以理聽斷者,監察御史廉訪司糾之。諸軍官軍人不歸營屯,到任官員不歸官舍,往來使臣不歸館驛,輒於民家居止,為民害者,行省行臺起遣究治。到任官無官舍,出私錢僦居者聽。諸造謀以已賣田宅,誣買主占奪,脅取錢物者,計贓論罪,仍紅泥粉壁書過於門。諸婚田訴訟,必於本年結絕,已經務停而不結絕者,從廉訪司及本管上司,正官吏之罪。累經務停,而不結絕者,即與歸結,不在務停之限,違者罪亦如之。其所爭田內租入,納稅之外,並從有司收貯,斷後隨田給付。



 諸以女子典雇於人,及典雇人之子女者,並禁止之。若已典雇,願以婚嫁之禮為妻妾者,聽。諸受錢典雇妻妾者,禁。其夫婦同雇而不相離者,聽。諸受財嫁賣妻妾及過房弟妹者,禁。諸乞養過房男女者,聽;轉賣為奴婢者,禁之。奴婢過房良民者,禁之。諸守宰抑取部民男女為奴婢者,杖七十七,期年後降二等雜職敘。諸妄認良人為奴,非理殘虐者,杖八十七,有官者罷之。諸訴良得實,給據居住,候元籍親屬收領,無親屬者聽令自便。諸奴婢背主在逃,杖七十七。



 諸男女議婚,有以指腹割衿為定者,禁之。諸嫁娶之家,飲食宴好,求足成禮,以華侈相尚,暮夜不休者,禁之。諸男女婚姻,媒氏違例多索聘財,及多取媒利者,諭眾決譴。諸女子已許嫁而未成婚,其夫家犯叛逆,應沒入者,若其夫為盜及犯流遠者,皆聽改嫁。已成婚有子,其夫雖為盜受罪,勿改嫁。諸男女既定婚,其女犯奸事覺,夫家欲棄,則追還聘財,不棄則減半成婚。若夫家輒詭以風聞奸事,恐脅成親者,笞五十七,離之。諸遭父母喪,忘哀拜靈成婚者,杖八十七,離之,有官者罷之,仍沒其聘財,婦人不坐。諸服內定婚,各減服內成親罪二等,仍離之,聘財沒官。諸有女許嫁,已報書及有私約,或已受聘財而輒悔者,笞三十七;更許他人者,笞四十七;已成婚者,五十七;後娶知情者,減一等,女歸前夫。男家悔者,不坐,不追聘財,五年無故不娶者,有司給據改嫁。諸有女納婿,復逐婿,納他人為婿者,杖六十七。後婿同其罪,女歸前夫,聘財沒官。諸職官娶娼為妻者,笞五十七,解職,離之。諸有妻妾復娶妻妾者,笞四十七,離之。在官者,解職記過,不追聘財。諸先通奸被斷,復娶以為妻妾者,雖有所生男女,猶離之。諸轉嫁已歸未成婚男婦者,杖六十七,婦歸宗,聘財沒官。諸受財以妻轉嫁者,杖六十七,追還聘財;娶者不知情,不坐,婦人歸宗。諸以書幣娶人女為妾,復受財轉嫁他人者,笞五十七,聘財沒官,妾歸宗,有官者罷之。諸僧道悖教娶妻者,杖六十七,離之,僧道還俗為民,聘財沒官。諸典賣佃戶者,禁。佃戶嫁娶,從其父母。諸兄收弟婦者,杖一百七,婦九十七,離之。雖出道,仍坐。主婚笞五十七,行媒三十七。諸居父母喪,奸收庶母者,各杖一百七,離之,有官者除名。諸漢人、南人,父沒子收其庶母,兄沒弟收其嫂者,禁之。諸姑表兄弟嫂叔不相收,收者以奸論。諸奴收主妻者,以奸論;強收主女者,處死。諸為子輒以亡父之妾與人,人輒受而私之,與者杖七十七,受者笞五十七。諸受財強嫁所監臨妻,以枉法論,杖七十七,除名,追財沒官,妻還前夫。諸良家女願與人奴為婚者,即為奴婢。娶良家女為妻,以為奴婢賣之者,即改正為良,賣主買主同罪,價沒官。諸以童養未成婚男婦轉配其奴者,笞五十七,婦歸宗,不追聘財。諸逃奴有女,嫁為良人妻,已有男女,而本主覺察者,追其聘財歸本主,婦人不離。諸棄妻,已歸宗改嫁者,從其後夫。諸棄妻改嫁,後夫亡,復納以為妻者,離之。諸夫婦不相睦,賣休買休者禁之,違者罪之,和離者不坐。諸出妻妾,須約以書契,聽其改嫁。以手模為徵者,禁之。諸婦人背夫、棄舅姑出家為尼者,杖六十七,還其夫。諸賣買良人為倡,賣主買主同罪,婦還為良,價錢半沒官,半付告者。或婦人自陳,或因事發覺,全沒入之。良家婦犯奸,為夫所棄,或倡優親屬,願為倡者聽。諸倡女孕,勒令墮胎者,犯人坐罪,倡放為良。諸勒妻妾為倡者,杖八十七。以乞養良家女,為人歌舞,給宴樂,及勒為倡者,杖七十七,婦人並歸宗。勒奴婢為倡者,笞四十七,婦人放從良。諸受財縱妻妾為倡者,本夫與奸婦奸夫各杖八十七,離之。其妻妾隨時自首者,不坐;若日月已久才自首者,勿聽。



\end{pinyinscope}