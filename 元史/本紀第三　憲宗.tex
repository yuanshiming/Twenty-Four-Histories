\article{本紀第三 憲宗}

\begin{pinyinscope}

 憲宗桓肅皇帝,諱蒙哥,睿宗拖雷之長子也。母曰莊聖太后,怯烈氏,諱唆魯禾帖尼。歲戊辰羅納德即「摩訶提婆·戈文德·羅納德」。,十二月三日生帝。時有黃忽答部知天象者,言帝後必大貴,故以蒙哥為名。蒙哥,華言長子也。太宗在潛邸,養以為子,屬昂灰皇后撫育之。既長,為娶火魯剌部女火裡差為妃,分之部民。及睿宗薨,乃命歸籓邸。從征伐,屢立奇功。嘗攻欽察部,其酋八赤蠻逃於海島。帝聞,亟進師,至其地,適大風刮海水去,其淺可渡。帝喜曰:「此天開道與我也。」遂進屠其眾,擒八赤蠻,命之跪,八赤蠻曰:「我為一國主,豈茍求生?且身非駝,何以跪人為?」乃命囚之。八赤蠻謂守者曰:「我之竄入於海,與魚何異,然終見擒,天也。今水回期且至,軍宜早還。」帝聞之,即班師,而水已至,後軍有浮渡者。復與諸王拔都征斡羅思部,至也烈贊城,躬自搏戰,破之。



 歲戊申,定宗崩,朝廷久未立君,中外洶洶,咸屬意於帝,而覬覦者眾,議未決。諸王拔都木哥、阿里不哥、唆亦哥禿、塔察兒,大將兀良合臺、速你帶、帖木迭兒、也速不花,咸會於阿剌脫忽剌兀之地,拔都首建議推戴。時定宗皇後海迷失所遣使者八剌在坐,曰:「昔太宗命以皇孫失烈門為嗣,諸王百官皆與聞之。今失烈門故在,而議欲他屬,將置之何地耶?」木哥曰:「太宗有命,誰敢違之。然前議立定宗,由皇后脫列忽乃與汝輩為之,是則違太宗之命者,汝等也,今尚誰咎耶?」八剌語塞。兀良合臺曰:「蒙哥聰明睿知,人咸知之,拔都之議良是。」拔都即申令於眾,眾悉應之,議遂定。



 元年辛亥夏六月,西方諸王別兒哥、脫哈帖木兒,東方諸王也古、脫忽、亦孫哥、按只帶、塔察兒、別里古帶,西方諸大將班裏赤等,東方諸大將也速不花等,復大會於闊帖兀阿闌之地,共推帝即皇帝位於斡難河。失烈門及諸弟腦忽等心不能平,有後言。帝遣諸王旭烈與忙可撒兒帥兵覘之。諸王也速忙可、不里、火者等後期不至,遣不憐吉鷿率兵備之。遂改更庶政:命皇弟忽必烈領治蒙古、漢地民戶;遣塔兒、斡魯不、察乞剌、賽典赤、趙璧等詣燕京,撫諭軍民;以忙哥撒兒為斷事官;以孛魯合掌宣發號令、朝覲貢獻及內外聞奏諸事;以晃兀兒留守和林宮闕、帑藏,阿藍答兒副之;以牙剌瓦赤、不只兒、斡魯不、睹答兒等充燕京等處行尚書省事,賽典赤、匿昝馬丁佐之;以訥懷、塔剌海、麻速忽等充別失八里等處行尚書省事,暗都剌兀尊、阿合馬、也的沙佐之;以阿兒渾充阿毋河等處行尚書省事,法合魯丁、匿隻馬丁佐之;以茶寒、葉了乾統兩淮等處蒙古、漢軍,以帶答兒統四川等處蒙古、漢軍,以和里鷿統土蕃等處蒙古、漢軍,皆仍前征進;以僧海雲掌釋教事,以道士李真常掌道教事。葉孫脫、按只鷿、暢吉、爪難、合答曲憐、阿里出及剛疙疸、阿散、忽都魯等,務持兩端,坐誘諸王為亂,並伏誅。遂頒便益事宜於國中:凡朝廷及諸王濫發牌印、詔旨、宣命,盡收之;諸王馳驛,許乘三馬,遠行亦不過四;諸王不得擅招民戶;諸官屬不得以朝覲為名賦斂民財;民糧遠輸者,許於近倉輸之。罷築和林城役千五百人。冬,以宴只吉帶違命,遣合丹誅之,仍籍其家。



 二年壬子春正月,幸失灰之地,遣乞都不花攻末來吉兒都怯寨。皇太后崩。夏,駐蹕和林。分遷諸王於各所:各丹於別石八里地,蔑里於葉兒的石河,海都於海押立地,別兒哥於曲兒只地,脫脫於葉密立地,蒙哥都及太宗皇後乞里吉忽帖尼於擴端所居地之西。仍以太宗諸后妃家貲分賜親王。定宗後及失烈門母以厭禳事覺,並賜死,謫失烈門、也速、孛裡等於沒脫赤之地,禁錮和只、納忽、也孫脫等於軍營。秋七月,命忽必烈征大理,諸王禿兒花、撒立征身毒,怯的不花征沒里奚,旭烈征西域素丹諸國。詔諭宋荊南、襄陽、樊城、均州諸守將,使求附。八月,忽必烈次臨洮,命總帥汪田哥以城利州聞,欲為取蜀之計。冬十月,命諸王也古征高麗。帝駐蹕月帖古忽闌之地。時帝因獵墜馬傷臂,不視朝百餘日。



 十月戊午,大赦天下。以帖哥紬、闊闊術等掌帑藏;孛闌合剌孫掌斡脫;阿忽察掌祭祀、醫巫、卜筮,阿剌不花副之。諸王合剌薨。以只兒斡帶掌傳驛所需,孛魯合掌必闍赤寫發宣詔及諸色目官職。徙諸匠五百戶修行宮。是歲,籍漢地民戶。諸王旭烈薨。



 三年癸丑春正月,汪田哥修治利州,且屯田,蜀人莫敢侵軼。帝獵於怯蹇義罕之地。諸王也古以怨襲諸王塔剌兒營。帝遂會諸王於斡難河北,賜予甚厚。罷也古征高麗兵,以札剌兒帶為征東元帥。遣必闍別兒哥括斡羅思戶口。三月,大兵攻海州,戍將王國昌逆戰於城下,敗之,獲都統一人。夏六月,命諸王旭烈兀及兀良合臺等帥師征西域哈裏發八哈塔等國。又命塔塔兒帶撒里、土魯花等征欣都思、怯失迷兒等國。帝幸火兒忽納要不兒之地。諸王拔都遣脫必察詣行在,乞買珠銀萬錠,以千錠授之,仍詔諭之曰:「太祖、太宗之財,若此費用,何以給諸王之賜!王宜詳審之。此銀就充今後歲賜之數。」秋,幸軍腦兒。以忙可撒兒為萬戶,哈丹為札魯花赤。九月,忽必烈次忒剌地,分兵三道以進。冬十二月,大理平。帝駐蹕汪吉地。命宗王耶虎與洪福源同領軍征高麗,攻拔禾山、東州、春州、三角山、楊根、天龍等城。是歲,斷事官忙哥撒兒卒。



 四年甲寅春,帝獵於怯蹇義罕。夏,幸月兒滅怯土之地。遣札剌亦兒部人火兒赤征高麗。秋七月,詔官吏之赴朝理算錢糧者,許自首不公,仍禁以後浮費。冬,大獵於也滅乾哈里義海之地。忽必烈還自大理,留兀良合臺攻諸夷之未附者,入覲於獵所。是歲,會諸王於顆顆腦兒之西,乃祭天於日月山。初籍新軍。帝謂大臣,求可以慎固封守、閑於將略者。擢史樞征行萬戶,配以真定、相、衛、懷、孟諸軍,駐唐、鄧。張柔移鎮亳州。權萬戶史權屯鄧州。張柔遣張信將八漢軍戍潁州。王安國將四千戶渡漢南,深入而還。張柔以連歲勤兵,兩淮艱於糧運,奏據亳之利。詔柔率山前八軍,城而戍之。柔又以渦水北隘淺不可舟,軍既病涉,曹、濮、魏、博粟皆不至,乃築甬路自亳抵汴,堤百二十里,流深而不能築,復為橋十五,或廣八十尺,橫以二堡戍之。均州總管孫嗣遣人齎蠟書降,且乞援,史權以精甲備宋人之要,遂援嗣而來。其後驍將鐘顯、王梅、杜柔、袁師信各帥所部來降。



 五年乙卯春,詔徵逋欠錢穀。夏,帝幸月兒滅怯土。秋九月,張柔會大帥於符離。以百丈口為宋往來之道,可容萬艘,遂築甬路,自亳而南六十餘里,中為橫江堡。又以路東六十里皆水,可致宋舟,乃立柵水中,惟密置偵邏於所達之路,由是鹿邑、寧陵、考、柘、楚丘、南頓無宋患,陳、蔡、潁、息皆通矣。是歲,改命札剌鷿與洪福源同征高麗。後此又連三歲,攻拔其光州、安城、忠州、玄風、珍原、甲向、玉果等城。



 六年丙辰春,大風起北方,砂礫飛揚,白日晦冥。帝會諸王、百官於欲兒陌哥都之地,設宴六十餘日,賜金帛有差,仍定擬諸王歲賜錢穀。忽必烈遣沒兒合石詣行在所,奏請續簽內郡漢軍,從之。夏四月,駐蹕於塔密兒。五月,幸昔剌兀魯朵。六月,太白晝見。幸鷿亦兒阿答。諸王亦孫哥、駙馬也速兒等請伐宋。帝亦以宋人違命囚使,會議伐之。秋七月,命諸王各還所部以居。諸王塔察兒、駙馬帖裏垓軍過東平諸處,掠民羊豕,帝聞,遣使問罪,由是諸軍無犯者。是歲,波麗國王細嵯甫、雲南酋長摩合羅嵯及素丹諸國來覲。兀良合臺討白蠻等,克之;遂自昔八兒地還至重慶府,敗宋將張都統。賜金縷織文衣一襲、銀五十兩、彩帛萬二百匹,以賚軍士。冬,帝駐蹕阿塔哈帖乞兒蠻。以阿木河回回降民分賜諸王百官。



 七年丁巳春,幸忽闌也兒吉。詔諸王出師征宋。乞都不花等討末來吉兒都怯寨,平之。夏六月,謁太祖行宮,祭旗鼓,復會於怯魯連之地,還幸月兒滅怯土。秋,駐蹕於軍腦兒,釃馬乳祭天。九月,出師南征。以駙馬剌真之子乞鷿為達魯花赤,鎮守斡羅思,仍賜馬三百、羊五千。回鶻獻水精盆、珍珠傘等物,可直銀三萬餘錠。帝曰:「方今百姓疲弊,所急者錢爾,朕獨有此何為?」卻之。賽典赤以為言,帝稍償其直,且禁其勿復有所獻。宗王塔察兒率諸軍南征,圍樊城,霖雨連月,乃班師。元帥卜鄰吉鷿軍自鄧州略地,遂渡漢江。冬十一月,兀良合臺伐交趾,敗之,入其國。安南主陳日煚竄海島,遂班師。遣阿藍答兒、脫因、囊加臺等詣陜西等處理算錢穀。冬,帝度漠南,至於玉龍棧。忽必烈及諸王阿里不哥、八里土、出木哈兒、玉龍塔失、昔烈吉、公主脫滅乾等來迎,大燕,既而各遣歸所部。



 八年戊午春正月朔,幸也裏本朵哈之地,受朝賀。二月,陳日煚傳國於長子光昺。光昺遣婿與其國人以方物來見,兀良合臺送詣行在所。諸王旭烈兀討回回哈裏發,平之,禽其王,遣使來獻捷。帝獵於也里海牙之地。師南征,次於河。適冰合,以土覆之而渡。帝自將伐宋,由西蜀以入。命張柔從忽必烈征鄂,趨杭州。命塔察攻荊山,分宋兵力。宋四川制置使蒲澤之攻成都,紐鄰率師與戰,敗之;進攻雲頂山,守將姚某等以眾相繼來降。詔以紐鄰為都元帥。帝由東勝渡河。遣參知政事劉太平括興元戶口。三月,命洪茶丘率師從札剌鷿同征高麗。夏四月,駐蹕六盤山,諸郡縣守令來覲。豐州千戶郭燧奏請續簽軍千人修治金州,從之。是時,軍四萬,號十萬,分三道而進:帝由隴州入散關,諸王莫哥由洋州入米倉關,孛里義萬戶由漁關入沔州。以明安答兒為太傅,守京兆。詔徵益都行省李亶兵,璮來言:「益都南北要沖,兵不可撤。」從之。璮還,擊海州、漣水等處。五月,皇子阿速帶因獵獨騎傷民稼,帝見讓之,遂撻近侍數人。士卒有拔民蔥者,即斬以徇。由是秋毫莫敢犯。仍賜所經郡守各有差。秋七月,留輜重於六盤山,率兵由寶雞攻重貴山,所至輒平。八月辛丑,璮與宋人戰,殺宋師殆盡。九月,駐蹕漢中。都元帥紐鄰留密里火者、劉黑馬等守成都,悉率餘兵渡馬湖,禽宋制置使張實,遂遣實招諭苦竹隘,實遁。冬十月壬午,帝次寶峰。癸未,如利州,觀其城池並非深固,以汪田哥能守,蜀不敢犯,賜卮酒獎諭之。帝渡嘉陵江,至白水江,命田哥造浮梁以濟,梁成,賜田哥等金帛有差。帝駐蹕劍門。戊子,攻苦竹隘,裨將趙仲竊獻東南門,師入,與其守將楊立戰,敗之,殺立,眾皆奔潰。詔毋犯趙仲家屬,仍賜仲衣帽,徙於隆慶。己亥,獲張實,支解之。賜田哥玉帶及犒賞士卒,留精兵五百守之。遣使招諭龍州。帝至高峰。庚子,圍長寧山,守將王佐、裨將徐昕等率兵出戰,敗之。十一月己酉,帝督軍先攻鵝頂堡。壬子,力戰於望喜門。薄暮,宋知縣王仲由鵝頂堡出降。是夜破其城,王佐死焉。癸丑,誅佐之子及徐昕等四十餘人。以彭天祥為達魯花赤治其事,王仲副之。丙辰,進攻大獲山,守將楊大淵降,命大淵為四川侍郎,仍以其兵從。庚午,次和溪口,遣驍騎略青居山。是月,龍州王知府降。諸王莫哥都攻禮義山不克,諸王塔察兒略地至江而還,並會於行在所。命忽必烈統諸路蒙古、漢軍伐宋。十二月壬午,楊大淵率所部兵與汪田哥分擊相如等縣。都元帥紐鄰攻簡州,以宋降將張威率眾為先鋒。乙酉,帝次於運山。大淵遣人招降其守將張大悅,仍以大悅為元帥。師至青居山,裨將劉淵等殺都統段元鑒降。庚寅,遣使招諭未附。丁酉,隆州守縣降。己亥,大良山守將蒲元圭降。詔諸軍毋俘掠。癸卯,攻雅州,拔之。石泉守將趙順降。甲辰,遣宋人晉國寶招諭合州守將王堅,堅辭之,國寶遂歸。是歲,皇子辨都薨於吉河之南。



 九年己未春正月乙巳朔,駐蹕重貴山北,置酒大會,因問諸王、駙馬、百官曰:「今在宋境,夏暑且至,汝等其謂可居否乎?」札剌亦兒部人脫歡曰:「南土瘴癘,上宜北還,所獲人民,委吏治之便。」阿兒剌部人八里赤曰:「脫歡怯,臣願往居焉。」帝善之。戊申,晉國寶歸次峽口,王堅追還殺之。諸王莫哥都復攻渠州禮義山,曳剌禿魯雄攻巴州平梁山。丁卯,大淵請攻合州,俘男女八萬餘。



 二月丙子,帝悉率諸兵渡雞爪灘,至石子山。丁丑,督諸軍戰城下。辛巳,攻一字城。癸未,攻鎮西門。三月,攻東新門、奇勝門、鎮西門小堡。夏四月丙子,大雷雨凡二十日。乙未,攻護國門。丁酉,夜登外城,殺宋兵甚眾。五月,屢攻不克。六月丁巳,汪田哥復選兵夜登外城馬軍寨,殺寨主及守城者。王堅率兵來戰。遲明,遇雨,梯折,後軍不克進而止。是月,帝不豫。秋七月辛亥,留精兵三千守之,餘悉攻重慶。癸亥,帝崩於釣魚山,壽五十有二,在位九年。追謚桓肅皇帝,廟號憲宗。



 帝剛明雄毅,沉斷而寡言,不樂燕飲,不好侈靡,雖后妃不許之過制。初,太宗朝,群臣擅權,政出多門。至是,凡有詔旨,帝必親起草,更易數四,然後行之。御群臣甚嚴,嘗諭旨曰:「爾輩若得朕獎諭之言,即志氣驕逸,志氣驕逸,而災禍有不隨至者乎?爾輩其戒之。」性喜畋獵,自謂遵祖宗之法,不蹈襲他國所為。然酷信巫覡卜筮之術,凡行事必謹叩之,殆無虛日,終不自厭也。



\end{pinyinscope}