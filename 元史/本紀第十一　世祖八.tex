\article{本紀第十一 世祖八}

\begin{pinyinscope}

 十七年春正月癸卯朔,高麗國王王睶遣其僉議中贊金方慶來賀,兼奉歲貢。丙午,命萬戶綦公直戍別失八里,賜鈔一萬二千五百錠。辛亥,磁州、永平縣水,給鈔貸之。丙辰,立遷轉官員法:凡無過者授見闕,物故及過犯者選人補之,滿代者令還家以俟。又定諸路差稅課程,增益者即上報,隱漏者罪之,不須履畝增稅,以搖百姓。詔括江淮銅及銅錢銅器。辛酉,以海賊賀文達所掠良婦百三十餘人還其家。廣西廉州海賊霍公明、鄭仲龍等伏誅。甲子,敕泉州行省所轄州郡山寨未即歸附者率兵拔之,已拔復叛者屠之。以總管張瑄、千戶羅璧收宋二王有功,升瑄沿海招討使,虎符;璧管軍總管,金符。丁卯,畋近郊。詔毋以侍衛軍供工匠役。戊辰,敕相威檢核阿里海牙、忽都帖木兒等所俘丁三萬二千餘人,並放為民。置行中書省於福州。改德慶路為總管府。賜開灤河五衛軍鈔。



 二月乙亥,張易言:「高和尚有秘術,能役鬼為兵,遙制敵人。」命和禮霍孫將兵與高和尚同赴北邊。丙子,立北京道二驛。丁丑,答里不罕以雲南行省軍攻定昌路,擒總管穀納,殺之。詔令答里不罕還,以阿答代之。敕非遠方歸附人毋入會同館。詔納速剌丁將精兵萬人征緬國。乙酉,賞納速剌丁所部征金齒功銀五千三百二十兩。己丑,命梅國賓襲其父應春瀘州安撫使職。瀘州嘗叛,應春為前重慶制置使張玨所殺。國賓詣闕訴冤,詔以玨畀國賓,使復其父仇。玨時在京兆,聞之自經死。國賓請贖還瀘州軍民之為俘者,從之。日本國殺國使杜世忠等,征東元帥忻都、洪茶丘請自率兵往討,廷議姑少緩之。丙申,詔諭真人祁志誠等焚毀《道藏》偽妄經文及板。庚子,阿里海牙及納速剌丁招緬國及洞蠻降臣,詔就軍前定錄其功以聞。江淮行省左丞夏貴請老,從之,仍官其子孫。合剌所部和州等城為叛兵所掠者,賜鈔給之,仍免其民差役三年。發侍衛軍三千浚通州運糧河。畏吾戶居河西界者,令其屯田。辛丑,以廣中民不聊生,召右丞塔出、左丞呂師夔廷詰壞民之由,命也的迷失、賈居貞行宣慰司往撫之。師夔至,廷辯無驗,復令還省治事。詔王相府於諸奧魯市馬二萬六千三百匹。遣使代祀岳瀆。賜諸王阿八合、那木干所部,及征日本行省阿剌罕、範文虎等西錦衣、銀鈔、幣帛各有差。又賜四川貧民及兀剌帶等馬牛羊價鈔。



 三月癸卯,命福建王積翁入領省事,中書省臣以為不可,改戶部尚書。甲辰,車駕幸上都。思、播州軍侵鎮遠、黃平界,命李德輝等往視之。罷通政院官不勝任者。丙午,敕東西兩川發蒙古、漢軍戍魚通、黎、雅。乙卯,立都功德使司,從二品,掌奏帝師所統僧人並吐番軍民等事。己未,詔討羅氏鬼國,命以蒙古軍六千,哈剌章軍一萬,西川藥剌海、萬家奴軍萬人,阿里海牙軍萬人,三道並進。癸亥,高郵等處饑,賑粟九千四百石。辛未,立畏吾境內交鈔提舉司。給月脫古思八部屯田牛具。賜忙古帶等羊馬及皇子南木合下羊馬價。



 夏四月壬申朔,中書省臣言:「唆都軍士擾民,故南劍等路民復叛。及忙古帶往招徠之,民始獲安。」詔以忙古帶仍行省福州。癸酉,南康杜可用叛,命史弼討擒之。定杭州宣慰司官四員,以游顯、管如德、忽都虎、劉宣充之。丙子,隆興路楊門站復為懷安縣。庚辰,四川宣慰使也罕的斤請賜海青符,命以二符給之。壬午,史弼入朝。乙酉,以宋太常樂付太常寺。改泗州靈壁縣仍隸宿州。丁亥,立杭州路金玉總管府。甲午,敕軍戶貧乏者還民籍。丙申,以羅佐山道梗,敕阿里海牙發軍千人戍守。以隆興、泉州、福建置三省不便,命廷臣集議以聞。己亥,諸王只必帖木兒請各投下設官,不從。庚子,歲星犯軒轅大星。敕權停百官俸。寧海、益都等四郡霜,真定七郡蟲,皆損桑。



 五月辛丑朔,樞密院調兵六百守居庸南、北口。甲辰,作行宮於察罕腦兒。丙午,升沙州為路。癸丑,括沙州戶丁,定常賦,其富戶餘田令所戍漢軍耕種。詔雲南行省發四川軍萬人,命藥剌海領之,與前所遣將同征緬國。高麗國王王春以民饑,乞貸糧萬石,從之。福建行省移泉州。甲寅,汀、漳叛賊廖得勝等伏誅。造船三千艘,敕耽羅發材木給之。庚申,賜諸王別乞帖木兒銀印。辛酉,賜國師掌教所印。賞伯顏將士戰功銀二萬八千七百五十兩。真定、咸平、忻州、漣、海、邳、宿諸州郡蝗。



 六月辛未朔,以忽都帶兒牧籍闌遺人民牛畜,撥荒地令屯田。壬申,復招諭占城國。丁丑,唆都部下顧總管聚黨於海道劫奪商貨,範文虎招降之,復議置於法,命文虎等集議處之。阿答海等請罷江南所立稅課提舉司,阿合馬力爭,詔御史臺選官檢核,具實以聞。阿合馬請立大宗正府。罷上都奧魯官,以留守司兼管奧魯事。安西王薨,罷其王相府。遣呂告蠻部安撫使王阿濟同萬戶昝坤招諭羅氏鬼國。壬辰,召範文虎議征日本。戊戌,高麗王王睶遣其將軍樸義來貢方物。江淮等處頒行鈔法,廢宋銅錢。遣不魯合答等檢核江淮行省阿里伯、燕帖木兒錢穀。改泗州隸淮安路。賜忽烈禿、忽不剌等將士力戰者銀鈔、及給折可察兒等軍士羊馬價鈔各有差。



 秋七月辛丑,廣東宣慰使帖木兒不花言:「諸軍官宜一例遷轉。江淮郡縣,首亂者誅,沒其家。官豪隱庇佃民,不供徭役,宜別立籍。各萬戶軍交參重役,宜發還元翼。」詔中書省、樞密院、翰林院集議以聞。敕思州安撫司還舊治。戊申,太陰犯房距星。以高麗國初置驛,站民乏食,命給糧一歲,仍禁使臣往來,勿求索飲食。己酉,立行省於京兆,以前安西相李德輝為參知政事,兼領錢穀事。徙泉州行省於隆興。以禿古滅軍劫食火拙畏吾城禾,民饑,命官給驛馬之費,仍免其賦稅三年。太陰犯南斗。甲寅,發衛兵八百治沙嶺橋,敕毋踐民田。戊午,從阿合馬言,以參知政事郝禎、耿仁並為中書左丞;用姚演言,開膠東河及收集逃民屯田漣、海。甲子,遣安南國王子倪還。括蒙古軍成丁者。敕亦來等率萬人入羅氏鬼國,如其不附,則入討之。乙丑,罷江南財賦總管府。丁卯,並大都鹽運司入河間為一,仍減汰冗員。割建康民二萬戶種稻,歲輸釀米三萬石,官為運至京師。戊辰,詔括前願從軍者及張世傑潰軍,使征日本。命範文虎等招集避罪附宋蒙古、回回等軍。己巳,遣中使咬難歷江南名山訪求高士,且命持香幣詣信州龍虎山、臨江閣皁山、建康三茅山,皆設醮。賜阿赤黑等及怯薛都等戰功銀鈔,賜招收散毛等洞官吏衣段。



 八月庚午朔,蕭簡等十人歷河南五路,擅招闌遺戶,事覺,謫其為首者從軍自效,餘皆杖之。乙亥,改蒙古侍衛總管府為蒙古侍衛親軍都指揮使司。丙子,太陰犯心東星。丁丑,唆都請招三佛齊等八國,不從。鎮守南劍路萬戶呂宗海竊兵亡去,詔追捕之。戊寅,占城、馬八兒國皆遣使奉表稱臣,貢寶物犀象。以前所括願從軍者為軍,付茶忽領之征日本。丁亥,許衡致仕,官其子師可為懷孟路總管,以便侍養。納碧玉盞六、白玉盞十五於太廟。癸巳,賜西平王所部糧。戊戌,高麗王王睶來朝,且言將益兵三萬征日本。以範文虎、忻都、洪茶丘為中書右丞,李庭、張拔突為參知政事,並行中書省事。賜闊里吉思等鈔,迷裏兀合等羊馬,怯魯憐等牛羊馬價,及東宮位下怯憐口等粟帛。大都、北京、懷孟、保定、南京、許州、平陽旱,濮州、東平、濟寧、磁州水。



 九月壬子,車駕至自上都。壬戌,也罕的斤進徵斡端。癸亥,命沿途廩食和林回軍。甲子,太陰掩右執法,並犯歲星。乙丑,守庫軍盜庫鈔,八剌合赤分其贓,縱盜遁去,詔誅之。丁卯,羅氏鬼國主阿察及阿裏降,安西王相李德輝遣人偕入覲。賜八剌合赤等羊馬價二萬八千三錠,及禿渾下貧民糧三月。



 冬十月庚午,塔剌不罕軍與賊力戰者,命給田賞之。癸酉,加高麗國王王春開府儀同三司、中書左丞相、行中書省事。甲戌,遣使括開元等路軍三千征日本。丙子,賜雲南王忽哥赤印。丁丑,以湖南兵萬人伐亦奚不薛,亦奚不薛降。戊寅,發兵十萬,命範文虎將之。賜右丞洪茶丘所將征日本新附軍鈔及甲。辛巳,立營田提舉司,從五品,俾置司柳林,割諸色戶千三百五十五隸之,官給牛種農具。壬午,詔立陜西四川等處行中書省,以不花為右丞,李德輝、汪惟正並左丞。時德輝已卒。甲申,詔龍虎山天師張宗演赴闕。己丑,命都實窮黃河源。辛卯,以漢軍屯田沙、甘。壬辰,亦奚不薛病,遣其從子入覲。帝曰:「亦奚不薛不稟命,輒以職授其從子,無人臣禮。宜令亦奚不薛出,乃還軍。」癸巳,詔諭和州諸城招集流移之民。丙申,命在官者,任事一月,後月乃給俸,或廢事者斥之。遣使諭瓜哇國及交趾國。始制象轎。給怯烈等糧。賜火察家貧乏者。



 十一月己亥朔,翰林學士承旨和禮霍孫等言:「俱藍、馬八、闍婆、交趾等國俱遣使進表,乞答詔。」從之,仍賜交趾使人職名及弓矢鞍勒。降詔招諭瓜哇國。乙巳,置泉府司,掌領御位下及皇太子、皇太后、諸王出納金銀事。敕別置局院以處童匠,有貧乏者,給以鈔幣。詔:「有罪配役者,量其程遠近;犯罪當死者,詳加審讞。」戊申,中書省臣議:「流通鈔法,凡賞賜宜多給幣帛,課程宜多收鈔。」制曰:「可。」庚戌,命和禮霍孫揀汰交趾國使,除可留者,餘皆放還。辛亥,敕緩營建工役。壬子,詔諭俱藍國使來歸附。甲寅,太原路堅州進嘉禾六莖。壬戌,詔江淮行中書省招巧匠。甲子,詔頒《授時歷》。丁卯,詔以末甘孫民貧,除倉站稅課外,免其役三年。復遣宣慰使教化、孟慶元等持詔諭占城國主,令其子弟或大臣入朝。詔江南、江北、陜西、河間、山東諸鹽場增撥灶戶,賜將作院呂合剌工匠銀、鈔、幣帛。



 十二月庚午,以江淮行省平章政事阿里伯、左丞燕鐵木兒擅易命官八百員,自分左右司官,鑄銀、銅印,復違命不散防守軍,敕誅之。辛未,以熟券軍還襄陽屯田。高麗國王王睶領兵萬人、水手萬五千人、戰船九百艘、糧一十萬石,出征日本,給右丞洪茶丘等戰具、高麗國鎧甲戰襖。諭諸道征日本兵取道高麗,毋擾其民。以高麗中贊金方慶為征日本都元帥,密直司副使樸球、金周鼎為管高麗國征日本軍萬戶,並賜虎符。癸酉,以高麗國王王睶為中書右丞相。甲戌,復授徵日本軍官元佩虎符。丁丑,用忽辛言,以民當站役,十戶為率,官給一馬,死則買馬補之。戊寅,以奉使木剌由國速剌蠻等為招討使,佩金符。己卯,羅氏鬼國土寇為患,思、播道路不通,發兵千人與洞蠻開道。甲申,甘州增置站戶,詔於諸王戶籍內簽之。乙酉,敕民避役竄名匠戶者復為民。淮西宣慰使昂吉兒請以軍士屯田,阿塔海等以發民兵非便,宜募民願耕者耕之,且免其租三年,從之。丁亥,復詔管民官兼管諸軍奧魯。戊子,以徵也可不薛軍千五百復還塔海,戍八番、羅甸。壬辰,陳桂龍據漳州反,唆都率兵討之,桂龍亡入畬洞。甲午,大都重建太廟成,自舊廟奉遷神主於祏室,遂行大享之禮。置鎮北庭都護府於畏吾境,以脫脫木兒等領其事。丙申,遼東路所益兵以妻子易馬,敕以合輸賦稅贖還之。敕鏤板印造帝師八合思八新譯《戒本》五百部,頒降諸路僧人。左丞相阿術巡歷西邊,至別十八里以疾卒。敕擅據江南逃亡民田者有罪。修桐柏山淮瀆祠。以三茅山上清四十三代宗師許道杞祈禱有驗,命別主道教。安南國來貢馴象。賜蠻洞主銀鈔衣物有差。賑鞏昌、常德等路饑民,仍免其徭役。改拱衛司為都指揮司;升尚舍監秩三品;立太倉提舉司,秩五品。改建寧、雷州、封州、廉州、化州、高州為路,以肇慶路隸廣南西道,遷峽州路於江北舊治。復置鄣縣,隸鞏昌路。宿州靈壁縣復隸歸德。是歲,斷死罪一百二人。



 十八年春正月戊戌朔,高麗國王王睶遣其簽議中贊金方慶來賀,兼奉歲幣。辛丑,召阿剌罕、範文虎、囊加帶同赴闕受訓諭,以拔都、張珪、李庭留後。命忻都、洪茶丘軍陸行抵日本,兵甲則舟運之,所過州縣給其糧食。用範文虎言,益以漢軍萬人。文虎又請馬二千給禿失忽思軍及回回砲匠。帝曰:「戰船安用此?」皆不從。癸卯,發鈔及金銀付孛羅,以給貧民。丁未,畋於近郊。敕江南州郡兼用蒙古、回回人。凡諸王位下合設達魯花赤,並令赴闕,仍詔諭諸王阿只吉等知之。己酉,改黃州陽羅堡復隸鄂州。辛亥,遣使代祀岳瀆后土。壬子,高麗王王睶遣使言日本犯其邊境,乞兵追之,詔以戍金州隘口軍五百付之。丙辰,車駕幸漷州。改符寶局為典瑞監,收天下諸司職印。丁巳,制以六祖李全祐嗣五祖李居壽祭斗。癸亥,邵武民高日新據龍樓寨為亂,擒之。賞忻都等戰功,賜征日本諸軍鈔。



 二月戊辰,發侍衛軍四千完正殿。賜征日本善射軍及高麗火長水軍鈔四千錠。辛未,車駕幸柳林。高麗王王睶以尚主,乞改宣命益駙馬二字。制曰:「可。」乙亥,敕以耽羅新造船付洪茶丘出征。詔以刑徒減死者付忻都為軍。揚州火,發米七百八十三石賑被災之家。詔諭範文虎等以征日本之意,仍申嚴軍律。立上都留守司。升敘州為路,隸安西省。移潭州省治鄂州,徙湖南宣慰司於潭州。乙酉,改畏吾斷事官為北庭都護府,升從二品。丙戌,征日本國軍啟行。浙東饑,發粟千二百七十餘石賑之。己丑,發肅州等處軍民鑿渠溉田。給徵日本軍衣甲、弓矢、海青符。敕通政院官渾都與郭漢傑整治水驛,自敘州至荊南凡十九站,增戶二千一百、船二百十二艘。詔諭烏瑣納空等毋擾羅氏鬼國,違者令國主阿利具以名聞。福建省左丞蒲壽庚言:「詔造海船二百艘,今成者五十,民實艱苦。」詔止之。乙未,貞懿順聖昭天睿文光應皇後弘吉剌氏崩。丙申,車駕還宮。詔三茅山三十八代宗師蔣宗瑛赴闕。遣丹八八合赤等詣東海及濟源廟修佛事。以中書右丞、行江東道宣慰使阿剌罕為中書左丞相,行中書省事,江西道宣慰使兼招討使也的迷失參知政事,行中書省事。以遼陽、懿、蓋、北京、大定諸州旱,免今年租稅之半。



 三月戊戌,許衡卒。己亥,敕黃平棣安西行省,鎮遠隸潭州行省,各遣兵戍守。甲辰,命天師張宗演即宮中奏赤章於天七晝夜。丙午,車駕幸上都。丙辰,升軍器監為三品。辛酉,立登聞鼓院,許有冤者撾鼓以聞。



 夏四月辛未,益云南軍征合剌章。癸酉,復頒中外官吏俸。辛巳,通、泰二州饑,發粟二萬一千六百石賑之。戊子,置蒙古漢人新附軍總管。甲午,命太原五戶絲就輸太原。自太和嶺至別十八里置新驛三十。賜征日本河西軍等鈔。



 五月癸卯,禁西北邊回回諸人越境為商。甲辰,遣使賑瓜、沙州饑。戊申,罷霍州畏兀按察司。己酉,禁甘肅瓜、沙等州為酒。壬子,免耽羅國今歲入貢白苧。丙辰,以烏蒙阿謀宣撫司隸雲南行省。歲星犯右執法。庚申,嚴鬻人之禁,乏食者量加賑貸。壬戌,詔括契丹戶。敕耽羅國達魯花赤塔兒赤,禁高麗全羅等處田獵擾民者。



 六月丙寅,敕賽典赤、火尼赤分管烏木、拔都怯兒等八處民戶。謙州織工百四十二戶貧甚,以粟給之,其所鬻妻子官與贖還。以太原新附軍五千屯田甘州。丁丑,以按察司所劾羨餘糧四萬八千石餉軍。己卯,以順慶路隸四川東道宣慰司。安西等處軍站,凡和顧和買,與民均役。增陜西營田糧十萬石,以充常費。壬午,命耽羅戍力田以自給。日本行省臣遣使來言:「大軍駐巨濟島,至對馬島獲島人,言太宰府西六十里舊有戍軍已調出戰,宜乘虛搗之。」詔曰:「軍事卿等當自權衡之。」癸未,命中書省會計姚演所領漣、海屯田官給之資與歲入之數,便則行之,否則罷去。丁亥,放乞赤所招獵戶七千為民。庚寅,以阿剌罕有疾,詔阿塔海統率軍馬征日本。壬辰,高麗國王王睶言,本國置驛四十,民畜凋弊,敕並為二十站,仍給馬價八百錠。奉使木剌由國苫思丁至占城船壞,使人來言,乞給舟糧及益兵,詔給米一千四百餘石。以中書左丞忽都帖木兒為中書右丞,行中書省事;御史中丞、行御史臺事忽剌出為中書左丞,行尚書省事。賜皇子南木合所部工匠羊馬價鈔。



 秋七月甲午朔,命萬戶綦公直分宣慰使劉恩所將屯肅州漢兵千人,入別十八里,以嘗過西川兵百人為向導。丁酉,敕甘州置和中所,以給兵糧。京兆四川分置行省於河西。己亥,阿剌罕卒。庚子,括回回砲手散居他郡者,悉令赴南京屯田。癸卯,太陰犯房距星。庚戌,以忻都戍大和嶺所將蒙古軍還,復令漢軍戍守。以松州知州僕散禿哥前後射虎萬計,賜號萬虎將軍。賜貴赤合八兒禿所招和、真、滁等戶二千八百二十,俾自領之。辛酉,唆都征占城,賜駝蓬以闢瘴毒。占城國來貢象犀。命天師張宗演等即壽寧宮奏赤章於天,凡五晝夜。



 八月甲子朔,招討使方文言擇守令、崇祀典、戢奸吏、禁盜賊、治軍旅、獎忠義六事,詔廷臣及諸老議舉行之。丙寅,熒惑犯諸侯西第三星。庚午,忙古帶為中書右丞,行中書省事。辛未,敕隆興行省參政劉合拔兒禿,凡金穀造作專領之。乙亥,甘州凡諸投下戶,依民例應站役。申嚴大都總管府、兵馬司、左右巡院斂民之禁。庚寅,以阿剌罕既卒,命阿塔海等分戍三海口,令阿塔海就招海中餘寇。高麗國王王睶遣其密直司使韓康來賀聖誕節。壬辰,以開元等路六驛饑,命給幣帛萬二千匹,其鬻妻子者官為贖之。詔征日本軍回,所在官為給糧。忻都、洪茶丘、範文虎、李庭、金方慶諸軍,船為風濤所激,大失利,餘軍回至高麗境,十存一二。設醮於上都壽寧宮。賜歡只兀部及滅乞裡等羊馬價,及眾家奴等助軍羊馬鈔。賜常河部軍貧乏者,給過西川軍糧。海南諸國來貢象犀方物。給怯薛丹糧,拘其所占田為屯田。



 閏月癸巳朔,熒惑犯司怪南第二星。阿塔海乞以戍三海口軍擊福建賊陳吊眼,詔以重勞不從。敕守縉山道侍衛軍還京師。壬辰,瓜州屯田進瑞麥一莖五穗。丙午,車駕至自上都。庚戌,太陰犯昴。丁巳,命播州每歲親貢方物,改思州宣撫司為宣慰司,兼管內安撫使。升高麗簽議府為從三品。敕中書省減執政及諸司冗員。遣兀良合帶運沙城等糧六千石入和林。括江南戶口稅課。庚申,安南國貢方物。江西行省薦舉兵官,命罷之。壬戌,詔諭斡端等三城官民及忽都帶兒,括不闌奚人口。兩淮轉運使阿剌瓦丁坐盜官鈔二萬一千五百錠,盜取和買馬三百四十四匹,朝廷宣命格而弗頒,又以官員所佩符擅與家奴往來貿易等事,伏誅。賜謙州屯田軍人鈔幣、衣裘等物,及給農具漁具。償站匠等助軍羊馬價。



 九月癸亥朔,畋於近郊。甲子,增大都巡兵千人。給鈔賑上都饑民。癸酉,商賈市舶物貨已經泉州抽分者,諸處貿易,止令輸稅。益耽羅戍兵,仍命高麗國給戰具。庚辰,還宮。辛巳,大都立蒙古站屯田,編戶歲輸包銀者及真定等路闌遺戶,並令屯田,其在真定者與免皮貨。癸未,京兆等路歲辦課額,自一萬九千錠增至五萬四千錠,阿合馬尚以為未實,欲核之。帝曰:「阿合馬何知。」事遂止。大都、新安縣民復和顧和買。甲申,太陰犯軒轅大星。壬辰,占城國來貢方物。賜修大都城侍衛軍鈔幣帛有差,賞北征軍銀鈔。賜怯憐口及四斡耳朵下與範文虎所部將士羊馬、衣服、幣帛有差。



 冬十月乙未,享於太廟,貞懿順聖昭天睿文光應皇后祔。丙申,募民淮西屯田。己亥,議封安南王號,易所賜安南國畏吾字虎符,以國字書之;仍降詔諭安南國,立日烜之叔遺愛為安南國王。庚子,溪洞新附官鎮安州岑從毅,縱兵殺掠,迫死知州李顯祖,召從毅入覲。壬寅,賜征日本將校衣裝、幣帛、靴帽等物有差。乙巳,命安西王府協濟戶及南山隘口軍,於安西、延安、鳳翔、六盤等處屯田。河西置織毛段匠提舉司。丁未,安南國置宣慰司,以北京路達魯花赤孛顏帖木兒參知政事,行安南國宣慰使,都元帥、佩虎符柴椿、忽哥兒副之。給鈔萬錠,付河西行省以備經費。己酉,張易等言:「參校道書,惟《道德經》系老子親著,餘皆後人偽撰,宜悉焚毀。」從之,仍詔諭天下。給隆興行省海青符。命失里咱牙信合八剌麻合迭瓦為占城郡王,加榮祿大夫,賜虎符。立行中書省占城。以唆都為右丞,劉深為左丞,兵部侍郎也裏迷失參知政事。庚戌,敕以海船百艘,新舊軍及水手合萬人,期以明年正月征海外諸番,仍諭占城郡王給軍食。以安南國王陳遺愛入安南,發新附軍千人衛送。詔諭乾不昔國來歸附。壬子,用和禮霍孫言,於揚州、隆興、鄂州、泉州四省,置蒙古提舉學校官各二員。以翰林學士承旨撒里蠻兼領會同館、集賢院事,以平章政事、樞密副使張易兼領秘書監、太史院、司天臺事,以翰林學士承旨和禮霍孫守司徒。改大都南陽真定等處屯田孛蘭奚總管府為農政院。癸丑,皇太子至自北邊。丙辰,以兀良合帶言,上都南四站人畜困乏,賜鈔給之。庚申,籍西川戶。辛酉,邵武叛人高日新降。給徵日本回侍衛新附軍冬衣。賜劉天錫等銀幣,勝兀剌等羊馬鈔,諸王阿只吉等馬牛羊,各有差。



 十一月癸亥朔,詔諭探馬禮,令歸附。甲子,敕誅陳吊眼首惡者,餘並收其兵仗,系送京師。己巳,敕軍器監給兵仗付高麗沿海等郡。奉使占城孟慶元、孫勝夫並為廣州宣慰使,兼領出徵調度。高麗國、金州等處置鎮邊萬戶府,以控制日本。高日新及其弟鼎新等至闕,以日新兩為叛首,授山北路民職。文慶之屬,遣還泉州。賜有功將校二百二十三員銀十萬兩及幣帛、弓矢、鞍勒有差。詔安南國王給占城行省軍食。高麗國王請完濱海城,防日本,不允。辛未,給諸王阿只吉糧六千石。甲戌,太陰犯五車次南星。乙亥,召法師劉道真,問祠太乙法。丁丑,太陰犯鬼。壬午,詔諭瓜哇國主,使親來覲。昌州及蓋里泊民饑,給鈔賑之。丙戌,給鈔二萬錠付和林貿易。敕征日本回軍後至者分戍沿海。丁亥,太陰掩心東星。給揚州行省新附軍將校鈔,人二錠。己酉,賜安南國出征新附軍鈔。賜禮部尚書留夢炎及出使馬八國俺都剌等鈔各有差。



 十二月甲午,以甕吉剌帶為中書右丞相。己亥,罷日本行中書省。丙午,太陰犯軒轅大星。丁未,議選侍衛軍萬人練習,以備扈從。升太常寺為正三品。辛亥,命西川行省給萬家奴所部兵仗。癸丑,敕免益都、淄萊、寧海開河夫今年租賦,仍給其傭直。乙卯,以諸王札忽兒所占文安縣地給付屯田。丙辰,調新附軍屯田。獲福州叛賊林天成,戮於市。免福州路今年稅二分,十八年以前租稅並免征。以漢州德陽縣隸成都府。改漳州為府。賜禮部尚書謝昌元鈔。賞捏古伯戰功銀有差。償阿只吉等助軍馬價。賜塔剌海籍沒戶五十。是歲,保定路清苑縣水,平陽路松山縣旱,高唐、夏津、武城等縣蟊害稼,並免今年租,計三萬六千八百四十石。斷死罪二十二人。



\end{pinyinscope}