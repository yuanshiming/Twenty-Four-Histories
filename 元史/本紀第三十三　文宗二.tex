\article{本紀第三十三 文宗二}

\begin{pinyinscope}

 天歷二年春正月己未朔,立都督府,以總左、右欽察及龍翊衛。庚申,封知樞密院事火沙為昭武王。床兀兒之子答鄰答里襲父封為句容郡王。高麗國遣使來朝賀。遣前翰林學士承旨不答失里北還皇兄行在所,仍命太府太監沙剌班奉金、幣以往。辛酉,封朵列帖木兒復為楚王。高昌王鐵木兒補化為中書左丞相,大司農王毅為平章政事,欽察臺知樞密院事。皇兄遣火里忽達孫、剌剌至京師。以伯帖木兒扈從有功,遣使以幣帛百匹即行在賜之。諸王渾都帖木兒薨,取其印及王傅印以賜斡即。武寧王徹徹禿遣使來言皇兄啟行之期。癸亥,燕鐵木兒為御史大夫,太平王如故。賜魯國大長公主鈔二萬錠營第宅。甲子,太白犯壘壁陣。時享於太廟。齊王月魯帖木兒薨。乙丑,中書省言:「度支今歲芻槁不足,常例支給外,凡陳乞者,宜勿予。」從之,仍命中書右丞徹里帖木兒總其事。丙寅,帝幸大崇恩福元寺。遣使賜西域諸王燕只吉臺海東鶻二。戊辰,遣使獻海東鶻於皇兄行在所。己巳,賜內外軍士四萬二千二百七十人鈔各一錠。作佛事。陜西告饑,賑以鈔五萬錠。辛未,以冊命皇后,告於南郊。賜豫王黃金印。回回人戶與民均當差役。中書省臣言:「近籍沒欽察家,其子年十六,請令與其母同居;仍請繼今臣僚有罪致籍沒者,其妻有子,他人不得陳乞,亦不得沒為官口。」從之。壬申,遣近侍星吉班以詔往四川招諭囊加臺。癸酉,命中書省、宣徽院臣稽考近侍、宿衛廩給,定其名籍。以遼陽省蒙古、高麗、肇州三萬戶將校從逆,舉兵犯京畿,拘其符印制敕。罷今歲柳林田狩。復鹽制每四百斤為引,引為鈔三錠。四川囊加臺乞師於鎮西武靖王搠思班,搠思班以兵守關隘。甲戌,復命太僕卿教化獻海東鶻於皇兄行在所。罷中瑞司。丙子,皇后媵臣張住童等七人授集賢侍講學士等官。丁丑,四川囊加臺攻破播州貓兒埡隘,宣慰使楊延里不花開關納之。陜西蒙古軍都元帥不花臺者,囊加臺之弟,囊加臺遣使招之,不花臺不從,斬其使。中書省臣言:「朝廷賞賚,不宜濫及罔功。鷹、鶻、獅、豹之食,舊支肉價二百餘錠,今增至萬三千八百錠;控鶴舊止六百二十八戶,今增二千四百戶。又,佛事歲費,以今較舊,增多金千一百五十兩、銀六千二百兩、鈔五萬六千二百錠、幣帛三萬四千餘匹;請悉揀汰。」從之。中政院臣言,皇后日用所需,鈔十萬錠、幣五萬匹、綿五千斤。詔鈔予所需之半,幣給一萬匹。賑大都路涿州房山、範陽等縣饑民糧兩月。己卯,以冊命皇后,告於太廟。庚辰,賜潛邸說書劉道衡等四人官從七品,薛允等十六人官從八品。辛巳,起復中書左丞史惟良為御史中丞。上都官吏,惟初入仕及驟升者黜之,余聽敘復。以御史臺贓罰鈔三百錠賜教坊司撒剌兒。壬午,以陜西行臺御史大夫阿不海牙為中書平章政事。皇兄遣常侍孛羅及鐵住訖先至京師,賞以金、幣、居宅,仍遣內侍禿教化如皇兄行在所。播州楊萬戶引四川賊兵至烏江峰,官軍敗之;八番元帥脫出亦破烏江北岸賊兵,復奪關口。諸王月魯帖木兒統蒙古、漢人、答剌罕諸軍及民丁五萬五千,俱至烏江。癸未,遣宣靖王買奴往行在所。丙戌,皇兄明宗即皇帝位於和寧之北。四川囊加臺焚雞武關大橋,又焚棧道。命中書省錄江陵、汴梁郡縣官扈從者三十四人,並升其階秩。陜西大饑,行省乞糧三十萬石、鈔三十萬錠,詔賜鈔十四萬錠,遣使往給之。大同路言,去年旱且遭兵,民多流殍,命以本路及東勝州糧萬三千石,減時直十之三賑糶之。奉元蒲城縣民王顯政五世同居,衛輝安寅妻陳氏、河間王成妻劉氏、冀寧李孝仁妻寇氏、濮州王義妻雷氏、南陽卻二妻張氏、懷慶阿魯輝妻翟氏皆以貞節,並旌其門。



 二月己丑,曲赦四川囊加臺。庚寅,燕鐵木兒復為中書右丞相。立繕工司,掌織御用紋綺,秩正三品。辛卯,帝御大明殿,冊命皇后雍吉剌氏。廣西思明路軍民總管黃克順來貢方物。壬辰,囊加臺據雞武關,奪三義、柴關等驛。癸巳,遣翰林侍講學士曹元用祀孔子於闕里。囊加臺以書誘鞏昌總帥汪延昌。丙申,命中書省、翰林國史院官祀太祖、太宗、睿宗御容於普慶寺。丁酉,遣晉邸部曲之在京師者還所部。囊加臺以兵至金州,據白土關,陜西行省督軍御之。樞密院言:「囊加臺阻兵四川,其亂未已,請命鎮西武靖王搠思班等皆調軍,以湖廣行省官脫歡、別薛、孛羅及鄭昂霄總其兵進討。」從之。戊戌,命察罕腦兒宣慰使撒忒迷失將本部蒙古軍,會鎮西武靖王等討四川。諸傭雇者,主家或犯惡逆及侵損己身,許訴官,餘非干己,不許告訐,著為制。頒行《農桑輯要》及《栽桑圖》。辛丑,中書省議追尊皇妣亦乞烈氏曰仁獻章聖皇后,唐兀氏曰文獻昭聖皇后,命有司具冊寶。建游皇城佛事。雲南行省蒙通蒙算甸土官阿三木,開南土官哀放,八百媳婦、金齒、九十九洞、銀沙羅甸,咸來貢方物。癸卯,賜吳王木楠子、西寧王忽答的迷失、諸王那海罕、闊兒吉思金銀有差。丙午,囊加臺分兵逼襄陽,湖廣行省調兵鎮播州及歸州。己酉,熒惑犯井宿。辛亥,帝謂廷臣曰:「撒迪還,言大兄已即皇帝位。凡二月二十一日以前除官者,速與制敕;後凡銓選,其詣行在以聞。」廬州路合肥縣地震。壬子,命有司造行在帳殿。癸丑,諸王月魯帖木兒等至播州,招諭土官之從囊加臺者,楊延里不花及其弟等皆來降。甲寅,立奎章閣學士院,秩正三品,以翰林學士承旨忽都魯都兒迷失、集賢大學士趙世延並為大學士,侍御史撒迪、翰林直學士虞集並為侍書學士,又置承制、供奉各一員。更鑄鈔版,仍毀其刓者。調河南、江浙、江西、山東兵萬一千,及左右翼蒙古侍衛軍二千,討四川。乙卯,置銀沙羅甸等處宣慰司都元帥府。丙辰,奉元臨潼、咸陽二縣及畏兀兒八百餘戶告饑,陜西行省以便宜發鈔萬三千錠賑咸陽,麥五千四百石賑臨潼,麥百餘石賑畏兀兒,遣使以聞,從之。永平、大同二路,上都雲需兩府,貴赤衛,皆告饑,永平賑糧五萬石,大同賑糶糧萬三千石,雲需府賑糧一月,貴赤衛賑糧二月。真定平山縣、河間臨津等縣、大名魏縣,有蟲食桑,葉盡,蟲俱死。



 三月辛酉,遣燕鐵木兒奉皇帝寶於明宗行在所,仍命知樞密院事禿兒哈帖木兒、御史中丞八即剌,翰林直學士馬哈某、典瑞使教化的、宣徽副使章吉、僉中政院事脫因、通政使那海、太醫使呂廷玉、給事中咬驢、中書斷事官忽兒忽答、右司郎中孛別出、左司員外郎王德明、禮部尚書八剌哈赤等從行。復命有司奉金千五百兩、銀七千五百兩、幣帛各四百匹及金腰帶二十,詣行在所,以備賜予。帝命廷臣曰:「寶璽既北上,繼今國家政事,其遣人聞於行在所。」癸亥,命有司造乘輿服御,北迎大駕。改潛邸所幸諸路名:建康曰集慶,江陵曰中興,瓊州曰乾寧,潭州曰天臨。甲子,減大官羊直。丙寅,躍裏帖木兒自行在還,諭旨曰:「朕在上都,宗王、大臣必皆會集,有司當備供張。上都積貯,已為倒剌沙所耗,大都府藏,聞亦悉虛。供億如有不足,其以御史臺、司農司、樞密、宣徽、宣政等院所貯充之。」蒙古饑民之聚京師者,遣往居庸關北,人給鈔一錠、布一匹,仍令興和路賑糧兩月,還所部。戊辰,雲南諸王答失不花、禿堅不花及平章馬思忽等集眾五萬,數丞相也兒吉尼專擅十罪,將殺之。也兒吉尼遁走八番,答失不花等偽署參知政事等官。己巳,命改集慶潛邸,建大龍翔集慶寺,以來歲興工。辛未,監察御史與扎魯忽赤等官錄囚。壬申,以去冬無雪,今春不雨,命中書及百司官分禱山川群祀。設奎章閣授經郎二員,職正七品,以勛舊、貴戚子孫及近侍年幼者肄業。甲戌,舊賜篤麟帖木兒平江田百頃,官嘗收其租米,詔特予之。開遼陽酒禁。乙亥,置行樞密院,以山東都萬戶也速臺兒知行樞密院事,與湖廣、河南兩省官進兵平四川,也速臺兒以病不往。命明裏董阿為蒙古巫覡立祠。丁丑,文獻昭聖皇后神御殿月祭,特命如列聖故事。僧、道、也裏可溫、術忽、合失蠻為商者,仍舊制納稅。丙戌,囊加臺所遣守隘碉門安撫使布答思監等降於雲南行省。丁亥,雨土,霾。



 夏四月己丑,時享於太廟。辛卯,命躍里鐵木兒、王不憐吉臺代也速臺兒討四川,不憐吉臺以母老辭,同僉樞密院事傅巖起請往,從之。壬辰,匠官年七十者,許致仕。浚漷州漕運河。甲午,四番衛士各分五十人直東宮。丁酉,給鈔萬錠,為集慶大龍翔寺置永業。戊戌,以陜西久旱,遣使禱西嶽、西鎮諸祠。賜衛士萬三千人鈔,人八十錠。四番衛士舊以萬人為率,至是增三千人。己亥,湖廣行省參知政事孛羅奉詔至四川,赦囊加臺等罪,囊加臺等聽詔,蜀地悉定,諸省兵皆罷。癸卯,明宗遣武寧王徹徹禿、中書平章政事哈八兒禿來錫命,立帝為皇太子,命仍置詹事院,罷儲慶司。陜西諸路饑民百二十三萬四千餘口,諸縣流民又數十萬,先是嘗賑之,不足;行省復請令商賈入粟中鹽,富家納粟補官,及發孟津倉糧八萬石及河南、漢中廉訪司所貯官租以賑,從之。德安府屯田饑,賑糧千石。常德、澧州、慈利州饑,賑糶糧萬石。賑衛輝路饑民萬七千五百餘戶。丙午,封孛羅不花為鎮南王。占臘國來貢羅香木及象、豹、白猿。戒翰林、典瑞兩院官,不許互相奏請璽書以護其家。諸王分邑達魯花赤受代,不得仍留官所,其父兄所居官,子弟不得再任。辛亥,賑鄧州諸縣被兵逃戶糧三千六百石。壬子,賑通州諸縣被兵之民糧兩月,被俘者四千五百一十人,命遼陽行省督所屬簿錄,護送歸其家。丙辰,行在所遣只兒哈郎等至京師。河南廉訪司言:「河南府路以兵、旱民饑,食人肉事覺者五十一人,餓死者千九百五十人,饑者一萬七千四百餘人。乞弛山林川澤之禁,聽民採食,行入粟補官之令,及括江淮僧道餘糧以賑。」從之。江浙行省言:「池州、廣德、寧國、太平、建康、鎮江、常州、湖州、慶元諸路及江陰州饑民六十餘萬戶,當賑糧十四萬三千餘石。」從之。諸王忽剌答兒言黃河以西所部旱蝗,凡千五百戶,命賑糧兩月。大都、興和、順德、大名、彰德、懷慶、衛輝、汴梁、中興諸路,泰安、高唐、曹、冠、徐、邳諸州,饑民六十七萬六千餘戶,賑以鈔九萬錠、糧萬五千石。大都宛平縣,保定遂州、易州,賑糧一月,靖州賑糶糧九千八百石。濮州鄄城縣蠶災,大寧興中州、懷慶孟州、廬州無為州蝗。廣西獠寇古縣。



 五月丁巳朔,復賜魯國大長公主鈔二萬錠,以構居第。賜燕鐵木兒祖父紀功碑銘。水達達路阿速古兒千戶所大水。己未,遣翰林學士承旨阿鄰帖木兒北迎大駕。命司天監翽星。昌王八剌失裡還鎮。庚申,太白犯鬼宿積尸氣。癸亥,復遣翰林學士承旨斡耳朵迎大駕。乙丑,命有司給行在宿衛士衣糧及馬芻豆。以儲慶司所貯金三十錠、銀百錠,建大承天護聖寺。給皇子宿衛之士千人鈔,四番宿衛增為萬三千人,至是又增千人。甲戌,命中書省臣擬注中書六部官,奏於行在所。乙亥,幸大聖壽萬安寺,作佛事於世祖神御殿,又於玉德殿及大天源延聖寺作佛事。丙子,武寧王徹徹禿、中書平章政事哈八兒禿至自行在所,致立皇太子之命。賜徹徹禿金五百兩,餘有差。改儲慶使司為詹事院。伯顏、鐵木兒補化及江南行臺御史大夫阿兒思蘭海牙、江浙行省平章政事曹立,並為太子詹事;又除副詹事、詹事丞及斷事官、家令司、典寶、典用、典醫等官。丁丑,帝發京師,北迎明宗皇帝。戊寅,次於大口。徵諸王鼎八入朝。庚辰,次香水園。置江淮財賦都總管府,秩正三品,隸詹事院。陜西行省言:「鳳翔府饑民十九萬七千九百人,本省用便宜賑以官鈔萬五千錠。又,豐樂八屯軍士饑,死者六百五十人,萬戶府軍士饑者千三百人,賑以官鈔百三十錠。」從之。給保定路定興驛車馬,又賑被兵之民百四十五戶糧一月,真定路民被兵者二千七百四十八戶,亦命賑之。上都迭只諸位宿衛士及開平縣民被兵者,並賑以糧。大名路蠶災。



 六月丁亥朔,明宗遣近侍馬駒、塔臺、別不花至。丁酉,鐵木兒補化以旱乞避宰相位,有旨諭之曰:「皇帝遠居沙漠,未能即至京師,是以勉攝大位。今亢陽為災,皆予闕失所致。汝其勉修厥職,祗修實政,可以上答天變。」仍命馳奏於行在。己亥,江浙行省言,紹興、慶元、臺州、婺州諸路饑民凡十一萬八千九十戶。乙巳,命中書省逮系也先捏以還。丙午,永平屯田府所隸昌國諸屯大風驟雨,平地出水。丁未,太白晝見。庚戌,次於上都之六十里店。辛亥,陜西行臺御史孔思迪言:「人倫之中,夫婦為重。比見內外大臣得罪就刑者,其妻妾即斷付他人,似與國朝旌表貞節之旨不侔、夫亡終制之令相反。況以失節之婦配有功之人,又與前賢所謂『娶失節者以配身是己失節』之意不同。今後凡負國之臣籍沒奴婢財產,不必罪其妻子。當典刑者,則孥戮之,不必斷付他人,庶使婦人均得守節。請著為令。」壬子,海運糧至京師,凡百四十萬九千一百三十石。是月,陜西雨。賜鳳翔府岐陽書院額。書院祀周文憲王,仍命設學官,春秋釋奠,如孔子廟儀。明宗遣吏部尚書別兒怯不花還京師。命中書集老臣議賑荒之策。時陜西、河東、燕南、河北、河南諸路流民十數萬,自嵩、汝至淮南,死亡相藉,命所在州縣官以便宜賑之。順元、思、播州諸驛,因兵興,馬多羸斃,驛戶貧乏,令有司市馬補之。益都莒、密二州春水,夏旱蝗,饑民三萬一千四百戶,賑糧一月。陜西延安諸屯,以旱免征舊所逋糧千九百七十石。永平屯田府昌國、濟民、豐贍諸署,以蝗及水災,免今年租。汴梁蝗,衛輝蠶災,峽州旱,淮東諸路、歸德府徐、邳二州大水。



 秋七月丙辰朔,日有食之。丁巳,次上都之三十里店。宗仁衛屯田大水,壞田二百六十頃。戊午,大都之東安、薊州、永清、益津、潞縣,春夏旱,麥苗枯;六月壬子雨,至是日乃止,皆水災。己未,更定遷徙法:凡應徙者,驗所居遠近,移之千里,在道遇赦,皆得放還;如不悛再犯,徙之本省不毛之地,十年無過,則量移之;所遷人死,妻子聽歸土著。著為令。征京師僧道商稅。癸亥,太白經天。丙子,帝受皇太子寶。辛巳,發諸衛軍六千完京城。冀寧陽曲縣雨雹,大者如雞卵。令諸王封邑達魯花赤,推擇本部年二十五以上、識達治體、廉慎無過者以充,或有冒濫,罪及王傅。遣使以上尊、臘羊、鈔十錠至大都國子監,助仲秋上丁釋奠。以淮安海寧州、鹽城、山陽諸縣去年水,免今年田租。真定、河間、汴梁、永平、淮安、大寧、廬州諸屬縣及遼陽之蓋州蝗。



 八月乙酉朔,明宗次於王忽察都。丙戊,帝入見,明宗宴帝及諸王、大臣於行殿。庚寅,明宗崩,帝入臨哭盡哀。燕鐵木兒以明宗後之命,奉皇帝寶授於帝,遂還。壬辰,次孛羅察罕,以伯顏為中書左丞相,依前太保;欽察臺、阿兒思蘭海牙、趙世延並中書平章政事;甘肅行省平章朵兒只為中書右丞;中書參議阿榮、太子詹事丞趙世安並中書參知政事;前右丞相塔失鐵木兒、知樞密院事鐵木兒補化及上都留守鐵木兒脫並為御史大夫。癸巳,帝至上都。乙未,賜護守大行皇帝山陵官、御史大夫孛羅等鈔有差。焚四川偽造鹽、茶引。丙申,監察御史徐奭言:「天下不可一日無君,神器不可一時而曠。先皇帝奄棄臣庶已逾數日,伏望聖上早正宸極,以安億兆之心,實宗社無疆之福。」流諸王忽剌出於海南。丁酉,命阿榮、趙世安提調通政院事,一切給驛事皆關白然後給遣。戊戌,四川囊加臺以指斥乘輿,坐大不道棄市。己亥,帝復即位於上都大安閣,大赦天下,詔曰:



 朕惟昔上天啟我太祖皇帝肇造帝業,列聖相承。世祖皇帝既大一統,即建儲貳,而我裕皇天不假年,成宗入繼,才十餘載。我皇考武宗歸膺大寶,克享天心,志存不私,以仁廟居東宮,遂嗣宸極。甫及英皇,降割我家。晉邸違盟構逆,據有神器,天示譴告,竟隕厥身。於是宗戚舊臣,協謀以舉義,正名以討罪,揆諸統緒,屬在眇躬。朕興念大兄播遷朔漠,以賢以長,歷數宜歸,力拒群言,至於再四。乃曰艱難之際,天位久虛,則眾志弗固,恐隳大業。朕雖從請而臨御,秉初志之不移,是以固讓之詔始頒,奉迎之使已遣。尋命阿剌忒納失里、燕鐵木兒奉皇帝寶璽,遠迓於途。受寶即位之日,即遣使授朕皇太子寶。朕幸釋重負,實獲素心,乃率臣民北迎大駕。而先皇帝跋涉山川,蒙犯霜露,道里遼遠,自春徂秋,懷艱阻於歷年,望都邑而增慨,徒御弗慎,屢爽節宣。信使往來,相望於道路,彼此思見,交切於衷懷。八月一日,大駕次王忽察都,朕欣瞻對之有期,獨兼程而先進,相見之頃,悲喜交集。何數日之間,而宮車弗駕,國家多難,遽至於斯!念之痛心,以夜繼旦。諸王、大臣以為祖宗基業之隆,先帝付托之重,天命所在,誠不可違,請即正位,以安九有。朕以先皇帝奄棄方新,摧怛何忍;銜哀辭對,固請彌堅,執誼伏闕者三日,皆宗社大計,乃以八月十五日即皇帝位於上都,可大赦天下,自天歷二年八月十五日昧爽以前,罪無輕重,咸赦除之。於戲!戡定之餘,莫急乎與民休息;丕變之道,莫大乎使民知義。亦惟爾中外大小之臣,各究乃心,以稱朕意。



 庚子,命阿榮、趙世安督造建康龍翔集慶寺。辛丑,立寧徽寺,掌明宗宮分事。壬寅,以鈔萬錠、幣帛二千匹,供明宗後八不沙費用。升奎章閣學士院秩正二品,更司籍郎為群玉署,秩正六品。癸卯,幸世祖所御幄殿祓祭。禁凡送諸王、駙馬恩賜者,毋受金幣,犯者以贓論;或以衣、馬為贈者聽。遣道士苗道一、吳全節修醮事於京師,毛穎遠祭遁甲神於上都南屏山、大都西山。甲辰,命司天監及回回司天監翽星。中書省臣言:「祖宗故事,即位之初,必恩賚諸王、百官。比因兵興,經費不足,請如武宗之制,凡金銀五錠以上減三之一,五錠以下全畀之,又以七分為率,其二分準時直給鈔。」制可。遣欽察臺先還京師,經理政務;燕鐵木兒、阿榮留上都,監給恩賚金幣。以仁宗、英宗潛邸宿衛士二百人還大都,備直宿。乙巳,立藝文監,秩從三品,隸奎章閣學士院;又立藝林庫、廣成局,皆隸藝文監。賜御史中丞史惟良沛縣地五十頃。發諸衛軍浚通惠河。丙子,自庚子至是日,晝霧夜晴。封牙納失里為遼王,以故遼王脫脫印賜之。出官米五萬石,賑糶京師貧民。丁未,以馬扎兒臺為上都留守。馬扎兒臺前為陜西行臺侍御史,坐塗毀詔書得罪,以其兄伯顏有功,故特官之。戊申,封諸王寬徹為肅王。己酉,車駕發上都。賜明宗北來衛士千八百三十人各鈔五十錠,怯薛官十二人各鈔二百錠;賜諸部曲出征者幣帛人各二匹,遣還。冀寧之忻州兵後薦饑,賑鈔千錠。庚戌,改詹事院為儲政院,伯顏兼儲政使,中政使哈撒兒不花、太子詹事丞霄雲世月思、前儲慶使姚煒並儲政使。河東宣慰使哈散托朝賀為名,斂所屬鈔千錠入己,事覺,雖會赦,仍徵鈔還其主。敕自今有以朝賀斂鈔者,依枉法論罪。癸丑,征吳王潑皮及其諸父木楠子赴京師。甲寅,置隆祥總管府,秩正三品,總建大承天護聖寺工役。監察御史劾:「前丞相別不花昔以贓罷,天歷初因人成功,遂居相位。既矯制以買驢家貲賜平章速速,又與速速等潛呼日者推測聖算。今奉詔已釋其罪,宜竄諸海島,以杜奸萌。」帝曰:「流竄海島,朕所不忍,其並妻子置之集慶。」河南府路旱、疫,又被兵,賑以本府屯田租及安豐務遞運糧三月。莒、密、沂諸州,饑民採草木實,盜賊日滋,賑以米二萬一千石,並賑晉寧路饑民鈔萬錠。大名、真定、河間諸屬縣及湖、池、饒諸路旱,保定之行唐縣蝗。加封大都城隍神為護國保寧王,夫人為護國保寧王妃。



 九月乙卯朔,作佛事於大明殿、興聖、隆福諸宮。市故宋太后全氏田為大承天護聖寺永業。戊午,賜武寧王徹徹禿金百兩、銀五百兩,西域諸王燕只吉臺金二千五百兩、銀萬五千兩,鈔幣有差。己未,立龍翔、萬壽營繕提點所、海南營繕提點所,並秩正四品,隸隆祥總管府。庚申,加封故領諸路道教事張留孫為上卿、大宗師、輔成贊化保運神德真君。辛酉,凡往明宗所送寶官吏,越次超升者皆從黜降。賑甘肅行省沙州、察八等驛鈔各千五百錠。癸亥,敕宣徽院所儲金、銀、鈔、幣,百司毋得奏請。甲子,賜雲南烏撒土官祿餘、曲靖土官舉精衣各一襲。丁卯,大駕至大都。戊辰,敕翰林國史院官同奎章閣學士採輯本朝典故,準《唐》、《宋會要》,著為《經世大典》。召威順王寬徹不花赴闕。敕:「使者頒詔赦,率日行三百餘里。既受命,逗留三日及所至飲宴稽期者治罪,取賂者以枉法論。」辛未,以控鶴士二十人賜宣靖王買奴。監察御史劾奏:「知樞密院事塔失帖木兒阿附倒剌沙,又與王禪舉兵犯闕。今既待以不死,而又付之兵柄,事非便。」詔罷之。壬申,怯薛官武備卿定住特授開府儀同三司。癸酉,帝御大明殿,受諸王、百官朝賀。鐵木迭兒諸子鎖住等,明宗嘗敕流於南方,燕鐵木兒言,鎖住天歷初有勞於國,請各遣還田里,從之。甲戌,命江浙行省明年漕運糧二百八十萬石赴京師。廣西思明州土官黃宗永遣其子來貢虎、豹、方物。乙亥,史惟良上疏言:「今天下郡邑被災者眾,國家經費若此之繁,帑藏空虛,生民凋瘵,此政更新百廢之時。宜遵世祖成憲,汰冗濫蠶食之人,罷土木不急之役,事有不便者,咸厘正之。如此則天災可弭,禎祥可致。不然,將恐因循茍且,其弊漸深,治亂之由,自此而分矣。」帝嘉納之。丙子,改太禧院為太禧宗禋院。立溫州路竹木場。以衛輝路旱,罷蘇門歲輸米二千石。鐵木兒補化加錄軍國重事。以翰林學士承旨也兒吉尼、元帥梁國公都列捏並知行樞密院事。立衛候司,秩正四品,隸儲政院。賑陜西臨潼等二十三驛各鈔五百錠。論也先捏以不忠不敬,伏誅。嵐、管、臨三州所居諸王八剌馬、忽都火者等部曲,乘亂為寇,遣省、臺、宗正府官往督有司捕治之。壬午,伯顏以病在告,居赤城,遣使召赴闕。封知樞密院事燕不鄰為興國公,以大司農卿燕赤為司徒。癸未,建顏子廟於曲阜所居陋巷。上都西按塔罕、闊干忽剌禿之地,以兵、旱,民告饑,賑糧一月。



 冬十月甲申朔,帝服袞冕,享於太廟。丙戌,命欽察臺兼領度支監,遣鎮南王孛羅不花還鎮楊州。禁奉元、永平釀酒。戊子,知樞密院事昭武王火沙知行樞密院事。己丑,立大承天護聖寺營繕提點所,秩正五品;又立大都等處、平江等處田賦提舉司二,秩從五品;皆隸隆祥總管府。辛卯,燕鐵木兒率群臣請上尊號,不許。雲南行省立元江等處宣慰司。申飭海道轉漕之禁。籍四川囊加臺家產,其黨楊靜等皆奪爵,杖一百七,籍其家,流遼東。封太禧宗禋使禿堅帖木兒為梁國公。甲午,以登極恭謝,遣官代祀於南郊、社稷。中書省臣言:「舊制,朝官以三十月為一考,外任則三年為滿。比年朝官率不久於其職,或數月即改遷,於典制不類,且治跡無從考驗。請如舊制為宜。」敕:「除風憲官外,其餘朝官,不許二十月內遷調。」監察御史劾奏:「吏部尚書八剌哈赤,先除陜西行臺侍御史,避難不行。」罷之。丙申,中書省臣言:「臣等謹集樞密院、御史臺、翰林、集賢院、奎章閣、太常禮儀院、禮部諸臣僚,議上大行皇帝尊謚曰翼獻景孝皇帝,廟號明宗,國言謚號曰護都篤皇帝。」是日,奉玉冊、玉寶於太廟,如常儀。命江西、湖廣分漕米四十萬石,以紓江浙民力;給鈔十五萬錠,賑陜西饑民。己亥,加封天妃為護國庇民廣濟福惠明著天妃,賜廟額曰靈慈,遣使致祭。申飭都水監河防之禁。辛丑,遣使括勘內外郡邑官久次事故應代者,歲終上名於中書省。以怯憐口諸色民匠總管府及所屬諸司隸徽政院者,悉隸儲政院。發中政院財賦總管府糧儲在江南者赴京師,以助經費,驗時直以鈔還之。諸王、公主、官府、寺觀撥賜田租,除魯國大長公主聽遣人徵收外,其餘悉輸於官,給鈔酬其直。壬寅,弛陜西山澤之禁以與民。大寧路地震。癸卯,命道士苗道一建醮於長春宮。改瓊州軍民安撫司為乾寧軍民安撫司,升定安縣為南建州,隸海北元帥府,以南建洞主王官知州事,佩金符,領軍民。監察御史劾奏:「張思明在仁宗朝,阿附權臣鐵木迭兒,間諜兩宮,仁宗灼見其奸,既行黜降。及英宗朝鐵木迭兒再相,復援為左丞,稔惡不悛,竟以罪廢。今又冒居是官,宜從黜罷。」詔罷之。敕刑部尚書察民之無賴者懲治之。甲辰,畏兀僧百八人作佛事於興聖殿。戊申,以江淮財賦都總管府隸儲政院,供皇后湯沐之用。作佛事於廣寒殿。征朵朵、王士熙等十二人於貶所,放還鄉里。庚戌,以親祀太廟禮成,詔天下。罷大承天護聖寺工役。囚在獄三年疑不能決者,釋之。民間拖欠官錢無可追徵者,盡行蠲免。命通政院官分職往所在官司,僉補逃亡驛戶。大都至上都並塔思哈剌、旭麥怯諸驛,自備首思,供給繁重,天歷三年官為應付。免征奉元路民間商稅一年,命所在官司設置常平倉。雲南八番為囊加臺所詿誤,反側未安者,並貰其罪。免各處煎鹽灶戶雜泛夫役二年。遣使代祀岳瀆山川。免永平屯田總管府田租,申禁天下私殺馬牛。明宗乳媼夫斡耳朵,在武宗時為大司徒,仁宗朝拘其印。燕鐵木兒以為言,詔給還之。雲南威楚路黃州土官哀放遣其子來朝貢。湖廣常德、武昌、澧州諸路旱饑,出官粟賑糶之。陜西鳳翔府饑民四萬七千戶,皆賑以鈔。



 十一月乙卯,以立皇后,詔天下。受佛戒於帝師,作佛事六十日。丙辰,以句容郡王答鄰答里知行樞密院事。詔列聖諸宮後妃陪從之臣,永給衣廩芻粟。後八不沙請為明宗資冥福,命帝師率群僧作佛事七日於大天源延聖寺,道士建醮於玉虛、天寶、太乙、萬壽四宮及武當、龍虎二山。戊午,遣使代祀天妃。賜燕鐵木兒宅一區。皇后以銀五萬兩,助建大承天護聖寺。冠州旱。命朵耳只亦都護為河南行省丞相。近制行省不設丞相,中書省以為言,帝有旨:「朵耳只先朝舊臣,不當以例拘。」武宗宿衛士歲賜,如仁宗衛士例。西夏僧總統封國公沖卜卒,其弟監藏班臧卜襲職,仍以璽書、印章與之。癸亥,以翰林學士承旨闊徹伯知樞密院事,位居眾知院事上。甲子,廬州旱饑,發糧五千石賑之。止鷹坊毋獵畿甸。江西龍興、南康、撫、瑞、袁、吉諸路旱。丙寅,升山東河北蒙古軍大都督府秩從二品。改普慶修寺人匠提舉司為營繕提點所,秩從五品,隸崇祥總管府。雲南威楚路土官暱放來朝貢。罷功德使司,以所掌事歸宣政院。己巳,撒迪為中書右丞。命中書左丞趙世安提調國子監學。庚午,諸王闊不花至自陜西,收其印,遣還。壬申,毀廣平王木剌忽印,命哈班代之,更鑄印以賜。癸酉,太陰犯填星。丙子,諸王阿剌忒納失里翊戴有勞,以其父越王禿剌印與之。丁丑,復立孟定路軍民總管府,復給元江路軍民總管府印。湖廣州縣為廣源等徭寇掠者二百八十餘所,命行省平章劉脫歡招捕之。造青木綿衣萬領,賜圍宿軍。乙卯,翰林國史院臣言:「纂修《英宗實錄》,請具倒剌沙款伏付史館。」從之。高麗國王王燾久病,不能朝,請命其子楨襲位。以平江官田百五十頃,賜大龍翔集慶寺及大崇禧萬壽寺。辛巳,遷山東河北蒙古軍大都督府於濮州,仍聽山東廉訪司按治。欽察臺兼右都威衛使。壬午,詔豫王阿剌忒納失里鎮雲南,賜其衛士鈔萬錠,仍每歲豫給其衣廩。



 十二月甲申,給豳王忽塔忒迷失王傅印,以西僧輦真吃剌思為帝師。詔僧尼徭役一切無有所與。丙戌,詔百官一品至三品,先言朝政得失一事;四品以下,悉聽敷陳。仍命趙世安、阿榮輯錄所上章疏,善者即議舉行。追封燕鐵木兒曾祖班都察為溧陽王,祖土土哈為升王,父床兀兒為楊王。庚寅,祓祭於太祖幄殿。以末吉為大司徒。中書省臣言:「舊制,凡有奏陳,眾議定共署,乃入奏。近年,事方議擬,一二省臣輒已上請,致多乖滯。今請如舊制。」御史臺臣言:「風憲官赴任,毋拘遠近,均給驛為宜。」並從之。辛卯,命帝師率其徒作佛事於凝暉閣。甲午,冀寧路旱饑,賑糧二千九百石。乙未,改封前鎮南王帖木兒不花為宣讓王。初,鎮南王脫不花薨,子孛羅不花幼,命帖木兒不花襲其爵。孛羅不花既長,帖木兒不花請以王爵歸之,乃特封宣讓王,以示褒寵。收諸王帖古思金印。詔諭廷臣曰:「皇姑魯國大長公主,蚤寡守節,不從諸叔繼尚,鞠育遺孤,其子襲王爵,女配予一人。朕思庶民若是者猶當旌表,況在懿親乎?趙世延、虞集等可議封號以聞。」詔:「諸僧寺田,自金、宋所有及累朝賜予者,悉除其租。其有當輸租者,仍免其役。僧還俗者,聽復為僧。」戊戌,以淮、浙、山東、河間四轉運司鹽引六萬,為魯國大長公主湯沐之資。己亥,遣使驛致故帝師舍利還其國,給以金五百兩、銀二千五百兩、鈔千五百錠、幣五千匹。加謚漢長沙王吳芮為長沙文惠王。壬寅,命江浙行省印佛經二十七藏。癸卯,蘄州路夏秋旱饑,賑米五千石。甲辰,以明年正月武宗忌辰,命高麗、漢僧三百四十人,預誦佛經二藏於大崇恩福元寺。丁未,造至元鈔四十五萬錠、中統鈔五萬錠,如歲例。中書省臣言:「在京酒坊五十四所,歲輸課十餘萬錠。比者間以賜諸王、公主及諸官寺,諸王、公主自有封邑、歲賜,官寺亦各有常產,其酒課悉令仍舊輸官為宜。」從之。開河東冀寧路、四川重慶路酒禁。罷土番巡捕都元帥府。賑上都留守司八剌哈赤二千二百餘戶、燭剌赤八百餘戶糧三月,鈔有差;牙連禿傑魯迭所居鷹坊八百七十戶糧三月。戊申,以玥璐不花為御史大夫,兼領隆祥總管府事。庚戌,詔興舉中政院事。辛亥,趣內外已授官者速赴任。改上都饅頭山為天歷山。壬子,織武宗御容成,即神御殿作佛事。敕:「凡階開府儀同三司者,班列居一品之前。」武昌江夏縣火,賑其貧乏者二百七十戶糧一月。黃州路及恩州旱,並免其租。是歲,會賦入之數:金三百二十七錠,銀千一百六十九錠,鈔九百二十九萬七千八百錠,幣帛四十萬七千五百匹,絲八十八萬四千四百五十斤,綿七萬六百四十五斤,糧千九十六萬五十三石。



\end{pinyinscope}