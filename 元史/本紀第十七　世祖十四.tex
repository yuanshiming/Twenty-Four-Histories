\article{本紀第十七 世祖十四}

\begin{pinyinscope}

 二十九年春正月甲午朔,以日食免朝賀。日食時,左右有珥,上有抱氣。丙申,雲南行中書省言:「羅甸歸附後改普定府,隸雲南省三十餘年。今創羅甸宣慰安撫司,隸湖南省,不便,乞罷之,仍以其地隸雲南省。」制曰:「可。」戊戌,清州饑,就陵州發粟四萬七千八百石賑之。己亥,命太史令郭守敬兼領都水監事,仍置都水監少監、丞、經歷、知事凡八員。八作司官舊制六員,今分為左右二司,增官二員。庚子,江西行省左丞高興言:「江西、福建汀、漳諸處連年盜起,百姓入山以避,乞降旨招諭復業。福建鹽課既設運司,又設四鹽使司,今若設提舉司專領鹽課,其酒稅課悉歸有司為便。福建銀鐵又各立提舉司,亦為冗濫,請罷去。」詔皆從之。禁商賈私以金銀航海。壬寅,以武平地震,全免去年稅四千五百三十六錠,今年量輸之,止征二千五百六十九錠。癸卯,命玉典赤阿里置司邕州以便糧餉,而以輕軍邏思明州。以漢天師張宗演男與棣嗣其教。升利用監正三品。甲辰,詔:「江南州縣學田,其歲入聽其自掌,春秋釋奠外,以廩師生及士之無告者。貢士莊田,則令核數入官。」乙巳,賜諸王失都兒金千兩。丙午,河南、福建行中書省臣請詔用漢語,有旨以蒙古語諭河南,漢語諭福建。罷河南宣慰司,以汴梁、襄陽、河南、南陽、歸德皆隸河南行省。復割湖廣省之德安、漢陽、信陽隸荊湖北道,蘄黃隸淮西道,並淮東道三宣慰司咸隸河南省。其荊湖北道宣慰司舊領辰、沅、澧、靖、歸、常德,直隸湖廣省。從葛蠻軍民安撫使宋子賢請,詔諭未附平伐、大甕眼、紫江、皮陵、潭溪、九堡等處諸洞貓蠻。戊申,太陰犯歲及軒轅左角。己酉,興州之興安、宜興兩縣饑,賑米五千石。罷南雄、韶州、惠州三路錄事司。壬子,桓州至赤城站戶告饑,給鈔計口賑之。癸丑,罷四賓庫,復會同館,初置織造段匹提舉司五。八番都元帥劉德祿言:「新附洞蠻十五寨,請置官府以統之。」詔設陳蒙、爛土軍民安撫司。江西行省伯顏、阿老瓦丁言:「蒙山歲課銀二萬五千兩。初制,煉銀一兩,免役夫田租五斗,今民力日困,每兩擬免一石。」帝曰:「重困吾民,民何以生!」從之。丙辰,播州洞蠻因籍戶懷疑竄匿,降詔招集之。以行播州軍民安撫使楊漢英為紹慶珍州南平等處沿邊宣慰使、行播州軍民宣撫使、播州等處管軍萬戶,仍佩虎符。壬戌,召嗣漢天師張與棣赴闕。



 二月甲子朔,金竹酋長騷驢貢馬、氈各二十有七,從其請減所部貢馬,降詔招諭之。賜新附黑蠻衣襖,遣回,命進所產硃砂、雄黃之精善者,無則止。遣使代祀岳瀆、后土、四海。乙丑,給輝州龍山、里州和中等縣饑民糧一月。丁卯,畋於近郊。命宿衛受月廩及蒙古軍以艱食受糧者,宣徽院仍領之。己巳,太陰犯畢。發通州、河西務粟,賑東安、固安、薊州、寶坻縣饑民。申禁鞭背。庚午,斡羅思招附桑州生貓、羅甸國古州等峒酋長三十一,所部民十二萬九千三百二十六戶,詣闕貢獻。壬申,敕遣使分行諸路,釋死罪以下輕囚。澤州獻嘉禾。乙亥,立總管高麗女直漢軍萬戶府,頒銀印,總軍六千人。以泉府太卿亦黑迷失、鄧州舊軍萬戶史弼、福建行省右丞高興並為福建行中書省平章政事,將兵征瓜哇,用海船大小五百艘、軍士二萬人。戊寅,立征行左、右軍都元帥府,都元帥四、副都元帥二。上萬戶府達魯花赤四、萬戶皆四、副萬戶八、鎮撫四,各佩虎符。詔加高麗王王睶太保,仍錫功臣之號。詔從諸王阿禿作亂者,朵羅帶以付闊里吉思,脫迭出以付阿里,抄兒赤以付月的迷失,合麥以付亦黑迷失,使從軍自效。又詔諸王從合丹作亂者,納答兒之鎮南王所,聶怯來之合剌合孫答剌罕所,阿禿之云南王所,朵列禿之阿里所,八里帶之月的迷失所,斡里羅、忽裏帶之東海。發義倉官倉糧,賑德州、齊河、清平、泰安州饑民。庚辰,月兒魯等言:「納速剌丁滅里、忻都、王巨濟黨比桑哥,恣為不法,楮幣、銓選、鹽課、酒稅,無不更張變亂之。銜命江南理算者,皆嚴急輸期,民至嫁妻賣女,禍及親鄰。維楊、錢塘,受害最慘,無故而隕其生五百餘人。其初士民猶疑事出國家,今乃知天子仁愛元元,而使民至此極者,實桑哥及其兇黨之為,莫不願食其肉。臣等議,此三人既已伏辜,乞依條論坐以謝天下。」從之。牙亦迷失招無籍民千四百三十六戶,請隸東宮,詔命之耕田。辛巳,從樞密院臣暗伯等請,就襄陽給曲先塔林合剌魯六百三十七戶田器種粟,俾耕而食。丁亥,以汪惟和為鞏昌等二十四處便宜都總帥,兼鞏昌府尹,仍佩虎符。御史臺月兒魯、崔彧等言:「馮子振、劉道元指陳桑哥同列罪惡,詔令省臺臣及董文用、留夢炎等議。其一言:翰林諸臣撰《桑哥輔政碑》者,廉訪使閻復近已免官,餘請聖裁。」帝曰:「死者勿論,其存者罰不可恕也。」乞臺不花等使緬國,詔令遙授左丞。廷議以尚書行使事,其副以郎中處之。制曰:「可。」戊子,禁杭州放鷹。己丑,歲星犯軒轅大星。庚寅,宣政院臣言,授諸路釋教都總統輦真術納思為太中大夫、土蕃等處宣慰使都元帥。敕畸零拔都兒三百四十七戶佃益都閑田,給牛種農具,官為屋居之。壬辰,山東廉訪司申:「棣州境內春旱且霜,夏復霖澇,饑民啖藜藿木葉,乞賑恤。」敕依東平例,發附近官廩,計口以給。



 三月甲午,詔遣脫忽思、儂獨赤昔烈門至合敦奴孫界,與駙馬闊里吉思議行屯田。己亥,樞密院臣言:「出征女直納裡哥,議於合思罕三千新附軍內選撥千人。」詔先調五百人,行中書省具舟給糧,仍設征東招討司。壬寅,御史大夫月兒魯等奏:「比監察御史商琥舉昔任詞垣風憲,時望所屬而在外者,如胡祗遹、姚燧、王惲、雷鷹、陳天祥、楊恭懿、高道、程文海、陳儼、趙居信十人,宜召置翰林,備顧問。」帝曰:「朕未深知,俟召至以聞。」丙午,中書省臣言:「京畿薦饑,宜免今歲田租。上都、隆興、平灤、河間、保定五路供億視他路為甚,宜免今歲公賦。漢地河泊隸宣徽院,除入太官外,宜弛其禁,便民取食。」並從之。丁未,納速剌丁滅里以盜取官民鈔一十三萬餘錠,忻都以征理逋負迫殺五百二十人,皆伏誅。王巨濟雖無贓,帝以與忻都同惡,並誅之。中書省與御史臺共定贓罪十三等,枉法者五,不枉法者八,罪入死者以聞。制曰:「可。」戊申,以威寧、昌等州民饑,給鈔二千錠賑之。己酉,以大司農、同知宣徽院事兼領尚膳監事鐵哥,翰林學士承旨、通政院使兼知尚乘寺事剌真,並為中書平章政事,兼領舊職。中書省臣言:「右丞何榮祖以疾,平章政事麥術丁以久居其任,乞令免署,惟食其祿,與議中書省事。」從之。以阿里為中書右丞,梁暗都剌為參知政事。中書省臣言:「亦奚不薛及八番羅甸既各設宣慰司,又復立都元帥府,其地甚狹而官府多,宜合二司帥府為一。」詔從之,且命亦奚不薛與思、播州同隸湖廣省,羅甸還隸雲南,以八番羅甸宣慰使斡羅思等並為八番順元等處宣慰使都元帥,佩虎符。以安南國王陳益稷遙授湖廣等處行中書省平章政事,佩虎符,居鄂州。庚戌,車駕幸上都。賜速哥、斡羅思、賽因不花蠻夷之長五十六人金紋綾絹各七十九匹,及弓矢、鞍轡。壬子,樞密院臣奏:「延安、鳳翔、京兆三路籍軍三千人,桑哥皆罷為民,今復其軍籍,屯田六盤。」從之。敕都水監分視黃河堤堰,罷河渡司。庚申,免寶慶路邵陽縣田租萬三千七百九十三斛。壬戌,給還楊璉真加土田、人口之隸僧坊者。初,璉真加重賂桑哥,擅發宋諸陵,取其寶玉,凡發塚一百有一所,戕人命四,攘盜詐掠諸贓為鈔十一萬六千二百錠,田二萬三千畝,金銀、珠玉、寶器稱是。省臺諸臣乞正典刑以示天下,帝猶貸之死,而給還其人口、土田。隆興府路饑,給鈔二千錠,復發粟以賑之。



 夏四月丙子,太陰犯氐。己卯,復典瑞監三品。弛甘肅酒禁,榷其酤。辛巳,弛太原酒禁,仍榷酤。辛卯,設雲南諸路學校,其教官以蜀士充。



 五月甲午,遼陽水達達、女直饑,詔忽都不花趣海運給之。丙午,敕:「雲南邊徼入朝,非初附者不聽乘傳,所進馬不給芻豆。」丁未,中書省臣言:「妄人馮子振嘗為詩譽桑哥,且涉大言,及桑哥敗,即告詞臣撰碑引諭失當,國史院編修官陳孚發其奸狀,乞免所坐,遣還家。」帝曰:「詞臣何罪!使以譽桑哥為罪,則在廷諸臣,誰不譽之!朕亦嘗譽之矣。」詔以楊居寬、郭佑死非其罪,給還其家資。改思州安撫司為軍民宣撫司,隸湖廣省,詔諭其民因閱戶驚逃者,各使安業。以陜西鹽運司酒稅等課已入州縣,罷諸子鹽司,並罷東平路河道提舉司事入都水監。己未,龍興路南昌、新建、進賢三縣水,免田租四千四百六十八石。是月,真定之中山新樂、平山、獲鹿、元氏、靈壽,河間之滄州無棣,景之阜城、東光,益都之濰州北海縣,有蟲食桑葉盡,無蠶。



 六月甲子,平江、湖州、常州、鎮江、嘉興、松江、紹興等路水,免至元二十八年田租十八萬四千九百二十八石。戊辰,詔聽僧食鹽不輸課。己巳,日本來互市,風壞三舟,惟一舟達慶元路。壬申,江西省臣言:「肇慶、德慶二路,封、連二州,宋時隸廣東,今隸廣西,不便,請復隸廣東。」從之。鐵旗城後察昔折一烈率其族類部曲三千餘戶來附。甲戌,設司籍庫,秩從五品,隸太府監,儲物之籍入者。丙子,太寧路惠州連年旱澇,加以役繁,民饑死者五百人,詔給鈔二千錠及糧一月賑之,仍遣使責遼陽省臣阿散。壬午,敕以海南新附四州洞寨五百一十九、民二萬餘戶,置會同、定安二縣,隸瓊州,免其田租二年。癸未,以征瓜哇,暫禁兩浙、廣東、福建商賈航海者,俟舟師已發後,從其便。丁亥,湖州、平江、嘉興、鎮江、楊州、寧國、太平七路大水,免田租百二十五萬七千八百八十三石。己丑,太白犯歲星。鐵木塔兒、薛闍禿、捏古帶、闊闊所部民饑,詔給米四千石付鐵木塔兒、薛闍禿,一千石付捏古帶、闊闊,俾以賑之。



 閏六月辛卯朔,升上都兵馬司四品,如大都。丁酉,遼陽、沈州、廣寧、開元等路雹害稼,免田租七萬七千九百八十八石。岳州華容縣水,免田租四萬九百六十二石。東昌路蝗。壬寅,以東安、海寧改隸淮安路。詔大都事繁,課稅改隸轉運司,通州造船畢,罷提舉司。罷福建歲造象齒鞶帶。戊申,熒惑犯狗國。庚戌,回回人忽不木思售大珠,帝以無用卻之。辛亥,河西務水,給米賑饑民。江北河南省既立,詔江北諸城悉隸其省。詔漢陽隸湖廣省。左江總管黃堅言:「其管內黃勝許聚眾二萬據忠州,乞調軍萬人、土兵三千人,命劉國傑討之。臣願調軍民萬人以從。」詔許之。太平、寧國、平江、饒、常、湖六路民艱食,發粟賑之。高麗饑,其王遣使來請粟,詔賜米十萬石。中書省臣言:「今歲江南海運糧至京師者一百五萬石,至遼陽者十三萬石,比往歲無耗折不足者。」甲寅,右江岑從毅降,從毅老疾,詔以其子鬥榮襲,佩虎符,為鎮安路軍民總管。廣南西路安撫副使賽甫丁等誹謗朝政,沙不丁復資給之,以風聞三十餘事,妄告省官,帝以有傷政體,捕惡黨下吏如法。乙卯,濟南、般陽蝗。是月,詔諭廉訪司巡行勸課農桑。禮部尚書張立道、郎中歪頭使安南回,以其使臣阮代乏、何維巖至闕。陳日燇拜表箋,修歲貢。



 秋七月庚申朔,詔以史弼代也黑迷失、高興,將萬人征瓜哇,仍召三人者至闕。遣使檢核竄名鷹坊受糧者。辛酉,河北河南道廉訪司還治汴梁。癸亥,完大都城。也裏嵬里、沙沙嘗簽僧、道、儒、也裏可溫、答赤蠻為軍,詔令止隸軍籍。甲子,降詔申嚴牛馬踐稼之禁。乙丑,阿里願自修船,同張存從征瓜哇軍,往招占城、甘不察,詔授阿里三珠虎符,張存一珠虎符,仍蠲阿里父布伯所負斡脫鈔三千錠。丙寅,罷徽州路錄事司。免屯田租一萬二千八百一十一石。辛未,太陰犯牛。壬申,建社稷和義門內,壇各方五丈,高五尺,白石為主,飾以五方色土,壇南植松一株,北墉瘞坎壝垣,悉仿古制,別為齋廬,門廡三十三楹。戊寅,黎兵百戶鄧志願謀叛,伏誅。庚辰,敕云南省擬所轄州縣官如福建、二廣例,省臺委官銓選以姓名聞,隨給授宣敕。



 八月己丑朔,賽甫丁處死,餘黨杖而徙之,仍籍其家產。壬辰,敕禮樂戶仍與軍站、民戶均輸賦。丁酉,辰星犯右執法。己亥,太白犯房。辛丑,寧夏府屯田成功,升其官脫兒赤。壬寅,括唐兀禿魯花所部闊彖赤及河西逃人入蠻地者。甲辰,車駕至自上都。討浙東孟總把等賊,敕諸軍之駐福建者,聽平章政事闍裏節度。乙巳,歲星犯右執法。丙午,用郭守敬言,浚通州至大都漕河十有四,役軍匠二萬人,又鑿六渠灌昌平諸水。以廣濟署屯田既蝗復水,免今年田租九千二百十八石。丁未,也黑迷失乞與高興等同征瓜哇,帝曰:「也黑迷失惟熟海道,海中事當付之,其兵事則委之史弼可也。」以史弼為福建等處行中省平章政事,統領出征軍馬。庚戌,高苑縣高希允以非所宜言,伏誅。壬子,詔塔剌赤、程鵬飛討黃聖許,劉國傑駐馬軍戍守。戊午,福建行省參政魏天祐獻計,發民一萬鑿山煉銀,歲得萬五千兩。天祐賦民鈔市銀輸官,而私其一百七十錠,臺臣請追其贓而罷煉銀事,從之。改燕南河北廉訪司還治真定。高麗、女直界首雙城告饑,敕高麗王於海運內以粟賑之。弛平灤州酒禁。詔不敦、忙兀魯迷失以軍徵八百媳婦國。



 九月己未朔,治書侍御史裴居安言:「月的迷失遇盜起不即加兵,盜去乃延誅平民。」詔臺院遣官按問之。辛酉,詔諭安南國陳日燇使親入朝。選湖南道宣慰副使梁會,授吏部尚書,佩三珠虎符,翰林國史院編修官陳孚,授禮部郎中,佩金符,同使安南。山東東西道廉訪司劾:「宣慰使樂實盜庫鈔百二十錠,買庫銀九百五十兩,官局私造弓勒等物,受屯田鈔百八十錠,樂實宜解職。」從之。丁卯,中書省臣言:「茆灊、十圍、安化等新附洞蠻凡八萬,宜設管軍民司,以其土人蒙意、蒙世、莫仲文為長官,以呂天佑、塔不帶為達魯花赤。八番斡羅思招附光蘭州洞蠻,宜置定遠府,就用其所舉禿干、高守文、黃世曾、燕只哥為達魯花赤、知府、同知、判官。」制曰:「可。」癸酉,徙沔州治鐸水縣,廢新得州置通江縣,復漢州綿竹縣。沙州、瓜州民徙甘州,詔於甘、肅兩界,畫地使耕,無力者則給以牛具農器。寧夏戶口繁多,而土田半紅花,詔令盡種穀麥,以補民食。丁丑,以平灤路大水且霜,免田租二萬四千四十一石。辛巳,太白犯南斗。罷雲南行臺,徙置西川,設雲南廉訪司。壬午,水達達、女直民戶由反地驅出者,押回本地,分置萬夫、千夫、百夫內屯田。甲申,烏思藏宣慰司言:「由必里公反後,站驛遂絕,民貧無可供億。」命給烏思藏五驛各馬百、牛二百、羊五百,皆以銀;軍七百三十六戶,戶銀百五十五。丁亥,從宣政院言,置烏思藏納里速古兒孫等三路宣慰使司都元帥。



 冬十月戊子朔,詔福建廉訪司知事張師道赴闕;師道至,乞汰內外官府之冗濫者。詔麥術丁、何榮祖、馬紹、燕公楠等與師道同區別之。數月,授師道翰林直學士。日本舟至四明,求互市,舟中甲仗皆具,恐有異圖,詔立都元帥府,令哈剌帶將之,以防海道。詔浚浙西河道,導水入海。庚寅,兩淮運使納速剌丁坐受商賈賄,多給之鹽,事覺,詔嚴加鞫問。癸巳,弛上都酒禁。燕公楠言:「歲終,各行省臣赴闕奏事,亦宜令行臺臣赴闕,奏一歲舉剌之數。」制曰:「可。」丙申,四川行省以洞蠻酋長向思聰等七人入朝。壬寅,從硃清、張瑄請,授高德誠管領海船萬戶,佩雙珠虎符,復以殷實、陶大明副之,令將出征水手。甲辰,信合納帖音國遣使入覲。廣東道宣慰司遣人以暹國主所上金冊詣京師。乙巳,太陰犯井。丁未,太陰犯鬼。己酉,樞密院臣言:「六衛內領漢軍萬戶,見存者六千戶,撥分為三:力足以備車馬者二千五百戶,每甲令備馬十五匹、牛車二輛;力足以備車者五百戶,每甲令備牛車三輛;其三千戶,惟習戰鬥,不他役之。六千戶外,則供他役。庶能各勤乃事,而兵亦精銳。」詔施行之。詔擇囚徒罪輕者釋之。癸丑,完澤等言:「凡賜諸人物有二十萬錠者,為數既多,先賜者盡得之,及後將賜,或無可給,不均為甚。今計怯薛帶、怯憐口、昔博赤、哈剌赤,凡近侍人,上等以二百戶為率,次等半之,下等又半之,於下等擇尤貧者歲加賞賜,則無不均之失矣。一歲天下所入,凡二百九十七萬八千三百五錠,今歲已辦者才一百八十九萬三千九百九十三錠,其中有未至京師而在道者,有就給軍旅及織造物料館傳俸祿者,自春至今,凡出三百六十三萬八千五百四十三錠,出數已逾入數六十六萬二百三十八錠矣。懷孟竹課,歲辦千九十三錠,尚書省分賦於民,人實苦之,宜停其稅。」帝皆嘉納其言。命趙德澤、吳榮領逃奴無主者二百四十戶,淘銀耕田於廣寧、沈州。乙卯,太陰犯氐。



 十一月庚申,岳州華容縣水,發米二千一百二十五石賑饑民。壬戌,太陰犯壘壁陣。戊寅,樞密院奏:「一衛萬人,嘗調二千屯田,木八剌沙上都屯田二年有成,擬增軍千人。」從之。己卯,太陰犯太微東垣上相。癸未,禁所在私渡,命關津譏察奸宄。丙戌,提省溪、錦州、銅人等洞酋長楊秀朝等六人入見,進方物。



 十二月庚寅,中書省臣言:「皇孫晉王甘麻剌昔鎮雲南,給梁王印,今進封晉王,請給晉王印。北安王府尉也裏古帶、司馬荒兀,並為晉王中尉,仍命不只答魯帶、狄琮並為司馬。金齒適當忙兀禿兒迷失出征軍馬之沖,資其芻糧,立為木來府。」敕應昌府給乞答帶糧五百石,以賑饑民。癸巳,中書省臣言:「寧國路民六百戶鑿山冶銀,歲額二千四百兩,皆市銀以輸官,未嘗採之山,乞罷之。」制曰:「可。」庚子,太陰犯井。甲辰,太陰犯太微西垣。己酉,故麓川路軍民總管達魯花赤阿散男布八同趙升等,招木忽魯甸金齒土官忽魯馬男阿魯來入見,貢方物。阿魯言其地東南鄰境未附者約二十萬民,慕化願附,請頒詔旨,命布八、趙升諭之,從之。壬子,敕中書省用烏思藏站例,給合里、忽必二站馬牛羊,凡為銀九千五百兩。丁巳,敕都水監修治保定府沙塘河堤堰。是歲,賜皇子、皇孫、諸王、籓戚、禁衛、邊庭將士等,鈔四十六萬六千七百十三錠。給軍士畸零口糧五千五百二十三石,賑其乏者為鈔三十六萬八千四百二十八錠。命國師、諸僧、咒師修佛事七十二會。斷死獄七十四。



 三十年春正月壬戌,詔遣使招諭漆頭、金齒蠻。乙丑,敕福建毋進鶻。戊戌,和林漢軍四百,留百人,餘令耕屯杭海。丙寅,太陰犯畢。命中書汰冗員,凡省內外官府二百五十五所,總六百六十九員。丁卯,安西王請仍舊設常侍,不允。罷雲南延慶司,以洛波、卜兒二蠻酋遙授知州,各賜璽書。戊辰,樞密院臣奏:「兀渾察部兀末魯罕軍,每歲運米六千四百二十六石以給之,計傭直為鈔萬二千八百五十二錠。」詔邊境無事,令本軍屯耕以食。庚午,驗洞酋長楊總國等來朝。捏怯烈女直二百人以漁自給,有旨:「與其漁於水,曷若力田,其給牛價、農具使之耕。」甲戌,河南江北行省平章伯顏言:「揚州忙兀臺所立屯田,為田四萬餘頃,官種外,宜聽民耕墾,揚州鹽轉運一司設三重官府,宜削去鹽司,止留管勾。襄陽舊食京兆鹽,以水陸難易計之,莫若改食揚州鹽。蔡州去汴梁地遠,宜升散府,以潁、息、信陽、光州隸之。」詔皆從其議。升廣州為上路總管府,罷納速剌丁滅里所立魚鹽局,割江西興國路隸湖廣行省。乙亥,謚皇太子曰明孝。丙子,西番一甸蠻酋三人來覲,各授以蠻夷軍民官,仍以招諭人張道明為達魯花赤。丁丑,太陰犯氐。戊寅,詔舊隸乃顏、勝納答兒女直戶四百,虛縻廩食,令屯田揚州。庚辰,歲星犯左執法。立豪、懿州七驛。辛巳,置遼陽路慶雲至合里賓二十八驛,驛給牛三十頭、車七輛。壬午,淮西道宣慰使昂吉兒,斂軍鈔六百錠、銀四百五十兩、馬二匹,敕省臺及扎魯火赤鞫問。丁亥,遣使代祀岳瀆、東海及後土。



 二月己丑,從阿老瓦丁、燕公楠之請,以楊璉真加子宣政院使暗普為江浙行省左丞。詔:「上都管倉庫者無資品俸秩,故為盜詐,宜於六品、七品內委用,以俸給之。」高麗國王王睶請易名曰昛,其簽議府請升簽議司,降二品印,從之。減河南、江浙海運米四十萬石。中書省添設檢校二員。免大都今歲公賦。益上都屯田軍千人,給農具、牛價鈔五千錠,以木八剌沙董之。詔以只速滅里與鬼蠻之民隸詹事院。壬辰,太陰犯畢。丙申,卻江淮行樞密院官不憐吉帶進鷹,仍敕自今禁戢軍官無從禽擾民,違者論罪。丁酉,回回孛可馬合謀沙等獻大珠,邀價鈔數萬錠,帝曰:「珠何為!當留是錢以賙貧者。」敕海運米十萬石給遼陽戍兵,仍諭其省官薛闍干,令伯鐵木部欽察等耕漁自養,糧不須給。甲辰,中書省臣言:「侍臣傳旨予官者,先後七十人,臣今欲加汰擇,不可用者不敢奉詔。」帝曰:「率非朕言,凡來奏者朕只令諭卿等,可用與否,卿等自處之。」又言:「今歲給餉上都、大都及甘州、西京,經費浩繁,自今賞賜悉宜姑止。」從之。乙巳,熒惑犯天街。丁未,車駕幸上都。以新附洞蠻吳動鰲為潭溪等處軍民官,佩金符。給新附軍三百人,人鈔十錠,屯田真定。庚戌,太陰犯牛。辛亥,詔發總帥汪惟和所部軍三千征土番,又發陜西、四川兵萬人,以行樞密官明安答兒統之,征西番。敕以韶、贛相去地遠,分贛州行院官一員鎮韶州。復立雲南行御史臺。詔沿海置水驛,自耽羅至鴨淥江口凡十一所,令洪君祥董之。癸丑,太白犯壘壁陣。江西行院官月的迷失言:「江南豪右多庇匿盜賊,宜誅為首者,餘徙內縣。」從之。申嚴江南兵器之禁。



 三月庚申,以同知樞密院事扎散知樞密院事,以平章政事範文虎董疏漕河之役。平章政事李庭率諸軍扈從上都。雨壞都城,詔發侍衛軍三萬人完之,仍命中書省給其傭直。甲子,括天下馬十萬匹。己巳,立行大司農司。洪澤、芍陂屯田舊委四處萬戶,詔存其二,立民屯二十。辛未,太陰犯氐。



 夏四月己亥,行大司農燕公楠、翰林學士承旨留夢炎言:「杭州、上海、澉浦、溫州、慶元、廣東、泉州置市舶司凡七所,唯泉州物貨三十取一,餘皆十五抽一,乞以泉州為定制。」從之。仍並溫州舶司入慶元,杭州舶司入稅務。江南行大司農司自平江徙揚州,兼管兩淮農事。省八番重設州縣官,罷徽州錄事司。皇孫晉王位立內史府。詔諸二品官府自今與各部文移相關。鞏昌二十四城,依舊例於總帥汪氏弟兄子侄內選用二人。壬寅,樞密院臣言:「去年征瓜哇軍二萬,各給鈔二錠,其後只以五千人往,宜徵元給鈔三萬錠入官。」帝曰:「非其人不行,乃朕中止之耳,勿征。」癸丑,太白犯填星。廣東肅政廉訪司復治廣州。甲寅,詔遣使招諭暹國。斡羅思請以八番見戶合思、播之民兼管,徙宣慰司治辰、沅、靖州,常賦外,歲輸鈔三千錠,不允。光州蠻人光龍等一十二人及邦崖王文顯等二十八人、金竹府馬麟等一十六人、大龍番禿盧忽等五十四人、永順路彭世強等九十人、安化州吳再榮等一十三人、師壁散毛洞勾答什王等四人,各授蠻夷官,賜以璽書遣歸。敕江南毀諸道觀聖祖天尊祠。



 五月丙辰朔,給四部更番衛士馬萬匹,又給其必闍赤四百匹。壬戌,定雲洞蠻酋長來附。癸亥,括思、播等處亡宋涅手軍。丙寅,詔委官與行省官閱核蠻夷軍民官。以江南民怨楊璉真加,罷其子江浙行省左丞暗普。詔以浙西大水冒田為災,令富家募佃人疏決水道。辛未,敕僧寺之邸店,商賈舍止,其物貨依例收稅。丁丑,中書省臣言:「上都工匠二千九百九十九戶,歲縻官糧萬五千二百餘石,宜擇其不切於用者,俾就食大都。」從之。甲申,真定路深州靜安縣大水,民饑,發義倉糧二千五百七十四石賑之。



 六月丙戌,敕選河西質子軍精銳者八百,給以鎧仗鞍勒、狐貉衣裘,遣赴皇孫阿難答所出征。己丑,歲星犯左執法。庚寅,詔雲南旦當仍屬西番宣慰司。改淮西蘄、黃等路隸河南江北行省。丙申,太陰犯鬥。乙巳,以皇太子寶授皇孫鐵穆耳,總兵北邊。己酉,詔浚太湖。壬子,大興縣蝗;易州雨雹,大如雞卵。



 秋七月丁巳,敕中書省官一員監修國史。己未,詔皇曾孫松山出鎮雲南,以皇孫梁王印賜之。詔免福建歲輸皮貨及泉州織作紵絲。庚申,命知鶴慶府昔寶赤齎璽書招諭農順未附蠻寨。甲子,太陰犯建星。己巳,命劉國傑從諸王亦吉里督諸軍征交趾。免雲南屯田軍逋租萬石。壬申,以月失察兒知樞密院事。丁丑,賜新開漕河名曰通惠。壬申,以只兒合忽所汰乞兒吉思戶七百,屯田合思合之地。辛巳,太陰犯鬼。



 八月丙戌,括所在荒田無主名者,令放良、漏籍等戶屯田。庚寅,奉使安南國梁曾、陳孚以安南使人陶子奇、梁文藻偕來。敕福建行省放瓜哇出征軍歸其家。甲午,辰星犯太微西垣上將。戊戌,給安西王府斷事官印。甲辰,太陰犯畢。丁未,湖廣行省臣言海南、海北多曠土,可立屯田,詔設鎮守黎蠻海北海南屯田萬戶府以董之。戊申,太陰犯鬼。營田提舉司所轄屯田百七十七頃為水所沒,免其租四千七百七十二石。



 九月癸丑朔,大駕至自上都。戊午,敕各路達魯花赤、總管董驛事。己未,明安答兒率軍萬人征土蕃,近遣使來言,乞引茂州先附寨官赴闕,不允。乙丑,立海北海南博易提舉司,稅依市舶司例。丙寅,遣金齒人還歸。丁卯,太陰犯畢。癸酉,敕以御史臺贓罰鈔五萬錠,給衛士之貧者。辛巳,登州蝗,恩州水,百姓闕食,賑以義倉米五千九百餘石。冬十月癸未朔,以侍衛親軍千戶張邦瑞為萬戶,佩虎符,將六盤山軍千人及皇子西平王等軍共為萬人,西征。賜冠城疏河董役軍官衣各一襲;賜交趾陶子奇等十七人冬衣,荊南安置。戊子,詔修汴堤。己丑,遣兵部侍郎忽魯禿花等使閣藍、可兒納答、信合納帖音三國,仍賜信合納帖音酋長三珠虎符。庚寅,饗於太廟。彗星入紫微垣,抵鬥魁,光芒尺許,凡一月乃滅。丙申,熒惑犯亢。己亥,太陰犯天關。辛丑,太陰犯井。壬寅,敕減米直,糶京師饑民,其鰥寡孤獨不能自存者給之。甲辰,赦天下。戊申,僧官總統以下有妻者罷之。以段貞董開河、修倉之役,加平章政事。庚戌,造象蹄掌甲。辛亥,禁江南州郡以乞養良家子轉相販鬻,及強將平民略賣者。平灤水,免田租萬一千九百七十七石。廣濟署水,損屯田百六十五頃,免田租六千二百一十三石。



 十一月壬子朔,改德安府隸黃州路。丁巳,孫民獻嘗附桑哥,助要束木為惡,及同知上都留守司事,又受贓減諸從臣糧,詔籍其家貲、妻奴;復因潭州呂澤訴其刻虐,械送民獻至湖廣,如澤所訴窮治之。立海北海南道肅政廉訪司,治雷州。庚申,敕中書省,凡出征軍,毋以和顧和買煩其家。乙丑,太陰犯畢。乙卯,太陰犯井。戊辰,以金齒木朵甸戶口增,立下路總管府,給其為長者雙珠虎符。真定路達魯花赤合散言:「廉訪司官檢責民官太苛,乞以民官復檢責廉訪司文卷。」從之。庚午,太陰犯鬼。免江南都作院軍匠出征。丙子,熒惑犯鉤鈐。戊寅,歲星犯亢。己卯,河南江北行省平章伯顏入為中書省平章政事,位帖哥、剌真、不忽木上。



 十二月丁亥,禁漢軍更番者毋鬻軍器。辛卯,武平路達魯花赤塔海言:「女直地至今未定,賊一人入境,百姓離散,臣願往安集之。」詔以塔海為遼東道宣慰使。壬辰,中書左丞馬紹疾,以詹事丞張九思代之。乙未,太陰犯井。遣使督思、播二州及鎮遠、黃平,發宋舊軍八千人,從征安南。庚子,平章政事亦黑迷失、史弼、高興等無功而還,各杖而恥之,仍沒其家貲三之一。癸卯,敕以桑哥沒入官田三百九十一頃八十餘廟,給阿合兀闌所司匠戶。丙午,以鐵赤、脫脫木兒、咬住、拜延四人,並安西王傅。是歲,天下路、府、州、縣等二千三十八:路一百六十九,府四十三,州三百九十八,縣千一百六十五,宣撫司十五,安撫司一,寨十一,鎮撫所一,堡一,各甸部管軍民官七十三,長官司五十一,錄事司百三,巡院三。官府大小二千七百三十三處,隨朝二百二十一;員萬六千四百二十五,隨朝千六百八十四。戶一千四百萬二千七百六十。賜皇后、親王、公主如歲例。賜諸臣羊馬價,鈔四十三萬四千五百錠、幣五萬五千四百一十錠。周貧乏,鈔三萬七千五百二十錠。作佛事祈福五十一。真定、寧晉等處,被水、旱、蝗、雹為災者二十九。斷死罪四十。



 三十一年春正月壬子朔,帝不豫,免朝賀。癸亥,知樞密院事伯顏至自軍中。庚午,帝大漸。癸酉,帝崩於紫檀殿。在位三十五年,壽八十。親王、諸大臣發使告哀於皇孫。乙亥,靈駕發引,葬起輦穀,從諸帝陵。



 夏四月,皇孫至上都。甲午,即皇帝位。丙午,中書右丞相完澤及文武百官議上尊謚。壬寅,始為壇於都城南七里。甲辰,遣司徒兀都帶、平章政事不忽木、左丞張九思,率百官請謚於南郊。



 五月戊午,遣攝太尉臣兀都帶奉冊上尊謚曰聖德神功文武皇帝,廟號世祖,國語尊稱曰薛禪皇帝。是日,完澤等議同上先皇後弘吉剌氏尊謚曰昭睿順聖皇后。



 世祖度量弘廣,知人善任使,信用儒術,用能以夏變夷,立經陳紀,所以為一代之制者,規模宏遠矣。



\end{pinyinscope}