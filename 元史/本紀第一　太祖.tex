\article{本紀第一 太祖}

\begin{pinyinscope}

 太祖法天啟運聖武皇帝,諱鐵木真,姓奇渥溫氏,蒙古部人。太祖其十世祖孛端義兒,母曰阿蘭果火器。,嫁脫奔咩哩犍,生二子,長曰博寒葛答黑,次曰博合睹撒裏直。既而夫亡,阿蘭寡居,夜寢帳中,夢白光自天窗中入,化為金色神人,來趨臥榻。阿蘭驚覺,遂有娠,產一子,即孛端義兒也。孛端義兒狀貌奇異,沉默寡言,家人謂之癡,獨阿蘭語人曰:「此兒非癡,後世子孫必有大貴者。」阿蘭沒,諸兄分家貲,不及之。孛端義兒曰:「貧賤富貴,命也,貲財何足道!」獨乘青白馬,至八里屯阿懶之地居焉。食飲無所得,適有蒼鷹搏野獸而食,孛端義兒以緡設機取之,鷹即馴狎,乃臂鷹,獵兔禽以為膳,或闕即繼,似有天相之。



 居月,有民數十家自統急里忽魯之野逐水草來遷。孛端義兒結茅與之居,出入相資,自此生理稍足。一日,仲兄忽思之,曰:「孛端義兒獨出而無齎,近者得無凍餒乎?」即自來訪,邀與俱歸。孛端義兒中路謂其兄曰:「統急里忽魯之民無所屬附,若臨之以兵,可服也。」兄以為然,至家,即選壯士,令孛端義兒帥之前行,果盡降之。



 孛端義兒歿,子八林昔黑剌禿合必畜嗣,生子曰咩撚篤敦。咩撚篤敦妻曰莫拿倫,生七子而寡。莫拿倫性剛急。時押剌伊而部有群小兒掘田間草根以為食,莫拿倫乘車出,適見之,怒曰:「此田乃我子馳馬之所,群兒輒敢壞之邪?」驅車徑出,輾傷諸兒,有至死者。押剌伊而忿怨,盡驅莫拿倫馬群以去。莫拿倫諸子聞之,不及被甲,往追之。莫拿倫私憂曰:「吾兒不甲以往,恐不能勝敵。」令子婦載甲赴之,已無及矣。既而果為所敗,六子皆死。押剌伊而乘勝殺莫拿倫,滅其家。唯一長孫海都尚幼,乳母匿諸積木中,得免。先是莫拿倫第七子納真,於八剌忽民家為贅婿,故不及難。聞其家被禍,來視之,見病嫗十數與海都尚在,其計無所出,幸驅馬時,兄之黃馬三次掣套竿逸歸,納真至是得乘之。乃偽為牧馬者,詣押剌伊而。路逢父子二騎先後行,臂鷹而獵。納真識其鷹,曰:「此吾兄所擎者也。」趨前紿其少者曰:「有赤馬引群馬而東,汝見之乎?」曰:「否。」少者乃問曰:「爾所經過有鳧雁乎?」曰:「有。」曰:「汝可為吾前導乎?」曰:「可。」遂同行。轉一河隈,度後騎相去稍遠,刺殺之。縶馬與鷹,趨迎後騎,紿之如初。後騎問曰:「前射鳧雁者,吾子也,何為久臥不起耶?」納真以鼻衄對。騎者方怒,納真乘隙刺殺之。復前行,至一山下,有馬數百,牧者唯童子數人,方擊髀石為戲。納真熟視之,亦兄家物也。紿問童子,亦如之。於是登山四顧,悄無來人,盡殺童子,驅馬臂鷹而還,取海都並病嫗,歸八剌忽之地止焉。海都稍長,納真率八剌忽怯穀諸民,共立為君。海都既立,以兵攻押剌伊而,臣屬之,形勢浸大,列營帳於八剌合黑河上,跨河為梁,以便往來。由是四傍部族歸之者漸眾。



 海都歿,子拜姓忽兒嗣。拜姓忽兒歿,子敦必乃嗣。敦必乃歿,子葛不律寒嗣。葛不律寒歿,子八哩丹嗣。八哩丹歿,子也速該嗣,並吞諸部落,勢愈盛大。也速該崩,至元三年十月,追謚烈祖神元皇帝。



 初,烈祖征塔塔兒部,獲其部長鐵木真。宣懿太後月倫適生帝,手握凝血如赤石。烈祖異之,因以所獲鐵木真名之,志武功也。族人泰赤烏部舊與烈祖相善,後因塔兒不臺用事,遂生嫌隙,絕不與通。及烈祖崩,帝方幼沖,部眾多歸泰赤烏。近侍有脫端火兒真者,亦將叛,帝自泣留之。脫端曰:「深池已乾矣,堅石已碎矣,留復何為!」竟帥眾馳去。宣懿太后怒其弱己也,麾旗將兵,躬自追叛者,驅其太半而還。時帝麾下搠只別居薩里河。札木合部人禿臺察兒居玉律哥泉,時欲相侵凌,掠薩里河牧馬以去。搠只麾左右匿群馬中,射殺之。札木合以為怨,遂與泰赤烏諸部合謀,以眾三萬來戰。帝時駐軍答蘭版硃思之野,聞變,大集諸部兵,分十有三翼以俟。已而札木合至,帝與大戰,破走之。



 當是時,諸部之中,唯泰赤烏地廣民眾,號為最強。其族照烈部,與帝所居相近。帝常出獵,偶與照烈獵騎相屬。帝謂之曰:「今夕可同宿乎?」照烈曰:「同宿固所願,但從者四百,因糗糧不具,已遣半還矣,今將奈何?」帝固邀與宿,凡其留者,悉飲食之。明日再合圍,帝使左右驅獸向照烈,照烈得多獲以歸。其眾感之,私相語曰:「泰赤烏與我雖兄弟,常攘我車馬,奪我飲食,無人君之度。有人君之度者,其惟鐵木真太子乎?」照烈之長玉律,時為泰赤烏所虐,不能堪,遂與塔海答魯領所部來歸,將殺泰赤烏以自效。帝曰:「我方熟寐,幸汝覺我,自今車轍人跡之途,當盡奪以與汝矣。」已而二人不能踐其言,復叛去。塔海答魯至中路,為泰赤烏部人所殺,照烈部遂亡。



 時帝功德日盛,泰赤烏諸部多苦其主非法,見帝寬仁,時賜人以裘馬,心悅之。若赤老溫、若哲別、若失力哥也不干諸人,若朵郎吉、若札剌兒、若忙兀諸部,皆慕義來降。



 帝會諸族薛徹、大丑等,各以旄車載湩酪,宴於斡難河上。帝與諸族及薛徹別吉之毋忽兒真之前,共置馬湩一革囊;薛徹別吉次毋野別該之前,獨置一革囊。忽兒真怒曰:「今不尊我,而貴野別該乎?」疑帝之主膳者失丘兒所為,遂笞之。於是頗有隙。時皇弟別里古臺掌帝乞列思事,乞列思,華言禁外系馬所也。播裡掌薛徹別吉乞列思事。播里從者因盜去馬靷,別里古臺執之。播里怒,斫別里古臺,傷其背。左右欲鬥,別里古臺止之,曰:「汝等欲即復仇乎?我傷幸未甚,姑待之。」不聽,各持馬乳橦疾鬥,奪忽兒真、火裏真二哈敦以歸。薛徹別吉遣使請和,因令二哈敦還。會塔塔兒部長蔑兀真笑裏徒背金約,金主遣丞相完顏襄帥兵逐之北走。帝聞之,發近兵自斡難河迎擊,仍諭薛徹別吉帥部人來助。候六日不至,帝自與戰,殺蔑兀真笑裏徒,盡虜其輜重。帝之麾下有為乃蠻部人所掠者,帝欲討之,復遣六十人徵兵於薛徹別吉。薛徹別吉以舊怨之故,殺其十人,去五十人衣而歸之。帝怒曰:「薛徹別吉曩笞我失丘兒,斫傷我別里古臺,今又敢乘敵勢以陵我耶?」因帥兵逾沙磧攻之,殺虜其部眾,唯薛徹、大醜僅以妻孥免。越數月,帝復伐薛徹、大醜,追至帖烈徒之隘,滅之。



 克烈部札阿紺孛來歸。札阿紺孛者,部長汪罕之弟也。汪罕名脫里,受金封爵為王,番言音重,故稱王為汪罕。初,汪罕之父忽兒札胡思杯祿既卒,汪罕嗣位,多殺戮昆弟。其叔父菊兒罕帥兵與汪罕戰,逼於哈剌溫隘,敗之,僅以百餘騎脫走,奔於烈祖。烈祖親將兵逐菊兒罕走西夏,復奪部眾歸汪罕。汪罕德之,遂相與盟,稱為按答。按答,華言交物之友也。烈祖崩,汪罕之弟也力可哈剌,怨汪罕多殺之故,復叛歸乃蠻部。乃蠻部長亦難赤為發兵伐汪罕,盡奪其部眾與之。汪罕走河西、回鶻、回回三國,奔契丹。既而復叛歸,中道糧絕,捋羊乳為飲,刺橐駝血為食,困乏之甚。帝以其與烈祖交好,遣近侍往招之。帝親迎撫勞,安置軍中振給之,遂會於土兀剌河上,尊汪罕為父。



 未幾,帝伐蔑里乞部,與其部長脫脫戰於莫那察山,遂掠其資財、田禾,以遺汪罕。汪罕因此部眾稍集。居亡何,汪罕自以其勢足以有為,不告於帝,獨率兵復攻蔑里乞部。部人敗走,脫脫奔八兒忽真之隘。汪罕大掠而還,於帝一無所遺,帝不以屑意。



 會乃蠻部長不欲魯罕不服,帝復與汪罕徵之。至黑辛八石之野,遇其前鋒也的脫孛魯者,領百騎來戰,見軍勢漸逼,走據高山,其馬鞍轉墜,擒之。曾未幾何,帝復與乃蠻驍將曲薛吾撒八剌二人遇,會日暮,各還營壘,約明日戰。是夜,汪罕多燃火營中,示人不疑,潛移部眾於別所。及旦,帝始知之,因頗疑其有異志,退師薩里河。既而汪罕亦還至土兀剌河,汪罕子亦剌合及札阿紺孛來會。曲薛吾等察知之,乘其不備,襲虜其部眾於道。亦剌合奔告汪罕,汪罕命亦剌合與卜魯忽鷿共追之,且遣使來曰:「乃蠻不道,掠我人民,太子有四良將,能假我以雪恥乎?」帝頓釋前憾,遂遣博爾術、木華黎、博羅渾、赤老溫四人,帥師以往。師未至,亦剌合已追及曲薛吾,與之戰,大敗,卜魯忽鷿成擒,流矢中亦剌合馬胯,幾為所獲。須臾,四將至,擊乃蠻走,盡奪所掠歸汪罕。已而與皇弟哈撒兒再伐乃蠻,拒鬥於忽蘭盞側山,大敗之,盡殺其諸將族眾,積尸以為京觀,乃蠻之勢遂弱。



 時泰赤烏猶強,帝會汪罕於薩里河,與泰赤烏部長沆忽等大戰斡難河上,敗走之,斬獲無算。哈答斤部、散只兀部、朵魯班部、塔塔兒部、弘吉剌部聞乃蠻、泰赤烏敗,皆畏威不自安,會於阿雷泉,斬白馬為誓,欲襲帝及汪罕。弘吉剌部長迭夷恐事不成,潛遣人告變。帝與汪罕自虎圖澤逆戰於杯亦烈川,又大敗之。汪罕遂分兵,自由怯綠憐河而行。札阿紺孛謀於按敦阿述、燕火脫兒等曰:「我兄性行不常,既屠絕我昆弟,我輩又豈得獨全乎?」按敦阿述洩其言,汪罕令執燕火脫兒等至帳下,解其縛,且謂燕火脫兒曰:「吾輩由西夏而來,道路饑困,其相誓之語,遽忘之乎?」因唾其面,坐上之人皆起而唾之。汪罕又屢責札阿紺孛,至於不能堪,札阿紺孛與燕火脫兒等俱奔乃蠻。



 帝駐軍於徹徹兒山,起兵伐塔塔兒部。部長阿剌兀都兒等來逆戰,大敗之。



 時弘吉剌部欲來附,哈撒兒不知其意,往掠之。於是弘吉剌歸札木合部,與朵魯班、亦乞剌思、哈答斤、火魯剌思、塔塔兒、散只兀諸部,會於犍河,共立札木合為局兒罕,盟於禿律別兒河岸,為誓曰:「凡我同盟,有洩此謀者,如岸之摧,如林之伐。」誓畢,共舉足蹋岸,揮刀斫林,驅士卒來侵。塔海哈時在眾中,與帝麾下抄吾兒連姻。抄吾兒偶往視之,具知其謀,即還至帝所,悉以其謀告之。帝即起兵,逆戰於海剌兒、帖尼火魯罕之地,破之,札木合脫走,弘吉剌部來降。



 歲壬戌,帝發兵於兀魯回失連真河,伐按赤塔塔兒、察罕塔塔兒二部。先誓師曰:「茍破敵逐北,見棄遺物,慎無獲,俟軍事畢散之。」既而果勝,族人按彈、火察兒、答力臺三人背約,帝怒,盡奪其所獲,分之軍中。



 初,脫脫敗走八兒忽真隘,既而復出為患,帝帥兵討走之。至是又會乃蠻部不欲魯罕約朵魯班、塔塔兒、哈答斤、散只兀諸部來侵。帝遣騎乘高四望,知乃蠻兵漸至,帝與汪罕移軍入塞。亦剌合自北邊來據高山結營,乃蠻軍沖之不動,遂還。亦剌合尋亦入塞。將戰,帝遷輜重於他所,與汪罕倚阿蘭塞為壁,大戰於闕奕壇之野,乃蠻使神巫祭風雪,欲因其勢進攻。既而反風,逆擊其陣,乃蠻軍不能戰,欲引還。雪滿溝澗,帝勒兵乘之,乃蠻大敗。是時札木合部起兵援乃蠻,見其敗,即還,道經諸部之立己者,大縱掠而去。



 帝欲為長子術赤求昏於汪罕女抄兒伯姬,汪罕之孫禿撒合亦欲尚帝女火阿真伯姬,俱不諧,自是頗有違言。初,帝與汪罕合軍攻乃蠻,約明日戰,札木合言於汪罕曰:「我於君是白翎雀,他人是鴻雁耳。白翎雀寒暑常在北方,鴻雁遇寒則南飛就暖耳。」意謂帝心不可保也。汪罕聞之疑,遂移部眾於別所。及議昏不成,札木合復乘隙謂亦剌合曰:「太子雖言是汪罕之子,嘗通信於乃蠻,將不利於君父子。君若能加兵,我當從傍助君也。」亦剌合信之。會答力臺、火察兒、按彈等叛歸亦剌合,亦說之曰:「我等願佐君討宣懿太后諸子也。」亦剌合大喜,遣使言於汪罕。汪罕曰:「札木合,巧言寡信人也,不足聽。」亦剌合力言之,使者往返者數四。汪罕曰:「吾身之存,實太子是賴。髭須已白,遺骸冀得安寢,汝乃喋喋不已耶?汝善自為之,毋貽吾憂可也。」札木合遂縱火焚帝牧地而去。



 歲癸亥,汪罕父子謀欲害帝,乃遣使者來曰:「向者所議姻事,今當相從,請來飲布渾察兒。」布渾察兒,華言許親酒也。帝以為然,率十騎赴之,至中道,心有所疑,命一騎往謝,帝遂還。汪罕謀既不成,即議舉兵來侵。圉人乞失力聞其事,密與弟把帶告帝。帝即馳軍阿蘭塞,悉移輜重於他所,遣折里麥為前鋒,俟汪罕至,即整兵出戰。先與硃力斤部遇,次與董哀部遇,又次與火力失烈門部遇,皆敗之;最後與汪罕親兵遇,又敗之。亦剌合見勢急,突來沖陣,射之中頰,即斂兵而退。怯里亦部人遂棄汪罕來降。



 汪罕既敗而歸,帝亦將兵還,至董哥澤駐軍,遣阿里海致責於汪罕曰:「君為叔父菊兒罕所逐,困迫來歸,我父即攻菊兒罕,敗之於河西,其土地人民盡收與君,此大有功於君一也。君為乃蠻所攻,西奔日沒處。君弟札阿紺孛在金境,我亟遣人召還。比至,又為蔑里乞部人所逼,我請我兄薛徹別及及我弟大醜往殺之,此大有功於君二也。君困迫來歸時,我過哈丁里,歷掠諸部羊、馬、資財,盡以奉君,不半月間,令君饑者飽,瘠者肥,此大有功於君三也。君不告我,往掠蔑里乞部,大獲而還,未嘗以毫發分我,我不以為意。及君為乃蠻所傾覆,我遣四將奪還爾民人,重立爾國家,此大有功於君四也。我征朵魯班、塔塔兒、哈答斤、散只兀、弘吉剌五部,如海東鷙禽之於鵝雁,見無不獲,獲則必致於君,此大有功於君五也。是五者皆有明驗,君不報我則已,今乃易恩為仇,而遽加兵於我哉?」汪罕聞之,語亦剌合曰:「我向者之言何如?吾兒宜識之。」亦剌合曰:「事勢至今日,必不可已,唯有竭力戰鬥。我勝則並彼,彼勝則並我耳。多言何為?」時帝諸族按彈、火察兒皆在汪罕左右,帝因遣阿里海誚責汪罕,就令告之曰:「昔者吾國無主,以薛徹、太丑二人實我伯祖八剌哈之裔,欲立之。二人既已固辭,乃以汝火察兒為伯父聶坤之子,又欲立之,汝又固辭。然事不可中輟,復以汝按彈為我祖忽都剌之子,又欲立之,汝又固辭。於是汝等推戴吾為之主,初豈我之本心哉,不自意相迫至於如此也。三河,祖宗肇基之地,毋為他人所有。汝善事汪罕,汪罕性無常,遇我尚如此,況汝輩乎?我今去矣,我今去矣!」按彈等無一言。



 帝既遣使於汪罕,遂進兵虜弘吉剌別部溺兒斤以行。至班硃尼河,河水方渾,帝飲之以誓眾。有亦乞烈部人孛徒者,為火魯剌部所敗,因遇帝,與之同盟。哈撒兒別居哈剌渾山,妻子為汪罕所虜,挾幼子脫虎走,糧絕,探鳥卵為食,來會於河上。時汪罕形勢盛強,帝微弱,勝敗未可知,眾頗危懼。凡與飲河水者,謂之飲渾水,言其曾同艱難也。汪罕兵至,帝與戰於哈闌真沙陀之地,汪罕大敗,其臣按彈、火察兒、札木合等謀弒汪罕,弗克,往奔乃蠻。答力臺、把憐等部稽顙來降。帝移軍斡難河源,謀攻汪罕,復遣二使往汪罕,偽為哈撒兒之言曰:「我兄太子今既不知所在,我之妻孥又在王所,縱我欲往,將安所之耶?王儻棄我前愆,念我舊好,即束手來歸矣。」汪罕信之,因遣人隨二使來,以皮囊盛血與之盟。及至,即以二使為向導,令軍士銜枚夜趨折折運都山,出其不意,襲汪罕,敗之,盡降克烈部眾,汪罕與亦剌合挺身遁去。汪罕嘆曰:「我為吾兒所誤,今日之禍,悔將何及!」汪罕出走,路逢乃蠻部將,遂為其所殺。亦剌哈走西夏,日剽掠以自資。既而亦為西夏所攻走,至龜茲國,龜茲國主以兵討殺之。帝既滅汪罕,大獵於帖麥該川,宣布號令,振凱而歸。



 時乃蠻部長太陽罕心忌帝能,遣使謀於白達達部主阿剌忽思曰:「吾聞東方有稱帝者,天無二日,民豈有二王耶?君能益吾右翼,吾將奪其弧矢也。」阿剌忽思即以是謀報帝,居無何,舉部來歸。



 歲甲子,帝大會於帖麥該川,議伐乃蠻。群臣以方春馬瘦,宜俟秋高為言。皇弟斡赤斤曰:「事所當為,斷之在早,何可以馬瘦為辭?」別里古臺亦曰:「乃蠻欲奪我弧矢,是小我也,我輩義當同死。彼恃其國大而言誇,茍乘其不備而攻之,功當可成也。」帝悅,曰:「以此眾戰,何憂不勝。」遂進兵伐乃蠻,駐兵於建忒該山,先遣虎必來、哲別二人為前鋒。太陽罕至自按臺,營於沆海山,與蔑里乞部長脫脫、克烈部長阿憐太石、猥剌部長忽都花別吉,暨禿魯班、塔塔兒、哈答斤、散只兀諸部合,兵勢頗盛。時我隊中羸馬有驚入乃蠻營中者,太陽罕見之,與眾謀曰:「蒙古之馬瘦弱如此,今當誘其深入,然後戰而擒之。」其將火力速八赤對曰:「先王戰伐,勇進不回,馬尾人背不使敵人見之。今為此遷延之計,得非心中有所懼乎?茍懼之,何不令后妃來統軍也。」太陽罕怒,即躍馬索戰。帝以哈撒兒主中軍。時札木合從太陽罕來,見帝軍容整肅,謂左右曰:「乃蠻初舉兵,視蒙古軍若甗珝羔兒,意謂蹄皮亦不留。今吾觀其氣勢,殆非往時矣。」遂引所部兵遁去。是日,帝與乃蠻軍大戰至晡,禽殺太陽罕。諸部軍一時皆潰,夜走絕險,墜崖死者不可勝計。明日,餘眾悉降。於是朵魯班、塔塔兒、哈答斤、散只兀四部亦來降。已而復徵蔑里乞部,其長脫脫奔太陽罕之兄卜欲魯罕,其屬帶兒兀孫獻女迎降,俄復



 叛去。帝至泰寒寨,遣孛羅歡、沈白二人領右軍往平之。



 "



 歲乙丑,帝征西夏,拔力吉里寨,經落思城,大掠人民及其橐駝而還。



 元年丙寅,帝大會諸王群臣,建九斿白旗,即皇帝位於斡難河之源,諸王群臣共上尊號曰成吉思皇帝。是歲實金泰和之六年也。帝既即位,遂發兵復徵乃蠻。時卜欲魯罕獵於兀魯塔山,擒之以歸。太陽罕子屈出律罕與脫脫奔也兒的石河上。帝始議伐金。初,金殺帝宗親咸補海罕,帝欲復仇。會金降俘等具言金主璟肆行暴虐,帝乃定議致討,然未敢輕動也。



 二年丁卯秋,再征西夏,克斡羅孩城。是歲,遣按彈、不兀剌二人使乞力吉思。既而野牒亦納里部、阿里替也兒部,皆通使來獻名鷹。



 三年戊辰春,帝至自西夏。夏,避暑龍庭。冬,再徵脫脫及屈出律罕。時斡亦剌部等遇我前鋒,不戰而降,因用為向導。至也兒的石河,討蔑里乞部,滅之,脫脫中流矢死,屈出律奔契丹。



 四年己巳春,畏吾兒國來歸。帝入河西,夏主李安全遣其世子率師來戰,敗之,獲其副元帥高令公。克兀剌海城,俘其太傅西壁氏。進至克夷門,復敗夏師,獲其將嵬名令公。薄中興府,引河水灌之,堤決,水外潰,遂撤圍還。遣太傅訛答入中興,招諭夏主,夏主納女請和。



 五年庚午春,金謀來伐,築烏沙堡。帝命遮別襲殺其眾,遂略地而東。初,帝貢歲幣於金,金主使衛王允濟受貢於凈州。帝見允濟不為禮。允濟歸,欲請兵攻之。會金主璟殂,允濟嗣位,有詔至國,傳言當拜受。帝問金使曰:「新君為誰?」金使曰:「衛王也。」帝遽南面唾曰:「我謂中原皇帝是天上人做,此等庸懦亦為之耶?何以拜為!」即乘馬北去。金使還言,允濟益怒,欲俟帝再入貢,就進場害之。帝知之,遂與金絕,益嚴兵為備。



 六年辛未春,帝居怯綠連河。西域哈剌魯部主阿昔蘭罕來降,畏吾兒國主亦都護來覲。二月,帝自將南伐,敗金將定薛於野狐嶺,取大水濼、豐利等縣。金復築烏沙堡。秋七月,命遮別攻烏沙堡及烏月營,拔之。八月,帝及金師戰於宣平之會河川,敗之。九月,拔德興府,居庸關守將遁去。遮別遂入關,抵中都。



 冬月,襲金群牧監,驅其馬而還。耶律阿海降,入見帝於行在所。皇子術赤、察合臺、窩闊臺分徇云內、東勝、武、朔等州,下之。是冬,駐蹕金之北境。劉伯林、夾谷長哥等來降。



 七年壬申春正月,耶律留哥聚眾於隆安,自為都元帥,遣使來附。帝破昌、桓、撫等州。金將紇石烈九斤等率兵三十萬來援,帝與戰於貛兒觜,大敗之。秋,圍西京。金元帥左都監奧屯襄率師來援,帝遣兵誘至密谷口,逆擊之,盡殪。復攻西京,帝中流矢,遂撤圍。九月,察罕克奉聖州。冬十二月甲申,遮別攻東京不拔,即引去,夜馳還,襲克之。



 八年癸酉春,耶律留哥自立為遼王,改元元統。秋七月,克宣德府,遂攻德興府。皇子拖雷、駙馬赤駒先登,拔之。帝進至懷來,及金行省完顏綱、元帥高琪戰,敗之,追至北口。金兵保居庸。詔可忒、薄剎守之,遂趨涿鹿。金西京留守忽沙虎遁去。帝出紫荊關,敗金師於五回嶺,拔涿、易二州。契丹訛魯不兒等獻北口,遮別遂取居庸,與可忒、薄剎會。八月,金忽沙虎弒其主允濟,迎豐王珣立之。是秋,分兵三道:命皇子術赤、察合臺、窩闊臺為右軍,循太行而南,取保、遂、安肅、安、定、邢、洺、磁、相、衛、輝、懷、孟,掠澤、潞、遼、沁、平陽、太原、吉、隰,拔汾、石、嵐、忻、代、武等州而還;皇弟哈撒兒及斡陳那顏、拙赤鷿、薄剎為左軍,遵海而東,取薊州、平、灤、遼西諸郡而還;帝與皇子拖雷為中軍,取雄、霸、莫、安、河間、滄、景、獻、深、祁、蠡、冀、恩、濮、開、滑、博、濟、泰安、濟南、濱、棣、益都、淄、濰、登、萊、沂等郡。復命木華黎攻密州,屠之。史天倪、蕭勃迭率眾來降,木華黎承制並以為萬戶。帝至中都,三道兵還,合屯大口。是歲,河北郡縣盡拔,唯中都、通、順、真定、清、沃、大名、東平、德、邳、海州十一城不下。



 九年甲戌春三月,駐蹕中都北郊。諸將請乘勝破燕,帝不從,乃遣使諭金主曰:「汝山東、河北郡縣悉為我有,汝所守惟燕京耳。天既弱汝,我復迫汝於險,天其謂我何?我今還軍,汝不能犒師以弭我諸將之怒耶?」金主遂遣使求和,奉衛紹王女岐國公主及金帛、童男女五百、馬三千以獻,仍遣其丞相完顏福興送帝出居庸。夏五月,金主遷汴,以完顏福興及參政抹撚盡忠輔其太子守忠,留守中都。六月,金糺軍斫答等殺其主帥,率眾來降。詔三摸合、石抹明安與斫答等圍中都。帝避暑魚兒濼。秋七月,金太子守忠走汴。冬十月,木華黎征遼東,高州盧琮、金樸等降。錦州張鯨殺其節度使,自立為臨海王,遣使來降。



 十年乙亥春正月,金右副元帥蒲察七斤以通州降,以七斤為元帥。二月,木華黎攻北京,金元帥寅答虎、烏古倫以城降,以寅答虎為留守,吾也而權兵馬都元帥鎮之。興中府元帥石天應來降,以天應為興中府尹。三月,金御史中丞李英等率師援中都,戰於霸州,敗之。夏四月,克清、順二州。詔張鯨總北京十提控兵從南征,鯨謀叛,伏誅。鯨弟致遂據錦州,僭號漢興皇帝,改元興龍。五月庚申,金中都留守完顏福興仰藥死,抹撚盡忠棄城走,明安入守之。是月,避暑桓州涼涇,遣忽都忽等籍中都帑藏。秋七月,紅羅山寨主杜秀降,以秀為錦州節度使。遣乙職里往諭金主以河北、山東未下諸城來獻,及去帝號為河南王,當為罷兵,不從。詔史天倪南征,授右副都元帥,賜金虎符。八月,天倪取平州,金經略使乞住降。木華黎遣史進道等攻廣寧府,降之。是秋,取城邑凡八百六十有二。



 冬月,金宣撫蒲鮮萬奴據遼東,僭稱天王,國號大真,改元天泰。十一月,耶律留哥來朝,以其子斜闍入侍。史天祥討興州,擒其節度使趙守玉。



 十一年丙子春,還廬朐河行宮。張致陷興中府,木華黎討平之。秋,撒裡知兀鷿三摸合拔都魯率師由西夏趨關中,遂越潼關,獲金西安軍節度使尼龐古薄魯虎,拔汝州等郡,抵汴京而還。冬十月,薄鮮萬奴降,以其子帖哥入侍。既而復叛,僭稱東夏。



 十二年丁丑夏,盜祁和尚據武平,史天祥討平之,遂擒金將巢元帥以獻。察罕破金監軍夾谷於霸州,金求和,察罕乃還。秋八月,以木華黎為太師,封國王,將蒙古、糺、漢諸軍南征,拔遂城、蠡州。冬,克大名府,遂東定益都、淄、登、萊、濰、密等州。是歲,禿滿部民叛,命缽魯完、朵魯伯討平之。



 十三年戊寅秋八月,兵出紫荊口,獲金行元帥事張柔,命還其舊職。木華黎自西京入河東,克太原、平陽及忻、代、澤、潞、汾、霍等州。金將武仙攻滿城,張柔擊敗之。是年,伐西夏,圍其王城,夏主李遵頊出走西涼。契丹六哥據高麗江東城,命哈真、札剌率師平之;高麗王皞遂降,請歲貢方物。



 十四年己卯春,張柔敗武仙,降祁陽、曲陽、中山等城。夏六月,西域殺使者,帝率師親征,取訛答剌城,擒其酋哈只兒只蘭禿。秋,木華黎克岢嵐、吉、隰等州,進攻絳州,拔其城,屠之。



 十五年庚辰春三月,帝克蒲華城。夏五月,克尋思乾城,駐蹕也兒的石河。秋,攻斡脫羅兒城,克之。木華黎徇地至真定,武仙出降。以史天倪為河北西路兵馬都元帥、行府事,仙副之。東平嚴實籍彰德、大名、磁、洺、恩、博、滑、浚等州戶三十萬來歸,木華黎承制授實金紫光祿大夫、行尚書省事。冬,金邢州節度使武貴降。木華黎攻東平,不克,留嚴實守之,撤圍趨洺州,分兵徇河北諸郡。是歲,授董俊龍虎衛上將軍、右副都元帥。



 十六年辛巳春,帝攻卜哈兒、薛迷思乾等城,皇子術赤攻養吉干、八兒真等城,並下之。夏四月,駐蹕鐵門關,金主遣烏古孫仲端奉國書請和,稱帝為兄,不允。金東平行省事忙古棄城遁,嚴實入守之。宋遣茍夢玉來請和。夏六月,宋漣水忠義統轄石珪率眾來降,以珪為濟、兗、單三州總管。秋,帝攻班勒紇等城,皇子術赤、察合臺、窩闊臺分攻玉龍傑赤等城,下之。冬十月,皇子拖雷克馬魯察葉可、馬魯、昔剌思等城。木華黎出河西,克葭、綏德、保安、鄜、坊、丹等州,進攻延安,不下。十一月,宋京東安撫使張琳以京東諸郡來降,以琳為滄、景、濱、棣等州行都元帥。是歲詔諭德順州。



 十七年壬午春,皇子拖雷克徒思、匿察兀兒等城,還經木剌夷國,大掠之,渡搠搠闌河,克也裏等城。遂與帝會,合兵攻塔裏寒寨,拔之。木華黎軍克乾、涇、邠、原等州,攻鳳翔,不下。夏,避暑塔裏寒寨。西域主札闌丁出奔,與滅里可汗合,忽都忽與戰不利。帝自將擊之,擒滅里可汗。札闌丁遁去,遣八剌追之,不獲。秋,金復遣烏古孫仲端來請和,見帝於回鶻國。帝謂曰:「我向欲汝主授我河朔地,令汝主為河南王,彼此罷兵,汝主不從。今木華黎已盡取之,乃始來請耶?」仲端乞哀,帝曰:「念汝遠來,河朔既為我有,關西數城未下者,其割付我,令汝主為河南王,勿復違也。」仲端乃歸。金平陽公胡天祚以青龍堡降。冬十月,金河中府來附,以石天應為兵馬都元帥守之。



 十八年癸未春三月,太師國王木華黎薨。夏,避暑八魯彎川。皇子術赤、察合臺、窩闊臺及八剌之兵來會,遂定西域諸城,置達魯花赤監治之。冬十月,金主珣殂,子守緒立。是歲,宋復遣茍夢玉來。



 十九年甲申夏,宋大名總管彭義斌侵河北,史天倪與戰於恩州,敗之。是歲,帝至東印度國,角端見,班師。



 二十年乙酉春正月,還行宮。二月,武仙以真定叛,殺史天倪。董俊判官李全亦以中山叛。三月,史天澤擊仙走之,復真定。夏六月,彭義斌以兵應仙,天澤御於贊皇,擒斬之。



 二十一年丙戌春正月,帝以西夏納仇人亦喝翔昆及不遣質子,自將伐之。



 二月,取黑水等城。夏,避暑於渾垂山。取甘、肅等州。秋,取西涼府搠羅、河羅等縣,遂逾沙陀,至黃河九渡,取應裡等縣。九月,李全執張琳,郡王帶孫進兵圍全於益都。冬十一月庚申,帝攻靈州,夏遣嵬名令公來援。丙寅,帝渡河擊夏師,敗之。丁丑,五星聚見於西南。駐蹕鹽州川。十二月,李全降。授張柔行軍千戶、保州等處都元帥。是歲,皇子窩闊臺及察罕之師圍金南京,遣唐慶責歲幣於金。



 二十二年丁亥春,帝留兵攻夏王城,自率師渡河攻積石州。二月,破臨洮府。



 三月,破洮、河、西寧二州。遣斡陳那顏攻信都府,拔之。夏四月,帝次龍德,拔德順等州,德順節度使愛申、進士馬肩龍死焉。五月,遣唐慶等使金。閏月,避暑六盤山。六月,金遣完顏合周、奧屯阿虎來請和。帝謂群臣曰:「朕自去冬五星聚時,已嘗許不殺掠,遽忘下詔耶。今可布告中外,令彼行人亦知朕意。」是月,夏主李晛降。帝次清水縣西江。秋七月壬午,不豫。己丑,崩於薩里川哈老徒之行宮。臨崩謂左右曰:「金精兵在潼關,南據連山,北限大河,難以遽破。若假道於宋,宋、金世仇,必能許我,則下兵唐、鄧,直搗大梁。金急,必徵兵潼關。然以數萬之眾,千里赴援,人馬疲弊,雖至弗能戰,破之必矣。」言訖而崩,壽六十六,葬起輦穀。至元三年冬十月,追謚聖武皇帝。至大二年冬十一月庚辰,加謚法天啟運聖武皇帝,廟號太祖。在位二十二年。



 帝深沉有大略,用兵如神,故能滅國四十,遂平西夏。其奇勛偉跡甚眾,惜乎當時史官不備,或多失於紀載雲。



 戊子年。是歲,皇子拖雷監國。



\end{pinyinscope}