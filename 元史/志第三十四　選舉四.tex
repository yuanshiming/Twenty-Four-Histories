\article{志第三十四 選舉四}

\begin{pinyinscope}

 ○考課



 凡隨朝職官:至元六年格,一考升一等,兩考通升二等止。六部侍郎正四品,依舊例通理八十月,升三品。左右司郎中、員外郎、都事,考滿升二等。六部郎中、員外郎、主事,三十月考滿升一等,兩考通升二等。



 凡官員考數:省部定擬:從九品擬歷三任,升從八。正九品歷兩任,升從八。正八品歷三任,升從七。從七歷三任,呈省。正七歷兩任,升從六。從六品通歷三任,升從五。正六歷兩任,升從五。從五轉至正五,緣四品闕少,通歷兩任,須歷上州尹一任,方入四品。內外正從四品,通理八十月,升三品。



 凡取會行止:中統三年,詔置簿立式,取會各官姓名、籍貫、年甲、入仕次第。至元十九年,諸職官解由到省部,考其功過,以憑黜陟。大德元年,外任官解由到吏部,止於刑部照過,將各人所歷,立行止簿,就檢照定擬。



 凡職官回降:至元十九年,定江淮官已受宣敕,資品相應,例升二等遷去。江淮官員依舊於江淮任用。其已考滿者,並免回降。不及考者,例存一等。有出身未合入流品受宣者,任回,三品擬同六品,四品擬同七品,正從五品同正八品;受敕者,正從六品同從八品,七品、八品同正從九品,正從九品同提領案牘、巡檢。無出身及白身人受宣者,三品同七品,四品同八品,正從五品同正九品;受敕者,正從六品同從九品,七品、八品同提領案牘、巡檢,正從九品擬院務監當官。其上項有資品人員,再於接連福建、兩廣溪洞州郡任用,擬升一等。兩廣、福建,別議升轉。至元十四年,都省未注江淮官已前,創立官府,招撫百姓,實有勞績者,其見受職名,若應受宣者,三品同七品,四品、五品擬同八品;若應受敕者,正從六品同正從九品,其七品、八品擬同提控案牘、巡檢,正從九品擬同院務監當官。無出身不應敘白身人,其見受職名,應受宣者,三品同八品,四品、五品同九品;應受敕者,正從六品同提控案牘、巡檢,七品以下擬院務監當官。其上項人員,若再於接連福建、兩廣溪洞州郡任用,擬升一等。兩廣、福建,別議升轉。至元十四年已後,新收撫州郡、準上例定奪。前資不應又升二等遷去江淮官員,任回,擬定前資合得品級,於上例升二等,止於江淮遷轉,若於腹里任用,並依上例。七品以下,已歷三品、四品者,比附上項有出身未入流品人員例,從一高。前三件於見擬資品上增一等銓注。二十一年,詔:「軍官轉入民職,已受宣敕不曾之任者,擬自準定資品換授,從禮任月日為始,理算資考升轉。若先受宣敕已經禮任,資品相應者,通理月日升轉外,據驟升人員前任所歷月日除一考外,餘月日與後任月日依準定資品通理升轉,不及考者,擬自準定資品換授,從禮任月日為始,理算資考升轉。腹裏常調官,除資品相應者依例升轉外,有前資未應入流品受宣敕者,六品以下人員,照勘有無出身,依驗職事品秩,自受敕以後歷一考者,同江淮例定擬,不及考者,更升一等。五品以上人員,斟酌比附議擬,呈省據在前已經除授者,任回通理定奪。」



 凡吏屬年勞差等:至元六年,吏部呈:「省部譯史、通事,舊以一百二十月出職,今案牘繁冗,合以九十月為滿。」十九年,部擬:「行省通事、譯史、令史、宣使或經例革替罷,所歷月日不等,如元經省掾發去,不及一考者,擬令貼補;及一考之上者,比臺院令史出身例定奪。自行踏逐者,降一等敘,不及一考者,發還本省區用。宣慰司人吏,經省院發,不及一考者,擬貼補;及一考之上者,比部令史出身降一等定奪。自行踏逐者,又降一等;不及一考者,別無定奪。」二十年,省擬:「雲南行省極邊重地令譯史人等,六十月考滿。甘肅行省令譯史人等,六十五月考滿,本土人員,依舊例用。」二十五年,省準:「緬中行省令史,依雲南行省一體出身。」大德元年,省臣奏:「以省、臺、院諸衙門令譯史、通事、知印、宣使等,舊以九十月為滿,升遷太驟,今以一百二十月為滿,於應得職事內升用。又寫聖旨、掌奏事選法、應辦刑名文字必闍赤等,以八月折十月,今後毋令折算。」四年,制以諸衙門令譯史、宣使人等一百二十月為滿。部議:「遠方令譯史人等,甘肅、福建、四川於此發去,九十月為滿。兩廣、海北海南道於此發去,八十月滿,雲南省八十月滿。土人一百二十月滿。」都省議:「俱以九十月為考滿,土人依例一百二十月為滿。」至大元年,部議:「和林行省即系遠方,其人吏比四川、甘肅行省九十月出職。」二年,詔:「中外吏員人等,依世祖定制,以九十月滿,參詳,歷一百二十月已受除者,依大德十一年內制,外任減一資。所有詔書已後在選未曾除受,並見告滿之人,歷一百二十月者,合同四考理算,外任一資不須再減。」省擬:「以九十月為滿,餘有月日,後任理算。應滿而不離役者,雖有役過月日,不準。」三年,省準:「河西廉訪司書吏人等月日。」部議:「合準舊例,雲南六十月,河西、西川六十五月,土人九十月為滿。」皇慶二年,部議:「凡內外諸司吏員,舊以九十月為滿,大德元年改一百二十月為滿,至大二年復舊制。一紀之間,受除者眾。其元除有以三十月為一考者,亦有四十月為一考者,以所除不等,往往援例陳訴,有礙選法。擬合依已降詔絳為格,系大德元年三月七日以後入役,至未復舊制之前,已除未除俱以四十月為一考,通理一百二十月為滿,減資升轉。其未滿受除者,一體理考定擬,餘二十六月已上,準升一等,十五月之上,減外任一資,十五月之下,後任理算。改格之後應滿而不離役者,役過月日,別無定奪。」



 凡吏員考滿授從六品:至元九年,省準:「省令史出身,中統四年已前,六品升遷,已後七品除授,至元之後,事繁責重,宜依準中統四年已前考滿一體注授。」三十一年,省議:「三師僚屬,蒙古必闍赤、掾史、宣使等,依都省設置,若不由臺院轉補者,降等敘。」元貞元年,省議:「監修國史僚屬,依三師所設,非臺院轉補者,降等敘。」大德五年,部呈考滿省掾各各資品。省議:「今後院臺並行省令史選充省掾者,雖理考滿,須歷三十月方許出職,仍分省發、自行踏逐者,各部令史毋得直理省掾月日。」



 凡吏員考滿授正七品:至元九年,部擬:「院、臺、大司農司令史出身,三考正七品。一考之上,驗月日定奪。一考之下,二十月以上為從八品;十五月以上正九品;十五月以下,十月之上為從九品,添一資,歷十月以下為巡檢。」十一年,部議:「扎魯火赤令史、譯史考滿,合依樞密院、御史臺令史、譯史出身,三考出為正七品,自用者降一等,有闕於部令史內選取。」十四年,部擬:「前諸站統領使司令史,同部令史出身,今既改通政院從二品,通事、譯史、令史人等,宜同臺、院人吏一體出身。」十五年,翰林國史院言:「本院令史系省準人員,其出身與御史臺一體,遇闕省掾時,亦合勾補。準吏部牒,本院令史以九十月考滿,同部令史出身,本院與御史臺皆隨朝二品,令史亦合與臺令史一體出身,有闕於部令史內選用。」十九年,部擬:「泉府司隨朝從二品,令史、譯史人等,由省部發者,考滿依通政院例定奪,自行用者降一等。」二十年,定擬安西王王相府首領官令史,與臺、院吏屬一體遷轉。二十二年,部擬:「宣徽院升為二品,與臺、院品秩相同,令史出身合依正七品遷除貢補,省、院有闕,於部令史內選取。」總制院與御史臺俱為正二品,部擬:「令譯史考滿,亦合一體出身。」二十三年,省準:「詹事院掾史,若六部選充者,考滿出為正七品,自用者降等。」二十四年,集賢院言:「本院與翰林國史院品級相同。」省議:「令史考滿,一體定奪。」二十五年,省議:「上都留守司兼本路總管府令史出身,三考正八品,其自部令史內選取者,同宣徽院、太醫院令史一體出身。上都留守司升為正二品,見設令史,自行踏逐者,考滿不為例,從七品內選用;部令史內選取,考滿宣徽院、大司農司令史一體出身。」部議:「都護府人吏依通政院令譯史人等出身,由省部發者,考滿出為正七品,自用者降一等。」二十六年,省準:「都功德使司隨朝二品,令譯史人等,比臺、院人吏一體升轉。」二十九年,部呈:「大司徒令史,若各部選發者,三考出為正七,自用者降等。崇福司與都護府、泉府司品秩相同,所設人吏,由省部發者,考滿出為正七品,自用者降一等。福建省征瓜哇所設人吏,出征回還,俱同考滿。」三十年,省準:「將作院令史,依通政院等衙門令史,考滿除正七品。」部議:「如系六部選發,考滿除正七品,自用者本衙門敘。」元貞元年,內史府秩正二品,令史亦於部令史內收補,考滿除正七品,自用者降等。大德九年,部擬:「闊闊出大司徒令史,若各部選發,考滿正七,自用者降等。」至大四年,省準:「會福院令史、知印、通事、譯史、宣使、典吏俱自用,前擬不拘常調,考滿本衙門區用。隆禧院令史人等,如常選者,考滿依例遷敘,自用者不入常調,於本衙門區用。「皇慶二年,部議:「崇祥院人吏,系部令史發補者,依例遷用,不應者降等敘。」延祐四年,部議:「隆禧院令史、譯史、通事、知印、典吏同五臺殊祥院人吏一體,常選內委付。其出身若有曾歷寺監並籍記各部令史人等,考滿同二品衙門出身,降等敘,白身者降等,添一資升轉;省部發去者,依例遷敘。後有闕,令史須於常選教授儒人職官並部令史見役上名內取補;宣使於職官並相應內參補;通事、知印從長官保選,仍參用職官,違例補充,別無定奪。殊祥院人吏,先未定擬,亦合一體。」



 凡吏員考滿授從七品:至元六年,省擬:「部令史、譯史、通事人等,中統四年正月以前收補者,擬九十月為滿,注從七品,回降正八一任,還入從七。以後充者,亦擬九十月為滿,正八品,仍免回降。」九年,吏、禮部擬:「凡部令史二考,注從七品。一考之上,驗月日定奪。一考之下,二十月以上者正九品。十五月以上從九品,十五月以下,令史提控案牘,通事、譯史巡檢。太府監改擬正三品,與六部同,人吏自行踏逐,將已歷月日準為資考,似為不倫,擬自改升月日為始,九十月為滿,同部令史出職,有闕於籍記部令史內挨次收補。」十一年,省議:「省斷事官令史,與六部令史一體出身,若是實歷俸月九十月,考滿遷除,有闕於應補部令史人內挨次補用。」省議:「中御府正三品,擬同太府監令史出身,九十月於從七品內除授,自行踏逐者降一等,歇下名闕,於應補部令史人內補填。」十三年,省議:「行工部令史,與六部令史一體出身。四怯薛令史,九十月同部令史出身,有闕以籍記部令史內補填。」二十年,部呈:「行省令、譯史人等,比臺、院一體出身。行臺、行院令譯史、通事人等,九十月考滿,元系都省臺院發去及應補之人,合降臺院一等。」二十三年,省判:「大都留守司兼少府監令史,如系省部發去相應人員,同部令史出身,九十月考滿,從七品,自行踏逐者降等。」二十四年,省判:「中尚監令史人等,若系省部發去人員,同太府監令譯史等出身,自行踏逐者降等。」太史院令史,部議:「如省部發去人員,從七品內遷除,自行踏逐者,降等敘用。」部擬:「行省臺院令史,九十月考滿,若系都省臺院發去腹裏請俸人員,行省令史同臺院令史出身,行臺、行院降一等,俱於腹裏選用,自行踏逐遞降一等,於江南任用。」二十九年,省判:「鞏昌等處便宜都總帥府令史人等出身,擬與各道宣慰司一體,自行踏逐者降等敘用。」大德三年,省準:「上都留守司令史,舊以見役部令史發補,以籍居懸遠,擬於籍記部令史內選發,與六部見役令史一體轉升二品衙門令史,轉補不盡者,考滿從七品敘用。」八年,部擬:「利用監自大德三年八月已前入役者,若充各衙門有俸令史,及本監奏差、典吏轉補,則於應得資品內選用;由庫子、本把就升,並白身人,於雜職內通理定奪;自用之人,本監委用。」皇慶元年,制:「典瑞監人吏俱與七品出身。」部議:「太府、利用等四監同。省發者考滿與六部一體敘,其餘寺監令譯史正八品,奏差正九品。令典瑞監、前典寶監人吏出身同大府等監,系奉旨事理。」省議:「已除者,依舊例定奪。」三年,省準:「章慶使司秩正二品,見役人吏,若同隨朝二品衙門,考滿除正七品,緣系徽政院所轄司屬,量擬考滿除從七品,自用者降等,如系及考部令史轉充,考滿正七品,未及考者止除從七品。有闕須依例補,不許自用。」



 凡吏員考滿授正八品:至元十一年,省議:「秘書監從三品,令史擬九十月出為正八品,自用者降一等,有闕諸衙門考滿典吏內補填。」省議:「太常寺正三品,令史以九十月出為從八品,有闕於應補監令史內取用。」省議:「少府監正四品,準軍器監令史出身,是省部發去者,三考於正八品任用,自行踏逐人員,考滿降一等。」省議:「尚牧監正四品,省部發去令史,擬九十月出為正八品,自用者降一等,有闕於諸衙門典吏內選補。」部擬:「河南等路宣慰司系外任從二品,與隨朝各部正三品衙門相同,準令史以九十月同部令史遷轉。開元等路宣撫司外任正三品,令譯史比前例降一等,九十月於正八品內遷轉。」十四年,部擬:「樞密院斷事官令史,擬以九十月出為從八品,有闕於諸衙門考滿典吏內補用。」十六年,部擬:「樞密院斷事官今改從三品,所設人吏,若系上司發去人員,歷九十月,比省斷事官令史降等於正八品內遷除,自用者降一等,遇闕於相應人內發遣。」二十一年,部擬:「廣西、海北海南道宣慰司令史、譯史、奏差人等,與嶺南廣西道等處按察司書吏人等一體,二十月理算一考,擬六十月同考滿。」省準:「廣東宣慰司其地倚山瀕海,極邊煙瘴,令史議合優升,依泉州行省令譯史等,以二十月理算一考。」二十二年,省準:「詹事院府正、家令二司,給侍宮闈,正班三品,令史即非各司自用人員,俸秩與六部同,若遇院掾史有闕,於兩司令史內選補,擬定資品出身,依樞密院所轄各衛令史出身,考滿出為正八品。尚醖監令史,與六部令史同議,諸監令史考滿,正八品內遷用,及非省部發去者例降一等,尚醖監令史亦合一體。」二十三年,省準:「太常寺令史,歷九十月,正八品內任用,有闕於呈準籍記人內選取。雲南省羅羅斯宣慰司兼管軍萬戶府首領官、令史人等,依雲南行省令史例,六十月考滿,首領官受敕,例以三十月為一考。武備寺正三品,令譯史等出身,擬先司農寺令譯史人等,依各監例,考滿出為正八品,武備寺令史亦合依例遷敘。尚舍監令史,擬同諸寺監令史,考滿授正八品,自行用者降一等,尚舍監亦如之。陜西四川行省順元等路軍民宣慰司,依雲南令譯史人等,六十月為滿遷轉。」二十四年,部擬:「太史院、武備寺、光祿寺等令史,九十月正八品內遷用,自用者降一等。太醫院系宣徽院所轄,令史人等,若系省部發去,考滿同諸監令史,擬正八品,自用者降等任用。」二十六年,省準:「給事中兼修起居注人吏,依諸寺監令史出身例,考滿一體定奪。侍儀司令史,依給事中兼起居注人吏遷轉。」二十七年,省準:「延慶司令史,九十月,依已準家令、府正兩司例,由省部發者出為正八品,自用者降等敘。」二十八年,省準:「太僕寺擬比尚乘等寺令史,以九十月出為正八品,自用者降一等。拱衛直都指揮使司與武備寺同品,令史考滿,出為從八品,自用者降一等遷用。蒙古等衛令史,即系在先考滿令史,合於正八品內遷敘,各衛令史有闕,由省部籍記選發者,考滿出為正八品。樞密院所轄都元帥府、萬戶府各衛並屯田等司官吏,俱從本院定奪、遷調,見役令史,自用者考滿,合從本院定奪。宣政院斷事官令史,與樞密院及蒙古必闍赤,由翰林院發者,以九十月為從七品,通事、令史以九十月為正八品,奏差以九十月為正九品,典吏九十月轉本府奏差,自用者降等。」二十九年,部擬:「左右兩江宣慰司都元帥府令譯史人等,依雲南、兩廣、福建人吏,六十月為滿。兩廣敘用譯史,除從七品,非翰林院選發,別無定奪。令史省發,考滿正八品,奏差省發,考滿正九品,自用者降等敘。儀鳳司令史,比同侍儀司令史,考滿為正八品,自用者降一等。哈迷為頭只哈赤八剌哈孫達魯花赤令史,吏部議,與阿速拔都兒達魯花赤必闍赤考滿正八品任用,雖必闍赤、令史月俸不同,各官隨朝近侍一體,比依例出身相應。」三十年,省準:「孛可孫系正三品,令譯史人等,比依各寺監令譯史出身相應。都水監從三品,令譯史等依寺監令史一體出身,考滿正八品敘,自用者降等。只兒哈忽昔寶赤八剌哈孫達魯花赤本處隨朝正三品,與只哈赤八剌哈孫達魯花赤令史等即系一體,擬合依例,考滿出為正八品。」元貞元年,省準:「闌遺監令譯史人等,省部發去者,考滿正八品內任用,自行踏逐者降等。家令司、府正司改內宰、宮正,其人吏依元定為當。拱衛直都指揮使司升為正三品,其令譯史等俸,俱與光祿寺相同,擬系相應人內發補者考滿與正八品,奏差正九,自用者降等敘。」大德三年,部議:「鷹坊總管府人吏,依隨朝三品,考滿正八品內遷用。」五年,部擬:「和林宣慰司都元帥府人吏,合與隨朝二品衙門一體,及量減月日。」部議:「各道宣慰司令史,一百二十月正八品敘,自用者降等遷用。其和林宣慰司無應取司屬,又系酷寒之地,人吏已蒙都省從優以九十月為滿,今擬考滿,不分自用,俱於正八品內遷用。」八年,部言:「行都水監準設人吏,令史八人,奏差六人,壕寨一十人,通事、知印各一人,譯史一人,公使人二十人。都水監令譯史、通事、知印考滿,俱於正八品遷用,奏差考滿,正九品,自用者降等,壕寨出身並俸給同奏差。行都水監系江南創立衙門,令史比例,合於行省所轄常調提控案牘內選取,奏差、壕寨人等亦須選相應人,考滿比都水監人吏降等江南遷用,典吏公使人,從本監自用。」九年,部言:「尚乘寺援武備寺、大府、章佩等監例,求升加其人吏出身俸給。議得,各監人吏皆系奉旨升加,尚乘寺人吏合依已擬。」至大三年,部言:「和林系邊遠酷寒之地,兵馬司司吏歷一考餘,轉本路總管府司吏。補不盡者,六十月升都目。總管府吏,再歷一考,轉青海宣慰司令史,考滿除正八品,不系本路司吏轉補者,降等敘,補不盡者,六十月,部劄提控案牘內任用,蒙古必霝赤比上例定奪。」部議:「晉王位下斷事官正三品,除怯里馬赤、知印例從長官所保,蒙古必劄赤翰林院發,令史以內史府考滿典吏並籍記寺監令史發補,九十月除正八品,與職官相參用。奏差亦須選相應人,九十月依例遷用,自用者,考滿本衙門定奪。」皇慶元年,部言:「衛率府勾當人員,令都省與常選出身。議得,令史系軍司勾當之人,未有轉受民職定奪,合自奏準日為格,系皇慶元年二月九日以前者,同典牧監一體遷敘,以後者若系籍記寺監令史,常選提控案牘補充,依上銓除,自用者不入常調。」部議:「徽政院繕珍司見役令史,若系籍記寺監令史、常調提控案牘、院兩考之上典吏補充,內宰司令史例,考滿除正八,通事、譯史、知印亦依上遷敘,自用者降等。後有闕,須依例發補,違例補充,別無定奪。」二年,部議:「徽政院延福司見役令史,若系籍記寺監令史、常調提控案牘、本院兩考之上典吏補充者,依內宰司令史例,考滿除正八品,通事、譯史、知印依上遷敘,自用者降等。後有闕,須依例發補,不許自用。」延祐三年,省準:「徽政院所轄衛候司,奉旨升正三品,與拱衛直都指揮使司同品,合設令譯史,考滿除正八,自用者降等。衛候司就用前衛候司人吏,擬自呈準月日理算,考滿同自用遷敘,後有闕,以相應人補,考滿依例敘。徽政院掌飲司人吏,部議常選發補令譯史,考滿從八,奏差從九,自用者降等,後有闕須以相應人補,違例補充,考滿本衙門用。」四年,省準:「屯儲總管萬戶府司吏譯史出身,至大二年尚書省劄,和林路司吏未定出身,和林系邊遠酷寒去處,兵馬司司吏如歷一考之上,轉補本路司吏並總管府司吏,再歷一考之上,轉補青海宣慰司令史,考滿正八品遷除,補不盡人數,從優,擬六十月於部劄提控案牘內任用,蒙古必闍赤比依上例定奪。其沙州ˇ瓜州立屯儲總管萬戶府衙門,即系邊遠酷寒地面,依和林路總管府司吏人員一體出身。」



 凡吏員考滿授正九品:至元二十年,省準:「宮籍監系隨朝從五品,令史擬九十月正九品,例革人員,驗月日定奪,自行踏逐,降一等。」二十八年,省擬:「廉訪司所設人吏,擬選取書吏,止依按察司舊例,上名者依例貢部,下名轉補察院,貢補不盡人數,廉訪司月日為始理算,考滿者正九品敘,須令回避本司分治及元籍路分。」部議:「察院書吏出身,除見役人三十月,轉補不盡者,九十月出為從八品。察院書吏有闕,止於各道廉訪司書吏內選取,依上三十月轉部,九十月從八品。如非廉訪司書吏取充者,四十五月轉部,補不盡者,九十月考滿,降一等,出為正九品。」三十年,省準:「行臺察院書吏歷一考之上者,轉江南宣慰司令史,並內臺察院書吏,於見役人內用之。若有用不盡人數,以九十月出為正九品。江南有闕,依內臺察院書吏,於各道廉訪司書吏內選取,依例轉補。」大德四年,省擬:「各道廉訪司書吏,至元二十八年七月元定出身,上名貢部,下名轉補察院書吏。貢補不盡者,廉訪司為始理算月日,考滿正九品用。今議廉訪司先役書吏,歷九十月依已定出身,正九品注,任回,添一資升轉。大德元年三月七日已後充廉訪司人吏,九十月考滿,須歷提控案牘一任,於從九品內用。通事、譯史,比依上例。察院書吏,至元二十八年十二月元定出身,於各道廉訪司書吏內選取,三十月轉部,九十月從八品內用。如非廉訪司書吏取充者,四十五月轉部。補用不盡者,九十月考滿,降一等,正九品用。今議先役書吏,九十月依已定出身遷用,任回,添一資升轉。大德元年三月七日為始創入役者,止依舊例轉部。行臺察院書吏,至元三十年正月元定出身,於廉訪司書吏內選取,歷一考之上,轉補江南宣慰司令史、並內臺察院書吏,用不盡者,九十月正九品,江南用。省議先役書吏,歷俸九十月,依已定出身,任回,添一資升轉。大德元年三月七日為始創入者,止依舊例,轉補江南宣慰司令史,北人貢內臺察院。」



 凡吏員考滿除錢穀官、案牘、都吏目:至元十三年,吏、禮部言:「各路司吏四十五以下,以次轉補按察司書吏。補不盡者,歷九十月,於都目內任用;六十月以上,於吏目內任用。」省議:「上都、大都路司吏,難同其餘路分出身,依按察司書吏遷用。」十四年,省準:「覆實司司吏,俱授吏部劄付,如歷九十月,擬於中州都目內遷,若不滿考及六十月,於下州吏目內任用,有闕以相應人發充。」二十一年,省準:「諸色人匠總管府與少府監不同,又其餘相體管匠衙門人吏,俱未定擬出身,量擬比外路總管府司吏,考滿於都目內任用。」二十二年,省準:「大都等路都轉運使司令史,與河間等路都轉運鹽使司書吏出身同。外路總管府司吏三名,貢舉儒吏二名,貢不盡,年四十五之上,考滿都目內任用。」二十三年,省準:「各路司吏、轉運司書吏,年四十五以上,歷俸六十月充吏目,九十月充都目,餘有役過月日不用。奏差宜從行省斟酌月日,量於錢穀官內就便銓用。」省準:「覆實司系正五品,令史出身比交鈔提舉司司吏出身,九十月務使,六十月都監,六十月之下、四十五月之上都監添一界遷用,四十五月之下轉補運司令史。」部擬:「京畿漕運司司吏轉補察院書吏,不盡,四十五以上,九十月依例於都目內任用。」二十四年,部議:「各道巡行勸農官書吏,於各路總管府上名司吏內選取,考滿於提控案牘內任用,奏差從大司農司選委。」省準:「諸司局人匠總管府令史,於都目內任用。」二十五年,省準:「大護國仁王寺、昭應宮財用規運總管府令譯史人等,比大都路總管府正三品司吏,九十月提控案牘內任用。」部議:「甘肅、寧夏等處巡行勸農司系邊陲遠地,人吏依甘肅行省並河西隴北道提刑按察司,以二十二月準一考,六十五月為滿。」省準:「供膳司司吏,比覆實司司吏,九十月出身,於務使內任用。」二十六年,省準:「巡行勸農司書吏,役過路司吏月日,三折二準算,通理九十月,於提控案牘內遷敘。尚書省右司郎中、管領大都等路打捕民匠等戶總管令史,比依諸司局人匠總管府令史例,九十月,於都目內任用。」省準:「諸路寶鈔都提舉司司吏,有闕於諸路轉運司、漕運司上名司吏內選取,三十月充吏目,四十五月之上、六十月之下都目,六十月已上轉提控案牘,充寺監令史者聽。諸路寶鈔提舉司同。」奏準:「大都路都總管府添設司吏一十名,委差五名。司吏六十月,於提控案牘內任用,委差於近上錢穀官內委用,有闕以有根腳請俸人補充,不及考滿,不許無故替換。」二十七年,省準:「京畿都漕運司令史,九十月充提控案牘,年四十五之上,比依都提舉萬億庫司吏,願充寺監令史者聽。」二十九年,部擬:「大都路令史四十五以上,六十月提控案牘內任用,任回減一資升轉,四十五以下、六十月之上選舉貢部,每歲二名。奏差六十月,酌中錢穀官內任用。」省準:「京畿都漕運司令史,比依諸路寶鈔提舉司司吏出身例,三十月吏目,四十五月之上、六十月之下都目,六十月之上提控案牘。」三十年,省準:「提舉八作司系正六品,司吏四十五月之上吏目,六十月之上都目。」元貞元年,省準:「大都等路都轉運司令史,九十月提控案牘。」大德三年,省準:「諸路寶鈔提舉司、都提舉萬億四庫司吏,九十月提控案牘內任用,如六十月之上,自願告敘者,於都目內遷除,有闕於平準行用庫攢典內挨次轉補。」省準:「寶鈔總庫司、提舉富寧庫司俱系從五品,其司吏九十月,都目內任用。如六十月之上,自願告敘,於吏目內遷除。有闕須於在京五品衙門及左右巡院、大興、宛平二縣,及諸州司吏並籍記各部典吏內選。」省準:「提舉左右八作司吏,九十月都目內任用,六十月之上,自願告敘,於吏目內遷除,有闕於在都諸倉攢典內選補。京畿都漕運使司令史,六十月之上,於提控案牘內用,遇闕於路府諸州並在京五品等衙門上名司吏內選。大都路司吏改為令史,六十月之上,年及四十五以下,貢部不過二名,四十五以上,六十月提控案牘內遷用,任回減資升轉。大都路都總管府令史,依舊六十月,於提控案牘內遷敘,不須減資,有闕於府州兵馬司、左右巡院、大興、宛平二縣上名司吏內選補。」大德五年,省準:「河東宣慰使司軍儲所司吏、譯史,九十月為滿,譯史由翰林院發補,司吏由州縣司吏取充,與各路總管府譯史、司吏一體升轉,自用譯史,別無定奪,司吏除酌中錢穀官,委差近下錢穀官。」七年,部擬:「濟南、萊蕪等處鐵冶都提舉司及廣平、彰德等處鐵冶都提舉司秩四品,司吏九十月比散府上州例,升吏目。蒙古必闍赤擬酌中錢穀官,奏差近下錢穀官,典吏三考,轉本司奏差。」省準:「陜西省敘州等處諸部蠻夷宣撫司正三品,其令譯史考滿,比各路司吏人等一體遷用奏差,行省定奪。」九年,宣慰司大同等處屯儲軍民總管萬戶府從三品,司吏、譯史、委差人等,九十月為滿,司吏除酌中錢穀官,委差近下錢穀官。大德十年,省準:「諸路吏六十月,須歷五萬石之上倉官一界,升吏目,一考升都目,一考升中州案牘或錢穀官,通理九十月入流。五萬石之下倉官一界,升吏目,兩考都目,一考依上升轉。補不盡路吏,九十月升吏目,兩考升都目,依上流轉,如非州縣司吏轉補者,役過月日,別無定奪。」



 凡通事、譯史考滿遷敘:至元二年,部議:「雲南行省極邊重地,令譯史等人員,擬二十月為一考,歷六十月,準考滿敘用。」九年,省準:「省部臺院所設知印人等,所請俸給,元擬出身,俱在勾當官之上,既將勾當官升作從八品,其各部知印考滿,亦合升正八品,據例減知印除有前資人員,驗前資定奪,無前資者,各驗實歷月日,定擬遷敘。」二十年,各道按察司奏差、通事、譯史、奏差已有定例,通事九十月考滿,擬同譯史一體遷敘。部議:「行省、行臺、行院五品以下官員並首領官,亦合比依臺院例,一考升一等任用。據行省人吏比同臺院人吏出身,已有定例,行院、行臺令史、譯史、通事、宣使人等,九十月滿考,元系都省臺院發及應補者,擬降臺院一等定奪。」部擬:「甘肅行省令譯史、通事、宣使人等,量擬以六十五月遷敘,若系都省發去人員,如部議,自用者仍舊例。」二十一年,部議:「四川行省人吏,比甘肅行省所歷月日,一體遷除。」二十三年,部擬:「福建、兩廣行省令譯史、通事、宣使人等,擬歷六十月同考滿,止於江南遷用,若行省咨保福建、兩廣必用人員,於資品上升一等。」二十四年,部議:「行省、行臺、行院令史,九十月考滿,若系都省臺院發去腹裏相應人員,行省令史同臺院令史出身,行臺、行院降臺院一等,俱於腹里遷用,自用者遞降一等,止於江南任用。」二十七年,省議:「中書省蒙古必闍赤俱系正從五品遷除,今蒙古字教授擬比儒學教授例高一等,其必闍赤擬高省掾一等,內外諸衙門蒙古譯史,一體升等遷敘。」二十八年,部擬:「諸路寶鈔都提舉司蒙古必闍赤,三十月吏目,四十五月都目,六十月提控案牘,役過月日,擬於巡檢內敘用。奏差九十月,近上錢穀官,六十月,酌中錢穀官內任用。翰林院寫聖旨必闍赤,比依都省蒙古必闍赤內管宣敕者,八月算十月遷轉正六品。」部議:「寫聖旨必闍赤比依管宣敕蒙古必闍赤一體,亦合八折十準算月日外據出身已有定例。崇福司令譯史、知印,省部發補者,考滿出為正七品,自用者降一等。宣使省部發去者,考滿出為正八品,自用者降一等。各道廉訪司通事、譯史出身,比依書吏擬合一體考滿正九。奏差考滿,依通事、譯史降二等量擬,於省劄錢穀官並巡檢內任用。」三十年,省準:「將作院令譯史人等,由省部選發者,考滿正七品遷敘,自用者止從本衙門定奪。大都路蒙古必闍赤若系例後入役人員,擬六十月於巡檢內遷用,任回減一資升轉。」大德三年,省議:「各路譯史如系翰林院選發人員,九十月考滿。除蒙古人依準所擬外,其餘色目、漢人先歷務使一界,升提控一界,於巡檢內遷用。」省議:「大都運司通事比依本司令史,滿考者於巡檢內任用。」四年,省準:「雲南諸路廉訪司寸白通事、譯史出身,比依書吏出身,九十月為滿,歷巡檢一任,轉升從九品,雲南地面遷用。」七年,宣慰司奏差,除應例補者,一百二十月考滿,依例自行保舉者降等,任回,添資定奪任用。廉訪司通事、譯史,大德元年三月七日已後創入補者,九十月歷巡檢一任,轉從九,如書吏役九十月,充巡檢者聽,如違不準。各路譯史,如系各道提舉學校官選發腹裏各路譯史,九十月考滿,先歷務使一界升提領,再歷一界充巡檢,三考從九,違者雖歷月日,不準。會同館蒙古必闍赤,九十月務提領內遷用。十年,省準:「中政院寫懿旨必闍赤,依寫聖旨必闍赤一體出身。八番順元、海北海南宣慰司都元帥府極邊重地令譯史人等,考滿依兩廣、福建例,於江南遷用。」



 凡官員致仕:至元二十八年,省議:「諸職官年及七十,精力衰耗,例應致仕。今到選官員,多有年已七十或七十之上者,合令依例致仕。」大德七年,省臣言:「內外官員年至七十者,三品以下,於應授品級,加散官一等,令致仕。」十年,省臣言:「官員年老不堪仕宦者,於應得資品,加散官、遙授職事,令致仕。」皇慶二年,省臣言:「蒙古、色目官員所授散官,卑於職事,擬三品以下官員,職事、散官俱升一等,令致仕。」



 凡封贈之制:至元初,唯一二勛舊之家以特恩見褒,雖略有成法,未悉行之。至元二十年,制:「考課雖以五事責辦管民官,為無激勸之方,徒示虛文,竟無實效。自今每歲終考課,管民官五事備具,內外諸司官職任內各有成效者,為中考。第一考,對官品加妻封號。第二考,令子弟承廕敘仕。第三考,封贈祖父母、父母。品格不及封贈者,量遷官品,其有政績殊異者,不次升擢,仰中書參酌舊制,出給誥命。」至大二年,詔:「流官五品以上父母、正妻,七品以上正妻,令尚書省議行封贈之制。」禮部集吏部、翰林國史院、集賢院、太常等官,議封贈謚號等第,制以封贈非世祖所行,其令罷之。至治三年,省臣言:「封贈之制,本以激勸將來,比因泛請者眾,遂致中輟。」詔從新設法議擬與行,毋致冗濫。禮部從新分立等第:正從一品封贈三代,爵國公,勛正上柱國,從柱國母、妻並國夫人。正從二品封贈二代,爵郡公,勛正上護軍,從護軍,母、妻並郡夫人。正從三品封贈二代,爵郡侯,勛正上輕車都尉,從輕車都尉母、妻並郡夫人。正從四品封贈父母,爵郡伯,勛正上騎都尉,從騎都尉,母、妻並郡君。正五品封贈父母,爵縣子,勛驍騎尉,母、妻並縣君。從五品封贈父母,爵縣男,勛飛騎尉,母、妻並縣君。正從六品封贈父母,父止用散官,母、妻並恭人。正從七品封贈父母,父止用散官,母、妻並宜人。正從一品至五品宣授,六品至七品敕牒。如應封贈三代者,曾祖父母一道,祖父母一道,父母一道,生者各另給降。封贈者,一品至五品並用散官勛爵,六品七品止用散官職事,從一高。封贈曾祖,降祖一等,祖降父一等,父母妻並與夫、子同。父母在仕者不封,已致仕並不在仕者封之,雖在仕棄職就封者聽。父母應封,而讓曾祖父母、祖父母者聽。諸子應封父母,嫡母在,所生之母不得封。嫡母亡,得並封。若所生母未封贈者,不得先封其妻。諸職官曾受贓,不許申請,封贈之後,但犯取受之贓,並行追奪。其父祖元有官進一階,不在追奪之例。父祖元有官者,隨其所帶文武官上封贈,若已是封贈之官,止於本等官上許進一階,階滿者更不在封贈之限。如子官至四品,其父祖已帶四品上階之類。或兩子當封者,從一高。文武不同者,從所請。婦人因其夫、子封贈,而夫、子兩有官者,從一高。封贈曾祖母、祖母並母,生封並加太字,若已亡歿或曾祖、祖父、父在者,不加太字。職官居喪,應封贈曾祖父母、祖父母、父母者聽。其應受封之人,居曾祖父母、祖父母、父母、舅姑、夫喪者,服闋申請。應封贈者,有使遠死節,有臨陳死事者,驗事特議加封。應封妻者,止封正妻一人,如正妻已歿,繼室亦止封一人,餘不在封贈之例。婦人因夫、子得封者,不許再嫁,如不遵守,將所受宣敕追奪,斷罪離異。父祖曾任三品以上官,亡歿,生前有勛勞,為上知遇者,子孫雖不仕,具實跡赴所在官司保結申請,驗事跡可否,量擬封贈。無後者,許有司保結申請。曾祖父母、祖父母、父母曾犯十惡奸盜除名等罪,及例所封妻不是以禮娶到正室,或系再醮倡優婢妾,並不許申請。凡告請封贈者,隨朝並京官行省、行臺、宣慰司、廉訪司見任官,各於任所申請。其餘官員,見任並已除未任,至得替日,隨其解由申請。致仕官於所在官司申請。正從七品至正從六品,止封一次。升至正從五品,封贈一次。升至正從四品,封贈一次。升至正從三品,封贈一次。升至正從二品,封贈一次。升至正從一品,封贈一次。凡封贈流官父祖曾任三品以上者,許請謚。如立朝有大節,功勛在王室者,許加功臣之號。至治三年,詔:「封贈之典,本以激勸忠孝,今後散官職事勛爵,依例加授,外任官員並許在任申請,其餘合行事理,仰各依舊制。」泰定元年,詔:「犯贓官員,不得封贈,沉鬱既久,宜許自新,有能滌慮改過,再歷兩任無過者,許所管上司正官從公保明,監察御史、廉訪司覆察是實,並聽依例申請。」



\end{pinyinscope}