\article{志第三十六 百官二}

\begin{pinyinscope}

 樞密院,秩從一品,掌天下兵甲機密之務。凡宮禁宿衛,邊庭軍翼,征討戍守,簡閱差遣,舉功轉官,節制調度,無不由之。世祖中統四年,置樞密副使二員,僉書樞密事一員。至元七年,置同知樞密院事一員,院判一員。二十八年,始置知院一員,增院判一員,又以中書平章商量院事。大德十年,增置知院二員,同知五員,副樞五員,僉院五員,同僉三員,院判二員。至大三年,知院七員,同知二員,副樞二員,僉院一員,同僉一員,院判二員,革去議事平章。延祐四年,以分鎮北邊,增知院一員。五年,增同知一員。後定置知院六員,從一品;同知四員,正二品;副樞二員,從二品;僉院二員,正三品;同僉二員,正四品;院判二員,正五品;參議二員,正五品;經歷二員,從五品;都事四員,正七品;承發兼照磨二員,正八品;架閣庫管勾一員,正九品;同管勾一員,從九品;掾史二十四人,譯史一十四人,通事三人,司印二人,宣使一十九人,銓寫二人,蒙古書寫二人,典吏一十七人,院醫二人。



 客省使,秩從五品。大使二員,副使二員。至元十四年,置大使一員。十六年,增一員。二十一年,置副使一員。延祐五年,增一員。天歷元年,又增一員。尋定置大使二員,從五品;副使二員,從六品;令史二人。



 斷事官,秩正三品,掌處決軍府之獄訟。至元元年,始置斷事官二員。八年,增二員。十九年,又增一員。二十年,又增二員。大德十一年,又增四員。皇慶元年,省二員。後定置斷事官八員,正三品;經歷一員,從七品;令史六人,譯史一人,通事、知印、奏差、典吏各一人。



 行樞密院。國初有征伐之事,則置行樞密院。大征伐,則止曰行院。為一方一事而設,則稱某處行樞密院,或與行省代設,事已則罷。



 西川行樞密院,中統四年始置,設官二員,管四川軍民課稅交鈔、打捕鷹房人匠,及各投下應管公事,節制官吏諸色人等,並軍官遷授徵進等事。始置於成都。至元十年,又於重慶別置東川行樞密院,設官一員。十三年,並為一院,尋復分東川行院。十六年,罷兩川行院。二十八年,復立四川行院於成都。



 江南行樞密院。至元十年,罷河南省統軍司、漢軍都元帥、山東行院,置荊湖等路行院,設官三員;淮西行院,設官二員。掌調度軍馬之事。十二年,罷行院。十九年,詔於楊州、岳州俱立行院,各設官五員。二十一年,立沿江行院。二十二年,立江西行院,馬軍戍江州,步軍戍撫州。二十八年,徙岳州行院於鄂州,徙江淮行院於建康,其後行院悉並歸行省。



 甘肅行樞密院。至大四年,置行院於甘州,為甘肅等處行樞密院,設官四員,提調西路軍馬。後以甘肅省丞相提調,遂罷行院。



 河南行樞密院,致和元年分置,專管調遣之事。天歷元年罷。



 嶺北行樞密院,天歷二年置。知院一員,同知二員,副樞一員,僉院二員,同僉一員,院判二員,經歷一員,都事二員,蒙古必闍赤四人,掾史二人,怯里馬赤一人,知印一人,宣使四人。掌邊庭軍務,凡大小事宜,悉從裁決。



 右衛,秩正三品。中統三年,初置武衛。至元元年,改為侍衛。八年,改為左、右、中三衛,掌宿衛扈從,兼屯田。國有大事,則調度之。二十年,增都指揮使一員、副都指揮使一員。二十一年,置僉事二員。大德十一年,增都指揮使二員、副都指揮使一員。至大元年,增都指揮使三員、副都指揮使一員。四年,省都指揮使五員、副都指揮使二員。後定置都指揮使三員,正三品;副都指揮使二員,從三品;僉事二員,正四品;經歷二員,從七品;知事二員,照磨一員,俱從八品;令史七人,譯史、通事、知印各一人。又其屬十有五:



 鎮撫所,鎮撫二員。



 行軍千戶所十,秩正五品。達魯花赤十員,副達魯花赤十員,千戶十員,副千戶十員,彈壓二十員,百戶二百員,知事十員。



 弩軍千戶所一,秩正五品。達魯花赤一員,千戶一員,彈壓二員,百戶十員。



 屯田左右千戶所二,秩正五品。達魯花赤二員,千戶二員,彈壓二員,百戶四十員。



 教官二,蒙古字教授一員,儒學教授一員。掌諸屯衛行伍耕戰之暇,使之習學國字,通曉書記。初由樞府選舉,後歸吏部。



 左衛,秩正三品。至元八年,以侍衛改置。掌宿衛扈從,兼屯田。國有大事,則調度之。是年,增副指揮使一員。十六年,增副都指揮使一員。二十年,置僉事一員。二十二年,增僉事一員。二十四年,省都指揮使、副都指揮使一員。大德十一年,增都指揮使五員、副都指揮使二員、僉事二員。至大四年,省都指揮使六員、副都指揮使二員。其後定制,衛官:都指揮使二員,正三品;副都指揮使二員,從三品;僉事二員,正四品;經歷二員,從七品;知事二員,照磨一員,俱從八品;令史七人,譯史、通事、知印各一人。其屬十有五:



 鎮撫所,鎮撫二員。



 行軍千戶所凡十,秩正五品。達魯花赤十員,副達魯花赤十員,千戶十員,副千戶十員,彈壓二十員,百戶二百員,知事十員。



 弩軍千戶所一,秩正五品。達魯花赤一員,千戶一員,彈壓二員,百戶十員。



 屯田左右千戶所二,秩正五品。達魯花赤一員,千戶二員,彈壓二員,百戶四十員。



 教官二,蒙古字教授一員,儒學教授一員。



 中衛,秩正三品。至元八年,以侍衛改置。掌宿衛扈從,兼營屯田。國有大事,則調度之。是年,置都指揮使一員、副都指揮使一員。二十年,增副都指揮使一員。二十一年,置僉事二員。二十三年,增都指揮使一員。大德十一年,增都指揮使二員、副使三員。至大元年,增都指揮使一員。四年,省都指揮使三員、副都指揮使三員。其後定置都指揮使三員,正三品;副都指揮使二員,從三品;僉事二員,正四品;經歷二員,從七品;知事二員,承發架閣照磨一員,俱從八品;令史七人,譯史、通事、知印各一人。其屬十有五:



 鎮撫所,鎮撫二員。



 行軍千戶所十,秩正五品。達魯花赤十員,副達魯花赤十員,千戶十員,副千戶十員,彈壓二十員,百戶二百員,知事十員。



 弩軍千戶一,秩正五品。達魯花赤一員,千戶一員,彈壓二員,百戶十員。



 屯田左右千戶所二,秩正五品。達魯花赤二員,千戶二員,彈壓二員,百戶四十員。



 教官二,蒙古字教授一員,儒學教授一員。



 前衛,秩正三品。至元十六年,以侍衛親軍創置前、後二衛。掌宿衛扈從,兼營屯田。國有大事,則調度之。是年,置都指揮使一員、副都指揮使二員。十八年,增都指揮使二員。二十年,置僉事一員。大德十一年,增都指揮使五員、副都指揮使一員、僉事三員。至大四年,省都指揮使五員、副都指揮使一員、僉事三員。後定置衛官,都指揮使三員,正三品;副都指揮使二員,從三品;僉事二員,正四品;經歷一員,從七品;知事二員,承發架閣照磨一員,俱從八品;令史七人,譯史、通事、知印各一人。又其屬十有七:



 鎮撫所,鎮撫二員。



 行軍千戶所十,秩正五品。達魯花赤十員,副達魯花赤十員,千戶十員,副千戶十員,彈壓二十員,百戶二百員。



 弩軍千戶一,秩正五品。達魯花赤一員,千戶一員,彈壓二員,百戶十員。



 屯田千戶所二,秩正五品。達魯花赤二員,千戶二員,彈壓二員,百戶四十員。



 門尉二,平則門尉一員,順承門尉一員。



 教官二,蒙古字教授一員,儒學教授一員。



 後衛,秩正三品。至元十六年,以侍衛親軍創置。掌宿衛扈從,兼營屯田。國有大事,則調度之。是年,置都指揮使二員、副都指揮使二員,後增設副都指揮使一員。十八年,增都指揮使二員。二十年,置僉事二員。大德十一年,增都指揮使五員、副都指揮使一員、僉事二員。至大四年,省都指揮使五員、副指揮使二員、僉事二員。後定置都指揮使三員,正三品;副都指揮使二員,從三品;僉事二員,正四品;經歷二員,從七品;知事二員,照磨一員,俱從八品;令史七人,譯史二人,知印一人,通事二人。其屬十有四:



 鎮撫所,鎮撫二員。



 行軍千戶所十,秩正五品。達魯花赤十員,副達魯花赤十員,千戶十員,副千戶十員,彈壓二十員,百戶二百員。



 弩軍千戶所一,秩正五品。達魯花赤一員,千戶一員,彈壓二員,百戶十員。



 屯田千戶所一,秩正五品。達魯花赤一員,千戶二員,彈壓二員,百戶四十員。



 教官二,蒙古字教授一員,儒學教授一員。



 武衛親軍都指揮使司,秩正三品,掌修治城隍及京師內外工役,兼大都屯田等事。至元二十六年,樞密院以六衛六千人,大都屯田三千人,近路迤南萬戶府一千人,總一萬人,立武衛,設官五員。元貞、大德年間,累增都指揮使四員。至大三年,省都指揮使四員、副都指揮使一員。後定置衛官,達魯花赤一員,正三品;都指揮使三員,正三品;副都指揮使二員,從三品;僉事二員,正四品;經歷二員,從七品;知事二員,照磨一員,俱從八品;令史七人,譯史、通事、知印各一人。其屬十有五:



 鎮撫所,鎮撫二員。



 行軍千戶所七,秩正五品。達魯花赤七員,副達魯花赤七員,千戶七員,副千戶七員,百戶一百四十員,彈壓一十四員。



 屯田千戶所六,秩正五品。達魯花赤各一員,千戶六員,百戶六十員,彈壓六員。



 教官二,蒙古字教授一員,儒學教授一員。



 隆鎮衛親軍都指揮使司,秩正三品,掌屯軍徼巡盜賊於居庸關南、北口,統領欽察、阿速護軍三千六百九十三人,屯駐東西四十三處。皇慶元年,升隆鎮萬戶府為隆鎮衛,置都指揮使三員、副都指揮使二員、僉事二員。延祐二年,又以哈兒魯軍千戶所,並隸東衛。四年,置色目經歷一員。至治二年,置愛馬知事一員。後定置衛官,都指揮使三員,正三品;副指揮使二員,從三品;僉事二員,正四品;經歷二員,從七品;知事二員,承發兼照磨一員,俱從八品;令史七人,譯史、通事、知印各一人。其屬十有二:



 鎮撫所,鎮撫二員。



 北口千戶所,秩正五品。達魯花赤一員,千戶一員,百戶七員。於上都路龍慶州東口置司。



 南口千戶所,秩正五品。達魯花赤一員,千戶一員,百戶一員,彈壓一員。於大都路昌平縣居庸關置司。



 白羊口千戶所,秩正五品。達魯花赤一員,千戶一員,百戶二員,彈壓一員。於大都路昌平縣東口置司。



 碑樓口千戶所,秩正五品。達魯花赤一員,千戶一員,百戶一員,彈壓一員。於應州金城縣東口置司。



 古北口千戶所,秩正五品。達魯花赤一員,千戶一員,百戶六員,彈壓一員。於檀州北面東口置司。



 遷民鎮千戶所,秩正五品。達魯花赤一員,千戶一員,百戶六員,彈壓一員。於大寧路東口置司。



 黃花鎮千戶所,秩正五品。達魯花赤一員,千戶一員,百戶六員,彈壓一員。於昌平縣東口置司。



 蘆兒嶺千戶所,秩五品。達魯花赤一員,千戶一員,百戶六員,彈壓一員。於昌平縣本口置司。



 太和嶺千戶所,秩五品。達魯花赤一員,千戶一員,百戶六員,彈壓一員。於大同路昌邑縣本隘置司。



 紫荊關千戶所,秩五品。達魯花赤一員,千戶一員,百戶六員,彈壓一員。於易州易縣本隘置司。



 隆鎮千戶所,秩五品。達魯花赤一員,千戶一員,百戶八員,彈壓一員。於龍慶州北口置司。



 左右翼屯田萬戶府二,秩從三品,分掌斡端、別十八里回還漢軍,及大名、衛輝新附之軍,並迤東回軍,合為屯田。至元二十六年置。延祐五年,隸詹事院,並入衛率府。復改隸樞密院。定置兩府達魯花赤各一員,萬戶各一員,副萬戶各一員,經歷各一員,知事各一員,提控案牘各一員,令史各五人,屬官鎮撫各二員。



 千戶八所,達魯花赤八員,千戶八員,副千戶八員,百戶五十九員,彈壓一十六員。



 千戶四所,達魯花赤四員,千戶四員,副千戶四員,百戶五十二員,彈壓八員。



 左衛率府,秩正三品。至大元年,撥江南行省萬戶府精銳漢軍為東宮衛軍,立衛率府,設官十一員。延祐四年,始改為忠翊府,又改為御臨親軍指揮司,又以御臨非古典,改為羽林。六年,復隸東宮,仍為左衛率府。定置率使三員,正三品;副使二員,從三品;僉事二員,正四品;經歷一員,從七品;知事一員,照磨一員,俱從八品;令史七人,譯史、通事、知印各二人。其屬十有五:



 鎮撫所,鎮撫二員。



 行軍千戶所十,秩正五品。達魯花赤一員,千戶十員,副千戶十員,百戶二百員,彈壓一十員。



 弩軍千戶所一,秩正五品。達魯花赤一員,千戶一員,百戶十員,彈壓一員。



 屯田千戶所三,秩正五品。達魯花赤三員,千戶三員,百戶六十員,彈壓三員。



 教官三員,蒙古字教授一員,儒學教授一員,陰陽教授一員。



 右衛率府,秩正三品。延祐五年,以速怯那兒萬戶府、迤東女直兩萬戶府、右翼屯田萬戶府兵,合為右衛率府,置官十二員。後定置率使二員,正三品;副使二員,從三品;僉事二員,正四品;經歷二員,從七品;知事二員,照磨一員,俱從八品;令史七人,譯史、通事、知印各二人。其屬七:



 鎮撫所,鎮撫二員。



 千戶所五,秩正五品。千戶五員,百戶四十五員,彈壓二員。



 教官一,儒學教授一員。



 河南淮北蒙古軍都萬戶府,秩正三品。至元二十四年,以四萬戶奧魯赤改為蒙古軍都萬戶府,設府官四員、奧魯官四員。大德七年後,改為河南淮北蒙古軍都萬戶府。延祐五年,罷奧魯官、副鎮撫等員,定置都萬戶一員,正三品;副都萬戶一員,從三品;經歷一員,從七品;知事一員,提控案牘一員,俱從八品;令史七人,譯史、通事各一人。屬官鎮撫二員。



 八撒兒萬戶府,萬戶一員,副萬戶一員,經歷、知事、提控案牘各一員。鎮撫一員。



 千戶所一十翼,達魯花赤一十員,千戶十員,副千戶十戶,百戶七十三員,彈壓一十員。



 札忽兒臺萬戶府,萬戶一員,經歷、知事、提控案牘各一員,鎮撫一員。



 千戶所七翼,千戶七員,百戶三十八員,彈壓七員。



 脫烈都萬戶府,萬戶一員,副萬戶一員,經歷一員,知事一員,提控案牘一員,鎮撫一員。



 千戶所九翼,千戶九員,百戶六十二員,彈壓九員。



 和尚萬戶府,萬戶一員,副萬戶一員,經歷一員,知事、提控案牘各一員,鎮撫一員。



 千戶所六翼,達魯花赤四員,千戶六員,副千戶四員,百戶四十七員,彈壓六員。



 砲手千戶所一翼,千戶一員,百戶六員,彈壓一員。



 哨馬千戶所一翼,達魯花赤一員,千戶一員,副千戶一員,彈壓二員,百戶九員,奧魯官二員。



 右阿速衛親軍都指揮使司,秩正三品,掌宿衛城禁,兼營潮河、蘇沽兩川屯田,供給軍儲。至元九年,初立阿速拔都達魯花赤,置屬官。二十三年,遂名為阿速之軍。至大二年,改立右阿速衛親軍都指揮使司,置達魯花赤三員、都指揮使三員、副都指揮使二員、僉事二員。四年,省達魯花赤三員。後定置達魯花赤一員,正三品;都指揮使三員,正三品;副都指揮使二員,從三品;僉事二員,正四品;經歷二員,從七品;知事二員,承發架閣照磨一員,從八品;令史七人,譯史、通事、知印各一人,鎮撫二員。其屬五:



 行軍千戶所,千戶七員,百戶九員。



 把門千戶二員,百戶五員,門尉一員。



 本投下達魯花赤一員,長官一員,副長官一員。



 廬江縣達魯花赤一員,主簿一員。



 教官,儒學教授一員。



 左阿速衛親軍都指揮使司,品秩職掌同右阿速衛。至元九年,初立阿速拔都達魯花赤,置屬官。二十三年,遂名為阿速之軍。至大二年,改立左衛阿速親軍都指揮使司,置達魯花赤二員、都指揮使六員、副都指揮使四員、僉事二員。四年,省達魯花赤一員、都指揮使三員。後定置達魯花赤一員,都指揮使三員,副都指揮使二員,僉事二員,經歷二員,知事二員,照磨一員,鎮撫二員。其屬四:



 本投下達魯花赤二員,長官二員。



 鎮巢縣達魯花赤二員,主簿一員。



 圍宿把門千戶所一十三翼,千戶二十六員,百戶一百三十員,彈壓一十三員。



 教官,儒學教授一員。



 回回砲手軍匠上萬戶府,秩正三品。至元十一年,置砲手總管府。十八年,始立為都元帥府。二十二年,改為萬戶府。後定置達魯花赤一員,萬戶一員,副萬戶一員,經歷、知事、提控案牘各一員,令史四人,譯史一人。鎮撫二員。



 千戶所三翼,達魯花赤三員,千戶三員,副千戶三員,百戶三十二員,彈壓三員。



 唐兀衛親軍都指揮使司,秩正三品。總領河西軍三千人,以備征討。至元十八年始立,置都指揮使二員、副都指揮使二員。二十二年,增都指揮使一員、僉事一員。大德五年,增指揮使二員。至大元年,增都指揮一員。四年,省都指揮使三員,副都指揮使一員。後定置都指揮使三員,正三品;副都指揮使二員,從三品;僉事二員,正四品;經歷一員,從七品;知事一員,照磨一員,俱從八品;令史七人,通事、譯史、知印各一人,鎮撫二員,奧魯官正副各一員。



 千戶所九翼,正千戶九員,副千戶九員,百戶七十五員,彈壓九員,奧魯官正副各九員。



 門尉三,建德門一,和義門一,肅清門一。



 教官二,儒學教授一員,蒙古字教授一員。



 貴赤衛親軍都指揮使司,秩正三品。至元二十四年立,置都指揮使二員、副都指揮二員、僉事二員。二十九年,置達魯花赤一員。大德十一年,增達魯花赤一員、都指揮使四員、副都指揮一員。至大元年,省達魯花赤一員、都指揮使四員、副都指揮使三員。後定置達魯花赤一員,正三品;都指揮使二員,正三品;副都指揮使二員,從三品;僉事二員,正四品;經歷二員,從七品;知事二員,照磨一員,令史七人,知印一人,通事、譯史各一人,鎮撫二員。



 千戶所八翼,每所置達魯花赤一員,千戶一十六員,百戶八十員,彈壓八員,門尉二員。##!#延安屯田打捕總管府,秩從三品。管析居放良人戶,並兀里吉思田地北來蒙古人戶。至元十八年始設,定置達魯花赤一員,總管一員,同知一員,經歷、知事各一員。屬官打捕屯田官一十二員。



 大寧海陽等處屯田打捕所,秩從七品,掌北京、平灤等路析居放良不蘭奚等戶。至元二十二年,置總管府。元貞元年,罷總管府,置打捕所。定置達魯花赤一員,長官一員。教官,蒙古字教授一員,儒學教授一員。



 忠翊侍衛親軍都指揮使司,秩正三品。至元二十九年,始立屯田府。大德十一年,增軍數,立為大同等處指揮使司。至大四年,屬徽政院。延祐元年,改中都威衛使司,仍隸徽政院,尋復改屬樞密院。至治元年,改為忠翊侍衛。後定置都指揮使三員,正三品;副都指揮使二員,從三品;僉事二員,正四品;經歷二員,從七品;知事二員,照磨一員,俱從八品;令史七人,譯史、通事、知印各一人,鎮撫二員。



 行軍千戶所一十翼,達魯花赤一十員,副達魯花赤一十員,千戶一十員,副千戶一十員,百戶二百六員,彈壓二十員。



 弩軍千戶所一翼,達魯花赤一員,千戶一員,百戶一十員,彈壓一十員。



 屯田左右手千戶所二翼,達魯花赤二員,千戶二員,百戶四十員,彈壓四員。



 西域親軍都指揮使司,秩正三品。元貞元年始立,設官十一員。大德十一年,增都指揮使二員,又增指揮使三員、副都指揮使二員、僉事二員。至大四年,省都指揮使五員、副都指揮使二員、僉事二員。後定置達魯花赤一員,正三品;都指揮使二員,正三品;副都指揮使二員,從三品;僉事二員,正四品;經歷二員,從七品;知事二員,承發架閣兼照磨一員,並從八品;令史七人,通事、譯史、知印各一人,鎮撫二員。



 行軍千戶所,千戶一十三員,百戶二十九員。



 把門千戶二員,百戶八員,門尉一員。



 教官,儒學教授一員。



 宗仁蒙古侍衛親軍都指揮使司,秩正三品。至治二年,以亦乞列思人氏二百戶,與所收蒙古子女通三千戶,及清州匠二千戶,屯田漢軍二千戶,立宗仁衛以統之。定置都指揮使三員,正三品;副都指揮使二員,從三品;僉事二員,正四品;經歷二員,從七品;知事二員,照磨一員,俱從八品;令史七人,知印二人,怯里馬赤二人,譯史二人,鎮撫二員。



 蒙古軍千戶所一十翼,千戶二十員,百戶一百員,彈壓一十員。



 屯田千戶所,千戶四員,百戶四十員,彈壓四員。



 教官二,儒學教授一員,蒙古字教授一員。



 山東河北蒙古軍大都督府,秩從二品,掌各路軍民科差征進,及調遣總攝軍馬公事。至元二十一年,罷統軍司都元帥府,立蒙古軍都萬戶府。大德七年,改山東河北蒙古軍都萬戶府。延祐五年罷。天歷二年,改立為大都督府。定置正官大都督三員,從二品;同知一員,從三品;副使一員,從四品;經歷一員,從六品;都事二員,從七品;承發兼照磨一員,正八品;令史八人,譯史、通事、知印各二人,宣使五人,典吏三人,鎮撫二員。



 左手萬戶府,萬戶一員,副萬戶一員,經歷一員,知事一員,提控案牘各一員,鎮撫一員。



 千戶九翼,千戶一十一員,百戶七十四員,彈壓一十一員。



 右手萬戶府,萬戶一員,副萬戶一員,經歷一員,知事一員,提控案牘一員,鎮撫一員。



 千戶九翼,千戶九員,百戶六十三員,彈壓九員。



 拔都萬戶府,達魯花赤一員,萬戶一員,副萬戶一員,經歷一員,知事一員,提控案牘一員,鎮撫一員。



 千戶六翼,千戶七員,百戶四十一員,彈壓五員。



 哈答萬戶府,達魯花赤一員,萬戶一員,經歷一員,知事一員,提控案牘一員,鎮撫一員。



 千戶八翼,千戶八員,百戶二十四員,彈壓八員。



 蒙古回回水軍萬戶府,達魯花赤一員,萬戶一員,副萬戶一員,經歷、知事、提控案牘各一員,鎮撫二員。



 千戶八翼,達魯花赤二員,千戶六員,百戶四十六員,彈壓九員。



 都哥萬戶府,初隸都府七千戶翼,延祐三年樞密院奏,改立萬戶府。達魯花赤一員,萬戶一員,副萬戶一員,經歷、知事、提控案牘各一員,鎮撫二員。



 千戶七翼,千戶九員,百戶三十五員,彈壓八員。



 哈必赤千戶翼,千戶一員,百戶四員,彈壓一員,直隸大都督府。



 洪澤屯田千戶趙國宏翼,達魯花赤一員,千戶一員,副千戶一員,百戶一十四員,彈壓二員,直隸大都督府。



 左翊蒙古侍衛親軍都指揮使司,秩正三品。至元十八年,以蒙古侍衛總管府依五衛之例,為指揮使司,設官十二員,奧魯官二員。大德七年,奏改為左翼蒙古侍衛親軍都指揮使司。延祐五年,罷奧魯官。後定置司官,都指揮使三員,正三品;副都指揮使二員,從三品;僉事二員,正四品;經歷二員,從七品;知事二員,承發架閣兼照磨一員,並從八品;令史七人,譯史、通事、知印各一人,典吏二人,鎮撫二員。



 千戶所七翼,正千戶七員,副千戶七員,知事七員,彈壓七員,百戶六十二員。



 教官二,蒙古字教授一員,儒學教授一員。



 右翊蒙古侍衛親軍都指揮使司,品秩同左衛。至元十八年,以蒙古侍衛總管府依五衛例,為指揮使司,設官犬二員,奧魯官二員。大德七年,奏改為右翊蒙古侍衛親軍都指揮使司。延祐五年,罷奧魯官。後定置司官,都指揮使三員,正三品;副都指揮使二員,從三品;僉事二員,正四品;經歷二員,從七品;知事二員,承發兼照磨架閣一員,並從八品;令史七人,譯史、通事、知印各一人,典吏二人,鎮撫二員。



 千戶所一十二翼,正千戶一十二員,副千戶一十二員,知事一十二員,彈壓一十二員,百戶一百九員。



 教官,蒙古字教授一員,儒學教授一員。



 虎賁親軍都指揮使司,秩正三品,管領上都路元籍軍人,兼奧魯之事。至元十六年,立虎賁軍,設官二員。十七年,置都指揮使二員、副都指揮使一員,又增置副都指揮使一員。元貞元年,以虎賁軍改為虎賁親軍都指揮使司。十一年,增置都指揮使六員。至大四年,省都指揮使九員。後定置司官,都指揮使三員,正三品;副都指揮使二員,從三品;僉事二員,正四品;經歷一員,從七品;知事、照磨兼承發各一員,並從八品;令史七人,譯史、通事、知印各一人,典吏二人,鎮撫二員,都目一員。



 撒的赤千戶翼,正達魯花赤一員,副達魯花赤一員,正千戶一員,副千戶一員,知事一員,百戶二十員,彈壓二員。



 不花千戶翼,正達魯花赤一員,副達魯花赤一員,正千戶一員,副千戶一員,百戶二十二員,彈壓二員。



 脫脫木千戶翼,正達魯花赤一員,副達魯花赤一員,正千戶一員,副千戶一員,知事一員,百戶二十八員,彈壓二員。



 大忽都魯千戶翼,正達魯花赤一員,副達魯花赤一員,正千戶一員,副千戶一員,知事一員,百戶二十四員,彈壓二員。



 楊千戶翼,正達魯花赤一員,副達魯花赤一員,正千戶一員,副千戶一員,知事一員,百戶二十二員,彈壓二員。



 迷裏火者千戶翼,正達魯花赤一員,副達魯花赤一員,正千戶一員,副千戶一員,知事一員,百戶二十員,彈壓二員。



 大都督府,正二品,管領左右欽察兩衛、龍翊侍衛、東路蒙古軍元帥府、東路蒙古軍萬戶府、哈剌魯萬戶府。天歷二年,始立欽察親軍都督府,秩從二品。後改大都督府。置大都督三員,正二品;同知二員,正三品;副都督三員,從三品;僉都督事二員,正四品;經歷二員,從六品;都事二員,從七品;管勾一員,照磨一員,俱正八品;令史八人,蒙古必闍赤二人,怯里馬赤二人,知印二人,宣使六人。



 右欽察衛,秩正三品。至元二十三年,依河西等衛例,立欽察衛,設官十員。至治二年,分為左右衛。天歷二年,撥隸大都督府。定置達魯花赤一員,正三品;都指揮二員,正三品;副使二員,從三品;僉事二員,正四品;經歷二員,從七品;知事二員,照磨二員,並從八品;令史七人,譯史、通事、知印各一人,鎮撫一員。



 行軍千戶十八所,達魯花赤各一員,千戶三十六員,百戶一百八十員,彈壓一十八員。



 屯田千戶所二,達魯花赤二員,千戶二員,百戶二十員,彈壓二員。



 門尉二員。



 儒學教授一員,至大四年始置;蒙古字教授一員,延祐四年始置。



 左欽察衛,秩正三品。至治二年,依阿速衛例,分為兩衛,設官十員。天歷二年,撥隸大都督府。定置衛官,都指揮使三員,正三品;副都指揮二員,從三品;僉事二員,正四品;經歷二員,從七品;知事二員,照磨一員,從八品;令史七人,譯史、通事、知印各一人,屬官鎮撫二員。



 行軍千戶所一十翼,千戶一十員,百戶八十二員,彈壓九員,奧魯官四員。



 守城千戶所一翼,達魯花赤一員,千戶一員,百戶九員,彈壓一員。



 屯田千戶所一翼,達魯花赤一員,千戶一員,百戶十員,彈壓一員。



 教官,儒學教授一員。



 龍翊侍衛親軍都指揮使司,秩正三品。天歷元年始立,設官十四員。二年,又置愛馬知事一員,又以左欽察衛唐吉失九千戶隸本衛。定置官,都指揮使三員,正三品;副都指揮使二員,從三品;僉事二員,正四品;經歷一員,從七品;知事二員,照磨一員,並從八品;令史七人,譯史二人,怯里馬赤二人,知印二人,鎮撫二員。



 行軍千戶所九翼,達魯花赤一員,千戶六員,副千戶一員,百戶四十五員,彈壓五員。



 屯田一翼欽察千戶所,達魯花赤一員,千戶一員,百戶二十二員,彈壓二員。



 教官二,蒙古字教授一員,儒學教授一員。



 哈剌魯萬戶府,掌守禁門等處應直宿衛。至元二十四年,招集哈剌魯軍人,立萬戶府。尋移屯襄陽,後征交趾。大德二年置司南陽。天歷二年,奏隸大都督府。定置官,達魯花赤一員,萬戶一員,經歷、知事各一員,提控案牘一員,鎮撫一員,吏目一員。



 千戶所三翼,千戶三員,百戶九員,彈壓三員。



 御史臺,秩從一品。大夫二員,從一品;中丞二員,正二品;侍御史二員,從二品;治書侍御史二員,正三品,掌糾察百官善惡、政治得失。至元五年,始立臺建官,設官七員。大夫從二品,中丞從三品,侍御史從五品,治書侍御史從六品,典事從七品,檢法二員,獄丞一員。七年,改典事為都事。十九年,罷檢法、獄丞。二十一年,升大夫為從一品,中丞為正三品,侍御史為正五品,治書為正六品。二十七年,大夫以下品從各升一等,始置蒙古經歷一員。大德十一年,升中丞為正二品,侍御史為從二品,治書侍御史為正三品。皇慶元年,增中丞為三員。二年,減一員。至治二年,大夫一員。後定置御史大夫二員、中丞二員、侍御史二員、治書侍御史二員,品秩如上;經歷一員,從五品;都事二員,正七品;照磨一員,正八品;承發管勾兼獄丞一員,正八品;架閣庫管勾兼承發一員,正九品;掾史一十五人,譯史四人,知印二人,通事二人,宣使十人,臺醫二人,蒙古書寫二人,典吏六人,庫子二人。其屬有二:



 殿中司,殿中侍御史二員,正四品。至元五年始置,秩正七品,後升正四品。凡大朝會,百官班序,其失儀失列,則糾罰之;在京百官到任假告事故,出三日不報者,則糾舉之;大臣入內奏事,則隨以入,凡不可與聞之人,則糾避之。知班四人,通事、譯史各一人。



 察院,秩正七品,監察御史三十二員,司耳目之寄,任刺舉之事。至元五年,始置御史十一員,悉以漢人為之。八年,增置六員。十九年,增置一十六員,始參用蒙古人為之。至元二十二年,參用南儒二人。書吏三十二人。



 江南諸道行御史臺,設官品秩同內臺。至元十四年,始置江南行御史臺於楊州,尋徙杭州,又徙江州。二十三年,遷於建康,以監臨東南諸省,統制各道憲司,而總諸內臺。初置大夫、中丞、侍御史、治書侍御史各一員,統淮東、淮西、湖北、浙東、浙西、江東、江西、湖南八道提刑按察司。十五年,增江南湖北、嶺南廣西、福建廣東三道。二十三年,以淮東、淮西、山南三道,撥隸內臺。三十年,增海北海南一道。大德元年,定為江南諸道行御史臺,設官九員,以監江浙、江西、湖廣三省,統江東、江西、浙東、浙西、湖南、湖北、廣東、廣西、福建、海南十道。大夫一員,中丞二員,侍御史二員,治書侍御史二員,經歷一員,都事二員,照磨一員,架閣庫管勾一員,承發管勾兼獄丞一員,令史一十六人,譯史四人,回回掾史、通事、知印各二人,宣使十人,典吏、庫子、臺醫各有差。



 察院,品秩如內察院。至元十四年,置監察御史十員,書吏十員。二十三年,增蒙古御史十四員、書吏十四人,又增漢人御史四員、書吏四人。後定置御史二十八員、書吏二十八人。



 陜西諸道行御史臺,設官品秩同內臺。至元二十七年,始置雲南諸路行御史臺,官止四員。大德元年,移雲南行臺於京兆,為陜西行臺,而雲南改立廉訪司。延祐元年罷。二年復立,統漢中、隴北、四川、雲南四道。定置大夫一員、御史中丞二員、侍御史二員、治書侍御史二員、經歷一員、都事二員、照磨一員、架閣庫管勾一員、承發司管勾兼獄丞一員、掾史一十二人、蒙古必闍赤二人、回回掾史一人、通事二人、知印一人、宣使十人、典吏五人、庫子二人。



 察院,品秩同內察院。監察御史二十員,書吏二十人。



 肅政廉訪司。國初,立提刑按察司四道:曰山東東西道,曰河東陜西道,曰山北東西道,曰河北河南道。至元六年,以提刑按察司兼勸農事。八年,置河東山西道、陜西四川道。十二年,分置燕南河北道。十三年,以省並衙門,罷按察司。十四年復置,增立八道:曰江北淮東道,曰淮西江北道,曰山南江北道,曰浙東海右道,曰江南浙西道,曰江東建康道,曰江西湖東道,曰嶺北湖南道。十五年,復增三道:曰江南湖北道,曰嶺南廣西道,曰福建廣東道。十九年,增西蜀四川道。二十年,增海北廣東道,改福建廣東道曰福建閩海道。以雲南七路,置雲南道。以女直之地,置海西遼東道。二十三年,以淮東、淮西、山南三道,撥隸內臺。二十四年,增河西隴右道。是年,罷雲南道。二十五年,罷海西遼東。二十七年,以雲南按察司所治,立雲南行御史臺。二十八年,改按察司曰肅政廉訪司。大德元年,徙雲南行臺於陜西,復立雲南道。三十年,增海北海南道,其後遂定為二十二道。每道廉訪使二員,正三品;副使二員,正四品;僉事四員,兩廣、海南止二員,正五品;經歷一員,從七品;知事一員,正八品;照磨兼管勾一員,正九品;書吏十六人,譯史、通事各一人,奏差五人,典吏二人。



 內道八,隸御史臺:



 山東東西道,濟南路置司。



 河東山西道,冀寧路置司。



 燕南河北道,真定路置司。



 江北河南道,汴梁路置司。



 山南江北道,中興路置司。



 淮西江北道,廬州路置司。



 江北淮東道,楊州路置司。



 山北遼東道,大寧路置司。



 江南十道,隸江南行臺:



 江東建康道,寧國路置司。



 江西湖東道,龍興路置司。



 江南浙西道,杭州路置司。



 浙東海右道,婺州路置司。



 江南湖北道,武昌路置司。



 嶺北湖南道,天臨路置司。



 嶺南廣西道,靜江府置司。



 海北廣東道,廣州路置司。



 海北海南道,雷州路置司。



 福建閩海道,福州路置司。



 陜西四道,隸陜西行臺:



 陜西漢中道,鳳翔府置司。



 河西隴北道,甘州路置司。



 西蜀四川道,成都路置司。



 雲南諸路道,中慶路置司。



\end{pinyinscope}