\article{列傳第二 睿宗}

\begin{pinyinscope}

 睿宗景襄皇帝,諱拖雷,太祖第四子,太宗母弟也。方太祖崩時,太宗留霍博之地,國事無所屬,拖雷實身任之。聞燕京盜賊白晝剽掠富民財物,吏不能禁,遂遣塔察、吾圖撒合里往窮治之,殺十有六人,盜始屏息。



 己丑夏,太宗還京。八月,即位。明年庚寅秋,太宗伐金,命拖雷帥師以從,破天城堡,拔蒲城縣,聞金平章合達、參政蒲阿守西邊,遂渡河,攻鳳翔。會前兵戰不利,從太宗援之,合達乃退。辛卯春,破洛陽、河中諸城。太宗還官山,大會諸侯王,謂曰:「人言耗國家者,實由寇敵。今金未殄,實我敵也。諸君寧無計乎?」拖雷進曰:「臣有愚計,非眾可聞。」太宗屏左右,亟臨問之,其言秘,人莫知也。鳳翔既下,有降人李昌國者言:「金主遷汴,所恃者黃河、潼關之險爾。若出寶雞,入漢中,不一月可達唐、鄧。金人聞之,寧不謂我師從天而下乎!」拖雷然之,言於太宗。太宗大喜,語諸王大臣曰:「昔太祖嘗有志此舉,今拖雷能言之,真賽因也。」賽因,猶華言大好云。遂大發兵。



 太宗以中軍自碗子城南下,渡河,由洛陽進;斡陳那顏以左軍由濟南進;而拖雷總右軍自鳳翔渡渭水,過寶雞,入小潼關,涉宋人之境,沿漢水而下。期以明年春,俱會於汴。遣搠不罕詣宋假道,且約合兵。宋殺使者,拖雷大怒曰:「彼昔遣茍夢玉來通好,遽自食言背盟乎!」乃分兵攻宋諸城堡,長驅入漢中,進襲四川,陷閬州,過南部而還。遂由金取房,前鋒三千人破金兵十餘萬於武當山,趨均州。乘騎浮渡漢水,遣夔曲涅率千騎馳白太宗。太宗方詣漢水,將分兵應之,會夔曲涅至,即遣慰諭拖雷,亟合兵焉。



 拖雷既渡漢,金大將合達設伏二十餘萬於鄧州之西,據隘待之。時拖雷兵不滿四萬,及得諜報,乃悉留輜重,輕騎以進。十二月丙子,及金人戰於禹山,佯北以誘之,金人不動。拖雷舉火夜行,金合達聞其且至,退保鄧州,攻之,三日不下。遂將而北,以三千騎命札剌等率之為殿。明旦,大霧迷道,為金人所襲,殺傷相當。拖雷以札剌失律,罷之,而以野裡知給歹代焉。未幾,敗金軍。壬辰春,合達等知拖雷已北,合步騎十五萬躡其後。拖雷按兵,遣其將忽都忽等誘之。日且暮,令軍中曰:「毋令彼得休息,宜夜鼓以擾之。」太宗時亦渡河,遣親王口溫不花等將萬餘騎來會。天大雨雪,金人殭凍無人色,幾不能軍,拖雷即欲擊之,諸將請俟太宗至破之未晚,拖雷曰:「機不可失,彼脫入城,未易圖也。況大敵在前,敢以遺君父乎!」遂奮擊於三峰山,大破之,追奔數十里,流血被道,資仗委積,金之精銳盡於此矣。餘眾迸走睢州,伏兵起,又敗之。合達走鈞州,僅遺數百騎。蒲阿走汴,至望京橋,復禽獲之。太宗尋至,按行戰地,顧謂拖雷曰:「微汝,不能致此捷也。」諸侯王進曰:「誠如聖諭,然拖雷之功,著在社稷。」蓋又指其定冊云爾。拖雷從容對曰:「臣何功之有,此天之威,皇帝之福也。」聞者服其不伐。從太宗攻鈞州,拔之,獲合達。攻許州,又拔之,遂從太宗收定河南諸郡。四月,由半渡入真定,過中都,出北口,住夏於官山。



 五月,太宗不豫。六月,疾甚。拖雷禱於天地,請以身代之,又取巫覡祓除釁滌之水飲焉。居數日,太宗疾愈,拖雷從之北還,至阿剌合的思之地,遇疾而薨,壽四十有闕。妃怯烈氏。子十一人,長憲宗,次四則世祖也。憲宗立,追謚曰英武皇帝,廟號睿宗。二年,合祭昊天后土,以太祖、睿宗配享。世祖至元二年,改謚景襄皇帝。



 ○裕宗



 裕宗文惠明孝皇帝,諱真金,世祖嫡子也。母昭睿順聖皇后,弘吉烈氏。少從姚樞、竇默受《孝經》,及終卷,世祖大悅,設食饗樞等。中統三年,封燕王,守中書令。丞相史天澤入啟事,王曰:「我幼,未嘗習祖宗典則,閑於政體,一旦當大任,惟汝耆德賴焉。」復諭贊善王恂曰:「省臣所啟,等國事也。爾宜入與聞之。」四年,兼判樞密院事。至元初,省臣奏請王署敕,每月必再至中書。於是王將入中書,乳母進新衣,笑卻之曰:「吾何事美觀也。」嘗從幸宜興,世祖違豫,憂形於色,夕不能寐。聞母皇后暴得風疾,即悲泣,衣不及帶而行。七年秋,受詔巡撫稱海,至冬還京。間謂諸王札剌忽及從官伯顏等曰:「吾屬適有茲暇,宜各悉乃心,慎言所守,俾吾聞之。」於是撒里蠻曰:「太祖有訓:欲治身,先治心;欲責人,先責己。」伯顏曰:「皇上有訓:欺罔盜竊,人之至惡。一為欺罔,則後雖出善言,人終弗信;一為盜竊,則事雖未覺,心常惴惴,若捕者將至。」札剌忽曰:「我祖有訓:長者梢,深者底。蓋言貴有終始,長必極其杪,深必究其底,不可中輟也。」王曰:「皇上有訓:毋持大心。大心一持,事即隳敗。吾觀孔子之語,即與聖訓合也。」至王恂陳說尤多,事見恂傳。



 十年二月,立為皇太子,仍兼中書令,判樞密院事。受玉冊:「皇帝若曰:咨爾皇太子真金,仰惟太祖皇帝遺訓,嫡子中有克嗣服繼統者,豫選定之。是用立太宗英文皇帝,以紹隆丕構。自時厥後,為不顯立塚嫡,遂啟爭端。朕上遵祖宗宏規,下協昆弟僉同之議,乃從燕邸,即立爾為皇太子,積有日矣。比者儒臣敷奏,國家定立儲嗣,宜有冊命,此典禮也。今遣攝太尉、左丞相伯顏持節授爾玉冊金寶。於戲!聖武燕謀,爾其承奉。昆弟宗親,爾其和協。使仁孝顯於躬行,抑可謂不負所托矣。尚其戒哉,勿替朕命。」九月丙戌,詔立宮師府,設官屬三十有八員。起處士楊恭懿於京兆。



 太子嘗有疾,世祖臨幸,親和藥以賜之。遣侍臣李眾馳祀岳瀆名山川,太子戒其所至郡邑,毋煩吏迎送,重擾民也。詔以侍衛親軍萬人益隸東宮,太子命王慶端、董士亨選其驍勇者,教以兵法,時閱試焉。太子服綾袷,為瀋所漬,命侍臣重加染治,侍臣請織綾更制之,太子曰:「吾欲織百端,非難也。顧是物未敝,豈宜棄之。」東宮香殿成,工請鑿石為池,如曲水流觴故事。太子曰:「古有肉林酒池,爾欲吾效之耶!」不許。每與諸王近臣習射之暇,輒講論經典,若《資治通鑒》、《貞觀政要》,王恂、許衡所述遼、金帝王行事要略,下至《武經》等書,從容片言之間,茍有允愜,未嘗不為之灑然改容。時侍經幄者,如王恂、白棟皆朝夕不出東宮,而待制李謙、太常宋尤加咨訪,蓋無間也。



 十八年正月,昭睿順聖皇后崩,太子自獵所奔赴,勺飲不入口者終日,設廬帳居之。命宋擇可備顧問者,以郭祐、何瑋、徐琰、馬紹、楊居寬、何榮祖、楊仁風等為言。太子曰:「是數人者,盡為我致之,宜自近者始。」遂召瑋於易州、琰於東平。贊善王恂卒,太子聞之嗟悼,賻鈔二千五百緡。一日,顧謂左右曰:「王贊善當言必言,未嘗顧惜,隨事規正,良多裨補,今鮮有其匹也。」時阿合馬擅國重柄,太子惡其奸惡,未嘗少假顏色。盜知阿合馬所畏憚者,獨太子爾,因為偽太子,夜入京城,召而殺之。及和禮霍孫入相,太子曰:「阿合馬死於盜手,汝任中書,誠有便國利民者,毋憚更張。茍或沮撓,我當力持之。」中書啟以何瑋參議省事,徐琰為左司郎中。瑋、琰入見,太子諭之曰:「汝等學孔子之道,今始得行,宜盡平生所學,力行之。」闢楊仁風於潞州、馬紹於東平,復闢楊恭懿置省中議事,以衛輝總管董文用練達官政,與恭懿同置省中。按察副使王惲進《承華事略》:一曰《廣孝》,二曰《立愛》,三曰《端本》,四曰《進學》,五曰《擇術》,六曰《謹習》,七曰《聽政》,八曰《達聰》,九曰《撫軍》,十曰《明分》,十一曰《崇儒》,十二曰《親賢》,十三曰《去邪》,十四曰《納誨》,十五曰《幾諫》,十六曰《從諫》,十七曰《推恩》,十八曰《尚儉》,十九曰《戒逸》,二十曰《審官》。太子聞漢成帝不絕馳道,唐肅宗改絳紗袍為硃明服,大喜曰:「使吾行之,亦當若此。」及說邢峙止齊太子食邪蒿,顧宮臣曰:「菜名邪蒿,未必果邪也。雖食之,豈遽使人不正邪?」張九思對曰:「古人設戒,義固當爾。」



 詔割江西龍興路為太子分地,太子謂左右曰:「安得治民如邢州張耕者乎!誠使之往治,俾江南諸郡取法,民必安集。」於是召宋大選署守長。江西行省以歲課羨餘鈔四十七萬緡獻,太子怒曰:「朝廷令汝等安治百姓,百姓安,錢糧何患不足,百姓不安,錢糧雖多,安能自奉乎!」盡卻之。阿里以民官兼課司,請歲附輸羊三百,太子以其越例,罷之。參政劉思敬遣其弟思恭以新民百六十戶來獻,太子問民所從來,對曰:「思敬征重慶時所俘獲者。」太子蹙然曰:「歸語汝兄,此屬宜隨所在放遣為民,毋重失人心。」烏蒙宣撫司進馬,逾歲獻之額,即諭之曰:「去歲嘗俾勿多進馬,恐道路所經,數勞吾民也。自今其勿復然。」



 二十年春,闢劉因於保定,因以疾辭,固闢之,乃至,拜右贊善大夫,以吏部郎中夾谷之奇為左贊善大夫。是時已立國子學,李棟、宋、李謙皆以東宮僚友,繼典教事,至是,命因專領之,而以等仍備咨訪。嘗曰:「吾聞金章宗時,有司論太學生廩費太多,章宗謂養出一範文正公,所償顧豈少哉。其言甚善。」會因復以疾乞去。二十二年,以長史耶律有尚為國子司業。中庶子伯必以其子阿八赤入見,諭令入學,伯必即令其子入蒙古學。逾年又見,太子問讀何書,其子以蒙古書對,太子曰:「我命汝學漢人文字耳,其亟入胄監。」遣使闢宋工部侍郎倪堅於開元,既至,訪以古今成敗得失,堅對言:「三代得天下以仁,其失也以不仁。漢、唐之亡也,以外戚閹豎。宋之亡也,以奸黨權臣。」太子嘉納,賜酒,日昃乃罷。諭德李謙、夾谷之奇嘗進言曰:「殿下睿性夙成,閱理久熟,方遵聖訓,參決庶務,如視膳問安之禮,固無待於贊諭。至於軍民之利病,政令之得失,事關朝廷,責在臺院,有非宮臣所宜言者。獨有澄原固本,保守成業,殿下所當留心,臣等不容緘口者也。敬陳十事:曰正心,曰睦親,曰崇儉,曰親賢,曰幾諫,曰戢兵,曰尚文,曰定律,曰正名,曰革敝。」其論正心有云:「太子之心,天下之本也。太子心正,則天心有所屬,人心有所系矣。唐太宗嘗言,人主一心,攻之者眾,或以勇力,或以辨口,或以諂諛,或以奸詐,或以嗜欲,輻輳攻之,各求自售。人主少懈,而受其一,則其害有不可勝言者。殿下至尊之儲貳,人求自售者亦不為少,須常喚醒此心,不使為物欲所撓,則宗社生靈之福。固本澄原,莫此為切。」論睦親,以「宗親為王室之籓屏,人主之所自衛者也。大分既定,尊卑懸殊,必恩意俯逮,然後得盡其歡心。宗親之歡心得,則遠近之歡心得矣」。其論正名、革敝,尤切中時政。太子在中書日久,明於聽斷,四方州郡科徵、挽漕、造作、和市,有系民休戚者,聞之,即日奏罷。右丞盧世榮以言利進,太子意深非之。嘗曰:「財非天降,安得歲取贏乎!恐生民膏血,竭於此也。豈惟害民,實國之大蠹。」其後世榮果坐罪。桑哥素主世榮,聞太子有言,訖箝口不敢救。



 至元以來,天下臻於太平,人材輩出,太子優禮遇之,在師友之列者,非朝廷名德,則布衣節行之士,德意未嘗少衰。宋目疾,賜鈔千五百緡。王磐告老而歸,官其婿於東平,以終養。孔洙自江南入覲,則責張九思學聖人之道,不知有聖人之後。其大雅不群,本於天性,中外歸心焉。於是世祖春秋高,江南行臺監察御史言事者請禪位於太子,太子聞之,懼。臺臣寢其奏,不敢遽聞,而小人以臺臣隱匿,乘間發之。世祖怒甚,太子愈益懼,未幾,遂薨,壽四十有三。成宗即位,追謚曰文惠明孝皇帝,廟號裕宗,祔於太廟。



 ○顯宗



 顯宗光聖仁孝皇帝,諱甘麻剌,裕宗長子也。母曰徽仁裕聖皇后,弘吉剌氏。甘麻剌少育於祖母昭睿順聖皇后,日侍世祖,未嘗離左右,畏慎不妄言,言必無隱。至元中,奉旨鎮北邊,叛王岳木忽兒等聞其至,望風請降。既而都阿、察八兒諸王遣使求和,邊境以寧。嘗出征駐金山,會大雪,擁火坐帳內,歡甚,顧謂左右曰:「今日風雪如是,吾與卿處猶有寒色,彼從士亦人耳,腰弓矢、荷刃周廬之外,其苦可知。」遂命饔人大為肉糜,親嘗而遍賜之。撫循部曲之暇,則命也滅堅以國語講《通鑒》。戒其近侍太不花曰:「朝廷以籓屏寄我,事有不逮,正在汝輩輔助。其或依勢作威,不用我命,輕者論遣,大者奏聞耳,宜各慎之。使百姓安業,主上無北顧之憂,則予與卿等亦樂處於此,乃所以報國家也。」



 二十六年,世祖以其居邊日久,特命獵於柳林之地。率眾至漷州,恐廩膳不均,令左右司之,分給從士,仍飭其眾曰:「汝等飲食既足,若復侵漁百姓,是汝自取罪謫,無悔。」眾皆如約,民賴以安。北還,覲世祖於上京,世祖勞之曰:「汝在柳林,民不知擾,朕實嘉焉。」明年冬,封梁王,授以金印,出鎮雲南。過中山,又明年春過懷、孟,從卒馬駝之屬不下千百計,所至未嘗橫取於民。



 二十九年,改封晉王,移鎮北邊,統領太祖四大斡耳朵及軍馬、達達國土,更鑄晉王金印授之。中書省臣言於世祖曰:「諸王皆置傳,今晉王守太祖肇基之地,視諸王宜有加,請置內史。」世祖從之,遂以北安王傅禿歸、梁王傅木八剌沙、雲南行省平章賽陽並為內史。明年,置內史府。又明年,世祖崩,晉王聞訃,奔赴上都。諸王大臣咸在,晉王曰:「昔皇祖命我鎮撫北方,以衛社稷,久歷邊事,願服厥職。母弟鐵木耳仁孝,宜嗣大統。」於是成宗即帝位,而晉王復歸籓邸。



 元貞元年,塔塔兒部年穀不熟,檄宣徽院賑之。又答答剌民饑,請朝廷賑之。詔賜王鈔千萬貫,及銀帛有差。皇太后復以雲南所貢金器,遣朵年來賜。是歲冬,奉詔以知樞密院事札散、同知徽政院事阿里罕為內史。大德二年,詔給秫米五百石。五年,成宗以邊士貧乏,分給鈔一千萬貫。



 六年正月乙巳,王薨,年四十。王天性仁厚,御下有恩。元貞初,籓邸屬官審伯年老,請以其子代之。內史言於王,王曰:「惟天子所命。」其自守如此,故尤為朝廷所重。然崇尚浮屠,命僧作佛事,歲耗財不可勝計。子三人:曰也孫帖木兒,曰松山,曰迭裡哥兒不花。王薨後十年,仁宗即位,謚王獻武。又十一年,英宗遇弒,也孫帖木兒以嗣晉王即皇帝位,追尊曰光聖仁孝皇帝,廟號顯宗,祔享太室。又六年,文宗即位,乃毀其廟室。



 ○順宗



 順宗昭聖衍孝皇帝,諱答剌麻八剌,裕宗第二子也。母曰徽仁裕聖皇后,弘吉剌氏。至元初,裕宗為燕王,答剌麻八剌生於燕邸。明年,詔裕宗居潮河。八月,召至京師。凡乘輿巡幸及歲時朝賀,未嘗不侍裕宗以行。稍長,世祖賜女侍郭氏,其後乃納弘吉剌氏為妃。二十二年,裕宗薨,答剌麻八剌以皇孫鐘愛,兩宮優其出閣之禮。二十八年,始詔出鎮懷州,以侍衛都指揮使梭都、尚書王倚從行,至趙州,從卒有伐民桑棗者,民庶訴於道,答剌麻八剌怒,杖從卒以懲眾,遣王倚入奏,世祖大悅。未至,以疾召還。明年春,世祖北幸,留治疾京師,越兩月而薨,年二十有九。



 子三人:長曰阿木哥,封魏王,郭出也;妃所生者曰海山,是為武宗;曰愛育黎拔力八達,是為仁宗。大德十一年秋,武宗即位,追謚曰昭聖衍孝皇帝,廟號順宗,祔享太廟。



\end{pinyinscope}