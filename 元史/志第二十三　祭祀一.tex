\article{志第二十三 祭祀一}

\begin{pinyinscope}

 禮之有祭祀,其來遠矣。天子者,天地宗廟社稷之主,於郊社禘嘗有事守焉,以其義存乎報本,非有所為而為之。故其禮貴誠而尚質,務在反本循古,不忘其初而已。漢承秦弊,郊廟之制,置《周禮》不用,謀議巡守封禪,而方士祠官之說興,兄弟相繼共為一代,而統緒亂。迨其季世,乃合南北二郊為一。雖以唐、宋盛時,亦莫之正,蓋未有能反其本而求之者。彼籩豆之事,有司所職,又豈足以盡仁人孝子之心哉!



 元之五禮,皆以國俗行之,惟祭祀稍稽諸古。其郊廟之儀,禮官所考日益詳慎,而舊禮初未嘗廢,豈亦所謂不忘其初者歟?然自世祖以來,每難於親其事。英宗始有意親郊,而志弗克遂。久之,其禮乃成於文宗。至大間,大臣議立北郊而中輟,遂廢不講。然武宗親享於廟者三,英宗親享五。晉王在帝位四年矣,未嘗一廟見。文宗以後,乃復親享。豈以道釋禱祠薦禳之盛,竭生民之力以營寺宇者,前代所未有,有所重則有所輕歟。或曰,北陲之俗,敬天而畏鬼,其巫祝每以為能親見所祭者,而知其喜怒,故天子非有察於幽明之故、禮俗之辨,則未能親格,豈其然歟?



 自憲宗祭天日月山,追崇所生與太祖並配,世祖所建太廟,皇伯術赤、察合帶皆以家人禮祔於列室。既而太宗、定宗以世天下之君俱不獲廟享,而憲宗亦以不祀。則其因襲之弊,蓋有非禮官之議所能及者。而況乎不禰所受國之君,而兄弟共為一世,乃有徵於前代者歟?夫郊廟,國之大祀也,本原之際既已如此,則中祀以下,雖有闊略,無足言者。



 其天子親遣使致祭者三:曰社稷,曰先農,曰宣聖。而岳鎮海瀆,使者奉璽書即其處行事,稱代祀。其有司常祀者五:曰社稷,曰宣聖,曰三皇,曰嶽鎮海瀆,曰風師雨師。其非通祀者五:曰武成王,曰古帝王廟,曰周公廟,曰名山大川、忠臣義士之祠,曰功臣之祠,而大臣家廟不與焉。其儀皆禮官所擬,而議定於中書。日星始祭於司天臺,而回回司天臺遂以翽星為職事。五福太乙有壇畤,以道流主之,皆所未詳。



 凡祭祀之事,其書為《太常集禮》,而《經世大典》之《禮典篇》尤備。參以累朝《實錄》與《六條政類》,序其因革,錄其成制,作《祭祀志》。



 郊祀上



 元與朔漠,代有拜天之禮,衣冠尚質,祭器尚純,帝後親之,宗戚助祭。其意幽深玄遠,報本反始,出於自然,而非強為之也。憲宗即位之二年秋八月八日,始以冕服拜天於日月山。其十二日,又用孔氏子孫元措言,合祭昊天后土,始大合樂作牌位,以太祖、睿宗配享。歲甲寅,會諸王於顆顆腦兒之西,丁巳秋,駐蹕於軍腦兒,皆祭天於其地。世祖中統二年,親征北方。夏四月己亥,躬祀天於舊桓州之西北,灑馬湩以為禮,皇族之外無得而與,皆如其初。



 至元十二年十二月,以受尊號,遣使豫告天地,下太常檢討唐、宋、金舊儀,於國陽麗正門東南七里建祭臺,設昊天上帝、皇地祇位二,行一獻禮。自後國有大典禮,皆即南郊告謝焉。十三年五月,以平宋,遣使告天地,中書下太常議定儀物以聞。制若曰:「其以國禮行事。」



 三十一年,成宗即位。夏四月壬寅,始為壇於都城南七里。甲辰,遣司徒兀都帶率百官為大行皇帝請謚南郊,為告天請謚之始。大德六年春三月庚戌,合祭昊天上帝、皇地祇、五方帝於南郊,遣左丞相哈剌哈孫攝事,為攝祀天地之始。



 大德九年二月二十四日,右丞相哈剌哈孫等言:「去年地震星變,雨澤愆期,歲比不登。祈天保民之事,有天子親祀者三:曰天,曰祖宗,曰社稷。今宗廟、社稷,歲時攝官行事。祭天,國之大事也,陛下雖未及親祀,宜如宗廟、社稷,遣官攝祭,歲用冬至,儀物有司豫備,日期至則以聞。」制若曰:「卿言是也,其豫備儀物以待事。」於是翰林、集賢、太常禮官皆會中書集議。博士疏曰:「冬至,圜丘惟祀昊天上帝,至西漢元始間,始合祭天地。歷東漢至宋千有餘年,分祭合祭,迄無定論。」集議曰:「《周禮》,冬至圜丘禮天,夏至方丘禮地,時既不同,禮樂亦異。王莽之制,何可法也。今當循唐、虞、三代之典,惟祀昊天上帝。其方丘祭地之禮,續議以聞。」按《周禮》,壇壝三成,近代增外四成,以廣天文從祀之位。集議曰:「依《周禮》三成之制。然《周禮》疏雲每成一尺,不見縱廣之度。恐壇上狹隘,器物難容,擬四成制內減去一成,以合陽奇之數。每成高八尺一寸,以合乾之九九。上成縱廣五丈,中成十丈,下成十五丈。四陛,陛十有二級。外設二壝,內壝去壇二十五步,外壝去內壝五十四步,壝各四門。壇設於丙巳之地,以就陽位」按古者,親祀冕無旒,服大裘而加袞。臣下從祀,冠服歷代所尚,其制不同。集議曰:「依宗廟見用冠服制度。」按《周禮·大司樂》云:「凡樂,圜鐘為宮,黃鐘為角,太簇為征,姑洗為羽,雷鼓雷鞀,孤竹之管,雲和之琴瑟,雲門之舞,冬至日於地上之圜丘奏之。若樂六變,則天神皆降,可得而禮矣。」集議曰:「樂者所以動天地,感鬼神,必訪求深知音律之人,審五聲八音,以司肄樂。」



 夏四月壬辰,中書復集議。博士言:「舊制神位版用木。」中書議,改用蒼玉金字,白玉為座。博士曰:「郊祀尚質,合依舊制。」遂用木主,長二尺五寸,闊一尺二寸,上圓下方,丹漆金字,木用松柏,貯以紅漆匣,黃羅帕覆之。造畢,有司議所以藏。議者復謂,神主廟則有之,今祀於壇,對越在上,非若他神無所見也。所制神主遂不用。



 七月九日,博士又言:「古者祀天,器用陶匏,席用槁鞂。自漢甘泉雍畤之祀,以迄後漢、魏、晉、南北二朝、隋、唐,其壇壝玉帛禮器儀仗,日益繁縟,浸失古者尚質之意。宋、金多循唐制,其壇壝禮器,考之於經,固未能全合,其儀法具在。當時名儒輩出,亦未嘗不援經而定也,酌古今以行禮,亦宜焉。今檢討唐、宋、金親祀、攝行儀注,並雅樂節次,合從集議。」太常議曰:「郊祀之事,聖朝自平定金、宋以來,未暇舉行,今欲修嚴,不能一舉而大備。然始議之際,亦須酌古今之儀,垂則後來。請從中書會翰林、集賢、禮官及明禮之士,講明去取以聞。」中書集議曰:「合行禮儀,非草創所能備。唐、宋皆有攝行之禮,除從祀受胙外,一切儀注,悉依唐制修之。」



 八月十二日,太常寺言:「尊祖配天,其禮儀樂章別有常典,若俟至日議之,恐匆遽有誤。」於是中書省臣奏曰:「自古漢人有天下,其祖宗皆配天享祭,臣等與平章何榮祖議,宗廟已依時祭享,今郊祀止祭天。」制曰:「可。」是歲南郊,配位遂省。



 十一年,武宗即位。秋七月甲子,命御史大夫鐵古迭兒即南郊告謝天地,主用柏素,質玄書,為即位告謝之始。



 至大二年冬十月乙酉,尚書省臣及太常禮官言:「郊祀者國之大禮,今南郊之禮已行而未備,北郊之禮尚未舉行。今年冬至南郊,請以太祖聖武皇帝配享;明年夏至北郊,以世祖皇帝配。」帝皆是之。十二月甲辰朔,尚書太尉右丞相、太保左丞相、田司徒、郝參政等復奏曰:「南郊祭天於圜丘,大禮已舉。其北郊祭皇地祇於方澤,並神州地祇、五岳四瀆、山林川澤及朝日夕月,此有國家所當崇禮者也。當聖明御極而弗舉行,恐遂廢弛。」制若曰:「卿議甚是,其即行焉。」



 至大三年春正月,中書禮部移太常禮儀院,下博士擬定北郊從祀、朝日夕月禮儀。博士李之紹、蔣汝礪疏曰:「按方丘之禮,夏以五月,商以六月,周以夏至,其丘在國之北。禮神之玉以黃琮,牲用黃犢,幣用黃繒,配以後稷。其方壇之制,漢去都城四里,為壇四陛。唐去宮城北十四里,為方壇八角三成,每成高四尺,上闊十六步,設陛。上等陛廣八尺,中等陛一丈,下等陛廣一丈二尺。宋至徽宗始定為再成。歷代制雖不同,然無出於三成之式。今擬取坤數用六之義,去都城北六里,於壬地選擇善地,於中為方壇,三成四陛,外為三壝。仍依古制,自外壝之外,治四面稍令低下,以應澤中之制。宮室、墻圍、器皿色,並用黃。其再成八角八陛,非古制,難用。其神州地祇以下從祀,自漢以來,歷代制度不一,至唐始因隋制,以岳鎮海瀆、山林川澤、丘陵墳衍原隰,各從其方從祀。今盍參酌舉行。」秋九月,太常禮儀院復下博士,檢討合用器物。十一月丙申,有事於南郊,以太祖配,五方帝日月星辰從祀。



 仁宗延祐元年夏四月丁亥,太常寺臣請立北郊。帝謙遜未遑,北郊之議遂輟。



 英宗至治二年九月,有旨議南郊祀事。中書平章買閭,御史中丞曹立,禮部尚書張野,學士蔡文淵、袁桷、鄧文原,太常禮儀院使王緯、田天澤,博士劉致等會都堂議:



 一曰年分,按前代多三年一祀,天子即位已及三年,常有旨欽依。



 二曰神位。《周禮·大宗伯》,「以禋祀祀昊天上帝」。注謂:「昊天上帝,冬至圜丘所祀天皇大帝也。」又曰「蒼璧禮天」。注云:「此禮天以冬至,謂天皇大帝也。在北極,謂之北辰。」又云:「北辰天皇耀魄寶也,又名昊天上帝,又名太一帝君,以其尊大,故有數名。」今按《晉書·天文志·中宮》「鉤陳口中一星曰天皇大帝,其神耀魄寶」。《周禮》所祀天神,正言昊天上帝。鄭氏以星經推之,乃謂即天皇大帝。然漢、魏以來,名號亦復不一。漢初曰上帝,曰太一,曰皇天上帝。魏曰皇皇帝天。梁曰天皇大帝。惟西晉曰昊天上帝,與《周禮》合。唐、宋以來,壇上既設昊天上帝,第一等復有天皇大帝,其五天帝與太一、天一等,皆不經見。本朝大德九年,中書圓議,止依《周禮》,祀昊天上帝。至大三年圓議,五帝從享,依前代通祭。



 三曰配位。《孝經》曰:「孝莫大於嚴父,嚴父莫大於配天。」又曰:「郊祀后稷以配天。」此郊之所以有配也。漢、唐已下,莫不皆然。至大三年冬十月三日,奉旨十一月冬至合祭南郊,太祖皇帝配,圓議取旨。



 四曰告配。《禮器》曰:「魯人將有事於上帝,必先有事於NT宮。」注:「告後稷也,告之者,將以配天也。」告用牛一。《宋會要》於致齋二日,宿廟告配,凡遣官犧尊豆籩,行一獻禮。至大三年十一月二十一日,質明行事。初獻攝太尉同太常禮儀院官赴太廟奏告,圓議取旨。



 五曰大裘冕。《周禮》司裘「掌為大裘,以共王祀天之服」,鄭司農云,黑羊裘,服以祀天,示質也。弁師「掌王之五冕」,注:「冕服有六,而言五者,大裘之冕蓋無旒,不聯數也。」《禮記·郊特牲》曰:「郊之祭也,迎長日之至也。祭之日,王被袞以象天,戴冕璪十有二旒,則天數也。」陸佃曰:「禮不盛服不充,蓋服大裘以袞襲之也。謂冬至服大裘,被之以袞。開元及開寶《通禮》,鸞駕出宮,服袞冕至大次,質明改服大裘冕而出次。《宋會要》紹興十三年,車駕自廟赴青城,服通天冠、絳紗袍,祀日服大裘袞冕。圓議用袞冕,取旨。



 六曰匏爵。《郊特牲》曰:「郊之祭也,器用陶匏,以象天地之性也。」注謂:「陶瓦器,匏用酌獻酒。」《開元禮》、《開寶禮》皆有匏爵。大德九年,正配位用匏爵有坫。圓議正位用匏,配位飲福用玉爵,取旨。



 七曰戒誓。唐《通典》引《禮經》,祭前期十日親戒百官及族人,太宰總戒群官。唐前祀七日,《宋會要》十日。《纂要》太尉南向,司徒、亞終獻、一品、二品從祀北向,行事官以次北向,禮直官以誓文授之太尉讀。今天子親行大禮,止令禮直局管勾讀誓文。圓議令管勾代太尉讀誓,刑部尚書蒞之。



 八曰散齋、致齋。《禮經》前期十日,唐、宋、金皆七日,散齋四日,致齋三日。國朝親祀太廟七日,散齋四日於別殿,致齋三日於大明殿。圓議依前七日。



 九曰藉神席。《郊特牲》曰:「莞簟之安,而蒲越槁鞂之尚。」注:「蒲越槁鞂,藉神席也。」《漢舊儀》高帝配天紺席,祭天用六彩綺席六重。成帝即位,丞相衡、御史大夫譚以為天地尚質,宜皆勿修,詔從焉。唐麟德二年,詔曰:「自處以厚,奉天以薄,改用裀褥。上帝以蒼,其餘各視其方色。」宋以褥加席上,禮官以為非禮。元豐元年,奉旨不設。國朝大德九年,正位槁鞂,配位蒲越,冒以青繒。至大三年,加青綾褥,青錦方座。圓議合依至大三年於席上設褥,各依方位。



 十曰犧牲。《郊特牲》曰:「郊特牲而社稷太牢。」又曰:「天地之牛角繭慄。」秦用騮駒。漢文帝五帝共一牲,武帝三年一祀,用太牢。光武採元始故事,天地共犢。隋上帝、配帝,蒼犢二。唐開元用牛。宋正位用蒼犢一,配位太牢一。國朝大德九年,蒼犢二,羊豕各九。至大三年,馬純色肥腯一,牲正副一,鹿一十八,野豬一十八,羊一十八,圓議依舊儀。神位配位用犢外,仍用馬,其餘並依舊日已行典禮。



 十一曰香鼎。大祭有三,始煙為歆神,始宗廟則焫蕭稞鬯,所謂臭陽達於墻屋者也。後世焚香,蓋本乎此,而非《禮經》之正。至大三年,用陶瓦香鼎五十,神座香鼎、香盒案各一。圓議依舊儀。



 十二曰割牲。《周禮·司士》「凡祭祀,帥其屬而割牲,羞俎豆」。又《諸子》,「大祭祀正六牲之體」。《禮運》云「腥其俎,熟其殽,體其犬豕牛羊」。注云:「腥其俎,謂豚解而腥之,為七體也。熟其殽,謂體解而爓之,為二十一體也。體其犬豕牛羊,謂分別骨肉之貴賤,以為眾俎也。」七體,謂脊、兩肩、兩拍、兩髀。二十一體,謂肩、臂、臑、膊、骼、正脊、脠脊、橫脊、正脅、短脅、代脅並腸三、胃三、拒肺一、祭肺三也。宋元豐三年,詳定禮文所言,古者祭祀用牲,有豚解,有體解。豚解則為七,以薦腥;體解則為二十一,以薦熟。蓋犬豕牛羊,分別骨肉貴賤,其解之為體,則均也。皇朝馬牛羊豕鹿,並依至大三年割牲用國禮。圓議依舊儀。



 十三曰大次、小次。《周禮·掌次》,「王旅上帝,張氈按皇邸。」唐《通典》前祀三日,尚舍直長施大次於外壝東門之內道北,南向。《宋會要》前祀三日,儀鸞司帥其屬,設大次於外壝東門之內道北,南向;小次於午階之東,西向。《曲禮》曰:「踐阼,臨祭祀。」《正義》曰:「阼主階也。天子祭祀履主階行事,故云踐阼。」宋元豐詳定禮文所言,《周禮》宗廟無設小次之文。古者人君臨位於阼階。蓋阼階者,東階也。惟人主得位主階行事。今國朝太廟儀注,大次、小次皆在西,蓋國家尚右,以西為尊也。圓議依祀廟儀注。



 續具末議:一曰禮神玉。《周禮·大宗伯》,「以禋祀祀昊天上帝」。注:「禋之言煙也。周人尚臭,煙氣之臭聞者。積柴實牲體焉,或有玉帛。」《正義》曰:「或有玉帛,或不用玉帛,皆不定之辭也。」崔氏云,天子自奉玉帛牲體於柴上,引《詩》「圭璧既卒」,是燔牲玉也。蓋卒者,終也。謂禮神既終,當藏之也。正經即無燔玉明證。漢武帝祠太乙,胙餘皆燔之,無玉。晉燔牲幣,無玉。唐、宋乃有之。顯慶中,許敬宗等修舊禮,乃云郊天之有四圭,猶宗廟之有圭瓚也,並事畢收藏,不在燔列。宋政和禮制局言:「古祭祀無不用玉,《周官》典瑞掌玉器之藏,蓋事已則藏焉,有事則出而復用,未嘗有燔瘞之文。今後大祀,禮神之玉時出而用,無得燔瘞。」從之。蓋燔者取其煙氣之臭聞。玉既無煙,又且無氣,祭之日但當奠於神座,既卒事,則收藏之。



 二曰飲福。《特牲饋食禮》曰,尸九飯,親嘏主人。《少牢饋食禮》尸十一飯,尸嘏主人。嘏,長也,大也。行禮至此,神明已饗,盛禮俱成,故膺受長大之福於祭之末也。自漢以來,人君一獻才畢而受嘏。唐《開元禮》太尉未升堂,而皇帝飲福。宋元豐三年,改從亞終獻,既行禮,皇帝飲福受胙。國朝至治元年親祀廟儀注,亦用一獻畢飲福。



 三曰升煙。禋之言煙也,升煙所以報陽也。祀天之有禋柴,猶祭地之瘞血,宗廟之祼鬯。歷代以來,或先燔而後祭,或先祭而後燔,皆為未允。祭之日,樂六變而燔牲首,牲首亦陽也。祭終,以爵酒饌物及牲體,燎於壇。天子望燎,柴用柏。



 四曰儀注。《禮經》出於秦火之後,殘闕脫漏,所存無幾。至漢,諸儒各執所見。後人所宗,惟鄭康成、王子NU,而二家自相矛盾。唐《開元禮》、杜祐《通典》,五禮略完。至宋《開寶禮》並《會要》與郊廟奉祠禮文,中間講明始備。金國大率依唐、宋制度。聖朝四海一家,禮樂之興,政在今日。況天子親行大禮,所用儀注,必合講求。大德九年,中書集議,合行禮儀依唐制。至治元年已有祀廟儀注,宜取大德九年、至大三年並今次新儀,與唐制參酌增損修之。侍儀司編排鹵簿,太史院具報星位,分獻官員數及行禮並諸執事官,合依至大三年儀制亞終獻官,取旨。



 是歲,太皇太后崩,有旨冬至南郊祀事,可權止。



 泰定四年春正月,御史臺臣言:「自世祖迄英宗,咸未親郊,惟武宗、英宗親享太廟,陛下宜躬祀郊廟。」制曰:「朕當遵世祖舊典,其命大臣攝行祀事。」閏九月甲戌,郊祀天地,致祭五岳四瀆、名山大川。



 至順元年,文宗將親郊。十月辛亥,太常博士言:「親祀儀注已具,事有未盡者,按前代典禮。親郊七日,百官習儀於郊壇。今既與受戒誓相妨,合於致齋前一日,告示與祭執事者,各具公服赴南郊習儀。親祀太廟雖有防禁,然郊外尤宜嚴戒,往來貴乎清肅。凡與祭執事齋郎樂工,舊不設盥洗之位,殊非涓潔之道。今合於饌殿齊班前及齋宿之所,隨宜設置盥洗數處,俱用鍋釜溫水置盆杓巾帨,令人掌管省諭,必盥洗然後行事,違者治之。祭日,太常院分官提調神廚,監視割烹。上下燈燭凡燎,已前雖有翦燭提調凡盆等官,率皆虛應故事;或減刻物料,燭燎不明。又嘗見奉禮贊賜胙之後,獻官方退,所司便服徹俎,壇上燈燭一時俱滅,因而雜人登壇攘奪,不能禁止,甚為褻慢。今宜禁約,省牲之前,凡入壝門之人,皆服窄紫,有官者公服。禁治四壝紅門,宜令所司添造關木鎖鑰,祭畢即令關閉,毋使雜人得入。其稿秸匏爵,事畢合依大德九年例焚之。」壬子,御史臺臣言:「祭日,宜敕股肱近臣及諸執事人毋飲酒。」制曰:「卿言甚善,其移文中書禁之。」丙辰,監察御史楊彬等言:「禮,享帝必以始祖為配,今未聞設配位,竊恐禮文有闕。又,先祀一日,皇帝必備法駕出宿郊次,其扈從近侍之臣未嘗經歷,宜申加戒敕,以達孚誠。」命與中書議行。十月辛酉,始服大裘袞冕,親祀昊天上帝於南郊,以太祖配。自世祖混一六合,至文宗凡七世,而南郊親祀之禮始克舉焉,蓋器物儀注至是益加詳慎矣。



 自至元十二年冬十二月,用香酒脯MZ行一獻禮。而至治元年冬二祭告,泰定元年之正月,咸用之。自大德九年冬至,用純色馬一,蒼犢一,羊鹿野豕各九。十一年秋七月,用馬一,蒼犢正副各一,羊鹿野豕各九。而至大中告謝五,皇慶至延祐告謝七,與至治三年冬告謝二,泰定元年之二月,咸如大德十一年之數。泰定四年閏九月,特加皇地祇黃犢一,將祀之夕敕送新獵鹿二。惟至大三年冬至,正配位蒼犢皆一,五方帝犢各一,皆如其方之色,大明青犢、夜明白犢皆一,馬一,羊鹿野豕各十有八,兔十有二,而四年四月如之。其犧牲品物香酒,皆參用國禮,而豐約不同。告謝非大祀,而用物無異,豈所謂未能一舉而大備者乎?



 南郊之禮,其始為告祭,繼而有大祀,皆攝事也,故攝祀之儀特詳。



 壇壝:地在麗正門外丙位,凡三百八畝有奇。壇三成,每成高八尺一寸,上成縱橫五丈,中成十丈,下成十五丈。四陛午貫地子午卯酉四位陛十有二級。外設二壝。內壝去壇二十五步,外壝去內壝五十四步。壝各四門,外垣南櫺星門三,東西櫺星門各一。圜壇周圍上下俱護以甓,內外壝各高五尺,壝四面各有門三,俱塗以赤。至大三年冬至,以三成不足以容從祀版位,以青繩代一成。繩二百,各長二十五尺,以足四成之制。燎壇在外壝內丙巳之位,高一丈二尺,四方各一丈,周圜亦護以甓,東西南三出陛,開上南出戶,上方六尺,深可容柴。香殿三間,在外壝南門之外,少西,南向。饌幕殿五間,在外壝南門之外,少東,南向。省饌殿一間,在外壝東門之外,少北,南向。



 外壝之東南為別院。內神廚五間,南向;祠祭局三間,北向;酒庫三間,西向。獻官齋房二十間,在神廚南垣之外,西向。外壝南門之外,為中神門五間,諸執事齋房六十間以翼之,皆北向。兩翼端皆有垣,以抵東西周垣,各為門,以便出入。齊班五間,在獻官齋房之前,西向。儀鸞局三間,法物庫三間,都監庫五間,在外垣內之西北隅,皆西向。雅樂庫十間,在外垣西門之內,少南,東向。演樂堂七間,在外垣內之西南隅,東向。獻官廚三間,在外垣內之東南隅,西向。滌養犧牲所,在外垣南門之外,少東,西向。內犧牲房三間,南向。



 神位:昊天上帝位天壇之中,少北,皇地祇位次東,少卻,皆南向。神席皆緣以繒,綾褥素座,昊天上帝色皆用青,皇地祇色皆用黃,藉皆以稿秸。配位居東,西向。神席綾褥錦方座,色皆用青,藉以蒲越。



 其從祀圜壇,第一等九位。青帝位寅,赤帝位巳,黃帝位未,白帝位申,黑帝位亥,主皆用柏,素質玄書;大明位卯,夜明位酉,北極位丑,天皇大帝位戌,用神位版,丹質黃書。神席綾褥座各隨其方色,藉皆以稿秸。



 第二等內官位五十有四。鉤星、天柱、玄枵、天廚、柱史位於子,其數五;女史、星紀、御女位於丑,其數三;自子至丑,神位皆西上。帝座、歲星、大理、河漢、析木、尚書位於寅,帝座居前行,其數六,南上。陰德、大火、天槍、玄戈、天床位於卯,其數五,北上。太陽守、相星、壽星、輔星、三師位於辰,其數五,南上。天一、太一、內廚、熒惑、鶉尾、勢星、天理位於巳,天一、太一居前行,其數七,西上。北斗、天牢、三公、鶉火、文昌、內階位於午,北斗居前行,其數六;填星、鶉首、四輔位於未,其數三;自午至未,皆東上。太白、實沈位於申,其數二,北上。八穀、大梁、杠星、華蓋位於酉,其數四;五帝內座、降婁、六甲、傳舍位於戌,五帝內座居前行,其數四;自酉至戌,皆南上。紫微垣、辰星、陬訾、鉤陳位於亥,其數四,東上。神席皆藉以莞席,內壝外諸神位皆同。



 第三等中官百五十九位。虛宿、女宿、牛宿、織女、人星、司命、司非、司危、司祿、天津、離珠、羅堰、天桴、奚仲、左旗、河鼓、右旗位於子,虛宿、女宿、牛宿、織女居前行,其數十有七;月星、建星、斗宿、箕宿、天雞、輦道、漸臺、敗瓜、扶筐、匏瓜、天弁、天棓、帛度、屠肆、宗星、宗人、宗正位於丑,月星、建星、斗宿、箕宿居前行,其數十有七;自子至丑,皆西上。日星、心宿、天紀、尾宿、罰星、東咸、列肆、天市垣、斛星、斗星、車肆、天江、宦星、市樓、候星、女床、天籥位於寅,日星、心宿、天紀、尾宿居前行,其數十有七,南上。房宿、七公、氐宿、帝席、大角、亢宿、貫索、鍵閉、鉤鈐、西咸、天乳、招搖、梗河、亢池、周鼎位於卯,房宿、七公、氐宿、帝席、大角、亢宿居前行,其數十有五,北上。太子星、太微垣、軫宿、角宿、攝提、常陳、幸臣、謁者、三公、九卿、五內諸侯、郎位、郎將、進賢、平道、天田位於辰,太子星、太微垣、軫宿、角宿、攝提居前行,其數十有六,南上。張宿、翼宿、明堂、四帝座、黃帝座、長垣、少微、靈臺、虎賁、從官、內屏位於巳,張宿、翼宿、明堂居前行,其數十有一,西上。軒轅、七星、三臺、柳宿、內平、太尊、積薪、積水、北河位於午,軒轅、七星、三臺、柳宿居前行,其數九;鬼宿、井宿、參宿、天尊、五諸侯、鉞星、座旗、司怪、天關位於未,鬼宿、井宿、參宿居前行,其數九;自午至未,皆東上。畢宿、五車、諸王、觜宿、天船、天街、礪石、天高、三柱、天潢、咸池位於申,畢宿、五車、諸王、觜宿居前行,其數十有一,北上。月宿、昴宿、胃宿、積水、天讒、卷舌、天河、積尸、太陵、左更、天大將軍、軍南門位於酉,月宿、昴宿、胃宿居前行,其數十有二;婁宿、奎宿、壁宿、右更、附路、閣道、王良、策星、天廄、土公、雲雨、霹靂位於戌,婁宿、壁宿居前行,其數十有二;自酉至戌,皆南上。危宿、室宿、車府、墳墓、虛梁、蓋屋、臼星、杵星、土公吏、造父、離宮、雷電、騰蛇位於亥,危宿、室宿居前行,其數十有三,東上。



 內壝內外官一百六位。天壘城、離瑜、代星、齊星、周星、晉星、韓星、秦星、魏星、燕星、楚星、鄭星位於子,其數十有二;越星、趙星、九坎、天田、狗國、天淵、狗星、鱉星、農丈人、杵星、糠星位於丑,其數十有一;自子至丑,皆西上。騎陣將軍、天輻、從官、積卒、神宮、傅說、龜星、魚星位於寅,其數八,南上。陣車、車騎、騎官、頡頏、折威、陽門、五柱、天門、衡星、庫樓位於卯,其數十,北上。土司空、長沙、青丘、南門、平星位於辰,其數五,南上。酒旗、天廟、東甌、器府、軍門、左右轄位於巳,其數六,西上。天相、天稷、爟星、天記、外廚、天狗、南河位於午,其數七;天社、矢星、水位、闕丘、狼星、弧星、老人星、四瀆、野雞、軍市、水府、孫星、子星位於未,其數十有三;自午至未,皆東上。天節、九州殊口、附耳、參旗、九斿、玉井、軍井、屏星、伐星、天廁、天矢、丈人位於申,其數十有二,北上。天園、天陰、天廩、天苑、天囷、芻槁、天庾、天倉、鈇鑕、天溷位於酉,其數十;外屏、大司空、八魁、羽林位於戌,其數四;自酉至戌,皆南上。哭星、泣星、天錢、天綱、北落師門、敗臼、斧鉞、壘壁陣位於亥,其數八,東上。



 內壝外眾星三百六十位,每辰神位三十自第二等以下,神位版皆丹質黃書。內官、中官、外官則各題其星名;內壝外三百六十位,惟題曰眾星位。凡從祀位皆內向,十二次微左旋,子居子陛東,午居午陛西,卯居卯陛南,酉居酉陛北。



 器物之等,其目有八:



 一曰圭幣。昊天上帝蒼璧一,有繅藉,青幣一,燎玉一。皇地祇黃琮一,有繅藉,黃幣一。配帝青幣一,黃帝黃琮一,青帝青圭一,赤帝赤璋一,白帝白琥一,黑帝玄璜一,幣皆如其方色。大明青圭有邸,夜明白圭有邸,天皇大帝青圭有邸,北極玄圭有邸,幣皆如其玉色。內官以下皆青幣。



 二曰尊罍。上帝大尊、著尊、犧尊、山罍各一,在壇上東南隅,皆北向,西上;設而不酌者,象尊、壺尊各二,山罍四,在壇下午陛之東,皆北向,西上。皇地祇亦如之,在上帝酒尊之東,皆北向,西上。配帝著尊、犧尊、象尊各二,在地祇酒尊之東,皆北向,西上。設而不酌者,犧尊、壺尊各二,山罍四,在壇下酉陛之北,東向,北上。五帝、日月、北極、天皇,皆太尊一,著尊二。內官十二次,各象尊二。中官十二次,各壺尊二。外官十二次,各概尊二。眾星十二次,各散尊二。凡尊各設於神座之左而右向,皆有坫,有勺,加冪,冪之繪以云,惟設而不酌者無勺。



 三曰籩豆登俎。昊天上帝、皇地祇及配帝,籩豆皆十二,登三,簋二,簠二,俎八,皆有匕箸,玉幣篚二,匏爵一,有坫,沙池一,青瓷牲盤一。從祀九位。籩豆皆八,簠一,簋一,登一,俎一,匏爵一,有坫,沙池一,玉幣篚一。內官位五十四,籩豆皆二,簋一,簠一,登一,俎一,匏爵有坫,沙池,幣篚,十二次各一。中官百五十八,皆籩一,豆一,簋一,簠一,俎一,匏爵有坫,沙池,幣篚,十二次各一。外官位一百六,皆籩一,豆一,簋一,簠一,俎一,匏爵,沙池,幣篚,十二次各一。眾星位三百六十,皆籩一,豆一,簋一,簠一,俎一,匏爵,沙池,幣篚,十二次各一。此籩、豆、簠、簋、登、爵、篚之數也。凡籩之設,居神位左,豆居右,登、簠簋居中,俎居後,籩皆有巾,巾之繪以斧。



 四曰酒齊。以太尊實泛齊,著尊實醴齊,犧尊實盎齊,山罍實三酒,皆有上尊。馬湩設於尊罍之前,注于器而LV之。設而不酌者,以象尊實醴齊,壺尊實沈齊,山罍二實三酒,皆有上尊,以祀昊天上帝。皇地祇亦如之。以著尊實泛齊,犧尊實醴齊,象尊實盎齊,山罍實清酒,皆有上尊。馬湩如前設之。設而不酌者,以犧尊實醍齊,壺尊實沈齊,山罍三實清酒,皆有上尊,以祀配帝。以太尊實泛齊,以著尊實醍齊,皆有上尊,九位同,以祀五帝、日月、北極、天皇大帝。以象尊實醴齊,有上尊,十二次同,以祀內官。以壺尊實沈齊,有上尊,十二次同,以祀中官。以概尊實清酒,有上尊,十二次同,以祀外官。以散尊實昔酒,有上尊,十二次同,以祀眾星。凡五齊之上尊,必皆實明水;山罍之上尊,必皆實玄酒;散尊之上尊,亦實明水。



 五曰牲齊庶器。昊天上帝蒼犢,皇地祇黃犢,配位蒼犢,大明青犢,夜明白犢,天皇大帝蒼犢,北極玄犢皆一,馬純色一,鹿十有八,羊十有八,野豕十有八,兔十有二,蓋參以國禮。割牲為七體:左肩臂臑兼代脅、長脅為一體,右肩臂臑、代脅、長脅為一體,右髀肫胳為一體,脊連背膚短脅為一體,膺骨臍腹為一體,項脊為一體,馬首報陽升煙則用之。毛血盛以豆,或青瓷盤,饌未入置俎上,饌入徹去之。籩之實,魚鱐、糗餌、粉糍、棗、乾尞、形鹽、鹿脯、榛、桃、菱、芡、慄。豆之實,芹菹、韭菹、菁菹、筍菹、脾折菹、裛食、魚醢、豚拍、鹿MZ、醓醢、糝食。凡籩之用八者,無糗餌、粉糍、菱、慄。豆之用八者,無脾折菹、裛食、兔醢、糝食。用皆二者,籩以鹿脯、乾棗,豆以鹿MZ、菁菹。用皆一者,籩以鹿脯,豆以鹿臡。凡簠簋用皆二者,簋以黍、稷,簠以稻、粱;用皆一者,簋以稷,簠以黍。實登以大羹。



 六曰香祝。洗位正位香鼎一,香合一,食案一,祝案一,皆有衣,拜褥一,盥爵洗位一,罍一,洗一,白羅巾一,親祀匜二,盤二。地祇配位咸如之。香用龍腦沉香。祝版長各二尺四寸,闊一尺二寸,厚三分,木用楸柏。從祀九位,香鼎、香合、香案、綾拜褥皆九,褥各隨其方之色,盥爵洗位二,罍二,洗二,巾二。第二等,盥爵洗位二,罍二,洗二,巾二。第三等亦如之。內壝內,盥爵洗位一,罍一,洗一,巾一。內壝外亦如之。凡巾,皆有篚。從祀而下,香用沈檀降真,鼎用陶瓦。第二等十二次而下,皆紫綾拜褥十有二。親祀禦板位一,飲福位及大小次盥洗爵洗板位各一,皆青質金書。亞獻、終獻飲福板位一,黑質黃書。禦拜褥八,亞終獻飲福位拜褥一,黃道裀褥寶案二,黃羅銷金案衣,水火鑒。



 七曰燭燎。天壇椽燭四,皆銷金絳紗籠。自天壇至內壝外及樂縣南北通道,絳燭三百五十,素燭四百四十,皆絳紗籠。御位椽,燭六,銷金絳紗籠。獻官椽燭四,雜用燭八百,凡盆二百二十,有架。黃桑條去膚一車,束之置燎壇,以焚牲首。



 八曰獻攝執事。亞獻官一,終獻官一,攝司徒一,助奠官一,大禮使一,侍中二,門下侍郎二,禮儀使二,殿中監二,尚輦官二,太僕卿二,控馬官六,近侍官八,導駕官二十有四,典寶官四,侍儀官五,太常卿丞八,光祿卿丞二,刑部尚書二,禮部尚書二,奉玉幣官一,定撰祝文官一,書讀祝冊官二,舉祝冊官二,太史令一,御奉爵官一,奉匜盤官二,御爵洗官二,執巾官二,割牲官二,溫酒官一,太官令一,太官丞一,良醖令丞二,廩犧令丞二,糾儀御史四,太常博士二,郊祀令丞二,太樂令一,太樂丞一,司尊罍二,亞終獻盥洗官二,爵洗官二,巾篚官二,奉爵官二,祝史四,太祝十有五,奉禮郎四,協律郎二,翦燭官四,禮直官管勾一,禮部點視儀衛官二,兵部清道官二,拱衛使二,大都兵馬使二,齋郎百,司天生二,看守凡盆軍官一百二十。



\end{pinyinscope}