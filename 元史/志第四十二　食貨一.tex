\article{志第四十二 食貨一}

\begin{pinyinscope}

 《洪範》八政,食為首而貨次之,蓋食貨者養生之源也。民非食貨則無以為生,國非食貨則無以為用。是以古之善治其國者,不能無取於民集》十五卷,現存十卷,以魯迅校本為善。參見「文學」、,亦未嘗過取於民,其大要在乎量入為出而已。《傳》曰:「生財有大道,生之者眾,食之者寡,為之者疾,用之者舒。」此先王理財之道也。後世則不然,以漢、唐、宋觀之,當其立國之初,亦頗有成法,及數傳之後,驕侈生焉,往往取之無度,用之無節。於是漢有告緡、算舟車之令,唐有借商、稅間架之法,宋有經、總制二錢,皆掊民以充國,卒之民困而國亡,可嘆也已。



 元初,取民未有定制。及世祖立法,一本於寬。其用之也,於宗戚則有歲賜,於兇荒則有賑恤,大率以親親愛民為重,而尤惓藐於農桑一事,可謂知理財之本者矣。世祖嘗語中書省臣曰:「凡賜與雖有朕命,中書其斟酌之。」成宗亦嘗謂丞相完澤等曰:「每歲天下金銀鈔幣所入幾何?諸王駙馬賜與及一切營建所出幾何?其會計以聞。」完澤對曰:「歲入之數,金一萬九千兩,銀六萬兩,鈔三百六十萬錠,然猶不足於用,又於至元鈔本中借二十萬錠矣。自今敢以節用為請。」帝嘉納焉。世稱元之治以至元、大德為首者,蓋以此。



 自時厥後,國用浸廣。除稅糧、科差二者之外,凡課之入,日增月益。至於天歷之際,視至元、大德之數,蓋增二十倍矣,而朝廷未嘗有一日之蓄,則以其不能量入為出故也。雖然,前代告緡、借商、經總等制,元皆無之,亦可謂寬矣。其能兼有四海,傳及百年者,有以也夫。故仿前史之法,取其出入之制可考者:一曰經理,二曰農桑,三曰稅糧,四曰科差,五曰海運,六曰鈔法,七曰歲課,八曰鹽法,九曰茶法,十曰酒醋課,十有一曰商稅,十有二曰市舶,十有三曰額外課,十有四曰歲賜,十有五曰俸秩,十有六曰常平義倉,十有七曰惠民藥局,十有八曰市糴,十有九曰賑恤,具著於篇,作《食貨志》。



 經理



 經界廢而後有經理,魯之履畝,漢之核田,皆其制也。夫民之強者田多而稅少,弱者產去而稅存,非經理固無以去其害;然經理之制,茍有不善,則其害又將有甚焉者矣。



 仁宗延祐元年,平章章閭言:「經理大事,世祖已嘗行之,但其間欺隱尚多,未能盡實。以熟田為荒地者有之,懼差而析戶者有之,富民買貧民田而仍其舊名輸稅者亦有之。由是歲入不增,小民告病。若行經理之法,俾有田之家,及各位下、寺觀、學校、財賦等田,一切從實自首,庶幾稅入無隱,差徭亦均。」於是遣官經理。以章閭等往江浙,尚書你咱馬丁等往江西,左丞陳士英等往河南,仍命行御史臺分臺鎮遏,樞密院以軍防護焉。



 其法先期揭榜示民,限四十日,以其家所有田,自實於官。或以熟為荒,以田為蕩,或隱占逃亡之產,或盜官田為民田,指民田為官田,及僧道以田作弊者,並許諸人首告。十畝以下,其田主及管幹佃戶皆杖七十七。二十畝以下,加一等。一百畝以下,一百七;以上,流竄北邊,所隱田沒官。郡縣正官不為查勘,致有脫漏者,量事論罪,重者除名。此其大略也。



 然期限猝迫,貪刻用事,富民黠吏,並緣為奸,以無為有,虛具於籍者,往往有之。於是人不聊生,盜賊並起,其弊反有甚於前者。仁宗知之,明年,遂下詔免三省自實田租。二年,時汴梁路總管塔海亦言其弊,於是命河南自實田,自延祐五年為始,每畝止科其半,汴梁路凡減二十二萬餘石。至泰定、天歷之初,又盡革虛增之數,民始獲安。今取其數之可考者,列於後云:



 河南省,總計官民荒熟田一百一十八萬七百六十九頃。



 江西省,總計官民荒熟田四十七萬四千六百九十三頃。



 江浙省,總計官民荒熟田九十九萬五千八十一頃。



 農桑



 農桑,王政之本也。太祖起朔方,其俗不待蠶而衣,不待耕而食,初無所事焉。世祖即位之初,首詔天下,國以民為本,民以衣食為本,衣食以農桑為本。於是頒《農桑輯要》之書於民,俾民崇本抑末。其睿見英識,與古先帝王無異,豈遼、金所能比哉?



 中統元年,命各路宣撫司擇通曉農事者,充隨處勸農官。二年,立勸農司,以陳邃、崔斌等八人為使。至元七年,立司農司,以左丞張文謙為卿。司農司之設,專掌農桑水利。仍分布勸農官及知水利者,巡行郡邑,察舉勤惰。所在牧民長官提點農事,歲終第其成否,轉申司農司及戶部,秩滿之日,注于解由,戶部照之,以為殿最。又命提刑按察司加體察焉。其法可謂至矣。



 是年,又頒農桑之制一十四條,條多不能盡載,載其所可法者:縣邑所屬村畽,凡五十家立一社,擇高年曉農事者一人為之長。增至百家者,別設長一員。不及五十家者,與近村合為一社。地遠人稀,不能相合,各自為社者聽。其合為社者,仍擇數村之中,立社長官司長以教督農民為事。凡種田者,立牌橛於田側,書某社某人於其上,社長以時點視勸誡。不率教者,籍其姓名,以授提點官責之。其有不敬父兄及兇惡者,亦然。仍大書其所犯於門,俟其改過自新乃毀,如終歲不改,罰其代充本社夫役。社中有疾病兇喪之家不能耕種者,眾為合力助之。一社之中災病多者,兩社助之。凡為長者,復其身,郡縣官不得以社長與科差事。農桑之術,以備旱為先。凡河渠之利,委本處正官一員,以時浚治。或民力不足者,提舉河渠官相其輕重,官為導之。地高水不能上者,命造水車。貧不能造者,官具材木給之。俟秋成之後,驗使水之家,俾均輸其直。田無水者鑿井,井深不能得水者,聽種區田。其有水田者,不必區種。仍以區田之法,散諸農民。種植之制,每丁歲種桑棗二十株。土性不宜者,聽種榆柳等,其數亦如之。種雜果者,每丁十株,皆以生成為數,願多種者聽。其無地及有疾者不與。所在官司申報不實者,罪之。仍令各社布種苜蓿,以防饑年。近水之家,又許鑿池養魚並鵝鴨之數,及種蒔蓮藕、雞頭、菱角、蒲葦等,以助衣食。凡荒閑之地,悉以付民,先給貧者,次及餘戶。每年十月,令州縣正官一員,巡視境內,有蟲蝗遺子之地,多方設法除之。其用心周悉若此,亦仁矣哉!



 九年,命勸農官舉察勤惰。於是高唐州官以勤升秩,河南陜縣尹王仔以惰降職。自是每歲申明其制。十年,令探馬赤隨處入社,與編民等。二十五年,立行大司農司及營田司於江南。二十八年,頒農桑雜令。是年,又以江南長吏勸課擾民,罷其親行之制,命止移文諭之。二十九年,以勸農司並入各道肅政廉訪司,增僉事二員,兼察農事。是年八月,又命提調農桑官帳冊有差者,驗數罰俸。故終世祖之世,家給人足。天下為戶凡一千一百六十三萬三千二百八十一,為口凡五千三百六十五萬四千三百三十七,此其敦本之明效可睹也已。



 成宗大德元年,罷妨農之役。十一年,申擾農之禁,力田者有賞,游惰者有罰,縱畜牧損禾稼桑棗者,責其償而後罪之。由是大德之治,幾於至元。然旱霖雨之災迭見,饑毀薦臻,民之流移失業者亦已多矣。



 武宗至大二年,淮西廉訪僉事苗好謙獻種蒔之法。其說分農民為三等,上戶地一十畝,中戶五畝,下戶二畝或一畝,皆築垣墻圍之,以時收採桑椹,依法種植。武宗善而行之。其法出《齊民要術》等書,茲不備錄。三年,申命大司農總挈天下農政,修明勸課之令,除牧養之地,其餘聽民秋耕。



 仁宗皇慶二年,復申秋耕之令,惟大都等五路許耕其半。蓋秋耕之利,掩陽氣於地中,蝗蝻遺種皆為日所曝死,次年所種,必盛於常禾也。延祐三年,以好謙所至,植桑皆有成效,於是風示諸道,命以為式。是年十一月,令各社出地,共蒔桑苗,以社長領之,分給各社。四年,又以社桑分給不便,令民各畦種之。法雖屢變,而有司不能悉遵上意,大率視為具文而已。五年,大司農司臣言:「廉訪司所具栽植之數,書於冊者,類多不實。」觀此,則惰於勸課者,又不獨有司為然也。致和之後,莫不申明農桑之令。天歷二年,各道廉訪司所察勤官內丘何主簿等凡六人,惰官濮陽裴縣尹等凡四人。其可考者,蓋止於此云。



 稅糧



 元之取民,大率以唐為法。其取於內郡者,曰丁稅,曰地稅,此仿唐之租庸調也。取於江南者,曰秋稅,曰夏稅,此仿唐之兩稅也。



 丁稅、地稅之法,自太宗始行之。初,太宗每戶科粟二石,後又以兵食不足,增為四石。至丙申年,乃定科征之法,令諸路驗民戶成丁之數,每丁歲科粟一石,驅丁五升,新戶丁驅各半之,老幼不與。其間有耕種者,或驗其牛具之數,或驗其土地之等征焉。丁稅少而地稅多者納地稅,地稅少而丁稅多者納丁稅。工匠僧道驗地,官吏商賈驗丁。虛配不實者杖七十,徒二年。仍命歲書其數於冊,由課稅所申省以聞,違者各杖一百。逮及世祖,申明舊制,於是輸納之期、收受之式、關防之禁、會計之法,莫不備焉。



 中統二年,遠倉之糧,命止於沿河近倉輸納,每石帶收腳錢中統鈔三錢,或民戶赴河倉輸納者,每石折輸輕齎中統鈔七錢。五年,詔僧、道、也裏可溫、答失蠻、儒人凡種田者,白地每畝輸稅三升,水地每畝五升。軍、站戶除地四頃免稅,餘悉徵之。至元三年,詔窵戶種田他所者,其丁稅於附籍之郡驗丁而科,地稅於種田之所驗地而取。漫散之戶逃於河南等路者,依見居民戶納稅。八年,又定西夏中興路、西寧州、兀剌海三處之稅,其數與前僧道同。



 十七年,遂命戶部大定諸例:全科戶丁稅,每丁粟三石,驅丁粟一石,地稅每畝粟三升。減半科戶丁稅,每丁粟一石。新收交參戶,第一年五斗,第三年一石二斗五升,第四年一石五斗,第五年一石七斗五升,第六年入丁稅。協濟戶丁稅,每丁粟一石,地稅每畝粟三升。隨路近倉輸粟,遠倉每粟一石,折納輕齎鈔二兩。富戶輸遠倉,下戶輸近倉,郡縣各差正官一員部之,每石帶納鼠耗三升,分例四升。凡糧到倉,以時收受,出給硃錢。權勢之徒結攬稅石者罪之,仍令倍輸其數。倉官、攢典、斗腳人等飛鈔作弊者,並置諸法。輸納之期,分為三限:初限十月,中限十一月,末限十二月。違者,初犯笞四十,再犯杖八十。成宗大德六年,申明稅糧條例,復定上都、河間輸納之期:上都,初限次年五月,中限六月,末限七月。河間,初限九月,中限十月,末限十一月。



 秋稅、夏稅之法,行於江南。初,世祖平宋時,除江東、浙西,其餘獨征秋稅而已。至元十九年,用姚元之請,命江南稅糧依宋舊例,折輸綿絹雜物。是年二月,又用耿左丞言,令輸米三之一,餘並人鈔以折焉。以七百萬錠為率,歲得羨鈔十四萬錠。其輸米者,止用宋斗斛,蓋以宋一石當今七斗故也。二十八年,又命江淮寺觀田,宋舊有者免租,續置者輸稅,其法亦可謂寬矣。



 成宗元貞二年,始定征江南夏稅之制。於是秋稅止命輸租,夏稅則輸以木綿布絹絲綿等物。其所輸之數,視糧以為差。糧一石或輸鈔三貫、二貫、一貫,或一貫五百文、一貫七百文。輸三貫者,若江浙省婺州等路、江西省龍興等路是已。輸二貫者,若福建省泉州等五路是已。輸一貫五百文者,若江浙省紹興路、福建省漳州等五路是已。皆因其地利之宜,人民之眾,酌其中數而取之。其折輸之物,各隨時估之高下以為直,獨湖廣則異於是。初,阿里海牙克湖廣時,罷宋夏稅,依中原例,改科門攤,每戶一貫二錢,蓋視夏稅增鈔五萬餘錠矣。大德二年,宣慰張國紀請科夏稅,於是湖、湘重罹其害。俄詔罷之。三年,又改門攤為夏稅而並徵之,每石計三貫四錢之上,視江浙、江西為差重雲。其在官之田,許民佃種輸租。江北、兩淮等處荒閑之地,第三年始輸。大德四年,又以地廣人稀更優一年,令第四年納稅。凡官田,夏稅皆不科。



 泰定之初,又有所謂助役糧者。其法命江南民戶有田一頃之上者,於所輸稅外,每頃量出助役之田,具書於冊,里正以次掌之,歲收其入,以助充役之費。凡寺觀田,除宋舊額,其餘亦驗其多寡令出田助役焉。民賴以不困,因並著於此云。



 天下歲入糧數,總計一千二百十一萬四千七百八石。



 腹裏,二百二十七萬一千四百四十九石。



 行省,九百八十四萬三千二百五十八石。



 遼陽省七萬二千六十六石。



 河南省二百五十九萬一千二百六十九石。



 陜西省二十二萬九千二十三石。



 四川省一十一萬六千五百七十四石。



 甘肅省六萬五百八十六石。



 雲南省二十七萬七千七百一十九石。



 江浙省四百四十九萬四千七百八十三石。



 江西省一百一十五萬七千四百四十八石。



 湖廣省八十四萬三千七百八十七石。



 江南三省天歷元年夏稅鈔數,總計中統鈔一十四萬九千二百七十三錠三十三貫。



 江浙省五萬七千八百三十錠四十貫。



 江西省五萬二千八百九十五錠一十一貫。



 湖廣省一萬九千三百七十八錠二貫。



 科差



 科差之名有二,曰絲料,曰包銀,其法各驗其戶之上下而科焉。絲料之法,太宗丙申年始行之。每二戶出絲一斤,並隨路絲線、顏色輸於官;五戶出絲一斤,並隨路絲線、顏色輸於本位。包銀之法,憲宗乙卯年始定之。初漢民科納包銀六兩,至是止征四兩,二兩輸銀,二兩折收絲絹、顏色等物。逮及世祖,而其制益詳。



 中統元年,立十路宣撫司,定戶籍科差條例。然其戶大抵不一,有元管戶、交參戶、漏籍戶、協濟戶。於諸戶之中,又有絲銀全科戶、減半科戶、止納絲戶、止納鈔戶;外又有攤絲戶、儲也速鷿兒所管納絲戶、復業戶,並漸成丁戶。戶既不等,數亦不同。元管戶內,絲銀全科系官戶,每戶輸系官絲一斤六兩四錢、包銀四兩;全科系官五戶絲戶,每戶輸系官絲一斤、五戶絲六兩四錢,包銀之數與系官戶同;減半科戶,每戶輸系官絲八兩、五戶絲三兩二錢、包銀二兩;止納系官絲戶,若上都、隆興、西京等路十戶十斤者,每戶輸一斤,大都以南等路十戶十四斤者,每戶輸一斤六兩四錢;止納系官五戶絲戶,每戶輸系官絲一斤、五戶絲六兩四錢。交參戶內,絲銀戶每戶輸系官絲一斤六兩四錢、包銀四兩。漏籍戶內,止納絲戶每戶輸絲之數,與交參絲銀戶同;止納鈔戶,初年科包銀一兩五錢,次年遞增五錢,增至四兩,並科絲料。協濟戶內,絲銀戶每戶輸系官絲十兩二錢、包銀四兩;止納絲戶,每戶輸系官絲之數,與絲銀戶同。攤絲戶,每戶科攤絲四斤。儲也速鷿兒所管戶,每戶科細絲,其數與攤絲同。復業戶並漸成丁戶,初年免科,第二年減半,第三年全科,與舊戶等。然絲料、包銀之外,又有俸鈔之科,其法亦以戶之高下為等,全科戶輸一兩,減半戶輸五錢。於是以合科之數,作大門攤,分為三限輸納。被災之地,聽輸他物折焉,其物各以時估為則。凡儒士及軍、站、僧、道等戶皆不與。



 二年,復定科差之期,絲料限八月,包銀初限八月,中限十月,末限十二月。三年,又命絲料無過七月,包銀無過九月。及平江南,其制益廣。至元二十八年,以《至元新格》定科差法,諸差稅皆司縣正官監視人吏置局均科。諸夫役皆先富強,後貧弱;貧富等者,先多丁,後少丁。



 成宗大德六年,又命止輸絲戶每戶科俸鈔中統鈔一兩,包銀戶每戶科二錢五分,攤絲戶每戶科攤絲五斤八兩;絲料限八月,包銀、俸鈔限九月,布限十月。大率因世祖之舊而增損云。



 科差總數:



 中統四年,絲七十一萬二千一百七十一斤,鈔五萬六千一百五十八百錠。



 至元二年,絲九十八萬六千九百一十二斤,包銀等鈔五萬六千八百七十四錠,布八萬五千四百一十二匹。



 至元三年,絲一百五萬三千二百二十六斤,包銀等鈔五萬九千八十五錠。



 至元四年,絲一百九萬六千四百八十九斤,鈔七萬八千一百二十六錠。



 天歷元年,包銀差發鈔九百八十九錠,一百一十三萬三千一百一十九索,絲一百九萬八千八百四十三斤,絹三十五萬五百三十匹,綿七萬二千一十五斤,布二十一萬一千二百二十三匹。



 海運



 元都於燕,去江南極遠,而百司庶府之繁,衛士編民之眾,無不仰給於江南。自丞相伯顏獻海運之言,而江南之糧分為春夏二運。蓋至於京師者一歲多至三百萬餘石,民無挽輸之勞,國有儲蓄之富,豈非一代之良法歟!



 初,伯顏平江南時,嘗命張瑄、硃清等,以宋庫藏圖籍,自崇明州從海道載入京師。而運糧則自浙西涉江入淮,由黃河逆水至中灤旱站,陸運至淇門,入御河,以達於京。後又開濟州泗河,自淮至新開河,由大清河至利津,河入海,因海口沙壅,又從東阿旱站運至臨清,入御河。又開膠、萊河道通海,勞費不貲,卒無成效。至元十九年,伯顏追憶海道載宋圖籍之事,以為海運可行,於是請於朝廷,命上海總管羅璧、硃清、張瑄等,造平底海船六十艘,運糧四萬六千餘石,從海道至京師。然創行海洋,沿山求奧,風信失時,明年始至直沽。時朝廷未知其利,是年十二月立京畿、江淮都漕運司二,仍各置分司,以督綱運。每歲令江淮漕運司運糧至中灤,京畿漕運司自中灤運至大都。二十年,又用王積翁議,命阿八赤等廣開新河。然新河候潮以入,船多損壞,民亦苦之。而忙兀鷿言海運之舟悉皆至焉。於是罷新開河,頗事海運,立萬戶府二,以硃清為中萬戶,張瑄為千戶,忙兀鷿為萬戶府達魯花赤。未幾,又分新河軍士水手及船,於揚州、平灤兩處運糧,命三省造船三千艘於濟州河運糧,猶未專於海道也。



 二十四年,始立行泉府司,專掌海運,增置萬戶府二,總為四府。是年遂罷東平河運糧。二十五年,內外分置漕運司二。其在外者於河西務置司,領接運海道糧事。二十八年,又用硃清、張瑄之請,並四府為都漕運萬戶府二,止令清、瑄二人掌之。其屬有千戶、百戶等官,分為各翼,以督歲運。



 至大四年,遣官至江浙議海運事。時江東寧國、池、饒、建康等處運糧,率令海船從揚子江逆流而上。江水湍急,又多石磯,走沙漲淺,糧船俱壞,歲歲有之。又湖廣、江西之糧運至真州泊入海船,船大底小,亦非江中所宜。於是以嘉興、松江秋糧,並江淮、江浙財賦府歲辦糧充運。海漕之利,蓋至是博矣。



 凡運糧,每石有腳價鈔。至元二十一年,給中統鈔八兩五錢,其後遞減至於六兩五錢。至大三年,以福建、浙東船戶至平江載糧者,道遠費廣,通增為至元鈔一兩六錢,香糯一兩七錢。四年,又增為二兩,香糯二兩八錢,稻穀一兩四錢。延祐元年,斟酌遠近,復增其價。福建船運糙粳米每石一十三兩,溫、臺、慶元船運糙粳、香糯每石一十一兩五錢,紹興、浙西船每石一十一兩,白粳價同,稻穀每石八兩,黑豆每石依糙白糧例給焉。



 初,海運之道,自平江劉家港入海,經揚州路通州海門縣黃連沙頭、萬里長灘開洋,沿山奧而行,抵淮安路鹽城縣,歷西海州、海寧府東海縣、密州、膠州界,放靈山洋投東北,路多淺沙,行月餘始抵成山。計其水程,自上海至楊村馬頭,凡一萬三千三百五十里。至元二十九年,硃清等言其路險惡,復開生道。自劉家港開洋,至撐腳沙轉沙觜,至三沙、洋子江,過匾擔沙、大洪,又過萬里長灘,放大洋至青水洋,又經黑水洋至成山,過劉島,至芝罘、沙門二島,放萊州大洋,抵界河口,其道差為徑直。明年,千戶殷明略又開新道,從劉家港入海,至崇明州三沙放洋,向東行,入黑水大洋,取成山轉西至劉家島,又至登州沙門島,於萊州大洋入界河。當舟行風信有時,自浙西至京師,不過旬日而已,視前二道為最便云。然風濤不測,糧船漂溺者無歲無之,間亦有船壞而棄其米者。至元二十三年始責償於運官,人船俱溺者乃免。然視河漕之費,則其所得蓋多矣。



 歲運之數:



 至元二十年,四萬六千五十石,至者四萬二千一百七十二石。二十一年,二十九萬五百石,至者二十七萬五千六百一十石。二十二年,一十萬石,至者九萬七百七十一石。二十三年,五十七萬八千五百二十石,至者四十三萬三千九百五石。二十四年,三十萬石,至者二十九萬七千五百四十六石。二十五年,四十萬石,至者三十九萬七千六百五十五石。二十六年,九十三萬五千石,至者九十一萬九千九百四十三石。二十七年,一百五十九萬五千石,至者一百五十一萬三千八百五十六石。二十八年,一百五十二萬七千二百五十石,至者一百二十八萬一千六百一十五石。二十九年,一百四十萬七千四百石,至者一百三十六萬一千五百一十三石。三十年,九十萬八千石,至者八十八萬七千五百九十一石。三十一年,五十一萬四千五百三十三石,至者五十萬三千五百三十四石。



 元貞元年,三十四萬五百石。二年,三十四萬五百石,至者三十三萬七千二十六石。



 大德元年,六十五萬八千三百石,至者六十四萬八千一百三十六石。二年,七十四萬二千七百五十一石,至者七十萬五千九百五十四石。三年,七十九萬四千五百石。四年,七十九萬五千五百石,至者七十八萬八千九百一十八石。五年,七十九萬六千五百二十八石,至者七十六萬九千六百五十石。六年,一百三十八萬三千八百八十三石,至者一百三十二萬九千一百四十八石。七年,一百六十五萬九千四百九十一石,至者一百六十二萬八千五百八石。八年,一百六十七萬二千九百九石,至者一百六十六萬三千三百一十三石。九年,一百八十四萬三千三石,至者一百七十九萬五千三百四十七石。十年,一百八十萬八千一百九十九石,至者一百七十九萬七千七十八石。十一年,一百六十六萬五千四百二十二石,至者一百六十四萬四千六百七十九石。



 至大元年,一百二十四萬一百四十八石,至者一百二十萬二千五百三石。二年,二百四十六萬四千二百四石,至者二百三十八萬六千三百石。三年,二百九十二萬六千五百三十三石,至者二百七十一萬六千九百十三石。四年,二百八十七萬三千二百一十二石,至者二百七十七萬三千二百六十六石。



 皇慶元年,二百八萬三千五百五石,至者二百六萬七千六百七十二石。二年,二百三十一萬七千二百二十八石,至者二百一十五萬八千六百八十五石。



 延祐元年,二百四十萬三千二百六十四石,至者二百三十五萬六千六百六石。二年,二百四十三萬五千六百八十五石,至者二百四十二萬二千五百五石。三年,二百四十五萬八千五百一十四石,至者二百四十三萬七千七百四十一石。四年,二百三十七萬五千三百四十五石,至者二百三十六萬八千一百一十九石。五年,二百五十五萬三千七百一十四石,至者二百五十四萬三千六百一十一石。六年,三百二萬一千五百八十五石,至者二百九十八萬六千一十七石。七年,三百二十六萬四千六石,至者三百二十四萬七千九百二十八石。



 至治元年,三百二十六萬九千四百五十一石,至者三百二十三萬八千七百六十五石。二年,三百二十五萬一千一百四十石,至者三百二十四萬六千四百八十三石。三年,二百八十一萬一千七百八十六石,至者二百七十九萬八千六百一十三石。



 泰定元年,二百八萬七千二百三十一石,至者二百七萬七千二百七十八石。二年,二百六十七萬一千一百八十四石,至者二百六十三萬七千五十一石。三年,三百三十七萬五千七百八十四石,至者三百三十五萬一千三百六十二石。四年,三百一十五萬二千八百二十石,至者三百一十三萬七千五百三十二石。



 天歷元年,三百二十五萬五千二百二十石,至者三百二十一萬五千四百二十四石。二年,三百五十二萬二千一百六十三石,至者三百三十四萬三百六石。



 鈔法



 鈔始於唐之飛錢、宋之交會、金之交鈔。其法以物為母,鈔為子,子母相權而行,即《周官》質劑之意也。元初仿唐、宋、金之法,有行用鈔,其制無文籍可考。



 世祖中統元年,始造交鈔,以絲為本。每銀五十兩易絲鈔一千兩,諸物之直,並從絲例。是年十月,又造中統元寶鈔。其文以十計者四:曰一十文、二十文、三十文、五十文。以百計者三:曰一百文、二百文、五百文。以貫計者二:曰一貫文、二貫文。每一貫同交鈔一兩,兩貫同白銀一兩。又以文綾織為中統銀貨。其等有五:曰一兩、二兩、三兩、五兩、十兩。每一兩同白銀一兩,而銀貨蓋未及行雲。五年,設各路平準庫,主平物價,使相依準,不至低昂,仍給鈔一萬二千錠,以為鈔本。至元十二年,添造厘鈔。其例有三:曰二文、三文、五文。初,鈔印用木為版,十三年鑄銅易之。十五年,以厘鈔不便於民,復命罷印。



 然元寶、交鈔行之既久,物重鈔輕。二十四年,遂改造至元鈔,自二貫至五文,凡十有一等,與中統鈔通行。每一貫文當中統鈔五貫文。依中統之初,隨路設立官庫,貿易金銀,平準鈔法。每花銀一兩,入庫其價至元鈔二貫,出庫二貫五分,赤金一兩,入庫二十貫,出庫二十貫五百文。偽造鈔者處死,首告者賞鈔五錠,仍以犯人家產給之。其法為最善。



 至大二年,武宗復以物重鈔輕,改造至大銀鈔,自二兩至二厘定為一十三等。每一兩準至元鈔五貫,白銀一兩,赤金一錢。元之鈔法,至是蓋三變矣。大抵至元鈔五倍於中統,至大鈔又五倍於至元。然未及期年,仁宗即位,以倍數太多,輕重失宜,遂有罷銀鈔之詔。而中統、至元二鈔,終元之世,蓋常行焉。



 凡鈔之昏爛者,至元二年,委官就交鈔庫,以新鈔倒換,除工墨三十文。三年,減為二十文。二十二年,復增如故。其貫伯分明,微有破損者,並令行用,違者罪之。所倒之鈔,每季各路就令納課正官,解赴省部焚毀,隸行省者就焚之。大德二年,戶部定昏鈔為二十五樣。泰定四年,又定焚毀之所,皆以廉訪司官監臨,隸行省者,行省官同監。其制之大略如此。



 若錢,自九府圜法行於成周,歷代未嘗或廢。元之交鈔、寶鈔雖皆以錢為文,而錢則弗之鑄也。武宗至大三年,初行錢法,立資國院、泉貨監以領之。其錢曰至大通寶者,一文準至大銀鈔一厘;曰大元通寶者,一文準至大通寶錢一十文。歷代銅錢,悉依古例,與至大錢通用。其當五、當三、折二,並以舊數用之。明年,仁宗復下詔,以鼓鑄弗給,新舊資用,其弊滋甚,與銀鈔皆廢不行,所立院、監亦皆罷革,而專用至元、中統鈔云。



 歲印鈔數:



 中統元年,中統鈔七萬三千三百五十二錠。二年,中統鈔三萬九千一百三十九錠。三年,中統鈔八萬錠。四年,中統鈔七萬四千錠。



 至元元年,中統鈔八萬九千二百八錠。二年,中統鈔一十一萬六千二百八錠。三年,中統鈔七萬七千二百五十二錠。四年,中統鈔一十萬九千四百八十八錠。五年,中統鈔二萬九千八百八十錠。六年,中統鈔二萬二千八百九十六錠。七年,中統鈔九萬六千七百六十八錠。八年,中統鈔四萬七千錠。九年,中統鈔八萬六千二百五十六錠。十年,中統鈔一十一萬一百九十二錠。十一年,中統鈔二十四萬七千四百四十錠。十二年,中統鈔三十九萬八千一百九十四錠。十三年,中統鈔一百四十一萬九千六百六十五錠。十四年,中統鈔一百二萬一千六百四十五錠。十五年,中統鈔一百二萬三千四百錠。十六年,中統鈔七十八萬八千三百二十錠。十七年,中統鈔一百一十三萬五千八百錠。十八年,中統鈔一百九萬四千八百錠。十九年,中統鈔九十六萬九千四百四十四錠。二十年,中統鈔六十一萬六百二十錠。二十一年,中統鈔六十二萬九千九百四錠。二十二年,中統鈔二百四萬三千八十錠。二十三年,中統鈔二百一十八萬一千六百錠。二十四年,中統鈔八萬三千二百錠,至元鈔一百萬一千一十七錠。二十五年,至元鈔九十二萬一千六百一十二錠。二十六年,至元鈔一百七十八萬九十三錠。二十七年,至元鈔五十萬二百五十錠。二十八年,至元鈔五十萬錠。二十九年,至元鈔五十萬錠。三十年,至元鈔二十六萬錠。三十一年,至元鈔一十九萬三千七百六錠。



 元貞元年,至元鈔三十一萬錠。二年,至元鈔四十萬錠。



 大德元年,至元鈔四十萬錠。二年,至元鈔二十九萬九千九百一十錠。三年,至元鈔九十萬七十五錠。四年,至元鈔六十萬錠。五年,至元鈔五十萬錠。六年,至元鈔二百萬錠。七年,至元鈔一百五十萬錠。八年,至元鈔五十萬錠。九年,至元鈔五十萬錠。十年,至元鈔一百萬錠。十一年,至元鈔一百萬錠。



 至大元年,至元鈔一百萬錠。二年,至元鈔一百萬錠。三年,至大銀鈔一百四十五萬三百六十八錠。四年,至元鈔二百一十五萬錠,中統鈔一十五萬錠。



 皇慶元年,至元鈔二百二十二萬二千三百三十六錠,中統鈔一十萬錠。二年,至元鈔二百萬錠,中統鈔二十萬錠。



 延祐元年,至元鈔二百萬錠,中統鈔一十萬錠。二年,至元鈔一百萬錠,中統鈔一十萬錠。三年,至元鈔四十萬錠,中統鈔一十萬錠。四年,至元鈔四十八萬錠,中統鈔一十萬錠。五年,至元鈔四十萬錠,中統鈔一十萬錠。六年,至元鈔一百四十八萬錠,中統鈔一十萬錠。七年,至元鈔一百四十八萬錠,中統鈔一十萬錠。



 至治元年,至元鈔一百萬錠,中統鈔五萬錠。二年,至元鈔八十萬錠,中統鈔五萬錠。三年,至元鈔七十萬錠,中統鈔五萬錠。



 泰定元年,至元鈔六十萬錠,中統鈔一十五萬錠。二年,至元鈔四十萬錠,中統鈔一十萬錠。三年,至元鈔四十萬錠,中統鈔一十萬錠。四年,至元鈔四十萬錠,中統鈔一十萬錠。



 天歷元年,至元鈔三十一萬九百二十錠,中統鈔三萬五百錠。二年,至元鈔一百一十九萬二千錠,中統鈔四萬錠。



\end{pinyinscope}