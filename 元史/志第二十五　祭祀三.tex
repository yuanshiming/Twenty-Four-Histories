\article{志第二十五 祭祀三}

\begin{pinyinscope}

 ○宗廟上



 其祖宗祭享之禮,割牲、奠馬湩,以蒙古巫祝致辭,蓋國俗也。世祖中統元年秋七月丁丑,設神位於中書省,用登歌樂,遣必闍赤致祭焉。必闍赤,譯言典書記者。十二月,初命制太廟祭器、法服。二年九月庚申朔,徙中書署,奉遷神主於聖安寺。辛巳,藏於瑞像殿。三年十二月癸亥,即中書省備三獻官,大禮使司徒攝祀事。禮畢,神主復藏瑞像殿。四年三月癸卯,詔建太廟於燕京。十一月丙戌,仍寓祀事中書,以親王合丹、塔察兒、王磐、張文謙攝事。



 至元元年冬十月,奉安神主於太廟,初定太廟七室之制。皇祖、皇祖妣第一室,皇伯考、伯妣第二室,皇考、皇妣第三室,皇伯考、伯妣第四室,皇伯考、伯妣第五室,皇兄、皇后第六室,皇兄、皇后第七室。凡室以西為上,以次而東。二年九月,初命滌養犧牲,取大樂工於東平,習禮儀。冬十月己卯,享於太廟,尊皇祖為太祖。三年秋九月,始作八室神主,設祏室。冬十月,太廟成。丞相安童、伯顏言:「祖宗世數、尊謚廟號、配享功臣、增祀四世、各廟神主、七祀神位、法服祭器等事,皆宜以時定」。乃命平章政事趙璧等集議,制尊謚廟號,定為八室。烈祖神元皇帝、皇曾祖妣宣懿皇后第一室,太祖聖武皇帝、皇祖妣光獻皇后第二室,太宗英文皇帝、皇伯妣昭慈皇后第三室,皇伯考術赤、皇伯妣別土出迷失第四室,皇伯考察合帶、皇伯妣也速倫第五室,皇考睿宗景襄皇帝、皇妣莊聖皇后第六室,定宗簡平皇帝、欽淑皇后第七室,憲宗桓肅皇帝、貞節皇后第八室。十一月戊申,奉安神主於祏室,歲用冬祀,如初禮。



 四年二月,初定一歲十二月薦新時物。六年冬,時享畢,十二月,命國師僧薦佛事於太廟七晝夜,始造木質金表牌位十有六,設大榻金椅奉安祏室前,為太廟薦佛事之始。七年十月癸酉,敕宗廟祝文書以國字。八年八月,太廟柱朽,從張易言,告於列室而後修,奉遷慄主金牌位與舊神主於饌幕殿,工畢安奉。自是修廟皆如之。丙子,敕冬享毋用犧牛。



 十二年五月,檢討張謙呈:「昔者因修太廟,奉遷金牌位於饌幕殿,設以金椅,其慄主卻與舊主牌位各貯箱內,安置金椅下,禮有非宜。今擬合以金牌位遷於八室內,其祏室慄主宜用彩輿遷納,舊主並牌位安置於箱為宜。」九月丁丑,敕太廟牲復用牛。十月己未,遷金牌位於八室內。太祝兼奉禮郎申屠致遠言:「竊見木主既成,又有金牌位,其日月山神主及中統初中書設祭神主,安奉無所。」博士議曰:「合存祏室慄主,舊置神主牌位,俱可隨時埋瘞,不致神有二歸。」太常少卿以聞,制曰:「其與張仲謙諸老臣議行之。」十三年九月丙申,薦佛事於太廟,命即佛事處便大祭。己亥,享於太廟,加薦羊鹿野豕。是歲,改作金主,太祖主題曰「成吉思皇帝」,睿宗題曰「太上皇也可那顏」,皇后皆題名諱。



 十四年八月乙丑,詔建太廟於大都。博士言:「古者廟制率都宮別殿,西漢亦各立廟,東都以中興崇儉,故七室同堂,後世遂不能革。」十五年五月九日,太常卿還自上都,為議廟制,據博士言同堂異室非禮,以古今廟制畫圖貼說,令博士李天麟齎往上都,分議可否以聞。



 一曰都宮別殿,七廟、九廟之制。《祭法》曰:「天子立七廟,三昭三穆與太祖之廟而七,諸侯、大夫、士降殺以兩。」晉博士孫毓以謂外為都宮,內各有寢廟,別有門垣。太祖在北,左昭右穆,以次而南是也。前廟後寢者,以象人君之居,前有朝而後有寢也。廟以藏主,以四時祭;寢有衣冠幾杖象生之具,以薦新物。天子太祖百世不遷,宗亦百世不遷,高祖以上,親盡則遞遷。昭常為昭,穆常為穆,同為都宮,則昭常在左,穆常在右,而外有以不失其序。一世自為一廟,則昭不見穆,穆不見昭,而內有以各全其尊,必祫享而會於太祖之廟,然後序其尊卑之次。蓋父子異宮,祖禰異廟,所以盡事亡如事存之義。然漢儒論七廟、九廟之數,其說有二。韋玄成等以謂周之所以七廟者,以後稷始封,文王、武王受命而王,是以三廟不毀,與親廟四而七也。如劉歆之說,則周自武王克商,以後稷為太祖,即增立高圉、亞圉二廟於公叔、太王、王季、文王二昭二穆之上,已為七廟矣。至懿王時始立文世室於三穆之上,至孝王時始立武世室於三昭之上,是為九廟矣。然先儒多是劉歆之說。



 二曰同堂異室之制。後漢明帝遵儉自抑,遺詔無起寢廟,但藏其主於光武廟中更衣別室。其後章帝又復如之,後世遂不敢加。而公私之廟,皆用同堂異室之制。先儒硃熹以謂至使太祖之位,下同孫子,而更僻處於一隅,無以見為七廟之尊;群廟之神,則又上厭祖考,不得自為一廟之主。以人情論之,生居九重,窮極壯麗,而設祭一室,不過尋丈,甚或無地以容鼎俎,而陰損其數,子孫之心,於此宜亦有所不安矣。且如命士以上,其父子婦姑,猶且異處,謹尊卑之序,不相褻瀆。況天子貴為一人,富有四海,而祖宗神位數世同處一堂,有失人子事亡如事存之意矣。



 十六年八月丁酉,以江南所獲玉爵及坫,凡四十九事,納於太廟。十七年十二月甲申,告遷於太廟。癸巳,承旨和禮霍孫,太常卿太出、禿忽思等,以祏室內慄主八位並日月山板位、聖安寺木主俱遷。甲午,和禮霍孫、太常卿撒里蠻率百官奉太祖、睿宗二室金主於新廟安奉,遂大享焉。乙未,毀舊廟。



 十八年二月,博士李時衍等議:「歷代廟制,俱各不同。欲尊祖宗,當從都宮別殿之制;欲崇儉約,當從同堂異室之制。」三月十一日,尚書段那海及太常禮官奏曰:「始議七廟,除正殿、寢殿、正門、東西門已建外,東西六廟不須更造,餘依太常寺新圖建之。」遂為前廟、後寢,廟分七室。二十一年三月丁卯,太廟正殿成,奉安神主。九月,廟室掛鐵綱釘鏨籠門告成。



 二十二年十二月丁未,皇太子薨。太常博士議曰:「前代太子薨,梁武帝謚統曰昭明,齊武帝謚長懋曰文惠,唐憲宗謚寧曰惠昭,金世宗謚允恭曰宣孝,又建別廟以奉神主,準中祀以陳登歌,例設令丞,歲供灑掃。斯皆累代之典,莫不追美洪休。」時中書、翰林諸老臣,亦議宜加謚,立別廟奉祀。遂謚曰明孝太子,作主用金。二十五年冬享,制送白馬一。三十年十月朔,皇太子祔於太廟。



 三十一年,成宗即位,追尊皇考為皇帝,廟號裕宗。元貞元年冬十月癸卯,有事於太廟。中書省臣言:「去歲世祖、皇后、裕宗祔廟,以綾代玉冊。今玉冊、玉寶成,請納諸各室。」帝曰:「親饗之禮,祖宗未嘗行之。其奉冊以來,朕躬祝之。」命獻官迎導入廟。大德元年十一月,太保月赤察兒等奏請廟享增用馬,制可。二年正月,特祭太廟,用馬一,牛一,羊鹿野豕天鵝各七,餘品如舊,為特祭之始。四年八月,以皇妣、皇后祔。六年五月戊申,太廟寢殿災。



 十一年,武宗即位,追尊皇考為皇帝,廟號順宗。太祖室居中,睿宗西第一室,世祖西第二室,裕宗西第三室,順宗東第一室,成宗東第二室。追尊先元妃為皇后,祔成宗室。至大二年春正月乙未,以受尊號,恭謝太廟,為親祀之始。十月,以將加謚太祖、睿宗,擇日請太祖、睿宗尊謚於天,擇日請光獻皇后、莊聖皇后尊謚於廟,改制金表神主,題寫尊謚廟號。十二月乙卯,親享太廟,奉玉冊、玉寶。加上太祖聖武皇帝尊謚曰法天啟運,廟號太祖,光獻皇后曰翼聖。加上睿宗景襄皇帝曰仁聖,廟號睿宗,莊聖皇后曰顯懿。其舊制,金表神主,以櫝貯兩旁,自是主皆範金作之,如金表之制。



 延祐七年,仁宗升祔,增置廟室。太常禮儀院下博士檢討歷代典故,移書禮部、中書集議曰:「古者天子祭七代,兄弟同為一代,廟室皆有神主,增置廟室。」又議:「大行皇帝升祔太廟,七室皆有神主,增室不及。依前代典故,權於廟內止設幄座,面南安奉。今相視得第七室近南對室地位,東西一丈五尺,除設幄座外,餘五尺,不妨行禮。」乃結彩為殿,置武宗室南,權奉神主。十月戊子,英宗將以四時躬祀太廟,命太常禮官與中書、翰林、集賢等官集議其禮制,曰:「此追遠報本之道也,毋以朕勞而有所損焉,其一遵典禮。」丙寅,中書以躬謝太廟儀注進。十一月丙子朔,帝御齋宮。丁丑,備法駕儀衛,躬謝太廟,至欞星門駕止,有司進輦不御,步至大次,服袞冕端拱以俟。禮儀使請署祝,帝降御座正立書名。及讀祝,敕高贊御名。至仁宗室,輒歔欷流涕,左右莫不感動。退至西神門,殿中監受圭,出降沒階乃授。甲辰,太常進時享太廟儀式。



 至治元年正月乙酉,始命於太廟垣西北建大次殿。丙戌,始以四孟月時享,親祀太室。禮成,坐大次謂群臣曰:「朕纘承祖宗丕緒,夙夜祗慄,無以報稱,歲惟四祀,使人代之,不能至如在之誠,實所未安。自今以始,歲必親祀,以終朕身。」五月,中書省臣言:「以廟制事,集御史臺、翰林院、太常院臣議。謹按前代廟室,多寡不同。晉則兄弟同為一室,正室增為十四間,東西各一間。唐九廟,後增為十一室。宋增室至十八,東西夾室各一間,以藏祧主。今太廟雖分八室,然兄弟為世,止六世而已。世祖所建前廟後寢,往歲寢殿災。請以今殿為寢,別作前廟十五間,中三間通為一室,以奉太祖神主,餘以次為室,庶幾情文得宜。謹上太常廟制。」制曰:「善,期以來歲營之。」



 二年春正月丁丑,始陳鹵簿,親享太廟。三月二十三日,以新作太廟正殿,夏秋二祭權止。秋八月丙辰,太皇太后崩,太常院官奏:「國哀以日易月,旬有二日外,乃舉祀事。有司以十月戊辰,有事於太廟,取聖裁。」制曰:「太廟禮不可廢,迎香去樂可也。」又言:「太廟興工未畢,有妨陳宮縣樂,請止用登歌。」從之。



 三年春三月戊申,祔昭獻元聖皇后於順宗室。夏四月六日,上都分省參議速速,以都堂旨,太廟夾室未有制度,再約臺院等官議定。博士議曰:「按《爾雅》曰『室有東西廂曰廟』,《注》:『夾室前堂。』同禮曰『西夾南向』,《注》曰『西廂夾室』。此東西夾室之正文也。賈公彥曰:『室有東西廂曰廟,其夾皆在序。』是則夾者,猶今耳房之類也。然其制度,則未之聞。東晉太廟正室一十六間,東西儲各一間,共十有八。所謂儲者,非夾室與?唐貞觀故事,遷廟之主,藏於夾室西壁,南北三間。又宋哲宗亦嘗於東夾室奉安,後雖增建一室,其夾室仍舊。是唐、宋夾室,與諸室制度無大異也。五帝不相沿樂,三王不相襲禮。今廟制皆不合古,權宜一時。宜取今廟一十五間,南北六間,東西兩頭二間,準唐南北三間之制,壘至棟為三間,壁以紅泥,以準東西序,南向為門,如今室戶之制,虛前以準廂,所謂夾室前堂也。雖未盡合於古,於今事為宜。」六月,上都中書省以聞,制若曰「可」。壬申,敕以太廟前殿十有五間,東西二間為夾室,南向。秋七月辛卯,太廟落成。俄,國有大故,晉王即皇帝位。十二月戊辰,追尊皇考晉王為皇帝,廟號顯宗,皇妣晉王妃為皇后。庚午,盜入太廟,失仁宗及慈聖皇后神主。壬申,重作仁廟二金主。丙午,御史趙成慶言:「太廟失神主,乃古今莫大之變。由太常禮官不恭厥職,宜正其罪,以謝宗廟,以安神靈。」制命中書定罪。



 泰定元年春正月甲午,奉安仁宗及慈聖皇后二神主。丁丑,御史宋本、趙成慶、李嘉賓言:「太廟失神主,已得旨,命中書定太常失守之罪。中書以為事在太廟署令,而太常官屬居位如故。昔唐陵廟皆隸宗正。盜斫景陵門戟架,既貶陵令丞,而宗正卿亦皆貶黜。且神門戟架比之太廟神主,孰為輕重?宜定其罪名,顯示黜罰,以懲不恪。」不報。



 先是,博士劉致建議曰:



 竊以禮莫大於宗廟。宗廟者,天下國家之本,禮樂刑政之所自出也。唐、虞、三代而下,靡不由之。聖元龍興朔陲,積德累功,百有餘年,而宗廟未有一定之制。方聖天子繼統之初,定一代不刊之典,為萬世法程,正在今日。



 周制,天子七廟,三昭三穆,昭處於東,穆處於西,所以別父子親疏之序,而使不亂也。聖朝取唐、宋之制,定為九世,遂以舊廟八室而為六世,昭穆不分,父子並坐,不合《禮經》。新廟之制,一十五間,東西二間為夾室,太祖室既居中,則唐、宋之制不可依,惟當以昭穆列之。父為昭,子為穆,則睿宗當居太祖之東,為昭之第一世,世祖居西,為穆之第一世。裕宗居東,為昭之第二世。兄弟共為一世,則成宗、順宗、顯宗三室皆當居西,為穆之第二世。武宗、仁宗二室皆當居東,為昭之第三世。英宗居西,為穆之第三世。昭之後居左,穆之後居右,西以左為上,東以右為上也。茍或如此,則昭穆分明,秩然有序,不違《禮經》,可為萬世法。



 若以累朝定制,依室次於新廟遷安,則顯宗躋順宗之上,順宗躋成宗之上。以禮言之,春秋閔公無子,庶兄僖公代立,其子文公遂躋僖公於閔公之上,史稱逆祀。及定公正其序,書曰「從祀先公」。然僖公猶是有位之君,尚不可居故君之上,況未嘗正位者乎?



 國家雖曰以右為尊,然古人所尚,或左或右,初無定制。古人右社稷而左宗廟,國家宗廟亦居東方。豈有建宗廟之方位既依《禮經》,而宗廟之昭穆反不應《禮經》乎?且如今朝賀或祭祀,宰相獻官分班而立,居西則尚左,居東則尚右。及行禮就位,則西者復尚右,東者復尚左矣。至職居博士,宗廟之事所宜建明,然事大體重,宜從使院移書集議取旨。



 四月辛巳,中書省臣言:「世祖皇帝始建太廟。太祖皇帝居中南向,睿宗、世祖、裕宗神主以次祔西室,順宗、成宗、武宗、仁宗以次祔東室。邇者集賢、翰林、太常諸臣言,國朝建太廟遵古制,古尚左,今尊者居右為少屈,非所以示後世。太祖皇帝居中南向,宜奉睿宗皇帝神主祔左一室,世祖祔右一室,裕宗祔睿宗室之左。顯宗、順宗、成宗兄弟也,以次祔世祖室之右,武宗、仁宗亦兄弟也,以祔裕宗室之左,英宗祔成宗室之右。臣等以其議近是,謹繪室次為圖以獻,惟陛下裁擇。」從之。五月戊戌,祔顯宗、英宗凡十室。



 四年夏四月辛未,盜入太廟,失武宗神位及祭器。壬申,重作武宗金主及祭器。甲午,奉安武宗神主。天歷元年冬十月丁亥,毀顯宗室。重改至元之六年六月,詔毀文宗室。其宗廟之事,本末因革,大概如此。



 凡大祭祀,尤貴馬湩。將有事,敕太僕寺挏馬官,奉尚飲者革囊盛送焉。其馬牲既與三牲同登於俎,而割奠之饌,復與籩豆俱設。將奠牲盤酹馬湩,則蒙古太祝升詣第一座,呼帝後神諱,以致祭年月日數、牲齊品物,致其祝語。以次詣列室,皆如之。禮畢,則以割奠之餘,撒於南欞星門外,名曰拋撒茶飯。蓋以國禮行事,尤其所重也。始至元初,金大祝魏友諒者仕於朝,詣中書言太常寺奉祀宗廟禮不備者數事。禮部移太常考前代典禮,以勘友諒所言,皆非是,由是禮官代有討論。割奠之禮,初惟太常卿設之。桑哥為初獻,乃有三獻等官同設之儀。博士議曰:「凡陳設祭品、實樽罍等事,獻官皆不與也,獨此親設之,然後再升殿,恐非誠愨專一之道。且大禮使等官,尤非其職。」大樂署長言:「割奠之禮,宜別撰樂章。」博士議曰:「三獻之禮,實依古制。若割肉,奠葡萄酒、馬湩,別撰樂章,是又成一獻也。」又議:「燔膋膟與今燒飯禮合,不可廢。形鹽、糗餌、粉糍、裛食、糝食非古。雷鼓、路鼓,與播鞀之制不同。攝祀大禮使終夕堅立,無其義。」知禮者皆有取於其言。英宗之初,博士又言:「今冬祭即烝也。天子親稞太室,功臣宜配享。」事亦弗果行。



 廟制:至元十七年,新作於大都。前廟後寢。正殿東西七間,南北五間,內分七室。殿陛二成三階,中曰泰階,西曰西階,東曰阼階。寢殿東西五間,南北三間。環以宮城,四隅重屋,號角樓。正南、正東、正西宮門三,門各五門,皆號神門。殿下道直東西神門曰橫街,直南門曰通街,甓之。通街兩旁井二,皆覆以亭。宮城外,繚以崇垣。饌幕殿七間,在宮城南門之東,南向。齊班五間,在宮城之東南,西向。省饌殿一間,在宮城東門少北,南向。初獻齋室在宮城之東,東垣門內少北,西向。其南為亞終獻、司徒、大禮使、助奠、七祀獻官等齋室,皆西向。雅樂庫在宮城西南,東向。法物庫、儀鸞庫在宮城之東北,皆南向。都監局在其東少南,西向。東垣之內,環築墻垣為別院。內神廚局五間,在北,南向。井在神廚之東北,有亭。酒庫三間,在井亭南,西向。祠祭局三間,對神廚局,北向。院門西向。百官廚五間,在神廚院南,西向。宮城之南,復為門,與中神門相值,左右連屋六十餘間,東掩齊班,西值雅樂庫,為諸執事齋房。築崇墉以環其外,東西南開欞星門三,門外馳道,抵齊化門之通衢。



 至治元年,詔議增廣廟制。三年,別建大殿一十五間於今廟前,用今廟為寢殿,中三間通為一室,餘十間各為一室,東西兩旁際墻各留一間,以為夾室。室皆東西橫闊二丈,南北入深六間,每間二丈。宮城南展後,鑿新井二於殿南,作亭。東南隅、西南隅角樓,南神門、東西神門,饌幕殿、省饌殿、獻官百執事齋室,中南門、齊班、雅樂庫、神廚、祠祭等局,皆南徙。建大次殿三間於宮城之西北,東西欞星門亦南徙。東西欞星門之內,鹵簿房四所,通五十間。



 神主:至元三年,始命太保劉秉忠考古制為之,高一尺二寸,上頂圜徑二寸八分,四廂合剡一寸一分。上下四方穿,中央通孔,徑九分,以光漆題尊謚於背上。匱趺底蓋俱方。底自下而上,蓋從上而下。底齊趺,方一尺,厚三寸。皆準元祐古尺圖。主及匱趺皆用慄木,匱趺並用玄漆,設祏室以安奉。帝主用曲幾,黃羅帕覆之。後主用直幾,紅羅帕覆之;祏室,每室紅錦厚褥一,紫錦薄褥一,黃羅復帳一,龜背紅簾一,緣以黃羅帶飾。六年十二月十八日,國師奉旨造木質金表牌位十有六,亦號神主。設大榻金椅位,置祏室前。帝位於右,後位於左,題號其面,籠以銷金絳紗,其制如櫝。



 祝有二:祝冊,親祀用之。制以竹,每副二十有四簡,貫以紅絨絳。面用膠粉塗飾,背飾以絳金綺。藏以楠木縷金雲龍匣。塗金鎖鑰,韜以紅錦囊,蒙以銷金雲龍絳羅覆。擬撰祝文、書祝、讀祝,皆翰林詞臣掌之。至大二年親祀,竹冊長一尺二寸,廣一寸二分,厚三分。至治二年正月親祀,竹冊八副,每冊二十有四簡,長一尺一寸,廣一寸,厚一分二厘。



 祝版,攝祀用之,制以楸木,長二尺四寸,廣一尺二寸,厚一分。其面背飾以精潔楮紙。



 祝文,至元時,享於太祖室,稱孝孫嗣皇帝臣某;睿宗室,稱孝子嗣皇帝臣某。天歷時,享自太祖至裕宗四室,皆稱孝曾孫嗣皇帝臣某;順宗室,稱孝孫嗣皇帝臣某;成宗至英宗三室,皆稱嗣皇帝臣某;武宗室,稱孝子嗣皇帝臣某。



 幣:以白繒為之,每段長一丈八尺。



 牲齊庶品:大祀,馬一,用色純者,有副;牛一,其角握,其色赤,有副;羊,其色白;豕,其色黑;鹿。凡馬、牛、羊、豕、鹿牲體,每室七盤,單室五盤。太羹,每室三登;和羹,每室三鉶。籩之實,每室十有二品;豆之實,每室十有二品。凡祀,先期命貴臣率獵師取鮮麞鹿兔,以供脯MZ醓醢。稻粱為飯,每室二簠;黍稷為飯,每室二簋。彞尊之實,每室十有一。明水玄酒,用陰監取水於月,與井水同,鬯用鬱金為之。五齊三酒,醖於光祿寺。膟膋蕭蒿,至元十八年五月弗用,後遂廢。茅香以縮酒,至元十七年,始用沅州麻陽縣包茅。天鵝、野馬、塔剌不花、其狀如貛。野雞、鶬、黃羊、胡寨兒、其狀如鳩。湩乳、葡萄酒,以國禮割奠,皆列室用之。羊一,豕一,籩之實二慄、鹿脯,豆之實二菁菹、鹿MZ,簠之實黍,簋之實稷,爵尊之實酒,皆七祀位各用之。薦新鮪、野彘,孟春用之。雁、天鵝,仲春用之。葑韭、鴨雞卵,季春用之。冰、羔羊,孟夏用之。櫻桃、竹筍、蒲筍、羊,仲夏用之。瓜、豚、大麥飯、小麥面,季夏用之。雛雞,孟秋用之。菱芡、慄、黃鼠,仲秋用之。梨、棗、黍、粱、鶿老,季秋用之。芝麻、兔、鹿、稻米飯,孟冬用之。麕、野馬,仲冬用之。鯉、黃羊、塔剌不花,季冬用之。至大元年春正月,皇太子言薦新增用影堂品物,羊羔、炙魚、饅頭、食其子、西域湯餅、圜米粥、砂糖飯羹,每月用以配薦。



 祭器:籩十有二,冪以青巾,巾繪彩雲。豆十有四,一實毛血,一實膟膋。登三,鉶三,有柶。簠二,簋二,有匕箸。俎七,以載牲體,皆有鼎。後以盤貯牲體,盤置俎上,鼎不用。香案一。銷金絳羅衣。銀香鼎一,銀香奩一,茅苴盤一,實以沙。已上並陳室內。燎爐一,實以炭。篚一,實以蕭蒿黍稷。祝案一,紫羅衣,置祝文於上,銷金絳羅覆之。雞彞一,有舟;鳥彞一,有舟,加勺;春夏用之。斝彞一,有舟;黃彞一,有舟,加勺;秋冬用之。虎彞一,有舟;蜼彞一,有舟,加勺;特祭用之。凡雞彞、斝彞、虎彞以實明水,鳥彞、黃彞、蜼彞以實鬯。犧尊二,象尊二,春夏用之。著尊二,壺尊二,秋冬用之。太尊二,山尊二,特祭用之。尊皆有坫勺,冪以白布巾,巾繪黼文。著尊二,山罍二,皆有坫加冪。已上並陳室外。壺尊二,太尊二,山罍四,皆有坫加冪,藉以莞席,並陳殿下,北向西上,設而不酌,每室皆同。通廊禦香案一,銷金黃羅衣,銀香奩一,貯禦祝香,銷金帕覆之,並陳殿中央。罍洗所罍二,洗二,一以供爵滌,一以供盥潔。篚二,實以璋瓚巾、塗金銀爵。七祀神位,籩二,豆二,簠一,簋一,俎一,爵一有坫,香案一,沙池一,壺尊二有坫加冪,七祀皆同。罍一、洗一、篚一,中統以來,雜金、宋祭器而用之。至治初,始造新器於江浙行省,其舊器悉置幾閣。



 親祀時享儀,其目有八:



 一曰齋戒。前祀七日,皇帝散齋四日於別殿,治事如故,不作樂,停奏刑名事,不行刑罰。致齋三日,惟專心祀事,其二日於大明殿,一日於大次。致齋前一日,尚舍監設御幄於大明殿西序,東向。致齋之日質明,諸衛勒所部屯列。晝漏下一刻,通事舍人引侍享執事文武四品以上官,俱公服詣別殿奉迎。二刻,侍中版奏請中嚴,皇帝服通天冠、絳紗袍。三刻,侍中版奏外辦,皇帝結佩出別殿,乘輿,華蓋傘扇侍衛如常儀,奉引至大明殿御幄,東向坐,侍臣夾侍如常。一刻頃,侍中前跪奏言請降就齋,俯伏興。皇帝降座入室,侍享執事官各還所司,宿衛者如常。凡應祀官受誓戒於中書省。散齋四日,致齋三日。光祿卿鑒取明水、火。火以供爨,水以實尊。



 二曰陳設。祀前三日,尚舍監陳大次於西神門外道北,南向。設小次於西階西,東向。設版位於西神門內,橫街南,東向。設飲福位於太室尊彞所,稍東,西向。設黃道裀褥於大次前,至西神門,至小次版位西階及殿門之外。設御洗位於御板位東,稍北,北向。設亞終獻位於西神門內御板位稍南,東向。以北為上,罍洗在其東北。設亞終獻飲福位於御飲福位後,稍南,西向。陳設八寶黃羅案於西階西,隨地之宜。設享官宮縣樂、省牲位、諸執事公卿御史位,並如常儀。殿上下及各室,設簠、簋、籩、豆、尊、罍、彞、斝等器,並如常儀。



 三曰車駕出宮。祀前一日,所司備法駕鹵簿於崇天門外,太僕卿率其屬備玉輅於大明門外。千牛將軍執刀於輅前,北向。其日質明,諸侍享執事官,先詣太廟祀所。諸侍臣直衛及導駕官於致齋殿前,左右分班立。通事舍人引侍中跪奏請中嚴,俯伏興。皇帝服通天冠、絳紗袍。少頃,侍中版奏外辦,皇帝出齋室,即御座。群臣起居訖,尚輦進輿,侍中奏請皇帝升輿。皇帝升輿,華蓋傘扇侍衛如常儀。導駕官前導至大明門外,侍中進當輿前,跪奏請皇帝降輿升輅。皇帝升輅,太僕執御,導駕官分左右步導。門下侍郎進當輅前,跪奏請車駕進發。車駕動,稱警蹕。千牛將軍夾而趨至崇天門外,門下侍郎跪奏請車駕少駐,敕眾官上馬。侍中承旨退,稱曰「制可」。門下侍郎退,傳制稱眾官上馬。贊者承傳敕眾官上馬。上馬訖,門下侍郎奏請敕車右升,侍中前承制,退稱曰「制可」。千牛將軍升訖,門下侍郎奏請車駕進發。車駕動,稱警蹕。符寶郎奉八寶與殿中監部從在黃鉞內,教坊樂前引,鼓吹不振作。將至太廟,禮直官引諸侍享執事官於廟門外,左右立班,奉迎駕至廟門,回輅南向。將軍降立於輅左,侍中於輅前奏稱侍中臣某請皇帝降輅,步入廟門。皇帝降輅,導駕官前導,皇帝步入廟門稍西。侍中奏請皇帝升輿,尚輦奉輿,華蓋傘扇如常儀。皇帝乘輿至大次,侍中奏請皇帝降輿入就大次。皇帝入就次,簾降,宿衛如式,尚食進膳如儀。禮儀使以祝版奏御署訖,奉出,太廟令受之,各奠於坫,置各室祝案上。通事舍人承旨,敕眾官各還齋次。



 四曰省牲器。祀前一日未後三刻,廩犧令丞、太官令丞、太祝以牲就位。禮直官引太常卿、光祿卿丞、監祭禮等官就位。禮直官請太常、監祭、監禮由東神門北偏門入,升自東階。每位視滌祭器,司尊彞舉冪曰「潔」。俱畢,降自東階,由東神門北偏門出,復位,立定。禮直官稍前曰「請省牲」,引太常卿視牲,退復位。次引廩犧令出班,巡牲一匝,西向折身曰「充」。諸太祝巡牲一匝,上一員出班西向折身曰「腯」畢,俱復位。蒙古巫祝致詞訖,禮直官稍前曰「請詣省饌位」,引太常卿、光祿卿、監祭、監禮、光祿丞、大官令丞詣省饌位,東西相向立定,以北為上。禮直官引太常卿詣饌殿內省饌。視饌訖,禮直官引太常卿還齋所。次引廩犧令丞、諸太祝以次牽牲詣廚,授太官令。次引光祿卿丞、監祭、監禮詣廚省鼎鑊,視滌溉訖,各還齋所。太官令帥宰人以鸞刀割牲,祝史各取毛血,每位共實一豆,以肝洗於鬱鬯及取膟膋,每位共實一豆,置於各位。饌室內,庖人烹牲。



 五曰晨稞。祀日醜前五刻,諸享陪位官各服其服。光祿卿、良醖令、太官令入,實籩、豆、簠、簋、尊、罍,各如常儀。太樂令率工人二舞,以次入。奉禮郎贊者先入就位,禮直官引御史、博士及執事者以次各入,就位,並如常儀。禮直官引司徒以下官升殿,分香設酒,如常儀。禮直官引太常官、御史、博士升殿,視陳設,就位。復與太廟令、太祝、宮闈令升殿。太祝出帝主,宮闈令出後主訖,御史及以上升殿官於當陛近西,北向立。奉禮於殿上贊奉神主訖,奉禮曰「再拜」,贊者承傳,諸官及執事者皆再拜,各就位。禮直官引亞終獻等官,由南神門東偏門入,就位,立定。禮直官贊有司謹具,請行事。協律郎俯伏興,舉麾,工鼓柷,宮縣樂作《思成之曲》,以黃鐘為宮,大呂為角,太簇為征,應鐘為羽,作文舞九成止。樂奏將終,通事舍人引侍中版奏請中嚴。皇帝服袞冕,坐少頃,禮直官引博士,博士引禮儀使,對立於大次門外,當門北向。侍中奏外辦,禮儀使跪奏請皇帝行禮,俯伏興,簾卷。符寶郎奉寶陳於西陛之西黃羅案上。皇帝出大次,博士、禮儀使前導,華蓋傘扇如儀,大禮使後從。至西神門外,殿中監跪進鎮圭,皇帝執圭,華蓋傘扇停於門外,近侍從入門。協律郎跪俯伏興,舉麾,工鼓柷,宮縣《順成之樂》作。至版位東向,協律郎偃麾,工戛敔,樂止。引禮官分左右侍立,禮儀使前奏請再拜,皇帝再拜。奉禮曰「眾官再拜」,贊者承傳,凡在位者皆再拜。禮儀使奏請皇帝詣盥洗位,宮縣樂作,至洗位,樂止。內侍跪取,興,沃水。又內侍跪取盤,興,承水。禮儀使奏請搢鎮圭,皇帝搢圭,盥手訖,內侍跪取巾於篚,興,以進,帨手訖,皇帝詣爵洗位,奉瓚官以瓚跪進,皇帝受瓚,內侍奉沃水。又內侍跪,奉盤承水,洗瓚訖,內侍奉巾以進,皇帝拭瓚訖,內侍奠盤,又奠巾於篚,奉瓚官跪受瓚。禮儀使奏請執鎮圭,前導皇帝升殿,宮縣樂作,至西階下,樂止。皇帝升自西階,登歌樂作,禮儀使前導皇帝詣太祖室尊彞所,東向立,樂止。奉瓚官以瓚蒞鬯,司尊者舉冪,侍中跪酌鬱鬯訖,禮儀使前導,入詣太祖神座前,北向立。禮儀使奏請搢鎮圭跪,奉瓚官西向立,以瓚跪進。禮儀使奏請執瓚、以鬯祼地,皇帝執瓚以鬯祼地,以瓚授奉瓚官。禮儀使奏請執鎮圭、俯伏興。皇帝俯伏興,禮儀使前導出戶外褥位。禮儀使奏請再拜。皇帝再拜訖,禮儀使前導詣第二室以下,祼鬯並如上儀。祼訖,禮儀使奏請還版位。登歌樂作,皇帝降自西階,樂止。宮縣樂作,至版位東向立,樂止。禮儀使奏請還小次,前導皇帝行,宮縣樂作。將至小次,禮儀使奏請釋鎮圭,殿中監跪受,皇帝入小次,簾降,樂止。



 六曰進饌。皇帝祼將畢,光祿卿詣饌殿視饌,復位。太官令率齋郎詣饌幕,以牲體設於盤,各對舉以行,自南神門入。司徒出迎饌,宮縣樂作,奏無射宮《嘉成之曲》。禮直官引司徒、齋郎奉饌升自太階,由正門入。諸太祝迎於階上,各跪奠於神座前。齋郎執笏俯伏興,遍奠訖,樂止。禮直官引司徒、太官令率齋郎降自東階,各復位。饌之升殿也,太官丞率七祀齋郎奉饌,以序跪奠於七祀神座前,退從殿上齋郎以次復位。諸太官令率割牲官詣各室,進割牲體置俎上,皆退。



 七曰酌獻。禮直官於殿上贊太祝立茅苴,禮儀使奏請詣盥洗位。簾卷,出次,宮縣樂作。殿中監跪進鎮圭,皇帝執鎮圭至盥洗位,樂止,北向立。禮儀使奏請搢鎮圭,執事者跪取醿,興,沃水,又跪取盤,承水。禮儀使奏請皇帝盥手,執事者跪取巾於篚,興,進。帨手訖,禮儀使奏請執鎮圭,請詣爵洗位,北向立。禮儀使奏請搢鎮圭,奉爵官以爵跪進。皇帝受爵,執事者奉醿沃水,奉盤承水。皇帝洗爵訖,執事者奉巾跪進。皇帝拭爵,執事者奠盤醿,又奠巾於篚,奉爵官受爵。禮儀使奏請執鎮圭,升殿。宮縣樂作,至西階下,樂止。升自西階,登歌樂作,禮儀使前導詣太祖室尊彞所,東向立,樂止。禮儀使奏請搢鎮圭執爵,奉爵官以爵跪進。皇帝受爵,司尊者舉冪,良醖令跪酌犧尊之泛齊,以爵授執事者。禮儀使奏請執鎮圭,皇帝執圭,入詣太祖神位前,北向立。宮縣樂作,奏《開成之曲》。禮儀使跪奏請搢鎮圭跪,又奏請三上香。三上香訖,奉爵官以爵授進酒官,進酒官東向以爵跪進。禮儀使奏請執爵,三祭酒於茅苴,以虛爵授進酒官,進酒官以授奉爵官,奉爵官退立尊彞所。進酒官進取神案上所奠玉爵馬湩,東向跪進,禮儀使奏請執爵祭馬湩。祭訖,以虛爵授進酒官,進酒官進奠神案上,退。禮儀使奏請執圭,俯伏興,司徒搢笏跪於俎前,奉牲西向以進。禮儀使奏請搢鎮圭,皇帝搢圭,俯受牲盤,北向跪奠神案上。蒙古祝史致辭訖,禮儀使奏請執鎮圭興,前導,出戶外褥位,北向立,樂止。舉祝官搢笏跪,對舉祝版,讀祝官北向跪,讀祝文訖,俯伏興,舉祝官奠祝版訖,先詣次室。禮儀使奏請再拜。拜訖,禮儀使前導詣各室,各奏本室之樂。其酌獻、進牲、祭馬湩,並如第一室之儀。既畢,禮儀使奏請詣飲福位。登歌樂作,至位,西向立,樂止。登歌《成之樂》作,禮直官引司徒立於飲福位側,太祝以爵酌上尊飲福酒,合置一爵,以奉侍中,侍中受爵,奉以立。禮儀使奏請皇帝再拜。拜訖,奏請搢鎮圭跪。侍中東向以爵跪進,禮儀使奏請執爵,三祭酒,又奏請啐酒。啐酒訖,以爵授侍中。禮儀使奏請受胙,太祝以黍稷飯籩授司徒,司徒東向跪進。皇帝受,以授左右。太祝又以胙肉俎跪授司徒,司徒跪進。皇帝受,以授左右。禮直官引司徒退立。侍中再以爵酒跪進,禮儀使奏請皇帝受爵飲福。飲福訖,侍中受虛爵,興,以授太祝。禮儀使奏請執鎮圭,俯伏興,又奏請再拜。拜訖,樂止。禮儀使前導還版位,登歌樂作,降自西階,樂止。宮縣樂作,至位樂止。禮儀使奏請還小次,宮縣樂作。將至小次,禮儀使奏請釋鎮圭,殿中監跪受。入小次,簾降,樂止。文舞退,武舞進。先是皇帝酌獻訖,將至小次,禮直官引亞獻官詣盥洗位。盥洗訖,升自阼階,酌獻並如常儀。酌獻訖,禮直官引亞獻官詣東序,西向立。諸太祝各以酌罍福酒,合置一爵,一太祝捧爵進亞獻之左,北向立。亞獻再拜受爵,跪祭酒,遂啐飲。太祝進受爵,退,復於坫上。亞獻興再拜,禮直官引亞獻官降復位。終獻如亞獻之儀。初終獻既升,禮直官引七祀獻官各詣盥洗位,搢笏盥帨訖,執笏詣神位,搢笏跪執爵,三祭酒,奠爵執笏,俯伏興,再拜訖,詣次位,如上儀。終獻畢,贊者唱「太祝徹籩豆」。諸太祝進徹籩豆,登歌《豐成之樂》作,卒徹樂止。奉禮曰「賜胙」,贊者唱「眾官再拜」,在位者皆再拜。禮儀使奏請詣版位。簾卷,出次,殿中監跪進鎮圭。皇帝執圭行,宮縣樂作,至位樂止。送神《保成之樂》作,一成止。禮儀使奏請皇帝再拜,贊者承傳,凡在位者皆再拜。禮儀使前奏禮畢,前導皇帝還大次。宮縣《昌寧之樂》作,出門樂止。禮儀使奏請釋鎮圭,殿中監跪受,華蓋傘扇引導如常儀。入大次,簾降。禮直官引太常卿、御史、太廟令、太祝、宮闈令升殿納神主,降就拜位,奉禮贊升納神主訖,再拜,御史以下諸執事者皆再拜,以次出。禮直官各引享官以次出,太樂令率工人二舞以次出,太廟令闔戶以降乃退。祝冊藏於匱。



 八曰車駕還宮。皇帝既還大次,侍中奏請解嚴。皇帝釋袞冕,停大次。五刻頃,尚食進膳。所司備法駕鹵簿,與侍祠官序立於太廟欞星門外,以北為上。侍中版奏請中嚴,皇帝改服通天冠、絳紗袍。少頃,侍中版奏皇帝出次升輿,導駕官前導,華蓋傘扇如儀。至廟門外,太僕卿率其屬進金輅如式。侍中前奏請皇帝降輿升輅。升輅訖,太僕御。門下侍郎奉請車駕進發,俯伏興,退。車駕動,稱警蹕。至欞星門外,門下侍郎奉請車駕權停,敕眾官上馬。侍中承旨退稱曰「制可」。門下侍郎退傳制,贊者承傳。眾官上馬畢,門下侍郎奏請敕車右升。侍中承旨退稱「制可」,千牛將軍升訖,導駕官分左右前導,門下侍郎奏請車駕進發。車駕動,稱警蹕。符寶郎奉八寶與殿中監從,教坊樂鼓吹振作。駕至崇天門外垣欞星門外,門下侍郎奏請車駕權停,敕眾官下馬。贊者承傳,眾官下馬。車駕動,眾官前引入內石橋,與儀仗倒卷而北,駐立。駕入崇天門,至大明門外降駕,升輿以入。駕既入,通事舍人承旨敕眾官皆退,宿衛官率衛士宿衛如式。



\end{pinyinscope}