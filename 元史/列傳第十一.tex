\article{列傳第十一}

\begin{pinyinscope}

 ○塔本



 塔本,伊吾廬人。人以其好揚人善,稱之曰揚公。父宋五設托陀,托陀者,其國主所賜號,猶華言國老也。塔本初從太祖討諸部,屢厄艱危。復從圍燕,征遼西,下平灤、白霫諸城。軍士有妄殺人者,塔本戒之曰:「國之本,民也。殺人得地,何益於國。且殺無罪以堅敵心,非上意。」太祖聞而喜之,賜金虎符,俾鎮撫白霫諸郡,號行省都元帥,管內得承制除縣吏,死囚得專決。久之,徙治興平。興平兵火傷殘,民慘無生意。塔本召父老問所苦,為除之,薄賦斂,役有時。民大悅,乃相與告教,無違約束,歸者四集。塔本始至,戶止七百,不一二年,乃至萬戶。出己馬以寬驛人;貸廉吏銀,其子錢不能償者,焚其券。農不克耕,亦與之牛,比歲告稔,民用以饒。庚寅,詔益中山、平定、平原隸行省。甲午,盜李仙、越小哥等作亂,塔本止誅首惡,宥其詿誤。癸卯立春日,宴群僚,歸而疾作,遂卒。是夕星隕,隱隱有聲。遺命葬以紙衣瓦棺。贈推誠定遠佐運功臣、太師、開府儀同三司、上柱國,追封營國公,謚忠武。子阿里乞失鐵木兒。



 阿里乞失鐵木兒,嗣父職,為興平等處行省都元帥。其為治一遵先政,興學養士,輕刑薄徭,雖同僚不敢私役一民。從大軍伐高麗有功。歲丙辰卒。贈宣忠輔義功臣、榮祿大夫、平章政事、柱國,追封營國公,謚武襄。子阿臺。



 阿臺,當襲父職,適罷行省為平灤路總管府,丁巳,憲宗命阿臺為平灤路達魯花赤。始至,請蠲銀、鹽、酒等稅課八之一,細民不征。世祖即位,來朝,賜金虎符。諸侯王道出平灤,供給費銀七千五百兩,戶部不即償,阿臺自陳上前,盡取償以歸。置甲乙籍,籍民丁力,民甚便之。至元十年,進階懷遠大將軍。歲饑,發粟賑民。或持不可,阿臺曰:「朝廷不允,願以家粟償官。」於是全活甚眾。僚屬始至,阿臺必遺之鹽、米、羊畜、什器,曰:「非有他也,欲其不剝民耳。」姻族窮者,月有常給;民有喪不能葬者,與之棺郭、布帛、資糧。灤為孤竹故國,乃廟祀伯夷、叔齊,以勵風俗。二十一年,進昭武大將軍。二十四年,乃顏叛,獻馬五百匹佐軍,世祖大喜。已而得乃顏銀甕,亟以賜之。二十五年入朝,以疾卒。賜宣力功臣、資德大夫、中書右丞、上護軍,追封永平郡公,謚忠亮。子迭里威失。



 迭里威失,少好讀書,成宗時入宿衛,授河西廉訪司僉事,拜監察御史,遷淮西廉訪副使,召為中書左司員外郎,改樞密院參議,升判官。延祐四年,授翰林侍講學士,出為河間路總管。屬歲饑,出俸金及官庫所積賑之,活數十萬人。河間當水陸要沖,四方供億皆取給焉,迭里威失立法調遣,民便之。復建言增置便習弓馬尉一人,益邏兵之數,於是盜賊屏息。陵州群兇為官民害,悉收系死獄中。後升遼陽行省參知政事。子鎖咬兒哈的迷失。



 鎖咬兒哈的迷失,年十二,宿衛英宗潛邸,掌服御諸物。英宗即位,拜監察御史。至治元年春,詔起大剎于京西壽安山,鎖咬兒哈的迷失與御史觀音保、成珪、李謙亨上章極諫,以為東作方始,而興大役,以耗財病民,非所以祈福也。且歲在辛酉,不宜興築。初,司徒劉夔妄獻浙右民田,冒出內帑鈔六百萬貫,丞相帖木迭兒分取其半,監察御史發其奸,由是疾忌臺諫。至是,帖木迭兒之子瑣南為治書侍御史,密奏曰:「彼宿衛舊臣,聞事有不便,弗即入白,今訕上以揚己之直,大不敬。」帝乃殺鎖咬兒哈的迷失與觀音保,杖珪、謙亨,黥之,竄諸遐裔。泰定初,贈鎖咬兒哈的迷失資德大夫、御史中丞、上護軍,追封永平郡公,謚貞愍。賜其妻子鈔五百貫、良田千畝,仍詔樹碑神道。



 ○哈剌亦哈赤北魯



 哈剌亦哈赤北魯,畏兀人也。性聰敏,習事。國王月仙帖木兒亦都護聞其名,自唆裏迷國徵為斷事官。月仙帖木兒卒,子八兒出阿兒忒亦都護年幼,西遼主鞠兒可汗遣使據其國,且召哈剌亦哈赤北魯,至則以為諸子師。八兒出阿兒忒聞太祖明聖,乃殺西遼使,更遣阿憐帖木兒都督等四人使西遼。阿憐帖木兒都督者,哈剌亦哈赤北魯婿也。具語以其故,於是與其子月朵失野訥馳歸太祖,一見大悅,即令諸皇子受學焉。仍令月朵失野訥以質子入宿衛。從帝西征。至別失八里東獨山,是城空無人,帝問:「此何城也?」對曰:「獨山城。往歲大饑,民皆流移之它所。然此地當北來要沖,宜耕種以為備。臣昔在唆裏迷國時,有戶六十,願移居此。」帝曰:「善。」遣月朵失野訥佩金符往取之,父子皆留居焉。後六年,太祖西征還,見田野墾闢,民物繁庶,大悅。問哈剌亦哈赤北魯,則已死矣。乃賜月朵失野訥都督印章,兼獨山城達魯花赤。月朵失野訥卒,子乞赤宋忽兒,在太宗時襲爵,賜號答剌罕。子四人:曰塔塔兒,曰忽棧,曰火兒思蠻,曰月兒思蠻。



 世祖命火兒思蠻從雪雪的斤鎮雲南。月兒思蠻事憲宗,襲父爵,兼領僧人。後因軍帥札忽兒臺據別失八里,盡室徙居平涼。與其子阿的迷失帖木兒入覲,世祖詔入宿衛為必闍赤,命從安西王忙哥剌出鎮六盤。安西王薨,其子阿難答嗣。成宗即位,遣使入朝,因奏:「阿的迷失帖木兒父子,本先帝舊臣,來事先王,服勤二十餘年矣。若終老王府,非所以盡其才也,願以歸陛下用之。」成宗可其奏,授阿的迷失帖木兒汝州達魯花赤,積官秘書太監。卒。子阿鄰帖木兒。



 阿鄰帖木兒,善國書,多聞識,歷事累朝,由翰林待制累遷榮祿大夫、翰林學士承旨。英宗時,以舊學日侍左右,陳說祖宗以來及古先哲王嘉言善行。翻譯諸經,紀錄故實,總治諸王、駙馬、番國朝會之事。天歷初,北迎明宗入正大統,一見歡甚,顧左右曰:「此朕師也。」天歷三年,進光祿大夫、知經筵事。



 子曰沙剌班,曰禿忽魯,曰六十,曰咱納祿。沙剌班,累拜中書平章政事、大司徒、宣政院使。



 ○塔塔統阿



 塔塔統阿,畏兀人也。性聰慧,善言論,深通本國文字。乃蠻■可汗尊之為傅,掌其金印及錢穀。太祖西征,乃蠻國亡,塔塔統阿懷印逃去,俄就擒。帝詰之曰:「大■人民疆土,悉歸於我矣,汝負印何之?」對曰:「臣職也,將以死守,欲求故主授之耳。安敢有他!」帝曰:「忠孝人也!」問是印何用,對曰:「出納錢穀,委任人材,一切事皆用之,以為信驗耳。」帝善之,命居左右。是後凡有制旨,始用印章,仍命掌之。帝曰:「汝深知本國文字乎?」塔塔統阿悉以所蘊對,稱旨,遂命教太子諸王以畏兀字書國言。太宗即位,命司內府玉璽金帛。命其妻吾和利氏為皇子哈剌察兒乳母,時加賜予。塔塔統阿召諸子諭之曰:「上以汝母鞠育太子,賜予甚厚,汝等豈宜有之,當先供太子用,有餘則可分受。」帝聞之,顧侍臣曰:「塔塔統阿以朕所賜先供太子,其廉介可知矣。」由是數加禮遇。以疾卒。至大三年,贈中奉大夫,追封雁門郡公。子四人:長玉笏迷失,次力渾迷失,次速羅海,次篤綿。



 玉笏迷失,少有勇略,渾都海叛於三盤,時玉笏迷失守護皇孫脫脫營壘,率其眾與渾都海戰,敗之。追至只必勒,適遇阿藍答兒與之合兵,復戰,玉笏迷失死之。



 力渾迷失,有膂力,嘗獵於野,與眾相失,遇盜三人,欲奪其衣,力渾迷失搏之,盡僕,遂縛以還。帝召見,選力士與之角,無與敵者。帝壯之,賜金,令行宿衛。



 速羅海,襲父職,仍命司內府玉璽金帛。



 篤綿,舊事皇子哈剌察兒,世祖即位,從其母入見,欲官之,以無功辭,命統宿衛。奉使遼東。卒,封雁門郡公。子阿必實哈,陜西行省平章政事。



 ○岳璘帖穆爾



 岳璘帖穆爾,回鶻人,畏兀國相暾欲谷之裔也。其兄仳理伽普華,年十六,襲國相、答剌罕。時西契丹方強,威制畏兀,命太師僧少監來臨其國,驕恣用權,奢淫自奉。畏兀王患之,謀於仳理伽普華曰:「計將安出?」對曰:「能殺少監,挈吾眾歸大蒙古國,彼且震駭矣。」遂率眾圍少監,斬之。以功加號仳理傑忽底,進授明別吉,妻號赫思迭林。左右有疾其功者,譖於其王曰:「少監珥珠,先王寶也,仳理伽普華匿之,盍急索勿失。」其王怒,索寶甚急。仳理伽普華度無以自明,乃亡附太祖,賜以金虎符、獅紐銀印、金螭椅一、衣金直孫校尉四人,仍食二十三郡。繼又賜銀五萬兩。以弟岳璘帖穆爾為質。仳理伽普華以疾卒。



 岳璘帖穆爾從太祖征討,多戰功。皇弟斡真求師傅,帝命岳璘帖穆爾往,訓導諸王子以孝弟敦睦、仁厚不殺為先,帝聞而嘉之。從平河南,徙贊縣民萬餘戶入樂安。俄授河南等處軍民都達魯花赤,佩金虎符,並賜宮女四人。所得上方賞賚,悉輦歸故郡,以散親舊。且盛陳漢官儀衛以激厲之,國人羨慕。道出河西,所過榛莽,或時乏水,為之鑿井置堠,居民使客相慶稱便。太祖即位,以中原多盜,選充大斷事官。從斡真出鎮順天等路,布德化,寬征徭,盜遁奸革,州郡清寧。尋復監河南等處軍民。年六十七,卒於保定。後贈宣力保德功臣、山東宣慰使,謚莊簡。子合剌普華,見《忠義傳》。



 ○李楨



 李楨,字幹臣。其先,西夏國族子也。金末,楨以經童中選。既長,入為質子,以文學得近侍,太宗嘉之,賜名玉出乾必闍赤。從皇子闊出伐金,帝命之曰:「凡軍中事,須訪楨以行。」及下河南諸郡,闊出遣楨偕吉登哥往唐、鄧二州數民實,兵餘歲兇,流散十八九。楨至,賑恤饑寒,歸者如市。十年,從大將察罕下淮甸。楨以功佩金符,授軍前行中書省左右司郎中。楨奏尋訪天下儒士,令所在優贍之。十三年,師圍壽春,天雨不止,楨言於察罕曰:「頓師城下,暑雨疫作,將有不利。且城久拒命,破必屠之,則生靈何辜。請退舍數里,身往招之。」從之。楨遂單騎入敵壘,曉以利害,明日,與其將二人率眾來降。以功賜銀五千兩。楨表言:「襄陽乃吳、蜀之要沖,宋之喉襟,得之則可為他日取宋之基本。」定宗嘉其言。庚戌,賜虎符,授襄陽軍馬萬戶。丙辰,憲宗命楨率師巡哨襄樊。戊午,帝親征,召楨同議事。秋九月,卒於合州,年五十九。



 ○速哥



 速哥,蒙古怯烈氏,世傳李唐外族。父懷都,事太祖,嘗從飲班術尼河水。速哥為人外若質直,而內實沉勇有謀,雅為太宗所知。命使金,因俾覘其虛實,語之曰:「即不還,子孫無憂不富貴也。」速哥頓首曰:「臣死,職耳。奉陛下威命以行,可無慮也。」帝悅,賜所常御馬。至河,金人閉之舟中,七日始登南岸,又三旬乃達汴。及見金主,曰:「天子念爾土地日狹,民力日疲,故遣我致命,爾能共修歲幣,通好不絕,則轉禍為福矣。」謁者令下拜,速哥曰:「我大國使,為爾屈乎!」金主壯之,取金卮飲之酒曰:「歸語汝主,必欲加兵,敢率精銳以相周旋,歲幣非所聞也。」速哥飲畢,即懷金卮以出。速哥雖佯為不智,而默識其地理厄塞、城郭人民之強弱。既復命,備以虛實告,且以所懷金卮獻。帝喜曰:「我得金於汝手中矣。」復以賜之。始下令徵兵南伐。兵至河北岸,方舟欲渡,金軍陳於河南,帝令儀衛導速哥居中行,親率偏師乘陣西策馬沙河。會睿宗軍亦由襄、鄧至,兩軍夾攻之。及金亡,詔賜金護駕士五人,曰:「此以旌汝為使之不辱也。」昔使過崞州,崞人盜殺其良馬,至是,兼以崞民賜之。歲乙未,帝從容謂速哥曰:「我將官汝,西域、中原,惟汝擇之。」速哥再拜曰:「幸甚!臣意中原為便。」帝曰:「西山之境,八達以北,汝其主之。汝於城中構大樓,居其上,使人皆仰望汝,汝俯而諭之,顧不偉乎!」乃以為山西大達魯花赤。受命方出,有回回六人訟事不實,將抵罪,遇諸途,急止監者曰:「姑緩其刑,當入奏。」復見帝曰:「此六人者,名著西域,徒以小罪盡誅之,恐非所以懷遠人也。願以賜臣,臣得困辱之,使自悔悟遷善,為他日用,殺之無益也。」帝意解,召六人謂之曰:「生汝者速哥也,其竭力事之。」至雲中,皆釋之。後有至大官者。其寬大愛人多類此。卒年六十二。贈推忠翊運同德功臣、太師、開府儀同三司、上柱國,追封宣寧王,謚忠襄。



 子六人:曰長罕,曰玉呂忽都,曰撒合裏都,曰忽蘭,曰忽都兒不花,曰不花。長罕、玉呂忽都、撒合裏都,皆從兀魯赤太子出征,以戰功顯。



 忽蘭之母以後戚故,得襲職。鉏強植弱,均役平刑,闔郡賴以安輯。乙未之抄戶籍也,前賜崞人已入官籍,更賜山西戶三百。西方多盜,郡縣捕不得,則法當計所失物直倍償,郡縣苦之。有甄軍判者,率群盜往來阜平、曲陽間,殺人渾源界而奪之財。縣以失捕當償,忽蘭曰:「此大盜也,縣豈能制哉!」即遣千人捕甄殺之,剿捕其餘黨,其害乃除。忽蘭性純篤,然酷好佛,嘗施千金修龍宮寺,建金輪大會,供僧萬人。卒年四十二。贈太保、金紫光祿大夫、上柱國,追封云國公,謚康忠。



 子天德於思,穎悟過人,世祖聞其賢,令襲父爵,養母完顏氏以孝聞。自中山北來,適有邊釁,天德於思督造兵甲,撫循其民,無有寧息,形容盡瘁。帝聞而嘉之,賜馴豹、名鷹,使得縱獵禁地,當時眷顧最號優渥。卒年三十九。贈太傅、儀同三司、上柱國,追封云國公,謚顯毅。子孫世多顯貴云。



 ○忙哥撒兒



 忙哥撒兒,察哈札剌兒氏。曾祖赤老溫愷赤,祖搠阿,父那海,並事烈祖。及太祖嗣位,年尚幼,所部多叛亡,搠阿獨不去。皇弟槊只哈撒兒陰擿之去,亦謝不從。搠阿精騎射,帝甚愛之,號為默爾傑,華言善射之尤者也。帝嘗與賊遇,將戰,有二飛鶩至,帝命搠阿謝之。請曰:「射其雄乎?抑雌者乎?」帝曰:「雄者。」搠阿一發墜其雄。賊望見,驚曰:「是善射若此,飛鳥且不能逃,況人乎!」不戰而去。從征乃蠻,敵率銳兵鼓而進,搠阿按兵屹不動,敵止。俄復鼓而進,搠阿亦不動,敵卒疑畏不敢前。太祖征蔑里吉,兵潰,搠阿與其弟左右力戰以衛帝。會兀良罕哲里馬來援,敵乃引退。那海事太祖,備歷艱險,未嘗形於言,帝嘉其忠,且念其世勛,詔封懷、洛陽百七十五戶。



 忙哥撒兒事睿宗,恭謹過其父。嘗從攻鳳翔,首立奇功。定宗升為斷事官,剛明能舉職。憲宗在籓邸,深知其人。從征斡羅思、阿速、欽察諸部,常身先諸將,及以所俘寶玉頒諸將,則退然一無所取。憲宗由是益重之,使治籓邸之分民。間出游獵,則長其軍士,動如紀律。雖太后及諸嬪御小有過失,知無不言,以故邸中人咸敬憚之。乃以為斷事官之長,其位在三公之上,猶漢之大將軍也。既拜命,出帳殿外,欹橐坐熊席,其僚列坐左右者四十人。忙哥撒兒問曰:「主上以我長此官,諸公其為我言,當以何道守官?」眾皆默然。又問之,有夏人和斡居下坐,進曰:「夫札魯忽赤之道,猶宰之刲羊也,解肩者不使傷其脊,在持平而已。」忙哥撒兒聞之,即起入帳內。眾不知所為,皆咎和斡失言。既入,乃為帝言和斡之言善。帝召和斡,命之步,曰:「是可用之才也。」和斡由是知名。



 定宗崩,宗王八都罕大會宗親,議立憲宗。畏兀八剌曰:「失烈門,皇孫也,宜立。且先帝嘗言其可以君天下。」諸大臣皆莫敢言。忙哥撒兒獨曰:「汝言誠是,然先皇后立定宗時,汝何不言耶?八都罕固亦遵先帝遺言也。有異議者,吾請斬之。」眾乃不敢異,八都罕乃奉憲宗立之。憲宗之幼也,太宗甚重之。一日行幸,天大風,入帳殿,命憲宗坐膝下,撫其首曰:「是可以君天下。」他日,用牸按豹,皇孫失烈門尚幼,曰:「以牸按豹,則犢將安所養?」太宗以為有仁心,又曰:「是可以君天下。」其後太宗崩,六皇后攝政,竟立定宗。故至是,二人各舉以為言云。



 憲宗既立,察哈臺之子及按赤臺等謀作亂,刳車轅,藏兵其中以入,轅折兵見,克薛傑見之,上變。忙哥撒兒即發兵迎之。按赤臺不虞事遽覺,倉卒不能戰,遂悉就擒。憲宗親簡其有罪者,付之鞫治。忙哥撒兒悉誅之。帝以其奉法不阿,委任益專。有當刑者,輒以法刑之,乃入奏,帝無不報可。帝或臥未起,忙哥撒兒入奏事,至帳前,扣箭房,帝問何言,即可其奏,以所御大帳行扇賜之。其見親寵如此。



 癸丑冬,病酒而卒。帝以忙哥撒兒當國時,多所誅戮,及是,咸騰謗言,乃為詔諭其子,略曰:



 汝高祖赤老溫愷赤暨汝祖搠阿,事我成吉思皇帝,皆著勞績,惟朕皇祖實褒嘉之。汝父忙哥撒兒,自其幼時,事我太宗,朝夕忠勤,罔有過咎。從我皇考,經營四方。迨事皇妣及朕兄弟,亦罔有過咎。暨朕討定斡羅思、阿速、穩兒別里欽察之域,濟大川,造方舟,伐山通道,攻城野戰,功多於諸將。俘厥寶玉,大賚諸將,則退然無欲得之心。惟朕言是用,修我邦憲,治我蒐田,輯我國家,罔不咸乂。惟厥忠,雖其私親,與朕嬪御,小有過咎,一是無有比私。故朕皇妣,迨朕昆弟,無不嘉賴。朝之老臣、宿衛耆舊,無不嚴畏。錄其勤勞,命為札魯忽赤,治朕皇考受民,布昭大公,以辨獄慎民,爰作朕股肱耳目,眾無嘩言,朕聽以安。



 自時厥後,察哈臺阿哈之孫,太宗之裔定宗、闊出之子,及其民人,越有他志。賴天之靈,時則有克薛傑者,以告於朕。汝父肅將大旅,以遏亂略,按赤臺等謀是用潰,悉就拘執。朕取有罪者,使辨治之,汝父體朕之公,其刑其宥,克比於法。又使治也速、不里獄,亦克比於法。



 惟爾脫歡、脫兒赤:自朕用汝父,用法不阿,兄弟親姻,咸麗於憲。今眾罔不怨,曰「爾亦有死耶」,若有慊志。人則雖死,朕將寵之如生。肆朕訓汝,爾克明時朕言,如是而有福,不如是而有禍。惟天惟君,能禍福人;惟天惟君,是敬是畏。立身正直,制行貞潔,是汝之福;反是勿思也。能用朕言,則不墜汝父之道,人亦不能間汝矣;不用朕言,則人將仇汝、伺汝、間汝。怨汝父者,必曰「汝亦與我夷矣」,汝則殆哉。汝於朕言,弗慎繹之,汝則有咎;克慎繹之,人將敬汝畏汝,無間伺汝,無慢汝怨汝者矣。又,而母而婦,有讒欺巧佞構亂之言,慎勿聽之,則盡善矣。



 至順四年,追封忙哥撒兒為兗國公。子四人:長脫歡,次脫兒赤,次也先帖木爾,次帖木兒不花。脫歡為萬戶,無子。脫兒赤子明禮帖木兒,累官翰林學士承旨,從征乃顏有功。明禮帖木兒子咬住,咬住子也先,延徽寺卿。也先帖木兒子曰哈剌合孫。帖木兒不花子曰塔術納,曰哈里哈孫,曰伯答沙。



 伯答沙幼入宿衛,為寶兒赤。歷事成宗、武宗,由光祿少卿擢同知宣徽院事,升銀青光祿大夫、宣徽院使,遙授左丞相。武宗崩,護梓宮葬於北,守山陵三年,乃還。仁宗即位,眷顧益厚。延祐二年,拜中書右丞相。時承平日久,朝廷清明,君臣端拱廟堂之上,而百姓乂安於下,一時號稱極治。仁宗崩,帖木迭兒執政,改授集賢大學士,仍開府儀同三司、錄軍國重事。未幾,以大宗正札魯忽赤出鎮北方,亦以清靜為治,邊民按堵。泰定間還朝,加太保。及倒剌沙構兵上都,兵潰,伯答沙奉璽紱來上,文宗嘉之。拜太傅,仍為札魯忽赤。至順三年薨。



 伯答沙為人清慎寬厚,號稱長者。其歿也,貧無以為斂,人皆嘆其廉。詔贈推忠佐理正德秉義功臣、開府儀同三司、太師、上柱國,追封威平王。



 三子:長馬馬的斤,次潑皮,次八郎。八郎期而孤,其母乞咬契氏二十而寡,守節不他適。八郎後為大宗正府札魯忽赤,能繼其先。有成立者,母氏之教也。



 ○孟速思



 孟速思,畏兀人,世居別失八里,古北庭都護之地。幼有奇質,年十五,盡通本國書。太祖聞之,召至闕下,一見大悅,曰:「此兒目中有火,它日可大用。」以授睿宗,使視顯懿莊聖皇后分邑歲賦。復事世祖於潛籓,日見親用。憲宗崩,孟速思言於世祖曰:「神器不可久曠,太祖嫡孫,唯王最長且賢,宜即皇帝位。」諸王塔察兒、也孫哥、合丹等,咸是其言。世祖即位,眷顧益重。南征時,與近臣不只兒為斷事官。及諸王阿里不哥叛,相拒漠北,不只兒有二心,孟速思知之,奏徙之於中都,親監護以往,帝以為忠。數命收召豪俊,凡所引薦,皆極其選。詔與安童並拜丞相,固辭。帝語安童及丞相伯顏、御史大夫月魯那演等曰:「賢哉孟速思,求之彼族,誠為罕也。」孟速思為人剛嚴謹信。蚤居帷幄,謀議世莫得聞。至元四年卒,年六十有二。帝尤哀悼,特謚敏惠。武宗朝,贈推忠同德佐理功臣、太師、開府儀同三司、上柱國,追封武都王,改謚智敏。子九人,多至大官。



\end{pinyinscope}