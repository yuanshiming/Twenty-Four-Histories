\article{志第八 歷五}

\begin{pinyinscope}

 ○庚午元歷上



 演紀上元庚午,距太祖庚辰歲,積年二千二十七萬五千二百七十算外,上考往古,每年減一算,下驗將來,每年加一算。



 步氣朔術



 日法,五千二百三十。



 歲實,一百九十一萬二百二十四。



 通餘,二萬七千四百二十四。



 朔實,一十五萬四千四百四十五。



 通閏,五萬六千八百八十四。



 歲策,三百六十五,餘一千二百七十四。



 朔策,二十九,餘二千七百七十五。



 氣策,一十五,餘一千一百四十二,秒六十。



 望策,一十四,餘四千二,秒四十五。



 象策,七,餘二千一,秒二十二半。



 沒限,四千八十七,秒三十。



 朔虛分,二千四百五十五。



 旬周,三十一萬三千八百。



 紀法,六十。



 秒母,九十。



 求天正冬至



 置上元庚午以來積年,以歲實乘之,為通積分;滿旬周,去之,不盡,以日法約之,為日,不盈,為餘;命壬戌算外,即得所求天正冬至大小餘也。先以里差加減通積分,然後求之。求里差術,具《月離》篇中。



 求次氣



 置天正冬至大小餘,以氣策及餘累加之,秒盈秒母從分,分滿日法從日,即得次氣日及餘分秒。



 求天正經朔



 置通積分,滿朔實去之,不盡,為閏餘;以減通積分,為朔積分;滿旬周,去之,不盡,如日法而一,為日,不盡,為餘,即得所求天正經朔大小餘也。



 求弦望及次朔



 置天正經朔大小餘,以象策累加之,即各得弦望及次朔經日及餘秒也。



 求沒日



 置有沒之氣恆氣小餘,如沒限以上,為有沒之氣;以秒母乘之,內其秒,用減四十七萬七千五百五十六;餘,滿六千八百五十六而一;所得並入恆氣大餘內,命壬戌算外,即得為沒日也。



 求滅日



 置有滅之朔小餘,經朔小餘不滿朔虛分者。六因之,如四百九十一而一;所得並經朔大餘,命為滅日。



 步卦候發斂術



 候策,五,餘三百八十,秒八十。



 卦策,六,餘四百五十七,秒六。



 貞策,三,餘二百二十八,秒四十八。



 秒母,九十。



 辰法,二千六百一十五。



 半辰法,一千三百七半。



 刻法,三百一十三,秒八十。



 辰刻,八,分一百四,秒六十。



 半辰刻,四,分五十二,秒三十。



 秒母,一百。



 求七十二候



 置節氣大小餘,命之為初候;以候策累加之,即得次候及末候也。



 求六十四卦



 置中氣大小餘,命之為公卦;以卦策累加之,得闢卦;又加,得候內卦;以貞策加之,得節氣之初,為候外卦;又以貞策加之,得大夫卦;又以卦策加之,為卿卦也。



 求土王用事



 以貞策減四季中氣大小餘,即得土王用事日也。



 求發斂



 置小餘,以六因之,如辰法而一,為辰數;不盡,以刻法除為刻,命子正算外,即得加時所在辰刻分也。如加半辰法,即命子初。



 求二十四氣卦候



 以下表格略



 步日躔術



 周天分,一百九十一萬二百九十二,秒九十八。



 歲差,六十八,秒九十八。



 秒母,一百。



 周天度,三百六十五,分二十五,秒六十七。



 象限,九十一,分三十一,秒九。



 分秒母,一百。



 二十四氣日積度盈縮



 表略



 二十四氣中積及朓朒



 表略



 求每日盈縮朓朒



 各置其氣損益率,求盈縮,用盈縮之損益;求朓朒,用朓朒之損益。六因,如象限而一,為其氣中率;與後氣中率相減,為合差;半合差,加減其氣中率,為元末泛率,至後,加初減末;分後,減初加末。又置合差,六因,如象限而一,為日差;半之,加減初末泛率,為初末定率;至後,減初加末;分後,加初減末。以日差累加減氣初定率,為每日損益分;至後,減;分後,加。



 各以每日損益分加減氣下盈縮朓朒,為每日盈縮朓朒。二分前一氣無後率相減為合差者,皆用前氣合差。



 求經朔弦望入氣



 置天正閏餘,以日法除為日,不滿,為餘。如氣策以下,以減氣策,為入大雪氣;以上,去之,餘亦以減氣策,為入小雪氣;即得天正經朔入氣日及餘也。以象策累加之,滿氣策去之,即為弦望入次氣日及餘;因加得後朔入氣日及餘也。便為中朔望入氣。



 求每日損益盈縮朓朒



 以日差益加損減其氣初損益率,為每日損益率;馴積損益其氣盈縮朓朒積,為每日盈縮朓朒積。



 求經朔弦望入氣朓朒定數



 以各所求入氣小餘,以乘其日損益率,如日法而一;所得,損益其下朓朒積,為定數。便為中朔弦望朓朒定數。



 赤道宿度



 斗二十五〓〓牛七少〓〓女十一少〓〓虛九少六十七秒〓〓危十五度半〓〓室十七〓〓壁八太



 右北方七宿,九十四度六十七秒。〓



 奎十六半〓〓婁十二〓〓胃十五〓〓昴十一少〓〓畢十七少〓觜半〓參十半〓



 右西方七宿,八十三度。



 井三十三少〓鬼二半〓柳十三太〓星六太〓張十七少〓翼十八〓軫十七



 右南方七宿,一百九度少。



 角十二〓亢九少〓氐十六〓房五太〓心六少〓尾十九少〓箕十半



 右東方七宿,七十九度。



 求冬至赤道日度



 置通積分,以周天分去之;餘,日法而一,為度,不滿,退除為分秒;以百為母,命起赤道虛宿六度外,去之,不滿宿,即得所求年天正冬至加時日躔赤道宿度及分秒。其在尋斯干之東西者,先以里差加減通積分。



 求春分夏至秋分赤道日度



 置天正冬至加時赤道日度,累加象限,滿赤道宿次,去之,即各得春分、夏至、秋分加時日在宿度及分秒。



 求四正赤道宿積度



 置四正赤道宿全度,以四正赤道日度及分秒減之,餘為距後度;以赤道宿度累加之,各得四正後赤道宿積度及分秒。



 求赤道宿積度入初末限



 視四正後赤道宿積度及分,在四十五度六十五分五十四秒半以下,為入初限;以上者,用減象限,餘為入末限。



 求二十八宿黃道度



 置四正後赤道宿入初末限度及分,減一百一度;餘,以初末限度及分乘之,進位,滿百為分,分滿百為度;至後以減、分後以加赤道宿積度,為其宿黃道積度;以前宿黃道積度減之,其四正之宿,先加象限,然後以前宿減之。



 為其宿黃道度及分。其分就近約為太半少。



 黃道宿度



 斗二十三〓牛七〓女十一〓虛九少六十七秒〓危十六〓室十八少〓壁九半



 右北方七宿,九十四度六十七秒。



 奎十七太〓婁十二太〓胃十五半〓昴十二〓畢十六半〓觜半〓參九太



 右西方七宿,八十三度太。



 井三十半〓鬼二半〓柳十三少〓星六太〓張十七太〓翼二十〓軫十八半



 右南方七宿,一百九度少。



 角十二太〓亢九太〓氐十六少〓房五太〓心六〓尾十八少〓箕九半



 右東方七宿,七十八度少。



 前黃道宿度,依今歷歲差所在算定。如上考往古,下驗將來,當據歲差,每移一度,依術推變當時宿度,然後可步七曜,知其所在。



 求天正冬至加時黃道日度



 以冬至加時赤道日度分秒,減一百一度,餘以冬至加時赤道日度及分秒乘之,進位,滿百為分,分滿百為度,命曰黃赤道差;用減冬至加時赤道日度及分秒,即得所求年天正冬至加時黃道日度及分秒。



 求二十四氣加時黃道日度



 置所求年冬至日躔黃赤道差,以次年黃赤道差減之,餘以所求氣數乘之,二十四而一;所得,以加其氣中積度及約分,以其氣初日盈縮數盈加縮減之,用加冬至加時黃道日度,依宿次去之,即各得其氣加時黃道日躔宿度及分秒。如其年冬至加時赤道宿度空分秒在歲差以下者,即加前宿全度,然求黃赤道差,餘依術算。



 求二十四氣及每日晨前夜半黃道日度



 副置其恆氣小餘,以其氣初日損益率乘之,盈縮之損益。萬約之,應益者盈加縮減,應損者盈減縮加,其副日法除之,為度,不滿,退除為分秒,以減其氣加時黃道日度,即得其氣初日晨前夜半黃道日度。每日加一度,以萬乘之,又以每日損益數,盈縮之損益。應益者盈加縮減,應損者盈減縮加,為每日晨前夜半黃道日度及分秒。



 求每日午中黃道日度



 置一萬分,以所求入氣日損益數加減,益者,盈加縮減;損者,盈減縮加。半之,滿百為分,不滿為秒,以加其日晨前夜半黃道日度,即其日午中日躔黃道宿度及分秒。



 求每日午中黃道積度



 以二至加時黃道日度,距至所求日午中黃道日度,為入二至後黃道日積度及分秒。



 求每日午中黃道入初末限



 視二至後黃道積度,在四十三度一十二分八十七秒之以下為初限;以上,用減象限,餘為入末限。其積度,滿象限去之,為二分後黃道積度;在四十八度一十八分二十一秒之以下,為初限;以上,用減象限,餘為入末限。



 求每日午中赤道日度



 以所求日午中黃道積度,入至後初限、分後末限度及分秒,進三位,加二十萬二千五十少,開平方除之,所得減去四百四十九半,餘在初限者,直以二至赤道日度加而命之;在末限者,以減象限,餘以二分赤道日度加而命之,即每日午中赤道日度。



 以所求日午中黃道積度,入至後末限、分後初限度及分秒,進三位,用減三十萬三千五十少,開平方除之,所得,以減五百五十半,其在初限者,以所減之餘,直以二分赤道日度加而命之;在末限者,以減象限,餘以二至赤道日度加而命之,即每日午中赤道日度。



 太陽黃道十二次入宮宿度



 危〓十三度三十九分五十九秒外入衛分陬訾之次,辰在亥。



 奎〓二度三十五分八十五秒外入魯分降婁之次,辰在戌。



 胃〓四度二十四分三十三秒外入趙分大梁之次,辰在酉。



 畢〓七度九十六分二十秒外入晉分實沈之次,辰在申。



 井〓九度四十七分一十秒外入秦分鶉首之次,辰在未。



 柳〓四度九十五分二十六秒外入周分鶉火之次,辰在午。



 張〓十五度五十六分三十五秒外入楚分鶉尾之次,辰在巳。



 軫〓十度四十四分五秒外入鄭分壽星之次,辰在辰。



 氐〓一度七十七分七十七秒外入宋分大火之次,辰在卯。



 尾〓三度九十七分七十二秒外入燕分析木之次,辰在寅。



 斗〓四度三十六分六十六秒外入吳越分星紀之次,辰在醜。



 女〓二度九十一分九十一秒外入齊分玄枵之次,辰在子。



 求入宮時刻



 各置入宮宿度及分秒,以其日晨前夜半日度減之,相近一度之間者求之。餘以日法乘其分,其秒從於下,亦通乘之。為實;以其日太陽行分為法;實如法而一,所得,依發斂加時求之,即得其日太陽入宮時刻及分秒。



 步晷漏術



 中限,一百八十二日六十二分一十八秒。



 冬至初限、夏至末限,六十二日二十分。



 夏至初限、冬至末限,一百二十日四十二分。



 冬至永安晷影常數,一丈二尺八寸三分。



 夏至永安晷影常數,一尺五寸六分。



 周法,一千四百二十八。



 內外法,一萬八百九十六。



 半法,二千六百一十五。



 日法四分之三,三千九百二十二半。



 日法四分之一,一千三百七半。



 昏明分,一百三十分七十五秒。



 昏明刻,二刻一百五十六分九十秒。



 刻法,三百一十三分八十秒。



 秒母,一百。



 求午中入氣中積



 置所求日大餘及半法,以所入氣大小餘減之,為其日午中入氣;以加其氣中積,為其日午中中積。小餘以日法除,為約分。



 求二至後午中入初末限



 置午中中積及分,如中限以下,為冬至後;以上,去中限,為夏至後。其二至後,如在初限以下,為初限;以上,覆減中限,餘為入末限也。



 求午中晷影定數



 視冬至後初限、夏至後末限,百通日內分,自相乘,副置之,以一千四百五十除之;所得,加五萬三百八,折半限分並之,除其副為分,分滿十為寸,寸滿十為尺,用減冬至地中晷影常數,為所求晷影定數。



 視夏至後初限、冬至後末限,百通日內分,自相乘,為上位;下置入限分,以二百二十五乘之,百約之,加一十九萬八千七十五,為法;夏至前後半限以上者,減去半限,列於上位,下置半限,各百通日內分,先相減,後相乘,以七千七百除之,所得以加其法。及除上位為分,分滿十為寸,寸滿十為尺,用加夏至地中晷影常數,為所求晷影定數。



 求四方所在晷影



 各於其處測冬夏二至晷數,乃相減之,餘為其處二至晷差;亦以地中二至晷數相減,為地中二至晷差。其所求日在冬至後初限、夏至後末限者,如在半限以下,倍之;半限以上,覆減全限,餘亦倍之;並入限日,三因,折半,以日為分,十分為寸,以減地中二至晷差,為法;置地中冬至晷影常數,以所求日地中晷影定數減之,餘以其處二至晷差乘之,為實;實如法而一,所得,以減其處冬至晷數,即得其處其日晷影定數。所求日在夏至後初限、冬至後末限者,如在半限以下,倍之;半限以上,覆減全限,餘亦倍之;並入限日,三因,四除,以日為分,十分為寸,以加地中二至晷差,為法;置所求日地中晷影定數,以地中夏至晷影常數減之,餘以其處二至晷差乘之,為實;實如法而一,所得,以加其處夏至晷數,即得其處其日晷影定數。



 二十四氣陟降及日出分



 以下表格略



 二分前後陟降率



 春分前三日,太陽入赤道內,秋分後三日,太陽出赤道外,故其陟降與他日不倫,今各別立數而用之。



 驚蟄,十二日陟四。六十七、一十六。此為末率,於此用畢。其減差亦止於此也。



 十三日陟四。四十一、六。十四日陟四。三十八、九十。



 十五日陟四。



 秋分,初日降四。三十八。一日降四。二十九。二日降四。五十九。三日降四。六十八。



 此為初率,始用之。其加差亦始於此也。



 求每日日出入晨昏半晝分



 各以陟降初率,陟減降加其氣初日日出分,為一日下日出分;以增損差仍加減加減差。增損陟降率,馴積而加減之,即為每日日出分;覆減日法,餘為日入分;以日出分減日入分,半之,為半晝分;以昏明分減日出分,為晨分;加日入分,為昏分。



 求日出入辰刻



 置日出入分,以六因之,滿辰法而一,為辰數;不盡,刻法除之,為刻,不滿為分。命子正算外,即得所求。



 求晝夜刻



 置日出分,十二乘之,刻法而一,為刻,不滿為分,即為夜刻;覆減一百,餘為晝刻及分秒。



 求更點率



 置晨分,四因之,退位,為更率;二因更率,退位,為點率。



 求更點所在辰刻



 置更點率,以所求更點數因之,又六因之,內加更籌刻,滿辰法而一,為辰數;不盡,滿刻法,除之,為刻數;不滿,為分;命其日辰刻算外,即得所求。



 求四方所在漏刻



 各於所在下水漏,以定其處冬至或夏至夜刻,乃與五十刻相減,餘為至差刻。置所求日黃道去赤道內外度及分,以至差刻乘之,進一位,如二百三十九而一,為刻;不盡,以刻法乘之,退除為分;內減外加五十刻,即得所求日夜刻;以減百刻,餘為晝刻。其日出入辰刻及更點差率等,並依前術求之。



 求黃道內外度



 置日出之分,如日法四分之一以上,去之,餘為外分;如日法四分之一以下,覆減之,餘為內分。置內外分,千乘之,如內外法而一,為度,不滿,退除為分秒,即為黃道去赤道內外度;內減外加象限,即得黃道去極度。



 求距中度及更差度



 置半法,以晨分減之,餘為距中分;百乘之,如周法而一,為距中度;用減一百八十三度一十二分八十三秒半,餘四因,退位,為每更差度。



 求昏明五更中星



 置距中度,以其日午中赤道日度加而命之,即昏中星所格宿次,因為初更中星;以更差度累加之,滿赤道宿次,去之,即得逐更及明中星。



 步月離術



 轉終分,一十四萬四千一百一十,秒六千二十,微六十。



 轉終日,二十七,餘二千九百,秒六千二十,微六十。



 轉中日,一十三,餘四千六十五,秒三千一十,微三十。



 朔差日,一,餘五千一百四,秒三千九百七十九,微四十。



 象策,七,餘二千一,秒二千五百。



 秒母,一萬。



 微母,一百。



 上弦度,九十一,分三十一,秒四十一太。



 望度,一百八十二,分六十二,秒八十三半。



 下弦度,二百七十三,分九十四,秒二十五少。



 月平行度,十三,分三十六,秒八十七半。



 分秒母,一百。



 七日初數,四千六百四十八,末數,五百八十二。



 十四日初數,四千六十五,末數,一千一百六十五。



 二十一日初數,三千四百八十三,末數,一千七百四十七。



 二十八日初數,二千九百一。



 求經朔弦望入轉凡稱秒者,微從之,他仿此。



 置天正朔積分,以轉終分及秒去之,不盡,如日法而一,為日,不滿為餘秒,即天正十一月經朔入轉日及餘秒;以象策累加之,去命如前,得弦望經日加時入轉及餘秒;徑求次朔入轉,即以朔差加之。加減里差,即得中朔弦望入轉及餘秒。



 以下表格略



 求中朔弦望入轉朓朒定數



 置入轉小餘,以其日算外損益率乘之,如日法而一,所得,以損益朓朒積,為定數。其四七日下餘,如初數以下,初率乘之,如初數而一,以損益朓朒積,為定數;如初數以上,以初數減之,餘乘末率,如末數而一,用減初率,餘如朓朒積,為定數。其十四日下餘,如初數以上,以初數減之,餘乘末率,如末數而一,為朓朒定數。



 求朔弦望中日



 以尋斯干城為準,置相去地里,以四千三百五十九乘之,退位,萬約為分,曰里差;以加減經朔弦望小餘,滿與不足,進退大餘,即中朔弦望日及餘。以東加之,以西減之。



 求朔弦望定日



 置中朔弦望小餘,朓減朒加入氣入轉朓朒定數,滿與不足,進退大餘,命壬戌算外,各得定朔弦望日辰及餘。定朔幹名與後朔同者,其月大;不同者,其月小;月內無中氣者,為閏。視定朔小餘,秋分後在日法四分之三以上者,進一日;春分後,定朔日出分與春分日出分相減之,餘者,三約之,用減四分之三;定朔小餘及此分以上者,亦進一日;或有交,虧初於日入前者,不進之。定弦望小餘,在日出分以下者,退一日;或有交,虧初於日出前者,小餘雖在日出後,亦退之。如望在十七日者,又視定朔小餘在四分之三以下之數,春分後用減定之數。與定望小餘在日出分以上之數相校之,朔少望多者,望不退,而朔猶進之;望少朔多者,朔不進,而望猶退之。日月之行,有盈縮遲疾;加減之數,或有四大三小。若循常當察加時早晚,隨所近而進退之,使不過四大三小。



 求定朔弦望中積



 置定朔弦望小餘,與中朔弦望小餘相減之,餘以加減經朔弦望入氣日餘,中朔弦望,少即加之,多即減之。即為定朔弦望入氣;以加其氣中積,即為定朔弦望中積。其餘,以日法退除為分秒。



 求定朔弦望加時日度



 置定朔弦望約餘,以所入氣日損益率乘之,盈縮之損益。萬約之,以損益其下盈縮積,乃盈加縮減定朔弦望中積,又以冬至加時日躔黃道宿度加之,依宿次去之,即得定朔弦望加時日所在度分秒。



 又法:置定朔弦望約餘,副之,以乘其日盈縮之損益率,萬約之,應益者盈加縮減,應損者盈減縮加,其副滿百為分,分滿百為度,以加其日夜半日度,命之,各得其日加時日躔黃道宿次。若先於歷中注定每日夜半日度,即用此法為準也。



 求定朔弦望加時月度



 凡合朔加時日月同度,其定朔加時黃道日度即為定朔加時黃道月度;弦望,各以弦望度加定朔弦望加時黃道日度,依宿次去之,即得定朔弦望加時黃道月度及分秒。



 求夜半午中入轉



 置中朔入轉,以中朔小餘減之,為中朔夜半入轉。又中朔小餘,與半法相減之,餘以加減中朔加時入轉,中朔少如半法,加之;多如半法,減之。為中朔午中入轉。若定朔大餘有進退者,亦加減轉日,否則因中為定,每日累加一日,滿轉終日及餘秒,去命如前,各得每日夜半午中入轉。求夜半,因定朔夜半入轉累加之;求午中,因定朔午中入轉累加之;求加時入轉者,如求加時入氣之術法。



 求加時及夜半月度



 置其日入轉算外轉定分,以定朔弦望小餘乘之,如日法而一,為加時轉分;分滿百為度。減定朔弦望加時月度,為夜半月度。以相次轉定分累加之,即得每日夜半月度。或朔至弦望,或至後朔,皆可累加之。然近則差少,遠則差多。置所求前後夜半相距月度為行度,計其日相距入轉積度,與行度相減,餘以相距日數除之,為日差行度。多日差加每日轉定分行度,少日差減每日轉定分而用之可也。欲求速,即用此數。欲究其微,而可用後術。



 求晨昏月度



 置其日晨分,乘其日算外轉定分,日法而一,為晨轉分;用減轉定分,餘為昏轉分。又以朔望定小餘,乘轉定分,日法而一,為加時分,以減晨昏轉分,為前;不足,覆減之,為後;乃前加後減加時月度,即晨昏月度所在宿度及分秒。



 求朔弦望晨昏定程



 各以其朔昏定月減上弦昏定月,餘為朔後昏定程。以上弦昏定月,減望昏定月,餘為上弦後昏定程。以望晨定月,減下弦晨定月,餘為望後晨定程。以下弦晨定月,減後朔晨定月,餘為下弦後晨定程。



 求每日轉定度



 累計每定程相距日下轉積度,與晨昏定程相減,餘以相距日數除之,為日差;定程多,加之;定程少,減之。以加減每日轉定分,為轉定度;因朔弦望晨昏月,每日累加之,滿宿次去之,為每日晨昏月度及分秒。凡注歷,朔日已後注昏月,望後一日注晨月。古歷有九道月度,其數雖繁,亦難削去,具其術。



 求正交日辰



 置交終日及餘秒,以其月經朔加時入交泛日及餘秒減之,餘為平交入其月經朔加時後日算及餘秒;中朔同。以加其月中朔大小餘,其大餘命壬戌算外,即得平交日辰及餘秒。求次交者,以交終日及餘秒加之,如大餘滿紀法,去之,命如前,即得次平交日辰及餘秒也。



 求平交入轉朓朒定數



 置平交小餘,加其日夜半入轉,餘以乘其日損益率,日法而一,所得,以損益其日下朓朒積,為定數。



 求平交日辰



 置平交小餘,以平交入轉朓朒定數朓減朒加之,滿與不足,進退日辰,即得正交日辰及餘秒;與定朔日辰相距,即得所在月日。



 求中朔加時中積



 各以其月中朔加時入氣日及餘,加其氣中積及餘,其日命為度,其餘,以日法退除為分秒,即其月中朔加時中積度及分秒。



 求正交加時黃道月度



 置平交入中朔加時後日算及餘秒,以日法通日內餘進二位,如三萬九千一百二十一為度,不滿,退除為分秒,以加其月中朔加時中積,然後以冬至加時黃道日度加而命之,即得其月正交加時月離黃道宿度及分秒。如求次交者,以交中度及分秒加而命之,即得所求。



 求黃道宿積度



 置正交加時黃道宿全度,以正交加時月離黃道宿度及分秒減之,餘為距後度及分秒;以黃道宿度累加之,即各得正交後黃道宿積度及分秒。



 求黃道宿積度入初末限



 置黃道宿積度及分秒,滿交象度及分秒去之,餘在半交象以下為初限;以上者,減交象度,餘為末限。入交積度、交象度,並在《交會篇》中。



 求月行九道宿度



 凡月行所交,冬入陰歷,夏入陽歷,月行青道;冬至夏至後,青道半交在春分之宿,當黃道東;立冬立夏後,青道半交在立春之宿,當黃道東南;至所沖之宿,亦皆如之也。宜細推。冬入陽歷,夏入陰歷,月行白道;冬至夏至後,白道半交在秋分之宿,當黃道西;立冬立夏後,白道半交在立秋之宿,當黃道西北;至所沖之宿,亦如之也。春入陽歷,秋入陰歷,月行硃道;春分秋分後,硃道半交在夏至之宿,當黃道南;立春立秋後,硃道半交在立夏之宿,當黃道西南;至所沖之宿,亦如之也。春入陰歷,秋入陽歷,月行黑道。春分秋分後,黑道半交在冬至之宿,當黃道北;立春立秋後,黑道半交在立冬之宿,當黃道東北;至所沖之宿,亦如之也。四時離為八節,至陰陽之所交,皆與黃道相會,故月行有九道。各以所入初入初末限度及分,減一百一度,餘以所入初入初末限度及分乘之,半而退位為分,分滿百為度,命為月道與黃道泛差。



 凡日以赤道內為陰,外為陽;月以黃道內為陰,外為陽。故月行正交,入夏至後宿度內為同名,入冬至後宿度內為異名。其在同名者,置月行與黃道泛差,九因之,八約之,為定差;半交後,正交前,以差減;正交後,半交前,以差加;此加減出入六度,正如黃赤道相交同名之差,若較之漸異,則隨交所在遷變不常。仍以正交度距秋分度數,乘定差,如象限而一,所得,為月道與赤道定差;前加者為減,減者為加。其在異名者,置月行與黃道泛差,七因之,八約之,為定差;半交後,正交前,以差加;正交後,半交前,以差減;此加減出入六度,正如黃赤道相交異名之差,若較之漸同,則隨交所在遷變不常。仍以正交度距春分度數,乘定差,如象限而一,所得,為月道與赤道定差;前加者為減,減者為加,各加減黃道宿積度,為九道宿積度;以前宿九道積度減之,為其宿九道度及分秒。其分就近約為太、半、少,論春夏秋冬,以四時日所在宿度為正。



 求正交加時月離九道宿度



 以正交加時黃道日度及分,減一百一度,餘以正交度及分乘之,半而退位為分,分滿百為度,命為月道與黃道泛差。其在同名者,置月行與黃道泛差,九因之,八約之,為定差,以加;仍以正交度距秋分度數乘定差,如象限而一,所得,為月道與赤道定差,以減。其異名者,置月行與黃道泛差,七因之,八約之,為定差,以減;仍以正交度距春分度數,乘定差,如象限而一,所得,為月道與赤道定差,以加。置正交加時黃道月度及分,以二差加減之,即為正交加時月離九道宿度及分。



 求定朔弦望加時月所在度



 置定朔加時日躔黃道宿次,凡合朔加時,月行潛在日下,與太陽同度,是為加時月離宿次;各以弦望度及分秒,加其所當弦望加時日躔黃道宿度,滿宿次,去之,命如前,各得定朔弦望加時月所在黃道宿度及分秒。



 求定朔弦望加時九道月度



 各以定朔弦望加時月離黃道宿度及分秒,加前宿正交後黃道積度,為定朔弦望加時正交後黃道積度;如前求九道積度,以前宿九道積度減之,餘為定朔弦望加時九道月離宿度及分秒。其合朔加時,若非正交,則日在黃道,月在九道,所入宿度雖多少不同,考其兩極若繩準。故云月行潛在日下,與太陽同度,即為加時。九道月度,求其晨昏夜半月度,並依前術。



\end{pinyinscope}