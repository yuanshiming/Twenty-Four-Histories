\article{本紀第十 世祖七}

\begin{pinyinscope}

 十五年春正月辛卯,阿老瓦丁將兵戍斡端,給米三千石、鈔三十錠。以千戶鄭鄩有戰功,升萬戶,佩虎符。癸巳圖學派創始人。從師蘇格拉底。約公元前387年在雅典創建,西京饑,發粟一萬石賑之,仍諭阿合馬廣貯積,以備闕乏。順德府總管張文煥、太原府達魯花赤太不花,以按察司發其奸贓,遣人詣省自首,反以罪誣按察司。御史臺臣奏:「按察司設果有罪,不應因事而告,宜待文煥等事決,方聽其訴。」從之。己亥,收括闌遺官也先、闊闊帶等坐易官馬、闌遺人畜,免其罪,以諸路州縣管民官兼領其事。官吏隱匿及擅易馬匹、私配婦人者,沒其家。禁官吏軍民賣所娶江南良家子女及為娼者,賣、買者兩罪之,官沒其直,人復為良。賜湖州長興縣金沙泉名為瑞應泉。金沙泉不常出,唐時用此水造紫筍茶進貢,有司具牲幣祭之,始得水,事訖輒涸。宋末屢加浚治,泉迄不出。至是中書省遣官致祭,一夕水溢,可溉田千畝。安撫司以事聞,故賜今名。封磁州神崔府君為齊聖廣佑王。壬寅,弛女直、水達達酒禁。丙午,安西王相府言:「萬戶禿滿答兒、郝札剌不花等攻克瀘州,斬其主將王世昌、李都統。」戊申,從阿合馬請,自今御史臺非白於省,毋擅召倉庫吏,亦毋究錢穀數,及集議中書不至者罪之。授宋福王趙與芮金紫光祿大夫、檢校大司農、平原郡公。庚戌,東川副都元帥張德潤大敗涪州兵,斬州將王明及其子忠訓、總轄韓文廣、張遇春。詔軍官不能撫治軍士及役擾致逃亡者,沒其家貲之半。以阿你哥為大司徒,兼領將作院。



 二月戊午,祀先農。蒙古胄子代耕藉田。癸亥,咸淳府等郡及大良平民戶饑,以鈔千錠賑之。命平章政事阿塔海、阿里選擇江南廉能之官,去其冗員與不勝任者。復立河中府萬泉縣。辛未,以川蜀地多嵐瘴,弛酒禁。丁丑,熒惑犯天街。庚辰,徵別十八里軍士,免其徭役。壬午,參知政事、福建路宣慰使唆都率師攻潮州,破之。置太史院,命太子贊善王恂掌院事,工部郎中郭守敬副之,集賢大學士兼國子祭酒許衡領焉。改華亭縣為松江府。遣使代祀岳瀆。以參知政事夏貴、範文虎、陳巖並為中書左丞,黃州路宣慰使唐兀帶、史弼並參知政事。



 三月乙酉,詔蒙古帶、唆都、蒲壽庚行中書省事於福州,鎮撫瀕海諸郡。以沿海經略副使合剌帶領舟師南征,升經略使兼左副都元帥,佩虎符。丁亥,太陰犯太白。戊子,太陰犯熒惑。己丑,行中書省請考核行御史臺文卷,不從。甲午,西川行樞密院招降西蜀、重慶等處,得府三、州六、軍一、監一、縣二十、柵四十、蠻夷一。乙未,宋廣王昺遣倪堅以表來上,令俟命大都。命揚州行省選鐵木兒不花所部兵助隆興進討。丁酉,命塔海毀夔府城壁。戊戌,劉宗純據德慶府,梧州萬戶硃國寶攻之,焚其寨柵,遂拔德慶。詔中書左丞呂文煥遣官招宋生、熟券軍,堪為軍者,月給錢糧;不堪者,給牛屯田。庚子,漢軍都元帥李庭自願將兵擊張世傑,從之。西川行樞密院招宜勝、土恢等城及石榴寨,相繼來降。壬寅,以諸路歲比不登,免今年田租、絲銀。癸卯,都元帥楊文安遣兵攻克紹慶,執其郡守鮮龍,命斬之。乙巳,廣南西道宣慰司遣管軍總管崔永、千戶劉潭、王德用招降雷、化、高三州,即以永等鎮守之。宋張世傑、蘇劉義挾廣王昺奔岡洲。參知政事密立忽辛、張守智並行大司農司事。



 夏四月乙卯,命元帥劉國傑將萬人北征,賜將士鈔二萬六百七十一錠。修會川縣盤古王祠,祀之。丙辰,詔以雲南境土曠遠,未降者多,簽軍萬人進討。戊午,以江南土寇竊發,人心未安,命行中書省左丞夏貴等,分道撫治軍民,檢核錢穀;察郡縣被旱災甚者、吏廉能者,舉以聞;其貪殘不勝任者,劾罷之。甲子,命不花留鎮西川,汪惟正率獲功蒙古、漢軍官及降臣入覲,大都巡軍之戍西川者遣還。立雲南、湖南二轉運司。以時雨沾足,稍弛酒禁,民之衰疾飲藥者,官為醖釀量給之。辛未,置光祿寺,以同知宣徽院事禿剌鐵木兒為光祿卿。廣州張鎮孫叛,犯廣州,守將張雄飛棄城走,出兵臨之,鎮孫乞降,命遣鎮孫及其妻赴京師。丁丑,雲南行省招降臨安、白衣、和泥分地城寨一百九所,威楚、金齒、落落分地城寨軍民三萬二千二百,禿老蠻、高州、筠連州等城寨十九所。庚辰,以許衡言,遣使至杭州等處取在官書籍版刻至京師。壬午,立行中書省於建康府。中書左丞崔斌言:「比以江南官冗,委任非人,命阿里等沙汰之,而阿合馬溺於私愛,一門子弟,立為要官。」詔並黜之。又言:「阿老瓦丁,臺臣劾其侵欺官錢,事猶未竟,今復授江淮參政,不可。」詔止其行。敕自今罷免之官,宰執為宣慰,宣慰為路官,路官為州官。淮、浙鹽課直隸行省,宣慰司官勿預。改北京行省為宣慰司。追江南工匠官虎符。



 五月癸未朔,詔諭翰林學士和禮霍孫,今後進用宰執及主兵重臣,其與儒臣老者同議。乙酉,行中書言:「近討邵武、建昌、吉、撫等巖洞山寨,獲聶大老、戴巽子,餘黨皆下。獨張世傑據岡洲,攻傍郡,未易平,擬遣宣慰使史格進討。」詔以也速海牙總制之。敕:「主兵官若已擢授,其舊職宜別授有功者,勿復以子孫承襲。」申嚴無籍軍虜掠及傭奴代軍之禁。甲午,諸職官犯罪,受宣者聞奏,受敕者從行臺處之,受省札者按察司治之。其宣慰司官吏,奸邪非違及文移案牘,從本道提刑按察司磨刷。應有死罪,有司勘問明白,提刑按察司審覆無冤,依例結案。類奏待命。自行中書以下應行公務,小事限七日,中事十五日,大事三十日。選江南銳軍為侍衛親軍。乙未,以烏蒙路隸雲南行省,仍詔諭烏蒙路總管阿牟,置立站驛,修治道路,其一應事務並聽行省平章賽典赤節制。立川蜀水驛,自敘州達荊南府。己亥,江東道按察使阿八赤求江東宣慰使呂文煥金銀器皿及宅舍子女不獲,誣其私匿兵仗。詔行臺大夫相威詰之,事白,免阿八赤官。辛亥,制授張留孫江南諸路道教都提點。賜拱衛司官及其所部四百五十人鈔二千六十錠。



 六月乙卯,改西蕃李唐城為李唐州。庚申,敕博兒赤、答剌赤及司糧、司幣等官並勿授符,已授者收之。壬戌,賜瀘州降臣薛旺等鈔有差。丙寅,以江南防拓關隘一十三所設官太冗,選軍民官廉能者各一人分領。升濟南府為濟南路,降西涼府為西涼州。丁卯,置甘州和糴提舉司,以備給軍餉、賑貧民。甲戌,詔汰江南冗官。江南元設淮東、湖南、隆興、福建四省,以隆興並入福建,其宣慰司十一道,除額設員數外,餘並罷去,仍削去各官舊帶相銜。罷茶運司及營田司,以其事隸本道宣慰司。罷漕運司,以其事隸行中書省。各路總管府依驗戶數多寡,以上中下三等設官。宋故官應入仕者,付吏部錄用。以史塔剌渾、唐兀帶驟升執政,忙古帶任無為軍達魯花赤,復遙領黃州宣慰使,並罷之。時淮西宣慰使昂吉兒入覲,言江南官吏太冗,故有是命。帝諭昂吉兒曰:「宰相明天道、察地理、盡人事,能兼此三者,乃為稱職。爾縱有功,宰相非可覬者。回回人中阿合馬才任宰相,阿里年少亦精敏,南人如呂文煥、範文虎率眾來歸,或可以相位處之。」又顧謂左右曰:「汝可諭姚樞等,江南官吏太冗,此卿輩所知,而皆未嘗言,昂吉兒乃為朕言之。」近侍劉鐵木兒因言:」阿里海牙屬吏張鼎,今亦參知政事。」詔即罷去。遂命平章政事哈伯等諭中書省、樞密院、御史臺:「翰林院及諸南儒今為宰相、宣慰,及各路達魯花赤佩虎符者,俱多謬濫,其議所以減汰之者。凡小大政事,順民之心所欲者行之,所不欲者罷之。」乙亥,敕省、院、臺諸司應聞奏事,必由起居注。丁丑,太廟殿柱朽腐,命太常少卿伯麻思告於太室,乃易之。戊寅,全州西延溪洞徭蠻二十所內附。己卯,發蒙古軍千人從江東宣慰使張弘範由海道討宋餘眾。參知政事蒙古帶請頒詔招宋廣王昺及張世傑等,不從。庚辰,處州張三八、章焱、季文龍等為亂,行省遣宣慰使謁只里率兵討之。辛巳,達實都收括中興等路闌遺。安南國王陳光昞遣使奉表來貢。



 秋七月壬午朔,湖南制置張烈良、提刑劉應龍與周隆、賀十二起兵,行省調兵往討,獲周隆、賀十二,斬之。烈良等舉家及餘兵奔思州烏羅洞,為官軍所襲,二人皆戰死。甲申,賜親王愛牙赤所部建都戍軍貧乏者鈔千二百七十七錠。行御史臺增設監察御史四員。江南湖北道、嶺南廣西道、福建廣東道並增設提刑按察司。乙酉,改江南諸路總管府為散府者七、為州者一,散府為州者二。丙戌,以江南事繁,行省官未有知書者,恐於吏治非便,分命崔斌至揚州行省,張守智至潭州行省。丁亥,詔虎符舊用畏吾字,今易以國字。癸巳,以塔海征夔軍旅之還戍者,及揚州、江西舟師,悉付水軍萬戶張榮實將之,守禦江口。丙申,以右丞塔出、左丞呂師夔、參知政事賈居貞行中書省事於贛州,福建、江西、廣東皆隸焉。丁酉,賜江西軍與張世傑力戰者三十人,各銀五十兩。以江西參知政事李恆為都元帥,將蒙古、漢軍征廣。命揚州行中書省分軍三千付李恆。復上都守城軍二千人為民。壬寅,改鑄高麗王王愖駙馬印。丙午,改開元宣撫司為宣慰司,太倉為御廩,資成庫為尚用監,皮貨局入總管府。定江南俸祿職田。戊申,濮州蝗。己酉,禁使人經行納憐驛。辛亥,改京兆府為安西府。詔江南、浙西等處毋非理徵科擾民。建漢祖天師正一祠於京城。以參知政事李恆為蒙古、漢軍都元帥,忙古帶為福建路宣慰使,張榮實、張鼎並為湖北道宣慰使,也的迷失為招討使。



 八月壬子朔,追毀宋故官所受告身。以嘉定、重慶、夔府既平,還侍衛親軍歸本司。遣禮部尚書柴椿等使安南國,詔切責之,仍俾其來朝。丁巳,沿海經略司、行左副都元帥劉深言:「福州安撫使王積翁既已降附,復通謀於張世傑。」積翁上言:「兵力單弱,若不暫從,恐為闔郡生靈之患。」詔原其罪。壬戌,有首高興匿宋金者,詔置勿問。兩淮運糧五萬石賑泉州軍民。乙丑,濟南總管張宏以代輸民賦,嘗貸阿里、阿答赤等銀五百五十錠,不能償,詔依例停征。辛未,復給漳州安撫使沈世隆家貲。世隆前守建寧府,有郭贊者受張世傑檄,誘世隆,世隆執贊斬之。蒙古帶以世隆擅殺,籍其家。帝曰:「世隆何罪,其還之。」仍授本路管民總管。中書省臣言:「近有旨追諸路管民官所授金虎符,其江南路臣宜仍所授。」從之。制封泉州神女號護國明著靈惠協正善慶顯濟天妃。甲戌,安西王相府言:「川蜀悉平,城邑山寨洞穴凡八十三,其渠州禮義城等處凡三十三所,宜以兵鎮守,餘悉徹毀。」從之。己卯,復立提刑按察司於畏吾兒分地。庚辰,以四川平,勞賞軍士鈔二萬一千三百三十九錠。辛巳,升洺磁為廣平府路。監察御史韓昺劾同知大都路總管府事舍裏甫丁毆部民至死,詔杖之,免其官,仍籍沒家貲十之二。詔行中書省唆都、蒲壽庚等曰:「諸蕃國列居東南島寨者,皆有慕義之心,可因蕃舶諸人宣布朕意,誠能來朝,朕將寵禮之。其往來互市,各從所欲。」詔諭軍前及行省以下官吏,撫治百姓,務農樂業,軍民官毋得占據民產,抑良為奴。以中書左丞董文炳簽書樞密院事,參知政事唆都、蒲壽庚並為中書左丞。



 九月壬午朔,敕以總管張子良所簽軍二千二百人為侍衛軍,俾張亨、陳瑾領之。癸未,省東西川行樞密院,其成都、潼川、重慶、利州四處皆設宣慰司。詔分揀諸路所括軍,驗事力乏絕者為民,其恃權豪避役者復為兵。所遣分揀官及本府州縣官,能核正無枉者,升爵一級。又減至元九年所括三萬軍半以為民,其商戶餘丁軍並除之。戊子,以征東元帥府治東京。庚寅,昭信達魯花赤李海剌孫言,願同張弘略取宋二王,調漢軍、水軍俾將之。以中書左丞、行江東道宣慰呂文煥為中書右丞。



 冬十月己未,享於太廟,常設牢醴外,益以羊、鹿、豕、蒲萄酒。庚申,車駕至自上都。辛酉,賑別十八里、日忽思等饑民鈔二千五百錠。分夔府漢軍二千、新軍一千付塔海將之。賜合答乞帶軍士馬價幣帛二千匹,其軍士力戰者賞齎有差。乙丑,正一祠成,詔張留孫居之。丁卯,弛山場樵採之禁。己巳,趣行省造海船付烏馬兒、張弘範,增兵四千俾將之。庚午,敕御史臺,凡軍官私役軍士者,視數多寡定其罪。詔:「河西、西京、南京、西川、北京等處宣慰司案牘,宜依江南近例,令按察司磨照。」移河南河北道提刑按察司治南京。御史臺臣言:「失里伯之弟阿剌與王權府等俘掠良民,失里伯縱弗問。及遣御史掾詰問,不伏。」詔執而鞫之。



 十一月庚辰朔,棗陽萬戶府言:「李均收撫大洪山寨為宋硃統制所害。」命賜銀千兩賙其家。丁亥,以辰、沅、靖、鎮遠等郡與蠻獠接壤,民不安業,命塔海、程鵬飛並為荊湖北道宣慰使,置司常德路,餘官屬留荊南府,供給糧食軍需。壬辰,江東道宣慰使囊加帶言:「江南既平,兵民宜各置官屬,蒙古軍宜分屯大河南北,以餘丁編立部伍,絕其虜掠之患。分揀官僚,本以革阿合馬濫設之弊。其將校立功者,例行沙汰,何以勸後?新附軍士,宜令行省賜其衣糧,無使闕乏。」帝嘉納之。征宋相馬廷鸞、章鑒赴闕。甲午,開酒禁。復阿合馬子忽辛、阿散先等官。始,忽辛等以崔斌論列而免,至是以張惠請,故復之。惠又請復其子麻速忽及其侄別都魯丁、苫思丁前職,帝疑惠,不從。敕已除官僚不之任者,除名為農。丁酉,召陳巖入覲。己亥,貸侍衛軍屯田者鈔二千錠市牛具。辛丑,建寧政和縣人黃華,集鹽夫,聯絡建寧、括蒼及纁民婦自稱許夫人為亂,詔調兵討之。丁未,行中書省自揚州移治杭州,立淮東宣慰司於揚州,以阿剌罕為宣慰使。詔諭沿海官司通日本國人市舶。以參知政事程鵬飛行荊湖北道宣慰使。閏月庚戌朔,羅氏鬼國主阿榨、西南蕃主韋昌盛並內附,詔阿榨、韋昌盛各為其地安撫使,佩虎符。辛亥,太白、熒惑、填星聚於房。甲寅,幸光祿寺。丙辰,詔禿魯赤同潭州行省官一員,察戍還病軍所過州縣不加顧恤者按之。甲子,發蒙古、漢軍都元帥張弘範攻漳州,得山寨百五十、戶百萬一。是日,諜報文天祥見屯潮陽港,亟遣先鋒張弘正、總管囊加帶率輕騎五百人,追及於五坡嶺麓中,大敗之,斬首七千餘,執文天祥及其將校四人赴都。



 十二月己卯,簽書西川行樞密院昝順招誘都掌蠻夷及其屬百一十人內附,以其長阿永為西南番蠻安撫使,得蘭紐為都掌蠻安撫使,賜虎符,餘授宣敕、金銀符有差。庚辰,思州安撫使田景賢、播州安撫使楊邦憲請歸宋舊借鎮遠、黃平二城,仍撤戍卒,不允。景賢等請降詔禁戍卒毋擾思、播之民,從之。鴨池等處招討使欽察所領南征新軍,不能自贍者千人,命屯田於京兆。乙酉,伯顏以渡江收撫沙陽、新城、陽羅堡、閩、浙等郡獲功軍士及降臣姓名來上,詔授虎符者入覲,千戶以下並從行省授官。丙戌,揚州行省上將校軍功凡百三十四人,授官有差。丙申,從播州安撫楊邦憲請,以鼎山仍隸播州。庚子,敕長春宮修金籙大醮七晝夜。丙午,禁玉泉山樵採漁弋。戊申,以敘州等處禿老蠻殺使臣撒里蠻,命發兵討之。封伯夷為昭義清惠公,叔齊為崇讓仁惠公。以十六年歷日賜高麗。海州贛榆縣雹傷稼,免今年田租。南寧、吉陽、萬安三郡內附。開成路置屯田總管府,廣安縣隸之。臨淄、臨朐、清河復為縣。導肥河入于酅,淤陂盡為良田。會諸王於大都,以平宋所俘寶玉器幣分賜之。賜諸王等金、銀、幣、帛如歲例。是歲,西京奉聖州及彰德等處水旱民饑,賑米八萬八百九十石、粟三萬六千四十石、鈔二萬四千八百八十錠有奇。斷死罪五十二人。



 十六年春正月己酉朔,高麗國王王愖遣其簽議中贊金方慶來賀,兼奉歲幣。壬子,罷五翼探馬赤重役軍。癸丑,汪良臣言:「西川軍官父死子繼,勤勞四十年,乞顯加爵秩。」詔從其請。詔以海南、瓊崖、儋、萬諸郡俱平,令阿里海牙入覲。瀘州降臣趙金、吳大才、袁禹繩等從征重慶,其家屬為叛者所殺,詔賜鈔有差,仍以叛者妻孥付金等。敕高麗國置大灰艾州、東京、柳石、孛落四驛。甲寅,無籍軍侵掠平民,而諸王只必帖木兒所部為暴尤甚,命捕為首者置之法。敕移贛州行省還隆興。高麗國來獻方物。辛酉,合州安撫使王立以城降。先是,立遣間使降安西王相李德輝,東川行院與德輝爭功,德輝單舸至城下,呼立出降,川蜀以平。東川行院遂言,立久抗王師,嘗指斥憲宗,宜殺之。樞密院以其事聞,而降臣李諒亦訟立前殺其妻子,有其財物,遂詔殺立,籍其家貲償諒。既而安西王具立降附本末來上,且言東川院臣憤李德輝受降之故,誣奏誅立。樞密院臣亦以前奏為非。帝怒曰:「卿視人命若戲耶!前遣使計殺立久矣,今追悔何及。卿等妄殺人,其歸待罪。」斥出之。會安西王使再至,言未殺立。即召立入覲,命為潼川路安撫使,知合州事。壬戌,分川蜀為四道:以成都等路為四川西道,廣元等路為四川北道,重慶等路為四川南道,順慶等路為四川東道,並立宣慰司。賞重慶等處從征蒙古、漢軍鈔三萬九千九百五十一錠。改播州鼎山縣為播川縣。丁卯,賜參知政事昝順田民百八十戶於江津縣。戊辰,立河西屯田,給畊具,遣官領之。甲戌,張弘範將兵追宋二王至崖山寨,張世傑來拒戰,敗之,世傑遁去,廣王昺偕其官屬俱赴海死,獲其金寶以獻。丙子,詔諭又巴、散毛等四洞番蠻酋長使降。以中書左丞別乞裏迷失同知樞密院事。禁中書省文冊奏檢用畏吾字書。賜異樣等局官吏工匠銀二千兩。賜皇子奧魯赤及諸王拜答罕下軍士與思州田師賢所部軍衣服及鈔有差。



 二月戊寅朔,祭先農於籍田。壬午,升溧州為路。遣使訪求通皇極數番陽祝泌子孫,其甥傅立持泌書來上,撥民萬戶隸明裏淘金。以江南漕運舊米賑軍民之饑者。癸未,增置五衛指揮司。詔遣塔黑麻合兒、撒兒答帶括中興戶。太史令王恂等言:「建司天臺於大都,儀象圭表皆銅為之,宜增銅表高至四十尺,則景長而真。又請上都、洛陽等五處分置儀表,各選監候官。」從之。甲申,平章阿里伯乞行中書省檢核行御史臺文案,且請行臺呈行省,比御史呈中書省例,從之。以征日本,敕揚州、湖南、贛州、泉州四省造戰船六百艘。移紹興宣慰司於處州。己丑,調潭州行省軍五千戍沿海州郡。庚寅,張弘範以降臣陳懿兄弟破賊有功,且出戰船百艘從征宋二王,請授懿招討使兼潮州路軍民總管,及其弟忠、義、勇三人為管軍總管,千夫長塔剌海獲文天祥有功,請授總管軍千戶,佩符,並從之。壬辰,詔諭宗師張留孫悉主淮東、淮西、荊襄等處道教。乙未,玉速帖木兒言:「行臺文卷令行省檢核,於事不便。」詔改之,其運司文卷聽御史臺檢核。饒州路達魯花赤玉古倫擅用羨餘糧四千四百石,杖之,仍沒其家。詔湖南行省於戍軍還途,每四五十里立安樂堂,疾者醫之,饑者廩之,死者槁葬之,官給其需。遣官核實益都、淄萊、濟南逃亡荒地之為行營牧地者。禁諸奧魯及漢人持弓矢,其出征之所持兵仗,即輸之官庫。壬寅,賜太史院銀一千七十八兩。癸卯,發嘉定新附軍千人屯田脫里北之地。甲辰,升大都兵馬都指揮使司秩四品。詔大都、河間、山東管鹽運司並兼管酒、醋、商稅等課程。中書省臣請以真定路達魯花赤蒙古帶為保定路達魯花赤,帝曰:「此正人也,朕將別以大事付之。」賞汪良臣所部蒙古、漢軍收附四川功鈔五萬錠。命嘉定以西新鄭州郡及田、楊二家諸貴官子,俱充質子入侍。車駕幸上都。乙巳,命同知太史院事郭守敬訪求精天文歷數者。西蜀四川道立提刑按察司。丙午,遣使代祀岳瀆后土。詔河南、西京、北京等路課程,令各道宣慰司領之。賞西川新附軍鈔三千八百五十錠。以斡端境內蒙古軍耗乏,並漢軍、新附軍等,賜馬牛羊及馬驢價鈔、衣服、弓矢、鞍勒各有差。



 三月戊申朔,詔禁歸德、亳、壽、臨淮等處畋獵。庚戌,敕郭守敬繇上都、大都,歷河南府抵南海,測驗晷景。壬子,囊加帶括兩淮造回回砲新附軍匠六百,及蒙古、回回、漢人、新附人能造砲者,俱至京師。庚申,給千戶馬乃部下拔突軍及土渾川軍屯田牛具。丙寅,敕中書省,凡掾史文移稽緩一日二日者杖,三日者死。甲戌,潭州行省遣兩淮招討司經歷劉繼昌招下西南諸番,以龍方零等為小龍蕃等處安撫使,仍以兵三千戍之。中書省下太常寺講究州郡社稷制度,禮官折衷前代,參酌《儀禮》,定擬祭祀儀式及壇壝祭器制度,圖寫成書,名曰《至元州縣社稷通禮》,上之。以保定路旱,減是歲租三千一百二十石。



 夏四月己卯,立江西榷茶運司及諸路轉運鹽使司、宣課提舉司。癸巳,以給事中兼起居注,掌隨朝諸司奏聞事。戊戌,以池州路達魯花赤阿塔赤戰功升招討使,兼本軍萬戶。癸卯,填星犯鍵閉。乙巳,汪良臣言:「昔昝順兵犯成都,掠其民以歸。今嘉定既降,宜還其民成都。」制曰「可。」敕以上都軍四千衛都城,凡他所來戍者皆遣歸。從唆都請,令泉州僧依宋例輸稅,以給軍餉。詔諭揚州行中書省,選南軍精銳者二萬人充侍衛軍,並發其家赴京師,仍給行費鈔萬六千錠。大都等十六路蝗。



 五月己酉,中書省請復授宣慰司官虎符,不允。又請各路設提舉、同提舉、副提舉各一員,專領課程,從之。辛亥,蒲壽庚請下詔招海外諸蕃,不允。詔諭漳、泉、汀、邵武等處暨八十四畬官吏軍民,若能舉眾來降,官吏例加遷賞,軍民按堵如故。以泉州經張世傑兵,減今年租賦之半。丙辰,以五臺僧多匿逃奴及逋賦之民,敕西京宣慰司、按察司搜索之。命畏吾界內計畝輸稅。以各道按察司地廣事繁,並勸農官入按察司,增副使、僉事各壹員,兼職勸農水利事。甲子,御史臺臣言:「先是省臣阿里伯言,有罪者與臺臣相威同問,有旨從之。臣等謂行省斷罪以意出入,行臺何由舉正。宜從行省問訖,然後體察為宜。」制曰:「可。」高興侵用宋二王金三萬一千一百兩有奇、銀二十五萬六百兩,詔遣使追理。詔漣、海等州募民屯田,置總管府及提舉司領之。乙丑,敕江陵等路拔突戶一萬,凡千戶置達魯花赤一員,直隸省部。丙寅,敕江南僧司文移,毋輒入遞。臨洮、鞏昌、通安等十驛,非有海青符,不聽乘傳。丁卯,改雲南寶山、莨渠二縣為州。己巳,詔沿路驛店民家,凡往來使臣不當乘傳者,毋給人畜飲食芻料。完都、河南七驛民貧乏,給其馬牛羊價鈔千八百錠。庚午,賜乃蠻帶戰功及攻圍重慶將士及宣慰使劉繼昌等鈔、衣服各有差。壬申,以呂虎來歸,授順慶府總管,佩虎符,仍賜鈔五十錠。徙丁子峪所駐侍衛軍萬人,屯田昌平。癸酉,兀里養合帶言:「賦北京、西京車牛俱至,可運軍糧。」帝曰:「民之艱苦汝等不問,但知役民。使今年盡取之,來歲禾稼何由得種。其止之。」甲戌,給要束合所領工匠牛二千,就令運米二千石供軍。詔諭脫兒赤等管甘州路宣課,諸人毋或沮擾。潭州行省上言:「瓊州宣慰馬旺已招降海外四州,尋有土寇黃威遠等四人為亂,今已擒獲。」詔置之極刑。丙子,進封桑乾河洪濟公為顯應洪濟公。命宗師張留孫即行宮作醮事,奏赤章於天,凡五晝夜。賜皇子奧魯赤、撥裡答等及千戶伯牙兀帶所部軍及和州站戶羊馬鈔各有差。



 六月丁丑朔,阿合馬言:「常州路達魯花赤馬恕告簽浙西按察司事高源不法四十事,源亦劾恕。」事聞,詔令廷辯。詔發新附軍五百人、蒙古軍百人、漢軍四百人戍碉門、魚通、黎、雅。詔諭王相府及四川行中書省,四道宣慰司撫治播川、務川西南諸蠻夷,官吏軍民各從其俗,無失常業。壬午,以浙東宣慰使陳祐沒王事,命其子夔為管軍總管,佩虎符。甲申,宋張世傑所部將校百五十八人,詣瓊、雷等州來降。敕造戰船征日本,以高麗材用所出,即其地制之,令高麗王議其便以聞。乙酉,榆林、洪贊、刁窩,每驛益馬百五十、車二百,牛如車數給之。丙戌,左右衛屯田蝗蝻生。庚寅,升濟寧府為路。壬辰,以參知政事、行河南等路宣慰使忽辛為中書左丞,行中書省事。癸巳,以新附軍二萬分隸六衛屯田。徹里帖木兒言其部軍多為盜劫掠貲財,有司不即理斷,乞遣官詰治,詔兀魯帶往治之。以不花行西川樞密院事,總兵入川,平宋諸城之未下者。仍令東川行樞密院調兵守釣魚山寨。西川既平,復立屯田,其軍官第功升擢,凡授宣敕、金銀符者百六十一人。詔以高州、筠連州騰川縣新附戶於漵州等處治道立驛。雲南都元帥愛魯、納速剌丁招降西南諸國。愛魯將兵分定亦乞不薛,納速剌丁將大理軍抵金齒、蒲驃、曲蠟、緬國界內,招忙木、巨木禿等寨三百,籍戶十一萬二百。詔定賦租,立站遞,設衛送軍。軍還,獻馴象十二。戊戌,改宣德府龍門鎮復為縣。庚子,拘括河西、西番闌遺戶。辛丑,以通州水路淺,舟運甚難,命樞密院發軍五千,仍令食祿諸官雇役千人開浚,以五十日訖工。癸卯,以臨洮、鞏昌、通安等十驛歲饑,供役繁重,有質賣子女以供役者,命選官撫治之。甲辰,以襄陽屯田戶四百代軍當驛役。賜征北諸郡蒙古軍闊闊八都等力戰有功者銀五十兩,戰歿者家給銀百兩,從行伍者鈔一錠,其餘衣物有差。禁伯顏察兒諸峪寨捕獵。詔免四川差稅。以參知政事、行中書省事別都魯丁為河南等路宣慰使。以阿合馬子忽辛為潭州行省左丞,忽失海牙等並復舊職。占城、馬八兒諸國遣使以珍物及象犀各一來獻。賜諸王所部銀鈔、衣服、幣帛、鞍勒、弓矢及羊馬價鈔等各有差。五臺山作佛事。



 秋七月戊申,寧國路新附軍百戶詹福謀叛,福論死,授告者何士青總把、銀符,仍賜鈔十錠。罷西川行省。庚戌,禁脫脫和孫搜取乘傳者私物。乙卯,應昌府依例設官。置東宮侍衛軍。定江南上、中路置達魯花赤二員,下路一員。敕發西川蒙古軍七千、新附軍三千,付皇子安西王。丁巳,交趾國遣使來貢馴象。己未,以朵哥麻思地之算木多城為鎮西府。敕以蒙古軍二千、益都軍二千、諸路軍一千、新附軍五千,合萬人,令李庭將之。壬戌,賞甕吉剌所部力戰軍人銀五十兩,死事者人百兩,給其家。阿里海牙入覲,獻金三千五百八十兩、銀五萬三千一百兩。罷潭州行省造征日本及交趾戰船。丙寅,填星犯鍵閉。癸酉,西南八番、羅氏等國來附,洞寨凡千六百二十有六,戶凡十萬一千一百六十有八。詔遣牙納術、崔彧至江南訪求藝術之人。以中書左丞、行四川行中書省事汪良臣為安西王相。賜諸王納里忽所部有功將校銀鈔、衣裝、幣帛、羊馬有差。以趙州等處水旱,減今年租三千一百八十一石。命散都修佛事十有五日。



 八月丁丑,車駕至自上都。庚辰,太陰犯房距星。戊子,範文虎言:「臣奉詔征討日本,比遣周福、欒忠與日本僧齎詔往諭其國,期以來年四月還報,待其從否,始宜進兵。」又請簡閱舊戰船以充用。皆從之。海賊賀文達率眾來歸文虎,文虎以所得銀三千兩來獻。有旨釋其前罪,官其徒四十八人,就以銀賜文虎。己丑,宋降臣王虎臣陳便宜十七事,令張易等議,可者行之。庚寅,敕沅州路蒙古軍總管乞答合征取桐木籠、犵狫、伯洞諸蠻未附者。調江南新附軍五千駐太原,五千駐大名,五千駐衛州。以每歲聖誕節及元辰日,禮儀費用皆斂之民,詔天下罷之。丁酉,以江南所獲玉爵及坫凡四十九事,納於太廟。己亥,海賊金通精死,獲其從子溫,有司欲論如法,帝曰:「通精已死,溫何預焉?」特赦其罪。庚子,歲星犯軒轅大星。甲辰,詔漢軍出征逃者罪死,且沒其家。置大護國仁王寺總管府,以散扎兒為達魯花赤,李光祖為總管。賜範文虎僚屬二十一人金紋綾及西錦衣。賞征重慶將校幣帛有差。賜諸王阿只吉糧五千石、馬六百匹、羊萬口。



 九月乙巳朔,範文虎薦可為守令者三十人。詔:「今後所薦,朕自擇之。凡有官守不勤於職者,勿問漢人、回回皆論誅之,且沒其家。」女直、水達達軍不出征者,令隸民籍輸賦。己酉,罷金州守船軍千人,量留監守,餘皆遣還。庚戌,詔行中書省左丞忽辛兼領杭州等路諸色人匠,以杭州稅課所入,歲造繒段十萬以進。杭、蘇、嘉興三路辦課官吏,額外多取分例,今後月給食錢,或數外多取者罪之。阿合馬言:「王相府官趙炳云,陜西課程歲辦萬九千錠,所司若果盡心措辦,可得四萬錠。」即命炳總之。同知揚州總管府事董仲威坐贓罪,行臺方按其事,仲威反誣行臺官以他事。詔免仲威官,仍沒其產十之二。戊午,王相府言:「四川宣慰司有籍無軍虛受賞者一萬七千三百八人。」命詰治之。議罷漢人之為達魯花赤者。御史臺臣言:「江南三路管課官,於分例外支用鈔一千九百錠。」命盡征之。詔遣使招諭西南諸蠻部族酋長,能率所部歸附者,官不失職,民不失業。乙丑,以忽必來、別速臺為都元帥,將蒙古軍二千人、河西軍一千人,戍斡端城。己巳,樞密院臣言:「有唐兀帶者冒禁引軍千餘人,於辰溪、沅州等處劫掠新附人千餘口及牛馬、金銀、幣帛等,而麻陽縣達魯花赤武伯不花為之鄉導。」敕斬唐兀帶、武伯不花,餘減死論,以所掠者還其民。給河西行省鈔萬錠,以備支用。



 冬十月己卯,享於太廟。辛巳,敘州、夔府至江陵界立水驛。乙酉,帝御香閣。命大樂署令完顏椿等肄文武樂。戊子,張融訴西京軍戶和買和雇,有司匿所給價鈔計萬八千餘錠;官吏坐罪,以融為侍衛軍總把。千戶脫略、總把忽帶擅引軍入婺州永康縣界,殺掠吏民,事覺,自陳扈從先帝出征有功,乞貸死。敕沒入其家貲之半,杖遣之。辛卯,賑和州貧民鈔。乙未,納碧玉爵於太廟。丙申,太陰犯太微西垣上將。辛丑,以月直元辰,命五祖真人李居壽作醮事,奏赤章,凡五晝夜。畢事,居壽請間言:「皇太子春秋鼎盛,宜預國政。」帝喜曰:「尋將及之。」明日,下詔皇太子燕王參決朝政,凡中書省、樞密院、御史臺及百司之事,皆先啟後聞。甲辰,賜高麗國王至元十七年歷日。



 十一月戊申,敕諸路所捕盜,初犯贓多者死,再犯贓少者從輕罪論。阿合馬言:「有盜以舊鈔易官庫新鈔百四十錠者,議者謂罪不應死,且盜者之父執役臣家,不論如法,寧不自畏。」詔處死。壬子,遣禮部尚書柴椿偕安南國使村中贊齎詔往諭安南國世子陳日烜,責其來朝。癸丑,太陰犯熒惑。乙卯,罷太原、平陽、西京、延安路新簽軍還籍。罷招討使劉萬奴所管無籍軍願從大軍征討者。趙炳言陜西運司郭同知、王相府郎中令郭叔云盜用官錢,敕尚書禿速忽、侍御史郭祐檢核之。戊辰,命湖北道宣慰使劉深教練鄂州、漢陽新附水軍。詔諭四川宣慰司括軍民戶數。己巳,以梧州妖民吳法受扇惑藤州、德慶府瀧水徭蠻為亂,獲其父,誅之。並教坊司入拱衛司。



 十二月戊寅,發粟鈔賑鹽司灶戶之貧者。括甘州戶。庚辰,安南國貢藥材。甲申,祀太陽。丙申,敕樞密、翰林院官,就中書省與唆都議招收海外諸番事。丁酉,八里灰貢海青。回回等所過供食,羊非自殺者不食,百姓苦之。帝曰:「彼吾奴也,飲食敢不隨我朝乎?」詔禁之。詔諭海內海外諸番國主。賜右丞張惠銀五千四百兩。敕自明年正月朔日,建醮於長春宮,凡七日,歲以為例。命李居壽告祭新歲。詔諭占城國主,使親自來朝。唆都所遣闍婆國使臣治中趙玉還。改單州、兗州隸濟寧路;復置萬泉縣,隸河中府;改垣曲縣隸絳州;降歸州路為州;升沔陽、安陸各為府;改京兆為安西路;改惠州、建寧、梧州、柳州、象州、邕州、慶遠、賓州、橫州、容州、潯州並為路。建聖壽萬安寺於京城。帝師亦憐吉卒。敕諸國教師禪師百有八人,即大都萬安寺設齋圓戒,賜衣。是歲,斷死罪百三十二人。保定等二十餘路水旱風雹害稼。



\end{pinyinscope}