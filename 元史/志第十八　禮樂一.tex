\article{志第十八 禮樂一}

\begin{pinyinscope}

 《傳》曰:「禮者,天地之序也;樂者,天地之和也。」致禮以治躬,外貌斯須不莊不敬,則慢易之心入之矣;致樂以治心,中心斯須不和不樂,則鄙詐之心入之矣。古之禮樂,壹本於人君之身心,故其為用,足以植綱常而厚風俗;後世之禮樂,既無其本,唯屬執事者從事其間,故僅足以美聲文而侈觀聽耳。此治之所以不如古也。



 前聖之制,至周大備。周公相成王,制禮作樂,而教化大行,邈乎不可及矣。秦廢先代典禮,漢因秦制,起朝儀,作宗廟樂。魏、晉而後,五胡雲擾,秦、漢之制亦復不存矣。唐初襲用隋禮,太常多肄者,教坊俗樂而已。至宋,承五季之衰,因唐禮,作《太常因革禮》,而所制《大晟樂》,號為古雅。及乎靖康之變,禮文樂器,掃蕩無遺矣。元之有國,肇興朔漠,朝會燕饗之禮,多從本俗。太祖元年,大會諸侯王於阿難河,即皇帝位,始建九斿白旗。世祖至元八年,命劉秉忠、許衡始制朝儀。自是,皇帝即位、元正、天壽節,及諸王、外國來朝,冊立皇后、皇太子,群臣上尊號,進太皇太后、皇太后冊寶,暨郊廟禮成、群臣朝賀,皆如朝會之儀;而大饗宗親、錫宴大臣,猶用本俗之禮為多。



 若其為樂,則自太祖徵用舊樂於西夏,太宗征金太常遺樂於燕京,及憲宗始用登歌樂,祀天於日月山,而世祖命宋周臣典領樂工,又用登歌樂享祖宗於中書省。既又命王鏞作《大成樂》,詔括民間所藏金之樂器。至元三年,初用宮縣、登歌、文武二舞於太廟,烈祖至憲宗八室,皆有樂章。三十年,又撰社稷樂章。成宗大德間,制郊廟曲舞,復撰宣聖廟樂章。仁宗皇慶初,命太常補撥樂工,而樂制日備。大抵其於祭祀,率用雅樂,朝會饗燕,則用燕樂,蓋雅俗兼用者也。



 元之禮樂,揆之於古,固有可議。然自朝儀既起,規模嚴廣,而人知九重大君之尊,重其樂聲雄偉而宏大,又足以見一代興王之象,其在當時,亦云盛矣。今取其可書者著於篇,作《禮樂志》。



 制朝儀始末



 世祖至元八年秋八月己未,初起朝儀。先是,至元六年春正月甲寅,太保劉秉忠、大司農孛羅奉旨,命趙秉溫、史杠訪前代知禮儀者肄習朝儀。既而秉忠奏曰:「二人習之,雖知之,莫能行也。」得旨,許用十人。遂徵儒生周鐸、劉允中、尚文、岳忱、關思義、侯祐賢、蕭琬、徐汝嘉,從亡金故老烏古倫居貞、完顏復昭、完顏從愈、葛從亮、於伯儀及國子祭酒許衡、太常卿徐世隆,稽諸古典,參以時宜,沿情定制,而肄習之,百日而畢。秉忠復奏曰:「無樂以相須,則禮不備。」奉旨,搜訪舊教坊樂工,得杖鼓色楊皓、笛色曹楫、前行色劉進、教師鄭忠,依律運譜,被諸樂歌,六月而成,音聲克諧,陳於萬壽山便殿,帝聽而善之。秉忠及翰林太常奏曰:「今朝儀既定,請備執禮員。」有旨,命丞相安童、大司農孛羅擇蒙古宿衛士可習容止者二百餘人,肄之期月。七年春二月,奏以丙子觀禮。前期一日,布綿蕝金帳殿前,帝及皇后臨觀於露階,禮文樂節,悉無遺失。冬十有一月戊寅,秉忠等奏請建官典朝儀,帝命與尚書省論定以聞。



 八年春二月,立侍儀司,以忽都於思、也先乃為左右侍儀,奉御趙秉溫為禮部侍郎兼侍儀司事,周鐸、劉允中為左右侍儀使,尚文、岳忱為左右直侍儀事,關思義、侯祐賢為左右侍儀副使,蕭琬、徐汝嘉為僉左右侍儀事,烏古倫居貞為承奉班都知,完顏復昭為引進副使,葛從亮為侍儀署令,於伯儀為尚衣局大使。夏四月,侍儀司奏請制內外仗,如歷代故事,從之。秋七月,內外仗成。遇八月帝生日,號曰天壽聖節,用朝儀自此始。



 元正受朝儀



 前期三日,習儀於聖壽萬安寺。或大興教寺。前二日,陳設於殿庭。至期大昕,侍儀使引導從護尉,各服其服,入至寢殿前,捧牙牌跪報外辦。內侍入奏,出傳制曰「可」,侍儀使俯伏興。皇帝出閣升輦,鳴鞭三。侍儀使拜通事舍人,分左右,引擎執護尉、劈正斧中行,導至大明殿外。劈正斧直正門北向立,導從倒卷序立,惟扇置於錡。侍儀使導駕時,引進使同內侍官,引宮人擎執導從,入至皇后宮庭,捧牙牌跪報外辦。內侍入啟,出傳旨曰「可」,引進使俯伏興。皇后出閣升輦,引進使引導從導至殿東門外,引進使分退押直至堊塗之次,引導從倒卷出。俟兩宮升御榻,鳴鞭三,劈正斧退立於露階東。司晨報時雞唱畢,尚引引殿前班,皆公服,分左右入日精、月華門,就起居位,相向立。通班舍人唱曰「左右衛上將軍兼殿前都點檢臣某以下起居」,尚引唱曰「鞠躬」,曰「平身」,引至丹墀拜位,知班報班齊。宣贊唱曰「拜」,通贊贊曰「鞠躬」,曰「拜」,曰「興」,曰「拜」,曰「興」,曰「都點檢稍前」。宣贊報曰「聖躬萬福」,通贊贊曰「復位」,曰「拜」,曰「興」,曰「拜」,曰「興」,曰「平身」,曰「搢笏」,曰「鞠躬」,曰「三舞蹈」,曰「跪左膝,三叩頭」,曰「山呼」,曰「山呼」,曰「再山呼」,凡傳「山呼」,控鶴呼噪應和曰「萬歲」,傳「再山呼」,應曰「萬萬歲」。後仿此。曰「出笏」,曰「就拜」,曰「興」,曰「拜」,曰「興」,曰「拜」,曰「興」,曰「平立」,宣贊唱曰「各恭事」。兩班點檢、宣徽將軍分左右升殿,宿直以下分立殿前,尚廄分立仗南,管旗分立大明門南楹。



 俟後妃、諸王、駙馬以次賀獻禮畢,典引引丞相以下,皆公服,入日精、月華門,就起居位。通班唱曰「文武百僚、開府儀同三司、錄軍國重事、監修國史、右丞相具官無常。臣某以下起居」,典引贊曰「鞠躬」,曰「平身」,引至丹墀拜位,知班報班齊。宣贊唱曰「拜」,通贊贊曰「鞠躬」,曰「拜」,曰「興」,曰「拜」,曰「興」,曰「平身」,曰「搢笏」,曰「鞠躬」,曰「三舞蹈」,曰「跪左膝,三叩頭」,曰「山呼」,曰「山呼」,曰「再山呼」,曰「出笏」,曰「就拜」,曰「興」,曰「拜」,曰「興」,曰「拜」,曰「興」,曰「平身」。侍儀使詣丞相前請進酒,雙引升殿。前行樂工分左右,引登歌者及舞童舞女,以次升殿門外露階上。登歌之曲各有名,音中本月之律。先期,儀鳳司運譜,翰林院撰辭肄之。丞相至宇下褥位立,侍儀使分左右北向立。俟前行色曲將半,舞旋列定,通贊唱曰「分班」,樂作。侍儀使引丞相由南東門入,宣徽使奉隨至御榻前。丞相跪,宣徽使立於東南,曲終。丞相祝贊曰:「溥天率土,祈天地之洪福,同上皇帝、皇后億萬歲壽。」宣徽使答曰:「如所祝。」丞相俯伏興,退詣進酒位。尚醖官以觴授丞相,丞相搢笏捧觴,北面立,宣徽使復位。前行色降,舞旋至露階上。教坊奏樂,樂舞至第四拍,丞相進酒,皇帝奉觴。宣贊唱曰「殿上下侍立臣僚皆再拜」,通贊贊曰「鞠躬」,曰「拜」,曰「興」,曰「拜」,曰「興」,曰「平身」。丞相三進酒畢,以觴授尚醖官,出笏,侍儀使雙引自南東門出,復位,樂止。至元七年進酒儀:班首至殿前褥位立,前行進曲,尚醖官執空杯,自正門出,授班首。班首搢笏執空杯,由正門入,至御榻前跪。俟曲終,以杯授尚醖官,出笏祝贊。宣徽使曰「諾」,班首俯伏興。班首、宣徽使由南東門出,各復位。班首以下舞蹈山呼五拜,百官分班,教坊奏樂,尚醖官進酒,殿上下侍立臣僚皆再拜。三進酒畢,班首降至丹墀。至元十八年十二月二十八日改今儀。



 通贊贊曰「合班」。禮部官押進奏表章、禮物二案至橫階下,宣禮物舍人進讀禮物目,至第二重階。俟進讀表章官等,翰林國史院屬官一人。至宇下齊跪。宣表目舍人先讀中外百司表目,翰林院官讀中書省表畢,皆俯伏興,退,降第一重階下立。俟進讀禮物舍人升階,至宇下,跪讀禮物目畢,俯伏興,退。同降至橫階,隨表章西行,至右樓下,侍儀仍領之,禮物東行至左樓下,太府受之。宣贊唱曰「拜」,通贊贊曰「鞠躬」,曰「拜」,曰「興」,曰「平身」,曰「搢笏」,曰「鞠躬」,曰「三舞蹈」,曰「跪左膝,三叩頭」,曰「山呼」,曰「山呼」,曰「再山呼」,曰「出笏」,曰「就拜」,曰「興」,曰「拜」,曰「興」,曰「拜」,曰「興」,曰「平立」。僧、道、耆老、外國籓客,以次而賀。



 禮畢,大會諸王宗親、駙馬、大臣,宴饗殿上,侍儀使引丞相等升殿侍宴。凡大宴,馬不過一,羊雖多,必以獸人所獻之鮮及脯鱐,折其數之半。預宴之服,衣服同制,謂之質孫。宴饗樂節,見宴樂篇。四品以上,賜酒殿上。典引引五品以下,賜酒於日精、月華二門之下。宴畢,鳴鞭三。侍儀使導駕,引進使導後,還寢殿,如來儀。



 天壽聖節受朝儀如元正儀



 郊廟禮成受賀儀如元正儀



 皇帝即位受朝儀



 前期三日,習儀於萬安寺;前二日,陳設於殿庭;前一日,設宣詔位於闕前。至期大昕,侍儀使引導從護尉,各服其服,至皇太子寢閣前,捧牙牌跪報外辦。內侍傳旨曰「可」,侍儀使俯伏興。皇太子出閣,侍儀使前導,由崇天門入,升大明殿。引進使引導從至皇太子妃閣前,跪報外辦。內侍出傳旨曰「可」,引進使俯伏興,前導由鳳儀門入。俟諸王以國禮扶皇帝登寶位畢,鳴鞭三。尚引引點檢以下,皆公服,入就起居位。起居贊拜,如元正朝儀。



 兩班點檢、宣徽將軍、宿直、尚廄、管旗,各恭事。俟後妃、諸王、駙馬以次賀獻禮畢,參議中書省事四人,以篚奉詔書,由殿左門入,至御榻前。參議中書省事跪奏詔文,俯伏興,以詔授典瑞使押寶畢,置於篚,對舉由正門出,樂作,至闕前,以詔置於案,文武百僚各公服就位北向立。侍儀使稱有制,宣贊唱曰「拜」,通贊贊曰「鞠躬」,曰「拜」,曰「興」,曰「拜」,曰「興」,曰「平身」,曰「班首稍前」,典引引班首至香案前。通贊贊曰「跪」,曰「在位官皆跪」,司香贊曰「搢笏」,通贊贊曰「上香」,曰「上香」,曰「三上香」,曰「出笏」,曰「就拜」,曰「興」,曰「復位」,宣贊唱曰「拜」,通贊贊曰「鞠躬」,曰「拜」,曰「興」,曰「拜」,曰「興」,曰「平身」。侍儀使以詔授左司郎中,郎中跪受,同譯史稍西,升木榻,東向宣讀。通贊贊曰「在位官皆跪」。讀詔,先以國語宣讀,隨以漢語譯之。讀畢,降榻,以詔授侍儀使,侍儀使置於案。通贊贊曰「就拜」,曰「興」,曰「拜」,曰「興」,曰「拜」,曰「興」,曰「搢笏」,曰「鞠躬」,曰「三舞蹈」,曰「跪左膝,三叩頭」,曰「山呼」,曰「山呼」,曰「再山呼」,曰「出笏」,曰「就拜」,曰「興」,曰「拜」,曰「興」,曰「拜」,曰「興」,曰「平立」。典引引丞相以下皆公服入起居位。起居拜舞,祝頌,進酒,獻表,賜宴,並同元正受朝儀。宴畢,鳴鞭三。侍儀使導駕,引進使導後,入寢殿,如來儀。次日,以詔頒行。



 群臣上皇帝尊號禮成受朝賀儀



 前期二日,儀鸞司設大次於大明門外,又設進冊案於殿內御座前之西,進寶案於其東,設受冊案於御座上之西,受寶案於其東。侍儀司設冊案於香案南,寶案又於其南。禮儀使位於前,冊使、冊副位於廷中,北面。引冊、奉冊、舉冊、讀冊、捧冊官,位於右,引寶、奉寶、舉寶、讀寶、捧寶官位於左,以北為上。百官自金玉府迎冊寶,奉安中書省,如常儀。



 前期一日,右丞相率公卿朝服,儀衛音樂,導冊寶二案出自中書,至闕前,控鶴奠案,方輿中道。冊使等奉隨入大次內,方輿奠案。侍儀使引冊使以下,由左門以出,百官趨退。



 至其大昕,右丞相以下百官,各公服集闕廷,儀仗護尉就位。侍儀使、禮儀使引導從導皇帝升大明殿,引進使引導從導皇后升殿。尚引引殿前班入起居位,起居山呼拜舞畢,宣贊唱曰「各恭事」。皇太子、諸王、后妃、公主以次升殿,鳴鞭三。侍儀使、引冊、引寶導冊寶由正門入,樂作。奉冊使、右丞相率冊官由右門入,奉寶使、御史大夫率寶官由左門入,至殿下,置冊案於香案南,寶案又奠於其南,樂止。侍儀使引冊使以下就起居位,典引引群臣入就位。通班舍人唱曰「文武百僚具官臣某以下起居」,典引贊曰「鞠躬」,曰「平身」,引至丹墀拜位。宣贊唱曰「拜」,通贊贊拜、舞蹈、山呼,如常儀。



 畢,承奉班都知唱曰「奉冊使以下進上冊寶」,侍儀司引冊使以下進就位,樂作。掌儀贊曰「奉冊寶官稍前,搢笏,捧冊寶」,侍儀使前導,由中道升正階,立於下。俟奉冊使諸冊官由右階隮,奉寶使諸寶官由左階隮畢,俱由左門入,奉冊寶至御榻褥位前,冊西寶東,樂止。掌儀贊曰「捧冊寶官稍前,以冊寶跪置於案」,曰「出笏」,曰「就拜」,曰「興」,曰「平身」,曰「復位」,曰「奉冊使以下皆跪」,曰「舉冊官興,俱至案前跪」,曰「搢笏,取冊於匣,置於盤,對舉」,曰「讀冊官興,俱至案前跪」,曰「讀冊」。讀冊官稱臣某謹讀冊。讀畢,舉冊官納冊於匣,興,以授典瑞使,出笏,立於冊案西南,典瑞使置於受冊案。掌儀贊曰「舉寶官興,俱至案前跪」,曰「搢笏,取寶於盝,對舉」,曰「讀寶官興,俱至案前跪」,曰「讀寶」。讀寶官稱臣某謹讀寶。讀畢,舉寶官納寶於盝,興,以授典瑞使,出笏,立於寶案東南,典瑞使置於受寶案。掌儀贊曰「奉冊使以下皆就拜」,曰「興」,曰「平身」。參議中書省事四人,以篚奉詔書,由殿左門入,至御榻前,跪讀詔文,如常儀,授典瑞使押寶畢,置於篚,對舉,由正門出,至丹墀北,置於詔案。冊使以下由南東門出,就位聽詔,如儀。儀鸞使四人,舁進冊寶案,由左門出。



 侍儀使引班首由左階隮,前行色樂作,至宇下,樂止,舞旋至露階立。班首入殿,宣徽使奉隨,班首跪,宣徽使西北向立。班首致詞曰:「冊寶禮畢,願上皇帝、皇后萬萬歲壽。」宣徽使應曰:「如所祝。」樂作。通贊唱曰「分班」。進酒畢,班首由東南門出,降階,復位,樂止。通贊唱曰「合班」。奏進表章禮物,贊拜、舞蹈、山呼、錫宴,並如元正之儀。



 冊立皇后儀



 前期二日,儀鸞司設發冊寶案於大明殿御座前稍西,設發寶案稍東。掌謁設香案於皇后殿前,設冊案於殿內座榻前稍西,寶案稍東,設受冊案於座塌上稍西,設受寶案於稍東。侍儀司設板位,冊使副位於廷中,北面,冊官位於右,寶官位於左,禮儀使位於冊案前,主節位於太尉左。皇后殿廷亦如之。



 至期大昕,引贊敘太尉以下於闕廷,各公服。侍儀使、禮儀使、引冊使,引冊、奉冊、舉冊、讀冊、捧冊官,由月華門入;侍儀使、禮儀使、引冊副,引寶、奉寶、舉寶、讀寶、捧寶官,由日精門入。至露階下,依板位立。侍儀使捧牙牌入至寢殿前,跪報外辦。內侍入奏,出傳制曰「可」,侍儀使俯伏興。皇帝出閣升輦,鳴鞭三。侍儀使引導從導皇帝入大明殿,升御座,鳴鞭三。



 司晨報時雞唱畢,尚引引殿前班入起居位,起居、贊拜、舞蹈、山呼,如儀。



 宣贊唱曰「各恭事」。引贊引冊使以下入就位,掌儀舍人引承奉班都知、侍儀使、禮儀使、主節、捧冊、捧寶官,升自左階,由南東門入,至御座前,分左右相向立。掌儀贊曰「禮儀使稍前跪」,曰「太尉以下皆跪」。禮儀使跪奏請進發皇后冊寶。掌儀贊曰「就拜」,曰「興」,曰「平身」,曰「太尉以下皆興」,曰「復位」。掌儀贊曰「內謁者稍前」,曰「搢笏」,曰「捧冊寶跪進皇帝」,曰「以冊寶授捧冊寶官」,捧冊寶官跪受,興。掌儀贊曰「主節官搢笏持節」,禮儀使引節導冊寶由正門出,至露階,南向立。禮儀使稱有制,承奉班都知唱曰「太尉以下皆再拜」,通贊曰「鞠躬」,曰「拜」,曰「興」,曰「拜」,曰「興」,曰「平身」。禮儀使宣制曰「命太尉某等持節授皇后冊寶」,通贊贊曰「鞠躬」,曰「拜」,曰「興」,曰「拜」,曰「興」,曰「平身」。降至露階下,依次就位。掌儀唱曰「以冊寶置於案」,曰「出笏」,曰「復位」。方輿舁以行,樂作。侍儀使、禮儀使引太尉及冊寶官,奉隨至皇后宮庭奠案,樂止。掌儀唱曰「捧冊寶官稍前,搢笏」。捧冊寶使、太尉以下奉隨由正階隮,至案前。掌儀贊曰「以冊寶置於案」,曰「出笏」,曰「復位」。侍儀使稍前跪報外辦,內侍入啟,出傳旨曰「可」,侍儀使俯伏興。



 皇后出閣,詣褥位。太尉稱制遣臣某等恭授皇后冊寶。內侍贊禮曰「跪」,掌儀贊曰「太尉以下皆跪」。內侍贊皇后曰「上香」,曰「上香」,曰「三上香」,曰「拜」,曰「興」,曰「拜」,曰「興」。掌儀贊曰「太尉以下皆興」。皇后升殿,立於座榻前。承奉班都知唱曰「太尉以下進冊寶」,掌儀唱曰「捧冊寶官稍前,搢笏」。捧冊寶由正門至殿內。掌儀贊曰「以冊寶跪置於案」,曰「捧冊寶官出笏,興,復位」,曰「太尉以下皆跪」,曰「舉冊官興,至案前跪」,曰「搢笏,取冊於匣,置於盤,對舉」,曰「讀冊官興,至案前跪」,曰「讀冊」。讀冊官稱臣某謹讀冊,讀畢,納冊於匣。掌儀贊曰「出笏,舉寶官興,至案前跪,搢笏,取寶於盝,對舉」,曰「讀寶官興,至案前跪」,曰「讀寶」。讀寶官稱臣某謹讀寶,讀畢,納寶於盝。掌儀贊曰「出笏」,曰「太尉以下皆就拜」,曰「興」,曰「平身」。捧冊寶官以冊寶授太尉,太尉以授掌謁,掌謁以冊寶置於受冊寶案。掌儀唱曰「太尉以下跪」,曰「眾官皆跪」。太尉致祝辭曰:「冊寶禮畢,伏願皇后與天同算。」司徒應曰:「如所祝。」就拜,興,平身。太尉進酒,樂作;皇后飲畢,樂止。禮儀使引節引主節由正門以出。侍儀使引太尉以下,由左門至階下,北面立。承奉班都知唱曰「太尉以下皆再拜」,通贊曰「鞠躬」,曰「拜」,曰「興」,曰「拜」,曰「興」,曰「平立」。侍儀使引太尉以下還詣皇帝御座前,跪奏曰:「奉制授皇后冊寶,謹以禮畢。」就拜,興,由左門出,降詣旁折位。



 侍儀使引導從導皇后詣大明殿前謝恩,掌謁贊曰「拜」,曰「興」,曰「拜」,曰「興」。侍儀使分退,掌謁導皇后升御座。典引引丞相以下入起居位,起居贊拜如儀。侍儀使詣右丞相前請進酒,雙引升殿,至宇下褥位立。侍儀使分左右北向立,俟前行色曲將半,舞旋列定,通贊唱曰「分班」樂作。侍儀使引右丞相由南東門入,宣徽使奉隨至御榻前,右丞相跪,宣徽使立於東南,曲終。右丞相祝贊曰:「冊寶禮畢,臣等不勝慶抃,同上皇帝、皇后萬萬歲壽。」宣徽使應曰:「如所祝。」右丞相俯伏興,退詣進酒位。進酒、進表章禮物、贊拜、僧道賀獻、大宴殿上,並如元正儀。宴畢,鳴鞭三。侍儀使導駕,引進使導後,還寢殿,如來儀。



 冊立皇太子儀



 前期三日,右丞相率百僚至金玉局冊寶案前,舍人贊曰「鞠躬」,曰「拜」,曰「興」,曰「拜」,曰「興」,曰「平身」。曰「班首稍前」,曰「跪」,曰「在位官皆跪」,曰「搢笏」,曰「上香」,曰「上香」,曰「三上香」,曰「出笏」,曰「就拜」,曰「興」,曰「拜」,曰「興」,曰「拜」,曰「興」,曰「平身」。侍儀使、舍人分引群臣,儀衛音樂導至中書省,正位安置。



 前期二日,儀鸞司設發冊案於大明殿御座西,發寶案於東。典寶官設香案於太子殿前階上,設冊案於西,寶案於東;又設受冊案於殿內座榻之西,受寶案於東。侍儀司設板位,太尉、冊使副位於大明殿廷,太尉位居中,冊官位於右,寶官位於左,禮儀使位於前,主節官位於太尉之左。太子殿廷亦如之,樂位布置亦如之。右丞相率百僚朝服,至中書省冊寶案前,敘立定。舍人贊曰「鞠躬」,曰「拜」,曰「興」,曰「拜」,曰「興」,曰「平身」。曰「班首稍前」,曰「跪」,曰「搢笏」,曰「在位官皆跪」,曰「上香」,曰「上香」,曰「三上香」,曰「出笏」,曰「就拜」,曰「興」,曰「拜」,曰「興」,曰「拜」,曰「興」,曰「平立」。舍人分引群臣,儀衛導從,音樂傘扇,導至闕前。控鶴奠案,方輿官舁之,由中道入崇天門,冊使以下奉隨至露階下。方輿官置冊案於西,寶案於東,分退立於兩廡。冊使副北面,引冊官舉冊官、讀冊官、捧冊官位於冊案西,東向;引寶官、舉寶官、讀寶官、捧寶官位於寶案東,西向。掌儀舍人贊曰「捧冊官稍前」,曰「搢笏」,曰「捧冊」。又贊曰「捧寶官稍前」,曰「搢笏」,曰「捧寶」。侍儀使、引進使、引冊官、引寶官前導,捧冊寶官次之,冊使副以下奉隨升大明殿午階,由正門入,至進發冊寶案前,冊使副北面立,引冊官、引寶官、舉冊官、舉寶官以下,分左右夾冊寶案立。掌儀贊曰「以冊寶置於案」,曰「出笏」,曰「復位」。侍儀使引奉冊使以下由左門出,百闢趨退。



 至期大昕,引贊引冊使以下,皆公服,敘位於闕廷。侍儀使導從皇帝出閣,鳴鞭三,升大明殿,登御座。尚引引殿前班入起居位,起居贊拜如儀,宣贊唱曰「各恭事」。引贊引冊使以下入就位,掌儀舍人引承奉班都知、侍儀使、禮儀使、主節郎、捧冊、捧寶官,升自左階,由左門入,至御座前,分左右立。掌儀贊曰「禮儀使稍前」,曰「跪」,曰「眾官皆跪」。禮儀使奏請發皇太子冊寶,掌儀唱曰「就拜」,曰「興」,曰「平身」,曰「眾官皆興」,曰「復位」。曰「內謁者稍前」,曰「搢笏」,曰「捧冊寶跪進皇帝」,曰「以冊寶授捧冊寶官」,捧冊寶官跪受,興。掌儀贊曰「主節郎搢笏持節」,禮儀使引節導冊寶由正門以出,至露階南向立。禮儀使稱有制,承奉班都知唱曰「太尉以下皆再拜」,掌儀贊曰「鞠躬」,曰「拜」,曰「興」,曰「拜」,曰「興」,曰「平身」。禮儀使宣制曰「上命太尉等持節授皇太子冊寶」,掌儀贊曰「鞠躬」,曰「拜」,曰「興」,曰「拜」,曰「興」,曰「平身」。禮儀使引節導冊寶,降至露階下,依次就位。掌儀贊曰「以冊寶置於案」,曰「出笏」,曰「復位」。方輿舁以行,樂作。侍儀使、禮儀使、主節前導,冊使以下奉隨由正門出。至闕前,方輿奠案,控鶴舁以行。至皇太子殿廷,控鶴奠案,方輿舁以行。入至露階下奠案,方輿退,樂止。冊使以下以次立,掌儀贊曰「捧冊寶官稍前,搢笏,捧冊寶」。侍儀使引節,主節導冊寶以行,冊使以下由正階隮,節立於香案之西。掌儀贊曰「捧冊寶官跪,以冊寶置於案」,曰「出笏」,曰「興」,曰「就位」。右庶子跪報外備,內侍入啟,出傳旨曰「可」,右庶子俯伏興。



 皇太子出閣,立於香案前。掌儀贊曰「皇太子跪」,曰「上香」,曰「上香」,曰「三上香」,曰「拜」,曰「興」,曰「拜」,曰「興」。太尉前稱制遣臣某等恭授皇太子冊寶,復位。掌儀贊曰「皇太子拜」,曰「興」,曰「拜」,曰「興」。請皇太子詣褥位,南向立。曰「皇太子跪」,曰「諸執事官皆跪」。曰「舉冊官興,至案前」,曰「跪」,曰「讀冊」。讀畢,曰「納冊於匣」,曰「出笏」。掌儀唱曰「舉寶官興,至案前」,曰「跪」,曰「讀寶」。讀畢,曰「納寶於盝」,曰「出笏」,曰「舉冊寶官、讀冊寶官皆興,復位。」掌儀贊曰「太尉進授冊寶」,侍儀使引太尉、司徒至冊寶案前,搢笏,以冊寶跪進。皇太子恭受,以授左、右庶子,左、右庶子搢笏跪受。掌儀贊曰「皇太子興,冊使以下皆興」。右庶子捧冊,左庶子捧寶,導皇太子入殿。右庶子奠冊於受冊案,左庶子奠寶於受寶案。引節引主節立於殿西北,引贊引太尉以下降階復位,北向立。承奉班都知唱曰「太尉以下皆再拜」,掌儀贊曰「鞠躬」,曰「拜」,曰「興」,曰「拜」,曰「興」,曰「平身」,樂作。侍儀使詣太尉前請進酒,太尉入至殿內,進酒畢,降復位,樂止。



 侍儀使、禮儀使、主節導太尉以下還詣大明殿御座前,跪奏曰:「奉制授皇太子冊寶,謹以禮畢。」俯伏興,降詣位。侍儀使、左右庶子導皇太子詣大明殿御座前謝恩,右庶子贊曰「拜」,曰「興」,曰「拜」,曰「興」。進酒,又贊曰「拜」,曰「興」,曰「拜」,曰「興」。降殿,還府。



 侍儀使詣右丞相前請進酒,雙引升殿,至宇下褥位立,侍儀使分左右,北向立。俟前行色曲將半,舞旋列定,通贊唱曰「分班」,樂作。侍儀使、右丞相由南東門入,宣徽使奉隨至御榻前。右丞相跪,宣徽使立於東南,曲終。右丞相祝贊曰:「皇太子冊寶禮畢,臣等不勝慶抃,同上皇帝、皇后萬萬歲壽。」宣徽使應曰:「如所祝。」右丞相俯伏興,退詣進酒位。進酒、進表章禮物、贊拜,如元正儀。駕興,鳴鞭三。侍儀使導駕還寢殿,如來儀。



 皇太子還府,升殿。典引引群臣入就起居位,通班,自班西行至中道,唱曰「具官某以下起居」,典引贊曰「鞠躬」,曰「平身」。進就拜位,宣贊唱曰「拜」,通贊贊曰「鞠躬」,曰「拜」,曰「興」,曰「拜」,曰「興」,曰「平身」。侍儀使詣班首前請進酒,雙引由左階至殿宇下褥位立,侍儀分左右,北向立。俟前行色曲將半,舞旋列定,通贊唱曰「分班」。班首入自左門,右庶子隨至座前。班首跪,右庶子立於東南。俟曲終,班首致祝詞曰:「冊寶禮畢,願上殿下千秋之壽。」右庶子應曰:「如所祝。」班首俯伏興,退至進酒位,搢笏,捧觴,北向立,右庶子退復位。俟舞旋至露階,樂舞至第四拍,班首進酒。宣贊唱曰「文武百僚皆再拜」,通贊贊曰「鞠躬」,曰「拜」,曰「興」,曰「拜」,曰「興」,曰「平身」。班首自東門出,復位,樂止。通贊唱曰「合班」。中書押進箋及禮物案至橫階下,進讀箋官由左階隮,進讀禮物官至階下。俟進讀箋官至宇下,先讀箋目,次讀箋,讀畢,俯伏興,降至階下。進讀禮物官升階,至宇下,跪讀禮物狀畢,俯伏興,退,同讀箋官至橫階,隨箋案西行,至右廡下,禮物案東行,至左廡下,各付所司。宣贊唱曰「拜」,通贊贊曰「鞠躬」,曰「拜」,曰「興」,曰「拜」,曰「興」,曰「平立」。右庶子導皇太子還閣。



 太皇太后上尊號進冊寶儀



 前期二日,儀鸞司設進發冊寶案於大明殿御座之前,掌謁設進冊寶案於太皇太后殿座榻前,設受冊寶案於座榻上,並冊西寶東。侍儀司設冊使副位於廷中,北面,冊官位右,寶官位左,禮儀使位於前,以北為上。太皇太后殿廷亦如之。



 至期大昕,群臣皆公服,敘位闕前。侍儀使、禮儀使、引冊使,引冊、奉冊、舉冊、讀冊、捧冊官,由月華門入,侍儀使、禮儀使、引冊副,引寶、奉寶、舉寶、讀寶、捧寶官,由日精門入。至露階下,依板位立。侍儀使捧牙牌入至寢殿前,跪報外辦,內侍入奏,出傳制曰「可」,侍儀使俯伏興。皇帝出閣升輦,鳴鞭三;入大明殿,升御座,鳴鞭三。司晨報時雞唱畢,侍儀使、禮儀使、引冊使以下升自東階,由左門入,至御榻前,相向立。掌儀贊曰「奏中嚴」,侍儀使捧牙牌跪奏曰「中嚴」,又贊曰「就拜」,曰「興」,曰「平身」,曰「復位」,曰「禮儀使稍前跪」,曰「冊使以下皆跪」。禮儀使奏請進發太皇太后冊寶,掌儀贊曰「就拜」,曰「興」,曰「平身」,曰「復位」,曰「內謁者稍前」,曰「搢笏,奉冊寶上進」,曰「冊使副、捧冊寶官稍前」,曰「搢笏」,曰「內謁者跪進冊寶」。皇帝興,以冊授冊使,冊使跪受,興,以授捧冊官,出笏;以寶授冊副,冊副跪受,興,以授捧寶官,出笏。侍儀使、禮儀使、引冊、引寶官,導冊寶由正門出,冊使以下奉隨,至階下。掌儀贊曰「以冊寶置於案」,曰「出笏,復位」。方輿舁行,樂作。侍儀使、禮儀使、引冊、引寶前導,冊使以下奉隨,至興聖宮前,奠案,樂止。



 侍儀使以導從入至太皇太后寢殿前,跪報外辦。掌謁入啟,出傳旨曰「可」,侍儀使俯伏興。侍儀使、掌謁前導太皇太后升殿。導太皇太后時,侍儀使入至大明殿,跪奏冊寶至興聖宮,請行禮。駕興,鳴鞭三,侍儀使前引導從至興聖宮,升御座。侍儀使出,至案所,樂作。方輿入,至露階下奠案。冊使副立於案前,冊官東向,寶官西向。方輿分退,立於兩廡,樂止。



 尚引引殿前班入起居位,相向立,起居拜舞,如元正儀。禮畢,宣贊唱曰「各恭事」,贊引冊使以下退至起居位。通班舍人唱曰「攝某官具官或太尉,具官無常。



 臣某以下起居」,引贊贊曰「鞠躬」,曰「平身」。進入丹墀,知班唱曰「班齊」,宣贊唱曰「拜」,通贊贊曰「鞠躬」,曰「拜」,曰「興」,曰「拜」,曰「興」,曰「平身」,宣贊唱曰「各恭事」。進至案前,依位立。宣贊唱曰「太尉以下進上冊寶」,掌儀贊曰「捧冊寶官稍前,搢笏,捧冊寶」。侍儀使引冊寶官前導,冊使奉隨,至御榻,進冊寶案前。掌儀唱曰「跪」,捧冊寶官不跪,曰「以冊寶置於案」,曰「捧冊寶官出笏復位」,曰「太尉以下皆跪」,曰「讀、舉冊寶官興,俱至案前跪」。掌儀贊曰「舉冊官搢笏,取冊於匣,置於盤,對舉」。曰「讀冊」,讀冊官稱臣某謹讀冊。讀畢,舉冊官納冊於匣。掌儀贊曰「出笏」,曰「舉寶官搢笏,取寶於盝,對舉」。曰「讀寶」,讀寶官稱臣某謹讀寶。讀畢,舉寶官納寶於盝。掌儀贊曰「出笏」,曰「就拜」,曰「興」,曰「平身」,曰「眾官皆興」,曰「復位」。曰「太尉、司徒、奉冊寶官稍前」,曰「捧冊寶官稍前」,曰「搢笏」,曰「捧冊寶上進」,曰「皇帝躬授太皇太后冊寶」,太皇太后以冊寶授內掌謁,內掌謁置於案。皇帝興,進酒。太皇太后舉觴飲畢,皇帝復御座畢,掌儀贊曰「眾官皆復位」。侍儀使、引冊使以下,分左右,出就位。皇帝率皇后及后妃、公主,降丹墀,北面拜賀,升殿。皇太子及諸王拜賀,升殿。典引引百官入就起居位,通班舍人唱曰「文武百僚具官臣某以下起居」,曰「鞠躬」,曰「平身」,引至丹墀拜位。知班報班齊,宣贊唱曰「拜」,通贊贊曰「鞠躬」,曰「拜」,曰「興」,曰「拜」,曰「興」,曰「平身」。侍儀使詣班首前請進酒,雙引至殿宇下褥位立,俟舞旋列定,通贊唱曰「分班」,樂作。侍儀使引班首由南東門入,宣徽使奉隨,至御榻前,班首跪,曲終。班首祝贊曰:「冊寶禮畢,臣等不勝欣抃,願上太皇太后、皇帝億萬歲壽。」宣徽使應曰:「如所祝。」班首俯伏興,退詣進酒位。以下並同元正儀。



 皇太后上尊號進冊寶儀同前儀



 太皇太后加上尊號進冊寶儀同前儀



 進發冊寶導從



 清道官二人,警蹕二人,並分左右,皆攝官,服本品朝服。



 雲和樂一部:署令二人,分左右。次前行戲竹二,次排簫四,次簫管四,次板二,次歌四,並分左右。前行內琵琶二十,次箏十六,次箜篌十六,次蓁十六,次方響八,次頭管二十八,次龍笛二十八,為三十三重。重四人。



 次杖鼓三十,為八重。次板八,為四重。板內大鼓二,工二人,舁八人。樂工服並與鹵簿同。法物庫使二人,服本品服。次硃團扇八,為二重。次小雉扇八,次中雉扇八,次大雉扇八,分左右,為十二重。次硃團扇八,為二重。次大傘二,次華蓋二,次紫方傘二,次紅方傘二,次曲蓋二,並分左右。執傘扇所服,並同立仗。



 圍子頭一人,中道。次圍子八人,分左右。服與鹵簿內同。



 安和樂一部:署令二人,服本品服。札鼓六,為二重,前四後二。次和鼓一,中道。次板二,分左右。次龍笛四,次頭管四,並為二重。次羌管二,次笙二,並分左右。次雲璈一,中道。次闉二,分左右。樂工服與鹵簿內同。



 傘一,中道。椅左,踏右。執人,皁巾,大團花緋錦襖,金塗銅束帶,行縢鞋襪。



 拱衛使一人,服本品服。



 舍人二人,次引寶官二人,並分左右,服四品服。



 香案,中道。輿士控鶴八人,服同立仗內表案輿士。侍香二人,分左右,服四品服。



 寶案,中道。輿士控鶴十有六人,服同香案輿士。方輿官三十人,夾香案寶案,分左右而趨。至殿門,則控鶴退,方輿官舁案以升。唐巾,紫羅窄袖衫,金塗銅束帶,烏靴。



 引冊二人,四品服。



 香案,中道。輿士控鶴八人,服同寶案輿士。侍香二人,分左右,服四品服。



 冊案,中道。輿士控鶴十有六人,服同寶案輿士。方輿官三十人,夾香案冊案,分左右而趨。至殿門,則控鶴退,方輿官舁案以升。巾服與寶案方輿官同。



 葆蓋四十人,次閱仗舍人二人,服四品服。次小戟四十人,次儀鍠四十人,夾雲和樂傘扇,分左右行,服同立仗。



 拱衛使二人,服本品朝服。次班劍十,次梧杖十二,次斧十二,次鐙杖二十,次列絲十,皆分左右。次水簹左,金盆右。次列絲十,次立瓜十。次金杌左,鞭桶右,蒙鞍左,散手右。次立瓜十,次臥瓜三十,並夾葆蓋、小戟、儀鍠,分左右行。服並同鹵簿內。



 拱衛外舍人二人,服四品服,引導冊諸官。次從九品以上,次從七品以上,次從五品以上,並本品朝服。



 金吾折沖二人,牙門旗二,每旗引執五人。次青槊四十人,赤槊四十人,黃槊四十人,白槊四十人,紫槊四十人,並兜鍪甲靴,各隨槊之色,行導冊官外。



 冊案後舍人二人,服四品服。次太尉右,司徒左。次禮儀使二人,分左右。次舉冊官四人右,舉寶官四人左。次讀冊官二人右,讀寶官二人左。次閣門使四人,分左右,並本品服。



 知班六人,分左右,服同立仗,往來視諸官之失儀者而行罰焉。



 冊寶攝官



 上尊號冊寶,凡攝官二百五十有六人:奉冊官四人,奉寶官四人,捧冊官二人,捧寶官二人,讀冊官二人,讀寶官二人,引冊官五人,引寶官五人,典瑞官三人,糾儀官四人,殿中侍御史二人,監察御史四人,閣門使三人,清道官四人,點試儀衛五人,司香四人,備顧問七人,代禮官三十人,拱衛使二人,押仗二人,方輿官一百六十人。



 上皇太后冊寶,凡攝官二百五十人:攝太尉一人,攝司徒一人,禮儀使四人,奉冊官二人,奉寶官二人,引冊官二人,引寶官二人,舉冊官二人,舉寶官二人,讀冊官二人,讀寶官二人,捧冊官二人,捧寶官二人,奏中嚴一人,主當內侍十人,閣門使六人,充內臣十三人,糾儀官四人,代禮官四十二人,掌謁四人,司香十二人,折沖都尉二人,拱衛使二人,清道官四人,警蹕官四人,方輿官百二十人。



 上太皇太后冊寶攝官,同前。



 授皇后冊寶,凡攝寶官百八十人:攝太尉一人,攝司徒一人,主節官二人,禮儀使四人,奉冊官二人,奉寶官二人,引冊官二人,引寶官二人,舉冊官二人,舉寶官二人,讀冊官二人,讀寶官二人,內臣職掌十人,宣徽使二人,閣門使四人,代禮官三十七人,侍香二人,清道官四人,折沖都尉二人,警蹕官四人,中宮內臣九人,糾儀官四人,接冊內臣二人,接寶內臣二人,方輿官七十四人。



 授皇太子冊,凡攝官四十有九人:攝太尉一人,奉冊官二人,持節官一人,捧冊官二人,讀冊官二人,引冊官二人,攝禮儀使二人,主當內侍六人,副持節官五人,侍從官十一人,代禮官十六人。



 攝行告廟儀如受尊號,上太皇太后、皇太后冊寶,



 冊立皇后、皇太子,凡國家大典禮,皆告宗廟。



 前期二日,太廟令掃除內外,翰林國史院學士撰寫祝文;前一日,告官等致齋一日。其日,告官等各服紫服,奉祝版,進請御署訖,差控鶴,用紅羅銷金案抬舁,覆以黃羅帕,並奉御香、御酒,如常儀,迎至祝所齋宿。告日質明前三刻,禮直官引太廟令,率其屬入廟殿,開室,陳設如儀。禮直官引告官等,各服紫服,以次入就位,東向立定。禮直官稍前,贊曰「有司謹具,請行事」。贊者曰「再拜」,在位者皆再拜。禮直官先引執事者各就位,次引告官詣盥洗、爵洗位,北向立。搢笏,盥手,帨手,洗爵、拭爵訖,執笏,請詣酒尊所,搢笏,執爵,司尊者舉LV,良醖令酌酒,以爵授奉爵官,執笏,詣太祖室,再拜。執事者奉香,告官搢笏跪,三上香,執爵三祭酒,以虛爵授奉爵官,執笏,俯伏興。舉祝官搢笏跪,對舉祝版,讀祝官跪讀祝文訖,奠祝於案,執笏,俯伏興。禮直官、贊告官再拜畢,每室並如上儀。告畢,引告官以下降,復位。再拜訖,詣望瘞燔祝,再拜,半燎,告官以下皆退。



 國史院進先朝實錄儀



 是日大昕,諸司官具公服,立於光天門外,侍儀使引《實錄》案以入,監修國史以下奉隨,至光天殿前,分班立,皇帝升御座。宣贊唱曰「拜」,通贊贊曰「鞠躬」,曰「拜」,曰「興」,曰「拜」,曰「興」,曰「平身」。待制四人奉《實錄》,升自午階,監修國史以下奉隨,至御前香案南立,眾官降,復位。應奉翰林文字升,至《實錄》前,跪讀表,讀畢,俯伏興,復位。翰林學士承旨升,至御前,分班立,俟御覽畢,降復位。宣贊唱曰「監修國史以下皆再拜」,通贊贊曰「鞠躬」,曰「拜」,曰「興」,曰「拜」,曰「興」,曰「平身」。待制升,取《實錄》,降自午階,置於案,由光天門以出,音樂儀從前導,還國史院,置於堂上。通贊贊曰「鞠躬」,曰「拜」,曰「興」,曰「拜」,曰「興」,曰「平身」,曰「搢笏」,曰「上香」,曰「上香,曰「三上香」,曰「出笏」,曰「就拜」,曰「興,」曰「拜」,曰「興」,曰「拜」,曰「興」,曰「平立」。百僚趨退。



\end{pinyinscope}