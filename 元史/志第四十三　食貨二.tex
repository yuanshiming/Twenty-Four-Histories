\article{志第四十三 食貨二}

\begin{pinyinscope}

 ○歲課



 山林川澤之產,若金、銀、珠、玉、銅、鐵、水銀、硃砂、碧甸子、鉛、錫、礬、硝、堿、竹、木之類,皆天地自然之利,有國者之所必資也,而或以病民者有之矣。元興,因土人呈獻,而定其歲入之課,多者不盡收,少者不強取,非知理財之道者,能若是乎?



 產金之所,在腹里曰益都、檀、景,遼陽省曰大寧、開元,江浙省曰饒、徽、池、信,江西省曰龍興、撫州,湖廣省曰嶽、澧、沅、靖、辰、潭、武岡、寶慶,河南省曰江陵、襄陽,四川省曰成都、嘉定,雲南省曰威楚、麗江、大理、金齒、臨安、曲靖、元江、羅羅、會川、建昌、德昌、柏興、烏撒、東川、烏蒙。



 產銀之所,在腹里曰大都、真定、保定、雲州、般陽、晉寧、懷孟、濟南、寧海,遼陽省曰大寧,江浙省曰處州、建寧、延平,江西省曰撫、瑞、韶,湖廣省曰興國、郴州,河南省曰汴梁、安豐、汝寧,陜西省曰商州,雲南省曰威楚、大理、金齒、臨安、元江。



 產珠之所,曰大都,曰南京,曰羅羅,曰水達達,曰廣州。



 產玉之所,曰于闐,曰匪力沙。



 產銅之所,在腹里曰益都,遼陽省曰大寧,雲南省曰大理、澄江。



 產鐵之所,在腹里曰河東、順德、檀、景、濟南,江浙省曰饒、徽、寧國、信、慶元、臺、衢、處、建寧、興化、邵武、漳、福、泉,江西省曰龍興、吉安、撫、袁、瑞、贛、臨江、桂陽,湖廣省曰沅、潭、衡、武岡、寶慶、永、全、常寧、道州,陜西省曰興元,雲南省曰中慶、大理、金齒、臨安、曲靖、澄江、羅羅、建昌。



 產硃砂、水銀之所,在遼陽省曰北京,湖廣省曰沅、潭,四川省曰思州。



 產碧甸子之所,曰和林,曰會川。



 產鉛、錫之所,在江浙省曰鉛山、臺、處、建寧、延平、邵武,江西省曰韶州、桂陽,湖廣省曰潭州。



 產礬之所,在腹里曰廣平、冀寧,江浙省曰鉛山、邵武,湖廣省曰潭州,河南省曰廬州、河南。



 產硝、堿之所,曰晉寧。若竹、木之產,所在有之,不可以所言也。



 初,金課之興,自世祖始。其在益都者,至元五年,命於從剛、高興宗以漏籍民戶四千,於登州棲霞縣淘焉。十五年,又以淘金戶二千簽軍者,付益都、淄萊等路淘金總管府,依舊淘金。其課於太府監輸納。在遼陽者,至元十年,聽李德仁於龍山縣胡碧峪淘採,每歲納課金三兩。十三年,又於遼東雙城及和州等處採焉。在江浙者,至元二十四年,立提舉司,以建康等處淘金夫凡七千三百六十五戶隸之,所轄金場凡七十餘所。未幾以建康無金,革提舉司,罷淘金戶,其徽、饒、池、信之課,皆歸之有司。在江西者,至元二十三年,撫州樂安縣小曹周歲辦金一百兩。在湖廣者,至元二十年,撥常德、澧、辰、沅、靖民萬戶,付金場轉運司淘焉。在四川者,元貞元年,以其病民罷之。在雲南者,至元十四年,諸路總納金一百五錠。此金課之興革可考者然也。



 銀在大都者,至元十一年,聽王庭璧於檀州奉先等洞採之。十五年,令關世顯等於薊州豐山採之。在雲州者,至元二十七年,撥民戶於望雲煽煉,設從七品官掌之。二十八年,又開聚陽山銀場。二十九年,遂立雲州等處銀場提舉司。在遼陽者,延祐四年,惠州銀洞三十六眼,立提舉司辦課。在江浙者,至元二十一年,建寧南劍等處立銀場提舉司煽煉。在湖廣者,至元二十三年,韶州路曲江縣銀場聽民煽煉,每年輸銀三千兩。在河南者,延祐三年,李允直包羅山縣銀場,課銀三錠。四年,李珪等包霍丘縣豹子崖銀洞,課銀三十錠,其所得礦,大抵以十分之三輸官。此銀課之興革可考者然也。



 珠在大都者,元貞元年,聽民於楊村、直沽口撈採,命官買之。在南京者,至元十一年,命滅怯、安山等於宋阿江、阿爺苦江、忽呂古江採之。在廣州者,採於大步海。他如兀難、曲朵剌、渾都忽三河之珠,至元五年,徙鳳哥等戶撈焉。勝州、延州、乃延等城之珠,十三年,命朵魯不鷿等撈焉。此珠課之興革可考者然也。



 玉在匪力沙者,至元十一年,迷兒、麻合馬、阿里三人言,淘玉之戶舊有三百,經亂散亡,存者止七十戶,其力不充,而匪力沙之地旁近有民戶六十,每同淘焉。於是免其差徭,與淘戶等所淘之玉,於忽都、勝忽兒、舍裏甫丁三人所立水站,遞至京師。此玉課之興革可考者然也。



 銅在益都者,至元十六年,撥戶一千,於臨朐縣七寶山等處採之。在遼陽者,至元十五年,撥採木夫一千戶,於錦、瑞州雞山、巴山等處採之。在澄江者,至元二十二年,撥漏籍戶於薩矣山煽煉,凡一十有一所。此銅課之興革可考者然也。



 鐵在河東者,太宗丙申年,立爐於西京州縣,撥冶戶七百六十煽焉。丁酉年,立爐於交城縣,撥冶戶一千煽焉。至元五年,始立洞冶總管府。七年罷之。十三年,立平陽等路提舉司。十四年又罷之。其後廢置不常。大德十一年,聽民煽煉,官為抽分。至武宗至大元年,復立河東都提舉司掌之。所隸之冶八:曰大通,曰興國,曰惠民,曰利國,曰益國,曰閏富,曰豐寧,豐寧之冶蓋有二云。在順德等處者,至元三十一年,撥冶戶六千煽焉。大德元年,設都提舉司掌之,其後亦廢置不常。至延祐六年,始罷兩提舉司,並為順德廣平彰德等處提舉司。所隸之冶六:曰神德,曰左村,曰豐陽,曰臨水,曰沙窩,曰固鎮。在檀、景等處者,太宗丙申年,始於北京撥戶煽焉。中統二年,立提舉司掌之,其後亦廢置不常。大德五年,始並檀、景三提舉司為都提舉司,所隸之冶有七:曰雙峰,曰暗峪,曰銀崖,曰大峪,曰五峪,曰利貞,曰錐山。在濟南等處者,中統四年,拘漏籍戶三千煽焉。至元五年,立洞冶總管府,其後亦廢置不常。至至大元年,復立濟南都提舉司,所隸之監有五:曰寶成,曰通和,曰昆吾,曰元國,曰富國。其在各省者,獨江浙、江西、湖廣之課為最多。凡鐵之等不一,有生黃鐵,有生青鐵,有青瓜鐵,有簡鐵,每引二百斤。此鐵課之興革可考者然也。



 硃砂、水銀在北京者,至元十一年,命蒙古都喜以恤品人戶於吉思迷之地採煉。在湖廣者,沅州五寨蕭雷發等每年包納硃砂一千五百兩,羅管賽包納水銀二千二百四十兩。潭州安化縣每年辦硃砂八十兩、水銀五十兩。碧甸子在和林者,至元十年,命烏馬兒採之。在會川者,二十一年,輸一千餘塊。此硃砂、水銀、碧甸子課之興革可考者然也。



 鉛、錫在湖廣者,至元八年,辰、沅、靖等處轉運司印造錫引,每引計錫一百斤,官收鈔三百文,客商買引,赴各冶支錫販賣。無引者,比私鹽減等杖六十,其錫沒官。此鉛、錫課之興革可考者然也。



 礬在廣平者,至元二十八年,路鵬舉獻磁州武安縣礬窯一十所,周歲辦白礬三千斤。在潭州者,至元十八年,李日新自具工本,於瀏陽永興礬場煎烹,每十斤官抽其二。在河南者,二十四年,立礬課所於無為路,每礬一引重三十斤,價鈔五兩。此礬課之興革可考者然也。



 竹之所產雖不一,而腹里之河南、懷孟,陜西之京兆、鳳翔,皆有在官竹園。國初,皆立司竹監掌之,每歲令稅課所官以時採斫,定其價為三等,易於民間。至元四年,始命制國用使司印造懷孟等路司竹監竹引一萬道,每道取工墨一錢,凡發賣皆給引。至二十二年,罷司竹監,聽民自賣輸稅。明年,又用郭畯言,於衛州復立竹課提舉司,凡輝、懷、嵩、洛、京襄、益都、宿、蘄等處竹貨皆隸焉。在官者辦課,在民者輸稅。二十三年,又命陜西竹課提領司差官於輝、懷辦課。二十九年,丞相完澤言:「懷孟竹課,頻年斫伐已損。課無所出,科民以輸。宜罷其課,長養數年。」世祖從之。此竹課之興革可考者也。若夫硝、堿、木課,其興革無籍可考,故不著焉。



 天歷元年歲課之數:



 金課:



 腹裏,四十錠四十七兩三錢。



 江浙省,一百八十錠一十五兩一錢。



 江西省,二錠四十兩五錢。



 湖廣省,八十錠二十兩一錢。



 河南省,三十八兩六錢。



 四川省,麩金七兩二錢。



 雲南省,一百八十四錠一兩九錢。



 銀課:



 腹裏,一錠二十五兩。



 江浙省,一百一十五錠三十九兩二錢。



 江西省,四百六十二錠三兩五錢。



 湖廣省,二百三十六錠九兩。



 雲南省,七百三十五錠三十四兩三錢。



 銅課:



 雲南省二千三百八十斤。



 鐵課:



 江浙省,額外鐵二十四萬五千八百六十七斤,課鈔一千七百三錠一十四兩。



 江西省,二十一萬七千四百五十斤,課鈔一百七十六錠二十四兩。



 湖廣省,二十八萬二千五百九十五斤。



 河南省,三千九百三十斤。



 陜西省,一萬斤。



 雲南省,一十二萬四千七百一斤。



 鉛錫課:



 江浙省,額外鉛粉八百八十七錠九兩五錢,鉛丹九錠四十二兩二錢,黑錫二十四錠一十兩二錢。



 江西省,錫一十七錠七兩。



 湖廣省,鉛一千七百九十八斤。



 礬課:



 腹裏,三十三錠二十五兩八錢。



 江浙省,額外四十二兩五錢。



 河南省,額外二千四百一十四錠三十三兩一錢。



 硝堿課:



 晉寧路,二十六錠七兩四錢。



 竹木課:



 腹裏,木六百七十六錠一十五兩四錢,額外木七十三錠二十五兩三錢;竹二錠四十兩,額外竹一千一百三錠二兩二錢。



 江浙省,額外竹木九千三百五十五錠二十四兩。



 江西省,額外竹木五百九十錠二十三兩三錢。



 河南省,竹二十六萬九千六百九十五竿,板木五萬八千六百條,額外竹木一千七百四十八錠三十兩一錢。



 ○鹽法



 國之所資,其利最廣者莫如鹽。自漢桑弘羊始榷之,而後世未有遺其利者也。元初,以酒醋、鹽稅、河泊、金、銀、鐵冶六色,取課於民,歲定白銀萬錠。太宗庚寅年,始行鹽法,每鹽一引重四百斤,其價銀一十兩。世祖中統二年,減銀為七兩。至元十三年既取宋,而江南之鹽所入尤廣,每引改為中統鈔九貫。二十六年,增為五十貫。元貞丙申,每引又增為六十五貫。至大己酉至延祐乙卯,七年之間,累增為一百五十貫。凡偽造鹽引者皆斬,籍其家產,付告人充賞。犯私鹽者徒二年,杖七十,止籍其財產之半;有首告者,於所籍之內以其半賞之。行鹽各有郡邑,犯界者減私鹽罪一等,以其鹽之半沒官,半賞告者。然歲辦之課,難易各不同。有因自凝結而取者,解池之顆鹽也。有煮海而後成者,河間、山東、兩淮、兩浙、福建等處之末鹽也。惟四川之鹽出於井,深者數百尺,汲水煮之,視他處為最難。今各因其所產之地言之。



 大都之鹽:太宗丙申年,初於白陵港、三叉沽、大直沽等處置司,設熬煎辦,每引有工本錢。世祖至元二年,又增寶坻三鹽場,灶戶工本,每引為中統鈔三兩,與清、滄等。八年,以大都民戶多食私鹽,因虧國課,驗口給以食鹽。十九年,罷大都及河間、山東三鹽運司,設戶部尚書、員外郎各一員,別給印,令於大都置局賣引,鹽商買引,赴各場關鹽發賣。每歲灶戶工本,省臺遣官逐季分給之。十九年,改立大都蘆臺越支三叉沽鹽使司一。二十五年,復立三叉沽、蘆臺、越支三鹽使司。二十八年,增灶戶工本,每引為中統鈔八兩。二十九年,以歲饑減鹽課一萬引,入京兆鹽運司添辦。大德元年,遂罷大都鹽運司,並入河間。



 河間之鹽:太宗庚寅年,始立河間稅課所,置鹽場,撥灶戶二千三百七十六隸之,每鹽一袋,重四百斤。甲午年,立鹽運司。庚子年,改立提舉鹽榷所,歲辦三萬四千七百袋。癸卯年,改立提舉滄清鹽課使所,歲辦鹽九萬袋。定宗四年,改真定河間等路課程所為提舉鹽榷滄清鹽使所。憲宗二年,又改河間課程所為提舉滄清深鹽使所。八年,每袋增鹽至四百五十斤。世祖中統元年,改立宣撫司提領滄清深鹽使所。四年,改滄清深鹽提領所為轉運司。是年,辦銀七千六十五錠,米三萬三千三百餘石。至元元年,又增三之一焉。二年,改立河間都轉運司,歲辦九萬五千袋。七年,始定例歲煎鹽十萬引,辦課銀一萬錠。十二年,改立都轉運使司,添灶戶九百餘,增鹽課二十萬引。十八年,以河間灶戶勞苦,增工本為中統鈔三貫。是年,又增灶戶七百八十六。十九年,罷河間都轉運司,改立清、滄鹽使司二。二十二年,復立河間等路都轉運鹽使司,增鹽課為二十九萬六百引。二十三年,改立河間都轉運司,通辦鹽酒稅課。二十五年,增工本為中統鈔五貫。二十七年,增灶戶四百七十,辦鹽三十五萬引。至大元年,又增至四十五萬引。延祐元年,以虧課,停煎五萬引。自是至天歷,皆歲辦四十萬引,所隸之場,凡二十有二。



 山東之鹽:太宗庚寅年,始立益都課稅所,撥灶戶二千一百七十隸之,每銀一兩,得鹽四十斤。甲午年,立山東鹽運司。中統元年,歲辦銀二千五百錠。三年,命課稅隸山東都轉運司。四年,令益都山東民戶,月買食鹽三斤;灶戶逃亡者,招民戶補之。是歲,辦銀三千三百錠。至元二年,改立山東轉運司,辦課銀四千六百錠一十九兩。是年,戶部造山東鹽引。六年,增歲辦鹽為七萬一千九百九十八引,自是每歲增之。至十二年,改立山東都轉運司,歲辦鹽一十四萬七千四百八十七引。十八年,增灶戶七百,又增鹽為一十六萬五千四百八十七引,灶戶工本錢亦增為中統鈔三貫。二十三年,歲辦鹽二十七萬一千七百四十二引。二十六年,減為二十二萬引。大德十年,又增為二十五萬引。至大元年之後,歲辦正、餘鹽為三十一萬引,所隸之場,凡一十有九。



 河東之鹽:出解州鹽池,池方一百二十里,每歲五月,場官伺池鹽生結,令夫搬摝鹽花。其法必值亢陽,池鹽方就,或遇陰雨,則不能成矣。太宗庚寅年,始立平陽府徵收課稅所,從實辦課,每鹽四十斤,得銀一兩。癸巳年,撥新降戶一千,命鹽使姚行簡等修理鹽池損壞處所。憲宗壬子年,又增撥一千八十五戶,歲撈鹽一萬五千引,辦課銀三千錠。世祖中統二年,初立陜西轉運司,仍置解鹽司於路村。三年,以太原民戶自煎小鹽,歲辦課銀一百五十錠。五年,又增小鹽課銀為二百五十錠。至元三年,諭陜西四川,以所辦鹽課赴行制國用使司輸納,鹽引令制國用使司給降。四年,立陜西四川轉運司。六年,立太原提舉鹽使司,直隸制國用使司。十年,命撈鹽戶九百八十餘,每丁撈鹽一石,給工價鈔五錢。歲辦鹽六萬四千引,計中統鈔一萬一千五百二十錠。二十三年,改立陜西都轉運司,兼辦鹽、酒、醋、竹等課。二十九年,減大都鹽課一萬引,入京兆鹽司添辦。是年五月,又革京兆鹽司一,止存鹽運司。大德十一年,增歲額為八萬二千引。至大元年,又增煎餘鹽為二萬引,通為一十萬二千引。延祐三年,以池為雨所壞,止辦課鈔八萬二千餘錠。於是晉寧、陜西之民改食常仁紅鹽,懷孟、河南之民改食滄鹽。五年,乃免河南、懷孟、南陽三路今歲陜西鹽課,仍授鹽運使暨所臨路府州縣正官兼知渠堰事,責以疏通壅塞。六年,改陜西運司為河東解鹽等處都轉運鹽使司,直隸中書省。十月,罷陜西行省所委巡鹽官六十八員,添設通判一員,別鑄分司印二。又罷撈鹽提領二十員,改立提領所二,增餘鹽五百料。是年,實撈鹽一十八萬四千五百引。天歷二年,辦課鈔三十九萬五千三百九十五錠。



 四川之鹽:為場凡一十有二,為井凡九十有五,在成都、夔府、重慶、敘南、嘉定、順慶、潼川、紹慶等路萬山之間。元初,設拘搉課稅所,分撥灶戶五千九百餘隸之,從實辦課。後為鹽井廢壞,四川軍民多食解鹽。至元二年,立興元四川鹽運司,修理鹽井,仍禁解鹽不許過界。八年,罷四川茶鹽運司。十六年,復立之。十八年,並鹽課入四川道宣慰司。十九年,復立陜西四川轉運司,通辦鹽課。二十二年,改立四川鹽茶運司,分京兆運司為二,歲煎鹽一萬四百五十一引。二十六年,一萬七千一百五十二引。皇慶元年,以灶戶艱辛,減煎餘鹽五千引。天歷二年,辦鹽二萬八千九百一十引,計鈔八萬六千七百三十錠。



 遼陽之鹽:太宗丁酉年,始命北京路征收課稅所,以大鹽泊硬鹽立隨車隨引載鹽之法,每鹽一石,價銀七錢半,帶納匠人米五升。癸卯年,合懶路歲辦課白布二千匹,恤品路布一千匹。至元四年,立開元等路運司。五年,禁東京懿州乞石兒硬鹽,不許過塗河界。是年,諭各位下鹽課如例輸納。二十四年,灤州四處鹽課,舊納羊一千者,亦令如例輸鈔。延祐二年,又命食鹽人戶,歲辦課鈔,每兩率加五焉。



 兩淮之鹽:至元十三年命提舉馬里範張依宋舊例辦課,每引重三百斤,其價為中統鈔八兩。十四年,立兩淮都轉運使司,每引始改為四百斤。十六年,額辦五十八萬七千六百二十三引。十八年,增為八十萬引。二十六年,減一十五萬引。三十年,以襄陽民改食揚州鹽,又增八千二百引。大德四年,諭兩淮鹽運司設關防之法,凡鹽商經批驗所發賣者,所官收批引牙錢,其不經批驗所者,本倉就收之。八年,以灶戶艱辛,遣官究議,停煎五萬餘引。天歷二年,額辦正餘鹽九十五萬七十五引,計中統鈔二百八十五萬二百二十五錠,所隸之場凡二十有九,其工本鈔亦自四兩遞增至十兩云。



 兩浙之鹽:至元十四年,立運司,歲辦九萬二千一百四十八引。每引分作二袋,每袋依宋十八界會子,折中統鈔九兩。十八年,增至二十一萬八千五百六十二引。十九年,每引於舊價之上增鈔四貫。二十一年,置常平局,以平民間鹽價。二十三年,增歲辦為四十五萬引。二十六年,減十萬引。三十年,置局賣鹽魚鹽於海濱漁所。三十一年,並煎鹽地四十四所為三十四場。大德三年,立兩浙鹽運司檢校所四。五年,增額為四十萬引。至大元年,又增餘鹽五萬引。延祐六年,罷四檢校所,立嘉興、紹興等處鹽倉官,三十四場各場鹽運官一員,歲辦五十萬引。七年,各運司鹽課以十分為率,收白銀一分,每銀一錠,準鹽課四十錠。其工本鈔,浙西一十一場正鹽每引遞增至二十兩,餘鹽至二十五兩;浙東二十三場正鹽每引遞增至二十五兩,餘鹽至三十兩云。



 福建之鹽:至元十三年,始收其課,為鹽六千五十五引。十四年,立市舶司,兼辦鹽課。二十年,增至五萬四千二百引。二十四年,改立福建等處轉運鹽使司,歲辦鹽六萬引。二十九年,罷福建鹽運司及鹽使司,改立福建鹽課提舉司,增鹽為七萬引。大德四年,復立鹽運司。九年,又罷之,並入本道宣慰司。十年,又立鹽課都提舉司,增鹽至十萬引。至大元年,又增至十三萬引。四年,改立福建鹽運司。至順元年,實辦課三十八萬七千七百八十三錠。其工本鈔,煎鹽每引遞增至二十貫,曬鹽每引至一十七貫四錢。所隸之場有七。



 廣東之鹽:至元十三年,克廣州,因宋之舊,立提舉司,從實辦課。十六年,立江西鹽鐵茶都轉運司,所轄鹽使司六,各場立管勾。是年,辦鹽六百二十一引。二十二年,分江西鹽隸廣東宣慰司,歲辦一萬八百二十五引。二十三年,並廣東鹽司及市舶提舉司為廣東鹽課市舶提舉司,每歲辦鹽一萬一千七百二十五引。大德四年,增至正餘鹽二萬一千九百八十二引。十年,又增至三萬引。十一年,三萬五千五百引。至大元年,又增餘鹽一萬五千引。延祐二年,歲煎五萬五百引。五年,又增至五萬五百五十二引。所隸之場凡十有三。



 廣海之鹽:至元十三年,初立廣海鹽課提舉司,辦鹽二萬四千引。三十年,又立廣西石康鹽課提舉司。大德十年,增一萬一千引。至大元年,又增餘鹽一萬五千引。延祐二年,正餘鹽通為五萬一百六十五引。



 凡天下一歲總辦之數,唯天歷為可考,今並著於後:



 鹽,總二百五十六萬四千餘引。



 鹽課鈔,總七百六十六萬一千餘錠。



 ○茶法



 榷茶始於唐德宗,至宋遂為國賦,額與鹽等矣。元之茶課,由約而博,大率因宋之舊而為之制焉。



 世祖至元五年,用運使白賡言,榷成都茶,於京兆、鞏昌置局發賣,私自採賣者,其罪與私鹽法同。六年,始立西蜀四川監榷茶場使司掌之。十三年,既平宋,復用左丞呂文煥言,榷江西茶,以宋會五十貫準中統鈔一貫。十三年,定長引短引之法,以三分取一。長引每引計茶一百二十斤,收鈔五錢四分二厘八毫。短引計茶九十斤,收鈔四錢二分八毫。是歲,徵一千二百餘錠。十四年,取三分之半,增至二千三百餘錠。十五年,又增至六千六百餘錠。十七年,置榷茶都轉運司於江州,總江淮、荊湖、福廣之稅,而遂除長引,專用短引。每引收鈔二兩四錢五分,草茶每引收鈔二兩二錢四分。十八年,增額至二萬四千錠。十九年,以江南茶課官為置局,令客買引,通行貨賣。歲終,增二萬錠。二十一年,廉運使言:「各處食茶課程,抑配於民,非便。」於是革之。而以其所革之數,於正課每引增一兩五分,通為三兩五錢。二十三年,又以李起南言,增為五貫。是年征四萬錠。二十五年,改立江西等處都轉運司。二十六年,丞相桑哥增引稅為一十貫。三十年,又改江南茶法。凡管茶提舉司一十六所,罷其課少者五所,並入附近提舉司。每茶商貨茶,必令齎引,無引者與私茶同。引之外,又有茶由,以給賣零茶者。初,每由茶九斤,收鈔一兩,至是自三斤至三十斤分為十等,隨處批引局同,每引收鈔一錢



 元貞元年有獻利者言:「舊法江南茶商至江北者又稅之,其在江南賣者,亦宜更稅,如江北之制。」於是朝議復增江南課三千錠,而弗稅。是年凡徵八萬三千錠。至大元年,以龍興、瑞州為皇太后湯沐邑,其課入徽政院。四年,增額至一十七萬一千一百三十一錠。皇慶二年,更定江南茶法,又增至一十九萬二千八百六十六錠。延祐元年,改設批驗茶由局官。五年,用江西茶副法忽魯丁言,立減引添課之法,每引增稅為一十二兩五錢,通辦鈔二十五萬錠。七年,遂增至二十八萬九千二百一十一錠。



 天歷二年,始罷榷司而歸諸州縣,其歲征之數,蓋與延祐同。至順之後,無籍可考。他如範殿帥茶、西番大葉茶、建寧胯茶,亦無從知其始末,故皆不著。



 ○酒醋課



 元之有酒醋課,自太宗始。其後皆著定額,為國賦之一焉,利之所入亦厚矣。初,太宗辛卯年,立酒醋務坊場官,榷沽辦課,仍以各州府司縣長官充提點官,隸徵收課稅所,其課額驗民戶多寡定之。甲午年,頒酒曲醋貨條禁,私造者依條治罪。世祖至元十六年,以大都、河間、山東酒醋商稅等課並入鹽運司。二十二年,詔免農民醋課。是年二月,命隨路酒課依京師例,每石取一十兩。三月,用右丞盧世榮等言,罷上都醋課,其酒課亦改榷沽之制,令酒戶自具工本,官司拘賣,每石止輸鈔五兩。二十八年,詔江西酒醋之課不隸茶運司,福建酒醋之課不隸鹽運司,皆依舊令有司辦之。二十九年,丞相完澤等言:「杭州省酒課歲辦二十七萬餘錠,湖廣、龍興歲辦止九萬錠,輕重不均。」於是減杭州省十分之二,令湖廣、龍興、南京三省分辦。



 大德八年,大都酒課提舉司設槽房一百所。九年,並為三十所,每所一日所醖,不許過二十五石之上。十年,復增三所。至大三年,又增為五十四所。其制之可考者如此。若夫累朝以課程撥賜諸王公主及各寺者,凡九所云。



 天下每歲總入之數:



 酒課:



 腹裏,五萬六千二百四十三錠六十七兩一錢。



 遼陽行省,二千二百五十錠一十一兩二錢。



 河南行省,七萬五千七十七錠一十一兩五錢。



 陜西行省,一萬一千七百七十四錠三十四兩四錢。



 四川行省,七千五百九十錠二十兩。



 甘肅行省,二千七十八錠三十五兩九錢。



 雲南行省,二十萬一千一百一十七索。



 江浙行省,一十九萬六千六百五十四錠二十一兩三錢。



 江西行省,五萬八千六百四十錠一十六兩八錢。



 湖廣行省,五萬八千八百四十八錠四十九兩八錢。



 醋課:



 腹裏,三千五百七十六錠四十八兩九錢。



 遼陽行省,三十四錠二十六兩五錢。



 河南行省,二千七百四十錠三十六兩四錢。



 陜西行省,一千五百七十三錠三十九兩二錢。



 四川行省,六百一十六錠一十二兩八錢。



 江浙行省,一萬一千八百七十錠一十九兩六錢。



 江西行省,九百五十一錠二十四兩五錢。



 湖廣行省,一千二百三十一錠二十七兩九錢。



 ○商稅



 商賈之有稅,本以抑末,而國用亦資焉。元初,未有定制。太宗甲午年,始立徵收課稅所,凡倉庫院務官並合干人等,命各處官司選有產有行之人充之。其所辦課程,每月赴所輸納。有貿易借貸者,並徒二年,杖七十;所官擾民取財者,其罪亦如之。世祖中統四年,用阿合馬、王光祖等言,凡在京權勢之家為商賈,及以官銀賣買之人,並令赴務輸稅,入城不吊引者同匿稅法。至元七年,遂定三十分取一之制,以銀四萬五千錠為額,有溢額者別作增餘。是年五月,以上都商旅往來艱辛,特免其課。凡典賣田宅不納稅者,禁之。二十年,詔各路課程,差廉幹官二員提調,增羨者遷賞,虧兌者陪償降黜。凡隨路所辦,每月以其數申部,違期不申及雖申不圓者,其首領官初犯罰俸,再犯決一十七,令史加一等,三犯正官取招呈省。其院務官俸鈔,於增餘錢內給之。是年,始定上都稅課六十分取一;舊城市肆院務遷入都城者,四十分取一。二十二年,又增商稅契本,每一道為中統鈔三錢。減上都稅課,於一百兩之中取七錢半。二十六年,從丞相桑哥之請,遂大增天下商稅,腹里為二十萬錠,江南為二十五萬錠。二十九年,定諸路輸納之限,不許過四孟月十五日。三十一年,詔天下商稅有增餘者,毋作額。元貞元年,用平章剌真言,又增上都之稅。至大三年,契本一道復增作至元鈔三錢。逮至天歷之際,天下總入之數,視至元七年所定之額,蓋不啻百倍云。



 商稅額數:



 大都宣課提舉司,一十萬三千六錠一十一兩四錢。



 大都路,八千二百四十二錠九兩七錢。



 上都留守司,一千九百三十四錠五兩。



 上都稅課提舉司,一萬五百二十五錠五兩。



 興和路,七百七十錠一十七兩一錢。



 永平路,二千二百七十二錠四兩五錢。



 保定路,六千五百七錠二十三兩五錢。



 嘉定路,一萬七千四百八錠三兩九錢。



 順德路,二千五百七錠九兩九錢。



 廣平路,五千三百七錠二十兩二錢。



 彰德路,四千八百五錠四十二兩八錢。



 大名路,一萬七百九十五錠八兩八錢。



 懷慶路,四千九百四十九錠二兩。



 衛輝路,三千六百六十三錠七兩。



 河間路,一萬四百六十六錠四十七兩二錢。



 東平路,七千一百四十一錠四十八兩四錢。



 東昌路,四千八百七十九錠三十二兩。



 濟寧路,一萬二千四百三錠四兩一錢。



 曹州,六千一十七錠四十六兩三錢。



 濮州,二千六百七十一錠七錢。



 高唐州,四千二百五十九錠六兩。



 泰安州,二千一十三錠二十五兩四錢。



 冠州,七百三十八錠一十九兩七錢。



 寧海州,九百四十四錠三錢。



 德州,二千九百一十九錠四十二兩八錢。



 益都路,九千四百七十七錠一十五兩。



 濟南路,一萬二千七百五十二錠三十六兩六錢。



 般陽路,三千四百八十六錠九兩。



 大同路,八千四百三十八錠一十九兩一錢。



 冀寧路,一萬七百一十四錠三十四兩六錢。



 晉寧路,二萬一千三百五十九錠四十兩二錢。



 嶺北行省,四百四十八錠四十五兩六錢。



 遼陽行省,八千二百七十三錠四十一兩四錢。



 河南行省,一十四萬七千四百二十八錠三十二兩三錢。



 陜西行省,四萬五千五百七十九錠三十九兩二錢。



 四川行省,一萬六千六百七十六錠四兩八錢。



 甘肅行省,一萬七千三百六十一錠三十六兩一錢。



 江浙行省,二十六萬九千二十七錠三十兩三錢。



 江西行省,六萬二千五百一十二錠七兩三錢。



 湖廣行省,六萬八千八百四十四錠九兩九錢。



 ○市舶



 互市之法,自漢通南粵始,其後歷代皆嘗行之,至宋置市舶司於浙、廣之地,以通諸蕃貨易,則其制為益詳矣。



 元自世祖定江南,凡鄰海諸郡與蕃國往還互易舶貨者,其貨以十分取一,粗者十五分取一,以市舶官主之。其發舶回帆,必著其所至之地,驗其所易之物,給以公文,為之期日,大抵皆因宋舊制而為之法焉。於是至元十四年,立市舶司一於泉州,令忙古[A156]領之。立市舶司三於慶元、上海、澉浦,令福建安撫使楊發督之。每歲招集舶商,於蕃邦博易珠翠香貨等物。及次年回帆,依例抽解,然後聽其貨賣。



 時客船自泉、福販土產之物者,其所征亦與蕃貨等,上海市舶司提控王楠以為言,於是定雙抽、單抽之制。雙抽者蕃貨也,單抽者土貨也。十九年,又用耿左丞言,以鈔易銅錢,令市舶司以錢易海外金珠貨物,仍聽舶戶通販抽分。二十年,遂定抽分之法。是年十月,忙古[A156]言,舶商皆以金銀易香木,於是下令禁之,唯鐵不禁。



 二十一年,設市舶都轉運司於杭、泉二州,官自具船、給本,選人入蕃,貿易諸貨。其所獲之息,以十分為率,官取其七,所易人得其三。凡權勢之家,皆不得用己錢入蕃為賈,犯者罪之,仍籍其家產之半。其諸蕃客旅就官船賣買者,依例抽之。



 二十二年,並福建市舶司入鹽運司,改曰都轉運司,領福建漳、泉鹽貨市舶。二十三年,禁海外博易者,毋用銅錢。二十五年,又禁廣州官民,毋得運米至占城諸蕃出糶。二十九年,命市舶驗貨抽分。是年十一月,中書省定抽分之數及漏稅之法。凡商旅販泉、福等處已抽之物,於本省有市舶司之地賣者,細色於二十五分之中取一,粗色於三十分之中取一,免其輸稅。其就市舶司買者,止於賣處收稅,而不再抽。漏舶物貨,依例斷沒。三十年,又定市舶抽分雜禁,凡二十二條,條多不能盡載,擇其要者錄焉。泉州、上海、澉浦、溫州、廣東、杭州、慶元市舶司凡七所,獨泉州於抽分之外,又取三十分之一以為稅。自今諸處,悉依泉州例取之,仍以溫州市舶司並入慶元,杭州市舶司並入稅務。凡金銀銅鐵男女,並不許私販入蕃。行省行泉府司、市舶司官,每年於回帆之時,皆前期至抽解之所,以待舶船之至,先封其堵,以次抽分,違期及作弊者罪之。



 三十一年,成宗詔有司勿拘海舶,聽其自便。元貞元年,以舶船至岸,隱漏物貨者多,命就海中逆而閱之。二年,禁海商以細貨於馬八兒、唄喃、梵答剌亦納三蕃國交易,別出鈔五萬錠,令沙不丁等議規運之法。大德元年,罷行泉府司。二年,並澉浦、上海入慶元市舶提舉司,直隸中書省。是年,又置制用院,七年,以禁商下海罷之。至大元年,復立泉府院,整治市舶司事。二年,罷行泉府院,以市舶提舉司隸行省。四年,又罷之。延祐元年,復立市舶提舉司,仍禁人下蕃,官自發船貿易,回帆之日,細物十分抽二,粗物十五分抽二。七年,以下蕃之人將絲銀細物易於外國,又並提舉司罷之。至治二年,復立泉州、慶元、廣東三處提舉司,申嚴市舶之禁。三年,聽海商貿易,歸徵其稅。泰定元年,諸海舶至者,止令行省抽分。其大略如此。



 若夫中買寶貨之制,泰定三年命省臣依累朝呈獻例給價。天歷元年,以其蠹耗國財,詔加禁止,凡中獻者以違制論云。



 額外課



 元有額外課。謂之額外者,歲課皆有額,而此課不在其額中也。然國之經用,亦有賴焉。課之名凡三十有二:其一曰歷日,二曰契本,三曰河泊,四曰山場,五曰窯冶,六曰房地租,七曰門攤,八曰池塘,九曰蒲葦,十曰食羊,十一曰荻葦,十二曰煤炭,十三曰撞岸,十四曰山查,十五曰曲,十六曰魚,十七曰漆,十八曰酵,十九曰山澤,二十曰蕩,二十一曰柳,二十二曰牙例,二十三曰乳牛,二十四曰抽分,二十五曰蒲,二十六曰魚苗,二十七曰柴,二十八曰羊皮,二十九曰磁,三十曰竹葦,三十一曰姜,三十二曰白藥。其歲入之數,唯天歷元年可考云。



 歷日:總三百一十二萬三千一百八十五本,計中統鈔四萬五千九百八十錠三十二兩五錢。內腹裏,七萬二千一十本,計鈔八千五百七十錠三十一兩一錢;行省,二百五十五萬一千一百七十五本,計鈔三萬七千四百一十錠一兩四錢。大歷,二百二十萬二千二百三本,每本鈔一兩,計四萬四千四十四錠三兩。小歷,九十一萬五千七百二十五本,每本鈔一錢,計一千八百三十一錠三十二兩五錢。回回歷,五千二百五十七本,每本鈔一兩,計一百五錠七兩。



 契本:總三十萬三千八百道,每道鈔一兩五錢,計中統鈔九千一百一十四錠。內腹裏,六萬八千三百三十二道,計鈔二千四十九錠四十八兩;行省,二十三萬五千四百六十八道,計鈔七千六十四錠二兩。



 河泊課:總計鈔五萬七千六百四十三錠二十三兩四錢。內腹裏,四百六錠四十六兩二錢;行省,五萬七千二百三十六錠二十七兩一錢。



 山場課:總計鈔七百一十九錠四十九兩一錢。內腹裏,二百三十九錠一十三兩四錢;行省,四百八十錠三十五兩六錢。



 窯冶課:總計鈔九百五十六錠四十五兩九錢。內腹裏,一百九十七錠三十二兩四錢;行省,七百五十九錠一十三兩。



 房地租錢:總計鈔一萬二千五十三錠四十八兩四錢。內腹裏,九百六十六錠五兩三錢;行省,一萬一千八十七錠四十三兩一錢。



 門攤課:總計鈔二萬六千八百九十九錠一十九兩一錢。內湖廣省,二萬六千一百六十七錠三兩四錢;江西省,三百六十錠一兩五錢;河南省,三百七十二錠一十四兩一錢。



 池塘課:總計鈔一千九錠二十六兩五錢。內江浙省,二十四錠二十二兩七錢;江西省,九百八十五錠三兩八錢。



 蒲葦課:總計鈔六百八十六錠三十三兩四錢。內腹裏,一百四十一錠五兩八錢;行省,五百四十五錠二十七兩六錢。



 食羊等課:總計鈔一千七百六十錠二十九兩七錢。內大都路,四百三十八錠;上都路,三百錠;興和路,三百錠;大同路,三百九十三錠;羊市,二百二十九錠二十九兩七錢;煤木所,一百錠。



 荻葦課:總計鈔七百二十四錠六兩九錢。內河南省,六百四十四錠五兩八錢;江西省,八十錠一兩八錢。



 煤炭課:總計鈔二千六百一十五錠二十六兩四錢。內大同路,一百二十九錠一兩九錢;煤木所,二千四百九十六錠二十四兩五錢。撞岸課:總計鈔一百八十六錠三十七兩五錢。內般陽路,一百六十錠二十四兩;寧海州,二十六錠一十三兩五錢;恩州,一十三兩八錢。



 山查課:總計鈔七十五錠二十六兩四錢。內真定路一錠二十五兩八錢;廣平路,四十錠五兩一錢;大同路,三十三錠四十五兩四錢。



 曲課:江浙省鈔五十五錠三十七兩四錢。



 魚課:江浙省鈔一百四十三錠四十兩四錢。



 漆課:總計鈔一百一十二錠二十六兩。內四川省廣元路一百一十一錠二十五兩八錢。



 酵課:總計鈔二十九錠三十七兩八錢。內腹里永平路二十三錠二十五兩四錢;江西行省,六錠一十二兩五錢。



 山澤課:總計鈔二十四錠二十一兩一錢。內彰德路,一十三錠四十兩;懷慶路,一十錠三十一兩一錢。



 蕩課:平江路,八百八十六錠七錢。



 柳課:河間路,四百二錠一十四兩八錢。



 牙例課:河間路,二百八錠三十三兩八錢。



 乳牛課:真定路,二百八錠三十兩。



 抽分課:黃州路,一百四十四錠四十四兩五錢。



 蒲課:晉寧路,七十二錠。



 魚苗課:龍興路,六十五錠八兩五錢。



 柴課:安豐路,三十五錠一十一兩七錢。



 羊皮課:襄陽路,一十錠四十八兩八錢。



 磁課:冀寧路,五十八錠。



 竹葦課:奉元路,三千七百四十六錠三兩六錢。



 姜課:興元路,一百六十二錠二十七兩九錢。



 白藥課:彰德路,一十四錠二十五兩。



\end{pinyinscope}