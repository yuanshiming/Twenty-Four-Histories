\article{志第七 歷四}

\begin{pinyinscope}

 ○授時歷經下



 步中星第五



 大都北極,出地四十度太強。



 冬至,去極一百一十五度二十一分七十三秒。



 夏至,去極六十七度四十一分一十三秒。



 冬至晝,夏至夜,三千八百一十五分九十二秒。



 夏至晝,冬至夜,六千一百八十四分八秒。



 昏明,二百五十分。



 黃道出入赤道內外去極度及半晝夜分



 表略



 求每日黃道出入赤道內外去極度



 置所求日晨前夜半黃道積度,滿半歲周,去之,在象限已下,為初限;已上,復減半歲周,餘為入末限;滿積度,去之,餘以其段內外差乘之,百約之,所得,用減內外度,為出入赤道內外度;內減外加象限,即所求去極度及分秒。



 求每日半晝夜及日出入晨昏分



 置所求入初末限,滿積度,去之,餘以晝夜差乘之,百約之,所得,加減其段半晝夜分,為所求日半晝夜分;前多後少為減,前少後多為加。以半夜分便為日出分,用減日周,餘為日入分;以昏明分減日出分,餘為晨分;加日入分,為昏分。



 求晝夜刻及日出入辰刻



 置半夜分,倍之,百約,為夜刻;以減百刻,餘為晝刻;以日出入分依發斂求之,即得所求辰刻。



 求更點率



 置晨分,倍之,五約,為更率;又五約更率,為點率。



 求更點所在辰刻



 置所求更點數,以更點率乘之,加其日昏分,依發斂求之,即得所求辰刻。



 求距中度及更差度



 置半日周,以其日晨分減之,餘為距中分;以三百六十六度二十五分七十五秒乘之,如日周而一,所得,為距中度;用減一百八十三度一十二分八十七秒半,倍之,五除,為更差度及分。



 求昏明五更中星



 置距中度,以其日午中赤道日度加而命之,即昏中星所臨宿次,命為初更中星;以更差度累加之,滿赤道宿次去之,為逐更及曉中星宿度及分秒。其九服所在晝夜刻分及中星諸率,並準隨處北極出地度數推之。已上諸率,與晷漏所推自相符契。



 求九服所在漏刻



 各於所在以儀測驗,或下水漏,以定其處冬至或夏至夜刻,與五十刻相減,餘為至差刻。置所求日黃道,去赤道內外度及分,以至差刻乘之,進一位,如二百三十九而一,所得內減外加五十刻,即所求夜刻;以減百刻,餘為晝刻。其日出入辰刻及更點等率,依術求之。



 步交會第六



 交終分,二十七萬二千一百二十二分二十四秒。



 交終,二十七日二千一百二十二分二十四秒。



 交中,十三日六千六十一分一十二秒。



 交差,二日三千一百八十三分六十九秒。



 交望,十四日七千六百五十二分九十六秒半。



 交應,二十六萬一百八十七分八十六秒。



 交終,三百六十三度七十九分三十四秒。



 交中,一百八十一度八十九分六十七秒。



 正交,三百五十七度六十四分。



 中交,一百八十八度五分。



 日食陽歷限,六度。定法,六十。



 陰歷限,八度。定法,八十。



 月食限,十三度五分。定法,八十七。



 推天正經朔入交



 置中積,加交應,減閏餘,滿交終分,去之;不盡,以日周約之為日,不滿為分秒,即天正經朔入交泛日及分秒。上考者,中積內加所求閏餘,減交應,滿交終去之,不盡,以減交終,餘如上。



 求次朔望入交



 置天正經朔入交泛日及分秒,以交望累加之,滿交終日,去之,即為次朔望入交泛日及分秒。



 求定朔望及每日夜半入交



 各置入交泛日及分秒,減去經朔望小餘,即為定朔望夜半入交。若定日有增損者,亦如之。否則因經為定,大月加二日,小月加一日,餘皆加七千八百七十七分七十六秒,即次朔夜半入交;累加一日,滿交終日,去之,即每日夜半入交泛日及分秒。



 求定朔望加時入交



 置經朔望入交泛日及分秒,以定朔望加減差加減之,即定朔望加時入交日及分秒。



 求交常交定度



 置經朔望入交泛日及分秒,以月平行度乘之,為交常度;以盈縮差盈加縮減之,為交定度。



 求日月食甚定分



 日食:視定朔分在半日周已下,去減半周,為中前;已上,減去半周,為中後;與半周相減、相乘,退二位,如九十六而一,為時差;中前以減,中後以加,皆加減定朔分,為食甚定分;以中前後分各加時差,為距午定分。



 月食:視定望分在日周四分之一已下,為卯前;已上,覆減半周,為卯後;在四分之三已下,減去半周,為酉前;已上,覆減日周,為酉後。以卯酉前後分自乘,退二位,如四百七十八而一,為時差;子前以減,子後以加,皆加減定望分,為食甚定分;各依發斂求之,即食甚辰刻。



 求日月食甚入盈縮歷及日行定度



 置經朔望入盈縮歷日及分,以食甚日及定分加之,以經朔望日及分減之,即為食甚入盈縮歷;依日躔術求盈縮差,盈加縮減之,為食甚入盈縮歷定度。



 求南北差



 視日食甚入盈縮歷定度,在象限已下,為初限;已上,用減半歲周,為末限;以初末限度自相乘,如一千八百七十而一,為度,不滿,退除為分秒;用減四度四十六分,餘為南北泛差;以距午定分乘之,以半晝分除之,所得,以減泛差,為定差。泛差不及減者,反減之為定差,應加者減之,應減者加之。在盈初縮末者,交前陰歷減,陽歷加,交後陰歷加,陽歷減;在縮初盈末者,交前陰歷加,陽歷減,交後陰歷減,陽歷加。



 求東西差



 視日食甚入盈縮歷定度,與半歲周相減相乘,如一千八百七十而一,為度,不滿,退除為分秒,為東西泛差;以距午定分乘之,以日周四分之一除之,為定差。若在泛差已上者,倍泛差減之,餘為定差,依其加減。在盈中前者,交前陰歷減,陽歷加;交後陰歷加,陽歷減;中後者,交前陰歷加,陽歷減;交後陰歷減,陽歷加。在縮中前者,交前陰歷加,陽歷減;交後陰歷減,陽歷加;中後者,交前陰歷減,陽歷加;交後陰歷加,陽歷減。



 求日食正交中交限度



 置正交、中交度,以南北東西差加減之,為正交、中交限度及分秒。



 求日食入陰陽歷去交前後度



 視交定度,在中交限已下,以減中交限,為陽歷交前度;已上,減去中交限,為陰歷交後度;在正交限已下,以減正交限,為陰歷交前度;已上,減去正交限,為陽歷交後度。



 求月食入陰陽歷去交前後度



 視交定度,在交中度已下,為陽歷;已上,減去交中,為陰歷。視入陰陽歷,在後準十五度半已下,為交後度;前準一百六十六度三十九分六十八秒已上,覆減交中,餘為交前度及分。



 求日食分秒



 視去交前後度,各減陰陽歷食限,不及減者不食。餘如定法而一,各為日食之分秒。



 求月食分秒



 視去交前後度,不用南北東西差者。用減食限,不及減者不食。餘如定法而一,為月食之分秒。



 求日食定用及三限辰刻



 置日食分秒,與二十分相減、相乘,平方開之,所得,以五千七百四十乘之,如入定限行度而一,為定用分;以減食甚定分,為初虧;加食甚定分,為復圓;依發斂求之,為日食三限辰刻。



 求月食定用及三限五限辰刻



 置月食分秒,與三十分相減、相乘,平方開之;所得,以五千七百四十乘之,如入定限行度而一,為定用分;以減食甚定分,為初虧;加食甚定分,為復圓;依發斂求之,即月食三限辰刻。



 月食既者,以既內分與一十分相減、相乘,平方開之,所得,以五千七百四十乘之,如入定限行度而一,為既內分;用減定用分,為既外分;以定用分減食甚定分,為初虧;加既外,為食既;又加既內,為食甚;再加既內,為生光;復加既外,為復圓;依發斂求之,即月食五限辰刻。



 求月食入更點



 置食甚所入日晨分,倍之,五約,為更法;又五約更法,為點法。乃置初末諸分,昏分已上,減去昏分,晨分已下,加晨分,以更法除之,為更數;不滿,以點法收之,為點數;其更點數,命初更初點算外,各得所入更點。



 求日食所起



 食在陽歷,初起西南,甚於正南,復於東南;食在陰歷,初起西北,甚於正北,復於東北;食八分已上,初起正西,復於正東。此據午地而論之。



 求月食所起



 食在陽歷,初起東北,甚於正北,復於西北;食在陰歷,初起東南,甚於正南,復於西南;食八分已上,初起正東,復於正西。此亦據午地而論之。



 求日月出入帶食所見分數



 視其日日出入分,在初虧已上、食甚已下者,為帶食。各以食甚分與日出入分相減,餘為帶食差;以乘所食之分,滿定用分而一,如月食既者,以既內分減帶食差,餘進一位,如既外分而一,所得,以減既分,即月帶食出入所見之分;不及減者,為帶食既出入。以減所食分,即日月出入帶食所見之分。其食甚在晝,晨為漸進,昏為已退;其食甚在夜,晨為已退,昏為漸進。



 求日月食甚宿次



 置日月食甚入盈縮歷定度,在盈,便為定積;在縮,加半歲周,為定積。望即更加半周天度。以天正冬至加時黃道日度,加而命之,各得日月食甚宿次及分秒。



 步五星第七



 歷度



 三百六十五度二十五分七十五秒。



 歷中



 一百八十二度六十二分八十七秒半。



 歷策



 一十五度二十一分九十秒六十二微半。



 木星



 周率,三百九十八萬八千八百分。



 周日,三百九十八日八十八分。



 歷率,四千三百三十一萬二千九百六十四分八十六秒半。



 度率,一十一萬八千五百八十二分。



 合應,一百一十七萬九千七百二十六分。



 歷應,一千八百九十九萬九千四百八十一分。



 盈縮立差,二百三十六加。



 平差,二萬五千九百一十二減。



 定差,一千八十九萬七千。



 伏見,一十三度。



 表略



 火星



 周率,七百七十九萬九千二百九十分。



 周日,七百七十九日九十二分九十秒。



 歷率,六百八十六萬九千五百八十分四十三秒。



 度率,一萬八千八百七分半。



 合應,五十六萬七千五百四十五分。



 歷應,五百四十七萬二千九百三十八分。



 盈初縮末立差,一千一百三十五減。



 平差,八十三萬一千一百八十九減。



 定差,八千八百四十七萬八千四百。



 縮初盈末立差,八百五十一加。



 平差,三萬二百三十五負減。



 定差,二千九百九十七萬六千三百。



 伏見,一十九度。



 表略



 土星



 周率,三百七十八萬九百一十六分。



 周日,三百七十八日九分一十六秒。



 歷率,一億七百四十七萬八千八百四十五分六十六秒。



 度率,二十九萬四千二百五十五分。



 合應,一十七萬五千六百四十三分。



 歷應,五千二百二十四萬五百六十一分。



 盈立差,二百八十三加。



 平差,四萬一千二十二減。



 定差,一千五百一十四萬六千一百。



 縮立差,三百三十一加。



 平差,一萬五千一百二十六減。



 定差,一千一百一萬七千五百。



 伏見,一十八度。



 表略



 金星



 周率,五百八十三萬九千二十六分。



 周日,五百八十三日九十分二十六秒。



 歷率,三百六十五萬二千五百七十五分。



 度率,一萬。



 合應,五百七十一萬六千三百三十分。



 歷應,一十一萬九千六百三十九分。



 盈縮立差,一百四十一加。



 平差,三減。



 定差,三百五十一萬五千五百。



 伏見,一十度半。



 表略



 水星



 周率,一百一十五萬八千七百六十分。



 周日,一百一十五日八十七分六十秒。



 歷率,三百六十五萬二千五百七十五分。



 度率,一萬。



 合應,七十萬四百三十七分。



 歷應,二百五萬五千一百六十一分。



 盈縮立差,一百四十一加。



 平差,二千一百六十五減。



 定差,三百八十七萬七千。



 晨伏夕見,一十六度半。



 夕伏晨見,一十九度。



 表略



 推天正冬至後五星平合及諸段中積中星



 置中積,加合應,以其星周率去之,不盡,為前合;復減周率,餘為後合;以日周約之,得其星天正冬至後平合中積中星。命為日,日中積;命為度,日中星。以段日累加中積,即諸段中積;以平度累加中星,經退則減之,即為諸段中星。



 上考者,中積內減合應,滿周率去之,不盡,便為所求後合分。



 推五星平合及諸段入歷



 各置中積,加歷應及所求後合分,滿歷率,去之;不盡,如度率而一為度,不滿,退除為分秒,即其星平合入歷度及分秒;以諸段限度累加之,即諸段入歷。上考者,中積內減歷應,滿歷率去之,不盡,反減歷率,餘加其年後合,餘同上。



 求盈縮差



 置入歷度及分秒,在歷中已下,為盈;已上,減去歷中,餘為縮。視盈縮歷,在九十一度三十一分四十三秒太已下,為初限;已上,用減歷中,餘為末限。



 其火星,盈歷在六十度八十七分六十二秒半已下,為初限;已上,用減歷中,餘為末限。



 置各星立差,以初末限乘之,去加減平差,得,又以初末限乘之,去加減定差,再以初末限乘之,滿億為度,不滿退除為分秒,即所求盈縮差。



 又術:置盈縮歷,以歷策除之,為策數,不盡為策餘;以其下損益率乘之,歷策除之,所得,益加損減其下盈縮積,亦為所求盈縮差。



 求平合諸段定積



 各置其星其段中積,以其盈縮差盈加縮減之,即其段定積日及分秒;以天正冬至日分加之,滿紀法去之,不滿,命甲子算外,即得日辰。



 求平合及諸段所在月日



 各置其段定積,以天正閏日及分加之,滿朔策,除之為月數,不盡,為入月已來日數及分秒。其月數,命天正十一月算外,即其段入月經朔日數及分秒;以日辰相距,為所在定朔月日。



 求平合及諸段加時定星



 各置其段中星,以盈縮差盈加縮減之,金星倍之,水星三之。即諸段定星;以天正冬至加時黃道日度加而命之,即其星其段加時所在宿度及分秒。



 求諸段初日晨前夜半定星



 各以其段初行率,乘其段加時分,百約之,乃順減退加其日加時定星,即其段初日晨前夜半定星;加命如前,即得所求。



 求諸段日率度率



 各以其段日辰距後段日辰為日率,以其段夜半宿次與後段夜半宿次相減,餘為度率。



 求諸段平行分



 各置其段度率,以其段日率除之,即其段平行度及分秒。



 求諸段增減差及日差



 以本段前後平行分相減,為其段泛差;倍而退位,為增減差;以加減其段平行分,為初末日行分。前多後少者,加為初,減為末;前少後多者,減為初,加為末。倍增減差,為總差;以日率減一,除之,為日差。



 求前後伏遲退段增減差



 前伏者,置後段初日行分,加其日差之半,為末日行分。



 後伏者,置前段末日行分,加其日差之半,為初日行分;以減伏段平行分,餘為增減差。



 前遲者,置前段末日行分,倍其日差,減之,為初日行分。



 後遲者,置後段初日行分,倍其日差,減之,為末日行分;以遲段平行分減之,餘為增減差。前後近留之遲段。



 木火土三星,退行者,六因平行分,退一位,為增減差。



 金星,前後退伏者,三因平行分,半而退位,為增減差。



 前退者,置後段初日行分,以其日差減之,為末日行分。



 後退者,置前段末日行分,以其日差減之,為初日行分;乃以本段平行分減之,餘為增減差。



 水星,退行者,半平行分,為增減差;皆以增減差加減平行分,為初末日行分。前多後少者,加為初,減為末;前少後多者,減為初,加為末。又倍增減差,為總差;以日率減一,除之,為日差。



 求每日晨前夜半星行宿次



 各置其段初日行分,以日差累損益之,後少則損之,後多則益之,為每日行度及分秒;乃順加退減,滿宿次去之,即每日晨前夜半星行宿次。



 求五星平合見伏入盈縮歷



 置其星其段定積日及分秒,若滿歲周日及分秒,去之,餘在次年天正冬至後。



 如在半歲周已下,為入盈歷;滿半歲周,去之,為入縮歷;各在初限已下,為初限;已上,反減半歲周,餘為末限;即得五星平合見伏入盈縮歷日及分秒。



 求五星平合見伏行差



 各以其星其段初日星行分,與其段初日太陽行分相減,餘為行差。若金、水二星退行在退合者,以其段初日星行分,並其段初日太陽行分,為行差;內水星夕伏晨見者,直以其段初日太陽行分為行差。



 求五星定合定見定伏泛積



 木火土三星,以平合晨見夕伏定積日,便為定合伏見泛積日及分秒。



 金水二星,置其段盈縮差度及分秒,水星倍之。各以其段行差除之,為日,不滿,退除為分秒。在平合夕見晨伏者,盈減縮加;在退合夕伏晨見者,盈加縮減;各以加減定積為定合伏見泛積日及分秒。



 求五星定合定積定星



 木火土三星,各以平合行差除其段初日太陽盈縮積,為距合差日;不滿,退除為分秒,以太陽盈縮積減之,為距合差度。各置其星定合泛積,以距合差日盈減縮加之,為其星定合定積日及分秒;以距合差度盈減縮加之,為其星定合定星度及分秒。



 金水二星,順合退合者,各以平合退合行差,除其日太陽盈縮積,為距合差日;不滿,退除為分秒,順加退減太陽盈縮積,為距合差度。順合者,盈加縮減其星定合泛積,為其星定合定積日及分秒;退合者,以距合差日盈減縮加、距合差度盈加縮減其星退定合泛積,為其星退定合定積日及分秒;命之,為退定合定星度及分秒。以天正冬至日及分秒,加其星定合定積日及分秒,滿旬周,去之,命甲子算外,即得定合日辰及分秒。以天正冬至加時黃道日度及分秒,加其星定合定星度及分秒,滿黃道宿次,去之,即得定合所躔黃道宿度及分秒。徑求五星合伏定日:木、火、土三星,以夜半黃道日度,減其星夜半黃道宿次,餘在其日太陽行分已下,為其日伏合;金、水二星,以其星夜半黃道宿次,減夜半黃道日度,餘在其日金、水二星行分已下者,為其日伏合。金、水二星伏退合者,視其日太陽夜半黃道宿次,未行到金、水二星宿次,又視次日太陽行過金、水二星宿次,金、水二星退行過太陽宿次,為其日定合伏退定日。



 求木火土三星定見伏定積日



 各置其星定見定伏泛積日及分秒,晨加夕減九十一日三十一分六秒,如在半歲周已下,自相乘,已上,反減歲周,餘亦自相乘,滿七十五,除之為分,滿百為度,不滿,退除為秒;以其星見伏度乘之,一十五除之;所得,以其段行差除之,為日,不滿,退除為分秒;見加伏減泛積,為其星定見伏定積日及分秒;加命如前,即得定見定伏日辰及分秒。



 求金水二星定見伏定積日



 各以伏見日行差,除其段初日太陽盈縮積,為日,不滿,退除為分秒;若夕見晨伏,盈加縮減;如晨見夕伏,盈減縮加;以加減其星定見定伏泛積日及分秒,為常積。如在半歲周已下,為冬至後;已上,去之,餘為夏至後。各在九十一日三十一分六秒已下,自相乘,已上,反減半歲周,亦自相乘。冬至後晨,夏至後夕,一十八而一,為分;冬至後夕,夏至後晨,七十五而一,為分;又以其星見伏度乘之,一十五除之;所得,滿行差,除之,為日,不滿,退除為分秒,加減常積,為定積。在晨見夕伏者,冬至後加之,夏至後減之;夕見晨伏者,冬至後減之,夏至後加之;為其星定見定伏定積日及分秒;加命如前,即得定見定伏日晨及分秒。



\end{pinyinscope}