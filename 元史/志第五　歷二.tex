\article{志第五 歷二}

\begin{pinyinscope}

 ○授時歷議下



 交食



 歷法疏密,驗在交食,然推步之術難得其密,加時有早晚,食分有淺深,取其密合,不容偶然。推演加時,必本於躔離朓朒;考求食分,必本於距交遠近;茍入氣盈縮、入轉遲疾未得其正,則合朔不失之先,必失之後。合朔失之先後,則虧食時刻,其能密乎?日月俱東行,而日遲月疾,月追及日,是為一會。交值之道,有陽歷陰歷;交會之期,有中前中後;加以地形南北東西之不同,人目高下邪直之各異,此食分多寡,理不得一者也。今合朔既正,則加時無早晚之差;氣刻適中,則食分無強弱之失;推而上之,自《詩》、《書》、《春秋》及三國以來所載虧食,無不合焉者。合於既往,則行之悠久,自可無弊矣。



 《詩》、《書》所載日食二事



 《書·胤征》:「惟仲康肇位四海。乃季秋月朔,辰弗集於房。」



 今按《大衍歷》作仲康即位之五年癸巳,距辛巳三千四百八年,九月庚戌朔,泛交二十六日五千四百二十一分入食限。



 《詩·小雅·十月之交》,大夫刺幽王也。「十月之交,朔日辛卯,日有食之,亦孔之醜。」



 今按梁太史令虞絪云:十月辛卯朔,在幽王六年乙丑朔。《大衍》亦以為然。以《授時歷》推之,是歲十月辛卯朔,泛交十四日五千七百九分入食限。



 《春秋》日食三十七事



 隱公三年辛酉歲,春王二月己巳,日有食之。



 杜預云:「不書朔,史官失之。」《公羊》云:「日食或言朔或不言朔,或日或不日,或失之前或失之後,失之前者朔在前也,失之後者朔在後也。」《穀梁》云:「言日不言朔,食晦日也。」姜岌校《春秋》日食云:「是歲二月己亥朔,無己巳,似失一閏。



 三月己巳朔,去交分入食限。」《大衍》與姜岌合。今《授時歷》推之,是歲三月己巳朔,加時在晝,去交分二十六日六千六百三十一入食限。



 桓公三年壬申歲,七月壬辰朔,日有食之。



 姜岌以為是歲七月癸亥朔,無壬辰,亦失閏。其八月壬辰朔,去交分入食限。《大衍》與姜岌合。以今歷推之,是歲八月壬辰朔,加時在晝,食六分一十四秒。



 桓公十七年丙戌歲,冬十月朔,日有食之。



 《左氏》云:「不書日,史官失之。」《大衍》推得在十一月交分入食限,失閏也。以今歷推之,是歲十一月加時在晝,交分二十六日八千五百六十入食限。



 莊公十八年乙巳歲,春王三月,日有食之。



 《穀梁》云:「不言日,不言朔,夜食也。」《大衍》推是歲五月朔,交分入食限,三月不應食。以今歷推之,是歲三月朔,不入食限。五月壬子朔,加時在晝,交分入食限,蓋誤五為三。



 莊公二十五年壬子歲,六月辛未朔,日有食之。



 《大衍》推之,七月辛未朔,交分入食限。以今歷推之,是歲七月辛未朔,加時在晝,交分二十七日四百八十九入食限,失閏也。



 莊公二十六年癸丑歲,冬十有二月癸亥朔,日有食之。



 今歷推之,是歲十二月癸亥朔,加時在晝,交分十四日三千五百五十一入食限。



 莊公三十年丁巳歲,九月庚午朔,日有食之。



 今歷推之,是歲十月庚午朔,加時在晝,去交分十四日四千六百九十六入食限,失閏也。《大衍》同。



 僖公十二年癸酉歲,春王三月庚午朔,日有食之。



 姜氏云:「三月朔,交不應食,在誤條;其五月庚午朔,去交分入食限。」《大衍》同。今歷推之,是歲五月庚午朔,加時在晝,去交分二十六日五千一百九十二入食限,蓋五誤為三。



 僖公十五年丙子歲,夏五月,日有食之。



 《左氏》云:「不書朔與日,史官失之也。」《大衍》推四月癸丑朔,去交分入食限,差一閏。今歷推之,是歲四月癸丑朔,去交分一日一千三百一十六入食限。



 文公元年乙未歲,二月癸亥朔,日有食之。



 姜氏云:「二月甲午朔,無癸亥。三月癸亥朔,入食限。」《大衍》亦以為然。今歷推之,是歲三月癸亥朔,加時在晝,去交分二十六日五千九百十七分入食限,失閏也。



 文公十五年己酉歲,六月辛丑朔,日有食之。



 今歷推之,是歲六月辛丑朔,加時在晝,交分二十六日四千四百七十三分入食限。



 宣公八年庚申歲,秋七月甲子,日有食之。



 杜預以七月甲子晦食。姜氏云:「十月甲子朔,食。」《大衍》同。今歷推之,是歲十月甲子朔,加時在晝,食九分八十一秒,蓋十誤為七。



 宣公十年壬戌歲,夏四月丙辰,日有食之。



 今歷推之,是月丙辰朔,加時在晝,交分十四日九百六十八分入食限。



 宣公十七年己巳歲,六月癸卯,日有食之。



 姜氏云:「六月甲辰朔,不應食。」《大衍》云:「是年五月在交限,六月甲辰朔,交分已過食限,蓋誤。」今歷推之,是歲五月乙亥朔,入食限。六月甲辰朔,泛交二日已過食限,《大衍》為是。



 成公十六年丙戌歲,六月丙寅朔,日有食之。



 今歷推之,是歲六月丙寅朔,加時在晝,去交分二十六日九千八百三十五分入食限。



 成公十七年丁亥歲,十有二月丁巳朔,日有食之。



 姜氏云:「十二月戊子朔,無丁巳,似失閏。」《大衍》推十一月丁巳朔,交分入食限。今歷推之,是歲十一月丁巳朔,加時在晝,交分十四日二千八百九十七分入食限,與《大衍》同。



 襄公十四年壬寅歲,二月乙未朔,日有食之。



 今歷推之,是歲二月乙未朔,加時在晝,交分十四日一千三百九十三分入食限也。



 襄公十五年癸卯歲,秋八月丁巳朔,日有食之。



 姜氏云:「七月丁巳朔,食,失閏也。」《大衍》同。今歷推之,是歲七月丁巳朔,加時在晝,去交分二十六日三千三百九十四分入食限。



 襄公二十年戊申歲,冬十月丙辰朔,日有食之。



 今歷推之,是歲十月丙辰朔,加時在晝,交分十三日七千六百分入食限。



 襄公二十一年己酉歲,秋七月庚戌朔,日有食之。



 今歷推之,是月庚戌朔,加時在晝,交分十四日三千六百八十二分入食限。



 冬十月庚辰朔,日有食之。



 姜氏云:「比月而食,宜在誤條。」《大衍》亦以為然。今歷推之,十月已過交限,不應頻食,姜說為是。



 襄公二十三年辛亥歲,春王二月癸酉朔,日有食之。



 今歷推之,是月癸酉朔,加時在晝,交分二十六日五千七百三分入食限。



 襄公二十四年壬子歲,秋七月甲子朔,日有食之,既。



 今歷推之,是月甲子朔,加時在晝,日食九分六秒。



 八月癸巳朔,日有食之。



 《漢志》:「董仲舒以為比食又既。」《大衍》云:「不應頻食,在誤條。」今歷推之,立分不葉,不應食,《大衍》說是。



 襄公二十七年乙卯歲,冬十有二月乙亥朔,日有食之。



 姜氏云:「十一月乙亥朔,交分入限,應食。」《大衍》同。今歷推之,是歲十一月乙亥朔,加時在晝,交分初日八百二十五分入食限。



 昭公七年丙寅歲,夏四月甲辰朔,日有食之。



 今歷推之,是月甲辰朔,加時在晝,交分二十七日二百九十八分入食限。



 昭公十五年甲戌歲,六月丁巳朔,日有食之。



 《大衍》推五月丁巳朔,食,失一閏。今歷推之,是歲五月丁巳朔,加時在晝,交分十三日九千五百六十七分入食限。



 昭公十七年丙子歲,夏六月甲戌朔,日有食之。



 姜氏云:「六月乙巳朔,交分不葉,不應食,當誤。」《大衍》云:「當在九月朔,六月不應食,姜氏是也。」今歷推之,是歲九月甲戌朔,加時在晝,交分二十六日七千六百五十分入食限。



 昭公二十一年庚辰歲,七月壬午朔,日有食之。



 今歷推之,是月壬午朔,加時在晝,交分二十六日八千七百九十四分入食限。



 昭公二十二年辛巳歲,冬十有二月癸酉朔,日有食之。



 今歷推之,是月癸酉朔,交分十四日一千八百入食限。杜預以長歷推之,當為癸卯,非是。



 昭公二十四年癸未歲,夏五月乙未朔,日有食之。



 今歷推之,是月乙未朔,加時在晝,交分二十六日三千八百三十九分入食限。



 昭公三十一年庚寅歲,十有二月辛亥朔,日有食之。



 今歷推之,是月辛亥朔,加時在晝,交分二十六日六千一百二十八分入食限。



 定公五年丙申歲,春三月辛亥朔,日有食之。



 今歷推之,三月辛卯朔,加時在晝,交分十四日三百三十四分入食限。



 定公十二年癸卯歲,十一月丙寅朔,日有食之。



 今歷推之,是歲十月丙寅朔,加時在晝,交分十四日二千六百二十二分入食限,蓋失一閏。



 定公十五年丙午歲,八月庚辰朔,日有食之。



 今歷推之,是月庚辰朔,加時在晝,交分十三日七千六百八十五分入食限。



 哀公十四年庚申歲,夏五月庚申朔,日有食之。



 今歷推之,是月庚申朔,加時在晝,交分二十六日九千二百一分入食限。



 右《詩》、《書》所載日食二事,《春秋》二百四十二年間,凡三十有七事,以《授時歷》推之,惟襄公二十一年十月庚辰朔及二十四年八月癸巳朔不入食限,蓋自有歷以來,無比月而食之理。其三十五食,食皆在朔,《經》或不書日,不書朔,《公羊》、《穀梁》以為食晦,二者非;《左氏》以為史官失之者,得之。其間或差一日二日者,蓋由古歷疏闊,置閏失當之弊,姜岌、一行已有定說。孔子作書,但因時歷以書,非大義所關,故不必致詳也。



 三國以來日食



 蜀章武元年辛丑,六月戊辰晦,時加未。



 《授時歷》,食甚未五刻。



 《大明歷》,食甚未五刻。



 右皆親。二歷推戊辰皆七月朔。



 魏黃初三年壬寅,十一月庚申晦食,時加西南維。



 《授時歷》,食甚申二刻。



 《大明歷》,食甚申三刻。



 右《授時》親,《大明》次親。二歷推庚申皆十二月朔。



 梁中大通五年癸丑,四月己未朔食,在丙。



 《授時歷》,虧初午四刻。



 《大明歷》,虧初午四刻。



 右皆親。



 太清元年丁卯,正月己亥朔食,時加申。



 《授時歷》,食甚申一刻。



 《大明歷》,食甚申三刻。



 右《授時》次親,《大明》親。



 陳太建八年丙申,六月戊申朔食,於卯甲間。



 《授時歷》,食甚卯二刻。



 《大明歷》,食甚卯四刻。



 右《授時》次親,《大明》疏遠。



 唐永隆元年庚辰,十一月壬申朔食,巳四刻甚。



 《授時歷》,食甚巳七刻。



 《大明歷》,食甚巳五刻。



 右《授時》疏,《大明》親。



 開耀元年辛巳,十月丙寅朔食,巳初甚。



 《授時歷》,食甚辰正三刻。



 《大明歷》,食甚辰正一刻。



 右《授時》親,《大明》疏。



 嗣聖八年辛卯,四月壬寅朔食,卯二刻甚。



 《授時歷》,食甚寅八刻。



 《大明歷》,食甚卯初刻。



 右皆次親。



 十七年庚子,五月己酉朔食,申初甚。



 《授時歷》,食甚申初二刻。



 《大明歷》,食甚申正初刻。



 右《授時》次親,《大明》疏遠。



 十九年壬寅,九月乙丑朔食,申三刻甚。



 《授時歷》,食甚申一刻。



 《大明歷》,食甚申四刻。



 右《授時》次親,《大明》親。



 景龍元年丁未,六月丁卯朔食,午正甚。



 《授時歷》,食甚午正二刻。



 《大明歷》,食甚未初初刻。



 右《授時》次親,《大明》疏遠。



 開元九年辛酉,九月乙巳朔食,午正後三刻甚。



 《授時歷》,食甚午正一刻。



 《大明歷》,食甚午正二刻。



 右《授時》次親,《大明》親。



 宋慶歷六年丙戌,三月辛巳朔食,申正三刻復滿。



 《授時歷》,復滿申正三刻。



 《大明歷》,復滿申正一刻。



 右《授時》密合,《大明》次親。



 皇祐元年己丑,正月甲午朔食,午正甚。



 《授時歷》,食甚午初三刻。



 《大明歷》,食甚午正初刻。



 右《授時》親,《大明》密合。



 五年癸巳歲,十月丙申朔食,未一刻甚。



 《授時歷》,食甚未三刻。



 《大明歷》,食甚未初刻。



 右《授時》次親,《大明》親。



 至和元年甲午,四月甲午朔食,申正一刻甚。



 《授時歷》,食甚申正一刻。



 《大明歷》,食甚申正二刻。



 右《授時》密合,《大明》親。



 嘉祐四年己亥,正月丙申朔食,未三刻復滿。



 《授時歷》,復滿未初二刻。



 《大明歷》,復滿未初二刻。



 右皆親。



 六年辛丑,六月壬子朔食,未初虧初。



 《授時歷》,虧初未初刻。



 《大明歷》,虧初未一刻。



 右《授時》親,《大明》次親。



 治平三年丙午,九月壬子朔食,未二刻甚。



 《授時歷》,食甚未三刻。



 《大明歷》,食甚未四刻。



 右《授時》親,《大明》次親。



 熙寧二年己酉,七月乙丑朔食,辰三刻甚。



 《授時歷》,食甚辰五刻。



 《大明歷》,食甚辰四刻。



 右《授時》次親,《大明》親。



 元豐三年庚申,十一月己丑朔食,巳六刻甚。



 《授時歷》,食甚巳五刻。



 《大明歷》,食甚巳二刻。



 右《授時》親,《大明》疏遠。



 紹聖元年甲戌,三月壬申朔食,未六刻甚。



 《授時歷》,食甚未五刻。



 《大明歷》,食甚未五刻。



 右皆親。



 大觀元年丁亥,十一月壬子朔食,未二刻虧初,未八刻甚,申六刻復滿。



 《授時歷》,虧初未三刻,食甚申初刻,復滿申六刻。



 《大明歷》,虧初未初刻,食甚未七刻,復滿申五刻。



 右《授時歷》虧初、食甚皆親,復滿密合;《大明》虧初次親,食甚、復滿皆親。



 紹興三十二年壬午,正月戊辰朔食,申初虧初。



 《授時歷》,虧初申一刻。



 《大明歷》,虧初未七刻。



 右皆親。



 淳熙十年癸卯,十一月壬戌朔食,巳正二刻甚。



 《授時歷》,食甚巳正二刻。



 《大明歷》,食甚巳正一刻。



 右《授時》密合,《大明》親。



 慶元元年乙卯,三月丙戌朔食,午初二刻虧初。



 《授時歷》,虧初午初一刻。



 《大明歷》,虧初午初二刻。



 右《授時》虧初親,《大明》虧初密合。



 嘉泰二年壬戌,五月甲辰朔食,午初一刻虧初。



 《授時歷》,虧初巳正三刻。



 《大明歷》,虧初午初三刻。



 右皆親。



 嘉定九年丙子,二月甲申朔食,申正四刻甚。



 《授時歷》,食甚申正三刻。



 《大明歷》,食甚申正二刻。



 右《授時》親,《大明》次親。



 淳祐三年癸卯,三月丁丑朔食,巳初二刻甚。



 《授時歷》,食甚巳初一刻。



 《大明歷》,食甚巳初初刻。



 右《授時》親,《大明》次親。



 本朝中統元年庚申,三月戊辰朔食,申正二刻甚。



 《授時歷》,食甚申正一刻。



 《大明歷》,食甚申初三刻。



 右《授時》親,《大明》疏。



 至元十四年丁丑,十月丙辰朔食,午正初刻虧初,未初一刻食甚,未正二刻復滿。



 《授時歷》,虧初午正初刻,食甚未初一刻,復滿未正一刻。



 《大明歷》,虧初午正三刻,食甚未正一刻,復滿申初二刻。



 右《授時》虧初、食甚皆密合,復滿親;《大明》虧初疏,食甚、復滿皆疏遠。



 前代考古交食,同刻者為密合,相較一刻為親,二刻為次親,三刻為疏,四刻為疏遠。今《授時》、《大明》校古日食,上自後漢章武元年,下訖本朝,計三十五事。密合者,《授時》七,《大明》二。親者,《授時》十有七,《大明》十有六。次親者,《授時》十,《大明》八。疏者,《授時》一,《大明》三。疏遠者,《授時》無,《大明》六。



 前代月食



 宋元嘉十一年甲戌,七月丙子望食,四更二唱虧初,四更四唱食既。



 《授時歷》,虧初四更三點,食既在四更四點。



 《大明歷》,虧初在四更二點,食既在四更五點。



 右《授時》虧初親,食既密合;《大明》虧初密合,食既親。



 十三年丙子,十二月癸巳望食,一更三唱食既。



 《授時歷》,食既在一更三點。



 《大明歷》,食既在一更四點。



 右《授時》密合,《大明》親。



 十四年丁丑,十一月丁亥望食,二更四唱虧初,三更一唱食既。



 《授時歷》,虧初在二更五點,食既在三更二點。



 《大明歷》,虧初在二更四點,食既在三更二點。



 右《授時》虧初、食既皆親;《大明》虧初密合,食既親。



 梁中大通二年庚戌,五月庚寅望月食,在子。



 《授時歷》,食甚在子正初刻。



 《大明歷》,食甚在子正初刻。



 右皆密合。



 大同九年癸亥,三月乙巳望食,三更三唱虧初。



 《授時歷》,虧初三更一點。



 《大明歷》,虧初三更三點。



 右《授時》次親,《大明》密合。



 隋開皇十二年壬子,七月己未望食,一更三唱虧初。



 《授時歷》,虧初在一更四點。



 《大明歷》,虧初在一更五點。



 右《授時》親,《大明》次親。



 十五年乙卯,十一月庚午望食,一更四點虧初,二更三點食甚,三更一點復滿。



 《授時歷》,虧初在一更三點,食甚在二更二點,復滿在二更五點。



 《大明歷》,虧初在一更五點,食甚在二更三點,復滿在二更五點。



 右《授時》虧初、食甚、復滿皆親;《大明》虧初、復滿皆親,食甚密合。



 十六年丙辰,十一月甲子望食,四更三籌復滿。



 《授時歷》,復滿在四更四點。



 《大明歷》,復滿在四更五點。



 右《授時》親,《大明》次親。



 後漢天福十二年丁未,十二月乙未望食,四更四點虧初。



 《授時歷》,虧初四更五點。



 《大明歷》,虧初四更一點。



 右《授時》親,《大明》次親。



 宋皇祐四年壬辰,十一月丙辰望食,寅四刻虧初。



 《授時歷》,虧初在寅二刻。



 《大明歷》,虧初在寅一刻。



 右《授時》次親,《大明》疏。



 嘉祐八年癸卯,十月癸未望食,卯七刻甚。



 《授時歷》,食甚在辰初刻。



 《大明歷》,食甚在辰初刻。



 右皆親。



 熙寧二年己酉,閏十一月丁未望食,亥六刻虧初,子五刻食甚,醜四刻復滿。



 《授時歷》,虧初在亥六刻,食甚在子五刻,復滿在丑三刻。



 《大明歷》,虧初在子初刻,食甚在子六刻,復滿在丑四刻。



 右《授時》虧初、食甚密合,復滿親;《大明》虧初次親,食甚親,復滿密合。



 四年辛亥,十一月丙申望食,卯二刻虧初,卯六刻甚。



 《授時歷》,虧初在卯初刻,食甚在卯五刻。



 《大明歷》,虧初在卯四刻,食甚在卯七刻。



 右虧初皆次親,食甚皆親。



 六年癸丑,三月戊午望食,亥一刻虧初,亥六刻甚,子四刻復滿。



 《授時歷》,虧初在戌七刻,食甚在亥五刻,復滿在子三刻。



 《大明歷》,虧初在亥二刻,食甚在亥七刻,復滿在子四刻。



 右《授時》虧初次親,食甚、復滿皆親;《大明》虧初、食甚皆親,復滿密合。



 七年甲寅,九月己酉望食,四更五點虧初,五更三點食既。



 《授時歷》,虧初在四更五點,食既在五更三點。



 《大明歷》,虧初在四更三點,食既在五更二點。



 右《授時》虧初、食既皆密合;《大明》虧初次親,食既親。



 崇寧四年乙酉,十二月戊寅望食,酉三刻甚,戌初刻復滿。



 《授時歷》,食甚在酉一刻,復滿在酉七刻。



 《大明歷》,食甚在酉三刻,復滿在戌二刻。



 右《授時》食甚、復滿皆次親;《大明》食甚密合,復滿次親。



 本朝至元七年庚午,三月乙卯望食,丑三刻虧初,寅初刻食甚,寅六刻復滿。



 《授時歷》,虧初在丑二刻,食甚在寅初刻,復滿在寅六刻。



 《大明歷》,虧初在丑四刻,食甚在寅一刻,復滿在寅七刻。



 右《授時》虧初親,食甚、復滿密合;《大明》虧初、食甚、復滿皆親。



 九年壬申,七月辛未望食,丑初刻虧初,丑六刻食甚,寅三刻復滿。



 《授時歷》,虧初在子七刻,食甚在丑四刻,復滿在寅一刻。



 《大明歷》,虧初在丑二刻,食甚在丑六刻,復滿在寅二刻。



 右《授時》虧初親,食甚、復滿皆次親;《大明》虧初次親,食甚密合,復滿親。



 十四年丁丑,四月癸酉望食,子六刻虧初,丑三刻食既,醜五刻甚,醜七刻生光,寅四刻復滿。



 《授時歷》,虧初在子六刻,食既在丑四刻,食甚在醜五刻,生光丑六刻,復滿寅四刻。



 《大明歷》,虧初在丑初刻,食既醜七刻,食甚在丑七刻,生光在丑八刻,復滿寅六刻。



 右《授時》虧初、食甚、復滿皆密合,食既、生光皆親;《大明》虧初、食甚、復滿皆次親,食既疏遠,生光親。



 十六年己卯,二月癸酉望食,子五刻虧初,丑二刻甚,醜七刻復滿。



 《授時歷》,虧初在子五刻,食甚在丑二刻,復滿在丑七刻。



 《大明歷》,虧初在子七刻,食甚在丑三刻,復滿在丑七刻。



 右《授時》虧初、食甚、復滿皆密合;《大明》虧初次親,食甚親,復滿密合。



 八月己丑望食,醜五刻虧初,寅初刻甚,寅四刻復滿。



 《授時歷》,虧初在丑三刻,食甚在寅初刻,復滿在寅四刻。



 《大明歷》,虧初在丑七刻,食甚在寅二刻,復滿在寅四刻。



 右《授時》虧初次親,食甚、復滿皆密合;《大明》虧初、食甚皆次親,復滿密合。



 十七年庚辰,八月甲申望食,在晝,戌一刻復滿。



 《授時歷》,復滿在戌一刻。



 《大明歷》,復滿在戌四刻。



 右《授時》密合,《大明》疏。



 已上四十五事,密合者,《授時》十有八,《大明》十有一;親者,《授時》十有八,《大明》十有七;次親者,《授時》九,《大明》十有四;疏者,《授時》無,《大明》二;疏遠者,《授時》無,《大明》一。



 定朔



 日平行一度,月平行十三度十九分度之七,一晝夜之間,月先日十二度有奇,歷二十九日五十三刻,復追及日,與之同度,是謂經朔。經朔雲者,謂合朔大量不出此也。日有盈縮,月有遲疾,以盈縮遲疾之數損益之,始為定朔。



 古人立法,簡而未密,初用平朔,一大一小,故日食有在朔二,月食有在望前後者。漢張衡以月行遲疾,分為九道;宋何承天以日行盈縮,推定小餘;故月有三大二小。隋劉孝孫、劉焯欲遵用其法,時議排抵,以為迂怪,卒不能行。唐傅仁均始採用之,至貞觀十九年九月後,四月頻大,復用平朔。訖麟德元年,始用李淳風《甲子元歷》,定朔之法遂行。淳風又以晦月頻見,故立進朔之法,謂朔日小餘在日法四分之三已上者,虛進一日,後代皆循用之。然虞絪嘗曰:「朔在會同,茍躔次既合,何疑於頻大;日月相離,何拘於間小。」一行亦曰:「天事誠密,雖四大三小,庸何傷。」今但取辰集時刻所在之日以為定朔,朔雖小餘在進限,亦不之進,甚矣,人之安於故習也。



 初歷法用平朔,止知一大一小,為法之不可易,初聞三大二小之說,皆不以為然。自有歷以來,下訖麟德,而定朔始行,四大三小,理數自然,唐人弗克若天,而止用平朔。迨本朝至元,而常議方革。至如進朔之意,止欲避晦日月見,殊不思合朔在酉戌亥,距前日之卯十八九辰矣,若進一日,則晦不見月,此論誠然。茍合朔在辰申之間,法不當進,距前日之卯已逾十四五度,則月見於晦,庸得免乎?且月之隱見,本天道之自然,朔之進退,出入為之牽強,孰若廢人用天,不復虛進,為得其實哉。至理所在,奚恤乎人言,可為知者道也。



 不用積年日法



 歷法之作,所以步日月之躔離,候氣朔之盈虛,不揆其端,無以測知天道,而與之吻合;然日月之行遲速不同,氣朔之運參差不一,昔人立法,必推求往古生數之始,謂之演紀上元。當斯之際,日月五星同度,如合璧連珠然。惟其世代綿遠,馴積其數至逾億萬,後人厭其布算繁多,互相推考,斷截其數而增損日法,以為得改憲之術,此歷代積年日法所以不能相同者也。然行之未遠,浸復差失,蓋天道自然,豈人為附會所能茍合哉?夫七政運行於天,進退自有常度,茍原始要終,候驗周匝,則象數昭著,有不容隱者,又何必舍目前簡易之法,而求億萬年宏闊之術哉?



 今《授時歷》以至元辛巳為元,所用之數,一本諸天,秒而分,分而刻,刻而日,皆以百為率,比之他歷積年日法,推演附會,出於人為者,為得自然。



 或曰:「昔人謂建歷之本,必先立元,元正然後定日法,法定然後度周天以定分至,然則歷之有積年日法尚矣。自黃帝以來,諸歷轉相祖述,殆七八十家,未聞舍此而能成者。今一切削去,無乃昧於本原,而考求未得其方歟?」是殆不然。晉杜預有云:「治歷者,當順天以求合,非為合以驗天。」前代演積之法,不過為合驗天耳。今以舊歷頗疏,乃命厘正,法之不密,在所必更,奚暇踵故習哉。遂取漢以來諸歷積年日法及行用年數,具列於後,仍附演積數法,以釋或者之疑。



 《三統歷》西漢太初元年丁丑鄧平造,行一百八十八年,至東漢元和乙酉,後天七十八刻。



 積年,十一四萬四千五百一十一。



 日法,八十一。



 《四分歷》,東漢元和二年乙酉編造,行一百二十一年,至建安丙戌,後天七刻。



 積年,一萬五百六十一。



 日法,四。



 《乾象歷》建安十一年丙戌劉洪造,行三十一年,魏景初丁巳,後天七刻。



 積年,八千四百五十二。



 日法,一千四百五十七。



 《景初歷》魏景初元年丁巳楊偉造,行二百六年,至宋元嘉癸未,先天五十刻。



 積年,五千八十九。



 日法,四千五百五十九。



 《元嘉歷》宋元嘉二十年癸未何承天造,行二十年,至大明七年癸卯,先天五十刻。



 積年,六千五百四十一。



 日法,七百五十二。



 《大明歷》宋大明七年癸卯宋祖沖之造,行五十八年,至魏正光辛丑,後天二十九刻。



 積年,五萬二千七百五十七。



 日法,三千九百三十九。



 《正光歷》後魏正光二年辛丑李業興造,行一十九年,至興和庚申,先天十三刻。



 積年,一十六萬八千五百九。



 日法,七萬四千九百五十二。



 《興和歷》興和二年庚申李業興造,行一十年,至齊天保庚午,先天九十九刻。



 積年,二十萬四千七百三十七。



 日法,二十萬八千五百三十。



 《天保歷》北齊天保元年庚午宋景業造,行一十七年,至周天和丙戌,後天一日八十七刻。



 積年,一十一萬一千二百五十七。



 日法,二萬三千六百六十。



 《天和歷》後周天和元年丙戌甄鸞造,行一十三年,至大象己亥,先天四十刻。



 積年,八十七萬六千五百七。



 日法,二萬三千四百六十。



 《大象歷》大象元年己亥馬顯造,行五年,至隋開皇甲辰,後天十刻。



 積年,四萬二千二百五十五。



 日法,一萬二千九百九十二。



 《開皇歷》隋開皇四年甲辰張賓造,行二十四年,至大業戊辰,後天七刻。



 積年,四百一十二萬九千六百九十七。



 日法,一十萬二千九百六十。



 《大業歷》大業四年戊辰張胄玄造,行一十一年,至唐武德己卯,後天七刻。



 積年,一百四十二萬八千三百一十七。



 日法,一千一百四十四。



 《戊寅歷》唐武德二年己卯道士傅仁均造,行四十六年,至麟德乙丑,後天四十七刻。



 積年,一十六萬五千三。



 日法,一萬三千六。



 《麟德歷》麟德二年乙丑李淳風造,行六十三年,至開元戊辰,後天一十二刻。



 積年,二十七萬四百九十七。



 日法,一千三百四十。



 《大衍歷》開元十六年戊辰僧一行造,行三十四年,至寶應壬寅,先天一十三刻。



 積年,九千六百九十六萬二千二百九十七。



 日法,三千四十。



 《五紀歷》寶應元年壬寅郭獻之造,行二十三年,至貞元乙丑,後天二十四刻。



 積年,二十七萬四百九十七。



 日法,一千三百四十。



 《貞元歷》貞元元年乙丑徐承嗣造,行三十七年,至長慶壬寅,先天十五刻。



 積年,四十萬三千三百九十七。



 日法,一千九十五。



 《宣明歷》長慶二年壬寅徐昂造,行七十一年,至景福癸丑,先天四刻。



 積年,七百七萬五百九十七。



 日法,八千四百。



 《崇玄歷》景福二年癸丑邊岡造,行十四年,後六十三年,至周顯德丙辰,先天四刻。



 積年,五千三百九十四萬七千六百九十七。



 日法,一萬三千五百。



 《欽天歷》五代周顯德三年丙辰王樸造,行五年,至宋建隆庚申,先天二刻。



 積年,七千二百六十九萬八千七百七十七。



 日法,七千二百。



 《應天歷》宋建隆元年庚申王處訥造,行二十一年,至太平興國辛巳,後天二刻。



 積年,四百八十二萬五千八百七十七。



 日法,一萬單二。



 《乾元歷》太平興國六年辛巳吳昭素造,行二十年,至咸平辛丑,合。



 積年,三千五十四萬四千二百七十七。



 日法,二千九百四十。



 《儀天歷》咸平四年辛丑史序造,行二十三年,至天聖甲子,合。



 積年,七十一萬六千七百七十七。



 日法,一萬一百。



 《崇天歷》天聖二年甲子宋行古造,行四十年,至治平甲辰,後天五十四刻。



 積年,九千七百五十五萬六千五百九十七。



 日法,一萬五百九十。



 《明天歷》治平元年甲辰周琮造,行一十年,至熙寧甲寅,合。



 積年,七十一萬一千九百七十七。



 日法,三萬九千。



 《奉元歷》熙寧七年甲寅衛樸造,行十八年,至元祐壬申,後天七刻。



 積年,八千三百一十八萬五千二百七十七。



 日法,二萬三千七百。



 《觀天歷》元祐七年壬申皇居卿造,行一十一年,至崇寧癸未,先天六刻。



 積年,五百九十四萬四千九百九十七。



 日法,一萬二千三十。



 《占天歷》崇寧二年癸未姚舜輔造,行三年,至丙戌,後天四刻。



 積年,二千五百五十萬一千九百三十七。



 日法,二萬八千八十。



 《紀元歷》崇寧五年丙戌姚舜輔造,行二十一年,至金天會丁未,合。



 積年,二千八百六十一萬三千四百六十七。



 日法,七千二百九十。



 《大明歷》金天會五年丁未楊級造,行五十三年,至大定庚子,合。



 積年,三億八千三百七十六萬八千六百五十七。



 日法,五千二百三十。



 《重修大明歷》大定二十年庚子趙知微重修,行一百一年,至元朝至元辛巳,後天一十九刻。



 積年,八千八百六十三萬九千七百五十七。



 日法,五千二百三十。



 《統元歷》後宋紹興五年乙卯陳得一造,行三十二年,至乾道丁亥,合。



 積年,九千四百二十五萬一千七百三十七。



 日法,六千九百三十。



 《乾道歷》乾道三年丁亥劉孝榮造,行九年,至淳熙丙申,後天一刻。



 積年,九千一百六十四萬五千九百三十七。



 日法,三萬。



 《淳熙歷》淳熙三年丙申劉孝榮造,行一十五年,至紹熙辛亥,合。



 積年,五千二百四十二萬二千七十七。



 日法,五千六百四十。



 《會元歷》紹熙二年辛亥劉孝榮造,行八年,至慶元己未,後天一十刻。



 積年,二千五百四十九萬四千八百五十七。



 日法,三萬八千七百。



 《統天歷》慶元五年己未楊忠輔造,行八年,至開禧丁卯,先天六刻。



 積年,三千九百一十七。



 日法,一萬二千。



 《開禧歷》開禧三年丁卯鮑浣之造,行四十四年,至淳祐辛亥,後天七刻。



 積年,七百八十四萬八千二百五十七。



 日法,一萬六千九百。



 《淳祐歷》淳祐十年庚戌李德卿造,行一年,至壬子,合。



 積年,一億二千二十六萬七千六百七十七。



 日法,三千五百三十。



 《會天歷》寶祐元年癸丑譚玉造,行十八年,至咸淳辛未,後天一刻。



 積年,一千一百三十五萬六千一百五十七。



 日法,九千七百四十。



 《成天歷》咸淳七年辛未陳鼎造,行四年,至至元辛巳,後天一刻。



 積年,七千一百七十五萬八千一百五十七。



 日法,七千四百二十。



 此下不曾行用,見於典籍經進者二歷。



 《皇極歷》大業間劉焯造,阻難不行,至唐武德二年己卯,先天四十三刻。



 積年,一百萬九千五百一十七。



 日法,一千二百四十二。



 《乙未歷》大定二十年庚子耶律履造,不曾行用,至辛巳,後天一十九刻。



 積年,四千四十五萬三千一百二十六。



 日法,二萬六百九十。



 《授時歷》元至元十八年辛巳為元。



 積年、日法不用。



 實測到至元十八年辛巳歲。



 氣應,五十五日六百分。



 閏應,二十日一千八百五十分。



 經朔,三十四日八千七百五十分。



 日法,二千一百九十,演紀上元己亥,距至元辛巳九千八百二十五萬一千四百二十二算。



 氣應,五十五日六百二分。



 閏應,二十日一千八百五十三分。



 經朔,三十四日八千七百四十九分。



 日法,八千二百七十,演紀上元甲子,距辛巳五百六十七萬五百五十七算,日命甲子。



 氣應,五十五日五百三十三分。



 閏應,二十日一千八百八分。



 經朔,三十四日八千七百二十五分。



 日法,六千五百七十,寅紀上元甲子,距辛巳三千九百七十五萬二千五百三十七算。



 氣應,五十五日六百三十一分。



 閏應,二十日一千九百一十九分。



 經朔,三十四日八千七百一十二分。



\end{pinyinscope}