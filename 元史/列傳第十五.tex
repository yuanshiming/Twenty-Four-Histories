\article{列傳第十五}

\begin{pinyinscope}

 ○阿術



 阿術,兀良氏,都帥兀良合臺子也。沉幾有智略,臨陣勇決,氣蓋萬人。憲宗時,從其父征西南夷,率精兵為候騎,所向摧陷,莫敢當其鋒。至平大理,克諸部,降交趾,無不在行。事見《兀良合臺傳》。憲宗嘗勞之曰:「阿術未有名位,挺身奉國,特賜黃金三百兩,以勉將來。」世祖即位,留典宿衛。中統三年,從諸王拜出、帖哥征李璮有功。九月,自宿衛將軍拜征南都元帥,治兵於汴。復立宿州。至元元年八月,略地兩淮,攻取戰獲,軍聲大振。



 四年八月,觀兵襄陽,遂入南郡,取仙人、鐵城等柵,俘生口五萬。軍還,宋兵邀襄、樊間。阿術乃自安陽灘濟江,留精騎五千陣牛心嶺,復立虛寨,設疑火。夜半,敵果至,斬首萬餘級。初,阿術過襄陽,駐馬虎頭山,指漢東白河口曰:「若築壘於此,襄陽糧道可斷也。」五年,遂築鹿門、新城等堡,繼又築臺漢水中,與夾江堡相應,自是宋兵援襄者不能進。六年七月,大霖雨,漢水溢,宋將夏貴、範文虎相繼率兵來援,復分兵出入東岸林谷間。阿術謂諸將曰:「此張虛形,不可與戰,宜整舟師備新堡。」諸將從之。明日,宋兵果趨新堡,大破之,殺溺生擒五千餘人,獲戰船百餘艘。於是治戰船,教水軍,築圓城,以逼襄陽。文虎復率舟師來救,來興國又以兵百艘侵百丈山,前後邀擊於湍灘,俱敗走之。九年三月,破樊城外郛,增築重圍以逼之。宋裨將張順、張貴裝軍衣百船,自上流入襄陽,阿術攻之,順死,貴僅得入城。俄乘輪船順流東走,阿術與元帥劉整分泊戰船以待,燃薪照江,兩岸如晝,阿術追戰至櫃門關,擒貴,餘眾盡死。是年九月,加同平章事。先是,襄、樊兩城,漢水出其間,宋兵植木江中,聯以鐵鎖,中造浮梁,以通援兵,樊恃此為固。至是,阿術以機鋸斷木,以斧斷鎖,焚其橋,襄兵不能援。十二月,遂拔樊城。襄守將呂文煥懼而出降。十年七月,奉命略淮東。抵揚州城下,宋以千騎出戰,阿術伏兵道左,佯北,宋兵逐之,伏發,擒其騎將王都統。



 十一年正月,入覲,與參政阿里海牙奏請伐宋。帝命相臣議,久不決。阿術進曰:「臣久在行間,備見宋兵弱於往昔,失今不取,時不再來。」帝即可其奏,詔益兵十萬,與丞相伯顏、參政阿里海牙等同伐宋。三月,進平章政事。



 秋九月,師次郢之鹽山,得俘民言:「宋沿江九郡精銳,盡聚郢江東、西兩城,今舟師出其間,騎兵不得護岸,此危道也。不若取黃家灣堡,東有河口,可由其中拖船入湖,轉以下江為便。」從之,遂舍攻郢而去。行大澤中,忽宋騎兵千人突至。時從騎才數十人,阿術即奮槊馳擊。所向畏避,追斬五百餘級,生擒其將趙、範二統制。進攻沙洋、新城,拔之。前次復州,守將翟貴迎降。時夏貴鎖大艦扼江、漢口,兩岸備御堅嚴。阿術用軍將馬福計,回舟淪河口,穿湖中,從陽羅堡西沙蕪口入大江。十二月,軍至陽羅堡,攻之不克。阿術謂伯顏曰:「攻城,下策也。若分軍船之半,循岸西上,對青山磯止泊,伺隙搗虛,可以得志。」從之。明日,阿術遙見南岸沙洲,即率眾趨之,載馬後隨。宋將程鵬飛來拒,大戰中流,鵬飛敗走。諸軍抵沙洲,急擊,攀岸步斗,開而復合者數四,敵小卻,出馬於岸,遂力戰破之,追擊至鄂東門而還。夏貴聞阿術飛渡,大驚,引麾下兵三百艘先遁,餘皆潰走,遂拔陽羅堡,盡得其軍實。



 伯顏議師所向,或欲先取蘄、黃,阿術曰:「若赴下流,退無所據,上取鄂、漢,雖遲旬日,師有所依,可以萬全。」己未,水陸並趨鄂、漢,焚其船三千艘,煙焰漲天,漢陽、鄂州大恐,相繼皆降。十二年正月,黃、蘄、江州降。阿術率舟師趨安慶,範文虎迎降。繼下池州。宋丞相賈似道擁重兵拒蕪湖,遣宋京來請和。伯顏謂阿術曰:「有詔令我軍駐守,何如?」阿術曰:「若釋似道而不擊,恐已降州郡今夏難守,且宋無信,方遣使請和,而又射我軍船,執我邏騎。今日惟當進兵,事若有失,罪歸於我。」二月辛酉,師次丁家洲,遂與宋前鋒孫虎臣對陣。夏貴以戰艦二千五百艘橫亙江中,似道將兵殿其後。時已遣騎兵夾岸而進,兩岸樹砲,擊其中堅,宋軍陣動,阿術挺身登舟,手自持柂,突入敵陣,諸軍繼進,宋兵遂大潰。以上詳見《伯顏傳》。



 世祖以宋重兵皆駐揚州,臨安倚之為重,四月,命阿術分兵圍守揚州。庚申,次真州,敗宋兵於珠金砂,斬首二千餘級。既抵揚州,乃造樓櫓戰具於瓜洲,漕粟於真州,樹柵以斷其糧道。宋都統姜才領步騎二萬來攻柵,敵軍夾河為陣,阿術麾騎士渡河擊之,戰數合,堅不能卻。眾軍佯北,才逐之,遂奮而回擊,萬矢雨集,才軍不能支,擒其副將張林,斬首萬八千級。七月庚午,宋兩淮鎮將張世傑、孫虎臣以舟師萬艘駐焦山東,每十船為一舫,聯以鐵鎖,以示必死。阿術登石公山,望之,舳艫連接,旌旗蔽江,曰:「可燒而走也。」遂選強健善射者千人,載以巨艦,分兩翼夾射,阿術居中,合勢進擊,繼以火矢燒其蓬檣,煙焰漲天。宋兵既碇舟死戰,至是欲走不能,前軍爭赴水死,後軍散走。追至圌山,獲黃鵠白鷂船七百餘艘,自是宋人不復能軍矣。十月,詔拜中書左丞相,仍諭之曰:「淮南重地,李庭芝狡詐,須卿守之。」時諸軍進取臨安,阿術駐兵瓜洲,以絕揚州之援。伯顏所以兵不血刃而平宋者,阿術控制之力為多。



 十三年二月,夏貴舉淮西諸城來附。阿術謂諸將曰:「今宋已亡,獨庭芝未下,以外助猶多故也。若絕其聲援,塞彼糧道,尚恐東走通、泰,逃命江海。」乃柵揚之西北丁村,以扼其高郵、寶應之饋運;貯粟灣頭堡,以備捍禦;留屯新城,以逼泰州。又遣千戶伯顏察兒率甲騎三百助灣頭兵勢,且戒之曰:「庭芝水路既絕,必從陸出,宜謹備之。如丁村烽起,當首尾相應,斷其歸路。」六月甲戌,姜才知高郵米運將至,果夜出步騎五千犯丁村柵。至曉,伯顏察兒來援,所將皆阿術牙下精兵,旗幟畫雙赤月。眾軍望其塵,連呼曰:「丞相來矣!」宋軍識其旗,皆遁,才脫身走,追殺騎兵四百,步卒免者不滿百人。壬辰,李庭芝以硃煥守揚州,挾姜才東走。阿術率兵追襲,殺步卒千人,庭芝僅入泰州,遂築壘以守之。七月乙巳,硃煥以揚州降。乙卯,泰州守將孫良臣開北門納降,執李庭芝、姜才,奉命戮揚州市。揚、泰既下,阿術申嚴士卒,禁暴掠。有武衛軍校掠民二馬,即斬以徇。兩淮悉平,得府二、州二十二、軍四、縣六十七。九月辛酉,入見世祖於大明殿,陳宋俘。第功行賞,實封泰興縣二千戶。



 二十三年,受命北伐叛王昔剌木等。明年凱旋。繼又西征,至哈剌霍州,以疾卒,年五十四,追封河南王。



 ○阿里海牙



 阿里海牙,畏吾兒人也。初生,胞中剖而出。其父以為不祥,將棄之,母不忍。比長,果聰辨,有膽略。家貧,嘗躬耕,舍耒嘆曰:「大丈夫當立功朝廷,何至效細民事畎畝乎!」去,求其國書讀之,逾月,又棄去。用薦者得事世祖於潛邸。世祖即位,漸見擢用,由左右司郎中遷參議中書省事。至元二年,立諸路行中書省,進僉河南行省事。



 五年,命與元帥阿術、劉整取襄陽,又加參知政事。始,帝遣諸將,命毋攻城,但圍之,以俟其自降。乃築長圍,起萬山,包百丈、楚山,盡鹿門,以絕之。宋兵入援者皆敗去。然城中糧儲多,圍之五年,終不下。九年三月,破樊城外郛,其將復閉內城守。阿里海牙以為襄陽之有樊城,猶齒之有脣也,宜先攻樊城,樊城下,則襄陽可不攻而得。乃入奏。帝始報可。會有西域人亦思馬因獻新砲法,因以其人來軍中。十年正月,為砲攻樊,破之。先是,宋兵為浮橋以通襄陽之援,阿里海牙發水軍焚其橋,襄援不至,城乃拔。詳具《阿術傳》。



 阿里海牙既破樊,移其攻具以向襄陽。一砲中其譙樓,聲如雷霆,震城中。城中洶洶,諸將多逾城降者。劉整欲立碎其城,執文煥以快其意。阿里海牙獨不欲攻,乃身至城下,與文煥語曰:「君以孤軍城守者數年,今飛鳥路絕,主上深嘉汝忠。若降,則尊官厚祿可必得,決不殺汝也。」文煥狐疑未決。又折矢與之誓,如是者數四,文煥感而出降。遂與入朝。帝以文煥為昭勇大將軍、侍衛親軍都指揮使、襄漢大都督;阿里海牙行荊湖等路樞密院事,鎮襄陽。阿里海牙奏曰:「襄陽自昔用武之地也,今天助順而克之,宜乘勝順流長驅,宋可必平。」平章阿術亦贊其說。帝命丞相史天澤議之。天澤曰:「朝廷若遣重臣,如丞相安童、同知樞密院事伯顏者一人,都督諸軍,則四海混同,可立待也。」帝曰:「伯顏可。」乃大徵兵,拜伯顏為行中書省左丞相,阿術為平章。阿里海牙進行省右丞,賞鈔二百錠。



 十一年九月,會師襄陽,遂破郢州及沙洋、新城。十二月,師出沙蕪口。宋制置夏貴守諸隘,甚固。阿里海牙麾兵攻武磯堡,貴趨援之。阿術遂以兵西渡青山磯,宋都統程鵬飛來迎戰,敗之江中。會貴兵亦敗走廬州,宣撫硃示異孫夜遁還江陵,知鄂州張晏然以城降,鵬飛以本軍降。伯顏與諸將會鄂城下,議曰:「鄂襟山帶江,江南之要區也,且兵糧皆備。今蜀、江陵、岳、鄂皆未下,不以一大將鎮撫之,上流一動,則鄂非我有也。」乃以兵四萬遣阿里海牙戍鄂,而與阿術將大兵以東。



 阿里海牙集鄂民,宣上德惠,禁將士毋侵掠。其下恐懼,無敢取民之菜者,民大悅。遣人徇壽昌、信陽、德安諸郡,皆下。進徇江陵。十有二年春三月,與安撫高世傑兵遇巴陵,命張榮實搗其中堅,解汝楫率諸翼兵左右角之。世傑敗走,追降之於桃花灘。遂下嶽州。四月,至沙市,城不下,縱火攻之,沙市立破,宣撫硃示異孫、制置高達恐即以城降。乃入江陵,釋系囚,放戍券軍,除其徭賦及法令之繁細者。傳檄郢、歸、峽、常德、澧、隨、辰、沅、靖、復、均、房、施、荊門及諸洞,無不降者。盡奏官其所降官,以兵守峽,籍其戶口財賦來上。帝喜,大宴三日,語近臣曰:「伯顏兵東,阿里海牙以孤軍戍鄂,朕甚憂之。今荊南定,吾東兵可無後患矣。」乃親作手詔褒之,命右丞廉希憲守江陵,促阿里海牙急還鄂,且以沿江諸城新附者委之。



 阿里海牙至鄂,招潭州守臣李芾,不聽。乃移兵長沙,拔湘陰。冬十月,至潭,為書射城中以示芾,曰:「速下,以活州民,否則屠矣。」不答。乃決隍水,部分諸將,以砲攻之,破其木堡。流矢中胸,瘡甚,督戰益急,奪其城。潭人復作月城以相拒。凡攻七十日,大小數十戰。十有三年春正月,芾力屈,及轉運使鐘蜚英、都統陳義皆自殺,其將劉孝忠以城降。諸將欲屠之,阿里海牙曰:「是州生齒數百萬口,若悉殺之,非上諭伯顏以曹彬不殺意也,其屈法生之。」復發倉以食饑者。



 遣使徇郴、全、道、桂陽、永、衡、武岡、寶慶、袁、韶、南雄諸郡,其守臣皆率其民來迎,曰:「聞丞相體皇帝好生之德,毋殺虜,所過皆秋毫無犯,民今復見太平,各奉表來降。」丞相,稱阿里海牙也。奏官其降官,皆如江陵。獨宋經略使馬既守靜江不下。使總管俞全等招之,皆為所殺。會宋主以國降,降手詔遣湘山僧宗勉諭既,既復殺之。阿里海牙又為書,以天命地利人心開既,許以廣西大都督,反覆千餘言,終不聽。因入朝賀平宋,拜平章政事,使持詔如靜江諭之。十一月,前兵至嚴關,既守關弗納,破其兵,又敗都統馬應麒於小溶江,遂逼靜江。錄上所賜靜江詔以示既,既焚之,斬其使。靜江以水為固,乃築堰斷大陽、小溶二江,以遏上流,決東南埭,以涸其隍,破其城。民聞城破,即縱火焚居室,多赴水死。既及其總制黃文政、總管張虎以殘兵突圍走,執之。阿里海牙以靜江民易叛,非潭比,不重刑之,則廣西諸州不服,因悉坑之,斬既於市。分遣萬戶脫溫不花徇賓、融、柳、欽、橫、邕、慶遠,齊榮祖徇鬱林、貴、廉、象,脫鄰徇潯、容、藤、梧,皆下之。特磨王儂士貴、南丹州牧莫大秀,皆奉表求內附,奏官其降官如潭州。以兵戍靜江、昭、賀、梧、邕、融,乃還潭。



 既而宋二王稱制海中,雷、瓊、全、永與潭屬縣之民文才喻、周隆、張虎、羅飛咸起兵應之,舒、黃、蘄相繼亦起,大者眾數萬,小者不下數千。詔命討之,且略地海外。阿里海牙既定才喻等,至雷州,使人諭瓊州安撫趙與珞降,不聽。遂自航大海五百里,執與珞、冉安國、黃之紀,皆裂殺之,盡定瓊南寧、萬安、吉陽地。降八蕃羅甸蠻,以其總管龍文貌入見,置宣慰司。八蕃羅甸、臥龍、羅蕃、大龍、退蠻、盧蕃、小龍、石蕃、方蕃、洪蕃、程蕃,並置安撫以鎮之。



 十八年,奏請徙省鄂州。所定荊南、淮西、江西、海南、廣西之地,凡得州五十八,峒夷山獠不可勝計。大率以口舌降之,未嘗專事殺戮。又其取民悉定從輕賦,民所在立祠祀之。



 二十三年,入朝,加光祿大夫、湖廣行省左丞相;卒,年六十。贈開府儀同三司、上柱國,封楚國公,謚武定。至正八年,進封江陵王。



 子忽失海牙,湖廣行中書省左丞;貫只哥,江西行中書省平章政事。



 ○相威



 相威,國王速渾察之子也。性弘毅重厚,不飲酒,寡言笑。喜延士大夫,聽讀經史,諭古今治亂,至直臣盡忠、良將制勝,必為之擊節稱善。以故臨大事,決大議,言必中節。



 至元十一年,世祖命相威總速渾察元統弘吉剌等五投下兵從伐宋。由正陽取安豐,略廬,克和,攻司空山,平野人原。道安慶,渡江東下,會丞相伯顏兵於潤州,分三道並進,相威率左軍,參政董文炳為副,部署將校,申明約束。江陰、華亭、澉浦、上海悉望風款附,吏民按堵如故。進屯鹽官,伯顏已駐師臨安城下,得宋幼主降表。相威乃移兵瓜洲,與阿術兵合。臨揚州,都統姜才以兵二萬攻揚子橋,率諸將擊敗之。



 十三年夏,驛召相威。秋,入覲,大饗,賚功,授金虎符、征西都元帥,仍賜弓矢甲鞍、文錦表裏四、鈔萬貫,從者賞賜有差。時親王海都叛,命領汪總帥兵以鎮西土。



 十四年,召拜江南諸道行臺御史大夫。乃上奏曰:「陛下以臣為耳目,臣以監察御史、按察司為耳目。倘非其人,是人之耳目先自閉塞,下情何由上達。」帝嘉之,命御史臺清其選。每除目至,必集幕僚御史議其可否,不協公論者即劾去之。繼陳便民一十五事,其略曰:並行省,削冗官,鈐鎮戍,拘官船,業流民,錄故官,贓饋遺,淮浙鹽運司直隸行省,行大司農營田司並入宣慰司,理訟勿分南北,公田召佃仍減其租,革宋公吏勿容作弊。帝皆納焉。浙東盜起,浙西宣慰使昔里伯縱兵肆掠,俘及平民,乃遣御史商琥據錢唐津渡閱治之,得釋者以數千計。昔里伯遁還都,奏執還揚州治其罪。



 十六年,入覲,會左丞崔斌等言平章阿合馬不法事,有旨命相威及知樞密院博羅自開平馳驛大都共鞫之。阿合馬稱疾不出,博羅欲回,相威厲聲色曰:「奉旨按問,敢回奏耶!」令輿疾赴對,首責數事。既引伏,有旨釋免,仍喻相威曰:「朕知卿不惜顏面。」復命還南行臺。十七年,有旨命相威檢核阿里海牙、忽都帖木兒等所俘三萬二千餘口,並放為民。



 十八年,右丞範文虎、參政李庭以兵十萬航海征倭。七晝夜至竹島,與遼陽省臣兵合。欲先攻太宰府,遲疑不發。八月朔,颶風大作,士卒十喪六七。帝震怒,復命行省左丞相阿塔海征之。一時無敢諫者。相威遣使入奏曰:「倭不奉職貢,可伐而不可恕,可緩而不可急。向者師行迫期,戰船不堅,前車已覆,後當改轍。今為之計,預修戰艦,訓練士卒,耀兵揚武,使彼聞之,深自備御。遲以歲月,俟其疲怠,出其不意,乘風疾往,一舉而下,萬全之策也。」帝意始釋,遂罷其役。又陳皇太子既令中書,宜領撫軍監國之任,選正人端士,立詹事、賓客、諭德、贊善,衛翼左右,所以樹國本也。帝深然之。



 十九年,又奏阿里海牙占降民一千八百戶為奴,阿里海牙以為征討所得,有旨:「果降民也,還之有司;若征討所得,令御史臺籍其數以聞,量賜有功者。」阿里海牙又自陳其功比伯顏,當賜養老戶,御史滕魯瞻劾之,阿里海牙自辨,有旨遣使赴行臺逮問。相威曰:「為臣敢爾欺誑邪,滕御史何罪。」即馳奏,使者竟歸。



 二十年,以疾請入覲,進譯語《資治通鑒》,帝即以賜東宮經筵講讀。拜江淮行省左丞相。二十一年,啟行。四月,卒於蠡州,年四十四。訃聞,帝悼惜不已。



 子阿老瓦丁,南行臺御史大夫;孫脫歡,集賢大學士。



 ○土土哈



 土土哈,其先本武平北折連川按答罕山部族,自曲出徙居西北玉里伯里山,因以為氏,號其國曰欽察。其地去中國三萬餘里,夏夜極短,日暫沒即出。曲出生唆末納,唆末納生亦納思,世為欽察國王。



 太祖征蔑里乞,其主火都奔欽察,亦納思納之。太祖遣使諭之曰:「汝奚匿吾負箭之麋?亟以相還,不然禍且及汝。」亦納思答曰:「逃鸇之雀,叢薄猶能生之,吾顧不如草木耶?」太祖乃命將討之。亦納思已老,國中大亂,亦納思之子忽魯速蠻遣使自歸於太宗。而憲宗受命帥師,已扣其境,忽魯速蠻之子班都察舉族迎降。從征麥怯斯有功。率欽察百人從世祖征大理,伐宋,以強勇稱。嘗侍左右,掌尚方馬畜,歲時挏馬乳以進,色清而味美,號黑馬乳,因目其屬曰哈剌赤。



 土土哈,班都察之子也。中統元年,父子從世祖北征,俱以功受上賞。班都察卒,乃襲父職,備宿衛。



 宗王海都構亂,世祖以國家根本之地,命皇太子北平王率諸王鎮守之。至元十四年,諸王脫脫木、失烈吉叛,寇抄諸部,掠祖宗所御大帳以去。土土哈率兵討之,敗其將脫兒赤顏於納蘭不剌,邀諸部以還。應昌部族只兒瓦臺構亂,脫脫木引兵應之,中途遇土土哈,將戰,先獲其候騎數十,脫脫木乃引去,遂滅只兒瓦臺。追脫脫木等至禿兀剌河,三宿而後返。尋復敗之於斡歡河,奪回所掠大帳,還諸部之眾於北平。



 十五年,大軍北征,詔率欽察驍騎千人以從。追失烈吉逾金山,擒扎忽臺等以獻。又敗寬折哥等,裹瘡力戰,獲羊馬輜重甚眾。還朝,帝召至榻前,親慰勞之,賜金銀酒器及銀百兩、金幣九、歲時預宴只孫冠服全、海東白鶻一,仍賜以奪回所掠大帳,而諭之曰:「祖宗武帳,非人臣所得御,以卿能歸之,故以授卿。」嘗有旨:「欽察人為民及隸諸王者,皆別籍之以隸土土哈,戶給鈔二千貫,歲賜粟帛,選其材勇,以備禁衛。」



 十九年,授昭勇大將軍、同知太僕院事。二十年,改同知衛尉院事,兼領群牧司。請以所部哈剌赤屯田畿內,詔給霸州文安縣田四百頃,益以宋新附軍人八百,俾領其事。二十一年,賜金虎符,並賜金貂、裘帽、玉帶各一,海東青鶻一,水磑壹區,近郊田二千畝,籍河東諸路蒙古軍子弟四千六百人隸其麾下。二十二年,拜鎮國上將軍、樞密院副使。二十三年,置欽察親軍衛,遂兼都指揮使,聽以宗族將吏備官屬。



 海都兵犯金山,詔與大將朵兒朵懷共御之。二十四年,宗王乃顏叛,陰遣使來結也不干、勝剌哈,為土土哈所執,盡得其情以聞。勝剌哈設宴邀二大將,朵兒朵懷將往,土土哈以為事不可測,遂止,勝剌哈計不得行。未幾,有旨令勝剌哈入朝,將由東道進,土土哈言於北安王曰:「彼分地在東,脫有不虞,是縱虎入山林也。」乃命從西道進。既而有言也不幹叛者,眾欲先聞於朝,然後發兵。土土哈曰:「兵貴神速,若彼果叛,我軍出其不意,可即圖之;否則與約而還。」即日啟行,疾驅七晝夜,渡禿兀剌河,戰於孛怯嶺,大敗之,也不干僅以身免。世祖時親征乃顏,聞之,遣使命土土哈收其餘黨,沿河而下。遇叛王也鐵哥軍萬騎,擊走之,獲馬甚眾,並擒叛王哈兒魯等,獻俘行在所,誅之。欽察、康里之屬,自叛所來歸者,即以付土土哈,置哈剌魯萬戶府,欽察之散處安西諸王部下者,悉令統之。時成宗以皇孫撫軍於北,詔以土土哈從。追乃顏餘黨於哈剌,誅叛王兀塔海,盡降其眾。二十五年,諸王也只里為叛王火魯哈孫所攻,遣使告急。復從皇孫移師援之,敗諸兀魯灰。還至哈剌溫山,夜渡貴烈河,敗叛王哈丹,盡得遼左諸部,置東路萬戶府。世祖多其功,以也只裏女弟塔倫妻之。



 二十六年,從皇孫晉王征海都。抵杭海嶺,敵先據險,諸軍失利,惟土土哈以其軍直前鏖戰,翼晉王而出。追騎大至,乃選精銳設伏以待之,寇不敢逼。秋七月,世祖巡幸北邊,召見慰諭之,曰:「昔太祖與其臣同患難者,飲班術河之水以記功。今日之事,何愧昔人,卿其勉之。」還至京師,大宴群臣,復謂土土哈曰:「朔方人來,聞海都言:『杭海之役,使彼邊將皆如土土哈,吾屬安所置哉!』」論功行賞,帝欲先欽察之士。土土哈言:「慶賞之典,蒙古將吏宜先之。」帝曰:「爾毋飾讓,蒙古人誠居汝右,力戰豈在汝右耶?」召諸將頒賞有差。



 初,世祖既取宋,命籍建康、廬、饒租戶千為哈剌赤戶,益以俘獲千七百戶賜土土哈,仍官一子,以督其賦。二十八年,土土哈奏:「哈剌赤軍以萬數,足以備用。」詔賜珠帽、珠衣、金帶、玉帶、海東青鶻各一,復賜其部曲毳衣、縑素萬匹。於是率哈剌赤萬人北獵於漢塔海,邊寇聞之,皆引去。二十九年秋,略地金山,獲海都之戶三千餘,還至和林。有詔進取乞里吉思。三十年春,師次欠河,冰行數日,始至其境,盡收其五部之眾,屯兵守之。奏功,加龍虎衛上將軍,仍給行樞密院印。海都聞取乞里吉思,引兵至欠河,復敗之,擒其將孛羅察。



 三十一年,成宗即位,詔以邊境事重,其免會朝,遣使就賜銀五百兩、七寶金壺盤盂各一、鈔萬貫、白氈帳一、獨峰駝五。冬,召至京師,賞賚有加,別賜其麾下士鈔千二百萬貫。元貞元年春,仍出守北邊。二年秋,諸王附海都者率眾來歸,邊民驚擾,身至玉龍罕界,饋餉安集之,導諸王岳木忽等入朝。帝解御衣以賜,又賜金五十兩、銀千五百兩、鈔五萬貫、轎輿各一。



 大德元年正月,拜銀青榮祿大夫、上柱國、同知樞密院事、欽察親軍都指揮使,奉命還北邊。二月,至宣德府,卒,年六十一。贈金紫光祿大夫、司空,追封延國公,謚武毅,後加封升王。子八人,其第三子曰床兀兒。



 床兀兒初以大臣子奉詔從太師月兒魯行軍,戰於百塔山,有功,拜昭勇大將軍、左衛親軍都指揮使。大德元年,襲父職,領征北諸軍帥師逾金山,攻八鄰之地。八鄰之南有答魯忽河,其將帖良臺阻水而軍,伐木柵岸以自庇,士皆下馬跪坐,持弓矢以待我軍,矢不能及,馬不能進。床兀兒命吹銅角,舉軍大呼,聲震林野。其眾不知所為,爭起就馬。於是麾師畢渡,水勇水拍岸,木柵漂散,因奮師馳擊,追奔五十里,盡得其人馬廬帳。還次阿雷河,與海都所遣援八鄰之將孛伯軍遇。河之上有高山,孛伯陣於山上,馬不利下馳。床兀兒麾軍渡河蹙之,其馬多顛躓,急擊敗之,追奔三十餘里,孛伯僅以身免。二年,北邊諸王都哇、徹徹禿等潛師襲火兒哈禿之地。其地亦有山甚高,敵兵據之。床兀兒選勇而善步者,持挺刃四面上,奮擊,盡覆其軍。三年,入朝,成宗親解御衣賜之,慰勞優渥,拜鎮國上將軍、僉樞密院事、欽察親軍都指揮使、太僕少卿。復還邊。



 是時武宗在潛邸,領軍朔方,軍事必咨於床兀兒。及戰,床兀兒嘗為先。四年秋,叛王禿麥、斡魯思等犯邊,床兀兒迎敵於闊客之地。及其未陣,直前搏之,敵不敢支,追之,逾金山乃還。五年,海都兵越金山而南,止於鐵堅古山,因高以自保。床兀兒急引兵敗之。復與都哇相持於兀兒禿之地。床兀兒以精銳馳其陣,左右奮擊,所殺不可勝計,都哇之兵幾盡。武宗親視其戰,乃嘆曰:「何其壯耶!力戰未有如此者。」事聞,詔遣御史大夫禿只等即赤訥思之地集諸王軍將問戰勝功狀,咸稱床兀兒功第一。武宗既命尚雅忽禿楚王公主察吉兒,及使者以功簿奏,帝復出御衣遣使臨賜之。七年秋,入朝,帝親諭之曰:「卿鎮北邊,累建大功,雖以黃金周飾卿身,猶不足以盡朕意。」賜以衣帽、金珠等物甚厚,拜驃騎衛上將軍、樞密院副使、欽察親軍都指揮使、太僕少卿,仍賜其軍萬人,鈔四千萬貫。



 九年,諸王都哇、察八兒、明裏帖木兒等相聚而謀曰:「昔我太祖艱難以成帝業,奄有天下,我子孫乃弗克靖恭,以安享其成,連年構兵,以相殘殺,是自隳祖宗之業也。今撫軍鎮邊者,皆吾世祖之嫡孫,吾與誰爭哉?且前與土土哈戰既弗能勝,今與其子床兀兒戰又無功,惟天惟祖宗意可見矣。不若遣使請命罷兵,通一家之好,使吾士民老者得以養,少者得以長,傷殘疲憊者得以休息,則亦無負太祖之所望於我子孫者矣。」使至,帝許之。於是明裏帖木兒等罷兵入朝,特為置驛以通往來。十年,拜榮祿大夫、同知樞密院事,尋拜光祿大夫、知樞密院事,欽察左衛指揮、太僕少卿皆如故。



 成宗崩,武宗時在渾麻出之海上,床兀兒請急歸定大業,以副天下之望。武宗納其言,即日南還。及即位,賜以先朝所御大武帳等物,加拜平章政事,仍兼樞密、欽察左衛、太僕。還邊,復封容國公,授以銀印,賜尚服衣段及虎豹之屬。至大二年,入朝,加封句容郡王,改授金印。帝曰:「世祖征大理時所御武帳及所服珠衣,今以賜卿,其勿辭。」翌日,又以世祖所乘安輿賜之,且曰:「以卿有足疾,故賜此。」床兀兒叩頭泣涕,固辭而言曰:「世祖所御之帳,所服之衣,固非臣所敢當,而乘輿尤非所宜蒙也。貪寵過當,臣實不敢。」帝顧左右曰:「他人不知辭此。」別命有司置馬轎賜之,俾得乘至殿門下。



 仁宗即位,入朝,特授光祿大夫、平章政事、知樞密院事、欽察親軍都指揮使、左衛親軍都指揮使、太僕少卿。延祐元年,敗叛王也先不花等軍於亦忒海迷失之地,遣使入報,賜以尚服。二年,敗也先不花所遣將也不干、忽都帖木兒於赤麥乾之地。追出其境,至鐵門關,遇其大軍於扎亦兒之地,又敗之。四年,帝念其功而憫其老,召入商議中書省事,知樞密院事。大理國進象牙、金飾轎,即以賜之。每見必賜坐,每食必賜食,待以宗室親王之禮。床兀兒常曰:「老臣受朝廷之賜厚矣,吾子孫當以死報國。」至治二年卒,年六十三。後累封揚王。



 子六人:燕帖木兒,答剌罕、太師、右丞相、太平王;撒敦,左丞相;答里,襲封句容郡王。



\end{pinyinscope}