\article{志第四十六 兵一}

\begin{pinyinscope}

 兵者,先王所以威天下,而折奪奸宄、戡定禍亂者也。三代之制遠矣,漢、唐而下,其法變更不一。大抵用得其道孔子春秋末期思想家,儒學創始人。學說以「仁」為核,則兵力富,而國勢強;用失其宜,則兵力耗,而國勢弱。故兵制之得失,國勢之盛衰系焉。



 元之有國,肇基朔漠。雖其兵制簡略,然自太祖、太宗,滅夏剪金,霆轟風飛,奄有中土,兵力可謂雄勁者矣。及世祖即位,平川蜀,下荊襄,繼命大將帥師渡江,盡取南宋之地,天下遂定於一,豈非盛哉!



 考之國初,典兵之官,視兵數多寡,為爵秩崇卑,長萬夫者為萬戶,千夫者為千戶,百夫者為百戶。世祖時,頗修官制,內立五衛,以總宿衛諸軍,衛設親軍都指揮使;外則萬戶之下置總管,千戶之下置總把,百戶之下置彈壓,立樞密院以總之。遇方面有警,則置行樞密院,事已則廢,而移都鎮撫司屬行省。萬戶、千戶、百戶分上中下。萬戶佩金虎符,符趺為伏虎形,首為明珠,而有三珠、二珠、一珠之別。千戶金符,百戶銀符。萬戶、千戶死陣者,子孫襲爵,死病則降一等。總把、百戶老死,萬戶遷他官,皆不得襲。是法尋廢,後無大小,皆世其官,獨以罪去者則否。



 若夫軍士,則初有蒙古軍、探馬赤軍。蒙古軍皆國人,探馬赤軍則諸部族也。其法,家有男子,十五以上、七十以下,無眾寡盡簽為兵。十人為一牌,設牌頭,上馬則備戰鬥,下馬則屯聚牧養。孩幼稍長,又籍之,曰漸丁軍。既平中原,發民為卒,是為漢軍。或以貧富為甲乙,戶出一人,曰獨戶軍,合二三而出一人,則為正軍戶,餘為貼軍戶。或以男丁論,嘗以二十丁出一卒,至元七年十丁出一卒。或以戶論,二十戶出一卒,而限年二十以上者充。士卒之家,為富商大賈,則又取一人,曰餘丁軍,至十五年免。或取匠為軍,曰匠軍。或取諸侯將校之子弟充軍,曰質子軍,又曰禿魯華軍。是皆多事之際,一時之制。



 天下既平,嘗為軍者,定入尺籍伍符,不可更易。詐增損丁產者,覺則更籍其實,而以印印之。病死戍所者,百日外役次丁;死陣者,復一年。貧不能役,則聚而一之,曰合並;貧甚者、老無子者,落其籍。戶絕者,別以民補之。奴得縱自便者,俾為其主貼軍。其戶逃而還者,復三年,又逃者杖之,投他役者還籍。其繼得宋兵,號新附軍。又有遼東之軍、契丹軍、女直軍、高麗軍,雲南之寸白軍,福建之畬軍,則皆不出戍他方者,蓋鄉兵也。又有以技名者,曰砲軍、弩軍、水手軍。應募而集者,曰答剌罕軍。



 其名數,則有憲宗二年之籍、世祖至元八年之籍、十一年之籍,而新附軍有二十七年之籍。以兵籍系軍機重務,漢人不閱其數。雖樞密近臣職專軍旅者,惟長官一二人知之。故有國百年,而內外兵數之多寡,人莫有知之者。



 今其典籍可考者,曰兵制,曰宿衛,曰鎮戍,而馬政、屯田、站赤、弓手、急遞鋪兵、鷹房捕獵,非兵而兵者,亦以類附焉,作《兵志》。



 兵制



 太宗元年十一月,詔:「兄弟諸王諸子並眾官人等所屬去處簽軍事理,有妄分彼此者,達魯花赤並官員皆罪之。每一牌子簽軍一名,限年二十以上、三十以下者充,仍定立千戶、百戶、牌子頭。其隱匿不實及知情不首並隱藏逃役軍人者,皆處死。」



 七年七月,簽宣德、西京、平陽、太原、陜西五路人匠充軍,命各處管匠頭目,除織匠及和林建宮殿一切合干人等外,應有回回、河西、漢兒匠人,並札魯花赤及札也、種田人等,通驗丁數,每二十人出軍一名。



 八年七月,詔:「燕京路保州等處,每二十戶簽軍一名,令答不葉兒統領出軍。真定、河間、邢州、大名、太原等路,除先簽軍人外,於斷事官忽都虎新籍民戶三十七萬二千九百七十二人數內,每二十丁起軍一名,亦令屬答不葉兒領之。」



 十三年八月,諭總管萬戶劉黑馬,據斜烈奏,忽都虎等元籍諸路民戶一百萬四千六百五十六戶,除逃戶外,有七十二萬三千九百一十戶,隨路總簽軍一十萬五千四百七十一名,點數過九萬七千五百七十五人,餘因近年蝗旱,民力艱難,往往在逃。有旨,今後止驗見在民戶簽軍,仍命逃戶復業者免三年軍役。



 世祖中統元年六月,詔罷解鹽司軍一百人。初,解鹽司元籍一千鹽戶內,每十戶出軍一人,後阿藍答兒倍其役。世祖以重困其民,罷之。七月,以張榮實從南征,多立功,命為水軍萬戶兼領霸州民戶。諸水軍將吏河陰縣達魯花赤胡玉、千戶王端臣軍七百有四人,八柳樹千戶斡來軍三百六十一人,孟州龐抄兒赤、張信軍一百九十人,濱棣州海口總把張山軍一百人,滄州海口達魯花赤塔剌海軍一百人,睢州李總管麾下孟春等五十五人,霸州蕭萬戶軍一百九十五人,悉聽命焉。



 三年三月,詔:「真定、彰德、邢州、洺磁、東平、大名、平陽、太原、衛輝、懷孟等路各處,有舊屬按札兒、孛羅、笑乃鷿、闊闊不花、不里合拔都兒等官所管探馬赤軍人,乙卯歲籍為民戶,亦有簽充軍者。若壬寅、甲寅兩次簽定軍,已入籍冊者,令隨各萬戶依舊出征;其或未嘗為軍,及蒙古、漢人民戶內作數者,悉簽為軍。」六月,以軍士訴貧乏者眾,命貧富相兼應役,實有不能自存者優恤三年。十月,諭山東東路經略司:「益都路匠軍已前曾經簽把者,可遵別路之例,俾令從軍。」以鳳翔府屯田軍人準充平陽軍數,仍於鳳翔屯田,勿遣從軍。刁國器所管重簽軍九百一十五人,即日放罷為民。陜西行省言:「士卒戍金州者,諸奧魯已嘗服役,今重勞苦。」詔罷之。並罷山東、大名、河南諸路新簽防城戍卒。



 四年二月,詔:「統軍司及管軍萬戶、千戶等,可遵太祖之制,令各官以子弟入朝充禿魯花。」其制:萬戶,禿魯花一名,馬一十匹,牛二具,種田人四名。千戶見管軍五百或五百已上者,禿魯花一名,馬六匹,牛一具,種田人二名。雖所管軍不及五百,其家富強子弟健壯者,亦出禿魯花一名,馬匹、牛具、種田人同。萬戶、千戶子弟充禿魯花者,挈其妻子同至,從人不拘定數,馬匹、牛具,除定去數目已上,復增餘者聽。若有貧乏不能自備者,於本萬戶內不該出禿魯花之人,通行津濟起發,不得因而科及眾軍。萬戶、千戶或無親子、或親子幼弱未及成人者,以弟侄充,候親子年及十五,卻行交換。若委有親子,不得隱匿代替,委有氣力,不得妄稱貧乏,及雖到來,氣力卻有不完者,並罪之。是月,帝以太宗舊制,設官分職,軍民之事,各有所司。後多故之際,不暇分別,命阿海充都元帥,專於北京、東京、平灤、懿州、蓋州路管領見管軍人,凡民間之事毋得預焉。五月,立樞密院,凡蒙古、漢軍並聽樞密節制。統軍司、都元帥府,除遇邊面緊急事務就便調度外,其軍情一切大小公事,並須申覆。合設奧魯官,並從樞密院設置。七月,詔免河南保甲丁壯、射生軍三千四百四十一戶雜泛科差,專令守把巡哨。八月,諭成都路行樞密院:「近年軍人多逃亡事故者,可於各奧魯內盡實簽補,自乙卯年定入軍籍之數,悉簽起赴軍。」十一月,女直、水達達及乞烈賓地合簽鎮守軍,命亦里不花簽三千人,付塔匣來領之;並達魯花赤官之子及其餘近上戶內,亦令簽軍,聽亦里不花節制。



 至元二年八月,陜西五路西蜀四川行省言:「新簽軍七千人,若發民戶,恐致擾亂。今鞏昌已有舊軍三千,諸路軍二千,餘二千人亦不必發民戶,當以便宜起補。」從之。十一月,省院官議,收到私走間道、盜販馬匹、曾過南界人三千八百四戶,悉令充軍,以一千九百七十八人與山東路統軍司,一千人與蔡州萬戶,餘八百二十六戶,有旨留之軍中。



 三年七月,添內外巡軍,外路每百戶選中產者一人充之,其賦令餘戶代輸,在都增武衛軍四百。



 四年正月,簽蒙古軍,每戶二丁、三丁者一人,四丁、五丁者二人,六丁、七丁者三人。二月,詔遣官簽平陽、太原人戶為軍,除軍、站、僧、道、也裏可溫、答失蠻、儒人等戶外,於系官、投下民戶、運司戶、人匠、打捕鷹房、金銀鐵冶、丹粉錫碌等,不以是何戶計,驗酌中戶內丁多堪當人戶,簽軍二千人,定立百戶、牌子頭,前赴陜西五路西蜀四川行中書省所轄東川出征。復於京兆、延安兩路簽軍一千人,如平陽、太原例。五月,詔:「河南路驗酌中戶內丁多堪當軍人戶,簽軍四百二十名,歸之樞密院,俾從軍,復其徭役。南京路,除邳州、南宿州外,依中書省分間定應簽軍人戶,驗丁數,簽軍二千五百八十名,管領出征。」十二月,簽女直、水達達軍三千人。



 五年閏正月,詔益都李璮元簽軍,仍依舊數充役。二月,詔諸路奧魯毋隸總管府,別設總押所官,聽樞密院節制。六月,省臣議:「簽起禿魯花官員,皆已遷轉,或物故黜退者,於內復有貧難蒙古人氏,除隨路總管府達魯花赤、總管及掌兵萬戶,合令應當,其次官員禿魯花,宜放罷,其自願留質者聽之。」十月,禁長軍之官不得侵漁士卒,違者論罪。十一月,簽山東、河南沿邊州城民戶為軍,遇征進,則選有力之家同元守邊城漢軍一體出征,其無力之家代守邊城及屯田勾當。



 六年二月,簽懷孟、衛輝路丁多人戶充軍,益都、淄萊所轄登、萊州李鋋舊軍內,起簽一萬人,差官部領出征。其淄萊路所轄淄、萊等處有非李鋋舊管者,簽五百二十六人,其餘諸色人戶,亦令酌驗丁數,簽軍起遣,至軍前赴役。十月,從山東路統軍司言,應系逃軍未獲者,令其次親丁代役,身死軍人亦令親丁代補,無親丁則以少壯驅丁代之。



 七年三月,定軍官等級,萬戶、千戶、百戶、總把以軍士為差。六月,成都府括民三萬一千七十五戶,簽義士軍八千六十七人。七月,分揀隨路砲手軍。始太祖、太宗征討之際,於隨路取發,並攻破州縣,招收鐵木金火等人匠充砲手,管領出征,壬子年俱作砲手附籍。中統四年揀定,除正軍當役外,其餘戶與民一體當差。後為出軍正戶煩難,至元四年取元充砲手民戶津貼,其間有能與不能者,影占不便,至是分揀之。



 八年二月,以瓜州、沙州鷹房三百人充軍。



 九年正月,河南省請益兵,敕諸路簽軍三萬,詔元帥府、統軍司、總管萬戶府閱實軍籍。二月,命阿術典行省蒙古軍,劉整、阿里海牙典漢軍。四月,詔:「諸路軍戶驅丁,除至元六年前從良入民籍者當差。七年後,凡從良文書寫從便為民者,亦如之。餘雖從良,並令津助本戶軍役。」七月,閱大都、京兆等處探馬赤戶名籍。九月,詔樞密:「諸路正軍貼戶及同籍親戚僮奴,丁年堪役,依諸王權要以避役者,並還之軍,惟匠藝精巧者以名聞。」十二月,命府州司縣達魯花赤及治民長官,不妨本職,兼管諸軍奧魯。各路總管府達魯花赤、總管,別給宣命印信,府州司縣達魯花赤長官止給印信,任滿則別具解由,申樞密院。



 十年正月,合剌請於渠江之北雲門山及嘉陵西岸虎頭山立二戍,以其圖來上,仍乞益兵二萬,敕給京兆新簽軍五千人益之。陜西京兆、延安、鳳翔三路諸色人戶,約六萬戶內,簽軍六千。五月,禁乾討虜人,其願充軍者,於萬戶、千戶內結成牌甲,與大軍一體征進。八月,禁軍吏之長舉債,不得重取其息,以損軍力,違者罪之。九月,襄陽生券軍至都釋械系免死,聽自立部伍,俾征日本,仍於蒙古、漢人內選官率領之。



 十一年正月,初立軍官以功升散官格。五月,便宜總帥府言:「本路軍經今四十年間,或死或逃,無丁不能起補,見在軍少,乞選擇堪與不堪丁力,放罷貧乏無丁者,於民站內別選充役。」從之。詔延安府、沙井、凈州等處種田白達達戶,選其可充軍者,簽起出征。六月,潁州屯田總管李珣言:「近為簽軍事,乞依徐、邳州屯田例,每三丁內,一丁防城,二丁納糧,可簽丁壯七百餘人,並元撥保甲丁壯,令珣通領,鎮守潁州,代見屯納合監戰軍馬別用。」從之。



 十二年三月,遣官往遼東,簽揀蒙古達魯花赤、千戶、百戶等官子弟出軍。詔隨處所置襄陽生券軍之為農者,或自願充軍,具數以聞。五月,正陽萬戶劉復亨言:「新下江南三十餘城,俱守以兵,及江北、淮南、潤、揚等處未降,軍力分散,調度不給,以致鎮巢軍、滁州兩處復叛。乞簽河西等戶為軍,並力剿除,庶無後患。」有旨,命肅州達魯花赤,並遣使同往驗各色戶計物力富強者簽起之。六月,簽平陽、西京、延安等路達魯花赤弟男為軍。萊州酒稅官王貞等上言:「國家討平殘宋,吊伐為事,何嘗以賄利為心。彼不紹事業小人,貪圖貨利,作乾討虜名目,侵掠彼地,所得人口,悉皆貨賣,以充酒食之費,勝則無益朝廷,敗則實為辱國。其招討司所收乾討虜人,可悉罷之,第其高下,籍為正軍,命各萬戶管領征進,一則得其實用,二則正王師吊伐之名,實為便益。」從之。



 十四年正月,詔:「上都、隆興、西京、北京四路編民捕獵等戶,簽選丁壯軍二千人,防守上都。」中書省議:「從各路搭配,二十五戶內取軍一名,選善騎射者充,官給行資中統鈔一錠,仍自備鞍馬衣裝器仗,編立牌甲,差官部領,前來赴役。」十二月,樞密院臣言:「收附亡宋州城,新附請糧官軍,並通事馬軍人等,軍官不肯存恤,多逃散者,乞招誘之。」命左丞陳巖等,分揀堪當軍役者,收系充軍,依舊例月支錢糧。其生券不堪當軍者,官給牛具糧食,屯田種養。



 十五年正月,定軍官承襲之制。凡軍官之有功者升其秩,元受之職,令他有功者居之,不得令子侄復代。陣亡者始得承襲,病死者降一等。總把、百戶老病死,不在承襲之例。凡將校臨陣中傷、還營病創者,亦令與陣亡之人一體承襲。禁長軍之官不恤士卒,及士卒亡命避役,侵擾初附百姓者,俱有罪。雲南行省言:「雲南舊屯駐蒙古軍甚少,遂取漸長成丁怯困都等軍,以備出征。雲南闊遠,多未降之地,必須用兵,已簽爨、僰人一萬為軍,續取新降落落、和泥等人,亦令充軍。然其人與中原不同,若赴別地出征,必致逃匿,宜令就各所居一方未降處用之。」九月,並軍士。初,至元九年簽軍三萬,止擇精銳年壯者,不復問其貲產,且無貼戶之助,歲久多貧乏不堪。樞密院臣奏,宜縱為民,遂並為一萬五千。諸軍戶投充諸侯王怯憐口、人匠,或托為別戶以避其役者,復令為軍,有良匠則別而出之。樞密臣又言:「至元八年,於各路軍之為富商大賈者一百四十三戶,各增一軍,號餘丁軍。今東平等路諸奧魯總管府言,往往人死產乏,不能充二軍,乞免餘丁充役者。」制可。十二月,樞密院官議:「諸軍官在軍籍者,除百戶、總把權準軍役,其元帥、招討、萬戶、總管、千戶或首領官,俱合再當正軍一名。」



 十六年正月,罷五翼探馬赤重役軍。三月,括兩淮造回回砲新附軍匠六百人,及蒙古、回回、漢人、新附人能造砲者,至京師。五月,淮西道宣慰司官昂吉兒請招諭亡宋通事軍,俾屬之麾下。初,亡宋多招納北地蒙古人為通事軍,遇之甚厚,每戰皆列於前行,願效死力。及宋亡,無所歸。朝議欲編入版籍未暇也,人人疑懼,皆不自安。至是,昂吉兒請招集,列之行伍,以備征戍。從之。九月,詔河西地未簽軍之官,及富強戶有物力者,簽軍六百人。十月,壽州等處招討使李鐵哥,請召募有罪亡命之人充軍,其言:「使功不如使過。始南宋未平時,蒙古、諸色人等,因得罪皆亡命往依焉,今已平定,尚逃匿林藪。若釋其罪而用之,必能效力,無不一當十者矣。」十一月,罷太原、平陽、西京、延安路新簽軍還籍。



 十七年七月,詔江淮諸路招集答剌罕軍。初平江南,募死士願從軍者,號答剌罕,屬之劉萬戶麾下。南北既混一,復散之,其人皆無所歸,率群聚剽掠。至是,命諸路招集之,令萬奴部領如故,聽範左丞、李拔都二人節制。



 十八年二月,並貧乏軍人三萬戶為一萬五千,取帖戶津帖正軍充役。四月,置蒙古、漢人、新附軍總管。六月,樞密院議:「正軍貧乏無丁者,令富強丁多帖戶權充正軍應役,驗正軍物力,卻令津濟貼戶,其正軍仍為軍頭如故。或正軍實系單丁者,許傭雇練習之人應役,丁多者不得傭雇,軍官亦不得以親從人代之。」



 十九年二月,諸侯王阿只吉遣使言:「探馬赤軍凡九處出征,各奧魯內復徵雜泛徭役,不便。」詔免之,並詔有司毋重役軍戶。六月,禁長軍之官,毋得占役士卒。散定海答剌罕軍還各營,及歸戍城邑。十月,簽發漸丁軍士。遵舊制,家止一丁者不作數,凡二丁至五丁、六丁之家,止存一人,餘皆充軍。



 二十年二月,命各處行樞密院造新附軍籍冊。六月,從丞相伯顏議,所括宋手號軍八萬三千六百人,立牌甲,設官以統之。十月,定出征軍人亡命之罪,為首者斬,餘令減死一等。



 二十一年八月,江東道僉事馬奉訓言:「劉萬奴乾討虜軍,私相糾合,結為徒黨,張弓挾矢,或詐稱使臣,莫若散之各翼萬戶、千戶、百戶、牌甲內管領為便。」省院官以聞,有旨,可令問此軍:「欲從脫歡出征虜掠耶?欲且放散還家耶?」回奏:「眾軍皆言,自圍襄樊渡江以來,與國效力,願令還家少息。」遂從之。籍亡宋手記軍。宋時有是軍,死則以兄弟若子承代。有旨,依漢軍例籍之,毋涅其手。



 二十二年正月,立行樞密院於江南三省,其各處行省見管軍馬悉以付焉。九月,詔福建黃華畬軍,有恆產者放為民,無恆產與妻子者編為守城軍。征交趾蒙古軍五百人、漢軍二千人,除留蒙古軍百人、漢軍四百人,為鎮南王脫歡宿衛,餘悉遣還,別以江淮行樞密院蒙古軍戍江西。十月,從月的迷失言,以乾討虜軍七百人,籍名數,立牌甲,命將官之無軍者領之。十一月,御史臺臣言:「昔宋以無室家壯士為鹽軍,內附之初,有五千人,除征占城運糧死亡者,今存一千一百二十二人。此徒皆性習兇暴,民患苦之,宜給以衣糧,使屯田自贍,庶絕其擾。」從之。十二月,從樞密院請,嚴立軍籍條例,選壯士及有力之家充軍。舊例,丁力強者充軍,弱者出錢,故有正軍、貼戶之籍。行之既久,而強者弱,弱者強,籍亦如故。其同戶異居者,私立年期,以相更代,故有老稚不免從軍,而強壯家居者,至是革焉。江浙省募鹽徒為軍,得四千七百六十六人,選軍官麾下無士卒者,相參統之,以備各處鎮守。



 二十四年閏二月,樞密院臣言:「諸軍貼戶,有正軍已死者,有充工匠者,放為民者,有元系各投下戶回付者,似此歇閑一千三百四十戶,乞差人分揀貧富,定貼戶、正軍。」制可。



 二十六年八月,樞密院議:「諸管軍官萬戶、千戶、百戶等,或治軍有法、鎮守無虞、鎧仗精完、差役均平、軍無逃竄者,許所司薦舉以聞,不次擢用。諸軍吏之長,非有上司之命,毋擅離職。諸長軍者,及蒙古、漢軍,毋得妄言邊事。」



 成宗大德二年十二月,定各省提調軍馬官員。凡用隨從軍士,蒙古長官三十名,次官二十名,漢人一十名;萬戶、千戶、百戶人等,俱不得占役。行省鎮撫止用聽探外,亦不得多餘役占。



 十一年四月,詔禮店軍還屬土番宣慰司。初,西川也速迭兒、按住奴、帖木兒等所統探馬赤軍,自壬子年屬籍禮店,隸王相府,後王相府罷,屬之陜西省,桑哥奏屬土番宣慰司,咸以為不便,大德十年命依壬子之籍,至是復改屬焉。



 武宗至大元年正月,以通惠河千戶劉粲所領運糧軍九百二十人,屬萬戶赤因帖木爾兵籍。十二月,丞相三寶奴等言:「國制,行省佐貳及宣慰使不得提調軍馬,若遙授平章、揚州宣慰使阿憐帖木兒者,嘗與成宗同乳母,故得行之,非常憲也。今命沙的代之,宜遵國制,勿令提調。」制可。



 仁宗皇慶元年三月,中書省臣奏李馬哥等四百戶為民。初,李馬哥等四百戶屬諸侯王脫脫,乙未年定籍為民,高麗林衍及乃顏叛,皆嘗簽為軍。至元八年置軍籍,以李馬哥等非七十二萬戶內軍數,復改為民。至大四年,樞密院復奏為軍。至是,省官以為言,命遵乙未年已定之籍。後樞密復奏,竟以為軍戶。十二月,省臣言:「先是樞密院奏準,雲南省宜遵各省制,其省官居長者二員,得佩虎符,提調軍馬,餘佐貳者不得預,已受虎符者悉收之。今雲南省言,本省籍軍士之力,以辦集錢穀,遇有調遣,則省官親率眾上馬,故舊制雖牧民官亦得佩虎符,領軍務,視他省為不同。臣等議,已受虎符者依故事,未受者宜頒賜之。」制可。



 二年正月,詔:「雲南省鎮遠方,掌邊務,凡事涉軍旅者,自平章至僚佐須同署押,其長官二員,復與哈必赤。」



 延祐元年二月,四川省軍官闕員,詔:「依民官遷調之制,差人與本省提調官及監察御史同銓注。」



 三年三月,命伯顏都萬戶府及紅胖襖總帥府各調軍九千五百人,往諸侯王所,更代守邊士卒。其屬都萬戶府者,軍一名,馬三匹;屬總帥府者,軍一名,馬二匹。令人自為計,其貧不能自備者,則命行伍之長及百戶、千戶等助之。悉遣精銳練習騎射之士。每軍一百名,百戶一員;五百名,千戶一員。復命買住、囊加鷿二人分左右部領之。



\end{pinyinscope}