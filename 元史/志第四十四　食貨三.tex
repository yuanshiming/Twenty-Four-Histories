\article{志第四十四 食貨三}

\begin{pinyinscope}

 ○歲賜



 自昔帝王於其宗族姻戚必致其厚者,所以明親親之義也。元之為制,其又厚之至者歟!凡諸王及后妃公主,皆有食採分地。其路府州縣得薦其私人以為監,秩祿受命如王官,而不得以歲月通選調。其賦則五戶出絲一斤,不得私征之,皆輸諸有司之府,視所當得之數而給與之。其歲賜則銀幣各有差,始定於太宗之時,而增於憲宗之日。及世祖平江南,又各益以民戶。時科差未定,每戶折支中統鈔五錢,至成宗復加至二貫。其親親之義若此,誠可謂厚之至矣。至於勛臣亦然,又所以大報功也。故詳著其所賜之人,及其數之多寡於後。



 諸王



 太祖叔答里真官人位:



 歲賜,銀三十錠,段一百匹。



 五戶絲,丙申年,分撥寧海州一萬戶。延祐六年,實有四千五百三十二戶,計絲一千八百一十二斤。



 江南戶鈔,至元十八年,撥南豐州一萬一千戶,計鈔四百四十錠。



 太祖弟搠只哈撒兒大王子淄川王位:



 歲賜,銀一百錠,段三百匹。



 五戶絲,丙申年,分撥般陽路二萬四千四百九十三戶。延祐六年,實有七千九百五十四戶,計絲三千六百五十六斤。



 江南戶鈔,至元十三年,分撥信州路三萬戶,計鈔一千二百錠。



 太祖弟哈赤溫大王子濟南王位:



 歲賜,銀一百錠,綿六百二十五斤,小銀色絲五千斤,段三百匹,羊皮一千張。



 五戶絲,丙申年,分撥濟南路五萬五千二百戶。延祐六年,實有二萬一千七百八十五戶,計絲九千六百四十八斤。



 江南戶鈔,至元十八年,分撥建昌路六萬五千戶,計鈔二千六百錠。



 太祖弟斡真那顏位:



 歲賜,銀一百錠,絹五千九十八匹,綿五千九十八斤,段三百匹,諸物折中統鈔一百二十錠,羊皮五百張,金一十六錠四十五兩。



 五戶絲,丙申年,分撥益都路等處六萬二千一百五十六戶。延祐六年,實有二萬八千三百一戶,計絲一萬一千四百二十五斤。



 江南戶鈔,至元十八年,分撥建寧路七萬一千三百七十七戶,計鈔二千八百五十五錠。



 太祖弟孛羅古鷿大王子廣寧王位:



 歲賜,銀一百錠,段三百匹。



 五戶絲,丙申年,分撥恩州一萬一千六百三戶。延祐六年,實有二千四百二十戶,計絲一千三百五十九斤。



 江南戶鈔,至元十八年,分撥鉛山州一萬八千戶,計鈔七百二十錠。。



 太祖長子術赤大王位:



 歲賜,段三百匹,常課段一千匹。



 五戶絲,丙申年,分撥平陽四萬一千三百二戶。戊戌年,真定晉州一萬戶。



 江南戶鈔,至元十八年,分撥永州六萬戶,計鈔二千四百錠。



 太祖次子茶合鷿大王位:



 歲賜,銀一百錠,段三百匹,綿六百二十五斤,常課金六錠六兩。



 五戶絲,丙申年,分撥太原四萬七千三百三十戶。戊戌年,真定深州一萬戶。延祐六年,實有一萬七千二百一十一戶,計絲六千八百三十八斤。



 江南戶鈔,至元十八年,分撥澧州路六萬七千三百三十戶,計鈔二千六百九十三錠。



 太祖第三子太宗子定宗位:



 歲賜,銀一十六錠三十三兩,段五十匹。



 五戶絲,丙申年,分撥大名六萬八千五百九十三戶。延祐六年,實有一萬二千八百三十五戶,計絲五千一百九十三斤。



 太祖第四子睿宗子阿里不哥大王位:



 歲賜,銀一百錠,段三百匹。



 五戶絲,丙申年,分撥真定路八萬戶。延祐六年,實有一萬五千二十八戶,計絲五千一十三斤。



 江南戶鈔,至元十八年,分撥撫州路一十萬四千戶,計鈔四千一百六十錠。



 太祖第五子兀魯赤太子。無嗣。



 太祖第六子闊列堅太子子河間王位:



 歲賜,銀一百錠,段三百匹。



 五戶絲,丙申年,分撥河間路四萬五千九百三十戶。延祐六年,實有一萬一百四十戶,計絲四千四百七十九斤。



 江南戶鈔,至元十八年,分撥衡州路五萬三千九百三十戶,計鈔二千一百五十七錠。



 太宗子合丹大王位:



 歲賜,銀一十六錠三十三兩,段五十匹。



 五戶絲,丁巳年,分撥汴梁在城戶。至元三年,改撥鄭州。延祐六年,實有二千三百五十六戶,計絲九百三十六斤。



 江南戶鈔,至元十八年,分撥常寧州二千五百戶,計鈔一百錠。



 太宗子滅里大王位:



 歲賜,銀一十六錠三十三兩,段五十匹。



 五戶絲,丁巳年,分撥汴梁在城戶。至元三年,改撥鈞州一千五百八十四戶。延祐六年,實有二千四百九十六戶,計絲九百九十七斤。



 太宗子合失大王位:



 歲賜,銀一十六錠三十三兩,段五十匹。



 五戶絲,丁巳年,分撥汴梁路在城戶。至元三年,改撥蔡州三千八百一十六戶。延祐六年,實有三百八十八戶,計絲一百五十四斤。



 太宗子闊出太子位:



 歲賜,銀六十六錠三十三兩,段一百五十匹。



 五戶絲,丁巳年,分撥汴梁路在城戶。至元三年,改撥睢州五千二百一十四戶。延祐六年,實有一千九百三十七戶,計絲七百六十四斤。



 太宗子闊端太子位:



 歲賜,銀一十六錠三十三兩,段五十匹。



 五戶絲,丙申年,分撥東平路四萬七千七百四十一戶。延祐六年,實有一萬七千八百二十五戶,計絲三千五百二十四斤。



 江南戶鈔,至元十八年,分撥常德路四萬七千七百四十戶,計鈔一千九百九錠。



 睿宗長子憲宗子阿速臺大王位:。



 歲賜,銀八十二錠,段三百匹。



 又泰定二年,晃兀帖木兒大王改封並王,增歲賜銀一十錠,班禿大王銀八錠。



 又泰定三年,明裏忽都魯皇后位下,添歲賜中統鈔一千錠,段五十匹,絹五十匹。



 五戶絲,癸丑年,查過衛輝路三千三百四十二戶。延祐六年,實有二千二百八十戶,計絲九百一十六斤。



 睿宗子世祖次子裕宗位:



 裕宗妃伯藍也怯赤:



 歲賜,銀五十錠。



 江南戶鈔,延祐三年,分撥江州路德化縣二萬九千七百五十戶,計鈔一千一百九十錠。



 裕宗子順宗子武宗:



 五戶絲,丁巳年,分撥懷孟一萬一千二百七十三戶。



 江南戶鈔,大德八年,分撥瑞州路六萬五千戶,計鈔二千六百錠。



 睿宗子旭烈大王位:



 歲賜,銀一百錠,段三百匹。



 五戶絲,丁巳年,分撥彰德路二萬五千五十六戶。延祐六年,實有二千九百二十九戶,計絲二千二百一斤。



 睿宗子阿里不哥大王位。見前。



 睿宗子末哥大王位:



 歲賜,銀五十錠,段三百匹。



 五戶絲,丁巳年,分撥河南府五千五百五十二戶。延祐六年,實有八百九戶,計絲三百三十三斤。



 江南戶鈔,至元十八年,分撥茶陵州八千五十二戶,計鈔三百二十四錠。



 睿宗子撥綽大王位:



 歲賜,銀五十錠,段三百匹。



 五戶絲,丁巳年,分撥真定蠡州三千三百四十七戶。延祐六年,實有一千四百七十二戶,計絲六百一十二斤。



 江南戶鈔,至元十八年,分撥耒陽州五千三百四十七戶,計鈔二百一十三錠。



 睿宗子歲哥都大王位:



 五戶絲,壬子年,元查認濟南等處五千戶。延祐六年,實有五十戶,計絲二十斤。



 世祖長子朵兒只太子位:



 腹裏、江南無分撥戶。



 世祖次子裕宗後位:



 歲賜,段一千匹,絹一千匹。



 江南戶鈔,至元十八年,分撥龍興路一十萬五千戶,計鈔四千二百錠。



 又四怯薛伴當江南戶鈔,至元十八年,撥瑞州上高縣八千戶,計鈔三百三十錠。



 世祖次子安西王忙哥剌位:



 歲賜,段一千匹,絹一千匹。



 江南戶鈔,至元十八年,分撥吉州路六萬五千戶,計鈔二千六百錠。



 世祖次子北安王那木罕位:



 歲賜,段一千匹,絹一千匹。



 江南戶鈔,至元二十二年,分撥臨江路六萬五千戶,計鈔二千六百錠。



 世祖次子寧遠王闊闊出位:



 歲賜,段匹物料,折鈔一千六百五十六錠;銀五十錠,折鈔一千錠。



 江南戶鈔,泰定元年,分撥永福縣一萬三千六百四戶,計鈔五百四十四錠。



 世祖次子西平王奧魯赤位:



 歲賜,段匹物料,折鈔一千六百五十六錠;銀五十錠,折鈔一千錠。



 江南戶鈔,大德七年,分撥南恩州一萬三千六百四戶,計鈔五百四十四錠。



 世祖次子愛牙赤大王位:



 歲賜,銀五十錠,折鈔一千錠;段匹物料,折鈔一千六百五十六錠。



 江南戶鈔,皇慶元年,分撥邵武路光澤縣一萬三千六百四戶,計鈔五百四十四錠。



 世祖次子鎮南王脫歡位:



 歲賜,銀五十錠;段匹物料,折鈔一千六百五十六錠。



 江南戶鈔,皇慶元年,分撥福州路寧德縣一萬三千六百四戶,計鈔五百四十四錠。



 世祖次子雲南王忽哥赤位:



 歲賜,銀五十錠,折鈔一千錠;段匹物料,折鈔一千六百五十六錠。



 江南戶鈔,皇慶元年,分撥福州路福安縣一萬三千六百四戶,計鈔五百四十四錠。



 世祖次子忽都帖木兒太子位:



 歲賜,銀五十錠,折鈔一千錠;段匹物料,折鈔一千六百五十六錠。



 江南戶鈔,皇慶元年,分撥泉州路南安縣一萬三千六百四戶,計鈔五百四十四錠。



 裕宗長子晉王甘麻剌位:



 歲賜,段一千匹,絹一千匹。



 又朵兒只,延祐元年為始,年例支中統鈔一千錠。



 五戶絲,闊闊不花所管益都二十九戶。



 江南戶鈔,皇慶元年,分撥南康路六萬五千戶。



 又迭裡哥兒不花湘寧王分撥湘鄉州、寧鄉縣六萬五千戶,計鈔二千六百錠。



 順宗子阿木哥魏王位:



 江南戶鈔,皇慶元年,分撥慶元路六萬五千戶,計鈔二千六百錠。



 順宗子武宗子明宗位:



 江南戶鈔,延祐二年,分撥湘潭州六萬五千戶,計鈔二千六百錠。



 合丹大王位:



 五戶絲,戊午年,分撥濟南漏籍二百戶。延祐六年,實有一百九十三戶,計絲七十七斤。



 阿魯渾察大王:



 五戶絲,丁巳年,分撥廣平三十戶。延祐三年,實有五戶,計絲二斤。



 霍里極大王:



 五戶絲,丁巳年,分撥廣平等處一百五十戶。延祐三年,實有八十七戶,計絲三十四斤。



 阿剌忒納失里豫王:



 天歷元年,分撥江西行省南康路。



 後妃公主



 太祖四大斡耳朵:



 大斡耳朵:



 歲賜,銀四十三錠,紅紫羅二十匹,染絹一百匹,雜色絨五千斤,針三千個,段七十五匹,常課段八百匹。



 五戶絲,乙卯年,分撥保定路六萬戶。延祐六年,實有一萬二千六百九十三戶,計絲五千二百七斤。



 江南戶鈔,至元十八年,分撥贛州路二萬戶,計鈔八百錠。



 第二斡耳朵:



 歲賜,銀五十錠,段七十五匹,常課段一千四百九十匹。



 五戶絲,丁巳年,分撥河間青城縣二千九百戶。延祐六年,實有一千五百五十六戶,計絲六百五十七斤。



 江南戶鈔,至元十八年,分撥贛州路一萬五千戶,計鈔六百錠。



 第三斡耳朵:



 歲賜,銀五十錠,段七十五匹,常課段六百八十二匹。



 五戶絲,壬子年,查認過真定等處畸零三百一十八戶。延祐六年,實有一百二十一戶,計絲四十八斤。



 江南戶鈔,至元十八年,分撥贛州路二萬一千戶,計鈔八百四十錠。



 第四斡耳朵:



 歲賜,銀五十錠,段七十五匹。



 五戶絲,壬子年,分撥真定等處二百八十三戶。延祐六年,實有一百一十六戶,計絲四十六斤。



 又八不別及妃子位,至元二十五年,分撥河間清州五百一十戶,計絲二百四斤。



 世祖四斡耳朵:



 大斡耳朵:



 歲賜,銀五十錠。



 江南戶鈔,大德三年,分撥袁州路宜春縣一萬戶,計鈔一千六百錠。



 第二斡耳朵:



 歲賜,銀五十錠,又七錠,段一百五十匹。



 江南戶鈔,至元二十一年,分撥袁州路分宜縣四千戶,計鈔一百六十錠。大德四年,分撥袁州路萍鄉州四萬二千戶,計鈔一千六百八十錠。



 第三斡耳朵:



 歲賜,銀五十錠。



 江南戶鈔,大德十年,分撥袁州路宜春縣二萬九千七百五十戶,計鈔一千一百九十錠。



 第四斡耳朵:



 歲賜,銀五十錠。



 江南戶鈔,大德十年,分撥袁州路萬載縣二萬九千七百五十戶,計鈔一千一百九十錠。



 順宗後位:



 歲賜,段五百匹。



 江南戶鈔,大德二年,分撥三萬二千五百戶。



 武宗斡耳朵:



 真哥皇后位:



 歲賜,銀五十錠,鈔五百錠。



 江南戶鈔,延祐二年,分撥湘陰州四萬二千戶,計鈔一千六百八十錠。



 完者臺皇后位:



 歲賜,銀五十錠。



 江南戶鈔,延祐二年,分撥潭州路衡山縣二萬九千七百五十戶,計鈔一千一百九十錠。



 阿昔倫公主位:



 至元六年,分撥葭州等處種田三百戶。



 趙國公主位:



 五戶絲,丙申年,分撥高唐州二萬戶。延祐六年,實有六千七百二十九戶,計絲二千三百九十九斤。



 江南戶鈔,至元十八年,分撥柳州路二萬七千戶,計鈔一千八十錠。



 魯國公主位:



 五戶絲,丙申年,分撥濟寧路三萬戶。延祐六年,實有六千五百三十戶,計絲二千二百九斤。



 江南戶鈔,至元十八年,分撥汀州四萬戶,計鈔一千六百錠。



 昌國公主位:



 五戶絲,丙申年,分撥一萬二千六百五十二戶。延祐六年,實有三千五百三十一戶,計絲二千七百六十六斤。



 江南戶鈔,至元十八年,分撥廣州路二萬七千戶,計鈔一千八十錠。



 鄆國公主位:



 五戶絲,丙申年,分撥濮州三萬戶。延祐六年,實有五千九百六十八戶,計絲一千八百三十六斤。



 江南戶鈔,至元十八年,分撥橫州等處四萬戶,計鈔一千六百錠。



 塔出駙馬:



 五戶絲,壬子年,元查真定等處畸零二百七十戶。延祐六年,實有二百三十二戶,計絲九十五斤。



 帶魯罕公主位:



 歲賜,銀四錠八兩,段一十二匹。



 五戶絲,延祐六年,實有代支戶六百三十戶,計絲二百五十四斤。



 火雷公主位:



 五戶絲,丙申年,分撥延安府九千七百九十六戶。延祐六年,實有代支戶一千八百九戶,計絲七百二十二斤。



 奔忒古兒駙馬:



 五戶絲,庚辰年,分撥眼戶五百七十三戶。延祐六年,實有五十六戶,計絲二十二斤。



 獨木干公主位:



 五戶絲,丁巳年,分撥平陽一千一百戶。延祐六年,實有五百六十戶,計絲二百二十四斤。



 江南戶鈔,至元十八年,分撥梅州程鄉縣一千四百戶,計鈔五十六錠。



 勛臣



 木華黎國王:



 五戶絲,丙申年,分撥東平三萬九千一十九戶。延祐六年,實有八千三百五十四戶,計絲三千三百四十三斤。



 江南戶鈔,至元十八年,分撥韶州等路四萬一千一十九戶,計鈔一千六百四十錠。



 孛羅先鋒:



 五戶絲,丙申年,分撥廣平等處種田一百戶。延祐六年,實有七十戶,計絲二十八斤。



 行醜兒:



 五戶絲,丙申年,分撥大名種田一百戶。延祐六年,實有三十八戶,計絲一十五斤。



 闊闊不花先鋒:



 五戶絲,壬子年,元查益都等處畸零二百七十五戶。延祐六年,實有一百二十七戶,計絲一十五斤。



 撒吉思不花先鋒:



 五戶絲,壬子年,元查汴梁等處二百九十一戶。延祐六年,實有一百二十七戶,計絲一十五斤。



 阿里侃斷事官:



 五戶絲,壬子年,元查濟寧等處三十五戶,計絲一十四斤。



 乞裡歹拔都:



 五戶絲,丙申年,分撥東平一百戶,計絲四十斤。



 孛羅海拔都:



 五戶絲,壬子年,元查德州等處一百五十三戶,計絲六十一斤。



 拾得官人:



 五戶絲,壬子年,元查東平等處畸零一百一十二戶,計絲八十四斤。



 伯納官人:



 五戶絲,壬子年,元查東平三十二戶。延祐六年,實有四十五戶,計絲一十八斤。



 笑乃帶先鋒:



 五戶絲,丙申年,分撥東平一百戶。延祐六年,實有七十八戶,計絲三十一斤。



 帶孫郡王:



 五戶絲,丙申年,分撥東平東阿縣一萬戶。延祐六年,實有一千六百七十五戶,計絲七百二十斤。



 江南戶鈔,至元十八年,分撥韶州路樂昌縣一萬七千戶,計鈔四百二十八錠。



 慍裡答兒薛禪:



 五戶絲,丙申年,分撥泰安州二萬戶。延祐六年,實有五千九百七十一戶,計絲二千四百二十五斤。



 江南戶鈔,至元十八年,分撥桂陽州二萬一千戶,計鈔八百四十錠。



 術赤臺郡王:



 五戶絲,丙申年,分撥德州二萬戶。延祐六年,實有七千一百四十六戶,計絲二千九百四十八斤。



 江南戶鈔,至元十八年,分撥連州路二萬一千戶,計鈔八百四十錠。



 阿兒思蘭官人:



 江南戶鈔,至元十八年,分撥潯州路三千戶,計鈔一百二十錠。



 孛魯古妻佟氏:



 五戶絲,丙申年,分撥真定一百戶。延祐六年,實有三十九戶,計絲一十五斤。



 八答子:



 五戶絲,丙申年,分撥順德路一萬四千八十七戶。延祐六年,實有四千四百四十六戶,計絲二千四百六斤。



 江南戶鈔,至元十八年,分撥欽州路一萬五千八十七戶,計鈔六百三錠。



 右手萬戶三投下孛羅臺萬戶:



 五戶絲,丙申年,分撥廣平路洺水縣一萬七千三百三十三戶。延祐六年,實有四千七百三十三戶,計絲一千七百三十八斤。



 江南戶鈔,至元十八年,分撥全州路清湘縣一萬七千九百一十九戶,計鈔七百一十六錠。



 忒木臺駙馬:



 五戶絲,丙申年,分撥廣平路磁州九千四百五十七戶。延祐六年,實有二千四百七戶,計絲九百八十九斤。



 江南戶鈔,至元二十二年,分撥全州路錄事司九千八百七十六戶,計鈔三百九十五錠。



 斡闊烈闍里必:



 五戶絲,丙申年,分撥廣平路一萬五千八百七戶。延祐六年,實有一千七百三戶,計絲六百八十斤。



 江南戶鈔,至元二十年,分撥全州路灌陽縣一萬六千一百五十七戶,計鈔六百四十六錠。



 左手九千戶合丹大息千戶:



 五戶絲,丙申年,分撥河間路齊東縣一千二十三戶。延祐六年,實有三百六十六戶,計絲一百六十斤。



 江南戶鈔,至元十八年,分撥藤州、蒼梧縣一千二百四十四戶,計鈔九錠。



 也速不花等四千戶:



 五戶絲,丙申年,分撥河間路陵州一千三百一十七戶。延祐六年,實有五百五十九戶,計絲二百二十三斤。



 也速兀兒等三千戶:



 五戶絲,丙申年,分撥河間路寧津縣一千七百七十五戶。延祐六年,實有七百二十二戶,計絲二百八十八斤。



 江南戶鈔,至元十八年,分撥藤州等處三千七百三十二戶,計絲二百八十八斤。



 帖柳兀禿千戶:



 五戶絲,丙申年,分撥河間路臨邑縣一千四百五十戶。延祐六年,實有三百五十四戶,計絲二百六斤。



 江南戶鈔,至元十八年,分撥藤州一千二百四十四戶,計鈔四十九錠。



 和斜溫兩投下一千二百戶:



 五戶絲,丙申年,分撥曹州一萬戶。延祐六年,實有一千九百二十八戶,計絲七百四十八斤。



 江南戶鈔,至元十八年,分撥貴州一萬五百戶,計鈔四百二十錠。



 忽都虎官人:



 五戶絲,壬子年,查認過廣平等處四千戶。



 江南戶鈔,至元十八年,分撥韶州曲江縣五千三百九戶,計鈔二百一十二錠。



 滅古赤:



 五戶絲,丙申年,分撥鳳翔府實有一百三十戶。



 江南戶鈔,至元二十二年,分撥永州路祁陽縣五千戶,計鈔二百錠。



 塔思火兒赤:



 五戶絲,丙申年分撥東平種田戶,並壬子年續查戶,共六百八十戶。延祐六年,實有三百八十九戶,計絲一百五十五斤。



 塔丑萬戶:



 五戶絲,壬子年,元查平陽等處一百八十六戶。延祐六年,實有八十一戶,計絲三十七斤。



 察罕官人:



 五戶絲,壬子年,元查懷孟等處三千六百六戶。延祐六年,實有五百六十戶,計絲二百二十四斤。



 孛羅渾官人:



 五戶絲,壬子年,元查保定等處四百一十五戶。丁巳年,分撥衛輝路淇州一千一百戶。延祐六年,實有一千九十九戶,計絲四百四十九斤。



 江南戶鈔,至元二十七年、大德六年,分撥四千戶,計鈔一百六十錠。



 速不臺官人:



 五戶絲,丁巳年,分撥汴梁等處一千一百戶。延祐六年,實有五百七十七戶,計絲二百三十斤。



 江南戶鈔,至元二十年,分撥欽州靈山縣一千六百戶,計鈔六十四錠。



 宿敦官人:



 五戶絲,丁巳年,分撥真定一千一百戶。延祐六年,實有六十四戶,計絲二十八斤。



 也苦千戶:



 五戶絲,丁巳年,分撥東平等處一千一百戶。延祐六年,實有二百九十五戶,計絲一百一十八斤。



 江南戶鈔,至元十八年,分撥梅州一千四百戶,計鈔五十六錠。



 阿可兒:



 五戶絲,癸丑年,分撥益都路高苑縣一千戶。延祐六年,實有一百九十六戶,計絲七十八斤。



 伯八千戶:



 五戶絲,丁巳年,分撥太原一千一百戶。延祐六年,實有三百五十一戶,計絲一百四十斤。



 兀里羊哈歹千戶:



 五戶絲,戊午年,分撥東平等處一千戶。延祐六年,實有四百七十九戶,計絲一百九十一斤。



 禿薛官人:



 五戶絲,丁巳年,分撥興元等處種田六百戶。延祐六年,實有二百戶,計絲八十斤。



 塔察兒官人:



 五戶絲,壬子年,元查平陽二百戶。延祐六年,實有二百戶,計絲八十斤。



 折米思拔都兒:



 五戶絲,丙申年,分撥懷孟等處一百戶。延祐六年,實有五十戶,計絲二十斤。



 猱虎官人:



 五戶絲,丁巳年,分撥平陽一千戶。延祐六年,實有六百戶,計絲二百四十斤。



 孛哥帖木兒:



 五戶絲,丙申年,分撥真定等處五十八戶,計絲二十三斤。



 也速魯千戶:



 五戶絲,壬子年,分撥真定路一百六十九戶。延祐六年,實有四十戶,計絲一十六斤。



 鎮海相公:



 五戶絲,壬子年,元查保定九十五戶。延祐六年,實有五十三戶,計絲二十一斤。



 按察兒官人:



 五戶絲,壬子年,分撥太原等處五百五十戶。延祐六年,實有九十八戶,計絲二十九斤。



 按攤官人:



 五戶絲,中統元年,元查平陽路種田戶六十戶。延祐六年,實有四十戶,計絲一十六斤。



 阿術魯拔都:



 五戶絲,壬子年,查大名等處三百一十戶。延祐六年,實有三百一戶,計絲一百二十斤。



 孛羅口下裴太納:



 五戶絲,壬子年,元查廣平等處八十二戶。延祐六年,實有三十戶,計絲一十二斤。



 忒木臺行省:



 五戶絲,壬子年,元查大同等處七百五十一戶。延祐六年,實有二百五十五戶,計絲一百一十斤。



 撒禿千戶:



 江南戶鈔,至元二十年,分撥潯州三千戶,計鈔一百二十錠。



 也可太傅:



 五戶絲,壬子年,元查上都五百四十戶。延祐六年,實有三百戶,計絲一百二十斤。



 迭哥官人:



 五戶絲,丙申年,分撥大名清豐縣一千七百一十三戶。延祐六年,實有一千三百七戶,計絲五百七斤。



 卜迭捏拔都兒:



 五戶絲,壬子年,元查懷孟八十八戶。延祐六年,實有四十戶,計絲一十六斤。



 黃兀兒塔海:



 五戶絲,丙申年,分撥平陽一百四十四戶。延祐六年,實有一百戶,計絲四十斤。



 怯來千戶:



 江南戶鈔,至元二十年,分撥潯州路三千戶,計鈔一百二十錠。



 哈剌口溫:



 五戶絲,壬子年,元查真定三十二戶。



 曳剌中書兀圖撒罕里:



 五戶絲,壬子年,元查大都等處八百七十戶。延祐六年,實有四百四十九戶,計絲一百一十七斤。



 欠帖木:



 五戶絲,壬子年,元查曹州三十四戶。延祐六年,實有三十四戶。



 欠帖溫:



 歲賜絹一百匹,弓弦一千條。



 江南戶鈔,至元十九年,分撥梅州、安仁縣四千戶,計鈔一百六十錠。



 扎八忽娘子:



 歲賜常課段四百七十匹。



 魚兒泊八剌千戶:



 五戶絲,大德元年,分撥真定等處一千戶。延祐三年,實有六百戶,計絲二百四十斤。



 昔寶赤:



 江南戶鈔,至元二十一年,分撥衡州路安仁縣四千戶,計鈔一百六十錠。



 八剌哈赤:



 江南戶鈔,至元二十一年,分撥臺州路天臺縣四千戶,計鈔一百六十錠。



 阿塔赤:



 江南戶鈔,至元二十一年,分撥常德路沅江縣四千戶,計鈔一百六十錠。



 必闍赤:



 江南戶鈔,至元二十一年,分撥袁州路萬載縣三千戶,計鈔一百二十錠。



 貴赤:



 江南戶鈔,至元二十一年,分撥和州歷陽縣四千戶,計鈔一百六十錠。



 厥列赤:



 江南戶鈔,至元二十一年,分撥婺州永康縣五十戶,計鈔二十錠。



 八兒赤、不魯古赤:



 江南戶鈔,至元二十一年,分撥衡州路酃縣六百戶,計鈔二十四錠。



 阿速拔都:



 江南戶鈔,至元二十一年,分撥盧州等處三千四百九戶,計鈔一百三十六錠。



 也可怯薛:



 江南戶鈔,至元二十一年,分撥武岡路武岡縣五千戶,計鈔二百錠。



 忽都答兒怯薛:



 江南戶鈔,至元二十一年,分撥武岡路新寧縣五千戶,計鈔二百錠。



 帖古迭兒怯薛:



 江南戶鈔,至元二十一年,分撥常德路龍陽縣五千戶,計鈔二百錠。



 月赤察兒怯薛:



 江南戶鈔,至元二十一年,分撥武岡路綏寧縣五千戶,計鈔二百錠。



 玉龍帖木兒千戶:



 江南戶鈔,至元二十年,分撥潯州三千戶,計鈔一百二十錠。



 別苦千戶:



 江南戶鈔,至元二十年,分撥潯州三千戶,計鈔一百二十錠。



 憧兀兒王:



 江南戶鈔,延祐二年為始,支中統鈔二百錠,無城池。



 霍木海:



 五戶絲,壬子年,元查大名等處三十三戶。



 哈剌赤禿禿哈:



 江南戶鈔,至元二十一年,分撥饒州路四千戶,計鈔一百六十錠。



 添都虎兒:



 五戶絲,丙申年,分撥真定一百戶。



 賈答剌罕:



 五戶絲,壬子年,元查大都一十四戶。



 阿剌博兒赤:



 五戶絲,壬子年,元查真定五十五戶。



 忽都那顏:



 五戶絲,壬子年,元查大名二十戶。



 忽辛火者:



 五戶絲,壬子年,元查真定二十七戶。



 大忒木兒:



 五戶絲,壬子年,元查真定二十二戶。



 布八火兒赤:



 五戶絲,壬子年,元查大都八十四戶。



 塔蘭官人:



 五戶絲,壬子年,元查大寧三戶。



 憨剌哈兒:



 五戶絲,壬子年,元查保定二十一戶。



 昔里吉萬戶:



 五戶絲,壬子年,元查大都七十九戶。



 清河縣達魯花赤也速:



 五戶絲,壬子年,元查大名二十戶。



 塔剌罕劉元帥:



 五戶絲,壬子年,元查順德一十九戶。



 怯薛臺蠻子:



 五戶絲,壬子年,元查泰安州七戶。



 必闍赤汪古臺:



 五戶絲,壬子年,元查汴梁等處四十六戶。



 阿剌罕萬戶:



 五戶絲,壬子年,元查保定一戶。



 徐都官人:



 五戶絲,壬子年,元查大都三十一戶。



 西川城左翼蒙古漢軍萬戶脫力失:



 歲賜,常課段三十三匹。



 伯要歹千戶:



 歲賜,段二十四匹。



 典迭兒:



 歲賜,常課段六十四匹。



 燕帖木兒太平王:



 歲賜,天歷元年,定金十錠、銀五十錠、鈔一萬錠,分撥江東道太平路地五百頃。



\end{pinyinscope}