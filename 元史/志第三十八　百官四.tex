\article{志第三十八 百官四}

\begin{pinyinscope}

 太常禮儀院,秩正二品,掌大禮樂、祭享宗廟社稷、封贈謚號等事。中統元年,中都立太常寺,設寺丞一員。至元二年「文學」、「教育」、「政治學」中的「亞里士多德」。,翰林兼攝太常寺。九年,立太常寺,設卿一員,正三品;少卿以下五員,品秩有差。十三年,省並衙門,以侍儀司並入太常寺。十四年,增博士一員。十六年,又增法物庫子,掌公服法服之藏。二十年,升正三品,別置侍儀司。至大元年,改升院,設官十二員,正二品。四年,復為太常寺,正三品。延祐元年,復改升院,正二品,以大司徒領之。七年,降從二品。天歷二年,復升正二品。定置院使二員,正二品;同知二員,正三品;僉院二員,從三品;同僉二員,正四品;院判二員,正五品;經歷一員,從五品;都事一員,從七品;照磨兼管勾承發架閣一員,正八品。屬官:博士二員,正七品;奉禮郎二員,奉禮兼檢討一員,並從八品;協律郎二員,從八品;太祝十員,從八品;禮直管勾一員,從九品;令史四人,通事、知印、譯史各二人,宣使四人,典吏三人。



 太廟署,秩從六品,掌宗廟行禮,兼廩犧署事。至元三年始置。令二員,從六品;丞一員,從七品。



 郊祀署,秩從六品。大德九年始置。掌郊祀行禮,兼廩犧署事。令二員,從六品;丞二員,從七品。



 社稷署,秩從六品。大德元年始置。令二員,從六品;丞一員,從七品。



 大樂署,秩從六品。中統五年始置。令二員,從六品;丞一員,從七品。掌管禮生樂工四百七十九戶。



 典瑞院,秩正二品。掌寶璽、金銀符牌。中統元年,始置符寶郎二員。至元十六年,立符寶局,給六品印。十七年,升正五品。十八年,改典瑞監,秩正三品。二十年,降為正四品,省卿二員。二十九年,復正三品,仍置監卿二員。大德十一年,升典瑞院,正二品。置院使四員,正二品;同知二員,正三品;僉院二員,從三品;同僉二員,正四品;院判二員,正五品;經歷二員,從五品;都事二員,從七品;照磨兼管勾承發架閣庫一員,正八品;令史四人,譯史四人,知印、通事各一人,宣使四人,典吏三人。



 太史院,秩正二品,掌天文歷數之事。至元十五年,始立院,置太史令等官一員。至大元年,升從二品,設官十員。延祐三年,升正二品,設官十五員。後定置院使五員,正二品;同知二員,正三品;僉院二員,從三品;同僉二員,正四品;院判二員,正五品;經歷一員,從五品;都事一員,從七品;管勾一員,從九品;令史三人,譯史一人,知印二人,通事一人,宣使二人,典吏二人。



 春官正兼夏官正一員,正五品。



 秋官正兼冬官正中官正一員,正五品。



 保章正五員,正七品。



 保章副五員,正八品。



 掌歷二員,正八品。



 腹里印歷管勾一員,從九品。



 各省司歷十二員,正九品。



 印歷管勾二員,從九品。



 靈臺郎一員,正七品。



 監候六員,從八品。



 副監候六員,正九品。



 星歷生四十四員。



 挈壺正一員,從八品。



 司辰郎二員,正九品。



 燈漏直長一人。



 教授一員,從八品。



 學正一員,從九品。



 校書郎二員,正八品。



 太醫院,秩正二品,掌醫事,制奉御藥物,領各屬醫職。中統元年,置宣差,提點太醫院事,給銀印。至元二十年,改為尚醫監,秩正四品。二十二年,復為太醫院,給銀印,置提點四員,院使、副使、判官各二員。大德五年,升正二品,設官十六員。十一年,增院使二員。皇慶元年,增院使二員。二年,增院使一員。至治二年,定置院使一十二員,正二品;同知二員,正三品;僉院二員,從三品;同僉二員,正四品;院判二員,正五品;經歷二員,從七品;都事二員,從七品;照磨兼承發架閣庫一員,正八品;令史八人,譯史二人,知印二人,通事二人,宣使七人。



 廣惠司,秩正三品,掌修制御用回回藥物及和劑,以療諸宿衛士及在京孤寒者。至元七年,始置提舉二員。十七年,增置提舉一員。延祐六年,升正三品。七年,仍正五品。至治二年,復為正三品,置卿四員,少卿、丞各二員。後定置司卿四員,少卿二員,司丞二員,經歷、知事、照磨各一員。



 大都、上都回回藥物院二,秩從五品,掌回回藥事。至元二十九年始置。至治二年,撥隸廣惠司,定置達魯花赤一員、大使二員、副使一員。



 御藥院,秩從五品,掌受各路鄉貢、諸蕃進獻珍貴藥品,修造湯煎。至元六年始置。達魯花赤一員,從五品;大使二員,從五品;副使三員,正七品;直長一員,都監二員。



 御藥局,秩從五品,掌兩都行篋藥餌。至元十年始置。大德九年,分立行御藥局,掌行篋藥物。本局但掌上都藥倉之事。定置達魯花赤一員,從五品;局使二員,從五品;副使二員,正七品。



 行御藥局,秩從五品。達魯花赤一員,大使二員,副使三員,品秩同上。掌行篋藥餌。大德九年始置。



 御香局,秩從五品,提點一員,司令一員,掌修合御用諸香。至大元年始置。



 大都惠民局,秩從五品,掌收官錢,經營出息,市藥修劑,以惠貧民。中統二年始置,受太醫院劄。至元十四年,定從六品秩。二十一年,升從五品。



 上都惠民司,提點一員,司令一員。中統四年始置,品秩並同上。



 醫學提舉司,秩從五品。至元九年始置。十三年罷,十四年復置。掌考校諸路醫生課義,試驗太醫教官,校勘名醫撰述文字,辨驗藥材,訓誨太醫子弟,領各處醫學。提舉一員,副提舉一員。



 官醫提舉司,秩從五品,掌醫戶差役、詞訟。至元二十五年置。



 大都、保定、彰德、東平四路,設提舉、同提舉、副提舉各一員。



 河間、大名、晉寧、大同、濟寧、廣平、冀寧、濟南、遼陽、興和十路,設提舉、副提舉各一員。



 衛輝、懷慶、大寧,設提舉一員。



 奎章閣學士院,秩正二品。天歷二年,立於興聖殿西,命儒臣進經史之書,考帝王之治。大學士二員,正三品。尋升為學士院。大學士,正二品;侍書學士,從二品;承制學士,正三品;供奉學士,正四品;參書,從五品。多以它官兼領其職。至順元年,增大學士二員,共四員。侍書學士二員,承制學士二員,供奉學士二員。首領官:參書二員,典簽二員,照磨一員,內掾四人,譯文內掾二人,知印二人,怯里馬赤一人,宣使四人,典書五人。屬官:授經郎二員。



 群玉內司,秩正三品,天歷二年始置,掌奎章圖書寶玩,及凡常御之物。監司一員,正三品;司尉一員,從三品;亞尉一員,正四品;僉司二員,從四品;司丞二員,正五品;典簿一員,正七品;令史二人,知印一人,怯里馬赤一人,奏差、典吏各二人,給使八人,司膳四人。



 藝文監,秩從三品。天歷二年置,專以國語敷譯儒書,及儒書之合校讎者俾兼治之。大監檢校書籍事二員,從三品;少監同檢校書籍事二員,從四品;監丞參檢校書籍事二員,從五品;典簿一員,照磨一員,令史四人,譯史一人,怯里馬赤一人,奏差二人,典吏三人。



 監書博士,秩正五品。天歷二年始置。品定書畫,擇朝臣之博識者為之。博士二員,正五品;書吏一人。



 藝林庫,秩從六品。提點一員,從六品;大使一員,副使一員,正七品;庫子二人,本把二人。掌藏貯書籍。天歷二年始置。



 廣成局,秩七品,掌傳刻經籍及印造之事。天歷二年始置。大使一員,從七品;副使一員,正八品;直長二人,正九品;司吏二人。



 侍正府,秩正二品。至順二年置。侍正一十四員,正二品;同知二員,正三品;參府二員,從三品;侍判二員,正四品;經歷一員,從六品;都事一員,從七品;照磨一員,從八品。掌內廷近侍之事,領速古兒赤四百人、奉御二十四員,拱衛直都指揮使司為其屬。掾史八人,譯史四人,通事、知印各二人,宣使八人,典吏五人。



 奉御二十四員,秩五品。尚冠奉御二員,從五品;尚冠副奉御二員,從六品;尚衣奉御二員,從五品;尚衣副奉御二員,從六品;尚鞶奉御二員,從五品;尚鞶副奉御二員,從六品;尚沐奉御二員,從五品;尚沐副奉御二員,從六品;尚飾兼尚輦奉御二員,正六品;尚飾兼尚輦副奉御二員,正七品;奉御掌簿四員,從七品。天歷初置,以四怯薛之速古兒赤為之。



 給事中,秩正四品。至元六年,始置起居注、左右補闕,掌隨朝省、臺、院、諸司凡奏聞之事,悉紀錄之,如古左右史。十五年,改升給事中兼修起居注,左右補闕改為左右侍儀奉御兼修起居注。皇慶元年,升正三品。延祐七年,仍正四品。後定置給事中兼修起居注二員,右侍儀奉御同修起居注一員,左侍儀奉御同修起居注一員,令史一人,譯史四人,通事兼知印一人。



 將作院,秩正二品,掌成造金玉珠翠犀象寶貝冠佩器皿,織造刺繡段匹紗羅,異樣百色造作。至元三十年始置。院使一員,經歷、都事各一員。三十一年,增院使二員。元貞元年,又增二員。延祐七年,省院使二員。後定置院使七員,正二品;同知二員,正三品;同僉二員,正四品;院判二員,正五品;經歷一員,從五品;都事一員,從七品;照磨管勾一員,正八品;令史六人,譯史、知印各二人,宣使四人。



 諸路金玉人匠總管府,秩正三品,掌造寶貝金玉冠帽、系腰束帶、金銀器皿,並總諸司局事。中統二年,初立金玉局,秩正五品。至元三年,改總管府,置總管一員,經歷、提控案牘各一員。十二年,又置同知、副總管各一員。二十五年,置達魯花赤一員。大德四年,又置副達魯花赤、副總管各一員。後定置達魯花赤二員,正三品;總管二員,正三品;副達魯花赤二員,正四品;同知二員,從四品;副總管二員,正五品;經歷一員,從七品;知事一員,從八品;照磨、管勾各一員,令史五人,譯史一人,奏差二人。



 玉局提舉司,秩從五品。提舉一員,正七品;同提舉一員,從七品;副提舉一員,正八品。中統二年,以和林人匠置局造作,始設直長。至元三年,立玉匠局,用正七品印。十五年,改提舉司。



 金銀器盒提舉司,秩從五品。提舉一員,同提舉一員,副提舉一員,品秩同上;吏目一員。至元十五年,始置金銀局,秩從七品。二十四年,改為提舉司,秩正六品。大德間,升從五品。



 瑪瑙提舉司,秩從五品。提舉一員,同提舉一員,吏目一員。至元九年,置大都等處瑪瑙局,秩從七品,管領瑪瑙匠戶五百有奇,置提舉三員,受金玉府劄。十五年,改立提舉司,領大都、宏州兩處造作,升從五品。三十年,減副提舉一員,定置如上。



 陽山瑪瑙提舉司,秩從五品。至元十五年置。提舉一員,同提舉一員,副提舉一員,品秩同前。



 金絲子局,秩從五品。大使一員,從五品;副使一員,正七品;直長一員。中統二年,設二局。二十四年,並為一。



 鞓帶斜皮局,秩從八品,至元十五年置,大使、副使各一員。



 瓘玉局,秩從八品,至元十五年置,大使一員。



 浮梁磁局,秩正九品,至元十五年立,掌燒造磁器,並漆造馬尾棕藤笠帽等事,大使、副使各一員。



 畫局,秩從八品,掌描造諸色樣制。至元十五年置,大使一員。



 管領珠子民匠官,正七品,掌採撈蛤珠於楊村、直沽等處。中統二年立,管領官子孫世襲。



 裝釘局,從八品,至元十五年置,大使一員。



 大小雕木局,秩從八品,至元十五年置,大使一員。



 宣德隆興等處瑪瑙人匠提舉司,秩正六品。至元十五年置。提舉一員,從七品;副提舉一員,從八品。



 溫犀玳瑁局,秩從八品,至元十五年置,大使一員。



 上都金銀器盒局,秩從六品,至元十六年置,大使一員,副使一員,直長一員。



 漆紗冠冕局,至元十五年置,大使、副使各一員。



 大同路採砂所,至元十六年置,管領大同路撥到民一百六戶,歲採磨玉夏水砂二百石,起運大都,以給玉工磨礲之用。大使一員。



 管匠都提領所,秩從七品,至元十三年置,掌金玉府諸人匠詞訟,都提領一員。



 監造諸般寶貝官,秩正五品,至元二十一年置,達魯花赤二員。



 收支諸物庫,秩從八品,至元十五年置,大使、副使各一員。



 行諸路金玉人匠總管府,秩從三品。至大間,始置於杭州路。達魯花赤、總管各一員,並從三品;同知一員,正五品;副總管一員,從五品;經歷一員,從七品;知事一員,從八品;提控案牘一員。



 異樣局總管府,秩正三品。中統二年,立提點所。至元六年,改為總管府,總管一員。十四年,置同知、副總管各一員。二十一年,增總管一員。二十九年,置達魯花赤一員。三十年,減同知、副總管各一員。後定置達魯花赤一員,總管一員,並正三品;同知一員,從四品;副總管一員,從五品;經歷一員,從七品;知事一員,從八品。



 異樣紋繡提舉司,秩從五品。中統二年立局。至元十四年,改提舉司。提舉一員,從五品;同提舉一員,正七品;副提舉一員,正八品。



 綾錦織染提舉司,秩從五品。至元二十四年,改局置提舉司。提舉一員,同提舉一員,副提舉一員,品秩同上。



 紗羅提舉司,秩從五品。至元十二年,改局置提舉司。提舉、同提舉、副提舉各一員,品秩同上。



 紗金顏料總庫,秩從九品。中統二年置,大使、副使各一員,從九品。



 大都等路民匠總管府,秩正三品。府官:總管一員,從三品;同知一員,正五品;副總管一員,從五品;經歷一員,從七品;知事一員,從八品;提控案牘一員。至元七年,初立府,秩從三品。十四年,改升正三品。



 備章總院,秩正六品,大使、副使各一員。至元十三年,省並楊藺等八局為總局。



 尚衣局,秩從五品。至元二年置。達魯花赤一員,從五品;提舉一員,從五品;同提舉一員,正七品;副提舉一員,正八品;都目一人。



 御衣局,秩從五品。至元二年置。達魯花赤、提舉各一員,從五品;同提舉一員,正七品;副提舉一員,正八品;都目一人。



 御衣史道安局,秩從六品。至元二年置。以史道安掌其職,因以名之。大使、副使各一員。



 高麗提舉司,秩從五品,至元二十二年置,提舉一員。



 織佛像提舉司,秩從五品。延祐四年,改提領所為提舉司。提舉、副提舉各二員。



 通政院,秩從二品。國初,置驛以給使傳,設脫脫禾孫以辨奸偽。至元七年,初立諸站都統領使司以總之,設官六員。十三年,改通政院。十四年,分置大都、上都兩院;二十九年,又置江南分院;大德七年罷。至大元年,升正二品。四年罷,以其事歸兵部。是年,兩都仍置,止管達達站赤。延祐七年,復從二品,仍兼領漢人站赤。大都院使四員,從二品;同知二員,正三品;副使二員,從三品;僉院一員,正四品;同僉一員,從四品;院判一員,正五品;經歷一員,從五品;都事一員,從七品;照磨兼管勾承發架閣一員,正八品;令史十三人,通事一人,知印二人,宣使十人。上都院使、同知、副使、僉院、判官各一員,經歷、都事各一員,品秩並同大都;令史四人,譯史三人,通事一人,知印一人,宣使十人。



 廩給司,秩從七品,掌諸王諸蕃各省四方邊遠使客飲食供張等事。至元十九年置,提領、司令、司丞各一員。



 中政院,秩正二品。院使七員,正二品;同知二員,正三品;僉院二員,從三品;同僉二員,正四品;院判二員,正五品。掌中宮財賦營造供給,並番衛之士,湯沐之邑。元貞二年,始置中御府,秩正三品。大德四年,升中政院,秩正二品。至大三年,升從一品,院使七員,同知、僉院、同僉、院判各二員。四年,省並入典內院。皇慶二年,復為中政院,設官如舊。其幕職則司議二員,從五品;長史二員,正六品;照磨兼管勾承發架閣一員,正八品。吏屬:蒙古必闍赤四人,掾史十二人,回回掾史二人,怯里馬赤二人,知印二人,宣使十人。



 中瑞司,秩正三品,掌奉寶冊。卿五員,正三品;丞二員,正四品;典簿二員,從七品;寫懿旨必闍赤四人,譯史一人,令史四人,知印一人,通事一人,奏差二人,典吏二人。



 內正司,秩正三品,掌百工營繕之役,地產孳畜之儲,以供膳服,備賜予。卿四員,正三品;少卿二員,正四品;丞二員,從五品;典簿二員,從七品;照磨兼管勾一員,正九品。吏屬各有差。領署二、提舉司一,及其司屬凡十有六。歲賦之額,工作之程,終歲則會其數以達焉。



 尚工署,秩從五品。令一員,從五品;丞二員,從六品;書史一人,書吏四人。掌營繕雜作之役,凡百工名數,興造程式,與其材物,皆經度之,而責其成功。皇慶元年始置,隸內正司。



 玉列赤局,秩從七品,提領一員,大使一員,副使一員,直長二員,掌裁制縫線之事。延祐六年始置,隸尚工署。



 贊儀署,秩正五品,提領一員,大使一員,副使一員,直長二員,掌車輿器備雜造之事。皇慶二年始置,隸內正司。



 管領六盤山等處怯憐口民匠都提舉司,秩正四品。達魯花赤一員,都提舉一員,同提舉二員,副提舉二員,知事一員,提控案牘一員,吏四人,奏差二人。至大四年始置。國初,未有官署,賦無所稽。後遣使核實,始著為籍,設司以領之。



 奉元等路、平涼等處、開城等處、甘肅寧夏等路、察罕腦兒等處長官司,凡五處,秩正五品。各設達魯花赤一員,長官一員,副長官一員,提控案牘一員,都目一員,吏十人。延祐二年,以民匠提舉司所領,地里闊遠,人戶散處,於政不便,乃酌遠近眾寡,立長官司提領所,以分理之。



 提領所凡十,並正七品,奉元等路、鳳翔等處、平涼寧環等處、開城等處、察罕腦兒等處、甘州等路、肅沙等路、永昌寧夏等路、長城等路,各設提領一員、同提領一員、副提領一員、典史一人,分掌怯憐口地方隸各長官司。



 翊正司,秩正三品。令五員,正三品;丞四員,正四品;典簿二員,從七品;照磨一員,從八品;譯史二人,令史六人,知印二人,通事、奏差、典吏各二人。掌怯憐口民匠五千餘戶,歲辦錢糧造作,以供公上。至元三十一年,始置御位下管領隨路民匠打捕鷹房納綿等戶總管府,正三品,復隸正宮位下。延祐六年,改翊正司。歲終,會其出納以達於院,而糾其弊。領提舉司二、提領所一:



 管領上都等處諸色人匠提舉司,秩從五品。達魯花赤一員,提舉一員,並從五品;同提舉一員,從六品;副提舉一員,從七品;直長一員,都目一員,吏目一員,司吏四人,部役二人。元貞元年始置,管戶二千五百有奇,隸翊正司。



 管領隨路打捕鷹房納綿等戶提舉司,秩從五品。達魯花赤一員,提舉一員,同提舉一員,副提舉一員,品秩同上;直長一員,都目一員,吏目一員,司吏四人,部役二人。元貞元年始置,隸翊正司。



 管領歸德亳州等處管民提領所,秩從七品。提領一員,同提領一員,副提領一員,典史一員,司吏一人。國初平江南,收附歸德楚通等三百五十六戶,令脫忽伯管領。大德二年,始置提領所,隸翊正司。



 典飲局,秩正七品,大使二員,副使二員,典史一員,攢典二人,掌醖造酒醴,以供內府,及祭祀宴享賓客賜頒之給。初置嘉醖局,秩六品,隸家令。至大二年,改典飲,兩都分置。皇慶元年,撥隸中宮。



 管領大都等路打捕民匠等戶總管府,秩正三品。達魯花赤一員,總管一員,並正三品;同知一員,正四品;副總管一員,正五品;經歷一員,從七品;知事一員,從八品;提控案牘照磨一員,譯史一人,令史、奏差各四人。掌錢糧造作之事。國初平定河南諸郡,收聚人戶一萬五千有奇,置官管領。至元八年,屬有司。二十年,改隸中尚監。二十六年,始置總管府。領提舉司十有一,提領所二十有五。



 在京提舉司二,秩從五品。達魯花赤一員,提舉一員,從五品;同提舉一員,從六品;副提舉一員,從七品;都目一員。分管各處人戶。至元十六年,給從七品印。大德四年,省並為十一處,改提舉司,升從五品。



 涿州、保定、真定、冀寧、河南、大名、東平、東昌、濟南等路提舉司,凡九處,各設達魯花赤一員、提舉一員、同提舉一員、副提舉一員、都目一員。



 提領所凡二十五處:大都等路、東安州、濟寧、曹州、沂州、完州、河間、濟南、濟陽、大同、元氏、冀寧、晉寧、歸德、南陽、懷孟、汝寧、衛輝、曹州、涿州、真定、中山、平山、大名、高唐等處,每處各設提領一員、同提領一員、副提領一員、典史一員。



 管領諸路打捕鷹房民匠等戶總管府,秩正三品。達魯花赤一員,總管一員,正三品;同知一員,正五品;副總管二員,從五品;經歷三員,從七品;知事一員,從八品;提控案牘一員,照磨一員,譯史一人,令史四人,奏差二人。掌錢糧造作之事。大德三年始置。元貞元年,撥隸中宮位下,領提舉司四、提領所十有一。



 管民提舉司,大都等路、冀寧等路、南陽唐州等處、河南路府等處,凡四司,秩從五品,每司設達魯花赤一員、提舉一員、同提舉一員、副提舉一員、都目一員、吏二人。



 提領所凡十有一:大都保定、河間真定、南陽鄧州、濟南嵩汝、汴梁裕州、汝濟陳州、唐州泌陽、襄陽湖陽、晉寧、冀寧等處各設所,秩正七品。每所提領二員,同提領一員,副提領一員,典史一員,司吏二人。至元十六年置。至大元年,改提領所。



 江浙等處財賦都總管府,秩正三品。達魯花赤一員,都總管一員,並正三品;同知一員,正五品;副總管一員,從五品;經歷一員,從七品;知事一員,從八品;照磨一員,提控案牘一員,從九品;譯史一人,令史一十五人,奏差一十五人,典吏二人。掌江南沒入貲產,課其所賦,以供內儲。至大元年置。領提舉司三,庫、局各一。



 平江、松江、建康等處提舉司凡三處,秩並正五品,每司各設達魯花赤一員、提舉一員、同提舉一員、副提舉一員、都目一員、吏目一員、司吏六人。



 豐盈庫,提領一員,大使一員,副使一員,典吏一人,掌收本府錢帛。



 織染局,局使一員,典吏一人,掌織染歲造段匹。



 管領種田打捕鷹房民匠等戶萬戶府,秩正三品,掌歸德、亳州、永、宿二十餘城各蒙古、漢軍種田戶差稅。中統二年置。初隸塔察兒王位下,其後改屬中宮。萬戶一員,經歷一員,知事一員,提控案牘一員,令史四人。領司屬凡十處。



 管領大名等處種田諸色戶總管府,秩正五品,總管一員,副總管一員,都目一員。中統二年置。至元二十三年,置府大名。



 管領本投下大都等處諸色戶計都達魯花赤,秩正五品,達魯花赤一員,提控案牘一員,都目一員。中統三年置。至元十五年,置司大都。



 管領大都河間等路打捕鷹房總管府,秩正五品,總管一員,副總管一員,都目一員,司吏二人。中統二年置,三年給印。



 管領東平等路管民官,秩正五品,總管一員,相副官一員,都目一員,吏一人。中統二年置,至元二十二年給印。



 管領大名等路宣撫司、燕京路管民千戶所,秩從七品,提領一員,副提領一員。中統二年置。



 管領曹州等處本投下民戶、管領東明等處本投下戶計、管領蒲城等處本投下諸色戶計、管領汴梁等路本投下種田打捕軀戶四提領所,秩正七品。提領各二員,同提領、副提領各一員,典史各一人,司吏各一人。中統二年置,至元十四年頒印。



 海西遼東哈思罕等處鷹房諸色人匠怯憐口萬戶府,秩正三品,達魯花赤一員,萬戶一員,副萬戶一員,經歷一員,知事一員,提控案牘兼照磨一員,譯史一人。掌錢糧造作之事,管領哈思罕等處、肇州、朵因溫都兒諸色人匠四千戶,仍領鎮撫所、千戶所。



 鎮撫司,鎮撫一員,吏一人。延祐四年始置。



 哈思罕等處打捕鷹房怯憐口千戶所,秩從五品,達魯花赤一員,千戶一員,副千戶一員,吏目一員,司吏四人,彈壓一人,部役二人。至大二年,置提舉司。延祐六年,改千戶所。



 諸色人匠怯憐口千戶所,秩從五品,達魯花赤一員,千戶一員,副千戶一員,都目一員,司吏四人,部役二人。初為提舉司,後改千戶所。



 肇州等處女直千戶所,達魯花赤一員,千戶一員,副千戶一員,吏目一員,司吏四人。延祐三年置。



 朵因溫都兒兀良哈千戶所,延祐三年置。



 灰亦兒等處怯憐口千戶所,至治元年置。



 開元等處怯憐口千戶所,至治元年置。



 石州等處怯憐口千戶所,延祐七年置。



 沈陽等處怯憐口千戶所,至治元年置。



 遼陽等處怯憐口千戶所,至治二年置。



 蓋州等處怯憐口千戶所,延祐五年置。



 乾盤等處怯憐口千戶所,至治元年置。



 遼陽等處金銀鐵冶都提舉司,秩正四品,都提舉一員,同提舉一員,副提舉一員,提控案牘一員,譯史一人,吏六人,奏差二人。掌辦金銀甘鐵等課,分納中書省及中政院。延祐七年,以其賦盡歸中宮。



 管領本位下怯憐口隨路諸色民匠打捕鷹房都總管府,秩正三品。達魯花赤一員,都總管一員,並正三品;同知一員,正五品;副總管一員,從五品。掌怯憐口二萬九千戶,田萬五千餘頃,出賦以備供奉營繕之事。中統二年置府。大德十年,隸詹事院。至大三年,隸徽政院。延祐三年,改善政司。至治二年,徽政院及其屬盡廢。天歷三年,復立府,仍正三品,設官如上。其首領官則經歷一員,從七品;知事一員,從八品;照磨一員,從九品。吏屬:令史一十二人,譯史四人,通事、知印各二人,奏差一十人,典吏六人。



 管領諸路打捕鷹房民匠等戶總管府,秩正三品。達魯花赤一員,總管一員,同知一員,副總管一員,品秩如上;經歷一員,知事一員,提控案牘一員,照磨一員,令史四人,譯史一人,奏差二人。大德三年置。其屬附見:



 大都等路管民提舉司,達魯花赤一員,同提舉一員,副提舉一員,都目一員。



 大都保定提領所,提領二員,同提領一員,副提領一員,典史一員。



 河間真定提領所,提領二員,同提領一員,副提領一員,典史一員。



 唐州提舉司,達魯花赤一員,提舉一員,同提舉一員,副提舉一員,都目一員。



 南陽鄧州提領所,提領二員,同提領一員,副提領一員,典史一員。



 唐州泌陽提領所,提領二員,同提領一員,副提領一員,典史一員。



 襄陽湖陽提領所,提領二員,同提領一員,副提領一員,典史一員。



 汝寧陳州提領所,提領二員,同提領一員,副提領一員,典史一員。



 河南提舉司,達魯花赤一員,提舉一員,同提舉一員,都目一員。



 汴梁裕州提領所,提領二員,同提領一員,副提領一員,典史一員。



 河南嵩汝提領所,提領二員,同提領一員,副提領一員,典史一員。



 南陽唐州提領所,提領二員,同提領一員,副提領一員,典史一員。



 冀寧提舉司,達魯花赤一員,提舉一員,都目一員。



 冀寧提領所,提領二員,同提領一員,副提領一員,典史一員。



 晉寧提領所,提領二員,同提領一員,副提領一員,典史一員。



 寶昌庫,提領一員,大使一員,掌受金銀甘鐵之課,以待儲運。



 金銀場提領所凡七,梁家寨銀場、明世銀場、密務銀場、寶山銀場、燒炭峪銀場、胡寶峪金場、七寶山甘炭場,俱從七品,每所各設提領一員、同提領一員、副提領一員。



 鐵冶管勾所凡二處,各設管勾一員、同管勾一員、副管勾一員。



 奉宸庫,秩五品,提點四員,副使二員,提控案牘一員,庫子六人,掌中藏寶貨錢帛給納之事。大德元年置。



 廣禧庫,達魯花赤一員,提舉一員,大使一員,副使二員,提控案牘一員,庫子四人,大德八年置,掌收支御膳野物,職視生料庫。



\end{pinyinscope}