\article{志第三十五 百官一}

\begin{pinyinscope}

 王者南面以聽天下之治,建邦啟土,設官分職,其制尚矣。漢、唐以來,雖沿革不同,恆因周、秦之故,以為損益,亦無大相遠。大要欲得賢才用之,以佐天子、理萬民也。



 元太祖起自朔土,統有其眾,部落野處,非有城郭之制,國俗淳厚,非有庶事之繁,惟以萬戶統軍旅,以斷事官治政刑,任用者不過一二親貴重臣耳。及取中原,太宗始立十路宣課司,選儒術用之。金人來歸者,因其故官,若行省,若元帥,則以行省、元帥授之。草創之初,固未暇為經久之規矣。



 世祖即位,登用老成,大新制作,立朝儀,造都邑,遂命劉秉忠、許衡酌古今之宜,定內外之官。其總政務者曰中書省,秉兵柄者曰樞密院,司黜陟者曰御史臺。體統既立,其次在內者,則有寺,有監,有衛,有府;在外者,則有行省,有行臺,有宣慰司,有廉訪司。其牧民者,則曰路,曰府,曰州,曰縣。官有常職,位有常員,其長則蒙古人為之,而漢人、南人貳焉。於是一代之制始備,百年之間,子孫有所憑藉矣。



 大德以後,承平日久,彌文之習勝,而質簡之意微,僥幸之門多,而方正之路塞。官冗於上,吏肆於下,言事者屢疏論列,而朝廷訖莫正之,勢固然也。



 大抵元之建官,繁簡因乎時,得失系乎人,故取其簡牘所載,而論次之。若其因事而置,事已則罷,與夫異教雜流世襲之屬,名類實繁,亦姑舉其大概。作《百官志》。



 三公,太師、太傅、太保各一員,正一品,銀印,以道燮陰陽,經邦國。有元襲其名號,特示尊崇。太祖十二年,以國王置太師一員。太宗即位,建三公,其拜罷歲月,皆不可考。世祖之世,其職常缺,而僅置太保一員。至成宗、武宗而後,三公並建,而無虛位矣。又有所謂大司徒、司徒、太尉之屬,或置,或不置。其置者,或開府,或不開府。而東宮嘗置三師、三少,蓋亦不恆有也。



 中書令一員,銀印,典領百官,會決庶務。太宗以相臣為之,世祖以皇太子兼之。至元十年,立皇太子,行中書令。大德十一年,以皇太子領中書令。延祐三年,復以皇太子行中書令。置屬,監印二人。



 右丞相、左丞相各一員,正一品,銀印,統六官,率百司,居令之次。令缺,則總省事,佐天子,理萬機。國初,職名未創,太宗始置右丞相一員、左丞相一員。世祖中統元年,置丞相一員。二年,復置右丞相二員、左丞相二員。至元二年,增置丞相五員。七年,立尚書省,置丞相二員。八年,罷尚書省,乃置丞相二員。二十四年,復立尚書省,其中書省丞相二員如故。二十九年,以尚書再罷,專任一相。武宗至大二年,復置尚書省,丞相二員,中書丞相二員。四年,尚書省仍歸中書,丞相凡二員,自後因之不易。文宗至順元年,專任右相,其一或置或不置。



 平章政事四員,從一品,掌機務,貳丞相,凡軍國重事,無不由之。世祖中統元年,置平章二員。二年,置平章四員。至元七年,置尚書省,設尚書平章二員。八年,尚書並入中書,平章復設三員。二十三年,詔清冗職,平章汰為二員。二十四年,復尚書省,中書、尚書兩省平章各二員。二十九年,罷尚書省,增中書平章為五員,而一員為商議省事。三十年,又增平章為六員。成宗元貞元年,改商議省事為平章軍國重事。武宗至大二年,再立尚書省,平章三員,中書五員。四年,罷尚書省歸中書,平章仍五員。文宗至順元年,定置四員,自後因之。



 右丞一員,正二品,左丞一員,正二品,副宰相裁成庶務,號左右轄。世祖中統二年,置左、右丞各一員。三年,增為四員。至元七年,立尚書省,中書右丞、左丞仍四員。八年,尚書並入中書省,右、左丞各一員。二十三年,汰冗職,右、左丞如故。二十四年,復立尚書省,右、左丞各一,而中書省缺員。二十八年,復罷尚書省。三十年,設右丞二員,而一員為商議省事。成宗元貞元年,右丞商議省事者,又以昭文大學士與中書省事。武宗至大二年,復立尚書省,右、左丞二員,中書右、左丞五員。四年,罷尚書右、左丞,中書右、左丞止設四員。文宗至順元年,定置右丞一員、左丞一員,而由是不復增損。



 參政二員,從二品,副宰相以參大政,而其職亞於右、左丞。世祖中統元年,始置參政一員。二年,增為二員。至元七年,立尚書省,參政三員。八年,尚書並入中書,參政二員。二十三年,汰冗職,參政二員如故。二十四年,復立尚書省,參政二員,中書參政二員。二十八年,罷尚書省參政。武宗至大二年,復置尚書省,參政二員,中書參政二員。四年,並尚書省入中書,參政三員。文宗至順元年,定參政為二員,自後因之。



 參議中書省事,秩正四品,典左右司文牘,為六曹之管轄,軍國重事咸預決焉。中統元年,始置一員。至元二十二年,累增至六員。大德元年,止置四員,後遂為定額。其治曰參議府,令史二人。



 左司,郎中二員,正五品;員外郎二員,正六品;都事二員,正七品。中統元年,置左右司。至元十五年,分置兩司。左司所掌:吏禮房之科有九,一曰南吏,二曰北吏,三曰貼黃,四曰保舉,五曰禮,六曰時政記,七曰封贈,八曰牌印,九曰好事。知除房之科有五,一曰資品,二曰常選,三曰臺院選,四曰見闕選,五曰別里哥選。戶雜房之科有七,一曰定俸,二曰衣裝,三曰羊馬,四曰置計,五曰田土,六曰太府監,七曰會總。科糧房之科有六,一曰海運,二曰儹運,三曰邊遠,四曰賑濟,五曰事故,六曰軍匠。銀鈔房之科有二,一曰鈔法,二曰課程。應辦房之科有二,一曰飲膳,二曰草料。令史二人,蒙古書寫二十人,回回書寫一人,漢人書寫七人,典吏十五人。



 右司,郎中二員,正五品;員外郎二員,正六品;都事二員,正七品。中統元年,置左右司。至元十五年,分置兩司。右司所掌:兵房之科有五,一曰邊關,二曰站赤,三曰鋪馬,四曰屯田,五曰牧地。刑房之科有六,一曰法令,二曰弭盜,三曰功賞,四曰禁治,五曰枉勘,六曰鬥訟。工房之科有六,一曰橫造軍器,二曰常課段匹,三曰歲賜,四曰營造,五曰應辦,六曰河道。令史二人,蒙古書寫三人,回回書寫一人,漢人書寫一人,典吏五人。



 中書省掾屬:



 監印二人,掌監視省印,有中書令則置。



 知印四人,掌執用省印。



 怯里馬赤四人。



 蒙古必闍赤二十二人,左司十六人,右司六人。



 漢人省掾六十人,左司三十九人,右司二十一人。



 回回省掾十四人,左司九人,右司五人。



 宣使五十人。



 省醫三人。



 玉典赤四十一人。



 斷事官,秩三品,掌刑政之屬。國初,嘗以相臣任之。其名甚重,其員數增損不常,其人則皆御位下及中宮、東宮、諸王各投下怯薛丹等人為之。中統元年,一十六位下置三十一員。至元六年,十七位下置三十四員。七年,十八位下置三十五員。八年,始給印。二十七年,分立兩省,而斷事官隨省並置。二十八年,十八位下置三十六員,並入中書。三十一年,增二員。後定置,自御位下及諸王位下共置四十一員。首領官:經歷一員,知事一員。吏屬:蒙古必闍赤二人,令史一十二人,回回令史一人,怯里馬赤二人,知印二人,奏差八人,典吏一人。



 客省使,秩正五品,使四員,正五品;副使二員,正六品;令史一人,掌直省舍人、宣使等員選舉差遣之事。至元九年,置使二員,一員兼通事,一員不兼。大德元年,增置四員,副二員。直省舍人二員,至元七年始置,後增至三十三員,掌奏事給使差遣之役。檢校官四員,正七品,掌檢校左右司、六部公事程期、文牘稽失之事,書吏六人,大德元年置。



 照磨一員,正八品,掌磨勘左右司錢穀出納、營繕料例,凡數計、文牘、簿籍之事。中統元年,置二員。至元八年,省為一員,典吏八人。



 管勾一員,正八品,掌出納四方文移緘縢啟拆之事,郵遞之程期,曹屬之承受,兼主之。中統元年,置二員。至元三年,定為一員,典吏八人。



 架閣庫管勾二員,正八品,掌庋藏省府籍帳案牘,凡備稽考之文,即掌故之任。至元三年,始置二員,其後增置員數不一。至順初,定為二員,典吏十人。蒙古架閣庫兼管勾一員,典吏二人。回回架閣庫管勾一員,典吏二人。



 吏部,尚書三員,正三品;侍郎二員,正四品;郎中二員,從五品;員外郎二員,從六品,掌天下官吏選授之政令。凡職官銓綜之典,吏員調補之格,封勛爵邑之制,考課殿最之法,悉以任之。世祖中統元年,以吏、戶、禮為左三部,尚書二員,侍郎二員,郎中四員,員外郎六員。至元元年,以吏禮自為一部,尚書三員,侍郎仍二員,郎中仍四員,員外郎三員。三年,復為左三部。五年,又合為吏禮部,尚書仍二員,侍郎、郎中、員外郎各一員。七年,始列尚書六部。吏部尚書一員,侍郎一員,郎中二員,員外郎二員。八年,仍為吏禮部,尚書、侍郎、郎中各一員,員外郎仍二員。十三年,分置吏部,尚書增置七員,侍郎三員,郎中二員,員外郎四員。十九年,尚書裁為二員,侍郎一員,郎中一員,員外郎二員。二十一年,尚書三員,侍郎一員,郎中、員外郎如故。二十三年,定六部尚書、侍郎、郎中、員外郎員額各二員。二十八年,增尚書為三員,主事三員,蒙古必闍赤三人,令史二十五人,回回令史二人,怯里馬赤一人,知印二人,奏差六人,蒙古書寫二人,銓寫五人,典吏一十九人。



 戶部,尚書三員,正三品;侍郎二員,正四品;郎中二員,從五品;員外郎三員,從六品,掌天下戶口、錢糧、田土之政令。凡貢賦出納之經,金幣轉通之法,府藏委積之實,物貨貴賤之直,斂散準駁之宜,悉以任之。中統元年,以吏、戶、禮為左三部,尚書二員,侍郎二員,郎中四員,員外郎六員。至元元年,分立戶部,尚書三員,侍郎、郎中四員,員外郎省為三員。三年,復為左三部。五年,復分為戶部,尚書一員,侍郎、郎中各一員,員外郎又省為二員。七年,始列尚書六部,尚書二員,侍郎二員,郎中二員,員外郎如故。十三年,尚書增置一員,侍郎、郎中、員外郎如故。十九年,郎中、員外郎俱增至四員。二十三年,六部尚書、侍郎、郎中定以二員為額。明年,以戶部所掌,視他部特為繁劇,增置二員。成宗大德五年,省尚書一員,員外郎亦省一員,各設三員,主事八員,蒙古必闍赤七人,令史六十一人,回回令史六人,怯里馬赤一人,知印二人,奏差三十二人,蒙古書寫一人,典吏二十二人,司計官四人。其屬附見於後:



 都提舉萬億寶源庫,掌寶鈔、玉器,至元二十五年始置。都提舉一員,正四品;提舉一員,正五品;同提舉一員,從五品;副提舉一員,從六品;知事一員,從八品,提控案牘一員,司吏二十三人,譯史二人,司庫四十六人,內以色目二人參之。



 都提舉萬億廣源庫,掌香藥、紙劄諸物,設置同上。提控案牘二員,司吏一十二人,譯史一人,司庫一十三人。



 都提舉萬億綺源庫,掌諸色段匹,設置並同上,而副提舉則增一員。提控案牘設三員,後省二員,司吏二十二人,譯史一人,司庫二十六人,內參用色目二人。



 都提舉萬億賦源庫,掌絲綿、布帛諸物,設置並同上。提控案牘二員,其後省一員,司吏一十七人,譯史一人,司庫一十五人,內參用色目二人。



 四庫照磨兼架閣庫,管勾一員,從九品。世祖至元二十八年,以四庫錢帛事繁,始置一員,仍給印。



 提舉富寧庫,至元二十七年始創。提舉一員,從五品;同提舉一員,從六品;副提舉一員,從七品,分掌萬億寶源庫出納金銀之事。吏目一人,其後司吏增至六人,譯史一人,司庫八人。



 諸路寶鈔都提舉司,達魯花赤一員,正四品;都提舉一員,正四品;副達魯花赤一員,正五品;提舉一員,正五品;同提舉一員,從五品;副提舉二員,從六品;知事一員,從八品;照磨一員,從九品。國初,戶部兼領交鈔公事。世祖至元,始設交鈔提舉司,秩正五品。二十四年,改諸路寶鈔都提舉司,升正四品,增副達魯花赤、提控案牘各一員。其後定置已上官員,提控案牘又增一員。設司吏十二人,蒙古必闍赤一人,回回令史一人,奏差七人。



 寶鈔總庫,達魯花赤一員,從五品;大使一員,從五品;副使三員,正七品。世祖至元二十五年,改元寶庫為寶鈔總庫,秩正六品。二十六年,升從五品,增大使、副使,設司庫。其後遂定置已上官員。司吏七人,譯史一人,司庫五十人。



 印造寶鈔庫,達魯花赤一員,正七品;大使二員,從七品;副使二員,正八品。中統四年始置,秩從八品。至元二十四年,升從七品,增達魯花赤一人。其後遂定置已上官員。



 燒鈔東西二庫,達魯花赤一員,正八品;大使一員,從八品;副使一員,從九品。至元元年,始置昏鈔庫,用正九品印,置監燒昏鈔官。二十四年,分立燒鈔東西二庫,秩從八品,各置達魯花赤、大使、副使等員。



 行用六庫。中統元年,初立中都行用庫,秩從七品,提領一員,從七品;大使一員,從八品;副使一員,從九品。至元二十四年,京師改置庫者三:曰光熙,曰文明,曰順承。因城門之名為額。二十六年,又置三庫:曰健德,曰和義,曰崇仁。並因城門以為名。



 大都宣課提舉司,掌諸色課程,並領京城各市。提舉二員,從五品;同提舉一員,從六品;副提舉一員,從七品;提控案牘一員,司吏六人。世祖至元十九年,並大都舊城兩稅務為大都稅課提舉司。至武宗至大元年,改宣課提舉司。其屬四:



 馬市、豬羊市,秩從七品。提領一員,從七品;大使一員,從八品;副使一員,從九品。世祖至元三十年始置。



 牛驢市、果木市,品秩、設官同上。



 角蟹市,大使一員,副使二員。至大元年始置。



 煤木所,提領一員,從八品;大使一員,從九品;副使一員。至元二十二年始置。



 大都酒課提舉司,掌酒醋榷酤之事,至元十九年始置。提舉一員,從五品;同提舉二員,從六品;副提舉二員,從七品;提控案牘二員,司吏五人。二十八年,省同提舉一員、副提舉一員,餘如故。



 抄紙坊,提領一員,正八品;大使一員,從八品;副使二員,從九品。中統四年始置,用九品印,止設大使、副使各一員。至元二十七年,升正八品,增置提領、副使各一員。



 印造茶鹽等引局,大使一員,副使一員,至元二十四年置,掌印造腹裏、行省鹽、茶、礬、鐵等引。仍置攢典、庫子各一人。



 右以上屬戶部。其萬億四庫,國初以太府掌內帑之出納,既設左藏等庫,而國計之領在戶部,仍置萬億等庫,為收藏之府。中統元年,置庫官六員,而未有品秩俸給。至元十六年,始為提舉萬億庫,秩正五品。二十四年,改升都提舉萬億庫,秩正四品。二十五年,分立四庫,以分掌出納。至二十七年,又別立富寧庫焉。



 京畿都漕運使司,秩正三品。運使二員,正三品;同知二員,正四品;副使二員,正五品;判官二員,正六品;經歷一員,正七品;知事一員,從八品,提控案牘兼照磨二員,掌凡漕運之事。世祖中統二年,初立軍儲所,尋改漕運所。至元五年,改漕運司,秩五品。十二年,改都漕運司,秩五品。十九年,改京畿都漕運使司,秩正三品。二十四年,內外分立兩運司,而京畿都漕運司之額如舊。止領在京諸倉出納糧斛,及新運糧提舉司站車攢運公事。省同知、運判、知事各一員,而押綱官隸焉。延祐六年,增同知、副使、運判各一員。其後定置官員已上正官各二員,首領官四員。吏屬:令史二十一人,譯史二人,回回令史一人,通事一人,知印二人,奏差一十六人,典吏二人。其屬二十有四:



 新運糧提舉司,秩正五品。至元十六年始置,管站車二百五十輛,隸兵部。開設運糧壩河,改隸戶部。定置達魯花赤一員,都提舉一員,同提舉二員,副提舉一員,吏目一員,司吏八人,奏差十二人。



 京師二十二倉,秩正七品。



 萬斯北倉,中統二年置。萬斯南倉,至元二十四年置。千斯倉,中統二年置。永平倉,至元十六年置。永濟倉,至元四年置。惟億倉,既盈倉,大有倉,並系皇慶元年置。屢豐倉,積貯倉。並系皇慶元年增置。



 已上十倉,每倉各置監支納一員,正七品;大使二員,從七品;副使二員,正八品。



 豐穰倉,皇慶元年置。廣濟倉,皇慶元年置。廣衍倉,至元二十九年置。大積倉,至元二十八年置。既積倉,盈衍倉,至元二十六年置。相因倉,中統二年置。順濟倉。至元二十九年置。



 已上八倉,每倉各置監支納一員,正七品;大使一員,從七品;副使二員,正八品。



 通濟倉,中統二年置。廣貯倉,至元四年置。豐潤倉,至元十六年置。



 豐實倉。



 已上四倉,每倉各置監支納一員,正七品;大使一員,從七品;副使一員,正八品。



 通惠河運糧千戶所,秩正五品,掌漕運之事。至元三十一年始置,中千戶一員,中副千戶二員。



 都漕運使司,秩正三品,掌御河上下至直沽、河西務、李二寺、通州等處人贊運糧斛。至元二十四年,自京畿運司分立都漕運司,於河西務置總司,分司臨清。運使二員,正三品;同知二員,正四品;副使二員,正五品;運判三員,正六品;經歷一員,從七品;知事一員,從八品。提控案牘二員,內一員兼照磨,司吏三十三人,通事、譯史各一人,奏差一十六人,典吏一人。其屬七十有五:



 河西務十四倉,秩正七品。



 永備南倉,永備北倉,廣盈南倉,廣盈北倉,充溢倉。



 已上五倉,各置監支納一員,正七品;大使二員,從七品;副使二員,正八品。



 崇墉倉,大盈倉,大京倉,大稔倉,足用倉,豐儲倉,豐積倉,恆足倉,既備倉。



 已上九倉,各置監支納一員,正七品;大使一員,從七品;副使一員,正八品。



 通州十三倉,秩正七品。



 有年倉,富有倉,廣儲倉,盈止倉,及秭倉,乃積倉,樂歲倉,慶豐倉,延豐倉。



 已上九倉,各置監支納一員,正七品;大使二員,從七品;副使二員,正八品。



 足食倉,富儲倉,富衍倉,及衍倉。



 已上四倉,各置監支納一員,正七品;大使二員,從七品;副使一員,正八品。



 河倉一十有七,用從七品印。



 館陶倉,舊縣倉,陵州倉,傅家池倉。



 已上各置監支納一員,從七品;大使一員,從八品;副使一員。



 秦家渡倉,尖塚西倉,尖塚東倉,長蘆倉,武強倉,夾馬營倉,上口倉,唐宋倉,唐村倉,安陵倉,四柳樹倉,淇門倉,伏恩倉。



 已上各置監支納一員,從八品;大使一員,從九品;副使一員。



 直沽廣通倉,秩正七品,大使一員。



 滎陽等綱,凡三十:曰濟源,曰陵州,曰獻州,曰白馬,曰滏陽,曰完州,曰河內,曰南宮,曰沂莒,曰霸州,曰東明,曰獲嘉,曰鹽山,曰武強,曰膠水,曰東昌,曰武安,曰汝寧,曰修武,曰安陽,曰開封,曰儀封,曰蒲臺,曰鄒平,曰中牟,曰膠西,曰衛輝,曰浚州,曰曹濮州,每綱皆設押綱官二員,計六十員。秩正八品。每編船三十只為一綱。船九百餘隻,運糧三百餘萬石,船戶八千餘戶,綱官以常選正八品為之。



 檀景等處採金鐵冶都提舉司,秩正四品。提舉一員,正四品;同提舉一員,正五品;副提舉一員,從六品,掌各冶採金煉鐵,榷貨以資國用。國初,中統始置景州提舉司,管領景州、灤陽、新匠三冶。至元十四年,又置檀州提舉司,管領雙峰、暗峪、大峪五峰等冶。大德五年,檀州、景州三提舉司,並置檀州等處採金鐵冶都提舉司,而灤陽、雙峰等冶悉隸焉。他如河東、山西、濟南、萊蕪等處鐵冶提舉司,及益都、般陽等處淘金總管府,其沿革蓋不一也。



 大都河間等路都轉運鹽使司,秩正三品,掌場灶榷辦鹽貨,以資國用。使二員,正三品;同知一員,正四品;副使一員,正五品;運判二員,正六品。首領官:經歷一員,從七品;知事一員,從八品;照磨一員,從九品。國初,立河間稅課達魯花赤清滄鹽使所,後創立運司,立提舉鹽榷所,又改為河間路課程所,提舉滄清課鹽使所。中統三年,改都提領拘榷滄清課鹽所。至元二年,以刑部侍郎、右三部郎中兼滄清課鹽使司,尋改立河間都轉運鹽使司,立清、滄課三鹽司。十二年,改為都轉運使司。十九年,以戶部尚書行河間等路都轉運使司事,尋罷,改立清、滄二鹽使司。二十三年,改立河間等路都轉運司。二十七年,改令戶部尚書行河間等路都轉運使司事。二十八年,改河間等路都轉運司。延祐六年,頒分司印,巡行郡邑,以防私鹽之弊。



 鹽場二十二所,每場設司令一員,從七品;司丞一員,從八品。辦鹽各有差。



 利國場,利民場,海豐場,阜民場,阜財場,益民場,潤國場,海阜場,海盈場,海潤場,嚴鎮場,富國場,興國場,厚財場,豐財場,三叉沽場,蘆臺場,越支場,石碑場,濟民場,惠民場,富民場。



 山東東路轉運鹽使司,品秩、職掌同上,運判止一員。國初,始置益都課稅所,管領山東鹽場,以總鹽課,後改置運司。中統四年,詔以中書左右部兼諸路都轉運司。至元二年,命有司兼辦其課,改立山東轉運司。至元十二年,改立都轉運司。延祐五年,以鹽法澀滯,降分司印,巡行各場,督收課程,罷膠萊鹽司所屬鹽場。



 鹽場一十九所,每場設司令一員,從七品;司丞一員,從八品;管勾一員,從八品。



 永利場,寧海場,官臺場,豐國場,新鎮場,豐民場,富國場,高家港場,永阜場,利國場,固堤場,王家岡場,信陽場,濤洛場,石河場,海滄場,行村場,登寧場,西由場。



 河東陜西等處轉運鹽使司,品秩、職掌同前,運判增一員。國初,設平陽府以徵課程之利。中統二年,改置轉運司,置提舉解鹽司。至元二年,罷運司,命有司掌其務,尋復置轉運司。二十二年,立陜西都轉運司,諸色稅課悉隸焉。二十九年,置鹽運司,專掌鹽課,其餘課稅歸有司,解鹽司亦罷。延祐六年,更為河東陜西等處都轉運鹽使司,隸省部。其屬三:



 解鹽場,管勾一員,正九品;同管勾一員,從九品。



 河東等處解鹽管民提領所,正提領一員,從八品;副提領一員,從九品。



 安邑等處解鹽管民提領所,正提領一員,從八品;副提領一員,從九品。



 禮部,尚書三員,正三品;侍郎二員,正四品;郎中二員,從五品;員外郎二員,從六品,掌天下禮樂、祭祀、朝會、燕享、貢舉之政令。凡儀制損益之文,符印簡冊之信,神人封謚之法,忠孝貞義之褒,送迎聘好之節,文學僧道之事,婚姻繼續之辨,音藝膳供之物,悉以任之。世祖中統元年,以吏、戶、禮為左三部,置尚書二員,侍郎二員,郎中四員,員外郎六員,總領三部之事。至元元年,分立為吏禮部,尚書三員,侍郎仍二員,郎中仍四員,員外郎四員。七年,別立禮部,尚書三員,侍郎一員,郎中二員,員外郎如舊。明年,又合為吏禮部。十三年,又別為禮部。二十三年,六部尚書、侍郎、郎中、員外郎定以二員為額。成宗元貞元年,復增尚書一員,領會同館事。主事二員,蒙古必闍赤二人,令史一十九人,回回令史二人,怯里馬赤一人,知印二人,奏差十二人,典吏三人。其屬附見:



 左三部照磨所,秩正八品,照磨一員,掌吏、戶、禮三部錢穀計帳之事。典吏八人。



 侍儀司,秩正四品,掌凡朝會、即位、冊后、建儲、奉上尊號及外國朝覲之禮。至元八年始置。左右侍儀奉御二員,禮部侍郎知侍儀事一員,引進使知侍儀事一員,左右侍儀使二員,左右直侍儀使二員,左右侍儀副使二員,左右侍儀僉事二員,引進副使、侍儀令、承奉班都知、尚衣局大使各一員。十二年,省左侍儀奉御,通曰左右侍儀。省引進副使及侍儀令、尚衣使等員,改置通事舍人十四員。三十年,減通事舍人七員為侍儀舍人。大德十一年,升秩正三品。至大二年,置典簿一員。延祐七年,定置侍儀使四員。至治元年,增置通事舍人六員、侍儀舍人四員。其後定置侍儀使四員,正三品;引進使知侍儀事二員,正四品。首領官:典簿一員,從七品。屬官:承奉班都知一員,正七品;通事舍人一十六員,從七品;侍儀舍人十四員,從九品。吏屬:令史二人,譯史一人,通事一人,知印一人。其屬法物庫,秩五品,掌大禮法物。提點一員,從五品;大使一員,從六品;副使一員,從七品;直長二員,正八品。



 拱衛直都指揮使司,秩從四品,掌控鶴六百餘戶,及儀衛之事。至元三年始置。都指揮使一員,副使一員,鈐轄一員,提控案牘一員。十六年,升正三品,降虎符,增置達魯花赤一員,隸宣徽院。二十年,復為從四品。二十五年,歸隸禮部。元貞元年,復升正三品。皇慶元年,置經歷一員。二年,改鈐轄為僉事。至順二年,撥隸侍正府,定置達魯花赤一員,正三品;都指揮使四員,正三品;副指揮使二員,從三品;僉事二員,正四品。首領官:經歷一員,從七品;知事一員,從八品。吏屬:令史四人,譯史一人,通事、知印各一人,奏差二人。其屬控鶴百戶所,秩從七品。色目百戶一十三員,漢人百戶一十三員。總十三所。



 儀從庫,秩從七品,掌收儀衛器仗。大使一員,從七品;副使一員,從八品。



 儀鳳司,秩正四品,掌樂工、供奉、祭饗之事。至元八年,立玉宸院,置樂長一員,樂副一員,樂判一員。二十年,改置儀鳳司,隸宣徽院,置大使、副使各一員,判官三員。二十五年,歸隸禮部,省判官三員。三十一年,置達魯花赤一員,副使一員。大德十一年,改升玉宸樂院,秩從二品,置院使、副使、僉事、同僉、院判。至大四年,復為儀鳳司,秩正三品。延祐七年,降從三品。定置大使五員,從三品;副使四員,從四品。首領官:經歷一員,從七品;知事一員,從八品。吏屬:令史二人,譯史、通事、知印各一人。其屬五:



 雲和署,秩正七品,掌樂工調音律及部籍更番之事。至元十二年始置。至大二年,撥隸玉宸樂院。皇慶元年,升正六品。二年,升從五品。署令二員,署丞二員,管勾二員,協音一員,協律一員,書史二人,書吏四人,教師二人,提控四人。



 安和署,秩正七品,職掌與雲和同。至元十三年始置。皇慶二年,升從五品。署令二員,署丞二員,管勾二員,協音一員,協律一員,書史二人,書吏四人,教師二人,提控四人。



 常和署,初名管勾司,秩正九品,管領回回樂人。皇慶元年初置。延祐三年,升從六品。署令一員,署丞二員,管勾二員,教師二人,提控二人。



 天樂署,初名昭和署,秩從六品,管領河西樂人。至元十七年始置。大德十一年,升正六品。至大四年,改為天樂署。皇慶元年,升從五品。署令二員,署丞二員,管勾二員,協音一員,協律一員,書史二人,書吏四人,教師二人,提控四人。



 廣樂庫,秩從九品,掌樂器等物。大使一員,副使一員。皇慶元年始置。



 教坊司,秩從五品,掌承應樂人及管領興和等署五百戶。中統二年始置。至元十二年,升正五品。十七年,改提點教坊司,隸宣徽院,秩正四品。二十五年,隸禮部。大德八年,升正三品。延祐七年,復正四品。達魯花赤一員,正四品;大使三員,正四品;副使四員,正五品;知事一員,從八品。令史四人,譯史、知印、奏差各二人,通事一人。其屬三:



 興和署,秩從六品。署令二員,署丞二員,管勾二員。



 祥和署,秩從六品。署令二員,署丞二員,管勾二員。



 廣樂庫,秩從九品。大使一員,副使一員。



 會同館,秩從四品,掌接伴引見諸番蠻夷峒官之來朝貢者。至元十三年始置。二十五年罷之。二十九年復置。元貞元年,以禮部尚書領館事,遂為定制。禮部尚書領會同館事一員,正三品;大使二員,正四品;副使二員,從六品。提控案牘一員,掌書四人,蒙古必闍赤一人,典給官八人。其屬有收支諸物庫,秩從九品,大使一員,副使一員。至元二十九年,以四賓庫改置。



 鑄印局,秩正八品,掌凡刻印銷印之事。大使一員,副使一員,直長一員。至元五年始置。



 白紙坊,秩從八品,掌造詔旨宣敕紙劄。大使一員,副使一員。至元九年始置。



 掌薪司,秩正七品。司令一員,正七品;司丞二員,正八品;典吏一人。



 兵部,尚書三員,正三品;侍郎二員,正四品;郎中二員,從五品;員外郎二員,從六品,掌天下郡邑郵驛屯牧之政令。凡城池廢置之故,山川險易之圖,兵站屯田之籍,遠方歸化之人,官私芻牧之地,駝馬、牛羊、鷹隼、羽毛、皮革之徵,驛乘、郵運、祗應、公廨、皁隸之制,悉以任之。世祖中統元年,以兵、刑、工為右三部,置尚書二員,侍郎二員,郎中五員,員外郎五員,總領三部之事。至元元年,別置工部,以兵刑自為一部,尚書四員,侍郎三員,郎中如舊,員外郎五員。三年,並為右三部。五年,復為兵刑部,尚書二員,省侍郎二員,郎中如故,員外郎一員。七年,始列六部,尚書一員,侍郎仍舊,郎中一員,員外郎仍一員。明年,又合為兵刑部。十三年,復析兵部。二十三年,定尚書、侍郎、郎中、員外郎以二員為額。至治三年,增尚書一員。主事二員,蒙古必闍赤二人,令史十四人,回回令史一人,怯里馬赤一人,知印二人,奏差八人,典吏三人。其屬附見:



 大都陸運提舉司,秩從五品,掌兩都陸運糧斛之事。至元十六年,始置運糧提舉司。延祐四年,改今名。提舉二員,從五品;副提舉一員,從七品。吏目一員,司吏六人,委差一十人。海王莊、七裏莊、魏家莊、臘八莊四所,各設提領一人,用從九品印。



 管領隨路打捕鷹房民匠總管府,秩從三品。達魯花赤一員,總管一員,副總管二員,經歷、知事各一員,提控案牘一員,吏屬令史六人。初,太祖以隨路打捕鷹房民戶七千餘戶撥隸旭烈大王位下。中統二年始置。至元十二年,阿八合大王遣使奏歸朝廷,隸兵部。



 管領本投下大都等路打捕鷹房諸色人匠都總管府,秩正三品,掌哈贊大王位下事。大德八年始置,官吏皆王選用。至大四年,省並衙門,以哈兒班答大王遠鎮一隅,別無官屬,存設不廢。定置府官,達魯花赤二員,總管一員,同知一員,副總管一員,知事一員,提控案牘一員,令史四人,譯史二人,奏差二人,典吏一人。其屬東局織染提舉司,秩從五品。達魯花赤一員,提舉一員,副達魯花赤一員,副提舉一員,提控案牘一員,司吏二人。



 隨路諸色民匠打捕鷹房等戶都總管府,秩從三品。達魯花赤一員,總管一員,同知一員,經歷一員,知事一員,提控案牘兼照磨一員,令史六人,譯史一人,知印通事一人,奏差二人,掌別吉大營盤事及管領大都路打捕鷹房等戶。至元三十年置。延祐四年,升正三品。



 管領本位下打捕鷹房民匠等戶都總管府,秩正三品。達魯花赤一員,總管一員,副達魯花赤一員,同知一員,副總管一員,判官一員,經歷一員,知事一員,提控案牘兼照磨一員,令史六人,譯史、通事、知印各一人,掌別吉大營盤城池阿哈探馬兒一應差發、薛徹干定王位下事。泰定元年始置。



 刑部,尚書三員,正三品;侍郎二員,正四品;郎中二員,從五品;員外郎二員,從六品,掌天下刑名法律之政令。凡大闢之按覆,系囚之詳讞,孥收產沒之籍,捕獲功賞之式,冤訟疑罪之辨,獄具之制度,律令之擬議,悉以任之。世祖中統元年,以兵、刑、工為右三部,置尚書二員,侍郎二員,郎中五員,員外郎五員。以郎中、員外郎各一員,專署刑部。至元元年,析置工部,而兵刑仍為一部。尚書四員,侍郎仍二員,郎中四員,員外郎置五員。三年,復為右三部。七年,始別置刑部。尚書一員,侍郎一員,郎中一員,員外郎二員。八年,改為兵刑部。十三年,又為刑部。二十三年,六部尚書、侍郎、郎中、員外郎定以二員為額。大德四年,尚書增置一員。其首領官則主事三員。吏屬則蒙古必闍赤四人,令史三十人,回回令史二人,怯里馬赤一人,知印二人,奏差十人,書寫三人,典吏七人。其屬附見:



 司獄司,司獄一員,正八品;獄丞一員,正九品;獄典一人。初以右三部照磨兼刑部系獄之任,大德七年始置專官。部醫一人,掌調視病囚。



 司籍所,提領一員,同提領一員。至元二十年,改大都等路斷沒提領所為司籍所,隸刑部。



 工部,尚書三員,正三品;侍郎二員,正四品;郎中二員,從五品;員外郎二員,從六品,掌天下營造百工之政令。凡城池之修浚,土木之繕葺,材物之給受,工匠之程式,銓注局院司匠之官,悉以任之。世祖中統元年,右三部置尚書二員,侍郎二員,郎中五員,員外郎五員,內二員專署工部事。至元元年,始分立工部。尚書四員,侍郎三員,郎中四員,員外郎五員。三年,復合為右三部。七年,仍自為工部。尚書二員,侍郎仍二員,郎中三員,員外郎如舊。二十三年,定尚書、侍郎、郎中、員外郎各以二員為額。明年,以曹務繁冗,增尚書二員。二十八年,省尚書一員。首領官:主事五員。蒙古必闍赤六人,令史四十二人,回回令史四人,怯里馬赤一人,知印一人,奏差三十人,蒙古書寫一人,典吏七人。又司程官四員,右三部照磨一員,典吏七人。其屬附見:



 左右部架閣庫,秩正八品。管勾二員,典吏十二人,掌六部文卷簿籍架閣之事。中統元年,左右部各置。二十三年,並為左右部架閣庫。



 諸色人匠總管府,秩正三品,掌百工之技藝。至元十二年始置,總管、同知、副總管各一員。十六年,置達魯花赤一員,增同知、副總管各一員。二十八年,省同知一員。三十年,省副總管一員。後定置達魯花赤一員,總管一員,同知二員,副總管二員,經歷一員,知事一員,提控案牘一員,令史五人,譯史一人,奏差四人。其屬十有一:



 梵像提舉司,秩從五品。提舉一員,同提舉一員,副提舉一員,吏目一員,董繪畫佛像及土木刻削之工。至元十二年,始置梵像局。延祐三年,升提舉司,設今官。



 出蠟局提舉司,秩從五品。提舉一員,同提舉一員,副提舉一員,吏目一員,掌出蠟鑄造之工。至元十二年,始置局。延祐三年,升提舉司,設今官。



 鑄瀉等銅局,秩從七品。大使一員,副使一員,掌鑄瀉之工。至元十年,始置官三員。二十八年,省管勾一員,後定置二員。



 銀局,秩從七品。大使一員,直長一員,掌金銀之工。至元十二年始置。



 鑌鐵局,秩從八品。大使一員,掌鏤鐵之工。至元十二年始置。



 瑪瑙玉局,秩從八品。直長一員。掌琢磨之工。至元十二年始置。



 石局,秩從七品。大使一員,管勾一員,董攻石之工。至元十二年始置。



 木局,秩從七品。大使一員,直長一員,董攻木之工。至元十二年始置。



 油漆局,副使一員,用從七品印,董髹漆之工。至元十二年始置。



 諸物庫,秩正九品。提領一員,副使一員,掌出納諸物之事。至元十二年始置。



 管領隨路人匠都提領所,提領一員,大使一員,俱受省檄,掌工匠詞訟之事。至元十二年始置。



 諸司局人匠總管府,秩正三品。達魯花赤一員,總管一員,副達魯花赤一員,同知一員,副總管一員,經歷一員,知事一員,提控案牘一員,令史四人,領兩都金銀器盒及符牌等一十四局事。至元十四年置。二十四年,以八局改隸工部及金玉府,止領五局一庫,掌氈毯等事。其屬有六:



 收支庫,秩正九品。大使一員,掌出納之物。



 大都氈局,秩從七品。大使、副使各一員,管人匠一百二十有五戶。



 大都染局,秩從九品。大使一員,管人匠六千有三戶。



 上都氈局,秩從五品。大使一員,副使一員,管人匠九十有七戶。



 隆興氈局,大使一員,副使一員,管人匠一百戶。



 剪毛花毯蠟布局,大使一員,副使一員,管人匠一百一十有八戶。



 提舉右八作司,秩正六品。提舉二員,同提舉一員,副提舉一員,吏目一人,司吏九人,司庫十三人,譯史一人,秤子一人,掌出納內府漆器、紅甕、捎只等,並在都局院造作鑌鐵、銅、鋼、鍮石,東南簡鐵,兩都支持皮毛、雜色羊毛、生熟斜皮、馬牛等皮、鬃尾、雜行沙裏陀等物。中統三年,始置提領八作司,秩正九品。至元二十五年,改升提舉八作司,秩正六品。二十九年,以出納委積,分為左右兩司。



 提舉左八作司,秩正六品,掌出納內府氈貨、柳器等物。其設置官員同上。



 諸路雜造總管府,秩正三品。至元元年,改提領所為提舉司。十四年,又改工部尚書行諸路雜造局總管府。定置達魯花赤一員,總管一員,同知一員,副總管一員,知事一員,提控案牘一員,令史六人,譯史一人。其屬二:



 簾網局,大使一員,副使一員,並受省劄。至元元年始置。



 收支庫,大使一員,副使一員,至元三十年始置。



 茶迭兒局總管府,秩正三品,管領諸色人匠造作等事。憲宗朝置。至元十六年,始設總管一員。二十七年,置同知一員。後定置府官,達魯花赤一員,總管一員,同知一員,知事一員,提控案牘一員,司吏四人。其屬二:



 諸司局,用從七品印。提領一員,相副官二員,中統三年始置。



 收支庫,提領一員,大使、副使各一員。掌造作出納之物。



 大都人匠總管府,秩從三品。至元六年始置。達魯花赤一員,總管一員,同知一員,經歷一員,提控案牘一員,令史十人,通事一人。其屬四:



 繡局,用從七品印。大使一員,副使一員,掌繡造諸王百官段匹。



 紋錦總院,提領一員,大使一員,副使一員,掌織諸王百官段匹。



 涿州羅局,提領一員,大使一員,掌織造紗羅段匹。



 尚方庫,提領一員,大使、副使各一員,掌出納絲金顏料等物。



 隨路諸色民匠都總管府,秩正三品,掌仁宗潛邸諸色人匠。延祐六年,撥隸崇祥院,後又屬將作院。至治三年,歸隸工部。後定置達魯花赤一員,總官一員,同知一員,副總管一員,經歷一員,知事一員,提控案牘一員,照磨一員,令史八人,譯史二人,知印、通事各一人,奏差四人。其屬五:



 織染人匠提舉司,秩從七品。至大二年設。達魯花赤一員,提舉一員,同提舉一員,副提舉一員,吏目一員。



 雜造人匠提舉司,秩從七品,設置官屬同上。



 大都諸色人匠提舉司,秩從五品。達魯花赤一員,提舉一員,同提舉一員,副提舉一員,吏目一員。



 大都等處織染提舉司,秩從五品,管阿難答王位下人匠一千三百九十八戶。達魯花赤一員,提舉一員,同提舉一員,副提舉一員,吏目一員。



 收支諸物庫,秩從七品。提領一員,大使一員,副使一員,庫子二人。



 提舉都城所,秩從五品。提舉二員,同提舉二員,副提舉二員,吏目一員,照磨一員,掌修繕都城內外倉庫等事。至元三年置。其屬一:



 左右廂,官四員,用從九品印。至元十三年置。



 受給庫,秩正八品。提領一員,大使一員,副使一員,掌京城內外營造木石等事。至元十三年置。



 符牌局,秩正八品。大使一員,副使一員,直長一員,掌造虎符等。至元十七年置。



 旋匠提舉司,秩從五品。提舉一員,副提舉一員。至元九年置。



 撒答剌欺提舉司,秩正五品。提舉一員,副提舉一員,提控案牘一員。至元二十四年,以札馬剌丁率人匠成造撒答剌欺,與絲紬同局造作,遂改組練人匠提舉司為撒答剌欺提舉司。



 別失八里局,秩從七品。大使一員,副使一員,掌織造御用領袖納失失等段。至元十三年始置。



 忽丹八里局,大使一員,給從七品印。至元三年置。



 平則門窯場,提領一員,大使一員,副使一員,給從六品印。至元十三年置。



 光熙門窯場,提領一員,大使一員,副使一員,給從八品印。至元二十五年置。



 大都皮貨所,提領一員,大使一員,副使一員,用從九品印。至元二十九年置。



 通州皮貨所,提領一員,大使一員,副使一員,用從九品印。延祐六年置。



 晉寧路織染提舉司,提舉一員,照略案牘一員,其屬:



 提領所一,系官織染人匠局一,雲內人匠東、西局二,本路人匠局一,河中府、襄陵、翼城、潞州、隰州、澤州、雲州等局七。每局各設提領一員,副提領一員,惟澤州、雲州則止設提領一員。



 冀寧路織染提舉司、真定路織染提舉司,各置提舉一員,同提舉一員,副提舉一員,照略案牘一員。其屬二:



 開除局,大使一員,副使一員,照略案牘一員。



 真定路紗羅兼雜造局,大使一員,副使一員。



 南宮、中山織染提舉司,各設提舉一員,同提舉、副提舉一員,照略案牘一員。



 中山劉元帥局,大使一員,副使一員。



 中山察魯局,大使一員,副使一員。



 深州織染局,大使一員,副使一員,照略案牘一員。



 深州趙良局,大使一員,副使一員。



 弘州人匠提舉司,提舉一員,同提舉一員,副提舉一員,照略案牘一員。



 納失失毛段二局,院長一員。



 雲內州織染局,大使一員,副使一員,照略案牘一員。



 大同織染局,大使一員,副使一員,照略案牘一員。



 朔州毛子局,大使一員。



 恩州織染局,大使一員,副使一員,照略案牘一員。



 恩州東昌局,提領一員。



 保定織染提舉司,提舉一員,同提舉一員,副提舉一員,照略案牘一員。



 大名人匠提舉司,提舉一員,同提舉一員,副提舉一員,照略案牘一員。



 永平路紋錦等局提舉司,提舉一員,同提舉一員,副提舉一員,照略案牘一員。



 大寧路織染局,大使一員,副使一員,照略案牘一員。



 雲州織染提舉司,提舉一員,同提舉一員,副提舉一員,照略案牘一員。



 順德路織染局,大使一員,副使一員,照略案牘一員。



 彰德路織染人匠局,大使一員,副使一員,照略案牘一員。



 懷慶路織染局,大使一員,副使一員,照略案牘一員。



 別失八里局,官一員。



 宣德府織染提舉司,提舉一員,同提舉一員,副提舉一員,照略案牘一員。



 東聖州織染局,院長一員,局副一員。



 宣德八魯局,提領一員,副使一員。



 東平路畽局,直長一員。



 興和路蕁麻林人匠提舉司,提舉一員,同提舉一員,副提舉一員,照略案牘一員。



 陽門天城織染局,提領一員,副使一員,照磨案牘一員。



 巡河提領所,提領二員,副提領一員。



\end{pinyinscope}