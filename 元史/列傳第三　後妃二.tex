\article{列傳第三 後妃二}

\begin{pinyinscope}

 睿宗顯懿莊聖皇后,名唆魯和帖尼,怯烈氏,生子憲宗、世祖,相繼為帝。至元二年,追上尊謚莊聖皇后,升祔睿宗廟。至大二年十二月,加謚顯懿莊聖皇后。三年十月,又上玉冊,其文曰:「祖功宗德,稱誄於天。內則閫儀,受成於廟。行之大者名必顯,恩之隆者報則豐。上以增佐定之光,下以伸遹追之孝。欽惟莊聖皇后,英明溥博,聖善柔嘉。尊儷景襄,陰教純被。逮事光獻,婦職勤修。勛聿著於承天,祥兩占於夢日。跡聖緒洪源之有漸,知深仁厚澤之無垠。玄符肇自塗山,顧前徽之未稱;蒼籙興於文母,豈後嗣之能忘。是用參考彞經,丕揚景鑠。敷繹寶慈之誼,形容青史之規。謹遣攝太尉某奉玉冊玉寶,加上尊謚曰顯懿莊聖皇后。伏惟睿靈,昭垂鑒格。禮嚴閟宮,樂歌夷則。億萬斯年,承休無斁。」



 裕宗徽仁裕聖皇后伯藍也怯赤,一名闊闊真,弘吉剌氏,生順宗、成宗。先是世祖出田獵,道渴,至一帳房,見一女子緝駝茸,世祖從覓馬湩。女子曰:「馬湩有之,但我父母諸兄皆不在,我女子難以與汝。」世祖欲去之。女子又曰:「我獨居此,汝自來自去,於理不宜。我父母即歸,姑待之。」須臾果歸。出馬湩飲世祖。世祖既去,嘆息曰:「得此等女子為人家婦,豈不美耶!」後與諸臣謀擇太子妃,世祖俱不允。有一老臣嘗知向者之言,知其未許嫁,言於世祖。世祖大喜,納為太子妃。後性孝謹,善事中宮,世祖每稱之為賢德媳婦。侍昭睿順聖皇后,不離左右,至溷廁所用紙,亦以面擦,令柔軟以進。一日,裕宗有病,世祖往視,見床上設織金臥褥。世祖慍而語之曰:「我嘗以汝為賢,何乃若此耶?」後跪答曰:「常時不曾敢用,今為太子病,恐有濕氣,因用之。」即時徹去。



 世祖崩,成宗至上都,諸王畢會。先是,御史中丞崔彧得玉璽於木華黎國王曾孫世德家,其文曰「受命於天,既壽永昌」,上之於後,至是,後手授成宗。即皇帝位,尊後為皇太后,冊文曰:「自家而國,治道必有所先;立愛惟親,君德莫先於孝。況恩深於鞠我,而禮重於正名。歷代以來,令儀可考。人子之職所在,天下之母宜尊。恭惟聖母,聖善本乎天資,靜專法乎地道。上以奉宗祏之重,下以敘倫紀之常。助我前人,守《卷耳》憂勤之志;保予沖子,成《思齊》雍肅之風。肆神器之有歸,知孫謀之素定。畀付雖由於歷數,規摹一出於庭闈。是用率籲眾心,章明巨度,不勝拳拳大願。謹奉冊寶,上尊稱曰皇太后。伏惟長信穆穆,周宗綿綿。備《洛書》之錫福,粲慈極之儀天。瑤圖寶運,於萬斯年。」命設官屬,置徽政院。後院官有受獻浙西田七百頃,籍於位下,太后曰:「我寡居婦人,衣食自有餘,況江南率土,皆國家所有,我曷敢私之。」即命中書省盡易院官之受獻者。後之弟欲因後求官,後語之曰:「若欲求官耶?汝自為之,勿以累我也。」其後,弟果被黜,人皆服後之先見。大德四年二月崩,祔葬先陵,謚曰裕聖皇后,升祔裕宗廟。至大三年十月,又追尊謚曰徽仁裕聖皇后。



 顯宗宣懿淑聖皇后,名普顏怯裏迷失,弘吉剌氏,顯宗居晉邸,納為元妃,生泰定帝。泰定元年,追尊宣懿淑聖皇后,其冊文曰:「祗纘皇圖,方弘仁孝之化;追崇聖母,永懷鞠育之恩。匪建鴻名,疇彰厚德。欽惟皇妣晉王妃弘吉剌氏,淑侔周姒,賢邁虞嬪。儷我先王,恪守肇基之地;昭其懿範,益恢正始之風。順坤道以承乾,炯月輝以溯日。陰功久積,衍聖緒於無疆;神器攸歸,知慶源之有自。仰徽音之如在,慨至養之莫加。聿選休辰,爰修縟典。謹遣攝太尉某奉玉冊玉寶,上尊謚曰宣懿淑聖皇后。伏惟淑靈在上,式垂鑒臨,合享太宮,永錫繁祉。」升祔皇考顯宗廟。天歷初,復祧顯宗廟祀。



 順宗照獻元聖皇后,名答己,弘吉剌氏,按陳孫渾都帖木兒之女。裕宗居燕邸及潮河,順宗俱在侍,稍長,世祖賜女侍郭氏,後乃納後為妃,生武宗及仁宗。大德九年,成宗不豫,卜魯罕皇后秉政,遣仁宗母子出居懷州。十年十二月,後至懷州。十一年正月,成宗崩。時武宗總兵北邊,右丞相答剌罕哈剌哈孫陰遣使報仁宗,與後奔還京師。後與仁宗入內哭,復出居舊邸,朝夕入奠。即遣使迎武宗還,以五月即位。先是,太后以兩太子星命付陰陽家推算,問所宜立,對曰:「重光大荒落有災,旃蒙作噩長久。」重光為武宗生年,旃蒙為仁宗生年。太后頗惑其言,遣近臣朵耳諭旨武宗曰:「汝兄弟二人,皆我所出,豈有親疏。陰陽家所言,運祚修短,不容不思也。」武宗聞之默然,進康裏脫脫而言曰:「我捍北邊十年,又胤次居長,太后以星命為言,茫昧難信。使我設施合於天心民望,雖一日之短,亦足垂名萬世。何可以陰陽家言,而乖祖宗之托哉!」脫脫以聞,太后愕然曰:「修短之說,雖出術家,吾為太子遠慮,所以深愛太子也。太子既如是言,今當速來耳。」詳見《康裏脫脫傳》中。



 五月,武宗既立,即日尊太后為皇太后。立仁宗為皇太子。三宮協和。十一月,帝朝太后於隆福宮,上皇太后玉冊玉寶。至大元年三月,帝為太后建興聖宮,給鈔五萬錠、絲二萬斤。二年正月,太后幸五臺山作佛事,詔高麗王璋從之。四月,立興聖宮江淮財賦總管府,以司太后錢糧。三年二月,以上皇太后尊號,告祀南郊。四月,以興聖宮鷹坊等戶四千,分處遼陽,建萬戶府統之。十月戊申,帝率皇太子諸王群臣朝興聖宮,上皇太后尊號冊寶曰儀天興聖慈仁昭懿壽元皇太后。庚戌,後恭謝太廟,以皇太后受尊號,詔赦天下。四年,仁宗即位。延祐二年三月,帝率諸王百官奉玉冊玉寶,加上皇太后尊號曰儀天興聖慈仁昭懿壽元全德泰寧福慶皇太后。延祐七年,英宗即位,十二月,上尊號太皇太后,冊文云:「王政之先,無以加孝,人倫之本,莫大尊親。肆予臨御之初,首舉推崇之典。恭惟太皇太后陛下,仁施溥博,明燭幽之微。爰自居淵潛之宮,已有母天下之望。方武宗之北狩,適成廟之賓天。旋克振於乾綱,諒再安於宗祏。雖有在躬之歷數,實司創業之艱難。儀式表於慈闈,動協謀於先帝。莫究補天之妙,允如扶日之升。位履至尊,兩翼成於聖子;嗣登大寶,復擁佑於眇躬。矧德邁塗山,功高文母。是宜加於四字,式益衍於徽稱。謹奉玉冊玉寶,加上尊號曰儀天興聖慈仁昭懿壽元全德泰寧福慶徽文崇佑太皇太后。於戲!茲雖涉於強名,庶庸申於善頌。九州四海,養未足於孝心;萬歲千秋,願永膺於壽祉。」丙辰,太后御大明殿,受朝賀。戊辰,告太廟。太后見明宗少時有英氣,而英宗稍柔懦,諸群小以立明宗必不利於己,遂擁立英宗。及既即位,太后來賀,英宗即毅然見於色,後退而悔曰:「我不擬養此兒耶!」遂飲恨成疾。至治三年二月崩,升祔順宗廟配食。



 後性聰慧,歷佐三朝,教宮中侍女皆執治女功,親操井臼。然不事檢飭,自正位東朝,淫恣益甚,內則黑驢母亦烈失八用事,外則幸臣失烈門、紐鄰及時宰迭木帖兒相率為奸,以至箠辱平章張珪等,濁亂朝政,無所不至。及英宗立,群幸伏誅,而後勢焰頓息焉。



\end{pinyinscope}