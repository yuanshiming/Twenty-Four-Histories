\article{本紀第八 世祖五}

\begin{pinyinscope}

 十年春正月乙卯朔,高麗國王王禃遣其世子愖來朝。戊午,敕自今並以國字書宣命。命忻都、鄭溫、洪茶丘征耽羅。宿州萬戶愛先不花請築堡牛頭山,以厄兩淮糧運,不允。愛先不花因言:「前宋人城五河機器》、《伊壁鳩魯的體系》等。,統軍司臣皆當得罪。今不築,恐為宋人所先。」帝曰:「汝言雖是,若坐視宋人戍之,罪亦不免也。」安南使者還,言陳光昞受詔不拜。中書移文責問,光昞稱從本俗。改回回愛薛所立京師醫藥院名廣惠司。己未,禁鷹坊擾民及陰陽圖讖等書。癸亥,阿里海牙等大攻樊城,拔之,守將呂文煥懼而請降,中書省驛聞,遣前所俘唐永堅持詔諭之。丁卯,立秘書監。戊辰,給皇子北平王甲一千。置軍器、永盈二庫,分典弓矢、甲胄。庚午,簽陜西探馬赤軍。己卯,川蜀省言:「宋昝萬壽攻成都,也速帶兒所部騎兵征建都未還,擬於京兆等路簽新軍六千為援。」從之。詔遣扎術呵押失寒、崔杓持金十萬兩,命諸王阿不合市藥獅子國。壬午,賞東川統軍合剌所部有功者。合剌請於渠江之北雲門山及嘉陵西岸虎頭山立二戍,以其圖來上,仍乞益兵二萬。詔給京兆新簽軍五千益之。



 二月丙戌,以皇后、皇太子受冊寶,遣太常卿合丹告於太廟。丙申,雲南羅羽酋長阿旭叛,詔有司安集其民,募能捕斬阿旭者賞之。遣斷事官麥肖勾校川陜行省錢穀。詔勘馬剌失里、乞帶脫因、劉源使緬國,諭遣子弟近臣來朝。高麗國王王禃以王師征耽羅,乞下令禁俘掠,聽自制兵仗,從之。丁未,宋京西安撫使、知襄陽府呂文煥以城降。



 三月甲寅朔,詔申諭大司農司遣使巡行勸課,務要農事有成。乙丑,敕樞密院以襄陽呂文煥率將吏赴闕;熟券軍並城居之民仍居襄陽,給其田牛;生券軍分隸各萬戶翼。文煥等發襄陽,擇蒙古、漢人有才力者護視以來。丙寅,帝御廣寒殿,遣攝太尉、中書右丞相安童授皇後弘吉剌氏玉冊玉寶,遣攝太尉、同知樞密院事伯顏授皇太子真金玉冊金寶。辛未,以皇后、皇太子受冊寶,詔告天下。劉整請教練水軍五六萬及於興元金、洋州、汴梁等處造船二千艘,從之。壬申,分金齒國為兩路。癸酉,客星青白如粉絮,起畢,度五車北,復自文昌貫斗杓,歷梗河,至左攝提,凡二十一日。以前中書左丞相耶律鑄平章軍國重事,中書左丞張惠為中書右丞。車駕幸上都。西蜀嚴忠範以罪罷,遣察不花等撫治軍民。罷中興等處行中書省。



 夏四月癸未朔,阿里海牙以呂文煥入朝,授文煥昭勇大將軍、侍衛親軍都指揮使、襄漢大都督,賜其將校有差。時將相大臣皆以聲罪南伐為請,驛召姚樞、許衡、徒單公履等問計。公履對曰:「乘破竹之勢,席卷三吳,此其時矣。」帝然之。詔罷河南等路行中書省,以平章軍國重事史天澤、平章政事阿術、參知政事阿里海牙行荊湖等路樞密院事,鎮襄陽;左丞相合丹,參知行中書省事劉整,山東都元帥塔出、董文炳行淮西等路樞密院事,守正陽。天澤等陛辭,詔諭以襄陽之南多有堡寨,可乘機進取。仍以鈔五千錠賜將士及賑新附軍民。甲申,免隆興路榷課三年。丁酉,敕南儒為人掠賣者官贖為民。辛丑,罷四川行省,以鞏昌二十四處便宜總帥汪良臣行西川樞密院,東川閬、蓬、廣安、順慶、夔府、利州等路統軍使合剌行東川樞密院,東川副統軍王仲仁同簽行樞密院事,仍令汪良臣就率所部軍以往。



 五月壬子朔,定內外官復舊制,三歲一遷。甲寅,禁無籍軍從大軍殺掠,其願為軍者聽。戊辰,詔:「天下獄囚,除殺人者待報,其餘一概疏放,限以八月內自至大都,如期而至者皆赦之。」乙亥,詔:「免民代輸簽軍戶絲銀,及伐木夫戶賦稅。負前朝官錢不能償者,毋徵。主守失陷官錢者,杖而釋之。陣亡軍及營繕工匠無丁產者,量加廩給。」以雄、易州復隸大都。庚辰,賞襄陽有功萬戶奧魯赤等銀鈔衣服有差。



 六月乙酉,賑諸王塔察兒部民饑。丁亥,以各路弓矢甲匠並隸軍器監。免大都、南京兩路賦役,以紓民力。賑甘州等處諸驛。辛卯,汰陜西貧難軍。以劉整、阿里海牙不相能,分軍為二,各統之。癸巳,敕襄陽造戰船千艘。甲午,改資用庫為利用監。丁酉,置光州等處招討司。戊申,經略忻都等兵至耽羅,撫定其地。詔以失里伯為耽羅國招討使,尹邦寶副之。升拱衛直為都指揮司。使日本趙良弼,至太宰府而還,具以日本君臣爵號、州郡名數、風俗土宜來上。閏月癸丑,敕諸道造甲一萬、弓五千,給淮西行樞密院。己巳,罷東西兩川統軍司。辛未,以翰林院纂修國史,敕採錄累朝事實以備編集。丙子,以平章政事賽典赤行省雲南,統合剌章、鴨赤、赤科、金齒、茶罕章諸蠻,賜銀二萬五千兩、鈔五百錠。



 秋七月辛巳,以金州軍八百人及統軍司還成都,忽朗吉軍千人隸東川。壬午,以修太廟,將遷神主別殿,遣兀魯忽奴帶、張文謙祭告。丙戌,敕樞密院:「襄陽生券軍無妻子者,發至京師,仍益兵衛送,其老疾者遣還家。」庚寅,河南水,發粟賑民饑,仍免今年田租。省西涼府入永昌路。戊申,高麗國王王禃遣其順安公王悰、同知樞密院事宋宗禮,賀皇后、皇太子受冊禮成。



 八月庚戌朔,前所釋諸路罪囚,自至大都者凡二十二人,並赦之。甲寅,鳳翔寶雞縣劉鐵妻一產三男,復其家三年。丁丑,聖誕節,高麗王王禃遣其上將軍金詵來賀。己卯,賜襄陽生熟券軍冬衣有差。



 九月辛巳,遼東饑,弛獵禁。以合伯為平章政事。壬午,立河南宣慰司,供給荊湖、淮西軍需。甲申,襄陽生券軍至大都,詔伯顏諭之,釋其械系,免死罪,聽自立部伍,俾征日本;仍敕樞密院具鎧仗,人各賜鈔娶妻,於蒙古、漢人內選可為率領者。丙戌,劉秉忠、姚樞、王磐、竇默、徒單公履等上言:「許衡疾歸,若以太子贊善王恂主國學,庶幾衡之規模不致廢墜。」又請增置生員,並從之。秉忠等又奏置東宮宮師府詹事以次官屬三十八人。戊子,遣官詣荊湖行省,差次有功將士。禁京畿五百里內射獵。己丑,敕自今秋獵鹿豕先薦太廟。壬辰,中書省臣奏:「高麗王王禃屢言小國地狹,比歲荒歉,其生券軍乞駐東京。」詔令營北京界,仍敕東京路運米二萬石以賑高麗。丁酉,立正陽諸驛。敕河南宣慰司運米三十萬石給淮西合答軍,仍給淮西、荊湖軍需有差。壬寅,敕會同館專居降附之入覲者。以翰林學士承旨和禮霍孫兼會同館事,以主朝廷咨訪,及降臣奏請。征東招討使塔匣剌請徵骨嵬部,不允。丙午,置御藥院。車駕至自上都。給諸王塔察兒所部布萬匹。



 冬十月乙卯,享於太廟。丙辰,以西川編民、東川義士軍屯田,餉潼川、青居戍兵。敕伯顏、和禮霍孫以史天澤、姚樞所定新格,參考行之。庚申,御史臺臣言,沒入贓罰,為鈔一千三百錠。詔有貧乏不能存者,以此賑之。有司斷死罪五十人,詔加審覆,其十三人因鬥毆殺人,免死充軍,餘令再三審覆以聞。禁牧地縱火。以合答帶為御史大夫,升襄陽府為路,罷廣寧府新簽軍。初建正殿、寢殿、香閣、周廡兩翼室。西蜀都元帥也速答兒與皇子奧魯赤合兵攻建都蠻,擒酋長下濟等四人,獲其民六百,建都乃降,詔賞將士有差。



 十一月癸未,命布只兒修《起居注》。丁未,大司農司言:「中書移文,以畿內秋禾始收,請禁農民覆耕,恐妨芻牧。」帝以農事有益,詔勿禁。



 十二月己酉朔,安童等言:「昔博赤伯都謂總管府權太重,宜立運司並諸軍奧魯以分之。臣以今之民官,循例遷徙,保無邪謀,別立官府,於民未便。」帝然之。壬子,賜襄樊被傷軍士鈔千錠。甲寅,宋夏貴攻正陽,淮西行院擊走之。壬戌,召阿術同呂文煥入覲。大司農司請罷西夏世官,括諸色戶,從之。安南國王陳光昞遣使來貢方物。諸王薛闍禿以罪從軍,累戰皆捷,召赴闕。己巳,省陜州虢略、硃陽二縣入靈寶。賜萬戶解汝楫銀萬五千兩。諸王孛兀兒出率所部兵與皇子北平王合軍討叛臣聶古伯,平之,賞立功將士有差。賜諸王金、銀、幣、帛如歲例。是歲,諸路蟲蝻災五分,霖雨害稼九分,賑米凡五十四萬五千五百九十石。天下戶一百九十六萬二千七百九十五。



 十一年春正月己卯朔,宮闕告成,帝始御正殿,受皇太子諸王百官朝賀。高麗國王王禃遣其少卿李義孫等來賀,兼奉歲貢。乙酉,以金州招討使欽察率襄陽生熟券軍千人戍鴨池。庚寅,初立軍官以功升散官格。免諸路軍雜賦。以忙古帶等新舊軍一萬一千五百人戍建都。立建都寧遠都護府,兼領互市監。壬辰,置西蜀四川屯田經略司。丁酉,長春宮設周天金籙醮七晝夜。敕荊湖行院以軍三萬、水弩砲手五千隸淮西行院。丙午,彰德趙當道等以謀逆伏誅,餘從者論罪有差。立於闐、鴉兒看兩城水驛十三,沙州北陸驛二,免於闐採玉工差役。阿里海牙言:「荊襄自古用武之地,漢水上流已為我有,順流長驅,宋必可平。」阿術又言:「臣略地江淮,備見宋兵弱於往昔,今不取之,時不能再。」帝趣召史天澤同議,天澤對曰:「此國大事,可命重臣一人如安童、伯顏,都督諸軍,則四海混同,可計日而待矣。臣老矣,如副將者,猶足為之。」帝曰:「伯顏可以任吾此事矣。」阿術、阿里海牙因言:「我師南征,必分為三,舊軍不足,非益兵十萬不可。」詔中書省簽軍十萬人。



 二月戊申朔,賜阿術所部將士及茶罕章阿吉老耆等銀鈔有差。甲寅,太陰犯井宿。庚申,新德副元帥楊堯元戰沒,以其子襲職。初立儀鸞局,掌宮門管鑰、供帳燈燭。壬申,造戰船八百艘於汴梁。以廉希憲為中書右丞、北京等處行中書省事。車駕幸上都。



 三月己卯,詔以勸課農桑諭高麗國王王禃,仍命安撫高麗軍民總管洪茶丘提點農事。己丑,呂文煥隨司千戶陳炎謀叛,誅首惡二人,其隨司軍並其妻子皆令內徙。庚寅,敕鳳州經略使忻都、高麗軍民總管洪茶丘等將屯田軍及女直軍,並水軍,合萬五千人,戰船大小合九百艘,征日本。移碉門兵戍合答城。辛卯,改荊湖、淮西二行樞密院為二行中書省,伯顏、史天澤並為左丞相,阿術為平章政事,阿里海牙為右丞,呂文煥為參知政事,行中書省於荊湖;合答為左丞相,劉整為左丞,塔出、董文炳為參知政事,行中書省於淮西。遣使代祀岳瀆后土。河南宣慰司言:「軍興轉輸煩重,宜賦軍匠諸戶,權助財用。」從之。癸巳,獲嘉縣尹常德,課最諸縣,詔優賞之。亦乞裡帶強取民租產、桑園、廬舍、墳墓,分為探馬赤軍牧地,詔還其民。萬戶阿里必嘗發李璮逆謀,為璮所殺,以其子剌剌吉襲職。改金州招討司為萬戶府。遣要速木、咱興憨失招諭八魯國。帝師八合思八歸土番國,以其弟亦鄰真襲位。建大護國仁王寺成。



 夏四月辛亥,分陜西隴右諸州置提刑按察司,治鞏昌。癸丑,初建東宮。甲寅,誅西京訛言惑眾者。括諸路馬五萬匹。辛未,詔安慰斡端、鴉兒看、合失合兒等城。賜襄樊戰死之士二百四十九人之家,每家銀百兩。乙亥,命也速帶兒將千人,同撒吉思所部五州丁壯戍益都。



 五月丙戌,汪惟正以所部軍逃亡,乞於民站戶選補,從之。敕北京、東京等路新簽軍恐不宜暑,權駐上都。乙未,樞密院臣言:「舊制,蒙古軍每十人月食糧者,惟拔都二人。今遣怯薛丹合丹核其數,多籍二千六百七十人。」敕杖合丹,斥無入宿衛,謫往西川效死軍中,餘定罪有差。丙申,以皇女忽都魯揭裏迷失下嫁高麗世子王愖。辛丑,敕隨路所簽新軍,其戶絲銀均配於民者,並除之。六月丙午朔,劉整乞益甲仗及水弩手,給之。庚戌,賜建都合馬里戰士銀鈔有差。癸丑,敕合答選部下蒙古軍五千人,與漢軍分戍沿江堡隘,為使傳往來之衛。仍以古不來拔都、翟文彬率兵萬人,掠荊南鴉山,以綴宋之西兵。丙辰,免上都、隆興兩路簽軍。庚申,問罪於宋,詔諭行中書省及蒙古、漢軍萬戶千戶軍士曰:



 爰自太祖皇帝以來,與宋使介交通。憲宗之世,朕以籓職奉命南伐,彼賈似道復遣宋京詣我,請罷兵息民。朕即位之後,追憶是言,命郝經等奉書往聘,蓋為生靈計也。而乃執之,以致師出連年,死傷相藉,系累相屬,皆彼宋自禍其民也。襄陽既降之後,冀宋悔禍,或起令圖,而乃執迷,罔有悛心,所以問罪之師,有不能已者。今遣汝等,水陸並進,布告遐邇,使咸知之。無辜之民,初無預焉,將士毋得妄加殺掠。有去逆效順,別立奇功者,驗等第遷賞。其或固拒不從及逆敵者,俘戮何疑。



 甲子,分遣忙古帶、八都、百家奴率武衛軍南征。丙寅,以合剌合孫為中書左丞,崔斌參知政事,仍行河南道宣慰司事。敕有司閱核延安新軍,貧無力者免之。戊辰,監察御史言:「江淮未附,將帥闕人。今首用阿里海牙子忽失海牙、劉整子垓,素不知兵,且缺人望,宜依弟男例罷去。」從之。秋七月乙亥朔,敕山北遼東道提刑按察使兀魯失不花同參知政事廉希憲行省北京,國王頭輦哥毋署事,有大事,則希憲等就議。乙酉,徙生券軍八十一人屯田和林。癸巳,高麗國王王禃薨,遣使以遺表來上,且言世子搔孝謹,可付後事。敕同知上都留守司事張煥冊愖為高麗國王。乙未,伯顏等陛辭,帝諭之曰:「古之善取江南者,唯曹彬一人。汝能不殺,是吾曹彬也。」興元鳳州民獻麥一莖四穗至七穗,穀一莖三穗。八月甲辰朔,頒諸路立社稷壇壝儀式。丁未,史天澤言:「今大師方興,荊湖、淮西各置行省,勢位既不相下,號令必不能一,後當敗事。」帝是其言,復改淮西行中書省為行樞密院。癸丑,行中書省言:「江漢未下之州,請令呂文煥率其麾下臨城諭之,令彼知我寬仁,善遇降將,亦策之善者也。」從之。甲寅,弛河南軍器之禁。辛未,高麗王愖遣其樞密使樸璆來賀聖誕節。詔太原新簽軍遠戍兩川,誠可憫恤,諭樞密院遣使分括廩粟,給其家。九月丙戌,行中書省以大軍發襄陽,檄諭宋州郡官吏將校士民。癸巳,師次鹽山,距郢州二十里。宋兵十餘萬當郢,夾漢水,城萬勝堡,兩岸戰艦千艘,鐵絙橫江,貫大艦數十,遏我舟師不得下。惟黃家灣有溪,經鷂子山入唐港,可達於江,宋又為壩,築堡其處,駐兵守之,系舟數百,與壩相依。伯顏督諸軍攻拔之,鑿壩挽舟入溪,出唐港,整列而進。車駕至自上都。冬十月己酉,享於太廟。庚申,長河西千戶必剌沖剽掠甲仗,集眾為亂,火你赤移戍未還,副元帥覃澄率屬吏赴之。帝曰:「澄不必獨往,趣益兵三千付火你赤,合力討之。」壬戌,歲星犯壘壁陣。乙丑,伯顏督諸將破沙洋堡,生擒守將串樓王。翌日,次新城,總制黃順縋城降。伯顏遣順招都統邊居誼,不出,總管李庭破其外堡,諸軍蟻附而登,拔之,居誼自焚死。辛未,賜北平王南木合馬三萬、羊十萬。十一月庚辰,斷死罪三十九人。壬午,敕西川行樞密院也速帶兒取嘉定府。癸未,符寶郎董文忠言:「比聞益都、彰德妖人繼發,其按察司、達魯花赤及社長不能禁止,宜令連坐。」詔行之。乙酉,軍次復州,宋安撫使翟貴出降。丁亥,詔宋嘉定安撫昝萬壽,及凡守城將校納款來降,與避罪及背主叛亡者,悉從原免。癸巳,東川元帥楊文安與青居山蒙古萬戶怯烈乃、也只裏等會兵達州,直趣雲安,軍至馬湖江與宋兵遇,大破之,遂拔雲安、羅拱、高陽城堡,賜文安等金銀有差。以香河荒地千頃置中衛屯。伯顏遣萬戶帖木兒、譯史阿裏奏沙洋、新城之捷,且以新城總制黃順來見,賜順黃金錦衣及細甲,授湖北道宣慰使,佩虎符。敕:「京師盜詐者眾,宜峻立治法。」召徵日本忽敦、忽察、劉復亨、三沒合等赴闕。壬寅,安童以阿合馬擅財賦權,蠹國害民,凡官屬所用非人,請別加選擇;其營作宮殿,夤緣為奸,亦宜詰問。帝命窮治之。起閣南直大殿及東西殿,增選樂工八百人,隸教坊司。十二月丙午,伯顏大軍次漢口。宋淮西制置使夏貴,都統高文明、劉儀以戰船萬艘,分據諸隘,都統王達守陽羅堡,荊湖宣撫硃禩孫以游擊軍厄中流,師不得進。用千戶馬福言,自漢口開壩,引船會淪河口,徑趨沙蕪,遂入大江。癸丑,以諸路逃奴之無主者二千人,隸行工部。甲寅,賞忻都等征耽羅功,銀鈔幣帛有差。乙卯,阿里海牙督萬戶張弘範等攻武磯堡,宋夏貴以兵來援,阿術率萬戶晏徹兒等四翼軍對青山磯泊。丙辰,萬戶史格以一軍先渡,為宋荊鄂諸軍都統程鵬飛所敗,總管史塔剌渾等率眾赴敵,鵬飛敗走。進軍沙州,抵觀音山,夏貴東走,遂破武磯堡,斬宋都統王達,始達南岸,追至鄂州南門而還。丁巳,伯顏登武磯山,宋硃禩孫遁歸江陵。己未,師次鄂州,宋直秘閣湖北提舉張晏然、權知漢陽軍王儀、知德安府來興國並以城降,程鵬飛以本軍降。伯顏承制以宋鄂州民兵總制王該知鄂州事,王儀、來興國仍舊任,撤其戍兵,分隸諸軍。下令禁侵暴,凡逃民悉縱還之。以阿里海牙兵四萬鎮鄂漢。伯顏、阿術將大軍,水陸東下,以侍衛親軍都指揮使禿滿帶為諸軍殿。以襄陽路總管賈居貞為宣撫使,商議行中書省事。庚申,淮西正陽火,廬舍甲仗焚蕩無餘,杖萬戶愛先不花等有差。癸亥,賜太一真人李居素第一區,仍賜額曰太一廣福萬壽宮。行中書省以渡江捷聞。敕縱呂文煥隨司軍悉還家。割南陽盧氏縣隸嵩州,置歸德永城縣,長武縣省入涇川,良原縣省入靈臺。是歲,天下戶一百九十六萬七千八百九十八。諸路虸蚄等蟲災凡九所,民饑,發米七萬五千四百一十五石、粟四萬五百九十九石以賑之。



 十二年春正月癸酉朔,高麗國王王愖遣其判閣事李信孫來賀,及奉歲幣。甲戌,大軍次黃州,宋沿江制置副使、知黃州陳奕以城降,伯顏承制授奕沿江大都督。其子巖知漣州,奕遣人以書諭之,書至,巖即出降。乙亥,徙襄陽新民七百戶於河北。東川副都元帥張德潤拔禮義城,殺宋安撫使張資,招降軍民千五百餘人。繼遣元帥張桂孫略地,俘總管郭武及都轄唐惠等六人以歸。賜德潤金五十兩及西錦、金鞍、細甲、弓矢,部下將士鈔三百錠。戊寅,劉整卒。安西王相府乞給鈔萬錠為軍需,敕以千錠給之。癸未,師次蘄州,宋安撫使管景模以城降。乙酉,敕樞密院以納忽帶兒、也速帶兒所統戍軍及再簽登萊丁壯八百人,付五州經略司,其郯城、十字路亦聽經略司節度。丙戌,大軍次江州,宋江西安撫使、知江州錢真孫及淮西路六安軍曹明以城降。丁亥,樞密院臣言:「宋邊郡如嘉定、重慶、江陵、郢州、漣海等處,皆阻兵自守,宜降璽書招諭。」從之。宋知南康軍葉閶以城降。敕以侍衛親軍指揮使札的失、囊加帶將蒙古軍二千,百家奴、唐古、忙兀兒將漢軍萬人,赴蔡州;禿滿帶、賈忙古帶復將餘兵赴闕。己丑,遣伯術、唐永堅齎詔招諭郢州,仍敕襄陽統軍司調兵三千人衛送永堅等。選蒙古、畏吾、漢人十四人赴行中書省,為新附州郡民官。庚寅,遣右衛指揮副使鄭溫、唐古、帖木兒率衛軍萬人,同札的失、囊加帶戍黃州。詔諭重慶府制置司並所屬州郡城寨官吏軍民舉城歸附。壬辰,以宣撫使賈居貞簽書行中書省事,戍鄂州。安南國使者還,敕以舊制籍戶、設達魯花赤、簽軍、立站、輸租及歲貢等事諭之。乙未,遣兵部尚書廉希賢、工部侍郎嚴忠範、秘書監丞柴紫芝奉國書使於宋。丁酉,以萬家奴所募願為軍者萬人南征。己亥,雲南總管信苴日、石買等刺殺合剌章舍裏威之為亂者,以金賞之。命土魯至雲南,趣阿魯帖木兒入覲。以蠻夷未附者尚多,命宣慰司兼行元帥府事,並聽行省節度,置郡縣,尹長選廉能者任之。置雲南諸路規措所,以贍思丁為使。益衛送唐永堅兵,永堅求拜都、忙古帶偕行,許之。敕追諸王海都、八剌金銀符三十四。



 二月癸卯,大軍次安慶府,宋殿前都指揮使、知安慶府範文虎以城降,伯顏承制授文虎兩浙大都督。甲辰,以中書右丞博魯歡為淮東都元帥,中書右丞阿里左副都元帥。仍命阿里、撒吉思等各部蒙古、漢軍會邳州。又發蘄、宿戍兵,將河南戰船千艘赴之。遣必闍赤孛羅檢核西夏榷課,命開元宣撫司賑吉裏迷新附饑民。敕畏吾地春夏毋獵孕字野獸。立后土祠於平陽之臨汾,伏羲、女媧、舜、湯、河瀆等廟於河中、解州、洪洞、趙城。丙午,大軍次池州,宋權州事趙昴發自經死,都統制張林以城降。省西夏中興都轉運司入總管府。議以中統鈔易宋交會,並發蔡州鹽,貿易藥材。丁未,禁無籍自效軍俘掠新附復業軍民。戊申,詔諭江、黃、鄂、岳、漢陽、安慶等處歸附官吏士民軍匠僧道人等,令農者就耒,商者就途,士庶緇黃,各安己業,如或鎮守官吏妄有搔擾,詣行中書省陳告。史天澤卒。召游顯、楊庭訓赴闕。賜陳言人霍升、張和鈔十錠,俾從淮東元帥府南征。庚戌,遣禮部侍郎杜世忠、兵部郎中何文著,齎書使日本國。辛亥,遣同知濟南府事張漢英持詔諭淮東制置使李庭芝。壬子,洺磁路總管姜毅捕獲農民郝進等四人,造妖言惑眾,敕誅進,餘減死流遠方。宋都督賈似道遣計議宋京、承宣使阮思聰詣行中書省,請還已降州郡,約貢歲幣。伯顏使囊加帶同阮思聰還報命,留宋京以待,使謂似道曰:「未渡江時,議貢議和則可,今沿江諸郡皆已內屬,欲和,則當來面議也。」囊加帶還,乃釋宋京。以同簽樞密院事倪德政赴鄂州省,治財賦。癸丑,御史臺臣劾前南京路總管田大成,以其弟婦趙氏為妻,廢絕人倫,敕杖八十,三年不齒。時大成已死,惟市杖趙氏八十。丙辰,賞東征元帥府日本戰功錦絹、弓矢、鞍勒。庚申,遣塔不帶、斡魯召鄂漢降臣張晏然等赴闕,仍諭之曰:「朕省卿所奏云:『宋之權臣不踐舊約,拘留使者,實非宋主之罪,儻蒙聖慈,止罪擅命之臣,不令趙氏乏祀者。』卿言良是。卿既不忘舊主,必能輔弼我家。比卿奏上,已遣伯顏按兵不進,仍遣兵部尚書廉希賢等持書往使,果能悔過來附,既往之愆,朕復何究?至於權臣賈似道,尚無罪之之心,況肯令趙氏乏祀乎?若其執迷罔悛,未然之事,朕將何言,天其鑒之。」辛酉,以闊闊出率其部下軍千人及親附軍五百,聽阿剌海牙節制。凡湖南州縣及瀕水之民有來附者,俾闊闊出統之,拒敵不降者,就為招集。詔令大洪山避兵民,還歸漢陽,復業農畝,命阿剌海牙鎮守之。復命阿失罕、唐永堅、綦公直等與脫烈將甲騎千人,持詔招諭郢州。大軍次丁家洲,戰船蔽江而下。宋賈似道分遣步帥孫虎臣及督府節制軍馬蘇劉義,集兵船於江之南北岸,似道與淮西制置使夏貴將後軍。戰船二千五百餘艘,橫亙江中。翌日,伯顏命左右翼萬戶率騎兵,夾岸而進,繼命舉巨砲擊之。宋兵陣動,夏貴先遁,似道錯愕失措,鳴鉦斥諸軍散,宋兵遂大潰。阿術與鎮撫何瑋、李庭等舟師及步騎,追殺百五十里,得船二千餘艘,及軍資器仗、督府圖籍符印,似道東走揚州。阿先不花言:「夏貴縱北軍岳全還,稱欲內附,宜降璽書招諭。」遂遣其甥胡應雷持詔往諭之。甲子,大軍次蕪湖縣,宋江東運判、知太平州孟之縉以城降。都元帥博魯歡次海州,知州丁順以城降。乙丑,阿里海牙言:「江陵宋巨鎮,地居大江上流,屯精兵不啻數十萬,若非乘此破竹之勢取之,江水泛溢,鄂漢之城亦恐難守。」從其請,仍降璽書,遣使諭江陵府制置司及高達已下官吏軍民。宋福州團練使、知特摩道事農士貴,率知那寡州農天或、知阿吉州農昌成、知上林州農道賢,州縣三十有七,戶十萬,詣雲南行中書省請降。丙寅,樞密院言:「渡江初,亳州萬戶史格、毗陽萬戶石抹紹祖,以輕進致敗,乞罪之。」有旨,或決罰降官,或以戰功自贖,其從行省裁處。禁民間賭博,犯者流之北地。戊辰,師次採石鎮,知和州王善以城降。都元帥博魯歡次漣州,宋知州孫嗣武以城降。己巳,復遣伯術、唐永堅等宣諭郢州官吏士庶。庚午,大軍次建康府,宋沿江制置使趙溍南走,都統、權兵馬司事徐王榮、翁福、茅世雄等及鎮軍曹旺以城降。宋賈似道至揚州,始遣總管段佑送國信使郝經、劉人傑等來歸。敕樞密院迎經等,由水路赴闕。詔安南國王陳光昞,仍以舊制六事諭之,趣其來朝。命怯薛丹察罕不花、侍儀副使關思義、真人李德和,代祀岳瀆后土。車駕幸上都。



 三月壬申朔,宋鎮江府馬軍總管石祖忠以城降。行中書省分遣淮西行樞密院阿塔海駐京口。宋誅殿帥韓震,其部將李大明等二百人,攜震母、妻並諸子文育、文炌自臨安來奔。甲戌,宋江陰軍僉判李世修以城降。乙亥,諭樞密院:「比遣建都都元帥火你赤征長河西,以副都元帥覃澄鎮守建都,付以璽書,安集其民。」仍敕安西王忙兀剌、諸王只必帖木兒、駙馬長吉,分遣所部蒙古軍從西平王奧魯赤征吐蕃。命萬執中、唐永堅同前所遣阿失罕等,將銳兵千人,同往招諭郢州:已降,則鎮之;不降,則從陸路與阿里海牙、忽不來會於荊南。丙子,國信使廉希賢等至建康,傳旨令諸將各守營壘,毋得妄有侵掠。宋知滁州王文虎以城降。戊寅,賜皇子安西王幣帛八千匹、絲萬斤。己卯,改平陰縣新鎮寨為肥城縣,隸濟寧府。庚辰,宋知寧國府顏紹卿以城降。江東路得府二、州五、軍二、縣四十三,戶八十三萬一千八百五十二,口一百九十一萬九千一百六。甲申,於中興路置懷遠、靈武二縣,分處新民四千八百餘戶。丙戌,宋常州安撫戴之泰、通判王虎臣以城降。國信使廉希賢、嚴忠範等至宋廣德軍獨松關,為宋人所殺。丁亥,免諸路軍雜賦。辛卯,宋將高世傑復據岳州,質知州孟之紹妻子;又取復州降將翟貴妻子,送之江陵。世傑會郢、復、岳三州及上流諸軍戰船數千艘,兵數萬人,扼荊江口。壬辰,阿里海牙以軍屯於東岸,世傑夜半遁去,黎明至洞庭湖口,兵船成列而陣。阿里海牙督諸翼萬戶及水軍張榮實、解汝楫等,逐世傑於湖口之夾灘,遣郎中張鼎召世傑,世傑降。阿里海牙以世傑招岳州,孟之紹亦以城降。以世傑力屈而降,誅之。賜北平王南木合所部馬二千一百八十、羊三百。癸巳,敕郯城、沂州、十字路戍兵從博魯歡征淮南。丙申,側布蕃官稅昔、確州蕃官莊寮男車甲等,率四十三族,戶五千一百六十,詣四川行樞密院來附。戊戌,遣山東路經略使王儼戍岳州。庚子,從王磐、竇默等請,分置翰林院,專掌蒙古文字,以翰林學士承旨撒的迷底裡主之。其翰林兼國史院,仍舊纂修國史、典制誥、備顧問,以翰林學士承旨兼修《起居注》和禮霍孫主之。辛丑,敕阿術分兵取揚州。



 夏四月壬寅朔,賞討長河西必剌充有功者及陣亡者金、銀、鈔、幣、帛各有差。乙巳,改西夏中興道按察司為隴右河西道。丙午,立漣州、新城、清河三驛。阿里海牙駐軍江陵城南沙市,攻其柵,破之,知荊門軍劉懋降。丁未,阿里海牙遣郎中張鼎齎詔入江陵,宋荊湖制置硃禩孫,湖北制置副使高達,京西湖北提刑青陽夢炎、李湜始出降。阿里海牙入江陵,分道遣使招諭未下州郡,知峽州趙真、知歸州趙仔、權豐州安撫毛浚、常德府新城總制魯希文、舊城權知府事周公明等,悉以城降。辛亥,遣使招諭宋五郡鎮撫使呂文福使降。甲寅,諭中書省議立登聞鼓,如為人殺其父母兄弟夫婦,冤無所訴,聽其來擊。其或以細事唐突者,論如法。辛酉,宋郢州安撫趙孟、復州安撫翟貴以城降。宋度支尚書吳浚移書建康徐王榮等,述其丞相陳宜中語,請罷兵通好。伯顏遣中書議事官張羽、淮西行院令史王章,同宋來使馬馭,持徐王榮復書至平江府驛亭,悉為宋所殺。癸亥,阿術師駐瓜洲,距揚州四十五里,宋淮東制置司盡焚城中廬舍,遷其居民而去。阿術創立樓櫓戰具以守之。丙寅,立尚牧監。賜降臣丁順等衣服。免京畿百姓今歲絲銀。丁卯,以大司農、御史中丞孛羅為御史大夫。罷隨路巡行勸農官,以其事入提刑按察司。括諸寺闌遺人口。庚午,以高達為參知政事,仍詔慰諭之。遣兵部郎中王世英、刑部郎中蕭鬱,持詔召嗣漢四十代天師張宗演赴闕。



 五月辛未朔,阿里海牙以所俘童男女千人、牛萬頭來獻。樞密院言:「峽州宜以戰船扼其津要。又郢、復二州戍兵不足,今擬襄陽等處選五千七百人,隸行中書省,聽阿里海牙調遣。」從之。詔中書右丞廉希憲、參知政事脫博忽魯禿花行中書省於江陵府,阿里海牙還鄂州。立襄陽至荊南三驛。丁丑,阿術立木柵於楊子橋,斷淮東糧道,且為瓜洲籓蔽。庚辰,詔諭參知政事高達曰:「昔我國家出征,所獲城邑,即委而去之,未嘗置兵戍守,以此連年征伐不息。夫爭國家者,取其土地人民而已,雖得其地而無民,其誰與居?今欲保守新附城壁,使百姓安業力農,蒙古人未之知也。爾熟知其事,宜加勉旃。湖南州郡皆汝舊部曲,未歸附者何以招懷,生民何以安業,聽汝為之。」宋嘉定安撫昝萬壽遣部將李立奉書請降,言累負罪愆,乞加赦免。詔遣使招諭之。辛巳,宋知辰州呂文興、黃仙洞行隋州事傅安國、仙人寨行均州事徐鼎、知沅州文用圭、知靖州康玉、知房州李鑒等,皆以城降。荊南湖北路凡得府三、州十一、軍四、縣五十七,戶八十萬三千四百一十五,口一百一十四萬三千八百六十。丙戌,以三衛新附生券軍赴八達山屯田。丁亥,召伯顏赴闕,以蒙古萬戶阿剌罕權行中書省事,遣肅州達魯花赤阿沙簽河西軍。萬戶愛先不花違伯顏節制,擅撤戍兵,詔追奪符印,使從軍自效。淮東宣撫陳巖乞解官,終喪三年,不許。申嚴屠牛馬之禁。庚寅,宋五郡鎮撫使呂文福來降。壬辰,宋都統制劉師勇、殿帥張彥據常州。癸巳,諭高麗國王王心甚,招珍島餘黨之在耽羅者。



 六月庚子朔,日有食之。宋嘉定安撫使昝萬壽以城降,賜名順。癸卯,遣兩浙大都督範文虎持詔往諭安豐、壽州、招信、五河等處鎮戍官吏軍民;遣刑部侍郎伯術諭硃禩孫,以年老多病,不任朝謁,權留大都,無自疑懼。諭廉希憲等,元沒青陽夢炎、李湜家貲,如籍還之,並徙其家赴都。甲辰,以萬戶阿剌罕為行中書省參知政事。獲知開州張章,赦其罪。章二子柱、楫先來降,以其子故,免死。敕失里伯、史樞率襄陽熟券軍二千、獵戶丁壯二千,同範文虎招安豐軍,各賜馬十匹。其故嘗從丞相史天澤者十九人,願宣勞軍中,令從樞以行。戊申,簽平陽、西京、延安等路達魯花赤弟男為軍。辛亥,賞諸王兀魯所部獲功建都者三十五人銀鈔有差,定兀魯衛士人各馬二匹,從者一匹。敕淮東元帥府發兵,及鄂州戍兵與李璮舊部曲,並前河南已簽軍萬人後免為民者,復籍為兵,並付行中書省。戊午,詔遣使招諭宋四川制置趙定應:「比者畢再興、青陽夢炎赴闕,面陳蜀閫事宜,奏請緩師,令自納款,姑從所請。今遣再興宣布大信,若能順時達變,可保富貴,毋為塗炭生靈,自貽後悔。」庚申,遣重慶府招討使畢再興持詔招諭宋合州節使張玨、江安潼川安撫張朝宗、涪州觀察陽立、梁山軍防禦馬野。辛酉,宋潼川安撫使、知江安州梅應春以城降。乙丑,以漣、海新附丁順等括船千艘,送淮東都元帥府。丙寅,宋揚州都統姜才、副將張林步騎二萬人,乘夜攻楊子橋木柵。守柵萬戶史弼來告急,阿術自瓜洲以兵赴之。詰旦至柵下,才軍夾水為陣,阿術麾騎兵渡水擊之,陣堅不動。阿術軍引卻,才軍來逼,我軍與力戰,才軍遂走。阿術步騎並進,大敗之,才僅以身免,生擒張林,斬首萬八千級。戊辰,敕塔出率阿塔海、也速帶兒兩軍赴漣水。以遜攤為耽羅國達魯花赤。罷山東經略司。



 秋七月庚午朔,阿術集行省諸翼萬戶兵船於瓜洲,阿塔海、董文炳集行院諸翼萬戶兵船於西津渡,宋沿江制置使趙溍、樞密都承旨張世傑、知泰州孫虎臣等陳舟師於焦山南北。阿術分遣萬戶張弘範等,以拔都兵船千艘,西掠珠金沙。辛未,阿術、阿塔海登南岸石公山,指授諸軍水軍萬戶劉琛循江南岸,東趣夾灘,繞出敵後;董文炳直抵焦山南麓,以掎其右;招討使劉國傑趣其左;萬戶忽剌出搗其中;張弘範自上流繼至,趣焦山之北。大戰自辰至午,呼聲震天地,乘風以火箭射其箬篷。宋師大敗,世傑、虎臣等皆遁走。追至圌山,獲黃鵠白鷂船數百艘。宋人自是不復能軍。翌日,宋平江都統劉師勇、殿帥張彥,以兩浙制司軍至呂城,復為阿塔海行院兵所敗。壬申,簽雲南落落、蒲納烘等處軍萬人,隸行中書省。癸酉,太白犯井。詔取茶罕章未附種落。丁丑,立衛州至楊村水驛五。己卯,增置燕南河北道提刑按察司。以蔡州驛蒙古軍四百隸阿里海牙,漢軍六百從萬戶宋都帶赴江西。壬午,遣使招宋淮安安撫使硃煥。癸未,詔遣使江南,搜訪儒、醫、僧、道、陰陽人等。敕左丞相伯顏率諸將直趣臨安;右丞阿里海牙取湖南;蒙古萬戶宋都帶,漢軍萬戶武秀、張榮實、李恆,兵部尚書呂師夔行都元帥府,取江西。罷淮西行樞密院,以右丞阿塔海、參政董文炳同署行中書省事。辛卯,太陰犯畢。甲午,遣使持詔招諭宋李庭芝及夏貴。以伯顏為中書右丞相,阿術為中書左丞相。



 八月己亥朔,免北京、西京、陜西等路今歲絲銀。癸卯,伯顏陛辭南行,奉詔諭宋君臣,相率來附,則趙氏族屬可保無虞,宗廟悉許如故。授故奉使大理王君侯子如珪正八品官。己未,升任城縣為濟州。辛酉,車駕至自上都。丙寅,高麗王王愖遣其樞密副使許珙、將軍趙珪來賀聖誕節。



 九月己巳,太白犯少民。庚午,阿合馬等以軍興國用不足,請復立都轉運司九,量增課程元額,鼓鑄鐵器,官為局賣,禁私造銅器。乙亥,賞清河、新城戰士及死事者銀千兩、鈔百錠。賜西平王所部鴨城戍兵,人馬三匹。丁丑,以襄陽官牛五千八百賜貧民。弛河南鬻馬之禁。賜東西川屯戍蒙古軍糧鈔有差。戊寅,諭太常卿合丹:「去冬享太宮,敕牲無用牛,今其復之。」己卯,太白犯太微西垣上將。壬午,阿術築灣頭堡。乙酉,罷襄陽統軍司。甲午,宋揚州都統姜才將步騎萬五千人攻灣頭堡,阿術、阿塔海擊敗之。賞淮安招討使乞裏迷失及有功將士錦衣銀鈔有差。丙申,以玉昔帖木兒為御史大夫。括江南諸郡書版及臨安秘書省《乾坤寶典》等書。



 冬十月戊戌朔,享於太廟。辛丑,弛北京、義、錦等處獵禁。癸丑,太陰犯畢。



 十一月丁卯,阿里海牙以軍攻潭州。乙亥,伯顏分軍為三,趣臨安;阿剌罕率步騎自建康、四安、廣德以出獨松嶺;董文炳率舟師循海趣許浦、澉浦,以至浙江;伯顏、阿塔海由中道節度諸軍,期並會於臨安。丙子,宋權融、宜、欽三州總管岑從毅,沿邊巡檢使、廣西節制軍馬李維屏等,詣雲南行中書省降。丁丑,阿合馬奏立諸路轉運司凡十一所。己卯,宋都帶等軍次隆興府,宋江西轉運使、知府劉槃以城降。都元帥府檄諭江西諸郡相繼歸附,得府州六、軍四、縣五十六,戶一百五萬一千八百二十九,口二百七萬六千四百。壬午,伯顏大軍至常州,督諸軍登城,四面並進,拔其城。劉師勇變服單騎南走。改順天府為保定府。樞密院言:「兩都、平灤獵戶新簽軍二千,皆貧無力者,宜存恤其家。又新附郡縣有既降復叛,及糾眾為盜犯罪至死者,既已款伏,乞聽權宜處決。」皆從之。中書省臣議斷死罪,詔:「今後殺人者死,問罪狀已白,不必待時,宜即行刑。其奴婢殺主者,具五刑論。」乙酉,阿剌罕克廣德,趣獨松關。丙戌,太陰犯軒轅大星。己丑,遣太常卿合丹以所獲塗金爵三,獻於太廟。庚寅,伯顏遣降人游介實奉璽書副本使於宋,仍以書諭宋大臣。甲午,以高麗國官制僭濫,遣使諭旨,凡省、院、臺、部官名爵號,與朝廷相類者改正之。



 十二月戊戌,填星犯亢。己亥,簽書四川行樞密院事昝順言:「紹慶府、施州、南平及諸蠻呂告、馬蒙、阿永等,有向化之心。又播州安撫楊邦憲、思州安撫田景賢,未知逆順,乞降詔使之自新,並許世紹封爵。」從之。辛丑,董文炳軍次許浦,宋都統制祁安以本軍降。宋主為書,介國信副使嚴忠範侄煥請和。甲辰,伯顏次平江府,宋都統王邦傑以城降。乙巳,免江陵等處今歲田租。丁未,改諸站提領司為通政院。戊申,中書左丞相忽都帶兒與內外文武百寮及緇黃耆庶,請上皇帝尊號曰憲天述道仁文義武大光孝皇帝,皇后曰貞懿順聖昭天睿文光應皇后,不許。太陰犯畢。庚子,宋主復遣尚書夏士林、右史陸秀夫奉書,稱侄乞和。西川滄溪知縣趙龍遣間使入宋,敕流遠方,籍其家。癸亥,敕樞密院:「靖州既降復叛,今已平定,其遣張通判、李信家屬並同叛者赴都。」甲子,答宋國主書,令其來降。丙寅,阿剌罕軍次安吉州,宋安撫使趙與可以城降。升高麗東寧府為路,割江東南康路隸江西省,置馬湖路總管府。省重慶路隆化縣入南川,灤州海山縣入昌黎縣。復華州鄭縣。是歲,衛輝、太原等路旱,河間霖雨傷稼,凡賑米三千七百四十八石、粟二萬四千二百六石。天下戶四百七十六萬四千七十七,斷死罪六十八人。



\end{pinyinscope}