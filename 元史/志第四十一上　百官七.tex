\article{志第四十一上 百官七}

\begin{pinyinscope}

 行中書省,凡十一,秩從一品,掌國庶務,統郡縣,鎮邊鄙,與都省為表裏。國初,有征伐之役,分任軍民之事,皆稱行省,未有定制。中統、至元間,始分立行中書省,因事設官,官不必備,皆以省官出領其事。其丞相,皆以宰執行某處省事系銜。其後嫌於外重,改為某處行中書省。凡錢糧、兵甲、屯種、漕運、軍國重事,無不領之。至元二十四年,改行尚書省,尋復如舊。至大二年,又改行尚書省,二年復如舊。每省丞相一員,從一品;平章二員,從一品;右丞一員,左丞一員,正二品;參知政事二員,從二品,甘肅、嶺北二省各減一員;郎中二員,從五品;員外郎二員,從六品;都事二員,從七品;掾史、蒙古必闍赤、回回令史、通事、知印、宣使,各省設員有差。舊制參政之下,有僉省、有同僉之屬,後罷不置。丞相或置或不置,尤慎於擇人,故往往缺焉。



 河南江北等處行中書省。至元五年,罷隨路奧魯官,詔參政阿里僉行省事,於河南等路立省。二十八年,以河南、江北系要沖之地,又新入版圖,宜於汴梁立省以控治之,遂署其地,統有河南十二路、七府。



 江浙等處行中書省。至元十三年,初置江淮行省,治揚州。二十一年,以地理民事非便,遷於杭州。二十二年,割江北諸郡隸河南,改曰江浙行省,統有三十路、一府。



 江西等處行中書省,至元十四年置。十五年,並入福建行省。十七年,仍置省於龍興府,而福建自為行省,治泉州。二十二年,以福建行省並入江西。二十三年,又以福建省並入江浙。本省統有十八路。



 湖廣等處行中書省。至元十一年,右丞相伯顏伐宋,行中書省事於襄陽,尋以別將分省鄂州,為荊湖等路行中書省。十三年,取潭州,即署省治之。十八年,復徙置鄂州,統有三十路、三府。



 陜西等處行中書省。中統元年,以商挺領秦蜀五路四川行省事。三年,改立陜西四川行中書省,治京兆。至元三年,移治利州。十七年,復還京兆。十八年,分省四川,尋改立四川宣慰司。二十一年,仍合為陜西四川行省。二十三年,四川立行樞密院。本省所轄之地,惟陜西四路、五府。



 四川等處行中書省。國初,其地總於陜西。至元十八年,以陜西行中書分省四川。二十三年,始置四川行省,署成都,統有九路、五府。



 遼陽等處行中書省,至元二十四年置,治遼陽路,統有七路、一府。



 甘肅等處行中書省。中統二年,立行省於中興。至元十年,罷之。十八年復立,二十二年復罷,改立宣慰司。二十三年,徙置中興省於甘州,立甘肅行省。三十一年,分省按治寧夏,尋並歸之。本省治甘州路,統有七路、二州。



 嶺北等處行中書省。國初,太祖定都於哈剌和林河之西,因名其城曰和林,立元昌路。中統元年,世祖遷都中興,始置宣慰司都元帥府。大德十一年,改立和林等處行中書省,右丞相、左丞相各一員。至大四年,省右丞相。皇慶元年,改嶺北等處行中書省,設官如上,治和寧路,統有北邊等處。



 雲南等處行中書省,即古南詔之地。初,世祖征取以為郡縣,嘗封建宗王鎮撫其軍民。至元十一年,始置行省,治中慶路,統有三十七路、五府。



 征東等處行中書省。至元二十年,以征日本國,命高麗王置省,典軍興之務,師還而罷。大德三年,復立行省,以中國之法治之。既而王言其非便,詔罷行省,從其國俗。至治元年復置,以高麗王兼領丞相,得自奏選屬官,治沈陽,統有二府、一司、五道。



 各省屬官:



 檢校所,檢校一員,從七品;書吏二人。



 照磨所,照磨一員,正八品。



 架閣庫,管勾一員,正八品。



 理問所,理問二員,正四品;副理問二員,從五品;知事一員,提控案牘一員。



 都鎮撫司,都鎮撫一員,副都鎮撫一員。



 宣慰司,掌軍民之務,分道以總郡縣,行省有政令則布於下,郡縣有請則為達於省。有邊陲軍旅之事,則兼都元帥府,其次則止為元帥府。其在遠服,又有招討、安撫、宣撫等使,品秩員數,各有差等。



 宣慰使司,秩從二品。每司宣慰使三員,從二品;同知一員,從三品;副使一員,正四品;經歷一員,從六品;都事一員,從七品;照磨兼架閣管勾一員,正九品。凡六道:山東東西道,益都路置。河東山西道,大同路置。淮東道,揚州置。浙東道,慶元路置。荊湖北道,中興路置。湖南道。天臨路置。



 宣慰使司都元帥府,秩從二品,使三員,同知二員,副使二員,經歷二員,知事二員,照磨兼架閣管勾一員。



 廣東道,廣州置。大理金齒等處,蒙慶等處。



 右二府,設官如上。唯蒙慶一府,使二員,同知、副使各一員,經歷、都事亦減一員。



 廣西兩江道,靜江路置。海北海南道,福建道,八番順元等處,察罕腦兒等。



 右五府,宣慰使都元帥三員,副都元帥、僉都元帥事各二員,餘同上。



 宣慰使兼管軍萬戶府,每府宣慰使三員,同知、副使各一員,經歷一員,都事二員,照磨兼管勾一員。



 曲靖等路,羅羅斯,臨安廣西道元江等處。



 都元帥府,都元帥二員,副元帥二員,經歷、知事各一員。



 北庭、隸土番宣慰司。曲先塔林,都元帥三員。蒙古軍,征東。二府,都元帥各一員,副一員。



 元帥府,秩正三品,達魯花赤一員,元帥一員,經歷、知事各一員。



 李店文州,帖城河裏洋脫,朵甘思,常陽,岷州,積石州,洮州路,脫思馬路,十八族。



 右九府,唯李店文州增置同知、副元帥各一員;其餘八府,隸土蕃宣慰司,設官並同。



 宣撫司,秩正三品,每司達魯花赤一員,宣撫一員,同知、副使各二員,僉事一員,計議、經歷、知事各一員,提控案牘架閣一員。損益不同者,各附見於後。



 廣南西道,不置副使、僉事。麗江路,以上隸雲南省。順元等處,播州,思州,以上隸湖廣省。斜南等處。隸四川行省,不置僉事、計議。



 安撫司,秩正三品。每司達魯花赤一員,安撫使一員,同知、副使、僉事各一員,經歷、知事各一員。損益不同者,各附見於後。



 師壁洞,不置達魯花赤。永順等處,散毛洞,以上隸四川省。羅番遏蠻軍,不置達魯花赤。程番武盛軍,金石番太平軍,臥龍番南寧州,小龍番靜蠻軍,不置同知、副使。大龍番應天府,洪番永盛軍,方番河中府,蘆番靜海軍,不置知事。新添葛蠻。以上隸湖廣省。



 招討司,秩正三品,達魯花赤一員,招討使一員,經歷一員。



 土番,剌馬剛等處,天全,倴不思,沿邊溪洞,以下各置副使一員,無達魯花赤。唆尼,諸番,徵沔,長河西里管軍,簷裏管軍,脫思馬田地。



 諸路萬戶府:



 上萬戶府,管軍七千之上。達魯花赤一員,萬戶一員,俱正三品,虎符;副萬戶一員,從三品,虎符。



 中萬戶府,管軍五千之上。達魯花赤一員,萬戶一員,俱從三品,虎符;副萬戶一員,正四品,金牌。



 下萬戶府,管軍三千之上。達魯花赤一員,萬戶一員,俱從三品,虎符;副萬戶一員,從四品,金牌。其官皆世襲,有功則升之。每府設經歷一員,從七品;知事一員,從八品;提控案牘一員。



 鎮撫司,鎮撫二員,蒙古、漢人參用。上萬戶府正五品,中萬戶府從五品,俱金牌;下萬戶府正六品,銀牌。



 上千戶所,管軍七百之上。達魯花赤一員,千戶一員,俱從四品,金牌;副千戶一員,正五品,金牌。



 中千戶所,管軍五百之上。達魯花赤一員,千戶一員,俱正五品,金牌;副千戶一員,從五品,金牌。



 下千戶所,管軍三百之上。達魯花赤一員,千戶一員,俱從五品,金牌;副千戶一員,正六品,銀牌。



 彈壓二員,蒙古、漢人參用。上千戶所從八品,中下二所正九從九品內銓注。



 上百戶所,百戶二員,蒙古一員,漢人一員,俱從六品,銀牌。



 下百戶所,百戶一員,從七品,銀牌。



 儒學提舉司,秩從五品。各處行省所署之地,皆置一司,統諸路、府、州、縣學校祭祀教養錢糧之事,及考校呈進著述文字。每司提舉一員,從五品;副提舉一員,從七品;吏目一人,司吏二人。



 蒙古提舉學校官,秩從五品。提舉一員,從五品;同提舉一員,從七品。至元十八年置。惟江浙、湖廣、江西三省有之,餘省不置。



 官醫提舉司,秩從六品,提舉一員,同提舉一員,副提舉一員,掌醫戶差役詞訟。至元二十五年置。河南、江浙、江西、湖廣、陜西五省各立一司,餘省並無。



 兩淮都轉運鹽使司,秩正三品。國初,兩淮內附,以提舉馬里範章專掌鹽課之事。至元十四年,始置司於揚州。使二員,正三品;同知二員,正四品;副使一員,正五品;運判二員,正六品;經歷一員,從七品;知事一員,從八品;照磨一員,從九品。三十年,悉罷所轄鹽司,以其屬置場官。大德四年,復置批驗所於真州、採石等處。



 鹽場二十九所,每場司令一員,從七品;司丞一員,從八品;管勾一員,從九品。辦鹽各有差。



 呂四場,餘東場,餘中場,餘西場,西亭場,金沙場,石塂場,掘港場,豐利場,馬塘場,拼茶場,角斜場,富安場,安豐場,梁垛場,東臺場,河垛場,丁溪場,小海場,草堰場,白駒場,劉莊場,五祐場,新興場,廟灣場,莞瀆場,板浦場,臨洪場,徐瀆浦場。



 批驗所,每所提領一員,正七品;大使一員,正八品;副使一員,正九品。掌批驗鹽引。



 兩浙都轉運鹽使司,秩正三品,使二員,同知二員,運判二員,經歷、知事各一員,照磨一員。至元十四年,置司杭州。大德三年,定其產鹽之地,立場有差,仍於杭州、嘉興、紹興、溫、臺等處,設檢校四所,專驗鹽袋,毋過常度。



 鹽場三十四所,每所司令一員,從七品;司丞一員,從八品;管勾一員,從九品。



 仁和場,許村場,西路場,下沙場,青村場,表部場,浦東場,橫浦場,蘆瀝場,海沙場,鮑郎場,西興場,錢清場,三江場,曹娥場,石堰場,鳴鶴場,清泉場,長山場,穿山場,岱山場,玉泉場,蘆花場,大嵩場,昌國場,永嘉場,雙穗場,天富南監,長林場,黃巖場,杜瀆場,天富北監,長亭場,龍頭場。



 福建等處都轉運鹽使司,秩正三品,使二員,同知二員,運判二員,經歷、知事各一員,照磨一員。至元十四年,始置市舶司,領煎鹽徵課之事。二十四年,改立鹽運司。二十九年罷,立提舉司。大德四年,復為運司。九年復罷,並入元帥府兼掌之。十年,復立都提舉司。至大四年,復升運司,徑隸行省。凡置鹽場七所:



 鹽場七所,每所司令一員,從七品;司丞一員,從八品;管勾一員,從九品。



 海口場,牛田場,上里場,惠安場,潯美場,浯洲場,水丙洲場。



 廣東鹽課提舉司。至元十三年,始從廣州煎辦鹽課。十六年,隸江西鹽鐵茶都轉運司。二十二年,並入宣慰司。二十三年,置市舶提舉司。大德四年,改廣東鹽課提舉司。提舉一員,從五品;同提舉一員,從六品;副提舉一員,從七品。其屬附見:



 鹽場十三所,每所司令一員,從七品;司丞一員,從八品;管勾一員,從九品。



 靖康場,歸德場,東莞場,黃田場,香山場,矬峒場,雙恩場,咸水場,淡水場,石橋場,隆井場,招收場,小江場。



 四川茶鹽轉運司。成都鹽井九十五處,散在諸郡山中。至元二年,置興元四川轉運司,專掌煎熬辦課之事。八年罷之。十六年,復立轉運司。十八年,並入四道宣慰司。十九年,復立陜西四川轉運司,通轄諸課程事。二十二年,置四川茶鹽運司,秩從三品,使一員,同知、副使、運判各一員,經歷、知事、照磨各一員。



 鹽場一十二所,每所司令一員,從七品;司丞一員,從八品;管勾一員,從九品。



 簡鹽場,隆鹽場,綿鹽場,潼川場,遂實場,順慶場,保寧場,嘉定場,長寧場,紹慶場,雲安場,大寧場。



 廣海鹽課提舉司,至元三十一年置,專職鹽課,秩正四品。都提舉二員,從四品;同提舉二員,從五品;副提舉二員,從六品;知事一員,提控案牘一員。



 市舶提舉司。至元二十三年,立鹽課市舶提舉司,隸廣東宣慰司。三十年,立海南博易提舉司。至大四年罷之,禁下番船只。延祐元年,弛其禁,改立泉州、廣東、慶元三市舶提舉司。每司提舉二員,從五品;同提舉二員,從六品;副提舉二員,從七品;知事一員。



 海道運糧萬戶府,至元二十年置,秩正三品,掌每歲海道運糧供給大都。達魯花赤一員,萬戶一員,並正三品;副萬戶四員,從三品;經歷一員,從七品;知事一員,從八品;照磨一員,從九品;鎮撫二員,正五品。其屬附見:



 海運千戶所,秩正五品。達魯花一員,千戶二員,並正五品;副千戶三員,從五品。若溫臺,若慶元紹興,若杭州嘉興,若昆山崇明、常熟江陰等處,凡五所,而平江又有海運香莎糯米千戶所。



 諸路總管府,至元初置。二十年,定十萬戶之上者為上路,十萬戶之下者為下路,當沖要者,雖不及十萬戶亦為上路。上路秩正三品,達魯花赤一員,總管一員,並正三品,兼管勸農事,江北則兼諸軍奧魯,同知、治中、判官各一員。下路秩從三品,不置治中員,而同知如治中之秩,餘悉同上。至元二十三年,置推官二員,專治刑獄,下路一員。經歷一員,知事一員或二員,照磨兼承發架閣一員,司吏無定制,隨事繁簡以為多寡之額;譯史、通事各一人。其屬附見:



 儒學教授一員,秩九品。諸路各設一員,及學正一員、學錄一員。其散府、上中州,亦設教授一員,下州設學正一員。



 蒙古教授一員,正九品。



 醫學教授一員。



 陰陽教授一員。



 司獄司,司獄一員,丞一員。



 平準行用庫,提領、大使、副使各一員。



 織染局,局使一員,副使一員。



 雜造局,大使一員,副使一員。



 府倉,大使一員,副使一員。



 惠民藥局,提領一員。



 稅務,提領一員,大使、副使各一員。



 錄事司,秩正八品。凡路府所治,置一司,以掌城中戶民之事。中統二年,詔驗民戶,定為員數。二千戶以上,設錄事、司候、判官各一員;二千戶以下,省判官不置。至元二十年,置達魯花赤一員,省司候,以判官兼捕盜之事,典史一員。若城市民少,則不置司,歸之倚郭縣。在兩京,則為警巡院。獨杭州置四司,後省為左、右兩司。



 散府,秩正四品,達魯花赤一員,知府或府尹一員,領勸農奧魯與路同;同知一員,判官一員,推官一員,知事一員,提控案牘一員。所在有隸諸路及宣慰司、行省者,有直隸省部者,有統州縣者,有不統縣者,其制各有差等。



 諸州。中統五年,並立州縣,未有等差。至元三年,定一萬五千戶之上者為上州,六千戶之上者為中州,六千戶之下者為下州。江南既平,二十年,又定其地五萬戶之上者為上州,三萬戶之上者為中州,不及三萬戶者為下州。於是升縣為州者四十有四。縣戶雖多,附路府者不改。上州:達魯花赤、州尹秩從四品,同知秩正六品,判官秩正七品。中州:達魯花赤、知州並正五品,同知從六品,判官從七品。下州:達魯花赤、知州並從五品,同知正七品,判官正八品,兼捕盜之事。參佐官:上州,知事、提控案牘各一員;中州,吏目、提控案牘各一員;下州,吏目一員或二員。



 諸縣。至元三年,合並江北州縣。六千戶之上者為上縣,二千戶之上者為中縣,不及二千戶者為下縣。二十年,又定江淮以南,三萬戶之上者為上縣,一萬戶之上者為中縣,一萬戶之下者為下縣。上縣,秩從六品,達魯花赤一員,尹一員,丞一員,簿一員,尉一員,典史二員。中縣,秩正七品,不置丞,餘悉如上縣之制。下縣,秩從七品,置官如中縣,民少事簡之地,則以簿兼尉。後又別置尉,尉主捕盜之事,別有印。典史一員。巡檢司,秩九品,巡檢一員。



 諸軍,唯邊遠之地有之,各統屬縣,其秩如下州,其設官置吏亦如之。



 諸蠻夷長官司。西南夷諸溪洞各置長官司,秩如下州,達魯花赤、長官、副長官,參用其土人為之。



 各處脫脫禾孫,掌辨使臣奸偽。正一員,從五品;副一員,正七品。



 勛一十階:



 上柱國正一品輕車都尉從三品



 柱國從一品上騎都尉正四品



 上護軍正二品騎都尉從四品



 護軍從二品驍騎尉正五品



 上輕車都尉正三品飛騎尉從五品



 爵八等:



 王正一品郡侯從三品



 郡王從一品郡伯正四品



 國公正二品郡伯從四品



 郡公從二品縣子正五品



 郡侯正三品縣男從五品



 右勛爵,若上柱國、郡王、國公,時有除拜者,餘則止於封贈用之。



 文散官四十二:



 開府儀同三司中憲大夫



 儀同三司中順大夫以上正四品



 特進朝請大夫



 崇進朝散大夫



 金紫光祿大夫朝列大夫以上從四品



 銀青榮祿大夫以上俱正一品奉政大夫



 光祿大夫奉議大夫以上正五品



 榮祿大夫以上從一品奉直大夫



 資德大夫奉訓大夫以上從五品



 資政大夫承德郎



 資善大夫以上正二品承直郎以上正六品



 正奉大夫儒林郎



 通奉大夫承務郎以上從六品



 中奉大夫以上從二品文林郎



 正議大夫承事郎以上正七品



 通議大夫徵事郎



 嘉議大夫以上正三品從事郎以上從七品



 太中大夫登仕郎



 中大夫將仕郎以上正八品



 亞中大夫以上從三品,舊為少中,延祐改亞中。登仕佐郎



 中議大夫將仕佐郎以上從八品



 右文散官四十二階,由一品至五品為宣授,六品至九品為敕授。敕授則中書署牒,宣授則以制命之。一品至五品者服紫,六品至七品者服緋,八品至九品者服綠,武官以下皆如之。其官常對品,惟九品無散官,則但舉其職而已,武官雜職亦如之。



 武散官三十四階:



 龍虎衛上將軍宣武將軍以上從四品



 金吾衛上將軍武節將軍



 驃騎衛上將軍以上正二品武德將軍以上正五品



 奉國上將軍武義將軍



 輔國上將軍武略將軍以上從五品



 鎮國上將軍以上從二品承信校尉



 昭武大將軍昭信校尉以上正六品



 昭勇大將軍忠武校尉



 昭毅大將軍以上正三品忠顯校尉以上從六品



 安遠大將軍忠勇校尉



 定遠大將軍忠翊校尉以上正七品



 懷遠大將軍以上從三品修武校尉



 廣威將軍敦武校尉以上從七品



 宣威將軍保義校尉



 明威將軍以上正四品進義校尉以上正八品



 信武將軍保義副尉



 顯武將軍進義副尉以上從八品



 右武散官三十四階,自龍虎衛上將軍至進義副尉,由正二品至從八品,其除授具前。



 內侍散官一十四:



 中散大夫正二品通御郎從五品



 中引大夫從二品侍直郎正六品



 中御大夫正三品內直郎從六品



 侍中大夫從三品司謁郎正七品



 中衛大夫正四品司閽郎從七品



 中涓大夫從四品司奉郎正八品



 通侍郎正五品司引郎從八品



 右內侍品秩一十四階,自中散至司引,由正二品至從八品,其除授具前。



 司天散官一十四:



 欽象大夫從三品候儀郎從六品



 明時大夫司正郎正七品



 頒朔大夫以上正四品平秩郎從七品



 保章大夫從四品正紀郎



 司玄大夫正五品挈壺郎以上正八品



 授時郎從五品司歷郎



 靈臺郎正六品司辰郎以上從八品



 右司天品秩一十四階,自欽象至司辰,由從三品至從八品,其除授具前。



 太醫散官一十五:



 保宜大夫成和郎從六品



 保康大夫以上從三品成全郎正七品



 保安大夫醫正郎從七品



 保和大夫以上正四品醫效郎



 保順大夫從四品醫候郎以上正八品



 保沖大夫正五品醫痊郎



 保全郎從五品醫愈郎以上從八品



 成安郎正六品



 右太醫品秩一十五階,自保宜至醫愈,亦由從三品至從八品,其除授具前。



 教坊司散官十五:



 雲韶大夫司樂郎從六品



 仙韶大夫以上從三品協樂郎正七品



 長寧大夫和樂郎從七品



 德和大夫以上正四品司音郎



 協律大夫從四品司律郎以上正八品



 嘉成大夫正五品和聲郎



 純和郎從五品和節郎以上從八品



 調音郎正六品



 右教坊品秩一十五階,自雲韶至和節,由從三品至從八品,其除授具前。



\end{pinyinscope}