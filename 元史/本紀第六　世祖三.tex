\article{本紀第六 世祖三}

\begin{pinyinscope}

 二年春正月辛未朔,日有食之。癸酉,山東廉訪使言:「真定路總管張宏,前在濟南,乘變盜用官物。」詔以宏嘗告李璮反駁佛道,非難朱陸,集永嘉之學而成一家。著作有《習學記,免宏死罪,罷其職,徵贓物償官。邳州萬戶張邦直等違制販馬,並處死。敕徙鎮海、百八里、謙謙州諸色匠戶於中都,給銀萬五千兩為行費。又徙奴懷、忒木帶兒砲手人匠八百名赴中都,造船運糧。己卯,北京路行省給札剌赤戶東徙行糧萬石。以鄧州監戰訥懷、新舊軍萬戶董文炳並為河南副統軍。甲申,詔申嚴越界販馬之禁,違者處死。乙酉,以河南北荒田分給蒙古軍耕種。戊子,諸王塔察兒使臣闊闊出至北京花道驛,手殺驛吏郝用、郭和尚,有旨征鈔十錠給其主贖死。庚寅,城西番匣答路。癸巳,八東乞兒部牙西來朝,貢銀鼠皮二千,賜金、素幣各九、帛十有八。武城縣王氏妻崔一產三男。丁酉,給親王玉龍答失部民糧二千石。高麗國王王禃遣其弟廣平公恂奉表來貢。



 二月辛丑朔,元帥按東與宋兵戰於釣魚山,敗之,獲戰艦百四十六艘。甲辰,初立宮闈局。戊申,賜親王兀魯帶河間王印,給所部米千石。丁巳,車駕幸上都。癸亥,並六部為四,以麥術丁為吏禮部尚書,馬亨戶部尚書,嚴忠範兵刑部尚書,別魯丁工部尚書。禁山東東路私煎硝堿。甲子,以蒙古人充各路達魯花赤,漢人充總管,回回人充同知,永為定制。以同知東平路宣慰使寶合丁為平章政事,山東廉訪使王晉為參知政事。廉希憲、商挺罷。詔並諸王只必帖木兒所設管民官屬。詔諭總統所:「僧人通五大部經者為中選,以有德業者為州郡僧錄、判、正副都綱等官,仍於各路設三學講、三禪會。」



 三月癸酉,骨嵬國人襲殺吉裏迷部兵,敕以官粟及弓甲給之。丁亥,敕邊軍習水戰、屯田。誅宋諜李富住。乙未,罷南北互市,括民間南貨,官給其直。遼東饑,發粟萬石、鈔百錠賑之。



 夏四月戊午,賜諸王合必赤、亦怯烈金、素幣各四,拜行金幣一。



 五月壬午,賞萬戶晃裡答兒所部征吐蕃功銀四百五十兩。戊子,禁北京、平灤等處人捕獵。庚寅,令:「軍中犯法,不得擅自誅戮,罪輕斷遣,重者聞奏。」敕上都商稅、酒醋諸課毋徵,其榷鹽仍舊;諸人自願徙居永業者,復其家。詔西川、山東、南京等路戍邊軍屯田。



 閏五月癸卯,升蓚縣為景州。辛亥,檢核諸王兀魯帶部民貧無孳畜者三萬七百二十四人,人月給米二斗五升,四閱月而止。丙辰,雅州碉門宣撫使請復碉門城邑,詔相度之。癸亥,移秦蜀行省於興元。丙寅,命四川行院分兵屯田。丁卯,分四親王南京屬州,鄭州隸合丹,鈞州隸明裏,睢州隸孛羅赤,蔡州隸海都,他屬縣復還朝廷。以平章政事趙璧行省於南京、河南府、大名、順德、洺磁、彰德、懷孟等路,平章政事廉希憲行省事於東平、濟南、益都、淄萊等路,中書左丞姚樞行省事於西京、平陽、太原等路。詔:「諸路州府,若自古名郡,戶數繁庶,且當沖要者,不須改並。其戶不滿千者,可並則並之,各投下者,並入所隸州城。其散府州郡戶少者,不須更設錄事司及司候司,附郭縣止令州府官兼領。括諸路未占籍戶任差職者以聞。」



 六月戊辰朔,新得州安撫向良言:「頃以全城內附,元領軍民流散南界者,多欲歸順,並乞招徠。」從之。又敕良以所領新降軍民移戍通江縣,行新得州事。辛未,賜阿術所部馬價鈔一千二十三錠有奇。丙子,太陰犯心大星。戊寅,移山東統軍司於沂州。萬戶重喜立十字路。復正陽,命禿剌戍之。己卯,以淇州隸懷孟路。高麗國王王禃遣其臣榮胤伯奉表來賀聖誕節。千戶闊闊出部民乏食,賜鈔賑之。王晉罷。樞密院臣言:「各路出征逃亡漢軍,及貧難未起戶,並投下隱匿事故者,宜一概發遣應役。」從之。敕行院及諸軍將校卒伍,須正身應役,違者罪之。



 秋七月辛酉,益都大蝗饑,命減價糶官粟以賑。癸亥,安南國王陳光昞遣使奉表來貢。甲子,詔賜光昞至元三年歷。



 八月丙子,濟南路鄒平縣進芝草一本。戊寅,高麗國王王禃遣使來貢方物。己卯,諸宰職皆罷,以安童為中書右丞相,伯顏為中書左丞相。戊子,召許衡於懷孟,楊誠於益都。車駕至自上都。



 九月戊戌,以將有事太廟,取大樂工於東平,預習儀禮。敕江淮沿邊樹柵,徐、宿、邳三州助役徒。庚子,皇孫鐵穆爾生。丁巳,賞諸王只必帖木兒麾下河西戰功銀二百五十兩。



 冬十月己卯,享於太廟。癸未,敕順天張柔、東平嚴忠濟、河間馬總管、濟南張林、太原石抹總管等戶,改隸民籍。統軍抄不花、萬戶懷都麾下軍士所俘宋人九十三口,官贖為民。其私越禁界掠獲者四十五人,許令親屬完聚,並種田內地。戊子,詔隨路私商曾入南界者,首實免罪充軍。



 十一月丙申,召李昶於東平。辛丑,賜諸王只必帖木兒銀二萬五千兩、鈔千錠。癸丑,賞楊文安戰功金五十兩,所部軍銀六百兩及幣帛有差。甲子,詔事故貧難軍不堪應役者,以兩戶或三戶合並正軍一名,其丁單力備者,許顧人應役。



 十二月己巳,省並州縣凡二百二十餘所。庚午,宋子貞言:「朝省之政,不宜數行數改。又刑部所掌,事干人命,尚書嚴忠範年少,宜選老於刑名者為之。」又請罷北京行中書省,別立宣慰司以控制東北州郡。並從之。禁朝省告訐以息爭訟。辛未,以諸王也速不花所部戍西番軍屢有戰功,賞銀三百兩。癸酉,召張德輝於真定,徒單公履於衛州。丁丑,詔諭高麗,賜至元三年歷日。癸未,賜劉秉忠金五十兩。甲申,賜伯顏、宋子貞、楊誠銀千兩、鈔六十錠。丁亥,敕選諸翼軍富強才勇者萬人,充侍衛親軍。己丑,瀆山大玉海成,敕置廣寒殿。是歲,戶一百五十九萬七千六百一,絲九十八萬六千二百八十八斤,包銀鈔五萬七千六百八十二錠。賜諸王金、銀、幣、帛如歲例。彰德、大名、南京、河南府、濟南、淄萊、太原、弘州雹,西京、北京、益都、真定、東平、順德、河間、徐、宿、邳蝗旱,太原霜災。斷死罪四十二人。



 三年春正月乙未朔,高麗國王王禃遣使來賀。丙午,遣朵端、趙璧持詔撫諭四川將吏軍民。壬子,立制國用使司,以阿合馬為使。癸丑,選女直軍二千為侍衛軍。四川行樞密院謀取嘉定,請益兵,命朵端、趙璧摘諸翼蒙古、漢軍六千人付之。



 二月丙寅,廉希憲、宋子貞為平章政事,張文謙復為中書左丞,史天澤為樞密副使。癸酉,立沈州以處高麗降民。壬午,平陽路僧官以妖言惑眾伏誅。以中書右丞張易同知制國用使司事,參知政事,張惠為制國用副使。癸未,車駕幸上都。甲申,罷西夏行省,立宣慰司。初制太常禮樂工冠服。立東京、廣寧、懿州、開元、恤品、合懶、婆娑等路宣撫司。乙酉,蠲中都今年包銀四分之一。詔理斷阿術部下所俘人口、畜牧及其草地為民侵種者。以制國用使司條畫諭中外官吏。



 三月辛巳,分衛輝路為親王玉龍答失分地。戊戌,賑水達達民戶饑。己未,王晉及侍中和哲斯、濟南益都轉運使王明,以隱匿鹽課,皆伏誅。



 夏四月丁卯,五山珍御榻成,置瓊華島廣寒殿。亳州水軍千戶胡進等領騎兵渡淝水,逾荊山,與宋兵戰,殺獲甚眾,賞鈔幣有差。庚午,敕僧、道祈福於中都寺觀。詔以僧機為總統,居廣壽寺。己卯,申嚴瀕海私鹽之禁。敕宮燭毋彩繪。



 五月乙未,遣使諸路慮囚。庚子,敕太醫院領諸路醫戶、惠民藥局。辛丑,以黃金飾渾天儀。丙午,浚西夏中興漢延、唐來等渠。凡良田為僧所據者,聽蒙古人分墾。丙辰,罷益都行省。蠲平灤、益都質子戶賦稅之半。



 六月丁卯,封皇子南木合為北平王,以印給之。辛未,徙歸化民於清州興濟縣屯田,官給牛具。壬申,賜劉整畿內地五十頃。癸酉,以千戶扎剌兒沒於王事,賜其妻銀二百五十兩。丙子,立漕運司。戊寅,以陜西行省平章賽典赤等政事修治,賜銀五千兩。命山東統軍副使王仲仁督造戰船於汴。申嚴陜西、河南竹禁。立拱衛司。



 秋七月丙申,罷息州安撫司。壬寅,詔上都路總管府遇車駕巡幸,行留守司事,車駕還,即復舊。丙午,遣使祀五岳四瀆。甲寅,添內外巡兵。外路每百戶選中產者一人充之,其賦令餘戶代輸,在都增武衛軍四百。己未,以崞、代、堅、臺四州隸忻州。詔令西夏避亂之民還本籍,成都新民為豪家所庇者皆歸之州縣。詔招集逃亡軍,限百日詣所屬陳首,原其罪,貧者並戶應役。



 八月癸亥,賜丞相伯顏第一區。丁卯,以兵部侍郎黑的、禮部侍郎殷弘使日本,賜書曰:「皇帝奉書日本國王:朕惟自古小國之君,境土相接,尚務講信修睦,況我祖宗受天明命,奄有區夏,遐方異域畏威懷德者,不可悉數。朕即位之初,以高麗無辜之民,久瘁鋒鏑,即令罷兵,還其疆埸,反其旄倪。高麗君臣,感戴來朝,義雖君臣,而歡若父子。計王之君臣,亦已知之。高麗,朕之東籓也。日本密邇高麗,開國以來,時通中國,至於朕躬,而無一乘之使以通和好。尚恐王國知之未審,故特遣使持書布告朕心,冀自今以往,通問結好,以相親睦。且聖人以四海為家,不相通好,豈一家之理哉?以至用兵,夫孰所好,王其圖之。」又詔高麗導去使至其國。戊子,高麗國王王禃遣其大將軍樸琪來賀聖誕節。阿術略地蘄、黃,俘獲以萬計。



 九月戊午,車駕至自上都。



 冬十月庚申朔,降德興府為奉聖州。癸亥,高麗使還,以王禃病,詔和藥賜之。丁丑,徙平陽經籍所於京師。更敕牒舊式。太廟成,丞相安童、伯顏言:「祖宗世數、尊謚廟號、增祀四世、各廟神主、配享功臣、法服祭器等事,皆宜定議。」命平章政事趙璧等集群臣議,定為八室。申禁京畿畋獵。壬午,命制國用使司造神臂弓千張、矢六萬。



 十一月辛卯,初給京、府、州、縣、司官吏俸及職田。戊戌,瀕御河立漕倉。丁未,申嚴殺牛馬之禁。宋子貞致仁。辛亥,以忽都答兒為中書左丞相。詔禁天文、圖讖等書。丙辰,千戶散竹帶以嗜酒失所守大良平,罪當死,錄其前功免死,令往東川軍前自效。詔建都使復歸朝。又詔嘉定等府沿江一帶城堡早降。又詔四川行樞密院遣人告諭江、漢、庸、蜀等效順,具官吏姓名,對階換授,有功者遷,有才者用;民無生理者以衣糧賑之,願遷內地者給以田廬,毋令失所。



 十二月庚申,給諸王合必赤行軍印。辛酉,詔改四川行樞密院為行中書省,以賽典赤、也速帶兒等僉行中書省事。甲子,立諸路洞冶所。以梁成生擒宋總轄官,授同知開府事,佩金符。減輝州竹課,先是官取十之六,至是減其二。丁亥,詔安肅公張柔、行工部尚書段天祐等同行工部事,修築宮城。並太府監入宣徽院,仍以宣徽使專領監事。詔賜高麗以至元四年歷日,仍慰諭之。建大安閣於上都。鑿金口,導盧溝水以漕西山木石。敕:「諸越界私商及諜人與偽造鈔者,送京師審核。」是歲,天下戶一百六十萬九千九百三。東平、濟南、益都、平灤、真定、洺磁、順天、中都、河間、北京蝗,京兆、鳳翔旱。斷死罪九十六人。賜諸王金、銀、幣、帛如歲例。



 四年春正月甲午,陜西行省以開州新得復失,請益兵,敕平陽、延安等處簽民兵三千人,山東、河南、懷孟、潼川調兵七千人益之。丁酉,申嚴平陽等處私鹽之禁。壬寅,立茶速禿水十四驛。癸卯,敕修曲阜宣聖廟。乙巳,百濟遣其臣梁浩來朝,賜以錦繡有差。禁僧官侵理民訟。辛亥,封安肅公張柔為蔡國公,以趙璧為樞密副使。立諸路洞冶都總管府。癸丑,敕封昔木土山為武定山,其神曰武定公;泉為靈淵,其神曰靈淵侯。僉蒙古軍,戶二丁三丁者出一人為軍,四丁五丁者二人,六丁七丁者三人。乙卯,高麗國王王禃遣使來朝,詔撫慰之。戊午,立提點宮城所。析上都隆興府自為一路,行總管府事;立開元等路轉運司。城大都。



 二月庚申,粘合南合復平章事,阿里復為中書右丞。丁卯,改經籍所為弘文院,以馬天昭知院事。丁亥,括西夏民田,徵其租。車駕幸上都。詔陜西行省招諭宋人。又詔嘉定、瀘州、重慶、夔府、涪、達、忠、萬及釣魚、禮義、大良等處官吏軍民有能率眾來降者,優加賞擢。



 三月己丑,復以耶律鑄為中書左丞相。辛卯,自潼關至蘄縣立河渡官八員,以察奸偽。乙未,敕中都路建習樂堂,使樂工隸業其中。己亥,賜皇子燕王、忙阿剌、那沒罕、忽哥赤銀三萬兩。辛丑,夏津縣大雨雹。壬寅,安童言:「比者省官員數,平章、左丞各一員,今丞相五人,素無此例。臣等議擬設二丞相,臣等蒙古人三員,惟陛下所命。」詔以安童為長,史天澤次之,其餘蒙古、漢人參用,勿令員數過多;又詔宜用老成人如姚樞等一二員同議省事。丁巳,耶律鑄制宮縣樂成,詔賜名《大成》。夏四月申子,新築宮城。辛未,遣使祀岳瀆。



 五月丁亥朔,日有食之。敕上都重建孔子廟。乙未,應州大水。丙申,威州山後大番弄麻等十一族來附,賜以璽書、金銀符。己酉,以捕獵戶達魯花赤偽造銀符,處死。壬子,敕諸路官吏俸,令包銀民戶,每四兩增納一兩以給之。丙辰,析東平之博州五城別為一路。六月壬戌,以中都、順天、東平等處蠶災,免民戶絲料輕重有差。乙丑,復以史天澤為中書左丞相,忽都答兒、耶律鑄並降平章政事,伯顏降中書右丞,廉希憲降中書左丞,阿里、張文謙並降參知政事。乙酉,賜諸王玉龍答失銀五千兩、幣三百,歲以為常。罷宣徽院。黑的、殷弘以高麗使者宋君斐、金贊不能導達至日本來奏,降詔責高麗王王禃,仍令其遣官至彼宣布,以必得要領為期。秋七月丙戌朔,敕自中興路至西京之東勝立水驛十。戊戌,罷息州安撫岳林,以其民隸南京路;罷懷孟路安撫李宗傑,以其民隸本路。發鞏昌、鳳翔、京兆等處未占籍戶一千,修治四川山路、橋梁、棧道。大名路達魯花赤愛魯、總管張弘範等盜用官錢,罷之。壬寅,申嚴京畿牧地之禁。甲寅,詔亦即納新附貧民,從人借貸困不能償者,官為償之,仍給牛具、種實及糧食。簽東京軍千八百人充侍衛軍。八月庚申,填星犯天樽。辛酉,申嚴平灤路私鹽酒醋之禁。丙寅,復立宣徽院,以前中書右丞相線真為使。丁丑,封皇子忽哥赤為雲南王,賜駝鈕金鍍銀印。壬午,太白犯軒轅大星。命怯綿征建都。高麗國王王禃遣其秘書監郭汝弼來賀聖誕節。阿術略地至襄陽,俘生口五萬、馬牛五千。宋人遣步騎來拒,阿術率騎兵敗之。九月壬辰,作玉殿於廣寒殿中。乙未,總帥汪良臣請立寨於毋章德山,控扼江南,以當釣魚之沖,從之。戊申,以許衡為國子祭酒。安南國王陳光昞遣使來貢,優詔答之。立大理等處行六部,以闊闊帶為尚書兼雲南王傅,柴禎尚書兼府尉,寧源侍郎兼司馬。庚戌,遣雲南王忽哥赤鎮大理、鄯闡、茶罕章、赤禿哥兒、金齒等處,詔撫諭吏民。又詔諭安南國,俾其君長來朝,子弟入質,編民出軍役、納賦稅,置達魯花赤統治之。癸丑,申嚴西夏中興等路僧尼、道士商稅、酒醋之禁。車駕至自上都。鶚請立選舉法,有旨令議舉行,有司難之,事遂寢。冬十月辛酉,制國用司言:「別怯赤山石絨織為布,火不能然。」詔採之。壬戌,賜駙馬不花銀印。魚通巖州等處達魯花赤李福招諭西番諸族酋長以其民入附,以阿奴版的哥等為喝吾等處總管,並授璽書及金銀符。鐵旗城後番官官折蘭遣其子天郎持先受憲宗璽書、金符,乞改授新命,從之。甲子,歲星犯軒轅大星。辛未,太原進嘉禾二本,異畝同穎。甲戌,賑新附民陳忠等鈔。丁丑,制國用使司請量節經用,從之。庚辰,定品官子孫廕敘格。十一月乙酉,享於太廟。戊戌,立新蔡縣,以忽察、李家奴統所部兵戍之。甲辰,立夔府路總帥府,戍開州。乙巳,填星犯天樽距星。申嚴京畿畋獵之禁。南京宣慰劉整赴闕,奏攻宋方略,宜先從事襄陽。十二月甲戌,賞河南路統軍使訥懷所部將士戰功銀九千六百五十兩,鈔幣、鞍勒有差。丙子,賑親王移相哥所部饑民。丁丑,給遼東新簽軍布六萬匹。己卯,立遼東路水驛七。賞元帥阿術部下有功將士二千二十五人,銀五萬五千三百兩、金五十兩,及錦彩、鞍勒有差。庚辰,簽女直、水達達軍三千人。立諸位斡脫總管府。省平陽路岳陽、和州二縣入冀氏,復置霸州益津縣,省安西路櫟陽縣入臨潼。是歲,天下戶口一百六十四萬四千三十。山東、河南北諸路蝗,順天束鹿縣旱,免其租。斷死罪一百十四人。賜諸王金、銀、幣、帛如歲例。



 五年春正月甲午,太陰犯井。庚子,上都建城隍廟。辛丑,敕陜西五路四川行省造戰艦五百艘付劉整。高麗國王王禃遣其弟淐來朝。詔以禃飾辭見欺,面數其事於淐,切責之。復遣北京路總管於也孫脫、禮部郎中孟甲持詔往諭,令具表遣海陽公金俊、侍郎李藏用與去使同來以聞。庚戌,賜高麗國新歷。閏月戊午,以陳、亳、潁、蔡等處屯田戶充軍;令益都漏籍戶四千淘金登州棲霞縣,每戶輸金歲四錢。



 二月戊子,太陰犯天關。己丑,太陰犯井。給河南、山東貧乏軍士鈔。戊戌,改軍器局為軍器監。辛丑,百戶渾都速駐營濟南路屬縣三年。脅取民飲食糧料當粟五千石,敕杖決之,仍償粟千石。析甘州路之肅州自為一路。



 三月丙寅,罷諸路四品以下子孫入質者。田禹妖言,敕減死流之遠方。禁民間兵器,犯者驗多寡定罪。甲子,敕怯綿率兵二千招諭建都。壬申,改毋章德山為定遠城,武群山為武勝軍。丁丑,敕阿里等詣軍前閱視軍籍。罷諸路女直、契丹、漢人為達魯花赤者,回回、畏兀、乃蠻、唐兀人仍舊。



 夏四月壬寅,遣使祀岳瀆。



 五月辛亥朔,以太醫院、拱衛司、教坊司及尚食、尚果、尚醖三局隸宣徽院。癸亥,都元帥百家奴拔宋嘉定五花、石城、白馬三寨。癸酉,賜諸王禾忽及八剌合幣帛六萬匹。



 六月辛巳朔,濟南王保和以妖言惑眾,謀作亂,敕誅首惡五人,餘勿論。甲申,中山大雨雹。阿術言:「所領者蒙古軍,若遇山水寨柵,非漢軍不可。宜令史樞率漢軍協力征進。」從之。戊申,東平等處蝗。己酉,封諸王習列吉為河平王,賜駝鈕金印。



 秋七月辛亥,召翰林直學士高鳴,順州知州劉瑜,中都郝謙、李天輔、韓彥文、李祐赴上都,以山東統軍副使王仲仁戍眉州。壬子,詔陜西統軍司兼領軍民錢穀。罷各路奧魯官,令管民官兼領。癸丑,立御史臺,以右丞相塔察兒為御史大夫,詔諭之曰:「臺官職在直言,朕或有未當,其極言無隱,毋憚他人,朕當爾主。」仍以詔諭天下。立高州北二驛。戊辰,罷西夏宣撫司。庚午,省諸路打捕鷹坊工匠洞冶總管府,令轉運司兼領之。丙子,立西夏惠民局。高麗國王王示直遣其臣崔東秀來言備兵一萬,造船千隻。詔遣都統領脫朵兒往閱之,就相視黑山日本道路,仍命耽羅別造船百艘以伺調用。詔四川行省賽典赤自利州還京兆,立東西二川統軍司,以劉整為都元帥,與都元帥阿術同議軍事。整至軍中,議築白河口、鹿門山,遣使以聞,許之。罷軍中諸司參議。



 八月乙酉,程思彬以投匿名書言斥乘輿,伏誅。己丑,亳州大水。庚子,敕京師瀕河立十倉。命忙古帶率兵六千征西番、建都。



 九月癸丑,中都路水,免今年田租。罷中都路和顧所。丁巳,阿術統兵圍樊城。敕長春宮修設金籙周天大醮七晝夜。建堯廟及後土太寧宮。庚申,賜安南國王陳光昞錦繡,及其諸臣有差。己丑,立河南屯田。命兵部侍郎黑的、禮部侍郎殷弘齎國書復使日本,仍詔高麗國遣人導送,期於必達,毋致如前稽阻。詔諭安南國陳光昞:「來奏稱占城、真臘二寇侵擾,已命卿調兵與不乾並力征討,今復命雲南王忽哥赤統兵南下,卿可遵前詔,遇有叛亂不庭為邊寇者,發兵一同進討,降服者善為撫綏。」車駕至自上都。益都路饑,以米三十一萬八千石賑之。復以史天澤為樞密副使。



 冬十月戊寅朔,日有食之。己卯,敕中書省、樞密院,凡有事與御史臺官同奏。立河南等路行中書省,以參知政事阿里行中書省事。庚辰,以御史中丞阿里為參知政事。壬午,詔恤沿邊諸軍,其橫科差賦,責奧魯官償之。庚寅,敕從臣禿忽思等錄《毛詩》、《孟子》、《論語》。乙未,享於太廟。中書省臣言:「前代朝廷必有起居注,故善政嘉謨不致遺失。」即以和禮霍孫、獨胡剌充翰林待制兼起居注。敕給黎、雅、嘉定新附民田。戊戌,宮城成。劉秉忠辭領中書省事,許之,為太保如故。



 十一月己酉,簽河南、山東邊城附籍諸色戶充軍。庚申,宋兵自襄陽來攻沿山諸寨,阿術分諸軍御之,斬獲甚眾,立功將士千三百四人。詔首立戰功生擒敵軍者,各賞銀五十兩,其餘賞賚有差。癸酉,御史臺臣言:「立臺數月,發擿甚多,追理侵欺糧粟近二十萬石,錢物稱是。」有詔褒諭。免南京、河南兩路來歲修築都城役夫。



 十二月戊寅,以中都、濟南、益都、淄萊、河間、東平、南京、順天、順德、真定、恩州、高唐、濟州、北京等處大水,免今年田租。敕二分、二至及聖誕節日,祭星於司天臺。詔諭四川行省沿邊屯戍軍士逃役者處死。復置乾州奉天縣,省好畤、永壽入焉。以鳳州隸興元路;德興府改奉聖州,隸宣德。是歲,京兆大旱。天下戶一百六十五萬二百八十六,斷死罪六十九人。賜諸王金、銀、幣、帛如歲例。



 六年春正月癸丑,高麗國王王禃遣使以誅權臣金俊來告,賜歷日、西錦。立四道按察司。戊午,阿術軍入宋境,至復州、德安府、荊山等處,俘萬人而還。庚申,以參知政事楊果為懷孟路總管。甲戌,益都、淄萊大水,恩州饑,命賑之。敕史天澤與樞密副使駙馬忽剌出董師襄陽。二月壬午,以立四道提刑按察司詔諭諸道。己丑,詔以新制蒙古字頒行天下。丙申,罷宣德府稅課所,以上都轉運司兼領。改河南、懷孟、順德三路稅課所為轉運司。丁酉,簽民兵二萬赴襄陽。賑欠州人匠貧乏者米五千九百九十九石。敕:「鞍、靴、箭鏃等物,自今不得以黃金為飾。」開元等路饑,減戶賦布二匹,秋稅減其半,水達達戶減青鼠二,其租稅被災者免征。免單丁貧乏軍士一千九百餘戶為民。癸卯,給河南行省鈔千錠犒軍。三月甲寅,詔益都路簽軍萬人,人給鈔二十五貫。戊午,賑曹州饑。築堡鹿門山。夏四月辛巳,制玉璽大小十紐。甲午,遣使祀岳瀆。大名等路饑,賑米十萬石。五月丙午,東平路饑,賑米四萬一千三百餘石。辛酉,詔禁戍邊軍士牧踐屯田禾稼。六月辛巳,以招討怯綿征建都敗績,又擅追唆火兒璽書、金符,處死。壬午,免益都新簽軍單丁者千六百二十一人為民。丁亥,河南、河北、山東諸郡蝗。癸巳,敕:「真定等路旱蝗,其代輸築城役夫戶賦悉免之。」丙申,高麗國王王禃遣其世子愖來朝,賜禃玉帶一,愖金五十兩,從官銀幣有差。壬寅,阿術率兵萬五千人厄宋萬山、射垛岡、鬼門關樵蘇之路。癸卯,詔董文炳等率兵二萬二千人南征。東昌路饑,賑米二萬七千五百九十石。秋七月丁巳,遣宋私商四十五人還其國。庚申,水軍千戶邢德立、張志等生擒宋荊鄂都統唐永堅,賞銀幣有差。辛酉,制太常寺祭服。壬戌,西京大雨雹。己巳,立諸路蒙古字學。癸酉,立國子學。詔遣官審理諸路冤滯,正犯死罪明白者,各正典刑,其雜犯死罪以下量斷遣之。又詔諭宋國官吏軍民,示以不欲用兵之意。復遣都統領脫朵兒、統領王國昌等往高麗點閱所備兵船,及相視耽羅等處道路。立西蜀四川監榷茶場使司。宋將夏貴率兵船三千至鹿門山,萬戶解汝楫、李庭率舟師敗之,俘殺二千餘人,獲戰艦五十艘。八月己卯,立金州招討司。丙申,以沙、肅州鈔法未行,降詔諭之。詔諸路勸課農桑。命中書省採農桑事,列為條目,仍令提刑按察司與州縣官相風土之所宜,講究可否,別頒行之。高麗國世子愖奏,其國臣僚擅廢國王王禃,立其弟安慶公淐。詔遣斡朵思不花、李諤等往其國詳問,條具以聞。九月癸丑,恩州進嘉禾,一莖三穗。戊午,敕民間貸錢取息,雖逾限止償一本息。己未,授高麗世子王愖特進上柱國、東安公。壬戌,豐州、雲內、東勝旱,免其租賦。戊辰,敕高麗世子愖率兵三千赴其國難,愖辭東安公,乃授特進上柱國。辛未,敕管軍萬戶宋仲義征高麗。以忽剌出、史天澤並平章政事,阿里中書右丞,行河南等路中書省事,賽典赤行陜西五路西蜀四川中書省事。車駕至自上都。斡朵思不花、李諤以高麗刑部尚書金方慶至,奉權國王淐表,訴國王禃遘疾,令弟淐權國事。冬十月己卯,定朝儀服色。壬午,升高唐、冠氏並為州。丁亥,廣平路旱,免租賦。詔遣兵部侍郎黑的、淄萊路總管府判官徐世雄,召高麗國王王禃、王弟淐及權臣林衍俱赴闕。命國王頭輦哥以兵壓其境,趙璧行中書省於東京,仍降詔諭高麗國軍民。庚子,太陰犯辰星。宋遣人饋鹽、糧入襄陽,我軍獲之。賜諸王奧魯赤駝鈕金鍍銀印。十一月癸卯,高麗都統領崔坦等,以林衍作亂,挈西京五十餘城來附。丁未,簽王綧、洪茶丘軍三千人往定高麗。高麗西京都統李延齡乞益兵,遣忙哥都率兵二千赴之。庚午,敕:「諸路鰥寡廢疾之人,月給米二斗。」安南國王陳光昞遣使來貢。濟南饑,以米十二萬八千九百石賑之。高麗國王王禃遣其尚書禮部侍郎樸烋從黑的入朝,表稱受詔已復位,尋當入覲。築新城於漢江西。十二月戊子,築東安渾河堤。己丑,作佛事於太廟七晝夜。高唐、固安二州饑,以米二萬六百石賑之。析彰德、懷孟、衛輝為三路,升林慮縣為林州,改楨州復為韓城縣,並省馮翊等州縣十所,以懿州、廣寧等府隸東京。是歲,天下戶一百六十八萬四千一百五十七。賜諸王金、銀、幣、帛如歲例。斷死罪四十二人。



\end{pinyinscope}