\article{列傳第十}

\begin{pinyinscope}

 ○布智兒



 布智兒,蒙古脫脫里臺氏。父紐兒傑,身長八尺,有勇力,善騎射,能造弓矢。嘗道逢太祖前驅騎士別那顏,邀與俱見太祖,視其所挾弓矢甚佳,問誰為造者,對曰:「臣自為之。」適有野鳧翔於前,射之,獲其二,並以二矢獻而退。別那顏隨之,至所居,布智兒出見,別那顏奇之,許以女妻之,父子遂俱事太祖。嘗從征討,賜紐兒傑拔都名。從征回回、斡羅思等國,每臨陣,布智兒奮身力戰。身中數矢,太祖親視之,令人拔其矢,血流滿體,悶僕幾絕。太祖命取一牛,剖其腹,納布智兒於牛腹,浸熱血中,移時遂更。紐兒傑卒,憲宗以布智兒為大都行天下諸路也可扎魯忽赤,印造寶鈔。賜七寶金帶燕衣十襲,又賜蔚州、定安為食邑。



 布智兒卒,有子四人。長好禮,事世祖,備宿衛。會丞相伯顏伐宋,奏好禮督水軍攻襄樊,從渡江入臨安,以功授昭毅大將軍、水軍翼萬戶府達魯花赤。別帖木兒,吏部尚書。補兒答思,雲南宣慰使。不蘭奚,襲父職,為水軍翼萬戶招討使,鎮守江陰,移通州。子完者不花,遼陽省理問。



 ○召烈臺抄兀兒



 召烈臺抄兀兒,初事太祖,時有哈剌赤、散只兀、朵魯班、塔塔兒、弘吉剌、亦乞列思等,居堅河之濱忽蘭也兒吉之地,謀奉扎木合為帝,將不利於太祖。抄兀兒知其謀,馳以告太祖,遂以兵收海剌兒阿帶亦兒渾之地,盡誅扎木合等。惟弘吉剌入降。太祖賜以答剌罕之名。



 其子那真,事世祖,為也可扎魯花赤。那真歿,子伴撒襲其職。伴撒卒,子火魯忽臺襲。致和元年八月,執倒剌沙起軍之使察罕不花,並其金字圓牌以獻。天歷元年十一月,帝賜金帶,仍復其職。嘗奏言:「有犯法者治之,當自貴人始;窮乏不給者救之,當自下始。如此則可得眾心矣。」其言良切於事弊云。



 闊闊不花



 闊闊不花者,按攤脫脫里氏,為人魁岸,有膂力,以善射知名。歲庚寅,太祖命太師木華黎伐金,分探馬赤為五部,各置將一人,闊闊不花為五部前鋒都元帥,所向莫能支。然不嗜殺,惟欲以威信懷附,故所至無殘破。略定濱、棣諸州,俘獲焦林諸處民四百餘,但籍其姓名,遣歸鄉里。徇益都,守將降,得其財物馬畜,悉以分賜士卒。歲壬辰,從太宗渡河,攻汴梁、歸德,分兵渡淮,攻壽州,守將無降意,射書城中諭之,城中人感泣,以彩輿奉金公主開門送款,闊闊不花下令軍中,輒入城虜掠者死,城中帖然。公主,義宗之姑也。歲丙申,太宗命五部將分鎮中原,闊闊不花鎮益都、濟南,按察兒鎮平陽、太原,孛羅鎮真定,肖乃臺鎮大名,怯烈臺鎮東平,括其民匠,得七十二萬戶,以三千戶賜五部將。闊闊不花得分戶六百,立官治其賦,得薦置長吏,歲從官給其所得五戶絲,以疾卒官。



 子黃頭,代領探馬赤為元帥,從丞相伯顏取宋,道死。子東哥馬襲其職,累遷右都威衛千戶,卒。



 ○拜延八都魯



 拜延八都魯,蒙古扎剌臺氏,幼事太祖,賜名八都魯。歲乙未,太宗命領扎剌軍一千六百人,與塔海甘卜同征關西,有功。癸丑,憲宗命與阿脫、總帥汪世顯創立利州城。甲寅,領兵紫金山,破宋軍鹿角寨,奪其軍餉器械。丁巳,從都元帥紐鄰城成都,及領兵圍雲頂山,下其城。帝親征,元帥紐鄰既進兵,涉馬湖江,留拜延八都魯鎮成都,降屬縣諸城,得其民,悉撫安之,賜黃金五十兩、衣九襲。諸王哈丹、朵歡、脫脫等征大理還,命拜延八都魯領兵迎之。道過新津寨,與宋潘都統遇,戰敗之,殺獲甚眾。中統二年,元帥紐鄰上其功,授蒙古奧魯官。



 子外貌臺,孫兀渾察。至元六年,拜延八都魯告老,兀渾察代其軍,從行省也速答兒徵諸國有功。十六年,從大軍征斡端,又有功,賞銀五十兩。二十一年,諸王術伯命兀渾察往乞失哈里之地為游擊軍。時敵軍二千餘,兀渾察以勇士五十人與戰,擒其將也班胡火者以獻。王壯之,以其功聞,賞銀六百兩、鈔四千五百貫,授蒙古軍萬戶,賜三珠虎符。三十年,以疾卒。次子襲授曲先塔林左副元帥,尋卒。弟塔海忽都襲,升鎮國上將軍都元帥,改授四川蒙古副都萬戶。至治二年,以疾退。子孛羅帖木兒襲。



 ○阿術魯



 阿術魯,蒙古氏。太祖時,命同飲班硃尼河之水,扈駕親征有功,命領兵收附遼東女直,還,賞金甲、珠衣、寶帶,他物稱是。復命總兵征西夏,與敵兵大戰於合剌合察兒之地。西夏勢蹙,其主懼,乞降,執之以獻,太宗殺之,賜以所籍貲產。繼領兵收附信安,下金二十餘城。其後告老,諸王塔察兒命其子不花代領其軍。



 ○紹古兒



 紹古兒,麥里吉臺氏。事太祖,命同飲班硃尼河之水,扈從親征。已而從破信安,略地河西,賜金虎符,授洺磁等路都達魯花赤。領軍出征,復從伐金,破河南。太宗命領濟南、大名、信安等處軍馬,復從國王答石出征。歲辛亥,卒。



 子拜都襲。拜都卒,子忽都虎襲,移睢州。從世祖渡江,攻鄂,還鎮恩州。中統三年,從征李璮有功,尋命修立邳州城,領兵鎮兩淮。十一年,從丞相伯顏渡江,有戰功。又從參政董文炳沿海出征,還,鎮嘉興,行安撫事。十二年,加昭勇大將軍,職如故。十四年,授嘉興路總管府達魯花赤,尋升鎮國上將軍、黃州路宣慰使,尋罷黃州宣慰司,復舊任。十六年,改授浙西道宣慰使,加招討使,仍鎮國上將軍。奉詔徵占城,以其國降表、貢物入見,帝嘉之,厚加賞賚。二十四年,從征交趾。明年還師,授邳州萬戶府萬戶。三十年,沒於軍。



 ○阿剌瓦而思



 阿剌瓦而思,回鶻八瓦耳氏,仕其國為千夫長。太祖征西域,駐蹕八瓦耳之地,阿剌瓦而思率其部曲來降。從帝親征,既破瀚海軍,又攻輪臺、高昌、于闐、尋斯干等,靡戰不克,沒於軍。



 子阿剌瓦丁,從世祖北征有功,至元二十九年卒,壽一百二歲。



 子贍思丁,有子五人:長烏馬兒,陳州達魯花赤;次不別,隆鎮衛都指揮使;次忻都,監察御史;次阿合馬,拱衛直司都指揮使;次阿散不別,驍勇善騎射,歷事成宗、武宗、仁宗,數被寵遇,計前後所賜楮幣餘四十萬緡,他物稱是,積官榮祿大夫,三珠虎符。



 子斡都蠻襲職。致和元年八月,自上都逃來,丞相燕帖木兒任為裨將,率壯士百人,圍滅裏帖木兒等於陀羅臺驛,擒之以獻,特賜衣一襲,及禿禿馬失甲、金束帶各一,白金一百兩,鈔二百錠。天歷元年九月,充行院同僉。十月,從擊忽剌臺、馬扎罕等軍於盧溝橋,敗之,追至紫荊關,多所俘獲,招降安童所將軍一千五百人,復以功受上賞。二年,進樞僉院。三年,以隆鎮衛都指揮使兼領拱衛司。



 ○抄兒



 抄兒,別速氏。世居汴梁陽武縣,從太祖收附諸國有功。又從征金,沒於陣。



 子抄海,從征河南、山東,復沒於陣。子別帖,將其父軍,從攻鄂州,以功賞銀帛衣甲等,繼從太子忽哥赤西征大理國,復沒於陣。子阿必察,至元五年授武略將軍、蒙古千戶,賜金符,從圍襄樊,復渡江,奪陽羅堡岸口,以功賞白金,進宣武將軍、蒙古軍總管,管領左右手兩萬戶軍。既下廣德,從平章阿里海牙征海外國,率死士鼓戰船進,奪岸口,擒勇士趙安等,以功賞銀帛。十六年,命管領蒙古侍衛軍,以疾卒於軍。



 ○也蒲甘卜



 也蒲甘卜,唐兀氏。歲辛巳,率眾歸太祖,隸蒙古軍籍。奉旨同所管河西人,從木華黎出征,以疾卒。



 子昂吉兒襲領其軍,徵諸國有功。至元六年,授金符千戶,從征蘄、黃、安慶等處。九年,易虎符,升信陽萬戶,從平章阿術南征,又有功,歷淮西道宣慰使、參知政事、都元帥、廬州蒙古漢軍萬戶府達魯花赤、行省左丞相、尚書左丞,積官龍虎衛上將軍。二十一年,攜其子昂阿禿入見。世祖命昂阿禿充速古兒赤。二十四年,隨駕征乃顏有功,奉旨代其父職。二十六年,授廬州蒙古漢軍萬戶府達魯花赤。大德六年,領兵討宋隆濟等,以功受上賞。還鎮廬州,以私財築室一百二十餘間,以居軍士之貧者,省臺以其事聞,特命升其秩,以金束帶賜之。泰定四年卒。昂阿禿之弟暗普,由速古兒赤授金符、唐兀禿魯花千戶,後改授海北海南道廉訪使。



 ○趙阿哥潘



 趙阿哥潘,土播思烏思臧掇族氏。始附宋,賜姓趙氏。世居臨洮。祖巴命,富甲諸羌。父阿哥昌,貌甚偉,有力兼人,金貞祐中,以軍功至熙河節度使。金亡,保蓮花山,以其眾來歸。皇子闊端之鎮西土也,承制以阿哥昌為疊州安撫使。時兵興,城無居人,至則招逃亡,立城壘,課耕桑以安輯之,年八十,卒於官。



 阿哥潘事親以孝聞,從伐蜀,與宋都統制曹友聞屢戰,勝負略相當,以破大安功最,授同知臨洮府事。斬朝天關,乘嘉陵江至閬州,獲蜀船三百艘。攻利州,生得其劉太尉,戰敗宋師於川。宋制置使劉雄飛進攻青居山,阿哥潘擊之,宵潰,四川大震。進逼成都,略嘉定,平峨眉太平寨,擒其將陳侍郎、田太尉,餘眾悉降。大小五十餘戰,皆先陷陣,皇子賜以金甲、銀器。歲壬子,世祖以皇弟南征大理,道出臨洮,見而奇之,命攝元帥,城益昌。時宋兵屯兩川,堡柵相望,矢石交擊,歷五年而城始完。憲宗出蜀,以阿哥潘為選鋒,攻西安,下之,賜金符,授臨洮府元帥。帝駐釣魚山,合州守將王堅夜來斫營,阿哥潘率壯士逆戰,手殺數十百人,堅遂引去。明日陛見,帝喜曰:「有臣如此,朕復何憂!」賜黃金五十兩,名曰拔都。中統建元,詔還鎮臨洮。歲饑,發私廩以賑貧乏。給民農種粟二千餘石、蕪菁子百石,人賴不饑。郡當孔道,傳置旁午,有司敝於供給。阿哥潘以私馬百匹充驛騎,羊千口代民輸。帝聞而嘉之,詔京兆行省酬其直。阿哥潘曰:「我豈以私惠而邀公賞耶!」卒不受。以軍事赴青居山,道為宋兵所邀,遂死於敵。



 阿哥潘好畜良馬,常千蹄,歲擇其上驥五駟貢於朝,子孫遵之不替。先是,勛臣子孫為祖父請謚者,帝每靳之,至是敕大臣以美謚謚之,謚曰桓勇。



 子重喜,始給侍皇子闊端為親衛。癸丑,從世祖征哈剌章,數有功。中統元年,渾都海反,從總帥汪良臣引兵至拔沙河納火石地逆戰,以功授征行元帥。四年,從討忽都、達吉、散竹臺等,克之,制必帖木兒王承制使襲父職為元帥。入覲,賜金虎符,為臨洮府達魯花赤。時解軍職而轉民官者,例納所佩符。有旨:「趙氏世世勤勞,其金符勿拘常例,使終佩之。」重喜在郡,勸農興學,省刑敦教,以善治聞。請致仕,不許,詔其長子官卓斯結襲為達魯花赤。升重喜鞏昌二十四處宣慰使。卒,謚桓襄。



 官卓斯結性靖退,辭官閑處二十餘年。仁宗聞其名,召不起。子德壽,雲南左丞。



 ○純只海



 純只海,散術臺氏。弱冠宿衛太祖帳下,從征西域諸國有功。歲癸巳,太宗命佩金虎符,充益都行省軍民達魯花赤,從大帥太出破徐州,擒金帥國用安。丁酉,以益都為皇太子分土,遷京兆行省都達魯花赤。至懷,值大疫,士卒困憊,有旨以本部兵就鎮懷孟。未幾,代察罕總軍河南,尋復懷孟。己亥,同僚王榮潛畜異志,欲殺純只海,伏甲縶之,斷其兩足跟,以帛緘純只海口,置佛祠中。純只海妻喜禮伯倫聞之,率其眾攻榮家,奪出之。純只海裹瘡從二子馳旁郡,請兵討榮,殺之。朝廷遣使以榮妻孥貲產賜純只海家,且盡驅懷民萬餘口郭外,將戮之。純只海力爭曰:「為惡者止榮一人耳,其民何罪。若果盡誅,徒守空城何為?茍朝廷罪使者以不殺,吾請以身當之。」使者還奏,帝是其言,民賴不死。純只海給榮妻孥券,放為民,遂以其宅為官廨,秋毫無所取。郡人德之。既入覲,太宗以純只海先朝舊臣,功績昭著,賜第一區於和林,尋以疾卒。敕葬山陵之側。



 皇慶初,贈推忠宣力功臣、金紫光祿大夫、上柱國、溫國公,謚忠襄。仍敕詞臣劉敏中制文樹碑於懷,以旌其功云。子昂阿剌嗣。



 ○苫徹拔都兒



 苫徹拔都兒,欽察人。初事太宗,掌牧馬。從攻鳳翔,戰潼關,皆有功。後從大將速不臺攻汴京,金人列木柵於河南,苫徹拔都兒率死士往拔之,賜良馬十匹。師還,金將高都尉率眾邀於中路,苫徹拔都兒迎擊,斬其首以歸,賜白金五十兩、幣四匹。從攻蔡州,前鋒答答兒與金將戰,金將捽其須,苫徹拔都兒進斫金將,乃得脫。蔡州破,金守將佩虎符立城上,苫徹拔都兒以鐵椎擊殺之,取虎符以獻。帝嘉其能,命從皇子攻棗陽。繼從宗王口溫不花攻光州,一日五戰,光州下。賜黃金五十兩、白金酒器一事、馬三十匹。百戶愛不怯赤自以臨陣不勇,乞苫徹拔都兒自代,遂升百戶。從攻滁州,與宋兵大戰,至暮,宋兵敗走西山,苫徹拔都兒與千戶忽孫追殺之。



 歲己未,世祖伐宋,募能先絕江者,苫徹拔都兒首應命,率眾逼南岸。詔苫徹拔都兒與脫歡領兵百人。同宋使諭鄂州使降。抵城下,鄂守將殺使者以軍來襲,苫徹拔都兒與之遇,奮擊大破之。復賜黃金五十兩。中統三年,授蔡州蒙古漢軍萬戶。冬,宋人犯西平,苫徹拔都兒逐北逾淮,獲其生口甚眾。至元二年秋,由安慶入廬州,聞宋兵至,亟設伏於竹林,擊殺之。四年秋九月,元帥阿術軍襄陽安陽灘。宋兵據渡口,苫徹拔都兒擊破其眾。五年,從阿術圍襄陽,擊奪宋將夏貴米舟。阿術入漢江,以其有戰功,俾與扎剌兒引軍南略,獲八十人。十年八月,略地淮東。十一年,遣招鄂州。十二年,遣招滁州,誅王安撫。改武略將軍、管軍千戶。五月,伏兵大江北岸,擊宋軍,敗走之。十三年,復略地淮東,獲其總管二人以獻。遷滁州總管府達魯花赤。宋都統姜才率軍取糧高郵,苫徹拔都兒從史萬戶奪其馬及糧橐二萬,淮東平,入朝。十四年,從討叛人只裏瓦歹於懷剌合都,改宣武將軍、滁州路總管府達魯花赤。



 十七年,率其子脫歡、孫麻兀入見。奏曰:「臣老矣,幸主上憐之。」帝命以脫歡為宣武將軍、管軍總管,佩金符;麻兀為滁州路總管府達魯花赤。其後脫歡以征倭功授明威將軍、滁州萬戶府達魯花赤,升昭勇大將軍、征行軍萬戶府達魯花赤,佩三珠虎符。又以征爪哇功升昭毅大將軍,鎮守無為滁州萬戶府達魯花赤。次子鎖住,襲其職。



 ○怯怯里



 怯怯里,斡耳那氏。太宗七年南伐,以千戶從闊端攻安豐、壽州。又從諸王塔察兒率蒙古軍二千攻荊山,破之,賜馬二匹。與萬戶納鷿以兵守沂、郯,略漣海,又從元帥懷都攻襄陽。卒。



 子相兀速襲父職,率本部兵從丞相阿術攻襄樊,又從塔出築正陽堡。瀘軍乘艦來窺壁壘,相兀速率征騎逆之,夾淮水而軍,射死者甚眾。至元十一年,賜金符,授武略將軍。明年,從御史大夫博羅罕平漣海。秋九月,從丞相伯顏渡淮,率兵一千騎攻淮安南門,破之。又從元帥博羅罕築灣頭堡。萬戶納兒鷿臥疾,令相兀速權領蒙古、女直、漢人三萬戶。夏五月,宋揚州都統姜才引兵來侵,相兀速率本部兵逆戰有功。又從丞相阿術襲制置使李庭芝及姜才於泰州,皆殺之。十四年,加宣武將軍、管軍總管。十八年,為蒙古侍衛親軍總管。二十三年,改千戶。三十年,升蒙古侍衛親軍副指揮使司事,易金虎符,加顯武將軍。



 子捏古鷿,元貞元年,為蒙古侍衛親軍百戶。大德六年,襲父職,佩金虎符,授宣武將軍。延祐四年,升左翊蒙古侍衛親軍都指揮使,仍所佩符,進懷遠大將軍。



 ○塔不已兒



 塔不已兒,束呂糺氏。太宗時以招討使將兵出征,破信安、河南,以功授金虎符、征行萬戶。歲甲寅,以疾卒。



 子脫察剌襲職。歲己未,率兵渡江,破十字寨。命其子重喜從行。重喜率先引弓,射中敵兵,又多殺獲。既而與敵兵戰於洋隘口,奪戰艦一,流矢中左足,勇氣愈倍。時世祖駐蹕洋隘口北,親勞之曰:「汝年幼能宣力如是,深可嘉尚。然繼今尤當勉之。」及脫察剌卒,以重喜襲職。中統三年,從征李璮有功。四年,以兵鎮莒州。至元二年,奉旨初築十字路城,以備守御。重喜率兵南巡,為游擊軍。四年,從抄不花出征,至泗州北古城。時蔡千戶為敵兵所圍,重喜奮戰,救而出之。五年,入覲。帝嘉其功,賜白金、納失失段及金鞍弓矢等。十年,修正陽城。明年,宋兵圍正陽,從戰敗之。十二年,從下漣海諸城。俄奉旨率五千人從出征,道過衡陽店,與宋將李提轄等戰,大敗之,殺掠幾盡,遂駐兵瓜洲。十三年夏六月,宋都統姜才領諸軍來圍城堡,敗之。秋七月,從兵襲擊李庭芝等於泰州。十四年,進昭勇大將軍、婺州路總管府達魯花赤,佩已降虎符。未幾卒。



 子慶孫襲職,初授宣武將軍、管軍總管,鎮守安樂州。十六年,移戍鎮江府。十八年,還鎮通州。二十年,進明威將軍。二十二年,移鎮十字路。二十四年,領諸翼軍鎮太湖,教習水戰。二十九年,從征爪哇,升昭勇大將軍、征行上萬戶。將行,有旨留之。皇慶二年卒。子孛蘭奚襲。



 ○直脫兒



 直脫兒,蒙古氏,父阿察兒,事太祖,為博兒赤。直脫兒從太宗徵欽察、康里、回回等部有功。四年,收河南、關西諸路,得民戶四萬餘,以屬莊聖皇太后,為脂粉絲線顏色戶。八年,建織染七局於涿州。明年,改涿州路,以直脫兒為達魯花赤。卒。



 子哈蘭術襲,佩虎符。李璮叛,世祖命領諸萬戶為監戰達魯花赤以討之。有功,授解萬戶翼監戰領軍。遷益都路蒙古萬戶,監戰密州,沒於軍。



 從子忽剌出襲職,授昭勇大將軍。至元十一年,攻宋六安軍,有功。行中書省命領諸軍戰艦沖宋軍,宋軍敗,有旨褒賞。九月,師次安慶。忽剌出及參政董文炳領山東諸軍順流東下,至丁家洲,遇宋臣夏貴、孫虎臣等,戰江中,宋軍大敗,擒其將校三十七人、軍五千餘、船四十艘。十二年三月,與宋軍戰硃金沙,復有功。七月,復與宋軍戰焦山江中。時丞相阿術等督戰,忽剌出與董文炳身冒矢石,沿流鏖戰八十餘里。忽剌出身被數傷,裹創力戰,遂勝之。九月,宋臣張殿帥攻奪呂城倉、丹陽縣。忽剌出與萬戶懷都往救,生擒之。十月,下常州,從丞相伯顏略蘇、湖、秀州,至長橋,遇宋軍,又敗之。十三年正月,師至杭州,丞相伯顏命忽剌出守浙江亭及宋北門。五月,揚州軍劫揚子橋堡,敗之。六月,敗真州軍。七月,追李庭芝至通海口,降揚州及高郵、寶應、真州、滁州等城,江南平。加昭毅大將軍,職如故。尋遷湖州路達魯花赤。十四年,進鎮國上將軍、淮東宣慰使。已而屯守上都。十五年,授嘉議大夫、行御史臺中丞。十九年,進資善大夫、福建行省左丞。黃華叛,平之。二十年,授江淮行省左丞。二十三年,遷右丞。三月,進榮祿大夫、江浙行省平章政事。六月,卒。



 ○月裡麻思



 月裡麻思,乃馬氏。歲丁酉,太宗命與斷事官忽都那顏同署。歲戊戌,又同阿術魯拔都兒充達魯花赤,破南宿州。歲辛丑,使宋議和。從行者七十餘人,月裡麻思語之曰:「吾與汝等奉命南下,楚人多詐,倘遇害,當死焉,毋辱君命。」已而馳抵淮上,宋將以兵脅之,曰:「爾命在我,生死頃刻間耳。若能降,官爵可立致。不然,必不汝貸。」月裡麻思曰:「吾持節南來,以通國好,反誘我以不義,有死而已。」言辭慷慨,不少屈。宋將知其不可逼,乃囚之長沙飛虎寨三十六年而死。世祖深悼之,詔復其家,以子忽都哈思為答剌罕,日給糧食其家人。忽都哈思自陳於帝曰:「臣願為國效死,為父雪恥。」帝嘉納之,授以上均州監戰萬戶。十八年,以招討使將兵征日本,死於敵。



 ○捏古剌



 捏古剌,在憲宗朝,與也裏牙阿速三十人來歸。後從征釣魚山,討李璮,皆有功。



 子阿塔赤,世祖時圍襄陽,下江南,敗失列及,徵乃顏,皆以功受賞。後事成宗、武宗,為札撒兀孫。仁宗時,歷官至左阿速衛千戶。卒。



 子教化,初為速古兒赤,繼襲父職。必里阿禿叛,奉旨往平之,凱還,賜衣一襲。天歷元年八月,從丞相燕帖木兒戰居庸北,有功。九月,進拱衛直都指揮使。尋遷章佩卿。



 子者燕不花,初事仁宗,為速古兒赤。英宗時為進酒寶兒赤。天歷元年,迎文宗於河南,賜白金、彩段,命為溫都赤。九月,往居庸關料敵,道逢二軍,謂探馬赤諸軍曰:「今北兵且至,其避之。」者燕不花恐搖眾心,即拔所佩刀斬之。授兵部郎中。招集阿速軍四百餘人。十月,進兵部尚書,授雙珠虎符,領軍六百人迎敵通州。會丞相燕帖木兒至檀子山,與禿滿迭兒戰,敗之。遷大司農丞。



 ○阿兒思蘭



 阿兒思蘭,阿速氏。初,憲宗以兵圍阿兒思蘭之城,阿兒思蘭偕其子阿散真迎謁軍門。帝賜手詔,命專領阿速人,且留其軍之半,餘悉還之,俾鎮其境內。以阿散真置左右。道遇闍兒哥叛軍,阿散真力戰死之。帝遣使裹尸還葬之。阿兒思蘭言於帝曰:「臣長子死,不能為國效力,今以次子捏古來獻之陛下,願用之。」捏古來至,帝命從兀良哈臺征哈剌章,有功,兀良哈臺賞以白金名馬。從伐宋,中流矢而死。



 子忽兒都答,充管軍百戶。世祖命從不羅那顏使哈兒馬某之地,以疾卒。



 子忽都帖木兒,武宗潛邸時從征海都,以功賞白金。至大元年,授宣武將軍、左衛阿速親軍副都指揮使。四年,卒。



 ○哈八兒禿



 哈八兒禿,薛亦氏。憲宗時,從攻釣魚山有功。還,又從親王塔察兒北征,充千戶所都鎮撫。從千戶脫倫伐宋,沒於陣。



 子察罕,從塔察兒攻樊城西門,領揚州等處游擊軍與宋兵戰,有功。至元十一年,從忽都帖木兒攻江陵東南城堡,又從阿剌罕敗宋兵於陽邏堡之南。阿剌罕選為本萬戶府副鎮撫。十二年,分隸脫脫總管,出廣德游擊軍,與宋兵戰,敗之,賜以白金酒器。又從攻獨松、千秋、撥出等關及諸山寨,其降民悉綏撫之,賜白金一百兩。十三年,中書省檄為瑞安縣達魯花赤。始至,招集逃移民十萬餘戶。十四年,升忠顯校尉、管軍總把,並領新附軍五百人,從宣慰唐兀臺戰於司空山,有功,命以其職兼都鎮撫。俄選充侍衛親軍。十六年,授銀符、忠武校尉、管軍總把。二十四年,賜金符,授承信校尉、蒙古衛軍屯田千戶。二十五年,進武義將軍、本所達魯花赤。二十七年,升左翼屯田萬戶府副萬戶。大德五年卒。子太納襲。



 ○艾貌



 艾貌拔都,康里氏。初從雪不臺那演徵欽察,攻河西城,收西關,破河南;繼從定宗略地阿奴,皆有功。又從四太子南伐,命充怯憐口阿答赤孛可孫。又從兵渡江攻鄂,以疾卒於軍。



 子也速臺兒,從討阿藍答、渾都海,徵李璮,伐宋,累功授管軍總把。至元十四年,從攻福建興化,招古田等處民五千餘戶,以功升武略將軍、千戶,賜金符。又招手號新軍二千五百餘人,升宣武將軍、總管,賜虎符。有旨征日本,也速臺兒願效力,賜以弓矢,進懷遠大將軍、萬戶。二十年,授泰州萬戶府達魯花赤。二十三年,遷昭勇大將軍、欽察親軍都指揮使。二十四年,從征乃顏有功。明年卒。後贈金吾衛上將軍,追封成武郡公,謚顯敏。



\end{pinyinscope}