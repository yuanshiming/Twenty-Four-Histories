\article{志第二十七 祭祀五}

\begin{pinyinscope}

 ○太社太稷



 至元七年十二月,有詔歲祀太社太稷。三十年正月,始用御史中丞崔彧言,於和義門內少南,得地四十畝,為壝垣,近南為二壇,壇高五丈,方廣如之。社東稷西,相去約五丈。社壇土用青赤白黑四色,依方位築之,中間實以常土,上以黃土覆之。築必堅實,依方面以五色泥飾之。四面當中,各設一陛道。其廣一丈,亦各依方色。稷壇一如社壇之制,惟土不用五色,其上四周純用一色黃土。壇皆北向,立北墉於社壇之北,以磚為之,飾以黃泥;瘞坎二於稷壇之北,少西,深足容物。



 二壇周圍壝垣,以磚為之,高五丈,廣三十丈,四隅連飾。內壝垣欞星門四所,外垣欞星門二所,每所門二,列戟二十有四。外壝內北垣下屋七間,南望二壇,以備風雨,曰望祀堂。堂東屋五間,連廈三間,曰齊班。之南,西向屋八間,曰獻官幕。又南,西向屋三間,曰院官齋所。又其南,屋十間,自北而南,曰祠祭局,曰儀鸞庫,曰法物庫,曰都監庫,曰雅樂庫。又其南,北向屋三間,曰百官廚。外垣南門西壝垣西南,北向屋三間,曰大樂署。其西,東向屋三間,曰樂工房。又其北,北向屋一間,曰饌幕殿。又北,南向屋三間,曰饌幕。又北稍東,南向門一間。院內南,南向屋三間,曰神廚。東向屋三間,曰酒庫。近北少卻,東向屋三間,曰犧牲房。井有亭。望祀堂後自西而東,南向屋九間,曰執事齋郎房。自北折而南,西向屋九間,曰監祭執事房。此壇壝次舍之所也。



 社主用白石,長五尺,廣二尺,剡其上如鐘,於社壇近南,北向,埋其半於土中。稷不用主。後土氏配社,後稷氏配稷。神位版二,用慄,素質黑書。社樹以松,於社稷二壇之南各一株。此作主樹木之法也。



 祝版四,以楸木為之,各長二尺四寸,闊一尺二寸,厚一分。文曰:「維年月日,嗣天子敬遣某官某,敢昭告於太社之神。」配位曰后土之神。稷曰太稷之神,配位曰后稷之神。玉幣,社稷皆黝圭一,繅藉瘞玉一,以黝石代之,玄幣一。配位皆玄幣一,各長一丈八尺。此祝文玉幣之式也。



 牛一,其色黝,其角握,有副。羊四,野豕四。籩之實皆十,無糗餌、粉糍。豆之實亦十,無厓食、糝食。簠簋之實皆四,鉶之實和羹五,齊皆以尚醖代之。香用沉龍涎。神席一,緣以黑綾,黑綾褥方七尺四寸。太尊、著尊、犧尊、山罍各二,有坫,加勺冪。象尊、壺尊、山罍各二,有坫冪,設而不酌。籩豆各十有一,其一設於饌幕。鉶三,簠三,簋三,其一設於饌幕。俎八,其二設於饌幕。盤一,毛血豆一,爵一,有坫。沙池一,玉幣篚一,木柶一,勺一,香鼎一,香盒一,香案一,祝案一,皆有衣。紅髹器一,以盛馬湩。盥洗位二,罍二,洗二。白羅巾四,實以篚。硃漆盤五。已上,社稷皆同。配位有象尊,無太尊。設而不酌者,無象尊。餘皆與正位同。此牲齊祭器之等也。



 饌幕、省饌殿、香殿,黃羅幕三,黃羅額四,黃絹帷一百九十五幅,獻攝板位三十有五,紫綾拜褥百,蒲、葦席各二百,木燈籠四十,絳羅燈衣百一十,紅挑燈十,剪燭刀二,鐵凡盆三十有架,黃燭二百,雜用燭二百,麻凡三百,松明、清油各百斤。此饌幕板位燭燎之用也。



 初獻官一,亞獻官一,終獻官一,攝司徒一,助奠官二,太常卿一,光祿卿一,廩犧令一,太官令一,巾篚官四,祝史四,監祭御史二,監禮博士二,司天監二,良醖令一,奉爵官一,司尊罍二,盥洗官二,爵洗官二,大社令一,大社丞一,大樂令一,大樂丞一,協律郎二,奉禮郎二,讀祝官一,舉祝官二,奉幣官四,剪燭官二,太祝七,齋郎四十有八,贊者一,禮直官三,與祭官無定員。此獻攝執事之人也。



 凡祭之日,以春秋二仲月上戊。延祐六年改用中戊。其儀注之節有六:



 一曰迎香。前一日,有司告諭坊市,灑掃經行衢路,設香案。至日質明,有司具香酒樓輿,三獻官以下及諸執事官各具公服,五品以下官、齋郎等皆借紫,詣崇天門。三獻官及太常禮儀院官入,奉祝及御香、尚尊酒、馬湩自內出。監祭御史、監禮博士、奉禮郎、太祝分左右兩班前導。控鶴五人,一人執傘,四人執儀仗,由大明門正門出。教坊大樂作。至崇天門外,奉香酒、馬湩者各安置於輿,導引如儀。至紅門外,百官乘馬分班行於儀仗之外,清道官行於儀衛之先,兵馬司巡兵夾道次之,金鼓又次之,京尹儀從左右成列又次之,教坊大樂一隊次之。控鶴弩手各服其服,執儀仗左右成列次之。拱衛使行其中,儀鳳司細樂又次之。太常卿與博士御史導於輿前,獻官、司徒、助奠官從於輿後。若駕幸上都,三獻官以下及諸執事官則詣健德門外,皆具公服於香輿前北向立,異位重行。俟奉香酒官驛至,太常官受而奉之,各置於輿。禮直官贊「班齊」,「鞠躬」,「再拜興」,「平立」。班首稍前搢笏跪,眾官皆跪,三上香,出笏就拜興,平立退復位,北向立,鞠躬,再拜興,平立。眾官上馬,分班前導如儀。至社稷壇北神門外皆下馬,分左右入自北門,序立如儀。太常卿、博士、御史前導,獻官、司徒、助奠等官後從。至望祀堂下,三獻奉香、酒、馬湩升階,置於堂中黃羅幕下。禮直官引三獻官以次而出,各詣齋次,釋服。



 二曰齋戒。前期三日質明,有司設三獻官以下行事執事官位於中書省。太尉南向,監祭御史位二於其西,東向,監禮博士位二於其東,西向,俱北上。司徒、亞獻、終獻位於其南,北向。次助奠,稍卻。次太常卿、光祿卿、大官令、司尊彞、良醖令、太社令、廩犧令、光祿丞、大樂令、太社丞。次讀祝官、奉爵官、太祝、祝史、奉禮郎、協律郎、司天生、諸執事齋郎。每等異位重行,俱北向,西上。贊者引行事執事官各就位,立定。禮直官引太尉、初獻就位,讀誓曰:「某年某月某日上戊日,祭於太社太稷,各揚其職,其或不敬,國有常刑。」散齋二日,宿於正寢,致齋一日於祠所。散齋日治事如故,不吊喪問疾,不作樂,不判署刑殺文字,不決罰罪人,不與穢惡事。致齋日,惟祭事得行,其餘悉禁。凡與祭之官已齋而闕者,通攝行事。七品以下官先退,餘官對拜。守壝門兵衛與大樂工人,俱清齋一日。行禮官,前期習儀於祠所。



 三曰陳設。前期三日,所司設三獻以下行事執事官次於齋房之內,又設饌幕四於西神門之外,稍南,西向,北上。今有饌幕殿在西壝門外,近北,南向。陳設如儀。前祭二日,所司設兵衛,各以其方色器服守衛壝門,每門二人,每隅一人。大樂令帥其屬設登歌之樂於兩壇上,稍北,南向。磬虡在東,鐘虡在西,柷一在鐘虡南稍東,敔一在磬虡南稍西。搏拊二,一在柷南,一在敔南,東西相向。歌工次之,餘工位在縣後。其匏竹者位於壇下,重行南向,相對為首。太社令帥其屬掃除壇之上下,為瘞坎二於壬地,方深足以容物,南出陛。前祭一日,司天監、太社令帥其屬升,設太社、太稷神座各於壇上,近南,北向。設后土神座於太社神座之左,後稷神座於太稷神座之左,俱東向。席皆以莞,裀褥如幣之色,設神位版各於座首。奉禮郎設三獻官位於西神門之內道南,亞獻、終獻位稍卻。司徒位道北,太常卿、光祿卿次之,稍卻。司天監、光祿丞又次之。太社令、大官令、良醖令、廩犧令、太社丞、讀祝官、奉爵官、太祝以次位於其北,諸執事者及祝史、齋郎位於其後。每等異位重行,俱東向,南上。又設監祭御史位二,監禮博士位二,於太社壇子陛之東北,俱東向,南上。設奉禮郎位於稷壇之西北隅,贊者位於東北隅,俱東向。協律郎位二,於各壇上樂虡東北,俱南向。太樂令位於兩壇樂虡之間南向,司尊彞位於酌尊所,俱南向。設望瘞位於坎之南,北向。又設牲榜於西神門外,東向。諸太祝位於牲西,祝史次之,東向。太常卿、光祿卿、大官令位在南,北向,東上。監祭、監禮位於太常卿之東稍卻,俱北向,東上。廩犧令位於牲東北,南向。又設禮饌於牲東,設省饌於禮饌之北,今有省饌殿設位於其北,東西相向,南上。太常卿、光祿卿、大官令位於西,東向,監祭、監禮位於東,西向,俱南上,禮部設版案各於神位之側,司尊彞、奉禮郎帥執事者設玉幣篚於酌尊所。次設籩豆之位,每位各籩十、豆十、簠二、簋二、鉶三、俎五、盤一。又各設籩一、豆一、簠一、簋一、俎三於饌幕內。毛血別置一豆。設尊罍之位,社稷正位各太尊二、著尊二、犧尊二、山罍二,於壇上酉陛之西北隅,南向,東上。設配位各著尊二、犧尊二、象尊二、山罍二,在正位酒尊之西,俱南向,東上。又設正位各象尊二、壺尊二、山罍二,於壇下子陛之東,南向,東上。配位各壺尊二、山罍二,在卯陛之南,西向,南上。又設洗位二,於各壇子陛之西北,南向。篚在洗東北肆,執罍篚者各位於其後。祭日醜前五刻,司天監、太社令各服其服,帥其屬升,設正配位神位版於壇上。又陳玉幣,正位禮神之玉一,兩圭有邸,置於匣。正配位幣皆以玄,各長一丈八尺,陳於篚。太祝取瘞玉加於幣,實於篚,瘞玉以玉石為之,及禮神之玉各置於神座前。光祿卿帥其屬,入實籩豆簠簋。每位籩三行,以右為上。第一行,乾尞在前,乾棗、形鹽、魚鱐次之。第二行,鹿脯在前,榛實、乾桃次之。第三行,菱在前,芡、慄次之。豆三行,以左為上。第一行,芹菹在前,筍菹、葵菹、菁菹次之。第二行,韭菹在前,魚醢、兔醢次之。第三行,豚拍在前,鹿MZ、醓醢次之。簠實以稻粱,簋實以黍稷,鉶實以羹。良醖令帥其屬,入實尊罍。正位太尊為上,實以泛齊,著尊實以醴齊,犧尊實以盎齊,象尊實以醍齊,壺尊實以沈齊,山罍實以三酒。配位著尊為上,實以泛齊,犧尊實以醴齊,象尊實以盎齊,壺尊實以醍齊,山罍實以三酒。凡齊之上尊實以明水,酒之上尊實以玄酒,酒齊皆以尚醖代之。太常卿設燭於神座前。



 四曰省牲器。前期一日午後八刻,諸衛之屬禁止行人。未後二刻,太社令帥其屬,掃除壇之上下。司尊彞、奉禮郎帥執事者,以祭器入設於位。司天監、太社令升,設神位版及禮神之玉幣如儀。俟告潔畢,權徹,祭日重設。未後二刻,廩犧令與諸太祝、祝史以牲就位,禮直官、贊者分引太常卿、監祭、監禮、大官令於西神門外省牲位,立定。禮直官引太常卿,贊者引監祭、監禮,入自西神門,詣太社壇,自西陛升,視滌濯於上,執事者皆舉冪曰「潔」。次詣太稷壇,如太社之儀訖,降復位。禮直官稍前曰「告潔畢,請省牲」,引太常卿稍前省牲訖,退復位。次引廩犧令出班巡牲一匝,東向折身曰「充」,復位。諸太祝俱巡牲一匝,上一員出班東向折身曰「腯」,復位。禮直官稍前曰「省牲畢,請就省饌位」,引太常卿以下各就位,立定。省饌畢,還齋所。廩犧令與太祝、祝史以次牽牲詣廚,授大官令。次引光祿卿以下詣廚省鼎鑊,視滌溉畢,乃還齋所。晡後一刻,大官令帥宰人以鸞刀割牲,祝史以豆取血各置於饌幕。祝史又取瘞血貯於盤,遂烹牲。



 五曰奠玉幣。祭日醜前五刻,三獻官以下行事執事官,各服其服。有司設神位版,陳玉幣,實籩豆簠簋尊罍。俟監祭、監禮按視壇之上下,及徹去蓋冪。未明二刻,大樂令帥工人入,奉禮郎、贊者入就位,禮直官、贊者入就位。禮直官、贊者分引監祭、監禮、諸太祝、祝史、齋郎及諸執事官,自西神門南偏門入,當太社壇北墉下,重行南向立,以東為上。奉禮曰「再拜」,贊者承傳,監祭、監禮以下皆再拜。次贊者分引各就壇上下位,祝史奉盤血,太祝奉玉幣,由西階升壇,各於尊所立。次引監祭、監禮按視壇之上下,糾察不如儀者,退復位。質明,禮直官、贊者各引三獻以下行禮執事官入就位,皆由西神門南偏門以入。禮直官進初獻之左,曰「有司謹具,請行事」,退復位。協律郎跪,俯伏舉麾興,工鼓柷,樂作八成,偃麾,戛敔樂止。禮直官引太常卿瘞血於坎訖,復位,祝史以盤還饌幕,以俟奉毛血豆。奉禮曰「眾官再拜」,在位者皆再拜。又贊諸執事者各就位,禮直官、贊者分引執事官各就壇上下位。諸太祝各取玉幣於篚,立於尊所。禮直官引初獻詣太社壇盥洗位,樂作,至位南向立,樂止。搢笏,盥手,帨手,執笏詣壇,樂作,升自北陛,至壇上,樂止。詣太社神座前,南向立,樂作,搢笏跪。太祝加玉於幣,東向跪以授初獻,初獻受玉幣奠訖,執笏俯伏興,少退,再拜訖,樂止。禮直官引初獻降自北陛,詣太稷壇盥洗位,樂作,至位樂止。盥洗訖,升壇奠玉幣,並如太社後土之儀。奠畢,降自北陛,樂作,復位樂止。初獻奠玉幣將畢,祝史各奉毛血豆立於西神門外,俟奠玉幣畢,樂止。祝史奉正位毛血入自中門,配位毛血入自偏門,至壇下,正位者升自北陛,配位者升自西陛,諸太祝迎取於壇上,各進奠於神位前,太祝、祝史俱退立於尊所。



 六曰進熟。初獻既奠玉幣,有司先陳鼎入於神廚,各在於鑊右。大官令出,帥進饌者詣廚,以匕升羊豕於鑊,各實於一鼎,冪之。祝史以扃對舉鼎,有司執匕以從,各陳於饌幕內。俟光祿卿出,帥其屬實籩豆簠簋訖,乃去鼎之扃冪,匕加於鼎。大官令以匕升羊豕,各載於俎,俟初獻還位,樂止。禮直官引司徒出詣饌所,帥進饌者各奉正配位之饌,大官令引以次自西神門入。正位之饌入自中門,配位之饌入自偏門。饌初入門,樂作,饌至陛,樂止。祝史俱進,徹毛血豆,降自西陛以出。正位之饌升自北陛,配位之饌升自西陛,諸太祝迎取於壇上,各跪奠於神座前訖,俯伏興。禮直官引司徒、大官令及進饌者,自西陛各復位。諸太祝還尊所,贊者曰「太祝立茅苴於沙池」。禮直官引初獻官詣太社壇盥洗位,樂作,至位南向立,樂止。搢笏,盥手,帨手,執笏詣爵洗位,至位南向立,搢笏,洗爵、拭爵,以爵授執事者,執笏詣壇,樂作,升自北陛,至壇上,樂止。詣太社酌尊所,東向立,執事者以爵授初獻,初獻搢笏執爵,司尊者舉冪,良醖令跪酌太尊之泛齊,樂作。初獻以爵授執事者,執笏詣太社神座前,南向立,搢笏跪。執事者以爵授初獻,初獻執爵三祭酒,奠爵,執笏俯伏興,少退立,樂止。舉祝官跪,對舉祝版。讀祝官西向跪,讀祝文。讀訖,俯伏興,舉祝官奠祝版於案,興。初獻再拜訖,樂止。次詣后土氏酌尊所,東向立。執事者以爵授初獻,初獻搢笏執爵,司尊彞舉冪,良醖令跪酌著尊之泛齊,樂作。初獻以爵授執事者,執笏詣后土神座前,西向立,搢笏跪。執事者以爵授初獻,初獻執爵三祭酒,奠爵訖,執笏俯伏興,少退立,樂止。舉祝官跪,對舉祝版。讀祝官南向跪,讀祝文。讀訖,俯伏興,舉祝官奠祝版於案,興。初獻再拜訖,樂止。降自北陛,詣太稷壇盥洗位,樂作,至位樂止。盥洗升獻並如太社後土之儀。降自北陛,樂作,復位,樂止。讀祝、舉祝官亦降復位。亞獻詣兩壇盥洗升獻,並如初獻之儀。終獻盥洗升獻,並如亞獻之儀。終獻奠獻畢,降復位,樂止,執事者亦復位。太祝各進徹籩豆,樂作,卒徹樂止。奉禮曰「賜胙,眾官再拜」。贊者承傳,在位者皆再拜訖,送神樂作,一成止。禮直官進初獻之左,曰「請詣望瘞位」,御史、博士從,樂作,至位北向立,樂止。初在位官將拜,諸太祝各執篚進於神座前,取瘞玉及幣,齋郎以俎載牲體並黍稷爵酒,各由其陛降,置於坎訖,贊者曰「可瘞」,東西各二人置土半坎。禮直官進初獻之左,曰「禮畢」,禮直官各引獻官以次出。禮直官引監祭、太祝以下執事官,俱復於壇北墉下,南向立定。奉禮曰「再拜」,監祭以下皆再拜訖,出。祝史、齋郎及工人以次出。祝版燔於齋所。光祿卿、監祭、監禮展視酒胙訖,乃退。



 其告祭儀,告前三日,三獻官以下諸執事官,各具公服,赴中書省受誓戒。告前一日,省牲器。告日質明,三獻官以下諸執事各服其服,禮直官引監祭、監禮以下諸執事官入自北墉下,南向立定。奉禮郎贊曰「再拜」。在位官皆再拜訖,奉禮郎贊曰「各就位」,「立定」。監祭、監禮視陳設畢,復位立定。禮直官引三獻、司徒、太常卿、光祿卿入就位,立定。禮直官贊「有司謹具,請行事」。降神樂作,八成止。太常卿瘞血,復位立定。奉禮郎贊「再拜」。皆再拜訖,禮直官引初獻官詣盥洗位,盥手訖,詣社壇正位神座前南向,搢笏跪,三上香,奠玉幣,執笏俯伏興。再拜訖,詣配位神座前西向,搢笏跪,三上香,奠幣,執笏俯伏興。再拜訖,詣稷壇盥洗位,盥手訖,升壇,並如上儀。俱畢,降復位。司徒率齋郎進饌,奠訖,降復位。禮直官引初獻官詣盥洗位,盥手訖,詣爵洗位,洗爵訖,詣酒尊所酌酒訖,詣社壇神位座前,南向立,搢笏跪,三上香,執爵,三祭酒於茅苴,爵授執事者,執笏俯伏興。俟讀祝官讀祝文訖,再拜興,詣酒尊所酌酒訖,詣配位神座前,西向,搢笏跪,三上香,執爵,三祭酒於茅苴,爵授執事者,執笏俯伏興。俟讀祝文訖,再拜興,詣稷壇盥洗位,盥手,洗爵,酌獻,並如上儀。俱畢,降復位。禮直官引亞獻,並如初獻之儀,惟不讀祝。俱畢,降復位。禮直官引終獻,並如亞獻之儀。俱畢,降復位。太祝徹籩豆訖,奉禮郎贊「賜胙」。眾官再拜訖,禮直官引三獻、司徒、太常卿詣瘞坎位,南向立定。禮直官贊「可瘞」,禮畢出。禮直官引監祭、監禮、太祝、齋郎至北墉下,南向立定。奉禮贊「再拜」,皆再拜訖,出。



 先農



 先農之祀,始自至元九年二月,命祭先農如祭社之儀。十四年二月戊辰,祀先農東郊。十五年二月戊午,祀先農,以蒙古胄子代耕籍田。二十一年二月丁亥,又命翰林學士承旨撒里蠻祀先農於籍田。武宗至大三年夏四月,從大司農請,建農、蠶二壇。博士議:二壇之式與社稷同,縱廣一十步,高五尺,四出陛,外壝相去二十五步,每方有欞星門。今先農、先蠶壇位在籍田內,若立外壝,恐妨千畝,其外壝勿築。是歲命祀先農如社稷,禮樂用登歌,日用仲春上丁,後或用上辛或甲日。祝文曰:「維某年月日,皇帝敬遣某官,昭告於帝神農氏。」配神曰「於後稷氏」。



 祀前一日未後,禮直官引三獻、監祭禮以下省牲饌如常儀。祀日醜前五刻,有司陳燈燭,設祝幣,大官令帥其屬入實籩豆尊罍。丑正,禮直官引先班入就位,立定,次引監祭禮按視壇之上下,糾察不如儀者。畢,退復位,東向立。奉禮曰「再拜」。贊者承傳再拜訖,奉禮又贊「諸執事者各就位」。禮直官各引執事官各就位,立定。次引三獻官並與祭等官以次入就位,西向立。禮直官於獻官之右,贊「請行事」,樂作三成止。奉禮贊「再拜」,在位者皆再拜。太祝跪取幣於篚,立於尊所。禮直官引初獻官詣盥洗位,北向立,盥手帨手畢,升自東階,詣神位前北向立,搢笏跪,三上香,受幣奠幣,執笏俯伏興,少退,再拜訖,降復位,立定。大官令率齋郎設饌於神位前畢,俯伏興,退復位。禮直官引初獻再詣盥洗位,北向立,盥手、帨手,詣爵洗位,洗爵拭爵,詣酒尊所酌酒畢,詣正位神位前,北向立。帨笏跪,三上香,三祭酒於沙池,爵授執事者,執笏俯伏興,北向立。俟讀祝畢,再拜興。次詣配位酒尊所,酌酒訖,詣神位前,東向立。搢笏跪,三上香,三祭酒於沙池,爵授執事者,執笏俯伏興,東向立。俟讀祝畢,再拜,退復位。次引亞終獻行禮,並如初獻之儀,惟不讀祝,退復位,立定。禮直官贊徹籩豆,樂作,卒徹,樂止。奉禮贊賜胙,眾官再拜。贊者承傳,在位者皆再拜訖,樂作送神之曲,一成止。禮直官引齋郎升自東階,太祝跪取幣祝,齋郎捧俎載牲體及籩豆簠簋,各由其階至坎位,北向立。俟三獻畢,至立定。各跪奠訖,執笏俯伏興。禮直官贊「可瘞」,乃瘞。焚瘞畢,三獻以次詣耕地所,耕訖而退。此其儀也。先蠶之祀未聞。



 宣聖



 宣聖廟,太祖始置於燕京。至元十年三月,中書省命春秋釋奠,執事官各公服如其品,陪位諸儒襴帶唐巾行禮。成宗始命建宣聖廟於京師。大德十年秋,廟成。至大元年秋七月,詔加號先聖曰大成至聖文宣王。延祐三年秋七月,詔春秋釋奠於先聖,以顏子、曾子、子思、孟子配享。封孟子父為邾國公,母為邾國宣獻夫人。皇慶二年六月,以許衡從祀,又以先儒周惇頤、程灝、程頤、張載、邵雍、司馬光、硃熹、張栻、呂祖謙從祀。至順元年,以漢儒董仲舒從祀。齊國公叔梁紇加封啟聖王,魯國太夫人顏氏啟聖王夫人;顏子,兗國復聖公;曾子,郕國宗聖公;子思,沂國述聖公;孟子,鄒國亞聖公;河南伯程灝,豫國公;伊陽伯程頤,洛國公。



 其祝幣之式,祝版三,各一尺二寸,廣八寸,木用楸梓柏,文曰:「維年月日,皇帝敬遣某官等,致祭於大成至聖文宣王。」於先師曰:「維年月日,某官等致祭於某國公。」幣三,用絹,各長一丈八尺。



 其牲齊器皿之數,牲用牛一、羊五、豕五。以犧尊實泛齊,象尊實醴齊,皆三,有上尊,加冪有勺,設堂上。太尊實泛齊,山罍實醴齊,有上尊。著尊實盎齊,犧尊實醴齊,象尊實沈齊,壺尊實三酒,皆有上尊,設堂下。盥洗位,在阼階之東。以象尊實醴齊,有上尊,加冪有勺,設於兩廡近北。盥洗位,在階下近南。籩十,豆十,簠二,簋二,登三,鉶三,俎三,有毛血豆,正配位同。籩豆皆二,簋一,簠一,俎一,從祀皆同。凡銅之器六百八十有一,宣和爵坫一,豆二百四十有八,簠簋各一百一十有五,登六,犧尊、象尊各六,山尊二,壺尊六,著尊、太尊各二,罍二,洗二。龍杓二十有七,坫二十有八,爵一百一十有八。竹木之器三百八十有四,籩二百四十有八,篚三,俎百三十有三。陶器三,瓶二,香爐一。籩巾二百四十有八,簠簋巾二百四十有八,俎巾百三十有三,黃巾蒙單十。



 其樂登歌。其日用春秋二仲月上丁,有故改用中丁。



 其釋奠之儀,省牲前期一日晡時,三獻官、監祭官各具公服,詣省牲所阼階,東西向立,以北為上。少頃,引贊者引三獻官、監祭官巡牲一匝,北向立,以西為上。俟禮牲者折身曰「充」,贊者曰「告充」畢,禮牲者又折身曰「腯」,贊者曰「告腯」畢,贊者復引三獻官、監祭官詣神廚,視滌溉畢,還齋所,釋服。釋奠,是日醜前五刻,初獻官及兩廡分奠官二員,各具公服於幕次,諸執事者具儒服,先於神門外西序東向立,以北為上。明贊、承傳贊先詣殿庭前再拜畢,明贊升露階東南隅西向立,承傳贊立於神門階東南隅西向立。掌儀先引諸執事者各司其事,引贊者引初獻官、兩廡分奠官點視陳設。引贊者進前曰「請點視陳設」。至階,曰「升階」,至殿簷下,曰「詣大成至聖文宣王神位前」,至位,曰「北向立」。點視畢,曰「詣兗國公神位前」。至位,曰「東向立」。點視畢,曰「詣鄒國公神位前」。至位,曰「西向立」。點視畢,曰「詣東從祀神位前」。至位,曰「東向立」。點視畢,曰「詣西從祀神位前」。至位,曰「西向立」。點視畢,曰「詣酒尊所」,曰「西向立」。點視畢,曰「詣三獻爵洗位」。至階,曰「降階」,至位,曰「北向立」。點視畢,曰「詣三獻官盥洗位」。至位,曰「北向立」。點視畢,曰「請就次」。



 方初獻點視時,引贊二人各引東西廡分奠官曰「請詣東西廡神位前」,至位東曰東,西曰西向立。點視畢,曰「詣先儒神位前」。至位,曰「南向立」。點視畢,曰「退詣酒尊所」。至酒尊所,東西向立。點視畢,曰「退詣分奠官爵洗位」。至位,曰「南向立」。點視畢,曰「請就次」。西廡分奠官點視畢,引贊曰「請詣望瘞位」。至位,曰「北向立」。點視畢,曰「請就次」。初獻官釋公服,司鐘者擊鐘,初獻以下各服其服,齊班於幕次。



 掌儀點視班齊,詣明贊報知,引禮者引監祭官、監禮官就位。進前曰「請就位」。至位,曰「就位,西向立。」明贊唱曰「典樂官以樂工進,就位」,承傳贊曰「典樂官以樂工進,就位」。明贊唱曰「諸執事者就位」,承傳贊曰「諸執事者就位」。明贊唱曰「諸生就位」,承傳贊曰「諸生就位」,引班者引諸生就位。明贊唱曰「陪位官就位」。承傳贊曰「陪位官就位」,引班者引陪位官就位。明贊唱曰「獻官就位」,承傳贊曰「獻官就位」,引贊者進前曰「請就位」,至位,曰「西向立」。明贊唱曰「闢戶」,俟戶闢,迎神之曲九奏。樂止,明贊唱曰「初獻官以下皆再拜」,承傳贊曰「鞠躬,拜,興,拜,興,平身」。明贊唱曰「諸執事者各司其事」。俟執事者立定,明贊唱曰「初獻官奠幣」。引贊者進前曰「請詣盥洗位」。盥洗之樂作,至位,曰「北向立」。搢笏,盥手,帨手,出笏,樂止。及階,曰「升階」。升殿之樂作。樂止,入門,曰「詣大成至聖文宣王神位前」。至位,曰「就位,北向立,稍前」。奠幣之樂作。搢笏跪,三上香,奉幣者以幣授初獻,初獻受幣奠訖,出笏就拜興,平身少退,再拜,鞠躬,拜興,拜興,平身。曰「詣兗國公神位前」。至位,曰「就位,東向立」,奠幣如上儀。曰「詣鄒國公神位前」。至位,曰「就位,西向立」,奠幣如上儀。樂止,曰「退復位」。及階,降殿之樂作。樂止,至位,曰「就位,西向立」。



 俟立定,明贊唱曰「禮饌官進俎」。奉俎之樂作,乃進俎,樂止,進俎畢。明贊唱曰「初獻官行禮」,引贊者進前曰「請詣盥洗位」。盥洗之樂作,至位,曰「北向立」。搢笏,盥手、帨手,出笏。請詣爵洗位,至位,曰「北向立」。搢笏,執爵、滌爵、拭爵,以爵授執事者,如是者三,出笏。樂止,曰「請詣酒尊所」。及階,升殿之樂作,曰「升階」。樂止,至酒尊所,曰「西向立」。搢笏,執爵舉冪,司尊者酌犧尊之泛齊,以爵授執事者,如是者三,出笏。曰「詣大成至聖文宣王神位前」。至位,曰「就位,北向立」。酌獻之樂作,稍前,搢笏跪,三上香,執爵三祭酒,奠爵,出笏,樂止。祝人東向跪讀祝,祝在獻官之左。讀畢興,先詣左配位,南向立。引贊曰「就拜興」,「平身」,「少退」,「再拜」,「鞠躬」,「拜,興」,「拜,興」,「平身」。曰「詣兗國公神位前」。至位,曰「就位,東向立」,酌獻之樂作。樂止,讀祝如上儀。曰「詣鄒國公神位前」。至位,曰「就位,西向立」,酌獻之樂作。樂止,讀祝如上儀。曰「退,復位」。至階,降殿之樂作。樂止,至位,曰「就位,西向立」。



 俟立定,明贊唱曰「亞獻官行禮」,引贊者進前曰「請詣盥洗位」。至位,曰「北向立」。搢笏,盥手,出笏。請詣爵洗位,至位,曰「北向立」。搢笏,執爵、滌爵、拭爵,以爵授執事者,如是者三,出笏。請詣酒尊所,曰「西向立」。搢笏,執爵舉冪,司尊者酌象尊之醴齊,以爵授執事者,如是者三,出笏。曰「詣大成至聖文宣王神位前」。至位,曰「就拜,北向立」。酌獻之樂作。稍前,搢笏跪,三上香,執爵三祭酒,奠爵出笏,就拜興,平身少退,鞠躬,拜興,拜興,平身。曰「詣兗國公神位前」。至位,曰「東向立」,酌獻如上儀。曰「詣鄒國公神位前」。至位,曰「西向立」,酌獻如上儀。樂止,曰「退,復位」。及階,曰「降階」,至位,曰「就位,西向立」。明贊唱曰「終獻官行禮」,引贊者進前曰「請詣盥洗位」,至位,曰「北向立」。搢笏,盥手,帨手,出笏。請詣爵洗位,至位,曰「北向立」。搢笏,執爵、滌爵、拭爵,以爵授執事者,如是者三,出笏。請詣酒尊所,至階,曰「升階」,至酒尊所,曰「西向立」。搢笏,執爵舉冪,司尊者酌象尊之醴齊,以爵授執事者,如是者三,出笏。曰「詣大成至聖文宣王神位前」。至位,曰「就位,北向立,稍前」。酌獻之樂作。搢笏跪,三上香,執爵三祭酒,奠爵,出笏,就拜興,平身少退,鞠躬,拜興,拜興,平身。曰「詣兗國公神位前」。至位,曰「東向立」,酌獻如上儀。曰「詣鄒國公神位前」。至位,曰「西向立」,酌獻如上儀。樂止,曰「退復位」。及階,曰「降階」,至位,曰「就位,西向立」。



 俟終獻將升階,明贊唱曰「分獻官行禮」。引贊者分引東西從祀分獻官進前曰「詣盥洗位」。至位,曰「北向立」。搢笏,盥手,帨手,出笏,詣爵洗位,至位,曰「北向立」。搢笏,執爵、滌爵、拭爵,以爵授執事者,出笏,詣酒尊所。至階,曰「升階」,至酒尊所,曰「西向立」。搢笏,執爵舉冪,司尊者酌象尊之醴齊,以爵授執事者,出笏,詣東從祀神位前。至位,曰「就位,東向立,稍前」。搢笏跪,三上香,執爵三祭酒,奠爵,出笏,就拜興,平身少退,鞠躬,拜興,拜興,平身,退復位。及階,曰「降階」,至位,曰「就位,西向立」。



 引西從祀分獻官同上儀,唯至神位前東向立。俟十哲分獻官離位,明贊唱曰「兩廡分奠官行禮」。引贊者進前曰「詣盥洗位」,至位,曰「南向立」。搢笏,盥手、帨手,出笏,詣爵洗位。至位,曰「南向立」。搢笏,執爵、滌爵、拭爵,以爵授執事者,出笏。曰「詣東廡酒尊所」。及階,曰「升階」,至酒尊所,曰「北向立」。搢笏,執爵舉冪,酌象尊之醴齊,以爵授執事者,出笏,詣東廡神位前,至位,曰「東向立,稍前」。搢笏跪,三上香,執爵三祭酒,奠爵,出笏,就拜興,平身稍退,鞠躬,拜興,拜興,平身,退復位。至階,曰「降階」,至位,曰「就位,西向立」。



 引西廡分奠官同上儀,唯至神位前,東向立作西向立。俟終獻十哲,兩廡分奠官同時復位。明贊唱曰「禮饌者徹籩豆」。徹豆之樂作,禮饌者跪,移先聖前籩豆,略離席,樂止。明贊唱曰「諸執事者退復位」。俟諸執事者至版位立定,送神之樂作。明贊唱曰「初獻官以下皆再拜」,承傳贊曰「鞠躬,拜,興,拜,興,平身」。樂止。明贊唱曰「祝人取祝,幣人取幣,詣瘞坎」。俟徹祝幣者出殿門,北向立。望瘞之樂作。明贊唱曰「三獻官詣望瘞位」,引贊者進前曰「請詣望瘞位」。至位,曰「就位,北向立」,曰「可瘞」。埋畢,曰「退,復位」。至殿庭前,候樂止,明贊唱曰「典樂官以樂工出就位」,明贊唱曰「闔戶」。又唱曰「初獻官以下退詣圓揖位」,引贊者引獻官退詣圓揖位。至位,初獻在西,亞終獻及分獻以下在東,陪位官東班在東,西班在西。俟立定,明贊唱曰「圓揖」。禮畢,退復位,引贊者各引獻官詣幕次更衣。



 其飲福受胙,除國學外,諸處仍依常制。



 闕里之廟,始自太宗九年,令先聖五十一代孫襲封衍聖公元措修之,官給其費。而代祠之禮,則始於武宗。牲用太牢,禮物別給白金一百五十兩,彩幣表裏各十有三匹。四年冬,復遣祭酒劉賡往祀,牲禮如舊。延祐之末,泰定、天歷初載,皆循是典,錦幣雜彩有加焉。



 岳鎮海瀆



 岳鎮海瀆代祀,自中統二年始。凡十有九處,分五道。後乃以東嶽、東海、東鎮、北鎮為東道,中嶽、淮瀆、濟瀆、北海、南嶽、南海、南鎮為南道,北嶽、西嶽、后土、河瀆、中鎮、西海、西鎮、江瀆為西道。既而又以驛騎迂遠,復為五道,道遣使二人,集賢院奏遣漢官,翰林院奏遣蒙古官,出璽書給驛以行。中統初,遣道士,或副以漢官。至元二十八年正月,帝謂中書省臣言曰:「五岳四瀆祠事,朕宜親往,道遠不可。大臣如卿等又有國務,宜遣重臣代朕祠之,漢人選名儒及道士習祀事者。」



 其禮物,則每處歲祀銀香合一重二十五兩,五岳組金幡二、鈔五百貫,四瀆織金幡二、鈔二百五十貫,四海、五鎮銷金幡二、鈔二百五十貫,至則守臣奉詔使行禮。皇帝登寶位,遣官致祭,降香幡合如前禮,惟各加銀五十兩,五嶽各中統鈔五百貫,四瀆、四海、五鎮各中統鈔二百五十貫。或他有禱,禮亦如之。



 其封號,至元二十八年春二月,加上東岳為天齊大生仁聖帝,南嶽司天大化昭聖帝,西岳金天大利順聖帝,北岳安天大貞玄聖帝,中嶽中天大寧崇聖帝。加封江瀆為廣源順濟王,河瀆靈源弘濟王,淮瀆長源溥濟王,濟瀆清源善濟王,東海廣德靈會王,南海廣利靈孚王,西海廣潤靈通王,北海廣澤靈佑王。成宗大德二年二月,加封東鎮沂山為元德東安王,南鎮會稽山為昭德順應王,西鎮吳山為成德永靖王,北鎮醫巫閭山為貞德廣寧王,中鎮霍山為崇德應靈王,敕有司歲時與岳瀆同祀。



 郡縣社稷



 至元十年八月甲辰朔,頒諸路立社稷壇壝儀式。十六年春三月,中書省下太常禮官,定郡縣社稷壇壝、祭器制度、祀祭儀式,圖寫成書,名《至元州郡通禮》。元貞二年冬,復下太常,議置壇於城西南二壇,方廣視太社、太稷,殺其半。壺尊二,籩豆皆八,而無樂。牲用羊豕,餘皆與太社、太稷同。三獻官以州長貳為之。



 郡縣宣聖廟



 中統二年夏六月,詔宣聖廟及所在書院有司,歲時致祭,月朔釋奠。八月丁酉,命開平守臣釋奠於宣聖廟。成宗即位,詔曲阜林廟,上都、大都諸路府州縣邑廟學、書院,贍學土地及貢士莊田,以供春秋二丁、朔望祭祀,修完廟宇。自是天下郡邑廟學,無不完葺,釋奠悉如舊儀。



 郡縣三皇廟



 元貞元年,初命郡縣通祀三皇,如宣聖釋奠禮。太皞伏羲氏以勾芒氏之神配,炎帝神農氏以祝融氏之神配,軒轅黃帝氏以風後氏、力牧氏之神配。黃帝臣俞跗以下十人,姓名載於醫書者,從祀兩廡。有司歲春秋二季行事,而以醫師主之。



 岳鎮海瀆常祀



 至元三年夏四月,定歲祀岳鎮海瀆之制。正月東嶽、鎮、海瀆,土王日祀泰山於泰安州,沂山於益都府界,立春日祀東海於萊州界,大淮於唐州界。三月南嶽、鎮、海瀆,立夏日遙祭衡山,土王日遙祭會稽山,皆於河南府界,立夏日遙祭南海、大江於萊州界。六月中嶽、鎮,土王日祀嵩山於河南府界,霍山於平陽府界。七月西嶽、鎮、海瀆,土王日祀華山於華州界,吳山於隴縣界,立秋日遙祭西海、大河於河中府界。十月北嶽、鎮、海瀆,土王日祀恆山於曲陽縣界,醫巫閭於遼陽廣寧路界,立冬日遙祭北海於登州界,濟瀆於濟源縣。祀官,以所在守土官為之。既有江南,乃罷遙祭。



 風雨雷師



 風、雨、雷師之祀,自至元七年十二月,大司農請於立春後丑日,祭風師於東北郊;立夏後申日,祭雷、雨師於西南郊。仁宗延祐五年,乃即二郊定立壇壝之制,其儀注闕。



 武成王



 武成王立廟於樞密院公堂之西,以孫武子、張良、管仲、樂毅、諸葛亮以下十人從祀。每歲春秋仲月上戊,以羊一、豕一、犧尊、象尊、籩、豆、俎、爵,樞密院遣官,行三獻禮。



 古帝王廟



 堯帝廟在平陽。舜帝廟,河東、山東濟南歷山、濮州、湖南道州皆有之。禹廟在河中龍門。至元元年七月,龍門禹廟成,命侍臣持香致敬,有祝文。十二年二月,立伏羲、女媧、舜、湯等廟於河中解州、洪洞、趙城。十五年四月,修會川縣盤古王祠,祀之。二十四年閏二月,敕春秋二仲丙日,祀帝堯廟。致和元年,禮部移太常送博士議,舜、禹之廟合依堯祠故事,每歲春秋仲月上旬卜日,有司蠲潔致祭,官給祭物。至順元年三月,從太常奉禮郎薛元德言,彰德路湯陰縣北故羑里城周文王祠,命有司奉祀如故事。



 周公廟



 周公廟在鳳翔府岐山之陽。天歷二年六月,以岐陽廟為岐陽書院,設學官,春秋釋奠周文憲王如孔子廟儀。凡有司致祭先代聖君名臣,皆有牲無樂。



 名山大川忠臣義士之祠



 凡名山大川、忠臣義士在祀典者,所在有司主之。惟南海女神靈惠夫人,至元中,以護海運有奇應,加封天妃神號,積至十字,廟曰靈慈。直沽、平江、周涇、泉、福、興化等處,皆有廟。皇慶以來,歲遣使齎香遍祭,金幡一合,銀一鋌,付平江官漕司及本府官,用柔毛酒醴,便服行事。祝文云:「維年月日,皇帝特遣某官等,致祭於護國庇民廣濟福惠明著天妃。」



 功臣祠



 功臣之祠,惟故淮安忠武王立廟於杭,春秋二仲月次戊,祀以少牢,用籩豆簠簋,行酌獻禮。若魏國文正公許衡廟在大名,順德忠獻王哈剌哈孫廟在順德、武昌者,皆歲時致祭。自古帝王而下,祭器不用籩豆簠簋,儀非酌奠者,有司便服行禮,三上香奠酒而已。



 大臣家廟



 大臣家廟,惟至治初右丞相拜住得立五廟,同堂異室,而牲器儀式未聞。



\end{pinyinscope}