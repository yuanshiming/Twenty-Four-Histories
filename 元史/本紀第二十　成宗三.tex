\article{本紀第二十 成宗三}

\begin{pinyinscope}

 三年春正月癸未朔,暹番、沒剌由、羅斛諸國各以方物來貢,賜暹番世子虎符。丙戌,太陰犯太白。己丑,中書省臣言:「天變屢見,大臣宜依故事引咎避位。」帝曰:「此漢人所說耳,豈可一一聽從耶?卿但擇可者任之。」庚寅,詔遣使問民疾苦。除本年內郡包銀、俸鈔,免江南夏稅十分之三,增給小吏俸米。置各路惠民局,擇良醫主之。封藥木忽而為定遠王,賜金印。命中書省:自今後妃、諸王所需,非奉旨勿給;各位擅置官府,紊亂選法者,戒飭之。辛卯,詔諸行省謹視各翼病軍。浙西肅政廉訪使王遇犯贓罪,托權幸規免,命御史臺鞫治之。壬辰,安置高麗陪臣趙仁規於安西、崔沖紹於鞏昌,並笞而遣之,以正其附王言原擅命妄殺之罪,復以王昛為高麗王,遣工部尚書也先鐵木而、翰林待制賈汝舟齎詔往諭之。追收別鐵木而、脫脫合兒魯行軍印。中書省臣言:「比年公帑所費,動輒巨萬,歲入之數,不支半歲,自餘皆借及鈔支。臣恐理財失宜,鈔法亦壞。」帝嘉納之。仍令諭月赤察而等自今一切賜與皆勿奏。癸巳,以江南軍數多闕,官吏因而作弊,詔禁飭之。以答剌罕哈剌哈孫為中書左丞相。丁酉,太陰犯西垣上將。戊戌,太陰犯右執法。辛丑,括諸路馬,隸蒙古軍籍者免之。乙巳,太白經天。



 二月癸丑朔,車駕幸柳林。丁巳,完澤等奏銓定省部官,以次引見,帝皆允之,仍諭六部官曰:「汝等事多稽誤,朕昔未知其人為誰。今既閱視,且知姓名,其洗心滌慮,各欽乃職。復蹈前失,罪不汝貸。」罷四川、福建等處行中書省,陜西行御史臺,江東、荊南、淮西三道宣慰司。置四川、福建宣慰司都元帥府及陜西漢中道肅政廉訪司。廣和林、甘州城。詔縉山縣民戶為勢家所蔽者,悉還縣定籍。壬戌,詔諭江浙、河南北兩省軍民。乙巳,熒惑犯五諸侯。壬申,加解州鹽池神惠康王曰廣濟,資寶王曰永澤;泉州海神曰護國庇民明著天妃;浙西鹽官州海神曰靈感弘祐公;吳大夫伍員曰忠孝威惠顯聖王。金齒國遣使來貢方物。庚辰,車駕幸上都。



 三月癸巳,緬國世子信合八的奉表來謝賜衣,遣還。命妙慈弘濟大師、江浙釋教總統補陀僧一山齎詔使日本,詔曰:「有司奏陳:向者世祖皇帝嘗遣補陀禪僧如智及王積翁等兩奉璽書通好日本,咸以中途有阻而還。爰自朕臨御以來,綏懷諸國,薄海內外,靡有遐遺,日本之好,宜復通問。今如智已老,補陀僧一山道行素高,可令往諭,附商舶以行,庶可必達。朕特從其請,蓋欲成先帝遺意耳。至於惇好息民之事,王其審圖之。」甲午,命何榮祖等更定律令。詔軍官受贓罪,重者罷職,輕者降其散官,或決罰就職停俸,期年許令自效。戊戌,熒惑犯輿鬼。升御史臺殿中司秩五品。乙巳,行御史臺劾平章教化受財三萬餘錠,教化復言平章的裡不花領財賦時盜鈔三十萬錠,及行臺中丞張閭受李元善鈔百錠,敕俱勿問。戊申,減江南諸道行臺御史大夫一員,賜和林軍鈔十萬錠。



 夏四月辛亥朔,駙馬蠻子臺所部匱乏,以糧十三萬石賑之。己未,太陰犯上將。丙寅,填星犯輿鬼,太陰犯心。庚午,申嚴江浙、兩淮私鹽之禁,巡捕官驗所獲遷賞。辛未,禁和林戍軍竄名他籍。自通州至兩淮漕河,置巡防捕盜司凡十九所。己卯,以禮部尚書月古不花為中書左丞。賜和林軍鈔五十萬錠、帛四十萬匹、糧二萬石,仍命和林宣慰司市馬五千匹給之。遼東開元、咸平蒙古、女直等人乏食,以糧二萬五百石、布三千九百匹賑之。



 五月壬午,罷江南諸路釋教總統所。丙申,太陰犯南斗。海南速古臺、速龍探、奔奚里諸番以虎象及桫羅木舟來貢。己亥,太白犯畢。庚子,免山東也速帶而牧地歲輸粟之半,禁阿而剌部毋於廣平牧馬。庚子,復征東行中書省,以福建平海省平章政事闊里吉思為平章政事。是月,鄂、岳、漢陽、興國、常、澧、潭、衡、辰、沅、寶慶、常寧、桂陽、茶陵旱,免其酒課、夏稅;江陵路旱、蝗,弛其湖泊之禁;仍並以糧賑之。六月辛亥,兀魯兀敦慶童擅殺所部軍之逃亡者,命樞密院戒之。癸丑,罷大名路所獻黃河故道田輸租。戊午,申禁海商以人馬兵仗往諸番貿易者。以福建州縣官類多色目、南人,命自今以漢人參用。禁福建民冒稱權豪佃戶,規免門役。庚申,太陰掩房。丁卯,熒惑犯右執法。壬申,歲星晝見。賜和林戍軍鈔一百四十萬錠,鷹師五十萬一千餘錠。



 秋七月己卯朔,太白犯井。庚辰,中書省臣言:「江南諸寺佃戶五十餘萬,本皆編民,自楊總攝冒入寺籍,宜加厘正。」從之。丙申,揚州、淮安屬縣蝗,在地者為鶖啄食,飛者以翅擊死,詔禁捕鶖。丁未,太陰犯輿鬼。



 八月己酉朔,日有食之。丁巳,太陰犯箕。戊辰,太白犯軒轅大星。己巳,太陰犯五車星。賜定遠王藥木忽而所部鈔萬五千錠。是月,汴梁、大都、河間水,隆興、平灤、大同、宣德等路雨雹。九月癸未,聖誕節,駐蹕古柵,受諸王百官賀。庚寅,置河東山西鐵冶提舉司。壬辰,流星色赤,尾長丈餘,其光燭地,起自河鼓,沒於牽牛之西,有聲如雷。癸巳,罷括宋手號軍。乙未,太陰犯昴距星。丁酉,太白犯左執法。己亥,車駕還大都。揚州、淮安旱,免其田租。



 冬十月戊申朔,有事於太廟。壬子,冊伯岳吾氏為皇后。甲寅,復立海北海南道肅政廉訪司。山東轉運使阿里沙等增課鈔四萬一千八百錠,賜錦衣人一襲。丙子,太陰犯房。賜禿忽魯不花等所部戶鈔三萬七千餘錠,橐駝戶十萬二千餘錠。以淮安、江陵、沔陽、揚、廬、隨、黃旱,汴梁、歸德水,隴、陜蝗,並免其田租。



 十一月庚辰,置浙西平江湖渠閘堰凡七十八所。禁和林釀酒。乙酉,太白犯房。戊子,釋囚二十人。丁酉,浚太湖及澱山湖。己亥,賜隆福宮牧駝者鈔十萬二千錠,諸王合帶部十萬錠,雲南王也先鐵木而及所部三萬八千錠,和林戍軍一百四十萬餘錠、幣帛二萬九千匹。杭州火,江陵路蝗,並發粟賑之。



 十二月己酉,徙鎮巢萬戶府戍沅、靖,毗陽萬戶府戍辰州,均州萬戶府戍常德、澧州。賜諸王岳忽難銀印。丙寅,詔各省戍軍輪次放還二年供役。升宣徽院為從一品。癸酉,詔中書省貨財出納,自今無券記者勿與。以守司徒、集賢院使、領太史院事阿魯渾撒里為平章政事。賜諸王六十、脫脫等鈔一萬三千餘錠,四怯薛衛士五萬二千餘錠,千戶撒而兀魯所部四萬錠。淮安、揚州饑,甘肅亦集乃路屯田旱,並賑以糧。



 四年春正月丙申,申嚴京師惡少不法之禁,犯者黥剌,杖七十,拘役。辛丑,詔蒙古都元帥也速答而非奉旨勿擅決重刑。命和林戍軍借斡脫錢者,止償其本。癸卯,復淮東漕渠。賜諸王塔失鐵木而金印。賜翰林承旨僧家鈔五百錠,以養其母。賜諸王木忽難所部一萬二千餘錠,八魯剌思等部六萬錠。



 二月丁未朔,日有食之。乙卯,遣使祠東嶽。丙辰,皇太后崩,明日祔葬先陵。戊午,太陰犯軒轅。壬戌,帝諭何榮祖曰:「律令良法也,宜早定之。」榮祖對曰:「臣所擇者三百八十條,一條有該三四事者。」帝曰:「古今異宜,不必相沿,但取宜於今者。」甲戌,發粟十萬石賑湖北饑民,仍弛山澤之禁。罷稱海屯田,改置於呵札之地,以農具、種實給之。乙亥,車駕幸上都。置西京太和嶺屯田。立烏撒、烏蒙等郡縣,並會理泗川西州為二,置維摩州。丙子,命李庭訓練各衛軍士。賜晉王所部鈔四萬錠。



 三月乙未,寧國、太平兩路旱,以糧二萬石賑之。



 夏四月丙午朔,詔雲南行省厘革積弊。壬子,高郵府寶應縣民孫奕妻硃一產三男,蠲復三年。丙辰,置五條河屯田。丁巳,免今年上都、隆興絲銀,大都差稅地租。賜諸王也滅乾鋈金印。緬國遣使進白象。戊午,參政張頤孫及其弟珪等伏誅於龍興市。頤孫初為新淦富人胡制機養子,後制機自生子而死,頤孫利其貲,與珪謀殺之,賂郡縣吏獲免。其僕胡忠訴主之冤於官,乃誅之,其貲悉還胡氏。以中書省斷事官不蘭奚為平章政事。賜皇侄海山所統諸王戍軍馬二萬二千九百餘匹。



 五月癸未,左丞相答剌罕遣使來言:「橫費不節,府庫漸虛。」詔自今諸位下事關錢穀者,毋輒入聞。帝諭集賢大學士阿魯渾撒裡等曰:「集賢、翰林乃養老之地,自今諸老滿秩者升之,勿令輒去,或有去者,罪將及汝。其諭中書知之。」增云南至緬國十五驛,驛給圓符四、驛券十二。甲午,太陰犯壘壁陣。辛丑,太白犯輿鬼,太陰犯昴。復延慶司。賜諸王也只里部鈔二萬錠,八憐脫列思所隸戶六萬五千餘錠。是月,同州、平灤、隆興雹,揚州、南陽、順德、東昌、歸德、濟寧、徐、濠、芍陂旱、蝗,真定、保定、大都通、薊二州水。六月己酉,詔立緬國王子窟麻剌哥撒八為緬國王,賜以銀印及金銀器皿衣服等物。丙辰,以太傅月赤察而為太師,完澤為太傅,皆賜之印。丁巳,太白犯填星。御史中丞不忽木卒,貧無以葬,賜鈔五百錠。甲子,置耽羅總管府。詔各省自今非奉命毋擅役軍。以和林都元帥府兼行宣慰司事。吊吉而、瓜哇、暹國、蘸八等國二十二人來朝,賜衣遣之。



 秋七月甲戌朔,右丞相完澤請上徽仁裕聖皇后謚寶冊。乙酉,緬國阿散哥也弟者蘇等九十一人各奉方物來朝,詔命餘人留安慶,遣者蘇來上都。辛卯,熒惑犯井。加乳母冀國夫人韓氏為燕冀國順育夫人,石抹氏為冀國夫人。杭州路貧民乏食,以糧萬石減其直糶之。



 八月癸卯朔,更定廕敘格,正一品子為正五,從五品子為從九,中間正從以是為差,蒙古、色目人特優一級。置廣東鹽課提舉司。癸丑,太陰犯井。庚申,緬國阿散吉牙等昆弟赴闕,自言殺主之罪;罷征緬兵。甲子,辰星犯靈臺上星。大名之白馬縣旱。



 閏八月庚辰,熒惑犯輿鬼。庚子,車駕還大都。以中書右丞賀仁傑為平章政事。賜晉王所部糧七萬石。



 九月戊午,太白犯鬥。壬戌,太陰犯輿鬼。曹州探馬赤軍與民訟地百二十頃,詔別以鄰近官田如數給之。廣東英德州達魯花赤脫歡察而招降群盜二千餘戶,升英德州為路,立三縣,以脫歡察而為達魯花赤兼萬戶以鎮之。甲子,太白犯鬥。改中御府為中政院。賜諸王出伯所部鈔萬五千四百餘錠。建康、常州、江陵饑民八十四萬九千六十餘人,給糧二十二萬九千三百九十餘石。



 冬十月癸酉朔,有事於太廟。



 十一月壬寅朔,詔頒寬令,免上都、大都、隆興大德五年絲銀、稅糧,附近秣養馬駝之郡免稅糧十分之三,其餘免十分之一;徒罪各減一半,杖罪以下釋之;江北荒田許人耕種者,元擬第三年收稅,今並展限一年,著為定例。並遼陽省所轄狗站、牛站為一,仍給鈔以賙其乏。命省、臺差官同昔寶赤鞫和林運糧稽遲未至者。真定路平棘縣旱。



 十二月癸酉,御史臺臣言:「所糾官吏與有司同審,所以事沮難行,乞依舊制。中書凡有改作,輒令監察御史同往,非宜,自今非奉旨勿遣。」皆從之。庚寅,熒惑犯軒轅。癸巳,太陰犯房距星。晉州達魯花赤捏古伯紿稱母喪,歸迎其妻。事聞,詔以其斁傷彞倫,罷職不敘。遣劉深、合剌帶、鄭祐將兵二萬人徵八百媳婦,仍敕云南省每軍十人給馬五匹,不足則補之以牛。賜諸王忻都部鈔五萬錠,兀魯思不花等四部二十一萬九千餘錠,西都守城軍二萬八千餘錠。賑建康、平江、浙東等處饑民糧二十二萬九千三百餘石。



 五年春正月己酉,太陰犯五車。庚戌,給徵八百媳婦軍鈔,總計九萬二千餘錠。壬子,太陰犯輿鬼積尸氣。奉安昭睿順聖皇后御容於護國仁王寺。罷檀、景兩州採金鐵冶提舉司,以其事入都提舉司。御史臺臣言:「官吏犯贓及盜官錢,事覺避罪逃匿者,宜同獄成。雖經原免,亦加降黜,庶奸偽可革。」從之。丙寅,以兩淮鹽法澀滯,命轉運司官兩員分司上江以整治之,仍頒印及驛券。辛酉,太陰犯心。



 二月己卯,太陰犯輿鬼。以劉深、合剌帶並為中書右丞,鄭祐為參知政事,皆佩虎符。分雲南諸路行中書省事,仍置理問官二員,郎中、員外郎、都事各一員,給圓符四、驛券二十。罷福建織繡提舉司。增河間轉運司鹽為二十八萬引,罷其所屬清、滄、深三鹽司。丁亥,立徵八百媳婦萬戶府二,設萬戶四員,發四川、雲南囚徒從軍。乙未,詔廉訪司官非親喪遷葬及以病給告者,不得離職;或以地遠職卑受任不赴者,臺憲勿復用。丙申,給脫脫等部馬萬匹。丁酉,車駕幸上都。詔飭云南行中書省減內外諸司官千五百一十四員,增江浙戍兵。戊戌,賜昭應宮、興教寺地各百頃,興教仍賜鈔萬五千錠;上都乾元寺地九十頃,鈔皆如興教之數;萬安寺地六百頃,鈔萬錠;南寺地百二十頃,鈔如萬安之數。己亥,凡軍士殺人奸盜者,令軍民官同鞫。永寧路總管雄挫來朝,獻馬三十餘匹,賜幣帛有差。



 三月甲辰,收故軍官金銀符。戊申,太陰犯御女。己酉,罷陜西路拘榷課稅所。壬子,賜諸王也孫等鈔一萬八千五百錠。戊午,馬來忽等海島遣使來朝,賜金素幣有差。給和林貧乏軍鈔二十萬錠,諸王藥木忽而所部萬五千九百餘錠。丁卯,熒惑犯填星。己巳,熒惑、填星相合。詔戒飭中外官吏。命遼陽行省平章沙藍將萬人駐夏山後,人備馬二匹,官給其直。



 夏四月壬申,太陰犯東井。癸酉,遣禿剌鐵木而等犒和林軍。壬午,以晉王甘麻剌所部貧乏,賜鈔四十萬錠。調雲南軍徵八百媳婦。癸巳,禁和林釀酒,其諸王、駙馬許自釀飲,不得沽賣。是月,大都、彰德、廣平、真定、順德、大名、濮州蟲食桑。



 五月,商州隕霜殺麥。河南妖賊丑斯等伏誅。己酉,給月裏可里軍駐夏山後者市馬鈔八萬八千七百餘錠。辛亥,遣怯列亦帶脫脫帥師征四川。癸丑,太陰犯南斗。乙卯,熒惑犯右執法。丙辰,曲靖等路宣慰使兼管軍萬戶忽林失來朝。壬戌,雲南土官宋隆濟叛。時劉深將兵由順元入雲南,雲南右丞月忽難調民供饋,隆濟因紿其眾曰:「官軍徵發汝等,將盡剪發黥面為兵,身死行陣,妻子為虜。」眾惑其言,遂叛。丙寅,詔雲南行省自願徵八百媳婦者二千人,人給貝子六十索。丁卯,太白犯井。六月乙亥,平江等十有四路大水,以糧二十萬石隨各處時直賑糶。開中慶路昆陽州海口。甲申,歲星犯司怪。丙戌,宋隆濟率貓、狫、紫江諸蠻四千人攻楊黃寨,殺掠甚眾。己丑,緬王遣使獻馴象九。壬辰,宋隆濟攻貴州,知州張懷德戰死。梁王遣雲南行省平章幢兀兒、參政不蘭奚將兵御之,殺賊酋撒月,斬首五百級。癸巳,太白犯輿鬼,歲星犯井。甲午,太白犯輿鬼。賜諸王念不烈妃札忽而真所部鈔二十萬錠。是月,汴梁、南陽、衛輝、大名、濮州旱,大都路水,順德、懷孟蝗。



 秋七月戊戌朔,晝晦,暴風起東北,雨雹兼發,江湖泛溢,東起通、泰、崇明,西盡真州,民被災死者不可勝計,以米八萬七千餘石賑之。己亥,增階、沙二州戍軍。庚子,籍安西王所侵占田、站等四百餘戶為民,賜寧遠王闊闊出所部鈔二萬三千餘錠。乙巳,遼陽省大寧路水,以糧千石賑之。丙午,歲星犯井。丁未,命御史大夫禿忽赤整飭臺事。詔軍官受贓者與民官同例,量罪大小殿黜。命監察御史審覆札魯忽赤罪囚,檢照蒙古翰林院案牘。戊申,立耽羅軍民萬戶府。諸王也滅乾薨,以其子八八剌嗣。己酉,詔諸司嚴禁盜賊。辛亥,太陰犯壘壁陣。賜諸王出伯等部鈔六萬錠,又給市馬直三十八萬四千錠。癸丑,詔禁畏吾兒僧、陰陽、巫覡、道人、咒師,自今有大祠禱必請而行,違者罪之。浙西積雨泛溢,大傷民田,詔役民夫二千人疏導河道,俾復其故。命雲南省分蒙古射士徵八百媳婦。庚申,辰星犯太白。癸亥,合丹之孫脫歡自北境來歸,其父母妻子皆遭殺虜,賜鈔一千四百錠。給諸王妃札忽而真及諸王出伯軍鈔四十萬錠。中書省臣言:「舊制京師州縣捕盜,止從兵馬司,有司不與,遂致淹滯。自今輕罪乞令有司決遣,重者從宗正府聽斷,庶不留獄,且民不冤。」從之。以暗伯、阿忽臺並知樞密院事。禁富豪之家役軍。詔封贈非中書省無輒奏請。稱海至北境十二站大雪,馬牛多死,賜鈔一萬一千餘錠。命御史臺檢照宣政院並僧司案牘。升太醫院為二品,以平章政事、大都護、提點太醫事脫因納為太醫院使。賜上都諸匠等鈔二十一萬七千四百錠。大都、保定、河間、濟寧、大名水,廣平、真定蝗。



 八月戊辰,給軍人羊馬價及定遠王所部鈔十四萬三千錠。己巳,平灤路霖雨,灤、漆、淝、汝河溢,民死者眾,免其今年田租,仍賑粟三萬石。庚午,禿剌鐵木而等自和林犒軍還,言:「和林屯田宜令軍官廣其墾闢,量給農具,倉官宜任選人,可革侵盜之弊。」從之。甲戌,遣薛超兀而等將兵征金齒諸國,時征緬師還,為金齒所遮,士多戰死。又接連八百媳婦諸蠻,相效不輸稅賦,賊殺官吏,故皆徵之。庚辰,詔:「遣官分道賑恤。凡獄囚禁系累年,疑不能決者,令廉訪司具其疑狀,申呈省、臺詳讞,仍為定例。各路被災重者,免其差稅一年,貧乏之家,計口賑恤,尤甚者優給之。小吏犯贓者,並罷不敘。」徵緬萬戶曳剌福山等進馴象六。壬辰,太陰犯軒轅御女。乙未,填星犯太微上將。順德路水,免其田租。九月癸丑,放稱海守倉庫軍還,令以次更代。丙辰,江陵、常德、澧州皆旱,並免其門攤、酒醋課。乙酉,自八月庚辰彗出井,歷紫微垣至天市垣,凡四十六日而滅。



 冬十月丙寅朔,以畿內歲饑,增明年海運糧為百二十萬石。己巳,緬王遣使入貢。戊寅,雲南武定路土官群則獻方物。癸未,太陰犯東井。壬午,車駕還大都。丙戌,以歲饑禁釀酒,弛山澤之禁,聽民捕獵。湖廣行省臣言:「海南海北道宣慰司都元帥府,不與軍務,遇有盜竊,惟行文移,比回,已不及事,今乞以其長二人領軍務。又鎮守官慢功當罰者,已有定例;獲功當賞者,乞或加散官,或授金、銀符。」皆從之。撥南陽府屯田地給新籍畏吾而戶,俾耕以自贍,仍給糧三月。丁亥,詔:「軍官既受命而不時赴者、病故不行者、被差事畢不即還者,準民官例,違限六月,選人代之,被代者期年始敘。」改鄂州路為武昌路。遣使就調雲南、四川、福建、廣東、廣西官,諭百司凡事關中書省者,毋得輒奏。權豪勢要之家佃戶貸糧者,聽於來歲秋成還之。癸巳,分碉門、黎、雅軍戍蠻夷,命陜西屯田萬戶也不乾等將之。辛卯,夜有流星大如杯,光燭地,自北起近東分為二星,沒於危宿。



 十一月己亥,歲星犯東井,詔諭中書,近因禁酒,聞年老需酒之人有預市而儲之者,其無釀具者勿問。罷湖南轉運司弘州種田提舉司,以其事入有司。降容、象、橫、賓路為州,平灤金丹提舉司為管勾,升昭州為平樂府,省泌縣入唐州。丁未,遣劉國傑及也先忽都魯將兵萬人,八剌及阿塔赤將兵五千人,徵宋隆濟。減直糶米,賑京師貧民,設肆三十六所,其老幼單弱不能自存者,廩給五月。選六衛扈從漢軍習武事,仍禁萬戶以下毋令私代,犯者斷罪有差。戊申,太陰犯昴。徭人藍賴率丹陽三十六洞來降,以賴等為融州懷遠縣簿、尉。立長信寺,秩三品。



 十二月甲戌,歲星犯司怪。給安西王所部軍士食,令各還其家,候春調遣。辛卯,太陰犯南斗。征東行省平章闊里吉思以不能和輯高麗罷。定強竊盜條格,凡盜人孳畜者,取一償九,然後杖之。是歲,汴梁、歸德、南陽、鄧州、唐州、陳州、和州、襄陽、汝寧、高郵、揚州、常州蝗,峽州、隨州、安陵、荊門、泰州、光州、揚州、滁州、高郵、安豐霖,汴梁之封丘、陽武、蘭陽、中牟、延津,河南澠池,蘄州之蘄春、廣濟、蘄水旱,大名、宣德、奉聖、歸德、寧海、濟寧、般陽、登州、萊州、益都、濰州、博興、東平、濟南、濱州、保定、河間、真定、大寧水。是歲,斷大闢六十一人。



 六年春正月癸卯,詔千戶、百戶等自軍逃歸,先事而逃者罪死,敗而後逃者,杖而罷之,沒入其男女。乙巳,中書省臣言:「廣東宣慰副使脫歡察而收捕盜賊,屢有勞績,近廉訪司劾其私置兵仗、擅殺土寇等事,遣官鞫問,實無私罪,乞加獎諭。」命賜衣二襲。晉王甘麻剌薨,命封其王印及內史府印。丙午,京畿二十一站闕食,命賜鈔萬二千七百餘錠。陜西旱,禁民釀酒。以雲南站戶貧乏,增馬及鈔以優恤之。中書省臣以硃清、張瑄屢致人言,乞罷其職,徙其諸子官江南者於京。丁未,命江浙平章阿裏專領其省財賦。庚戌,詔官吏犯罪已經赦宥者,仍從核問。海道漕運船,令探馬赤軍與江南水手相參教習,以防海寇。江南僧石祖進告硃清、張瑄不法十事,命御史臺詰問之。帝語臺臣曰:「朕聞江南富戶侵占民田,以致貧者流離轉徙,卿等嘗聞之否?」臺臣言曰:「富民多乞護持璽書,依倚以欺貧民,官府不能詰治,宜悉追收為便。」命即行之,毋越三日。詔自今僧官、僧人犯罪,御史臺與內外宣政院同鞫,宣政院官徇情不公者,聽御史臺治之。增諸王塔赤鐵木而歲賜銀二百五十兩、雜幣百匹。乙卯,築渾河堤長八十里,仍禁豪家毋侵舊河,令屯田軍及民耕種。增劉國傑等軍,仍令屯戍險隘,俟秋進師。命札忽而帶、阿里等整治江南影占稅民地土者。中書省臣言:「御史臺、廉訪司,體察、體覆,前後不同。初立臺時,止從體察,後立按察司,事無大小,一皆體覆。由是憲司之事,積不能行。請自今除水旱災傷體覆,餘依舊例體察為宜。」從之。以大都、平灤等路去年被水,其軍應赴上都駐夏者,免其調遣一年。詔軍官除邊遠出征,其餘遇祖父母、父母喪,依民官例,立限奔赴。禁畜養鷹、犬、馬、駝等人擾民。乙未,以諸王真童誣告濟南王,謫置劉國傑軍中自效。壬戌,鎮星犯太微垣上將。



 二月庚午,太陰犯昴。謫諸王孛羅於四川八剌軍中自效。癸酉,增諸王出伯軍三千人,人備馬二匹,官給其直。丙戌,遣陜西省平章也速帶而、參政汪惟勤將川陜軍,湖廣平章劉國傑將湖廣軍,徵亦乞不薛,一切軍務,並聽也速帶而、劉國傑節制。罷徵八百媳婦右丞劉深等官,收其符印、驛券。以京師民乏食,命省、臺委官計口驗實,以鈔十一萬七千一百餘錠賑之。癸巳,帝有疾,釋京師重囚三十八人。



 三月丁酉,以旱、溢為災,詔赦天下。大都、平灤被災尤甚,免其差稅三年,其餘災傷之地,已經賑恤者免一年。今年內郡包銀、俸鈔,江淮已南夏稅,諸路鄉村人戶散辦門攤課程,並蠲免之。壬寅,太陰犯輿鬼。命僧設水陸大會七晝夜。癸卯,歲星犯井。甲寅,太陰犯鉤鈐。合祭昊天上帝、皇地祇於南郊,遣中書左丞相答剌罕哈剌哈孫攝事。



 夏四月乙丑朔,太白犯東井。丁卯,詔曲赦雲南諸部蠻夷;發通州倉粟三百石賑貧民;釋輕重囚三十八人,人給鈔五錠。乙亥,浚永清縣南河。戊寅,太陰犯心。庚辰,上都大水民饑,減價糶糧萬石賑之。戊子,修盧溝上流石徑山河堤。釋重囚。車駕幸上都。庚寅,太白犯輿鬼。真定、大名、河間等路蝗。



 五月乙巳,給貧乏漢軍地,及五丁者一頃,四丁者二頃,三丁者三頃,其孤寡者存恤六年,逃散者招諭復業。戊申,太廟寢殿災。癸丑,謫和林潰軍征雲南,其戰傷而歸及嘗奉晉王令旨、諸王藥木忽而免者,不遣。丁巳,福州路饑,賑以糧一萬四千七百石。濟南路大水,揚州、淮安路蝗,歸德、徐州、邳州水。六月癸亥朔,日有食之。太史院失於推策,詔中書議罪以聞。填星犯太微西垣上將。甲子,建文宣王廟於京師。辛未,享於太廟。乙亥,太陰犯鬥。安南國以馴象二及硃砂來獻。甲申,賜諸王合答孫、脫歡、脫列鐵木而、伯牙倫、完者所部鈔四萬五千八百餘錠。湖州、嘉興、杭州、廣德、饒州、太平、婺州、慶元、紹興、寧國等路饑,賑糧二十五萬一千餘石。大同路、寧海州亦饑,以糧一萬六千石賑之。廣平路大水。



 秋七月癸巳朔,熒惑、鎮星、辰星聚井。庚子,太陰犯心。己酉,亦乞不薛土官三人棄家來歸,賜金銀符、衣服。戊午,太陰犯熒惑。辛酉,賜諸王八八剌、脫脫灰、也只里、也滅乾等鈔四萬三千九百餘錠。以江浙行省參知政事忽都不丁為中書右丞。建康民饑,以米二萬石賑之。大都諸縣及鎮江、安豐、濠州蝗,順德水。



 八月甲子,詔御史臺凡有司婚姻、土田文案,遇赦依例檢覆。乙丑,熒惑犯歲星。己巳,熒惑犯輿鬼。辛巳,太陰犯昴。壬午,太白犯軒轅。九月乙未,遣阿牙赤、撒罕禿會計稱海屯田歲入之數,仍自今令宣慰司官與阿剌臺共掌之。甲午,賜諸王兀魯思不花所部鈔六萬錠。丙午,熒惑犯軒轅。丁未,中書省臣言:「羅裡等擾民,宜依例決遣置屯田所。」從之。賜諸王八撒而等鈔八萬六千三百餘錠。己酉,龍興民訛言括童男女,至有殺其子者,命誅其為首者三人。癸丑,太陰犯輿鬼。丁巳,太白犯右執法。賜諸王捏苦迭而等鈔五千八百四十錠。



 冬十月甲子,改浙東宣慰司為宣慰司都元帥府,徙治慶元,鎮遏海道。置大同路黃花嶺屯田。罷軍儲所,立屯儲軍民總管萬戶府,設官六員,仍以軍儲所宣慰使法忽魯丁掌之。南人林都鄰告浙西廉訪使張珪收藏禁書及推算帝五行,江浙運使合只亦言珪沮撓鹽法,命省、臺官同鞫之。丙子,車駕還大都。壬午,熒惑犯太微西垣上將。濟南濱、棣、泰安、高唐州霖雨,米價騰湧,民多流移,發粟賑之,並給鈔三萬錠。



 十一月辛卯,填星犯左執法。甲午,劉國傑裨將宋光率兵大敗蛇節,賜衣二襲,仍授以金符。乙未,辰星犯房。癸卯,太陰犯昴。己酉,太陰犯軒轅。庚戌,禁和林軍釀酒,惟安西王阿難答、諸王忽剌出、脫脫、八不沙、也只里、駙馬蠻子臺、弘吉列帶、燕裡幹許釀。辛亥,以同知樞密院事合答知樞密院事。詔江南寺觀凡續置民田及民以施入為名者,並輸租充役。戊午,籍河西寧夏善射軍隸親王阿木哥,甘州軍隸諸王出伯。己未,詔諸驛使輒枉道者罪之。



 十二月庚申朔,熒惑犯填星。辛酉,御史臺臣言:「自大德元年以來,數有星變及風水之災,民間乏食。陛下敬天愛民之心,無所不盡,理宜轉災為福;而今春霜殺麥,秋雨傷稼,五月太廟災,尤古今重事。臣等思之,得非荷陛下重任者不能奉行聖意,以致如此。若不更新,後難為力。乞令中書省與老臣識達治體者共圖之。」復請禁諸路釀酒,減免差稅,賑濟饑民。帝皆嘉納,命中書郎議行之。雲南地震。戊辰,又震。甲子,衡州袁舜一等誘集二千餘人侵掠郴州,湖南宣慰司發兵討之,獲舜一及其餘黨,命誅其首謀者三人,餘者配洪澤、芍陂屯田,其脅從者招諭復業。乙丑,歲星犯輿鬼。乙亥,太陰犯輿鬼。丙子,劉國傑、也先忽都魯來獻蛇節、羅鬼等捷。庚辰,熒惑犯太微東垣上相。命中書省更定略賣良人罪例。癸未,太陰犯房。保定等路饑,以鈔萬錠賑之。是歲,斷大闢三人。



\end{pinyinscope}