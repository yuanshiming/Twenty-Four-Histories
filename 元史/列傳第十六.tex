\article{列傳第十六}

\begin{pinyinscope}

 ○來阿八赤



 來阿八赤,寧夏人。父術速忽裏,歸太祖,選居宿衛,繼命掌膳事。憲宗即位,大舉伐宋,攻釣魚山,命諸將議進取之計,術速忽裏言於帝曰:「川蜀之地,三分我有其二,所未附者巴江以下數十州而已,地削勢弱,兵糧皆仰給東南,故死守以抗我師。蜀地巖險,重慶、合州又其籓屏,皆新築之城,依險為固,今頓兵堅城之下,未見其利。曷若城二城之間,選銳卒五萬,命宿將守之,與成都舊兵相出入,不時擾之,以牽制其援師。然後我師乘新集之銳,用降人為鄉導,水陸東下,破忠、涪、萬、夔諸小郡,平其城,俘其民,俟冬水涸,瞿唐三峽不日可下,出荊楚,與鄂州渡江諸軍合勢,如此則東南之事一舉可定。其上流重慶、合州,孤危無援,不降即走矣。」諸將曰「攻城則功在頃刻」,反以其言為迂,卒不用。於是博選宿衛中材力可任用者,以阿八赤奉命往監元帥紐鄰軍,遏宋人援兵,駐重慶下流之銅羅峽,夾江據崖為壘。宋都統甘順自夔州溯流西上,乘舟來攻。阿八赤預積薪於二壘,明火鼓噪,矢石如雨,順流而進。宋人力戰不能支,退保西岸,斂兵自固。黎明復至,阿八赤身率精兵,緣崖而下,戰艦復進,宋人敗走,殺傷數千人。帝聞而壯之,賜銀二鋌。憲宗崩,阿八赤從父倍道歸燕。世祖即位,問以川蜀之事,阿八赤歷陳始末,誦其父前所言以對,世祖撫掌曰:「當時若從此策,東南其足平乎!朕在鄂渚,日望上流之聲勢耳。」



 至元七年,南征襄樊,發河南、北器械糧儲悉聚於淮西之義陽。慮宋人剽掠,命阿八赤督運,二日而畢。既還,世祖大悅,以銀一鋌賜之。十四年,立尚膳院,授中順大夫、同知尚膳院事。十八年,佩三珠虎符,授通奉大夫、益都等路宣慰使、都元帥。發兵萬人開運河,阿八赤往來督視,寒暑不輟。有兩卒自傷其手,以示不可用,阿八赤檄樞密並行省奏聞,斬之以懲不律。運河既開,遷膠萊海道漕運使。二十一年,調同僉宣徽院事。遼左不寧,復降虎符,授征東招討使。阿八赤招徠降附,期以自新,遠近帖然。二十二年,授征東宣慰使、都元帥。



 皇子鎮南王征交趾,授湖廣等處行中書省右丞,召見,世祖親解衣衣之,並金玉束帶及弓矢甲胄賜焉。二十四年,改湖廣等處行尚書省右丞,詔四省所發士馬,俾阿八赤閱視。九月,領中衛親軍千人,翊導皇子至思明州。賊阻險拒守,於是選精銳與賊戰於女兒關,斬馘萬計,餘兵棄關走。於是大軍深入,進至交州,陳日烜空其城而遁。阿八赤曰:「賊棄巢穴而匿山海者,意待吾之敝而乘之耳。將士多北人,春夏之交瘴癘作,賊弗就擒,吾不能持久矣。今出兵分定其地,招降納附,勿縱士卒侵掠,急捕日烜,此策之善者也。」時日烜屢遣使約降,欲以賂緩我師。諸將皆信其說,且修城以居而待其至。久之,軍乏食,日烜不降,擁眾據竹洞、安邦海口。阿八赤率兵往攻之,屢與賊遇,晝夜迎戰,賊兵敗遁。會將士多疫不能進,而諸蠻復叛,所得關厄皆失守,乃議班師。選諸軍步騎,命先啟行,且戰且行,日數十合。賊據高險,射毒矢,將士裹瘡以戰,諸軍護皇子出賊境,阿八赤中毒矢三,首項股皆腫,遂卒。



 子寄僧,為水達達屯田總管府達魯花赤。乃顏叛,戰於高麗雙城。調萬安軍達魯花赤。平黎蠻有功,遷雷州路總管,卒。孫完者不花,同知潮州路總管府事;次禿滿不花、也先不花、太不花。



 ○紐璘也速答兒附



 紐璘,珊竹帶人。祖孛羅帶,為太祖宿衛,從太宗平金,戍河南。父太答兒,佐憲宗征阿速、欽察等國有功,拜都元帥。歲壬子,率陜西西海、鞏昌諸軍攻宋,入蜀。癸丑,與總帥汪田哥立利州。甲寅,攻碉門、黎、雅等城。乙卯,入重慶,獲都統制張實。是歲卒。



 紐璘偉貌長身,勇力絕人,且多謀略,常從父軍中。丁巳歲,憲宗命將兵萬人略地,自利州下白水,過大獲山,出梁山軍直抵夔門。戊午,還釣魚山,引軍欲會都元帥阿答胡等於成都。宋制置使蒲擇之,遣安撫劉整、都統制段元鑒等,率眾據遂寧江箭灘渡以斷東路。紐璘軍至,不能渡,自旦至暮大戰,斬首二千七百餘級,遂長驅至成都。帝聞,賜金帛勞之。蒲擇之命楊大淵等守劍門及靈泉山,自將四川兵取成都。會阿答胡死,諸王阿卜干與諸將脫林帶等謀曰:「今宋兵日逼,聞我帥死,必悉眾來攻,其鋒不可當。我軍去朝廷遠,待上命建大帥,然後禦敵,恐無及已。不若推紐璘為長,以號令諸將,出彼不意,敵可必破。」眾然之,遂推紐璘為長。璘率諸將大破宋軍於靈泉山,乘勝追擒韓勇,斬之,蒲擇之兵潰。進圍雲頂山城,扼宋軍歸路。其主將倉卒失計,遂以其眾降。城中食盡,亦殺其守將以降。成都、彭、漢、懷、綿等州悉平,威、茂諸蕃亦來附。紐璘奉金銀、竹箭、銀銷刀,遣速哥入獻。帝賜黃金五十兩,即軍中真拜都元帥。



 時紐璘軍止二萬,以五千命拜延八都魯等守成都,自將萬五千人從馬湖趨重慶。冬,帝進軍至大獲山,紐璘率步騎號五萬,戰船二百艘,發成都。遣張威以五百人為前鋒,水陸並進,謀鎖重慶江,以絕吳、蜀之路,縛橋資州之口以濟師。千戶暗都剌率舟師而下,紐璘將步騎而南,旌旗輜重百里不絕,鼓噪渡瀘,放舟而東。蒲擇之以兵分道要遮,遇輒敗之。紐璘至涪,造浮橋,駐軍橋南北,以杜宋援兵。聞大軍多虐癘,遣人進牛犬豕各萬頭。明年春,朝行在所,還討思、播二州,獲其將一人。宋將呂文煥攻涪浮橋,時新立成都,士馬不耐其水土,多病死,紐璘憂之。密旨督戰,不得已出師,大敗文煥軍,獲其將二人,斬之,遂班師。文煥以兵襲其後,紐璘戰卻之。



 中統元年,世祖即位,紐璘入朝,賜虎符及黃金五十兩、白金二千五百兩、馬二匹。紐璘遣梁載立招降黎、雅、碉門、巖州、偏林關諸蠻,得漢、番二萬餘戶。未幾,詔速哥分西川兵及陜西諸軍屬紐璘,鎮秦、鞏、唐兀之地。三年,宋將劉整以瀘州降,呂文煥圍之,詔以兵往援,文煥敗走,遂徙瀘州民於成都、潼川。四年,為劉整所譖,徵至上都,驗問無狀,詔釋之。還至昌平,卒。子也速答兒。



 也速答兒勇智類其父,至元十一年,入見世祖,以屬行樞密院火都赤,使習兵事。從圍嘉定,以三千人至三龜、九頂山相地形勢,敗宋安撫昝萬壽兵,斬首五百級,以功賜虎符,授六翼達魯花赤。昝萬壽尋遣部將李立以嘉定、三龜、九頂、紫雲諸城寨降。又從行樞密副使忽敦率兵徇下流諸城,皆望風來附。忽敦以兵二萬會東川行樞密院合答圍重慶,歲餘不下,帝命行樞密副使不花代將。不花將兵萬餘至城下,也速答兒率二十餘騎攻其門。宋都統趙安出戰,也速答兒三入其軍,再挾猛士以出。大兵四集,斬首五百餘級。趙安開門降,制置使張玨遁,追至涪州擒之。捷聞,帝賜玉帶、鈔五千貫,授西川蒙古軍馬六翼新附軍招討使,遷四川西道宣慰使,加都元帥。



 羅氏鬼國亦奚不薛叛,詔以四川兵會雲南、江南兵討之。至會靈關,亦奚不薛遣先鋒阿麻、阿豆等將數萬眾迎敵,也速答兒馳入其軍,挾阿麻、阿豆出,斬之。亦奚不薛懼,率所部五萬餘戶降。以功拜西川等處行中書省右丞,加賜金帛鞍轡。西南夷雄左、都掌蠻得蘭右叛,詔以兵討降之,改四川等處行樞密副使。冬,烏蒙蠻陰連都掌蠻以叛,詔以兵會雲南行院拜答力進討。也速答兒擒烏蒙蠻,帝賜玉帶、織金服,遷蒙古軍都萬戶,復賜銀鼠裘,鎮唐兀之地。進同知四川等處行樞密院事,仍居鎮。成宗即位,拜四川等處行中書省平章政事。武宗時,由四川遷雲南,加左丞相,仍為平章政事。南征叛蠻,感瘴毒,還至成都卒。



 弟八剌,襲為蒙古軍萬戶。八剌卒,次子拜延襲,拜四川行省左丞;長子南加臺,官至四川行省平章政事。



 ○阿剌罕



 阿剌罕,札剌兒氏。祖撥徹,事太祖,為火而赤,又為博而赤,攻城掠地,數有戰功。太宗即位,仍以其職從征隴北、陜西,身先戰士,死焉。父也柳干,幼隸皇子岳里吉為衛士長。歲乙未,從皇子闊出、忽都禿南征,累功授萬戶,遷天下馬步禁軍都元帥。及大將察罕卒,也柳乾領其職,拜諸翼軍馬都元帥,統大軍攻淮東、西諸郡。戊午,戰死揚州。阿剌罕襲為諸翼蒙古軍馬都元帥。己未,從世祖渡江,至鄂而還。



 世祖即位,從至末黎伯顏孛剌。宗王阿里不哥稱兵內向,阿剌罕以所部軍擊破阿藍帶兒、渾都海之兵於昔門禿,追至河西,以功賜金五十兩。中統三年,李璮叛,據濟南,大軍討之。阿剌罕與璮戰於老倉口,敗之。璮伏誅,授都元帥,賜金虎符、銀印。



 至元四年春,改上萬戶,從都元帥阿術伐宋。九月,師次襄陽西安陽灘,逆戰宋兵,敗之。五年,大軍圍襄樊,阿剌罕守南面百丈山、漫河灘,兵累交,宋不能師。十年春,樊城破,襄陽降。十一年秋,丞相伯顏與阿術會師襄陽,遣阿剌罕率諸翼軍攻郢、復諸州。十月,奪郢州南門堡。丞相伯顏、阿術親率騎兵行視漢陽城壁,欲取漢口渡江。宋人以精兵扼漢口,乃遣阿剌罕帥蒙古騎兵倍道兼行,擊破沙蕪堡,遂入江,取鄂州。阿剌罕同斷事官楊仁風東略壽昌,得米四十萬斛,遂統左翼軍順流東下,沿江州郡悉降,乃撫輯其人民。



 十二年六月,加昭毅大將軍、蒙古漢軍上萬戶,屯駐建康。丞相伯顏受詔赴闕,以阿剌罕留治省事,拜中奉大夫、參知政事。丞相伯顏還軍中,分軍為三道並進。阿剌罕由西道趨溧水、溧陽,攻破銀樹東壩,至護牙山慶豐圬,敗宋軍,斬首七千級,又擒其將祝亮,並裨校七十二人,斬首三千級。又與宋兵戰,斬首七千級,逐其援兵退走數十里。又敗其都統等三人,斬首三千級。破建平縣,殺其守吏。進攻廣德軍獨松關。先是,宋廣德守張濡殺國信使廉希賢、嚴忠範等於獨松關,及阿剌罕軍次安吉州上柏鎮,濡率兵來拒戰,大敗之,斬首二千級,生擒其副將馮翼,戮於軍前。濡遁走,追斬之。



 十三年春,宋以國降,詔阿剌罕同左丞董文炳率高興等,攻浙東溫、臺、衢、婺、處、明、越及閩中諸郡,降其運使、提刑等五百人。追襲宋嗣秀王趙與BT至安福縣,與BT以軍三萬來拒戰,阿剌罕身先士卒,率高興、撒里蠻等渡江,鏖戰四十餘里,斬其步帥觀察使李世達,生擒與BT及其將吏百八十人,悉斬之,獲其銅印五、軍資器仗無算。泉州蒲壽庚降。江南平,以參知政事佩金虎符,行江東宣慰使。十四年,入覲,進資善大夫、行中書省左丞,俄遷右丞,仍宣慰江東。十八年,召拜光祿大夫、中書左丞相、行中書省事,統蒙古軍四十萬征日本,行次慶元,卒於軍中。



 子拜降襲,累遷江浙行中書省平章政事,仍領本軍萬戶。拜降卒,弟也速迭兒襲,由左手蒙古軍萬戶累遷河南江北行省平章政事,兼山東河北蒙古軍大都督。



 ○阿塔海



 阿塔海,遜都思人。祖塔海拔都兒,驍勇善戰,嘗從太祖同飲黑河水,以功為千戶。父卜花襲職,卒。阿塔海魁偉有大度,才略過人。既襲千戶,從大帥兀良合歹征雲南,身先行陣。師還,事世祖於潛邸。



 至元九年,命馳驛督諸軍攻襄陽。襄陽下,第功授鎮國上將軍、淮西行樞密院副使。築正陽東西城。五月霖雨,宋將夏貴乘淮水溢,來爭正陽。阿塔海率眾御之,貴走,追至安豐城下而還。拜中書右丞、行樞密院事。渡江,與丞相伯顏軍合。克池州。十二年,師次建康。宋鎮江攝守石祖忠遣使乞降。揚州守將李庭芝聞之,遣兵突圍出擊。阿塔海率師救之,宋兵望風退走。時真、泰諸城尚為宋守,鎮江地扼襟喉,城壁不固,阿塔海乃立木柵,以保障居民。又分兵屯瓜洲,以絕揚州之援。宋將張世傑、孫虎臣帥舟師陳於江中焦山下,其勢甚張,阿塔海與平章阿術登南岸督諸軍大破之。宋殿帥張彥與平江都統劉師勇襲呂城,遣萬戶懷都擊之,斬彥。十月,並行樞密院於行中書省,仍以阿塔海為右丞。克常州,降平江、嘉興。十三年正月,會兵臨安,宋降,以其幼主、母後入覲。詔復趨瓜洲,與阿術議淮南事宜,淮南平。詳見伯顏、阿術傳。



 十四年,授榮祿大夫、平章政事、行中書省事。十五年二月,召赴闕,拜光祿大夫、行中書省左丞相,移治臨安。二十年,遷征東行省丞相,征日本。遇風,舟壞,喪師十七、八。二十二年,行同知沿江樞密院事。二十三年,行江西中書省事,入朝。二十四年,扈從征乃顏。師還,奉朝請居京師。二十六年十二月卒,年五十六,贈推忠翊運宣力功臣、開府儀同三司、太師、上柱國,追封順昌郡王,謚武敏。



 子阿里麻,江淮行樞密副使,累官至江南諸道行御史臺御史大夫,卒。



 ○唆都百家奴



 唆都,扎剌兒氏。驍勇善戰,入宿衛,從征花馬國有功。李璮叛山東,從諸王哈必赤平之。還,言於朝曰:「郡縣惡少年,多從間道鬻馬於宋境,乞免其罪,籍為兵。」從之,得兵三千人。以千人隸唆都,為千戶,命守蔡州。



 至元五年,阿術等兵圍襄陽,命唆都出巡邏,奪宋金剛臺寨、筲基窩、青澗寨、大洪山、歸州洞諸隘。嘗猝遇宋兵千餘,持羈勒欲竊馬,唆都戰敗之,斬首三百級。六年,宋將範文虎率舟師駐灌子灘,丞相史天澤命唆都拒卻之。升總管,分東平卒八百隸之。九年,攻樊城,唆都先登,城遂破。襄陽降,再與卒五千,賜弓矢、襲衣、金鞍、白金等物。入見,升郢復等處招討使。十一年,移戍郢州之高港,敗宋師,斬首三百級,獲裨校九人。從大軍濟江,鄂、漢降。



 十二年,建康降,參政塔出命唆都入城招集,改建康安撫使。攻平江、嘉興,皆下之。帥舟師會伯顏於阜亭山。宋平,詔伯顏以宋主入朝,留參政董文炳守臨安,令其自擇可副者,文炳請留唆都,從之。時衢、婺諸州皆復起兵,文炳謂唆都曰:「嚴州不守,臨安必危,公往鎮之。」至嚴方十日,衢、婺、徽連兵來攻,唆都戰卻之,獲章知府等二十二人。復婺州,敗宋將陳路鈐於梅嶺下,斬首三千級。又復龍游縣。攻衢州,衢守備甚嚴,唆都親率諸軍鼓噪登城,拔之,宋丞相留夢炎降。攻處州,斬首七百級。又攻建寧府松溪縣、懷安縣,皆下之。



 十四年,升福建道宣慰使,行征南元帥府事,聽參政塔出節制。塔出令唆都取道泉州,泛海會於廣州之富場。將行,信州守臣來求援曰:「元帥不來,信不可守。今邵武方聚兵觀釁,元帥旦往,邵武兵夕至矣。」唆都告於眾曰:「若邵武不下,則腹背受敵,豈獨信不可守乎!」乃遣周萬戶等往招降之。唆都趨建寧,遇宋兵於崇安,軍容甚盛。令其子百家奴及楊庭璧等數隊夾擊之,範萬戶以三百人伏祝公橋,移剌答以四百人伏北門外。庭璧陷陣深入,宋兵敗走,伏兵起,邀擊之,斬首千餘級。宋丞相文天祥、南劍州都督張清合兵將襲建寧,唆都夜設伏敗之。轉戰至南劍,敗張清,奪其城。至福州,王積翁以城降。攻興化軍,知軍陳瓚乞降,復閉城拒守。唆都臨城諭之,矢石雨下。乃造雲梯砲石,攻破其城。巷戰終日,斬首三萬餘級,獲瓚,支解以徇。至漳州,漳州亦拒守,先遣百家奴往會塔出,留攻之,斬首數千級,知府何清降。攻潮州,知府馬發不降,唆都恐失富場之期,乃舍之而去。十五年,至廣州,塔出令還攻潮。發城守益備,唆都塞塹填濠,造雲梯、鵝車,日夜急攻。發潛遣人焚之,二十餘日不能下,唆都令於眾曰:「有能先登者拜爵,已仕者增秩。」總管兀良哈耳先登,諸將繼之,戰至夕,宋兵潰,潮州平。進參知政事,行省福州。徵入見,帝以江南既定,將有事於海外,升左丞,行省泉州,招諭南夷諸國。十八年,改右丞,行省占城。



 十九年,率戰船千艘,出廣州,浮海伐占城。占城迎戰,兵號二十萬。唆都率敢死士擊之,斬首並溺死者五萬餘人。又敗之於大浪湖,斬首六萬級。占城降,唆都造木為城,闢田以耕。伐烏里、越里諸小夷,皆下之,積穀十五萬以給軍。二十一年,鎮南王脫歡征交趾,詔唆都帥師來會,敗交趾兵於清化府,奪義安關,降其臣彰憲、昭顯。脫歡命唆都屯天長以就食,與大營相距二百餘里。俄有旨班師,脫歡引兵還,唆都不知也。交趾使人告之,弗信,及至大營,則空矣。交趾遮之於乾滿江,唆都戰死。事聞,贈榮祿大夫,謚襄愍。子百家奴。



 百家奴至元五年從元帥阿術攻襄陽,築新城,數立功。七年,以質子從郡王合達,敗宋兵於灌子灘。八年夏四月,宋殿帥範文虎等督促糧運,輸之襄陽,晝夜不絕。百家奴乘戰船順流至鹿門山,欲塞宋糧道,出擊範文虎軍,累獲戰功,於是河南行省命為管軍總把。後隸丞相伯顏麾下,擢為知印。從攻鄂州,宋都統趙五帥諸軍來迎戰,百家奴深入卻敵,身被數瘡。攻沙洋,立雲梯於東角樓,登城力戰,破之,奪其旗幟、弓矢、衣甲。攻新城,先登,拔之,宋將王安撫棄城宵遁。伯顏以百家奴前後戰功上聞,世祖大悅,曰:「此人之名,朕心不忘,兵還時大用之,朕不食言也。今且以良家女及銀碗一賜之,以為左驗。」



 從圍漢陽,自沙武口曳船入江。宋制置夏貴來迎戰,百家奴與暗答孫突入敵陣擊之,宋兵奔潰,遂登江南岸,獲其戰船、器甲甚多。轉戰至黃州,會日暮,追擊夏貴至白虎山,夜分乃還。未幾,復攻破金牛壩。十二年春正月,與千戶薛赤乾取雞籠洞,還至瑞昌縣,遇夏貴潰兵,復擊敗之。是時,宋遣兵救瑞昌,未至而縣已下矣。復擊宋救兵,得宋所執北兵五人來歸。圍江州,宋安撫呂師夔以城降。東定池州,擊宋平章賈似道及孫虎臣於丁家洲,追逐百里餘,奪戰船五艘及旗幟器甲,擒宋統制王文虎,因定黃池。略地宣州,百家奴為前鋒,與敵兵戰喃呢湖,敗之,奪其戰船三百艘。太平州亦望風款附。其父唆都因說下建康。於是伯顏令謁只裏論諸將功。遂賞百家奴銀二錠以旌之,仍命為管軍總把。俄從伯顏入朝,加進義校尉,賜銀符,為管軍總把。攻丹陽、呂城,破常州,皆有功。至蘇州,宋守臣王安撫以城降。秀州、湖州皆不煩兵而下。諸軍乘勝直趨臨安,宋主出降。十三年,領新附軍守鎮江。未幾,復從平章博魯歡攻泰、壽二州,中瘡,遂罷攻。後數日,與萬戶葉了虔將兵攻泰州新城,百家奴力疾先登,破之,復被兩瘡。已而從阿術攻下揚州諸郡,得宋制置李庭芝、都統姜才,以功升武略將軍,賜金符,為管軍總管,鎮高郵白馬湖。是時,行省以百家奴襲父唆都郢復州招討使、建康宣撫使,仍領本翼軍。



 頃之,徇地福建,行定衢、婺、信等州城邑。至新安縣,擊斬宋趙監軍、詹知縣,擒江通判。道與畬軍遇,疾戰敗之。鼓行而東,沈安撫以建寧府降。攻陷南劍州,張清、聶文慶遁去。閩清、懷安二縣傳檄而定。至福州,諭以威德,王安撫率眾出降。攻破興化,擒陳安撫及白牒都統。別擊東華鄉。張世傑軍於泉州,俄領諸軍乘戰船入海,追逐張世傑於惠州甲子門。進至同安縣答關寨,瀕海縣鎮悉招諭下之。白望丹、五虎陳以戰船三千餘艘來降。冬十二月,宋二王遣倪宙奉表詣軍門降,遂進兵至廣州,諸郡縣以次降附。明年春正月,振旅而還,復攻下德勝等寨。至蒲仙江,聶文慶復敗走。攻潮州,破之,誅馬發等數人,廣東遂平。三月,引宙奉降表來朝,未至,授昭勇大將軍,賜虎符,管軍萬戶。七月,遂朝於上都,升鎮國上將軍、海外諸蕃宣慰使,兼福建道市舶提舉,仍領本翼軍守福建,俄兼福建道長司宣慰使都元帥。是時,福建多水災,百家奴出私錢市米以賑,貧民全活者甚眾。十七年,朝京師,加正奉大夫、宣慰使、都元帥。



 二十二年,從父唆都征交趾,唆都力戰死之,百家奴遂與脫歡引兵薄交趾境,水陸轉戰,戰輒有功。二十五年,驛召至南京宣慰司,命括五路民馬。二十七年,除建康路總管。武宗即位,遷鎮江路總管。至大四年,金瘡發,卒於家。



 ○李恆



 李恆,字德卿,其先姓於彌氏,唐末賜姓李,世為西夏國主。太祖經略河西,有守兀納剌城者,夏主之子也,城陷不屈而死。子惟忠,方七歲,求從父死,主將異之,執以獻宗王合撒兒,王留養之。及嗣王移相哥立,惟忠從經略中原,有功。淄川王分地,以惟忠為達魯花赤,佩金符。惟忠生恆,恆生有異質,王妃撫之猶己子。中統三年,命恆為尚書斷事官,恆以讓其兄。李璮反漣海,恆從其父棄家入告變,璮怒,系恆闔門獄中。璮誅,得出。世祖嘉其功,授淄萊路奧魯總管,佩金符,並償其所失家資。



 至元七年,改宣武將軍、益都淄萊新軍萬戶,從伐宋。襄陽守將呂文煥時出拒敵,殿帥範文虎復援之。恆率本軍築堡萬山扼城西,絕其陸路。文煥等又以漁舟渡漢水窺伺軍形,恆設伏敗之,水路亦絕,遂進攻樊城。十年春,恆以精兵渡漢,自南面先登,樊城破,襄陽亦降。捷聞,帝賜以寶刀,遷明威將軍,佩金虎符。十一年,丞相伯顏大會師襄陽,進至郢州。宋以舟師截漢水,伯顏由唐港入漢,舍郢而進攻沙洋、新城,留恆為後拒,敗其追兵。至陽羅堡,宋制置夏貴遣其子松來逆戰,恆先陷陣,額中流矢,伯顏止之,恆戰益力,卒射松殺之。諸軍渡江,恆與宋兵戰,自寅至申,夏貴敗走,鄂州、漢陽俱下。以功遷宣威將軍,賜白金五百兩。遂從伯顏東下。



 十二年春,宋將高世傑復窺漢、沔,乃遣恆還守鄂州。時豪民聚眾侵江陵,省命恆往討之,恆斂兵不動,但諭使出降,得生口十餘萬,悉縱為民;仍禁軍毋得虜掠,饋獻充積一無所受。十二年,從右丞阿里海牙至洞庭,擒高世傑。下嶽州,進攻沙市,拔之。宋制置高達以江陵降,留恆鎮守。傳檄歸、峽、辰、沅、靖、澧、常德諸州,皆下。未幾,徙鎮常德,以扼湖南之沖。俄有詔分三道出師,以恆為左副都元帥,從都元帥遜都臺出江西。九月,開府於江州。師次建昌縣,擒都統熊飛。遂圍隆興,轉運使劉盤請降,恆察其詐,密為之備。盤果以銳兵空至,恆擊敗之,殺獲殆盡,盤乃降。下撫、瑞、建昌、臨江。軍中有得宋相文天祥與建昌故吏民書,恆焚之,人心乃安。進攻吉州,知州周天驥降,遂定贛、南安。廣東經略徐直諒奉蠟書納其所部十四郡,前江西制置黃萬石亦以邵武降。隆興帥府誣富民與敵連,已誅百三十家,恆還,審其非罪,盡釋之。



 宋丞相陳宜中及其大將張世傑立益王鸑於閩中,郡縣豪傑爭起兵應之。恆遣將破吳浚兵於南豐。世傑遣都統張文虎與浚合兵十萬,期必復建昌。恆復遣將敗之兜港。浚走從文天祥於瑞金,又破之,天祥走汀州。遣鎮撫孔遵追之,並破趙孟瀯軍,取汀州。元帥府罷,授昭勇大將軍、同知江西宣慰司事,加鎮國上將軍,遷福建宣慰使,改江西宣慰使。天祥復取汀州,兵出興國縣,連破諸邑,圍贛州尤急。或言天祥墳墓在吉州者,若遣兵發之,則必下矣。恆曰:「王師討不服耳,豈有發人墳墓之理!」乃分兵援贛,自率精兵潛至興國。天祥走,追至空坑,獲其妻女,擒招討使趙時賞已下二十餘人,降其眾二十萬。有旨令與右丞阿里罕、左丞董文炳合兵追益王。眾議所向,皆謂宜趨福建,恆曰:「不可。若諸軍俱在福建,彼必竄廣東,則梅嶺、江西非我有矣,宜從廣東夾攻之。」眾以為然。兵至梅嶺,果與宋兵遇,出其不意敗之,乃遁走岡州。十四年,拜參知政事,行省江西。



 十五年,益王殂,其樞密張世傑、陸秀夫等復立衛王昺,守廣東諸郡,詔以恆為蒙古漢軍都元帥經略之。恆進兵取英德府、清遠縣,敗其制置凌震、運使王道夫,遂入廣州,世傑等移屯崖山。時都元帥張弘範舟師未至,恆按兵不動,分遣諸將略定梅、循諸州。凌震等復抵廣州,恆擊敗之,皆棄舟走,赴水死,奪其船三百艘,擒將吏宋邁以下二百餘人,又破其餘軍於茭塘越。十六年二月,弘範至自漳州,直指崖山,恆率所部赴之。張世傑集海艦千餘艘,貫以巨索,為柵以自固。恆遣斷其汲路,其勢日迫,諭降不可,乃陣於船尾,由北面逆行,搗其柵。索絕,世傑猶死戰,自朝至晡,弘範督南面諸軍合擊,大敗之。陸秀夫先沉妻子於海,乃抱衛王赴海死。從死者十餘萬人。獲其金璽、後宮及文武之臣。其大將翟國秀、凌震等皆解甲降。焚溺之餘,尚得八百餘艘。是日,黑氣如霧,有乘舟南遁者,恆以為衛王,追至高、化,詢之降人,始知衛王已死,遁者乃世傑也。世傑繼亦溺死於海陵港。嶺海悉平,功成入覲,帝賞勞甚厚,將士預賜宴者二百餘人。



 十七年,拜資善大夫、中書左丞,行省荊湖。掠民為奴婢者,禁之;常德、澧、辰、沅、靖五郡之饑者,賑之;獵戶之籍於官者,奏請一千戶之外,悉放散之。



 十九年,乞解軍職,乃命其長子同知江西宣慰司事散木[A156]襲為本軍萬戶。占城之役,恆奉旨給其糧餉器械、海艦百艘,久留瘴鄉,冒疾而還。俄有詔命恆從皇子鎮南王征交趾,結筏渡海,奪天長府。交趾遂空其國,航海而遁。恆封其宮庭府庫,追襲於海洋,敗之,得船二百艘,幾獲其世子。會盛夏,軍中疾作,霖潦暴漲,浸濯營地。議者謂交趾且降,請班師,恆弗能奪,遂還。蠻兵追敗後軍,王乃改命恆殿後,且戰且行。毒矢貫恆膝,一卒負恆而趨。至思明州,毒發,卒,年五十。後贈銀青榮祿大夫、平章政事,謚武愍;再贈推忠靖遠功臣、太保、儀同三司,追封滕國公。



 子散木鷿,江西行省平章政事;囊加真,益都淄萊萬戶;遜都臺,同知湖南宣慰使司事。孫薛徹干,兵部侍郎;薛徹禿,益都般陽萬戶。



\end{pinyinscope}