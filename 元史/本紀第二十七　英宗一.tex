\article{本紀第二十七 英宗一}

\begin{pinyinscope}

 英宗睿聖文孝皇帝,諱碩德八剌,仁宗嫡子也。母莊懿慈聖皇后,弘吉剌氏,以大德七年二月甲子生。仁宗欲立為太子分析了王權、行政權、立法權、君主制和民主制、君主主權,帝入謁太后,固辭,曰:「臣幼無能,且有兄在,宜立兄,以臣輔之。」太后不許。延祐三年十二月丁亥,立為皇太子,授金寶,開府置官屬。監察御史段輔、太子詹事郭貫等,首請近賢人,擇師傅,帝嘉納之。六年十月戊午,受玉冊,詔命百司庶務必先啟太子,然後奏聞。帝謂中書省臣曰:「至尊委我以天下事,日夜寅畏,惟恐弗堪。卿等亦當洗心滌慮,恪勤乃職,勿有隳壞,以貽君父憂。」



 七年春正月戊戌,仁宗不豫,帝憂形於色,夜則焚香,泣曰:「至尊以仁慈御天下,庶績順成,四海清晏。今天降大厲,不如罰殛我身,使至尊永為民主。」辛丑,仁宗崩,帝哀毀過禮,素服寢於地,日歠一粥。癸卯,太陰犯鬥。甲辰,太子太師鐵木迭兒以太后命為右丞相。丙午,遣使分讞內外刑獄。戊申,賑通、漷二州蒙古貧民,汰知樞密院事四員。禁巫、祝、日者交通宗戚、大官。



 二月壬子,罷造永福寺。賑大同、豐州諸驛饑。以江浙行省左丞相黑驢為中書平章政事。丁巳,修佛事。戊午,祭社稷。建御容殿於永福寺。汰富民竄名宿衛者,給役蒙古諸驛。己未,命儲糧於宣德、開平、和林諸倉,以備賑貸供億。復以都水監隸中書。辛丑,太陰犯軒轅御女。平章政事赤斤鐵木兒、御史大夫脫歡罷為集賢大學士。壬戌,太陰犯靈臺。甲子,鐵木迭兒、阿散請捕逮四川行省平章政事趙世延赴京。參議中書省事乞失監坐鬻官,刑部以法當杖,太后命笞之,帝曰:「不可。法者天下之公,徇私而輕重之,非示天下以公也。」卒正其罪。丙寅,以陜西行省平章政事趙世榮為中書平章政事,江西行省右丞木八剌為中書右丞,參知政事張思明為中書左丞,中書左丞換住罷為嶺北行省右丞。丁卯,太陰犯日星。白云宗總攝沈明仁為不法坐罪,詔籍江南冒為白雲僧者為民。己巳,修鎮雷佛事於京城四門,罷上都乾元寺規運總管府。庚午,太陰犯鬥。辛未,括民間系官山場、河泊、窯冶、廬舍。壬申,召陜西行臺御史大夫答失鐵木兒赴闕。以遼陽、大同、上都、甘肅官牧羊馬牛駝給朔方民戶,仍給曠地屯種。癸酉,括勘崇祥院地,其冒以官地獻者追其直,以民地獻者歸其主。決開平重囚。丙子,定京城環衛更番法,準五衛漢軍歲例。丁丑,奪前中書平章政事李孟所受秦國公制命,仍僕其先墓碑。戊寅,中書平章政事兀伯都剌罷為甘肅行省平章政事,阿禮海牙罷為湖廣行省平章政事。鐵木迭兒以前御史中丞楊朵兒只、中書平章政事蕭拜住違太后旨,矯命殺之,並籍其家。徽政院使失列門,以太后命請更朝官,帝曰:「此豈除官時耶?且先帝舊臣,豈宜輕動。俟予即位,議於宗親、元老,賢者任之,邪者黜之可也。」司農卿完者不花言:「先帝以土田頒賜諸臣者,宜悉歸之官。」帝問曰:「所賜為誰?」對曰:「左丞相阿散所得為多。」帝曰:「予常諭卿等,當以公心輔弼。卿於先朝嘗請海泊之稅,以阿散奏而止。今卿所言,乃復私憾耳,非公議也,豈輔弼之道耶?」遂出完者不花為湖南宣慰使。奪僧輦真吃剌思等所受司徒、國公制,仍銷其印。



 三月辛巳,以中書禮部領教坊司。壬午,賑陳州、嘉定州饑。瓜哇遣使入貢。戊子,太陰犯酒旗上星,熒惑犯進賢。徵諸王、駙馬流竄者,給侍從,遣就分邑。庚寅,帝即位,詔曰:



 洪惟太祖皇帝膺期撫運,肇開帝業;世祖皇帝神機睿略,統一四海。以聖繼聖,迨我先皇帝,至仁厚德,涵濡群生,君臨萬國,十年於茲。以社稷之遠圖,定天下之大本,葉謀宗親,授予冊寶。方春宮之與政,遽昭考之賓天。諸王貴戚,元勛碩輔,咸謂朕宜體先帝付托之重,皇太后擁護之慈,既深系於人心,詎可虛於神器,合辭勸進,誠意交孚。乃於三月十一日,即皇帝位於大明殿。可赦天下。



 尊太后為太皇太后。是夜,太陰犯明堂。壬辰,太皇太后受百官朝賀於興聖宮。鐵木迭兒進開府儀同三司、上柱國、太師。敕群臣超授散官者,朝會毋越班次。賜諸王也孫鐵木兒、脫脫那顏等金銀、幣帛有差。賑寧夏路軍民饑。甲午,作佛事於寶慈殿,賑木憐、渾都兒等十一驛饑。乙未,日明暈若連環。丙申,斡羅思等內附,賜鈔萬四千貫,遣還其部。遣知樞密事也兒吉尼檢核鞏昌等路屯戍,選甘州戍卒。戊戌,汰上都留守司留守五員,定吏員秩止從七品如前制。庚子,降太常禮儀院、通政院、都護府、崇福司,並從二品;蒙古國子監、都水監、尚乘寺、光祿寺,並從三品;給事中、闌遺監、尚舍寺、司天監,並正四品;其官遞降一等有差,七品以下不降。賜邊戍諸王、駙馬及將校士卒金銀、幣帛有差。市羊五十萬、馬十萬,贍北邊貧乏者。辛丑,禁擅奏璽書。以樞密院兼領左、右衛率府。壬寅,降前中書平章政事李孟為集賢侍講學士,悉奪前所受制命。御史臺臣請降詔諭百司以肅臺綱,帝曰:「卿等但守職盡言,善則朕當服行,否亦不汝罪也。」甲辰,詔中外毋沮議鐵木迭兒。敕罷醫、卜、工匠任子,其藝精絕者擇用之。丙午,有事於南郊,告即位。丁未,罷崇祥院,以民匠都總管府隸將作院。



 夏四月庚戌,有事於太廟,告即位。追奪佛速司徒官。罷少府監,復儀鳳、教坊、廣惠諸司品秩。罷行中書省丞相,河南行省丞相也先鐵木兒、湖廣行省丞相朵兒只的斤、遼陽行省丞相,並降為本省平章政事,惟征東行省丞相高麗王不降。賜諸王鐵木兒不花鈔萬五千貫。甲寅,太白犯填星。乙卯,復國子監、都水監,秩正三品。罷回回國子監、行通政院。封諸王徹徹禿為寧遠王。申詔京師勢家與民均役。那懷、渾都兒驛戶饑,賑之。戊午,祀社稷,告即位。己未,紹慶路洞蠻為寇,命四川行省捕之。祭遁甲神於香山。命平章政事王毅等征理在京諸倉庫糧帛虧額,申嚴和林酒禁。庚申,降百官越階者,並依所受之職。以太常禮儀院使拜住為中書平章政事;以西僧牙八的裡為元永延教三藏法師,授金印。壬戌,太陰犯房。以即位,賞宿衛軍。括馬三萬匹,給蒙古流民,遣還其部。給通、漷二州蒙古戶夏布。鐵木迭兒請參決政務,禁諸臣毋隔越擅奏,從之。乙丑,仁宗喪卒哭,作佛事七日。戊辰,車駕幸上都。海運至直沽,調兵千人防戍。封王煦為雞林郡公。議祔仁宗,以陰陽拘忌,權結彩殿於太室東南,以奉神主。己巳,河間、真定、濟南等處蒙古軍饑,賑之。罷市舶司,禁賈人下番。課回回散居郡縣者,戶歲輸包銀二兩。增兩淮、荊湖、江南東西道田賦,斗加二升。賑大都、凈州等處流民,給糧馬,遣還北邊。戊寅,以蒙古、漢人驛傳復隸通政院。有獻七寶帶者,因近臣以進,帝曰:「朕登大位,不聞卿等薦賢而為人進帶,是以利誘朕也,其還之。」是月,左衛屯田旱、蝗,左翊屯田蟲食麥苗,亳州水。



 五月己卯,禁僧馳驛,仍收元給璽書。庚辰,上都留守賀伯顏坐便服迎詔棄市,籍其家。辛巳,汝寧府霖雨傷麥禾,發粟五千石賑糶之。丁亥,罷沅陵縣浦口千戶所。己丑,中書省臣請禁擅奏除拜,帝曰:「然恐朕遺忘,或乘間奏請,濫賜名爵,汝等當復以聞。」復置稱海、五條河屯田。命僧禱雨。大同雲內、豐、勝諸郡縣饑,發粟萬三千石貸之。左丞相阿散罷為嶺北行省平章政事,以拜住為中書左丞相,乃剌忽、塔失海牙並為中書平章政事,只兒哈郎為中書參知政事。庚寅,太陰犯心。辛卯,參知政事欽察罷為集賢學士。賑上都城門及駐冬衛士。遣使榷廣東番貨,弛陜西酒禁。壬辰,和林民閻海瘞殍死者三千餘人,旌其門。癸巳,太陰犯天狗。甲午,沈陽軍民饑,給鈔萬二千五百貫賑之。乙未,請大行皇帝謚於南郊。丙申,太白犯畢。禁宗戚權貴避徭役及作奸犯科。戊戌,有告嶺北行省平章政事阿散、中書平章政事黑驢及御史大夫脫忒哈、徽政使失列門等與故要束謀妻亦列失八謀廢立,拜住請鞫狀,帝曰:「彼若借太皇太后為詞,奈何?」命悉誅之,籍其家。追封隴西公汪世顯為隴右王。辛丑,以知樞密院事鐵木兒脫為中書平章政事。壬寅,監察御史請罷僧、道、工、伶濫爵及建寺、豢獸之費。甲辰,以誅阿散、黑驢、賀伯顏等詔天下。敕百司日勤政務,怠者罪之。丙午,御史劉恆請興義倉及奪僧、道官。敕捕亦列失八子江浙行省平章政事買驢,仍籍其家。丁未,封王禪為雲南王,往鎮其地。饒州番陽縣進嘉禾,一莖六穗。以賀伯顏、失列門、阿散家貲、田宅賜鐵木迭兒等。



 六月己酉,流徽政院使米薛迷於金剛山。以脫忒哈、失列門故奪人畜產歸其主。甲寅,前太子詹事床兀兒伏誅。京師疫,修佛事於萬壽山。乙卯,昌王阿失部饑,賜鈔千萬貫賑之。賞誅阿散等功,賜拜住以下金銀、鈔有差。丙辰,召河南行省平章政事也仙帖木兒至京師,收脫忒哈廣平王印。丁巳,以江西行省左丞相脫脫為御史大夫,宗正扎魯火赤鐵木兒不花知樞密院事。戊午,罷徽政院。廣東採珠提舉司罷,以有司領其事。封知樞密院事塔失鐵木兒為薊國公。己未,定邊地盜孳畜罪犯者,令給各部力役,如不悛,斷罪如內地法。庚申,太陰犯鬥。賜角牴百二十人鈔各千貫。辛酉,詔免僧人雜役。壬戌,敕諸使至京者,大事五日、小事三日遣還。是夜,月食既。癸亥,太陰犯壘壁陣。乙丑,賑北邊饑民,有妻子者鈔千五百貫,孤獨者七百五十貫。新作太祖幄殿。西番盜洛各目降。丁卯,太白犯井。賜諸王阿木里臺宴服、珠帽。戊辰,賑雷家驛戶鈔萬五千貫。辛未,太陰犯昴。甲戌,賜北邊諸王伯要臺等十人鈔各二萬五千貫。邊民賑米三月。修寧夏欽察魯佛事,給鈔二百一十二萬貫。丁丑,改紅城中都威衛為忠翊侍衛親軍都指揮使司,隸樞密院。罷章慶司、延福司、群牧監、宮正司、遼陽萬戶府,復徽儀司為繕珍司,善政司為都總管府,內宰司、延慶司、甄用監復為正三品。益都蝗,荊門州旱,棣州、高郵、江陵水。



 秋七月戊寅,賜諸王曲魯不花鈔萬五千貫。命玄教宗師張留孫修醮事於崇真宮。壬午,立普定路屯田,分烏撒、烏蒙屯田卒二千赴之。運和林糧於扎昆倉,以便邊軍,市馬三萬、羊四萬給邊軍貧乏者。癸未,括馬於大同、興和、冀寧三路,以頒衛士。甲申,車駕將北幸,調左右翊軍赴北邊浚井。以知樞院事買驢、哈丹並為遼陽行省平章政事。丙戌,賜諸王買奴等鈔二十五萬貫。丁亥,太陰犯鬥。諸王告住等部火,賑糧三月、鈔萬五千貫。晉王也孫鐵木兒部饑,賑鈔五千萬貫。壬辰,罷女直萬戶府及狗站脫脫禾孫,散遼陽紅花萬戶府兵。遣扈從諸營還大都,禁踐民禾。安南內附人陳巖言其國貢使多為覘伺,敕湖廣行省汰遣之。乙未,賜西僧沙加鈔萬五千貫,以甘肅行省平章欽察臺知樞密院事。回回太醫進藥曰打里牙,給鈔十五萬貫。丙申,以昌平、灤陽十二驛供億繁重,給鈔三十萬貫賑之。中書平章政事乃剌忽罷。降封安王兀都不花為順陽王。禁獻珍寶制袞冕。戊戌,熒惑犯房。樞密院臣言:「塔海萬戶部不剌兀赤與北兵戰,拔軍士三百人以還,棄其子於野,殺所乘馬以啖士卒,請賞之。」賜鈔五千貫。斡魯思辰告諸王月兒魯鐵木兒謀變,賞鈔萬五千貫,敕中外希賞自請者勿予。己亥,太陰犯昴。賜女巫伯牙臺鈔萬五千貫。庚子,以江南行御史臺中丞廉恂為中書平章政事。辛丑,賜公主扎牙八剌等鈔七萬五千貫。晉王也孫鐵木兒遣使以地七千。頃歸朝廷,請有司徵其租,歲給糧鈔,從之。以遼陽金銀鐵冶歸中政院。癸卯,賜伶人鈔二萬五千貫,酒人十五萬貫。己巳,以知樞密院事也兒吉尼為江西行省平章政事。是月,後衛屯田及潁、息、汝陽、上蔡等縣水,霸州及堂邑縣蝻。



 八月丁未朔,嶺北省臣忻都坐以官錢犒軍免官,詔復其職。戊申,祭社稷。罷曲靖路人匠提舉司。賑晉王部軍民鈔二百五十萬貫。翽星於司天監。辛亥,賑龍居河諸軍。乙卯,賜上都駐冬衛士鈔四百萬貫。諸王木南即部饑,興聖宮牧駝戶貧乏,並賑之。丙辰,祔仁宗聖文欽孝皇帝、莊懿慈聖皇后於太廟,鐵木迭兒攝太尉,奉玉冊行事。太白犯靈臺。戊午,鐵木迭兒以趙世延嘗劾其奸,誣以不敬下獄,請殺之,並究省、臺諸臣,不允。帝幸涼亭,從容謂近侍曰:「頃鐵木迭兒必欲置趙世延於死地,朕素聞其忠良,故每奏不納。」左右咸稱萬歲。乙丑,熒惑犯天江。丁卯,太白犯太微垣右執法。宮人官奴,坐用日者請太皇太后翽星,杖之,籍其資。脫思馬部宣慰使亦憐真坐違制不發兵,杖流奴兒乾之地。庚午,發米十萬石賑糶京師貧民。壬申,太陰犯軒轅御女。甲戌,廣東新州饑,賑之,河間路水。



 九月甲申,建壽安山寺,給鈔千萬貫。括興和馬以贍北部貧民,禁五臺山樵採。罷上都、嶺北、甘肅、河南諸郡酒禁。乙酉,太陰犯壘壁陣。丙戌,熒惑犯鬥。壬辰,敕議玉華宮歲享睿宗登歌大樂。土番利族、阿俄等五種寇成穀,遣鞏昌總帥以兵討之。循州溪蠻秦元吉為寇,遣守將捕之。癸巳,太陰犯昴。沈陽水旱害稼,弛其山場河泊之禁。戊戌,太陰犯鬼。己亥,太白犯亢。庚子,常澧州洞蠻貞公合諸洞為寇,命土官追捕之。癸卯,親王脫不花、搠思班遣使來賀登極。甲辰,雲南木邦路土官紿邦子忙兀等入貢,賜幣有差。遣馬扎蠻等使占城、占臘、龍牙門,索馴象。以廩藏不充,停諸王所部歲給。



 冬十月丁未,時享太廟。庚戌,太陰犯熒惑於斗。將作院使也速坐董制珠衣怠工,杖之,籍其家。壬子,作佛事於文德殿四十日。申嚴兩淮鹽禁。丁巳,酉陽聳儂洞蠻田謀遠為寇,命守臣招捕之。戊午,車駕至自上都。詔太常院臣曰:「朕將以四時躬祀太室,宜與群臣集議其禮。此追遠報本之道,毋以朕勞於對越而有所損,其悉遵典禮。」安南國遣其臣鄧恭儉來貢方物。庚申,敕譯佛書。辛酉,賜勞探馬赤宿衛者,遣還所部。癸亥,太陰犯井。乙丑,幸大護國仁王寺,帝師請以醮八兒監藏為土番宣慰司都元帥,從之。酉陽土官冉世昌遣其子冉朝率大、小石堤洞蠻入貢。丙寅,定恭謝太廟儀式。丁卯,為皇后作鹿頂殿於上都。己巳,罷玉華宮祀睿宗登歌樂。敕翰林院譯詔,關白中書。庚午,命拜住督造壽安山寺。癸酉,流諸王阿刾鐵木兒於雲南。



 十一月丙子朔,帝齋齊宮。丁丑,恭謝太廟,至仁宗太室,即流涕,左右感動。戊寅,以海運不給,命江浙行省以財賦府租益之,還其直,歸宣徽、中政二院。檢勘沙、凈二州流民,勒還本部。以登極,大賚諸王、百官,中書會其數,計金五千兩、銀七十八萬兩、鈔百二十一萬一千貫、幣五萬七千三百六十四匹、帛四萬九千三百二十二匹、木綿九萬二千六百七十二匹、布二萬三千三百九十八匹、衣八百五十九襲,鞍勒、弓矢有差。給嶺北驛牛馬。造今年鈔本,至元鈔五千萬貫、中統鈔二百五十萬貫。汰衛士冒受歲賜者。庚辰,並永平路灤邑縣於石城。遣定住等括順陽王兀都思不花邸財物,入章佩監、中政院。禁京城諸寺邸舍匿商稅。辛巳,以親祀太廟禮成,御大明殿受朝賀。甲申,敕翰林國史院纂修《仁宗實錄》。丁亥,作佛事於光天殿。戊子,幸隆福宮。己丑,宣德蒙古驛饑,命通政院賑之。丁酉,詔各郡建帝師八思巴殿,其制視孔子廟有加。戊戌,交趾蠻儂志德寇脫零那乞等六洞,命守將討之。遣使閱實各行省戍兵。己亥,計京官俸鈔,給米三分。癸卯,熒惑犯壘壁陣。甲辰,鐵木迭兒言:「和市織幣薄惡,由董事者不謹,請免右丞高昉等官,仍令郡縣更造,徵其元直。」不允。太常禮儀院擬進時享太廟儀式。十二月乙巳朔,詔曰:「朕祗遹眙謀,獲承丕緒,念付托之惟重,顧繼述之敢忘。爰以延祐七年十一月丙子,被服袞冕,恭謝於太廟。既大禮之告成,宜普天之均慶。屬茲逾歲,用易紀元,於以導天地之至和,於以法春秋之謹始,可以明年為至治元年。減天下租賦二分,包銀五分。免大都、上都、興和三路差稅三年。優復煮鹽、煉鐵等戶二年。開燕南、山東河泊之禁,聽民採取。命官家屬流落邊遠者,有司資給遣之;其子女典鬻於人者,聽還其家。監察御史、廉訪司歲舉可任守令者二人。七品以上官,有偉畫長策可以濟世安民者,實封上之。士有隱居行義,明治體,不求聞達者,有司具狀以聞。」丁未,播州蜒蠻的羊籠等來降。庚戌,鑄銅為佛像,置玉德殿。壬子,賜壽寧公主鈔七萬五千貫。癸丑,以天壽節,預遣使修醮於龍虎山。乙卯,率百官奉玉冊、玉寶,加上太皇太后尊號曰儀天興聖慈仁昭懿壽元全德泰寧福慶徽文崇祐太皇太后。翰林學士忽都魯都兒迷失譯進宋儒真德秀《大學衍義》,帝曰:「修身治國,無逾此書。」賜鈔五萬貫。河南饑,帝問其故,群臣莫能對,帝曰:「良由朕治道未洽,卿等又不盡心乃職,委任失人,致陰陽失和,災害薦至。自今各務勤恪,以應天心,毋使吾民重困。」太陰掩昴。丙辰,以太皇太后加號禮成,御大明殿受朝賀。丁巳,詔諭中外。戊午,太陰犯井。庚申,太陰犯鬼。辛酉,作延春閣後殿。壬戌,召西僧輦真哈剌思赴京師,敕所過郡縣肅迎。乙丑,翽星於回回司天監四十晝夜。丙寅,以典瑞院使闊徹伯知樞密院事。修秘密佛事於延春閣。丁卯,鐵木迭兒、拜住言:「比者詔內外言得失,今上封事者,或直進御前。乞令臣等開視,乃入奏聞。」帝曰:「言事者直至朕前可也,如細民輒訴訟者則禁之。」給武宗皇后鈔七十五萬貫。以《大學衍義》印本頒賜群臣。戊辰,以太皇太后加號禮成,告太廟。己巳,敕罷明年二月八日迎佛。中書右丞木八剌罷為江西行省右丞,以中書參知政事只兒哈郎為右丞,江南浙西道廉訪使薛處敬為中書參知政事。遣使閱奉元路軍需庫。辛未,拜住進鹵簿圖,帝以唐制用萬二千三百人耗財,乃定大駕為三千二百人,法駕二千五百人。上思州徭結交趾寇忠州。癸酉,帝聞賀伯顏母老,憫之,以所籍京兆田磑還其家。江浙行省平章政事伯顏察兒、江西行省平章政事白撒都並坐貪墨免官。是歲,決獄輕重七千六百三十事。河決汴梁原武,浸灌諸縣;滹沱決文安、大城等縣;渾河溢,壞民田廬。秦州成紀縣暴雨,山崩,朽壞墳起,覆沒畜產。汴梁延津縣大風晝晦,桑多損。大同雨雹,大者如雞卵。諸衛屯田隕霜害稼,益津縣雨黑霜。



 至治元年春正月丁丑,修佛事於文德殿。壬午,增置漷州都漕運司同知、運判各一員。甲申,召高麗王王章赴上都。丙戌,帝服袞冕,享太廟,以左丞相拜住亞獻,知樞密院事闊徹伯終獻。詔群臣曰:「一歲惟四祀,使人代之,不能致如在之誠,實所未安。歲必親祀,以終朕身。」廷臣或言祀事畢宜赦天下,帝諭之曰:「恩可常施,赦不可屢下。使殺人獲免,則死者何辜?」遂命中書陳便宜事,行之。丁亥,帝欲以元夕張燈宮中,參議中書省事張養浩上書諫止,帝遽命罷之,曰:「有臣若此,朕復何憂?自今朕凡有過,豈獨臺臣當諫,人皆得言。」賜養浩帛二匹。諸王忽都答兒來朝。癸巳,諸王斡羅思部饑,發凈州、平地倉糧賑之。蘄州蘄水縣饑,賑糧三月。奉元路饑,禁酒。乙未,太陰掩房。己亥,降延福監為延福提舉司,廣福監為廣福提舉司,秩從五品。以壽安山造佛寺,置庫掌財帛,秩從七品。甲辰,辰星犯外屏。水、金、火、土四星聚奎。



 二月,汴梁、歸德饑,發粟十萬石賑糶。河南、安豐饑,以鈔二萬五千貫、粟五萬石賑之。戊申,祭社稷。改中都威衛為忠翊侍衛親軍都指揮使司。己酉,作仁宗神御殿於普慶寺。辛亥,調軍三千五百人修上都華嚴寺。壬子夜,金、火、土三星聚於奎。大永福寺成,賜金五百兩、銀二千五百兩、鈔五十萬貫、幣帛萬匹。丁巳,畋於柳林,敕更造行宮。監察御史觀音保、鎖咬兒哈的迷失、成珪、李謙亨諫造壽安山佛寺,殺觀音保、鎖咬兒哈的迷失,杖珪、謙亨,竄於奴兒乾地。己未,樞密院臣請授副使吳元珪榮祿大夫,以階高不允,授正奉大夫。賑木憐道三十一驛貧戶。辛酉,太白犯熒惑。癸亥,太陰犯心。甲子,置承徽寺,秩正三品,割常州、宜興民四萬戶隸之。丁卯,以僧法洪為釋源宗主,授榮祿大夫、司徒。禁越臺、省訴事,罷先朝傳旨濫選者。戊辰,賜公主扎牙八剌從者鈔七十五萬貫。



 三月甲戌,營王也先帖木兒部畜牧死損,賜鈔五十萬貫。丙子,建帝師八思巴寺於京師。丁丑,御大明殿,受緬國使者朝貢。太陰掩昴。賜公主買的鈔五萬貫,駙馬滅憐鈔二萬五千貫。召諸王太平於汴。發民丁疏小直沽白河。庚辰,廷試進士泰普化、宋本等六十四人,賜及第、出身有差。辛巳,車駕幸上都。遣使賜西番撒思加地僧金二百五十兩、銀二千二百兩、袈裟二萬,幣、帛、幡、茶各有差。壬午,遣咒師朵兒只往牙濟、班卜二國取佛經。癸未,制御服珠袈裟。甲申,敕纂修《仁宗實錄》,《後妃》、《功臣傳》。乙酉,寶集寺金書西番《波若經》成,置大內香殿,益壽安山造寺役軍。己丑,大同路麒麟生。甲午,置雲南王府。己亥,宦者孛羅鐵木兒坐罪,流奴兒乾地。庚子,賑寧國路饑。辛丑,以鐵失為御史大夫,佩金符,領忠翊侍衛親軍都指揮使。癸卯,益都、般陽饑,以粟賑之。



 夏四月丙午,給喃答失王府銀印,秩正三品;寬徹、忽塔迷失王府銅印,秩從三品。庚戌,享太廟。江州、贛州、臨江霖雨,袁州、建昌旱,民皆告饑,發米四萬八千石賑之。丁巳,廣德路旱,發米九千石減直賑糶。戊午,太陰犯心。己未,造象駕金脊殿。吉陽黎蠻寇寧遠縣。庚申,太陰犯鬥。戊辰,敕賜鐵木迭兒父祖碑。命宦者孛羅臺為太常署令,太常官言刑人難與大祭,遂罷之。



 五月丙子,毀上都回回寺,以其地營帝師殿。賑益都、膠州饑。丁丑,霸州蝗。戊寅,太白犯鬼積尸氣,太陰犯軒轅。庚辰,太陰犯明堂。濮州大饑,命有司賑之。壬午,遷親王圖帖穆爾於海南。禁日者毋交通諸王、駙馬,掌陰陽五科者毋洩占候。以興國路去歲旱,免其田租。丁亥,修佛事於大安閣。庚寅,賑諸王哈賓鐵木兒部。沂州民張昱坐妖言,濟南道士李天祥坐教人兵藝,杖之。女直蠻赤興等十九驛饑,賑之。辛卯,海漕糧至直沽,遣使祀海神天妃。作行殿於縉山流杯池。高郵府旱。癸巳,寶定路飛蟲食桑。乙未,命世家子弟成童者入國學。辛丑,太常禮儀院進太廟制圖。壬寅,開元路霖雨。六月癸卯朔,日有食之。作金浮屠於上都,藏佛舍利。乙卯,以鐵木迭兒領宣政院事。丁巳,參知政事敬儼罷為陜西行御史臺中丞。戊午,涇州雨雹。己未,太陰犯虛梁。滁州霖雨傷稼,蠲其租。辛酉,太白經天。趙弘祚等言事,勒歸鄉里,仍禁妄言時政。壬戌,龍虎山張嗣成來朝,授太玄輔化體仁應道大真人。乙丑,遣使往銓江浙、江西、湖廣、四川、雲南五省邊郡官選。丁卯,翽星於司天臺。大同路雨雹。戊辰,衛輝、汴梁等處蝗。己巳,以上都留守只兒哈郎為中書平章政事。臨江路旱,免其租。通濟屯霖雨傷稼,霸州大水,渾河溢,被災者二萬三千三百戶。



 秋七月壬申,賜晉王也孫鐵木兒鈔百萬貫。遼陽、開元等路及順州、邢臺等縣大水。癸酉,衛輝路胙城縣蝗。乙亥,賑南恩、新州饑。丙子,淮安路屬縣水。丁丑,享太廟。戊寅,通州潞縣榆棣水決。庚辰,鹵簿成。滹沱河及範陽縣巨馬河溢。辛巳,盩厔縣僧圓明作亂,遣樞密院判官章臺督兵捕之。壬午,通許、臨淮、盱眙等縣蝗。癸未,對太尉孛蘭奚為和國公。乙酉,大雨,渾河防決。庚寅,清池縣蝗。癸巳,太陰犯昴。黃平府蠻盧砰為寇,削萬戶何之祺等官一級。遣吏部尚書教化、禮部郎中文矩使安南,頒登極詔。諸王闊別薨,賻鈔萬五千貫。丙申,禁服色逾制。己亥,奉仁宗及帝御容於大聖壽萬安寺。蒲陰縣大水。庚子,修上都城。詔河南、江浙流民復業。淮西蒙城等縣饑,郃陽道士劉志先以妖術謀亂,復命章臺捕之。薊州平谷、漁陽等縣大水,大都、保定、真定、大名、濟寧、東平、東昌、永平等路,高唐、曹、濮等州水,順德、大同等路雨雹,乞兒吉思部水。



 八月壬寅,修都城。安陸府水,壞民廬舍。癸卯,賑膠州饑。甲辰,高郵興化縣水,免其租。丙午,泰興、江都等縣蝗。丁未,太陰犯心。戊申,祭社稷。上都鹿頂殿成。己酉,太陰犯鬥。庚戌,以軍士貧乏,遣知樞密院事鐵木兒不花整治,仍詔諭中外,有敢擾害者罪之。賑北部孤寡糧、鈔。賜公主速哥八剌鈔五十萬貫。兀兒速、憨哈納思等部貧乏,戶給牝馬二匹。壬子,熒惑犯軒轅。乙卯,中書平章政事鐵木兒脫罷為上都留守。壬戌,淮安路鹽城、山陽縣水,免其租。車駕駐蹕興和,左右以寒甚,請還京師,帝曰:「兵以牛馬為重,民以稼穡為本。朕遲留,蓋欲馬得芻牧,民得刈獲,一舉兩得,何計乎寒?」雷州路海康、遂溪二縣海水溢,壞民田四千餘頃,免其租。秦州成紀縣山崩。九月乙亥,熒惑犯靈臺。京師饑,發粟十萬石減價糶之。丙子,駐蹕昂兀嶺。壬午,熒惑犯太微西垣上將。賜諸王撒兒蠻鈔五萬貫。壬辰,中書平章政事塔失海牙坐受贓杖免。丁酉,熒惑犯太微垣右執法。車駕還大都。庚子,安陸府漢水溢,壞民田,賑之。



 冬十月辛丑朔,修佛事於大內。妖僧圓明等伏誅。甲辰,太白經天。戊申,熒惑犯太微垣左執法。庚戌,親享太廟。壬子,拜住獻嘉禾,兩莖同穗。癸丑,敕翰林、集賢官年七十者毋致仕。以內郡水,罷不急工役。敕蒙古子女鬻為回回、漢人奴者,官收養之。禁中書掾曹毋洩機事,命樞密遣官整視各郡兵馬。戊午,置趙王馬札罕部錢糧總管府,秩正三品。己未,肇慶路水,賑之。丙寅,河南行省參知政事你咱馬丁坐殘忍免官。丁卯,增置侍儀司通事舍人六員,侍儀舍人四員。己巳,遣燕鐵木兒巡邊。



 十一月辛未,熒惑犯進賢。己亥,幸大護國仁王寺。丙子,太陰犯虛梁。戊寅,御大明殿,群臣上尊號曰繼天體道敬文仁武大昭孝皇帝。是夜,辰星犯房。己卯,以受尊號詔天下,拜住請釋囚,不允。庚辰,益壽安山寺役卒三千人。辛巳,命御史大夫鐵失領左、右阿速衛。丙戌,太陰犯井。丁亥,以教官待選者借注廣海巡檢。己丑,太陰犯酒旗,又犯軒轅。庚寅,拜住等言:「受尊號,宜謝太廟,行一獻禮。世祖亦嘗議行,武宗則躬行謝禮。」詔曰:「朕當親謝。」命太史卜日,樞密選兵肄鹵簿。辛卯,太陰犯明堂。癸巳,以營田提舉司徵酒稅擾民,命有司兼榷之。甲午,以遼陽行省管內山場隸中政院。丙申,敕立故丞相安童碑於保定新城。戊戌,鞏昌成州饑,發義倉賑之。己亥,太白犯西咸。



 十二月庚子,給蒙古子女冬衣。辛丑,立亦啟烈氏為皇后,遣攝太尉、中書右丞相鐵木迭兒持節授玉冊、玉寶。癸卯,以立後詔天下。慶遠路饑,真定路疫,並賑之。甲辰,熒惑犯亢。戊申,躬謝太廟。庚戌,太陰犯昴。作太廟正殿。甲寅,疏玉泉河。車駕幸西僧灌項寺。己未,封唆南藏卜為白蘭王,賜金印。真定、保定、大名、順德等路水,民饑,禁釀酒。以金虎符頒各行省平章政事。辛酉,熒惑入氐。甲子,置田糧提舉司,掌薊、景二州田賦,以給衛士貧乏者,秩從五品。命帝師公哥羅古羅思監藏班藏卜詣西番受具足戒,賜金千三百五十兩、銀四千五十兩、幣帛萬匹、鈔五十萬貫。以諸王怯伯使者數入朝,發兵守北口及盧溝橋。河間路饑,賑之。復以馬家奴為司徒。乙丑,置中瑞司。冶銅五十萬斤作壽安山寺佛像。寧海州蝗,歸德、遼陽、通州等處水。



\end{pinyinscope}