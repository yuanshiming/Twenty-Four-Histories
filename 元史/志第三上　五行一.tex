\article{志第三上 五行一}

\begin{pinyinscope}

 人與天地,參為三極,災祥之興,各以類至。天之五運,地之五材格拉底的大弟子柏拉圖有影響,故得此名,以示與柏拉圖學,其用不窮,其初一陰陽耳,陰陽一太極耳。而人之生也,全付畀有之,具為五性,著為五事,又著為五德,修之則吉,不修則兇,吉則致福焉,不吉則致極焉。征之於天,吉則休徵之所應也,不吉則咎徵之所應也。天地之氣,無感不應,天地之氣應,亦無物不感,而況天子建中和之極,身為神人之主,而心範圍天地之妙,其精神常與造化相流通,若桴鼓然。故軒轅氏治五氣,高陽氏建五官,夏后氏修六府,自身而推之於國,莫不有政焉。其後箕子因之,以衍九疇,其言天人之際備矣。漢儒不明其大要,如夏侯勝、劉向父子,競以災異言之,班固以來採為《五行志》,又不考求向之論著本於伏生。生之《大傳》言:「六沴作見,若是共御,五福乃降;若不共御,六極其下。禹乃共闢厥德,爰用五事,建用王極。」後世君不建極,臣不加省,顧乃執其類而求之,惑矣。否則判而二焉,如宋儒王安石之論,亦過也。天人感應之機,豈易言哉!故無變而無不修省者,上也;因變而克自修省者,次之;災變既形,修之而莫知所以修,省之而莫知所以省,又次之;其下者,災變並至,敗亡隨之,訖莫修省者,刑戮之民是已。歷考往古存亡之故,不越是數者。



 元起朔漠,方太祖西征,角端見於東印度,為人語云「汝主宜早還」,意者天告之以止殺也。憲宗討八赤蠻於寬田吉思海,會大風,吹海水盡涸,濟師大捷,憲宗以為「天導我也」。以此見五方不殊性,其於畏天,有不待教而能者。世祖兼有天下,方地既廣,郡邑災變,蓋不絕書,而妖孽禍眚,非有司言狀,則亦不得具見。



 昔孔子作《春秋》,所紀災異多矣,然不著其事應;聖人之知猶天也,故不妄意天,欲人深自謹焉。乃本《洪範》,仿《春秋》之意,考次當時之災祥,作《五行志》。



 五行,一曰水。潤下,水之性也。失其性為沴,時則霧水暴出,百川逆溢,壞鄉邑,溺人民,及凡霜雹之變,是為水不潤下。其徵恆寒,其色黑,是為黑眚黑祥。



 至元元年,真定、順天、河間、順德、大名、東平、濟南等郡大水。四年五月,應州大水。五年八月,亳州大水。六年十二月,獻、莫、清、滄四州及豐州、渾源縣大水。九年九月,南陽、懷孟、衛輝、順天等郡,洺、磁、泰安、通、灤等州淫雨,河水並溢,圮田廬,害稼。十三年十二月,濟寧及高麗沈州水。十四年六月,濟寧路雨水,平地丈餘,損稼。曹州定陶、武清二縣,濮州、堂邑縣雨水,沒禾稼。十二月,冠州、永年縣水。十六年十二月,保定等路水。十七年正月,磁州、永平縣水。八月,大都、北京、懷孟、保定、東平、濟寧等路水。十八年二月,遼陽懿州、蓋州水。十一月,保定清苑縣水。二十年六月,太原、懷孟、河南等路沁河水湧溢,壞民田一千六百七十餘頃。衛輝路清河溢,損稼。南陽府唐、鄧、裕、嵩四州河水溢,損稼。十月,涿州巨馬河溢。二十一年六月,保定、河間、濱、棣大水。二十二年秋,南京、彰德、大名、河間、順德、濟南等路河水壞田三千餘頃。高郵、慶元大水,傷人民七百九十五戶,壞廬舍三千九十區。二十三年六月,安西路華州華陰縣大雨,潼谷水湧,平地三丈餘。杭州、平江二路屬縣水,壞民田一萬七千二百頃。大都涿、漷、檀、順、薊五州,汴梁、歸德七縣水。二十四年六月,霸州益津縣雨水。九月,東京義、靜、威遠、婆娑等處水。二十五年七月,膠州大水,民採橡為食。十二月,太原、汴梁二路河溢,害稼。二十六年二月,紹興大水。十月,平灤路水,壞田稼一千一百頃。二十七年正月,甘州、無為路大水。五月,江陰州大水。六月,河溢太康縣,沒民田三十一萬九千畝。八月,沁水溢。廣州清遠縣大水。十一月,河決祥符義唐灣,太康、通許二縣,陳、潁二州,大被其患。二十八年二月,常德路水。八月,浙東婺州水。九月,平灤、保定、河間三路大水。二十九年五月,龍興路南昌、新建、進賢三縣水。六月,鎮江、常州、平江、嘉興、湖州、松江、紹興等路府水。揚州、寧國、太平三郡大水。岳州華容縣水。三十年五月,深州靜安縣大水。十月,平灤路水。三十一年八月,趙州寧晉縣水。十月,遼陽路水。



 元貞元年五月,建康溧陽州,太平當塗縣,鎮江金檀、丹徒等縣,常州無錫州,平江長洲縣,湖州烏程縣,鄱陽餘干州,常德沅江、澧州安鄉等縣水。六月,泰安州奉符、曹州濟陰、兗州嵫陽等縣水。歷城縣大清河水溢,壞民居。七月,遼東和州、大都武衛屯田水。九月,廬州、平江二郡大水。二年五月,太原平晉縣,獻州交河、樂壽二縣,莫州任丘、莫亭等縣,湖南醴陵州水。六月,大都路益津、保定、大興三縣水,損田稼七千餘頃。真定鼓城、獲鹿、槁城等縣,保定葛城、歸信、新安、束鹿等縣,汝寧潁州,濟寧沛縣,揚、廬、岳、澧四郡,建康、太平、鎮江、常州、紹興五郡水。八月,棣州、曹州水。九月,河決河南杞、封丘、祥符、寧陵、襄邑五縣。十月,河決開封縣。十二月,江陵潛江縣,沔陽玉沙縣,淮安海寧朐山、鹽城等縣水。



 大德元年三月,歸德徐州,邳州宿遷、睢寧,鹿邑三縣,河南許州臨潁、郾城等縣,睢州襄邑,太康、扶溝、陳留、開封、杞等縣,河水大溢,漂沒田廬。五月,河決汴梁,發民夫三萬五千塞之。漳水溢,害稼。龍興、南康、澧州、南雄、饒州五郡水。六月,和州歷陽縣江水溢,漂廬舍一萬八千五百區。七月,郴州耒陽縣、衡州酃縣大水,溺死三百餘人。九月,溫州平陽、瑞安二州水,溺死六千八百餘人。十一月,常德武陵縣大水。二年六月,河決蒲口,凡九十六所,泛溢汴梁、歸德二郡。大名、東昌、平灤等路水。三年八月,河間郡水。四年五月,保定、真定二郡,通、薊二州水。六月,歸德睢州大水。五年五月,宣德、保定、河間屬州水。寧海州水。六月,濟寧、般陽、益都、東平、濟南、襄陽、平江七郡水。七月,江水暴風大溢,高四五丈,連崇明、通、泰、真州定江之地,漂沒廬舍,被災者三萬四千八百餘戶。遼陽大寧路水。八月,平灤郡雨,灤河溢。順德路水。六年四月,上都大水。五月,濟南路大水。歸德府徐州、邳州睢寧縣雨五十日,沂、武二河合流,大水溢。東安州渾河溢,壞民田一千八十餘頃。六月,廣平路大水。七年五月,濟南、河間等路水。六月,遼陽、大寧、平灤、昌國、沈陽、開元六郡雨水,壞田廬,男女死者百十有九人。修武、河陽、新野、蘭陽等縣趙河、湍河、白河、七里河、沁河、潦河皆溢。臺州風水大作,寧海、臨海二縣死者五百五十人。八年五月,太原陽武縣、衛輝獲嘉縣、汴梁祥符縣河溢。大名滑州、濬州雨水,壞民田六百八十餘頃。八月,潮陽颶風海溢,漂民廬舍。九年六月,汴梁陽武縣思齊口河決。東昌博平、堂邑二縣雨水。潼川郪縣雨,綿江、中江溢,水決入城。龍興、撫州、臨川三郡水。七月,沔陽玉沙縣江溢。嶧州水。揚州泰興縣,淮安山陽縣水。八月,歸德府寧陵,陳留、通許、扶溝、太康、杞縣河溢。大名元城縣大水。十年五月,雄州、漷州水。平江、嘉興二郡水,害稼。六月,保定滿城、清苑二縣雨水。大名、益都、定興等路大水。七月,平江路大風,海溢。吳江州大水。十一年六月,靖海、容城、束鹿、隆平、新城等縣水。七月,冀寧文水縣汾水溢。十一月,盧龍、灤河、遷安、昌黎、撫寧等縣水。



 至大元年七月,濟寧路雨水,平地丈餘,暴決入城,漂廬舍,死者十有八人。真定路淫雨,大水入南門,下注槁城,死者百七十人。彰德、衛輝二郡水,損稻田五千三百七十頃。二年七月,河決歸德府,又決汴梁封丘縣。三年六月,洧川、鄄城、汶上三縣水。峽州大雨,水溢,死者萬餘人。七月,循州、惠州大水,漂廬舍二百九十區。四年六月,大都三河縣、潞縣,河東祁縣、懷仁縣,永平豐盈屯雨水害稼。七月,東平、濟寧、般陽、保定等路大水。江陵松滋縣、桂陽臨武縣水。



 皇慶元年五月,歸德睢陽縣河溢。六月,大寧、水達達路雨,宋瓦江溢,民避居亦母兒乞嶺。八月,松江府大風,海水溢。二年五月,辰州沅陵縣水。六月,涿州範陽縣,東安州,宛平縣,固安州,霸州益津、永清、永安等縣雨水,壞田稼七千六百九十餘頃。河決陳、亳、睢三州,開封、陳留等縣。八月,崇明、嘉定二州大風,海溢。



 延祐元年五月,常德路武陵縣雨水,壞廬舍,溺死者五百人。六月,涿州範陽、房山二縣渾河溢,壞民田四百九十餘頃。七月,沅陵、盧溪二縣水。八月,肇慶、武昌、建康、杭州、建德、南康、江州、臨江、袁州、建昌、贛州、安豐、撫州等路水。二年六月,河決鄭州,壞汜水縣治。七月,京師大雨。漷州、昌平、香河、寶坻等縣水。全州、永州江水溢,害稼。三年四月,潁州太和縣河溢。七月,婺源州雨水,溺死者五千三百餘人。四年正月,解州鹽池水。五年四月,廬州合肥縣大雨水。六年六月,河間路漳河水溢,壞民田二千七百餘頃。益都、般陽、濟南、東昌、東平、濟寧等路,曹、濮、泰安、高唐等州大雨水害稼。遼陽、廣寧、沈陽、永平、開元等路水。大名路屬縣水,壞民田一萬八千頃。汴梁、歸德、汝寧、彰德、真定、保定、衛輝、南陽等郡大雨水。七年四月,安豐、廬州淮水溢,損禾麥一萬頃。城父縣水。五月,江陵縣水。六月棣州、德州大雨水,壞田四千六百餘頃。七月,上蔡、汝陽、西平等縣水。八月,霸州文安、文成二縣滹沱河溢,害稼。汾州平遙縣水。是歲,河決汴梁原武縣。



 至治元年六月,霸州大水,渾河溢,被災者三萬餘戶。七月,薊州平谷、漁陽二縣,順州邢臺、沙河二縣,大名魏縣,永平石城縣大水。彰德臨漳縣漳水溢。大都固安州,真定元氏縣,東安,寶坻縣,淮安清河、山陽等縣水。東平、東昌二路,高唐、曹、濮等州雨水害稼。乞里吉思部江水溢。八月,安陸府雨七日,江水大溢,被災者三千五百戶。雷州海康、遂溪二縣海水溢,壞民田四千頃。九月,京山、長壽二縣漢水溢。十月,遼陽、肇慶等郡水。二年正月,儀封縣河溢。二月,濮州大水。閏五月,睢陽縣亳社屯大水。六月,奉元郿縣,邠州新平、上蔡二縣水。八月,廬州六安、舒城二縣水。十一月,平江路大水,損民田四萬九千六百頃。三年五月,東安州水,壞民田一千五百餘頃。真定武邑縣水害稼。六月,大都永清縣雨水,損田四百頃。七月,漷州雨水害稼。九月,漳州、建昌、南康等郡水。



 泰定元年五月,漷州、固安州水。隴西縣大雨水,漂死者五百餘家。龍慶路雨水傷稼。六月,益都、濟南、般陽、東昌、東平、濟寧等郡二十有二縣,曹、濮、高唐、德州等處十縣淫雨,水深丈餘,漂沒田廬。大同渾源河溢。陳、汾、順、晉、恩、深六州雨水害稼。真定滹沱河溢,漂民廬舍。陜西大雨,渭水及黑水河溢,損民廬舍。渠州江水溢。七月,真定、河間、保定、廣平等郡三十有七縣大雨水五十餘日,害稼。大都路固安州清河溢。順德路任縣沙、灃、洺水溢。奉元朝邑縣、曹州楚丘縣、開州濮陽縣河溢。九月,延安路洛水溢。奉元長安縣大雨,灃水溢。濮州館陶縣水。十二月,杭州鹽官州海水大溢,壞堤塹,侵城郭,有司以石囤木櫃捍之不止。二年正月,大都寶坻縣、肇慶高要縣雨水。鞏昌路水。



 閏閏月,雄州歸信縣大水。二月,甘州路大雨水,漂沒行帳孳畜。三月,咸平府清、寇二河合流,失故道,隳堤堰。四月,涿州房山、範陽二縣水。岷、洮、文、階四州雨水。五月,檀州大水,平地深丈有五尺。高郵興化、江陵公安二縣水。河溢汴梁,被災者十有五縣。六月,冀寧路汾河溢。潼川府綿江、中江水溢入城,深丈餘。衛輝汲縣、歸德宿州雨水。濟寧路虞城、碭山、單父、豐、沛五縣水。七月,睢州河決。八月,霸州,涿州,永清、香河二縣大水,傷稼九千五十餘頃。九月,開元路三河溢,沒民田,壞廬舍。十月,寧夏鳴沙州大雨水。三年正月,恩州水。二月,歸德府河決。六月,大同縣大水。汝寧光州水。七月,河決鄭州,漂沒陽武等縣民一萬六千五百餘家。東安、檀、順、漷四州雨,渾河決,溫榆水溢,傷稼。延安路膚施縣水,漂民居九十餘戶。八月,鹽官州大風,海溢,捍海堤崩,廣三十餘里,袤二十里,徙居民千二百五十家以避之。真定蠡州,奉元蒲城縣,無為州,歷陽、含山等縣水。九月,平遙縣汾水溢。十一月,崇明州三沙鎮海溢,漂民居五百家。十二月,遼陽大水。大寧路瑞州大水,壞民田五千五百頃,廬舍八百九十所,溺死者百五十人。四年正月,鹽官州潮水大溢,捍海堤崩二千餘步。四月,復崩十九里,發丁夫二萬餘人,以木柵竹落磚石塞之,不止。六月,大都東安、固安、通、順、薊、檀、漷七州,永清、良鄉等縣雨水。七月,上都雲州大雨。北山黑水河溢。雲安縣水。八月,汴梁扶溝、蘭陽二縣河溢,漂民居一千九百餘家。濟寧虞城縣河溢,傷稼。十二月,夏邑縣河溢。汴梁中牟、開封、陳留三縣,歸德邳、宿二州雨水。



 致和元年三月,鹽官州海堤崩,遣使禱祀,造浮圖二百十六,用西僧法壓之。河決碭山、虞城二縣。四月,鹽官州海溢,益發軍民塞之,置石囤二十九里。六月,南寧、開元、永平等路水。河間臨邑縣雨水。益都、濟南、般陽、濟寧、東平等郡三十縣,濮、德、泰安等州九縣雨水害稼。七月,廣西兩江諸州水。



 天歷元年八月,杭州、嘉興、平江、湖州、建德、鎮江、池州、太平、廣德九郡水,沒民田萬四千餘頃。二年六月,大都東安、通、薊、霸四州,河間靖海縣雨水害稼。永平昌國諸屯水。



 至順元年六月,河決大名路長垣、東明二縣,沒民田五百八十餘頃。曹州、高唐州水。



 七月,海潮溢,漂沒河間運司鹽二萬六千七百引。閏七月,平江、嘉興、湖州、松江三路一州大水,壞民田三萬六千六百餘頃,被災者四十萬五千五百餘戶。杭州、常州、慶元、紹興、鎮江、寧國等路,望江、銅陵、長林、寶應、興化等縣水,沒民田一萬三千五百餘頃。大都、保定、大寧、益都屬州縣水。二年四月,潞州潞城縣大雨水。五月,河間莫亭縣、寧夏河渠縣、紹慶彭水縣及德安屯田水。六月,彰德屬縣漳水決。十月,吳江州大風,太湖水溢,漂民居一千九百七十餘家。十二月,深州、晉州水。三年三月,奉元朝邑縣洛水溢。五月,汴梁河水溢。江都、泰興、雲夢、應城等縣水。六月,汾州大水。



 至元十四年九月,湖州長興縣金沙泉,自唐、宋以來,用以造茶,其泉不常有,今瀵然湧出,溉田可數百頃。有司以聞,錫名瑞應泉。十五年十二月,河水清,自孟津東柏谷至汜水縣蓼子穀,上下八十餘里,澄瑩見底,數月始如故。



 元貞元年閏四月,蘭州上下三百餘里,河清三日。



 中統二年五月,西京隕霜殺禾。三年五月,宣德、威寧等路隕霜。八月,河間、平灤等路隕霜害稼。四年四月,武州隕霜殺麥禾。



 至元二年八月,太原隕霜。七年四月,檀州隕霜。八年七月,鞏昌會、蘭等州霜殺稼。十七年四月,益都隕霜。二十一年三月,山東隕霜殺桑,蠶盡死,被災者三萬餘家。二十七年七月,大同、平陽、太原隕霜殺禾。二十九年三月,濟南、般陽等郡及恩州屬縣霜殺桑。



 元貞二年八月,金、復州隕霜殺禾。



 大德五年三月,湯陰縣霜殺麥。五月,商州霜殺麥。六年八月,大同、太原霜殺禾。七年四月,霜殺麥。八年三月,濟陽、灤城二縣霜殺桑。八月,隕霜殺稼。九年三月,河間、益都、般陽三郡屬縣隕霜殺桑。清、莫、滄、獻四州霜殺桑二百四十一萬七千餘本,壞蠶一萬二千七百餘箔。十年七月,大同渾源縣霜殺禾。八月,綏德州米脂縣霜殺禾二百八十頃。



 至大元年八月,大同隕霜殺禾。



 皇慶二年三月,濟寧霜殺桑。



 延祐元年三月,東平、般陽等郡,泰安、曹、濮等州大雨雪三日,隕霜殺桑。閏三月,濟寧、汴梁等路及隴州、開州、青城、渭源諸縣霜殺桑,無蠶。七月,冀寧隕霜殺稼。四年夏,六盤山隕霜殺稼五百餘頃。五年五月,雄州歸信縣隕霜。六年三月,奉元路同州隕霜。七年八月,益津縣雨黑霜。



 至治三年七月,冀寧陽曲縣、大同路大同縣、興和路威寧縣隕霜。八月,袁州宜春縣隕霜害稼。



 泰定二年三月,雲需府大雪,民饑。



 天歷三年二月,京師大霜,晝雺。



 至順元年閏七月,奉元西和州,寧夏應理州、鳴沙州,鞏昌靜寧、邠、會等州,鳳翔麟游,大同山陰,晉寧潞城、隰川等縣隕霜殺稼。



 中統二年四月,雨雹,大如彈丸。三年五月,順天、平陽、真定、河南等郡雨雹。四年七月,燕京昌平縣,景州蓚縣,開平路興、松、雲三州雨雹害稼。



 至元二年八月,彰德、大名、南京、河南、濟南、太原等郡雨雹。四年三月,夏津縣大雨雹。五年六月,中山大雨雹。六年七月,西京大同縣雨雹。七年五月,河內縣大雨雹。十五年閏十一月,海州贛榆縣雨雹傷稼。十九年八月,雨雹,大如雞卵。二十年四月,河南風雷雨雹傷稼。五月,安西路風雷雨雹。八月,真定元氏縣大風雹,禾盡損。二十二年七月,冠州雨雹。二十四年九月,大定、金源、高州、武平、興中等處雨雹。二十五年三月,靈壁、虹縣雨雹,如雞卵,害麥。十二月,靈壽、陽曲、天成等縣雨雹。二十六年夏,平陽、大同、保定等郡大雨雹。二十七年四月,靈壽縣大風雹。六月,棣州厭次,濟陽二縣大風雹,傷禾黍菽麥桑棗。二十九年閏六月,遼陽、沈州,廣寧、開元等路雨雹。三十一年四月,即墨縣雨雹。八月,德州安德縣大風雨雹。



 元貞元年五月,鞏昌金州、會州、西和州雨雹大,無麥禾。七月,隆興路雨雹。二年五月,河中猗氏縣雨雹。六月,隆興威寧縣,順德邢臺縣,太原交城、離石、壽陽等縣雨雹。八月,懷孟武陟縣雨雹。



 大德元年六月,太原崞州雨雹害稼。二年二月,檀州雨雹。八月,彰德安陽縣雨雹。四年三月,宣州涇縣、臺州臨海縣風雹。八年五月,大寧路建州、蔚州靈仙縣雨雹。太原、大同、隆興屬縣陽曲、天成、懷安、白登風雹害稼。八月,管州、嵐州,交城、陽曲、懷仁等縣雨雹。九年六月,晉寧、冀寧、宣德、隆興、大同等郡大雨雹,害稼。十年四月,鄭州管城縣風雹,大如雞卵,積厚五寸。五月,大雨雹。七月,宣德縣雨雹。十一年五月,建州雨雹。



 至大元年四月,般陽新城縣、濟南厭次縣、益都高苑縣風雹。五月,管城縣大雹,深一尺,無麥禾。八月,大寧縣雨雹害稼,斃畜牧。二年三月,濟陰、定陶等縣雨雹。六月,崞州、源州、金城縣雨雹。延安神木縣大雹一百餘里,擊死人畜。三年四月,靈壽、平陰等縣雨雹。四年四月,南陽雨雹。閏七月,宣寧縣雨雹。



 皇慶元年四月,大名濬州、彰德安陽縣、河南孟津縣雨雹。六月,開元路風雹害稼。二年七月,冀寧平定州雨雹。景州阜城縣風雹。八月,大同懷仁縣雨雹。



 延祐元年五月,膚施縣大風雹,損稼並傷人畜。六月,宣平、仁壽、白登等縣雨雹。二年五月,大同、宣德等郡雷雹害稼。三年五月,薊州雹深一尺。五年四月,鳳翔府雹傷麥禾。六年六月,大同縣雨雹,大如雞卵。七月,鞏昌隴西縣雹害稼。七年八月,大同路雷風雨雹。



 至治元年六月,武州雨雹害稼。永平路大雹深一尺,害稼。七月,真定、順德等郡雨雹。二年四月,涇州涇川縣雨雹。六月,思州大風雨雹。三年五月,大風雨雹,拔柳林行宮大木。



 泰定元年五月,冀寧陽曲縣雨雹傷稼。思州龍泉平雨雹傷麥。六月,順元、太平軍、定西州雨雹。七月,龍慶路雨雹,大如雞卵,平地深三尺餘。八月,大同白登縣雨雹。二年四月,奉元白水縣雨雹。五月,洮州路可當縣、臨洮府狄道縣雨雹。六月,興州、鄜州、靜寧州及成紀、通渭、白水、膚施、安塞等縣雨雹。七月,檀州雨雹。三年六月,鞏昌路大雨雹。中山府安喜縣、乾州永壽縣雨雹。七月,房山、寶坻、玉田、永平等縣大風雹,折木傷稼。八月,龍慶州雨雹一尺,大風損稼。四年七月,彰德湯陰縣,冀寧定襄縣,大同武、應二州雨雹害稼。



 致和元年四月,濬州、涇州大雹傷麥禾。五月,冀寧陽曲縣、威州井陘縣雨雹。六月,涇川、湯陰等縣大雨雹。大寧、永平屬縣雨雹。



 天歷二年七月,大寧惠州雨雹。八月,冀寧陽曲縣大雹如雞卵,害稼。三年七月,順州、東安州及平棘、肥鄉、曲陽、行唐等縣風雹害稼。開元路雨雹。



 至順二年十二月,冀寧清源縣雨雹。三年五月,甘州雨雹。乙巳,天鼓鳴於西北。



 中統二年九月,河南民王四妻靳氏一產三男。《唐志》云:「物反常為妖,陰氣盛則母道壯也。」



 至元元年八月,武城縣王氏妻崔一產三男。十年八月甲寅,鳳翔寶雞縣劉鐵牛妻一產三男。二十年二月,高州張醜妻李氏一產四子,三男一女。四月,固安州王得林妻張氏懷孕五月生一男,四手四足,圓頭三耳,一耳附腦後,生而即死,具狀有司上之。二十八年九月,襄陽南漳縣民李氏妻王一產三子。



 大德元年五月,遂寧州軍戶任福妻一產三男。十一月,遼陽打雁孛蘭奚戶那懷妻和裏迷一產四男。四年,寶應縣民孫奕妻硃氏一產三男。十年正月,江州湖口縣方丙妻甘氏一產四男。



 泰定元年十月乙卯,秦州成紀縣趙思直妻張氏一產三子。



 致和元年三月壬辰,太平當塗縣楊太妻吳氏一產三子。



 五行,二曰火。炎上,火之性也,失其性為沴。董仲舒云:「陽失節,則火災出。」於是而濫炎妄起,災宗廟,燒宮館,雖興師眾弗能救也。是為火不炎上。其徵恆燠,其色赤,是為赤眚赤祥。



 定宗三年戊申,野草自焚,牛馬十死八九,民不聊生。



 至元十一年十二月,淮西正陽火,廬舍、鎧仗悉毀。十八年二月,揚州火。



 元貞二年,杭州火,燔七百七十家。



 大德八年五月,杭州火,燔四百家。九年三月,宜黃縣火。十年十一月,武昌路火。



 延祐元年二月,真州揚子縣火。三年六月,重慶路火,郡舍十焚八九。六年四月,揚州火,燔官民廬舍一萬三千三百餘區。



 至治二年四月,揚州、真州火。十二月,杭州火。三年五月,奉元路行宮正殿火,上都利用監庫火。九月,揚州江都縣火,燔四百七十餘家。



 泰定元年五月,江西袁州火,燔五百餘家。三年六月,龍興路寧州高市火,燔五百餘家。七月,龍興奉新縣、辰州辰溪縣火。八月,杭州火,燔四百七十餘家。四年八月,龍興路火。十二月,杭州火,燔六百七十家。



 天歷二年三月,四川紹慶彭水縣火。四月,重慶路火,延二百四十餘家。七月,武昌路江夏縣火,延四百家。十二月,江夏縣火,燔四百餘家。三年二月,河內諸縣火。



 皇慶元年,冬無雪,詔禱嶽瀆。



 延祐元年,大都檀、薊等州冬無雪,至春草木枯焦。



 至元二年八月丙寅,濟南鄒平縣進芝一本。八年八月癸酉,益都濟州進芝二本。十五年四月,濟南歷城縣進芝。十九年六月,芝生眉州青神縣景德寺。二十三年四月丁未,江東宣慰司進芝一本。十月,濟寧進芝一本。二十六年三月癸未,東流縣獻芝。四月,池州貴池縣民王逸進紫芝十二本。六月,汲縣民硃良進紫芝。二十八年三月,芝生鈞州陽翟縣。二十九年六月,芝生賀州。



 大德五年十二月,興元西鄉縣進芝一本,色如珊瑚。六年正月,濟南鄒平縣進芝一本,五枝五葉,色皆赤。



 至大四年八月,芝生國學大成殿。



 延祐二年三月,芝生大成殿。五年七月,芝生大成殿。



 中統二年正月辛未,御帳殿受朝賀,是夜,東北有赤氣照人,大如席。



 五行,三曰木。曲直,木之性也,失其性為沴,故生不暢茂,為變異者有之,是為木不曲直。其徵恆雨,其色青,是為青眚青祥。



 大德七年十一月辛酉,木冰。



 至順二年十一月丁巳,雨木冰。十二月癸亥,雨木冰。



 元貞元年,太平路蕪湖縣進榆木,有文曰「天下太平年」。



 至治三年五月庚子,柳林行宮大木風拔三千七百株。



 至元十七年二月,真定七郡桑有蟲食之。二十九年五月,滄州、濰州,中山,元氏、無棣等縣桑蟲食葉,蠶不成。



 元貞元年四月,真定中山、靈壽二縣桑有蟲食之。



 大德五年四月,彰德、廣平、真定、順德、大名等郡蟲食桑。



 至大元年五月,大名、廣平、真定三郡蟲食桑。



 致和元年六月,河南德安屯蠖食桑。



 天歷二年三月,滄州、高唐州及南皮、鹽山、武城等縣桑,蟲食之如枯株。



 至順二年三月,冠州蟲食桑四萬株。晉、冀、深、蠡等州及鄆城、延津二縣蟲夜食桑,晝匿土中,人莫捕之。五月,曹州禹城、保定博野、東昌封丘等縣蟲食桑,皆既。



 至元九年六月丁亥,京師大雨。二十四年九月,太原、河間、河南等路霖雨害稼。二十五年七月,保定郡,霸、漷二州淫雨害稼。八月,嘉祥、魚臺、金鄉三縣淫雨,九月,莫、獻二州淫雨。保定路淫雨。二十六年六月,濟寧、東平、汴梁、濟南、順德、真定、平灤、棣州霖雨害稼。二十八年八月,大名、清河、南樂諸縣霖雨為災。九月,河間郡淫雨。



 至大四年七月,河間、順德、大名、彰德、廣平等路,德、濮、恩、通等州及河東祁縣霖雨害稼。



 皇慶元年,龍興路新建縣雨害稼。



 延祐四年四月,遼陽蓋州雨水害稼。六年七月,霸州文成縣雨害稼三千餘頃。



 至治元年,江州、贛州淫雨。二年閏五月,安豐路雨傷稼。三年五月,大名魏縣淫雨。保定定興縣,濟南無棣、厭次縣,濟寧碭山縣,河間齊東縣霖雨害稼。



 泰定元年七月,真定、廣平、廬州十一郡雨傷稼。八月,汴梁考城、儀封,濟南沾化、利津等縣霖雨,損禾稼。



 五行,四曰金。從革,金之性也,失其性為沴,時則冶鑄不成,變異者有之,是為金不從革。金石同類,故古者以類附見。其徵恆絜,其色白,是為白眚白祥。



 至元十三年,霧靈山伐木官劉氏言,檀州大峪錐山出鐵礦,有司覆視之,尋立四冶。



 大德元年,雲州聚陽山等冶言,礦石煽煉銀貨不出,詔減其課額。二年六月,撫州崇仁縣辛陂村有星隕於地,為綠色隕石,邑人張椿以狀聞。



 泰定四年八月,天全道山崩,飛石擊人,中者輒死。



 庶徵之恆絜,劉向以為《春秋》大旱也。京房《易傳》曰:「欲得不用,茲謂張,厥災荒。」荒,旱也。



 中統三年五月,濱、棣二州旱。四年八月,真定郡及洺、磁等州旱。



 至元元年二月,東平、太原、平陽旱,分命西僧禱雨。五年十二月,京兆大旱。八年四月,蔚州靈仙、廣靈二縣旱。九年六月,高麗旱。十三年十二月,平陽路旱。十六年七月,趙州旱。十八年二月,廣寧、北京大定州旱。二十三年五月,汴梁旱,京畿旱。二十四年春,平陽旱,二麥枯死。二十五年,東平路須城等六縣,安西路商、耀、乾、華等十六州旱。二十六年,絳州大旱。



 元貞元年六月,環州、葭州及咸寧、伏羌、通渭等縣旱。七月,河間肅寧、樂壽二縣旱,泗州、賀州旱。二年八月,大名開州、懷孟武陟縣、河間肅寧縣旱。九月,莫州、獻州旱。十月,化州旱。十二月,遼東、開元二路旱。



 大德元年六月,汴梁、南陽大旱,民鬻子女。九月,鎮江丹陽、金壇二縣旱。十二月,平陽曲沃縣旱。二年五月,衛輝、順德、平灤等路旱。三年五月,荊湖諸郡及桂陽、寶慶、興國三路旱。十月,揚、廬、隨、黃等州旱。四年,平棘、白馬二縣旱。五年六月,汴梁、南陽、衛輝、大名等路旱。九月,江陵旱。八年六月,鳳翔扶風、岐山、寶雞三縣旱。九年七月,晉州饒陽縣、漢陽漢川縣旱。八月,象州、融州、柳州屬縣旱。十年五月,京畿旱。安西春夏大旱,二麥枯死。



 至大三年夏,廣平亢旱。



 皇慶元年六月,濱、棣、德三州及蒲臺、陽信等縣旱。二年九月,京畿大旱。



 延祐二年春,檀、薊、濠三州旱。夏,鞏昌蘭州旱。四年四月,德安府旱。五年七月,真定、河間、廣平、中山大旱。七年六月,黃、蘄二郡及荊門州旱。



 至治元年六月,大同路旱。二年十一月,岷州旱。三年夏,順德、真定、冀寧大旱。



 泰定元年六月,景、清、滄、莫等州,臨汾、涇川、靈臺、壽春、六合等縣旱。九月,建昌郡旱。二年五月,潭州、茶陵州、興國永興縣旱。七月,隨州、息州旱。三年夏,燕南、河南州縣十有四亢陽不雨。七月,關中旱。四年二月,奉元醴泉、順德唐山、邠州淳化等縣旱。六月,潞、霍、綏德三州旱。八月,藤州旱。



 致和元年二月,廣平、彰德等郡旱。



 天歷元年八月,陜西大旱,人相食。二年夏,真定、河間、大名、廣平等四州四十一縣旱。峽州二縣旱。八月,浙西湖州,江東池州、饒州旱。十二月,冀寧路旱。



 至順元年七月,肇州、興州、東勝州及榆次、滏陽等十三縣旱。二年,霍、隰、石三州,阜城、平地二縣旱。



 恆絜,則有介蟲之孽。釋者謂小蟲有甲飛揚之類,陽氣所生也,於《春秋》為螽,今謂之蝗。按劉歆云,貪虐取民則螽與魚同占。劉向以為介蟲之孽,當屬言不從。今仿之。



 中統三年五月,真定、順天、邢州蝗。四年六月,燕京、河間、益都、真定、東平蝗。



 八月,濱、棣等州蝗。



 至元二年七月,益都大蝗。十二月,西京、北京、順德、徐、宿、邳等州郡蝗。五年六月,東平等郡蝗。七年七月,南京、河南諸路大蝗。八年六月,上都、中都、大名、河間、益都、順天、懷孟、彰德、濟南、真定、衛輝、平陽、歸德、順德等路,淄、萊、洺、磁等州蝗。十六年四月,大都十六路蝗。十七年五月,忻州及漣、海、邳、宿等州蝗。十九年四月,別十八里部東三百餘里蝗害麥。二十五年七月,真定、汴梁蝗。八月,趙、晉、冀三州蝗。二十七年四月,河北十七郡蝗。二十九年六月,東昌、濟南、般陽、歸德等郡蝗。三十一年六月,東安州蝗。



 元貞元年六月,汴梁陳留、太康、考城等縣,睢、許等州蝗。二年六月,濟寧任城、魚臺縣,東平須城、汶上縣,開州長垣、清豐縣,德州齊河縣,滑州,太和縣,內黃縣蝗。八月,平陽、大名、歸德、真定等郡蝗。



 大德元年六月,歸德邳州、徐州蝗。二年四月,燕南、山東、兩淮、江浙、燕南屬縣百五十處蝗。三年五月,淮安屬縣蝗,有鶖食之。十月,隴、陜蝗。五年六月,順德路、淇州蝗。七月,廣平、真定等路蝗。八月,河南、淮南、睢、陳、唐、和等州,新野、汝陽、江都、興化等縣蝗。六年四月,真定、大名、河間等路蝗。七月,大都涿、順、固安三州及濠州鐘離、鎮江丹徒二縣蝗。七年五月,益都、濟南等路蝗。六年,大寧路蝗。八年四月,益都臨朐、德州齊河縣蝗。六月,益津縣蝗。九年六月,通、泰、靖海、武清等州縣蝗。八月,涿州,良鄉、河間南皮、泗州天長等縣及東安、海鹽等州蝗。十年四月,大都、真定、河間、保定、河南等郡蝗。六月,龍興、南康等郡蝗。



 至大元年五月,晉寧路蝗。六月,保定、真定二郡蝗。八月,淮東蝗。二年四月,益都、東平、東昌、順德、廣平、大名、汴梁、衛輝等郡蝗。六月,檀、霸、曹、濮、高唐、泰安等州,良鄉、舒城、歷陽、合肥、六安、江寧、句容、溧水、上元等縣蝗。七月,濟南,濟寧,般陽,河中,解、絳、耀、同、華等州蝗。八月,真定、保定、河間、懷孟等郡蝗。三年四月,寧津、堂邑、茌平、陽谷、平原、齊河、禹城七縣蝗。七月,磁州、威州,饒陽、元氏、平棘、滏陽、元城、無棣等縣蝗。



 皇慶元年,彰德安陽縣蝗。



 延祐七年六月,益都路蝗。



 至治元年五月,霸州蝗。六月,衛輝、汴梁等處蝗。七月,江都、泰興、胙城、通許、臨淮、盱眙、清池等縣蝗。十二月,寧海州蝗。二年,汴梁祥符縣蝗,有群鶖食蝗,既而復吐,積如丘垤。三年五月,保定路歸信縣蝗。



 泰定元年六月,大都、順德、東昌、衛輝、保定、益都、濟寧、彰德、真定、般陽、廣平、大名、河間、東平等郡蝗。二年五月,彰德路蝗。六月,德、濮、曹、景等州,歷城、章丘、淄川、柳城、茌平等縣蝗。九月,濟南、歸德等郡蝗。三年六月,東平須城縣、興國永興縣蝗。七月,大名、順德、廣平等路,趙州,曲陽、滿城、慶都、修武等縣蝗。淮安、高郵二郡,睢、泗、雄、霸等州蝗。八月,永平、汴梁、懷慶等郡蝗。四年五月,洛陽縣有蝗五畝,群烏盡食之,越數日,蝗又集,又食之。七月,籍田蝗。八月,冠州、恩州蝗。十二月,保定、濟南、衛輝、濟寧、廬州五路,南陽、河南二府蝗。博興、臨淄、膠西等縣蝗。



 致和元年四月,大都薊州、永平路石城縣蝗。鳳翔岐山縣蝗,無麥苗。五月,潁州及汲縣蝗。六月,武功縣蝗。



 天歷二年四月,大寧興中州、懷慶孟州、廬州無為州蝗。六月,益都莒、密二州蝗。七月,真定、汴梁、永平、淮安、廬州、大寧、遼陽等郡屬縣蝗。三年五月,廣平、大名、般陽、濟寧、東平、汴梁、南陽、河南等郡,輝、德、濮、開、高唐五州蝗。



 至順元年六月,漷、薊、固安、博興等州蝗。七月,解州、華州及河內、靈寶、延津等二十二縣蝗。二年三月,陜州諸路蝗。六月,孟州濟源縣蝗。七月,河南閿鄉、陜縣,奉元蒲城、白水等縣蝗。



 至元十五年四月,濟南無棣縣獲白雉以獻。



 元貞三年正月,寧海州牟平縣獲白鹿於聖水山以獻。



 至元二十四年七月癸丑,日暈連環,白虹貫之。



 至大元年七月,流星起勾陳,化為白氣,員如車輪,至貫索始滅。



 皇慶元年六月丁卯,天雨毛。



 延祐元年二月己亥,白暈亙天,連環貫日。



 至順三年五月丁酉,白虹並日出,其長竟天。



 五行,五曰土。土,中央生萬物者也,而莫重於稼穡。土氣不養,則稼穡不成,金木水火沴之,沖氣為異,為地震,為天雨土。其徵恆風,其色黃,是為黃眚黃祥。



 中統元年五月,澤州、益州饑。二年六月,塔察兒部饑。七月,桓州饑。三年五月,甘州饑。閏九月,濟南郡饑。



 至元二年四月,遼東饑。五年九月,益都饑。六年十一月,濟南饑。十一月,固安、高唐二州饑。七年五月,東京饑。七月,山東淄、萊等州饑。八年正月,西京、益都饑。九年四月,京師饑。七月,水達達部饑。十七年三月,高郵郡饑。十八年二月,浙東饑。四月,通、泰、崇明等州饑。十九年九月,真定路饑,民流徙鄂州。二十三年七月,宣寧縣饑。二十四年九月,平灤路饑。十二月,蘇、常、湖、秀四州饑。二十五年十一月,兀良合部饑。二十六年二月,合木里部饑。三月,安西、甘州等路饑。四月,遼陽路饑。閏十月,武平路饑,檀州饑。十二月,蠡州饑,河間、保定二路饑。二十七年二月,開元路寧遠縣饑。四月,浙東婺州饑,河間任丘、保定定興二縣饑。九月,河東山西道饑。二十八年三月,真定、河間、保定、平灤、太原、平陽等路饑,杭州、平江、鎮江、廣德、太平、徽州饑。九月,武平路饑。十二月,洪寬女直部饑,大都內郡饑。二十九年正月,清州、興州饑。三月,輝州龍山縣、里州和中縣饑,東安、固安、薊、棣四州饑。三月,威寧、昌州饑。閏六月,南陽、懷孟、衛輝等路饑。三十年十月,京師饑。



 元貞二年四月,平陽絳州、太原陽曲、臺州黃巖饑。



 大德元年六月,廣德路饑。七月,寧海州文登、牟平等縣饑。三年八月,揚州、淮安等郡饑。四年二月,湖北饑。三月,寧國、太平二路饑。九月,建康、常州、江陵等郡饑。六年五月,福州饑。六月,杭州、嘉興、湖州、廣德、寧國、饒州、太平、紹興、慶元、婺州等郡饑,大同路饑。七月,建康路饑。十一月,保定路饑。七年二月,真定路饑。五月,太原、龍興、南康、袁州、瑞州、撫州等路,高唐、南豐等州饑。六月,浙西饑。七月,常德路饑。八年六月,烏撒、烏蒙、益州、忙部、東川等路饑。九年三月,常寧州饑。五月,寶慶路饑。八月,揚州饑。十年三月,濟州任城饑。四月,漢陽、淮安、道州、柳州饑。七月,黃州、沅州、永州饑。八月,成都饑。十一月,揚州、辰州饑。



 至大元年二月,益都、般陽、濟寧、濟南、東平、泰安大饑。六月,山東、河南、江淮等郡大饑。二年七月,徐州、邳州饑。



 皇慶元年六月,鞏昌、河州路饑。二年三月,晉寧、大同、大寧、四川、鞏昌、甘肅等郡饑。四月,真定、保定、河間等路饑。五月,順德、冀寧二路饑。六月,上都饑。



 延祐元年六月,衡州饑。七月,臺州饑。十二月,歸德、汝寧、沔陽、安豐等郡饑。二年正月,晉寧、宣德、懷孟、衛輝、益都、般陽等路饑。二年十二月,漢陽路饑。三年二月,河間,濟南濱、棣等處饑。四月,遼陽蓋州及南豐州饑。五月,寶慶、桂陽、澧州、潭州、永州、道州、袁州饑。四年正月,汴梁饑。五年四月,上都饑。六年八月,山東濟寧饑。七年五月,大同、雲內、豐、勝諸郡邑饑,沈陽路饑。八月,廣東新州新興縣饑。



 至治元年正月,蘄州蘄水縣饑。二月,河南汴梁、歸德、安豐等路饑。五月,膠州、濮州饑。七月,南恩、新州饑。十一月,鞏昌成州饑。十二月,慶遠、真定二路饑。二年三月,河南、淮東、淮西諸郡饑,延安延長、宜川二縣饑,奉元路饑。四月,東昌、霸州饑。九月,臨安河西縣饑。三年二月,京師饑。三月,平江嘉定州饑。崇明、黃巖二州饑。十一月,鎮江丹徒、沅州黔陽縣饑。十二月,歸、澧二州饑。



 泰定元年正月,惠州、新州、南恩州,信州上饒縣,廣德路廣德縣,岳州臨湘、華容等縣饑。二月,慶元、紹興二路,綏德州米脂、清澗二縣饑。三月,臨洮狄道縣、石州離石縣饑。四月,江陵、荊門軍、監利縣饑。五月,贛州、吉安、臨江等郡,昆山、南恩等州饑。八月,冀寧、延安、江州、安陸、杭州、建昌、常德、全州、桂陽、辰州、南安等路屬州縣饑。九月,紹興、南康二路饑。十一月,泉州饑,中牟、延津二縣饑。二年正月,梅州饑,祿施、英德二州饑。閏正月,河間、真定、保定、瑞州四郡饑。二月,鳳翔路饑。三月,薊、漷、徐、邳等州饑,濟南、肇慶、江州、惠州饑。四月,杭州、鎮江、寧國、南安、潯州、潭州等路饑。五月,廣德、袁州、撫州饑。六月,寧夏路饑。九月,瓊州、成州饑,德慶路饑。十二月,濟南、延川等郡饑。三年三月,河間、保定、真定三路饑。三月,大都、永平、奉元饑。十一月,沈陽、大寧、永平、廣寧,金、復州,甘肅亦集乃路饑。四年正月,遼陽諸郡饑。



 二月,奉符、長清、萊蕪三縣饑,建康、淮安、蘄州屬縣饑。四月,通、薊等州,漁陽、永清等縣饑。七月,武昌江夏縣饑。



 致和元年二月,乾州饑。三月,晉寧、冀寧、奉元、延安等路饑。四月,保定、東昌、般陽、彰德、大寧五路屬縣饑。五月,河南、東平、大同等郡饑。七月,威寧、長安縣、涇州靈臺縣饑。



 天歷二年正月,大同及東勝州饑,涿州房山、範陽等縣饑。四月,奉元耀州、乾州、華州及延安、邠、寧諸縣饑,流民數十萬。大都、興和、順德、大名、彰德、懷慶、衛輝、汴梁、中興等路,泰安、高唐、曹、冠、徐、邳等州饑。江東、浙西二道饑。八月,忻州饑。十月,漢陽、武昌、常德、澧州等路饑,鳳翔府大饑。三年正月,寧海州文登、牟平縣饑,懷慶、衡州二路饑,真定、汝寧、揚、廬、蘄、黃、安豐等郡饑。二月,河南大饑。三月,東昌須城、堂邑縣饑。沂、莒、膠、密、寧海五州,臨清、定陶、光山等縣饑。鞏昌蘭州、定西州饑。四月,德州清平縣饑。



 至順二年二月,集慶、嘉興二郡及江陰州饑,檀、順、維、密、昌平五州饑。六月,興和路高原,咸平等縣饑。九月,思州鎮遠府饑。十二月,河南大饑。三年四月,大理、中慶路饑。五月,常寧州饑。七月,勝州饑。八月,大都寶坻縣饑。



 至大元年春,紹興、慶元、臺州疫死者二萬六千餘人。



 皇慶二年冬,京師大疫。《唐志》云:「國將有恤,則邪亂之氣先被於民,故疫。」



 太宗五年癸巳十二月,大風霾,凡七晝夜。



 至元二十年正月,汴梁延津、封丘二縣大風,麥苗盡拔。



 延祐七年八月,延津縣大風,晝晦,桑隕者十八九。



 至治元年三月,大同路大風,走沙土,壅沒麥田一百餘頃。三年三月,衛輝路大風,桑雕蠶死。



 泰定三年七月,寶坻、房山二縣大風折木。八月,大都昌平等縣大風一晝夜,壞民居九百餘家。四年五月,衛輝路輝州大風九日,禾盡偃。



 天歷三年二月,胙城縣、新鄉縣大風。



 按《漢志》云:「溫而風則生螟丱,有裸蟲之孽。」



 至元八年六月,遼州和順縣、解州聞喜縣虸蚄生。十八年,高唐、夏津、武城縣蟊。二十三年五月,霸州、漷州蝻。二十四年,鞏昌虸蚄為災。二十七年四月,婺州螟害稼,雷雨大作,螟盡死,歲乃大稔。



 元貞元年六月,利州龍山縣、蓋州明山縣螟。二年五月,濟州任城縣螟。隨州野蠶成繭,亙數百里,民取為纊。



 大德七年五月,濟南、東昌、般陽、益都等路蟲食麥。閏五月,汴梁開封縣蟲食麥。九年七月,桂陽郡蝝。



 至大元年五月,東平、東昌、益都等郡蝝。



 皇慶二年五月,檀州及獲鹿縣蝻。



 延祐七年七月,霸州及堂邑縣蝻。



 泰定四年七月,奉元路咸陽、興平、武功三縣,鳳翔府岐山等縣虸蚄害稼。



 天歷二年,淮安、廬州、安豐三路屬縣蝻。



 至元十六年四月,益都樂安縣硃五十家,牛生牸犢,兩頭四耳三尾,其色黃,既生即死。



 大德九年二月,大同平地縣迷兒的斤家,牛生麒麟而死。



 至大四年,大同宣寧縣民滅的家,牛生一犢,其質有鱗無毛,其色青黃,類若麟者,以其郭上之。



 泰定三年九月,湖州長興州民王俊家,牛生一獸,鱗身牛尾,口目皆赤,墮地即大鳴,母不乳之。具圖以上,不知何獸,或曰:「此瑞也,宜俾史臣紀錄。」



 至元二十四年,諸王薛徹都部雨土七晝夜,沒死牛畜。



 大德十年二月,大同平地縣雨沙黑霾,斃牛馬二千。



 至治三年二月丙戌,雨土。



 致和元年三月壬申,雨霾。



 天歷二年三月丁亥,雨土霾。



 至順二年三月丙戌,雨土霾。



 至元二十一年九月戊子,京師地震。按《傳》云:「陽伏而不能出,陰迫而不能升,於是有地震。」二十六年正月丙戌,地震。二十七年二月癸未,泉州地震。丙戌,泉州地復震。八月癸未,武平路地大震。二十八年八月己丑,平陽路地震,壞廬舍萬八百區。



 元貞元年三月壬戌,地震。



 大德六年十二月辛酉,雲南地震,戊辰亦如之。七年八月辛卯夕,地震,太原、平陽尤甚,壞官民廬舍十萬計。平陽趙城縣範宣義郇堡徙十餘里。太原徐溝、祁縣及汾州平遙、介休、西河、孝義等縣地震成渠,泉湧黑沙。汾州北城陷,長一里,東城陷七十餘步。八年正月,平陽地震不止。九年四月己酉,大同路地震,有聲如雷,壞廬舍五千八百,壓死者一千四百餘人。懷仁縣地震,二所湧水盡黑,其一廣十八步,深十五丈,其一廣六十六步,深一丈。五月癸亥,以地震,改平陽路為晉寧,太原路為冀寧。十一月壬子,大同地震。十二月丙子,地震。十年正月,晉寧、冀寧地震不止。十一年三月,道州營道縣暴雨,山裂百三十餘處。八月壬寅,開成路地震。



 至大元年六月丁酉,鞏昌隴西、寧遠縣地震。雲南烏撒、烏蒙地三日而大震者六。九月己酉,蒲縣地震。十月癸巳,蒲縣、靈縣地震。二年十二月壬戌,陽曲縣地震有聲。三年十二月戊申,冀寧路地震。四年三月己亥,寧夏路地震。七月癸未,甘州地震,大風,有聲如雷。閏七月甲子,寧夏地震。



 皇慶二年六月,京師地震。己未,京師地震,丙寅又震,壬寅又震。



 延祐元年二月戊辰,大寧路地震。四月甲申朔,大寧地震,有聲如雷。八月丁未。冀寧、汴梁等路,涉縣、武安縣地震。十一月戊辰,大寧地震如雷。二年五月乙丑,秦州成紀縣北山移至夕川河,明日再移,平地突如土阜,高者二三丈,陷沒民居。三年八月己未,冀寧、晉寧等郡地震。十月壬午,河南地震。四年正月壬戌,冀寧地震。七月己丑,成紀縣山崩。辛卯,冀寧地震。九月,嶺北地震三日。五年正月甲戌,懿州地震。二月癸巳,和寧路地震。丁酉,秦安縣山崩。三月己卯,德慶路地震。七月戊子,寧遠縣山崩。八月,伏羌縣山崩。秦州成紀縣暴雨,山崩,朽壤墳起,覆沒畜產。



 至治二年九月癸亥,地震。十一月癸卯,地震。



 泰定元年八月,成紀縣大雨,山崩水溢,壅土至來穀河成丘阜。十二月庚申,奉元路同州地震,有聲如雷。三年十二月丁亥,寧夏路地震如雷,發自西北,連震者三。四年三月癸卯,和寧路地震如雷。八月,鞏昌通渭縣山崩。碉門地震,有聲如雷,晝晦。鳳翔、興元、成都、陜州、江陵等郡地同日震。九月壬寅,寧夏地震。



 致和元年七月辛酉朔,寧夏地震。己卯,大寧路地震。十月壬寅,大寧路地震。



 至順二年四月丁亥,真定涉縣地一日五震或三震,月餘乃止。四年四月戊申,大寧路地震。五月戊寅,京師地震有聲。八月己酉,隴西地震。



 至元元年十月壬子,恩州歷亭縣進嘉禾,一莖九穗。十一月丁酉,太原臨州進嘉禾二莖。四年十月庚午,太原進嘉禾二本,異畝同穎。六年九月癸丑,恩州進嘉禾,一莖三穗。七年夏,東平府進瑞麥,一莖五穗。十一年,興元鳳州進麥,一莖七穗;穀一莖三穗。十四年八月,嘉禾生襄陽。十七年十月,太原堅州進嘉禾六莖。十八年八月壬寅,瓜州屯田進瑞麥,一莖五穗。二十年十月癸巳,斡端宣慰司劉恩進嘉禾,同穎九穗、七穗、六穗者各一。二十三年五月,廣元路閬中麥秀兩岐。二十四年八月,濬州進瑞麥,一莖五穗。二十五年八月,袁州萍鄉縣進嘉禾。二十六年十二月,寧州民張安世進嘉禾二本。三十一年,嘉禾生京畿,一莖九穗。



 大德元年十一月辛未,曹州禹城縣進嘉禾,一莖九穗。大德九年,嘉禾生應州山陰縣。



 至大三年九月,河間路獻嘉禾,有異畝同穎及一莖數穗者,敕繪為圖。



 皇慶二年八月,嘉禾生渾源州,一莖四穗。



 延祐四年七月,南城產嘉禾。七年五月,鄱陽進嘉禾,一莖六穗。



 至治二年八月,蔚州獻嘉禾。



 泰定元年十月,成都縣穀一莖九穗。



\end{pinyinscope}