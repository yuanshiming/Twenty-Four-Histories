\article{本紀第三十二 文宗一}

\begin{pinyinscope}

 文宗聖明元孝皇帝,諱圖帖睦爾,武宗之次子,明宗之弟也。母曰文獻昭聖皇后,唐兀氏。大德三年分》。,武宗總兵北邊,帝以八年春正月癸亥生。



 十一年,武宗入繼大統。至大四年,武宗崩,傳位於弟仁宗。延祐三年,丞相鐵木迭兒等議立英宗為皇太子,明宗以武宗長子,乃出之,居於朔漠。及英宗即位,鐵木迭兒復為丞相,懷私固寵,構釁骨肉,諸王大臣,莫不自危。至治元年五月,中政使咬住告脫歡察兒等交通親王,於是出帝居於海南。三年六月,英宗在上都,謂丞相拜住曰:「朕兄弟實相友愛,曩以小人譖訴,俾居遠方,當亟召還,明正小人離間之罪。」未幾,鐵失、也先鐵木兒等為逆,而晉王遂立為皇帝,改元泰定。召帝於海南之瓊州,還至潭州,復命止之。居數月,乃還京師。十月,封懷王,賜黃金印。二年正月,又命出居於建康,以殊祥院使也先捏掌其衛士。初,晉王既為皇帝,以內史倒剌沙為中書平章政事,遂為丞相,狡愎自用,災異數見,而帝兄弟播越南北,人心思之。



 致和元年春,大駕出畋柳林,以疾還宮。諸王滿禿、阿馬剌臺,太常禮儀使哈海,宗正扎魯忽赤闊闊出等,與僉樞密院事燕鐵木兒謀曰:「今主上之疾日臻,將往上都。如有不諱,吾黨扈從者執諸王、大臣殺之。居大都者,即縛大都省、臺官,宣言太子已至,正位宸極,傳檄守禦諸關,則大事濟矣。」



 三月,大駕至上都,滿禿、闊闊出等扈從。西安王阿剌忒納失里居守,燕鐵木兒亦留大都。時也先捏私至上都,與倒剌沙等圖弗利於帝,乃遣宗正扎魯忽赤雍古臺遷帝居江陵。



 七月庚午,泰定皇帝崩於上都。倒剌沙及梁王王禪、遼王脫脫,因結黨害政,人皆不平。時燕鐵木兒實掌大都樞密符印,謀於西安王阿剌忒納失里,陰結勇士,以圖舉義。八月甲午,黎明,百官集興聖宮,燕鐵木兒率阿剌鐵木兒、孛倫赤等十七人,兵皆露刃,號於眾曰:「武宗皇帝有聖子二人,孝友仁文,天下正統當歸之。今爾一二臣,敢紊邦紀,有不順者斬!」乃手縛平章政事烏伯都剌、伯顏察兒,分命勇士執中書左丞朵朵,參知政事王士熙,參議中書省事脫脫、吳秉道,侍御史鐵木哥、丘世傑,治書侍御史脫歡,太子詹事丞王桓等,皆下之獄。燕鐵木兒與西安王阿剌忒納失里共守內廷,籍府庫,錄符印,召百官入內聽命。即遣前河南行省參知政事明裏董阿、前宣政使答里麻失里,馳驛迎帝於江陵,密以意諭河南行省平章政事伯顏,令簡兵以備扈從。是日,前湖廣行省左丞相別不花為中書左丞相,太子詹事塔失海涯為中書平章政事,前湖廣行省右丞速速為中書左丞,前陜西行省參知政事王不憐吉臺為樞密副使,與中書右丞趙世延、同僉樞密院事燕鐵木兒、翰林學士承旨亦列赤、通政院使寒食分典機務,調兵守禦關要,徵諸衛兵屯京師,下郡縣造兵器,出府庫犒軍士。燕鐵木兒直宿禁中,達旦不寐,一夕或再徙,人莫知其處。乙未,以西安王令,給宿衛京城軍士鈔有差,調諸衛兵守居庸關及盧兒嶺。丙申,遣左衛率使禿魯將兵屯白馬甸,隆鎮衛指揮使斡都蠻將兵屯泰和嶺。丁酉,發中衛兵守遷民鎮。又遣撒里不花等往迎帝,且令塔失帖木兒矯為使者自南來,言帝已次近郊,使民毋驚疑。戊戌,徵宣靖王買奴、諸王燕不花於山東。己亥,徵兵遼陽。明裏董阿至汴梁,執行省臣,皆下之獄,又收肅政廉訪司、萬戶府及郡縣印。庚子,發宗仁衛兵增守遷民鎮。辛丑,遣萬戶徹里帖木兒將兵屯河中。壬寅,河南行省以郡縣闕人,權署官攝其事。癸卯,燕鐵木兒之弟撒敦、子唐其勢,自上都來歸。河南行省殺平章曲烈、右丞別鐵木兒。是日,明裏董阿等至江陵。甲辰,帝發江陵,遣使召鎮南王鐵木兒不花、威順王寬徹不花、湖廣行省平章政事高昌王鐵木兒補化來會。執湖廣行省左丞馬合某送京師,以別薛代之。河南行省出府庫金千兩、銀四千兩、鈔七萬一千錠,分給官吏、將士。又命有司造乘輿、供張、儀仗等物。乙巳,遣隆鎮衛指揮使也速臺兒將兵守碑樓口。河南行省殺其參政脫孛臺。召陜西行臺侍御史馬扎兒臺及行省平章政事探馬赤,不至。丙午,諸王按渾察至京師。遣前西臺御史剌馬黑巴等諭陜西。丁未,撒敦守居庸關,唐其勢屯古北口。命河南行省造銀符,以給軍士有功者。戊申,燕鐵木兒又令乃馬臺矯為使者北來,言周王整兵南行,聞者皆悅。帝命河南行省平章政事伯顏為本省左丞相。河南行省遣前萬戶孛羅等將兵守潼關。己酉,諸王滿禿、阿馬剌臺,宗正扎魯忽赤闊闊出,前河南行平章政事買閭,集賢侍讀學士兀魯思不花,太常禮儀院使哈海赤等十八人,同謀援大都,事覺,倒剌沙殺之。庚戌,帝至汴梁,伯顏等扈從北行。以前翰林學士承旨阿不海牙為河南行省平章政事。發平灤民塹遷民鎮,以御遼東軍。辛亥,以燕鐵木兒知樞密院事,亦列赤為御史中丞。壬子,阿速衛指揮使脫脫木兒帥其軍自上都來歸,即命守古北口。癸丑,鑄樞密分院印。是日,上都諸王及用事臣,以兵分道犯京畿,留遼王脫脫、諸王孛羅帖木兒、太師朵帶、左丞相倒剌沙、知樞密院事鐵木兒脫居守。甲寅,剌馬黑巴等至陜西,皆見殺。乙卯,脫脫木兒及上都諸王失剌、平章政事乃馬臺、詹事欽察戰於宜興,斬欽察於陣,禽乃馬臺送京師,戮之,失剌敗走。丙辰,燕鐵木兒奉法駕郊迎。丁巳,帝至京師,入居大內。貴赤衛指揮使脫迭出自上都,率其軍來歸,命守古北口。戊午,以速速為中書平章政事,前御史中丞曹立為中書右丞,江浙行省參知政事張友諒為中書參知政事,河南行省左丞相伯顏為御史大夫,中書右丞趙世延為御史中丞。己未,以河南萬戶也速臺兒同知樞密院事。罷回回掌教哈的所。上都梁王王禪、右丞相塔失鐵木兒、太尉不花、平章政事買閭、御史大夫紐澤等,兵次榆林。升宜興縣為州。隆鎮衛指揮使黑漢謀附上都,坐棄市,籍其家。九月庚申朔,燕鐵木兒督師居庸關,遣撒敦以兵襲上都兵於榆林,擊敗之,追至懷來而還。隆鎮衛指揮使斡都蠻以兵襲上都諸王滅里鐵木兒、脫木赤於陀羅臺,執之,歸於京師。遣使即軍中賜脫脫木兒等銀各千兩,以分給軍士有功者。賜京師耆老七十人幣帛。命有司括馬。中書左丞相別不花言:「回回人哈哈的,自至治間貨官鈔,違制別往番邦,得寶貨無算,法當沒官,而倒剌沙私其種人,不許,今請籍其家。」從之。燕鐵木兒請釋馬合某,從之。陜西兵入河中府,劫行用庫鈔萬八千錠,殺同知府事不倫禿。壬戌,遣使祭五岳、四瀆。命速速宣諭中外曰:「昔在世祖以及列聖臨御,咸命中書省綱維百司,總裁庶政,凡錢穀、銓選、刑罰、興造,罔不司之。自今除樞密院、御史臺,其餘諸司及左右近侍,敢有隔越中書奏請政務者,以違制論,監察御史其糾言之。」以高昌王鐵木兒補化知樞密院事,也先捏為宣徽使。給居庸關軍士糗糧,賜鎮南王鐵木兒不花等鈔有差。征五衛屯田兵赴京師。安南國來貢方物。賜上都將士來歸者鈔各有差。樞密院臣言:「河南行省軍列戍淮西,距潼關、河中不遠,湖廣行省軍,唯平陽、保定兩萬戶號稱精銳,請發蘄、黃戍軍一萬人及兩萬戶軍,為三萬,命湖廣參政鄭昂霄、萬戶脫脫木兒將之,並黃河為營,以便徵遣。」從之。召燕鐵木兒赴闕。上都諸王也先帖木兒、平章禿滿迭兒,自遼東以兵入遷民鎮,諸王八剌馬、也先帖木兒以所部兵入管州,殺掠吏民。丙寅,命造兵器,江浙、江西、湖廣三省六萬事,內郡四萬事。丁卯,燕鐵木兒率諸王、大臣伏闕請早正大位,以安天下,帝固辭曰:「大兄在朔方,朕敢紊天序乎?」燕鐵木兒曰:「人心向背之機,間不容發,一或失之,噬臍無及。」帝曰:「必不得已,必明著朕意以示天下而後可。」賜西安王阿剌忒納失里、鎮南王帖木兒不花、威順王寬徹不花、宣靖王買奴等,金各五十兩、銀各五百兩、幣各三十匹。遣撒敦拒遼東兵於薊州東流沙河,元帥阿兀剌守居庸關。上都軍攻碑樓口,指揮使也速臺兒御之,不克。戊辰,大司農明裏董阿、大都留守闊闊臺,並為中書平章政事。募勇士從軍。遣使分行河間、保定、真定及河南等路,括民馬。征鄢陵縣河西軍赴闕。命襄陽萬戶楊克忠、鄧州萬戶孫節,以兵守武關。命海道萬戶府來年運米三百一十萬石。造金符八十。己巳,鑄御寶成。立行樞密院於汴梁,以同知樞密院事也速臺兒知行樞密院事,將兵行視太行諸關,西擊河中、潼關軍,以折疊弩分給守關軍士。上都諸王忽剌臺等引兵犯崞州。庚午,命有司和市粟豆十六萬五千石,分給居庸等關軍馬。遣軍民守歸、峽諸隘。辛未,常服謁太廟。雲南孟定路土官來貢方物。烏伯都剌、鐵木哥棄市,朵朵、王士熙、伯顏察兒、脫歡等各流於遠州,並籍其家。同知樞密院事脫脫木兒與遼東禿滿迭兒戰於薊州兩家店。壬申,帝即位於大明殿,受諸王、百官朝賀,大赦,詔曰:



 洪惟我太祖皇帝混一海宇,爰立定制,以一統緒,宗親各受分地,勿敢妄生覬覦,此不易之成規,萬世所共守者也。世祖之後,成宗、武宗、仁宗、英宗,以公天下之心,以次相傳,宗王、貴戚,咸遵祖訓。至於晉邸,具有盟書,願守籓服,而與賊臣鐵失、也先帖木兒等潛通陰謀,冒干寶位,使英宗不幸罹於大故。朕兄弟播越南北,備歷艱險,臨御之事,豈獲與聞!



 朕以叔父之故,順承惟謹,於今六年,災異迭見。權臣倒剌沙、烏伯都剌等,專權自用,疏遠勛舊,廢棄忠良,變亂祖宗法度,空府庫以私其黨類。大行上賓,利於立幼,顯握國柄,用成其奸。宗王、大臣,以宗社之重,統緒之正,協謀推戴,屬於眇躬。朕以菲德,宜俟大兄,固讓再三。宗戚、將相,百僚、耆老,以為神器不可以久虛,天下不可以無主,周王遼隔朔漠,民庶遑遑,已及三月,誠懇迫切。朕故從其請,謹俟大兄之至,以遂朕固讓之心。已於致和元年九月十三日,即皇帝位於大明殿,其以致和元年為天歷元年,可大赦天下。自九月十三日昧爽已前,除謀殺祖父母、父母,妻妾殺夫,奴婢殺主,謀故殺人,但犯強盜,印造偽鈔不赦外,其餘罪無輕重,咸赦除之。



 於戲,朕豈有意於天下哉!重念祖宗開創之艱,恐隳大業,是以勉徇輿情。尚賴爾中外文武臣僚,協心相予,輯寧億兆,以成治功。咨爾多方,體予至意!



 癸酉,翰林院增給驛璽書。命燕鐵木兒將兵擊遼東軍,封燕鐵木兒為太平王,以太平路為食邑,賜金五百兩、銀二千五百兩、鈔萬錠、平江官地五百頃。中書右丞曹立為江浙行省平章政事,福建廉訪使易釋董阿為右丞,前中書左丞張思明為左丞。諸王塔術、只兒哈郎、佛寶等自恩州來朝。賜按灰鈔百錠,以祀天神。括河東馬。甲戌,燕鐵木兒加開府儀同三司、上柱國、錄軍國重事、中書右丞相、監修國史,依前知樞密院事,伯顏加太尉,以江南行臺御史大夫朵兒只為江浙行省左丞相,淮西道肅政廉訪使阿兒思蘭海牙為江南行臺御史大夫。諸王孛羅、忽都火者來朝。徵左右兩阿速衛軍老幼赴京師,不行者斬,籍其家。乙亥,立太禧院,以奉祖宗神御殿祠祭,秩正二品,罷會福、殊祥兩院。江西行省平章政事禿堅帖木兒、江浙行省右丞易釋董阿並為太禧院使,中書平章速速、御史中丞亦列赤兼太禧院使。上都王禪兵襲破居庸關,將士皆潰。燕鐵木兒軍次三河。丙子,王禪游兵至大口,燕鐵木兒還軍次榆河,帝出齊化門視師。丁丑,燕鐵木兒來見曰:「乘輿一出,民心必驚,軍旅之事,臣請以身任之。」即日還宮。命司天監翽星。戊寅,諭中外曰:「近以奸臣倒剌沙、烏伯都剌潛通陰謀,變易祖宗成憲,既已明正其罪。凡回回種人不預其事者,其安業勿懼;有因而扇惑其人者,罪之。」又敕:「軍中逃歸,及京城游民敢攘民財者斬。」命高昌僧作佛事於延春閣。又命也裏可溫於顯懿莊聖皇后神御殿作佛事。諸王阿兒八忽、按灰、脫脫來朝。命留守司完京城,軍士乘城守御。燕鐵木兒與王禪前軍戰於榆河,敗之,追奔紅橋北。其樞密副使阿剌帖木兒、指揮使忽都帖木兒以兵會王禪,復來戰,又敗之,我師據紅橋。增給大都驛馬百匹。庚辰,太白犯亢宿。詔諭御史臺:「今後監察御史、廉訪司,凡有刺舉,並著其實,無則勿妄以言。廉訪司書吏,當以職官、教授、吏員、鄉貢進士參用。」加封漢將軍關羽為顯靈義勇武安英濟王,遣使祠其廟。辛巳,命司天監翽星。以別不花知樞密院事,依前中書左丞相。括山東馬。燕鐵木兒與上都軍大戰白浮之野,燕鐵木兒手刃七人於陣,敗之。脫脫木兒與遼東軍戰薊州之檀子山。壬午,大霧。王禪等遁昆山州。獲上都頒詔使者及遼東征兵使者以聞,詔誅之。癸未,以同知樞密院事禿兒哈帖木兒知樞密院事,中書平章政事明裏董阿為江浙行省平章政事。王禪收集散亡,復來戰,我師列陣白浮之西,敵不敢犯。至夜,撒敦、脫脫木兒前後夾攻,敗走之,追及於昌平北,斬首數千級,降者萬餘人。帝遣使賜燕鐵木兒上尊,諭旨曰:「丞相每臨陣,躬冒矢石,脫有不虞,奈何?自今第以大將旗鼓督戰可也。」燕鐵木兒對曰:「凡戰,臣必以身先之,敢後者,論以軍法。若委之諸將,萬一失利,悔將何及?」甲申,慶雲見。王禪單騎亡,撒敦追之不及而還。命御史臺:「凡各道廉訪司官,用蒙古二人,畏兀、河西、回回、漢人各一人。各司書吏十六人,用職官五,各路司吏五,教授二,鄉貢進士四人。本臺經歷品秩相當者,除各道廉訪使,都事除副使。今臺譯史通事考滿不得除御史。」靖安王闊不花等將陜西兵潛由潼關南水門入,萬戶孛羅棄關走,闊不花等分據陜州等縣,縱兵四劫。乙酉,以明裏董阿為中書平章政事,嶺北行省左丞燕不鄰知樞密院事。募丁壯千人守捍城郭。上都兵入古北口,將士皆潰,其知樞密院事竹溫臺以兵掠石槽。追封乳母完者云國夫人,其夫乾羅思贈太保,封云國公,謚忠懿;子鎖乃贈司徒,封云國公,謚貞閔。燕鐵木兒遣撒敦倍道趨石槽,掩其不備,擊之。燕鐵木兒大兵繼至,轉戰四十餘里,至牛頭山,擒駙馬孛羅帖木兒,平章蒙古塔失、雅失帖木兒,將作院使撒兒討溫,送闕下戮之,將校降者萬人,餘兵奔竄,夜遣撒敦出古北口逐之。脫脫木兒與遼東軍戰薊州南,殺獲無算。調河南蒙古軍老幼五萬人,增守京師,募丁壯守直沽。調臨清萬戶府運糧軍三千五百並御河分守,山東丁壯萬人守禦益都、般陽諸處海港。居庸關壘石以為固。丁亥,遼東軍抵京城,燕鐵木兒引兵拒之,令京城裏長召募丁壯及百工合萬人,與兵士為伍,乘城守御,月給鈔三錠、米三斗。冀寧、晉寧兩路所轄:代州之雁門關,崞州之陽武關,嵐州之天澗口、皮庫口,保德州之寨底、天橋、白羊三關,石州之塢堡口,汾州之向陽關,隰州之烏門關,吉州之馬頭、秦王嶺二關,靈石縣之陰地關,皆令穿塹壘石以為固,調丁壯守之。戊子,上都諸王忽剌臺等兵入紫荊關,將士皆潰,行樞密院官卜顏、斡都蠻,指揮使也速臺兒將兵援之。陜西行臺御史大夫也先帖木兒引兵從大慶關渡河,擒河中府官,殺之。萬戶徹里帖木兒軍潰而遁,河南廉訪副使萬家閭言:「徹里帖木兒身為大將,紀律不嚴,望風奔潰,宜加重罰,以示勸懲。」不報。河東聞也先帖木兒軍至,官吏皆棄城走,也先帖木兒悉以其黨代之。召雲南行省左丞相也兒吉尼,不至。前尚書左丞相三寶奴以罪誅,其二子上都、哈剌八都兒近侍,命以所籍家貲及制命還之。



 冬十月己丑朔,命西僧作佛事。燕鐵木兒引兵至通州,擊遼東軍敗之,皆渡潞水走。遣脫脫木兒等將兵四千,西援紫荊關。調江浙兵萬人,西御潼關。紫荊關潰卒南走保定,因肆剽掠,同知路事阿里沙及故平章張珪子武昌萬戶景武等率民持挺擊死數百人。河南行省調兵守虎牢關。庚寅,我師與遼東軍夾潞水而陣,遼東軍宵遁,我師渡而襲之。辛卯,禮官言:「即位之始,當告祭郊廟、社稷,時享之禮,請改用仲月。」從之。紫荊關兵進逼涿州,同知州事教化的調丁壯御之。壬辰,也先捏以軍至保定,殺阿里沙等及張景武兄弟五人,並取其家貲。倒剌沙貸其姻家長蘆鹽運司判官亦剌馬丹鈔四萬錠,買鹽營利於京師,詔追理之。癸巳,立壽福、會福、隆禧、崇祥四總管府,分奉祖宗神御殿,秩正三品,並隸太禧院。忽剌臺游兵進逼南城,令京城居民戶出壯丁一人,持兵仗從軍士乘城,仍於諸門列甕貯水以防火。燕鐵木兒及陽翟王太平、國王朵羅臺等戰於檀子山之棗林,唐其勢陷陣,殺太平,死者蔽野,餘皆宵遁,遣撒敦追之,弗及。甲午,命有司市馬千匹,賜軍士出征者。脫脫木兒、章吉與也先捏合擊敵軍於良鄉南,轉戰至瀘溝橋,忽剌臺被創,據橋而宿。乙未,燕鐵木兒率軍循北山而西,趣良鄉,諸將時與忽剌臺、阿剌帖木兒等戰於瀘溝橋,聲言燕鐵木兒大軍至,敵兵皆遁。使者頒詔於甘肅,至陜西,行省、行臺官塗毀詔書,械使者送上都。湘寧王八剌失裡引兵入冀寧,殺掠吏民。時太行諸關守備皆闕,冀寧路來告急,敕萬戶和尚將兵由故闕援之。冀寧路官募民丁迎敵,和尚以兵為殿,殺獲甚眾。會上都兵大至,和尚退保故關,冀寧遂破。丙申,燕鐵木兒入朝,賜宴興聖殿。賑通州被兵之家。命速速等董度支芻粟。中書省臣言:「上都諸王、大臣,不思祖宗成憲,惑於奸臣倒剌沙之言,輒以兵犯京畿。賴陛下洪福,王禪遂致潰亡,生擒諸王孛羅帖木兒及諸用事臣蒙古答失、雅失帖木兒等,既已明正典刑,宜傳首四方以示眾。」從之。丁酉,以縉山縣民十人嘗為王禪向導,誅其為首者四人,餘杖一百七,籍其家貲,妻子分賜守關軍士。戊戌,命湖廣行省平章政事乞住調兵守歸、峽,左丞別薛守八番,以御四川軍。諸將追阿剌帖木兒等至紫荊關,獲之,送京師,皆棄市。己亥,幸大聖壽萬安寺,謁世祖、裕宗神御殿。賜燕鐵木兒太平王黃金印,並降制書,及賜玉盤、龍衣、珠衣、寶珠、金腰帶、海東白鶻青鶻各一。河南行中書省、行樞密院,皆聽便宜行事。禿滿迭兒軍復入古北口,燕鐵木兒引軍御之,大戰於檀州南,敗之,其萬戶以兵萬人降,禿滿迭兒遂走還遼東。使者頒詔於陜西,行省、行臺官焚詔書,下使者獄,告於上都。庚子,以梁王王禪第賜諸王帖木兒不花。廷臣言:「保定萬戶張昌,其諸父景武等既受誅,宜罷其所將兵,而奪其金虎符。」不許。辛丑,以同知樞密院事脫脫木兒、通政使也不倫並知樞密院事,御史中丞亦列赤為御史大夫。還給伯顏察兒、朵朵家貲。齊王月魯帖木兒、東路蒙古元帥不花帖木兒等以兵圍上都,倒剌沙等奉皇帝寶出降。梁王王禪遁,遼王脫脫為齊王月魯帖木兒所殺,遂收上都諸王符印。壬寅,以宣徽使也先捏知行樞密院事,宣徽副使章吉為行樞密院副使,與知樞密院事也速臺兒等將兵西擊潼關軍。中書省臣言:「野理牙舊以贓罪除名,近復命為太醫使,臣等不敢奉詔。」帝曰:「往者勿咎,比兵興之時,朕已錄用,其依朕命行之。」以張珪女歸也先捏。癸卯,以故徽政使失烈門妻賜燕鐵木兒。以通州知州趙義能禦敵,賜幣二匹。也先鐵木兒軍至晉寧,本路官皆遁。甲辰,晉邸及遼王所轄路、府、州、縣達魯花赤並罷免禁錮,選流官代之。給淮東宣慰司銀字圓符。命有司收將士所遺符印、兵仗。賑糶京城米十萬石,石為鈔十五貫。丙午,中書省臣言:「凡有罪者,既籍其家貲,又沒其妻子,非古者罪人不孥之意,今後請勿沒人妻子。」制可。丁未,告祭於南郊。以中書平章政事塔失海涯為大司農,復以欽察臺為中書平章政事,侍御史玥璐不花為中丞。以度支芻豆經用不足,凡諸王、駙馬來朝並節其給,宿衛官已有廩祿者及內侍宮人歲給芻豆,皆權止之。糴豆二十萬石於瀕御河州縣,以河間、山東鹽課鈔給其直。放還防河運糧軍。陜西兵至鞏縣黑石渡,遂據虎牢,我師皆潰,儲仗悉為所獲。河南行省來告急,戒有司修城壁,嚴守衛。雲南銀沙羅甸土官哀贊等來貢方物。己酉,別不花加太保,落知樞密院事。命刑部郎中大都、前廣東僉事張世榮追理烏伯都剌家貲。開居庸關。陜西軍奪武關,萬戶楊克忠等兵潰。庚戌,帝御興聖殿,齊王月魯帖木兒、諸王別思帖木兒、阿兒哈失里、那海罕及東路蒙古元帥不花帖木兒等奉上皇帝寶。倒剌沙等從至京師,下之獄,分遣使者檄行省、內郡罷兵以安百姓。以宦者伯帖木兒妻及奴婢田宅賜撒敦。辛亥,雲南徹里路土官刁賽等來貢方物。詔:「自今朝廷政務及籍沒田宅賜人者,非與燕鐵木兒議,諸人不許奏陳。」以宦者米薛迷奴婢家貲賜伯顏。壬子,以河南、江西、湖廣入貢駕鵝太頻,令減其數以省驛傳。以諸王火沙第賜燕鐵木兒繼母公主察吉兒。癸丑,燕鐵木兒辭知樞密院事,命其叔父東路蒙古元帥不花帖木兒代之。燕鐵木兒請以蒙古塔失等三十人田宅賜徹里鐵木兒等三十人,從之。以所括河北諸路馬,四百匹給四宿衛阿塔赤,二百匹給中宮阿塔赤,餘二千匹分牧於內郡。核上都倉庫錢穀。御史臺臣言:「近北兵奪紫荊關,官軍潰走,掠保定之民。本路官與故平章張珪子景武五人,率其民擊官軍死,也先捏不俟奏聞,輒擅殺官吏及珪五子。珪父祖三世為國勛臣,設使珪子有罪,珪之妻女又何罪焉!今既籍其家,又以其女妻也先捏,誠非國家待遇勛臣之意。」帝曰:「卿等言是。」命中書革正之。命御史臺擇人充各道廉訪司官。遣官賑良鄉、涿州、定興、保定驛戶之被兵者。甲寅,罷徽政院,改立儲慶使司,秩正二品。平章政事速速、明裏董阿並領儲慶司事,鷹坊伯撒里、河南行省左丞姚煒並為儲慶使,元帥也速答兒執湘寧王八剌失里送京師。八剌失里及趙王馬扎罕、諸王忽剌臺,承上都之命,各起所部兵南侵冀寧,還次馬邑至是被執,其所俘男女千人,悉還其家。遣使止江浙軍士之往潼關者,就還鎮。也先鐵,木兒兵至潞州。乙卯,以倒剌沙宅賜不花帖木兒,倒剌沙子潑皮宅賜斡都蠻,內侍王伯顏宅賜唐其勢。丙辰,燕鐵木兒請以所沒逆臣赤斤鐵木兒家貲還其妻。鐵木哥兵入鄧州。丁巳,毀顯宗室,升順宗祔右穆第二室,成宗祔右穆第三室,武宗祔左昭第三室,仁宗祔左昭第四室,英宗祔右穆第四室。加命燕鐵木兒為答剌罕,仍命子孫世襲其號。燕鐵木兒請以河南平章曲列等二十三人田宅賜西安王阿剌忒納失裡等二十三人,從之。戊午,詔諭廷臣曰:「凡今臣僚,唯丞相燕鐵木兒、大夫伯顏許兼三職署事,餘者並從簡省。百司事當奏者,共議以聞,或私任己意者,不許獨請。上都官吏,自八月二十一日以後擢用者,並追收其制。」敕:「天下僧道有妻者,皆令為民。」也先捏軍次順德。令廣平、大名兩路括馬。盜殺太尉不花。初,不花乘國家多事,率眾剽掠,居庸以北皆為所擾,至是盜入其家殺之。興和路當盜以死罪,刑部議以為:「不花不道,眾所聞知,幸遇盜殺,而本路隱其殘剽之罪,獨以盜聞,於法不當。」中書以聞,帝嘉其議。



 十一月己未,詔諭中外曰:「諸王王禪及禿滿迭兒、阿剌不花、禿堅等,兵敗而逃,有能擒獲者,授五品官;同黨之人,若能去逆效順,擒王禪等來歸者,免本罪,依上授官;家奴獲之者,得備宿衛;敢有隱匿者,事覺,與犯人同罪。」給殿中侍御史及冀寧路印,凡內外百司印,因兵興而失者,令中書如品秩鑄給之。命太保伯答沙升太傅,兼宗正扎魯忽赤,總兵北邊。中書省臣言:「侍御史左吉非才,不當任風憲。」御史臺臣伯顏等言:「左吉,御史所薦,若既用之,又以人言而止,臺綱不能振矣。必如省臣所言,臣等乞辭避。」帝曰:「汝等其勿為是言。左吉果不可用,省臣何不先言之。其令左吉仍為侍御史。」帝謂中書省臣曰:「朕在瓊州、建康時,撒迪皆從,備極艱苦,其賜鹽引六萬,俾規利以贍其家。」命郡縣招集被兵流亡之民,貧者賑給之。遼東降軍,給行糧遣還。京畿及四方民為兵所掠而奴於人者,令有司追理送還。山北、京東驛被兵者,賑以鈔二萬一千五百錠。放高麗宦者米薛迷、剛答里歸田里。庚申,中書錄用前御史臺官亦憐真、蔡文淵。用江南行臺御史王琚仁言,汰近歲白身入官者。敕行御史臺:「凡有糾劾,必由御史臺陳奏,勿徑以封事聞。」命中書省追理倒剌沙及其兄馬某沙,子潑皮、木八剌沙等家貲。辛酉,燕鐵木兒請以紐澤田宅賜欽察臺。也先捏兵至武安,也先鐵木兒以軍降,河東州縣聞之,盡殺其所署官吏。癸亥,帝宿齋宮。甲子,服袞冕,享於太廟。陜西兵進逼汴梁,聞朝廷傳檄罷兵,乃解去。乙丑,燕鐵木兒請以烏伯都剌等三十人田宅賜斡魯思等三十人,從之。丁卯,伯顏兼忠翊侍衛都指揮使。庚午,復立察罕腦兒宣慰司。命總宿衛官分簡所募勇士,非舊嘗宿衛者皆罷去。汴梁、河南等路及南陽府頻歲蝗旱,禁其境內釀酒。日本舶商至福建博易者,江浙行省選廉吏徵其稅。中書省臣言:「今歲既罷印鈔本,來歲擬印至元鈔一百一十九萬二千錠、中統鈔四萬錠。」監察御史言:「戶部鈔法,歲會其數,易故以新,期於流通,不出其數。邇者倒剌沙以上都經費不足,命有司刻板印鈔;今事既定,宜急收毀。」從之。監察御史撒里不花、鎖南八、於欽、張士弘言:「朝廷政務,賞罰為先,功罪既明,天下斯定。國家近年自鐵木迭兒竊位擅權,假刑賞以遂其私,綱紀始紊。迨至泰定,爵賞益濫。比以兵興,用人甚急,然而賞罰不可不嚴。夫功之高下,過之重輕,皆系天下之公論。願命有司,務合公議,明示黜陟。功罪既明,賞罰攸當,則朝廷肅清,紀綱振舉,而天下治矣。」帝嘉納之。辛未,遣西僧作佛事於興和新內。鐵木哥兵入襄陽,本路官皆遁。襄陽縣尹穀庭珪、主簿張德獨不去,西軍執使降,不屈,死之。時僉樞密院事塔海擁兵南陽不救。壬申,遣官告祭社稷。以故平章黑驢平江田三百頃及嘉興蘆地賜西安王阿剌忒納失里。癸酉,八百媳婦國使者昭哀,雲南威楚路土官胒放等,九十九寨土官必也姑等,各以方物來貢。燕鐵木兒言:「向者上都舉兵,諸王失剌、樞密同知阿乞剌等十人,南望宮闕鼓噪,其黨拒命逆戰,情不可恕。」詔各杖一百七,流遠,籍其家貲。甲戌,居泰定後雍吉剌氏於東安州。杭州火,命江浙行省賑被災之家。乙亥,賜西安王阿剌忒納失里、齊王月魯帖木兒、知樞密院事不花帖木兒金各五百兩、銀各二千五百兩、鈔各萬錠,諸王朵列帖木兒金五十兩、銀五百兩、鈔千錠,從者及軍士有差。丙子,速速坐受賂,杖一百七,徙襄陽;以母年老,詔留之京師。丁丑,以躬祀太廟禮成,御大明殿,受諸王、文武百官朝賀。荊王也速也不干遣使傳檄至襄陽,鐵木哥引兵走。戊寅,以御史中丞玥璐不花為太禧使。監察御史撒里不花等言:「玥璐不花素稟直氣,操履端正,陛下欲振憲綱,非任斯人不可。」乃復以玥璐不花為中丞,兼太禧使。作佛事於五臺寺。命河南、江浙兩省以兵五萬益湖廣。己卯,中書省臣言:「內外流官年及致仕者,並依階敘授以制敕,今後不須奏聞。」制可。以也先鐵木兒、烏伯都剌珠衣賜撒迪、趙世安。諸衛漢軍及州縣丁壯所給甲胄兵仗,皆令還官。庚辰,遣使奉迎皇兄明宗皇帝於漠北。以中政院使敬儼為中書平章政事,同知樞密院事徹里帖木兒為中書左丞。辛巳,遣欽察百戶及其軍士還鎮。以脫脫等三人妻賜闊闊出等三人,以朵臺等十一人田宅賜駙馬朵必兒等十一人。壬午,第三皇子寶寧易名太平訥,命大司農買住保養於其家。詔行樞密院罷兵還。以御史中丞玥璐不花為中書右丞。癸未,倒剌沙伏誅,磔其尸於市,王禪亦賜死,馬某沙、紐澤、撒的迷失、也先鐵木兒等皆棄市。以所賜速速、也先捏宅改賜駙馬謹只兒及乳媼也孫真。甲申,命威順王寬徹不花還鎮湖廣。御史中丞趙世延以老疾辭職,不許,用故中丞崔彧故事,加平章政事居前職。御史臺臣言:「行宣政院、行都水監宜罷。」從之。丙戌,作水陸會。以阿魯灰帖木兒等六人在上都欲舉義,不克而死,並賜贈謚,恤其家。燕鐵木兒言:「晉王及遼王等所轄府縣達魯花赤既已黜罷,其所舉宗正府扎魯忽赤、中書斷事官,皆其私人,亦宜革去。」從之。敕趙世延及翰林直學士虞集制御史臺碑文。遣諸衛兵各還鎮。別不花罷。命有司追理上都官吏預借俸。遼王脫脫之子八都聚黨出剽掠,敕宣德府官捕之。四川行省平章囊加臺自稱鎮西王,以其省左丞脫脫為平章,前雲南廉訪使楊靜為左丞,殺其省平章寬徹等官,稱兵燒絕棧道。烏蒙路教授杜巖肖謂:「聖明繼統,方內大寧,省臣當罷兵入朝,庶免一方之害。」囊加臺以其妄言惑眾,杖一百七,禁錮之。



 十二月己丑朔,監察御史言,伯顏宜與燕鐵木兒一體論功行賞,帝曰:「伯顏之功,朕心知之,御史不必言。」庚寅,令內外諸司,天壽節聽具肉食,民間禁屠宰如舊制。命通政院整飭蒙古驛。諸關隘嘗毀民屋以塞者,賜民鈔,俾完之。甲午,以王禪奴婢賜鎮南王鐵木兒不花及燕鐵木兒。乙未,以王禪弓矢賜燕鐵木兒、伯顏。燕鐵木兒請以馬某沙等九人田宅賜燕不鄰等九人,從之。丙午,幸大崇恩福元寺,謁武宗神御殿。分命諸僧於大明殿、延春閣、興聖宮、隆福宮、萬歲山作佛事。雲南土官普雙等來貢方物。御史臺臣言:「也先捏將兵所至,擅殺官吏,俘掠子女貨財。」詔刑部鞫之,籍其家,杖一百七,竄於南寧,命其妻歸父母家。己亥,造皇后玉冊、玉寶。庚子,赦天下。賜諸王滿禿為果王,阿馬剌臺為毅王,宗正札魯忽赤闊闊出等十七人並賜功臣號及階官爵謚,仍命有司刻其功於碑,賜鈔恤其家。中書省臣言:「陜西行省、行臺官,焚棄詔書,坐罪當流,雖經赦宥,永不錄用為宜。」制可。辛丑,立龍翊侍衛親軍都指揮使司,分掌欽察軍士,秩正三品;指揮使三人,命燕鐵木兒及卜蘭奚、卯罕為之,餘官悉聽燕鐵木兒選人以聞。命高昌僧作佛事於寶慈殿。江南行臺御史言:「遼王脫脫,自其祖父以來,屢為叛逆,蓋因所封地大物眾,宜削王號,處其子孫遠方,而析其元封分地。」詔中書與勛舊大臣議其事。火兒忽答等十三人從湘寧王八剌失裡用兵,既伏誅,命皆籍其家貲。西僧百人作佛事於徽猷閣七日。癸卯,欽察、阿速二部,依宿衛軍士例給芻豆。乙巳,伯顏加太尉、開府儀同三司,與亦列赤並為御史大夫,同振臺綱,詔天下。立內宰司,隸儲慶使司,秩正三品。以阿伯等六人田宅賜諸王老的等六人。雲南姚州知州高明來貢方物。戊申,以潛邸所用工匠百五十人付皇子阿剌忒納答剌,立異樣局以司之,秩從六品。加伯顏為太保,知樞密院事不花帖木兒為太尉,香山為司徒。己酉,開上都酒禁。壬子,以諸路民匠提領所合為提舉司,秩從五品。甲寅,復遣治書侍御史撒迪、內侍不顏禿古思奉迎皇兄於漠北。西安王阿剌忒納失里及燕鐵木兒、鐵木兒補化,請各遣人送名鷹於行在所。以王禪妻金珠首飾歸中宮。丙辰,升太禧院從一品,中書左丞玥璐不花為太禧使。丁巳,封西安王阿剌忒納失里為豫王,賜南康路為食邑。徹里鐵木兒升右丞,參知政事躍里鐵木兒為左丞,參議省事趙世安為參知政事。戊午,詔:「被兵郡縣免雜役,禁釀酒,弛山場河濼之禁;私相假貸者,俟秋成責償。蒙古、色目人願丁父母憂者,聽如舊制。」御史臺言:「囊加臺拒命西南,罪不可宥,所授制敕,宜從追奪。」中書省臣言:「令方許囊加臺等自新,則御史言宜勿行。」從之。教坊司達魯花赤撒剌兒,在武宗時遙授參知政事,階中奉大夫,詔落遙授之職,而仍其舊階。是月,復遣使者召雲南行省左丞相也兒吉你,又不至。加謚唐司徒顏真卿正烈文忠公,令有司歲時致祭。陜西自泰定二年至是歲不雨,大饑,民相食。杭州、嘉興、平江、湖州、鎮江、建德、池州、太平、廣德等路水,沒民田萬四千餘頃。河北、山東有年。



\end{pinyinscope}