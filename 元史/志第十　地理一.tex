\article{志第十 地理一}

\begin{pinyinscope}

 自封建變為郡縣,有天下者,漢、隋、唐、宋為盛,然幅員之廣,咸不逮元。漢梗於北狄整個世界作為自由活動的對象。認為它是現象學本質的客觀,隋不能服東夷,唐患在西戎,宋患常在西北。若元,則起朔漠,並西域,平西夏,滅女真,臣高麗,定南詔,遂下江南,而天下為一,故其地北逾陰山,西極流沙,東盡遼左,南越海表。蓋漢東西九千三百二里,南北一萬三千三百六十八里,唐東西九千五百一十一里,南北一萬六千九百一十八里,元東南所至不下漢、唐,而西北則過之,有難以里數限者矣。



 初,太宗六年甲午,滅金,得中原州郡。七年乙未,下詔籍民,自燕京、順天等三十六路,戶八十七萬三千七百八十一,口四百七十五萬四千九百七十五。憲宗二年壬子,又籍之,增戶二十餘萬。世祖至元七年,又籍之,又增三十餘萬。十三年,平宋,全有版圓。二十七年,又籍之,得戶一千一百八十四萬八百有奇。於是南北之戶總書於策者,一千三百一十九萬六千二百有六,口五千八百八十三萬四千七百一十有一,而山澤溪洞之民不與焉。立中書省一,行中書省十有一:曰嶺北,曰遼陽,曰河南,曰陜西,曰四川,曰甘肅,曰雲南,曰江浙,曰江西,曰湖廣,曰征東,分鎮籓服,路一百八十五,府三十三,州三百五十九,軍四,安撫司十五,縣一千一百二十七。文宗至順元年,戶部錢糧戶數一千三百四十萬六百九十九,視前又增二十萬有奇,漢、唐極盛之際,有不及焉。蓋嶺北、遼陽與甘肅、四川、雲南、湖廣之邊,唐所謂羈縻之州,往往在是,今皆賦役之,比於內地;而高麗守東籓,執臣禮惟謹,亦古所未見。地大民眾,後世狃於治安,而不知詰戎兵、慎封守,積習委靡,一旦有變,而天下遂至於不可為。嗚呼!盛極而衰,固其理也。



 唐以前以郡領縣而已,元則有路、府、州、縣四等。大率以路領州、領縣,而腹里或有以路領府、府領州、州領縣者,其府與州又有不隸路而直隸省者,具載於篇,而其沿革則溯唐而止焉。作《地理志》。凡路,低於省一字。府與州直隸省者,亦低於省一字。其有宣慰司、廉訪司,亦止低於省一字。各路錄事司與路所親領之縣與府、州之隸路者,低於路一字。府與州所領之縣,低於府與州一字。府領州、州又領縣者,又低於縣一字。路所親領之縣若府若州,曰領縣若干、府若干、州若干;府與州所領之縣,則曰若干縣,所以別之也。



 中書省統山東西、河北之地,謂之腹裏,為路二十九,州八,屬府三,屬州九十一,屬縣三百四十六。各路立站,總計一百九十八處。



 大都路,唐幽州範陽郡。遼改燕京。金遷都,為大興府。元太祖十年,克燕,初為燕京路,總管大興府。太宗七年,置版籍。世祖至元元年,中書省臣言:「開平府闕庭所在,加號上都,燕京分立省部,亦乞正名。」遂改中都,其大興府仍舊。四年,始於中都之東北置今城而遷都焉。京城右擁太行,左挹滄海,枕居庸,奠朔方。城方六十里,十一門:正南曰麗正,南之右曰順承,南之左曰文明,北之東曰安貞,北之西曰健德,正東曰崇仁,東之右曰齊化,東之左曰光熙,正西曰和義,西之右曰肅清,西之左曰平則。海子在皇城之北、萬壽山之陰,舊名積水潭,聚西北諸泉之水,流入都城而匯於此,汪洋如海,都人因名焉。恣民漁採無禁,擬周之靈沼雲。九年,改大都。十九年,置留守司。二十一年,置大都路總管府。戶一十四萬七千五百九十,口四十萬一千三百五十。用至元七年抄籍數。領院二、縣六、州十。州領十六縣。



 右警巡院。



 左警巡院。初設警巡院三,至元四年,省其一,止設左右二院,分領坊市民事。



 縣六



 大興,赤。宛平,赤。與大興分治郭下。金水河源出玉泉山,流入皇城,故名金水。良鄉,下。永清,下。寶坻下。至元十六年,於縣立屯田所,收子粒赴太倉及醴源倉輸納。昌平。下。



 州十



 涿州,下。唐範陽縣,復改涿州。宋因之。元太宗八年,為涿州路。中統四年,復為涿州。領二縣:



 範陽,下。倚郭。房山。下。金奉先縣,至元二十七年,改今名。



 霸州,下。唐隸幽州。周始置霸州。宋升永清郡。金置信安軍。元仍為霸州。領四縣:



 益津,下。倚郭。中統四年省,至元二年置。文安,下。大城,下。



 保定。下。至元二年,省入益津,四年置。



 通州,下。唐為潞縣。金改通州,取漕運通濟之義,有豐備、通濟、太倉以供京師。領二縣:



 潞縣,倚郭。三河。下。



 薊州,下。唐置,後改漁陽郡,仍改薊州。宋為廣川郡。金為中都。元太祖十年,定其地,仍為薊州。領五縣:



 漁陽,下。倚郭。豐閏,下。至元二年,省入玉田,四年,以路當沖要復置。二十一年,立豐閏署,領屯田八百三十七戶。玉田,下。遵化,下。



 平谷。下。至元二年,省入漁陽,十三年復置。



 漷州,下。遼、金為漷陰縣。元初為大興府屬邑,至元十三年,升漷州,割大興府之武清、香河二邑來屬。領二縣:



 香河,下。武清。



 順州,下。唐初改燕州,復為歸德郡,復為順州,復為歸順州。遼為歸化軍。宋為順興軍。金仍為順州,置溫陽縣。元廢縣存州。



 檀州,下。唐改密雲郡,又復為檀州。遼為武威軍。宋為鎮遠軍。金仍為檀州。元因之。



 東安州,下。唐以前為安次縣。遼、金因之,元初隸大興府。太宗七年,隸霸州。中統四年,升為東安州,隸大都路。



 固安州,下。唐仍隋舊為固安縣,隸幽州。宋隸涿水郡。金隸涿州。元憲宗九年,隸霸州,又改隸大興府。中統四年,升固安州。



 龍慶州,唐為媯川縣。金為縉山縣。元至元三年,省入懷來縣,五年復置,本屬上都路宣德府奉聖州。二十二年,仁宗生於此。延祐三年,割縉山、懷來來隸大都,升縉山為龍慶州。領一縣:



 懷來。下。



 上都路,唐為奚、契丹地。金平契丹,置桓州。元初為札剌兒部、兀魯郡王營幕地。憲宗五年,命世祖居其地,為巨鎮。明年,世祖命劉秉忠相宅於桓州東、灤水北之龍岡。中統元年,為開平府。五年,以闕庭所在,加號上都,歲一幸焉。至元二年,置留守司。五年,升上都路總管府。十八年,升上都留守司,兼行本路總管府事。戶四萬一千六十二,口一十一萬八千一百九十一。領院一、縣一、府一、州四,州領三縣,府領三縣、二州,州領六縣。



 警巡院。



 縣一



 開平。上。



 府一



 順寧府,唐為武州。遼為德州。金為宣德州。元初為宣寧府。太宗七年,改山東路總管府。中統四年,改宣德府,隸上都路。至元三年,以地震改順寧府。領三縣、二州。



 三縣



 宣德,下。倚郭。至元二年,省本府之錄事司並龍門縣並入焉。二十八年,又割龍門去屬雲州。宣平,下。順聖。下。本隸弘州,今來屬。



 二州



 保安州,下。唐新州。遼改奉聖州。金為德興府。元初因之。舊領永興、縉山、懷來、礬山四縣。至元二年,省礬山入永興。三年,省縉山入懷來,仍改為奉聖州,隸宣德府。五年,復置縉山。延祐三年,以縉山、懷來仍隸大都。至元三年,以地震改保安州。領一縣:永興。下。倚郭。



 蔚州,下。唐改為安邊郡,又改為興唐縣,又仍為蔚州。遼為忠順軍。金仍為蔚州。元至元二年,省州為靈仙縣,隸弘州。其年,復改為蔚州,隸宣德府。領五縣:靈仙,下。靈丘,下。飛狐,下。定安,下。廣靈。下。



 州四



 興州,下。唐為奚地。金初為興化軍,隸北京,後為興州。元中統三年,屬上都路。領二縣:



 興安,下。至元二年置。宜興。中。至元二年置。



 松州,下。本松林南境,遼置松山州。金為松山縣,隸北京路大定府。元中統三年,升為松州,仍存縣。至元二年,省縣入州。



 桓州,下。本上谷郡地,金置桓州。元初廢,至元二年復置。



 雲州,下。古望雲川地,契丹置望雲縣。金因之。元中統四年,升縣為雲州,治望雲縣。至元二年,州存縣廢。二十八年,復升宣德之龍門鎮為望雲縣,隸雲州。領一縣:



 望雲。



 興和路,上。唐屬新州。金置柔遠鎮,後升為縣,又升撫州,屬西京。元中統三年,以郡為內輔,升隆興路總管府,建行宮。戶八千九百七十三,口三萬九千四百九十五。領縣四、州一。



 縣四



 高原,下。倚郭。中統二年隸宣德府,三年來屬。懷安,下。元初隸宣德府,中統三年來屬。天成,下。元初隸宣德府,中統三年來屬。咸寧。下。元初隸宣德府,中統三年來屬。



 州一



 寶昌州,下。金置昌州。元初隸宣德府,中統三年隸本路,置鹽使司。延祐六年,改寶昌州。



 永平路,下。唐平州。遼為盧龍軍。金為興平軍。元太祖十年,改興平府。中統元年,升平灤路,置總管府,設錄事司。大德四年,以水患改永平路。戶一萬三千五百一十九,口三萬五千三百。領司一、縣四、州一。州領二縣。



 錄事司。



 縣四



 盧龍,下。倚郭。遷安,下。至元二年,省入盧龍縣,後復置。撫寧,下。至元二年,與海山俱省入昌黎。三年復置。四年,又與海山俱入昌黎。七年復置,仍省昌黎、海山入焉。十一年,復置昌黎,以屬灤州,今昌黎屬本縣。昌黎。下。至元十一年復置,仍並海山入焉。詳見撫寧縣。



 州一



 灤州,下。在盧龍塞南,金領義豐、馬城、石城、樂亭四縣。元至元二年,省義豐入州。三年復置,先以石城省入樂亭,其年改入義豐。四年,馬城亦省。領二縣:



 義豐,下。倚郭。至元二年省入州,三年復置。樂亭。下。元初嘗於縣置漠州,尋廢,復為樂亭縣,隸灤州。



 德寧路,下。領縣一:德寧。下。



 凈州路,下。領縣一:天山。下。



 泰寧路,下。領縣一:泰寧。下。



 集寧路,下。領縣一:集寧。下。



 應昌路,下。領縣一:應昌。下。



 全寧路,下。領縣一:全寧。下。



 寧昌路,下。領縣一:寧昌。下。



 砂井總管府,領縣一:砂井。



 以上七路、一府、八縣皆闕。



 保定路,上。本清苑縣,唐隸鄚州。宋升保州。金改順天軍。元太宗十一年,升順天路,置總管府。至元十二年,改保定路,設錄事司。戶七萬五千一百八十二,口一十三萬九百四十。領司一、縣八、州七。州領十一縣。



 錄事司。



 縣八



 清苑,中。附郭。滿城,中。唐縣,下。金隸定州,後來屬。慶都,下。元初隸真定府,太宗十一年來屬。行唐,下。曲陽,中。古恆州地,唐為曲陽縣。宋屬中山府。金因之。元初改恆州,立元帥府,割阜平、靈壽、行唐、慶都、唐縣以隸之。逮移鎮歸德,還隸中山府,復為曲陽縣,後隸保定,北岳恆山在焉。新安,下。金置新安州渥城縣。元至元二年,州縣俱廢,改為新安鎮,入歸信縣。四年,割入容城。九年,置新安縣來屬。博野。下。至元三十一年立。



 州七



 易州,中。唐改上谷郡,又復為易州。元太宗十一年,割隸順天府。至元十年,隸大都路。二十三年,還隸保定。領三縣:



 易縣,中。倚郭。元初存州廢縣,至元三年復置。淶水,下。定興。下。金隸涿州,今來屬。



 祁州,中。唐為義豐縣,屬定州。宋改為蒲陰縣。金於縣置祁州,屬真定路。元至元三年,立附郭蒲陰縣及以束鹿、深澤二縣來屬,隸保定。領三縣:



 蒲陰,中。倚郭。深澤,下。至元二年,並入束鹿縣,三年又來屬。束鹿。中。



 雄州,下。唐歸義縣。五代為瓦橋關,周世宗克三關,於關置雄州。宋為易陽郡。金為永定軍。元太宗十一年,割雄州三縣屬順天路。至元十年,改屬大都路。十二年改順天路為保定路,二十三年,復以雄州隸之。領三縣:



 歸信,下。容城,下。金隸安肅州,今來屬。新城。太宗二年,改新泰州。七年,復為縣,隸大都路。十一年,隸順天路。至元二年,隸雄州。十年,隸大都。二十三年復來屬。



 安州,下。唐為唐興縣,隸鄚州。宋升順安軍。金改安州,治渥城縣。元初移治葛城。至元二年,廢為鎮,入高陽縣,後復改安州,隸保定。領二縣:



 葛城,下。倚郭。高陽。下。



 遂州,下。唐為遂城縣,屬易州。宋改廣信軍。金廢為遂城縣,隸保州。元至元二年,省入安肅州為鎮,後復置州而縣廢,隸保定。



 安肅州,下。本易州宥戎鎮地,宋創立靜戎軍,又改安肅軍。金為安肅州。元隸保定。



 完州,下。唐為北平縣,隸定州。宋升北平軍。金更為永平縣,又改完州。元至元二年,改永平縣,後復為完州。



 燕南河北道肅政廉訪司



 真定路,唐恆山郡,又改鎮州。宋為真定府。元初置總管府,領中山府,趙、邢、洺、磁、滑、相、浚、衛、祁、威、完十一州。後割磁、威隸廣平,浚、滑隸大名,祁、完隸保定,又以邢入順德,洺入廣平,相入彰德,衛入衛輝;又以冀、深、晉、蠡四州來屬。戶一十三萬四千九百八十六,口二十四萬六百七十。領司一、縣九、府一、州五。府領三縣,州領十八縣。



 錄事司。



 縣九



 真定,中。倚郭。槁城,中。太宗六年,為永安州,無極、寧晉、新樂、平棘四縣隸焉。七年,廢州為槁城縣,屬真定。欒城,下。元氏,中。獲鹿,中。太宗在潛邸改西寧州,既即位七年,復為獲鹿縣,隸真定。平山。下。靈壽,下。阜平,下。涉縣。元初為崇州,隸真定路,後廢州復置涉縣。至元二年,省入磁州,後復來屬。



 府一



 中山府,唐定州。宋為中山郡。金為中山府。元初因之。舊領祁、完二州,太宗十一年,割二州隸順天府,後為散府,隸真定。領三縣:



 安喜,中。新樂,下。無極。中。



 州五



 趙州,中。唐趙州。宋為慶源軍。金改沃州。元仍為趙州。舊領平棘、臨城、欒城、元氏、高邑、贊皇、寧晉、隆平、柏鄉九縣,太祖十五年,割欒城、元氏隸真定。領七縣:



 平棘,中。寧晉,下。隆平,下。臨城,中。柏鄉,下。高邑,下。贊皇。下。至元二年,並入高邑。七年復置。



 冀州,上。唐改魏州,後仍為冀州。宋升安武軍。元仍為冀州。領五縣:



 信都,中。至元初與冀州錄事司俱省入冀州,後復置。三年,省錄事司入焉,為冀州治所。南宮,上。棗強,中。武邑,中。新河。中。太宗四年置。



 深州,下。唐改饒陽郡,後仍為深州。元初隸河間,置帥府。太宗十年,隸真定路,領饒陽、安平、武強、束鹿、靜安五縣。後割安平、饒陽、武強隸晉州,束鹿隸祁州,以冀州之衡水來屬。領二縣:



 靜安,中。衡水。下。



 晉州,唐、宋皆為鼓城縣。元太祖十年,改晉州。太宗十年,立鼓城等處軍民萬戶府。中統二年,復為晉州。領四縣:



 鼓城,中。倚郭。饒陽,中。安平,下。太祖十九年,為南平州,於此行千戶總管府事,領饒陽一縣。太宗七年,復改為縣,隸深州。憲宗在潛,隸鼓城等處軍民萬戶府。中統二年,改立晉州,仍為安平縣隸焉。武強。下。元初創立東武州,領武邑、靜安。太宗六年,廢州復為縣,改隸深州。十一年,割屬祁州。憲宗在潛,隸鼓城等處軍民萬戶府。中統二年,置晉州,縣隸焉。



 蠡州,下。唐始置。宋改永寧軍。金仍為蠡州。元初隸真定,領司候司、博野縣。至元三年,省司候司、博野縣入蠡州。十七年,直隸省部。二十一年,仍屬真定。



 順德路,下。唐邢州。宋為信德府。金改邢州。元初置元帥府,後改安撫司。憲宗分洺水民戶之半於武道鎮,置司總管。五年,以武道鎮置廣宗縣,並以來屬。中統三年,升順德府。至元元年,以洺州、磁州來屬。二年,洺、磁自為一路,以順德為順德路總管府。戶三萬五百一,口一十二萬四千四百六十五。領司一、縣九。



 錄事司。



 縣九



 邢臺,中。倚郭。鉅鹿,中。內丘,中。至元二年,並唐山縣入焉,後復置唐山,與內丘並。平鄉,中。廣宗,中。憲宗五年置。中統三年以後屬順德府。至元二年,省入平鄉縣,後復置,隸順德路。沙河,下。至元二年,省南和縣入焉。後復置南和,與沙河並。南河,下。唐山。下。任縣。下。至元二年,省入邢臺縣,後復置。



 廣平路,下。唐洺州,又為廣平郡。元太宗八年,置邢洺路總管府,以邢、磁、威隸之。憲宗二年,為洺磁路,止領磁、威二州。至元十五年,升廣平路總管府。戶四萬一千四百四十六,口六萬九千八十二。領司一、縣五、州二。州領六縣。



 錄事司。



 縣五



 永平,中。倚郭。曲周,中。肥鄉,中。雞澤,下。元初並入永年,後復置。廣平。下。



 州二



 磁州,中。唐磁州。宋為滏陽郡。金以隸彰德。元太祖十年,升為滏源軍節度,隸真定路。太宗八年,隸邢洺路。憲宗二年,改邢洺路為洺磁路。至元二年,以真定之涉縣及成安縣並入滏陽,武安縣並入邯鄲,止以滏陽、邯鄲二縣及錄事司來屬。後復置涉縣歸真定,以滏陽、武安、邯鄲、成安、錄事司隸焉。至元三年,並錄事司入滏陽縣。至元十五年,改洺磁路為廣平路總管府,磁州仍隸焉。領四縣:



 滏陽,中。倚郭。武安,中。邯鄲,下。成安。下。



 威州,中。舊無此州,金始置。元太宗六年,割隸邢洺路,以洺水縣來屬。憲宗二年,隸洺磁路,徙州治於洺水。領二縣:



 洺水,中。倚郭。太宗八年,隸洺州。定宗二年,改隸威州。憲宗二年,徙威州治此。井陘。下。威州本治此,憲宗二年,移州治於洺水縣,井陘為屬縣。



 彰德路,下。唐相州,又改鄴郡。石晉升彰德軍。金升彰德府。元太宗四年,立彰德總帥府,領衛、輝二州。憲宗二年,割出衛、輝,以彰德為散府,屬真定路。至元二年,復立彰德總管府,領懷、孟、衛、輝四州,及本府安陽、臨漳、湯陰、輔巖、林慮五縣。四年,又割出懷、孟、衛、輝,仍立總管,以林慮升為林州,復立輔巖縣隸之。六年,並輔巖入安陽。戶三萬五千二百四十六,口八萬八千二百六。領司一、縣三、州一。



 錄事司。



 縣三



 安陽,上。至元六年,並輔巖入焉。湯陰,中。臨漳。中。



 州一



 林州,下。本林慮縣,金升為州。元太宗七年,行縣事。憲宗二年,復為州。至元二年,復為縣,又並輔巖入焉。未幾復為州,割輔巖入安陽,仍以州隸彰德路。



 大名路上。唐魏州。五代南漢改大名府。金改安武軍。元因舊名,為大名府路總管府。戶六萬八千六百三十九,口一十六萬三百六十九。領司一、縣五、州三。州領六縣。



 錄事司。



 縣五



 元城,中。倚郭。至元二年,並入大名縣,後復置。大名,中。倚郭。太宗六年,立縣治。憲宗二年,遷縣事於府城內。至元二年,省元城來屬,尋析大名、元城為二縣。九年,還縣治於故所。南樂,中。魏縣,中。清河。本恩州地,太宗七年,籍為清河縣,隸大名路。



 州三



 開州,上。唐澶州。宋升開德府。金為開州。元割開封之長垣、曹州之東明來屬。領四縣:



 濮陽,上。倚郭。東明,中。太宗七年,割隸大名路。至元二年來屬。長垣,中。初隸大名路,至元二年始隸開州。清豐。中。



 滑州,中。唐改靈昌郡。宋改武成軍。元仍為滑州。領二縣:



 白馬,上。為州治所。內黃。



 浚州,下。唐置黎州,後廢。石晉置浚州。宋為通利軍,又改平川軍。金復為浚州。元初隸真定。至元二年,隸大名。



 懷慶路,下。唐懷州,復改河內郡,又仍為懷州。宋升為防禦。金改南懷州,又改沁南軍。元初復為懷州。太宗四年,行懷孟州事。憲宗六年,世祖在潛邸,以懷孟二州為湯沐邑。七年,改懷孟路總管府。至元元年,以懷孟路隸彰德路。二年,復以懷孟自為一路。延祐六年,以仁宗潛邸改懷慶路。戶三萬四千九百九十三,口一十七萬九百二十六。領司一、縣三、州一。州領三縣。



 錄事司。



 縣三



 河內,中。修武,中。武陟。中。



 州一



 孟州,下。唐置河陽軍,又升孟州。宋隸河北道。金大定中,為河水所害,北去故城十五里,築今城,徙治焉。故城謂之下孟州,新城謂之上孟州。元初治下孟州。憲宗八年,復立上孟州,河陽、濟源、王屋、溫四縣隸焉,設司候司。至元三年,省王屋入濟源,並司候司入河陽。領三縣:



 河陽,下。濟源,下。太宗六年,改濟源為原州。七年,州廢,復為縣。至元三年,省王屋縣入焉。溫縣。



 衛輝路,下。唐義州,又為衛州,又為汲郡。金改河平軍。元中統元年,升衛輝路總管府,設錄事司。戶二萬二千一百一十九,口一十二萬七千二百四十七。領司一、縣四、州二。



 錄事司。



 縣四



 汲縣,下。倚郭。新鄉,中。獲嘉,下。胙城。下。舊以胙城為倚郭。憲宗元年,還州治於汲,以胙城為屬邑。



 州二



 輝州,下。唐以共城縣置共州。宋隸衛州。金改為河平縣,又改蘇門縣,又升蘇門縣為輝州,置山陽縣屬焉。至元三年,省蘇門縣,廢山陽為鎮,入本州。



 淇州,下。唐、宋、金並為衛縣之域,曰鹿臺鄉。元憲宗五年,以大名、彰德、衛輝籍餘之民,立為淇州,因又置縣曰臨淇,為倚郭。中統元年,隸大名路宣撫司。至元三年,立衛輝路,以州隸之,而臨淇縣省。



 河間路,上。唐瀛州。宋河間府。元至元二年,置河間路總管府。戶七萬九千二百六十六,口一十六萬八千五百三十六。領司一、縣六、州六。州領十七縣。



 錄事司。



 縣六



 河間,中。倚郭。肅寧,下。至元二年,廢為鎮,入河間縣,後復舊。齊東,下。憲宗三年,隸濟南路。至元二年,還屬河間路。寧津,下。憲宗二年,屬濟南路,至元二年,隸河間。臨邑,下。本屬濟南府,太宗七年,割屬河間。憲宗三年,還屬濟南。至元二年,復屬河間。青城。下。本青平鎮,太宗七年,析臨邑、寧津地置縣,隸濟南。中統置青城縣,隸陵州。至元二年,隸河間。



 州六



 滄州,中。唐改景城郡,復仍為滄州。金升臨海軍。元復為滄州。領五縣:



 清池,中。樂陵,中。南皮,下。無棣,下。至元二年,並入樂陵縣,以縣治入濟南之棣州,尋復置。鹽山。下。



 景州,中。唐觀州,又改景州。宋改永靜軍。金仍改觀州。元因之。至元二年,復為景州。領五縣:



 蓚縣,中。舊屬觀州,元初升元州,後復為蓚縣。故城,中。元初隸河間路。至元二年,並為故城鎮,屬景州。是年,復置縣還來屬。阜城,下。東光,下。吳橋。中。



 清州,下。五代置乾寧軍。宋為乾寧郡,大觀間以河清,改清州。金為乾寧軍。元太宗二年,改清寧府。七年,又改清州。至元二年,以靖海、興濟兩縣及本州司候司並為會川縣,後復置清州。領三縣:



 會川,中。靖海,下。興濟。下。



 獻州,下。本樂壽縣,宋隸瀛州,又隸河間府。金改為壽州,又改獻州。元至元二年,以州並入樂壽,直隸河間路,未幾復舊。領二縣:



 樂壽,中。附郭。交河。中。至元二年,入樂壽,未幾如故。



 莫州,下。唐置鄚州,尋改為莫。舊領二縣,至元二年,省入河間,未幾仍領二縣:



 莫亭,下。倚郭。至元二年,與任丘俱省入河間縣,後復置。任丘。下。



 陵州,下。本將陵縣,宋、金皆隸景州。憲宗三年,割隸河間府。是年升陵州,隸濟南路。至元二年,復為縣。三年,復為州,仍隸河間路。



 東平路,下。唐鄆州,又改東平郡,又號天平軍。宋改東平府,隸河南道。金隸山東西路。元太祖十五年,嚴實以彰德、大名、磁、洺、恩、博、浚、滑等戶三十萬來歸,以實行臺東平,領州縣五十四。實沒,子忠濟為東平路管軍萬戶總管,行總管府事,州縣如舊。至元五年,以東平為散府。九年,改下路總管府。戶四萬四千七百三十一,口五萬一百四十七。領司一、縣六。



 錄事司。



 縣六



 順城,下。為東平治所。東阿,中。陽谷,中。汶上,中。壽張,下。平陰。下。至元十一年,以縣之辛鎮寨、孝德等四鄉分析他屬。明年,改寨為肥城,作中縣,隸濟寧路,以平陰為下縣,仍屬東平。



 東昌路,下。唐博州。宋隸河北東路。金隸大名府。元初隸東平路。至元四年,析為博州路總管府。十三年,改東昌路,仍置總管府。戶三萬三千一百二,口一十二萬五千四百六。領司一、縣六。



 錄事司。



 縣六



 聊城,中。倚郭。堂邑,中。莘縣,中。宋隸大名府,元割以來屬。



 博平,中。茌平,中。丘縣。下。本為鎮,隸曲周。至元二年,並入堂邑。二十六年,山東宣慰司言:「丘縣並入堂邑,差稅詞訴相去二百餘里,往復非便。平恩有戶二千七百,升縣為宜。」遂立丘縣,隸東昌。



 濟寧路,下。唐麟州。周於此置濟州。元太宗七年,割屬東平府。至元六年,以濟州還治巨野,仍析鄆城之四鄉來屬。八年,升濟寧府,治任城,尋還治巨野。十二年,復立濟州,治任城,屬濟寧府。十五年,遷府於濟州,卻以巨野行濟州事。其年又以府治歸巨野,而濟州仍治任城,但為散州。十六年,濟寧升為路,置總管府。戶一萬五百四十五,口五萬九千八百一十八。領司一、縣七、州三。州領九縣。



 錄事司。



 縣七



 巨野,中。倚郭。金廢,屬鄆州。至元六年復立。鄆城,上。金以水患,徙置盤溝村。元至元八年,復來屬。肥城,中。宋、金為平陰縣。元至元十二年,以平陰莘鎮寨東北十五里舊城改設今縣。金鄉,下。初隸濟州,至元二年來屬。碭山,金為水蕩沒。元憲宗七年,始復置縣治,隸東平路。至元二年,以戶口稀少,並入單父縣。三年復置,屬濟州。八年,屬濟寧路。虞城,下。金圮于水。元憲宗二年,始復置縣,隸東平路。至元二年,以戶口稀少,並入單父。三年,復立縣,屬濟州。八年,隸濟寧路。豐縣。唐屬徐州。元憲宗二年,屬濟州。至元二年,以沛縣並入豐縣。三年,復立沛縣。八年,以豐縣直隸濟寧路。



 州三



 濟州,下。唐以前為濟北郡,治單父。唐初為濟州,又為濟陽郡,仍改濟州。周瀕濟水立濟州。宋因之。金遷州治任城,以河水湮沒故也。元至元二年,以戶不及千數,並隸任城。六年,遷州於巨野,而任城為屬邑。八年,升州為濟寧府,治任城,復還府治巨野。十二年,以任城當江淮水陸沖要,復立濟州,屬濟寧府而任城廢。十五年,遷府於濟州,以巨野行濟州事。其年復於巨野立府,仍於此為州。二十三年,復置任城,隸州。領三縣:



 任城,倚郭。魚臺,太宗七年,屬濟州。至元二年,並入金鄉。三年復故。八年,屬濟寧府。十三年來屬。沛縣。太宗七年,移滕州治此。憲宗二年,州廢,復為縣。至元二年,省入豐縣。三年復置。八年,隸濟寧府。十三年來屬。



 兗州,下。唐初為兗州,復升泰寧軍。宋改襲慶府。金改泰定軍。元初復為兗州,屬濟州。憲宗二年,分隸東平路。至元五年,復屬濟州。十六年,隸濟寧路總管府。二十三年,立尚珍署,領屯田四百五十六戶,收子粒赴濟州官倉輸納,餘糧糶賣,所入鈔納於光祿寺。領四縣:



 嵫陽,曲阜,泗水,至元二年,省入曲阜。三年復置。寧陽。至元二年,省入嵫陽。大德元年復置。



 單州,下。唐置輝州,治單父。後唐改為單州。宋升團練州。金隸歸德府。元初屬濟州。憲宗二年,屬東平府。至元五年,復屬濟州。十六年,隸濟寧路。領二縣:



 單父,縣在郭下。元初與單州並屬濟州。憲宗二年,隸東平府。至元二年,復立單父縣。三年,還屬濟州,今屬單州。嘉祥。舊屬濟州。憲宗二年,割隸東平路。至元三年,還屬濟州。今為單州屬縣。



 曹州,上。唐初為曹州,後改濟陰郡,又仍為曹州。宋改興仁府。金復為曹州。元初隸東平路總管府。至元二年,直隸省部。戶三萬七千一百五十三,口一十九萬五千三百三十五。領縣五:



 濟陰,上。成武,中。定陶,中。禹城,中。楚丘。中。



 濮州,上。唐初為濮州,後改濮陽郡,又仍為濮州。宋升防禦郡。金為刺史州。元初隸東平路,後割大名之館陶、朝城,恩州之臨清,開州之觀城來屬。至元五年,析隸省部。戶一萬七千三百一十六,口六萬四千二百九十三。領縣六:



 鄄城,上。朝城,中。初隸東平府,至元五年來屬。館陶。中。初屬東平路,至元三年來屬。臨清,觀城,下。金屬開州,元初來屬。範縣。下。初屬東平府路,至元二年來屬。



 高唐州,中。唐為縣,屬博州。宋、金因之。元初隸東平,至元七年升州。戶一萬九千一百四,口二萬三千一百二十一。領縣三:



 高唐,中。夏津,中。初隸東平,至元七年來屬。武城。中。初隸東平,至元七年來屬。



 泰安州,中。本博城縣,唐初於縣置東泰州,後廢州,改為乾封縣,屬兗州。宋改奉符縣。金置泰安州。元初屬東平路。至元二年,省新泰縣入萊蕪縣。五年,析隸省部。三十一年,復立新泰縣。東嶽泰山在焉。戶九千五百四十,口一萬七百九十五。領縣四:



 奉符,中。長清,中。舊屬濟南府,元初來屬。萊蕪,下。新泰。金屬泰安州,至元二年,省入萊蕪,三十一年復立。



 德州,唐初為德州,後改平原郡,又仍為德州。金屬山東西路。元初隸東平路總管府,割大名之清平、濟南之齊河縣來屬。戶二萬四千四百二十四,口一十五萬六千九百五十二。領縣五:



 安德,下。平原,下。齊河,金創置此縣,隸濟南府,至元二年來屬。



 清平,宋、金隸大名府,元初來屬。德平。



 恩州,中。唐貝州,又為清河郡。宋改恩州。金隸大名府路。元初割清河縣隸大名府,以武城隸高唐,惟存歷亭一縣及司候司。至元二年,縣及司俱省入州。七年,自東平析隸省部。戶一萬五百四十五,口三萬七千四百七十九。



 冠州,本冠氏縣,唐因隋舊,置毛州,後州廢,縣屬魏州。宋、金並屬大名府。元初屬東平路。至元六年,升冠州,直隸省。戶五千六百九十七,口二萬三千四十。



 山東東西道宣慰司



 益都路,唐青州,又升盧龍軍。宋改鎮海軍。金為益都路總管府。戶七萬七千一百六十四,口二十一萬二千五百二。領司一、縣六、州八。州領十五縣。



 錄事司。



 縣六



 益都,中。倚郭。至元二年,以行淄州及行淄川縣並入。三年,又並臨淄、臨朐二縣入焉。十五年,割臨淄、臨朐復置縣,並屬本路。臨淄,下。臨朐,下。高苑,下。舊屬淄州。樂安,下。壽光。下。



 州八



 濰州,下。唐初為濰州,後廢。宋為北海軍,復升濰州。金屬益都路。元初領北海、昌邑、昌樂三縣及司候司。憲宗三年,省司候司入北海。至元三年,省昌樂縣入北海。領二縣:



 北海,下。昌邑。下。



 膠州,下。唐初為膠西縣。宋置臨海軍。金仍改為膠西縣,屬密州。元太祖於縣置膠州。領三縣:



 膠西,中。即墨,下。宋、金皆隸萊州,元太祖二十二年來屬。高密。下。宋、金並隸密州。



 密州,唐初改為高密郡,後仍為密州。宋為臨海軍,復為密州。元初因之,以膠西、高密屬膠州。憲宗三年,省司候司入諸城縣,隸益都。領二縣:



 諸城,州治所。安丘。下。



 莒州,下。唐廢莒州,以莒縣隸密州。宋沿其舊。金復為莒州,隸益都府。元初因之。領四縣:



 莒縣,下。州治所。憲宗三年,省司候司入焉。沂水,下。有沂山,為東鎮。日照,下。蒙陰。下。元初,因舊名為新泰縣。中統三年,以李璮亂,人民逃散,省入沂水。皇慶二年,復置為蒙陰縣。



 沂州,下。唐初改為瑯邪郡,後仍為沂州。宋屬京東東路。金屬山東東路。元屬益都路。領二縣:



 臨沂,中。州治所。憲宗三年,省司候司入焉。費縣。下。



 滕州,下。唐為滕縣,屬徐州。宋仍舊。金改為滕州,屬兗州。元隸益都路。領二縣:



 滕縣,下。憲宗三年,省司候司入焉。鄒縣。下。



 嶧州,下。唐置鄫州,又改蘭陵縣為承縣,後州廢,以縣屬沂州。宋仍舊。金改蘭陵縣,於縣置嶧州。元初以嶧州隸益都路,至元二年,省蘭陵入本州。



 博興州,下。唐博昌縣。後唐改博興。宋屬青州。金屬益都府。元初升為州。



 山東東西道肅政廉訪司



 濟南路,上。唐濟州,又改臨淄郡,又改濟南郡,又為青州。宋為濟南府。金因之。元初改濟南路總管府,舊領淄、陵二州。至元二年,淄州割入淄萊路,陵州割入河間路,又割臨邑縣隸河間路,長清縣入泰安州,禹城縣隸曹州,齊河縣入德州,割淄州之鄒平縣來屬,置總管府。戶六萬三千二百八十九,口一十六萬四千八百八十五。領司一、縣四、州二。州領七縣。



 錄事司。



 縣四



 歷城,中。倚郭。章丘,上。鄒平,上。唐、宋皆屬淄州,至元間來屬。濟陽。中。



 州二



 棣州,上。唐析滄州之陽信、商河、樂陵、厭次置棣州。宋、金因之。元初濱、棣自為一道,中統三年,改置濱棣路安撫司。至元二年,與濱州俱棣濟南路。領四縣:



 厭次,中。倚郭。初立司候司,至元二年,省入本縣。商河,中。陽信,中。無棣。下。宋、金屬滄州,元初割無棣之半屬滄州,半以來屬。



 濱州,中。唐屬棣州。周始置濱州。金隸益都。元初以棣州為濱棣路。至元二年,省路為州,隸濟南路。領三縣:



 渤海,中。初設司候司,至元二年,省入此縣。利津,下。沾化。下。



 般陽府路,下。唐淄州,宋屬河南道。金屬山東東路。元初太宗在潛,置新城縣。中統四年,割濱州之蒲臺來屬。先是,淄州隸濟南路總管府;五年,升淄州路,置總管府。是歲改元至元,割鄒平屬濟南路、高苑屬益都路。二年,改淄州路為淄萊路。二十四年,改般陽路,取漢縣以為名。戶二萬一千五百三十,口一十二萬三千一百八十五。領司一、縣四、州二。州領八縣。



 錄事司。



 縣四



 淄川,中。倚郭。長山,中。初屬濟南路,中統三年來屬。新城,中。本長山縣驛臺,太宗在潛,以人民完聚,創置城曰新城,以田、索二鎮屬焉。蒲臺。下。金屬濱州,元初隸濱棣路。中統五年,屬淄州。至元二年,改屬淄萊路,升中縣。



 州二



 萊州,中。唐初改東萊郡為萊州。宋為防禦州。金升定海軍,屬山東東路。元初屬益都路。中統五年,屬淄萊路。舊設錄事司。至元二年,省入掖縣,又省即墨入掖與膠水,仍隸般陽路。領四縣:



 掖縣,中。倚郭。至元二年,省錄事司,析即墨縣入焉。膠水,下。至元二年,析即墨縣入焉。招遠,下。萊陽。下。



 登州,下。唐初為牟州,復改登州,宋屬河南道。元初屬益都路。中統五年,別置淄萊路,以登州隸之。至元二十四年,改屬般陽路。領四縣:



 蓬萊,下。黃縣,下。福山,下。偽齊以登州之雨水鎮為福山縣,楊畽鎮為棲霞縣。棲霞。下。



 寧海州,下。偽齊劉豫以登州之文登、牟平二縣立寧海軍。金升寧海州。元初隸益都路。至元九年,直隸省部。戶五千七百一十三,口一萬五千七百四十三。領縣二:



 牟平,中。文登。下。



 河東山西道宣慰使司



 大同路,上。唐為北恆州,又為雲州,又改雲中郡。遼為西京大同府。金改總管府。元初置警巡院。至元二十五年,改西京為大同路。戶四萬五千九百四十五,口一十二萬八千四百九十六。領司一、縣五、州八。州領四縣。大德四年,於西京黃華嶺立屯田。六年,立萬戶府,所屬山陰、雁門、馬邑、鄯陽、洪濟、金城、寧武凡七屯。



 錄事司。



 縣五



 大同,中。倚郭。至元二年,省西縣入焉。白登,下。至元二年,廢為鎮,屬大同縣,尋復置。宣寧,下。平地,下。本號平地裊,至元二年,省入豐州。三年,置縣,曰平地。懷仁。下。



 州八



 弘州,下。唐為清塞軍,隸蔚州。遼置弘州。金仍舊。舊領襄陰、順聖二縣。元至元中,割順聖隸宣德府,惟領襄陰及司候司,後並省入州。



 渾源州,下。唐為渾源縣,隸應州。金升為州,仍置縣在郭下,並置司候司。元至元四年省入州。



 應州,下。唐末置。後唐升彰國軍。元初仍為應州。領二縣:



 金城,下。州治所。山陰。下。至元二年,並入金城,後復置。



 朔州,下。唐改馬邑郡為朔州。後唐升振武軍。宋為朔寧府。金為朔州。元因之。領二縣:



 鄯陽,下。至元四年,省錄事司入焉。馬邑。下。



 武州,下。唐隸定襄、馬邑二郡。遼置武州宣威軍。元至元二年,割寧邊州之半來屬。舊領寧邊一縣及司候司,四年省入州。



 豐州,下。唐初為豐州,又改九原郡,又仍為豐州。金為天德軍。元復為豐州。舊有錄事司並富民縣,元至元四年省入州。



 東勝州,下。唐勝州,又改榆林郡,又復為勝州。張仁願築三受降城,東城南直榆林,後以東城濱河,徙置綏遠峰南郡今東勝州是也。金初屬西夏,後復取之。元至元二年,省寧邊州之半入焉。舊有東勝縣及錄事司,四年省入州。



 雲內州,下。唐初立雲中都督府,復改橫塞軍,又改天德軍,即中受降城之地。金為雲內州。舊領雲川、柔服二縣,元初廢雲川,設錄事司。至元四年,省司、縣入州。



 河東山西道肅政廉訪司



 冀寧路,上。唐並州,又為太原府。宋、金因之。元太祖十三年,立太原路總管府。大德九年,以地震改冀寧路。戶七萬五千四百四,口一十五萬五千三百二十一。領司一、縣十、州十四。州領九縣。



 錄事司。



 縣十



 陽曲,中。倚郭。文水,中。平晉,下。祁縣,下。舊隸晉州,後州廢,隸太原路。榆次,下。至元二年,隸太原路。太谷,下。清源,下。壽陽,下。交城,下。徐溝。下。



 州十四



 汾州,中。唐改西河郡為浩州,又改汾州,又改西河郡,又為汾州。金置汾陽軍。元初立汾州元帥府,割靈石縣隸平陽路之霍州,仍析置小靈石縣,後廢府。至元二年,復行州事,省小靈石入介休。三年,並溫泉入孝義。領四縣:



 西河,中。孝義,下。至元三年,割溫泉縣之半置巡檢司,隸本縣。平遙,下。元初屬太原府,至元二年來屬。



 介休。下。元初置,隸太原府,至元二年來屬,仍省小靈石縣入焉。



 石州,下。唐初改離石郡為石州,又改昌化郡,又仍為石州。宋、金因其名。元中統二年,省離石縣入本州。三年,復立。至元三年,省溫泉入孝義,以臨泉為臨州。舊置司候司,後與孟門、方山俱省入離石。領二縣:



 離石,下。倚郭。寧鄉。下。太宗九年,隸太原府。定宗三年,隸石州。憲宗九年,又隸太原府。至元三年,復來屬。



 忻州,下。唐初置新興郡,後改忻州,又改定襄郡,又為忻州。金隸太原府。元因之。領二縣:



 秀容,下。倚郭。至元二年,省入忻州。四年復置。定襄。下。



 平定州,下。唐為廣陽縣。宋為平定軍。金為平定州。元至元二年,省倚郭平定、樂平二縣入本州。七年,復立樂平。領一縣:



 樂平。下。倚郭。至元二年,省縣為鄉,入本州,立巡檢司。七年復立。



 臨州,下。唐置臨泉縣,又置北和州,後州廢,隸石州。宋置晉寧軍。金廢軍,置臨水縣,隸石州。元中統二年,仍改臨泉縣,直隸太原府。三年,升臨州。



 保德州,下。本嵐州地,宋始置州。舊有倚郭縣,元憲宗七年廢縣。至元二年,省隩州、芭州入本州。三年,又並岢嵐軍入焉。四年,割岢嵐隸管州,隩州仍來屬。



 崞州,下。本崞縣,元太祖十四年升崞州。



 管州,下。唐以靜樂縣置,後州廢,屬嵐州。後又為憲州。宋為靜樂軍。金為靜樂郡,又改為管州。元太祖十六年,以嵐州之岢嵐、寧化、樓煩並入本州。至元二十二年,割岢嵐隸嵐州,而寧化、樓煩並入本州。



 代州,下。唐置代州總管府。金改都督府。元中統四年,並雁門縣入州。



 臺州,下。唐為五臺縣,隸代州。金升臺州,隸太原路。元因之。



 興州,下。唐臨津縣,隸嵐州,又改合河縣。金升興州,隸太原路。元因之。



 堅州,下。唐繁畤縣。金為堅州,隸太原路。元因之。



 嵐州,下。唐、宋並為嵐州。金升鎮西節度。至元二年,省入管州。五年復立。



 盂州,下。本盂縣,金升為州。元因之。



 晉寧路,上。唐晉州。金為平陽府。元初為平陽路,大德九年,以地震改晉寧路。戶一十二萬六百二十,口二十七萬一百二十一。領司一、縣六、府一、州九。府領六縣,州領四十縣。



 錄事司。



 縣六



 臨汾,中。倚郭。襄陵,中。洪洞,中。浮山,下。汾西,下。岳陽。下。本猗氏縣,屬平陽府。至元三年,省入岳陽縣。四年,以縣當東西驛路之要復置,並岳陽、和川二縣入焉。後復改為岳陽縣。



 府一



 河中府,唐蒲州,又改河中府,又改河東郡,又仍為河中府。宋為護國軍。金復為河中府。元憲宗在潛,置河解萬戶府,領河、解二州。河中府領錄事司及河東、臨晉、虞鄉、猗氏、萬泉、河津、榮河七縣。至元三年,省虞鄉入臨晉,省萬泉入猗氏,並錄事司入河東,罷萬戶府,而河中府仍領解州。八年,割解州直隸平陽路,河中止領五縣。十五年,復置萬泉縣來屬。領六縣:



 河東,下。府治所。萬泉,下。猗氏,下。榮河,下。金隸榮州,元初廢榮州,復為榮河縣。臨晉,下。河津。下。



 州九



 絳州,中。唐初為絳郡,又改絳州。宋置防禦。金改晉安府。元初為絳州行元帥府,河、解二州諸縣皆隸焉。後罷元帥府,仍為絳州,隸平陽路。領七縣:



 正平,下。倚郭。至元二年,省錄事司入焉。太平,中。曲沃,下。



 翼城,下。金為翼州,元初復為翼城縣,隸絳州。稷山,下。絳縣,下。至元二年,省垣曲縣入焉。十六年,復立垣曲縣,絳縣如故。



 垣曲。下。



 潞州,下。唐初為潞州,後改上黨郡,又仍為潞州。宋改隆德軍。金復為潞州。元初為隆德府,行都元帥府事。太宗三年,復為潞州,隸平陽路。至元三年,以涉縣割入真定府,以錄事司並入上黨縣。領七縣:



 上黨,下。壼關,下。長子,下。潞城,下。屯留,下。至元三年,省入襄垣。十五年復置。襄垣,下。黎城。下。至元二年,並涉縣偏城等十三村入焉。



 澤州,下。唐初為澤州,後為高平郡,又仍為澤州。宋屬河東道。金為平陽府。元初置司候司及領晉城、高平、陽城、沁水、端氏、陵川六縣。至元三年,省司候司、陵川縣入晉城,省端氏入沁水。後復置陵州。領五縣:



 晉城,下。高平,下。陽城,下。沁水,下。陵川。下。至元三年,省入晉城,後復置。



 解州,下。本唐蒲州之解縣。五代漢乾祐中置解州。宋屬京兆府。金升寶昌軍。元至元四年,並司候司入解縣。有鹽池,方一百二十里。領六縣:



 解縣,下。安邑,下。聞喜,下。夏縣,下。平陸,下。芮城。下。



 霍州,下。唐初為霍山郡,又改呂州,又廢州而以縣隸晉州。金改霍州。元因之。領三縣:



 霍邑,下。倚郭。有霍山為中鎮。趙城,舊屬平陽府。靈石。下。舊屬汾州。



 隰州,下。唐初為隰州,又改大寧郡,又仍為隰州。元以州隸晉寧路。領五縣:



 隰川,中。州治所。至元三年,省大寧、蒲、溫泉三縣入焉。大寧,下。至元三年,省入隰川,二十三年復置。石樓,下。永和,下。蒲縣。下。



 沁州,下。唐初為沁州,又改陽城郡,又仍為沁州。宋置威勝軍。金仍為沁州。元因之。領三縣:



 銅鞮,下。州治所。至元十年,省錄事司、武鄉縣入焉。沁源,下。至元十年,省綿上縣入焉。武鄉。下。至元三年,省入銅鞮,後復立。



 遼州,下。唐初置遼州,又改箕州,又改儀州。宋復為遼州。元隸晉寧路。領三縣:



 遼山,下。倚郭。榆社,下。至元三年,省入遼山,六年復立。和順。下。至元三年,省儀城縣入焉。



 吉州,下。唐初置西汾州,又為南汾州,又改慈州。宋置吉鄉軍。金改耿州,又改吉州。元初領司候司、吉鄉、鄉寧二縣。中統二年,並司候司入吉鄉縣。至元二年,省吉鄉。三年,又省鄉寧並入州。後復置鄉寧。領一縣:



 鄉寧。下。



 嶺北等處行中書省統和寧路總管府



 和寧路,上。始名和林,以西有哈剌和林河,因以名城。太祖十五年,定河北諸郡,建都於此。初立元昌路,後改轉運和林使司,前後五朝都焉。太宗乙未年,城和林,作萬安宮。丁酉,治迦堅茶寒殿,在和林北七十餘里。戊戌,營圖蘇胡迎駕殿,去和林城三十餘里。世祖中統元年,遷都大興,和林置宣慰司都元帥府。後分都元帥府於金山之南,和林止設宣慰司。至元二十六年,諸王叛兵侵軼和林,宣慰使怯伯等乘隙叛去。二十七年,立和林等處都元帥府。大德十一年,立和林等處行中書省,以淇陽王月赤察兒為右丞相,太傅答剌罕為左丞相,罷和林宣慰司都元帥府,置和林總管府。至大二年,改行中書省為行尚書省。四年,罷尚書省,復為行中書省。皇慶元年,改嶺北等處行中書省,改和林路總管府為和寧路總管府。至元二十年,令西京宣慰司送牛一千,赴和林屯田。二十二年,並和林屯田入五條河。三十年,命戍和林漢軍四百,留百人,餘令耕屯杭海。元貞元年,於六衛漢軍內撥一千人赴青海屯田。北方立站帖裡乾、木憐、納憐等一百一十九處。



\end{pinyinscope}