\article{本紀第四十一 順帝四}

\begin{pinyinscope}

 三年春正月丙子,中書左丞許有壬辭職。丁丑,享於太廟。乙酉,中書平章政事納麟辭職。庚寅,沙汰怯薛丹名數。



 二月戊戌,祭社稷。甲辰,太陰犯井宿,填星犯牛宿,熒惑犯羅堰。丁未,立四川省檢校官。遼陽吾者野人叛。乙卯,太陰犯氐宿。是月,汴梁路新鄭、密二縣地震。寶慶路饑,判官文殊奴以所受敕牒貸官糧萬石賑之。秦州成紀縣,鞏昌府寧遠、伏羌縣山崩,水湧,溺死人無算。



 三月壬申,造鹿頂殿。監察御史成遵等言:「可用終場下第舉人充學正、山長,國學生會試不中者,與終場舉人同。」戊寅,詔:「作新風憲。在內之官有不法者,監察御史劾之;在外之官有不法者,行臺監察御史劾之。歲以八月終出巡,次年四月中還司。」壬午,太陰犯氐宿。是月,詔修遼、金、宋三史,以中書右丞相脫脫為都總裁官,中書平章政事鐵木兒塔識、中書右丞太平、御史中丞張起巖、翰林學士歐陽玄、侍御史呂思誠、翰林侍講學士揭傒斯為總裁官。



 夏四月丙申朔,日有食之。乙巳,享於太廟。是月,兩都桑果葉皆生黃色龍文。車駕時巡上都。



 五月,河決白茅口。六月壬子,命經筵官月進講者三。是月,回回剌里五百餘人渡河寇掠解、吉、隰等州。中書戶部以國用不足,請撙節浮費。



 秋七月丁卯,享於太廟。戊辰,修大都城。戊寅,立永昌等處宣慰司。庚辰,太白犯右執法。是月,興國路大旱,河南自四月至是月霖雨不止。戶部復言撙節錢糧。



 八月甲午朔,晉寧路臨汾縣獻嘉禾,一莖有八穗者。命朵思麻同知宣慰司事鎖兒哈等討四川上蓬瑣吃賊。戊戌,祭社稷。山東有賊焚掠兗州。是月,車駕還自上都。九月甲子,湖廣行省平章政事鞏卜班擒道州、賀州徭賊首唐大二、蔣仁五至京,誅之。其黨蔣丙,自號順天王,攻破連、桂二州。甲申,修理太廟,遣官告祭,奉遷神主於後殿。



 冬十月乙未,增立巡防捕盜所於永昌。丁酉,告祭太廟,奉安神主。戊戌,帝將祀南郊,告祭太廟。至寧宗室,問曰:「朕,寧宗兄也,當拜否?」太常博士劉聞對曰:「寧宗雖弟,其為帝時,陛下為之臣。春秋時,魯閔公弟也,僖公兄也,閔公先為君,宗廟之祭,未聞僖公不拜。陛下當拜。」帝乃拜。丁未,月食。己酉,帝親祀上帝於南郊,以太祖配。癸丑,命僉樞密院事韓元善為中書參知政事,中書參議買木丁同知宣徽院事。己未,以郊祀禮成,詔大赦天下,文官普減一資,武官升散官一等,蠲民間田租五分,賜高年帛。以湖廣行省平章政事鞏卜班為宣徽院使,行樞密院知院剌剌為翰林學士承旨。



 十一月辛未,享於太廟。



 十二月丙申,詔寫金字《藏經》。丁未,以別兒怯不花為中書左丞相。是月,膠州及屬邑高密地震。河南等處民饑,賑糶麥十萬石。是歲,詔立常平倉,罷民間食鹽。征遺逸脫因、伯顏、張瑾、杜本,本辭不至。



 四年春正月辛未,享於太廟。辛巳,詔:「定守令黜陟之法,六事備者升一等,四事備者減一資,三事備者平遷,六事俱不備者降一等。」庚寅,河決曹州,雇夫萬五千八百修築之。是月,河又決汴梁。



 二月戊戌,祭社稷。辛丑,四川行省立惠民藥局。是月,中書右丞太平升平章政事。



 閏月辛酉朔,永平、澧州等路饑,賑之。乙亥,月食。三月丁酉,復立武功縣。壬寅,特授八禿麻朵兒只征東行省左丞相,嗣高麗國王。癸丑,以河南行省平章政事納麟為中書平章政事,集賢大學士姚庸為中書左丞。



 夏四月丁亥,復立廣樣局。是月,車駕時巡上都。



 五月乙未,右丞相脫脫辭職,不許;甲辰,許之,以阿魯圖為中書右丞相。乙巳,封脫脫為鄭王,食邑安豐,賜金印及海青、文豹等物,俱辭不受。是月,大霖雨,黃河溢,平地水二丈,決白茅堤、金堤,曹、濮、濟、兗皆被災。六月戊辰,鞏昌隴西縣饑,每戶貸常平倉粟三斗,俟年豐還官。己巳,賜脫脫松江田,為立松江等處稻田提領所。



 秋七月戊子朔,溫州颶風大作,海水溢,地震。益都瀕海鹽徒郭火你赤作亂。己丑,享於太廟。是月,灤河水溢。



 八月戊午,祭社稷。丁卯,山東霖雨,民饑相食,賑之。丙戌,賜脫脫金十錠、銀五十錠、鈔萬錠、幣帛二百匹,辭不受。是月,陜西行省立惠民藥局。莒州蒙陰縣地震。郭火你赤上太行,由陵川入壺關,至廣平,殺兵馬指揮,復還益都。車駕還自上都。九月丁亥朔,日有食之。丙午,命太平提調都水監。辛亥,以南臺治書侍御史秦從德為江浙行省參知政事,提調海運。癸丑,命御史大夫也先帖木兒、平章政事鐵木兒塔識知經筵事,右丞達識帖睦邇提調宣文閣、知經筵事。



 冬十月乙酉,議修黃河、淮河堤堰。



 十一月丁亥朔,以各郡縣民饑,不許抑配食鹽。復令民入粟補官,以備賑濟。戊子,禁內外官民宴會不得用珠花。己亥,保定路饑,以鈔八萬錠、糧萬石賑之。戊申,河南民饑,禁酒。



 十二月己未,四川廉訪司建言:「廣元等五路,廣安等三府,永寧等兩宣撫司,請依內郡設置推官一員。」從之。壬戌,太陰犯外屏。癸亥,漢陽地震。戊寅,徭賊寇靖州。



 是月,東平地震。禁淫祠。賑東昌、濟南、般陽、慶元、撫州饑民。是歲,徭賊寇潯州,同知府事保童率民兵擊走之。



 五年春正月辛卯,享於太廟。是月,薊州地震。



 二月戊午,祭社稷。



 三月辛卯,帝親試進士七十有八人,賜普顏不花、張士堅進士及第,其餘賜出身有差。是月,以陳思謙參議中書省事。先是,思謙建言:「所在盜起,蓋由歲饑民貧,宜大發倉廩賑之,以收人心,仍分布重兵鎮撫中夏。」不聽。大都、永平、鞏昌、興國、安陸等處並桃溫萬戶府各翼人民饑,賑之。



 夏四月丁卯,大都流民,官給路糧,遣其還鄉。是月,汴梁、濟南、邠州、瑞州等處民饑,賑之。募富戶出米五十石以上者,旌以義士之號。車駕時巡上都。



 五月己丑,詔以軍士所掠雲南子女一千一百人放還鄉里,仍給其行糧,不願歸者聽。丁未,河間轉運司灶戶被水災,詔權免餘鹽二萬引,候年豐補還官。六月,廬州張順興出米五百餘石賑饑,旌其門。



 秋七月丁亥,河決濟陰。己丑,享於太廟。丙午,命也先帖木兒、鐵木兒塔識並為御史大夫。詔作新風紀。



 八月戊午,祭社稷。是月,車駕還自上都。九月辛巳朔,日有食之。戊戌,開酒禁。辛丑,以中書右丞達識帖睦邇為翰林學士承旨,中書參知政事搠思監為右丞,資政院使朵兒直班為中書參知政事。是月,革罷奧魯。



 冬十月壬子,以中書平章政事太平為御史大夫。乙卯,享於太廟。辛酉,命奉使宣撫巡行天下,詔曰:



 朕自踐祚以來,至今十有餘年,托身億兆之上,端居九重之中,耳目所及,豈能周知?故雖夙夜憂勤,覬安黎庶,而和氣未臻,災眚時作,聲教未洽,風俗未淳,吏弊未祛,民瘼滋甚。豈承宣之寄,糾劾之司,奉行有所未至歟?若稽先朝成憲,遣官分道奉使宣撫,布朕德意,詢民疾苦,疏滌冤滯,蠲除煩苛。體察官吏賢否,明加黜陟,有罪者,四品以上停職申請,五品以下就便處決。民間一切興利除害之事,悉聽舉行。



 命江西行省左丞忽都不丁、吏部尚書何執禮巡兩浙江東道,前雲南行省右丞散散、將作院使王士弘巡江西福建道,大都路達魯花赤拔實、江浙行省參知政事秦從德巡江南湖廣道,吏部尚書定僧、宣政僉院魏景道巡河南江北道,資政院使蠻子、兵部尚書李獻巡燕南山東道,兵部尚書不花、樞密院判官靳義巡河東陜西道,宣政院同知伯家奴、宣徽僉院王也速迭兒巡山北遼東道,荊湖北道宣慰使阿乞剌、兩淮運使杜德遠巡云南省,上都留守阿牙赤、陜西行省左丞王紳巡甘肅永昌道,大都留守答爾麻失里、河南行省參知政事王守誠巡四川省,前西臺中丞定定、集賢侍講學士蘇天爵巡京畿道,平江路達魯花赤左答納失里、都水監賈惟貞巡海北海南廣東道。黃河泛溢。辛未,遼、金、宋三史成,右丞相阿魯圖進之,帝曰:「史既成書,前人善者,朕當取以為法,惡者取以為戒,然豈止激勸為君者,為臣者亦當知之。卿等其體朕心,以前代善惡為勉。」己卯,監察御史不答失裡請罷造作不急之務。是月,以呂思誠為中書參知政事。



 十一月甲午,《至正條格》成。奉元路陳望叔偽稱燕帖古思太子,伏誅。



 十二月丁巳,詔定薦舉守令法。是歲,宣徽院使篤憐鐵穆邇知樞密院事,馮思溫為御史中丞。



 六年春二月庚戌朔,日有食之。辛未,興國雨雹,大者如馬首。是月,山東地震,七日乃止。



 三月辛未,盜扼李開務之閘河,劫商旅船。兩淮運使宋文瓚言:「世皇開會通河千有餘里,歲運米至京者五百萬石。今騎賊不過四十人,劫船三百艘而莫能捕,恐運道阻塞,乞選能臣率壯勇千騎捕之。」不聽。戊申,京畿盜起,範陽縣請增設縣尉及巡警兵,從之。山東盜起,詔中書參知政事鎖南班至東平鎮遏。八番龍宜進馬。



 夏四月壬子,遼陽為捕海東青煩擾,吾者野人及水達達皆叛。癸丑,以長吉為皇太子宮傅官。頒《至正條格》於天下。甲寅,以中書參知政事呂思誠為左丞。乙卯,享於太廟。丁卯,車駕時巡上都。發米二十萬石賑糶貧民。萬戶買住等討吾者野人遇害,詔恤其家。以中書左丞呂思誠知經筵事。命左右二司、六部吏屬於午後講習經史。



 五月壬午,陜西饑,禁酒。象州盜起。江西田賦提舉司擾民,罷之。丁亥,盜竊太廟神主。遣火兒忽答討吾者野人。丁酉,以黃河決,立河南山東都水監。六月己酉,汀州連城縣民羅天麟、陳積萬叛,陷長汀縣,福建元帥府經歷真寶、萬戶廉和尚等討之。丁巳,詔以雲南賊死可伐盜據一方,侵奪路甸,命亦禿渾為雲南行省平章政事討之。



 秋七月己卯,享於太廟。丙戌,以遼陽吾者野人等未靖,命太保伯撒里為遼陽行省左丞相鎮之。丁亥,降詔招諭死可伐。散毛洞蠻覃全在叛,招降之,以為散毛誓厓等處軍民宣撫使,置官屬,給宣敕、虎符,設立驛鋪。癸巳,詔選怯薛官為路、府、縣達魯花赤。丙申,以朵兒直班為中書右丞,答兒麻為參知政事。壬寅,以御史大夫亦憐真班等知經筵事。甲辰,京畿奉使宣撫定定奏言御史撒八兒等罪,杖黜之。時諸道奉使,皆與臺憲互相掩蔽,惟定定與湖廣道拔實糾舉無避。



 八月丙午,命江浙行省右丞忽都不花、江西行省右丞禿魯統軍合討羅天麟。戊申,祭社稷。是月,車駕還自上都。九月乙酉,克復長汀。戊子,邵武地震,有聲如鼓,至夜復鳴。



 冬十月,思、靖徭寇犯武岡,詔湖廣省臣及湖南宣慰元帥完者帖木兒討之,俘斬數百級,徭賊敗走。



 閏月乙亥,詔赦天下,免差稅三分,水旱之地全免。靖州徭賊吳天保陷黔陽。癸未,汀州賊徒羅德用殺首賊羅天麟、陳積萬,以首級送官,餘黨悉平。



 十二月丁丑,省臣改擬明宗母壽章皇后徽號曰莊獻嗣聖皇后。己卯,改立山東東西道宣慰使司都元帥府,開設屯田,駐軍馬。甲申,詔復立大護國仁王寺昭應宮財用規運總管府,凡貸民間錢二十六萬餘錠。辛卯,有司以賞賚泛濫,奏請恩賜必先經省、臺、院定擬。甲午,設立海剌禿屯田二處。詔:「犯贓罪之人,常選不用。」復立八百宣慰司,以土官韓部襲其父爵。辛丑,以吉剌班為太尉,開府,置僚屬。壬寅,山東、河南盜起,遣左、右阿速衛指揮不兒國等討之。是歲,黃河決。尚書李絅請躬祀郊廟,近正人,遠邪佞,以崇陽抑陰,不聽。



 七年春正月甲辰朔,日有食之。大寒而風,朝官僕者數人。己酉,享於太廟。壬子,命中書左丞相別兒怯不花為右丞相,尋辭職。丁巳,復立東路都蒙古軍都元帥府。庚申,雲南老丫等蠻來降,立老丫耿凍路軍民總管府。丙寅,以廣西宣慰使章伯顏討徭、獠有功,升湖廣行省左丞。詔以怯薛丹支給浩繁,除累朝定額外,悉罷之。



 二月甲戌朔,興聖宮作佛事,賜鈔二千錠。己卯,山東地震,壞城郭,棣州有聲如雷。河南、山東盜蔓延濟寧、滕、邳、徐州等處。庚辰,以中書參知政事鎖南班為中書右丞,道童為中書參知政事。丙戌,以宦者伯帖木兒為司徒。是月,徭賊吳天保寇沅州。以阿吉剌為知樞密院事,整治軍務。



 三月甲辰,中書省臣言:「世祖之朝,省、臺、院奏事,給事中專掌之,以授國史纂修。近年廢弛,恐萬世之後,一代成功無從稽考,乞復舊制。」從之。乙巳,遣使銓選云南官員。修光天殿。庚戌,試國子監,會食弟子員,選補路府及各衛學正。戊午,詔編《六條政類》。庚申,監察御史王士點劾集賢大學士吳直方躐進官階,奪其宣命,乙丑,雲南王孛羅來獻死可伐之捷。壬申,遣使修上都大乾元寺。命有司定吊賻諸王、公主、駙馬禮儀之數。



 夏四月乙亥,命江浙省臣講究役法。己卯,享於太廟。辛巳,遣達本、賀方使於占城。以通政院使朵郎吉兒為遼陽行省參知政事,討吾者野人。己丑,發米二十萬石賑糶貧民。以翰林學士承旨定住為中書右丞。庚寅,復命別兒怯不花為中書右丞相。以中書平章政事鐵木兒塔識為左丞相。臨清、廣平、灤河等處盜起,遣兵捕之。通州盜起,監察御史言:「通州密邇京城,而盜賊蜂起,宜增兵討之,以杜其源。」不聽。是月,河東大旱,民多饑死,遣使賑之。車駕時巡上都。



 五月庚戌,徭賊吳天保陷武岡路,詔遣湖廣行省右丞沙班統軍討之。乙丑,右丞相別兒怯不花,以調燮失宜、災異迭見罷,詔以太保就第。是月,臨淄地震,七日乃止。六月,詔免太師馬札兒臺官,安置西寧州,其子脫脫請與父俱行。以御史大夫太平為中書平章政事。彰德路大饑,民相食。



 秋七月甲寅,召隱士完者圖、執禮哈瑯為翰林待制,張樞、董立為翰林修撰,李孝先為著作郎。張樞不至。丙辰,太陰犯壘壁陣。丁巳,以江南行臺大夫納麟為御史大夫。



 是月,徭賊吳天保復寇沅州,陷漵浦、辰溪縣,所在焚掠無遺。徙馬札兒臺於甘肅,以別兒怯不花之譖也。九月癸卯,八憐內哈剌那海、禿魯和伯賊起,斷嶺北驛道。甲辰,遼陽霜早傷禾,賑濟驛戶。戊申,車駕還自上都。癸丑,上都斡耳朵成,用鈔九千餘錠。甲寅,詔舉才能學業之人,以備侍衛。丁巳,中書左丞相鐵木兒塔識薨。辛西,以御史大夫朵兒只為中書左丞相。甲子,集慶路盜起,鎮南王孛羅不花討平之。丁卯,徭寇吳天保復陷武岡,延及寶慶,殺湖廣行省右丞沙班於軍中。



 冬十月辛未,享於太廟。丁丑,詔:「左右丞相、平章、樞密知院、御史大夫,得賜玉押字印,餘官不與。」庚辰,詔建木華黎、伯顏祠堂於東平。丙戌,亦憐只答兒反,遣兵討之。辛卯,開東華射圃。戊戌,西番盜起,凡二百餘所,陷哈剌火州,劫供御蒲萄酒,殺使臣。是月,徭賊吳天保復寇沅州,州兵擊走之。



 十一月辛丑,監察御史曲曲,以宦者隴普憑藉寵幸,驟升榮祿大夫,追封三代,田宅逾制,上疏劾之。甲辰,沿江盜起,剽掠無忌,有司莫能禁。兩淮運使宋文瓚上言:「江陰、通泰,江海之門戶,而鎮江、真州次之,國初設萬戶府以鎮其地。今戍將非人,致使賊艦往來無常。集慶花山劫賊才三十六人,官軍萬數,不能進討,反為所敗,後竟假手鹽徒,雖能成功,豈不貽笑!宜亟選智勇,以任兵柄,以圖後功。不然,東南五省租稅之地,恐非國家之有。」不聽。撥山東地土十六萬二千餘頃屬大承天護聖寺。乙巳,中書戶部言:「各處水旱,田禾不收,湖廣、雲南盜賊蜂起,兵費不給,而各位怯薛冗食甚多,乞賜分柬。」帝牽於眾請,令三年後減之。庚戌,太陰犯天廩。懷慶路饑。徭賊吳天保復陷武岡,命湖廣行省平章政事茍爾領兵討之。以河決,命工部尚書迷兒馬哈謨行視金堤。甲寅,徭賊吳天保陷靖州,命威順王寬徹不花、鎮南王孛羅不花及湖廣、江西二省以兵討之。丁巳,命中書平章政事太平為左丞相,辭,不允。戊午,命河南、山東都府發兵討湖廣洞蠻。己未,以中書省平章政事韓嘉訥為陜西行臺御史大夫。迤北荒旱缺食,遣使賑濟驛戶。丁卯,海北、湖南徭賊竊發,兩月餘,有司不以聞,詔罪之,並降散官一等。是月,馬札兒臺薨,召脫脫還京師。



 十二月庚午,以中書左丞相朵兒只為右丞相,平章政事太平為左丞相,詔天下。丙子,以連年水旱,民多失業,選臺閣名臣二十六人出為郡守縣令,仍許民間利害實封呈省。壬午,晉寧、東昌、東平、恩州、高唐等處民饑,賑鈔十四萬錠、米六萬石。丙戌,中書省臣建議,以河南盜賊出入無常,宜分撥達達軍與楊州舊軍於河南水陸關隘戍守,東至徐、邳,北至夾馬營,遇賊掩捕,從之。是月,陜西行御史臺臣劾奏,別兒怯不花乃逆臣之親子,不可居太保之職,不從。是歲,置中書議事平章四人。隆福宮三皇後弘吉剌氏木納失里薨。



 八年春正月戊戌朔,命也先帖木兒知樞密院事。丁未,享於太廟。辛亥,黃河決,遷濟寧路於濟州。詔:「各官府諳練事務之人,毋得遷調。」詔翰林國史院纂修后妃、功臣列傳,學士承旨張起巖、學士楊宗瑞、侍講學士黃溍為總裁官,左丞相太平、左丞呂思誠領其事。甲子,木憐等處大雪,羊馬凍死,賑之。是月,詔給銅虎符,以宮尉完者不花、貴赤衛副指揮使壽山監湖廣軍。命湖廣行省右丞禿赤、湖南宣慰都元帥完者帖木兒討莫磐洞諸蠻,斬首數百級,其餘二十餘洞,縛其洞首楊鹿五赴京師。



 二月癸酉,御史大夫納麟加太尉致仕。乙亥,以北邊沙土苦寒,罷海剌禿屯田。丙子,命太子愛猷識理達臘習讀畏吾兒文字。庚辰,太陰犯軒轅。癸未,太陰犯平道。甲申,命星吉為江南行臺御史大夫。壬辰,太平言:「孛答、乃禿、忙兀三處屯田,世祖朝以行營舊站撥屬虎賁司,後為豪有力者所奪,遂失其利,今宜仍前撥還。」從之。是月,以前奉使宣撫賈惟貞稱職,特授永平路總管。會歲饑,惟貞請降鈔四萬餘錠賑之。詔濟寧鄆城立行都水監,以賈魯為都水。



 三月丁酉,詔以束帛旌郡縣守令之廉勤者。遼東鎖火奴反,詐稱大金子孫,水達達路脫脫禾孫唐兀火魯火孫討擒之。壬寅,土番盜起,有司請不拘資級,委官討之。福建盜起,地遠,難於討捕,詔汀、漳二州立分元帥府轄之。癸卯,帝親試進士七十有八人,賜阿魯輝帖木兒、王宗哲進士及第,餘出身有差。己酉,湖廣行省遣使獻石壁洞蠻捷。丙辰,太陰犯建星。己未,遣使詣江浙、江西、湖廣、四川、雲南銓福建、番、廣蠻夷等處官員選。辛酉,遼陽兀顏撥魯歡妄稱大金子孫,受玉帝符文,作亂,官軍討斬之。壬戌,《六條政類》書成。京畿民饑。徽州路達魯花赤哈剌不花以政績聞,詔賜金帛旌之。是月,徭賊吳天保復寇沅州。



 夏四月辛未,河間等路以連年河決,水旱相仍,戶口消耗,乞減鹽額,詔從之。乙亥,帝幸國子學,賜衍聖公銀印,升秩從二品。定弟子員出身及奔喪、省親等法。詔:「守令選立社長,專一勸課農桑。」詔:「京官三品以上,歲舉守令一人,守令到任三月,亦舉一人自代。其玉典赤、拱衛百戶,不得授縣達魯花赤,止授佐貳,久著廉能則用之。」平江、松江水災,給海運糧十萬石賑之。丁丑,遼陽董哈剌作亂,鎮撫欽察討擒之。己卯,海寧州沐陽縣等處盜起,遣翰林學士禿堅不花討之。是月,享於太廟。車駕時巡上都。命脫脫為太傅。湖廣章伯顏引兵捕土寇莫萬五、蠻雷等,已而廣西峒賊乘隙入寇,伯顏退走。



 五月丁酉朔,大霖雨,京城崩。庚子,廣西山崩,水湧,漓江溢,平地水深二丈餘,屋宇、人畜漂沒。壬子,寶慶大水。丁巳,四川旱,饑,禁酒。六月丙寅朔,升徐州為總管府,以邳、宿、滕、嶧四州隸之。丙戌,立司天臺於上都。是月,山東大水,民饑,賑之。



 秋七月丙申朔,日有食之。辛丑,復立五道河屯田。乙巳,享於太廟。旌表大都節婦鞏氏門。戊申,西北邊軍民饑,遣使賑之。壬子,量移竄徙官於近地安置,死者聽歸葬。乙卯,遣使祭曲阜孔子廟。江州路總管劉恆有政績,升授山東宣慰使。丙辰,以阿剌不花為大司徒。



 八月丙子,太陰犯壘壁陣。己卯,山東雨雹。是月,車駕還自上都。九月己未,太陰犯靈臺。



 冬十月丁亥,廣西蠻掠道州。



 十一月辛亥,徭賊吳天保率眾六萬掠全州。是歲,詔賜高年帛,設分元帥府於沂州,以買列的為元帥,備山東寇。臺州方國珍為亂,聚眾海上,命江浙行省參知政事朵兒只班討之。監察御史張楨劾太尉阿乞剌欺罔之罪,又言:「明裏董阿、也裏牙、月魯不花,皆陛下不共戴天之仇,伯顏賊殺宗室嘉王、郯王一十二口,稽之古法,當伏門誅,而其子、兄弟尚仕於朝,宜急誅竄。別兒怯不花阿附權奸,亦宜遠貶。今災異迭見,盜賊蜂起,海寇敢於要君,閫帥敢於玩寇,若不振舉,恐有唐末籓鎮噬臍之禍。」不聽。監察御史李泌言:「世祖誓不與高麗共事,陛下踐世祖之位,何忍忘世祖之言,乃以高麗奇氏亦位皇后。今災異屢起,河決地震,盜賊滋蔓,皆陰盛陽微之象,乞仍降為妃,庶幾三辰奠位,災異可息。」不聽。



\end{pinyinscope}