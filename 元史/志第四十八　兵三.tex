\article{志第四十八 兵三}

\begin{pinyinscope}

 ○馬政



 西北馬多天下,秦、漢而下,載籍蓋可考已。元起朔方,俗善騎射,因以弓馬之利取天下,古或未之有。蓋其沙漠萬里,牧養蕃息,太僕之馬,殆不可以數計,亦一代之盛哉。



 世祖中統四年,設群牧所,隸太府監。尋升尚牧監,又升太僕院,改衛尉院。院廢,立太僕寺,屬之宣徽院。後隸中書省,典掌御位下、大斡耳朵馬。其牧地,東越耽羅,北𧾷俞火裏禿麻,西至甘肅,南暨雲南等地,凡一十四處,自上都、大都以至玉你伯牙、折連怯呆兒,周回萬里,無非牧地。



 馬之群,或千百,或三五十,左股烙以官印,號大印子馬。其印有兵古、貶古、闊卜川、月思古、斡欒等名。牧人曰哈赤、哈剌赤;有千戶、百戶,父子相承任事。自夏及冬,隨地之宜,行逐水草,十月各至本地。朝廷歲以九月、十月遣寺官馳驛閱視,較其多寡,有所產駒,即烙印取勘,收除見在數目,造蒙古、回回、漢字文冊以聞,其總數蓋不可知也。凡病死者三,則令牧人償大牝馬一,二則償二歲馬一,一則償牝羊一,其無馬者以羊、駝、牛折納。



 太廟祀事暨諸寺影堂用乳酪,則供牝馬;駕仗及宮人出入,則供尚乘馬。車駕行幸上都,太僕卿以下皆從,先驅馬出健德門外,取其肥可取乳者以行,汰其羸瘦不堪者還於群。自天子以及諸王百官,各以脫羅氈置撒帳,為取乳室。車駕還京師,太僕卿先期遣使征馬五十醖都來京師。醖都者,承乳車之名也。既至,俾哈赤、哈剌赤之在朝為卿大夫者,親秣飼之,日釀黑馬乳以奉玉食,謂之細乳。每醖都,牝馬四十。每牝馬一,官給芻一束、菽八升。駒一,給芻一束、菽五升。菽貴,則其半以小稻充。自諸王百官而下,亦有馬乳之供,醖都如前之數,而馬減四之一,謂之細乳。芻粟要旬取給於度支,寺官亦以旬詣閑MI閱肥瘠。又自世祖而下山陵,各有醖都,取馬乳以供祀事,號金陵擠馬,越五年,盡以與守山陵使者。



 凡御位下、正宮位下、隨朝諸色目人員,甘肅、土番、耽羅、雲南、占城、蘆州、河西、亦奚卜薛、和林、斡難、怯魯連、阿剌忽馬乞、哈剌木連、亦乞里思、亦思渾察、成海、阿察脫不罕、折連怯呆兒等處草地,內及江南、腹裏諸處,應有系官孳生馬、牛、駝、驢、羊點數之處,一十四道牧地,各千戶、百戶等名目如左:



 東路折連怯呆兒等處,玉你伯牙、上都周圍,哈剌木連等處,阿剌忽馬乞等處,斡斤川等處,阿察脫不罕等處,甘州等處,左手永平等處,右手固安州等處,雲南亦奚卜薛,蘆州,益都,火裏禿麻,高麗耽羅國。



 一,折連怯呆兒等處御位下:折連怯呆兒地哈剌赤千戶買買、買的、撒臺、怯兒八思、闊闊來、塔失鐵木兒、哈剌那海、伯要鷿、也兒的思、撒的迷失、教化、太鐵木兒、塔都、也先、木薛肥、不思塔八、不兒都、麻失不顏臺、撒敦。按赤、忽裏哈赤千戶下百戶脫脫木兒。兀魯兀內土阿八剌哈赤闊闊出。徹徹地撒剌八。薛里溫、你裡溫、斡脫忽赤、哈剌鐵木兒。哈思罕地僧家奴。玉你伯牙斷頭山百戶哈只。



 一,玉你伯牙等處御位下:玉你伯牙地哈剌赤百戶忽兒禿哈、兀都蠻、燕鐵木兒、暗出忽兒、也先禿滿、玉龍鐵木兒、月思哥、明裏不蘭。



 大斡耳朵位下:乞剌裏郭羅赤馬某等。哈里牙兒茍赤別鐵木兒。伯只剌茍赤阿藍答兒。阿察兒伯顏茍赤教化的等。塔魯內亦兒哥赤、塔裏牙赤等。伯只剌阿塔赤忽兒禿哈。桃山太師月赤察兒分出鐵木兒等。伯顏只魯干阿塔赤禿忽魯等。玉你伯牙奴禿赤、火你赤。



 一,哈剌木連等處御位下:阿失溫忽都地八都兒。希徹禿地吉兒鷿。哈察木敦。火石腦兒哈塔、咬羅海牙、撒的。換撒裡真按赤哈答。須知忽都哈剌赤別乞。軍腦兒哈剌赤火羅思。玉龍占徹。雲內州拙里牙赤昌罕。察罕腦兒欠昔思。棠樹兒安魯罕。石頭山禿忽魯。牙不罕你裡溫脫脫木兒。開成路黑水河不花。



 大斡耳朵位下:完者。



 一,阿剌忽馬乞等處御位下:阿剌忽馬乞地哈剌赤百戶按不憐、乾鐵哥、火石鐵木兒、末赤、卯罕、不蘭奚、孛羅罕。怯魯連地哈剌赤千戶床八失,百戶怯兒的、小薛干、別鐵列不作、孛羅、串都、也速、典列、坦的裡、也裏迷失、忙兀鷿。斡難地蘭盞兒、未者、哈只不花等。



 大斡耳朵位下:阿剌忽馬乞按灰等。闊苦地闊赤斤等。



 一,斡斤川等處御位下:斡斤川地哈剌赤千戶月魯、阿剌鐵木兒、塔塔塔察兒。拙里牙赤斡羅孫,馬塔哈兒哈地哈剌為千戶當失、燕忽裏,歡差太難。闊闊地兀奴忽赤忙兀鷿。怯魯連八剌哈赤八兒麻思。



 大斡耳朵位下:馬塔哈兒哈怯連口只兒哈忽。



 一,阿察脫不罕等處御位下:阿察脫不罕地哈赤守納。斡川札馬昔寶赤忙哥撒兒。火羅罕按赤禿忽赤。青海後火義罕塔兒罕、按赤也先。黃兀兒不剌按赤末兒哥、忽林失。應裡哥地按赤哈丹、忽臺迷失。應吉列古哈剌赤不魯。亦兒渾察西哈剌赤。答蘭速魯哈剌赤八隻吉兒。哈兒哈孫不剌哈剌赤阿兒禿。



 大斡耳朵位下:怯魯連火你赤塔剌海。



 一,甘州等處御位下:口千子哈剌不花一所。奧魯赤一所。阿剌沙阿蘭山兀都蠻。亦不剌金一所。寬徹干。塔塔安地普安。勝回地劉子總管。闊闊思地太鐵木兒等。甘州等處楊住普。撥可連地撒兒吉思。只哈禿屯田地安童一所。哈剌班忽都拙里牙赤耳眉。



 一,左手永平等處御位下:永平地哈剌赤千戶六十。樂亭地拙里牙赤、阿都赤、答剌赤迷裏迷失,亦兒哥赤馬某撒兒答。香河按赤定住、亦馬赤速哥鐵木兒。河西務愛牙赤孛羅鷿。漷州哈剌赤脫忽察。桃花島青昔寶赤赤班等。



 大斡耳朵位下:河西務玉提赤百戶馬札兒。



 一,右手固安州四怯薛八剌哈赤平章那懷為長:固安州哈剌赤脫忽察,哈赤忽裏哈赤、按赤不都兒。真定昔寶赤脫脫。左衛哈剌赤塔不鷿。青州哈剌赤阿哈不花。涿州哈剌赤不魯哈思。



 一,雲南亦奚卜薛鐵木兒不花為長。



 一,蘆州。



 一,益都哈剌赤忽都鐵木兒。



 一,火裏禿麻太勝忽兒為長。



 一,高麗耽羅。



 ○屯田



 古者寓兵於農,漢、魏而下,始置屯田為守邊之計。有國者善用其法,則亦養兵息民之要道也。國初,用兵征討,遇堅城大敵,則必屯田以守之。海內既一,於是內而各衛,外而行省,皆立屯田,以資軍餉。或因古之制,或以地之宜,其為慮蓋甚詳密矣。大抵芍陂、洪澤、甘、肅、瓜、沙,因昔人之制,其地利蓋不減於舊;和林、陜西、四川等地,則因地之宜而肇為之,亦未嘗遺其利焉。至於雲南八番,海南、海北,雖非屯田之所,而以為蠻夷腹心之地,則又因制兵屯旅以控扼之。由是而天下無不可屯之兵,無不可耕之地矣。今故著其建置增損之概,而內外所轄軍民屯田,各以次列焉。



 樞密院所轄



 左衛屯田:世祖中統三年三月,調樞密院二千人,於東安州南、永清縣東荒土及本衛元占牧地,立屯開耕,分置左右手屯田千戶所,為軍二千名,為田一千三百一十頃六十五畝。



 右衛屯田:世祖中統三年三月,調本衛軍二千人,於永清、益津等處立屯開耕,分置左右手屯田千戶所。其屯軍田畝之數,與左衛同。



 中衛屯田:世祖至元四年,於武清、香河等縣置立。十一年,以各屯地界,相去百餘里,往來耕作不便,遷於河西務、荒莊、楊家口、青臺、楊家白等處。其屯軍之數,與左衛同,為田一千三十七頃八十二畝。



 前衛屯田:世祖至元十五年九月,以各省軍人備侍衛者,於霸州、保定、涿州荒閑地土屯種,分置左右手屯田千戶所。屯軍與左衛同,為田一千頃。



 後衛屯田:置立歲月與前衛同。後以永清等處田畝低下,遷昌平縣之太平莊。泰定三年五月,以太平莊乃世祖經行之地,營盤所在,春秋往來,牧放衛士頭匹,不宜與漢軍立屯,遂罷之,止於舊立屯所,耕作如故。屯軍與左衛同,為田一千四百二十八頃一十四畝。



 武衛屯田:世祖至元十八年,發迤南軍人三千名,於涿州、霸州、保定、定興等處置立屯田,分設廣備、萬益等六屯,別立農政院以領之。二十二年,罷農政院為司農寺,自後與民相參屯種。二十五年,別立屯田萬戶府,分管屯種軍人。二十六年,以屯軍屬武衛親軍都指揮使司,兼領屯田事。仁宗皇慶元年,改屬衛率府,後復歸之武衛。英宗至治元年,命以廣備、利民二千戶軍人所耕地土,與左衛率府忙古鷿屯田千戶所互相更易。屯軍三千名,為田一千八百四頃四十五畝。



 左翼屯田萬戶府:世祖至元二十六年二月,罷蒙古侍衛軍從人之屯田者,別以斡端、別十八里回還漢軍,及大名、衛輝兩翼新附軍,與前、後二衛迤東還戍士卒合並屯田,設左、右翼屯田萬戶府以領之。遂於大都路霸州及河間等處立屯開耕,置漢軍左右手二千戶、新附軍六千戶所,為軍二千五十一名,為田一千三百九十九頃五十二畝。



 右翼屯田萬戶府:其置立歲月與左翼同。成宗大德元年十一月,發真定軍人三百名,於武清縣崔黃口增置屯田。仁宗延祐五年四月,立衛率府,以本府屯田並屬詹事院,後復歸之樞密,分置漢軍千戶所三,別置新附軍千戶所一,為軍一千五百四十人,為田六百九十九頃五十畝。



 忠翊侍衛屯田:世祖至元二十九年十一月,命各萬戶府,摘大同、隆興、太原、平陽等處軍人四千名,於燕只哥赤斤地面及紅城周回,置立屯田,開耕荒田二千頃,仍命西京宣慰司領其事,後改立大同等處屯儲萬戶府以領之。成宗大德十一年,改侍衛親軍都指揮使司,仍領屯田。武宗至大四年,以黃華嶺新附屯田軍一千人並歸本衛,別立屯署。是年,改大同侍衛為中都威衛,屬之徽政院,分屯軍二千置弩軍翼,止以二千人分置左右手屯田千戶所,黃華嶺新附軍屯如故。仁宗延祐二年,遷紅城屯軍於古北口、太平莊屯種。五年,復簽中都威衛軍八百人,於左都威衛所轄地內別立屯署。七年十二月,罷左都威衛及太平莊、白草營等處屯田,復於紅城周回立屯,仍屬中都威衛。英宗至治元年,始改為忠翊侍衛,屯田如故,為田二千頃。後移置屯所,不知其數。



 左、右欽察衛屯田:世祖至元二十四年,發本衛軍一千五百一十二名,分置左右手屯田千戶所及欽察屯田千戶所,於清州等處屯田。英宗至治二年,始分左、右欽察衛,以左右手屯田千戶所分屬之。文宗天歷二年,創立龍翊侍衛,復以隸焉。為軍左手千戶所七百五名,右手千戶所四百三十七名,欽察千戶所八百名。為田左手千戶所一百三十七頃五十畝,右手千戶所二百一十八頃五十畝,欽察千戶所三百頃。



 左衛率府屯田:武宗至大元年六月,命於大都路漷州武清縣及保定路新城縣置立屯田。英宗至治元年,以武衛與左衛率府屯田地界相離隔絕,不便耕作,命以兩衛屯地互更易之,分置三翼屯田千戶所,為軍三千人,為田一千五百頃。



 宗仁衛屯田:英宗至治二年八月,發五衛漢軍二千人,於大寧等處創立屯田,分置兩翼屯田千戶所,為田二千頃。



 宣忠扈衛屯田:文宗至順元年十二月,命收聚訖一萬斡羅斯,給地一百頃,立宣忠扈衛親軍萬戶府屯田,依宗仁衛例。



 大司農司所轄



 永平屯田總管府:世祖至元二十四年八月,以北京採取材木百姓三千餘戶,於灤州立屯,設官署以領其事,為戶三千二百九十,為田一萬一千六百一十四頃四十九畝。



 營田提舉司:不詳其建置之始,其設立處所在大都漷州之武清縣,為戶軍二百五十三,民一千二百三十五,析居放良四百八十,不蘭奚二百三十二,火者一百七十口,獨居不蘭奚一十二口,黑瓦木丁八十二名,為田三千五百二頃九十三畝。



 廣濟署屯田:世祖至元二十二年正月,以崔黃口空城屯田,歲澇不收,遷於清、滄等處。後大司農寺以尚珍署舊領屯夫二百三十戶歸之,既又遷濟南、河間五百五十戶,平灤、真定、保定三路屯夫四五百十戶,並入本屯,為戶共一千二百三十,為田一萬二千六百頃三十八畝。



 宣徽院所轄



 淮東淮西屯田打捕總官府:世祖至元十六年,募民開耕漣、海州荒地,官給禾種,自備牛具,所得子粒官得十之四,民得十之六,仍免屯戶徭役,屢欲中廢不果。二十七年,所轄提舉司一十九處並為十二。其後再並,止設八處,為戶一萬一千七百四十三,為田一萬五千一百九十三頃三十九畝。



 豐閏署:世祖至元二十二年,創立於大都路薊州之豐閏縣,為戶八百三十七,為田三百四十九頃。



 寶坻屯:世祖至元十六年,簽大都屬邑編民三百戶,立屯於大都之寶坻縣,為田四百五十頃。



 尚珍署:世祖至元二十三年,置立於濟寧路之兗州,為戶四百五十六,為田九千七百一十九頃七十二畝。



 腹里所轄軍民屯田



 大同等處屯儲總管屯田:成宗大德四年,以西京黃華嶺等處田土頗廣,發軍民九千餘人,立屯開耕。六年,始設屯儲軍民總管萬戶府。十一年,放罷漢軍還紅城屯所,止存民夫在屯。仁宗時,改萬戶府為總管府,為戶軍四千二十,民五千九百四十五,為田五千頃。



 虎賁親軍都指揮使司屯田:世祖至元十七年十二月,月兒魯官人言:「近於滅捏怯土、赤納赤、高州、忽蘭若班等處,改置驛傳,臣等議,可於舊置驛所設立屯田。」從之。二十八年,發虎賁親軍二千人入屯。二十九年,增軍一千,凡立三十四屯,於上都置司,為軍三千人,佃戶七十九,為田四千二百二頃七十九畝。



 嶺北行省屯田



 世祖至元二十一年,並和林阿剌鷿元領軍一千人入五條河。成宗元貞元年,摘六衛漢軍一千名,赴稱海屯田。大德三年,以五條河漢軍悉並入稱海。仁宗延祐三年,罷青海屯田,復立屯於五條河。六年,分揀蒙古軍五千人,復屯田青海。七年,命依世祖舊制,青海、五條河俱設屯田,發軍一千人於五條河立屯。英宗時,立屯田萬戶府,為戶四千六百四十八,為田六千四百餘頃。



 遼陽等處行中書省所轄屯田



 大寧路海陽等處打捕屯田所:世祖至元二十三年,以大寧、遼陽、平灤諸路拘刷漏籍、放良、孛蘭奚人戶,及僧道之還俗者,立屯於瑞州之西瀕海荒地開耕,設打捕屯田總管府。成宗大德四年,罷之,止立打捕屯田所,為戶元撥並召募共一百二十二,為田二百三十頃五十畝。



 浦峪路屯田萬戶府:世祖至元二十九年十月,為蠻軍三百戶、女直一百九十戶,於咸平府屯種。三十年,命本府萬戶和魯古鷿領其事,仍於茶剌罕、剌憐等處立屯。三十一年,罷萬戶府屯田。仁宗大德二年,撥蠻軍三百戶屬肇州蒙古萬戶府,止存女直一百九十戶,依舊立屯,為田四百頃。



 金復州萬戶府屯田:世祖至元二十一年五月,發新附軍一千二百八十一戶,於忻都察置立屯田。二十六年,分京師應役新附軍一千人,屯田哈思罕關東荒地。三十年,以玉龍帖木兒、塔失海牙兩萬戶新附軍一千三百六十戶,並入金復州,立屯耕作,為戶三千六百四十一,為田二千五百二十三頃。



 肇州蒙古屯田萬戶府:成宗元貞元年七月,以乃顏不魯古赤及打魚水達達、女直等戶,於肇州旁近地開耕,為戶不魯古赤二百二十戶,水達達八十戶,歸附軍三百戶,續增漸丁五十二戶。



 河南行省所轄軍民屯田



 南陽府民屯:世祖至元二年正月,詔孟州之東,黃河之北,南至八柳樹、枯河、徐州等處,凡荒閑地土,可令阿、阿剌罕等所領士卒,立屯耕種,並摘各萬戶所管漢軍屯田。六年,以攻襄樊軍餉不足,發南京、河南、歸德諸路編民二萬餘戶,於唐、鄧、申、裕等處立屯。八年,散還元屯戶,別簽南陽諸色戶計,立營田使司領之。尋罷,改立南陽屯田總管府。後復罷,止隸有司,為戶六千四十一,為田一萬六百六十二頃七畝。



 洪澤萬戶府屯田:世祖至元二十三年,立洪澤南北三屯,設萬戶府以統之。先是,江淮行省言:「國家經費,糧儲為急,今屯田之利,無過兩淮,況芍陂、洪澤皆漢、唐舊嘗立屯之地,若令江淮新附漢軍屯田,可歲得糧百五十餘萬石。」至是從之。三十一年,罷三屯萬戶,止立洪澤屯田萬戶府以統之。其置立處所,在淮安路之白水塘、黃家畽等處,為戶一萬五千九百九十四名,為田三萬五千三百一十二頃二十一畝。



 芍陂屯田萬戶府:世祖至元二十一年二月,江淮行省言:「安豐之芍陂,可溉田萬餘頃,乞置三萬人立屯。」中書省議:「發軍士二千人,姑試行之。」後屯戶至一萬四千八百八名。



 德安等處軍民屯田總管府:世祖至元十八年,以各翼取到漢軍,及各路拘收手號新附軍,分置十屯,立屯田萬戶府。三十一年,改立總管府,為民九千三百七十五名,軍五千九百六十五名,為田八千八百七十九頃九十六畝。



 陜西等處行中書省所轄軍民屯田



 陜西屯田總管府:世祖至元十一年正月,以安西王府所管編民二千戶,立櫟陽、涇陽、終南、渭南屯田。十八年,立屯田所。十九年,以軍站屯戶拘收為怯憐口戶計,放還而無所歸者,籍為屯戶,立安西、平涼屯田,設提領所以領之。二十九年,立鳳翔、鎮原、彭原屯田,放罷至元十年所簽接應成都、延安軍人,置立民屯,設立屯田所,尋改為軍屯,令千戶所管領。三十年,復更為民屯,為戶鳳翔一千一百二十七戶;鎮原九百一十三戶;櫟陽七百八十六戶,後存六百五十戶;涇陽六百九十六戶,後存六百五十八戶;彭原一千二百三十八戶;安西七百二十四戶,後存二百六十二戶;平涼二百八十八戶;終南七百七十一戶,後存七百一十三戶;渭南八百一十一戶,後存七百六十六戶。為田鳳翔九十頃一十二畝,鎮原四百二十六頃八十五畝,櫟陽一千二十頃九十九畝,涇陽一千二十頃九十九畝,彭原五百四十五頃六十八畝,安西四百六十七頃七十八畝,平涼一百一十五頃二十畝,終南九百四十三頃七十六畝,渭南一千二百二十二頃三十一畝。



 陜西等處萬戶府屯田:世祖至元十九年二月,以盩厔南系官荒地,發歸附軍,立孝子林、張馬村軍屯。二十年,以南山把口子巡哨軍人八百戶,於盩厔之杏園莊、寧州之大昌原屯田。二十一年,發文州鎮戍新附軍九百人,立亞柏鎮軍屯,復以燕京戍守新附軍四百六十三戶,於德順州之威戎立屯開耕。為戶孝子林屯三百一戶,張馬村屯三百一十三戶,杏園莊屯二百三十三戶,大昌原屯四百七十四戶,亞柏鎮屯九百戶,威戎屯四百六十三戶。為田孝子林二十三頃八十畝,張馬村七十三頃八十畝,杏園莊一百一十八頃三十畝,大昌原一百五十八頃七十九畝,亞柏鎮二百六十八頃五十九畝,威戎一百六十四頃八十畝。



 貴赤延安總管府屯:世祖至元十九年,以拘收贖身、放良、不蘭奚及漏籍戶計,於延安路探馬赤草地屯田,為戶二千二十七,為田四百八十六頃。



 甘肅等處行中書省轄所軍民屯田



 寧夏等處新附軍萬戶府屯田:世祖至元十九年三月,發迤南新附軍一千三百八十二戶,往寧夏等處屯田。二十一年,遣塔塔裏千戶所管軍人九百五十八戶屯田,為田一千四百九十八頃三十三畝。



 管軍萬戶府屯田:世祖至元十八年正月,命肅州、沙州、瓜州置立屯田。先是,遣都元帥劉恩往肅州諸郡,視地之所宜,恩還言宜立屯田,遂從之。發軍於甘州黑山子、滿峪、泉水渠、鴨子翅等處立屯,為戶二千二百九十,為田一千一百六十六頃六十四畝。



 寧夏營田司屯田:世祖至元八年正月,簽發己未年隨州、鄂州投降人民一千一百七戶,往中興居住。十一年,編為屯田戶,凡二千四百丁。二十三年,續簽漸丁,得三百人,為田一千八百頃。



 寧夏路放良官屯田:世祖至元十一年,從安撫司請,以招收放良人民九百四戶,編聚屯田,為田四百四十六頃五十畝。



 亦集乃屯田:世祖至元十六年,調歸附軍人於甘州,十八年,以充屯田軍。二十二年,遷甘州新附軍二百人,往屯亦集乃合即渠開種,為田九十一頃五十畝。



 江西等處行中書省所轄屯田



 贛州路南安寨兵萬戶府屯田:成宗大德二年正月,以贛州路所轄信豐、會昌、龍南、安遠等處,賊人出沒,發寨兵及宋舊役弓手,與抄數漏籍人戶,立屯耕守,以鎮遏之,為戶三千二百六十五,為田五百二十四頃六十八畝。



 江浙等處行中書省所轄屯田



 汀、漳屯田:世祖至元十八年,以福建調軍糧儲費用,依腹裏例,置立屯田,命管軍總管鄭楚等,發鎮守士卒年老不堪備征戰者,得百有十四人,又募南安等縣居民一千八百二十五戶,立屯耕作。成宗元貞三年,命於南詔、黎、畬各立屯田,摘撥見戍軍人,每屯置一千五百名,及將所招陳吊眼等餘黨入屯,與軍人相參耕種。為戶汀州屯一千五百二十五名,漳州屯一千五百一十三名。為田汀州屯二百二十五頃,漳州屯二百五十頃。



 高麗國立屯



 高麗屯田:世祖至元七年創立,是時東征日本,欲積糧餉,為進取之計,遂以王綧、洪茶丘等所管高麗戶二千人,及發中衛軍二千人,合婆娑府、咸平府軍各一千人,於王京東寧府、鳳州等一十處,置立屯田,設經略司以領其事,每屯用軍五百人。



 四川行省所轄軍民屯田二十九處



 廣元路民屯:世祖至元十三年,從利州路元帥言,廣元實東西兩川要沖,支給浩繁,經理系官田畝,得九頃六十畝,遂以褒州刷到無主人口,偶配為十戶,立屯開種。十八年,發新得州編民七十七戶屯田,為戶共八十七。



 敘州宣撫司民屯:世祖至元十一年,命西蜀四川經略使起立屯田。十五年,簽長寧軍、富順州等處編民四百七十五戶,立屯耕種。十九年,續簽一百六十戶。二十年,敘州簽民一千九百戶。二十五年,富順州復簽民六百八戶,增人舊屯。二十七年,取勘析出屯戶,得二百八十四。成宗元貞二年,復放罷站戶一千一十七戶,依舊屯田。總之為戶四千四百四十四。



 紹慶路民屯:世祖至元十九年,於本路未當差民戶內,簽二十三戶,置立屯田。二十年,於彭水縣籍管萬州寄戶內,簽撥二十戶。二十一年,簽彭水縣未當差民戶三十二戶增入。二十六年,屯戶貧乏者多負逋,復簽彭水縣編民一十六戶補之。為戶九十一。



 嘉定路民屯:世祖至元十九年,簽亡宋編民四戶,置立屯田。成宗元貞元年,撥成都義士軍八戶增入。為戶一十二。



 順慶路民屯:世祖至元十二年,簽順慶民三千四百六十八戶,置立屯田。十九年,復於民戶內差撥一千三百三十六戶置民屯。二十年,復簽二百一十二戶增入。總之五千一十六戶。



 潼州府民屯:世祖至元十一年,簽本府編民及義士軍二千二百二十四戶,立屯。十三年,復簽民一百四十二戶。二十一年,行省遣使於遂寧府擇監夫之老弱廢疾者,得四十六戶,簽充屯戶。總之二千四百一十二戶。



 夔路總管府民屯:世祖至元十一年置,累簽本路編民至五千二十七戶,續於新附軍內簽老弱五十六戶增入。



 重慶路民屯:世祖至元十一年置,累於江津、巴縣、瀘州、忠州等處,簽撥編民二千三百八十七戶,並召募,共三千五百六十六戶。



 成都路民屯:世祖至元十三年,簽陰陽人四十戶,辦納屯糧。二十二年,續簽瀘州編民九十七戶,充屯田戶。三十一年,續簽千戶高德所管民一十四戶。



 保寧萬戶府軍屯:世祖至元二十六年,保寧府言:「本管軍人,一戶或二丁三丁,父兄子弟應役,實為重並,若又遷於成都屯種,去家隔遠,逃匿必多。乞令本府在營士卒,及夔路守鎮軍人,止於保寧沿江屯種。」從之。簽軍一千二百名。二十七年,發屯軍一百二十九人,從萬戶也速迭兒西征,別簽漸丁軍人入屯,為戶一千三百二十九名,為田一百一十八頃二十七畝。



 敘州等處萬戶府軍屯:成宗元貞二年,改立敘州軍屯,遷遂寧屯軍二百三十九人,於敘州宣化縣喁口上下荒地開耕,為田四十一頃八十三畝。



 重慶五路守鎮萬戶府軍屯:仁宗延祐七年,發軍一千二百人,於重慶路三堆、中嶆、趙市等處屯耕,為田四百二十頃。



 夔路萬戶府軍屯:世祖至元二十一年,從四川行省議,除沿邊重地,分軍鎮守,餘軍一萬人,命官於成都諸處擇膏腴地,立屯開耕,為戶三百五十一人,為田五十六頃七十畝,凡創立十四屯。



 成都等路萬戶府軍屯:於本路崇慶州義興鄉楠木園置立,為戶二百九十九人,為田四十二頃七十畝。



 河東陜西等路萬戶府軍屯:置立於灌州之青城、陶壩及崇慶州之大柵頭等處,為戶一千三百二十八名,為田二百八頃七畝。



 廣安等處萬戶府軍屯:置立於成都路崇慶州之七寶壩,為戶一百五十名,為田二十六頃二十五畝。



 保寧萬戶府軍屯:置立於崇慶州晉原縣之金馬,為戶五百六十四名,為田七十五頃九十五畝。



 敘州萬戶府軍屯:置立於灌州之青城,為戶二百二十一名,為田三十八頃六十七畝。



 五路萬戶府軍屯:置立於成都路崇慶州之大柵鎮孝感鄉及灌州青城縣之懷仁鄉,為戶一千一百六十一名,為田二百三頃一十七畝。



 興元金州等處萬戶府軍屯:置立於崇慶州晉原縣孝感鄉,為戶三百四十四名,為田五十六頃。



 隨路八都萬戶府軍屯:置立於灌州青城、溫江縣,為戶八百三十二名,為田一百六十二頃五十七畝。



 舊附等軍萬戶府軍屯:置立於灌州青城縣、崇慶州等處,為戶一千二名,為田一百二十九頃五十畝。



 砲手萬戶府軍屯:置立於灌州青城縣龍池鄉,為戶九十六名,為田一十六頃八十畝。



 順慶軍屯:置立於晉原縣義興鄉、江源縣將軍橋,為戶五百六十五名,為田九十八頃八十七畝。



 平陽軍屯:置立於灌州青城、崇慶州大柵頭,為戶三百九十八名,為田六十九頃六十五畝。



 遂寧州軍屯:為戶二千名,為田三百五十頃。



 嘉定萬戶府軍屯:世祖至元二十一年,摘蒙古、漢軍及嘉定新附軍三百六十人,於崇慶州、青城等處屯田。二十八年,還之元翼,止餘屯軍一十三名,為田二頃二十七畝。



 順慶等處萬戶府軍屯:世祖至元二十六年,發軍於沿江下流漢初等處屯種,為戶六百五十六名,為田一百一十四頃八十畝。



 廣安等處萬戶府軍屯:世祖至元二十七年,撥廣安舊附漢軍一百一十八名,於新明等處立屯開耕,為田二十頃六十五畝。



 雲南行省所轄軍民屯田一十二處



 威楚提舉司屯田:世祖至元十五年,於威楚提舉鹽使司拘刷漏籍人戶充民屯,本司就領其事,與中原之制不同,為戶三十三,為田一百六十五雙。



 大理金齒等處宣尉司都元帥府軍民屯:世祖至元十二年,命於所轄州縣拘刷漏籍人戶,得二千六十有六戶,置立屯田。十四年,簽本府編民四百戶益之。十八年,續簽永昌府編民一千二百七十五戶增入。二十六年,立大理軍屯,於爨僰軍內撥二百戶。二十七年,復簽爨僰軍人二百八十一戶增入。二十八年,續增一百一十九戶。總之民屯三千七百四十一戶,軍屯六百戶,為田軍民己業二萬二千一百五雙。



 鶴慶路軍民屯田:世祖至元十二年,簽鶴慶路編民一百戶立民屯。二十七年,簽爨僰軍一百五十二戶立軍屯,為田軍屯六百八雙,民屯四百雙,俱己業。



 武定路總管府軍屯:世祖至元二十七年,以雲南戍軍糧餉不足,於和曲、祿勸二州爨僰軍內,簽一百八十七戶,立屯耕種,為田七百四十八雙。



 威楚路軍民屯田:世祖至元十二年,立威楚民屯,拘刷本路漏籍人戶,得一千一百雙。二十七年,始立屯軍,於本路爨僰軍內簽三百九十九戶,內一十五戶官給荒田六十雙,餘戶自備己業田一千五百三十六雙。



 中慶路軍民屯田:世祖至元十二年,置立中慶民屯,於所屬州縣內拘刷漏籍人戶,得四千一百九十七戶,官給田一萬七千二十二雙,自備己業田二千六百二雙。二十七年,始立軍屯,用爨僰軍人七百有九戶,官給田二百三十四雙,自備己業田二千六百一雙。



 曲靖等處宣慰司兼管軍萬戶府軍民屯田:世祖至元十二年,立曲靖路民屯,拘刷所轄州郡諸色漏籍人戶七百四十戶立屯。十八年,續簽民一千五百戶增入,其所耕之田,官給一千四百八十雙,自備己業田三千雙。十二年,立澂江民屯,所簽屯戶,與曲靖同,凡一千二百六十戶。二十六年,始立軍屯,於爨僰軍內簽一百六十九戶。二十七年,復簽二百二十六戶增入。十二年,立仁德府民屯,所簽屯戶,與澂江同,凡八十戶,官給田一百六十雙。二十六年,始立軍屯,簽爨僰軍四十四戶。二十七年,續簽五十六戶增入,所耕田畝四百雙,俱系軍人己業。



 烏撒宣慰司軍民屯田:世祖至元二十七年,立烏撒路軍屯,以爨僰軍一百一十四戶屯田。又立東川路民屯,屯戶亦系爨僰軍人,八十六戶,皆自備己業。



 臨安宣慰司兼管軍萬戶府軍民屯田:世祖至元十二年,立臨安民屯二處,皆於所屬州縣拘刷漏籍人戶開耕。宣慰司所管民屯三百戶,田六百雙。本路所管民屯二千戶,田三千四百雙。二十七年,續立爨僰軍屯,為戶二百八十八,為田一千一百五十二雙。



 梁千戶翼軍屯:世祖至元三十年,梁王遣使詣雲南行省言,以漢軍一千人置立屯田。三十一年,發三百人備鎮戍巡邏,止存七百人,於烏蒙屯田,後遷於新興州,為田三千七百八十九雙。



 羅羅斯宣慰司兼管軍萬戶府軍民屯田:世祖至元二十七年,立會通民屯,屯戶系爨僰土軍二戶。十六年,立建昌民屯,撥編民一百四戶。二十三年,發爨僰軍一百八十戶,立軍屯。是年,又立會川路民屯,發本路所轄州邑編民四十戶。十六年,立德昌路民屯,發編民二十一戶。二十年,始立軍屯,發爨僰軍人一百二十戶。



 烏蒙等處屯田總管府軍屯:仁宗延祐三年,立烏蒙軍屯。先是雲南行省言:「烏蒙乃云南咽喉之地,別無屯戍軍馬,其地廣闊,土脈膏腴,皆有古昔屯田之跡,乞發畏吾兒及新附漢軍屯田鎮遏。」至是從之。為戶軍五千人,為田一千二百五十頃。



 湖廣等處行中書省所轄屯田三處



 海北海南道宣慰司都元帥府民屯:世祖至元三十年,召募民戶並發新附士卒,於海南、海北等處置立屯田。成宗元貞元年,以其地多瘴癘,縱屯田軍二千人還各翼,留二千人與召募民之屯種。大德三年,罷屯田萬戶府,屯軍悉令還役,止令民戶八千四百二十八戶屯田,瓊州路五千一十一戶,雷州路一千五百六十六戶,高州路九百四十八戶,化州路八百四十三戶,廉州路六十戶。為田瓊州路二百九十二頃九十八畝,雷州路一百六十五頃五十一畝,高州路四十五頃,化州路五十五頃二十四畝,廉州路四頃八十八畝。



 廣西兩江道宣慰司都元帥撞兵屯田:成宗大德二年,黃聖許叛,逃之交趾,遺棄水田五百四十五頃七畝。部民有呂瑛者,言募牧蘭等處及融慶溪洞徭、撞民丁,於上浪、忠州諸處開屯耕種。十年,平大任洞賊黃德寧等,以其地所遺田土,續置藤州屯田。為戶上浪屯一千二百八十二戶,忠州屯六百一十四戶,那扶屯一千九戶,雷留屯一百八十七戶,水口屯一千五百九十九戶。續增藤州屯,二百八頃一十九畝。



 湖南道宣慰司衡州等處屯田:世祖至元二十五年,調德安屯田萬戶府軍士一千四百六十七名,分置衡州之清化、永州之烏符、武岡之白倉,置立屯田。二十七年,募衡陽縣無土產居民,得九戶,增入清化屯。為戶清化屯軍民五百九戶;烏符屯軍民五百戶,白倉屯同。為田清化屯一百二十頃一十九畝,烏符屯一百三頃五十畝,白倉屯八十六頃九十二畝。



\end{pinyinscope}