\article{志第十四 地理五}

\begin{pinyinscope}

 江浙等處行中書省,為路三十、府一、州二,屬州二十六,屬縣一百四十三。本省陸站一百八十處,水站八十二處。



 江南浙西道肅政廉訪司



 杭州路,上。唐初為杭州,後改餘杭郡,又仍為杭州。五代錢鏐據兩浙,號吳越國。宋高宗南渡,都之,為臨安府。元至元十三年,平江南,立兩浙都督府,又改為安撫司。十五年,改為杭州路總管府。二十一年,自揚州遷江淮行省來治於杭,改曰江浙行省。本路戶三十六萬八百五十,口一百八十三萬四千七百一十。至元二十七年抄籍數。領司二、縣八、州一。



 左、右錄事司。宋高宗建炎三年,遷都杭州,設九廂。元至元十四年,分為四隅錄事司。泰定二年,並為左右二錄事司。



 縣八



 錢塘,上。仁和,上。與錢塘分治城下。餘杭,中。臨安,中。



 新城,中。富陽,中。於潛,中。昌化。中。



 州一



 海寧州,中。唐以來為鹽官縣。元元貞元年,以戶口繁多,升為鹽官州。是年,升江南平陽等縣為州,以戶為差,戶至四萬五萬者為下州,五萬至十萬者為中州。凡為中州者二十八,下州者十五。泰定四年,海圮鹽官。天歷二年,改海寧州。海寧東南皆濱巨海,自唐、宋常有水患,大德、延祐間亦嘗被其害。泰定四年春,其害尤甚,命都水少監張仲仁往治之,沿海三十餘里下石囤四十四萬三千三百有奇,木櫃四百七十餘,工役萬人。文宗即位,水勢始平,乃罷役,故改曰海寧云。



 湖州路,上。唐改吳興郡,又改湖州。宋改安吉州。至元十三年,升湖州路。戶二十五萬四千三百四十五。抄籍戶口數闕,用至順錢糧數。領司一、縣五、州一。



 錄事司。舊設東西南北四廂。至元十三年,立總督四廂。十四年,改錄事司。



 縣五



 烏程,上。歸安,上。與烏程皆為倚郭。安吉,中。德清,下。



 武康。中。



 州一



 長興州,中。唐為綏州,又更名雉州,又為長城縣。硃梁改曰長興。宋因之。元元貞元年,升為州。



 嘉興路,上。唐為嘉興縣。石晉置秀州。宋為嘉禾郡,又升嘉興府。戶四十二萬六千六百五十六,口二百二十四萬五千七百四十二。領司一、縣一、州二。



 錄事司。舊置廂官,元初改為兵馬司。至元十四年,置錄事司。



 縣一



 嘉興。上。倚郭。



 州二



 海鹽州,中。唐為縣,宋因之。元元貞元年升州。



 崇德州,中。石晉置,宋因之。元元貞元年升州。



 平江路,上。唐初為蘇州,又改吳郡,又仍為蘇州。宋為平江府。元至元十三年升平江路。戶四十六萬六千一百五十八,口二百四十三萬三千七百。領司一、縣二、州四。



 錄事司。



 縣二



 吳縣,上。長洲。上。與吳縣並為倚郭。



 州四



 昆山州,中。唐以來為縣,元元貞元年升州。



 常熟州,中。唐以來為縣,元元貞元年升州。



 吳江州,中。唐以來為縣,元元貞元年升州。



 嘉定州,中。本昆山縣地,宋置縣,元元貞元年升州。



 常州路,上。唐初為常州,又改晉陵郡,又復為常州,宋因之。元至元十四年升為路。戶二十萬九千七百三十二,口一百二萬一十一。領司一、縣二、州二。



 錄事司。



 縣二



 晉陵,中。倚郭。武進。中。倚郭。



 州二



 宜興州,中。唐義興縣。宋改義為宜。元至元十五年,升宜興府。二十年,仍為縣。二十一年,復升為府,仍置宜興縣以隸之。元貞元年,府縣俱廢,止立宜興州。



 無錫州,中。唐無錫縣。元元貞元年升州。



 鎮江路,下。唐潤州,又改丹陽郡,又為鎮海軍。宋為鎮江府。元至元十三年,升為鎮江路。戶一十萬三千三百一十五,口六十二萬三千六百四十四。領司一、縣三。



 錄事司。



 縣三



 丹徒,中。倚郭。丹陽,中。金壇。中。



 建德路,上。唐睦州,又為嚴州,又改新定郡。宋為建德軍,又為遂安軍,後升建德府。元至元十三年,改建德府安撫司。十四年,改建德路。戶一十萬三千四百八十一,口五十萬四千二百六十四。領司一、縣六。



 錄事司。



 縣六



 建德,中。倚郭。淳安,中。遂安,下。桐廬,中。分水,中。壽昌。中。



 松江府,唐為蘇州屬邑。宋為秀州屬邑。元至元十四年,升為華亭府。十五年,改松江府,仍置華亭縣以隸之。戶一十六萬三千九百三十一。至順錢糧數。領縣二:



 華亭,上。倚郭。上海。上。本華亭縣地,至元二十七年,以戶口繁多,置上海縣,屬松江府。



 江陰州,上。唐初為暨州,後為江陰縣,隸常州。宋為軍。元至元十二年,依舊置軍,行安撫司事。十四年,升為江陰路總管府,今降為江陰州。戶五萬三千八百二十一,口三十萬一百七十七。



 浙東道宣慰司都元帥府元治婺州,大德六年移治慶元。



 慶元路,上。唐為鄞州,又為明州,又為餘姚郡。宋升慶元府。元至元十三年,改置宣慰司。十四年,改為慶元路總管府。戶二十四萬一千四百五十七,口五十一萬一千一百一十三。領司一、縣四、州二。



 錄事司。



 縣四



 鄞縣,上。倚郭。象山,中。慈溪,中。定海。中。



 州二



 奉化州,下。唐析鄮縣地置奉化縣,隸明州。元元貞元年,升為奉化州,隸慶元。



 昌國州,下。宋置昌國縣。元至元十四年,升為州,仍置昌國縣以隸之。後止立昌國州,隸慶元。



 衢州路,上。本太末地,唐析婺州之西境置衢州,又改信安郡,又改為衢州。元至元十三年,改衢州路總管府。戶一十萬八千五百六十七,口五十四萬三千六百六十。領司一、縣五。



 錄事司。



 縣五



 西安,中。倚郭。龍游,上。江山,下。常山,下。宋改信安,今復舊名。開化。中。



 浙東海右道肅政廉訪司



 婺州路,上。唐初為婺州,又改東陽郡。宋為保寧軍。元至元十三年,改婺州路。戶二十二萬一千一百一十八,口一百七萬七千五百四十。領司一、縣六、州一。



 錄事司。



 縣六



 金華,上。倚郭。東陽,上。義烏,上。永康,中。武義,中。浦江。中。



 州一



 蘭溪州,下。本金華之西部三河戍,唐析置蘭溪縣,宋因之。元元貞元年,升州。



 紹興路,上。唐初為越州,又改會稽郡,又仍為越州。宋為紹興府。元至元十三年,改紹興路。戶一十五萬一千二百三十四,口五十二萬一千五百八十八。領司一、縣六、州二。



 錄事司。



 縣六



 山陰,上。會稽,中。與山陰俱倚郭。有會稽山為南鎮。上虞,上。



 蕭山,中。嵊縣,上。新昌。中。



 州二



 餘姚州,下。唐餘姚縣,宋因之。元元貞元年升州。



 諸暨州,下。宋諸暨縣。元元貞元年升州。



 溫州路,上。唐初為東嘉州,又改永嘉郡,又為溫州。宋升瑞安府。元至元十三年,置溫州路。戶一十八萬七千四百三,口四十九萬七千八百四十八。領司一、縣二、州二。



 錄事司。



 縣二



 永嘉,上。倚郭。樂清。下。



 州二



 瑞安州,下。唐瑞安縣,宋因之。元元貞元年升州。



 平陽州,下。唐平陽縣,宋因之。元元貞元年升州。



 臺州路,上。唐初為海州,復改臺州,又改臨海郡,又為德化軍,宋因之。元至元十三年,置安撫司。十四年,改臺州路總管府。戶一十九萬六千四百一十五,口一百萬三千八百三十三。領司一、縣四、州一。



 錄事司。



 縣四



 臨海,上。倚郭。仙居,上。寧海,上。天臺。中。



 州一



 黃巖州,下。唐為縣,宋因之。元元貞元年升州。



 處州路,上。唐初為括州,又改縉雲郡,又為處州,宋因之。元至元十三年,立處州路總管府。戶一十三萬二千七百五十四,口四十九萬三千六百九十二。領司一、縣七。



 錄事司。



 縣七



 麗水,中。倚郭。龍泉,中。松陽,中。遂昌,中。青田,中。縉雲,中。慶元。中。



 江東建康道肅政廉訪司



 寧國路,上。唐為宣州,又為宣城郡,又升寧國軍。宋升寧國府。元至元十四年,升寧國路總管府。戶二十三萬二千五百三十八,口一百一十六萬二千六百九十。領司一、縣六。



 錄事司。舊立四廂,元至元十四年,廢四廂創立。



 縣六



 宣城,上。倚郭。南陵,中。涇縣,中。寧國,中。旌德,中。太平。中。



 徽州路,上。唐歙州。宋改徽州。元至元十四年,升徽州路。戶一十五萬七千四百七十一,口八十二萬四千三百四。領司一、縣五、州一。



 錄事司。舊設四廂,至元十四年改置。



 縣五



 歙縣,上。倚郭。休寧,中。祈門,中。黟縣,下。績溪。中。



 州一



 婺源州,下。本休寧縣之回玉鄉,唐析之置婺源縣。元元貞元年升州。



 饒州路,上。唐改鄱陽郡,仍改饒州,宋因之。元至元十四年,升饒州路總管府。戶六十八萬二百三十五,口四百三萬六千五百七十。領司一、縣三、州三。



 錄事司。舊設三廂,至元十四年改立。



 縣三



 鄱陽,上。倚郭。德興,上。安仁。中。



 州三



 餘干州,中。唐以來為縣,元元貞元年升州。



 浮梁州,中。唐以來為縣,元元貞元年升州。



 樂平州,中。唐以來為縣,元元貞元年升州。



 江南諸道行御史臺



 集慶路,上。唐武德初,置揚州東南道行臺尚書省。後復為蔣州,罷行臺,移揚州江都,改金陵曰白下,以其地隸潤州。貞觀中,更白下曰江寧。至德中,置江寧郡。乾元中,改升州。其後楊氏有其地,改為金陵府。南唐李氏又改為江寧府。宋平南唐,復為升州。仁宗以升王建國,升建康軍。高宗改建康府,建行都,又為沿江制置司治所。元至元十二年歸附。十四年,升建康路。初立行御史臺於揚州,既而徙杭州,又徙江州,又還杭州;二十三年,自杭州徙治建康。天歷二年,以文宗潛邸,改建康路為集慶路。戶二十一萬四千五百四十八,口一百七萬二千六百九十。領司一、縣三、州二。



 錄事司。



 縣三



 上元,中。倚郭。江寧,中。倚郭。句容。中。



 州二



 溧水州,中。唐以來皆為縣,元元貞元年升州。



 溧陽州,中。唐以來並為縣,元至元十六年,升為溧陽路。二十七年,復降為縣,後復升為州。



 太平路,下。唐置南豫州。宋為太平州。至元十四年,升為太平路。戶七萬六千二百二,口四十四萬六千三百七十一。領司一、縣三。



 錄事司。舊設四廂,至元十四年改立。



 縣三



 當塗,中。倚郭。蕪湖,中。繁昌。下。



 池州路,下。唐於秋浦縣置池州,後廢,以縣隸宣州,未幾復置。宋仍為池州。元至元十四年,升為路。戶六萬八千五百四十七,口三十六萬六千五百六十七。領司一、縣六。



 錄事司。



 縣六



 貴池,下。倚郭。即秋浦縣,吳改為貴池。青陽,下。建德,下。銅陵,下。石埭,中。東流。下。



 信州路,上。唐乾元以前,為衢、饒、撫、建四州之地。乾元元年,始割衢之玉山、常山,饒之弋陽及撫、建二州之地置信州。宋因之。元至元十四年,升為路。戶一十三萬二千二百九十,口六十六萬二千二百五十八。領司一、縣五。



 錄事司。



 縣五



 上饒,上。倚郭。玉山,中。弋陽,中。貴溪,中。永豐。中。



 廣德路,下。唐初,以綏安縣置桃州,後廢州,改綏安為廣德縣。宋為廣德軍。元至元十四年,升為路。戶五萬六千五百一十三,口三十三萬九千七百八十。領司一、縣二。



 錄事司。



 縣二



 廣德,中。倚郭。建平。中。



 鉛山州,中。本建、撫二州之地,山產銅鉛。後唐析上饒、弋陽五鄉為銅場,繼升為縣,屬信州。宋因之。元至元二十九年,割上饒之乾元、永樂二鄉,弋陽之新政、善政二鄉來屬,升為鉛山州,直隸行省。戶二萬六千三十五。至順錢糧數。



 福建道宣慰使司都元帥府大德元年立。



 福建閩海道肅政廉訪司



 福州路,上。唐為閩州,後改福州,又為長樂郡,又為威武軍。宋為福建路。元至元十五年,為福州路。十八年,遷泉州行省於本州。十九年,復還泉州。二十年,仍遷本州。二十二年,並入杭州。戶七十九萬九千六百九十四,口三百八十七萬五千一百二十七。領司一、縣九、州二。州領二縣。



 錄事司。至元十五年,行中書省於在城十二廂分四隅,置錄事司。十六年,並其二,置東西二司。二十年,復並為一。



 縣九



 閩縣,中。倚郭。侯官,中。倚郭。懷安,中。古田,上。閩清,中。長樂,中。連江,中。羅源,中。永福。中。



 州二



 福清州,下。唐析長樂八鄉置萬安縣,又改福唐,又改福清。元元貞元年升為州。



 福寧州,上。唐長溪縣,元升為福寧州。領二縣:



 寧德,中。福安。中。



 建寧路,下。唐初為建州,又改建安郡。宋升建寧軍。元至元二十六年,升為路。戶一十二萬七千二百五十四,口五十萬六千九百二十六。領司一、縣七。



 錄事司。



 縣七



 建安,中。甌寧,中。與建安俱倚郭。浦城,中。建陽,中。崇安,中。松溪,下。政和。下。



 泉州路,上。唐置武榮州,又改泉州。宋為平海軍。元至元十四年,立行宣慰司,兼行征南元帥府事。十五年,改宣慰司為行中書省,升泉州路總管府。十八年,遷行省於福州路。十九年,復還泉州。二十年,仍遷福州路。戶八萬九千六十,口四十五萬五千五百四十五。領司一、縣七。



 錄事司。至元十五年,立南北二司。十六年,並為一。



 縣七



 晉江,中。倚郭。南安,中。惠安,下。同安,下。永春,下。安溪,下。德化。下。



 興化路,下。宋置太平軍,又改興化軍,先治興化,後遷莆田。元至元十四年,升興化路。戶六萬七千七百三十九,口三十五萬二千五百三十四。領司一、縣三。



 錄事司。



 縣三



 莆田,中。宋置興化軍,遷治莆田。元至元十三年,割左右二廂屬錄事司,縣如故。仙游,下。興化。下。軍治元在此,後移於莆田,此縣為屬邑。



 邵武路,下。唐邵武縣,屬建州。宋置邵武軍。元至元十三年,為邵武路。戶六萬四千一百二十七,口二十四萬八千七百六十一。領司一、縣四。



 錄事司。



 縣四



 邵武,中。倚郭。光澤,中。泰寧,中。建寧。中。



 延平路,下。五代為延平鎮,王延政始以鎮為鐔州。南唐置劍州。宋以利州路亦有劍州,乃稱此為南劍州。元至元十五年,升南劍路,後改延平路。戶八萬九千八百二十五,口四十三萬五千八百六十九。領司一、縣五。



 錄事司。



 縣五



 南平,中。倚郭。尤溪,中。沙縣,中。順昌,中。將樂。中。



 汀州路,下。唐開福、撫二州山洞置州,治新羅,後改臨汀郡,又仍為汀州。宋隸福建路。元至元十五年,升為汀州路。戶四萬一千四百二十三,口二十三萬八千一百二十七。領司一、縣六。本路屯田二百二十五頃。



 錄事司。



 縣六



 長汀,中。倚郭。寧化,中。清流,下。蓮城,下。上杭,下。武平。下。



 漳州路,下。唐析閩州西南境置,後改漳浦郡,又復為漳州。宋因之。元至元十六年,升漳州路。戶二萬一千六百九十五,口一十萬一千三百六。領司一、縣五。本路屯田二百五十頃。



 錄事司。



 縣五



 龍溪,下。倚郭。漳浦,下。龍巖,下。長泰,下。南靖。下。本南勝,改今名。



 江西等處行中書省,為路一十八、州九,屬州十三,屬縣七十八。本省馬站八十五處,水站六十九處。



 江西湖東道肅政廉訪司



 龍興路,上。唐初為洪州,又為豫章郡,又仍為洪州。宋升隆興府。元至元十二年,設行都元帥府及安撫司,仍領南昌、新建、豐城、進賢、奉新、靖安、分寧、武寧八縣,置錄事司。十四年,改元帥府為江西道宣慰司,本路為總管府,立行中書省。十五年,立江西湖東道提刑按察司,移省於贛州。十六年,復還隆興。十七年,並入福建行省,止立宣慰司。十九年復立,罷宣慰司,隸皇太子位。二十一年,改隆興府為龍興。二十三年,豐城縣升富州,武寧縣置寧州,領武寧、分寧二縣。大德五年,以分寧縣置寧州,武寧縣隸龍興路。戶三十七萬一千四百三十六,口一百四十八萬五千七百四十四。至元二十七年抄籍數。領司一、縣六、州二。



 錄事司。宋以南昌、新建二縣分置九廂。元至元十三年,廢城內六廂,置錄事司。



 縣六



 南昌,上。倚郭。至元二十年,割錄事司所領城外二廂、東南兩關來屬。新建,上。倚郭。進賢,中。奉新,中。靖安,中。武寧。中。至元二十三年,置寧州,縣為倚郭。大德五年,於分寧縣置寧州,武寧直隸本路。



 州二



 富州,上。本富城縣,又曰豐城。唐自豐水之西徙治章水東,即今治所。宋屬隆興府。元至元十九年,隸皇太子位。二十三年,升為富州。



 寧州,中。唐分寧縣。宋因之。元至元二十三年,於武寧縣置寧州,武寧為倚郭縣。大德八年,割武寧直隸本路,遂徙州治於分寧。



 吉安路,上。唐為吉州,又為廬陵郡。宋升為上州。元至元十四年,升吉州路總管府,置錄事司,領一司、八縣。元貞元年,吉水、安福、太和、永新四縣升州,改吉州為吉安路。戶四十四萬四千八十三,口二百二十二萬四百一十五。領司一、縣五、州四。大德二年,吉、贛立屯田。



 錄事司。



 縣五



 廬陵,上。倚郭。永豐,上。萬安,中。龍泉,中。永寧。下。至順間,分永新州立。



 州四



 吉水州,中。舊為縣。元元貞元年升州。



 安福州,中。唐初以縣置潁州,後廢,復為縣。元元貞元年升州。



 太和州,下。唐初置南平州,後廢為縣。元元貞元年升州。



 永新州,下。唐為縣。元元貞元年升州。



 瑞州路,上。唐改建成縣曰高安,即其地置靖州,又改筠州。宋為高安郡,又改瑞州。元至元十四年,升瑞州路,領一司、三縣。元貞元年,升新昌縣為州。戶一十四萬四千五百七十二,口七十二萬二千三百二。領司一、縣二、州一。



 錄事司。至元十四年始立。



 縣二



 高安,上。倚郭。上高。中。



 州一



 新昌州,下。唐為建成縣,屬靖州,後省入高安。宋割高安、上高二縣地,升鹽步鎮為新昌縣。元元貞元年升州。



 袁州路,上。唐為袁州,又為宜春郡。元至元十三年,置安撫司。十四年,改總管府,領四縣,設錄事司,隸湖南行省。十九年,升路,隸江西行省。元貞元年,萍鄉縣升州。戶一十九萬八千五百六十三,口九十九萬二千八百一十五。領司一、縣三、州一。



 錄事司。至元十三年,設兵馬司。十四年,改錄事司。



 縣三



 宜春,上。倚郭。分宜,上。萬載。中。



 州一



 萍鄉州,中。本為縣。元貞元年升州。



 臨江路,上。唐改建成為高安,而蕭灘鎮實高安境內。南唐升鎮為清江縣,屬洪州,後又屬筠州。宋即清江縣置臨江軍,隸江南西道。元至元十三年,隸江西行都元帥府。十四年,改臨江路總管府。元貞元年,新淦、新喻二縣升州。戶一十五萬八千三百四十八,口七十九萬一千七百四十。領司一、縣一、州二。



 錄事司。宋隸都監司。元至元十三年,設兵馬司。十五年,改錄事司。



 縣一



 清江。上。宋即縣治置臨江軍。元至元十四年,升軍為路,而縣為倚郭。



 州二



 新淦州,中。唐以來為縣。元元貞元年升州。



 新喻州,中。唐以來為縣。元元貞元年升州。



 撫州路,上。唐初為撫州,又為臨川郡,又仍為撫州。元至元十二年,復為撫州。十四年,升撫州路總管府。戶二十一萬八千四百五十五,口一百九萬二千二百七十五。領司一、縣五。



 錄事司。至元十四年,廢宋三廂立。



 縣五



 臨川,上。崇仁,上。金溪,上。宜黃,中。樂安。中。



 江州路,下。唐初為江州,又改潯陽郡,又仍為江州。宋為定江軍。元至元十二年,置江東西宣撫司。十三年,改為江西大都督府,隸揚州行省。十四年,罷都督府,升江州路,隸龍興行都元帥府,後置行中書省,江州直隸焉。十六年,隸黃蘄等路宣慰司。二十二年,復隸行省。戶八萬三千九百七十七,口五十萬三千八百五十二。領司一、縣五。



 錄事司。宋隸都監司。元至元十二年,設兵馬司。十四年,置錄事司。



 縣五



 德化,中。唐潯陽縣。瑞昌,中。彭澤,中。湖口,中。德安。中。



 南康路,下。唐屬江州。宋置南康軍,治星子縣。元至元十四年,升南康路,隸江淮行省。二十二年,割屬江西,領一司、三縣。戶九萬五千六百七十八,口四十七萬八千三百九十。領司一、縣二、州一。



 錄事司。



 縣二



 星子,下。南康治所。都昌。上。



 州一



 建昌州,下。唐初置南昌州,後廢,屬洪州。宋屬南康軍。元元貞元年升州。



 贛州路,上。唐初為虔州,又為南康郡,又仍為虔州。宋改贛州。元至元十四年,升贛州路總管府。十五年,設錄事司,領一司、十縣,隸江西省。二十四年,並龍南入信豐,安遠入會昌。大德元年,寧都、會昌二縣升州,割瑞金隸會昌。至大三年,復置龍南、安遠二縣,屬寧都。戶七萬一千二百八十七,口二十八萬五千一百四十八。領司一、縣五、州二。州領三縣。本路屯田五百一十餘頃。



 錄事司。



 縣五



 贛縣,上。州治所。興國,中。信豐,下。雩都,下。石城。下。



 州二



 寧都州,下。唐為縣。元大德元年,升寧都州。領二縣:



 龍南,下。至元二十四年,並入信豐縣。至大三年復置。安遠。下。至元二十四年,省入會昌縣。至大三年復置。



 會昌州,下。本雩都地。唐屬虔州。宋升縣之九州鎮為會昌縣,復升為軍。元大德元年,升會昌州。領一縣:



 瑞金。下。舊屬虔州,大德元年來屬。



 建昌路,下。本南城縣,屬撫州。南唐升建武軍。宋升建昌軍。元至元十四年,改建昌路總管府,割南城置錄事司。十九年,南豐縣升州,直隸行省。戶九萬二千二百二十三,口五十五萬三千三百三十八。領司一、縣三。



 錄事司。至元十四年立。



 縣三



 南城,中。新城,中。廣昌。中。



 南安路,下。唐升大庾鎮為縣,屬虔州。宋以縣置南安軍。元至元十四年,改南安路總管府。十五年,割大庾縣在城四坊,設錄事司。十六年,廢錄事司。戶五萬六百一十一,口三十萬三千六百六十六。領縣三:



 大庾,中。倚郭。南康,中。上猶。下。南唐為上猶。宋改南安。至元十六年,改永清,後復為上猶。



 南豐州,下。唐為南豐縣,隸撫州。宋改隸建昌軍。元至元十九年,升為州,直隸行省。戶二萬五千七十八,口一十二萬八千九百。



 廣東道宣慰使司都元帥府



 海北廣東道肅政廉訪司



 廣州路,上。唐以廣州為嶺南五府節度五管經略使治所,又改南海郡,又仍為廣州。宋升為帥府。元至元十三年內附,後又叛。十五年克之,立廣東道宣慰司,立總管府並錄事司。元領八縣,而懷集一縣割屬賀州。戶一十七萬二百一十六,口一百二萬一千二百九十六。領司一、縣七。



 錄事司。至元十六年立,以州之東城、西城、子城並番禺、南海二縣在城民戶隸之。



 縣七



 南海,中。番禺,下。與南海俱倚郭。東筦,中。增城,中。香山,下。新會,下。清遠。下。



 韶州路,下。唐初為番州,又更名東衡州,又改韶州,又為始興郡,又仍為韶州。元至元十三年內附,未幾廣人叛,十五年始定,立總管府,設錄事司。戶一萬九千五百八十四,口一十七萬六千二百五十六。領司一、縣四。



 錄事司。



 縣四



 曲江,中。元初分縣城西廂地及城外三廂,屬錄事司。樂昌,下。仁化,下。乳源。下。



 惠州路,下。唐循州。宋改惠州,又改博羅郡,又復為惠州。元至元十六年,改惠州路總管府。戶一萬九千八百三,口九萬九千一十五。領縣四:



 歸善,下。倚郭。博羅,下。海豐,下。河源。下。



 南雄路,下。本始興縣。唐初屬韶州。五代劉氏割韶之湞昌、始興二縣置雄州。宋以河北有雄州,改為南雄州。元至元十五年,改南雄路總管府。戶一萬七百九十二,口五萬三千九百六十。領縣二:



 保昌,下。本湞昌,宋改今名。始興。下。



 潮州路,下。唐初為潮州,又改潮陽郡,又復為潮州。元至元十五年歸附。十六年,改為總管府,以孟招討鎮守,未幾移鎮漳州,土豪各據其地。二十一年,廣東道宣慰使月的迷失以兵來招諭。二十三年,復為江西等處行樞密院副使兼廣東道宣慰使以鎮之,始定。戶六萬三千六百五十,口四十四萬五千五百五十。領司一、縣三。



 錄事司。至元二十二年始立。



 縣三



 海陽,下。倚郭。潮陽,下。揭陽。下。



 德慶路,下。唐初為南康州,又名康州,又改晉康郡。宋升德慶府。元至元十三年,徇廣東,既取廣州,而德慶未下。十四年,廣西宣慰司以兵取之,改隸廣西道。十七年,立德慶路總管府,後仍屬廣東道。戶一萬二千七百五,口三萬二千九百九十七。領縣二:



 端溪,下。瀧水。下。



 肇慶路,下。唐初為端州,又改高要郡,又仍為端州。宋升肇慶府。元至元十三年,徇廣東,惟肇慶未附。十六年,廣南西道宣慰司定之,因隸廣西。十七年,改為下路總管府,仍屬廣東。戶三萬三千三百三十八,口五萬五千四百二十九。領縣二:



 高要,中。倚郭。四會。中。



 英德州,下。唐洭州。五代南漢為英州。宋升英德府。元至元十三年歸附。十五年,立英德路總管府。二十三年,降為散州。大德四年,復為路。本州素為寇盜淵藪。大德四年,達魯花赤脫歡察兒此歲招降群盜至二千餘戶,遂升英德為路,命脫歡察兒為達魯花赤兼萬戶以鎮之。至大元年,復降為州。領縣一:



 翁源。大德五年置。



 梅州,下。唐為程鄉縣,屬潮州。五代南漢置敬州。宋改梅州。元至元十三年歸附。十六年,置總管府。二十三年,改為散州。戶二千四百七十八,口一萬四千八百六十五。領縣一:



 程鄉。



 南恩州,下。唐恩州,又為齊安郡。宋改南恩州。元至元十三年置南恩路總管府,十九年降為散州。戶一萬九千三百七十三,口九萬六千八百六十五。領縣二:



 陽江,下。陽春。下。



 封州,下。唐改為臨封郡,後復為封州。元至元十三年歸附。明年,廣人叛,廣西宣慰司以兵定之,遂隸西道。十六年,立封州路總管府,後又降為散州,仍屬東道。戶二千七十七,口一萬七百四十二。領縣二:



 封川,下。開建。下。



 新州,下。唐改為新昌郡,後復為新州。元至元十六年,置新州路總管府。十九年,降為散州。戶一萬一千三百一十六,口六萬七千八百九十六。領縣一:



 新興。下。



 桂陽州,下。本桂陽縣,唐、宋因之。元至元十三年內附。十九年,升桂陽縣為散州,割連州陽山縣來屬,為蒙古鷿忽都虎郡王分地,元隸湖南道宣慰司,後隸廣東道。戶六千三百五十六,口二萬五千六百五十五。領縣一:



 陽山。下。唐屬連州,宋因之。至元十九年割以來屬。



 連州,下。唐改連山郡,復改連州。元至元十三年,置安撫司,直隸行中書省。十七年,廢安撫司,升為連州路總管府,隸湖南道宣慰司。十九年,降為散州,隸廣東道。戶四千一百五十四,口七千一百四十一。領縣一:



 連山,下。



 循州,下。唐改為海豐郡,仍改循州。宋為博羅郡。元至元十三年,立總管府。二十三年,降為散州。戶一千六百五十八,口八千二百九十。領縣三:



 龍川,下。興寧,下。長樂。下。



\end{pinyinscope}