\article{本紀第三十一 明宗}

\begin{pinyinscope}

 明宗翼獻景孝皇帝,諱和世束,武宗長子也。母曰仁獻章聖皇后,亦乞烈氏。成宗大德三年,命武宗撫軍北邊,帝以四年十一月壬子生。成宗崩,十一年,武宗入繼大統,立仁宗為皇太子,命以次傳於帝。武宗崩,仁宗立,延祐三年春,議建東宮,時丞相鐵木迭而欲固位取寵,乃議立英宗為皇太子,又與太后幸臣識烈門譖帝於兩宮,浸潤久之,其計遂行。於是封帝為周王,出鎮雲南。置常侍府官屬,以遙授中書左丞相禿忽魯、大司徒斡耳朵、中政使尚家奴、山北遼陽等路蒙古軍萬戶孛羅、翰林侍講學士教化等並為常侍,中衛親軍都指揮使唐兀、兵部尚書賽罕八都魯為中尉,仍置諮議、記室各二員,遣就鎮。是年冬十一月,帝次延安,禿忽魯、尚家奴、孛羅及武宗舊臣厘日、沙不丁、哈八兒禿等皆來會。教化謀曰:「天下者,我武皇之天下也,出鎮之事,本非上意,由左右構間致然。請以其故白行省,俾聞之朝廷,庶可杜塞離間,不然,事變叵測。」遂與數騎馳去。先是,阿思罕為太師,鐵木迭兒奪其位,出之為陜西行省丞相,及教化等至,即與平章政事塔察兒、行臺御史大夫脫里伯、中丞脫歡,悉發關中兵,分道自潼關、河中府入。已而塔察兒、脫歡襲殺阿思罕、教化於河中,帝遂西行,至北邊金山。西北諸王察阿臺等聞帝至,咸率眾來附。帝至其部,與定約束,每歲冬居扎顏,夏居斡羅斡察山,春則命從者耕於野泥,十餘年間,邊境寧謐。



 延祐七年,仁宗崩,英宗嗣立。是歲夏四月丙寅,子妥歡帖木爾生,是為至正帝。至治三年八月癸亥,御史大夫鐵失等弒英宗,晉王也孫鐵木兒自立為皇帝,改元泰定。五月,遣使扈從皇后八不沙至自京師。二年,帝弟圖帖睦爾以懷王出居於建康。三年三月癸酉,子懿璘質班生,是為寧宗。



 歲戊辰七月庚午,泰定皇帝崩於上都,倒剌沙專權自用,逾月不立君,朝野疑懼。時僉樞密院事燕鐵木兒留守京師,遂謀舉義。八月甲午黎明,召百官集興聖宮,兵皆露刃,號於眾曰:「武皇有聖子二人,孝友仁文,天下歸心,大統所在,當迎立之,不從者死!」乃縛平章烏伯都剌、伯顏察兒,以中書左丞朵朵、參知政事王士熙等下於獄。燕鐵木兒與西安王阿剌忒納失裏固守內廷。於是帝方遠在沙漠,猝未能至,慮生他變,乃迎帝弟懷王於江陵,且宣言已遣使北迎帝,以安眾心。復矯稱帝所遣使者自北方來,云周王從諸王兵整駕南轅,旦夕即至矣。丁巳,懷王入京師,群臣請正大統,固讓曰:「大兄在北,以長以德,當有天下。必不得已,當明以朕志播告中外。」九月壬申,懷王即位,是為文宗,改元天歷,詔天下曰:「謹俟大兄之至,以遂朕固讓之心。」時倒剌沙在上都,立泰定皇帝子為皇帝,乃遣兵分道犯大都,而梁王王禪、右丞相答失鐵木兒、御史大夫紐澤、太尉不花等兵皆次於榆林,燕帖木兒與其弟撒敦、子唐其勢等,帥師與戰,屢敗之。上都兵皆潰。十月辛丑,齊王月魯帖木兒、元帥不花帖木兒以兵圍上都,倒剌沙乃奉皇帝寶出降,兩京道路始通。於是文宗遣哈散及撒迪等相繼來迎,朔漠諸王皆勸帝南還京師,遂發北邊。諸王察阿臺、沿邊元帥朵烈捏、萬戶買驢等,咸帥師扈行,舊臣孛羅、尚家奴、哈八兒禿皆從。至金山,嶺北行省平章政事潑皮奉迎,武寧王徹徹禿、僉樞密院事帖木兒不花繼至。乃命孛羅如京師,兩京之民聞帝使者至,歡呼鼓舞曰:「吾天子實自北來矣!」諸王、舊臣爭先迎謁,所至成聚。



 天歷二年正月乙丑,文宗復遣中書左丞躍裏帖木兒來迎。乙酉,撒迪等至,入見帝於行幄,以文宗命勸進。丙戌,帝即位於和寧之北,扈行諸王、大臣咸入賀,乃命撒迪遣人還報京師。是月,前翰林學士承旨不答失里以太府太監沙剌班輦金銀幣帛至。遣撒迪等還京師,帝命之曰:「朕弟曩嘗覽觀書史,邇者得無廢乎?聽政之暇,宜親賢士大夫,講論史籍,以知古今治亂得失。卿等至京師,當以朕意諭之。」



 二月壬辰,宣靖王買奴自京師來覲。辛丑,追尊皇妣亦乞烈氏曰仁獻章聖皇后。是月,文宗立奎章閣學士院於京師,遣人以除目來奏,帝並從之。



 三月戊午朔,次潔堅察罕之地。辛酉,文宗遣右丞相燕鐵木兒奉皇帝寶來上,御史中丞八即剌、知樞密院事禿兒哈帖木兒等,各率其屬以從。壬戌,造乘輿服御及近侍諸服用。丙寅,帝謂中書左丞躍裏帖木兒曰:「朕至上都,宗籓諸王必皆來會,非尋常朝會比也,諸王察阿臺今亦從朕遠來,有司供張,皆宜豫備。卿其與中書臣僚議之。」丁亥,雨土,霾。四月癸巳,燕鐵木兒見帝於行在,率百官上皇帝寶,帝嘉其勛,拜太師,仍命為中書右丞相,開府儀同三司、上柱國、錄軍國重事、監修國史、答剌罕、太平王並如故。復諭燕鐵木兒等曰:「凡京師百官,朕弟所用者,並仍其舊,卿等其以朕意諭之。」燕鐵木兒奏:「陛下君臨萬方,國家大事所系者,中書省、樞密院、御史臺而已,宜擇人居之。」帝然其言,以武宗舊人哈八兒禿為中書平章政事,前中書平章政事伯帖木兒知樞密院事,常侍孛羅為御史大夫。甲午,立行樞密院,命昭武王、知樞密院事火沙領行樞密院事,賽帖木兒、買奴並同知行樞密院事。是日,帝宴諸王、大臣於行殿,燕鐵木兒、哈八兒禿、伯帖木兒、孛羅等侍。帝特命臺臣曰;「太祖皇帝嘗訓敕臣下云:『美色、名馬,人皆悅之,然方寸一有系累,即能壞名敗德。』卿等居風紀之司,亦嘗念及此乎?世祖初立御史臺,首命塔察兒、奔帖傑兒二人協司其政。天下國家,譬猶一人之身,中書則右手也,樞密則左手也。左右手有病,治之以良醫,省、院闕失,不以御史臺治之可乎?凡諸王、百司,違法越禮,一聽舉劾。風紀重則貪墨懼,猶斧斤重則入木深,其勢然也。朕有闕失,卿亦以聞,朕不爾責也。」乙未,特命孛羅等傳旨,宣諭燕鐵木兒、伯答沙、火沙、哈八兒禿、八即剌等曰:「世祖皇帝立中書省、樞密院、御史臺及百司庶府,共治天下,大小職掌,已有定制。世祖命廷臣集律令章程,以為萬世法。成宗以來,列聖相承,罔不恪遵成憲。朕今居太祖、世祖所居之位,凡省、院、臺、百司庶政,詢謀僉同,摽譯所奏,以告於朕。軍務機密,樞密院當即以聞,毋以夙夜為間而稽留之。其他事務,果有所言,必先中書、院、臺,其下百司及紘御之臣,毋得隔越陳請。宜宣諭諸司,咸俾聞知。儻違朕意,必罰無赦。」丁酉,以陜西行臺御史大夫鐵木兒脫為上都留守。辛丑,文宗立都督府於京師,遣使來奏,又以臺憲官除目來上,並從之。癸卯,遣使如京師,卜日命中書左丞相鐵木兒補化攝告即位於郊廟、社稷。遣武寧王徹徹禿及哈八兒禿立文宗為皇太子,仍立詹事院,罷儲慶司,以徹里鐵木兒為中書平章政事,闊兒吉司為中書右丞,怯來、只兒哈郎並為甘肅行省平章政事,忽剌臺為江浙行省平章政事,那海為嶺北行省平章政事。甲辰,敕中書省賜官吏送寶者秩一等,從者賚以幣帛。乙巳,監察御史言:「嶺北行省,控制一方,廣輪萬里,實為太祖肇基之地,國家根本系焉。方面之寄,豈可輕任。平章塔即吉素非勛舊,奴事倒剌沙,倔起宿衛,輒為右丞,俄升平章,年已七十,眊昏殊甚。左丞馬謀,本晉邸部民,以女妻倒剌沙,引為都水,遂除左丞。郎中羅里,市井小人,禿魯忽乃晉邸衛卒,不諳政務。並宜黜退。」臺臣以聞,帝曰:「御史言甚善,其並黜之。」又諭臺臣曰:「御史劾嶺北省臣,朕甚嘉之。繼今所當言者,毋有所憚。被劾之人,茍營求申訴,朕必罪之。或廉非其實,毋輒以聞。」五月丁巳朔,次朵里伯真之地。戊午,遣西安王阿剌忒納失裡還京師,封帖木兒為保德郡王。賜扈駕宿衛士等幣帛有差。己未,皇太子遣翰林學士承旨阿鄰帖木兒來覲。庚申,次斡耳罕木東。辛酉,御史大夫孛羅、中政使尚家奴,並特授開府儀同三司,以典四番宿衛。癸亥,次必忒怯禿之地,翰林學士承旨斡耳朵自京師來覲。命有司新武宗幄殿、車輿。庚午,命燕鐵木兒升用嶺北行省官吏,其餘官吏並賜散官一級。選用潛邸舊臣及扈從士,受制命者八十有五人,六品以下二十有六人。壬申,次探禿兒海之地。封亦憐真八為柳城郡王,以八即剌為陜西行臺御史大夫,眾家奴為御史中丞。乙亥,次禿忽剌。敕大都省臣鑄皇太子寶。時求太子故寶不知所在,近侍伯不花言寶藏於上都行幄,遣人至上都索之,無所得,乃命更鑄之。西木鄰等四十三驛旱災,命中書以糧賑之,計八千二百石。丁丑,皇太子發京師。鎮南王帖木兒不花,諸王也速、斡即、答來不花、朵來只班、伯顏也不干,駙馬別闍里及扈衛百官,悉從行。戊寅,京師市馬二百八十匹,載乘輿服御送行在所。己卯,次禿忽剌河東。加翰林學士承旨唐兀為太尉。趙王馬札罕部落旱,民五萬五千四百口不能自存,敕河東宣慰司賑糧兩月。庚辰,賜諸王燕只哥臺鈔二百錠、幣帛二千匹。辛巳,次斡羅斡禿之地。壬午,次不魯通之地。是日,左丞相鐵木兒補化等以帝即位,攝告南郊。甲申,次忽剌火失溫之地。六月丁亥朔,次坤都也不剌之地。是日,鐵木兒補化等以帝即位,攝告於宗廟、社稷。戊子,燕鐵木兒等奏:「中政院越中書擅奏除授,移文來征制敕,已如所請授之,然於大體非宜,乞申命禁止,庶使政權歸一。」從之。庚寅,次撒里之地。陜西行省告饑,遣使還都,與諸老臣議賑救之。丁酉,次兀納八之地。升都督府為大都督府。己亥,次闊朵之地。樞密院奏:「皇太子遣使來言,近已頒敕,四川諸省兵悉遣還營,惟雲南逆謀叵測,兵未可即罷,令臣等以聞。」帝曰:「可仍屯戍,俟平定而後罷。」辛丑,次撒里怯兒之地。壬寅,戒近侍毋得輒有奏請。甲辰,賜駙馬脫必兒鈔千錠,往雲南。丁未,次哈里溫。戊申,次闊朵傑阿剌倫。辛亥,次哈兒哈納禿之地。詔諭中書省臣:「凡國家錢穀、銓選諸大政事,先啟皇太子,然後以聞。」癸丑,次忽禿之地。甲寅,賑陜西臨潼、華陰二十三驛鈔一千八百錠,晉寧路十五驛鈔八百錠。是月,鐵木兒補化以久旱啟於皇太子,辭相位,乞更選賢德,委以燮理,皇太子遣使以聞。帝諭闊兒吉思等曰:「修德應天,乃君臣當為之事,鐵木兒補化所言良是。天明可畏,朕未嘗斯須忘於懷也。皇太子來會,當與共圖其可以澤民利物者行之。卿等其以朕意諭群臣。」七月丙辰朔,日有食之。甲子,次孛羅火你之地。壬申,監察御史把的於思言:「朝廷自去秋命將出師,戡定禍亂,其供給軍需,賞賚將士,所費不可勝紀。若以歲入經賦較之,則其所出已過數倍。況今諸王朝會,舊制一切供億,俱尚未給,而陜西等處饑饉薦臻,餓殍枕籍,加以冬春之交,雪雨愆期,麥苗槁死,秋田未種,民庶遑遑,流移者眾。臣伏思之,此正國家節用之時也。如果有功必當賞賚者,宜視其官之崇卑而輕重之,不惟省費,亦可示勸。其近侍諸臣奏請恩賜,宜悉停罷,以紓民力。」臺臣以聞,帝嘉納之,仍敕中書省以其所言示百司。乙亥,次不羅察罕之地。丙子,文宗受皇太子寶。戊寅,次小只之地。壬午,遣使詣京師,敕中書平章政事哈八兒禿同翰林國史院官祭太祖、太宗、睿宗三朝御容。發諸衛軍六千完京城。八月乙酉朔,次王忽察都之地。丙戌,皇太子入見。是日,宴皇太子及諸王、大臣於行殿。庚寅,帝暴崩,年三十,葬起輦穀,從諸陵。是月己亥,皇太子復即皇帝位。十二月乙巳,知樞密院事臣也不倫等議請上尊謚曰翼獻景孝皇帝,廟號明宗。三年三月壬申,祔於太廟。



\end{pinyinscope}