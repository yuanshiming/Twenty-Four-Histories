\article{志第四十五上 食貨四}

\begin{pinyinscope}

 ○俸秩



 官必有祿,所以養廉也。元初未置祿秩,世祖既位之初,首命給之。內而朝臣百司,外而路府州縣,微而府史胥徒,莫不有祿。大德中,以外有司有職田,於是無職田者,復益之以俸米。其所以養官吏者,不亦厚乎!



 祿秩之制,凡朝廷職官,中統元年定之;六部官,二年定之;隨路州縣官,是年十月定之。至元六年,又分上中下縣,為三等。提刑按察司官吏,六年定之。自經歷以下,七年復增之。轉運司官及諸匠官,七年定之。其運司依民官例,於差發內支給。至十七年,定奪俸祿,凡內外官吏皆住支。十八年,更命公事畢而無罪者給之,公事未畢而有罪者逐之。二十二年,重定百官俸,始於各品分上中下三例,視職事為差,事大者依上例,事小者依中例。二十三年,又命內外官吏俸以十分為率,添支五分。二十九年,定各處儒學教授俸,與蒙古、醫學同。



 成宗大德三年,詔益小吏俸米。六年,又定各處行省、宣慰司、致用院、宣撫司、茶鹽運司、鐵冶都提舉司、淘金總管府、銀場提舉司等官循行俸例。七年,始加給內外官吏俸米。凡俸一十兩以下人員,依小吏例,每一兩給米一斗。十兩以上至二十五兩,每員給米一石。餘上之數,每俸一兩給米一升。無米,則驗其時直給價,雖貴每石不過二十兩。上都、大同、隆興、甘肅等處,素非產米之地,每石權給中統鈔二十五兩,俸三錠以上者不給。至大二年,詔隨朝官員及軍官等俸改給至元鈔,而罷其俸米。延祐七年,又命隨朝官吏俸以十分為率,給米三分。



 凡諸官員上任者不過初二日,罷任者已過初五日,給當月俸。各路官擅割官吏俸者罪之。諸職官病假百日之外,及因病求醫、親老告侍者,不給祿。後官已至,而前官被差者,其俸兩給之。隨朝官吏每月給俸,如告假事故,當官立限者全給,違限托故者追罰。軍官差出者許借俸,歿於王事者借俸免征。各投下保充路府州縣等官,其俸與王官等。



 職田之制,路府州縣官至元三年定之,按察司官十四年定之,江南行省及諸司官二十一年定之,其數減腹里之半。至武宗至大二年,外官有職田者,三品給祿米一百石,四品給六十石,五品五十石,六品四十五石,七品以下四十石;俸鈔改支至元鈔,其田拘收入官。四年,又詔公田及俸皆復舊制。延祐三年,外官無職田者,量給粟麥。凡交代官芒種已前去任者,其租後官收之,已後去任者前官分收。後又以爭競者多,俾各驗其俸月以為多寡。



 其大略如此。今取其制之可考者,具列於後。



 至元二十二年百官俸例,各品分上中下三等:



 從一品六錠五錠正二品四錠二十五兩四錠一十五兩從二品四錠三錠三十五兩三錠二十五兩正三



 品三錠二十五兩三錠一十五兩三錠從三品



 三錠



 二錠三十五兩二錠二十五兩正四品二錠二十五兩



 二錠一十五兩二錠從四品二錠一錠四十五兩



 一錠四十兩正五品一錠四十兩一錠三十兩從五品一錠三十兩一錠二十兩正六品一錠二十兩一錠一十五兩從六品一錠一十五兩一錠一十兩正七品一錠一十兩一錠五兩從七品一錠五



 兩一錠正八品一錠



 四十五兩從八品四十五兩四十兩正九品四十兩



 三十五兩從九品三十五兩內外官俸數:



 太師府:太師,俸一百四十貫,米一十五石。諮議、參軍,俸四十五貫,米四石五斗。長史,俸三十四貫六錢六分,米三石。太傅、太保府同。監修國史、參軍、長史同。



 中書省:右丞相,俸一百四十貫,米一十五石;左丞相同。平章政事,俸一百二十八貫六錢六分六厘,米一十二石。右丞,俸一百一十八貫六錢六分六厘,米一十二石;左丞同。參知政事,俸九十五貫三錢三分三厘,米九石五斗。參議,俸五十九貫,米六石。郎中,俸四十二貫,米四石五斗。員外郎,俸三十四貫六錢六分六厘,米三石。都事,俸二十八貫,米三石。承發管勾,俸二十五貫三錢三分三厘,米二石;照磨、省架閣庫管勾、回回架閣庫管勾並同。檢校官,俸二十八貫,米三石五斗。斷事官,內一十八員俸各八十二貫六錢六分六厘,米八石五斗;一十四員俸各五十九貫三錢三分三厘,米六石;一員俸五十四貫六錢六分六厘,米五石五斗;一員俸四十貫六錢六分六厘,米四石。經歷,俸二十三貫六錢六分六厘,米二石五斗。知事,俸二十二貫,米二石。客省使,俸三十九貫三錢三分三厘,米三石五斗;副使,俸二十八貫,米三石。直省舍人,俸三十四貫六錢六分六厘,米三石。六部尚書,俸七十八貫,米八石。侍郎,俸五十三貫三錢三分三厘,米五石。郎中,俸三十四貫六錢六分六厘,米三石。員外郎,俸二十八貫,米三石。主事,俸二十六貫六錢六分六厘,米二石五斗。戶部司計,俸二十八貫,米三石。工部司程,俸一十八貫,米二石五斗。刑部獄丞,俸一十一貫,米一石。司籍提領,俸一十二貫六錢六分六厘,米一石。同提領,俸一十一貫三錢三分三厘,米五斗。



 樞密院:知院,俸一百二十九貫三錢三分三厘,米一十三石五斗。同知,俸一百六貫,米一十一石。副樞,俸九十五貫三錢三分三厘,米九石五斗。僉院,俸九十貫一錢八分六厘,米九石五斗。同僉,俸五十九貫三錢三分三厘,米六石。院判,俸四十二貫,米四石五斗。參議,俸三十九貫三錢三分三厘,米三石五斗。經歷,俸三十四貫六錢六分六厘,米三石。都事,俸二十八貫,米二石。照磨,俸二十二貫,米二石;管勾同。斷事官,俸五十九貫三錢三分三厘,米六石。經歷,俸二十五貫三錢三分三厘,米二石。知事,俸二十貫六錢六分六厘,米一石五斗。客省使,俸三十一貫三錢三分三厘,米三石;副使俸二十二貫,米二石。右衛都指揮使,俸七十貫,米七石五斗。副都指揮使,俸五十九貫三錢三分三厘,米六石。僉事,俸四十八貫六錢六分六厘,米四石五斗。經歷,俸二十五貫三錢三分三厘,米二石。知事,俸二十貫六錢六分六厘,米一石五斗。照磨,俸一十八貫六錢六分六厘,米一石五斗。鎮撫,俸二十貫六錢六分六厘,米一石五斗。行軍官:千戶,俸二十五貫三錢三分三厘,米二石。副千戶,俸二十貫六錢六分六厘,米一石五斗。百戶,俸一十七貫三錢三分三厘,米一石五斗。彈壓,俸一十二貫六錢六分六厘,米一石。知事,俸一十一貫三錢三分三厘,米一石。弩軍官:千戶,俸二十貫六錢六分六厘,米一石五斗。百戶,俸一十二貫六錢六分六厘,米一石。彈壓,俸一十一貫三錢三分三厘,米五斗。都目,俸一十貫,米五斗。屯田千戶所同弩軍官例。左衛、前衛、後衛、中衛、武衛、左阿速衛、右阿速衛、左都威衛、右都威衛、左欽察衛、右欽察衛、左衛率府、宗仁衛、西域司、唐兀司、貴赤司並同右衛例。忠翊侍衛都指揮使,俸一百貫。副使,俸八十三貫三錢三分三厘。僉事,俸六十六貫六錢六分六厘。經歷,俸三十三貫三錢三分三厘。知事,俸二十六貫六錢六分六厘。照磨,俸二十四貫六錢六分六厘。行軍官:千戶,俸三十三貫三錢三分三厘。副千戶,俸二十六貫六錢六分六厘。百戶,俸二十三貫三錢三分三厘。彈壓,俸一十六貫六錢六分六厘。知事,俸一十五貫三錢三分三厘。弩軍官:千戶,俸二十六貫六錢六分六厘。百戶,俸一十六貫六錢六分六厘。彈壓,俸一十三貫三錢三分三厘。右手屯田千戶所:千戶,俸二十六貫六錢六分六厘。百戶,俸一十六貫六錢六分六厘。左手屯田千戶所同。隆鎮衛、右翊蒙古侍衛並同忠翊侍衛例。



 御史臺:御史大夫,俸一百一十八貫六錢六分,米一十二石。中丞,俸一百六貫,米一十一石。侍御史,俸九十六貫三錢五分,米九石五斗。治書侍御史,俸九十貫一錢八分,米九石五斗。經歷,俸三十四貫六錢六分,米三石。都事,俸二十八貫,米三石。殿中,俸四十八貫六錢六分,米四石五斗。知班,俸一十四貫,米一石五斗。監察御史,俸二十八貫,米三石。



 奎章閣學士院:大學士,俸一百一貫三錢三分三厘,米一十石五斗。侍書學士,俸九十五貫三錢三分三厘,米九石五斗。承制學士,俸七十八貫,米八石。供奉學士,俸五十九貫三錢三分三厘,米六石。參書,俸三十四貫三錢三分三厘,米三石。典簽,俸二十八貫,米三石。鑒書博士,俸四十一貫,米四石五斗。授經郎,俸二十八貫,米三石。



 太禧宗禋院:院使,俸一百一十八貫六錢六分六厘,米一十二石。同知,俸一百貫,米一十石。副使,俸九十五貫三錢三分三厘,米九石五斗。僉院,俸九十貫一錢八分,米九石。同僉,俸五十九貫三錢三分三厘,米六石。院判,俸四十二貫,米四石五斗。參議,俸三十九貫三錢三分三厘,米三石五斗。經歷,俸三十四貫六錢六分六厘,米三石。都事,俸二十八貫,米三石。照磨,俸二十二貫,米二石;管勾同。斷事官,俸五十九貫三錢三分,米六石。經歷,俸二十五貫三錢三分,米二石。知事,俸二十貫六錢六分,米一石五斗。客省使,俸三十一貫三錢三分,米三石。副使,俸二十二貫,米二石。



 宣政院:院使,俸一百一十八貫六錢六分,米一十二石。同知,俸一百六貫,米一十一石。副使,俸九十五貫三錢三分,米九石五斗。僉院,俸九十貫一錢八分,米九石五斗。同僉,俸五十九貫三錢三分,米六石。院判,俸四十二貫,米四石五斗。參議,俸三十九貫三錢三分,米三石五斗。經歷,俸三十四貫六錢六分,米三石五斗。都事,俸二十八貫,米三石。照磨,俸二十二貫,米二石;管勾同。斷事官、客省使並同太禧宗禋院例。宣徽院同。



 翰林國史院:承旨,俸一百一十八貫六錢六分,米一十二石。學士,俸一百六貫,米一十一石。侍讀學士,俸九十五貫三錢三分,米九石五斗;侍講學士同。直學士,俸五十九貫三錢三分三厘,米六石。經歷,俸三十四貫六錢六分六厘,米三石。都事,俸二十八貫,米三石。待制,俸三十九貫三錢三分三厘,米三石五斗。修撰,俸二十八貫,米三石。應奉,俸二十五貫三錢三分三厘,米二石。編修,俸二十二貫,米二石;檢閱同。典籍,俸二十貫六錢六分六厘,米一石五斗。翰林院、集賢院,大學士同承旨,餘並同上例。



 中政院:院使,俸一百一貫三錢三分三厘,米一十石五斗。同知,俸八十二貫六錢六分六厘,米八石五斗。僉院,俸七十貫,米七石五斗。同僉,俸五十九貫三錢三分三厘,米六石。院判,俸四十三貫,米四石五斗。司議,俸三十四貫六錢六分六厘,米三石。長史,俸二十八貫,米三石。照磨,俸二十二貫,米二石;管勾同。太醫院、典瑞院、將作院、太史院、儲政院並同。



 太常禮儀院:院使,俸八十二貫六錢六分,米八石五斗。同知,俸七十二貫,米七石五斗。僉院,俸四十八貫六錢六分六厘,米四石五斗。同僉,俸四十二貫,米四石五斗。院判,俸三十七貫三錢三分三厘,米四石。經歷,俸二十八貫,米三石。都事,俸二十五貫三錢三分,米二石。照磨,俸二十二貫,米二石。太祝,俸二十貫六錢六分,米一石五斗;奉禮、協律同。



 通政院:院使,俸八十二貫六錢六分六厘,米八石五斗。同知,俸七十貫,米七石五斗。副使,俸五十九貫三錢三分三厘,米六石。僉院,俸四十八貫六錢六分六厘,米四石五斗。同僉,俸四十四貫,米四石五斗。院判,俸三十九貫三錢三分三厘,米三石五斗。經歷,俸三十四貫六錢六分六厘,米三石。都事,俸二十六貫六錢六分六厘,米二石五斗。照磨,俸二十二貫,米二石。



 大宗正府:也可扎魯忽赤,內一員俸一百一十八貫六錢六分六厘,米一十二石;二十七員俸八十二貫六錢六分六厘,米八石;五員俸六十七貫三錢三分三厘,米六石五斗。郎中,俸三十六貫,米三石五斗。員外郎,俸三十一貫三錢三分三厘,米三石。都事,俸二十六貫六錢六分六厘,米二石五斗。照磨,俸二十二貫,米二石;管勾同。



 大司農司:大司農,俸一百一十八貫六錢六分,米一十二石。大司農卿,俸一百三貫,米一十一石。大司農少卿,俸九十五貫三錢三分,米九石五斗。大司農丞,俸九十貫一錢八分,米九石五斗。經歷,俸三十四貫六錢六分,米三石。都事,俸二十八貫,米三石。照磨,俸二十二貫,米二石;管勾同。



 內史府:內史,俸一百四十三貫三錢三分。中尉,俸一百一十六貫六錢六分六厘。司馬,俸八十三貫三錢三分三厘。諮議,俸四十六貫六錢六分六厘。記室,俸四十貫。照磨,俸三十貫。



 大都留守司:留守,俸一百一貫三錢三分,米一十石五斗。同知,俸八十二貫六錢六分,米八石五斗。副留守,俸五十九貫三錢三分三厘,米六石。留判,俸四十二貫,米四石五斗。經歷,俸三十四貫六錢六分六厘,米三石。都事,俸二十八貫,米三石。照磨,俸二十二貫,米二石。



 都護府:大都護,俸八十二貫六錢六分六厘,米八石五斗。同知,俸七十二貫,米七石五斗。副都護,俸五十九貫三錢三分三厘,米六石。經歷,俸二十八貫,米三石。都事,俸二十六貫六錢六分六厘,米二石五斗。照磨,俸二十二貫,米二石。



 崇福司:司使,俸八十二貫六錢六分六厘,米八石。同知,俸七十貫,米七石五斗。副使,俸五十九貫三錢三分,米六石。司丞,俸三十九貫三錢三分,米三石五斗。經歷,俸二十八貫,米三石。都事,俸二十六貫六分六厘,米二石五斗。照磨,俸二十二貫,米二石。



 給事中,俸五十三貫三錢三分三厘,米五石。左右侍儀奉御,俸四十八貫六錢六分六厘,米四石五斗。



 武備寺:卿,俸七十貫,米七石五斗。同判,俸五十九貫三錢三分三厘,米六石。少卿,俸四十二貫,米四石五斗。寺丞,俸三十九貫三錢三分三厘,米三石五斗。經歷,俸二十五貫三錢三分三厘,米二石。知事,俸二十四貫,米二石。照磨,俸二十二貫,米二石。



 太僕寺:卿,俸七十貫,米七石五斗。少卿,俸四十二貫,米四石五斗。寺丞,俸三十九貫三錢三分,米三石五斗。經歷,俸二十五貫三錢三分三厘,米二石。知事,俸二十二貫,米二石。照磨,俸二十貫六錢六分,米一石五斗。光祿、長慶、長新、長秋、承徽、長寧、尚乘、長信等寺並同。



 尚舍寺:太監,俸四十八貫六錢六分,米四石。少監,俸三十九貫三錢三分,米三石五斗。監丞,俸三十一貫三錢三分,米二石。知事,俸二十二貫,米二石。



 侍儀司:侍儀使,俸七十貫,米七石五斗。引進使,俸四十八貫六錢六分,米四石五斗。典簿,俸二十五貫三錢三分,米二石。承奉班都知,俸二十六貫六錢六分,米二石五斗。通事舍人,俸二十五貫三錢三分,米二石。侍儀舍人,俸一十七貫三錢三分,米一石五斗。



 拱衛司:都指揮使,俸七十貫,米七石五斗。副都指揮使,俸五十九貫三錢三分三厘,米六石。僉事,俸四十八貫六錢六分六厘,米四石五斗。經歷,俸二十五貫三錢三分三厘,米二石。知事,俸二十貫六錢六分六厘,米一石五斗。



 內宰司:內宰,俸七十貫,米七石五斗。司丞,俸四十五貫,米四石五斗。典簿,俸二十五貫三錢三分,米二石。照磨,俸二十貫六錢六分,米一石五斗。翊正司同。



 延慶司:延慶使,俸一百貫。同知,俸六十三貫三錢三分三厘。副使,俸四十六貫六錢六分六厘。司丞,俸三十四貫六錢六分六厘,米三石。典簿,俸二十五貫三錢三分三厘,米二石。照磨,俸二十貫六錢六分六厘,米一石五斗。



 內正司:司卿,俸七十貫,米七石五斗。少卿,俸四十七貫,米四石五斗。司丞,俸三十九貫三錢三分三厘,米三石五斗。典簿,俸二十五貫三錢三分三厘,米二石。照磨,俸二十貫六錢六分,米一石五斗。中瑞司同。



 京畿運司:運使,俸五十六貫,米六石。同知,俸三十九貫三錢三分,米三石五斗。運副,俸三十四貫六錢六分,米三石。運判,俸二十六貫六錢六分,米二石五斗。經歷,俸二十貫六錢六分,米一石五斗。知事,俸一十四貫,米一石五斗。提控案牘,俸一十四貫六錢六分,米一石。



 太府監:卿,俸七十貫,米七石五斗。太監,俸五十九貫三錢三分,米六石。少監,俸四十二貫,米四石五斗。監丞,俸三十九貫三錢三分,米三石五斗。經歷,俸二十五貫三錢三分,米二石。知事,俸二十四貫,米二石。照磨,俸二十二貫,米二石。秘書、章佩、利用、中尚、度支等監並同。



 國子監:祭酒,俸五十九貫三錢三分,米六石。司業,俸三十九貫三錢三分,米三石五斗。監丞,俸三十貫三錢三分,米三石。典簿,俸一十五貫三錢三分,米二石。博士,俸二十六貫六錢六分,米二石五斗;太常博士、回回國子博士同。助教,俸二十二貫,米二石;教授同。學錄,俸一十一貫三錢三分,米五斗。蒙古國子監同。



 經正監:卿,俸七十貫,米七石五斗。太監,俸五十貫,米五石。少監,俸四十二貫,米四石五斗。監丞,俸三十四貫六錢六分六厘,米三石。經歷,俸二十五貫三錢三分三厘,米二石。知事,俸二十二貫,米二石。



 闌遺監:太監,俸四十八貫六錢六分,米四石。少監,俸三十九貫三錢三分三厘,米三石。監丞,俸三十一貫三錢三分,米三石。知事,俸二十二貫,米二石。提控案牘,俸二十貫六錢六分,米一石五斗。



 司天監:提點,俸五十九貫三錢三分,米六石。司天監,俸五十三貫三錢三分,米五石。監丞,俸三十一貫三錢三分,米三石。知事,俸二十貫六錢六分六厘,米一石五斗。教授,俸一十貫六錢六分,米一石;管勾同。司辰,俸八貫六錢六分,米五斗;學正、押宿並同。回回司天監:少監,俸四十二貫,米四石五斗;餘同上。



 都水監:都水卿,俸五十三貫,米六石。少監,俸三十九貫三錢三分,米三石五斗。監丞,俸三十貫,米三石。經歷,俸二十五貫三錢三分,米二石。知事,俸二十二貫,米二石。



 大都路達魯花赤,俸一百三十貫;總管同。副達魯花赤,一百二十貫。同知八十貫;治中同。判官,五十五貫。推官,五十貫。經歷,四十貫。知事,三十貫。提控案牘,二十五貫;照磨同。並中統鈔。



 行省:左丞相,俸二百貫。平章政事,一百六十六貫六錢六分六厘;右丞、左丞同。參知政事,一百三十三貫三錢三分三厘。郎中,四十六貫六錢六分六厘。員外郎,三十貫。都事,二十六貫六錢六分六厘;檢校同。管勾,二十三貫三錢三分三厘。理問所:理問,俸四十六貫六錢六分六厘。副理問,俸三十貫。知事,俸一十六貫六錢六分六厘;提控案牘同。



 宣慰司:腹裏宣慰使,俸中統鈔五百八十貫三錢三分。同知,五百貫。副使,四百一十六貫六錢六分。經歷,四百貫。都事,一百八十三貫三錢三分。照磨,一百五十貫。行省宣慰使,俸至元鈔八十七貫五錢。同知,四十九貫。副使,四十二貫。經歷,二十八貫。都事,二十四貫。照磨,一十七貫五錢。



 廉訪司:廉訪使,俸中統鈔八十貫。副使,四十五貫。僉事,三十貫。經歷,二十貫。知事,一十五貫。照磨,一十二貫。



 鹽運司:腹裏運使,俸一百二十貫。同知,五十貫。副使,三十五貫。判官,三十貫。經歷,二十貫。知事,一十五貫。照磨,一十三貫。行省運使,八十貫。同知,五十貫。運副,四十貫。判官,三十貫。經歷,二十五貫。知事,一十七貫。提控案牘,一十五貫。



 上路達魯花赤,俸八十貫;總管同。同知,四十貫。治中,三十貫。判官,二十貫。推官,一十九貫。經歷,一十七貫。知事,一十二貫。提控案牘,一十貫。下路達魯花赤,俸七十貫;總管同。同知,三十五貫。判官,二十貫。推官,一十九貫。經歷,一十七貫。知事,一十二貫。提控案牘,一十貫。



 散府達魯花赤,俸六十貫;知府同。同知,三十貫。判官,一十八貫;推官同。知事,一十二貫。提控案牘,一十貫。



 上州達魯花赤,俸五十貫;州尹同。同知,二十五貫。判官,一十八貫。知事,一十二貫。提控案牘,一十貫。中州達魯花赤,俸四十貫;知州同。同知,二十貫。判官,一十五貫。提控案牘,一十貫。都目,八貫。下州達魯花赤,俸三十貫;知州同。同知,一十八貫。判官,一十三貫。吏目,四十貫。



 上縣達魯花赤,俸二十貫;縣尹同。縣丞,一十五貫。主簿,一十三貫。縣尉,一十二貫。典史,三十五貫。巡檢,一十貫。中縣達魯花赤,俸一十八貫;縣尹同。主簿,一十三貫。縣尉,一十二貫。典史,三十五貫。下縣達魯花赤,俸一十七貫;縣尹同。主簿,一十二貫;縣尉同。典史,三十五貫。



 諸署、諸局、諸庫等官及掾吏之屬,其目甚多,不可勝書。然其俸數之多寡,亦皆以品級之高下為則。觀者可以類推,故略而不錄。



 職田數:



 至元三年,定隨路府州縣官員職田:上路達魯花赤一十六頃,總管同,同知八頃,治中六頃,府判五頃。下路達魯花赤一十四頃,總管同,同知七頃,府判五頃。散府達魯花赤一十二頃,知府同,同知六頃,府判四頃。上州達魯花赤一十頃,州尹同,同知五頃,州判四頃。中州達魯花赤八頃,知州同,同知四頃,州判三頃。下州達魯花赤六頃,知州同,州判三頃。警巡院達魯花赤五頃,警使同,警副四頃,警判三頃。錄事司達魯花赤三頃,錄事同,錄判二頃。縣達魯花赤四頃,縣尹同,縣丞三頃,主簿二頃,縣尉、主簿兼尉並同,經歷四頃。



 至元十四年,定按察司職田:各道按察使一十六頃,副使八頃,僉事六頃。



 至元二十一年,定江南行省及諸司職田比腹裏減半。上路達魯花赤八頃,總管同,同知四頃,治中三頃,府判二頃五十畝。下路達魯花赤七頃,總管同,同知三頃五十畝,府判二頃五十畝,經歷二頃,知事一頃,提控案牘同。散府達魯花赤六頃,知府同,同知三頃,府判二頃,提控案牘一頃。上州達魯花赤五頃,知州同,同知二頃,州判同,提控案牘一頃。中州達魯花赤四頃,知州同。同知二頃,州判一頃五十畝,都目五十畝。下州達魯花赤三頃,知州同,同知二頃,州判一頃五十畝。上縣達魯花赤二頃,縣尹同,縣丞一頃五十畝,主簿一頃,縣尉同。中縣同上。無縣丞。下縣達魯花赤一頃五十畝,縣尹同,主簿兼尉一頃。錄事司達魯花赤一頃五十畝,錄事同,錄判一頃。司獄一頃,巡檢同。



 按察司使八頃,副使四頃,僉事三頃,經歷二頃,知事一頃。運司官:運使八頃,同知四頃,運副三頃,運判同,經歷二頃,知事二頃,提控案牘同。鹽司官:鹽使二頃,鹽副二頃,鹽判一頃,各場正、同、管勾各一頃。



 常平義倉



 常平起於漢之耿壽昌,義倉起於唐之戴胄,皆救荒之良法也。元立義倉於鄉社,又置常平於路府,使饑不損民,豐不傷農,粟直不低昂,而民無菜色,可謂善法漢、唐者矣。



 今考其制,常平倉世祖至元六年始立。其法:豐年米賤,官為增價糴之;歉年米貴,官為減價糶之。於是八年以和糴糧及諸河倉所撥糧貯焉。二十三年定鐵法,又以鐵課糴糧充焉。義倉亦至元六年始立。其法:社置一倉,以社長主之,豐年每親丁納粟五斗,驅丁二斗,無粟聽納雜色,歉年就給社民。於是二十一年新城縣水,二十九年東平等處饑,皆發義倉賑之。皇慶二年,復申其令。然行之既久,名存而實廢,豈非有司之過與?



 惠民藥局



 《周官》有醫師,掌醫之政令,凡邦有疾病NY者造焉,則使醫分而治之,此民所以無夭折之患也。元立惠民藥局,官給鈔本,月營子錢,以備藥物,仍擇良醫主之,以療貧民,其深得《周官》設醫師之美意者與。



 初,太宗九年,如於燕京等十路置局,以奉御田闊闊、太醫王璧、齊楫等為局官,給銀五百錠為規運之本。世祖中統二年,又命王祐開局。四年,復置局於上都,每中統鈔一百兩,收息錢一兩五錢。至元二十五年,以陷失官本,悉罷革之。至成宗大德三年,又準舊例,於各路置焉。凡局皆以各路正官提調,所設良醫,上路二名,下路府州各一名,其所給鈔本,亦驗民戶多寡以為等差。今並著於後:



 腹裏,三千七百八十錠。



 河南行省,二百七十錠。



 湖廣行省,一千一百五十錠。



 遼陽行省,二百四十錠。



 四川行省,二百四十錠。



 陜西行省,二百四十錠。



 江西行省,三百錠。



 江浙行省,二千六百一十五錠。



 雲南行省,真一萬一千五百索。



 甘肅行省,一百錠。



 市糴



 和糴自唐始,所以備邊庭軍需也,其弊至於害民者,蓋有之矣。元和糴之名有二,曰市糴糧,曰鹽折草,率皆增其直而市於民。於是邊庭之兵不乏食,京師之馬不乏芻,而民亦用以不困,其為法不亦善乎!



 市糴糧之法,世祖中統二年,始以鈔一千二百錠,於上都、北京、西京等處糴三萬石。四年,以解鹽引一萬五千道,和中陜西軍儲。是年三月,又命扎馬剌丁糴糧,仍敕軍民官毋沮。五年,諭北京、西京等路市糴軍糧。至元三年,以南京等處和糴四十萬石。四年,命沔州等處中納官糧,續還其直。八年,驗各路糧粟價直,增十分之一,和糴三十九萬四千六百六十石。十六年,以兩淮鹽引五萬道,募客旅中糧。十九年,以鈔三萬錠,市糴於隆興等處。二十年,以鈔五千錠市於北京,六萬錠市於上都,二千錠市於應昌。二十一年,以河間、山東、兩浙、兩淮鹽引,募諸人中糧。是年四月,以鈔四千錠,於應昌市糴。九月,發鹽引七萬道、鈔三萬錠,於上都和糴。二十二年,以鈔五萬錠,令木八剌沙和糴於上都。是年二月,詔江南民田秋成,官為定例收糴,次年減價出糶。二十三年,發鈔五千錠,市糴沙、凈、隆興軍糧。二十四年,官發鹽引,聽民中糧。是年十二月,以揚州、杭州鹽引五十萬道,兌換民糧。二十七年,和糴西京糧,其價每一十兩之上增一兩。延祐三年,中糴和林糧二十三萬石。五年、六年,又各和中二十萬石。



 鹽折草之法,成宗大德八年,定其則例。每年以河間鹽,令有司於五月預給京畿郡縣之民,至秋成,各驗鹽數輸草,以給京師秣馬之用。每鹽二斤,折草一束,重一十斤。歲用草八百萬束,折鹽四萬引云。



 賑恤



 救荒之政,莫大於賑恤。元賑恤之名有二:曰蠲免者,免其差稅,即《周官·大司徒》所謂薄征者也;曰賑貸者,給以米粟,即《周官·大司徒》所謂散利者也。然蠲免有以恩免者,有以災免者。賑貸有以鰥寡孤獨而賑者,有以水旱疫癘而賑者,有以京師人物繁湊而每歲賑糶者。若夫納粟補官之令,亦救荒之一策也。其為制各不同,今並著於後,以見其仁厚愛民之意云。



 恩免之制:世祖中統元年,量減絲料、包銀分數。二年,免西京、北京、燕京差發。是年二月,以真定、大名、河南、陜西、東平、益都、平陽等路,兵興之際,勞於轉輸,其差發減輕科取。三年,北京等路以兵興供給繁重,免本歲絲料、包銀。是年閏九月,以濟南路遭李鋋之亂,軍民皆饑,盡除差發。四年,以西涼民戶值渾都海、阿藍鷿兒之亂,人民流散,免差稅三年。至元元年,詔減明年包銀十分之三,全無業者十之七。是年四月,逃戶復業者,免差稅三年。三年,減中都包銀四分之一。十二年,蠲免包銀、絲線、俸鈔。是年八月,免河南路包銀三分之二,其餘路府亦免十之五。十九年,免諸路民戶明年包銀、俸鈔,及逃移戶差稅。二十年,免大都、平灤民戶絲線、俸鈔。二十二年,除民間包銀三年,不使帶納俸鈔,盡免大都軍民地稅。二十四年,免東京軍民絲線、包銀、俸鈔。是年九月,除北京馬五百匹。二十五年,免遼陽、武平等處差發。二十七年,減河間、保定、平灤三路絲線之半,大都全免。二十八年,詔免腹裏諸路包銀、俸鈔;其大都、上都、隆興、平灤、大同、太原、河間、保定、武平、遼陽十路絲線並除之。二十九年,免上都、隆興、平灤、保定、河間五路包銀、俸鈔。三十年,免大都差稅。三十一年,成宗即位,詔免天下差稅有差。是年六月,免腹里軍、站、匠、船、鹽、鐵等戶稅糧,及江南夏稅之半。元貞元年,除大都民戶絲線、包銀、稅糧。大德元年,以改元免大都、上都、隆興民戶差稅三年。三年,詔免腹裏包銀、俸鈔,及江南夏稅十分之三。四年,詔免上都、大都、隆興明年絲銀稅糧,其數亦如之,江南租稅減十分之一。九年,又下寬免之令,以恤大都、上都、隆興、腹裏、江淮之民。十年,逃移民戶復業者,免差稅三年。十一年,武宗即位,詔免內外郡縣差稅有差。至大二年,上尊號,詔免腹裏、江淮差稅。三年,又免大都、上都、中都秋稅,及民間差稅之負欠者。四年,免腹裏包銀及江南夏稅十分之三。是年四月,免大都、上都、中都差稅三年。延祐元年,以改元免大都、上都差稅二年,其餘被災經賑者免一年,流民復業者免差稅三年。二年,免各路差稅、絲料。七年,免腹裏絲綿十分之五,外郡十分之三,江淮夏稅所免之數,與外郡絲綿同,民間逋欠差稅並除之。是年,免丁地稅糧、包銀、絲料各有差。至治二年,寬恤軍民站戶。三年,免臨清萬戶府軍民船戶差稅三年,福建蜑戶差稅一年。泰定三年,罷江淮以南包銀。天歷元年,免諸路差稅、絲料有差,及海北鹽課三年。二年,免達達軍站之貧乏者及各路差稅有差。是年十月,免人民逋欠官錢,及奉元商稅,各處灶戶雜役。至順元年,以改元免諸路差稅有差,減方物之貢,免河南府、懷慶路門攤、海北鹽課,存恤紅城兒屯田軍三年。



 災免之制:世祖中統元年,以各處被災,驗實減免科差。三年,以蠻寇攻掠,免三叉沽灶戶一百六十五戶其年絲料、包銀。四年,以秋旱霜災,減大名等路稅糧。至元三年,以東平等處蠶災,減其絲料。五年,以益都等路禾損,蠲其差稅。六年,以濟南、益都、懷孟、德州、淄萊、博州、曹州、真定、順德、河間、濟州、東平、恩州、南京等處桑蠶災傷,量免絲料。七年,南京、河南蝗旱,減差徭十分之六。十九年,減京師民戶科差之半。二十年,以水旱相仍,免江南稅糧十分之二。二十四年,免北京饑民差稅。是年,揚州及浙西水,其地稅在揚州者全免,浙西減二分。二十五年,南安等處被寇兵者,稅糧免征。二十六年,紹興路水,免地稅十之三。是年六月,以禾稼不收,免遼陽差稅。二十七年,大都、遼陽被災,免其包銀、俸鈔。是年六月,以霖雨免河間等路絲料之半。十月,以興、松二州霜,免其地稅。二十八年,遼陽被災者,稅糧皆免征,其餘量徵其半。是年五月,以太原去歲不登,杭州被水,其太原丁地稅糧、杭州地稅並除之。九月,又免州路所負歲糧。二十九年,以北京地震,量減歲課。是年,以大都去歲不登,流移者眾,免其稅糧及包銀、俸鈔。元貞元年,以供給繁重及水傷禾稼,免咸平府邊民差稅。大德三年,以旱蝗,除揚州、淮安兩路稅糧。五年,各路被災重者,其差稅並除之。六年,免大都、平灤差稅。七年,以內郡饑,荊湖、川蜀供給軍餉,其差稅減免各有差。八年,以平陽、太原地震,免差稅三年。至大元年,以江南、江北水旱民饑,其科差、夏稅並免之。二年,以腹裏、江淮被災,其科差、夏稅亦並免之。皇慶二年,免益都饑民貸糧。延祐二年,河南、歸德、南陽、徐、邳、陳、蔡、許州、荊門、襄陽等處水,三年,肅州等處連歲被災,皆免其民戶稅糧。天歷元年,陜西霜旱,免其科差一年;鹽官州海潮,免其秋糧夏稅。是年十二月,詔經寇盜剽掠州縣,免差稅一年。二年,以關陜旱,免差稅三年。至順元年,以河南、懷慶旱,其門攤課程及逋欠差稅皆免征。



 鰥寡孤獨賑貸之制:世祖中統元年,首詔天下,鰥寡孤獨廢疾不能自存之人,天民之無告者也,命所在官司,以糧贍之。至元元年,又詔病者給藥,貧者給糧。八年,令各路設濟眾院以居處之,於糧之外,復給以薪。十年,以官吏破除入己,凡糧薪並敕於公給散。十九年,各路立養濟院一所,仍委憲司點治。二十年,給京師南城孤老衣糧房舍。二十八年,給寡婦冬夏衣。二十九年,給貧子柴薪,日五斤。三十一年,特賜米絹。元貞二年,詔各處孤老,凡遇寬恩,人給布帛各一。大德三年,詔遇天壽節,人給中統鈔二貫,永為定例。六年,給死者棺木錢。



 水旱疫癘賑貸之制:中統元年,平陽旱,遣使賑之。二年,遷曳捏即地貧民就食河南、平陽、太原。三年,濟南饑,以糧三萬石賑之。是年七月,以課銀一百五十錠濟甘州貧民。四年,以錢糧幣帛賑東平濟河貧民,鈔四千錠賑諸王只必帖木兒部貧民。至元二年,以鈔百錠賑闊闊出所部軍。五年,益都民饑,驗口賑之。六年,東平、河間一十五處饑,亦驗口賑之。八年,以糧賑西京路急遞鋪兵卒。十二年,濮州等處饑,貸糧五千石。十六年,以江南所運糯米不堪用者賑貧民。十九年,真定饑,賑糧兩月。二十年,以帛千匹、鈔三百錠,賑水達達地貧民。二十三年,大都屬郡六處饑,賑糧三月。二十四年,斡端民饑,賑鈔萬錠。是年四月,以陳米給貧民。七月,以糧給諸王阿只吉部貧民,大口二斗,小口一斗。二十六年,京兆旱,以糧三萬石賑之。是年,又賑左右翼屯田蠻軍及月兒魯部貧民糧,各三月。二十七年,大都民饑,減直糶糧五萬石。二十八年,以去歲隕霜害稼,賑宿衛士怯憐口糧二月,以饑賑徽州、溧陽等路民糧三月。三十一年,復賑宿衛士怯憐口糧三月。元貞元年,諸王阿難答部民饑,賑糧二萬石。是年六月,以糧一千三百石賑隆興府饑民,二千石賑千戶滅禿等軍。七月,以遼陽民饑,賑糧二月。大德元年,以饑賑遼陽、水達達等戶糧五千石,公主囊加真位糧二千石。是年,臨江、揚州等路亦饑,賑糧有差;腹裏並江南災傷之地賑糧三月。二年,賑龍興、臨江兩路饑民,又賑金復州屯田軍糧二月。四年,鄂州等處民饑,發湖廣省糧十萬石賑之。七年,以鈔萬錠賑歸德饑民。九年,澧陽縣火,賑糧二月。十一年,以饑賑安州高陽等縣糧五千石,漷州穀一萬石,奉符等處鈔二千錠,兩浙、江東等處鈔三萬餘錠、糧二十萬餘石。又勸率富戶賑糶糧一百四十餘萬石,凡施米者,驗其數之多寡,而授以院務等官。是年,又以鈔一十四萬七千餘錠、鹽引五千道、糧三十萬石,賑紹興、慶元、臺州三路饑民。皇慶元年,寧國饑,賑糧兩月。自延祐之後,腹裏、江南饑民歲加賑恤,其所賑或以糧,或以鹽引,或以鈔。



 京師賑糶之制:至元二十二年始行。其法於京城南城設鋪各三所,分遣官吏,發海運之糧,減其市直以賑糶焉。凡白米每石減鈔五兩,南粳米減鈔三兩,歲以為常。成宗元貞元年,以京師米貴,益廣世祖之制,設肆三十所,發糧七萬餘石糶之,白粳米每石中統鈔一十五兩,白米每石一十二兩,糙米每石六兩五錢。二年,減米肆為一十所,其每年所糶,多至四十餘萬石,少亦不下二十餘萬石。至大元年,增兩城米肆為一十五所,每肆日糶米一百石。四年,增所糶米價為中統鈔二十五貫。自是每年所糶,率五十餘萬石。泰定二年,減米價為二十貫。致和元年,又減為一十五貫雲。賑糶糧之外,復有紅帖糧。紅帖糧者,成宗大德五年始行。初,賑糶糧多為豪強嗜利之徒,用計巧取,弗能周及貧民。於是令有司籍兩京貧乏戶口之數,置半印號簿文貼,各書其姓名口數,逐月封貼以給。大口三斗,小口半之。其價視賑糶之直,三分常減其一,與賑糶並行。每年撥米總二十萬四千九百餘石,閏月不與焉。其愛民之仁,於此亦可見矣。



 入粟補官之制:元初未嘗舉行。天歷三年,內外郡縣亢旱為災,於是用太師答剌罕等言,舉而行之。凡江南、陜西、河南等處定為三等,令其富實民戶依例出米,無米者折納價鈔。陜西每石八十兩,河南並腹裏每石六十兩,江南三省每石四十兩,實授茶鹽流官,如不仕讓封父母者聽。錢穀官考滿,依例升轉。陜西省:一千五百石之上,從七品;一千石之上,正八品;五百石之上,從八品;三百石之上,正九品;二百石之上,從九品;一百石之上,上等錢穀官;八十石之上,中等錢穀官;五十石之上,下等錢穀官;三十石之上,旌表門閭。河南並腹裏:二千石之上,從七品;一千五百石之上,正八品;一千石之上,從八品;五百石之上,正九品;三百石之上,從九品;二百石之上,上等錢穀官;一百五十石之上,中等錢穀官;一百石之上,下等錢穀官。江南三省:一萬石之上,正七品;五千石之上,從七品;三千石之上,正八品;二千石之上,從八品;一千石之上,正九品;五百石之上,從九品;三百石之上,上等錢穀官;二百五十石之上,中等錢穀官;二百石之上,下等錢穀官。先已入粟,遙授虛名,今再入粟者,驗其糧數,照依資品,實授茶鹽流官。陜西:一千石之上,從七品;六百六十石之上,正八品;三百三十石之上,從八品;二百石之上,正九品;一百三十石之上,從九品。河南並腹裏:一千三百三十石之上,從七品;一千石之上,正八品;六百六十石之上,從八品;三百三十石之上,正九品;二百石之上,從九品。江南三省:六千六百六十石之上,正七品;三千三百三十石之上,從七品;二千石之上,正八品;一千三百三十石之上,從八品;六百六十石之上,正九品;三百三十石之上,從九品。先已入粟,實授茶鹽流官,今再入粟者,驗其糧數,加等升除。陜西:七百五十石之上,五百石之上,二百五十石之上,一百五十石之上,一百石之上。河南並腹裏:一千石之上,七百五十石之上,五百石之上,二百五十石之上,一百五十石之上。僧道入粟:三百石之上,賜六字師號,都省給之;二百石之上,四字師號,一百石之上,二字師號,禮部給之。四川省富實民戶,有能入粟赴江陵者,依河南省補官例行之。夫入粟補官,雖非先王之政,然荒札之餘,民賴其助者多矣,故特識於篇末而不敢略云。



\end{pinyinscope}