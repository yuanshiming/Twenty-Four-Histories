\article{本紀第十八 成宗一}

\begin{pinyinscope}

 成宗欽明廣孝皇帝,諱鐵穆耳,世祖之孫,裕宗真金第三子也。母曰徽仁裕聖皇后,弘吉烈氏。至元二年九月庚子生。



 二十四年,諸王乃顏反,世祖自將討平之。其後合丹復叛,命帝往征之,合丹敗亡。三十年六月乙巳,受皇太子寶,撫軍於北邊。



 三十一年春



 正月,世祖崩,親王、諸大臣遣使告哀軍中。



 夏四月壬午,帝至上都,左右部諸王畢會。先是,御史中丞崔彧得玉璽於故臣之家,其文曰「受命天於,既壽永昌」,上之徽仁裕聖皇后。至是手授於帝。甲午,既皇帝位,受諸王宗親、文武百官朝於大安閣,詔曰:



 朕惟太祖聖武皇帝受天明命,肇造區夏,聖聖相承,光熙前緒。迨我先皇帝體元居正以來,然後典章文物大備。臨御三十五年,薄海內外,罔不臣屬,宏規遠略,厚澤深仁,有以衍皇元萬世無疆之祚。



 我昭考早正儲位,德盛功隆,天不假年,四海缺望。顧惟眇質,仰荷先皇帝殊眷,往歲之夏,親授皇太子寶,付以撫軍之任。今春宮車遠馭,奄棄臣民,乃有宗籓昆弟之賢,戚畹官僚之舊,謂祖訓不可以違,神器不可以曠,體承先皇帝夙昔付托之意,合辭推戴,誠切意堅。朕勉徇所請,於四月十四日既皇帝位,可大赦天下。



 尚念先朝庶政,悉有成規,惟慎奉行,罔敢失墜。更賴祖親勛戚,左右忠良,各盡乃誠,以輔臺德。布告遠邇,咸使聞知。



 詔除大都、上都兩路差稅一年,其餘減丁地稅糧十分之三。系官逋欠,一切蠲免。民戶逃亡者,差稅皆除之。追尊皇考曰皇帝,尊太母元妃曰皇太后。庚子,遣攝太尉兀都帶等請謚於南郊。遣禮部侍郎李衎、兵部郎中蕭泰登齎詔使安南。中書省臣言:「陛下新即大位,諸王、駙馬賜與,宜依往年大會之例,賜金一者加四為五,銀一者加二為三。又江南分土之賦,初止驗其版籍,令戶出鈔五百文,今亦當有所加,然不宜增賦於民,請因五百文加至二貫,從今歲官給之。」從之。乙巳,賜駙馬蠻子帶銀七萬六千五百兩,闊里吉思一萬五千四百五十兩,高麗王王昛三萬兩。丁未,湖廣行省所屬寇盜竊發,復令劉國傑討之。戊申,太白晝見,又犯鬼。詔存恤征黎蠻、瓜哇等軍。己酉,雲南行省以所定路、府、州、縣來上:上路二,下路十一,下州四十九,中縣一,下縣五十。以金齒歸附官阿魯為孟定路總管,佩虎符。是月,即墨縣雹。



 五月庚戌朔,太白犯輿鬼。壬子,始開醮祠於壽寧宮,祭太陽、太歲、火、土等星於司天臺。戊午,遣攝太尉兀都帶奉玉冊玉寶,上大行皇帝尊謚曰聖德神功文武皇帝,廟號世祖;皇后尊謚曰昭睿順聖皇后;皇考尊謚曰文惠明孝皇帝,廟號裕宗。賜國王和童金二百五十兩,月兒魯百五十兩,伯顏、月赤察而各五十兩,銀、鈔、錦各有差。庚申,祭紫微星於雲仙臺。雲南部長適習、四川散毛洞主覃順等來貢方物,升其洞為府。丁卯,八番宣慰使斡羅思犯法,為人所訟,懼罪逃還京師。賜安西王阿難答鈔萬錠。己巳,改皇太后所居舊太子府為隆福宮,詹事院為徽政院,司議曰中議,府正曰宮正,家令曰內宰,典醫署曰掌醫,典寶曰掌謁,典設曰掌儀,典膳曰掌膳,仍增控鶴至三百人。詔各處轉運司官,欺隱奸詐為人所訟者,聽廉訪司即時追問,其案牘仍舊例於歲終檢之。升福建鹽提舉司為鹽轉運司,增捕私鹽人賞格。庚午,諸王亦里不花來朝,以瘠馬輸官,官酬其直,為鈔十有一萬五千錠。賜也速帶而、汪惟正兩軍將士糧五萬石,餉北征軍。壬申,御史臺臣言:「內外官府增置愈多,在京食祿者萬人,在外尤眾,理宜減並。」命與中書議之。用崔彧言,肅政廉訪司案牘,勿令總管府檢劾。詔議增官吏祿。以也速帶而所統將士貧乏,給鈔萬錠。乙亥,以扎珊知樞密院事。戊寅,封皇姑高麗王王昛妃忽都魯揭裏迷失為安平公主。賜亦都護金五百五十兩、銀七千五百兩,合迷裏的斤帖林金五十兩、銀四百五十兩。西平王奧魯赤言:「汪總帥之軍,多庇其富實,而令貧弱者應役。」命更易之。以月兒魯為太師,伯顏為太傅,月赤察而為太保。禁諸司豪奪鹽船遞運官物,僧道權勢之家私匿盜販。是月,密州路諸城縣、大都路武清縣雹,峽州路大水。



 六月庚辰朔,日有食之。辛巳,御史臺臣言:「名分之重,無逾宰相,惟事業顯著者可以當之,不可輕授。廉訪司官歲以五月分按所屬,次年正月還司。職官犯贓,敕授者聽總司議,宣授者上聞。其本司聲跡不佳者代之,受賂者依舊例比諸人加重。」帝曰:「其與中書同議。」乙酉,雲南金齒路進馴象三。丙戌,以雲南歲貢馬二千五百匹給梁王,數太多,命量減之。庚寅,必察不里城敢木丁遣使來貢。詔罷功德使司及泉府司官冗員。壬辰,立晉王內史府,復以光祿寺隸宣徽院。中書省臣言:「朝會賜與之外,餘鈔止有二十七萬錠。凡請錢糧者,乞量給之。」定西平王奧魯赤、寧遠王闊闊出、鎮南王脫歡及也先帖木而大會賞賜例,金各五百兩、銀五千兩、鈔二千錠、幣帛各二百匹;諸王帖木而不花、也只裏不花等,金各四百兩、銀四千兩、鈔一千六百錠、幣帛各一百六十匹。以帖木而復為平章政事。諸王阿只吉部玉速福屢叛,伏誅。以甘肅等處米價踴貴,詔禁釀酒。命月赤察而提調群牧事。乙未,以世祖、皇后、裕宗謚號播告天下。免所在本年包銀、俸鈔,及內郡地稅、江淮以南夏稅之半。乙亥,以乳保勞,封完顏伯顏為冀國公,妻何氏為冀國夫人。完澤貸民錢,多取其息,命依世祖定制。辛丑,浙西道提刑按察使弘吉烈帶阿魯灰受賂,遇赦免,復以為河西隴北道肅政廉訪使。御史臺臣言:「先朝決獄,隨罪輕重,笞杖異施,今止用杖,乞如舊制。」不允。宋使家鉉翁安置河間,年逾八十,賜衣服,遣還其家。癸卯,封駙馬闊里吉思為唐王,給金印。甲辰,詔翰林國史院修《世祖實錄》,以完澤監修國史。乙巳,給困赤禿出征軍士鈔各千戶千錠。丙午,太陰犯井。以昔寶赤從征諸軍自備馬一千一百九十餘匹,命給還其直。戊申,詔宗籓內外官吏人等,咸聽丞相完澤約束。以合剌思八斡節而為帝師,賜玉印。賜雪雪的斤公主鈔千錠,諸王伯答罕、末察合而部貧乏者三千錠,伯牙兀真、赤裏、由柔伯牙伯剌麻、闊怯倫、忙哥真各金五十兩,銀、鈔、幣有差。是月,東安州蝗。



 秋七月壬子,詔御史大夫月兒魯振臺綱,禁內外諸司減官吏俸為宴飲費。置隆福宮衛候司。癸丑,詔軍民各隸所司,無相侵越。乙卯,以諸王出伯所部四百餘戶乏食,徙其家屬就食內郡,仍賜以奧魯軍年例鈔三千錠。給瓜、沙之民徙甘州屯田者牛價鈔二千六百錠。以也的迷失為東昌路達魯花赤,中書省臣言其嘗官是郡,犯法五百餘款,今不宜復官,帝曰:「姑試之。」己未,復立平陽路之蒲、武鄉,保定路之博野,泰安州之新泰等縣。賜諸王出伯奧魯軍、也速帶而紅襖軍,幣帛各六萬匹。庚申,改侍衛都指揮使司為隆福宮左都威衛使司、右都威衛使司;以陜西道廉訪司沒入贓罰錢舊給安西王者,令行省別貯之。壬戌,詔中外崇奉孔子。癸亥,罷肇州宣慰司,並入遼東道。戊辰,減八番等處所設官二百一十六員。八番稱新附九十萬戶,設官四百二十四員,及遣官核實,止十六萬五千餘戶,故減之。行樞密院月的迷失、程鵬飛各加平章政事,中書省臣言:「樞密之臣不宜重與相銜。」帝命以軍職尊崇者授之。辛未,中書省臣言:「向御史臺劾右丞阿里嘗與阿合馬同惡,論罪抵死,幸得原免,不當任以執政。臣謂阿里得罪之後,能自警省,乞令執政如故。」從之。以軍戶所棄田產歲入及管軍官吏贖罪等鈔,復輸樞密院。癸酉,以陜西行省平章不忽木為中書平章政事。甲戌,立隨路民匠、打捕、鷹房、納綿等戶總管府,秩正三品。詔招諭暹國王敢木丁來朝,或有故,則令其子弟及陪臣入質。扎魯花赤言:「諸王之下有罪者,不聞於朝,輒自決遣。」詔禁治之。詔月兒魯守北邊,賜其所統軍士幣帛各萬匹,及西征軍士幣三萬匹、鈔三萬六千六百錠。賜不魯花真公主及諸王阿只吉女弟伯禿銀、鈔有差。是月,棣州陽信縣雹,大風拔木發屋,真定路之南宮、新河,易州之淶水等縣雹。



 八月庚辰,太白晝見。癸未,平灤路遷安等縣水,蠲其田租。戊子,初祀社稷,用堂上樂,歲以為常。己丑,以大都留守段貞、平章政事範文虎監浚通惠河,給二品銀印。令軍士復浚浙西太湖,澱山湖溝港,立新河運糧千戶所。詔諸路平準交鈔庫所貯銀九十三萬六千九百五十兩,除留十九萬二千四百五十兩為鈔母,餘悉運至京師。復立平陽之芮城、陵川等縣。辛卯,以忙哥撒而妻子為敵所掠,賜鈔八千錠。戊戌,太陰犯畢,太白犯軒轅。是月,德州之安德縣大風雨雹。九月壬子,聖誕節,帝駐蹕三部落,受諸王、百官賀。癸丑,詔有司存恤征瓜哇軍士死事之家。甲寅,口授諸王傅阿黑不花為丞相。丁巳,太白經天。庚申,以合魯剌及乃顏之黨七百餘人隸同知樞密院事不憐吉帶,習水戰。丙寅,太陰掩填星。辛未,太陰犯軒轅。乙亥,太白犯右執法,太陰犯平道。遣禿古鐵木兒等使閣藍。是月,趙州之寧晉等縣水。



 冬十月戊寅,車駕還大都。辛巳,江浙行省臣言:「陛下即位之初,詔蠲今歲田租十分之三。然江南與江北異,貧者佃富人之田,歲輸其租,今所蠲特及田主,其佃民輸租如故,則是恩及富室而不被於貧民也。宜令佃民當輸田主者,亦如所蠲之數。」從之。遼陽行省所屬九處大水,民饑,或起為盜賊,命賑恤之。江西行省臣言:「銀場歲辦萬一千兩,而未嘗及數,民不能堪。」命自今從實辦之,不為額。壬午,太白犯左執法。有事於太廟。癸巳,太陰掩填星。乙未,太陰犯井。金齒新附孟愛甸酋長遣其子來朝,即其地立軍民總管府。硃清、張瑄從海道歲運糧百萬石,以京畿所儲充足,詔止運三十萬石。辛丑,帝諭右丞阿里、參政梁德珪曰:「中書職務,卿等皆懷怠心。朕在上都,令還也的迷沙已沒財產,任明裏不花,皆至今未行。又不約束吏曹,使選人留滯。桑哥雖奸邪,然僚屬憚其威,政事無不立決。卿等其約束曹屬,有不事事者笞之。仍以朕意諭右丞相完澤。」壬寅,緬國遣使貢馴象十。乙巳,遣南巫里、速木答剌、繼沒剌矛、毯陽使者各還其國,賜以二珠虎符及金銀符,金、幣、衣服有差。初,也黑迷失征瓜哇時,嘗招其瀕海諸國,於是南巫里等遣人來附,以禁商泛海留京師,至是弛商禁,故皆遣之。



 十一月丁未朔,帝朝皇太后於隆福宮,上玉冊、玉寶。庚戌,行樞密院臣劉國傑討辰州賊,詔選州民刀弩手助其軍,他不為例。京師犯贓罪者三百人,帝命事無疑者,準世祖所定十三等例決之。己酉,太陰犯亢。庚戌,廣西鹽先給引於民,而徵其直,私鹽日橫,及官自鬻鹽,民復不售。詔先以鹽與民,而後征之。辛亥,中書省臣言:「國賦歲有常數,先帝嘗曰:『凡賜與,雖有朕命,中書其斟酌之。』由是歲務節約,常有贏餘。今諸王籓戚費耗繁重,餘鈔止一百十六萬二千餘錠。上都、隆興、西京、應昌、甘肅等處糴糧鈔計用二十餘萬錠,諸王五戶絲造作顏料鈔計用十餘萬錠,而來會諸王尚多,恐無以給。乞俟其還部,臣等酌量定擬以聞。」從之。壬子,詔以軍民不相統壹,罷湖廣、江西行樞密院,並入行省。乙卯,令河西僧人依舊助役。丁巳,以伯顏察而參議中書省事,其兄伯顏言曰:「臣叨平章政事,兄弟宜相嫌避。」帝曰:「卿勿復言。兄平章於上,弟參議於下,何所嫌也。」罷貴赤屯田總管府;罷宣政院所刻河西《藏經》板。庚申,太陰犯畢。甲子,詔禁作奸犯科者。以湖南道宣慰使何偉為中書參知政事。罷海北海南市舶提舉司。壬申,立覆實司。濟寧路立諸色戶計諸總管府,秩四品。癸酉,太白犯房。詔改明年為元貞元年。



 十二月辛巳,賜諸王亦思麻殷金五十兩。癸未,歲星犯房。丙戌,罷遼河等處人匠正副達魯花赤。丁亥,歲星犯鉤鈐。甲午,以諸王晃兀而、駙馬阿失等皆在軍,加賜金銀、鞍勒、弓矢、衣服各有差。乙未,以伯遙帶忽剌出所隸一千戶饑,賜鈔萬錠。壬辰,太陰犯鬼。戊戌,禁侵擾農桑者。庚子,太陰犯房,又犯歲星。選各衛精兵千人,命孛羅曷答兒等將之,戍和林,聽太師月兒魯節度,三年而更。用帝師奏,釋京師大闢三十人,杖以下百人;賜諸鰥寡貧民鈔二百錠。曲靜、澂江、普安等路夷官各以方物來貢。以東勝等處牛遞戶貧乏,賜鈔三千餘錠。卜阿里使麻八兒還都。阿思民為海都所虜,賜鈔三萬九千九百錠。是月,常德、岳、鄂、漢陽四州水,免其田租。是歲,斷大闢三十一人。



 元貞元年春正月戊申,諸王阿失罕來朝,賜金五十兩、銀四百五十兩。癸丑,以太僕卿只而合朗為御史大夫。甲寅,以從世祖狩杭海功,賜諸王忽剌出金五十兩、珠一串。乙卯,太陰犯填星,又犯畢。壬戌,以國忌,即大聖壽萬安寺飯僧七萬。癸亥,安西王阿難答、寧遠王闊闊出皆言所部貧乏,賜安西王鈔二十萬錠、寧遠王六萬錠。又以隕霜殺禾,復賑安西王山後民米一萬石。詔道家復行《金籙》、《科範》。以雲南行省左丞楊炎龍為中書左丞。乙丑,以亦奚不薛復隸雲南行省;以行樞密院既罷,賜行中書省長官虎符,領其軍。庚午,以江浙行省平章阿老瓦丁為參知政事。壬申,立北庭都元帥府,以平章政事合伯為都元帥,江浙行省右丞撒里蠻為副都元帥,皆佩虎符。立曲先塔林都元帥府,以釁都察為都元帥,佩虎符。饒州路達魯花赤阿剌紅、治中趙良不法,僉江東廉訪司事昔班、季讓受金縱之,事覺,昔班自殺,杖季讓,除名,仍沒其財產奴婢之半。罷瓜、沙等州屯田。癸酉,歲星犯東咸。甲戌,有飛書妄言硃清、張瑄有異圖者,詔中外慰勉之。乙亥,追封皇國舅按只那演為濟寧王,謚忠武,封皇姑囊家真公主為魯國大長公主,駙馬蠻子臺為濟寧王,仍賜金印。詔飭諸道鹽運司。



 二月丙子朔,安西王相鐵赤等請復立王相府,不許。令陜西省臣給其所需,仍以廉訪司沒入贓罰鈔與之。丁丑,翰林學士承旨留夢炎告老,帝以其在先朝言無所隱,厚賜遣之。命曷伯、撒里蠻、孛來將探馬赤軍萬人出征,聽諸王出伯節度。壬午,罷江南茶稅,以其數三千錠添入江西榷茶都轉運司歲額。詔貸斡脫錢而逃隱者罪之,仍以其錢賞首告者。癸未,熒惑犯太陰。丁亥,雲南行省平章也先不花言:「敢麻魯有兩夷未附,金齒亦叛服不常,乞調兵六千鎮撫金齒,置驛入緬。」從之。復以拱衛司為正三品。以濟寧王蠻子臺所部弘吉烈人貧乏,賜鈔一十八萬錠。戊子,思州田曷剌不花、雲南夷卜木、四川洞主查閭王、金齒帶梅混冬等來見。緬國阿剌扎高微班的來獻舍利、寶玩。甲午,以探馬赤軍出征,馬不足,詔除軍民官吏所乘,凡有馬者盡括之。壬辰,太陰犯平道。丁酉,車駕幸上都。癸卯,太陰犯歲星。以諸王亦憐真部馬牛驛人貧乏,賜鈔千錠。以工部尚書兼諸路金玉人匠總管府達魯花赤呂天麟為中書參知政事。立雲州銀場都提舉司,秩四品。中書省臣言:「近者阿合馬、桑哥怙勢賣官,不別能否,止憑解由選調,由是選法大壞。宜令廉訪司體覆以聞,省臺選官核實,定其殿最,以明黜陟。其廉訪司官,亦令省臺同選為宜。」從之。罷河西軍,聽各還其所屬。賜駙馬那懷鈔萬五千錠。以醮延春閣,賜天師張與棣、宗師張留孫、真人張志仙等十三人玉圭各一。制寶玉五方佛冠賜帝師。



 三月乙巳朔,安南世子陳日燇遣使上表慰國哀,又上書謝寬貰恩,並獻方物。丙午,遣密剌章以鈔五萬錠授征西元帥,令市馬萬匹,分賜二十四城貧乏軍校。庚戌,太陰犯填星。壬子,禁來朝官斂所屬俸。丙辰,給月兒魯、禿禿軍炒米萬石。金齒夷洞蠻來見,賜衣遣之。戊午,罷福建銀場提舉司,其歲額銀以有司領之。中書省臣言:「樞密院、御史臺例應奏舉官屬,其餘諸司不宜奏請,今皆請之,非便。」詔自今已後,專令中書擬奏。以東作方殷,罷諸不急營造,惟帝師塔及張法師宮不罷。壬戌,地震。太陰犯房。丙寅,國王和童隱所賜本部貧民鈔三百五十錠,命臺臣遣人按問以愧之。詔免醫工門徭。增置蒙古學正,以各道肅政廉訪司領之。



 夏四月辛巳,妖人蒙蟲僭擬,及其黨十三人伏誅。賜章河至苦鹽貧乏驛戶,鈔一萬二千九百餘錠。丙戌,諸王也只里以兵五千人戍兀魯思界,遣使來求馬,帝不允。庚寅,太陰犯東咸。封乳母楊氏為趙國安翼夫人。癸巳,以同知烏撒烏蒙等處宣慰使司事牙那木假兵部尚書,佩虎符,使馬答兒的陰。戊戌,給扈從探馬赤軍市馬鈔十二萬錠。庚子,立掌謁司,掌皇太后寶,秩四品,以宦者為之。賜貴赤親軍貧乏戶鈔四萬一千五百餘錠。癸卯,以諸王出伯所統探馬赤、紅襖軍各千人,隸西平王奧魯赤。設各路陰陽教授,仍禁陰陽人不得游於諸王、駙馬之門。以貴赤萬戶忽禿不花等所部為敵所掠,賜鈔有差。是月,真定路之平山、靈壽等縣有蟲食桑。



 閏四月丙午,為皇太后建佛寺於五臺山,以前工部尚書涅只為將作院使,領工部事;燕南河北道肅政廉訪使宋德柔為工部尚書,董其役;以大都、保定、真定、平陽、太原、大同、河間、大名、順德、廣平十路,應其所需。癸丑,歲星犯房。甲寅,太陰犯平道。立梭厘招討使司,以答而忽帶為使,佩虎符。乙卯,太陰犯亢。丁巳,太陰掩房。己未,罷打捕鷹房總管府,及司籍、周用、薄斂等庫,及徽州路銀場。各處鹽使司鹽場,改設司令、司丞。仍免大都今歲田租。弛甘州酒禁。庚申,河南行省虧兩淮歲辦鹽十萬引、鈔五千錠,遣扎剌而帶等往鞫實,命隨其罪之輕重治之。陜西行省增羨鹽鈔一萬二千五百餘錠,山東都轉運使司別思葛等增羨鹽鈔四千餘錠,各賜衣以旌其能。南人洪幼學上封事,妄言五運,笞而遣之。壬戌,塔即古阿散從不法伏誅。詔禁行省、行泉府司抽分市舶船貨,而固匿其珍細者。戊辰,遣愛牙赤核實高麗國儲糧。平陽民訴諸王小薛、曲列失伯部曲恣橫,遣官鞫之。賜安南國王陳益稷鈔千錠。是月,蘭州上下三百餘里河清三日。



 五月戊寅,以魯國大長公主建佛寺於應昌,給鈔千錠、金五十兩。命麥術丁、何榮祖等厘正選法。己卯,竄忙兀部別闍於江西,俾從月底迷失討賊。庚辰,詔各省止存儒學提舉司一,餘悉罷之。升江南平陽等縣為州。以戶為差,戶至四萬五萬者為下州,五萬至十萬者為中州,下州官五員,中州六員。凡為中州者二十八,下州者十五。又以戶不及額,降連州路為連州。增重挑補鈔人罪,告捕者仍優其賞,令犯人給之。辛巳,罷行大司農司。加平章政事麥術丁為平章軍國重事,中書左丞、議中書省事何榮祖為昭文館大學士,與中書省事。甲申,詔自元貞元年五月以前逋欠錢糧者,皆罷征。丁亥,太陰犯南斗。甲午,以諸王阿只吉部貧乏,賜鈔二十萬錠。江浙行省臣鐵木而不聽詔,遣官責之。丙申,以伯顏之子買的為僉書樞密院事。太后言其父盡心王室,欲令代其父官,帝以其年尚小,故有是命。詔以農桑水利諭中外。鞏昌府金州、西和州、會州雨雹,無麥禾。饒州、鎮江、常州、湖州、平江、建康、太平、常德、澧州皆水。六月戊申,濟南路之歷城縣大清河水溢,壞民居。壬子,高麗王王昛乞為太師中書令,不允。以近邊役煩及水災,免咸平府民八百戶今年賦稅。詔遼陽省進海東青鶻二十四驛,每驛給牛六頭,使者食米五石,鷹食羊五口;又狗遞十二驛,每戶給鈔十錠。甲寅,翰林承旨董文用等進《世祖實錄》。乙卯,江西行省所轄郡大水無禾,民乏食,令有司與廉訪司官賑之,仍弛江河湖泊之禁,聽民採取。升沅州為路,以靖州隸之。遣使與各省官就遷調邊遠六品以下官,並左右兩江宣慰司都元帥府、宣撫司,為廣西兩江道宣慰司都元帥府,以靜江為治所,仍分司邕州。敕:「凡上封事者,命中書省發緘視之,然後以聞。」詔河西僧納租稅。癸亥,立蒙古軍都元帥府於西川,徑隸樞密院,以阿剌鐵木而、岳樂罕並為都元帥,佩虎符。河西隴北道廉訪司鞫張萬戶不法,西平王奧魯赤沮撓其事,帝命諭之。甲子,以安西王所部出征軍妻孥乏食,給糧二千石。昭、賀、藤、邕、澧、全、衡、柳、吉、贛、南安等處蠻寇竊發,以軍民官備御不嚴,撫字不至,皆責而降之。駙馬濟寧王蠻子臺私殺罪人,御史臺臣言其專擅,有旨諭蠻子臺令知之。庚午,立西域衛親軍都指揮使司,以迷而的斤為都指揮使。是月,汴梁路蝗,利州、蓋州螟,泰安、曹州、濟寧路水,鞏昌、環州、慶陽、延安、安西旱。



 秋七月乙亥,徙甘、涼禦匠五百餘戶於襄陽。詔江南地稅輸鈔。丁丑,太陰犯亢。罷追問已原逋欠。普顏怯裏迷失公主等,俱以其部貧乏來告,賜鈔計四十九萬餘錠。御史臺臣言:「內地盜賊竊發者眾,皆由國家赦宥所致,乞命中書立為條格,督責所屬,期至盡滅。」制曰:「可。」乙卯,詔申飭中外:「有儒吏兼通者,各路舉之,廉訪司每道歲貢二人,省臺委官立法考試,中程者用之,所貢不公,罪其舉者。職官坐贓論斷,再犯者加二等。倉庫官吏盜所守錢糧,一貫以下笞之,至十貫杖之,二十貫加十等,一百二十貫徒一年,每三十貫加半年,二百四十貫徒三年,滿三百貫者死。計贓以至元鈔為則。」給江南行御史臺守護軍百人。減海南屯田軍之半,還其元翼。詔增給諸軍藥餌價直。壬午,立肇州屯田萬戶府,以遼陽行省左丞阿散領其事。甲申,歲星犯房。給塞下貧民鈔二萬四千錠。己丑,賜劉國傑玉帶錦衣,旌其戰功。辛卯,以禿禿合所部貧乏,賜鈔十萬錠。戊戌,硃永福、邊珍裕以妖言伏誅。扎魯忽赤文移舊用國語,敕改從漢字。壬寅,詔易江南諸路天慶觀為玄妙觀,毀所奉宋太祖神主。大都、遼東、東平、常德、湖州武衛屯田大水,隆興路雹,太原、平陽、安豐、河間等路旱。



 八月乙酉,太陰犯牛。壬子,太陰犯壘壁陣。辛酉,緬國進馴象三。癸亥,賑遼陽民被水者糧兩月。己巳,以駙馬那懷知樞密院事。金、復州屯田有蟲食禾,汴梁、安西、真定等路旱,平江、安豐等路大水。九月甲戌,帝至自上都。乙亥,用帝師奏,釋大闢三人、杖以下四十七人。戊寅,以八撒而治私第,給鹽萬引。詔輸米十萬石於榷場故廩,以備北塞。以探馬赤軍士所至擾民,令合伯鎮之,犯者罪其主將。乙卯,罷四川淘金戶四千,還其元籍,罪初獻言者。庚辰,罷寧夏路行中書省,以其事並入甘肅行省。丁亥,瓜哇遣使來獻方物。己丑,給桓州甲匠糧千石。壬辰,湖州司獄郭訴浙西廉訪司僉事張孝思多取廩餼,孝思系於獄。行臺令監察御史楊仁往鞫,而江浙行省平章鐵木而逮孝思至省訊問,又令其屬官與仁同鞫事,仁不從,行臺以聞。詔省臺遣官鞫問,既引服,皆杖之。諸王小薛部眾擾民,遣官按問,杖其所犯重者,余聽小薛責之。甲午,太陰犯軒轅。戊戌,太陰犯平道。宣德府大水,軍民乏食,給糧兩月。武衛萬盈屯及延安路隕霜殺禾,高郵府、泗州、賀州旱,平江、廬州等路大水。



 冬十月癸卯,有事於太廟。中書省臣言:「去歲世祖、皇后、裕宗祔廟,以綾代玉冊。今玉冊、玉寶成,請納諸各室。」帝曰:「親享之禮,祖宗未嘗行之,其奉冊以來,朕躬祝之。」命獻官迎導入廟。給江浙、河南巡邏私鹽南軍兵仗。癸丑,以西北叛王將入自土蕃,命平章軍國重事答失蠻往征之,仍敕便宜總帥發兵千人從行,聽其節度。甲寅,中書省、御史臺臣言:「江浙行省平章明裏不花陳臺憲非便事,臣等議,乞自今監察御史廉訪司有所按核,州縣官與本路同鞫,路官與宣慰司同鞫,宣慰司官與行省同鞫。」制曰:「可。」詔諸王、駙馬部民既隸軍籍者,毋奪回本部。己未,賜各衛士貧乏者鈔二萬九千三百餘錠。辛酉,辰星犯房。壬戌,辰星犯鍵閉。癸亥,賜諸王巴撒而、火而忽答孫、禿剌三部鈔四萬八千五百餘錠。丁卯,以博而赤、答剌赤等貧乏,賜鈔二萬九千餘錠。戊辰,太白晝見,太陰犯房。遣安南朝貢使陳利用等還其國,降詔諭陳日燇。



 十一月甲戌,太白經天及犯壘壁陣。辛巳,置江浙行省檢校官二員,立江浙金銀洞冶轉運使司。乙酉,太陰犯井。丙戌,毯陽酋長之兄脫杭捧於、法而剌酋長之弟密剌八都、阿魯酋長之弟脫杭忽先等,各奉金表來覲。丁亥,太陰犯鬼。戊子,賜阿魯酋長虎符。癸巳,賜安西王甲胄、槍撾、弓矢、橐鞬等十五萬八千二百餘事。戊戌,升贛州路之寧都、會昌二縣為州,以石城縣隸寧都,瑞金縣隸會昌。詔江浙行省括隱漏官田及檢劾富強避役之戶。



 十二月庚子朔,遣集賢院使阿里渾撒裡等祭星於司天臺。癸卯,以駙馬阿不花所部民貧,賜鈔萬錠。賜諸王押忽禿、忽剌出、阿失罕等金各二百五十兩、鈔五百錠。丙辰,太陰犯軒轅。荊南僧普昭等偽撰佛書,有不道語,伏誅。己未,詔大都路,凡和顧和買及一切差役,以諸色戶與民均當。賜諸王不顏鐵木而、阿八也不干金各五百兩、銀五千兩、鈔二千錠、幣帛各二百匹,其幼王減五分之一。以各道廉訪司官八員,員一印,命收其三。甲子,太陰犯天江。賜帝師雙龍紐玉印。也速帶而之軍因李璮亂去山東,其元駐之地為人所墾,歲久成業,爭訟不已;命別以境內荒田給之,正軍五頃,餘丁二頃,已滿數者不給。減海運腳價鈔一貫,計每石六貫五百文,著為令。徙縉山所居乞里乞思等民於山東,以田與牛、種給之。丁卯,禁諸王輒召有司官吏。己巳,詔免軍器匠門徭。是歲,斷大闢三十人。



\end{pinyinscope}