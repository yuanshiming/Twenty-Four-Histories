\article{志第十二 地理三}

\begin{pinyinscope}

 ○陜西諸道行御史臺



 陜西等處行中書省,為路四、府五、州二十七,屬州十二,屬縣八十八。本省陸站八十處,水站一處。



 奉元路,上。唐初為雍州,後改關內道,又改京兆府,又以京城為西京,又曰中京,又改上都。宋分陜西永興、秦鳳、熙河、涇原、環慶、鄜延為六路。金並陜西為四路。元中統三年,立陜西四川行省,治京兆。至元初,並雲陽縣入涇陽,櫟陽縣入臨潼,終南縣入盩啡。十六年,改京兆為安西路總管府。二十三年,四川置行省,改此省為陜西等處行中書省。大德元年,移雲南行臺於此,為陜西行臺。皇慶元年,改安西為奉元路。戶三萬三千九百三十五,口二十七萬一千三百九十九。壬子年數。領司一、縣十一、州五。州領十五縣。



 錄事司。



 縣十一



 咸寧,下。長安,下。咸陽,下。興平,下。臨潼,下。屯田一千二十頃有奇。藍田,下。涇陽,下。至元二年,並入高陵縣。三年復立。屯田一千三十頃有奇。高陵,下。鄠縣,下。盩啡,下。屯田九百四十三頃有奇。郿縣。下。舊為郿州,添置柿林縣。至元元年,省郿州為郿縣,廢柿林。



 州五



 同州,下。唐初為同州,又改馮翊郡,又復為同州。宋為定國軍。金因之。元仍為同州。領五縣:



 朝邑,下。白水,下。郃陽,下。澄城,下。韓城。下。唐、宋為韓城縣,金曰楨州。至元元年,州廢。二年再立。六年,州又廢,止設縣。



 華州,下。唐改鎮國軍。宋改鎮潼軍。金改金安軍。元復為華州。西嶽華山在焉。領三縣:



 華陰,下。蒲城,下。渭南。下。屯田一千二百二十二頃有奇。



 耀州,下。唐初立宜州,後為華原縣,後又為耀州。宋為感義軍,又改感德軍,又為耀州如故。金因之。元至元元年,並華原縣入州,又並美原入富平。領三縣:



 三原,下。富平,下。同官。下。



 乾州,下。唐以高宗乾陵所在,改醴泉縣為奉天,又升為乾州。宋改醴州。金復改乾州。元至元元年,並奉天縣入州。五年,復置奉天,省好畤入焉,又割永壽來屬,後又改奉天為醴泉。領三縣:



 醴泉,下。武功,下。永壽。下。宋、金屬邠州。至元十五年,徙縣治於麻亭。



 商州,下。唐初為商州,又改上洛郡,又復為商州。宋及元皆因之。領一縣:



 洛南。下。



 延安路,下。唐初為延州,又改延安郡,又為延州。宋為延安府。金為鄜延路。元改延安路。戶六千五百三十九,口九萬四千六百四十一。壬子年數。領縣八、州三。州領八縣。本路屯田四百八十餘頃。



 縣八



 膚施,下。甘泉,下。宜川,下。元初置司候司。至元六年,省入宜川。延長,下。延川,下。安定,下。本宋舊堡,元壬子年升為安定縣。至元元年,析置丹頭縣。四年,並丹頭入本縣。安塞,下。本金舊堡,壬子年升為縣。保安。下。金為保安州,至元六年,降為縣。



 州三



 鄜州,下。唐初為鄜州,又改洛交郡,又復為鄜州。宋、金因之。舊領洛交、洛川、鄜城、直羅四縣。元至元四年,並鄜城入洛川,又並洛交、直羅入州。六年,廢坊州,以中部、宜君二縣來屬。領三縣:



 洛川,下。中部,下。宜君。下。



 綏德州,下。唐綏州,又改上郡,又為綏州。宋為綏德軍。金為州,領八縣。歸附後,並嗣武入米脂,綏平入懷寧。至元四年,並定戎入米脂,懷寧入青澗,又並義合、綏德入本州。領二縣:



 青澗,下。米脂。下。



 葭州,下。唐銀州。宋為晉寧軍。金改為葭州。元至元六年,並通秦、彌川、葭盧入州,並太和入神木,建寧入府穀。領三縣:



 神木,下。元初創立雲州於古麟州之神木寨。至元六年,廢州為縣。吳堡,下。府穀。下。後唐為府州。元初建州治。至元六年,廢為縣。



 興元路,下。唐為梁州,又改漢中郡,又為興元府。宋仍舊名。元立興元路總管府,久之,以鳳、金、洋三州隸焉。宋時領南鄭、西縣、褒城、廉水、城固五縣,後廢廉水入南鄭。元初割出西縣屬沔州,以洋州西鄉縣來屬。戶二千一百四十九,口一萬九千三百七十八。至元二十七年數。領縣四、州三。



 縣四



 南鄭,下。城固,下。褒城,下。西鄉,下。



 州三



 鳳州,下。唐初為鳳州,後升節度府。宋為團練州。至元五年,以在郭梁泉縣並入州,隸興元路。



 洋州,下。唐改洋川郡,又復為洋州,後更革不常。宋復為洋州。元至元二年,省興道、真符二縣入州。



 金州,下。唐改西城郡為金州。宋升為金房開達四州路。元為散州。



 陜西漢中道肅政廉訪司



 鳳翔府,唐為扶風郡,又為鳳翔府,號西京。宋、金因其名。元初割平涼府、秦、隴、德順、西寧、鎮原州隸鞏昌路,廢恆州,以所領盩啡縣隸安西府路,尋立鳳翔路總管府。至元九年,更為散府。戶二千八十一,口一萬四千九百八。壬子年數。領縣五:



 鳳翔,下。屯田九十頃有奇。扶風,下。岐山,下。寶雞,下。



 麟游。下。



 邠州,下。唐豳州,以字類幽,改為邠。宋、金以來皆因之。領縣二:



 新平,下。淳化。下。至元七年,並三水入本縣。



 涇州,下。唐改安定郡,後仍為涇州。宋改彰化軍。舊領保定、長武、靈臺、良原四縣。金改保定縣為涇川。元初以隸都元帥府,立總司轄邠州,後屬鞏昌都總帥府,或隸平涼府、陜西省,所隸不一,今直隸省。領縣二:



 涇川,下。涇州治此,即保定。靈臺。下。至元七年,並歸涇川。十一年復立,以良原並入,而長武仍並於涇川。



 開成州,下。唐原州。宋為鎮戎軍。金升鎮戎州。元初仍為原州。至元十年,皇子安西王分治秦、蜀,遂立開成府,仍視上都,號為上路。至治三年,降為州。領縣一、州一。



 縣一



 開成。



 州一



 廣安州。本鎮戎地,金升為縣,隸鎮戎州,經亂荒廢。元至元十年,安西王封守西土,既立開成路,遂改為廣安縣,募民居止,未幾戶口繁夥。十五年升為州,仍隸本路。



 莊浪州。下。沿革闕。成宗大德八年二月,降莊浪路為州。



 鞏昌等處總帥府



 鞏昌府,唐初置渭州,後曰隴西郡,尋陷入吐蕃。宋復得其地,置鞏州。金為鞏昌府。元初改鞏昌路便宜都總帥府,統鞏昌、平涼、臨洮、慶陽、隆慶五府及秦、隴、會、環、金、德順、徽、金洋、安西、河、洮、岷、利、巴、沔、龍、大安、褒、涇、邠、寧、定西、鎮原、階、成、西和、蘭二十七州,又於成州行金洋州事。至元五年,割安西州屬脫思麻路總管府。六年,以河州屬吐蕃宣慰司都元帥府。七年,並洮州入安西州。八年,割岷州屬脫思麻路。十三年,立鞏昌路總管府。十四年,復行便宜都總帥府事,其年割隆慶府,利、巴、大安、褒、沔、龍等州隸廣元路。二十一年,又以涇、邠二州隸陜西漢中道宣慰司,而帥府所統者,鞏昌、平涼、臨洮、慶陽,府凡四;秦、隴、寧、定西、鎮原、階、成、西和、蘭、會、環、金、德順、徽、金洋,州凡十有五。戶四萬五千一百三十五,口三十六萬九千二百七十二。壬子年數。領司一、縣五。



 錄事司。



 縣五



 隴西,下。寧遠,下。伏羌,下。本舊寨,至元十三年升縣。通渭,下。鄣縣。下。宋名鹽川寨,金為鎮,至元十七年,置今縣。



 平涼府,唐為馬監,隸原州。宋為涇原路,升平涼軍。金立平涼府。元初並潘原縣入平涼,化平入華亭,隸鞏昌帥府。領縣三:



 平涼,下。屯田一百一十五頃。崇信,下。華亭。下。



 臨洮府,唐臨洮軍。宋為鎮洮軍,又為熙州。金為臨洮府。元至元十三年,復以渭源堡升為縣。領縣二:



 狄道,下。渭源,下。



 慶陽府,唐慶州。宋環慶路,改慶陽軍,又升府。金為慶原路。元初改為慶陽散府,至元七年,並安化、彭原入焉。領縣一:



 合水。下。



 秦州,中。唐初為秦州。宋為天水郡。金為秦州。舊領六縣。元至元七年,並雞川、隴城入秦安,治坊入清水。領縣三:



 成紀,中。清水,中。秦安。下。



 隴州,中。唐改汧陽郡,復為隴州。宋、金置防禦使。舊領四縣。元至元七年,省吳山、隴安入汧源,十三年,罷防禦使為散郡。有吳山為西鎮。領縣二:



 汧源,中。汧陽。下。



 寧州,下。唐初改北地郡為寧州。宋、金因之。元至元七年,並襄樂、安定、定平入州。領縣一:



 真寧。下。



 定西州,下。本唐渭州西市,五代淪於先零。宋置定西城。金改定西縣,復升為州,仍置安西縣,倚郭,通西二寨,並置縣來屬。元至元三年,並三縣入本州。屯田四百六十七頃。



 鎮原州,下。唐原州,又為平涼郡。宋、金因之。元改鎮原州,以鎮戎州之東山、三川二縣來屬。至元七年,例並州縣,遂以臨涇、彭陽及東山、三川四縣入本州。屯田四百二十六頃有奇。



 西和州,下。唐岷州,又改和政郡,又仍為岷州。宋改曰西和。舊領縣三,大潭、祐川軍興久廢,惟有長道一縣,元至元七年,亦並入本州。



 環州,下。唐改威州。宋復為環州,後與慶州定為環慶路。金隸慶陽府。元初為散郡。舊領通遠一縣,元至元七年並入本州。



 金州,下。本蘭州龕穀寨,金升寨為縣,以龕穀為金州治所。元至元七年,並縣入州。



 靜寧州,下。宋慶歷中,以渭州隴干城置德順軍,復置隴干縣。金升為州。元初並治平、水洛入隴干,後復省隴干,改為靜寧州。領縣一:



 隆德。下。



 蘭州,下。唐初置,後改金城郡,又仍為蘭州。宋、金因之。元初領阿干一縣及司候司,至元七年並司縣入本州。



 會州,下。唐初改西會州,又為粟州,又為會寧郡,又為會州。宋置敷川縣。金置保川縣,陷於河西,僑治州西南百里會川城,名新會州。元初棄新會州,遷於所隸西寧縣。至元七年,並縣入州。



 徽州,下。元兵入蜀,鳳州二縣首降,以鳳州仍治梁泉,別置南鳳州治於河池。後又升永寧鄉為縣,與兩當同為屬邑。至元元年,改為徽州。七年,並河池、永寧二縣入州。領縣一:



 兩當。下。



 階州,下。唐初置武州,又改武都郡,又更名階州。宋因之。今州治在柳樹城,距舊城東八十里。舊領福津、將利二縣,至元七年並入本州。



 成州,下。唐初為成州,又改同谷郡,後仍為成州。宋因之。舊領同谷、慄亭二縣。元初歲壬寅,以田世顯挈成都府歸附,今遷於慄亭,行慄亭管民司事,不隸成州,割天水縣來屬。至元七年,並同谷、天水二縣入州。



 金洋州,本隸興元路,戊戌歲,有雷、李二將挈民戶歸附,令遷至成州,自行金洋州事。



 土蕃等處宣慰司都元帥府至元九年,於土蕃西川界立寧河站。



 河州路。下。領縣三:



 定羌,下。寧河,下。安鄉,下。



 雅州。下。憲宗戊午歲,攻破雅州,石泉守將趙順以城降。領縣五:



 名山。下。瀘山,下。百丈,下。榮經,下。嚴道。下。



 黎州。下。至元十八年,給黎、雅州民千一百五十四戶、鈔二千三百八錠,以資牛具種實。領縣一:



 漢源。下。



 洮州。下。領縣一:



 可當。下。



 貴德州。下。



 茂州。下。領縣二:



 汶山,下。汶川。下。



 脫思麻路。



 岷州。下。



 鐵州。下。



 碉門魚通黎雅長河西寧遠等處宣撫司,至元二年,授雅州碉門安撫使高保四虎符,高保四言:「碉門舊有城邑,中統初為宋人所廢,眾依山為柵,去碉門半舍,欲復戍故城,便於守佃。」敕秦蜀行省:「彼中緩急,卿等相度,順得其宜,城如可復,當助成之。」三年,諭四川行樞密院,遣人於碉門、巖州西南沿邊,丁寧告諭官吏軍民,有願來歸者,方便接納,用意存恤,百姓貧者賑之,願徙近裏城邑者以屋舍給之。



 禮店文州蒙古漢兒軍民元帥府自河州以下至此多闕,其餘如朵甘思、烏思藏、積石州之類尚多,載籍疏略,莫能詳錄也。



 四川等處行中書省,為路九、府三,屬府二,屬州三十六,軍一,屬縣八十一。蠻夷種落,不在其數。本省陸站四十八處,水站八十四處。鹽場十二處,俱鹽井所出。井凡九十五眼,在成都、夔府、重慶、敘南、嘉定、順慶、廣元、潼川、紹慶等路所管州縣萬山之間。



 西蜀四川道肅政廉訪司



 成都路,上。唐改蜀郡為益州,又改成都府。宋為益州路,又為成都府路。元初撫定,立總管府,設錄事司。至元十三年,領成都、嘉定、崇慶三府,眉、邛、隆、黎、雅、威、茂、簡、漢、彭、綿十一州,後嘉定自為一路,以眉、雅、黎、邛隸之。二十年,又割黎、雅屬吐蕃招討司,降崇慶為州,隆州並入仁壽縣,隸本府。戶三萬二千九百一十二,口二十一萬五千八百八十八。至元二十七年數。領司一、縣九、州七。州領十一縣。



 錄事司。



 縣九



 成都,下。唐、宋為成都府治所。至元十三年,以本縣元管大城內西北隅並入錄事司。華陽,下。新都,下。郫縣,下。溫江,下。雙流,下。新繁,下。仁壽,下。唐為陵州。宋為隆州。元至元二十年,以此州地荒民散,並為仁壽縣,隸成都府路。金堂。下。宋屬懷安軍。元初升為懷州,而縣屬如故。至元二十年,並州入金堂縣,隸成都府路。



 州七



 彭州,下。唐置蒙州,又為彭州。宋及元因之。領二縣:



 蒙陽,下。崇寧。下。



 漢州,下。唐為德陽郡,又為漢州。自唐至宋,苦於兵革,民不聊生。元中統元年,復立漢州。領三縣:



 什邡,下。德陽,下。至元八年,升為德州。十三年,仍為縣,隸成都路。十八年,復來屬。綿竹。下。至元十三年,以戶少並入州,後復置。



 安州,下。唐置石泉縣。宋升為軍。元中統五年,升為安州。領一縣:



 石泉。下。



 灌州,下。唐導江縣。五代為灌州。宋為永康軍,後廢為灌口寨。元初復立灌州。至元十三年,以導江、青城二縣戶少,省入州。青城陶壩立屯田萬戶府。



 崇慶州,下。唐為唐安郡,又為蜀州。宋為崇慶軍。元至元十二年,立總管府。二十年,改為崇慶州,並江原縣入州。本州有屯田萬戶府。領二縣:



 晉原,下。新津。下。



 威州,下。唐維州。宋改威州,領保寧、通化二縣。元至元十九年,並保寧入州。領一縣:



 通化。下。



 簡州,下。唐析益州置。宋因之。元至元二十年,並附郭陽安縣入州。二十二年,並成都府所屬靈泉縣來隸。而本州有平泉,以地荒,竟廢之。



 嘉定府路,下。唐初為嘉州,又改犍為郡,又仍為嘉州。宋升嘉定府。元至元十三年,立總管府。舊領龍游、夾江、峨眉、犍為、洪雅五縣。二十年,並洪雅入夾江。領司一、縣四、州二。州領三縣。戶口數闕。



 錄事司。



 縣四



 龍游,下。夾江,下。峨眉,下。犍為。下。



 州二



 眉州,下。唐改嘉州,又仍為眉州。元至元十四年,隸嘉定路。領二縣:



 彭山,下。青神。下。



 邛州,唐孵置邛州,又改臨邛郡,又仍為邛州。元至元十四年,立安撫司,兼行州事。二十一年,並臨邛、依政、蒲江三縣入州。領一縣:



 大邑。下。



 廣元路,下。唐初為利州,又改益昌郡,又復為利州。宋為利州路,端平後兵亂無寧歲,地荒民散者十有七年,元憲宗三年,立利州治,設都元帥府。至元十四年,罷帥府,改為廣元路。戶一萬六千四百四十二,口九萬六千四百六。至元二十七年數。領縣二、府一、州四。府領三縣,州領七縣。本路屯田九頃有奇。



 縣二



 綿谷,下。昭化。下。元初並葭萌入焉。



 府一



 保寧府,下。唐隆州,又改閬州,又為閬中郡。後唐為保寧軍。元初立東川路元帥府。至元十三年,升保寧府。二十年,罷元帥府,改保寧路。初領新得、小寧二州,後並入閬中縣,又並奉國入蒼溪縣,新井、新政、西水總入南部縣,仍改為府,隸廣元路。本府屯田一百一十八頃有奇。領三縣:



 閬中,下。倚郭。蒼溪,下。南部。下。



 州四



 劍州,下。唐為始州,後改劍州。宋升普安軍,又為隆慶府。元至元二十年,改劍州。領二縣:



 普安,下。至元二十年,並普城、劍門入焉。梓潼。下。



 龍州,下。唐初為龍門郡,又改龍州,又改江油郡,又改應靈郡。宋改政州,繼復舊。元憲宗歲戊午,宋守將王知府以城降。至元二十二年,並江油、清川二縣入焉。



 巴州,下。唐初改巴州,又改清化郡,又為巴州。宋領化城、難江、恩陽、曾口、上通江、下通江六縣。元至元二十年,並難江、恩陽二縣入化城,上、下通江二縣入曾口。領二縣:



 化城,下。曾口。下。



 沔州,下。唐初為興州,又為順政郡,又改興州。宋改沔州。元至元十四年,隸廣元路。二十年,廢褒州,止設鐸水縣,遷沔州而治焉。領三縣:



 鐸水,下。倚郭。大安,下。本大安州,至元二十年,降為縣以來屬。略陽。下。至元二十年,並長舉及西縣入焉。



 順慶路,下。唐為南充郡,又改梁州,又改充州。宋升順慶府。元中統元年,立征南都元帥府。至元四年,置東川路統軍司,後改東川府。十五年,復為順慶。二十年,升為路,設錄事司。戶二千八百二十一,口九萬五千一百五十六。至元二十七年數。領司一、縣二、府一、州二。府領二縣,州領五縣。



 錄事司。



 縣二



 南充,下。至元二十年,並漢初入焉。西充。下。至元二十年,並流溪舊縣入焉。



 府一



 廣安府,唐屬宕渠、巴西、洛陵三郡。宋置廣安軍,又改寧西軍。元至元十五年,廢寧西軍。二十年,升為廣安府。舊領渠江、岳池、和溪、新明四縣,後並和溪、新明入岳池。領二縣:



 渠江,下。倚郭。岳池。下。



 州二



 蓬州,下。唐改蓬山郡,又仍為蓬州。元初立宣撫都元帥府,後罷。至元二十年,立蓬州路總管府,後復為蓬州。領三縣:



 相如,至元二十年,以金城寨入焉。營山,下。至元二十年,並良山入焉。



 儀隴。下。至元二十年,並蓬池、伏虞入焉。



 渠州,下。唐初為渠州,又改濆山郡,又為渠州。宋屬潼川府。元至元十一年,立渠州安撫司。二十年,罷安撫司,以渠州為散郡。領二縣:



 流江,下。大竹。下。至元二十年,並鄰山、鄰水入焉。



 潼川府,唐梓州,又改梓潼郡,又為梓州。宋改靜戎軍,又改靜安軍,又升潼川府。兵後地荒,元初復立府治。至元二十年,並涪城及錄事司入郪縣,通泉入射洪,東關入鹽亭,銅山入中江。領縣四、州二。戶口闕。



 縣四



 郪縣,下。倚郭。中江,下。射洪,下。鹽亭。下。



 州二



 遂寧州,下。唐遂州,又改遂寧郡。宋為遂寧府。元初因之。至元十九年,並遂寧、青石二縣入小溪,長江入蓬溪,後復改為州。領二縣:



 小溪,下。蓬溪。下。



 綿州,下。唐更改不常。元初隸成都路。元至元二十年,並魏城入本州,改隸潼川路。領二縣:



 彰明,下。羅江。下。



 永寧路。下。闕。領州一。



 筠連州。下。闕。至元十七年,樞密院言:「四川行省參政行諸蠻夷部宣慰司昝順言,先是奉旨以高州,筠連州騰川縣隸安撫郭漢傑立站,今漢傑已並蠻洞五十六。有旨昝順所陳,卿等與中書議,臣等以為宜遣使行視之。」帝曰:「此五十六洞如舊隸高州、筠連,則與郭漢傑立站,否則還之昝順。」領一縣:



 騰川。下。



 四川南道宣慰司至元十六年立。



 重慶路,上。唐渝州。宋更名恭州,又升重慶府。元至元十六年,立重慶路總管府。二十一年,升為上路,割忠、涪二州為屬郡。二十二年,又割瀘、合來屬,省壁山入巴縣,廢南平軍入南川縣為屬邑,置錄事司。戶二萬二千三百九十五,口九萬三千五百三十五。至元二十七年數。領司一、縣三、州四。州領十縣。本路三堆、中嶆、趙市等處屯田四百二十頃。



 錄事司。



 縣三



 巴縣,下。倚郭。江津,下。至元十六年,賜四川行省參政昝順田民百八十戶於江津縣。南川。下。



 州四



 瀘州,下。唐改瀘川郡為瀘州。宋為瀘川軍。元至元二十年,並瀘川縣入焉。二十二年,隸重慶路。領三縣:



 江安,下。納溪,下。合江。下。



 忠州,下。唐改為南賓郡,又為忠州。宋升咸淳府。元仍為忠州。領三縣:



 臨江,下。南賓,下。豐都。下。



 合州,下。唐為合州,又改巴川郡,又仍為合州。宋因之。元至元十五年,宋安撫使王立以城降。二十年,為散郡,並錄事司、赤水入石照縣。二十二年,改為州,隸重慶路。領三縣:



 銅梁,下。元初並巴川入焉。定遠,下。本宋地,名女菁平。元至元四年,便宜都總帥部兵創為武勝軍,後為定遠州。二十四年,降為縣。石照。下。



 涪州,下。唐改為涪陵郡,又改涪州。宋因之。元至元二十年,並涪陵、樂溫二縣入焉。領一縣:



 武龍。下。



 紹慶府,下。唐黔州,又黔中郡。宋升為紹慶府。元至元二十年,仍置府。戶三千九百四十四,口一萬五千一百八十九。至元二十七年數。領縣二:



 彭水,下。黔江。下。



 懷德府。領州四。闕。



 來寧州,下。柔遠州,下。酉陽州,下。服州。下。皆闕。



 夔路,下。唐初為信州,又為夔州,又為雲安郡,又仍為夔州。宋升為帥府。元至元十五年,立夔州路總管府,以施、雲安、萬、大寧四州隸焉。二十二年,又以開、達、梁山三州來屬。戶二萬二十四,口九萬九千五百九十八。至元二十七年數。領司一、縣二、州七。州領五縣。本路屯田五十六頃。



 錄事司。



 縣二



 奉節,下。巫山。下。



 州七



 施州,下。唐改清江郡,又改清化郡,又復為施州。宋因之。舊領清江、建始二縣。元至元二十二年,並清江入州。領一縣:



 建始。下。



 達州,下。唐為通州,又改通川郡,又仍為通州。宋更名達州。元至元十五年,隸四川東道宣慰司。二十二年,改隸夔路。領二縣:



 通川,下。新寧。下。



 梁山州,下。本梁山縣,宋升梁山軍。元至元二十年,升為州。領一縣:



 梁山。下。



 萬州,下。唐改浦州為萬州,又改南浦郡。宋為浦州,元至元二十年,以南浦為萬州。領一縣:



 武寧。



 雲陽州,下。唐雲安監。宋置安義縣,後復為監。元至元十五年,立雲安軍。二十年,升雲陽州,並雲陽縣入焉。



 大寧州,下。舊大昌縣,宋置監。元至元二十年,升為州,並大昌縣入焉。



 開州,下。唐改為盛山郡,又復為開州。宋及元皆因之。



 敘南等處蠻夷宣撫司



 敘州路,古僰國,唐戎州。貞觀初徙治僰道,在蜀江之西三江口。宋升為上州,屬東川路,後易名敘州,咸淳中城登高山為治所。元至元十二年,郭漢傑挈城歸附。十三年,立安撫司。未幾,毀山城,復徙治三江口,罷安撫司,立敘州。十八年,復升為路,隸諸部蠻夷宣撫司。領縣四、州二。



 縣四



 宜賓,下。慶符,下。南溪,下。宣化。下。元貞二年,於本縣置萬戶府,領軍屯田四十餘頃。



 州二



 富順州,下。唐富義縣。宋富義監,後改富順縣。元至元十二年,改立富順監安撫司。二十年,罷安撫司,升富順州。



 高州,下。古夜郎之屬境,鄰烏蠻,與長寧軍地相接,均為西南羌族,前代以為化外,置而不論。唐開拓邊地,於本部立高州。宋設長寧軍,十州族姓俱效順。元至元十五年,雲南行省遣官招諭內附。十七年,知州郭安復行州事,蠻人散居村囤,無縣邑鄉鎮。



 馬湖路,下。古牂牁屬地,漢、唐以下名馬湖部。宋時蠻主屯湖內。元至元十三年內附後,立總管府,遷於夷部溪口,瀕馬湖之南岸創府治。其民散居山箐,無縣邑鄉鎮。領軍一、州一。初,馬湖蠻來朝,嘗以獨本蔥為獻,由是歲至,郡縣疲於遞送,元貞二年敕罷之。



 軍一



 長寧軍,唐置長寧等羈縻十四州、五十六縣,並隸瀘州都督府。宋以長寧地當沖要,升為長寧軍,立安寧縣。元至元十二年,郡守黃立挈城效順。二十二年,設錄事司,後與安寧縣俱省入本軍。



 州一



 戎州,下。本夜郎國西南蠻種,號大壩都掌,分族十有九,前代以化外,置而弗論。唐武后時,恢拓蠻徼,設十四州、五團、二十九縣,於本部置晏州。元至元十三年,以昝順為蠻夷部宣撫司,遣官招諭。十七年,本部官得蘭紐來見,授以大壩都總管。二十二年,升為戎州。叛服不常,州治在箐前。所領俱村囤,無縣邑鄉鎮。



 上羅計長官司,領蠻地羅計、羅星,乃古夜郎境,為西南種族,前代置之化外。宋設長寧軍,十州族姓俱效順,各命之官。其後分姓他居,遂有上、下羅計之分,蓋亦如唐羈縻之,以為西蜀後戶屏蔽。元至元十三年,蠻夷部宣撫昝順引本部夷酋得賴阿當歸順。十五年,授得賴阿當千戶。十八年,黎州同知李奇以武恩將軍來充羅星長官。二十二年,夷人叛,誘訹上羅星夷,行樞密院討平之。其民人散居村箐,無縣邑鄉鎮。



 下羅計長官司,領蠻地。其境近烏蠻,與敘州、長寧軍相接,均為西南夷族,與上羅計同。至元十二年,長寧知軍率先內附。十三年,昝順引本部夷酋得顏個詣行樞密院降,奏充下羅計蠻夷千戶。二十二年,諸蠻皆叛,惟本部無異志。



 四十六囤蠻夷千戶所,領豕蛾夷地,在慶符向南抵定川,古夜郎之屬,唐羈縻定州之支江縣也。至元十三年收附,於慶符縣僑置千戶所,領四十六囤:



 黃水口上下落骨,山落牟許滿吳,麼落財,麼落賢,騰息奴,屯莫面,落搔,麼落梅,麼得幸,上落松,麼得會,麼得惡,落魂,落昧下村,落島,麼得享,落燕,落得慮,麼得了,麼騰斛,許宿,麼九色,落搔屯右,麼得晏,落能,山落寡,水落寡,落得擂,麼得具,麼得淵,騰日彯,落昧上村,賴扇,許焰,騰郎,周頭,賣落炎,落女,愛答落,愛答速,麼得奸,阿郎頭,下得辛,上得辛,愛得婁,落鷗。



 諸部蠻夷:



 秦加大散等洞。以下各設蠻夷官。



 敘崖冒硃等洞。



 隴堤紂皮等洞。



 石耶洞。



 散毛洞。



 彭家洞。



 黑土石等處。



 市備洞。



 樂化兀都剌布白享羅等處。



 洪望冊德等族。



 大江九姓羅氏。



 水西。



 鹿朝。



 阿永蠻部。至元二十一年,酋長阿泥入覲,自言阿永鄰境烏蒙等蠻悉隸皇太子



 位,願依例附屬。詔從其請,以阿永蠻隸宮府。



 師壁洞安撫司。



 永順等處軍民安撫司。



 阿者洞。以下各設蠻夷官。



 謝甲洞。



 上安下壩。



 阿渠洞。



 下役洞。



 驢虛洞。



 錢滿等處。



 水洞下曲等寨。



 必藏等處。



 酌宜等處。



 雍邦等寨。



 崖筍等寨。



 冒硃洞。



 麻峽柘歌等寨。



 新附嵬羅金井。



 沙溪等處。



 宙窄洞。



 新容米洞。



 甘肅等處行中書省,為路七、州二,屬州五。本省馬站六處。



 河西隴北道肅政廉訪司



 甘州路,上。唐為甘州,又為張掖郡,宋初為西夏所據,改鎮夷郡,又立宣化府。元初仍稱甘州。至元元年,置甘肅路總管府。八年,改甘州路總管府。十八年,立行中書省,以控制河西諸郡。戶一千五百五十,口二萬三千九百八十七。至元二十七年數。本路黑山、滿峪、泉水渠、鴨子翅等處屯田,計一千一百六十餘頃。



 永昌路,下。唐涼州。宋初為西涼府,景德中陷入西夏。元初仍為西涼府。至元十五年,以永昌王宮殿所在,立永昌路,降西涼府為州隸焉。



 西涼州。下。



 肅州路,下。唐為肅州,又為酒泉郡。宋初為西夏所據。元太祖二十一年,西征,攻肅州下之。世祖至元七年,置肅州路總管府。戶一千二百六十二,口八千六百七十九。至元二十七年數。



 沙州路,下。唐為沙州,又為敦煌郡。宋仍為沙州,景祐初,西夏陷瓜、沙、肅三州,盡得河西故地。金因之。元太祖二十二年,破其城以隸八都大王。至元十四年,復立州。十七年,升為沙州路總管府,瓜州隸焉。沙州去肅州千五百里,內附貧民欲乞糧沙州,必須白之肅州,然後給與,朝廷以其不便,故升沙州為路。



 瓜州,下。唐改為晉昌郡,復為瓜州。宋初陷於西夏。夏亡,州廢。元至元十四年復立。二十八年徙居民於肅州,但名存而已。



 亦集乃路,下。在甘州北一千五百里,城東北有大澤,西北俱接沙磧,乃漢之西海郡居延故城,夏國嘗立威福軍。元太祖二十一年內附。至元二十三年,立總管府。二十三年,亦集乃總管忽都魯言:「所部有田可以耕作,乞以新軍二百人鑿合即渠于亦集乃地,並以傍近民西僧餘戶助其力。」從之。計屯田九十餘頃。



 寧夏府路,下。唐屬靈州。宋初廢為鎮,領番部。自唐末有拓拔思恭者鎮夏州,世有銀、夏、綏、宥、靜五州之地。宋天禧間,傳至其孫德明,城懷遠鎮為興州以居,後升興慶府,又改中興府。元至元二十五年,置寧夏路總管府。至元八年,立西夏中興等路行尚書省。元貞元年,革寧夏路行中書省,並其事於甘肅行省。



 領州三。本路棗園、納憐站等處屯田一千八百頃。



 靈州,下。唐為靈州,又為靈武郡。宋初陷於夏國,改為翔慶軍。



 鳴沙州,下。隋置環州,立鳴沙縣。唐革州以縣隸靈州。宋沒於夏國,仍舊名。元初立鳴沙州。屯田四百四十餘頃。



 應理州,下。與蘭州接境,東阻大河,西據沙山。考之圖志,乃唐靈武郡地。其州城未詳建立之始,元初仍立州。



 山丹州,下。唐為刪丹縣,隸甘州。宋初為夏國所有,置甘肅軍。元初為阿只吉大王分地。至元六年,行山丹城事,刪訛為山。二十二年,升為州,隸甘肅行省。



 西寧州,下。唐置鄯州,理湟水縣,上元間沒於土蕃,號青唐城。宋改為西寧州。元初為章吉駙馬分地。至元二十三年,立西寧州等處拘榷課程所。二十四年,封章吉為寧濮郡王,以鎮其地。



 兀剌海路。闕。太祖四年,由黑水城北兀剌海西關口入河西,獲西夏將高令公,克兀剌海城。



\end{pinyinscope}