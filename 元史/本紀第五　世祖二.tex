\article{本紀第五 世祖二}

\begin{pinyinscope}

 三年春正月癸亥,修宣聖廟成。庚午,罷高麗互市。諸王塔察兒請置鐵冶,從之;請立互市,不從。忽剌忽兒所部民饑節變化。戰國末鄒衍開始把朝代更替和陰陽消息、五行生成、,罷上供羊。命銀冶戶七百、河南屯田戶百四十,賦稅輸之州縣。命匠戶為軍者仍為軍,其軍官當考第富貧,存恤無力者。耶律鑄詣北京餉諸王軍,仍遣宣撫使柴禎等增價糴米三萬石益之。賜高麗國歷。辛未,禁諸道戍兵及勢家縱畜牧犯桑棗禾稼者。癸酉,以軍興人民勞苦,敕停公私逋負毋徵。癸未,賜廣寧王爪都駝鈕金鍍銀印,及諸王合必赤行軍印。宋制置使賈似道以書誘總管張元等,李璮獲其書上之。丙戌,命江漢大都督史權、亳州萬戶張弘彥將兵八千赴燕。備宮懸鐘磬、樂舞、籥翟,凡用三百六十二人。高麗遣使奉表來謝,優詔答之。李璮質子彥簡逃歸。



 二月丁亥朔,元籍軍竄名為民者,命有司還正之。括諸道逃亡軍。己丑,李璮反,以漣、海三城獻於宋,盡殺蒙古戍軍,引麾下趨益都。前宣撫副使王磐脫身走至濟南,驛召磐,令姚樞問計,磐對:「豎子狂妄,即成擒耳。」帝然之。庚寅,宋兵攻新蔡。辛卯,始定中外官俸,命大司農姚樞講定條格。甲午,李亶入益都,發府庫犒其將校。乙未,詔諸道以今歲民賦市馬。丙申,郭守敬造寶山漏成,徙至燕京。以興、松、雲三州隸上都。辛丑,李璮遣騎寇蒲臺。癸卯,詔發兵討之。以趙璧為平章政事。修深、冀、南宮、棗強四城。甲辰,發諸蒙古、漢軍討李璮,命水軍萬戶解成、張榮實、大名萬戶王文干及萬戶嚴忠範會東平,濟南萬戶張宏、歸德萬戶邸浹、武衛軍砲手元帥薛軍勝等會濱棣,詔濟南路軍民萬戶張宏、濱棣路安撫使韓世安,各修城塹,盡發管內民為兵以備。召張柔及其子弘範率兵二千詣京師。丙午,命諸王合必赤總督諸軍,以不只愛不干及趙璧行中書省事於山東,宋子貞參議行中書省事,以董源、高逸民為左右司郎中,許便宜從事。真定、順天、河間、平灤、大名、邢州、河南諸路兵皆會濟南。以中書左丞闊闊、尚書怯烈門、宣撫游顯行宣慰司於大名,洺滋、懷孟、彰德、衛輝、河南東西兩路皆隸焉。己酉,王文統坐與李璮同謀伏誅,仍詔諭中外。王演等以妖言誅。辛亥,敕元帥阿海分兵戍平灤、海口及東京、廣寧、懿州,以餘兵詣京師。詔諸道括逃軍還屯田,嚴其禁。壬子,李璮據濟南。癸丑,詔大名、洺滋、彰德、衛輝、懷孟、河南、真定、邢州、順天、河間、平灤諸路皆籍兵守城。宋兵攻滕州。丙辰,詔拔都抹臺將息州戍兵詣濟南,移其民於蔡州,東平萬戶嚴忠範留兵戍宿州及蘄縣,以餘兵自隨。



 三月戊午,有旨:「非中書省文移及兵民官申省者,不許入遞。」己未,括木速蠻、畏吾兒、也裏可溫、答失蠻等戶丁為兵。庚申,括北京鷹坊等戶丁為兵,蠲其賦,令趙炳將之。辛酉,宗拔突言河南有自願從軍者,命即令將之。遣鄭鼎、贍思丁、答裡帶、三島行宣慰司事於平陽、太原。簽見任民官及捕鷹坊、人匠等軍。徙弘州錦工繡女於京師。敕河東兩路元括金州兵付鄭鼎將之。詔以平章政事祃祃、廉希憲,參政商挺,斷事官麥肖,行中書省於陜西、四川。獲私商南界者四十餘人,命釋之。敕燕京至濟南置海青驛凡八所。壬申,命戶部尚書劉肅專職鈔法,平章政事賽典赤兼領之。以撒吉思、柴楨行宣慰司事於北京。免今歲絲銀,止輸田租。癸酉,命史樞、阿術各將兵赴濟南。遇李璮軍,邀擊,大破之,斬首四千,璮退保濟南。乙亥,宋將夏貴攻符離。戊寅,萬戶韓世安率鎮撫馬興、千戶張濟民,大破李璮兵於高苑,獲其權府傅珪,賜濟民、興金符。詔以李亶兵敗諭諸路。禁民間私藏軍器。壬午,始以畏吾字書給驛璽書。免西京今年絲銀稅。甲申,免高麗酒課。乙酉,宋夏貴攻蘄縣。諭諸路管民官,毋令軍馬、使臣入州城、村居、鎮市,擾及良民。



 夏四月丙戌朔,大軍樹柵鑿塹,圍璮於濟南。丁亥,詔博興、高苑等處軍民嘗為李璮脅從者,並釋其罪。庚寅,命怯烈門、安撫張耕分邢州戶隸兩答剌罕。辛卯,修河中禹廟,賜名建極宮。壬辰,以大梁府渠州路軍民總帥蒲元圭為東夔路經略使。丙申,宋華路分、湯太尉攻徐、邳二州。詔分張柔軍千人還戍亳州。庚子,江漢大都督史權以趙百戶潔眾逃歸,斬之。詔:「自今部曲犯重罪,鞫問得實,必先奏聞,然後置諸法。」詔安輯徐、邳民,禁征戍軍士及勢官,毋縱畜牧傷其禾稼桑棗。以米千石、牛三百給西京蒙古戶。癸卯,宋兵攻亳州。甲辰,命行中書省、宣慰司、諸路達魯花赤、管民官,勸誘百姓,開墾田土,種植桑棗,不得擅興不急之役,妨奪農時。乙巳,以北京、廣寧、豪、懿州軍興勞弊,免今歲稅賦。命諸路詳讞冤獄。詔河東兩路並平陽、太原路達魯花赤及兵民官,撫安軍民,各安生業,毋失歲計。丁未,李璮遣柴牛兒招諭部民盧廣,廣縛以獻,殺之;以廣權威州軍判,兼捕盜官。戊申,賜諸王也相哥金印。庚戌,賜諸王合必赤金銀海青符各二。免松州、興州、望雲州新舊差賦,以望雲、松山、興州課程隸開平府。壬子,敕非軍情毋行望雲驛。乙卯,河南路王豁子、張無僧、杜信等謀為不軌,並伏誅。詔右丞相史天澤專征,諸將皆受節度。



 五月戊午,蘄縣陷,權萬戶李義、千戶張好古死之。庚申,築環城圍濟南,璮不復得出。詔撒吉思安撫益都路百姓,各務農功,仍禁蒙古、漢軍剽掠。癸亥,史權妄奏徐、邳總管李杲哥完復邳州城,詔由杲哥以下並原其罪。時宋將夏貴攻邳州,杲哥出降,貴既去,杲哥自陳能保全州城,史權以聞,故有是命。甲子,宋兵攻利津縣。蠲濱棣今歲田租之半,東平蠲十之三。自燕至開平立牛驛,給鈔市車牛。戊辰,以左丞相忽魯不花兼中書省都斷事官,賜虎符。真定、順天、邢州蝗。以平章政事賽典赤兼領工部及諸路工作,以孟烈所獻蹶張弩藏於中都。丙子,縉山至望雲立海青驛。丁丑,李杲哥等伏誅,命史天澤選考徐、邳總管。甲申,真定路不眼裡海牙擅殺造偽鈔者三人,詔詰其違制之罪。西京、宣德、威寧、龍門霜,順天、平陽、河南、真定雨雹,東平、濱棣旱。詔核實逃戶、輸納絲銀稅租戶,口增者賞之,隱匿者罪之,逃民茍免差稅重加之罪。大司農姚樞辭赴省議事,帝勉留之,命樞與左三部尚書劉肅依前商議中書省事。



 六月乙酉朔,宋兵攻滄州、雅州、瀘山,民既降復叛,命誅其首亂者七人,餘令安業。割遼河以東隸開元路。戊子,濱棣安撫使韓世安敗宋兵於濱州丁河口。己丑,遣塔察兒帥兵擊宋軍,仍安諭瀕海軍民。乙未,禁女直侵軼高麗國民,其使臣往還,官為護送。送婆娑府屯田軍移駐鴨綠江之西,以防海道。丙申,高麗國王王禃遣使來貢。壬寅,陜西行省言西京、宣德、太原匠軍困乏,乞以民代之。有旨:「軍籍已定,不宜動搖,宜令貧富相資,果甚貧者,令休息一歲。」癸卯,太原總管李毅奴哥、達魯花赤戴曲薛等領李璮偽檄,傳行旁郡,事覺誅之。敕寧武軍歲輸所產鐵。河西民及諸王忽撒吉所部軍士乏食,給鈔賑之。壬子,申嚴軍官及兵伍擾民之禁。癸丑,立小峪、蘆子、寧武軍、赤泥泉鐵冶四所。東平嚴忠濟向為民貸錢輸賦四十三萬七千四百錠,借用課程、鈔本、鹽課銀萬五千餘兩,詔勿征。



 秋七月戊午,復蒙古軍站戶差賦,農民包銀徵其半,俘戶止令輸絲,民當輸賦之月,毋徵私債。敕私市金銀應支錢物,止以鈔為準。丙寅,賜夔州路行省楊大淵金符十、銀符十九,賞麾下將士;別給海青符二,事有急速,馳以上聞。立槍桿嶺驛,以便轉輸。癸酉,甘州饑,給銀以賑之。甲戌,李璮窮蹙,入大明湖,投水中不即死,獲之,並蒙古軍囊家伏誅,體解以徇。戊寅,以夔府行省劉整行中書省於成都、潼川兩路,仍賜銀萬兩,分給軍士之失業者。



 八月己丑,郭守敬請開玉泉水以通漕運,廣濟河渠司王允中請開邢、洺等處漳、滏、澧河、達泉以溉民田,並從之。甲午,博都歡等奏請以宣德州、德興府等處銀冶付其匠戶,歲取銀及石綠、丹粉輸官,從之。丙午,立諸路醫學教授。戊申,敕王鶚集廷臣商榷史事,鶚等乞以先朝事跡錄付史館。河間、平灤、廣寧、西京、宣德、北京隕霜害稼。



 九月戊午,亳州萬戶張弘略破宋兵於蘄縣,復宿、蘄二城。以侍衛親軍都指揮使董文炳兼山東路經略使,收集益都舊軍充武衛軍,戍南邊,詔益都行省大都督撒吉思與董文炳會議兵民籍,每十戶惟取其二充武衛軍;其海州、東海、漣水移入益都者,亦隸本衛。己未,罷霸州海青驛。安南國陳光昞遣使貢方物。壬戌,改邢州為順德府,立安撫司,洺、磁、威三州隸焉。聽太原民食小鹽,歲輸銀七千五百兩。己巳,以馬月合乃餉軍功,授禮部尚書,賜金符。壬申,授安南國王陳光昞及達魯花赤訥剌丁虎符。敕濟南官吏,凡軍民公私逋負,權閣毋徵。癸酉,都元帥闊闊帶卒於軍,以其兄阿術代之,授虎符,將南邊蒙古、漢軍。閏月甲申朔,沙、肅二州乏食,給米、鈔賑之。丁亥,立古北口驛。己丑,濟南民饑,免其賦稅。免諸路軍戶他徭。庚寅,敕京師順州至開平置六驛。辛卯,嚴忠範奏請補東平路廟學太常樂工,從之。敕武衛軍及黑軍會於京師。庚子,中翼千戶九住破宋兵於虎腦山。庚戌,發粟三十萬石賑濟南饑民。



 冬十月丙辰,放金州所屯軍士二千人及大名、河南新簽防城軍為民。庚申,分益都軍民為二,董文炳領軍,撒吉思治民。禁諸王、使臣、師旅敢有恃勢擾民者,所在執以聞。詔以李璮所掠民馬還其主。以郝經、劉人傑使宋未還,廩其家。中書省奏與宋互市,庶止私商,及復逋民之陷於宋者,且覘漣、海二州,不允。以劉仁傑不附李璮,擢益都路總管,仍以金帛賜之。壬戌,授益都行中書省都督府所統州郡官金符十七、銀符十一。乙丑,詔禁京畿畋獵。丙寅,分東西兩川都元帥府為二,以帖的及劉整等為都元帥及左右副都元帥。詔責高麗欺慢之罪。又詔賜高麗王禃歷。以戰功賞渠州達魯花赤王章等金五十兩、銀一千五百五十兩。賞閬、蓬等路都元帥合州戰功銀五千兩。丁卯,詔鳳翔府屯田軍隸兵籍,仍屯田鳳翔。放刁國器所簽平陽軍九百一十五人為民。閬、蓬、廣安、順慶、夔府等路都元帥欽察戍青居山,請益兵,詔陜西行省及鞏昌總帥汪惟正以兵益之。戊辰,楊大淵乞於利州大安軍以鹽易軍糧,從之。庚午,敕鞏昌總帥汪惟正將戍青居軍還,屯田利州。乙亥,分中書左右部。丁丑,敕宿州百戶王達等所擒宋王用、夏珍等八人赴京師。命百家奴所將質子軍入侍。戊寅,命不里剌所統固安、平灤質子軍自益都徙還故地。詔益都府路官吏軍民為李璮脅從者,並赦其罪。敕萬戶嚴忠範修復宿州、蘄縣,萬戶忽都虎、懷都、何總管修完邳州城郭。



 十一月乙酉,太白犯鉤鈐。丁亥,敕聖安寺作佛頂金輪會,長春宮設金籙周天醮。辛丑,日有背氣重暈三珥。敕濟南人民為李璮裨校掠取財物者,詣都督撒吉思所訟之。真定民郝興仇殺馬忠,忠子榮受興銀,令興代其軍役。中書省以榮納賂忘仇,無人子之道,杖之,沒其銀。事聞,詔論如法。有司失出之罪,俾中書省議之。三義沽灶戶經宋兵焚掠,免今年租賦。汰少府監工匠,存其良者千二百戶。遣官審理陜西重刑。敕河西民徙居應州,其不能自贍者百六十戶,給牛具及粟麥種,仍賜布,人二匹。乙巳,詔都元帥阿術分兵三千人同阿鮮不花、懷都兵馬,復立宿州、蘄縣、邳州。有旨諭史天澤:「朕或乘怒欲有所誅殺,卿等宜遲留一二日,覆奏行之。」丙午,詔特徵人員,宜令乘傳。戊申,升撫州為隆興府,以昔剌斡脫為總管,割宣德之懷安、天成及威寧、高原隸焉。



 十二月甲寅,封皇子真金為燕王,守中書令。丙辰,敕諸王塔察兒等所部獵戶止收包銀,其絲稅輸之有司。立河南、山東統軍司,以塔剌渾火兒赤為河南路統軍使,盧升副之,東距亳州,西至均州,諸萬戶隸焉;茶不花為山東路統軍使,武秀副之,西自宿州,東至寧海州,諸萬戶隸焉。罷各路急遞鋪。丁巳,立十路宣慰司,以真定路達魯花赤趙瑨等為之。己未,犯罪應死者五十三人,詔重加詳讞。辛酉,詔給懷州新民耕牛二百,俾種水田。立諸路轉運司,以燕京路監榷官曹澤等為之使。癸亥,享太廟。詔:「各路總管兼萬戶者,止理民事,軍政勿預。其州縣官兼千戶、百戶者,仍其舊。」乙丑,復立息州城以安其民。召真定、順德等路宣慰使王磐乘傳赴京師。丙寅,申嚴屠殺牛馬之禁。己巳,詔:「諸路管民總管子弟,有分管州、府、司、縣及鷹坊、人匠諸色事務者,罷之。」壬申,遣使收輯諸路軍民官海青牌及驛券。戊寅,詔:「諸路管民官理民事,管軍官掌兵戎,各有所司,不相統攝。」作佛事於昊天寺七晝夜,賜銀萬五千兩。割北京、興州隸開平府。建行宮於隆興路。升太原臨泉縣為臨州;降寧陵為下縣,仍隸歸德。賜諸王金、銀、幣、帛如歲例。是歲,天下戶一百四十七萬六千一百四十六,斷死罪六十六人。



 四年春正月乙酉,禁蒙古軍馬擾民。宋賈似道遣楊琳齎空名告身及蠟書、金幣,誘大獲山楊大淵南歸,大淵部將執琳,詔誅之。以宋忽兒、滅里及沙只回回鷹坊等兵戍商州、藍田諸隘。軍民官各從統軍司及宣慰司選舉。岳天輔乞復立息州,不允。丙戌,以姚樞為中書左丞。改諸路監榷課稅所為轉運司。甲午,給公主拜忽符印,其所屬設達魯花赤。給鈔賑益都路貧民之無牛者。立十路奧魯總管。丁酉,益都路行省大都督撒吉思上李璮所傷漣水軍民及陷宋蒙古、女直、探馬赤軍數,男女凡七千九百二十二人。癸卯,領部阿合馬請興河南等處鐵冶及設東平等路巡禁私鹽軍,從之。召商挺、趙良弼赴闕。乙巳,敕李平陽以所部西川出征軍士戍青居山,其各翼軍在青居山者悉還成都。詔陜西行省塔剌海等收恤離散軍戶。詔:「以諸路漢軍奧魯毋隸各萬戶管領,其科征差稅,山東、河南隸統軍司,東西兩川隸征東元帥府,陜西隸行戶部。凡奧魯官內有各萬戶弟、男及私人,皆罷之。」敕總帥汪忠臣、都元帥帖的及劉整等益兵付都元帥欽察,戍青居山,仍以解州鹽課給軍糧。丙午,詔諸翼萬戶簡精兵四千充武衛軍。罷古北口新置驛。增萬戶府監戰一員、參議一員。以馬合麻所俘濟南老僧口之民文面為奴者,付元籍為民。汪忠臣、史權系宋諜者六人至京師,有旨釋之。辛亥,申禁民家兵器及蒙古軍擾民者。陵州達魯花赤蒙哥戰死濟南,以其子忙兀帶襲職。召雲頂山侍郎張威赴闕。



 二月壬子朔,命河東宣慰司市馬百二十九匹,賜諸王八剌軍士之無馬者。甲寅,詔諸路官員子弟入質。以高麗不答詔書,詰其使者。以民杜了翁先朝舊功,復其家。庚申,賞萬戶怯來所部將士討李璮有功者銀二千七百五十兩。甲子,車駕幸開平。以王德素充國信使,劉公諒副之,使於宋,致書宋主,詰其稽留郝經之故。詔:「諸路置局造軍器,私造者處死;民間所有,不輸官者,與私造同。」



 三月戊子,沂州胡節使、範同知陷於宋,命存恤其家。或言其嘗為宋兵向導,乃分其妻孥資產,賜有功將士。辛卯,敕撒吉思招集益都逃民。命董文炳以所獲宋諜及俘八十一人赴隆興府。聽諸路獵戶及捕盜巡鹽者執弓矢。壬辰,遣扎馬剌丁和糴東京。己亥,諸路包銀以鈔輸納,其絲料入本色,非產絲之地,亦聽以鈔輸入。凡當差戶包銀鈔四兩,每十戶輸絲十四斤,漏籍老幼鈔三兩、絲一斤。庚子,亦黑迭兒丁請修瓊華島,不從。壬寅,關東蒙古、漢軍官未經訓敕者,令各乘傳赴開平。癸卯,初建太廟。乙巳,賜迭怯那延等銀七千九十兩。命北京元帥阿海發漢軍二千人赴開平。己酉,高麗國王王禃遣其臣硃英亮入貢,上表謝恩。復立宿州。



 夏四月庚戌朔,以漏籍戶一萬一千八百、附籍戶四千三百於各處起冶,歲課鐵四百八十萬七千斤。癸丑,選益都兵千人充武衛軍。甲寅,償河西阿沙賑贍所部貧民銀三千七百兩。己未,以完顏端田宅賜益都千戶傅國忠。國忠父天祐為端所殺,故命以其田宅賜之。宣德至開平置驛。罷開元路宣慰司。丙寅,西京武州隕霜殺稼。戊寅,召竇默、許衡乘驛赴開平。諸王阿只吉所部貧民遠徙者,賜以馬牛車幣。以東平為軍行蹂踐,賑給之。改滄清深鹽提領所為轉運司。王鶚請延訪太祖事跡付史館。



 五月癸未,詔北京運米五千石赴開平,其車牛之費並從官給。乙酉,初立樞密院,以皇子燕王守中書令,兼判樞密院事。戊子,升開平府為上都,其達魯花赤兀良吉為上都路達魯花赤,總管董銓為上都路總管兼開平府尹。辛卯,詔立燕京平準庫,以均平物價,通利鈔法。乙未,敕商州民就戍本州,毋禁弓矢。丙申,立上都馬、步驛。丁酉,以元帥楊大淵、張大悅復神山有功,降詔獎諭。戊戌,以禮部尚書馬月合乃兼領潁州、光化互市,及領已括戶三千,興煽鐵冶,歲輸鐵一百三萬七千斤,就鑄農器二十萬事,易粟四萬石輸官,河南隨處城邑市鐵之家,令仍舊鼓鑄。庚子,河南路總管劉克興矯制括戶,罷其職,籍家資之半。升上都路望雲縣為雲州,松山縣為松州。賞前討渾都海戰功,撒裡都、闊闊出等鈔二千一百七十四錠、幣帛一千四百二十匹。



 六月壬子,河間、益都、燕京、真定、東平諸路蝗。乙卯,以管民官兼統懷孟等軍俺撒戰歿汴梁,命其子忙兀帶為萬戶,佩金符。戊午,賜線真田戶六百。己未,賜高麗國王王禃羊五百。癸酉,賜拜忽公主所部鈔千錠。立上都惠民藥局。建帝堯廟於平陽,仍賜田十五頃。以線真為中書右丞相,塔察兒為中書左丞相。



 秋七月癸未,詔諸投下毋擅勾攝燕京路州縣官吏。乙酉,禁野狐嶺行營民,毋入南、北口縱畜牧,損踐桑稼。給公主拜忽銀五萬兩,合剌合納銀千兩。乙未,以故東平權萬戶呂義死王事,賜謚貞節。戊戌,詔弛河南沿邊軍器之禁。升燕京屬縣安次為東安州,固安為固安州。河南統軍司言:「屯田民為保甲丁壯射生軍,凡三千四百人,分戍沿邊州郡,乞蠲他徭。」從之。庚子,詔賜諸王爪都牛馬價銀六萬三千一百兩。壬寅,詔禁益都路探馬赤擾民。以成都經略司隸西川行院。禁蒙古、漢軍諸人煎、販私鹽。詔山東經略司徙膠、萊、莒、密之民及灶戶居內地。中書省臣以妨煮鹽為言,遂令統軍司完復邊戍,居民灶戶毋徙。詔阿術戒蒙古軍,不得以民田為牧地。燕京、河間、開平、隆興四路屬縣雨雹害稼。



 八月戊申朔,詔霍木海總管諸路驛,佩金符。辛亥,置元帥府於大理。詔東平、大名、河南宣慰司市馬千五百五十匹,給阿術等軍。升宣德州為宣德府,隸上都。以淄、萊、登三州為總管府,治淄州。命昔撒昔總制鬼國、大理兩路。兵部郎中劉芳前使大理,至吐蕃遇害,命恤其家。壬子,命中書省給北京、西京轉運司車牛價鈔。彰德路及洺、磁二州旱,免彰德今歲田租之半,洺、磁十之六。冀州蒙古百戶阿昔等犯鹽禁,沒入馬百二十餘匹,以給軍士之無馬者。甲寅,命成都路運米萬石餉潼川。給鈔付劉整市牛屯田。分劉元禮等軍戍潼川,命按敦將之。丙辰,詔以成都路綿州隸潼川。戊午,以阿脫、商挺行樞密院於成都,凡成都、順慶、潼川都元帥府並聽節制。庚申,以史天倪前為武仙所殺,以武仙第賜其子楫。癸亥,敕京兆路給賜劉整第一區、田二十頃。以夢八剌所部貧乏,賜銀七千五百兩給之。甲子,以西涼經兵,居民困弊,給鈔賑之,仍免租賦三年。敕諸臣傳旨,有疑者須覆奏。丙寅,以諸王只必帖木兒部民困乏,賜銀二萬兩給之。壬申,復置急遞鋪。濱、棣二州蝗,真定路旱。詔西涼流民復業者,復其家三年。車駕至自上都。



 九月壬午,河南、大名兩道宣慰司所獲宋諜王立、張達、刁俊等十八人,遇赦釋免,給衣服遣還。乙酉,立漕運河渠司。己丑,賜諸王阿只吉所部種食、牛具。庚寅,諭高麗、上京等處毋重科斂民。招諭濟南、濱棣流民。遣使徵諸路賦稅錢帛。民間所賣布帛有疏薄狹短者,禁之。



 冬十月戊午,初置隆興路驛。



 十一月甲申,詔以歲不登,量減阿述、怯烈各軍行餉。東平、大名等路旱,量減今歲田租。丙戌,享於太廟,以合丹、塔察兒、王磐、張文謙行事。高麗國王王禃以免置驛、籍民等事,遣其臣韓就奉表來謝,賜中統五年歷並蜀錦一,仍命禃入朝。立禦衣、尚食二局。



 十二月丁未朔,以鳳翔屯軍、汪惟正青居等軍、刁國器平陽軍,令益都元帥欽察統之,戍虎嘯寨。甲戌,敕駙馬愛不花蒲萄戶依民例輸賦。也裏可溫、答失蠻、僧、道種田入租,貿易輸稅。丙子,賜諸王金、銀、幣、帛如歲例。是歲,天下戶一百五十七萬九千一百一十;賦,絲七十萬六千四百一斤,鈔四萬九千四百八十七錠;斷死罪七人。



 至元元年春正月丁丑朔,高麗國王王禃遣使奉表來賀。壬午,敕諸路宣慰司,非奉旨無輒入覲。以千戶張好古歿王事,命其弟好義、好禮並襲職為千戶。癸巳,以益都武衛軍千人屯田燕京,官給牛具。以鄧州保甲軍二千三百二十九戶隸統軍司。戊戌,楊大淵進花羅、紅邊絹各百五十段,優詔諭之。己亥,立諸路平準庫。癸卯,命諸王位下工匠已籍為民者,並徵差賦;儒、釋、道、也裏可溫、達失蠻等戶,舊免租稅,今並徵之;其蒙古、漢軍站戶所輸租減半。西北諸王率部民來歸。敕北京、西京宣慰司、隆興總管府和糴以備糧餉。築泠水河城,命千戶土虎等戍之。罷南邊互市。申嚴持軍器、販馬、越境私商之禁。



 二月辛亥,賀福等六人告平陽、太原漏籍戶,詔賞以官,廷臣以非材對,給鈔與之。敕選儒士編修國史,譯寫經書,起館舍,給俸以贍之。壬子,修瓊花島。發北京都元帥阿海所領軍疏雙塔漕渠。甲寅,以故亳州千戶邸閏陷於宋,命其子榮祖襲職。丙辰,罷陜西行戶部。丁卯,太陰犯南斗。癸酉,車駕幸上都。詔諸路總管史權等二十三人赴上都大朝會。弛邊城軍器之禁。



 三月庚辰,設周天醮於長春宮。己亥,命尚書宋子貞陳時事,子貞條具以聞,詔獎諭,命中書省議行之。辛丑,詔四川行院,命阿脫專掌軍政,其刑名錢穀商挺任之。立漕運司,以王光益為使。



 夏四月戊申,以彰德、洺磁路引漳、滏、洹水灌田,致御河淺澀,鹽運不通,塞分渠以復水勢。辛亥,太陰犯軒轅御女星。壬子,東平、太原、平陽旱,分遣西僧祈雨。乙卯,詔高麗國王王禃來朝上都,修世見之禮。辛酉,以四川茶、鹽、商、酒、竹課充軍糧。楊大淵以部將王仲得宋將昝萬壽書殺之,詔以其事未經鞫問,或墮宋人行間之計,豈宜輒施刑戮,詰責大淵,仍存恤仲家。御苑官南家帶請修駐蹕涼樓並廣牧地,詔涼樓俟農隙,牧地分給農之無田者。丁卯,追治李璮逆黨萬戶張邦直兄弟及姜鬱、李在等二十七人罪。戊辰,給新附戍軍糧餉。高麗國王王禃遣其臣金祿來貢。



 五月乙亥,詔遣唆脫顏、郭守敬行視西夏河渠,俾具圖來上。庚辰,敕劍州守將分軍守劍門,置驛於人頭山。丙戌,太陰犯房。丁亥,釋宋私商五十七人,給糧遣歸其國。己丑,以平陰縣尹馬欽發私粟六百石贍饑民,又給民粟種四百餘石,詔獎諭,特賜西錦五端以旌其義。乙未,初置四川急遞鋪。丙申,賜諸王欽察銀萬兩,濟其所部貧乏者。己亥,太陰犯昴。以中書右丞粘合南合為平章政事。邛部川六番安撫招討使都王明亞為鄰國建都所殺,敕其子伯佗襲職,賜金符。



 六月乙巳,召王鶚、姚樞赴上都。宋制置夏貴率兵欲攻虎嘯山,敕以萬戶石抹糺札剌一軍益欽察戍之。戊申,高麗國王王禃來朝。



 秋七月甲戌,彗星出輿鬼,昏見西北,貫上臺,掃紫微、文昌及北斗,旦見東北,凡四十餘日。以阿合馬言,益解州鹽課,均賦諸色僧道軍匠等戶,其太原小鹽,聽從民便。癸未,改新鳳州為徽州。以西番十八族部立安西州,行安撫司事。丁亥,諸王算吉所部營帳軍民被火,發粟賑之。庚寅,給諸王也速不花印。壬辰,特詔諭鞏昌路總帥汪惟正勞勉之,賜元寶交鈔三萬貫,仍戍青居。賜諸王玉龍答失印,仍以先朝獵戶賜之。丁酉,龍門禹廟成,命侍臣阿合脫因代祀。己亥,定用御寶制:凡宣命,一品、二品用玉,三品至五品用金。其文曰「皇帝行寶」者,即位時所鑄,惟用之詔誥;別鑄宣命金寶行之。庚子,阿里不哥自昔木土之敗,不復能軍,至是與諸王玉龍答失、阿速帶、昔裡給,其所謀臣不魯花、忽察、禿滿、阿里察、脫忽思等來歸。詔諸王皆太祖之裔,並釋不問,其謀臣不魯花等皆伏誅。



 八月壬寅朔,陜西行省臣上言:「川蜀戍兵軍需,請令奧魯官徵入官庫,移文於近戍官司,依數取之。宋新附民宜撥地土衣糧,給其牛種,仍禁邊將分匿人口。商州險要,乞增戍兵。陜西獵戶移獵商州。河西、鳳翔屯田軍遷戍興元。四川各翼軍,有地者徵其稅,給無田者糧。」皆從之。甲辰,詔秦蜀行省發銀二十五萬兩給沿邊歲用。乙巳,立山東諸路行中書省,以中書左丞相耶律鑄、參知政事張惠等行省事。詔新立條格:省並州縣,定官吏員數,分品從官職,給俸祿,頒公田,計月日以考殿最;均賦役,招流移;禁勿擅用官物,勿以官物進獻,勿借易官錢;勿擅科差役;凡軍馬不得停泊村坊,詞訟不得隔越陳訴;恤鰥寡,勸農桑,驗雨澤,平物價;具盜賦、囚徒起數,月申省部。又頒陜西四川、西夏中興、北京三處行中書條格。定立諸王使臣驛傳稅賦差發,不許擅招民戶,不得以銀與非投下人為斡脫,禁口傳敕旨及追呼省臣官屬。詔:「蒙古戶種田,有馬牛羊之家,其糧住支;無田者仍給之。」庚戌,命燕王署敕、諸王設僚屬及說書官。諸站戶限田四頃免稅,供驛馬及祗應;命各路總管府兼領其事。癸丑,命僧子聰同議樞密院事。詔子聰復其姓劉氏,易名秉忠,拜太保,參領中書省事。乙卯,詔改燕京為中都,其大興府仍舊。增都省參佐掾史月俸。丙辰,劉秉忠、王鶚、張文謙、商挺言,燕王既署相銜,宜於省中別置幕位,每月一再至,判署朝政。其說書官,皇子忙安以李磐為之,南木合以高道為之。丁巳,以改元大赦天下,詔曰:



 應天者惟以至誠,拯民者莫如實惠。朕以菲德,獲承慶基,內難未戡,外兵未戢,夫豈一日,於今五年。賴天地之畀矜,暨祖宗之垂裕,凡我同氣,會於上都。雖此日之小康,敢朕心之少肆。比者星芒示儆,雨澤愆常,皆闕政之所繇,顧斯民之何罪。宜布惟新之令,溥施在宥之仁。據不魯花、忽察、禿滿、阿里察、脫火思輩,構禍我家,照依太祖皇帝扎撒正典刑訖。可大赦天下,改中統五年為至元元年。於戲!否往泰來,迓續亨嘉之會;鼎新革故,正資輔弼之良。咨爾臣民,體予至意!



 戊午,給益都武衛軍千人冬衣。己未,鳳翔府龍泉寺僧超過等謀亂遇赦,沒其財,羈管京兆僧司;同謀蘇德,責令從軍自效。發萬戶石抹糺札剌所部千人赴商州屯田,亳州軍六百八人及河南府軍六十人助欽察戍青居。敕山東經略副使武秀選益都新軍千人充武衛軍,赴中都。城郯,以沂州監戰塔思、萬戶孟義所部兵戍之。太原路總管攸忙兀帶坐藏甲匿戶,罷職為民。



 九月壬申朔,立翰林國史院。以改元詔諭高麗國,並赦其境內。辛巳,車駕至自上都。庚寅,益都毛璋謀逆,二子及其黨崔成並伏誅,籍其家貲,賜行省撒吉思。



 冬十月壬寅朔,高麗國王王禃來朝。乙巳,禁上都畿內捕獵。庚戌,有事於太廟。壬子,恩州歷亭縣進嘉禾,一莖五穗。戊辰,改武衛軍為侍衛親軍。



 十一月丙子,詔宋人歸順及北人陷沒來歸者,皆月給糧食。辛巳,徵骨嵬。先是,吉裏迷內附,言其國東有骨嵬、亦里於兩部,歲來侵疆,故往征之。己丑,以至元二年歷日賜高麗國王王禃。禁登州、和州等處並女直人入高麗界剽掠。辛卯,召衛州太一五代度師李居壽赴闕。壬辰,罷領中書左右部,並入中書省。以領中書省左右部兼諸路都轉運使、知太府監事阿合馬為平章政事,領中書省左右部兼諸路都轉運使阿里為中書右丞。丁酉,太原路臨州進嘉禾二莖。以元帥按敦、劉整、劉元禮、欽察等將士獲功,賞賚有差。



 十二月乙巳,罷各投下達魯花赤,定中外百官儀從。丁未,敕遣宋諜者四人還其國。戊午,賞拔都軍人銀五十萬兩。甲子,太陰犯房。乙丑,以王鑒昔使大理沒於王事,其子天赦不能自存,優恤之。丁卯,敕鄧州沿邊增立茱萸、常平、建陵、季陽四堡。戊辰,命選善水者一人,沿黃河計水程達東勝可通漕運,馳驛以聞。庚午,詔罷樞密院斷事官及各路奧魯官,令總管府兼總押所。始罷諸侯世守,立遷轉法。是歲,真定、順天、洺、磁、順德、大名、東平、曹、濮州、泰安、高唐、濟州、博州、德州、濟南、濱、棣、淄、萊、河間大水。賜諸王金、銀、幣帛如歲例。戶一百五十八萬八千一百九十五,斷死罪七十三人。



\end{pinyinscope}