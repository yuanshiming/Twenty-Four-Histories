\article{本紀第七 世祖四}

\begin{pinyinscope}

 七年春正月辛丑朔,高麗國王王禃遣使來賀。丙午,耶律鑄、廉希憲罷。立尚書省,罷制國用使司。以平章政事忽都答兒為中書左丞相,國子祭酒許衡為中書左丞魏弁、陳仲、史鰌、墨翟、宋鈃、慎到、田駢、惠施、鄧析、,制國用使阿合馬平章尚書省事,同知制國用使司事張易同平章尚書省事,制國用使司副使張惠、簽制國用使司事李堯咨、麥術丁並參知尚書省事。己酉,太陰犯畢。敕諸投下官隸中書省。壬子,敕驛卷無印者不許乘傳。甲寅,高麗國王王禃遣使來言:「比奉詔臣已復位,今從七百人入覲。」詔令從四百人來,餘留之西京。詔高麗西京內屬,改東寧府,畫慈悲嶺為界。丁巳,以蒙哥都為安撫高麗使,佩虎符,率兵戍其西境。戊午,均、房州總管孫嗣擒宋統制硃興祖等。丙寅,賑兀魯吾民戶鈔。丁卯,定省、院、臺文移體式。



 二月辛未朔,以前中書右丞相伯顏為樞密副使。甲戌,築昭應宮於高梁河。丙子,帝御行宮,觀劉秉忠、孛羅、許衡及太常卿徐世隆所起朝儀,大悅,舉酒賜之。丁丑,以歲饑罷修築宮城役夫。甲申,置尚書省署。乙酉,立紙甲局。申嚴畜牧損壞禾稼桑果之禁。壬辰,立司農司,以參知政事張文謙為卿,設四道巡行勸農司。乙未,宋襄陽出步騎萬餘人、兵船百餘艘,來趣萬山堡,萬戶張弘範、千戶脫脫擊卻敗之,事聞,各賜金紋綾有差。高麗國王王禃來朝,求見皇子燕王,詔曰:「汝一國主也,見朕足矣。」禃請以子愖見,從之。詔諭禃曰:「汝內附在後,故班諸王下。我太祖時亦都護先附,即令齒諸王上,阿思蘭後附,故班其下,卿宜知之。」又詔令國王頭輦哥等舉軍入高麗舊京,以脫朵兒、焦天翼為其國達魯花赤,護送禃還國。仍下詔:「林衍廢立,罪不可赦。安慶公淐本非得已,在所寬宥。有能執送衍者,雖舊在其黨,亦必重增官秩。」世子愖奏乞隨朝及尚主,不許,命隨其父還國。



 三月庚子朔,日有食之。改河南等路及陜西五路西蜀四川、東京等路行中書省為行尚書省。尚書省臣言:「河西和糴,應僧人、豪官、富民一例行之。」制可。甲寅,車駕幸上都。丙辰,浚武清縣御河。丁巳,定醫官品從。戊午,益都、登、萊蝗旱,詔減其今年包銀之半。阿術與劉整言:「圍守襄陽,必當以教水軍、造戰艦為先務。」詔許之。教水軍七萬餘人,造戰艦五千艘。



 夏四月壬午,檀州隕黑霜二夕。設諸路蒙古字學教授。敕:「諸路達魯花赤子弟廕敘充散府諸州達魯花赤,其散府諸州子弟充諸縣達魯花赤,諸縣子弟充巡檢。」改御史臺典事為都事。癸未,定軍官等級,萬戶、總管、千戶、百戶、總把以軍士為差。己丑,省終南縣入盩厔,復真定贊皇縣、太原樂平縣。高麗行省遣使來言:「權臣林衍死,其子惟茂擅襲令公位,為尚書宋宗禮所殺。島中民皆出降,已遷之舊京。衍黨裴仲孫等復集餘眾,立禃庶族承化侯為王,竄入珍島。」



 五月辛丑,懷州河內縣大雨雹。癸卯,陜西簽省也速帶兒、嚴忠範與東西川統軍司率兵及宋兵戰於嘉定、重慶、釣魚山、馬湖江,皆敗之,拔三寨,擒都統牛宣,俘獲人民及馬牛戰艦無算。甲辰,威州汝鳳川番族八千戶內附,其酋長來朝,授宣命,賜金符。丁未,東京路饑,兼運糧造船勞役,免今年絲銀十之三。以同知樞密院事合答為平章政事。乙卯,復平灤路撫寧縣,以海山、昌黎入之。丙辰,括天下戶。尚書省臣言:「諸路課程,歲銀五萬錠,恐疲民力,宜減十分之一。運司官吏俸祿,宜與民官同,其院務官量給工食,仍禁所司多取於民,歲終,較其增損而加黜陟。上都地里遙遠,商旅往來不易,特免收稅以優之,惟市易莊宅、奴婢、孳畜,例收契本工墨之費。管民官遷轉,以三十月為一考,數於變易,人心茍且,自今請以六十月遷轉。諸王遣使取索諸物及鋪馬等事,自今並以文移,毋得口傳教令。」並從之。改宣徽院為光祿司,秩正三品,以宣徽使線真為光祿使。庚申,命樞密院閱實軍數。壬戌,東平府進瑞麥,一莖二穗、三穗、五穗者各一本。省中都打捕鷹坊總管府入工部。大名、東平等路桑蠶皆災,南京、河南等路蝗,減今年銀絲十之三。



 六月丙子,敕西夏中興市馬五百匹。庚辰,敕:「戍軍還,有乏食及病者,令所過州城村坊主者給飲食醫藥。」丁亥,罷各路洞冶總管府,以轉運司兼領。徙謙州甲匠於松山,給牛具。賜皇子南木合馬六千、牛三千、羊一萬。賜北邊戍軍馬二萬、牛一千、羊五萬。丙申,立籍田大都東南郊。禁民擅入宋境剽掠。



 秋七月辛丑,設上林署。乙卯,賜諸王拜答寒印及海青、金符二。庚申,初給軍官俸。壬戌,簽諸道回回軍。乙丑,閱實諸路砲手戶。都元帥也速帶兒等略地光州,敗宋兵於金剛臺。以遼東開元等路總管府兼本路轉運司事。山東諸路旱蝗。免軍戶田租,戍邊者給糧。命達魯花赤兀良吉帶給上都扈從畋獵糧。



 八月戊辰朔,築環城以逼襄陽。己巳,賑應昌府饑。諸王拜答寒部曲告饑,命有車馬者徙居黃忽兒玉良之地,計口給糧,無車馬者就食肅、沙、甘州。戊寅,隆興府總管昔剌斡脫以盜用官錢罷。庚辰,以御史大夫塔察兒同知樞密院事,御史中丞帖只為御史大夫。高麗世子王愖來賀聖誕節。辛巳,設應昌府官吏。辛卯,保定路霖雨,傷禾稼。



 九月庚子,敕僧、道、也裏可溫有家室不持戒律者,占籍為民。丁巳,太陰犯井。丙寅,括河西戶口,定田稅。宋將範文虎以兵船二千艘來援襄陽,阿術、合答、劉整率兵逆戰於灌子灘,殺掠千餘人,獲船三十艘,文虎引退。西京饑,敕諸王阿只吉所部就食太原。山東饑,敕益都、濟南酒稅以十之二收糧。



 冬十月戊辰朔,敕兩省以已奏事報御史臺。庚午,太白犯右執法。癸酉,敕宗廟祭祀祝文,書以國字。乙亥,宋人攻莒州。乙酉,享於太廟。丁亥,以南京、河南兩路旱蝗,減今年差賦十之六。發清、滄鹽二十四萬斤,轉南京米十萬石,並給襄陽軍。己丑,敕來年太廟牲牢,勿用豢豕,以野豕代之,時果勿市,取之內園。車駕至自上都。降興中府為州。賑山東淄萊路饑。



 十一月壬寅,熒惑犯太微西垣上將。壬子,河西諸郡諸王頓舍,僧、民協力供給。丁巳,敕益兵二千,合前所發軍為六千,屯田高麗,以忻都及前左壁總帥史樞,並為高麗金州等處經略使,佩虎符,領屯田事。仍詔諭高麗國王立侍儀司。安南國王陳光昞遣使來貢,優詔答之。復賑淄萊路饑。閏月丁卯朔,高麗世子王愖還,賜王禃至元八年歷。戊辰,禁繒段織日月龍虎,及以龍犀飾馬鞍者。己巳,給河西行省鈔萬錠,以充歲費。以義州隸婆娑府。癸未,詔諭西夏提刑按察司管民官,禁僧徒冒據民田。壬辰,申明勸課農桑賞罰之法。詔設諸路脫脫禾孫。



 十二月丙申朔,改司農司為大司農司,添設巡行勸農使、副各四員,以御史中丞孛羅兼大司農卿。安童言孛羅以臺臣兼領,前無此例。有旨:「司農非細事,朕深諭此,其令孛羅總之。」命陜西等路宣撫使趙良弼為秘書監,充國信使,使日本。敕歲祀太社、太稷、風師、雨師、雷師。戊戌,徙懷孟新民千八百餘戶居河西。壬寅,升御史大夫秩正二品。降河南韶州為澠池縣。宋重慶制置硃禩孫遣諜者持書榜來誘安撫張大悅等,大悅不發封,並諜者送致東川統軍司。丁未,金齒、驃國三部酋長阿匿福、勒丁、阿匿爪來內附,獻馴象三、馬十九匹。己酉,魚通路知府高曳失獲宋諜者,詔賞之。辛酉,以都水監隸大司農司。以諸王伯忽兒為札魯忽赤之長。建大護國仁王寺於高良河。敕更定僧服色。是歲,天下戶一百九十二萬九千四百四十九。賜先朝后妃及諸王金、銀、幣、帛如歲例。斷死刑四十四人。



 八年春正月乙丑朔,高麗國王王禃遣其秘書監樸恆、郎將崔有淹來賀,兼奉歲貢。丙寅,太陰犯畢。己卯,以同僉河南等路行中書省事阿里海牙參知尚書省事。中書省臣言:「前有旨令臣與樞密院、御史臺議河南行省阿里伯等所置南陽等處屯田,臣等以為凡屯田人戶,皆內地中產之民,遠徙失業,宜還之本籍。其南京、南陽、歸德等民賦,自今悉折輸米糧,貯於便近地,以給襄陽軍食。前所屯田,阿里伯自以無效引伏,宜令州郡募民耕佃。」從之。史天澤告老,不允。敕:「前築都城,徙居民三百八十二戶,計其直償之。」設樞密院斷事官。遣兀都蠻率蒙古軍鎮西方當當。丙戌,高麗安撫阿海略地珍島,與逆黨遇,多所亡失。中書省臣言:「諜知珍島餘糧將竭,宜乘弱攻之。」詔不許,令巡視險要,常為之備。丁亥,管如仁、費正寅以國機事為書,謀遣崔繼春、賈靠山、路坤入宋,事覺窮治,正寅、如仁、繼春皆正典刑,靠山、坤並流遠方。壬辰,敕:「諸路鰥寡孤獨疾病不能自存者,官給廬舍、薪米。」高麗國王王禃遣使奉表,為世子愖請昏。詔禁邊將受賂放軍及科斂。賑北京、益都饑。二月乙未朔,定民間婚聘禮幣,貴賤有差。丁酉,發中都、真定、順天、河間、平灤民二萬八千餘人築宮城。己亥,罷諸路轉運司入總管府。以尚書省奏定條畫頒天下。移陜蜀行中書省於興元。癸卯,四川行省也速帶兒言:「比因饑饉,盜賊滋多,宜加顯戮。」詔令群臣議,安童以為:「強竊盜賊,一皆處死,恐非所宜。罪至死者,仍舊待命。」以中書左丞、東京等路行尚書省事趙璧為中書右丞。甲辰,添設監察御史六員。命忽都答兒持詔招諭高麗林衍餘黨裴仲孫。乙巳,大理等處宣慰都元帥寶合丁、王傅闊闊帶等,協謀毒殺雲南王,火你赤、曹楨發其事,寶合丁、闊闊帶及阿老瓦丁、亦速夫並伏誅,賞楨、火你赤及證左人金銀有差。以沙州、瓜州鷹坊三百人充軍。戊申,詔以治事日程諭中外官吏。敕往畏吾兒地市米萬石。庚戌,申嚴東川井鹽之禁。己未,敕軍官佩金銀符,其民官、工匠所佩者,並拘入,勿復給。敕海青符用太祖皇帝御署。庚申,奉御九住舊以梳櫛奉太祖,奉所落須發束上,詔櫝之,藏於太廟夾室。辛酉,敕:「凡訟而自匿及誣告人罪者,以其罪罪之。」分歸德為散府,割宿、亳、邳、徐等州隸之。升申州為南陽府,割唐、鄧、裕、嵩、汝等隸之。賑西京饑。三月乙丑,增治河東山西道按察司,改河東陜西道為陜西四川道,山北東西道為山北遼東道。甲戌,敕:「元正、聖節、朝會,凡百官表章、外國進獻、使臣陛見、朝辭禮儀,皆隸侍儀司。」丙子,改山東、河間、陜西三路鹽課都轉運司為都轉運鹽使司。乙卯,中書省臣言:「高麗叛臣裴仲孫乞諸軍退屯,然後內附;而忻都未從其請,今願得全羅道以居,直隸朝廷。」詔以其飾詞遷延歲月,不允。辛巳,復立夏邑縣,以碭山入焉。省穀熟入睢陽。濱棣萬戶韓世安,坐私儲糧食、燒毀軍器、詐乘驛馬及擅請諸王塔察兒益都四縣分地等事,有司屢以為言,詔誅之,仍籍其家。甲申,車駕幸上都。乙酉,許衡以老疾辭中書機務,除集賢大學士、國子祭酒,衡納還舊俸,詔別以新俸給之。命設國子學,增置司業、博士、助教各一員,選隨朝百官近侍蒙古、漢人子孫及俊秀者充生徒。丁亥,熒惑犯太微西垣上將。己丑,立西夏中興等路行尚書省,以趁海參知行尚書省事。命尚書省閱實天下戶口,頒條畫,諭天下。賑益都等路饑。敕:「有司毋留獄滯訟,以致越訴,違者官民皆罪之。」制封皇子燕王乳母趙氏豳國夫人,夫鞏德祿追封德育公。夏四月壬寅,高麗鳳州經略司忻都言:「叛臣裴仲孫,稽留使命,負固不服,乞與忽林赤、王國昌分道進討。」從之。平灤路昌黎縣民生子,中夜有光,詔加鞠養。或以為非宜,帝曰:「何幸生一好人,毋生嫉心也。」命高麗簽軍征珍島。癸卯,給河南行中書省歲用銀五十萬兩,仍敕襄樊軍士自今人月給米四斗。甲辰,簽壯丁備宋。戊午,阿術率萬戶阿剌罕等與宋將範文虎等戰於湍灘,敗之,獲統制硃勝等百餘人,奪其軍器,賞阿術、阿剌罕等金帛有差。以至元七年諸路災,蠲今歲絲料輕重有差。五月乙丑,以東道兵圍守襄陽,命賽典赤、鄭鼎提兵,水陸並進,以趨嘉定,汪良臣、彭天祥出重慶,札剌不花出瀘州,曲立吉思出汝州,以牽制之。改簽省也速帶兒、鄭鼎軍前行尚書事,賽典赤行省事於興元,轉給軍糧。丙寅,牢魚國來貢。己巳,修佛事於瓊華島。辛未,分大理國三十七部為三路,以大理八部蠻酋新附,降詔撫諭。壬申,造內外儀仗。丁丑,賑蔚州饑。乙卯,命史天澤平章軍國重事。升太府監為正三品。忻都、史樞表言珍島賊徒敗散,餘黨竄入耽羅。辛巳,賜河西行省金符、銀海青符各一。令蒙古官子弟好學者,兼習算術。癸未,升濟州為濟寧府。以玉宸院棣宣徽院。高麗國王王禃遣使貢方物。六月甲午,敕樞密院:「凡軍事徑奏,不必經由尚書省,其幹錢糧者議之。」上都、中都、河間、濟南、淄萊、真定、衛輝、洺磁、順德、大名、河南、南京、彰德、益都、順天、懷孟、平陽、歸德諸州縣蝗。癸卯,宋將範文虎率蘇劉義、夏松等舟師十萬援襄陽,阿術率諸將迎擊,奪其戰船百餘艘,敵敗走。平章合答又遣萬戶解汝楫等邀擊,擒其總管硃日新、鄭皋,大破之。辛亥,敕:「凡管民官所領錢穀公事,並俟年終考較。」乙卯,招集河西、斡端、昂吉呵等處居民。己未,山東統軍司塔出、董文炳偵知宋人欲據五河口,請築城守之,既而坐失事機,宋兵已樹柵其地。事聞,敕決罰塔出、文炳等有差。遼州和順縣、解州聞喜縣虸蚄生。秋七月壬戌朔,尚書省請增太原鹽課,歲以鈔千錠為額,仍令本路兼領,從之。設回回司天臺官屬,以札馬剌丁為提點。簽女直、水達達軍。以鄭元領祠祭岳瀆,授司禋大夫。丁卯,南人李忠進言,運山侍郎張大悅嘗與宋交通,以其事無實,詔諭大悅:「宋善用間,朕不輕信,毋懷疑懼。」以國王頭輦哥行尚書省於北京、遼東等路。辛未,置左、右、中三衛親軍都指揮使司。乙亥,鞏昌、臨洮、平涼府、會、蘭等州隕霜殺禾。乙酉,宋將來興國攻百丈山營,阿術擊破之,追至湍灘,斬首二千餘級。高麗世子王愖入質,珍島脅從民戶來降。八月壬辰朔,日有食之。癸巳,敕:「軍站戶地四頃以上,依例輸租。」己亥,詔招諭宋襄陽守臣呂文煥。壬子,車駕至自上都。遷成都統軍司於眉州。己未,聖誕節,初立內外仗及雲和署樂位。東川統軍司引兵攻宋銅鈸寨,守寨總管李慶等降,以慶知梁山軍事。九月壬戌朔,敕都元帥阿術以所部兵略地漢南。癸亥,高麗世子王愖辭歸,賜國王王禃西錦,優詔諭之。甲子,賜劉整鈔五百錠、鄧州田五百頃,整辭,改賜民田三百戶,科調如故。給河南行省歲用鈔二萬八千六百錠。丙寅,罷陜西五路西蜀四川行尚書省,以也速答兒行四川尚書省事於興元,京兆等路直隸尚書省。敗宋軍於渦河。戊辰,升成都府德陽縣為德州,降虢州為虢略縣。壬申,選胄子脫脫木兒等十人肄業國學。癸酉,益都府濟州進芝二本。甲戌,簽西夏回回軍。太廟殿柱朽壞,監察御史劾都水劉晸監造不敬,晸以憂卒。張易請先期告廟,然後完葺,從之。丙子,敕今歲享太廟毋用犧牛。太陰犯畢。庚辰,右衛親軍都指揮使忽都等言:「五河城堡已成,唯廬舍未完,凡材甓皆出宋境,請率精兵分道抄掠。」從之。壬午,山東路統軍司言宋兵攻膠州,千戶蔣德等逆戰敗之,俘統制範廣等五十餘人,獲戰船百艘。癸未,詔忙安倉失陷米五千餘石,特免征,仍禁諸王非理需索。詔以四川民力困弊,免茶鹽等課稅,以軍民田租給沿邊軍食。仍敕:「有司自今有言茶鹽之利者,以違制論。」冬十月癸巳,大司農臣言:「高唐州達魯花赤忽都納、州尹張廷瑞、同知陳思濟勸課有效,河南府陜縣尹王仔怠於勸課,宜加黜陟,以示勸懲。」從之。丁酉,享於太廟。己未,檀、順等州風潦害稼。賜高麗至元九年歷。十一月辛酉朔,敕品官子孫儤直,敕遣阿魯忒兒等撫治大理。壬戌,罷諸路交鈔都提舉司。乙亥,劉秉忠及王磐、徒單公履等言:「元正、朝會、聖節、詔赦及百官宣敕,具公服迎拜行禮。」從之。禁行金《泰和律》。建國號曰大元,詔曰:



 誕膺景命,奄四海以宅尊;必有美名,紹百王而紀統。肇從隆古,匪獨我家。且唐之為言蕩也,堯以之而著稱;虞之為言樂也,舜因之而作號。馴至禹興而湯造,互名夏大以殷中。世降以還,事殊非古。雖乘時而有國,不以利而制稱。為秦為漢者,著從初起之地名;曰隋曰唐者,因即所封之爵邑。是皆徇百姓見聞之狃習,要一時經制之權宜,概以至公,不無少貶。



 我太祖聖武皇帝,握乾符而起朔土,以神武而膺帝圖,四震天聲,大恢土宇,輿圖之廣,歷古所無。頃者耆宿詣庭,奏章申請,謂既成於大業,宜早定於鴻名。在古制以當然,於朕心乎何有。可建國號曰大元,蓋取《易經》「乾元」之義。茲大冶流形於庶品,孰名資始之功;予一人底寧於萬邦,尤切體仁之要。事從因革,道協天人。於戲!稱義而名,固匪為之溢美;孚休惟永,尚不負於投艱。嘉與敷天,共隆大號。



 丙戌,置四川省於成都。上都萬安閣成。十二月辛卯朔,詔天下興起國字學。宣徽院請以闌遺、漏籍等戶淘金,帝曰:「姑止,毋重勞吾民也。」乙巳,減百官俸。括西夏田。召塔出、董文炳赴闕。辛亥,並太常寺入翰林院,宮殿府入少府監。甲寅,詔尚書省遷入中書省。是歲,天下戶一百九十四萬六千二百七十。賜先朝后妃及諸王金、銀、幣、帛如歲例,賜囊家等羊馬價鈔萬千一百六十七錠。斷死罪一百五人。



 九年春正月庚申朔,高麗國王王禃遣其臣禮賓卿宣文烈來賀,兼奉歲貢。甲子,並尚書省入中書省,平章尚書省事阿合馬、同平章尚書省事張易並中書平章政事,參知尚書省事張惠為中書左丞,參知尚書省事李堯咨、麥術丁並參知中書政事。罷給事中、中書舍人、檢正等官,仍設左右司,省六部為四,改稱中書。丙寅,詔遣不花及馬璘諭高麗具舟糧助征耽羅。河南省請益兵,敕諸路簽軍三萬。丁丑,敕皇子西平王奧魯赤、阿魯帖木兒、禿哥及南平王禿魯所部與四川行省也速帶兒部下,並忙古帶等十八族、欲速公弄等土番軍,同征建都。新安州初隸雄州,詔為縣入順天。庚辰,改北京、中興、四川、河南四路行尚書省為行中書省。京兆復立行省,仍命諸王只必帖木兒設省斷事官。給西平王奧魯赤馬價弓矢,賜南平王禿魯銀印及金銀符各五。辛巳,移鳳州屯田於鹽、白二州。敕董文炳時巡掠南境,毋令宋人得立城堡。敕:「軍民訟田者,民田有餘則分之軍,軍田有餘亦分之民。仍遣能臣聽其直,其軍奴入民籍者,還正之。」敕燕王遣使持香幡,祠岳瀆、后土、五臺興國寺。命劉整總漢軍。壬午,改山東東路都元帥府統軍司為行樞密院,以也速帶兒、塔出並為行樞密院副使。乙酉,定受宣敕官禮儀。詔元帥府統軍司、總管萬戶府閱實軍籍。



 二月庚寅朔,奉使日本趙良弼遣書狀官張鐸同日本二十六人,至京師求見。辛卯,詔:「札魯忽赤乃太祖開創之始所置,位百司右,其賜銀印,立左右司。」壬辰,高麗國王王禃遣其臣齊安侯王淑來賀改國號。改中都為大都。甲午,命阿術典蒙古軍,劉整、阿里海牙典漢軍。戊戌,以去歲東平及西京等州縣旱蝗水潦,免其租賦。庚子,復唐州泌陽縣。建中書省署於大都。戊申,始祭先農如祭社之儀。詔諸路開浚水利。車駕幸上都。



 三月乙丑,諭旨中書省,日本使人速議遣還。安童言:「良弼請移金州戍兵,勿使日本妄生疑懼。臣等以為金州戍兵,彼國所知,若復移戍,恐非所宜。但開諭來使,此戍乃為耽羅暫設,爾等不須疑畏也。」帝稱善。甲戌,括民間《四教經》,焚之。蒙古都元帥阿術、漢軍都元帥劉整、阿里海牙督本軍破樊城外郛,斬首二千級,生擒將領十六人,增築重圍守之。賑濟南路饑。詔免醫戶差徭。



 夏四月己丑,詔於土番、西川界立寧河驛。辛卯,賜皇子愛牙赤所部馬。丙午,給西平王奧魯赤所部米。甲寅,賑大都路饑。



 五月戊午朔,立和林轉運司,以小雲失別為使,兼提舉交鈔使。己未,給闊闊出海青銀符二。辛酉,罷簽回回軍。癸亥,敕拔都軍於怯鹿難之地開渠耕田。丙寅,簽徐、邳二州丁壯萬人戍邳州。庚午,減鐵冶戶,罷西蕃禿魯乾等處金銀礦戶為民。禁漢人聚眾與蒙古人鬥毆。詔議取耽羅及濟州。辛巳,敕修築都城,凡費悉從官給,毋取諸民,並蠲伐木役夫稅賦。甲申,敕諸路軍戶驅丁,除至元七年前從良入民籍者當差,餘雖從良,並令助本戶軍力。乙酉,太白犯畢距星。宮城初建東西華、左右掖門。詔安集答里伯所部流民。



 六月壬辰,遣高麗國西京屬城諸達魯花赤及質子金鎰等歸國。減乞里吉思屯田所入租,仍遣南人百名,給牛具以往。是夜,京師大雨,壞墻屋,壓死者眾。癸巳,敕以籍田所儲糧賑民,不足,又發近地官倉濟之。甲午,高麗告饑,轉東京米二萬石賑之。己亥,山東路行樞密院塔出於四月十三日遣步騎趨漣州,攻破射龍溝、五港口、鹽場、白頭河四處城堡,殺宋兵三百餘人,虜獲人牛萬計,第功賞賚有差。辛亥,高麗國王王禃請討耽羅餘寇。



 秋七月丁巳朔,河南省臣言:「往歲徙民實邊屯耕,以貧苦悉散還家。今唐、鄧、蔡、息、徐、邳之民,愛其田廬,仍守故屯,願以絲銀準折輸糧,而內地州縣轉粟餉軍者,反厭苦之。臣議今歲沿邊州郡,宜仍其舊輸糧,內地州郡,驗其戶數,俾折鈔就沿邊和糴,庶幾彼此交便。」制曰:「可。」拘括開元、東京等路諸漏籍戶。禁私鬻《回回歷》。賑水達達部饑。戊寅,賜諸王八八部銀鈔。集都城僧誦《大藏經》九會。壬午,和禮霍孫奏:「蒙古字設國子學,而漢官子弟未有學者,及官府文移猶有畏吾字。」詔自今凡詔令並以蒙古字行,仍遣百官子弟入學。乙酉,免徙大羅鎮居民,令倍輸租米給鷹坊。詔分閱大都、京兆等處探馬赤奴戶名籍。



 八月丙戌朔,日有食之。戊子,立群牧所,掌牧馬及尚方鞍勒。壬辰,敕忙安倉及靖州預儲糧五萬石,以備弘吉剌新徙部民及西人內附者廩給。調兵增戍全羅州。乙未,禁諸人以己事輒呼至尊稱號者。丁酉,立斡脫所。己亥,諸王闊闊出請以分地寧海、登、萊三州自為一路,與他王比,歲賦惟入寧海,無輸益都,詔從之。癸卯,千戶崔松敗宋襄陽援兵,斬其將張順,賜松等將士有差。乙巳,車駕至自上都。丁未,改延州為延津縣,與陽武同隸南京。癸丑,賑遼東等路饑。



 九月甲子,宋襄陽將張貴以輪船出城,順流突戰,阿術、阿剌海牙等舉烽燃火,燭江如晝,率舟師轉戰五十餘里,至櫃門關,生獲貴及將士二千餘人。丙寅,敕樞密院:「諸路正軍帖戶及同籍親戚奴僕,丁年既長,依諸王權要以避役者,並還之軍,惟匠藝精巧者以名聞。」癸酉,同簽河南省事崔斌訟右丞阿里妄奏軍數二萬,敕杖而罷之。甲戌,罷水軍總管府。東川元帥李吉等略地開州,拔石羊寨,擒宋將一人。統軍使合剌等兵掠合州及渠江口,獲戰船五十艘,賞銀幣有差。丙子,民夫三千人伐巨木遼東,免其家徭賦。戊寅,太陰犯御女。賑益都路饑。



 冬十月丙戌朔,封皇子忙哥剌為安西王,賜京兆為分地,駐兵六盤山。遣使持詔諭扮卜、忻都國。壬辰,享於太廟。癸巳,趙璧為平章政事,張易為樞密副使。乙未,築渾河堤。戊戌,熒惑犯填星。己亥,敕自七月至十一月終聽捕獵,餘月禁之。癸卯,立文州。初立會同館。



 十一月乙卯朔,詔以至元十年歷賜高麗。壬戌,發北京民夫六千,伐木乾山,蠲其家徭賦。諸王只必帖木兒築新城成,賜名永昌府。丙寅,蠲昔剌斡脫所負官錢。丁卯,太陰犯畢。城光州。遣無籍軍掠宋境。己巳,敕發屯田軍二千、漢軍二千、高麗軍六千,仍益武衛軍二千,徵耽羅。辛未,召高陵儒者楊恭懿,不至。癸酉,以前拔樊城外郛功,賞千戶劉深等金銀符。己卯,並中書省左右司為一。宋荊湖制置李庭芝為書,遣永寧僧齎金印、牙符,來授劉整盧龍軍節度使,封燕郡王。僧至永寧,事覺,上聞,敕張易、姚樞雜問。適整至自軍中,言:「宋患臣用兵襄陽,欲以是殺臣,臣實不知。」敕令整為書復之,賞整,使還軍中,誅永寧僧及其黨友。參知行省政事阿里海牙言:「襄陽受圍久未下,宜先攻樊城,斷其聲援。」從之。回回亦思馬因創作巨石砲來獻,用力省而所擊甚遠,命送襄陽軍前用之。



 十二月乙酉朔,詔諸路府州司縣達魯花赤管民長官,兼管諸軍奧魯。丁亥,立肅州等處驛。以東平府民五萬餘戶復為東平路。辛丑,諸王忽剌出拘括逃民高麗界中,高麗達魯花赤上其事,詔高麗之民猶未安集,禁罷之。遣宋議互市使者南歸。戊午,賜北平王南木合軍馬一萬二千九百九十一、羊六萬一千五百三十一,及諸王塔察兒軍幣帛。辛亥,宋將昝萬壽來攻成都,簽省嚴忠範出戰失利,退保子城,同知王世英等八人棄城遁。詔以邊城失守,罪在主將,世英雖遁,與免其罪,惟遣使縛忠範至京師。癸丑,升拱衛司為拱衛直都指揮使司。是歲,天下戶一百九十五萬五千八百八十。賜先朝后妃及諸王金、銀、幣、帛如歲例。斷死罪三十九人。建大聖壽萬安寺。



\end{pinyinscope}