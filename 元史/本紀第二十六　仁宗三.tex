\article{本紀第二十六 仁宗三}

\begin{pinyinscope}

 四年春正月庚子,帝謂左右曰:「中書比奏百姓乏食,宜加賑恤。朕默思之,民饑若此,豈政有過差以致然歟?向詔百司務遵世祖成憲,宜勉力奉行,輔朕不逮,然嘗思之,唯省刑薄賦,庶使百姓各遂其生也。」乙卯,諸王脫脫駐雲南,擾害軍民,以按灰代之。丙辰,以知樞密院事完者為雲南行省平章政事。己未,給帝師寺廩食鈔萬錠。壬戌,冀寧路地震。戊辰,給諸王也速也不干、明安答兒部糧三月。



 閏月庚辰,封諸王孛羅為冀王。丙戌,以立皇太子詔天下,給賜鰥寡孤獨鈔,減免各路租稅有差。賜諸王、宗戚朝會者,金三百兩、銀二千五百兩、鈔四萬三千九百錠。辛卯,封別鐵木兒為汾陽王。壬辰,給豳王南忽里部鈔十二萬錠買馬。汴梁、揚州、河南、淮安、重慶、順慶、襄陽民皆饑,發廩賑之。



 二月庚子,賜諸王買閭部鈔三萬錠。甲辰,敕郡縣各社復置義倉。戊申,特授近侍完者不花翰林侍讀學士、知制誥、同修國史。癸亥,升泰寧府為泰寧路,仍置泰寧縣。乙丑,升蒙古國子監秩正三品,賜銀印。丙寅,以諸王部值脫火赤之亂,百姓貧乏,給鈔十六萬六千錠、米萬石賑之。曹州水,免今年租。三月丁卯朔,升靖州為路。庚午,給趙王阿魯禿部糧四千石。乙酉,太陰犯箕。辛卯,車駕幸上都。



 夏四月戊戌,給安王兀都思不花部軍糧三月。己亥,德安府旱,免屯田租。壬寅,加授太常禮儀院使拜住大司徒,賜趙王阿魯禿金五十兩、銀五百兩、鈔千錠,割懷來縣隸龍慶州。甲辰,以太寧路隸遼陽省。戊申,答合孫寇邊,吳王朵列納等敗之於和懷,賜金玉束帶、黃金、幣帛有差。己未,諸王紐憐薨。乙丑,禁嶺北酒。常嘗夜坐,謂侍臣曰:「雨暘不時,奈何?」蕭拜住對曰:「宰相之過也。」帝曰:「卿不在中書耶?」拜住惶愧。頃之,帝露香默禱。既而大雨,左右以雨衣進,帝曰:「朕為民祈雨,何避焉!」翰林學士承旨忽都魯都兒迷失、劉賡等譯《大學衍義》以進,帝覽之,謂群臣曰:「《大學衍義》議論甚嘉,其令翰林學士阿憐鐵木兒譯以國語。」五月辛未,授上都留守闊闊出開府儀同三司、大司徒。壬申,賜出征諸王丑漢等金銀、鈔幣有差。乙亥,加封大長公主忙哥臺為皇姑大長公主,給金印。戊寅,改衛率府為中翊府。壬午,黃州、高郵、真州、建寧等處,流民群聚,持兵抄掠,敕所在有司,其傷人及盜者罪之,餘並給糧遣歸。以翰林學士承旨赤因鐵木兒為中書平章政事,中書平章兀伯都剌為集賢大學士。己丑,升中書左丞阿卜海牙為平章政事,參政乞塔為右丞,高昉為左丞,參議中書省事換住、張思明並參知政事。



 六月乙巳,太陰犯心。內外監察御史四十餘人劾鐵木迭兒奸貪不法。戊申,鐵木迭兒罷,以左丞相合散為中書右丞相。己酉,兀伯都剌復為中書平章政事。壬子,以工部尚書王桂為中書參知政事。安遠王丑漢、趙王阿魯禿為叛王脫火赤所掠,各賜金銀、幣帛。丙辰,敕:「諸王、駙馬、功臣分地,仍舊制自闢達魯花赤。」丁巳,安南國遣使來貢。戊午,置冀王孛羅王傅二員,中尉、司馬各一員,都總管府秩正三品。己未,給嶺北行省經費鈔九十萬錠、雜彩五萬匹。癸亥,禁總攝沈明仁所佩司空印毋移文有司。



 秋七月乙亥,李孟罷,以江浙行省左丞王毅為中書平章政事。庚辰,賜皇姑大長公主忙哥臺金百兩、銀千兩、鈔二千錠、幣帛各百匹。賞討叛王有功句容郡王床兀兒等金銀、幣帛、鈔各有差。壬午,敕赤因鐵木兒頒賚諸王、駙馬,及賑濟所部貧乏。特授中衛親軍都指揮使孛蘭奚太尉。己丑,成紀縣山崩,土石潰徙,壞田稼廬舍,壓死居民。辛卯,冀寧路地震。帝諭省臣曰:「比聞蒙古諸部困乏,往往鬻子女於民家為婢僕,其命有司贖之還各部。」帝出,見衛士有敝衣者,駐馬問之,對曰:「戍守邊鎮餘十五年,以故貧耳。」帝曰:「此輩久勞於外,留守臣未常以聞,非朕親見,何由知之!自今有類此者,必言於朕。」因命賜之錢帛。八月丙申,車駕至自上都。熒惑犯輿鬼。壬子,太陰犯昴。庚申,合散奏事畢,帝問曰:「卿等日所行者何事?」合散對曰:「臣等第奉行詔旨而已。」帝曰:「卿等何嘗奉行朕旨,雖祖宗遺訓,朝廷法令,皆不遵守。夫法者,所以辨上下,定民志,自古及今,未有法不立而天下治者。使人君制法,宰相能守而勿失,則下民知所畏避,綱紀可正,風俗可厚。其或法弛民慢,怨言並興,欲求治安,豈不難哉?」九月丙寅,合散言:「故事,丞相必用蒙古勛臣;合散回回人,不厭人望。」遂懇辭,制以宣徽使伯答沙為中書右丞相,合散為左丞相。己巳,大都南城產嘉禾,一莖十一穗。庚午,太陰犯鬥。壬辰,詔戒飭海漕,諭諸司毋得沮撓。嶺北地震三日。



 冬十月甲午朔,有事於太廟。戊戌,給諸王晃火鐵木兒等部糧五千石。壬寅,敕刑部尚書舉林柏監大都兵馬司防遏盜賊,仍嚴飭軍校,制其出入。遣御史大夫伯忽、參知政事王桂祭陜西嶽鎮名山,賑恤秦州被災之民。己酉,監察御史言:「官吏丁憂起復,人情驚惑,請禁止以絕僥幸。惟朝廷耆舊特旨起復者,不在禁例。」制曰:「可。」給兩淮屯田總管府職田。壬子,給鈔五萬錠、糧五萬石,賑察罕腦兒。戊午,海外婆羅公之民往賈海番,遇風濤,存者十四人漂至溫州永嘉縣,敕江浙省資遣還鄉。改潮州路所統梅州隸廣東道宣慰司。



 十一月己卯,復浚揚州運河。己丑,並汧源縣入隴州。壬辰,諭:「諸宿衛入直,各居其次,非有旨不得上殿,闌入禁中者坐罪。大臣許從二人,他官一人,門者譏其出入。」十二月丁酉,復廣州採金銀珠子都提舉司,秩正四品,官三員。乙巳,置詹事院,從一品,太子詹事四員,副詹事、詹事丞並二員,家令府、延慶司設官並四員,典寶監八員。遣官即興和路及凈州發廩賑給北方流民。己酉,盧溝橋、澤畔店、琉璃河並置巡檢司。壬子,置安王王傅。丁巳,賜諸王禿滿鐵木兒等及駙馬忽剌兀帶各部,金一千二百兩、銀七千七百兩、鈔一萬七千七百錠、幣帛二千匹。以內宰領延福司事禿滿迭兒知樞密院事,特授晉王內史按攤出金紫光祿大夫、魯國公。辛酉,改怯憐口民匠總管府為繕用司。



 五年春正月辛未,賜諸王禿滿鐵木兒等所部鈔四萬錠。甲戌,懿州地震。丙子,安南國遣其臣尹世才等以方物來貢。乙酉,敕諸王位下民在大都者,與民均役。丁亥,會試進士。湖廣平章買住加魯國公、大司農。賑晉王也孫鐵木兒等部貧乏者。



 二月癸巳朔,日有食之。和寧路地震。丁酉,敕:「廣寧、開元等萬戶府軍入侍衛,有兄弟子侄五人者,三人留,四人三人者,二人留,著為籍。」秦州秦安縣山崩。封諸王晃火鐵木兒為嘉王,禿滿鐵木兒為武平王,並賜印。丁未,敕云南、四川歸還所侵順元宣撫司民地。戊申,升內史府秩正二品。建鹿頂殿於文德殿後。辛亥,敕杭州守臣春秋祭淮安忠武王伯顏祠。王子諸王答失蠻部乏食,敕甘肅行省給糧賑之。賜諸王察吉兒鈔萬錠。甲寅,置寧昌府。乙卯,命中書省汰不急之役,增置河東宣慰司副使一員。敕上都諸寺、權豪商販貨物,並輸稅課。戊午,以者連怯耶兒萬戶府為右衛率府。給書西天字《維摩經》金三千兩。庚申,罷封贈。賞討叛王脫火赤戰功,賜諸王部察罕等金銀幣鈔有差。



 三月戊辰,御試進士,賜忽都達兒、霍希賢以下五十人及第、出身有差。己巳,賜寧海王八都兒金印。庚午,立諸王斡羅溫孫部打捕鷹坊諸色人匠怯憐口總管府,秩從四品。改靜安路為德寧路,靜安縣為德寧縣。癸酉,晉王也孫鐵木兒部貧乏,賑米四千一百五十石,仍賜鈔二萬錠買牛羊孳畜。乙亥,增給兩淮運司分司印一。特授安遠王丑漢開府儀同三司、錄軍國重事、知樞密院事。戊寅,以湖州路為安王兀都思不花分地,其戶數視衛王阿木哥。癸未,和寧、凈州路禁酒。賜鈔萬錠,命晉王也孫鐵木兒賑濟遼東貧民。晉王內史拾得閭加榮祿大夫,封桓國公。給金九百兩、銀百五十兩,書金字《藏經》。甲申,免鞏昌等處經賑濟者差稅鹽課。乙酉,御史臺臣言:「諸司近侍隔越中書聞奏者,請如舊制論罪。」制曰:「可。」己丑,敕以紅城屯田米賑凈州、平地等處流民。置汾陽王別鐵木兒王傅四員,賜丑驢答剌罕平江路田百頃。



 夏四月壬辰,安吉王乞臺普濟薨。丁酉,諸王雍吉剌帶部乏食,賑米三千石。己亥,耽羅捕獵戶成金等為寇,敕征東行省督兵捕之。庚子,賜諸王察吉兒部鈔萬錠,布帛稱是。給中翊府閻臺順州屯田鈔萬錠,置牛種農具。庚戌,敕:「安遠王丑漢分地隸建寧者七縣、汀州者三縣,達魯花赤聽其自闢。」升印經提舉司為延福監,秩正三品。遣官分汰各部流民,給糧賑濟。免懷孟、河南、南陽居民所輸陜西鹽課。是時解州鹽池為水所壞,命懷孟等處食陜西紅鹽;後以地遠,改食滄鹽,而仍輸課陜西,民不堪命,故免之。木鄰、鐵裡乾驛困乏,濟以馬五千匹。遼陽饑,海漕糧十萬石於義、錦州,以賑貧民。甲寅,樞密院臣言:「各省調度軍馬,惟長官二人領其事。今四川省諸臣皆預,非便,請如舊制。」從之。以千奴、史弼並為中書平章政事,侍御史敬儼為中書參知政事。戊午,車駕幸上都。



 五月辛酉朔,順元等處軍民宣撫使阿晝以洞蠻酋黑沖子子昌奉方物來覲。丁卯,賜安王兀都思不花金五百兩、銀五千兩。以御史中丞亦列赤為中書右丞相。戊辰,遣平章政事王毅翽星於司天臺三晝夜。諸王按塔木兒、不顏鐵木兒部乏食,賑糧兩月。壬申,監察御史言:「比年名爵冒濫,太尉、司徒、國公接跡於朝。昔奉詔裁罷,中外莫不欣悅。近聞禮部奉旨鑄太尉、司徒、司空等印二十有六,此輩無功於國,載在史冊,貽笑將來。請自今門閥貴重、勛業昭著者存留一二,餘並革去。」制曰:「可。」癸酉,遣官分道減決笞以下罪。己卯,德慶路地震。鞏昌隴西縣大雨,南土山崩,壓死居民,給糧賑之。六月辛卯,御史臺臣言:「昔遣張驢等經理江浙、江西、河南田糧,虛增糧數,流毒生民,已嘗奉旨俟三年征租。今及其期,若江浙、江西當如例輸之,其河南請視鄉例減半征之。」制曰:「可。」癸巳,以典瑞院使斡赤為集賢大學士、領典瑞院事、大司徒。己亥,北地諸部軍士乏食,給糧賑之。庚子,遣阿尼八都兒、只兒海分汰凈州北地流民,其隸四宿衛及諸王、駙馬者,給資糧遣還各部。癸卯,賜諸王桑哥班金束帶一、銀百兩、鈔五百錠。乙巳,術者趙子玉等七人伏誅。時衛王阿木哥以罪貶高麗,子玉言於王府司馬曹脫不臺等曰:「阿木哥名應圖讖。」於是潛謀備兵器、衣甲、旗鼓,航海往高麗取阿木哥至大都,俟時而發,行次利津縣,事覺,誅之。西番土寇作亂,敕甘肅省調兵捕之。丁巳,賜安王兀都思不花等金束帶及金二百兩、銀一千五十兩、鈔二千二百錠、幣帛二百八十匹。



 秋七月己未朔,李邦寧加開府儀同三司。癸亥,賜諸王八里帶等金二百兩、銀八百五十兩、鈔二千錠、幣帛二百匹。甲子,給欽察衛馬羊價鈔一十四萬五千九百九十二錠。丙寅,調軍五千烏蒙等處屯田,置總管萬戶府,秩正三品,給銀印。丁卯,給鈔二十萬錠、糧萬石,命晉王分賚所部宿衛士。壬申,御史中丞趙簡言:「皇太子春秋鼎盛,宜選耆儒敷陳道義。今李銓侍東宮說書,未諳經史,請別求碩學,分進講讀,實宗社無疆之福。」制曰:「可。」諸王不里牙敦之叛,諸王也舍、失列吉及衛士朵帶、伯都坐持兩端,不助官軍進討,敕流也舍江西,失列吉湖廣,朵帶衡州,伯都潭州。癸酉,拘衛王阿木哥王傅印。置餼廩司,秩正八品,隸上都留守司。豐州石泉店置巡檢司。賜諸王別失帖木兒等金、銀,並賑其部米萬石、鈔萬錠。己卯,諸王雍吉剌帶、曲春鐵木兒來朝,賜金二百兩、銀一千兩、鈔五千錠、幣帛一百匹,仍給鈔萬錠、米萬石,分賚其所部。辛巳,立受給庫,秩九品,隸工部。壬午,罷河南省左丞陳英等所括民田,止如舊例輸稅。戊子,鞏昌路寧遠縣山崩。加封楚三閭大夫屈原為忠節清烈公。



 八月戊子,車駕至自上都。乙卯,並翁源縣入曲江縣。九月癸亥,大司農買住等進司農丞苗好謙所撰《栽桑圖說》,帝曰:「農桑衣食之本,此圖甚善。」命刊印千帙,散之民間。丙寅,廣西兩江龍州萬戶趙清臣、太平路總管李興隆率土官黃法扶、何凱,並以方物來貢,賜以幣帛有差。豳王南忽裏等部貧乏,命甘肅省市馬萬匹給之。丁卯,中書右丞、宣徽使亦列赤為中書平章政事,左丞高昉為右丞,參知政事換住為左丞,吏部尚書燕只乾為參知政事。壬申,以鈔給北邊軍為馬價。甲戌,以作佛事,釋重囚三人,輕囚五十三人。己卯,以江浙省所印《大學衍義》五十部賜朝臣。辛巳,置大永福寺都總管府,秩三品。壬午,敕:「軍官犯罪,行省咨樞密院議擬,毋擅決遣。」丙戌,以僉太常禮儀院事狗兒為中書參知政事。丁亥,立行宣政院於杭州,設官八員。大同路金城縣大雨雹。



 冬十月己丑,以大寧路隸遼陽省,宣德府隸大都路。敕:「僧人除宋舊有及朝廷撥賜土田免租稅,餘田與民一體科征。」播州南寧長官洛麼作亂,思州守臣換住哥招諭之,洛麼遣人以方物來覲。罷膠、萊、莒、密鹽使司,復立濤洛場。辛卯,禁大同、冀寧、晉寧等路釀酒。壬辰,建帝師巴思八殿於大興教寺,給鈔萬錠。癸巳,改中翊府為羽林親軍都指揮使司。甲午,有事於太廟。癸丑,贛州路雩都縣裏胥劉景周,以有司徵括田新租,聚眾作亂,敕免征新租,招諭之。



 十一月辛酉,開成、莊浪等處禁酒。壬戌,改黃花嶺屯儲軍民總管府為屯儲總管府,設官四員。山後民饑,增海漕四十萬石。增置大都南、北兩兵馬司指揮使,色目、漢人各二員,給分司印二。丁卯,用監察御史乃蠻帶等言,追奪建康富民王訓等白身濫受宣敕,仍禁冒籍貫宿衛及巧受遠方職官、不赴任求別調者,隱匿不自首者罪之。己巳,升同知樞密院事忠嘉知樞密院事。丙子,集賢大學士、太保曲出言:「唐陸淳著《春秋纂例》、《辨疑》、《微旨》三書,有益後學,請令江西行省鋟梓,以廣其傳。」從之。癸未,敕江西茶運司歲課以二十五萬錠為額。敕大永福寺創殿,安奉順宗皇帝御容。



 十二月壬辰,特授集賢大學士脫列大司徒。辛亥,置重慶路江津、巴縣等處屯田,省成都歲漕萬二千石。甲寅,敕樞密院核實蒙古軍貧乏者,存恤五年。



 六年春正月丁巳朔,暹國遣使奉表來貢方物。丁卯,敕:「福建、兩廣、雲南、甘肅、四川軍官致仕還家,官給驛傳如民官例。」戊辰,賑晉王部貧民。癸酉,特授同知徽政院事醜驢答剌罕金紫光祿大夫、太尉,給銀印。甲戌,監察御史孛術魯翀等言:「皇太子位正東宮,既立詹事院以總家政,宜擇年德老成、道義崇重者為師保賓贊,俾盡心輔導,以廣緝熙之學。」制曰:「可。」戊寅,太陰犯心。己卯,翽星於司天臺。廣東南恩、新州徭賊龍郎庚等為寇,命江西行省發兵捕之。帝御嘉禧殿,謂扎魯忽赤買閭曰:「扎魯忽赤人命所系,其詳閱獄辭,事無大小,必謀諸同僚,疑不能決者,與省、臺臣集議以聞。」又顧謂侍臣曰:「卿等以朕居帝位為安邪?朕惟太祖創業艱難,世祖混一疆宇,兢業守成,恆懼不能當天心,繩祖武,使萬方百姓樂得其所,朕念慮在茲,卿等固不知也。」



 二月丁亥朔,日有食之。改釋奠於中丁,祀社稷於中戊。翽星於回回司天臺。丁酉,雲南闍裏愛俄、永昌蒲蠻阿八剌等並為寇,命雲南省從宜剿捕。戊戌,改陜西轉運鹽使司為河東陜西都轉運鹽使司,直隸省部。己亥,太陰犯靈臺。乙巳,敕:「諸司不由中書奏官輒署事者悉罷之。」特授僧從吉祥榮祿大夫、大司空,加榮祿大夫、大司徒僧文吉祥開府儀同三司。



 三月丁巳,以天壽節,釋重囚一人。乙未,給鈔賑濟上都、西番諸驛。辛酉,斡端地有叛者入寇,遣鎮西武靖王搠思班率兵討之。詔以御史中丞禿禿合為御史大夫,諭之曰:「御史大夫職任至重,以卿勛舊之裔,故特授汝。當思乃祖乃父忠勤王室,仍以古名臣為法,否則將墜汝家聲,負朕委任之意矣。」丙寅,改懷孟路為懷慶路。特授翰林學士承旨八兒思不花開府儀同三司、大司徒。己巳,太陰犯明堂。敕:「諸王、駙馬、宗姻諸事,依舊制領於內八府官,忽徑移文中書。」封諸王月魯鐵木兒為恩王,給印,置王傅官。免大都、上都、興和、大同今歲租稅。癸酉,太陰犯日星。甲戌,太陰犯心。壬午,賜大興教寺僧齋食鈔二萬錠,禁甘肅行省所屬郡縣釀酒。



 夏四月壬辰,中書省臣言:「雲南土官病故,子侄兄弟襲之,無則妻承夫職。遠方蠻夷,頑獷難制,必任土人,可以集事。今或闕員,宜從本俗,權職以行。」制曰:「可。」丙辰,命京師諸司官吏運糧輸上都、興和,賑濟蒙古饑民。庚子,車駕幸上都。以鐵木迭兒為太子太師。內外監察御史四十餘人,劾其逞私蠹政,難居師保之任,不聽。諸王合贊薨。丙午,命宣政院賑給西番諸驛。壬子,伯顏鐵木兒部貧乏,給鈔賑之。



 五月辛酉,太陰犯靈臺。丁卯,太陰犯房。丙子,太陰犯壘壁陣。加安南國王陳益稷儀同三司。六月戊子,以莊浪巡檢司為莊浪縣,移巡檢司於北卜渡。癸巳,以米五千石賑大長公主所隸貧民。甲午,改繕珍司為徽儀使司,秩二品。己亥,歲星犯東咸。辛丑,置河南田賦總管府,隸內史府,設達魯花赤、總管、同知各一員,副總管二員,秩從三品。戊申,置勇校署,以角牴者隸之。庚戌,大同縣雨雹,大如雞卵。詔以駝馬牛羊分給朔方蒙古民戍守邊徼者,俾牧養蕃息以自贍,仍命議興屯田。壬子,賜大乾元寺鈔萬錠,俾營子錢,供繕修之費,仍升其提點所為總管府,給銀印,秩正三品。給鈔四十萬錠,賑合剌赤部貧民;三十萬錠,賑諸位怯憐口被災者;諸有俸祿及能自贍者勿給。癸丑,以羽林親軍萬人隸東宮。丙子,升廣惠司秩正三品,掌回回醫藥。丁丑,以濟寧等路水,遣官閱視其民,乏食者賑之,仍禁酒,開河泊禁,聽民採食。晉陽、西涼、鈞等州,陽翟、新鄭、密等縣大雨雹,汴梁、益都、般陽、濟南、東昌、東平、濟寧、泰安、高唐、濮州、淮安諸處大水。



 秋七月丙辰,緬國趙欽撒以方物來覲。來安路總管岑世興叛,據唐興州,賜璽書招諭之。諸王闊慳堅部貧乏,給糧賑之。壬戌,太陰犯心。以者連怯耶兒萬戶府軍萬人隸東宮,置右衛率府,秩正三品。丁卯,詔諭江西官吏、豪民毋沮撓茶課。甲戌,皇姊大長公主祥哥剌吉作佛事,釋全寧府重囚二十七人,敕按問全寧守臣阿從不法,仍追所釋囚還獄。命分簡奴兒流囚罪稍輕者,屯田肇州。乙亥,通州、



 漷州增置三倉。丙子,太白犯太微垣右執法。增置上都警巡院、開平縣官各二員。己卯,晉王也孫鐵木兒所部民,經剽掠災傷,為盜者眾,敕扎魯忽赤囊加帶往,與晉王內史審錄罪囚,重者就啟晉王誅之,當流配者加等杖之。庚辰,賜木憐、麥該兩驛鈔一萬二千一百二十錠,俾市馬給驛。辛巳,賜左右鷹坊及合剌赤等貧乏者鈔一十四萬錠。



 八月甲申,以河東山西道宣慰使張思明為中書參知政事。乙酉,熒惑犯輿鬼。甲午,以授皇太子玉冊,告祭於南郊。庚子,車駕至自上都。丁未,告祭於太廟。是月,伏羌縣山崩。



 閏八月丙辰,辰星犯太微垣右執法。賜嘉王晃火鐵木兒部羊十萬、馬萬匹。庚申,增置興和路預備倉,秩正八品;升廣盈庫從八品。癸亥,熒惑犯軒轅。甲子,太陰犯壘壁陣。浚會通河。壬申,以太傅、御史大夫伯忽為太師。癸酉,敕:「河東山西道宣慰司官,給俸同隨朝。」敕:「諸司有受命不之官及避繁劇托故去職者,奪其宣敕。」乙亥,太白犯東咸。並永興縣入奉聖州。



 九月甲申,以徽政使朵帶為太傅,升參議中書省事欽察為參知政事。辛卯,鐵裡乾等二十八驛被災,給鈔賑之。壬辰,翽星於司天臺。癸巳,以作佛事,釋大闢囚七人,流以下囚六人。戊戌,增海漕十萬石。置雲南縣,隸云內州。以故昌州寶山縣置寶昌州,隸興和路。庚子,並順德、廣平兩鐵冶提舉司為順德廣平彰德等處鐵冶提舉司。癸卯,御史臺臣言:「比者官以幸求,罪以賂免,乞凡內外官非勛舊有資望者,不許驟升。諸犯贓罪已款伏及當鞫而幸免者,悉付元問官以竟其罪;其貪污受刑,奪職不敘者,夤緣近侍,出入內庭,覬幸名爵,宜斥逐之。」帝皆納其言。詔謂四宿衛嘗受刑者,勿令造禁廷。山東諸路禁酒。浚鎮江練湖。發粟賑濟寧、東平、東昌、高唐、德州、濟南、益都、般陽、揚州等路饑。十月甲寅,省都功德使四員,止存六員。乙卯,東平、濟寧路水陸十五驛乏食,戶給麥十石。中書省臣言:「白云宗總攝沈明仁,強奪民田二萬頃,誑誘愚俗十萬人,私賂近侍,妄受名爵,已奉旨追奪,請汰其徒,還所奪民田。其諸不法事,宜令核問。」有旨:「朕知沈明仁奸惡,其嚴鞫之。」戊午,遣中書右丞相伯答沙持節授皇太子玉冊。辛酉,以扎魯忽赤鐵木兒不花為御史大夫。癸亥,熒惑犯太微垣左執法。上都民饑,發官粟萬石減價賑糶。置兩浙鹽倉六所,秩從八品,官二員,惟杭州、嘉興二倉設官三員,秩從七品;鹽場三十四所,場設監運一員,正八品。罷檢校所。乙丑,太陰犯昴。丁卯,賑北方諸驛。戊辰,太陰犯東井。庚午,太白晝見。辛未,太陰犯軒轅。丙子,以皇太子受玉冊,詔天下。己卯,浚通惠河。增河東、陜西鹽運司判官一員,給分司印二;置提領所二,秩從八品,官各二員;鹽場二,增管勾各二員;罷漉鹽戶提領二十人。濟南濱、棣州、章丘等縣水,免其田租。



 十一月辛卯,熒惑犯進賢。木邦路帶邦為寇,敕云南省招捕之。乙巳,以秘書卿苫思丁為大司徒。庚子,敕晉王部貧民二千居稱海屯田。增京畿漕運司同知、副使各一員,給分司印。中書省臣言:「曩賜諸王阿只吉鈔三萬錠,使營子錢以給畋獵廩膳,毋取諸民。今其部阿魯忽等出獵,恣索於民,且為奸事,宜令宗正府、刑部訊鞫之,以正典刑。」制曰:「可。」禁民匿蒙古軍亡奴。帝諭臺臣曰:「有國家者,以民為本。比聞百姓疾苦銜冤者眾,其令監察御史、廉訪司審察以聞。」河間民饑,發粟賑之。



 十二月壬戌,命皇太子參決國政。封宋儒周惇頤為道國公。甲子,遣宗正府扎魯忽赤二員,審決興和、平地等處獄囚。省雲南大理、大、小徹里等地同知、相副官及儒學、蒙古教授等官百二十四員。丙寅,太陰犯軒轅。己巳,復吏人出身舊制,其犯贓者止從七品。免大都、上都、興和延祐七年差稅。河西塔塔剌地置屯田,立軍民萬戶府。壬申,太陰犯星。平章政事王毅以親老辭職,從之,仍賜其父幣帛。癸酉,是夜風雪甚寒,帝謂侍臣曰:「朕與卿等居暖室,宗戚、昆弟遠戍邊陲,曷勝其苦!歲賜錢帛,可不遍及耶?」敕上都、大都冬夏設食於路,以食饑者。



 七年春正月辛巳朔,日有食之。帝齋居損膳,輟朝賀。壬午,御史臺臣言:「比賜不兒罕丁山場、完者不花海舶稅,會計其鈔,皆數十萬錠,諸王軍民貧乏者,所賜未嘗若是,茍不撙節,漸致帑藏虛竭,民益困矣。」中書省臣進曰:「臺臣所言良是,若非振理朝綱,法度愈壞。臣等乞賜罷黜,選任賢者。」帝曰:「卿等不必言,其各共乃事。」癸未,帝御大明殿,受諸王、百官朝賀。辛卯,江浙行省丞相黑驢言:「白雲僧沈明仁,擅度僧四千八百餘人,獲鈔四萬餘錠,既已辭伏,今遣其徒沈崇勝潛赴京師行賄求援,請逮赴江浙,並治其罪。」從之。乙未,太陰犯明堂上星。丁亥,帝不豫。辛丑,帝崩於光天宮,壽三十有六,在位十年。癸卯,葬起輦穀,從諸帝陵。五月乙未,群臣上謚曰聖文欽孝皇帝,廟號仁宗,國語曰普顏篤皇帝。



 仁宗天性慈孝,聰明恭儉,通達儒術,妙悟釋典,嘗曰:「明心見性,佛教為深;修身治國,儒道為切。」又曰:「儒者可尚,以能維持三綱五常之道也。」平居服御質素,澹然無欲,不事游畋,不喜征伐,不崇貨利。事皇太后,終身不違顏色;待宗戚勛舊,始終以禮。大臣親老,時加恩賚;太官進膳,必分賜貴近。有司奏大闢,每慘惻移時。其孜孜為治,一遵世祖之成憲云。



\end{pinyinscope}