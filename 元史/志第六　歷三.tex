\article{志第六 歷三}

\begin{pinyinscope}

 ○授時歷經上



 步氣朔第一



 至元十八年歲次辛巳為元。上考往古,下驗將來,皆距立元為算。周歲消長,百年各一,其諸應等數,隨時推測,不用為元。



 日周,一萬。



 歲實,三百六十五萬二千四百二十五分。



 通餘,五萬二千四百二十五分。



 朔實,二十九萬五千三百五分九十三秒。



 通閏,十萬八千七百五十三分八十四秒。



 歲周,三百六十五日二千四百二十五分。



 朔策,二十九日五千三百五分九十三秒。



 氣策,十五日二千一百八十四分三十七秒半。



 望策,十四日七千六百五十二分九十六秒半。



 弦策,七日三千八百二十六分四十八秒少。



 氣應,五十五萬六百分。



 閏應,二十萬一千八百五十分。



 沒限,七千八百一十五分六十二秒半。



 氣盈,二千一百八十四分三十七秒半。



 朔虛,四千六百九十四分七秒。



 旬周,六十萬。



 紀法,六十。



 推天正冬至



 置所求距算,以歲實上推往古,每百年長一;下算將來,每百年消一。乘之,為中積。加氣應,為通積。滿旬周,去之;不盡,以日周約之為日,不滿為分。其日命甲子算外,即所求天正冬至日辰及分。如上考者,以氣應減中積,滿旬周,去之;不盡,以減旬周。餘同上。



 求次氣



 置天正冬至日分,以氣策累加之,其日滿紀法,去之,外命如前,各得次氣日辰及分秒。



 推天正經朔



 置中積,加閏應,為閏積。滿朔實,去之不盡,為閏餘,以減通積,為朔積。滿旬周,去之;不盡,以日周約之,為日,不滿為分,即所求天正經朔日及分秒。上考者,以閏應減中積,滿朔實,去之不盡,以減朔實,為閏餘。以日周約之為日,不滿為分,以減冬至日及分,不及減者,加紀法減之,命如上。



 求弦望及次朔



 置天正經朔日及分秒,以弦策累加之,其日滿紀法,去之,各得弦望及次朔日及分秒。



 推沒日



 置有沒之氣分秒,如沒限已上為有沒之氣。以十五乘之,用減氣策,餘滿氣盈而一,為日,並恆氣日,命為沒日。



 推滅日



 置有滅之朔分秒,在朔虛分已下為有滅之朔。以三十乘之,滿朔虛而一,為日,並經朔日,命為滅日。



 步發斂第二



 土王策,三日四百三十六分八十七秒半。



 月閏,九千六十二分八十二秒。



 辰法,一萬。



 半辰法,五千。



 刻法,一千二百。



 推五行用事



 各以四立之節,為春木、夏火、秋金、冬水首用事日。以土王策減四季中氣,各得其季土始用事日。



 氣候



 正月



 立春,正月節東風解凍蟄蟲始振魚陟負冰



 雨水,正月中獺祭魚候雁北草木萌動



 二月



 驚蟄,二月節桃始華倉鶊鳴鷹化為鳩



 春分,二月中玄鳥至雷乃發聲始電



 三月



 清明,三月節桐始華田鼠化為釐虹始見



 穀雨,三月中萍始生鳴鳩拂其羽戴勝降於桑



 四月



 立夏,四月節螻蟈鳴蚯蚓出王瓜生



 小滿,四月中苦萊秀靡草死麥秋至



 五月



 芒種,五月節螳螂生鵙始鳴反舌無聲



 夏至,五月中鹿角解蜩始鳴半夏生



 六月



 小暑,六月節溫風至蟋蟀居壁鷹始摯



 大暑,六月中腐草為螢土潤溽暑大雨時行



 七月



 立秋,七月節涼風至白露降寒蟬鳴



 處暑,七月中鷹乃祭鳥天地始肅禾乃登



 八月



 白露,八月節鴻雁來玄鳥歸群鳥養羞



 秋分,八月中雷始收聲蟄蟲壞戶水始涸



 九月



 寒露,九月節鴻雁來賓雀入大水為蛤菊有黃華



 霜降,九月中豺乃祭獸草木黃落蟄蟲咸俯



 十月



 立冬,十月節水始冰地始凍雉入大水為蜃



 小雪,十月中虹藏不見天氣上升,地氣下降閉塞而成冬



 十一月



 大雪,十一月節鶡鴠不鳴虎始交荔挺出



 冬至,十一月中蚯蚓結麋角解水泉動



 十二月



 小寒,十二月節雁北鄉鵲始巢雉雊



 大寒,十二月中雞乳征鳥厲疾水澤腹堅



 推中氣去經朔



 置天正閏餘,以日周約之,為日,命之,得冬至去經朔。以月閏累加之,各得中氣去經朔日算。滿朔策,去之,乃全置閏,然俟定朔無中氣者裁之。



 推發斂加時



 置所求分秒,以十二乘之,滿辰法而一,為辰數;餘以刻法收之,為刻;命子正算外,即所在辰刻。如滿半辰法,通作一辰,命起子初。



 步日躔第三



 周天分,三百六十五萬二千五百七十五分。



 周天,三百六十五度二十五分七十五秒。



 半周天,一百八十二度六十二分八十七秒半。



 象限,九十一度三十一分四十三秒太。



 歲差,一分五十秒。



 周應,三百一十五萬一千七十五分。



 半歲周,一百八十二日六千二百一十二分半。



 盈初縮末限,八十八日九千九十二分少。



 縮初盈末限,九十三日七千一百二十分少。



 推天正經朔弦望入盈縮歷



 置半歲周,以閏餘日及分減之,即得天正經朔入縮歷。冬至後盈,夏至後縮。



 以弦策累加之,各得弦望及次朔入盈縮歷日及分秒。滿半歲周去之,即交盈縮。



 求盈縮差



 視入歷盈者,在盈初縮末限已下,為初限,已上,反減半歲周,餘為末限;縮者,在縮初盈末限已下,為初限,已上,反減半歲周,餘為末限。其盈初縮末者,置立差三十一,以初末限乘之,加平差二萬四千六百,又以初末限乘之,用減定差五百一十三萬三千二百,餘再以初末限乘之,滿億為度,不滿退除為分秒。縮初盈末者,置立差二十七,以初末限乘之,加平差二萬二千一百,又以初末限乘之,用減定差四百八十七萬六百,餘再以初末限乘之,滿億為度,不滿退除為分秒,即所求盈縮差。



 又術:置入限分,以其日盈縮分乘之,萬約為分,以加其下盈縮積,萬約為度,不滿為分秒,亦得所求盈縮差。



 赤道宿度



 角十二一十亢九二十氐十六三十房五六十



 心六五十尾十九一十箕十四十



 右東方七宿,七十九度二十分。



 斗二十五二十牛七二十女十一三十五虛八九十五太



 危十五四十室十七一十壁八六十



 右北方七宿,九十三度八十分太。



 奎十六六十婁十一八十胃十五六十昴十一三十



 畢十七四十觜初五參十一一十



 右西方七宿,八十三度八十五分。



 井三十三三十鬼二二十柳十三三十星六三十



 張十七二十五翼十八七十五軫十七三十



 右南方七宿,一百八度四十分。



 右赤道宿次,並依新制渾儀測定,用為常數,校天為密。若考往古,即用當時宿度為準。



 推冬至赤道日度



 置中積,以加周應為通積,滿周天分,上推往古,每百年消一;下算將來,每百年長一。去之,不盡,以日周約之為度,不滿,退約為分秒。命起赤道虛宿六度外,去之,至不滿宿,即所求天正冬至加時日躔赤道宿度及分秒。上考者,以周應減中積,滿周天,去之;不盡,以減周天,餘以日周約之為度;餘同上。如當時有宿度者,止依當時宿度命之。



 求四正赤道日度



 置天正冬至加時赤道日度,累加象限,滿赤道宿次,去之,各得春夏秋正日所在宿度及分秒。



 求四正赤道宿積度



 置四正赤道宿全度,以四正赤道日度及分減之,餘為距後度;以赤道宿度累加之,各得四正後赤道宿積度及分。



 黃赤道率



 表略



 推黃道宿度



 置四正後赤道宿積度,以其赤道積度減之,餘以黃道率乘之,如赤道率而一;所得,以加黃道積度,為二十八宿黃道積度;以前宿黃道積度減之,為其宿黃道度及分。其秒就近為分。



 黃道宿度



 角十二八十七亢九五十六氐十六四十房五四十八



 心六二十七尾十七九十五箕九五十九



 右東方七宿,七十八度一十二分。



 斗二十三四十七牛六九十女十一十二虛九分空太



 危十五九十五室十八三十二壁九三十四



 右北方七宿,九十四度一十分太。



 奎十七八十七婁十二三十六胃十五八十一昴十一0八



 畢十六五十觜初0五參十二十八



 右西方七宿,八十三度九十五分。



 井三十一0三鬼二一十一柳十三星六三十一



 張十七七十九翼二十0九軫十八七十五



 右南方七宿,一百九度八分。



 右黃道宿度,依今歷所測赤道準冬至歲差所在算定,以憑推步。若上下考驗,據歲差每移一度,依術推變,各得當時宿度。



 推冬至加時黃道日度



 置天正冬至加時赤道日度,以其赤道積度減之,餘以黃道率乘之,如赤道率而一;所得,以加黃道積度,即所求年天正冬至加時黃道日度及分秒。



 求四正加時黃道日度



 置所求年冬至日躔黃赤道差,與次年黃赤道差相減,餘四而一,所得,加象限,為四正定象度。置冬至加時黃道日度,以四正定象度累加之,滿黃道宿次,去之,各得四正定氣加時黃道度及分。



 求四正晨前夜半日度



 置四正恆氣日及分秒,冬夏二至,盈縮之端,以恆為定。以盈縮差命為日分,盈減縮加之,即為四正定氣日及分。置日下分,以其日行度乘之,如日周而一;所得,以減四正加時黃道日度,各得四正定氣晨前夜半日度及分秒。



 求四正後每日晨前夜半黃道日度



 以四正定氣日距後正定氣日為相距日,以四正定氣晨前夜半日度距後正定氣晨前夜半日度為相距度,累計相距日之行定度,與相距度相減;餘如相距日而一,為日差;相距度多為加,相距度少為減。以加減四正每日行度率,為每日行定度;累加四正晨前夜半黃道日度,滿宿次,去之,為每日晨前夜半黃道日度及分秒。



 求每日午中黃道日度



 置其日行定度,半之,以加其日晨前夜半黃道日度,得午中黃道日度及分秒。



 求每日午中黃道積度



 以二至加時黃道日度距所求日午中黃道日度,為二至後黃道積度及分秒。



 求每日午中赤道日度



 置所求日午中黃道積度,滿象限,去之,餘為分後;內減黃道積度,以赤道率乘之,如黃道率而一;所得,以加赤道積度及所去象限,為所求赤道積度及分秒;以二至赤道日度加而命之,即每日午中赤道日度及分秒。



 黃道十二次宿度



 危,十二度六十四分九十一秒。入娵訾之次,辰在亥。



 奎,一度七十三分六十三秒。入降婁之次,辰在戌。



 胃,三度七十四分五十六秒。入大梁之次,辰在酉。



 畢,六度八十八分五秒。入實沈之次,辰在申。



 井,八度三十四分九十四秒。入鶉首之次,辰在未。



 柳,三度八十六分八十秒。入鶉火之次,辰在午。



 張,十五度二十六分六秒。入鶉尾之次,辰在巳。



 軫,十度七分九十七秒。入壽星之次,辰在辰。



 氐,一度一十四分五十二秒。入大火之次,辰在卯。



 尾,三度一分一十五秒。入析木之次,辰在寅。



 斗,三度七十六分八十五秒。入星紀之次,辰在醜。



 女,二度六分三十八秒。入玄枵之次,辰在子。



 求入十二次時刻



 各置入次宿度及分秒,以其日晨前夜半日度減之,餘以日周乘之,為實;以其日行定度為法;實如法而一,所得,依發斂加時求之,即入次時刻。



 步月離第四



 轉終分,二十七萬五千五百四十六分。



 轉終,二十七日五千五百四十六分。



 轉中,十三日七千七百七十三分。



 初限,八十四。



 中限,一百六十八。



 周限,三百三十六。



 月平行,十三度三十六分八十七秒半。



 轉差,一日九千七百五十九分九十三秒。



 弦策,七日三千八百二十六分四十八秒少。



 上弦,九十一度三十一分四十三秒太。



 望,一百八十二度六十二分八十七秒半。



 下弦,二百七十三度九十四分三十一秒少。



 轉應,一十三萬一千九百四分。



 推天正經朔入轉



 置中積,加轉應,減閏餘,滿轉終分,去之,不盡,以日周約之為日,不滿為分,即天正經朔入轉日及分。上考者,中積內加所求閏餘,減轉應,滿轉終,去之,不盡,以減轉終,餘同上。



 求弦望及次朔入轉



 置天正經朔入轉日及分,以弦策累加之,滿轉終,去之,即弦望及次朔入轉日及分秒。如徑求次朔,以轉差加之。



 求經朔弦望入遲疾歷



 各視入轉日及分秒,在轉中已下,為疾歷;已上,減去轉中,為遲歷。



 遲疾轉定及積度



 表略



 求遲疾差



 置遲疾歷日及分,以十二限二十分乘之,在初限已下為初限,已上覆減中限,餘為末限。置立差三百二十五,以初末限乘之,加平差二萬八千一百,又以初末限乘之,用減定差一千一百一十一萬,餘再以初末限乘之,滿億為度,不滿退除為分秒,即遲疾差。



 又術:置遲疾歷日及分,以遲疾歷日率減之,餘以其下損益分乘之,如八百二十而一,益加損減其下遲疾度,亦為所求遲疾差。



 求朔弦望定日



 以經朔弦望盈縮差與遲疾差,同名相從,異名相消,盈遲縮疾為同名,盈疾縮遲為異名。以八百二十乘之,以所入遲疾限下行度除之,即為加減差,盈遲為加,縮疾為減。以加減經朔弦望日及分,即定朔弦望日及分。若定弦望分在日出分已下者,退一日,其日命甲子算外,各得定朔弦望日辰。定朔幹名與後朔乾同者,其月大;不同者,其月小;內無中氣者,為閏月。



 推定朔弦望加時日月宿度



 置經朔弦望入盈縮歷日及分,以加減差加減之,為定朔弦望入歷,在盈,便為中積,在縮,加半歲周,為中積;命日為度,以盈縮差盈加縮減之,為加時定積度;以冬至加時日躔黃道宿度加而命之,各得定朔弦望加時日度。



 凡合朔加時,日月同度,便為定朔加時月度,其弦望各以弦望度加定積,為定弦望月行定積度,依上加而命之,各得定弦望加時黃道月度。



 推定朔弦望加時赤道月度



 各置定朔弦望加時黃道月行定積度,滿象限,去之,以其黃道積度減之,餘以赤道率乘之,如黃道率而一,用加其下赤道積度及所去象限,各為赤道加時定積度;以冬至加時赤道日度加而命之,各為定朔弦望加時赤道月度及分秒。象限已下及半周,去之,為至後;滿象限及三象,去之,為分後。



 推朔後平交入轉遲疾歷



 置交終日及分,內減經朔入交日及分,為朔後平交日;以加經朔入轉,為朔後平交入轉;在轉中已下,為疾歷;已上,去之,為遲歷。



 求正交日辰



 置經朔,加朔後平交日,以遲疾歷依前求到遲疾差,遲加疾減之,為正交日及分,其日命甲子算外,即正交日辰。



 推正交加時黃道月度



 置朔後平交日,以月平行度乘之,為距後度;以加經朔中積,為冬至距正交定積度;以冬至日躔黃道宿度加而命之,為正交加時月離黃道宿度及分秒。



 求正交在二至後初末限



 置冬至距正交積度及分,在半歲周已下,為冬至後;已上,去之,為夏至後。其二至後,在象限已下,為初限,已上,減去半歲周,為末限。



 求定差距差定限度



 置初末限度,以十四度六十六分乘之,如象限而一,為定差;反減十四度六十六分,餘為距差。以二十四乘定差,如十四度六十六分而一;所得,交在冬至後名減,夏至後名加,皆加減九十八度,為定限度及分秒。



 求四正赤道宿度



 置冬至加時赤道度,命為冬至正度;以象限累加之,各得春分、夏至、秋分正積度;各命赤道宿次去之,為四正赤道宿度及分秒。



 求月離赤道正交宿度



 以距差加減春秋二正赤道宿度,為月離赤道正交宿度及分秒。冬至後,初限加,末限減,視春正;夏至後,初限減,末限加,視秋正。



 求正交後赤道積度入初末限



 各置春秋二正赤道所當宿全度及分,以月離赤道正交宿度及分減之,餘為正交後積度;以赤道宿次累加之,滿象限去之,為半交後;又去之,為中交後;再去之,為半交後;視各交積度在半象已下,為初限;已上,用減象限,餘為末限。



 求月離赤道正交後半交白道舊名九道出入赤道內外度及定差



 置各交定差度及分,以二十五乘之,如六十一而一;所得,視月離黃道正交在冬至後宿度為減,夏至後宿度為加,皆加減二十三度九十分,為月離赤道後半交白道出入赤道內外度及分;以周天六之一,六十度八十七分六十二秒半,除之,為定差。月離赤道正交後為外,中交後為內。



 求月離出入赤道內外白道去極度



 置每日月離赤道交後初末限,用減象限,餘為白道積;用其積度減之,餘以其差率乘之;所得,百約之,以加其下積差,為每日積差;用減周天六之一,餘以定差乘之,為每日月離赤道內外度;內減外加象限,為每日月離白道去極度及分秒。



 求每交月離白道積度及宿次



 置定限度,與初末限相減相乘,退位為分,為定差;正交、中交後為加,半交後為減。以差加減正交後赤道積度,為月離白道定積度;以前宿白道定積度減之,各得月離白道宿次及分。



 推定朔弦望加時月離白道宿度



 各以月離赤道正交宿度距所求定朔弦望加時月離赤道宿度,為正交後積度;滿象限,去之,為半交後;又去之,為中交後;再去之,為半交後;視交後積度在半象已下,為初限;已上,用減象限,為末限;以初末限與定限度相減相乘,退位為分,分滿百為度,為定差;正交中交後為加,半交後為減。以差加減月離赤道正交後積度,為定積度,以正交宿度加之,以其所當月離白道宿次去之,各得定朔弦望加時月離白道宿度及分秒。



 求定朔弦望加時及夜半晨昏入轉



 置經朔弦望入轉日及分,以定朔弦望加減差加減之,為定朔弦望加時入轉;以定朔弦望日下分減之,為夜半入轉;以晨分加之,為晨轉;昏分加之,為昏轉。



 求夜半月度



 置定朔弦望日下分,以其入轉日轉定度乘之,萬約為加時轉度,以減加時定積度,餘為夜半定積度;依前加而命之,各得夜半月離宿度及分秒。



 求晨昏月度



 置其日晨昏分,以夜半入轉日轉定度乘之,萬約為晨昏轉度;各加夜半定積度,為晨昏定積度;加命如前,各得晨昏月離宿度及分秒。



 求每日晨昏月離白道宿次



 累計相距日數轉定度,為轉積度;與定朔弦望晨昏宿次前後相距度相減,餘以相距日數除之,為日差;距度多為加,距度少為減。以加減每日轉定度,為行定度;以累加定朔弦望晨昏月度,加命如前,即每日晨昏月離白道宿次。朔後用晨,望後用昏,朔望晨昏俱用。



\end{pinyinscope}