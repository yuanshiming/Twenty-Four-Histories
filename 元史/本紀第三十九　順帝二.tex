\article{本紀第三十九 順帝二}

\begin{pinyinscope}

 二年春正月壬戌,太陰犯右執法。甲子,太陰犯角宿。乙丑,宿松縣地震,山裂。丁卯,太陰犯房宿。是月,置都水庸田使司於平江。



 二月戊寅朔,祭社稷。辛巳,太陰犯昴宿。甲申,太白經天。戊子,詔以世祖所賜王積翁田八十頃還其子都中。初,積翁齎詔諭日本,死於王事,嘗受賜,後收入官,故復賜之。己丑,立穆陵關巡檢司。壬辰,日赤如赭,乙未、丙申,復如之。丁酉,追尊帝生母邁來迪為貞裕徽聖皇后。庚子,分衡州路衡陽縣,立新城縣。進封宣靖王買奴為益王。甲辰,宗王也可札魯忽赤添孫薨,賜鈔一百錠以葬。乙巳,詔賞勞廣海征徭將卒,有官者升散階,歿於王事者優加褒贈,金山甘肅兵士在逃者,聽復業,免其罪。



 三月戊申,以阿里海牙家藏書畫賜伯顏。甲寅,以按灰為大宗正府也可札魯忽赤,總掌天下奸盜詐偽。丁巳,以累朝御服硃衣、七寶項牌賜伯顏。庚申,日赤如赭,壬戌,復如之。賜征東元帥府軍士冬衣及甲。諸軍討廣西徭,久無功,敕行省、行臺、廉訪司官共督之。順州民饑,以鈔四千錠賑之。夜,太陰犯星宿。癸亥,日赤如赭。甲子,太陰犯箕宿。乙丑,太陰犯南斗。賜宗王火兒灰母答里鈔一千錠。以撒敦上都居第賜太保定住,仍敕有司籍撒敦家財。甲戌,復四川鹽井之禁。以按答木兒家人田宅賜太保定住。以汪家奴為宣政院使,加金紫光祿大夫。造武宗、英宗、明宗三朝皇后玉冊、玉寶。是月,陜西暴風,旱,無麥。



 夏四月丁丑朔,日赤如赭。禁民間私造格例。戊寅,封駙馬孛羅帖木兒為毓德王。丙戌,太陰犯角宿。丁亥,禁服麒麟、鸞鳳、白兔、靈芝、雙角五爪龍、八龍、九龍、萬壽、福壽字、赭黃等服。庚寅,以知樞密院事帖木兒不華為中書平章政事,撒迪為御史大夫。甲午,遣使以香、幣賜武當、龍虎二山。詔以太平路為郯王徹徹禿食邑;以集慶、廬州、饒州禿禿哈民戶賜伯顏,仍於句容縣設長官所領之。戊戌,車駕時巡上都。拜中書左丞耿煥為侍御史,王懋德為中書左丞。賜宗室灰裏王金一錠、鈔一千錠,毓德王孛羅帖木兒鈔三千錠,公主八八鈔二千錠。



 五月丙午朔,黃河復於故道。庚戌,太陰犯靈臺。乙卯,南陽、鄧州大霖雨,自是日至於六月甲申,湍河、白河大溢,水為災。丙辰,太白晝見,丁巳,亦如之。壬申,秦州山崩。是月,婺州不雨,至於六月。



 六月丁丑,禁諸王、駙馬從衛服只孫衣,系絳環。贈宗王忽都答兒為雲安王,謚忠武;羅羅歹為保寧王,謚昭勇。庚辰,命中書平章政事阿吉剌知經筵事。戊子,以鐵木兒補化為江浙行省左丞相。太白犯井宿。辛卯,以汴梁、大名諸路脫別臺地土賜伯顏。禮部侍郎忽裏臺請復科舉取士之制,不聽。庚子,涇水溢。辛丑,以鈔五千錠賜吳王搠失江。



 秋七月丙午,詔以公主奴倫引者思之地五千頃賜伯顏。以衛輝路賜衛王寬徹哥為食邑。己酉,太白犯鬼宿。庚戌,以定住、鎖南參議中書省事。壬子,發阿魯哈、不蘭奚駱駝一百一十上供太皇太后乘輿之用。乙卯,太白犯熒惑。庚申,禁隔越中書口傳敕旨,冒支錢糧。甲子,命有司以所籍撒敦寶器分賜伯顏及太保定住。乙丑,中書平章政事孛羅徙宅,賜金二錠、銀十錠。庚午,敕賜上都孔子廟碑,載累朝尊崇之意。省諸王、公主、駙馬從衛糧賜之數。癸酉,命宗王不蘭奚,駙馬月魯不花、帖古思、教化鎮薛連可、怯魯連之地,各賜鈔六百錠及銀牌遣之。是月,黃州蝗,督民捕之,人日五斗。以鈔二千錠賑新收阿速軍扈從車駕者,每戶鈔二錠,死者人一錠。



 八月甲戌朔,日有食之。高郵大雨雹。詔:「雲南、廣海、八番及甘肅、四川邊遠官死而不能歸葬者,有司給糧食舟車護送還鄉。去鄉遠者,加鈔二十錠;無親屬者,官為瘞之。」命威順王寬徹不花還鎮湖廣。先是,伯顏矯制召之至京,至是帝遣歸籓。戊寅,祭社稷。大都至通州霖雨,大水,敕軍人修道。己卯,太陰犯心宿。辛巳,太陰犯箕宿。辛卯,以徽政院、中政院財賦府田租六萬三千三百石補本年海運未敷之數,令有司歸其直。壬辰,立屯衛於馬札罕之地。庚子,詔:「強盜皆死,盜牛馬者劓,盜驢騾者黥額,再犯劓,盜羊豕者墨項,再犯黥,三犯劓;劓後再犯者死。盜諸物者,照其數估價。省、院、臺、五府官三年一次審決。著為令。」辛丑,減馬湖路泥溪、平夷、蠻夷、夷都、沐川、雷坡六長官司,並為三。九月庚戌,熒惑犯太微垣。癸亥,弛鞏昌總帥府漢人軍器之禁。戊辰,車駕還自上都。海運糧至京,遣官致祭天妃。是月,臺州路饑,發義倉、募富人出粟賑之。沅州路盧陽縣饑,賑糶米六千石。



 冬十月丙子,熒惑犯左執法。己卯,享於太廟。丙申,命參知政事納麟監繪明宗皇帝御容。丁酉,太陰犯昴宿。己亥,詔:「每日,右丞相伯顏,太保定住,中書平章政事孛羅、阿吉剌聚議於內廷。平章政事塔失海牙,右丞鞏卜班,參知政事納麟、許有壬等聚議於中書。」太陰犯進賢。是月,撫州、袁州、瑞州諸路饑,發米六萬石賑糶之。



 十一月己酉,太陰犯壘壁陣。壬子,以那海為湖廣行省平章政事,討廣西叛徭。武宗、英宗、明宗三朝皇后升祔入廟,命官致祭。丁巳,遣河南行省平章政事玥珞普華於西番為僧。己未,太陰犯壘壁陣。辛酉,賜宣讓王帖木兒不花市宅錢四千錠,詔帖木兒不花王府官屬,朝賀班次列於有司之右。壬戌,命同知樞密院事者燕不花兼宮相都總管府達魯花赤,領隆鎮衛、左阿速衛諸軍。癸亥,安置宗王不蘭奚於梧州。丁卯,太陰犯房宿。辛未,禁彈弓、弩箭、袖箭。壬申,國公買住卒,賜鈔三百錠。印造至元三年鈔本一百五十萬錠。是月,松江府上海縣饑,發義倉糧及募富人出粟賑之。安豐路饑,賑糶麥四萬二千四百石。



 十二月甲戌,日赤如赭。丙子,命興元府鳳州留壩鎮及晉寧路遼山縣十八盤各立巡檢司。宗王也孫帖木兒進西馬三匹。賜文濟王蠻子金印、驛券及從衛者衣並糧五千石。詔省、院、臺、翰林、集賢、奎章閣、太常禮儀院、禮部官定議寧宗皇帝尊謚、廟號。是月,江州諸縣饑,總管王大中貸富人粟以賑貧民,而免富人雜徭以為息,約年豐還之,民不病饑。慶元慈溪縣饑,遣官賑之。是歲,詔整治驛傳。以甘肅行省白城子屯田之地賜宗王喃忽裏。以燕鐵木兒居第賜灌頂國師曩哥星吉,號大覺海寺,塑千佛於其內。江浙旱,自春至於八月不雨,民大饑。



 三年春正月癸卯,廣州增城縣民硃光卿反,其黨石昆山、鐘大明率眾從之,偽稱大金國,改元赤符。命指揮狗札里、江西行省左丞沙的討之。戊申,大都南北兩城設賑糶米鋪二十處。辛亥,升祔懿璘只班皇帝於廟,謚沖聖嗣孝皇帝,廟號寧宗。豫王阿剌忒納失裡買池州銅陵產銀地一所,請用私財煆煉,輸納官課,從之。癸丑,立宣鎮侍衛屯田萬戶府於寧夏。丙辰,月食。丁巳,日有交暈,左右珥上有白虹貫之。戊午,帝獵於柳林,凡三十五日。監察御史醜的、宋紹明進諫,帝嘉納之,賜金、幣。醜的等固辭,帝曰:「昔魏征進諫,唐太宗未嘗不賞,汝其受之。」是月,臨江路新淦州、新喻州,瑞州民饑,賑糶米二萬石。封晉郭璞為靈應侯。



 二月壬申朔,日有食之。棒胡反於汝寧信陽州。棒胡本陳州人,名閏兒,以燒香惑眾,妄造妖言作亂,破歸德府鹿邑,焚陳州,屯營於杏岡,命河南行省左丞慶童領兵討之。紹興路大水。丙子,立船戶提舉司十處,提領二十處。定船戶科差,船一千料之上者,歲納鈔六錠,以下遞減。壬午,以上太皇太后玉冊、玉寶,恭謝太廟。甲申,定服色、器皿、輿馬之制。己丑,汝寧獻所獲棒胡彌勒佛、小旗、偽宣敕並紫金印、量天尺。辛卯,發鈔四十萬錠,賑江浙等處饑民四十萬戶,開所在山場、河泊之禁,聽民樵採。廣西徭賊復反,命湖廣行省平章那海、江西行省平章禿兒迷失海牙總兵捕之。丙申,太保定住薨,給賜殯葬諸物。庚子,中書參知政事納麟等請立採珠提舉司。先是,常立提舉司,泰定間以其煩擾罷去,至是納麟請復立之,且以採珠戶四萬賜伯顏。是月,發義倉米賑蘄州及紹興饑民。



 三月辛亥,太陰犯靈臺。發鈔一萬錠,賑大都寶坻饑民。戊午,以玉寶、玉冊立弘吉剌氏伯顏忽都為皇后,因雨輟賀。詔以完者帖木兒蘇州之田二百頃賜郯王徹徹禿。己未,大都饑,命於南北兩城賑糶糙米。癸亥,加封晉周處為英義武惠正應王。己丑,命宗王燕帖木兒為大宗正府札魯忽赤。是月,天雨線。發義倉糧賑溧陽州饑民六萬九千二百人。



 夏四月壬申,遣使降香於龍虎、三茅、閣皁諸山。癸酉,禁漢人、南人、高麗人不得執持軍器,凡有馬者拘入官。甲戌,有星孛於王良,至七月壬寅沒於貫索。皇后以受玉冊、玉寶,恭謝太廟。命伯顏領宣鎮侍衛軍,賜鈔三千錠,建宣鎮侍衛府。以太皇太后受冊、寶詔天下。己卯,車駕時巡上都。壬午,高麗王阿剌忒納失里朝賀還國,賜金一錠、鈔二千錠,從官賜與有差。辛卯,合州大足縣民韓法師反,自稱南朝趙王。太陰犯壘壁陣。丁酉,謚唐杜甫為文貞。己亥,惠州歸善縣民聶秀卿、譚景山等造軍器,拜戴甲為定光佛,與硃光卿相結為亂,命江西行省左丞沙的捕之。庚子,太白晝見。是月,詔:「省、院、臺、部、宣慰司、廉訪司及部府幕官之長,並用蒙古、色目人。禁漢人、南人不得習學蒙古、色目文字。」以米八千石、鈔二千八百錠賑哈剌奴兒饑民。龍興路南昌、新建縣饑,太皇太后發徽政院糧三萬六千七百七十石賑糶之。



 五月辛丑,民間訛言朝廷拘刷童男、童女,一時嫁娶殆盡。壬寅,太白犯鬼宿。癸卯,給平伐、都云定雲二處安撫司達魯花赤暗都剌等虎符。乙巳,以興州、松州民饑,禁上都、興和造酒。太陰犯軒轅。戊申,詔:「汝寧棒胡,廣東硃光卿、聶秀卿等,皆系漢人。漢人有官於省、臺、院及翰林、集賢者,可講求誅捕之法以聞。」太白晝見。壬子,太陰犯心宿。甲寅,詔哈八兒禿及禿堅帖木兒為太尉,各設僚屬幕官。西番賊起,殺鎮西王子黨兀班,立行宣政院,以也先帖木兒為院使,往討之。戊午,太白晝見。己未,太陰犯壘壁陣。辛酉,太白晝見。壬戌,命四川行省參知政事舉理等捕反賊韓法師。丁卯,彗星見於東北,大如天船星,色白,約長尺餘,彗指西南,至八月庚午始滅。六月庚午,太白經天,辛未、甲戌,復如之。乙亥,太白犯靈臺。戊寅,贈丞相安童推忠佐運開國元勛、東平忠憲王,於所封城內建立祠廟,官為致祭。己卯,太白經天。夜,太白犯太微垣。辛巳,大霖雨,自是日至癸巳不止。京師、河南、北水溢,御河、黃河、沁河、渾河水溢,沒人畜、廬舍甚眾。壬午,太白晝見,太陰犯鬥宿。癸未,設醮長春宮。丁亥,太白犯太微垣。戊子,加封文始尹真人為無上太初博文文始真君,徐甲為垂玄感聖慈化應御真君,庚桑子洞靈感化超蹈混然真君,文子通玄光暢升元敏秀真君,列子沖虛至德遁世游樂真君,莊子南華至極雄文弘道真君。己丑,太白晝見,庚寅,復如之,至七月辛酉方息。壬辰,彰德大水,深一丈。立高密縣濰川鄉景芝社巡檢司。



 秋七月己亥,漳河泛溢至廣平城下。賜鞏卜班西平王印。癸卯,車駕出獵。太白經天。乙巳,復如之。丙午,車駕幸失剌斡耳朵。太白復經天。丁未,車駕幸龍岡,灑馬乳以祭。戊申,召朵兒只國王入朝。庚戌,太白晝見。河南武陟縣禾將熟,有蝗自東來,縣尹張寬仰天祝曰:「寧殺縣尹,毋傷百姓。」俄有魚鷹群飛啄食之。壬子,車駕幸乾元寺。甲寅,太白經天。乙卯,懷慶水。庚申,詔:「除人命重事之外,凡盜賊諸罪,不須候五府官審錄,有司依例決之。」辛酉,太白晝見。壬戌,賜宗王桑哥八剌七寶系腰。太白經天。癸亥、甲子,復如之。是月,狗札里、沙的擒硃光卿,尋追擒石昆山、鐘大明。



 八月戊辰,祭社稷。遣使賑濟南饑民九萬戶。庚午,彗星不見,自五月丁卯始見,至是凡六十三日,自昴至房,凡歷一十五宿而滅。甲戌,太陰犯心宿。辛巳,京畿盜起。壬午,京師地大震,太廟梁柱裂,各室墻壁皆壞,壓損儀物,文宗神主及御床盡碎;西湖寺神御殿壁僕,壓損祭器。自是累震,至丁亥方止,所損人民甚眾。癸未,日有交暈,左右珥白虹貫之。河南地震。弛高麗執持軍器之禁,仍令乘馬。戊子,漢人鎮遏生番處,亦開軍器之禁。修理文宗神主並廟中諸物。是月,車駕至自上都。九月己亥,熒惑犯鬥宿。甲辰,太陰犯鬥宿。丁未,太陰犯壘壁陣。己酉,立皮貨所於寧夏,設提領使、副主之。立四川、湖廣江西、江浙行樞密院。文宗新主、玉冊及一切神御之物皆成,詔依典禮祭告。太陰犯壘壁陣。辛亥,太陰犯軒轅。丙寅,大都南北兩城添設賑糶米鋪五所。



 冬十月庚午,太白晝見。癸酉,日赤如赭。乙亥,命江浙行省丞相搠思監提調海運。丙子,太陰犯壘壁陣。壬午,太陰犯昴宿。丁亥,太白晝見,太陰犯鬼宿。庚寅,太白晝見,辛卯,亦如之,丙申,復如之。



 十一月丁酉,太白經天。戊戌,太白犯亢宿。己亥,太白經天。壬寅,太陰犯熒惑。癸卯,太陰犯壘壁陣。丙午,立屯田於雄州。丁未,填星犯鍵閉。辛亥,太陰犯五車。甲寅,太陰犯鬼宿。丙辰,太陰犯軒轅。丁巳,太白經天,太陰犯太微垣。詔脫脫木兒襲脫火赤荊王位,仍命其妃忽剌灰同治兀魯思事。戊午,太白經天。癸亥,發鈔萬五千錠,賑宣德等處地震死傷者。太白經天。甲子、乙丑,復如之。



 十二月己巳,享於太廟。歲星退犯天樽,填星犯罰星。甲戌,熒惑犯壘壁陣,太白犯東咸。乙亥,吏部仍設考功郎中、員外郎、主事各一員。庚辰,命阿魯圖襲廣平王爵。壬午,集賢大學士羊歸等言:「太上皇、唐妃影堂在真定玉華宮,每年宜於正月二十日致祭。」從之。丙戌,命阿速衛探馬赤軍屯田。是月,以馬札兒臺為太保,分樞密院鎮北邊。是歲,詔賜孝子靳昺碑。伯顏請殺張、王、劉、李、趙五姓漢人,帝不從。征西域僧伽剌麻至京師,號灌頂國師,賜玉印。



 四年春正月丙申,以地震,赦天下。詔:「內外廉能官,父母年七十無侍丁者,附近銓注,以便侍養。」以宣政院使不蘭奚年七十致仕,授大司徒,給全俸終身。癸卯,太白犯建星,甲辰,復如之。丙午,太陰犯五車。辛亥,太陰犯軒轅。己未,填星犯東咸。江浙海運糧數不足,撥江西、河南五十萬石補之。庚申,太陰入南斗,太白犯牛宿。辛酉,分命宗王乃馬歹為知行樞密院事。癸亥,印造鈔本百二十萬錠。是月,詔修曲阜孔子廟。



 二月丁卯,罷河南、江浙、湖廣江西、四川等處行樞密院。戊辰,祭社稷。庚午,車駕獵於柳林。戊寅,太陰犯軒轅。己卯,太陰犯靈臺。乙酉,奉聖州地震。是月,賑京師、河南、北被水災者。龍興路南昌州饑,以江西海運糧賑糶之。



 三月戊申,填星退犯東咸。辛酉,命中書平章政事阿吉剌監修《至正條格》。告祭南郊。以國王朵兒只為遼陽行省左丞相,宗王玉裡不花為知樞密院事,賜鈔一千錠、金一錠、銀十錠。



 夏四月辛未,京師天雨紅沙,晝晦。以探馬赤、只兒瓦歹為中書平章政事。癸酉,以脫脫為御史大夫。乙亥,命阿吉剌為奎章大學士兼知經筵事。己卯,車駕時巡上都。河南執棒胡至京師,誅之。癸巳,車駕薄暮至八里塘,雨雹,大如拳,其狀有小兒、環夬、獅、象、龜、卵之形。



 五月乙未,立五臺山等處巡檢司。庚戌,升兩淮屯田打捕總管府為正三品。甲寅,贈湖廣行省平章政事燕赤推誠翊戴安邊制勝功臣、太傅、開府儀同三司、上柱國,追封永平王,謚忠襄。辛酉,詔:「土番宣慰司軍士,許令乘馬,執兵器。」湖廣行省元領新化洞、古州、潭溪、龍里、洪州諸洞三百餘處,洞民六萬餘戶,分隸靖州,立敘南、橫江巡檢司。是月,命佛家閭為考功郎中,喬林為考功員外郎,魏宗道為考功主事,考較天下郡縣官屬功過。命阿剌吉復為中書平章政事。彰德獻瑞麥,一莖三穗。臨沂、費縣水,發米三萬石賑糶之。六月庚午,廣東廉訪司僉事恩莫綽言:「處決重囚,宜命五府官斟酌地里遠近,預選官分行各道,比到秋分時畢事。」從之。辛巳,袁州民周子旺反,僭稱周王,偽改年號,尋擒獲,伏誅。填星退犯鍵閉。壬午,立重慶路墊江縣。己丑,邵武路大雨,水入城郭,平地二丈。是月,信州路靈山裂。漳州路南勝縣民李志甫反,圍漳城,守將搠思監與戰,失利。詔江浙行省平章別不花,總浙閩、江西、廣東軍討之。



 秋七月壬寅,詔以伯顏有功,立生祠於涿州、汴梁。己酉,奉聖州地大震,損壞人民廬舍。丙辰,鞏昌府山崩,壓死人民。戊午,為伯顏立打捕鷹房諸色人戶總管府。



 八月癸亥朔,日有食之。戊辰,祭社稷。己巳,申取高麗女子及閹人之禁。贈伯顏察兒守誠佐治安惠世美功臣、太師、開府儀同三司、上柱國,追封奉元王,謚忠宣。辛未,宣德府地大震。癸酉,山東鹽運司於濟南歷城立濱洛鹽倉東西二場。丙子,京師地震,日二三次,至乙酉乃止。丁酉,白虹貫天。癸未,改宣德府為順寧府,奉聖州為保安州。贈太保曲出推忠翊運保寧一德功臣、太師、開府儀同三司、上柱國,追封廣陽王,謚忠惠。贈平章伯帖木兒宣忠濟美協誠正德功臣、太傅、開府儀同三司、上柱國,追封文安王,謚忠憲。甲申,雲南老告土官八那遣侄那賽齎象馬來朝,為立老告軍民總管府。是月,車駕還自上都。



 閏八月戊戌,日赤如赭,己亥,復如之。填星犯罰星,太陰犯鬥宿。壬寅,日赤如赭。庚戌,太陰犯鬥宿。乙卯,太陰犯鬼宿。



 九月丙寅,太陰犯鬥宿。戊辰,太白犯東咸。癸酉,奔星如杯大,色白,起自右旗之下,西南行,沒於近濁。甲申,太陰犯軒轅。乙酉,太陰犯靈臺。庚寅,日赤如赭,太白犯鬥宿。



 冬十月辛卯,享於太廟。辛亥,太陰犯酒旗。



 十一月丙寅,改英宗殿名昭融。丁卯,立紹熙府軍民宣撫都總使司,命御史大夫脫脫兼都總使,治書侍御史吉當普為副都總使,世襲其職。本府元領六州、二十縣、一百五十二鎮,國初,以其地荒而廢之;至是居民二十餘萬,故立府治之。乙巳,命平章政事孛羅領太常禮儀院使。熒惑犯氐宿。丁丑,太陰犯鬼宿。戊寅,太白犯壘壁陣。壬午,四川散毛洞蠻反,遣使賑被寇人民。



 十二月甲午,大都南城等處設米鋪二十,每鋪日糶米五十石,以濟貧民,俟秋成乃罷。戊戌,立邦牙等處宣慰司都元帥府並總管府。先是,世祖既定緬地,以其處雲南極邊,就立其酋長為帥,令三年一入貢,至是來貢,故立官府。庚子,熒惑犯房宿。壬寅,以宣徽使別兒怯不花為御史大夫。癸卯,太白經天,己酉,復如之。庚戌,加荊王脫脫木兒元德上輔廣中宣義正節振武佐運功臣之號。太白經天。辛亥,復如之。壬子,熒惑犯東咸。乙卯,太白犯外屏,太陰犯鬥宿。丙辰,太白經天。



\end{pinyinscope}