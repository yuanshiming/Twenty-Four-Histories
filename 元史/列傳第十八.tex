\article{列傳第十八}

\begin{pinyinscope}

 ○速哥



 速哥,蒙古人。父忽魯忽兒,國王木華黎麾下卒也。後更隸塔海、帖哥軍。以善馳馬,有口辯,慎重不洩,令佩銀符,常居軍中。奏白機務,往返未嘗失期。太宗以為才,賜名動哥居。詔:「動哥居奏事,朝至朝入奏,夕至夕入奏。」嘗出金盤龍袍及宮女賜之。憲宗時,以疾卒。速哥亦以壯勇居軍中,歲甲寅,憲宗命從都元帥帖哥火魯赤等入蜀。乙卯,萬戶劉七哥、阿剌魯阿力與宋兵戰巴州,失利,陷敵中。速哥馳入其軍,奪劉七哥等以歸。以功賜白金五十兩、馬二匹、紫羅圈甲一注。又從都元帥紐璘敗宋將劉整,破雲頂山城。紐璘受詔會涪州,至馬湖江,速哥以革為舟,夜渡江,至大獲山行在所,陳道梗失期,帝慰遣之。未幾,復自涪州入奏事,遇宋軍於三曹山,速哥眾僅百餘,奮兵疾戰,敗之,奪其器械旗鼓以歸。己未,宋兵攻涪州浮橋,部將火尼赤戰陷,速哥破圍出之。又以白事諸王穆哥所,復敗宋軍於三曹山,還至石羊,與劉整遇,復擊敗之。



 世祖即位,賜白金、弓刀、鞍勒。中統二年,賜銀符,命隸紐璘軍。至元二年,四川行省遣速哥招收降民,得三千餘人。三年,從行院帖赤戰九頂山。四年,行省也速帶兒署為本軍總管,從征瀘州,取瀘川。五年,立德州,以速哥為達魯花赤,擢陜西五路四川行省左右司員外郎。從也速帶兒入朝,賞賚加厚。七年,從也速帶兒敗宋軍於馬湖江。用平章政事賽典赤薦,遷行尚書省員外郎。九年,建都蠻叛,詔諸王奧魯赤及也速帶兒討之。速哥將千人為先鋒,破黎州火尾寨,攻連雲關,克之。軍至建都,戰於東山,斬其酋布庫。復與元帥八兒禿迎合剌軍於不魯思河,所過城邑皆下。十年,討碉樓諸蠻,襲破連環城,還敗宋軍於七盤山,闢新軍萬戶。



 十一年,賜虎符,真授管軍萬戶,領成都高哇哥等六翼及京兆新軍,教習水戰。也速帶兒進圍嘉定,速哥率舟師會平康城,修築懷遠等寨,守其要害。十二年,遣兵敗宋將昝萬壽於麻平。既而行樞密副使忽敦等軍至,與也速帶兒會於紅崖,遣速哥守龍壩。城中大震,宋將陳都統、鮮於團練率舟師遁。速哥追擊,溺死者不可勝計,遂與中使沈答罕徇下流諸城,紫雲、瀘、敘皆降。進圍重慶,速哥以所部兵鎮白水、馬湖江口。十三年,帝遣脫術、教化的持詔諭其守臣使降,不聽,乃分兵為五道,水陸並進攻之。眾軍不利,唯速哥獲戰艦三百艘,俘其眾百三十人。涪州守將遣書納降,速哥率千人往察其情偽。速哥至涪州,果降,遂入其城。重慶守臣張萬率眾來襲,速哥一日夜出兵凡與十八戰,斬首三百餘級,萬敗走。未幾,萬復以積兵三千人來攻,又戰敗之。十四年,行院闢為鎮守萬戶、嘉定總管府達魯花赤。時瀘州復叛,速哥從大軍討平之。重慶受圍久,其守將趙安開門出降,制置使張玨遁,速哥追破之,虜百餘人及其舟二十餘艘。以功授成都水軍萬戶,尋改重慶夔府等路宣撫、招討兩司軍民達魯花赤。十六年,除四川南道宣慰使,依前成都水軍萬戶,鎮重慶、夔、施、黔、忠、萬、雲、涪、瀘等州。



 十九年,亦奚不薛蠻叛,置順元等路軍民宣慰司,以速哥為宣慰使,經理諸蠻。二十四年,遷河東陜西等路萬戶府達魯花赤,播州宣撫賽因不花等赴闕請留之。降八番金竹等百餘寨,得戶三萬四千,悉以其地為郡縣,置順元路、金竹府、貴州以統之。東連九溪十八峒,南至交趾,西至雲南,咸受節制。二十九年,入朝,加都元帥,改河東陜西等處萬戶府達魯花赤。三十一年,僉書四川行樞密院事,詔開土番道。土番叛,以兵圍茂州,速哥率師敗之。元貞元年,行院罷,速哥家居數歲卒。



 子壽不赤,襲河東陜西等處萬戶府達魯花赤。



 ○囊加歹



 囊加歹,乃蠻人。曾祖不蘭伯,仕其國,位群臣之右。祖合折兒,管帳前軍,兼統國政,仕至太師。太祖平乃蠻,父麻察來歸。太宗命與察剌同總管蒙古、漢軍,由是從世祖伐宋,破阿里不哥於失門禿,從諸王哈必赤及闊闊歹平李璮,皆有功,賞賚甚厚,賜金符。後以子貴,贈太傅,追封梁國公,謚桓武。囊加歹幼從麻察習戰陣,有謀略,佩金符,為都元帥府經歷。從阿術圍襄陽,襄陽降,以功授漢軍千戶。從丞相伯顏攻復州,與宋人戰,敗宋兵於風波湖。渡江後,伯顏南攻鄂州,阿術北攻漢陽,分戰艦五十,囊加歹與張弘範等焚其蒙沖三千艘,兩城大恐,皆出降。伯顏軍次安慶。賈似道督師江上,遣宋京來請和。軍至池州,遣囊加歹偕宋京報似道。似道復遣阮思聰偕囊加歹至軍中,仍請議和。時暑雨方漲,世祖慮士卒不習水土,遣使令緩師。伯顏、阿術與諸將議,乘勢徑前,遂進軍至丁家洲,似道師潰,大軍次建康。帝聞囊加歹親與賈似道語,召赴闕,具陳其說,遣還諭旨於伯顏,以北邊未靖,勿輕入敵境,而大軍已入平江矣。宋使柳岳、夏士林、呂師孟、劉岊等踵至,皆命囊加歹同往報之。師逼臨安,復遣囊加歹入取降表、玉璽,徵宋將相文武百官出迎王師。宋主乃遣賈餘慶等同囊加歹以降表、玉璽至皋亭山,伯顏遣囊加歹馳獻世祖。還傳密旨,遷宋君臣北上。賜金符,授懷遠大將軍、安撫司達魯花赤。與阿剌罕、董文炳等取臺、溫、福州,尋領蒙古軍副都萬戶、江東道宣慰使,佩金虎符如故。擢江東道按察使,復為本道宣慰使,領萬戶如故。召為都元帥,管領通事軍馬,東征日本,未至而還。詔以元管出役軍與孛羅迷兒見管軍合為一翼,充萬戶,守建康。改賜三珠虎符,拜雲南行省參知政事,討金齒、緬國,得疾,召還京師。授南京等路宣慰使,改河南道宣慰使,特旨命襲父職為蒙古軍都萬戶。



 武宗在潛邸,囊加歹嘗從北征,與海都戰於帖堅古。明日又戰,海都圍之山上,囊加歹力戰決圍而出,與大軍會。武宗還師,囊加歹殿,海都遮道不得過,囊加歹選勇敢千人直前沖之,海都披靡,國兵乃由旭哥耳溫、稱海與晉王軍合。是役也,囊加歹戰為多,以疾而歸。成宗崩,昭獻元聖太后與仁宗在懷州,太后召囊加歹、不憐吉歹、脫因不花、八思臺等諭之曰:「今宮車晏駕,皇后欲立安西王阿難答,爾等當毋忘世祖、裕宗在天之靈,盡力奉二皇子。」囊加歹頓首曰:「臣等雖碎身,不能仰報兩朝之恩,願效死力。」既至京師,仁宗遣囊加歹與八思臺詣諸王禿剌議事宜。時內外洶洶,猶豫莫敢言,囊加歹獨贊禿剌,定計先發。歸白仁宗,意猶遲疑,固問可否,對曰:「事貴速成,後將受制於人矣。」太后與仁宗意乃決。內難既平,仁宗監國,命同知樞密院事。武宗即位,真拜同知樞密事,階資德大夫,賜以七寶束帶、鞍轡、衣甲、弓矢、黃金五十兩,以旌其定策之功。尋授蘄縣萬戶府達魯花赤,仍同知樞密院事。仁宗嘗語近臣曰:「今春之事,吾與太后疑不能主,賴囊加歹一語而定。吾聞周文王有姜太公,囊加歹亦予家姜太公也。」其見稱許如此。尋以老病乞骸骨,不允。仁宗即位,以其家河南,特授河南江北行省平章政事,佩金虎符,終其身。封浚都王。



 子教化,山東河北蒙古軍副都萬戶;執禮和臺,河南江北行省平章政事。孫脫堅,山東河北軍大都督,世襲有位。



 ○忙兀臺



 忙兀臺,蒙古達達兒氏。祖塔思火兒赤,從太宗定中原有功,為東平路達魯花赤,位在嚴實上。忙兀臺事世祖,為博州路奧魯總管。至元七年,又為監戰萬戶,佩金虎符。八年,改鄧州新軍蒙古萬戶,治水軍於萬山南岸。九月,以兵攻樊,拔古城,繼敗宋軍於安陽灘。轉戰八十里,禽其將鄭高。十月,大軍攻樊,分軍為五道,忙兀臺當其一。率五翼軍以進,焚南岸舟,豎雲梯於北岸,登櫃子城,奪西南角入城,命部將據倉粟。功在諸將右,賞金百兩。襄陽降,同宋安撫呂文煥入覲,賜銀五十兩及翎根甲等物。



 十一年,從丞相伯顏、平章阿術南征,命與萬戶史格率麾下會鹽山嶺。遇宋兵,忙兀臺突陣殺一人,諸軍繼進,與戰,敗之。自郢州黃家原蕩舟入湖,至沙洋堡,立砲座十有二,豎雲梯先登,焚其樓櫓,拔羊角壩,破沙洋堡,擒宋將四人。直抵新城,鏖戰自晨至晡,大敗之,宋復州守將翟貴以城降。將由漢口入江,至蔡店,聞宋兵屯漢口,乃率舟師經鬥龍口至沙步入江。遇宋兵三百餘艘分道來拒,進擊走之。次武磯堡,宋將夏貴堅守不下。十月乙卯,平章阿術率萬戶晏徹兒、史格、賈文備同忙兀臺四軍雪夜溯流西上,黎明至青山磯北岸,萬戶史格先渡,宋將程鵬飛拒敵,格被三創,喪卒二百人。諸將繼進,大戰中流,鵬飛被七創,敗走。舟泊中洲,宋兵阻水不得近,伯顏復遣萬戶張榮實等率舟來援。夏貴率麾下數千將奔,大軍乘之,大敗,走黃州,遂拔武磯堡,斬守將王達。阿術既渡南岸,翼日丞相伯顏視師,則大江南北皆北軍旗幟,宋制置使硃示異孫遁還江陵。語在《阿術傳》。己未,伯顏次鄂州,遣忙兀臺諭宋守臣張晏然以城降,程鵬飛以本軍降,知漢陽軍王儀、知德安府來興國繼降,乃留軍鎮鄂、漢,率諸將水陸東下。十二年正月,忙兀臺諭蘄、黃、安慶、池州諸郡,皆下之。次丁家洲,宋賈似道、孫虎臣來拒,忙兀臺擊之,奪虎臣所乘巨舟,與宋降將範文虎以兵五百諭降和州及無為、鎮巢二軍。九月,攻常州,拔其木城。宋降將趙潛叛於溧陽,伯顏命忙兀臺擊之,戰於豐登莊,斬首五百餘級,擒其將三人,復招降湖州守將二人。十二月,行省第其功,承制授行兩浙大都督府事。



 十四年,改閩廣大都督,行都元帥府事。時宋二王逃遁入海,忙兀臺奉旨率諸軍,與江西右丞塔出會兵收之,次漳州,諭降宋守將何清。十五年,師還福州,拜參知政事,詔與唆都等行省於福,鎮撫瀕海八郡。十月,召赴闕,升左丞。十六年七月,沙縣盜起,詔忙兀臺復行省事,討平之。初,忙兀臺北還,左丞唆都行省福建。一日,帝命召唆都,李庭言:「若召唆都,則行省無人,宜令建康阿剌罕往。」帝曰:「何必阿剌罕,其命忙兀臺即往,候唆都還,則令移潭州可也。」未幾,中書言:「唆都在福建,麾下擾民,致南劍等路往往殺長吏叛。及忙兀臺至,招來七十二寨,建寧、漳、汀稍獲安集,若移之他處,而唆都復往,恐重勞民。」有旨,忙兀臺仍鎮閩。十八年,轉右丞。時宣慰使王剛中以土人饒貲,頗擅作威福,忙兀臺慮其有變,奏移之他道。



 二十一年,拜江淮行省平章政事。初,宋降將五虎陳義嘗助張弘範擒文天祥,助完者都討陳大舉,又資阿塔海征日本戰艦三千艘。福建省臣言其有反側意,請除之。帝使忙兀臺察之。至是忙兀臺攜義入朝,保其無事,且乞寵以官爵,丞相伯顏亦以為言。乃授義同知廣東道宣慰司事,授明珠虎符,其從林雄等十人並上百戶。



 二十二年,脫忽思、樂實傳旨中書省,令悉代江浙省臣。中書復奏,帝曰:「朕安得此言,傳者妄也。如忙兀臺之通曉政事,亦可代耶?」俄以言者召赴闕,封其家貲,遣使按驗無狀。未幾,拜銀青榮祿大夫、行省左丞相,還鎮江浙。時浙西大饑,乃弛河泊禁,發府庫官貨,低其直,貿粟以賑之。浙東盜起,蠲田租,以紓民力。二十三年,奏:「以販鬻私鹽者皆海島民,今征日本,可募為水工。」從之,賜鈔五千貫。役既罷,請以戰艦付海漕。又言:「省治在杭州,其兩淮、江東財賦軍實,既南輸至杭,復自杭北輸京城,往返勞頓不便,請移省治於揚州。」復言:「淮東近地,宜置屯田,歲入糧以給軍,所餘餉京師。」帝悉從其言。二十五年,詔江淮管內,並聽忙兀臺節制。



 二十六年,朝廷以中原民轉徙江南,令有司遣還,忙兀臺言其不可,遂止。閩、越盜起,詔與不魯迷失海牙等合兵討之,御史大夫玉速帖木兒奏宜選將,帝曰:「忙兀臺已往,無慮也。」未幾,悉平之。屢以病,上疏乞骸骨,乃召還。二十七年,以江西平章奧魯赤不稱職,特命為丞相,兼樞密院事,出鎮江西。謹約束,鋤強暴,尊卑殊服,軍民安業,威德並著,在官四十日卒。



 忙兀臺之在江浙,專愎自用,又易置戍兵,平章不憐吉臺言其變更伯顏、阿術成法,帝每戒敕之。既死,臺臣劾郎中張斯立罪狀,而忙兀臺迫死劉宣及其屯田無成事,始聞於帝云。



 子三人:帖木兒不花;孛蘭奚,襲萬戶;亦剌出,中書參知政事。



 ○奧魯赤



 奧魯赤,札剌臺人。曾祖豁火察,驍果善騎射,太祖出征,每提精兵為前驅。祖朔魯罕,有膽力,嘗被讒不許入見,一日俟駕出,趨前曰:「臣無罪。若果有罪,速殺臣,臣將從先帝於地下;不然赦臣,願得自效。」帝笑而復用之。辛未,與金人戰於野孤嶺,中流矢,戰愈力,克之。既還,拔矢,血出昏眩。帝親撫視,傅以藥,竟不起。帝悲悼曰:「朔魯罕朕之一臂,今亡矣!」賜其家馬四百匹,錦綺萬段。父忒木臺,從太宗征杭里部,俘部長以獻。復從征西夏有功,特命行省事,領兀魯、忙兀、亦怯烈、弘吉剌、札剌兒五部軍。平河南,以功賜戶二千。嘗駐兵太原、平陽、河南,土人德之,皆為立祠。



 奧魯赤性樸魯,智勇過人。早事憲宗,帶御器械,特見親任。戊午,扈駕征蜀,攻釣魚山。至元五年,攻襄陽,授金符、蒙古軍萬戶。明年,賜虎符,襲父職,領蒙古軍四萬戶。十一年春,詔丞相伯顏大舉伐宋,以所部從,渡江圍鄂。宋兵固守,奧魯赤白丞相,可遣使諭降,乃遣許千戶同所獲宋將持金符抵其城東南門,懸金符以招之。其夜,守門將崔立啟門出,遂引立見丞相。復遣入城,諭守臣張晏然。明日,晏然以城降。遷奧魯赤昭毅大將軍,諸郡望風而靡。分兵出獨松關,宋兵堅守,奧魯赤令將校益樹旗幟於山上,率精騎突之,守兵驚潰,棄關走,追逐百餘里,斬馘不可勝計。



 十三年,宋主降,分討未下州郡,詔加鎮國上將軍、行中書省參知政事。未幾,以參知政事行湖北道宣慰使,兼領蒙古軍。時州郡初附,戍以重兵,民驚懼,往往逃匿山澤間。奧魯赤止侵暴,恤單弱,號令嚴明,民悉復業。會詔所在括逃俘,有司拘男女千餘人。時軍士已還部,所括者無所歸,眾議悉以隸官。奧魯赤曰:「斯民不幸被兵,幸而骨肉完聚,復羈之,是重被兵也,不若籍之為民。」眾從之。俄徵詣闕,賜賚優渥。及還,帝曰:「武昌襟帶江湖,實要害地。朕嘗用師於彼,故遣卿往治,為朕耳目。」升驃騎衛上將軍、中書左丞,行宣慰使。



 十八年,詔移行省於鄂、宣慰司於潭。時湖南劇賊周龍、張虎聚黨行劫,隨宜招捕,梟二賊首,餘悉縱遣。復召入見,拜行省右丞,改荊湖等處行樞密院副使。二十三年春,拜湖廣等處行中書省平章政事。夏四月,赴召上都,命佐鎮南王征交趾,帝慰諭之曰:「昔木華黎等戮力王室,榮名迄今不朽,卿能勉之,豈不並美於前人乎!」仍命其子脫桓不花襲萬戶。至交趾,啟王分軍為三,因險制變,蠻不能支,竄匿海島。餘寇扼師歸路,奧魯赤轉戰以出。改江西行省平章政事。二十六年,以疾求退,不允。俄授同知湖廣等處行樞密院事。成宗即位,進光祿大夫、上柱國、江西等處行中書省平章政事。大德元年春三月卒,年六十六。贈金紫光祿大夫、大司徒、上柱國,追封鄭國公,謚忠宣。



 子拜住,明威將軍、蒙古侍衛親軍副都指揮使;脫桓不花,驃騎衛上將軍、行中書省左丞、蒙古軍都萬戶。



 ○完者都



 完者都,欽察人。父哈剌火者,從憲宗征討有功。完者都廣顙豐頷,髯長過腹,為人驍勇,而樂善好施,聽讀史書,聞忠良則喜,遇奸諛則怒。歲丙辰,以材武從軍。己未,從攻鄂州,先登,賞銀五十兩。中統三年,從諸王合必赤討李璮於濟南,凡兩戰,皆有功。至元元年,合必赤因樞密臣以其武勇聞,帝特賞賜之。四年十月,從萬戶木花裏略地荊南,還至襄陽西安陽灘,遇宋軍,敗之。既而從丞相阿術圍襄樊,水陸大戰者四,皆有功。嘗梯樊城,焚樓櫓,勇敢出諸軍右,幕府上其功。十一年,授武略將軍,為彰德南京新軍千戶。九月,從丞相伯顏南征。十一月,攻沙洋、新城。始授金符,領丞相帳前合必赤軍。十二月,統舟師由沙蕪口渡江。十二年春,與宋將孫虎臣戰於丁家洲,大捷,進武義將軍。攻泰州,戰揚子橋,戰焦山,破常州。十三年春,入臨安,下揚州,皆有功。江南平,入見,帝顧謂侍臣曰:「真壯士也!」因賜名拔都兒,授信武將軍、管軍總管、高郵軍達魯花赤,佩虎符。既而軍升為路,遂進懷遠大將軍、高郵路總管府達魯花赤。



 十六年,授昭勇大將軍,遷管軍萬戶。漳州陳吊眼聚黨數萬,劫掠汀、漳諸路,七年未平。十七年八月,樞密副使孛羅請命完者都往討,從之,加鎮國上將軍、福建等處征蠻都元帥,率兵五千以往。賜翎根甲,面慰遣之,且曰:「賊茍就擒,聽汝施行。」時黃華聚黨三萬人,擾建寧,號頭陀軍。完者都先引兵鼓行壓其境,軍聲大震,賊驚懼納款。完者都許以為副元帥,凡征蠻之事,一以問之。且慮其奸詐莫測,因大獵以耀武。適有一雕翔空,完者都仰射之,應弦而落,遂大獵,所獲山積,華大悅服。乃聞於朝,請與之俱討賊,朝廷從之,制授華征蠻副元帥,與完者都同署。華遂為前驅,至賊所,破其五寨。十九年三月,追陳吊眼至千壁嶺,擒之,斬首漳州市,餘黨悉平。軍還至揚州,奉旨賞賜有差。至高郵,病。七月,入覲,帝嘉之,賜鈔及銀、金綺、鞍勒、弓矢,復授管軍萬戶、高郵路總管府達魯花赤。有虎為害,完者都挾弓矢出郊,射殺之。



 二十二年八月,以疾召入朝。帝屢遣中使存問,仍命良醫視之。疾平,帝大喜,賜醫者鈔萬貫,拜完者都驃騎上將軍、江浙行省左丞,兼管軍萬戶。初,浙西私鹽,吏莫能禁,完者都躬詣松江上海,收鹽徒五千,隸軍籍。九月,授中書左丞,行浙西道宣慰使。二十五年,遙授尚書省左丞。二十六年,升資德大夫、江西等處行樞密院副使,兼廣東宣慰使。疾復作,召還。成宗即位,入見,賜玉帶,授榮祿大夫、江浙行省平章政事。大德二年十一月卒,年五十九。贈效忠宣力定遠功臣、開府儀同三司、太尉、上柱國,追封林國公,謚武宣。



 子十四人,皆仕,而帖木禿古思、別里怯都尤顯。孫二十四人,仕者亦多云。



 ○伯帖木兒



 伯帖木兒,欽察人也。至元中,充哈剌赤,入備宿衛,以忠謹,授武節將軍,僉左衛親軍都指揮使司事。二十四年,征叛王乃顏,隸御史大夫玉速帖木兒麾下,敗乃顏兵於忽爾阿剌河,追至海剌兒河,又敗之。乃顏黨金家奴、別不古率眾走山前,從大夫追戰於札剌馬禿河,殺其將二人,追至夢哥山,並擒金家奴。二十五年,超授顯武將軍。冬,哈丹王叛,從諸王乃麻歹討之。至斡麻站、兀剌河等處,連敗其黨阿禿八剌哈赤軍,轉戰至帖麥哈必兒哈,又敗之。進至明安倫城,哈丹迎戰,敗走,追至忽蘭葉兒,又與阿禿一日三戰,手殺五人,擒裨將一人。至帖裏揭,突擊哈丹,挺身陷陣,身中三十餘箭而還。大夫親視其創,而罪潰軍之不救者。車駕親征,駐蹕兀魯灰河,伯帖木兒以兵從大夫至貴列兒河。哈丹拒王師,伯帖木兒首戰卻之,獲其黨駙馬阿剌渾,帝悅,以所獲賊將兀忽兒妻賜之。至霸郎兒,與忽都禿兒乾戰,殺其裨將五人,生擒曲兒先。九月,大夫令率師往納兀河東等處,招集逆黨乞答真一千戶、達達百姓及女直押兒撒等五百餘戶。二十六年春正月,師還,復遣戍也真大王之境。五月,海都謀擾邊,有旨令伯帖木兒以其軍先來。行至怯呂連河,值拜要叛,伯帖木兒即移兵致討,擒其黨伯顏以獻。帝深加獎諭,賜以所得伯顏女茶倫。是年冬,立東路蒙古軍上萬戶府,統欽察、乃蠻、捏古思、那亦勤等四千餘戶。升懷遠大將軍、上萬戶,佩三珠虎符。



 二十七年,哈丹復入高麗,伯帖木兒奉命偕徹里帖木兒進討。二十八年正月,至鴨綠江,與哈丹子老的戰,失利。伯帖木兒以聞,帝命乃麻歹、薛徹干等征之,仍命伯帖木兒為先鋒。薛徹干軍先至禪定州,擊敗哈丹,逾數日,乃麻歹以兵至,合攻哈丹,又敗之。伯帖木兒將百騎追至一大河,虜其妻孥,追奔逐北。哈丹尚有八騎,伯帖木兒止餘三騎,再戰,兩騎士皆重傷不能進,伯帖木兒單騎追之。至一大山,日暮,遂失哈丹所在。乃麻歹嘉其勇,賞以老的妻完者,上其功於朝,賜金帶、衣服、鞍馬、弓矢、銀器等物,並厚賚其軍。二十九年,聞叛王捏怯烈尚在濠來倉,伯帖木兒率兵擊,虜其妻子畜產,追至陳河,捏怯烈以二十餘騎脫身走,遂定其地。得所管女直戶五百餘以聞,帝命以充漁戶。伯帖木兒度地置馬站七所,令歲捕魚,馳驛以進。成宗即位,俾仍其官。車駕幸上京,徵其兵千人從,歲以為常云。



 ○懷都



 懷都,斡魯納臺氏。祖父阿術魯,與太祖同飲黑河水,屢從征討,賜銀印,總大軍伐遼東女直諸部。復帥師討西夏,大戰於合剌合察兒,擒夏主,太祖命盡賜以夏主遺物。繼總軍南伐,攻拔信安,下宿、泗等州。諸王塔察兒以阿術魯年老,俾其子不花襲職。中統二年,不花卒,子幼,兄子懷都繼領其職。



 中統三年春,李璮叛,詔懷都從親王哈必赤討之,圍璮濟南。夏四月,璮夜出兵,四面沖突求出。懷都直前奮擊,斬百餘級,俘二百餘人,奪兵仗數百。璮退走入城,懷都晝夜勒兵與戰。秋七月,破濟南,誅璮。哈必赤第其功,居最,詔賜金虎符,領蒙古、漢軍。攻海州,略淮南廬州。至元三年,充邳州監戰萬戶。四年,領山東路統軍司,從主帥南征。至襄陽,西渡漢江,宋遣水軍絕歸路,懷都選士卒浮水殺宋軍,奪戰艦二十餘艘,斬首千餘級。六年,軍次淮南天長,至五河口,與宋兵戰,敗之。七年,詔守鹿門山、白河口、一字城。九年春,懷都請攻樊之古城堡。堡高七層,懷都夜勒士卒,親冒矢石,攻奪之,斬宋將韓撥發,擒蔡路鈐。襄陽既降,帥師屯蔡、息,出巡淮安,還城正陽,略地安豐,獲生口無算。



 十一年夏,宋將夏貴來攻正陽,懷都領步卒薄淮西岸,至橫河口,逆戰退之。九月,略地安慶。十二年,北渡,至柵江堡,值宋軍三千餘,懷都與戰,敗之。復南渡江,駐兵鎮江。諜報宋平江軍出常州,懷都領兵千人,至無錫,與宋兵遇,大戰,殲其眾。秋七月,行省檄懷都領軍護焦山江岸,仍往揚州灣頭立木城,以兵守之。九月,權樞密院事,復守鎮江。宋殿帥張彥、安撫劉師勇攻呂城,懷都與萬戶忽剌出、帖木兒追戰至常州,奪舟百餘艘,擒張殿帥、範總管。冬十月,從右丞阿塔海攻常州。宋硃都統自蘇州赴援,懷都提兵至橫林店,與之遇,奮擊,大破之。十一月,取蘇州,徇秀州,仍撫治臨安迤東新附軍民。十三年秋,同元帥撒里蠻、帖木兒、張弘範徇溫州、福建,所至州郡迎降。十四年,授鎮國上將軍、浙東宣慰使。討臺、慶叛者,戰於黃奢嶺,又戰於溫州白塔屯寨,轉戰至於漳、泉、興化,平之。十六年,召至闕下,賜玉帶、弓矢,授行省參知政事,至處州,以疾卒。



 子八忽臺兒,官至通奉大夫、浙東道宣慰使都元帥,平浙東、建寧盜賊,數有功。不花子忽都答兒既長,分襲蒙古軍千戶,從平宋有功,授浙西招討使,改邳州萬戶,後加榮祿大夫、平章政事,卒。



 ○亦黑迷失



 亦黑迷失,畏吾兒人也。至元二年,入備宿衛。九年,奉世祖命使海外八羅孛國。十一年,偕其國人以珍寶奉表來朝,帝嘉之,賜金虎符。十二年,再使其國,與其國師以名藥來獻,賞賜甚厚。十四年,授兵部侍郎。十八年,拜荊湖占城等處行中書參知政事,招諭占城。二十一年,召還。復命使海外僧迦剌國,觀佛缽舍利,賜以玉帶、衣服、鞍轡。二十一年,自海上還,以參知政事管領鎮南王府事,復賜玉帶。與平章阿里海牙、右丞唆都征占城,戰失利,唆都死焉。亦黑迷失言於鎮南王,請屯兵大浪湖,觀釁而後動。王以聞,詔從之,竟全軍而歸。二十四年,使馬八兒國,取佛缽舍利,浮海阻風,行一年乃至。得其良醫善藥,遂與其國人來貢方物,又以私錢購紫檀木殿材並獻之。嘗侍帝於浴室,問曰:「汝逾海者凡幾?」對曰:「臣四逾海矣。」帝憫其勞,又賜玉帶,改資德大夫,遙授江淮行尚書省左丞,行泉府太卿。



 二十九年,召入朝,盡獻其所有珍異之物。時方議征爪哇,立福建行省,亦黑迷失與史弼、高興並為平章。詔軍事付弼,海道事付亦黑迷失,仍諭之曰:「汝等至爪哇,當遣使來報。汝等留彼,其餘小國即當自服,可遣招徠之。彼若納款,皆汝等之力也。」軍次占城,先遣郝成、劉淵諭降南巫里、速木都剌、不魯不都、八剌剌諸小國。三十年,攻葛郎國,降其主合只葛當。又遣鄭珪招諭木來由諸小國,皆遣其子弟來降。爪哇主婿土罕必闍耶既降,歸國復叛,事並見《弼傳》。諸將議班師,亦黑迷失欲如帝旨,先遣使入奏,弼與興不從,遂引兵還,以所俘及諸小國降人入見。帝罪其與弼縱土罕必闍耶,沒家貲三之一。尋復還之。以榮祿大夫、平章政事為集賢院使,兼會同館事,告老家居。仁宗念其屢使絕域,詔封吳國公,卒。



 ○拜降



 拜降,北庭人。父忽都,武勇過人,由宿衛為南宿州鎮將,分守蘄縣。後從世祖南征,年幾七十,每率先士卒,冒矢石,身被數十瘡,戰功居多。徙居大名路清豐縣,卒。贈廣平路總管,封漁陽郡侯。忽都卒時,拜降生甫數月,母徐氏鞠育教誨甚至,每曰:「吾惟一子,已童丱矣,不可使不知學。」顧縣僻左,無良師友,遂遣從師大名城中。郡守每旦望入學,見拜降容止講解,大異群兒,甚愛獎之。比弱冠,美髭髯,儀表甚偉。



 丞相阿術南攻襄陽、江陵諸郡,以偏裨隸麾下。軍行至安陽灘,與宋軍遇。宋騎直前突陣,陣為卻。拜降躍馬出陣前,引弓連斃數人,宋騎稍卻。復率眾戰良久,宋師大潰。至元五年,圍襄樊,戰有功。十一年,從阿術渡江,水陸遇敵,嘗先登陷陣,勇冠一軍。宋平,以功授江浙省理問官。時事方草創,省臣有所建白,及事有不可便宜自決須奏聞者,以拜降善敷奏,數令馳驛往咨於朝。及引見,世祖遙識之,喜曰:「黑髯使臣復來耶!」其見器使如此。



 二十七年,遷江西行尚書省都鎮撫。適徭、獠擾邊,拜降從丞相忙兀臺討定之。二十九年,遷慶元路治中。歲大饑,狀累上行省,不報。拜降曰:「民饑如是而不賑之,豈為民父母意耶!」即躬詣行省力請,得發粟四萬石,民賴全活。



 元貞間,兩浙鹽運司同知範某陰賊為奸,州縣吏以賂,咸聽驅役,由是數侵暴細民。民有珍貨腴田,必奪為己有,不與,則朋結無賴,妄訟以羅織之,無不蕩破家業者。兇焰鑠人,人咸側目。里人欲殺之,不果,顧被誣訴逮系者亡慮數十人,俱死獄中。蘭溪州民葉一、王十四有美田宅,範欲奪之,不可,因誣以事,系獄十年不決。事聞於省,省下理問所推鞫之,適拜降至官,冤遂得直。置範於刑,而七人者先瘐死矣,惟葉一、王十四得釋,時論多焉。大德元年,遷浙東廉訪副使,令行禁止,豪強懾伏。同寅有貪穢者,拜降抗章核之於臺,遂免其官。後轉工部侍郎,賜侍燕服一襲,升工部尚書,有能聲。



 至大二年,仁宗奉皇太后避暑五臺,拜降供給道路,無有闕遺,恩賚尤渥。比至都,改資國院使。母徐氏卒,遂奔喪於杭。時酒禁方嚴,帝特命以酒十MG,官給傳致墓所,以備奠禮。初,徐氏盛年守節,教子甚嚴,比拜降貴,事上於朝,特旌其門。及老,見拜降歷官有聲譽,喜曰:「有子如是,吾死可瞑目矣。」拜降居喪盡禮,未及起復,延祐二年,卒於家。贈資政大夫、江浙左丞,謚貞惠。



\end{pinyinscope}