\article{志第四十五下 食貨五}

\begin{pinyinscope}

 食貨前志,據《經世大典》為之目,凡十有九,自天歷以前,載之詳矣。若夫元統以後,海運之多寡,鈔法之更變,鹽茶之利害,其見於《六條政類》之中,及有司採訪事跡,凡有足徵者,具錄於篇,以備參考;而喪亂之際,其亡逸不存者,則闕之。



 ○海運



 元自世祖用伯顏之言,歲漕東南粟,由海道以給京師,始自至元二十年,至於天歷、至順,由四萬石以上增而為三百萬以上,其所以為國計者大矣。歷歲既久,弊日以生,水旱相仍,公私俱困,疲三省之民力,以充歲運之恆數,而押運監臨之官,與夫司出納之吏,恣為貪黷,腳價不以時給,收支不得其平,船戶貧乏,耗損益甚。兼以風濤不測,盜賊出沒,剽劫覆亡之患,自仍改至元之後,有不可勝言者矣。由是歲運之數,漸不如舊。至正元年,益以河南之粟,通計江南三省所運,止得二百八十萬石。二年,又令江浙行省及中政院財賦總管府,撥賜諸人寺觀之糧,盡數起運,僅得二百六十萬石而已。及汝、潁倡亂,湖廣、江右相繼陷沒,而方國珍、張士誠竊據浙東、西之地,雖縻以好爵,資為籓屏,而貢賦不供,剝民以自奉,於是海運之舟不至京師者積年矣。



 至十九年,朝廷遣兵部尚書伯顏帖木兒、戶部尚書齊履亨征海運於江浙,由海道至慶元,抵杭州。時達識帖睦邇為江浙行中書省丞相,張士誠為太尉,方國珍為平章政事,詔命士誠輸粟,國珍具舟,達識帖睦邇總督之。既達朝廷之命,而方、張互相猜疑,士誠慮方氏載其粟而不以輸於京也,國珍恐張氏掣其舟而因乘虛以襲己也。伯顏帖木兒白於丞相,正辭以責之,巽言以諭之,乃釋二家之疑,克濟其事。先率海舟俟於嘉興之澉浦,而平江之粟展轉以達杭之石墩,又一舍而後抵澉浦,乃載於舟。海灘淺澀,躬履艱苦,粟之載於舟者,為石十有一萬。二十年五月赴京。是年秋,又遣戶部尚書王宗禮等至江浙。二十一年五月,運糧赴京,如上年之數。九月,又遣兵部尚書徹徹不花、侍郎韓祺往征海運一百萬石。二十二年五月,運糧赴京,視上年之數,僅加二萬而已。九月,遣戶部尚書脫脫歡察爾、兵部尚書帖木至江浙。二十三年五月,仍運糧十有三萬石赴京。九月,又遣戶部侍郎博羅帖木兒、監丞賽因不花往征海運。士誠托辭以拒命,由是東南之粟給京師者,遂止於是歲雲。



 鈔法



 至正十年,右丞相脫脫欲更鈔法,乃會中書省、樞密院、御史臺及集賢、翰林兩院官共議之。先是,左司都事武祺嘗建言云:「鈔法自世祖時已行之後,除撥支料本、倒易昏鈔以布天下外,有合支名目,於寶鈔總庫料鈔轉撥,所以鈔法疏通,民受其利。比年以來,失祖宗元行鈔法本意。不與轉撥,故民間流轉者少,致偽鈔滋多。」遂準其所言,凡合支名目,已於總庫轉支。至是,吏部尚書偰哲篤及武祺,俱欲迎合丞相之意。偰哲篤言更鈔法,以楮幣一貫文省權銅錢一千文為母,而錢為子。眾人皆唯唯,不敢出一語,惟集賢大學士兼國子祭酒呂思誠獨奮然曰:「中統、至元自有母子,上料為母,下料為子。比之達達人乞養漢人為子,是終為漢人之子而已,豈有故紙為父,而以銅為過房兒子者乎!」一坐皆笑。思誠又曰:「錢鈔用法,以虛換實,其致一也。今歷代錢及至正錢,中統鈔及至元鈔、交鈔,分為五項,若下民知之,藏其實而棄其虛,恐非國之利也。」偰哲篤、武祺又曰:「至元鈔多偽,故更之爾。」思誠曰:「至元鈔非偽,人為偽爾,交鈔若出,亦有偽者矣。且至元鈔猶故戚也,家之童稚皆識之矣。交鈔猶新戚也,雖不敢不親,人未識也,其偽反滋多爾。況祖宗成憲,豈可輕改。」偰哲篤曰:「祖宗法弊,亦可改矣。」思誠曰:「汝輩更法,又欲上誣世皇,是汝又欲與世皇爭高下也。且自世皇以來,諸帝皆謚曰孝,改其成憲,可謂孝乎?」武祺又欲錢鈔兼行,思誠曰:「錢鈔兼行,輕重不倫,何者為母,何者為子?汝不通古今,道聽途說,何足以行,徒以口舌取媚大臣,可乎?」偰哲篤曰:「我等策既不可行,公有何策?」思誠曰:「我有三字策,曰行不得,行不得。」又曰:「丞相勿聽此言。如向日開金口河,成則歸功汝等,不成則歸罪丞相矣。」脫脫見其言直,猶豫未決。御史大夫也先帖木兒言曰:「呂祭酒言有是者,有非者,但不當坐廟堂高聲厲色。若從其言,此事終不行耶!」明日,諷御史劾之,思誠歸臥不出,遂定更鈔之議而奏之。下詔云:「朕聞帝王之治,因時制宜,損益之方,在乎通變。惟我世祖皇帝,建元之初,頒行中統交鈔,以錢為文,雖鼓鑄之規未遑,而錢幣兼行之意已具。厥後印造至元寶鈔,以一當五,名曰子母相權,而錢實未用。歷歲滋久,鈔法偏虛,物價騰踴,奸偽日萌,民用匱乏。爰詢廷臣,博採輿論,僉謂拯弊必合更張。其以中統交鈔壹貫文省權銅錢一千文,準至元寶鈔二貫,仍鑄至正通寶錢與歷代銅錢並用,以實鈔法。至元寶鈔,通行如故。子母相權,新舊相濟,上副世祖立法之初意。」



 十一年,置寶泉提舉司,掌鼓鑄至正通寶錢、印造交鈔,令民間通用。行之未久,物價騰踴,價逾十倍。又值海內大亂,軍儲供給,賞賜犒勞,每日印造,不可數計。舟車裝運,軸轤相接,交料之散滿人間者,無處無之。昏軟者不復行用。京師料鈔十錠,易鬥粟不可得。既而所在郡縣,皆以物貨相貿易,公私所積之鈔,遂俱不行,人視之若弊楮,而國用由是遂乏矣。



 鹽法



 大都之鹽:元統二年四月,御史臺備監察御史言:「竊睹京畿居民繁盛,日用之中,鹽不可闕。大德中,因商販把握行市,民食貴鹽,乃置局設官賣之。中統鈔一貫,買鹽四斤八兩。後雖倍其價,猶敷民用。及泰定間,因所任局官不得其人,在上者失於鈐束,致有短少之弊。於是巨商趨利者營屬當道,以局官侵盜為由,輒奏罷之,復從民販賣。自是鈔一貫,僅買鹽一斤。無籍之徒,私相犯界,煎賣獨受其利,官課為所侵礙。而民食貴鹽益甚,貧者多不得食,甚不副朝廷恤小民之意。如朝廷仍舊設局,官為發賣,庶課不虧,而民受賜矣。」



 既而大都路備三巡院及大興、宛平縣所申,又戶部尚書建言,皆如御史所陳。戶部乃言,以謂「榷鹽之法,本以裕國而便民。始自大德七年罷大都運司,令河間運司兼辦。每歲存留鹽數,散之米鋪,從其發賣。後因富商專利,遂於南北二城設局,凡十有五處,官為賣之。當時立法嚴明,民甚便益。泰定二年,因局官綱船人等多有侵盜之弊,復從民販賣,而罷所置之局。未及數載,有司屢言富商高抬價直之害。運司所言綱船作弊,蓋因立法不嚴,失於關防所致。且各處俱有官設鹽鋪,與商賈販賣並無窒礙,豈有京城之內,乃革罷官賣之局。宜準本部尚書所言,及大都路所申,依舊制於南北二城置局十有五處。每局日賣十引,設賣鹽官二員,以歲一周為滿,責其奉公發賣。每中統鈔一貫,買鹽二斤四兩,毋令雜灰土其中,及權衡不得其平。凡買鹽過十貫者禁之,不及貫者從所買與之。如滿歲無短少失陷及元定分數者,減一界升用之;若有侵盜者,依例追斷其合賣鹽數。令河間運司分為四季,起赴京廒,用官定法物,兩平稱收,分給各局。其所賣價鈔,逐旬起解,委本部官輪次提調之。仍委官巡視,如有豪強兼利之徒,頻買局鹽而增價轉賣於外者,從提調巡督官痛治之。仍令運司嚴督押運之人,設法防禁,毋致縱令綱船人等作弊。其客商鹽貨,從便相參發賣。」四月二十六日,中書省上奏,如戶部所擬行之。



 至元三年三月,大都京廒申戶部云:「近奉文帖,起運至元二年京廒發賣食鹽一萬五千引,令兩平稱收,如數具實申部。除各綱淹沒短少鹽計八百四十八引,本廒實收一萬四千一百五十有二引,已支一萬一百引付各局發賣,見存鹽四千五十有二引,支撥欲盡。所據至元三年食鹽,宜依例於河間運司起運一萬五千引赴都,庶民間食用不闕。」戶部準其所言,乃議:「京廒食鹽,今歲宜從河間運一萬五千引,其腳價席索等費,令運司於鹽課錢內通算支用。仍召募有產業船戶,互相保識,每一千引為一綱,就差各該場官一員,並本司奏差或監運巡鹽官,每名管押一綱,於大都興國等場見收鹽內驗數,分派分司官監視,如數兩平支收,限三月內赴京廒交卸,取文憑赴部銷照。但有雜和沙土,濕潤短少數,並令本綱船戶、押運場官、奏差監運諸人,如數均賠,依例坐罪。」中書如戶部所議行之。



 至正三年,監察御史王思誠、侯思禮等建言:「京師自大德七年罷大都鹽運司,設官賣鹽,置局十有五處,泰定二年以其不便罷之,元統二年又復之,迨今十年,法久弊生。在船則有侵盜滲溺之患,入局則有和雜灰土之奸。名曰一貫二斤四兩,實不得一斤之上。其潔凈不雜,而斤兩足者,唯上司提調數處耳。又常白鹽一千五百引,用船五十艘,每歲以四月起運,官鹽二萬引,用船五十艘,每歲以七月起運,而運司所遣之人,擅作威福,南抵臨清,北自通州,所至以索截河道,舟楫往來,無不被擾。名為和顧,實乃強奪。一歲之中,千里之內,凡富商巨賈之載米粟者,達官貴人之載家室者,一概遮截,得重賄而放行,所拘留者,皆貧弱無力之人耳。其舟小而不固,滲溺侵盜,弊病多端。既達京廒,又不得依時交收,淹延歲月,困守無聊,鬻妻子、質舟楫者,往往有之。此客船所以狼顧不前,使京師百物湧貴者,實由於此。竊計官鹽二萬引,每引腳價中統鈔七貫,總為鈔三千錠,而十五局官典俸給,以一歲計之又五百七十六錠,其就支賃房之資,短腳之價,席草諸物,又在外焉。當時置局設官,但為民食貴鹽,殊不料官賣之弊,反不如商販之賤,豈忍徒費國家,而使百物貴也。宜從憲臺具呈中書省,議罷其鹽局,及來歲起運之時,出榜文播告鹽商,從便入京興販。若常白鹽所用船五十艘,亦宜於江南造小料船處如數造之。既成之後,付運司顧人運載,庶舟楫通而商賈集,則京師百物賤,而鹽亦不貴矣。」御史臺以其言具呈中書,而河間運司所申,亦如前議。



 戶部言:「運司及大都路講究,即同監察御史所言,元設鹽局,合準革罷,聽從客旅興販。其常白鹽系內府必用之物,起運如故,宜從都省聞奏。」二月初五日,中書省上奏,如戶部所擬行之。



 河間之鹽:至正二年,河間運司申戶部云:「本司歲辦額餘鹽共三十八萬引,計課鈔一百一十四萬錠,以供國用,不為不重。近年以來,各處私鹽及犯界鹽販賣者眾,蓋因軍民官失於禁治,以致侵礙官課,鹽法澀滯,實由於此。乞轉呈都省,頒降詔旨,宣諭所司,欽依規辦。」本部具呈中書省,遂於四月十七日上奏,降旨戒飭之。



 七月,又據河間運司申:「本司辦課,全藉郡縣行鹽地方買食官鹽。去歲河間等路旱蝗闕食,累蒙賑恤,民力未蘇,食鹽者少。又因古北口等處,把隘官及軍人不為用心詰捕,大都路所屬有司,亦不奉公巡禁,致令諸人裝載疙疸鹽於街市賣之,或量以斗,或盛以盤,明相饋送。今紫荊關捕獲犯人張狡群等所載疙疸鹽,計一千六百餘斤。自至元六年三月迄今犯者,將及百起。若不申聞,恐年終課不如數,虛負其咎。」本部具呈中書省,照會樞密院給降榜文禁治之。



 三年,又據河間運司申:「生財節用,固治國之常經;薄賦輕徭,實理民之大本。本司歲額鹽三十五萬引,近年又添餘鹽三萬引,元簽灶戶五千七百七十四戶,除逃亡外,止存四千三百有一戶。每年額鹽,勒令見在疲乏之戶勉強包煎。今歲若依舊煎辦,人力不足。又兼行鹽地方旱蝗相仍,百姓焉有買鹽之資。如蒙矜閔,自至正二年為始,權免餘鹽三萬引,俟豐稔之歲,煎辦如舊。」本部以錢糧支用不敷,權擬住煎一萬引,具呈中書省。正月二十八日上奏,如戶部所擬行之。



 既而運司又言:「至元三十一年,本司辦鹽額二十五萬引,自後累增至三十有五萬。元統元年,又增餘鹽三萬引,已經具呈。蒙都省奏準,住煎一萬引。外有二萬引,若依前勒令見戶包煎,實為難堪。如並將餘鹽二萬引住煎,誠為便益。」戶部又以所言具呈中書省,權擬餘鹽二萬引住煎一年,至正四年煎辦如故。四月十二日上奏,如戶部所擬行之。



 山東之鹽:元統二年,戶部呈:「據山東運司準濟南路牒,依副達魯花赤完者、同知闍裏帖木兒所言,比大都、河間運司,改設巡鹽官一十二員,專一巡禁本部。詳山東運司,歲辦鈔七十五萬餘錠,行鹽之地,周圍三萬餘里,止是運判一員,豈能遍歷,恐私鹽來往,侵礙國課。本司既與濟南路講究便益,宜準所言。」中書省令戶部復議之,本部言:「河間運司定設奏差一十二名,巡鹽官一十六名,山東運司設奏差二十四名,今既比例添設巡鹽官外,據元設奏差內減去一十二名。」具呈中書省,如所擬行之。



 三年二月,又據山東運司備臨朐、沂水等縣申:「本縣十山九水,居民稀少,元系食鹽地方,後因改為行鹽,民間遂食貴鹽,公私不便。如蒙仍舊改為食鹽,令居民驗戶口多寡,以輸納課鈔,則官民俱便,抑且可革私鹽之弊。」運司移文分司,並益都路及下滕、嶧等州,從長講究,互言食鹽為便。及準本司運使辛朝列牒云:「所據零鹽,擬依登、萊等處,銓注局官,給印置局,散賣於民,非惟大課無虧,官釋私鹽之憂,民免刑配之罪。」戶部議:「山東運司所言,於滕、嶧等處增置十有一局,如登、萊三十五局之例,於錢穀官內通行銓注局官,散賣食鹽,官民俱便。既經有司講究,宜從所議。」具呈中書省,如所擬行之。



 至元二年,御史臺據山東肅政廉訪司申:「準濟南路備章丘縣申『見奉山東運司為本司額辦鹽課二十八萬引,除客商承辦之外,見存十三萬引,絕無買者,將及年終,歲課不能如數。所據新城、章丘、長山、鄒平、濟南俱近鹽場,與大、小清河相接,客旅興販,宜依商河、滕、嶧等處,改為食鹽,權派八千引,責付本處有司自備席索腳力,赴已擬固堤等場,於元統三年依例支出,均散於民』等事,竊照山東運司,初無上司明文,輒擅散民食鹽,追納課鈔,使民不得安業。今於至元元年正月、二月,兩次奉到中書戶部符文,行鹽食鹽地分已有定例,毋得樁配於民。本司不遵省部所行,寢匿符文,依前差人馳驛,督責州縣,臨逼百姓,追徵食鹽課鈔,不無擾害。據本司恣意行事,玩法擾民,理應取問,緣系辦課之時,宜從憲臺區處。又據監察御史所呈,亦為茲事。若便行取問,即系辦課時月,具呈中書省區處。」戶部議呈:「行鹽食鹽已有定所,宜從改正。若準御史臺所呈,取問運司,卻緣鹽法例應從長規畫,似難別議。」中書省如所擬行之。



 陜西之鹽:至元二年九月,御史臺準陜西行臺咨備監察御史帖木兒不花建言:「近蒙委巡歷奉元東道,至元元年各州縣戶口額辦鹽課,其陜西運司官不思轉運之方,每年豫期差人,分道齎引,遍散州縣,甫及旬月,杖限追鈔,不問民之無有。竊照諸處運司之例,皆運官召商發賣,惟陜西等處鹽司,近年散於民戶。且如陜西行省食鹽之戶,該辦課二十萬三千一百六十四錠有餘。於內鞏昌、延安等處認定課鈔一萬六千二百七十一錠,慶陽、環州、鳳翔、興元等處歲辦課一萬七千九百八十五錠,其餘課鈔,先因關陜旱饑,民多流亡,準中書省咨,至順三年鹽課,十分為率,減免四分,於今三載,尚有虧負。蓋因戶口凋殘,十亡八九,縱或有復業者,家產已空,爾來歲頗豐收,而物價甚賤,得鈔為艱。本司官皆勒有司徵辦,無分高下,一概給散,少者不下二三引,每一引收價三錠,富家無以應辦,貧下安能措畫?糶終歲之糧,不酬一引之價,緩則輸息而借貸,急則典鬻妻子。縱引目到手,力窘不能裝運,止從各處鹽商,勒價收買。舊債未償,新引又至,民力有限,官賦無窮。又寧夏所產韋紅鹽池,不辦課程,除鞏昌等處循例認納乾課,從便食用外,其池鄰接陜西環州百餘里,紅鹽味甘而價賤,解鹽味苦而價貴,百姓私相販易,不可禁約。以此參詳,河東鹽池,除撈鹽戶口食鹽外,辦課引數,今後宜從運官設法,募商興販。但遇行鹽之處,諸人毋得侵擾韋紅鹽法。運司每歲分輪官吏監視,聽民採取,立法抽分,依例發賣,每引收價鈔三錠。自黃河以西,從民食用,通辦運司元額課鈔。因時夾帶至黃河東南者,同私鹽法罪之,陜西興販解鹽者不禁。如此庶望官民兩便,而課亦無虧矣。」



 又據陜西漢中道肅政廉訪使胡通奉所陳云:「陜西百姓,許食解鹽,近脫荒儉,流移漸復,正宜安輯,而鹽吏不察民瘼,止以恢辦為名,不論貧富,散引收課,或納錢入官,動經歲月,猶未得鹽。蓋因地遠,腳力艱澀。今後若令大河以東之民,分定課程,買食解鹽,其以西之民,計口攤課,任食韋紅之鹽,則官不被擾,民無蕩產之禍矣。且解鹽結之於風,韋紅之鹽產之於地,東鹽味苦,西鹽味甘,又豈肯舍其美而就其惡乎?使陜西百姓,一概均攤解鹽之課,令食韋紅之鹽,則鹽吏免巡禁之勞,而民亦受惠矣。」本臺詳所言鹽法,宜從省部定擬,具呈中書省,送戶部議之。本部議云:「陜西行臺所言鹽事,宜從都省選官,前赴陜西,與行省、行臺及河東運司官一同講究,是否便益,明白咨呈。」



 三年,都省移咨陜西行省,仍摘委河東運司正官一員赴省,一同再行講究。三月初二日,陜西行省官及李御史、運司同知郝中順會鞏昌、延安、興元、奉元、鳳翔、邠州等官,與總帥汪通議等,俱稱當從御史帖木兒不花及廉使胡通奉所言,限以黃河為界,令陜西之民從便食用韋紅二鹽,解鹽依舊西行,紅鹽不許東渡。其咸寧、長安錄事司三處未散者,依已散州縣,一體斟酌,認納乾課,與運司已散食鹽引價同。見納乾課,辦鈔七萬錠,通行按季輸納,運司不須散引。如此則民不受害,而課以無虧矣。郝同知獨言:「運司每歲辦課四十五萬錠,陜西該辦二十萬錠,今止認七萬錠,餘十三萬錠,從何處恢辦?」議不合而散。本省檢照運司逐年申報文冊,陜西止辦七萬二千六十餘錠,郝遂稱疾不出,其後訖無定論。



 戶部參照至順二年中書省嘗遣兵部郎中井朝散,與陜西行省官一同講究,以涇州白家河永為定界,聽民食用。仍督所在軍民官嚴行禁約,毋致韋紅二鹽犯境侵課。中書如所擬行之。



 兩淮之鹽:至元六年八月,兩淮運司準行戶部尚書運使王正奉牒:「本司自至元十四年創立,當時鹽課未有定額,但從實恢辦,自後累增至六十五萬七十五引。客人買引,自行赴場支鹽,場官逼勒灶戶,加其斛面,以通鹽商,壞亂鹽法。大德四年,中書省奏準,改法立倉,設綱攢運,撥袋支發,以革前弊。本司行鹽之地,江浙、江西、河南、湖廣所轄路分,上江下流,鹽法通行。至大間,煎添正額餘鹽三十萬引,通九十五萬七十五引。客商運至揚州東關,俱於城河內停泊,聽候通放,不下三四十萬餘引,積疊數多,不能以時發放。至順四年,前運使韓大中等又言:『歲賣額鹽九十五萬七十五引。客商買引,關給勘合,赴倉支鹽,雇船腳力,每引遠倉該鈔十二三貫,近倉不下七八貫,運至揚州東關,俟候以次通放。其船梢人等,恃以鹽主不能照管,視同己物,恣為侵盜,弊病多端。及事敗到官,非不嚴加懲治,莫能禁止。其所盜鹽,以鈔計之,不過折其舊船以償而已,安能如數征之?是以里河客商,虧陷資本,外江興販,多被欺侮,而百姓高價以買不潔之鹽,公私俱受其害。』竊照揚州東關城外,沿河兩岸,多有官民空閑之地。如蒙聽從鹽商自行賃買基地,起造倉房,支運鹽袋到場,籍定資次,貯置倉內,以俟通放。臨期用船,載往真州發賣,既防侵盜之患,可為悠久之利,其於鹽法非小補也。」



 既申中書戶部及河南行省,照勘議擬,文移往復,紛紜不決。久之,戶部乃定議,令運司於已收在官客商帶納挑河錢內,撥鈔一萬錠,起蓋倉房,仍從都省移咨河南行省,委官與運司偕往,相視空地,果無違礙,而後行之。



 兩浙之鹽:至元五年,兩浙運司申中書省云:



 本司自至元十三年創立,當時未有定額。至十五年始立額,辦鹽十五萬九千引。自後累增至四十五萬引,元統元年又增餘鹽三萬引,每歲總計四十有八萬。每引初定官價中統鈔五貫,自後增為九貫、十貫,以至三十、五十、六十、一百,今則為三錠矣。每年辦正課中統鈔一百四十四萬錠,較之初年,引增十倍,價增三十倍。課額愈重,煎辦愈難,兼以行鹽地界所拘戶口有限。前時聽從客商就場支給,設立檢校所,稱檢出場鹽袋。又因支查停積,延祐七年,比兩淮之例,改法立倉,綱官押船到場,運鹽赴倉收貯,客旅就倉支鹽。始則為便,經今二十餘年,綱場倉官任非其人,惟務掊克。況淮、浙風土不同,兩淮跨涉四省,課額雖大,地廣民多,食之者眾,可以辦集。本司地界,居江枕海,煎鹽亭灶,散漫海隅,行鹽之地,里河則與兩淮鄰接,海洋則與遼東相通,番舶往來,私鹽出沒,侵礙官課,雖有刑禁,難盡防禦。鹽法隳壞,亭民消廢,其弊有五:



 本司所轄場司三十四處,各設令、丞、管勾、典史,管領灶戶火丁。用工之時,正當炎暑之月,晝夜不休。才值陰雨,束手徬徨。貧窮小戶,餘無生理,衣食所資,全籍工本,稍存抵業之家,十無一二。有司不體其勞,又復差充他役。各場元簽灶戶一萬七千有餘,後因水旱疫癘,流移死亡,止存七千有餘。即今未蒙簽補,所據拋下額鹽,唯勒見戶包煎而已。若不早為簽補,優加存恤,將來必致損見戶而虧大課。此弊之一也。



 又如所設三十五綱監運綱司,專掌召募船戶,照依隨場日煎月辦課額,官給水腳錢,就場支裝所煎鹽袋,每引元額四百斤,又加折耗等鹽十斤,裝為二袋,綱官押運前赴所撥之倉而交納焉。客人到倉支鹽,如自二月至於十月河凍之時,以運足為度,其立法非不周密也。今各綱運鹽船戶,經行歲久,奸弊日滋。凡遇到場裝鹽之時,私屬鹽場官吏司秤人等,重其斤兩,裝為硬袋,出場之後,沿途盜賣,雜以灰土,補其所虧。及到所赴之倉,而倉官司秤人又各受賄,既不加辨,秤盤又不如法。在倉日久,又復消折。袋法不均,誠非細故。不若仍舊令客商就場支給,既免綱運俸給水腳之費,又鹽法一新。此弊之二也。



 本司歲辦額鹽四十八萬引,行鹽之地,兩浙、江東凡一千九百六萬餘口。每日食鹽四錢一分八厘,總而計之,為四十四萬九千餘引。雖賣盡其數,猶剩鹽三萬一千餘引。每年督勒有司,驗戶口請買。又值荒歉連年,流亡者眾,兼以瀕江並海,私鹽公行,軍民官失於防禦,所以各倉停積累歲未賣之鹽,凡九十餘萬引,無從支散。如蒙早降定制,以憑遵守,賞罰既明,私鹽減少,戶口食鹽,不致廢弛。此弊之三也。



 又每季拘收退引,凡遇客人運鹽到所賣之地,先須住報水程及所止店肆,繳納退引。豈期各處提調之官,不能用心檢舉,縱令吏胥坊里正等,需求分例錢,不滿所欲,則多端留難。客人或因發賣遲滯,轉往他所,水程雖住,引不拘納,遂有埋沒,致容奸民藏匿在家,影射私鹽,所司亦不檢勘拘收。其懦善者,賣過官鹽之後,即將引目投之鄉胥。又有狡猾之徒,不行納官,通同鹽徒,執以為憑,興販私鹽。如蒙將有司官吏,明定黜降罪名,使退引盡實還官,不致影射私鹽。此弊之四也。



 本司自延祐七年改立杭州等七倉,設置部轄,掌收各綱船戶,運到鹽袋,貯頓在倉,聽候客人,依次支鹽,俱有定制。比年以來,各倉官攢,肆其貪欲,出納之間,兩收其利。凡遇綱船到倉,必受船戶之賄,縱其雜和灰土,收納入倉。或船戶運至好鹽,無錢致賄,則故生事留難,以致停泊河岸,侵欺盜賣。其倉官與鹽運人等為弊多端,是以各倉積鹽九十餘萬引,新舊相並,充溢廊屋,不能支發,走鹵消折,利害非輕。雖系客人買過之物,課鈔入官,實恐年復一年,為患益甚。若仍舊令客商自備腳力,就場支裝,庶免停積。此弊之五也。



 五者之中,各倉停積,最為急務。驗一歲合賣之數,止該四十四萬餘引,盡賣二年,尚不能盡,又復煎運到倉,積累轉多。如蒙特賜奏聞,選委德望重臣,與拘該官府,從長講究,參酌時宜,更張法制,定為良規,惠濟黎元,庶望大課無虧。見為住煎餘鹽三萬引,差人齎江浙行省咨文赴中書省,請照詳焉。



 戶部詳運司所言,除餘鹽三萬引別議外,其餘事理,未經行省明白定擬,呈省移咨,從長講究。六年五月,中書省奏,選官整治江浙鹽法,命江浙行省右丞納麟及首領官趙郎中等提調,既而納麟又以他故辭。



 至正元年,運使霍亞中又言:「兩淮、福建運司,俱有餘鹽,已行住免。本司系同一體,如蒙依例住煎三萬引,庶大課易為辦集。」中書省上奏,得旨權將餘鹽三萬引倚閣,俟鹽法通行而後辦之。



 二年十月,中書右丞相脫脫、平章鐵木兒塔識等奏:「兩浙食鹽,害民為甚,江浙行省官、運司官屢以為言。擬合欽依世祖皇帝舊制,除近鹽地十里之內,令民認買,革罷見設鹽倉綱運,聽從客商赴運司買引,就場支鹽,許於行鹽地方發賣,革去派散之弊。及設檢校批驗所四處,選任廉幹之人,直隸運司,如遇客商載鹽經過,依例秤盤,均平袋法,批驗引目,運司官常行體究。又自至元十三年歲辦鹽課,額少價輕,今增至四十五萬,額多價重,轉運不行。今戶部定擬,自至正三年為始,將兩浙額鹽量減一十萬引,俟鹽法流通,復還元額,散派食鹽,擬合住罷。」有旨從之。



 福建之鹽:至元六年正月,江浙行省據福建運司申:「本司歲辦額課鹽,十有三萬九引一百八十餘斤,今查勘得海口等七場,至元四年閏八月終,積下附餘增辦等鹽十萬一千九百六十二引二百六十二斤。看詳,既有積攢附餘鹽數,據至元五年額鹽,擬合照依天歷元年住煎正額五萬引,不給工本,將上項餘鹽五萬,準作正額,省官本鈔二萬錠,免致亭民重困。本年止辦額鹽八萬九引一百八十餘斤,計鹽十有三萬九引有奇,通行發賣,辦納正課。除留餘鹽五萬餘引,預支下年軍民食鹽,實為官民便益。」本省如所擬,咨呈中書省。送戶部參詳,亦如所擬。其下餘鹽五萬一千九百六十二引,發賣為鈔,通行起解。回咨本省,從所擬行之。



 至正元年,詔:「福建、山東表賣食鹽,病民為甚。行省、監察御史、廉訪司拘該有司官,宜公同講究。」二年六月,江浙行省左丞與行臺監察御史、福建廉訪司官及運使常山李鵬舉、漳州等八路正官講究得食鹽不便,其目有三:一曰餘鹽三萬引,難同正額,擬合除免。二曰鹽額太重,比依廣海例,止收價二錠。三曰住罷食鹽,並令客商通行。



 福建鹽課始自至元十三年,見在鹽六千五十五引,每引鈔九貫。二十年,煎賣鹽五萬四千二百引,每引鈔十四貫。二十五年,增為一錠。三十一年,始立鹽運司,增鹽額為七萬引。元貞二年,每引增價十五貫。大德八年,罷運司,並入宣慰使司恢辦。十年,立都提舉司,增鹽額為十萬引。至大元年,各場煎出餘鹽三萬引。四年,復立運司,遂定額為十三萬引,增價鈔為二錠。延祐元年,又增為三錠,運司又從權改法,建、延、汀、邵仍舊客商興販,而福、興、漳、泉四路樁配民食,流害迄今三十餘年。本道山多田少,土瘠民貧,民不加多,鹽額增重。八路秋糧,每歲止二十七萬八千九百餘石,夏稅不過一萬一千五百餘錠,而鹽課十三萬引,該鈔三十九萬錠。民力日弊,每遇催征,貧者質妻鬻子以輸課,至無可規措,往往逃移他方。近年漳寇擾攘,亦由於此。運司官耳聞目見,蓋因職專恢辦,惠無所施。如蒙欽依詔書事意,罷餘鹽三萬引,革去散賣食鹽之弊,聽從客商八路通行發賣,誠為官民兩便。其正額鹽,若依廣海鹽價,每引中統鈔二錠,宜從都省區處。



 江浙行省遂以左丞所講究,咨呈中書省,送戶部定擬,自至正三年為始,將餘鹽三萬引,權令減免,散派食鹽擬合住罷。其減正額鹽價,即與廣海提舉司事例不同,別難更議。十月二十八日,右丞相脫脫、平章帖木兒達失等,以所擬奏而行之。



 廣東之鹽:至元二年,御史臺準江南諸道行御史臺咨備監察御史韓承務建言:「廣東道所管鹽課提舉司,自至元十六年為始,止辦鹽額六百二十一引,自後累增至三萬五千五百引,延祐間又增餘鹽,通正額計五萬五百五十二引。灶戶窘於工程,官民迫於催督,呻吟愁苦,已逾十年。泰定間,蒙憲臺及奉使宣撫,交章敷陳,減免餘鹽一萬五千引。元統元年,都省以支持不敷,權將已減餘鹽,依舊煎辦,今已三載,未蒙住罷。竊意議者,必謂廣東控制海道,連接諸蕃,船商輳集,民物富庶,易以辦納,是蓋未能深知彼中事宜。本道所轄七路八州,平土絕少,加以嵐瘴毒癘,其民刀耕火種,巢顛穴岸,崎嶇辛苦,貧窮之家,經歲淡食,額外辦鹽,賣將誰售。所謂富庶者,不過城郭商賈與舶船交易者數家而已。灶戶鹽丁,十逃三四,官吏畏罪,止將見存人戶,勒令帶煎。又有大可慮者,本道密邇蠻獠,民俗頑惡,誠恐有司責辦太嚴,斂怨生事,所系非輕。如蒙捐此微利,以示大信,疲民幸甚。」具呈中書省,送戶部定擬,自元統三年為始,廣東提舉司所辦餘鹽,量減五千引。十月初九日,中書省以所擬奏聞,得旨從之。



 廣海之鹽:至元五年三月,湖廣行省咨中書省云:「廣海鹽課提舉司額鹽三萬五千一百六十五引,餘鹽一萬五千引。近因黎賊為害,民不聊生,正額積虧四萬餘引,臥收在庫。若復添辦餘鹽,困苦未蘇,恐致不安。事關利害,如蒙憐憫,聞奏除免,庶期元額可辦,不致遺患邊民。」戶部議云:「上項餘鹽,若全恢辦,緣非元額,兼以本司僻在海隅,所轄灶民,累遭劫掠,死亡逃竄,民物凋弊,擬於一萬五千引內,量減五千引,以舒民力。」中書以所擬奏聞,得旨從之。



 四川之鹽:元統三年,四川行省據鹽茶轉運使司申:「至順四年,中書坐到添辦餘鹽一萬引外,又帶辦兩浙運司五千引,與正額鹽通行煎辦,已後支用不闕,再行議擬。卑司為各場別無煎出餘鹽,不免勒令灶戶承認規劃,幸已足備。以後年分,若不申覆,誠恐灶戶逃竄,有妨正課。如蒙憐憫,備咨中書省,於所辦餘鹽一萬引內,量減帶辦兩浙之數。」又準分司運官所言云:「四川鹽井,俱在萬山之間,比之腹裏、兩淮,優苦不同,又行帶辦餘鹽,灶民由此而疲矣。」行省咨呈中書省,上奏得旨,權以帶辦餘鹽五千引倚閣之。



 茶法



 至元二年,江西、湖廣兩行省具以茶運司同知萬家閭所言添印茶由事,咨呈中書省云:「本司歲辦額課二十八萬九千二百餘錠,除門攤批驗鈔外,數內茶引一百萬張,每引十二兩五錢,共為鈔二十五萬錠。末茶自有官印筒袋關防,其零斤草茶由帖,每年印造一千三百八萬五千二百八十九斤,該鈔二萬九千八十餘錠。茶引一張,照茶九十斤,客商興販。其小民買食及江南產茶去處零斤採賣,皆須由帖為照。春首發賣茶由,至於夏秋,茶由盡絕,民間闕用。以此考之,茶由數少課輕,便於民用而不敷,茶引課重數多,止於商旅興販,年終尚有停閑未賣者。每歲合印茶由,以十分為率,量添二分,計二百六十一萬七千五十八斤。算依引目內官茶,每斤收鈔一錢三分八厘八毫八絲,計增鈔七千二百六十九錠七兩,比驗減去引目二萬九千七十六張,庶幾引不停閑,茶無私積。中書戶部定擬,江西茶運司歲辦公據十萬道,引一百萬,計鈔二十八萬九千二百餘錠。茶引便於商販,而山場小民全憑茶由為照,歲辦茶由一千三百八萬五千二百八十九斤,每斤一錢一分一厘一毫二絲,計鈔五千八百一十六錠七兩四錢一分,減引二萬三千二百六十四張。茶引一張,造茶九十斤,納官課十二兩五錢。如於茶由量添二分,計二百六十一萬七千五十八斤,每斤添收鈔一錢三分八厘八毫八絲,計鈔七千二百六十九錠七兩,積出餘零鈔數,官課無虧,而便於民用。」合準本省所擬,具呈中書省,移咨行省,如所擬行之。



 至正二年,李宏陳言內一節,言江州茶司據引不便事云:「榷茶之制,古所未有,自唐以來,其法始備。國朝既於江州設立榷茶都轉運司,仍於各路出茶之地設立提舉司七處,專任散據賣引,規辦國課,莫敢誰何。每至十二月初,差人勾集各處提舉司官吏,關領次年據引。及其到司,旬月之間,司官不能偕聚。吏貼需求,各滿所欲,方能給付據引。此時春月已過。及還本司,方欲點對給散,又有分司官吏,到各處驗戶散據賣引。每引十張,除正納官課一百二十五兩外,又取要中統鈔二十五兩,名為搭頭事例錢,以為分司官吏饋食盡之資。提舉司雖以榷茶為名,其實不能專散據賣引之任,不過為運司官吏營辦資財而已。上行下效,勢所必然。提舉司既見分司官吏所為若是,亦復仿效遷延。及茶戶得據還家,已及五六月矣。中間又存留茶引二三千本,以茶戶消乏為名,轉賣與新興之戶。每據又多取中統鈔二十五兩,上下分派,各為己私。不知此等之錢,自何而出,其為茶戶之苦,有不可言。至如得據在手,碾磨方興,吏卒踵門,催並初限。不知茶未發賣,何從得錢?間有充裕之家,必須別行措辦。其力薄者,例被拘監,無非典鬻家私,以應官限。及終限不能足備,上司緊並,重復勾追,非法苦楚。此皆由運司給引之遲,分司苛取之過。茶戶本圖求利,反受其害,日見消乏逃亡,情實堪憫。今若申明舊制,每歲正月,須要運司盡將據引給付提舉司,隨時派散,無得停留在庫,多收分例,妨誤造茶時月;如有過期,別行定罪。仍不許運司似前分司自行散賣據引,違者從肅政廉訪司依例糾治。如此,庶茶司少革貪黷之風,茶戶免損乏之害。」中書省以其言送戶部定擬,復移咨江西行省,委官與茶運司講究,如果便益,如所言行之。



\end{pinyinscope}