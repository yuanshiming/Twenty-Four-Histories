\article{志第一 天文一}

\begin{pinyinscope}

 司天之說尚矣,《易》曰:「天垂象,見吉兇,聖人象之。」又曰:「觀乎天文,以察時變。」自古有國家者斷超越自身而又否定自身,並給世界以意義。聲稱自為的這,未有不致謹於斯者也。是故堯命羲、和,歷象日月星辰,舜在璇璣、王衡,以齊七政,天文於是有測驗之器焉。然古之為其法者三家:曰周髀,曰宣夜,曰渾天。周髀、宣夜先絕,而渾天之學至秦亦無傳,漢洛下閎始得其術,作渾儀以測天。厥後歷世遞相沿襲,其有得有失,則由乎其人智術之淺深,未易遽數也。



 宋自靖康之亂,儀象之器盡歸於金。元興,定鼎於燕,其初襲用金舊,而規環不協,難復施用。於是太史郭守敬者,出其所創簡儀、仰儀及諸儀表,皆臻於精妙,卓見絕識,蓋有古人所未及者。其說以謂:昔人以管窺天,宿度餘分約為太半少,未得其的。乃用二線推測,於餘分纖微皆有可考。而又當時四海測景之所凡二十有七,東極高麗,西至滇池,南逾硃崖,北盡鐵勒,是亦古人之所未及為者也。自是八十年間,司天之官遵而用之,靡有差忒。而凡日月薄食、五緯凌犯、彗孛飛流、暈珥虹霓、精昆雲氣等事,其系於天文占候者,具有簡冊存焉。



 若昔司馬遷作《天官書》,班固、範曄作《天文志》,其於星辰名號、分野次舍、推步候驗之際詳矣。及晉、隋二《志》,實唐李淳風撰,於夫二十八宿之躔度,二曜五緯之次舍,時日災祥之應,分野休咎之別,號極詳備,後有作者,無以尚之矣。是以歐陽修志《唐書·天文》,先述法象之具,次紀日月食、五星凌犯及星變之異;而凡前史所已載者,皆略不復道。而近代史官志宋《天文》者,則首載儀象諸篇;志金《天文》者,則唯錄日月五星之變。誠以璣衡之制載於《書》,日星、風雨、霜雹、雷霆之災異載於《春秋》,慎而書之,非史氏之法當然,固所以求合於聖人之經者也。今故據其事例,作元《天文志》。



 ○簡儀



 簡儀之制,四方為趺,縱一丈八尺,三分去一以為廣。趺面上廣六寸,下廣八寸,厚如上廣。中布橫輄三、縱輄三。南二,北抵南輄;北一,南抵中輄。趺面四周為水渠,深一寸,廣加五分。四隅為礎,出趺面內外各二寸。繞礎為渠,深廣皆一寸,與四周渠相灌通。又為礎於卯酉位,廣加四維,長加廣三之二,水渠亦如之。北極雲架柱二,徑四寸,長一丈二尺八寸。下為鰲云,植於乾艮二隅礎上,左右內向,其勢斜準赤道,合貫上規。規環徑二尺四寸,廣一寸五分,厚倍之。中為距,相交為斜十字,廣厚如規。中心為竅,上廣五分,方一寸有半,下二寸五分,方一寸,以受北極樞軸。自雲架柱斜上,去趺面七尺二寸,為橫輄。自輄心上至竅心六尺八寸。又為龍柱二,植於卯酉礎中分之北,皆飾以龍,下為山形,北向斜植,以柱北架。南極雲架柱二,植於卯酉礎中分之南,廣厚形制,一如北架。斜向坤巽二隅,相交為十字,其上與百刻環邊齊,在辰巳、未申之間,南傾之勢準赤道,各長一丈一尺五寸。自趺面斜上三尺八寸為橫輄,以承百刻環。下邊又為龍柱二,植於坤巽二隅礎上,北向斜柱,其端形制,一如北柱。



 四游變環,徑六尺,廣二寸,厚一寸,中間相離一寸,相連於子午卯酉。當子午為圓竅,以受南北極樞軸。兩面皆列周天度分,起南極,抵北極,餘分附於北極。去南北樞竅兩旁四寸,各為直距,廣厚如環。距中心各為橫關,東西與兩距相連,廣厚亦如之。關中心相連,厚三寸,為竅方八分,以受窺衡樞軸。窺衡長五尺九寸四分,廣厚皆如環,中腰為圓竅,徑五分,以受樞軸。衡兩端為圭首,以取中縮。去圭首五分,各為側立橫耳,高二寸二分,廣如衡面,厚三分,中為圓竅,徑六分。其中心,上下一線界之,以知度分。



 百刻環,徑六尺四寸,面廣二寸,周布十二時、百刻,每刻作三十六分,厚二寸,自半已上廣三寸。又為十字距,皆所以承赤道環也。百刻環內廣面臥施圓軸四,使赤道環旋轉無澀滯之患。其環陷入南極架一寸,仍釘之。赤道環徑廣厚皆如四游,環面細刻列舍、周天度分。中為十字距,廣三寸,中空一寸,厚一寸。當心為竅,竅徑一寸,以受南極樞軸。界衡二,各長五尺九寸四分,廣三寸。衡首斜剡五分,刻度分以對環面。中腰為竅,重置赤道環、南極樞軸。其上衡兩端,自長竅處邊至衡首底,厚倍之,取二衡運轉,皆著環面,而無低昂之失,且易得度分也。二極樞軸皆以鋼鐵為之,長六寸,半為本,半為軸。本之分寸一如上規距心,適取能容軸徑一寸。北極軸中心為孔,孔底橫穿,通兩旁,中出一線,曲其本,出橫孔兩旁結之。孔中線留三分,亦結之。上下各穿一線,貫界衡兩端,中心為孔,下洞衡底,順衡中心為渠以受線,直入內界長竅中。至衡中腰,復為孔,自衡底上出結之。



 定極環,廣半寸,厚倍之,皆勢穹窿,中徑六度,度約一寸許。極星去不動處三度,僅容轉周。中為斜十字距,廣厚如環,連於上規。環距中心為孔,徑五厘。下至北極軸心六寸五分,又置銅板,連於南極雲架之十字,方二寸,厚五分。北面剡其中心,存一厘以為厚,中為圜孔,徑一分,孔心下至南極軸心亦六寸五分。又為環二:其一陰緯環,面刻方位,取趺面縱橫輄北十字為中心,臥置之。其一曰立運環,面刻度分,施於北極雲架柱下,當臥環中心,上屬架之橫輄,下抵趺輄之十字,上下各施樞軸,令可旋轉。中為直距,當心為竅,以施窺衡,令可俯仰,用窺日月星辰出地度分。右四游環,東西運轉,南北低昂,凡七政、列舍、中外官去極度分皆測之。赤道環旋轉,與列舍距星相當,即轉界衡使兩線相對,凡日月五星、中外官入宿度分皆測之。百刻環,轉界衡令兩線與日相對,其下直時刻,則晝刻也,夜則以星定之。比舊儀測日月五星出沒,而無陽經陰緯云柱之映。



 其渾象之制,圜如彈丸,徑六尺,縱橫各畫周天度分。赤道居中,去二極,各周天四之一。黃道出入赤道內外,各二十四度弱。月行白道,出入不常,用竹篾均分天度,考驗黃道所交,隨時遷徙。先用簡儀測到入宿去極度數,按於其上,校驗出入黃赤二道遠近疏密,了然易辨,仍參以算數為準。其象置於方匱之上,南北極出入匱面各四十度太強,半見半隱,機運輪牙隱於匱中。



 ○仰儀



 仰儀之制,以銅為之,形若釜,置於磚臺。內畫周天度,脣列十二辰位,蓋俯視驗天者也。其《銘》辭云:「不可體形,莫天大也。無競維人,仰釜載也。六尺為深,廣自倍也。兼深廣倍,晙釜兌也。環鑿為沼,準以溉也。辨方正位,曰子卦也。衡縮度中,平斜再也。斜起南極,平釜鐓也。小大必周,入地畫也。始周浸斷,浸極外也。極入地深,四十太也。北九十一,赤道齘也。列刻五十,六時配也。衡竿加卦,巽坤內也。以負縮竿,子午對也。首旋璣板,曌納芥也。上下懸直,與鐓會也。視日透光,何度在也。絜穀朝賓,夕餞昧也。寒暑發斂,驗進退也。薄蝕起自,鑒生殺也。以避赫曦,奪目害也。南北之偏,亦可概也。極淺十五,林邑界也。黃道夏高,人所載也。夏永冬短,猶少差也。深五十奇,鐵勒塞也。黃道浸平,冬晝晦也。夏則不沒,永短最也。安渾宣夜,昕穹蓋也。六天之書,言殊話也。一儀一揆,孰善悖也。以指為告,無煩喙也。暗資以明,疑者沛也。智者是之,膠者怪也。古今巧歷,不億輩也。非讓不為,思不逮也。將窺天朕,造化愛也。其有俊明,昭聖代也。泰山礪乎,河如帶也。黃金不磨,悠久賴也。鬼神禁訶,勿銘壞也。」



 大明殿燈漏



 燈漏之制,高丈有七尺,架以金為之。其曲梁之上,中設雲珠,左日右月。雲珠之下,復懸一珠。梁之兩端,飾以龍首,張吻轉目,可以審平水之緩急。中梁之上,有戲珠龍二,隨珠俯仰,又可察準水之均調。凡此皆非徒設也。燈球雜以金寶為之,內分四層,上環布四神,旋當日月參辰之所在,左轉日一周。次為龍虎鳥龜之象,各居其方,依刻跳躍,鐃鳴以應於內。又次周分百刻,上列十二神,各執時牌,至其時,四門通報。又一人當門內,常以手指其刻數。下四隅,鐘鼓鉦鐃各一人,一刻鳴鐘,二刻鼓,三鉦,四鐃,初正皆如是。其機發隱於櫃中,以水激之。



 ○正方案



 正方案,方四尺,厚一寸。四周去邊五分為水渠。先定中心,畫為十字,外抵水渠。去心一寸,畫為圓規,自外寸規之,凡十九規。外規內三分,畫為重規,遍布周天度。中為圓,徑二寸,高亦如之。中心洞底植臬,高一尺五寸,南至則減五寸,北至則倍之。



 凡欲正四方,置案平地,注水於渠,眡平,乃植臬於中。自臬景西入外規,即識以墨影,少移輒識之,每規皆然,至東出外規而止。凡出入一規之交,皆度以線,屈其半以為中,即所識與臬相當,且其景最短,則南北正矣。復遍閱每規之識,以審定南北。南北既正,則東西從而正。然二至前後,日軌東西行,南北差少,即外規出入之景以為東西,允得其正。當二分前後,日軌東西行,南北差多,朝夕有不同者,外規出入之景或未可憑,必取近內規景為定,仍校以累日則愈真。



 又測用之法,先測定所在北極出地度,即自案地平以上度,如其數下對南極入地度,以墨斜經中心界之,又橫截中心斜界為十字,即天腹赤道斜勢也。乃以案側立,懸繩取正。凡置儀象,皆以此為準。



 ○圭表



 圭表以石為之,長一百二十八尺,廣四尺五寸,厚一尺四寸,座高二尺六寸。南北兩端為池,圓徑一尺五寸,深二寸,自表北一尺,與表梁中心上下相直。外一百二十尺,中心廣四寸,兩旁各一寸,畫為尺寸分,以達北端。兩旁相去一寸為水渠,深廣各一寸,與南北兩池相灌通以取平。表長五十尺,廣二尺四寸,厚減廣之半,植於圭之南端圭石座中,入地及座中一丈四尺,上高三十六尺。其端兩旁為二龍,半身附表上擎橫梁,自梁心至表顛四尺,下屬圭面,共為四十尺。梁長六尺,徑三寸,上為水渠以取平。兩端及中腰各為橫竅,徑二分,橫貫以鐵,長五寸,系線合於中,懸錘取正,且防傾墊。



 按表短則分寸短促,尺寸之下所謂分秒太半少之數,未易分別;表長則分寸稍長,所不便者景虛而淡,難得實影。前人欲就虛景之中考求真實,或設望筒,或置小表,或以木為規,皆取端日光,下徹表面。今以銅為表,高三十六尺,端挾以二龍,舉一橫梁,下至圭面共四十尺,是為八尺之表五。圭表刻為尺寸,舊一寸,今申而為五,厘毫差易分別。



 ○景符



 景符之制,以銅葉,博二寸,長加博之二,中穿一竅,若針芥然。以方跂為趺,一端設為機軸,令可開闔,耆其一端,使其勢斜倚,北高南下,往來遷就於虛梁之中。竅達日光,僅如米許,隱然見橫梁於其中。舊法一表端測晷,所得者日體上邊之景。今以橫梁取之,實得中景,不容有毫末之差。至元十六年己卯夏至晷景,四月十九日乙未景一丈二尺三寸六分九厘五毫。至元十六年己卯冬至晷景,十月二十四日戊戌景七丈六尺七寸四分。



 窺幾



 窺幾之制,長六尺,廣二尺,高倍之。下為趺,廣三寸,厚二寸,上跂廣四寸,厚如趺。以板為面,厚及寸,四隅為足,撐以斜木,務取正方。面中開明竅,長四尺,廣二寸。近竅兩旁一寸分畫為尺,內三寸刻為細分,下應圭面。幾面上至梁心二十六尺,取以為準。窺限各各長二尺四寸,廣二寸,脊厚五分,兩刃斜閷,取其於幾面相符,著限兩端,厚廣各存二寸,銜入幾跂。俟星月正中,從幾下仰望,視表梁南北以為識,折取分寸中數,用為直景。又於遠方同日窺測取景數,以推星月高下也。



 ○西域儀象



 世祖至元四年,扎馬魯丁造西域儀象:



 咱禿哈剌吉,漢言混天儀也。其制以銅為之,平設單環,刻周天度,畫十二辰位,以準地面。側立雙環而結於平環之子午,半入地下,以分天度。內第二雙環,亦刻周天度,而參差相交,以結於側雙環,去地平三十六度以為南北極,可以旋轉,以象天運為日行之道。內第三、第四環,皆結於第二環,又去南北極二十四度,亦可以運轉。凡可運三環,各對綴銅方釘,皆有竅以代衡簫之仰窺焉。



 咱禿朔八臺,漢言測驗周天星曜之器也。外周圓墻,而東面啟門,中有小臺,立銅表高七尺五寸,上設機軸,懸銅尺,長五尺五寸,復加窺測之簫二,其長如之,下置橫尺,刻度數其上,以準掛尺。下本開圖之遠近,可以左右轉而周窺,可以高低舉而遍測。



 魯哈麻亦渺凹只,漢言春秋分晷影堂。為屋二間,脊開東西橫罅,以斜通日晷。中有臺,隨晷影南高北下,上仰置銅半環,刻天度一百八十,以準地上之半天,斜倚銳者銅尺,長六尺,闊一寸六分,上結半環之中,下加半環之上,可以往來窺運,側望漏屋晷影,驗度數,以定春秋二分。



 魯哈麻亦木思塔餘,漢言冬夏至晷影堂也。為屋五間,屋下為坎,深二丈二尺,脊開南北一罅,以直通日晷。隨罅立壁,附壁懸銅尺,長一丈六寸。壁仰畫天度半規,其尺亦可往來規運,直望漏屋晷影,以定冬夏二至。



 苦來亦撒麻,漢言渾天圖也。其制以銅為丸,斜刻日道交環度數於其腹,刻二十八宿形於其上。外平置銅單環,刻周天度數,列於十二辰位以準地。而側立單環二,一結於平環之子午,以銅丁象南北極,一結於平環之卯酉,皆刻天度。即渾天儀而不可運轉窺測者也。



 苦來亦阿兒子,漢言地理志也。其制以木為圓球,七分為水,其色綠,三分為土地,其色白。畫江河湖海,脈絡貫串於其中。畫作小方井,以計幅圓之廣袤、道里之遠近。



 兀速都兒剌不,定漢言,晝夜時刻之器。其制以銅如圓鏡而可掛,面刻十二辰位、晝夜時刻,上加銅條綴其中,可以圓轉。銅條兩端,各屈其首為二竅以對望,晝則視日影,夜則窺星辰,以定時刻,以測休咎。背嵌鏡片,三面刻其圖凡七,以辨東西南北日影長短之不同、星辰向背之有異,故各異其圖,以畫天地之變焉。



 ○四海測驗



 南海,北極出地一十五度,夏至景在表南,長一尺一寸六分,晝五十四刻,夜四十六刻。



 衡岳,北極出地二十五度,夏至日在表端,無景,晝五十六刻,夜四十四刻。



 岳臺,北極出地三十五度,夏至晷景長一尺四寸八分,晝六十刻,夜四十刻。



 和林,北極出地四十五度,夏至晷景長三尺二寸四分,晝六十四刻,夜三十六刻。



 鐵勒,北極出地五十五度,夏至晷景長五尺一分,晝七十刻,夜三十刻。



 北海,北極出地六十五度,夏至晷景長六尺七寸八分,晝八十二刻,夜一十八刻。



 大都,北極出地四十度太強,夏至晷景長一丈二尺三寸六分,晝六十二刻,夜三十八刻。



 上都,北極出地四十三度少。



 北京,北極出地四十二度強。



 益都,北極出地三十七度少。



 登州,北極出地三十八度少。



 高麗,北極出地三十八度少。



 西京,北極出地四十度少。



 太原,北極出地三十八度少。



 安西府,北極出地三十四度半強。



 興元,北極出地三十三度半強。



 成都,北極出地三十一度半強。



 西涼州,北極出地四十度強。



 東平,北極出地三十五度太。



 大名,北極出地三十六度。



 南京,北極出地三十四度太強。



 河南府陽城,北極出地三十四度太弱。



 揚州,北極出地三十三度。



 鄂州,北極出地三十一度半。



 吉州,北極出地二十六度半。



 雷州,北極出地二十度太。



 瓊州,北極出地一十九度太。



 ○日薄食暈珥及日變



 世祖中統二年三月壬戌朔,日有食之。三年十一月辛丑,日有背氣,重暈三珥。至元二年正月辛未朔,日有食之。四年五月丁亥朔,日有食之。五年十月戊寅朔,日有食之。七年三月庚子朔,日有食之。八年八月壬辰朔,日有食之。九年八月丙戌朔,日有食之。十二年六月庚子朔,日有食之。十四年十月丙辰朔,日有食之。十九年六月己丑朔,日有食之。七月戊午朔,日有食之。二十四年七月癸丑,日暈連環,白虹貫之。十月戊午朔,日有食之。二十六年三月庚辰朔,日有食之。二十七年八月辛未朔,日有食之。二十九年正月甲午朔,日有食之。有物漸侵入日中,不能既,日體如金環然,左右有珥,上有抱氣。三十一年六月庚辰朔,日食。



 成宗大德三年八月己酉朔,日食。四年二月丁未朔,日食。六年六月癸亥朔,日食。七年閏五月戊午朔,日食。八年五月壬子朔,日食。



 武宗至大三年正月丁亥,白虹貫日。八月甲寅,白虹貫日。四年正月壬辰,日赤如赭。



 仁宗皇慶元年六月乙丑朔,日有食之。延祐元年三月己亥,白暈亙天,連環貫日。二年四月戊寅朔,日有食之。五月甲戌,日赤如赭。乙亥,亦如之。九月甲寅,日赤如赭。戊午,亦如之。三年五月戊申,日赤如赭。五年二月癸巳朔,日有食之。六年二月丁亥朔,日有食之。七年正月辛巳朔,日有食之。三月乙未,日有暈若連環然。



 英宗至治元年三月己丑,交暈如連環貫日。六月癸卯朔,日有食之。二年十一月甲午朔,日有食之。



 泰定帝泰定四年二月辛卯,白虹貫日。九月丙申朔,日食。



 文宗天歷二年七月丙辰朔,日有食之。至順元年九月癸巳,白虹貫日。二年正月己酉,白虹貫日。八月甲辰朔,日有食之。十一月壬申朔,日有食之。三年五月丁酉,白虹並日出,長竟天。



 順帝元統元年三月癸巳,日赤如赭。閏三月丙申、癸丑、甲寅,皆如之。二年四月戊午朔,日有食之。至元元年十二月戊午,日赤如赭。閏十二月丁亥、戊子、己丑,皆如之。二年二月壬辰,日赤如赭。乙未、丙申,亦如之。三月庚申、壬戌、癸亥,四月丁丑朔,皆如之。八月甲戌朔,日有食之。十二月甲戌,日赤如赭。三年正月丁巳,日有交暈,左右珥上有白虹貫之。二月壬申朔,日有食之。八月癸未,日有交暈,左右珥上有白虹貫之。十月癸酉,日赤如赭。四年閏八月戊戌,日赤如赭。己亥、壬寅,亦如之。



 九月庚寅,皆如之。五年正月丙寅,日有交暈,左右珥上有白虹貫之。二月辛亥,日赤如赭。三月庚申、辛酉,四月丁未,皆如之。至正元年三月壬申,日赤如赭。三年四月丙申朔,日有食之。四年九月丁亥朔,日有食之。十年十一月壬子朔,日有食之。十三年九月乙丑朔,日有食之。十四年三月癸亥朔,日有食之。十五年二月丙子,日赤如赭。十七年七月己丑,日有交暈,連環貫之。十八年六月戊辰朔,日有食之。十二月乙丑朔,日有食之。二十一年四月辛巳朔,日有食之。二十五年三月壬戌,日有暈,內赤外青,白虹如連環貫之。二十六年二月丁卯,日有暈,左珥上有背氣一道。七月辛巳朔,日有食之。二十七年十二月癸卯朔,日有食之。



 ○月五星凌犯及星變上



 憲宗六年六月,太白晝見。



 世祖中統元年五月乙未,熒惑入南斗,留五十餘日。



 二年二月丁酉,太陰掩昴。六月戊戌,太陰犯角。八月丙午,太白犯歲星。十一月庚午,太陰犯昴。十二月辛卯,熒惑犯房。壬寅,熒惑犯鉤鈐。



 三年十一月乙酉,太白犯鉤鈐。



 至元元年二月丁卯,太陰犯南斗。四月辛亥,太陰犯軒轅御女星。五月丙戌,太陰犯房。己亥,太陰犯昴。七月甲戌,彗星出輿鬼,昏見西北,貫上臺,掃紫微、文昌及北斗,旦見東北,凡四十餘日。十二月甲子,太陰犯房。



 二年六月丙子,太陰犯心宿大星。



 四年八月庚申,填星犯天罇距星。壬午,太白犯軒轅大星。甲子,歲星犯軒轅大星。



 十一月乙巳,填星犯天罇距星。



 五年正月甲午,太陰犯井。二月戊子,太陰犯天關。己丑,太陰犯井。



 六年十月庚子,太陰犯辰星。



 七年正月己酉,太陰犯畢。九月丁巳,太陰犯井。十月庚午,太白犯右執法。十一月壬寅,熒惑犯太微西垣上將。



 八年正月辛未,太陰犯畢。三月丁亥,熒惑犯太微西垣上將。九月丙子,太陰犯畢。



 九年五月乙酉,太白犯畢距星。九月戊寅,太陰犯御女。十月戊戌,熒惑犯填星。十一月丁卯,太陰犯畢。



 十年三月癸酉,客星青白如粉絮,起畢,度五車北,復自文昌貫斗杓,歷梗河,至左攝提,凡二十一日。



 十一年二月甲寅,太陰犯井宿。十月壬戌,歲星犯壘壁陣。



 十二年七月癸酉,太白犯井。辛卯,太陰犯畢。九月己巳,太白犯少民。己卯,太白犯太微西垣上將。十月癸丑,太陰犯畢。十一月丙戌,太陰犯軒轅大星。十二月戊戌,填星犯亢。戊申,太陰犯畢。



 十三年九月辛亥,太白犯南斗。甲寅,太白入南斗。十一月乙卯,太陰犯填星。十二月辛酉朔,熒惑掩鉤鈐。



 十四年二月癸亥,彗出東北,長四尺餘。



 十五年二月丁丑,熒惑犯天街。三月丁亥,太陰犯太白。戊子,太陰犯熒惑。閏十一月辛亥,太白、熒惑、填星聚於房。



 十六年四月癸卯,填星犯鍵閉。七月丙寅,填星犯鍵閉。八月庚辰,太陰犯房宿距星。庚子,歲星犯軒轅大星。十月丙申,太陰犯太微西垣上將。十一月癸丑,太陰犯熒惑。



 十七年四月庚子,歲星犯軒轅大星。七月戊申,太陰掩房宿距星。己酉,太陰犯南斗。



 八月丙子,太陰犯心宿東星。九月甲子,太陰犯右執法並犯歲星。



 十八年七月癸卯,太陰犯房宿距星。閏八月癸巳朔,熒惑犯司怪南第二星。庚戌,太陰犯昴。九月甲申,太陰犯軒轅大星。十一月甲戌,太陰犯五車次南星。丁丑,太陰犯鬼。丁亥,太陰掩心。十二月丙午,太陰犯軒轅大星。



 二十年正月己巳,太陰犯軒轅御女。庚辰,太陰入南斗,犯距星。二月庚寅,太陰掩昴。庚子,太白犯昴。壬寅,太白犯昴。乙巳,太陰犯心。三月己未,歲星犯鍵閉。庚申,太陰犯井。壬戌,太陰犯鬼。己巳,歲星犯房。癸酉,歲星掩房。四月己亥,太陰犯房。壬寅,太陰犯南斗。五月丙寅,太陰掩心。七月丙辰,太白犯井。癸亥,太陰犯南斗。乙丑,太白犯井。庚午,熒惑犯司怪。八月丙午,太白犯軒轅。丁未,歲星犯鉤鈐。九月壬子,太白犯軒轅少女。戊午,太陰犯鬥。己巳,太白犯右執法。壬申,太陰掩井。癸酉,熒惑犯鬼。甲戌,太陰犯鬼,熒惑犯積尸氣,太白犯左執法。十月丙申,太陰犯昴。十一月戊寅,太白、歲星相犯。十二月甲辰,太陰掩熒惑。



 二十一年閏五月戊寅朔,填星犯鬥。七月甲申,太白犯熒惑。九月癸巳,太白犯南斗第四星。乙未,太陰犯井。十月己酉,太陰犯軫。十一月丙戌,太陰犯昴。己丑,太陰掩輿鬼。庚子,太陰犯心。



 二十二年二月辛亥,太陰犯東井。癸丑,太陰犯鬼。壬戌,太陰犯心。八月癸丑,太陰入東井。十二月己亥,歲星犯填星。



 二十三年正月壬午,太陰犯軒轅太民。乙酉,太陰犯氐。二月丙午,太陰犯井。三月己巳,太陰犯婁。五月己巳,熒惑犯太微西垣上將。庚辰,歲星犯壘壁陣。乙酉,熒惑犯太微右執法。六月丙申朔,太白犯御女。八月乙卯,太白犯軒轅右角星。九月甲申,太陰犯天關。十月甲午朔,太白犯右執法。戊戌,太陰犯建星。辛亥,太陰犯東井。甲寅,太白犯進賢。十一月戊辰,太白犯亢。己卯,太陰犯東井。辛巳,歲星犯壘壁陣。十二月戊戌,太白犯東咸。丁未,太陰犯東井。丁巳,太陰犯氐。



 二十四年正月甲戌,太陰犯東井。乙酉,太陰犯房。二月庚子,太陰犯天關。辛丑,太陰犯東井。閏二月癸亥,太陰犯辰星。甲申,太陰犯牽牛。三月丙申,太陰犯東井。四月癸酉,太陰犯氐。甲戌,太陰犯房。七月戊戌,太陰犯南斗。辛丑,太陰犯牽牛。壬寅,熒惑犯輿鬼積尸氣。甲辰,熒惑犯輿鬼。壬子,太陰犯司怪。八月癸亥,太白犯亢。丙子,填星南犯壘壁陣。己卯,太陰犯天關。辛巳,太陰犯東井。甲申,太白犯房。九月丁酉,熒惑犯長垣。庚子,太白犯天江。乙巳,太陰犯畢。辛亥,熒惑犯太微西垣上將。壬子,太白犯南斗。十月壬戌,太陰犯牽牛大星。乙酉,熒惑犯左執法。十一月壬辰,太白犯壘壁陣,太陰暈太白、填星。丙申,熒惑犯太微東垣上將。庚子,太白晝見。丙辰,熒惑犯進賢。十二月丙寅,太陰犯畢,太白晝見。



 二十五年正月乙巳,太陰犯角。戊申,太陰犯房。三月丁亥,熒惑犯太微東垣上相。戊子,太陰犯畢。己亥,太陰掩角。四月戊午,太陰犯井。五月戊申,太白犯畢。六月甲戌,太白犯井。丁丑,太陰犯歲星。七月己亥,熒惑犯氐。庚子,太白犯鬼。乙巳,太陰掩畢。八月丙辰,熒惑犯房。己未,太白犯軒轅大星。九月癸未朔,熒惑犯天江。庚子,太陰犯畢。癸卯,熒惑犯南斗。十二月辛酉,太陰犯畢。甲子,太陰犯井。甲戌,太陰犯亢,熒惑犯壘壁陣。



 二十六年正月辛丑,太陰犯氐。三月甲午,太陰犯亢。五月壬辰,太白犯鬼。七月戊子,太白經天四十五日。辛卯,太陰犯牛。乙未,太陰犯歲星。八月辛未,歲星晝見。



 九月戊寅,歲星犯井。乙未,太陰犯畢。丙申,熒惑犯太微西垣上將。十月癸丑,太陰犯牛宿距星。甲寅,熒惑犯右執法。閏十月丁亥,辰星犯房。己丑,太陰犯畢,熒惑犯進賢,太陰犯井。十一月丁巳,熒惑犯亢。戊辰,太陰犯亢。



 二十七年正月庚戌,太白犯牛。癸丑,太陰犯井。丁卯,熒惑犯房。壬申,熒惑犯鍵閉。二月戊寅,太陰犯畢。庚寅,太陰犯亢。三月壬子,熒惑犯鉤鈐。四月丙子,太陰犯井。壬辰,熒惑守氐十餘日。五月乙丑,太陰犯填星。六月己丑,熒惑犯房。七月辛酉,熒惑犯天江。九月癸卯,歲星犯鬼。十月辛巳,太白犯鬥。十一月戊申,太陰掩填星。辛酉,太陰掩左執法。十二月辛卯,太陰犯亢。



 二十八年正月壬寅,太白、熒惑、填星聚奎。二月癸未,太陰犯左執法。甲申,太白犯昴。三月丁未,太陰犯御女。己酉,太陰犯右執法。庚戌,太陰犯太微東垣上相。乙卯,太白犯五車。四月乙未,歲星犯輿鬼積尸氣。五月壬寅,太陰犯少民。甲寅,太陰犯牛。六月辛卯,太陰犯畢。七月己亥,太白犯井。八月丙寅,太白犯輿鬼。丙子,太陰犯牽牛。癸未,歲星犯軒轅大星。戊子,太白犯軒轅大星,並犯歲星。癸巳,太陰掩熒惑。九月丙辰,熒惑犯左執法。戊午,太白犯熒惑。辛酉,歲星犯少民。十月丙戌,太陰犯軒轅大星並御女。己丑,太陰犯太微東垣上相。十一月甲辰,太白犯房。丙午,熒惑犯亢。丁未,太陰犯畢。庚申,熒惑犯氐。十二月庚辰,太陰犯御女。癸未,太陰犯東垣上相。己丑,熒惑犯房。庚寅,熒惑犯鉤鈐。



 二十九年正月戊申,太陰犯歲星及軒轅左角。二月己巳,太陰犯畢。四月丙子,太陰犯氐。六月己丑,太白犯歲星。閏六月戊申,熒惑犯狗國。七月辛未,太陰犯牛。八月丁酉,辰星犯右執法。己亥,太白犯房。乙巳,歲星犯右執法。九月壬戌,熒惑犯壘壁陣。辛巳,太白犯南斗。十月乙巳,太陰犯井。丁未,太陰犯鬼。乙卯,太陰犯氐。十一月壬戌,太陰犯壘壁陣。己卯,太陰犯太微東垣上將。十二月庚子,太陰犯井。甲辰,太陰犯太微西垣上將。



 三十年正月丙寅,太陰犯畢。丁丑,太陰犯氐。庚辰,歲星犯左執法。二月壬辰,太陰犯畢。乙巳,熒惑犯天街。庚戌,太陰犯牛。癸丑,太白犯壘壁陣。三月辛未,太陰犯氐。四月癸丑,太白犯填星。六月己丑,歲星犯左執法。丙申,太陰犯鬥。七月甲子,太陰犯建星。辛巳,太陰犯鬼。八月甲午,辰星犯太微西垣上將。甲辰,太陰犯畢。戊申,太陰犯鬼。九月丁卯,太陰犯畢。十月庚寅,彗星入紫微垣,抵鬥魁,光芒尺許,凡一月乃滅。丙申,熒惑犯亢。己亥,太陰犯天關。辛丑,太陰犯井。十一月乙丑,太陰犯畢。丁卯,太陰犯井。庚午,太陰犯鬼。丙子,熒惑犯鉤鈐。戊寅,歲星犯亢。十二月乙未,太陰犯井。



 三十一年四月戊申,太白晝見,又犯鬼。五月庚戌朔,太白犯輿鬼。六月丙午,太陰犯井。八月庚辰,太白晝見。戊戌,太陰犯畢,太白犯軒轅。九月丁巳,太白經天。丙寅,太陰掩填星。辛未,太陰犯軒轅。乙亥,太白犯右執法,太陰犯平道。十月壬午,太白犯左執法。癸巳,太陰掩填星。乙未,太陰犯井。十一月己酉,太陰犯亢。庚申,太陰犯畢。癸酉,太白犯房。十二月癸未,歲星犯房。丁亥,歲星犯鉤鈐。壬辰,太陰犯鬼。庚子,太陰犯房,又犯歲星。



 成宗元貞元年正月乙卯,太陰犯填星,又犯畢。癸酉,歲星犯東咸。二月癸未,熒惑犯太陰。壬辰,太陰犯平道。癸卯,太陰犯歲星。三月庚戌,太陰犯填星。壬戌,太陰犯房。四月庚寅,太陰犯東咸。閏四月癸丑,歲星犯房。甲寅,太陰犯平道。乙卯,太陰犯亢。丁巳,太陰掩房。五月丁亥,太陰犯南斗。七月丁丑,太陰犯亢。甲申,歲星犯房。八月乙酉,太陰犯牛。壬子,太陰犯壘壁陣。九月甲午,太陰犯軒轅。戊戌,太陰犯平道。十月辛酉,辰星犯房。壬戌,辰星犯鍵閉。戊辰,太白晝見,太陰犯房。十一月甲戌,太白經天及犯壘壁陣。乙酉,太陰犯井。丁亥,太陰犯鬼。十二月丙辰,太陰犯軒轅。甲子,太陰犯天江。



 二年正月壬午,太陰犯輿鬼。丙戌,太白晝見。丁亥,太陰犯平道。庚寅,太陰犯鉤鈐。二月丁未,太陰犯井。三月乙酉,太陰犯鉤鈐。五月丁丑,太陰犯平道。六月乙巳,太白犯天關。丁巳,太白犯填星。癸亥,太陰犯井。七月壬午,填星犯井,太白犯輿鬼。八月庚子,太陰犯亢,太白犯軒轅。癸卯,太陰犯天江。乙卯,太陰犯天街,太白犯上將。九月戊辰,太白犯左執法。壬申,太陰掩南斗。丁丑,太陰犯壘壁陣。己丑,太陰犯軒轅。十一月丁丑,太陰犯月星,又犯天街。庚辰,太陰犯井。丁亥,太陰犯上相。戊子,太陰犯平道。壬辰,太陰犯天江。十二月丁未,太陰犯井。乙卯,太陰犯進賢。



 大德元年三月戊辰,熒惑犯井。癸酉,太陰掩軒轅大星。五月癸酉,太白犯鬼積尸氣。乙亥,太陰犯房。六月乙未,太白晝見。七月庚午,太陰犯房。八月丁巳,妖星出奎。



 九月辛酉朔,妖星復犯奎。十月戊午,太白經天。十一月戊子,太白經天。十二月甲辰,太白經天,又犯東咸。丙午,太陰犯軒轅。甲寅,太陰犯心。閏十二月癸酉,太白犯建星。丙子,太白犯建星。



 二年二月辛酉,歲星、熒惑、太白聚危,熒惑犯歲星。辛未,太陰犯左執法。丙子,太陰犯心。五月戊戌,太陰犯心。六月壬戌,太陰犯角。七月癸巳,太陰犯心。八月壬戌,太陰犯箕。九月辛丑,太陰犯五車南星。癸卯,太陰犯五諸侯。己酉,太陰犯左執法。十月壬戌,太白犯牽牛。戊寅,太陰犯角宿距星。十一月己亥,太陰犯輿鬼。辛丑,辰星犯牽牛。壬寅,太陰犯右執法。十二月戊午,太白經天。己未,填星犯輿鬼。乙丑,太白犯歲星,太陰犯熒惑。庚午,填星入輿鬼,太陰犯上將。甲戌,彗出子孫星下。己卯,太陰犯南斗。



 三年正月丙戌,太陰犯太白。丁酉,太陰犯西垣上將。戊戌,太陰犯右執法。乙巳,太白經天。三月乙巳,熒惑犯五諸侯。戊戌,熒惑犯輿鬼。四月己未,太陰犯上將。丙寅,填星犯輿鬼,太陰犯心。五月丙申,太陰犯南斗。己亥,太白犯畢。六月庚申,太陰掩房。丁卯,熒惑犯右執法。壬申,歲星晝見。七月己卯朔,太白犯井。丁未,太陰犯輿鬼。八月丁巳,太陰犯箕。戊辰,太白犯軒轅大星。己巳,太陰犯五車星。九月壬辰,流星色赤,尾長尺餘,其光燭地,起自河鼓,沒於牽牛之西,有聲如雷。乙未,太陰犯昴宿距星。丁酉,太白犯左執法。十月丙子,太陰犯房。十一月乙酉,太白犯房。



 四年二月戊午,太陰犯軒轅。五月甲午,太陰犯壘壁陣。辛丑,太白犯輿鬼,太陰犯昴。六月丁巳,太白犯填星。七月辛卯,熒惑犯井。八月癸丑,太陰犯井。甲子,辰星犯靈臺上星。閏八月庚辰,熒惑犯輿鬼。九月戊午,太白犯鬥。壬戌,太陰犯輿鬼。甲子,太白犯鬥。十二月庚寅,熒惑犯軒轅。癸巳,太陰犯房宿距星。



 五年正月己酉,太陰犯五車。壬子,太陰犯輿鬼積尸氣。辛酉,太陰犯心。二月己卯,太陰犯輿鬼。三月戊申,太陰犯御女。丁卯,熒惑犯填星。己巳,熒惑、填星相合。四月壬申,太陰犯東井。五月癸丑,太陰犯南斗。乙卯,熒惑犯右執法。丁卯,太白犯井。六月甲申,歲星犯司怪。己酉,太白犯輿鬼,歲星犯井。甲午,太白犯輿鬼。七月丙午,歲星犯井。辛亥,太陰犯壘壁陣。庚申,辰星犯太白。八月壬辰,太陰犯軒轅御女。乙未,填星犯太微上將。九月乙丑,自八月庚辰,彗出井二十四度四十分,如南河大星,色白,長五尺,直西北,後經文昌斗魁,南掃太陽,又掃北斗、天機、紫微垣、三公、貫索,星長丈餘,至天市垣巴蜀之東、梁楚之南、宋星上,長盈尺,凡四十六日而滅。十月癸未,太陰犯東井。辛卯,夜有流星,大如杯,色赤,尾長丈餘,光燭地,自北起,近東徐徐而行,分為二星,前大後小,相離尺餘,沒於危宿。十一月己亥,歲星犯東井。戊申,太陰犯昴。十二月甲戌,歲星犯司怪。辛卯,太陰犯南斗。



 六年正月壬戌,填星犯太微西垣上將。二月庚午,太陰犯昴。三月壬寅,太陰犯輿鬼。癸卯,歲星犯井。甲寅,太陰犯鉤鈐。四月乙丑朔,太白犯東井。戊寅,太陰犯心。庚寅,太白犯輿鬼。六月癸亥朔,填星犯太微西垣上將。乙亥,太陰犯鬥。七月癸巳朔,熒惑、填星、辰星聚井。庚子,太陰犯心。戊午,太陰犯熒惑。八月乙丑,熒惑犯歲星。己巳,熒惑犯輿鬼。辛巳,太陰犯昴。壬午,太白犯軒轅。九月丙午,熒惑犯軒轅。癸丑,太陰犯輿鬼。丁巳,太白犯右執法。十月壬午,熒惑犯太微西垣上將。十一月辛卯,填星犯左執法。乙未,辰星犯房。癸卯,太陰犯昴。己酉,太陰犯軒轅。十二月庚申朔,熒惑犯填星。乙丑,歲星犯輿鬼。乙亥,太陰犯輿鬼。庚辰,熒惑犯太微東垣上相。癸未,太陰犯房。



 七年正月戊戌,太陰犯昴。甲辰,太陰犯軒轅。二月戊寅,太陰犯心。四月癸亥,太陰犯東井。丙寅,太陰犯軒轅。乙亥,歲星犯輿鬼,太陰犯南斗。甲申,熒惑犯太微垣右執法。丁亥,歲星犯輿鬼。五月壬辰,辰星犯東井。閏五月戊辰,太陰犯心。七月戊寅,歲星犯軒轅。己卯,太陰犯井。乙酉,熒惑犯房。八月癸巳,太白犯氐。甲午,熒惑犯東咸,太陰犯牽牛。乙巳,歲星犯軒轅。辛亥,熒惑犯天江。九月丙寅,太白晝見。辛未,熒惑犯南斗。甲戌,太陰犯東井。乙亥,太白犯南斗。壬午,辰星犯氐。十月丁亥,太白經天。辛丑,太陰犯東井。十一月己未,太白經天。丙寅,填星犯進賢。戊辰,太陰犯東井。己卯,太陰犯東咸。十二月丙戌,太白經天。夜,熒惑犯壘壁陣。丙申,太陰犯東井。辛丑,太陰犯明堂。丁未,太陰犯天江。



 八年三月乙丑,自去歲十二月庚戌,彗星見,約盈尺,指東南,色白,測在室十一度,漸長尺餘,復指西北,掃騰蛇,入紫微垣,至是滅,凡七十四日。



 九年正月丁巳,太陰犯天關。甲子,太陰犯明堂。己巳,太陰犯東咸。三月甲寅,熒惑犯氐。戊午,歲星犯左執法。四月庚辰,太陰犯井。壬辰,太白犯井。五月癸亥,歲星掩左執法。七月丙午,熒惑犯氐。甲寅,太白經天。丁卯,熒惑犯房。八月辛巳,太陰犯東咸。乙未,熒惑犯天江。九月丁巳,熒惑犯鬥。十月丙戌,太白經天。十一月庚戌,歲星、太白、填星聚於亢。癸丑,歲星犯亢。丙寅,歲星晝見。壬申,太白經天。十二月丙子,太白犯西咸。庚寅,熒惑犯壘壁陣。己亥,辰星犯建星。



 十年正月丁巳,太白犯建星。閏正月癸酉,太白犯牽牛。己丑,太白犯壘壁陣。二月戊午,太陰犯氐。三月戊寅,歲星犯亢。四月辛酉,填星犯亢。六月癸丑,太陰犯羅堰上星。己未,歲星犯亢。七月庚辰,太陰犯牽牛。八月壬寅,歲星犯氐,熒惑犯太微垣上將。九月己巳,熒惑犯太微垣右執法。壬午,熒惑犯太微垣左執法。十月甲辰,太白犯鬥。辛亥,太陰犯畢。甲寅,太陰犯井。十一月辛未,歲星犯房。壬申,太陰犯虛。甲戌,熒惑犯亢。戊子,熒惑犯氐。辛卯,太陰犯熒惑。十二月壬寅,太白晝見。乙巳,歲星犯東咸。戊午,太陰犯氐。



 十一年六月丙午,太陰犯南斗杓星。七月己巳,太陰犯亢。壬午,熒惑犯南斗。九月癸酉,太白犯右執法。己卯,太白犯左執法。十月乙巳,太白犯亢。己酉,熒惑犯壘壁陣。甲寅,太陰犯明堂。己未,太陰犯太白。十一月丁卯,太白犯房。丙子,太陰犯東井。乙酉,太陰犯亢。辛卯,辰星犯歲星。十二月丁巳,填星犯鍵閉。



 武宗至大元年正月辛未,太陰犯井。甲申,太陰犯填星。二月丁未,太陰犯亢。甲寅,太陰犯牛距星。三月乙丑,太陰犯井。五月癸未,太白犯輿鬼。七月庚申,流星起自勾陳,南至於大角傍,尾跡約三尺,化為白氣,聚於七公,南行,圓若車輪,微有銳,經貫索滅。壬申,太白犯左執法。八月壬子,太陰犯軒轅太民。九月壬申,填星犯房。丙子,太陰犯井。癸未,太陰犯熒惑。十月辛丑,太白犯南斗。十一月庚申,太白晝見。癸亥,熒惑犯亢。己巳,太陰掩畢。甲戌,熒惑犯氐。乙亥,辰星犯填星。閏十一月壬寅,熒惑犯房。丁未,太陰犯亢。十二月甲子,太陰犯畢。丙子,太陰犯氐。戊寅,太白掩建星。



 二年二月己巳,太陰犯亢。辛未,太陰犯氐。庚辰,太陰犯太白。三月戊戌,太陰犯氐。己亥,熒惑犯歲星。丙午,熒惑犯壘壁陣。五月辛卯,太陰犯亢。六月乙卯,太白犯井。癸酉,辰星犯輿鬼。乙亥,太陰掩畢。八月乙亥,太陰犯軒轅。丁丑,太陰犯右執法。九月丙午,太陰犯進賢。十月壬申,太陰犯左執法。十一月己亥,太陰犯右執法。庚子,太陰犯上相。辛丑,熒惑犯外屏。十二月庚申,太陰犯參。癸亥,辰星犯歲星。辛未,太白犯壘壁陣。



 三年正月壬辰,太陰犯軒轅御女。甲午,太陰犯右執法。丙申,太陰犯平道。二月辛亥,熒惑犯月星。庚申,熒惑犯天街,太陰犯軒轅少民。壬戌,太陰犯左執法。甲戌,太白犯月星。三月甲申,太陰犯井。庚寅,太陰犯氐。丙申,太陰犯南斗。丁未,太白犯井。甲寅,太陰犯軒轅御女。戊辰,太白晝見。五月乙酉,太陰犯平道。癸巳,熒惑犯輿鬼。六月乙卯,太陰犯氐。七月戊寅,太陰犯右執法。己卯,太陰犯上相。八月甲子,太白犯軒轅太民。乙丑,太陰掩畢大星。九月辛巳,太陰犯建星。辛卯,太陰犯天廩。十月甲辰朔,太白經天。丙午,太白犯左執法。癸丑,熒惑犯亢。十一月甲戌朔,太白犯亢。丁亥,太陰犯畢。十二月甲辰朔,太陰犯羅堰。庚申,太陰犯軒轅大星。辛酉,太白犯填星。丙寅,太白犯氐。



 四年二月甲子,太陰犯填星。三月丙戌,太陰犯太微上將。四月甲寅,太陰犯亢,熒惑犯壘壁陣。五月癸未,太陰犯氐。乙未,太陰犯太微東垣上相。六月庚戌,太陰犯氐。



 七月癸巳,太陰掩畢。丁酉,太陰犯鬼宿距星。閏七月丙寅,太陰犯軒轅。九月乙卯,太陰犯畢。十月丙申,太白犯壘壁陣。十一月甲寅,太陰犯輿鬼。十二月庚辰,太白經天。癸未,亦如之。甲申,太陰犯太微西垣上將。壬辰,太白經天。



 仁宗皇慶元年正月癸丑,太陰犯太微東垣上相。二月壬午,太陰犯亢。三月丁酉朔,熒惑犯東井。壬寅,太陰犯東井。四月丙子,太白晝見。壬午,熒惑犯輿鬼。癸未,熒惑犯積尸氣。庚寅,太白經天。六月己巳,太陰犯天關。七月戊午,太陰犯東井。八月戊辰,太白犯軒轅。辛未,太陰犯填星。壬午,辰星犯右執法。乙酉,太白犯右執法。丁亥,辰星犯左執法。九月丁巳,太白犯亢。十月丁亥,太陰犯平道。戊子,太陰犯亢。十一月己亥,太陰掩壘壁陣。十二月甲申,熒惑、填星、辰星聚鬥。戊子,太陰犯熒惑。



 二年正月戊申,太陰犯三公。三月庚子,熒惑犯壘壁陣。丁未,彗出東井。七月己丑朔,歲星犯東井。辛卯,太白晝見。乙未、丙辰,皆如之。丁巳,太白經天。八月戊午朔,太白晝見。壬戌,歲星犯東井。壬午,太陰犯輿鬼。



 延祐元年二月癸酉,熒惑犯東井。三月壬辰,太陰掩熒惑。



 閏閏月辛酉,太陰犯輿鬼。丙寅,太陰犯太微東垣。五月戊午,辰星犯輿鬼。六月乙未,熒惑犯右執法。十月庚戌,辰星犯東咸。十二月甲午,太陰犯輿鬼。癸卯,太陰犯房。甲辰,太陰犯天江。



 二年正月乙卯,歲星犯輿鬼。己未,太白晝見。癸亥,太陰犯軒轅。丁卯,太陰犯進賢。二月戊子,太白晝見。癸巳,太白經天。丙午,亦如之。三月丙辰,太陰色赤如赭。四月庚子,太陰犯壘壁陣。五月辛酉,太陰犯天江。庚午,太白晝見。六月甲申,太白晝見。是夜,太陰犯平道。癸卯,太白犯東井。丙午,辰星犯輿鬼。九月己酉,太陰犯房。辛酉,太白犯左執法。十月丙子朔,客星見太微垣。十一月丙午,客星變為彗,犯紫微垣,歷軫至壁十五宿,明年二月庚寅乃滅。



 三年九月癸丑,太白晝見。丙寅,太白經天。十月甲申,太白犯鬥。



 四年三月乙酉,太陰犯箕。六月乙巳,太陰犯心。八月丙申,熒惑犯輿鬼。壬子,太陰犯昴。九月庚午,太陰犯鬥。



 六年正月戊寅,太陰犯心。二月己亥,太陰犯靈臺。三月己巳,太陰犯明堂。癸酉,太陰犯日星。甲戌,太陰犯心。五月辛酉,太陰犯靈臺。丁卯,太陰犯房。丙子,太陰犯壘壁陣。六月己亥,歲星犯東咸。七月壬戌,太陰犯心。丙子,太白犯太微垣右執法。



 八月乙酉,熒惑犯輿鬼。閏八月丙辰,辰星犯太微垣右執法。丁巳,太陰犯心。癸亥,熒惑犯軒轅。甲子,太陰犯壘壁陣。乙亥,太白犯東咸。十月癸亥,熒惑犯太微垣左執法。乙丑,太陰犯昴。戊辰,太陰犯東井。庚午,太白晝見。辛未,太陰犯軒轅。辛卯,熒惑犯進賢。庚子,太陰犯明堂。十二月丙寅,太陰犯軒轅。



 七年正月乙未,太陰犯明堂上星。癸卯,太陰犯鬥宿東星。二月辛酉,太陰犯軒轅御女。壬戌,太陰犯靈臺。丁卯,太陰犯日星。庚午,太陰犯鬥宿距星。三月戊子,太陰犯酒旗上星,熒惑犯進賢。庚寅,太陰犯明堂上星。四月甲寅,太白犯填星。壬戌,太陰犯房宿距星。五月庚寅,太陰犯心宿東星。癸巳,太陰犯狗宿東星。丙申,太白犯畢宿距星。六月庚申,太陰犯鬥宿東星。癸亥,太陰犯壘壁陣西二星。丁卯,太白犯井宿東扇第三星。辛未,太陰犯昴宿。七月丁亥,太陰犯鬥宿東三星。戊戌,熒惑犯房宿上星。己亥,太陰犯昴宿距星。八月丙辰,太白犯靈臺上星。乙丑,熒惑犯天江。丁卯,太白犯太微垣右執法。壬申,太陰犯軒轅御女。九月乙酉,太陰犯壘壁陣西二星。丙戌,熒惑犯鬥宿。癸巳,太陰犯昴宿東星。己亥,太白犯亢星。十月庚戌,太陰犯熒惑於斗。癸亥,太陰犯井宿。十一月癸卯,熒惑犯壘壁陣。乙卯,太陰掩昴宿。戊午,太陰犯井宿東星。庚申,太陰犯鬼宿。



 英宗至治元年正月乙未,太陰掩房宿距星。甲辰,辰星犯外屏西第一星,辰星、太白、熒惑、填星聚於奎宿。二月壬子,太白、熒惑、填星聚於奎宿。辛酉,太白犯熒惑。癸亥,太陰犯心宿大星,又犯心宿東星。三月丁丑,太陰掩昴宿。四月戊午,太陰犯心宿大星。庚申,太陰犯鬥宿東第三星。五月戊寅,太白犯鬼宿積尸氣,太陰犯軒轅右角。庚辰,太陰犯明堂中星。六月己未,太陰犯虛梁東第二星。辛酉,太白經天。七月癸巳,太陰犯昴宿。八月丁未,太陰犯心宿前星。己酉,太陰犯鬥宿西第二星。壬子,熒惑犯軒轅大星。九月乙亥,熒惑犯靈臺東北星。壬午,熒惑犯太微西垣上將。丁酉,熒惑犯太微垣右執法。十月甲辰,太白經天。戊申,熒惑犯太微垣左執法。十一月辛未,熒惑犯進賢。丙子,太陰犯虛梁東第一星。戊寅,辰星犯房宿上星。丙戌,太陰犯井宿東扇北第二星。己丑,太陰犯酒旗西星,又犯軒轅右角。辛卯,太陰犯明堂中星。己亥,太白犯西咸南第一星。十二月甲辰,熒惑犯亢宿南第一星。庚戌,太陰犯昴宿東第一星。辛酉,熒惑入氐宿。



 二年正月丁丑,太陰犯昴宿距星。庚辰,太白犯建星西第二星。辛巳,太白犯建星西第三星。辛卯,太陰犯心宿大星。甲午,熒惑犯房宿上星。丁酉,太白犯牛宿南第一星。



 二月己亥朔,熒惑犯鍵閉星。丙午,熒惑犯罰星南一星。戊申,太陰犯井宿東扇北第二星。庚戌,熒惑犯東咸北第二星。辛亥,太陰犯酒旗西第一星及軒轅右角星。壬子,太白犯壘壁陣西方第二星。癸丑,太陰犯明堂中星。己未,太陰犯天江南第一星。壬戌,太白犯壘壁陣西方第二星。癸丑,太陰犯明堂中星。己未,太陰犯天江南第一星。壬戌,太白犯壘壁陣第六星。五月丙子,熒惑退犯東咸南第一星。六月壬申,熒惑犯心宿距星。七月己亥,熒惑犯天江南第一星。戊午,太陰犯井宿鉞星。九月己未,太陰犯明堂中星。十月庚辰,太陰犯井宿距星。辛巳,太陰犯井宿東扇北第二星及第三星。己丑,熒惑犯壘壁陣西第六星。十一月甲辰,太白犯壘壁陣第一星。乙巳,熒惑犯壘壁陣西第八星。戊申,太陰掩井宿東扇北第二星。己未,太陰犯東咸南第一星。庚申,太陰犯天江上第二星。辛酉,熒惑犯歲星。十二月乙丑,太白、歲星、熒惑聚於室,太白犯壘壁陣西第八星。乙亥,太陰掩井宿距星。戊寅,太白犯歲星。己丑,熒惑犯外屏西第三星,太陰犯建星西第二星。



 三年正月壬寅,太陰犯鉞星,又犯井宿距星。癸卯,太陰犯井宿東扇南第二星。二月癸亥朔,熒惑、太白、填星聚於胃宿。癸酉,太白犯昴宿。辛巳,太陰犯東咸南第一星、第二星。五月戊戌,太白經天。癸卯,太陰犯房宿第二星。庚戌,太白犯畢宿右股第三星。六月癸未,填星犯畢宿距星。九月辛卯,填星退犯畢。十月己巳,太白犯亢。丙子,太白犯氐。十一月己丑朔,熒惑犯亢。庚寅,太白犯鉤鈐。乙未,太白犯東咸。壬寅,熒惑犯氐。十二月己巳,辰星犯壘壁陣。辛未,熒惑犯房。辛巳,熒惑犯東咸。



 泰定帝泰定元年五月丙午,太白犯鬼宿。丁未,太白又犯鬼宿積尸氣。十月丙寅,太白犯鬥宿距星。己巳,太白入斗宿魁,太陰犯填星。庚午,太白犯鬥。壬午,熒惑犯壘壁陣。十二月庚午,熒惑犯外屏。乙亥,太白經天。



 二年正月丙戌,辰星犯天雞。壬寅,太白犯建星。二月庚寅,熒惑、歲星、填星聚於畢宿。六月丙戌,填星犯井宿鉞星。丙午,填星犯井宿。八月癸巳,歲星犯天罇。十月壬辰,熒惑犯氐宿。癸巳,填星退犯井宿。十一月戊午,填星退犯井宿鉞星。十二月乙酉,熒惑犯天江,辰星犯建星。甲午,太白犯壘壁陣。



 三年正月辛酉,太白犯外屏。三月丙午,填星犯井宿鉞星。戊辰,熒惑犯壘壁陣,填星犯井宿。庚午,填星、太白、歲星聚於井。四月戊戌,太白犯鬼宿。壬寅,熒惑犯壘壁陣。七月戊辰,太白經天,至於十二月。九月壬戌,太白犯太微垣右執法。十月辛巳,太白犯進賢。



 四年正月己酉,太白犯牛宿。三月丁卯,熒惑犯井宿。九月壬子,太白犯房宿。閏九月己巳,太白經天,至十二月。十月乙巳,晝有流星。戊午,辰星犯東咸。十一月癸酉,太白犯壘壁陣,熒惑犯天江。十二月己未,歲星退犯太微西垣上將。



 致和元年二月壬戌,太白晝見。五月庚辰,流星如缶大,光明燭地。七月丙戌,太白犯軒轅大星。



 文宗天歷元年九月庚辰,太白犯亢宿。



 二年正月甲子,太白犯壘壁陣。二月己酉,熒惑犯井宿。五月庚申,太白犯鬼宿積尸氣。六月丁未,太白晝見。七月癸亥,太白經天。十一月癸酉,太陰犯填星。



 至順元年七月庚午,歲星犯氐宿。八月戊辰,太白犯氐宿。九月己丑,熒惑犯鬼宿。甲午,熒惑犯鬼宿。十一月甲申,熒惑退犯鬼宿。丙戌,太白犯壘壁陣。



 二年二月壬子,太白晝見。三月丙子朔,熒惑犯鬼宿。己卯,熒惑犯鬼宿積尸氣。五月丁丑,熒惑犯軒轅左角。甲午,太白犯畢宿。庚子,太陰犯太白。辛丑,太白經天。六月丁未,太白晝見。丁卯,太陰犯畢,太白犯井。八月乙卯,太白犯軒轅大星。庚申,太白犯軒轅左角。九月丙子,太白犯填星。十一月壬申朔,太白犯鉤鈐。



 三年五月癸酉,熒惑犯東井。



\end{pinyinscope}