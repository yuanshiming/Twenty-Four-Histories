\article{本紀第三十八 順帝一}

\begin{pinyinscope}

 順帝名妥歡帖睦爾,明宗之長子。母罕祿魯氏,名邁來迪,郡王阿兒廝蘭之裔孫也。初,太祖取西北諸國,阿兒廝蘭率其眾來降,乃封為郡王,俾領其部族。及明宗北狩,過其地,納罕祿魯氏。延祐七年四月丙寅,生帝於北方。



 當泰定帝之崩,太師燕鐵木兒與諸王、大臣迎立文宗。文宗既即位,以明宗嫡長,復遣使迎立之。明宗即位於和寧之北,而立文宗為皇太子。及明宗崩,文宗復正大位。至順元年四月辛丑,明宗後八不沙被讒遇害,遂徙帝於高麗,使居大青島中,不與人接。閱一載,復詔天下,言明宗在朔漠之時,素謂非其己子,移於廣西之靜江。



 三年八月己酉,文宗崩,燕鐵木兒請文宗後立太子燕帖古思,後不從,而命立明宗次子懿璘只班,是為寧宗。十一月壬辰,寧宗崩,燕鐵木兒復請立燕帖古思,文宗後曰:「吾子尚幼,妥歡貼睦爾在廣西,今年十三矣,且明宗之長子,禮當立之。」乃命中書左丞闊里吉思迎帝於靜江。至良鄉,具鹵簿以迓之。燕鐵木兒既見帝,並馬徐行,具陳迎立之意。帝幼且畏之,一無所答。於是燕鐵木兒疑之,故帝至京,久不得立。適太史亦言帝不可立,立則天下亂,以故議未決。遷延者數月,國事皆決於燕鐵木兒,奏文宗後而行之。俄而燕鐵木兒死,後乃與大臣定議立帝,且曰:「萬歲之後,其傳位於燕帖古思,若武宗、仁宗故事。」諸王宗戚奉上璽綬勸進。



 四年六月己巳,帝即位於上都,詔曰:



 洪惟我太祖皇帝,受命於天,肇造區夏;世祖皇帝,奄有四海,治功大備;列聖相傳,丕承前烈。我皇祖武宗皇帝入纂大統,及致和之季,皇考明宗皇帝遠居沙漠,札牙篤皇帝戡定內難,讓以天下。我皇考賓天,札牙篤皇帝復正宸極。治化方隆,奄棄臣庶。今皇太后召大臣燕鐵木兒、伯顏等曰:「昔者闊徹伯、脫脫木兒、只兒哈郎等謀逆,以明宗太子為名,又先為八不沙始以妒忌,妄構誣言,疏離骨肉。逆臣等既正其罪,太子遂遷於外。札牙篤皇帝後知其妄。尋至大漸,顧命有曰:『朕之大位,其以朕兄子繼之。』」時以朕遠征南服,以朕弟懿璘只班登大位,以安百姓,乃遽至大故。皇太后體承札牙篤皇帝遺意,以武宗皇帝之元孫,明宗皇帝之世嫡,以賢以長,在予一人,遣使迎還。徵集宗室諸王來會,合辭推戴。今奉皇太后勉進之篤,宗親大臣懇請之至,以至順四年六月初八日,即皇帝位於上都。於戲!惟天、惟祖宗全付予有家,慄慄危懼,若涉淵冰,罔知攸濟。尚賴宗親臣鄰,交修不逮,以底隆平。其赦天下。



 時有阿魯輝帖木兒者,明宗親臣也,言於帝曰:「天下事重,宜委宰相決之,庶可責其成功;若躬自聽斷,則必負惡名。」帝信之,由是深居宮中,每事無所專焉。辛未,命伯顏為太師、中書右丞相、上柱國、監修國史,兼奎章閣大學士,領學士院、太史院、回回、漢人司天監事;撒敦為太傅、左丞相。是月,大霖雨,京畿水平地丈餘,饑民四十餘萬,詔以鈔四萬錠賑之。涇河溢,關中水災。黃河大溢,河南水災。兩淮旱,民大饑。



 秋七月,霖雨。潮州路水。己亥,太陰犯房宿。



 八月壬申,鞏昌徽州山崩。是月,立燕鐵木兒女伯牙吾氏為皇后。



 九月甲午,太陰犯填星。乙未,太陰犯天江。甲寅,中書省臣言:「官員遞升,窒礙選法。今請自省、院、臺官外,其餘不許遞升。」從之。丁巳,太陰犯填星。己未,太陰犯氐宿。庚申,詔太師、右丞相伯顏,太傅、左丞相撒敦,專理國家大事,其餘官不得兼領三職。秦州山崩。賑恤寧夏饑民五萬三千人一月。詔免儒人役。



 冬十月甲子,太陰犯鬥宿。丙寅,鳳州山崩。戊辰,改元,詔曰:「在昔世祖皇帝,紹開丕圖,稽古建元,立經陳紀,列聖相承,恪遵成憲。肆予沖人,嗣大歷服,茲圖治之雲初,嘉與民而更始。乃新紀號,誕告多方,其以至順四年為元統元年。於戲!一元運於四時,惟裁成之有道;大統綿於萬世,思保佑於無疆。」中書省臣言:「凡朝賀遇雨,請便服行禮。」從之。己巳,加知樞密院事、答剌罕答里金紫光祿大夫。庚午,詔以察罕腦兒宣慰司人民,止令應當徽政院差發。癸酉,雲南人嵬羅土官渾鄧馬弄來貢方物,詔以其地升立散府。丁丑,依皇太后行年之數,釋放罪囚二十七人。庚辰,奉文宗皇帝及太皇太后御容於大承天護聖寺,命左丞相撒敦為隆祥使,奉其祭祀。乙酉,詔以高郵府為伯顏食邑。戊子,封撒敦為榮王,食邑廬州。唐其勢襲父封為太平王,進階金紫光祿大夫。庚寅,中書省臣請集議武宗、英宗、明宗三朝皇后升祔。



 十一月辛卯朔,罷富州金課。甲午,太陰犯壘壁陣。丙申,鞏昌成紀縣地裂山崩,令有司賑被災人民。丁酉,享於太廟。辛丑,起棕毛殿。丙午,申飭鹽運司。辛亥,江西、湖廣、江浙、河南復立榷茶運司。追謚札牙篤皇帝為聖明元孝皇帝,廟號文宗。時寢廟未建,於英宗室次權結彩殿,以奉安神主。封伯顏為秦王,錫金印。是日,秦州山崩地裂。夜,太陰犯太微東垣上相。壬子,太陰犯填星。癸丑,太陰犯亢宿。乙卯,以燕鐵木兒平江所賜田五百頃,復賜其子唐其勢。罷河間大報恩寺諸色人匠總管府。江浙旱饑,發義倉糧、募富人入粟以賑之。詔秦王、右丞相伯顏,榮王、左丞相撒敦,統百官,總庶政。



 十二月庚申,命伯顏提調彰德威武衛。乙丑,廣西徭寇湖南,陷道州,千戶郭震戰死,寇焚掠而去。壬申,遣省、臺官分理天下囚,罪狀明者處決,冤者辨之,疑者讞之,淹滯者罪其有司。以奴列你他代其父塔剌赤為耽羅國軍民安撫使司達魯花赤,錫三珠虎符。癸酉,太陰犯鬼宿。甲戌,禿堅帖木兒致仕,錫太尉印,置僚屬。乙亥,為皇太后置徽政院,設官屬三百六十有六員。太白犯壘壁陣,太陰犯軒轅。己卯,太陰犯進賢。癸未,太陰犯東咸。



 元統二年春正月庚寅朔,雨血於汴梁,著衣皆赤。辛卯,東平須城縣、濟寧濟州、曹州濟陰縣水災,民饑,詔以鈔六萬錠賑之。以御史大夫脫別臺為中書平章政事,阿里海牙為河南行省左丞相。丁酉,享於太廟。戊戌,四川大盤洞蠻謀穀什用遣男謀者什用來貢方物,即其地立盤順府,命謀穀什用為知府。遣吏部尚書帖住、禮部郎中智熙善使交趾,以《授時歷》賜之。太陰犯軒轅。癸卯,敕僧道與民一體充役。己酉,以上文宗皇帝謚號,遣官告祭於南郊。庚戌,太陰犯房宿。甲寅,罷廣教總管府,立行宣政院。乙卯,雲南土酋姚安路總管高明來獻方物,錫符印遣之。



 二月己未朔,詔內外興舉學校。癸亥,廣西徭寇邊,殺官吏。廣海官已除而未上者罪之。甲子,塞北東涼亭雹,民饑,詔上都留守發倉廩賑之。乙丑,命有司以時給宿衛冬衣。以燕不鄰為太保,置僚屬。戊辰,封也真也不乾為昌寧王,錫金印。癸酉,太陰犯太微上相。丁丑,封皇姑妥妥輝為英壽大長公主。癸未,安豐路旱饑,敕有司賑糶麥萬六千七百石。甲申,太廟木陛壞,遣官告祭。丁亥,太白經天。是月,灤河、漆河溢,永平諸縣水災,賑鈔五千錠。瑞州路水,賑米一萬石。



 三月己丑朔,詔:「科舉取士,國子學積分、膳學錢糧,儒人免役,悉依累朝舊制;學校官選有德行學問之人以充。」辛卯,以陰陽家言,罷造作四年。太陰犯填星。癸巳,廣西徭賊復起,殺同知元帥吉列思,掠庫物,遣右丞脫魯迷失將兵討之。復立西番巡捕都元帥府。罷廣誼司,復立覆實司。贈吉烈思官,令其子孫襲職。庚子,杭州、鎮江、嘉興、常州、松江、江陰水旱疾疫,敕有司發義倉糧,賑饑民五十七萬二千戶。癸卯,月食既。甲辰,中書省臣言:「興和路起建佛事,一路所費,為鈔萬三千五百三十餘錠。請依上都、大都例,給膳僧錢,節其冗費。」從之。乙巳,中書省臣言:「益都、真定盜起,請選省、院官往督捕之,仍募能擒獲者倍其賞,獲三人者與一官。」從之。丁未,以河南行省左丞相阿里海牙為江浙行省左丞相。壬子,廣西慶遠府徭賊寇全州,詔平章政事探馬赤統兵二萬人擊之。丁巳,詔:「蒙古、色目犯奸盜詐偽之罪者,隸宗正府;漢人、南人犯者,屬有司。」是月,山東霖雨,水湧,民饑,賑糶米二萬二千石。淮西饑,賑糶米二萬石。湖廣旱,自是月不雨至於八月。



 夏四月戊午朔,日有食之。庚申,封宗室蠻子為文濟王。乙丑,命順元等處軍民宣撫使、八番等處沿邊宣慰使伯顏溥花承襲父職。丙寅,罷龍慶州黑峪道上勝火兒站。庚午,詔:「雲南出征軍士亡歿者,人賜鈔二錠以葬。」壬申,命唐其勢為總管高麗女直漢軍萬戶府達魯花赤,與馬札兒臺並為御史大夫。丁丑,太白經天。戊寅,太白晝見。己卯,奉聖明元孝皇帝文宗神主祔於太廟,躬行告祭之禮,樂用宮懸,禮三獻。先是,御史臺臣言:「郊廟,國之大典,王者必行親祀之禮,所以盡尊尊、親親之誠,宜因升祔,有事於太廟。」帝從之。是日,罷夏季時享。詔加榮王、左丞相撒敦開府儀同三司、上柱國、錄軍國重事,食邑廬州。復立杭州四隅錄事司。太白晝見。壬午,復如之。帝嘉許衡輔世祖以不殺一天下,特錄其孫從宗為章佩監畢珍庫提點。癸未,立鹽局於京師南北城,官自賣鹽,以革專利之弊。乙酉,中書省臣言:「佛事布施,費用太廣,以世祖時較之,歲增金三十八錠、銀二百三錠四十兩、繒帛六萬一千六百餘匹、鈔二萬九千二百五十餘錠。請除累朝期年忌日之外,餘皆罷。」從之。是月,車駕時巡上都。益都、東平路水,設酒禁。大名路桑麥災。成州旱饑,詔出庫鈔及發常平倉米賑之。河南旱,自是月不雨至於八月。



 五月己丑,詔威武西寧王阿哈伯之子亦里黑赤襲其父封。宦者孛羅帖木兒傳皇后旨,取鹽一十萬引入中政院。辛卯,以唐其勢代撒敦為中書左丞相,撒敦仍商量中書省事。壬辰,命中書平章政事撒的領蒙古國子監。癸巳,罷洪教提點所。戊申,詔文濟王蠻子鎮大名,雲南王阿魯鎮雲南,給銀字團牌。是月,中書省臣言:「江浙大饑,以戶計者五十九萬五百六十四,請發米六萬七百石、鈔二千八百錠,及募富人出粟,發常平、義倉賑之,並存海運糧七十八萬三百七十石以備不虞。」從之。詔:「王侯宗戚軍站、人匠、鷹坊、控鶴,但隸京師諸縣者,令所在一體役之。」贈故中書平章政事王泰亨謚清憲。舊令,三品以上官,立朝有大節及有大功勛於王室者,得賜功臣號及謚。時浸冗濫失實,惟泰亨在中書時,安南請佛書,乞以《九經》賜之,使高麗不受禮遺,為尚書貧不能自給,故特賜是謚。贈漳州萬戶府知事闞文興英毅侯,妻王氏貞烈夫人,廟號雙節。六月丁巳朔,中書省臣言:「雲南大理、中慶諸路,曩因脫肩、敗狐反叛,民多失業,加以災傷,民饑,請發鈔十萬錠,差官賑恤。」從之。戊午,淮河漲,淮安路山陽縣滿浦、清岡等處民畜房舍多漂溺。丙寅,宣德府水災,出鈔二千錠賑之。乙亥,唐其勢辭左丞相不拜,復命撒敦為左丞相。辛巳,詔蒙古、色目人行父母喪。癸未,復立繕工司,造繒帛。乙酉,贈燕鐵木兒公忠開濟弘謨同德翊運佐命功臣、開府儀同三司、太師、中書右丞相,追封德王,謚忠武。是月,彰德雨白毛。大寧、廣寧、遼陽、開元、沈陽、懿州水旱蝗,大饑,詔以鈔二萬錠,遣官賑之。



 秋七月丁亥,戒陰陽人毋得於貴戚之家妄言禍福。辛卯,祭太祖、太宗、睿宗三朝御容。罷秋季時享。壬辰,帝幸大安閣。是日,宴侍臣於奎章閣。甲午,太白晝見。己亥,太白經天。壬寅,詔:「蒙古、色目人犯盜者免刺。」甲辰,太白經天,丙午,復如之。帝幸楠木亭。己酉,太白晝見。夜,有流星大如酒杯,色赤,長五尺餘,光明燭地,起自天津,沒於離宮之南。庚戌,太白經天,壬子,復如之。夜,熒惑犯鬼宿。癸丑、甲寅,太白復經天。是月,池州青陽、銅陵饑,發米一千石及募富民出粟賑之。



 八月丙辰朔,太白經天,凡四日。戊午,祭社稷。癸亥,太白經天。丙寅至戊辰,太白復經天。辛未,赦天下。京師地震。雞鳴山崩,陷為池,方百里,人死者甚眾。自是日至甲戌,太白經天,丁丑、己卯,復如之;夜,犯軒轅。庚辰至壬午,太白復經天。癸未,中書平章政事阿里海牙罷。是月,南康路諸縣旱蝗,民饑,以米十二萬三千石賑糶之。九月庚寅,太白經天。辛卯,車駕還自上都。壬辰,太陰入南斗。癸巳,太白犯靈臺。甲午,太白經天。徭賊陷賀州,發河南、江浙、江西、湖廣諸軍及八番義從軍,命廣西宣慰使、都元帥章伯顏將以擊之。乙未,太白經天,己亥、壬寅,復如之。乙巳,太白犯太微垣。壬子,吉安路水災,民饑,發糧二萬石賑糶。夜,太白犯太微垣。



 冬十月乙卯朔,正內外官朝會儀班次,一依品從。戊午,享於太廟。辛酉,以侍御史許有壬為中書參知政事。癸亥,太白犯太微上相,復犯進賢。丁卯,立湖廣黎兵屯田萬戶府,統千戶一十三所,每所兵千人,屯戶五百,皆土人為之,官給田土、牛、種、農器,免其差徭。又創立武安縣。移石山寨巡檢司於清水寨,立霍丘縣淮陰鄉臨水山巡檢司,改乾寧軍民安撫司曰乾寧安撫司。乙亥,太陰犯軒轅,太白犯填星。己卯,奉玉冊、玉寶,上皇太后尊號曰贊天開聖仁壽徽懿昭宣皇太后,詔曰:「朕登大寶,君臨萬方,永惟大母擁佑之勤;神器奠安,海宇寧謐,實慈訓之致然也。爰協眾議,再舉徽稱,而皇太后以文宗皇帝未祔於廟,至誠謙抑,弗賜俞允。今告祔禮成,亦既閱歲,始徇所請。乃以吉日奉上尊號,思與普天同茲大慶,其赦天下。」免今年民租之半,內外官四品以下減一資。卻天鵝之獻。癸未,命臺憲部官各舉材堪守令者一人。



 十一月戊子,中書省臣請發兩宗船下番,為皇後營利。濟南萊蕪縣饑,罷官冶鐵一年。辛卯,賜行宣政院廢寺錢一千錠以營公廨。乙未,填星犯亢宿。庚戌,熒惑犯太微垣。



 是月,鎮南王孛羅不花來朝。



 十二月,立道州永明縣白面墟、江華縣濤墟巡檢司各一,以鎮遏徭賊。甲戌,詔整治學校。是歲,禁私創寺觀庵院。僧道入錢五十貫,給度牒方出家。



 至元元年春正月癸巳,申命廉訪司察郡縣勸農官勤惰,達大司農司以憑黜陟。乙未,立徽政院屬官侍正府。丙午,雲南婦人一產三男。



 二月甲寅朔,革冗官。乙卯,車駕將田於柳林,御史臺臣諫曰:「陛下春秋鼎盛,宜思文皇付托之重,致天下於隆平。況今赤縣之民,供給繁勞,農務方興,而馳騁冰雪之地,倘有銜橛之變,奈宗廟社稷何!」遂止。丁巳,立縹甸散府一,穆由甸、範陵甸軍民長官司二。以薊州寶坻縣稻田提舉司所轄田土賜伯顏。戊午,祭社稷。甲戌,熒惑逆行入太微。己卯,以上皇太后冊、寶,遣官告祭天地。



 三月癸未朔,詔遣五府官決天下囚。御史臺臣言:「丞相已領軍國重事,省、院、臺官俱不得兼領各衛。」從之。平伐、都雲、定雲、酋長寶郎、天都蟲等來降,即其地復立宣撫司,參用其土酋為官。辛卯,以上皇太后寶、冊,遣官告祭太廟。壬辰,河州路大雪十日,深八尺,牛羊駝馬凍死者十九,民大饑。丙申,中書省臣言:「甘肅甘州路十字寺奉安世祖皇帝母別吉太后於內,請定祭禮。」從之。丁酉,以沾益州所轄羅山、石梁、交水三縣並歸巡檢司。月食。己亥,龍興路饑,出糧九萬九千八百石賑其民。庚子,御史臺臣言:「高麗為國首效臣節,而近年屢遣使往選取媵妾,至使生女不舉,女長不嫁,乞賜禁止。」從之。中書省臣言,帝生母太后神主宜於太廟安奉,命集議其禮。甲辰,山東、河間、兩淮、福建四處增鹽課一十八萬五千引,中書請權罷征,止令催辦正額。乙巳,以中書左丞王結、參知政事許有壬知經筵事。封安南世子陳端午為安南國王。是月,益都路沂水、日照、蒙陰、莒縣旱饑,賑米一萬石。



 夏四月癸丑朔,詔:「諸官非節制軍馬者,不得佩金虎符。」辛酉,享於太廟。以江南行御史臺中丞不花為中書省參知政事。壬戌,太陰犯左執法。丙寅,詔以鈔五十萬錠,命徽政院散給達達兀魯思、怯薛丹、各愛馬。己巳,加唐其勢開府儀同三司。己卯,詔翰林國史院纂修累朝實錄及后妃、功臣列傳。庚辰,罷功德、典瑞、營繕、集慶、翊正、群玉、繕工、金玉珠翠諸提舉司。以撒的為御史大夫。禁犯御名。是月,河南旱,賑恤芍陂屯軍糧兩月。



 五月壬午朔,皇太后以膺受寶、冊,恭謝太廟。丙戌,占城國遣其臣剌忒納瓦兒撒來獻方物,且言交趾遏其貢道,詔遣使宣諭交趾。戊子,車駕時巡上都。遣使者詣曲阜孔子廟致祭。加伯撒裏金紫光祿大夫。壬辰,命嚴謚法,以絕冒濫。京畿民饑,詔有司議賑恤。癸卯,太陰犯壘壁陣。甲辰,伯顏請以右丞相讓唐其勢,詔不允,命唐其勢為左丞相。是月,永新州饑,賑之。六月辛酉,有司言甘肅撒里畏兀產金銀,請遣官稅之。壬戌,太陰犯心宿。癸酉,禁服色不得僭上。乙亥,罷江淮財賦總管府所管杭州、平江、集慶三處提舉司,以其事歸有司。詔湖南宣慰使司兼都元帥府,總領所轄諸路鎮守軍馬。庚辰,伯顏奏唐其勢及其弟塔剌海謀逆,誅之。執皇后伯牙吾氏幽於別所。大霖雨。



 秋七月辛巳朔,以馬札兒臺、阿察赤並為御史大夫。壬午,伯顏殺皇后伯牙吾氏於開平民舍。丁亥,享於太廟。壬辰,加馬札兒臺銀青榮祿大夫、開府儀同三司,領承徽寺。乙未,太陰犯壘壁陣。壬寅,專命伯顏為中書右丞相,罷左丞相不置。癸卯,立脫脫禾孫於察罕腦兒之地。乙巳,罷燕鐵木兒、唐其勢舉用之人。戊申,誅答里及剌剌等於市,詔曰:「曩者文宗皇帝以燕鐵木兒嘗有勞伐,父子兄弟顯列朝廷,而輒造事釁,出朕遠方。文皇尋悟其妄,有旨傳次於予。燕鐵木兒貪利幼弱,復立朕弟懿璘質班,不幸崩殂。今丞相伯顏,追奉遺詔,迎朕於南,既至大都,燕鐵木兒猶懷兩端,遷延數月,天隕厥躬。伯顏等同時翊戴,乃正宸極。後撒敦、答里、唐其勢相襲用事,交通宗王晃火帖木兒,圖危社稷,阿察赤亦嘗與謀,賴伯顏等以次掩捕,明正其罪。元兇構難,貽我皇太后震驚,朕用兢惕。永惟皇太后後其所生之子,一以至公為心,親挈大寶,畀予兄弟,跡其定策兩朝,功德隆盛,近古罕比。雖嘗奉上尊號,揆之朕心,猶為未盡,已命大臣特議加禮。伯顏為武宗捍禦北邊,翼戴文皇,茲又克清大憝,明飭國憲,爰賜答剌罕之號,至於子孫,世世永賴。可赦天下。」是月西和州、徽州雨雹,民饑,發米賑貸之。



 八月辛亥朔,熒惑犯氐宿。戊午,祭社稷。癸亥,詔以岐陽王完者帖木兒、知樞密院事帖木兒不花並為御史大夫。甲子,加完者帖木兒太傅。戊寅,道州、永興水災,發米五千石及義倉糧賑之。己卯,議尊皇太后為太皇太后,許有壬諫以為非禮,不從。是月,廣西徭反,命湖廣行省右丞完者討之。沅州等處民饑,賑米二萬七千七百石。九月庚辰朔,車駕駐扼胡嶺。丙戌,赦。丁亥,封知樞密院事闊里吉思為宜國公,太保、中書平章政事定住為宣德王。夜,太陰犯鬥宿。庚寅,太陰犯壘壁陣。庚子,加中書平章政事徹里帖木兒銀青榮祿大夫。命有司造太皇太后玉冊、玉寶。御史臺臣言:「國朝初用宦官,不過數人,今內府執事不下千餘。乞依舊制,裁減冗濫,廣仁愛之心,省糜費之患。」從之。丙午,詔以烏撒、烏蒙之地隸四川行省。是月,耒陽、常寧、道州民饑,以米萬六千石並常平米賑糶之。車駕還自上都。以京畿鹽換羊二萬口。



 冬十月甲寅,熒惑犯南斗。丙辰,以大司農塔失海牙為太尉,置僚屬,商議中書省事。丁巳,以塔失帖木兒為太禧院使,議軍國重事;流晃火帖木兒、答里、唐其勢子孫於邊地。詔海道都漕運萬戶府船戶與民一體充役。壬戌,加御史大夫帖木兒不花銀青榮祿大夫。癸亥,流御史大夫完者帖木兒於廣海安置。完者帖木兒乃賊臣也先鐵木兒骨肉之親,監察御史以為言,故斥之。選省、院、臺、宗正府通練刑獄之官,分行各道,與廉訪司審決天下囚。甲子,太陰犯昴宿。丁卯,太陰犯鬥宿。戊辰,太白晝見。以宗王亦思乾兒弟撒昔襲其兄封。監察御史呂思誠等十九人劾奏徹里帖木兒之罪,不聽,皆辭去,惟陳允文以不署名留。辛未,太皇太后玉冊、玉寶成,遣官告祭於太廟。是月,以伯顏獨任中書右丞相詔天下。



 十一月庚辰,敕以所在儒學貢士莊田租給宿衛衣糧,詔罷科舉。甲申,太白經天。乙酉,伯顏請內外官悉循資銓注,今後無得保舉,澀滯選法,從之。癸巳,命知樞密院事馬札兒臺領武備寺。丙戌,太白經天。己丑,辰星犯房宿。甲午,以燕鐵木兒、唐其勢、答里所奪高麗田宅,還其王阿剌忒納失里。丁酉,以戶部尚書徐奭、吏部尚書定住參議中書省事。戊戌,召前知樞密院事福丁、失剌不花、撒兒的哥還京師。初,二人以帝未立,謀誅燕鐵木兒,為所誣貶,故正之。己亥,太陰犯太微垣。庚子,太陰犯左執法。辛丑,下詔改元,詔曰:



 朕祗紹天明,入纂丕緒,於今三年,夙夜寅畏,罔敢怠荒。茲者年穀順成,海宇清謐,朕方增修厥德,日以敬天恤民為務,屬太史上言,星文示儆。將朕德菲薄,有所未逮歟?天心仁愛,俾予以治,有所告戒歟?弭災有道,善政為先。更號紀年,實惟舊典。惟世祖皇帝在位長久,天人協和,諸福咸至,祖述之志,良切朕懷。今特改元統三年仍為至元元年。遹遵成憲,誕布寬條,庶格禎祥,永綏景祚。赦天下。



 立常平倉。丁未,賜知樞密院事徹里帖木兒三珠虎符。



 十二月己酉朔,荊門州獻紫芝。以廩給司屬通政院。加知樞密院事闊里吉思銀青榮祿大夫,兼左翊蒙古侍衛親軍都指揮使。壬子,太陰犯壘壁陣。乙卯,命雲南行省造軍士錢糧新舊之籍。丙辰,制省諸王、公主、駙馬飲膳之費。詔征高麗王阿剌忒納失里入朝。丁巳,詔伯顏領宮相府。戊午,日赤如赭。辛酉,太白犯壘壁陣。壬戌,撥廬州、饒州牧地一百頃,賜宣讓王帖木兒不花。命四川、雲南、江西行省保選蠻夷官以俟銓注。乙丑,奉玉冊、玉寶,上太皇太后尊號曰贊天開聖徽懿宣昭貞文慈佑儲善衍慶福元太皇太后,詔曰:「欽惟太皇太后,承九廟之托,啟兩朝之業,親以大寶,付之眇躬。尚依擁佑之慈,恪遵仁讓之訓,爰極尊崇之典,以昭報本之忱。庸上徽稱,宣告中外。」命宣政使末吉以司徒就第。太白犯軒轅夫人星。丙寅,太白經天,丁卯,復如之。夜,太陰犯右執法。庚午,太白經天,壬申,復如之。癸酉,歲星晝見。乙亥,太白、歲星皆晝見。丙子,安慶、蘄、黃地震。丁丑,西番賊起,遣兵擊之。戊寅,蒙古國子監成。是日,太白經天,歲星晝見。是月,寶慶路饑,賑糶米三千石。



 閏月乙酉,詔:「四川鹽運司於鹽井仍舊造鹽,餘井聽民煮造,收其課十之三。」熒惑犯壘壁陣。丁亥,日赤如赭,凡三日。戊子,復以宗正府為大宗正府。壬辰,詔宗室脫脫木兒襲封荊王,賜金印,命掌忙來諸軍,設立王府官屬。丁酉,御史大夫撒的加銀青榮祿大夫,領奎章閣,知經筵事。戊戌,御史臺臣復劾奏中書平章政事徹里帖木兒罪,罷之。庚子,太陰犯心星。壬寅,流徹里帖木兒於南安。太陰犯箕宿。癸卯,太陰犯南斗。丙午,詔平章政事塔失海牙領都水、度支二監。是年,江西大水,民饑,賑糶米七萬七千石。賜天下田租之半。凡有妻室之僧,令還俗為民,既而復聽為僧。移犍為縣還舊治。



\end{pinyinscope}