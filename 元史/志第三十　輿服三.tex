\article{志第三十 輿服三}

\begin{pinyinscope}

 ○儀衛



 殿上執事



 挈壺郎二人,掌直漏刻,冠學士帽,服紫羅窄袖衫,塗金束帶,烏靴。漏刻直御榻南。



 司香二人,掌侍香,以主服御者國語曰速古兒赤。攝之,冠服同挈壺。香案二,在漏刻東西稍南。司香侍案側,東西相向立。



 酒人,凡六十人,主酒國語曰答剌赤。二十人,主湩國語曰郃剌赤。二十人,主膳國語曰博兒赤。二十人。冠唐帽,服同司香。酒海直漏南,酒人北面立酒海南。



 護尉四十人,以質子在宿衛者攝之。質子,國語曰睹魯花。冠交角襆頭,紫梅花羅窄袖衫,塗金束帶,白錦汗胯,帶弓矢,佩刀,執骨朵,分立東西宇下。



 警蹕三人,以控鶴衛士為之。冠交角襆頭,服紫羅窄袖衫,塗金束帶,烏靴,捧立於露階。每乘輿出入,則鳴其鞭以警眾。



 殿下執事



 司香二人,亦以主服御者攝之,冠服同殿上司香。香案直露階南,司香東西相向立。



 護尉,凡四十人,以戶郎國語曰玉典赤。二十人、質子二十人攝之,服同宇下護尉,夾立階NX。



 右階之下,伍長凡六人,都點檢一人,右點檢一人,左點檢一人。凡宿衛之人及諸門者、戶者,皆屬焉。如怯薛歹、八剌哈赤、玉典赤之類是也。殿內將軍一人,凡殿內佩弓矢者、佩刀者、諸司御者皆屬焉。如火兒赤、溫都赤之類是也。



 殿外將軍一人,宇下護尉屬焉。宿直將軍一人,黃麾立仗及殿下護尉屬焉。右無常官,凡朝會,則以近侍重臣攝之。服白帽,白衲襖,行縢,履襪,或服其品之公服,恭事則侍立。舍人授以骨朵而易笏,都點檢以玉,右點檢以瑪瑙,左點檢以水精,殿內將軍以瑪瑙,殿外將軍以水精,宿直將軍以金。



 左階之下,伍長凡三人,殿內將軍一人,殿外將軍一人,宿直將軍一人,冠服同右,恭事則侍立。舍人授以骨朵而易笏,殿內將軍以瑪瑙,殿外將軍以水精,宿直將軍以金。



 司辰郎二人,一人立左樓上,服視六品,候時,北面而雞唱;一人立樓下,服視八品,候時,捧牙牌趨丹墀跪報。露階之下,左黃麾仗內,設表案一,禮物案一,輿士凡八人,每案四人。前二人冠縷金額交角襆頭,緋錦寶相花窄袖襖,塗金束帶,行縢,鞋襪。後二人冠服同前,惟襖色青。



 圉人十人,國語曰阿塔赤。冠唐巾,紫羅窄袖衫,青錦緣白錦汗胯,銅束帶,烏靴,馭立仗馬十,覆以青錦緣緋錦鞍復,分左右,立黃麾仗南。



 侍儀使二人,引進使一人,通班舍人一人,尚引舍人一人,閱仗舍人一人,奉引舍人一人,先輿舍人一人。糾儀官凡四人,尚書一人,侍郎一人,監察御史二人。知班三人,視班內失儀者,白糾儀官而行罰焉。皆東向,立右仗之東,以北為上。



 侍儀使二人,引進使一人,承奉班都知一人,宣表目舍人一人,宣表修撰一人,宣禮物舍人一人,奉表舍人一人,奉幣舍人一人,尚引舍人一人,閱仗舍人一人,奉引舍人一人,先輿舍人一人。押禮物官凡二人,工部侍郎一人,禮部侍郎一人。糾儀官凡四人,尚書一人,侍郎一人,監察御史二人。知班三人,視班內如左右輦路。宣贊舍人一人,通贊舍人一人,戶郎二人,承傳贊席前,皆西向,立左仗之西,以北為上。凡侍儀使、引進使、尚書、侍郎、御史,各服其本品之服。承奉班都知、舍人,借四品服。知班,冠展角襆頭,服紫羅窄袖衫,塗金束帶,烏靴。



 護尉三十人,以質子在宿衛者攝之,立大明門闑外,冠服同宇下護尉。



 承傳二人,控鶴衛士為之,立大明門楹間,以承傳於外仗。冠服同警蹕,執金柄小骨朵。



 殿下黃麾仗



 黃麾仗凡四百四十有八人,分布於丹墀左右,各五行。



 右前列,執大蓋二人,執華蓋二人,執紫方蓋二人,執紅方蓋二人,執曲蓋二人,冠展角襆頭,服緋絁生色寶相花袍,勒帛,烏靴。



 次二列,執硃團扇八人,執大雉扇八人,執中雉扇八人,執小雉扇八人,執硃團扇八人,冠武弁,服同前執蓋者。



 次三列,執黃麾幡十人,武弁,青絁生色寶相花袍,青勒帛,烏靴。執絳引幡十人,武弁,緋絁生色寶相花袍,緋勒帛,烏靴。執信幡十人,冠服同上,其色黃。執傳教幡十人,冠服同上,其色白。執告止幡十人,冠服同上,其色紫。



 次四列以下,執葆蓋四十人,武弁,服緋絁生色寶相花袍,勒帛,烏靴。執儀鍠斧四十人,冠服同上,其色黃。執小戟蛟龍掌四十人,冠服同上,其色青。左列亦如之。皆以北為上。押仗四人,行視仗內而檢校之,冠服同警蹕者。



 殿下旗仗



 旗仗執護引屏,凡五百二十有八人,分左右以列。



 左前列,建天下太平旗第一,牙門旗第二,每旗執者一人,護者四人,皆五色緋巾,五色絁生色寶相花袍,勒帛,雲頭靴,執人佩劍,護人加弓矢;後屏五人,執槊,硃兜鍪,硃甲,雲頭靴。



 左二列,日旗第三,龍君旗第四,每旗執者一人,護者四人,後屏五人,巾服執佩同前列。



 右前列,建皇帝萬歲旗第一,牙門旗第二,每旗執者一人,護者四人,後屏五人,巾服執佩同左前列。



 右二列,月旗第三,虎君旗第四,每旗執者一人,護者四人,後屏五人,巾服執佩同前列。



 左次三列,青龍旗第五,執者一人,黃絁巾,黃絁生色寶相花袍,勒帛,花靴,佩劍;護者二人,硃白二色絁巾,二色絁生色寶相花袍,勒帛,花靴,佩劍,加弓矢。天王旗第六,執者一人,巾服同上;護者二人,青白二色絁巾,二色生色寶相花袍,勒帛,花靴,佩劍,加弓矢。後屏五人,執槊,硃兜鍪,硃甲,雲頭靴。風伯旗第七,執者一人,護者二人,後屏五人,巾服佩執同天王旗。雨師旗第八,執者一人,護者二人,後屏五人,巾服佩執同青龍旗。雷公旗第九,執者一人,巾服佩同上;護者二人,青紫二色絁巾,二色絁生色寶相花袍,勒帛,花靴,佩劍,加弓矢;後屏五人,執槊,白兜鍪,白甲,雲頭靴。電母旗第十,執者一人,護者二人,後屏五人,巾服執佩同風伯旗。吏兵旗第十一,執者一人,護者二人,巾服佩同雷公旗;後屏五人,執槊,黃兜鍪,黃甲,雲頭靴。



 右次三列,白虎旗第五,執者一人,黃絁巾,黃絁生色寶相花袍,勒帛,花靴,佩劍;護者二人,青硃二色絁巾,二色絁生色寶相花袍,勒帛,花靴,佩劍,加弓矢。後屏五人,執槊,硃兜鍪,硃甲,雲頭靴。江瀆旗第七,執者一人,護者二人,後屏五人,巾服執佩同天王旗。河瀆旗第八,執者一人,巾服佩同上;護者二人,青紫二色絁巾,二色絁生色寶相花袍,勒帛,花靴,佩劍,加弓矢;後屏五人,執槊,黃兜鍪,黃甲,雲頭靴。淮瀆旗第九,執者一人,巾服佩同上;護者二人,青硃二色絁巾,二色絁生色寶相花袍,勒帛,花靴,佩劍,加弓矢;後屏五人,巾服執佩同白虎旗。濟瀆旗第十,執者一人,巾服佩同上;護者二人,硃白二色絁巾,二色絁生色寶相花袍,勒帛,花靴,佩劍,加弓矢;後屏五人,執槊,青兜鍪,青甲,雲頭靴。力士旗第十一,執者一人,護者二人,後屏五人,巾服佩執同河瀆旗。二十二旗內,拱衛直指揮使二人,分左右立,服本品朝服,執玉斧。次臥瓜一列,次立瓜一列,次列絲一列,冠縷金額交角襆頭,緋錦寶相花窄袖襖,塗金荔枝束帶,行縢,履襪。次鐙仗一列,次吾仗一列,次班劍一列,並分左右立,冠縷金額交角襆頭,青錦寶相花窄袖襖,塗金荔枝束帶,行縢,履襪。



 左次四列,硃雀旗第十二,執者一人,黃絁巾,黃絁生色寶相花袍,勒帛,花靴,佩劍,護者二人,青白二色絁巾,二色絁生色寶相花袍,勒帛,花靴,佩劍,加弓矢;後屏五人,執槊,硃兜鍪,硃甲,雲頭靴。木星旗第十三,執者一人,巾服佩同上;護者二人,青硃二色絁巾,二色絁生色寶相花袍,勒帛,花靴,佩劍,加弓矢,後屏五人,執槊,青兜鍪,青甲,雲頭靴。熒惑旗第十四,執者一人,巾服佩同上;護者二人,青紫二色絁巾,二色絁生色寶相花袍,勒帛,花靴,佩劍,加弓矢;後屏五人,巾服執佩同硃雀旗。土星旗第十五,執者一人,護者二人,巾服佩同熒惑旗;後屏五人,執槊,黃兜鍪,黃甲,雲頭靴。太白旗第十六,執者一人,護者二人,巾服佩同木星旗;後屏五人,執槊,白兜鍪,白甲,雲頭靴。水星旗第十七,執者一人,護者二人,巾服佩同太白旗;後屏五人,執槊,紫兜鍪,紫甲,雲頭靴。鸞旗第十八,執者一人,巾服佩同上;護者二人,硃白二色絁巾,二色絁生色寶相花袍,勒帛,花靴,佩劍,加弓矢;後屏五人,巾服執同木星旗。



 右次四列,玄武旗第十二,執者一人,黃絁巾,黃絁生色寶相花袍,勒帛,花靴,佩劍,護者二人,硃白二色絁巾,二色絁生色寶相花袍,勒帛,花靴,佩劍,加弓矢;後屏五人,紫兜鍪,紫甲,雲頭靴,執槊。東嶽旗第十三,執者一人,護者二人,巾服佩同玄武旗;後屏五人,執槊,青兜鍪,青甲,雲頭靴。南嶽旗第十四,執者一人,巾服佩同上;護者二人,青白二色絁巾,二色絁生色寶相花袍,勒帛,花靴,佩劍,加弓矢;後屏五人,執槊,硃兜鍪,硃甲。中嶽旗第十五,執者一人,巾服佩同上;護者二人,紫青二色絁巾,二色絁生色寶相花袍,勒帛,花靴,佩劍,加弓矢;後屏五人,執槊,黃兜鍪,黃甲,雲頭靴。西嶽旗第十六,執者一人,巾服佩同上;護者二人,硃青二色絁巾,二色絁生色寶相花袍,勒帛,花靴,佩劍,加弓矢;後屏五人,執槊,白兜鍪,白甲。北嶽旗第十七,執者一人,護者二人,巾服佩同南嶽旗;後屏五人,巾服執同玄武旗。麟旗第十八,執者一人,護者二人,後屏五人,巾服執佩同西嶽旗。



 左次五列,角宿旗第十九,亢宿旗第二十,氐宿旗第二十一,房宿旗第二十二,心宿旗第二十三,尾宿旗第二十四,箕宿旗第二十五。每旗,執者一人,黃絁巾,黃絁生色寶相花袍,勒帛,花靴,佩劍;護者二人,青硃二色絁巾,二色絁生色寶相花袍,勒帛,花靴,佩劍,加弓矢;後屏五人,青兜鍪,青甲,執槊。



 右次五列,奎宿旗第十九,婁宿旗第二十,胃宿旗第二十一,昴宿旗第二十二,畢宿旗第二十三,觜宿旗第二十四,參宿旗第二十五。每旗,執者一人,黃絁巾,黃絁生色寶相花袍,勒帛,花靴,佩劍;護者二人,青硃二色絁巾,二色絁生色寶相花袍,勒帛,花靴,佩劍,加弓矢;後屏五人,執槊,白兜鍪,白甲。



 左次六列,鬥宿旗第二十六,牛宿旗第二十七,女宿旗第二十八,虛宿旗第二十九,危宿旗第三十,室宿旗第三十一,壁宿旗第三十二。每旗,執者一人,黃絁巾,黃絁生色寶相花袍,勒帛,花靴,佩劍;護者二人,硃白二色絁巾,二色絁生色寶相花袍,勒帛,花靴,佩劍,加弓矢;後屏五人,執槊,紫兜鍪,紫甲。



 右次六列,井宿旗第二十六,鬼宿旗第二十七,柳宿旗第二十八,星宿旗第二十九,張宿旗第三十,翼宿旗第三十一,軫宿旗第三十二。每旗,執者一人,黃絁巾,黃絁生色寶相花袍,勒帛,花靴,佩劍;護者二人,硃白二色絁巾,二色絁生色寶相花袍,勒帛,花靴,佩劍,加弓矢;後屏五人,執槊,硃兜鍪,硃甲。



 宮內導從



 警蹕三人,以控鶴衛士為之,並列而前行,掌鳴其鞭以警眾。服見前。天武二人,執金鉞,分左右行,金兜鍪,金甲,蹙金素汗胯,金束帶,綠雲靴。



 舍人二人,服視四品。



 主服御者凡三十人,速古兒赤也。執骨朵二人,執幢二人,執節二人,皆分左右行。攜金盆一人,由左;負金椅一人,由右。攜金水瓶、鹿盧一人,由左;執巾一人,由右。捧金香球二人,捧金香合二人,皆分左右行。捧金唾壺一人,由左;捧金唾盂一人,由右。執金拂四人,執升龍扇十人,皆分左右行。冠交角襆頭,服紫羅窄袖衫,塗金束帶,烏靴。



 劈正斧官一人,由中道,近侍重臣攝之。侍儀使四人,分左右行。



 佩弓矢十人,國語曰火兒赤。分左右,由外道行,服如主服御者。



 佩寶刀十人,國語曰溫都赤。分左右行,冠鳳翅唐巾,服紫羅辮線襖,金束帶,烏靴。



 中宮導從



 舍人二人,引進使二人,中政院判二人,同僉中政院事二人,僉中政院事二人,中政院副使二人,同知中政院事二人,中政院使二人,皆分左右行,各服其本品公服。內侍二人,分左右行,服視四品。



 押直二人,冠交角襆頭,紫羅窄袖衫,塗金束帶,烏靴。小內侍凡九人,執骨朵二人,執葆蓋四人,皆分左右行;執傘一人,由中道行;攜金盆二人由左,負金椅二人由右。服紫羅團花窄袖衫,冠、帶、靴如押直。



 中政使一人,由中道,捧外辦象牌,服本品朝服。



 宮人,凡二十二人。攜水瓶、金鹿盧一人,由右;執銷金凈巾一人,由左。捧金香球二人,捧金香合二人,分左右。捧金唾壺一人,由左;捧金唾盂一人,由右。執金拂四人,執雉扇十人,各分左右行。冠鳳翅縷金帽,銷金緋羅襖,銷金緋羅結子,銷金緋羅系腰,紫羅衫,五色嵌金黃雲扇,瓘玉束帶。



 進發冊寶



 清道官二人,警蹕二人,並分左右,皆攝官,服本品朝服。



 雲和樂一部,署令二人,分左右。次前行戲竹二,次排簫四,次簫管四,次板二,次歌四,並分左右。前行內琵琶二十,次箏十六,次箜篌十六,次緌十六,次方響八,次頭管二十八,次龍笛二十八,為三十三重。重四人。



 次杖鼓三十,為八重。次板八,為四重。板內大鼓二,工二人,舁八人。本工服並與鹵簿同。法物庫使二人,服本品服。次硃團扇八,為二重。次小雉扇八,次中雉扇八,次大雉扇八,分左右,為十二重。次硃團扇八,為二重。次大傘二,次華蓋二,次紫方傘二,次紅方傘二,次曲蓋二,並分左右。執傘扇所服,並同立仗。



 圍子頭一人,中道。次圍子八人,分左右。服與鹵簿內同。



 安和樂一部,署令二人,服本品服。札鼓六,為二重,前四,後二。次和鼓一,中道。次板二,分左右。次龍笛四,次頭管四,並為二重。次羌管二,次笙二,並分左右。次雲璈一,中道。次緌二,分左右。樂工服與鹵簿內同。



 傘一,中道,椅左,踏右,執人皁巾,大團花緋錦襖,金塗銅束帶,行縢,鞋襪。



 拱衛使一人,服本品服。



 舍人二人,次引寶官二人,並分左右,服四品服。



 香案,中道,輿士控鶴八人,服同立仗內表案輿士。侍香二人,分左右,服四品服。



 寶案,中道,輿士控鶴十有六人,服同香案輿士。方輿官三十人,夾香案寶案,分左右而趨,至殿門,則控鶴退,方輿官舁案以升。唐巾,紫羅窄袖衫,金塗銅束帶,烏靴。



 引冊二人,四品服。



 香案,中道,輿士控鶴八人,服同寶案輿士。侍香二人,分左右,服四品服。



 冊案,中道,輿士控鶴十有六人,服同寶案輿士。方輿官三十人,夾香案冊案,分左右而趨,至殿門,則控鶴退,方輿官舁案以升。巾服與寶案方輿官同。



 葆蓋四十人,次閱仗舍人二人,服四品服。次小戟四十人,次儀鍠四十人,夾雲和樂傘扇,分左右行,服同立仗。



 拱衛使二人,服本品朝服。次班劍十,次吾仗十二,次斧十二,次鐙仗二十,次列絲十,皆分左右。次水瓶左,金盆右。次列絲十,次立瓜十。次金杌左,鞭桶右;蒙鞍左,傘手右。次立瓜十,次臥瓜三十。並夾葆蓋、小戟、儀鍠,分左右行。服並同鹵簿內。



 拱衛外舍人二人,服四品服,引導冊諸官。次從九品以上,次從七品以上,次從五品以上,並本品朝服。



 金吾折沖二人,牙門旗二,每旗引執五人。次青槊四十人,赤槊四十人,黃槊四十人,白槊四十人,紫槊四十人,並兜鍪甲靴,各隨槊之色,行導冊官外。



 冊案後,舍人二人,服四品服。次太尉右,司徒左。次禮儀使二人,分左右。次舉冊官四人右,舉寶官四人左;次讀冊官二人右,讀寶官二人左。次閣門使四人,分左右。並本品服。



 知班六人,分左右,服同立仗,往來視諸官之失儀者而行罰焉。



 冊寶攝官



 上尊號冊寶,凡攝官二百五十有六人,奉冊官四人,奉寶官四人,捧寶官二人,讀冊官二人,讀寶官二人,引冊官五人,引寶官五人,典瑞官三人,糾儀官四人,殿中侍御史二人,監察御史四人,閣門使三人,清道官四人,點試儀衛五人,司香四人,備顧問七人,代禮三十人,拱衛使二人,押仗二人,方輿一百六十人。



 上皇太后冊寶,凡攝官二百五十人,攝太尉一人,攝司徒一人,禮儀使四人,奉冊官二人,奉寶官二人,引冊官二人,引寶官二人,舉冊官二人,舉寶官二人,讀冊官二人,讀寶官二人,捧冊官二人,捧寶官二人,奏中嚴一人,主當內侍十人,閣門使六人,充內臣十三人,糾儀官四人,代禮官四十二人,掌謁四人,司香十二人,折沖都尉二人,拱衛使二人,清道官四人,警蹕官四人,方輿官百二十人。



 太皇太后冊寶,攝官同前。



 授皇后冊寶,凡攝官百八十人,攝太尉一人,攝司徒一人,主節官二人,禮儀使四人,奉冊官二人,奉寶官二人,引冊官二人,引寶官二人,舉冊官二人,舉寶官二人,讀冊官二人,讀寶官二人,內臣職掌十人,宣徽使二人,閤門使四人,代禮官三十七人,侍香二人,清道官四人,折沖都尉二人,警蹕官四人,中宮內臣九人,糾儀官四人,接冊內臣二人,接寶內臣二人,方輿官七十四人。



 授皇太子冊,凡攝官四十有九人,攝太尉一人,奉冊官二人,持節官一人,捧冊官二人,讀冊官二人,引冊官二人,攝禮儀使二人,主當內侍六人,副持節官五人,侍從官十一人,代禮官十六人。



 班序



 先期,侍儀使糾庀陳設。



 殿內兩楹北,香案二。



 殿門內,殿內將軍板位二。其外,殿外將軍板位二。宇下,斜界護尉板位二。軒溜前斜外出畫白蓮六,右點檢板位三,左宣徽板位三。蓮南一步,橫列鳴鞭板位三。左右階南兩隅,天武板位二。宇下左右第一第三重,敘界導從板位二。



 殿東門兩磌斜界出導從二道三層,各圈十五,先扇錡各五,寶蓋錡各二。



 殿東階下各圈十,直至東門階下,為回倒導從位。



 正階下二十四甓,香案一。護尉席內各所迤內第四螭首取直,邊北,左右護尉第五席相向布席,北二席宿直。次殿中,次典瑞,次起居,每席函丈五尺。設殿前板位八,各以左右NX道內邊丹墀迤內第五甓縱直,北空路五丈五尺,東西走路各違四丈九尺,中布席四十,席函九尺,設護尉板位二。



 輦路東西各五道,袤二丈一仞五寸。南北兩道,廣丈有奇。北至道當中,第一北三南一,自兩端各函六丈。第二北起十一,各函丈咫,南起九,各函丈三尺。第三北起十三,各函丈五尺,南起十二,各函丈五寸。第四北起十六,各函丈二尺,南起十四,各函九尺。第五北起,同上南起,各函八尺,北頭曲尺路內,各函九尺,設黃麾仗錡二百二十。仗南畫闌約丈許,左右同,中央置席,設尚廄板位二。仗內丹墀橫界一十八道,道函五尺,縱引橫引三丈,中設九品板位一十八。尚廄南左右縱畫各一十八道,道函仞,左右向,設起居旁折板位三十六,以內為上。



 大明門中兩楹外,斜界二道,護尉板位二,外設管旗板位二。門下左右闕邊各六丈,南北各畫一道,廣一引七丈一仞六寸,空各二丈一仞,內橫二引二丈五寸,空各三丈五尺。每錡後丈五尺屏風渠一道,長五尺,坐各違四壁丈五尺,設牙旗錡七十四。闕下兩觀內各六丈,縱各界一十八道,道違仞,左右設外序班板位三十六。自序班北入闕左右門邊兩外仗往北折,西至月華門,東至日精門,道中央入至起居旁折界一道導引。



\end{pinyinscope}