\article{列傳第十二}

\begin{pinyinscope}

 ○賽典赤贍思丁子納速剌丁忽辛附



 賽典赤贍思丁,一名烏馬兒,回回人,別庵伯爾之裔。其國言賽典赤,猶華言貴族也。太祖西征,贍思丁率千騎以文豹白鶻迎降,命入宿衛,從征伐,以賽典赤呼之而不名。太宗即位,授豐凈云內三州都達魯花赤;改太原、平陽二路達魯花赤;入為燕京斷事官。憲宗即位,命同塔剌渾行六部事,遷燕京路總管,多惠政,擢採訪使。帝伐蜀,賽典赤主饋餉,供億未嘗闕乏。



 世祖即位,立十路宣撫司,擢燕京宣撫使。中統二年,拜中書平章政事,皆降制獎諭。至元元年,置陜西五路西蜀四川行中書省,出為平章政事。蒞官三年,增戶九千五百六十五、軍一萬二千二百五十五、鈔六千二百二十五錠、屯田糧九萬七千二十一石,撙節和買鈔三百三十一錠。中書以聞,詔賞銀五千兩,仍命陜西五路四川行院大小官屬並聽節制。七年,分鎮四川,宋將昝萬壽擁強兵守嘉定,與賽典赤軍對壘,一以誠意待之,不為侵掠,萬壽心服。未幾,賽典赤召還,萬壽請置酒為好,左右皆難之,賽典赤竟往不疑。酒至,左右復言未可飲,賽典赤笑曰:「若等何見之小耶。昝將軍能毒我,其能盡毒我朝之人乎!」萬壽嘆服。八年,有旨:大軍見圍襄陽,各道宜進兵以牽制之。於是賽典赤偕鄭鼎率兵水陸並進,至嘉定,獲宋將二人,順流縱筏,斷其浮橋,獲戰艦二十八艘。尋命行省事於興元,專給糧餉。



 十一年,帝謂賽典赤曰:「雲南朕嘗親臨,比因委任失宜,使遠人不安,欲選謹厚者撫治之,無如卿者。」賽典赤拜受命,退朝,即訪求知雲南地理者,畫其山川城郭、驛舍軍屯、夷險遠近為圖以進,帝大悅,遂拜平章政事,行省雲南,賜鈔五十萬緡、金寶無算。時宗王脫忽魯方鎮雲南,惑於左右之言,以賽典赤至,必奪其權,具甲兵以為備。賽典赤聞之,乃遣其子納速剌丁先至王所,請曰:「天子以雲南守者非人,致諸國背叛,故命臣來安集之,且戒以至境即加撫循,今未敢專,願王遣一人來共議。」王聞,遽罵其下曰:「吾幾為汝輩所誤!」明日,遣親臣撒滿、位哈乃等至,賽典赤問以何禮見,對曰:「吾等與納速剌丁偕來,視猶兄弟也,請用子禮見。」皆以名馬為贄,拜跪甚恭,觀者大駭。乃設宴陳所賜金寶飲器,酒罷,盡以與之,二人大喜過望。明日來謝,語之曰:「二君雖為宗王親臣,未有名爵,不可以議國事,欲各授君行省斷事官,以未見王,未敢擅授。」令一人還,先稟王,王大悅。由是政令一聽賽典赤所為。



 十二年,奏:「雲南諸夷未附者尚多,今擬宣慰司兼行元帥府事,並聽行省節制。」又奏:「哈剌章、雲南壤地均也,而州縣皆以萬戶、千戶主之,宜改置令長。」並從之。十三年,以所改雲南郡縣上聞。雲南俗無禮儀,男女往往自相配偶,親死則火之,不為喪祭。無秔稻桑麻,子弟不知讀書。賽典赤教之拜跪之節,婚姻行媒,死者為之棺郭奠祭,教民播種,為陂池以備水旱,創建孔子廟明倫堂,購經史,授學田,由是文風稍興。雲南民以貝代錢,是時初行鈔法,民不便之,賽典赤為聞於朝,許仍其俗。又患山路險遠,盜賊出沒,為行者病,相地置鎮,每鎮設土酋吏一人、百夫長一人,往來者或值劫掠,則罪及之。



 有土吏數輩,怨賽典赤不已,用至京師誣其專僭數事。帝顧侍臣曰:「賽典赤憂國愛民,朕洞知之,此輩何敢誣告!」即命械送賽典赤處治之。既至,脫其械,且諭之曰:「若曹不知上以便宜命我,故訴我專僭,我今不汝罪,且命汝以官,能竭忠自贖乎?」皆叩頭拜謝曰:「某有死罪,平章既生之而又官之,誓以死報。」交趾叛服不常,湖廣省發兵屢征不利,賽典赤遣人諭以逆順禍福,且約為兄弟。交趾王大喜,親至雲南,賽典赤郊迎,待以賓禮,遂乞永為籓臣。蘿盤甸叛,往征之,有憂色,從者問故,賽典赤曰:「吾非憂出征也,憂汝曹冒鋒鏑,不幸以無辜而死;又憂汝曹劫虜平民,使不聊生,及民叛,則又從而征之耳。」師次蘿盤城,三日不降,諸將請攻之,賽典赤不可,遣使以理諭之。蘿盤主曰:「謹奉命。」越三日又不降,諸將奮勇請進兵,賽典赤又不可。俄而將卒有乘城進攻者,賽典赤大怒,遽鳴金止之,召萬戶叱責之曰:「天子命我安撫雲南,未嘗命以殺戮也。無主將命而擅攻,於軍法當誅。」命左右縛之。諸將叩首,請俟城下之日從事。蘿盤主聞之曰:「平章寬仁如此,吾拒命不祥。」乃舉國出降。將卒亦釋不誅。由是西南諸夷翕然款附。夷酋每來見,例有所獻納,賽典赤悉分賜從官,或以給貧民,秋毫無所私;為酒食勞酋長,制衣冠襪履,易其卉服草履。酋皆感悅。



 賽典赤居雲南六年,至元十六年卒,年六十九,百姓巷哭,葬鄯闡北門。交趾王遣使者十二人,齊絰為文致祭,其辭有「生我育我,慈父慈母」之語,使者號泣震野。帝思賽典赤之功,詔雲南省臣盡守賽典赤成規,不得輒改。大德元年,贈守仁佐運安遠濟美功臣、太師、開府儀同三司、上柱國、咸陽王,謚忠惠。



 子五人:長納速剌丁;次哈散,廣東道宣慰使都元帥;次忽辛;次苫速丁兀默里,建昌路總管;次馬速忽,雲南諸路行中書省平章政事。



 納速剌丁,累官中奉大夫、雲南諸路宣慰使都元帥。至元十六年,遷帥大理,以軍抵金齒、蒲、驃、曲蠟、緬國,招安夷寨三百,籍戶十二萬二百,定租賦,置郵傳,立衛兵,歸以馴象十二入貢,有旨賞金五十兩、衣二襲,麾下士賞銀有差。會其父贍思丁歿,雲南省臣於諸夷失撫綏之方,世祖憂之,近臣以納速剌丁為言。十七年,授資德大夫、雲南行中書省左丞,尋升右丞。建言三事:其一謂雲南省規措所造金簿貿易病民,宜罷;其一謂雲南有省,有宣慰司,又有都元帥府,近宣慰司已奏罷,而元帥府尚存,臣謂行省既兼領軍民,則元帥府亦在所當罷;其一謂雲南官員子弟入質,臣謂達官子弟當遣,餘宜罷。奏可。二十一年,進榮祿大夫、平章政事。奏減合剌章冗官,歲省俸金九百餘兩;屯田課程專人掌之,歲得五千兩。二十三年,以合剌章蒙古軍千人,從皇太子脫歡征交趾,論功賞銀二千兩。二十八年,進拜陜西行省平章政事。二十九年,以疾卒。贈推誠佐理協德功臣、太師、開府儀同三司、上柱國、中書左丞相,封延安王。



 子十二人:伯顏,中書平章政事;烏馬兒,江浙行省平章政事;扎法兒,荊湖宣慰使;忽先,雲南行省平章政事;沙的,雲南行省左丞;阿容,太常禮儀院使;伯顏察兒,中書平章政事,佩金虎符,贈太師、開府儀同三司、上柱國、中書左丞相、奉元王,謚忠憲。



 忽辛,至元初以世臣子備宿衛,世祖善其應對。至元十四年,授兵部郎中。明年,出為河南等路宣慰司同知。河南多強盜,往往群聚山林,劫殺行路,官軍收捕失利,忽辛以招安自任,遣土豪持檄諭之。未幾,賊二人來自歸,忽辛賜之冠巾,且諭之曰:「汝昔為賊,今既自歸,即良民矣。」俾侍左右,出入房闥無間,悉放還,令遍諭其黨。數日後,招集其為首者十輩來,身長各七尺餘,羅拜庭下,顧視異常,眾悉驚怖失措。忽辛命吏籍其姓名為民,俾隨侍左右,夜則令臥戶外,時呼而飲食之,各得其歡心。群盜聞之,相繼款附。



 二十一年,授雲南諸路轉運使。明年,轉陜西道。又明年,授燕南河北道宣慰司同知,尋除南京總管。三十年,授兩浙鹽運使。大德九年,進江東道宣慰使,改陜西行臺御史中丞,再改雲南行省右丞。既至,條具諸不便事言於宗王,請更張之,王不可,忽辛與左丞劉正馳還京師,有旨令宗王協力施行。由是一切病民之政,悉革而新之。豪民規避徭役,往往投充王府宿衛,有司不勝供給,忽辛按朝廷元額所無者,悉籍為民,去其宿衛三分之二。馬龍州酋謀叛,陰與外賊通,持所受宣敕納賊以示信,事覺,宗王為左右所蔽,將釋不問,忽辛與劉正反覆研鞫,反狀盡得,竟斬之。軍糧支給,地理遠近不同,吏夤緣為奸,忽辛籍軍戶姓名及倉廩處所,為更番支給,吏奸始除。先是,贍思丁為雲南平章時,建孔子廟為學校,撥田五頃,以供祭祀教養。贍思丁卒,田為大德寺所有,忽辛按廟學舊籍奪歸之。乃復下諸郡邑遍立廟學,選文學之士為之教官,文風大興。王府畜馬繁多,悉縱之郊,敗民禾稼,而牧人又在民家宿食,室無寧居。忽辛度地置草場,構屋數十間,使為牧所,民得以安。



 廣南酋沙奴素強悍,宋時嘗賜以金印,雲南諸部悉平,獨此梗化。忽辛遣使誘致,待之以禮,留數月不遣,酋請還,忽辛曰:「汝欲還,可納印來。」酋不得已,齎印以納,忽辛置酒宴勞,諷令偕印入覲,帝大悅。大德五年,緬國主負固不臣,忽辛遣人諭之曰:「我老賽典赤平章子也,惟先訓是遵,凡官府於汝國所不便事,當一切為汝更之。」緬國主聞之,遂與使者偕來,獻白象一,且曰:「此象古來所未有,今聖德所致,敢效方物。」既入,帝賜緬國主以世子之號。烏蠻等租賦,歲發軍征索乃集,忽辛以利害榜諭諸蠻,不遣一卒,而租賦咸足。俄有為飛語及符讖以惑宗王者,忽辛引劉正密為奏馳報,朝廷遣使臨問,凡造言之徒悉誅之,忽辛偕使者還覲。



 大德八年,出為四川行省左丞,改江浙行省。至大元年,拜榮祿大夫、江西行省平章政事。明年,以母老謝職歸養。又明年正月卒。天歷元年,贈守德宣惠敏政功臣、上柱國、雍國公,謚忠簡。



 子二人:伯杭,中慶路達魯花赤;曲列,湖南道宣慰使。



 ○布魯海牙



 布魯海牙,畏吾人也。祖牙兒八海牙,父吉臺海牙,俱以功為其國世臣。布魯海牙幼孤,依舅氏家就學,未幾,即善其國書,尤精騎射。年十八,隨其主內附,充宿衛。太祖西征,布魯海牙扈從,不避勞苦,帝嘉其勤,賜以羊馬氈帳,又以居里可汗女石抹氏配之。太祖崩,諸王來會,選使燕京總理財幣。使還,莊聖太后聞其廉謹,以名求之於太宗,凡中宮軍民匠戶之在燕京、中山者,悉命統之,又賜以中山店舍園田、民戶二十,授真定路達魯花赤。



 辛卯,拜燕南諸路廉訪使,佩金虎符,賜民戶十。未幾,授斷事官,使職如故。時斷事官得專生殺,多倚勢作威,而布魯海牙小心謹密,慎於用刑。有民誤毆人死,吏論以重法,其子號泣請代死,布魯海牙戒吏,使擒於市,懼則殺之。既而不懼,乃曰:「誤毆人死,情有可宥,子而能孝,義無可誅。」遂並釋之,使出銀以資葬埋,且呼死者家諭之,其人悅從。是時法制未定,奴有罪者,主得專殺,布魯海牙知其非法而不能救,嘗出金贖死者數十人。征討之際,隸軍籍者,憚於行役,往往募人代之,又軍中多逃歸者,朝廷下制:募代者杖百,逃歸者死。命布魯海牙與斷事官卜只兒按順天等路,及至州縣,得募人代者萬一千戶、逃者十二人。然募者聞命將下,已潛遣家人易代募者。布魯海牙聞之,嘆曰:「募者已懼罪往易,逃者因單弱思歸,情皆可矜,吾可不伸理耶?」遂奏其狀,皆得經減。有丁多產富而家人不往,及未至役所而即逃者,則曰:「此而不殺,何以戒後!」有竊妓逃者,吏論當死,布魯海牙曰:「敗亂綱常,罪固宜死;此妓也,豈可例論!」命杖之。其執法平允類如此。



 世祖即位,擇信臣宣撫十道,命布魯海牙使真定。真定富民出錢貸人者,不逾時倍取其息,布魯海牙正其罪,使償者息如本而止,後定為令。中統鈔法行,以金銀為本,本至,乃降新鈔。時莊聖太后已命取真定金銀,由是真定無本,鈔不可得。布魯海牙遣幕僚邢澤往謂平章王文統曰:「昔奉太后旨,金銀悉送至上京。真定南北要沖之地,居民商賈甚多,今舊鈔既罷,新鈔不降,何以為政?且以金銀為本,豈若以民為本。又太后之取金帛,以賞推戴之功也,其為本不亦大乎!」文統不能奪,立降鈔五千錠,民賴以便。俄遷順德等路宣慰使,佩金虎符。來朝,帝命坐,慰勞之,賜以海東青鶻。至元二年秋卒,年六十九。



 布魯海牙性孝友,造大宅於燕京,自畏吾國迎母來居,事之,得祿不入私室。幼時叔父阿里普海牙欺之,盡有其產,及貴顯,築室宅旁,迎阿里普海牙居之。弟益特思海牙以宿憾為言,常慰諭之,終無間言。帝嘗賜以太府綾絹五千匹,絲絮相等,弟求四之一納其國賦,盡與之,無吝色。初布魯海牙拜廉使,命下之日,子希憲適生,喜曰:「吾聞古以官為姓,天其以廉為吾宗之姓乎!」故子孫皆姓廉氏。後或奏廉氏仕進者多,宜稍汰去,世祖曰:「布魯海牙功多,子孫亦朕所知,非汝當預。」大德初,贈儀同三司、大司徒,追封魏國公,謚孝懿。



 子希閔、希憲、希恕、希尹、希顏、希願、希魯、希貢、希中、希括,孫五十三人,登顯仁者代有之,希憲自有傳。



 ○高智耀子睿附



 高智耀,河西人,世仕夏國。曾祖逸,大都督府尹;祖良惠,右丞相。智耀登本國進士第,夏亡,隱賀蘭山。太宗訪求河西故家子孫之賢者,眾以智耀對,召見將用之,遽辭歸。皇子闊端鎮西涼,儒者皆隸役,智耀謁籓邸,言儒者給復已久,一旦與廝養同役,非便,請除之。皇子從其言。欲奏官之,不就。憲宗即位,智耀入見,言:「儒者所學堯、舜、禹、湯、文、武之道,自古有國家者,用之則治,不用則否,養成其材,將以資其用也。宜蠲免徭役以教育之。」帝問:「儒家何如巫醫?」對曰:「儒以綱常治天下,豈方技所得比。」帝曰:「善。前此未有以是告朕者。」詔復海內儒士徭役,無有所與。世祖在潛邸已聞其賢,及即位,召見,又力言儒術有補治道,反覆辯論,辭累千百。帝異其言,鑄印授之,命凡免役儒戶,皆從之給公文為左驗。時淮、蜀士遭俘虜者,皆沒為奴,智耀奏言:「以儒為驅,古無有也。陛下方以古道為治,宜除之,以風厲天下。」帝然之,即拜翰林學士,命循行郡縣區別之,得數千人。貴臣或言其詭濫,帝詰之,對曰:「士,譬則金也,金色有淺深,謂之非金不可,才藝有淺深,謂之非士亦不可。」帝悅,更寵賚之。智耀又言:「國初庶政草創,綱紀未張,宜仿前代,置御史臺以糾肅官常。」至元五年立御史臺,用其議也。擢西夏中興等路提刑按察使。會西北籓王遣使入朝,謂:「本朝舊俗與漢法異,今留漢地,建都邑城郭,儀文制度,遵用漢法,其故何如?」帝求報聘之使以析其問,智耀入見,請行,帝問所答,畫一敷對,稱旨,即日遣就道。至上京,病卒,帝為之震悼。後贈崇文贊治功臣、金紫光祿大夫、司徒、柱國,追封寧國公,謚文忠。子睿。



 睿,資廩直亮,智耀之北使也,攜之以行。及卒,帝問其子安在,近臣以睿見,時年十六。授符寶郎,出入禁闥,恭謹詳雅。久之,授唐兀衛指揮副使,歷翰林待制、禮部侍郎。除嘉興路總管,境內有宿盜,白晝掠民財,捕者積十數輩莫敢近。睿下令,不旬日,生擒之,一郡以寧。擢江東道提刑按察使,部內草竊陸梁,聲言圍宣城。郡將怯懦,城門不開,睿召責之曰:「寇勢方熾,官先示弱,民何所憑?」即命密治兵衛,而洞開城門,聽民出入貿易自便。既而寇以有備,不敢進,遂討平之。除同僉行樞密院事,遷浙西道肅政廉訪使。鹽官州民,有連結黨與,持郡邑短長,其目曰十老,吏莫敢問,睿悉按以法,闔境快之。拜江南行臺侍御史,進御史中丞,除淮東道肅政廉訪使。盜竊真州庫鈔三萬緡,有司大索,追逮平民數百人,吏因為奸利,睿躬自詳讞而得其情,即縱遣之。未幾,果得真盜。復拜南臺御史中丞,務持大體,有儒者之風焉。延祐元年卒,年六十有六。累贈推忠佐理功臣、太傅、開府儀同三司、上柱國,追封寧國公,謚貞簡。



 子納麟,官至太尉、江南諸道行御史臺大夫。



 ○鐵哥



 鐵哥,姓伽乃氏,迦葉彌兒人。迦葉彌兒者,西域築乾國也。父斡脫赤與叔父那摩俱學浮屠氏。斡脫赤兄弟相謂曰:「世道擾攘,吾國將亡,東北有天子氣,盍往歸之。」乃偕入見,太宗禮遇之。定宗師事那摩,以斡脫赤佩金符,奉使省民瘼。憲宗尊那摩為國師,授玉印,總天下釋教。斡脫赤亦貴用事,領迦葉彌兒萬戶,奏曰:「迦葉彌兒西陲小國,尚未臣服,請往諭之。」詔偕近侍以往。其國主不從,怒而殺之,帝為發兵誅國主,元貞元年封代國公,謚忠遂。



 斡脫赤之歿,鐵哥甫四歲,性穎悟,不為嬉戲。從那摩入見,帝問誰氏子,對曰:「兄斡脫赤子也。」帝方食雞,輟以賜鐵哥。鐵哥捧而不食,帝問之,對曰:「將以遺母。」帝奇之,加賜一雞。世祖即位,幸香山永安寺,見書畏吾字於壁,問誰所書,僧對曰:「國師兄子鐵哥書也。」帝召見,愛其容儀秀麗,語音清亮,命隸丞相孛羅備宿衛。先是,世祖事憲宗甚親愛,後以讒稍疏,國師導世祖宜加敬慎,遂友愛如初。至是,帝將用鐵哥,曰:「吾以酬國師也。」於是鐵哥年十七,詔擇貴家女妻之,辭曰:「臣母漢人,每欲求漢人女為婦,臣不敢傷母心。」乃為娶冉氏女。久之,命掌饔膳湯藥,日益親密。



 至元十六年,鐵哥奏曰:「武臣佩符,古制也。今長民者亦佩符,請省之,以彰武職。」從之。十七年,進正議大夫、尚膳監。帝嘗諭之曰:「朕聞父飲藥,子先嘗之,君飲藥,臣先嘗之。今卿典朕膳,凡飲食湯藥,宜先嘗之。」又曰:「朕以宿衛士隸卿,其可任使者,疏其才能,朕將用之。「詔賜第於大明宮之左。留守段圭言:「逼木局,不便。」帝曰:「俾居近禁闥,以便召使。木局稍隘,又何害焉。」



 高州人言,州境多野獸害稼,願捕以充貢。鐵哥曰:「捕獸充貢,徒濟其私耳,且擾民,不可聽。」從之。十九年,遷同知宣徽院事,領尚膳監。有食尚食餘餅者,帝察知之,怒。鐵哥曰:「失餅之罪在臣,食者何與焉。」內府食用圓米,鐵哥奏曰:「計粳米一石,僅得圓米四斗,請自今非御用,止給常米。」帝皆善之。進中奉大夫、司農寺達魯花赤。從獵百杳兒之地,獵人亦不剌金射兔,誤中名駝,駝死,帝怒,命誅之。鐵哥曰:「殺人償畜,刑太重。」帝驚曰:「誤耶,史官必書。」亟釋之。庾人有盜鑿粳米者,罪當死。鐵哥諫曰:「臣鞫庾人,其母病,盜粳欲食母耳,請貸之。」牧人有盜割駝峰者,將誅之。鐵哥曰:「生割駝峰,誠忍人也。然殺之,恐乖陛下仁恕心。」詔皆免死。二十二年,進正奉大夫,奏:「司農寺宜升為大司農司,秩二品,使天下知朝廷重農之意。」制可。進資善大夫、司農。時司農供膳,有司多擾民,鐵哥奏曰:「屯田則備諸物,立供膳司甚便。」從之。桓州饑民鬻子女以為食,鐵哥奏以官帑贖之。



 二十四年,從征乃顏,至撒兒都之地,叛王塔不臺率兵奄至。鐵哥奏曰:「昔李廣一將耳,尚能以疑退敵,況陛下萬乘之威乎!今彼眾我寡,不得地利,當設疑以退之。」於是帝張曲蓋,據胡床,鐵哥從容進酒。塔不臺按兵覘伺,懼有伏,遂引去。帝以金章宗玉帶賜之。二十九年,進榮祿大夫、中書平章政事。以病足,聽輿轎入殿門。帝嘗憶北征事,不能悉記,鐵哥條舉甚詳,帝悅,以金束帶賜之。初,詔遣宋新附民種蒲萄於野馬川晃火兒不剌之地,既獻其實,鐵哥以北方多寒,奏歲賜衣服,從之。



 成宗即位,以鐵哥先朝舊臣,賜銀一千兩、鈔十萬貫。他日,又賜以瑪瑙碗,謂鐵哥曰:「此器先皇所用,朕今賜卿,以卿久侍先皇故也。」大德元年,加光祿大夫。三年,乞解機務,從之。仍授平章政事、議中書省事。時諸王朝見,未有知典故者,帝曰:「惟鐵哥知之,俾專其事,凡廩餼金帛之數,皆遵世祖制詔,自今懷諸王之禮,悉命鐵哥掌之。」七年,復拜中書平章政事。平灤大水,鐵哥奏曰:「散財聚民,古之道也。今平灤水災,不加賑恤,民不聊生矣!」從之。十年,丁母憂,詔奪情起復。遼王脫脫入朝,從者執兵入大明宮,鐵哥劾止之,王懼謝。從幸縉山,饑民相望,鐵哥輒發廩賑之,既乃陳疏自劾,帝稱善不已。武宗即位,賜金一百兩,加金紫光祿大夫,遙授中書右丞相。有訴寧遠王闊闊出有逆謀者,命誅之。鐵哥知其誣,廷辨之,由是得釋,徙高麗。二年,領度支院。尋賜江州稻田五千畝。仁宗皇慶元年,授開府儀同三司、太傅、錄軍國重事。乃進奏:世祖子惟寧遠王在,宜賜還。從之。二年,奉命詣萬安寺祀世祖,感疾歸,皇太后令內臣問疾,鐵哥附奏曰:「臣死無日,願太后輔陛下布惟新之政,社稷之福也。」是年薨,賜賻禮加厚,敕有司治喪事,贈太師、開府儀同三司、上柱國,追封秦國公,謚忠穆。加贈推誠守正佐理翊戴功臣,封延安王,改謚忠獻。



 子六人:忽察,淮東宣慰使;平安奴,太平路達魯花赤;也識哥,同知山東宣慰司事;虎里臺,同知真定總管府事;亦可麻,同知都護府事;重喜,隆禧院副使。孫八人,伯顏,中書平章政事;餘多居宿衛。



\end{pinyinscope}