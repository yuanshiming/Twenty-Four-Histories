\article{列傳第九}

\begin{pinyinscope}

 ○巴而術阿而忒的斤



 巴而術阿而忒的斤亦都護,亦都護者,高昌國主號也。先世居畏兀兒之地,有和林山,二水出焉,曰禿忽剌,曰薛靈哥。一夕,有神光降於樹,在兩河之間,人即其所而候之,樹乃生癭,若懷妊狀,自是光常見。越九月又十日,而樹癭裂,得嬰兒者五,土人收養之。其最稚者曰不古可罕。既壯,遂能有其民人土田,而為之君長。傳三十餘君,是為玉倫的斤,數與唐人相攻戰,久之議和親,以息民罷兵。於是唐以金蓮公主妻的斤之子葛勵的斤,居和林別力跛力答,言婦所居山也。又有山曰天哥里於答哈,言天靈山也。南有石山曰胡力答哈,言福山也。唐使與相地者至其國,曰:「和林之盛強,以有此山也。盍壞其山,以弱其國?」乃告諸的斤曰:「既為婚姻,將有求於爾,其與之乎?福山之石,於上國無所用,而唐人願見。」的斤遂與之石,大不能動,唐人以烈火焚之,沃以醲醋,其石碎,乃輦而去。國中鳥獸為之悲號。後七日,玉倫的斤卒,災異屢見,民弗安居,傳位者又數亡,乃遷於交州。交州即火州也。統別失八里之地,北至阿術河,南接酒泉,東至兀敦、甲石哈,西臨西蕃。居是者凡百七十餘載,而至巴而術阿而忒的斤,臣於契丹。歲己巳,聞太祖興朔方,遂殺契丹所置監國等官,欲來附。未行,帝遣使使其國。亦都護大喜,即遣使入奏曰:「臣聞皇帝威德,即棄契丹舊好,方將通誠,不自意天使降臨下國,自今而後,願率部眾為臣僕。」是時帝征大陽可汗,射其子脫脫殺之。脫脫之子火都、赤剌溫、馬札兒、禿薛乾四人,以不能歸全尸,遂取其頭涉也兒的石河,將奔亦都護,先遣使往,亦都護殺之。四人者至,與大戰於襜河。亦都護遣其國相來報,帝復遣使還諭亦都護,遂以金寶入貢。辛未,朝帝於怯綠連河,奏曰:「陛下若恩顧臣,使臣得與陛下四子之末,庶幾竭其犬馬之力。」帝感其言,使尚公主也立安敦,且得序於諸子。與者必那演征罕勉力、鎖潭回回諸國,將部曲萬人以先。紀律嚴明,所向克捷。又從帝征你沙卜里,征河西,皆有大功。既卒,而次子玉古倫赤的斤嗣。



 玉古倫赤的斤卒,子馬木剌的斤嗣。將探馬軍萬人,從憲宗伐宋合州,攻釣魚山有功,還火州卒。至元三年,世祖命其子火赤哈兒的斤嗣為亦都護。海都、帖木迭兒之亂,畏兀兒之民遭亂解散,於是有旨命亦都護收而撫之,其民人在宗王近戚之境者,悉遣還其部,畏兀兒之眾復輯。



 十二年,都哇、卜思巴等率兵十二萬圍火州,聲言曰:「阿只吉、奧魯只諸王以三十萬之眾,猶不能抗我而自潰,爾敢以孤城當吾鋒乎?」亦都護曰:「吾聞忠臣不事二主,吾生以此城為家,死以此城為墓,終不能從爾也。」受圍凡六月,不解。都哇以書系矢射城中曰:「我亦太祖皇帝諸孫,何以不附我?且爾祖嘗尚公主矣。爾能以女與我,我則休兵,不然則急攻爾。」其民相與言曰:「城中食且盡,力已困,都哇攻不止,則相與俱亡矣。」亦都護曰:「吾豈惜一女而不以救民命乎!然吾終不能與之相見。」以其女也立亦黑迷失別吉厚載以茵,引繩縋城下而與之,都哇解去。其後入朝,帝嘉其功,錫以重賞,妻以公主曰巴巴哈兒,定宗之女也。又賜鈔十萬錠以賑其民。還鎮火州,屯於州南哈密力之地,兵力尚寡,北方軍忽至其地,大戰力盡,遂死之。



 子紐林的斤,尚幼,詣闕請兵北征,以復父讎。帝壯其志,賜金幣巨萬,妻以公主曰不魯罕,太宗之孫女也。公主薨,又尚其妹曰八卜叉。有旨師出河西,俟北征諸軍齊發,遂留永昌。會吐蕃脫思麻作亂,詔以榮祿大夫平章政事,領本部探馬等軍萬人鎮吐蕃宣慰司。威德明信,賊用斂跡,其民賴以安。武宗召還,嗣為亦都護,賜之金印,復署其部押西護司之官。仁宗始稽故實,封為高昌王,別以金印賜之,設王傅之官。其王印行諸內郡,亦都護印行諸畏兀兒之境。八卜叉公主薨,復尚公主曰兀剌真,安西王之女也。領兵火州,復立畏兀兒城池。延祐五年薨。子二人,長曰帖木兒補化,次曰篯吉,皆八卜叉公主所生也。



 帖木兒補化,大德中,尚公主曰朵兒只思蠻,闊端太子孫女也。至大中,從父入覲,備宿衛。又事皇太后於東朝,拜中奉大夫,領大都護事。又以資善大夫出為鞏昌等處都總帥達魯花赤。奔父喪於永昌,請以王爵讓其叔父欽察臺,叔父力辭,乃嗣為亦都護高昌王。至治中,領甘肅諸軍,仍治其部。泰定中召還,與威順王寬徹不花、宣靖王買奴、靖安王闊不花分鎮襄陽。俄拜開府儀同三司、湖廣行省平章政事。文宗召至京師,佐平大難。時湖廣左丞有以忌嫉害政者,詔命誅之。帖木兒補化乃為申請曰:「是誠有罪,然不至死。」人服其雅量。天歷元年,拜開府儀同三司、上柱國、錄軍國重事、知樞密院事。明年正月,以舊官勛封拜中書左丞相。三月,加太子詹事;十月,拜御史大夫。其弟篯吉乃以讓嗣為亦都護高昌王。



 ○鐵邁赤虎都鐵木祿



 塔海



 鐵邁赤,合魯氏。善騎射,初事忽蘭皇後帳前,嘗命為挏馬官。從太祖定西夏。又從皇子闊出、忽都禿、行省鐵木答兒定河南,累有戰功。憲宗之伐宋也,世祖以皇弟受命攻鄂。大駕征西川,遣元帥兀良哈臺自交趾搗宋,與諸軍合。歲己未,皇弟駐兵鄂渚,聞兀良哈臺由廣西至長沙,遣鐵邁赤將練卒千人、鐵騎三千迎兀良哈臺於岳州。兀良哈臺得援,抵江夏,北涉黃州,鐵邁赤與有力焉。世祖即位,命從征叛王於失木土之地,勞績益著。至元七年,授蒙古諸萬戶府奧魯總管。十九年,以疾卒。子八人,虎都鐵木祿最顯。



 虎都鐵木祿好讀書,與學士大夫游,字之曰漢卿。仁宗嘗顧左右曰:「虎都鐵木祿字漢卿,漢名卿不讓也,汝等以漢卿名之宜矣。」其母姓劉氏,故人又稱之曰劉漢卿云。至元十一年,從丞相伯顏渡江。既取宋,遣視宋故宮室,護帑藏。諭下明、越等州。從平章奧魯入覲,授忠顯校慰總把,再轉昭信校尉。二十二年,授奉訓大夫,荊湖占城等處行中書省理問官。時行省之名曰荊湖占城,曰荊湖,曰湖廣,凡三改。理問一日以軍事入奏,敷陳辨白有指趣,世祖大悅,若曰:「辭簡意明,令人樂於聽受,昔以其兄阿里警敏捷給,令侍左右,斯人顧不勝耶?」敕都護脫因納志之。平章政事程鵬飛建議征日本,奏漢卿為征東省郎中。帝顧脫因納,若曰:「鵬飛南士也,猶知其能。姑聽之,候還,朕自錄任。」征東省罷,征漢卿還。丞相阿里海牙以湖廣行省機密事重,舍漢卿無可用者,遣郎中嶽洛也奴奏留,從之。二十一年,從皇子鎮南王征交趾。比還鄂時,權臣方擅威福,遂退處於家。二十八年,詔太傅、右丞相順德王答剌罕擒權奸於鄂。答剌罕遂拜湖廣行中書省平章政事,詢舊人知方面之務者,眾薦漢卿,遣使即南陽家居驛致武昌,奏事京師,帝嘉之,擢給事中。居再歲,提刑按察司改肅政廉訪司,臺臣奏授奉議大夫、廣西海北道副使,陛辭,留之仍舊職。既而湖廣行省平章政事劉國傑奏伐交趾,造戰船五百於廣東,帝曰:「此重事也,須才幹臣乃濟用。」以漢卿督匠南方,敕曰:「汝還,當顯汝於眾。」因頓首謝。事既集,帝崩,遷福建行省郎中,朝列大夫、漢陽監府,中順大夫、湖南宣慰副使。峒酋岑雄叛,奉詔開諭,頑獷帖服。改太中大夫、河南行中書省郎中,通議大夫、同僉樞密院事,拜禮部尚書。大臣奏核實江南民田,漢卿奉詔使江西,以田額舊定,重擾民不便,置不問。止奏茶、漕置局十有七所,以七品印章敕授局官五十一員,增中統課緡五十萬。轉正議大夫、兵部尚書。未幾命為中奉大夫、荊湖北道宣慰使,已命,復留之。延祐三年,大臣以浙東倭奴商舶貿易致亂,奏遣漢卿宣慰閩、浙,撫戢兵民,海陸為之靜謐云。從子塔海。



 塔海,漢卿兄子也。世祖時,從土土哈充哈剌赤。至元二十四年,扈駕征乃顏。二十六年,入覲,帝命充寶兒赤,扈駕至和林,賜只孫冠服。大德四年,授中書直省舍人。遷中書客省副使。武宗即位,賜中統鈔五百錠,以旌其能。尋進和林行省理問所官,改通政僉院。歷和寧路總管,改汴梁。先是,朝廷令民自實田土,有司繩以峻法,民多虛報以塞命,其後差稅無所於征,民多逃竄流移者。塔海以其弊言於朝。由是省民間虛糧二十二萬,民賴以安。後改任廬州,時有飛蝗北來,民患之,塔海禱於天,蝗乃引去,亦有墮水死者,人皆以為異。民乏食,開廩減直,俾民糴之,所活甚眾。天歷元年冬十月,樞密院臣奏以塔海充樞密僉院,守潼關及河中府。帝遣人馳賜白金鈔幣,宣授僉書樞密院事。未幾西軍犯南陽,督諸衛兵往平之。至其地,首率勇士與帖木哥等戰,摧其前鋒將,奪其旗鼓,西軍敗走。賜三珠虎符,進大都督,累官資善大夫。



 ○按扎兒



 按扎兒,拓跋氏,嘗扈從太祖南征。歲丙子,復從定諸部有功,命領蒙古軍為前鋒,時木華黎暨博爾術為左右萬戶長,各以其屬為翊衛。太祖命木華黎為太師國王都行省承制行事,兵臨燕、遼、營、青、齊、魯、趙、韓、魏,皆下。歲己卯,河中府降,兵北還,以按扎兒領前鋒總帥,仍統所部兵屯平陽以備金,攝國王事。時金將乞石烈氏擁兵數為邊患,然畏按扎兒威名,不敢輕犯其境。歲壬午,元帥石天應守河中府,屯中條山,金侯將軍率昆弟兵十餘萬夜襲河中,天應遣偏裨吳權府率五百兵出東門,伏兩谷間。諭之曰:「俟其半過,即翼擊之,俾腹背受敵,即成禽矣。」吳醉,敵至,聲援弗繼,城遂陷,天應死焉,遂燔其城,屠其民。將趨中條,按扎兒進兵擊之,斬首數萬級,逃免者僅十數。歲癸未春,至聞喜縣西下馬村,木華黎卒,詔以子孛魯襲其爵,時平陽重地,令按扎兒居守。歲庚寅,孛魯由雲中圍衛州,金將武仙恐,退保潞東十餘里原上,孛魯馳至沁南,未立鼓,乞石烈引兵襲其後,孛魯戰失利,輜重人口皆陷沒,按扎兒妻奴丹氏亦被獲,拘於大梁。金主聞按扎兒威名,召奴丹氏見,奴丹氏色莊言正,不為動。金主因謂之曰:「今縱爾還,能偕爾夫來,當厚賞爾。」奴丹氏佯諾之,遂得還。太宗聞而義之,召見,褒賚甚厚,遂詔預其夫前鋒事。帝率從弟按只吉歹、口溫不花大王、皇弟四太子暨國王孛魯征潞州、鳳翔。至鈞州三峰山,金將完顏合達引兵十五萬來戰,俘其同僉移剌不花等,悉誅之。明年壬辰春,三月,帝班師北還,命偕都元帥唆伯臺圍汴。城中識按扎兒旗幟,懼曰:「其妻猶勇且義,況其夫乎!」歲甲午,金亡,詔封功臣,賜平陽戶六百一十有四、驅戶三十、獵戶四。未幾,以疾卒。子忙漢、拙赤哥。



 至元十五年,忙漢為管軍千戶。二十四年,從征乃顏。二十六年,從征海都。二十七年,宣授蒙古侍衛親軍千戶,佩金符。元貞元年,有旨命領探馬赤軍,偕哈伯元帥從宗王出伯西征,改授昭信校尉、右都威衛千戶。大德元年,召還。至大四年卒。子乃蠻襲。



 拙赤哥入宿衛,從世祖征鄂漢,以功賜白金。至元三年,從征李璮,戰死之。子闊闊術為御史臺都事。至元三十一年,國王速渾察之子拾得既沒,其家有故璽,王將鬻之,命闊闊術以示中丞崔彧、御史楊桓,辨其文曰:「受命於天,既壽永昌。」蓋秦璽也。彧請獻之徽仁裕聖皇后,後以鈔二千五百貫賜拾得家,金織文段二賜闊闊術。成宗即位,近臣以其事聞,授闊闊術漢中廉訪僉事,仕至湖南廉訪使。



 ○雪不臺



 雪不臺,蒙古部兀良罕氏。遠祖捏里弼生孛忽都,雄勇有智略。曾孫合飭溫生哈班、哈不里。哈班生二子:長虎魯渾,次雪不臺。太祖初建興都於班硃泥河,今龍居河也。哈班驅群羊入貢,遇盜見執,雪不臺及兄虎魯渾隨至,剌盜殺之,眾潰去,哈班得以羊進帝所,由是父子兄弟以義勇稱。虎魯渾以百夫長西征,破乃蠻,立戰功。



 雪不臺以質子襲職,七年,攻桓州,先登,下其城,賜金幣凡一車。十一年,戰滅里吉眾於蟾河,追其部長玉峪,大破之,遂有其地。扈從征回鶻,其主棄國去,雪不臺率眾追之,回鶻竟走死。其帑藏之積盡入內府,賜寶珠一銀罌。十八年,討定欽察,鏖戰斡羅思大小密赤思老,降之,奏滅里吉、乃蠻、怯烈、杭斤、欽察部千戶,通立一軍。十九年,獻馬萬匹。二十一年,取馺里畏吾、特勤、赤憫等部,德順、鎮戎、蘭、會、洮等州,獻牝馬三千匹。太宗二年,大舉伐金,渡河而南,睿宗以太弟將兵渡漢水而北,會河南之三峰山。金大臣合達諸將步騎數十萬待戰,雪不臺從睿宗出牛頭關,謀曰:「城邑兵野戰不利,易破耳。」師集三峰,金圍之數匝,將士頗懼。俄而風雪大作,金卒殭踣,士氣遂奮,敵眾盡殪。河南諸州以次降破。四年夏,雪不臺總諸道兵攻汴,金義宗走衛州,又走歸德,又走蔡州。癸巳秋,汴將以城降,其冬攻蔡。六年春,金亡。雪不臺以汴民饑,縱使渡河就食,民德之。是年詔宗王拔都西征,雪不臺為先鋒,戰大捷。十三年,討兀魯思部主野力班,禽之。攻馬札部,與其酋怯憐戰漷寧河,遣偏師由下流搗其城,拔之。是時,北庭、西域、河南北、關隴皆底定,雪不臺功力居多。初,太祖征西夏,閔其久於行間,敕還省覲。雪不臺對曰:「君勞臣佚,情所未安。」帝壯而聽之。又金帥合達見獲,以不屈死,猶問雪不臺安在,請一識之。雪不臺出謂曰:「汝須臾人耳,識我何為?」曰:「人臣亦各為其主,卿勇蓋諸將,天生英豪,其偶然邪。吾見卿,甘心瞑目矣。」定宗三年,卒於篤列河之地,年七十有三。至大元年,贈效忠宣力佐命功臣、開府儀同三司、上柱國、河南王,謚忠定。



 ○唵木海



 唵木海,蒙古八剌忽鷿氏,與父孛合出俱事太祖,征伐有功。帝嘗問攻城略地,兵仗何先,對曰:「攻城以砲石為先,力重而能及遠故也。」帝悅,即命為砲手。歲甲戌,太師國王木華黎南伐,帝諭之曰:「唵木海言,攻城用砲之策甚善,汝能任之,何城不破。即授金符,使為隨路砲手達魯花赤。唵木海選五百餘人教習之,後定諸國,多賴其力。太宗即位,留為近侍,以講武藝。歲壬辰,從攻河南有功。壬子,憲宗特授虎符,升都元帥。癸丑,從宗王旭烈兀征剌里西番、斜巨山、桃裏寺、河西諸部,悉下之。卒,子忒木臺兒以從戰功授金符,襲砲手總管。



 至元十年,修立正陽東西二城,置砲二百餘座,與宋人戰,卻之。十三年,從丞相伯顏伐宋,駐軍臨安之阜亭山,同忙古歹等八人率甲三百入宋宮,取傳國寶。宋太后請解兵延見內殿,期明日奉寶乞降,至期,果遣賈餘慶等奉寶至軍前。以功授行省斷事官,復令其子忽都答兒襲砲手總管。十四年,進昭勇大將軍砲手萬戶,佩元降虎符,鎮平江之常熟。有叛民擁眾自號太尉者,行省會諸軍討之,與忽都答兒父子自為一軍,奮戈陷陣,斬賊酋戴太尉,擒硃太尉,帝嘉其功。十五年,兼平江路達魯花赤,尋遷徽州、湖州,卒。忽都答兒後升砲手萬戶,改授達魯花赤,卒。



 ○昔裏鈐部



 昔裏鈐部,唐兀人,昔里氏。鈐部亦云甘卜,音相近而互用也。太祖時,西夏既臣服,大軍西征,復懷貳心。帝聞之,旋師致討。命鈐部同忽都鐵穆兒招諭沙州。州將偽降,以牛酒犒師,而設伏兵以待之。首帥至,伏發馬躓,鈐部以所乘馬與首帥使奔,自乘所躓馬而殿後,擊敗之。他日,帝聞曰:「卿臨死地,而易馬與人,何也?」鈐部對曰:「小臣陣死,不足重輕,首帥乃陛下器使宿將,不可失也。」帝以為忠。進兵圍肅州,守者乃鈐部之兄,懼城破害及其家,先以為請。帝怒城久不下,有旨盡屠之,惟聽鈐部求其親族家人於死所,於是得免死者百有六戶,歸其田業。歲乙未,定宗、憲宗皆以親王與速卜帶征西域,明年啟行,鈐部亦在中。又明年,至寬田吉思海,鈐部從諸王拔都征斡羅斯,至也裏贊城,大戰七日,拔之。己亥冬十有一月,至阿速滅怯思城,負固久不下。明年春正月,鈐部率敢死士十人,躡雲梯先登,俘十一人,大呼曰:「城破矣!」眾蟻附而上,遂拔之。賜西馬、西錦,錫名拔都。明年班師,授鈐部千戶,賜只孫為四時宴服,尋遷斷事官。丙午,定宗即位,進秩大名路達魯花赤。憲宗以卜只兒來蒞行臺,命鈐部同署,既又別錫虎符,出監大名。己未,世祖南征,供給軍餉,未嘗乏絕。以疾輿歸,卒於家,年六十九。子愛魯。



 愛魯襲為大名路達魯花赤。至元五年,從雲南王征金齒諸部。蠻兵萬人絕縹甸道,擊之,斬首千餘級,諸部震服。六年,再入,定其租賦,平火不麻等二十四寨,得七馴象以還。七年,改中慶路達魯花赤,兼管爨僰軍。十年,平章賽典赤行省雲南,令愛魯疆理永昌,增田為多。十一年,閱中慶版籍,得隱戶萬餘,以四千戶即其地屯田。十三年,詔開烏蒙道,帥師至玉連等州,所過城寨,未附者盡擊下之,水陸皆置驛傳,由是大為賽典赤信任。十四年,忙部、也可不薛叛,以兵二千討平之,遷廣南西道左右兩江宣撫使,兼招討使。十六年,遷雲南諸路詔與西川都元帥也速答兒、湖南行省脫里察會師進討,禽也可不薛送京師,仁普諸酋長皆降,得戶四千。諸王相吾答兒帥諸將征緬,愛魯供饋餉,無乏絕。二十二年,烏蒙阿謀殺宣撫使以叛,與右丞拜答兒往征之,拜答兒以愛魯習知其山川道里,令諸軍悉聽指授,分道進擊,生擒阿蒙以歸。二十四年,進右丞。朝廷立尚書省,復改行尚書右丞。鎮南王征交趾,詔愛魯將兵六千人從之。自羅羅至交趾境,交趾將昭文王以兵四萬守木兀門,愛魯與戰破之,擒其將黎石、何英。比三月,大小一十八戰,乃至其王城,與諸軍會戰又二十餘合,功為多。二十五年,感瘴癘卒。贈平章政事,謚毅敏。



 子教化,中書平章政事,請於朝,贈其祖昔裏鈐部太師,謚貞獻,加贈愛魯太師,追封魏國公,改謚忠節。



 ○槊直腯魯華



 槊直腯魯華,蒙古克烈氏。初,以其部人二百從太祖征乃蠻、西夏有功,命將萬人,為太師國王木華黎前鋒。下金桓州,得其監馬幾百萬匹,分屬諸軍,軍勢大振。歲辛未,破遼東、西諸州,唯東京未下,獲金使,遣往諭之。槊直盾魯華曰:「東京,金舊都,備嚴而守固,攻之未易下,以計破之可也。請易服與其使偕往說之,彼將不疑,俟其門開,繼以大軍赴之,則可克矣。」卒如其計。徇地河北,攻大名,小大數十戰,城垂陷,中流矢而卒。武宗時,贈太傅,追封衛國公,謚武敏。



 子撒吉思卜華,嗣將其軍。太宗元年己丑,錫金符,安輯河北、山東諸州。先是真定同知武仙攻滅都元帥史天倪家,其弟天澤擊仙走,復真定。以天澤為真定、河間、濟南、東平、大名五路萬戶。庚寅,命撒吉思卜華佩金虎符,以總師行省監其軍。金宣宗之徙都於汴也,立河平軍於新衛以自固,恃為北門。撒吉思卜華數攻之,不拔。壬辰正月,太宗自白坡濟河而南,睿宗由峭石灘涉漢而北。撒吉思卜華集西都水之舟,渡自河陰。至鄭,鄭守馬伯堅降。及金義宗勢力窮蹙出奔,帝命撒吉思卜華追躡之,會其節度斜捻阿卜棄衛入汴,撒吉思卜華遂據而有之。十二月,義宗自黃陵岡濟河,謀復衛。撒吉思卜華與其將白撒戰白公廟五日夜,俘斬萬計,餘眾盡潰。義宗竄歸德。撒吉思卜華追躡其後,薄北門而軍。左右皆水,其舟師日至。癸巳四月,其將官奴夜來斫營,腹背受敵,撒吉思卜華與一軍皆沒。



 嗣國王塔思承制,以其弟明安答兒領其行營,尋有旨以為蒙古漢軍萬戶。明安答兒善騎射,從征淮安,因糧於敵,未嘗匱乏,軍士免負擔之勞,咸樂為用。癸丑,憲宗遣從昔烈門太子南伐,死於鈞州。五子,長腯虎,幼普闌溪。



 腯虎從世祖北征叛王,挺戈出入其陣,帝壯之,賜號拔都,賞白金四百五十兩。及平李璮之亂,亦有戰功。普闌溪,光祿大夫、徽政使。金亡,命大臣忽都虎料民分封功臣,撒吉思卜華妻楊氏自陳曰:「吾舅及夫皆死國事,而獨爾見遺。」事聞,帝曰:「彼家再世死難,宜賜新衛民二百戶。」撒吉思卜華贈太師,謚忠武。明安答兒贈太保,謚武毅,爵皆衛國公。



 ○昔兒吉思



 昔兒吉思,幼從太祖征回回、河西諸國,俱有戰功。太宗時,從睿宗西征,師次京兆府,會亦來哈鷿率諸部兵作亂,昔兒吉思挺身斫賊陣,下馬搏戰,賊眾莫不披靡。俄失所乘馬,步走至睿宗軍中。賊退,睿宗嘉其勤勞,妻以侍女唆火臺。世祖尤愛之,軍旅田獵,未嘗不在左右。初,昔兒吉思之妻為皇子乳母,於是皇太后待以家人之禮,得同飲白馬湩。時朝廷舊典,白馬湩非宗戚貴胄不得飲也。昔兒吉思子塔出,為寶兒赤、迭只斡耳朵千戶。塔出子千家奴、撒里蠻。千家奴從征乃顏,力戰而死,帝命籍乃顏人口、財物以賜之。撒里蠻年十六,從世祖討阿里不哥,戰於失門禿,有功,賜號拔都兒,賞賚尤厚,授光祿少卿,仍襲為迭只斡耳朵千戶,改同僉宣徽院,進僉院事。以管軍千戶從征乃顏有功,賞金盞二、金五十兩,復入為同知宣徽院事。成宗時,拜宣徽使,加大司徒,卒。子帖木迭兒襲為迭只斡耳朵千戶,累遷宣徽院使,遙授左丞相。



 ○哈散納



 哈散納,怯烈亦氏。太祖時,從征王罕有功,命同飲班硃尼河之水,且曰:「與我共飲此水者,世為我用。」後管領阿兒渾軍,從太祖征西域,下薛迷則乾、不花剌等城。至太宗時,仍命領阿兒渾軍,並回回人匠三千戶駐於蕁麻林。尋授平陽、太原兩路達魯花赤,兼管諸色人匠,後以疾卒。子捏古伯襲,從憲宗攻釣魚山,有功,以疾卒。子撒的迷失襲。撒的迷失卒,子木八剌襲,充貴赤千戶,遷西域親軍副都指揮使,大德元年卒。弟禿滿答襲,禿滿答卒,子哈剌章襲。



\end{pinyinscope}