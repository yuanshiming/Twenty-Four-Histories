\article{志第九 歷六}

\begin{pinyinscope}

 ○庚午元歷下



 步交會術



 交終分,一十四萬二千三百一十九,秒九千三百六,微二十。



 交終日,二十七,餘一千一百九,秒九千三百六,微二十。



 交中日,一十三,餘三千一百六十九,秒四千六百五十三,微一十。



 交朔日,二,餘一千六百六十五,秒六百九十三,微八十。



 交望日,一十四,餘四千二,秒五千。



 秒母,一萬。



 微母,一百。



 交終度,三百六十三,分七十九,秒三十六。



 交中度,一百八十一,分八十九,秒六十八。



 交象度,九十,分九十四,秒八十四。



 半交象度,四十五,分四十七,秒四十二。



 日食既前限,二千四百。定法,二百四十八。



 日食既後限,三千一百。定法,三百二十。



 月食限,五千一百。



 月食既限,一千七百。定法,三百四十。



 分秒母,皆一百。



 求朔望入交先置裡差,半之,如九而一,所得依其加減天正朔積分,然後求之。



 置天正朔積分,以交終分去之,不盡,如日法而一,為日,不滿為餘,即得天正十一月中朔入交泛日及餘秒。便為中朔加時入交泛日及餘。交朔加之,得次朔;交望加之,得望;再加交望,亦得次朔;各為朔望入交泛日及餘秒。凡稱餘秒者,微亦從之,餘仿此。



 求定朔及每日夜半入交



 各置入交泛日及餘秒,減去中朔望小餘,即為定朔望夜半入交泛日及餘秒。若定朔望有進退者,亦進退交日,否則因中為定,大月加二日,小月加一日,餘皆加四千一百二十,秒六百九十三,微八十,即次朔夜半入交;累加一日,滿交終日及餘秒,去之,即每日夜半入交泛日及餘秒。



 求定朔望加時入交



 置中朔望加時入交泛日及餘秒,以入氣入轉朓朒定數朓減朒加之,即得定朔望加時入交泛日及餘秒。



 求定朔望加時入交積度及陰陽歷



 置定朔望加時入交泛日,以日法通之,內餘進二位,如三萬九千一百二十一而一,為度,不滿,退除為分秒,即得定朔望加時月行入交積度;以定朔望加時入轉遲疾度遲減疾加之,即為月行入定交積度;如交中度以下,為入陽歷積度,以上,去之,為入陰歷積度。每日夜半準此求之。



 求月去黃道度



 視月入陰陽歷積度及分,交象以下,為少象;以上,覆減交中,餘為老象。置所入老少象度於上位,列交象度於下,相減,相乘,倍之,退位為分,分滿百為度,用減所入老少象度及分;餘,又與交中度相減、相乘,八因之,以一百一十除之,為分,分滿百為度,即得月去黃道度及分。



 求朔望加時入交常日及定日



 置朔望入交泛日,以入氣朓朒定數朓減朒加,為入交常日。又置入轉朓朒定數,進一位,以一百二十七而一,所得,朓減朒加交常日,為入交定日及餘秒。



 求入交陰陽歷交前後分



 視入交定日,如交中以下,為陽歷;以上,去之,為陰歷。如一日上下,以日法通日內分,內餘為交後分;十三日上下,覆減交中日,餘為交前分。



 求日月食甚定餘



 置朔望入氣入轉朓朒定數,同名相從,異名相消,以一千三百三十七乘之,以定朔望加時入轉算外轉定分除之,所得,以朓減朒加中朔望小餘,為泛餘。日食,視泛餘,如半法以下,為中前,半法以上,去之,為中後。置中前後分,與半法相減、相乘,倍之,萬約為分,曰時差。中前以時差減泛餘,為定餘;覆減半法,餘為午前分;中後以時差加泛餘,為定餘;減去半法,餘為午後分。月食,視泛餘,在日入後夜半前,如日法四分之三以下,減去半法,為酉前分;四分之三以上,覆減日法,餘為酉後分。又視泛餘,在夜半後日出前者,如日法四分之一以下,為卯前分;四分之一以上,覆減半法,餘為卯後分。其卯酉前後分,自相乘,四因,退位,萬約為分,以加泛餘,為定餘。各置定餘,以發斂加時法求之,即得日月食甚辰刻及分秒。



 求日月食甚日行積度



 置定朔望食甚大小餘,與中朔望大小餘相減之,餘以加減中朔望入氣日餘,以中朔望少加多減。即為食甚入氣;以加其氣中積,為食甚中積。又置食甚入氣餘,以所入氣日損益率盈縮之損益。乘之,如日法而一,以損益其日盈縮積,盈加縮減食甚中積,即為食甚日行積度及分。先以食甚中積經分為約分,然後加減之,餘類此者,依而求之。



 求氣差



 置日食食甚日行積度及分,滿中限去之,餘在象限以下,為初限;以上,覆減中限,為末限;皆自相乘,進二位,以四百七十八而一,所得,用減一千七百四十四,餘為氣差恆數;以午前後分乘之,半晝分除之,所得,以減恆數,為定數。如不及減者,覆減為定數,應加者減之,應減者加之。春分後,陽歷減陰歷加;秋分後,陽歷加陰歷減。春分前秋分後,各二日二千一百分為定氣,於此宜加減之。



 求刻差



 置日食食甚日行積度及分,滿中限去之,餘與中限相減、相乘,進二位,如四百七十八而一,所得,為刻差恆數;以午前後分乘之,日法四分之一除,所得,為定數。若在恆數以上者,倍恆數,以所得之數減之,為定數,依其加減。冬至後,午前陽加陰減,午後陽減陰加;夏至後,午前陽減陰加,午後陽加陰減。



 求日食去交前後定分



 置氣刻二差定數,同名相從,異名相消,為食差;依其加減去交前後分,為去交前後定分。視其前後定分,如在陽歷,即不食;如在陰歷,即有食之。如交前陰歷不及減,反減之,反減食差。為交後陽歷;交後陰歷不及減,反減之,為交前陽歷;即不食。交前陽歷不及減,反減之,為交後陰歷;交後陽歷不及減,反減之,為交前陰歷;即日有食之。



 求日食分



 視去交前後定分,如二千四百以下,為既前分;以二百四十八除,為大分;二千四百以上,覆減五千五百,不足減者不食。為既後分;以三百二十除,為大分,不盡,退除為秒。其一分以下者,涉交太淺,太陽光盛,或不見食。



 求月食分



 視去交前後分,不用氣刻差者。一千七百以下者,食既;以上,覆減五千一百,不足減者不食。餘以三百四十除之,為大分;不盡,退除為秒,即月食之分秒。去交分在既限以下,覆減既限,亦以三百四十除之,為既內之大分。



 求日食定用分



 置日食之大分,與二十分相減、相乘,又以二千四百五十乘之,如定朔入轉算外轉定分而一,所得,為定用分;減定餘,為初虧分;加定餘,為復圓分;各以發斂加時法求之,即得日食三限辰刻也。



 求月食定用分



 置月食之大分,與三十五分相減、相乘,又以二千一百乘之,如定望入轉算外轉定分而一,所得,為定用分;加減定餘,為初虧復圓分。各如發斂加時法求之,即得月食三限辰刻。



 月食既者,以既內大分,以一十五分相減相乘,又以四千二百乘之,如定望入轉算外轉定分而一,所得為既內分;用減定用分,為既外分。置月食定餘,減定用分,為初虧分;因加既外分,為食既分;又加既內分,為食甚分;即定餘分是也。



 再加既內分,為生光分;復加既外分,為復圓分。各以發斂加時法求之,即得月食五限辰刻及分。如月食既者,以十分並既內大分,如其法而求其定用分也。



 求月食所入更點



 置食甚所入日晨分,倍之,五約之,為更法;又五約之,為點法。乃置月食初末諸分,昏分以上者,減昏分;晨分以下者,加晨分;如不滿更法,為初更;不滿點法,為一點。依法以次求之,即得更點之數。



 求日食所起



 食在既前,初起西南,甚於正南,復於東南。食在既後,初起西北,甚於正北,復於東北。其食八分以上者,皆起正西,復正東。此據正午地而論之。



 求月食所起



 月在陽歷,初起東北,甚於正北,復於西北。月在陰歷,初起東南,甚於正南,復於西南。其食八分以上,皆起正東,復正西。此亦據正午地而論之。



 求日月出入帶食所見分數



 各以食甚小餘,與日出入分相減,餘為帶食差;以乘所食之分,滿定用分而一,月食既者,以既內分減帶食差,餘乘所食分,如既外分而一,不及減者,為帶食既出入。以減所食分,即日月出入帶食所見之分。其食甚在晝,晨為漸進,昏為已退;食甚在夜,晨為已退,昏為漸進也。



 求日月食甚宿次



 置日月食甚日行積度,望即更加望度。以天正冬至加時黃道日度加而命之,依黃道宿次去之,即各得日月食甚宿度及分秒。



 步五星術



 木星



 周率,二百八萬六千一百四十二,秒九。



 歷率,二千二百六十五萬五百五十七。



 歷度法,六萬二千一十四。



 周日,三百九十八日八十八分。



 歷度,三百六十五度二十四分九十秒。



 歷中,一百八十二度六十二分四十五秒。



 歷策,一十五度二十一分八十七秒。



 伏見,一十三度。



 以下表格略



 火星



 周率,四百七萬九千四十二,秒一十四半。



 歷率,三百五十九萬二千七百五十七,秒四十四少。



 歷度法,九千八百三十六半。



 周日,七百七十九日九十三分一十六秒。



 歷度,三百六十五度二十四分七十五秒。



 歷中,一百八十二度六十二分三十七秒半。



 歷策,一十五度二十一分八十六秒。



 伏見,一十九度。



 以下表格略



 土星



 周率,一百九十七萬七千四百一十一,秒六十九。



 歷率,五千六百二十二萬三千二百四十八半。



 歷度法,一十五萬三千九百二十八。



 周日,三百七十八日九分二秒。



 歷度,三百六十五度二十五分六十八秒。



 歷中,一百八十二度六十二分八十四秒。



 歷策,一十五度二十一分九十秒。



 伏見,一十七度。



 以下表格略



 金星



 周率,三百五萬三千八百四,秒六十三太。



 歷率,一百九十一萬二百四十,秒七十六半。



 歷度法,五千二百三十。



 周日,五百八十三日九十分一十四秒。



 合日,二百九十一日九十五分七秒。



 歷度,三百六十五度二十四分六十八秒。



 歷中,一百八十二度六十二分三十四秒。



 歷策,一十五度二十一分八十六秒。



 伏見,一十度半。



 以下表格略



 水星



 周率,六十萬六千三十一,秒七十七半。



 歷率,一百九十一萬二百四十二,秒一十三半。



 歷度法,五千二百三十。



 周日,一百一十五日八十七分六十秒。



 合日,五十七日九十三分八十秒。



 歷度,三百六十五度二十四分七十秒。



 歷中,一百八十二度六十二分三十五秒。



 歷策,一十五度二十一分八十五秒。



 晨伏夕見,一十四度。



 夕伏晨見,一十九度。



 以下表格略



 求五星天正冬至後平合及諸段中積中星



 置通積分,先以里差加減之。各以其星周率去之,不盡,為前合分;覆減周率,餘為後合分;如日法而一,不滿,退除為分秒,即得其星天正冬至後平合中積中星。命為日,曰中積;命為度,曰中星。以段日累加中積,即為諸段中積;以平度累加中星,經退則減之,即為諸段中星。



 求五星平合及諸段入歷



 置通積分,各加其星後合分,以歷率去之,不盡,各以其歷度法除為度,不滿,退除為分秒,即為其星平合入歷度及分秒;以諸段限度累加之,即得諸段入歷度及分秒。



 求五星平合及諸段盈縮定差



 各置其星段入歷度及分秒,如在歷中以下,為盈;以上,減去歷中,餘為縮。以其星歷策除之,為策數;不盡,為入策度及分。命策數算外,以其策損益率乘之,餘歷策而一,為分,以損益其下盈縮積度,即為其星段盈縮定差。



 求五星平合及諸段定積



 各置其星段中積,以其段盈縮定差盈加縮減之,即得其段定積日及分;加天正冬至大餘及約分,滿紀法,去之,不滿,命壬戌算外,即得日辰也。



 求五星平合及諸段所在月日



 各置其段定積,以加天正閏日及約分,以朔策及約分除之,為月數;不盡,為入月以來日數及分。其月數,命天正十一月算外,即得其段入月中朔日數及分;乃以日辰相距,為所在定朔月日。



 求五星平合及諸段加時定星



 各置中星,以盈縮定差盈加縮減,金星倍之,水星三之,然後加減。即為五星諸段定星;以加天正冬至加時黃道日度,依宿次命之,即其星其段加時所在宿度及分秒。



 求五星諸段初日晨前夜半定星



 各以其段初行率,乘其段定積日下加時分,百約之,乃順減退加其日加時定星,即其段初日晨前夜半定星所在宿度及分秒。



 求諸段日率度率



 各以其段日辰,距後段日辰為日率。以其段夜半宿次,與後段夜半宿次相減,餘為度率。



 求諸段平行分



 各置其段度率及分秒,以其段日率除之,即得其段平行度日及分秒。



 求諸段總差及日差



 本段前後平行分相減,為其段泛差;假令求木星次疾泛差,乃以順疾順遲平行分相減,餘為次疾泛差,他皆仿此。倍而退位,為增減差;加減其段平行分,為初末日行分;前多後少者,加為初,減為末;前少後多者,減為初,加為末。倍增減差,為總差;以日率減一除之,為日差。



 求前後伏遲退段增減差



 前伏者,置後段初日行分,加其日差之半,為末日行分;後伏者,置前段末日行分,加其日差之半,為初日行分;以減伏段平行分,餘為增減差。前遲者,置前段末日行分,倍其日差減之,為初日行分;後遲者,置後段初日行分,倍其日差減之,為末日行分;以遲段平行分減之,餘為增減差。前後近留遲段。木火土三星,退行者,六因平行分,退一位,為增減差。金星,前後伏退者,三因平行分,半而退位,為增減差。前退者,置後段初日之行分,以其日差減之,為末日行分。後退者,置前段末日之行分,以其日差減之,為初日行分;以本段平行分減之,餘為增減差。水星,平行分為增減差,皆以增減差加減平行分,為初末日行分。前多後少,加初減末;前少後多,減初加末。又倍增減差為總差,以日率減一,除之,為日差。



 求每日晨前夜半星行宿次



 各置其段初日行分,以日差累損益之,後少則損之,後多則益之。為每日行度及分秒;乃順加退減之,滿宿次去之,即得每日晨前夜半星行宿次。視前段末日後段初日行分相較之數,不過一二日差為妙;或多日差數倍,或顛倒不倫,當類同前後增減差稍損益之,使其有倫,然後用之。或前後平行分俱多俱少,則平注之;或總差之秒不盈一分,亦平注之;若有不倫而平注得倫者,亦平注之。



 求五星平合及見伏入氣



 置定積,以氣策及約分除之,為氣數;不滿,為入氣日及分秒;命天正冬至算外,即得所求平合及見伏入氣日及分秒。



 求五星平合及見伏行差



 各以其段初日星行分與太陽行分相減,餘為行差。若金在退行、水在退合者,相並為行差。如水星夕伏晨見者,直以太陽行分為行差。



 求五星定合及見伏泛積



 木火土三星,各以平合晨疾夕伏定積,為定合定見定伏泛積。金水二星,置其段盈縮定差,水星倍之。各以行差除之,為日,不滿,退除為分秒;若在平合夕見晨伏者,盈減縮加;如在退合夕伏晨見,盈加縮減;皆以加減定積為定合定見定伏泛積。



 求五星定合定積定星



 木火土三星,各以平合行差除其日太陽盈縮差,為距合差日;以太陽盈縮差減之,為距合差度;日在盈縮,以差日差度減之;在縮歷,加之;加減其星定合泛積,為定合定積定星。金水二星,順合退合,各以平合退合行差,除其日太陽盈縮差,為距合差日;順加退減太陽盈縮差,為距合差度;順在盈歷,以差日差度加之;在縮歷,減之;退在盈歷,以差日減之,差度加之;在縮歷,以差日加之,差度減之;皆以加減其定星定合再定合泛積,為定合再定合定積定星;以冬至大餘及約分加定積,滿紀法,去之,命得定合日辰;以冬至加時黃道日度加定星,滿宿次,去之,即得定合所在宿次。其順退所在盈縮,即太陽盈縮。



 求木火土三星定見伏定日



 各置其星定見伏泛積,晨加夕減象限日及分秒;半中限為象限。如中限以下,自相乘;以上,覆減歲周日及分秒,餘亦自相乘;滿七十五而一,所得,以其星伏見度乘之,一十五除之,為差。其差,如其段行差而一,為日,不滿,退除為分秒;見加伏減泛積,為定積;加命如前,即得日辰。



 求金水二星定見伏定日



 各以伏見日行差,除其日太陽盈縮差,為日。若晨伏夕見,日在盈歷,加之;在縮歷,減之;如夕伏晨見,日在盈縮,減之,在縮歷,加之;加減其星泛積,為常積。視常積,如中限以下,為冬至後;以上,去之,餘為夏至後。其二至後,如象限以下,自相乘;以上,覆減中限,餘亦自相乘;各如法而一為分,冬至後晨,夏至後夕,以一十八為法;冬至後夕,夏至後晨,以七十五為法。以伏見度乘之,一十五除之,為差。其差,滿行差而一,為日,不滿,退除為分秒;加減常積,為定積;冬至後,晨見夕伏,加之;夕見晨伏,減之。夏至後,晨見夕伏,減之;夕見晨伏,加之。加命如前,即得定見伏日辰。



 其水星,夕疾在大暑氣初日至立冬氣九日三十五分以下者,不見;晨留在大寒氣初日至立夏氣九日三十五分以下者,不見。春不晨見,秋不夕見者,亦舊歷有之。



\end{pinyinscope}