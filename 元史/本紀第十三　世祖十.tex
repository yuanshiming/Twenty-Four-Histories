\article{本紀第十三 世祖十}

\begin{pinyinscope}

 二十一年春正月乙卯,帝御大明殿,右丞相和禮霍孫率百官奉玉冊玉寶,上尊號曰憲天述道仁文義武大光孝皇帝,諸王百官朝賀如朔旦儀,赦天下。丁巳,敕:「自今凡奏事者,必先語同列以所奏。既奏,其所奉旨云何,令同列知而後書之簿;不明以告而輒書簿者,杖必闍赤。」己未,罷雲南都元帥府,所管軍民隸行省。甲子,罷揚州等處理算官,以其事付行省。江浙行省平章忙忽帶進真珠百斤。丙寅,闊闊你敦言:「屯田芍陂兵二千,布種二千石,得粳糯二萬五千石有奇,乞增新附軍二千。」從之。丁卯,建都王、烏蒙及金齒一十二處俱降。建都先為緬所制,欲降未能。時諸王相吾答兒及行省右丞太卜、參知政事也罕的斤分道征緬,於阿昔、阿禾兩江造船二百艘,順流攻之,拔江頭城,令都元帥袁世安戍之。遂遣使招諭緬王,不應。建都太公城乃其巢穴,遂水陸並進,攻太公城,拔之,故至是皆降。庚午,立江淮、荊湖、江西、四川行樞密院,治建康、鄂州、撫州、成都。立耽羅國安撫司。辛未,相吾答兒遣使進緬國所貢珍珠、珊瑚、異彩及七寶束帶。甲戌,遣蒙古官及翰林院官各一人祠岳瀆后土。遣王積翁齎詔使日本,賜錦衣、玉環、鞍轡。積翁由慶元航海至日本近境,為舟人所害。御史臺臣言:「罪黜之人,久忘其名又復奏用,乞戒約。」帝曰:「卿等所言固是,然其間豈無罪輕可錄用者?」御史大夫玉速帖木兒對曰:「以各人所犯罪狀明白敷奏,用否當取聖裁。」從之。丙子,建寧叛賊黃華自殺。丁丑,雲南諸路按察司官陛辭,詔諭之曰:「卿至彼,當宣明朕意,勿求貨財,名成則貨財隨之,徇財則必失其名,而性命亦不可保矣。」己卯,馬八兒國遣使貢珍珠、異寶、縑段。



 二月辛巳,以福建宣慰使管如德為泉州行省參知政事,徵緬。浚揚州漕河。罷高麗造征日本船。丁亥,命翰林學士承旨撒里蠻祀先農於藉田。壬辰,以江西叛寇妻子賜鷹坊養虎者。以別速帶逃軍七百餘人付安西王屯田,給以牛具。邕州、賓州民黃大成等叛,梧州、韶州、衡州民相挻而起,湖南宣慰使撒里蠻將兵討之。甲午,罷群牧所。己亥,瑞州獲叛民晏順等三十二人,並妻孥送京師。罷阿八赤開河之役,以其軍及水手各萬人運海道糧。放檀州淘金五百人還家。丁未,括江南樂工。命阿塔海發兵萬五千人、船二百艘助征占城,船不足,命江西省益之。戊申,徙江淮行省於杭州,徙浙西宣慰司於平江,省黃州宣慰司入淮西道。立法輪竿於大內萬壽山,高百尺。漳州盜起,命江浙行省調兵進討。秦州總管劉發有罪,嘗欲歸黃華,事覺伏誅,遷故宋宗室及其大臣之仕者於內地。



 三月辛亥,敕思、播管軍民官自今勿遷。丁巳,皇子北平王南木合至自北邊。王以至元八年建幕庭於和林北野裡麻里之地,留七年,至是始歸,右丞相安童繼至。以張弘範等將新附軍。壬戌,更定虎符。丙寅,乘輿幸上都。丁卯,太廟正殿成,奉安神主。甲戌,置潮、贛、吉、撫、建昌戍兵。乙亥,高麗國王王睶以皇帝尊號禮成,遣使來賀。



 夏四月壬午,令軍民同築堤堰,以利五衛屯田。乙酉,省泉府司入戶部,立大都留守司兼少府監,立大都路總管府,立西川、延安、鳳翔、興元宣課司。從迷裏火者、蜜剌裡等言,以鈔萬錠為市於別十八里及河西、上都。以火者赤依舊揚州鹽運使,歲市鹽八十萬石以贖過。己亥,涿州巨馬河決,沖突三十餘里。庚子,湖廣行省平章阿里海牙請身至海濱收集占城散軍,復使南征,且趣其未行者,許之。壬寅,江淮行省進各翼童男女百人。忽都鐵木兒征緬之師為賊沖潰。戊申,高麗王王睶及公主以其世子謜來朝。敕發思、播田、楊二家軍二千從征緬。籍江南鹽徒軍,藏匿者有罪。火兒忽等所部民戶告饑,帝曰:「饑民不救,儲糧何為?」發萬石賑之。命開元等路宣慰司造船百艘,付狗國戍軍。雲南行省為破緬國江頭城,進童男女八十人,並銀器幣帛。



 五月己酉,從禿禿合言,立二千戶,總欽察、康里子弟願為國宣勞者。壬子,拘征東省印。癸丑,樞密院臣言:「唆都潰軍已令李恆收集,江淮、江西兩省潰軍,別遣使招諭,凡至者皆給之糧,舟楫損者修之,以俟阿里海牙調用。」從之。戊午,敕中書省:「奏目及文冊,皆不許用畏吾字,其宣命、札付並用蒙古書。」己未,荊湖占城行省言:「忽都虎、忽馬兒等將兵征占城,前鋒舟師至舒眉蓮港不知所向,令萬戶劉君慶進軍次新州,獲占蠻,始知我軍已還矣。就遣占蠻向導至占城境,其國主遣阿不蘭以書降,且言其國經唆都軍馬虜掠,國計已空,俟來歲遣嫡子以方物進。繼遣其孫路司理勒蟄等奉表詣闕。」乙丑,取高麗所產鐵。蠲江南今年田賦十分之二,其十八年已前逋欠未征者,盡免之。阿魯忽奴言:「曩於江南民戶中撥匠戶三十萬,其無藝業者多,今已選定諸色工匠,餘十九萬九百餘戶宜縱令為民。」從之。詔諭各道提刑按察司分司事宜。庚午,荊湖占城行省以兵進據烏馬境,地近安南,請益兵,命鄂州達魯花赤趙翥等奉璽書往諭安南。河間任丘縣民李移住謀叛,事覺伏誅。括天下私藏天文圖讖《太乙雷公式》、《七曜歷》、《推背圖》、《苗太監歷》,有私習及收匿者罪之。丁丑,忽都虎、烏馬兒、劉萬戶等率揚州省軍二萬赴唆都軍前,遇風船散,其軍皆潰。敕追烏馬兒等誥命、虎符及部將所受宣敕,以河西孛魯合答兒等代之,聽阿里海牙節制。



 閏五月己卯,封法里剌王為郡王,佩虎符。改思、播二州隸順元路宣撫司,罷西南番安撫司,立總管府。給西川蒙古軍鈔,使備鎧仗,耕遂寧沿江曠土以食,四頃以下者免輸地稅。命總帥汪惟正括四川民戶。辛巳,加封衛輝路小清河神曰洪濟威惠王。壬午,蒙古侍衛親軍都指揮使八忽帶征黃華回,進人口百七十一。乙酉,以雲南境內洪城並察罕章,隸皇太子。丙戌,行御史臺自揚州遷於杭州。庚寅,賜歸附洞蠻官十八人衣,遣還。癸巳,賜北安王螭紐金印。罷皮貨所。理算江南諸行省造征日本船隱幣,詔按察司毋得沮撓。甲辰,安南國王世子陳日烜遣其中大夫陳謙甫貢玉杯、金瓶、珠絳、金領及白猿、綠鳩、幣帛等物。丙午,以侍衛親軍萬人修大都城。



 六月壬子,遣使分道尋訪測驗晷景、日月交食、歷法。增官吏俸,以十分為率,不及一錠者量增五分。甲寅,詔封皇子脫歡為鎮南王,賜塗金銀印,駐鄂州。庚申,改蒙古都元帥府為蒙古都萬戶府,砲手元帥府為砲手萬戶府,砲手都元帥府為回回砲手軍匠萬戶府。甲子,命也速帶兒所部軍六十人淘金雙城。從憨答孫請,移阿剌帶和林屯田軍與其所部相合,屯田五河。乙丑,中衛屯田蝗。甲戌,賜皇子愛牙赤怯薛帶孛折等及兀剌海所部民戶鈔二萬一千六百四十三錠,皇子南木合怯薛帶、怯憐口一萬二百四十六錠。以馬一萬一百九十五、羊一萬六十,賜朵魯朵海扎剌伊兒所部貧軍。



 秋七月丁丑朔,敕荊湖、西川兩省合兵討義巴、散毛洞蠻。雲南省臣言:「騰越、永昌、羅必丹民心攜貳,宜令也速帶兒或汪總帥將兵討之。」制曰:「可。」命樞密院差軍修大都城。己卯,立衍福司。中書省臣言:「宰相之名,不宜輕授。今占城省臣已及七人,宜汰之。」詔軍官勿帶相銜。賜皇子北安王印。復揚州管匠提舉司。丁亥,江淮行省以占城所遣太半達連扎赴闕,及其地圖來上。塔剌赤言:「頭輦哥國王出戍高麗,調旺速等所部軍四百以往,今頭輦哥已回,留軍耽羅,去其妻子已久,宜令他軍更戍。」伯顏等議,以高麗軍千人屯耽羅,其留戍四百人縱之還家,從之。戊子,詔鎮南王脫歡征占城。遣所留安南使黎英等還其國,日烜遣其中大夫阮道學等以方物來獻。總帥汪惟正言:「一門兄弟從仕者眾,乞仍於秦、鞏州置便宜都總帥府,仍用元帥印,即其兄弟四人擇一人為總帥,總帥之下總管府令其兼之。汪氏二人西川典兵者,亦擇其一為萬戶,餘皆依例遷轉。」從之。賜貧乏者阿魯渾、玉龍帖木兒等鈔,共七千四百八十錠。



 八月丁未,雲南行省言:「華帖、白水江、鹽井三處土老蠻叛,殺諸王及行省使者。」調兵千人討之。定擬軍官格例,以河西、回回、畏吾兒等依各官品充萬戶府達魯花赤,同蒙古人,女直、契丹同漢人。若女直、契丹生西北不通漢語者,同蒙古人;女直生長漢地,同漢人。己酉,御史臺臣言:「無籍之軍願從軍殺掠者,初假之以張渡江兵威,今各持弓矢,剽劫平民,若不分隸各翼,恐生他變。」詔遣之還家。辛亥,征東招討司聶古帶言:「有旨進討骨嵬,而阿里海牙、朵剌帶、玉典三軍皆後期。七月之後,海風方高,糧仗船重,深虞不測,姑宜少緩。」從之。占城國王乞回唆都軍,願以土產歲修職貢,使太盤亞羅日加翳、大巴南等十一人奉表詣闕,獻三象。甲子,放福建畬軍,收其軍器,其部長於近處州郡民官遷轉。庚午,車駕至自上都。甲戌,搠完上言:「建都女子沙智治道立站有功,已授虎符,管領其父元收附民為萬戶。今改建昌路總管,仍佩虎符。」從之。



 九月甲申,京師地震。並市舶司入鹽運司,立福建等處鹽課市舶都轉運司。中書省言:「福建行省軍餉絕少,必於揚州轉輸,事多遲誤。若並兩省為一,分命省臣治泉州為便。」詔以中書右丞、行省事忙兀臺為江淮等處行中書省平章政事,其行省左丞忽剌出、蒲壽庚,參政管如德分省泉州。癸巳,太白犯南斗。丙申,以江南總攝楊璉真加發宋陵塚所收金銀寶器修天衣寺。甲辰,海南貢白虎、獅子、孔雀。



 冬十月丁未,享於太廟。戊申,四川行省言金齒遺民尚多未附,以要剌海將探馬赤軍二千討之。己酉,敕:「管軍萬戶為行省宣慰使者,毋兼管軍事;仍為萬戶者,毋兼蒞民政。」壬子,定漣海等處屯田法。辛酉,征東招討司以兵征骨嵬。宋有手記軍,死則以兄弟若子繼,詔依漢軍籍之,毋文其手。丁卯,和禮霍孫請設科舉,詔中書省議,會和禮霍孫罷,事遂寢。以招討使張萬為征緬招討使,佩三珠虎符。戊辰,立常平倉,以五十萬石價鈔給之。甲戌,詔諭行中書省,凡征日本船及長年篙手,並官給鈔增價募之。賜貧乏者押失、忻都察等鈔一萬四千三錠。



 十一月甲申,封南木里、忙哥赤郡公。戊子,命北京宣慰司修灤河道。己丑,江西行省參知政事也的迷失禽獲海盜黎德及招降餘黨百三十三人,即其地誅黎德以徇,以黎德弟黎浩及偽招討吳興等檻送京師。遷轉官員薄而不就者,其令歸農當役。庚寅,占城國王遣使大羅盤亞羅日加翳等奉表來賀聖誕節,獻禮幣及象二,占城舊州主寶嘉婁亦奉表入附。庚子,以範文虎為左丞,商量樞密院事。太陰犯心。辛丑,和禮霍孫、麥術丁、張雄飛、溫迪罕皆罷。前右丞相安童復為右丞相,前江西榷茶運使盧世榮為右丞,前御史中丞史樞為左丞,不魯迷失海牙、撒的迷失並參知政事,前戶部尚書拜降參議中書省事。敕中書省整治鈔法,定金銀價,禁私自回易,官吏奉行不虔者罪之。壬寅,安童、盧世榮言:「阿合馬專政時所用大小官員,例皆奏罷,其間豈無通才?宜擇可用者仍用之。」詔依所言汰選,毋徇私情。癸卯,福建行省遣使入八合魯思招降南巫里、別里剌、理倫、大力等四國,各遣其相奉表以方物來貢。以江淮間自襄陽至東海多荒田,命司農司立屯田法,募人開耕,免其六年租稅並一切雜役。賜蒙古貧乏者也裏古、薛列海、察吉兒等鈔十二萬四千七百二十二錠。



 十二月甲辰朔,中書省臣言:「江南官田為權豪寺觀欺隱者多,宜免其積年收入,限以日期,聽人首實,逾限為人所告者,徵以其半給告者。」從之。立常平鹽局。乙巳,崔彧言盧世榮不可為相,忤旨罷。以丁壯萬人開神山河,立萬戶府以總之。辛亥,以儀鳳司隸衛尉院。癸亥,盧世榮言:「京師富豪戶釀酒,價高而味薄,以致課不時輸,宜一切禁罷,官自酤賣,向之歲課,一月可辦。」從之。甲子,以高麗提舉司隸工部。乙丑,祀太一。丙寅,荊湖占城行省遣八番劉繼昌諭降龍昌寧、龍延萬等赴闕,奉羊馬、白氈來貢,各授本處安撫使。立宣慰司,招撫西南諸蕃等處酋長。癸酉,命翰林承旨撒里蠻、翰林集賢大學士許國禎,集諸路醫學教授增修《本草》。是月,鎮南王軍至安南,殺其守兵,分六道以進,安南興道王以兵拒於萬劫,進擊敗之,萬戶倪閏戰死於劉村。以涇州隸都總帥府、賜蒙古貧乏者兀馬兒等鈔二千八百八十五錠、銀四十錠。



 二十二年春正月戊寅,以命相詔天下。民間買賣金銀、懷孟諸路竹貨、江淮以南江河魚利,皆弛其禁。諸處站赤飲食,官為支給。遣官諸路慮囚,罪輕者釋之。徙屯衛輝新附軍六千家,廩之京師,以完倉廩。發五衛軍及新附軍浚蒙村漕渠。庚辰,立別十八里驛傳。毀宋郊天臺。桑哥言:「楊璉真加雲,會稽有泰寧寺,宋毀之以建寧宗等攢宮;錢唐有龍華寺,宋毀之以為南郊。皆勝地也,宜復為寺,以為皇上、東宮祈壽。」時寧宗等攢宮已毀建寺,敕毀郊天臺,亦建寺焉。壬午,詔立市舶都轉運司。立上都等路群牧都轉運使司、諸路常平鹽鐵坑冶都轉運司。甲申,遣使代祀五岳、四瀆、東海、后土。戊子,闊闊你敦言:「先有旨遣軍二千屯田芍陂,試土之肥磽,去秋已收米二萬餘石,請增屯士二千人。」從之。徙江南樂工八百家於京師。封駙馬唆郎哥為寧昌郡王,賜龜紐銀印。西川趙和尚自稱宋福王子廣王以誑民,民有信者;真定民劉驢兒有三乳,自以為異,謀不軌;事覺,皆磔裂以徇。移五條河屯田軍五百於兀失蠻、札失蠻。辛卯,發諸衛軍六千八百人給護國寺修造。廣御史臺贓罰庫。癸巳,樞密臣言:「舊制四宿衛各選一人參決樞密院事,請以脫列伯為僉院。」從之。詔括京師荒地,令宿衛士耕種。乙未,中書省臣請以御史大夫玉速帖木兒為左丞相,中書撒里蠻為御史大夫。罷行御史臺,以其所屬按察司隸御史臺,行御史臺大夫撥魯罕為中書省平章政事。帝曰:「玉速帖木兒朕當思之,撥魯罕寬緩,不可。」安童對曰:「阿必赤合何如?」帝曰:「此事朕自處之。罷行御史臺者,當如所奏。」盧世榮請罷福建行中書省,立宣慰司,隸江西行中書省。又言:「江南行中書省事繁,恐致壅滯,今隨行省立行樞密院總兵,以分其務為便。」帝曰:「行院之事,前日已言,由阿合馬欲其子忽辛兼兵柄而止,今議行之。」流徵占城擅還將帥二十三人於遠方。丙申,帝畋於近郊。升武備監為武備寺,尚醫監為太醫院,職俱三品,升六部為二品。以合必赤合為中書平章政事,命禮部領會同館。初,外國使至,常令翰林院主之,至是改正。荊湖占城行省平叛蠻百六十六洞。詔禁私酒。己亥,分江浙行省所治南康隸江西省。辛丑,以楊兀魯帶為征骨嵬招討使,佩二珠虎符。壬寅,造大樽於殿,樽以木為質,銀內而金外,鏤為雲龍,高一丈七寸。是月壬午,烏馬兒領兵與安南興道王遇,擊敗之,兵次富良江北。乙酉,安南世子陳日烜領戰船千餘艘以拒。丙戌,與戰,大破之,日烜遁去,入其城。還屯富良江北,唆都、唐古帶等引兵與鎮南王會。



 二月乙巳,駐蹕柳林。增濟州漕舟三千艘,役夫萬二千人。初,江淮歲漕米百萬石於京師,海運十萬石,膠、萊六十萬石,而濟之所運三十萬石,水淺舟大,恆不能達,更以百石之舟,舟用四人,故夫數增多。塞渾河堤決,役夫四千人。詔改江淮、江西元帥招討司為上中下三萬戶府,蒙古、漢人、新附諸軍相參,作三十七翼。上萬戶:宿州、蘄縣、真定、沂郯、益都、高郵、沿海七翼;中萬戶:棗陽、十字路、邳州、鄧州、杭州、懷州、孟州、真州八翼;下萬戶:常州、鎮江、潁州、廬州、亳州、安慶、江陰水軍、益都新軍、湖州、淮安、壽春、揚州、泰州、弩手、保甲、處州、上都新軍、黃州、安豐、松江、鎮江水軍、建康二十二翼。翼設達魯花赤、萬戶、副萬戶各一人,以隸所在行院。江西盜黎德等餘黨悉平。以應放還五衛軍穿河西務河。舊例,五衛軍十人為率,七人三人,分為二番,十月放七人者還,正月復役,正月放三人者還,四月復役,更休息之。丙午,以荊湖行省所隸八番、羅甸隸西川行省,分嵐、管為二州。加封桑乾河神洪濟公為顯應洪濟公。己酉,為皇孫阿難答立衍福司,職四品,使、同知、副使各一員。辛亥,廣東宣慰使月的迷失討潮、惠二州盜郭逢貴等四十五寨,皆平,降民萬餘戶、軍三千六百一十人,請將所獲渠帥入覲,面陳事宜,從之。丙辰,詔罷膠、萊所鑿新河,以軍萬人隸江浙行省習水戰,萬人載江淮米泛海由利津達於京師。辛酉,御史臺臣言:「近中書奏罷行御史臺,改按察司為提刑轉運司,俾兼錢穀,而糾彈之職廢矣。請令安童與老臣議。」從之。壬戌,太陰犯心。中書省臣盧世榮請立規措所,經營錢穀,秩五品,所用官吏以善賈為之,勿限白身人,帝從之。參知政事不魯迷失海牙等因奏世榮姻黨有牛姓者,前為提舉,今浙西運司課程頗多,擬升轉運副使,亦從之。詔舊城居民之遷京城者,以貲高及居職者為先,仍定制以地八畝為一分;其或地過八畝及力不能作室者,皆不得冒據,聽民作室。升御帶庫為章佩監。徙右千戶只兒海迷失分地泉州。賜合剌失都兒新附民五千戶,合剌赤、阿速、阿塔赤、昔寶赤、貴由赤等嘗從征者,亦皆賜之。以民八十戶賜皇太子宿衛臣嘗從征者。用盧世榮言,回買江南民土田。詔天下拘收銅錢。申禁私造酒曲。戊辰,車駕幸上都。帝問省臣:「行御史臺何故罷之?」安童曰:「江南盜賊屢起,行御史臺鎮遏居多,臣以為不可罷。然與江浙行中書省並在杭州,地甚遠僻,徙之江州,居江浙、湖南、江西三省之中為便。」從之。立真定、濟南、太原、甘肅、江西、江淮、湖廣等處宜慰司兼都轉運使司,以治課程,仍立條制。禁諸司不得擅追管課官吏,有敢沮擾者,具姓名以聞。增濟州漕運司軍萬二千人。立江西、江淮、湖廣造船提舉司。令江浙行省參政馮珪,湖廣行省右丞要束木、參政潘傑,龍興行省左丞伯顏、參政楊居寬、簽省陳文福,專領課程事。以甕吉剌帶為中書左丞相。己巳,復立按察司。撥民二萬七千戶與駙馬唆郎哥。以忽都魯為平章政事。詔:「各道提刑按察司,能遵奉條畫,蒞事有成者,任滿升職,贓污不稱任者,罷黜除名。」詔立供膳司,職從五品,達魯花赤、令、丞各一員。罷融州總管府為州。



 三月丙子,遣太史監候張公禮、彭質等往占城測候日晷。癸未,罷甘州行中書省,立宣慰司,隸寧夏行中書省。荊湖占城行省請益兵,時陳日烜所逃天長、長安二處兵力復集,興道王船千餘艘聚萬劫,阮盝在永平,而官兵遠行久戰,懸處其中,唆都、唐古帶之兵又不以時至,故請益兵。帝以水行為危,令遵陸以往。庚子,詔依舊制,凡鹽一引四百斤,價銀十兩,以折今鈔為二十貫,商上都者,六十而稅一。增契本為三錢。立上都規措所回易庫,增壞鈔工墨費每貫二分為三分。



 夏四月癸卯,立行樞密院都鎮撫司,置畏兀驛六所。丙午,以征日本船運糧江淮及教軍水戰。庚戌,監察御史陳天祥劾中書右丞盧世榮罪惡,詔世榮、天祥皆赴上都。壬子,江陵民張二妻鄧氏一產三男。癸丑,詔追捕宋廣王及陳宜中。遣中書省、樞密院、御史臺官各一員,決大都及諸路罪囚。大都、汴梁、益都、廬州、河間、濟寧、歸德、保定蝗。辛酉,以耽羅所造征日本船百艘賜高麗。壬戌,御史中丞阿剌帖木兒、郭佑,侍御史白禿剌帖木兒,參知政事撒的迷失等以盧世榮所招罪狀奏。阿剌帖木兒等與世榮對於帝前,世榮悉款服。改六部依舊為三品。詔:「安童與諸老臣議世榮所行,當罷者罷之,更者更之,其所用人實無罪者,朕自裁決。」癸亥,敕以麥術丁所行清潔,與安童治省事。



 五月甲戌,以御史中丞郭佑為中書省參知政事。丁丑,減上都商稅。戊寅,廣平、汴梁、鈞、鄭旱。以遠方歷日取給京師,不以時至,荊湖等處四行省所用者隆興印之,合剌章、河西、西川等處所用者京兆印之。詔甘州每地一頃輸稅三石。壬午,以軍千人修阿失鹽場倉。以忻都為踢裡玉招討使,佩虎符,有旨:「不可興兵遠攻,近地有不服者討之。」右巴等洞蠻平。甲申,立汴梁宣慰司,依安西王故事,汴梁以南至江,以親王鎮之。丁亥,中書省臣言:「六部官冗甚,可止以六十八員為額,餘悉汰去。」詔擇其廉潔有幹局者存之。分漢地及江南所拘弓箭兵器為三等,下等毀之,中等賜近居蒙古人,上等貯於庫;有行省、行院、行臺者掌之,無省、院、臺者達魯花赤、畏兀、回回居職者掌之,漢人、新附人雖居職無有所預。戊子,改升江、烏定、朵裏滅該等府為路。雲南行省臣脫帖木兒言蠲逋賦、征侵隱、戍叛民、明黜陟、罷轉運、給親王、賦豪戶、除重稅、決盜賊、增驛馬、取質子、定俸祿、教農桑、優學者、恤死事、捕逃亡十餘事,命中書省議其可者行之。庚寅,真定、廣平、河間、恩州、大名、濟南蠶災。增大都諸門尉、副各一人。敕朵兒只招集甘、沙、速等州流徙饑民。行御史臺復徙於杭州。丁酉,徙行樞密院於建康。戊戌,汴梁、懷孟、濮州、東昌、廣平、平陽、彰德、衛輝旱。罷江南造船提舉司。陳日烜走海港,鎮南王命李恆追襲,敗之。適暑雨疫作,兵欲北還思明州,命唆都等還烏里。安南以兵追躡,唆都戰死;恆為後距,以衛鎮南王,藥矢中左膝,至思明,毒發而卒。



 六月庚戌,命女直、水達達造船二百艘及造征日本迎風船。辛亥,揚州進芝草。丙辰,遣馬速忽、阿里齎鈔千錠往馬八圖求奇寶,賜馬速忽虎符,阿裏金符。高麗遣使來貢方物。庚午,詔減商稅,罷牙行,省市舶司入轉運司。左丞呂師夔乞假五月,省母江州,帝許之,因諭安童曰:「此事汝蒙古人不知,朕左右復無漢人,可否皆自朕決。汝當盡心善治百姓,無使重困致亂,以為朕羞。」參知政事張德潤獻其家人四百戶於皇太子。馬湖部田鼠食稼殆盡,其總管祠而祝之,鼠悉赴水死。



 秋七月壬申,造溫石浴室及更衣殿。癸酉,詔禁捕獵。甲戌,敕秘書監修《地理志》。乙亥,安南降者昭國王、武道、文義、彰憲、彰懷四侯赴闕。戊寅,京師蝗。分甘州屯田新附軍三百人,田於亦集乃之地。己卯,以米千石廩甕吉剌貧民。壬午,陜西四川行中書省左丞汪惟正入見。甲申,改闊里吉思等所平大小十溪洞悉為府、州、縣。修汴梁城。丁亥,廣東宣慰使月的迷失入覲,以所降渠帥郭逢貴等至京師,言山寨降者百五十餘所。帝問:「戰而後降邪,招之即降邪?」月的迷失對曰:「其首拒敵者臣已磔之矣,是皆招降者也。」因言:「塔術兵後未嘗撫治其民,州縣官復無至者,故盜賊各據土地,互相攻殺,人民漸耗,今宜擇良吏往治之。」從之。庚寅,樞密院言:「鎮南王脫歡所總征交趾兵久戰力疲,請於奧魯赤等三萬戶分蒙古軍千人,江淮、江西、荊湖三行院分漢軍、新附軍四千人,選良將將之,取鎮南王脫歡、阿里海牙節制,以征交趾。」從之。復以唐兀帶為荊湖行省左丞。唐兀帶請放征交趾軍還家休憩,詔從脫歡、阿里海牙處之。給諸王阿只吉分地貧民農具牛種,令自耕播。乙未,雲南行省言:「今年未暇征緬,請收獲秋禾,先伐羅北甸等部。」從之。庚子,改開、達、梁山三州隸夔州路。給鈔萬二千四百錠為本,取息以贍甘、肅二州屯田貧軍。



 八月辛酉,命有司祭斗三日。戊申,分四川鎮守軍萬人屯田成都。丙辰,車駕至自上都。己未,詔復立泉府司,秩從二品,以答失蠻領之。初,和禮霍孫以泉府司商販者,所至官給飲食,遣兵防衛,民實厭苦不便,奏罷之。至是,答失蠻復奏立之。丙寅,遣蒙古軍三千人屯田清、滄、靖海。戊辰,罷禁海商。省合剌章、金齒二宣撫司為一,治永昌。立臨安廣西道宣撫司。中書省臣奏:「近奉旨括江淮水手,江淮人皆能游水,恐因此動搖者眾。」從之。罷榷酤。初,民間酒聽自造,米一石官取鈔一貫。盧世榮以官鈔五萬錠立榷酤法,米一石取鈔十貫,增舊十倍。至是罷榷酤,聽民自造,增課鈔一貫為五貫。敕拘銅錢,餘銅器聽民仍用。令福建黃華畬軍有恆產者為民,無恆產與妻子者編為守城軍。汪惟正言鞏昌軍民站戶並諸人奴婢,因饑歲流入陜西、四川者,彼即括為軍站。帝曰:「信如所言,當鳩集與之,如非己有而強欲得之者,豈彼於法不懼邪?」



 九月乙亥,聽民自實兩淮荒地,免稅三年。中書省以江北諸城課程錢糧聽杭、鄂二行省節制,道途迂遠,請改隸中書,從之。永昌、騰沖二城在緬國、金齒間,摧圮不可禦敵,敕修之。敕:「自今貢物惟地所產,非所產者毋輒上。」丙子,真蠟、占城貢樂工十人及藥材、鱷魚皮諸物。辛巳,收集工匠之隱匿者。丙戌,速木都剌、馬答二國遣使來朝。庚寅,敕征交趾諸軍,除留蒙古軍百、漢軍四百為鎮南王脫歡宿衛,餘悉遣還,別以江淮行樞密院所總蒙古兵戍江西。癸巳,雲南貢方物。烏蒙叛,命四川行院也速帶兒將兵討之,馬湖總管汝作以蠻軍三百為助。降西崖門酋長阿者等百餘戶。



 冬十月己亥,以鈔五千錠和糴於應昌府。復分河間、山東鹽課轉運司為二。遣合撒兒海牙使安南。遣雪雪的斤領畏兀兒戶一千戍合剌章。庚子,享於太廟。甲辰,修南嶽廟。乙巳,樞密院臣言:「脫脫木兒遣使言,阿沙、阿女、阿則三部欲叛,宜遣人往召,如不至,乘隙伐之。」不允,因敕諭之:「事不議於雲南王也先帖木兒者,毋輒行。」詔征東招討使塔塔兒帶、揚兀魯帶以萬人征骨嵬,因授揚兀魯帶三珠虎符,為征東宣慰使都元帥。壬子,長葛、郾城各進芝草。癸丑,立征東行省,以阿塔海為左丞相,劉國傑、陳巖並左丞,洪茶丘右丞,征日本。賜脫里察安、答即古阿散等印,令考核中書省,其制如三品。丙辰,以參議帖木兒為參知政事,位郭佑上,且命之曰:「自今之事,皆責於汝。」馬法國入貢。戊午,以江淮行省平章忙兀帶為江浙省左丞相。初,西川止立四路,阿合馬濫用官,增而為九,臺臣言其地民少,留廣元、成都、順慶、重慶、夔府五路,餘悉罷去。後以山谷險要,蠻夷雜處,復置嘉定路、敘州宣撫司以控制之。升大理寺為都護府,職從二品。都護府言合剌禾州民饑,戶給牛二頭、種二石,更給鈔一十一萬六千四百錠,糴米六萬四百石,為四月糧賑之。癸亥,以答即古阿散理算積年錢穀,別置司署,與省部敵,干擾政務,並入省中。丁卯,敕樞密院計膠、萊諸處漕船,高麗、江南諸處所造海舶,括傭江淮民船,備征日本。仍敕習泛海者,募水工至千人者為千戶,百人為百戶。塔海弟六十言:「今百姓及諸投下民,俱令造船於女直,而女直又復發為軍,工役繁甚。乃顏、勝納合兒兩投下鷹坊、採金等戶獨不調。」有旨遣使發其民。烏蒙蠻夷宣撫使阿蒙叛,詔止征羅必丹兵,同雲南行省出兵討之。郭佑言:「自平江南,十年之間,凡錢糧事八經理算。今答即古阿散等又復鉤考,宜即罷去。」帝嘉納之。十一月己巳朔,廣東宣慰使月的迷失以英德、循、梅三路民少,請改為州,又請以管軍總管於躍為惠州總管,蔚州知州木八剌為潮州達魯花赤。帝疑其專,不允。御史臺臣言:「御史臺、按察司以糾察百官為職,近鉤校錢穀者恐發其奸,私聚群不逞之徒,欲沮其事,願陛下依舊制諭之。」制曰:「可。」庚午,賜皇子愛牙赤銀印。壬申,以討日本,遣阿八剌督江淮行省軍需,遣察忽督遼東行省軍需。甲戌,置合剌章、四川、建都等驛。戊寅,遣使告高麗發兵萬人、船六百五十艘,助征日本,仍令於近地多造船。己丑,籍重慶府不花家人百二十三戶為民。御史臺臣奏:「昔宋以無室家壯士為鹽軍,數凡五千,今存者一千一百二十二人,性習兇暴,民患苦之,宜給以衣糧,使屯田自贍。」詔議行之。癸巳,敕漕江淮米百萬石,泛海貯於高麗之合浦,仍令東京及高麗各貯米十萬石,備征日本。諸軍期於明年三月以次而發,八月會於合浦。乙未,以禿魯歡為參知政事,盧世榮伏誅。丙申,赦囚徒,黥其面,及招宋時販私鹽軍習海道者為水工,以征日本。



 十二月,敕減天下罪囚。以占城遁還忽都虎、劉九、田二復舊職,從征日本。增阿塔海征日本戰士萬人、回回砲手五十人。己亥,從樞密院請,嚴立軍籍條例,選壯士及有力家充軍。敕樞密院:「向以征日本故,遣五衛軍還家治裝,今悉選壯士,以正月一日到京師。」江淮行省以戰船千艘習水戰江中。辛丑,誅答即古阿散黨人蔡仲英、李蹊。丁未,皇太子薨。戊午,以中衛軍四千人伐木五萬八千六百,給萬安寺修造。己未,丹太廟楹。乙酉,立集賢院,以札里蠻領之。戊子,罷合剌章打金規運所及都元帥。敕合剌章酋長之子入質京師,千戶、百戶子留質云南王也先帖木兒所。中書省臣奏:「納速丁言,減合剌章冗官,可歲省俸金九百四十六兩;又屯田課程專人主之,可歲得金五千兩。」皆從之。遣只必哥等考核雲南行省。庚寅,詔毋遷轉工匠官。辛卯,敕有司祭北斗。



 是歲,命江浙轉運司通管課程。集諸路僧四萬於西京普恩寺,作資戒會七日夜。並省重慶等處州縣。占城行省參政亦黑迷失等以軍還,駐海外四州,遣使以聞,敕放其軍還。賜皇子脫歡,諸王阿魯灰、只吉不花,公主囊家真等,鈔計七千七百三十二錠、馬六百二十九匹、衣段百匹、弓千、矢二萬發。賜諸王阿只吉、合兒魯、忙兀帶、宋忽兒、阿沙、合丹、別合剌等及官戶散居河西者,羊馬價鈔三萬七千七百五十七錠、布四千匹、絹二千匹。以伯八剌等貧乏,給鈔七萬六千五百二錠。賞諸王阿只吉、小廝、汪總帥、別速帶、也先等所部及徵緬、占城等軍,鈔五萬三千五百四十一錠、馬八千一百九十七匹、羊一萬六千六百三十四、牛十一、米二萬二千一百石、絹帛八萬一千匹、綿五百三十斤、木綿二萬七千二百七十九匹、甲千被、弓千張、衣百七十九襲。命帝師也憐八合失甲自羅二思八等遞藏佛事於萬安、興教、慶壽等寺,凡一十九會。斷死罪二百七十一人。



\end{pinyinscope}