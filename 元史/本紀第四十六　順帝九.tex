\article{本紀第四十六 順帝九}

\begin{pinyinscope}

 二十一年春正月癸丑朔,詔赦天下。命中書參知政事七十往諭孛羅帖木兒罷兵還鎮,復遣使往諭察罕帖木兒,亦令罷兵。孛羅帖木兒縱兵掠冀寧等處,察罕帖木兒以兵拒之,故有是命。庚申,太陰犯歲星。乙丑,河南賊犯杞縣,察罕帖木兒討平之。丁卯,李思齊進兵平伏羌縣等處。癸酉,石州大風拔木,六畜俱鳴,民所持槍,忽生火焰,抹之即無,搖之即有。



 二月癸未朔,填星退犯太微垣。甲申,同僉樞密院事迭裏帖木兒復永平、灤州等處。己丑,察罕帖木兒駐兵霍州,攻孛羅帖木兒。壬寅,太陰犯天江。是月,江南行臺侍御史八撒剌不花殺廣東廉訪使完者篤、副使李思誠、僉事迭麥赤,以兵自衛,據廣州。時八撒剌不花以廉訪使久居廣東,專恣自用,詔乃以完者篤等為廉訪司官,而除八撒剌不花侍御史。八撒剌不花不受命,怒完者篤等代己,即誣以罪,盡殺之,惟廉訪使董鑰哀請得免。



 三月丙辰,太陰犯井宿。癸酉,察罕帖木兒調兵討永城縣,又駐兵宿州,擒賊將梁綿住。庚辰,熒惑犯鬼宿。是月,張士誠海運糧一十一萬石至京師。孛羅帖木兒罷兵還,遣脫列伯等引兵據延安,以謀入陜。張良弼出南山義穀,駐藍田,受節制於察罕帖木兒。良弼又陰結陜西行省平章政事定住,聽丞相帖裏帖木兒調遣,營於鹿臺。



 夏四月辛巳朔,日有食之。是月,以張良弼為陜西行省參知政事。察罕帖木兒遣其子副詹事擴廓帖木兒貢糧至京師,皇太子親與定約,遂不復疑。



 五月癸丑,四川明玉珍陷嘉定等路,李思齊遣兵擊敗之。壬戌,太陰犯房宿。癸酉,太白犯軒轅。甲戌,熒惑犯太白。乙亥,察罕帖木兒以兵侵孛羅帖木兒所守之地。是月,李思齊受李武、崔德等降。六月乙未,熒惑、歲星、太白聚於翼宿。丙申,察罕帖木兒總兵討山東,發晉軍,下井陘,出邯鄲,過磁、相、懷、衛,逾白馬津,發其軍之在汴梁者繼之,水陸並進。戊戌,太陰犯雲雨。甲辰,太白晝見。



 秋七月辛亥,察罕帖木兒平東昌。己巳,沂州西北有赤氣蔽天如血。



 是月,察罕帖木兒進兵復冠州。八月乙酉,大同路北方夜有赤氣蔽天,移時方散。庚子,以福建行省平章政事普化帖木兒為江南行臺御史大夫。癸卯,大明兵取江州路。時偽漢陳友諒據江州為都,至是退都武昌。是月,察罕帖木兒遣其子擴廓帖木兒、閻思孝等,會關保、虎林赤等,將兵由東河造浮橋以濟,賊以二萬餘眾奪之,關保、虎林赤且戰且渡,拔長清,討東平,東平偽丞相田豐遣崔世英等出戰,大破之。乃遣使招諭田豐,豐降,東平平,令豐為前鋒,從大軍東討。棣州俞寶降,東平王士誠、東昌楊誠等皆降,魯地悉定。進兵濟南,劉珪降,遂圍益都。九月戊午,陽翟王阿魯輝帖木兒伏誅。阿魯輝帖木兒以宗親,見天下盜賊並起,遂乘間隙,肆為異圖,詔少保、知樞密院事老章率諸軍討之。老章遂敗其眾,尋為部將同知太常禮儀院事脫歡所擒,送闕下,詔誅之。於是詔加老章太傅、和寧王,以阿魯輝帖木兒之弟忽都帖木兒襲封陽翟王。宗王囊加、玉樞虎兒吐華與脫歡悉議加封。壬戌,四川賊兵陷東川郡縣,李思齊調兵擊之。壬申,命孛羅帖木兒於保定以東,河間以南,從便屯種。是月,命兵部尚書徹徹不花、侍郎韓祺征海運糧於張士誠。大明取建昌、饒州二路。



 冬十月癸巳,絳州有赤氣見北方如火。以察罕帖木兒為中書平章政事,兼知河南、山東等處行樞密院事、陜西行御史臺中丞。察罕帖木兒調參知政事陳秉直、劉珪等守禦河南。



 十一月戊申朔,溫州樂清縣雷。庚戌,太陰犯建星。癸亥,太陰犯井宿。戊辰,黃河自平陸三門磧下至孟津,五百餘里皆清,凡七日。命秘書少監程徐祀之。壬申,太陰犯氐宿。是月,察罕帖木兒、李思齊遣兵圍鹿臺,攻張良弼,詔和解之,俾各還信地,兵乃解。是歲,京師大饑,屯田成,收糧四十萬石。賜司農丞胡秉彞尚尊、金幣,以旌其功。



 二十二年春正月戊申朔,太白犯建星。甲寅,詔李思齊討四川,張良弼平襄漢。時兩軍不和,故有是命。乙卯,填星退犯左執法。庚申,大明取江西龍興諸路。時江西諸路皆陳友諒所據。丁卯,詔以太尉完者帖木兒為陜西行省左丞相。仍命察罕帖木兒屯種於陜西。申諭李思齊、張良弼等各以兵自效。以也先不花為中書右丞。



 二月丁丑朔,盜殺陜西行省右丞塔不歹。己卯,太白犯壘壁陣。乙酉,彗星見於危宿,光芒長丈餘,色青白。丁酉,彗星犯離宮西星,至二月終,光芒長二丈餘。是月,知樞密院事禿堅帖木兒奉詔諭李思齊討四川。時思齊退保鳳翔,使至,思齊進兵益門鎮;使還,思齊復歸鳳翔。



 三月戊申,彗星不見星形,惟有白氣,形曲竟天,西指掃大角。壬子,彗星行過太陽前,惟有星形,無芒,在昴宿,至戊午始滅。甲寅,四川明玉珍陷雲南省治,屯金馬山,陜西行省參知政事車力帖木兒等擊敗之,擒明玉珍弟明二。己未,御史大夫老的沙辭職,不許。是月,命孛羅帖木兒為中書平章政事,位第一,加太尉。張良弼受節制於孛羅帖木兒。李思齊遣兵攻良弼,至於武功,良弼以伏兵大破之。



 夏四月丙子朔,長星見,其形如練,長數十丈,在虛、危之間,後四十餘日乃滅。丁亥,熒惑離太陽三十九度,不見,當出不出。己丑,詔諸王、駙馬、御史臺各衙門,不許占匿人民不當差役。乙未,賊新橋張陷安州,孛羅帖木兒來請援兵。是月,紹興路大疫。



 五月乙巳朔,泉州賽甫丁據福州路,福建行省平章政事燕只不花擊敗之,餘眾航海還據泉州。福建行省參知政事陳有定復汀州路。己未,中書參知政事陳祖仁上章,乞罷修上都宮闕。辛酉,太陰犯建星。辛未,明玉珍據成都,自稱隴蜀王,遣偽將楊尚書守重慶,分兵寇龍州、青州,犯興元、鞏昌等路。是月,張士誠海運糧一十三萬石至京師。六月辛巳,彗星見紫微垣,光芒長尺餘,東南指,西南行。戊子,彗星光芒掃上宰。田豐及王士誠刺殺察罕帖木兒,遂走入益都城,眾乃推察罕帖木兒之子擴廓帖木兒為總兵官,復圍益都。詔贈察罕帖木兒推誠定遠宣忠亮節功臣、開府儀同三司、上柱國、河南行省左丞相,追封忠襄王,謚獻武,食邑沈丘縣;令河南、山東等處立廟,長吏歲時致祭。其父司徒阿都溫賜良田二百頃;其子擴廓帖木兒授光祿大夫、中書平章政事,兼知河南山東等處行樞密院事、同知詹事院事,一應軍馬,並聽節制。仍詔諭其將士曰:「凡爾將佐,久為察罕帖木兒從事,惟恩與義,實同骨肉,視彼逆黨,不共戴天,當力圖報復,以伸大義。」己亥,益都賊兵出戰,擴廓帖木兒生擒六百餘人,斬首八百餘級。



 秋七月乙卯,彗星滅跡。丙辰,熒惑見西方,須臾,成白氣如長蛇,光炯有文,橫亙中天,移時乃滅。是月,河決範陽縣,漂民居。八月己亥,擴廓帖木兒言:「孛羅帖木兒、張良弼據延安,掠黃河上下,欲東渡以奪晉寧,乞賜詔諭。」癸巳,太陰犯畢宿。九月癸卯朔,劉福通以兵援田豐,至火星埠,擴廓帖木兒遣關保邀擊,大破之。甲辰,以山北廉訪司權置於惠州。丁未,太白犯亢宿。己酉,太陰犯鬥宿。癸亥,歲星犯軒轅。丙寅,熒惑犯鬼宿。戊辰,以也速為遼陽行省左丞相,依前總兵,撫安迤東郡縣。己巳,有流星如酒杯,色青白,光明燭地。熒惑犯鬼宿積尸氣。



 冬十月壬申朔,江西行省平章朵列不花移檄討八撒剌不花。時朵列不花分省廣州,適邵宗愚陷廣州,執八撒剌不花,殺之。甲戌,孛羅帖木兒南侵擴廓帖木兒所守之地,遂據真定路。己卯,太陰犯牛宿。丁亥,辰星犯亢宿。戊子,太陰犯畢宿。



 十一月乙巳,擴廓帖木兒復益都,田豐等伏誅。自擴廓帖木兒既襲父職,身率將士,誓必復仇,人心亦思自奮,圍城益急。賊悉力拒守,乃以壯士穴地通道而入,遂克之,盡誅其黨,取田豐、王士誠之心以祭察罕帖木兒。庚戌,擴廓帖木兒遣關保復莒州,山東悉平。庚申,詔授擴廓帖木兒太尉、銀青榮祿大夫、中書平章政事、知樞密院事、太子詹事,便宜行事,襲總其父兵;將校、士卒,論賞有差;察罕帖木兒父阿魯溫進封汝陽王,察罕帖木兒改贈宣忠興運弘仁效節功臣,追封潁川王,改謚忠襄。癸亥,四川賊兵陷清州。



 十二月壬辰,太陰犯角宿。庚子,以中書平章政事佛家奴為御史大夫。是歲,樞密副使李士瞻上疏極言時政,凡二十條:一曰悔己過,以詔天下;二曰罷造作,以快人心;三曰御經筵,以講聖學;四曰延老成,以詢治道;五曰去姑息,以振乾剛;六曰開言路,以求得失;七曰明賞罰,以厲百司;八曰公選舉,以息奔競;九曰察近幸,以杜奸弊;十曰嚴宿衛,以備非常;十一曰省佛事,以節浮費;十二曰絕濫賞,以足國用;十三曰罷各宮屯種,俾有司經理;十四曰減常歲計置,為諸宮用度;十五曰招集散亡,以實八衛之兵;十六曰廣給牛具,以備屯田之用;十七曰獎勵守令,以勸農務本;十八曰開誠布公,以禮待籓鎮;十九曰分遣大將,急保山東;二十曰依唐廣寧故事,分道進取。先是薊國公脫火赤上言乞罷三宮造作,帝為減軍匠之半,還隸宿衛,而造作如故,故士瞻疏首及之。皇太子嘗坐清寧殿,分布長席,列坐西番、高麗諸僧。皇太子曰:「李好文先生教我儒書多年,尚不省其義。今聽佛法,一夜即能曉焉。」於是頗崇尚佛學。帝以讒廢高麗王伯顏帖木兒,立塔思帖木兒為王。國人上書言舊王不當廢、新王不當立之故。初,皇后奇氏宗族在高麗,恃寵驕橫,伯顏帖木兒屢戒飭不悛,高麗王遂盡殺奇氏族。皇后謂太子曰:「爾年已長,何不為我報仇!」時高麗王昆弟有留京師者,乃議立塔思帖木兒為王,而以奇族子三寶奴為元子,以將作同知崔帖木兒為丞相,以兵萬人送之國,至鴨綠江,為高麗兵所敗,僅餘十七騎還京師。詔加封唐撫州刺史南庭王危全諷為南庭忠烈靈惠王。



 二十三年春正月壬寅朔,四川明玉珍僭稱皇帝,建國號曰大夏,紀元曰天統。乙巳,大寧陷。庚戌,歲星犯軒轅。



 二月戊戌,太白晝見。庚子,亦如之。是月,擴廓帖木兒自益都領兵還河南,留鎖住以兵守益都,以山東州縣立屯田萬戶府。



 三月辛丑朔,彗星見東方,經月乃滅。詔中書平章政事愛不花分省冀寧,擴廓帖木兒遣兵據之。丙午,大赦天下。丁未,親試進士六十二人,賜寶寶、楊牴進士及第,餘出身有差。丙辰,太陰犯氐宿。壬戌,大同路夜有赤氣亙天,中侵北斗。是月,立廣西行中書省,以廉訪使也兒吉尼為平章政事。時南方郡縣多陷沒,惟也兒吉尼獨保廣西者十五年。立膠東行中書省及行樞密院,總制東方事。以袁宏為參知政事。是春,關先生餘黨復自高麗還寇上都,孛羅帖木兒擊降之。



 夏四月辛丑,熒惑犯歲星。孛羅帖木兒、李思齊互相交兵。庚申,歲星犯軒轅。是月,擴廓帖木兒遣貊高等以兵擊張良弼。



 五月己巳朔,張士誠海運糧十三萬石至京師。壬午,太白晝見。甲午,亦如之。乙未,熒惑犯右執法。是月,爪哇遣使淡蒙加加殿進金表,貢方物。六月戊戌朔,孛羅帖木兒遣方脫脫迎匡福於彰德,擴廓帖木兒遣兵追之,敗還。匡福遂據保定路。己亥,擴廓帖木兒部將歹驢等駐兵藍田、七盤,李思齊攻圍興平,遂據盩厔。孛羅帖木兒時奉詔進討襄漢,而歹驢阻道於前,思齊踵襲於後,乃請催督擴廓帖木兒東出潼關,道路既通,即便南討。戊申,孛羅帖木兒遣竹貞等入陜西,據其省治。時陜西行省右丞答失鐵木兒與行臺有隙,且恐陜西為擴廓帖木兒所據,陰結於孛羅帖木兒,請竹貞入城,劫御史大夫完者帖木兒及監察御史張可遵等印。其後屢使召完者帖木兒,貞拘留不遣。擴廓帖木兒遣部將貊高與李思齊合兵攻之,竹貞出降,遂從擴廓帖木兒。庚戌,星隕於濟南龍山,入地五尺。甲寅,詔授江南下第及後期舉人為路、府、州儒學教授。乙卯,太白犯井宿。丁巳,絳州有白虹二道,沖鬥牛間。庚申,平陽路有白氣三道,一貫北極,一貫北斗,一貫天漢,至夜分乃滅。壬戌,太白晝見,夜犯井宿。



 秋七月戊辰朔,京師大雹,傷禾稼。丁丑,以馬良為中書參知政事。乙酉,太白晝見。有星墜於慶元路西北,聲如雷,光芒數十丈,久之乃滅。



 八月丁酉朔,倭人寇蓬州,守將劉暹擊敗之。自十八年以來,倭人連寇瀕海郡縣,至是海隅遂安。辛丑,擴廓帖木兒遣兵侵孛羅帖木兒所守之境。壬寅,太白犯軒轅。乙巳,太陰犯建星。丁未,太白犯軒轅。己酉,太白晝見。丙辰,太陰犯畢宿。沂州有赤氣亙天,中有白色如蛇形,徐徐西行,至夜分乃滅。戊午,孛羅帖木兒言:「擴廓帖木兒踵襲父惡,有不臣之罪,乞賜處置。」己未,太白晝見。辛酉,太白犯歲星。乙丑,太白犯右執法。是月,大明兵與偽漢兵大戰於鄱陽湖,陳友諒敗績而死。其子理自立,仍據武昌為都,改元德壽,大明兵遂進圍武昌。九月丁卯朔,遣爪哇使淡蒙加加殿還國,詔賜其國主三珠金虎符及織金紋幣。辛未,太白犯左執法。乙亥,歲星犯右執法。丁丑,辰星犯填星,丁亥,太白犯填星。辰星犯亢宿。是月,張士誠自稱吳王,來請命,不報。遣戶部侍郎博羅帖木兒等征海運於張士誠,士誠不與。



 冬十月丙申朔,青齊一方赤氣千里。癸卯,太白犯氐宿。甲辰,湖廣偽姚平章、張知院陰遣人言於擴廓帖木兒,設計擒殺偽漢主陳理及偽夏主明玉珍,不果。己酉,監察御史米只兒海牙劾奏太傅太平罪狀,詔安置太平於陜西之西,仍拘收宣命並御賜等物。戊午,太白犯房宿。是月,擴廓帖木兒遣僉樞密院事任亮復安陸府。孛羅帖木兒遣兵攻冀寧,至石嶺關,擴廓帖木兒大破走之,擒其將烏馬兒、殷興祖。孛羅帖木兒軍由是不振。



 十一月壬申,御史臺臣言:「故右丞相脫脫有大臣之體,向在中書,政務修舉,深懼滿盈,自求引退,加封鄭王,固辭不受。再秉鈞軸,克濟艱危,統軍進徵,平徐州,收六合,大功垂成,浮言構難,奉詔謝兵,就貶以沒。已蒙錄用其子,還所籍田宅,更乞憫其勛舊,還其所授宣命。」從之。癸未,太陰犯軒轅,歲星犯左執法。是歲,御史大夫老的沙與知樞密院事禿堅帖木兒,得罪於皇太子,皆奔大同,孛羅帖木兒匿之營中。



 二十四年春正月戊寅,太陰犯軒轅。庚辰,保德州民家產豬一頭兩身。



 二月壬子,歲星犯右執法。癸丑,太陰犯西咸池。是月,大明滅偽漢,其所據湖南北、江西諸郡皆降於大明。



 三月乙亥,監察御史王朵列禿、崔卜顏帖木兒等諫皇太子勿親征。辛卯,詔以孛羅帖木兒匿老的沙,謀為悖逆,解其兵權,削其官爵,候道路開通,許還四川田里。孛羅帖木兒拒命不受。



 夏四月甲午朔,命擴廓帖木兒討孛羅帖木兒。乙未,太陰犯西咸池。孛羅帖木兒悉知詔令調遣之事非出帝意,皆右丞相搠思監所為,遂令禿堅帖木兒舉兵向闕。壬寅,禿堅帖木兒兵入居庸關。癸卯,知樞密院事也速、詹事不蘭奚迎戰於皇后店。不蘭奚力戰,也速不援而退,不蘭奚幾為所獲,脫身東走。甲辰,皇太子率侍衛兵出光熙門,東走古北口,趨興、松。乙巳,禿堅帖木兒兵至清河列營。時都城無備,城中大震,令百官吏卒分守京城,使達達國師至其軍問故,以必得搠思監及宦官樸不花為對,詔慰解之,不聽。丁未,詔屏搠思監於嶺北,竄樸不花於甘肅,執而與之。復孛羅帖木兒前官,仍總兵。以也速為左丞相。庚戌,禿堅帖木兒陳兵自健德門入,覲帝於延春閣,慟哭請罪,帝就宴賚之。加孛羅帖木兒太保,依前守禦大同,禿堅帖木兒為中書平章政事。辛亥,禿堅帖木兒軍還。皇太子至路兒嶺。詔追及之,還宮。癸丑,太白犯井宿。



 五月甲子朔,黃河清。戊辰,擴廓帖木兒奉命討孛羅帖木兒,屯兵冀寧,其東道以白鎖住領兵三萬,守禦京師,中道以貊高、竹貞領兵四萬,西道以關保領軍五萬,合擊之。關保等兵逼大同,孛羅帖木兒留兵守大同,而自率兵與禿堅帖木兒、老的沙復大舉向闕。甲戌,太白犯鬼宿。乙亥,又犯積尸氣,歲星犯右執法。六月癸卯,三星晝見,白氣橫突其中。甲辰,河南府有大星夜見南方,光如晝。丁未,大星隕,照夜如晝,及旦,黑氣晦暗如夜。甲寅,白鎖住以兵至京師,請皇太子西行。丁巳,太白犯右執法。是月,保德州黃龍見井中。



 秋七月癸亥,太白與歲星合於翼宿。甲子,歲星犯左執法。丙戌,孛羅帖木兒前鋒軍入居庸關,皇太子親率軍御於清河,也速軍於昌平,軍士皆無鬥志。皇太子馳還都城,白鎖住引兵入平則門。丁亥,白鎖住扈從皇太子出順承門,由雄、霸、河間,取道往冀寧。戊子,孛羅帖木兒駐兵健德門外,與禿堅帖木兒、老的沙入見帝於宣文閣,訴其非罪,皆泣,帝亦泣,乃賜宴。孛羅帖木兒欲追襲皇太子,老的沙止之。庚寅,詔以孛羅帖木兒為中書左丞相,老的沙為中書平章政事,禿堅帖木兒為御史大夫,其部屬布列省臺百司。以也速知樞密院事。詔諭:「孛羅帖木兒、擴廓帖木兒俱朕股肱,視同心膂,自今各棄宿忿,弼成大勛。」是月,大明兵取廬州路。



 八月壬辰朔,日有食之。乙未,熒惑犯鬼宿。壬寅,詔以孛羅帖木兒為中書右丞相、監修國史,節制天下軍馬。乙巳,皇太子至冀寧。乙卯,張士誠自以其弟士信代達識帖睦邇為江浙行省左丞相。是月,孛羅帖木兒請誅狎臣禿魯帖木兒、波迪哇兒祃,罷三宮不急造作,沙汰宦官,減省錢糧,禁止西番僧人好事。九月辛酉朔,宦官思龍宜潛送宮女伯忽都出自順承門,以達於皇太子。乙丑,太白晝見。癸酉,夜,天西北有紅光,至東而散。甲申,太陰犯軒轅。是月,大明兵取中興及歸、峽、潭、衡等路。



 冬十月丙午,太陰犯畢宿。己酉,太陰犯井宿。己未,詔皇太子還京師。命也速、老的沙分道總兵。



 十二月乙卯,太陰犯太白。



 二十五年春正月癸亥,封李思齊為許國公。丙寅,太白晝見。戊辰,亦如之。己巳,大明兵取寶慶路,守將唐隆道遁走。偽漢守將熊天瑞以贛州及韶州、南雄降於大明。甲戌,太白犯建星。壬午,監察御史孛羅帖木兒、賈彬等辯明哈麻、雪雪之罪。



 二月辛丑,汴梁路見日傍有一月一星。丙午,太陰犯填星。戊午,皇太子在冀寧,命甘肅行省平章政事朵兒只班以岐王阿剌乞兒軍馬,會平章政事臧卜、李思齊,各以兵守寧夏。



 三月庚申,皇太子下令於擴廓帖木兒軍中曰:「孛羅帖木兒襲據京師,餘既受命總督天下諸軍,恭行顯罰,少保、中書平章政事擴廓帖木兒,躬勒將士,分道進兵,諸王、駙馬及陜西平章政事李思齊等,各統軍馬,尚其奮義戮力,克期恢復。」丙寅,孛羅帖木兒幽置皇后奇氏於諸色總管府。丁卯,命老的沙、別帖木兒並為御史大夫。戊辰,太白犯壘壁陣。



 夏四月庚寅,孛羅帖木兒至諸色總管府見皇后奇氏,令還宮取印章,作書遺皇太子,遣內侍官完者禿持往冀寧,復出皇后,幽之。乙巳,關保等兵進圍大同。壬子,熒惑犯靈臺。乙卯,關保入大同。



 五月辛酉,熒惑犯太微垣。甲子,京師天雨氂,長尺許,或言於帝曰:「龍絲也。」命拾而祀之。乙亥,大明兵破安陸府,守將任亮迎戰,被執。己卯,大明兵破襄陽路。是月,侯卜延答失奉威順王自雲南經蜀轉戰而出,至成州,欲之京師,李思齊俾屯田於成州。六月戊子朔,以黎安道為中書參知政事。辛丑,湖廣行省左丞周文貴復寶慶路。乙巳,皇后奇氏自幽所還宮。乙卯,以太尉火你赤為御史大夫。是月,皇太子加李思齊銀青榮祿大夫、邠國公、中書平章政事、皇太子詹事,兼四川行樞密院事、虎符招討使。分中書四部。



 秋七月丁丑,填星、歲星、熒惑聚於角、亢。己卯,太陰犯畢宿。乙酉,孛羅帖木兒伏誅,禿堅帖木兒、老的沙皆遁走。丙戌,遣使函孛羅帖木兒首往冀寧,召皇太子還京師。大赦天下。黎安道、方脫脫、雷一聲皆伏誅。是月,京師大水。河決小流口,達於清河。八月丁亥朔,京城門至是不開者三日。竹貞、貊高軍至城外,命軍士緣城而上,碎平則門鍵,悉以軍入,占民居,奪民財。乙未,太陰犯建星。己亥,太陰犯壘壁陣。癸卯,詔命皇太子分調將帥,戡定未復郡邑,即還京師,行事之際,承制用人,並準正授。丁未,皇後弘吉剌氏崩。壬子,以洪寶寶、帖古思不花、捏烈禿並為中書平章政事。九月,擴廓帖木兒扈從皇太子至京師。丁丑,太陰犯井宿。壬午,詔以伯撒里為太師、中書右丞相、監修國史;擴廓帖木兒為太尉、中書左丞相、錄軍國重事、同監修國史、知樞密院事,兼太子詹事。是月,以方國珍為淮南行省左丞相,分省慶元。



 冬十月辛卯,熒惑犯天江。壬寅,以哈剌章為知樞密院事。丁未,益王渾都帖木兒、樞密副使觀音奴擒老的沙,誅之。禿堅帖木兒以餘兵往八兒思之地,命嶺北行省左丞相山僧及知樞密院事魏賽因不花同討之。戊申,以資政院使禿魯為御史大夫。己酉,熒惑犯鬥宿。太陰犯右執法。庚戌,太陰犯太微垣。閏月庚申,以賓國公五十八為知樞密院事。詔張良弼、俞寶、孔興等悉聽調於擴廓帖木兒。戊辰,太白、辰星、熒惑聚於斗宿。太陰犯畢宿。辛未,詔封擴廓帖木兒河南王,代皇太子親征,總制關陜、晉冀、山東等處並迤南一應軍馬,諸王各愛馬應該總兵、統兵、領兵等官,凡軍民一切機務、錢糧、名爵、黜陟、予奪,悉聽便宜行事。壬申,太白犯辰星。辛巳,以脫脫木兒為中書右丞,達識帖木兒為參知政事。



 十一月己丑,太白犯熒惑,太陰犯壘壁陣。丙申,太陰犯畢宿。癸卯,太陰犯太微垣。是月,大明兵取泰州。時泰州、通州、高郵、淮安、徐州、宿州、泗州、濠州、安豐諸郡,皆張士誠所據。



 十二月乙卯,詔立次皇后奇氏為皇后,改奇氏為肅良合氏,詔天下,仍封奇氏父以上三世皆為王爵。癸亥,太陰犯畢宿。以帖林沙為中書參知政事。庚午,歲星掩房宿。辛未,太陰犯右執法。是月,禿堅帖木兒伏誅。



\end{pinyinscope}