\article{志第十六 河渠一}

\begin{pinyinscope}

 水為中國患,尚矣。知其所以為患,則知其所以為利,因其患之不可測而能先事而為之備,或後事而有其功,斯可謂善治水而能通其利者也。昔者禹堙洪水,疏九河,陂九澤,以開萬世之利,而《周禮·地官》之屬,所載瀦防溝遂之法甚詳。當是之時,天下蓋無適而非水利也。自先王疆理井田之制壞,而後水利之說興。魏史起鑿漳河,秦鄭國引涇水,漢鄭當時、王安世輩或獻議穿漕渠,或建策防水決,是數君子者,皆嘗試其術而卒有成功,太史公《河渠》一書猶可考。自時厥後,凡好事喜功之徒,率多為興利之言,而其患顧有不可勝言者矣。夫潤下,水之性也,而欲為之防,以殺其怒,遏其沖,不亦甚難矣哉。惟能因其勢而導之,可蓄則儲水以備旱之災,可洩則瀉水以防水潦之溢,則水之患息,而於是蓋有無窮之利焉。



 元有天下,內立都水監,外設各處河渠司,以興舉水利、修理河堤為務。決雙塔、白浮諸水為通惠河,以濟漕運,而京師無轉餉之勞;導渾河,疏灤水,而武清、平灤無墊溺之虞;浚冶河,障滹沱,而真定免決嚙之患。開會通河於臨清,以通南北之貨;疏陜西之三白,以溉關中之田;洩江湖之淫潦,立捍海之橫塘,而浙右之民得免於水患。當時之善言水利,如太史郭守敬等,蓋亦未嘗無其人焉。一代之事功,所以為不可泯也。今故著其開修之歲月,工役之次第,歷敘其事而分紀之,作《河渠志》。



 通惠河



 通惠河,其源出於白浮、甕山諸泉水也。世祖至元二十八年,都水監郭守敬奉詔興舉水利,因建言:「疏鑿通州至大都河,改引渾水溉田,於舊閘河蹤跡導清水,上自昌平縣白浮村引神山泉,西折南轉,過雙塔、榆河、一畝、玉泉諸水,至西水門入都城,南匯為積水潭,東南出文明門,東至通州高麗莊入白河,總長一百六十四里一百四步。塞清水口一十二處,共長三百一十步。壩閘一十處,共二十座,節水以通漕運,誠為便益。」從之。首事於至元二十九年之春,告成於三十年之秋,賜名曰通惠。凡役軍一萬九千一百二十九,工匠五百四十二,水手三百一十九,沒官囚隸百七十二,計二百八十五萬工,用楮幣百五十二萬錠,糧三萬八千七百石,木石等物稱是。役興之日,命丞相以下皆親操畚鍤為之倡。置閘之處,往往於地中得舊時磚木,時人為之感服。船既通行,公私兩便。先時通州至大都五十里,陸挽官糧,歲若干萬,民不勝其悴,至是皆罷之。



 其壩閘之名曰:廣源閘;西城閘二,上閘在和義門外西北一里,下閘在和義水門西三步;海子閘,在都城內;文明閘二,上閘在麗正門外水門東南,下閘在文明門西南一里;魏村閘二,上閘在文明門東南一里,下閘西至上閘一里;籍東閘二,在都城東南王家莊;郊亭閘二,在都城東南二十五里銀王莊;通州閘二,上閘在通州西門外,下閘在通州南門外;楊尹閘二,在都城東南三十里;朝宗閘二,上閘在萬億庫南百步,下閘去上閘百步。



 成宗元真元年四月,中書省臣言:「新開運河閘,宜用軍一千五百,以守護兼巡防往來船內奸宄之人。」從之。七月,工部言:「通惠河創造閘壩,所費不貲,雖已成功,全藉主守之人,上下照略修治。今擬設提領三員,管領人夫,專一巡護,降印給俸。其西城閘改名會川,海子閘改名澄清,文明閘仍用舊名,魏村閘改名惠和,籍東閘改名慶豐,郊亭閘改名平津,通州閘改名通流,河門閘改名廣利,楊尹閘改名溥濟。」



 武宗至大四年六月,省臣言:「通州至大都運糧河閘,始務速成,故皆用木,歲久木朽,一旦俱敗,然後致力,將見不勝其勞。今為永固計,宜用磚石,以次修治。」從之。後至泰定四年,始修完焉。



 文宗天歷三年三月,中書省臣言:「世祖時,開挑通惠河,安置閘座,全藉上源白浮、一畝等泉之水以通漕運。今各枝及諸寺觀權勢,私決堤堰,澆灌稻田、水碾、園圃,致河淺妨漕事,乞禁之。」奉旨:白浮、甕山直抵大都運糧河堤堰泉水,諸人毋挾勢偷決,大司農司、都水監可嚴禁之。



 壩河



 壩河,亦名阜通七壩。成宗大德六年三月,京畿漕運司言:「歲漕米百萬,全藉船壩夫力。自冰開發運至河凍時止,計二百四十日,日運糧四千六百餘石,所轄船夫一千三百餘人,壩夫七百三十,占役俱盡,晝夜不息。今歲水漲,沖決壩堤六十餘處,雖已修畢,恐霖雨沖圮,走洩運水,以此點視河堤淺澀低薄去處,請加修理。」自五月四日入役,六月十二日畢,深溝壩九處,計一萬五千一百五十三工。王村壩二處,計七百十三工;鄭村壩一處,計一千一百二十五工;西陽壩三處,計一千二百六十二工;郭村壩三處,計一千九百八十七工。千斯壩下一處,計一萬工;總用工三萬二百四十。



 金水河



 金水河,其源出於宛平縣玉泉山,流至和義門南水門入京城,故得金水之名。



 至元二十九年二月,中書右丞馬速忽等言:「金水河所經運石大河及高良河、西河俱有跨河跳槽,今已損壞,請新之。」是年六月興工,明年二月工畢。



 至大四年七月,奉旨引金水河水注之光天殿西花園石山前舊池,置閘四以節水。閏七月興工,九月成,凡役夫匠二十九,為工二千七百二十三,除妨工,實役六十五日。



 隆福宮前河



 隆福宮前河,其水與太液池通。英宗至治二年五月,奉敕云:「昔在世祖時,金水河濯手有禁,今則洗馬者有之。比至秋疏滌,禁諸人毋得污穢。」於是會計修浚,三年四月興工,五月工畢,凡役軍八百,為工五千六百三十五。



 海子岸



 海子岸,上接龍玉堂,以石甃其四周。海子一名積水潭,聚西北諸泉之水,流行入都城而匯於此,汪洋如海,都人因名焉。



 仁宗延祐六年二月,都水監計會前後,興元修舊石岸相接,凡用石三百五,各長四尺,闊二尺五寸,厚一尺,石灰三千斤,該三百五工,丁夫五十,石工十,九月五日興工,十一日工畢。



 至治三年三月,大都河道提舉司言:「海子南岸東西道路,當兩城要沖,金水河浸潤於其上,海子風浪沖嚙於其下,且道狹,不時潰陷泥濘,車馬艱於往來,如以石砌之,實永久之計也。」



 泰定元年四月,工部應副工物,七月興工,八月工畢,凡用夫匠二百八十七人。



 雙塔河



 雙塔河,源出昌平縣孟村一畝泉,經雙塔店而東,至豐善村,入榆河。至元三年四月六日,巡河官言:「雙塔河時將泛溢,不早為備,恐至潰決,臨期卒難措手。乃計會閉水口工物,開申都水監,創開雙塔河,未及堅久。今已及水漲之時,倘或決壞,走洩水勢,誤運船不便。」省準制國用司給所需,都水監差夫修治焉。凡合閉水口五處,用工二千一百五十五。



 盧溝河



 盧溝河,其源出於代地,名曰小黃河,以流濁故也。自奉聖州界流入宛平縣境,至都城四十里東麻穀,分為二派。



 太宗七年歲乙未八月敕:「近劉沖祿言:『率水工二百餘人,已依期築閉盧溝河元破牙梳口,若不修堤固護,恐不時漲水沖壞,或貪利之人盜決溉灌,請令禁之。』劉沖祿可就主領,毋致沖塌盜決,犯者以違制論,徒二年,決杖七十。如遇修築時,所用丁夫器具,應差處調發之。其舊有水手人夫內,五十人差官存留不妨。已委管領,常切巡視體究,歲一交番,所司有不應副者罪之。」



 白浮甕山



 白浮甕山,即通惠河上源之所出也。白浮泉水在昌平縣界,西折而南,經甕山泊,自西水門入都城焉。



 成宗大德七年六月,甕山等處看閘提領言:「自閏五月二十九日始,晝夜雨不止,六月九日夜半,山水暴漲,漫流堤上,沖決水口。」於是都水監委官督軍夫,自九月二十一日入役,至是月終輟工,實役軍夫九百九十三人。十一年三月,都水監言:「巡視白浮甕山河堤,崩三十餘里,宜編荊笆為水口,以洩水勢。」計修笆口十一處,四月興工,十月工畢。



 仁宗皇慶元年正月,都水監言:「白浮甕山堤,多低薄崩陷處,宜修治。」來春二月入役,八月修完,總修長三十七里二百十五步,計七萬三千七百七十三工。延祐元年四月,都水監言:「自白浮甕山下至廣源閘堤堰,多淤澱淺塞,源泉微細,不能通流,擬疏滌。」由是會計工程,差軍千人疏治。



 泰定四年八月,都水監言:「八月三日至六日,霖雨不止,山水泛溢,沖壞甕山諸處笆口,浸沒民田。」計料工物,移交工部關支修治。自八月二十六日興工,九月十二日工畢,役軍夫二千名,實役九萬工,四十五日。



 渾河



 渾河,本盧溝水,從大興縣流至東安州、武清縣,入漷州界。至大二年十月,渾河水決左都威衛營西大堤,泛溢南流,沒左右二翊及後衛屯田麥,由是左都威衛言:「十月五日,水決武清縣王甫村堤,闊五十餘步,深五尺許,水西南漫平地流,環圓營倉局,水不沒者無幾。恐來春冰消,夏雨水作,沖決成渠,軍民被害,或遷置營司,或多差軍民修塞,庶免墊溺。」三年二月十二日,省準下左右翊及後衛、大都路委官督工修治,至五月二十日工畢。



 皇慶元年二月十七日,東安州言:「渾河水溢,決黃堝堤一十七所。」都水監計工物移文工部。二十七日,樞密知院塔失帖木兒奏:「左衛言渾河決堤口二處,屯田浸不耕種,已發軍五百修治。臣等議,治水有司職耳,宜令中書戒所屬用心修治。」從之。七月,省委工部員外郎張彬言:「巡視渾河,六月三十日霖雨,水漲及丈餘,決堤口二百餘步,漂民廬,沒禾稼,乞委官修治,發民丁刈雜草興築。」



 延祐元年六月十七日,左衛言:「六月十四日,渾河決武清縣劉家莊堤口,差軍七百與東安州民夫協力同修之。」三年三月,省議:「渾河決堤堰,沒田禾,軍民蒙害,既已奏聞。差官相視,上自石徑山金口,下至武清縣界舊堤,長計三百四十八里,中間因舊修築者大小四十七處,漲水所害合修補者一十九處,無堤創修者八處,宜疏通者二處,計工三十八萬一百,役軍夫三萬五千,九十六日可畢。如通築則役大難成,就令分作三年為之,省院差官先發軍民夫匠萬人,興工以修其要處。」是月二十日,樞府奏撥軍三千,委中衛僉事督修治之。七年五月,營田提舉司言:「去歲十二月二十一日,屯戶巡視廣賦屯北渾河堤二百餘步將崩,恐春首土解水漲,浸沒為患,乞修治。」都水監委濠寨,會營田提舉司官、武清縣官,督夫修完廣武屯北陷薄堤一處,計二千五百工;永興屯北堤低薄一處,計四千一百六十六工;落褵村西沖圮一處,計三千七百三十三工;永興屯北崩圮一處,計六千五百十八工;北王村莊西河東岸至白墳兒,南至韓村西道口,計六千九十三工;劉邢莊西河東岸北至寶僧百戶屯,南至白墳兒,計三萬七百十二工。總用工五萬三千七百二十二。



 泰定四年四月,省議:「三年六月內霖雨,山水暴漲,泛沒大興縣諸鄉桑棗田園,移文樞府,於七衛屯田及見有軍內,差三千人修治。」



 白河



 白河,在漷州東四里,北出通州潞縣,南入於通州境,又東南至香河縣界,又流入於武清縣境,達於靜海縣界。



 至元三十年九月,漕司言:「通州運糧河全仰白、榆、渾三河之水,合流名曰潞河,舟楫之行有年矣。今歲新開閘河,分引渾、榆二河上源之水,故自李二寺至通州三十餘里,河道淺澀。今春夏天旱,有止深二尺處,糧船不通,改用小料船搬載,淹延歲月,致虧糧數。先是,都水監相視白河,自東岸吳家莊前,就大河西南,斜開小河二里許,引榆河合流至深溝壩下,以通漕舟。今丈量,自深溝、榆河上灣,至吳家莊龍王廟前白河,西南至壩河八百步。及巡視,知榆河上源築閉,其水盡趨通惠河,止有白佛、靈溝、一子母三小河水入榆河,泉脈微,不能勝舟。擬自吳家莊就龍王廟前閉白河,於西南開小渠,引水自壩河上灣入榆河,庶可漕運。又深溝樂歲五倉,積貯新舊糧七十餘萬石,站車挽運艱緩,由是訪視通州城北通惠河積水,至深溝村西水渠,去樂歲、廣儲等倉甚近,擬自積水處由舊渠北開四百步,至樂歲倉西北,以小料船運載甚便。」都省準焉。通惠河自通州城北,至樂歲西北,水陸共長五百步,計役八萬六百五十工。



 大德二年五月,中書省札付都水監:運糧河堤自楊村至河西務三十五處,用葦一萬九千一百四十束,軍夫二千六百四十九名,度三十日畢。於是本監分官率濠寨至楊村歷視壞堤,督巡河夫修理,以霖雨水溢,故工役倍元料,自寺洵口北至蔡村、清口、孫家務、辛莊、河西務堤,就用元料葦草,修補卑薄,創築月堤,頗有成功。其楊村兩岸相對出水河口四處,葦草不敷,就令軍夫採刈,至九月住役。楊村河上接通惠諸河,下通滹沱入江淮,使官民舟楫直達都邑,利國便民。奈楊村堤岸隨修隨圮,蓋為用力不固,徒煩工役,其未修者,候來春水涸土干,調軍夫修治。



 延祐六年十月,省臣言:「漕運糧儲及南來諸物商賈舟楫,皆由直沽達通惠河。今岸崩泥淺,不早疏浚,有礙舟行,必致物價翔湧。都水監職專水利,宜分官一員,以時巡視,遇有頹圮淺澀,隨宜修築,如功力不敷,有司差夫助役,怠事者究治。」從之。



 至治元年正月十一日,漕司言:「夏運海糧一百八十九萬餘石,轉漕往返,全藉河道通便,今小直沽水義河口潮汐往來,淤泥壅積七十餘處,漕運不能通行,宜移文都水監疏滌。」工部議:「時農作方興,兼民多艱食,若不差軍助役,民力有所不逮。」樞密院言:「軍人不敷。」省議:「若差民丁,方今東作之時,恐妨歲事。其令大都募民夫三千,日給傭鈔一兩、糙粳米一升,委正官提調,驗日支給,令都水監暨漕司官同督其事。」四月十一日入役,五月十日工畢。



 泰定元年二月,樞府臣奏:「臨清萬戶府言,至治元年霖雨,決壞運糧河岸,宜差軍修築。臣等議,誠利益事,令本府差軍三百執役。」從之。三年三月,都水監言:「河西務菜市灣水勢沖嚙,與倉相近,將來為患,宜於劉二總管營相對河東岸,截河築堤,改水道與舊河合,可杜後患。」四年正月,省臣奏準,樞府差軍五千,大都路募夫五千人,日支糙米五升、中統鈔一兩,本監工部委官與前衛董指揮同監役,是年三月十八日興工,六月十一日工畢。



 致和元年六月六日,臨清禦河萬戶府言:「泰定四年八月二日,河溢,壞營北門堤約五十步,漂舊樁木百餘,崩圮猶未已。」工部議:「河岸崩摧,理宜修治,既都水監會計工物,各處支給,其役夫三千人,若擬差民,方春恐妨農務,宜移文樞密院撥軍。」省準修舊堤岸,展闊新河口東岸,計工五萬九千九百三十七,用軍三千、木匠十人。



 天歷二年三月,漕司言:「元開劉二總管營相對河,比舊河運糧迂遠,乞委官相視,復開舊河便。」四月九日,奏準,差軍七千,委兵部員外郎鄧衡、都水監丞阿里、漕使太不花等督工修浚。後以冬寒,候凍解興役。三年,工部移文大都,於近甸募民夫三千,日支糙粳米三升、中統鈔一兩,兵部改委辛侍郎暨元委官修闢。



 至順元年六月,都水監言:「二十三日夜,白河水驟漲丈餘,觀音寺新修護倉堤,已督有司差夫救護,今水落尺餘,宜候伏槽興作。」



 御河



 御河,自大名路魏縣界經元城縣泉源鄉於村度,南北約十里,東北流至包家渡,下接館陶縣界三口。御河上從交河縣,下入清池縣界。又永濟河在清池縣西三十里,自南皮縣來,入清州,今呼為御河也。



 至元三年七月六日,都水監言:「運河二千餘里,漕公私物貨,為利甚大。自兵興以來,失於修治,清州之南,景州以北,頹闕岸口三十餘處,淤塞河流十五里。至癸巳年,朝廷役夫四千,修築浚滌,乃復行舟。今又三十餘年,無官主領。滄州地分,水面高於平地,全藉堤堰防護。其園圃之家掘堤作井,深至丈餘,或二丈,引水以溉蔬花。復有瀕河人民就堤取土,漸至闕破,走洩水勢,不惟澀行舟,妨運糧,或致漂民居,沒禾稼。其長蘆以北,索家馬頭之南,水內暗藏樁橛,破舟船,壞糧物。」部議以濱河州縣佐貳之官兼河防事,於各地分巡視,如有闕破,即率眾修治,拔去樁橛,仍禁園圃之家毋穿堤作井,栽樹取土。都省準議。七年,省臣言:「御河水泛武清縣,計疏浚役夫一十,工八十日可畢。」從之。



 至大元年六月二十九日,左翼屯田萬戶府呈:「五月十八日申時,水決會川縣孫家口岸約二十餘步,南流灌本管屯田,已移文河間路、武清縣、清州有司,多發丁夫,管領修治。」由是樞密院檄河間路、左翊屯田萬戶府,差軍並工築塞。十月,大名路浚州言:「七月十一日連雨至十七日,清、石二河水溢李家道,東南橫流。詢社長高良輩,稱水源自衛輝路汲縣東北,連本州淇門西舊黑蕩泊,溢流出岸,漫黃河古堤,東北流入本州齊賈泊,復入御河,漂及門民舍。竊計今歲水勢逆行,及下流漳水漲溢遏絕不能通,以致若此,實非人力可勝。又西關水手佐聚稱,七月十二日卯時,御河水驟漲三尺,十八日復添四尺,其水逆流,明是下流漲水壅逆,擬差官巡治。」



 延祐三年七月,滄州言:「清池縣民告,往年景州吳橋縣諸處御河水溢,沖決堤岸,萬戶千奴為恐傷其屯田,差軍築塞舊洩水郎兒口,故水無所洩,浸民廬及已熟田數萬頃,乞遣官疏闢,引水入海。及七月四日,決吳橋縣柳斜口東岸三十餘步,千戶移僧又遣軍閉塞郎兒口,水壅不得洩,必致漂蕩張管、許河、孟村三十餘村黍谷廬舍,故本州摘官相視,移文約會開闢,不從。」四年五月,都水監遣官與河間路官相視元塞郎兒口,東西長二十五步,南北闊二十尺,及堤南高一丈四尺,北高二丈餘,復按視郎兒口下流故河,至滄州約三十餘里,上下古跡寬闊,及減水故道,名曰盤河。今為開闢郎兒口,增浚故河,決積水,由滄州城北達滹沱河,以入於海。



 泰定元年九月,都水監遣官督丁夫五千八百九十八人,是月二十八日興工,十月二日工畢。



 灤河



 灤河,源出金蓮川中,由松亭北,經遷安東、平州西,瀕灤州入海也。王曾《北行錄》云:「自偏槍嶺四十里,過烏灤河,東有灤州,因河為名。」



 至元二十八年八月,省臣奏:「姚演言,奉敕疏浚灤河,漕運上都,乞應副沿河蓋露囷工匠什物,仍預備來歲所用漕船五百艘,水手一萬,牽船夫二萬四千。臣等集議,近歲東南荒歉,民力凋弊,造舟調夫,其事非輕,一時並行,必致重困。請先造舟十艘,量撥水手試行之,如果便,續增益。」制可其奏,先以五十艘行之,仍選能人同事。



 大德五年八月十三日,平灤路言:「六月九日霖雨,至十五日夜,灤河與淝、洳三河並溢,沖圮城東西二處舊護城堤、東西南三面城墻,橫流入城,漂郭外三關瀕河及在城官民屋廬糧物,沒田苗,溺人畜,死者甚眾,而雨猶不止。至二十四日夜,灤、漆、淝、洳諸河水復漲入城,餘屋漂蕩殆盡。」乃委吏部馬員外同都水監官修之,東西二堤,計用工三十一萬一千五十,鈔八千八十七錠十五兩,糙粳米三千一百一十石五斗,樁木等價鈔二百七十四錠二十六兩四錢。



 延祐四年六月十六日,上都留守司言:「正月一日,城南御河西北岸為水沖嚙,漸至頹圮,若不修治,恐來春水泛漲,漂沒民居。又開平縣言,四月二十六日霖雨,至二十八日夜,東關灤河水漲,沖損北岸,宜擬修築。本司議,即目仲夏霖雨,其水復溢,必大為害,乃委官督夫匠興役。開平發民夫,幼小不任役,請調軍供作,庶可速成。」五月二十一日,留守司言:「灤河水漲決堤,計修築用軍六百,宜令樞密院差調,官給其食。」制曰:「今維其時,移文樞密院發軍速為之。」虎賁司發軍三百治焉。



 泰定二年三月十三日,永平路屯田總管府言:「國家經費咸出於民,民之所生,無過農作。本屯闢田收糧,以供億內府之用,不為不重。訪馬城東北五里許張家莊龍灣頭,在昔有司差夫築堤,以防灤水,西南連清水河,至公安橋,皆本屯地分。去歲霖雨,水溢,沖蕩皆盡,浸死屯民田苗,終歲無收。方今農隙,若不預修,必致為害。」工部移文都水監,差濠寨泊本屯官及灤州官新詣相視,督令有司差夫補築。三年五月十日,上都留守司及本路總管府言:「巡視大西關南馬市口灤河遞北堤,侵嚙漸崩,不預治,恐夏霖雨水泛,貽害居民。」於是送都城所丈量,計用物修治,工部移文上都分部施行。七月二日,右丞相塔失帖木兒等奏:「斡耳朵思住冬營盤,為灤河走凌河水沖壞,將築護水堤,宜令樞密院發軍千二百人以供役。」從之。樞密院請遣軍千二百人。



 河間河



 河間河,在河間路界。泰定三年三月,都水監言:「河間路水患,古儉河,自北門外始,依舊疏通,至大成縣界,以洩上源水勢,引入鹽河,古陳玉帶河,自軍司口浚治,至雄州歸信縣界,以導澱濼積潦,注之易河。黃龍港,自鎖井口開鑿,至文安縣玳瑁口,以通濼水,經火燒澱,轉流入海。計河宜疏者三十處,總役夫三萬,三十日可畢。」是月省臣奏準,遣斷事官定住同元委都水孫監丞洎本處有司官,於旁近州縣發丁夫三萬,日給鈔一兩、米一升,先詣古陳玉帶河。尋以歲旱民饑,役興人勞罷,候年登為之。



 冶河



 冶河,在真定路平山縣西門外,經井陘縣流來本縣東北十里,入滹沱河。



 元貞元年正月十八日,丞相完澤等言:「往年先帝嘗命開真定冶河,已發丁夫人役,適值先帝升遐,以聚眾罷之。今請遵舊制,俾卒其事。」從之。



 皇慶元年七月二日,真定路言:「龍花、判官莊諸處壞堤,計工物,申請省委都水監及本路官,自平山縣西北,歷視滹沱、冶河合流,急注真定西南關,由是再議,照冶河故道,自平山縣西北河內,改修滾水石堤,下修龍塘堤,東南至水碾村,改引河道一里,蒲吾橋西,改闢河道一里。上至平山縣西北,下至寧晉縣,疏其淤澱,築堤分其上源入舊河,以殺其勢。復有程同、程章二石橋阻咽水勢,擬開減水月河二道,可久且便。下相欒城縣,南視趙州寧晉縣,諸河北之下源,地形低下,恐水泛,經欒城、趙州,壞石橋,阻河流為害。由是議於欒城縣北,聖母堂東冶河東岸,開減水河,可去真定之患。」省準,於二年二月都水監委官與本路及廉訪司官,同詣平山縣相視,會計修治,總計冶河,始自平山縣北關西龍神廟北獨石,通長五千八百六步,共役夫五千,為工十八萬八百七,無風雨妨工,三十六日可畢。



 滹沱河



 滹沱河,源出於西山,在真定路真定縣南一里,經槁城縣北一里,經平山縣北十里,《寰宇記》載經靈壽縣西南二十里。此河連貫真定諸郡,經流去處,皆曰滹沱水也。



 延祐七年十一月,真定路言:「真定縣城南滹沱河,北決堤,浸近城,每歲修築。聞其源本微,與冶河不相通,後二水合,其勢遂猛,屢壞金大堤為患。本路達魯花赤哈散於至元三十年言,準引闢冶河自作一流,滹沱河水十退三四。至大元年七月,水漂南關百餘家,淤塞冶河口,其水復滹河。自後歲有潰決之患,略舉大德十年至皇慶元年,節次修堤,用卷掃葦草二百餘萬,官給夫糧備傭直百餘萬錠。及延祐元年三月至五月,修堤二百七十餘步,其明堂、判官、勉村三處,就用橋木為樁,征夫五百餘人,執役月餘不能畢。近年米價翔貴,民匱於食,有丁者正身應役,單丁者必須募人,人日傭直不下三五貫,前工未畢,後役迭至。至七月八日,又沖塌李玉飛等莊及木方、胡營等村三處堤,長一千二百四十步,申請委官相視,差夫築月堤。延祐二年,本路前總管馬思忽嘗闢冶河,已復湮塞。今歲霖雨,水溢北岸數處,浸沒田禾。其河元經康家莊村南流,不記歲月,徙於村北。數年修築,皆於堤北取土,故南高北低,水愈就下侵嚙。西至木方村,東至護城堤,數約二千餘步,比來春,必須修治。用樁梢築土堤,亦非永久之計。若浚木方村南舊湮枯河,引水南流,閘閉北岸河口,於南岸取土築堤,下至合頭村北與本河合,如此去城稍遠,庶可無患。」都水監差官相視,截河築堤,闊千餘步,新開古岸,止闊六十步,恐不能制御千步之勢。若於北岸闕破低薄處,比元料,增夫力,葦草卷掃補築,便計葦草丁夫,若令責辦民間,緣今歲旱澇相仍,民食匱乏,擬均料各州縣上中戶,價錢及食米於官錢內支給。限二月二十日興工,役夫五千,為工十六萬七百一十九,度三十二日可畢。總計補築滹沱河北岸防水堤十處,長一千九百一十步,高闊不一,計三百四十萬七千七百五十尺,用推掃梯二十五,每梯用大檁三、小檁三,計大小檁一百五十,草三十五萬八百束,葦二十八萬六百四十束,梢柴七千二百束。



 至治元年三月,真定路言:「真定縣滹沱河,每遇水泛,沖堤岸,浸沒民田,已差募丁夫修築,與廉訪司官相視講究,如將木方村南舊堙河道疏闢,導水東南行,閘閉北岸,卻於河南取土,修築至合頭村,合入本河,似望可以民安。」都水監與真定路官相視議:「夫治水者,行其所無事,蓋以順其性也。閘閉滹沱河口,截河築堤一千餘步,開掘故河老岸,闊六十步,長三十餘里,改水東南行流,霖雨之時,水拍兩岸,截河堤堰,阻逆水性,新開故河,止闊六十步,焉能吞授千步之勢?上咽下滯,必致潰決,徒糜官錢,空勞民力。若順其自然,將河北岸舊堤比之元料,增添工物,如法卷掃,堅固修築,誠為官民便益。」省準補築滹沱河北岸縷水堤一十處,通長一千九百一十步,役夫五百名,計一十六萬七百三十九工。



 泰定四年八月七日,省臣奏:「真定路言,滹沱河水連年泛溢為害,都水監、廉訪司、真定路及瀕河州縣官洎耆老會議,其源自五臺諸山來,至平山縣王母村山口下,與平定州娘子廟石泉冶河合。夏秋霖雨水漲,拶漫城郭,每年勞民築堤,莫能除害,宜自王子村、辛安村鑿河,長四里餘,接魯家灣舊澗,復開二百餘步,合入冶河,以分殺其勢。又木方村滹沱河南岸故道,疏滌三十里,北岸下樁卷掃,築堤捍水,令東流。今歲儲材,九月興役,期十一月功成。所用石鐵石灰諸物,夫匠工糧,官為供給,力省功多,可永無害。工部議,若從所請,二河並治,役大民勞,擬先開冶河,其真定路征民夫,如不敷,可於鄰郡順德路差募人夫,日給中統鈔一兩五錢,如侵礙民田,官酬其直。中書省都水監差官,率知水利濠寨,督本路及當該州縣用工,廉訪司添力咸就,滹河近後再議。」從之。九月,委都水監官洎本道廉訪司真定路同監督有司並工修治。後真定路言:「閏九月五日為始興工間,據趙州臨城諸縣申,天寒地凍,難於用工,候春暖開闢便,已於十月七日放散人民。」部議,人夫既散,宜準所擬。凡已給夫鈔二萬六千八百三十二錠,地價錢六百三十錠。



 會通河



 會通河,起東昌路須城縣安山之西南,由壽張西北至東昌,又西北至於臨清,以逾於御河。



 至元二十六年,壽張縣尹韓仲暉、太史院令史邊源相繼建言,開河置閘,引汶水達舟於御河,以便公私漕販。省遣漕副馬之貞與源等按視地勢,商度工用,於是圖上可開之狀。詔出楮幣一百五十萬緡、米四萬石、鹽五萬斤,以為傭直,備器用,徵旁郡丁夫三萬,驛遣斷事官忙速兒、禮部尚書張孔孫、兵部尚書李處巽等董其役。首事於是年正月己亥,起於須城安山之西南,止於臨清之御河,其長二百五十餘里,中建閘三十有一,度高低,分遠邇,以節蓄洩。六月辛亥成,凡役工二百五十一萬七百四十有八,賜名曰會通河。



 二十七年,省以馬之貞言霖雨岸崩,河道淤淺,宜加修浚,奏撥放罷輸運站戶三千,專供其役,仍俾採伐木石等以充用。是後,歲委都水監官一員,佩分監印,率令史、奏差、濠寨官往職巡視,且督工,易閘以石,而視所損緩急為後先。至泰定二年,始克畢事。



 會通鎮閘三、土壩二,在臨清縣北。頭閘長一百尺,闊八十尺,兩直身各長四十尺,兩雁翅各斜長三十尺,高二尺,閘空闊二丈,自至元三十年正月一日興工,凡役夫匠六百六十名,至十月二十九日工畢。中閘南至隘船閘三里,元貞二年七月二十三日興工,至大德二年三月十三日工畢,夫匠四百四十三,長廣與上閘同。隘船閘南至李海務閘一百五十二里,延祐元年八月十五日興工,九月二十五日工畢,夫匠五百,閘空闊九尺,長廣同上。土壩二。



 李海務閘南至周家店閘一十二里,元貞二年二月二日興工,五月二十日工畢,夫匠五百二十七名,長廣與會通鎮閘同。



 周家店閘南至七級閘一十二里,大德四年正月二十一日興工,八月二十日工畢,夫匠四百四十二,長廣與上同。



 七級閘二:北閘南至南閘三里,大德元年五月一日興工,十月六日工畢,夫匠四百四十三名,長廣如周家店閘。南閘南至阿城閘一十二里,元貞二年正月二十日興工,十月五日工畢,夫匠四百五十名,長廣同北閘。



 阿城閘二:北閘南至南閘三里,大德三年三月五日興工,七月二十八日工畢,夫匠四百四十一名,長廣上同。南閘南至荊門北閘一十里,大德二年正月二十五日興工,十月一日工畢,夫匠四百四十六名,長廣上同。



 荊門閘二:北閘南至荊門南閘二里半,大德三年六月初一日興工,至十月二十五日工畢,役夫三百一十名,長廣同。南閘南至壽張閘六十三里,大德六年正月二十三日興工,六月二十九日工畢,長廣同北閘。



 壽張閘南至安山閘八里,至元三十一年正月一日興工,五月二十日工畢。



 安山閘南至開河閘八十五里,至元二十六年建。



 開河閘南至濟州閘一百二十四里。



 濟州閘三:上閘南至中閘三里,大德元年三月十二日興工,七月二十八日工畢。中閘南至下閘二里,至治元年三月一日興工,六月六日工畢。下閘南至趙村閘六里,大德七年二月十三日興工,五月二十一日工畢。



 趙村閘南至石佛閘七里,泰定四年二月十八日興工,五月二十日工畢。



 石佛閘南至辛店閘一十三里,延祐六年二月十日興工,四月二十九日工畢。



 辛店閘南至師家店閘二十四里,大德元年正月二十七日興工,四月一日工畢。



 師家店閘南至棗林閘一十五里,大德二年二月三日興工,五月二十三日工畢。



 棗林閘南至孟陽泊閘九十五里,延祐五年二月四日興工,五月二十二日工畢。



 孟陽泊閘南至金溝閘九十里,大德八年正月四日興工,五月十七日工畢。



 金溝閘南至隘船閘一十二里,大德十年閏正月二十五日興工,四月二十三日工畢。



 沽頭閘二:北隘船閘南至下閘二里,延祐二年二月六日興工,五月十五日工畢。南閘南至徐州一百二十里,大德十一年二月興工,五月十四日工畢。



 三水義口閘入鹽河,南至土山閘一十八里,泰定二年正月十九日興工,四月十三日工畢。土山閘南至三水義口閘二十五里,入鹽河。



 兗州閘。



 堈城閘。



 延祐元年二月二十日,省臣言:「江南行省起運諸物,皆由會通河以達於都,為其河淺澀,大船充塞於其中,阻礙餘船不得來往。每歲省臺差人巡視,其所差官言,始開河時,止許行百五十料船,近年權勢之人,並富商大賈,貪嗜貨利,造三四百料或五百料船,於此河行駕,以致阻滯官民舟楫,如於沽頭置小石閘一,止許行百五十料船便。臣等議,宜依所言,中書及都水監差官於沽頭置小閘一,及於臨清相視宜置閘處,亦置小閘一,禁約二百料之上船,不許入河行運。」從之。



 至治三年四月十日,都水分監言:「會通河沛縣東金溝、沽頭諸處,地形高峻,旱則水淺舟澀,省部已準置二滾水堰。近延祐二年,沽頭閘上增置隘閘一,以限巨舟,每經霖雨,則三閘月河、截河土堰,盡為沖決。自秋摘夫刈薪,至冬水落,或來歲春首修治,工夫浩大,動用丁夫千百,束薪十萬之餘,數月方完,勞費萬倍。又況延祐六年雨多水溢,月河、土堰及石閘雁翅日被沖嚙,土石相離,深及數丈,其工倍多,至今未完。今若運金溝、沽頭並隘閘三處見有石,於沽頭月河內修堰閘一所,更將隘閘移置金溝閘月河、或沽頭閘月河內,水大則大閘俱開,使水得通流,小則閉金溝大閘,上開隘閘,沽頭則閉隘閘,而啟正閘行舟。如此歲省修治之費,亦可免丁夫冬寒入水之苦,誠為一勞永逸。」移文工部,令委官與有司同議。於是差濠寨約會濟寧路官相視,就問金溝閘提領周德興,言每歲夏秋霖雨,沖失閘堤,必候水落,役夫採薪修治,不下三兩月方畢,冬寒水作,苦不勝言。會驗監察御史言:「延祐初,元省臣亦嘗請置隘閘以限巨舟,臣等議,其言當,請從之。」於是議:梭板等船乃御河、江、淮可行之物,宜遣出任其所之,於金溝、沽頭兩閘中置隘閘二,各闊一丈,以限大船。若欲於通惠、會通河行運者,止許一百五十料,違者罪之,仍沒其船。其大都、江南權勢紅頭花船,一體不許來往,準擬拆移沽頭隘閘,置於金溝大閘之南,仍作運環閘,其間空地北作滾水石堰,水漲即開大小三閘,水落即鎖閉大閘,止於隘閘通舟。果有小料船及官用巨物,許申稟上司,權開大閘,仍添金溝閘板積水,以便行舟。其沽頭截河土堰,依例改修石堰,盡除舊有土堰三道。金溝閘月河內創建滾水石堰,長一百七十尺,高一丈,闊一丈。沽頭閘月河內修截河堰,長一百八十尺,高一丈一尺,底闊二丈,上闊一丈。



 泰定四年四月,御史臺臣言:「巡視河道,自通州至真、揚,會集都水分監及瀕河州縣官民,詢考利病,不出兩端,一曰壅決,二曰經行。卑職參詳,自古立國,引漕皆有成式。自世祖屈群策,濟萬民,疏河渠,引清、濟、汶、泗,立閘節水,以通燕薊、江淮,舟楫萬里,振古所無。後人篤守成規,茍能舉其廢墜而已,實萬世無窮之利也。蓋水性流變不常,久廢不修,舊規漸壞,雖有智者,不能善後。以故詳歷考視,酌古準今,參會眾議,輒有管見,倘蒙採錄,責任水監,謹守勿失,能事畢矣。不窮利病之源,頻歲差人,具文巡視,徒為煩擾,無益於事。都水監元立南北隘閘,各闊九尺,二百料下船梁頭八尺五寸,可以入閘。愚民嗜利無厭,為隘閘所限,改造減舷添倉長船至八九十尺,甚至百尺,皆五六百料,入至閘內,不能回轉,動輒淺閣,阻礙餘舟,蓋緣隘閘之法,不能限其長短。今卑職至真州,問得造船作頭,稱過閘船梁八尺五寸船,該長六丈五尺,計二百料。由是參詳,宜於隘閘下岸立石則,遇船入閘,必須驗量,長不過則,然後放入,違者罪之。閘內舊有長船,立限遣出。」省下都水監,委濠寨官約會濟寧路委官同歷視議擬,隘閘下約八十步河北立二石則,中間相離六十五尺,如舟至彼,驗量如式,方許入閘,有長者罪遣退之。又與東昌路官親詣議擬,於元立隘閘西約一里,依已定丈尺,置石則驗量行舟,有不依元料者罪之。



 天歷三年三月,詔諭中外:「都水監言:世祖費國家財用,開闢會通河,以通漕運。往來使臣、下番百姓及隨從使臣、各枝幹脫權勢之人,到閘不候水則,恃勢捶撻看閘人等,頻頻啟放。又漕運糧船,凡遇水淺,於河內築土壩,積水以漸行舟,以故壞閘。乞禁治事。命後諸王駙馬各枝往來使臣及干脫權勢之人、下番使臣等,並運官糧船,如到閘,依舊定例啟閉。若似前不候水則,恃勢捶拷守閘人等,勒令啟閘,及河內用土築壩壞閘之人,治其罪。如守閘之人,恃有聖旨,合啟閘時,故意遲延,阻滯使臣客旅,欺要錢物,乃不畏常憲也。」仍令監察御史、廉訪司常加體察。



 兗州閘



 兗州閘已見前。至元二十七年四月,都漕運副使馬之貞言:



 準山東東西道宣慰使司牒文,相視兗州閘堰事。先於至元十二年蒙丞相伯顏訪問自江淮達大都河道,之貞乃言,宋、金以來,汶、泗相通河道,郭都水按視,可以通漕。於二十年中書省奏準,委兵部李尚書等開鑿,擬修石閘十四。二十一年,省委之貞與尚監察等同相視,擬修石閘八、石堰二,除已修畢外,有石閘一、石堰一、堽城石堰一,至今未修。據濟州以南,徐、邳沿河纖道橋梁,二十三年添立邳州水站,移文沿河州縣,修治已完。二十三年調之貞充漕運副使,委管閘接放綱船。沿河纖道,元無崩損去處,在前年例,當麻麥盛時,差官修理閘道,督責地主割刈麻麥,並滕州開決稻堰,泗源磨堰,差人於呂梁百步等祇,及濟州閘監督江淮綱運船隻,過祇出閘,不令阻滯客旅,茍取錢物。據新開會通並濟州汶、泗相通河,非自然長流河道,於兗州立閘堰,約泗水西流,堽城立閘堰,分汶水入河,南會於濟州,以六閘撙節水勢,啟閉通放舟楫,南通淮、泗,以入新開會通河,至於通州。近去歲四月,江淮都漕運使司言,本司糧運,經濟河至東阿交割,前者濟州運司,不時移文瀕河官司,修治纖道,若有緩急處所,正官取招呈省,路經歷、縣達魯花赤以下就便斷罪。今濟州漕司革罷,其河道撥屬都漕運司管領,本司糧運未到東阿,凡有阻滯,並是本司遲慢。迤南河道,從此無人管領,不時水勢泛溢,堤岸摧塌,澀滯河道。又濟州閘,前濟州運司正官親臨監視,其押綱船戶不敢分爭。即目各處官司差人管領,與綱官船戶各無統攝,爭要水勢,及攙越過閘,互相毆打,以致損壞船隻,浸沒官糧。擬將東阿河道撥付江淮都漕運司提調管領,庶幾不誤糧運,都省準焉。又準江淮都漕運司副使言,除委官看管閘堰外,據汶、泗、堽城二閘一堰、泗河兗州閘堰、濟州城南閘,乃會通河上源之喉衿,去歲流水沖壞堽城汶河土堰、兗州泗河土堰,必須移文兗州、泰安州差夫修閉。又被漲水沖破梁山一帶堤堰,走洩水勢,通入舊河,以致新河水小,澀糧船,乞移文斷事等官,轉下東平路修閉,上流撥屬江淮漕運司,下流屬之貞管領。若已後新河水小,直下濟州監閘官,並泰安、兗州、東平修理。據兗州石閘一所、石堰一道,堽城石閘一道,合用材物已行措置完備,必須修理,雖初經之貞相視會計,即令不隸管領,乞移文江淮漕司修治。其泰安州堽城安、梁山一帶堤岸,濟州閘等處,雖是撥屬江淮漕司,今後倘若水漲,沖壞堤堰,亦乞照會東平、濟寧、泰安,如承文字,亦仰奉行。又東阿、須城界安山閘,為糧船不由舊河來往,江淮所委監閘官已去,目今無人看管,必須之貞修理,以此權委人守焉。



\end{pinyinscope}