\article{本紀第三十四 文宗三}

\begin{pinyinscope}

 至順元年春正月丙辰,命趙世延、趙世安領纂修《經世大典》事。懷慶路饑,賑鈔四千錠。丁巳,賜明宗妃按出罕、月魯沙、不顏忽魯都鈔幣有差。以知樞密院事伯帖木兒為遼陽行省左丞相。戊午,頒璽書諭云南。辛酉一性原理貫徹到底。,時享太廟。命回回司天監翽星。壬戌,中興路饑,賑糶糧萬石,貧者仍賙其家。甲子,燕鐵木兒、伯顏並辭丞相職,不允,仍命阿榮、趙世安慰諭之。丁卯,雲南諸王禿堅及萬戶伯忽、阿禾、怯朝等叛,攻中慶路,陷之,殺廉訪司官,執左丞忻都等,迫令署諸文牘。庚午,芍陂屯及鷹坊軍士饑,賑糧一月。辛未,中書省臣言:「科舉會試日期,舊制以二月一日、三日、五日,近歲改為十一、十三、十五。請依舊制。」從之。壬申,衡陽徭為寇,劫掠湘鄉州。癸酉,以宣徽使撒敦復知樞密院事,與欽察臺並領長密卿。乙亥,賜燕鐵木兒質庫一。密海州文登、牟平縣饑,賑以糧三千石。丙子,衡州路饑,總管王伯恭以所受制命質官糧萬石賑之。丁丑,追封三寶奴為郢城王,謚榮敏。召荊王之子脫脫木兒赴闕。趙世延請致仕,不允。命中書省制玉帶二十,賜臣僚官一品者。遣使齎金千五百兩、銀五百兩,詣杭州書佛經。賜海南大興龍普明寺鈔萬錠,市永業地。戊寅,賜隆禧總管府田千頃。立荊襄等處、平松等處田賦提舉司,並隸太禧宗禋院。命陜西行省以鹽課鈔十萬錠賑流民之復業者。徭賊八百餘人寇石康縣。己卯,封太醫院使野理牙為秦國公。庚辰,升群玉署為群玉內司,秩正三品,置司尉、亞尉、僉司、司丞,仍隸奎章閣學士院。禮部尚書颭颭兼監群玉內司事。辛巳,改大都田賦提舉司為宣農提舉司,荊襄田賦提舉司為荊襄濟農香戶提舉司,平江提舉司為平江善農提舉司。遣使齎鈔三千錠,往甘肅市髦牛。濠州去年旱,賑糧一月。大名路及江浙諸路俱以去年旱告,永平路以去年八月雹災告。加封秦蜀郡太守李冰為聖德廣裕英惠王,其子二郎神為英烈昭惠靈顯仁祐王。



 二月壬午朔,以趙世安為御史中丞,史惟良為中書左丞。癸未,加知樞密院事燕不鄰開府儀同三司。籍張珪子五人家資。乙酉,以西僧加瓦藏卜、蘸八兒監藏並為烏思藏土蕃等處宣慰使都元帥。雲南麓川等土官來貢方物。楊州、安豐、廬州等路饑,以兩淮鹽課鈔五萬錠、糧五萬石賑之。真定、蘄、黃等路,汝寧府、鄭州饑,各賑糧一月。丁亥,命江南、陜西、河南等處富民輸粟補官,江南萬石者官正七品,陜西千五百石、河南二千石、江南五千石者從七品,自餘品級有差。四川富民有能輸粟赴江陵者,依河南例,其不願仕,乞封父母者聽。僧、道輸己粟者,加以師號。征江浙、江西、湖廣賑糶糧價鈔赴京師。己丑,禿堅、伯忽等攻陷仁德府,至馬龍州,調八番元帥完澤將八番答剌罕軍千人、順元土軍五百人御之。庚寅,改萬聖祐國、興龍普明、龍翔萬壽三提點所並為營繕都司,秩正四品;萬安規運、普慶營繕等八提點所並為營繕司,秩正五品。以修《經世大典》久無成功,專命奎章閣阿鄰帖木兒、忽都魯都兒迷失等譯國言所紀典章為漢語,纂修則趙世延、虞集等,而燕鐵木兒如國史例監修。開元路胡里改萬戶府軍士饑,給糧賑之。辛卯,以御史臺贓罰鈔萬錠、金千兩、銀五千兩付太禧宗禋院,供祭祀之需。賜燕鐵木兒給驛璽書,以徵其食邑租賦。奎章閣學士忽都魯都兒迷失、撒迪、虞集辭職,詔諭之曰:「昔我祖宗睿知聰明,其於致理之道,自然生知。朕以統緒所傳,實在眇躬,夙夜憂懼,自惟早歲跋涉艱阻,視我祖宗,既乏生知之明,於國家治體,豈能周知?故立奎章閣,置學士員,日以祖宗明訓、古昔治亂得失陳說於前,使朕樂於聽聞。卿等其推所學以稱朕意,其勿復辭。」帖麥赤驛戶及建康、廣德、鎮江諸路饑,賑糧一月。衛輝、江州二路饑,賑鈔二萬錠。寧國路饑,嘗賑糧二萬石,不足,復賑萬五千石。癸巳,衛輝路胙城、新鄉縣大風雨災。甲午,自庚寅至是日,京師大霜晝雺。立諸色民匠打捕鷹坊都總管府,秩正二品。置奎章閣監書博士二人,秩正五品。禿堅、伯忽等攻晉寧州,禿堅自立為雲南王,伯忽為丞相,阿禾、忽剌忽等為平章等官,立城柵,焚倉庫以拒命。乙未,中書省言:「江浙民饑,今歲海運為米二百萬石,其不足者來歲補運。」從之。丙申,雲南蒲蠻來朝。賑常德、澧州路饑。丁酉,帝及皇后、燕王阿剌忒納答剌並受佛戒。己亥,命明宗皇子受佛戒。監察御史言:「中書平章朵兒只,職任臺衡,不思報效,銓選之際,紊亂綱紀,貪污著聞,恬不知恥,黜罷為宜。」從之。徭賊入灌陽縣,劫民財。庚子,以兵興所收諸王也先帖木兒、搠思監等印還給之。壬寅,玥璐不花辭御史大夫職,不允。土蕃等處民饑,命有司以糧賑之。新安、保定諸驛孳畜疫死,命中書給鈔濟其乏。癸卯,汴梁路封丘、祥符縣霜災。甲辰,流王禪之子於吉陽軍。乙巳,封明宗皇子亦璘真班為鄜王。豫王阿剌忒納失里所部千六百餘人饑,賑糧二月。淮安路民饑,以兩淮鹽課鈔五萬錠賑之。丙午,復以阿兒思蘭海牙為江南行臺御史大夫。命中尚卿小雲失以兵討雲南。御史臺臣言:「欽察臺天歷初在上都,常與闊闊出等謀執倒剌沙,事洩,同謀者皆死,欽察臺以出征獲免。頃臺臣疑而劾之,不稱事情,宜雪其枉。」制曰:「可。」丁未,以伯顏知樞密院事,依前太保?⒙季墈皶厥隆Z祔橢惺樵緩「昔在世祖,嘗以宰相一人總領庶務,故治出於一,政有所統。今燕鐵木兒為右丞相,愛顏既知樞密院事,左丞相其勿復置。」太禧宗禋院所隸總管府,各置副達魯花赤一人。賜豫王王傅官金虎符。戊申,命中書省及翰林國史院官祭太祖、太宗、睿宗三朝御容。以太禧宗禋使阿不海牙為中書平章政事,命史惟良及參知政事和尚總督建言之事。中書省臣言:「舊制,正旦、天壽節,內外諸司各有贄獻,頃者罷之。今江浙省臣言,聖恩公溥,覆幬無疆,而臣等殊無補報,凡遇慶禮,進表稱賀,請如舊制為宜。」從之。降璽書申鹽法之禁。以嘉興路崇德縣民四萬戶所輸租稅,供英宗后妃歲賜錢帛。詔諭樞密院,以屯田子粒鈔萬錠助建佛寺,免其軍卒土木之役。庚戌,茶陵州民饑,同知萬家奴、江存禮以所受敕質糧三千石賑之。辛亥,迤西蒙古驛戶饑,給芻粟有差。賑河南流民復歸者鈔五千錠。泰安州饑民三千戶,真定南宮縣饑民七千七百戶,松江府饑民萬八千二百戶,及土蕃朵裡只失監萬戶部內饑,命所在有司從宜賑之。濟寧路饑民四萬四千九百戶,賑以山東鹽課鈔萬錠。杭州火,賑糧一月。命市故瀛國公趙鳷田,為大龍翔集慶寺永業。御史臺臣言不必予其直,帝曰:「吾建寺為子孫黎民計,若取人田而不予直,非朕志也。」察罕腦兒宣慰司所部千戶察剌等衛饑者萬四千四百五十六人,人給鈔一錠。



 三月甲寅,命宣政院供顯懿莊聖皇后神御殿祭祀。乖西鷿蠻三千人入松梨山,燒沿邊官軍營堡。東平路須城縣饑,賑以山東鹽課鈔。安慶、安豐、蘄、黃、廬五路饑,以淮西廉訪司贓罰鈔賑之。丁巳,徙封濟陽王木楠子為吳王,吳王潑皮為濟陽王。賜八番順元、曲靖、烏撒、烏蒙、蒙慶、羅羅斯、嵩明州土官幣帛各一。禁泛濫給驛。四川官吏脅從囊加臺者皆復故職。戊午,封皇子阿剌忒納答剌為燕王,立宮相府總其府事,秩正二品,燕鐵木兒領之。廷試進士,賜篤列圖、王文燁等九十七人及第、出身有差。命彰德路歲祭羑里周文王祠。以河南行省平章乞住為雲南行省平章,八番順元宣慰使帖木兒不花為雲南行省左丞,從豫王由八番道討雲南。賜明宗近侍七十人官有差。裕宗及昭獻元聖皇后位宿衛三千人,命儲政院給其衣糧芻粟。發米十萬石賑糶京師貧民。癸亥,遣諸王桑哥班、撒忒迷失、買哥分使西北諸王燕只吉臺、不賽因、月即別等所。甲子,詔諭中外,命御史大夫鐵木兒補化、玥璐不花振舉臺綱。丁卯,木八剌沙來貢蒲萄酒,賜鈔幣有差。以山東鹽課鈔萬錠賑東昌饑民三萬三千六百戶。己巳,議明宗升祔,序於英宗之上,視順宗、成宗廟遷之例。辛未,群臣請上皇帝尊號,不許,固請不已,乃許之。封知樞密院事不花帖木兒為武平郡王。錄討雲南禿堅、伯忽之功,雲南宣慰使土官舉宗、祿餘並遙授雲南行省參知政事,余賜賚有差。分龍慶州隸大都路。諸王也孫臺部七百餘人入天山縣,掠民財產,遣樞密院、宗正府官往捕之。壬申,奉玉冊、玉寶,祔明宗神主於太廟。濮州臨清、館陶二縣饑,賑鈔七千錠。光州光山縣饑,出官粟萬石,下其直賑糶。信陽、息州及光之固始縣饑,並以附近倉糧賑之。甲戌,封諸王速來蠻為西寧王。乙亥,西番哈剌火州來貢蒲萄酒。諸王、駙馬還鎮,錫賚有差。丙子,改山東都萬戶府為都督府。雲南木邦路土官渾都來貢方物。河南登封、偃師、孟津諸縣饑,賑以兩淮鹽課鈔三萬錠。鞏昌、臨洮、蘭州、定西州饑,賑鈔三千五百錠。沂、莒、膠、密、寧海五州饑,賑糧五千石。中興、峽州、歸州、安陸、沔陽饑戶三十萬有奇,賑糧四月。丁丑,升太常禮儀院秩正二品。敕有司供明宗後八不沙宮分幣帛二百匹,及阿梯里、脫忽思幣帛有差。賜燕鐵木兒功勛之碑。廣平路饑,以河間鹽課鈔萬三千錠賑之。辛巳,諸王哈兒蠻遣使來貢蒲萄酒。廣德、太平、集慶等路饑,凡數百萬戶。濮州諸縣蟲食桑葉將盡。



 夏四月壬午朔,命西僧作佛事於仁智殿,自是日始,至十二月終罷。癸未,置怯憐口錢糧都總管府,秩正三品。中書省臣言:「各宮分及宿衛士歲賜錢帛,舊額萬人,去歲增四千人,邇者增數益廣,請依舊額為宜。」詔命阿不海牙裁省以聞。甲申,時享太廟。丙戌,封也真也不乾為桓國公。燕鐵木兒言:「天歷初,阿速軍士為國有勞,請以鈔十萬錠、米十萬石分給其家。」從之。戊子,四川行省調重慶五路萬戶以兵救雲南。庚寅,中書省臣言:「邇者諸處民饑,累常賑救,去歲賑鈔百三十四萬九千六百餘錠、糧二十五萬一千七百餘石。今汴梁、懷慶、彰德、大名、興和、衛輝、順德、歸德及高唐、泰安、徐、邳、曹、冠等州饑民六十七萬六千戶,一百一萬二千餘口,請以鈔九萬錠、米萬五千石,命有司分賑。」制曰:「可。」以陜西饑,敕有司作佛事七日。壬辰,以所籍張珪諸子田四百頃,賜大承天護聖寺為永業。沿邊部落蒙古饑民八千二百,人給鈔三錠、布二匹、糧二月,遣還其所部。癸巳,置豫王王傅、副尉、司馬各二員。丁酉,遣諸王桑兀孫還雲南。金蘭等驛馬牛死,賑鈔五百錠。庚子,降璽書申諭太禧宗禋院。天臨之醴陵、湘陰等州、臺州之臨海等縣饑,各賑糴米五千石。辛丑,明宗後八不沙崩。壬寅,括益都、般陽、寧海閑田十六萬二千九十頃,賜大承天護聖寺為永業。立益都廣農提舉司及益都、般陽、寧海諸提領所,並隸隆祥總管府。烏撒土官祿餘殺烏撒宣慰司官吏,降於伯忽。羅羅諸蠻俱叛,與伯忽相應,平章帖木兒不花為其所害。晉寧、建昌二路民饑,賑糧五萬五千石、鈔二萬三千錠。戊申,陜西行臺言:「奉元、鞏昌、鳳翔等路以累歲饑,不能具五穀種,請給鈔二萬錠,俾分糴於他郡。」從之。雲南賊祿餘以蠻兵七百餘人拒烏撒、順元界,立關固守。重慶五路萬戶軍至雲南境,值羅羅蠻,萬餘人遇害,千戶祝天祥等引餘眾遁還。詔江浙、河南、江西三省調兵二萬,命諸王云都思帖木兒及樞密判官洪浹將之,與湖廣行省平章脫歡會兵討雲南。己酉,作佛事。是月,滄州、高唐州屬縣蟲食桑葉盡。芍陂屯饑,賑糧二月。土蕃等處脫思麻民饑,命有司賑之。賑懷慶承恩、孟州等驛鈔千錠。



 五月乙卯,遣宣徽使定住等,以受尊號告祭南郊。故四川行省平章寬徹、四川道廉訪使忽都魯養阿等,皆為囊加臺所害,並贈官賜謚。榆次縣主簿太帖木兒、河中府判官禿塔兒,皆為遼東軍所害,並加褒贈。戊午,帝御大明殿,燕帖木兒率文武百官及僧道、耆老,奉玉冊、玉寶,上尊號曰欽天統聖至德誠功大文孝皇帝。是日,改元至順,詔天下。河南、懷慶、衛輝、晉寧四路曾經賑濟人戶,今歲差發全行蠲免,其餘被災路分人民已經賑濟者,腹裏差發、江淮夏稅,亦免三分。己未,羅羅斯權土官宣慰撒加伯、阿漏土官阿剌、里州土官德益叛,附於祿餘。庚申,以受尊號恭謝太廟。辛酉,四川行省討雲南,進軍至烏蒙。壬戌,歸德府之譙縣霧傷麥。癸亥,四川軍至雲南之雪山峽,遇羅羅斯軍,敗之。德州饑,賑以山東鹽課鈔三千錠。武昌路饑,賑以糧五萬石、鈔二千錠。甲子,申命燕鐵木兒為中書右丞相,詔天下。以鈔四萬錠分給宮人,賜魯國大長公主鈔萬錠。丁卯,翰林國史院修《英宗實錄》成。戊辰,車駕發大都,次大口。升尚舍寺秩正三品。命阿鄰帖木兒為大司徒。遣豫王阿剌忒納失里鎮西番,授以金印。賜諸王脫歡金印,大司徒不蘭奚銀印。加趙世延翰林學士承旨,封魯國公。賑衛輝、大名、廬州饑民鈔六千錠、糧五千石。開元路胡里該萬戶府、寧夏路哈赤千戶所軍士饑,各賑糧二月。己巳,次龍虎臺。置肅王寬徹傅、尉、司馬各一員。辛未,置宣忠扈衛親軍都萬戶府,秩正三品,總幹羅思軍士,隸樞密院。以太禧宗禋使亦列赤為中書平章政事。左、右欽察、龍翊侍衛軍士五千三百七十戶饑,戶賑鈔二錠、布一匹、糧一月。癸酉,遣使勞軍於雲南。時諸王禿剌率萬戶忽都魯沙、怯列、孛羅等,皆領兵進討禿堅、伯忽。甲戌,八番乖西鷿苗阿馬、察伯秩等萬人侵擾邊境,詔樞密臣分兵討之。乙亥,置順元宣撫司,統答剌罕軍征雲南,人賜鈔五錠。衛輝路之輝州,以荒乏穀種,給鈔三千錠,俾糴於他郡。己卯,遣使詣五臺山作佛事。庚辰,命湖廣行省以鈔五萬錠給雲南軍需。是月,右衛左右手屯田大水,害禾稼八百餘頃。廣平、河南、大名、般陽、南陽、濟寧、東平、汴梁等路,高唐、開、濮、輝、德、冠、滑等州,及大有、千斯等屯田蝗。以浙東宣慰使陳天祐、湖廣參知政事樊楫死於王事,贈封特加一級。龍興張仁興妻鄒氏、奉元李鬱妻崔氏以志節,汴梁尹華以孝行,皆旌其門。六月辛巳朔,燕鐵木兒言:「向有旨,惟許臣及伯顏兼領三職。今趙世延以平章政事兼翰林學士承旨、奎章閣大學士,引疾以辭。」帝曰:「朕重老成人,其令世延仍視事中書,果病,無預銓選可也。」丙戌,大駕至上都。戊子,給左、右欽察、龍翊侍衛軍士糧。壬辰,鎮江饑,賑糧四萬石。饒州饑,亦命有司賑之。癸巳,御史臺臣言:「宣徽院錢穀,出納無經,以上供飲膳,冒昧者多,不稽其案牘,則弊日滋。宜如舊制,具實上之省部,以備考核。」從之。丙申,立行樞密院討雲南,賜給驛璽書十五、銀字圓符五。以河南行省平章徹里鐵木兒知行樞密院事,陜西行省平章探馬赤、近侍教化為同知、副使。發朵甘思、朵思麻及鞏昌諸處軍萬三千人,人乘馬三匹。徹里鐵木兒同鎮西武靖王搠思班等由四川,教化從豫王阿剌忒納失裡等由八番,分道進軍。黃河溢,大名路之屬縣沒民田五百八十餘頃。庚子,以內侍中瑞卿撒里為大司徒,賜四川行省左丞孛羅金虎符。以鹽課鈔二十萬錠供雲南軍需。命河南、湖廣、江西、甘肅行省誦《藏經》六百五十部,施鈔三萬錠。知樞密院事闊徹伯、脫脫木兒,通政使只兒哈郎,翰林學士承旨教化的、伯顏也不干,燕王宮相教化的、斡羅思,中政使尚家奴、禿烏臺,右阿速衛指揮使那海察、拜住,以謀變有罪,並棄市,籍其家。癸卯,四川孛羅以蒙古漸丁軍五千往雲南。乙巳,羅羅斯土官撒加伯合烏蒙蠻兵萬人攻建昌縣,雲南行省右丞躍裏帖木兒拒之,斬首四百餘級,四川軍亦敗撒加伯於蘆古驛。丙午,朵思麻蒙古民饑,賑糧一月。丁未,改東路蒙古軍元帥府為東路欽察軍萬戶府。是月,高唐、曹州及前、後、武衛屯田水災。大都、益都、真定、河間諸路,獻、景、泰安諸州,及左都威衛屯田蝗。迤北蒙古饑民三千四百人,人給糧二石、布二匹。旌表真定梁子益妻李氏等貞節,徐州胡居仁孝行。



 秋七月辛亥,封諸王按渾察為廣寧王,授以金印。壬子,命西僧禜星。丙辰,以闊徹伯大司徒印授撒里。丁巳,命中書省、翰林國史院官祀太祖、太宗、睿宗御容於大普慶寺。命西僧為皇子燕王作佛事。西域諸王不賽因遣使來朝賀。監察御史請以所籍闊徹伯衣物分賜宿衛軍士,從之。己未,以闊徹伯宅賜太禧宗禋院,衣服賜群臣。通渭山崩,壓民舍,命陜西行省賑被災者十二家。庚申,籍脫脫木兒家貲,輸內府。辛酉,改哈思罕萬戶府為總管府,秩四品。詔:「僧、道、獵戶、鷹坊合得璽書者,翰林院無得越中書省以聞。」真定路之平棘,廣平路之肥鄉,保定路之曲陽、行唐等縣,大風雨雹傷稼。許失臺速怯、月謹真孛可等部獻人口牧畜,命酬其直。江西建昌萬戶府軍戍廣海者,一歲更役,來往勞苦,詔仍至元舊制,二歲一更。乙丑,翰林學士承旨也兒吉尼知樞密院事。調諸衛卒築漷州柳林海子堤堰。丙寅,蒙古百姓以饑乏至上都者,閱口數給以行糧,俾各還所部。增大都賑糶米五萬石。大都之順州、東安州大風雨雹傷稼。戊辰,壽寧公主薨,收其印。己巳,命江浙行省以鈔十萬錠至雲南增其軍需。庚午,歲星犯氐宿。開平路雨雹傷稼。中書省臣言:「近歲帑廩虛空,其費有五:曰賞賜,曰作佛事,曰創置衙門,曰濫冒支請,曰續增衛士鷹坊。請與樞密院、御史臺、各怯薛官同加汰減。」從之。御史臺臣劾奏新除河南府總管張居敬避難不之官,有旨免所授官,加其罪笞。甲戍,賜諸王養怯帖木兒、孛欒臺、征棘斯、察阿兀罕等金銀鈔幣有差。丙子,敕中書省、御史臺遣官詣江浙、江西、湖廣、四川、雲南諸行省,遷調三品以下官。命四川行省於明年茶鹽引內給鈔八萬錠增軍需,以討雲南。賑木鄰、扎里至苦鹽泊等九驛,每驛鈔五百錠。增給戍居庸關軍士糧。海潮溢,漂沒河間運司鹽二萬六千七百餘引。丁丑,以給驛璽書五、銀字圓符二,增給陜西蒙古都萬戶府,以討雲南。故丞相鐵木迭兒子將作使鎖住與其弟觀音奴、姊夫太醫使野理牙,坐怨望、造符錄、祭北斗、咒咀,事覺,詔中書鞫之。事連前刑部尚書烏馬兒、前御史大夫孛羅、上都留守馬兒及野理牙姊阿納昔木思等,俱伏誅。雲南禿堅、伯忽等勢愈猖獗,烏撒祿餘亦乘勢連約烏蒙、東川、茫部諸蠻,欲令伯忽弟拜延等兵攻順元。樞密臣以聞,詔即遣使督豫王阿納忒剌失里及行樞密院、四川、雲南行省亟會諸軍分道進討,以烏蒙、烏撒及羅羅斯地接西番,與碉門安撫司相為脣齒,命宣政院督所屬軍民嚴加守備,又命鞏昌都總帥府調兵千人戍四川。開元、大同、真定、冀寧、廣平諸路及忠翊侍衛左右屯田,自夏至於是月不雨。奉元、晉寧、興國、揚州、淮安、懷慶、衛輝、益都、般陽、濟南、濟寧、河南、河中、保定、河間等路及武衛、宗仁衛、左衛率府諸屯田蝗。永平龐遵以孝行,福州王薦以隱逸,大同李文實妻齊氏、河南閻遂妻楊氏、大都潘居敬妻陳氏、王成妻高氏以志節,順德馬奔妻胡閏奴、真定民妻周氏、冀寧民妻魏益紅以夫死自縊殉葬,並旌其門。



 閏七月庚辰朔,封諸王卯澤為永寧王,授金印,及給銀字圓符、給驛璽書,並以所隸封邑歲賦賜之。癸未,遣諸王篤憐、渾禿、孛羅等共銀千兩、幣二百匹,賜諸王朵列鐵木兒。監察御史葛明誠言:「中書平章政事趙世延,年逾七十,智慮耗衰,固位茍容,無補於事,請斥歸田里。」臺臣以聞,詔令中書議之。雲南茫部路九村夷人阿斡、阿里詣四川行省自陳:「本路舊隸四川,今土官撒加伯與雲南連叛,願備糧四百石、民丁千人,助大軍進徵。」事聞,詔嘉其去逆效順,厚慰諭之。衛士上都駐冬者,所給糧以三分為率,二分給鈔。大駕將還,敕上都兵馬司官二員,率兵士由偏嶺至明安巡邏,以防盜賊。市橐駝百、牛三百,充扈從屬軍之用。丙戌,忠翊衛左右屯田隕霜殺稼。籍鎖住、野里牙等庫藏、田宅、奴僕、牧畜,給大承天護聖寺為永業。鑄黃金神仙符命印,賜掌全真教道士苗道一。己丑,立掌醫署,秩正五品。庚寅,以所籍野理牙宅為都督府公署。辛卯,以陜西行臺御史中丞脫亦納為中書參知政事。燕鐵木兒言:「趙世延向自言年老,屢乞致仕,臣等以聞,嘗有旨,世延舊人,宜與共政中書。御史之言,不知前有旨也。」帝曰:「如御史言,世延固難任中書矣,其仍任以翰林、奎章之職。」四川行省平章汪壽昌言:「雲南伯忽叛逆,興兵進討,調遣饋餉,皆壽昌領之。頃以市馬、造器械、軍官俸給、軍士行糧,已給鈔十五萬錠。今伯忽未及殄滅,而烏撒、烏蒙相繼為亂,大兵深入,去朝廷益遠,元請軍需,早乞頒降,從本省酌其緩急,便宜以行,庶不稽誤。」從之。寧夏、奉元、鞏昌、鳳翔、大同、晉寧諸路屬縣隕霜殺稼。癸巳,以月魯帖木兒為大司徒。賜哈剌赤軍士鈔一萬錠、糧十萬石。察罕腦兒並東、西涼亭諸衛士九百五十人,人賜鈔五錠、糧二月;朔漠軍士,人鈔三錠、布二匹、糧二月。命燕鐵木兒以鈔萬錠,分賜天歷初諸王、群臣死事之家。行樞密院言:「征戍雲南軍士二人逃歸,捕獲,法當死。」詔曰:「如臨戰陣而逃,死宜也。非接戰而逃,輒當以死,何視人命之易耶?其杖而流之。」丁酉,大駕發上都。授阿憐帖木兒大司徒印。戊戌,甘肅平章政事乃馬臺封宣寧郡王,授以金印;駙馬謹只兒封鄆國公,授以銀印;並知行樞密院事。贈安南國王陳益稷儀同三司、湖廣行省平章政事,王爵如故,謚忠懿。益稷在世祖時自其國來歸,遂授以國王,即居於漢陽府,天歷二年卒,至是加贈、謚。庚子,魯王阿剌哥識里所部三萬餘人告饑,賑鈔萬錠、糧二萬石。中書省臣言:「內外佛寺三百六十七所,用金、銀、鈔、幣不貲,今國用不充,宜從裁省。」命省人及宣政院臣裁減。上都歲作佛事百六十五所,定為百四所,令有司永為歲例。乙巳,雲南使來報捷,遣使賜雲南、四川省臣、行樞密院臣以上尊。丙午,諸王卜顏帖木兒請給鞍馬,願從諸軍擊雲南,帝嘉其意,從之。戊申,加封孔子父齊國公叔梁紇為啟聖王,母魯國太夫人顏氏為啟聖王夫人,顏子兗國復聖公,曾子成阜國宗聖公,子思沂國述聖公,孟子鄒國亞聖公,河南伯程顥豫國公,伊陽伯程頤洛國公。羅羅斯土官撒加伯及阿陋土官阿剌、里州土官德益兵八千撤毀棧道,遣把事曹通潛結西番,欲據大渡河進寇建昌。四川行省調碉門安撫司軍七百人,成都、保寧、順慶、廣安諸屯兵千人,令萬戶周戡統領,直抵羅羅斯界,以控扼西番及諸蠻部。又遣成都、順慶二翼萬戶昝定遠等,以軍五千同邛部知州馬伯所部蠻兵,會周戡等,從便道共討之,發成都沙糖戶二百九十人防遏敘州。征重慶、夔州逃亡軍八百人赴成都。廣西徭於國安率千五百人寇修仁、荔浦等縣,廣西元帥府發兵捕之,賊眾潰走,生擒國安。大都、太寧、保定、益都諸屬縣及京畿諸衛、大司農諸屯水,沒田八十餘頃。杭州、常州、慶元、紹興、鎮江、寧國諸路及常德、安慶、池州、荊門諸屬縣皆水,沒田一萬三千五百八十餘頃。松江、平江、嘉興、湖州等路水,漂民廬,沒田三萬六千六百餘頃,饑民四十萬五千五百七十餘戶,詔江浙行省以入粟補官鈔三千錠及勸率富人出粟十萬石賑之。寶慶、衡、永諸處,田生青蟲,食禾稼。冠州鬱世復、大都趙祥及弟英,以孝行旌其門。大都愛祖丁、塔術,漷州劉仲溫,以輸米賑貧旌其門。



 八月庚戌,河南府路新安、沔池等十五驛饑疫,人給粟、馬給芻粟各一月。辛亥,雲南躍里鐵木兒以兵屯建昌,執羅羅斯把事曹通斬之。丁巳,北邊諸王月即別遣使來京師。燕鐵木兒由西道田獵未至,詔以機務至重,遣使趣召之。己未,大駕至京師,勞遣人士還營。有言蔚州廣靈縣地產銀者,詔中書、太禧院遣人蒞其事,歲所得銀歸大承天護聖寺。辛酉,以世祖是月生,命京師率僧百七十人作佛事七日。御史臺臣請立燕王為皇太子,帝曰:「朕子尚幼,非裕宗為燕王時比,俟燕帖木兒至,共議之。」甲子,忠州土官黃祖顯遣其子宗忠來朝,獻方物。乙丑,遣使詣真定玉華宮,祀睿宗及顯懿莊聖皇后神御殿。戊辰,太白犯氐宿。壬申,詔興舉蒙古字學。中書省、樞密院、御史臺言:「臣等比奉旨裁省衛士,今定大內四宿衛之士,每宿衛不過四百人;累朝宿衛之士,各不過二百人。鷹坊萬四千二十四人,當減者四千人。內饔九百九十人,四怯薛當留者各百人。累朝舊邸宮分饔人三千二百二十四人,當留者千一百二十人。媵臣、怯憐口共萬人,當留者六千人。其汰去者,斥歸本部著籍應役。自裁省之後,各宿衛復有容匿漢、南、高麗人及奴隸濫充者,怯薛官與其長杖五十七,犯者與曲給散者皆杖七十七,沒家貲之半,以籍入之半為告者賞。仍令監察御史察之。」制可。九月庚辰,江浙行省言:「今歲夏秋霖雨大水,沒民田甚多,稅糧不滿舊額,明年海運本省止可二百萬石,餘數令他省補運為便。」從之罷入粟補官例。糴豆二十三萬石於河間、保定等路,冠、恩、高唐等州,出馬八萬匹,令諸路分牧之。大寧路地震。甲申,授不蘭奚及月魯鐵木兒大司徒印。史惟良辭中書左丞職,不允。命藝文監以《燕鐵木兒世家》刻板行之。命河南行省給湖廣行省鈔四千錠為軍需,討雲南。遼陽諸王老的、蠻子臺諸部擾民,敕樞密院、宗五府及行省,每歲遣官偕往巡問,以治其獄訟。監察御史葛明誠劾奏:「遼陽行省平章哈剌鐵木兒,嘗坐贓被杖罪,今復任以宰執,控制東籓,亦足見國家名爵之濫,黜罷為宜。」從之。丙戌,邛部州土官馬伯向導征雲南軍有功,以為征進招討,知本州事。江西、湖廣蒙古軍進征雲南者,人給鈔五錠。雲南羅羅斯叛,與成都甚邇,而成都軍馬俱進征雲南,詔四川鄰境諸王,發籓部丁壯二千人戍成都。廣源賊弗道閉覆寇龍州羅回洞,龍州萬戶府移文詰安南國,其國回言:「本國自歸順天朝,恪共臣職,彼疆我界,盡歸一統。豈以羅回元隸本國,遂起爭端?此蓋邊吏生釁,假閉覆為名爾,本府宜自加窮治。」湖廣行省備其言以聞,命龍州萬戶府申嚴邊防。己丑,熒惑犯鬼宿。辛卯,賜陜西蒙古軍之征雲南者三十人,人鈔六錠。監察御史朵羅臺、王文若言:「嶺北行省乃太祖肇基之地,武宗時,太師月赤察兒為右丞相,太傅答剌罕為左丞相,保安邊境,朝廷遂無北顧之患。今天子臨御,及命哈八兒禿為平章政事,其人無正大之譽,有鄙俚之稱,錢穀甲兵之事,懵無所知,豈能昭宣皇猷,贊襄國政?且以月赤察兒輩居於前,而以斯人繼其後,賢不肖固不待辯而明,理宜黜罷。」制曰:「可。」癸巳,白虹貫日。置麓川路軍民總管府,復立總管府於哈剌火州。甲午,熒惑犯鬼宿積尸氣。封魏王阿木哥子阿魯為西靖王。乙未,以立冬祀五福十神、太一真君。御史臺臣劾奏:「前中書平章速速,叨居臺鼎,專肆貪淫,兩經杖斷一百七,方議流竄,幸蒙恩宥,量徙湖廣。不復畏法自守,而乃攜妻娶妾,濫污百端。況湖廣乃屯兵重鎮,豈宜居此?乞屏之遠裔,以示至公。」詔永竄雷州,湖廣行省遣人械送其所。丙申,以魯國大長公主邸第未完,復給鈔萬錠,命中書平章亦列赤董其役。己亥,以奎章閣纂修《經世大典》,命省、院、臺諸司以次宴其官屬。以平江等處官田五百頃,賜魯國大長公主。敕:「諸人非其本俗,敢有弟收其嫂、子收庶母者,坐罪。」壬寅,核實諸衛軍戶物力。賜魯國大長公主鈔萬錠,命燕鐵木兒詣其邸第送之。丙午,命西僧作佛事於大明殿。史惟良復乞辭職歸養,允其請,仍賜鈔二百錠。丁未,中書參知政事張友諒為左丞,知樞密院事脫別臺為陜西行臺御史大夫。鐵裡乾、木鄰等三十二驛,自夏秋不雨,牧畜多死,民大饑,命嶺北行省人賑糧二石。至治初以白云宗田給壽安山寺為永業,至是其僧沈明琦以為言,有旨令中書省改正之。敕有司繕治南郊齋宮。遼陽行省水達達路,自去夏霖雨,黑龍、宋瓦二江水溢,民無魚為食。至是,末魯孫一十五狗驛,狗多餓死,賑糧兩月,狗死者,給鈔補市之。辰州萬戶圖格里不花母石抹氏以志節,漳州龍溪縣陳必達以孝行,並旌其門。



 冬十月戊申朔,降璽書申飭衍聖公崇奉孔子廟事。賜雲南行省參政忽都沙三珠虎符。辛亥,命湖廣行省給諸王云都思鐵木兒幣百匹,以賞將士捕徭賊有功者。壬子,諸王、大臣復請立燕王為皇太子,帝曰:「卿等所言誠是。但燕王尚幼,恐其識慮未弘,不克負荷,徐議之未晚也。」立宣忠扈衛親軍都萬戶營於大都北,市民田百三十餘頃賜之。戊午,致齋於大明殿。己未,遣亞獻官中書右丞相燕鐵木兒、終獻官貼木爾補化率諸執事告廟,請以太祖皇帝配享南郊。庚申,出次郊宮。辛酉,帝服大裘、袞冕,祀昊天上帝於南郊,以太祖皇帝配,禮成,是日大駕還宮。甲子,以奉元驛馬瘠死,命陜西行省給鈔三千錠補市之。木納火失溫所居諸牧人三千戶、瀕黃河所居鷹坊五千戶,各賑糧兩月。乙丑,廣西徭賊寇橫州及永淳縣,敕廣西元帥府率兵捕之。樞密院臣言:「每歲大駕幸上都,發各衛軍士千五百人扈從,又發諸衛漢軍萬五千人駐山後,蒙古軍三千人駐官山,以守關梁。乞如舊數調遣,以俟來年。」從之。辛未,烏蒙路土官阿朝歸順,遣其通事阿累等貢方物。壬申,御史臺臣言:「內外官吏令家人受財,以其幹名犯義,罪止四十七、解任。今貪污者緣此犯法愈多,請依十二章計贓多寡論罪。」從之。甲戌,敕:「累朝宮分官署,凡文移無得稱皇后,止稱某位下娘子。其委用官屬,並由中書擬聞。」乙亥,改打捕鷹坊總管府為仁虞都總管府。知樞密院事撒敦、宣徽使唐其勢,並賜答剌罕之號。中書省臣言:「近討雲南,已給鈔二十萬錠為軍需,今費用已盡,鎮西武靖王搠思班及行省、行院復求鈔如前數。臣等議,方當進討之際,宜依所請給之。」制曰:「可。」賜伯夷、叔齊廟額曰聖清,歲春秋祠以少牢。遣使趣四川、雲南行省兵進討。於是四川行省平章塔出引兵由永寧,左丞孛羅引兵由青山、茫部並進,陳兵周泥驛,及祿餘等戰,殺蠻兵三百餘人。祿餘眾潰,即奪其關隘,以導順元諸軍。時雲南行省平章乞住等俱失期不至。



 十一月庚辰,命中書賑糶糧十萬石,濟京師貧民。辛巳,御史臺臣言:「陜西行省左丞怯列,坐受人僮奴一人及鸚鵡,請論如律。」詔曰:「位至宰執,食國厚祿,猶受人生口,理宜罪之。便鸚鵡微物,以是論贓,失於太苛,其從重者議罪。今後凡饋禽鳥者,勿以贓論,著為令。」癸未,賑上都灤河駐冬各宮分怯憐口萬五千七百戶糧二萬石。甲申,熒惑退犯鬼宿。命帝師率西僧作佛事,內外凡八所,以是日始,歲終罷。丙戌,太白犯壘壁陣。中書省臣言:「至元間,安豐、安慶、廬州等路有未附籍戶千四百三十六,世祖命以其歲賦賜床兀兒。後既附籍,所輸歲賦皆入官,別令萬億庫歲給以鈔二百錠。今乞停所給鈔,復以其戶還賜床兀兒之子燕鐵木兒。」從之。羅羅斯撒加伯、烏撒阿答等合諸蠻萬五千人攻建昌,躍里鐵木兒等引兵追戰於木托山下,敗之,斬首五百餘級。賑襄、鄧畏兀民被西兵害者六十三戶,戶給鈔十五錠、米二石;被西兵掠者五百七十七戶,戶給鈔五錠、米二石。廣西廉訪司言:「今討叛徭,各行省官將兵二萬人,皆屯駐靜江,遷延不進,曠日持久,恐失事機。」詔遣使趣之。知樞密院事燕不憐,請依舊制全給鷹坊芻粟,使毋貧乏。帝曰:「國用皆百姓所供,當量入為出,朕豈以鷹坊失所,重困吾民哉?」不從。辛卯,以闊闊臺知樞密院事。給山東鹽課鈔三千錠,賑曹州濟陰等縣饑民。癸巳,以臨江、吉安兩路天源延聖寺田千頃所入租稅,隸太禧宗禋院。戊戌,立打捕鷹坊紅花總管府於遼陽行省,秩四品。辛丑,征河南行省民間自實田土糧稅,不通舟楫之處得以鈔代輸。命陜西行省賑河州蒙古屯田衛士糧兩月。甲辰,命司天監翽星。丙午,恩州諸王按灰,坐擊傷巡檢張恭,杖六十七,謫還廣寧王所部充軍役。



 十二月戊申,遣伯顏等以將立燕王阿剌忒納答剌為皇太子,告祭於郊、廟。己酉,以董仲舒從祀孔子廟,位列七十子之下。國子生積分及等者,省、臺、集賢院、奎章閣官同考試,中式者以等第試官,不中者復入學肄業。以粟十萬石,米、豆各十五萬石,給河北諸路牧官馬之家,宣忠扈衛斡羅思屯田,給牛、種、農具。辛亥,立燕王阿剌忒納答剌為皇太子,詔天下。甲寅,西域軍士居永平、灤州、豐閏、玉田者,人給鈔三錠、布二匹、糧兩月。監察御史言:「昔裕宗由燕邸而正儲位,世祖擇耆舊老臣如王顒、姚燧、蕭等為之師、保、賓客。今皇太子仁都聰睿,出自天成,誠宜慎選德望老成、學行純正者,俾之輔導於左右,以宏養正之功,實宗社生民之福也。」帝嘉納其言。詔:「龍翔集慶寺工役、佛事,江南行臺悉給之。」戊午,以十月郊祀禮成,帝御大明殿受文武百官朝賀,大赦天下。癸亥,知樞密院事闊闊臺兼大都留守。乙丑,遣集賢侍讀學士珠遘詣真定,以明年正月二十日祀睿宗及後於玉華宮之神御殿。丁卯,命西僧於興聖、光天宮十六所作佛事。癸酉,詔宣忠扈衛親軍都萬戶府:「凡立營司境內所屬山林川澤,其鳥獸魚鱉悉供內膳,諸獵捕者坐罪。」甲戌,御史中丞和尚坐受婦人為賂,遇赦原罪。監察御史言:「和尚所為貪縱,有污臺綱,罪雖見原,理宜追奪所受制命,禁錮元籍終其身。」臺臣以聞,制可。敕各行省:「凡遇邊防有警,許令便宜發兵,事緩則驛聞。」賑龍慶州懷來縣前歲被兵萬一千八百六十戶糧兩月。冀寧路梁世明妻程氏、中興路伯顏妻阿迭的以志節,大都宛平縣鄭珪以行義,並旌其門。賑遼陽行省所居鷹坊戶糧一月。



\end{pinyinscope}