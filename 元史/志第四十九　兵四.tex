\article{志第四十九 兵四}

\begin{pinyinscope}

 ○站赤



 元制站赤者,驛傳之譯名也。蓋以通達邊情,布宣號令,古人所謂置郵而傳命,未有重於此者焉。凡站,陸則以馬以牛,或以驢,或以車,而水則以舟。其給驛傳璽書,謂之鋪馬聖旨。遇軍務之急,則又以金字圓符為信,銀字者次之;內則掌之天府,外則國人之為長官者主之。其官有驛令,有提領,又置脫脫禾孫於關會之地,以司辨詰,皆總之於通政院及中書兵部。而站戶闕乏逃亡,則又以時簽補,且加賑恤焉。於是四方往來之使,止則有館舍,頓則有供帳,饑渴則有飲食,而梯航畢達,海宇會同,元之天下,視前代所以為極盛也。今故著其驛政之大者,然後紀各省水陸凡若干站,而遼東狗站,亦因以附見雲。



 太宗元年十一月,敕:「諸牛鋪馬站,每一百戶置漢車一十具。各站俱置米倉,站戶每年一牌內納米一石,令百戶一人掌之。北使臣每日支肉一斤、面一斤、米一升、酒一瓶。」



 四年五月,諭隨路官員並站赤人等:「使臣無牌面文字,始給馬之驛官及元差官,皆罪之。有文字牌面,而不給驛馬者,亦論罪。若系軍情急速,及送納顏色、絲線、酒食、米粟、段匹、鷹廑,但系禦用諸物,雖無牌面文字,亦驗數應付車牛。」



 世祖中統四年三月,中書省定議乘坐驛馬,長行馬使臣、從人及下文字曳剌、解子人等分例。乘驛使臣換馬處,正使臣支粥食、解渴酒,從人支粥。宿頓處,正使臣白米一升,面一斤,酒一升,油鹽雜支鈔一十文,冬月一行日支炭五斤,十月一日為始,正月三十日終住支;從人白米一升,面一斤。長行馬使臣齎聖旨、令旨及省部文字,乾當官事者,其一二居長人員,支宿頓分例,次人與粥飯,仍支給馬一匹、草一十二斤、料五升,十月為始,至三月三十日終止,白米一升,面一斤,油鹽雜用鈔一十文。投呈公文曳剌、解字,依部擬宿頓處批支。五月,雲州設站戶,取迤南州城站戶籍內,選堪中上戶應當。馬站戶,馬一匹,牛站戶,牛二雙,於各戶選堪當站役之人,不問親軀,每戶取二丁,及家屬於立站去處安置。



 五年八月,詔:「站戶貧富不等,每戶限四頃,除免稅石,以供鋪馬祗應;已上地畝,全納地稅。」



 至元六年二月,詔:「各道憲司,如總管府例,每道給鋪馬劄子三道。」



 七年正月,省部官定議:「各路總管府在城驛,設官二員,於見役人員內選用;州縣驛,設頭目二名,如見役人即是相應站戶,就令依上任事,不系站戶,則就本站馬戶內別行選用;除脫脫禾孫依舊存設,隨路見設總站官罷之。」十一月,立諸站都統領使司,往來使臣,令脫脫禾孫盤問。



 八年正月,中書省議:「鋪馬劄子,初用蒙古字,各處站赤未能盡識,宜繪畫馬匹數目,復以省印覆之,庶無疑惑。」因命今後各處取給鋪馬標附文籍,其馬匹數付譯史房書寫畢,就左右司用墨印,印給馬數目,省印印訖,別行附籍發行墨印,左右司封掌。



 九年八月,諸站都統領使司言:「朝省諸司局院,及外路諸官府應差馳驛使臣所齎劄子,從脫脫禾孫辨詰,無脫脫禾孫之處,令總管府驗之。」



 十一年十月,命隨處站赤,直隸各路總管府,其站戶家屬,令元籍州縣管領。



 十三年正月,改諸站都統領使司為通政院,命降鑄印信。



 十七年二月,詔:「江淮諸路增置水站。除海青使臣,及事干軍務者,方許馳驛。餘者自濟州水站為始,並令乘船往來。」



 十八年閏八月,詔:「除上都、榆林迤北站赤外,隨路官錢,不須支給,驗其閑劇,量增站戶,協力自備首思當站。」



 十九年四月,詔給各處行省鋪馬聖旨,揚州行省、鄂州行省、泉州行省、隆興行省、占城行省、安西行省、四川行省、西夏行省、甘州行省,每省五道。南方驗田糧及七十石者,準當站馬一匹。九月,通政院臣言:「隨路站赤三五戶,共當正馬一匹,十三戶供車一輛,自備一切什物公用。近年以來,多為諸王公主及正宮太子位下頭目識認招收,或冒入投下戶計者,遂致站赤損弊,乞換補站戶。」從之。十月,增給各省鋪馬聖旨,西川、京兆、泉州十道,甘州、中興各五道。



 二十年二月,和林宣慰司給鋪馬聖旨二道。五月,江淮行省增給十道,都省遣使繁多,亦增二十道給之。七月,免站戶和顧和買、一切雜泛差役,仍令自備首思。十一月,增給甘州行省鋪馬聖旨十道,總之為二十道。十二月,增各省及轉運司、宣慰司鋪馬聖旨三十五道:江淮行省十道,四川行省十道,安西轉運司分司二道,荊湖行省所轄湖南宣慰司三道,福建行省十道。



 二十一年二月,增給各處鋪馬劄子:荊湖、占城等處本省一十道,荊湖北道宣慰司二道,所轄路分一十六處,每處二道;山東運司二道;河間運司七道;宣德府三道;江西行省五道;福建行省所轄路分七處,每處二道;司農司五道;四川行省所轄順元路宣慰司三道,思州、播州兩處宣撫司各三道;都省二十道。四月,定增使臣分例:正使宿頓支米一升、面一斤、羊肉一斤、酒一升、柴一束、油鹽雜支增鈔二分,通作三分,經過減半。從者每名支米一升,經過減半。九月,給阿里海牙所治之省鋪馬聖旨十道,所轄宣慰司二處,各三道。



 二十二年四月,給陜西行省並各處宣慰司、行工部等處鋪馬劄子一百二十六道。



 二十三年四月,福建、東京兩行省各給圓牌二面。奧魯赤出使交趾,先給圓牌二面,今再增二面,於脫歡太子位下給發。南京行省起馬三十匹,給圓牌二面。創立三處宣慰司,給劄子起馬三十匹。



 二十四年四月,增給尚書省鋪馬聖旨一百五十道,並先給降一百五十道,共三百道。五月,揚州省言:「徐州至揚州水馬站,兩各分置,夏月水潦,使臣勞苦。請徙馬站附並水站一處安置,馳驛者白日馬行,夜則經由水路,況站戶皆是水濱居止者,庶幾官民兩便。」從之。七月,給中興路、陜西行省、廣東宣慰司、沙不丁等官鋪馬聖旨一十三道。



 二十五年正月,腹里路分三十八處,年銷祗應錢不敷,增給鈔三千九百八十一錠,並元額七千一百六十九錠,總中統鈔一萬一千一百五十錠,分上下半年給降。二月,命南方站戶,以糧七十石出馬一匹為則,或十石之下八九戶共之,或二三十石之上兩三戶共之,惟求稅糧僅足當站之數,不至多餘,卻免其一切雜泛差役。若有納糧百石之下、七十石之上,自請獨當站馬一匹者聽之。五月,增給遼陽行省鋪馬劄子五道。十一月,福建行省元給鋪馬聖旨二十四道,增給劄子六道。



 二十六年正月,給光祿寺鋪馬劄子四道。二月,從沿海鎮守官蔡澤言,以舊有水軍二千人,於海道置立水站。三月,給海道運糧萬戶府鋪馬聖旨五道。四月,四川紹慶路給鋪馬劄子二道,成都府六道。龍興行省增給鋪馬聖旨五道,太原府宣慰司及儲峙提舉司給降二道。八月,給遼東宣慰司鋪馬聖旨五道,大理、金齒宣慰司四道。九月,增給西京宣慰司鋪馬劄子五道,江淮行省所轄浙東道宣慰司三道,紹興路總管府給降二道,甘肅行省所轄亦集乃總管府、沙州、肅州三路給六道。十一月,增給甘肅行省鋪馬聖旨七道。



 二十七年正月,增給陜西行省鋪馬聖旨五道。二月,都省增給鋪馬聖旨一百五十道,江淮行省一十五道。六月,給營田提舉司鋪馬聖旨二道。九月,江淮行省所轄徽州路水道不通,給鋪馬聖旨二道。



 二十八年六月,隨處設站官二員,大都至上都置司吏三名,餘設二名,祗應頭目、攢典各一名。站戶及百者,設百戶一名。七月,詔各路府州縣達魯花赤長官,依軍戶例,兼管站赤奧魯,非奉通政院明文,不得擅科差役。十二月,增給省除之任官鋪馬聖旨三百五十道。



 二十九年三月,命通政院分官四員,於江南四省整理站赤,給印與之。



 三十年正月,南丹州洞蠻來朝,立安撫司於其地,給鋪馬聖旨二道。三月,兩淮都轉運鹽使司增給鋪馬聖旨起馬五匹。五月,給淘金運司鋪馬聖旨起馬五匹,大司農司起馬二十匹。六月,江浙行省言:「各路遞運站船,若止以六戶供船一艘,除苗不過十四五石,力寡不能當役。請令各路除苗不過元額二十四石,自六戶之上,或至十戶,通融簽撥。」從之。八月,給劉二拔都兒圓牌三面,鋪馬聖旨一十五道。十月,增給濟南府鹽運司鋪馬聖旨一道。



 三十一年六月,給福建運司鋪馬聖旨起馬五匹。



 成宗大德八年正月,御史臺臣言:「各處站赤合用祗應官錢,多不依時撥降,又或數少不給,遂令站戶輸當庫子,陪備應辦。莫若驗使臣起數,實支官錢,所在官司,依時撥降,令各站提領收掌祗待,毋得科配小民,似為便益。」詔都省定議行之。



 十年,從江浙省言,命站官仍領祗待,選站戶之有餘糧者,以充庫子,止設一名,上下半年更代,就準本戶里正、主首身役。



 武宗至大三年五月,給嘉興、松江、瑞州三路及汴梁等處管民總管府鋪馬聖旨各三道。



 四年三月,詔拘收各衙門鋪馬聖旨,命中書省定議以聞。省臣言:「始者站赤隸兵站,後屬通政院,今通政院怠於整治,站赤消乏,合依舊命兵部領之。」制可。四月,中書省臣又言:「昨奉旨以站赤屬兵部,今右丞相鐵木迭兒等議,漢地之驛,命兵部領之,其鐵烈干、納鄰、末鄰等處蒙古站赤,仍付通政院。」帝曰:「何必如此,但令罷通政院,悉隸兵部可也。」閏七月,復立通政院,領蒙古站赤。八月,詔:「大都至上都,每站除設驛令、丞外,設提領三員、司吏三名。腹里路分,沖要水陸站赤,設提領二員、司吏二名。其餘閑慢驛分,止設提領一員、司吏一名。如無驛令,量擬提領二員。每一百戶,設百戶一名,從拘該路府州縣提調正官,於站戶內選用,三歲為滿。凡濫設官吏頭目人等,盡罷之。」十一月,給中政院鋪馬聖旨二十道。



 仁宗皇慶二年四月,增給陜西行臺鋪馬聖旨八道。



 延祐元年六月,中書省臣言:「典瑞監掌金字圓牌及鋪馬聖旨三百餘道。至大四年,凡聖旨皆納之於翰林院,以金字圓牌不敷,增置五十面。蓋圓牌遣使,初為軍情大事而設,不宜濫給,自今求給牌面,不經中書省、樞密院者,宜勿與。」從之。十月,沙、瓜州立屯儲總管萬戶府,給鋪馬聖旨六道。



 五年十月,中書兵部言:「各站設置提領,止受部劄,行九品印,職專車馬之役,所領站赤多者三二千,少者五七百戶,比之軍民,體非輕細。奈何俸祿不給,三年一更,貪邪得以自縱。今擬各處館驛,除令、丞外,見役提領不許交換。」從之。



 七年四月,詔蒙古、漢人站,依世祖舊制,悉歸之通政院。十一月,從通政院官請,詔腹裏、江南漢地站赤,依舊制,命各路達魯花赤、總管提調,州縣官勿得預。



 泰定元年三月,遣官賑給帖裡乾、木憐、納憐等一百一十九站鈔二十一萬三千三百錠,糧七萬六千二百四十四石八斗。北方站赤,每加津濟,至此為最盛。



 中書省所轄腹裏各路站赤,總計一百九十八處:



 陸站一百七十五處,馬一萬二千二百九十八匹,車一千六十九輛,牛一千九百八十二隻,驢四千九百八頭。水站二十一處,船九百五十隻,馬二百六十六匹,牛二百隻,驢三百九十四頭,羊五百口。牛站二處,牛三百六隻,車六十輛。



 河南江北等處行中書省所轄,總計一百七十九處,該一百九十六站:



 陸站一百六處,馬三千九百二十八匹,車二百一十七輛,牛一百九十二隻,驢五百三十四頭。水站九十處,船一千五百一十二隻。



 遼陽等處行中書省所轄,總計一百二十處:



 陸站一百五處,馬六千五百一十五匹,車二千六百二十一輛,牛五千二百五十九只。狗站一十五處,元設站戶三百,狗三千隻,後除絕亡倒死外,實在站戶二百八十九,狗二百一十八隻。



 江浙等處行中書省所轄,總計二百六十二處:



 馬站一百三十四處,馬五千一百二十三匹。轎站三十五處,轎一百四十八乘。步站一十一處,遞運夫三千三十二戶。水站八十二處,船一千六百二十七只。



 江西等處行中書省所轄,總計一百五十四處:



 馬站八十五處,馬二千一百六十五匹,轎二十五乘。水站六十九處,船五百六十八隻。



 湖廣等處行中書省所轄,總計一百七十三處:



 陸站一百處,馬二千五百五十五匹,車七十輛,牛五百四十五只,坐轎一百七十五乘,臥轎三十乘。水站七十三處,船五百八十只。



 陜西行中書省所轄八十一處:



 陸站八十處,馬七千六百二十九匹。水站一處,船六隻。



 四川行中書省所轄:



 陸站四十八處,馬九百八十六匹,牛一百五十頭。水站八十四處,船六百五十四隻,牛七十六頭。



 雲南諸路行中書省所轄站赤七十八處:



 馬站七十四處,馬二千三百四十五匹,牛三十只。水站四處,船二十四隻。



 甘肅行中書省所轄三路:



 脫脫禾孫馬站六處,馬四百九十一匹,牛一百四十九頭,驢一百七十一頭,羊六百五十口。



 ○弓手



 元制,郡邑設弓手,以防盜也。內而京師,有南北兩城兵馬司,外而諸路府所轄州縣,設縣尉司、巡檢司、捕盜所,置巡軍弓手,而其數則有多寡之不同。職巡邏,專捕獲。官有綱運及流徙者至,則執兵仗導道,以轉相授受。外此則不敢役,示專其職焉。



 世祖中統五年,隨州府驛路設置巡馬及馬步弓手,驗民戶多寡,定立額數。除本管頭目外,本處長官兼充提控官。其夜禁之法,一更三點,鐘聲絕,禁人行;五更三點,鐘聲動,聽人行。有公事急速及喪病產育之類,則不在此限。違者笞二十七下,有官者笞七下,準贖元寶鈔一貫。州縣城池相離遠處,其間五七十里,所有村店及二十戶以上者,設立巡防弓手,合用器仗,必須完備,令本縣長官提調。不及二十戶者,依數差補。若無村店去處,或五七十里,創立聚落店舍,亦須及二十戶數。其巡軍別設,不在戶數之內。關津渡口,必當設立店舍弓手去處,不在五七十里之限。於本路不以是何投下當差戶計,及軍站人匠、打捕鷹房、斡脫、窯冶諸色人等戶內,每一百戶內取中戶一名充役,與免本戶合著差發,其當戶推到合該差發數目,卻於九十九戶內均攤。若有失盜,勒令當該弓手,定立三限盤捉,每限一月。如限內不獲,其捕盜官,強盜停俸兩月,竊盜一月。外據弓手,如一月不獲,強盜決一十七下,竊盜七下;兩月不獲,強盜二十七下,竊盜一十七下;三月不獲者,強盜三十七下,竊盜二十七下。如限內獲賊,數及一半者,全免正罪。



 至元三年,省部議:「隨路戶數,多寡不同,兼軍站不該差發,似難均攤。擬合斟酌京府司縣合用人數,止於本處包銀絲線,並止納包銀戶計內,每一百戶選差中戶一名當役,本戶合當差發稅銀,卻令九十九戶包納。」從之。



 四年,除上都、中都已有巡軍,其所轄州縣合設弓手,俱於本路包銀等戶選丁多強壯者充,驗各處州縣戶數多寡、驛程緊慢設置,合用器仗,各人自備。



 八年,御史臺言:「諸路宜選年壯熟閑弓馬之人,以備巡捕之職。弓手數少者,亦宜增置。除捕盜防轉,不得別行差占。」



 十六年,分大都南北兩城兵馬司,各主捕盜之任。南城三十二處,弓手一千四百名;北城一十七處,弓手七百九十五名。



 二十三年,省臺官言:「捕賊巡馬,先令執持悶棍以行,賊眾多有弓箭,反致巡軍被傷。今議給各路弓箭十副,府州七副,司縣五副,各令置備防盜。」從之。



 仁宗延祐二年,從江南行御史臺請,以各處弓手人等,往往致害人命,役三年者罷之,還當民役,別於相應戶內補換。



 急遞鋪兵



 古者置郵而傳命,示速也。元制,設急遞鋪,以達四方文書之往來,其所系至重,其立法蓋可考焉。



 世祖時,自燕京至開平府,復自開平府至京兆,始驗地里遠近,人數多寡,立急遞站鋪。每十里或十五里、二十五里,則設一鋪,於各州縣所管民戶及漏籍戶內,簽起鋪兵。



 中統元年,詔:「隨處官司,設傳遞鋪驛,每鋪置鋪丁五人。各處縣官,置文簿一道付鋪,遇有轉遞文字,當傳鋪所即注名件到鋪時刻,及所轄轉遞人姓名,置簿,令轉送人取下鋪押字交收時刻還鋪。本縣官司時復照刷,稽滯者治罪。其文字,本縣官司絹袋封記,以牌書號。其牌長五寸,闊一寸五分,以綠油黃字書號。若系邊關急速公事,用匣子封鎖,於上重別題號,及寫某處文字,發遣時刻,以憑照勘遲速。其匣子長一尺,闊四寸,高三寸,用黑油紅字書號。已上牌匣俱系營造小尺,上以千字文為號,仍將本管地境、置立鋪驛卓望地名,遞相傳報。」鋪兵一晝夜行四百里。各路總管府委有俸正官一員,每季親行提點。州縣亦委有俸末職正官,上下半月照刷。如有怠慢,初犯事輕者笞四十,贖銅,再犯罰俸一月,三犯者決。總管府提點官比總管減一等,仍科三十,初犯贖銅,再犯罰俸半月,三犯者決。鋪兵鋪司,痛行斷罪。



 至元八年,申命州縣官,用心照刷及點視闕少鋪司鋪兵。凡有遞轉文字到,鋪司隨即分明附籍,速令當該鋪兵,裹以軟絹包袱,更用油絹卷縛,夾版束系,齎小回歷一本,作急走遞,到下鋪交割附歷訖,於回歷上令鋪司驗到鋪時刻,並文字總計角數,及有無開拆、磨擦損壞,或亂行批寫字樣,如此附寫一行,鋪司畫字,回還。若有違犯,易為挨問。隨路鋪兵,不許顧人領替,須要本戶少壯人力正身應役。每鋪安置十二時輪子一枚、紅綽楔一座,並牌額及上司行下、諸路申上鋪歷二本。每遇夜,常明燈燭。其鋪兵每名備夾版、鈴攀各一付,纓槍一,軟絹包袱一,油絹三尺,蓑衣一領,回歷一本。各處往來文字,先用凈檢紙封裹於上,更用厚夾紙印信封皮。各路承發文字人吏,每日逐旋發放,及將承發到文字,驗視有無開拆、磨擦損壞、批寫字樣,分朗附簿。



 九年,左補闕祖立福合言:「諸路急遞鋪臺,不合人情。急者急速也,國家設官署名字,必須吉祥者為美,宜更定之。」遂更為通遠鋪。



 二十年,留守司官言:「初立急遞鋪時,取不能當差貧戶,除其差發充鋪兵,又不敷者,於漏籍戶內貼補。今富人規避差發,求充鋪兵,乞擇其富者,令充站戶,站戶之貧者,卻充鋪兵。」從之。



 二十八年,中書省定議:「近年入遞文字,封緘雜亂,發遣無時,今後省部並諸衙門入遞文字,其常事皆付承發司隨所投下去處,類為一緘。如往江淮行省者,凡江淮行省不以是何文字,通為一緘。其他官府同。省部臺院,凡有急速之事,別置匣子發遣,其匣子入遞,隨到即行。鋪司須能附寫文歷,辨定時刻,鋪兵須壯健善走者,不堪之人,隨即易換。」



 三十一年,大都設置總急遞鋪提領所,降九品銅印,設提領三員。



 英宗至治三年,各處急遞鋪,每十鋪設一郵長,於州縣籍記司吏內差充,使之專督其事。一歲之內,能盡職者,從優補用;不能者,提調官量輕重罪之。



 凡鋪卒皆腰革帶,懸鈴,持槍,挾雨衣,齎文書以行。夜則持炬火,道狹則車馬者,負荷者,聞鈴避諸旁,夜亦以驚虎狼也。響及所之鋪,則鋪人出以俟其至。囊板以護文書不破碎、不襞積,折小漆絹以禦雨雪,不使濡濕之。及各鋪得之,則又展轉遞去。



 鷹房捕獵



 元制,自御位及諸王,皆有昔寶赤,蓋鷹人也。是故捕獵有戶,使之致鮮食以薦宗廟,供天庖,而齒革羽毛,又皆足以備用,此殆不可闕焉者也。然地有禁,取有時,而違者則罪之。冬春之交,天子或親幸近郊,縱鷹隼搏擊,以為游豫之度,謂之飛放。故鷹房捕獵,皆有司存。而打捕鷹房人戶,多取析居、放良及漏籍孛蘭奚、還俗僧道與凡曠役無賴者,乃招收亡宋舊役等戶為之。其差發,除納地稅、商稅,依例出軍等六色宣課外,並免其雜泛差役。自太宗乙未年,抄籍分屬御位下及諸王公主駙馬各投下。及世祖時,行尚書省嘗重定其籍,厥後永為定制焉。



 御位下打捕鷹房官:一所,權官張元,大都路寶坻縣置司,無額七十七戶。一所,王阿都赤,世襲祖父職,掌十投下、中都、順天、真定、宣德等路諸色人匠打捕等戶,元額一百四十七戶。一所,管領大都等處打捕鷹房民戶達魯花赤石抹也先,世襲祖父職,元額一百一十七戶。一所,管領大都路打捕鷹房等官李脫歡帖木兒,世襲祖父職,元額二百二十八戶。一所,宣授管領大都等處打捕鷹房人匠等戶達魯花赤黃也速鷿兒,世襲祖父職,元額五十戶。一所,管領鷹房打捕人匠等戶達魯花赤移剌帖木兒,世襲祖父職,元額一百五十七戶。一所,宣授管領打捕鷹房等戶達魯花赤阿八赤,世襲祖父職,元額三百五十五戶。一所,宣授管領大都等路打捕鷹房人戶達魯花赤寒食,世襲祖父職,元額二百四十三戶。



 諸王位下:汝寧王位下,管領民匠打捕鷹房等戶官,元額二百一戶。普賽因大王位下,管領本投下大都等路打捕鷹房諸色人匠達魯花赤都總管府,元額七百八十戶。



 天下州縣所設獵戶:腹裏打捕戶,總計四千四百二十三戶。河東宣慰司打捕戶,五百九十八戶。晉寧路打捕戶,三百三十二戶。大同路打捕戶,一十五戶。冀寧路打捕戶,二百五十一戶。上都留守司打捕戶,三百九十七戶。宣德提領所打捕戶,一百八十二戶。山東宣慰司打捕戶,三百九十七戶。宣德提領所打捕戶,一百八十二戶。山東宣慰司打捕戶,一百戶。益都路打捕戶,四十三戶。濟南路打捕戶,三十六戶。般陽路二十一戶。東平路三十四戶。曹州八十四戶。德州一十戶。濮州三十一戶。泰安州五戶。東昌路一戶。真定路九十一戶。順德路一十九戶。廣平路一十九戶。冠州五戶。恩州二戶。彰德三十七戶。衛輝路一十六戶。大名路二百八十六戶。保定路三十一戶。河間路二百五十二戶。隨路提舉司一千一百九十一戶。河間鷹戶府二百七十六名。都總管府七百五十六戶。



 遼陽大寧等處打捕鷹房官捕戶,七百五十九戶。東平等路打捕鷹房官捕戶,三百九戶。隨州德安河南襄陽懷孟等處打捕鷹房官捕戶,一百七十二戶。扠捕提領所捕戶,四十戶。高麗鷹房總管捕戶,二百五十戶。河南等路打捕鷹房官捕戶,一千一百四十二戶。益都等處打捕鷹房官捕戶,五百二十一戶。河北河南東平等處打捕鷹房官捕戶,三百戶。隨路打捕鷹房總管捕戶,一百五十九戶。真定保定等處打捕鷹房官捕戶,五十戶。淮安路鷹房官捕戶,四十七戶。揚州等處打捕鷹房官捕戶,七十二戶。



 宣徽院管轄淮東淮西屯田打捕總管府司屬打捕衙門,提舉司十處,千戶所一處,總一萬四千三百二戶。淮安提舉司八百五十八戶。安東提舉司九百一十二戶。招泗提舉司四百六十五戶。鎮巢提舉司二千五百四十戶。蘄黃提舉司一千一百一十二戶。通泰提舉司七百四十九戶。塔山提舉司六百四十四戶。魚網提舉司二千五百一十九戶。打捕手號軍上千戶所打捕軍,六百四戶。



\end{pinyinscope}